
\documentclass[10pt]{book}
\usepackage[utf8]{inputenc}

\usepackage{latexsym,amssymb,amsfonts,amsmath,amsxtra,dsfont,
indentfirst,shapepar,%fleqn,%
picinpar,shadow,floatflt,enumerate,multicol,colortbl,moreverb,cite,ipi}

\usepackage{rotating}
\usepackage{mathrsfs}
\usepackage[noend]{algorithmic}
\usepackage{ulem}
\usepackage{graphicx}
%\usepackage{algorithm2e}
\usepackage[linesnumbered,boxed,ruled]{algorithm2e}
%\usepackage{xypic}
\usepackage{oldgerm}
\usepackage{epic}
\usepackage{eepic}


\SetAlgorithmName{Algorithm}{алгоритм}{Список алгоритмов}

%из Дюковой

\newcommand{\algKeyword}[1]{{\bf #1}}
\newcommand{\Proc}[1]{\text{\tt #1}}
\def\CALL{\algKeyword{call}~}

\newenvironment{AlgProcedure}[1]
{
    \small
    \medskip
    %    \hrule
    \medskip
    \algKeyword{PROCEDURE} #1
    \begin{algorithmic}[1]}
    {\end{algorithmic}
    %    \hrule
    \bigskip
}

\def\CALL{\algKeyword{call}~}

%конец для Дюковой

%\RequirePackage[ruled]{algorithm}


\input{epsf}

%\nofiles

%\includeonly{avtor}             %+pdf+
%\includeonly{obchak,avtor}
%\includeonly{pred}                 %+
%\includeonly{podgot-rus-site,podgot-eng-site}  
%\includeonly{ocherk} 
%\includeonly{nekrol} 
%\includeonly{ipi-ind} 
%\includeonly{index13}
%\includeonly{toc-rus, toc-en}
%\includeonly{toc-rus}
%\includeonly{toc-en} 



%\includeonly{flerov}                      %1pdf
%\includeonly{bosov}                       %2pdf
%\includeonly{lange}                        %3pdf
%\includeonly{krivenko}                    %4pdf
%\includeonly{senko}                       %5pdf
%\includeonly{agasan}                      %6pdf
%\includeonly{ushakovi}                    %7pdf
%\includeonly{shest}                       %8pdf
%\includeonly{konovalov}                   %9pdf
%\includeonly{meih}                        %10pdf
%\includeonly{borisov}                     %11pdf
%\includeonly{kudr}                        %12pdf
%\includeonly{grusho-zab}                  %13pdf
%\includeonly{grusho-timonina}             %14pdf
%\includeonly{goncharov}                   %15pdf
%\includeonly{zatsman}                     %16pdf+цвет
%\includeonly{seif}                        %17pdf




%\includeonly{nekrol}             %+


%\includeonly{obchak}
%\includeonly{rekl}
%\includeonly{rekl-1}
%\includeonly{reshal}  %
%\includeonly{cover3}

\usepackage{acad}
%\usepackage{courier}
\usepackage{decor}
\usepackage{newton}
\usepackage{pragmatica}
\usepackage{zapfchan}
\usepackage{petrotex}
\usepackage{bm}                     % полужирные греческие буквы
\usepackage{upgreek}                % прямые греческие буквы
\usepackage{eufrak}
\usepackage{verbatim}

\renewcommand{\bottomfraction}{0.99}
\renewcommand{\topfraction}{0.99}
\renewcommand{\textfraction}{0.01}

\setcounter{secnumdepth}{1} %здесь - 3 + chapter = 4

\arraycolsep=1.5pt

%\usepackage[pdftex]{graphicx}

%\usepackage{oz}

%NEW COMMANDS


\renewcommand*{\hm}[1]{#1\nobreak\discretionary{}%
            {\hbox{$\mathsurround=0pt #1$}}{}} %% Дублирует знаки операций
                               %при переносе в формуле (перед знаком, который
                               %надо продублировать ставится команда \hm)

%\newcommand{\endproof}{\hfill$\Box$}
\renewcommand{\r}{\mathbb{R}}
%\newcommand{\I}{{\rm I\hspace{-0.7mm}I}}
%\newcommand{\Ikl}{{\tt{1}}\hspace*{-1.44mm}\mathtt{1}}
\newcommand{\Ik}{\mbox{{\small \tt {1}}\hspace{-1.3mm}{\tt 1}}}
\newcommand{\argmin}{\mathop{\mathrm{arg}\,\mathrm{min}}}
\newcommand{\argmax}{\mathop{\mathrm{arg}\,\mathrm{max}}}
%\newcommand{\capr}{\mathop{\cap\,}}
%\newcommand{\cupr}{\mathop{\cup\,}}
%\def\argmin{\mathop{arg\,min}}

\def\vrp{\varphi}
\def\prt{\partial}
\def\mm{{\sf M}}
\def\modnop#1{\mathop{#1}\limits_{n}}
\def\eam{\mathbin{{\mathop{=}\limits^{\mathrm{def}}}}}
\def\dey#1#2{#1 (#2)}
\def\deyc#1#2{#1 \cdot  #2}
\def\ra#1{\;\mathop{\to}\limits^{#1}\;}
\def\raz#1{\;\mathop{\longrightarrow}\limits^{\!\!\!#1}\;}
\def\ral#1{\;\mathop{\longrightarrow}\limits^{#1}\;}

\newcommand{\Nor}{\mathcal{N}}
\newcommand{\T}{\mathbb{T}}
\newcommand{\Z}{\mathbb{Z}}



\newcommand{\il}[2]{\int\limits_{#1}^{#2}}%интеграл с пределами #1 и #2

\def\sm2{\mathop {\sum\limits^{n^\Theta}\sum\limits^{n^\Theta}}}
\def\sss{\sum\limits}
\def\tr{,\,\ldots\,,\,}
\def\rk{\right]}
\def\lk{\left[}
\def\rf{\right\}}
\def\lf{\left\{}
\def\lv{\,\left\vert}
\def\rv{\right\vert\,}
\def\iii{\int\limits}
\def\iin{\int\limits_{-\infty}^\infty}
\def\rrv{\right\vert}


\def\ee{{\cal E}}
\def\ww{{\cal W}}
\def\yy{{\cal Y}}
\def\vv{{\cal V}}

\newcommand{\R}{\mathbb R}
\newcommand{\E}{\mathbb E}
\newcommand{\N}{\mathbb N}

\renewcommand{\P}{\mathbb{P}}

\newcommand{\h}{{\bf H}}
\newcommand{\p}{{\sf P}}  % вероятность

\newcommand{\e}{{\sf E}}  % мат. ожидание
\newcommand{\D}{{\sf D}}  % дисперсия
\newcommand{\eps}{\varepsilon}
\newcommand{\vp}{{\mathbf p}}
\newcommand{\vz}{{\mathbf z}}
\newcommand{\vx}{{\mathbf x}}
\newcommand{\vf}{{\mathbf f}}
\newcommand{\F}{{\mathcal F}}
\def\ap{{\mathrm{ЭР}}}
\newcommand{\ud}{\Delta_n} %uniform ditance
\newcommand{\nud}{\Delta_n(x)}
%\renewcommand{\Re}{\mathrm{Re}\,}

\newcommand{\abs}[1]{\left\vert#1\right\vert}

\newcommand{\norm}[1]{\left\Vert#1\right\Vert}
\def\da{(\Delta_t,A)}

\newcommand{\corr}{\mathrm{corr}}

\newcommand{\cov}{\mathrm{cov}}
\newcommand{\Expect}{\mathbb{E}}

\def\w{\omega}
\def\W{\Omega}

\def\inh{\int\limits_{nh}^{(n+1)h}}

\def\sumin{\sum_{i=1}^N}


\def\bxt{(Y,t)}
\def\xt{(y,t)}

\def\ovth{{\fr{\tau-nh}{h}}}
\def\ov{\overline}
\def\tm{\tilde m}
\def\tl{\tilde\lambda}
\def\tB{\widetilde B}
\def\tb{\tilde b}
\def\ld{\ldots}
\def\cd{\cdots}


\DeclareMathOperator{\sign}{sign}

%\newcommand{\gr}{{\geqslant}}


\newcommand{\g}{\mbox{\textit{g}}}

\renewcommand{\la}{\lambda}
\newcommand{\si}{\sigma}
\newcommand{\alp}{\alpha}

\newcommand{\pto}{\stackrel{P}{\longrightarrow}} % сходимость по веpоятности

\newcommand{\eqd}{\stackrel{\mathrm{d}}{=}} % равенство по pаспpеделению
\newcommand{\eqdelta}{\stackrel{\triangle}{=}} % равенство по pаспpеделению

\def\be#1{\begin{equation}\label{#1}}
\def\ee{\end{equation}}
\def\re#1{(\ref{#1})}

\def\bn{\begin{enumerate}}
\def\en{\end{enumerate}}
\def\bi{\begin{itemize}}
\def\ei{\end{itemize}}
%\def\i{\item}

%\newcommand{\kp}{\kappa}
%\def\Q{{\cal Q}} \def\H{{\cal H}}
%\newcommand{\bet}{\beta_{2+\delta}}


%\newtheorem{definition}{Определение}
%\renewcommand{\thedefinition}{\arabic{definition}.}
%END NEW COMMANDS

%\renewcommand{\baselinestretch}{1.2}

%\pagestyle{myheadings}

\setlength{\textwidth}{167mm}      % 122mm
\setlength{\textheight}{658pt}
%\setlength{\textheight}{635.6pt}
\setlength{\columnsep}{4.5mm}

\setcounter{secnumdepth}{4}

%\addtolength{\headheight}{2pt}
%\addtolength{\headsep}{-2mm}

\addtolength{\topmargin}{-7mm}  % for printing


%\hoffset=-30mm  % From Yap
\hoffset=-23mm  % From Acrobat

%\voffset=0mm % From Yap
\voffset=-5mm   % From Acrobat

%\addtolength{\evensidemargin}{-2.5mm} % for printing
%\addtolength{\oddsidemargin}{2.5mm}  % for printing

\addtolength{\evensidemargin}{-12mm} % for printing
\addtolength{\oddsidemargin}{8mm}  % for printing

%\renewcommand{\thefootnote}{\fnsymbol{footnote}}
%\renewcommand{\thefootnote}{\arabic{footnote}}
\renewcommand{\figurename}{\protect\bf Рис.}
\renewcommand{\tablename}{\protect\bf Таблица}

\newcommand{\Caption}[1]{\caption{\protect\small %\baselineskip=2.5ex
#1}}

\renewcommand{\thefigure}{\arabic{figure}}
\renewcommand{\thetable}{\arabic{table}}
\renewcommand{\theequation}{\arabic{equation}}
\renewcommand{\thesection}{\arabic{section}}

\renewcommand{\contentsname}{СОДЕРЖАНИЕ}
\newcommand{\fr}[2]{\displaystyle\frac{\displaystyle #1\mathstrut}{\displaystyle #2\mathstrut}}

%\renewcommand{\thefootnote}{\fnsymbol{footnote}}
%\newcommand{\g}{\mbox{\textit{g}}}

%\newcommand{\Caption}[1]{\caption{\protect\small\baselineskip=2ex #1}}
\newcounter{razdel}
\setcounter{razdel}{0}


\newcommand{\titel}[4]{%
\

\vspace*{5pt}

\ifodd\therazdel {\raggedright\noindent\Large\textrm\textbf
 \lineskip .75em
  \baselineskip=3.2ex #1 \par}
\vskip 1em {\noindent\large\textrm\textbf #2 \par}
\addcontentsline{toc}{subsection}{{\textrm\textbf #1}\protect\newline #2}
\def\rightheadline{\underline{\noindent\hbox to \textwidth{\hfill\small\textrm{#4}
%\hfill \large\bf\thepage
}}}
\def\leftheadline{\underline{\noindent\parbox{\textwidth}{
%\raggedleft\large\bf\thepage \hfill
\small\textit{#3}\hfill}}}
\def\leftfootline{\small{\textbf{\thepage}
\hfill ИНФОРМАТИКА И ЕЁ ПРИМЕНЕНИЯ\ \ \ том~13\ \ \ выпуск 4\ \ \ 2019}
}%
 \def\rightfootline{\small{ИНФОРМАТИКА И ЕЁ ПРИМЕНЕНИЯ\ \ \ том~13\ \ \ выпуск~4\ \ \ 2019
\hfill \textbf{\thepage}}}
\vskip 2em \setcounter{figure}{0}
\setcounter{table}{0}
\setcounter{equation}{0}
\setcounter{section}{0}
\setcounter{subsection}{0}
\setcounter{subsubsection}{0}
\setcounter{footnote}{0}
\setcounter{razdel}{0}
%\end{flushleft}
\else {
 \raggedright\noindent\Large\textrm\textbf
 \lineskip .75em
\baselineskip=3.2ex #1 \par} \vskip 1em
%\begin{flushleft}
{\noindent\large\textrm\textbf #2 \par}
\addcontentsline{toc}{subsection}{{\textrm\textbf #1}\protect\newline #2}
\def\rightheadline{\underline{\noindent\hbox to \textwidth{\hfill\small\textrm{#4}
%\hfill \large\bf\thepage
}}}
\def\leftheadline{\underline{\noindent\parbox{\textwidth}{%\raggedleft\large\bf\thepage \hfill
\small\textit{#3}\hfill}}}
\def\leftfootline{\small{\textbf{\thepage}
\hfill ИНФОРМАТИКА И ЕЁ ПРИМЕНЕНИЯ\ \ \ том~13\ \ \ выпуск~4\ \ \ 2019}
}%
 \def\rightfootline{\small{ИНФОРМАТИКА И ЕЁ ПРИМЕНЕНИЯ\ \ \ том~13\ \ \ выпуск~4\ \ \ 2019
\hfill \textbf{\thepage}}} \vskip 2em \setcounter{figure}{0}
\setcounter{table}{0} \setcounter{equation}{0} \setcounter{section}{0}
\setcounter{subsection}{0} \setcounter{subsubsection}{0}
\setcounter{footnote}{0}
%\end{flushleft}
\fi}

\newcommand{\titelr}[2]{%
\

\vspace*{5pt}

\ifodd\therazdel {\raggedright\noindent%\Large\textrm\textbf
 \lineskip .75em
  \baselineskip=3.2ex #1 \par}
\vskip 1em {\noindent\normalsize\textrm\textbf #2 \par}
\else {
 \raggedright\noindent\Large\textrm\textbf
 \lineskip .75em
\baselineskip=3.2ex #1 \par} \vskip 1em
%\begin{flushleft}
{\noindent\large\textrm\textbf #2 \par
%\noindent\normalsize\textrm\textbf #2 \par
} \fi}

\newcommand{\titele}[5]{%
\

%\vspace*{5pt}

\ifodd\therazdel {\raggedright\noindent\large
\textrm\textbf
 \lineskip .75em
%  \baselineskip=3.2ex
#1 \par}
\vskip .5em {\noindent\large\textrm\textbf #2 \par}
\vskip .5em
 {\noindent\textrm #3 \par}
\addcontentsline{toc}{subsection}{{\textrm\textbf #1}\protect\newline #2}
\def\rightheadline{\underline{\noindent\hbox to \textwidth{\hfill\small\textrm{#4}
%\hfill \large\bf\thepage
}}}
\def\leftheadline{\underline{\noindent\parbox{\textwidth}{
%\raggedleft\large\bf\thepage \hfill
\small\textrm{#5}\hfill}}}
\def\leftfootline{\small{\textbf{\thepage}
\hfill ИНФОРМАТИКА И ЕЁ ПРИМЕНЕНИЯ\ \ \ том~13\ \ \ выпуск~4\ \ \ 2019}
}%
 \def\rightfootline{\small{ИНФОРМАТИКА И ЕЁ ПРИМЕНЕНИЯ\ \ \ том~13\ \ \ выпуск~4\ \ \ 2019
\hfill \textbf{\thepage}}} \vskip 1em \setcounter{figure}{0}
\setcounter{table}{0} \setcounter{equation}{0} \setcounter{section}{0}
\setcounter{subsection}{0} \setcounter{subsubsection}{0}
\setcounter{footnote}{0} \setcounter{razdel}{0}
%\end{flushleft}
\else {
 \raggedright\noindent\large
 \textrm\textbf
 \lineskip .75em
%\baselineskip=3.2ex
#1 \par} \vskip .5em
%\begin{flushleft}
{\noindent\large\textrm\textbf #2 \par} \vskip .5em
 {\noindent\textrm #3 \par}
\addcontentsline{toc}{subsection}{{\textrm\textbf #1}\protect\newline #2}
\def\rightheadline{\underline{\noindent\hbox to \textwidth{\hfill\small\textrm{#4}
%\hfill \large\bf\thepage
}}}
\def\leftheadline{\underline{\noindent\parbox{\textwidth}{%\raggedleft\large\bf\thepage \hfill
\small\textrm{#5}\hfill}}}
\def\leftfootline{\small{\textbf{\thepage}
\hfill ИНФОРМАТИКА И ЕЁ ПРИМЕНЕНИЯ\ \ \ том~13\ \ \ выпуск~4\ \ \ 2019}
}%
 \def\rightfootline{\small{ИНФОРМАТИКА И ЕЁ ПРИМЕНЕНИЯ\ \ \ том~13\ \ \ выпуск~4\ \ \ 2019
\hfill \textbf{\thepage}}} \vskip 1em \setcounter{figure}{0}
\setcounter{table}{0} \setcounter{equation}{0} \setcounter{section}{0}
\setcounter{subsection}{0} \setcounter{subsubsection}{0}
\setcounter{footnote}{0}
%\end{flushleft}
\fi}

\def\Abst#1{
\begin{center}\small\nwt
\parbox{150mm}{%\baselineskip=2.5ex
\textbf{Аннотация:}\ \
%\hspace*{\parindent}
#1}
\end{center}}
\def\Abste#1{
\begin{center}\small\nwt
\parbox{150mm}{%\baselineskip=2.5ex
\textbf{Abstract:}\ \
%\hspace*{\parindent}
#1}
\end{center}}

\def\DOI#1{
\begin{center}\small\nwt
\parbox{150mm}{%\baselineskip=2.5ex
\textbf{DOI:}\ \
%\hspace*{\parindent}
#1}
\end{center}}

\def\Abstend#1{
\begin{center}\small\nwt
\parbox{150mm}{%\baselineskip=2.5ex
%\hspace*{\parindent}
#1}
\end{center}}


\def\KW#1{
\begin{center}\small\nwt
\parbox{150mm}{%\baselineskip=2.5ex
\textbf{Ключевые слова:}\ \ #1}
\end{center}}

\def\KWE#1{
\begin{center}\small\nwt
\parbox{150mm}{%\baselineskip=2.5ex
\textbf{Keywords:}\ \ #1}
\end{center}}


\def\KWN#1{
%\begin{center}
%\small
%\parbox{150mm}\end{center}
}

\newcommand{\Avtors}[1]{%\smallskip
%\vspace*{.5pt}
\hangindent=23pt\noindent
%\nwt
{\bfseries#1}\
}


\renewcommand{\thesubsection}{\thesection.\arabic{subsection}\hspace*{-5pt}}
\renewcommand{\thesubsubsection}{\thesubsection\hspace*{5pt}.\arabic{subsubsection}\hspace*{-3pt}}

\newcommand{\Ack}{\section*{\protect\rmfamily Acknowledgments}\noindent}
\newcommand{\Contr}{\section*{\protect\rmfamily Contributors}\noindent}
\newcommand{\Contrl}{\section*{\protect\rmfamily Contributor}\noindent}

\makeindex


\begin{document}
\Rus

\nwt
%\ptb


%\renewcommand{\contentsname}{\protect\Large\bf Содержание}

\setcounter{tocdepth}{2}

%\tableofcontents

\renewcommand{\bibname}{\protect\rmfamily Литература}
  \def\Au#1{{\it #1}}
    \def\Aue#1{{#1}}

%\newcommand{\No}{№}
  \newcommand{\tg}{\,\mathrm{tg}\,}
    \newcommand{\ctg}{\,\mathrm{ctg}\,}
  \newcommand{\arctg}{\,\mathrm{arctg}\,}

\def\forallb{\mathop{\forall}}
\def\cupb{\mathop{\cup}}
\def\existsb{\mathop{\exists}}


\newpage
\addtocounter{razdel}{1}
%\def\razd{РЕГУЛИРУЕМЫЙ ЭЛЕКТРОПРИВОД ДЛЯ ЭЛЕКТРОЭНЕРГЕТИКИ}


\setcounter{page}{3}

%   { %\Large  
   { %\baselineskip=16.6pt
   
   \vspace*{-48pt}
   \begin{center}\LARGE
   \textit{Предисловие}
   \end{center}
   
   %\vspace*{2.5mm}
   
   \vspace*{25mm}
   
   \thispagestyle{empty}
   
   { %\small 

    
Вниманию читателей журнала <<Информатика и её применения>> предлагается 
очередной тематический выпуск <<Вероятностно-статистические методы и 
задачи информатики и информационных технологий>>. Предыдущие тематические 
выпуски журнала по данному направлению вышли в 2008~г.\ (т.~2, вып.~2), 
в 2009~г.\ (т.~3, вып.~3) и в 2010~г.\ (т.~4, вып.~2). 

Статьи, собранные в данном журнале, посвящены разработке новых вероятностно-статистических 
методов, ориентированных на применение к решению конкретных задач информатики и информационных 
технологий, а также~--- в ряде случаев~--- и других прикладных задач. Проблематика, охватываемая 
публикуемыми работами, развивается в рамках научного сотрудничества между Институтом проблем 
информатики Российской академии наук (ИПИ РАН) и Факультетом вычислительной математики и 
кибернетики Московского государственного университета им.\ М.\,В.~Ломоносова в ходе работ 
над совместными научными проектами (в том числе в рамках функционирования 
Научно-образовательного центра <<Вероятностно-статистические методы анализа рисков>>). 
Многие из авторов статей, включенных в данный номер журнала, являются активными участниками 
традиционного международного семинара по проблемам устойчивости стохастических моделей, 
руководимого В.\,М.~Золотаревым и В.\,Ю.~Королевым; регулярные сессии этого семинара 
проводятся под эгидой МГУ и ИПИ РАН (в 2011~г.\ указанный семинар проводится в октябре 
в Калининградской области РФ). 

Наряду с представителями ИПИ РАН и МГУ в число авторов данного выпуска журнала входят 
ученые из Научно-исследовательского института системных исследований РАН, Института 
проблем технологии микроэлектроники и особочистых материалов РАН, Института 
прикладных математических исследований Карельского НЦ РАН, Московского 
авиационного института, Вологодского государственного педагогического университета, 
НИИММ им.\ Н.\,Г.~Чеботарева, Казанского государственного университета, Дебреценского 
университета (Венгрия).

Несколько статей выпуска посвящено разработке и применению стохастических методов и 
информационных технологий для решения различных прикладных задач. В~работе В.\,Г.~Ушакова 
и О.\,В.~Шестакова рассмотрена задача определения вероятностных характеристик случайных 
функций по распределениям интегральных преобразований, возникающих в задачах эмиссионной 
томографии. В~статье Д.\,О.~Яковенко и М.\,А.~Целищева рассмотрены некоторые вопросы 
математической теории риска и предложен новый подход к диверсификации инвестиционных 
портфелей. Работа И.\,А.~Кудрявцевой и А.\,В.~Пантелеева посвящена построению и 
исследованию математической модели, описывающей динамику сильноионизованной плазмы. 
В~статье П.\,П.~Кольцова изучается качество работы ряда алгоритмов сегментации изображений. 
Статья А.\,Н.~Чупрунова и И.~Фазекаша посвящена вероятностному анализу числа без\-оши\-бочных 
блоков при помехоустойчивом кодировании; получены усиленные законы больших чисел для указанных 
величин.

В данном выпуске традиционно присутствует тематика, весьма активно разрабатываемая в течение 
многих лет специалистами ИПИ РАН и МГУ,~--- методы моделирования и управления для 
информационно-телекоммуникационных и вычислительных систем, в частности методы 
теории массового обслуживания. В~статье А.\,И.~Зейфмана с соавторами рассматриваются 
модели обслуживания, описываемые марковскими цепями с непрерывным временем в случае 
наличия катастроф. В~работе М.\,М.~Лери и И.\,А.~Чеплюковой рассматриваются случайные 
графы Интернет-типа, т.\,е.\ графы, степени вершин которых имеют степенные распределения; 
такие задачи находят применение при исследовании глобальных сетей передачи данных. 
Работа Р.\,В.~Разумчика посвящена исследованию систем массового обслуживания специального 
вида~--- с отрицательными заявками и хранением вытесненных заявок.

Ряд статей посвящен развитию перспективных теоретических 
вероятностно-статистических методов, которые находят широкое применение в различных 
задачах информатики и информационных технологий. В~работе В.\,Е.~Бенинга, А.\,К.~Горшенина 
и В.\,Ю.~Королева рассмотрена задача статистической проверки гипотез о числе компонент 
смеси вероятностных распределений, приводится конструкция асимптотически наиболее мощного 
критерия. Результаты этой работы найдут применение в ряде прикладных задач, использующих 
математическую модель смеси вероятностных распределений (в информатике, моделировании 
финансовых рынков, физике турбулентной плазмы и~т.\,д.). В~статье В.\,Ю.~Королева, 
И.\,Г.~Шевцовой и С.\,Я.~Шоргина строится новая, улучшенная оценка точности нормальной 
аппроксимации для пуассоновских случайных сумм; как известно, указанные случайные суммы 
широко используются в качестве моделей многих реальных объектов, в том числе в информатике, 
физике и других прикладных областях. Работа В.\,Г.~Ушакова и Н.\,Г.~Ушакова посвящена 
исследованию ядерной оценки плотности распределения; эти результаты могут применяться, 
в част\-ности, при анализе трафика в телекоммуникационных системах. Серьезные приложения 
в статистике могут получить результаты работы О.\,В.~Шестакова, в которой доказаны оценки 
скорости сходимости распределения выборочного абсолютного медианного отклонения к нормальному 
закону. 

\smallskip

Редакционная коллегия журнала выражает надежду, что данный тематический  выпуск 
будет интересен специалистам в области теории вероятностей и математической статистики 
и их применения к решению задач информатики и информационных технологий.
     
     %\vfill 
     \vspace*{20mm}
     \noindent
     Заместитель главного редактора журнала <<Информатика и её 
применения>>,\\
     директор ИПИ РАН, академик  \hfill
     \textit{И.\,А.~Соколов}\\
     
     \noindent
     Редактор-составитель тематического выпуска,\\
     профессор кафедры математической статистики факультета\\
      вычислительной математики и кибернетики МГУ им.\ М.\,В.~Ломоносова,\\
     ведущий научный сотрудник ИПИ РАН,\\ 
доктор физико-математических наук \hfill
      \textit{В.\,Ю.~Королев}
     
     } }
     }

\def\stat{flerov}

\def\tit{АВТОМАТИЗИРОВАННАЯ СИСТЕМА ВЕСОВОГО 
ПРОЕКТИРОВАНИЯ САМОЛЕТОВ}

\def\titkol{Автоматизированная система весового 
проектирования самолетов}

\def\aut{Л.\,Л.~Вышинский$^1$, Ю.\,А.~Флеров$^2$, Н.\,И.~Широков$^1$}

\def\autkol{Л.\,Л.~Вышинский, Ю.\,А.~Флеров, Н.\,И.~Широков}

\titel{\tit}{\aut}{\autkol}{\titkol}

\index{Вышинский Л.\,Л.}
\index{Флеров Ю.\,А.}
\index{Широков Н.\,И.}
\index{Vyshinsky L.\,L.}
\index{Flerov Yu.\,A.}
\index{Shirokov N.\,I.}




%{\renewcommand{\thefootnote}{\fnsymbol{footnote}} \footnotetext[1]
%{Работа выполнена при финансовой поддержке РФФИ (проект 17-01-00816).}}


\renewcommand{\thefootnote}{\arabic{footnote}}
\footnotetext[1]{Вычислительный центр им.\ А.\,А.~Дородницына Федерального исследовательского 
центра <<Информатика и~управ\-ле\-ние>> Российской академии наук, 
\mbox{Wysh@ccas.ru}}
\footnotetext[2]{Вычислительный центр им.\ А.\,А.~Дородницына Федерального исследовательского 
центра <<Информатика и~управ\-ле\-ние>> Российской академии наук, 
fler@ccas.ru}
%\footnotetext[3]{Вычислительный центр им.\ А.\,А.~Дородницына Федерального исследовательского 
%центра <<Информатика и~управ\-ле\-ние>> Российской академии наук, 
%\mbox{Wysh@ccas.ru}}

%\vspace*{-6pt}


 
  \Abst{Статья посвящена вопросам автоматизации задач весового проектирования 
самолетов. Весовые и~мас\-со\-во-инер\-ци\-он\-ные параметры являются одними из основных 
величин, влияющих на эксплуатационные характеристики самолетов. Информационной 
основой системы служит весовая модель самолета. Описывается структура весовой 
модели и~даны характеристики отдельным ее компонентам. Показана программная 
реализация системы, которая выполнена в~рамках архитектуры кли\-ент--сер\-вер. 
Автоматизированная система весового проектирования (АСВП)
реализована с~использованием 
про\-грам\-мно-ин\-стру\-мен\-таль\-но\-го комплекса <<Генератор проектов>> (технология ГП), 
который был разработан в~Вычислительном центре Российской академии наук. Создание 
ин\-фор\-ма\-ци\-он\-но-вы\-чис\-ли\-тель\-ных сис\-тем в~рамках технологии ГП базируется на так 
называемом <<проектном подходе>>, когда по формальному описанию системы автоматически 
генерируются тексты программного кода для клиентских и~серверных компонент системы.}
   
  \KW{математическое моделирование; автоматизация проектирования; самолет; весовое 
проектирование; весовая модель; дерево конструкции; генератор проектов; генерация 
программного кода; архитектура кли\-ент--сер\-вер}

  \DOI{10.14357/19922264180103} 
  
\vspace*{12pt}


\vskip 10pt plus 9pt minus 6pt

\thispagestyle{headings}

\begin{multicols}{2}

\label{st\stat}
   
\section{Введение}

  Развитие и~повсеместное использование информационных технологий за 
последние несколько десятилетий существенно изменили традиционный 
процесс проектирования и~создания различных инженерных систем, 
сооружений, машин. Во многих проектных организациях давно отказались от 
ко\-гда-то привычных инструментов конструктора~--- кульмана 
и~логарифмической линейки. 
%
Сейчас первые эскизы новых проектов 
появляются чаще не на бумаге, как было всегда, а~на экране монитора. Этому 
способствует широкий спектр имеющихся систем автоматизированного 
проектирования. В~российских авиационных конструкторских бюро, например, уже давно 
применяются такие CAD (computer aided design)
сис\-те\-мы, как NX (Unigraphics), CATIA и~др. 
%
Эти развитые системы геометрического трех\-мер\-но\-го (3D) мо\-де\-ли\-ро\-ва\-ния позволяют 
создавать сложные по\-верх\-ности, конструировать любые детали, осуществлять 
сборку узлов, агрегатов и~сложнейших изделий. Однако построение 
геометрических моделей изделий является финальной стадией проектирования, 
за которой следует их реализация <<в~металле>>. Построению электронных 
геометрических макетов предшествует и~сопутствует решение множества 
расчетных задач, а~также задач анализа и~оптимизации в~разных областях инженерных 
знаний. В~авиастроении это аэродинамика, динамика полета, прочность, 
системы управления, двигателестроение и~пр. Все эти задачи 
требуют разработки разноплановых математических моделей и~специальных 
вычислительных программ. 
  
  Одной из важнейших технических характеристик самолета является его вес. 
При решении подавляющего большинства проектных и~конструкторских задач 
весовые параметры в~том или ином виде участвуют в~расчетах. Необходимость 
проведения весовых расчетов возникает на самых ранних шагах 
проектирования и~сопровождает все дальнейшие стадии разработки 
и~эксплуатации. 

В~процессе создания и~эксплуатации самолетов постоянно 
контролируются вес и~другие мас\-со\-во-инер\-ци\-он\-ные характеристики (МИХ)
всех размещаемых на борту систем, агрегатов, узлов и~деталей. Количество 
агрегатов, узлов и~деталей современных самолетов исчисляется 
десятками тысяч, поэтому в~авиастроении весовые расчеты, весовой анализ, 
весовой контроль выливаются в~сложную инженерную проблему и~выделяются 
в~целое направление инженерной деятельности, которое принято называть 
весовым проектированием~[1].
  
  Данная статья посвящена вопросам автоматизации задач весового 
проектирования самолетов. В~разные годы Вычислительным центром РАН\linebreak был 
разработан и~внедрен в~эксплуатацию ряд \mbox{программ}, решающих отдельные 
задачи весовых рас\-че\-тов летательных аппаратов (ЛА)~[2--4]. В~настоящей статье 
представлено описание интегрированной АСВП, предназначенной для использования на всех 
этапах жизненного цикла изделий. Она разработана как интерактивная 
многопользовательская информационная система кли\-ент-сер\-вер\-ной 
архитектуры с~централизованной базой данных. Информационным ядром 
и~основой АСВП является единая струк\-тур\-но-па\-ра\-мет\-ри\-че\-ская весовая модель 
самолета, описание которой дает довольно полное представление о~задачах, 
решаемых с~помощью АСВП.

\section{Структурно-параметрическая весовая модель самолета}

  Самолет является сложным техническим объ\-ектом, состоящим из множества 
различных \mbox{ком\-понентов}, функционально и~конструктивно связанных между 
собой. Под струк\-тур\-но-па\-ра\-мет\-ри\-че\-ской весовой моделью самолета 
здесь понимается база данных, которая содержит всю необходи\-мую 
информацию для проведения комплекса расчетов 
МИХ и~мас\-со\-во-цент\-ро\-воч\-ных данных (МЦД) 
самолета. Весовая модель состоит из нескольких структур, ориентированных на 
определенные группы параметров и~задач весового проектирования. Ниже 
перечислены основные структуры весовой модели, реализованные в~системе 
АСВП:
  \begin{itemize}
\item дерево конструкции самолета;
\item иерархия систем координат, связанных с~самолетом и~его агрегатами;
\item геометрические структуры весовой модели самолета;
\item каталог целевой нагрузки, размещаемой во внут\-рен\-них отсеках и~на 
подвесках;
\item реестр допустимых вариантов загрузки само\-лета;
\item таблицы тарировочных характеристик топливных баков;
\item таблицы характеристик выработки топлива.
\end{itemize}


  \subsection{Дерево конструкции самолета}

  Дерево конструкции самолета является центральной структурой весовой 
модели, которая отражает членение изделия на его составные части~--- 
системы, агрегаты, узлы, детали. В~базе данных весовой модели эта структура 
представлена в~виде многоуровневого корневого дерева $W \hm= (U, V)$, где 
вершинам $U \hm= \{U_i\}$ соответствуют различные\linebreak
 элементы конструкции. 
Ориентированные дуги дере\-ва, идущие из~$U_i$ в~$U_j$, означают вхождение 
конструкции~$U_j$ в~конструкцию~$U_i$ в~качестве ее составной части. 
Терминальными или висячими вершинами дерева конструкции будем называть 
вершины, у которых нет ни одной подчиненной конструкции.
  
  Многолетний опыт самолетостроения выработал устоявшиеся 
конструктивные схемы самолетов различного назначения. Существуют 
отраслевые стандарты и~классификаторы, которые вводят определения 
основных элементов конструкции самолетов. На рис.~1 показан пример 
представления в~АСВП нескольких верхних уровней дерева конструкции 
самолета. 


    

  Существующие классификаторы отражают лишь самые общие принципы 
построения конструкции самолетов. Разумеется, каждый новый проект 
самолета имеет свои конструктивные особенности, которые отражаются на 
структуре весовой модели. Дерево конструкции строится постепенно, сверху 
вниз, в~течение всего процесса проектирования самолета. 

 { \begin{center}  %fig1
 \vspace*{9pt}
\mbox{%
 \epsfxsize=77.216mm 
 \epsfbox{fle-1.eps}
 }

\vspace*{6pt}


\noindent
{{\figurename~1}\ \ \small{Дерево конструкции самолета}}
\end{center}
}

\addtocounter{figure}{1}
  
  Понятие <<конструкции>> в~данном контексте используется и~для 
обозначения любой вершины графа, и~для всего поддерева подчиненных 
конструкций этой вершине. Каждая конструкция дерева имеет уникальное 
в~рамках весовой модели обозначение, которое может быть шифром, кодом, 
идентификатором или чертежным номером конструкции. Разумеется, для более 
полного и~наглядного вербального представления конструкции  
в~струк\-тур\-но-па\-ра\-мет\-ри\-че\-ской модели можно задать ее текстовое 
описание.
  
  \textbf{Масса конструкции.} Основную содержательную и~необходимую 
информацию весовой модели содержит список значений  
МИХ, соответствующих каждой 
вершине дерева конструкций. Центральным параметром является масса. 
  
  На разных стадиях создания самолета, когда неизвестно точное значение 
массы, прибегают к~различным оценкам.  
В~струк\-тур\-но-па\-ра\-мет\-ри\-че\-ской весовой модели фиксируются 
перечисленные ниже оценки массы, которые соответствуют разным этапам 
проектирования:
  \begin{description}
\item[\,]  $M_{\mathrm{теор}}$~--- теоретическая масса~--- оценка массы, вычисленная на 
основании некоторой математической модели конструкции; 
  
\item[\,]  $M_{\mathrm{лим}}$~--- лимитная масса конструкции, уста\-нав\-ли\-ва\-емая на 
основании теоретических оценок и~используемая для весового контроля 
в~процессе детальной разработки конструкции;
  
\item[\,]  $M_{\mathrm{черт}}$~--- чертежная масса конструкции, рассчитанная по чертежу или по 
электронной гео\-мет\-ри\-че\-ской модели конструкции;
  
\item[\,]  $M_{\mathrm{креп}}$~--- масса крепежа конструкции~--- дополнение к~чертежной массе, 
которое учитывает мелкие детали конструкции, предназначенные для 
соединения подчиненных деталей (заклепки, болты, гайки, шайбы и~т.\,п.). 
Введение такой дополнительной массы позволяет избавить дерево конструкции 
от десятков и~сотен тысяч вершин, которые несут относительно небольшую 
нагрузку в~весовых характеристиках, но тем не менее их учет необходим при 
контроле веса. Масса крепежа распределяется по подчиненным конструкциям;  
\item[\,]  $M_{\mathrm{факт}}$~--- фактическая масса изготовленной 
и~взвешенной конструкции. 
Фактическая масса может задаваться не только для изготавливаемых 
конструкций ЛА, но и~для готовых по\-став\-ля\-емых 
изделий при их установке на борту.
\end{description}
  
  Порядок задания оценок массы диктуется логикой развития проекта. 
В~дереве конструкции все оценки массы, кроме $M_{\mathrm{лим}}$ и~$M_{\mathrm{креп}}$, 
суммируются по подчиненным вершинам снизу вверх. Однако если для 
некоторых терминальных значений одна или несколько оценок не определены, 
например некоторые детали конструкции не изготовлены и, стало быть, 
$M_{\mathrm{факт}}$ не определена, то и~для всех вышестоящих конструкций эти оценки не 
определены. При задании $M_{\mathrm{лим}}$ это правило может не соблюдаться. 
  
  На основании оценок массы определяется то расчетное значение массы 
конструкции, которое используется во всех расчетах на текущей стадии 
проекта: 
  $M$~--- текущая масса конструкции. Значение текущей массы \textit{для 
нетерминальных} конструкций определяется суммированием по подчиненным 
конструкциям. \textit{Для терминальных} вершин дерева конструкций 
применяется процедура определения текущей массы по первому известному 
значению из следующего списка в~указанном порядке: $M_{\mathrm{факт}}$, 
$M_{\mathrm{черт}}$\;+\;$M_{\mathrm{креп}}$, $M_{\mathrm{теор}}$, $M_{\mathrm{лим}}$.
  
  \textbf{Геометрия масс конструкции.} Кроме собственно массы в~весовой 
модели задаются или вычисляются значения характеристик, которые принято 
называть характеристиками геометрии масс: 
  \begin{description}
  \item[\,] $X$, $Y$ и $Z$~--- положение центра масс конструкции; 
  \item[\,] $L_x$, $L_y$ и $L_z$~--- габаритные параметры конструкции;
  \item[\,] $I_x$, $I_y$ и $I_z$~--- полные плоскостные моменты инерции;
  \item[\,]  $I_{xy}$, $I_{xz}$ и $I_{yz}$~--- полные центробежные моменты 
инерции;
  \item[\,] $I^c_x$, $I^c_y$ и  $I^c_z$~--- собственные плоскостные моменты 
инерции:
  \begin{align*}
  I^c_x &= I_x - M X^2\,;\\ 
  I^c_y &= I_y - M Y^2\,;\\ 
  I^c_z &= I_z - M Z^2\,;
 \end{align*}
  \item[\,] $I^c_{xy}$, $I^c_{xz}$ и~$I^c_{yz}$~--- собственные центробежные 
моменты инерции:
 \begin{align*}
  I^c_{xy} &= I_{xy}- M X Y\,;\\
   I^c_{xz} &= I_{xz}- M X Z\,;\\
   I^c_{yz} &= I_{yz}- M Y Z\,;
\end{align*}
  \item[\,] $J_x$, $J_y$ и $J_z$~--- собственные осевые моменты инерции 
конструкции:
  \begin{align*}
  J_x &= I^c_y + I^c_z\,;\\ 
  J_y &= I^c_x + I^c_z\,;\\
   J_z &= I^c_y + I^c_x\,;
  \end{align*}
  \item[\,] СК~--- система координат конструкции, в~которой задаются 
характеристики геометрии масс.
  \end{description}
  
  \begin{figure*} %fig2
  \vspace*{1pt}
 \begin{center}
 \mbox{%
 \epsfxsize=162mm 
 \epsfbox{fle-2.eps}
 }
 \end{center}
\vspace*{-9pt}
  \Caption{Основные параметры конструкций весовой модели самолета}
  \end{figure*}
  
  Каждая конструкция привязывается к~одной из систем координат, которые 
описаны в~весовой модели. В~весовой модели изделия для удобства описания 
различных агрегатов может быть описано несколько систем координат. Все 
описанные сис\-те\-мы координат организованы в~иерархическую структуру. 
Считается предописанной глобальная самолетная система координат, в~которой 
могут быть заданы или вычислены координаты всех объектов весовой модели. 
Каждая система координат в~весовой модели задается уникальным именем, 
положением начала координат относительно вышестоящей системы координат 
и~тремя углами поворота относительно вышестоящей. 

Параметр, 
обозначенный как СК,~--- это имя одной из сис\-тем координат весовой модели. 
Если СК не задано, то считается, что характеристики гео\-мет\-рии масс заданы 
в~глобальной системе координат. Каж\-дая сис\-те\-ма координат весовой модели 
содержит матрицу преобразования координат из самолетной (глобальной) 
системы координат в~данную и~обратно. Эта матрица для каждой системы 
координат есть произведение локальных матриц преобразований 
в~соответствии с~положением данной системы в~иерархии систем координат. 
Любое изменение параметров систем координат требует пе\-ре\-вы\-чис\-ле\-ния 
матриц преобразования как измененной сис\-те\-мы, так и~всех подчиненных. На 
рис.~2 показана панель параметрического пред\-став\-ле\-ния конструкций весовой 
модели.
  
  Так же как и~масса, центры тяжести и~моменты инерции вычисляются снизу
вверх от терминальных конструкций к~вышестоящим. При этом осуществляется 
пересчет характеристик по заданной иерархии систем координат от 
нижестоящих к~вышестоящим и~к~самолетной системе координат. Расчет 
МИХ терминальных конструкций 
осуществляется на основании гео\-мет\-ри\-че\-ских моделей. Геометрические модели 
на этапе рабочего проекта строятся в~системах гео\-мет\-ри\-че\-ско\-го 
моделирования. В~процессе их построения автоматически вычисляются 
объемы, массы, положение центра тяжести и~другие характеристики гео\-мет\-рии 
масс. Рассчитанная в~системах гео\-мет\-ри\-че\-ско\-го моделирования масса 
с~по\-мощью соответствующих интерфейсных средств может быть загружена как 
$M_{\mathrm{черт}}$ в~весовую модель. (Раньше документация была представлена в~виде 
чертежей на бумажных носителях и~$M_{\mathrm{черт}}$ вручную вычислялась по этим 
чертежам.) Однако на более ранних этапах проектирования, когда еще не 
проработана гео\-мет\-рия многих элементов конструкции, весовые расчеты 
проводятся на основании эскизов и~наборов гео\-мет\-ри\-че\-ских и~конструктивных 
параметров агрегатов изделия. Для этого в~весовой модели должны быть 
предусмотрены средства параметрического представления гео\-мет\-рии 
конструкций. Геометрическое пред\-став\-ле\-ние конструкций 
в~автоматизированной системе весового проектирования выполняет 
и~немаловажную функцию визуализации конструкций, их компоновки, 
размещения нагрузки и~т.\,д. В~АСВП реализовано несколько форм 
представления гео\-мет\-ри\-че\-ской информации, ориентированных не только на 
расчет МИХ, но и~на визуализацию выполняемых расчетов. Это чертежи 
геометрических проекций изделия, это таб\-лич\-ное задание типовых 
геометрических конструкций, это каркасное представление трехмерных 
геометрических моделей и, наконец, задание объемных конструкций 
триангуляционной (фасеточной) поверхностью. Последний вид представления 
является наиболее перспективным для точного вычисления МИХ. В~АСВП он 
применяется для расчета тарировочных характеристик топливных баков, о~чем 
будет сказано ниже.
  
  \textbf{Классификационные признаки конструкции.} В~весовой модели 
кроме числовых параметров опре\-делен ряд классификационных признаков 
конструкций, по которым проводится весовой анализ.\linebreak
 Таки\-ми маркерами могут 
быть подразделения, ответст\-вен\-ные за разработку конструкции, поставщики 
или изготовители готовых изделий, принадлежность конструкции 
к~определенным функциональным системам, конструкционные материалы 
и~пр.
  
  \textbf{Функциональные подсистемы изделия.} Конст\-рук\-тив\-ное членение 
самолета может не совпадать с~его функциональной структурой. Отдельные\linebreak 
элементы функциональных подсистем самолета удобнее описывать в~составе 
конструкции ка\-ко\-го-ни\-будь агрегата планера. Например, некоторая деталь 
может конструктивно входить в~состав крыла, а принадлежать 
к~функциональной подсистеме гидравлики или электрооборудования. Для того 
чтобы иметь возможность выполнять весовые расчеты, учитывая разные 
подходы к~классификации конструкции самолета, в~АСВП отдельно от дерева 
конструкции ведется реестр подсистем, для которых может быть проведен 
специальный расчет весовых параметров. В~этом реестре ведется полный 
перечень конструкций весовой модели, входящих в~подсистемы реестра, 
независимо от того, в~какой ветви дерева конструкции они находятся. Любая 
конструкция может быть включена только в~одну из подсистем реестра. 
В~зависимости от режима расчетов МИХ
конструкций, входящих в~под\-сис\-те\-му, будут учтены либо в~со\-ста\-ве 
вышестоящих агрегатов дерева конструкции, либо отдельно в~под\-сис\-теме. 
{\looseness=1

}
  
  \textbf{Распределенные характеристики изделия.} Задача вычисления 
распределенных характеристик изделия является родственной задачей 
вычисления характеристик геометрии масс. Основное отличие состоит в~том, 
что в~данной задаче рассчитываются не интегральные характеристики 
распределения материала, а сама функция распределения массы по объему 
конструкции. Такие функции рассчитываются по заданному геометрическому 
разбиению конструкции на пространственные отсеки. Функции распределения 
массы по объему конструкции в~процессе проектирования используются при 
построении динамически подобных моделей для проведения некоторых видов 
испытаний и~продувок, а~также для выполнения прочностных расчетов. 
  
  Каждый отсек разбиения для расчета распределенных характеристик 
представляет собой вы\-пук\-лый многогранник, ограниченный конечным набором 
плоскостей. Задача построения распределенных весовых характеристик состоит 
в~вычислении для каждого отсека массы и~положения центра тяжести той части 
конструкции самолета, которая геометрически расположена внутри этого 
отсека. Эта задача решается путем нахождения геометрического пересечения 
каждой терминальной конструкции с~каждым отсеком разбиения, и~если такое 
пересечение не пусто, то вычисление массы и~центра тяжести той части 
конструкции, которая попадает в~отсек. Некоторые конструкции могут быть 
объявлены сосредоточенными массами. Использование сосредоточенных масс 
позволяет исключить все подчиненные конструкции из распределения по 
отсекам и~рассматривать их отдельно для задания сосредоточенных нагрузок. 
Список сосредоточенных масс с~уникальными именами представляет собой 
отдельную структуру весовой модели. Каждая сосредоточенная масса содержит 
список ссылок на конструкции весовой модели. Любая конструкция может 
быть включена не более чем в~одну сосредоточенную массу.
  
  \textbf{Весовые сводки.} Одной из основных задач \mbox{АСВП} является 
построение так называемых весовых сводок. Весовые сводки являются 
документами, сопровождающими построение весовой модели самолета 
в~процессе его создания. В АСВП реализовано несколько форм весовых 
сводок, которые с~разных сторон отражают дерево конструкции самолета или 
отдельных ветвей этого дерева. Назначение этих сводок и~форма представления 
зависят от ре\-ша\-емых задач. Весовые данные в~сводках могут быть 
представлены либо в~табличном виде, либо в~виде иерархии конструкций. 
Могут содержать информацию в~детализированном или в~укрупненном виде. 
Отдельные виды весовых сводок пред\-став\-ля\-ют распределенные 
характеристики по отсекам. Весовые сводки предназначены для решения задач 
весового контроля и~весового анализа. 
  
  Весовой контроль при проектировании самолетов представляет собой  
ор\-га\-ни\-за\-ци\-он\-но-тех\-ническую сис\-те\-му, нацеленную на создание 
конструк\-ции минимального веса. Для эффективного \mbox{весового} контроля 
необходима оперативная информация о текущей массе изделия и~любой его 
части. Весовая информация для весового контроля в~АСВП представляется 
в~виде оперативных весовых сводок по отдельным подразделениям 
предприятия. В~таких весовых сводках содержится информация о текущей, 
теоретической, лимитной,\linebreak чертежной и~фактической массах конструкций, 
разрабатываемых данным подразделением. Могут также выпускаться 
оперативные сводки по группе подразделений или по всему проекту. Сводки 
весового контроля предназначены для использования руководителями проекта.
  
  Весовой анализ также связан с~выпуском определенного вида весовых 
сводок. Для решения задач весового анализа в~АСВП осуществляется 
сортировка и~выборки конструкций по определенному классификационному 
признаку. Например, могут быть рассчитаны массы силового и~несилового 
набора конструкции, массы продольного и~поперечного набора, массы 
конструкций определенного материала, массы готовых изделий или изделий 
конкретного поставщика и~т.\,д. Весовой контроль и~анализ позволяют 
выявить резервы конструкции, узкие места, тренды в~изменении веса 
кон\-ст\-рук\-ции.
{\looseness=1

}
  
  \subsection{Постоянные и~переменные структуры весовой модели 
самолета}
  
  Дерево конструкции весовой модели готового изделия не является 
статической структурой. Конфигурация самолета зависит от конкретных 
условий его применения. Мас\-со\-во-инер\-ци\-он\-ные характеристики при 
взлете и~посадке отличаются от тех же характеристик в~полете, когда убраны 
стойки шасси. Конфигурация меняется и~в~полете у~самолетов с~изменяемым 
углом стреловидности или с~измененяемым вектором тяги. Текущая 
конфигурация является одним из параметров весовой модели и~параметров 
весовых расчетов. По самому смыс\-лу создания самолета как транспортного 
средства предполагается, что кроме собственно конструкции, которая 
обеспечивает выполнение основных задач, на его  
МИХ существенным образом влияет 
перевозимая нагрузка. Перевозимая нагрузка есть переменная часть структуры 
дерева конструкции. Самолетные весовые классификаторы выделяют 
постоянную часть массы изделия и~переменную, состоящую из снаряжения, 
топлива и~целевой нагрузки:
  \begin{multline*}
{M} = M_{\mathrm{пустого}} + 
M_{\mathrm{снаряжения}} + {}\\
{}+M_{\mathrm{топлива}} + 
M_{\mathrm{целевой\_нагрузки}}\,.
  \end{multline*}
  
  Все переменные и~постоянные компоненты самолета составляют единое 
целое, и~расчет мас\-со\-во-инер\-ци\-он\-ных и~центровочных характеристик 
допусти\-мых конфигураций является одной из главных задач проектирования 
самолетов любого назначения. Переменные структуры в~весовой модели могут 
задаваться альтернативными конструкциями, когда у некоторой вершины 
дерева есть несколько вариантов поддеревьев и~когда любой из вариантов, но 
только один из них, может быть активирован в~конкретный момент времени. 
Существует своя специфика задания переменных структур весовой модели для 
разных содержательных задач. 
  
  \textbf{Пустой самолет}~--- это постоянная часть конструкции самолета, 
которая не меняется в~процессе эксплуатации готового изделия. Компонентами 
пустого самолета являются конструкция планера самолета, силовая установка 
и~ее системы, другие самолетные системы, обеспечивающие управление 
самолетом, а~также специальные системы бортового оборудования, 
предназначенные для решения целевых задач самолета. В~процессе 
проектирования и~при эксплуатации самолетов рассматриваются различные 
варианты отдельных конструкций планера, а~чаще~--- различные варианты 
по\-став\-ля\-емых готовых изделий. В~связи с~этим в~весовой модели АСВП 
рассматриваются возможные комбинации вариантов пустого самолета, 
вариантов снаряжения и~полезной нагрузки. 

\begin{figure*} %fig3
\vspace*{1pt}
 \begin{center}
 \mbox{%
 \epsfxsize=162mm 
 \epsfbox{fle-3.eps}
 }
 \end{center}
\vspace*{-9pt}
\Caption{Тарировочная таблица топливного бака}
\end{figure*}
  
  \textbf{Снаряжение самолета} устанавливается на борту в~процессе 
предполетной подготовки. Снаряжение самолета принято разделять на 
основное и~дополнительное. Основное снаряжение включает несколько 
позиций. Это экипаж и~системы жизнеобеспечения экипажа, системы 
жизнеобеспечения пассажиров, заправляемые компоненты и~расходуемые 
материалы, несливаемый остаток топлива и~другие возможные компоненты. 
Использование различных вариантов экипажа и~другого снаряжения самолета 
связано с~различным характером выполняемых задач. Как правило, существует 
несколько типовых вариантов комплектации экипажа 
и~элементов снаряжения. Весовая модель должна содержать перечень 
альтернативных вариантов снаряжения и~их характеристик. Естественно, что 
этот перечень может модифицироваться. К~дополнительному снаряжению 
относят временное оборудование и~средства, связанные с~установкой на борту 
и~закреплением на подвесках перевозимых грузов. Временно устанавливаемое 
оборудование, как правило, связано со спецификой полетных заданий. Это 
может быть специальная измерительная аппаратура или оборудование, которое 
необходимо проверить в~условиях реального полета. Перечень такого 
оборудования и~его характеристики в~весовой модели должны быть 
пред\-став\-ле\-ны в~специальном реестре, или в~каталоге. Для установки 
оборудования, размещения любой коммерческой нагрузки и~вооружения в~конструкции самолета
должны быть  предусмотрены специальные места 
размещения и~узлы крепления. Точки размещения оборудования и~любых 
элементов целевой нагрузки задаются своими координатами и~установочными 
углами закрепления. 

\begin{figure*} %fig4
  \vspace*{1pt}
 \begin{center}
 \mbox{%
 \epsfxsize=162mm 
 \epsfbox{fle-4.eps}
 }
 \end{center}
\vspace*{-11pt}
\Caption{Варианты размещения целевой нагрузки самолета на подвесках}
\end{figure*}
  
  \textbf{Топливо}~--- величина переменная и~на земле, при подготовке 
самолета к~вылету, и~в~воздухе, при выработке топлива, и, если это 
предусмотрено, при дозаправке в~воздухе. Одной из самых сложных и~важных 
задач построения весовой модели является отражение изменяющихся в~полете  
МИХ топлива, находящегося 
в~топливных баках. Топливные баки современных ЛА
могут иметь довольно сложные геометрические формы. В~процессе выработки 
топлива все характеристики располагаемого запаса топлива меняются. 
Необходимо отслеживать эти изменения в~процессе произвольных допустимых 
эволюций траектории полета. Функции изменения МИХ в~зависимости от 
объема оставшегося топлива задаются тарировочными характеристиками баков. 
Для расчета тарировочных характеристик топливных баков при произвольных 
углах атаки, углах тангажа и~крена в~весовой модели наиболее удобно 
триангуляционное задание баков. В~тарировочной таблице вычисляется масса 
оставшегося топлива в~зависимости от уровня поверхности жидкости 
в~топливном баке. На рис.~3 приведен пример расчета тарировочной таблицы 
крыльевого топливного бака.



  Если МИХ топлива в~конкретном баке по 
мере его выработки определяются тарировочной характеристикой, то 
зависимость МИХ оставшегося топлива определяется последовательностью, 
в~которой осуществляется выработка из разных баков. Топливная система 
самолета состоит из нескольких баков~--- как внутренних, так и~размещенных 
на подвесках, а~также из системы трубопроводов, перекачивающих насосов и~управляющей автоматики. Основой управления расходом топлива является 
программа, определяющая порядок расходования топлива из разных баков. 
Переключение перекачки топлива между разными баками осуществляется для 
обеспечения центровки самолета в~заданных границах. Одним из критериев при 
разработке алгоритмов перекачки является число переключений и~обеспечение 
бесперебойной подачи топлива при любых допустимых параметрах траектории 
полета. Массово-инерционные характеристики топлива в~процессе тарировки 
баков задаются их разбиением плоскопараллельными сечениями на тонкие 
слои. Для каждого слоя указывается масса, координаты центра тяжести 
и~плоскостные моменты инерции. Программа выработки топлива пред\-став\-ля\-ет 
собой последовательность выработки слоев из разных баков в~соответствии 
с~диаграммой переключений. В~весовой модели может быть задано несколько 
вариантов программ расходования топлива. Разумеется, в~процессе выполнения 
полетного задания программа расходования топлива фиксирована. 
Предварительный расчет характеристик для разных вариантов порядка 
выработки топлива необходим для выбора наилучшего, удовле\-тво\-ря\-юще\-го 
всем ограничениям.
  
  \textbf{Целевая нагрузка} зависит от назначения самолета и~от конкретного 
полетного задания. Для пасса\-жирских самолетов целевая нагрузка~--- это 
пассажи\-ры с~багажом, для транспортных са\-мо\-летов~--- это коммерческие 
грузы, для военных~--- подвесное или размещаемое в~специальных \mbox{отсеках} 
вооружение. В~полете возможен сброс и~десантирование целевой нагрузки. 
Комплектация и~установка целевой нагрузки представляет собой довольно 
сложный процесс. Выбор состава грузов и~их размещение могут проходить 
в~несколько этапов. Сложность выбора обусловлена большим количеством 
типов перевозимой нагрузки, наличием большого числа вспомогательных 
специальных устройств закрепления грузов как во внутренних отсеках 
самолета, так и~на внешних подвесках. На рис.~4 приведена панель 
формирования различных расчетных вариантов целевой нагрузки самолета. 
Визуализация этого процесса существенно облегчает решение различных задач 
анализа допустимой нагрузки как на этапе проектирования самолета, так и~при 
эксплуатации во время подготовки полетных заданий.
  
  \begin{figure*} %fig5
\vspace*{1pt}
 \begin{center}
 \mbox{%
 \epsfxsize=162mm 
 \epsfbox{fle-5.eps}
 }
 \end{center}
\vspace*{-9pt}
\Caption{Область допустимых центровок}
\end{figure*}

  Для удобства выбора и~проведения расчетов множества вариантов загрузки 
самолета в~рамках весовой модели реализованы каталоги нагрузки~--- 
специального оборудования, коммерческой нагрузки, вооружения. В~этих 
каталогах ведутся клас\-си\-фи\-ка\-то\-ры, позволяющие в~громадных переч\-нях 
находить нужные позиции и~их характеристики. Кроме  
МИХ размещаемой нагрузки в~каталогах 
даются ссылки на их геометрические модели, задаются габариты, другие 
геометрические па\-ра\-мет\-ры. Эти данные нужны для визуализации размещения 
и~компоновки, для вычисления их МИХ. 
Как правило, существуют довольно жесткие ограничения на 
размещение нагрузки на борту, а~также на внешних узлах крепления. Эти 
ограничения должны указываться в~каталоге и~учитываться в~процессе 
формирования вариантов загрузки самолета. 
  
  Ограничения, которые проверяются при анализе различных вариантов 
снаряжения самолета, программы выработки топлива и~допустимых вариантов 
целевой нагрузки, задают область допустимых центровок самолета. 
  
  \textbf{Область допустимых центровок} является неотъемлемой частью 
весовой модели и~служит одной из основных весовых характеристик самолета, 
особенно важной и~контролируемой в~процессе его эксплуатации. На рис.~5 
проиллюстрированы ограничения, образующие область допустимых центровок, 
и~приведен график изменения центровки самолета при выработке топлива. 



  По оси абсцисс на этом графике откладывается центровка самолета, которая 
определяется как положение центра тяжести самолета на средней 
аэро\-ди\-на\-ми\-че\-ской хорде в~процентах от ее длины. По оси ординат 
откладывается текущая масса самолета с~учетом массы снаряжения, массы 
целевой нагрузки и~текущего запаса топлива. Точки излома на графиках 
центровки соответствуют моментам переключения подачи топлива с~одного 
бака на\linebreak другой, которые определяются программой выработки топлива или 
моментами сброса целевой нагрузки. Двойной график изменения центровки 
соответствует двум полетным конфигурациям~--- с~убранными 
и~выпущенными стойками шасси. Ограничения, которые обеспечивают 
устой\-чи\-вость и~управ\-ля\-емость полета, задаются предельными значениями 
центровки. Предельно передняя и~предельно задняя центровки на графике 
показаны вертикаль\-ными штриховыми линиями. Горизонтальные линии задают 
ограничения на взлетную и~посадочную массы. Ограничения максимальной 
взлетной и~посадочной массы при определенных условиях могут нарушаться, 
но эти нарушения допускаются в~исключительных условиях и~сказываются на 
ресурсных характеристиках самолета.\linebreak Превышение \textbf{предельных} 
значений взлетной и~посадочной массы не допускается. Наклонные штриховые 
линии на графике задают ограничения, связанные с~максимально допустимыми 
нагрузками на переднюю и~главную опоры шасси.  

\begin{figure*} %fig6
\vspace*{1pt}
 \begin{center}
 \mbox{%
 \epsfxsize=165mm 
 \epsfbox{fle-6.eps}
 }
 \end{center}
\vspace*{-9pt}
\Caption{Архитектура программной реализации исполнительных модулей АСВП}
\end{figure*} 

%\vspace*{-12pt}

\section{Программная реализация автоматизированной системы весового
проектирования}

  Представленная здесь струк\-тур\-но-па\-ра\-мет\-ри\-че\-ская весовая модель 
самолета позволяет решать широкий круг задач весового проектирования. 
Весовая модель составляет информационную основу,\linebreak на базе которой могут 
быть построены различные вычислительные программы и~пользовательские 
модули. Рассматриваемая в~данной работе АСВП построена по 
кли\-ент-сер\-вер\-ной архитектуре, где весовая модель служит единым хранилищем 
информации, базой данных системы. Непосредственно с~информацией, 
хранящейся в~этой базе данных, взаимодействуют различные вычислительные, 
расчетные программы~--- серверы, которые кроме расчетных функций 
обеспечивают информационную связь клиентских модулей с~весовой моделью 
самолета. Непосредственными пользователями клиентских модулей являются 
конструкторы и~проектировщики, решающие различные задачи весового 
проектирования.  Построена АСВП как многопользовательская интерактивная 
система. На рис.~6 представлена архитектура АСВП, ее основные программные 
и~информационные компоненты.




  Ниже перечислены основные функции программных модулей АСВП:
 \begin{description}
 \item[\,] 
Сервер ПУСТОЙ ЛА\;+\;Модуль расчета МИХ пус\-то\-го самолета:
\begin{itemize}
\item создание и~модификация дерева конструкции пустого самолета;
\item расчет МИХ пустого изделия, всех его сис\-тем, узлов, агрегатов и~деталей 
на любых уровнях дерева конструкции;
\item весовой анализ и~контроль текущего состояния проекта, выполнения 
лимитных ограничений по весу, осуществление выборок весовой информации 
по различным признакам~--- сис\-те\-мам, агрегатам, типу конструкции 
(си\-ло\-вая/не\-си\-ло\-вая),  материалу конструкции, подразделениям и~т.\,д.;
\item расчет распределения массы самолета по различным разбиениям на 
отсеки; эта информация используется для построения динамически подобных 
моделей и~при прочностных расчетах;
\item расчет МИХ при различных вариантах полетной конфигурации при 
убранных и~выпущенных стойках шасси, при отклонениях консолей крыла для 
самолетов с~из\-ме\-ня\-емой геометрией, при отклонении органов управления.
\end{itemize}
\begin{figure*} %fig7
\vspace*{1pt}
 \begin{center}
 \mbox{%
 \epsfxsize=155.86mm 
 \epsfbox{fle-7.eps}
 }
 \end{center}
\vspace*{-1pt}
\Caption{Проектный подход~--- технология ГП}
\vspace*{6pt}
\end{figure*}
 \item[\,]
Сервер НАГРУЗКА ЛА\;+\;Модуль расчета МИХ самолета с~переменной 
массой:
\begin{itemize}
\item создание и~модификация реестра допустимых вариантов нагрузки 
самолета;
\item расчеты МИХ снаряженного и~загруженного самолета для разных 
вариантов компоновки и~размещения на борту полезной нагрузки;
\item расчет изменения МИХ самолета в~полете при выработке топлива, 
дозаправке в~воздухе, сбросе нагрузки;
\item расчет МИХ самолета в~виде табличных зависимостей для различных 
вариантов снаряжения и~размещения нагрузки;
\item расчет МИХ самолета в~виде графических зависимостей от массы 
самолета и/или от массы топлива;
\item проверка выполнения установленных эксплуатационных ограничений по 
центровке, взлетной и~посадочной массе, нагрузке на опоры шасси для 
различных вариантов снаряжения и~размещения нагрузки; сигнализация 
в~случае нарушения ограничений, а~также для различных вариантов программ 
выработки топлива.
\end{itemize}

\pagebreak

 \item[\,]
Сервер КАТАЛОГ\;+\;Модуль ведения каталога элементов нагрузки:\\[-9pt]
\begin{itemize}
\item создание и~модификация каталога элементов целевой нагрузки самолета;\\[-9pt]
\item создание и~модификация базы данных вариантов размещения 
и~закрепления элементов нагрузки каталога на борту самолета или на подвесках;\\[-9pt]
\item создание и~модификация базы данных вспомогательных элементов 
конструкции установки элементов нагрузки.\\[-9pt]
\end{itemize}
 \item[\,]
Сервер ТОПЛИВО\;+\;Модуль расчета порядка выработки топлива:\\[-9pt]
\begin{itemize}
\item создание и~модификация базы данных различных вариантов программы 
выработки топлива;\\[-9pt]
\item расчет МИХ и~МЦД для различных вариантов переключения выработки 
топлива из внутренних, закладных и~подвесных баков;\\[-9pt]
\item расчет МИХ и~МЦД при различных программах заливки и~дозаправки 
топлива во внутренние, закладные и~подвесные баки.\\[-9pt]
\end{itemize}
 \item[\,]
Сервер БАКИ\;+\;Модуль расчета тарировки топливных баков:\\[-9pt]
\begin{itemize}
\item создание и~модификация базы данных гео\-мет\-рии топливных баков;\\[-9pt]
\item расчет тарировочных характеристик топливных баков при различных 
углах тангажа и~крена.\\[-9pt]
  \end{itemize}
  \end{description}
  
  Программная реализация АСВП велась с~использованием инструментального комплекса 
<<Генератор проектов>> (технология ГП)~\cite{5-fl}. Технология ГП 
обеспечивает возможность разработки приклад\-ных систем многоуровневой  
кли\-ент-сер\-вер\-ной архитектуры с~использованием реляционных и~сетевых 
баз данных со сложным пользовательским и~межпрограммным интерфейсом. 
Создание ин\-фор\-ма\-ци\-он\-но-вы\-чис\-ли\-тель\-ных сис\-тем в~рамках 
технологии ГП базируется на так называемом <<проектном подходе>>. Под 
проектом здесь понимается пакет документов (файлов), содержащий описание 
структуры проекта, описание логической структуры баз данных, спецификации 
пользовательского интерфейса, перечень команд и~сценарии работы 
пользователей, описание функций и~процедур обработки пользовательских 
запросов. Исходное описание проекта подается на вход <<Генератора 
проекта>>, который строит в~памяти модель проекта, осуществляет ее анализ 
на предмет корректности и~целостности, а затем на основании этой модели 
генерирует тексты программного кода для клиентских и~серверных компонент 
системы, а~так\-же ге\-нерирует утилиты, необходимые для сборки, инсталляции 
и~сопровождения системы. 

На рис.~7 показана общая архитектура 
программной конструкции, связанной с~применением технологии ГП.
  


  В приведенной цепочке разработчик прикладной информационной системы 
имеет дело только с~первым ее звеном~--- проектом системы. При этом он 
избавлен от необходимости иметь дело с~системным программным окружением 
вычислительной среды, в~которой должна функционировать разрабатываемая 
прикладная система. Все связи прикладных информационных процессов 
с~конкретной системной вычислительной средой привносит 
в~результирующую рабочую программу <<Генератор проектов>> на стадии 
анализа и~генерации итогового программного кода. Естественно, что при этом 
объем описания проекта оказывается существенно короче программного кода, 
который создается автоматически. Экономия трудозатрат разработчика 
оказывается существенной. В~частности, объем описания проекта АСВП на 
порядок меньше, чем объем сгенерированного программного кода. Даже если 
предположить, что написанный вручную программный код благодаря 
искусству программистов будет весьма экономным, то все равно трудоемкость 
разработки прикладных систем будет в~разы меньше. 

Но главное даже не 
в~числе строк программ, а~прежде всего в~экономии интеллектуальных затрат 
разработчиков прикладных систем и,~в~итоге, автоматически созданные 
программы более надежны и~свободны от нечаянных ошибок и~опечаток.\linebreak 
И~кроме того, разрабатываемые в~рамках технологии ГП прикладные системы 
обеспечивают-\linebreak\vspace*{-12pt}

\pagebreak

\noindent
ся эффективными средствами сопровождения, т.\,е.\linebreak достаточно 
простой процедурой внесения ис\-прав\-ле\-ний и~развития программ в~процессе их 
эксплу\-а\-тации. 

Прикладные программные комплексы в~рамках технологии ГП 
разрабатываются как автономные системы и~не требуют для своей работы 
специальной среды и~дорогостоящих программных продуктов (кроме 
использующихся систем управления базами данных
(СУБД) и~общесистемного обеспечения). Разрабатываемые 
в~рамках технологии ГП прикладные системы допускают масштабирование 
и~портирование на различные вычислительные платформы и~СУБД.
  
  \bigskip
  
  Как уже говорилось, система АСВП разрабатывалась в~течение ряда лет, 
многие ее компоненты и~версии были апробированы и~использовались 
в~реальном проектировании. 
%
Авторы выражают благодарность 
С.\,И.~Скобелеву, М.\,К.~Курьянскому, Д.\,Ю.~Стрель\-цу, П.\,В.~Плунскому 
и~К.\,Н.~Ерасову за плодотворные обсуждения проблем весового проектирования 
самолетов, за постановку многих задач и~за апробацию разработанных 
программ.

%\vspace*{-12pt}

{\small\frenchspacing
 {%\baselineskip=10.8pt
 \addcontentsline{toc}{section}{References}
 \begin{thebibliography}{9}
\bibitem{1-fl}
\Au{Шейнин В.\,М., Козловский~В.\,И.} Весовое проектирование и~эффективность 
пассажирских самолетов.~--- М.: Машиностроение, 1977.   Т.~1. 343~с.

%\columnbreak

\bibitem{2-fl}
\Au{Скобелев С.\,И., Широков~Н.\,И.} Весовой анализ и~контроль в~САПР ЛА~// Задачи 
и~методы автоматизированного проектирования.~--- М.: ВЦ РАН, 1991. С.~92--100.
\bibitem{3-fl}
\Au{Широков Н.\,И.} Автоматизированная система весовых расчетов в~САПР ЛА~// 
Автоматизация проектирования инженерных и~финансовых информационных систем 
средствами Генератора проектов~/ Отв. ред. Ю.\,А.~Флеров.~--- М.: ВЦ РАН, 
2010. С.~55--66.

\vspace*{6pt}

\bibitem{4-fl}
\Au{Вышинский Л.\,Л., Широков~Н.\,И.} Система автоматизации расчетов 
массово-инерционных характеристик ЛА с~переменной массой~// Развитие и~применение 
инструментального комплекса Генератор проектов~/ Отв. ред. Ю.\,А.~Флеров.~--- 
М.: ВЦ РАН, 2014. С.~20--31.
{\looseness=1

}

\vspace*{6pt}

\bibitem{5-fl}
\Au{Вышинский Л.\,Л., Гринев~И.\,Л., Флеров~Ю.\,А., Широков~А.\,Н., Широков~Н.\,И.} 
Генератор проектов~--- инструментальный комплекс для разработки  
<<кли\-ент-сер\-вер\-ных>> сис\-тем~// Информационные технологии и~вычислительные 
системы, 2003. №\,1-2. С.~6--25.
 \end{thebibliography}

 }
 }

\end{multicols}

\vspace*{-6pt}

\hfill{\small\textit{Поступила в~редакцию 24.05.17}}

\vspace*{8pt}

%\newpage

%\vspace*{-24pt}

\hrule

\vspace*{2pt}

\hrule

%\vspace*{8pt}


\def\tit{COMPUTER-AIDED SYSTEM OF~AIRCRAFT WEIGHT DESIGN}

\def\titkol{Computer-aided system of~aircraft weight design}

\def\aut{L.\,L.~Vyshinsky, Yu.\,A.~Flerov, and~N.\,I.~Shirokov}

\def\autkol{L.\,L.~Vyshinsky, Yu.\,A.~Flerov, and~N.\,I.~Shirokov}

\titel{\tit}{\aut}{\autkol}{\titkol}

\vspace*{-9pt}


\noindent
A.\,A.~Dorodnicyn Computing Centre, Federal Research Center ``Computer Science and 
Control'' of the Russian Academy of Sciences,  40~Vavilov Str., Moscow 119333, Russian 
Federation 



\def\leftfootline{\small{\textbf{\thepage}
\hfill INFORMATIKA I EE PRIMENENIYA~--- INFORMATICS AND
APPLICATIONS\ \ \ 2018\ \ \ volume~12\ \ \ issue\ 1}
}%
 \def\rightfootline{\small{INFORMATIKA I EE PRIMENENIYA~---
INFORMATICS AND APPLICATIONS\ \ \ 2018\ \ \ volume~12\ \ \ issue\ 1
\hfill \textbf{\thepage}}}

\vspace*{3pt}
   

\Abste{The article is devoted to the problems of computer-aided weight design of 
aircraft. Weight and mass-inertial parameters are one of the basic values that affect 
the performance characteristics of aircraft. The informational basis of the system is 
the weight model of the aircraft. The paper describes the structure of the weight 
model and its individual components. The program implementation of the system, 
which is executed within the framework of the client-server architecture, is shown. 
The automated system of weight design is implemented using the software tool 
complex ``Project Generator'' (GP technology), which was developed at the 
Computing Centre of the Russian Academy of Sciences. The creation of information 
and computing systems within the framework of the GP technology is based on the 
so-called ``project approach,'' when the formal description of the system 
automatically generates code for the client and server components of the system.}

\KWE{math modeling; design automation; aircraft; weight design; weighting model; 
design tree; project generator; code generation; client-server architecture}

  \DOI{10.14357/19922264180103} 

%\vspace*{-12pt}

%\Ack
%\noindent




%\vspace*{8pt}

  \begin{multicols}{2}

\renewcommand{\bibname}{\protect\rmfamily References}
%\renewcommand{\bibname}{\large\protect\rm References}

{\small\frenchspacing
 {%\baselineskip=10.8pt
 \addcontentsline{toc}{section}{References}
 \begin{thebibliography}{9} 
 
 %\vspace*{-6pt}
 
 \bibitem{1-fl-1}
\Au{Sheynin, V.\,M., and V.\,I.~Kozlovskiy}. 1977. \textit{Vesovoe 
proektirovanie i~effektivnost' passazhirskikh samoletov} [Weight design and 
efficiency of passenger aircraft]. Moscow: Mechanical Engineering. Vol.~1. 343~p.
\bibitem{2-fl-1}
\Aue{Skobelev, S.\,I., and N.\,I.~Shirokov.} 1991. Vesovoy analiz i~kontrol' v~SAPR 
LA [Weight analysis and control in CAD of aircraft]. \textit{Zadachi i~metody 
avtomatizirovannogo proektirovaniya} [Tasks and methods of computer-aided 
design]. Moscow: Computing Centre of the USSR Academy of Sciences.  
92--100.
\bibitem{3-fl-1}
\Aue{Shirokov, N.\,I.} 2010. Avtomatizirovannaya sistema vesovykh raschetov 
v~SAPR LA [Automated system weight calculations in CAD].  
\textit{Avtomatizatsiya proektirovaniya inzhenernykh i~finansovykh 
informatsionnykh system sredsvami Generatora proektov} [Computer 
aided  design of engineering and financial information systems by the means of the 
Project Generator]. Moscow: Computing Centre of RAS. 
55--66.
\bibitem{4-fl-1}
\Aue{Vyshinskiy, L.\,L., and N.\,I.~Shirokov.} 2014. Sistema avtomatizatsii 
raschetov massovo-inertsionnykh kharakteristik LA s~peremennoy massoy [CAD 
system of calculation  aircraft mass-inertial characteristics with variable mass].  
\textit{Razvitie i~primenenie instrumental'nogo kompleksa Generator proektov} 
[The development and application of a tool set Project Generator]. 
Moscow: Computing Centre of RAS. 20--31.
{\looseness=1

}

\bibitem{5-fl-1}
\Aue{Vyshinskiy, L.\,L., I.\,L.~Grinev, Yu.\,A.~Flerov, A.\,N.~Shirokov, and 
N.\,I.~Shirokov.} 2003. Generator proektov~--- instrumental'nyy kompleks dlya 
razrabotki ``klient--servernykh'' sistem [The project generator~--- tool complex for 
development of ``client--server'' systems]. 
\textit{Informatsionnye tekhnologii i~vychislitel'nye sistemy} [Information 
Technologies and Computer Systems] 1-2:6--25.

\end{thebibliography}

 }
 }

\end{multicols}

\vspace*{-6pt}

\hfill{\small\textit{Received May 24, 2017}}

%\vspace*{-10pt}

\Contr

\noindent
\textbf{Vyshinsky Leonid L.} (b.\ 1941)~--- Candidate of Sciences (PhD) in physics and 
mathematics, Head of Laboratory, A.\,A.~Dorodnicyn Computing 
Centre, Federal Research Center ``Computer Science and Control'' of the Russian 
Academy of Sciences, 40~Vavilov Str., Moscow 119333, Russian Federation; 
\mbox{Wysh@ccas.ru} 

\vspace*{3pt}

\noindent
\textbf{Flerov Yuri A.} (b.\ 1942)~--- Corresponding Member of the Russian 
Academy of Science, Doctor of Science in physics and mathematics, professor, 
Deputy Director, A.\,A.~Dorodnicyn Computing Centre, Federal Research Center 
``Computer Science and Control'' of the Russian Academy of Sciences, 40~Vavilov 
Str., Moscow 119333, Russian Federation; \mbox{fler@ccas.ru}

\vspace*{3pt}

\noindent
\textbf{Shirokov Nikolai I.} (b.\ 1963)~--- Candidate of Sciences (PhD) in physics and 
mathematics, senior scientist, A.\,A.~Dorodnicyn Computing Centre, Federal 
Research Center ``Computer Science and Control'' of the Russian Academy of 
Sciences, 40~Vavilov Str., Moscow 119333, Russian Federation; 
\mbox{Wysh@ccas.ru} 



\label{end\stat}


\renewcommand{\bibname}{\protect\rm Литература}    %1 
\def\stat{bosov+stef}

\def\tit{УПРАВЛЕНИЕ ВЫХОДОМ СТОХАСТИЧЕСКОЙ ДИФФЕРЕНЦИАЛЬНОЙ СИСТЕМЫ 
ПО~КВАДРАТИЧНОМУ КРИТЕРИЮ. I.~ОПТИМАЛЬНОЕ РЕШЕНИЕ МЕТОДОМ 
ДИНАМИЧЕСКОГО ПРОГРАММИРОВАНИЯ$^*$}

\def\titkol{Управление выходом стохастической дифференциальной системы 
по~квадратичному критерию. I}
%.~Оптимальное решение методом 
%динамического программирования}

\def\aut{А.\,В.~Босов$^1$, А.\,И.~Стефанович$^2$}

\def\autkol{А.\,В.~Босов, А.\,И.~Стефанович}

\titel{\tit}{\aut}{\autkol}{\titkol}

\index{Босов А.\,В.}
\index{Стефанович А.\,И.}
\index{Bosov A.\,V.}
\index{Stefanovich A.\,I.}




{\renewcommand{\thefootnote}{\fnsymbol{footnote}} \footnotetext[1]
{Работа выполнена при частичной поддержке РФФИ (проект 16-07-00677).}}


\renewcommand{\thefootnote}{\arabic{footnote}}
\footnotetext[1]{Институт проблем информатики Федерального исследовательского центра <<Информатика 
и~управление>> Российской академии наук, \mbox{AVBosov@ipiran.ru}}
\footnotetext[2]{Институт проблем информатики Федерального исследовательского центра <<Информатика 
и~управление>> Российской академии наук, \mbox{AStefanovich@frccsc.ru}}

%\vspace*{8pt}



  
  \Abst{Решается задача оптимального управления для диффузионного процесса 
Ито и~линейного управ\-ля\-емо\-го выхода. Рассматриваемая постановка близка 
к~классической ли\-ней\-но-квад\-ра\-тич\-ной гауссовской задаче управления 
(linear-quadratic Gaussian (LQG) control). Отличия состоят в~том, что состояние описывается нелинейным 
дифференциальным уравнение Ито $dy_t\hm= A_t(y_t) \,dt\hm+ \Sigma_t(y_t)\,dv_t$ 
и~не зависит от управ\-ле\-ния~$u_t$, оптимизации подлежит управ\-ля\-емый 
линейный выход $dz_t\hm= a_t y_t\,dt\hm+ b_t z_t \,dt\hm+ c_t u_t \,dt\hm+ \sigma_t\, 
dw_t$. Дополнительные обобщения внесены в~квад\-ра\-тич\-ный критерий качества 
с~целью воз\-мож\-ности постановки таких задач, как отслеживание выходом 
состояния или управ\-ле\-ни\-ем~--- линейной комбинации состояния и~выхода. Для 
решения используется метод динамического программирования. Функцию 
Беллмана позволяет найти предположение о~ее структуре вида $V_t(y,z)\hm= 
\alpha_t z^2\hm+ \beta_t(y)z \hm+\gamma_t(y)$. Решение дают три 
дифференциальных уравнения для коэффициентов~$\alpha_t$, $\beta_t(y)$ 
и~$\gamma_t(y)$. Эти уравнения со\-став\-ля\-ют оптимальное решение 
рас\-смат\-ри\-ва\-емой задачи.}
  
  \KW{стохастическое дифференциальное уравнение; оптимальное управ\-ле\-ние; 
динамическое программирование; функция Беллмана; уравнение Риккати; 
линейные уравнения параболического типа}

\DOI{10.14357/19922264180314}
  
%\vspace*{4pt}


\vskip 10pt plus 9pt minus 6pt

\thispagestyle{headings}

\begin{multicols}{2}

\label{st\stat}

\section{Введение}

     Ключевые результаты в~области оптимизации стохастических 
динамических систем, со\-став\-ля\-ющие классическую теорию управления, 
получены более~40~лет назад (такова работа~[1] в~отношении задачи 
управ\-ле\-ния ли\-ней\-но-гаус\-сов\-ски\-ми стохастическими сис\-те\-ма\-ми по 
квад\-ра\-тич\-но\-му критерию). К~классической тео\-рии следует относить 
линейные модели стохастических сис\-тем и~квадратичный критерий качества. 
Это исходный базис, на котором основано множество успешно 
исследованных и~решенных задач стохастического управ\-ле\-ния 
и~оптимизации. 

Дальнейшее развитие~--- это новые модели и~критерии, но 
прежде всего это новые методы: от тео\-рии линейных регуляторов, метода 
динамического программирования и~принципа максимума к~адаптивному 
и~минимаксному подходу, импульсному управ\-ле\-нию и~т.\,д. Множество 
инноваций как в~час\-ти моделей, так и~в~час\-ти математического аппарата, 
имевших мес\-то в~по\-сле\-ду\-ющие годы, существенно обогатили тео\-рию 
управ\-ле\-ния. Но и~до настоящего времени линейные модели и~квадратичный 
критерий, несмотря на всю справедливую критику в~отношении их 
аде\-кват\-ности и~гиб\-кости, сохраняют исследовательский интерес и~находят 
современные области приложения.
     
     Не претендуя на сколь\-ко-ни\-будь полное обосно\-ва\-ние последнего 
тезиса, приведем несколько примеров, показавшихся наиболее ин\-те\-рес\-ными. 

Так, в~[2] решается ли\-ней\-но-квад\-ра\-тич\-ная за\-да\-ча в~игровой 
постановке с~запаздыванием. В~близ\-кой по модели работе~[3] задача 
управ\-ле\-ния ставится в~терминах $H_\infty$-ро\-баст\-ности. Точнее \mbox{называть} 
эту тематику $H_2/H_\infty$-управ\-ле\-ни\-ем, и~работ по этой теме очень 
много. Аккуратности ради следует уточнить, что под линейными 
понимаются модели с~мультипликативными по состоянию воз\-му\-ще\-ниями. 

Совсем другой класс моделей, особо популярных в~по\-след\-ние годы, 
составляют скачкообразные процессы. Например, линейные уравнения 
в~сочетании с~пуассоновскими скачками в~[4] используются в~моделях, 
описывающих различные показатели функционирования сетевых протоколов 
передачи данных транспортного уровня. Телекоммуникации представляют 
в~последние годы самый популярный прикладной материал для 
исследований, работ по этой проб\-ле\-ма\-ти\-ке множество, математические 
техники привлекаются самые разные и~самые современные, но и~линейным 
моделям место находится. Еще один любопытный пример исследования 
скачкообразного процесса и~оптимизации на основе квад\-ра\-тич\-но\-го критерия 
можно найти в~[5] применительно к~задаче инвестирования на финансовом 
рынке. Наконец, упомянем еще работу~[6], подводящую итог исследований 
в~отношении классической детерминированной  
ли\-ней\-но-квад\-ра\-тич\-ной задачи с~использованием техники матричных 
неравенств.
     
     В данной работе также эксплуатируются привлекательные свойства 
линейных моделей и~квад\-ра\-тич\-но\-го критерия, причем в~стохастической 
постановке. На\-прав\-ле\-ни\-ем для обобщения \mbox{выбрана} модель динамики 
сис\-те\-мы: основные усилия на\-прав\-ле\-ны на то, чтобы сделать ее нелинейной. 
Кроме того, пред\-став\-лен\-ная постановка может рас\-смат\-ри\-вать\-ся и~как 
обобщение ранее решенной задачи в~дискретном времени~[7, 8] на время 
непрерывное. В~упомянутых работах помимо собственно модельной 
постановки важна еще и~привлекаемая прикладная об\-ласть~--- 
функционирование сложных программных сис\-тем. Результатов, 
ориентированных непосредственно на такие приложения, к~настоящему 
времени пренебрежимо мало, поэтому~[7, 8]~--- это еще и~прикладное 
обоснование рас\-смат\-ри\-ва\-емой далее задачи.
     
     Оптимизируемая динамическая сис\-те\-ма описывается двумя 
уравнениями. Состояние задается нелинейным стохастическим 
дифференциальным уравнением Ито, не содержащим управ\-ля\-емой 
переменной. Возмущение здесь описывается стандартным винеровским 
процессом, накладываются простые условия существования 
и~един\-ст\-вен\-ности решения. Поскольку состояние не управ\-ля\-ет\-ся, то уместно 
его интерпретировать как слож\-ное внешнее возмущение. Вторая 
переменная~--- управ\-ля\-емый выход~--- задается линейным стохастическим 
дифференциальным уравнением. Цель оптимизации выхода формируется 
квадратичным критерием общего вида. Формальная постановка задачи 
приведена в~сле\-ду\-ющем разделе.
     
     Для решения задачи используется метод динамического 
программирования, решается уравнение Беллмана~[9]. Соответственно, 
в~результате получаются аналитические выражения и~для оптимального 
управ\-ле\-ния, и~для значения функционала качества. Технически 
традиционный, стандартный подход к~задаче обременен, пожалуй, 
единственной проблемой~--- поиском верного пред\-став\-ле\-ния структуры 
функции Беллмана. Справиться с~этой проблемой в~большей степени удается 
за счет результата, полученного при решении дискретного по времени 
аналога рассматриваемой постановки~\cite{8-bos}. Конечные соотношения 
для оптимального решения, как и~во всех подобных задачах, включая 
классическую ли\-ней\-но-квад\-ра\-тич\-ную, содержат решения 
определенных дифференциальных уравнений (обыкновенных и~в~частных 
производных). Вывод этих уравнений и~со\-став\-ля\-ет содержание первой час\-ти 
данной работы. Во второй части будет обсуждаться их приближенное 
чис\-лен\-ное решение и~компьютерные эксперименты.
     
     Кратко обозначим основные положения, при\-вле\-ка\-емые далее 
к~решению задачи, следуя в~основном обозначениям 
и~терминологии~\cite{9-bos}, а~именно: будем рассматривать задачу 
оптимального управления в~стохастической динамической сис\-те\-ме по полной 
информации, применяя метод динамического программирования. В~качестве 
целевого функционала, опре\-де\-ля\-юще\-го качество управ\-ле\-ния $U_0^T\hm= \{ 
u_t,\ 0\leq t\leq T\}$, выступает
     \begin{equation}
     J\left(U_0^T\right)={\sf E}\left\{ \int\limits_0^T L_t \left(x_t, u_t\right)\,dt+ 
l\left(x_T\right)\right\}\,.
     \label{e1-bos}
     \end{equation}
Здесь ${\sf E}\{\cdot\}$~--- оператор математического ожидания; $x_t$~--- 
случайный процесс, описываемый стохастическим дифференциальным 
уравнением Ито
     \begin{equation}
     dx_t=m_t\left( x_t, u_t\right) dt+ \sigma_t\left( x_t\right)dW_t\,,\enskip 
x_0=X\,,
     \label{e2-bos}
     \end{equation}
где $W_t$~--- стандартный винеровский процесс подходящей раз\-мер\-ности; 
$X$~--- случайный вектор.

     $U_0^T$ будем выбирать из класса допустимых неупреждающих (по 
отношению к~$W_t$) управлений~\cite{9-bos}. Соответственно, 
относительно функций сноса и~диффузии~$m_t$ и~$\sigma_t$  
в~(\ref{e2-bos}) будем предполагать выполненными ка\-кие-ли\-бо условия 
существования сильного решения для заданного до\-пус\-ти\-мо\-го управ\-ле\-ния. 
Например, для управ\-ле\-ния с~обратной связью $u_t\hm= u_t(x_t)$ будем 
считать, что $m_t(x,u_t(x))$ и~$\sigma_t(x)$ удовлетворяют условию 
линейного рос\-та и~локальному условию Липшица по~$x$ равномерно 
по~$t$ (т.\,е.\ условиям Ито).
     
     Для поиска оптимального управления, минимизирующего $J(U_0^T)$, 
рас\-смат\-ри\-ва\-ет\-ся функция Беллмана
     \begin{equation}
     V_t(x)=\left.\mathop{\mathrm{inf}}\limits_{U_t^T} {\sf E} \left\{ \int\limits_t^T 
L_t \left( x_t, u_t\right)\,dt+l\left( x_T\right) \right\vert \mathcal{F}_t^x\right\}\,,
     \label{e3-bos}
     \end{equation}
где $\mathcal{F}_t^x$~--- $\sigma$-ал\-геб\-ра, по\-рож\-ден\-ная~$x_\tau$, 
$0\hm\leq \tau\hm\leq t$, ${\sf E}\{\cdot\vert \mathcal{F}\}$~--- оператор условного 
математического ожидания относительно~$\mathcal{F}$. Соответственно, 
в~качестве достаточного условия оп\-ти\-маль\-ности воспользуемся уравнением 
динамического программирования
\begin{multline}
\fr{\partial V_t(x)}{\partial t} +\fr{1}{2}\sum\limits^n_{i,j=1} \sigma^2_{t_{ij}}
\fr{\partial^2 V_t(x)}{\partial x_i \partial x_j}+{}\\
{}+\min\limits_u\left[  
\sum\limits^n_{i=1} m_{t_i} \fr{\partial V_t(x)}{\partial x_i} + L_t(x,u)\right] 
=0\,,\\
V_T(x)=l(x)\,,
\label{e4-bos}
\end{multline}
где $m_{t_i}$~--- $i$-й элемент век\-тор-функ\-ции~$m_t(x,u)$; 
$\sigma^2_{t_{ij}} \hm= \sum\nolimits^m_{k=1} 
\sigma_{t_{ik}}\sigma_{t_{ki}}$, $\sigma_{t_{ij}}$~--- $i$-й по строке, $j$-й 
по столб\-цу элемент мат\-рич\-ной функции~$\sigma_t(x)$; $n$ и~$m$~--- 
размерности~$x_t$ и~$W_t$ соответственно.

     Традиционно в~рамках применения метода динамического 
программирования будем предполагать, что функции~$L_t$, $l$, $m_t$ 
и~$\sigma_t$ обеспечивают существование хотя бы одного решения 
уравнения~(\ref{e4-bos}), а~следовательно, и~оптимального 
управления~$u_t^*$, $0\hm\leq t\hm\leq T$, до\-став\-ля\-юще\-го минимум 
целевому функционалу~(\ref{e1-bos}). Задача оптимизации далее получается 
путем указания конкретных выражений для~$L_t$, $l$, $m_t$ и~$\sigma_t$.

\section{Постановка задачи управления выходом}

     Рассматриваемые далее случайные функции будут предполагаться 
скалярными. Такое упрощение позволит разгрузить выкладки и~итоговые 
выражения от не самых существенных деталей.
     
     Рассмотрим стохастическую дифференциальную сис\-те\-му, со\-сто\-яние 
которой представляет диффузи\-он\-ный процесс~$y_t$, описываемый 
нелинейным стохастическим дифференциальным уравнением Ито
     \begin{equation}
     dy_t=A_t\left( y_t\right) dt +\Sigma_t \left( y_t\right) dv_t\,,\enskip 
y_0=Y\,,
     \label{e5-bos}
     \end{equation}
где $v_t$~--- стандартный (одномерный) винеровский процесс; $Y$~--- 
случайная величина с~конечным вторым моментом; функции~$A_t$ 
и~$\Sigma_t$ удовлетворяют условиям Ито:
\begin{equation*}
\left\vert A_t(y)\right\vert +\left\vert \Sigma_t(y)\right\vert \leq C(1+\vert y\vert )\ 
\mbox{для\ всех } 0\leq t\leq T\,;
\end{equation*}

\vspace*{-12pt}

\noindent
\begin{multline*}
\hspace*{-2.10051pt}\left\vert A_t\left(y_1\right) -A_t \left( y_2\right) \right\vert +\left\vert 
\Sigma_t\left( y_1\right) -\Sigma_t \left(y_2\right)\right\vert \leq
C\left\vert y_1-y_2\right\vert\\
 \mbox{для\ всех\ } 0\leq t\leq T\ \mbox{и } 
y_1,y_2\in \mathbb{R}^1\,,
\end{multline*}
обеспечивающим существование единственного сильного (потраекторного) 
решения уравнения.
     
     Будем считать, что~$y_t$ описывает состояние некоторой 
динамической системы. Соответственно, поведение этой сис\-те\-мы опишем 
выходом, линейно связанным с~со\-сто\-янием:
     \begin{equation}
     dz_t=a_t y_t \,dt+ b_t z_t \,dt+ c_t u_t \,dt+\sigma_t \,dw_t\,,\enskip
     z_0=Z\,.
     \label{e6-bos}
     \end{equation}
Здесь $w_t$~--- не зависящий от~$v_t$, $Y$ и~$Z$ стандартный (одномерный) 
винеровский процесс; $Z$~--- случайная величина с~конечным вторым 
моментом; $u_t$~--- допустимое неупреждающее управ\-ле\-ние, качество 
которого определяется целевым функционалом следующего вида:
\begin{multline}
\!\hspace*{-3.98538pt}J\left( U_0^T\right) ={\sf E}\left\{ \int\limits_0^T \!\left( S_t\left( s_ty_t-g_t z_t -h_t 
u_t\right)^2 +G_t z_t^2+{}\right.\right.\\
\left.\left.{}+ H_t u_t^2
\vphantom{S_t\left( s_ty_t-g_t z_t -h_t 
u_t\right)^2}
\right) dt+S_T\left( s_T y_T -g_T 
z_T\right)^2+G_T z_T^2
\vphantom{\int\limits_0^T}\right\}\,,
\label{e7-bos}
\end{multline}
где $S_t$, $G_t$ и~$H_t$~--- неотрицательные функции\linebreak
$0\hm\leq t\hm\leq T$. 
Такой критерий отражает физический смысл задачи распределения ресурсов 
со\-глас\-но аналогичной~(\ref{e5-bos})--(\ref{e7-bos}) задаче для дис\-крет\-но\-го 
времени, рас\-смот\-рен\-ной в~\cite{7-bos}. В~част\-ности,  
функци\-онал~(\ref{e7-bos}) поз\-во\-ля\-ет ставить задачи отслеживания
 выходом 
со\-сто\-яния сис\-те\-мы, используя сла\-га\-емое $(y_t\hm- z_t)^2$, или 
управлением~--- линейной комбинации со\-сто\-яния и~выхода, сла\-га\-емое типа\linebreak 
$(y_t\hm+ z_t\hm- u_t)^2$. Поскольку задача формулируется 
в~предположении наличия пол\-ной информации о~со\-сто\-янии~$y_t$ 
и~выходе~$z_t$ (соответствующую $\sigma$-ал\-геб\-ру 
обозначим~$\mathcal{F}_t^{y,z}$), то допустимое управ\-ле\-ние ищется 
в~классе~$\mathcal{F}_t^{y,z}$-из\-ме\-ри\-мых неупреждающих функций 
(и,~как будет показано далее, оказывается управ\-ле\-ни\-ем с~обратной связью).

     Функции~$a_t$, $b_t$, $c_t$ и~$\sigma_t$ будем предполагать 
ограниченными: $\vert a_t\vert \hm+ \vert b_t\vert \hm+\vert c_t\vert \hm+ \vert 
\sigma_t \vert \hm\leq C$ для всех $0\hm\leq t\hm\leq T$, процесс  
управления~--- допустимым не\-упреж\-да\-ющим~\cite{9-bos}, обеспечивая, 
таким образом, существование сильного решения урав\-не\-ния~(\ref{e6-bos}) 
для любого допустимого управ\-ления.
     
     Задачу составляет поиск~$u_t^*$~--- допустимого управ\-ле\-ния, 
доставляющего минимум квад\-ра\-тич\-но\-му функционалу~$J(U_0^T)$.
      
     Поставленная задача очевидным образом формулируется в~терминах 
введенных выше в~(\ref{e1-bos})--(\ref{e3-bos}) обозначений, а~именно: 
     требуется обозначить
     \begin{gather*}
      x_t=\begin{pmatrix}
     y_t\\ z_t\end{pmatrix};\quad  m_t(x_t, u_t)=\begin{pmatrix}
     A_t(y_t)\\ a_t y_t +b_t z_t +c_t u_t\end{pmatrix};\\
     \sigma_t(x_t)= \begin{pmatrix}
     \Sigma_t(y_t)& 0\\
     0& \sigma_t\end{pmatrix};\quad W_t=\begin{pmatrix}
     v_t \\ w_t\end{pmatrix}
     %     \label{e8-bos}
     \end{gather*}
для записи уравнения со\-сто\-яния типа~(\ref{e2-bos}) и
\begin{align*}
L_t(x,u)&= L_t(y,z,u) ={}\\
&\hspace*{3mm}{}=S_t\left( s_t y-g_t z -h_t u\right)^2 +G_t z^2 +H_t  u^2\,;\\
l(x)&= l(y,z) =S_T \left( S_T y-g_T z\right)^2 +G_T z^2
%\label{e9-bos}
\end{align*}
для записи целевого функционала в~виде~(\ref{e1-bos}).

     Функция Беллмана~(\ref{e3-bos}) принимает вид 
     $V_t(x)\hm= V_t(y,z)$. Для записи со\-от\-вет\-ст\-ву\-юще\-го~(\ref{e4-bos}) 
уравнения Беллмана для~$V_t(y,z)$ заметим, что
     $$
     \left( \sigma^2_{t_{ij}}\right)_{i,j=1,2}= \begin{pmatrix}
     \Sigma_t^2(y) & 0\\
     0 & \sigma_t^2\end{pmatrix}\,.
     $$
     
     С~учетом перечисленных обозначений урав\-не\-ние динамического 
программирования~(\ref{e4-bos}) принимает вид:
     \begin{multline}
     \fr{\partial V_t(y,z)}{\partial t} +\fr{1}{2}\left( \Sigma_t^2(y) \fr{\partial^2 
V_t(y,z)} {\partial y^2}+\sigma_t^2\fr{\partial^2 V_t(y,z)} {\partial 
z^2}\right)+{}\\
    {}+\min\limits_u\! \left[ A_t(y) \fr{\partial V_t(y,z)}{\partial y}+\left( a_t 
y+b_t z+c_t u\right) \fr{\partial V_t(y,z)}{\partial z} +{}\right.\hspace*{-3pt}\\
\left.{}+ S_t\left( s_t y-g_t z-h_t 
u\right)^2+G_t z^2+H_t u^2
     \vphantom{\fr{\partial V_t(y,z)}{\partial y}}\right] =0\,,\\
     V_T(y,z)=S_T\left( s_T y-g_T z\right)^2+G_T z^2\,.
     \label{e10-bos}
     \end{multline}
     Это и~есть то самое уравнение, которое требуется решить: 
существование решения данного урав\-не\-ния суть достаточное условие 
оптимальности; оптимальное управ\-ле\-ние при этом~--- точ\-ка минимума 
со\-от\-вет\-ст\-ву\-юще\-го сла\-га\-емого.
     
\section{Динамическое программирование и~оптимальное 
управление}

     В рассматриваемой постановке линейность\linebreak выхода и~квадратичность 
критерия дают те же преимущества, что и~в~классической  
ли\-ней\-но-квад\-ра\-тич\-ной задаче управ\-ле\-ния~\cite{1-bos}, а~именно: 
позволяют сразу определить вид оптимального управ\-ле\-ния и~фактические 
условия его существования. Действительно, со\-хра\-няя в~(\ref{e10-bos}) под 
знаком $\min\nolimits_u$ только члены, зависящие от~$u$, получаем
     \begin{multline*}
     \fr{\partial V_t(y,z)}{\partial t} +\fr{1}{2}\left( \Sigma_t^2(y) \fr{\partial^2 
V_t(y,z)} {\partial y^2}+\sigma_t^2\fr{\partial^2 V_t(y,z)} {\partial 
z^2}\right)+{}\\
     {}+A_t(y)\fr{\partial V_t(y,z)}{\partial y}+\left( a_t y+b_t z\right) 
\fr{\partial V_t(y,z)}{\partial z}+{}\\
{}+S_t\left( s_t y-g_t z\right)^2 +G_t z^2+{}
\end{multline*}

\noindent
\begin{multline*}
     {}+\min\limits_u \left[ \left( c_t \fr{\partial V_t(y,z)}{\partial z}-2S_t \left( 
s_t y-g_t z\right) h_t\right)u +{}\right.\\
\left.{}+\left( S_t h_t^2+H_t\right) u^2
\vphantom{\fr{\partial V_t(y,z)}{\partial z}}
\right]=0\,,
     %\label{e11-bos}
     \end{multline*}
откуда в~предположении $S_t h_t^2\hm+ H_t\hm>0$ следует, что существует 
оптимальное управ\-ле\-ние, которое определяется равенством
\begin{multline}
u_t^* = u_t^*(y,z)=-\fr{1}{2}\left( S_t h_t^2 +H_t\right)^{-1} \left( c_t 
\fr{\partial V_t(y,z)}{\partial z}-{}\right.\\
\left.{}-2S_t\left( s_t y-g_t z\right) h_t
\vphantom{\fr{\partial V_t(y,z)}{\partial z}}
\right)
\label{e12-bos}
\end{multline}
и доставляет минимум соответствующему сла\-га\-емо\-му в~урав\-не\-нии Беллмана, 
равный
$-\left( S_t h_t^2\hm+\right.$\linebreak
$\left.{}+H_t\right)^{-1} \left( c_t 
{\partial V_t(y,z)}/{\partial 
z}\hm-2S_t\left( s_t y \hm-g_t z\right) h_t \right)^2/4.
$ 
     
     Отметим, что, как и~в~классической ли\-ней\-но-квад\-ра\-тич\-ной 
задаче, управ\-ле\-ние из класса до\-пус\-ти\-мых не\-упреж\-да\-ющих получилось 
управ\-ле\-ни\-ем с~обратной связью.
     
     Таким образом, функция Беллмана описывается сле\-ду\-ющим 
дифференциальным уравнением:
     \begin{multline}
     \fr{\partial V_t(y,z)}{\partial t} +\fr{1}{2}\left( \Sigma_t^2(y) \fr{\partial^2 
V_t(y,z)} {\partial y^2}+\sigma_t^2\fr{\partial^2 V_t(y,z)} {\partial 
z^2}\right)+{}\\
     {}+ A_t(y) \fr{\partial V_t(y,z)}{\partial y}+\left( a_t y+b_t z\right) 
\fr{\partial V_t(y,z)}{\partial z}+{}\\
{}+ S_t \left( s_t y- g_t z\right)^2 +G_t z^2-
 \fr{1}{4}\left( S_t h_t^2+H_t\right)^{-1}\times{}\\
 {}\times \left( c_t \fr{\partial V_t(y,z)} 
{\partial z}-2S_t\left( s_t y -g_t z\right) h_t \right)^2=0\,.
     \label{e13-bos}
     \end{multline}
     
     Возводя в~квадрат по\-след\-нее сла\-га\-емое в~(\ref{e13-bos}), перепишем 
его в~виде:
     \begin{multline}
     \fr{\partial V_t(y,z)}{\partial t} +\fr{1}{2}\left( \Sigma_t^2(y) \fr{\partial^2 
V_t(y,z)} {\partial y^2}+\sigma_t^2\fr{\partial^2 V_t(y,z)} {\partial 
z^2}\!\right)+{}\\
{}+A_t(y) \fr{\partial V_t(y,z)}{\partial y}
+ \left( 
\vphantom{\left( S_t h_t^2 +H_t\right)^{-1}}
a_t y+b_t z+{}\right.\\
\left.{}+\left( S_t h_t^2 +H_t\right)^{-1}
 c_t S_t \left( s_t y-g_t z\right) h_t
\right) 
     \fr{\partial V_t(y,z)}{\partial z}+{}\\
     {}+\left( S_t-\left( S_t h_t^2 +H_t\right)^{-1} S_t^2 h_t^2\right)\left( s_t y -
g_t z\right)^2+{}\\
     \!\!{}+
     G_t z^2 -\fr{1}{4}\left( S_t h_t^2+H_t\right)^{-1}\! c_t^2
     \left(\fr{\partial V_t(y,z)}{\partial z}\right)^{\!2}=0\,.\!\!
     \label{e14-bos}
     \end{multline}
     
     Рассматривая полученное уравнение, заметим, что его решение может 
быть пред\-став\-ле\-но в~виде:
   \begin{equation}
     V_t(y,z)= \alpha_t z^2+\beta_t(y) z +\gamma_t(y)\,,
     \label{e15-bos}
     \end{equation}
т.\,е.\ будем искать решение при дополнительном предположении 
о~квад\-ра\-тич\-ности функции Белл\-ма\-на по переменной~$z$, и~сведем, таким 
образом, поиск оптимального решения к~уравнениям относительно функций 
$\alpha_t$, $\beta_t(y)$ и~$\gamma_t(y)$. Отметим сразу, что явный вид 
функции~$\gamma_t(y)$ для реализации оптимального управ\-ле\-ния не 
требуется, однако далее будет предложен вариант вы\-чис\-ле\-ния и~этой 
функции, что пред\-став\-ля\-ет\-ся небесполезным, поскольку позволит выполнять 
расчет минимума целевого функционала. Источником для 
предложения~(\ref{e15-bos}) является уже упоминавшаяся аналогичная 
задача для случая дис\-крет\-но\-го времени~\cite{7-bos, 8-bos}. В~той задаче 
выражение для функции Беллмана получается формально без 
дополнительных усилий. При этом форма~(\ref{e15-bos}) обнаруживается 
как свойство оптимального решения. В~рассматриваемом случае 
непрерывного времени~(\ref{e15-bos}) постулируется, а~пра\-виль\-ность 
постулата под\-тверж\-да\-ет\-ся далее ре\-зуль\-ти\-ру\-ющи\-ми уравнениями 
для~$\alpha_t$, $\beta_t(y)$ и~$\gamma_t(y)$ Кроме того, данное 
предположение пред\-став\-ля\-ет\-ся вы\-те\-ка\-ющим из линейной структуры задачи 
в~отношении переменной~$z$, в~част\-ности, тем фактом, что такой вид 
функции Беллмана обеспечивает линейность оптимального 
управ\-ле\-ния~(\ref{e12-bos}) по~$z$.

     Граничное условие при выбранном предположении~(\ref{e15-bos}) 
принимает вид:

\noindent
     \begin{multline*}
     V_T(y,z)= S_T\left( s_T y- g_T z\right)^2+G_T z^2 ={}\\[-0.5pt]
     {}=\alpha_T z^2 
+\beta_T(y) z +\gamma_T(y)\,,
    \end{multline*}
т.\,е.

\noindent
\begin{align*}
\alpha_T&= S_T g_T^2 +G_T\,;\\[-0.5pt]
\beta_T(y)&=-2S_T s_T g_T y\,;\\[-0.5pt]
\gamma_T(y)&=S_T s_T^2 y^2\,.
%\label{e16-bos}
\end{align*}
          При этом само оптимальное управ\-ле\-ние, определенное 
выражением~(\ref{e12-bos}), оказывается управ\-ле\-ни\-ем с~обратной связью 
по~$y_t$ и~$z_t$:

\noindent
     \begin{multline}
     u_t^*=u_t^*(y,z) ={}\\[-0.5pt]
     {}=
     -\fr{1}{2}\left( S_t h_t^2 +H_t\right)^{-1}
     \left( c_t \left( 2\alpha_t z +\beta_t(y)\right) +{}\right.\\[-0.5pt]
    \left. {}+2S_t\left( s_t y-g_t z\right) 
h_t\right)\,.
     \label{e17-bos}
     \end{multline}
          Подставляем $V_t(y,z)\hm= \alpha_t z^2 \hm+ \beta_t(y) 
z\hm+\gamma_t(y)$ в~(\ref{e14-bos}):

\noindent
     \begin{multline*}
     \fr{\partial \alpha_t}{\partial t}\, z^2 +
     \fr{\partial \beta_t(y)}{\partial t}\,z +
     \fr{\partial \gamma_t(y)}{\partial t}+{}\\[-0.5pt]
     {}+\fr{1}{2}\left( \Sigma_t^2(y) \left( 
\fr{\partial^2\beta_t(y)}{\partial y^2}\,z +\fr{\partial^2 \gamma_t(y)}{\partial 
y^2}\right) +2\sigma_t^2\alpha_t\right)+{}\\[-0.5pt]
 {}+A_t(y)\left(\fr{\partial \beta_t(y)}{\partial y}\,z + \fr{\partial 
\gamma_t(y)}{\partial y}\right) +{}\\[-0.5pt]
\hspace*{-0.22987pt}{}+\left( a_t y+b_t z+\left( S_t h_t^2 +H_t\right)^{-1} c_t S_t \left( s_t y-
g_t z\right) h_t\right)\times{}
\end{multline*}

\noindent
\begin{multline*}
         {}\times \left( 2\alpha_t z+\beta_t(y)\right)+{}\\
     {}+\left( S_t-\left( S_t h_t^2 +H_t\right)^{-1} S_t^2 h_t^2\right)\left( s_t y-
g_t z\right)^2+{}\\
     {}+ G_t z^2 -\fr{1}{4}\left( S_t h_t^2 +H_t\right)^{-1} c_t^2 \left( 
2\alpha_t z+\beta_t(y)\right)^2=0\,.
     \end{multline*}
          Далее выделяем слагаемые при~$z^2$, $z$ и~$z^0$
          
          \noindent
     \begin{multline*}
     \fr{\partial \alpha_t}{\partial t}\, z^2 +\fr{\partial \beta_t(y)}{\partial t}\,z +
     \fr{\partial \gamma_t(y)}{\partial 
t}+\fr{1}{2}\,\Sigma_t^2(y)\fr{\partial^2\beta_t(y)}{\partial y^2}\,z+ {}\\
{}+
\fr{1}{2}\,\Sigma_t^2(y)\fr{\partial^2\gamma_t(y)}{\partial 
y^2}+\sigma_t^2\alpha_t+A_t(y)\fr{\partial \beta_t(y)}{\partial y}\,z +{}\\
{}+A_t(y) \fr{\partial 
\gamma_t(y)}{\partial y}+{}\\
{}+ 2\alpha_t \left( b_t -\left( S_t h_t^2+H_t\right)^{-1} c_t 
S_t h_t g_t \right)z^2+{}\\
     {}+
     \left( 2\alpha_t\left( \alpha_t+\left( S_t h_t^2+H_t\right)^{-1} c_t S_t h_t 
s_t\right)y +{}\right.\\
\left.{}+\beta_t(y) \left( b_t-\left( S_t h_t^2+H_t\right)^{-1} c_t S_t h_t 
g_t\right) \right) z+{}\\
     {}+\beta_t(y)\left( a_t +\left( S_t h_t^2+H_t\right)^{-1} c_t S_t h_t s_t\right) 
y+{}\\
{}+ \left( S_t -\left( S_t h_t^2+H_t\right)^{-1} S_t^2 h_t^2\right) g_t^2 z^2-{}\\
     {}- 2\left( S_t -\left( S_t h_t^2+H_t\right)^{-1} S_t^2 h_t^2\right) s_t g_t yz 
+{}\\
{}+
     \left( S_t-\left( S_t h_t^2+H_t\right)^{-1} S_t^2 h_t^2\right) s_t^2 y^2+{}\\
     {}+G_t z^2 -\left( S_t h_t^2 +H_t\right)^{-1} c_t^2 \alpha_t^2 z^2 -{}\\
     {}-\left( 
S_t h_t^2+H_t\right)^{-1} c_t^2 \alpha_t \beta_t(y) z-{}\\
{}-
\fr{1}{4}\left( S_t h_t^2+H_t\right)^{-1}  c_t^2 \beta_t^2(y)=0\,,
     \end{multline*}
группируем их и~получаем сле\-ду\-ющие уравнения:
\begin{itemize}
\item  для~$\alpha_t$:

\noindent
\begin{multline}
\fr{\partial\alpha_t}{\partial t}+2\alpha_t\left( b_t-\left( S_t h_t^2+H_t\right)^{-1} c_t 
S_t h_t g_t\right)+{}\\
{}+ \left( S_t- \left( S_t h_t^2+H_t\right)^{-1} S_t^2 h_t^2\right) 
g_t^2+G_t-{}\\
\hspace*{-8mm}{}-\left( S_t h_t^2+H_t\right)^{-1} c_t^2 \alpha_t^2 =0\,,\enskip \alpha_T=S_T 
g_t^2+G_T\,;\!\!
\label{e18-bos}
\end{multline}
\item для $\beta_t$:

\noindent
\begin{multline}
\fr{\partial\beta_t(y)}{\partial 
t}+\fr{1}{2}\,\Sigma_t^2(y)\fr{\partial^2\beta_t(y)}{\partial y^2} 
+A_t(y)\fr{\partial \beta_t(y)}{\partial y}+{}\\
{}+ 2\alpha_t\left( a_t +\left( S_t h_t^2+H_t\right)^{-1} c_t S_t h_t s_t\right) y+{}\\
{}+
\beta_t(y)\left( b_t -\left( S_t h_t^2 +H_t\right)^{-1} c_t S_t h_t g_t\right)-{}\\
{}-2\left( S_t-\left( S_t h_t^2+H_t\right)^{-1} S_t^2 h_t^2\right) s_t g_t y-{}
\\
{}-
\left( S_t h_t^2+H_t\right)^{-1} c_t^2 \alpha_t \beta_t(y)=0\,,\\
\beta_T(y)=-2S_T s_T g_T y\,;
\label{e19-bos}
\end{multline}
\item  для $\gamma_t$:
\begin{multline}
\hspace*{-0.8pt}\fr{\partial \gamma_t(y)}{\partial t}+\fr{1}{2}\,\Sigma_t^2(y)
\fr{\partial^2 \gamma_t(y)}{\partial y^2} +\sigma_t^2 \alpha_t +A_t(y)
\fr{\partial \gamma_t(y)}{\partial y}+{}\\
{}+ \beta_t(y)\left( a_t +\left( S_t h_t^2+H_t\right)^{-1} c_t S_t h_t s_t\right) y+{}\\
{}+
\left( S_t-\left( S_t h_t^2+H_t\right)^{-1} S_t^2 h_t^2\right)  s_t^2 y^2-{}\\
{}-\fr{1}{4}\left( S_t h_t^2+H_t\right)^{-1} c_t^2 \beta_t^2(y) =0\,,\\
\gamma_T(y)=S_T s_T^2 y^2\,.
\label{e20-bos}
\end{multline}
\end{itemize}
     
     Уравнение~(\ref{e18-bos}), легко заметить, является уравнением 
Риккати, которое в~силу сформулированного выше условия   
имеет единственное неотрицательное решение для всех $0\hm\leq t\hm\leq T$. 
Этот факт требует дополнительного комментария. Для получения 
уравнения~(\ref{e18-bos}) рас\-смот\-рим исходную задачу при дополнительных 
условиях $a_t\hm=0$ и~$s_t\hm=0$ для всех $0\hm\leq t\hm\leq T$. Нетрудно 
видеть, что эти условия рассматриваемую по\-ста\-нов\-ку сводят фактически 
к~классической ли\-ней\-но-квад\-ра\-тич\-ной задаче. Имеющуюся 
в~рассматриваемой формулировке чуть более общую форму целевой 
функции (принципиального значения это обобщение, конечно, не имеет) 
сведем к~классической еще одним предположением: $S_t\hm=0$ для всех 
$0\hm\leq t\hm\leq T$. Теперь уравнение для~$\alpha_t$ принимает хорошо 
известный вид:
     \begin{equation}
     \fr{\partial \alpha_t}{\partial t}+2\alpha_t b_t +G_t- H_t^{-1} c_t^2 
\alpha_t^2=0\,,\enskip \alpha_T=G_T\,.
     \label{e21-bos}
     \end{equation}

     В таком случае, как известно~\cite{10-bos}, существует единственное 
оптимальное управление~--- линейное с~обратной связью по выходу~$z_t$, 
с~коэффициентом усиления, опи\-сы\-ва\-емым уравнением  
Риккати~(\ref{e21-bos}). Именно этот результат дают  
уравнения~(\ref{e18-bos})--(\ref{e20-bos}) и~описываемая ими функция 
Беллмана~(\ref{e15-bos}), так как из $a_t\hm=0$ и~$s_t\hm=0$ немедленно 
следует, что $\beta_t(y)\hm=0$, откуда, в~свою очередь, с~учетом 
не\-за\-ви\-си\-мости решения от~$y_t$ следует, что $\gamma_t(y)\hm=\gamma_t$, 
т.\,е.\ не зависит от~$y$ и~задается уравнением: 
     $$
     \fr{\partial \gamma_t(y)}{\partial t} +\sigma^2_t \alpha_t=0\,,\enskip 
\gamma_T=0\,.
     $$ 
     Оптимальное управ\-ле\-ние при этом 
     $$
     u_t^*= -H_t^{-1} c_t \alpha_t z_t\,,
     $$
      т.\,е.\ все полностью совпадает с~известным классическим решением.
     
     С уравнениями~(\ref{e19-bos}) и~(\ref{e20-bos}) ситуация, естественно, 
обстоит сложнее. Это линейные уравнения второго порядка параболического 
типа, поскольку\linebreak
 $\Sigma_t^2(y)\hm>0$. Фактически отсутствуют 
конструктивные условия, гарантирующие существование их\linebreak
 решений 
(требовать, чтобы все фигурирующие в~уравнениях коэффициенты были 
представлены аналитическими функциями на всем пространстве значений, 
вряд ли целесообразно), поэтому далее будем предполагать, что данные 
уравнения имеют на рас\-смат\-ри\-ва\-емом интервале $0\hm\leq t\hm\leq T$ хотя 
бы одно ограниченное решение и~именно эти условия будем рас\-смат\-ри\-вать 
как достаточные условия существования оптимального решения 
рассматриваемой задачи.
     
     Таким образом, доказана следующая тео\-рема.
     
     \smallskip
     
     \noindent
     \textbf{Теорема.}\ \textit{Пусть для диффузионного 
процесса}~(\ref{e5-bos}) \textit{выполнены условия Ито, для 
     процесса}~(\ref{e6-bos})~--- \textit{ограничены коэффициенты, 
уравнения}~(\ref{e18-bos})--(\ref{e20-bos}) \textit{имеют ограниченные 
решения для $0\hm\leq t\hm\leq T$. Тогда минимум  
функционалу}~(\ref{e7-bos}) \textit{доставляет оптимальное 
управ\-ле\-ние}~(\ref{e17-bos}), \textit{где} $y\hm= y_t$; $z\hm=z_t$.
     
\section{Заключение}

     Рассмотренная задача оптимизации в~целом близка и~по модели, и~по 
критерию к~классической ли\-ней\-но-квад\-ра\-тич\-ной постановке. 
Принципиальным отличием является нелинейная модель для описания 
со\-сто\-яния динамической сис\-те\-мы, в~которой отсутствует управ\-ля\-ющее 
воздействие.\linebreak
 Такую модель наряду с~традиционной интер\-пре\-тацией  
<<со\-сто\-яние--вы\-ход>> мож\-но понимать как\linebreak модель неконтролируемого 
слож\-но\-го внешнего воздействия. Небольшое дополнительное отличие дает 
предложенная форма квад\-ра\-тич\-но\-го критерия, поз\-во\-ля\-ющая, в~част\-ности, 
ставить такие задачи, как отслеживание выходом или управ\-ле\-ни\-ем со\-сто\-яния 
сис\-те\-мы или ее выхода.
     
     Поскольку обсуждать возможности точного решения уравнений, 
определяющих оптимальное управ\-ле\-ние, не имеет смыс\-ла, наиболее 
актуальной далее является задача их приближенного чис\-лен\-но\-го решения 
и~анализа воз\-мож\-ности практической реализации. Этому посвящена вторая 
часть данной работы, пла\-ни\-ру\-емая к~выходу в~ближайшее время.

{\small\frenchspacing
 {%\baselineskip=10.8pt
 \addcontentsline{toc}{section}{References}
 \begin{thebibliography}{99}
\bibitem{1-bos}
\Au{Athans M.} Editorial on the LQG problem~// IEEE~T. Automat. Contr., 1971. Vol.~16. 
No.\,6. P.~528--552. doi: 10.1109/TAC.1971.1099845.
\bibitem{2-bos}
\Au{Wu Z.} Forward-backward stochastic differential equations, linear quadratic stochastic 
optimal control and nonzero sum differential games~// J.~Syst. Sci. Complex., 2005. Vol.~18. 
No.\,2. P.~179--192.
\bibitem{3-bos}
\Au{Chen B.\,S., Zhang~W.} Stochastic H2/H1 control with state-dependent noise~// IEEE 
T.~Automat. Contr., 2004. Vol.~49. No.\,1. P.~45--56. doi: 10.1109/TAC.2003.821400.
\bibitem{4-bos}
\Au{Bohacek S.} A~stochastic model of TCP and fair video transmission~// IEEE 
INFOCOM, 2003. Vol.~2. P.~1134--1144. doi: 10.1109/INFCOM.2003.1208950.
\bibitem{5-bos}
\Au{Домбровский В.\,В., Объедко~Т.\,Ю.} Управление с~прогнозированием системами 
с~марковскими скачками при ограничениях и~применение к~оптимизации 
инвестиционного портфеля~// Автомат. телемех., 2011. №\,5. С.~96--112. doi: 
10.1134/S0005117911050079.
\bibitem{6-bos}
\Au{Баландин Д.\,В., Коган~М.\,М.} Оптимальное линейно-квад\-ра\-тич\-ное управление: от 
матричных уравнений к~линейным матричным неравенствам~// Автомат. телемех., 2011. 
№\,11. С.~60--69. doi: 10.1134/ S0005117911110038.
\bibitem{7-bos}
\Au{Босов А.\,В.} Обобщенная задача распределения ресурсов программной системы~// 
Информатика и~её применения, 2014. Т.~8. Вып.~2. С.~39--47. doi: 
10.14357/19922264140204.
\bibitem{8-bos}
\Au{Босов А.\,В.} Управление линейным выходом дискретной стохастической системы по 
квадратичному критерию~// Изв. РАН. Теория и~системы управления, 2016. №\,3.  
С.~19--35. doi: 10.1134/S1064230716030060.
\bibitem{9-bos}
\Au{Флеминг У., Ришел~Р.} Оптимальное управление детерминированными 
и~стохастическими системами~/ Пер. с~англ.~--- М.: Мир, 1978. 316~с. 
(\Au{Fleming~W.\,H., Rishel~R.\,W.} Deterministic and stochastic optimal control.~--- New 
York, NY, USA: Springer-Verlag, 1975. 222~p.)
\bibitem{10-bos}
\Au{Девис М.\,Х.\,А.} Линейное оценивание и~стохастическое управление~/ Пер. с~англ.~--- 
М.: Наука, 1984. 206~с. (\Au{Davis~M.\,H.\,A.} Linear estimation and stochastic control.~--- 
London: Chapman and Hall, 1977. 224~p.)

 \end{thebibliography}

 }
 }

\end{multicols}

\vspace*{-6pt}

\hfill{\small\textit{Поступила в~редакцию 30.03.18}}

\vspace*{4pt}

%\newpage

%\vspace*{-24pt}

\hrule

\vspace*{2pt}

\hrule

\vspace*{-2pt}


\def\tit{STOCHASTIC DIFFERENTIAL SYSTEM OUTPUT CONTROL 
BY~THE~QUADRATIC CRITERION.~I.~DYNAMIC\\ PROGRAMMING 
OPTIMAL SOLUTION}


\def\titkol{Stochastic differential system output control 
by~the~quadratic criterion. I.~Dynamic programming 
optimal solution}

\def\aut{A.\,V.~Bosov and~A.\,I.~Stefanovich}

\def\autkol{A.\,V.~Bosov and~A.\,I.~Stefanovich}

\titel{\tit}{\aut}{\autkol}{\titkol}

\vspace*{-11pt}


\noindent
Institute of Informatics Problems, Federal Research Center ``Computer Science 
and Control'' of the Russian Academy of Sciences, 44-2~Vavilov Str., Moscow 
119333, Russian Federation


\def\leftfootline{\small{\textbf{\thepage}
\hfill INFORMATIKA I EE PRIMENENIYA~--- INFORMATICS AND
APPLICATIONS\ \ \ 2018\ \ \ volume~12\ \ \ issue\ 3}
}%
 \def\rightfootline{\small{INFORMATIKA I EE PRIMENENIYA~---
INFORMATICS AND APPLICATIONS\ \ \ 2018\ \ \ volume~12\ \ \ issue\ 3
\hfill \textbf{\thepage}}}

\vspace*{3pt}



\Abste{The problem of optimal control for the Ito diffusion 
process and a~controlled linear output is solved. The considered 
statement is close to the classical linear-quadratic Gaussian 
control  (LQG control) problem. Differences consist in the fact 
that the state is described by the nonlinear differential Ito equation  $dy_y = A_t(y_t) 
\,dt+\Sigma_t(y_t)\,dv_t$ and does not depend on the control~$u_t$, 
optimization subject is controlled linear output 
 $dz_t=a_ty_t\,dt +b_tz_t\,dt +c_t u_t\,dt +\sigma_t \,dw_t$. 
Additional generalizations are included in the quadratic 
quality criterion for the purpose of statement such problems 
as state tracking by output or a linear combination of state 
and output tracking by control. The method of dynamic programming 
is used for the solution. 
The assumption about Bellman function in the form  $V_t(y,z)= \alpha_t 
z^2+\beta_t(y) z+\gamma_t(y)$ allows one to find it. 
Three differential equations for the coefficients $\alpha_t$,  $\beta_t(y)$,
and $\gamma_t(y)$ give the solution. 
These equations constitute the optimal solution of the problem under consideration.}

\KWE{stochastic differential equation; optimal control; dynamic programming; 
Bellman function; Riccati equation; linear differential equations of parabolic type}


\DOI{10.14357/19922264180314}

\vspace*{-12pt}

\Ack
\noindent
This work was partially supported by the Russian Science Foundation (grant  
16-07-00677).



%\vspace*{6pt}

  \begin{multicols}{2}

\renewcommand{\bibname}{\protect\rmfamily References}
%\renewcommand{\bibname}{\large\protect\rm References}

{\small\frenchspacing
 {%\baselineskip=10.8pt
 \addcontentsline{toc}{section}{References}
 \begin{thebibliography}{99}
\bibitem{1-bos-1}
\Aue{Athans, M.} 1971. Editorial on the LQG problem. \textit{IEEE~T. 
Automat. Contr.} 16(6):528--552. doi: 10.1109/ TAC.1971.1099845.
\bibitem{2-bos-1}
\Aue{Wu, Z.} 2005. Forward-backward stochastic differential equations, linear 
quadratic stochastic optimal control and\linebreak\vspace*{-12pt}

\columnbreak

\noindent
 nonzero sum differential games. 
\textit{J.~Syst. Sci. Complex.} 18(2):179--192.
\bibitem{3-bos-1}
\Aue{Chen, B.\,S. and W.~Zhang.} 2004. Stochastic H2/H1 control with  
state-dependent noise. \textit{IEEE~T. Automat. Contr.} 49(1):45--56.
doi: 10.1109/TAC.2003.821400.
\bibitem{4-bos-1}
\Aue{Bohacek, S.} 2003. A~stochastic model of TCP and fair video 
transmission. \textit{IEEE INFOCOM}. 2:1134--1144.
doi: 10.1109/INFCOM.2003.1208950.
\bibitem{5-bos-1}
\Aue{Dombrovskii, V.\,V., and T.\,Yu.~Ob''edko.} 2011. Predictive control of 
systems with Markovian jumps under constraints and its application to the 
investment portfolio optimization. \textit{Automat. Rem. Contr.}  
72(5):989--1003.
\bibitem{6-bos-1}
\Aue{Balandin, D.\,V., and M.\,M.~Kogan.} 2011. Optimal linear-quadratic 
control: From matrix equations to linear matrix inequalities. \textit{Automat. 
Rem. Contr.} 72(11):2276--2284.
\bibitem{7-bos-1}
\Aue{Bosov, A.\,V.} 2014. Obobshchennaya zadacha raspredeleniya resursov 
programmnoy sistemy [The generalized problem of software system resources 
distribution]. \textit{Informatika i~ee Primeneniya~--- Inform. Appl.}  
8(2):39--47. doi: 
10.14357/19922264140204.
\bibitem{8-bos-1}
\Aue{Bosov, A.\,V.} 2016. Discrete stochastic system linear output control 
with respect to a quadratic criterion. \textit{J.~Comput. Syst. Sc. 
Int.} 55(3):349--364.
\bibitem{9-bos-1}
\Aue{Fleming, W.\,H., and R.\,W.~Rishel.} 1975. \textit{Deterministic and 
stochastic optimal control.} New York, NY: Springer-Verlag. 222~p.
\bibitem{10-bos-1}
\Aue{Davis, M.\,H.\,A.} 1977. \textit{Linear estimation and stochastic 
control.} London: Chapman and Hall. 224~p.
\end{thebibliography}

 }
 }

\end{multicols}

\vspace*{-6pt}

\hfill{\small\textit{Received March 30, 2018}}

%\pagebreak

%\vspace*{-18pt}
     
     \Contr
     
       \noindent
       \textbf{Bosov Alexey V.} (b.\ 1969)~--- Doctor of Science in technology, 
principal scientist, Institute of Informatics Problems, Federal Research 
Center ``Computer Science and Control'' of the Russian Academy of Sciences, 
44-2~Vavilov Str., Moscow 119333, Russian Federation; 
\mbox{AVBosov@ipiran.ru}
       
       \vspace*{3pt}
       
       \noindent
       \textbf{Stefanovich Alexey I.} (b.\ 1983)~--- principal specialist, 
Institute of Informatics Problems, Federal Research Center ``Computer Science 
and Control'' of the Russian Academy of Sciences, 44-2~Vavilov Str., Moscow 
119333, Russian Federation; \mbox{AStefanovich@frccsc.ru}
\label{end\stat}

\renewcommand{\bibname}{\protect\rm Литература}       

          %2
\renewcommand{\figurename}{\protect\bf Figure}
\renewcommand{\tablename}{\protect\bf Table}

\def\stat{lange}


\def\tit{ON COMPARATIVE EFFICIENCY OF~CLASSIFICATION SCHEMES IN~AN~ENSEMBLE 
OF~DATA SOURCES USING AVERAGE MUTUAL INFORMATION}

\def\titkol{On comparative efficiency of~classification schemes in~an~ensemble 
of~data sources using average mutual information}

\def\autkol{M.\,M.~Lange}

\def\aut{M.\,M.~Lange$^1$}

\titel{\tit}{\aut}{\autkol}{\titkol}

%{\renewcommand{\thefootnote}{\fnsymbol{footnote}}
%\footnotetext[1] {The study was carried out under state order to the Karelian Research 
%Centre of the Russian Academy of Sciences (Institute of Applied Mathematical 
%Research KarRC RAS) and supported by the Russian Foundation for Basic Research, 
%projects 18-07-00187, 18-07-00147, 18-07-00156, 19-07-00303.}}

\renewcommand{\thefootnote}{\arabic{footnote}}
\footnotetext[1]{Federal Research Center ``Computer Science and Control'' of the Russian Academy of Sciences, 
44-2~Vavilov Str., Moscow 119333, Russian Federation; \mbox{lange\_mm@ccas.ru}}


\index{Lange M.\,M.}
\index{Ланге M.\,M.}


\def\leftfootline{\small{\textbf{\thepage}
\hfill INFORMATIKA I EE PRIMENENIYA~--- INFORMATICS AND
APPLICATIONS\ \ \ 2019\ \ \ volume~13\ \ \ issue\ 4}
}%
 \def\rightfootline{\small{INFORMATIKA I EE PRIMENENIYA~---
INFORMATICS AND APPLICATIONS\ \ \ 2019\ \ \ volume~13\ \ \ issue\ 4
\hfill \textbf{\thepage}}}

%\vspace*{-2pt}





%The research is partially supported by the Russian Foundation for Basic Research 
%(grants Nos.\,18-07-01231 and 18-07-01385).




\Abste{Given ensemble of data sources and different fusion schemes, an accuracy of multiclass 
classification of the collections of the source objects is investigated. Using the average mutual 
information between the datasets of the sources and a~set of the classes, a~new approach to 
comparing lower bounds to an error probability in two fusion schemes is developed. The authors 
consider the WMV (Weighted Majority Vote) scheme which uses a~composition of the class 
decisions on the objects of the individual sources and the GDM (General Dissimilarity Measure) 
scheme based on a~composition of metrics in datasets of the sources.  For the above fusion 
schemes, the mean values of the average mutual information per one source are estimated. It is 
proved that the mean in the WMV scheme is less than the similar mean in the GDM scheme. As a~corollary, the lower bound to the error probability in the WMV scheme exceeds the similar 
bound to the error probability in the GDM scheme. This theoretical result is confirmed by 
experimental error rates in face recognition of HSI color images that yield the 
ensemble of H, S, and~I sources.} 

\KWE{multiclass classification; ensemble of sources; fusion scheme; composition of decisions; 
composition of metrics; average mutual information; error probability
}


\DOI{10.14357/19922264190403} 


%\vspace*{8pt}


\vskip 12pt plus 9pt minus 6pt

 \thispagestyle{myheadings}

 \begin{multicols}{2}

 \label{st\stat}

\section{Introduction }

\noindent
There are plenty of multiclass classification schemes that use input data from an 
ensemble of different modality sources. Such ensemble of data  sources produces 
the composite objects as the collections of the same class objects taken by one per 
each source. An example is the ensemble of biometric images such as faces, finger-
prints, signatures, palms, irises, and the like for a~given set of persons or classes. In 
this case, the composite objects are the collections of the same person images taken 
by one per each modality. In any correct classification scheme that makes the 
decisions on the submitted composite objects, an error probability decreases with 
increasing a~number of the sources~[1]. The decisions can be obtained using the 
different fusion schemes and the principal question is: What scheme is better? 

      The classification problem in the ensemble of sources is similar to the source 
coding problem based on quantization~[2].  There are known scalar and vector 
quantization for the continuous values.  The scalar quantization is used for the
individual values while the vector quantization is used for blocks of the values. In 
both cases, the above quantization schemes yield the code vectors for the 
appropriate blocks of the continuous values. 
   
    It should be noted that the optimal vector quantization is constructed with 
covering a~multidimensional space of the values by general spheres whose shape is 
adjusted to a~given dissimilarity measure between any pair of blocks of the 
values~[3]. In scalar quantization, the same multidimensional space is covered by 
cubes whose edge size is an optimal quantization step for any dimension. Thus, the 
code vectors are represented by the centers of the above spheres or cubes. Since
 for the same volume the spheres are more compact than the cubes, the vector 
quantization yields a~smaller error with respect to the scalar quantization. 

Also, for classification in a~given ensemble of the sources, an error probability is 
waited to be smaller in a~scheme of joint classifying each composite object as 
compared to an error probability in the scheme of combining the decisions on the 
objects of the individual sources. The proposed paper is focused on both developing a~theoretical validity of this idea and supporting it by a~computing experiment.  
    
Two fusion schemes that use the different data compositions for 
making the class decisions on the composite objects in the ensemble of the sources
have been investigated. 
They are the traditional WMV scheme by weighted majority voting the decisions 
on the objects of the individual sources~[4] and the original GDM scheme by 
combining the sources with a~general dissimilarity measure between any pair of the 
composite objects~[5]. Notice that WMV scheme is based on a~composition of 
decisions on the objects of individual sources while GDM scheme uses 
a~composition of metrics in datasets of the sources. Thus, ideologically, WMV 
and GDM fusion schemes are similar to the above scalar and vector quantization.

The specified similarity allows one to expect a~smaller error probability in GDM 
scheme as against WMV scheme. Some limits on the majority vote accuracy have 
been obtained in~[6]. Intuitively, it is clear that the minimal error probability of 
any classifier should depend on the average mutual information~[7] between a~set 
of the source objects and a~set of the classes. Moreover, the more average mutual 
information, the less error probability can be attained. So, our goal is to introduce 
the mutual information-based characteristics for WMV and GDM fusion schemes 
and, using these characteristics, to show an advantage of GDM scheme as against 
WMV scheme in the error probability. 

\section{Formalization of~the~Problem}


\subsection{Basic definitions and classification schemes}

\noindent
Let $\Omega=\{\omega_1, \ldots ,\omega_c\}$, $c\hm\geq 2$, be a~set of classes 
of the prior probabilities ${\sf P}(\omega_i)> 0$: $\sum\nolimits^c_{i=1} 
{\sf P}(\omega_i)=1$, and $\mathbf{X}^M= \mathbf{X}_1\cdots \mathbf{X}_M$ be 
an ensemble of sources, where the set $\mathbf{X}_m =\{\mathbf{x}_m= 
(x_{m1}, \ldots , x_{mN_m})\}$, $m=1,\ldots, M$, of $N_m$-dimensional 
vectors gives the $m$th source objects. In the ensemble , the components of 
any vector $\mathbf{x}_m\in \mathbf{X}_m$ take real values in $(-\infty, \infty)$, 
and any composite object $\mathbf{x}^M=(\mathbf{x}_1, \ldots ,
\mathbf{x}_M)\in \mathbf{X}^M$ is produced by a~collection of the vectors by 
one per source belonging to the same class in~$\Omega$.

In each set~$\mathbf{X}_m$, $m=1,\ldots , M$, a~dissimilarity measure between 
any pair of the objects $\mathbf{x}_m\in \mathbf{X}_m$ and 
$\hat{\mathbf{x}}_m\in \mathbf{X}_m$ is defined by
\begin{equation}
d\left( \mathbf{x}_m, \hat{\mathbf{x}}_m\right) =\sum\limits_{n=1}^{N_m} 
\fr{(x_{mn}-\hat{x}_{mn})^2}{\sigma^2_{mn}}
\label{e1-l}
\end{equation}
where $0<\sigma^2_{mn} <\infty$, $n=1,\ldots , N_m$, are unknown parameters. 
Also, for any pair of the composite objects $\mathbf{x}^M\in \mathbf{X}^M$ and 
$\hat{\mathbf{x}}^M\in \mathbf{X}^M$, let us define a~general dissimilarity 
measure as a~weighted composition of the metrics of the form~(1) taken with the 
weights $W=\{w_m>0,\ m=1,\ldots , M\}$ as follows:
\begin{equation}
D\left( \mathbf{x}^M, \hat{\mathbf{x}}^M\right) =\sum\limits^M_{m=1} w_m
d\left( 
\mathbf{x}_m, \hat{\mathbf{x}}_m\right)\,.
\label{e2-l}
\end{equation}

Let
\begin{equation}
\left\{\mathbf{x}_{im},\ i=1,\ldots ,c \right\} \subset \mathbf{X}_m,\enskip 
m=1,\ldots ,M\,,
\label{e3-l}
\end{equation}
be the subsets of the source template objects that represent the classes by one 
object from~$\mathbf{X}_m$ per each class. The subsets~(3) produce the subset 
of the template composite objects 
\begin{equation}
\left\{ \mathbf{x}_i^M=\left( \mathbf{x}_{i1},\ldots , \mathbf{x}_{iM}\right),\ 
i=1,\ldots ,c\right\} \subset \mathbf{X}^M\,.
\label{e4-l}
\end{equation}
Using the dissimilarity measure~(1) and assuming a~compactness of the objects 
in~$\mathbf{X}_m$, $m=1,\ldots , M$, relative to the corresponding template 
objects in~(3), let us define class-conditional densities of the~$m$th source objects 
as follows:
\begin{equation}
p\left(\mathbf{x}_m\vert \omega_i\right) =\fr{e^{-d(\mathbf{x}_m, 
\mathbf{x}_{im})}} {\int\nolimits_{\mathbf{X}_m}\!\!\! e^{-d(\mathbf{x}_m, 
\mathbf{x}_{im})}d\mathbf{x}_m}\,,\enskip i=1, \ldots , c\,.\!\!
\label{e5-l}
\end{equation}
Also, assuming a~compactness of the composite objects in~$\mathbf{X}^M$ 
relative to the corresponding templates in~(\ref{e4-l}) and using the general 
dissimilarity measure~(2), let us define class-conditional densities of the composite 
objects by
\begin{multline}
p_W\left(\mathbf{x}^M\vert \omega_i\right) \fr{e^{-D\left(\mathbf{x}^M, 
\mathbf{x}_i^M\right)}} {\int\nolimits_{\mathbf{X}^M} e^{-D\left(\mathbf{x}^M, 
\mathbf{x}_i^M\right)} d\mathbf{x}^M}\\
{} = \prod\limits^M_{m=1} \fr{e^{-w_m 
d(\mathbf{x}_m, \mathbf{x}_{im})}} {e^{-w_m d(\mathbf{x}_m, 
\mathbf{x}_{im})} d\mathbf{x}_m}\,,\enskip i=1,\ldots , c\,.
\label{e6-l}
\end{multline}
Under the product in~(\ref{e6-l}), there are the weighted class-conditional 
densities 
\begin{multline}
p_{w_m}\left(\mathbf{x}_m\vert \omega_i\right) =\fr{e^{-w_m d(\mathbf{x}_m, 
\mathbf{x}_{im})}} {\int\nolimits_{\mathbf{X}_m} e^{-w_m d(\mathbf{x}_m, 
\mathbf{x}_{im})} d\mathbf{x}_m}\,,\\
 i=1,\ldots ,c\,,
\label{e7-l}
\end{multline}
that give the densities of the form~(\ref{e5-l}) when $w_m=1$. In terms of 
information theory,  the densities~(\ref{e7-l}) define the $m$th source observation 
channel between input set~$\Omega$ and the output set~$\mathbf{X}_m$ as well 
as the  densities~(\ref{e6-l}) yield the observation multichannel 
between~$\Omega$ and~$\mathbf{X}^M$.

\begin{figure*} %fig1
  \vspace*{1pt}
    \begin{center}  
  \mbox{%
 \epsfxsize=107.799mm 
 \epsfbox{lan-1.eps}
 }
\end{center}
\vspace*{-9pt}
\Caption{Schemes of WMV-based~(\textit{a}) and GDM-based~(\textit{b}) classifiers}
\end{figure*}

Let $g_i^d(\mathbf{x}_m)$, $i=1,\ldots , c$, be the discriminant functions that are 
defined in the sets~$\mathbf{X}_m$, $m=1,\ldots , M$, using the dissimilarity 
measure of the form~(1). Then, WMV-based  class label decision on a~composite 
object $\mathbf{x}^M\in \mathbf{X}^M$ is defined by  
\begin{equation}
j^{\mathrm{WMV}}\left(\mathbf{x}^M\right) =\mathrm{arg}\,\max\limits^c_{i=1} 
\sum\limits^M_{m=1} w_m g_i^d\left(\mathbf{x}_m\right)
\label{e8-l}
\end{equation}
where the discriminant functions are independent on the source weights. Similarly, 
using in the ensemble~$\mathbf{X}^M$ the discriminant 
functions~$g_i^D(\mathbf{x}^M)$, $i=1,\ldots , c$, that depend on the 
weights~$W$ of all sources, GDM-based class label decision on the same 
composite object $\mathbf{x}^M\in \mathbf{X}^M$ is the following:
\begin{equation}
j^{\mathrm{GDM}}\left(\mathbf{x}^M\right) =\mathrm{arg}\,\max^c_{i=1} 
g_i^D\left(\mathbf{x}^M\right)\,.
\label{e9-l}
\end{equation}

        The classification schemes by the decision rules~(\ref{e8-l}) and~(\ref{e9-l}) 
are shown in Fig.~1. Here, $\hat{\Omega}=\Omega$ provided that the decisions 
in~$\hat{\Omega}$ can be differed from the real classes in~$\Omega$. The 
appropriate class-conditional densities yield the observation multichannels in 
WMV and GDM fusion schemes, respectively.




\subsection{Information criterion of efficiency for~the~fusion schemes}

\noindent  
Given the prior distribution $\{ P(\omega_i),\ i=1,\ldots\linebreak \ldots ,c\}$ and the weighted 
class-conditional densities $\{ p_{w_m}(\mathbf{x}_m\vert\omega_i), i=1,\ldots 
,c\}$ of the form~(\ref{e7-l}),the average mutual information 
between~$\mathbf{X}_m$ and~$\Omega$ is defined according to~\cite{7-l} by 
\begin{equation}
I_{w_m}\left(\mathbf{X}_m;\Omega\right) =H_{w_m}\left(\mathbf{X}_m\right) 
-H_{w_m} \left(\mathbf{X}_m\vert\Omega\right)\,.
\label{e10-l}
\end{equation}
Here,
\begin{align*}
H_{w_m}\left(\mathbf{X}_m\right) &=-\int\limits_{\mathbf{X}_m} p_{w_m} 
\left(\mathbf{x}_m\right) \ln p_{w_m} 
\left(\mathbf{x}_m\right)\,d\mathbf{x}_m\,;\\
H_{w_m}\left(\mathbf{X}_m\vert \Omega\right) &\\
&\hspace*{-11mm}{}=-\sum\limits^c_{i=1} 
P\left(\omega_i\right) \int\limits_{ \mathbf{X}_m} 
p_{w_m}\left(\mathbf{x}_m\vert \omega_i\right) \ln 
\left(\mathbf{x}_m\vert\omega_i\right)\,d\mathbf{x}_m
\end{align*}
are the differential entropies, and 
$p_{w_m}(\mathbf{x}_m)\linebreak =\sum\nolimits^c_{i=1} P(\omega_i) 
p_{w_m}(\mathbf{x}_m\vert \omega_i)$ is the marginal density 
in~$\mathbf{X}_m$, $m=1,\ldots ,M$. Notice that the average mutual information 
in~(\ref{e10-l}) does not exceed the entropy $H(\Omega)=-\sum\nolimits^c_{i=1} 
P(\omega_i)\ln P(\omega_i)$ of the set of the classes. For $w_m=1$, there is valid 
$p_{w_m}(\mathbf{x}_m\vert \omega_i)=p(\mathbf{x}_m\vert \omega_i)$  that 
yields $I_{w_m}(\mathbf{X}_m;\Omega)=I(\mathbf{X}_m;\Omega)$. 

Taking the means of the values 
$I\left(\mathbf{X}_m; \Omega\right)$  and $I_{w_m}(\mathbf{X}_m;\Omega)$ 
over all $m=1,\ldots , 
M$, one obtains the efficiency characteristics for WMV-based decision~(\ref{e8-l}) 
and GDM-based decision~(\ref{e9-l}), respectively. These means are defined as 
follows: 
\begin{align}
\hspace*{-2mm}I^{\mathrm{WMV}}_{W\_\mathrm{mean}}\left(\mathbf{X}^M;\Omega\right) &= 
\sum\limits^M_{m=1} I\left(\mathbf{X}_m;\Omega\right) 
\fr{w_m}{\sum\nolimits^M_{m=1} w_m}\,;\!\!\label{e11-l}\\
\hspace*{-2mm}I^{\mathrm{GDM}}_{W\_\mathrm{mean}} \left(\mathbf{X}^M;\Omega\right) &= \fr{1}{M} 
\sum\limits^M_{m=1} I_{w_m} \left(\mathbf{X}_m;\Omega\right)\,.
\label{e12-l}
\end{align}
Our goal is to prove the inequality 
\begin{equation}
\max\limits_W I^{\mathrm{WMV}}_{W\_\mathrm{mean}} \left(\mathbf{X}^M;\Omega\right) \leq 
I^{\mathrm{GDM}}_{W^*\_\mathrm{mean}} \left(\mathbf{X}^M;\Omega\right)
\label{e13-l}
\end{equation}
where~$W^*$ is the set of the source weights providing the maximum in the left 
part. 

\begin{figure*} %fig2
 \vspace*{1pt}
    \begin{center}  
  \mbox{%
 \epsfxsize=154.826mm 
 \epsfbox{lan-2.eps}
 }
\end{center}
\vspace*{-9pt}
\Caption{Sketches of the lower bounds to the average mutual information as the function 
of the error probability in WMV and GDM fusion schemes}
\end{figure*}  


\subsection{Average mutual information and~classification error probability}

\noindent
The criterion of the form~(\ref{e13-l}) assumes a~dependence of the average mutual 
information~$I(\mathbf{X}^M;\hat{\Omega})$ between the 
ensemble~$\mathbf{X}^M$ and the set of the class decisions~$\hat{\Omega}$ on 
a~lower bound to the error probability~$\varepsilon$ in the schemes shown in 
Fig.~1. Given observation multichannel, such function has been defined 
in~\cite{8-l} as a~generalization of the rate-distortion function for the source 
coding model with a~noisy observation channel~\cite{9-l}.  According  
to~\cite{8-l}, this function is lower bounded by 
\begin{multline}
R_L(\varepsilon) =I\left(\mathbf{X}^M;\Omega\right) -h\left(\varepsilon-
\varepsilon_{\min} \right)\\
{} -\left( \varepsilon -\varepsilon_{\min} \right) \ln (c-
1)\,,\enskip \varepsilon_{\min}\leq \varepsilon \leq \varepsilon_{\max}\,.
\label{e14-l}
\end{multline}
Here, $h(z) = -z\ln z -(1-z) \ln (1-z)$; 
$R_L(\varepsilon_{\min})\linebreak =I(\mathbf{X}^M;\Omega)$; 
$R_L(\varepsilon_{\max})= 0$; and $I(\mathbf{X}^M;\Omega) 
=H(\mathbf{X}^M)\linebreak - H(\mathbf{X}^M\vert\Omega)$ is the average mutual 
information between the input and the output of the observation multichannel in 
Fig.~1. Function~(\ref{e14-l}) has the largest value 
$I(\mathbf{X}^M;\Omega)$ at the point $\varepsilon=\varepsilon_{\min}$ and 
decreases as $\varepsilon$~increases. It is not difficult to show that the minimal error 
probability~$\varepsilon_{\min}$ is lower estimated by the conditional entropy 
$H(\Omega\vert \mathbf{X}^M)$ and~$\varepsilon_{\min}$ tends to zero 
when $H(\Omega\vert \mathbf{X}^M)$  decreases by increasing the size~$M$ of 
the ensemble. Taking into account the symmetry of the average mutual information 
\begin{multline*}
I(\mathbf{X}^M;\Omega)=H(\mathbf{X}^M)-H(\mathbf{X}^M\vert\Omega)\\
{}= 
H(\Omega)-H(\Omega\vert \mathbf{X}^M),
\end{multline*}
 in case of $\varepsilon_{\min}\to0$,  
function~(\ref{e14-l}) yields the Shannon bound of the form  
$H(\Omega)-h(\varepsilon) -\varepsilon\ln (c-1)$~\cite{7-l}. 

In the bound~(\ref{e14-l}), the average mutual information 
$I(\mathbf{X}^M;\Omega)$ is calculated in the product 
$\Omega*\mathbf{X}^M$ using the prior probabilities of the classes and the 
class-conditional densities of the form~(\ref{e6-l}). According to Fig.~1,  
the class-conditional densities in GDM scheme depend on the source weights and, 
therefore, $I(\mathbf{X}^M;\Omega)=I_W^{\mathrm{GDM}}(\mathbf{X}^M;\Omega)$ is 
the function of~$W$. In WMV scheme, the corresponding average mutual 
information $I(\mathbf{X}^M;\Omega)=I^{\mathrm{WMV}}(\mathbf{X}^M;\Omega)$ is 
equal to $I_W^{\mathrm{GDM}}(\mathbf{X}^M;\Omega)$ taken with the weights 
$w_m=1$, $m=1,\ldots , M$. The values $I^{\mathrm{WMV}}(\mathbf{X}^M;\Omega)$ 
and $I_W^{\mathrm{GDM}}(\mathbf{X}^M;\Omega)$ correspond to the minimal error 
probabilities~$\varepsilon_{\min}^{\mathrm{WMV}}$ and~$\varepsilon_{\min}^{\mathrm{GDM}}$ 
in WMV and GDM fusion schemes, respectively.  These error probabilities are 
achieved by the Bayes decisions of the form~(\ref{e9-l}) when the discriminant 
functions are given by the posterior probabilities of the classes~\cite{10-l}.


In general, the source sets $\mathbf{X}_1,\ldots, \mathbf{X}_M$ are statistically 
dependent on each other and there are valid the relations 
\begin{gather*}
I^{\mathrm{WMV}}_{W\_\mathrm{mean}} 
(\mathbf{X}^M;\Omega) < I^{\mathrm{WMV}}(\mathbf{X}^M;\Omega);\\[6pt]
I_{W\_\mathrm{mean}}^{\mathrm{GDM}} (\mathbf{X}^M;\Omega) 
< I^{\mathrm{GDM}}_{W} 
(\mathbf{X}^M;\Omega).
\end{gather*}

 Thus, for the weights~$W^*$ giving the maximum 
in~(\ref{e13-l}), the means $I^{\mathrm{WMV}}_{W^*\_\mathrm{mean}} 
(\mathbf{X}^M;\Omega)$ 
and $I^{\mathrm{GDM}}_{W^*\_\mathrm{mean}}(\mathbf{X}^M;\Omega)$ yield the error 
probabilities $\varepsilon^{\mathrm{WMV}}\linebreak
>\varepsilon^{\mathrm{WMV}}_{\min}$ and 
$\varepsilon^{\mathrm{GDM}}>\varepsilon^{\mathrm{GDM}}_{\min}$ that belong to the 
corresponding lower bounds of the form~(\ref{e14-l}). Also, taking into account 
that $I^{\mathrm{WMV}}(\mathbf{X}^M;\Omega)\leq 
I^{\mathrm{GDM}}_{W^*}(\mathbf{X}^M;\Omega)$, the inequality~(\ref{e13-l}) 
provides the following relation: $\varepsilon^{\mathrm{WMV}}
\geq \varepsilon^{\mathrm{GDM}}$. 
This fact is illustrated  in Fig.~2. 

\section{Calculation of~the~Average Mutual Information}

\noindent
In this section, an upper estimate of the functional 
$I_{w_m}(\mathbf{X}_m;\Omega)$ given in~(\ref{e10-l}) is obtained as 
a~function of the variable~$w_m^{1/2}$.  At the value $w q_m^{1/2}=1$, this 
function yields the upper estimate for~$I(\mathbf{X}_m;\Omega)$. Using the 
marginal density $p_{w_m}(\mathbf{x}_m)$  and taking into account that $-\ln z$ 
is the convex downwards function of~$z$, it is valid the Jensen 
inequality~\cite{11-l} as follows:
\begin{multline*}
-\ln p_{w_m} \left(\mathbf{x}_m\right) = -\ln \sum\limits^c_{i=1} P(\omega_i) 
p_{w_m} \left(\mathbf{x}_m\vert   \omega_i\right)\\
{} \leq -\sum\limits^c_{i=1} 
P(\omega_i) \ln p_{w_m}\left(\mathbf{x}_m\vert \omega_i\right)\,.
\end{multline*}
Applying this inequality in~(\ref{e10-l}), one obtains the upper estimated 
differential entropy:
\begin{multline}
H_{w_m}\left(\mathbf{X}_m\right) \leq -\sum\limits^c_{i=1} P(\omega_i) 
\sum\limits^c_{j=1} P(\omega_j)\\
{}\times \int\limits_{\mathbf{X}_m} 
p_{w_m}\left(\mathbf{x}_m\vert\omega_i\right) \ln p_{w_m} 
\left(\mathbf{x}_m\vert \omega_j\right)\,d\mathbf{x}_m\,.
\label{e15-l}
\end{multline}

Given the dissimilarity measures~(\ref{e1-l}) and~(\ref{e2-l}), the conditional 
density $p_{w_m}(\mathbf{x}_m\vert\omega_i)$ of the form~(\ref{e7-l}) is the 
Gaussian density of~$N_m$ independent variables that have the 
means~$x_{imn}$ and the variances $\sigma^2_{imn}/(2w_m)$, $n=1,\ldots , 
N_m$, subject to $w_m>0$. It allows us to express the integral in~(\ref{e15-l}) 
over the interval $(-\infty, +\infty)$  as the Euler integral~\cite{12-l}. The 
calculation yields the upper estimated differential entropy:
\begin{multline}
H_{w_m}(\mathbf{X}_m)\leq \fr{1}{2}\ln 
\fr{\pi}{w_m}+\fr{1}{2}\sum\limits^c_{j=1} P(\omega_j)  
\sum\limits_{n=1}^{N_m} \ln \sigma^2_{jmn}\\
{}+w_m \sum\limits^c_{i=1} P(\omega_i) \sum\limits^c_{j=1} P(\omega_j) 
\sum\limits_{n=1}^{N_m} \fr{(x_{imn}-x_{jmn})^2}{\sigma^2_{jmn}}\\
+2\fr{w_m^{1/2}}{\sqrt{\pi}}\sum\limits^c_{i=1} 
P(\omega_i)\sum\limits^c_{j=1} P(\omega_j) \sum\limits_{n=1}^{N_m} 
\fr{\vert x_{imn}-x_{jmn}\vert \sigma_{imn}}{\sigma^2_{jmn}}\\
+\fr{1}{2}\sum\limits^c_{i=1} P(\omega_i) \sum\limits^c_{j=1} P(\omega_j) 
\sum\limits_{n=1}^{N_m} \fr{\sigma^2_{imn}}{\sigma^2_{jmn}}
\label{e16-l}
\end{multline}
and the following conditional differential entropy:
\begin{multline}
H_{w_m}\left(\mathbf{X}_m\vert\Omega\right) \\
{}=\fr{1}{2}\ln \fr{\pi e}{w_m} 
+\fr{1}{2} \sum\limits^c_{i=1} P(\omega_i) \sum\limits_{n=1}^{N_m} \ln 
\sigma^2_{imn}\,.
\label{e17-l}
\end{multline}
The substitutions of the differential entropy and the conditional differential entropy 
in~(\ref{e10-l}) by~(\ref{e16-l}) and~(\ref{e17-l}) yield the upper 
estimated average mutual information:

\noindent
\begin{multline}
I_{w_m}\left(\mathbf{X}_m;\Omega\right) \\
{}\leq w_m \sum\limits^c_{i=1} 
P(\omega_i) \sum\limits^c_{j=1} P(\omega_j) \sum\limits_{n=1}^{N_m} 
\fr{(x_{imn}-x_{jmn})^2}{\sigma^2_{jmn}}\\
+2\fr{w_m^{1/2}}{\sqrt{\pi}} \sum\limits^c_{i=1} P(\omega_i) 
\sum\limits^c_{j=1} P(\omega_j) \sum\limits_{n=1}^{N_m} \fr{\vert x_{imn} -
x_{jmn})^2}{\sigma^2_{jmn}}\\
+\fr{1}{2}\sum\limits^c_{i=1}P(\omega_i) \sum\limits_{j=1}^c P(\omega_j) 
\sum\limits^{N_m}_{n=1} \left( \fr{\sigma^2_{imn}} {\sigma^2_{jmn}}-
1\right)\,.
\label{e18-l}
\end{multline}
The right part in~(\ref{e18-l}) is a~parabolic function 
$a_mw_m\linebreak +b_mw_m^{1/2}+c_m$ of the variable $w_m^{1/2}>0$ for 
$m\linebreak =1,\ldots , M$. Since $a_m>0$, $b_m>0$, and $c_m\geq0$, the parabola 
exceeds the value~$c_m$ and grows when~$w_m^{1/2}$ increases.  For 
$w_m^{1/2}=1$, this function gives the upper estimate $a_m+b_m+c_m$ for 
$I(\mathbf{X}_m;\Omega)$. The weights of interest are defined by the values 
$w_m^{1/2}\geq 1$ that satisfy the condition 
$a_mw_m+b_mw_m^{1/2}+c_m\leq H(\Omega)$, $m=1, \ldots , M$. Setting 
$\delta_m=(a_m+b_m+c_m)/H(\Omega)\leq 1$, we assign the parametric source 
weights 
\begin{equation}
w_m(s)=e^{s\delta_m}\,,\enskip m=1,\ldots , M,
\label{e19-l}
\end{equation}
where $s\geq 0$ is a~free parameter that yields $w_m(s)\geq 1$. In what follows, 
we denote the upper estimates~(\ref{e18-l}) taken with the weights~(\ref{e19-l}) 
by $I_s(\mathbf{X}_m;\Omega)$, $m=1,\ldots , M$.

\section{Main Results}

\noindent
   Using in the right part of the form~(\ref{e11-l}) the estimates 
$I(\mathbf{X}_m;\Omega)\leq a_m+b_m+c_m$, $m=1,\ldots , M$, taken with the 
weights~(\ref{e19-l}), one obtains the upper estimated mean value 
$I^{\mathrm{WMV}}_{s\_\mathrm{mean}} (\mathbf{X}^M;\Omega)$.  Also, the estimates 
$I_{w_m}(\mathbf{X}_m;\Omega)\linebreak \leq a_mw_m+b_m w_m^{1/2}+c_m$, 
$m=1,\ldots , M$, taken with the similar weights in the right part of~(\ref{e12-l}) 
yield the upper estimated mean value $I^{\mathrm{GDM}}_{s\_\mathrm{mean}} 
(\mathbf{X}^M;\Omega)$. Then, for $s\to 0$, we calculate an asymptotic 
maximum $I^{\mathrm{WMV}}_{s^*\_\mathrm{mean}}(\mathbf{X}^M;\Omega)$ at the point~$s^*$ 
and show that this maximum satisfies the inequality 
$I^{\mathrm{WMV}}_{s^*\_\mathrm{mean}}(\mathbf{X}^M;\Omega) \leq 
I^{\mathrm{GDM}}_{s^*\_\mathrm{mean}}(\mathbf{X}^M;\Omega)$.

In subsequent statements, we use the following notations:  
\begin{gather*}
\mu=\fr{1}{M}\sum\limits_{m=1}^M \delta_m\,;\quad 
\Delta_1=\fr{1}{M}\sum\limits^M_{m=1} \delta^2_m-\mu^2\,;\\
\Delta_2=\fr{1}{M}\sum\limits^M_{m=1} \delta_m^3-
\mu\fr{1}{M}\sum\limits^M_{m=1} \delta^2_m\,.
\end{gather*}

\noindent
\textbf{Theorem~1.}\ \textit{For $(2\mu\Delta_1-\Delta_2)>\Delta_1 >0$ and 
$s\to 0$, the value $s^*=\Delta_1/(2\mu \Delta_1-\Delta_2)$ yields}
$$
\max\limits_s I^{\mathrm{WMV}}_{s\_\mathrm{mean}} \left(\mathbf{X}^M;\Omega\right) =
\left( 
\mu+\fr{1}{2}\,\Delta_1 s^*\right) H(\Omega)\,.
$$
\textit{For $\Delta_1=0$, there is valid $I^{\mathrm{WMV}}_{s\_\mathrm{mean}} 
(\mathbf{X}^M;\Omega) =\mu H(\Omega)$ for all $s\geq 0$}.

\smallskip

\noindent
P\,r\,o\,o\,f\,.\ \  Using $q_s(\delta_m) =e^{s\delta_m}/\sum\nolimits^M_{m=1} 
e^{s\delta_m}$,  the upper estimated mean value defined in~(\ref{e11-l}) takes the 
form: 
\begin{equation}
I^{\mathrm{WMV}}_{s\_\mathrm{mean}} (\mathbf{X}^M;\Omega) =H(\Omega) 
\sum\limits^M_{m=1} \delta_m q_s(\delta_m)\,.
\label{e20-l}
\end{equation}
For $s\to 0$, there is valid the asymptotic equation: 
\begin{equation}
\sum\limits^M_{m=1} \delta_m q_s(\delta_m) \approx \mu+ \Delta_1 s-
\fr{1}{2}\left( 2\mu \Delta_1-\Delta_2\right) s^2\,.
\label{e21-l}
\end{equation}
Using the assumption of the theorem, the parabola in the right part of~(\ref{e21-l}) 
takes the maximal value $\mu+\Delta_1 s^*/2$ at the point $s^*=\Delta_1/(2\mu 
\Delta_1-\Delta_2)$. Notice that the same values $\delta_m=\delta$, $m=1,\ldots , 
M$, provide $\Delta_1=0$ and $\Delta_2=0$. In this case, $q_s(\delta_m)=1/M$ 
and the sum in~(\ref{e21-l}) is equal to $\mu=\delta$ for all $s\geq0$. Thus, the 
substitution of the sum in~(\ref{e20-l}) by $\mu+\Delta_1 s^*/2$ in case of 
$\Delta_1>0$ or by~$\mu$ in case of $\Delta_1=0$ completes the proof.


\smallskip

\noindent
\textbf{Theorem~2.}\ \textit{For $\Delta_1>0$ and on condition that $a_m\geq 
c_m$, $m=1,\ldots , M$, there is valid the inequality $I^{\mathrm{WMV}}_{s^*\_\mathrm{mean}} 
(\mathbf{X}^M;\Omega)<I^{\mathrm{GDM}}_{s^*\_\mathrm{mean}} (\mathbf{X}^M;\Omega)$ at 
the optimal point $s^*>0$.  For $\Delta_1=0$ and a~given $s\geq 0$, there is valid 
the inequality $I^{\mathrm{WMV}}_{s\_\mathrm{mean}} (\mathbf{X}^M;\Omega) \leq 
I^{\mathrm{GDM}}_{s\_\mathrm{mean}} (\mathbf{X}^M;\Omega)$ which passes into the equality at 
the point $s=0$.}

\smallskip\

\noindent
P\,r\,o\,o\,f\,.\ The estimates~(\ref{e18-l}) taken with the weights~(\ref{e19-l}) 
give the upper estimated mean value~(\ref{e12-l}) as follows:
\begin{multline}
I^{\mathrm{GDM}}_{s\_\mathrm{mean}} \left(\mathbf{X}^M;\Omega\right)\\ 
{}=\fr{1}{M}\sum\limits^M_{m=1} \left( a_m e^{s\delta_m} +b_m 
s^{s\delta_m/2} +c_m\right)\,.
\label{e22-l}
\end{multline}
Taking the square approximations of the exponential terms in~(\ref{e22-l}), one 
obtains the following inequality: 
\begin{multline}
I^{\mathrm{GDM}}_{s\_\mathrm{mean}} \left(\mathbf{X}^M;\Omega\right) \geq \mu H(\Omega) \\
{}+\left( \fr{1}{M} \sum\limits^M_{m=1} 
a_m\delta_m+\fr{1}{2M}\sum\limits^M_{m=1} b_m\delta_m\right) s\\
{}+ \left( \fr{1}{2M}\sum\limits^M_{m=1} 
a_m\delta_m^2+\fr{1}{4M}\sum\limits^M_{m=1} b_m\delta_m^2\right) s^2\,.
\label{e23-l}
\end{multline}
In case of $\Delta_1>0$, the inequality~(\ref{e23-l}) together with the 
estimates~(\ref{e20-l}) and~(\ref{e21-l}) yield: 
\begin{multline}
I^{\mathrm{GDM}}_{s\_\mathrm{mean}}\left(\mathbf{X}^M;\Omega\right)-
I^{\mathrm{WMV}}_{s\_\mathrm{mean}} 
\left(\mathbf{X}^M;\Omega\right)\\
\geq \left( \fr{1}{M} \sum\limits^M_{m=1}a_m\delta_m 
+\fr{1}{2M}\sum\limits^M_{m=1} b_m\delta_m-\Delta_1H(\Omega)\right)s\\
{}+\left( \fr{1}{2M}\sum\limits^M_{m=1} 
a_m\delta_m^2+\fr{1}{4M}\sum\limits^M_{m=1} 
b_m\delta_m^2\right.\\
\left.{}+\fr{1}{2}\left( 2\mu \Delta_1-\Delta_2\right) 
H(\Omega)
\vphantom{\sum\limits^M_{m=1}}
\right)s^2\,.
\label{e24-l}
\end{multline}
Assuming 
\begin{equation}
\fr{1}{M}\sum\limits^M_{m=1} a_m\delta_m+\fr{1}{2M} 
\sum\limits^M_{m=1} b_m\delta_m\geq \fr{1}{2}\Delta_1 H(\Omega),
\label{e25-l}
\end{equation}
the right part in~(\ref{e24-l}) is lower estimated by the parabola
\begin{multline*}
-\fr{1}{2}\,\Delta_1H(\Omega) s+\left( \fr{1}{2M}\sum\limits^M_{m=1} 
a_m\delta_m^2+\fr{1}{4M} \sum\limits^M_{m=1} b_m 
\delta_m^2\right.\\
\left.{}+\fr{1}{2}\left( 2\mu \Delta_1-\Delta_2\right) H(\Omega) 
\vphantom{\sum\limits^M_{m=1}}
\right) s^2
\end{multline*}
that has a~positive root
\begin{multline*}
s_0=
\Delta_1H(\Omega)\Bigg/
\left(
\vphantom{\sum\limits^M_{m=1}}
(2\mu\Delta_1-\Delta_2)H(\Omega)\right.\\
\left.{} +\fr{1}{M} 
\sum\limits^M_{m=1} a_m\delta_m^2+\fr{1}{2M} \sum\limits^M_{m=1} 
b_m\delta_m^2\right)\\
{}<\fr{\Delta_1}{2\mu\Delta_1-\Delta_2}=s^*\,.
\end{multline*}
Since this parabola is positive for $s>s_0$,  the lower estimate of the right part 
in~(\ref{e24-l}) is positive at the point $s^*>0$ of the maximal value 
$I^{\mathrm{WMV}}_{s^*\_\mathrm{mean}} (\mathbf{X}^M;\Omega)$ that provides the inequality 
$I^{\mathrm{GDM}}_{s^*\_\mathrm{mean}} (\mathbf{X}^M;\Omega)- I^{\mathrm{WMV}}_{s^*\_\mathrm{mean}} 
(\mathbf{X}^M;\Omega) >0$. 

Notice that the assumption of the form~(\ref{e25-l}) is equivalent to the inequality 
$$
\fr{1}{M}\sum\limits^M_{m=1} \left( c_m-a_m\right) \delta_m \leq \mu^2 
H(\Omega)
$$
that is valid under the conditions $a_m\geq c_m$, $m\linebreak =1,\ldots , M$. These 
conditions are held if the templates in different classes are sufficiently distinct 
from each other. Formally, the parameters in~(\ref{e18-l}) should satisfy the 
following relation:
\begin{multline*}
\left( x_{imn}-x_{jmn}\right)^2\geq \fr{1}{2}\left\vert \sigma^2_{imn} -
\sigma^2_{jmn}\right\vert \,,\\
 m=1,\ldots , M\,,\enskip n=1,\ldots , N_m\,.
\end{multline*}
In case of $\Delta_1=0$, one has $I^{\mathrm{WMV}}_{s\_\mathrm{mean}} 
(\mathbf{X}^M;\Omega) =\mu H(\Omega)$ and 
$I^{\mathrm{GDM}}_{s\_\mathrm{mean}}(\mathbf{X}^M;\Omega)\geq \mu H(\Omega)$ for a~given 
$s\geq 0$. So, there is valid the inequality $I^{\mathrm{WMV}}_{s\_\mathrm{mean}} 
(\mathbf{X}^M;\Omega)\linebreak \leq I^{\mathrm{GDM}}_{s\_\mathrm{mean}} (\mathbf{X}^M;\Omega)$ 
which passes into the equality at the point $s=0$. The theorem is proved.

\smallskip


Sketches of the graphics in Fig.~3 interpret the theorems~1 and~2.


\begin{figure*} %fig3
 \vspace*{1pt}
    \begin{center}  
  \mbox{%
 \epsfxsize=162.134mm 
 \epsfbox{lan-3.eps}
 }
\end{center}
\vspace*{-9pt}
\Caption{Graphical interpretation of the results for cases of $\Delta_1>0$~(\textit{a}) and 
$\Delta_1=0$~(\textit{b})}
\end{figure*}

\begin{figure*}[b] %fig4
 \vspace*{1pt}
    \begin{center}  
  \mbox{%
 \epsfxsize=163mm 
 \epsfbox{lan-4.eps}
 }
\end{center}
\vspace*{-9pt}
\Caption{Examples of the 8th level representations for the face HSI images}
\end{figure*}
  



\noindent
\textbf{Corollary.}\ For the optimal value~$s^*$ in the case of $\Delta_1>0$ and any 
$s>0$ in the case of $\Delta_1=0$, the mean values of the average mutual information 
per one source in WMV and GDM fusion schemes provide the lower bounds to the 
error probabilities satisfying the inequality 
$\varepsilon^{\mathrm{WMV}}>\varepsilon^{\mathrm{GDM}}$.

\section{Experimental Results}

\noindent
The efficiency of WMV and GDM fusion schemes is shown by comparative error 
rates for face recognition of HSI color images.The components H, S, and~I produce the 
objects of the individual sources and the ensemble HSI produces the composite 
objects. The color images are taken from~25~persons (classes) per 40~images in 
each class~\cite{13-l}.  The prior probability distribution of the classes is uniform. 
Face recognition has been performed in a~space of multilevel tree-structured 
pattern representations with elliptic primitives~\cite{5-l}. The error rates have 
been obtained for multiclass NN (nearest neighbor) and SVM (support vector 
machine) classifiers that are the collections of elementary ``class-vs-all'' classifiers. 
The experiments have been performed using 100~times, 2~fold cross validation. 

The examples of the tree-structured representations for the face components H, S, and I 
are shown in Fig.~4. The image components correspond to the source numbers 
$m=1, 2, 3$.


Using the above representations, the dissimilarity measure 
$d(\mathbf{x}_m, \hat{\mathbf{x}}_m)\geq 0$  for any pair 
of the objects~$\mathbf{x}_m$ and~$\hat{\mathbf{x}}_m$ has been introduced 
in~\cite{14-l}.
The weighted sum of the above measures taken over the components 
H, S, and I yields the general dissimilarity measure $D(\mathbf{x}^3, 
\hat{\mathbf{x}}^3)$  of the form~(\ref{e2-l}) between the corresponding 
composite objects~$\mathbf{x}^3$ and~$\hat{\mathbf{x}}^3$. 

The dissimilarity 
measures $d(\mathbf{x}_m, \hat{\mathbf{x}}_m)$, $m=1,2,3$, and 
$D(\mathbf{x}^3, \hat{\mathbf{x}}^3)$ have allowed us to construct the 
discriminant functions~$g_i^d(\mathbf{x}_m)$ and~$g_i^D(\mathbf{x}^3)$, 
$i=1, \ldots$\linebreak $\ldots , c$, for making the decisions of the form~(\ref{e8-l}) and~(\ref{e9-l}) 
by the appropriate NN and SVM classifiers.
{\looseness=1

}

\begin{table*}\small
\begin{center}
\tabcolsep=8pt
\begin{tabular}{cccccc}
\multicolumn{6}{c}{Error rates for HSI face recognition by NN and SVM classifiers}\\
\multicolumn{6}{c}{\ }\\[-6pt]
\hline
\multicolumn{1}{c}{\raisebox{-6pt}[0pt][0pt]{Classifier}}&
\multicolumn{3}{c}{Sources} &\multicolumn{2}{c}{Fusion schemes}\\ 
\cline{2-6} 
&H&S&I&\hspace*{2mm}WMV&GDM\\ 
\hline 
NN&0.022&0.017&0.015&\hspace*{2mm}0.009&0.006\\ 
SVM&0.019&0.012&0.011&\hspace*{2mm}0.007&0.003\\ 
\hline 
\end{tabular} 
\end{center} 
\vspace*{-12pt}
\end{table*}
The table summarizes the cross-validation error rates for both the individual sources 
and their ensemble using GDM and WMV fusion schemes. The experimental 
results demonstrate a~decrease of the error rates in the ensemble HSI as against the 
error rates for the sources H, S, and~I. Also, the obtained error rates confirm 
some advantage of GDM scheme as compared with the WMV scheme.
 
\vspace*{-9pt}

\section{Concluding Remarks}

\noindent
To compare the potentially achievable  classification error probabilities for two 
fusion schemes in the ensemble of data sources, the information-based criterion 
has been suggested. The proposed  criterion is based on comparing the mean 
values of the average mutual information between the set of the classes and the 
datasets of the sources. These means  are independent on a~decision algorithm 
and  they are defined in the WMV scheme of fusion of the decisions on the source 
objects and in the GDM scheme of fusion of the metrics in datasets of the sources. 
Taking the above mean values as the points of the appropriate rate distortion 
functions, it has been shown that the  lower bound to GDM-based error probability is 
smaller as compared with the similar WMV-based error probability. The advantage 
in accuracy of the GDM scheme relative to the WMV scheme is confirmed by the error 
rates for NN and SVM decision algorithms in experiments on recognition of HSI 
face images given by the ensemble of the sources Н, S, and I.
In future, we plan to 
extend the ensemble of biometric sources and the set of the decision algorithms.  
For the above fusion schemes and the different decision algorithms, we plan 
to estimate a~redundancy of the error rates relative to the appropriate lower 
bounds. 

\vspace*{-9pt}

\Ack
\noindent
The research is partially supported by the Russian Foundation for Basic Research 
(grants Nos.\,18-07-01231 and 18-07-01385).

\renewcommand{\bibname}{\protect\rmfamily References}


\vspace*{-9pt}

{\small\frenchspacing
{\baselineskip=10.45pt
\begin{thebibliography}{99}
\bibitem{1-l}
\Aue{Kuncheva, L.} 2014. \textit{Combining pattern classifiers, methods and algorithms}. 2nd ed. 
New York, NY: John Wiley and Sons. 384~p.
\bibitem{2-l}
\Aue{Gray, R., and D.~Neuhoff.} 1998. Quantization. 
\textit{IEEE T.~Inform. Theory} 44(6):2325--2383.
\bibitem{3-l}
\Aue{Kolmogorov, A.\,N., and V.\,M.~Tikhomirov.} 1961. 
\mbox{$\varepsilon$-entropy} and  $\varepsilon$-capacity of sets in 
functional spaces. \textit{AMS Transl.} 17(2):277--364.
\bibitem{4-l}
\Aue{Lam, L., and C.~Suen.} 1997. 
Application of majority voting to pattern recognition: An 
analysis of its behavior and performance. \textit{IEEE T.~Syst. Man. Cyb.}
27(5):553--568.
\bibitem{5-l}
\Aue{Lange, M.\,M., and D.\,Y.~Stepanov.} 
2014. Recognition of objects given by collections of 
multichannel images. \textit{Pattern Recogn. Image Anal.} 24(3):431--442.
\bibitem{6-l}
\Aue{Kuncheva, L., C.~Whitaker, C.~Shipp, and R.~Duin.} 2003. Limits on the majority
vote accuracy in classifier fusion. \textit{Pattern Anal. Appl.} 6(1):22--31.
\bibitem{7-l}
\Aue{Gallager, R.} 1968. 
\textit{Information theory and reliable communication}. New York, NY: John Wiley and 
Sons. 608~p.
\bibitem{8-l}
\Aue{Lange, M.\,M., and A.\,M.~Lange.} 2018. 
O~teoretiko-informatsionnoy modeli klassifikatsii
dannykh [On information theoretical model for data classification]. 
\textit{Mashinnoe obuchenie i~analiz dannykh}  [J.~Machine Learning Data Analysis] 
4(3):165--179.
\bibitem{9-l}
\Aue{Dobrushin, R.\,L., and B.\,S.~Tsybakov.} 1962. 
Information transmission with additional noise. 
\textit{IRE T.~Inform. Theor.} 8(5):293--304.
\bibitem{10-l}
\Aue{Duda, R., P. Hart, and D.~Stork.}
 2001. \textit{Pattern classification}. 2nd ed. New York, NY: John Wiley and Sons. 
688~p.
\bibitem{11-l}
\Aue{Beckenbach, E., and R.~Bellman.} 1961. 
\textit{Inequalities}. New York, NY: Springer-Verlag. 55~p.
\bibitem{12-l}
\Aue{Gradshteyn, I.\,S., and I.\,M.~Ryzhik.}
 2007. \textit{Table of integrals, series, and products}. 7th ed. 
Academic Press. 1221~p.
\bibitem{13-l}
Database of face images. Available at:
{\sf http://\linebreak sourceforge.net/projects/colorfaces} (accessed 
October~9, 2019).
\bibitem{14-l}
\Aue{Lange, M.\,M., and S.\,N.~Ganebnykh.} 
2018. On fusion schemes for multiclass object 
classification with reject in a~given ensemble of sources. 
\textit{J.~Phys. Conf. Ser.} 1096:012048. 12~p. Available at: 
{\sf https://\linebreak iopscience.iop.org/article/10.1088/1742-6596/1096/1/ 012048}
 (accessed October~7,  2019).
 \end{thebibliography} } }

\end{multicols}

\vspace*{-9pt}

\hfill{\small\textit{Received July 01, 2019}}

\vspace*{-16pt}

\Contrl

\vspace*{-3pt}

\noindent
\textbf{Lange Mikhail M.} (b.\ 1945)~--- Candidate of Science (PhD) in technology, leading 
scientist, Federal Research Center ``Computer Sciences and Control'' of the Russian Academy of 
Sciences, 44-2~Vavilov Str., Moscow 119333, Russian Federation; 
\mbox{lange\_mm@ccas.ru}

 

\newpage

%\vspace*{8pt}

%\hrule

%\vspace*{2pt}

%\hrule

%\vspace*{-7pt}

%\newpage

\vspace*{-28pt}

\def\tit{О СРАВНИТЕЛЬНОЙ ЭФФЕКТИВНОСТИ СХЕМ КЛАССИФИКАЦИИ ДАННЫХ НА~АНСАМБЛЕ 
ИСТОЧНИКОВ С~ИСПОЛЬЗОВАНИЕМ СРЕДНЕЙ ВЗАИМНОЙ ИНФОРМАЦИИ$^*$}

\def\titkol{О сравнительной эффективности схем классификации данных на~ансамбле 
источников} % с~использованием средней взаимной информации}

\def\aut{M.\,M.~Ланге}

\def\autkol{M.\,M.~Ланге}

{\renewcommand{\thefootnote}{\fnsymbol{footnote}} \footnotetext[1]
{Работа частично поддержана РФФИ (проекты 18-07-01231 и 18-07-01385).}}



\titel{\tit}{\aut}{\autkol}{\titkol}

\vspace*{-11pt}

\noindent
Федеральный исследовательский центр <<Информатика и управление>> Российской академии наук, 
\mbox{lange\_mm@ccas.ru}

\vspace*{1pt}

\def\leftfootline{\small{\textbf{\thepage}
\hfill ИНФОРМАТИКА И ЕЁ ПРИМЕНЕНИЯ\ \ \ том\ 13\ \ \ выпуск\ 4\ \ \ 2019}
}%
 \def\rightfootline{\small{ИНФОРМАТИКА И ЕЁ ПРИМЕНЕНИЯ\ \ \ том\ 13\ \ \ выпуск\ 4\ \ \ 2019
\hfill \textbf{\thepage}}}

\vspace*{-1pt}




\Abst{Исследуется точность многоклассовой классификации наборов объектов от 
ансамбля источников при различных схемах комплексирования данных. Предлагается 
новый подход к~сравнению нижних границ вероятности ошибки для двух схем 
классификации с~использованием средней взаимной информации между данными 
источников и множеством классов. Рассмотрена схема WMV (Weighted Majority Vote) на 
основе композиции решений по объектам источников и~схема GDM (General Dissimilarity 
Measure) на основе композиции метрик на множествах объектов источников. Для 
исследуемых схем получены оценки усредненных значений средней взаимной 
информации на один источник. Доказано, что указанная характеристика схемы WMV не 
превосходит аналогичной характеристики схемы GDM, при этом нижняя граница 
вероятности ошибки в~схеме WMV превосходит нижнюю границу вероятности ошибки 
в~схеме GDM. Полученный теоретический результат подтвержден экспериментальными 
оценками вероятности ошибки распознавания цветных HSI изображений лиц для двух 
схем комплексирования данных от источников H, S и~I.} 

\KW{многоклассовая классификация; ансамбль источников; схема комплексирования; 
композиция решений; композиция метрик; средняя взаимная информация; вероятность 
ошибки}

\DOI{10.14357/19922264190403} 



%\vspace*{-3pt}


 \begin{multicols}{2}

\renewcommand{\bibname}{\protect\rmfamily Литература}
%\renewcommand{\bibname}{\large\protect\rm References}

{\small\frenchspacing
{\baselineskip=10.5pt
\begin{thebibliography}{99}
%\vspace*{-3pt}
\bibitem{1-l-1}
\Au{Kuncheva L.} Combining pattern classifiers, methods and algorithms.~--- 2nd ed.~---
  New York, NY, USA: John Wiley and Sons, 2014. 384~p.
\bibitem{2-l-1}
\Au{Gray R., Neuhoff~D.} Quantization~// IEEE T. Inform. Theory, 1998. 
Vol.~44. Iss.~6. P.~2325--2383.
\bibitem{3-l-1}
\Au{Колмогоров А.\,Н.,  Тихомиров~В.\,М.} 
$\varepsilon$-энтропия и~$\varepsilon$-ем\-кость 
множеств в функциональных пространствах~// УМН, 1959. Т.~14. №\,2(86). С.~3--86.

\bibitem{4-l-1}
\Au{Lam L., Suen~C.} Application of majority voting to pattern recognition: An analysis of its behavior and 
performance~// IEEE T. Syst. Man Cyb., 1997. Vol.~27. Iss.~5. P.~553--568.
\bibitem{5-l-1}
\Au{Lange M.\,M., Stepanov~D.\,Y.}
 Recognition of objects given by collections of multichannel images~// 
Pattern Recogn. Image Anal., 2014. Vol.~24. Iss.~3. P.~431--442.
\bibitem{6-l-1}
\Au{Kuncheva L., Whitaker~C., Shipp~C., Duin~R.}
 Limits on the majority vote accuracy in classifier fusion~// 
Pattern Anal. Appl., 2003. Vol.~6. Iss.~1. P.~22--31.
\bibitem{7-l-1}
\Au{Gallager R.} Information theory and reliable communication.~---
  New York, NY, USA: John Wiley and Sons, 1968. 
608~p.
\bibitem{8-l-1}
\Au{Ланге М.\,М., Ланге~А.\,М.} О~тео\-ре\-ти\-ко-ин\-фор\-ма\-ци\-он\-ной 
модели классификации данных~// 
Машинное обучение и анализ данных, 2018. Т.~4. Вып.~3. С.~165--179.
\bibitem{9-l-1}
\Au{Dobrushin R.\,L., Tsybakov~B.\,S.}
 Information transmission with additional noise~// IRE T. 
Inform. Theor., 1962. Vol.~8. Iss.~5. P.~293--304.
\bibitem{10-l-1}
\Au{Duda R., Hart~P., Stork~D.}
 Pattern classification.~--- 2nd ed.~--- New York, NY, USA: John Wiley and Sons, 2001. 688~p.
\bibitem{11-l-1}
\Au{Beckenbach E., Bellman~R.} Inequalities.~--- New York, NY, USA: Springer-Verlag, 1961. 55~p.
\bibitem{12-l-1}
\Au{Gradshteyn I.\,S., Ryzhik~I.\,M.} Table of integrals, series, and products.~---
7th ed.~--- Academic Press, 
2007. 1221~p.
\bibitem{13-l-1}
Database of face images. {\sf http://sourceforge.net/\linebreak projects/colorfaces}.
\bibitem{14-l-1}
\Au{Lange M.\,M., Ganebnykh~S.\,N.} 
On fusion schemes for multiclass object classification with reject in 
a~given ensemble of sources~// J.~Phys. Conf. Ser., 2018. Vol.~1096.
 Art. ID: 012048.  P.~1--12. 
\end{thebibliography}
} }

\end{multicols}

 \label{end\stat}

 \vspace*{-9pt}

\hfill{\small\textit{Поступила в~редакцию 01.07.2019}}


%\renewcommand{\bibname}{\protect\rm Литература}
\renewcommand{\figurename}{\protect\bf Рис.}
\renewcommand{\tablename}{\protect\bf Таблица}



 
 
%Ланге Михаил Михайлович (р.\ 1945)~--- кандидат технических наук, ведущий научный 
%сотрудник Федерального исследовательского центра <<Информатика и управление>> 
%Российской академии наук

 
 
 
     %3
\def\stat{krivenko}

\def\tit{МНОГОМЕРНЫЙ РЕФЕРЕНСНЫЙ РЕГИОН\\ ВЫСОКОЙ ПЛОТНОСТИ}

\def\titkol{Многомерный референсный регион высокой плотности}

\def\aut{М.\,П.~Кривенко$^1$}

\def\autkol{М.\,П.~Кривенко}

\titel{\tit}{\aut}{\autkol}{\titkol}

\index{Кривенко М.\,П.}
\index{Krivenko M.\,P.}


%{\renewcommand{\thefootnote}{\fnsymbol{footnote}} \footnotetext[1]
%{Работа выполнена при финансовой поддержке РФФИ (проекты 16-07-00677 
%и~15-37-20611-мол\_а\_вед).}}


\renewcommand{\thefootnote}{\arabic{footnote}}
\footnotetext[1]{Институт проблем информатики Федерального исследовательского центра <<Информатика и~управление>> Российской академии наук,
\mbox{mkrivenko@ipiran.ru}}

\vspace*{4pt}



\Abst{Рассматриваются принципы построения многомерных референсных регионов
(MRR~--- multivariate reference region). 
Предложен оригинальный метод построения региона на основе областей с~высокой 
плотностью точек и~аппроксимации распределения данных с~помощью смеси нормальных 
распределений. Для оценки порога для плотности распределения используется  
бут\-стреп-ме\-тод. В~качестве эксперимента рассмотрена задача построения 
и~использования эталонной области для прогнозирования типа мочевого камня. Обработка 
реальных данных продемонстрировала преимущества предлагаемых решений.}

\KW{многомерный референсный регион; область высокой плотности; бут\-стреп-ме\-тод; 
смесь многомерных нормальных распределений}

\vspace*{6pt}

\DOI{10.14357/19922264170207} 


\vskip 10pt plus 9pt minus 6pt

\thispagestyle{headings}

\begin{multicols}{2}

\label{st\stat}

\section{Введение}

     Многомерный референсный регион 
был предложен в~литературе по клинической химии в~начале 1970-х~гг.\ как 
альтернатива одномерным референсным интервалам~[1]. Там излагались 
преимущества предлагаемых множественных тестов, хоть и~имеющих 
упрощенный вид, но снижающих (по отношению к~одномерным вариантам) 
число ложных положительных результатов. Появление MRR оказалось 
особенно привлекательным для интерпретации результатов наборов 
медицинских тестов. Тем не менее возникали трудности в~построении 
и~использовании процедур многомерного анализа (см., например,~[2]), 
связанные, в~частности, с~быстрым увеличением числа параметров, которые 
должны быть оценены. Немногие лаборатории использовали MRR в~своей 
практике, причем в~экспериментальном режиме, и,~как следствие, на 
сегодняшний день имеется относительно малое количество соответствующих 
публикаций. 

\vspace*{-6pt}

\section{Многомерный референсный регион на основе расстояния Махалонобиса}

\vspace*{-2pt}

     Одномерный референсный интервал, полученный статистическим путем, 
использует центральную часть значений анализируемого показателя, обычно 
соответствующую~95\% некоторой популяции~--- совокупности особей 
определенного вида (например, здоровой части населения определенного пола 
из некоторого диапазона возрастов). Одномерные референсные интервалы 
применялись в~течение многих лет в~качестве стандартного приема 
интерпретации лабораторных данных. Они легко формируются, хранятся, 
извлекаются и~передаются в~лабораторных информационных системах, просты 
в~понимании, хорошо воспринимаются медицинским сообществом в~ходе 
длительного использования. Тем не менее одномерные референсные интервалы 
при классификации данных могут дать большое число ложно аномальных 
результатов. Этот далеко не единственный недостаток однофакторного 
референсного интервала может быть полностью или частично устранен 
с~помощью MRR.
     
     Простейшим и~весьма распространенным способом построения MRR 
является использование прямого произведения отдельных референсных 
интервалов в~предположении, что они статистически независимы. Пусть 
$(1\hm-\alpha)$~--- вероятность попадания в~MRR, а~$p_0$~--- вероятность 
попадания в~референсный интервал для любого из~$d$~признаков, тогда 
$p_0\hm= \sqrt[d]{1-\alpha}$. С~ростом размерности~$d$ значения~$p_0$ 
быстро приближаются к~1, что фактически лишает смысла применение MRR.
     
     Как и~в одномерном случае, отправной точкой для построения MRR 
может стать нормальное распределение. Идеи центрального расположения 
референсного региона и~заданной вероятности попадания в~него приводят для 
$d$-мер\-но\-го нормального распределения, имеющего плотность 
распределения
     \begin{multline*}
     \varphi(y,\mu,\Sigma) ={}\\
     {}=(2\pi)^{-d/2}\vert\Sigma\vert^{-1/2}\exp \left( -\fr{\left(y-
\mu\right)^{\mathrm{T}} \Sigma^{-1}(y-\mu)}{2}\right),
   \end{multline*}
где величина $(y-\mu)^{\mathrm{T}} \Sigma^{-1} (y-\mu)$ есть квадрат так 
называемого расстояния Махаланобиса между~$y$ и~$\mu$, к~использованию 
многомерного эллипсоида
\begin{multline*}
(2\pi)^{-d/2}\vert\Sigma\vert^{-1/2}\exp \left( -\fr{\left(y-\mu\right)^{\mathrm{T}}
\Sigma^{-1} 
(y-\mu)}{2}\right) ={}\\
{}=const
\end{multline*}
или, что то же самое, 
$$ 
(y-\mu)^{\mathrm{T}} \Sigma^{-1}(y-\mu)=const\,.
$$
Его называют эллипсоидом равной плотности распределения (или просто 
эллипсоидом равной вероятности). 
     
     Если задаться вероятностью $(1\hm-\alpha)$ попадания в~эллипсоид 
равной вероятности вида $(y\hm-\mu)^{\mathrm{T}}\Sigma^{-1} (y\hm-\mu)\hm= 
\rho$, то параметр~$\rho$ удовлетворяет уравнению $\mathrm{Pr}\left\{ 
\chi_d^2\leq \rho\right\} \hm=1\hm-\alpha$.
     
     Использование эллипсоида в~качестве MRR будет оправдано только 
тогда, когда исходное распределение данных есть многомерное нормаль-\linebreak ное. 
Поэтому становятся актуальными критерии\linebreak подгонки, а~также использование 
процедур норма\-ли\-зации распределения данных в~многомерном\linebreak случае.
 Если 
с~помощью тестов выявляется, что распределение не является нормальным, то 
Международная федерация клинической химии и~лабораторной медицины 
рекомендует, согласно~[3], использовать двухступенчатую процедуру 
нормализации. Следует обратить внимание, что многошаговость здесь 
относится не к~многомерности, а касается лишь покоординатного 
преобразования распределения данных к~нормальному.
     
     Первые же попытки применения MRR на основе расстояния 
Махалонобиса (фактически это означает принятие модели нормального 
распределения референсных значений) выявили ряд недостатков (более 
подробно смотри в~\cite[разд.~6.2]{4-kri}):
     \begin{itemize}
\item проявление <<проклятий>> размерности при механическом 
увеличении~$d$, в~особенности если игнорируется этап анализа состава 
признаков~[1, 5, 6];
\item из-за небольших объемов обучающей выборки невысокая устойчивость 
при применении, в~частности чувствительность к~увеличению неточностей 
измерений после того, как регион был установлен~\cite{5-kri, 7-kri}. 
\item предположение о нормальном распределении и~попытки <<подправить>> 
действительность с~помощью преобразований реальных данных для их 
нормализации при увеличении размерности данных становятся все более 
шаткими~\cite{5-kri};
\item представление и~интерпретация выводов на основе MRR трудно 
понимаемы не только для специалистов в~предметной области~[8].
\end{itemize}

\vspace*{-9pt}

\section{Многомерный референсный регион высокой плотности}

\vspace*{-2pt}

     Заметим, что в~случае нормального распределения референсных значений 
для точек внут\-ри построенного эллипсоида значения плотности\linebreak распределения 
больше, чем на границе, а~вне~--- меньше. Это замечание позволяет 
предложить другой подход к~построению MRR.
     
     \smallskip
     
     \noindent
     \textbf{Определение.}\ Eсли плотность распределения референсных 
значений есть $f(y)$, то MRR есть область $A_t\hm= \left\{ y\in 
\mathcal{R}^d\vert f(y)\hm\geq t\right\}$ для некоторого порогового 
значения~$t$. 
     
     \smallskip
     
     Для нормального распределения это уже упомянутый эллипсоид равной 
вероятности. Если задается вероятность $(1\hm-\alpha)$ попадания в~$A_t$, то 
пороговое значение~$t$ есть решение уравнения $\int\nolimits_{A_t} 
f(u)\,du\hm=1\hm-\alpha$, получить которое аналитически в~случае 
произвольной плотности распределения вряд ли возможно. Здесь присутствуют 
две проблемы: вычисление многомерного интеграла и~зависимость области 
интегрирования от неизвестного значения. Для решения их предлагается 
привлечь метод моделирования.
     
     Сгенерируем выборку из $f(y)$, которую обозначим как $Y^f\hm= \left\{ 
y_1^f, \ldots, y_m^f\right\}$. Для оценки $\int\nolimits_{A_t} f(u)\,du$ 
используем отношение:

\noindent
\begin{multline*}
     \fr{\left\vert \left\{ y_i^f\vert y_i^f\in A_t\right\}\right\vert }{m} =
      \fr{\left\vert\left\{ y_i^f\vert 
f\left(y_i^f\right) \geq t\right\}\right\vert }{m} ={}\\
{}= 1-\fr{\left\vert \left\{ y_i^f\vert f(y_i^f)<t\right\}\right\vert }{m}=1-
F_m(t)\,,
     \end{multline*}
где $F_m(t)$~--- эмпирическая функция распределения случайной 
величины~$f(y)$, т.\,е.\ случайной величины, являющейся результатом 
преобразования с~помощью функции~$f(\cdot)$ случайной величины, име\-ющей 
плотность распределения~$f(u)$.

     Таким образом, искомая оценка~$t^*$ должна удовле\-тво\-рять уравнению 
$F_m(t^*)\hm=\alpha$ и~может быть получена как непараметрическая оценка 
квантиля\linebreak\vspace*{-12pt}

\pagebreak

\noindent
 порядка~$\alpha$ из распределения $F_m(\cdot)$. Если обозначить 
$f_i\hm= f(y_i^f)$, то~$t^*$ есть~$f_{(r)}$, где
     $$
     r= \begin{cases}
     m\alpha, &\ m\alpha~\mbox{---~целое}\,;\\
     \lfloor m\alpha+1\rfloor\,, & m\alpha~\mbox{--- не целое}\,.
     \end{cases}
     $$
     Заметим, что для такой оценки можно указать доверительный интервал.
     
     Для построения MRR необходимо знать распределение данных. При 
реализации принципа точек высокой плотности в~первую очередь следует 
обратиться к~параметрическим моделям, в~част\-ности к~смеси нормальных 
распределений, име\-ющей плотность распределения
     $$
     f(u) =\sum\limits_{j=1}^k p_j \varphi\left (u,\mu_j, \Sigma_j\right)\,.
     $$
Если $\hat{f}(u)$~--- оценка смеси, то~$t^*$ строится сле\-ду\-ющим образом:
\begin{itemize}
\item генерируется выборка $\left\{ y_1^f,\ldots , y_m^f\right\}$ из $\hat{f}(u)$ и~
для каждого ее $i$-го элемента подсчитывается значение $\hat{f}\left( 
y_i^f\right)$;
\item в~качестве~$t^*$ берется непараметрическая оценка квантиля 
порядка~$\alpha$ (в случае необходимости дополнительно находится 
непараметрическая оценка доверительного интервала для~$t^*$, что 
может характеризовать правильность выбранного объема для 
генерируемой выборки).
\end{itemize}

     Пусть для $f(u)$ имеется~$A_t$, а также получена $\hat{f}(u)$ 
и~соответствующий MRR вида~$\hat{A}_t$. Качество аппроксимации~$A_t$ 
с~по\-мощью~$\hat{A}_t$ можно оценить с~по\-мощью вероятности совпадения 
этих областей, т.\,е. 
     $$
     P_c= \int\limits_{\{ u\in A_t\}\cup \{u\in \hat{A}_t\}} \hspace*{-6mm}
f(u)\,du+\int\limits_{\{u\not\in A_t\} \cup\{ u\not\in \hat{A}_t\}}\hspace*{-6mm} f(u)\,du\,.
     $$
     
     Для оценки  $P_c$ можно использовать величину
     \begin{multline*}
     \hat{P}_c= \fr{\left\vert \left\{ 
     y_i^f\vert y_i^f \in \left\{\left\{ y_i^f\in A_t\right\}\cup \left\{y_i^f\in 
\hat{A}_t\right\}\right\}\right\}\right\vert}{m}+{}\\
{}+\fr{\left\vert \left\{ y_i^f\vert y_i^f \in \left\{\left\{ y_i^f\not\in A_t\right\}\cup 
\left\{ y_i^f\not\in \hat{A}_t\right\}\right\}\right\}\right\vert}{m}\,.
     \end{multline*}
     
     Использование MRR высокой плотности для диагностирования сводится 
к~реализации так называемого слабого критерия значимости для наблюденного 
значения~$x$: нулевая гипотеза заключается в~том, что $x\hm\in A_t$, 
статистика критерия есть $\hat{f}(x)$ и~решение о~принадлежности 
критической об\-ласти~$A_t$ принимается при больших значениях~$\hat{f}(x)$.
     
     Для медицинской практики важна возможность использования 
референсного региона при интерпретации результатов обследования 
некоторого пациента с~вектором признаков~$x$. В~подобных случаях 
сложившейся практикой для слабых критериев значимости является 
использование критического уровня~$\alpha_{\mathrm{cr}}$ (более распространенным 
в~медицине является употребление термина $p$-зна\-че\-ние)  $\alpha_{\mathrm{cr}}\hm= 
\mathrm{Pr}\left\{ \hat{f}(y)\hm\leq \hat{f}(x)\right\}$, где $y$~--- случайная 
величина, имеющая плотность распределения~$\hat{f}(u)$, а $\hat{f}(x)$~--- 
значение плотности распределения~$\hat{f}(u)$ в~точке~$x$. Эта 
характеристика дает представление о~том, насколько сильно данное 
наблюденное значение~$x$ противоречит гипотезе (или подкрепляет ее) 
о~принадлежности данных MRR. При выбранном же заранее уровне 
значимости с~помощью~$\alpha_{\mathrm{cr}}$ сразу же можно принять конкретное 
решение. 

\vspace*{-9pt}

\section{Эксперименты}

\vspace*{-2pt}

     Для демонстрации возможностей MRR использовались данные по 
прогнозу химического состава мочевых камней по метаболическим 
показателям мочи и~сыворотки крови, а также антропологическим 
характеристикам пациентов~[9]. В качестве исходной классификации камней 
рассматривалась следующая: чисто оксалатные (далее обозначены как O), чисто 
уратные (U), чисто фосфатные (P), смесь только оксалатных и~уратных (OU), 
смесь только оксалатных и~фосфатных (OP), смесь только уратных 
и~фосфатных (UP), все остальные. Данная классификация была построена 
в~[10] на основе доминирующих частот встречаемости основных компонентов. 
В~качестве референсных значений рассматривались наборы метаболических 
и~антропологических показателей (их всего было~14), соответствующих 
определенному классу камней.

\begin{table*}\small
\begin{center}


\begin{tabular}{|c|c|c|c|c|c|c|}
\multicolumn{7}{c}{Качество классификации с~помощью MRR}\\
\multicolumn{7}{c}{\ }\\[-6pt]
\hline
\multicolumn{1}{|c|}{\raisebox{-6pt}[0pt][0pt]{\tabcolsep=0pt\begin{tabular}{c}Тип\\ камня\end{tabular}}}&
\multicolumn{1}{c|}{\raisebox{-6pt}[0pt][0pt]{$N$}}&$(1-\alpha)$, 
&\multicolumn{2}{c|}{MRR(5)}&\multicolumn{2}{c|}{MRR(1)}\\
\cline{4-7}
&&&&&&\\[-9pt]
&&\%&$(1-\hat{\alpha})$, \%&$\hat{\beta}$, \%&$(1-\hat{\alpha})$, \%&$\hat{\beta}$, \%\\
\hline
\multicolumn{1}{|c|}{\raisebox{-18pt}[0pt][0pt]{O}}&
\multicolumn{1}{c|}{\raisebox{-18pt}[0pt][0pt]{82}}
&95&100\hphantom{9}&71&90&24\\
&&85&96&78&89&36\\
&&75&91&85&77&44\\
&&65&76&88&74&50\\
\hline
\multicolumn{1}{|c|}{\raisebox{-18pt}[0pt][0pt]{U}}&
\multicolumn{1}{c|}{\raisebox{-18pt}[0pt][0pt]{76}}&95&100\hphantom{9}&75&91&24\\
&&85&99&85&80&35\\
&&75&82&89&74&48\\
&&65&71&91&68&56\\
\hline
\multicolumn{1}{|c|}{\raisebox{-18pt}[0pt][0pt]{P}}&
\multicolumn{1}{c|}{\raisebox{-18pt}[0pt][0pt]{83}}&95&100\hphantom{9}&66&87&25\\
&&85&94&78&86&33\\
&&75&86&82&82&41\\
&&65&77&87&75&47\\
\hline
\end{tabular}
\end{center}
\end{table*}
     
     
     Для каждого из основных классов O, U, P, OU, OP и~UP перед построением 
MRR проводилась селекция признаков и~принималось то значение размерности 
признакового пространства~$d$ и~соответствующий набор показателей, 
которые позволяли прогнозировать состав камней без потери качества 
(методика описана в~\cite{9-kri} и~привела к~значению $d\hm=9$). В~качестве 
модели данных в~первую очередь рассматривалась смесь многомерных 
нормальных распределений из пяти элементов (подбор числа элементов смеси 
проводился с~по\-мощью AIC~--- Akaike information criterion), для соответствующего региона было принято 
обозначение MRR(5). Для сравнения также использовалась модель 
нормального распределения, которой соответствовал MRR(1). Полученные 
результаты приводятся час\-тич\-но в~таблице, где $N$~--- объем 
классифицируемых данных; $\hat{\alpha}$~--- оценка для~$\alpha$; 
$\hat{\beta}$~--- оценка мощности критерия при определении типа камня на 
основании MRR.


     Одной из базовых характеристик является вероятность попадания в~MRR 
$(1\hm-\alpha)$ и~ее оценка $(1\hm-\hat{\alpha})$. Сравнение соответствующих 
столбцов с~учетом значений~$N$ и~ориентировочных значений разброса 
(стандартные отклонения на основе биномиального распределения) не 
позволило выявить явных отклонений. Необходимо, правда, отметить, что во 
всех проанализированных случаях для MRR(5) оказалось, что $1\hm-
\hat{\alpha}\hm\geq 1\hm-\alpha$.
     
     Назначение MRR, заключающееся в~сжатом представлении референсных 
значений, в~многомерном случае практически не проявляется. Для задания 
MRR(5) необходимо указать следующие величины: $1\hm-\alpha$, $t$, 
$p_1,\ldots, p_{k-1}$, $\mu_1, \Sigma_1,\ldots , \mu_k,\Sigma_k$, общее 
количество которых равно  $[2\hm+ (k\hm-1)\hm+ k(d\hm+ d(d\hm+1)/2)]$ 
и,~в~частности, в~рассматриваемых экспериментах~--- 276. Для MRR(1) это 
значение меньше и~равно~56. При этом для обрабатываемой обучающей 
выборки в~зависимости от класса камней речь идет о~порядка~10$^2$ векторах 
данных (см.\ столбец со значениями~$N$), что приблизительно 
дает~10$^3$~скалярных величин.
     
     Другое назначение MRR состоит в~его использовании для 
диагностирования (классификации). В~этой связи в~первую очередь 
проводился сравнительный анализ MRR(1) (фактически это означает, что 
построение региона осуществляется на основе расстояния Махаланобиса) 
и~MRR(5) (модель смеси нормальных распределений и~предложенный 
в~данной работе метод оценивания па\-ра\-мет\-ров региона). Показателем 
информативности метода построения многомерного региона выступала 
мощность соответствующего слабого критерия значимости, а~именно: 
вероятность не попасть в~MRR при условии, что данные берутся из дополнения 
к~классу, для которого построена MRR. Сравнение соответствующих столбцов 
говорит о~явном преимуществе двух предложенных моментов: усложнение 
модели данных путем перехода от нормального распределения к~смеси 
нормальных распределений и~построение региона высокой плотности.
     
     Использование критического уровня можно продемонстрировать  
с~по\-мощью зависимости результатов сравнения двух классов от того, какой 
класс взять за основу. Введем для возможных значений $p$-ве\-ли\-чи\-ны три 
интервала: $(-\infty, 1\%)$, $[1\%, 5\%)$, $[5\%, 100\%)$ с~соответствующей 
интерпретацией положения наблюденного набора показателей для пациента 
относительно построенного MRR: уверенное непопадание, неуверенное 
попадание, уверенное попадание. Если MRR построить для оксалатных камней, 
то результаты для анализа пациентов с~фосфатными камнями дадут следующий 
вектор относительных частот попадания $p$-ве\-ли\-чин в~указанные 
интервалы: $(60\%, 18\%, 22\%)$. Если же MRR строить для фосфатных 
камней, то получим $(71\%, 5\%, 24\%)$. Таким образом, для классификации 
указанных камней при приблизительно одинаковых частотах попадания в~MRR 
(22\% или~24\%) уверенный отказ от референсного региона происходит чаще, 
если принять за базовый MRR регион для фосфатных камней. Построение 
шкалы, подобной рассмотренной, является прерогативой специалистов 
в~предметной области, в~данной работе она использовалась только для 
иллюстрации. 

\vspace*{-6pt}

\section{Заключение}

\vspace*{-2pt}

     На настоящий момент имеется относительно мало примеров применения 
MRR в~клинической практике. Тому есть несколько причин. Математическое 
обеспечение, необходимое для получения и~применения MRR, не отвечает 
возможностям большинства клинических лабораторий. Лаборатории слабо 
оснащены программными средствами\linebreak для реализации достаточно сложного 
математического аппарата многомерного анализа, а~еще важнее, что 
отсутствуют методики, инструкции по\linebreak использованию соответствующих 
средств. Лишь немногие клинические применения демонстрируют 
преимущества MRR, хотя свидетельств неудачных попыток больше.
     
     Несмотря на сложности внедрения мно\-го\-мерно\-го анализа референсных 
значений, можно сформулировать некоторые рекомендации по иссле\-до\-ва\-нию 
и~разработке MRR. Во-пер\-вых, эффективная размерность в~MRR должна 
быть как можно меньше, чтобы избежать затенения диагностически полезной 
информации тестами, со\-зда\-ющи\-ми шум. Низкая размерность также должна 
уменьшить неблагоприятные последствия увеличения неточности результатов 
в~связи с~ростом числа анализируемых показателей. Во-вто\-рых, показатели 
(тес\-ты), включенные в~MRR, должны быть физиологически релевантными 
исследуемому кругу расстройств, чтобы максимизировать информацию, 
полученную от MRR. В-треть\-их, чтобы учесть эффекты долгосрочной 
лабораторной из\-мен\-чи\-вости, данные, используемые для получения MRR, 
долж\-ны быть собраны и~проанализированы в~течение достаточно большого 
периода времени (от нескольких недель до нескольких месяцев).  
В-чет\-вер\-тых, представление результатов лабораторных исследований 
следует осуществлять в~графическом виде, чтобы помочь врачам лучше понять 
MRR. Различные подходы к~уменьшению размерности помогут выполнить это 
требование.
     
     Необходима дальнейшая разработка пояснительных инструментов, 
способных воспринять результаты анализа MRR. При этом дополнительно 
необходима информация о~том, какие именно тес\-ты являются важнейшими 
факторами нарушения нормы. Надо признать, что соответствующий 
математический аппарат еще предстоит разработать. Решение перечисленных 
вопросов играет важную роль для обеспечения постоянного клинического 
применения MRR. 

\vspace*{-6pt}
     
{\small\frenchspacing
 {%\baselineskip=10.8pt
 \addcontentsline{toc}{section}{References}
 \begin{thebibliography}{99}
 
 \vspace*{-2pt}
 
\bibitem{1-kri}
\Au{Boyd J.\,C.} Reference regions of two or more dimensions~// Clin. Chem. Lab. 
Med., 2004. Vol.~42. No.\,7. P.~739--746.
\bibitem{2-kri}
\Au{Winkel P.} Patterns and clusters~--- multivariate approach for interpreting 
clinical chemistry results~// Clin. Chem., 1973. Vol.~19. No.\,12. P.~1329--1333.
\bibitem{3-kri}
IFCC. Expert panel on theory of reference values. Approved recommendation on the 
theory of reference values. Part~5. Statistical treatment of collected reference values. 
Determination of reference limits~// J.~Clin. Chem. Clin. Biochem., 1987. Vol.~25. 
No.\,9. P.~645--656.
\bibitem{4-kri}
\Au{Кривенко М.\,П.} Статистические методы представления и~предварительной 
обработки референсных значений.~--- М.: ФИЦ ИУ РАН, 2016. 160~с.
\bibitem{5-kri}
\Au{Boyd J.\,C., Lacher~D.\,A.} The multivariate reference range: An alternative 
interpretation of multi-test profiles~// Clin. Chem., 1982. Vol.~28. No.\,2.  
P.~259--265.
\bibitem{6-kri}
\Au{Albert A., Harris~E.\,K.} Multivariate interpretation of clinical laboratory  
data.~--- New York, NY, USA: CRC Press, 1987. 328~p.
\bibitem{7-kri}
\Au{Linnet K.} Influence of sampling variation and analytical errors on the 
performance of the multivariate reference region~// Meth. Inf. Med., 1988. Vol.~27. 
No.\,1. P.~37--42.
\bibitem{8-kri}
\Au{Durbridge T.\,C.} Clinical acceptance of a multi-test reference region for 
biochemical-panel results~// Clin. Chem., 1983. Vol.~29. No.\,10. P.~1724--1726.
\bibitem{9-kri}
\Au{Кривенко М.\,П.} Критерии значимости отбора признаков классификации~// 
Информатика и~её применения, 2016. Т.~10. Вып.~3. С.~32--40.
\bibitem{10-kri}
\Au{Кривенко М.\,П., Голованов~С.\,А., Сивков~А.\,В.} Анализ однородности 
данных о химическом составе камней при уролитиазе~// Информатика и~её 
применения, 2013. Т.~7. Вып.~4. С.~94--104.
 \end{thebibliography}

 }
 }

\end{multicols}

\vspace*{-10pt}

\hfill{\small\textit{Поступила в~редакцию 5.12.16}}

\vspace*{4pt}

%\newpage

%\vspace*{-24pt}

\hrule

\vspace*{2pt}

\hrule

\vspace*{-3pt}


\def\tit{HIGH-DENSITY MULTIVARIATE REFERENCE REGION\\[-5pt]}

\def\titkol{High-density multivariate reference region}

\def\aut{M.\,P.~Krivenko\\[-7pt]}

\def\autkol{M.\,P.~Krivenko}

\titel{\tit}{\aut}{\autkol}{\titkol}

\vspace*{-16pt}


\noindent
Institute of Informatics Problems, Federal Research Center 
``Computer Science and Control'' of the Russian
Academy of Sciences,  44-2~Vavilov Str., Moscow 119333, Russian Federation



\def\leftfootline{\small{\textbf{\thepage}
\hfill INFORMATIKA I EE PRIMENENIYA~--- INFORMATICS AND
APPLICATIONS\ \ \ 2017\ \ \ volume~11\ \ \ issue\ 2}
}%
 \def\rightfootline{\small{INFORMATIKA I EE PRIMENENIYA~---
INFORMATICS AND APPLICATIONS\ \ \ 2017\ \ \ volume~11\ \ \ issue\ 2
\hfill \textbf{\thepage}}}

\vspace*{2pt}




\Abste{The paper considers the principles of construction of multivariate 
reference regions. An original method of construction of 
a~region on the basis of areas of high density of points and approximation 
of data distribution with a~mixture of normal distributions is suggested. 
To estimate the threshold for the probability density, the bootstrap method is used. 
As an experiment, the paper considers the problem of description and use of 
the reference region for predicting the type of urinary stones. 
Real data treatment demonstrated the benefits of the proposed solutions.}

\KWE{multivariate reference region; high-density region; bootstrap method; 
multivariate normal mixture}

\DOI{10.14357/19922264170207} 

%\vspace*{-18pt}

%\Ack
%\noindent



%\vspace*{3pt}

  \begin{multicols}{2}

\renewcommand{\bibname}{\protect\rmfamily References}
%\renewcommand{\bibname}{\large\protect\rm References}

{\small\frenchspacing
 {%\baselineskip=10.8pt
 \addcontentsline{toc}{section}{References}
 \begin{thebibliography}{99}
\bibitem{1-kri-1}
\Aue{Boyd, J.\,C.} 2004. Reference regions of two or more dimensions. \textit{Clin. 
Chem. Lab. Med.} 42(7):739--746.

\bibitem{2-kri-1}
\Aue{Winkel, P.} 1973. Patterns and clusters~--- multivariate approach for interpreting 
clinical chemistry results. \textit{Clin. Chem.} 19(12):1329--1333.
\bibitem{3-kri-1}
IFCC. 1987. Expert panel on theory of reference values. Approved recommendation on the 
theory of reference values. Part~5. Statistical treatment of collected reference values. 
Determination of reference limits. \textit{J.~Clin. Chem. Clin. Biochem.} 
25(9):645--656.
\bibitem{4-kri-1}
\Aue{Krivenko, M.\,P.} 2016. \textit{Statisticheskie metody predstavleniya 
i~predvaritel'noy obrabotki referensnykh znacheniy}
[Statistical methods for representation and preliminary processing of
reference values]. Moscow: FRC CSC RAS. 160~p.

\bibitem{5-kri-1}
\Aue{Boyd, J.\,C., and D.\,A.~Lacher.} 1982. The multivariate reference range: An 
alternative interpretation of multi-test profiles. \textit{Clin. Chem.}  
28(2):259--265.
\bibitem{6-kri-1}
\Aue{Albert, A., and E.\,K.~Harris.} 1987. \textit{Multivariate interpretation of 
clinical laboratory data}. New York, NY: CRC Press. 328~p.
\bibitem{7-kri-1}
\Aue{Linnet, K.} 1988. Influence of sampling variation and analytical errors on the 
performance of the multivariate reference region. \textit{Meth. Inf. Med.}  
27(1):37--42.
\bibitem{8-kri-1}
\Aue{Durbridge, T.\,C.} 1983. Clinical acceptance of a multi-test reference region 
for biochemical-panel results. \textit{Clin. Chem.} 29(10):1724--1726.
\bibitem{9-kri-1}
\Aue{Krivenko, M.\,P.} 2016. Kriterii znachimosti otbora priznakov klassifikatsii
[Significance tests of feature selection for~classification]. \textit{Informatika i~ee 
Primeneniya~--- Inform. Appl.} 10(3):32--40.
\bibitem{10-kri-1}
\Aue{Krivenko, M.\,P., S.\,A.~Golovanov, and A.\,V.~Sivkov}. 2013. Analiz 
odnorodnosti dannykh o~khimicheskom sostave kamney pri urolitiaze
[Analysis of data homogeneity of~the~chemical compositions 
of~stones in~case of~urolithiasis]. \textit{Informatika i~ee Primeneniya~---
Inform Appl.} 7(4):94--104.
\end{thebibliography}

 }
 }

\end{multicols}

\vspace*{-3pt}

\hfill{\small\textit{Received December 5, 2016}}


\Contrl

\noindent
\textbf{Krivenko Michail P.} (b.\ 1946)~--- Doctor of Science in technology, 
professor, leading scientist, Institute of Informatics Problems, Federal Research 
Center ``Computer Science and Control'' of the Russian Academy of Sciences, 
\mbox{44-2}~Vavilov Str., Moscow 119333, Russian Federation; \mbox{mkrivenko@ipiran.ru}

\label{end\stat}


\renewcommand{\bibname}{\protect\rm Литература}  %4
\def\stat{senko}

\def\tit{ИССЛЕДОВАНИЕ ВОЗМОЖНОСТИ ПРОГНОЗИРОВАНИЯ ИЗМЕНЕНИЯ ФИНАНСОВОГО 
СОСТОЯНИЯ КРЕДИТНОЙ ОРГАНИЗАЦИИ НА~ОСНОВЕ ПУБЛИКУЕМОЙ ОТЧЕТНОСТИ$^*$}

\def\titkol{Исследование возможности прогнозирования изменения финансового 
состояния кредитной организации} % на основе публикуемой отчетности}

\def\aut{Ю.\,И.~Журавлев$^1$, О.\,В.~Сенько$^2$, Н.\,Н.~Бондаренко$^3$, 
В.\,В.~Рязанов$^4$, А.\,А.~Докукин$^5$, А.\,П.~Виноградов$^6$}

\def\autkol{Ю.\,И.~Журавлев, О.\,В.~Сенько, Н.\,Н.~Бондаренко и~др.}


\titel{\tit}{\aut}{\autkol}{\titkol}

\index{Журавлев Ю.\,И.}
\index{Сенько О.\,В.}
\index{Бондаренко Н.\,Н.}
\index{Рязанов В.\,В.}
\index{Докукин А.\,А.}
\index{Виноградов А.\,П.}
\index{Zhuravlev Yu.\,I.}
\index{Sen'ko O.\,V.} 
\index{Bondarenko N.\,N.}
\index{Ryazanov V.\,V.}
\index{Dokukin A.\,A.}
\index{Vinogradov A.\,P.}


{\renewcommand{\thefootnote}{\fnsymbol{footnote}} \footnotetext[1]
{Работа выполнена при частичной финансовой поддержке РФФИ (проект 18-29-03151).}}


\renewcommand{\thefootnote}{\arabic{footnote}}
\footnotetext[1]{Федеральный исследовательский центр <<Информатика и~управ\-ле\-ние>>
  Российской академии наук; 
Московский государственный университет им.\ М.\,В.~Ломоносова, \mbox{zhur@ccas.ru}}
\footnotetext[2]{Федеральный исследовательский центр <<Информатика и~управ\-ле\-ние>>
  Российской академии наук, \mbox{senkoov@mail.ru}}
\footnotetext[3]{Московский государственный университет им.\ М.\,В.~Ломоносова, \mbox{kolianmos1@gmail.com}}
\footnotetext[4]{Федеральный исследовательский центр <<Информатика и~управ\-ле\-ние>>
  Российской академии наук, 
\mbox{rvvccas@mail.ru}}
\footnotetext[5]{Федеральный исследовательский центр <<Информатика и~управ\-ле\-ние>>
  Российской академии наук, 
\mbox{dalex@ccas.ru}}
\footnotetext[6]{Федеральный исследовательский центр <<Информатика и~управ\-ле\-ние>>
  Российской академии наук, 
\mbox{vngrccas@mail.ru}}

\vspace*{-3pt}

   
       
       \Abst{Рассматривается математическая модель для прогноза отзыва лицензии 
кредитной организации на период до 6~месяцев по данным из публикуемой отчетности 
кредитных организаций. Модель является коллективным решением по набору 
ком\-би\-на\-тор\-но-ло\-ги\-че\-ских методов распознавания и~ре\-ша\-ющих лесов различного типа. 
Оценка эффективности разработанной коллективной модели по показателю ROC AUC 
(area under receiver operating characteristic curve)
составила~0,74. Модель позволяет выделять группы кредитных организаций 
с~повышенным и~пониженным риском отзыва лицензии. Было проведено ранжирование 
различных показателей, показавшее важность величины ликвидных и~высоколиквидных 
активов.}
        
       \KW{прогнозирование; коллективные методы; финансовое состояние; кредитная 
организация}

\DOI{10.14357/19922264190405} 
  
\vspace*{-3pt}


\vskip 10pt plus 9pt minus 6pt

\thispagestyle{headings}

\begin{multicols}{2}

\label{st\stat}
     
      
\section{Введение }

Банковский сектор представляет собой крупнейшую по объему средств часть 
финансового рынка России, и~трудно переоценить важность деятельности 
кредитных организаций для экономики\linebreak страны в~целом и~отдельных 
компаний и~граждан в~частности. Банком России на протяжении последних 
лет проводилась работа по выводу с~рынка недобросовестных и~финансово 
несостоятельных участников: у кредитных организаций отзывались\linebreak лицензии 
на осуществление банковских операций или проводились мероприятия по 
финансовому оздо\-ров\-ле\-нию с~приостановкой полномочий бывших 
собственников и~руководства. Как правило, данные события были связаны 
с~ухудшением финансового состояния кредитной организации. Однако для 
клиентов банков момент наступления\linebreak такого негативного события мог стать 
полной неожиданностью. 

Естественно, что деятельность банковского %\linebreak 
надзора, связанная с~банковской тайной и~информацией ограниченного 
доступа, является не\-пуб\-лич\-ной. Поэтому невозможно заранее сказать, 
безопас\-но ли хранить средства (в~размере, превышающем объем страховой 
ответственности Государственной корпорации
<<Агентство по страхованию вкладов>>) в~данном банке или нет. Однако существуют 
раз\-личные методики, в~частности у кредитных рейтинговых агентств, 
позволяющие на основе количественных и~качественных показателей 
оценить вероятность наступления дефолта кредитной организации на 
определенном горизонте. Следует отметить, что отзыв лицензии может 
произойти и~без наступления дефолта кредитной организации. В~связи 
с~этим возникает потребность исследования возможности прогнозирования 
ухудшения финансового состояния кредитной организации заранее.

 Для 
решения этой задачи объективно могут использо\-вать\-ся средства машинного 
обучения. Их очевидными преимуществами по сравнению с~экспертными 
оценками являются объ\-ек\-тив\-ность, универсаль\-ность, относительно низкая 
сто\-и\-мость, воз\-мож\-ность быст\-рой коррекции алгоритмов прогнозирования по 
мере поступления новой информации. 

Перечисленные преимущества не 
могут не привлекать внимание исследователей и~вызывают появление работ 
по тематике использования методов компьютерного обучения для решения 
задач прогнозирования в~банковской сфере~[1, 2]. 

      В настоящее время существует большое число технологий обучения, 
а~также средств статистически корректной оценки эффективности 
полученных решений~[3, 4]. Большое значение имеют также сопутствующие 
обучению методы ранжирования показателей по их значимости при 
прогнозировании. Такое разнообразие технологий вызывает необходимость 
исследования как их эф\-фек\-тив\-ности по отдельности, так и~выбора 
оптимальной схемы построения коллективных решений~\cite{3-sen}.
      
\section{Использование для~прогноза методов распознавания }

 Сформулируем поставленную выше задачу прогнозирования ухудшения 
состояния кредитных организа\-ций как задачу предсказания отзыва лицензии 
по данным из публикуемой отчетности кредитных организаций. 

Имеется 
набор объектов (кредитных организаций) с~признаковым описанием 
(показатели отчетности\footnote{Отдельные показатели деятельности 
кредитной организации, используемые для расчета обязательных нормативов 
из разд.~2 отчетности банков по форме 0409135 <<Информация об 
обязательных нормативах>>, составляемой в~соответствии с~Указанием 
Банка России от 24.11.2016 №\,4212-У <<О~перечне, формах и~порядке 
составления и~представления форм отчетности кредитных организаций 
в~Центральный банк Российской Федерации>>.}, с~ежемесячной 
периодичностью раз\-ме\-ща\-емые на сайте Банка России\footnote{\sf 
www.cbr.ru.}), для которых\linebreak
 известно значение бинарного признака~--- будет 
ли негативное событие отзыва лицензии у~кредитной организации в~течение 
ближайших 6~месяцев или нет (со\-став\-лен\-ное по информации  
пресс-ре\-ли\-зов с~сайта Банка России). Задача заключается 
в~прогнозировании значения функции, зависящей от признакового описания 
объекта и~принимающей бинарное значений <<да>> или <<нет>> на 
ана\-ли\-зи\-ру\-емую дату с~горизонтом прогнозирования 6~месяцев 
в~зависимости от данных отчетности банка. 

Важно отметить, что причиной 
отзыва лицензии могут стать не только проблемы, связанные с~финансовым 
состоянием кредитной организации, но и~нарушения в~об\-ласти ПОД/ФТ
(противодействия отмыванию доходов и~финансированию
терроризма). 
В~связи с~этим из обучающей выборки были исключены кредитные 
организации, у~которых лицензии были отозваны в~связи с~указанными 
нарушениями.

      Был проведен эксперимент по оценке возможности прогнозирования 
отзыва лицензии, связанного с~финансовым состоянием банка. Прогноз 
проводился на период с~января по июнь 2015~г. В~анализ были включены 
24~банка, для которых отзыв лицензии был осуществлен в~указанный 
период. При этом по экспертной оценке сотрудников Банка России отзыв 
лицензии для этих банков определенно был связан с~их финансовым 
со\-сто\-яни\-ем. Также в~анализ были включены 692~банка, которые продолжали 
действовать без отзыва лицензии в~течение 2~лет с~декабря 2014~г. Для 
прогнозирования использовались параметры банковской отчетности, 
известные на момент времени, в~который производился прогноз. Всего 
в~анализе использовался 31~уникальный показатель банковской отчетности. 
Однако при прогнозировании применялись банковские показатели, 
рассчитанные для месяца, предшествующего дате составления прогноза, 
а~также для месяцев, отстоящих от даты прогноза на~1, 2 и~3 месячных 
интервала. Таким образом, общее число используемых для прогноза 
показателей составило~124.
      
      Задача прогнозирования, очевидно, может быть сведена к~задаче 
распознавания с~двумя классами. При этом задача усложняется малым 
размером целевого класса, а также высокой размерностью данных, связанной 
с~необходимостью учета динамики показателей финансового состояния 
банков.\linebreak Высокая эффективность в~этих условиях может достигать\-ся при 
использовании коллективных решений. Существенным требованием является 
необходимость анализа информативности различных показателей. Данная 
задача осложняется тем, что информативность показателей проявляется 
только в~рамках их взаимодействия. Можно предположить, что для 
повышения достоверности могут быть использованы коллективные методы 
оценивания информативности. Разработка таких методов также является 
одной из целей представляемого исследования.
      
      Для вычисления оптимальных прогнозных решений был 
протестирован ряд разнообразных технологий распознавания. Однако 
возможность получе\-ния эффективного алгоритма прогнозирования удалось 
показать только для следующих методов:
      \begin{itemize}
      \item логистическая регрессия (LogReg);
      \item алгоритм вычисления оценок с~использованием всевозможных 
наборов признаков в~качестве опорных множеств (АВО)~\cite{3-sen, 5-sen};
      \item решающий лес, использующий бэггинг для генерации ансамблей 
деревьев (RF)~\cite{4-sen};
      \item решающий лес, основанный на процедуре адап\-тив\-но\-го бустинга 
(АdaBoost)~\cite{4-sen, 7-sen};
      \item решающий лес, основанный на процедуре градиентного 
бустинга (GradBoost)~\cite{4-sen, 8-sen};
      \item метод статистически взвешенных синдромов  
(СВС)~\cite{3-sen, 6-sen}.
      \end{itemize}
      
      Оценка точности прогноза проводилась с~использованием метода 
кросс-валидации со 100~фолдами. Для оценивания результатов была 
использована известная метрика ROC AUC. Результаты приведены в~табл.~1.

  
      
%\begin{table*}
{ %\small %tabl1
\begin{center}

\parbox{50mm}{{{\tablename~1}\ \ \small{Оценка точности прогноза различными методами}}

}

\vspace*{6pt}

\small 
\begin{tabular}{|l|c|c|}
\hline
\multicolumn{1}{|c|}{Метод}& ROC AUC & Ранг\\
\hline
LogReg& 0,626& 1\\
RF& 0,707& 5\\
GradBoost& 0,662& 2\\
АdaBoost& 0,716& 6\\
АВО& 0,679& 3\\
СВС& 0,698& 4\\
\hline
\end{tabular}
\end{center}
}
%\end{table*}
      
\section{Коллективное решение }

      На основе полученных данных строилось коллективное решение. На 
первом этапе проводился поиск оптимального порога~$b_*$в решающем 
правиле для каждого из 6~алгоритмов. Подбор порога проводился из условия 
минимальности различия между чувствительностью и~специфичностью. 
Коллективная оценка объекта, описываемая вектором 
признаков~$\mathbf{x}_j$, вычислялась в~два этапа. На первом этапе по 
оценке $\gamma_*(\mathbf{x}_j)$, полученной с~помощью алгоритма~$A_*$, 
вычислялось значение бинарного показателя~$\beta_*$, указывающего на 
негативный прогноз для объекта~$\mathbf{x}_j$ при 
$\beta_*(\mathbf{x}_j)\hm=1$ и~на положительный прогноз при 
$\beta_*(\mathbf{x}_j)\hm=0$. Вычисление~$\beta_*$ проводилось по схеме: 
$$
\beta_*(\mathbf{x}_j)=
\begin{cases}
1 & \mbox{при } \gamma_*(\mathbf{x}_j)>b_*\,;\\
0 & \mbox{в~противном~случае}.
\end{cases}
$$

 Каждому из алгоритмов 
сопоставлялся весовой коэффициент~$\theta_*$. Рассматривался следующий 
способ задания весовых коэффициентов: алгоритмы ранжировались по 
величине ROC AUC. Значение коэффициента~$\theta_*$ для 
алгоритма~$A_*$ приравнивалось рангу~$A_*$ из табл.~1. Коллективная 
оценка $\gamma_{\mathrm{int}}(\mathbf{x}_j)$ вычислялась по формуле:
      \begin{multline*}
      \gamma_{\mathrm{int}}(\mathbf{x}_j) =
      \theta_{\mathrm{свс}} \beta_{\mathrm{свс}} 
(\mathbf{x}_j) +\theta_{\mathrm{GB}}\beta_{\mathrm{GB}}(\mathbf{x}_j) +{}\\
{}+\theta_{\mathrm{AB}}  \beta_{\mathrm{AB}} (\mathbf{x}_j) +
\theta_{\mathrm{RF}}  \beta_{\mathrm{RF}} (\mathbf{x}_j) +
\theta_{\mathrm{LR}}\beta_{\mathrm{LR}}(\mathbf{x}_j)+{}\\
{}+
\theta_{\mathrm{ABO}}\beta_{\mathrm{ABO}}  (\mathbf{x}_j)\,.
      \end{multline*}
      
      %\begin{table*}
{%tabl2
\begin{center}
%\vspace*{1pt}

\parbox{70mm}{{{\tablename~2}\ \ \small{Уровень риска отзыва лицензии для различных интервалов балльных оценок}}

}

\vspace*{6pt}

\small
\begin{tabular}{|c|c|c|}
\hline
\multicolumn{1}{|c|}{\raisebox{-6pt}[0pt][0pt]{Баллы}}& \multicolumn{2}{c|}{Количество отозванных лицензий}\\
\cline{2-3}
&\hspace*{10mm}шт.\hspace*{10mm} &\%\\
\hline
$\geq 20$& 5 из 19\hphantom{9} & 26,3\hphantom{9}\\
От 17 до 19& 12 из 63\hphantom{99} & 19\hphantom{99,}\\
От 3 до 16 & 9 из 401& 2,2\\
$<3$& 3 из 252 & 1,2\\
\hline
\end{tabular}
\end{center}}
%\end{table*}
\vspace*{12pt}

\noindent
Величина ROC AUC для коллективного решения составила~0,74. 
     
     Коллективные оценки риска оценивались по шкале от~0 до~21~балла, 
где 21~балл соответствовал негативному прогнозу, а~0~баллов 
соответствовали позитивному прогнозу. Из табл.~2 видно, что из 19~банков 
с~21 баллом лицензия была отозвана у~5, что составляет~26,3\%. В~группе 
из 252~банков с~менее чем тремя баллами лицензия была отозвана только 
у~трех банков (1,2\%). Таким образом, коллективное решение отчетливо 
выделяет группы с~пониженным и~повышенным риском отзыва лицензии. 


      
      Для оценивания информативности признаков использовались 
коллективные оценки ин\-фор\-ма\-тив\-ности, включающие оценки, полученные %\linebreak 
с~помощью всех трех используемых вариантов ре\-ша\-ющих лесов, а~также 
метода СВС. 

В~методах {решающих} лесов информативность признаков 
вычисляется как среднее значение показателей информативности, 
рассчитанных для отдельных %\linebreak 
деревьев и~характеризующих улучшение 
аппроксимации данных после включения признака в~модель~\cite{9-sen}. 

В~методе СВС показателем информативности служит значение 
статистики~$\chi^2$ при сравнении распределения целевого класса в~группах 
слева и~справа от рассчитанного для признака оптимального  
порога~\cite{6-sen}.
      
      Таким образом, информативность признаков рассчитывалась отдельно 
для решающих лесов, основанных на бэггинге, адаптивном или градиентном 
бустинге, а также для СВС. Обозначим показатели информативности по этим 
методам соответственно как $I_{\mathrm{RF}}$, $I_{\mathrm{AB}}$, $I_{\mathrm{GB}}$ 
и~$I_{\mathrm{свс}}$. Далее проводилось ранжирование признаков по 
величине каждого из четырех перечисленных показателей. Ранги 
по~$I_{\mathrm{RF}}$, $I_{\mathrm{AB}}$, $I_{\mathrm{GB}}$ и~$I_{\mathrm{свс}}$ обозначим 
как~$R_{\mathrm{RF}}$, $R_{\mathrm{AB}}$, $R_{\mathrm{GB}}$ и~$R_{\mathrm{свс}}$.
      
      Интегральный показатель информативности для $I_{\mathrm{int}}(X_j)$ 
признака~$X_j$ вычислялся как сумма рангов по каждому из четырех 
показателей информативности: 
\begin{multline*}
I_{\mathrm{int}}\left(X_i\right)= {}\\
{}=
R_{\mathrm{свс}}\left(X_i\right)+ R_{\mathrm{GB}}\left(X_i\right)+ 
R_{\mathrm{AB}}\left(X_i\right)+R_{\mathrm{RF}}\left(X_i\right)\,.
\end{multline*}
      
      Признаки, имеющие ранги от одного до пяти, приведены в~табл.~3.
      
\setcounter{table}{2}
\begin{table*}\small %tabl3
\begin{center}
\Caption{Наиболее информативные показатели}
\vspace*{2ex}

\begin{tabular}{|c|c|p{115mm}|}
\hline
Ранг&\tabcolsep=0pt\begin{tabular}{c}Интегральный\\ 
показатель\\ информативности\end{tabular}&\multicolumn{1}{c|}{Показатель 
банковской отчетности}\\
\hline
1&\hphantom{9}4&Высоколиквидные активы за месяц, предшествующий моменту прогноза\\
\hline
2&18&Ликвидные активы за месяц, предшествующий моменту прогноза\\
\hline
\multicolumn{1}{|c|}{\raisebox{-6pt}[0pt][0pt]{3}} &
\multicolumn{1}{c|}{\raisebox{-6pt}[0pt][0pt]{19}}&Активы II группы, взвешенные с~коэффициентом 20\% (мера риска), за месяц, 
отстоящий на один месячный интервал от даты прогноза\\
\hline
\multicolumn{1}{|c|}{\raisebox{-6pt}[0pt][0pt]{4}}&
\multicolumn{1}{c|}{\raisebox{-6pt}[0pt][0pt]{40}}&Активы, имеющие нулевой коэффициент риска за месяц, отстоящий на четыре 
месячных интервала от даты прогноза\\
\hline
\multicolumn{1}{|c|}{\raisebox{-6pt}[0pt][0pt]{5}}&
\multicolumn{1}{c|}{\raisebox{-6pt}[0pt][0pt]{49}}&Активы II группы, взвешенные с~коэффициентом 20\% (мера риска) за месяц, 
отстоящий на три месячных интервала от даты прогноза\\
\hline
\end{tabular}
\end{center}
\vspace*{-3pt}
\end{table*}

\vspace*{-6pt}
      
\section{Заключение}

\vspace*{-2pt}

      Как видно из полученных результатов, разработанная коллективная 
модель по группе методов распознавания по данным отчетности позволяет 
с~некоторой долей уверенности предсказать отзыв лицензии у кредитной 
организации. Возможность эффективного прогнозирования можно связать 
с~проводимой Банком России работой над достоверностью отчетности 
участников финансового рынка. Разработанная методика может быть полезна 
как для банковского надзора, так и~для участников финансового рынка. 
Однако для повышения точности прогнозирования негативных событий, 
безусловно, недостаточно следить только за значениями показателей 
отчетности: важно использовать механизм риск-ана\-ли\-ти\-ки, а~также 
формировать доверительную среду на финансовом рынке.

\vspace*{-6pt}
      
   {\small\frenchspacing
 {%\baselineskip=10.8pt
 \addcontentsline{toc}{section}{References}
 \begin{thebibliography}{9}
 
 \vspace*{-2pt}
 
 
\bibitem{1-sen}
\Au{Ясницкий Л.\,Н., Иванов~Д.\,В., Липатова~Е.\,В.} Нейросетевая система оценки 
вероятности банкротства банков~// Биз\-нес-ин\-фор\-ма\-ти\-ка, 2014.  Т.~3. №\,29. 
С.~49--56.
{\looseness=1

}
\bibitem{2-sen}
\Au{Синельникова-Мурылева Е.\,В., Горшкова~Т.\,Г., Ма\-ке\-ева~Н.\,В.} Прогнозирование 
дефолтов в~российском банковском секторе~// Экономическая политика, 2018. Т.~2. 
№\,13. С.~8--27.
\bibitem{3-sen}
\Au{Журавлев Ю.\,И., Рязанов~В.\,В., Сенько~О.\,В.} Распознавание: Математические 
методы. Программная система. Применения.~--- M.: Фазис, 2006. 159~c.
\bibitem{4-sen}
\Au{Hastie T., Tibshirani~R., Friedman~J.} The elements of statistical learning: Data 
mining, inference, and prediction.~--- Springer, 2009. 745~p.
\bibitem{5-sen}
\Au{Журавлев Ю.\,И.} Об алгебраическом подходе к~решению задач распознавания или 
классификации~// Проб\-ле\-мы кибернетики, 1978. №\,33. С.~5--68.

\bibitem{7-sen} %6
\Au{Freund Y., Schapire~R.} A~decision-theoretic generalization of on-line learning and an 
application to boosting~// J.~Comput. Syst. Sci., 1997. Vol.~55. P.~119--139.
\bibitem{8-sen} %7
\Au{Friedman J.} Greedy function approximation: A~gradient boosting machine~//  
Ann. Stat., 2001. Vol.~5. Iss.~29. P.~1189--1232.

\bibitem{6-sen} %8
\Au{Кузнецов В.\,А., Сенько~О.\,В., Кузнецова~А.\,В. и~др.} Распознавание нечетких 
систем по методу статистически взвешенных синдромов и~его применение для 
иммуногематологической нормы и~хронической патологии~// Хим. физика, 
1996. Т.~15. №\,1. С.~81--100.

\bibitem{9-sen}
\Au{Louppe G.} Understanding random forests: From theory to practice.~--- 
Liege: University of Liege, 2014.  PhD Thesis. 223~p.
 \end{thebibliography}

 }
 }

\end{multicols}

\vspace*{-6pt}

\hfill{\small\textit{Поступила в~редакцию 04.02.19}}

%\vspace*{8pt}

%\pagebreak

\newpage

\vspace*{-28pt}

%\hrule

%\vspace*{2pt}

%\hrule

%\vspace*{-2pt}

\def\tit{RESEARCH OF~THE~POSSIBILITY TO~FORECAST CHANGES 
IN~FINANCIAL STATE OF~A~CREDIT ORGANIZATION\\ ON~THE~BASIS 
OF~PUBLIC FINANCIAL STATEMENTS}


\def\titkol{Research of~the~possibility to~forecast changes in 
financial state of a credit organization on~the~basis 
of~public financial statements}

\def\aut{Yu.\,I.~Zhuravlev$^{1,2}$, O.\,V.~Sen'ko$^1$, 
N.\,N.~Bondarenko$^2$, V.\,V.~Ryazanov$^1$, A.\,A.~Dokukin$^1$, 
and~A.\,P.~Vinogradov$^1$}

\def\autkol{Yu.\,I.~Zhuravlev, O.\,V.~Sen'ko, 
N.\,N.~Bondarenko, et al.}
%V.\,V.~Ryazanov$^1$, A.\,A.~Dokukin$^1$, 
%and~A.\,P.~Vinogradov$^1$}

\titel{\tit}{\aut}{\autkol}{\titkol}

\vspace*{-11pt}



       \noindent
      $^1$Federal Research Center ``Computer Science and Control'' of the 
Russian Academy of Sciences, 44-2~Vavilov\linebreak
$\hphantom{^1}$Str., Moscow 119333, Russian 
            Federation
        
        \noindent
        $^2$M.\,V.~Lomonosov Moscow State University, 1-52~Leninskie Gory, GSP-1, 
Moscow 119991, Russian Federation

\def\leftfootline{\small{\textbf{\thepage}
\hfill INFORMATIKA I EE PRIMENENIYA~--- INFORMATICS AND
APPLICATIONS\ \ \ 2019\ \ \ volume~13\ \ \ issue\ 4}
}%
 \def\rightfootline{\small{INFORMATIKA I EE PRIMENENIYA~---
INFORMATICS AND APPLICATIONS\ \ \ 2019\ \ \ volume~13\ \ \ issue\ 4
\hfill \textbf{\thepage}}}

\vspace*{3pt} 


      
      \Abste{The mathematical model for forecasting of license revocation of 
a~credit organization in the 6-month period based on public financial statements 
is considered. The model represents an ensemble of combinatorial and logical 
methods and decision trees of different types. Its effectiveness estimated by ROC 
AUC 
(area under receiver operating characteristic curve)
is~0.74. The model allows distinguishing groups of credit organizations with 
higher and lower license revocation risks. Also, the ranking of different financial 
statement indicators has been performed which marked the importance of liquid 
and highly liquid assets.}
      
      \KWE{forecasting; algorithm ensembles; financial state; credit 
organization}
      
      
      
       \DOI{10.14357/19922264190405} 

%\vspace*{-14pt}

 \Ack
      \noindent
       The research has been carried out with the partial financial support of the 
Russian Foundation for Basic Research (project 18-29-03151).



%\vspace*{-6pt}

  \begin{multicols}{2}

\renewcommand{\bibname}{\protect\rmfamily References}
%\renewcommand{\bibname}{\large\protect\rm References}

{\small\frenchspacing
 {%\baselineskip=10.8pt
 \addcontentsline{toc}{section}{References}
 \begin{thebibliography}{9}
\bibitem{1-sen-1}
\Aue{Yasnitskiy, L.\,N., D.\,V.~Ivanov, and E.\,V.~Lipatova.} 2014. Neyrosetevaya  
sistema otsenki veroyatnosti bankrotstva bankov [Neural network designed 
to estimate probability of bank bankruptcies]. \textit{Biznes-informatika} [Business Informatics] 
3(29):49--56.
\bibitem{2-sen-1}
\Aue{Sinel'nikova-Muryleva, E.\,V., T.\,G.~Gorshkova, and N.\,V.~Makeeva}. 2018. 
Prognozirovanie defoltov v~rossiyskom bankovskom sektore [Default 
forecasting in the Russian banking sector]. \textit{Ekonomicheskaya politika}
[Economic Policy] 2(13):8--27.
\bibitem{3-sen-1}
\Aue{Zhuravlev, Yu.\,I., V.\,V.~Ryazanov, and O.\,V.~Sen'ko}. 2006. \textit{Raspoznavanie. 
Matematicheskie metody. Programmnaya sistema. Primeneniya} 
[Recognition. Mathematical methods. Software system. Applications].
Moscow: Fazis. 159~p.
\bibitem{4-sen-1}
\Aue{Hastie, T., R.~Tibshirani, and J.~Friedman.} 2009. 
\textit{The elements of statistical 
learning: Data mining, inference, and prediction}. Springer. 745~p.
\bibitem{5-sen-1}
\Aue{Zhuravlev, Yu.\,I.} 1978. Ob algebraicheskom podkhode k~re\-she\-niyu zadach 
raspoznavaniya ili klassifikatsii [On algebraic approach to recognition and 
classification problems]. \textit{Problemy kibernetiki} [Cybernetic Problems] 
33:5--68.

\bibitem{7-sen-1} %6
\Aue{Freund, Y., and R.~Schapire}. 1997. A~decision-theoretic generalization of 
on-line learning and an application to boosting. \textit{J.~Comput. Syst. Sci.} 
55:119--139.
\bibitem{8-sen-1} %7
\Aue{Friedman, J.} 2001. Greedy function approximation: A~gradient boosting 
machine. \textit{Ann. Stat.} 5(29):1189--1232.

\bibitem{6-sen-1} %8
\Aue{Kuznetsov, V.\,A., O.\,V.~Sen'ko, A.\,V.~Kuznetsova, \textit{et al.}} 1996. 
Recognition of fuzzy systems by the method of 
statistically weighed syndromes and its application to immunohematological 
characterization of the
norm and chronical pathology]. \textit{Chem. Phys. Rep.} 
15(1):87--107.

\bibitem{9-sen-1}
\Aue{Louppe, G.} 2014. Understanding random forests: From theory to 
practice.  Liege: University of Liege.  PhD Thesis. 223~p.
 \end{thebibliography}

 }
 }

\end{multicols}

%\vspace*{-7pt}

\hfill{\small\textit{Received February 4, 2019}}

%\pagebreak

%\vspace*{-22pt}

\Contr


\noindent
\textbf{Zhuravlev Yuri  I.} (b.\ 1935)~--- Doctor of Science in physics and 
mathematics, Academician of RAS, principal scientist, Federal Research Center 
``Computer Science and Control'' of the Russian Academy of Sciences,  
44-2~Vavilov Str., Moscow 119333, Russian Federation; honorary professor, head 
of the Mathematical Methods of Forecasting Department, M.\,V.~Lomonosov 
Moscow State University, 1-52~Leninskie Gory, GSP-1, Moscow 119991, 
Russian Federation; \mbox{zhur@ccas.ru}

\vspace*{3pt}

\noindent
\textbf{Sen'ko Oleg V.} (b.\ 1957)~--- Doctor of Science in physics and 
mathematics, leading scientist, Federal Research Center 
``Computer Science and Control'' of the Russian Academy of Sciences,  
44-2~Vavilov Str., Moscow 119333, Russian Federation; 
\mbox{senkoov@mail.ru}

\vspace*{3pt}

\noindent
\textbf{Bondarenko Nikolaj N.} (b.\ 1990)~--- postgraduate student, 
M.\,V.~Lomonosov Moscow State University, \mbox{1-52}~Leninskie Gory, GSP-1, 
Moscow 119991, Russian Federation; \mbox{kolianmos1@gmail.com} 

\vspace*{3pt}

\noindent
\textbf{Ryazanov Vladimir V.} (b.\ 1950)~--- Doctor of Science in physics and 
mathematics, principal scientist, Federal Research Center 
``Computer Science and Control'' of the Russian Academy of Sciences,  
44-2~Vavilov Str., Moscow 119333, Russian Federation; 
\mbox{rvvccas@mail.ru}

\vspace*{3pt}

\noindent
\textbf{Dokukin Alexander A.} (b.\ 1980)~--- Candidate of Science (PhD) in 
physics and mathematics, senior scientist, Federal Research Center ``Computer 
Science and Control'' of the Russian Academy of Sciences, 44-2~Vavilov Str., 
Moscow 119333, Russian Federation; \mbox{dalex@ccas.ru}


\vspace*{3pt}

\noindent
\textbf{Vinogradov Alexander P.}  (b.\ 1951)~--- Candidate of Science (PhD) in 
physics and mathematics, senior scientist, Federal Research Center ``Computer 
Science and Control'' of the Russian Academy of Sciences, 44-2~Vavilov Str., 
Moscow 119333, Russian Federation; \mbox{vngrccas@mail.ru}


\label{end\stat}

\renewcommand{\bibname}{\protect\rm Литература}  

          %5
\def\stat{agasan}

\def\tit{ТЕОРЕТИЧЕСКИЕ ОСНОВЫ ОПТИМИЗАЦИИ ПО~КОНТИНУАЛЬНОМУ КРИТЕРИЮ 
VaR\\ НА~СОВОКУПНОСТИ  РЫНКОВ$^*$}

\def\titkol{Теоретические основы оптимизации по
континуальному критерию VaR на совокупности 
рынков}

\def\aut{Г.\,А.~Агасандян$^1$}

\def\autkol{Г.\,А.~Агасандян}

\titel{\tit}{\aut}{\autkol}{\titkol}

\index{Агасандян Г.\,А.}
\index{Agasandyan G.\,A.}


{\renewcommand{\thefootnote}{\fnsymbol{footnote}} \footnotetext[1]
{Работа выполнена при финансовой поддержке РФФИ (проект 17-01-00816).}}


\renewcommand{\thefootnote}{\arabic{footnote}}
\footnotetext[1]{Вычислительный центр им.~А.\,А.~Дородницына Федерального исследовательского 
центра <<Информатика и управление>> Российской академии наук, 
\mbox{agasand17@yandex.ru}}

\vspace*{-12pt}

  
  
  \Abst{Работа продолжает изучение проблем использования континуального критерия VaR 
(CC-VaR) на финансовых рынках. Речь идет о применении CC-VaR на совокупности 
нескольких рынков разных размерностей, связанных между собой базовыми активами. 
В~типовой модели совокупности одного двумерного и двух одномерных теоретических 
рынков рассматривается наиболее общий случай их совместного функционирования. 
Приводится правило построения оптимального по CC-VaR комбинированного портфеля 
с~тремя компонентами. Оно основывается на расхождениях в относительных доходах между 
рынками с~сохранением требований критерия. Оптимальный портфель строится из базисных 
инструментов всех рынков с~ использованием в их конструкциях идей рандомизации. 
Приводятся также его идеалистичная и суррогатная версии, которые могут быть полезными 
при проверке расчетов и для графической иллюстрации платежных функций. Теоретически 
модель без труда распространяется на рынки большей размерности. Возможны и две 
усеченные постановки задачи, в одной из которых исключается один одномерный рынок, 
в~другой~--- двумерный.}
  
  \KW{базовые активы; функция рисковых предпочтений; континуальный критерий VaR; 
стоимостная и прогнозная плотности; функция относительных доходов; процедура  
Ней\-ма\-на--Пир\-со\-на; комбинированный портфель; рандомизация; суррогатный 
портфель; идеалистичный портфель} 

\DOI{10.14357/19922264190406} 
  
\vspace*{-3pt}


\vskip 10pt plus 9pt minus 6pt

\thispagestyle{headings}

\begin{multicols}{2}

\label{st\stat}
  
  \section{Введение}
  
  Работа продолжает исследования по применению введенного автором 
континуального критерия VaR  на финансовых рынках~[1--5] как 
с~одним, так и~с~несколькими базовыми активами~\cite{3-ag}. В~настоящей 
работе изучается оптимальное поведение инвестора, приверженного CC-VaR, 
одновременно на нескольких рынках разных размерностей, связанных между 
собой базовыми активами. В~типовой модели речь идет о совокупности трех 
рынков, один из которых двумерный, а~два других одномерные. Базовые 
активы одномерных рынков образуют пару базовых активов двумерного. 
Такую схему назовем \textit{комбинированным} (и~\textit{тройственным}) 
рынком. Она естественным образом распространяется на совокупности рынков 
больших размерностей, хотя с~их ростом, разумеется, многое в схеме 
технически усложняется.
  
  Для выявления сущности проблемы все рынки рассматриваются 
однопериодными (с~двумя моментами времени~--- началом и концом периода), 
теоретическими (страйки опционов образуют континуальное множество) 
и~идеальными (комиссионные равны нулю, а цены покупателя и продавца 
совпадают). 
  
  Решение ищется в форме совмещения трех портфелей и основывается на 
анализе поточечных расхождений в относительных доходах на рынках, 
обусловленных расхождениями в ценах на разных \mbox{рынках}. 

\vspace*{-6pt}
  
  \section{Исходные теоретические рынки}
  
  \vspace*{-2pt}
  
  Рассматриваются три однопериодных рынка: два одномерных \#X и \#Y 
  с~базовыми активами~$\boldsymbol{X}$ и~$\boldsymbol{Y}$ соответственно 
и~один двумерный \#0 с~парой активов ($\boldsymbol{X}, \boldsymbol{Y}$). 
Цены двух базовых активов обозначаются~$x$ и~$y$, параметры 
инструментов~--- $s$ и~$t$, при этом $x, s \hm\in {\sf X}\hm = [a_1, b_1)$, $y, t 
\hm\in {\sf Y}\hm = [a_2, b_2)$. Рыночная стоимость произвольного 
инструмента~$\boldsymbol{G}$ записывается как~$\vert\boldsymbol{G}\vert$, 
а~средний доход~--- $\|\boldsymbol{G}\|$.
  
  Вкратце напомним обозначения и конструкции рынков с введением 
необходимых дополнений, обусловленных их совместной работой. 
  
  Для рынка \#X заданы \textit{прогнозная} $p_{\mathrm{X}}(x)$ 
и~\textit{стоимостная}~$c_{\mathrm{X}}(x)$, $x\hm\in {\sf X}$, плотности, 
порождающие меры ${\sf C}_{\mathrm{X}}(\cdot)$ и~${\sf 
P}_{\mathrm{X}}(\cdot)$ соответственно. Первая сформирована рынком на 
начало периода, а вторая дает прогноз инвестора на его конец. Важный для 
оптимизации относительный доход $\rho_{\mathrm{X}}(\cdot)\hm=  
p_{\mathrm{X}}(\cdot)/c_{\mathrm{X}}(\cdot)$. На рынке, называемом  
$\delta$-\textit{рын\-ком}, можно торговать любым 
инструментом~$\boldsymbol{G}_{\mathrm{X}}$ с доходом, представимым в виде 
произвольной неотрицательной измеримой функции~$g(x)$, $x\hm\in {\sf X}$. 
Ее называем \textit{платежной функцией} инструмента и обозначаем $\pi(x; 
\boldsymbol{G}_{\mathrm{X}})$, т.\,е.\ $g(x) \hm= \pi(x; 
\boldsymbol{G}_{\mathrm{X}})$, $x\hm\in {\sf X}$, в частности $\pi(x; 
\boldsymbol{X})\hm = x$. 
  
  Базисными на рынке являются инструменты 
$\boldsymbol{D}_{\mathrm{X}}(s)$, $s \hm\in {\sf  X}$, с обобщенной  
$\delta$-функ\-ци\-ей в качестве платежной: $\pi(x; 
\boldsymbol{D}_{\mathrm{X}}(s)) \hm\equiv  \delta(x \hm- s)$. Для них 
  $$\left\vert \boldsymbol{D}_{\mathrm{X}}(s)\right\vert 
=c_{\mathrm{X}}(s)\,,\enskip \left\| 
\boldsymbol{D}_{\mathrm{X}}(s)\right\| =p_{\mathrm{X}}(s)\,,\enskip s\in 
{\sf X}\,.
  $$
  
  Для инструмента $\boldsymbol{G}_{\mathrm{X}}$ с платежной 
функцией~$g(x)$ имеют место соотношения: 
  \begin{align*}
  \boldsymbol{G}_{\mathrm{X}} &= \int\limits_{{\sf X}\times{\sf Y}} 
g_{\mathrm{X}}(s) \boldsymbol{D}_{\mathrm{X}}(s)\,ds\,; \\
  \left\vert \boldsymbol{G}_{\mathrm{X}}\right\vert &=\int\limits_{\sf X} 
g_{\mathrm{X}}(s)c_{\mathrm{X}}(s)\,ds\,;\\
  \left\| \boldsymbol{G}_{\mathrm{X}}\right\| &=\int\limits_{\sf X} 
g_{\mathrm{X}}(s) p_{\mathrm{X}}(s)\,ds\,.
  \end{align*}
  
  Определяются и такие важные для сценарных рынков инструменты, как 
индикаторы множеств $\boldsymbol{H}_{\mathrm{X}}\{M\}$, $M \hm\subset 
{\sf X}$, с~их характеристическими функциями в качестве платежных, а также 
\textit{единичный безрисковый} актив~$\boldsymbol{U}_{\mathrm{X}}$, и~для 
них
  \begin{align*}
  \boldsymbol{H}_{\mathrm{X}}\{M\} &= \int\limits_M 
\boldsymbol{D}_{\mathrm{X}}(s)\,ds\,;\\  
\boldsymbol{U}_{\mathrm{X}} = 
\boldsymbol{H}_{\mathrm{X}}\{{\sf X}\}& =\int\limits_{\sf X} 
\boldsymbol{D}_{\mathrm{X}}(s)\,ds\,;\\
  \left\vert \boldsymbol{H}_{\mathrm{X}}\{M\}\right\vert &= \int\limits_M 
c_{\mathrm{X}}(s)\,ds\,;\\
  \left\vert 
\boldsymbol{U}_{\mathrm{X}}\right\vert
   = \boldsymbol{C}_{\mathrm{X}}\{{\sf X}\} &=\int\limits_{\sf X} 
c_{\mathrm{X}}(s)\,ds\,.
  \end{align*}
  
  Аналогично вводятся агрегаты второго одномерного рынка с очевидной 
заменой $\mathrm{X}\leftrightarrow\mathrm{Y}$, $x\leftrightarrow y$, ${\sf 
X}\leftrightarrow {\sf Y}$, $s\leftrightarrow t$: плотности $p_{\mathrm{Y}}(y)$ 
и~$c_{\mathrm{Y}}(y)$; меры ${\sf P}_{\mathrm{Y}}\{\cdot\}$ и~${\sf 
C}_{\mathrm{Y}}\{\cdot\}$; относительный доход $\rho_{\mathrm{Y}}(y)$; 
инструменты $\boldsymbol{D}_{\mathrm{Y}}(t)$, $y, t \hm\in {\sf Y}$, 
$\boldsymbol{H}_{\mathrm{Y}}\{M\}$, $M \hm\subset {\sf Y}$, 
и~$\boldsymbol{U}_{\mathrm{Y}}$.
  
  Для \textit{двумерного} рынка \#0 задаются двумерные плотности $p(x, y)$ 
и~$c(x, y)$, $x \hm\in {\sf  X}$, $y \hm\in {\sf  Y}$, по\-рож\-да\-ющие меры ${\sf 
P}\{\cdot,\cdot\}$ и~${\sf  C}\{\cdot,\cdot\}$ соответственно, а~относительный 
доход $\rho(\cdot,\cdot)\hm = p(\cdot,\cdot)/c(\cdot,\cdot)$. Базисными 
инструментами служат $\boldsymbol{D}(s, t)$, $s \hm\in{\sf X}$, $t\hm\in{\sf  
Y}$, и для них $\pi(x, y; \boldsymbol{D}(s, t))\hm\equiv \delta(x \hm- s, y \hm- t)$, 
а~также 
  $$
  \left\vert \boldsymbol{D}(s,t)\right\vert =c(s,t)\,,\enskip\!
  \left\| \boldsymbol{D}(s,t)\right\| =p(s,t)\,,\enskip\!
  s\in {\sf X}\,,\ t\in {\sf Y}\,.
  $$
  
  Для инструмента $\boldsymbol{G}$ с платежной функцией $\pi(x, y; 
\boldsymbol{G}) \hm\equiv g(x, y)$ имеем: 

\noindent
  \begin{align*}
  \boldsymbol{G} &= \int\limits_{{\sf X}\times{\sf Y}} g(s,t) 
\boldsymbol{D}(s,t)\,dsdt\,;\\
  \vert \boldsymbol{G}\vert &=\int\limits_{{\sf X}\times {\sf Y}} 
g(s,t)c(s,t)\,dsdt\,;\\
\|\boldsymbol{G}\| &=\int\limits_{{\sf 
X}\times{\sf Y}} g(s,t) p(s,t)\,dsdt\,.
  \end{align*}
  
\vspace*{-2pt}
  
  Для плотностей $p(x, y)$ и $c(x, y)$, $x \hm\in{\sf X}$, $y \hm\in{\sf  Y}$, 
выполняются соотношения:
  $$
  \int\limits_{{\sf X}\times{\sf Y}} p(x,y)\,dxdy=1\,;\enskip
  \int\limits_{{\sf X}\times {\sf Y}} c(x,y)\,dxdy=\fr{1}{r}\,,
  $$
где $r$~--- безрисковый относительный доход за период. Двумерные плотности 
порождают маргинальные плотности $p_1(x)$, $p_2(y)$ и $c_1(x)$, $c_2(y)$, $x 
\hm\in{\sf X}$, $y \hm\in{\sf  Y}$: 

\noindent
\begin{equation}
\left.
\begin{array}{rlrl}
p_1(x) &= \displaystyle\int\limits_{\sf Y} p(x,y)\,dy\,; & p_2(y)&=\displaystyle
 \int\limits_{\sf X} 
p(x,y)\,dx\,;\\[6pt]
     c_1(x) &= \displaystyle\int\limits_{\sf Y} c(x,y)\,dy\,; & c_2(y)&= 
     \displaystyle\int\limits_{\sf X} c(x,y)\,dx\,.
     \end{array}
     \right\}
     \label{e1-ag}
     \end{equation}
     
     \vspace*{-2pt}
  
 \noindent
  При этом
  
  \noindent
  \begin{align*}
  \int\limits_{\sf X} p_1(x)\,dx=\int\limits_{\sf Y} p_2(y)\,dy=\int\limits_{{\sf 
X}\times{\sf Y}} p(x,y)\,dxdy=1\,;\\
  \int\limits_{\sf X} c_1(x)\,dx=\int\limits_{\sf Y} c_2(y)\,dy=\int\limits_{{\sf 
X}\times{\sf Y}} c(x,y)\,dxdy=\fr{1}{r}\,.
  \end{align*}
  
  Вновь, как обычно, без ограничения общности принимаем для простоты 
$r\hm = 1$ для рынка~\#0, что позволяет интерпретировать 
\textit{стоимостную} плотность $c(x, y)$, $x \hm\in{\sf X}$, $y \hm\in{\sf  Y}$, 
как плотность вероятности, порождаемую рынком. 
  
  Однако распространить такое же упрощение на все рынки представленной 
совокупности нельзя, так как на них могут возникать свои безрисковые ставки 
относительного дохода. И появляются новые \textit{параметры} 
$\chi_{\mathrm{X}}$ и~$\chi_{\mathrm{Y}}$~--- ставки безрискового 
относительного дохода на рынках~\#X и \#Y соответственно, дающие только 
интегральные ограничения на одномерные стоимостные плотности: 
  \begin{equation}
  \left.
  \begin{array}{rl}
  \left\vert \boldsymbol{U}_{\mathrm{X}}\right\vert &=\int\limits_{\sf X} 
c_{\mathrm{X}} (x)\,dx=\chi^{-1}_{\mathrm{X}}\,;\\[6pt] 
  \left\vert \boldsymbol{U}_{\mathrm{Y}}\right\vert &=\int\limits_{\sf Y} 
c_{\mathrm{Y}} (y)\,dy=\chi^{-1}_{\mathrm{Y}}\,.
\end{array}
\right\}
  \label{e2-ag}
  \end{equation}
  
  Поскольку в общем случае ценообразование на совместно 
функционирующих трех рынках (как на самостоятельных, хотя и родственных) 
производит-\linebreak\vspace*{-12pt}

\pagebreak

\noindent
ся раздельно, стоимостные плотности $c_{\mathrm{X}}(\cdot)$ 
и~$c_{\mathrm{Y}}(\cdot)$ одномерных рынков~\#X и \#Y не следует 
отождествлять с маргинальными~(1). И,~вообще говоря, 
  $$
  c_{\mathrm{X}}(x)\not= c_1(x)\,,\ c_{\mathrm{Y}}(y)\not= c_2(y)\,,\ x\in {\sf 
X}\,,\ y\in {\sf Y}\,,
  $$
  хотя при этом естественно считать, что 
  $$
  p_{\mathrm{X}}(x)\equiv p_1(x)\,,\ p_{\mathrm{Y}}(y)\equiv p_2(y)\,,\ x\in{\sf 
X}\,,\ y\in{\sf Y}\,,
  $$
так как все прогнозные плотности $p(x, y)$, $p_{\mathrm{X}}(x)$ 
и~$p_{\mathrm{Y}}(y)$ являются предметом единого цельного прогноза инвестора. 

  Для рынка \#0 определяются также инструментальные \textit{индикаторы} 
множеств $\boldsymbol{H}\{M\}$, $M \hm\subset {\sf X}\times{\sf Y}$, 
и~\textit{единичный безрисковый} актив~$\boldsymbol{U}$, и для них 
  \begin{align*}
  \boldsymbol{H}\{M\} &=\int\limits_M \boldsymbol{D}(s,t)\,dsdt\,;\\
  \boldsymbol{U}=\boldsymbol{H}\{{\sf X}\times {\sf Y}\} &=\int\limits_{{\sf 
X}\times{\sf Y}} \boldsymbol{D}(s,t)\,dsdt\,;\\
  \left\vert \boldsymbol{H}\{M\}\right\vert
  & =\int\limits_M c(s,t)\,dsdt\,;\\
     \vert \boldsymbol{U}\vert ={\sf C}\{{\sf X}\times {\sf Y}\}& =\int\limits_{{\sf 
X}\times{\sf Y}} c(s,t)\,dsdt =\fr{1}{r}\,.
  \end{align*}
  
  Наряду с введенными инструментами рынка~\#0 рассматриваются и его 
\textit{маргинальные} инструменты $\boldsymbol{D}_1(\cdot)$, 
$\boldsymbol{D}_2(\cdot)$, $\boldsymbol{U}_1$, $\boldsymbol{U}_2$, 
$\boldsymbol{H}_1\{\cdot\}$ и~$\boldsymbol{H}_2\{\cdot\}$, но каждый из них не 
самостоятелен и обретает смысл лишь в~\textit{произведении}  
с~ка\-ким-ли\-бо инструментом по другой координате.
  
  Наконец, критерий CC-VaR требует, чтобы выполнялись неравенства 
   ${\sf P}\{q\geq\phi(\varepsilon)\}\hm\geq 1\hm-\varepsilon$ сразу для \textit{всех} 
$\varepsilon \hm\in  [0, 1]$,
где $q$~--- доход инвестора; $\phi(\varepsilon)$~--- неотрицательная монотонно 
возрастающая и непрерывная \textit{функция рисковых предпочтений} (ф.р.п.)\ 
инвестора. 

  В связи с соотношениями~(2) следует также иметь в виду проблемы 
взаимодействия рынков.\linebreak Обмен между рынками инструментальными 
средствами не предусмотрен. Это значит, что не допус\-ти\-мо, например, 
расщепление двумерного единичного безрискового актива на два 
компонентных\linebreak инструмента с~целью последующих операций с~ними на двух 
других одномерных рынках. Естественно, что при этом сохраняется 
возможность использования денежных средств, полученных от продажи актива 
на двумерном рынке, для покупки других активов на одномерных рынках. 
Проясним на простейшем примере, как сказываются такие особенности 
многомерных рынков на исходах сделок. 
  
  Рассмотрим последовательность двух рыночных сделок: 
  \begin{enumerate}[(1)]
  \item продажа единицы инструмента $\boldsymbol{U}\hm= 
\boldsymbol{U}_1\times \boldsymbol{U}_2$ на рынке~\#0 по цене $S \hm= 1$; 
  \item приобретение на сумму~$S$ на рынках~\#X и \#Y по отдельности~$u$ 
и~$v$~единиц инструментов $\boldsymbol{U}_{\mathrm{X}}$ 
и~$\boldsymbol{U}_{\mathrm{Y}}$ по ценам $1/\chi_{\mathrm{X}}$ 
и~$1/\chi_{\mathrm{Y}}$~(2) для каждой единицы соответственно. 
  \end{enumerate}
  
  Для определения количеств~$u$ и~$v$ имеем уравнение $S \hm= 
u/\chi_{\mathrm{X}}\hm + v/\chi_{\mathrm{Y}}$. Во вполне приемлемом 
предположении, что $\chi_{\mathrm{X}}\hm = \chi_{\mathrm{Y}}\hm = 1$, одним из 
его решений будет, например, $u \hm= v \hm= 1/2$. Таким образом, в этом 
случае один двумерный инструмент~$\boldsymbol{U}$ эквивалентен по 
стоимости комбинации $\boldsymbol{U}_{\mathrm{X}}/2\hm+ 
\boldsymbol{U}_{\mathrm{Y}}/2$ (а~не $\boldsymbol{U}_{\mathrm{X}}\hm + 
\boldsymbol{U}_{\mathrm{Y}}$!). 
  
  \section{Оптимизация на~тройственном рынке}
  
  Предлагаются алгоритмы построения на совокупности трех теоретических 
рынков оптимального по CC-VaR комбинированного портфеля вместе 
с~некоторыми его версиями. Алгоритмы, как и~в~[1--5], основываются на 
анализе относительных доходов для всех трех исходных рынков 
с~континуальным применением процедуры Ней\-ма\-на--Пир\-со\-на из 
математической статистики~\cite{6-ag}. 
  
  В общей схеме тройственного рынка для целей оптимизации будем 
формировать единую функцию относительного дохода для комбинации трех 
рынков. Это производится путем поточечной замены значений 
функции~$\rho(\cdot,\cdot)$ рынка~\#0 ровно теми значениями 
функций~$\rho_{\mathrm{X}}(\cdot)$ или~$\rho_{\mathrm{Y}}(\cdot)$ для 
рынков~\#X и~\#Y (с~сопоставимыми по вероятностям весами), которые 
оказываются наибольшими из всех трех функций. 
  
  Формально правила замещения задаются разбиением множества ${\sf 
X}\times {\sf Y}$ на подмножества~$M_0$, $M_1$ и~$M_2$, определяемые 
соотношениями эквивалентности: 
  \begin{align}
  (s,t)\in M_0 &\Leftrightarrow \left\{ \rho(s,t)\geq 
\rho_{\mathrm{X}}(s)\&\rho(s,t)\geq \rho_{\mathrm{Y}}(t)\right\};\!\!
  \label{e3-ag}\\
  (s,t)\in M_1 &\Leftrightarrow \left\{ \rho_{\mathrm{X}}(s)> 
\rho(s,t)\&\rho_{\mathrm{X}}(s)\geq \rho_{\mathrm{Y}}(t)\right\};\!\!
  \label{e4-ag}\\
  (s,t)\in M_2 &\Leftrightarrow \left\{ \rho_{\mathrm{Y}}(t)> 
\rho(s,t)\&\rho_{\mathrm{Y}}(t)> \rho_{\mathrm{X}}(s)\right\}.\!\!
  \label{e5-ag}
  \end{align}
  
  Множества $M_0$, $M_1$ и~$M_2$ взаимно не пересекаются, в объединении 
дают полное множество ${\sf X}\times{\sf Y}$ и состоят из тех и только тех пар 
$(s,t) \hm\in {\sf X}\times{\sf Y}$, для которых максимальным является 
относительный доход соответственно $\rho(s, t)$, $\rho_{\mathrm{X}}(s)$ 
и~$\rho_{\mathrm{Y}}(t)$. В~случае равенства этих доходов приоритет 
в~отношении принадлежности множеству устанавливается в порядке 
рынков~\#0, \#X и~\#Y. 
  
  Результат классификации~(\ref{e3-ag})--(\ref{e5-ag}) можно записывать 
посредством принимающей всего три значения \textit{функции замещений} (для 
всех $s \hm\in{\sf X}$, $t \hm\in {\sf Y}$):
  \begin{equation}
  A(s,t)=k\Leftrightarrow (s,t)\in M_k\,,\enskip k=0,1,2\,.
  \label{e6-ag}
  \end{equation}
    Она просто помечает все точки множеств $M_0$, $M_1$ и~$M_2$ их 
индексами~--- цифрами~0, 1 и~2 соответственно. 
  
  Обозначим через $M_{1;s}(\subset {\sf Y})$ и $M_{2;t}(\subset {\sf X})$ 
сечения множеств $M_{1}$~(\ref{e4-ag}) и~$M_2$~(\ref{e5-ag}) для фиксированных 
значений $s\hm\in{\sf X}$ и $t\hm\in {\sf Y}$ соответственно: 
  $$
  M_1=\bigcup\limits_{s\in{\sf X}} M_{1;s}\,;\quad M_2=\bigcup_{t\in {\sf Y}} 
M_{2;t}\,.
  $$
  
  Рассмотрим индикатор $\boldsymbol{M}_1(s)$, $s \hm\in{\sf X}$, рынка~\#0 
как объединение базисных инструментов $\boldsymbol{D}(s, t)$ по $t \hm\in 
M_{1;s}$: 
  \begin{equation}
  \boldsymbol{M}_1(s)=\int\limits_{M_{1;s}} \boldsymbol{D}(s,t)\,dt 
=\boldsymbol{D}_1(s)\times \boldsymbol{H}_2\left\{ M_{1;s}\right\}\,.
  \label{e7-ag}
  \end{equation}
    Это $\delta$-ин\-стру\-мент на~${\sf X}$ и индикатор 
множества~$M_{1;s}$ на~${\sf Y}$, и для него 
  $$
  \left\vert \boldsymbol{M}_1(s)\right\vert =\int\limits_{M_{1;s}} 
c(s,t)\,dt\,;\enskip
  \left\| \boldsymbol{M}_1(s)\right\| =\int\limits_{M_{1;s}} p(s,t)\,dt\,.
  $$
  
  Его двумерную платежную функцию можно представить в виде 
произведения $\delta(x \hm- s)$, $x \hm\in {\sf X}$, на характеристическую 
функцию множества~$M_{1;s}$ по $y \hm\in {\sf  Y}$. Она сингулярна по~$x$ 
и~конечна по~$y$. 
  
  Индикаторы $\boldsymbol{M}_1(s)$~(\ref{e7-ag}) для каждого  $s \hm \in   
{\sf  X}$ являются ровно теми инструментами рынка~\#0, которые согласно 
условиям~(\ref{e4-ag}) для относительных доходов следовало бы заместить 
инструментами~$\boldsymbol{D}_{\mathrm{X}}(s)$ рынка~\#X. Однако действие 
инструмента~$\boldsymbol{D}_{\mathrm{X}}(s)$ распространяется на полное 
множество~${\sf Y}$, а~не только на его подмножество~$M_{1;s}$. Поэтому 
подобное замещение должно быть ограниченным, и инструменты~(\ref{e7-ag}) 
желательно было бы заместить совмещающими рынки~\#0 и~\#X 
<<гибридными>> инструментами: 
  \begin{equation}
  \boldsymbol{M}_{\mathrm{X}}(s)\equiv 
\boldsymbol{D}_{\mathrm{X}}(s)\times \boldsymbol{H}_2\left\{ 
M_{1;s}\right\}\,,\enskip s\in {\sf X}\,.
  \label{e8-ag}
  \end{equation}
  
  Но таких инструментов нет ни на одном из рассматриваемых рынков. Тем не 
менее рыночную реализацию такого замещения можно осуществить, если 
воспользоваться услугами \textit{рандомизации}. Это делается следующим 
образом. 
  
  Вводятся биномиальные случайные величины $\vartheta_{\mathrm{X}}(s)$, 
$s\hm\in {\sf  X}$, с вероятностью успеха (замещения)~$\theta_{\mathrm{X};s}$, 
равной условной вероятности 
  \begin{equation}
  \theta_{\mathrm{X};s} ={\sf P}\left\{ M_{1;s}\vert \mathrm{X}=s\right\} 
=\int\limits_{M_{1;s}} \fr{p(s,t)\,dt}{p_1(s)}\,,\enskip  s\in {\sf X}\,.
  \label{e9-ag}
  \end{equation}
Эти вероятности служат в модели параметрами рандомизации. 
  
  В соответствии с предположениями о вероятностях и ценообразовании для 
инструментов $\boldsymbol{M}_{\mathrm{X}}(s)$ должны были бы выполняться 
равенства: 

\vspace*{2pt}

\noindent
  \begin{equation}
  \left.
  \begin{array}{rl}
  \hspace*{-3mm}\left\vert \boldsymbol{M}_{\mathrm{X}}(s)\right\vert &=\theta_{\mathrm{X};s} 
c_{\mathrm{X}}(s);\\[6pt]
    \hspace*{-3mm}\left\| \boldsymbol{M}_{\mathrm{X}}(s)\right\| 
&=\theta_{\mathrm{X};s} p_{\mathrm{X}}(s)\,,\
  \rho_{\mathrm{X}}(s)=\fr{c_{\mathrm{X}}(s)}{p_{\mathrm{X}}(s)}\,,\ s\in{\sf 
X}.
\end{array}\!
\right\}\!
  \label{e10-ag}
  \end{equation}
  
  В качестве базисных для части~\#X комбинированного портфеля 
предлагается использовать рандомизированные инструменты: 
  \begin{equation}
  \boldsymbol{D}_{\mathrm{X}}^{\mathrm{cmb}}(s) =\vartheta_{\mathrm{X}}(s) 
\boldsymbol{D}_{\mathrm{X}}(s)\,,\enskip s\in{\sf X}\,.
  \label{e11-ag}
  \end{equation}
    Эти инструменты являются случайными, принимающими облик инструмента 
$\boldsymbol{D}_{\mathrm{X}}(s)$ с~ве\-ро\-ят\-ностью~$\theta_{\mathrm{X};s}$ 
и~\textit{нулевого} инструмента~$\boldsymbol{N}_{\mathrm{X}}(s)$\linebreak 
(с~тож\-де\-ст\-вен\-но равным нулю доходом и нулевой сто\-и\-мостью) с вероятностью 
$1\hm- \theta_{\mathrm{X};s}$, $s \hm\in {\sf  X}$. 
  
  Их средние цены и средние доходы (ве\-ро\-ят\-ности) соответственно 
  
  \vspace*{2pt}
  
  \noindent
  \begin{equation}
  \left.
  \begin{array}{rl}
  \left\vert \boldsymbol{D}^{\mathrm{cmb}}_{\mathrm{X}}(s)\right\vert &=
\theta_{\mathrm{X};s} c_{\mathrm{X}}(s)\,;\\[6pt]
  \left\| \boldsymbol{D}^{\mathrm{cmb}}_{\mathrm{X}}(s)\right\| 
&=\displaystyle \theta_{\mathrm{X};s} p_1(s)=\int\limits_{M_{1;s}} p(s,y)\,dy\,.
\end{array}
\right\}
  \label{e12-ag}
  \end{equation}
  
  Выбор~(\ref{e9-ag}) параметров~$\theta_{\mathrm{X};s}$ уравнивает 
вероятности, связанные 
с~инструментами~$\boldsymbol{M}_{\mathrm{X}}(s)$~(\ref{e8-ag}) 
и~$\boldsymbol{M}_1(s)$~(\ref{e7-ag}), поскольку вероятности, с которыми на 
рынке~\#0 инструменты~$\boldsymbol{M}_1(s)$ порождают ненулевой (именно 
единичный) доход, определяются плотностью $p(s, t)$ и~вторым соотношением 
в~(\ref{e10-ag}). 
  
  Свойства~(\ref{e12-ag}) инструментов~(\ref{e11-ag}) позволяют назначить 
их, несмотря на составную структуру, новыми цельными базисными 
инструментами комбинированного рынка, фактически реплицирующими 
инструменты~$\boldsymbol{M}_{\mathrm{X}}(s)$. 
  
  Подобные конструкции, введенные для рынка~\#X, в полной мере 
распространяются на рынок~\#Y. При этом они получаются из 
со\-от\-вет\-ст\-ву\-ющих аналогов рынка~\#X заменой $1\leftrightarrow2$, 
$s\leftrightarrow t$, $i\leftrightarrow j$, $\mathrm{X}\leftrightarrow\mathrm{Y}$. 
Так определяются уже связанные с~множеством $M_2$~(\ref{e5-ag}) замещения 
инструменты~$\boldsymbol{D}_{\mathrm{Y}}(t)$, случайные 
величины~$\vartheta_{\mathrm{Y}}(t)$ с~па\-ра\-мет\-ра\-ми~$\theta_{\mathrm{Y};t}$ 
успеха и~рандомизированные базисные инструменты 

\noindent
  \begin{multline}
  \boldsymbol{D}^{\mathrm{cmb}}_{\mathrm{Y}}(t)=\vartheta_{\mathrm{Y}}(t) 
\boldsymbol{D}_{\mathrm{Y}}(t)\,,\
  \theta_{\mathrm{Y};t} ={\sf P}\left\{ M_{2;t}\vert \mathrm{Y}=t\right\} 
={}\\
{}=\int\limits_{M_{2;t}} \fr{p(s,t)\,dt}{p_2(t)}\,,\enskip t\in{\sf Y}\,.
  \label{e13-ag}
  \end{multline}
  
  Инструменты~(\ref{e11-ag}) и~(\ref{e13-ag}) на рынках~\#X и~\#Y со своими 
ценами и средними доходами~(\ref{e12-ag}) вместе\linebreak\vspace*{-12pt}

\pagebreak

\noindent 
с~инструментами~$\boldsymbol{D}(\cdot,\cdot)$ на рынке~\#0 
с~плотностями~$c(\cdot,\cdot)$ и $p(\cdot,\cdot)$ на множестве~$M_0$ 
образуют полный \textit{комбинированный} базис. 
  
  Для этого базиса формируется единая функция \textit{относительных 
доходов}, и~к~ней применяется общий теоретический алгоритм оптимизации. 
В~результате его работы с новой функцией относительных доходов 
производится новое назначение всех вероятностей и~строится новая весовая 
функция базисных инструментов. Оптимальный \textit{комбинированный} 
портфель вследствие случайности величин $\vartheta_{\mathrm{X}}(s)$ 
и~$\vartheta_{\mathrm{Y}}(t)$ оказывается в итоге случайным и приобретает вид: 

\noindent
  \begin{multline*}
  \boldsymbol{G}^{\mathrm{cmb}}=\int\limits_{M_0} g^{\mathrm{cmb}}(s,t) 
\boldsymbol{D}(s,t)\,ds dt+{}\\
  {}+\int\limits_{\sf X}\! \!g^{\mathrm{cmb}}_{\mathrm{X}}(s) \vartheta_{\mathrm{X}}(s) 
\boldsymbol{D}_{\mathrm{X}}(s)\,ds\,+\!\int\limits_{\sf Y} \!
g^{\mathrm{cmb}}_{\mathrm{Y}}(t) \vartheta_{\mathrm{Y}}(t) 
\boldsymbol{D}_{\mathrm{Y}}(t)\,dt.\hspace*{-7.25882pt}
 % \label{e14-ag}
  \end{multline*}
  
  Нелишне рассмотреть и упрощенную, хотя и нереализуемую на 
тройственном рынке, \textit{идеалистичную} версию портфеля в эквивалентной 
по платежной функции и ценам форме двумерного портфеля с теми же весами: 

\noindent
  \begin{multline}
  \boldsymbol{G}^{\mathrm{idl}}=\int\limits_{M_0} g^{\mathrm{cmb}}(s,t) 
\boldsymbol{D}(s,t)\,dsdt+{}\\
{}+\int\limits_{\sf X} 
g^{\mathrm{cmb}}_{\mathrm{X}}(s)\boldsymbol{M}_{\mathrm{X}}(s)\,ds+
  \int\limits_{\sf Y} 
g^{\mathrm{cmb}}_{\mathrm{Y}}(t)\boldsymbol{M}_{\mathrm{Y}}(t)\,dt\,.
  \label{e15-ag}
  \end{multline}
  
  При всей условности такого представления его можно использовать для 
графической иллюстрации платежной функции в виде единой двумерной 
функции:
  \begin{equation*}
  \pi\left( x,y;\boldsymbol{G}^{\mathrm{idl}}\right)= \max \left( g^{\mathrm{cmb}}(x,y), 
g^{\mathrm{cmb}}_{\mathrm{X}}(x), g^{\mathrm{cmb}}_{\mathrm{Y}}(y)\right)\,.
 % \label{e16-ag}
  \end{equation*}
  
  Наряду с комбинированным можно построить и~портфель, который назовем 
\textit{суррогатным}. Он получается в результате формальной 
\textit{поточечной} замены базисных инструментов $\boldsymbol{D}(s,t)$ 
рынка~\#0 инструментами $\boldsymbol{D}^{\mathrm{srg}}(s, t)$, $s \hm\in {\sf X}$, $t 
\hm\in {\sf Y}$, с~теми же платежными функциями и~вероятностями, но 
с~ценами, скорректированными в~соответствии с~правилами 
замещения~(\ref{e3-ag})--(\ref{e5-ag}) и~с~учетом цен рынков~\#X и~\#Y. Для 
всех $s\hm\in {\sf X}$ и~$t\hm\in {\sf Y}$ и~при $A(s, t)\hm = 0$, 1, 2 
(см.~(\ref{e6-ag})) соответственно 
  $$
  \left\vert \boldsymbol{D}^{\mathrm{srg}}(s,t)\right\vert =c^{\mathrm{srg}} (s,t) =c(s,t), 
\fr{p(s,t)}{p_{\mathrm{X}}(s)}\,, \fr{p(s,t)}{p_{\mathrm{Y}}(t)}\,.
  $$

  
  Далее вновь образуется функция относительных доходов, и на ее основе 
алгоритм находит весовую функцию портфеля $g^{\mathrm{srg}}(s, t)$, $s \hm\in {\sf X}$, 
$t \hm\in {\sf Y}$. И~тогда

\noindent
  $$
  \boldsymbol{G}^{\mathrm{srg}}=\int\limits_{\sf X} \int\limits_{\sf Y} g^{\mathrm{srg}}(s,t) 
\boldsymbol{D}^{\mathrm{srg}}(s,t)\,dsdt\,.
  $$
  %
  Суррогатный портфель, как и портфель~(\ref{e15-ag}), не реализуем на 
рассматриваемом рынке, но ввиду своей простоты вполне может служить 
средством проверки правильности алгоритма в его дискретной версии, тем 
более по графикам доходов.

\vspace*{-6pt}
  
  \section{Заключение}
  
  В работе предложен подход к оптимизации поведения инвестора, 
придерживающегося CC-VaR, на совокупности финансовых рынков разной 
размерности. Изложение ведется для теоретических рынков, на которых 
базисными служат $\delta$-ин\-стру\-мен\-ты. Для целей оптимизации 
приводится правило замещения базисных инструментов двумерного рынка 
более доходными базисными инструментами двух одномерных 
с~использованием механизма рандомизации. Предлагается способ построения 
оптимального комбинированного портфеля из базисных инструментов всех 
рынков вместе с его идеалистичной и суррогатной версиями. Для проверки 
действенности модели и всех ее компонентов необходимо дополнительно 
адаптировать построенные теоретические конструкции к дискретным 
сценарным рынкам, рассмотреть характерные примеры с~проведением 
численных расчетов и демонстрацией результатов на графиках. 

\vspace*{-6pt}
  
{\small\frenchspacing
 {%\baselineskip=10.8pt
 \addcontentsline{toc}{section}{References}
 \begin{thebibliography}{9}
  \bibitem{1-ag}
  \Au{Agasandian G.\,A.} Optimal behavior of an investor in option market~//  
Joint Conference (International) on Neural Networks Proceedings.~--- 
 IEEE, 2002. P.~1859--1864. 
  \bibitem{2-ag}
  \Au{Агасандян Г.\,А.} Применение континуального критерия VaR на 
финансовых рынках.~--- М.: ВЦ РАН, 2011. 299~с. 
  \bibitem{3-ag}
  \Au{Агасандян Г.\,А.} Континуальный критерий VaR на многомерных рынках 
опционов.~--- М.: ВЦ РАН, 2015. 297~с. 
  \bibitem{4-ag}
  \Au{Агасандян Г.\,А.} Континуальный критерий VaR на сценарных рынках~// 
Информатика и её применения, 2018. Т.~12. Вып.~1. С.~32--40. 
  \bibitem{5-ag}
  \Au{Агасандян Г.\,А.} Континуальный критерий VaR и оптимальный 
портфель инвестора~// Управление большими системами, 2018. Вып.~73.  
С.~6--26.
  \bibitem{6-ag}
  \Au{Крамер Г.} Математические методы статистики~/ Пер. с~англ.~--- М.: 
Мир, 1975. 750~с. (\Au{Cramer~H.} Mathematical methods of statistics.~--- 
Princeton, NJ, USA: Princeton University Press, 1946. 575~p.)
 \end{thebibliography}

 }
 }

\end{multicols}

\vspace*{-7pt}

\hfill{\small\textit{Поступила в~редакцию 27.03.19}}

%\vspace*{8pt}

%\pagebreak

\newpage

\vspace*{-28pt}

%\hrule

%\vspace*{2pt}

%\hrule

%\vspace*{-2pt}

\def\tit{THEORETICAL FOUNDATIONS OF~CONTINUOUS VaR CRITERION OPTIMIZATION 
IN~THE~COLLECTION OF~MARKETS}


\def\titkol{Theoretical foundations of~continuous VaR criterion  optimization 
in~the~collection of~markets}

\def\aut{G.\,A.~Agasandyan}

\def\autkol{G.\,A.~Agasandyan}

\titel{\tit}{\aut}{\autkol}{\titkol}

\vspace*{-11pt}


\noindent
  A.\,A.~Dorodnicyn Computing Center, Federal Research Center ``Computer 
Science and Control'' of the Russian Academy of Sciences, 40~Vavilov Str., Moscow 
119333, Russian Federation

\def\leftfootline{\small{\textbf{\thepage}
\hfill INFORMATIKA I EE PRIMENENIYA~--- INFORMATICS AND
APPLICATIONS\ \ \ 2019\ \ \ volume~13\ \ \ issue\ 4}
}%
 \def\rightfootline{\small{INFORMATIKA I EE PRIMENENIYA~---
INFORMATICS AND APPLICATIONS\ \ \ 2019\ \ \ volume~13\ \ \ issue\ 4
\hfill \textbf{\thepage}}}

\vspace*{3pt} 
  
  
   
  \Abste{The work continues studying the problems of using continuous 
  VaR criterion (CC-VaR) in 
financial markets. The application of CC-VaR in a collection of theoretical markets of different 
dimensions that are mutually connected by their underliers is concerned. In a typical model of the 
collection of one two-dimensional market and two one-dimensional markets, the most general case 
of their conjoint functioning is considered. The rule of constructing a combined portfolio optimal on 
CC-VaR in these markets is submitted. This rule is founded on misbalance in returns relative 
between markets with maintaining optimality on CC-VaR. The optimal combined portfolio with 
three components is constructed from basis instruments of all markets and by using ideas of 
randomization in their composition. Also, the idealistic and surrogate versions of this combined 
portfolio, which are useful in testing all algorithmic calculations and in graphic illustrating 
portfolio's payoff functions, are adduced. The model can be extended without academic difficulties 
onto markets of greater dimensions. Also, two truncated variants of problem setting with excluded 
either one of one-dimensional markets or the two-dimensional market are fully justified.} 
  
  \KWE{underliers; risk preferences function; continuous VaR criterion; cost and forecast 
densities; return relative function; Newman--Pearson procedure; combined portfolio; 
randomization; surrogate portfolio; idealistic portfolio}
 
 
  \DOI{10.14357/19922264190406} 

%\vspace*{-14pt}

 \Ack
 \noindent
 The work was supported by the Russian Foundation for Basic 
Research (project 17-01-00816).
 


\vspace*{6pt}

  \begin{multicols}{2}

\renewcommand{\bibname}{\protect\rmfamily References}
%\renewcommand{\bibname}{\large\protect\rm References}

{\small\frenchspacing
 {%\baselineskip=10.8pt
 \addcontentsline{toc}{section}{References}
 \begin{thebibliography}{9}
 
 \vspace*{-18pt}
 
  \bibitem{1-ag-1}
  \Aue{Agasandian, G.\,A.} 2002. Optimal behavior of an investor in option market. 
\textit{Joint Conference (International) on Neural Networks Proceedings.}
 IEEE. 1859--1864. 
  \bibitem{2-ag-1}
  \Aue{Agasandyan, G.\,A.} 2011. \textit{Primenenie kontinual'nogo kriteriya VaR 
na finansovykh rynkakh} [Application of continuous VaR-criterion in financial 
markets]. Moscow: CC RAS. 299 p. 
  \bibitem{3-ag-1}
  \Aue{Agasandyan, G.\,A.} 2015. \textit{Kontinual'nyy kriteriy VaR na 
mnogomernykh rynkakh optsionov} [Continuous VaR-criterion\linebreak\vspace*{-12pt}

\columnbreak

\noindent
 in multidimensional 
option markets]. Moscow: CC RAS. 297~p. 

\vspace*{-2.5pt}

  \bibitem{4-ag-1}
  \Aue{Agasandyan, G.\,A.} 2018. Kontinual'nyy kriteriy VaR na stsenarnykh 
rynkakh [Continuous VaR-criterion in scenario markets]. \textit{Informatika i~ee 
Primeneniya~--- Inform. Appl.} 12(1):32--40. 

\vspace*{-2.5pt}

  \bibitem{5-ag-1}
  \Aue{Agasandyan, G.\,A.} 2018. Kontinual'nyy kriteriy VaR i~optimal'nyy 
portfel' investora [Continuous VaR-criterion and  investor's 
optimal portfolio]. 
\textit{Upravlenie bol'shimi sistemami} [Large-Scale Systems Control] 73:6--26. 

\vspace*{-2.5pt}

  \bibitem{6-ag-1}
  \Aue{Cramer, H.} 1946. \textit{Mathematical methods of statistics}. Princeton, 
NJ: Princeton University Press. 575~p.
 \end{thebibliography}

 }
 }

\end{multicols}

%\vspace*{-7pt}

\hfill{\small\textit{Received March 27, 2019}}

%\pagebreak

%\vspace*{-22pt}

\Contrl

\noindent
\textbf{Agasandyan Gennady A.} (b.\ 1941)~--- Doctor of Science in physics and mathematics, leading 
scientist, A.\,A.~Dorodnicyn Computing Center, Federal Research Center ``Computer Science and Control'' 
of the Russian Academy of Sciences, 40~Vavilov Str., Moscow 119333, Russian Federation; 
\mbox{agasand17@yandex.ru}

   
\label{end\stat}

\renewcommand{\bibname}{\protect\rm Литература}  
      %6
\def\stat{ushakovi}

\def\tit{ВЫХОДЯЩИЕ ПОТОКИ В~ОДНОЛИНЕЙНОЙ СИСТЕМЕ С~ОТНОСИТЕЛЬНЫМ ПРИОРИТЕТОМ$^*$}

\def\titkol{Выходящие потоки в~однолинейной системе с~относительным приоритетом}

\def\aut{В.\,Г.~Ушаков$^{1}$, Н.\,Г.~Ушаков$^2$}

\def\autkol{В.\,Г.~Ушаков, Н.\,Г.~Ушаков}

\titel{\tit}{\aut}{\autkol}{\titkol}

\index{Ушаков В.\,Г.}
\index{Ушаков Н.\,Г.}
\index{Ushakov V.\,G.}
\index{Ushakov N.\,G.}


{\renewcommand{\thefootnote}{\fnsymbol{footnote}} \footnotetext[1]
{Работа выполнена при финансовой поддержке РФФИ (проект 18-07-00678).}}


\renewcommand{\thefootnote}{\arabic{footnote}}
\footnotetext[1]{Факультет вычислительной математики и~кибернетики Московского государственного 
университета им.~М.\,В.~Ломоносова; 
Федеральный исследовательский центр <<Информатика и~управление>>  
Российской академии наук, \mbox{vgushakov@mail.ru}}
\footnotetext[2]{Институт проблем технологии микроэлектроники и~особочистых материалов 
Российской академии наук, Черноголовка;
Норвежский на\-уч\-но-тех\-но\-ло\-ги\-че\-ский университет, Тронхейм, 
\mbox{ushakov@math.ntnu.no}}

%\vspace*{-2pt}




\Abst{Изучена однолинейная система массового обслуживания с~бесконечным числом
 мест для ожидания, произвольным распределением времени обслуживания и~двумя 
 пуассоновскими входящими потоками требований. Требования первого потока 
 обладают относительным приоритетом перед требованиями второго потока. 
 Методом вложенных цепей Маркова исследуется многомерный случайный 
 процесс, компоненты которого~--- число требований каждого 
 приоритета в~системе и~длительность интервала времени между последовательными 
 моментами ухода из системы требований одного приоритета. Найдены 
 конечномерные распределения указанных процессов. 
 В~качестве следствия получены преобразования Лап\-ла\-са--Стилть\-еса 
 одномерных и~двумерных распределений выходящего потока требований каждого 
 приоритета в~стационарном режиме.}

\KW{выходящий поток; относительный приоритет; 
вложенная цепь Маркова; одноканальная система}

\DOI{10.14357/19922264190407} 
  
%\vspace*{1pt}


\vskip 10pt plus 9pt minus 6pt

\thispagestyle{headings}

\begin{multicols}{2}

\label{st\stat}

\section{Введение} 

Одной из важных характеристик функционирования систем массового обслуживания 
служит выходящий из нее после завершения обслуживания поток требований. 
Знание характеристик выходящего потока бывает необходимо при изучении 
сетей обслуживания, в~которых потоки требований в~узлы содержат в~себе часть 
требований, выходящих из других узлов. Другой важной задачей, в~которой 
рассматриваются выходящие потоки, является задача восстановления 
структуры и~параметров сис\-те\-мы по наблюдению за различными ее 
характеристиками (так называемые обратные задачи).

Вероятностные свойства выходящих потоков в~приоритетных системах 
обслуживания изучены пока недостаточно полно. Полученные к~на\-сто\-яще\-му 
времени результаты касаются свойств одно\-мер\-ных распределений интервалов 
между уходами\linebreak из сис\-те\-мы требований различных приоритетов
 (см., например,~[1--3]). 
В~настоящей работе найде\-ны одномерные и~двумерные распределения выходящих 
потоков каждого приоритета в~однолинейной системе обслуживания с~ожиданием, 
двумя пуассоновскими потоками
требований, в~которой требования первого потока (первого приоритета)
 имеют относительный приоритет перед требованиями второго потока.



\section{Обозначения и~определения}



Пусть $a_1$ и~$a_2$~--- интенсивности, а~$B_1(x)$ и~$B_2(x)$~--- 
функции распределения времен обслуживания приоритетных и~неприоритетных 
требований соответственно.  Обозначим
\begin{align*}
  \beta_i(s)&=\int\limits_0^{\infty}e^{-sx}dB(x)\,;\\ 
  \beta_{ij}&=\int\limits_0^{\infty}x^jdB_i(x)\,;\\
   \sigma&=a_1+a_2\,.
\end{align*}
Пусть далее $t_{iN}$~--- момент ухода из системы \mbox{$N$-го} требования приоритета~$i$ 
(нумерация требований производится для каждого приоритета отдельно в~порядке 
их ухода из системы), $t_{i0}\hm=0$, $\tau_{iN}\hm=t_{iN}\hm-t_{i,N-1}$, $L_i(t)$~--- 
число требований в~системе в~момент времени~$t,$
$i\hm=1,2$, $N\hm=1,2,\ldots$

Всюду в~дальнейшем будем считать выполненным условие эргодичности 
$\rho\hm=a_1\beta_{11}\hm+a_2\beta_{21}\hm<1.$
Положим
\begin{multline*}
P_i\left(n_1,n_2,x\right)=\lim\limits_{N\rightarrow\infty}
\mathbf{P}\left(L_1\left(t_{iN}+0\right)=n_1,\right.\\
\left.L_2
\left(t_{iN}+0\right)=n_2,\tau_{iN}<x\right);
\end{multline*}

%\vspace*{-12pt}

\noindent
\begin{multline*}
Q_i\left(n_1,n_2,m_1,m_2,x,y\right)={}\\
{}+
\lim\limits_{N\rightarrow\infty}\mathbf{P}
\left(L_1(t_{iN}+0)=n_1,L_2(t_{iN}+0)=n_2,\right.\\
 L_1(t_{i,N-1}+0)=m_1,L_2(t_{i,N-1}+0)=m_2,\\
\left.\tau_{iN}<x,\tau_{i,N-1}<y\right);
\end{multline*}
$$
p_i\left(z_1,z_2,s\right)=\!
\int\limits_0^{\infty}\!\!e^{-sx}\sum\limits_{n_1=0}^{\infty}\sum\limits_{n_2=0}^{\infty}
z_1^{n_1}z_2^{n_2}d_xP_i\left(n_1,n_2,x\right),
$$

\vspace*{-12pt}

\noindent
\begin{multline*}
q_i\left(w_1,w_2,z_1,z_2,s_1,s_2\right)={}\\
{}=
\int\limits_0^{\infty}\int\limits_0^{\infty}e^{-s_1x}e^{-s_2y}
\sum\limits_{n_1=0}^{\infty}\sum\limits_{n_2=0}^{\infty}
\sum\limits_{m_1=0}^{\infty}\sum\limits_{m_2=0}^{\infty}w_1^{n_1}w_2^{n_2}\times
\\
\times
z_1^{m_1}z_2^{m_2}d_xd_yQ_i\left(n_1,n_2,m_1,m_2,x,y\right);
\end{multline*}
$$
f_i(s)=\lim\limits_{N\rightarrow\infty}
\int\limits_0^{\infty}e^{-sx}d\mathbf{P}(\tau_{iN}<x);
$$

\vspace*{-12pt}

\noindent
\begin{multline*}
g_i(s_1,s_2)={}\\
{}=\!\!\lim\limits_{N\rightarrow\infty}
\int\limits_0^{\infty}\int\limits_0^{\infty}\!\!e^{-s_1x}e^{-s_2y}d_xd_y
\mathbf{P}(\tau_{iN}<x,\tau_{i,N-1}<y).\hspace*{-3.5pt}
\end{multline*}

\section{Предварительные результаты}

В дальнейшем понадобятся некоторые результаты для системы 
массового обслуживания типа $M|G|1|\infty.$
Обозначим~$a$~--- интенсивность входящего потока; $B(x)$~--- 
функцию распределения времени обслуживания; 
$\Pi(x)$~--- функцию распределения периода занятости:
$$
\beta(s)=\int\limits_0^{\infty}e^{-sx}dB(x)\,;\enskip 
\pi(s)=\int\limits_0^{\infty}e^{-sx}d\Pi(x)\,.
$$
Тогда $\pi(s)$ будет единственным решением уравнения 
$\pi(s)\hm=\beta(s\hm+a\hm-a\pi(s)),$ аналитическим в~области $\mathrm{Re}\, s\hm>0.$

Пусть в~начальный момент времени $t\hm=0$  в~сис\-те\-ме~$i$ требований. 
Обозначим $W^{(i)}(x,t)$~--- функцию распределения виртуального 
времени ожидания в~момент времени~$t$; $p^{(i)}(0,t)$~--- вероятность 
свободного состояния системы в~момент времени~$t$; $p^{(i)}(k,v,t)dv$~--- 
вероятность того, что в~момент времени~$t$ в~системе 
$k\hm\geqslant 1$ требований, а~с~начала обслуживания требования, 
находящегося на приборе, прошло время, лежащее в~интервале $(v,v+dv).$
Тогда
\begin{multline}
\label{n1}
\hspace*{-1.80583pt}\int\limits_0^{\infty}\int\limits_0^{\infty}\!\!e^{-sx}e^{-qt}\,d_xW^{(i)}
(x,t)\,dt=\fr{\beta^i(s)}{q-s+a-a\beta(s)}-{}\\
{}-
\fr{s\pi^i(q)}{(q+a-a\pi(q))(q-s+a-a\beta(s))}\,;
\end{multline}


\noindent
\begin{equation*}
%\label{n2}
\int\limits_0^{\infty}e^{-st}p^{(i)}(0,t)dt=\fr{\pi^i(s)}{s+a-a\pi(s)}\,;
\end{equation*}

\vspace*{-18pt}

\noindent
\begin{multline*}
%\label{n3}
\sum\limits_{k=1}^{\infty}z^k\int\limits_0^{\infty}
e^{-st}p^{(i)}(k,v,t)dt={}\\[-2pt]
{}=\fr{(1-B(v))e^{-(s+a-az)v}}{1-z^{-1}\beta(s+a-az)}
\!\left(\!z^i-\fr{(s+a-az)\pi^i(s)}{s+a-a\pi(s)}\!\right).\hspace*{-4.77934pt}
\end{multline*}

\vspace*{-15pt}


\section{Основные результаты}

\vspace*{-6pt}

Основные результаты работы содержатся в~приводимых ниже четырех теоремах.

%\smallskip

\noindent
\textbf{Теорема~1.}\
\textit{Функции $q_2\left(w_1,w_2,z_1,z_2,s_1,s_2\right)$ 
и~$p_2\left(z_1,z_2,s\right)$  определяются соотношениями}:

\vspace*{-6pt}

\noindent
\begin{multline}
\label{n4}
q_2\left(w_1,w_2,z_1,z_2,s_1,s_2\right)={}\\
{}=
w_2^{-1}\beta_2(s_1+\sigma-a_1w_1-a_2w_2)\times{}\\
{}\times
\left(
\vphantom{\fr{s_1+a_2-a_2w_2+a_1-a_1\pi_1(s_1+a_2-a_2w_2)}{s_1+\sigma-a_1\pi_1(s_1+a_2)}}
p_2\left(z_1\pi_1(s_1+a_2-a_2w_2),w_2z_2,s_2\right)-{}\right.\\
{}-
\fr{s_1+a_2-a_2w_2+a_1-a_1\pi_1(s_1+a_2-a_2w_2)}{s_1+\sigma-a_1\pi_1(s_1+a_2)}\times{}\\
\left.{}\times
p_2(z_1\pi_1(s_1+a_2),0,s_2)
\vphantom{\fr{s_1+a_2-a_2w_2+a_1-a_1\pi_1(s_1+a_2-a_2w_2)}{s_1+\sigma-a_1\pi_1(s_1+a_2)}}
\right);
\end{multline}

\vspace*{-20pt}

\noindent
\begin{multline}
\label{n5}
p_2\left(z_1,z_2,s\right)=z_2^{-1}\beta_2\left(s+\sigma-a_1z_1-a_2z_2\right)
\times{}\\
{}\times
\left(
\vphantom{\fr{s_1+a_2-a_2w_2+a_1-a_1\pi_1(s_1+a_2-a_2w_2)}{s_1+\sigma-a_1\pi_1(s_1+a_2)}}
p_2(\pi_1(s+a_2-a_2z_2),z_2,0)-{}\right.\\
{}-
\fr{s+a_2-a_2z_2+a_1-a_1\pi_1(s+a_2-a_2z_2)}{s+\sigma-a_1\pi_1(s+a_2)}\times{}\\
\left.{}\times
p_2(\pi_1(s+a_2),0,0)
\vphantom{\fr{s_1+a_2-a_2w_2+a_1-a_1\pi_1(s_1+a_2-a_2w_2)}{s_1+\sigma-a_1\pi_1(s_1+a_2)}}
\right).
\end{multline}

\vspace*{-4pt}

\noindent
\textit{Функция $p_2\left(z_1,z_2,0\right)$ равна}

\vspace*{-9pt}

\noindent
\begin{multline}
\label{n6}
p_2\left(z_1,z_2,0\right)=\beta_2(\sigma-a_1z_1-a_2z_2)\,\fr{1-\rho}{a_2}
\times{}\\
{}\times \fr{a_2-a_2z_2+a_1-a_1\pi_1(a_2-a_2z_2)}{\beta_2(\sigma-a_2z_2-a_1\pi_1(a_2-a_2z_2))-z_2},
\end{multline}
\textit{а $\pi_1(s)$  есть преобразование Лап\-ла\-са--Стилть\-еса функции распределения периода занятости системы $M|G|1|\infty$ с~интенсивностью входящего
потока~$a_1$ и~функцией распределения времени обслуживания~$B_1(x).$}


\noindent
Д\,о\,к\,а\,з\,а\,т\,е\,л\,ь\,с\,т\,в\,о\,.\ \
Рассматривая два соседних момента ухода требований второго
 приоритета из системы, имеем:
 
 \vspace*{-6pt}
 
 \noindent
\begin{multline*}
P_2(n_1,n_2,x)=\sum\limits_{i_1=0}^{\infty}
\sum\limits_{i_2=1}^{n_2+1}P\left(i_1,i_2,\infty\right)\times{}\\[-1pt]
{}\times \sum\limits_{k_2=0}^{n_2+1-i_2}
\int\limits_0^x \!\! e^{-a_2u}\fr{(a_2u)^{k_2}}{k_2!}\,
d\Pi_1^{*i_1}(u)\!\int\limits_0^{x-u}  \!\!e^{-\sigma u}\fr{(a_1v)^{n_1}}{n_1!}
\times{}\hspace*{-3.60439pt}
\end{multline*}

\noindent
\begin{multline}
{}\times\fr{(a_2v)^{n_2+1-i_2-k_2}}{(n_2+1-i_2-k_2)!}\,dB_2(v)+
\sum\limits_{i=0}^{\infty}P_2(i,0,\infty)\times{}\\
{}\times
\int\limits_0^xG_i(u,n_1,n_2,x-u)d\left(1-e^{-a_2u}\right),
\label{n7}
\end{multline}
где
$G_j(u,n_1,n_2,v)$~--- вероятность того, что первое требование второго 
приоритета покинет систему к~моменту времени~$u\hm+v,$
в~момент его ухода в~системе останется~$n_1$ и~$n_2$ требований первого и~второго 
приоритетов при условии, что это требование поступает
в~момент времени~$u,$ а в~начальный момент
в~системе есть~$j$  требований первого приоритета.  Переходя 
в~\eqref{n7} к~производящим функциям и~преобразованиям Лап\-ла\-са--Стилть\-еса 
и~учитывая~\eqref{n1} и~то, что

\vspace*{-3pt}

\noindent
\begin{multline*}
\sum\limits_{n_1=0}^{\infty}\sum\limits_{n_2=0}^{\infty}
z_1^{n_1}z_2^{n_2}\int\limits_0^{\infty}e^{-sx}
d_xG_i(u,n_1,n_2,x)={}\\
{}=\beta_2(s+\sigma-a_1z_1-a_2z_2)\times\\
\times \int\limits_0^{\infty}\!
\exp\left(-\left(s+a_2-a_2z_2+a_1-a_1\pi_1\left(s+a_2-{}\right.\right.\right.\\
\left.\left.\left.{}-a_2z_2\right)\right)x\right)
d_x W^{(i)}(x,t),
\end{multline*}

\vspace*{-3pt}

\noindent
получаем~\eqref{n5}. Подставляя в~\eqref{n5} $s\hm=0,$ получаем

\vspace*{-3pt}

\noindent
\begin{multline}
\label{n8}
p_2\left(z_1,z_2,0\right)=z_2^{-1}\beta_2(\sigma-a_1z_1-a_2z_2)\times{}\\
{}\times \left(
\vphantom{\fr{s_1+a_2-a_2w_2+a_1-a_1\pi_1(s_1+a_2-a_2w_2)}{s_1+\sigma-a_1\pi_1(s_1+a_2)}}
p_2(\pi_1(a_2-a_2z_2),z_2,0)-{}\right.\\
{}-
\fr{a_2-a_2z_2+a_1-a_1\pi_1(a_2-a_2z_2)}{\sigma-a_1\pi_1(a_2)}\times{}\\
\left.{}\times
p_2\left(\pi_1(a_2),0,0\right)
\vphantom{\fr{s_1+a_2-a_2w_2+a_1-a_1\pi_1(s_1+a_2-a_2w_2)}{s_1+\sigma-a_1\pi_1(s_1+a_2)}}
\right).
\end{multline}

\vspace*{-3pt}

\noindent
Из~\eqref{n8} следует, что
$$
p_2(z_1,z_2,0)=\beta_2\left(\sigma-a_1z_1-a_2z_2\right)\eta\left(z_2\right)\,,
$$
где

\vspace*{-2pt}

\noindent
\begin{multline*}
\eta(z_2)=-\fr{a_2-a_2z_2+a_1-a_1\pi_1(a_2-a_2z_2)}
{z_2-\beta_2(\sigma-a_2z_2-a_1\pi_1(a_2-a_2z_2))}\times{}\\
{}\times
\fr{\beta_2(\sigma-a_1\pi_1(a_2))}{\sigma-a_1\pi_1(a_2)}\,\eta(0)\,.
\end{multline*}
Устремляя в~последнем соотношении~$z_2$ к~единице, находим

\noindent
$$
\eta(0)=\fr{1-\rho}{a_2}\,\fr{\sigma-a_1\pi_1(a_2)}{\beta_2(\sigma-a_1\pi_1(a_2))}\,.
$$
Отсюда следует~\eqref{n6}.

Рассмотрим теперь три последовательных момента ухода из системы требований 
второго приоритета. Имеем

\noindent
\begin{multline*}
Q_2\left(n_1,n_2,m_1,m_2,x,y\right)=P_2\left(m_1,m_2,y\right)\times{}\\
{}\times \sum\limits_{k_2=0}^{\max(0,n_2+1-m_2)}
\int\limits_0^xe^{-a_2u}\fr{(a_2u)^{k_2}}{k_2!}\,d\Pi_1^{*m_1}(u)
\times{}\\
{}\times \int\limits_0^{x-u}e^{-\sigma v}\fr{(a_1v)^{n_1}}{n_1!}
\fr{(a_2v)^{n_2+1-m_2-k_2}}{(n_2+1-m_2-k_2)!}dB_2(v),\\ m_2\geqslant 1\,;
\end{multline*}

\vspace*{-12pt}

\noindent
\begin{multline*}
Q_2\left(n_1,n_2,m_1,0,x,y\right)=
P_2\left(m_1,0,y\right)\times{}\\
{}\times
\int\limits_0^xG_{m_1}\left(u,n_1,n_2,x-u\right)d\left(1-e^{-a_2u}\right).
\end{multline*}
Отсюда следует~\eqref{n4}.

\smallskip

\noindent
\textbf{Теорема~2.}\
\textit{Справедливы следующие соотношения}:
\begin{multline*}
f_2(s)=\beta_2(s)\beta_2
\left(a_1-a_1\pi_1(s)\right)-{}\\
{}-\beta_2(s)
\fr{s+a_1-a_1\pi_1(s)}{s+\sigma-a_1\pi_1(s+a_2)}\times{}\\
{}\times\fr{(1-\rho)\beta_2(\sigma-a_1\pi_1(s+a_2))
\left(\sigma-a_1\pi_1(a_2)\right)}{a_2\beta_2(\sigma-a_1\pi_1(a_2))}\,;
\end{multline*}


\vspace*{-12pt}

\noindent
\begin{multline*}
g_2(s_1,s_2)=\beta_2(s_1)\left(
\vphantom{\fr{s_1+a_1-a_1\pi_1(s_1)}{s_1+\sigma-a_1\pi_1(s_1+a_2)}}
p_2\left(\pi_1(s_1),1,s_2\right)-{}\right.\\
\left.{}-
\fr{s_1+a_1-a_1\pi_1(s_1)}{s_1+\sigma-a_1\pi_1(s_1+a_2)}
p_2(\pi_1(s_1+a_2),0,s_2)\right).
\end{multline*}

\noindent
Д\,о\,к\,а\,з\,а\,т\,е\,л\,ь\,с\,т\,в\,о\ \
 непосредственно вытекает из результатов теоремы~1 и~соотношений
$f_2(s)\hm=p_2\left(1,1,s\right)$ и~$g_2(s_1,s_2)\hm=q_2\left(1,1,1,1,s_1,s_2\right).
$


\smallskip

\noindent
\textbf{Теорема~3.}\ 
\textit{Функции $q_1\left(w_1,w_2,z_1,z_2,s_1,s_2\right)$, 
$p_1\left(z_1,z_2,s\right)$ и~$p_1\left(z_1,z_2,0\right)$  определяются по формулам}:
\begin{multline}
\label{n9}
q_1\left(w_1,w_2,z_1,z_2,s_1,s_2\right)={}\\
{}=
w_1^{-1}\beta_1\left(s_1+\sigma-a_1w_1-a_2w_2\right)\times{}\\
{}\times
\left(p_1\left(z_1w_1,z_2w_2,s_2\right)-
p_1\left(0,z_2w_2,s_2\right)\right)+{}\\
{}+\fr{a_1\beta_1\left(s_1+\sigma-a_1w_1-a_2w_2\right)}{s_1+\sigma-a_2\pi_2(s_1+a_1)}
\times{}\\
{}\times p_1\left(0,z_2\pi_2(s_1+a_1),s_2\right)+{}\\
{}+
w_1^{-1}w_2^{-1}\beta_1\left(s_1+\sigma-a_1w_1-a_2w_2\right)\times{}\\
{}\times
\left(p_1(0,z_2w_2,s_2)-\fr{s_1+\sigma-a_2w_2}
{s+\sigma-a_2\pi_2(s_1+a_1)}\times{}\right.\\
\left.{}\times p_1(0,z_2\pi_2(s_1+a_1),s_2)
\vphantom{\fr{1-\beta_2(s)}{1-\beta_2(s+a_1)}}
\right)
\times{}\\
{}\times
\fr{1}{1-w_2^{-1}\beta_2(s_1+\sigma-a_2w_2)}
\left( \beta_2\left(s_1+\sigma-\right.\right.\\
\left.\left.{}-a_1w_1-a_2w_2\right)-
\beta_2\left(s_1+\sigma-a_2w_2\right)\right)\,;
\end{multline}

%\vspace*{-12pt}

\noindent
\begin{multline}
\label{n10}
p_1\left(z_1,z_2,s\right)=
z_1^{-1}\beta_1\left(s+\sigma-a_1z_1-a_2z_2\right)\times{}\\
{}\times \left(p_1\left(z_1,z_2,0\right)-
\fr{z_2-\beta_2(s+\sigma-a_1z_1-a_2z_2)}{z_2-\beta_2(s+\sigma-a_2z_2)}\times{}\right.\\
\left.{}\times
p_1(0,z_2,0)
\vphantom{\fr{z_2-\beta_2(s+\sigma-a_1z_1-a_2z_2)}{z_2-\beta_2(s+\sigma-a_2z_2)}}
\right)-
\fr{\beta_1(s+\sigma-a_1z_1-a_2z_2)}{s+\sigma-a_2\pi_2(s+a_1)}
\times{}\\
{}\times p_1\left(0,\pi_2(s+a_1),0\right)\times{}\\
{}\times\left(\fr{\beta_2(s+\sigma-a_1z_1-a_2z_2)-\beta_2(s+\sigma-a_2z_2)}
{z_1(z_2-\beta_2(s+\sigma-a_2z_2))}\times{}\right.\\
\left.{}\times \left(s+\sigma-a_2z_2\right)-a_1
\vphantom{\fr{z_2-\beta_2(s+\sigma-a_1z_1-a_2z_2)}{z_2-\beta_2(s+\sigma-a_2z_2)}}
\right);
\end{multline}

\vspace*{-12pt}

\noindent
\begin{multline}
\label{n11}
\left(z_1-\beta_1(\sigma-a_1z_1-a_2z_2)\right)
p_1\left(z_1,z_2,0\right)+{}\\
{}+\left(z_2-\beta_2(\sigma-a_1z_1-a_2z_2)\right)
\fr{\beta_1(\sigma-a_1z_1-a_2z_2)}{z_2-\beta_2(\sigma-a_2z_2)}\times{}
\\
{}\times p_1\left(0,z_2,0\right)=
\fr{\beta_1(\sigma-a_1z_1-a_2z_2)}{\sigma-a_2\pi_2(a_1)}\,p_1\left(0,\pi_2(a_1)\right)\times{}\\
{}\times\left(
\vphantom{\fr{z_2-\beta_2(s+\sigma-a_1z_1-a_2z_2)}{z_2-\beta_2(s+\sigma-a_2z_2)}}
a_1z_1+a_2z_2-\sigma+\left(\sigma-a_2z_2\right)\times{}\right.\\
\left.{}\times
\fr{z_2-\beta_2(\sigma-a_1z_1-a_2z_2)}
{z_2-\beta_2(\sigma-a_2z_2)}\right),
\end{multline}
\textit{где}
\begin{multline}
\label{n12}
\fr{z_2-h_2(a_2-a_2z_2)}{z_2-\beta_2(\sigma-a_2z_2)}\,
p_1(0,z_2,0)={}\\
{}=\fr{p_1(0,\pi_2(a_1),0)}{\sigma-a_2\pi_2(a_1)}
\left(
\vphantom{\fr{z_2-\beta_2(s+\sigma-a_1z_1-a_2z_2)}{z_2-\beta_2(s+\sigma-a_2z_2)}}
a_1\pi_1\left(a_2-a_2z_2\right)+
a_2z_2-\sigma+{}\right.\\
\left.{}+\left(a_2z_2-\sigma\right)
\fr{h_2(a_2-a_2z_2)-z_2}
{z_2-\beta_2(\sigma-a_2z_2)}\right);
\end{multline}

\vspace*{-12pt}

\noindent
\begin{gather*}
p_1\left(0,\pi_2(a_1),0\right)=
a_1^{-1}(1-\rho)\left(\sigma-a_2\pi_2(a_1)\right);\\
 h_2(s)=\beta_2\left(s+a_1-a_1\pi_1(s)\right),
\end{gather*}
\textit{а $\pi_2(s)$  есть преобразование Лап\-ла\-са--Стилть\-еса функции распределения периода занятости системы $M|G|1|\infty$ с~интенсивностью входящего
потока~$a_2$ и~функцией распределения времени обслуживания~$B_2(x).$}

\smallskip

\noindent
Д\,о\,к\,а\,з\,а\,т\,е\,л\,ь\,с\,т\,в\,о\,.\ \
 Рассматривая два соседних момента ухода требований первого приоритета 
 из системы, имеем:
 
 \noindent
\begin{multline*}
P_1\left(n_1,n_2,x\right)=
\sum\limits_{i_1=1}^{n_1+1}\sum\limits_{i_2=0}^{n_2}P_1(i_1,i_2,\infty)\times{}\\
{}\times
\int\limits_0^xe^{-\sigma u}
\fr{(a_1u)^{n_1-i_1+1}}{(n_1-i_1+1)!}\,
\fr{(a_2u)^{n_2-i_2}}{(n_2-i_2)!}\,dB_1(u)+{}\\
{}+
\sum\limits_{i_2=0}^{n_2}P_1(0,i_2,\infty)\int\limits_0^x
\sum\limits_{k_2=1}^{n_2+1}\int\limits_0^u
p^{(i_2)}\left(k_2,v,u\right)\times{}\\
{}\times \int\limits_0^{x-u}e^{-\sigma\tau}
\fr{(a_1\tau)^{n_1}}{n_1!}\,
\fr{(a_2\tau)^{n_2-k_2+1}}{(n_2-k_2+1)!}\times{}
\end{multline*}

%\vspace*{-12pt}

\noindent
\begin{multline}
{}\times
d\left(1-e^{-a_1u}\right)d_{\tau}\left(B_2^{(v)}*B_1(\tau)\right)dv+{}\\
{}+
\sum\limits_{i_2=0}^{n_2}P_1(0,i_2,\infty)\int\limits_0^x p^{(i_2)}(0,u)
\int\limits_0^{x-u}e^{-\sigma\tau}\times{}\\
{}\times\fr{(a_1\tau)^{n_1}}{n_1!}\,\fr{(a_2\tau)^{n_2}}{(n_2)!}\,
d\left(1-e^{-a_1u}\right)dB_1(\tau),
\label{n13}
\end{multline}
где
$$
B_2^{(v)}(x)=\fr{B_2(x+v)-B_2(v)}{1-B_2(v)},
$$
а функции $p^{(i_2)}(k_2,v,u)$ и~$p^{(i_2)}(0,u)$ вычисляются при $a\hm=a_2$ 
и~$B(x)\hm=B_2(x).$

Переходя в~\eqref{n13} к~производящим функциям и~преобразованиям Лап\-ла\-са--Стилть\-еса, 
получаем~\eqref{n10}.
Подставляя в~\eqref{n10} $s\hm=0,$ получаем~\eqref{n11}. При $z_1\hm=\pi_1(a_2\hm-a_2z_2)$ 
первое слагаемое в~левой
час\-ти~\eqref{n11} обращается в~нуль. Отсюда следует~\eqref{n12}. 
Устремляя в~\eqref{n12} $z_2\hm\rightarrow 1,$
находим $p_1(0,\pi_2(a_1),0)\hm=a_1^{-1}(1-\rho)(\sigma\hm-a_2\pi_2(a_1)).$

Рассмотрим три последовательных момента ухода из системы требований 
первого приоритета. Имеем:

\noindent
\begin{multline*}
Q_1\left(n_1,n_2,m_1,m_2,x,y\right)=P_1\left(m_1,m_2,y\right)\times{}\\
{}\times
\int\limits_0^xe^{-\sigma u}
\fr{(a_1u)^{n_1-m_1+1}}{(n_1-m_1+1)!}\,\fr{(a_2u)^{n_2-m_2}}{(n_2-m_2)!}\,dB_1(u)\\
\mbox{при } m_1\geqslant 1,\ n_1\geqslant m_1-1\,\ n_2\geqslant m_2;
\end{multline*}

\vspace*{-12pt}

\noindent
\begin{multline*}
Q_1\left(n_1,n_2,0,m_2,x,y\right)=
P_1\left(0,m_2,y\right)\times{}\\
{}\times \int\limits_0^x\sum\limits_{k_2=1}^{n_2+1}
\int\limits_0^u p^{(m_2)}\left(k_2,v,u\right)
\int\limits_0^{x-u}e^{-\sigma\tau}\fr{(a_1\tau)^{n_1}}{n_1!}\times{}\\
{}\times
\fr{(a_2\tau)^{n_2}}{n_2!}\,d\left(1-e^{-a_1u}\right)d_\tau\left(B_2^{(v)}*B_1(\tau)\right)
dv+{}\\
{}+
P_1\left(0,m_2,y\right)\int\limits_0^x p^{(m_2)}(0,u)
\int\limits_0^{x-u}e^{-\sigma\tau}
\fr{(a_1\tau)^{n_1}}{n_1!}\times{}\\
{}\times
\fr{(a_2\tau)^{n_2}}{n_2!}\,d\left(1-e^{-a_1u}\right)
dB_1(\tau)\\ 
\mbox{при}\ n_1\geqslant 0,\ n_2\geqslant 0,\ m_2\geqslant 0,
\end{multline*}

\vspace*{-12pt}

\noindent
\begin{multline*}
Q_1\left(n_1,n_2,m_1,m_2,x,y\right)=0 \\
 \mbox{при остальных}\ n_1,\ n_2,\ m_1,\ m_2.
\end{multline*}
Переходя в~этих соотношениях к~преобразования Лап\-ла\-са--Стилть\-еса 
и~производящим функциям, получаем~\eqref{n9}.

%\pagebreak

%\smallskip

\noindent
\textbf{Теорема~4.}\
\textit{Справедливы следующие соотношения}:
\begin{multline*}
f_1(s)=\beta_1(s)\left(1-\fr{1-\beta_2(s)}{1-\beta_2(s+a_1)}\times{}\right.\\
\left.{}\times
\left(1-\rho+a_1^{-1}a_2\left(1-\beta_2(a_1)\right)+
(1-\rho)\times{}\right.\right.
\\
{}\times
\fr{a_1(1-\beta_2(s))-s(\beta_2(s)-\beta_2(s+a_1))}
{(1-\beta_2(s+a_1))(s+\sigma-a_2\pi_2(s+a_1))}\times{}\\
{}\times 
\left(a_1^{-1}
\fr{\pi_2(s+a_1)-\beta_2(\sigma-a_2\pi_2(s+a_1))}
{\pi_2(s+a_1)-h_2(a_2-a_2\pi_2(s+a_1))}\times{} \right.
\\
{}\times
\left(a_1\pi_1\left(a_2-a_2\pi_2(s+a_1)\right)+a_2\pi_2(s+a_1)-\sigma\right)+{}\\
\left.\left.{}+a_1^{-1}\left(\sigma-a_2\pi_2(a_1)\right)
\vphantom{\fr{1-\beta_2(s)}{1-\beta_2(s+a_1)}}
\right)\right);
\end{multline*}

\vspace*{-22pt}

\noindent
\begin{multline*}
g_1(s_1,s_2)=\beta_1(s_1)\left(
\vphantom{\fr{1-\beta_2(s_1)}{1-\beta_2(s_1+a_1)}}
p_1(1,1,s_2)-{}\right.\\
{}-
\fr{1-\beta_2(s_1)}{1-\beta_2(s_1+a_1)}
p_1\left(0,1,s_2\right)+{}
\end{multline*}

\noindent
\begin{multline*}
{}+
\fr{p_1(0,\pi_2(s_1+a_1),s_2)}
{s_1+\sigma-a_2\pi_2(s_1+a_1)}\times{}\\[6pt]
\left.{}\times
\fr{a_1(1-\beta_2(s_1))-s_1\left(\beta_2(s_1)-\beta_2(s_1+a_1)\right)}
{1-\beta_2\left(s_1+a_1\right)}\right).
\end{multline*}

\vspace*{-16pt}

{\small\frenchspacing
 {%\baselineskip=10.8pt
 \addcontentsline{toc}{section}{References}
 \begin{thebibliography}{9}
\bibitem{1-us}
\Au{Nain P.} 
Interdeparture times from a queuing system with preemptive resume priority~// 
Perform. Evaluation, 1984. Vol.~4. Iss.~2. P.~93--98.
\bibitem{2-us}
\Au{Stanford D. A.} Interdeparture time distributions in the 
non-preemptive priority $\Sigma\ M_i|G_i|1$ queue~// Perform. Evaluation, 1991. 
Vol.~12. Iss.~2.   P.~43--60.
\bibitem{3-us}
\Aue{Stanford D.\,A.} Waiting and interdeparture times in priority queues with 
Poisson and general arrival streams~// Oper.
Res., 1995. Vol.~45. Iss.~5. P.~725--735.

 \end{thebibliography}

 }
 }

\end{multicols}

\vspace*{-12pt}

\hfill{\small\textit{Поступила в~редакцию 12.09.19}}

\vspace*{6pt}

%\pagebreak

%\newpage

%\vspace*{-28pt}

\hrule

\vspace*{2pt}

\hrule

\vspace*{-4pt}

\def\tit{THE OUTPUT STREAMS IN~THE~SINGLE SERVER QUEUEING SYSTEM WITH~A~HEAD 
OF~THE~LINE PRIORITY\\[-5pt]}


\def\titkol{The output streams in~the~single server queueing system with~a~head 
of~the~line priority}

\def\aut{V.\,G.~Ushakov$^{1,2}$ and N.\,G.~Ushakov$^{3,4}$\\[-5pt]}

\def\autkol{V.\,G.~Ushakov and N.\,G.~Ushakov}

\titel{\tit}{\aut}{\autkol}{\titkol}

\vspace*{-24pt}


\noindent
$^1$Department of Mathematical Statistics, Faculty of Computational 
Mathematics and Cybernetics, M.\,V.~Lomo-\linebreak
$\hphantom{^1}$nosov Moscow State University, 
1-52~Leninskiye Gory, Moscow 119991, GSP-1, Russian Federation

\noindent
$^2$Institute of Informatics Problems, Federal Research Center ``Computer Science 
and Control'' of the Russian \linebreak
$\hphantom{^1}$Academy of Sciences, 44-2~Vavilov Str., 
Moscow 119333, Russian Federation

\noindent
$^3$Institute of Microelectronics Technology and High-Purity Materials of the 
Russian Academy of Sciences,\linebreak
$\hphantom{^1}$6~Academician Osipyan Str., Chernogolovka, 
Moscow Region 142432, Russian Federation

\noindent
$^4$Norwegian University of Science and Technology, 
15A~S.\,P.~Andersensvei, Trondheim 7491, Norway

\def\leftfootline{\small{\textbf{\thepage}
\hfill INFORMATIKA I EE PRIMENENIYA~--- INFORMATICS AND
APPLICATIONS\ \ \ 2019\ \ \ volume~13\ \ \ issue\ 4}
}%
 \def\rightfootline{\small{INFORMATIKA I EE PRIMENENIYA~---
INFORMATICS AND APPLICATIONS\ \ \ 2019\ \ \ volume~13\ \ \ issue\ 4
\hfill \textbf{\thepage}}}

\vspace*{3pt}  


 

\Abste{The paper studies a single server queuing system with two types of 
customers, head of the line priority, and an infinite number of positions in the queue.  The arrival stream of customers of each type is a Poisson stream.
Each type has its own generally distributed service time characteristics. 
The main result is the Laplace--Stieltjes
transform  of one- and two-dimensional stationary distribution functions 
of the interdeparture time for each type of
customers.
The analysis of the output process is carried out
 by the method of embedded Markov chains. As embedded times,
  successive moments of the end of service of the same type of 
  customers are selected. From the practical perspective, an accurate 
  characterization of the interdeparture time process is necessary
   when studying open networks of queues.}

\KWE{output stream; head of the line priority; embedded Markov chain; single server}





  \DOI{10.14357/19922264190407} 

\vspace*{-22pt}

\Ack
\noindent
The reported study was funded by the Russian Foundation for Basic
Research (project number 18-07-00678).



\vspace*{-6pt}

  \begin{multicols}{2}

\renewcommand{\bibname}{\protect\rmfamily References}
%\renewcommand{\bibname}{\large\protect\rm References}

{\small\frenchspacing
 {%\baselineskip=10.8pt
 \addcontentsline{toc}{section}{References}
 \begin{thebibliography}{9}

\bibitem{1-us-1}
\Aue{Nain, P.} 1984. Interdeparture times from 
a~queuing system with preemptive resume priority. 
\textit{Perform. Evaluation} 4(2):93--98.

\bibitem{2-us-1}
\Aue{Stanford, D.\,A.} 1991. Interdeparture time distributions in the 
non-preemptive priority $\Sigma\ M_i|G_i|1$ queue. 
\textit{Perform. Evaluation} 12(2):43--60.

\bibitem{3-us-1}
\Aue{Stanford, D.\,A.} 1995. Waiting and interdeparture times in priority
 queues with Poisson and general arrival streams. 
 \textit{Oper. Res.} 45(5):725--735.
\end{thebibliography}

 }
 }

\end{multicols}

\vspace*{-7pt}

\hfill{\small\textit{Received September 12, 2019}}

%\pagebreak

%\vspace*{-22pt}

\Contr

\noindent
\textbf{Ushakov Vladimir G.} (b.\ 1952)~--- 
Doctor of Science in physics and mathematics, professor, Department of 
Mathematical Statistics, Faculty of Computational Mathematics and Cybernetics, 
M.\,V.~Lomonosov Moscow State University, 1-52~Leninskiye Gory, 
Moscow 119991, GSP-1, Russian Federation; senior scientist, Institute 
of Informatics Problems, Federal Research Center ``Computer Science 
and Control'' of the Russian Academy of Sciences, 44-2~Vavilov Str., 
Moscow 119333, Russian Federation; \mbox{vgushakov@mail.ru}

\vspace*{6pt}

\noindent
\textbf{Ushakov Nikolai G.} (b.\ 1952)~--- 
Doctor of Science in physics and mathematics, leading scientist, 
Institute of Microelectronics Technology and High-Purity Materials 
of the Russian Academy of Sciences, 6~Academician Osipyan Str., Chernogolovka, 
Moscow Region 142432, Russian Federation; professor, Norwegian University
 of Science and Technology, 15A~S.\,P.~Andersensvei, Trondheim 7491, 
 Norway; \mbox{ushakov@math.ntnu.no}
\label{end\stat}

\renewcommand{\bibname}{\protect\rm Литература}   %7
\def\stat{shestakov}

\def\tit{ОБРАЩЕНИЕ ОДНОРОДНЫХ ОПЕРАТОРОВ С~ПОМОЩЬЮ
СТАБИЛИЗИРОВАННОЙ ЖЕСТКОЙ ПОРОГОВОЙ ОБРАБОТКИ
ПРИ~НЕИЗВЕСТНОЙ ДИСПЕРСИИ ШУМА$^*$}

\def\titkol{Обращение однородных операторов с~помощью
стабилизированной жесткой пороговой обработки}
%при~неизвестной дисперсии шума}

\def\aut{О.\,В.~Шестаков$^1$}

\def\autkol{О.\,В.~Шестаков}

\titel{\tit}{\aut}{\autkol}{\titkol}

\index{Шестаков О.\,В.}
\index{Shestakov O.\,V.}


{\renewcommand{\thefootnote}{\fnsymbol{footnote}} \footnotetext[1]
{Работа выполнена при частичной финансовой поддержке РФФИ (проект 19-07-00352).}}


\renewcommand{\thefootnote}{\arabic{footnote}}
\footnotetext[1]{Московский государственный университет им.\ М.\,В.~Ломоносова, 
кафедра математической статистики факультета вычислительной математики и~кибернетики; 
Институт проб\-лем информатики Федерального исследовательского центра 
<<Информатика и~управ\-ле\-ние>> Российской академии наук, \mbox{oshestakov@cs.msu.su}}


\vspace*{-6pt}


\Abst{При обращении линейных однородных операторов обычно необходимо использовать 
методы регуляризации, поскольку наблюдаемые данные, как правило, зашумлены. 
Для подавления шума часто используется пороговая обработка 
вейвлет-ко\-эф\-фи\-ци\-ен\-тов функции наблюдаемого сигнала. 
Пороговая обработка стала популярным инструментом подавления 
шума благодаря своей простоте, вы\-чис\-ли\-тель\-ной эффективности и~воз\-мож\-ности 
адаптации к~функциям, имеющим на разных участках разную степень регулярности. 
Рассматривается предложенный недавно стабилизированный метод жесткой 
пороговой обработки, в~котором устранены основные недостатки мягкой и~жесткой 
пороговой обработки, и~исследуются статистические свойства этого метода. 
В~модели данных с~аддитивным гауссовским шумом с~неизвестной дисперсией 
проведен анализ несмещенной оценки среднеквадратичного риска и~показано, 
что при определенных условиях данная оценка является асимптотически нормальной, 
при этом дисперсия предельного распределения зависит от способа оценивания 
дисперсии шума.}

\KW{вейвлеты; пороговая обработка; несмещенная оценка риска; 
асимптотическая нормальность; сильная состоятельность}

\DOI{10.14357/19922264190107}
  
%\vspace*{4pt}


\vskip 10pt plus 9pt minus 6pt

\thispagestyle{headings}

\begin{multicols}{2}

\label{st\stat}

\section{Введение}

В медицинских, физических, астрономических и~других научных проблемах часто 
возникает задача получить представление об объекте, который описывается 
некоторой функцией~$f$, имея возможность наблюдать только функцию~$Kf$, где~$K$~--- 
некоторый линейный оператор. При этом часто нельзя просто применить 
к~наблюдаемым данным обратный оператор~$K^{-1}$, поскольку эти данные, как правило, 
содержат шум и~задача обращения оператора~$K$ некорректно поставлена. 
К~тому же обычно дис\-пер\-сия шума неизвестна и~ее необходимо оценивать 
по наблюдаемым данным. 

Одним из популярных инструментов при регуляризации 
процедуры обращения служит вейв\-лет-раз\-ло\-же\-ние с~последующей 
пороговой обработкой вейв\-лет-ко\-эф\-фи\-ци\-ен\-тов. Наиболее распростра\-нен\-ные 
виды пороговой обработки~--- жесткая и~мягкая. В~работе~\cite{HL10} 
был предложен метод стабилизированной жесткой пороговой обработки, который 
объединяет в~себе преимущества этих двух видов. 
В~ситуации, когда дисперсия шума предполагается известной, в~работе~\cite{SH18} 
доказана асимптотическая нормальность оценки среднеквадратичного риска пороговой 
обработки. 

В~данной работе исследуется влияние способов оценивания дисперсии шума 
на характеристики предельного распределения оценки среднеквадратичного риска. 
Для метода мягкой пороговой обработки подобные исследования проводились 
в~работах~\cite{KS11-1, KS11-2}.

\section{Обращение линейных однородных операторов с~помощью вейглет-вейвлет-разложения}

В данной работе рассматривается метод обращения линейных однородных операторов, 
основанный на вейг\-лет-вейв\-лет-раз\-ло\-же\-нии~\cite{AS98}. Линейный оператор~$K$ 
называется однородным, если
$$
K\left[f\left(a\left(x-x_0\right)\right)\right]=a^{-\alpha}(Kf)\left[a\left(x-x_0\right)\right]
$$
для любого $x_0$ и~любого $a\hm>0$. Параметр~$\alpha$ называется показателем 
однородности. Примерами линейных однородных операторов служат оператор 
интегрирования, преобразование Гильберта и~преобразование Абеля.

Относительно наблюдаемой функции~$Kf$ будем предполагать, что она определена на 
конечном отрезке и~равномерно регулярна по Липшицу с~некоторым показателем $\gamma\hm>0$. 
Вейв\-лет-разложение~$Kf$ представляет собой ряд по ортонормированному базису
\begin{equation}
\label{wavelet_decomp}
Kf = \sum\limits_{j,k \in Z} \langle Kf,\psi_{j,k} \rangle \psi_{j,k}\,,
\end{equation}
где $\psi(t)$~--- некоторая материнская вейв\-лет-функ\-ция, 
а~$\psi_{j,k}(t) \hm= 2^{j/2}\psi(2^jt \hm- k)$. Индекс~$j$ в~(\ref{wavelet_decomp}) 
называется масштабом, а~индекс~$k$~--- сдвигом. Если вейв\-лет-функ\-ция 
обладает определенными свойствами регулярности~\cite{Mal99}, 
то для коэффициентов разложения в~(\ref{wavelet_decomp}) справедливо
\begin{equation}
\label{wavelet_decay}
\abs{\langle Kf, \psi_{j,k} \rangle} \leqslant \fr{C_f}
{2^{j \left( \gamma + 1/2 \right)}}\,,
\end{equation}
где $C_f$~--- некоторая положительная константа.

Поскольку оператор~$K$ линеен и~однороден, существуют такие функции~$u_{j,k}$, 
что $\langle f,u_{j,k}\rangle\hm=\langle Kf,\psi_{j,k}\rangle$. При этом функция~$f$ 
представляется в~виде ряда
\begin{equation}
\label{VWD}
f = \sum\limits_{j,k \in Z}\beta_{j,k}\langle Kf,\psi_{j,k}\rangle u_{j,k},
\end{equation}
где $u_{j,k} = K^{-1}\psi_{j,k}/\beta_{j,k}$, $\beta_{j,k}\hm=2^{\alpha j}\beta_{00}$, 
$\beta_{00} \hm= \norm{K^{-1}\psi}$ (функции~$u_{j,k}$, как и~$\psi_{j,k}$, 
представляют собой сдвиги и~растяжения одной материнской функции~$u$ и~называются 
вейглетами). При соответствующем выборе~$\psi(t)$ последовательность~$\{u_{j,k}\}$ 
образует устойчивый базис~\cite{L97}. Формула~(\ref{VWD}) и~есть основа метода 
вейг\-лет-вейв\-лет-раз\-ло\-же\-ния.

\section*{Пороговая обработка эмпирических коэффициентов}

При фактических измерениях значения функции сигнала регистрируются 
в~дискретных отсчетах, при этом такие значения, как правило, зашумлены. 
Рассмотрим сле\-ду\-ющую модель данных \mbox{с~шумом}:
\begin{equation*}
%\label{Data_Model}
X_i = (Kf)_i + \epsilon_i\,, \enskip i = 1, \dots, 2^J\,, %\notag
\end{equation*}
где $2^J$~--- число отсчетов; $(Kf)_i$~--- незашумленные значения функции сигнала; 
$\epsilon_i$~--- независимые нормально распределенные случайные величины с~нулевым 
средним и~дисперсией~$\sigma^2$.
После применения дискретного вейв\-лет-пре\-об\-ра\-зо\-ва\-ния 
получается следующая модель зашумленных вейв\-лет-ко\-эф\-фи\-ци\-ен\-тов:
\begin{equation*}
Y_{j,k}=\mu_{j,k}+\epsilon^W_{j,k},\enskip 
j=0,\ldots,J-1,\ k=0,\ldots,2^{j}-1\,,
\end{equation*}
где $\epsilon^W_{j,k}$ независимы и~распределены так же, как и~$\epsilon_i$, 
а~$\mu_{j,k}\hm= 2^{J/2}\langle Kf,\psi_{j,k}\rangle$~\cite{Mal99}.

Для подавления шума и~построения оценки функции сигнала к~коэффициентам~$Y_{j,k}$ 
обычно применяется функция жесткой пороговой обработки 
$\rho_{H}(y,T)\hm=x\textbf{I}(\abs{y}>T)$ или мягкой пороговой 
обработки $\rho_{S}(y,T)\hm=\textbf{sgn}(x)\left(\abs{y}-T\right)_{+}$ 
с~порогом~$T$. При таком подходе обнуляются коэффициенты, абсолютная величина 
которых ниже порога, так как в~силу~(\ref{wavelet_decay}) основная часть
 полезного сигнала содержится в~относительно небольшом числе больших по 
 модулю коэффициентов.

Каждому из этих видов пороговой обработки присущи свои недостатки. 
Жесткая пороговая функция разрывна, и~это приводит к~отсутствию устойчивости 
при выборе порога~\cite{B96} и~невозможности построения несмещенной оценки 
среднеквадратичного риска~\cite{J01}. При мягкой пороговой обработке в~оценке 
функции появляется дополнительное смещение. Чтобы частично избежать этих недостатков, 
в~работе~\cite{HL10} был предложен новый вид пороговой обработки, представляющий 
собой сглаженный (стабилизированный) аналог жесткой пороговой обработки. 
В~этом методе оценки~$\mu_{j,k}$ вычисляются по формулам:
\begin{equation*}
\widehat{\mu}_{j,k}=\Expect 
\left[\rho_{H}(Y_{j,k}+\lambda\xi_{j,k},T_j)|Y_{j,k}\right], %\notag
\end{equation*}
где случайные величины~$\xi_{j,k}$ имеют стандартное нормальное распределение и~не 
зависят от~$Y_{j,k}$, а~$\lambda\hm>0$~--- 
параметр стабилизации, отвечающий за степень сглаживания. Вычисляя математическое 
ожидание, получаем:
\begin{multline*}
\hspace*{-8.37947pt}\widehat{\mu}_{j,k}=Y_{j,k}\left[\Phi\!\left(-\fr{T_j+Y_{j,k}}
{\lambda}\right)+1-\Phi\left(\fr{T_j-Y_{j,k}}{\lambda}\right)\!\right]+{}\\
{}+
\lambda\left[\phi\left(\fr{T_j-Y_{j,k}}{\lambda}\right)-
\phi\left(\fr{T_j+Y_{j,k}}{\lambda}\right)\right]. %\notag
\end{multline*}
Достоинством такого метода является бесконечная дифференцируемость~$\widehat{\mu}_{j,k}$ 
по~$Y_{j,k}$, что приводит к~более робастным оценкам~\cite{HL10}. Заметим также, 
что при $\lambda\hm\to0$ получается обычный метод жесткой пороговой обработки. 
В~данной работе параметр~$\lambda$ предполагается фиксированным, а~в~качестве~$T_j$ 
для каждого масштаба~$j$ выбирается порог $T_j\hm=\sigma\sqrt{2\ln 2^j}$. 
Такой порог получил название <<универсальный>>, так как он не зависит 
от наблюдаемых данных. И~при жесткой, и~при мягкой пороговой обработке этот 
порог обеспечивает близость среднеквадратичного риска к~минимальному~\cite{Mal99}.

\section{Несмещенная оценка среднеквадратичного риска}

Среднеквадратичный риск метода пороговой обработки определяется по формуле:
\begin{equation}
\label{Risk}
R_J(\sigma)=\sum\limits_{j=0}^{J-1}\sum\limits_{k=0}^{2^j-1}\beta^2_{j,k}
\Expect\left(\widehat{\mu}_{j,k}(\sigma)-\mu_{j,k}\right)^2.
\end{equation}
В~\cite{HL10} показано, что при стабилизированной жесткой пороговой обработке
\begin{multline*}
\Expect\left(\widehat{\mu}_{j,k}(\sigma)-\mu_{j,k}\right)^2={}\\
{}=
\Expect\left[(Y_{j,k}-\widehat{\mu}_{j,k}(\sigma))^2+
2\sigma^2\fr{\partial}{\partial Y_{j,k}}\,\widehat{\mu}_{j,k}(\sigma)\right]-
\sigma^2, %\notag
\end{multline*}
где
\begin{multline*}
\fr{\partial}{\partial Y_{j,k}}\widehat{\mu}_{j,k}(\sigma)={}\\
{}=\Phi\left(-\fr{T_j+Y_{j,k}}{\lambda}\right)+1-
\Phi\left(\fr{T_j-Y_{j,k}}{\lambda}\right)+{}\\
{}+
\fr{T_j}{\lambda}\left[\phi\left(\fr{T_j-Y_{j,k}}{\lambda}\right)+
\phi\left(\fr{T_j+Y_{j,k}}{\lambda}\right)\right]. %\notag
\end{multline*}
Таким образом, величина
\begin{multline}
\label{Risk_Estimate}
\widehat{R}_J(\sigma)=\sum\limits_{j=0}^{J-1}\sum\limits_{k=0}^{2^j-1}
\beta^2_{j,k}
\Bigg[
\left(
Y_{j,k}-
\widehat{\mu}_{j,k}(\sigma)\right)^2+{}\\
{}+2\sigma^2\fr{\partial}{\partial Y_{j,k}}\,\widehat{\mu}_{j,k}(\sigma)-
\sigma^2
\Bigg]
\end{multline}
является несмещенной оценкой~$R_J$, не зависящей от ненаблюдаемых значений~$\mu_{j,k}$.

В работе~\cite{SH18} доказано следующее утверждение, устанавливающее 
асимптотическую нормальность оценки~(\ref{Risk_Estimate}) и~позволяющее строить 
асимптотические доверительные интервалы для риска~(\ref{Risk}).

\smallskip

\noindent
\textbf{Теорема 1.} 
\textit{Пусть $K$~--- линейный однородный оператор с~показателем 
однородности $\alpha\hm>0$, а~$Kf$ задана на конечном отрезке и~равномерно 
регулярна по Липшицу с~показателем $\gamma\hm>0$. Тогда}
\begin{equation*}
%\label{Normality}
{\sf P}\left(\fr{\widehat{R}_J(\sigma)-
R_J(\sigma)}{D_J}<x\right)\Rightarrow\Phi(x)\,, %\notag
\end{equation*}
\textit{где}
$$
D^2_J=\fr{2\sigma^4\beta_{0,0}^4}{2^{4\alpha+1}-1}2^{(4\alpha+1)J}\,.
$$

\section{Виды оценок дисперсии шума}

Как правило, дисперсия~$\sigma^2$ неизвестна и~вместо ее точного значения 
необходимо использовать некоторую оценку~$\hat{\sigma}^2$, которая обычно 
строится по половине всех вейв\-лет-ко\-эф\-фи\-ци\-ен\-тов для $j\hm=J\hm-1$, 
так как в~силу~(\ref{wavelet_decay}) эти коэффициенты фактически содержат только шум. 
При этом порог вычисляется по формуле $\hat{T}_j\hm=\hat{\sigma}\sqrt{2\ln 2^j}$.

В качестве оценки~$\sigma^2$ (или $\sigma$) в~данной работе 
рассматривается выборочная дисперсия
\begin{equation}
\label{SampleVarianceDef}
\widehat{\sigma}_S^2=\fr{1}{2^{J-1}}
\sum\limits_{k=0}^{2^{J-1}-1}Y_{J-1,k}^2-\overline{Y}^2,
\end{equation}
где
\begin{equation*}
\overline{Y}=\fr{1}{2^{J-1}}\sum\limits_{k=0}^{2^{J-1}-1}Y_{J-1,k}\,,
\end{equation*}
а также соответствующим образом нормированный выборочный интерквартильный 
размах~$\widehat{\sigma}_{R}$ и~выборочное абсолютное медианное 
отклонение~$\widehat{\sigma}_{M}$, которые определяются сле\-ду\-ющим образом:
\begin{align}
\widehat{\sigma}_{R}&=\fr{Y_{(J-1,3/4)}-Y_{(J-1,1/4)}}{2\xi_{3/4}}\,;
\label{IQR_Definition}
\\
\widehat{\sigma}_{M}&=\fr{\mathop{\mbox{med}}\limits_{0\leqslant k\leqslant 2^{J-1}-1}|Y_{J-1,k}-\mathop{\mbox{med}}\limits_{0\leqslant l\leqslant 2^{J-1}-1} Y_{J-1,l}|}{\xi_{3/4}}\,.
\label{MAD_Definition}
\end{align}
Здесь $Y_{(J-1,1/4)}$ и~$Y_{(J-1,3/4)}$~--- выборочные квантили порядка~$1/4$ и~$3/4$, 
построенные по выборке из половины всех вейв\-лет-ко\-эф\-фи\-ци\-ен\-тов при 
$j\hm=J\hm-1$; $\xi_{3/4}$~--- теоретическая квантиль порядка~$3/4$ 
стандартного нормального распределения ($\xi_{3/4}\hm\approx0,6745$); $\mbox{med}$ 
обозначает выборочную медиану.

Выборочная дисперсия служит самой популярной оценкой величины~$\sigma^2$, и~в~случае 
отсутствия выбросов она наиболее предпочтительна. Однако в~случае, когда 
оценка дисперсии строится по выборке сигнала, естественно ожидать, 
что выборка не будет однородной. Преимущество использования последних 
двух оценок заключается в~их ро\-баст\-ности, т.\,е.\ нечувствительности к~выбросам.

\section{Предельная дисперсия оценки среднеквадратичного риска}

Способ оценивания дисперсии шума влияет на вид предельной дисперсии 
оценки среднеквадратичного риска. Подобный эффект наблюдается и~при 
мягкой пороговой обработке~[4].

\noindent
\textbf{Теорема~2.}\ \textit{Пусть $Kf$ задана на конечном отрезке и~равномерно 
регулярна по Липшицу с~показателем $\gamma\hm>1/4$, а оценка дисперсии 
шума задана соотношением}~\eqref{SampleVarianceDef}. \textit{Тогда}
\begin{equation}
\label{CLT_Operator_SampleVar_Sigma}
\mathsf{P}\left(\frac{\widehat{R}_J(\widehat{\sigma}_S)-R_J(\sigma)}{D_J}<x\right)
\Rightarrow \Phi_{\Upsilon_1}(x),\notag
\end{equation}
\textit{где $\Phi_{\Upsilon_1}(x)$~--- функция распределения нормального 
закона с~нулевым средним и~дисперсией}
$$
\Upsilon_1^2=\fr{1}{2^{4\alpha+1}}+
\fr{2^{4\alpha+1}-1}{2^{4\alpha+1}\left(2^{2\alpha+1}-1\right)^2}\,.
$$

\noindent
Д\,о\,к\,а\,з\,а\,т\,е\,л\,ь\,с\,т\,в\,о\,.\ \ Обозначим
\begin{multline*}
\widehat{U}_J(\sigma)=\sum\limits_{j=0}^{J-1}\sum\limits_{k=0}^{2^j-1}
\beta^2_{j,k}\Bigg[
\left(Y_{j,k}-\widehat{\mu}_{j,k}(\sigma)\right)^2+{}\\
{}+2\sigma^2\fr{\partial}{\partial Y_{j,k}}\widehat{\mu}_{j,k}(\sigma)\Bigg] %\notag
\end{multline*}
и запишем $\widehat{R}_J(\hat{\sigma}_S)-R_J(\sigma)$ в~виде
\begin{multline*}
%\label{Three_Sums}
\widehat{R}_J(\hat{\sigma}_S)-R_J(\sigma)={}\\
{}=\left[\widehat{U}_J(\hat{\sigma}_S)-\widehat{U}_J(\sigma)\right]+
\left[\widehat{R}_J(\sigma)-R_J(\sigma)\right]+{}\\
{}+
\fr{2^{(2\alpha+1)J}-1}{2^{2\alpha+1}-1}(\sigma^2-\hat{\sigma}^2_S)
\equiv S_1+S_2+S_3\,.
\end{multline*}

Повторяя рассуждения из работ~\cite{KS11-1, KS11-2} и~учитывая, что если $\gamma\hm>1/4$, 
то выполнено $2^{J/2}\overline{Y}^2\stackrel{{\sf P}}{\to} 0$ при 
$J\hm\rightarrow\infty$~\cite{KS11-2}, можно показать, что
\begin{equation*}
{\sf P}\left(\fr{S_2+S_3}{D_J}<x\right)\Rightarrow\Phi_{\Upsilon_1}(x)\,.%\notag
\end{equation*}
% на самом деле с~условием Линдеберга чуть по-другому (без ограниченности слагаемых). Но дисперсия равномерно ограничена -- значит выполнено.

Докажем, что $D_J^{-1}S_1\stackrel{{\sf P}}{\to}0$ при $J\hm\rightarrow\infty$. 
Пусть $C_\delta\hm>0$~--- некоторая константа, а $\delta_J\hm=C_\delta J^{1/2}2^{-J/2}$. 
Запишем
\begin{multline*}
S_1=\mathbf{1}\left(\abs{\sigma^2-\hat{\sigma}^2_S}>\delta_J\right)S_1+{}\\
{}+
\mathbf{1}\left(\abs{\sigma^2-\hat{\sigma}^2_S}\leqslant\delta_J\right)
S_1\equiv S'_1+S''_1. %\notag
\end{multline*}
Для произвольного $\varepsilon\hm>0$
\begin{equation*}
{\sf P}\left(S'_1>\varepsilon\right)\leqslant{\sf P}
\left(\abs{\sigma^2-\hat{\sigma}^2_S}>\delta_J\right). %\notag
\end{equation*}
При выполнении условий теоремы, если константа~$C_\delta$ достаточно велика, 
то найдется константа~$\tilde{C}_\delta>0$ такая, что~\cite{KS11-2}
\begin{equation*}
{\sf P}\left(\abs{\sigma^2-\hat{\sigma}^2_S}>\delta_J\right)
\leqslant\tilde{C}_\delta2^{-J/2}. %\notag
\end{equation*}
%% комментарии по поводу этого неравенства и~загрязнения выборки есть в~диссертации
Следовательно, $S'_1\stackrel{P}{\to}0$ при $J\hm\rightarrow\infty$.

Обозначим слагаемые в~сумме~$S''_1$ через~$F_{j,k}(\hat{\sigma}_S)$. Пусть 
$A_j\hm=\sqrt{A\ln 2^j}$, где $0\hm<A\hm<2(\sigma^2\hm-\delta_J)$. Имеем:

\noindent
\begin{multline*}
\hspace*{-9.9pt}\sum\limits_{j=0}^{J-1}\sum\limits_{k=0}^{2^j-1}F_{j,k}\left(\hat{\sigma}_S\right)=
\sum\limits_{j=0}^{J-1}\sum\limits_{k=0}^{2^j-1}
\mathbf{1}(\abs{Y_{j,k}}\leqslant A_j)F_{j,k}(\hat{\sigma}_S)+{}\\
{}+
\sum\limits_{j=0}^{J-1}\sum\limits_{k=0}^{2^j-1}
\mathbf{1}\left(\abs{Y_{j,k}}>A_j\right)F_{j,k}(\hat{\sigma}_S)
\equiv  W_1+W_2. %\notag
\end{multline*}
Рассмотрим $W_1$. Учитывая определения $\widehat{\mu}_{j,k}(\sigma)$, 
$({\partial}/{\partial Y_{j,k}})\widehat{\mu}_{j,k}(\sigma)$ и~$A_j$, 
можно убедиться, что найдут\-ся константы $C_1\hm>0$ и~$\theta\hm>0$ такие, что
\begin{equation*}
\abs{\mathbf{1}\left(\abs{Y_{j,k}}\leqslant A_J\right)
F_{j,k}(\hat{\sigma}_S)}\leqslant C_1 
J^{5/2}2^{(2\alpha-\theta)j-J/2}\;\;\mbox{п.в.} %\notag
\end{equation*}
% поскольку выполнено \mathbf{1}(\abs{\sigma^2-\hat{\sigma}^2_S}\leqslant\delta_J). В логарифме степень: от Y идет 1, от T идет 1, от \delta_J идет 1/2 но для J, а не для j, поэтому берем для всех J^{5/2}. В степени 2: 2\alpha от \beta{j,k}, \theta из-за выбора A, J/2 от \delta_J
Следовательно, $D_J^{-1}W_1\hm\rightarrow 0$ п.в.\ при $J\hm\rightarrow\infty$.

Далее для слагаемых~$W_2$ имеем:
\begin{multline*}
\left\vert \mathbf{1}\left(
\left\vert Y_{j,k}\right\vert
> A_J\right)F_{j,k}
\left(\hat{\sigma}_S\right)\right\vert
\leqslant{}\\
{}\leqslant C_2 J^{3/2}2^{2\alpha j-J/2} 
\mathbf{1}\left( \left\vert Y_{j,k}\right\vert > A_J\right) 
\left\vert Y_{j,k}\right\vert^2\;\;\mbox{п.в.},
%\notag
\end{multline*}
% поскольку выполнено \mathbf{1}(\abs{\sigma^2-\hat{\sigma}^2_S}\leqslant\delta_J). В логарифме от T идет 1, от \delta_J идет 1/2.
где $C_2>0$~--- некоторая константа. Учитывая распределение~$Y_{j,k}$, 
нетрудно убедиться, что
\begin{equation*}
\Expect\frac{1}{D_J} \sum\limits_{j=0}^{J-1}
\sum\limits_{k=0}^{2^j-1} J^{3/2}2^{2\alpha j-J/2} 
\mathbf{1}\left(\abs{Y_{j,k}}> A_j\right)
\abs{Y_{j,k}}^2\to 0
\end{equation*}
при $J\rightarrow\infty$. %\notag
Следовательно, используя неравенство Маркова, получаем, что
\begin{equation*}
D_J^{-1}W_2\stackrel{{\sf P}}{\to}0\;\;\mbox{при}\;J\rightarrow\infty\,. %\notag
\end{equation*}
Таким образом, $D_J^{-1}S_1\stackrel{{\sf P}}{\to}0$ при $J\hm\rightarrow\infty$.

Теорема доказана.

\smallskip

Рассмотрим теперь ситуацию, когда в~качестве оценки~$\sigma$ используется 
величина~$\widehat{\sigma}_{R}$ или~$\widehat{\sigma}_{M}$. 
В~этом случае повышаются требования к~гладкости функции сигнала.

\smallskip

\noindent
\textbf{Теорема~3.}\
\textit{Пусть~$Kf$ задана на конечном отрезке и~равномерно регулярна по 
Липшицу с~показателем $\gamma\hm>1/2$, а оценка дисперсии шума~$\hat{\sigma}$ 
задана соотношением}~\eqref{IQR_Definition} 
\textit{или соотношением}~\eqref{MAD_Definition}. \textit{Тогда}
\begin{equation*}
\label{CLT_Operator_RobVar_Sigma}
\mathsf{P}\left(\fr{\widehat{R}_J(\widehat{\sigma})-R_J(\sigma)}{D_J}<x\right)
\Rightarrow \Phi_{\Upsilon_2}(x)\,, %\notag
\end{equation*}
где $\Phi_{\Upsilon_2}(x)$~--- функция распределения нормального закона 
с~нулевым средним и~дисперсией
\begin{multline*}
\Upsilon_2^2=1+\fr{2^{4\alpha+1}-1}{4(2^{2\alpha+1}-1)^2
\xi_{3/4}^2(\phi(\xi_{3/4}))^2}-{}\\
{}-
\fr{2^{4\alpha+1}-1 }{2^{2\alpha-1}(2^{2\alpha+1}-1)}\,.
\end{multline*}

\noindent
Д\,о\,к\,а\,з\,а\,т\,е\,л\,ь\,с\,т\,в\,о\,.\ \
Как и~в~предыдущей теореме, запишем
$\widehat{R}_J(\hat{\sigma})\hm-R_J(\sigma)\hm=S_1\hm+S_2\hm+S_3.$
Учитывая,\linebreak\vspace*{-12pt}

\pagebreak

\noindent
 что $\gamma\hm>1/2$, и~поступая, как в~работах~\cite{SH18, KS11-2, SH12}, 
с~использованием разложения Бахадура для выборочных квантилей~\cite{S80} и~выборочного 
абсолютного медианного отклонения~\cite{SM09}, можно показать, что
\begin{equation*}
{\sf P}\left(\fr{S_2+S_3}{D_J}<x\right)\Rightarrow\Phi_{\Upsilon_2}(x)\,. %\notag
\end{equation*}
% на самом деле с~условием Линдеберга чуть по-другому (без ограниченности слагаемых). Но дисперсия равномерно ограничена -- значит выполнено.

Используя экспоненциальные неравенства для выборочных квантилей~\cite{S80} 
и~выборочного абсолютного медианного отклонения~\cite{SM09}, получаем, что при 
выполнении условий теоремы найдется такая константа $C_\delta\hm>0$, что при 
$\delta_J\hm=C_\delta J^{1/2}2^{-J/2}$ для некоторой константы~$\widetilde{C}_\delta>0$ 
выполнено:
\begin{align*}
\mathsf{P}\left(\abs{\widehat{\sigma}_{R}-\sigma}>\delta_J\right)
&\leqslant\widetilde{C}_\delta2^{-J/2}\,;
\\
\mathsf{P}\left(\abs{\widehat{\sigma}_{M}-\sigma}>\delta_J\right)
&\leqslant\widetilde{C}_\delta2^{-J/2}\,. %\notag
\end{align*}
%% комментарии по поводу этого неравенства и~загрязнения выборки есть в~диссертации
Далее, повторяя рассуждения предыдущей теоремы, заключаем, что 
$D_J^{-1}S_1\stackrel{{\sf P}}{\to}0$ при $J\hm\rightarrow\infty$.


Теорема доказана.



{\small\frenchspacing
 {%\baselineskip=10.8pt
 \addcontentsline{toc}{section}{References}
 \begin{thebibliography}{99}

\bibitem{HL10}
\Au{Huang H.-C., Lee~T.\,C.\,M.} 
Stabilized thresholding with generalized sure for image denoising~// 
IEEE 17th  Conference (International) on Image Processing
Proceedings.~--- IEEE, 2010. P.~1881--1884.

\bibitem{SH18}
\Au{Shestakov O.\,V.} 
Nonlinear regularization of inverse problems for linear homogeneous transforms 
by the stabilized hard thresholding~// J.~Math. Sci., 2018. Vol.~234. No.\,6. P.~780--785.

\bibitem{KS11-1}
\Au{Кудрявцев А.\,А., Шестаков~О.\,В.} 
Асимптотика оценки риска при вейг\-лет-вейв\-лет разложении наблюдаемого сигнала~// 
T-Comm~--- телекоммуникации и~транспорт, 2011. №\,2. С.~54--57.

\bibitem{KS11-2}
\Au{Кудрявцев А.\,А., Шестаков~О.\,В.} 
Асимптотическое распределение оценки риска пороговой обработки 
вейг\-лет-ко\-эф\-фи\-ци\-ен\-тов сигнала при неизвестном уровне шума~// 
T-Comm~--- телекоммуникации и~транспорт, 2011. №\,5. С.~24--30.

\bibitem{AS98}
\Au{Abramovich F., Silverman~B.\,W.} 
Wavelet decomposition approaches to statistical inverse problems~// 
Biometrika, 1998. Vol.~85. No.\,1. P. 115--129.

\bibitem{Mal99}
\Au{Mallat S.} A~Wavelet tour of signal processing.~--- 
New York, NY, USA: Academic Press, 1999. 857~p.

\bibitem{L97}
\Au{Lee N.} Wavelet-vaguelette decompositions and homogenous equations.~--- 
West Lafayette, IN, USA: Purdue University, 1997.  PhD Thesis. 103~p.

\bibitem{B96}
\Au{Breiman L.} Heuristics of instability and stabilization in model selection~// 
Ann. Stat., 1996. Vol.~24. No.\,6. P.~2350--2383.

\bibitem{J01}
\Au{Jansen M.} Noise reduction by wavelet thresholding.~--- 
Lecture notes in statistics ser.~--- New York, NY, USA: Springer Verlag,
2001. Vol.~161. 196~p.

\bibitem{SH12}
\Au{Шестаков О.\,В.} О~скорости сходимости оценки риска пороговой обработки 
вейв\-лет-ко\-эф\-фи\-ци\-ен\-тов к~нормальному закону при использовании 
робастных оценок дисперсии~// Информатика и~её применения, 2012. Т.~6. Вып.~2. 
С.~122--128.

\bibitem{S80}
\Au{Serfling R.} Approximation theorems of mathematical statistics.~--- 
New York, NY, USA: John Wiley \& Sons, 1980. 371~p.

\bibitem{SM09}
\Au{Serfling R., Mazumder~S.} 
Exponential probability inequality and convergence results for the median 
absolute deviation and its modifications~// Stat. Probabil. Lett., 2009. 
Vol.~79. No.\,16. P.~1767--1773.
 \end{thebibliography}

 }
 }

\end{multicols}

\vspace*{-3pt}

\hfill{\small\textit{Поступила в~редакцию 14.12.18}}

\vspace*{8pt}

%\pagebreak

%\newpage

%\vspace*{-28pt}

\hrule

\vspace*{2pt}

\hrule

%\vspace*{-2pt}

\def\tit{INVERSION OF~HOMOGENEOUS OPERATORS USING~STABILIZED HARD THRESHOLDING 
WITH~UNKNOWN NOISE VARIANCE}

\def\titkol{Inversion of~homogeneous operators using~stabilized hard thresholding 
with~unknown noise variance}

\def\aut{O.\,V.~Shestakov}

\def\autkol{O.\,V.~Shestakov}

\titel{\tit}{\aut}{\autkol}{\titkol}

\vspace*{-11pt}


\noindent
Department of Mathematical Statistics, Faculty of Computational Mathematics and Cybernetics, M.V. Lomonosov Moscow State University, 1-52 Leninskiye Gory, GSP-1, Moscow 119991, Russian Federation
Institute of Informatics Problems, Federal Research Center 
``Computer Science and Control'' of the Russian Academy of Sciences, 44-2~Vavilov Str., 
Moscow 119333, Russian Federation

\def\leftfootline{\small{\textbf{\thepage}
\hfill INFORMATIKA I EE PRIMENENIYA~--- INFORMATICS AND
APPLICATIONS\ \ \ 2019\ \ \ volume~13\ \ \ issue\ 1}
}%
 \def\rightfootline{\small{INFORMATIKA I EE PRIMENENIYA~---
INFORMATICS AND APPLICATIONS\ \ \ 2019\ \ \ volume~13\ \ \ issue\ 1
\hfill \textbf{\thepage}}}

\vspace*{6pt}



\Abste{When inverting linear homogeneous operators, it is necessary to use 
regularization methods, since observed data are usually noisy. For noise suppression, 
threshold processing of  wavelet coefficients of the observed signal function 
is often used. Threshold processing has become a~popular noise suppression tool 
due to its simplicity, computational efficiency, and ability to adapt to functions 
that have different degrees of regularity at different domains. The paper 
discusses the recently proposed stabilized hard thresholding method that eliminates 
the main
drawbacks of soft and hard thresholding methods and studies statistical 
properties of this method. In the data model\linebreak\vspace*{-12pt}}

\Abstend{with an additive Gaussian noise with 
unknown variance, an unbiased estimate of the mean square risk is analyzed and it 
is shown that under certain conditions, this estimate is asymptotically normal and 
the variance of the limit distribution depends on the type of estimate of noise variance.}


\KWE{wavelets; threshold processing; unbiased risk estimate; asymptotic normality;
strong consistency}




\DOI{10.14357/19922264190107}

%\vspace*{-14pt}

\Ack
\noindent
This research was partly supported by the Russian  
Foundation for Basic Research (project No.\,19-07-00352).




%\vspace*{6pt}

  \begin{multicols}{2}

\renewcommand{\bibname}{\protect\rmfamily References}
%\renewcommand{\bibname}{\large\protect\rm References}

{\small\frenchspacing
 {%\baselineskip=10.8pt
 \addcontentsline{toc}{section}{References}
 \begin{thebibliography}{99}
\bibitem{1-sh-1}
\Aue{Huang, H.-C., and T.\,C.\,M.~Lee.} 2010. 
Stabilized thresholding with generalized sure for image denoising. 
\textit{IEEE 17th Conference (International) on Image Processing}. IEEE. 1881--1884.

 

\bibitem{2-sh-1}
\Aue{Shestakov, O.\,V.} 2018. 
Nonlinear regularization of inverse problems for linear homogeneous transforms 
by the stabilized hard thresholding. 
\textit{J.~Math. Sci.} 234(6):780--785.

\bibitem{3-sh-1}
\Aue{Kudryavtsev, A.\,A., and O.\,V.~Shestakov.} 2011. Аsimptotika otsenki riska pri 
veyglet-veyvlet razlozhenii nablyuda\-emo\-go signala [The average risk assessment 
of the wavelet decomposition of the signal].
\textit{T-Comm~--- Telecommunications and Their Application in
Transport Industry} 2:54--57.

\bibitem{4-sh-1}
\Aue{Kudryavtsev, A.\,A., and O.\,V.~Shestakov.} 2011. Аsimptoticheskoe raspredelenie 
otsenki riska porogovoy ob\-ra\-bot\-ki veyglet-koeffitsientov signala pri 
neizvestnom urovne shuma [Asymptotic distribution of the risk estimate of 
the signal vaguelette coefficients thresholding at the unknown noise level]. 
\textit{T-Comm~--- Telecommunications and Their Application in
Transport Industry} 5:24--30.

\bibitem{5-sh-1}
\Aue{Abramovich, F., and B.\,W.~Silverman.} 1998. Wavelet 
decomposition approaches to statistical inverse problems. 
\textit{Biometrika} 85(1):115--129.

\bibitem{6-sh-1}
\Aue{Mallat, S.} 1999. \textit{A~wavelet tour of signal processing.} New York, NY: 
Academic Press. 857 p.

\bibitem{7-sh-1}
\Aue{Lee, N.} 1997. Wavelet-vaguelette decompositions and homogenous equations. 
 West Lafayette, IN: Purdue University. PhD Thesis. 103~p.

\bibitem{8-sh-1}
\Aue{Breiman, L.} 1996. 
Heuristics of instability and stabilization in model selection. 
\textit{Ann. Stat.} 24(6):2350--2383.

\bibitem{9-sh-1}
\Aue{Jansen, M.} 2001. \textit{Noise reduction by wavelet thresholding.} 
Lecture notes in statistics ser.
New York, NY: Springer Verlag.  Vol.~161. 196~p.

\bibitem{10-sh-1}
\Aue{Shestakov, O.\,V.} 2012. O~skorosti skhodimosti otsenki riska porogovoy 
obrabotki veyvlet-koeffitsientov k~nor\-mal'\-no\-mu zakonu pri ispol'zovanii robastnykh 
otsenok dispersii [On the rate of convergence to the normal law of risk estimate for 
wavelet coefficients thresholding when using robust variance estimates]. 
\textit{Informatika i~ee Primeneniya~--- Inform. Appl.}  6(2):122--128.

\bibitem{11-sh-1}
\Aue{Serfling, R.} 1980. \textit{Approximation theorems of mathematical statistics}.
New York, NY: John Wiley \& Sons. 371~p.

\bibitem{12-sh-1}
\Aue{Serfling, R., and S.~Mazumder.} 2009. Exponential probability inequality 
and convergence results for the median absolute deviation and its modifications. 
\textit{Stat. Probabil. Lett.} 79(16):1767--1773.
\end{thebibliography}

 }
 }

\end{multicols}

\vspace*{-6pt}

\hfill{\small\textit{Received December 14, 2018}}

%\pagebreak

%\vspace*{-18pt}  

\Contrl

\noindent
\textbf{Shestakov Oleg V.} (b.\ 1976)~--- 
Doctor of Science in physics and mathematics, professor, Department of 
Mathematical Statistics, Faculty of Computational Mathematics and Cybernetics, 
M.\,V.~Lomonosov Moscow State University, 1-52~Leninskiye Gory, GSP-1, Moscow 119991, 
Russian Federation; senior scientist, Institute of Informatics Problems, 
Federal Research Center ``Computer Science and Control'' 
of the Russian Academy of Sciences, 44-2~Vavilov Str., Moscow 119333, 
Russian Federation; \mbox{oshestakov@cs.msu.su}
\label{end\stat}

\renewcommand{\bibname}{\protect\rm Литература} 
        %8

\def\stat{konovalov}

\def\tit{ОБ АДАПТИВНЫХ СТРАТЕГИЯХ И~УСЛОВИЯХ~ИХ~СУЩЕСТВОВАНИЯ$^*$}

\def\titkol{Об адаптивных стратегиях и~условиях их 
существования}

\def\autkol{М.\,Г.~Коновалов}

\def\aut{М.\,Г.~Коновалов$^1$}

\titel{\tit}{\aut}{\autkol}{\titkol}

{\renewcommand{\thefootnote}{\fnsymbol{footnote}}\footnotetext[1]
{Работа выполнена при поддержке РФФИ, грант № 11-07-00112.}}

\renewcommand{\thefootnote}{\arabic{footnote}}
\footnotetext[1]{Институт проблем информатики Российской академии наук, mkonovalov@ipiran.ru}



\Abst{Рассматривается задача оптимального управления в отсутствие априорной 
информации об управляемом объекте. Решением задачи является построение адаптивных 
стратегий на основе наблюдений, доступных в процессе управления. Изучаются 
некоторые условия адаптивной управляемости объекта. В~качестве математической 
модели используются управляемые случайные последовательности.}

\KW{управляемые случайные последовательности; адаптивные стратегии; условия 
существования}

\vskip 14pt plus 9pt minus 6pt

      \thispagestyle{headings}

      \begin{multicols}{2}

            \label{st\stat}


\section{Введение}

  Тема статьи относится к области адаптивных методов обработки информации с целью 
принятия оптимальных решений. Потребность в адаптивном\linebreak
подходе возникает в задачах 
с большой информационной неопределенностью, что наиболее характерно для 
телекоммуникационных систем, автоматизированных производственных процессов, 
робототех\-ни\-ки и других сфер, неразрывно связанных с компьютерной обработкой 
информации. Понятие неопределенности многозначно и связано с отсутствием априорных 
сведений, недетерминированностью, а также с неполнотой наблюдений. 
К~перечисленным факторам в нарастающей степени добавляется <<избыточность>> 
информации, которая порождается чрезмерно прогрессирующими объемами 
передаваемой и хранимой информации и обусловлена экспоненциальным ростом 
пропускной способности телекоммуникационных сетей, а также емкостей носителей 
информации.
  
  Идея адаптации (приспособления, самоорганизации), заимствованная из 
биологического мира, начала активно эксплуатироваться в науке примерно с середины 
прошлого века. Кратко, она заключается в том, чтобы, целенаправленно взаимодействуя с 
окружающей средой, отбирать и использовать поступающую информацию, необходимую 
для принятия оптимальных решений с точки зрения поставленной цели.
  
  Данная статья посвящена теоретическим аспектам адаптации. В~качестве исходного 
пред\-став\-ле\-ния использована схема, которая опирается на пред\-став\-ление о паре 
  <<объект--субъект>>, взаимодействующей в дискретном времени путем 
попеременного обмена сигналами. При этом субъект воздействует на объект с помощью 
управлений, получая в ответ сигналы, называемые наблюдениями. Действия субъекта 
преследуют цель, выраженную в наличии определенных свойств у траектории 
наблюдений.
  
  Основная отличительная особенность заключается в предположении, что действия 
субъекта происходят при недостаточной информации об объекте. В~качестве 
математической модели объекта взята конструкция управляемой случайной 
последовательности. В~терминах этого аппарата легко очерчиваются четыре аспекта 
информационной неопределенности:
  \begin{enumerate}[(1)]
\item недетерминированность понимается как стохастичность;
\item недостаток информации об объекте трактуется как неполное знание вероятностного 
распределения, задающего процесс;
  \item неполнота наблюдений означает, что состояния процесса наблюдаются лишь 
частично;
  \item недостаток знаний выражается в неумении \mbox{найти} или рассчитать ту или иную 
характеристику, связанную со случайной последовательностью, даже при наличии 
априорной информации о распределении процесса и полной его наблюдаемости.
  \end{enumerate}
  
  Субъект ассоциируется с алгоритмом, согласно которому выбираются управления, 
регулирующие траекторию случайной последовательности. Такой алгоритм принято 
называть стратегией управ\-ле\-ния. Задача заключается в том, чтобы выбрать стратегию, 
достигающую цели в ситуации, когда информация субъекта об объекте ограничена. 
По-дру\-го\-му можно сказать, что речь идет о построении стратегии, достигающей цели (в 
данном случае~--- максимизации предельного среднего дохода) для любого процесса из 
некоторого заданного класса объектов. Такие стратегии называют адаптивными по 
отношению к заданному классу объектов~[1].
  
  В разд.~2 даются формальные определения объекта, цели и адаптивной стратегии 
управления.
  
  В разд.~3 анализируются условия существования адаптивной стратегии. В~качестве 
необходимых условий обсуждаются два требования, которые, как представляется, должны 
выполняться из интуитивных соображений.
  
  Первое из необходимых условий связано с принципиальной особенностью адаптивных 
стратегий, которые, прежде чем выйти на <<оптимальный режим>>, должны затратить 
некоторое время на <<обуче\-ние>>. (На самом деле в рассматриваемой постановке процесс 
обучения для адаптивных стратегий длится даже неограниченно долго.) Естественно 
предположить, что подобные стратегии могут реализоваться, только если в процессе 
обучения не будут совершены <<непоправимые ошибки>>. Это соображение 
раскрывается на примерах и получает формальное описание.
  
  Второе необходимое условие является менее очевидным. Оно связано с гипотезой о 
том, что адаптивная стратегия управления классом случайных последовательностей 
существует лишь тогда, когда для данного класса возможно построение так называемой 
адаптивной стратегии перебора. Это выражается в том, что существует и заранее известно 
некоторое счетное множество вариантов поведения, среди которого для данного класса 
обязательно найдется оптимальный или близкий к нему вариант. Данное соображение 
также иллюстрировано примерами и приведена теорема о критерии существования 
адаптивной стратегии для определенного класса объектов.
  
  Подход, использованный в статье, а также полученные результаты являются 
продолжением направления, представленного в работе~[2].
  
\section{Постановка задачи адаптивного управления}
  
  Пусть  время $t$ пробегает значения 0, 1, \ldots\ и пусть заданы измеримые 
пространства $(X,\mathbf{X})$, $(Y,\mathbf{Y})$, $(Z,\mathbf{Z})$ (соответственно 
пространства \textit{состояний}, \textit{управлений} и \textit{наблюдений}).
  
  Общая траектория процесса упорядочена в виде последовательности $x_0, y_1, 
z_1,x_1,\ldots$\linebreak $\ldots , x_{t-1},y_t,z_t,x_t,\ldots$ Предыстория процесса до момента~$t$ 
включительно обозначается как

\noindent
  \begin{gather*}
 \! x^t=x_0^t=(x_0,\ldots , x_{t-1});\ \ \ y^t=y_1^t=(y_1, \ldots , y_{t-1});\\
  z^t=z_1^t=(z_1,  \ldots , z_{t-1})\,.
  \end{gather*}
  
  Траектории процесса определяются последовательностями условных вероятностных 
распределений~$\mu$, $\nu$ и~$\sigma$.
  
  Последовательность $\mu\hm=(\mu_0,\mu_1,\ldots ,\mu_t, \ldots)$ задает механизм 
смены состояний. В~этой последовательности $\mu_0$~--- вероятностное распределение 
на $(X,\mathbf{X})$; $\mu_t=\mu_t(A\vert x^{t-1},y^t)$, $t\hm>0$~---  условная 
(переходная) вероятность, которая при любых наборах $(x^{t-1},y^t)$ является 
вероятностной мерой на $(X,\mathbf{X})$ и при любом $A\hm\in X$ является измеримой 
функцией относительно $x^{t-1},y^t$.
  
  Последовательность $\nu\hm=(\nu_1, \ldots , \nu_t, \ldots)$ задает механизм появления 
наблюдений. В~этой последовательности каждый элемент $\nu_t\hm=\nu_t(C\vert x^{t-1}, 
y^t)$, $t\hm>0$, представляет собой условное распределение, которое при любом условии 
является вероятностной мерой на $(Z,\mathbf{Z})$ и для любого $C\hm\in Z$ является 
измеримой функцией относительно переменных, стоящих в условии. Пара $o\hm= 
(\mu,\nu)$ называется объектом.
  
  Последовательность $\sigma\hm= (\sigma_1, \ldots , \sigma_t. \ldots)$ называется 
(допустимой) \textit{стратегией} и определяет выбор управлений. В~этой 
последовательности:
%\smallskip
   $\sigma_1\hm=\sigma_1(\cdot)$~--- вероятностная мера на $(Y,\mathbf{Y})$; 
      $\sigma_{t+1}\hm=\sigma_{t+1}(B\vert y^t,z^t)$, $t\hm>0$,~--- условная вероятность, 
которая при любых $y^t,z^t$ является вероятностной мерой на $(Y,\mathbf{Y})$ и при 
любом $B\hm\in Y$ является измеримой функцией относительно $y^t,z^t$. Элементы 
последовательности~$\sigma$ называются (допустимыми) \textit{правилами}.

%\smallskip
  
  Введем обозначение для прямых произведений множеств:
  $$
  \Omega_0=X\,;\enskip \Omega_t=X^{t+1}\times Y^t\times Z^t\,,\enskip t>0\,,
  $$
а также для наименьших $\sigma$-ал\-гебр, порожденных соответствующими 
$\sigma$-ал\-геб\-рами:
$$
\mathbf{F}_0=\mathbf{X}\,;\enskip \mathbf{F}_t=\mathbf{X}\otimes \mathbf{Y}\otimes 
\mathbf{Z}\otimes \mathbf{X}\otimes \cdots \otimes \mathbf{Y}\otimes \mathbf{Z}\otimes 
\mathbf{X}
$$
($\mathbf{X}$ повторяется $t+1$ раз, $\mathbf{Y}$ и $\mathbf{Z}$~--- $t$ раз, $t\hm>0$).
  
  Положим
  
  \vspace*{3pt}
  
  \noindent
  $$
  \Omega =\prod\limits_{t\geq 0}\Omega_t\,;\enskip 
\mathbf{F}=\mathop{\otimes}\limits_{t\geq0}\mathbf{F}_t\,.
  $$ 
  
  Согласно общей теории~\cite{3-kon} последовательности $o\hm=(\mu,\nu)$ и~$\sigma$ 
порождают на пространстве $(\Omega, \mathbf{F})$ вероятностную меру $\mathbf{P}\hm= 
\mathbf{P}_{o,\sigma}\hm=\mathbf{P}_{\mu,\nu,\sigma}$, которая согласована с 
элементами этих последовательностей следующим образом. Случайные 
последова\-тель\-ности

\vspace*{-3pt}

\noindent
  \begin{gather*}
  x_t=x_t(\omega)\,;\enskip  
  y_{t+1}=y_{t+1}(\omega)\,;\\
  z_{t+1}= z_{t+1}(\omega)\,,\enskip  \omega\in \Omega\,,\  t\geq 0\,,
  \end{gather*}
удовлетворяют соотношениям:

\pagebreak

\noindent
$$
\mathbf{P}(x_0(\omega)\in A_0)=\int\limits_{A_0} \mu_0(dx_0)\,;
$$

\vspace*{-12pt}

\noindent
\begin{multline*}
\mathbf{P}\left(x_0(\omega)\in A_0\,,\  y_1(\omega)\in B_1\,,\ 
z_1(\omega)\in C_1, \ldots \right.\\[1pt]
\left.{}\ldots\,,
y_t(\omega)\in B_t\,,\  z_t(\omega)\in C_t\,,\  x_t(\omega)\in A_t\right)={}\\[1pt]
{}=\int\limits_{A_0}\mu_0(dx_0)\int\limits_{B_1}\sigma_1(dy_1)\int\limits_{C_1}\nu_1(dz_1
\vert x_0, y_1)\cdots{}\\[1pt]
{}\cdots
\int\limits_{B_t}\sigma_t\left(dy_t\vert y^{t-1},z^{t-1}\right) 
\int\limits_{C_t} \nu_t\left( dz_t\vert x^{t-
1},y^t\right) \times{}\\[1pt]
{}\times
\int\limits_{A_t} \mu_t\left( dx_t\vert x^{t-1},y^t\right)
\end{multline*}
для любых $A_t\in X$, $B_{t+1}\hm\in Y$, $C_{t+1}\hm\in Z$, $t\hm\geq 0$.
  
  По определению стратегии, ее правила зависят от предыдущих управлений и 
наблюдений, но не от предыдущих состояний. Это соответствует предположению о том, 
что состояния объекта не наблюдаемы в ходе процесса управления. В~частных случаях 
объект $o\hm=(\mu,\nu)$ может, конечно, описывать полностью наблюдаемый процесс. 
Например, если все множества $X_t$ содержат один и тот же единственный элемент. 
Другой простой пример~--- когда наблюдения тождественны состояниям. Однако на 
самом деле, как показывает лемма~1, с формальной точки зрения рассмотрение объекта с 
<<ненаблюдаемой>> компонентой всегда можно заменить изучением полностью 
наблюдаемого процесса.
  
  \medskip
  
  \noindent
  \textbf{Лемма 1.} \textit{Для любого объекта $o\hm=(\mu,\nu)$ условная вероятность 
$\mathbf{P}\left(dz_t\vert y^t,z^{t-1}\right)$ не зависит от стратегии~$\sigma$ при любых 
$t\hm>0$.}
  
  \medskip
  
  \noindent
  Д\,о\,к\,а\,з\,а\,т\,е\,л\,ь\,с\,т\,в\,о\,.\ Согласно отмеченной выше согласованности 
условных распределений $\mu,\nu,o$ и порождаемой ими меры~\textbf{P} имеем 
соотношения:
  \begin{multline*}
  I_1=\mathbf{P}\left(
  y_1(\omega)\in B_1,\ z_1(\omega)\in C_1\right) ={}\\[1pt]
  {}=
  \mathbf{P}\left( x_0(\omega)\in X_0\,,\ y_1(\omega)\in B_1\,,\ z_1(\omega)\in 
C_1\right)={}\\[1pt]
  {}=\int\limits_{X_0} \int\limits_{B_1} \int\limits_{C_1} \mu_0\left(dx_0\right) 
\sigma_1\left(dy_1\right) \nu_1\left(dz_1\vert x_0,y_1\right)={}\\[1pt]
  {}= \int\limits_{B_1}\int\limits_{C_1}\sigma_1\left(dy_1\right) \int\limits_{X_0}\mu_0\left( 
dx_0\right) \nu_1\left( dz_1\vert x_0,y_1\right)\,,
  \end{multline*}
справедливые при любых $B_1\hm\in Y$ и $C_1\in Z$. Кроме того, по определению 
условной вероятности
$$
I_1=\int\limits_{B_1}\int\limits_{C_1}\sigma_1\left(dy_1\right) \mathbf{P}\left(dz_1\vert 
y_1\right)\,.
$$
  
  Сравнивая оба выражения для~$I_1$, получаем, что
  $$
  \mathbf{P}\left( dz_1\vert 
y_1\right)=\int\limits_{X_0}\mu_0\left(dx_0\right)\nu_1\left(dz_1\vert x_0, y_1\right)\,,
  $$
т.\,е.\ утверждение леммы справедливо для $t\hm=1$. Пусть оно верно для $n\hm=1, 2, 
\ldots , t\hm-1$. Для любых $B_1\hm\in Y$, $C_1\hm\in Z$, \ldots , $B_{t-1}\hm\in Y$, 
$C_t\hm\in Z$ имеем:

\noindent
\begin{multline*}
I_t=\mathbf{P}\left( y_1(\omega)\in B_1\,,\ z_1(\omega)\in C_1, \ldots{}\right.\\[1pt]
\left.{}\ldots , y_t(\omega)\in B_t\,,\ 
z_t(\omega) \in C_t\right)={}\\[1pt]
{}=
\mathbf{P}\left( x_0(\omega)\in X\,,\ y_1(\omega)\in B_1\,,\ z_1(\omega)\in C_1\,, \ldots\right.\\[1pt]
\left.{}\ldots , x_{t-
1}(\omega)\in X\,,\ y_t(\omega)\in B_t\,,\ z_t(\omega)\in C_t\right)={}\\[1pt]
{}=
\int\limits_X \int\limits_{B_1} \int\limits_{C_1}\ldots \\[1pt]
\ldots\int\limits_X \int\limits_{B_t} 
\int\limits_{C_t} \mu_0\left( dx_0\right) \sigma_1\left( dy_1\right) \nu_1\left( dz_1\vert 
x_0,y_1\right)\cdots{}\\[1pt]
\cdots \mu_{t-1}\left( dx_{t-1}\vert x^{t-2} y^{t-1}\right) \sigma_t \left( dy_t\vert y^{t-
1},z^{t-1}\right)\times{}\\[1pt]
{}\times \nu_t\left( dz_t\vert x^{t-1},y^t\right)={}\\[1pt]
{}=\int\limits_{B_1} \sigma_1\left( dy_1\right) \int\limits_{C_1} \int\limits_{B_2} 
\sigma_2\left( dy_2\vert z_1\right)\cdots\\[1pt]
\cdots \int\limits_{C_{t-1}}\int\limits_{B_t} \sigma_t \left( 
dy_t\vert y^{t-1},z^{t-1}\right)\times{}\\[1pt]
{}\times \int\limits_X \mu_0\left(dx_0\right) \nu_1\left( dz_1\vert 
x_o,y_1\right)\cdots{}\\[1pt]
{}\cdots \int\limits_{X_{t-1}}\mu_{t-1}\left( dx_{t-1}\vert x^{t-2} y^{t-1}\right) \nu_t \left( 
dz_t\vert x^{t-1}, y^t\right)={}\\[1pt]
{}=\int\limits_{B_1} \sigma_1\left( dy_1\right) \int\limits_{C_1} 
\int\limits_{B_2}\sigma_2\left( dy_2\vert z_1\right)\cdots\\[1pt]
\cdots \int\limits_{C_{t-1}} 
\int\limits_{B_t} \sigma_t\left( dy_t\vert y^{t-1},z^{t-1}\right) \int\limits_{C_t} 
\mathbf{P}\left( dz_1\vert y_1\right)\ldots{}\\[1pt]
{}\cdots \mathbf{P}\left( dz_{t-1}\vert y^{t-1},z^{t-2}\right) \mathbf{P}\left( dz_t\vert y^t, 
z^{t-1}\right)\,.
\end{multline*}
Отсюда получаем, что

\noindent
  \begin{multline*}
\hspace*{-6.95218pt}\mathbf{P}\left( dz_1\vert y_1\right)\cdots \mathbf{P}\left( dz_{t-1}\vert y^{t-1},z^{t-
2}\right) \mathbf{P}\left( dz_t\vert y^t,z^{t-1}\right)={}\\[1pt]
  {}=\int\limits_X \mu_0\left( dx_0\right) \nu_1\left( dz_1\vert x_o,y_1\right)\cdots \\[1pt]
  \cdots
\int\limits_X \mu_{t-1}\left( dx_{t-1}\vert x^{t-2}y^{t-1}\right) \nu_t\left( dz_t\vert x^{t-
1},y^t\right)\,.
  \end{multline*}
  
  Следовательно, по предположению индукции $\mathbf{P}\left( dz_t\vert y^t,z^{t-
1}\right)$ не зависит от~$\sigma$.
  
  Таким образом, не уменьшая общности, можно ограничиться (что и будет сделано в 
оставшейся части текста) рассмотрением полностью наблюда-\linebreak\vspace*{-12pt}

\pagebreak

\noindent
емых объектов $o\hm=\mu$, 
управляемых (допустимыми) стратегиями~$\sigma$ c правилами вида
  $$
  \sigma_1=\sigma_1\left(\cdot\right)\,;\enskip \sigma_{t+1}=\sigma_{t+1}\left( \cdot \vert 
y^t,x^t\right)\,,\enskip t>0\,.
  $$
(Множество всех таких стратегий при заданных пространствах состояний и управлений 
далее обозначается через~$\Sigma$.) В~этом случае вероятностная мера 
$\mathbf{P}\hm=\mathbf{P}_{\mu,\sigma}$ определена на пространстве $(\Omega, 
\mathbf{F})$, в котором $\Omega\hm=\prod\limits_{t\geq0} X^{t+1}\times Y^t$, 
$\mathbf{F}\mathop{\otimes}\limits_{t\geq0} \mathbf{F}_t$, где $\mathbf{F}_0\hm=\mathbf{X}$; 
$\mathbf{F}_t=\mathbf{X}\otimes \mathbf{Y}\otimes \mathbf{X}\otimes \cdots \otimes 
\mathbf{Y}\otimes \mathbf{X}$ и согласована с последовательностями~$\mu$ и~$\sigma$. 
Через $\mathbf{F}_t$ обозначена $\sigma$-ал\-геб\-ра, порожденная предысторией 
$(x^t,y^t)$ до момента~$t$ включительно.
  
  В то же время необходимо заметить, что предположение о наличии 
<<двухступенчатой>> структуры у объектов (со\-сто\-яние--наблю\-де\-ние) может 
принести пользу при их изучении. Так происходит, например, в теории частично 
наблюдаемых управляемых марковских процессов.
  
  Предположим далее, что на наблюдаемой части траектории процесса задан 
одношаговый доход (в момент~$t$), и будем считать, что этот доход имеет вид 
$g_t\hm=g(x_t)$, где $g:\ X\rightarrow (0,\,1)\subset \mathbb{R}$~--- измеримая числовая 
функция со значениями из интервала (0,\,1).
  
  Обозначим через $v_{t,s}\hm=s^{-1}\sum\limits_{n=1}^s g_{t+n}$ среднее 
арифметическое доходов на промежутке от $t+1$ до $t\hm+s$ ($t\hm\geq0$, $s\hm\geq 1$).
  
  Если объект~$\mu$ управляется согласно стратегии~$\sigma$, то число
  $$
  w_t(\mu,\sigma) =\sup \left\{ c:\ \mathbf{P}_{\mu,\sigma} \left( 
\lim\limits_{\overline{s\rightarrow\infty}} v_{t,s}>c\right) =1\right\}
  $$
характеризует получаемый при этом гарантированный предельный средний доход 
начиная с момента $t=1$. Поскольку $\lim\limits_{\overline{s\rightarrow\infty}} v_{t,s}$ не 
зависит от~$t$, то $w_0(\mu,\sigma)\hm=w_1(\mu,\sigma)\hm=w_2(\mu,\sigma)\hm=\cdots$. 
Величина $w(\mu,\sigma)\hm=w_0(\mu,\sigma)$ играет в дальнейшем роль целевой 
функции и называется просто \textit{доходом} (при управлении объектом~$\mu$ с 
помощью стратегии~$\sigma$).
  
  Из определения дохода следует, что для любого $t>0$ выполняется условие
  $$
  \mathbf{P}_{\mu,\sigma}\left( \lim\limits_{\overline{s\rightarrow\infty}} v_{t,s}\geq 
w(\mu,\sigma)\vert \mathbf{F}_{t-1}\right)=1
  $$
почти наверное.
  Столь общее определение дохода, без предположений об эргодичности, оказывается 
полезным в теоретических рассмотрениях, однако на практике все же среднее 
арифметическое ведет себя более или менее регулярным образом. Поэтому введем 
следующее определение.
{ %\looseness=1

}
  
  Стратегия~$\sigma$ называется \textit{эргодической} по отношению к классу~$M$, 
если для любого объекта $\mu\hm\in M$ и любого $\varepsilon\hm>0$ выполняется 
условие $\sum\limits_{s=1}^\infty a_s\hm<\infty$, где $a_s\hm= 
a_s(\mu,\sigma,\varepsilon)\hm=\sup\limits_{t\geq0} \mathbf{P}_{\mu,\sigma}\left( \left\vert 
v_{t,s}-w(\mu,\sigma)\right\vert >\varepsilon\vert \mathbf{F}_t\right)$. Обозначим еще
  $$
  W=W(\mu) =\sup\limits_\sigma w(\mu,\sigma)\,,
  $$
где точная верхняя грань берется по всем допустимым стратегиям. Стратегия~$\sigma$ 
называется $\varepsilon$-\textit{оп\-ти\-маль\-ной}, если выполняется неравенство
$$
w(\mu,\sigma)\geq W-\varepsilon\,,\enskip \varepsilon\geq 0\,.
$$
  
  Далее объекты будут объединяться в множества объектов (классы объектов). При этом 
без дополнительных оговорок всюду предполагается, что
  \begin{itemize}
  \item все объекты из класса имеют одинаковые пространства состояний, управлений (и 
наблюдений);
  \item в качестве множества допустимых стратегий берется определенное выше 
множество~$\Sigma$;
  \item функция одношаговых доходов~$g$ одна и та же для всех объектов.
  \end{itemize}
  
  Пусть $M$~--- класс объектов. Стратегия~$\sigma$ является равномерно 
  $\varepsilon$-оп\-ти\-маль\-ной относительно этого класса, если последнее неравенство 
выполняется для всех $\mu\hm\in M$. Такую стратегию будем называть также 
  $\varepsilon$-\textit{адап\-тив\-ной} по отношению к классу~$M$. Класс объектов, для 
которого существует $\varepsilon$-адап\-тив\-ная стратегия, называется 
  $\varepsilon$-\textit{адап\-тив\-но управ\-ля\-емым}. (Если $\varepsilon\hm=0$, то 
приставка <<$\varepsilon$->> в этих определениях опускается.)
  
  Основная задача адаптивного управления заключается в построении адаптивных 
стратегий для различных классов объектов. 

К~настоящему вре\-ме\-ни получено много 
решений для многочисленных вариантов этой задачи. Подобные результаты являются 
фактически достаточными условиями адаптивной управ\-ля\-емости. Ниже, однако, будет 
уделено внимание также необходимым условиям существования адаптивных стратегий. 
Подчеркнем, что рассматриваемая постановка задачи предполагает, по сути, наличие 
лишь минимальной априорной информации об объекте управления~--- необходимо знать 
множество управлений~$Y$.

\section{Некоторые условия адаптивной управляемости}

  Пусть $\mu\in M$~--- фиксированный объект, а $\sigma\hm\in \Sigma$~--- 
фиксированная стратегия из некоторой среды. Набор, состоящий из первых $t$ правил 
стратегии~$\sigma$, будем обозначать через $\sigma^t\hm=(\sigma_1, \ldots , \sigma_t)$. 
Таким образом, $\sigma\hm=(\sigma^t, \sigma_{t+1},\sigma_{t+2}, \ldots)$. Положим
  $$
  w_t^*(\mu,\sigma) =w_t^*(\mu,\sigma^t)=\sup\limits_{\sigma_{t+1},\sigma_{t+2}, \ldots} 
w_t(\mu,\sigma)\,,
  $$
где верхняя грань берется по всем допустимым правилам начиная с момента $t\hm+1$. В 
этих обозначениях $w_0^*(\mu,\sigma) \hm=W(\mu)$. Ясно, что $W(\mu)\hm\geq 
w_1^*(\mu,\sigma)\hm\geq w_2^*(\mu,\sigma)\geq \cdots$
  
  Стратегию~$\sigma$ назовем $\varepsilon$-\textit{по\-вреж\-да\-ющей} для 
объекта~$\mu$, если
  $$
  \inf\left\{ t:\ w_t^*(\mu,\sigma)<W(\mu)-\varepsilon\right\} <\infty\,,\enskip \varepsilon>0\,.
  $$
  
  Пример~1 показывает, что существуют объекты, для которых каждая стратегия~--- 
$\varepsilon$-по\-вреж\-да\-ющая (с разными значениями~$\varepsilon$).
  
  \medskip
  
  \noindent
  \textbf{Пример~1.} Множество~$X$ состояний объекта~$\mu$ образовано точками с 
неотрицательными целочисленными координатами на плоскости, $X\hm=\{ (i,j), 
i\hm\geq0,\ j\hm\geq0\}$. Множество управлений $Y\hm=\{1;2\}$. Начальное состояние 
$x_0=(0,\,0)$. Детерминированные переходы между состояниями заданы следующим 
образом ($t\hm>0$, $i\hm\geq0$):
  \begin{align*}
  \mu_t\left( x_t=(i+1{,}0)\vert x_{t-1}=(i,0),y_t=1\right)&=1\,;\\
  \mu_t\left( x_t=(i,j+1)\vert x_{t-1}=(i,j),y_t=1\right)&=1\,,\ j>0\,;\\
  \mu_t\left( x_t=(i,j+1)\vert x_{t-1}=(i,j),y_t=2\right)&=1\,, j\geq 0\,.
  \end{align*}
  
  Одношаговые доходы определены как $g(i,0)\hm=0$, $g(i,j)\hm=1-2^{-i}$ для $i\geq 0$, 
$j\hm>0$.
  
  Стратегия, состоящая из бесконечного повторения управления~1, приносит доход~0. 
Стратегия, в которой управление~2 первый раз применяется (детерминировано) в 
момент~$t$, приносит доход $1\hm-2^{t-1}$, что меньше максимально возможного 
на~$2^{t-1}$. Рандомизация правил и их зависимость от предыстории не вносит 
принципиальных изменений~--- каждая стратегия остается 
  $\varepsilon$-по\-вреж\-да\-ющей относительно предельно наибольшего, но 
недостижимого значения~1.
  
  В примере~2 оптимальная стратегия для любого объекта из класса является 
повреждающей для остальных объектов.
  
  \medskip
  
  \noindent
  \textbf{Пример~2.} Пусть $X\hm= \{0, 1, 2, \ldots\}\cup \{a,b\}$; $Y\hm=\{0;\,1\}$; 
$g(a)\hm=1$; $g(b)\hm=g(i)\hm=0$, $i\hm\geq0$. Зададим счетное множество объектов 
$M\hm=\{\mu^{(k)},\ k\hm=0, 1, \ldots\}$. Пусть для всех~$k$:
  \begin{align*}
  \mu^{(k)}(x_0=0)&=1\,;\\
  \mu^{(k)}(x_{t+1}=i+1\vert x_t=i, y_t=0)&=1\,,\enskip i\geq0\,;\\
     \mu^{(k)}(x_{t+1}=a\vert x_t=k,y_t=1)&=1\,;\\
     \mu^{(k)}(x_{t+1}=b\vert x_t=i,y_t=1) &=1\,,\enskip i\not=k\,;\\
     \mu^{(k)}(x_{t+1}=a\vert x_t=a,y_t=j)&={}\\
&\hspace*{-45mm}{}=\mu^{(k)}(x_{t+1}=b\vert 
x_t=b,y_t=j)=1\,,\enskip j=0\vee 1\,.
     \end{align*}
  
  Таким образом, состояния $a$ и $b$~--- погло\-ща\-ющие, причем в состояние~$a$, 
приносящее максимальный доход, объект~$\mu^{(k)}$ может попасть, только если 
применить управление~1, находясь в со\-сто\-янии~$k$. Первые (существенные) правила 
оптимальной стратегии для объекта~$\mu^{(k)}$ требуют применения управления~0 до 
достижения состояния~$k$, а затем применения в этом состоянии управления~1. Однако 
такая стратегия является повреждающей для всех остальных объектов. Следовательно, для 
класса~$M$ не существует равномерно оптимальной стра\-тегии.
{\looseness=1

}
  
  Пусть $M$~--- класс объектов. Обозначим через $\Sigma_\varepsilon(\mu)$ множество 
$\varepsilon$-по\-вреж\-да\-ющих стратегий для объекта~$\mu$, $\mu\hm\in M$. Положим 
$\Sigma_\varepsilon(M)\bigcap\limits_{\mu\in M}\left( \Sigma\backslash 
\Sigma_\varepsilon(\mu)\right)$.
  
  \medskip
  
  \noindent
  \textbf{Лемма~2.} \textit{Для того чтобы существовала $\varepsilon$-адап\-тив\-ная 
стратегия, необходимо, чтобы $\Sigma_\varepsilon(M)\not=\emptyset$.}
  \medskip
  
  \noindent
  Д\,о\,к\,а\,з\,а\,т\,е\,л\,ь\,с\,т\,в\,о\,.\ Если $\Sigma_\varepsilon\not= \emptyset$, то любая 
допустимая стратегия хотя бы для одного из объектов является 
  $\varepsilon$-по\-вреж\-да\-ющей и, следовательно, не является 
  $\varepsilon$-оп\-ти\-маль\-ной, а потому не может быть равномерно 
  $\varepsilon$-оп\-ти\-маль\-ной по отношению к классу~$M$.
  
  В примере~3, несмотря на наличие по\-вреж\-да\-ющих стратегий, адаптивная стратегия 
существует.
  
  \medskip
  
  \noindent
  \textbf{Пример~3.} Пусть $X\hm=Y\hm=\{1, \ldots , K\}$ и пусть задана 
детерминированная функция~$f:\ X\hm\rightarrow X$, которая представляет собой 
циклическую подстановку на множестве~$X$,  т.\,е.\ $f(i)\not= f(j)$, если $i\not= j$; 
$i,j\hm=1, \ldots , K$. Рассмотрим следующий неоднородный во времени 
детерминированный объект. Положим
  \begin{align*}
  \mu_0(x_0=1)&=1\,;\\
  \mu_t(x_t=f(k)\vert x^{t-1},y^t) &= I_{\{y_t=k\}}\,,\ 0<k\,,\ t\leq K\,;\\
  \mu_t(x_t=f(k)\vert x^{t-1},y^t) &=I_{\{y_{K+1}=k}\,,\\
  & \hspace*{10mm}0<k\leq K\,,\enskip t>K
  \end{align*}
($I_A$~--- индикатор события~$A$).
  
  Одношаговые доходы определим как $g(i)\hm=i$, $i\hm\in X$.
  
  Так определенный объект обозначим через~$\mu^f$. Ясно, что для этого объекта 
траектория управ\-ля\-емо\-го процесса, начиная с момента $K+1$, и, следовательно, доход 
зависят исключительно от управ\-ле\-ния, примененного в момент $K+1$. Доход будет 
максимален (и равен~$K$) тогда и только тогда, когда $y_{K+1}\hm= k^\prime \hm= 
k^*(f)\hm=\argmax\limits_{1\leq k\leq K} f(k)$.
  
  Пусть $M=\{\mu^f\}$~--- совокупность всех объектов данного вида (которая содержит 
$K!$ элементов). Очевидно, для класса~$M$ существует равномерно оптимальная 
стратегия, доставляющая доход, равный~$K$. Например, достаточно вначале в моменты 
$t\hm=1, \ldots , K$ по одному разу применить каждое из управлений, а затем в момент 
$K+1$ применить управление~$k^*$, которое будет выявлено путем наблюдения за 
полученными одношаговыми доходами. Таким образом, на первых тактах необходимо совершить 
<<обучение>>~--- выявить управление, приносящее наибольший одношаговый доход. 
В~то же время существуют и повреждающие стратегии. Например, стратегия, в которой 
первые $K$ правил заключаются в применении управления~1. Правило~$\sigma_{K+1}$ 
такой стратегии может быть построено только в виде зависимости от управления~1 и от 
значения $f(1)$, поэтому при любом его определении найдется объект~$\mu^f$, для 
которого в момент $K+1$ будет с положительной вероятностью предписано применение 
неоптимального управления, и, следовательно, доход будет меньше~$K$.
  
  В примере 3 <<обучение>> оказалось возможным только благодаря знанию структуры 
процессов. Если бы заранее не было известно, что необходимо на первых тактах по разу 
<<испробовать>> все управ\-ле\-ния, то легко можно было пропустить период, когда 
возможно обучение, и совершить тем самым <<непоправимую ошибку>>. Следовательно, 
для того чтобы конструктивно построить равномерно оптимальную стратегию, 
необходима дополнительная информация. Это противоречит избранному принципу 
постановки задачи~--- минимальности априорной информации об объекте. 

Введем более 
жесткое определение адаптивной стратегии, которое, в част\-ности, устраняет указанное 
несоответствие.
  
  Пусть $M$~--- некоторый класс объектов. Эргодическая стратегия~$\sigma$ (ее 
определение дано в конце разд.~2) называется \textit{устойчивой} по отношению к 
классу~$M$, если для любого объекта $\mu\hm\in M$ стратегия~$\tilde{\sigma}$, 
полученная из стратегии~$\sigma$ путем произвольной (допустимой) замены конечного 
числа правил, (1)~имеет одинаковый со стратегией доход 
$w(\mu,\sigma)\hm=w(\mu,\tilde{\sigma})$ и (2)~является эргодической по отношению к 
классу~$M$.

%\columnbreak
  
  Адаптивная стратегия для класса~$M$ называется \textit{строго адаптивной}, если она 
устойчивая по отношению к этому классу.
  
  \medskip
  
  \noindent
  \textbf{Пример~4.} Легко показать, что строго адаптивными являются 
многочисленные адаптивные стратегии для класса управляемых конечных связных 
марковских цепей~[1, 2].
  
  Рассмотрим еще один мотив, выдвигаемый в качестве необходимого условия 
адаптивной управ\-ля\-емости.
  
  \medskip
  
  \noindent
  \textbf{Пример~5.} Пусть класс объектов состоит из функций вещественного 
аргумента~$u$ вида $\mu^y\hm=\mu^y(u)\hm=I_{\{u=y\}}$, $y\hm\in [0,\,1]$. (В~терминах 
управляемых случайных последовательностей: $X\hm= \{0;1]\}$, $Y\hm=[0,1]$; 
$\mu_t(x_t\vert x^{t-1},y^t)\hm=x_t I_{\{y_t=y\}}+ (1-x_t)I_{\{y_t=y\}}$; $g(x)\hm=x$, 
$x\hm\in X$.) Интуитивно представляется очевидным, что невозможно найти максимум 
такой функции за счетное число шагов, если не знать значение, в котором она обращается 
в единицу. В~то же время формально для каждого объекта~$\mu^y$ существует 
оптимальная стратегия. Например, можно постоянно повторять управление~$y$. Однако 
не существует стратегии, равномерно оптимальной по отношению к классу 
$M\hm=\{\mu^y\}$. В~такой стратегии для каждого $y\hm\in [0,\,1]$ необходимо должно 
было бы выполняться следующее условие: $\sigma_t(y_t=y\vert \cdot)>0$ хотя бы для 
одного значения~$t$. Но это невозможно, поскольку для фиксированного значения~$t$ 
данное неравенство может быть выполнено лишь для счетного множества значений~$y$, а 
$t$ также пробегает счетное множество значений. Счетное объединение счетных 
множеств само счетно, поэтому необходимое неравенство не может быть выполнено для 
всех точек на отрезке [0,\,1].
  
  Аналогичные рассуждения показывают, что в данном примере не существует счетного 
множества стратегий, обладающего тем свойством, что для любого объекта найдется 
$\varepsilon$-оп\-ти\-маль\-ная стратегия из этого множества.
  
  Конечное или счетное множество стратегий $\Sigma\hm=\{\sigma(1),\sigma(2), \ldots \}$ 
назовем \textit{базовым} по отношению к классу объектов $M\hm\in \mathcal{M}$, если:
  \begin{enumerate}[(1)]
  \item для любого объекта из $M$ и любого $\varepsilon\hm>0$ существует оптимальная 
стратегия из множества~$\Sigma$;
  \item любая стратегия $\sigma(i)$ является устойчивой по отношению к классу~$M$.
  \end{enumerate}
  
  \smallskip
  
  \noindent
  \textbf{Теорема.} \textit{Строго адаптивная стратегия для класса объектов~$M$ 
существует тогда и только тогда, когда для этого класса существует базовое 
множество стратегий~$\Sigma$.}


%\hfill {\large Приложение~1}

\bigskip

%\pagebreak

\noindent
Д\,о\,к\,а\,з\,а\,т\,е\,л\,ь\,с\,т\,в\,о\ \ теоремы.

Необходимость условий в данном случае является тривиальной, поскольку строго 
адаптивная стратегия, если она существует, образует базовое множество 
стратегий~$\Sigma$, состоящее из одного элемента.
  
  Докажем достаточность. Определим с по\-мощью стратегий из~$\Sigma$ новую 
стратегию $a$ следующим образом. Обозначим
  $$
  \theta_{t,n}=\mathrm{Int}\left(\left( 1-v_{t,n}\right)^{-n}\right)\,,
  $$
где $\mathrm{Int}\left(a\right)$ означает целую часть числа~$a$, и зададим 
последовательность марковских моментов $\tau\hm=\{\tau_n\}$ с помощью рекуррентных 
соотношений

\pagebreak

\noindent
$$
\tau_0=0\,,\enskip \tau_n=\tau_{n-1}+n+\theta_n\,,
$$
где $\theta_n\hm=\theta_{\tau_{n-1},n}$. Соответствующие $\sigma$-ал\-геб\-ры обозначим 
$\mathbf{F}_{(n)}\hm=\mathbf{F}_{\tau_{n-1}}$.
  
  Будем считать, что на пространстве $(\Omega,\mathbf{F})$ задана последовательность 
случайных величин $\beta\hm=\{\beta_n\}$, независимых 
относительно~$\mathbf{F}_{(n)}$. Каждая случайная величина имеет одно и то же 
невырожденное распределение $\{b_i\}$ на множестве номеров стратегий из~$\Sigma$.
  
  Определим правила стратегии $a\hm=a(\Sigma,\beta)$ формулой
  $$
  a_t=\sum\limits_{n=1}^\infty \sigma_t(\beta_n) I_{\{\tau_{n-1}<t\leq \tau_n\}}\,,
  $$
где $\sigma_t(\beta_n)$~--- правило стратегии $\sigma(i)\hm\in\Sigma$ в момент~$t$, если 
$\beta_n\hm=i$.
  
  Наглядно работа стратегии~$a$ выглядит следующим образом. Процесс управления 
разбивается на этапы. Этап с номером $n$ начинается в момент $\tau_{n-1}+1$ и 
оканчивается в момент~$\tau_n;\tau_0\hm=0$. В~момент, предшествующий началу 
очередного этапа, определяется номер стратегии в множестве~$\Sigma$, из которой будут 
взяты правила для применения на данном этапе. Этот номер равен значению случайной 
величины~$\beta_n$. Продолжительность $n$-го этапа равна $n\hm+\theta_n$ и зависит, 
следовательно, от номера этапа и от оценки качества применяемой стратегии, полученной 
в течение первых $n$ тактов этапа. Стратегия~$a$ называется стратегией перебора~[2]. 
Таким образом, последовательность~$\beta$ определяет на каждом этапе выбор стратегии 
из множества~$\Sigma$, правила из которой применяются на этом этапе.
  
  Пусть задан объект $\mu\hm\in M$ и пусть $W\hm=W(\mu)$~--- точная верхняя грань 
доходов для этого объекта, взятая по всем допустимым стратегиям, и пусть %также
  \begin{alignat*}{2}
  W_i&=w(\mu,\sigma(i))\,; &\enskip v_n^{(1)}&=v_{\tau_{n-1},n}\,;\\
  v_n^{(2)}&=v_{\tau_{n-1},n+\theta_n}\,; &\enskip \Delta_n&=\tau_n-\tau_{n-1}=n+\theta_n\,.
  \end{alignat*}
  
  Для произвольного $\varepsilon>0$ определим множества
  $$
  A_n^{(k)}(\varepsilon)=\left\{ v_n^{(k)}\geq W-\varepsilon\right\}\,,
  $$
обозначая их дополнения $\overline{A_n^{(k)}(\varepsilon)}$, $k=1, 2$.
  
  Обозначим
  \begin{align*}
  s_n^{(1)} &= \sum\limits_{l=1}^n I_{A_l^{(1)}(\varepsilon)\cap 
{A_l^{(2)}(2\varepsilon)}} \Delta_l\,;\\
  s_n^{(2)} &= \sum\limits_{l=1}^n I_{A_l^{(1)}\cap 
\overline{A_l^{(2)}(2\varepsilon)}}\Delta_l\,;\\
  s_n^{(3)} &= \sum\limits_{l=1}^n I_{\overline{A_l^{(1)}(\varepsilon)}}\Delta_l\,,
  \end{align*}
так что $\tau_n\hm=\sum\limits_{l=1}^n \Delta_l\hm= s_n^{(1)}\hm+ s_n^{(2)}\hm+ 
s_n^{(3)}$.

\columnbreak

  
  С~помощью введенных обозначений запишем оценку для усредненного дохода к 
моменту~$\tau_n$:
  \begin{multline}
  w_n=\fr{1}{\tau_n}\sum\limits_{t=1}^{\tau_n} g_t=\fr{\sum\limits_{l=1}^n 
v_l^{(2)}\Delta_l} {\sum\limits_{l=1}^n \Delta_l}\geq{}\\
{}\geq (W-2\varepsilon) \fr{s_n^{(1)}} 
{s_n^{(1)}+s_n^{(2)}+s_n^{(3)}}\,.
  \label{e1-kon}
  \end{multline}
  
  Для оценки суммы $s_n^{(1)}$ запишем неравенство
  $$
  s_n^{(1)}\geq \Delta_{v_n}\,,
  $$
в котором обозначено
$$
v_n=\max\left\{ l:\ l\leq n,\ A_l^{(1)}(\varepsilon)\cap A_l^{(2)}(2\varepsilon)\right\}\,.
$$
  
  Оценим вероятность события $B_n\hm=\{v_n\hm\leq n-\ln n\}$, для которого выполняется 
включение
  $$
  B_n\subset \bigcap\limits_{n-\ln n<l\leq n} 
  \overline{A_l^{(1)}(\varepsilon)}\cap \overline{A_l^{(2)}(2\varepsilon)}\,.
  $$
  
  Согласно определениям эргодической стратегии, базового множества стратегий и 
семейства случайных величин~$\beta$ имеем:
  \begin{multline*}
  \mathbf{P}_{a} \left( \overline{A_l^{(1)}(\varepsilon)}\cup\overline{A_l^{(2)} 
(2\varepsilon)}\,\Big\vert \mathbf{F}_{(l)}\right)\leq{}\\
  {}\leq
  \sum\limits_{\substack{{i\in \mathcal{I};}\\ {W_i\leq W-\varepsilon/2}}}\!\!\!\!
   \mathbf{P}_{a}\left(\beta_l=i\vert 
\mathbf{F}_{(l)}\right)+{}\\
{}+  %\substack{{i=\overline{1,n}}\\ {j=\overline{1,l}}}
\sum\limits_{\substack{{i\in \mathcal{I};}\\ {W_i\leq W-\varepsilon/2}}}\!\!\!\!
\mathbf{P}_{a}\left( \overline{A_l^{(1)}(\varepsilon)}, \ \beta_l=i
\vert \mathbf{F}_{(l)}\right)\leq{}\\
  {}\leq \sum\limits_{\substack{{i\in \mathcal{I};}\\ {W_i\leq W-\varepsilon/2}}}\!\!\!\!
  \mathrm{P}_{a}(\beta_l=i)+{}\\
{}+\sum\limits_{\substack{{i\in \mathcal{I};}\\ {W_i> W-
\varepsilon/2}}}
\!\!\!\!\mathbf{P}_{a}\left( v_l^{(1)}\leq W_i-\fr{\varepsilon}{2}, \beta_l=i\vert 
\mathbf{F}_{(l)}\right) \leq{}\\
  {}\leq \sum\limits_{\substack{{i\in \mathcal{I};}\\ {W_i\leq W-\varepsilon/2}}}\!\!\!\!
   b_i+a_l\left( 
\fr{\varepsilon}{2}\right) \leq q<1
  \end{multline*}
при всех достаточно больших~$l$. Отсюда следует, что для всех достаточно больших 
значений~$n$ выполняется неравенство
$$
\mathbf{P}_a(B_n)\leq q^{n-\ln n}\,.
$$
  
  Следовательно, согласно лемме Бо\-ре\-ля--Кан\-тел\-ли
  \begin{equation}
  \mathbf{P}_{a}\left( \overline{\lim\limits_{n\rightarrow\infty}} B_n\right)=0\,.
  \label{e2-kon}
  \end{equation}
  
  Это означает, что
  $$
  s_n^{(1)}\geq \Delta_{v_n}\geq (1-W-\varepsilon)^{-n+\ln n}\,.
  $$
  
  Оценим сумму $s_n^{(2)}$. Обозначив 
$C_n\hm=A_n^{(1)}(\varepsilon)\cap$\linebreak 
$\cap\overline{A_n^{(2)}(2\varepsilon)}$ и $W_{(n)}\hm=\sum\limits_{i\in 
I} W_i I_{\{\beta_n=i\}}$, получим:
  \begin{multline*}
  \mathrm{P}_{a}\left(C_n\vert \mathrm{ F}_{(n)}\right)=
  \mathrm{P}_{a|} \left( C_n, W_{(n)}<W-\fr{3\varepsilon}{2}\vert \mathrm{
  F}_{(n)}\right) +{}\\
  {}+ \mathrm{P}_{a}\left( 
  C_n, W_{(n)}\geq W-\fr{3\varepsilon}{2}\vert \mathrm{
  F}_{(n)}\right)\leq{}\\
  {}\leq \mathrm{P}_{a}\left( v_n^{(1)}>W-\varepsilon,\, W_{(n)}<W-\fr{3\varepsilon}{2}\vert \mathrm{
  F}_{(n)}\right)+{}\\
  {}+
  \mathrm{P}_{a} \left( v_n^{(2)}\leq W-2\varepsilon,\, W_{(n)}\geq W-
\fr{3\varepsilon}{2}\vert \mathrm{
  F}_{(n)}\right)\leq{}\\
  {}\leq \sum\limits_{i\in \mathcal{I}; W_i\leq W- \varepsilon/2} \mathrm{P}_{a}\left(
  v_{\tau_n,n}>W_i+\fr{\varepsilon}{2},\, \beta_l=i\vert\mathrm{F}_{(n)}\right)+{}\\
  {}+\sum\limits_{\substack{{i\in \mathcal{I};}\\ {W_i> W- 3\varepsilon/2}}}\!\!\!\!
   \mathbf{P}_{a} \left( 
v_{\tau_n,n+\theta_n}\leq W_i-\fr{\varepsilon}{2},\,\beta_l=i\vert\mathbf{F}_{(n)}\right)\leq {}\\
{}\leq
a_n\left( \fr{\varepsilon}{2}\right)\,.
  \end{multline*}
  
  Из определения базового множества стратегий следует, что
  $$
  \sum\limits_{n=1}^\infty \mathbf{P}_{a} (C_n)<\infty\,,
  $$
поэтому согласно лемме Бо\-ре\-ля--Кан\-тел\-ли полу\-чаем:
\begin{equation}
\mathbf{P}_{a}\left( \overline{\lim\limits_{n\rightarrow\infty}} C_n\right) =0\,.
\label{e3-kon}
\end{equation}
  
  Отсюда следует, что
  $$
  \sup\limits_n s_n^{(2)}\leq c<\infty\,.
  $$
  
  Для суммы $s_n^{(3)}$ имеем следующую оценку:
  $$
  s_n^{(3)}\geq \sum\limits_{l=1}^n \left(n+(1-W+\varepsilon)^{-l}\right)< n^2+n(1-
W+\varepsilon)^{-n}.
  $$
  
  Подставляя оценки, полученные для сумм $s_n^{(k)}$, в неравенство~(\ref{e1-kon}), 
получаем:
  \begin{multline*}
  w_n\geq (W-\varepsilon) \left( 1+\fr{s_n^{(2)}+s_n^{(3)}}{s_n^{(1)}}\right)^{-1}\geq 
{}\\
  {}\geq (W-\varepsilon)\left( 1+\fr{c+n^2+n(1-W+\varepsilon)^{-n}}{(1-W-\varepsilon/2)^{-
n+\ln n}}\right)^{-1}\geq{}\\
{}\geq W-3\varepsilon
  \end{multline*}
для всех достаточно больших значений~$n$. Отсюда
\begin{equation}
\lim\limits_{\overline{n\rightarrow\infty}} w_n\geq W\,.
\label{e4-kon}
\end{equation}
  
  Рассмотрим далее множество
  $$
  \Omega^\prime =\left\{ \lim\limits_{n\rightarrow\infty} w_n =W\right\}\cap 
\overline{B}\cap\overline{C}\,,
  $$
где $\overline{B}$ и $\overline{C}$ означают соответственно дополнения к множествам 
$B\hm= \overline{\lim\limits_{n\rightarrow\infty}} B_n$ и $C\hm= 
\overline{\lim\limits_{n\rightarrow\infty}} C_n$.
  
  Согласно формулам~(\ref{e2-kon})--(\ref{e4-kon})
  $$
  \mathbf{P}_{a}\left(\Omega^\prime\right) =1\,.
  $$
  
  Определим следующие события:
  
  \noindent
  \begin{align*}
  D_{n,t}^{(1)} &= \left\{ \tau_{n-1}<t\leq \tau_{n-1}+n\right\} \cap \Omega^\prime\,;\\
  D_{n,t}^{(2)} &= \left\{\tau_{n-1}+n<t\leq \tau_n\right\}\cap \Omega^\prime\,;\\
  D_{n,t}^{(3)} &= \left\{ \tau_{n-1}<t\leq \tau_n\right\} \cap \Omega^\prime\,.
  \end{align*}
  
  На множестве $D_{n,t}^{(1)}$ усредненный доход $v_t\hm=v_{0,t}\hm=
  t^{-1}\sum\limits_{s=1}^t g_s$ оценивается с помощью формулы~(\ref{e1-kon}) как
  
    \noindent
  $$
  v_t\geq \fr{\tau_{n-1} w_n}{\tau_{n-1}+n+\theta_n}\geq W-\varepsilon_n^{(1)}\,,
  $$
где $\varepsilon_n^{(1)}\hm\rightarrow0$ при $n\hm\rightarrow\infty$.
  
  Пусть событие $D_{n,t}^{(2)}$ имеет место. Тогда $\theta_n\geq (1\hm- 
W\hm+\varepsilon)^{-n}$. Кроме того, из определения событий $B_n$, $B$, 
$D_{n,t}^{(2)}$ следует, что для всех достаточно больших значений~$n$ выполняется 
неравенство $v_n\hm> n-\ln n$. Следовательно, на множестве~$D_n^{(2)}$ справедлива 
оценка

  \noindent
  $$
  v_t\geq \fr{\tau_{n-1} w_n}{\tau_{n-1}+n+\theta_n}\geq W-\varepsilon_n^{(2)}\,,
  $$
где $\varepsilon_n^{(2)}\hm\rightarrow0$ при $n\hm\rightarrow\infty$.
  
  Из определения событий $C_n$, $C$, $D_{n,t}^{(3)}$ вытекает, что
  
    \noindent
  $$
  D_{n,t}^{(3)} \subset \left\{ \min\limits_{n<m\leq n+\theta_n} v_{n,m}\geq W-
2\varepsilon\right\}\,,
  $$
поэтому на множестве $D_n^{(3)}$ справедливы неравенства:

  \noindent
\begin{multline*}
\!\!v_t\geq \fr{\tau_{n-1} w_n}{t}+\left(1- \fr{\tau_{n-1}}{t}\right) \left( 1-\tau_n\right)^{-1} 
\!\!\sum\limits_{s=\tau_{n-1}+1}^t \!\!\!\!g_s\geq{}\\
{}\geq \fr{\tau_{n-1} w_n}{t}+\left( 1-\fr{\tau_{n-1}}{t}\right)\left( W-2\varepsilon\right) \geq 
W-2\varepsilon -\varepsilon_n^{(3)},
\end{multline*}
где $\varepsilon_n^{(3)}\rightarrow0$ при $n\hm\rightarrow\infty$.

\pagebreak
  
  Таким образом, на множестве
  $$
  D_{n,t}=\bigcup\limits_{k=1}^3 D_{n,t}^{(k)} = \left\{ \tau_{n-1}<t\leq \tau_n\right\} \cap 
\Omega^\prime
  $$
имеет место оценка $v_n\hm\geq W-\varepsilon-\varepsilon_n$, где 
$\varepsilon_n\hm\rightarrow 0$ при $n\hm\rightarrow\infty$. Достаточность утверждения 
теоремы следует из соотношений $\Omega\hm= \bigcup\limits_{n=1}^\infty \left\{ \tau_{n-
1}\hm<t\hm\leq \tau_n\right\}$ и $\lim\limits_{t\rightarrow\infty} I_{D_{n,t}}\hm=0$.

\section{Заключение}

  Адаптивные стратегии, позволяющие достигать цели в условиях информационной 
неопреде\-лен\-ности, основываясь на <<обучении>> в процессе взаимодействия с объектом, 
находят все более широкое практическое применение. 

В~этой работе было уделено 
внимание теоретическим аспектам адаптивного подхода. Сформулированы определения 
адаптивных стратегий и приведена формальная постановка задачи адаптивного 
управления. Сформулированы и доказаны некоторые утверждения о необходимых 
условиях и достаточных условиях адап\-тив\-ной управляемости. 

Продолжение исследований 
в данном на\-прав\-ле\-нии позволит найти ответы на принципиальные вопросы, в каких 
ситуациях можно рассчитывать на <<приспособление к неизвестной среде>> и сколь 
универсальными могут быть <<обучающиеся>> алгоритмы.



{\small\frenchspacing
{%\baselineskip=10.8pt
\addcontentsline{toc}{section}{Литература}
\begin{thebibliography}{9}


  \bibitem{1-kon}
  \Au{Sragovich~V.\,G.}
  Mathematical theory of adaptive control.~--- Singapore: World Scientific, 2006.
  \bibitem{2-kon}
  \Au{Коновалов~М.\,Г.}
  Методы адаптивной обработки информации и их приложения.~--- М.: ИПИ РАН, 2007.
  
  \label{end\stat}
  
  \bibitem{3-kon}
  \Au{Неве~Ж.}
  Математические основы теории вероятностей.~--- М.: Мир, 1969.
\end{thebibliography}
}
}


\end{multicols}  %9
\def\stat{meih}

\def\tit{СТАЦИОНАРНЫЕ ВЕРОЯТНОСТИ СОСТОЯНИЙ В~СИСТЕМЕ ОБСЛУЖИВАНИЯ КОНЕЧНОЙ ЕМКОСТИ 
С~ИНВЕРСИОННЫМ ПОРЯДКОМ
ОБСЛУЖИВАНИЯ И~ОБОБЩЕННЫМ ВЕРОЯТНОСТНЫМ
ПРИОРИТЕТОМ$^*$}

\def\titkol{Стационарные вероятности состояний в~системе обслуживания конечной емкости} 
%с~инверсионным порядком обслуживания и~обобщенным вероятностным приоритетом}

\def\aut{Л.\,А.~Мейханаджян$^1$}

\def\autkol{Л.\,А.~Мейханаджян}

\titel{\tit}{\aut}{\autkol}{\titkol}

\index{Мейханаджян Л.\,А.}
\index{Meykhanadzhyan L.\,A.}

{\renewcommand{\thefootnote}{\fnsymbol{footnote}} \footnotetext[1]
{Работа выполнена при поддержке РФФИ (проект 15-07-03007).}}


\renewcommand{\thefootnote}{\arabic{footnote}}
\footnotetext[1]{Российский университет дружбы народов, lameykhanadzhyan@gmail.com}


\Abst{Рассматривается система
$M/G/1/(r-1)$ с~дисциплиной инверсионного порядка обслуживания
и обобщенного вероятностного приоритета.
Предполагается, что в~момент поступления новой заявки в~систему
становится известной ее длина и,~кроме того, в~любой момент времени
известна остаточная длина каждой заявки в~системе.
В~момент поступления очередной заявки в~непустую систему ее
исходная длина сравнивается с~остаточной длиной заявки на приборе, и~в зависимости
от результатов сравнения наступает одно из следующих событий:
обе заявки покидают систему; только одна из заявок
покидает систему (другая остается на приборе);
обе заявки остаются в~системе (одна попадает на прибор, другая~--- в~очередь).
Заявки, оставшиеся в~системе, приобретают новую
(случайную) длину в~соответствии с~заданным распределением, зависящим в~общем случае от
исходных длин заявок.
Заявки, застающие систему полностью заполненной, теряются
и не оказывают на нее никакого воздействия.
В~статье предложены математические соотношения
для вычисления совместного стационарного распределения
числа заявок в~системе и~остаточного времени обслуживания заявки на приборе,
периода занятости системы, стационарного распределения
времени ожидания и~пребывания заявки длины~$x$ (в~терминах преобразования 
Лап\-ла\-са--Стил\-тье\-са (ПЛС)).}

\KW{система массового обслуживания; специальные
дисциплины; инверсионный порядок
обслуживания; вероятностный приоритет}

\DOI{10.14357/19922264160214} 

\vspace*{6pt}

\vskip 12pt plus 9pt minus 6pt

\thispagestyle{headings}

\begin{multicols}{2}

\label{st\stat}

\section{Введение}

В этой работе, являющейся продолжением работ~\cite{n1, n2}, будет
рассматриваться та же однолинейная система
массового обслуживания (СМО), что и~в~\cite{n1}, но ограниченной емкости.
Основной результат работ~\cite{n1, n2} состоит в~нахождении
совместного стационарного распределения вероятностей состояний %\linebreak 
системы
$M/G/1$ с~дисциплиной инверсионного %\linebreak 
порядка
обслуживания и~обобщенного вероятностного приоритета, а~также основных
стационарных вероятностных характеристик в~терминах ПЛС. %\linebreak
Сейчас же задача заключается в~исследовании стационарных 
ве\-ро\-ят\-ност\-но-вре\-мен\-н$\acute{\mbox{ы}}$х характеристик указанной системы
в~случае, когда присутствует ограничение на размер очереди.

\vspace*{-4pt}

\section{Описание системы}

Рассмотрим СМО
с~одним прибором,
одной очередью для ожидающих заявок емкости $(r\hm-1)\hm<\infty$, $r \hm\ge 2$,
и~входящим потоком заявок, который для простоты будем называть здесь
потоком пуассоновского типа. Отличие этого потока от пуассоновского
заключается в~следующем: интенсивность поступления заявок равна~$\lambda$,
если на приборе имеется заявка, и~$\tl$, если система пуста.


Если в~момент поступления заявки в~систему
на приборе имеется заявка, то исходное распределение времени обслуживания поступающей
заявки является произвольным с~функцией распределения (ФР) $B(x)$.
Если же заявка поступает в~систему в~тот момент, когда система пуста, то исходное
распределение времени обслуживания поступающей
заявки является произвольным с~ФР~$\tB(x)$.

Далее для простоты изложения будем считать, что ФР $B(x)$ и~$\tB(x)$ имеют непрерывные
ограниченные плотности распределения $b(x)\hm=B'(x)$ и~$\tb(x)\hm=\tB'(x)$,
причем $\tb \hm= \int_0^\infty x \tb(x)\,dx \hm< \infty$
и~$b \hm= \int_0^\infty x b(x)\,dx \hm< \infty$.

Обобщенный инверсионный порядок обслуживания с~вероятностным приоритетом (LCFS BPP)
заключается в~следующем.
Предполагается, что в~любой момент времени известна остаточная длина (далее будем говорить
просто длина) каждой заявки в~системе.
В~момент поступления в~систему новой заявки ее
исходная длина~$u$ сравнивается с~(остаточной) длиной~$v$ заявки на приборе.
С~вероятностью~$D(x,y|u,v)$,
зависящей только от~$u$ и~$v$, обслуживавшаяся ранее заявка продолжает обслуживаться, причем
ее длина становится меньше~$y$, а~вновь
поступившая становится на первое место в~очереди и~ее длина становится меньше~$x$.
Кроме того, с~вероятностью~$D^*(x,y|u,v)$,
зависящей только от~$u$ и~$v$, вновь поступившая заявка занимает прибор, вытесняя обслуживавшуюся
ранее на первое место в~очереди, причем длина заявки, бывшей ранее на приборе, становится
меньше~$y$, а~вновь поступившей~--- меньше~$x$.

Если на приборе находится заявка остаточной длины~$v$ и~в~систему поступает заявка
длины~$u$, то с~вероятностью $D_0(x|u,v)$ заявка, находящаяся на приборе, покидает
систему, а~поступившая заявка становится на
прибор, причем ее длина становится меньше~$x$.
Кроме того, с~вероятностью
$D_0^*(y|u,v)$ поступившая заявка сразу же покидает систему, а~заявка, находящаяся на
приборе, продолжает обслуживаться, причем ее длина становится меньше~$y$.
Введем также обозначение:
\begin{equation*}
%\label{(2.1)}
D(x|u,v) = D_0(x|u,v) + D_0^*(x|u,v)\,.
\end{equation*}
Здесь $D(x|u,v)$~--- вероятность того, что одна из двух заявок покинет систему, а~вторая встанет
на прибор и~примет длину меньше~$x$.

Наконец, предполагается, что с~вероят\-ностью~$d_0(u,v)$ обе заявки покидают
систему, а~на прибор становится первая заявка из очереди.

Будем считать для удобства изложения, что все ФР 
$D(x,y|u,v)$, $D^*(x,y|u,v)$, $D_0(x|u,v)$,
$D_0^*(y|u,v)$, $D(y|u,v)$ и~$D_0(u,v)$
имеют непрерывные ограниченные плотности
$d(x,y|u,v)\hm=\partial^2 D(x,y|u,v)/(\partial x \partial y)$,
$d^*(x,y|u,v)\hm=\partial^2 D^*(x,y|u,v)/(\partial x \partial y)$,
$d_0(x|u,v)\hm= \partial D_0(x|u,v)/\partial x$,
$d_0^*(y|u,v)=\partial D_0^*(y|u,v)/\partial y$
и~$d(x|u,v)\hm=\partial D(x|u,v)/\partial x$.


Естественно, для любых~$u$ и~$v$ выполнено условие:
\begin{multline*}
%\label{e2.1-m}
\int\limits_0^\infty \int\limits_0^\infty
\left[d(x,y|u,v) + d^*(x,y|u,v)\right]\,dxdy+{}\\
{}+ \int\limits_0^\infty d(x|u,v) \,dx
+ d_0(u,v) =1\,.
\end{multline*}

Если длина заявки на приборе становится
равной нулю, то она мгновенно покидает систему и~на прибор переходит первая
заявка из очереди. Остальная очередь сдвигается на единицу.


Для конечного накопителя необходимо также задать
дисциплину принятия заявок в~систему при отсутствии в~нем свободных мест.
Здесь для простоты изложения будет рассмотрен только
тот случай, когда поступающая в~заполненную систему заявка теряется.
Заметим, что в~этом случае принятая в~систему заявка
будет обязательно обслужена полностью.
Для всех СМО с~такой дисциплиной принятия заявок в~систему
при отсутствии в~накопителе свободных мест стационарные
вероятности\linebreak $p_n(x_1,\ldots,x_n)$ при $n\hm<r$ совпадают
с~точностью до\linebreak постоянной с~аналогичными вероятностями
для системы с~бесконечным накопителем, различие заключается
только в~вероятностях $p_{r}(x_1,\ldots,x_{r})$.
Однако несколько более сложно вычисляются стационарные
распределения, связанные с~временем пребывания заявки 
в~системе, поскольку даже заявки, принятые в~систему, могут покидать
ее недообслуженными.

Далее будем предполагать, что система
функционирует в~стационарном режиме
и~$\tb \hm= \int_0^\infty x\tb(x)\,dx \hm< \infty$
и~$b \hm= \int_0^\infty x b(x)\,dx \hm< \infty$.
Отметим, что параметр $\rho \hm= \lambda b$ для данной системы не
является загрузкой в~традиционном смысле и~может существенно от нее отличаться.


\section{Стационарные вероятностные характеристики}

Обозначим через $\nu(t)$ число заявок в~системе
в~момент~$t$, а~через $\vec\xi(t)\hm =(\xi_{1}(t),\ldots,\xi_{\nu(t)}(t))$~---
вектор, координатой $\xi_{1}(t)$ которого
является (остаточное) время обслуживания
заявки, находящейся в~этот момент на приборе,
$\xi_{2}(t)$~--- первой заявки в~очереди$,\ldots,$ $\xi_{\nu(t)-1}(t)$~---
последней, \mbox{$(\nu(t)-1)$-й} заявки в~очереди.
При $\nu(t)\hm=0$ вектор $\vec\xi(t)$ не определяется.
Тогда $\eta(t)\hm=(\nu(t),\vec\xi(t))$ представляет
собой марковский процесс, описывающий поведение числа заявок в~рассматриваемой системе.

Положим 
\begin{align*}
p_{0}(t)&= \mathbf{P}\{\nu(t)=0\}\,;
\\
P_{n}\left(t;x_1,\ldots,x_{n}\right) &=
\mathbf{P}\{\nu(t)=n\,,\\
&\hspace*{-20pt}\xi_{1}(t)<x_{1},\ldots,\xi_{n}(t)<x_{n}\}
\,,\enskip 1 \le n \le r\,.
\end{align*}
Обозначим через
\begin{equation*}
%\label{(2.1)}
p_{0} = \lim\limits_{t\to\infty}
p_{0}(t) \,;
\end{equation*}
\begin{equation*}
%\label{(2.1)}
P_{n}(x_1,\ldots,x_{n}) = \lim\limits_{t\to\infty}
P_{n}(t;x_1,\ldots,x_{n}) \,,\enskip 1 \le n \le r\,,
\end{equation*}
стационарное распределение процесса $\eta(t)$.
В~силу сделанных в~предыдущем пункте
предположений относительно параметров системы,
можно показать (см., например,~[3; 4, с.~273]), что существуют
непрерывные и~ограниченные плотности 

\noindent
\begin{multline*}
p_n(x_1,\ldots,x_{n}) = \fr{\partial^n }{\partial x_1\cdots \partial x_n}
P_n(x_1,\ldots,x_{n}) \,,\\[-1pt]
1 \le n \le r\,.
\end{multline*}

Выпишем систему интегродифференциальных
уравнений, которой удовлетворяют стационарные
плотности $p_n(x_1,\ldots,x_{n})$ и~которую
для краткости по аналогии с~простейшими СМО будем называть системой уравнений
равновесия (СУР). Для этого рассмотрим вспомогательную систему
с~$(n\hm-1)$ мес\-та\-ми ожидания, отличающуюся от исходной
сис\-те\-мы только тем, что если в~очереди
находится $(n\hm-1)$ заявок, заявка на приборе имеет
остаточную длину~$v$ и~поступает новая заявка
длины~$u$, то с~ве\-ро\-ят\-ностью $d(x,y|u,v)$ на
приборе остается вновь поступившая заявка,
длина которой становится равной~$x$, а~обслуживавшаяся ранее заявка покидает
сис\-те\-му, и~наоборот: с~вероятностью $d^*(y,x|u,v)$ систему покидает вновь
поступившая заявка, а~находившаяся ранее на приборе заявка продолжает обслуживаться, но
ее длина становится равной~$x$.

В силу метода исключения состояний~\cite{ppav}
стационарные вероятности состояний в~исходной
и~вспомогательной системах отличаются лишь на
постоянный множитель (за исключением вероятности $p_{r}(x_1,\ldots,x_{r})$).
Это дает возможность при составлении
уравнений для $p_n(x_1,\ldots,x_{n})$, $n\hm\ge 1$,
воспользоваться вспомогательной системой и~получить следующие соотношения:

\noindent
\begin{multline}
\label{e3.1-mei}
-p'_1(x) = \tl \tb(x) p_0 - \lambda p_1(x)
+ {}\\[-1pt]
{}+\lambda \Bigg( \int\limits_0^\infty \int\limits_0^\infty
d(x|u,v) b(u) p_1(v) \,dudv +{}
\\[-1pt]
{}+ \int\limits_0^\infty \int\limits_0^\infty
\int\limits_0^\infty \left[d(x,y|u,v) b(u) p_1(v)  +{}\right.\\[-1pt]
\left.{}+
d^*(y,x|u,v) b(u) p_1(v)\right] \,dydudv
\Bigg)\,;
\end{multline}

\vspace*{-16pt}

\noindent
\begin{multline*}
-p'_{n}\left(x_1,\ld,x_n\right) ={}
\\
{}=
\lambda \Bigg(
\int\limits_0^\infty \int\limits_0^\infty
\left[d\left(x_2,x_1|u,v\right) b(u) p_{n-1}\left(v,x_3\ld,x_n\right)
+ {}\right.\\
\left.{}+
d^*\left(x_1,x_2|u,v\right) b(u) p_{n-1}\left(v,x_3,\ld,x_n\right)\right]
\,dudv \Bigg)
-{}\\
{}-
\lambda p_{n}\left(x_1,\ld,x_n\right)
+{}\\
{}+ \lambda \Bigg(
\int\limits_0^\infty \int\limits_0^\infty
d\left(x_1|u,v\right) b(u) p_{n}\left(v,x_2,\ld,x_n\right)
\,dudv +{}
\end{multline*}

\noindent
\begin{multline}
{}+
\int\limits_0^\infty \int\limits_0^\infty
\int\limits_0^\infty \left[d\left(x_1,y|u,v\right) b(u) p_{n}\left(v,x_2,\ld,x_n\right)
+{}\right.
\\
\left.{}+
d^*\left(y,x_1|u,v\right) b(u) p_{n}\left(v,x_2,\ld,x_n\right)\right]
\,dy du dv \Bigg)\,,
\\
 1 \le n \le r-1\,;
 \label{e3.2-mei}
\end{multline}

\vspace*{-12pt}

\noindent
\begin{multline}
\label{e3.3-mei}
-p'_{r}\left(x_1,\ld,x_n\right) ={}\\
{}=
\lambda \Bigg(
\int\limits_0^\infty \int\limits_0^\infty
\left[d\left(x_2,x_1|u,v\right) b(u) p_{n-1}\left(v,x_3\ld,x_n\right)
+{}\right.
\\
\!\!\!\!\left.{}+
d^*\left(x_1,x_2|u,v\right) b(u) p_{n-1}\left(v,x_3,\ld,x_n\right)\right]
\,du dv \!\Bigg).\!\!
\end{multline}

Остановимся подробнее на выводе уравнения
для плотности $p_{r}(x_1,\ldots,x_{r})$ (остальные уравнения
получаются так же, как и~в случае накопителя бесконечной емкости~\cite{n1}).
Рассмотрим моменты времени~$t$ и~$(t\hm+\Delta)$.
Тогда для того, чтобы в~момент времени
$(t\hm+\Delta)$ в~системе находилось~$r$~заявок, причем
на приборе заявка длины~$x_1$, а~в~очереди
заявки длин $x_2,\ldots,x_r$, нужно, чтобы произошло одно из следующих событий:
\begin{itemize}
\item в~момент~$t$ в~системе находилось $(r-1)$
заявок, причем заявка на приборе имела
длину~$v$, первая заявка в~очереди имела
длину $x_3,\ldots,$ последняя заявка в~очереди
имела\linebreak
 длину~$x_n$ (с~плотностью вероятностей $p_{r-1}(t;v,x_3,\ldots,x_r)$),
и~за время~$\Delta$ поступила заявка (с~вероятностью $\lambda\Delta$) длины~$u$
(с~плотностью вероятностей $b(u)$).
Заявка на приборе продолжает обслуживаться,
но ее длина становится равной~$x_1$, а~вновь
поступившая заявка занимает первое мес\-то в~очереди и~ее длина становится равной~$x_2$
(с~плот\-ностью вероятностей $d(x_2,x_1|u,v)$);
\item
в момент~$t$ в~системе находилось $(r\hm-1)$
заявок, причем заявка на приборе имела
длину~$v$, первая заявка в~очереди имела
длину $x_3,\ldots,$ последняя заявка в~очереди имела\linebreak
 длину~$x_n$ (с~плот\-ностью
вероятностей $p_{r-1}(t;v,x_3,\ldots,x_r)$),
и~за время~$\Delta$ поступила заявка (с~вероятностью $\lambda\Delta$) длины~$u$
(с~плот\-ностью вероятностей $b(u)$).
Поступившая заявка занимает прибор и~ее длина
становится равной~$x_1$, а~заявка,
обслуживавшаяся до поступления новой заявки,
занимает первое мес\-то в~очереди и~ее длина
становится равной~$x_2$ (с~плотностью вероятностей $d^*(x_1,x_2|u,v)$);
\item
в момент~$t$ в~системе находилось~$r$~заявок,
причем заявка на приборе имела длину $x_1\hm+\Delta$, первая заявка в~очереди
имела дли-\linebreak\vspace*{-12pt}

\pagebreak

\noindent
ну $x_2,\ldots,$ последняя заявка в~очереди имела длину~$x_r$ (с плотностью
вероятностей $p_r(t;x_1\hm+\Delta,x_2,\ldots,x_r)$).
\end{itemize}


Вероятности других событий равны $o(\Delta)$.
Применяя формулу полной вероятности, имеем:
\begin{multline*}
p_{r}\left(t+\Delta;x_1,\ld,x_r\right) ={}\\
\!{}=\!
\lambda\Delta \Bigg(\!
\int\limits_0^\infty \!\int\limits_0^\infty\!
\left[d\left(x_2,x_1|u,v\right) b(u) p_{r-1}\left(t;v,x_3,\ld,x_r\right)
+{}\right.\hspace*{-3.62766pt}
\\
\left.{}+
d^*\left(x_1,x_2|u,v\right) b(u) p_{r-1}\left(t;v,x_3,\ld,x_r\right)\right]
\,du dv \Bigg)
+ {}\\
{}+p_{r}\left(t;x_1+\Delta,x_2,\ld,x_r\right)\,,
\end{multline*}
откуда, перенося слагаемое
$p_r(t;x_1+\Delta,x_2,\ldots,x_{r})$ в~левую часть равенства, деля на~$\Delta$,
устремляя~$\Delta$ к~нулю и~учитывая стационарный режим функционирования системы,
получаем уравнение~\eqref{e3.3-mei}.


К системе уравнений~\eqref{e3.1-mei}--\eqref{e3.3-mei} 
нужно добавить начальные условия, которые удобно записать\linebreak \mbox{в~виде}:
\begin{align}
p_{1}(\infty) &= \lim\limits_{X\to \infty} p_{1}(X)
= 0\,; \label{e3.33-mei}
\\
p_{n}(\infty,x_2,\ld,x_r)
&= {}\notag\\
&\hspace*{-20mm}{}=\lim\limits_{X\to \infty} p_{n}\left(X,x_2,\ld,x_r\right)
= 0\,,\enskip
1 \le n \le r\,.
\label{e3.4-mei}
\end{align}
Как получаются соотношения~\eqref{e3.33-mei} и~\eqref{e3.4-mei},
показано в~\cite{n1}.
Оставшаяся неизвестной стационарная вероятность~$p_0$ отсутствия заявок в~системе
находится, как обычно, из условия нормировки:
\begin{equation}
\label{e3.6-mei}
\sum\limits_{n=0}^r p_n = 1\,, 
\end{equation}
где
$p_n=P_n(\infty,\ld,\infty)$, $1 \hm\le n\hm \le r$,~---
стационарная вероятность наличия в~системе~$n$~заявок.

Как и~в случае системы бесконечной емкости,
полученные соотношения~\eqref{e3.1-mei}--\eqref{e3.6-mei} позволяют
теоретически последовательно по~$n$
находить стационарные плотности вероятностей $p_n(x_1,\ldots,x_{n})$.
Однако на практике такие расчеты связаны с~серьезными вычислительными сложностями.


Как показано, например, в~\cite{n3}, в~практических случаях
бывает достаточно знать только маргинальные плотности
\begin{multline*}
%\label{(2.1)}
p_{n}(x) = \mathop{\int\cd\int}\limits_{x_2,\ld,x_n>0}
p_{n}\left(x,x_2\ld,x_n\right)\,dx_2\cdots dx_n\,,
\\ 
2 \le n \le r\,.
\end{multline*}

Интегрируя~\eqref{e3.2-mei} и~\eqref{e3.3-mei} по
$x_2,\ldots ,x_r$ в~пределах от нуля до бесконечности и~вспоминая равенство~\eqref{e3.1-mei}, 
получаем следующую систему интегродифференциальных уравнений
для $p_{n}(x)$, $1 \hm\le n \hm\le r$:
\begin{align}
-p'_{n}(x) &= a_n(x) - \lambda p_{n}(x) +
\int\limits_0^\infty K_n(x,v) p_{n}(v)\,dv \,,\notag\\ 
&\hspace*{35mm}1  \le n \le r-1\,; \label{e3.7-mei}\\
-p'_{r}(x) &= a_r(x)\,,  \label{e3.7-1-mei}
\end{align}
где $a_1(x)=\tl \tb(x) p_0$ и
\begin{multline*}
a_{n}(x) = \lambda \Bigg( \!\int\limits_0^\infty\!
p_{n-1}(v)\,dv \! \int\limits_0^\infty\!
b(u)\,du \!\int\limits_0^\infty\!
\left[d(y,x|u,v) +{}\right.\\
\left.{}+ d^*(x,y|u,v)\right] \,dy
\Bigg)\,,\enskip 1 \le n \le r\,;
\end{multline*}

\vspace*{-12pt}

\noindent
\begin{multline*}
%\label{(2.1)}
K_n(x,v) = \lambda \int\limits_0^\infty b(u)\,du
\Bigg( d(x|u,v) +{}\\
{}+
\int\limits_0^\infty \!\left[d(x,y|u,v) + d^*(y,x|u,v)\right]
\,dy\! \Bigg),
\enskip 1 \le n \le r-1.
\end{multline*}
Начальные условия для уравнений~\eqref{e3.7-mei} и~\eqref{e3.7-1-mei}
по аналогии с~\eqref{e3.33-mei} запишем в~виде:
\begin{equation}
\label{e3.8-mei}
p_{n}(\infty) = \lim\limits_{X\to \infty} p_{n}(X)
= 0 \,,\enskip 1 \le n \le r\,. 
\end{equation}


Решать систему~\eqref{e3.7-mei} и~\eqref{e3.7-1-mei}
с~начальными условиями~\eqref{e3.8-mei} можно различными способами.
Воспользуемся методом, предложенным в~\cite{n1}.
Прежде всего заметим, что из~\eqref{e3.7-1-mei} немедленно следует, что
\begin{equation*}
%\label{(3.7-1)}
p_r(x) = \int\limits_x^\infty a_r(u) \,du\,.
\end{equation*}
Решение уравнений~\eqref{e3.7-mei} будем искать в~виде:
\begin{equation}
\label{e4.1-mei}
p_n(x) = e^{\lambda x} q_n(x)\,,\enskip 1 \le n \le r-1\,.
\end{equation}
Подставляя в~\eqref{e3.7-mei} вместо $p_n(x)$ ее выражение
по формуле~\eqref{e4.1-mei}, получаем новое интегродифференциальное уравнение:
\begin{multline*}
- q'_n(x) = e^{-\lambda x} a_n(x) +
\int\limits_0^\infty e^{\lambda v} e^{-\lambda x} K_n(x,v) q_n(v)\, dv\,, \\
1 \le n \le r-1\,.
\end{multline*}
Интегрируя последнее равенство по~$x$ в~пределах от~$y$ до~$\infty$ и~учитывая
начальное условие~\eqref{e3.8-mei}, получаем
интегральное уравнение Фредгольма 2-го рода:
\begin{multline}
\label{e2.1n-mei}
q_n(y)= b_n(y) + \int\limits_0^\infty
G_n(y,v) q_n(v)\, dv \,, \\ 
1 \le n \le r-1\,,
\end{multline}
где
\begin{align*}
b_n(y) &= \int\limits_y^\infty e^{-\lambda x} a_n(x)\, dx\,; \\
G_n(y,v) &= \int\limits_y^\infty e^{\lambda (v-x)} K_n(x,v)\, dx\,.                              %       (4.2)
\end{align*}
Отметим, что свободный член $b_n(y)$ и~ядро
$G_n(y,v)$ интегрального уравнения являются неотрицательными функциями.
Далее для расчета $q_n(y)$ можно применить
подходящий метод решения интегральных уравнений Фредгольма 2-го рода
(см., например,~[6--8]).

В~некоторых частных случаях решения уравнений~\eqref{e2.1n-mei}  могут быть выписаны в~явном виде.
 Например, это возможно в~случае, когда  известны сепарабельные аппроксимации для функций
 $d(x,y|u,v)$, $d^*(x,y|u,v)$, $d_0(x|u,v)$ и~$d_0^*(x|u,v)$,
 т.\,е.\ разложения вида:
\begin{align*}
d(x,y|u,v)&=\sum\limits_{i=1}^{N_1} \alpha_{1i}(x)\beta_{1i}(y)\gamma_{1i}(u)\delta_{1i}(v)\,;
\\
d^*(x,y|u,v)&=\sum\limits_{i=1}^{N_2} \alpha_{2i}(x)\beta_{2i}(y)\gamma_{2i}(u)\delta_{2i}(v)\,;
\\
d_0(x|u,v)&=\sum\limits_{i=1}^{N_3} \alpha_{3i}(x)\gamma_{3i}(u)\delta_{3i}(v)\,;
\\
d_0^*(x|u,v)&=\sum\limits_{i=1}^{N_4} \alpha_{4i}(x)\gamma_{4i}(u)\delta_{4i}(v)\,,
\end{align*}
где $N_1$, $N_2$, $N_3$ и~$N_4$~--- некоторые натуральные чис\-ла,
а $\alpha_{ij}(x)$, $\beta_{ij}(x)$, $\gamma_{ij}(x)$ и~$\delta_{ij}(x)$~--- некоторые
известные функции.
Тогда решение уравнения~\eqref{e2.1n-mei} при фиксированном~$n$
сводится к~решению системы линейных уравнений относительно
$(N_1\hm+N_2\hm+N_3\hm+N_4)$ неизвестных.


\section{Стационарные временные характеристики}

\subsection{Стационарное распределение времени ожидания начала обслуживания}

Для того чтобы найти показатели функционирования СМО,
связанные с~временем пребывания в~системе, нужно прежде
всего найти ПЛС периода занятости (ПЗ) системы.

Обозначим через $u_n(s;x)$, $1 \hm\le n \hm\le r$,  ПЛС
времени до того момента, когда в~системе впервые останется $(n\hm-1)$ заявок,
при условии что на приборе начала обслуживаться заявка
(остаточной) длины~$x$ и~в~системе находится~$n$~заявок.

Учитывая, что по принятому соглашению поступающая в~заполненную
систему заявка сразу теряется, ПЛС  $u_r(s;x)$ удовлетворяет уравнению:
\begin{equation}
\label{t1-mei}
u_r(s;x)=e^{-s x}\,.
\end{equation}
Воспользовавшись свойствами ПЛС, получаем, что $u_{n}(s;x)$ равно:
\begin{itemize}
\item $e^{-s x}$, если до момента времени~$x$
окончания обслуживания заявки на приборе
новая заявка не поступила (с~вероятностью $e^{-\lambda x}$);

\item  $e^{-s t}$, если в~момент $0<t<x$
поступила новая заявка и~обе заявки покинули
систему (с плотностью
вероятностей
$ \lambda e^{-\lambda t}
\int\nolimits_0^\infty d_0(y,x\hm-t)b(y)\,dy$);

\item  $e^{-s t} u_{n}(s;v)$, если в~момент времени
$0\hm<t\hm<x$ поступила новая заявка длины~$y$, одна из двух
заявок (поступившая заявка или
заявка на приборе) покинула систему, а~оставшаяся приняла длину~$v$ и,~значит,
время до того момента, как в~системе останется $(n\hm-1)$ заявок,
равно $u_{n}(s;v)$
(плотность вероятности данного события равна
$ \lambda e^{-\lambda t}
\int\nolimits_0^\infty d(v|y,x-t) b(y)\, dy$);

\item  $e^{-s t} u_{n+1}(s;w) u_{n}(s;v)$, если в~момент
времени $0\hm<t\hm<x$ поступила новая заявка длины~$y$,
обе заявки остаются в~системе (новая встает в~очередь),
причем длина новой заявки становится равной~$v$,
а~на приборе --- $w$ (с~плот\-ностью вероятностей
$
\lambda e^{-\lambda t} \int\nolimits_0^\infty
d(v,w|y,x-t) b(y)\, dy$);

\item  $e^{-s t} u_{n+1}(s;v) u_{n}(s;w)$, если в~момент
времени $0\hm<t\hm<x$ поступила новая заявка длины~$y$,
обе заявки остаются в~системе (новая встает в~очередь),
причем длина новой заявки становится равной~$w$,
а~на приборе --- $v$ (с~плот\-ностью вероятностей
$\lambda e^{-\lambda t} \int\nolimits_0^\infty d^*(v,w|y,x-t)\, b(y)\, dy$).
\end{itemize}


По формуле полной вероятности окончательно получаем:
\begin{multline*}
u_{n}(s;x)=e^{-(s+\lambda) x} + {}\\
{}+\int\limits_0^x \lambda e^{-(\lambda+s) t}\,dt
\int\limits_0^\infty d_0(y,x-t)b(y)\,dy+{}
\\
{}+
\int\limits_0^x \lambda e^{-(\lambda+s) t} \,dt \int\limits_0^\infty
 u_{n}(s;v) \, dv \int\limits_0^\infty d(v|y,x-t)\, b(y)\, dy
+{}\\
{}+
\int\limits_0^x \lambda e^{-(\lambda+s) t} \, dt
\int\limits_0^\infty u_{n+1}(s;w)\,dw
\int\limits_0^\infty u_{n}(s;v) \,dv\times{}
\end{multline*}

\noindent
\begin{multline}
{}\times{}
\int\limits_0^\infty d(v,w|y,x-t) b(y)\, dy
+ 
\int\limits_0^x \lambda e^{-(\lambda+s) t} \,dt\times{}\\
{}\times
\int\limits_0^\infty u_{n+1}(s;v) \,dv
\int\limits_0^\infty  u_{n}(s;w)\, dw\times{}\\
{}\times
\int\limits_0^\infty d^*(v,w|y,x-t) b(y)\, dy\,,\\
 1 \le n \le r-1\,.
 \label{t2-mei}
\end{multline}

Система уравнений~\eqref{t1-mei}--\eqref{t2-mei} решается
рекуррентно, начиная с~$n\hm=r\hm-1$.

Зная значения $u_{n}(s;x)$, можно найти основные стационарные
временн$\acute{\mbox{ы}}$е характеристики заявок.
Пусть в~начальный момент в~системе находится~$n$~заявок, $1 \hm\le n \hm\le r-1$,
на приборе обслуживается заявка длины~$y$ и~в~этот момент
в~систему поступает заявка длины~$x$.
Обозначим через $w_n(s;x,y)$ ПЛС времени ожидания
начала обслуживания этой заявки. В~соответствии 
с~дисциплиной обслуживания имеет место равенство:
\begin{multline*}
w_n(s;x,y)=\int\limits_0^\infty \int\limits_0^\infty d^*(v,w|x,y)\, dv dw
+{}\\
{}+\int\limits_0^\infty\! d_0(v|x,y) \,dv
+ \int\limits_0^\infty \!\int\limits_0^\infty \!u_{n+1}(s;w) d(v,w|x,y)\, dv dw.
\end{multline*}
Заметим, что вероятность того, что поступающая заявка 
длины~$x$ будет потеряна при поступлении в~систему, равна:
\begin{multline*}
\pi(x)= {}\\
{}=\int\limits_0^\infty  \sum\limits_{n=1}^{r-1} p_n(y)
\left( d_0(x,y)+\int\limits_0^\infty d_0^*(w|x,y)\, dw\right)\,dy
+{}\\
{}+ \int\limits_0^\infty p_r(y)\, dy\,.
\end{multline*}
Тогда ПЛС $w(s)$ стационарного распределения времени ожидания начала обслуживания принятой
в~систему заявки определяется формулой:
\begin{multline*}
w(s) =\fr{1}{1-\pi} \biggl (
p_0 + {}\\
{}+\int\limits_0^\infty  \sum\limits_{n=1}^{r-1} p_n(y)\, dy
\int\limits_0^\infty b(x)  w_n(s;x,y)\, dx
\biggl )\,, 
\end{multline*}
где $\pi=\int_0^\infty \pi(x) b(x) \,dx$~--- безусловная вероятность потери заявки.


\subsection{Стационарное распределение времени пребывания заявки в~системе}


Распределение полного времени пребывания заявки в~системе вычисляется
несколько сложнее из-за того, что заявка, попавшая на прибор,
может покидать его и~возвращаться на него обратно,
менять свою длину, а~также уйти из системы недообслуженной.

Остановимся на нахождении следующих характеристик, которые
понадобятся в~дальнейшем:
\begin{itemize}
\item стационарное распределение времени пребывания на приборе заявки,
которая была обслужена до конца (с~учетом возможных смен длин
и~прерываний), при условии что в~момент поступления на прибор
ее длина равнялась~$x$, а~в~очереди было~$n$, $0\hm\le n\hm\le r\hm-1$, других заявок.
Через $V_{1,n}(s;x)$ будем обозначать ПЛС этого распределения;

\item стационарное распределение времени пребывания на приборе заявки,
которая могла быть и~не обслужена до конца (с~учетом возможных смен длин
и~прерываний), при условии что в~момент поступления на прибор
ее длина равнялась~$x$, а~в~очереди было~$n$, $0\hm\le n\hm\le r-1$, других заявок.
Через $V_{2,n}(s;x)$ будем обозначать ПЛС этого распределения.
\end{itemize}

Отметим, что здесь подразумевается, что время пребывания поступившей
на прибор заявки включает все времена,
на которые ее обслуживание было прервано.

Ввиду того что поступающая в~заполненную сис\-те\-му заявка теряется, выпишем
$ V_{1,r-1}(s;x)\hm=e^{-s x}$.
Далее, воспользовавшись свойством ПЛС, находим, что $V_{1,n}(s;x)$  равно:
\begin{itemize}
\item  $e^{-s x}$, если до момента времени~$x$
окончания обслуживания заявки на приборе
новая заявка не поступила (с~вероятностью~$e^{-\lambda x}$);
\item  $e^{-s t}V_{1,n}(s;w)$, если в~момент времени
$0\hm<t\hm<x$ поступила новая заявка длины~$y$,
изменила длину заявки на приборе на~$w$, а~сама\linebreak покинула систему 
(с~плотностью вероятностей~$\lambda e^{-\lambda t}
\int\nolimits_0^\infty d^*_0(w|y,x-t) b(y)\, dy$);

\item $e^{-s t}V_{1,n+1}(s;w)$, если в~момент
времени $0\hm<t\hm<x$ поступила новая заявка длины~$y$, которая
встала на первое место в~очереди, причем новая заявка
получила новую длину~$v$, а~заявка на приборе новую
длину~$w$ (с~плотностью вероятностей
$\lambda e^{-\lambda t} \int\nolimits_0^\infty d(v,w|y,x-t) b(y)\, dy$);

\item  $e^{-s t}u_{n+2}(s;v)V_{1,n}(s;w)$, если в~момент
времени $0\hm<t\hm<x$ поступила новая заявка длины~$y$, которая встала на 
прибор, получив \mbox{новую} длину~$v$, а~заявка с~прибора вытеснена на первое место 
в~очереди и~получила новую длину~$w$ (с~плотностью вероятностей
$\lambda e^{-\lambda t} \int\nolimits_0^\infty
d^*(v,w|y,x-t) b(y)\, dy $).
\end{itemize}

Воспользовавшись снова формулой полной вероятности, получаем, что
уравнение для определения ПЛС $V_{1,n}(s;x)$ имеет следующий вид:
\begin{multline*}
\!\!V_{1,n}(s;x)= e^{-(\lambda+s)x} +\int\limits_0^\infty 
V_{1,n+1}(s;w) f(s;x,w) \, dw
+{}\\
{}+\int\limits_0^\infty V_{1,n}(s;w) g_{n+2}(s;x,w) \, dw\,, \enskip 0\le n\le r-2\,,
\end{multline*}
где
\begin{multline*}
f(s;x,w)= {}\\
{}=\int\limits_0^x \lambda e^{-(\lambda+s) t}\,dt \int\limits_0^\infty  \, dv
\int\limits_0^\infty d(v,w|y,x-t)\, b(y)\, dy\,; 
\end{multline*}

\vspace*{-12pt}

\noindent
\begin{multline*}
g_{n+2}(s;x,w) = \int\limits_0^x \lambda e^{-(\lambda+s) t}\,dt
\int\limits_0^\infty u_{n+2}(s;v)  \, dv\times{}\\
{}\times
\int\limits_0^\infty d^*(v,w|y,x-t)\, b(y)\, dy
+{}\\
{}+
\int\limits_0^x \lambda e^{-(\lambda+s) t}\,dt \int\limits_0^\infty d^*_0(w|y,x-t)\, b(y)\, dy\,.
\end{multline*}

Уравнение для определения $V_{2,n}(s;x)$ получается  аналогичным образом.
Действительно, $V_{2,r-1}(s;x)\hm=e^{-s x}$.
Далее, ПЛС $V_{2,n}(s;x)$ равно:
\begin{itemize}
\item $e^{-s x}$, если до момента времени~$x$
окончания обслуживания заявки на приборе
новая заявка не поступила (с~вероятностью~$e^{-\lambda x}$);

\item  $e^{-s t}$, если в~момент $0\hm<t\hm<x$
поступила новая заявка и~она вместе с~заявкой на приборе покинула
систему (с~плотностью вероятностей~$\lambda e^{-\lambda t}
\int\nolimits_0^\infty d_0(y,x-t)b(y)\,dy$);

\item  $e^{-s t}$, если в~момент времени
$0\hm<t\hm<x$ поступила новая заявка длины~$y$, 
сама встала на прибор, а~заявка с~прибора покинула систему 
(с~плотностью вероятностей~$\lambda e^{-\lambda t}
\int\nolimits_0^\infty d_0(v|y,x-t) b(y)\, dy$);

\item $e^{-s t}V_{2,n}(s;w)$, если в~момент времени
$0\hm<t\hm<x$ поступила новая заявка длины~$y$,
изменила длину заявки на приборе на~$w$, 
а~сама\linebreak покинула систему (с~плотностью вероятностей~$\lambda e^{-\lambda t}
\int\nolimits_0^\infty d^*_0(w|y,x-t) b(y)\, dy$);

\item  $e^{-s t}V_{2,n+1}(s;w)$, если в~момент
времени $0\hm<t\hm<x$ поступила новая заявка длины~$y$, которая
встала на первое место в~очереди, причем новая заявка
получила новую длину~$v$, а~заявка на приборе новую
длину~$w$ (с~плотностью вероятностей~$\lambda e^{-\lambda t}
\int\nolimits_0^\infty d(v,w|y,x-t) b(y)\, dy$);

\item $e^{-s t}u_{n+2}(s;v)V_{2,n}(s;w)$, если в~момент
времени $0\hm<t\hm<x$ поступила новая заявка длины~$y$, которая встала на прибор, 
получив новую длину~$v$, а~заявка с~прибора вытеснена на первое место 
в~очереди и~получила новую длину~$w$ (с~плотностью вероятностей~$\lambda e^{-\lambda t}
\int\nolimits_0^\infty d^*(v,w|y,x-t) b(y)\, dy$).
\end{itemize}

Воспользовавшись снова формулой полной вероятности, получаем, что
уравнение для определения ПЛС $V_{2,n}(s;x)$ имеет следующий вид:

\noindent
\begin{multline*}
\!\!V_{2,n}(s;x)=h(s,x)+ \int\limits_0^\infty V_{2,n+1}(s;w) f(s;x,w) \, dw
+{}\\
{}+ \int\limits_0^\infty V_{2,n}(s;w) g_{n+2}(s;x,w) \, dw\,,\enskip 0\le n\le r-2\,,
\end{multline*}
где

\noindent
\begin{multline*}
h(s,x)= e^{-(\lambda+s) x}+{}\\
{}+\int\limits_0^x \lambda e^{-(\lambda+s) t}\,dt
\int\limits_0^\infty d_0(y,x-t)b(y)\,dy
+ {}\\
{}+
\int\limits_0^x \lambda e^{-(\lambda+s) t}\,dt
\int\limits_0^\infty \, dv
\int\limits_0^\infty d_0(v|y,x-t) b(y)\, dy\,.
\end{multline*}

Решение полученных уравнений осуществляется рекуррентным образом,
начиная с~$n\hm=r\hm-1$.
Естественно, ПЛС безусловных распределений получаются усреднением
 $V_{1,n}(s;x)$ и~$V_{2,n}(s;x)$ по распределению длины заявки $B(x)$.

Наконец, перейдем к~нахождению полного времени пребывания заявки 
в~системе. Будем различать два случая: первый~--- когда заявка не может
уходить из системы недообслуженной; второй~--- когда заявка на
приборе может покинуть систему не обслуженной до конца. В~обоих
случаях, как обычно, полное время пребывания заявки в~системе
складывается из времени ожидания заявкой начала обслуживания 
и~времени пребывания заявки на приборе (которое включает времена
прерываний обслуживания).

В первом случае ПЛС стационарного распределения полного времени
пребывания в~системе поступающей заявки длины~$x$ обозначим через
$V_1(s;x)$, во втором~--- через $V_2(s;x)$.

\pagebreak

Рассмотрим первый случай.

Во-первых, заявка длины~$x$ может с~вероят\-ностью~$p_0$ поступить
в~свободную систему, и~тогда время ее пребывания в~системе будет совпадать с~временем 
ее пребывания на приборе (с~учетом прерываний).

Во-вторых, с~плотностью вероятностей $p_n(y)$ поступающая заявка
длины~$x$ может застать в~сис\-те\-ме $1 \hm\le n \hm\le r-1$ заявок,
причем на приборе будет находиться заявка длины~$y$. 
В~этом случае возможны следующие варианты:
\begin{itemize}
\item либо с~вероятностью $d_0(v|x,y)\hm+d^*(v,w|x,y)$ поступающая заявка
встанет на прибор, причем ее длина станет равной~$v$ и~тогда полное время
ее пребывания в~системе будет совпадать 
с~временем ее пребывания на приборе (с~учетом прерываний);

\item либо с~вероятностью $d(v,w|x,y)$ поступающая заявка станет на первое место 
в~очереди, получит новую длину~$v$, а~заявка на приборе~--- новую длину~$w$; 
при этом время пребывания в~системе поступившей заявки будет равно сумме двух 
времен: времени до того момента, когда в~системе снова станет~$n$~заявок, 
и~времени пребывания на приборе (с~учетом прерываний) заявки длины~$v$.
\end{itemize}

Применяя формулу полной вероятности, приходим к~следующему выражению для ПЛС $V_1(s;x)$
стационарного распределения полного времени пребывания принятой заявки в~систему, в~которой
не допускается уход заявок недообслуженными:
\begin{multline}
\label{eq4-mei}
V_1(s;x)= \fr{1}{1-\pi}
\Biggl (
p_0 V_{1,0}(s;x) +{}
\\ 
{}+
\int\limits_0^\infty  \sum\limits_{n=1}^{r-1} p_n(y)
\Biggl [
\int\limits_0^\infty V_{1,n-1}(s;v)
\Biggl ( d_0(v|x,y)+{}\\
{}+\int\limits_0^\infty d^*(v,w|x,y)\,dw \Biggr ) \,dv
\Biggr ]\, dy
+{}\\
\int\limits_0^\infty  \sum\limits_{n=1}^{r-1} p_n(y)
\Biggl [
\int\limits_0^\infty 
\int\limits_0^\infty u_{n+1}(s;w) \times{}\\
{}\times V_{1,n-1}(s;v) d(v,w|x,y) \,dv  dw
\Biggr ]\,dy
\Biggr )\,.
\end{multline}

Наконец, ПЛС $V_1(s)$ стационарного распределения полного
времени пребывания в~системе заявки произвольной длины
получается усреднением $V_1(s;x)$ по распределению длины заявки $B(x)$ и~равно
$V_1(s)\hm=\int_0^\infty V_1(s;x)b(x)\,dx$.
Выражение для $V_2(s;x)$ получается путем замены
в соответству\-ющих местах формулы~\eqref{eq4-mei} $V_{1,n}(s;x)$
на $V_{2,n}(s;x)$.

\section{Заключение}

В заключение скажем несколько слов об условии существования
стационарного режима.
Для рассмотренной системы общего необходимого и~достаточного
условия его существования выписать не удается.
Оно зависит от конкретных параметров
системы и~в~каждом отдельном случае нуждается в~специальном исследовании.
Конечность среднего времени обслуживания
является только необходимым условием
и,~даже несмотря на присутствие ограничения на размер очереди,
не является достаточным\footnote{Например,
если положить $d(x,y|u,v) = e^{-v} b(x)b(y e^{-v})$,
$d^*(x,y|u,v)\hm=0$, $d(x|u,v)\hm=0$, $d_0(u,v)\hm=0$, $u, v\hm>0$,
то среднее время до того момента, когда в~системе останется
$(r\hm-2)$ заявки, при условии что в~начальный момент в~системе
было $(r\hm-1)$ заявок, без дополнительных
ограничений на функцию $b(x)$ может быть равно бесконечности. При этом,
учитывая пуассоновость входящего потока,
с~ненулевой вероятностью система
переходит в~состояние $(r\hm-2)$
и,~вообще говоря, с~ненулевой вероятностью может успеть выполнить
до прихода очередной заявки
любую находящуюся в~ней работу (при условии ее конечности), т.\,е.\
полностью опустошиться.}.


{\small\frenchspacing
 {%\baselineskip=10.8pt
 \addcontentsline{toc}{section}{References}
 \begin{thebibliography}{9}


\bibitem{n1} 
\Au{Мейханаджян Л.\,А., Милованова~Т.\,А., Печинкин~А.\,В., Разумчик~Р.\,В.}
Стационарные вероятности состояний в~системе обслуживания 
с~инверсионным порядком обслуживания и~обобщенным вероятностным
приоритетом~// Информатика и~её применения, 2014. Т.~8. Вып.~3.
С.~16--26.

\bibitem{n2} %2
\Au{Мейханаджян Л.\,А., Милованова~Т.\,А., Разумчик~Р.\,В.}
Время ожидания в~системе обслуживания с~инверсионным порядком
обслуживания и~обобщенным вероятностным приоритетом~// Информатика 
и~её применения, 2015. Т.~9. Вып.~2. С.~14--22.


%\bibitem{shrage} {\it Schrage L.} A proof of the
%optimality of the shortest remaining processing
%time discipline //
%Oper.\ Res., 1968. Vol.~16. P.~687--690.
%


\bibitem{bsev} %3
\Au{Севастьянов Б.\,А.}
Эргодическая теорема для марковских процессов и~ее приложение 
к~телефонным системам с~отказами~// ТВП, 1957. Т.~2. Вып.~1. С.~106--116.

\bibitem{ppav}  %4
\Au{Бочаров  П.\,П., Печинкин~А.\,В.}
Теория массового обслуживания.~--- М.: РУДН, 1995. 529~с.

\bibitem{n3} %5
\Au{Meykhanadzhyan L., Razumchik~R.}
New scheduling policy for estimation of stationary performance
characteristics in single server queues with inaccurate job size
information~// 30th European Conference on Modelling and Simulation Proceedings.~--- 
Dudweiler, Germany: Digitaldruck Pirrot GmbHP, 2016. P.~710--716.



%\bibitem{aaa1} {\it Нагоненко В.\ А.}
%О характеристиках одной нестандартной системы
%массового обслуживания.~I, II //
%Изв.\ АН СССР. Технич.\ кибернет., 1981.
%№~1. С.~187--195; №~3. С.~91--99.
%
%\bibitem{aaa2} {\it Печинкин А.\ В.} Об одной
%инвариантной системе массового обслуживания //
%Math.\ Operationsforsch.\ und Statist.
%Ser.\ Optimization, 1983. Vol.~14. №~3. S.~433--444.
%
%\bibitem{aaa3} {\it Нагоненко В.\ А., Печинкин А.\ В.}
%О большой загрузке в~системе с~инверсионным
%обслуживанием и~вероятностным приоритетом //
%Изв.\ АН СССР. Технич.\ кибернет., 1982. №~1. С.~86--94.
%
%\bibitem{aaa4} {\it Нагоненко В.\ А., Печинкин А.\ В.}
%О малой загрузке в~системе с~инверсионным порядком
%обслуживания и~вероятностным приоритетом //
%Изв.\ АН СССР. Технич.\ кибернет., 1984. №~6. С.~82--89.
%
%\bibitem{av1}{\it Печинкин А.\ В., Стальченко И.\ В.}
%Система $MAP/G/1/\infty$ с~инверсионным порядком
%обслуживания и~вероятностным приоритетом,
%функционирующая в~дискретном времени //
%Вестник Российского университета дружбы народов.
%Сер.\ Математика. Информатика. Физика, 2010.
%№~2. С.~26--36.
%
%\bibitem{av2}{\it Касконе А., Манзо Р.,
%Печинкин А.\ В., Салерно С.}
%Система $MAP/G/1/\infty$ в~дискретном
%времени с~инверсионной вероятностной дисциплиной
%обслуживания //
%Автоматика и~телемеханика, 2010. №~12. С.~57--69.
%
%\bibitem{av3}{\it Милованова Т.\ А., Печинкин А.\ В.}
%Стационарные характеристики системы обслуживания с
%инверсионным порядком обслуживания, вероятностным
%приоритетом и~гистерезисной политикой //
%Информатика и~ее применения, 2013. Т.~7. Вып.~1. С.~22--36.
%
%
\bibitem{jerri} %6
\Au{Jerri A.}
Introduction to integral equations with
applications.~--- New York, NY, USA: John Wiley \& Sons, 1999. 272~p.
%P. 433.

\bibitem{wh} %7
\Au{Press W.\,H., Teukolsky~S.\,A.,
Vetterling~W.\,T., Flannery~B.\,P.}
Numerical recipes:
The art of scientific computing.~--- 3rd ed.~---
 2007. 1256~p.

\bibitem{adav} %8
\Au{Полянин А.\,Д., Манжиров~А.\,В.}
Справочник по интегральным уравнениям.~---
Бока-Ратон\,--\,Лондон: Chapman \& Hall, CRC Press, 2008. 1108~p.


\end{thebibliography}

 }
 }

\end{multicols}

\vspace*{-6pt}

\hfill{\small\textit{Поступила в~редакцию 19.04.16}}

\vspace*{4pt}

%\newpage

%\vspace*{-24pt}

\hrule

\vspace*{2pt}

\hrule

\vspace*{-2pt}



\def\tit{STATIONARY CHARACTERISTICS OF~THE~FINITE CAPACITY QUEUEING SYSTEM
WITH~INVERSE SERVICE ORDER AND~GENERALIZED
PROBABILISTIC PRIORITY}

\def\titkol{Stationary characteristics of~the~finite capacity queueing system
with~inverse service order and~generalized
probabilistic priority}

\def\aut{L.\,A.~Meykhanadzhyan}

\def\autkol{L.\,A.~Meykhanadzhyan}

\titel{\tit}{\aut}{\autkol}{\titkol}

\vspace*{-9pt}

\noindent
Peoples' Friendship University of Russia,
6~Miklukho-Maklaya Str., 
Moscow 117198, Russian Federation

\def\leftfootline{\small{\textbf{\thepage}
\hfill INFORMATIKA I EE PRIMENENIYA~--- INFORMATICS AND
APPLICATIONS\ \ \ 2016\ \ \ volume~10\ \ \ issue\ 2}
}%
 \def\rightfootline{\small{INFORMATIKA I EE PRIMENENIYA~---
INFORMATICS AND APPLICATIONS\ \ \ 2016\ \ \ volume~10\ \ \ issue\ 2
\hfill \textbf{\thepage}}}

%\vspace*{3pt}


\Abste{Consideration is given to the $M/G/1/(r-1)$
queueing system with LIFO (last in, first out) preemptive
generalized probabilistic priority policy.
It is assumed that customer's service time becomes known
upon its arrival at the system
and at any time instant remaining service times
of all customers present in the system
are available. On arrival of a~customer at a~nonempty system,
its service time is compared to the (remaining) service time of the customer in
service and one of the following events occurs:
both customers leave the system at once,
one of the customers leaves the system (the other
occupies the server), or both customers stay in the system (one occupies the server,
the other~--- one place in the queue). Those customers which stay in the system
acquire new service time according to a~known
distribution, which can depend on their initial service times.
Arriving customers which find the queue full, leave the system and have no influence on it.
Analytical expressions for the computation of the
joint stationary distribution of the number of customers
in the system and the remaining service time of the customer
in the server, of the  busy period and the stationary sojourn time
(in terms of Laplace--Stieltjes transform) are proposed.}


\KWE{queueing system; special discipline; LIFO; probabilistic priority}



\DOI{10.14357/19922264160214}

\vspace*{-16pt}

\Ack

\vspace*{-2pt}

\noindent
The work is supported by the
Russian Foundation for Basic Research (project 15-07-03007).


  \vspace*{-1pt}

  \begin{multicols}{2}
  

  

\renewcommand{\bibname}{\protect\rmfamily References}
%\renewcommand{\bibname}{\large\protect\rm References}



{\small\frenchspacing
 {%\baselineskip=10.8pt
 \addcontentsline{toc}{section}{References}
 \begin{thebibliography}{9}

\vspace*{-2pt}
\bibitem{n1-1} 
\Aue{Meykhanadzhyan, L.\,A., T.\,A.~Milovanova, A.\,V.~Pechinkin, 
and R.\,V.~Ra\-zum\-chik}. 2014.
Statsionarnye veroyatnosti so\-sto\-yaniy v~sisteme obslu\-zhi\-va\-niya 
s~inversionnym po\-ryad\-kom
ob\-slu\-zhi\-va\-niya i~obob\-shchen\-nym veroyatnostnym pri\-o\-ri\-te\-tom
[Stationary distribution in a~queueing system with inverse service order and
generalized probabilistic priority].
\textit{Informatika i~ee Primeneniya}~--- \textit{Inform.Appl.}
8(3):16--26.

\bibitem{n2-1} 
\Aue{Meykhanadzhyan, L.\,A., T.\,A.~Milovanova, and R.\,V.~Ra\-zum\-chik}. 2015.
Vremya ozhidaniya v~sis\-te\-me ob\-slu\-zhi\-va\-niya s~inversionnym poryadkom obsluzhivaniya 
i~obobshchennym veroyatnostnym prioritetom
[Stationary\linebreak waiting time in a~queueing system with inverse service order and
generalized probabilistic priority].
\textit{Informatika i~ee Primeneniya}~--- \textit{Inform.Appl.}
9(2):14--22.

\bibitem{bsev-1} %3
\Aue{Sevastyanov, B.\,A.} 1957.
Ergodicheskaya teorema dlya markovskikh protsessov i~ee prilozhenie k~telefonnym 
sistemam s~otkazami
[An ergodic theorem for markov processes and its application to telephone systems with refusals].
\textit{Teor. Veroyatnost. i Primenen.} 
[Probability Theory and Its Applications] 2(1):106--116.



\bibitem{ppav-1} %4
\Aue{Bocharov,  P.\,P., and A.\,V.~Pechinkin}. 1995.
\textit{Teoriya massovogo obsluzhivaniya} [Queueing theory].
Moscow: RUDN. 529~p.

\bibitem{n3-1} %5
\Aue{Meykhanadzhyan, L., and R.~Razumchik}. 2016.
New scheduling policy for estimation of stationary performance
characteristics in single server queues with inaccurate job size
information. \textit{30th European Conference on Modelling and Simulation Proceedings}.
Dudweiler, Germany: Digitaldruck Pirrot GmbHP. 710--716.



\bibitem{jerri-1} %6
\Aue{Jerri, A.} 1999.
\textit{Introduction to integral equations with applications}.
New York, NY: John Wiley \& Sons. 272~p.



\bibitem{wh-1} %7
\Aue{Press, W.\,H., S.\,A.~Teukolsky, W.\,T.~Vetterling,
and B.\,P.~Flannery}. 2007.
\textit{Numerical recipes:  
The art of Scientific computing}. 3rd ed. 1256~p.

\bibitem{adav-1} %8
\Aue{Polyanin, A.\,D., and A.\,V.~Manzhirov}. 2008.
\textit{Handbook of integral equations}.
Boca Raton\,--\,London: Chapman \& Hall,
CRC Press. 1108~p.
\end{thebibliography}

 }
 }

\end{multicols}

\vspace*{-7pt}

\hfill{\small\textit{Received April 19, 2016}}

\vspace*{-17pt}
   

\Contrl

\vspace*{-2pt}

\noindent
\textbf{Meykhanadzhyan Lusine A.} (b.\ 1990)~---
PhD student, Peoples' Friendship University of Russia, 6~Miklukho-Maklaya Str., 
Moscow 117198, Russian Federation; lameykhanadzhyan@gmail.com

 
\label{end\stat}


\renewcommand{\bibname}{\protect\rm Литература}     %10
%\newcommand {\ff}{{\mathcal F}}
\newcommand {\ebd}{\triangleq}
\newcommand{\me}[2]{\mathbf{E}_{ #1 }\left\{ \mathop{#2} \right\} }



\def\stat{borisov}

\def\tit{ФИЛЬТРАЦИЯ СОСТОЯНИЙ МАРКОВСКИХ СКАЧКООБРАЗНЫХ ПРОЦЕССОВ 
ПО~ДИСКРЕТИЗОВАННЫМ НАБЛЮДЕНИЯМ$^*$}

\def\titkol{Фильтрация состояний марковских скачкообразных процессов 
по~дискретизованным наблюдениям}

\def\aut{А.\,В.~Борисов$^1$}

\def\autkol{А.\,В.~Борисов}

\titel{\tit}{\aut}{\autkol}{\titkol}

\index{Борисов А.\,В.}
\index{Borisov A.\,A.}




{\renewcommand{\thefootnote}{\fnsymbol{footnote}} \footnotetext[1]
{Работа выполнена при частичной поддержке РФФИ (проект 16-07-00677).}}


\renewcommand{\thefootnote}{\arabic{footnote}}
\footnotetext[1]{Институт проблем информатики Федерального исследовательского центра <<Информатика 
и~управление>> Российской академии наук,
\mbox{aborisov@frccsc.ru}}

%\vspace*{8pt}



\Abst{Статья посвящена решению задачи оптимальной 
фильтрации состояний однородного марковского скачкообразного процесса (МСП). 
Наблюдения представляют собой приращения случайных процессов~--- интегральных 
преобразований состояний, зашумленные винеровскими процессами, интенсивность 
которых также зависит от оцениваемого состояния. Оптимальная оценка в~моменты 
получения нового наблюдения вычисляется как функция предыдущей оценки и~новых 
наблюдений, а~между моментами наблюдений~--- простейшим прогнозом в~силу системы 
уравнений Колмогорова. Рекуррентная формула пересчета ресурсозатратна, так как 
содержит  интегралы~--- мас\-штаб\-но-сдви\-го\-вые смеси многомерных гауссиан, 
где в~качестве смешивающих выступают распределения времени пребывания 
состояния в~каждом из возможных значений. Предложены более простые аппроксимации, 
основанные на предположении об ограниченности числа скачков состояния за время между 
наблюдениями. Получены универсальные локальная и~глобальная характеристики точности 
аппроксимаций, зависящие от па\-ра\-мет\-ров оцениваемого процесса, величины 
временн$\acute{\mbox{о}}$го шага  между наблюдениями и~максимального числа учитываемых скачков.}

\KW{марковский скачкообразный процесс; оптимальная фильтрация; мультипликативные 
шумы в~наблюдениях; стохастическое дифференциальное уравнение; численная аппроксимация}

\DOI{10.14357/19922264180316}
  
%\vspace*{4pt}


\vskip 10pt plus 9pt minus 6pt

\thispagestyle{headings}

\begin{multicols}{2}

\label{st\stat}



 \section{Введение}
 
 Фильтр Вонэма~\cite{Won_65}~--- один из редких удачных случаев, когда 
 оценка оптимальной фильтрации состо\-яния стохастической системы наблюдения 
 выражается в~виде решения некоторой замк\-ну\-той\linebreak конечномерной сис\-те\-мы 
 стохастических дифференциальных уравнений. 
 
 Алгоритм данного фильт\-ра 
 позволяет вычислить оценку фильт\-ра\-ции со\-сто\-яния \textit{марковского скачкообразного 
 процесса} с~\mbox{конечным} множеством состояний по наблюдениям в~присутствии 
 аддитивных винеровских шумов. Теоретически оптимальная оценка со\-сто\-яния~--- 
 его условное распределение в~текущий момент времени~--- 
 обладает очевидными свойствами неотрицательности и~нормировки. 
 При чис\-лен\-ной реализации данного фильтра классическим методом 
 Эй\-ле\-ра--Ма\-ру\-ямы~\cite{KP_92} данные свойства могут не сохраняться и~процедура 
 вы\-чис\-ле\-ния становится неустойчивой.  В~связи с~этим обстоятельством разрабатывались 
 другие алгоритмы чис\-лен\-но\-го решения уравнения фильтра Вонэма, обладающие 
 требуемыми свойствами устойчивости (см.~\cite{YZL_04, PR_10} и~библиографию в~них). 
 В~час\-ти этих работ доказана лишь слабая сходимость пред\-ла\-га\-емых аппроксимационных 
 схем к~оценке фильт\-ра Вонэма, в~то время как ка\-кая-ли\-бо 
 характеризация точ\-ности этих приближений отсутствует.
 
 В~\cite{B_18} было представлено распространение фильт\-ра Вонэма на случай 
 наблюдений с~мультипликативными шумами. При этом уравнение обобщенного 
 фильт\-ра содержит в~правой части квадратическую характеристику шумов в~наблюдениях. 
 Данный процесс на практике никогда не наблюдается непосредственно, а~является лишь 
 некоторым нелинейным интегральным преобразованием наблюдений. Очевидно, что 
 имеющиеся в~настоящий момент времени алгоритмы приближенного вычисления оценки 
 фильтрации Вонэма для данной системы не подходят. 
 
 Целью предлагаемой работы является ис\-поль\-зование результатов оптимальной 
 фильтрации со\-стояний сис\-тем с~дискретным временем для аппроксимации решения 
 аналогичной задачи для\linebreak стохастических дифференциальных сис\-тем. 
 
 Статья организована следующим образом. Раздел~2 содержит формальную постановку 
 задачи фильт\-ра\-ции со\-сто\-яний однородного МСП с~конечным множеством со\-сто\-яний 
 по наблюдениям, полученным путем временн$\acute{\mbox{о}}$й дискретизации процессов с~непрерывным 
 временем~--- интегральных преобразований со\-сто\-яния сис\-те\-мы в~присутствии 
 мультипликативных винеровских шумов.\linebreak
  В~разд.~3 пред\-став\-ле\-но решение поставленной 
 задачи фильт\-ра\-ции: пересчет оценок со\-сто\-яний в~момент получения новых 
 дискретизованных наблюдений выполняется в~соответствии с~некоторыми\linebreak 
 рекуррентными интегральными соотношениями, в~то время как между 
 моментами наблюдений оценка корректируется в~соответствии с~прогнозом в~силу 
 сис\-те\-мы уравнений Колмогорова. Вы\-чис\-ли\-тель\-ная слож\-ность 
 упомянутых выше интегральных\linebreak 
 соотношений связана с~тем, что в~расчет принимается воз\-мож\-ность того, что между 
 моментами наблюдений оцениваемое со\-сто\-яние может совершить произвольное чис\-ло 
 скачков. В~разд.~4 пред\-став\-лен более простой алгоритм приближенного вы\-чис\-ле\-ния 
 оценки фильт\-ра\-ции, основанный на ограничении возможного числа учитываемых скачков 
 МСП. Доказана тео\-ре\-ма, опре\-де\-ля\-ющая как\linebreak
  локальную (одношаговую), так и~глобальную 
 (многошаговую) характеристики точ\-ности предложенного при\-бли\-же\-ния~--- 
 $\ell_1$-нор\-мы ошибки аппроксимации. Полученные характеристики являются\linebreak 
 универсальными, т.\,е.\ не асимптотическими по шагу дискретизации, и~зависят от характеристик 
 самого МСП, %\linebreak
  шага временн$\acute{\mbox{о}}$й дискретизации и~чис\-ла
  скачков со\-сто\-яния, учи\-ты\-ва\-емых 
 на шаге. Об\-суж\-де\-ние результатов и~заключительные комментарии пред\-став\-ле\-ны 
 в~разд.~5.
 
 \section{Постановка задачи фильтрации}
 
 На полном вероятностном пространстве с~фильт\-ра\-цией 
 $(\Omega,\mathcal{F},\mathcal{P},\{\mathcal{F}_{t}\}_{t \geqslant 0})$ рассматривается система наблюдений
\begin{equation}
 \left.
 \begin{array}{rl}
 \displaystyle X_t &=X_0 +  \displaystyle
 \int\limits_0^t \Lambda^{\top}X_{s}\,ds + \mu_s\,;  \\[6pt]
 \displaystyle Y_k &=  \displaystyle\int\limits_{t_{k-1}}^{t_k}fX_s\,ds+
 \int\limits_{t_{k-1}}^{t_k} 
 \sum\limits_{n=1}^NX_s^ng_n \,dW_s, \\[6pt]
 &\hspace*{10mm}\{t_k\}_{k \geqslant 0}: \; 0 = t_0 < t_1 < t_2\cdots,
 \end{array}
 \right\}
 \label{eq:obsys_1}
 \end{equation}
 где
  \begin{itemize}
  \item
  $X_t \ebd \mathrm{col}\left(X_t^1,\ldots,X_t^N\right) \hm\in \mathbb{S}^N$~--- 
  ненаблюда\-емое состояние системы, являющееся однородным МСП с~конечным 
  множеством состояний $ \mathbb{S}^N \ebd$\linebreak $\ebd \{e_1,\ldots,e_N\}$ ($\mathbb{S}^N$~--- 
  множество единичных векторов евклидова пространства~$\mathbb{R}^N$), 
  матрицей интенсивностей переходов~$\Lambda$ и~начальным распределением~$\pi$;
  \item
  $\mu_t \ebd \mathrm{col}\left(
  \mu_t^1,\ldots,\mu_t^N\right)\hm\in \mathbb{R}^N$~--- 
  ${\mathcal{F}}_t$-со\-гла\-со\-ван\-ный мартингал;
  \item
  $\{Y_k\}_{k \in \mathbb{N}}:\;  Y_k \ebd \mathrm{col}\left(Y_k^1,\ldots,Y_k^M\right) 
  \hm\in \mathbb{R}^M$~--- последовательность дискретизованных наблюдений, 
  доступных в~известные неслучайные  моменты времени~$\{t_k\}_{k \in \mathbb{N}}$,
в~которых $W_t \ebd$\linebreak $\ebd \mathrm{col}\left(W_t^1,\ldots,W_t^M\right) \hm\in \mathbb{R}^M$
 является ${\mathcal{F}}_t$-со\-гла\-со\-ван\-ным стандартным винеровским процессом, 
 определяющим шумы в~наблюдениях,\linebreak  $f$~--- $(M \times N)$-мер\-ная 
 мат\-ри\-ца плана наблюдений, а~набор мат\-риц~$\{g_n\}_{n=\overline{1,N}}$ 
 характеризует интенсивности шумов в~зависимости от текущего состояния~$X_t$.
  \end{itemize}
  
  Введем также в~рассмотрение неубывающие семейства $\sigma$-ал\-гебр 
  $\mathcal{O}_k \ebd \sigma\{ Y_{\ell}: \; 1 \hm\leqslant \ell \hm\leqslant k\}$ 
  и~$\mathcal{O}_t \ebd  \mathcal{O}_{k(t)}$, где 
  $k(t) \ebd \sum\nolimits_{j \in \mathbb{N}}\mathbf{I}(t-t_{j})$; 
  $\mathcal{O}_0 \ebd \{\varnothing,\; \Omega\}$.
  
   \textit{Задача оптимальной фильтрации состояния~$X$ по наблюдениям~$Y$} 
   заключается в~нахождении \textit{условного математического ожидания} (УМО)
  \begin{equation*}
  \widehat{X}_t \ebd {\sf E}\left\{X_t|\mathcal{O}_{t} \right\}\,.
 % \label{eq:fest_1}
  \end{equation*}
  
  Относительно системы~(\ref{eq:obsys_1})  сделаны следующие предположения:
   \begin{itemize}
 \item[(а)]
 ${\mathcal{F}}_t \equiv {\mathcal{F}}_{t}^X \bigvee 
 {\mathcal{F}}_{t}^W $ для любого $t \hm\geqslant 0$;
 \item[(б)]
 шумы в~наблюдениях равномерно невырожденные, т.\,е.\
  $g_ng_n^{\top} \hm\geqslant \alpha I \hm> 0$ для всех $n\hm=\overline{1,N}$ 
  и~некоторого $\alpha\hm>0$.
% \item
 % Верно неравенство
  %\begin{equation}
  %\min_{1\leqslant k \leqslant N}|\lambda_{kk}| > 0.
  %\label{eq:ineq_0}
  % \end{equation}
 %\item
 %Для любого $t \geqslant 0$ все компоненты вектора $p_t \ebd \me{}{X_t}$ строго %положительны. 
 \end{itemize} 

 \section{Уравнения оптимального фильтра} 
 
 Для получения уравнений оптимального фильт\-ра воспользуемся подходом, 
 применяемым для решения аналогичной задачи в~стохастических сис\-те\-мах 
 наблюдения с~дискретным временем~\cite{BSh_85}. 
 Воспользу\-ем\-ся методом математической индукции. 
 
 При $r=0$ 
 \begin{equation}
 \widehat{X}_{t_0}={\sf E}\{X_0|\mathcal{O}_0\}={\sf E}\{X_0\}=\pi\,.
 \label{eq:in_cond}
 \end{equation} 
 
 Пусть для некоторого $ r \hm\geqslant 0$ известна оценка оптимальной 
 фильтрации~$\widehat{X}_{t_r} \hm= {\sf E}{X_{t_r} |\mathcal{O}_r}$. 
 Определим оценку оптимальной фильтрации~$\widehat{X}_{t} $ для $t\hm \in (t_r,t_{r+1}]$. 
 
 Для произвольного момента $t \hm\in (t_r,t_{r+1})$ в~силу мартингального 
 разложения МСП~$X_t$ и~свойств УМО верна следующая цепочка равенств:
 \begin{multline*}
 \widehat{X}_{t} = {\sf E}\left\{X_t | \mathcal{O}_r\right\}={}\\
 {}=
 {\sf E}\left\{{\cal P}^{\top}(t_r,t)X_{t_r}+
 \int\limits_{t_r}^t{\cal P}^{\top}(t_r,s)\,dM_s\big\vert \mathcal{O}_r\right\} = {}
\end{multline*}

\noindent
   \begin{multline}
 \hspace*{-11.66pt}{}=\mathcal{P}^{\top}(t_r,t)\widehat{X}_{t_r} + {\sf E}\hspace*{-2pt}
 \left\{{\sf E}\hspace*{-2pt}\left\{\int\limits_{t_r}^t\hspace*{-2pt}\mathcal{P}^{\top}(t_r,s)\,dM_s |
 {\mathcal{F}}_{t_r}\right\}\!\big\vert 
 \mathcal{O}_r\!\right\} ={}\hspace*{-4.24124pt}\\
 {}=
  \mathcal{P}^{\top}(t_r,t)\widehat{X}_{t_r}\,,
 \label{eq:bw_obs}
 \end{multline}
 где $\mathcal{P}(s,t)$ $(s \hm\leqslant t)$~--- матрица переходной ве\-ро\-ят\-ности МСП 
 на промежутке $[s,t]$, являющаяся решением сис\-те\-мы дифференциальных 
 уравнений Колмогорова
 \begin{equation*}
 \mathcal{P}'_t(s,t) = \mathcal{P}(s,t) \Lambda, \enskip t > s, \enskip \mathcal{P}(s,s) = I.
 \end{equation*}
 В случае однородного МСП $\mathcal{P}(s,t) \hm= e^{(t-s)\Lambda}$.
 
 Далее необходимо определить совместное распределение $(X_{t_{r+1}},Y_{r+1})$ 
 относительно~$ \mathcal{O}_r$. Из модели наблюдений следует, что 
 распределение~$Y_{r+1}$ относительно 
 $\sigma$-ал\-геб\-ры~$\mathcal{F}^X_{t_{r+1}} \vee \mathcal{O}_r$~---
 гауссовское с~параметрами 
 \begin{align*}
{\sf E}\left\{Y_{r+1}|{\mathcal{F}}^X_{t_{r+1}}\right\}& = f \tau_{r+1}\,; \\[6pt]
 \mathrm{cov} \left(Y_{r+1},Y_{r+1}|{\mathcal{F}}^X_{t_{r+1}}\right) &= 
 \displaystyle\sum\limits_{n=1}^N \tau_{r+1}^n g_ng_n^{\top}\,,
% \label{eq:occup_1}
 \end{align*}
 где $\tau_{r+1} \hm= \tau_{r+1}(X(\omega))=
 \mathrm{col}\left(\tau_{r+1}^1,\ldots,\tau_{r+1}^N\right) \ebd$\linebreak
 $\ebd 
 \int\nolimits_{t_r}^{t_{r+1}}X_s\,ds$~--- случайный вектор, $n$-я 
 компонента которого равна времени пребывания процесса~$X$ в~со\-сто\-янии~$e_n$ 
 на  интервале времени $[t_r, t_{r+1}]$. 
 Обозначим через $\mathcal{D}_{r+1} \ebd \{u=\mathrm{col}\,(u^1,\ldots,u^N):\; 
 u_m \hm\geqslant 0,\; \sum\nolimits_{m=1}^Mu_m\hm= t_{r+1}-t_r\}$ $(M-1)$-мер\-ный 
 симплекс в~пространстве~$\mathbb{R}^M$, являющийся носителем распределения 
 вектора~$\tau_{r+1}$. Пусть $\rho^{k,\ell}_{r+1}(du)$~--- 
 распределение вектора $\tau_{r+1} X_{t_{r+1}}^{\ell}$ при условии $X_{t_r}\hm=e_k$, 
 т.\,е.\ 
 для любого $\mathcal{A} \hm\in \mathcal{B}(\mathbb{R}^M)$ верно тождество:
\begin{multline*}
 \mathbf{P}\left\{\omega: \; X_{t_{r+1}}(\omega)=e_{\ell},\right.\\
 \left. 
 \tau_{r+1}(X(\omega)) \in \mathcal{A}\;|\;X_{t_r}=e_k\right\} \equiv
   \rho^{k,\ell}_{r+1}(\mathcal{A})\,.
\end{multline*}
 
Обозначим через
\begin{multline*}
 \mathcal{N}(y,m,K) \ebd (2\pi)^{-M/2} \mathrm{ det}^{-1/2} K\times{}\\
 {}\times\exp
 \left\{ -\fr{1}{2}\left(y-m\right)^{\top}K^{-1}(y-m)\right\}
\end{multline*}
 $M$-мер\-ную плот\-ность гауссовского распределения с~математическим 
 ожиданием~$m$ и~ковариационной матрицей~$K$.
 
 Из марковского свойства  $\{X_{t_{r}},Y_{r})\}_{r \geqslant 0}$ 
 относительно~${\mathcal{F}}_{t_{r}}$~\cite{ZhSh_95} и~теоремы Фубини следует, что 
 для любого  множества $\mathcal{A} \hm\in \mathcal{B}(\mathbb{R}^M)$ 
 верна следующая цепочка равенств:
 \begin{multline*}
 {\sf E}\left\{X_{t_{r+1}}\mathbf{I}_{\mathcal{A}}
 \left(Y_{r+1}\right)\big|\mathcal{O}_r\right\}={}\\
 {}=
{\sf E}\left\{{\sf E}\left\{X_{t_{r+1}}\mathbf{I}_{\mathcal{A}}
\left(Y_{r+1}\right)\big|
\mathcal{F}^X_{t_{r+1}} \vee \mathcal{O}_r\right\}
 \big|\mathcal{O}_r\right\} = {}
\end{multline*}

\noindent
\begin{multline*}
 %{}=
% {\sf E}\left\{{\sf E}\left\{X_{t_{r+1}}\mathbf{I}_{\mathcal{A}}
% \left(Y_{r+1}\right)\vert X_{t_r}\right\}
% \vert\mathcal{O}_r\right\} = {}\\
% {}=
%{\sf E}\left\{\sum\limits_{k=1}^N {\sf E}\left\{X_{t_{r+1}}\mathbf{I}_{\mathcal{A}}
%\left(Y_{r+1}\right)  \big| X_{t_r}=e_k\right\}X_{t_r}^k
% \big|\mathcal{O}_r\right\} = {}\\ 
% {}=
% \sum\limits_{k=1}^N{\sf E}
% \left\{X_{t_{r+1}}\mathbf{I}_{\mathcal{A}}\left(Y_{r+1}\right)\bigl| X_{t_r}=e_k\right\} 
% \widehat{X}_{t_r}^k ={}\\
% {}=\!
% \sum\limits_{k=1}^N{\sf E}
% \left\{{\sf E}\left\{X_{t_{r+1}}\mathbf{I}_{\mathcal{A}}
% \left(Y_{r+1}\right)\!\bigl| {\mathcal{F}}_{t_{r+1}}\right\}\!\bigl| 
% X_{t_r}\!=e_k\right\} \widehat{X}_{t_r}^k ={}\\
% {}=
% \sum\limits_{k=1}^N {\sf E}\left\{
% \vphantom{\int\limits_A\left(\sum\limits_{p=1}^N\right)}
% X_{t_{r+1}} \times{}\right.\\
% {}\times\int\limits_{\mathcal{A}}  
% \mathcal{N}\left(y,f \tau_{r+1}(X),\sum\limits_{p=1}^N \tau_{r+1}^p(X) g_pg_p^{\top}\right)dy
% \Biggl| X_{t_r}={}\\
%\left. {}=e_k
% \vphantom{\int\limits_A\left(\sum\limits_{p=1}^N\right)}
%\right\} \widehat{X}_{t_r}^k = 
% \sum\limits_{k=1}^N \int\limits_{\mathcal{A}}{\sf E}\left\{ 
% \vphantom{\sum\limits_{p=1}^N}
% X_{t_{r+1}} \times{}\right.\\
% {}\times\mathcal{N}\left(y,f \tau_{r+1}(X),\sum\limits_{p=1}^N \tau_{r+1}^p(X) 
% g_p g_p^{\top}\right)
% \Biggl| X_{t_r}={}\\
%\left. {}=e_k
%\vphantom{\sum\limits^N_{p=1}}
%\right\} \widehat{X}_{t_r}^k\, dy
 %={}\\
 {}=
 \sum\limits_{\ell=1}^N e_{\ell} \int\limits_{\mathcal{A}} 
 \left[ \sum\limits_{k=1}^N 
 \int\limits_{\mathcal{D}_{r+1}} 
 \mathcal{N}\left(y,f u,\sum_{p=1}^N u^p g_pg_p^{\top}\right)\times{}\right.\\
\left. {}\times
 \rho^{k,\ell}_{r+1}(du)\widehat{X}_{t_r}^k
 \vphantom{\int\limits_A\sum\limits_{p=1}^N}
 \right] 
 dy,
 \end{multline*}
 из чего следует, что интегранд в~квадратных скобках в~последнем выражении 
 определяет искомое совместное распределение $(X_{t_{r+1}},Y_{r+1})$ 
 относительно~$ \mathcal{O}_r$. Оценка~$\widehat{X}_{t_{r+1}}$ покомпонентно 
 определяется~\cite{BSh_85} с~помощью обобщенного варианта формулы Байеса:
 \begin{multline}
 \widehat{X}_{t_{r+1}}^j = {}\\
 \hspace*{-1mm}{}=
 \fr{\int\nolimits_{\mathcal{D}_{r+1}}\hspace*{-6mm} 
 \mathcal{N}\left(Y_{r+1},f u,\sum\nolimits_{p=1}^N \hspace*{-2mm}
 u^p g_pg_p^{\top}\!\right)\hspace*{-1mm}
 \sum\nolimits_{k=1}^N \hspace*{-2mm}
 \widehat{X}_{t_r}^k
 \rho^{k,j}_{r+1}(du)
 }
 { \int\nolimits_{\mathcal{D}_{r+1}} \hspace*{-6mm}
 \mathcal{N}\left(Y_{r+1},f v,\sum\nolimits_{q=1}^N \hspace*{-2mm}
 v^q g_qg_q^{\top}\!\right)\hspace*{-1mm}
 \sum\nolimits_{i,\ell=1}^N \hspace*{-2mm}
 \widehat{X}_{t_r}^i
 \rho^{i,\ell}_{r+1}(dv)
  },  \\ 
  j = \overline{1,N}\,.
 \label{eq:filt_1}
 \end{multline}
 Таким образом, доказана следующая
 
 %\smallskip
 
 \noindent
 \textbf{Лемма~1.}
\textit{Если для системы наблюдения}~(\ref{eq:obsys_1}) 
\textit{верны условия~(а) и~(б), то оценка~$\widehat{X}_t$ оптимальной фильтрации 
определяется формулой}~(\ref{eq:in_cond}) 
\textit{при $t\hm=0$, рекуррентным соотношением}~(\ref{eq:filt_1})~---
\textit{в~моменты~$t_{r+1}$ получения наблюдений~$Y_{r+1}$ 
и~формулой}~(\ref{eq:bw_obs})~--- 
\textit{в~промежутках времени между моментами получения наблюдений}.


\smallskip
 

 
 Несмотря на компактную запись~(\ref{eq:filt_1}), их прямая численная реализация 
 ресурсозатратна. Во-пер\-вых, в~(\ref{eq:filt_1}) требуется вычислять 
 распределения мас\-штаб\-но-сдви\-го\-вых смесей многомерных нормальных 
 распределений, что является трудоемкой\linebreak процедурой. Во-вто\-рых, 
 распределения~$\rho^{k,j}_{r+1}$ вре-\linebreak мени пребывания представляют собой 
 сумму\linebreak бесконечного ряда, слагаемые которого вычис\-ляются с~помощью 
 некоторой рекуррентной про\-це\-дуры~\cite{S_00}. В-третьих, 
 распределения~$\rho^{k,j}_{r+1}$ не являются абсолютно непрерывными 
 относительно меры Ле\-бега.
 { %\looseness=1
 
 }
 
 Следующий раздел посвящен численной аппроксимации~(\ref{eq:filt_1}) и~исследованию 
 ее точностных характеристик.
 
 \section{Приближенное вычисление оценки фильтрации}
 
 Без ограничения общности будем считать, что сетка~$\{t_r\}_{r \geqslant 0}$ 
 является равномерной с~шагом~$\Delta$, т.\,е.\ $t_r \hm= r \Delta$ 
 и~$\mathcal{D}_r \hm\equiv \mathcal{D}$.
 Обозначим через~$N_{r+1}$ об-\linebreak\vspace*{-12pt}
 
 \pagebreak
 
 \noindent
 щее число скачков процесса~$X_t$, имевших место 
 на промежутке $(t_r,t_{r+1}]$. Тогда из формулы полной вероятности следует, 
 что~(\ref{eq:filt_1}) представима в~виде:
 \begin{multline}
 \widehat{X}_{t_{r+1}}^j =  \left(
 \int\limits_{\mathcal{D}} 
 \mathcal{N}\left(Y_{r+1},f u,\sum\limits_{p=1}^N u^p g_pg_p^{\top}\right)\times{}\right.\\
\left. {}\times
 \sum\limits_{h=0}^{\infty}\sum\limits_{k=1}^N \widehat{X}_{t_r}^k
 \rho^{k,j,h}_{r+1}(du)
 \right)\Bigg/ \\
 \left(
 \vphantom{\sum\limits_{m=0}^{\infty}
 \sum\limits_{i,\ell=1}^N \widehat{X}_{t_r}^i
 \rho^{i,\ell,m}_{r+1}(dv)}
 \int\limits_{\mathcal{D}} 
 \mathcal{N}\left(Y_{r+1},f v,\sum\limits_{q=1}^N v^q g_qg_q^{\top}\right)\times{}\right.\\
\left.{}\times \sum\limits_{m=0}^{\infty}
 \sum\limits_{i,\ell=1}^N \widehat{X}_{t_r}^i
 \rho^{i,\ell,m}_{r+1}(dv)
 \right)
  \,, \enskip j = \overline{1,N}\,,
  \label{eq:filt_1_1}
 \end{multline}
 где 
 $ \rho^{k,j,h}_{r+1}(du)$~--- распределение вектора 
 $\tau_{r+1}X_{t_{r+1}}^{j}\mathbf{I}_{\{h\}}(N_{r+1})$ при 
 условии $X_{t_r}\hm=e_k$, т.\,е.\ 
 для любого $\mathcal{A} \hm\in \mathcal{B}(\mathbb{R}^M)$ верно тождество
\begin{multline*}
 \mathbf{P}\left\{\omega: \; X_{t_{r+1}}(\omega)=e_{j}, \; N_{r+1} = h,\right.\\ 
\left. \tau_{r+1}(X(\omega)) \in \mathcal{A}\;|\;X_{t_r}=e_k\right\} \equiv
  \rho^{k,j,h}_{r+1}(\mathcal{A}).
\end{multline*}
В качестве аппроксимации оценок можно использовать  
 $\overline{X}_{t_{r+1}}^n \ebd 
 \mathrm{col}\,(\overline{X}_{t_{r+1}}^{n,1},\ldots,\overline{X}_{t_{r+1}}^{n,N})$, 
 полученные из~(\ref{eq:filt_1_1}) путем урезания сумм ряда в~числителе и~знаменателе:
 
 \noindent
 \begin{multline}
 \overline{X}_{t_{r+1}}^{n,j} = 
 \left(
 \int\limits_{\mathcal{D}} 
 \mathcal{N}\left(Y_{r+1},f u,\sum\limits_{p=1}^N u^p g_pg_p^{\top}\right)\times{}\right.\\[-1pt]
\left.{}\times \sum\limits_{h=0}^{n}\sum\limits_{k=1}^N \overline{X}_{t_r}^k
 \rho^{k,j,h}_{r+1}(du)
 \right)\Bigg/ \\[-1pt]
 \left(
 \int\limits_{\mathcal{D}} 
 \mathcal{N}\left(Y_{r+1},f v,\sum\limits_{q=1}^N v^q g_qg_q^{\top}\right)\times{}\right.\\[-1pt]
\left. {}\times
 \sum\limits_{m=0}^{n}
 \sum\limits_{i,\ell=1}^N \overline{X}_{t_r}^i
 \rho^{i,\ell,m}_{r+1}(dv)
  \right)\,, \enskip
   j = \overline{1,N}.
  \label{eq:filt_2}
 \end{multline}
 Ниже по формуле полной вероятности получены интегралы из~(\ref{eq:filt_2}) для 
 $h\hm=0,1,2$:
 
\vspace*{-3pt}

 \noindent
  \begin{multline*}
 \int\limits_{\mathcal{D}}  \mathcal{N}
 \left(Y_{r+1},f u,\sum\limits_{p=1}^N u^p g_pg_p^{\top}\right) 
 \rho^{k,j,0}_{r+1}(du) = {}\\[-1pt]
 {}=
 \delta_{kj}\mathcal{N}\left(Y_{r+1},\Delta f^j,\Delta g_jg_j^{\top}\right)
 e^{\lambda_{jj}\Delta};
 %\label{eq:h0}
\\[-1pt]
 \int\limits_{\mathcal{D}}  \mathcal{N}\left(
 Y_{r+1},f u,\sum\limits_{p=1}^N u^p g_pg_p^{\top}\right) 
 \rho^{k,j,1}_{r+1}(du) ={} 
 \end{multline*}
 
 \noindent
 \begin{multline}
 \hspace*{-6.7pt}{}=\left(1-\delta_{kj}\right)\lambda_{kj}e^{\lambda_{jj}\Delta}
\! \int\limits_0^{\Delta}\!
 e^{(\lambda_{kk}-\lambda_{jj})u^k}
 \mathcal{N}\left(Y_{r+1},u^kf^k +{}\right.\hspace*{-0.28818pt}\\[-1pt]
\hspace*{-3mm}\left. {}+ \left(\Delta - u^k\right)f^j, u^k g_kg_k^{\top}+
 \left(\Delta-u^k\right)g_jg_j^{\top}\right)\,du^k;
 \label{eq:h1}
 \end{multline}
 
 \vspace*{-12pt}
 
 \noindent
 \begin{multline}
 \int\limits_D \mathcal{N}\left( 
Y_{r+1},f u,\sum\limits_{p=1}^N u^p g_pg_p^{\top}\right)du ={}\\[-1pt]
{}=
\sum\limits_{\substack{{\ell:\ell \neq k,}\\ {\ell \neq j}}}
 \lambda_{k\ell}\lambda_{\ell j} e^{\lambda_{jj}\Delta}\times {}\\[-1pt] 
 {}\times
 \int\limits_0^{\Delta} \int\limits_0^{\Delta-u^k} \!
e^{(\lambda_{kk}-\lambda_{\ell\ell})u^k+(\lambda_{\ell\ell}-
 \lambda_{jj})u^{\ell}}\times{} \\[-1pt] 
{}  \times
 \mathcal{N}\left(Y_{r+1},u^k f^k+u^{\ell}f^{\ell}+\left(
 \Delta-u^k-u^{\ell} \right)f^j,\right.\\[-1pt]
 \hspace*{-1mm}\left.
 u^k g_kg_k^{\top}+u^{\ell}g_{\ell}g_{\ell}^{\top}+\left(
 \Delta-u^k-u^{\ell} \right)
 g_jg_j^{\top}
 \right) du^{\ell}du^{k}, \!\!
  \label{eq:h2}
 \end{multline} 
 
\vspace*{-2pt}
 
 \noindent
  где  $\delta_{ij}$~--- символ Кронекера. Интегралы для $h\hm>2$ также могут 
  быть получены в~явном виде, однако их сложность резко возрастает.
 

   Так как система~(\ref{eq:obsys_1}) является автономной, то в~качестве локальной 
   характеристики бли\-зости~$\{\overline{X}_{t_r}\}$ 
   к~$\{\widehat{X}_{t_r}\}$ может быть выбрана величина
   
\noindent
 \begin{multline*}
 \overline{\sigma}(\pi) \ebd {\sf E}\left\{
 \|\widehat{X}_{t_{1}}(\pi, Y_{1}) - \overline{X}_{t_{1}}
 \left(\pi,Y_{1}\right)\|_{1}\right\} = {}\\
 {}=
 \sum\limits_{j=1}^N{\sf E}
 \left\{\left\vert \widehat{X}^j_{t_{1}}\left(\pi, Y_{1}\right) - \overline{X}^{n,j}_{t_{1}}
 \left(\pi,Y_{1}\right)\right\vert\right\}.
 %\label{eq:prec_1}
 \end{multline*}
 При этом начальное распределение $\pi \hm\in \mathcal{D}_1 \ebd $\linebreak $\ebd
 \{\mathrm{col}\,(\pi^1,\ldots,\pi^N):\;\pi^j > 0$, 
 $\sum\nolimits_{j=1}^N\pi^j\hm=1\}$ является начальным условием применения 
 одного шага рекурсии~(\ref{eq:filt_1}) или~(\ref{eq:filt_2}) для вычисления 
 оценки~$\widehat{X}_{t_{1}}$
   или~$\overline{X}_{t_{1}}$ соответственно. Фактически, 
 характеристика~$\overline{\sigma}(\pi)$ определяет, насколько сильно 
 рекурсивные схемы~(\ref{eq:filt_1}) и~(\ref{eq:filt_2}) разойдутся за 
 один шаг, стартуя из общей точки~$\pi$.
 
 Рекуррентные схемы~(\ref{eq:filt_1}) и~(\ref{eq:filt_2}), примененные~$r$~раз, 
 позволяют вычислить оценки~$\widehat{X}_{t_r}$ и~$\overline{X}_{t_r}$ 
 в~точке~$t_r$. В~качестве характеристики точности глобальной аппроксимации в~этом 
 случае естественно рассмотреть величину
 
 \vspace*{-2pt}
 
 \noindent
 \begin{equation*}
 \overline{\Sigma}_{t_r}(\pi) \ebd {\sf E}
 \left\{\|\widehat{X}_{t_{r}} - \overline{X}_{t_{r}}\|_{1}\right\} = 
 \!\sum\limits_{j=1}^N\!{\sf E}
 \left\{\left\vert \widehat{X}^j_{t_{r}} - 
 \overline{X}^{n,j}_{t_{r}}\right\vert \right\}.
% \label{eq:prec_2}
 \end{equation*}
 
 Следующее утверждение определяет оценки локальной и~глобальной 
 точности схемы аппроксимации~(\ref{eq:filt_2}).
 
 %\smallskip
 
 \noindent
 \textbf{Теорема~1.}\
\textit{Выполняются неравенства} 

%\vspace*{-2pt}

\noindent
 \begin{equation}
 \sup_{\pi \in \mathcal{D}_1} \overline{\sigma}(\pi) 
 \leqslant 2 \fr{(\overline{\lambda}\Delta)^{n+1}}{(n+1)!}\,;
 \label{eq:prec_loc}
\end{equation}

\noindent
\begin{align}
  \sup\limits_{\pi \in \mathcal{D}_1} \overline{\Sigma}_{t_r}(\pi)
   &\leqslant 2r \fr{(\overline{\lambda}\Delta)^{n+1}}{(n+1)!} +{}\notag\\[-0.5pt]
   &\hspace*{-20mm}{}+
  r(r-1)\left(
  \fr{(\overline{\lambda}\Delta)^{n+1}}{(n+1)!}
  \right)^2
  \left(
  1-\fr{(\overline{\lambda}\Delta)^{n+1}}{(n+1)!}
  \right)^{r-2},
 \label{eq:prec_glob}
 \end{align}
 
 \vspace*{-2pt}
 
 \noindent
 \textit{где} $\overline{\lambda} \ebd \max_{1 \leqslant j \leqslant N}|\lambda_{jj}|$.


%\smallskip

 Доказательство теоремы~1 приведено в~приложении.
 
 Данное утверждение представляет полезные оценки точности. Во-пер\-вых, 
 они являются равномерными по начальному распределению $\pi \hm\in \mathcal{D}_1$. 
 Во-вто\-рых, оценки носят универсальный, а~не асимптотический характер. Это 
 существенно в~практических задачах оценивания по дискретизованным 
 наблюдениям с~физическими или алгоритмическими ограничениями на шаг 
 по времени. Например, в~случае наблюдаемого процесса восстановления в~силу 
 центральной предельной теоремы для процессов восстановления~\cite{B_80} его
  приращения можно рассматривать как гауссовские случайные величины. 
  Однако данная аппроксимация обладает удовлетворительной точностью 
  только в~случае, когда шаг дискретизации по времени достаточно большой. 
 %
 В-третьих, неравенство~(\ref{eq:prec_glob}) позволяет получить порядок 
 аппроксимации при $\Delta \hm\to 0$. Зафиксируем момент времени $t\hm=T$ и~рассмотрим 
 характеристику $\sup\nolimits_{\pi \in \mathcal{D}_1} 
 \overline{\Sigma}_{T}(\pi)$ при $r\hm={T}/{\Delta}$ и~$\Delta \hm\to 0$. 
 Как только~$\Delta$ становится настолько мало, что 
 $\max\left({(\overline{\lambda}\Delta)^{n+1}}/{(n+1)!}, 
 \Delta ({T\lambda^{n+1}}/{(n+1)!})\right)\hm< 1$, из~(\ref{eq:prec_glob}) 
 следует неравенство
  %\begin{equation}
  $\sup\nolimits_{\pi \in \mathcal{D}_1} \overline{\Sigma}_{T}(\pi) 
  \hm\leqslant  ({3\overline{\lambda}^{n+1}}/{(n+1)!}) T\Delta^n.$
 %\label{eq:prec_asympt}
 %\end{equation}
 Это значит, что с~ростом времени~$T$ 
 ошибка аппроксимации копится пропорционально~$T$ и~при этом порядок точности 
 по~$\Delta$ равен~$n$.
 
 %\vspace*{-7pt}
 
  \section{Заключение}
  
  \vspace*{-4pt}
 
  В работе решена задача оценивания состояния однородного МСП по 
  дискретизованным наблюдениям. Получено аналитическое решение и~его 
  чис\-лен\-ные аппроксимации. Локальные и~глобальные показатели точ\-ности этих 
  приближений в~статье так\-же пред\-став\-ле\-ны. Примечательно, что  част\-ный случай 
  аппроксимаций~(\ref{eq:filt_2}) при $n\hm=0$ и~$\Lambda\hm=0$ был ранее 
  пред\-став\-лен в~\cite{B_17_1,B_17_2} для решения задачи байесовской классификации 
  случайного вектора по непрерывным наблюдениям с~мультипликативными шумами. 
 % 
Алгоритм оптимальной фильт\-ра\-ции и~его субоптимальные версии могут 
рас\-смат\-ри\-вать\-ся в~качестве основы чис\-лен\-ной реализации обобщения фильт\-ра 
Вонэма для сис\-тем с~мультипликативными шумами в~наблюдениях. 
Однако для их непосредственного использования необходимо решить 
следующие проб\-ле\-мы. Во-пер\-вых, в~(\ref{eq:h1}) и~(\ref{eq:h2}) присутствуют
 многомерные интегралы. Следует выяснить, какую результирующую погрешность 
 будут вносить ошибки их вы\-чис\-ле\-ния. Во-вто\-рых, представляется интересным 
 определить характеристики точ\-ности оптимальной фильт\-ра\-ции по дискретизованным 
 наблюдениям по отношению к~оптимальной фильт\-ра\-ции по непрерывным наблюдениям: 
 каков порядок точ\-ности по шагу временной дискретизации~$\Delta$? Для случая 
 вы\-чис\-ле\-ния классического фильт\-ра Вонэма с~по\-мощью алгоритма Эй\-ле\-ра--Ма\-ру\-ямы 
 подобный результат известен: порядок глобальной ошибки равен~${1}/{2}$. 
 Перечисленные задачи являются предметом дальнейших исследований.
 
 
  \vspace*{-10pt}
 
{\small
\subsection*{\raggedleft Приложение} 

\vspace*{-2pt}


\noindent
Д\,о\,к\,а\,з\,а\,т\,е\,л\,ь\,с\,т\,в\,о\ \ теоремы~1.\ \ Введем следующие 
обозначения для случайных величин и~мат\-риц, составленных из них:
\begin{align*}
\xi^{ji}(\ell)&\ebd 
\sum\limits_{h=0}^n \int\limits_{\mathcal{D}} 
 \mathcal{N}\left(Y_{\ell},f u,\sum\limits_{p=1}^N u^p g_pg_p^{\top}\right)
 \rho^{j,i,h}_{1}(du)\,; \\
  \theta^{ji}(\ell)&\ebd 
\sum\limits_{h=n+1}^{\infty} \int\limits_{\mathcal{D}} 
 \mathcal{N}\left(Y_{\ell},f u,\sum\limits_{p=1}^N u^p g_pg_p^{\top}\right)
 \rho^{j,i,h}_{1}(du)\,;
\\
 \xi(\ell)&\ebd \|\xi^{ji}(\ell)\|_{j,i=\overline{1,N}}\,,\quad 
 \Xi(r) \ebd \xi(r) \xi(r-1)\cdots \xi(1)\,;
 \\
 \theta(\ell)&\ebd \|\theta^{ji}(\ell)\|_{j,i=\overline{1,N}}\,, \quad 
 \Theta(r) \ebd \theta(r) \theta(r-1)\cdots \theta(1)\,.
%\label{eq:not_1}
\end{align*}
 
 Рекуррентные формулы~(\ref{eq:filt_1}) и~(\ref{eq:filt_2}) можно записать в~явной 
 форме
 
 
\noindent
\begin{align*}
 \widehat{X}_{t_r}& = \left( \mathbf{1}\left(\Xi(r) + 
 \Theta(r)\right)\pi\right)^{-1} \left(\Xi(r) + \Theta(r)\right)\pi\,;
\\
 \overline{X}_{t_r} &= \left( \mathbf{1}\Xi(r)\pi\right)^{-1} \Xi(r) \pi,
\end{align*}

\vspace*{-2pt}

\noindent
где $\mathbf{1} \ebd (1,\ldots,1)$~--- век\-тор-стро\-ка 
подходящей раз\-мер\-ности, составленная из единиц.

%Далее для краткости записи зависимость от~$r$ в~обозначениях~$\Xi(r)$ 
%и~$\Theta(r)$ будет опущена. 
Верна следующая цепочка неравенств:

 \vspace*{-3pt}

\noindent
\begin{multline}
\overline{\Sigma}_{t_r}(\pi)=%
%\me{}{\left\| 
%\widehat{X}_{t_r}(\pi, Y_1,\ldots,Y_r) - \overline{X}_{t_r}(\pi, Y_1,\ldots,Y_r)
%\right\|_1} =\\=
{\sf E}\left\{\left\| 
\fr{1}{\mathbf{1}\left(\Xi(r) + \Theta(r)\right)\pi} \left(\Xi(r) +{}\right.\right.\right.\\[-1pt]
\left.\left.\left.{}+ \Theta(r)\right)\pi
- \fr{1}{\mathbf{1}\Xi(r)\pi}\,\Xi(r) \pi
\right\|_1\right\} ={} \\[-1pt]
{}=
{\sf E}\left\{\fr{1}{\mathbf{1}\left(\Xi(r) + \Theta(r)\right)\pi \mathbf{1}\Xi(r)\pi}
\left\|
 \mathbf{1}\Xi(r) \pi \Theta(r)\pi -{}\right.\right.\\[-1pt]
\left.\left. {}- \mathbf{1}\Theta(r)\pi \Xi(r) \pi
 \right\|_1
 \vphantom{\fr{1}{\mathbf{1}\left(\Xi(r) + \Theta(r)\right)\pi \mathbf{1}\Xi(r)\pi}}
\right\} \leqslant {}\\[-1pt]
{}\leqslant 
{\sf E}\left\{\fr{1}{\mathbf{1}\left(\Xi(r) + \Theta(r)\right)\pi \mathbf{1}\Xi(r)\pi}
\left(
\mathbf{1}\Xi(r)\pi \| \Theta(r)\pi \|_1 +{}\right.\right.\\[-1pt]
\left.\left.{}+ \mathbf{1}\Theta(r)\pi 
\|
\Xi(r) \pi
\|_1
\right)
 \vphantom{\fr{1}{\mathbf{1}\left(\Xi(r) + \Theta(r)\right)\pi \mathbf{1}\Xi(r)\pi}}
\right\} ={}\\[-1pt]
{}=
2\,{\sf E}\left\{\fr{1}{\mathbf{1}\left(\Xi(r) + \Theta(r)\right)\pi}\mathbf{1}\Theta(r)\pi 
\right\}.
\label{eq:ineq_1}
\end{multline}

 
 \noindent
 Рассмотрим случайные события $a_{\ell} \ebd \{\omega \in \Omega: 
 N_{\ell}(\omega) \hm\leqslant n\}$, $\ell \hm= \overline{1,r}$, и~$A_r \ebd \{
 \omega\hm \in \Omega: \max_{1 \leqslant {\ell} \leqslant r}N_{\ell}(\omega) 
 \hm\leqslant n
 \}\hm=\prod\nolimits_{\ell=1}^r a_{\ell}$ и~оценку 
 $
 \widetilde{X}_{t_r}(\pi, Y_1,\ldots,Y_r)\ebd$\linebreak $\ebd
 {\sf E}\left\{X_{t_r}(\omega)\mathbf{I}_{A_r}(\omega)|\mathcal{O}_r\right\}.
 $
 Используя введенные выше обозначе\-ния и~абстрактный вариант формулы Байеса, 
 получаем, что
 
 \noindent
\begin{align}
\widetilde{X}_{t_r}& = \fr{1}{{\mathbf{1}\left(\Xi(r) + 
 \Theta(r)\right)\pi}}\,\Xi(r)\pi\,;\notag
 \\
\widehat{X}_{t_r} - \widetilde{X}_{t_r} &=
{\sf E}\left\{X_{t_r}(\omega)\mathbf{I}_{\overline{A}_r}(\omega)|\mathcal{O}_r\right\} ={}\notag\\[-1pt]
&\hspace*{17mm}{}= 
\fr{1}{\mathbf{1}\left(\Xi(r) + \Theta(r)\right)\pi}\Theta(r)\pi\,. 
\label{eq:eq_2}
 \end{align}
 Из (\ref{eq:ineq_1}) и~(\ref{eq:eq_2}) для $r\hm=1$ следует, что
 
 \vspace*{-4pt}
 
 \noindent
 \begin{multline}
 \overline{\sigma}(\pi) \leqslant 2\,{\sf E}
 \left\{\|{\sf E}\left\{X_{t_1}(\omega)\mathbf{I}_{\overline{a}_1}(\omega)|\mathcal{O}_1
 \right\}\|_1
 \right\} ={}\\[-1.5pt]
 {}=
 2\,{\sf E}\left\{\sum\limits_{n=1}^N {\sf E}
 \left\{X^n_{t_1}(\omega)\mathbf{I}_{\overline{a}_1}
 (\omega)|\mathcal{O}_1\right\}\right\} ={} \\[-2pt] 
 {}=
  2\,{\sf E}\left\{{\sf E}\left\{\mathbf{I}_{\overline{a}_1}(\omega)|\mathcal{O}_1
  \right\}\right\} =
   2 \mathbf{P}\left\{\overline{a}_1(\omega)\right\}.
\label{eq:ineq_3}
\end{multline}

 \vspace*{-2pt}
 
 \noindent
 Процесс $N^X_t$ общего числа скачков состояния~$X_t$ является считающим, и~его
  квадратическая характеристика равна 
  
\vspace*{-2pt}
  
  \noindent
 $$
 \langle N^X, N^X\rangle_t = - \int\limits_0^t \sum\limits_{n=1}^N \lambda_{nn} X_s^n\,ds\,,
 $$
 поэтому искомая вероятность ограничена сверху:
 $$ 
 \mathbf{P}\left\{\overline{a}_1(\omega)\right\} \leqslant 
 e^{-\overline{\lambda}\Delta}\sum\limits_{k=n+1}^{\infty} 
 \fr{(\overline{\lambda}\Delta)^{k}}{k!} <
 \fr{(\overline{\lambda}\Delta)^{n+1}}{(n+1)!}.
 $$
 
  \vspace*{-2pt}
  
  \noindent
 Из последнего неравенства и~(\ref{eq:ineq_3}) следует, что  для любого 
 начального распределения~$\pi$ выполняется неравенство $\overline{\sigma}(\pi)  
 \hm< 2({(\overline{\lambda}\Delta)^{n+1}}/{(n+1)!})$, т.\,е.\ 
 локальная оценка~(\ref{eq:prec_loc}) верна.
 
 С помощью марковского свойства пары $(X_t, N^X_t)$ и~последнего 
 неравенства можно оценить сверху вероятность 
 $\mathbf{P}\left\{\overline{A}_r(\omega)\right\}$:
 
  \vspace*{-2pt}
 
 \noindent
 \begin{multline*}
 \mathbf{P}\left\{\overline{A}_r(\omega)\right\} \leqslant 1 - \left(
 1- \fr{(\overline{\lambda}\Delta)^{n+1}}{(n+1)!}
 \right)^r \leqslant r \fr{(\overline{\lambda}\Delta)^{n+1}}{(n+1)!} + {}\\[-1pt]
 {}+\left|
 \sum\limits_{k=2}^r C_r^k \left(-\fr{(\overline{\lambda}\Delta)^{n+1}}{(n+1)!}
 \right)^k
 \right| \leqslant
 r \fr{(\overline{\lambda}\Delta)^{n+1}}{(n+1)!} +{}\\[-1pt]
 {}+\fr{r(r-1)}{2}
 \left(
 \fr{(\overline{\lambda}\Delta)^{n+1}}{(n+1)!}
 \right)^2
 \left(
 1-\fr{(\overline{\lambda}\Delta)^{n+1}}{(n+1)!}
 \right)^{r-2},
 \end{multline*} 
 из чего следует истинность глобальной оценки~(\ref{eq:prec_glob}).
Теорема~1 доказана.

}

%\vspace*{-12pt}

{\small\frenchspacing
 {%\baselineskip=10.8pt
 \addcontentsline{toc}{section}{References}
 \begin{thebibliography}{99}

\bibitem{Won_65}
\Au{Wonham W.} 
Some applications of stochastic differential equations to optimal
  nonlinear filtering~//
SIAM~J.~Control, 1965. Vol.~2. P.~347--369. 

\bibitem{KP_92}
\Au{Kloeden P., Platen E.} Numerical solution of stochastic
differential equations.~--- Berlin: Springer, 1992.~636~p.

\bibitem{YZL_04}
\Au{Yin G., Zhang Q., Liu Y.} 
Discrete-time approximation of Wonham filters~//
J.~Control Theory Applications, 2004. Iss.~2. P.~1--10.

\bibitem{PR_10}
\Au{Platen E., Rendek R.}
Quasi-exact approximation of hidden Markov chain filters~//
Communicat.~Stoch.~Analys., 2010. Vol.~4. Iss.~1. P.~129--142.

\bibitem{B_18}
\Au{Борисов А.} Фильтрация Вонэма по наблюдениям с~мультипликативными шумами~// 
Автоматика и~телемеханика, 2018.
№~1. C.~52--65. 
 
  \bibitem{BSh_85} %6
\Au{Бертсекас Д., Шрив С.} Стохастическое оптимальное управление. 
Случай дискретного времени~/ Пер. с~англ.~--- М.: Наука, 1985.~280~c.
(\Au{Betsekas~D.\,P., Shreve~S.\,E.} Stochastic optimal control:
The discrete-time case.~--- Orlando, FL, USA:
Academic Press Inc., 1978. 323~p.)

  \bibitem{ZhSh_95} %7
\Au{Жакод Ж., Ширяев А.} Предельные теоремы для случайных процессов,~I.~/
Пер. с~англ.~--- 
М.: Физматлит, 1995.~544~c.
(\Au{Jacod~J., Shiryaev~A.} Limit theorems for stochastic processes.~---
Berlin: Springer, 2003. 664~p.)

\bibitem{S_00}
\Au{Sericola B.} Occupation times in Markov processes~//
Commun. Stat. Stochastic Models, 2000. Vol.~16. Iss.~5. P.~479--510. 

  \bibitem{B_80}
\Au{Боровков А.} Асимптотические методы в~тео\-рии массового обслуживания.~--- 
М.: Физматлит, 1995.~384~c.

  \bibitem{B_17_1}
\Au{Борисов А.} Классификация по непрерывным наблюдениям с~мультипликативными шумами.~I. 
Формулы байесовской оценки~// Информатика и~её применения, 2017. Т.~11. Вып.~1. C.~11--19.
doi: 10.14357/19922264170102.

  \bibitem{B_17_2}
\Au{Борисов А.} Классификация по непрерывным наблюдениям с~мультипликативными 
шумами.~II. Алгоритм численной реализации оценки~// Информатика и~её 
применения, 2017. Т.~11. Вып.~2. C.~33--41.
doi: 10.14357/19922264170204.

 \end{thebibliography}

 }
 }

\end{multicols}

\vspace*{-4pt}

\hfill{\small\textit{Поступила в~редакцию 10.07.18}}

\vspace*{6pt}

%\pagebreak

%\newpage

%\vspace*{-28pt}

\hrule

\vspace*{2pt}

\hrule

%\vspace*{-2pt}

\def\tit{FILTERING OF~MARKOV JUMP PROCESSES\\ BY~DISCRETIZED OBSERVATIONS}

\def\titkol{Filtering of Markov jump processes by discretized observations}

\def\aut{A.\,V.~Borisov}

\def\autkol{A.\,V.~Borisov}

\titel{\tit}{\aut}{\autkol}{\titkol}

\vspace*{-11pt}


\noindent
Institute of Informatics Problems, Federal Research Center ``Computer Science 
and Control'' of the Russian Academy of Sciences, 44-2~Vavilov Str., Moscow 
119333, Russian Federation


\def\leftfootline{\small{\textbf{\thepage}
\hfill INFORMATIKA I EE PRIMENENIYA~--- INFORMATICS AND
APPLICATIONS\ \ \ 2018\ \ \ volume~12\ \ \ issue\ 3}
}%
 \def\rightfootline{\small{INFORMATIKA I EE PRIMENENIYA~---
INFORMATICS AND APPLICATIONS\ \ \ 2018\ \ \ volume~12\ \ \ issue\ 3
\hfill \textbf{\thepage}}}

\vspace*{6pt}



\Abste{The article is devoted to a~solution of the optimal filtering problem 
of a~homogenous Markov
jump process state. The available observations represent 
time increments of the integral transformations of the Markov\linebreak\vspace*{-12pt}}

\Abstend{state corrupted by 
Wiener processes. The noise intensity is also state-dependent. At the instant of 
the consecutive
observation obtaining, the optimal estimate is calculated recursively 
as a~function of previous estimate and the new observation, meanwhile between 
observations the filtering estimate is a simple forecast by virtue of the Kolmogorov 
differential system. The recursion is rather expensive because of  need to calculate 
the integrals, which are the location-scale mixtures of Gaussians. The mixing 
distributions represent the occupation of the state in each of possible values 
during the mid-observation intervals. The paper contains numerically cheaper 
approximations, based on the restriction of the state transitions number between 
the observations. Both the local and global characteristics of approximation 
accuracy are obtained as functions of the dynamics parameters, mid-observation 
interval length, and upper bound of transitions number.}

\KWE{Markov jump process; optimal filtering; multiplicative observation noises; 
stochastic differential equation; numerical approximation}




\DOI{10.14357/19922264180316}

%\vspace*{-14pt}

\Ack
\noindent
The work was supported in part by the Russian Foundation
for Basic Research (Project No.\,16-07-00677).



%\vspace*{6pt}

  \begin{multicols}{2}

\renewcommand{\bibname}{\protect\rmfamily References}
%\renewcommand{\bibname}{\large\protect\rm References}

{\small\frenchspacing
 {%\baselineskip=10.8pt
 \addcontentsline{toc}{section}{References}
 \begin{thebibliography}{99}
\bibitem{Won_65-1}
\Aue{Wonham, W.} 1965.
Some applications of stochastic differential equations to optimal
  nonlinear filtering.
\textit{SIAM~J.~Control} 2:347--369. 

\bibitem{KP_92-1}
\Aue{Kloeden,~P., and E.~Platen.} 1992. \textit{Numerical solution of stochastic
differential equations.} Berlin: Springer. 636~p.

\bibitem{YZL_04-1}
\Aue{Yin,~G., Q.~Zhang, and Y.~Liu.} 2004.
Discrete-time approximation of Wonham filters.
\textit{J.~Control Theory Applications} 2:1--10.

\bibitem{PR_10-1}
\Aue{Platen, E., and R.~Rendek.} 2010.
Quasi-exact approximation of hidden Markov chain filters.
\textit{Communicat. Stoch. Analys.} 4(1):129--142.

\bibitem{B_18-1}
\Aue{Borisov, A.} 2018. Wonham filtering by observations
with multiplicative noises. \textit{Automat.~Rem.~Contr.} 79(1):39--50.  
doi: 10.1134/ S0005117918010046.
 
  \bibitem{BSh_85-1}
\Aue{Bertsekas, D., and S.~Shreve.} 1996.
\textit{Stochastic optimal control: The discrete-time case}.
Nashua, NH: Athena Scientific. 330~p.
  
  \bibitem{ZhSh_95-1}
  \Aue{Jacod,~J., and A.~Shiryaev.} 2003.
\textit{Limit theorems for stochastic processes.}
Berlin: Springer. 664~p.

\bibitem{S_00-1}
\Aue{Sericola, B.}
2000. Occupation times in Markov processes.
\textit{Commun. Stat.} 16(5):479--510. 

  \bibitem{B_80-1}
\Aue{Borovkov, A.} 1984.
 \textit{Asymptotic methods in queueing theory}. 
 Hoboken, NJ: Wiley-Blackwell.~304~p.

  \bibitem{B_17_1-1}
  \Aue{Borisov, A.} 2017. 
  Klassifikatsiya po ne\-pre\-ryv\-nym nablyu\-de\-miyam s~mul'tiplikativnymi shumami. I. 
  Formuly bayesov\-skoy otsenki [Classification by continuous-time observations
in multiplicative noise. I.~Formulae for Bayesian 
estimate]. \textit{Informatika i~ee Primeneniya~--- Inform.~Appl.}
11(1):11--19. doi: 10.14357/19922264170102.

  \bibitem{B_17_2-1}
\Aue{Borisov, A.} 2017. Klassifikatsiya po nepreryvnym nablyudemiyam 
s~mul'tiplikativnymi summami. II.~Formuly bayesovskoy otsenki 
[Classification by continuous-time observations
in multiplicative noise. II.~Numerical algorithm].
\textit{Informatika i~ee Primeneniya~--- Inform.~Appl.}
11(2):33--41. doi: 10.14357/19922264170204.

\end{thebibliography}

 }
 }

\end{multicols}

\vspace*{-6pt}

\hfill{\small\textit{Received July 10, 2018}}

%\pagebreak

%\vspace*{-18pt}

\Contrl

\noindent
\textbf{Borisov Andrey V.} (b.\ 1965)~--- 
Doctor of Science in physics and mathematics, principal scientist, Institute of
Informatics Problems, Federal Research Center ``Computer Science and Control''
 of the Russian Academy of
Sciences, 44-2 Vavilov Str., Moscow 119333, Russian Federation; 
\mbox{aborisov@frccsc.ru}
\label{end\stat}

\renewcommand{\bibname}{\protect\rm Литература}         %11
\def\stat{kudr}

\def\tit{ПРИБЛИЖЕННЫЕ МЕТОДЫ РЕШЕНИЯ ЗАДАЧИ ДИАГНОСТИКИ ПЛОСКИМ 
ЗОНДОМ СИЛЬНОИОНИЗОВАННОЙ ПЛАЗМЫ С~УЧЕТОМ КУЛОНОВСКИХ 
СТОЛКНОВЕНИЙ}

\def\titkol{Приближенные методы решения задачи диагностики плоским 
зондом сильноионизованной плазмы} %с~учетом Кулоновских  столкновений}

\def\autkol{И.\,А.~Кудрявцева, А.\,В.~Пантелеев}
\def\aut{И.\,А.~Кудрявцева$^1$, А.\,В.~Пантелеев$^2$}

\titel{\tit}{\aut}{\autkol}{\titkol}

%{\renewcommand{\thefootnote}{\fnsymbol{footnote}}\footnotetext[1]
%{Работа поддержана Российским фондом фундаментальных исследований
%(проекты 11-01-00515а и 11-07-00112а), а также Министерством
%образования и науки РФ в рамках ФЦП <<Научные и
%научно-педагогические кадры инновационной России на 2009--2013~годы>>.}}


\renewcommand{\thefootnote}{\arabic{footnote}}
\footnotetext[1]{Московский авиационный институт, irina.home.mail@mail.ru}
\footnotetext[2]{Московский авиационный институт, avpanteleev@inbox.ru}

\vspace*{-2pt}

\Abst{Сформирована математическая модель, описывающая динамику сильноионизованной 
плазмы с учетом столкновений заряженных частиц вблизи плоского зонда. Модель включает уравнение 
Фоккера--Планка и уравнение Пуассона. Предложено два подхода к решению задачи: на основе метода 
статистических испытаний Мон\-те-Кар\-ло и на основе композиции метода крупных частиц и метода 
расщепления.} 

\vspace*{-2pt}

\KW{телекоммуникационные системы; метод Монте-Карло; метод крупных частиц; метод 
расщепления; зонд; уравнение Фоккера--Планка; уравнение Пуассона} 

\vspace*{-4pt}

 \vskip 8pt plus 9pt minus 6pt

      \thispagestyle{headings}

      \begin{multicols}{2}
      
            \label{st\stat}

\section{Введение}

В настоящее время в области телекоммуникаций все более востребованными становятся 
информационные технологии, основанные на использовании математических моделей и численных 
методов физики плазмы. Поэтому особенно актуальным является решение разнообразных задач анализа 
поведения плазмы, включающих в себя формирование новых моделей и методов их исследования. 
Помимо этого, в разработке телекоммуникационного оборудования эффективно используются 
собственно физические свойства плазмы. В~частности, изготовлена антенна, работа которой основана 
на газовом разряде низкотемпературной плазмы~[1], интенсивно ведутся разработки по созданию и 
усовершенствованию источников бесперебойного питания на основе плазменных элементов~[2, 3]. 
      
      Одним из наиболее перспективных направлений для построения систем оптической 
беспроводной связи является использование лазеров~\cite{4-k, 5-k}. В~этой связи большое внимание 
уделяется использованию плазмы при разработке импульсных сильноточных коммутаторов~\cite{6-k}, 
так как практическое применение подобных разработок требует повышения уровня надежности и 
быстродействия лазерных систем.
      
      Исследования низкотемпературной плазмы также связаны с разработками в области дальней 
космической связи, так как моделирование процессов взаимодействия заряженного тела с верхними 
слоями атмосферы позволяет предлагать способы улучшения существующих систем радиосвязи с 
космическими летательными аппаратами~\cite{7-k}. 
      
      Наряду с этим актуальными также являются задачи диагностики плазмы, поскольку перспективы 
ее использования в области телекоммуникаций после более полного изучения физических свойств 
могут значительно расшириться. 

Для диагностики плазмы применяют зондовые методы исследования~[8--11]. Эти методы относятся к 
классу контактных методов; как следствие, возникает сложность в исследовании пристеночной области 
вблизи зонда, которая характеризуется достаточно сложным распределением потенциала и функциями 
распределения, отличными от максвелловских. 

Данная работа посвящена исследованию переходного режима обтекания заряженного тела плазмой. Для 
переходного режима выполняется следующее условие: длина свободного пробега иона до столкновения 
с нейтральным атомом или другим ионом невелика по сравнению с характерными размерами тела. 
В~этом случае возникает необходимость учета столкновений заряженных частиц с нейтральными 
атомами и кулоновских столкновений. В~работах~\cite{10-k, 11-k} подробно рассмотрена модель с 
учетом столкновений заряженных частиц с нейтральными атомами. В~настоящей статье представлена 
теоретическая модель, описывающая влияния ион-ионных и ион-элек\-т\-рон\-ных столкновений на 
измеряемые характеристики плазмы, что ранее детально не исследовалось.
      
      В~рамках данной работы предлагается модель, описывающая динамику сильноионизованной 
плазмы с учетом кулоновских столкновений. Эта модель учитывает такие процессы взаимодействия, 
как перенос частиц и столкновения между заряженными частицами типа <<ион--ион>> и 
      <<ион--электрон>> под влиянием макроскопического электрического поля. Перечисленные 
процессы описываются самосогласованной системой уравнений, включающей уравнение 
      Фок\-ке\-ра--План\-ка и уравнение Пуассона~[12].
      
      Вычислительная модель задачи строится на основе двух методов: метода статистических 
испытаний Мон\-те-Кар\-ло и композиции метода крупных частиц и метода расщепления. Приведены 
результаты численного моделирования, полученные с использованием вышеперечисленных методов.

\vspace*{-4pt}

\section{Постановка задачи}

\vspace*{-2pt}

Рассматривается следующая физическая постановка зондовой задачи~[11]. В~невозмущенную 
бесконечно протяженную плазму, состоящую из электронов и однозарядных ионов, внесена большая\linebreak 
заряженная до потенциала $\varphi_p$ плоскость. Плоскость, расположенная поперек потока плазмы, 
является идеально поглощающей для электронов. Ионы при ударе о плоскость нейтрализуются. 
Предполагается, что частицы в плазме движутся под действием внешнего электрического поля, 
магнитное поле отсутствует. Концентрации ионов $n_{i\infty}$ и электронов $n_{e\infty}$, а также 
температуры данных час\-тиц~$T_{i\infty}$ 
и~$T_{e\infty}$ в невозмущенной плазме заданы. За начальные 
функции распределения обоих типов час\-тиц принимаются функции распределения Максвелла. 
      
      Требуется с учетом столкновений между заряженными частицами найти напряженность 
самосогласованного электрического поля $\vec{E}(\vec{r},t)$, функции распределения однозарядных 
ионов $f_i(\vec{r}, \vec{v}, t)$ и электронов $f_e(\vec{r}, \vec{v}, t)$, 
а также их моменты (плотности 
токов ионов и электронов  $j_i(\vec{r},t)\hm
=q\int f_i(\vec{r}, \vec{v}, t)\vec{v}\,d\vec{v}$, $j_e(\vec{r},t) 
\hm={\sf e}\int f_e(\vec{r},\vec{v},t)\vec{v}\,d\vec{v}$, где $q=Z_i{\sf e}$, $Z_i=1$~--- заряд иона, ${\sf 
e}$~--- заряд электрона; концентрации ионов и электронов $n_i(\vec{r},t)\hm=\int 
f_i(\vec{r},\vec{v},t)\,d\vec{v}$, $n_e(\vec{r},t)\hm=\int f_e(\vec{r},\vec{v}, t)\,d\vec{v}$). 
Поведение частиц во 
времени~$t$ характеризуется ра\-ди\-ус-век\-то\-ром~$\vec{r}$ и вектором скорости~$\vec{v}$.
      
      Математическая модель, соответствующая данной физической постановке задачи, имеет 
вид~\cite{11-k, 13-k}:

\noindent
      \begin{equation}
      \left.
      \begin{array}{c}
      \fr{\partial f_\alpha (\vec{r},\vec{v},t)}{\partial t}+
      \vec{v}\fr{\partial f_\alpha (\vec{r},\vec{v},t)}{ 
\partial \vec{r}}+
\fr{\vec{F}_\alpha(\vec{r},t)}{m_\alpha}\times{}\\[4pt]
{}\times\fr{\partial f_\alpha(\vec{r},\vec{v},t)}{ \partial 
\vec{v}}=
\left(\fr{\partial f_\alpha(\vec{r},\vec{v},t)}{ \partial t}\right)_{\mathrm{с}}+S_\alpha 
(\vec{r},\vec{v},t)\,;\\[6pt]
      \Delta\varphi(\vec{r},t)=-\fr{{\sf e}}{\varepsilon_0}\left( n_i(\vec{r},t)-n_e(\vec{r},t)\right)\,;\\[6pt]
      \vec{E}(\vec{r},t)=-\nabla \varphi(\vec{r},t)\,.
      \end{array}\!\!
      \right\}\!\!
      \label{e1-k}
      \end{equation}
Здесь первое уравнение~--- уравнение Фок\-ке\-ра--План\-ка для частиц сорта~$\alpha$ ($\alpha=i,e$), 
второе~--- уравнение Пуассона для самосогласованного электрического поля; 
$f_\alpha(\vec{r},\vec{v},t)$~--- функция\linebreak
распределения час\-тиц сорта~$\alpha$; $(\partial 
f_\alpha(\vec{r},\vec{v},t)/\partial t)_{\mathrm{с}}$~--- 
оператор столкновений Фок\-ке\-ра--План\-ка; 
функция~$S_\alpha(\vec{r},\vec{v},t)$ описывает источники или стоки\linebreak
 час\-тиц; 
$\vec{F}_\alpha(\vec{r},t)=q_\alpha\vec{E}(\vec{r},t)$, где $\vec{E}(\vec{r},t)$~--- напряженность 
самосогласованного электрического поля, 
$$
q_\alpha =
\begin{cases}
-{\sf e}\,, & \alpha=e\,,\\
{\sf e}\,, & \alpha=i\,;
\end{cases}
$$
$\varphi(\vec{r},t)$~--- потенциал самосогласованного электрического поля; $n_\alpha(\vec{r},t)$ ($\alpha 
\hm=i,e$)~--- концентрация частиц сорта~$\alpha$; $m_\alpha$~--- масса частицы сорта~$\alpha$; 
$\varepsilon_0$~--- электрическая постоянная. 

Оператор столкновений Фок\-ке\-ра--План\-ка имеет вид~\cite{13-k, 14-k}
\begin{multline*}
\fr{1}{\Gamma_\alpha}\left( \fr{\partial f_\alpha}{\partial t}\right)_{\mathrm{с}} 
=\fr{1}{2}\,\nabla_v\nabla_v:\left(f_\alpha\nabla_v\nabla_vg_\alpha(\vec{r},\vec{v},t)\right)-{}\\
{}-
\nabla_v\cdot\left(f_\alpha\nabla_v h_\alpha\right)\,,
\end{multline*}
где $\nabla_v\nabla_v g_\alpha(\vec{r},\vec{v},t)$~--- ковариантная тензорная производная второго ранга, 
знак двоеточия ($:$) обозначает операцию двойного суммирования:
\begin{gather*}
\Gamma_\alpha=\fr{Z_\alpha^4 {\sf e}^4}{4\pi \varepsilon_0^2 m^2_\alpha}\,\ln D_\alpha\,;
\\
D_\alpha =\fr{12\pi\varepsilon_0 kT_{\alpha\infty}}{Z_\alpha^2 {\sf e}^2}\left( \fr{\varepsilon_0 k 
T_{e\infty}}{n_{e\infty} {\sf e}^2}\right)^{1/2}\,;\\
g_\alpha (\vec{r},\vec{v},t)=\sum\limits_{b=i,e}\left( \fr{Z_b}{Z_\alpha}\right) \int f_b 
(\vec{r},{\vec{v}}^{\,\prime},t)\left\vert \vec{v}-{\vec{v}}^{\,\prime}\right\vert\,d\vec{v}^{\,\prime}\,;\\
h_\alpha (\vec{r},\vec{v},t)=\sum\limits_{b=i,e} \fr{m_\alpha+m_b}{m_b} 
\left(\fr{Z_b}{Z_\alpha}\right)
\int
\fr{f_b(\vec{r},{\vec{v}}^{\,\prime}, t)}{\vert \vec{v}-{\vec{v}}^{\,\prime}\vert}
\,d{\vec{v}}^{\,\prime}\,;\\
Z_\alpha =1\,, \quad \alpha=i,e\,.
\end{gather*}
 
К системе уравнений~(\ref{e1-k}) необходимо добавить начальные и краевые условия:
\begin{equation}
\!\left.
\begin{array}{rrl}
t=0:\ & f_\alpha(\vec{r},\vec{v},0)&=f_\alpha^{\mathrm{maksv}}\,,\enskip \alpha=i,e;\\[9pt]
\vec{r}\in \Omega_p:\ & f_\alpha(\vec{r},\vec{v},t)\big\vert_{\vec{r}\in\Omega_p}&=0\,,\enskip \alpha=i,e\,;\\[9pt]
&\varphi(\vec{r},t)\big\vert_{\vec{r}\in\Omega_p}&=\varphi_p\,;\\[9pt]
\vec{r}\in\Omega_\infty:\ & 
f_\alpha(\vec{r},\vec{v},t)\big\vert_{\vec{r}\in\Omega_\infty}&= %{}\\[9pt]
f_\alpha^{\mathrm{maksv}}\,,\enskip \alpha=i,e\,;\\[9pt]
&\varphi(\vec{r},t)\big\vert_{\vec{r}\in\Omega_\infty}&=0\,,
\end{array}\!\!
\right\}\!\!\!\!
\label{e2-k}
\end{equation}
    где 
    
    \noindent
    \begin{multline*}
    f_\alpha^{\mathrm{maksv}}=n_{\alpha\infty}\left(\fr{m_\alpha}{2k\pi T_{\alpha\infty}}\right)^{3/2}\times{}\\
    {}\times
    \exp\left( -
\fr{m_\alpha}{2kT_{\alpha\infty}}\left\vert\vec{v}-\vec{v}_\infty\right\vert^2\right)\,,
\enskip \alpha=i, e\,;
\end{multline*} 
$\Omega_p$ и $\Omega_\infty$~--- множество радиус-векторов час\-тиц, концы которых принадлежат плоскости зонда и 
границе возмущенной зоны соответственно.

Для решения поставленной задачи введем декартову систему координат таким образом, чтобы 
заряженная плоскость совпала с плоскостью~$0xz$. Тогда положение частицы в пространстве будет 
определяться координатами $x,y,z$, а скорость~--- координатами $v_x, v_y, v_z$. В~силу того что 
плоскость является бесконечно большой в сравнении с характерным размером задачи, функции 
распределения частиц будут зависеть только от переменных $y, v_y, t$.

Поставленную задачу предлагается решать независимо двумя методами. Первый метод основывается на 
методе статистических испытаний Мон\-те-Кар\-ло, второй метод является композицией метода 
расщепления и метода крупных частиц.

\section{Применение метода Монте-Карло}

Запишем самосогласованную систему уравнений~(\ref{e1-k}) и~(\ref{e2-k}) в декартовой системе 
координат с учетом сделанных предположений:
\begin{equation}
\left.
\begin{array}{l}
\fr{\partial f_\alpha}{\partial t}+
v_y\fr{\partial f_\alpha}{\partial y}+\fr{F_y^\alpha}{m_\alpha}\,\fr{\partial 
f_\alpha}{\partial v_y}=\fr{1}{2}\,\fr{\partial^2 }{\partial [v_y]^2}\times{}\\
{}\times \left( 
f_\alpha\fr{\partial^2 g_\alpha  }{\partial [v_y]^2}\right) -
\fr{\partial}{\partial v_y}\left( f_\alpha\fr{\partial h_\alpha}{\partial v_y}\right)\,,
\enskip \alpha=i,e\,;\\[6pt]
    \fr{\partial^2\varphi}{\partial y^2} =-\fr{{\sf e}}{\varepsilon_0}\left(n_i-n_e\right)\,;
    \enskip E_y=-
\fr{\partial\varphi}{\partial y}\,;\\[6pt]
\hspace*{3.1mm}    t=0:\  \hspace*{2.6mm}f_\alpha(y,v_y,0)=f_\alpha^{\mathrm{maksv}}\,,\ \alpha=i,e\,;\\[9pt]
\hspace*{2.9mm} y=0:\ \hspace*{2.8mm}f_\alpha(0,v_y,t)=0\,,\ \alpha=i,e\,;\\[9pt]
\hspace*{24.3mm}\varphi(0,t)=\varphi_p\,;\\[9pt]
y=y_\infty:\ f_\alpha(y_\infty, v_y, t)=f_\alpha^{\mathrm{maksv}}\,,\ \alpha=i,e\,;\\[9pt]
\hspace*{21.5mm}\varphi(y_\infty, t)=0\,.
\end{array}
\right \}
\label{e3-k}
\end{equation}

В полученной системе уравнений~(\ref{e3-k}) перейдем к безразмерным величинам, применив 
соотношение $X=M_X \hat{X}$, где $M_X$~--- масштаб размерной величины~$X$, $\hat{X}$~--- 
безразмерная величина~$X$. В~качестве используемых масштабов были взяты следующие: радиус 
Дебая, скорость теплового движения частиц, концентрация частиц в невозмущенной плазме, потенциал, 
возникающий при разделении зарядов в дебаевской сфере, и производные от них величины.

Система безразмерных уравнений имеет следующий вид:
%\noindent
\begin{equation}
\left.
\begin{array}{l}
\fr{\partial 
\hat{f}_\alpha}{\partial\hat{t}}+A_\alpha\fr{\partial\hat{f}_\alpha}{\partial\hat{y}}+
B_\alpha\hat{E}_y\fr{\partial\hat{f}_\alpha}{\partial \hat{v}_y}={}\\
\!{}=
\fr{\partial^2}{\partial[\hat{v}_y]^2}\left(D_\alpha 
\hat{f}_\alpha\right)-\fr{\partial}{\partial\hat{v}_y}\left(K_\alpha \hat{f}_\alpha\right),\enskip 
\alpha=i,e;\\[9pt]
\fr{\partial^2\hat{\varphi}}{\partial\hat{y}^2}=-\left(\hat{n}_i-\hat{n}_e\right)\,;\enskip \hat{e}_y=-
\fr{\partial\hat\varphi}{\partial\hat{y}}\,;\\[9pt]
\hspace*{3.1mm}\hat{t}=0:\ \hspace*{2.6mm}\hat{f}_\alpha(\hat{y},\hat{v}_y,0)=\hat{f}_\alpha^{\mathrm{maksv}}\,,\enskip \alpha-i,e\,;\\[9pt]
\hspace*{2.9mm}\hat{y}=0:\ \hspace*{2.8mm}\hat{f}_\alpha(0,\hat{v}_y,\hat{t})=0\,,\enskip \alpha=i,e\,;\\[9pt]
\hspace*{24.3mm}\hat\varphi(0,\hat{t})=\hat{\varphi}_p\,;\\[9pt]
\hat{y}=\hat{y}_\infty:\ \hat{f}_\alpha(\hat{y}_\infty, \hat{v}_y, \hat{t})=\hat{f}^{\mathrm{maksv}}_\alpha\,,\enskip 
\alpha=i,e\,;\\[9pt]
\hspace*{21.5mm}\hat\varphi(\hat{y}_\infty,\hat{t})=0\,.
\end{array}
\right\}
\label{e4-k}
\end{equation}
Здесь 

\vspace*{-2pt}

\noindent
\begin{gather*}
A_\alpha=\sqrt{\delta_\alpha }\,\hat{v}_y\,;\enskip 
B_\alpha=\sqrt{\delta_\alpha}\,\fr{z_\alpha}{2\varepsilon_\alpha}\,;\\
\delta_\alpha=\fr{\varepsilon_\alpha}{\mu_\alpha}\,;\enskip 
\varepsilon_\alpha=\fr{T_{\alpha\infty}}{T_{i\infty}}\,;\\
\mu_\alpha=\fr{m_\alpha}{m_i}\,;\enskip 
D_\alpha=A_g^\alpha\fr{\partial^2\hat{g}_\alpha}{\partial  [\hat{v}_y]^2}\,;\\
K_\alpha=A_h^\alpha \fr{\partial \hat{h}_\alpha}{\partial \hat{v}_y}\,,\enskip \alpha=i,e\,,
\end{gather*}
где $A_g^\alpha$ и $A_h^\alpha$~--- коэффициенты, определяемые характерными параметрами 
задачи~\cite{15-k}.

Поиск решения самосогласованной системы уравнений~(\ref{e4-k}) осуществляется по следующей 
схе-\linebreak ме. Вначале находятся значения напряженности\linebreak
 электрического поля по значениям потенциала, 
полученным из граничной задачи для уравнения Пуассона. Далее, используя найденные значения 
напряженности, решается уравнение Фок\-ке\-ра--План\-ка путем перехода к стохастическому 
дифференциальному уравнению (СДУ) Ито:

\noindent
\begin{multline*}
d\Theta_\alpha(\hat{t}) = a_\alpha \left(\hat{t},\Theta_\alpha(\hat{t})\right)+{}\\
{}+\sigma\left(
\hat{t},\Theta_\alpha(\hat{t})\right)\,dW(\hat{t})\,,\quad \alpha=i,e\,,
%\label{e5-k}
\end{multline*}
где 

\noindent
\begin{align*}
\Theta_\alpha(\hat{t})&=\begin{bmatrix}
\hat{y}(\hat{t})\\ \hat{v}_y(\hat{t})
\end{bmatrix}\,;\\
a_\alpha\left(\hat{t},\Theta_\alpha(\hat{t})\right)&=\begin{bmatrix}
-A_\alpha\\ -K_\alpha -B_\alpha \hat{E}_y
\end{bmatrix}\,;\\
\sigma_\alpha\left(\hat{t},\Theta_\alpha(\hat{t})\right)\sigma_\alpha^{\mathrm{T}}\left( 
\hat{t},\Theta_\alpha(\hat{t})\right)&=D_\alpha\,,\enskip \alpha=i,e\,;
\end{align*} 
$W(\hat{t})$~--- стандартный винеровский случайный процесс.
\pagebreak

Для нахождения значений вектора состояния~$\Theta_\alpha(\hat{t})$ применим явную разностную 
схему стохастического метода Эйлера~\cite{16-k}:
\begin{multline*}
\Theta_\alpha^{n+1}=\Theta_\alpha^n +h_\tau a_\alpha \left( \hat{t}_n, \Theta_\alpha^n\right)+\sigma_\alpha 
\left( \hat{t}_n, \Theta_\alpha^n\right)\Delta W_n\,,\\ 
n=0,\ldots , N\,,\ \alpha=i,e\,,
%\label{e6-k}
\end{multline*}
где $\Theta_\alpha^n$, $n=0,\ldots , N$,~--- приближенное значение вектора 
состояния~$\Theta_\alpha(\hat{t})$, $\alpha=i,e$, в момент времени $\hat{t}\hm=\hat{t}_n$, 
$\hat{t}_n\hm=n h_\tau$, $n=0,\ldots , N$; $h_\tau$~--- достаточно малый шаг интегрирования; $\Delta 
W_n$, $n=0,\ldots ,N$,~--- величина приращения винеровского процесса~$W(\hat{t})$ на отрезке $\left[ 
\hat{t}_n,\,\hat{t}_{n+1}\right]$, по определению независимая от~$\Theta_\alpha^0$, 
$\Delta W_0,\ldots , 
\Delta W_{n-1}$: $\Delta W_n\hm=W(\hat{t}_{n-1})\hm-W(\hat{t}_n)$; $\Delta W_n\hm\sim N(0,\,h_\tau)$, 
т.\,е.\ $\Delta W_n$ представляют собой гауссовские случайные величины с нулевыми математическими 
ожиданиями и дисперсиями, равными шагу интегрирования; $\Theta_\alpha^0$~--- значение вектора 
состояния $\Theta_\alpha(\hat{t})$, $\alpha\hm=i,e$, в момент времени $\hat{t}=0$, 
$\Theta_\alpha^0\hm\sim \hat{f}_\alpha^{\mathrm{maksv}}$. 

Частные производные $\partial^2\hat{g}_\alpha/\partial[\hat{v}_y]^2$ и $\partial \hat{h}_\alpha/\partial 
\hat{v}_y$, являющиеся составляющими матрицы $\sigma_\alpha (\hat{t}_n, 
\Theta_\alpha^n)\sigma_\alpha^{\mathrm{T}}(\hat{t}_n,\Theta_\alpha^n)$ и вектора $a_\alpha(\hat{t}_n, 
\Theta_\alpha^n)$ соответственно, аппроксимируются со вторым порядком точности на трехточечном 
шаблоне на основе значений~$\hat{g}_\alpha$ и~$\hat{h}_\alpha$~\cite{17-k}.
      
      В выражения для функций~$\hat{g}_\alpha$ и~$\hat{h}_\alpha$ входят интегралы, которые 
вычисляются методом Мон\-те-Кар\-ло с использованием набора значений скоростной компоненты 
вектора состояния~$\hat{v}_y$, полученных из решения СДУ Ито:
      \begin{equation*}
      \int \hat{f}_\alpha \left\vert \hat{v}_y-
\hat{v}_y^\prime\right\vert\,dv_y^\prime=M\left(\zeta\left(\hat{V}_y\right)\right)\,,
\end{equation*}
где
$$
      \zeta\left(\hat{V}_y\right)=\left\vert \hat{v}_y-\hat{V}_y\right\vert\,,\enskip \hat{V}_y\sim 
\hat{f}_\alpha\,.
  $$
      
      Для вычисления напряженности самосогласованного электрического поля $\hat{E}_y=-
\partial\hat{\varphi}/\partial\hat{y}$, входящей в вектор $a_\alpha(\hat{t}_n, \Theta_\alpha^n)$, необходимо 
аналогично аппроксимировать со вторым порядком точности производную 
$\partial\hat{\varphi}/\partial\hat{y}$ на трехточечном шаблоне с использованием значений 
потенциала~$\hat{\varphi}$~\cite{17-k}. Значения потенциала~$\hat\varphi$ находятся из решения 
уравнения Пуассона. 
      
      Граничную задачу для уравнения Пуассона 
      \begin{align*}
      \fr{\partial^2 \hat\varphi}{\partial \hat{y}^2} & = -\left(\hat{n}_i-\hat{n}_e\right)\,;\\
      \hat{\varphi}\big|_{\hat{y}=0} &=\hat{\varphi}_p\,;\\
      \hat{\varphi}\big|_{\hat{y}_\infty=0} &=0
      \end{align*}
    предлагается решать путем перехода к конечно-разностной системе с последующим ее решением 
методом прогонки~\cite{17-k}:

\noindent
\begin{gather*}
\hat{\varphi}^n_{l-1}+2\hat{\varphi}_l^n+\hat{\varphi}^n_{l+1}=
h_y\hat{\delta}_l^n\,,\enskip l=1,\ldots , 
N_y\,;\\
\hat{\delta}_l^n=-\left( \hat{n}^n_{i,l}-\hat{n}^n_{e,l}\right)\,;\enskip 
\hat{\varphi}_0=\hat{\varphi}_p\,;\enskip \hat{\varphi}_{N_y}=0\,,
\end{gather*}
где $N_y$~--- число шагов по переменной~$\hat{y}$, $h_y$~--- величина шагов разбиения по~$\hat{y}$. 
      
      Концентрации $\hat{n}_\alpha$, $\alpha=i,e$, и плотности токов частиц на зонд~$\hat{f}_\alpha$, 
$\alpha=i,e$, вычисляются согласно описанному выше методу Мон\-те-Карло.

\section{Применение метода расщепления и~метода крупных~частиц}

Решение задачи в данном случае предлагается начать с записи правой части уравнения 
Фок\-ке\-ра--План\-ка в декартовой системе координат в виде:
$$
\mathbf{Q} f_\alpha = \fr{1}{2}\,\fr{\partial^2 f_\alpha}{\partial [v_y]^2}\,\fr{\partial^2 g_\alpha}{\partial 
[v_y]^2}+\fr{\partial f_\alpha}{\partial v_y}\,\fr{\partial C_\alpha}{\partial v_y}+H_\alpha\,,\enskip 
\alpha=i,e\,,
$$  
где 
\begin{align*}
C_\alpha(\vec{r},\vec{v},t)&=
\begin{cases}
\fr{1-\gamma}{Z_i^2}\int\fr{f_e(\vec{r},{\vec{v}}^{\,\prime},t)}{|\vec{v}-{\vec{v}}^{\,\prime} |}\,d{\vec{v}}^{\,\prime}\,, 
&\alpha=i\,;\\[9pt]
\fr{Z_i^2(\gamma-1)}{\gamma}\int \fr{f_i(\vec{r},{\vec{v}}^{\,\prime}, t)}
{|\vec{v}-{\vec{v}}^{\,\prime} 
|}\,d{\vec{v}}^{\,\prime}\,, &\alpha=e\,;
\end{cases} 
\\
H_\alpha&=
\begin{cases}
4\pi \left( \fr{\gamma f_e}{Z_i^2}+f_i\right)f_i\,, & \alpha=i\,;\\[9pt]
4\pi\left(\fr{Z_i^2 f_i}{\gamma}+f_e\right)f_e\,, &\alpha=e\,.
\end{cases}
\end{align*}
Тогда при переходе к безразмерным величинам (см.\ разд.~3) система~(\ref{e1-k}) запишется 
следующим образом:
      \begin{equation}
      \left.
\!\!\begin{array}{l}
      \fr{\partial 
\hat{f}_\alpha}{\partial\hat{t}}+A_\alpha\fr{\partial\hat{f}_\alpha}{\partial\hat{y}}+
B_\alpha  \hat{E}_y
\fr{\partial\hat{f}_\alpha}{\partial\hat{v}_\alpha}=\tilde{\mathbf{Q}}\hat{f}_\alpha\,,\enskip 
\alpha=i,e;\\[9pt]
      \fr{\partial^2\hat{\varphi}}{\partial\hat{y}^2}=-\left( \hat{n}_i-\hat{n}_e\right)\,,\enskip \hat{E}_y=-
\fr{\partial\hat\varphi}{\partial\hat{y}}\,,\\[9pt]
\hspace*{3.1mm}\hat{t}=0:\ \hspace*{2.6mm}\hat{f}_\alpha(\hat{y},\hat{v}_y, 0)=\hat{f}_\alpha^{\mathrm{maksv}}\,,\enskip \alpha=i,e\,,\\[9pt]
\hspace*{2.9mm} \hat{y}=0:\ \hspace*{2.8mm}\hat{f}_\alpha(0,\hat{v}_y,\hat{t})=0\,,\enskip \alpha=i,e\,;\\[9pt]
\hspace*{24.3mm}\hat\varphi(0,\hat{t})=\hat{\varphi}_p\,;\\[9pt]
      \hat{y}=\hat{y}_\infty:\ \hat{f}_\alpha(\hat{y}_\infty, 
\hat{v}_y,\hat{t})=\hat{f}_\alpha^{\mathrm{maksv}}\,,\enskip \alpha=i,e\,;\\[9pt]
\hspace*{21.5mm}\hat{\varphi}(\hat{y}_\infty,\hat{t})=0\,,\\[9pt]
    \end{array}
\right\}\!\!
\label{e7-k}
\end{equation}
где 
\begin{gather*}
\tilde{\mathbf{Q}} \hat{f}_\alpha=D_\alpha\fr{\partial^2\hat{f}_\alpha}{\partial 
[\hat{v}_y]^2}+K_\alpha\fr{\partial\hat{f}_\alpha}{\partial\hat{v}_y}+H_\alpha\,;\\
D_\alpha=A_g^\alpha\fr{\partial^2\hat{g}_\alpha}{\partial [\hat{v}_y]^2}\,;\enskip 
K_\alpha=A_h^\alpha \fr{\partial \hat{h}_\alpha}{\partial\hat{v}_y}\,,\ \alpha=i,e\,.
\end{gather*}

Для решения системы уравнений~(\ref{e7-k}) применяется модификация метода 
расщепления~\cite{17-k}, согласно которой исходная задача разбивается на две вспомогательные. Такое 
разбиение можно осуществить, переписав уравнение Фок\-ке\-ра--План\-ка в следующем виде:
$$
\fr{\partial\hat{f}_\alpha}{\partial\hat{t}} =
\tilde{\mathbf{Q}}_1\hat{f}_\alpha+\tilde{\mathbf{Q}}_2\hat{f}_\alpha\,,
$$
где 
\begin{align*}
\tilde{\mathbf{Q}}_1\hat{f}_\alpha &=-
\left(A_\alpha\fr{\partial\hat{f}_\alpha}{\partial\hat{y}}+
B_\alpha\fr{\partial\hat{f}_\alpha}{\partial\hat{y}}
\right)\,;\\
\tilde{\mathbf{Q}}_2\hat{f}_\alpha 
&=\left(D_\alpha\fr{\partial^2\hat{f}_\alpha}{\partial[\hat{v}_y]^2}+K_\alpha\fr{\partial 
\hat{f}_\alpha}{\partial\hat{v}_y}+H_\alpha\right)\,.
\end{align*}

      Правая часть уравнения Фок\-ке\-ра--План\-ка представляет собой сумму двух операторов, 
первый из которых отвечает за перенос частиц, второй~--- за столкновения заряженных частиц. 
В~результате образуются следующие задачи, которые решаются последовательно:
      \begin{itemize}
\item первая задача:
\begin{align*}
&\fr{\partial w_\alpha(\hat{y},\hat{v}_y,\hat{t})}{\partial\hat{t}} =\mathbf{Q}_1 
w_\alpha(\hat{y},\hat{v}_y,\hat{t})\,,\enskip \alpha=i,e\,;\\[9pt]
&\fr{\partial^2\hat\varphi}{\partial\hat{y}^2}=-\left(\hat{n}_i-\hat{n}_e\right)\,;\enskip
\hat{E}_y=-
\fr{\partial\hat\varphi}{\partial\hat{y}}\,;\\[9pt]
&w_\alpha(\hat{y},\hat{v}_y,\hat{t}^n)=\hat{f}_\alpha(\hat{y},\hat{v}_y,\hat{t}^n)\,,\enskip n=0,\ldots ,N-
1\,;\\[9pt]
&\hspace{2.9mm}\hat{y}=0:\ \hspace*{2.9mm}w_\alpha(0,\hat{v}_y,\hat{t})=0\,,\enskip \alpha=i,e\,;\\[9pt]
&\hspace*{25.1mm}\hat\varphi(0,\hat{t})=\hat{\varphi}_p\,;\\[9pt]
&\hat{y}=\hat{y}_\infty:\ w_\alpha(\hat{y}_\infty, \hat{v}_y, \hat{t})=
\hat{f}_\alpha^{\mathrm{maksv}}\,,\enskip 
\alpha=i,e\,;\\[9pt]
&\hspace*{22.5mm}\hat\varphi(\hat{y}_\infty,\hat{t})=0\,;
\end{align*}
\item вторая задача:
\begin{align*}
\!\!\!\!\!\!\!\fr{\partial s_\alpha(\hat{y},\hat{v}_y,\hat{t})}{\partial \hat{t}} &=\mathbf{Q}_2 
s_\alpha(\hat{y},\hat{v}_y,\hat{t})\,, & \alpha&=i,e\,;\\
\!\!\!\!\!\!\!s_\alpha (\hat{y},\hat{v}_y,\hat{t}^n) &=w_\alpha (\hat{y},\hat{v}_y, \hat{t}^{n+1}),& n&=0,\ldots ,N-
1.
\end{align*}
\end{itemize}

Первая задача представляет собой систему безразмерных уравнений Вла\-со\-ва--Пуас\-со\-на. Для ее 
решения применяется метод крупных частиц~\cite{18-k}. Согласно этому методу решение задачи 
осуществляется путем расщепления на два этапа: на первом этапе не учитываются конвективные члены 
и решение получается обычным интегрированием на неподвижной эйлеровой сетке, а на втором этапе 
рассматривается система, которая описывает перенос частиц в лагранжевой системе координат. Кроме 
того, на первом этапе необходимо решить уравнение Пуассона для получения значений потенциала 
самосогласованного электрического поля. Для этого применяется метод, описанный в разд.~3. 

Вторая задача решается путем перехода к ко\-неч\-но-раз\-ност\-ной сис\-те\-ме. При этом частные 
производные $\partial^2\hat{g}_\alpha/\partial[\hat{v}_y]^2$ и $\partial\hat{h}_\alpha/\partial\hat{v}_y$ 
аппроксимируются со вторым порядком точности с использованием трехточечного шаблона, а 
производная $\partial s_\alpha/\partial\hat{t}$ аппроксимируется на двухточечном шаблоне с первым 
порядком точности~\cite{16-k}. К~полученной системе разностных уравнений предлагается применить 
один из классических методов решения систем линейных уравнений, например метод 
Гаусса~\cite{19-k}.
      
      Решением первой задачи является функция $w_\alpha(\hat{y}, \hat{v}_y, \hat{t}^n)$, 
$n\hm=0,\ldots ,N$, , которая дает начальное условие для второй задачи. Решая вторую задачу, находим 
функцию $s_\alpha(\hat{y},\hat{v}_y,\hat{t}^n)\hm=\hat{f}_\alpha(\hat{y},\hat{v}_y,\hat{t}^n)$, 
$n=1,\ldots ,N$, $\alpha=i,e$, которая определяет решение $\hat{f}_\alpha(\hat{y},\hat{v}_y,\hat{t}^n)$, 
$\alpha=i,e$, исходной системы~(\ref{e7-k}) для рассматриваемых моментов времени $n=1,\ldots ,N$.

Моменты функций распределения $\hat{f}_\alpha$, $\alpha=i,e$, находятся с помощью методов 
численного интегрирования, например метода трапеций~\cite{19-k}.

\section{Результаты численного моделирования}

Для двух описанных выше методов реализованы две отдельные программы в среде {Matlab~7.0}. 
Эти программы позволяют по заданным значениям концентраций и температур частиц $n_{i\infty}$, 
$n_{e\infty}$, $T_{i\infty}$ и~$T_{e\infty}$ в невозмущенной плазме, а также потенциала~$\varphi_p$, 
подаваемого на зонд, изучить эволюцию во времени плотностей тока частиц~$j_i$ и~$j_e$, концентраций 
частиц~$n_i$  и~$n_e$ в произвольной точке пространства в возмущенной зоне, а также динамику 
изменения напряженности~$E_y$ самосогласованного электрического поля во времени и пространстве.

С использованием разработанных программ проведены серии расчетных экспериментов, в которых 
значение концентраций варьировалось в пределах $n_{i\infty} \hm = n_{e\infty}\hm =10^{18}\div 
10^{22}$~м$^{-3}$. Значение температур было выбрано неизменным и равным $T_{i\infty}\hm = 
T_{e\infty}\hm=3000$~K, а значения потенциала, подаваемого на зонд, изменялись в пределах 
$\varphi_p\hm=0\div 2{,}6$~В.

На рис.~1  и~2 приведены графики изменения напряженности самосогласованного электрического
 поля (см.\ рис.~1) и плотности токов ионов (см.\linebreak\vspace*{-12pt}

\pagebreak

\end{multicols}

\begin{figure} %fig1
\vspace*{1pt}
\begin{center}
\mbox{%
\epsfxsize=162.594mm
\epsfbox{kud-1.eps}
}
\end{center}
\vspace*{-9pt}
\Caption{Динамика изменения плотности тока ионов во времени в фиксированной точке возмущенной 
зоны для значений потенциала: \textit{1}~--- $\varphi_p=-6$; 
\textit{2}~--- $\varphi_p=-16$; \textit{3}~--- $\varphi_p=- 30$ 
в случае применения методов Монте-Карло~(\textit{а}) 
и крупных частиц~(\textit{б})}
\end{figure}

\begin{figure} %fig2
\vspace*{1pt}
\begin{center}
\mbox{%
\epsfxsize=162.713mm
\epsfbox{kud-2.eps}
}
\end{center}
\vspace*{-9pt}
\Caption{Динамика изменения напряженности электрического поля во времени в фиксированной точке 
возмущенной зоны для значений потенциала: 
\textit{1}~--- $\varphi_p=-6$; \textit{2}~--- $\varphi_p=-16$; 
\textit{3}~--- $\varphi_p=-30$ в случае применения методов Монте-Карло~(\textit{а}) и
крупных частиц~(\textit{б})
}
\end{figure}

\begin{multicols}{2}

\noindent
 рис.~2) во времени в фиксированной точке пространства 
возмущенной зоны в случае применения обоих разработанных алгоритмов.


На основании полученных результатов можно отметить похожее поведение зависимостей 
напряженности электрического поля и плотности тока от времени в двух рассматриваемых случаях. 
Графики кривых сначала убывают, затем начинают возрастать, выходя в некоторый момент 
времени~$t^\prime$ (момент установления) на стационарные значения. 

Одинаковое поведение 
напряженности и плот\-ности тока можно объяснить из следующих соображений: плотность тока ионов в 
данной области пространства равна произведению концентрации ионов на их направленную скорость и 
на заряд иона. Скорость ионов, в свою очередь, зависит от заряда, массы и напряженности 
электрического поля. 
%\columnbreak

При внесении в плазму отрицательно заряженного зонда возникает электрическое поле, которое 
нарушает квазинейтральность плазмы. Для того чтобы компенсировать действие внешнего 
электрического поля, ионы устремляются к зонду, а электроны~--- от зонда. Это приводит к дисбалансу 
концентраций вблизи зонда и, как следствие, к увеличению разности потенциалов; график 
напряженности электрического поля убывает. Вскоре разделение зарядов компенсирует внешнее 
электрическое поле; график выходит на стационарное значение. 

Также можно отметить, что значения 
напряженности электрического поля и плотности тока частиц на зонд в момент установления для двух 
методов совпадают. 

Момент установления~$t^\prime$ зависит от при\-ме\-ня\-емо\-го метода решения. В~случае метода 
Мон\-те-Кар\-ло $t^\prime=3{,}5\div 4$~ед., а для метода крупных частиц совместно с методом 
расщепления $t^\prime\hm=5\div 5{,}5$~ед. Используя ко\-неч\-но-раз\-ност\-ный метод, можно 
получить динамику изменения функций распределения частиц~$f_\alpha$, $\alpha=i,e$, во времени и 
пространстве. Функции распределения позволяют наглядно представить влияние на картину 
распределения частиц вблизи зонда самой поверхности зонда и электрического поля.

\section{Заключение}
      
      В работе найдено решение задачи диагностики плоским зондом сильноионизованной плазмы с 
учетом столкновений заряженных частиц. Разработана математическая модель исследуемого явления, 
описываемая уравнениями Фок\-ке\-ра--План\-ка и Пуассона. Решение получено двумя методами:\linebreak 
статистическим и ко\-неч\-но-раз\-ност\-ным на основе\linebreak сформированных алгоритмов. Приведены 
резуль-\linebreak таты численного моделирования при различных\linebreak характерных параметрах задачи.
 Из  проведенных 
вычислительных экспериментов вытекает, что искомые величины: напряженность 
электрического поля, плотности токов частиц на зонд, концентрации частиц вблизи зонда~--- как по 
характеру зависимости, так и по числовым значениям совпадают. При применении метода 
      Мон\-те-Кар\-ло момент установления наступает быстрее по сравнению с конечно-разностным 
методом, однако конечно-разностный метод позволяет получить более наглядные результаты.

{\small\frenchspacing
{%\baselineskip=10.8pt
\addcontentsline{toc}{section}{Литература}
\begin{thebibliography}{99}

\bibitem{1-k}
\Au{Alexeff I., Anderson T.}
Experimental and theoretical results with plasma antenna~// IEEE Trans. Plasma Sci., 2006. Vol.~34. 
No.\,2. P.~166--172.

\bibitem{2-k}
\Au{Сысун В.\,И.}
Сильноионизованная низкотемпературная плазма в приборах электронной техники: Методы 
исследования, свойства, применение. Дисс. \ldots д-ра физ.-мат. наук в форме науч. докл.: 
01.04.08.~--- Пет\-ро\-за\-водск, 1996.

\bibitem{3-k}
\Au{Тухас В.\,А.}
Методология создания средств измерений и испытаний на устойчивость к кондуктивным помехам~// 
Мат-лы VI Междунар. симп. по электромагнитной совместимости и 
электромагнитной экологии.~--- СПб., 2005. С.~231--234.

\bibitem{4-k}
\Au{Гудзенко Л.\,И., Яковленко С.\,И.}
Плазменные лазеры.~--- М.: Атомиздат, 1978.  256~с.

\bibitem{5-k}
\Au{Звелто О.}
Принципы лазеров.~--- М.: Мир, 1990.  560~с.

\bibitem{6-k}
\Au{Сысун В.\,И., Хромой Ю.\,Д.}
Расширение канала мощного импульсного разряда в парах ртути~// Электронная техника, 1974. 
Сер.~4. Вып.~10. С.~80--85. 

\bibitem{7-k}
\Au{Винклер Дж.\,Р.}
Искусственные пучки частиц в космической плазме.~--- М.: Мир, 1985.  451~с.

\bibitem{8-k}
\Au{Bernstein I.\,B., Rabinowitz I.\,N.}
Theory of electrostatic probes in low-density plasma~// Phys. Fluids, 1959. Vol.~2. No.\,2. P.~112--121. 

\bibitem{9-k}
\Au{Альперт Я.\,Л., Гуревич А.\,В., Питаевский~Л.\,П.}
Искусственные спутники в разреженной плазме.~--- М.: Наука, 1964.  282~с.

\bibitem{10-k}
\Au{Чан П., Тэлбот Л., Турян~К.}
Электрические зонды в неподвижной и движущейся плазме.~--- М.: Мир, 1978.  202~с.

\bibitem{11-k}
\Au{Алексеев Б.\,В., Котельников В.\,А.}
Зондовый метод диагностики плазмы.~--- М.: Энергоатомиздат, 1989.  240~с.

\bibitem{12-k}
\Au{Пантелеев А.\,В., Кудрявцева И.\,А.}
Формирование математической модели двухкомпонентной плазмы с учетом столкновений 
заряженных частиц в случае плоского зонда~// Теоретические вопросы вычислительной техники и 
программного обеспечения: Межвузовский сб. научн. тр.~--- М.: МИРЭА, 2006. С.~11--21.

\bibitem{13-k}
\Au{Олдер Б.}
Вычислительные методы в физике плазмы.~--- М.: Мир, 1974.  111~с.

\bibitem{14-k}
\Au{Montgomery D.\,C., Tidman D.\,A.}
Plasma kinetic theory.~--- New York, 1964. 

\bibitem{15-k}
\Au{Кудрявцева И.\,А., Пантелеев А.\,В.}
Применение метода Мон\-те-Кар\-ло для анализа поведения двухкомпонентной плазмы с учетом 
столкновений между заряженными частицами~// Теоретические вопросы\linebreak
вычислительной техники и 
программного обеспечения: Межвузовский сб. научн. тр.~--- М.: МИРЭА, 2008. С.~122--128. 

\bibitem{16-k}
\Au{Семенов В.\,В., Пантелеев А.\,В., Руденко~Е.\,А., Бор\-та\-ков\-ский~А.\,С.}
Методы описания, анализа и синтеза нелинейных систем управления.~--- М.: МАИ, 1993.  312~с.

\bibitem{17-k}
\Au{Киреев В.\,И., Пантелеев А.\,В.}
Численные методы в примерах и задачах.~--- М.: Высшая школа, 2006.  480~с.

\bibitem{18-k}
\Au{Белоцерковский О.\,М., Давыдов~Ю.\,М.}
Метод крупных частиц в газовой динамике. Вычислительный эксперимент.~--- М.: Наука, 
Физматгиз, 1982.

\label{end\stat}

\bibitem{19-k}
\Au{Вержбицкий В.\,М.}
Основы численных методов.~--- М.: Высшая школа, 2002.  840~с.
 \end{thebibliography}
}
}


\end{multicols}            %12
\def\stat{gr+zab}

\def\tit{ФОРМИРОВАНИЕ КОНЦЕПТОВ НА~ОСНОВЕ МАЛЫХ ВЫБОРОК$^*$}

\def\titkol{Формирование концептов на основе малых выборок}

\def\aut{А.\,А.~Грушо$^1$, М.\,И.~Забежайло$^2$, Н.\,А.~Грушо$^3$, 
Е.\,Е.~Тимонина$^4$}

\def\autkol{А.\,А.~Грушо, М.\,И. Забежайло, Н.\,А.~Грушо, 
Е.\,Е.~Тимонина}

\titel{\tit}{\aut}{\autkol}{\titkol}

\index{Грушо А.\,А.}
\index{Забежайло М.\,И.}
\index{Грушо Н.\,А.} 
\index{Тимонина Е.\,Е.}
\index{Grusho A.\,A.}
\index{Zabezhailo M.\,I.}
\index{Grusho N.\,A.}
\index{Timonina E.\,E.}



{\renewcommand{\thefootnote}{\fnsymbol{footnote}} \footnotetext[1]
{Работа частично поддержана РФФИ (проект 18-29-03081).}}


\renewcommand{\thefootnote}{\arabic{footnote}}
\footnotetext[1]{Институт проблем информатики Федерального исследовательского центра <<Информатика и~управление>> 
Российской академии наук, \mbox{grusho@yandex.ru}}
\footnotetext[2]{Институт проблем информатики Федерального исследовательского центра <<Информатика и~управление>> 
Российской академии наук, m.zabezhailo@yandex.ru}
\footnotetext[3]{Институт проблем информатики Федерального исследовательского центра <<Информатика и~управление>> 
Российской академии наук, info@itake.ru}
\footnotetext[4]{Институт проблем информатики Федерального исследовательского центра <<Информатика и~управление>> 
Российской академии наук, eltimon@yandex.ru}

\vspace*{-12pt}
  
  
  \Abst{Системы мониторинга информационной безопасности ин\-фор\-ма\-ци\-он\-но-вы\-чис\-ли\-тель\-ных систем получают информацию в~виде цепочек коротких сообщений, которые 
можно считать цепочками малых выборок. Часто в~силу инерционности информационных 
систем эти цепочки отражают близкие состояния вычислительной системы или сети. 
Предполагается, что работу системы можно представить в~виде конечного набора режимов, 
которые называются концептами. Нарушения безопасности выявляются с~помощью аномалий, 
которые ассоциируются с~появлением новых концептов. 
  Известные технологии выявления аномалий основаны на построении модели нормального 
поведения системы. Концепты соответствуют нормальным типам поведения системы. В~работе 
рассмотрена задача построения концептов на основе машинного обучения, опирающегося на 
цепочки малых выборок. Построен алгоритм формирования концептов и~доказана его 
эффективность.}
   
  \KW{мониторинг информационной безопасности; малые выборки; обучение на малых 
выборках; формирование концептов}

\DOI{10.14357/19922264190413} 
  
%\vspace*{1pt}


\vskip 10pt plus 9pt minus 6pt

\thispagestyle{headings}

\begin{multicols}{2}

\label{st\stat}
  
  
\section{Введение }

  Многие системы мониторинга информационной безопасности~[1, 2] и~других 
аспектов работы ин\-фор\-ма\-ци\-он\-но-вы\-чис\-ли\-тель\-ных систем получают 
информацию в~виде коротких сообщений, которые можно считать малыми 
выборками. В~силу инерционности информационных систем часто эти 
сообщения поступают сериями, отражая близкие состояния вычислительной 
системы или сети. 
  
  Целью работы мониторинговых систем ставится выявление аномалий в~работе 
отслеживаемых объектов. Известны технологии выявления аномалий, основанные 
на построении моделей нормального поведения~[3]. Однако поступающие 
сообщения не всегда имеют простую структуру~[4]. Не всегда методы 
регрессии~[3] можно применять: например, когда сеть изменяет свое поведение, 
происходит изменение многих параметров функционирования сети. Если 
устройство демонстрирует несколько режимов работы, их описание необходимо 
строить на основе анализа поступающих малых выборок с~помощью процедур 
машинного обучения. 
  
  В последнее время методы машинного обучения получили большое развитие 
(см., например,~[5, 6]). Методы машинного обучения на основе малых выборок 
также подробно изучались~[7]. Один из главных сценариев в~таком обучении 
основан на Concept Learning. Цель этого подхода состоит в~распознавании 
концептов по небольшому числу малых выборок на основе ранее наблюденных 
концептов. Вторая цель этого подхода состоит в~формировании множества 
концептов. В~дальнейшем будет использована терминология теории обучения на 
малых выборках, где под концептами понимаются классы выборок, 
принадлежность к~которым необходимо определять для вновь поступающих 
малых выборок. 
  
  Далее будем предполагать, что данные поступают с~помощью цепочек малых 
выборок. Каждая цепочка однозначно связана с~некоторым кон\-цеп\-том. При этом 
чис\-ло кон\-цеп\-тов неизвестно, но оно конечно. Каждый концепт будет описываться 
множеством выборок. Как отмечалось в~обзоре~[7], наиболее сложная задача 
состоит в~формировании концептов. 
  
  В статье построен и~описан алгоритм формирования концептов и~доказана его 
эффективность.

\vspace*{-6pt}
  
  \section{Математическая модель}
  
  \vspace*{-2pt}
  
  Будем считать, что каждая малая выборка есть слово длины~$N$ в~алфавите 
из~$m$~букв. Каждая цепочка малых выборок конечна, и~для простоты все 
цепочки имеют одинаковую длину~$n$. Концепты формируются с~помощью 
кластеров. 
  
  Цель работы~--- построение корректного алгоритма определения числа 
концептов и~самих концептов. 
  
  Примем следующие условия.\\[-14pt]
  \begin{enumerate}[1.]
\item Каждая малая выборка относится к~одному и~только к~одному концепту.\\[-14pt]
\item Концепты не пересекаются между собой.\\[-14pt]
\item Любая цепочка малых выборок относится только к~одному концепту.\\[-14pt]
  \end{enumerate}
  
  Поскольку концепты будут формироваться постепенно на основании текущей 
кластерной структуры, то все изолированные кластеры будем называть 
\textit{промежуточными концептами}. Каждый промежуточный концепт состоит из:\\[-14pt]
  \begin{itemize}
\item \textit{видимого концепта}, т.\,е.\ малых выборок, которые в~него 
вошли;\\[-14pt] 
\item \textit{невидимого концепта}, т.\,е.\ малых выборок, которые можно было 
бы отнести к~данному промежуточному концепту, но они ранее не встретились;\\[-14pt] 
\item \textit{запретов}, т.\,е.\ малых выборок, которые в~принципе не могут 
входить в~данных концепт.\\[-14pt]
\end{itemize}

  Из сделанных ранее предположений вытекают следующие выводы.\\[-14pt] 
  \begin{enumerate}[1.]
\item Если в~цепочке есть хотя бы одна малая выборка из существующего 
промежуточного концепта, то вся цепочка относится к~этому промежуточному 
концепту, хотя почти все ее элементы могут быть невидимыми для данного 
промежуточного концепта.\\[-14pt]
\item Если в~цепочке встретились по крайней мере две малые выборки, 
принадлежащие разным промежуточным концептам, то эти два промежуточных 
концепта объединяются в~единый промежуточный концепт. При этом остальные 
элементы цепочки принадлежат этому объединенному промежуточному 
концепту.\\[-14pt]
\item Если в~полученной цепочке нет ни одной малой выборки, принадлежащей 
одному из существующих промежуточных концептов, то такая цепочка образует 
новый промежуточный концепт. При этом надо помнить, что эта цепочка может 
состоять из невидимых элементов какого-то существующего промежуточного 
концепта.\\[-14pt] 
\end{enumerate}
  
  Указанные шаги~1--3 фактически формируют алгоритм обработки цепочек 
малых выборок и~преобразования промежуточных концептов. 
  
  Предположим, что существует некоторое семейство концептов $M_1,\ldots , 
M_k$ такое, что каждая малая выборка из пространства возможных малых 
выборок принадлежит одному из этих концептов. 
  
  Покажем, что предложенный алгоритм позволяет определить число концептов 
и~определить их содержание. Для простоты будем считать, что $n\hm= 2$. Выбор 
цепочек осуществляется случайно следующим образом. Сначала выбирается 
концепт, из которого выбирается цепочка для простоты в~соответствии 
с~равномерным распределением на множестве целых чисел $\{1,\ldots , k\}$. По 
условию каждая цепочка выбирается из одного концепта. 
  
  Обозначим через $\vert M_i\vert \hm=s_i$, $i\hm=1,\ldots , k$. При равномерном 
выборе малой выборки из кон\-цеп\-та~$M_i$ получим, что вероятность 
$$
{\sf P}\left(x_1,x_2\right)=\fr{1}{s_i(s_i-1)}\,,
$$
 где $x_, x_2\hm\in M_i$. Тогда $1\hm- 1/(s_i(s_i\hm-
1))$~--- вероятность того, что данная цепочка не встретится на фиксированном 
месте в~последовательности выбора цепочек из~$M_i$, $i\hm=1,\ldots , k$. 
  
  Пусть в~последовательности длины~$t$ выбранных из~$M_i$ цепочек ни разу 
не встретится цепочка $(x_1,x_2)$. Вероятность такого события равна $(1-
1/(s_i(s_i\hm-1)))^t$. 
  
  Из леммы Бореля--Кан\-тел\-ли и~полученных выше оценок следует, что 
в~бесконечной последовательности выборок из множества~$M_i$ появление 
цепочки $(x_1,x_2)$ произойдет бесконечное число раз. Кроме того, из той же 
леммы следует, что с~вероятностью~1 существует бесконечная 
последовательность появления концепта~$M_i$ в~указанной выше вероятностной 
схеме. 
  
  Рассмотрим бесконечную схему преобразования данных в~кластеры с~целью 
построения концептов. Пусть время дискретно и~в данный момент сформированы 
кластеры $K_1,\ldots ,K_r$ промежуточных концептов. Пусть получена очередная 
цепочка $(x_1,x_2)$. Тогда:
  \begin{enumerate}[(1)]
\item если~$x_1$ и~$x_2$ принадлежат некоторому клас\-те\-ру~$K_i$, то 
кластерная структура не изменяется; 
\item если один элемент $x_1$ или~$x_2$ ранее не встречался, а~второй 
элемент принадлежит промежуточному концепту~$K_i$, то клас\-тер~$K_i$ 
увеличивается на один ранее не встречавшийся элемент; 
\item если элемент~$x_1$ принадлежит некоторому клас\-те\-ру~$K_i$, 
а~элемент~$x_2$ принадлежит некоторому кластеру~$K_j$, $i\not= j$, то 
в~новой клас\-тер\-ной структуре вместо кластеров~$K_i$ и~$K_j$ появляется 
новый кластер $K_i\cup K_j$;
\item если ни один из элементов цепочки $(x_1,x_2)$ ранее не встречался 
и~не принадлежит ни одному кластеру, то элементы этой цепочки образуют 
новый кластер. 
\end{enumerate}
  
  Пусть первая цепочка, полученная для по\-стро\-ения концептов,~--- $(x_1, x_2)$, 
где $x_1, x_2\hm\in M_1$. Тогда для каждого элемента $x_3\hm\in M_1$ пара 
$(x_1, x_3)$ встречается с~ве\-ро\-ят\-н\-остью~1. Поскольку концепт~$M_1$ по 
определению конечен, то элемент~$x_1$ встретится в~сочетании со всеми 
элементами~$M_1$ с~ве\-ро\-ят\-ностью~1. Таким образом, концепт~$M_1$ будет 
однозначно восстановлен. Так как каждый из концептов выбирается 
в~бесконечной последовательности бесконечное число раз, то с~вероятностью~1 
будут восстановлены все другие концепты. 
  
  Докажем, что полученная структура не может быть противоречивой. Пусть 
цепочка $(x_1, x_2)$ такова, что $x_1\hm\in M_1$, а~$x_2\hm\in M_2$. Это 
противоречит условию, что каждая цепочка выбирается из одного концепта. 
  
  Докажем, что ни одна выборка~$x_1$ не может быть пропущена в~результате 
работы алгоритма. Пусть элемент~$x_2$ принадлежит тому же концепту, что 
и~$x_1$. Тогда, как было показано выше, цепочка $(x_1, x_2)$ появляется 
с~вероятностью~1 в~последовательности цепочек из~$M_1$. Если при этом 
известно, что $x_2\hm\in M_1$, то и~$x_1\hm\in M_1$, т.\,е.~$x_1$ не может быть 
пропущен.

\vspace*{-6pt}
  
 \section{Эффективность алгоритма построения концептов}
 
 \vspace*{-2pt}
  
  Рассмотрим задачу определения того, что клас\-тер\-ная структура соответствует 
структуре концептов. Определим граф~$G$ с~ребрами $(x_i, x_j)$, которые 
соответствуют появившимся цепочкам малых выборок. После того как все 
концепты~$M_i$, $i\hm=1,\ldots ,k$, определены, граф~$G$ представляет 
собой~$k$~компонентов связности, каждый их которых является полным графом. 
Таким образом, появление полных графов в~кластерной структуре компонентов 
связности служит признаком (недостаточным) того, что концепты построены. 
Изоляция кластеров, которые являются полными графами,~--- признак 
восстановления структуры концептов. 
  
  Число ребер в~графе, соответствующем концепту~$M_i$, равно 
$\begin{pmatrix} s_i\\ 2\end{pmatrix}$. Минимальное число цепочек, которое 
необходимо для появления признаков формирования концептов, равно 

\noindent
  $$
  R= \sum\limits_{i=1}^k \begin{pmatrix} s_i\\ 2\end{pmatrix}\,. 
  $$
  
  \vspace*{-2pt}
  
  Устойчивость структуры восстановленных концептов будет видна, когда число 
цепочек станет равным~$rR$, где $r\hm>1$. Таким образом, получена нижняя 
оценка сложности алгоритма восстановления концептов, и~она имеет 
квадратичный порядок.

\vspace*{-6pt}
  
  \section{Заключение }
  
  Построен алгоритм формирования концептов, не использующий семантический 
анализ содержания малых выборок. Это делает алгоритм универсальным 
в~подобных задачах. 
  
  Условие принадлежности малой выборки одному концепту можно заменить на 
меру близости, связанной с~содержанием малой выборки. Тогда при построении 
концептов возможны ошибки. Однако предложенный контроль при построении 
полного графа может скомпенсировать эти ошибки.
  
  В данной работе не рассмотрена задача снижения сложности алгоритма 
построения концептов. Возможно, что рассмотренную выше идею можно 
реализовать с~меньшей сложностью.

\vspace*{-6pt}
  
{\small\frenchspacing
 {%\baselineskip=10.8pt
 \addcontentsline{toc}{section}{References}
 \begin{thebibliography}{9}

\bibitem{2-gz}
\Au{Грушо А., Грушо~Н., Тимонина~Е., Шоргин~С.} Возможности построения безопасной 
архитектуры для динамически изменяющейся информационной системы~// Системы и~средства 
информатики, 2015. Т.~25. №\,3. С.~78--93.
\bibitem{1-gz}
\Au{Grusho A., Grusho~N., Timonina~E.}
 The bans in finite probability spaces and the problem of 
small samples~// Distributed computer and communication networks~/ Eds. V.\,M.~Vishnevskiy, K.\,E.~Samouylov, 
D.\,V.~Kozyrev.~--- Lecture notes in computer 
science ser.~--- Springer,  2019. Vol.~11965. P.~578--590.

\bibitem{3-gz}
\Au{Тьюки Дж.} Анализ результатов наблюдений. Разведочный анализ приложения~/ Пер. 
с~англ.~--- М.: Мир, 1981. 694~с. (\Au{Tukey~J.\,W.} Exploratory data analysis.~--- Addison 
Wesley, 1977. 711~р.)
\bibitem{4-gz}
\Au{Grusho A., Grusho~N., Timonina~E.} Detection of anomalies in non-numerical data~// 8th 
Congress (International) on Ultra Modern Telecommunications and Control Systems and Workshops 
Proceedings.~--- Piscataway, NJ, USA: IEEE, 2016. P.~273--276.
\bibitem{5-gz}
\Au{Jordan M.\,I., Mitchell~T.\,M.} Machine learning: Trends, perspectives, and prospects~// Science, 
2015. Vol.~349. Iss.~6245. P.~255--260.
\bibitem{6-gz}
\Au{Bramley N.\,R.} Constructing the world: Active causal learning in cognition.~--- 
London: University College London, 2017.  PhD thesis. 361~p. 
\bibitem{7-gz}
\Au{Shu~J., Zongben~X., Deyu~M.} Small sample learning in big data era~// arXiv.org, 
2018. 76~p. arXiv:1808.04572v3 [cs.LG].
 \end{thebibliography}

 }
 }

\end{multicols}

\vspace*{-6pt}

\hfill{\small\textit{Поступила в~редакцию 30.09.19}}

%\vspace*{8pt}

\pagebreak

\newpage

\vspace*{-28pt}

%\hrule

%\vspace*{2pt}

%\hrule

%\vspace*{-2pt}

\def\tit{CONCEPTS FORMING ON~THE~BASIS OF~SMALL SAMPLES}


\def\titkol{Concepts forming on~the~basis of~small samples}

\def\aut{A.\,A.~Grusho, M.\,I.~Zabezhailo, N.\,A.~Grusho, and~E.\,E.~Timonina}

\def\autkol{A.\,A.~Grusho, M.\,I.~Zabezhailo, N.\,A.~Grusho, and~E.\,E.~Timonina}

\titel{\tit}{\aut}{\autkol}{\titkol}

\vspace*{-11pt}


 \noindent
   Institute of Informatics Problems, Federal Research Center ``Computer Sciences and 
Control'' of the Russian Academy of Sciences; 44-2~Vavilov Str., Moscow 119133, 
Russian Federation

\def\leftfootline{\small{\textbf{\thepage}
\hfill INFORMATIKA I EE PRIMENENIYA~--- INFORMATICS AND
APPLICATIONS\ \ \ 2019\ \ \ volume~13\ \ \ issue\ 4}
}%
 \def\rightfootline{\small{INFORMATIKA I EE PRIMENENIYA~---
INFORMATICS AND APPLICATIONS\ \ \ 2019\ \ \ volume~13\ \ \ issue\ 4
\hfill \textbf{\thepage}}}

\vspace*{3pt}  


      
   
   \Abste{Monitoring systems of information security of information systems obtain information in 
the form of chains of short messages which can be considered as chains of small samples. Often, 
owing to an inertance of information systems, these chains reflect close statuses of the computing 
system or network. In the paper, it is supposed that work of the system can be presented in the form of 
a finite set of modes which are called concepts. Violations of security are detected by means of 
anomalies that are associated with emergence of new concepts. 
   The known technologies of identification of anomalies are based on creation of a model of a 
normal system's behavior. Concepts correspond to normal types of a~system's behavior. In the paper, 
the problem of creation of concepts on the basis of machine learning based on chains of small samples 
is considered. The algorithm of concepts forming is constructed and its efficiency is proved.} 
   
   \KWE{information security monitoring; small samples; small sample learning; concepts forming}
   
  

\DOI{10.14357/19922264190413} 

%\vspace*{-14pt}

 \Ack
   \noindent
   The paper was partially supported by the Russian Foundation for Basic Research (project  
18-29-03081).


%\vspace*{-6pt}

  \begin{multicols}{2}

\renewcommand{\bibname}{\protect\rmfamily References}
%\renewcommand{\bibname}{\large\protect\rm References}

{\small\frenchspacing
 {%\baselineskip=10.8pt
 \addcontentsline{toc}{section}{References}
 \begin{thebibliography}{9}

\bibitem{2-gz-1}
\Aue{Grusho, A., N.~Grusho, E.~Timonina, and S.~Shorgin.} 2015. Vozmozhnosti 
postroeniya 
bezopasnoy arkhitektury dlya dinamicheski izmenyayushcheysya informatsionnoy sistemy 
[Possibilities of secure architecture creation for dynamically changing information systems]. 
\textit{Sistemy i~Sredstva Informatiki~--- Systems and Means of Informatics} 25(3):78--93.

\bibitem{1-gz-1}
\Aue{Grusho, A., N.~Grusho, and E.~Timonina.} 2019. 
The bans in finite probability spaces and 
the problem of small samples. \textit{Distributed computer and communication networks}.
Eds. V.\,M.~Vishnevskiy, 
K.\,E.~Samouylov, and D.\,V.~Kozyrev. Lecture notes
in computer science ser. Springer. 11965:578--590.



\bibitem{3-gz-1}
\Aue{Tukey, J.\,W.} 1977. \textit{Exploratory data analysis}. Addison Wesley. 
711~р.
\bibitem{4-gz-1}
\Aue{Grusho, A., N.~Grusho, and E.~Timonina.} 2016. Detection of anomalies in non-numerical 
data. \textit{8th  Congress (International) on Ultra Modern Telecommunications and Control 
Systems and Workshops Proceedings}. Piscataway, NJ: IEEE. 273--276.

\vspace*{2pt}

\bibitem{5-gz-1}
\Aue{Jordan, M.\,I., and T.\,M.~Mitchell.} 2015. 
Machine learning: Trends, perspectives, 
and prospects. \textit{Science} 349(6245):\linebreak 255--260.

\vspace*{2pt}

\bibitem{6-gz-1}
\Aue{Bramley, N.\,R.} 2017. Constructing the world: 
Active causal learning in cognition.  London: University College London. PhD  Thesis. 361~p.

\vspace*{2pt}

\bibitem{7-gz-1}
\Aue{Shu, J., X.~Zongben, and M.~Deyu.} 2018. Small sample learning in big data era. Available 
at: {\sf https://arxiv.org/ abs/1808.04572} (accessed October~9, 2019).
\end{thebibliography}

 }
 }

\end{multicols}

\vspace*{-6pt}

\hfill{\small\textit{Received September 30, 2019}}

%\pagebreak

\vspace*{-22pt}

\Contr


\noindent
\textbf{Grusho Alexander A.} (b.\ 1946)~--- Doctor of Science in physics and 
mathematics, professor, principal scientist, Institute of Informatics Problems, Federal 
Research Center ``Computer Sciences and Control'' of the Russian Academy of 
Sciences; 44-2~Vavilov Str., Moscow 119133, Russian Federation; 
\mbox{grusho@yandex.ru}

\vspace*{3pt} 

\noindent
\textbf{Zabezhailo Michael I.} (b.\ 1956)~--- Doctor of Science in physics and 
mathematics, principal scientist, Institute of Informatics Problems, Federal Research 
Center ``Computer Sciences and Control'' of the Russian Academy of Sciences;  
44-2~Vavilov Str., Moscow 119133, Russian Federation; 
\mbox{m.zabezhailo@yandex.ru} 

\vspace*{3pt}

\noindent
\textbf{Grusho Nikolai A.} (b.\ 1982)~--- Candidate of Science (PhD) in physics 
and mathematics, senior scientist, Institute of Informatics Problems, Federal 
Research Center ``Computer Sciences and Control'' of the Russian Academy of 
Sciences; 44-2~Vavilov Str., Moscow 119133, Russian Federation; 
\mbox{info@itake.ru} 
 
\vspace*{3pt}

\noindent
\textbf{Timonina Elena E.} (b.\ 1952)~--- Doctor of Science in technology, professor, 
leading scientist, Institute of Informatics Problems, Federal Research Center 
``Computer Sciences and Control'' of the Russian Academy of Sciences; 44-2~Vavilov 
Str., Moscow 119133, Russian Federation; \mbox{eltimon@yandex.ru}
\label{end\stat}

\renewcommand{\bibname}{\protect\rm Литература}   %13
\def\stat{gr+timon}

\def\tit{ИСПОЛЬЗОВАНИЕ МЕТАДАННЫХ ДЛЯ~РЕАЛИЗАЦИИ ТРЕБОВАНИЙ ПОЛИТИКИ 
БЕЗОПАСНОСТИ MLS$^*$}

\def\titkol{Использование метаданных для~реализации требований политики 
безопасности MLS}

\def\aut{А.\,А.~Грушо$^1$,  Н.\,А.~Грушо$^2$, 
Е.\,Е.~Тимонина$^3$}

\def\autkol{А.\,А.~Грушо, Н.\,А.~Грушо, 
Е.\,Е.~Тимонина}

\titel{\tit}{\aut}{\autkol}{\titkol}

\index{Грушо А.\,А.}
\index{Грушо Н.\,А.} 
\index{Тимонина Е.\,Е.}
\index{Grusho A.\,A.}
\index{Grusho N.\,A.}
\index{Timonina E.\,E.}



{\renewcommand{\thefootnote}{\fnsymbol{footnote}} \footnotetext[1]
{Работа частично поддержана РФФИ (проект 18-07-00274).}}


\renewcommand{\thefootnote}{\arabic{footnote}}
\footnotetext[1]{Институт проблем информатики Федерального исследовательского центра <<Информатика и~управление>> 
Российской академии наук, \mbox{grusho@yandex.ru}}
\footnotetext[2]{Институт проблем информатики Федерального исследовательского центра <<Информатика и~управление>> 
Российской академии наук, info@itake.ru}
\footnotetext[3]{Институт проблем информатики Федерального исследовательского центра <<Информатика и~управление>> 
Российской академии наук, eltimon@yandex.ru}

\vspace*{2pt}

   


  \Abst{Рассматривается распределенная информационная система, объекты которой содержат 
как ценную информацию (или сами являются ценными), так и~открытую (не ценную) 
информацию. Для защиты ценной информации используется политика безопасности (ПБ) MLS
(Multilevel Security), 
которая запрещает информационные потоки от объектов с~ценной информацией к~объектам 
с~открытой информацией. Объекты с~ценной информацией образуют класс объектов уровня High, 
а~объекты c~открытой информацией образуют класс объектов уровня Low. 
  Метаданные (МД) создаются для управления соединениями в~сетях. Метаданные являются 
упрощением математических моделей биз\-нес-про\-цес\-сов и~служат основой разрешительной 
системы для соединений хостов в~распределенной 
ин\-фор\-ма\-ци\-он\-но-вы\-чис\-ли\-тель\-ной сис\-те\-ме
(РИВС).
  В~работе сформулированы правила ПБ MLS и~на основе 
инфраструктуры, связанной с~МД, показана возможность реализации этой 
ПБ в~РИВС. 
Единственный доверенный процесс, необходимый для реализации ПБ MLS, 
функционирует на уровне управления соединениями. Этот уровень не связан с~плос\-костью 
передачи данных и~может быть изолирован с~\mbox{целью} обеспечения его информационной 
безопасности.}
  
  \KW{политика безопасности MLS; информационные потоки; метаданные}
  
  \DOI{10.14357/19922264190414} 
  
%\vspace*{1pt}


\vskip 10pt plus 9pt minus 6pt

\thispagestyle{headings}

\begin{multicols}{2}

\label{st\stat}

\section{Введение}

  Политика безопасности в~компьютерной системе и~сети~--- это набор 
требований по ограничению доступа, хранению и~распределению 
информации~[1]. Обычно ПБ определяет требования по защите 
конфиденциальности, целостности и~доступности информации. Политика
безопас\-ности опирается на 
четкую классификацию ценных информационных ресурсов и~открытых 
информационных ресурсов. 
  
  Один из общих подходов к~описанию требований информационной 
безопасности к~конкретному информационному ресурсу~--- это неотделимая 
привязка к~информационному объекту вектора характеристик, определяющих 
обращение с~этой информацией. Каждый такой вектор содержит результаты 
классификации объекта, а~именно: требования по конфиденциальности, 
целостности и~доступности информации. 
  
  Для простоты будем классифицировать информацию как конфиденциальную 
  и~как открытую. Объекты, содержащие конфиденциальную информацию (ценную 
информацию), будем помечать символом~$(*)$. Это ограничение не умаляет 
общности, так как легко обобщается на требования защиты целостности 
и~доступности. 
  
  Для защиты конфиденциальности широко используется политика 
  MLS~[1], которая запрещает информационные потоки от 
объектов~$(*)$ к~объектам, не помеченным~$(*)$. Объекты с~меткой~$(*)$ 
образуют класс объектов уровня High, а объекты без метки образуют класс 
объектов уровня Low. 
  
  Метаданные создаются для управления соединениями в~сетях~[2]. Как 
правило, МД являются упрощением математических моделей  
биз\-нес-про\-цес\-сов и~служат основой разрешительной системы для соединений 
хостов в~РИВС. Разрешительная система строится на основе порядка взаимодействий 
задач, реализующих бизнес-процесс. Для ее работы вводятся две специальные 
задачи~$\mathcal{M}$ и~$\mathcal{N}$. Задача~$\mathcal{M}$ распределяет 
задачи по хостам, т.\,е.\ определяет бинарное отношение $H(A)$, где~$A$~--- 
задача, а~$H$~--- хост сети. Задача~$\mathcal{N}$ реализует разрешительную 
систему, которая разрешает и~организует соединения хостов $H(A)$ и~$H(B)$, 
если в~МД отражена необходимость инициализации или взаимодействия 
задач~$A$ и~$B$. 
  
  Как было показано в~[3, 4], МД не несут информации о значениях данных. 
Поэтому требования ПБ в~МД отражаются косвенно, т.\,е.\ МД могут содержать 
только информацию о том, куда нельзя направлять информационный поток. 
  
  Как правило, информационная технология (ИТ) представима в~виде составной 
задачи~[5], а~так\-же может описываться DAG (Directed Acyclic Graph)~[6], 
в~котором вершины~--- это задачи ИТ, а~дуги указывают направления 
информационных потоков, передающих исходные данные сле\-ду\-ющим задачам ИТ. 
В~таком представлении возможно существование дуг извне DAG как некоторых 
внешних информационных потоков с~исходными данными и~дуг, выходящих из 
DAG, но не входящих в~задачи ИТ (внешнее распределение информации). 
Передача ценных информационных ресурсов также осуществляется через дуги 
DAG. Поэтому передача ценного информационного ресурса соответствует 
метке~$(*)$ на соответствующей дуге. 
  
\section{Отражение требований MLS в~графах задач и~метаданных} 
  
  Правила MLS могут быть выражены следующим образом. Если в~вершину 
входит хоть одна дуга, помеченная~$(*)$, то эта вершина уже имеет метку~$(*)$ 
и~далее все дуги, выходящие из этой вершины, приобретают метку~$(*)$. 
Возможно, что ценная информация может порождаться в~результате решения 
задачи. Тогда эта задача помечается~$(*)$. 
  
  Справедливо следующее утверждение. 
  
  \smallskip
  
  \noindent
  \textbf{Утверждение~1.}\ \textit{При выполнении правил расстановки 
меток~$(*)$ в~данной ИТ выполняется политика MLS. }
  
  \smallskip
  
  \noindent
  Д\,о\,к\,а\,з\,а\,т\,е\,л\,ь\,с\,т\,в\,о\,.\ \ Допустим, что существует 
информационный поток с~уровня High на уровень Low. Тогда вершина, из которой 
исходит данный поток, имеет метку~$(*)$. По определению любая дуга, 
выходящая из вершины, помеченной~$(*)$, также имеет метку~$(*)$. Но такая 
дуга может входить только в~вершину, которая уже помечена (*). Но вершины 
уровня Low не могут иметь таких меток. Следовательно, предположение о 
существовании информационного потока с~уровня High на уровень Low неверно. 
Утверждение~1 доказано.
  
  \smallskip
  
  Метаданные содержат порядок решения задач, определяемый DAG. Поэтому 
если задачи в~этом порядке используют или порождают ценную информацию, то 
они имеют метку~$(*)$. Тогда все дальнейшие задачи также имеют такую метку. 
При таком дополнении МД несут информацию о требованиях~ПБ. 
  
  Ясно, что появление ценой информации означает дополнительные требования 
  к~хосту, на котором решается эта задача. Поэтому хост, на котором решается задача 
с~меткой~$(*)$, также должен иметь метку~$(*)$. Задача должна иметь 
метку~$(*)$, если она будет содержать ценную информацию. Отметим, что 
протокол управления соединениями в~РИВС основан на криптографии~[7] 
и~является безопасным. 
  
  Особенности хоста с~меткой~$(*)$ основаны на том, что на такой хост должен 
быть загружен <<чис\-тый>> образ операционной сис\-те\-мы, безопасный агент хоста для связи 
с~задачей~$\mathcal{N}$, <<чис\-тое>> программное
обеспечение для задач с~метками~$(*)$, и~на нем 
реализована процедура доверенной загрузки. В~MLS разрешены информационные 
потоки от уровня Low к~уровню High. Для предотвращения попадания на 
хост~$H^*$ вредоносного кода с~уровня Low необходимо обеспечить безопасный 
однонаправленный канал~[8] с~уровня Low на уровень High. Ясно, что на 
хосте~$H^*$ могут решаться различные (доверенно загруженные) задачи 
с~меткой~$(*)$. 
  
  Рассмотрим процедуру решения задачи~$A$ на уровне Low. В~этом случае 
возможны два варианта:
  \begin{enumerate}[(1)]
\item задача~$\mathcal{M}$ устанавливает задачу~$A$ на хосте, к~которому не 
допускается ни один информационный поток с~меткой~$(*)$;
\item на хосте $H^*(A^*)$ реализуется ИТ очистки от информации 
с~меткой~$(*)$ (новая доверенная загрузка). 
\end{enumerate}
  
\section{Отражение требований MLS в~инфраструктуре метаданных}

  Поскольку ИТ может использовать информацию с~меткой~$(*)$, то в~МД задача 
имеет метку~$(*)$. Если ИТ не использует информацию, помеченную~$(*)$, то 
в~МД отсутствуют задачи с~такой меткой. 
  
  Как было отмечено ранее, для передачи ценной информации задаче она должна 
иметь метку~$(*)$. Тогда хост, на котором находится эта задача, также должен 
иметь метку~$(*)$. Отсюда получается простейшее решение 
задачи~$\mathcal{M}$, обеспечивающее выполнение ПБ
MLS. Так как заранее известны все задачи с~меткой~$(*)$, то 
задача~$\mathcal{M}$ размещает их на хостах, помеченных~$(*)$, и~по правилам 
MLS задача~$\mathcal{N}$ реализует разрешительную систему на данных хостах. 
Это означает, что в~задаче~$\mathcal{N}$ выделяется подзадача~$\mathcal{N}_H$, 
реа\-ли\-зу\-ющая взаимодействие только на хостах с~меткой~$(*)$. Остальные хосты 
относятся к~уровню Low, и~задача~$\mathcal{N}$ реализует разрешительную 
систему только этих хостов. Таким образом, в~задаче~$\mathcal{N}$ выделяется 
независимая подзадача~$\mathcal{N}_L$, реализующая разрешительную систему 
взаимодействия хостов на уровне Low. 
  
  Отметим, что задачи~$\mathcal{N}_H$ и~$\mathcal{N}_L$ могут 
рассматриваться как два экземпляра задачи~$\mathcal{N}$, функционирующие на 
непересекающихся доменах РИВС. 
  
  Наиболее сложный вопрос состоит в~безопасной передаче информации с~уровня 
Low на уровень High. Согласно модели Bell-LaPadula~\cite{9-tt}, такую передачу 
можно осуществить только с~по\-мощью доверенного процесса, реализующего 
од\-но\-на\-прав\-лен\-ный канал с~уровня Low на уровень High. Такой однонаправленный 
канал можно реализовать на базе задачи~$\mathcal{N}$. Этот доверенный канал 
может быть организован на основе инфраструктуры метаданных следующим 
образом. 
  
  Пусть задача~$A$ запрашивает соединение с~задачей~$B^*$ через 
задачу~$\mathcal{N}_L$. Исходя из метаданных, задача~$\mathcal{N}$ 
определяет необходимость передачи исходных данных из задачи~$A$ 
в~задачу~$B^*$. При получении разрешения эти данные передаются из задачи~$A$ в~задачу~$\mathcal{N}_L$. После проверки их безопасности данные из 
задачи~$\mathcal{N}_L$ передаются в~задачу~$\mathcal{N}_H$ для дальнейшей 
передачи данных в~задачу~$B^*$. Отметим, что непосредственного соединения 
хоста уровня Low с~хостом уровня High не происходит. Отсюда следует 
утверждение. 
  
  \smallskip
  
  \noindent
  \textbf{Утверждение~2.}\ \textit{Пусть выполняются следующие условия}:
  \begin{enumerate}[(1)]
  \item \textit{ сформирована подсистема РИВС уровня High с~помощью задач 
  и~хостов с~метками~$(*)$ и~разрешительная система на основе МД 
и~задачи}~$\mathcal{N}_H$;
\item \textit{сформирована подсистема РИВС уровня Low с~помощью задач 
и~хостов без меток~$(*)$ и~разрешительная система на основе метаданных 
и~задачи}~$\mathcal{N}_L$;
  \item \textit{взаимодействие уровней Low и~High осуществляется только через 
однонаправленное взаимодействие задач~$\mathcal{N}_L$ 
и~~$\mathcal{N}_H$}. 
  \end{enumerate}
  \textit{Тогда в~РИВС выполняется политика MLS}.
  
  \smallskip
  
  \noindent
  Д\,о\,к\,а\,з\,а\,т\,е\,л\,ь\,с\,т\,в\,о\,.\ \ Метаданные
   и~задача~$\mathcal{N}_H$ не 
допускают взаимодействия уровня High с~уровнем Low. Аналогично МД и~задача 
NL не допускают непосредственного взаимодействия уровня Low с~уровнем High. 
Безопасный интерфейс уровня Low с~уровнем High реализован 
однонаправленным каналом между задачами~$\mathcal{N}_L$ 
и~$\mathcal{N}_H$. Функционал, реализующий этот канал, как и~вся 
задача~$\mathcal{N}$, могут быть изолированы от остального функционала РИВС и~поэтому могут считаться доверенным субъектом. Таким образом, все условия 
реализации политики MLS выполнены. Утверждение~2 доказано. 
  
  \section{Поиск информации с~уровня High на~уровне Low}
  
  Для решения задач с~меткой~$(*)$ может возникнуть необходимость 
дополнительного поиска информации в~памяти о предшествующих задачах. 
Метаданные сохраняют информацию о цепочке решенных задач, и~обращение 
к~ним на уровне High не представляет сложности. 
  
  Если задаче~$A^*$ необходимо найти дополнительную информацию на уровне 
Low, то тогда также можно использовать доверенное взаимодействие между 
задачами~$\mathcal{N}_H$ и~$\mathcal{N}_L$. Задача~$\mathcal{N}_H$ 
обращается к~задаче~$\mathcal{N}_L$ с~запросом на поиск данных в~решенных на 
уровне Low задачах. Задача~$\mathcal{N}_L$, используя обратный обзор 
решенных на уровне Low задач, ищет искомую информацию. В~данном случае 
возможен скрытый канал с~уровня High на уровень Low в~задании поиска 
информации на уровне Low~\cite{8-tt}. Этот канал можно перекрыть с~помощью 
последовательного опроса задач на уровне Low и~выявления признаков искомой 
информации уже на уровне задачи~$\mathcal{N}_L$. В~случае появления 
необходимых признаков задача~$\mathcal{N}_L$ передает 
задаче~$\mathcal{N}_H$ данные для задачи~$A^*$. 
  
  Данный способ не является доказательством перекрытия скрытого канала. 
Однако обращение с~уровня High на уровень Low считается запрещенным 
информационным потоком в~политике MLS, и~реализация такого поиска с~учетом 
возможности скрытого канала является сложной задачей~\cite{10-tt}. 
  
  \section{Заключение }
  
  В работе рассмотрена РИВС, в~которой управ\-ле\-ние соединениями 
осуществляется с~помощью МД. Показана возможность реализации 
ПБ MLS в~рассматриваемой РИВС на основе 
инфраструктуры, связанной с~МД. Единственный доверенный процесс, 
необходимый для реализации ПБ MLS, функционирует на 
уровне управ\-ле\-ния соединениями. Этот уровень не связан с~плос\-костью передачи 
данных и~может быть изолирован с~\mbox{целью} обеспечения его информационной 
безопас\-ности. 
  
  В работе рассмотрена только одна ИТ, для которой необходимо выполнить 
требования политики безопасности MLS. Однако рассмотренный метод легко 
обобщается на случай множества ИТ.
  

{\small\frenchspacing
 {%\baselineskip=10.8pt
 \addcontentsline{toc}{section}{References}
 \begin{thebibliography}{99}
\bibitem{1-tt}
Department of Defense trusted computer system evaluation criteria.~--- U.S.\ National Institute of 
Standards and Technology, Department of Defense, 1985. {\sf 
http://csrc.nist.gov/publications/history/dod85.\linebreak pdf}. 
\bibitem{2-tt}
\Au{Grusho A., Grusho N., Zabezhailo~M., Zatsarinny~A., Timonina~E.} Information security of SDN 
on the basis of meta data~// Computer 
network security~/ Eds. J.~Rak, J.~Bay, I.~Kotenko, \textit{et al.}~---
Lecture notes in computer science ser.~--- Springer, 2017. 
Vol.~10446.  P.~339--347. doi: 10.1007/978-3-319-65127-9\_27.
\bibitem{3-tt}
\Au{Грушо А.\,А., Грушо~Н.\,А., Левыкин~М.\,В., Тимонина~Е.\,Е.} Методы идентификации 
захвата хоста в~распределенной ин\-фор\-ма\-ци\-он\-но-вы\-чис\-ли\-тель\-ной системе, 
защищенной с~по\-мощью метаданных~// Информатика и~её применения, 2018. Т.~12. Вып.~4. 
С.~41--45.
\bibitem{4-tt}
\Au{Grusho A.\,A., Grusho~N.\,A., Timonina~E.\,E.} 
Information flow control on the basis of meta 
data~// Distributed computer and communication networks~/ 
Eds. V.\,M.~Vishnevskiy, K.\,E.~Samouylov, D.\,V.~Kozyrev.~--- 
Lecture notes
in computer science ser.~--- Springer,
 2019. Vol.~11965. P.~548--562.

\bibitem{5-tt}
\Au{Грушо А.\,А., Тимонина~Е.\,Е., Шоргин~С.\,Я.} Иерархический метод порождения 
метаданных для управления сетевыми соединениями~// Информатика и~её применения, 2018. 
Т.~12. Вып.~2. С.~44--49.
\bibitem{6-tt}
\Au{Грушо А.\,А., Зацаринный~А.\,А., Тимонина~Е.\,Е.} Электронная бухгалтерская книга на базе 
ситуационных центров для цифровой экономики~// Системы и~средства информатики, 2019. 
Т.~29. №\,2. С.~4--11.
\bibitem{7-tt}
\Au{Grusho A.\,A., Timonina~E.\,E., Shorgin~S.\,Ya.} Modelling for ensuring information security of 
the distributed information systems~// 31th European Conference on Modelling and Simulation 
Proceedings.~--- Dudweiler, Germany: Digitaldruck Pirrot GmbH, 2017. P.~656--660. {\sf 
http://www.scs-europe.net/dlib/2017/\linebreak  
ecms2017acceptedpapers/0656-probstat\_ECMS2017\_ 0026.pdf}.
\bibitem{8-tt}
\Au{Тимонина Е.\,Е.} Анализ угроз скрытых каналов и~методы построения гарантированно 
защищенных распределенных автоматизированных систем: Дис.\ \ldots\ д-ра техн. наук.~--- 
М., 2004. 204~с.
\bibitem{9-tt}
\Au{Грушо А.\,А., Применко~Э.\,А., Тимонина~Е.\,Е.} Теоретические основы компьютерной 
безопасности.~--- М.: Академия, 2009. 272~с.
\bibitem{10-tt}
\Au{Grusho A.\,A., Grusho~N.\,A., Zabezhailo~M.\,I., Timonina~E.\,E.} Protection of valuable 
information in public information space~// Communications of the ECMS: 33th European Conference 
on Modelling and Simulation Proceedings.~--- 
Dudweiler, Germany: Digitaldruck Pirrot GmbH, 2019. 
Vol.~33. No.\,1. P.~451--455. 
{\sf 
http://www.scs-europe.net/dlib/2019/ecms2019acceptedpapers/0451\_\linebreak pstat\_ecms2019\_0018.pdf}.
 \end{thebibliography}

 }
 }

\end{multicols}

\vspace*{-6pt}

\hfill{\small\textit{Поступила в~редакцию 13.10.19}}

\vspace*{8pt}

%\pagebreak

%\newpage

%\vspace*{-28pt}

\hrule

\vspace*{2pt}

\hrule

%\vspace*{-2pt}

\def\tit{USING METADATA TO~IMPLEMENT MULTILEVEL SECURITY POLICY 
REQUIREMENTS}


\def\titkol{Using metadata to~implement multilevel security policy 
requirements}

\def\aut{A.\,A.~Grusho, N.\,A.~Grusho, and~E.\,E.~Timonina}

\def\autkol{A.\,A.~Grusho, N.\,A.~Grusho, and~E.\,E.~Timonina}

\titel{\tit}{\aut}{\autkol}{\titkol}

\vspace*{-11pt}


 \noindent
   Institute of Informatics Problems, Federal Research Center ``Computer Sciences and 
Control'' of the Russian Academy of Sciences; 44-2~Vavilov Str., Moscow 119133, 
Russian Federation

\def\leftfootline{\small{\textbf{\thepage}
\hfill INFORMATIKA I EE PRIMENENIYA~--- INFORMATICS AND
APPLICATIONS\ \ \ 2019\ \ \ volume~13\ \ \ issue\ 4}
}%
 \def\rightfootline{\small{INFORMATIKA I EE PRIMENENIYA~---
INFORMATICS AND APPLICATIONS\ \ \ 2019\ \ \ volume~13\ \ \ issue\ 4
\hfill \textbf{\thepage}}}

\vspace*{3pt}  
 
  
   \Abste{A distributed information computing system which objects contain both valuable 
information (or are themselves valuable) and open (non-valuable) information is considered. To protect 
valuable information, multilevel  security (MLS) policy is used that prohibits information flows from objects with 
valuable information to objects with open information. Objects with valuable information form a~class 
of high-level objects, and objects with open information form a class of low-level objects.
   Metadata is created to manage network connections. Metadata is a simplification of mathematical 
models of business processes and is the basis of a permission system for host connections in 
a~distributed information computing system.
   The paper constructs MLS security policy rules, and based on metadata-related infrastructure, 
shows the ability to implement this security policy in the distributed information computing system. 
The only trusted process required to implement the MLS security policy is at the connection 
management level. This layer is unrelated to the data plane and can be isolated to ensure its 
information security.}
    
   \KWE{MLS security policy; information flows; metadata}
   
   
   

\DOI{10.14357/19922264190414} 

%\vspace*{-14pt}

 \Ack
   \noindent
   The paper was partially supported by the Russian Foundation for Basic Research (project  
18-07-00274).


%\vspace*{-6pt}

  \begin{multicols}{2}

\renewcommand{\bibname}{\protect\rmfamily References}
%\renewcommand{\bibname}{\large\protect\rm References}

{\small\frenchspacing
 {%\baselineskip=10.8pt
 \addcontentsline{toc}{section}{References}
 \begin{thebibliography}{99}
\bibitem{1-tt-1}
U.S.\ National Institute of Standards and Technology, Department of Defence. 1985.
Department of Defense trusted computer system evaluation criteria. Available at: {\sf 
http://csrc.nist.gov/publications/history/dod85.pdf} (accessed October~6, 2019).
\bibitem{2-tt-1}
\Aue{Grusho, A., N.~Grusho, M.~Zabezhailo, A.~Zatsarinny, and E.~Timonina.} 2017. Information 
security of SDN on the basis of meta data. 
\textit{Computer network security}. Eds. J.~Rak, J.~Bay, I.~Kotenko, 
\textit{et al.}
Lecture notes in computer science ser. Springer. 
10446:339--347. doi: 10.1007/978-3-319-65127-9\_27.
\bibitem{3-tt-1}
\Aue{Grusho, A.\,A., N.\,A.~Grusho, M.\,V.~Levykin, and E.\,E.~Timonina.} 2018. Metody 
identifikatsii zakhvata khosta v~raspredelennoy informatsionno-vychislitel'noy sisteme, 
zashchishchennoy s~pomoshch'yu metadannykh [Methods of identification of host capture in the 
distributed information system which is protected on the base of meta data]. \textit{Informatika i~ee 
Primeneniya~--- Inform. Appl.} 12(4):41--45. 
\bibitem{4-tt-1}
\Aue{Grusho, A.\,A., N.\,A.~Grusho, and E.\,E.~Timonina.} 2019. Information flow control 
on the basis of meta data. \textit{Distributed computer and communication networks}.
 Eds. V.\,M.~Vishnevskiy, 
K.\,E.~Samouylov, and D.\,V.~Ko\-zy\-rev. 
Lecture notes
in computer science ser. Springer. 11965:548--562.
\bibitem{5-tt-1}
\Aue{Grusho, A.\,A., E.\,E.~Timonina, and S.\,Ya.~Shorgin.} 2018. Ierarkhicheskiy metod 
porozhdeniya metadannykh dlya upravleniya setevymi soedineniyami [Hierarchical method of meta 
data generation for control of network connections]. \textit{Informatika i~ee Primeneniya~--- Inform. 
Appl.} 12(2):44--49.
\bibitem{6-tt-1}
\Aue{Grusho, A.\,A., A.\,A.~Zatsarinny, and E.\,E.~Timonina.} 2019. Elektronnaya bukhgalterskaya 
kniga na baze situatsionnykh tsentrov dlya tsifrovoy ekonomiki [The electronic ledger on the basis of 
the situational centers for digital economy]. \textit{Sistemy i~Sredstva Informatiki~--- Systems and 
Means of Informatics} 29(2):4--11.
\bibitem{7-tt-1}
\Aue{Grusho, A.\,A., E.\,E.~Timonina, and S.\,Ya.~Shorgin.} 2017. Modelling for ensuring 
information security of the distributed information systems. \textit{31th European Conference on 
Modelling and Simulation Proceedings}. Dudweiler, Germany: Digitaldruck Pirrot GmbH. 656--660. 
Available at: {\sf  
http://www.scs-europe.net/dlib/2017/\linebreak ecms2017acceptedpapers/0656-probstat\_ECMS2017\_ 0026.pdf} (accessed 
October~6, 2019).
\bibitem{8-tt-1}
\Aue{Timonina, E.\,E.} 2004. Analiz ugroz skrytykh kanalov i~metody postroeniya garantirovanno 
zashchishchennykh raspredelennykh avtomatizirovannykh sistem [The analysis of threats of covert 
channels and methods of creation of guaranteed protected distributed automated 
systems]. Moscow.  D.Sc. Diss.  204~p.
\bibitem{9-tt-1}
\Aue{Grusho, A., E.~Primenko, and E.~Timonina.} 2009. \textit{Teoreticheskie osnovy 
komp'yuternoy bezopasnosti} [Theoretical bases of computer security]. Moscow: Academy. 272~р.
\bibitem{10-tt-1}
\Aue{Grusho, A.\,A., N.\,A.~Grusho, M.\,I.~Zabezhailo, and E.\,E.~Timonina.} 2019. Protection of 
valuable information in public information space. \textit{Communications of the ECMS:  33th 
European Conference on Modelling and Simulation Proceedings}. 
Dudweiler, Germany: Digitaldruck Pirrot GmbH.
33(1):451--455. Available at: {\sf 
http://www.scs-europe.net/dlib/2019/ecms2019acceptedpapers/0451\_\linebreak pstat\_ecms2019\_0018.pdf} (accessed 
October~6, 2019).
\end{thebibliography}

 }
 }

\end{multicols}

\vspace*{-6pt}

\hfill{\small\textit{Received October 13, 2019}}

%\pagebreak

%\vspace*{-22pt}


\Contr


\noindent
\textbf{Grusho Alexander A.} (b.\ 1946)~--- Doctor of Science in physics and 
mathematics, professor, principal scientist, Institute of Informatics Problems, Federal 
Research Center ``Computer Sciences and Control'' of the Russian Academy of 
Sciences; 44-2~Vavilov Str., Moscow 119133, Russian Federation;  
\mbox{grusho@yandex.ru}

\vspace*{3pt} 

\noindent
\textbf{Grusho Nikolai A.} (b.\ 1982)~--- Candidate of Science (PhD) in physics 
and mathematics, senior scientist, Institute of Informatics Problems, Federal 
Research Center ``Computer Sciences and Control'' of the Russian Academy of 
Sciences;  
44-2~Vavilov Str., Moscow 119133, Russian Federation; info@itake.ru 

\vspace*{3pt}

\noindent
\textbf{Timonina Elena E.} (b.\ 1952)~--- Doctor of Science in technology, 
professor, leading scientist, Institute of Informatics Problems, Federal Research 
Center ``Computer Sciences and Control'' of the Russian Academy of Sciences;  
44-2~Vavilov Str., Moscow 119133, Russian Federation; 
\mbox{eltimon@yandex.ru}
\label{end\stat}

\renewcommand{\bibname}{\protect\rm Литература}    %14
\def\stat{goncharov}

\def\tit{ВЫРАВНИВАНИЕ ДЕКАРТОВЫХ ПРОИЗВЕДЕНИЙ УПОРЯДОЧЕННЫХ МНОЖЕСТВ$^*$}

\def\titkol{Выравнивание декартовых произведений упорядоченных множеств}

\def\aut{А.\,В.~Гончаров$^1$, В.\,В.~Стрижов$^2$}

\def\autkol{А.\,В.~Гончаров, В.\,В.~Стрижов}

\titel{\tit}{\aut}{\autkol}{\titkol}

\index{Гончаров А.\,В.}
\index{Стрижов В.\,В.}
\index{Goncharov A.\,V.}
\index{Strijov V.\,V.}


{\renewcommand{\thefootnote}{\fnsymbol{footnote}} \footnotetext[1]
{Работа выполнена при частичной финансовой поддержке РФФИ 
(проекты 19-07-1155 и~19-07-00885). Настоящая статья содержит 
результаты проекта <<Статистические методы машинного обучения>>, 
выполняемого в~рамках реализации Программы Центра компетенций 
Национальной технологической инициативы <<Центр хранения 
и~анализа больших данных>>, поддерживаемого Министерством науки 
и~высшего образования Российской Федерации по договору МГУ им.\ 
М.\,В.~Ломоносова  с~Фондом поддержки проектов Национальной 
технологической инициативы от 11.12.2018 №\,13/1251/2018.}}


\renewcommand{\thefootnote}{\arabic{footnote}}
\footnotetext[1]{Московский физико-технический институт, alex.goncharov@phystech.edu}
\footnotetext[2]{Вычислительный центр им.\ А.\,А.~Дородницына Федерального исследовательского 
центра <<Информатика и~управ\-ле\-ние>> Российской академии наук; 
Московский фи\-зи\-ко-тех\-ни\-че\-ский институт, \mbox{strijov@ccas.ru}}

%\vspace*{-12pt}



\Abst{Работа посвящена исследованию метрических методов анализа 
объектов сложной структуры. Предлагается обобщить метод динамического 
выравнивания двух временных рядов на случай объектов, определенных на 
двух и~более осях времени. В~дискретном представлении такие объекты 
являются матрицами. Метод динамического выравнивания временных рядов 
обобщается как метод динамического выравнивания матриц. Предложена 
функция расстояния, устойчивая к~монотонным нелинейным деформациям 
декартова произведения двух и~более временных шкал. Определен выравнивающий 
путь между объектами. В~дальнейшем объектом называется матрица, 
в~которой строки и~столбцы соответствуют осям времени. Исследованы 
свойства предложенной функции расстояния. Для иллюстрации метода 
решаются задачи метрической классификации объектов на модельных 
данных и~данных из набора MNIST.}

\KW{функция расстояния; динамическое выравнивание; расстояние между матрицами; 
нелинейные деформации времени; про\-стран\-ст\-вен\-но-вре\-мен\-ные ряды}

\DOI{10.14357/19922264200105} 
  
\vspace*{-3pt}


\vskip 10pt plus 9pt minus 6pt

\thispagestyle{headings}

\begin{multicols}{2}

\label{st\stat}


\section{Введение}

Временн$\acute{\mbox{ы}}$е ряды представляют собой набор измерений, упорядоченных 
по оси времени. Анализ временн$\acute{\mbox{ы}}$х рядов производится при решении задач, 
связанных с~классификацией активности человека по измерениям акселерометра 
телефона, поиском паттернов в~EEG-сиг\-на\-лах (электроэнцефалограмма), 
кластеризации набора ECoG (электрокортикограмма) данных и~во многих других 
задачах~\cite{0}. Рассматриваются объекты, для которых время между измерениями 
фиксированно. В~данной работе для построения адекватной функции 
расстояния между объектами требуется учесть нелинейные деформации 
относительно оси времени: глобальные и~локальные сдвиги, растяжения 
и~сжатия~\cite{1}.

В~\cite{2} приводятся различные методы решения задач анализа 
временн$\acute{\mbox{ы}}$х рядов: классификации, детектирования паттернов, 
кластеризации и~др. В~\cite{3} описание временных рядов 
строится с~по\-мощью анализа параметров моделей, в~\cite{4} 
используется их признаковое описание, в~\cite{5} анализируется их форма. 
Комбинации этих подходов описаны в~\cite{2}.

Метрические методы находят схожие объекты в~наборе. Используются 
функции расстояния над временн$\acute{\mbox{ы}}$ми рядами: расстояние Хаусдорфа~\cite{10}, 
MODH~\cite{11}, расстояние, основанное на HMM
(hiden Markov model)~\cite{6}, евклидово расстояние 
в~исходном пространстве или в~пространстве сниженной размерности~\cite{5}, 
\mbox{LCSS} (longest common\linebreak subsequence)~\cite{7}. Показано~\cite{8}, что в~случае локальных или глобальных 
деформаций времени при решении задач, требующих анализа исходной формы 
временн$\acute{\mbox{о}}$го ряда, метод динамического выравнивания оси времени 
DTW (Dynamic Time Warping) 
превосходит другие функции расстояния~\cite{9} по качеству итогового 
решения задачи, так как при наличии смещений двух объектов относительно 
друг друга требуется выравнивать их оптимальным образом для вычисления 
расстояния между ними.

В данной работе предлагается перейти от рас\-смот\-ре\-ния объекта~$\textbf{s}(t)$, 
временн$\acute{\mbox{о}}$го ряда, к~более общему случаю $\textbf{s}(\textbf{t})$, 
в~котором компоненты вектора~$\textbf{t}$~--- оси времени. Из-за 
существенного рос\-та вы\-чис\-ли\-тель\-ной слож\-ности при увеличении чис\-ла 
осей времени предлагается рас\-смот\-реть объекты $\textbf{s}(t_1, t_2)$, 
определенные на двух осях времени. Оси времени считаются независимыми. 
В~случае единственной дискретной и~ограниченной сверху шкалы времени 
объект представим вектором фиксированной размерности. 
Аналогично объект настоящего исследования представим мат\-ри\-цей.

Вводятся ограничения на зависимости осей времени в~декартовом 
произведении для таких объектов. Определена гипотеза порождения данных: 
объекты одного класса эквивалентности получены при помощи допустимых 
преобразований, а~именно: локальных деформаций (растяжений и~сжатий) 
каждой из осей времени по отдельности. В~дискретном случае преобразование 
представимо дуп\-ли\-ци\-ро\-ва\-ни\-ем строк и~столбцов матриц. 
В~число допустимых преобразований попадают и~глобальные деформации: 
сдвиги по осям времени, представимые добавлением и~удалением крайних 
строк и~столбцов исходных матриц. Для каждой из осей времени выполняются 
свойства времени: монотонность и~непрерывность. Похожими на описанные 
свойствами обладает, например, частотный спектр сигнала, где одна ось 
определяет время, а другая~--- частоту, величину, обратную времени.


Между двумя объектами, матрицами, в~случае допустимых преобразований 
требуется определить инвариантную к~преобразованиям осей времени функцию 
расстояния, которая сможет выделить классы эквивалентности множества 
преобразованных объектов. Работа посвящена определению такой функции 
расстояния, как обобщения метода динамического выравнивания временных рядов 
DTW для матриц.

Цель данной работы~--- построение метода, основанного на динамическом 
выравнивании осей времени для матриц. Метод динамического выравнивания 
временн$\acute{\mbox{ы}}$х рядов~\cite{33} определен только для объектов с~одной осью времени, 
что делает его неприменимым для описанного случая. Однако концепции, 
используемые на каждой стадии вы\-чис\-ле\-ния оптимального выравнивания, обобщены 
на рассматриваемый случай. Работа исследует свойства предложенного 
метода и~сравнивает результаты применения метода к~задачам классификации 
изображений~\cite{12} с~результатами функции расстояния~$L_2$.

Для иллюстрации и~анализа результатов решается задача метрической 
классификации объектов (матриц низкой размерности). Используются наборы данных: 
модельные данные, которые согласуются с~выдвинутой гипотезой порождения 
данных для временн$\acute{\mbox{ы}}$х рядов, подмножество набора MNIST сниженной 
размерности и~частотный спектр сигнала.

\vspace*{-10pt}

\section{Постановка задачи построения функции расстояния}

\vspace*{-2pt}

Рассмотрим задачу построения функции расстояния между объектами. 
Функция расстояния инвариантна к~допустимым преобразованиям осей времени: 
глобальным и~локальным линейным и~нелинейным деформациям временн$\acute{\mbox{о}}$й шкалы. 
Ниже приведены две постановки задачи, с~помощью которых определены свойства 
предложенной функции расстояния, оценено ее качество и~проведено сравнение 
нескольких функций расстояния: предложенной и~$L_2$.

Первая постановка задачи использует общее свойство функций расстояния: 
объединение схожих объектов и~разделение непохожих объектов. 
Вводится определение свойства инвариантности функции расстояния к~допустимым 
преобразованиям осей времени.
Вторая постановка задачи уточняет первую и~заключается в~проведении метрической 
классификации методом ближайшего соседа.

\textbf{Постановка задачи выбора функции расстояния между двумя объектами.}
На двух временн$\acute{\mbox{ы}}$х осях заданы объекты вида 
$\textbf{A}(t_1,t_2)\hm \in \mathbb{R}^{n \times n}$. 
Функция $G_w(\textbf{A}):\mathbb{R}^{n \times n} \hm\rightarrow 
\mathbb{R}^{\hat{n} \times \hat{n}}$ задает допустимые преобразования 
исходного объекта~$\textbf{A}$: глобальные сдвиги, локальные линейные 
и~нелинейные деформации, а~именно: растяжения и~сжатия оси времени, 
сдвиги значений по оси времени. Скалярный параметр $w \hm\in \mathbb{R}^+$
 функции~$G$ фиксирует набор этих преобразований.

Допустимым элементарным преобразованием матрицы~$\textbf{A}$ назовем 
дуплицирование случайных строк и~столбцов исходной матрицы, добавление 
или удаление крайних строк и~столбцов. Допустимым преобразованием 
примем некоторую последовательность допустимых элементарных 
преобразований матрицы~$\textbf{A}$ и~обозначим как~$G_w(\textbf{A})$.

Будем называть объект~$\textbf{B} \hm\in \mathbb{R}^{\hat{n} \times \hat{n}}$ 
полученным из объекта~$\textbf{A}$ при помощи допустимых 
преобразований~$G_{\hat{w}}$, если существует $\hat{w}\hm\in \mathbb{R}^+ : 
\textbf{B} \hm= G_{\hat{w}}(\textbf{A})$.

Функцию расстояния между двумя объектами $\rho: 
\mathbb{R}^{{n} \times {n}} \times \mathbb{R}^{\hat{n} \times \hat{n}} 
\hm\rightarrow  \mathbb{R}^+$ оценим на выборке $\mathfrak{D } \hm= 
\{ \textbf{A}_i \}_{i=1}^m$ объектов вида $\textbf{A}_i \hm\in 
\mathbb{R}^{n \times n}$.

Для каждого объекта выборки~$\textbf{A}_i$ и~объекта~$\textbf{B}_j$ его 
класса эквивалентности $\{\textbf{B}_j\}_i \hm= \{  \textbf{B} 
\hm\in \mathfrak{D} | \exists w_i,w_j: G_{w_i}(\textbf{A}_i) \hm= G_{w_j}
(\textbf{B}_j)   \}$ заданы допустимые трансформации с~параметрами~$w_i$ 
и~$w_j$, такие что $G_{w_i}(\textbf{A}_i)\hm = G_{w_j}(\textbf{B}_j)$. 
Для каждого объекта выборки~$\textbf{A}_i$ и~объекта~$\textbf{C}_j$ 
из других классов эквивалентности $\{ \textbf{C}_k\}_i \hm= 
\{  \textbf{C} \hm\in \mathfrak{D} | \nexists w_i,w_k: G_{w_i}(\textbf{A}_i)
\hm = G_{w_k}(\textbf{C})   \}$ не существует таких $ w_i, w_k : G_{w_i}
(\textbf{A}_i) \hm= G_{w_k}(\textbf{C}_k)$.

Решается задача поиска функции расстояния~$\rho$, значение
 которой на паре объектов одного класса эквивалентности меньше, 
 чем на любой паре объектов из разных: для любых $i,j,k \hm\in 
 \{1,\dots,m\}$ $\quad \rho(\textbf{A}_i,\textbf{B}_j) \hm< 
 \rho(\textbf{A},\textbf{C}_k)$. Функцию расстояния, обладающую 
 таким свойством, назовем инвариантной на классах эквивалентности.

Критерием качества для функции расстояния~$\rho$ на выборке~$\mathfrak{D}$ 
примем долю объектов, для которых указанное неравенство выполняется:
$$
S_{\rho}(\mathfrak{D}) = \fr{1}{m} \sum\limits_{i=1}^m 
\prod\limits_{\{ \textbf{B}_j\}_i} 
\prod\limits_{\{ \textbf{C}_k\}_i}  
\left[  \rho(\textbf{A}_i,\textbf{B}_j) < \rho(\textbf{A}_i,\textbf{C}_k)  
 \right].
 $$
Постановка задачи выбора функции расстояния~$\rho$ 
сводится к~задаче максимизации критерия качества.

\textbf{Прикладное использование функции расстояния.}
Задана выборка $\mathfrak{D}\hm = \{(\textbf{A}_i,y_i)\}^m_{i=1}$, 
состоящая из пар объ\-ект--от\-вет. Объектами служат объекты сложной 
структуры: $\textbf{A}_i\hm \in \mathbb{R}^{n\times n}$, 
а~ответами выступают метки класса~---~$y_i\hm \in Y \hm= \{1,\ldots,E\}$, 
где $E \hm\ll m$. Выборка разделена на обучение $\mathfrak{D}_l \hm= 
\{(\textbf{A}_i,y_i)\}^{m_1}_{i=1}$ и~контроль $\mathfrak{D}_t \hm= 
\{(\textbf{A}_i,y_i)\}_{m_1}^{m_1+m_2}$.

Модель классификации~$f$ принадлежит множеству моделей метрической 
классификации 1NN, которые классифицируемому объекту ставят 
в~соответствие метку класса ближайшего объекта из обучающей 
выборки по заданной функции расстояния~$\rho$:
$$ 
\hat{y} = f(\textbf{B} | \rho) = y \argmin\limits_{i = 1,\dots, m_1} 
\rho\left(B,A_i\right)\,.
$$
Критерий качества $S$ модели~$f$ для задачи классификации~--- 
доля правильно проставленного класса на контрольной выборке:
 $$ 
 S(f | \rho) = \fr{1}{m_2}\sum\limits_{i=m_1}^{m_1+m_2} 
 \left[f(\textbf{A}_i | \rho) = y_i\right].
 $$

Требуется выбрать функцию расстояния~$\rho$ для модели 
классификации~$f:~\mathbb{R}^{n\times n} \hm\rightarrow~Y$, 
максимизируюшую критерий качества~$S$ на контрольной выборке:
\begin{equation*}
f =  \argmax\limits_{\rho \in \{\mathrm{mDTW}, L_2\}}\left(S(f | \rho)\right).
\end{equation*}

\section{Вычисление матричного расстояния mDTW}

Предлагается использовать функцию расстояния DTW, 
модифицированную для случая выравнивания двойной шкалы времени.

\smallskip

\noindent
\textbf{Определение~1.} {Даны два объекта~$\textbf{A},\textbf{B}\hm \in 
\mathbb{R}^{n\times n}$. Тензор 
невязок~$\boldsymbol{\Omega}^{n \times n \times n \times n}$~--- 
такой тензор, что его элемент~$\boldsymbol{\Omega}(i,j,k,l)$ 
равен квадрату разности между элементами~$\textbf{A}(i,j)$ и~$\textbf{B}(k,l)$:}
\begin{equation*}
\boldsymbol{\Omega}(i,j,k,l)=(\textbf{A}(i,j) - \textbf{B}(k,l))^2.
\end{equation*}

\noindent
\textbf{Определение 2.} {Путем~$\boldsymbol{\pi}$ между двумя 
объектами $\textbf{A},\textbf{B} \hm\in \mathbb{R}^{n\times n}$ 
назовем множество индексов тензора~$\boldsymbol{\Omega}$: }
$$
\boldsymbol{\pi} = \{(i,j,k,l)\},\quad i,j,k,l \in \{1,\ldots,n\} ,
$$
\textit{удовлетворяющее следующим условиям:}

{\bfseries\textit{Частичный порядок.}}
Для элементов пути~$\boldsymbol{\pi}$ с~фиксированными значениями~$i,k$ 
задан порядок: выравнивающий путь для фиксированных строк двух 
матриц упорядочен~--- $\{(i,j_r,k,l_r))\}_{r=1}^{R} \hm\subset 
\boldsymbol{\pi}$ мощностью~$R$. Аналогично для фиксированных столбцов 
с~индексами~$j,l$.

{\bfseries\textit{Граничные условия.}}
 Пусть $(i,j,k,l) \in \boldsymbol{\pi}$, тогда $(1,j,1,l) \hm\in 
 \boldsymbol{\pi}$ и~$(i,1,k,1) \hm\in \boldsymbol{\pi}$.
Путь $\boldsymbol{\pi}$ содержит элементы тензора~$\boldsymbol{\Omega}$: 
$(1,1,1,1) \hm\in \boldsymbol{\pi}$ и~$(n,n,n,n) \hm\in \boldsymbol{\pi}$.

{\bfseries\textit{Непрерывность по направлению.}}
Для упорядоченного подмножества пути $\{(i,j_r,k,l_r)\}_{r=1}^{R}
\hm\subset\boldsymbol{\pi}$ выполняется условие непрерывности:
$$
j_{r}-j_{r-1}\leq1\,,\quad l_r-l_{r-1}\leq1\,, \quad r = 2,\ldots,R\,.
$$
На~шаге пути~$\boldsymbol{\pi}$ по фиксированному направлению времени~$i,k$ 
встречаются только соседние элементы матрицы (включая соседние по диагонали). 
Аналогично для фиксированных~$j,l$.

{\bfseries\textit{Монотонность по направлению.}}
Для упорядоченного подмножества пути  $\{(i,j_r,k,l_r)\}_{r=1}^{R}
\hm\subset\boldsymbol{\pi}$ выполняется хотя бы одно из условий 
монотонности функции выравнивания времени: 
$$
j_{r}-j_{r-1}\geq1\,,\quad l_r-l_{r-1}\geq1\,, \quad r = 2,\ldots,R\,.
$$

Свойства пути между матрицами обобщают свойства пути между двумя 
временными рядами.

\smallskip

\noindent
\textbf{Определение~3.}\ {Стоимость 
$\mathrm{Cost}\,(\textbf{A},\textbf{B},{\boldsymbol{\pi}})$ пути $\boldsymbol{\pi}$ 
между объектами $\textbf{A}, \textbf{B}$:
\begin{equation*}
\mathrm{Cost}\,(\textbf{A},\textbf{B},{\boldsymbol{\pi}}) = 
\sum\limits_{(i,j,k,l) \in \boldsymbol{\pi}}{\boldsymbol{\Omega}}(i,j,k,l).
\end{equation*}}

\noindent
\textbf{Определение~4.}\ 
{Выравнивающий путь~$\hat{\boldsymbol{\pi}}$ между 
объектами $\textbf{A},\textbf{B}$~--- путь наименьшей стоимости 
среди всех возможных путей между объектами:
\begin{equation*}
\hat{\boldsymbol{\pi}} = 
\argmin\limits_{{\boldsymbol{\pi}}} \mathrm{Cost}
\left(\textbf{A},\textbf{B},{\boldsymbol{\pi}}\right).
\end{equation*}}
Функция расстояния~$\rho (\textbf{A},\textbf{B})\hm = \mathrm{mDTW}\,
(\textbf{A},\textbf{B})$ между объектами~$\textbf{A}$ и~$\textbf{B}$ 
рассчитывается как стоимость выравнивающего пути~$\hat{\boldsymbol{\pi}}$:
\begin{equation}
\mathrm{mDTW}(\textbf{A},\textbf{B}) = \mathrm{Cost}\left(\textbf{A},
\textbf{B},\hat{\boldsymbol{\pi}}\right).
\end{equation}

\setcounter{figure}{1}
\begin{figure*}[b] %fig2
{\small 
\begin{center}
\begin{tabular}{l}
\hline
DTW(\textbf{s},\textbf{c}):\\
\hspace*{3mm}$\boldsymbol{D}$(1:n+1,1:m+1) = inf;\\
\hspace*{3mm}$\boldsymbol{D}$(1,1) = 0;\\
\hspace*{3mm}for $i = 2$: $n+1$\\
\hspace*{6mm}for $j = 2$ : $m+1$\\
\hspace*{9mm}$d = (\textbf{s}(i-1)-\textbf{c}(j-1))^2$;\\
\hspace*{9mm}$\boldsymbol{D}(i,j) = d + \min( 
[ \boldsymbol{D}(i-1,j), \boldsymbol{D}(i,j-1), \boldsymbol{D}(i-1,j-1) ])$;\\
return\ sqrt$(\boldsymbol{D}(n+1,m+1))$\\
\hline
\end{tabular}
\end{center}}
\vspace*{-9pt}

\Caption{Алгоритм вычисления DTW для временных рядов
\label{ris:dtwts}}
%\end{figure*}
%\begin{figure*} %fig3
\vspace*{6pt}
{\small 
\begin{center}
\begin{tabular}{l}
\hline
\\[-9pt]
Correction $(\overline{i,j,k,l}, \boldsymbol{\pi}(\overline{i,j,k,l})):$\\
\hspace*{3mm}if $\overline{i,j,k,l} \in \{ (i-1, j, k,l)  ;  
(i, j, k-1, l)  ;  (i-1, j, k-1, l) \}$:\\
\hspace*{6mm}$ \widehat{\pi} = \{ (\overline{i}, r, \overline{k}, f) \in 
\boldsymbol{\pi}(\overline{i, j, k, l}) \vert r, f \in \mathbb{N} \}$\\
\hspace*{3mm}elif $\overline{i,j,k,l}\in \{  
(i, j-1, k, l); (i, j, k, l-1); (i, j-1, k, l-1) \}$:\\
\hspace*{6mm}$\widehat{\pi} = \{ (r, \overline{j}, f, \overline{l}) 
\in \boldsymbol{\pi}(\overline{i, j, k, l}) \vert r, f \in \mathbb{N} \}$\\
\hspace*{3mm}elif $\overline{i,j,k,l} =  i-1,j-1,k-1,l-1:$\\
\hspace*{6mm}$\widehat{\pi} = \{ (\overline{i}, r, \overline{k}, f) 
\in \boldsymbol{\pi}(\overline{i, j, k, l}) \vert r,f \in \mathbb{N} \} \cup$\\
\hspace*{6mm}$\cup \{ (r, \overline{j}, f, \overline{l}) \in \boldsymbol{\pi}
(\overline{i, j, k, l}) \vert r,f \in \mathbb{N} \}$\\
\hspace*{3mm}$\boldsymbol{d\pi} = \{ \mathrm{element} \in \widehat{\pi}: 
\mbox{произведены\ замены\ индексов } 
\overline{i} = i,\ \overline{j} = j,\ \overline{k} = k,\ \overline{l} = l \}$\\
return $\boldsymbol{d\pi}$\\
\hline
\end{tabular}
\end{center}
}
\vspace*{-9pt}

\Caption{Алгоритм вычисления поправки $\boldsymbol{d\pi}$ 
пути $\boldsymbol{\pi}$
\label{ris:codedpi}}
\end{figure*}


\textbf{Алгоритм вычисления значения расстояния~(4).}
Построение алгоритма вычисления значения функции расстояния 
между матрицами основан на алгоритме расчета функции расстояния 
между временн$\acute{\mbox{ы}}$ми рядами. В~случае выравнивания одной\linebreak\vspace*{-12pt}

{ \begin{center}  %fig1
 \vspace*{-3pt}
    \mbox{%
 \epsfxsize=79mm 
 \epsfbox{gon-1.eps}
 }


\end{center}


\noindent
{{\figurename~1}\ \ \small{Матрица стоимости оптимального выравнивания, по обеим 
осям отложены временные отсчеты}}
}

\vspace*{12pt}


\noindent 
временн$\acute{\mbox{о}}$й шкалы
 итоговая матрица расстояний~$\boldsymbol{D}$ (рис.~1) в~каждом 
 элементе~$\boldsymbol{D}(i,j)$ содержит рас\-сто\-яние между подрядом 
 первого временн$\acute{\mbox{о}}$го ряда и~подрядом второго временн$\acute{\mbox{о}}$го ряда. 
 Рас\-смот\-рим алгоритм динамического выравнивания двух временн$\acute{\mbox{ы}}$х 
 рядов $\textbf{s} \hm\in R^n$ и~$\textbf{c} \hm\in R^m$ на рис.~2.
 
 

Элемент $\boldsymbol{D}(i,j)$ матрицы~$\boldsymbol{D}$ соответствует 
стоимости выравнивающего пути между подпоследовательностями 
исходных временн$\acute{\mbox{ы}}$х рядов: $\textbf{s}(1:i) \hm= \textbf{s}(t)$, 
$t \hm= 1,\ldots,i,$ и~$\textbf{c}(1:j) \hm= \textbf{c}(t)$, $t \hm= 1,\ldots,j$. 
Алгоритм построения наилучшего выравнивания времени 
подразумевает, что выравнивающий путь между этими 
подпоследовательностями получен одним из трех способов~--- 
если стоимость выравнивающего пути между 
подпоследовательностями~$\textbf{s}(1:\overline{i}) $ 
и~$\textbf{c}(1:\overline{j})$ минимальна для~$\overline{i,j}$ из множества
$$
\overline{i,j} \in \left\{ \{i-1,j\},\{i,j-1\},\{i-1,j-1\} \right\},$$
тогда выравнивающий путь между $\textbf{s}(1:i)$ и~$\textbf{c}(1:j)$ получен добавлением пары~$(i,j)$ к~выбранному 
выравнивающему пути с~минимальной стоимостью из трех.



Предложенный алгоритм переносит эти рас\-суж\-де\-ния на случай 
выравнивания двух матриц~$\textbf{A}$ и~$\textbf{B}$. 
Элемент~$\boldsymbol{D}(i,j,k,l)$ четырехиндексного
 тензора расстояний~$\boldsymbol{D}$ соответствует стоимости выравнивающего 
 пути между $\textbf{A}(1:i,1:j) \hm= \textbf{A}(t_1,t_2)$, 
 $t_1 \hm= 1,\ldots, i$, $t_2 \hm= 1,\ldots, j,$ 
 и~$\textbf{B}(1:k,1:l) \hm= \textbf{B}(t_1,t_2)$, $t_1 \hm= 1,\ldots, k$,
 $t_2 \hm= 1,\ldots, l$. Выравнивающий путь между этими 
 подматрицами получен одним из семи способов~--- 
 если стоимость выравнивающего пути между 
 подматрицами $\textbf{A}(1:\overline{i},1:\overline{j})$ 
 и~$\textbf{B}(1:\overline{k},1:\overline{l})$ 
 минимальна для~$\overline{i,j,k,l}$ из множества
\begin{multline*} 
\overline{i,j,k,l} \in 
\left\{ \{i-1,j,k,l\},\{i,j-1,k,l\},\right.\\
\{i,j,k-1,l\},
\{i,j,k,l-1\}, \{i-1,j,k-1,l\},\\
\left.
\{i,j-1,k,l-1\},\{i-1,j-1,k-1,l-1\}\right\},
\end{multline*}

\setcounter{figure}{3}
\begin{figure*} %fig4
{\small 
\begin{center}
\begin{tabular}{l}
\hline
$\mathrm{mDTW}\left(\textbf{A},\textbf{B}\right):$\\
\hspace*{3mm}$\textbf{D}(1:n+1,1:n+1, 1:n+1, 1:n+1) = inf$;\\
\hspace*{3mm}$\textbf{D}(1,1,1,1) = 0;$\\
\hspace*{3mm}$\boldsymbol{\pi}(1,1,1,1) = ((1,1),(1,1))$\\
\hspace*{3mm}$for\ i,j,k,l  \in \mathbb{N}^{2 : n+1} \times 
\mathbb{N}^{2 : n+1} \times \mathbb{N}^{2 : n+1} \times \mathbb{N}^{2 : n+1}:$\\
\hspace*{6mm}$\overline{i,j,k,l} = \argmin($ [ \textbf{D}(i-1, j, k, l), 
\textbf{D}(i, j-1, k, l), \textbf{D}(i, j, k-1, l), 
\textbf{D}(i, j, k, l-1),    \\
\hspace*{9mm}$\textbf{D}(i-1, j, k-1, l), \textbf{D}(i, j-1, k, l-1), 
\textbf{D}(i-1, j-1, k-1, l-1) ])$;\\
\hspace*{3mm}$\boldsymbol{d \pi} = \mathrm{Correction}\,(\overline{i,j,k,l}, 
\boldsymbol{\pi}(\overline{i,j,k,l}))$\\
\hspace*{3mm}$\boldsymbol{\pi}(i, j, k, l) = \boldsymbol{d \pi} \cup 
\{(\overline{i,j,k,l})\}$\\
\hspace*{3mm}$\mathrm{cost} = (\textbf{A}(i, j)-\textbf{B}(k, l))^2 + 
\sum\nolimits_{(r,f,t,g) \in \boldsymbol{d \pi}}
(\textbf{A}(r, f)-\textbf{B}(t, g))^2$;\\
\hspace*{3mm}$\textbf{D}(i,j,k,l) = \mathrm{cost} + \textbf{D}
(\overline{i,j,k,l})$\\
return  sqrt$(\textbf{D}(n+1,n+1,n+1,n+1))$\\
\hline
\end{tabular}
\end{center}
}
\vspace*{-9pt}

\Caption{Алгоритм вычисления расстояния между матрицами
\label{ris:matrixdtw}}
\end{figure*}

\begin{table*}[b]\small
\begin{center}
\begin{tabular}{|l|c|c|c|c|}
\multicolumn{5}{c}{Снижение расстояний при выполнении преобразований 
для различных наборов данных}\\
\multicolumn{5}{c}{\ }\\[-6pt]
\hline
 &\multicolumn{4}{c|}{Метод}\\
 \cline{2-5}
\multicolumn{1}{|c|}{Данные}  & \multicolumn{2}{c|}{$L_2$} & \multicolumn{2}{c|}{MatrixDTW} \\
\cline{2-5}
& $S(f|p)$  &  $S_{\rho}(\mathfrak{D})$ &  $S(f|p)$ & $S_{\rho}(\mathfrak{D})$ \\
\hline
Модельные данные без преобразований& 92\% & 78\% & 100\%\hphantom{9} & 85\% \\
Модельные данные с~преобразованиями & 86\% & 65\% &  100\%\hphantom{9} & 82\% \\
Модельные данные с~преобразованиями и~шумом& 69\% & 61\% &  92\% & 78\% \\
MNIST без преобразований& 95\% & --- & 95\% & --- \\
MNIST с~преобразованиями & 53\% & --- & 92\% & --- \\
Спектр сигнала& 83\% & --- & 96\% & --- \\
\hline
\end{tabular}
\end{center}
\end{table*}

\noindent
то к~выравнивающему пути между этими под\-мат\-ри\-ца\-ми 
добавляется элемент пути $(i,j,k,l)$ и~поправка~$\boldsymbol{d\pi} $ 
пути~$\boldsymbol{\pi}$, алгоритм вычисления которой приведен ниже.

Обозначим выравнивающий путь между $\textbf{A}(1:i,\linebreak 1:j)$
 и~$\textbf{B}(1:k,1:l)$ как~$\boldsymbol{\pi}(i,j,k,l)$, тогда 
 поправка~$\boldsymbol{d\pi} $ пути~$\boldsymbol{\pi}(i,j,k,l)$ 
 при фиксированных~$\overline{i,j,k,l}$ вычисляется приведенным на рис.~3 
 образом.





Алгоритм динамического выравнивания двух матриц и~вычисления 
расстояния $\mathrm{mDTW}$ между ними с~учетом приведенного выше 
алгоритма примет вид, представленный на рис.~4.





\begin{figure*} %fig5
\vspace*{1pt}
    \begin{center}  
  \mbox{%
 \epsfxsize=161.412mm 
 \epsfbox{gon-5.eps}
 }
\end{center}
\vspace*{-12.5pt}
\Caption{Выравнивание модельных данных: (\textit{а})~один класс без шума; 
(\textit{б})~разные классы без шума; 
(\textit{в})~один класс с~шумом; (\textit{г})~разные классы с~шумом
\label{ris:random}}
%\end{figure*}
%\begin{figure*} %fig6
\vspace*{1pt}
    \begin{center}  
  \mbox{%
 \epsfxsize=163mm 
 \epsfbox{gon-6.eps}
 }
\end{center}
\vspace*{-12.5pt}
\Caption{Выравнивание данных MNIST: левый столбец~--- один класс; 
правый столбец~--- разные 
классы;
(\textit{а})~$\mathrm{mDTW}\hm=720{,}1$; 
(\textit{б})~948,6;
(\textit{в})~2017,0;
(\textit{г})~$\mathrm{mDTW}\hm=2071{,}4$
\label{ris:mnist}}
\end{figure*}


Следует отметить, что алгоритм~\cite{15} имеет\linebreak высокую сложность 
вычисления~--- $O(n^4)$. Предполагается ускорение метода 
с~использованием ограниче\-ния Sakoe-Chiba band, что сократит 
вычислительную сложность алгоритма до $O(n^2k^2)$, где~$k$~--- 
параметр ограничения.


\section{Вычислительный эксперимент}

Вычислительный эксперимент проведен на модельных данных с~допустимыми 
преобразованиями и~на реальных данных: объектах коллекции MNIST с~допустимыми 
преобразованиями и~на спектрограммах зашумленных сигналов.





Решается задача метрической классификации методом ближайшего соседа. В~таблице 
приведены значения критерия качества функции расстояния 
$S_{\rho}(\mathfrak{D})$ и~критерия качества метрической классификации $S(f|p)$ 
при использовании двух функций расстояния: предложенной в~работе $\mathrm{mDTW}$ 
и~$L_2$.

Модельные данные~--- это нулевые матрицы со случайными ненулевыми 
строками, столбцами, подпрямоугольниками с~наложенным шумом. 
К~ним применены допустимые преобразования, согласованные с~гипотезой 
наличия локальных и~глобальных искажений. На рис.~\ref{ris:random} 
показан пример оптимального выравнивания двух объектов. 
Линиями показаны элементы пути~$\boldsymbol{\pi}$.

Подготовлена подвыборка набора данных MNIST. Она 
состоит из~100 объектов классов 0 и~1 сниженной размерности
 с~допустимыми преобразованиями. На рис.~\ref{ris:mnist} 
 показан пример оптимального выравнивания объектов.


Аналогичный эксперимент проведен для решения задачи метрической 
классификации спектров различных сигналов, пример которых приведен на 
рис.~\ref{ris:spectr}. На рисунке показаны примеры Фурье-спект\-ров 
этих сигналов. Спектр получен путем применения быстрого преобразования 
Фурье к~исходному сигналу для различных окон с~фиксированным размером и~сдвигом. 
Исходные временн$\acute{\mbox{ы}}$е ряды обладали свойством периодичности, период выбирался 
случайным образом.



Тестирование проведено на разного рода данных: исходных 
модельных данных без наложения\linebreak\vspace*{-12pt}

\pagebreak

\end{multicols}

\begin{figure*} %fig7
\vspace*{1pt}
    \begin{center}  
  \mbox{%
 \epsfxsize=149.062mm 
 \epsfbox{gon-7.eps}
 }
\end{center}
\vspace*{-8pt}
\Caption{Данные спектров сигнала: (\textit{а})~класс~1; (\textit{б})~спектр 
класса~1; (\textit{в})~класс~2; (\textit{г})~спектр класса~2; 
(\textit{д})~класс~3; (\textit{е})~спектр класса~3
\label{ris:spectr}}
\vspace*{9pt}
\end{figure*}

\begin{multicols}{2}

\noindent допустимых преобразований, с~ними, а~также 
на модельных данных с~наложенным поверх объектов случайным шумом.



В каждом из проведенных экспериментов была продемонстрирована 
устойчивость предложенного подхода к~допустимым преобразованиям. 
Наилучшее значение критерия качества задачи классификации было 
достигнуто при использовании предложенной функции расстояния.

\vspace*{-5pt}

\section{Заключение}

В работе предложено обобщение метода динамического выравнивания
 временн$\acute{\mbox{ы}}$х рядов для случая объектов, определенных на двух осях времени. 
 Существует теоретическое обобщение предлагаемых методов на случай 
 конечного множества осей времени. Вычислительный эксперимент позволил 
 проанализировать свойства подхода: устойчивость к~допустимым 
 преобразованиям и~разделяющая способность функции расстояния как 
 на реальных, так и~на модельных данных. Качество решения задачи 
 метрической классификации выше решения, основанного на евклидовом 
 расстоянии. Вычислительная сложность метода высокая, что ограничивает 
 его применимость на объектах высокой размерности.

\vspace*{-2pt}

{\small\frenchspacing
 {%\baselineskip=10.8pt
 \addcontentsline{toc}{section}{References}
 \begin{thebibliography}{99}
%\bibitem{Karasikov2016}
%\Au{Карасиков~М.\,Е., Стрижов~В.\,В.} Классификация временных рядов 
%в~пространстве параметров по\-рож\-да\-ющих моделей~// Информатика и~её 
%применения,~2016. T.~10. Вып.~4. С.~121--131.

\bibitem{0}
\Au{Hill~N.\,J., Lal~T.\,N., Schroder~M., Hinterberger~T., 
Wilhelm~B., Nijboer~F., Mochty~U., Widman~G., Elger~C., 
Scholkopf~B., Kubler~A., Birbaumer~N.} Classifying EEG and 
ECoG signals without subject training for fast BCI implementation: 
Comparison of nonparalyzed and completely paralyzed subjects~//  
IEEE~T. Neur. Sys. Reh., 2006. Vol.~14. 
Iss.~2. P.~183--186.

\bibitem{1}
\Au{Sakoe~H., Chiba~S.} 
A~dynamic programming approach to continuous speech recognition~// 
7th  Congress (International) on Acoustics Proceedings, 1971. Vol.~3. P.~65--69.

\bibitem{2} %3
\Au{Aghabozorgi~S., Ali~S.\,S., Wah~T.\,Y.} 
Time-series clustering~--- a~decade review~// Inform. Syst., 
2015. Vol.~53. P.~16--38.

\bibitem{3} %4
\Au{Warrenliao~T.} Clustering of time series data~--- a~survey~// 
Pattern Recogn., 2005. Vol.~38. Iss.~11. P.~1857--1874.



\bibitem{4} %5
\Au{Hautamaki~V., Nykanen~P., Franti~P.} 
Time-series clustering by approximate prototypes~// 
19th  Conference (International) on Pattern Recognition Proceedings, 2008. No.\,D. 
P.~1--4.

\bibitem{5} %6
\Au{Faloutsos~C., Ranganathan~M., Manolopoulos~Y.} 
Fast subsequence matching in time-series databases~// \mbox{SIGMOD} Rec., 1994. 
Vol.~23. Iss.~2. P.~419--429.

\bibitem{10} %7
\Au{Basalto~N., Bellotti~R., Carlo~F.\,D., Facchi~P., 
Pascazio~S.} Hausdorff clustering of financial time series~// 
Physica~A, 2007. Vol.~379. Iss.~2. P.~635--644.

\bibitem{11} %8
\Au{Gorelick~L., Blank~M., Shechtman~E., Irani~M., Basri~R.} 
Actions as space-time shapes~// IEEE~T. Pattern Anal., 
2007. Vol.~29. Iss.~12. P.~2247--2253.

\bibitem{6} %9
\Au{Smyth~P.} Clustering sequences with hidden Markov models~// 
Adv. Neural In., 1997. Vol.~9. P.~648--654.

\bibitem{7} %10
\Au{Banerjee~A., Ghosh~J.} Clickstream clustering using weighted 
longest common subsequences~// 
Workshop on Web Mining, SIAM Conference on Data Mining
Proceedings, 2001. P.~33--40.

\bibitem{8} %11
\Au{Aach~J., Church~G.M.} Aligning gene expression time series
 with time warping algorithms~// Bioinformatics, 2001. Vol.~17. Iss.~6. P.~495--508.

\bibitem{9} %12
\Au{Yi~B.\,K., Faloutsos~C.} Fast time sequence indexing 
for arbitrary $\mathcal{L}_p$ norms~// 
26th  Conference (International) on Very Large Data Bases Proceedings, 2000. P.~385--394.

\bibitem{33} %13
\Au{Goncharov~A.\,V., Strijov~V.\,V.} 
Analysis of dissimilarity set between time series~// Computational 
Mathematics Modeling, 2018. Vol.~29. Iss.~3. P.~359--366.

\bibitem{12} %14
\Au{Alon~J., Athitsos~V., Sclaroff~S.}
 Online and offline character recognition using alignment to prototypes~// 
 8th  Conference (International) on Document Analysis and Recognition, 2005. 
 Vol.~2. P.~839--843.

\bibitem{15} %15
\Au{Гончаров~А.\,В.} 
Выравнивания декартовых произведений упорядоченных множеств mDTW. 
Про\-грам\-мная реализация алгоритма, 2019. 
{\sf https://github.
com/Intelligent-Systems-Phystech/PhDThesis/tree/\linebreak  master/Goncharov2019/MatrixDTW/code}.
 \end{thebibliography}

 }
 }

\end{multicols}

\vspace*{-9pt}

\hfill{\small\textit{Поступила в~редакцию 24.04.19}}

\vspace*{6pt}

%\pagebreak

%\newpage

%\vspace*{-28pt}

\hrule

\vspace*{2pt}

\hrule

\vspace*{-4pt}

\def\tit{ALIGNMENT OF~ORDERED SET CARTESIAN PRODUCT\\[-5pt]}


\def\titkol{Alignment of~ordered set cartesian product}

\def\aut{A.\,V.~Goncharov$^1$ and~V.\,V.~Strijov$^{1,2}$}

\def\autkol{A.\,V.~Goncharov and~V.\,V.~Strijov}

\titel{\tit}{\aut}{\autkol}{\titkol}

\vspace*{-13pt}


\noindent
$^1$ Moscow Institute of Physics and Technology, 
9~Institutskiy Per., Dolgoprudny, Moscow Region 141700, Russian\linebreak
$\hphantom{^1}$Federation


\noindent
$^2$A.\,A.~Dorodnicyn Computing Center, Federal Research Center 
``Computer Science and Control'' of the Russian\linebreak
$\hphantom{^1}$Academy of Sciences, 
40~Vavilov Str., Moscow 119333, Russian Federation

\def\leftfootline{\small{\textbf{\thepage}
\hfill INFORMATIKA I EE PRIMENENIYA~--- INFORMATICS AND
APPLICATIONS\ \ \ 2020\ \ \ volume~14\ \ \ issue\ 1}
}%
 \def\rightfootline{\small{INFORMATIKA I EE PRIMENENIYA~---
INFORMATICS AND APPLICATIONS\ \ \ 2020\ \ \ volume~14\ \ \ issue\ 1
\hfill \textbf{\thepage}}}

\vspace*{2pt} 



\Abste{The work is devoted to the study of metric methods for analyzing 
objects with complex structure. It proposes to generalize the dynamic 
time warping method of two time series for the case of objects defined 
on two or more time axes. Such objects are matrices in the discrete 
representation. The DTW (Dynamic Time Warping) method of time series is generalized as 
a~method of matrices dynamic alignment. The paper proposes 
a~distance function resistant to monotonic nonlinear deformations of the 
Cartesian product of two time scales. The alignment path between objects is 
defined. An object is called a~matrix in which the rows and columns correspond 
to the axes of time. The properties of the proposed distance function 
are investigated. To illustrate the method, the problems of metric 
classification of objects are solved on model data and data from the 
MNIST dataset.}

\KWE{distance function; dynamic alignment; distance between matrices; 
nonlinear time warping; space--time series}



\DOI{10.14357/19922264200105} 

%\vspace*{-14pt}

\Ack
\noindent
This work was supported by the Russian Foundation for Basic
Research (projects 19-07-1155 and 19-07-00885). 
The paper contains results of the project Statistical 
methods of machine learning, which is carried out within the 
framework of the Program ``Center of Big Data Storage and Analysis'' 
of the National Technology Initiative Competence Center. 
It is supported by the Ministry of Science and Higher Education 
of the Russian Federation according to the agreement between the
 M.\,V.~Lomonosov Moscow State University and the Foundation 
 of project support of the National Technology Initiative from 11.12.2018, 
 No.\,13/1251/2018.
 


%\vspace*{6pt}

  \begin{multicols}{2}

\renewcommand{\bibname}{\protect\rmfamily References}
%\renewcommand{\bibname}{\large\protect\rm References}

{\small\frenchspacing
 {%\baselineskip=10.8pt
 \addcontentsline{toc}{section}{References}
 \begin{thebibliography}{99}

 \bibitem{0-1}   
\Aue{Hill, N.\,J., T.\,N.~Lal, M.~Schroder, T.~Hinterberger, B.~Wilhelm, 
F.~Nijboer, U.~Mochty, G.~Widman, C.~Elger, B.~Scholkopf, A.~Kubler, and 
N.~Birbaumer.} 2006. Classifying EEG and ECoG signals without subject 
training for fast BCI implementation: Comparison of nonparalyzed and completely 
paralyzed subjects. \textit{IEEE~T. Neur. Sys. 
Reh.} 14(2):183--186.

\bibitem{1-1}   
\Aue{Sakoe, H., and S.~Chiba.} 1971. A~dynamic programming approach 
to continuous speech recognition. \textit{7th 
 Congress (International) on Acoustics Proceedings}. 3:65--69.

\bibitem{2-1}    %2
\Aue{Aghabozorgi,~S., S.\,S.~Ali, and T.\,Y.~Wah.} 2015. 
Time-series clustering~--- a~decade review.  \textit{Inform. Syst.} 
53:16--38.

\bibitem{3-1}   %4 
\Aue{Warrenliao,~T.} 2005. Clustering of time series data~--- a~survey. 
\textit{Pattern Recogn.}
38(11):1857--1874.



\bibitem{4-1}    %5
\Aue{Hautamaki,~V., P.~Nykanen, and P.~Franti.} 2008. 
Time-series clustering by approximate prototypes. 
 \textit{19th  Conference (International) on Pattern Recognition Proceedings}. 
 D:1--4.

\bibitem{5-1}    %6
\Aue{Faloutsos,~C., M.~Ranganathan, and Y.~Manolopoulos.} 1994. 
Fast subsequence matching in time-series databases.  \textit{SIGMOD Rec}. 
23(2):419--429.

\bibitem{10-1}    %7
\Aue{Basalto, N., R.~Bellotti, F.\,D.~Carlo, P.~Facchi, and S.~Pascazio.} 
2007. Hausdorff clustering of financial time series. 
\textit{Physica~A} 379(2):635--644.

\bibitem{11-1}   %8
\Aue{Gorelick, L., M.~Blank, E.~Shechtman, M.~Irani, and R.~Basri.} 
2007. Actions as space-time shapes.
\textit{IEEE~T. Pattern Anal.} 29(12):2247--2253.

\bibitem{6-1}    %9
\Aue{Smyth, P.} 1997. 
Clustering sequences with hidden Markov models. \textit{Adv. Neural In.} 9:648--654.

\bibitem{7-1}    %10
\Aue{Banerjee,~A., and J.~Ghosh.} 2001. 
Clickstream clustering using weighted longest common subsequences.  
\textit{Workshop on Web Mining, SIAM Conference 
on Data Mining Proceedings.} 33--40.

\bibitem{8-1}    %11
\Aue{Aach, J., and G.\,M.~Church.} 2001. 
Aligning gene expression time series with time warping algorithms. 
\textit{Bioinformatics} 17(6):495--508.

\bibitem{9-1}   %12
\Aue{Yi, B.\,K., and C.~Faloutsos.} 2000. 
Fast time sequence indexing for arbitrary $\mathcal{L}_p$ norms. 
\textit{26th  Conference (International) 
on Very Large Data Bases Proceedings}. 385--394.

\bibitem{33-1}   %13 
\Aue{Goncharov,~A.\,V., and V.\,V.~Strijov.} 2018. 
Analysis of dissimilarity set between time series. 
\textit{Computational Mathematics Modeling } 29(3):359--366.



\bibitem{12-1}    %14
\Aue{Alon, J., V.~Athitsos, and S.~Sclaroff.} 2005.
 Online and offline character recognition using alignment to prototypes. 
 \textit{8th  Conference (International) on Document Analysis and Recognition}. 
 2:839--843.

\bibitem{15-1}    %15
\Aue{Goncharov, A.\,V.} Alignment of 
Ordered Set Cartesian Product mDTW. Software implementation of the algorithm. 
Available at: {\sf https://github.com/Intelligent-\linebreak 
Systems-Phystech/PhDThesis/tree/master/Goncharov\linebreak 2019/MatrixDTW/code} 
(accessed December~27, 2019).
\end{thebibliography}

 }
 }

\end{multicols}

%\vspace*{-7pt}

\hfill{\small\textit{Received April 24, 2019}}

%\pagebreak

%\vspace*{-22pt}



\Contr

\noindent
\textbf{Goncharov Alexey V.} (b.\ 1995)~--- 
PhD student, Moscow Institute of Physics and Technology, 
9~Institutskiy Per., Dolgoprudny, Moscow Region 141701, 
Russian Federation; \mbox{alex.goncharov@phystech.edu}

\vspace*{3pt}

\noindent
\textbf{Strijov Vadim V.} (b.\ 1967)~--- 
Doctor of Science in physics and mathematics, leading scientist, 
A.\,A.~Dorodnicyn Computing Centre, Federal Research Center 
``Computer Science and Control'' of the Russian Academy of Sciences, 
40~Vavilov Str., Moscow 119333, Russian Federation;
 professor, Moscow Institute of Physics and Technology, 
 9~Institutskiy Per., Dolgoprudny, Moscow Region 141701, Russian Federation; 
 \mbox{strijov@ccas.ru}
\label{end\stat}

\renewcommand{\bibname}{\protect\rm Литература} %15
\def\stat{zatsman}

\def\tit{ТРАНСФОРМАЦИИ ОБЪЕКТОВ ПЕРВОГО И~ВТОРОГО ПОРЯДКА 
В~ЛЕКСИКОГРАФИЧЕСКОЙ ИНФОРМАЦИОННОЙ СИСТЕМЕ$^*$}

\def\titkol{Трансформации объектов первого и~второго порядка 
в~лексикографической информационной системе}

\def\aut{И.\,М.~Зацман$^1$}

\def\autkol{И.\,М.~Зацман}

\titel{\tit}{\aut}{\autkol}{\titkol}

\index{Зацман И.\,М.}
\index{Zatsman I.\,M.}


{\renewcommand{\thefootnote}{\fnsymbol{footnote}} \footnotetext[1]
{Исследование выполнено в~ФИЦ ИУ РАН за счет гранта Российского научного фонда №\,24-18-00155, {\sf 
https://rscf.ru/project/24-18-00155}. Работа выполнялась с~использованием инфраструктуры Центра 
коллективного пользования <<Высокопроизводительные вычисления и~большие данные>> (ЦКП 
<<Информатика>>) ФИЦ ИУ РАН (г.\ Москва).}}


\renewcommand{\thefootnote}{\arabic{footnote}}
\footnotetext[1]{ Федеральный исследовательский центр <<Информатика и~управление>> Российской академии наук, 
\mbox{izatsman@yandex.ru}}

\vspace*{-12pt}


  
  \Abst{Рассматриваются теоретические основания проектирования информационных 
технологий (ИТ) интеграции двуязычных словарей и~параллельных корпусов. Дано описание 
первых результатов создания третьего уровня классификации трансформаций объектов 
предметной области информатики, которую предполагается использовать при создании 
концепции лексикографической информационной системы, обеспечивающей интеграцию. 
Все сущности информатики в~статье разделены на два глобальных класса: объекты и~их 
трансформации. Для каждого такого класса конструируется своя классификация. Ранее были 
описаны два верхних уровня классификации трансформаций объектов предметной области. 
В~данной статье рассматривается третий уровень этой классификации. Основанием для 
построения самого верхнего ее уровня служило деление предметной области информатики 
на среды (ментальная, сенсорно воспринимаемая, цифровая и~ряд других сред), каждая из 
которых по определению включает объекты одной природы. Основанием для построения 
второго уровня классификации трансформаций объектов служила типология знаковых  
сис\-тем А.~Соломоника. Цель статьи состоит в~систематизации трансформаций первого 
и~второго порядка объектов предметной области на третьем уровне этой классификации. 
Основанием для систематизации служит средовая версия иерархии Акоффа.}
  
  \KW{объекты предметной области; трансформации объектов; классификация; данные; 
информация; знание; лексикографическая информационная сис\-тема}

\DOI{10.14357/19922264240211}{VZTGVV}
  
\vspace*{3pt}


\vskip 10pt plus 9pt minus 6pt

\thispagestyle{headings}

\begin{multicols}{2}

\label{st\stat}
  
\section{Введение}

\vspace*{-9pt}

  Возникновение параллельных корпусов, в~которых предложениям 
оригинального текста со\-по\-став\-ле\-ны предложения его перевода, обеспечило 
возможность контрастивного лингвистического\linebreak \mbox{анализа} на принципиально 
новом уровне полноты и~точности, недостижимом в~докорпусную эпоху. 
Пионерскими в~этой области стали работы \mbox{1990-х~гг}. Стига Йоханссона  
с~анг\-ло-нор\-веж\-ским корпусом~[1]. В России параллельные корпусы стали 
формироваться в~начале XXI~века в~рамках Национального корпуса русского 
языка~[2].
  
  Создатели двуязычных словарей используют параллельные корпусы для 
сбора материала и~эмпирической проверки своих гипотез, касающихся 
межъязы\-ко\-вой эквивалентности. Ценность параллельных корпусов 
определяется тем, что в~лингвистике этап сбора исходного материала считается 
наиболее трудоемким и~наименее творческим, а~параллельные корпусы 
позволяют значительно сэкономить время и~силы для творческого этапа 
создания словарей~[3].
 % 
  При этом двуязычные словари, создаваемые на основе исходного материала, 
извлеченного из параллельных корпусов, сейчас формируются без связей с~их 
текстами. Другими словами, онлайновые связи созданных словарей 
с~параллельными корпусами, которые служили источниками исходного 
материала, отсутствуют. 

Параллельные корпусы постоянно пополняются 
новыми текстами, в~предложениях которых можно обнаружить новые значения 
слов и~устойчивых словосочетаний. Однако при этом отсутствуют методы 
и~средства оперативного обновления словарей по корпусным данным. 
В~настоящее время проблема установления связей между двуязычными 
словарями и~параллельными корпусами (далее~--- проблема интеграции) 
находится на стадии поиска концептуальных подходов к~их интеграции на 
уровне значений.
  
  Подход к~решению проблемы интеграции, предлагаемый в~статье, учитывает 
  и~появление новых значений слов и~устойчивых словосочетаний, и~динамику 
смысловых значений, которая обусловлена развитием и~пополнением знания 
лингвистов, фиксирующих эти значения в~результате семантического анализа 
пополняемых корпусных данных. Проведенные эксперименты показали, что 
обнаружение нового лингвистического знания обусловливает и~формирование 
дефиниций новых значений, и~пересмотр уже существующих дефиниций~[4, 5].
  
  Например, в~проведенных экспериментах с~использованием ЦКП 
<<Информатика>> ФИЦ ИУ РАН фиксировалась эволюция значений немецких 
модальных глаголов, исходное состояние значений которых было описано 
в~не\-мец\-ко-рус\-ском словаре. В~экспериментальном массиве текстов как 
потенциальных источниках нового знания 16\,268 предложений содержали 
немецкие модальные глаголы и~в~2041 из них встречался глагол sollen. 
В~начале эксперимента в~словаре были описаны~12~значений этого модального 
глагола. По окончании эксперимента лингвисты обнаружили два новых его 
значения, согласовали их дефиниции и~описали эволюцию дефиниций~[6, 7].
  
  Таким образом, для решения проблемы интеграции требуется фиксировать 
новое знание, обнаруженное лингвистами в~текстовых данных параллельных 
корпусов, отслеживать эволюцию знания, представленного в~виде дефиниций 
значений слов и~устойчивых словосочетаний, и,~соответственно, 
актуализировать электронные двуязычные словари. Предлагаемый 
концептуальный подход к~интеграции, который планируется реализовать 
в~процессе проектирования лексикографической информационной сис\-те\-мы, 
фиксирующей эволюцию лингвистического знания, основан на решении 
следующих задач:\\[-14pt]
  \begin{itemize}
  \item категоризация трех базовых понятий информатики, включенных 
  в~иерархию Акоффа~[8] (данные, информация, знание), на объекты 
проектируемой сис\-те\-мы, которая необходима, чтобы фиксировать 
<<кванты>> нового знания и~отслеживать его эволюцию в~этой сис\-теме;\\[-15pt]
  \item  систематизация трансформаций объектов этой сис\-темы.\\[-14pt]
  \end{itemize}
  
  Цель статьи и~состоит в~решении двух задач: категоризации трех базовых 
понятий информатики на объекты лексикографической информационной  
сис\-те\-мы и~сис\-те\-ма\-ти\-за\-ции трансформаций первого и~второго порядка 
ее объектов.
  
  Трансформациями первого порядка, о которых сказано в~формулировке цели 
статьи, называются взаимные преобразования между двумя объектами  
сис\-те\-мы одной природы. Например, перевод в~сис\-те\-ме текста с~русского 
языка на английский относится к~ним. Трансформациями второго порядка 
и~выше называются взаимные преобразования между двумя и~более объектами 
разной природы. Например, кодирование символов текс\-та компьютерными 
кодами и~их декодирование относятся по определению к~трансформациям 
второго порядка.

%\vspace*{-9pt}
  
\section{Процессы трансформаций в~информатике}

%\vspace*{-3pt}

Процессы трансформаций, рассматриваемые в~статье, относятся к~теоретическому ядру информатики, а~не 
только к~проектированию лексикографической информационной сис\-те\-мы. Например, из трех основных 
подходов к~описанию предметной об\-ласти информатики\footnote{В статье предметная область информатики 
трактуется согласно концепции полиадического компьютинга Пола Розенблума~\cite{9-zac}.} (объектный, 
трансформационный и~синтетический) сис\-те\-ма\-ти\-за\-ция трансформаций ближе всего ко второму 
подходу. Примерами первого подхода, в~рамках которого основное внимание уделяется объектам предметной 
области информатики и~в~меньшей степени отношениям\linebreak между ними, могут служить  
работы~\cite{8-zac, 10-zac, 11-zac}; \mbox{примерами} второго подхода, в~рамках которого основное внимание 
уделяется трансформациям и~в~меньшей степени трансформируемым объектам,~---  
работы~\cite{12-zac, 13-zac}; примерами третьего, синтетического подхода, в~котором уделяется внимание 
и~объектам предметной об\-ласти информатики, и~отношениям между ними, могут служить работы~\cite{14-zac, 
15-zac, 16-zac, 17-zac, 18-zac}.

  Таким образом, для описания трансформаций объектов лексикографической 
информационной\linebreak системы предпочтительнее всего трансформационный 
подход, который упоминается и~в определениях информатики. Например, 
в~2009~г.\ П.~Деннинг и~П.~Розенблум сформулировали суть \mbox{информатики} как 
компьютинга следующим образом: <<$\ldots$информатика~--- это не просто 
алгоритмы и~структуры данных; это преобразования [трансформации] 
представлений>>~\cite{12-zac}. Чуть позже, в~контексте краткого описания 
парадигмы информатики как компьютинга, П.~Деннинг и~П.~Фриман изменили 
эту формулировку на такую: <<Центральный объект внимания в~информатике 
можно определить как информационные процессы~--- \textit{естественные или 
искусственные процессы, преобразующие информацию} (курсив мой~--- 
И.\,З.)>>~\cite{13-zac}. Согласно парадигме, предлагаемой авторами этой 
статьи, на начальном этапе проектирования автоматизированных систем 
базовыми элементами моделей их функционирования служат 
\textit{информационные про\-цессы}.
  
  Однако если 15~лет назад в~формулировке из работы~\cite{13-zac} шла речь 
о~процессах, преобразующих информацию, то в~последние~10~лет в~спектр 
процессов трансформаций все чаще стали включать процессы, преобразующие 
не только информацию, но также и~другие объекты автоматизированных 
систем, в~первую очередь данные и~знания~[19--21]. Например, Виктория 
Стодден, позиционируя науку о~данных как одну из дисциплин информатики, 
говорит, что центральный объект исследований в~науке о~данных~--- это 
<<изучение обобщаемого извлечения знания из данных>>~\cite{21-zac}. 
Увеличение и~чис\-ла объектов, и~спект\-ра процессов их трансформаций 
в~автоматизированных сис\-те\-мах обуслов\-ли\-ва\-ет не\-об\-хо\-ди\-мость 
систематизации и~объектов, и~процессов их трансформаций на начальном этапе 
проектирования сис\-тем.
  
  Для создания концепции лексикографической информационной сис\-те\-мы 
и~проектирования ИТ, обеспечивающих интеграцию 
двуязычных словарей и~параллельных корпусов, сначала выполним 
категоризацию на объекты этой сис\-те\-мы трех базовых понятий информатики 
(данные, информация, знание) в~контексте построения классификаций 
сущностей ее предметной об\-ласти.
  
  Необходимость использования классификаций информатики в~процессе 
создания концепции проиллюстрируем, используя иерархию  
Акоффа~\cite{8-zac}. Он использовал принцип их вертикального размещения 
в~иерархии снизу вверх: данные, информация и~знание. Еще в~ней есть термин 
<<мудрость>>, который в~статье не рассматривается. Такое размещение Акофф 
прокомментировал так: <<Каждое из пе\-ре\-чис\-лен\-ных понятий [кроме данных] 
содержит в~себе нижестоящие$\ldots$>>~\cite{8-zac}.
  
  Этому принципу размещения и~комментарию Акоффа свойственны 
недостатки, проанализированные, в~частности, в~работе~\cite{10-zac}. Главный 
вывод, к~которому пришла Роули после изучения иерархии Акоффа, 
заключается в~следующем: <<$\ldots$информация определяется в~терминах 
данных, знание~--- в~терминах информации$\ldots$ но существует меньше 
консенсуса в~описании трансформаций, которые преобразуют сущности, 
расположенные ниже в~иерархии, в~те, которые находятся над ними, что 
приводит к~их терминологической неопределенности>>~\cite{10-zac}. Причина 
этой неопределенности, скорее всего, в~том, что базовые понятия информатики 
включены в~иерархию Акоффа изолированно от общего контекста 
классификаций сущностей ее предметной об\-ласти.

%\vspace*{-9pt}
  
\section{Классификации сущностей информатики}


%\vspace*{-2pt}

  Все сущности предметной области информатики в~работах~[22, 23] 
разделены на два глобальных класса: ее объекты и~их трансформации. Для 
каждого такого класса была предложена своя классификация. 
В~работе~\cite{22-zac} дано описание классификации объектов предметной 
области информатики, первый уровень которой содержит базовые понятия ее 
предметной области (данные, информация, знания и~др.).  
В~работе~\cite{23-zac} дано описание двух верхних уровней классификации 
трансформаций объектов предметной об\-ласти (см.\ рисунок 
в~работе~\cite{23-zac}). Основанием для построения самого верхнего ее уровня послужило деление 
предметной области информатики на среды\footnote{В~работе~\cite{24-zac} дано описание пяти сред 
предметной области информатики (ментальная; сенсорно воспринимаемая, или информационная; 
цифровая; нейро- и~ДНК-среда), каждая из которых по определению включает объекты одной и~той же 
природы.} и~степень разнообразия природы объектов, вовлеченных в~трансформации:
\begin{itemize}
\item  первый класс верхнего уровня классификации включает 
трансформации объектов в~пределах среды только одной природы 
(трансформации первого порядка);
\item  второй класс включает трансформации объектов, относящихся 
к~двум средам разной природы (трансформации второго порядка);
\item третий и~последующие классы включают трансформации объектов, 
относящихся к~трем и~более средам разной природы (трансформации 
третьего и~более высоких порядков).
\end{itemize}

  В работе~\cite{23-zac} были приведены примеры для трех первых классов 
трансформаций, включая пример трансформаций объектов, относящихся 
к~двум средам разной природы (компьютерное кодирование символов текстов 
с~по\-мощью таб\-лиц Unicode).
  
Основанием для построения второго уровня классификации трансформаций объектов послужила типология 
знаковых сис\-тем А.~Соломоника~\cite[c.~131]{25-zac}: естественные знаковые сис\-те\-мы, образные,  
ес\-тест\-вен\-но-язы\-ко\-в$\acute{\mbox{ы}}$е,  
вер\-баль\-но-не\-сло\-вес\-ные сис\-те\-мы записи\footnote{Под системой записи понимается знаковая 
система, сочетающая вербальные знаки с~несловесными (языки нотной записи, карт, таблиц и~др.).} 
и~формализованные знаковые сис\-те\-мы, включая математические. Введем понятие обобщенного текста~--- 
это текст, который может быть создан в~любой из перечисленных знаковых систем. Тогда обобщенные тексты 
могут быть естественными, образными, ес\-тест\-вен\-но-язы\-ко\-в$\acute{\mbox{ы}}$\-ми,  
вер\-баль\-но-не\-сло\-вес\-ны\-ми и~формализованными. Второй уровень классификации трансформаций 
охватывает не все виды объектов предметной  
об\-ласти информатики, а~только перечисленные~5~видов текс\-тов и~их представления, вовлеченные 
в~процессы трансформаций в~одной или более средах вместе с~данными, знанием и~его концептами.

\begin{figure*}[b] %fig1
\vspace*{6pt}
      \begin{center}
     \mbox{%
\epsfxsize=121.191mm 
\epsfbox{zac-1.eps}
}
\end{center}
\vspace*{-6pt}
\Caption{Средовая версия иерархии Акоффа}
\end{figure*}

\section{Классификация трансформаций: построение~третьего 
уровня}

  Основанием для систематизации трансформаций первого и~второго порядка 
на третьем уровне этой классификации служит иерархия Акоффа~\cite{8-zac}, 
на основе которой и~была создана ее средов$\acute{\mbox{а}}$я версия~[26, 
27]. Для создания средов$\acute{\mbox{о}}$й версии была выполнена 
категоризация трех базовых понятий информатики (данные, информация, 
знания) на объекты лексикографической информационной сис\-те\-мы 
в~процессе создания ее концепции\linebreak (рис.~1).
  


  В отличие от классической иерархии Акоффа, в~ее 
средов$\acute{\mbox{о}}$й версии различаются три вида данных: сенсорно 
воспринимаемые, цифровые и~те данные, которые генерируются 
искусственными нейронными сетями (ИНС) в~системах искусственного интеллекта 
(далее~--- ИИ-дан\-ные). Последний вид данных необходим, например, для 
различения входа и~выхода процесса применения обученной 
ИНС в~цифровой модели генерации знания, описанию которой 
посвящена работа~\cite{27-zac}.
  
  Также предлагается различать два вида информации: сенсорно 
воспринимаемая и~цифровая. Кроме знания в~средов$\acute{\mbox{у}}$ю 
версию добавлены концепты и~ментальные образы сенсорно воспринимаемых 
данных. Последние служат промежуточной сущностью между сенсорно 
воспринимаемыми данными и~генерируемым знанием при описании процессов 
извлечения знания из текстовых данных лексикографической информационной 
системы. Описание объектов средов$\acute{\mbox{о}}$й версии иерархии 
Акоффа (см.\ рис.~1) и~отношений между ними дано в~работах~\cite{26-zac, 28-zac}.
  
  В средов$\acute{\mbox{о}}$й версии число объектов равно восьми. Если 
учитывать направления трансформаций, то между восемью объектами на 
рис.~1 она включает~16 их видов (трансформации на границе между сенсорно 
воспринимаемыми данными и~информацией, обозначенные символом~<<?>>, 
в~статье не рас\-смат\-ри\-ва\-ют\-ся). В~будущем число объектов 
в~средов$\acute{\mbox{о}}$й версии, которая выбрана как основание для 
сис\-те\-ма\-ти\-за\-ции трансформаций первого и~второго порядка, может быть 
увеличено. Для построения классификации трансформаций 
важ\-но не возможное увеличение числа объектов 
и~трансформаций между ними, а то, что их виды в~средов$\acute{\mbox{о}}$й 
версии распределены между трансформациями первого и~второго порядка. Из 
16~видов на рис.~1 шесть относятся к~трансформациям первого порядка, это\linebreak 
виды с~номерами~7, 8, 13--16 (далее~--- типология трансформаций первого 
порядка), а~десять~--- к~трансформациям второго порядка, это виды 
с~\mbox{номерами}~1--6 и~9--12 (далее~--- типология трансформаций второго 
порядка). Разместим обе типологии на третьем уровне классификации (см.\ ее 
схему на рис.~2). Перечислим виды трансформаций первой типологии, вводя 
в~скобках их краткие названия, используемые ниже на рис.~3:
  \begin{description}
  \item[\,] 7~--- членение знания на концепты с~помощью одной или нескольких 
знаковых систем (далее~--- членение знания);
  \item[\,] 8~--- формирование знания на основе концептов (формирование 
знания);
  \item[\,] 13~--- обучение ИНС;
  \end{description}
  
  \vspace*{-6pt}
  
  \pagebreak
  
  \end{multicols}
  
  \begin{figure*} %fig2
\vspace*{1pt}
      \begin{center}
     \mbox{%
\epsfxsize=127.513mm 
\epsfbox{zac-2.eps}
}
\end{center}
\vspace*{-9pt}
\Caption{Схема трех верхних уровней классификации трансформаций объектов (объединены 
по три слоя и~для второго, и~для третьего уровней этой классификации)}
\end{figure*}
  
  \begin{multicols}{2}
  
  \noindent
  \begin{description}
  \item[\,] 14~--- восстановление обучающей информации на основе 
содержания обученной ИНС (обращение ИНС);
  \item[\,] 15~--- использование обученной ИНС (использование ИНС);



  \item[\,] 16~--- восстановление исходных данных, соответствующих 
полученным результатам работы обучен\-ной ИНС (восстановление исходных данных 
по результатам ИНС).
  \end{description}
  
  
  Не все виды трансформаций 13--16 поддерживаются в~конкретных системах 
искусственного интеллекта, но с~теоретической точки зрения все их 
предлагается включить в~первую типологию для полноты спектра видов 
трансформаций.
  
  Перечислим виды трансформаций второй типологии:
  \begin{description}
  \item[\,] 1~--- декодирование цифровых данных в~компьютерных системах 
(декодирование данных);
  \item[\,]  2~--- кодирование сенсорно воспринимаемых данных (кодирование 
данных);
  \item[\,] 3~--- ментальное копирование сенсорно воспринимаемых данных 
(ментальное копирование);
  \item[\,] 4~--- восстановление сенсорно воспринимаемых данных по 
ментальным образам (восстановление по образам);
  \item[\,] 5~--- смысловая интерпретация без деления на концепты ментальных 
образов сенсорно воспринимаемых данных (смысловая интерпретация);
  \item[\,] 6~--- восстановление ментальных образов (восстановление образов);
  \item[\,] 9~--- представление концептов в~виде сенсорно воспринимаемой 
информации, например текс\-та\-ми, формулами, таблицами, рисунками и~т.\,д.\ 
(представление концептов);
  \item[\,] 10~--- понимание смысла сенсорно воспринимаемой информации 
(понимание смысла);
  \item[\,] 11~--- кодирование сенсорно воспринимаемой информации 
(кодирование информации);
\end{description}

\vspace*{-6pt}

\pagebreak

\end{multicols}

\begin{figure*} %fig3
\vspace*{1pt}
      \begin{center}
     \mbox{%
\epsfxsize=163mm 
\epsfbox{zac-3.eps}
}
\end{center}
\vspace*{-9pt}
\Caption{Схема частного случая классификации трансформаций объектов (трансформации 
пронумерованы согласно рис.~1)}
\end{figure*}

\begin{multicols}{2}

\noindent
\begin{description}

  \item[\,] 12~--- декодирование цифровой информации (декодирование 
информации).
  \end{description}
  
  Отметим, что в~существующих ИТ
  и~компьютерных системах наиболее часто используются виды 
трансформаций~13 и~15 типологии первого порядка и~1, 2, 11 и~12 типологии 
второго порядка. На рис.~2 в~первом слое третьего уровня классификации 
показаны типологии первого порядка без указания числа трансформаций в~них 
и~без детализации трансформируемых объектов.
  
  Во втором слое третьего уровня классификации условно (без названий) 
показаны типологии второго порядка. Также на рис.~2 в~третьем слое третьего 
уровня классификации условно (также без названий) показаны типологии 
третьего порядка, которые планируется рассмотреть в~отдельной статье. По 
определению они должны включать трансформации между тремя объектами 
разной природы, но средов$\acute{\mbox{а}}$я версия иерархии Акоффа 
включает трансформации только между двумя объектами разной природы. 
Поэтому потребуется другое основание для их систематизации (ранее были 
рассмотрены отдельные примеры трансформаций третьего 
порядка\footnote{Далеко не всегда трансформации третьего и~более высоких порядков можно 
рассматривать как последовательность трансформаций второго порядка. Примером этого могут 
служить трансформации в~процессе обучения пациента пользованию роботизированной рукой, 
охватывающие личностные концепты пациента, релевантные его намерениям, сигналы активности 
мозга как объекты нейросреды и~компьютерные коды~\cite{29-zac}.}~\cite{29-zac}).

\section{Классификация трансформаций: частный~случай}

  Выше было отмечено, что в~будущем число объектов 
в~средов$\acute{\mbox{о}}$й версии иерархии Акоффа может быть увеличено. 
Это означает, что увеличатся и~чис\-ло объектов, и~чис\-ло трансформаций между 
ними в~классификации трансформаций, так как эта средов$\acute{\mbox{а}}$я 
версия служит по определению основанием для систематизации 
трансформаций первого и~второго порядка. Поэтому на третьем уровне рис.~2 
указаны типологии без детализации объектов и~без указания числа 
трансформаций в~каждой из них. С~одной стороны, при таком подходе 
получаем достаточно общий вид этой классификации, так как она не зависит от 
числа объектов в~том или ином варианте средов$\acute{\mbox{о}}$й версии 
(и~это существенно упрощает рис.~2). С~другой стороны, на третьем уровне 
такой общей классификации подразумевается, но не эксплицируется природа 
трансформируемых объектов и~их возможные сочетания в~трансформациях. 

При проектировании лексикографической информационной системы важно 
эксплицировать природу трансформируемых объектов и~их возможные 
сочетания.
  %
  Поэтому в~парадигму информатики~\cite{30-zac} кроме общей 
классификации трансформаций предлагается включать и~ее частные случаи, 
эксплицирующие природу трансформируемых объектов. 

В~этом разделе 
рассмотрим один частный случай, когда используются только естественные 
знаковые сис\-те\-мы из типологии А.~Соломоника~\cite{25-zac} вместе 
с~данными, знанием и~его концептами. Чис\-ло естественных языков при этом не 
ограничено. И~этот частный случай классификации включает только три 
класса природных трансформаций (первого, второго и~третьего порядка, см.\ 
схему классификации на рис.~3).
  
  Первый и~второй уровни схемы общей классификации (см.\ рис.~2) можно 
объединить в~один уровень в~этом частном случае. Ниже этого уровня 
приведено содержание типологий первого и~второго порядка без содержания 
типологий третьего по\-рядка.




  Наполнение типологий первого и~второго порядка соответствует 
средов$\acute{\mbox{о}}$й версии иерархии Акоффа на рис.~1, содержащей 
6~видов трансформаций типологии первого порядка и~10~видов 
трансформаций типологии второго порядка (на рис.~3 стрелки указывают 
направления трансформаций согласно средов$\acute{\mbox{о}}$й версии на рис.~1).
  
  Таким образом, частный случай классификации содержит для этих двух 
типологий 16~теоретически возможных трансформаций, 6 из которых 
в~настоящее время в~существующих ИТ применяются наиболее часто: виды 
трансформаций~1, 2, 11 и~12 типологии второго порядка реализуются 
с~помощью тех или иных методов ко\-ди\-ро\-ва\-ния/де\-ко\-ди\-ро\-ва\-ния 
(например, с~использованием таблиц Unicode), а~виды трансформаций~13 и~15
 в~типологии первого порядка реализуются полностью с~по\-мощью процессов 
цифровой обработки компьютерами.
  
  Остальные виды трансформаций или применяются намного реже (это 
виды~3, 5, 7, 9 и~10), или находятся в~стадии поиска и~разработки (14 и~16) или 
в~настоящее время носят только теоретический характер, обеспечивая полноту 
первой и~второй типологий (4, 6 и~8). Знаком~<<?>> обозначены те виды 
трансформаций, которые по определению не существуют в~используемой 
парадигме информатики~\cite{30-zac}. Однако возможно, что в~других 
будущих подходах к~построению ее парадигмы эти виды трансформаций будут 
существовать.
  
\section{Заключение}

  На сегодняшний день процесс построения классификаций объектов 
предметной области информатики~\cite{22-zac} и~их  
трансформаций~\cite{23-zac} еще не завершен. Однако первые результаты их 
построения уже используются для создания концепции лексикографической 
информационной сис\-те\-мы, обеспечивающей интеграцию двуязычных 
словарей и~параллельных корпусов.
  
  \bigskip
  
  
  Автор признателен рецензентам за помощь в~улучшении статьи.
  
{\small\frenchspacing
 { %\baselineskip=10.6pt
 %\addcontentsline{toc}{section}{References}
 \begin{thebibliography}{99}
\bibitem{1-zac}
\Au{Aijmer K., Altenberg~B.} Advances in corpus-based contrastive linguistics. Studies in honour 
of Stig Johansson.~--- Amsterdam: John Benjamins, 2013. 295~p.  doi: 10.1075/scl.54.
\bibitem{2-zac}
\Au{Добровольский Д.\,О., Кретов~А.\, А., Шаров~С.\,А.} Корпус параллельных текстов~// 
Научная и~техническая информация. Сер.~2: Информационные процессы и~сис\-те\-мы, 2005. 
№\,6. С.~16--27.
\bibitem{3-zac}
\Au{Добровольский Д.\,О.} Корпус параллельных текстов и~сопоставительная 
лексикология~// Труды Института русского языка им.\ В.\,В.~Виноградова, 2015. №\,6. 
С.~413--449. EDN: VJQBHP.
\bibitem{4-zac}
\Au{Гончаров А.\,А., Зацман~И.\,М., Кружков~М.\,Г.} Эволюция классификаций 
в~надкорпусных базах данных~// Информатика и~её применения, 2020. Т.~14. Вып.~4. 
С.~108--116. doi: 10.14357/19922264200415.  
EDN: \mbox{GKWBZT}.
\bibitem{5-zac}
\Au{Гончаров А.\, А., Зацман И. \,М., Кружков~М.\, Г}. Представление новых 
лексикографических знаний в~динамических классификационных сис\-те\-мах~// 
Информатика и~её применения, 2021. Т.~15. Вып.~1. С.~86--93.  doi: 10.14357/19922264210112. EDN: OPEFXW.
\bibitem{6-zac}
\Au{Zatsman I.} Finding and filling lacunas in linguistic typologies~// 15th Forum (International) 
on Knowledge Asset Dynamics Proceedings.~--- Matera, Italy: Institute of Knowledge Asset 
Management, 2020. P.~780--793.
\bibitem{7-zac}
\Au{Zatsman I.} Three-dimensional encoding of emerging meanings in AI-systems~// 21st 
European Conference on Knowledge Management Proceedings.~--- Reading, U.K.: Academic 
Publishing International Ltd., 2020. P.~878--887.
\bibitem{8-zac}
\Au{Ackoff R.} From data to wisdom~// J.~Applied Systems Analysis, 1989. Vol.~16. No.\,1. P.~3--9.
\bibitem{9-zac}
\Au{Rosenbloom P.\,S.} On computing: The fourth great scientific domain.~--- Cambridge, MA, 
USA: MIT Press, 2013. 307~p.
\bibitem{10-zac}
\Au{Rowley J.} The wisdom hierarchy: Representations of the DIKW hierarchy~// J.~Inf. 
Sci., 2007. Vol.~33. Iss.~2. P.~163--180. doi: 10.1177/0165551506070706.
\bibitem{11-zac} 
\Au{Frick$\acute{\mbox{e}}$~M.\,H.} Data--Information--Knowledge--Wisdom (DIKW) pyramid, 
framework, continuum~// Encyclopedia of big data~/ Eds. L.~Schintler, C.~McNeely.~--- Cham: 
Springer, 2018. 4~p. doi: 10.1007/978-3-319-32001-4\_331-1.
\bibitem{12-zac}
\Au{Denning P., Rosenbloom~P.} Computing: The fourth great domain of science~// Commun. 
ACM, 2009. Vol.~52. Iss.~9. P.~27--29.
\bibitem{13-zac}
\Au{Denning P., Freeman~P.} Computing's paradigm~// Commun.  ACM, 2009. Vol.~52. 
Iss.~12. P.~28--30. doi: 10.1145/ 1610252.1610265.
\bibitem{17-zac} %14
\Au{Farradane J.} Knowledge, information, and information science~// J.~Inf. Sci., 
1980. Vol.~2. Iss.~2. P.~75--80. doi: 10.1177/01655515800020020.

\bibitem{15-zac}
\Au{Шрейдер Ю.\,А.} Информация и~знание~// Сис\-тем\-ная концепция информационных 
процессов.~--- М.: ВНИИСИ, 1988. С.~47--52.
\bibitem{16-zac}
\Au{Ingwersen P.} Information and information science~// Enclyclopaedie of library and 
information science~/ Eds. J.\,D.~McDonald, 
M.~Levine-Clark.~--- New York, NY, USA: Marcel Dekker Inc., 1992. Vol.~56. Sup.~19. 
P.~137--174.

\bibitem{14-zac} %17
Информатика как наука об информации: Информационный, документальный, 
технологический, экономический, социальный и~организационный аспекты~/ Под ред. 
Р.\,С.~Гиляревского.~--- М.: Фаир-Пресс, 2006. 592~с.

\bibitem{18-zac}
\Au{Hjorland B.} Library and information science: practice, theory, and philosophical basis~// 
Inform. Process. Manag., 2000. Vol.~36. Iss.~3. P.~501--531. doi:  
10.1016/S0306-\mbox{4573(99)00038-2}.
\bibitem{19-zac}
Deep shift~--- technology tipping points and societal impact.~--- Geneva: WE Forum, 2015. 44~p. 
{\sf http://www3.weforum.org/docs/WEF\_GAC15\_ Technological\_Tipping\_Points\_report\_2015.pdf}.
\bibitem{20-zac}
\Au{Berman F., Rutenbar~R., Hailpern~B., Christensen~H., Davidson~S., Estrin~D., 
Franklin~M., Martonosi~M., Raghavan~P., Stodden~V., Szalay~A.\,S.} Realizing the potential of 
data science~// Commun.  ACM, 2018. Vol.~61. Iss.~4. P.~67--72. doi: 10.1145/3188721.

\bibitem{21-zac}
\Au{Stodden V.} The data science life cycle: A~disciplined approach to advancing data science as 
a~science~// Commun.  ACM, 2020. Vol.~63. Iss.~7. P.~58--66. doi: 10.1145/ 3360646.


\bibitem{23-zac} %22
\Au{Зацман И.\,М.} Научная парадигма информатики: классификация трансформаций 
объектов предметной об\-ласти~// Системы и~средства информатики, 2023. Т.~33. №\,4. 
С.~126--138. doi: 10.14357/08696527230412. EDN: ZIKUWO.

\bibitem{22-zac} %23
\Au{Зацман И.\,М.} Научная парадигма информатики: классификация объектов предметной  
об\-ласти~// Информатика и~её применения, 2023. Т.~17. Вып.~4. С.~96--103. doi: 
10.14357/19922264230413. EDN: FIUQAT.

\bibitem{24-zac}
\Au{Зацман И.\,М.} О~научной парадигме информатики: верхний уровень классификации 
объектов ее предметной об\-ласти~// Информатика и~её применения, 2022. Т.~16. Вып.~4. 
С.~73--79. doi: 10.14357/ 19922264220411. EDN: XZNKVI.

\bibitem{25-zac}
\Au{Соломоник А.\,Б.} Философия знаковых систем и~язык.~--- М.: ЛКИ, 2011. 408~с.
\bibitem{26-zac}
\Au{Зацман И.\,М.} Трансформация иерархии Акоффа в~научной парадигме информатики~// 
Информатика и~её применения, 2023. Т.~17. Вып.~3. С.~107--113. doi: 
10.14357/19922264230315. EDN: UMVRRV.

\bibitem{27-zac}
\Au{Zatsman I.} Building digital spiral models of knowledge generation~// 19th Forum 
(International) on Knowledge Asset Dynamics Proceedings.~--- Matera, Italy: Arts for Business 
Institute, 2024. P.~2185--2196.
\bibitem{28-zac}
\Au{Zatsman I.} Digital spiral model of knowledge creation and encoding its dynamics~// 18th 
Forum (International) on Knowledge Asset Dynamics Proceedings.~--- Matera, Italy: Arts for 
Business Institute, 2023. P.~581--596.
\bibitem{29-zac}
\Au{Зацман И.\,М.} Интерфейсы третьего порядка в~информатике~// Информатика и~её 
применения, 2019. Т.~13. Вып.~3. С.~82--89. doi: 10.14357/19922264190312. EDN: 
EHRQLF.

\bibitem{30-zac}
\Au{Зацман И.\,М.} Научная парадигма информатики как третьей культуры~//  
На\-уч\-но-тех\-ни\-че\-ская информация. Сер.~1: Организация и~методика информационной 
работы, 2023. №\,11. С.~1--14.

\end{thebibliography}

 }
 }

\end{multicols}

\vspace*{-9pt}

\hfill{\small\textit{Поступила в~редакцию 14.04.24}}

\vspace*{4pt}

%\pagebreak

%\newpage

%\vspace*{-28pt}

\hrule

\vspace*{2pt}

\hrule



\def\tit{OBJECT TRANSFORMATIONS OF~THE~FIRST AND~SECOND ORDER
IN~A~LEXICOGRAPHIC INFORMATION SYSTEM\\[-5pt]}


\def\titkol{Object transformations of~the~first and~second order
in~a~lexicographic information system}


\def\aut{I.\,M.~Zatsman}

\def\autkol{I.\,M.~Zatsman}

\titel{\tit}{\aut}{\autkol}{\titkol}

\vspace*{-13pt}


\noindent
Federal Research Center ``Computer Science and Control'' of the Russian Academy of Sciences, 
44-2~Vavilov Str., Moscow 119133, Russian Federation


\def\leftfootline{\small{\textbf{\thepage}
\hfill INFORMATIKA I EE PRIMENENIYA~--- INFORMATICS AND
APPLICATIONS\ \ \ 2024\ \ \ volume~18\ \ \ issue\ 2}
}%
 \def\rightfootline{\small{INFORMATIKA I EE PRIMENENIYA~---
INFORMATICS AND APPLICATIONS\ \ \ 2024\ \ \ volume~18\ \ \ issue\ 2
\hfill \textbf{\thepage}}}

\vspace*{2pt}



\Abste{The theoretical foundations of the design of information technologies used for 
the integration of bilingual dictionaries and parallel corpora are considered. The 
description of the first outcomes of the creation of the third\linebreak\vspace*{-12pt}}

\Abstend{ level of object 
transformations classification in the subject domain of informatics, which is supposed 
to be used
in creating the lexicographic information system providing integration, is 
given. All the entities of informatics are divided into two global classes: objects and 
their transformations. For each such class, its own classification is constructed. 
Previously, the two upper levels of the object transformation classification in the subject 
domain have been described. The present paper discusses the third level of this classification. The 
basis for the construction of its highest level was the division of the subject domain of 
informatics into media (mental, sensory, digital, and a~number of other media), each 
of which by definition includes objects of the same nature. The Solomonick's 
typology of sign systems served as the basis for constructing the second level of the 
object transformation classification. The aim of the paper is to systematize object 
transformations of the first and second orders at the third level of this classification. 
The basis for systematization is the medium version of the Ackoff's hierarchy.}

\KWE{subject domain objects; object transformations; classification; data; 
information; knowledge; lexicographic information system}


\DOI{10.14357/19922264240211}{VZTGVV}

\vspace*{-12pt}

\Ack

\vspace*{-3pt}


\noindent
The reported study was funded by the Russian Science Foundation, project  
No.\,24-18-00155, {\sf 
https://rscf.ru/project/24-18-00155}. The research was carried out using the infrastructure of the Shared 
Research Facilities ``High Performance Computing and Big Data'' (CKP 
``Informatics'') of FRC CSC RAS (Moscow) .
   


  \begin{multicols}{2}

\renewcommand{\bibname}{\protect\rmfamily References}
%\renewcommand{\bibname}{\large\protect\rm References}

{\small\frenchspacing
 {%\baselineskip=10.8pt
 \addcontentsline{toc}{section}{References}
 \begin{thebibliography}{99} 
\bibitem{1-zac-1}
\Aue{Aijmer, K., and B.~Altenberg.} 2013. \textit{Advances in corpus-based 
contrastive linguistics. Studies in honour of Stig Johansson}. Amsterdam: John 
Benjamins. 295~p. doi: 10.1075/scl.54.
\bibitem{2-zac-1}
\Aue{Dobrovolskiy, D.\,O., A.\,A.~Kretov, and S.\,A.~Sharov.} 2005. Korpus 
parallel'nykh tekstov [Corpus of parallel texts]. \textit{Nauchnaya i~tekhnicheskaya 
informatsiya. Ser. 2. Informatsionnye protsessy i~sistemy} [Scientific and Technical 
Information. Ser.~2: Information Processes and Systems] 6:16--27.
\bibitem{3-zac-1}
\Aue{Dobrovolskiy, D.\,O.} 2015. Korpus parallel'nykh tekstov i~sopostavitel'naya 
leksikologiya [The corpus of parallel texts and contrastive lexicology]. \textit{Trudy 
Instituta russkogo yazyka im. V.\,V.~Vinogradova} [Proceedings of the 
V.\,V.~Vinogradov Russian Language Institute] 6:413--449. EDN: VJQBHP.
\bibitem{4-zac-1}
\Aue{Goncharov, A.\,A., I.\,M.~Zatsman, and M.\,G.~Kruzhkov.} 2020. Evolyutsiya 
klassifikatsiy v~nadkorpusnykh ba\-zakh dannykh [Evolution of classifications in 
supracorpora databases]. \textit{Informatika i~ee Primeneniya~--- Inform. \mbox{Appl.}}  
14(4):108--116. doi: 10.14357/19922264200415.  
EDN: GKWBZT.
\bibitem{5-zac-1}
\Aue{Goncharov, A.\,A., I.\,M.~Zatsman, and M.\,G.~Kruzhkov.} 2021. 
Predstavlenie novykh leksikograficheskikh znaniy v~dinamicheskikh 
klassifikatsionnykh sistemakh [Representation of new lexicographical knowledge in 
dynamic classification systems]. \textit{Informatika i~ee Primeneniya~--- Inform. 
Appl.} 15(1):86--93. doi: 10.14357/19922264210112. EDN: OPEFXW.
\bibitem{6-zac-1}
\Aue{Zatsman, I.} 2020. Finding and filling lacunas in linguistic typologies. 
\textit{15th Forum (International) on Knowledge Asset Dynamics Proceedings}. 
Matera, Italy: Institute of Knowledge Asset Management. 780--793.
\bibitem{7-zac-1}
\Aue{Zatsman, I.} 2020. Three-dimensional encoding of emerging meanings in  
AI-systems. \textit{21st European Conference on Knowledge Management 
Proceedings}. Reading, U.K.: Academic Publishing International Ltd. 878--887.
\bibitem{8-zac-1}
\Aue{Ackoff, R.} 1989. From data to wisdom. \textit{J.~Applied Systems Analysis} 
16(1):3--9.
\bibitem{9-zac-1}
\Aue{Rosenbloom, P.\,S.} 2013. \textit{On computing: The fourth great scientific 
domain}. Cambridge, MA: MIT Press. 307~p.
\bibitem{10-zac-1}
\Aue{Rowley, J.} 2007. The wisdom hierarchy: Representations of the DIKW 
hierarchy. \textit{J.~Inf. Sci.} 33(2):163--180. doi: 10.1177/0165551506070706.
\bibitem{11-zac-1}
\Aue{Frick$\acute{\mbox{e}}$, M.\,H.} 2018.  
Data-Information-Knowledge-Wisdom (DIKW) pyramid, framework, continuum. 
\textit{Encyclopedia of big data}. Eds. L.~Schintler and C.~McNeely. Cham: 
Springer. 4~p. doi: 10.1007/978-3-319-32001- 4\_331-1.
\bibitem{12-zac-1}
\Aue{Denning, P., and P.~Rosenbloom.} 2009. Computing: The fourth great domain 
of science. \textit{Commun. ACM} 52(9):27--29.
\bibitem{13-zac-1}
\Aue{Denning, P., and P.~Freeman.} 2009. Computing's paradigm. \textit{Commun. 
ACM} 52(12):28--30. doi: 10.1145/ 1610252.1610265.

\bibitem{17-zac-1} %14
\Aue{Farradane, J.} 1980. Knowledge, information, and information science. 
\textit{J.~Inf. Sci.} 2(2):75--80. doi: 10.1177/ 01655515800020020.

\bibitem{15-zac-1}
\Aue{Shreyder, Yu.\,A.} 1988. Informatsiya i~znanie [Information and knowledge]. 
\textit{Sistemnaya kontseptsiya in\-for\-ma\-tsi\-on\-nykh protsessov} [System concept of 
information processes]. Moscow: VNIISI. 47--52.
\bibitem{16-zac-1}
\Aue{Ingwersen, P.} 1995. Information and information science. 
\textit{Encyclopedia of library and information science}. Eds. J.\,D.~McDonald and 
M.~Levine-Clark. New York, NY: Marcel Dekker Inc. 56(19):137--174.

\bibitem{14-zac-1} %17
Gilyarevskiy, R.\,S., ed. 2006. \textit{Informatika kak nauka ob informatsii: 
informatsionnyy, dokumental'nyy, tekh\-no\-lo\-gi\-che\-skiy, ekonomicheskiy, sotsial'nyy 
i~organizatsionnyy aspekty} [Informatics as information science: Informational, 
documentary, technological, economic, social, and organizational dimensions]. 
Moscow: FAIR-PRESS. 592~p.

\bibitem{18-zac-1}
\Aue{Hjorland, B.} 2000. Library and information science: Practice, theory, and 
philosophical basis. \textit{Inform. Process. Manag.} 36(3):501--531. doi:  
10.1016/S0306-\mbox{4573(99)00038-2}.
\bibitem{19-zac-1}
Deep shift~--- technology tipping points and societal impact. 2015. \textit{World Economic 
Forum}. Geneva. 44~p. Available at: {\sf 
http://www3.weforum.org/docs/WEF\_ GAC15\_Technological\_Tipping\_Points\_report\_2015.pdf} (accessed May~20, 
2024).
\bibitem{20-zac-1}
\Aue{Berman, F., R.~Rutenbar, B.~Hailpern, H.~Christensen, S.~Davidson, 
D.~Estrin, M.~Franklin, M.~Martonosi, P.~Raghavan, V.~Stodden, and 
A.\,S.~Szalay.} 2018. Realizing the potential of data science. \textit{Commun. ACM} 
61(4):67--72. doi: 10.1145/3188721.
\bibitem{21-zac-1}
\Aue{Stodden, V.} 2020. The data science life cycle: A~disciplined approach to 
advancing data science as a~science. \textit{Commun. ACM} 
 63(7):58--66. doi: 10.1145/3360646.

\bibitem{23-zac-1} %22
\Aue{Zatsman, I.\,M.} 2023. Nauchnaya paradigma informatiki: klassifikatsiya 
transformatsiy ob''ektov predmetnoy oblasti [Scientific paradigm of informatics: 
Transformation classification of domain objects]. \textit{Sistemy i~Sredstva 
Informatiki~--- Systems and Means of Informatics} 33(4):126--138. doi: 
10.14357/08696527230412. EDN: ZIKUWO.

\bibitem{22-zac-1} %23
\Aue{Zatsman, I.\,M.} 2023. Nauchnaya paradigma informatiki: klassifikatsiya 
ob''ektov predmetnoy oblasti [Scientific paradigm of informatics: Classification of 
domain objects]. \textit{Informatika i~ee Primeneniya~--- Inform. Appl.} 
 17(4):96--103. doi: 10.14357/19922264230413. EDN: FIUQAT.
 
\bibitem{24-zac-1}
\Aue{   Zatsman, I.\,M.} 2022. O nauchnoy paradigme informatiki: verkhniy uroven' 
klassifikatsii ob''ektov ee predmetnoy oblasti [On the scientific paradigm of 
informatics: The classification high level of its objects]. \textit{Informatika i~ee 
Primeneniya~--- Inform. Appl.} 16(4):73--79. doi: 10.14357/19922264220411. EDN: 
XZNKVI.
\bibitem{25-zac-1}
\Aue{Solomonick, A.\,B.} 2011. \textit{Filosofiya znakovykh system i~yazyk} 
[Philosophy of sign systems and language]. Moscow: LKI. 408~p.
\bibitem{26-zac-1}
\Aue{Zatsman, I.\,M.} 2023. Transformatsiya ierarkhii Akoffa v~nauchnoy 
paradigme informatiki [Transformation of the Ackoff's hierarchy in the scientific 
paradigm of informatics]. \textit{Informatika i~ee Primeneniya~--- Inform. \mbox{Appl.}} 
17(3):107--113. doi: 10.14357/19922264230315. EDN: UMVRRV.
\bibitem{27-zac-1}
\Aue{Zatsman, I.} 2024. Building digital spiral models of knowledge 
generation. \textit{19th Forum (International) on Knowledge Asset Dynamics 
Proceedings}. Matera, Italy: Arts for Business Institute. 2185--2196.
\bibitem{28-zac-1}
\Aue{Zatsman, I.} 2023. Digital spiral model of knowledge creation and encoding its 
dynamics. \textit{18th Forum (International) on Knowledge Asset Dynamics 
Proceedings}. Matera, Italy: Arts for Business Institute. 581--596.
\bibitem{29-zac-1}
\Aue{Zatsman, I.\,M.} 2019. Interfeysy tret'ego poryadka v~informatike 
 [Third-order interfaces in informatics]. \textit{Informatika i~ee Primeneniya~--- 
Inform. Appl.} 13(3):82--89. doi: 10.14357/19922264190312. EDN: EHRQLF.
\bibitem{30-zac-1}
\Aue{Zatsman, I.} 2023. Scientific paradigm of informatics as a~third culture. 
\textit{Scientific Technical Information Processing} 50(4):246--258. doi: 
10.3103/S0147688223040111. EDN: CKHMYS.

\end{thebibliography}

 }
 }

\end{multicols}

\vspace*{-6pt}

\hfill{\small\textit{Received April 14, 2024}} 


\vspace*{-12pt}


\Contrl

\vspace*{-3pt}

\noindent
\textbf{Zatsman Igor M.} (b.\ 1952)~--- Doctor of Science in technology, head of 
department, Federal Research Center ``Computer Science and Control'' of the 
Russian Academy of Sciences, 44-2~Vavilov Str., Moscow 119333, Russian 
Federation; \mbox{izatsman@yandex.ru}





\label{end\stat}

\renewcommand{\bibname}{\protect\rm Литература}    %16
\def\stat{seif-mul}

\def\tit{ЗАКОНЫ ИНФОРМАТИКИ И СИНЕРГЕТИКИ В~ПОЗНАНИИ~СЛОЖНЫХ~СИСТЕМ}

\def\titkol{Законы информатики и~синергетики в~познании сложных систем}

\def\aut{Р.\,Б.~Сейфуль-Мулюков$^1$}

\def\autkol{Р.\,Б.~Сейфуль-Мулюков}

\titel{\tit}{\aut}{\autkol}{\titkol}


\index{Сейфуль-Мулюков Р.\,Б.}
\index{Seyful-Mulyukov R.\,B.}




%{\renewcommand{\thefootnote}{\fnsymbol{footnote}} \footnotetext[1]
%{Работа выполнена в~Институте проблем информатики ФИЦ ИУ РАН при поддержке РФФИ 
%(проект 18-07-00192).}}


\renewcommand{\thefootnote}{\arabic{footnote}}
\footnotetext[1]{Институт проблем информатики Федерального исследовательского центра <<Информатика 
и~управление>> Российской академии наук, \mbox{rust@ipiran.ru}}

%\vspace*{-12pt}

    
    
    \Abst{В~статье основные законы информатики и~синергетики применены для 
объяснения возникновения и~развития такой сложной природной системы, какой 
представляется нефть. Законы информатики предполагают неопределенность 
проявления квантового поведения электронов при создании матрицы 
углеводородной молекулы. Динамическая и~статическая неопределенность 
проявляется при поиске месторождений нефти. Рассматриваются законы 
синергетики, показывающие способность молекул к~самоорганизации. Новые 
молекулы в~углеводородном флюиде образуются вблизи точек бифуркации, 
ассоциированных с~изменением термодинамики, структуры и~состава пород 
геологической среды. В~приложении к~образованию нефти рассматривается 
понятие аттрактора. Он представляется как бассейн притяжения всех 
образовавшихся молекул флюида, в~котором формируется состав нефти 
конкретного месторождения.} 
    
    \KW{синергетика и~образование нефти; информатика и~образование 
и~поиски нефти; бифуркация и~образование молекул углеводородов; аттрактор 
и~формирование состава нефти}

\DOI{10.14357/19922264190417} 
  
%\vspace*{1pt}


\vskip 10pt plus 9pt minus 6pt

\thispagestyle{headings}

\begin{multicols}{2}

\label{st\stat}

\section{Информатика и~сложные системы (на примере нефти)}


    Информатика понимается в~России как на\-уч\-но-при\-клад\-ная дисциплина. 
Прикладная часть примерно соответствует Information Science в~западной 
терминологии. Она имеет дело с~информацией\linebreak в~ее историческом значении как 
данных, сведений, фактов, которые используются, хранятся, обрабатываются, 
распространяются, передаются\linebreak с~по\-мощью информационных технологий, средств 
связи и~компьютерных систем. Научная часть, наряду с~использованием ее 
исторического значения, представляет информацию как категорию, 
сопоставимую с~материей, энергией, временем, движением и~пространством 
и~выражающую их соотношение и~меру, количество и~качество в~данном 
природном или социальном сложном явлении или системе. Связь этих категорий 
реальности с~информацией впервые всесторонне обосновал и~применил для 
объяснения образования и~развития сложных природных систем И.~Гуревич~[1]. 

Впоследствии законы информатики были применены к~нефти как сложной 
сис\-те\-ме неорганической природы~\cite[c.~104--124]{2-s}. Ее свойства, 
вытекающие из законов информатики, включая\linebreak \textit{существование, развитие, 
познаваемость, простоту, неопределенность и~необходимое разнообразие}, дают 
возможность понять природу нефти. Ниже рассмотрены только два свойства, 
наиболее важных в~данном случае.
    
    \textit{Неопределенность}~--- понятие математики, философии, кибернетики, 
физики и~базовое понятие квантовой механики, которое показывает 
взаимодействие между сопряженными переменными состояниями элементарных 
частиц. Неопределенность нефти выражает квантовое поведение электронов 
атомов углерода и~водорода, проявляющееся при их соединении в~молекулу 
углеводорода (УВ). При гибридизации перекрываются волновые функции 
электронов, создается скрепляющее атомы облако электронов с~общим зарядом. 
Формируется элект\-рон\-но-вол\-но\-вая (квантовая) матрица, своеобразный 
генетический код молекулы~\cite{3-s}. Она сохраняет структурную идентичность 
данного типа молекулы УВ при всех изменениях геологических 
условий, в~которых осуществляется ее миграция к~поверхности. Матрица~--- 
физическое явление субатомного уровня, поэтому в~изучении процессов создания 
сложной системы макроуровня это неопределенность. 
    
    \textit{Неопределенность} проявляется и~при поиске промышленного 
скопления нефти. При этом необходимо установить наличие нефти 
в~обязательной для аккумуляции совокупности трех элементов геологической 
среды (ловушка, по\-ро\-да-кол\-лек\-тор и~экран плохо проницаемых пород). Это 
своеобразная квартира для месторождения нефти. Неопределенность в~том, что не 
во всякой квартире, точно установленной в~недрах геофизическими методами, 
аккумулируется нефть. Эту проблему можно решить разными техническими 
и~математическими методами, разделив неопределенность на динамическую 
и~статическую~\cite{4-s}. 
    
    \textit{Необходимое разнообразие} как закон ввел  
Эшби~\cite[с.~202--216]{5-s}. Он справедлив для сложных природных систем, 
включая нефть. Закон уста\-нав\-ли\-ва\-ет сопряженность сложности и~разнообразия 
управляющей и~управляемой систем. Нефть не образуется и~не существует сама 
по себе, автономно и~независимо от состояния и~состава окружающей 
геологической среды, в~которой она развивается. Взаимодействие и~развитие ее 
частей на всех этапах развития определяется совокупностью факторов среды, 
которая является управ\-ля\-ющей для сис\-те\-мы нефти. Современное разнообразие 
условий среды~--- это основа слож\-но\-го со\-ста\-ва, химического и~структурного 
разнообразия углеводородных молекул. Согласно закону Эшби соответствие 
необходимого разнообразия и~сложности управ\-ля\-емой сис\-те\-мы разнообразию 
управляющей есть непременное условие самоорганизации и,~следовательно, 
создания слож\-ной сис\-темы.
    
    Для понимания роли законов информатики и~синергетики в~познании 
сложной системы нефти приведем краткое описание состава и~стро\-ения ее частей, 
т.\,е.\ молекул УВ. Нефть на 95\% со\-сто\-ит из атомов углерода 
и~водорода. Они образуют молекулы трех структурных типов: цепные 
и~разветвленные алканы, алкановые циклы (циклоалканы или нафтены) и~арены 
(ароматические кольца), а также их гомологи и~комбинации трех основных типов 
молекул~\cite[с.~34--175]{6-s}. Следовательно, эти атомы~--- исходное вещество 
для образования нефти. Проб\-ле\-ма в~том, как и~где из них образовалась первая 
молекула УВ. Автор считает\footnote{Значения температуры и~давления вычислены по 
геотермическому градиенту ($+30$~$^\circ$C на 100~м) и~геостатическому градиенту (2,31~МПа 
на~100~м при средней плотности пород~2,3~г/см$^3$ и~3,1~МПа на 100~м при плотности~3,1~г/см$^3$), 
отнесенным к~платформенным областям. (\textit{Прим.\ автора.})}, что это произошло при 
ковалентном соединении этих атомов на глубине с~температурой не более 
1200~$^\circ$С и~давлением 1150~Мпа (см.\  рисунок). 
    
    \begin{figure*}
    \vspace*{1pt}
    \begin{center}  
  \mbox{%
 \epsfxsize=162.209mm 
 \epsfbox{sei-1.eps}
 }
\end{center}

\vspace*{6pt}

\noindent
{\small Точки бифуркации и~положение бассейна притяжения в~схеме последовательности 
генерации молекул углеводородов нефти: \textit{1}~--- метан; \textit{2}~--- цепные алканы 
(парафины); \textit{3}~--- ветвистые алканы (изоалканы); \textit{4}~--- циклоалканы (нафтены); 
\textit{5}~--- ароматические (арены); \textit{6}~--- гетероуглеводородные соединения}
\end{figure*}

    Первая молекула УВ~--- метан СН$_4$~--- образовалась при гибридизации четырех 
электронных орбиталей ($2s$, $2p_x$, $2p_y$ и~$2p_z$) атома углерода с~$1s$-ор\-би\-та\-ля\-ми 
четырех атомов водорода (см.\ рисунок). При этом образовалась матрица (см. выше). 
Образование метана~--- современный процесс геосферного масштаба, 
вызывающий его глобальную увеличивающуюся эмиссию в~атмосферу~\cite{7-s}. 
Метан известен в~залежах недр суши и~скоплениях на дне океана в~виде 
газогидратов в~количестве триллионов кубометров. Он родоначальник генерации 
миллиардов тонн нефти, установленных в~месторождениях на всех континентах 
и~их шельфах. Образование метана знаменует переход от состояния беспорядка 
отдельных атомов углерода и~водорода в~мантии Земли к~их порядку в~молекуле 
в~верхней астеносфере. По Николису и~Пригожину, это первый шаг к~созданию 
сложности и~сложному поведению час\-тей (молекул) и~способности развиваться 
в~иное, более сложное состояние~\cite[с.~25--35]{8-s}. 
    
    В данном случае переход к~созданию сложности означает изменение 
состояния самой молекулы метана. На глубинах с~температурой более 
$+1200$~$^\circ$C и~давлением 1150~МПа незначительный сдвиг равновесия 
в~сторону температуры или давления служит импульсом для метана. Импульс 
давления, не изменяя метан, выдавливает его в~область более низких давлений. 
Он мигрирует в~первозданном составе к~поверхности между кристаллами или по 
трещинам в~породах размером более 0,56~нм (максимальный размер молекул  
\textit{н}-ал\-ка\-нов). Встретив на пути миграции в~земной коре ловушку,  
по\-ро\-ду-кол\-лек\-тор и~экран (плохо проницаемые породы), газ аккумулируется в~залежи преимущественно метанового состава. 
    
    Термический импульс вызывает гомолитический разрыв $2p_z$-свя\-зи атома 
углерода и~связи атома водорода, последний удаляется. Молекула метана 
преобразуется в~метильный радикал --CН$_3$, активную частицу с~одной 
свободной валентной \mbox{$2p_z$-связью}. В~дальнейших преобразованиях частица играет 
роль матрицы. К~ней присоединяется другая подобная частица. Так образуется 
молекула этана С$_2$Н$_6$ (Н$_3$С--СН$_3$). Из этана при разрыве связи \mbox{С--Н} 
образуется новая свободная частица~--- этильный радикал --C$_2$Н$_5$, 
одновалентная свободная частица, к~которой присоединяется радикал. Так 
образуется пропан С$_3$Н$_8$ (Н$_3$С--СН$_2$--СН$_3$). Начинается жизнь 
метана как совокупности метильного и~других радикалов, составляющих  
мо\-ле\-ку\-лы УВ. Раздвоение потока метана на его миграцию в~чистом виде и~в 
форме сцепленных радикалов в~цепи алканов происходит вблизи второго уровня 
усложнения системы УВ-флюида (см.\ рисунок).
    
    На глубинах \textbf{60--50}~км миграция УВ-флюида происходит при 
снижении температуры от~1200 до~800~$^\circ$C, а давления 
геологической среды~--- от~1150 до~950~МПа. Это создает условия для роста цепи 
алканов и~одновременного присоединения к~ней, в~строго определенном порядке, 
радикалов, но уже сбоку цепи. Так появляются разветвленные алканы. Начало их 
формирования знаменует третью точку резкого изменения состава углеводородного флю\-ида 
и~сложной системы в~целом (см.\ рисунок).
    
    На глубинах \textbf{50--40}~км температура литосферы Земли снижается 
с~800 до~600~$^\circ$C, а~давление~--- с~950 до~720~МПа. 
Внутренний механизм самоорганизации системы создает молекулу с~более 
плот\-ной упаковкой атомов, более прочными межмолекулярными связами, 
энергоемкую и~мобильную. Для этого замыкается цепь с~четырьмя, пятью или 
шестью атомами углерода в~цикл. Формируются молекулы циклоалканов, 
называемые нафтенами (см.\ рисунок). При появлении нафтенов ранее 
сформированные молекулы цепных и~разветвленных алканов сохраняются  
в~углеводородном флюиде.
    
    На глубинах \textbf{40--30}~км температура среды снижается 
с~600 до~480~$^\circ$C, а~давление~--- с~720 до~690~МПа. Это позволяет 
системе создавать особый, ароматический тип связи и~одноименную молекулу~--- 
бензол С$_6$Н$_6$. Каждый из шести атомов углерода бензольного кольца 
сохраняет три равные $\sigma$-свя\-зи~--- две с~соседними атомами углерода 
и~одну  
$\sigma$-связь с~атомом водорода. Четвертый $p$-элект\-рон каждого из шести 
атомов углерода не связан ни с~одним атомом. Эти электроны образуют  
$\pi$-свя\-зи в~виде двух электронных облаков, над и~под кольцом, которые 
выполняют роль валентных орбиталей. К~ним могут присоединяться 
радикалы и~другие типы молекул УВ (см.\ рисунок). 
    
    В интервале \textbf{30--20}~км температура среды снижается с~480 
    до~370~$^\circ$C, а~давление~--- с~690 до~460~МПа. На этих глубинах более разнородная 
структура и~текстура пород, меньшие температуры и~давление и~большая 
подвижность блоков земной коры создают условия для более свободного 
взаимодействия молекул и~их совместной миграции. Открывается возможность 
усложнять ранее созданные молекулы путем соединения цепных алканов 
с~нафтеновыми\linebreak циклами и~бензольными кольцами, соединять нафтены 
и~ароматические УВ в~гетероциклы и~гетероароматические молекулы. Примером 
преобразования молекул может служить электрофильное \mbox{замещение}. 
В~молекулах циклического или ароматического типа атом углерода или водорода 
замещается по $\pi$-ти\-пу связи гетероатомами (электрофилами). Ими могут 
быть атомы серы, азота или кислорода, поскольку имеют свободный валентный 
электрон на внешней орбитали (см.\ рисунок).

\vspace*{-6pt}
    
\section{Синергетика и~сложные системы (на примере нефти)}

    Синергетика, представленная Хакеном~\cite{9-s},~--- это наука о сложности 
как феномене, самоорганизации частей в~сложные системы. Признание нефти 
сложной природной системой открывает возможность использовать законы 
и~положения синергетики для того, чтобы наряду с~информатикой, геологией 
и~геохимией представить дополнительные доказательства неорганической, 
глубинной природы нефти.
    
    Одно из основных положений синергетики~--- \textit{самоорганизация} 
частей в~сложную систему. Как феномен ее показал Эшби~\cite{10-s}. Хакен 
определил самоорганизацию как процесс упорядочения в~пространстве и~времени 
открытой системы за счет взаимодействия составляющих ее 
частей~\cite[с.~147]{11-s}.  
Самоорганизация создает порядок из беспорядка~\cite[с.~235--240]{12-s}. 
В~общем виде самоорганизация в~приложении к~углеводородным мо\-ле\-ку\-лам нефти 
знаменует начало их сложного поведения. Это означает корреляцию между 
молекулами в~между-зерновом пространстве и~трещинах в~породах геологической 
среды (до десятых долей миллиметра). Это расстояния, на многие порядки 
превышающие короткодействующие (доли нанометра) межмолекулярные силы. 
Сложное поведение и~корреляция между частями~--- это и~есть самоорганизация 
ради достижения какой-то цели. Достижение как процесс~--- это миграция 
флюида, состоящего из УВ (но не нефти!) в~геологической среде. 
    
    Н.~Винер считал, что существование и~развитие сложной системы имеет 
цель. Согласно этой идее углеводородный флю\-ид имеет цель, т.\,е.\ конечное состояние, 
достигнув которого совокупность молекул УВ превращается в~сложную систему. 
Появление новых молекул УВ прекращается. 
    
    С другой стороны, имеет место нелинейность саморазвития; преобразование и~усложнение молекул не совершается без смысла, просто так. Изменением 
состояния системы углеводородного флю\-ида и~отбором молекул для дальнейшей миграции 
к~цели руководит \textit{системный разум}. В~системе жидкого углеводородного флю\-ида 
системный разум~--- это\linebreak
 алгоритм считывания природных кодов, своеобразный 
выбор типов молекул, которые в~точках изменения состояния сис\-те\-мы 
привлекаются (допускаются) для дальнейших преобразований или \mbox{продолжают} 
миграцию в~преж\-нем виде. Природными кодами служат квантовые мат\-ри\-цы 
и~информация молекул (см.\ рисунок). Их совокупность формирует каждый тип 
молекул УВ. Природный алгоритм оценивает смысл (необходимость) и~условия 
наращивания цепи атомов углерода,  необходимость 
и~последовательность присоединения к~цепи радикалов, а~также смысл 
преобразования цепи в~циклы, объединения циклов и~бензольных колец 
в~полициклические структуры.
    
    Когерентное (согласованное, коррелированное) поведение молекул в~данном 
случае означает обмен информацией (сигналами) молекул друг с~другом. 
Молекула построена по квантовой матрице~\cite{3-s} и~имеет объем информации 
в~битах~\cite[с.~129--132]{2-s}. Это набор природных кодов, по которым 
молекулы распознаются в~точках бифуркации (см.\ ниже). Роль информации 
в~создании сложности раскрыта в~\cite[с.~212--217]{8-s}.
    
    Закон синергетики определяет \textit{нелинейность} развития сложных 
систем. На пути миграции возникают переломные точки, в~которых постепенные 
изменения в~системе и~в~окружающей среде переходят в~скачкообразное 
изменение качества системы. В~синергетике это \textit{точки бифуркации}. 
Феномен бифуркации имеет прямое отношение к~процессу формирования со\-ста\-ва  
углеводородного флю\-ида. На схеме (см.\ рисунок) показано, что на определенных глубинных 
уровнях (точках бифуркации) система переходит в~новое, количественно 
и~качественно иное состояние. Наряду с~ранее созданными в~сис\-те\-ме  
углеводородного флю\-ида образуются новые типы молекул, более приспособленные для 
преобразований и~дальнейшей миграции.
    
    Нелинейность развития системы флюида проявляется 
в~непропорциональности результата и~причины, вызвавшей бифуркацию. 
Небольшой, практически не регистрируемый тепловой или/и динамический 
импульс среды, вызванный геотектоническими процессами в~литосфере, начинает 
процесс генерации нового качества системы, несопоставимый по масштабу 
с~импульсом.
    
    К понятиям синергетики относится \textit{аттрактор} (attractor), 
применяемый в~математике, метеорологии, экономике и~многих других 
дисциплинах. Соответственно, существует множество его определений 
и~толкований. В~зависимости от области приложения аттрактором может быть 
точка, линия, фигура сложной конфигурации или некая область фазового 
пространства. Их развернутая характеристика дана в~работе~\cite{13-s}. Для 
представления значения аттрактора для сис\-те\-мы углеводородного
флю\-ида вернемся 
к~понятию <<цель>>, которой является его конечное состояние, к~которому он 
стремится,~--- это и~есть аттрактор (attraction~--- притяжение). 
    
    В данном случае это сравнительно тонкая сфера геологической среды на 
глубинах не более~20~км, своеобразный бассейн притяжения (attractor basin), 
к~которому направлены траектории миграции всех типов и~разновидностей 
молекул углеводородного флю\-ида. Это своеобразный природный склад, в~котором 
аккумулируются все час\-ти, и~одновременно цех сборки уже конкретной сис\-те\-мы, 
т.\,е.\ месторождения с~определенным набором молекул УВ. Схематично три их 
основных типа показаны на схеме (см.\ рисунок). Углеводородный
флю\-ид в~бассейне 
притяжения, как и~на любой стадии миграции в~литосфере,~--- еще не нефть 
в~залежи в~обычном, тривиальном смысле, а полная совокупность всех ранее 
образованных мо\-ле\-кул УВ, некий молекулярный хаос. 
    
    Необходим отбор, чтобы их совокупность соответствовала геологическому 
строению, составу пород, геотектонике, гидрогеологии, тепловому режиму 
и~многим другим факторам геологической среды данного, конкретного региона. 
Это будет означать, что сложность и~разнообразие нефти в~данном 
месторождении соответствуют слож\-ности и~разнообразию среды недр, где 
расположено месторождение. Соотношение показателей слож\-ности 
и~разнообразия двух сис\-те\-мы будут соответствовать закону необходимого 
разнообразия Эшби (см.\ выше). В~этом кроется причина уни\-каль\-ности  
углеводородного со\-ста\-ва нефти в~месторождениях~--- преимущественно парафиновый, 
нафтеновый, ароматический или смешанный.

    Бассейн притяжения, в~котором аккумулируются созданные в~литосфере все 
типы молекул УВ, с~точки зрения второго закона классической  
термодинамики~--- это молекулярный хаос с~повышенной энтропией. Однако 
в~синергетике энтропия сложных систем может быть мерой их разнообразия, или 
многоуровневых состояний со сложным порядком~\cite[с.~79--80]{12-s}. 
В~бассейне притяжения из всего молекулярного хаоса происходит выбор их 
совокупности, которая соответствует всем геологическим условиям недр данного 
конкретного региона. Месторождения нефти~--- это более высокий уровень 
самоорганизации, поскольку каждое из них имеет уникальный набор частей 
сложной системы. Поэтому бассейн притяжения представляет высшую точку 
бифуркации, за которой формируются новые сложные сис\-те\-мы в~виде 
месторождений, каждое из которых~--- уникальная сложность.

\vspace*{-6pt}
    
\section{Заключение}

    Основные теоретические положения и~законы информатики и~синергетики 
развиты создателями этих наук во второй половине ХХ~в. Они открыли 
законы и~сформулировали положения, по которым стало понятно, что биты 
информации, атомы кристаллов, молекулы веществ, клетки живых организмов, 
физические предметы макроуровня, социальные и~экономические категории 
и~явления трансформируются/самоорганизуются в~сложность и~сложные системы. 
В~статье действие этих законов показано на примере конкретной природной 
сложной системы, состоящей из частей~--- молекул УВ. Ее природа, изучаемая 
только по законам геологии и~геохимии, ошибочно трактуется в~нефтяной 
геологии мира как продукт термолиза биосферного мусора в~верхних горизонтах 
земной коры. Законы и~положения информатики и~синергетики в~совокупности 
с~законами геологии и~геохимии позволяют раскрыть иную, минеральную 
природу нефти, подтверждая идею, высказанную еще Менделеевым~\cite{14-s} 
и~показанную Кудрявцевым~\cite{15-s}. 
    
    Казалось бы, геологические, экономические и~иные аспекты науки о нефти 
далеки от информатики и~синергетики. Однако, как показано выше, только 
конвергенция далеких друг от друга наук дает возможность понять истинную 
природу и~законы развития нефти как сложной системы. 
    
    {\small\frenchspacing
 {%\baselineskip=10.8pt
 \addcontentsline{toc}{section}{References}
 \begin{thebibliography}{99}
    \bibitem{1-s}
    \Au{Гуревич И.\,М.} Законы информатики~--- основа строения и~познания 
сложных систем.~--- М.: ТОРУС ПРЕСС, 2007. 399~с.
    \bibitem{2-s}
    \Au{Сейфуль-Мулюков Р.\,Б.} Нефть и~газ, глубинная природа и~ее 
прикладное значение.~--- М.: ТОРУС ПРЕСС, 2012. 215~с.
    \bibitem{3-s}
    \Au{Сейфуль-Мулюков Р.\,Б.} Квантовая матрица и~информация 
углеводородной молекулы~// Докл.\ Акад. наук. Сер. Геология, 2016. Т.~467. №\,3.  
С.~311--313.
    \bibitem{4-s}
    \Au{Сейфуль-Мулюков Р.\,Б.} Образование нефти и~газа, тео\-рия 
и~прикладные аспекты~// Геология нефти и~газа, 2017. №\,4. С.~89--95.
    \bibitem{5-s}
    \Au{Ashby W.\,R.} An introduction to cybernetics.~--- London: Chapman \& Hall, 
Ltd., 1957. 289~р.
    \bibitem{6-s}
    \Au{Петров А.\,А.} Углеводороды нефти.~--- М.: Наука, 1984. 264~с.
    \bibitem{7-s}
    Earth's methane emissions are rising~//
     New scientist, May~24, 2019. {\sf 
https://www.newscientist.com/article/ 2204466-earths-methane-emissions-are-rising-and-we-dont-know-why}.
    \bibitem{8-s}
    \Au{Николис Г., Пригожин~И.} Познание сложного. Введение~/
    Пер. с~англ.~--- М.: Мир,  1990. 344~с.
    (\Au{Nicolis~G., Prigogine~I.}  {Exploring 
complexity: An introduction}.~--- 1st ed.~--- St.\ Martin's Press, 1989. 328~p.)
    \bibitem{9-s}
    \Au{Haken H.} Information and self-organization: A~macroscopic approach to 
complex system.~--- New York, NY, USA: Springer-Verlag, 2000. 276~р.
    \bibitem{10-s}
    \Au{Ashby W.\,R.} Principles of the self-organizing dynamic system~// J.~Gen. 
Psychol., 1947. Vol.~37. P.~125--128.
    \bibitem{11-s}
    \Au{Haken H.} The Brain as a~synergetic and physical system~// Symposium 
(International) ``Selforganization in Complex Systems: The Past, Present, and Future of 
Synergetics'' Proceedings.~--- Delmenhorst, 2012. P.~147--165.
    \bibitem{12-s}
    \Au{Пригожин И., Стенгерс~И.} Порядок из хаоса: Новый диалог человека 
с~природой~/ Пер. с~англ.~--- М.: Прогресс, 1986. 432~с.
(\Au{Prigogine~I., Stengers~I.} {Order out of chaos: Man's
new dialogue with nature}.~--- Bantam Books, 1984. 349~p.)
    \bibitem{13-s}
    \Au{Бекман И.\,Н.} Нелинейная динамика сложных систем: теория 
и~практика.~--- М.: МГУ, 2018. 89~с.
{\sf 
http://\linebreak profbeckman.narod.ru/NelDin/NelDinText2.pdf}.
    \bibitem{14-s}
    \Au{Менделеев Д.\,И.} Неорганическое происхождение нефти: Доклад на 
заседании Русского химического общества~// Rev. Sci., 1876. Ser.~VIII. 
P.~409--416.
    \bibitem{15-s}
    \Au{Кудрявцев Н.\,А.} Генезис нефти и~газа.~--- Л.: Недра, 1973. 216~с.
     \end{thebibliography}

 }
 }

\end{multicols}

\vspace*{-6pt}

\hfill{\small\textit{Поступила в~редакцию 15.10.19}}

\vspace*{8pt}

%\pagebreak

%\newpage

%\vspace*{-28pt}

\hrule

\vspace*{2pt}

\hrule

%\vspace*{-2pt}

\def\tit{UNDERSTANDING OF~COMPLEX SYSTEMS USING~THE~LAWS OF~SYNERGETICS 
AND~INFORMATICS}


\def\titkol{Understanding of complex systems using~the~laws of~synergetics 
and~informatics}

\def\aut{R.\,B.~Seyful-Mulyukov}

\def\autkol{R.\,B.~Seyful-Mulyukov}

\titel{\tit}{\aut}{\autkol}{\titkol}

\vspace*{-11pt}


 \noindent
   Institute of Informatics Problems, Federal Research Center ``Computer Sciences and 
Control'' of the Russian Academy of Sciences; 44-2~Vavilov Str., Moscow 119133, 
Russian Federation

\def\leftfootline{\small{\textbf{\thepage}
\hfill INFORMATIKA I EE PRIMENENIYA~--- INFORMATICS AND
APPLICATIONS\ \ \ 2019\ \ \ volume~13\ \ \ issue\ 4}
}%
 \def\rightfootline{\small{INFORMATIKA I EE PRIMENENIYA~---
INFORMATICS AND APPLICATIONS\ \ \ 2019\ \ \ volume~13\ \ \ issue\ 4
\hfill \textbf{\thepage}}}

\vspace*{3pt}  



\Abste{The author shows how the laws of informatics and synergetics can be used to explain the 
genesis and evolution of such a complex natural system as petroleum. When one creates the matrix 
of hydrocarbon molecules using the laws of informatics, the latter imply the 
ambiguity in the 
quantum behavior of the electrons. This dynamic and static uncertainty comes 
into play during the 
oil field location process. Consideration is given to the laws of synergetics, 
which demonstrate the 
self-organization ability of the molecules. A~new type of molecules is formed in the 
hydrocarbon fluid 
near the bifurcation points, associated with the variation of the thermodynamics, 
structure, and 
mixture of the geological environment. In the analysis of the petroleum formation process, 
consideration is also given to the notion of attractor. It serves as the basin of attraction for all 
hydrocarbon molecules, in which the exact petroleum molecular composition is formed.}

\KWE{synergetics; petroleum formation; informatics and oil location; bifurcation and composition of 
hydrocarbon molecules; attractor and petroleum molecular composition}

 \DOI{10.14357/19922264190417} 

%\vspace*{-14pt}

% \Ack
  % \noindent
  


%\vspace*{-6pt}

  \begin{multicols}{2}

\renewcommand{\bibname}{\protect\rmfamily References}
%\renewcommand{\bibname}{\large\protect\rm References}

{\small\frenchspacing
 {%\baselineskip=10.8pt
 \addcontentsline{toc}{section}{References}
 \begin{thebibliography}{99}
\bibitem{1-s-1}
\Aue{Gurevich, I.\,M.} 2007. \textit{Zakony informatiki~--- osnova stroeniya i~poznaniya slozhnykh 
sistem} [The laws of computer science are the basis of the structure and knowledge of complex 
systems]. Moscow: TORUS PRESS. 399~p.
\bibitem{2-s-1}
\Aue{Seyful-Mulyukov, R.\,B.} 2012. \textit{Neft' i~gaz, glubinnaya priroda i~ee 
prikladnoe 
znachenie} [Petroleum and gas, the underlying nature and its practical significance]. Moscow: 
TORUS PRESS. 215~p.
\bibitem{3-s-1}
\Aue{Seyful-Mulyukov, R.\,B.} 2016. The 
quantum matrix and information from the hydrocarbon oil molecule. 
\textit{Dokl. Earth Sci.}  467(1):246--248.
\bibitem{4-s-1}
\Aue{Seyful-Mulyukov, R.\,B.} 2017. Obrazovanie nefti i~gaza, teoriya i~prikladnye aspekty 
[Oil and gas formation. Theory and practical aspects]. \textit{Geologiya nefti i~gaza} [Geology 
of Oil and Gas] 6:89--96.
\bibitem{5-s-1}
\Aue{Ashby, W.\,R.} 1957. \textit{An introduction to cybernetics}. London: Chapman \& Hall, Ltd.  
289~р.
\bibitem{6-s-1}
\Aue{Petrov, A.\,A.} 1984. \textit{Uglevodorody nefti} [Petoleum hydrocarbons]. Мoscow: Nauka. 
264~p.
\bibitem{7-s-1}
Earth's methane emissions are rising. May~24, 2019. Available at: {\sf 
 https://www.newscientist.com/article/2204466-earths-methane-emissions-are-rising-and-we-dont-know-why/} (accessed September~12, 2019).
\bibitem{8-s-1}
\Aue{Nicolis, G., and I.~Prigogine.} 1989. \textit{Exploring 
complexity: An introduction}. 1st ed. St.\ Martin's Press. 328~p.
\bibitem{9-s-1}
\Aue{Haken, H.} 2000. \textit{Information and self-organization: A~macroscopic approach to complex 
system}. New York, NY: Springer-Verlag. 276~р.
\bibitem{10-s-1}
\Aue{Ashby, W.\,R.} 1947. Principles of the self-organizing dynamic system. 
\textit{J.~Gen. Psychol.} 37:125--128.
\bibitem{11-s-1}
\Aue{Haken, H.} 2012. The brain as a synergetic and physical system. \textit{Symposium  
(International) ``Selforganization in Complex Systems: The Past, Present, and Future of Synergetics'' 
Proceedings}. Delmenhorst. 147--165.
\bibitem{12-s-1}
\Aue{Prigogine, I., and I.~Stengers.} 1984. \textit{Order out of chaos: Man's
new dialogue with nature}. Bantam Books. 349~p.
\bibitem{13-s-1}
\Aue{Bekman, I.\,N.} 2018. Nelineynaya dinamika slozhnykh sistem: teoriya i~praktika 
[Nonlinear 
dynamics of a~complex systems]. Moscow: MSU. 89~p. Available at: {\sf 
http://profbeckman.narod.ru/NelDin/NelDinText2.pdf} (accessed September~29, 2019).
\bibitem{14-s-1}
\Aue{Mendeleyev, D.\,I.} 1876. Neorganicheskoe proiskhozhdenie nefti: Doklad na zasedanii 
russkogo khimicheskogo obshchestva [Inorganic origin of oil: Report at a meeting of the Russian 
chemical society]. \textit{Rev. Sci.} Ser.~VIII:409--416.
\bibitem{15-s-1}
\Aue{Kudryavtsev, N.\,A.} 1973. \textit{Genezis nefti i~gaza} [Genesis of oil and gas]. Leningrad: 
Nedra. 216~p.
\end{thebibliography}

 }
 }

\end{multicols}

\vspace*{-6pt}

\hfill{\small\textit{Received October 15, 2019}}

%\pagebreak

%\vspace*{-22pt}

\Contrl

\noindent
\textbf{Seyful-Mulyukov Rustem B.} (b.\ 1928)~--- Doctor of Science in geology, professor, principal scientist, 
Institute of Informatics Problems, Federal Research Center ``Computer Science and Control'' of the Russian 
Academy of Sciences, 44-2~Vavilov Str,Moscow 119333, Russian Federation; \mbox{rust@ipiran.ru}
\label{end\stat}

\renewcommand{\bibname}{\protect\rm Литература}  
          %17






%%%%%%%%%%%%%%%%%%%%%%%%%%%%%%%%%%%%%%%%%%%%%%%

%\def\stat{rez}
{%\hrule\par
%\vskip 7pt % 7pt
\raggedleft\Large \bf%\baselineskip=3.2ex
Р\,Е\,Ц\,Е\,Н\,З\,И\,И \vskip 17pt
    \hrule
    \par
\vskip 6pt plus 6pt minus 3pt }

%\thispagestyle{headings} %с верхним колонтитулом
%\thispagestyle{myheadings} %с нижним колонтитулом, но в верхнем РЕЦЕНЗИИ

\def\tit{НОВАЯ КНИГА И.\,Н.~СИНИЦЫНА, А.\,С.~ШАЛАМОВА <<ЛЕКЦИИ ПО ТЕОРИИ 
ИНТЕГРИРОВАННОЙ ЛОГИСТИЧЕСКОЙ ПОДДЕРЖКИ>> (М.: ТОРУС ПРЕСС, 2012. 624~с.)}

%1
\def\aut{Д.ф.-м.н., профессор С.\,Я.~Шоргин}

\def\auf{\ }

\def\leftkol{\ % РЕЦЕНЗИИ
}

\def\rightkol{ \ } 

%\def\leftkol{\ } % ENGLISH ABSTRACTS}

%\def\rightkol{\ } %ENGLISH ABSTRACTS}

%\def\leftkol{РЕЦЕНЗИИ}

%\def\rightkol{РЕЦЕНЗИИ}

\titele{\tit}{\aut}{\auf}{\leftkol}{\rightkol}
\vspace*{-18pt}


     \label{st\stat}

     \begin{multicols}{2}
     {\small
     {\baselineskip=10.1pt
     

      В книге представлено системное изложение теоретических основ одного из новейших 
направлений в \mbox{об\-ласти} экономики послепродажного обслуживания изделий наукоемкой 
продукции (ИНП) длительного пользования~--- интегрированной логистической поддержки
(ИЛП). 
{\looseness=1

}

Приведены также результаты новых работ, выполненных в Институте проблем информатики 
Российской академии наук в рамках научного направления <<Информационные технологии и 
анализ сложных сис\-тем>>.
 {%\looseness=1

}
     
      Излагаемые в книге научные подходы позво\-ляют карди\-наль\-но реформировать 
существующие системы производства и эксплуатации ИНП путем создания и внед\-ре\-ния 
методов рационального и оптимального управ\-ле\-ния процессами расходования 
вре\-мен\-н$\acute{\mbox{ы}}$х, 
мате\-ри\-аль\-ных, трудовых и других ресурсов на всех стадиях жизненного цикла изделий (ЖЦИ) по 
критериям экономической целесообразности и эф\-фек\-тив\-ности.
  {\looseness=1

}
    
      В книге приведен краткий обзор причин возник\-новения и
      развития CALS-методологии как основы 
современных международных стандартов по созданию и функционированию глобальных 
ин\-фор\-ма\-ци\-он\-но-ком\-му\-ни\-ка\-ци\-он\-ных систем, ее ключевых возможностей и эффективности 
результатов ее использования. 
Авторы %\linebreak 
предлагают ряд научных обоснований для разработки 
единой теории проектирования и управления систем ИЛП для полноценного использования 
преимуществ %\linebreak
 суще\-ст\-ву\-ющей методологии, определяют \mbox{общую} структурную схему 
комплексной системы <<ИНП-СППО>> и необходимость разработки для ее описания 
гибридных стохастических моделей.
{%\looseness=1

}

%\columnbreak
      
      Книга состоит из пяти частей, где последовательно излагается материал по каждой из 
следующих тем: <<Интегрированная логистическая поддержка>>, <<Теория гибридных 
стохастических систем и компьютерная поддержка исследований и разработок>>, <<Основы 
математического моделирования, анализа и синтеза систем послепродажного обслуживания>>, 
<<Определение и анализ показателей экспортного потенциала ИНП при проектировании>>, 
<<Задачи управления поддержкой послепродажного обслуживания>>, а также 
<<Моделирование инвестиционных процессов ИЛП в условиях неравновесных финансовых 
рынков>>. 
   
      В конце каждой главы приведены выводы и даны вопросы и задания для 
самоконтроля. В~приложениях содержатся основные определения по программам работ по 
анализу ИЛП, логистическим базам данных и компьютерным решениям, эквивалентной статистической 
линеаризации нелинейных преобразований ИЛП, справочный материал, а также развернутые 
уравнения для вероятностных характеристик.


      \def\leftkol{РЕЦЕНЗИИ}

\def\rightkol{РЕЦЕНЗИИ} 

      
      Книга заинтересует широкий круг специалистов и может быть использована научными 
проектными организациями в сфере промышленного производства ИНП. Большое количество 
иллюстраций, примеров и вопросов, обращенных к читателю, позволяет использовать книгу 
также в качестве учебного пособия для студентов и аспирантов машиностроительных, 
транспортных и~других специальностей, а также для самостоятельного изучения. 
{%\looseness=-1

}

Книга 
представляет несомненный интерес для специалистов и студентов в области прикладной 
математики и информатики.
    

}

}
\end{multicols}

%\newpage

\def\stat{authorsrus}
{%\hrule\par
%\vskip 7pt % 7pt
\raggedleft\Large \bf%\baselineskip=3.2ex
О\,Б\ \ А\,В\,Т\,О\,Р\,А\,Х \vskip 17pt
    \hrule
    \par
\vskip 21pt plus 8pt minus 4pt }


\def\tit{\ }

\def\aut{\ }

\def\auf{\ }

\def\leftkol{\ } % ENGLISH ABSTRACTS}

\def\rightkol{ОБ АВТОРАХ} %ENGLISH ABSTRACTS}

\titele{\tit}{\aut}{\auf}{\leftkol}{\rightkol}
      
            \label{st\stat}



\vspace*{24pt}

\begin{multicols}{2}




\noindent
\textbf{Архипов Олег Петрович} (р.\ 1948)~---
кандидат технических наук, директор Орловского филиала Института проб\-лем информатики
Российской академии наук
%302025, г.Орел, Московское шоссе, д.137

\vspace*{3pt}

\noindent
\textbf{Бирюкова Татьяна Константиновна} (р.\ 1968)~---
кандидат фи\-зи\-ко-ма\-те\-ма\-ти\-че\-ских наук, старший научный сотрудник Института проб\-лем информатики
Российской академии наук

\vspace*{3pt}

\noindent 
\textbf{Бобков  Сергей Геннадьевич} (р.\ 1955)~---
доктор технических наук,  заведующий отделением На\-уч\-но-ис\-сле\-до\-ва\-тель\-ско\-го 
института системных исследований Российской академии наук
%117218, Москва, Нахимовский просп., 36, к.1 

\vspace*{3pt}

\noindent \textbf{Васильев Николай Семенович} (р.\ 1952)~--- доктор 
фи\-зи\-ко-ма\-те\-ма\-ти\-че\-ских наук, профессор, 
МГТУ им.\ Н.\,Э.~Баумана 
%, Москва 105005, 2-я Бауманская ул., д.~5,

\vspace*{3pt}

\noindent
\textbf{Гершкович Максим Михайлович} (р.\ 1968)~---
старший научный сотрудник Института проб\-лем информатики
Российской академии наук

\vspace*{3pt}

\noindent 
\textbf{Дьяченко Юрий Георгиевич} (р.\ 1958)~--- кандидат технических наук, 
старший научный сотрудник Института проб\-лем информатики
Российской академии наук

\vspace*{3pt}

\noindent 
\textbf{Ерошенко Александр Андреевич} (р.\ 1989)~--- аспирант кафедры 
математической статистики факультета вычисли\-тельной математики и кибернетики 
Московского государственного университета им.\ М.\,В.~Ломоносова
%119991, Москва ГСП-1, Ленинские горы, д.\ 1, стр. 52

\vspace*{3pt}
 
\noindent 
\textbf{Захаров Виктор Николаевич} (р.\ 1948)~--- 
доктор технических наук, доцент, ученый секретарь Института проб\-лем информатики
Российской академии наук

\vspace*{3pt}

\noindent
\textbf{Зейфман Александр Израилевич} (р.\ 1954)~---
доктор фи\-зи\-ко-ма\-те\-ма\-ти\-че\-ских наук, профессор, 
заведующий кафедрой Вологодского государственного университета; 
старший научный сотрудник Института проб\-лем информатики
Российской академии наук; главный научный сотрудник ИСЭРТ Российской академии наук

\vspace*{3pt}

\noindent
\textbf{Зыкин Сергей Владимирович} (р.\ 1959)~--- 
доктор технических наук, профессор, заведующий лабораторией Института математики 
им.\ С.\,Л.~Соболева Сибирского отделения Российской академии наук, Новосибирск 
%630090, пр.\ ак.\ Коптюга, 4 

\vspace*{4pt}

\noindent
\textbf{Киреев Владимир Иванович} (р.\ 1938)~---
доктор фи\-зи\-ко-ма\-те\-ма\-ти\-че\-ских наук, профессор Московского 
государственного горного университета
%Адрес: Россия, 119991, г. Москва, Ленинский проспект, д. 6

%\columnbreak

\vspace*{4pt}

\noindent
\textbf{Козеренко Елена Борисовна} (р.\ 1959)~---
кандидат филологических наук, заведующая лабораторией Института проб\-лем информатики
Российской академии наук

\vspace*{4pt}

\noindent
\textbf{Королев Виктор Юрьевич} (р.\ 1954)~--- доктор
фи\-зи\-ко-ма\-те\-ма\-ти\-че\-ских наук, профессор кафедры математической 
статистики факультета вычисли\-тельной математики и кибернетики 
Московского государственного университета; 
ведущий научный сотрудник Института проб\-лем информатики
Российской академии наук

\vspace*{4pt}

\noindent
\textbf{Коротышева Анна Владимировна} (р.\ 1988)~---
старший преподаватель Вологодского государственного университета

\vspace*{4pt}

\noindent 
\textbf{Кун Де Турк} (р.\ 1981)~--- научный сотрудник 
исследовательской группы SMACS факультета телекоммуникаций и обработки информации
Университета Гента, Бельгия
%В-9000 Гент, Бельгия

\vspace*{4pt}

\noindent
\textbf{Лупенцов Олег Сергеевич} (р.\ 1986)~---
аспирант Омского государственного института сервиса
%Омск 644043, ул.\ Певцова 13

\vspace*{4pt}

\noindent
\textbf{Лучко Олег Николаевич} (р.\ 1961)~---
кандидат педагогических наук, профессор, заведующий кафедрой 
Омского государственного института сервиса
%Омск 644043, ул.\ Певцова 13

\vspace*{4pt}

\noindent
\textbf{Малашенко Юрий Евгеньевич} (р.\ 1946)~---
доктор фи\-зи\-ко-ма\-те\-ма\-ти\-че\-ских наук, заведующий сектором 
Вычислительного центра им.\ А.\,А.~Дородницына Российской академии наук
%Адрес: 119333, Москва, ул. Вавилова, 40,

\vspace*{4pt}

\noindent
\textbf{Маньяков Юрий Анатольевич} (р.\ 1984)~---
кандидат технических наук, научный сотрудник Орловского филиала Института проб\-лем информатики
Российской академии наук
%302025, г.Орел, Московское шоссе, д.137

\vspace*{4pt}

\noindent
\textbf{Маренко Валентина Афанасьевна} (р.\ 1951)~---
кандидат технических наук, доцент, старший научный сотрудник 
Института математики им.\ С.\,Л.~Соболева Сибирского отделения Российской академии наук
%Новосибирск 630090, пр. ак. Коптюга, 4 

\vspace*{3pt}

\noindent 
\textbf{Морозов Евсей Викторович} (р.\ 1947)~--- доктор 
фи\-зи\-ко-ма\-те\-ма\-ти\-че\-ских, профессор, ведущий научный сотрудник 
Института прикладных математических исследований Карельского научного центра Российской
академии наук; 
%%185910 Россия, Республика Карелия, г.\ Петрозаводск, ул.\ Пушкинская, 11
профессор Петрозаводского государственного университета, Петрозаводск
%185910 Россия, Республика Карелия, г.\ Петрозаводск, пр.\ Ленина, 33

%\pagebreak

\vspace*{3pt}

\noindent
\textbf{Назарова Ирина Александровна} (р.\ 1966)~---
кандидат фи\-зи\-ко-ма\-те\-ма\-ти\-че\-ских наук, 
научный сотрудник Вычислительного центра им.\ А.\,А.~Дородницына Российской академии наук 
%Адрес: 119333, Москва, ул. Вавилова, 40

\vspace*{3pt}

\noindent
\textbf{Павлов Игорь Валерианович} (р.\ 1945)~--- 
доктор фи\-зи\-ко-ма\-те\-ма\-ти\-че\-ских наук, профессор МГТУ им.\ Н.\,Э.~Баумана 
%Москва 105005, 2-я Бауманская ул., д.~5 

%\pagebreak

\vspace*{3pt}

\noindent 
\textbf{Потахина Любовь Викторовна} (р.\ 1989)~--- аспирантка
Института прикладных математических исследований Карельского научного центра
Российской академии наук; 
%%185910 Россия, Республика Карелия, г.\ Петрозаводск, ул.\ Пушкинская, 11
инженер Петрозаводского государственного университета, Петрозаводск
%185910 Россия, Республика Карелия, г.\ Петрозаводск, пр.\ Ленина, 33

\vspace*{3pt}

\noindent 
\textbf{Рождественский Юрий Владимирович} (р.\ 1952)~--- 
кандидат технических наук, заведующий сектором Института проб\-лем информатики
Российской академии наук

\vspace*{3pt}

\noindent 
\textbf{Синицын Игорь Николаевич} (р.\ 1940)~--- доктор технических наук,
профессор, заслуженный деятель\linebreak\vspace*{-12pt}

\columnbreak

\noindent
 науки РФ, заведующий отделом Института проб\-лем информатики
Российской академии наук

\vspace*{7pt}


\noindent
\textbf{Сиротинин Денис Олегович} (р.\ 1984)~---
кандидат технических наук, научный сотрудник Орловского филиала Института проб\-лем информатики
Российской академии наук
%302025, г.Орел, Московское шоссе, д.137

\vspace*{7pt}

%\columnbreak

\noindent 
\textbf{Соколов  Игорь Анатольевич} (р.\ 1954)~--- академик (действительный член) Российской 
академии наук, доктор технических наук, директор Института проб\-лем информатики
Российской академии наук

\vspace*{7pt}

\noindent
\textbf{Степченков Юрий Афанасьевич} (р.\ 1951)~---
кандидат технических наук, заведующий отделом Института проб\-лем информатики
Российской академии наук

\vspace*{7pt}

\noindent
\textbf{Сурков Алексей Викторович} (р.\ 1978)~--- 
старший научный сотрудник На\-уч\-но-ис\-сле\-до\-ва\-тель\-ско\-го 
института системных исследований Российской академии наук
%117218, Москва, Нахимовский просп., 36, к.1 

\vspace*{7pt}

\noindent 
\textbf{Шестаков Олег Владимирович} (р.\ 1976)~--- доктор 
фи\-зи\-ко-ма\-те\-ма\-ти\-че\-ских, доцент кафедры математической статистики 
факультета вычисли\-тельной математики и кибернетики Московского 
государственного университета им.\ М.\,В.~Ломоносова; 
%119991, Москва ГСП-1, Ленинские горы, д.\ 1, стр. 52
старший научный сотрудник Института проб\-лем информатики
Российской академии наук
%, Москва 119333, ул. Вавилова, д.~44, корп.~2

\vspace*{7pt}

\noindent 
\textbf{Шоргин Сергей Яковлевич} (р.\ 1952.)~--- доктор
фи\-зи\-ко-ма\-те\-ма\-ти\-че\-ских наук, профессор, заместитель директора Института 
проб\-лем информатики Российской академии наук





%%%%%%%%%%%%%%%%%%%%%%%%%%%%%%%%%%%%%%%%%%%%%%%%%%%%%%%%%%%%%%%%%%%%%%%%%%%%%%%




%\def\rightkol{ОБ АВТОРАХ}
%\def\leftkol{ОБ АВТОРАХ}

 \label{end\stat}





%\def\leftfootline{\small{\textbf{\thepage}
%\hfill ИНФОРМАТИКА И ЕЁ ПРИМЕНЕНИЯ\ \ \ том~7\ \ \ выпуск~1\ \ \ 2013}
%}%
% \def\rightfootline{\small{ИНФОРМАТИКА И ЕЁ ПРИМЕНЕНИЯ\ \ \ том~7\ \ \ выпуск~1\ \ \ 2013
%\hfill \textbf{\thepage}}}


%\thispagestyle{myheadings}



\end{multicols}

\newpage  

%\def\stat{cont}
{%\hrule\par
%\vskip 7pt % 7pt
\raggedleft\Large \bf%\baselineskip=3.2ex
А\,В\,Т\,О\,Р\,С\,К\,И\,Й\ \ У\,К\,А\,З\,А\,Т\,Е\,Л\,Ь\ \ З\,А\ \ 2\,0\,0\,7 г. \vskip 17pt
    \hrule
    \par
\vskip 21pt plus 6pt minus 3pt }

\label{st\stat}

\def\tit{\ }

\def\aut{\ }
\def\auf{\ }

\def\leftkol{\ } % ENGLISH ABSTRACTS}

\def\rightkol{\ } %ENGLISH ABSTRACTS}

\titele{\tit}{\aut}{\auf}{\leftkol}{\rightkol}


\contentsline {chapter}{\ }{Выпуск \quad Стр.} 
\contentsline {section}{\textbf{Батракова Д.\,А., Королев В.\,Ю., Шоргин С.\,Я.}\ \ Новый метод вероятностно-ста\-ти\-сти\-че\-ско\-го анализа информационных потоков в\nobreakspace {}телекоммуникационных сетях}{\qquad 1 \qquad 40} 
\contentsline {section}{\textbf{Борисов А.\,В.}\ \ Байесовское оценивание в системах наблюдения с\nobreakspace {}марковскими скачкообразными процессами: игровой подход}{\qquad 2 \qquad 65}
\contentsline {section}{\textbf{Босов А.\,В., Иванов А.\,В.}\ \ Программная инфраструктура информационного Web-пор\-тала}{\qquad 2 \qquad 50}
\contentsline {section}{\textbf{Захаров В.\,Н., Калиниченко Л.\,А., Соколов И.\,А., Ступников С.\,А.}\ \ Конструирование канонических информационных моделей для интегрированных информационных систем}{\qquad 2 \qquad 15}
\contentsline {section}{\textbf{Захаров В.\,Н., Козмидиади В.\,А.}\ \ Средства обеспечения отказоустойчивости при\-ло\-жений}{\qquad 1 \qquad 14} 
\contentsline {section}{\textbf{Иванов А.\,В.}\ \ см. Босов А.\,В.\hfill\hfill\hfill\hfill\hfill\hfill\hfill\hfill\hfill\hfill\hfill\hfill\hfill\hfill\hfill\hfill\hfill\hfill\hfill\hfill\hfill\hfill\hfill\hfill\hfill\hfill\hfill\hfill\hfill\hfill\hfill\hfill\hfill\hfill\hfill}{\ }
\contentsline {section}{\textbf{Ильин В.\,Д., Соколов И.\,А.}\ \ Символьная модель системы знаний информатики в\nobreakspace {}че\-ло\-ве\-ко-автоматной среде}{\qquad 1 \qquad 66} 
\contentsline {section}{\textbf{Калиниченко Л.\,А.}\ \ см. Захаров В.\,Н.\hfill\hfill\hfill\hfill\hfill\hfill\hfill\hfill\hfill\hfill\hfill\hfill\hfill\hfill\hfill\hfill\hfill\hfill\hfill\hfill\hfill\hfill\hfill\hfill\hfill\hfill\hfill\hfill\hfill\hfill\hfill\hfill\hfill\hfill\hfill}{\ }
\contentsline {section}{\textbf{Козеренко Е.\,Б.}\ \ Лингвистическое моделирование для систем машинного перевода и обработки знаний}{\qquad 1 \qquad 54} 
\contentsline {section}{\textbf{Козмидиади В.\,А.}\ \ см. Захаров В.\,Н.\hfill\hfill\hfill\hfill\hfill\hfill\hfill\hfill\hfill\hfill\hfill\hfill\hfill\hfill\hfill\hfill\hfill\hfill\hfill\hfill\hfill\hfill\hfill\hfill\hfill\hfill\hfill\hfill\hfill\hfill\hfill\hfill\hfill\hfill\hfill }{\ } 
\contentsline {section}{\textbf{Королев В.\,Ю.}\ \ см. Батракова Д.\,А.\hfill\hfill\hfill\hfill\hfill\hfill\hfill\hfill\hfill\hfill\hfill\hfill\hfill\hfill\hfill\hfill\hfill\hfill\hfill\hfill\hfill\hfill\hfill\hfill\hfill\hfill\hfill\hfill\hfill\hfill\hfill\hfill\hfill\hfill\hfill}{\ } 
\contentsline {section}{\textbf{Кудрявцев А.\,А., Шоргин С.\,Я.}\ \ Байесовский подход к\nobreakspace {}анализу систем массового обслуживания и\nobreakspace {}показателей надежности}{\qquad 2 \qquad 76}
\contentsline {section}{\textbf{Печинкин А.\,В., Соколов И.\,А., Чаплыгин В.\,В.}\ \ Многолинейная система массового обслуживания с конечным накопителем и ненадежными приборами}{\qquad 1 \qquad 27} 
\contentsline {section}{\textbf{Печинкин А.\,В., Соколов И.\,А., Чаплыгин В.\,В.}\ \ Стационарные характеристики многолинейной\nobreakspace {}системы массового обслуживания с\nobreakspace {}одновременными отказами приборов}{\qquad 2 \qquad 39}
\contentsline {section}{\textbf{Синицын И.\,Н.}\ \ Корреляционные методы построения аналитических информационных моделей флуктуаций полюса Земли по априорным данным}{\qquad 2 \qquad \hphantom{9}2}
\contentsline {section}{\textbf{Синицын И.\,Н.}\ \ Развитие теории фильтров Пугачева для оперативной обработки информации в стохастических системах}{{\qquad 1 \qquad \hphantom{9}3}} 
\contentsline {section}{\textbf{Соколов И.\,А.}\ \ см. Захаров В.\,Н.\hfill\hfill\hfill\hfill\hfill\hfill\hfill\hfill\hfill\hfill\hfill\hfill\hfill\hfill\hfill\hfill\hfill\hfill\hfill\hfill\hfill\hfill\hfill\hfill\hfill\hfill\hfill\hfill\hfill\hfill\hfill\hfill\hfill\hfill\hfill}{\ }
\contentsline {section}{\textbf{Соколов И.\,А.}\ \ см. Ильин В.\,Д.\hfill\hfill\hfill\hfill\hfill\hfill\hfill\hfill\hfill\hfill\hfill\hfill\hfill\hfill\hfill\hfill\hfill\hfill\hfill\hfill\hfill\hfill\hfill\hfill\hfill\hfill\hfill\hfill\hfill\hfill\hfill\hfill\hfill\hfill\hfill}{\ } 
\contentsline {section}{\textbf{Соколов И.\,А.}\ \ см. Печинкин А.\,В.\hfill\hfill\hfill\hfill\hfill\hfill\hfill\hfill\hfill\hfill\hfill\hfill\hfill\hfill\hfill\hfill\hfill\hfill\hfill\hfill\hfill\hfill\hfill\hfill\hfill\hfill\hfill\hfill\hfill\hfill\hfill\hfill\hfill\hfill\hfill}{\ } 
\contentsline {section}{\textbf{Соколов И.\,А.}\ \ см. Печинкин А.\,В.\hfill\hfill\hfill\hfill\hfill\hfill\hfill\hfill\hfill\hfill\hfill\hfill\hfill\hfill\hfill\hfill\hfill\hfill\hfill\hfill\hfill\hfill\hfill\hfill\hfill\hfill\hfill\hfill\hfill\hfill\hfill\hfill\hfill\hfill\hfill}{\ }
\contentsline {section}{\textbf{Ступников С.\,А.}\ \ см. Захаров В.\,Н.\hfill\hfill\hfill\hfill\hfill\hfill\hfill\hfill\hfill\hfill\hfill\hfill\hfill\hfill\hfill\hfill\hfill\hfill\hfill\hfill\hfill\hfill\hfill\hfill\hfill\hfill\hfill\hfill\hfill\hfill\hfill\hfill\hfill\hfill\hfill}{\ }
\contentsline {section}{\textbf{Чаплыгин В.\,В.}\ \ см. Печинкин А.\,В.\hfill\hfill\hfill\hfill\hfill\hfill\hfill\hfill\hfill\hfill\hfill\hfill\hfill\hfill\hfill\hfill\hfill\hfill\hfill\hfill\hfill\hfill\hfill\hfill\hfill\hfill\hfill\hfill\hfill\hfill\hfill\hfill\hfill\hfill\hfill}{\ } 
\contentsline {section}{\textbf{Чаплыгин В.\,В.}\ \ см. Печинкин А.\,В.\hfill\hfill\hfill\hfill\hfill\hfill\hfill\hfill\hfill\hfill\hfill\hfill\hfill\hfill\hfill\hfill\hfill\hfill\hfill\hfill\hfill\hfill\hfill\hfill\hfill\hfill\hfill\hfill\hfill\hfill\hfill\hfill\hfill\hfill\hfill}{\ }
\contentsline {section}{\textbf{Шоргин С.\,Я.}\ \ см. Батракова Д.\,А.\hfill\hfill\hfill\hfill\hfill\hfill\hfill\hfill\hfill\hfill\hfill\hfill\hfill\hfill\hfill\hfill\hfill\hfill\hfill\hfill\hfill\hfill\hfill\hfill\hfill\hfill\hfill\hfill\hfill\hfill\hfill\hfill\hfill\hfill\hfill}{\ } 
\contentsline {section}{\textbf{Шоргин С.\,Я.}\ \ см. Кудрявцев А.\,А.\hfill\hfill\hfill\hfill\hfill\hfill\hfill\hfill\hfill\hfill\hfill\hfill\hfill\hfill\hfill\hfill\hfill\hfill\hfill\hfill\hfill\hfill\hfill\hfill\hfill\hfill\hfill\hfill\hfill\hfill\hfill\hfill\hfill\hfill\hfill}{\ }
%\thispagestyle{myheadings}
\def\leftfootline{\small{\textbf{\thepage}
\hfill ИНФОРМАТИКА И ЕЁ ПРИМЕНЕНИЯ\ \ \ том~1\ \ \ выпуск~2\ \ \ 2007}
}%
 \def\rightfootline{\small{ИНФОРМАТИКА И ЕЁ ПРИМЕНЕНИЯ\ \ \ том~1\ \ \ выпуск~2\ \ \ 2007
 \hfill \textbf{\thepage}}}
 \label{end\stat} 
                     
%\def\stat{cont-e}
{%\hrule\par
%\vskip 7pt % 7pt
\raggedleft\Large \bf%\baselineskip=3.2ex
2\,0\,0\,7\ \ A\,U\,T\,H\,O\,R\ \ I\,N\,D\,E\,X \vskip 17pt
    \hrule
    \par
\vskip 21pt plus 6pt minus 3pt }

\label{st\stat}

\def\tit{\ }

\def\aut{\ }
\def\auf{\ }

\def\leftkol{\ } % ENGLISH ABSTRACTS}

\def\rightkol{\ } %ENGLISH ABSTRACTS}

\titele{\tit}{\aut}{\auf}{\leftkol}{\rightkol}


\contentsline {chapter}{\ }{Issue \quad Page} 
\contentsline {subsection}{\textbf{Batrakova D.\,A., Korolev V.\,Yu., Shorgin S.\,Ya.}\ \ A New Method for the Probabilistic and Statistical Analysis of Information Flows in Telecommunication Networks}{\qquad 1 \qquad 40} 
\contentsline {subsection}{\textbf{Borisov A.\,V.}\ \ Bayesian Estimation in\nobreakspace {}Observation Systems with\nobreakspace {}Markov Jump Processes: Game-Theoretic Approach}{\qquad 2 \qquad 65} 
\contentsline {subsection}{\textbf{Bosov A.\,V., Ivanov A.\,V.}\ \ Linguistic Simulation for Machine Translation and Knowledge Management Systems}{\qquad 2 \qquad 50} 
\contentsline {subsection}{\textbf{Chaplygin V.\,V.} see Pechinkin A.\,V.\hfill\hfill\hfill\hfill\hfill\hfill\hfill\hfill\hfill\hfill\hfill\hfill\hfill\hfill\hfill\hfill\hfill\hfill\hfill\hfill\hfill\hfill\hfill\hfill\hfill\hfill\hfill\hfill\hfill\hfill\hfill\hfill\hfill\hfill\hfill}{\ }
\contentsline {subsection}{\textbf{Chaplygin V.\,V.} see Pechinkin A.\,V.\hfill\hfill\hfill\hfill\hfill\hfill\hfill\hfill\hfill\hfill\hfill\hfill\hfill\hfill\hfill\hfill\hfill\hfill\hfill\hfill\hfill\hfill\hfill\hfill\hfill\hfill\hfill\hfill\hfill\hfill\hfill\hfill\hfill\hfill\hfill}{\ }
\contentsline {subsection}{\textbf{Ilyin V.\,D., Sokolov I.\,A.}\ \ The Symbol Model of Informatics Knowledge System in Human-Automaton Environment}{\qquad 1 \qquad 66} 
\contentsline {subsection}{\textbf{Ivanov A.\,V.} see Bosov A.\,V.\hfill\hfill\hfill\hfill\hfill\hfill\hfill\hfill\hfill\hfill\hfill\hfill\hfill\hfill\hfill\hfill\hfill\hfill\hfill\hfill\hfill\hfill\hfill\hfill\hfill\hfill\hfill\hfill\hfill\hfill\hfill\hfill\hfill\hfill\hfill}{\ }
\contentsline {subsection}{\textbf{Kalinichenko L.\,A.} see Zakharov V.\,N.\hfill\hfill\hfill\hfill\hfill\hfill\hfill\hfill\hfill\hfill\hfill\hfill\hfill\hfill\hfill\hfill\hfill\hfill\hfill\hfill\hfill\hfill\hfill\hfill\hfill\hfill\hfill\hfill\hfill\hfill\hfill\hfill\hfill\hfill\hfill}{\ }
\contentsline {subsection}{\textbf{Korolev V.\,Yu.} see Batrakova D.\,A.\hfill\hfill\hfill\hfill\hfill\hfill\hfill\hfill\hfill\hfill\hfill\hfill\hfill\hfill\hfill\hfill\hfill\hfill\hfill\hfill\hfill\hfill\hfill\hfill\hfill\hfill\hfill\hfill\hfill\hfill\hfill\hfill\hfill\hfill\hfill}{\ }
\contentsline {subsection}{\textbf{Kozerenko E.\,B.}\ \ Linguistic Simulation for Machine Translation and Knowledge Management Systems}{\qquad 1 \qquad 54} 
\contentsline {subsection}{\textbf{Kozmidiady V.\,A.} see Zakharov V.\,N.\hfill\hfill\hfill\hfill\hfill\hfill\hfill\hfill\hfill\hfill\hfill\hfill\hfill\hfill\hfill\hfill\hfill\hfill\hfill\hfill\hfill\hfill\hfill\hfill\hfill\hfill\hfill\hfill\hfill\hfill\hfill\hfill\hfill\hfill\hfill}{\ }
\contentsline {subsection}{\textbf{Kudryavtsev A.\,A., Shorgin S.\,Ya.}\ \ Bayesian Approach to Queueing Systems and Reliability Characteristics}{\qquad 2 \qquad 76} 
\contentsline {subsection}{\textbf{Pechinkin A.\,V., Sokolov I.\,A., Chaplygin V.\,V.}\ \ Multichannel Queuing System with Finite Buffer and Unreliable Servers}{\qquad 1 \qquad 27} 
\contentsline {subsection}{\textbf{Pechinkin A.\,V., Sokolov I.\,A., Chaplygin V.\,V.}\ \ Stationary Characteristics of a Multichannel Queueing System with\nobreakspace {}Simultaneous Refusals of Servers}{\qquad 2 \qquad 39} 
\contentsline {subsection}{\textbf{Shorgin S.\,Ya.} see Batrakova D.\,A.\hfill\hfill\hfill\hfill\hfill\hfill\hfill\hfill\hfill\hfill\hfill\hfill\hfill\hfill\hfill\hfill\hfill\hfill\hfill\hfill\hfill\hfill\hfill\hfill\hfill\hfill\hfill\hfill\hfill\hfill\hfill\hfill\hfill\hfill\hfill}{\ }
\contentsline {subsection}{\textbf{Shorgin S.\,Ya.} see Kudryavtsev A.\,A.\hfill\hfill\hfill\hfill\hfill\hfill\hfill\hfill\hfill\hfill\hfill\hfill\hfill\hfill\hfill\hfill\hfill\hfill\hfill\hfill\hfill\hfill\hfill\hfill\hfill\hfill\hfill\hfill\hfill\hfill\hfill\hfill\hfill\hfill\hfill}{\ }
\contentsline {subsection}{\textbf{Sinitsyn I.\,N.}\ \ Correlational Methods for Analytical Informational Models of the Earth Pole Fluctuations Design Based on a priori Data}{\qquad 2 \qquad \hphantom{9}2}
\contentsline {subsection}{\textbf{Sinitsyn I.\,N.}\ \ Development of Pugachev Filtering for Stochastic Systems}{\qquad 1 \qquad \hphantom{9}3}
\contentsline {subsection}{\textbf{Sokolov I.\,A.} see Ilyin V.\,D.\hfill\hfill\hfill\hfill\hfill\hfill\hfill\hfill\hfill\hfill\hfill\hfill\hfill\hfill\hfill\hfill\hfill\hfill\hfill\hfill\hfill\hfill\hfill\hfill\hfill\hfill\hfill\hfill\hfill\hfill\hfill\hfill\hfill\hfill\hfill}{\ }
\contentsline {subsection}{\textbf{Sokolov I.\,A.} see Pechinkin A.\,V.\hfill\hfill\hfill\hfill\hfill\hfill\hfill\hfill\hfill\hfill\hfill\hfill\hfill\hfill\hfill\hfill\hfill\hfill\hfill\hfill\hfill\hfill\hfill\hfill\hfill\hfill\hfill\hfill\hfill\hfill\hfill\hfill\hfill\hfill\hfill}{\ }
\contentsline {subsection}{\textbf{Sokolov I.\,A.} see Pechinkin A.\,V.\hfill\hfill\hfill\hfill\hfill\hfill\hfill\hfill\hfill\hfill\hfill\hfill\hfill\hfill\hfill\hfill\hfill\hfill\hfill\hfill\hfill\hfill\hfill\hfill\hfill\hfill\hfill\hfill\hfill\hfill\hfill\hfill\hfill\hfill\hfill}{\ }
\contentsline {subsection}{\textbf{Sokolov I.\,A.} see Zakharov V.\,N.\hfill\hfill\hfill\hfill\hfill\hfill\hfill\hfill\hfill\hfill\hfill\hfill\hfill\hfill\hfill\hfill\hfill\hfill\hfill\hfill\hfill\hfill\hfill\hfill\hfill\hfill\hfill\hfill\hfill\hfill\hfill\hfill\hfill\hfill\hfill}{\ }
\contentsline {subsection}{\textbf{Stupnikov S.\,A.} see Zakharov V.\,N.\hfill\hfill\hfill\hfill\hfill\hfill\hfill\hfill\hfill\hfill\hfill\hfill\hfill\hfill\hfill\hfill\hfill\hfill\hfill\hfill\hfill\hfill\hfill\hfill\hfill\hfill\hfill\hfill\hfill\hfill\hfill\hfill\hfill\hfill\hfill}{\ }
\contentsline {subsection}{\textbf{Zakharov V.\,N., Kalinichenko L.\,A., Sokolov I.\,A., Stupnikov S.\,A.}\ \ Development of Canonical Information Models for Integrated Information Systems}{\qquad 2 \qquad 15} 
\contentsline {subsection}{\textbf{Zakharov V.\,N., Kozmidiady V.\,A.}\ \ Means Providing Applications Fault Tolerance}{\qquad 1 \qquad 14} 
\def\leftfootline{\small{\textbf{\thepage}
\hfill ИНФОРМАТИКА И ЕЁ ПРИМЕНЕНИЯ\ \ \ том~1\ \ \ выпуск~2\ \ \ 2007}
}%
 \def\rightfootline{\small{ИНФОРМАТИКА И ЕЁ ПРИМЕНЕНИЯ\ \ \ том~1\ \ \ выпуск~2\ \ \ 2007
 \hfill \textbf{\thepage}}}
 \label{end\stat} 


%\end{document}

%
\def\stat{rekl}
%\label{preobr}

%\def\tit{АКАДЕМИК ПУГАЧЁВ  ВЛАДИМИР СЕМЁНОВИЧ\\
%25.03.1911--25.03.1998}


%   \vspace*{-48pt}
%   \begin{center}\LARGE
%Академик Пугачёв  Владимир Семёнович\\ (25.03.1911--25.03.1998)
%   \end{center}

   %\vspace*{2.5mm}

   \begin{center}

{\prgsh\LARGE
ЮБИЛЕИ}

\end{center}
%\hrule

\vspace*{6pt}


   \vspace*{8mm}

   \thispagestyle{empty}


%\def\stat{emel}


\section*{К 70-летию заместителя директора ИПИ РАН,\\ члена редколлегии журнала
<<Информатика и её применения>>\\ доктора технических наук В.\,И.~Будзко}

\vspace*{18pt}




          \begin{multicols}{2}

%            \label{st\stat}

\begin{center}
\vspace*{1pt}
\mbox{%
\epsfxsize=78mm
\epsfbox{bud-1.eps}
}
\end{center}

\vspace*{12pt}

      14 августа 2014~г.\ исполнилось 70~лет за\-мес\-ти\-те\-лю директора ИПИ РАН по
научной работе доктору технических наук Владимиру Игоревичу Будзко.

      Владимир Игоревич Будзко родился в г.~Москве. Высшее образование получил на факультете
элект\-рон\-но-вы\-чис\-ли\-тель\-ных устройств в Московском
ин\-же\-нер\-но-фи\-зи\-че\-ском институте
(МИФИ), который он окончил в 1968~г., после чего был на\-прав\-лен для прохождения
службы в одну из войс\-ко\-вых частей, где прошел путь от инженера до первого заместителя
командира войсковой части.

      С приходом В.\,И.~Будзко в ИПИ РАН (2001~г.)\ в институте
сформировалось новое научное на\-прав\-ле\-ние теоретических исследований~--- <<Постро\-ение
ин\-фор\-ма\-ци\-он\-но-те\-ле\-ком\-му\-ни\-ка\-ци\-он\-ных\linebreak сис\-тем
высокой до\-ступ\-ности>>. В~рамках этого
направления выполнен широкий круг фундаментальных исследований по поиску подходов и
определению принципов построения средств обеспечения доступности, конфиденциальности
и целостности современных крупномасштабных
ин\-фор\-ма\-ци\-он\-но-те\-ле\-ком\-му\-ни\-ка\-ци\-он\-ных
сис\-тем (ИТС). Разработаны основные сис\-тем\-но-тех\-ни\-че\-ские принципы и базовые
архитектурные решения построения перспективных для условий России ИТС с
централизованной обработкой и хранением информации, сочетающих в себе свойства
высокой доступности, отказо- и катастрофоустойчивости, информационной защищенности.
Определены принципы, методы и математические основы рационального построения и
оптимизации средств восстановления функционирования центров обработки данных (ЦОД)
после возникновения отказов и катастроф, передачи и хранения данных, обеспечения
информационной безопасности при достижении минимальной совокупной стоимости
владения такими системами. Результаты нашли практическое воплощение при реализации
проектов в интересах ряда отечественных государственных и негосударственных
организаций, таких как Банк России (БР), Внешторгбанк, ОАО <<ГМК <<Норильский Никель>>,
<<Газпром>>, Минэкономразвития России, Правительство Москвы, а также ряд силовых
ведомств.

      Под руководством В.\,И.~Будзко начиная с 2001~г.\ выполнен комплекс
      на\-уч\-но-ис\-сле\-до\-ва\-тель\-ских и
      опыт\-но-кон\-ст\-рук\-тор\-ских работ (свыше 100~проектов),
направленных на развитие электронной информационной технологии БР.
Разработаны концепции развития ИТС БР сначала до 2008~г., а затем до 2013~г., которые
были приняты в качестве основы проведения технической политики. За реализацию проекта
<<Катастрофоустойчивая тер\-ри\-то\-ри\-аль\-но-рас\-пре\-де\-лен\-ная
      ин\-фор\-ма\-ци\-он\-но-те\-ле\-ком\-му\-ни\-ка\-ци\-он\-ная сис\-те\-ма централизованной
обработки банковской информации>> В.\,И.~Будзко удостоен Премии Правительства РФ в
области науки и техники за 2010~г.

      В.\,И.~Будзко возглавлял и возглавляет работы по ряду других прикладных проектов,
связанных с созданием, совершенствованием и развитием крупномасштабных ИТС.

      В.\,И.~Будзко~--- генерал-майор, доктор технических наук, член-кор\-рес\-пон\-дент
Академии криптографии РФ, известный ученый в области информатики и применения
информационных технологий при построении территориально распределенных ИТС
различного назначения. Является автором свыше 250~научных работ, опубликованных в
на\-уч\-но-тех\-ни\-че\-ских и специальных изданиях.

    \thispagestyle{empty}

      В.\,И.~Будзко уделяет большое внимание подготовке научных кадров. Под его
руководством защищено 6~диссертаций на соискание ученой степени кандидата
технических наук. Свыше 30~лет он читает лекции в ИКСИ Академии ФСБ, профессор
кафедры НИЯУ МИФИ. Является членом двух диссертационных советов, главным
редактором журнала <<Системы высокой доступности>> и членом редколлегии журнала
<<Информатика и её применения>>.

      \bigskip

      Редакционный совет и Редакционная коллегия журнала <<Информатика и её
применения>> сердечно поздравляют Владимира Игоревича Будзко с 70-ле\-ти\-ем и желают
крепкого здоровья и новых научных достижений.

\end{multicols}

%Информатика и её применения
%Том 13 Выпуск 1-4 Год 2019

\def\stat{cont}
{%\hrule\par
%\vskip 7pt % 7pt
\raggedleft\Large \bf%\baselineskip=3.2ex
А\,В\,Т\,О\,Р\,С\,К\,И\,Й\ \ У\,К\,А\,З\,А\,Т\,Е\,Л\,Ь\ \ З\,А\ \ 2\,0\,1\,9 г. \vskip 17pt
 \hrule
 \par
\vskip 21pt plus 6pt minus 3pt }

\label{st\stat}

\def\tit{\ }

\def\aut{\ }
\def\auf{\ }

\def\leftkol{\ } % ENGLISH ABSTRACTS}

\def\rightkol{\ } %АВТОРСКИЙ УКАЗАТЕЛЬ ЗА 2019 г.} %ENGLISH ABSTRACTS}

\titele{\tit}{\aut}{\auf}{\leftkol}{\rightkol}
\addcontentsline{toc}{subsection}{\textrm\textbf Авторский указатель за 2019 г.}

%\vspace*{-12pt}

\noindent
{\tabcolsep=3pt
\begin{tabular}{p{397pt}cc}
&\textbf{Вып.} & \textbf{Стр.}\\[6pt]
\Avtors{Абгарян~К.\,К., Осипова~В.\,А.} Применение методов поддержки принятия решений для\linebreak
\\[-12pt]
\hspace*{23pt}многокритериальной задачи отбора многомасштабных композиций&2&47--53\\
\Avtors{Агаларов~Я.\,М., Коновалов~М.\,Г.} Доказательство унимодальности целевой функции\linebreak
\\[-12pt]
\hspace*{23pt}в~задаче порогового управления нагрузкой на~сервер&2&2--6\\
\Avtors{Агаларов~Я.\,М., Ушаков~В.\,Г.} Об унимодальности функции дохода системы массового\linebreak
\\[-12pt]
\hspace*{23pt}обслуживания типа $G|M|s$ с~управляемой очередью&1&55--61\\
\Avtors{Агасандян~Г.\,А.} Вычисление показателей оптимальных по CC-VaR портфелей на~рынках\linebreak
\\[-12pt]
\hspace*{23pt}опционов&3&72--81\\
\Avtors{Агасандян~Г.\,А.} Теоретические основы оптимизации по континуальному критерию VaR на совокупности рынков&4&36--41\\
\Avtors{Анашин~В.\,С.} О теоретико-автоматных моделях блокчейн-среды&2&29--36\\
\Avtors{Аникеев~Д.\,А., Пенкин~Г.\,О., Стрижов~В.\,В.} Классификация физической активности\linebreak
\\[-12pt]
\hspace*{23pt}человека с~помощью локальных аппроксимирующих моделей&1&40--48\\
\Avtors{Арутюнов~Е.\,Н., Кудрявцев~А.\,А., Титова~А.\,И.} Байесовские модели баланса факторов, \linebreak
\\[-12pt]
\hspace*{23pt}имеющих априорные распределения Вейбулла и~Накагами&2&71--75\\
\Avtors{Бахтеев~О.\,Ю.} см.\ Грабовой~А.\,В.&&\\
\Avtors{Бондаренко~Н.\,Н.} см.\ Журавлев~Ю.\,И.&&\\
\Avtors{Борисов~А.\,В.} Численные схемы фильтрации марковских скачкообразных процессов по\linebreak
\\[-12pt]
\hspace*{23pt}дискретизованным наблюдениям~I: характеристики точности&4&68--75\\
\Avtors{Босов~А.\,В., Миллер~Г.\,Б.} О развитии концепции условно-минимаксной нелинейной\linebreak
\\[-12pt]
\hspace*{23pt}фильтрации: модифицированный фильтр и~его анализ&2&\hphantom{1}7--15\\
\Avtors{Босов~А.\,В., Мхитарян~Г.\,А., Наумов~А.\,В., Сапунова~А.\,П.} Использование модели гамма-распределения в~задаче формирования ограниченного по времени теста в~системе\linebreak
\\[-12pt]
\hspace*{23pt}дистанционного обучения&4&11--17\\
\Avtors{Босов~А.\,В., Стефанович~А.\,И.} Управление выходом стохастической дифференциальной системы по~квадратичному критерию. II.~Численное решение уравнений динами-\linebreak
\\[-12pt]
\hspace*{23pt}ческого программирования&1&\hphantom{1}9--15\\
\Avtors{Босов~А.\,В., Стефанович~А.\,И.} Управление выходом стохастической дифференциальной системы по квадратичному критерию. III.~Анализ свойств оптимального управ-\linebreak
\\[-12pt]
\hspace*{23pt}ления&3&41--49\\
\Avtors{Бурлуцкий~В.\,В., Якимчук~А.\,В., Мельников~А.\,В., Царегородцев~А.\,Л., Волошин~С.\,В.} Разработка метода формирования признакового пространства и~модели для оценки и~прогнозирования антропогенного влияния на окружающую среду (на примере\linebreak
\\[-12pt]
\hspace*{23pt}лесного фонда нефтедобывающего региона)&3&131--136\\
\Avtors{Вахтанов~Н.\,А.} см.\ Шнурков~П.\,В.&&\\
\Avtors{Вахтанов~Н.\,А.} см.\ Шнурков~П.\,В.&&\\
\Avtors{Виноградов~А.\,П.} см.\ Журавлев~Ю.\,И.&&\\
\Avtors{Волошин~С.\,В.} см.\ Бурлуцкий~В.\,В.&&\\
\Avtors{Вышинский~Л.\,Л., Курьянский~М.\,К., Флеров~Ю.\,А.} Цифровая модель весового паспорта\linebreak
\\[-12pt]
\hspace*{23pt}летательного аппарата&4&\hphantom{1}3--10\\
\Avtors{Гайдамака~А.\,А., Чухно~Н.\,В., Чухно~О.\,В., Самуйлов~К.\,Е., Шоргин~С.\,Я.} Формализация метода ранжирования альтернатив для процесса группового принятия решений при\linebreak
\\[-12pt]
\hspace*{23pt}анализе социальных сетей&3&63--71\\
\Avtors{Гайдамака~Ю.\,В.} см.\ Горбунова~А.\,В.&&\\
\end{tabular}
}

\pagebreak

\def\leftkol{АВТОРСКИЙ УКАЗАТЕЛЬ ЗА 2019 г.} % ENGLISH ABSTRACTS}

\def\rightkol{АВТОРСКИЙ УКАЗАТЕЛЬ ЗА 2019 г.} %ENGLISH ABSTRACTS}

%\thispagestyle{myheadings}
\def\leftfootline{\small{\textbf{\thepage}
\hfill ИНФОРМАТИКА И ЕЁ ПРИМЕНЕНИЯ\ \ \ том~13\ \ \ выпуск~4\ \ \ 2019}
}%
 \def\rightfootline{\small{ИНФОРМАТИКА И ЕЁ ПРИМЕНЕНИЯ\ \ \ том~13\ \ \ выпуск~4\ \ \ 2019
 \hfill \textbf{\thepage}}}


\noindent
{\tabcolsep=3pt
\begin{tabular}{p{394pt}cc}
&\textbf{Вып.} & \textbf{Стр.}\\[3pt]
\Avtors{Гольская~А.\,А.} см.\ Маркова~Е.\,В.&&\\
\Avtors{Гончаров~А.\,А., Зацман~И.\,М., Кружков~М.\,Г.} Темпоральные данные в~лексикографиче-\linebreak
\\[-12pt]
\hspace*{23pt}ских базах знаний&4&90--96\\
\Avtors{Гончаров~А.\,А., Инькова~О.\,Ю.} Методика поиска имплицитных логико-семантических\linebreak
\\[-12pt]
\hspace*{23pt}отношений в~тексте&3&\hphantom{1}97--104\\
\Avtors{Горбунова~А.\,В., Наумов~В.\,А., Гайдамака~Ю.\,В., Самуйлов~К.\,Е.} Ресурсные системы\linebreak
\\[-12pt]
\hspace*{23pt}массового обслуживания с~произвольным обслуживанием&1&\hphantom{1}99--107\\
\Avtors{Горшенин~А.\,К., Кузьмин~В.\,Ю.} Оптимизация гиперпараметров нейронных сетей с~ис-\linebreak
\\[-12pt]
\hspace*{23pt}пользованием высокопроизводительных вычислений для~предсказания осадков&1&75--81\\
\Avtors{Горшенин~А.\,К., Кузьмин~В.\,Ю.} Применение рекуррентных нейронных сетей для\linebreak
\\[-12pt]
\hspace*{23pt}прогнозирования моментов конечных нормальных смесей&3&114--121\\
\Avtors{Горшенин~А.\,К., Мартынов~О.\,П.} Гибридные модели экстремального градиентного\linebreak
\\[-12pt]
\hspace*{23pt}бустинга для восстановления пропущенных значений в~данных об~осадках&3&34--40\\
\Avtors{Грабовой~А.\,В., Бахтеев~О.\,Ю., Стрижов~В.\,В.} Определение релевантности параметров\linebreak
\\[-12pt]
\hspace*{23pt}нейросети&2&62--70\\
\Avtors{Гринченко~С.\,Н.} О генезисе информационного общества: информатико-кибернетиче-\linebreak
\\[-12pt]
\hspace*{23pt}ское модельное представление&2&100--108\\
\Avtors{Грушо~А.\,А., Грушо~Н.\,А., Тимонина~Е.\,Е.} Использование метаданных для реализации\linebreak
\\[-12pt]
\hspace*{23pt}требований политики безопасности MLS&4&85--89\\
\Avtors{Грушо~А.\,А., Грушо~Н.\,А., Тимонина~Е.\,Е.} Методы выявления <<слабых>> признаков\linebreak
\\[-12pt]
\hspace*{23pt}нарушений информационной безопасности&3&3--8\\
\Avtors{Грушо~А.\,А., Забежайло~М.\,И., Грушо~Н.\,А., Тимонина~Е.\,Е.} Архитектурные решения в~задаче выявления мошенничества при анализе информационных потоков\linebreak
\\[-12pt]
\hspace*{23pt}в~цифровой экономике&2&22--28\\
\Avtors{Грушо~А.\,А., Забежайло~М.\,И., Грушо~Н.\,А., Тимонина~Е.\,Е.} Формирование концептов\linebreak
\\[-12pt]
\hspace*{23pt}на основе малых выборок&4&81--84\\
\Avtors{Грушо~Н.\,А.} см.\ Грушо~А.\,А.&&\\
\Avtors{Грушо~Н.\,А.} см.\ Грушо~А.\,А.&&\\
\Avtors{Грушо~Н.\,А.} см.\ Грушо~А.\,А.&&\\
\Avtors{Грушо~Н.\,А.} см.\ Грушо~А.\,А.&&\\
\Avtors{Гудкова~И.\,А.} см.\ Маркова~Е.\,В.&&\\
\Avtors{Дзантиев~И.\,Л.} см.\ Маркова~Е.\,В.&&\\
\Avtors{Докукин~А.\,А.} см.\ Журавлев~Ю.\,И.&&\\
\Avtors{Дулин~С.\,К., Дулина~Н.\,Г., Кожунова~О.\,С.} Синтез геоданных в пространственных\linebreak
\\[-12pt]
\hspace*{23pt}инфраструктурах на~основе связанных данных&1&82--90\\
\Avtors{Дулина~Н.\,Г.} см.\ Дулин~С.\,К.&&\\
\Avtors{Дюкова~Е.\,В., Масляков~Г.\,О., Прокофьев~П.\,А.} О числе максимальных независимых\linebreak
\\[-12pt]
\hspace*{23pt}элементов частичных порядков (случай цепей)&1&25--32\\
\Avtors{Журавлев~Ю.\,И., Сенько~О.\,В., Бондаренко~Н.\,Н., Рязанов~В.\,В., Докукин~А.\,А., Виноградов~А.\,П.} Исследование возможности прогнозирования изменения финансового\linebreak
\\[-12pt]
\hspace*{23pt}состояния кредитной организации на основе публикуемой отчетности&4&30--35\\
\Avtors{Забежайло~М.\,И.} см.\ Грушо~А.\,А.&&\\
\Avtors{Забежайло~М.\,И.} см.\ Грушо~А.\,А.&&\\
\Avtors{Захарова~Т.\,В., Тархов~А.\,А.} Оценка уровня значимости критерия Шуирманна для\linebreak
\\[-12pt]
\hspace*{23pt}проверки гипотезы биоэквивалентности при наличии пропущенных данных&3&58--62\\
\Avtors{Зацаринный~А.\,А., Коротков~В.\,В., Матвеев~М.\,Г.} Моделирование процессов сетевого планирования портфеля проектов с~неоднородными ресурсами в~условиях нечет-\linebreak
\\[-12pt]
\hspace*{23pt}кой информации&2&92--99\\
\Avtors{Зацман~И.\,М.} Интерфейсы третьего порядка в~информатике&3&82--89\\
\Avtors{Зацман~И.\,М.} Кодирование концептов в~цифровой среде&4&\hphantom{1}97--106\\
\Avtors{Зацман~И.\,М.} Целенаправленное развитие систем лингвистических знаний: выявление\linebreak
\\[-12pt]
\hspace*{23pt}и~заполнение лакун&1&91--98\\
\Avtors{Зацман~И.\,М.} см.\ Гончаров~А.\,А.&&\\
\end{tabular}
}

\pagebreak

\def\leftkol{АВТОРСКИЙ УКАЗАТЕЛЬ ЗА 2019 г.} % ENGLISH ABSTRACTS}

\def\rightkol{АВТОРСКИЙ УКАЗАТЕЛЬ ЗА 2019 г.} %ENGLISH ABSTRACTS}

%\thispagestyle{myheadings}
\def\leftfootline{\small{\textbf{\thepage}
\hfill ИНФОРМАТИКА И ЕЁ ПРИМЕНЕНИЯ\ \ \ том~13\ \ \ выпуск~4\ \ \ 2019}
}%
 \def\rightfootline{\small{ИНФОРМАТИКА И ЕЁ ПРИМЕНЕНИЯ\ \ \ том~13\ \ \ выпуск~4\ \ \ 2019
 \hfill \textbf{\thepage}}}


\noindent
{\tabcolsep=3pt
\begin{tabular}{p{394pt}cc}
&\textbf{Вып.} & \textbf{Стр.}\\[3pt]
\Avtors{Зейфман~А.\,И., Сатин~Я.\,А., Киселева~К.\,М.} Об оценках скорости сходимости для некоторых моделей массового обслуживания с~неполно заданными интенсивно-\linebreak
\\[-12pt]
\hspace*{23pt}стями&3&14--19\\
\Avtors{Инькова~О.\,Ю., Кружков~М.\,Г.} Сочетаемость логико-семантических отношений: коли-\linebreak
\\[-12pt]
\hspace*{23pt}чественные методы анализа&2&83--91\\
\Avtors{Инькова~О.\,Ю.} см.\ Гончаров~А.\,А.&&\\
\Avtors{Кириков~И.\,А.} см.\ Румовская~С.\,Б.&&\\
\Avtors{Киселева~К.\,М.} см.\ Зейфман~А.\,И.&&\\
\Avtors{Ковалёв~Д.\,Ю., Тарасов~Е.\,А.} Виртуальные эксперименты в~исследованиях с~интенсив-\linebreak
\\[-12pt]
\hspace*{23pt}ным использованием данных&2&117--125\\
\Avtors{Кожунова~О.\,С.} см.\ Дулин~С.\,К.&&\\
\Avtors{Колесников~А.\,В., Листопад~С.\,В.} Протокол гетерогенного мышления гибридной интеллектуальной многоагентной системы для решения проблемы восстановления\linebreak
\\[-12pt]
\hspace*{23pt}распределительной электросети&2&76--82\\
\Avtors{Коновалов~М.\,Г., Разумчик~Р.\,В.} Комплексное управление в~одном классе систем\linebreak
\\[-12pt]
\hspace*{23pt}с~параллельным обслуживанием&4&54--59\\
\Avtors{Коновалов~М.\,Г.} см.\ Агаларов~Я.\,М.&&\\
\Avtors{Коротков~В.\,В.} см.\ Зацаринный~А.\,А.&&\\
\Avtors{Кривенко~М.\,П.} Выбор модели данных в~задачах медицинской диагностики&4&27--29\\
\Avtors{Кружков~М.\,Г.} см.\ Гончаров~А.\,А.&&\\
\Avtors{Кружков~М.\,Г.} см.\ Инькова~О.\,Ю.&&\\
\Avtors{Кудрявцев~А.\,А.} Априорное обобщенное гамма-распределение в~байесовских моделях\linebreak
\\[-12pt]
\hspace*{23pt}баланса&3&27--33\\
\Avtors{Кудрявцев~А.\,А.} О представлении 
гамма-экспоненциального и~обобщенного отрица-\linebreak
\\[-12pt]
\hspace*{23pt}тельного биномиального распределений&4&76--80\\
\Avtors{Кудрявцев~А.\,А., Палионная~С.\,И., Шоргин~В.\,С.} Априорные Фреше и масштабированное\linebreak
\\[-12pt]
\hspace*{23pt}обратное хи-распределение в~байесовских моделях баланса&1&62--66\\
\Avtors{Кудрявцев~А.\,А.} см.\ Арутюнов~Е.\,Н.&&\\
\Avtors{Кузьмин~В.\,Ю.} см.\ Горшенин~А.\,К.&&\\
\Avtors{Кузьмин~В.\,Ю.} см.\ Горшенин~А.\,К.&&\\
\Avtors{Курьянский~М.\,К.} см.\ Вышинский~Л.\,Л.&&\\
\Avtors{Ланге~M.\,M.} О~сравнительной эффективности схем классификации данных на~ансамбле\linebreak
\\[-12pt]
\hspace*{23pt}источников с~использованием средней взаимной информации&4&18--26\\
\Avtors{Лебедев~А.\,В.} Нетранзитивные триплеты непрерывных случайных величин и~их прило-\linebreak
\\[-12pt]
\hspace*{23pt}жения&3&20--26\\
\Avtors{Листопад~С.\,В.} см.\ Колесников~А.\,В.&&\\
\Avtors{Логачев~О.\,А., Сукаев~А.\,А., Федоров~С.\,Н.} Об одном методе решения систем 
квад\-ра\-тич\-ных булевых уравнений, использующем локальные аффинности булевых\linebreak
\\[-12pt]
\hspace*{23pt}функций&2&37--46\\
\Avtors{Логачев~О.\,А., Сукаев~А.\,А., Федоров~С.\,Н.} Полиномиальные алгоритмы вычисления\linebreak
\\[-12pt]
\hspace*{23pt}локальных аффинностей квадратичных булевых функций&1&67--74\\
\Avtors{Лукашенко~О.\,В., Морозов~Е.\,В., Пагано~М.} Гауссовская аппроксимация процесса\linebreak
\\[-12pt]
\hspace*{23pt}распределенных вычислений&2&109--116\\
\Avtors{Малашенко~Ю.\,Е., Назарова~И.\,А., Новикова~Н.\,М.} Анализ уязвимости многополюсных\linebreak
\\[-12pt]
\hspace*{23pt}сетей при~структурных повреждениях&1&33--39\\
\Avtors{Маркова~Е.\,В., Гольская~А.\,А., Дзантиев~И.\,Л., Гудкова~И.\,А., Шоргин~С.\,Я.} Сравнительный анализ показателей эффективности модели беспроводной сети меж\-ма\-шин\-ного взаимодействия, работающей в~рамках двух политик разделения радиоре-\linebreak
\\[-12pt]
\hspace*{23pt}сурсов&1&108--116\\
\Avtors{Мартынов~О.\,П.} см.\ Горшенин~А.\,К.&&\\
\Avtors{Масляков~Г.\,О.} см.\ Дюкова~Е.\,В.&&\\
\Avtors{Матвеев~М.\,Г.} см.\ Зацаринный~А.\,А.&&\\
\Avtors{Мейханаджян~Л.\,А., Разумчик~Р.\,В.} Система массового обслуживания Geo$/G/1/\infty$\linebreak
\\[-12pt]
\hspace*{23pt}синверсионным порядком обслуживания и~ресамплингом в~дискретном времени&4&60--67\\
\end{tabular}
}

\pagebreak

\def\leftkol{АВТОРСКИЙ УКАЗАТЕЛЬ ЗА 2019 г.} % ENGLISH ABSTRACTS}

\def\rightkol{АВТОРСКИЙ УКАЗАТЕЛЬ ЗА 2019 г.} %ENGLISH ABSTRACTS}

%\thispagestyle{myheadings}
\def\leftfootline{\small{\textbf{\thepage}
\hfill ИНФОРМАТИКА И ЕЁ ПРИМЕНЕНИЯ\ \ \ том~13\ \ \ выпуск~4\ \ \ 2019}
}%
 \def\rightfootline{\small{ИНФОРМАТИКА И ЕЁ ПРИМЕНЕНИЯ\ \ \ том~13\ \ \ выпуск~4\ \ \ 2019
 \hfill \textbf{\thepage}}}


\noindent
{\tabcolsep=3pt
\begin{tabular}{p{394pt}cc}
&\textbf{Вып.} & \textbf{Стр.}\\[3pt]
\Avtors{Мельников~А.\,В.} см.\ Бурлуцкий~В.\,В.&&\\
\Avtors{Миллер~Г.\,Б.} см.\ Босов~А.\,В.&&\\
\Avtors{Морозов~Е.\,В.} см.\ Лукашенко~О.\,В.&&\\
\Avtors{Мхитарян~Г.\,А.} см.\ Босов~А.\,В.&&\\
\Avtors{Назарова~И.\,А.} см.\ Малашенко~Ю.\,Е.&&\\
\Avtors{Наумов~А.\,В.} см.\ Босов~А.\,В.&&\\
\Avtors{Наумов~В.\,А.} см.\ Горбунова~А.\,В.&&\\
\Avtors{Новикова~Н.\,М.} см.\ Малашенко~Ю.\,Е.&&\\
\Avtors{Нуриев~В.\,А.} Архитектура системы нейронного машинного перевода&3&90--96\\
\Avtors{Осипова~В.\,А.} см.\ Абгарян~К.\,К.&&\\
\Avtors{Павлов~Ю.\,Л.} Об асимптотике кластерного коэффициента конфигурационного графа\linebreak
\\[-12pt]
\hspace*{23pt}с~неизвестным распределением степеней вершин&3&\hphantom{1}9--13\\
\Avtors{Пагано~М.} см.\ Лукашенко~О.\,В.&&\\
\Avtors{Палионная~С.\,И.} см.\ Кудрявцев~А.\,А.&&\\
\Avtors{Панов~А.\,И.} см.\ Смирнов~И.\,В.&&\\
\Avtors{Пенкин~Г.\,О.} см.\ Аникеев~Д.\,А.&&\\
\Avtors{Прокофьев~П.\,А.} см.\ Дюкова~Е.\,В.&&\\
\Avtors{Разумчик~Р.\,В.} см.\ Коновалов~М.\,Г.&&\\
\Avtors{Разумчик~Р.\,В.} см.\ Мейханаджян~Л.\,А.&&\\
\Avtors{Румовская~С.\,Б., Кириков~И.\,А.} Методы моделирования и~визуального представления\linebreak
\\[-12pt]
\hspace*{23pt}конфликта в~малом коллективе экспертов, решающих проблемы (обзор)&3&122--130\\
\Avtors{Рыбаков~К.\,А.} Об одном классе задач фильтрации на многообразиях&1&16--24\\
\Avtors{Рязанов~В.\,В.} см.\ Журавлев~Ю.\,И.&&\\
\Avtors{Самуйлов~К.\,Е.} см.\ Гайдамака~А.\,А.&&\\
\Avtors{Самуйлов~К.\,Е.} см.\ Горбунова~А.\,В.&&\\
\Avtors{Сапунова~А.\,П.} см.\ Босов~А.\,В.&&\\
\Avtors{Сатин~Я.\,А.} см.\ Зейфман~А.\,И.&&\\
\Avtors{Сейфуль-Мулюков~Р.\,Б.} Законы информатики и~синергетики в~познании сложных\linebreak
\\[-12pt]
\hspace*{23pt}систем&4&107--113\\
\Avtors{Сенько~О.\,В.} см.\ Журавлев~Ю.\,И.&&\\
\Avtors{Синицын~И.\,Н.} Интерполяционное аналитическое моделирование распределений\linebreak
\\[-12pt]
\hspace*{23pt}в~сложных стохастических системах&1&2--8\\
\Avtors{Скрынник~А.\,А.} см.\ Смирнов~И.\,В.&&\\
\Avtors{Смирнов~И.\,В., Панов~А.\,И., Скрынник~А.\,А., Чистова~Е.\,В.} Персональный когнитивный\linebreak
\\[-12pt]
\hspace*{23pt}ассистент: концепция и~принципы работы&3&105--113\\
\Avtors{Стефанович~А.\,И.} см.\ Босов~А.\,В.&&\\
\Avtors{Стефанович~А.\,И.} см.\ Босов~А.\,В.&&\\
\Avtors{Стрижов~В.\,В.} см.\ Аникеев~Д.\,А.&&\\
\Avtors{Стрижов~В.\,В.} см.\ Грабовой~А.\,В.&&\\
\Avtors{Сукаев~А.\,А.} см.\ Логачев~О.\,А.&&\\
\Avtors{Сукаев~А.\,А.} см.\ Логачев~О.\,А.&&\\
\Avtors{Сучков~А.\,П.} Научный результат как информационный объект в~контексте системы\linebreak
\\[-12pt]
\hspace*{23pt}управления научными сервисами&3&137--144\\
\Avtors{Тарасов~Е.\,А.} см.\ Ковалёв~Д.\,Ю.&&\\
\Avtors{Тархов~А.\,А.} см.\ Захарова~Т.\,В.&&\\
\Avtors{Тимонина~Е.\,Е.} см.\ Грушо~А.\,А.&&\\
\Avtors{Тимонина~Е.\,Е.} см.\ Грушо~А.\,А.&&\\
\Avtors{Тимонина~Е.\,Е.} см.\ Грушо~А.\,А.&&\\
\Avtors{Тимонина~Е.\,Е.} см.\ Грушо~А.\,А.&&\\
\Avtors{Титова~А.\,И.} см.\ Арутюнов~Е.\,Н.&&\\
\Avtors{Ушаков~В.\,Г., Ушаков~Н.\,Г.} Выходящие потоки в~однолинейной системе с~относитель-\linebreak
\\[-12pt]
\hspace*{23pt}ным приоритетом&4&42--47\\
\Avtors{Ушаков~В.\,Г.} см.\ Агаларов~Я.\,М.&&\\
\Avtors{Ушаков~Н.\,Г.} см.\ Ушаков~В.\,Г.&&\\
\end{tabular}
}

\pagebreak

\def\leftkol{АВТОРСКИЙ УКАЗАТЕЛЬ ЗА 2019 г.} % ENGLISH ABSTRACTS}

\def\rightkol{АВТОРСКИЙ УКАЗАТЕЛЬ ЗА 2019 г.} %ENGLISH ABSTRACTS}

%\thispagestyle{myheadings}
\def\leftfootline{\small{\textbf{\thepage}
\hfill ИНФОРМАТИКА И ЕЁ ПРИМЕНЕНИЯ\ \ \ том~13\ \ \ выпуск~4\ \ \ 2019}
}%
 \def\rightfootline{\small{ИНФОРМАТИКА И ЕЁ ПРИМЕНЕНИЯ\ \ \ том~13\ \ \ выпуск~4\ \ \ 2019
 \hfill \textbf{\thepage}}}


\noindent
{\tabcolsep=3pt
\begin{tabular}{p{394pt}cc}
&\textbf{Вып.} & \textbf{Стр.}\\[3pt]
\Avtors{Федоров~С.\,Н.} см.\ Логачев~О.\,А.&&\\
\Avtors{Федоров~С.\,Н.} см.\ Логачев~О.\,А.&&\\
\Avtors{Флеров~Ю.\,А.} см.\ Вышинский~Л.\,Л.&&\\
\Avtors{Царегородцев~А.\,Л.} см.\ Бурлуцкий~В.\,В.&&\\
\Avtors{Чистова~Е.\,В.} см.\ Смирнов~И.\,В.&&\\
\Avtors{Чухно~Н.\,В.} см.\ Гайдамака~А.\,А.&&\\
\Avtors{Чухно~О.\,В.} см.\ Гайдамака~А.\,А.&&\\
\Avtors{Шестаков~О.\,В.} Обращение однородных операторов с помощью стабилизированной\linebreak
\\[-12pt]
\hspace*{23pt}жесткой пороговой обработки при неизвестной дисперсии шума&1&49--54\\
\Avtors{Шестаков~О.\,В.} Свойства вейвлет-оценок сигналов, регистрируемых в~случайные\linebreak
\\[-12pt]
\hspace*{23pt}моменты времени&2&16--21\\
\Avtors{Шестаков~О.\,В.} Среднеквадратичный риск нелинейной регуляризации задачи обраще-\linebreak
\\[-12pt]
\hspace*{23pt}ния линейных однородных операторов при случайном объеме выборки&4&48--53\\
\Avtors{Шнурков~П.\,В., Вахтанов~Н.\,А.} Исследование проблемы оптимального управления запасом дискретного продукта в~стохастической модели регенерации с~непрерывно\linebreak
\\[-12pt]
\hspace*{23pt}происходящим потреблением и~случайной задержкой поставки&2&54--61\\
\Avtors{Шнурков~П.\,В., Вахтанов~Н.\,А.} О~решении проблемы оптимального управления запасом дискретного продукта в~стохастической модели регенерации с непрерывно\linebreak
\\[-12pt]
\hspace*{23pt}происходящим потреблением&3&50--57\\
\Avtors{Шоргин~В.\,С.} см.\ Кудрявцев~А.\,А.&&\\
\Avtors{Шоргин~С.\,Я.} см.\ Гайдамака~А.\,А.&&\\
\Avtors{Шоргин~С.\,Я.} см.\ Маркова~Е.\,В.&&\\
\Avtors{Якимчук~А.\,В.} см.\ Бурлуцкий~В.\,В.&&\\
\end{tabular}
}

%\thispagestyle{myheadings}
\def\leftfootline{\small{\textbf{\thepage}
\hfill ИНФОРМАТИКА И ЕЁ ПРИМЕНЕНИЯ\ \ \ том~13\ \ \ выпуск~4\ \ \ 2019}
}%
 \def\rightfootline{\small{ИНФОРМАТИКА И ЕЁ ПРИМЕНЕНИЯ\ \ \ том~13\ \ \ выпуск~4\ \ \ 2019
 \hfill \textbf{\thepage}}}

 \label{end\stat}

\newpage

\def\stat{cont-e}
{%\hrule\par
%\vskip 7pt % 7pt
\raggedleft\Large \bf%\baselineskip=3.2ex
2\,0\,1\,9\ \ A\,U\,T\,H\,O\,R\ \ I\,N\,D\,E\,X \vskip 17pt
 \hrule
 \par
\vskip 21pt plus 6pt minus 3pt }

\label{st\stat}

\def\tit{\ }

\def\aut{\ }
\def\auf{\ }

\def\leftkol{\ } %2019 AUTHOR INDEX} % ENGLISH ABSTRACTS}

\def\rightkol{\ } %2019 AUTHOR INDEX} %ENGLISH ABSTRACTS}

\titele{\tit}{\aut}{\auf}{\leftkol}{\rightkol}
\addcontentsline{toc}{subsection}{\textrm\textbf 2019 Author Index}

\def\leftfootline{\small{\textbf{\thepage}
\hfill INFORMATIKA I EE PRIMENENIYA~--- INFORMATICS AND APPLICATIONS\ \ \ 2019\
\ \ volume~13\ \ \ issue\ 4}
}%
 \def\rightfootline{\small{INFORMATIKA I EE PRIMENENIYA~--- INFORMATICS AND APPLICATIONS\ \ \ 2019\ \ \ volume~13\ \ \ issue\ 4
\hfill \textbf{\thepage}}}

%\vspace*{-12pt}

\noindent
{\tabcolsep=3pt
\begin{tabular}{p{396pt}cc}
&\textbf{Issue} & \textbf{Page}\\[6pt]
\Avtors{Abgaryan~K.\,K.\ and Osipova~V.\,A.} Application of decision support methods for the multicriterial\linebreak
\\[-12pt]
\hspace*{23pt}selection of multiscale compositions&2&47--53\\
\Avtors{Agalarov~Ya.\,M.\ and Konovalov~M.\,G.} Proof of the unimodality of the objective function in\linebreak
\\[-12pt]
\hspace*{23pt}$M/M/N$ queue with threshold-based congestion control&2&2--6\\
\Avtors{Agalarov~Ya.\,M.\ and Ushakov~V.\,G.} On the unimodality of the~income function of a~type $G|M|s$\linebreak
\\[-12pt]
\hspace*{23pt}queueing system with controlled queue&1&55--61\\
\Avtors{Agasandyan~G.\,A.} Performance estimations for optimal-on-CC-VaR portfolios in option markets&3&72--81\\
\Avtors{Agasandyan~G.\,A.} Theoretical foundations of~continuous VaR criterion optimization in~the~col-\linebreak
\\[-12pt]
\hspace*{23pt}lection of~markets&4&36--41\\
\Avtors{Anashin~V.\,S.} On automata models of blockchain&2&29--36\\
\Avtors{Anikeyev~D.\,A., Penkin~G.\,O., and Strijov~V.\,V.} Local approximation models for~human physical\linebreak
\\[-12pt]
\hspace*{23pt}activity classification&1&40--48\\
\Avtors{Arutyunov~E.\,N., Kudryavtsev~A.\,A., and Titova~A.\,I.} Bayesian models of factors balance with\linebreak
\\[-12pt]
\hspace*{23pt}\textit{a~priori} Weibull and Nakagami distributions&2&71--75\\
\Avtors{Bakhteev~O.\,Yu.} see Grabovoy~A.\,V.&&\\
\Avtors{Bondarenko~N.\,N.} see Zhuravlev~Yu.\,I.&&\\
\Avtors{Borisov~A.\,V.} Numerical schemes of markov jump process filtering given discretized observa-\linebreak
\\[-12pt]
\hspace*{23pt}tions~I:~Accuracy characteristics&4&68--75\\
\Avtors{Bosov~A.\,V.\ and Miller~G.\,B.} On the conditionally minimax nonlinear filtering concept\linebreak
\\[-12pt]
\hspace*{23pt}development: Filter modification and analysis&2&\hphantom{1}7--15\\
\Avtors{Bosov~A.\,V., Naumov~A.\,V., Mkhitaryan~G.\,A., and Sapunova~A.\,P.} Using the model of~gamma\linebreak
\\[-12pt]
\hspace*{23pt}distribution in~the~problem of~forming a~time-limited test in~a~distance learning system&4&11--17\\
\Avtors{Bosov~A.\,V.\ and Stefanovich~A.\,I.} Stochastic differential system output control by~the~quadratic\linebreak
\\[-12pt]
\hspace*{23pt}criterion. II.~Dynamic programming equations numerical solution&1&\hphantom{1}9--15\\
\Avtors{Bosov~A.\,V.\ and Stefanovich~A.\,I.} Stochastic differential system output control by~the~quadratic\linebreak
\\[-12pt]
\hspace*{23pt}criterion. III.~Optimal control properties analysis&3&41--49\\

\Avtors{Burlutskiy~V.\,V., Yakimchuk~A.\,V., Melnikov~A.\,V., Tsaregorodtsev~A.\,L., and Voloshin~S.\,V.} Development of a method for the formation of~attribute space and a~model for~the~assessment and prediction of anthropogenic influence on~the~environment (on~the~example of~the~forest fund of the~oil-producing region)&3& 131--136\\
\Avtors{Chistova~E.\,V.} see Smirnov~I.\,V.&&\\
\Avtors{Chukhno~N.\,V.} see Gaidamaka~A.\,A.&&\\
\Avtors{Chukhno~O.\,V.} see Gaidamaka~A.\,A.&&\\
\Avtors{Djukova~E.\,V., Maslyakov~G.\,O., and Prokofyev~P.\,A.} On the number of maximal independent\linebreak
\\[-12pt]
\hspace*{23pt}elements of~partially ordered sets (the case of~chains)&1&25--32\\
\Avtors{Dokukin~A.\,A.} see Zhuravlev~Yu.\,I.&&\\
\Avtors{Dulin~S.\,K., Dulina~N.\,G., and Kozhunova~O.\,S.} Synthesis of geodata in spatial infrastructures\linebreak
\\[-12pt]
\hspace*{23pt}based on related data&1&82--90\\
\Avtors{Dulina~N.\,G.} see Dulin~S.\,K.&&\\
\Avtors{Dzantiev~I.\,L.} see Markova~E.\,V.&&\\
\Avtors{Fedorov~S.\,N.} see Logachev~O.\,A.&&\\
\Avtors{Fedorov~S.\,N.} see Logachev~O.\,A.&&\\
\end{tabular}
}
\pagebreak

\def\leftfootline{\small{\textbf{\thepage}
\hfill INFORMATIKA I EE PRIMENENIYA~--- INFORMATICS AND APPLICATIONS\ \ \ 2019\
\ \ volume~13\ \ \ issue\ 4}
}%
 \def\rightfootline{\small{INFORMATIKA I EE PRIMENENIYA~---
INFORMATICS AND APPLICATIONS\ \ \ 2019\ \ \ volume~13\ \ \ issue\ 4
\hfill \textbf{\thepage}}}

\def\leftkol{2019 AUTHOR INDEX} % ENGLISH ABSTRACTS}

\def\rightkol{2019 AUTHOR INDEX} %ENGLISH ABSTRACTS}


\noindent
{\tabcolsep=3pt
\begin{tabular}{p{395.48108pt}cc}
&\textbf{Issue} & \textbf{Page}\\[6pt]
\Avtors{Flerov~Yu.\,A.} see Vyshinsky~L.\,L.&&\\
\Avtors{Gaidamaka~A.\,A., Chukhno~N.\,V., Chukhno~O.\,V., Samouylov~K.\,E., and Shorgin~S.\,Ya.} Formalization of the alternatives ranking method for group decision making in social net-\linebreak
\\[-12pt]
\hspace*{23pt}works&3&63--71\\
\Avtors{Gaidamaka~Yu.\,V.} see Gorbunova~A.\,V.&&\\
\Avtors{Golskaia~A.\,A.} see Markova~E.\,V.&&\\
\Avtors{Goncharov~A.\,A.\ and Inkova~O.\,Yu.} Methods for identification of implicit logical-semantic\linebreak
\\[-12pt]
\hspace*{23pt}relations in~texts&3&\hphantom{1}97--104\\
\Avtors{Goncharov~A.\,A., Zatsman~I.\,M., and Kruzhkov~M.\,G.} Temporal data in~lexicographic databases&4&90--96\\
\Avtors{Gorbunova~A.\,V., Naumov~V.\,A., Gaidamaka~Yu.\,V., and Samouylov~K.\,E.} Resource queuing\linebreak
\\[-12pt]
\hspace*{23pt}systems with general service discipline&1&\hphantom{1}99--107\\
\Avtors{Gorshenin~A.\,K.\ and Kuzmin~V.\,Yu.} Application of recurrent neural networks to~forecasting\linebreak
\\[-12pt]
\hspace*{23pt}the~moments of~finite normal mixtures&3&114--121\\
\Avtors{Gorshenin~A.\,K.\ and Kuzmin~V.\,Yu.} Optimization of hyperparameters of neural networks using\linebreak
\\[-12pt]
\hspace*{23pt}high-performance computing for prediction of precipitation&1&75--81\\
\Avtors{Gorshenin~A.\,K.\ and Martynov~O.\,P.} Hybrid extreme gradient boosting models to~impute\linebreak
\\[-12pt]
\hspace*{23pt}the~missing data in~precipitation records&3&34--40\\
\Avtors{Grabovoy~A.\,V., Bakhteev~O.\,Yu., and Strijov~V.\,V.} Estimation of the relevance of the neural\linebreak
\\[-12pt]
\hspace*{23pt}network parameters&2&62--70\\
\Avtors{Grinchenko~S.\,N.} On the genesis of the information society: Informatics-cybernetic model\linebreak
\\[-12pt]
\hspace*{23pt}representation&2&100--108\\
\Avtors{Grusho~A.\,A., Grusho~N.\,A., and Timonina~E.\,E.} Methods of identification of ``weak'' signs of\linebreak
\\[-12pt]
\hspace*{23pt}violations of information security&3&3--8\\
\Avtors{Grusho~A.\,A., Grusho~N.\,A., and Timonina~E.\,E.} Using metadata to~implement multilevel security\linebreak
\\[-12pt]
\hspace*{23pt}policy requirements&4&85--89\\
\Avtors{Grusho~A.\,A., Zabezhailo~M.\,I., Grusho~N.\,A., and Timonina~E.\,E.} Architectural decisions in the problem of identification of~fraud in~the~analysis of~information flows in~digital eco-\linebreak
\\[-12pt]
\hspace*{23pt}nomy&2&22--28\\
\Avtors{Grusho~A.\,A., Zabezhailo~M.\,I., Grusho~N.\,A., and Timonina~E.\,E.} Concepts forming on~the~basis\linebreak
\\[-12pt]
\hspace*{23pt}of~small samples&4&81--84\\
\Avtors{Grusho~N.\,A.} see Grusho~A.\,A.&&\\
\Avtors{Grusho~N.\,A.} see Grusho~A.\,A.&&\\
\Avtors{Grusho~N.\,A.} see Grusho~A.\,A.&&\\
\Avtors{Grusho~N.\,A.} see Grusho~A.\,A.&&\\
\Avtors{Gudkova~I.\,A.} see Markova~E.\,V.&&\\
\Avtors{Inkova~O.\,Yu.\ and Kruzhkov~M.\,G.} Compatibility of logical semantic relations: Methods\linebreak
\\[-12pt]
\hspace*{23pt}of~quantitative analysis&2&83--91\\
\Avtors{Inkova~O.\,Yu.} see Goncharov~A.\,A.&&\\
\Avtors{Kirikov~I.\,A.} see Rumovskaya~S.\,B.&&\\
\Avtors{Kiseleva~K.\,M.} see Zeifman~A.\,I.&&\\
\Avtors{Kolesnikov~A.\,V.\ and Listopad~S.\,V.} Heterogeneous thinking protocol of hybrid intelligent\linebreak
\\[-12pt]
\hspace*{23pt}multiagent system for~solving distributional power grid recovery problem&2&76--82\\
\Avtors{Konovalov~M.\,G.\ and Razumchik~R.\,V.} Mixed policies for~online job allocation in~one class\linebreak
\\[-12pt]
\hspace*{23pt}of~systems with~parallel service&4&54--59\\
\Avtors{Konovalov~M.\,G.} see Agalarov~Ya.\,M.&&\\
\Avtors{Korotkov~V.\,V.} see Zatsarinny~A.\,A.&&\\
\Avtors{Kovalev~D.\,Y.\ and Tarasov~E.\,A.} Virtual experiments in data intensive research&2&117--125\\
\Avtors{Kozhunova~O.\,S.} see Dulin~S.\,K.&&\\
\Avtors{Krivenko~M.\,P.} Data model selection in~medical diagnostic tasks&4&27--29\\
\Avtors{Kruzhkov~M.\,G.} see Goncharov~A.\,A.&&\\
\Avtors{Kruzhkov~M.\,G.} see Inkova~O.\,Yu.&&\\
\Avtors{Kudryavtsev~A.\,A.} \textit{A priori} generalized gamma distribution in Bayesian balance models&3&27--33\\
\Avtors{Kudryavtsev~A.\,A.} On the representation of gamma-exponential and~generalized negative\linebreak
\\[-12pt]
\hspace*{23pt}binomial distributions&4&76--80\\
\end{tabular}
}
\pagebreak

\def\leftfootline{\small{\textbf{\thepage}
\hfill INFORMATIKA I EE PRIMENENIYA~--- INFORMATICS AND APPLICATIONS\ \ \ 2019\
\ \ volume~13\ \ \ issue\ 4}
}%
 \def\rightfootline{\small{INFORMATIKA I EE PRIMENENIYA~---
INFORMATICS AND APPLICATIONS\ \ \ 2019\ \ \ volume~13\ \ \ issue\ 4
\hfill \textbf{\thepage}}}

\def\leftkol{2019 AUTHOR INDEX} % ENGLISH ABSTRACTS}

\def\rightkol{2019 AUTHOR INDEX} %ENGLISH ABSTRACTS}


\noindent
{\tabcolsep=3pt
\begin{tabular}{p{395.48108pt}cc}
&\textbf{Issue} & \textbf{Page}\\[6pt]
\Avtors{Kudryavtsev~A.\,A., Palionnaia~S.\,I., and Shorgin~V.\,S.} \textit{A priori} Frechet and~scaled inverse chi\linebreak
\\[-12pt]
\hspace*{23pt}distribution in~Bayesian balance models&1&62--66\\
\Avtors{Kudryavtsev~A.\,A.} see Arutyunov~E.\,N.&&\\
\Avtors{Kuryansky~M.\,K.} see Vyshinsky~L.\,L.&&\\
\Avtors{Kuzmin~V.\,Yu.} see Gorshenin~A.\,K.&&\\
\Avtors{Kuzmin~V.\,Yu.} see Gorshenin~A.\,K.&&\\
\Avtors{Lange~M.\,M.} On comparative efficiency of classification schemes in an ensemble of data\linebreak
\\[-12pt]
\hspace*{23pt}sources using average mutual information&4&18--26\\
\Avtors{Lebedev~A.\,V.} Nontransitive triplets of continuous random variables and their applications&3&20--26\\
\Avtors{Listopad~S.\,V.} see Kolesnikov~A.\,V.&&\\
\Avtors{Logachev~O.\,A., Sukayev~A.\,A., and Fedorov~S.\,N.} On local affinity based method of solving\linebreak
\\[-12pt]
\hspace*{23pt}systems of quadratic Boolean equations&2&37--46\\
\Avtors{Logachev~O.\,A., Sukayev~A.\,A., and Fedorov~S.\,N.} Polynomial algorithms for~constructing local\linebreak
\\[-12pt]
\hspace*{23pt}affinities of~quadratic Boolean functions&1&67--74\\
\Avtors{Lukashenko~O.\,V., Morozov~E.\,V., and Pagano~M.} A~Gaussian approximation of~the~distributed\linebreak
\\[-12pt]
\hspace*{23pt}computing process&2&109--116\\
\Avtors{Malashenko~Yu.\,E., Nazarova~I.\,A., and Novikova~N.\,M.} Vulnerability analysis of multipolar\linebreak
\\[-12pt]
\hspace*{23pt}networks after structural damages&1&33--39\\
\Avtors{Markova~E.\,V., Golskaia~A.\,A., Dzantiev~I.\,L., Gudkova~I.\,A., and Shorgin~S.\,Ya.} Comparative analysis of performance measures for a wireless machine-to-machine network model\linebreak
\\[-12pt]
\hspace*{23pt}operating within two radio resource management policies&1&108--116\\
\Avtors{Martynov~O.\,P.} see Gorshenin~A.\,K.&&\\
\Avtors{Maslyakov~G.\,O.} see Djukova~E.\,V.&&\\
\Avtors{Matveev~M.\,G.} see Zatsarinny~A.\,A.&&\\
\Avtors{Melnikov~A.\,V.} see Burlutskiy~V.\,V.&&\\
\Avtors{Meykhanadzhyan~L.\,A.\ and Razumchik~R.\,V.} Discrete-time $\mathrm{GEO}/G/1/\infty$ LIFO queue with\linebreak
\\[-12pt]
\hspace*{23pt}resampling policy&4&60--67\\
\Avtors{Miller~G.\,B.} see Bosov~A.\,V.&&\\
\Avtors{Mkhitaryan~G.\,A.} see Bosov~A.\,V.&&\\
\Avtors{Morozov~E.\,V.} see Lukashenko~O.\,V.&&\\
\Avtors{Naumov~A.\,V.} see Bosov~A.\,V.&&\\
\Avtors{Naumov~V.\,A.} see Gorbunova~A.\,V.&&\\
\Avtors{Nazarova~I.\,A.} see Malashenko~Yu.\,E.&&\\
\Avtors{Novikova~N.\,M.} see Malashenko~Yu.\,E.&&\\
\Avtors{Nuriev~V.\,A.} Architecture of a~machine translation system&3&90--96\\
\Avtors{Osipova~V.\,A.} see Abgaryan~K.\,K.&&\\
\Avtors{Pagano~M.} see Lukashenko~O.\,V.&&\\
\Avtors{Palionnaia~S.\,I.} see Kudryavtsev~A.\,A.&&\\
\Avtors{Panov~A.\,I.} see Smirnov~I.\,V.&&\\
\Avtors{Pavlov~Yu.\,L.} On the asymptotics of clustering coefficient in~a~configuration graph with unknown\linebreak
\\[-12pt]
\hspace*{23pt}distribution of~vertex degrees&3&\hphantom{1}9--13\\
\Avtors{Penkin~G.\,O.} see Anikeyev~D.\,A.&&\\
\Avtors{Prokofyev~P.\,A.} see Djukova~E.\,V.&&\\
\Avtors{Razumchik~R.\,V.} see Konovalov~M.\,G.&&\\
\Avtors{Razumchik~R.\,V.} see Meykhanadzhyan~L.\,A.&&\\
\Avtors{Rumovskaya~S.\,B.\ and Kirikov~I.\,A.} Methods of modeling and visual representation of~a~conflict\linebreak
\\[-12pt]
\hspace*{23pt}in~a~small collective of experts solving problems (review)&3&122--130\\
\Avtors{Ryazanov~V.\,V.} see Zhuravlev~Yu.\,I.&&\\
\Avtors{Rybakov~K.\,A.} On a class of filtering problems on~manifolds&1&16--24\\
\Avtors{Samouylov~K.\,E.} see Gaidamaka~A.\,A.&&\\
\Avtors{Samouylov~K.\,E.} see Gorbunova~A.\,V.&&\\
\Avtors{Sapunova~A.\,P.} see Bosov~A.\,V.&&\\
\Avtors{Satin~Y.\,A.} see Zeifman~A.\,I.&&\\
\end{tabular}
}
\pagebreak

\def\leftfootline{\small{\textbf{\thepage}
\hfill INFORMATIKA I EE PRIMENENIYA~--- INFORMATICS AND APPLICATIONS\ \ \ 2019\
\ \ volume~13\ \ \ issue\ 4}
}%
 \def\rightfootline{\small{INFORMATIKA I EE PRIMENENIYA~---
INFORMATICS AND APPLICATIONS\ \ \ 2019\ \ \ volume~13\ \ \ issue\ 4
\hfill \textbf{\thepage}}}

\def\leftkol{2019 AUTHOR INDEX} % ENGLISH ABSTRACTS}

\def\rightkol{2019 AUTHOR INDEX} %ENGLISH ABSTRACTS}


\noindent
{\tabcolsep=3pt
\begin{tabular}{p{395.48108pt}cc}
&\textbf{Issue} & \textbf{Page}\\[6pt]
\Avtors{Sen'ko~O.\,V.} see Zhuravlev~Yu.\,I.&&\\
\Avtors{Seyful-Mulyukov~R.\,B.} Understanding of~complex systems using~the~laws of~synergetics\linebreak
\\[-12pt]
\hspace*{23pt}and~informatics&4&107--113\\
\Avtors{Shestakov~O.\,V.} Inversion of homogeneous operators using stabilized hard thresholding with\linebreak
\\[-12pt]
\hspace*{23pt}unknown noise variance&1&49--54\\
\Avtors{Shestakov~O.\,V.} Properties of wavelet estimates of signals recorded at random time points&2&16--21\\
\Avtors{Shestakov~O.\,V.} The mean square risk of~nonlinear regularization in~the~problem of~inversion\linebreak
\\[-12pt]
\hspace*{23pt}of~linear homogeneous operators with~a~random sample size&4&48--53\\
\Avtors{Shnurkov~P.\,V.\ and Vakhtanov~N.\,A.} On the solution of the optimal control problem of inventory of~a~discrete product in~the~stochastic model of~regeneration with continuously\linebreak
\\[-12pt]
\hspace*{23pt}occuring consumption&3&50--57\\
\Avtors{Shnurkov~P.\,V.\ and Vakhtanov~N.\,A.} Research of the optimal control problem of~inventory of~a~discrete product in~the~stochastic regeneration model with continuously\linebreak
\\[-12pt]
\hspace*{23pt}occuring consumption and random delivery delay&2&54--61\\
\Avtors{Shorgin~S.\,Ya.} see Gaidamaka~A.\,A.&&\\
\Avtors{Shorgin~S.\,Ya.} see Markova~E.\,V.&&\\
\Avtors{Shorgin~V.\,S.} see Kudryavtsev~A.\,A.&&\\
\Avtors{Sinitsyn~I.\,N.} Interpolatonal analytical modeling in~complex stochastic systems&1&2--8\\
\Avtors{Skrynnik~A.\,A.} see Smirnov~I.\,V.&&\\
\Avtors{Smirnov~I.\,V., Panov~A.\,I., Skrynnik~A.\,A., and Chistova~E.\,V.} Personal cognitive assistant: \linebreak
\\[-12pt]
\hspace*{23pt}Concept and key principals&3&105--113\\
\Avtors{Stefanovich~A.\,I.} see Bosov~A.\,V.&&\\
\Avtors{Stefanovich~A.\,I.} see Bosov~A.\,V.&&\\
\Avtors{Strijov~V.\,V.} see Anikeyev~D.\,A.&&\\
\Avtors{Strijov~V.\,V.} see Grabovoy~A.\,V.&&\\
\Avtors{Suchkov~A.\,P.} The scientific result as~the~information object in~the~context of~the~scientific\linebreak
\\[-12pt]
\hspace*{23pt}services system management&3&137--144\\
\Avtors{Sukayev~A.\,A.} see Logachev~O.\,A.&&\\
\Avtors{Sukayev~A.\,A.} see Logachev~O.\,A.&&\\
\Avtors{Tarasov~E.\,A.} see Kovalev~D.\,Y.&&\\
\Avtors{Tarkhov~A.\,A.} see Zakharova~T.\,V.&&\\
\Avtors{Timonina~E.\,E.} see Grusho~A.\,A.&&\\
\Avtors{Timonina~E.\,E.} see Grusho~A.\,A.&&\\
\Avtors{Timonina~E.\,E.} see Grusho~A.\,A.&&\\
\Avtors{Timonina~E.\,E.} see Grusho~A.\,A.&&\\
\Avtors{Titova~A.\,I.} see Arutyunov~E.\,N.&&\\
\Avtors{Tsaregorodtsev~A.\,L.} see Burlutskiy~V.\,V.&&\\
\Avtors{Ushakov~N.\,G.} see Ushakov~V.\,G.&&\\
\Avtors{Ushakov~V.\,G.\ and Ushakov~N.\,G.} The output streams in~the~single server queueing system\linebreak
\\[-12pt]
\hspace*{23pt}with~a~head of~the~line priority&4&42--47\\
\Avtors{Ushakov~V.\,G.} see Agalarov~Ya.\,M.&&\\
\Avtors{Vakhtanov~N.\,A.} see Shnurkov~P.\,V.&&\\
\Avtors{Vakhtanov~N.\,A.} see Shnurkov~P.\,V.&&\\
\Avtors{Vinogradov~A.\,P.} see Zhuravlev~Yu.\,I.&&\\
\Avtors{Voloshin~S.\,V.} see Burlutskiy~V.\,V.&&\\
\Avtors{Vyshinsky~L.\,L., Kuryansky~M.\,K., and Flerov~Yu.\,A.} Digital model of the aircraft's weight\linebreak
\\[-12pt]
\hspace*{23pt}passport&4&\hphantom{1}3--10\\
\Avtors{Yakimchuk~A.\,V.} see Burlutskiy~V.\,V.&&\\
\Avtors{Zabezhailo~M.\,I.} see Grusho~A.\,A.&&\\
\Avtors{Zabezhailo~M.\,I.} see Grusho~A.\,A.&&\\
\Avtors{Zakharova~T.\,V.\ and Tarkhov~A.\,A.} Evaluation of the significance level in schuirmann's test for\linebreak
\\[-12pt]
\hspace*{23pt}checking the~bioequivalence hypothesis in~missing data conditions&3&58--62\\
\end{tabular}
}
\pagebreak

\def\leftfootline{\small{\textbf{\thepage}
\hfill INFORMATIKA I EE PRIMENENIYA~--- INFORMATICS AND APPLICATIONS\ \ \ 2019\
\ \ volume~13\ \ \ issue\ 4}
}%
 \def\rightfootline{\small{INFORMATIKA I EE PRIMENENIYA~---
INFORMATICS AND APPLICATIONS\ \ \ 2019\ \ \ volume~13\ \ \ issue\ 4
\hfill \textbf{\thepage}}}

\def\leftkol{2019 AUTHOR INDEX} % ENGLISH ABSTRACTS}

\def\rightkol{2019 AUTHOR INDEX} %ENGLISH ABSTRACTS}


\noindent
{\tabcolsep=3pt
\begin{tabular}{p{395.48108pt}cc}
&\textbf{Issue} & \textbf{Page}\\[6pt]
\Avtors{Zatsarinny~A.\,A., Korotkov~V.\,V., and Matveev~M.\,G.} Modeling the process of network planning\linebreak
\\[-12pt]
\hspace*{23pt}of~a~portfolio of~projects with heterogeneous resources under fuzziness&2&92--99\\
\Avtors{Zatsman~I.\,M.} Digital encoding of~concepts&4&\hphantom{1}97--106\\
\Avtors{Zatsman~I.\,M.} Goal-oriented development of~linguistic knowledge systems: Identifying and\linebreak
\\[-12pt]
\hspace*{23pt}filling of~lacunae&1&91--98\\
\Avtors{Zatsman~I.\,M.} Third-order interfaces in informatics&3&82--89\\
\Avtors{Zatsman~I.\,M.} see Goncharov~A.\,A.&&\\
\Avtors{Zeifman~A.\,I., Satin~Y.\,A., and Kiseleva~K.\,M.} On the bounds of the rate of convergence for\linebreak
\\[-12pt]
\hspace*{23pt}some queueing models with incompletely defined intensities&3&14--19\\
\Avtors{Zhuravlev~Yu.\,I., Sen'ko~O.\,V., Bondarenko~N.\,N., Ryazanov~V.\,V., Dokukin~A.\,A., and Vinogradov~A.\,P.} Research of~the~possibility to~forecast changes in~financial state of~a~credit\linebreak
\\[-12pt]
\hspace*{23pt}organization on~the~basis of~public financial statements&4&30--35\\
\end{tabular}
}

%\thispagestyle{myheadings}
\def\leftfootline{\small{\textbf{\thepage}
\hfill INFORMATIKA I EE PRIMENENIYA~--- INFORMATICS AND APPLICATIONS\ \ \ 2019\
\ \ volume~13\ \ \ issue\ 4}
}%
 \def\rightfootline{\small{INFORMATIKA I EE PRIMENENIYA~---
INFORMATICS AND APPLICATIONS\ \ \ 2019\ \ \ volume~13\ \ \ issue\ 4
\hfill \textbf{\thepage}}}

 \label{end\stat}

\newpage

%   \vspace*{-48pt}

\begin{center}
\vspace*{6pt}
\mbox{%
\epsfxsize=53.502mm
\epsfbox{foto-1.eps}
}
\end{center}

\vspace*{6pt} %Академик


   \begin{center}
\fbox{\Large\textbf{Профессор Игорь Алексеевич Ушаков}}\\[12pt]
\textbf{\large 22.01.1935--27.02.2015}
   \end{center}


   %\vspace*{2.5mm}

   \vspace*{5mm}

   \thispagestyle{empty}

%\

%\vspace*{-12pt}


Редакционный совет и редакционная коллегия журнала <<Информатика и~её применения>> с~глубоким прискорбием извещают, что 27~февраля 2015~г.\ после тяжелой
и~продолжительной болезни скончался Игорь Алексеевич Ушаков~--- доктор технических наук, профессор, член редколлегии журнала <<Информатика и ее применения>>.

Игорь Алексеевич Ушаков окончил Московский авиационный институт, в~1963~г.\ защитил кандидатскую, а~в~1968~г.~--- докторскую диссертацию. С~1958 по 1989~гг.\ работал в~ряде научно-исследовательских организаций СССР, в~том числе руководил отделами в~НИИ АА и~ВЦ АН СССР; с 1969 по 1989 гг. преподавал в~МФТИ (был профессором, а~затем заведующим кафедрой) и~в~МЭИ. С~1989~г.~---- в~США: являлся профессором университета Дж.\ Вашингтона, университета Дж.\ Мэйсона и~Калифорнийского университета, сотрудником компаний MCI, Qualcomm и Hughes.

И.\,А.~Ушаков с момента основания журнала <<Надежность и~контроль качества>> был заместителем ответственного редактора, а~затем на протяжении многих лет членом редколлегии. В~2006~г.\ основал электронный международный журнал ``Reliability: Theory \& Application'', главным редактором которого оставался до конца жизни.

Учебниками и справочниками по теории надежности, написанными И.\,А.~Ушаковым, пользовались и~пользуются несколько поколений ученых и~специалистов в~разных странах мира.

Игорь Алексеевич всегда уделял огромное внимание работе с~молодежью; более~50 его учеников защитили докторские и~кандидатские диссертации.

И.\,А.~Ушаков вел активную научно-про\-све\-ти\-тель\-скую деятельность. В~частности, он был одним из организаторов и~руководителей Московского кабинета качества и~надежности при Политехническом музее (целью этого Кабинета было оказание консультаций работникам промышленных предприятий и~чтение курсов лекций для инженеров, занимающихся проблемой надежности). Находясь в~США, И.\,А.~Ушаков создал международный ин\-тер\-нет-фо\-рум им.\ Б.\,В.~Гнеденко, объединивший около~400~видных специалистов по приложениям теории вероятностей и~математической статистики, преимущественно в~об\-ласти теории надежности и~анализа риска, из десятков стран мира; коллективным членов этого Форума является и~наш журнал. Цели Форума~--- содействие контактам между специалистами из разных стран, организация обмена профессиональными 
новостями и~информацией (новые публикации, предстоящие события и~др.). Также необходимо отметить большое число на\-уч\-но-по\-пу\-ляр\-ных работ, опубликованных И.\,А.~Ушаковым.

И.\,А.~Ушаков обладал большим личным обаянием, имел широкий круг интересов. Все знавшие И.\,А.~Ушакова всегда будут помнить его как замечательного ученого и~прекрасного человека.

\bigskip

Редакционный совет и редакционная коллегия журнала <<Информатика и~её применения>> 
выражают глубокие соболезнования родным и близким покойного, всем, кто его знал и~работал с~ним.


%\def\stat{cont}
{%\hrule\par
%\vskip 7pt % 7pt
\raggedleft\Large \bf%\baselineskip=3.2ex
А\,В\,Т\,О\,Р\,С\,К\,И\,Й\ \ У\,К\,А\,З\,А\,Т\,Е\,Л\,Ь\ \ З\,А\ \ 2\,0\,1\,0 г. \vskip 17pt
    \hrule
    \par
\vskip 21pt plus 6pt minus 3pt }

\label{st\stat}

\def\tit{\ }

\def\aut{\ }
\def\auf{\ }

\def\leftkol{\ } % ENGLISH ABSTRACTS}

\def\rightkol{\ } %АВТОРСКИЙ УКАЗАТЕЛЬ ЗА 2010 г.} %ENGLISH ABSTRACTS}

\titele{\tit}{\aut}{\auf}{\leftkol}{\rightkol}

\vspace*{-12pt}

{\tabcolsep=3pt
\begin{tabular}{p{388pt}rr}
&\textbf{Выпуск} & \textbf{Стр.}\\[6pt]
\hangindent=23pt\noindent\textbf{Арутюнян~А.\,Р.} Моделирование влияния деформаций отпечатков пальцев на 
точность\linebreak
\vspace*{-12pt}\\
\hspace*{23pt}дактилоскопической идентификации$\dotfill$&1&51\\
\hangindent=23pt\noindent\textbf{Архипов~О.\,П., Зыкова~З.\,П.} Интеграция гетерогенной информации о цветных 
пикселях\linebreak
\vspace*{-12pt}\\
\hspace*{23pt}и их цветовосприятии$\dotfill$&4&15\\
\hangindent=23pt\noindent\textbf{Баранов~С.\,И., Френкель~С.\,Л., Захаров~В.\,Н.} Полуформальная верификация 
цифрового устройства с конвейером, основанная на использовании алгоритмических машин\linebreak
\vspace*{-12pt}\\
\hspace*{23pt}состояния$\dotfill$&4&49\\
\textbf{Бекетова~И.\,В.} см.~Каратеев~С.\,Л.&&\\
\textbf{Белоусов~В.\,В.} см.~Синицын~И.\,Н.&&\\
\hangindent=23pt\noindent\textbf{Бенинг~В.\,Е., Королев~Р.\,А.} О предельном поведении мощностей критериев в 
случае\linebreak
\vspace*{-12pt}\\
\hspace*{23pt}распределения Лапласа$\dotfill$&2&63\\
\hangindent=23pt\noindent\textbf{Бенинг~В.\,Е., Сипина~А.\,В.} Асимптотическое разложение для мощности 
критерия,\linebreak
\vspace*{-12pt}\\
\hspace*{23pt}основанного на выборочной медиане, в случае распределения Лапласа$\dotfill$&1&18\\
\textbf{Бондаренко~А.\,В.} см.~Каратеев~С.\,Л.&&\\
\hangindent=23pt\noindent\textbf{Бородина~А.\,В., Морозов~Е.\,В.} Об оценивании асимптотики вероятности 
большого\linebreak
\vspace*{-12pt}\\
\hspace*{23pt}уклонения стационарной регенеративной очереди с одним прибором$\dotfill$&3&29\\
\hangindent=23pt\noindent\textbf{Бунтман~Н.\,В., Минель~Ж.-Л., Ле~Пезан~Д., Зацман~И.\,М.} Типология и 
компьютерное\linebreak
\vspace*{-12pt}\\
\hspace*{23pt}моделирование трудностей перевода$\dotfill$&3&77\\
\textbf{Визильтер~Ю.\,В.} см.~Каратеев~С.\,Л.&&\\
\hangindent=23pt\noindent\textbf{Гавриленко~С.\,В.} Оценки скорости сходимости распределений случайных сумм с 
безгранично делимыми индексами к нормальному закону$\dotfill$&4&81\\
\hangindent=23pt\noindent\textbf{Григорьева~М.\,Е., Шевцова~И.\,Г.} Уточнение неравенства 
Каца--Берри--Эссеена$\dotfill$&2&75\\
\hangindent=23pt\noindent\textbf{Грушо~А.\,А., Грушо~Н.\,А., Тимонина~Е.\,Е.} Поиск конфликтов в политиках 
безопасности: модель случайных графов$\dotfill$&3&38\\
\textbf{Грушо~Н.\,А.} см.~Грушо~А.\,А.&&\\
\hangindent=23pt\noindent\textbf{Гудков~В.\,Ю.} Математические модели изображения отпечатка пальца на основе 
описания линий$\dotfill$&1&58\\
\textbf{Гуртов~А.\,В.} см.~Лукьяненко~А.\,С.&&\\
\textbf{Желтов~С.\,Ю.} см.~Каратеев~С.\,Л.&&\\
\hangindent=23pt\noindent\textbf{Захаров~А.\,А., Серебряков~В.\,А.} Система управления электронной библиотекой 
LibMeta$\dotfill$&4&2\\
\textbf{Захаров~В.\,Н.} см.~Баранов~С.\,И.&&\\
\textbf{Захарова~Т.\,В.} см.~Матвеева~С.\,С.&&\\
\hangindent=23pt\noindent\textbf{Зацаринный~А.\,А., Чупраков~К.\,Г.} Некоторые аспекты выбора технологии для 
постро-\linebreak
\vspace*{-12pt}\\
\hspace*{23pt}ения систем отображения информации ситуационного центра$\dotfill$&3&59\\
\textbf{Зацман~И.\,М.} см.~Бунтман~Н.\,В.&&\\
\hangindent=23pt\noindent\textbf{Зейфман~А.\,И., Коротышева~А.\,В., Сатин~Я.\,А., Шоргин~С.\,Я.} Об 
устойчивости нестаци-\linebreak
\vspace*{-12pt}\\
\hspace*{23pt}онарных систем обслуживания с катастрофами$\dotfill$&3&9\\
\textbf{Зыкова~З.\,П.} см.~Архипов~О.\,П.&&\\
\hangindent=23pt\noindent\textbf{Илюшин~Г.\,Я., Соколов~И.\,А.} Организация управляемого доступа пользователей 
к\linebreak
\vspace*{-12pt}\\
\hspace*{23pt}разнородным ведомственным информационным ресурсам$\dotfill$&1&24\\
\hangindent=23pt\noindent\textbf{Кавагучи~Ю., Ульянов~В.\,В., Фуджикоши~Я.} Приближения для статистик, 
описывающих\linebreak
\vspace*{-12pt}\\
\hspace*{23pt}геометрические свойства данных большой размерности, с оценками 
ошибок$\dotfill$&1&12\\
\hangindent=23pt\noindent\textbf{Каратеев~С.\,Л., Бекетова~И.\,В., Ососков~М.\,В., Князь~В.\,А., 
Визильтер~Ю.\,В., Бондаренко~А.\,В., Желтов~С.\,Ю.} Автоматизированный контроль 
качества цифровых\linebreak
\vspace*{-12pt}\\
\hspace*{23pt}изображений для персональных документов$\dotfill$&1&65\\
\end{tabular}
}

\pagebreak

\def\leftkol{АВТОРСКИЙ УКАЗАТЕЛЬ ЗА 2010 г.} % ENGLISH ABSTRACTS}

\def\rightkol{АВТОРСКИЙ УКАЗАТЕЛЬ ЗА 2010 г.} %ENGLISH ABSTRACTS}

{\tabcolsep=3pt
\begin{tabular}{p{388pt}rr}
&\textbf{Выпуск} & \textbf{Стр.}\\[3pt]
\hangindent=23pt\noindent\textbf{Козеренко~Е.\,Б.} Лингвистические фильтры в статистических моделях машинного\linebreak
\vspace*{-12pt}\\
\hspace*{23pt}перевода$\dotfill$&2&83\\
\hangindent=23pt\noindent\textbf{Козеренко~Е.\,Б., Кузнецов~И.\,П.} Когнитивно-лингвистические представления в 
систе-\linebreak
\vspace*{-12pt}\\
\hspace*{23pt}мах обработки текстов$\dotfill$&3&69\\
\textbf{Князь~В.\,А.} см.~Каратеев~С.\,Л.&&\\
\hangindent=23pt\noindent\textbf{Колесников~А.\,В., Солдатов~С.\,А.} Алгоритм координации для гибридной 
интеллектуальной системы решения сложной задачи оперативно-производственного\linebreak
\vspace*{-12pt}\\
\hspace*{23pt}планирования$\dotfill$&4&61\\
\hangindent=23pt\noindent\textbf{Коновалов~М.\,Г.} О планировании потоков в системах вычислительных 
ресурсов$\dotfill$&2&3\\
\textbf{Конушин~А.\,С.} см.~Конушин~В.\,С.&&\\
\hangindent=23pt\noindent\textbf{Конушин~В.\,С., Кривовязь~Г.\,Р., Конушин~А.\,С.} Алгоритм распознавания людей 
в видео-\linebreak
\vspace*{-12pt}\\
\hspace*{23pt}последовательности по одежде$\dotfill$&1&74\\
\textbf{Корепанов~Э.\, Р.} см.~Синицын~И.\,Н.&&\\
\textbf{Королев~В.\,Ю.} см.~Соколов~И.\,А.&&\\
\textbf{Королев~Р.\,А.} см.~Бенинг~В.\,Е.&&\\
\textbf{Коротышева~А.\,В.} см.~Зейфман~А.\,И.&&\\
\hangindent=23pt\noindent\textbf{Кривенко~М.\,П.} Непараметрическое оценивание элементов байесовского 
клас\-си-\linebreak
\vspace*{-12pt}\\
\hspace*{23pt}фикатора$\dotfill$&2&13\\
\textbf{Кривовязь~Г.\,Р.} см.~Конушин~В.\,С.&&\\
\textbf{Крылов~А.\,С.} см.~Павельева~Е.\,А.&&\\
\hangindent=23pt\noindent\textbf{Крылов~В.\,А.} Моделирование и классификация многоканальных дистанционных\linebreak
\vspace*{-12pt}\\
\hspace*{23pt}изображений с использованием копул$\dotfill$&4&34\\
\hangindent=23pt\noindent\textbf{Крючин~О.\,В.} Разработка параллельных эвристических алгоритмов подбора 
весовых\linebreak
\vspace*{-12pt}\\
\hspace*{23pt}коэффициентов искусственной нейтронной сети$\dotfill$&2&53\\
\hangindent=23pt\noindent\textbf{Кудрявцев~А.\,А., Шоргин~С.\,Я.} Байесовские модели массового обслуживания и 
надеж-\linebreak
\vspace*{-12pt}\\
\hspace*{23pt}ности: характеристики среднего числа заявок в системе $M\vert M \vert 1\vert 
\infty$$\dotfill$&3&16\\
\hangindent=23pt\noindent\textbf{Кузнецов~А.\,А.} Связь между временными и структурно-топологическими 
характери-\linebreak
\vspace*{-12pt}\\
\hspace*{23pt}стиками диаграмм ритма сердца здоровых людей$\dotfill$&4&39\\
\textbf{Кузнецов~И.\,П.} см.~Козеренко~Е.\,Б.&&\\
\textbf{Ле~Пезан~Д.} см.~Бунтман~Н.\,В.&&\\
\hangindent=23pt\noindent\textbf{Лукьяненко~А.\,С., Морозов~Е.\,В., Гуртов~А.\,В.} Анализ сетевого протокола с общей 
функ-\linebreak
\vspace*{-12pt}\\
\hspace*{23pt}цией расширения окна передачи сообщения при конфликтах$\dotfill$&2&46\\
\hangindent=23pt\noindent\textbf{Лямин~О.\,О.} О предельном поведении мощностей критериев в случае обобщенного\linebreak
\vspace*{-12pt}\\
\hspace*{23pt}распределения Лапласа$\dotfill$&3&47\\
\hangindent=23pt\noindent\textbf{Маркин~А.\,В., Шестаков~О.\,В.} Асимптотики оценки риска при пороговой 
обработке\linebreak
\vspace*{-12pt}\\
\hspace*{23pt}вейвлет-вейглет коэффициентов в задаче томографии$\dotfill$&2&36\\
\hangindent=23pt\noindent\textbf{Матвеева~С.\,С., Захарова~Т.\,В.} Сети массового обслуживания с наименьшей 
длиной\linebreak
\vspace*{-12pt}\\
\hspace*{23pt}очереди$\dotfill$&3&22\\
\hangindent=23pt\noindent\textbf{Матюшенко~С.\,И.} Стационарные характеристики двухканальной системы 
обслужива-\linebreak
\vspace*{-12pt}\\
\hspace*{23pt}ния с переупорядочиванием заявок и распределениями фазового типа$\dotfill$&4&68\\
\textbf{Минель~Ж.-Л.} см.~Бунтман~Н.\,В.&&\\
\textbf{Морозов~Е.\,В.} см.~Бородина~А.\,В.&&\\
\textbf{Морозов~Е.\,В.} см.~Лукьяненко~А.\,С.&&\\
\textbf{Ососков~М.\,В.} см.~Каратеев~С.\,Л.&&\\
\hangindent=23pt\noindent\textbf{Павельева~Е.\,А., Крылов~А.\,С.} Поиск и анализ ключевых точек радужной 
оболочки\linebreak
\vspace*{-12pt}\\
\hspace*{23pt}глаза методом преобразования Эрмита$\dotfill$&1&79\\
\textbf{Печинкин~А.\,В.} см.~Френкель~С.\,Л.,&&\\
\hangindent=23pt\noindent\textbf{Протасов~В.\,И.} Составление субъективного портрета с использованием 
эволюционно-\linebreak
\vspace*{-12pt}\\
\hspace*{23pt}го морфинга и квалиметрия метода$\dotfill$&1&83\\
\hangindent=23pt\noindent\textbf{Рудаков~К.\,В., Торшин~И.\,Ю.} Вопросы разрешимости задачи распознавания 
вторичной\linebreak
\vspace*{-12pt}\\
\hspace*{23pt}структуры белка$\dotfill$&2&25\\
\textbf{Сатин~Я.\,А.} см.~Зейфман~А.\,И.&&\\
\hangindent=23pt\noindent\textbf{Сейфуль-Мулюков~Р.\,Б.} Нефть как носитель информации о своем 
происхождении,\linebreak
\vspace*{-12pt}\\
\hspace*{23pt}структуре и эволюции$\dotfill$&1&41\\
\end{tabular}
}

{\tabcolsep=3pt
\begin{tabular}{p{388pt}rr}
&\textbf{Выпуск} & \textbf{Стр.}\\[6pt]
\textbf{Семендяев~Н.\,Н.} см.~Синицын~И.\,Н.&&\\
\textbf{Серебряков~В.\,А.} см.~Захаров~А.\,А.&&\\
\textbf{Синицын~В.\,И.} см.~Синицын~И.\,Н.&&\\
\hangindent=23pt\noindent\textbf{Синицын~И.\,Н., Синицын~В.\,И., Корепанов~Э.\, Р., Белоусов~В.\,В., 
Семендяев~Н.\,Н.} Оперативное построение информационных моделей движения полюса 
Земли\linebreak
\vspace*{-12pt}\\
\hspace*{23pt}методами линейных и линеаризованных фильтров$\dotfill$&1&2\\
\textbf{Сипина~А.\,В.} см.~Бенинг~В.\,Е.&&\\
\hangindent=23pt\noindent\textbf{Соколов~И.\,А.} О работах заслуженного деятеля науки Российской Федерации 
И.\,Н.~Синицына в области информационных технологий и автоматизации (к 70-летию\linebreak
\vspace*{-12pt}\\
\hspace*{23pt}со дня рождения)$\dotfill$&3&84\\
\textbf{Соколов~И.\,А.} см.~Илюшин~Г.\,Я.&&\\
\hangindent=23pt\noindent\textbf{Соколов~И.\,А., Королев~В.\,Ю.} Предисловие$\dotfill$&2&2\\
\textbf{Солдатов~С.\,А.} см.~Колесников~А.\,В.&&\\
\hangindent=23pt\noindent\textbf{Степанов~С.\,Ю.} Использование координатного метода фрагментации 
коммутаторной\linebreak
\vspace*{-12pt}\\
\hspace*{23pt}нейронной сети для сокращения трафика$\dotfill$&2&57\\
\textbf{Тимонина~Е.\,Е.} см.~Грушо~А.\,А.&&\\
\textbf{Торшин~И.\,Ю.} см.~Рудаков~К.\,В.&&\\
\textbf{Ульянов~В.\,В.} см.~Кавагучи~Ю.&&\\
\textbf{Фазекаш~И.} см.~Чупрунов~А.\,Н.&&\\
\textbf{Френкель~С.\,Л.} см.~Баранов~С.\,И.&&\\
\hangindent=23pt\noindent\textbf{Френкель~С.\,Л., Печинкин~А.\,В.} Оценка времени самовосстановления в 
цифровых\linebreak
\vspace*{-12pt}\\
\hspace*{23pt}системах после сбоев, вызываемых переходными помехами$\dotfill$&3&2\\
\textbf{Фуджикоши~Я.} см.~Кавагучи~Ю.&&\\
\hangindent=23pt\noindent\textbf{Цискаридзе~А.\,К.} Математическая модель и метод восстановления позы человека 
по\linebreak
\vspace*{-12pt}\\
\hspace*{23pt}стереопаре силуэтных изображений$\dotfill$&4&27\\
\hangindent=23pt\noindent\textbf{Чупраков~К.\,Г.} К вопросу о размещении коллективных средств отображения в 
ситуа-\linebreak
\vspace*{-12pt}\\
\hspace*{23pt}ционном зале с заданными параметрами$\dotfill$&4&89\\
\textbf{Чупраков~К.\,Г.} см.~Зацаринный~А.\,А.&&\\
\hangindent=23pt\noindent\textbf{Чупрунов~А.\,Н., Фазекаш~И.} Законы повторного логарифма для числа 
безошибочных\linebreak
\vspace*{-12pt}\\
\hspace*{23pt}блоков при помехоустойчивом кодировании$\dotfill$&3&42\\
\textbf{Шевцова~И.\,Г.} см.~Григорьева~М.\,Е.&&\\
\hangindent=23pt\noindent\textbf{Шестаков~О.\,В.} Аппроксимация распределения оценки риска пороговой 
обработки вейвлет-коэффициентов нормальным распределением при использовании 
выбо-\linebreak
\vspace*{-12pt}\\
\hspace*{23pt}рочной дисперсии$\dotfill$&4&73\\
\textbf{Шестаков~О.\,В.} см.~Маркин~А.\,В.&&\\
\textbf{Шоргин~С.\,Я.} см.~Зейфман~А.\,И.&&\\
\textbf{Шоргин~С.\,Я.} см.~Кудрявцев~А.\,А.&&\\
\end{tabular}
}

%\thispagestyle{myheadings}
\def\leftfootline{\small{\textbf{\thepage}
\hfill ИНФОРМАТИКА И ЕЁ ПРИМЕНЕНИЯ\ \ \ том~4\ \ \ выпуск~4\ \ \ 2010}
}%
 \def\rightfootline{\small{ИНФОРМАТИКА И ЕЁ ПРИМЕНЕНИЯ\ \ \ том~4\ \ \ выпуск~4\ \ \ 2010
 \hfill \textbf{\thepage}}}
 \label{end\stat}
%
%Том 10 Выпуск 1-4 Год 2016

\def\stat{cont-e}
{%\hrule\par
%\vskip 7pt % 7pt
\raggedleft\Large \bf%\baselineskip=3.2ex
2\,0\,1\,6\ \ A\,U\,T\,H\,O\,R\ \ I\,N\,D\,E\,X \vskip 17pt
 \hrule
 \par
\vskip 21pt plus 6pt minus 3pt }

\label{st\stat}

\def\tit{\ }

\def\aut{\ }
\def\auf{\ }

\def\leftkol{\ } %2016 AUTHOR INDEX} % ENGLISH ABSTRACTS}

\def\rightkol{\ } %2016 AUTHOR INDEX} %ENGLISH ABSTRACTS}

\titele{\tit}{\aut}{\auf}{\leftkol}{\rightkol}

\def\leftfootline{\small{\textbf{\thepage}
\hfill INFORMATIKA I EE PRIMENENIYA~--- INFORMATICS AND APPLICATIONS\ \ \ 2016\
\ \ volume~10\ \ \ issue\ 4}
}%
 \def\rightfootline{\small{INFORMATIKA I EE PRIMENENIYA~--- INFORMATICS AND APPLICATIONS\ \ \ 2016\ \ \ volume~10\ \ \ issue\ 4
\hfill \textbf{\thepage}}}

\vspace*{-12pt}
\vspace*{-18pt}

{\tabcolsep=2.8pt
\begin{tabular}{p{382pt}cc}
&\textbf{Issue} & \textbf{Page}\\[6pt]
\Avtors{Agalarov~M.\,Ya.} see~Agalarov~Ya.\,M.&&\\
\Avtors{Agalarov~Ya.\,M., Agalarov~M.\,Ya., and
Shorgin~V.\,S.} About the optimal threshold of queue\linebreak
\\[-12pt]
\hspace*{23pt}length in a~particular problem of profit maximization
in the $M/G/1$ queuing system&2&70--79\\
\Avtors{Alexeyevsky~D.\,A.} BioNLP ontology extraction from 
a~restricted language corpus with\linebreak
\\[-12pt]
\hspace*{23pt}context-free grammars&1&119--128\\
\Avtors{Andreev~S.\,D.} see~Gaidamaka~Yu.\,V.&&\\
\Avtors{Andreev~S.\,D.} see~Ometov~A.\,Ya.&&\\
\Avtors{Arkhipov~O.\,P., Arkhipov~P.\,O., and Sidorkin~I.\,I.} The
option to create a~local coordinate\linebreak
\\[-12pt]
\hspace*{23pt}system for synchronization of selected images&3&91--97\\
\Avtors{Arkhipov~P.\,O.} see~Arkhipov~O.\,P.&&\\
\Avtors{Belousov~V.\,V.} see~Shnurkov~P.\,V.&&\\
\Avtors{Belousov~V.\,V.} see~Shnurkov~P.\,V.&&\\
\Avtors{Bening~V.\,E.} Calculation of~the~asymptotic deficiency
of~some statistical procedures based\linebreak
\\[-12pt]
\hspace*{23pt}on~samples with~random sizes&4&34--45\\
\Avtors{Borisov~A.\,V., Bosov~A.\,V., and Miller~G.\,B.} Modeling and
monitoring of VoIP connection&2&\hphantom{1}2--13\\
\Avtors{Bosov~A.\,V.} see~Borisov~A.\,V.&&\\
\Avtors{Briukhov~D.\,O.} see~Stupnikov~S.\,A.&&\\
\Avtors{Callaos~N.\,K.\ and Seyful-Mulyukov~R.\,B.} Complexity and
its information content&1&129--139\\
\Avtors{Chertok~A.\,V., Kadaner~A.\,I., Khazeeva~G.\,T., and
Sokolov~I.\,A.} Regime switching detection\linebreak
\\[-12pt]
\hspace*{23pt}for~the~Levy driven
Ornstein--Uhlenbeck process using CUSUM methods&4&46--56\\
\Avtors{Chichagov~V.\,V.} Asymptotic expansions of mean absolute
error of uniformly minimum variance unbiased and maximum likelihood
estimators on the one-parameter exponential\linebreak
\\[-12pt]
\hspace*{23pt}family model of lattice distributions&3&66--76\\
\Avtors{Danishevsky~V.\,I.} see~Kolesnikov A.\,V.&&\\
\Avtors{Fazliev~A.\,Z.} see~Kalinichenko~L.\,A.&&\\
\Avtors{Fedoseev~A.\,A.} What is behind the concept of ``knowledge in
small packages''&3&105--110\\
\Avtors{Gaidamaka~Yu.\,V., Andreev~S.\,D., Sopin~E.\,S.,
Samouylov~K.\,E., and Shorgin~S.\,Ya.} Interference analysis
of~the~device-to-device communications model with~regard to~a~signal\linebreak
\\[-12pt]
\hspace*{23pt}propagation environment&4&\hphantom{1}2--10\\
\Avtors{Gasilov~A.\,V.} see~Yakovlev~O.\,A.&&\\
\Avtors{Goncharov~A.\,V.\ and Strijov~V.\,V.} Metric time series
classification using weighted dynamic\linebreak
\\[-12pt]
\hspace*{23pt}warping relative to centroids of classes&2&36--47\\
\Avtors{Gordov~E.\,P.} see~Kalinichenko~L.\,A.&&\\
\Avtors{Gorshenin~A.\,K.} Concept of online service for stochastic
modeling of real processes&1&72--81\\
\Avtors{Gorshenin~A.\,K.} see~Shnurkov~P.\,V.&&\\
\Avtors{Gorshenin~A.\,K.} see~Shnurkov~P.\,V.&&\\
\Avtors{Grusho~A.\,A., Grusho~N.\,A., Zabezhailo~M.\,I., and
Timonina~E.\,E.} Integration of statistical and\linebreak
\\[-12pt]
\hspace*{23pt}deterministic methods for
analysis of information security&3&2--8\\
\Avtors{Grusho~A.\,A., Zabezhailo~M.\,I., and Zatsarinny~A.\,A.} On
the advanced procedure to reduce\linebreak
\\[-12pt]
\hspace*{23pt}calculation of Galois closures&4&\hphantom{1}96--104\\
\Avtors{Grusho~N.\,A.} see~Grusho~A.\,A.&&\\
\Avtors{Havanskov~V.\,A.} see~Minin~V.\,A.&&\\
\Avtors{Inkova~O.\,Yu.} see~Zatsman~I.\,M.&&\\
\Avtors{Isachenko~R.\,V.\ and Strijov~V.\,V.} Metric learning in
multiclass time series classification\linebreak
\\[-12pt]
\hspace*{23pt}problem&2&48--57\\
\end{tabular}
}
\pagebreak

\def\leftfootline{\small{\textbf{\thepage}
\hfill INFORMATIKA I EE PRIMENENIYA~--- INFORMATICS AND APPLICATIONS\ \ \ 2016\
\ \ volume~10\ \ \ issue\ 4}
}%
 \def\rightfootline{\small{INFORMATIKA I EE PRIMENENIYA~---
INFORMATICS AND APPLICATIONS\ \ \ 2016\ \ \ volume~10\ \ \ issue\ 4
\hfill \textbf{\thepage}}}

\def\leftkol{2016 AUTHOR INDEX} % ENGLISH ABSTRACTS}

\def\rightkol{2016 AUTHOR INDEX} %ENGLISH ABSTRACTS}


{\tabcolsep=2.83pt
\begin{tabular}{p{382pt}cc}
&\textbf{Issue} & \textbf{Page}\\[6pt]
\Avtors{Kadaner~A.\,I.} see~Chertok~A.\,V.&&\\[.255pt]
\Avtors{Kalinichenko~L.\,A., Volnova~A.\,A., Gordov~E.\,P.,
Kiselyova~N.\,N., Kovaleva~D.\,A., Malkov~O.\,Yu., Okladnikov~I.\,G.,
Podkolodnyy~N.\,L., Pozanenko~A.\,S., Ponomareva~N.\,V.,
Stupnikov~S.\,A.,} \textbf{and Fazliev~A.\,Z.} Data access challenges for data
intensive\linebreak
\\[-12pt]
\hspace*{23pt}research in Russia&1& 2--22\\[.255pt]
\Avtors{Karasikov~M.\,E.\ and Strijov~V.\,V.} Feature-based
time-series classification&4&121--131\\[.255pt]
\Avtors{Khazeeva~G.\,T.} see~Chertok~A.\,V.&&\\[.255pt]
\Avtors{Khokhlov~Yu.\,S.} Multivariate fractional Levy motion and its
applications&2&\hphantom{1}98--106\\[.255pt]
\Avtors{Kirikov~I.\,A., Kolesnikov~A.\,V., Listopad~S.\,V., and
Rumovskaya~S.\,B.} Fine-grained hybrid\linebreak
\\[-12pt]
\hspace*{23pt}intelligent systems. Part 2:
Bidirectional hybridization&1&\hphantom{1}96--105\\[.255pt]
\Avtors{Kirikov~I.\,A., Kolesnikov~A.\,V., Listopad~S.\,V., and
Rumovskaya~S.\,B.} ``Virtual council''~---\linebreak
\\[-12pt]
\hspace*{23pt}source environment
supporting complex diagnostic decision making&3&81--90\\[.255pt]
\Avtors{Kiselyova~N.\,N.} see~Kalinichenko~L.\,A.&&\\[.255pt]
\Avtors{Kolesnikov A.\,V., Listopad~S.\,V., Rumovskaya~S.\,B., and
Danishevsky~V.\,I.} Informal axiomatic\linebreak
\\[-12pt]
\hspace*{23pt}theory of~the~role visual models&4&114--120\\[.255pt]
\Avtors{Kolesnikov~A.\,V.} see~Kirikov~I.\,A.&&\\[.255pt]
\Avtors{Kolesnikov~A.\,V.} see~Kirikov~I.\,A.&&\\[.255pt]
\Avtors{Kolin~K.\,K.} Humanitarian aspects of information
security&3&111--121\\[.255pt]
\Avtors{Konovalov~M.\,G.\ and Razumchik~R.\,V.} Dispatching
to~two parallel nonobservable queues using\linebreak
\\[-12pt]
\hspace*{23pt}only static
information&4&57--67\\[.255pt]
\Avtors{Korchagin~A.\,Yu.} see~Korolev~V.\,Yu.&&\\[.255pt]
\Avtors{Korchagin~A.\,Yu.} see~Korolev~V.\,Yu.&&\\[.255pt]
\Avtors{Korepanov~E.\,R.} see~Sinitsyn~I.\,N.&&\\[.255pt]
\Avtors{Korepanov~E.\,R.} see~Sinitsyn~I.\,N.&&\\[.255pt]
\Avtors{Korolev~V.\,Yu., Korchagin~A.\,Yu., and Zeifman~A.\,I.} The
Poisson theorem for Bernoulli trials\linebreak
\\[-12pt]
\hspace*{23pt}with~a~random probability
of~success and~a~discrete analog of~the~Weibull distribution&4&11--20\\[.255pt]
\Avtors{Korolev~V.\,Yu., Zeifman~A.\,I., and Korchagin~A.\,Yu.}
Asymmetric Linnik distributions as~limit\linebreak
\\[-12pt]
\hspace*{23pt}laws for~random sums
of~independent random variables with~finite variances&4&21--33\\[.255pt]
\Avtors{Koucheryavy~E.\,A.} see~Ometov~A.\,Ya.&&\\[.255pt]
\Avtors{Kovaleva~D.\,A.} see~Kalinichenko~L.\,A.&&\\[.255pt]
\Avtors{Kovalyov~S.\,P.} Metaprogramming to increase
manufacturability of large-scale software-\linebreak
\\[-12pt]
\hspace*{23pt}intensive systems&1&56--66\\[.255pt]
\Avtors{Krivenko~M.\,P.} Significance tests of feature selection for
classification&3&32--40\\[.255pt]
\Avtors{Kruzhkov~M.\,G.} see~Zalizniak~Anna~A.&&\\[.255pt]
\Avtors{Kruzhkov~M.\,G.} see~Zatsman~I.\,M.&&\\[.255pt]
\Avtors{Kudryavtsev~A.\,A.} Bayesian queueing and reliability models:
\textit{A~priori} distributions with\linebreak
\\[-12pt]
\hspace*{23pt}compact support&1&67--71\\[.255pt]
\Avtors{Kudryavtsev~A.\,A.} Characteristics dependent on the balance
coefficient in Bayesian models\linebreak
\\[-12pt]
\hspace*{23pt}with compact support of \textit{a priori}
distributions&3&77--80\\[.255pt]
\Avtors{Kudryavtsev~A.\,A.\ and Palionnaia~S.\,I.} Bayesian recurrent
model of reliability growth:\linebreak
\\[-12pt]
\hspace*{23pt}Parabolic distribution of parameters&2&80--83\\[.255pt]
\Avtors{Kudryavtsev~A.\,A.\ and Titova~A.\,I.} Bayesian queuing
and~reliability models: Degenerate-\linebreak
\\[-12pt]
\hspace*{23pt}Weibull case&4&68--71\\[.255pt]
\Avtors{Leontyev~N.\,D.\ and Ushakov~V.\,G.} Analysis of a queueing
system with autoregressive arrivals\linebreak
\\[-12pt]
\hspace*{23pt}and nonpreemptive priority&3&15--22\\[.255pt]
\Avtors{Listopad~S.\,V.} see~Kirikov~I.\,A.&&\\[.255pt]
\Avtors{Listopad~S.\,V.} see~Kirikov~I.\,A.&&\\[.255pt]
\Avtors{Listopad~S.\,V.} see~Kolesnikov A.\,V.&&\\[.255pt]
\Avtors{Malkov~O.\,Yu.} see~Kalinichenko~L.\,A.&&\\[.255pt]
\Avtors{Markov~A.\,S., Monakhov~M.\,M., and
Ulyanov~V.\,V.} Generalized Cornish--Fisher expansions\linebreak
\\[-12pt]
\hspace*{23pt}for distributions of statistics based on samples
of random size&2&84--91\\[.255pt]
\Avtors{Melnikov~A.\,K.\ and Ronzhin~A.\,F.} Generalized statistical
method of~text analysis based\linebreak
\\[-12pt]
\hspace*{23pt}on~calculation of~probability distributions
of~statistical values&4&89--95\\
\end{tabular}
}
\pagebreak

\def\leftfootline{\small{\textbf{\thepage}
\hfill INFORMATIKA I EE PRIMENENIYA~--- INFORMATICS AND APPLICATIONS\ \ \ 2016\
\ \ volume~10\ \ \ issue\ 4}
}%
 \def\rightfootline{\small{INFORMATIKA I EE PRIMENENIYA~---
INFORMATICS AND APPLICATIONS\ \ \ 2016\ \ \ volume~10\ \ \ issue\ 4
\hfill \textbf{\thepage}}}

\def\leftkol{2016 AUTHOR INDEX} % ENGLISH ABSTRACTS}

\def\rightkol{2016 AUTHOR INDEX} %ENGLISH ABSTRACTS}


{\tabcolsep=3pt
\begin{tabular}{p{381pt}cc}
&\textbf{Issue} & \textbf{Page}\\[6pt]
\Avtors{Meykhanadzhyan~L.\,A.} Stationary characteristics of the finite
capacity queueing system with\linebreak
\\[-12pt]
\hspace*{23pt}inverse service order and generalized
probabilistic priority&2&123--131\\[.23pt]
\Avtors{Miller~G.\,B.} see~Borisov~A.\,V.&&\\[.23pt]
\Avtors{Minin~V.\,A., Zatsman~I.\,M., Havanskov~V.\,A., and
Shubnikov~S.\,K.} Intensity of citation of scientific publications in
inventions on information and computer technologies patented\linebreak
\\[-12pt]
\hspace*{23pt}in Russia by domestic and foreign applicants&2&107--122\\[.23pt]
\Avtors{Monakhov~M.\,M.} see~Markov~A.\,S.&&\\[.23pt]
\Avtors{Naumov~V.\,A.\ and Samouylov~K.\,E.} On relationship
between queuing systems with resources\linebreak
\\[-12pt]
\hspace*{23pt}and Erlang networks&3&\hphantom{1}9--14\\[.23pt]
\Avtors{Okladnikov~I.\,G.} see~Kalinichenko~L.\,A.&&\\[.23pt]
\Avtors{Ometov~A.\,Ya., Andreev~S.\,D., Turlikov~A.\,M., and
Koucheryavy~E.\,A.} Performance analysis of\linebreak
\\[-12pt]
\hspace*{23pt}a wireless data
aggregation system with contention for contemporary sensor
networks&3&23--31\\[.23pt]
\Avtors{Palionnaia~S.\,I.} see~Kudryavtsev~A.\,A.&&\\[.23pt]
\Avtors{Podkolodnyy~N.\,L.} see~Kalinichenko~L.\,A.&&\\[.23pt]
\Avtors{Ponomareva~N.\,V.} see~Kalinichenko~L.\,A.&&\\[.23pt]
\Avtors{Popkova~N.\,A.} see~Zatsman~I.\,M.&&\\[.23pt]
\Avtors{Pozanenko~A.\,S.} see~Kalinichenko~L.\,A.&&\\[.23pt]
\Avtors{Razumchik~R.\,V.} see~Konovalov~M.\,G.&&\\[.23pt]
\Avtors{Ronzhin~A.\,F.} see~Melnikov~A.\,K.&&\\[.23pt]
\Avtors{Rumovskaya~S.\,B.} see~Kirikov~I.\,A.&&\\[.23pt]
\Avtors{Rumovskaya~S.\,B.} see~Kirikov~I.\,A.&&\\[.23pt]
\Avtors{Rumovskaya~S.\,B.} see~Kolesnikov A.\,V.&&\\[.23pt]
\Avtors{Samouylov~K.\,E.} see~Gaidamaka~Yu.\,V.&&\\[.23pt]
\Avtors{Samouylov~K.\,E.} see~Naumov~V.\,A.&&\\[.23pt]
\Avtors{Serebryanskii~S.\,M.} see~Tyrsin~A.\,N.&&\\[.23pt]
\Avtors{Seyful-Mulyukov~R.\,B.} see~Callaos~N.\,K.&&\\[.23pt]
\Avtors{Shestakov~O.\,V.} Statistical properties of the denoising method
based on the stabilized hard\linebreak
\\[-12pt]
\hspace*{23pt}thresholding&2&65--69\\[.23pt]
\Avtors{Shestakov~O.\,V.} The strong law of large numbers for the risk
estimate in the problem of\linebreak
\\[-12pt]
\hspace*{23pt}tomographic image reconstruction from
projections with a correlated noise&3&41--45\\[.23pt]
\Avtors{Shestakov~O.\,V.} see~Zakharova~T.\,V.&&\\[.23pt]
\Avtors{Shnurkov~P.\,V., Gorshenin~A.\,K., and Belousov~V.\,V.}
Analytical solution of~the~optimal control\linebreak
\\[-12pt]
\hspace*{23pt}task of~a~semi-Markov
process with~finite set of~states&4&72--88\\[.23pt]
\Avtors{Shnurkov~P.\,V., Zasypko~V.\,V., Belousov~V.\,V., and
Gorshenin~A.\,K.} Development of the algorithm of numerical solution
of the optimal investment control problem\linebreak
\\[-12pt]
\hspace*{23pt}in the closed dynamical model of three-sector economy&1&82--95\\[.23pt]
\Avtors{Shorgin~S.\,Ya.} see~Gaidamaka~Yu.\,V.&&\\[.23pt]
\Avtors{Shorgin~V.\,S.} see~Agalarov~Ya.\,M.&&\\[.23pt]
\Avtors{Shubnikov~S.\,K.} see~Minin~V.\,A.&&\\[.23pt]
\Avtors{Sidorkin~I.\,I.} see~Arkhipov~O.\,P.&&\\[.23pt]
\Avtors{Sinitsyn~I.\,N.} Analytical modeling of processes in stochastic
systems with complex fractional\linebreak
\\[-12pt]
\hspace*{23pt}order Bessel nonlinearities&3&55--65\\[.23pt]
\Avtors{Sinitsyn~I.\,N.} Orthogonal supoptimal filters for nonlinear
stochastic systems on manifolds&1&34--44\\[.23pt]
\Avtors{Sinitsyn~I.\,N.\ and Korepanov~E.\,R.} Normal Pugachev
conditionally-optimal filters and extra-\linebreak
\\[-12pt]
\hspace*{23pt}polators for state linear stochastic systems&2&14--23\\[.23pt]
\Avtors{Sinitsyn~I.\,N.\ and Sinitsyn~V.\,I.} Analytical modeling of
distributions in stochastic systems on\linebreak
\\[-12pt]
\hspace*{23pt}manifolds based on ellipsoidal approximation&1&45--55\\[.23pt]
\Avtors{Sinitsyn~I.\,N., Sinitsyn~V.\,I., and
Korepanov~E.\,R.} Ellipsoidal suboptimal filters for nonlinear\linebreak
\\[-12pt]
\hspace*{23pt}stochastic systems on manifolds&2&24--35\\[.23pt]
\Avtors{Sinitsyn~V.\,I.} see~Sinitsyn~I.\,N.&&\\[.23pt]
\Avtors{Sinitsyn~V.\,I.} see~Sinitsyn~I.\,N.&&\\[.23pt]
\Avtors{Skvortsov~N.\,A.} see~Stupnikov~S.\,A.&&\\[.23pt]
\Avtors{Sokolov~I.\,A.} see~Chertok~A.\,V.&&\\
\end{tabular}
}
\pagebreak

\def\leftfootline{\small{\textbf{\thepage}
\hfill INFORMATIKA I EE PRIMENENIYA~--- INFORMATICS AND APPLICATIONS\ \ \ 2016\
\ \ volume~10\ \ \ issue\ 4}
}%
 \def\rightfootline{\small{INFORMATIKA I EE PRIMENENIYA~---
INFORMATICS AND APPLICATIONS\ \ \ 2016\ \ \ volume~10\ \ \ issue\ 4
\hfill \textbf{\thepage}}}

\def\leftkol{2016 AUTHOR INDEX} % ENGLISH ABSTRACTS}

\def\rightkol{2016 AUTHOR INDEX} %ENGLISH ABSTRACTS}


{\tabcolsep=3pt
\begin{tabular}{p{382pt}cc}
&\textbf{Issue} & \textbf{Page}\\[6pt]
\Avtors{Sopin~E.\,S.} see~Gaidamaka~Yu.\,V.&&\\
\Avtors{Strijov~V.\,V.} see~Goncharov~A.\,V.&&\\
\Avtors{Strijov~V.\,V.} see~Isachenko~R.\,V.&&\\
\Avtors{Strijov~V.\,V.} see~Karasikov~M.\,E.&&\\
\Avtors{Stupnikov~S.\,A., Briukhov~D.\,O., and Skvortsov~N.\,A.}
Co-lending systemic risk analysis over\linebreak
\\[-12pt]
\hspace*{23pt}heterogeneous data collections&1&23--33\\
\Avtors{Stupnikov~S.\,A.} see~Kalinichenko~L.\,A.&&\\
\Avtors{Suchkov~A.\,P.} see~Zatsarinny~A.\,A.&&\\
\Avtors{Timonina~E.\,E.} see~Grusho~A.\,A.&&\\
\Avtors{Titova~A.\,I.} see~Kudryavtsev~A.\,A.&&\\
\Avtors{Turlikov~A.\,M.} see~Ometov~A.\,Ya.&&\\
\Avtors{Tyrsin~A.\,N.\ and Serebryanskii~S.\,M.} Recognition of
dependences on the basis of inverse\linebreak
\\[-12pt]
\hspace*{23pt}mapping&2&58--64\\
\Avtors{Ulyanov~V.\,V.} see~Markov~A.\,S.&&\\
\Avtors{Ushakov~V.\,G.} Queueing system with working vacations and
hyperexponential input stream&2&92--97\\
\Avtors{Ushakov~V.\,G.} see~Leontyev~N.\,D.&&\\
\Avtors{Volnova~A.\,A.} see~Kalinichenko~L.\,A.&&\\
\Avtors{Yakovlev~O.\,A.\ and Gasilov~A.\,V.} Speeded-up stereo
matching using geodesic support weights&3&\hphantom{1}98--104\\
\Avtors{Zabezhailo~M.\,I.} see~Grusho~A.\,A.&&\\
\Avtors{Zabezhailo~M.\,I.} see~Grusho~A.\,A.&&\\
\Avtors{Zakharova~T.\,V.\ and Shestakov~O.\,V.} Precision analysis of
wavelet processing of aerodynamic\linebreak
\\[-12pt]
\hspace*{23pt}flow patterns&3&46--54\\
\Avtors{Zalizniak~Anna~A.\ and Kruzhkov~M.\,G.} Database
of~Russian impersonal verbal constructions&4&132--141\\
\Avtors{Zasypko~V.\,V.} see~Shnurkov~P.\,V.&&\\
\Avtors{Zatsarinny~A.\,A.\ and Suchkov~A.\,P.} Systems engineering
approaches to~the~establishment of\linebreak
\\[-12pt]
\hspace*{23pt}a~system for~decision support based
on~situational analysis&4&105--113\\
\Avtors{Zatsarinny~A.\,A.} see~Grusho~A.\,A.&&\\
\Avtors{Zatsman~I.\,M., Inkova~O.\,Yu., Kruzhkov~M.\,G., and
Popkova~N.\,A.} Representation of cross-\linebreak
\\[-12pt]
\hspace*{23pt}lingual knowledge about
connectors in supracorpora databases&1&106--118\\
\Avtors{Zatsman~I.\,M.} see~Minin~V.\,A.&&\\
\Avtors{Zeifman~A.\,I.} see~Korolev~V.\,Yu.&&\\
\Avtors{Zeifman~A.\,I.} see~Korolev~V.\,Yu.&&\\
\end{tabular}
}

%\thispagestyle{myheadings}
\def\leftfootline{\small{\textbf{\thepage}
\hfill INFORMATIKA I EE PRIMENENIYA~--- INFORMATICS AND APPLICATIONS\ \ \ 2016\
\ \ volume~10\ \ \ issue\ 4}
}%
 \def\rightfootline{\small{INFORMATIKA I EE PRIMENENIYA~---
INFORMATICS AND APPLICATIONS\ \ \ 2016\ \ \ volume~10\ \ \ issue\ 4
\hfill \textbf{\thepage}}}

 \label{end\stat}

\newpage

%\def\stat{rekl}
%\label{preobr}

%\def\tit{АКАДЕМИК ПУГАЧЁВ  ВЛАДИМИР СЕМЁНОВИЧ\\
%25.03.1911--25.03.1998}


%   \vspace*{-48pt}
%   \begin{center}\LARGE
%Академик Пугачёв  Владимир Семёнович\\ (25.03.1911--25.03.1998)
%   \end{center}
   
   %\vspace*{2.5mm}
   
   \begin{center}

{\prgsh\LARGE
ОБЪЯВЛЕНИЯ О КОНФЕРЕНЦИЯХ}

\end{center}
%\hrule

\vspace*{6pt}

   
   \vspace*{10mm}
   
   \thispagestyle{empty}

\noindent
\begin{tabular}{cc}
%\begin{center}
\multicolumn{1}{c}{\raisebox{-40pt}[0pt][0pt]{\mbox{%
\epsfxsize=33mm
\epsfbox{vspu.eps}
}}}
%\end{center}
&
\tabcolsep=0pt\begin{tabular}{c}
{\prg{\Large\textbf{XII Всероссийское совещание}}}\\[6pt]
{\prg{\Large\textbf{по проблемам управления}}}\\[12pt]
{\prg{\large 16--19 июня 2014~г.}}\\[6pt] 
{\prg{\large Институт проблем управления имени В.\,А.~Трапезникова РАН}}\\[6pt]
{\prg{\large Москва, Россия}}
\end{tabular}
\end{tabular}

\vspace*{60pt}

     
 { %\large    
 XII Всероссийское совещание по проблемам управления (ВСПУ XII), посвященное 75-летию 
Института проблем управления (ИПУ) имени В.\,А.~Трапезникова РАН, проводится 16--19~июня 
2014~г.\ 
в ИПУ РАН (г.~Москва, Россия). ВСПУ XII организуется ИПУ РАН при поддержке РФФИ, Отделения 
энергетики, машиностроения, механики и процессов управления Российской академии наук, 
Российского 
национального комитета по автоматическому управлению, Академии навигации и управ\-ле\-ния 
движением, 
Научного совета РАН по комплексным проблемам управления и автоматизации, Совета по 
мехатронике и робототехнике РАН. Официальный язык Совещания~--- русский.

\vspace*{24pt}
     
     \textbf{Направления работы}
     \begin{enumerate}[1.]
\item Теория систем управления
\item Управление подвижными объектами и навигация
\item Интеллектуальные системы управления
\item Управление в промышленности, транспортом и логистикой
\item Управление системами междисциплинарной природы
\item Средства измерения, вычислений и контроля в управлении
\item Системный анализ и принятие решений в задачах управления
\item Информационные технологии в управлении
\item Проблемы образования в области управления: современное содержание и технологии обучения
\end{enumerate}

\vspace*{24pt}

     Подробная информация о Совещании находится на сайте {\sf http://vspu2014.ipu.ru}. Срок 
окончательной подачи докладов через систему подачи докладов на сайте~--- \textbf{30~ноября} 
2013~г.
}

%\include{rekl-1}

%\end{document}

%   \vspace*{-48pt}

\begin{center}
\vspace*{6pt}
\mbox{%
\epsfxsize=53.502mm
\epsfbox{foto-1.eps}
}
\end{center}

\vspace*{6pt} %Академик


   \begin{center}
\fbox{\Large\textbf{Профессор Игорь Алексеевич Ушаков}}\\[12pt]
\textbf{\large 22.01.1935--27.02.2015}
   \end{center}


   %\vspace*{2.5mm}

   \vspace*{5mm}

   \thispagestyle{empty}

%\

%\vspace*{-12pt}


Редакционный совет и редакционная коллегия журнала <<Информатика и~её применения>> с~глубоким прискорбием извещают, что 27~февраля 2015~г.\ после тяжелой
и~продолжительной болезни скончался Игорь Алексеевич Ушаков~--- доктор технических наук, профессор, член редколлегии журнала <<Информатика и ее применения>>.

Игорь Алексеевич Ушаков окончил Московский авиационный институт, в~1963~г.\ защитил кандидатскую, а~в~1968~г.~--- докторскую диссертацию. С~1958 по 1989~гг.\ работал в~ряде научно-исследовательских организаций СССР, в~том числе руководил отделами в~НИИ АА и~ВЦ АН СССР; с 1969 по 1989 гг. преподавал в~МФТИ (был профессором, а~затем заведующим кафедрой) и~в~МЭИ. С~1989~г.~---- в~США: являлся профессором университета Дж.\ Вашингтона, университета Дж.\ Мэйсона и~Калифорнийского университета, сотрудником компаний MCI, Qualcomm и Hughes.

И.\,А.~Ушаков с момента основания журнала <<Надежность и~контроль качества>> был заместителем ответственного редактора, а~затем на протяжении многих лет членом редколлегии. В~2006~г.\ основал электронный международный журнал ``Reliability: Theory \& Application'', главным редактором которого оставался до конца жизни.

Учебниками и справочниками по теории надежности, написанными И.\,А.~Ушаковым, пользовались и~пользуются несколько поколений ученых и~специалистов в~разных странах мира.

Игорь Алексеевич всегда уделял огромное внимание работе с~молодежью; более~50 его учеников защитили докторские и~кандидатские диссертации.

И.\,А.~Ушаков вел активную научно-про\-све\-ти\-тель\-скую деятельность. В~частности, он был одним из организаторов и~руководителей Московского кабинета качества и~надежности при Политехническом музее (целью этого Кабинета было оказание консультаций работникам промышленных предприятий и~чтение курсов лекций для инженеров, занимающихся проблемой надежности). Находясь в~США, И.\,А.~Ушаков создал международный ин\-тер\-нет-фо\-рум им.\ Б.\,В.~Гнеденко, объединивший около~400~видных специалистов по приложениям теории вероятностей и~математической статистики, преимущественно в~об\-ласти теории надежности и~анализа риска, из десятков стран мира; коллективным членов этого Форума является и~наш журнал. Цели Форума~--- содействие контактам между специалистами из разных стран, организация обмена профессиональными 
новостями и~информацией (новые публикации, предстоящие события и~др.). Также необходимо отметить большое число на\-уч\-но-по\-пу\-ляр\-ных работ, опубликованных И.\,А.~Ушаковым.

И.\,А.~Ушаков обладал большим личным обаянием, имел широкий круг интересов. Все знавшие И.\,А.~Ушакова всегда будут помнить его как замечательного ученого и~прекрасного человека.

\bigskip

Редакционный совет и редакционная коллегия журнала <<Информатика и~её применения>> 
выражают глубокие соболезнования родным и близким покойного, всем, кто его знал и~работал с~ним.



%\end{document}

%\include{IPPM-25}

\def\stat{cont-rus}
{%\hrule\par
%\vskip 7pt % 7pt
\vspace*{-24pt}
\raggedleft\Large \bf%\baselineskip=3.2ex
Правила подготовки рукописей  для публикации в журнале
<<Информатика~и~её~применения>> \vskip 8pt
    \hrule
    \par
\vskip 14pt plus 6pt minus 3pt }

\label{st\stat}

\def\tit{\ }

\def\aut{\ }
\def\auf{\ }

\def\leftkol{\ }
% Правила подготовки рукописей  для публикации в журнале
%<<Информатика и её применения>>

\def\rightkol{\ }
%Правила подготовки рукописей  для публикации в журнале
%<<Информатика и её применения>>}


\titele{\tit}{\aut}{\auf}{\leftkol}{\rightkol}


\vspace*{-60pt}
{ %\small

Журнал <<Информатика и её применения>>
публикует теоретические, обзорные и дискуссионные статьи,
посвященные научным исследованиям и разработкам в области
информатики и ее приложений.

Журнал издается на русском языке. По специальному решению
редколлегии отдельные статьи могут печататься на английском языке.

Тематика журнала охватывает следующие направления:
\begin{itemize}
\item теоретические основы информатики;\\[-15pt]
      \item
математические методы исследования сложных систем и процессов;\\[-15pt]
           \item
информационные системы и сети;\\[-15pt]
                \item
информационные технологии;\\[-15pt]
                     \item
архитектура и программное обеспечение вычислительных комплексов и сетей.\\[-15pt]
\end{itemize}


\noindent
\begin{enumerate}[1.]
\item В журнале печатаются статьи, содержащие результаты, ранее не опубликованные и
не предназначенные к одновременной публикации в других изданиях.

%Публикация не должна нарушать закон об авторских правах.
Публикация предоставленной автором(ами) рукописи не должна нарушать 
положений глав~69, 70 раздела~VII части~IV Гражданского кодекса, 
которые определяют права на результаты интеллектуальной деятельности 
и~средства индивидуализации, в~том числе авторские права, в~РФ.

Ответственность за нарушение авторских прав, в~случае предъявления претензий к~редакции журнала,  
несут авторы статей.



Направляя рукопись в редакцию, авторы сохраняют свои права на данную
рукопись и при этом передают учредителям и редколлегии журнала неисключительные права на
издание статьи на русском языке 
(или на языке статьи, если он отличен от рус\-ско\-го) и~на перевод ее на английский
язык, а~также на
ее распространение в России и за рубежом. 
Каждый автор должен представить в~редакцию подписанный 
с~его стороны <<Лицензионный договор о~передаче неисключительных прав 
на использование произведения>>, текст которого размещен по адресу 
{\sf http://www.ipiran.ru/publications/licence.doc}. 
Этот договор может быть пред\-став\-лен в~бумажном (в~2-х экз.)\ 
или в~электронном виде (отсканированная копия заполненного и~подписанного документа).




Редколлегия вправе запросить у авторов экспертное заключение о возможности
пуб\-ли\-ка\-ции пред\-став\-лен\-ной статьи в открытой печати.\\[-13.5pt]

\item К статье прилагаются данные автора (авторов) (см.\ п.~8). При наличии нескольких
авторов указывается фамилия автора, ответственного за переписку с редакцией.\\[-13.5pt]

\item Редакция журнала осуществляет экспертизу присланных статей в соответствии с
принятой в журнале процедурой рецензирования.

Возвращение рукописи на доработку не означает ее принятия к печати.

Доработанный вариант с ответом на замечания рецензента необходимо прислать в
редакцию.\\[-13.5pt]

\item Решение редколлегии о публикации статьи или ее отклонении сообщается авторам.

Редколлегия может также направить авторам текст рецензии на их статью. Дискуссия по
поводу отклоненных статей не ведется.\\[-13.5pt]

%\pagebreak

\item Редактура статей высылается авторам для просмотра. Замечания к редактуре должны
быть присланы авторами в кратчайшие сроки.\\[-13.5pt]

\item Рукопись предоставляется в электронном виде в форматах MS WORD (.doc или
.docx) или \LaTeX\  (.tex), дополнительно~--- в формате .pdf, на дискете, лазерном диске
или электронной почтой. Предоставление бумажной рукописи необязательно.\\[-13.5pt]

\item При подготовке рукописи в MS Word рекомендуется использовать следующие
настройки.

Параметры страницы:
формат~--- А4; ориентация~--- книжная; поля (см): внутри~--- 2,5, снаружи~--- 1,5,
сверху~--- 2, снизу~--- 2, от края до нижнего колонтитула~--- 1,3.

Основной текст: стиль~--- <<Обычный>>, шрифт~--- Times New Roman, размер~---
14~пунк\-тов, абзацный отступ~--- 0,5~см, 1,5~интервала, выравнивание~--- по ширине.

\pagebreak

\def\leftkol{Правила подготовки рукописей  для публикации в журнале
<<Информатика и её применения>>}

\def\rightkol{Правила подготовки рукописей  для публикации в журнале
<<Информатика и её применения>>}



Рекомендуемый объем рукописи~--- не свыше 10~страниц указанного формата.
При превышении указанного объема редколлегия вправе потребовать от 
автора сокращения объема рукописи.


Сокращения слов, помимо стандартных, не допускаются. Допускается минимальное
количество аббревиатур.


Все страницы рукописи нумеруются.

Шаблоны оформления представлены в интернете:

\noindent
 {\sf
http://www.ipiran.ru/journal/template\_iiep\_ssi\_2024.zip}\\[-14pt]

\item Статья должна содержать следующую информацию на {\bfseries\textit{русском и
английском языках}}:\\[-16pt]

\begin{itemize}
\item название статьи;\\[-15pt]
\item Ф.И.О.\ авторов, на английском можно только имя и фамилию;\\[-15pt]
\item место работы, с указанием почтового адреса организации и электронного адреса каждого
автора;\\[-15pt]
\item сведения об авторах, в соответствии с форматом, образцы которого
представлены на страницах:



\def\leftfootline{\small{\textbf{\thepage}
\hfill ИНФОРМАТИКА И ЕЁ ПРИМЕНЕНИЯ\ \ \ том\ 18\ \ \ выпуск\ 3\ \ \ 2024}
}%
 \def\rightfootline{\small{ИНФОРМАТИКА И ЕЁ ПРИМЕНЕНИЯ\ \ \ том\ 18\ \ \ выпуск\ 3\ \ \ 2024
\hfill \textbf{\thepage}}}



{\sf http://www.ipiran.ru/journal/issues/2013\_07\_01/authors.asp} и

{\sf http://www.ipiran.ru/journal/issues/2013\_07\_01\_eng/authors.asp};
\item аннотация (не менее 100~слов на каждом из языков). Аннотация~--- это краткое
резюме работы, которое может публиковаться отдельно. Она является основным
источником информации в~ин\-фор\-ма\-ци\-он\-ных системах и базах данных. Английская
аннотация должна быть оригинальной, может не быть дословным переводом русского
текста и должна быть написана хорошим английским языком. В~аннотации не должно
быть ссылок на литературу и, по возможности, формул;\\[-15pt]
\item ключевые слова~--- желательно из принятых в мировой
на\-уч\-но-тех\-ни\-че\-ской литературе тематических тезаурусов. Предложения не
могут быть ключевыми словами;\\[-15pt]
\item источники финансирования работы (ссылки на гранты, проекты,
поддерживающие организации и~т.\,п.).
\end{itemize}



%\pagebreak

\item  Требования к спискам литературы.\\[-14pt]

Ссылки на литературу в тексте статьи нумеруются (в квадратных скобках) и
располагаются в каждом из списков литературы в порядке  первых упоминаний. Если источник имеет DOI и/или EDN,
то их необходимо указывать.

Списки литературы представляются в двух вариантах:\\[-14pt]


\noindent
\begin{enumerate}[(1)]
\item \textbf{Список литературы к русскоязычной части}. Русские и английские
работы~---  на языке и в алфавите оригинала;\\[-14.5pt]
\item  \textbf{References}. Русские работы и работы на других языках~--- в латинской
транслитерации с переводом на английский язык; английские работы и работы на других
языках~--- на языке оригинала.
\end{enumerate}

Необходимо для составления списка ``References'' пользоваться размещенной на сайте
{\sf http://www. translit.net/ru/bgn/} бесплатной программой транслитерации русского
 текста в~латиницу. %, при этом в~за\-клад\-ке <<варианты\ldots>> следует выбратьопцию BGN.

Список литературы ``References'' приводится полностью отдельным блоком, повторяя все
позиции из списка литературы к русскоязычной части, независимо от того, имеются или
нет в нем иностранные источники. Если в списке литературы к русскоязычной части есть
ссылки на иностранные публикации, набранные латиницей, они полностью повторяются в
списке ``References''.

Ниже приведены примеры ссылок на различные виды публикаций в списке ``References''.

\def\leftfootline{\small{\textbf{\thepage}
\hfill ИНФОРМАТИКА И ЕЁ ПРИМЕНЕНИЯ\ \ \ том\ 18\ \ \ выпуск\ 3\ \ \ 2024}
}%
 \def\rightfootline{\small{ИНФОРМАТИКА И ЕЁ ПРИМЕНЕНИЯ\ \ \ том\ 18\ \ \ выпуск\ 3\ \ \ 2024
\hfill \textbf{\thepage}}}

{\small

\noindent
\textbf{Описание статьи из журнала:}

\Aue{Zagurenko, A.\,G., V.\,A.~Korotovskikh, A.\,A.~Kolesnikov, A.\,V.~Timonov, and D.\,V.~Kardymon}. 2008.
Tekhniko-ekonomicheskaya optimizatsiya dizayna gidrorazryva plasta [Technical and
economic optimization of the design
of hydraulic fracturing]. \textit{Neftyanoe hozyaystvo} [\textit{Oil Industry}] 11:54--57.

\Aue{Zhang, Z., and D.~Zhu}. 2008. Experimental research on the localized
electrochemical micromachining. \textit{Russ. J.~Electrochem.}  44(8):926--930.
{\sf doi:10.1134/S1023193508080077}.

\noindent
\textbf{Описание статьи из электронного журнала:}

\Aue{Swaminathan, V., E.~Lepkoswka-White, and B.\,P.~Rao}. 1999. Browsers or buyers in cyberspace? An
investigation of electronic factors influencing electronic exchange. \textit{JCMC}
5(2). Available at: {\sf http://www.ascusc.org/jcmc/vol5/issue2/} (accessed April~28, 2011).

\def\leftkol{Правила подготовки рукописей  для публикации в журнале
<<Информатика и её применения>>}

\def\rightkol{Правила подготовки рукописей  для публикации в журнале
<<Информатика и её применения>>}


\noindent
\textbf{Описание статьи из продолжающегося издания (сборника трудов):}

\Aue{Astakhov, M.\,V., and T.\,V.~Tagantsev}. 2006. Eksperimental'noe
issledovanie prochnosti soedineniy ``stal'--kompozit'' [Experimental study of
the strength of joints ``steel--composite'']. \textit{Trudy MGTU
``Matematicheskoe modelirovanie slozhnykh tekh\-ni\-che\-skikh sistem''}
[\textit{Bauman MSTU ``Mathematical Modeling of Complex Technical
Systems'' Proceedings}]. 593:125--130.


\pagebreak



\noindent
\textbf{Описание материалов конференций:}

\Aue{Usmanov, T.\,S., A.\,A.~Gusmanov, I.\,Z.~Mullagalin, R.\,Ju.~Muhametshina, A.\,N.~Chervyakova, and
A.\,V.~Sveshnikov}. 2007. Osobennosti proektirovaniya razrabotki mestorozhdeniy
s primeneniem gidrorazryva
plasta [Features of the design of field development with the use of hydraulic fracturing].
\textit{Trudy 6-go
Mezhdu\-na\-rod\-no\-go Simpoziuma ``Novye resursosberegayushchie tekhnologii nedropol'zovaniya i povysheniya
neftegazootdachi''} [\textit{6th  Symposium (International) ``New Energy Saving Subsoil Technologies and
the Increasing of the Oil and Gas Impact'' Proceedings}]. Moscow. 267--272.



\def\leftfootline{\small{\textbf{\thepage}
\hfill ИНФОРМАТИКА И ЕЁ ПРИМЕНЕНИЯ\ \ \ том\ 18\ \ \ выпуск\ 3\ \ \ 2024}
}%
 \def\rightfootline{\small{ИНФОРМАТИКА И ЕЁ ПРИМЕНЕНИЯ\ \ \ том\ 18\ \ \ выпуск\ 3\ \ \ 2024
\hfill \textbf{\thepage}}}



\noindent
\textbf{Описание книги (монографии, сборники):}



Lindorf, L.\,S., and L.\,G.~Mamikoniants, eds. 1972.
\textit{Ekspluatatsiya turbogeneratorov s neposredstvennym
okhlazhdeniem} [\textit{Operation of turbine generators with direct cooling}].
Moscow: Energy Publs. 352~p.


\Aue{Latyshev, V.\,N.} 2009. \textit{Tribologiya rezaniya. Kn.~1: Friktsionnye protsessy
pri rezanii metallov}
[\textit{Tribology of cutting. Vol.~1: Frictional processes in metal cutting}]. Ivanovo: Ivanovskii
State Univ. 108~p.

\def\leftkol{Правила подготовки рукописей  для публикации в журнале
<<Информатика и её применения>>}

\def\rightkol{Правила подготовки рукописей  для публикации в журнале
<<Информатика и её применения>>}

\noindent
\textbf{Описание переводной книги}
(в списке литературы к русскоязычной части необходимо указать:~/ Пер.\ с англ.~---
после названия книги, а в конце ссылки указать оригинал книги в круглых скобках):
\begin{enumerate}[1.]
\item  В русскоязычной части:

\def\leftfootline{\small{\textbf{\thepage}
\hfill ИНФОРМАТИКА И ЕЁ ПРИМЕНЕНИЯ\ \ \ том\ 18\ \ \ выпуск\ 3\ \ \ 2024}
}%
 \def\rightfootline{\small{ИНФОРМАТИКА И ЕЁ ПРИМЕНЕНИЯ\ \ \ том\ 18\ \ \ выпуск\ 3\ \ \ 2024
\hfill \textbf{\thepage}}}

\Au{Тимошенко С.\,П., Янг Д.\,Х., Уивер~У.}
Колебания в инженерном деле~/ Пер.\ с англ.~--- М.: Машиностроение, 1985. 472~с.
(\Au{Timoshenko~S.\,P., Young~D.\,H., Weaver~W.}
Vibration problems in engineering.~--- 4th ed.~--- New York, NY, USA: Wiley, 1974. 521~p.)\\[-13.5pt]
\item  В англоязычной части:

\Aue{Timoshenko, S.\,P., D.\,H.~Young, and W.~Weaver}.
1974. \textit{Vibration problems in engineering}. 4th ed. New York: 
Wiley. 521~p.
\end{enumerate}

\vspace*{-3pt}


\noindent
\textbf{Описание неопубликованного документа:}


\Aue{Latypov, A.\,R., M.\,M.~Khasanov, and V.\,A.~Baikov}.
2004 (unpubl.). Geologiya i~dobycha (NGT GiD) [Geology and production (NGT GiD)]. Certificate on official registration of the computer program
No.\,2004611198. 

\noindent
\textbf{Описание интернет-ресурса:}


Pravila tsitirovaniya istochnikov [Rules for the citing of sources]. Available at: {\sf
http://www.scribd.com/doc/1034528/} (accessed February~7, 2011).

%\pagebreak

\noindent
\textbf{Описание диссертации или автореферата диссертации:}

\Aue{Semenov, V.\,I.}
2003. Matematicheskoe modelirovanie plazmy v sisteme kompaktnyy tor [Mathematical
modeling of the plasma in the compact torus].  Moscow.  D.Sc.\ Diss. 272~p.

\Aue{Kozhunova, O.\,S.} 2009. Tekhnologiya razrabotki semanticheskogo
slovarya informatsionnogo monitoringa [Technology of development of
semantic dictionary of information monitoring system].  Moscow: IPI RAN. PhD Thesis. 23~p.


\noindent
\textbf{Описание ГОСТа:}

GOST 8.586.5-2005. 2007. Metodika vypolneniya izmereniy. Izmerenie raskhoda i~kolichestva zhidkostey i~gazov
s~pomoshch'yu standartnykh suzhayushchikh ustroystv [Method of measurement.
Measurement of flow rate and volume of liquids and gases by means of orifice devices]. Moscow:
Standardinform  Publs. 10~p.

\noindent
\textbf{Описание патента:}

\Aue{Bolshakov, M.\,V., A.\,V.~Kulakov, A.\,N.~Lavrenov, and M.\,V.~Palkin}.
2006. Sposob orientirovaniya po krenu letatel'nogo
apparata s opti\-che\-skoy golovkoy
samonavedeniya [The way to orient on the roll of aircraft with optical homing head].
Patent RF No.\,2280590.
}

\item Присланные в редакцию материалы авторам не возвращаются.\\[-13.5pt]

\item При отправке файлов по электронной почте просим придерживаться следующих
правил:
\begin{itemize}
\item указывать в поле subject (тема) название журнала и фамилию автора;\\[-13.5pt]
\item указывать в тексте письма название статьи, авторов и~журнал, в~который направляется статья;\\[-13.5pt]
\item использовать attach (присоединение);\\[-13.5pt]
\item в состав электронной версии статьи должны входить: файл, содержащий текст
статьи, и файл(ы), содержащий(е) иллюстрации.\\[-13.5pt]
\end{itemize}

\item Журнал <<Информатика и её применения>> является некоммерческим изданием.
Плата за публикацию не взимается, гонорар авторам не выплачивается.
\end{enumerate}



\def\leftfootline{\small{\textbf{\thepage}
\hfill ИНФОРМАТИКА И ЕЁ ПРИМЕНЕНИЯ\ \ \ том\ 18\ \ \ выпуск\ 3\ \ \ 2024}
}%
 \def\rightfootline{\small{ИНФОРМАТИКА И ЕЁ ПРИМЕНЕНИЯ\ \ \ том\ 18\ \ \ выпуск\ 3\ \ \ 2024
\hfill \textbf{\thepage}}}


\vspace*{-1mm}

\begin{center}

\textbf{Адрес редакции журнала <<Информатика и её применения>>:} \\




Москва 119333, ул.~Вавилова, д.~44, корп.~2, ФИЦ ИУ РАН\\[-10pt]

\

Тел.: +7\,(499)\,135-86-92\ \ Факс:  +7\,(495)\,930-45-05\\[-10pt]

 \

e-mail:   {\sf iiep@frccsc.ru} (Стригина Светлана Николаевна)\\[-10pt]

\

{\sf http://www.ipiran.ru/journal/issues/}
\end{center}
}


\def\leftkol{Правила подготовки рукописей  для публикации в журнале
<<Информатика и её применения>>}

\def\rightkol{Правила подготовки рукописей  для публикации в журнале
<<Информатика и её применения>>}


\def\leftfootline{\small{\textbf{\thepage}
\hfill ИНФОРМАТИКА И ЕЁ ПРИМЕНЕНИЯ\ \ \ том\ 18\ \ \ выпуск\ 3\ \ \ 2024}
}%
 \def\rightfootline{\small{ИНФОРМАТИКА И ЕЁ ПРИМЕНЕНИЯ\ \ \ том\ 18\ \ \ выпуск\ 3\ \ \ 2024
\hfill \textbf{\thepage}}} 
\def\stat{podg-e}
{%\hrule\par
%\vskip 7pt % 7pt
\vspace*{-24pt}
\raggedleft\Large \bf%\baselineskip=3.2ex
Requirements for manuscripts submitted to Journal
``Informatics~and~Applications'' \vskip 8pt
    \hrule
    \par
\vskip 21pt plus 6pt minus 3pt }

\label{st\stat}

\def\tit{\ }

\def\aut{\ }
\def\auf{\ }

\def\leftkol{\ }

\def\rightkol{\ }
%Requirements for manuscripts submitted to Journal
%``Informatics~and~Applications''}

\titele{\tit}{\aut}{\auf}{\leftkol}{\rightkol}

\def\leftfootline{\small{\textbf{\thepage}
\hfill INFORMATIKA I EE PRIMENENIYA~--- INFORMATICS AND APPLICATIONS\ \ \ 2019\
\ \ volume~13\ \ \ issue\ 4}
}%
 \def\rightfootline{\small{INFORMATIKA I EE PRIMENENIYA~--- INFORMATICS AND APPLICATIONS\ \ \ 2019\ \ \ volume~13\ \ \ issue\ 4
\hfill \textbf{\thepage}}}

\vspace*{-60pt}

{\small

\noindent
Journal ``Informatics and Applications'' (Inform.\ Appl.)
publishes theoretical, review, and discussion
articles on the research and development in the
field of informatics and its applications.

The journal is published in Russian.
By a special decision of the editorial
board, some articles can be published in English.


The topics covered include the following areas:
\begin{itemize}
               \item
     theoretical fundamentals of informatics; \\[-14pt]
\item
mathematical methods for studying complex systems and processes; \\[-14pt]
\item
information systems and networks;\\[-14pt]
\item
information technologies; and \\[-14pt]
\item
architecture and software of computational complexes and networks. \\[-14pt]
\end{itemize}

\noindent
\begin{enumerate}[1.]
\item The Journal publishes original articles which have not been published before and are not
intended for simultaneous publication in other editions. An article submitted to the Journal must not violate the
Copyright law. Sending the manuscript to the Editorial Board, the authors retain all rights of the
owners of the manuscript and transfer the nonexclusive rights to publish the article in Russian
(or the language of the article, if not Russian) and its distribution in Russia and abroad to the
Founders and the Editorial Board. Authors should submit a letter to the Editorial Board in the
following form:

{\bfseries\textit{Agreement on the transfer of rights to publish:}}

``\textit{We, the undersigned authors of the manuscript ``\ldots'', pass to the
Founder and the Editorial Board of the Journal ``Informatics and Applications''
the nonexclusive right to publish the manuscript of the article in Russian (or
in English) in both print and electronic versions of the Journal. We affirm
that this publication does not violate the Copyright of other persons or
organizations.}

\textit{Author(s) signature(s): (name(s), address(es), date).}

This agreement should be submitted in paper form or in the form of a scanned copy (signed by
the authors).


%The Editorial Board has the right to request from the authors an official expert conclusion that
%the submitted article has no secret data prohibited for publication. \\[-13.5pt]
\item
A submitted article should be attached with \textbf{the data on the author(s)} (see item~8). If
there are several authors, the contact person should be indicated who is responsible for
correspondence with the Editorial Board and other authors about revisions and final approval
of the proofs.\\[-13.5pt]

\item The Editorial Board of the Journal examines the article according to the established
reviewing procedure. If the authors receive their article for correction after reviewing, it does not
mean that the article is approved for publication. The corrected article should be sent to the
Editorial Board for the subsequent review and approval.\\[-13.5pt]

\item The decision on the article publication or its rejection is communicated to the authors. The
Editorial Board may also send the reviews on the submitted articles to the authors. Any
discussion upon the rejected articles is not possible.\\[-13.5pt]

\item The edited articles will be sent to the authors for proofread. The comments of the authors
to the edited text of the article should be sent to the Editorial Board as soon as possible.\\[-13.5pt]

\item The manuscript of the article should be presented electronically in the MS WORD (.doc or
.docx) or \LaTeX\ (.tex) formats, and additionally in the .pdf format. All documents
 may be sent
by e-mail or provided on a CD or diskette. A~hard copy submission is not necessary.\\[-13.5pt]

\item The recommended typesetting instructions for manuscript.

Pages parameters: format A4, portrait orientation, document margins (cm): left~--- 2.5, right~---
1.5, above~--- 2.0, below~--- 2.0, footer 1.3.

Text: font~---Times New Roman, font size~--- 14, paragraph indent~--- 0.5, line spacing~--- 1.5,
justified alignment.

The recommended manuscript size: not more than 15~pages of the specified format.
If the specified size exceeded, the editorial board is entitled to require the author
to reduce the manuscript.

Use only standard abbreviations. Avoid  abbreviations in the title and
abstract. The full term for which an abbreviation stands should precede
its first use in the text unless it is a standard unit of measurement.

All pages of the manuscript should be numbered.

The templates for the manuscript typesetting are presented on site: {\sf
http://www.ipiran.ru/journal/template.doc}.\\[-13.5pt]


%\def\leftkol{Requirements for manuscripts submitted to Journal
%``Informatics~and~Applications''}

\item The articles should enclose data both in \textbf{Russian and English}:
\begin{itemize}
\item title;\\[-13.5pt]
\item author's name and surname;\\[-13.5pt]
\item affiliation~--- organization, its address with ZIP code, city, country, and
official e-mail address;\\[-13.5pt]
\item data on authors according to the format: (see site)

{\sf http://www.ipiran.ru/journal/issues/2013\_07\_01/authors.asp}  and

{\sf  http://www.ipiran.ru/journal/issues/2013\_07\_01\_eng/authors.asp};\\[-13.5pt]

\pagebreak

\def\leftfootline{\small{\textbf{\thepage}
\hfill INFORMATIKA I EE PRIMENENIYA~--- INFORMATICS AND APPLICATIONS\ \ \ 2019\
\ \ volume~13\ \ \ issue\ 4}
}%
 \def\rightfootline{\small{INFORMATIKA I EE PRIMENENIYA~--- INFORMATICS AND APPLICATIONS\ \ \ 2019\ \ \ volume~13\ \ \ issue\ 4
\hfill \textbf{\thepage}}}


%\def\leftkol{Requirements for manuscripts submitted to Journal
%``Informatics~and~Applications''}

%\def\rightkol{Requirements for manuscripts submitted to Journal
%``Informatics~and~Applications''}



\item abstract (not less than 100 words) both in Russian and in English. Abstract is a short
summary of the article that can be published separately. The abstract is the
main source of information on the article and it could be included in leading information
systems and data bases. The abstract in English has to be an original text and should
not be an exact translation of the Russian one. Good English is required.
In abstracts, avoid references and formulae;\\[-13.5pt]
\item indexing is performed on the basis of keywords. The use of keywords from the
internationally accepted thematic Thesauri is recommended.

%\def\leftkol{Requirements for manuscripts submitted to Journal
%``Informatics~and~Applications''}

%\def\rightkol{Requirements for manuscripts submitted to Journal
%``Informatics~and~Applications''}

Important! Keywords must not be sentences;
\item Acknowledgments.
\end{itemize}

\item References. Russian references have to be presented both in English translation and Latin
transliteration (refer {\sf http://www.translit.net/ru/bgn/}).

Please take into account the following examples of Russian references appearance:

\noindent
\textbf{Article in journal:}

\Aue{Zhang, Z., and D.~Zhu}. 2008. Experimental research on the localized electrochemical
micromachining.
\textit{Rus. J.~Electrochem.}  44(8):926--930. {\sf doi:10.1134/S1023193508080077}.


\noindent
\textbf{Journal article in electronic format:}

\Aue{Swaminathan, V., E.~Lepkoswka-White, and B.\,P.~Rao}. 1999. Browsers or buyers in
cyberspace? An
investigation of electronic factors influencing electronic exchange. \textit{JCMC}
5(2). Available at: {\sf http://www.ascusc.org/jcmc/vol5/issue2/} (accessed April~28, 2011).




\noindent
\textbf{Article from the continuing publication (collection of works, proceedings):}

\Aue{Astakhov, M.\,V., and T.\,V.~Tagantsev}. 2006. Eksperimental'noe
issledovanie prochnosti soedineniy ``stal'--kompozit'' [Experimental study of
the strength of joints ``steel--composite'']. \textit{Trudy MGTU
``Matematicheskoe modelirovanie slozhnykh tekh\-ni\-che\-skikh sistem''}
[\textit{Bauman MSTU ``Mathematical Modeling of Complex Technical
Systems'' Proceedings}]. 593:125--130.

\def\leftfootline{\small{\textbf{\thepage}
\hfill INFORMATIKA I EE PRIMENENIYA~--- INFORMATICS AND APPLICATIONS\ \ \ 2019\
\ \ volume~13\ \ \ issue\ 4}
}%
 \def\rightfootline{\small{INFORMATIKA I EE PRIMENENIYA~--- INFORMATICS AND APPLICATIONS\ \ \ 2019\ \ \ volume~13\ \ \ issue\ 4
\hfill \textbf{\thepage}}}

\def\leftkol{Requirements for manuscripts submitted to Journal
``Informatics~and~Applications''}

\def\rightkol{Requirements for manuscripts submitted to Journal
``Informatics~and~Applications''}

\noindent
\textbf{Conference proceedings:}

\Aue{Usmanov, T.\,S., A.\,A.~Gusmanov, I.\,Z.~Mullagalin, R.\,Ju.~Muhametshina,
A.\,N.~Chervyakova, and
A.\,V.~Sveshnikov}. 2007. Osobennosti proektirovaniya razrabotki mestorozhdeniy
s primeneniem gidrorazryva
plasta [Features of the design of field development with the use of hydraulic fracturing].
\textit{Trudy 6-go
Mezhdu\-na\-rod\-no\-go Simpoziuma ``Novye resursosberegayushchie tekhnologii
nedropol'zovaniya i povysheniya
neftegazootdachi''} [\textit{6th  Symposium (International) ``New Energy Saving Subsoil
Technologies and
the Increasing of the Oil and Gas Impact'' Proceedings}]. Moscow. 267--272.


\noindent
\textbf{Books and other monographs:}




Lindorf, L.\,S., and L.\,G.~Mamikoniants, eds. 1972.
\textit{Ekspluatatsiya turbogeneratorov s neposredstvennym
okhlazhdeniem} [\textit{Operation of turbine generators with direct cooling}].
Moscow: Energy Publs. 352~p.


%\Aue{Latyshev, V.\,N.} 2009. \textit{Tribologiya rezaniya. Kn.~1: Frikcionnye prosessy
%pri rezanii metallov}
%[\textit{Tribology of cutting. Vol.~1: Frictional processes in metal cutting}]. Ivanovo: Ivanovskii
%State Univ. 108~p.


%\noindent
%\textbf{Unpublished material:}

%\Aue{Latypov, A.\,R., M.\,M.~Khasanov, and V.\,A.~Baikov}.
%2004. Geology and production (NGT GiD). Certificate on official registration of the computer
%program
%No.\,2004611198. (In Russian, unpubl.)

%\noindent
%\textbf{Internet-source:}

%APA Style. 2011. Available at: {\sf http://www.apastyle.org/apa-style-help.aspx} (accessed
%February~5, 2011).

%Pravila citirovaniya istochnikov [Rules for the citing of sources]. Available at: {\sf
%http://www.scribd.com/doc/1034528/} (accessed February~7, 2011).


\noindent
\textbf{Dissertation and Thesis:}

%\Aue{Semenov, V.\,I.}
%2003. Matematicheskoe modelirovanie plazmy v sisteme kompaktnyy tor. [Mathematical
%modeling of the plasma in the compact torus]. D.Sc.\ Diss. Moscow. 272~p.

\Aue{Kozhunova, O.\,S.} 2009. Tekhnologiya razrabotki semanticheskogo
slovarya informatsionnogo monitoringa [Technology of development of
semantic dictionary of information monitoring system]. PhD Thesis. Moscow: IPI RAN. 23~p.


\noindent
\textbf{State standards and patents:}

GOST 8.586.5-2005. 2007. Metodika vypolneniya izmereniy. Izmerenie raskhoda i~kolichestva
zhidkostey i gazov 
s~pomoshch'yu standartnykh suzhayushchikh ustroystv [Method of measurement.
Measurement of flow rate and volume of liquids and gases by means of orifice devices]. M.:
Standardinform
Publs. 10~p.

%\noindent
%\textbf{Patent:}

\Aue{Bolshakov, M.\,V., A.\,V.~Kulakov, A.\,N.~Lavrenov, and M.\,V.~Palkin}.
2006. Sposob orientirovaniya po krenu letatel'nogo
apparata s opti\-che\-skoy golovkoy
samonavedeniya [The way to orient on the roll of aircraft with optical homing head].
Patent RF No.\,2280590.

References in Latin transcription are presented in the original language.

References in the text are numbered according to the order of their
first appearance; the number is
placed in square brackets. All items from the reference list should be
cited.\\[-13.5pt]

\item Manuscripts and additional materials are not returned to Authors by the Editorial Board.\\[-13.5pt]

\item Submissions of files by e-mail must include:\\[-13.5pt]
\begin{itemize}
\item   the journal title and author's name in the ``Subject'' field; \\[-13.5pt]
\item   an article and additional materials have to be attached using the ``attach'' function;\\[-13.5pt]
\item   an electronic version of the article should contain the file with the text and a separate file
with figures.\\[-13.5pt]
\end{itemize}

\item ``Informatics and Applications'' journal is not a profit publication. There are no
charges for the authors as well as there are no royalties.\\[-13.5pt]
\end{enumerate}

\def\leftfootline{\small{\textbf{\thepage}
\hfill INFORMATIKA I EE PRIMENENIYA~--- INFORMATICS AND APPLICATIONS\ \ \ 2019\
\ \ volume~13\ \ \ issue\ 4}
}%
 \def\rightfootline{\small{INFORMATIKA I EE PRIMENENIYA~--- INFORMATICS AND APPLICATIONS\ \ \ 2019\ \ \ volume~13\ \ \ issue\ 4
\hfill \textbf{\thepage}}}

\def\leftkol{Requirements for manuscripts submitted to Journal
``Informatics~and~Applications''}

\def\rightkol{Requirements for manuscripts submitted to Journal
``Informatics~and~Applications''}


%\vspace*{5mm}


\begin{center}
\textbf{Editorial Board address:} \\

%ABOUT AUTHORS



FRC CSC RAS, 44, block~2, Vavilov Str., Moscow 119333, Russia\\[-10pt]

\

Ph.: +7\,(499)\,135\,86\,92,\ \ Fax: +7\,(495)\,930\,45\,05\\[-10pt]

\

 e-mail: {\sf rust@ipiran.ru} (to Prof.\ Rustem Seyful-Mulyukov)\\[-10pt]

\

 {\sf http://www.ipiran.ru/english/journal.asp}
\end{center}
 }
%\thispagestyle{myheadings}

\def\leftkol{Requirements for manuscripts submitted to Journal
``Informatics~and~Applications''}

\def\rightkol{Requirements for manuscripts submitted to Journal
``Informatics~and~Applications''}

\def\leftfootline{\small{\textbf{\thepage}
\hfill INFORMATIKA I EE PRIMENENIYA~--- INFORMATICS AND APPLICATIONS\ \ \ 2019\
\ \ volume~13\ \ \ issue\ 4}
}%
 \def\rightfootline{\small{INFORMATIKA I EE PRIMENENIYA~--- INFORMATICS AND APPLICATIONS\ \ \ 2019\ \ \ volume~13\ \ \ issue\ 4
\hfill \textbf{\thepage}}}

 \label{end\stat}

\newpage

%\vspace*{-60pt} {\small
{\baselineskip=9.1pt
\section*{Правила подготовки рукописей статей для публикации в журнале
<<Информатика и её применения>>}

\thispagestyle{empty}

 Журнал <<Информатика и её применения>> публикует
теоретические, обзорные и дискуссионные статьи, посвященные научным
исследованиям и разработкам в области информатики и ее приложений. Журнал
издается на русском языке. По специальному решению редколлегии отдельные статьи,
в виде исключения, могут печататься на английском языке.
Тематика журнала охватывает следующие направления:
\begin{itemize}
\item теоретические основы информатики; %\\[-13.5pt]
\item математические методы исследования сложных систем и процессов; %\\[-13.5pt]
\item информационные системы и сети; %\\[-13.5pt]
\item информационные технологии; %\\[-13.5pt]
\item архитектура и программное
обеспечение вычислительных комплексов и сетей.
\end{itemize}
\begin{enumerate}
\item В журнале печатаются результаты, ранее не
опубликованные и не предназначенные к одновременной публикации в других
изданиях. Публикация не должна нарушать закон об авторских правах. Направляя
свою рукопись в редакцию, авторы автоматически передают учредителям и
редколлегии неисключительные права на издание данной статьи на русском языке и
на ее распространение в России и за рубежом. При этом за авторами сохраняются
все права как собственников данной рукописи. В связи с этим авторами должно
быть представлено в редакцию письмо в следующей форме:
Соглашение о передаче права на публикацию:

\textit{<<Мы, нижеподписавшиеся, авторы рукописи <<$\qquad\qquad$>>, передаем
учредителям и редколлегии журнала <<Информатика и её применения>>
неисключительное право опубликовать данную рукопись статьи на русском языке как
в печатной, так и в электронной версиях журнала. Мы подтверждаем, что данная
публикация не нарушает авторского права других лиц или организаций. Подписи
авторов: (ф.\,и.\,о., дата, адрес)>>.}

Указанное соглашение может быть представлено 
как в бумажном виде, так и в виде отсканированной копии (с подписями авторов).


Редколлегия вправе запросить у авторов экспертное заключение о возможности
опубликования представленной статьи в открытой печати. %\\[-13.5pt]
\item Статья
подписывается всеми авторами. На отдельном листе представляются данные автора
(или всех авторов): фамилия, полные имя и отчество, телефон, факс, e-mail,
почтовый адрес. Если работа выполнена несколькими авторами, указывается фамилия
одного из них, ответственного за переписку с редакцией. %\\[-13.5pt]
\item Редакция журнала
осуществляет самостоятельную экспертизу присланных статей. Возвращение рукописи
на доработку не означает, что статья уже принята к печати. Доработанный вариант
с ответом на замечания рецензента необходимо прислать в редакцию. %\\[-13.5pt]
\item Решение
редакционной коллегии о принятии статьи к печати или ее отклонении сообщается
авторам. Редколлегия не обязуется направлять рецензию авторам отклоненной
статьи. %\\[-13.5pt]
\item Корректура статей высылается авторам для просмотра. Редакция
просит авторов присылать свои замечания в кратчайшие сроки. %\\[-13.5pt]
\item При
подготовке рукописи в MS Word рекомендуется использовать следующие настройки.
Параметры страницы: формат~--- А4; ориентация~--- книжная; поля (см): внутри~---
2,5, снаружи~--- 1,5, сверху~--- 2, снизу~--- 2, от края до нижнего
колонтитула~--- 1,3. Основной текст: стиль~--- <<Обычный>>: шрифт Times New
Roman, размер 14~пунктов, абзацный отступ~--- 0,5~см, 1,5 интервала,
выравнивание~--- по ширине. Рекомендуемый объем рукописи~--- не свыше
25~страниц указанного формата. Ознакомиться с шаблонами, содержащими примеры
оформления, можно по адресу в Интернете:
\textsf{http://www.ipiran.ru/journal/template.doc}.
\item К рукописи, предоставляемой в 2-х
экземплярах, обязательно прилагается электронная версия статьи (как правило, в
форматах MS WORD (.doc) или \LaTeX\ (.tex), а также~--- дополнительно~--- в
формате .pdf) на дискете, лазерном диске или по электронной почте. Сокращения
слов, кроме стандартных, не применяются. Все страницы рукописи должны быть
пронумерованы. %\\[-13.5pt]
\item Статья должна содержать следующую информацию на русском и
английском языках: название, Ф.И.О. авторов, места работы авторов и их
электронные адреса, подробные сведения об авторах, оформленные в соответствии с форматом, 
определяемым файлами {\sf http://www.ipiran.ru/journal/issues/2011\_05\_01/authors.asp} и 
{\sf http://www.ipiran.ru/journal/issues/2011\_01\_eng/authors.asp},
аннотация (не более 100~слов), ключевые слова. Ссылки на
литературу в тексте статьи нумеруются (в квадратных скобках) и располагаются в
порядке их первого упоминания. В~списке литературы не должно быть позиций, на которые нет ссылки в тексте статьи.
Все фамилии авторов, заглавия статей, названия
книг, конференций и~т.\,п.\ даются на языке оригинала, если этот язык
использует кириллический или латинский алфавит. %\\[-13.5pt]
\item Присланные в редакцию материалы авторам не возвращаются.
\item При отправке файлов по электронной
почте просим придерживаться следующих правил:
\begin{itemize}
\item указывать в поле subject (тема) название журнала и фамилию автора; %\\[-13.5pt]
\item использовать attach (присоединение); %\\[-13.5pt]
\item в случае больших объемов информации возможно
использование общеизвестных архиваторов (ZIP, RAR); %\\[-13.5pt]
\item в состав электронной версии статьи должны входить: файл, содержащий текст статьи, и файл(ы),
содержащий(е) иллюстрации. %\\[-13.5pt]
\end{itemize}
\item Журнал <<Информатика и её применения>> является некоммерческим изданием. 
Плата за публикацию с авторов не взимается, гонорар авторам не выплачивается.
\end{enumerate}
\thispagestyle{empty}
\textbf{Адрес редакции:} Москва 119333,
ул.~Вавилова, д.~44, корп.~2, ИПИ РАН\\
\hphantom{\textbf{Адрес редакции:} }Тел.: +7 (499) 135-86-92\ \
Факс:  +7 (495) 930-45-05\ \  E-mail:   rust@ipiran.ru }
}

%\include{ipi-ind}

%\tableofcontents

\end{document}

%\tableofcontents

%\end{document}

%\tableofcontents


\end{document}

\newcommand{\Ack}{\subsection*{\protect\large\bf Acknowledgments}}