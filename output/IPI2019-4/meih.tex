%\def\l{\lambda}
%\def\b{\overline b}

\def\stat{meikh}

\def\tit{СИСТЕМА МАССОВОГО ОБСЛУЖИВАНИЯ $\mathrm{Geo}/G/1/\infty$ 
С~ИНВЕРСИОННЫМ ПОРЯДКОМ ОБСЛУЖИВАНИЯ
И~РЕСАМПЛИНГОМ В~ДИСКРЕТНОМ ВРЕМЕНИ$^*$}

\def\titkol{Система массового обслуживания $\mathrm{Geo}/G/1/\infty$ с~инверсионным порядком обслуживания
и~ресамплингом} % в~дискретном времени}

\def\aut{Л.\,А.~Мейханаджян$^1$, Р.\,В.~Разумчик$^2$}

\def\autkol{Л.\,А.~Мейханаджян, Р.\,В.~Разумчик}

\titel{\tit}{\aut}{\autkol}{\titkol}

\index{Мейханаджян Л.\,А.}
\index{Разумчик Р.\,В.}
\index{Meykhanadzhyan L.\,A.}
\index{Razumchik R.\,V.}



{\renewcommand{\thefootnote}{\fnsymbol{footnote}} \footnotetext[1]
{Работа выполнена при поддержке РФФИ (проект 18-37-00283).}}


\renewcommand{\thefootnote}{\arabic{footnote}}
\footnotetext[1]{Финансовый университет при Правительстве РФ,
 \mbox{lamejkhanadzhyan@fa.ru}}
\footnotetext[2]{Институт проблем информатики Федерального исследовательского 
центра <<Информатика и~управление>> Российской академии наук; Российский
университет дружбы народов, \mbox{rrazumchik@ipiran.ru} %\mbox{razumchik-rv@rudn.ru}
}

\vspace*{-2pt}




\Abst{Рассматривается задача нахождения оценки для
фактического стационарного среднего времени
пребывания в~дискретной однолинейной системе с~геометрическим
входящим потоком известной интенсивности, циклической дисциплиной
обслуживания и~неточной априорной информацией о временах обслуживания.
Показано, что новой оценкой может служить значение стационарного среднего
времени пребывания в~дискретной однолинейной системе с~инверсионным
порядком обслуживания и~дисциплиной ресамплинга,
предполагающей, что каждая поступающая в~непустую систему заявка
назначает новое остаточное время обслуживания заявке на приборе.
Для случая погрешности мультипликативного типа и~геометрически
распределенного фактического времени обслуживания
найдены легко проверяемые на практике условия, при которых
предлагаемая оценка является наилучшей из известных
оценок сверху для фактического
среднего времени пребывания во всем диапазоне возможных значений загрузки.}


\KW{дискретное время; инверсионный порядок обслуживания;
неточное время обслуживания; дисциплина циклического обслуживания; ресамплинг}


\DOI{10.14357/19922264190410} 
  
%\vspace*{1pt}


\vskip 10pt plus 9pt minus 6pt

\thispagestyle{headings}

\begin{multicols}{2}

\label{st\stat}

\section{Введение}


Рассмотрим дискретную систему массового обслуживания (СМО) $\mathrm{Geo}/G/1$--RR
с~вероятностью~$a$ поступления заявки на одном такте и~дисциплиной циклического
 обслуживания~\cite{Schass2}.
Дискретное время вводится обычным образом~\cite{distime},
и~все изменения состояния СМО происходят в~конце такта.
Распределение времени~$\hat{S}$ обслуживания заявок является
произвольным дискретным. Стационарное среднее время пребывания
заявки в~такой системе рассчитывается по формуле (см.~(1) в~\cite{Schass}):
\begin{equation}
\label{shassb}
v_{\hat S}= \fr{\overline{a} \mathsf{E}{\hat S}}{1- a  \mathsf{E}{\hat S}}\,.
\end{equation}
%$a \mathsf{E}{\hat S}<1$
Если в~информации о распределении случайной величины (с.в.)~$\hat{S}$
содержится ошибка (погрешность)\footnote[3]{В~этом случае $\hat{S}$
можно трактовать как прогнозное время
обслуживания, т.\,е.\ число тактов, которое, как ожидается,
потребуется затратить на обслуживание заявки.}, т.\,е.\
распределение фактического времени~$S$ обслуживания
другое, причем $\mathsf{E}{\hat S}\hm\neq \mathsf{E}{S}$,
то прогнозное среднее время обслуживания и~фактическое не совпадают.
При $\mathsf{E}{\hat S}\hm>(<)\mathsf{E}{S}$ значение~$v_{\hat S}$
служит оценкой сверху (снизу) для фактического, неизвестного среднего~$v_{S}$.
В~связи с~этим обстоятельством возникает следующий вопрос:
возможно ли только на основе информации о значении~$a$
и~распределении~$\hat{S}$ получать более точные оценки, чем~$v_{\hat S}$?
Оказывается, что в~некоторых случаях ответ на этот вопрос
положителен. В~этой статье показано,
что при $\mathsf{E}{\hat S}\hm>\mathsf{E}{S}$ и~некоторых
других ограничениях на распределение~$\hat{S}$, существует~$v_*$ такое, что
\begin{equation}
\label{ner}
v_{S} \le v_* \le v_{\hat S}
\end{equation}
при любых (допустимых) значениях загрузки, и~это~$v_*$
совпадает со стационарным средним временем пребывания в~СМО
$\mathrm{Geo}/G/1$ с~ресамплингом и~инверсионным порядком обслуживания (далее~--- 
$\mathrm{Geo}/G/1$--Re)
с~вероятностью~$a$ поступления заявки на одном такте
и~временем обслуживания, распределенном как~${\hat S}$ (см.\ формулу~\eqref{vre} ниже).
Доказательству неравенства~\eqref{ner} и~обсуждению
связанных с~ним вопросов посвящен последний раздел статьи.
В~предваряющих его разделах дается подробное описание СМО $\mathrm{Geo}/G/1$--Re
и~анализ ее основных стационарных характеристик.

Отметим, что по тематике СМО со специальными дисциплинами обслуживания
имеется обширная литература (см., 
например,~\cite{n0,n5,nm1,new1,wfb,new2,new3,new4,new5,stal,mey}
и~ссылки в~них).
Поэтому некоторые из приводимых ниже аналитических результатов для СМО $\mathrm{Geo}/G/1$--Re,
по-ви\-ди\-мо\-му, могут получаться как следствия уже известных\footnote[1]{Это замечание относится только к~формулам~\eqref{pn}---\eqref{us} и~\eqref{ppsi}.}.

\vspace*{-6pt}

\section{Описание системы}

\vspace*{-2pt}

Рассмотрим функционирующую в~дискретном времени однолинейную СМО
с~очередью неограниченной емкости,
в~которую поступает геометрический поток заявок
с~вероятностью~$a$ поступления заявки на одном 
такте\footnote[2]{Будем считать, что сначала систему покидает заявка на
приборе, если закончилось ее обслуживание, затем в~систему
принимается поступившая заявка, если таковая имеется, и,~наконец,
выбирается на обслуживание заявка из очереди, если очередь непуста.}.
Распределение времени обслуживания заявки является
произвольным дискретным с~вероятностью~$b_i$, $i \hm\ge 0$, 
того, что обслуживание заявки
продлится~$i$~тактов (предполагается, что $b_0\hm=0$).

Далее будем использовать следующие обозначения:
\begin{description}
\item $\overline{a}=1-a$~--- вероятность непоступления заявки на такте;
\item
$B_i=\sum\nolimits_{j=i}^\infty b_j$, $i \hm\ge 1$,~--- вероятность того, что длина заявки не менее $i$;
\item
$\beta(z)=\sum\nolimits_{i=1}^\infty  z^i b_i$~--- производящая функция (ПФ)
длины заявки.
\end{description}

В системе реализована дисциплина ресамплинга с~инверсионным порядком обслуживания.
Предполагается, что в~любой момент времени для каждой заявки, находящейся в~системе,
известна остаточная длина (т.\,е.\ число тактов, необходимое
для окончания обслуживания заявки).
Если новая заявка поступает в~непустую систему,
то она прерывает обслуживание заявки на приборе 
и~назначает ей новую остаточную длину (в~соответствии с~распределением~$\{b_i, 
\ i\hm\ge 0\}$). Далее заявка,
обслуживание которой было прервано, возвращается на прибор
и~продолжает обслуживаться (с~новой остаточной длиной), а новая
становится на первое место в~очереди.
Как только длина заявки на приборе становится равной нулю,
она моментально покидает систему. Если очередь непуста,
то заявка из первого места в~очереди поступает на прибор;
остальная очередь сдвигается на единицу.

Всюду в~дальнейшем будем предполагать, что выполнено необходимое
и~достаточное условие существования стационарного режима, которое
для рассматриваемой системы имеет вид:
$$
\beta\left(\overline{a}\right)
>\overline{a} \left(1+\overline{a}\right)^{-1}\,.
$$



Действительно, свяжем с~рассматриваемой сис\-те\-мой процесс Галь\-то\-на--Ват\-со\-на,
где первоначально имеется одна частица, которая в~конце
жизни производит случайное число потомков в~соответствии
с~распределением ${\{g_k, \ k\hm\ge 0\}}$: 

\noindent
\begin{align*}
g_0&=\beta\left(\overline{a}\right);\\
g_k&=\fr{\beta(\overline{a})(1\hm-\beta(\overline{a}))
(\overline{a}-\beta(\overline{a}))^{k-1}}{\overline{a}^k}\,,\enskip k\ge 1\,.
\end{align*}

\vspace*{-2pt}

Заметим, что число заявок, обслуженных рассматриваемой системой за период занятости,
\mbox{равно} общему числу частиц, появившихся во введенном процессе Галь\-то\-на--Ват\-со\-на 
до его вы\-рож\-де\-ния.
Последнее же  имеет место
с~вероятностью единица (за конечное среднее время) тогда и~только
тогда, когда среднее число $\sum\nolimits_{k=1}^\infty  k g_k$
потомков от одной частицы меньше единицы, что равносильно приведенному выше 
условию\footnote[3]{Заметим,
что это условие не зависит от моментов длины заявки ка\-ко\-го-ли\-бо порядка, т.\,е.\ для любого
распределения длины заявки при достаточно малой вероятности~$a$ поступления заявки на такте
существует стационарное распределение.}.

Наконец, заметим, что для рассматриваемой системы справедлив закон стационарной
очереди Хинчина (см.~\cite[п.~4.1.1]{distime}); это свойство используется
при описании выходящего из системы потока.

\vspace*{-6pt}


\section{Стационарное распределение очереди}

\vspace*{-2pt}

Найдем стационарное распределение числа заявок в~системе. Введем
обозначения:
\begin{description}
\item
$p_0$~--- стационарная вероятность того, что непосредственно после очередного такта система будет
пуста;
\item
$p_n(i)$, $i \ge 1$,~---
стационарная вероятность того,
что непосредственно после очередного такта в~системе будет~$n$~заявок 
и~до окончания обслуживания
заявки на приборе останется~$i$~тактов.
\end{description}

Заметим, что $p_n = \sum\nolimits_{i=1}^{\infty} p_n(i)$, $n \hm\ge 1$.
Используя метод исключения состояний,
приходим к~сле\-ду\-ющей системе уравнений равновесия:
\begin{align}
\label{ver1-new}
\!p_1(i) &=\left(p_0 + p_1 \right)a b_i+
p_1(i+1)\overline{a},\enskip  i \ge 1\,;
\\
\!p_n(i) &=
\left(p_{n-1}-p_{n-1}(1)\right) a b_i +
p_{n}(i+1) \overline{a} + p_{n} a b_i,\notag\\
& \hspace*{40mm}n \ge 2\,, \enskip i \ge 1\,,
\label{vern-new}
\end{align}
к которой необходимо добавить условие нор\-ми\-ровки

\vspace*{-2pt}

\noindent
$$
\sum\limits_{n=0}^\infty p_n =1\,.
$$
Для решения этой системы воспользуемся аппаратом ПФ. Введем обозначение:
$$
P(z,u) = \sum\limits_{n=1}^{\infty}  \sum\limits_{i=1}^{\infty} u^n z^i p_n(i)\,,\
0\le u,\ z\le 1\,.
$$
 Умножая левые и~правые части~\eqref{ver1-new} 
и~\eqref{vern-new}
на~$u^nz^i$ в~соответствующих степенях и~суммируя по всем возможным значениям~$n$ и~$i$,
получаем:
\begin{multline}
P(z,u)= {}\\
{}=\fr{z}{z-\overline{a}}
\sum\limits_{n=1}^{\infty}  u^n \left (
p_n(1)\overline{a} \left( \beta(z) -1\right)
+a \beta(z) p_n
\right ), 
\label{eq_Pzu}
\end{multline}
откуда по теореме Руше следует, что
$$
p_n(1) = \fr{p_n a \beta(\overline{a})}{\overline{a} 
( 1\hm- \beta(\overline{a}))}\,,\enskip n \ge 1\,.
$$
Воспользовавшись этим соотношением, условием нормировки
и~уравнениями локального баланса, которые для данной системы имеют
вид:
$$
p_0 a = \overline{a} p_1(1)\,; \enskip
(p_n- p_n(1))a = \overline{a} p_{n+1}(1)\,, \enskip n \geq 1\,,
$$
путем элементарных преобразований получаем:
\begin{align}
\label{pn}
p_n &= \left( \fr{\overline{a} - \beta(\overline{a})}{\overline{a} \beta(\overline{a})}
\right)^{n-1}
\fr{1- \beta(\overline{a})}{\beta(\overline{a})}\, p_0, \enskip n\geq 1\,;
\\
\label{p0}
p_0 &= \fr{\overline{a} \beta(\overline{a}) - \overline{a} + \beta(\overline{a})}{\beta(\overline{a})}.
\end{align}

Таким образом, стационарное распределение
общего числа заявок в~системе определяется начиная с~$p_1$
геометрической прогрессией, что упрощает
нахождение моментов. Например, среднее число~$N$~заявок в~системе равно
$$
N = \fr{\overline{a}^2 (1\hm- \beta(\overline{a}))}{\overline{a}
 \beta(\overline{a})\hm- \overline{a} \hm+ \beta(\overline{a})}\,.
 $$

Наконец, из~\eqref{eq_Pzu}, приравнивая коэффициенты
при~$u^n z^i$ в~левой и~правой частях,
находим выражение для совместного стационарного распределения~$p_n(i)$:
$$
p_n(i)=p_n \sum\limits_{j=i}^{\infty} 
\fr{a(\overline{a})^{j-i} }{1-\beta(\overline{a})}\,b_j , \enskip  n\geq 1\,, \ 
 i \geq 1\,.
$$

В заключение раздела отметим, что если загрузка системы близка к~критической,
т.\,е.\ 
$$
\beta(\overline{a}) \downarrow \overline{a} (1\hm+\overline{a})^{-1},
$$
то общее число заявок в~системе, нормированное величиной 
$(\overline{a} \beta(\overline{a}) \hm- \overline{a} \hm+ \beta(\overline{a}))^{-1}$,
имеет экспоненциальное распределение с~параметром 
$(1\hm+\overline{a})(\overline{a})^{-2}$.
Этот результат следует (по методу из~\cite{opech}) из анализа явной формулы для 
ПФ распределения $\{ p_n, \ n \hm\ge 0 \}$, задаваемого~\eqref{pn} и~\eqref{p0}.


\section{Стационарное распределение времени пребывания заявки в~системе}

Время пребывания заявки в~рассматриваемой системе складывается из
двух независимых с.в.: времени ожидания
начала обслуживания и~времени пребывания на приборе.
Ввиду инверсионного порядка обслуживания время пребывания в~очереди совпадает с~длительностью
периода занятости (ПЗ) системы, которая в~терминах ПФ имеет вид:

\vspace*{4pt}

\noindent
\begin{equation}
\label{us}
u(z) = \fr{(\overline{a}-a\beta(\overline{a} z))(1-\overline{a} z) - \sqrt{D}
}{2 a (\overline{a} z - \beta(\overline{a} z))}\,,
\end{equation}

\vspace*{-3pt}

\noindent
где 

\vspace*{-3pt}

\noindent
\begin{multline*}
D = (\overline{a}-a\beta(\overline{a} z))^2(1-\overline{a} z)^2 
-{}\\
{}- 4a \overline{a}\beta(\overline{a} z)(\overline{a} z - 
\beta(\overline{a} z))(1\hm-\overline{a} z)\,.
\end{multline*}

\vspace*{-2pt}

Для нахождения времени пребывания заявки на приборе поступим следующим образом.
Обозначим через~$w_j(i)$ вероятность того, что длительность пребывания
на приборе поступающей на него заявки равно~$i$~тактам и~за это время
в~систему поступит~$j$~новых заявок. По формуле полной ве\-ро\-ят\-ности имеем:

\vspace*{-3pt}

\noindent
\begin{align}
\label{gix1}
w_0(i)&= \overline{a}^i b_i, \enskip i \ge 0\,;
\\
w_1(i)&=\sum\limits_{n=1}^{i-1} \overline{a}^{n-1}a B_{n+1} 
w_0(i-n)+\overline{a}^{i-1}a b_i, \notag\\
&\hspace*{50mm}i \ge 1\,; 
\label{gix2}
\\
w_j(i)&=\sum\limits_{n=1}^{i-j+1} \overline{a}^{n-1}a B_{n+1} w_{j-1}(i-n), \notag \\
& \hspace*{40mm}i \ge j\,, \enskip j \ge 2\,. \label{gix3}
\end{align}

\vspace*{-2pt}

\noindent
Тогда
распределение времени пребывания заявки на приборе с~учетом возможных прерываний
есть $\{\sum\nolimits_{j=0}^\infty w_j(i), i \hm\ge 1\}$ 
и,~с~учетом~\eqref{gix1}--\eqref{gix3},
в терминах ПФ имеет вид:
\begin{equation}
\label{ppsi}
\psi(z)=
\fr{( 1- \overline{a} z) \beta(\overline{a} z) }
{(1-z)\overline{a} + a \beta(\overline{a} z)}\,.
\end{equation}
Замечая теперь, что с~вероятностью
$$
p_0+\sum\limits_{n=1}^\infty p_n(1)=
\fr{2\beta(\overline{a})
\hm-\overline{a}}{\beta(\overline{a})}
$$
поступающая заявка сразу занимает прибор,
а~с~дополнительной вероятностью занимает место в~очереди,
получаем вид ПФ стационарного времени пребывания произвольной заявки
в~системе:
\begin{equation}
\label{vpgf}
\chi(z)=\fr{ 2\beta(\overline{a})-\overline{a}}{\beta(\overline{a})}\, \psi(z) +
\fr{ \overline{a} - \beta(\overline{a}) }{\beta(\overline{a})} \,u(z) \psi(z).
\end{equation}

\vspace*{-6pt}

\pagebreak

Дифференцируя~\eqref{vpgf} по~$z$ в~точке $z\hm=1$,
с учетом формул~\eqref{us} и~\eqref{ppsi} нетрудно определять моменты времени
пребывания заявки в~системе. В~частности, среднее\footnote{Заметим, что
$v\hm=\overline{a} u'(1)$. Таким образом, в~отличие
от непрерывного случая, в~котором стационарное среднее время пребывания
заявки в~системе равно средней длине ПЗ системы~\cite[Corollary~2]{Mat-2},
в дискретном случае среднее время пребывания в~$\overline{a}^{-1}$ раз меньше.}
 время пребывания равно:
 \noindent
\begin{equation}
\label{vre}
v=\fr{\overline{a}^2 (1- \beta(\overline{a}))}
{ a \left (\overline{a}
 \beta(\overline{a})- \overline{a} + \beta(\overline{a})\right )}\,.
\end{equation}
Замечая, что $av\hm=N$, убеждаемся, что для рас\-смат\-ри\-ва\-емой системы
выполняется закон Литтла.

Вопрос поведения стационарного распределения времени пребывания заявки
в~системе в~условиях большой загрузки
остается открытым. При-\linebreak менение общих принципов асимптотического \mbox{анализа}
не дает содержательных результатов (см.~[17; 18, Remark ~2.6]).



\section{Выходящий поток}

В стационарном режиме выходящий из системы поток не является геометрическим.
Единственное исключение составляет случай, когда распределение $\{ b_i, \ i \hm\ge 0 \}$
является геометрическим. Поскольку доказательство этого факта в~основных чертах\linebreak
 повторяет
доказательство для непрерывного случая, изложенного в~\cite{finch}, здесь оно не 
приводится.\linebreak
Таким образом, в~общем случае промежутки между последовательными окончаниями
обслуживания являются зависимыми величинами.


Совместное стационарное распределение длительностей двух последовательных промежутков
в~терминах ПФ имеет вид:
\begin{multline*}
%\label{interval2}
D_2\left(z_1,z_2\right)
= \psi(z_2) \left ( \psi(z_1) -
\left (
\fr{(1 - z_1) \psi(z_1) }{1 -  z_1 \overline{a}}
+ {}\right.\right.\\[6pt]
\left.\left.{}+\fr{a z_1 (1 - z_2) \beta(\overline{a} z_1)}{(1 - z_1 \overline{a})
(1 - z_2 \overline{a})}
\right)
 p_0 -
 \fr{ (1 -z_2) \beta(\overline{a} z_1)}{1 - z_2 \overline{a}}\,
 p_1
\right).
\end{multline*}

По свойствам ПФ нетрудно получить выражения для основных
 характеристик выходящего потока:
 
 \noindent
\begin{gather*}
\mathsf{E}l=\fr{1}{a},
\quad
\mathsf{E}l(l-1) =
\fr{2\overline{a} (\beta(\overline{a}) - a \overline{a} \beta'(\overline{a}))
}{(a \beta(\overline{a}))^2}\,;
\\[6pt]
\mathsf{Cov}(l_1,l_2)=
\fr{ a \overline{a} \beta'(\overline{a}) + \beta(\overline{a})^2 - 
\beta(\overline{a})}{a^2 \beta(\overline{a})}\, p_0.
\end{gather*}

\noindent
Заметим, что при большой загрузке выходящий поток становится
практически некоррелированным.

%\columnbreak

\section{Оценка сверху для~среднего времени пребывания}

Предположим, что прогнозное время обслуживания~$\hat S$
представимо в~виде\footnote[2]{О практических основаниях
для предположения о мультипликативной, а~не аддитивной погрешности 
см.~\cite[Secion 6.3]{Mat}
и~\cite{LusMilR}.} $\hat S\hm=SX$,
где $S$~--- фактическое время обслуживания, а~$X$~--- 
некоторая положительная целочисленная
с.в.\ (ошибка), причем с.в.~$S$ и~$X$ независимы и~$\mathsf{E}X\hm>1$.

Зафиксируем произвольным образом среднее значение~$\mathsf{E}S$ 
и,~взяв вероятность~$a$ поступления заявки на такте за независимую переменную,
проиллюстрируем~\eqref{ner} при следующих предположениях о распределениях~$S$ и~$X$:
\begin{itemize}
\item[---] $S$ имеет геометрическое распределение, а $X$~--- логарифмическое, т.\,е.\
$\mathsf{P}(S=k)\hm=b(1-b)^{k-1}$, $k\hm\ge 1$,
$\mathsf{P}(X=k)\hm=-p^k/(k\ln(1-p))$, $k\hm \ge 1$ (рис.~1);
\item[---] $S$ имеет дискретное распределение Вейбулла с~<<легким хвостом>>, а~$X$~--- 
логарифмическое, т.\,е.\ $\mathsf{P}(S=k)\hm=q^{(k-1)^{\beta}}
\hm-q^{k^{\beta}}$, $k\hm\ge 1$, $0\hm<q\hm<1$, $\beta\hm>1$ (рис.~2);
\item[---] $S$ и~$X$ имеют одинаковое дискретное распределение 
Вейбулла с~<<тяжелым хвостом>>, т.\,е.\ $0\hm<\beta\hm<1$ (рис.~3).
\end{itemize}

На рис.~1--3 кривые~\textit{1}~--- $v_{S}$, \textit{2}~--- $v_{SX}$; 
\textit{3}~---~$v_*$.
Во всех примерах $\mathsf{E}S\hm=5$.




Допуская некоторую вольность речи, результаты численных экспериментов
можно резюмировать следующим образом: \eqref{ner} не выполняется, 
когда~$SX$ имеет либо <<легкий хвост>>, либо\linebreak слишком <<тяжелый хвост>>.

Нетрудно показать, что если распределение с.в.~$SX$ таково,
что для любого $i\hm\ge 0$ выполняется условие\footnote[3]{Это
условие (см.~\cite{new8-2}) выполняется для распределений
типа НХСС (<<новое хуже старого в~среднем>>), <<новое хуже старого>>
и~распределений с~убывающей функцией интенсивности 
(см.~[23; 24, с.~35; 25]).
В~последнем случае удобный критерий проверки для монотонных функций
 приводится в~\cite[Proposition~1]{Conti}.}
$$
\sum\limits_{j=i}^\infty \mathsf{P}({SX}>i)
\ge \mathsf{E}(SX)\left (1 - (\mathsf{E}(SX))^{-1}\right )^i,
$$
то правое неравенство в~\eqref{ner} имеет место всегда,
когда сис\-те\-ма $\mathrm{Geo}/G/1$--RR
с неточным временем обслуживания стационарна\footnote[4]{Примечательно, что 
область стационарности $\mathrm{Geo}/G/1$--Re шире,
чем $\mathrm{Geo}/G/1$--RR с~неточным временем обслуживания.} (т.\,е.\
 при $0\hm<a \hm<(\mathsf{E}(SX))^{-1}$).
В~противном случае оценка~$v_*$ для среднего времени пребывания хуже
значения, рас-\linebreak\vspace*{-12pt}

\pagebreak



\end{multicols}

\begin{figure*} %fig1
\vspace*{1pt}
    \begin{center}  
  \mbox{%
 \epsfxsize=162.991mm 
 \epsfbox{mei-1.eps}
 }
\end{center}
\vspace*{-15pt}
\Caption{Случай, когда $S$ имеет геометрическое распределение с~параметром
${b^{-1}\hm=\mathsf{E}S=5}$, $X$~--- логарифмическое распределение с~параметром  
$p\hm=0{,}2$ ($\mathsf{E}X \hm\approx 1{,}12$)~(\textit{a}) и~$p\hm=0{,}95$ 
($\mathsf{E}X \approx 6{,}34$)~(\textit{б})}
\label{fig1}
\end{figure*}


\begin{figure*} %fig2
\vspace*{1pt}
    \begin{center}  
  \mbox{%
 \epsfxsize=162.991mm 
 \epsfbox{mei-2.eps}
 }
\end{center}
\vspace*{-15pt}
\Caption{Случай, когда $S$ имеет дискретное распределение Вейбулла
с параметрами $\beta\hm=1{,}1$, $q\hm\approx 0{,}8316$ ($\mathsf{E}S \hm\approx 5$), 
$X$~--- логарифмическое распределение с~параметром  $p\hm=0{,}2$~(\textit{a}) 
и~$p\hm=0{,}95$~(\textit{б})}
\label{fig2}
\vspace*{-3pt}
\end{figure*}




\begin{multicols}{2}

  { \begin{center}  %fig3
 \vspace*{-6pt}
 \mbox{%
 \epsfxsize=78.7mm 
 \epsfbox{mei-3.eps}
 }

\end{center}

\vspace*{-6pt}

\noindent
{{\figurename~3}\ \ \small{Случай, когда~$S$ и~$X$ имеют одинаковое дискретное распределение Вейбулла
с~параметрами $\beta\hm=0{,}38$, $q \hm\approx 0{,}3834$ ($\mathsf{E}S\hm=\mathsf{E}X
\hm \approx 5$)}}
}

%\vspace*{9pt}

\addtocounter{figure}{1}




\noindent
считываемого по формуле~\eqref{shassb}, т.\,е.\
 $v_* \hm> v_{SX} \hm\ge v_{X}$ (см.\ рис.~2,\,\textit{а}).



Проверка левого неравенства в~\eqref{ner} невозможна без
дополнительной информации\footnote{Отметим, что
если с.в.~$S$ имеет геометрическое распределение,
то левое неравенство в~\eqref{ner} выполняется всегда,
вне зависимости от значения параметра геометрического распределения
и вида распределения~$X$.}
о~распределениях с.в.~$S$ и~$X$, так как, как видно на рис.~3,
даже если с.в.~$SX$ имеет НХСС-рас\-пре\-де\-ле\-ние, то
(2)~может и~не выполняться.
 
 В~связи с~этим интересен (и~остается открытым) вопрос о выборе подходящей для дискретного времени
модели неточного времени обслуживания\footnote{Укажем на работы~\cite{sx1,sx2}, 
модели из которых могут оказаться плодотворными.}
или, другими словами, вопрос построения дискретного аналога непрерывной 
модели из~\cite{Mat-2}.

В~отличие от случая непрерывного времени, где мультипликативной модели 
удается придать наблюдаемые
на практике черты\footnote{Более подробное обсуждение этого вопроса 
можно найти в~\cite[Secion 6.3]{Mat}.},
в~дискретном времени это трудноосуществимо.

{\small\frenchspacing
 {%\baselineskip=10.8pt
 \addcontentsline{toc}{section}{References}
 \begin{thebibliography}{99}

\bibitem{Schass2}
\Au{Daduna H., Schassberger~R.}
A~discrete-time round-Robin queue with Bernoulli input
and general arithmetic service time distributions~//
Acta Inform., 1981. Vol.~15. Iss.~3. P.~251--263.


\bibitem{distime}
\Au{Печинкин А.\,В., Разумчик~Р.\,В.} 
Системы массового обслуживания в~дискретном времени.~--- M.: Физматлит, 2018. 432~с.

\bibitem{Schass} %3
\Au{Schassberger R.} On the response time distribution in a~discrete round-robin queue~//
Acta Inform., 1981. Vol.~16. Iss.~1. P.~57--62.

\bibitem{new3} %4
\Au{Таташев А.\,Г.} 
Одна инверсионная дисциплина обслуживания в~одноканальной системе с~разнотипными
 заявками~// Автомат. и~телемех., 1999. №\,7. С. 177--181.

\bibitem{new4} %5
\Au{Таташев А.\,Г.} Система обслуживания с~инверсионной дисциплиной, 
двумя типами заявок и~марковским входящим потоком~//  Автомат. и~телемех., 2003. №\,11. С.~122--127.


\bibitem{new1} %6
\Au{Fiems D., Steyaert~B., Bruneel~H.}
Discrete-time queues with generally distributed service times and renewal-type 
server interruptions~// Perform. Eval., 2004. Vol.~55. Iss.~\mbox{3-4}. P.~277--298.


\bibitem{wfb} %7
\Au{Walraevens J., Flems~D., Bruneel~H.}
The discrete-time preemptive repeat identical priority queue~// Queueing~Sy., 2006. 
Vol.~53. Iss.~4. P.~231--243.

\bibitem{new2} %8
\Au{Pechinkin A., Shorgin~S.} 
The discrete-time queueing system with inversive service order and probabilistic
 priority~//
3rd Conference (International) on Performance Evaluation Methodologies and Tools 
 Proceedings.~--- Brussels: ICST, 2008. Art. No.\,20. 6~p.


\bibitem{nm1} %9
\Au{Милованова Т.\,А.} Система ${\mathrm{BMAP}/G/1}$ 
с~инверсионным порядком обслуживания и~вероятностным приоритетом~// 
Автомат. и~телемех., 2009. №\,5. С.~155--168.


\bibitem{stal} %10
\Au{Печинкин А.\,В., Стальченко~И.\,В.}
Система $\mathrm{MAP}/G/1/\infty$ с~инверсионным порядком
обслуживания и~вероятностным приоритетом,
функционирующая в~дискретном времени~//
Вестник РУДН. Сер.: Математика. Информатика. Физика, 2010.
№\,2. С.~26--36.

\bibitem{n5} %11
\Au{Милованова Т.\,А., Печинкин~А.\,В.}
Стационарные характеристики системы обслуживания 
с~инверсионным порядком обслуживания, вероятностным
приоритетом и~гистерезисной политикой~//
Информатика и~её применения, 2013. Т.~7. Вып.~1. С.~22--35.

\bibitem{mey} %12
\Au{Мейханаджян Л.\,А.} 
Стационарные вероятности состояний в~системе обслуживания конечной 
емкости с~инверсионным порядком обслуживания и~обобщенным вероятностным приоритетом~// 
Информатика и~её применения, 2016. Т.~10. Вып.~62. С.~123--131.

\bibitem{new5} %13
\Au{Афанасьева Л.\,Г., Ткаченко~А.\,В.} 
Условия ста\-биль\-ности систем с~очередью и~регенерирующим 
процессом прерываний обслуживания~// Теория вероятн. и~её примен., 2018. Т.~63. 
Вып.~4. С.~623--653.

\bibitem{n0} %14
\Au{Razumchik R.} Two-priority queueing system with LCFS service, 
probabilistic priority and batch arrivals~// AIP Conf. Proc., 2019. 
Vol.~2116. Iss.~1. P.~090011-1--090011-3.

\bibitem{opech} %15
\Au{Печинкина О.\,А.} 
Асимптотическое распределение длины очереди в~системе $M/G/1$ с~инверсионной
вероятностной дисциплиной обслуживания~// Вестник РУДН. Сер.: Прикладн. матем. и~информ.,
1995. №\,1. C.~87--100.

\bibitem{Mat-2} %16
\Au{Horv$\acute{\mbox{a}}$th I., Razumchik~R., Telek~M.}
The resampling M/G/1 non-preemptive LIFO queue and its 
application to systems with uncertain service time~// Perform. Evaluation, 2019. 
Vol.~134. Art. ID: 102000. 13~p.




\bibitem{limic} %17
\Au{Limic V.} A~LIFO queue in heavy traffic~// Ann. Appl. Probab., 2001. Vol.~11. Iss.~2. P.~301--331.

\bibitem{asmus} %18
\Au{Asmussen S., Glynn~P.\,W.} On 
preemptive-repeat LIFO queues~//  Queueing Sy., 2017. 
Vol.~87. Iss.~1-2. P.~1--22.


\bibitem{finch} %19
\Au{Finch P.\,D.}
The output process of the queueing system $M/G/1$~//
J.~Roy. Stat. Soc.~B Met., 1959. Vol.~21. Iss.~2. P.~375--380.

\bibitem{Mat} %20
\Au{Dell'Amico M., Carra~D., Michiardi~P.} PSBS: Practical size-based scheduling~// 
IEEE T. Comput., 2016.
Vol.~65. Iss.~7. P.~2199--2212.

\bibitem{LusMilR} %21
\Au{Milovanova T.\,A., Meykhanadzhyan~L.\,A., Razumchik~R.\,V.}
Bounding moments of Sojourn time in $M/G/1$ FCFS queue with 
inaccurate job size information and additive error: 
Some observations from numerical experiments~// CEUR Workshop Proceedings, 2018. 
Vol.~2236. P.~24--30.


\bibitem{new8-2} %22
\Au{Klefsj$\ddot{\mbox{o}}$~B.}
A~useful ageing property based on the Laplace transform~//
J.~Appl. Probab., 1983. Vol.~20. Iss.~3. P.~615--626.




\bibitem{new7} %23
\Au{Барлоу Р., Прошан~Ф.} Математическая теория надежности~/ Пер. с~англ. под ред. 
Б.\,В.~Гнеденко.~--- М.: Советское радио, 1969. 488~с.
(\Au{Barlow~R., Proschan~F.}  {Mathematical theory of reliability}.~---
New York, NY, USA: Wiley, 1965. 274~p.)

\bibitem{new6} %24
\Au{Штойян Д.} Качественные свойства и~оценки стохастических моделей~/
Пер. с~нем.~--- М.: Мир, 1979. 268~с.
(\Au{Stoyan~D.}  {Qualitative Eigenschaften und Abschtzungen 
stochastischer Modelle}.~--- Berlin: Akademie-Verlag, 1977. 198~p.)


\bibitem{new8-1} %25
\Au{Klefsj$\ddot{\mbox{o}}$~B.} The hnbue and hnwue classes of life distributions~// 
Nav. Res. Log., 1982.
Vol.~29. Iss.~2. P.~331--344.

\bibitem{Conti} %26
\Au{Conti P.\,L.\,J.} An asymptotic test for a geometric process against 
a~lattice distribution with monotone hazard~// Ital. Stat. Soc., 1997. 
Vol.~6. Iss.~3. P.~213--231.

\bibitem{sx1} %27
\Au{Artikis T., Voudouri~A., Malliaris~M.}
Certain selecting and underreporting processes~// Math. Comput. Model., 1994. 
Vol.~20. Iss.~1. P.~103--106.

\bibitem{sx2} %28
\Au{Королев В.\,Ю., Корчагин~А.\,Ю., Зейфман~А.\,И.} Теорема Пуассона для схемы
испытаний Бернулли со случайной вероятностью успеха и~дискретный аналог
распределения Вейбулла~// Информатика и~её применения, 2016. Т.~10. Вып.~4. С.~11--20.

 \end{thebibliography}

 }
 }

\end{multicols}

\vspace*{-6pt}

\hfill{\small\textit{Поступила в~редакцию 15.10.19}}

%\vspace*{8pt}

%\pagebreak

\newpage

\vspace*{-28pt}

%\hrule

%\vspace*{2pt}

%\hrule

%\vspace*{-2pt}

\def\tit{DISCRETE-TIME $\mathrm{Geo}/G/1/\infty$ LIFO QUEUE\\
 WITH~RESAMPLING POLICY}


\def\titkol{Discrete-time $\mathrm{Geo}/G/1/\infty$ LIFO queue with~resampling policy}

\def\aut{L.\,A.~Meykhanadzhyan$^1$ and R.\,V.~Razumchik$^{2,3}$}

\def\autkol{L.\,A.~Meykhanadzhyan and R.\,V.~Razumchik}

\titel{\tit}{\aut}{\autkol}{\titkol}

\vspace*{-11pt}


\noindent
$^1$Financial University under the Government of the Russian Federation,
49~Leningradsky Prosp., Moscow 125993,\linebreak
$\hphantom{^1}$Russian Federation

\noindent
$^2$Institute of Informatics Problems, 
Federal Research Center ``Computer Science and Control'' of 
the Russian\linebreak
$\hphantom{^1}$Academy of Sciences, 44-2~Vavilov Str., Moscow 119333, Russian Federation

\noindent
$^3$Peoples' Friendship University of Russia (RUDN University), 
6~Miklukho-Maklaya Str., Moscow 117198, Russian\linebreak
$\hphantom{^1}$Federation

\def\leftfootline{\small{\textbf{\thepage}
\hfill INFORMATIKA I EE PRIMENENIYA~--- INFORMATICS AND
APPLICATIONS\ \ \ 2019\ \ \ volume~13\ \ \ issue\ 4}
}%
 \def\rightfootline{\small{INFORMATIKA I EE PRIMENENIYA~---
INFORMATICS AND APPLICATIONS\ \ \ 2019\ \ \ volume~13\ \ \ issue\ 4
\hfill \textbf{\thepage}}}

\vspace*{3pt}  




\Abste{Consideration is given to the problem of estimation of 
the true stationary mean response time 
in the discrete-time single-server queue of infinite capacity,
with Bernoulli input, round-robin scheduling,
and inaccurate information about the 
service time distribution which is considered to be general arithmetic.
It is shown that the upper bound for the true value
may be provided by the mean response time in the discrete-time 
single-server queue with LIFO (last in, first out) service discipline and
resampling policy. The latter implies that 
a~customer arriving to the nonidle system
assigns new remaining service time for the customer in the server.
For the case when the true service time distribution 
is geometric and the error in the service times
is of multiplicative type, conditions are provided 
which, when satisfied, guarantee that the proposed method 
yields the upper bound across all possible values of the system's load.} 


\KWE{discrete time; inverse service order; inaccurate service time; 
round robin scheduling; resampling policy}




\DOI{10.14357/19922264190410} 

%\vspace*{-14pt}

 \Ack
\noindent
The reported study was partly supported by the
Russian Foundation for Basic Research according
to the research project No.\,18-37-00283.


%\vspace*{-6pt}

  \begin{multicols}{2}

\renewcommand{\bibname}{\protect\rmfamily References}
%\renewcommand{\bibname}{\large\protect\rm References}

{\small\frenchspacing
 {%\baselineskip=10.8pt
 \addcontentsline{toc}{section}{References}
 \begin{thebibliography}{99}


\bibitem{Schass2-1}
\Aue{Daduna, H., and R.~Schassberger.} 1981.
A discrete-time round-Robin queue with Bernoulli input
and general arithmetic service time distributions.
\textit{Acta Inform.} 15(3):251--263.


\bibitem{distime-1} %2
\Aue{Pechinkin, A.\,V., and R.\,V.~Razumchik.} 2018.
\textit{Sistemy massovogo obsluzhivaniya v~diskretnom vremeni}
[Discrete time queuing systems]. Moscow: Fizmatlit. 432~p.


\bibitem{Schass-1} %3
\Aue{Schassberger, R.} 1981. On the 
response time distribution in a discrete round-Robin queue. 
\textit{Acta Inform.} 16(1):57--62.

\bibitem{new3-1} %4
\Aue{Tatashev, A.\,G.}
 1999. On an inverse servicing discipline in 
 a~queue with customers of different types. \textit{Automat. Rem. Contr.} 
 60(7):1050--1053.

\bibitem{new4-1} %5
\Aue{Tatashev, A.\,G.} 2003. A~queueing system with inverse discipline, 
two types of customers, and Markov input flow. \textit{Automat. Rem. Contr.} 64(11):1755--1759.


\bibitem{new1-1} %6
\Aue{Fiems, D., B.~Steyaert, and H.~Bruneel.} 2004.
Discrete-time queues with generally distributed service times and 
renewal-type server interruptions. \textit{Perform. Eval.}  55(\mbox{3-4}):277--298. 


\bibitem{wfb-1} %7
\Aue{Walraevens, J., D. Fiems, and H.~Bruneel.} 2006.
The discrete-time preemptive repeat identical priority queue. 
\textit{Queueing Sy.} 53(4):231--243.

\bibitem{new2-1} %8
\Aue{Pechinkin, A., and S.~Shorgin.} 2008. 
The discrete-time queueing system with inversive service order and probabilistic priority. 
\textit{3rd  Conference (International) on Performance Evaluation Methodologies 
and Tools Proceedings}. Brussels: ICST. Art. No.\,20. 6~p. 

\bibitem{nm1-1}  %9
\Aue{Milovanova, T.\,A.} 2009. ${\mathrm{BMAP}/G/1/\infty}$ system with last
come first served probabilistic priority. \textit{Automat. Rem.
\mbox{Contr.}} 70(5):885--896.



\bibitem{stal-1} %10
\Aue{Pechinkin, A.\,V., and I.\,V.~Stalchenko.} 2010. Sistema
$\mathrm{MAP}/G/1/\infty$ s~inversionnym poryadkom obsluzhivaniya 
i~veroyatnostnym prioritetom, funktsioniruyushchaya v~diskretnom vremeni 
[The $\mathrm{MAP}/G/1/\infty$ discrete-time
queueing system with inversive service order and probabilistic priority]. 
\textit{Vestnik Rossiyskogo Universiteta Druzhby Narodov. Ser.
 Matematika. Informatika. Fizika} [Bulletin of
Peoples' Friendship University of Russia. Ser. Mathematics. 
Information Sciences. Physics] 2:26--36.

\bibitem{n5-1}  %11
\Aue{Milovanova, T.\,A., and A.\,V.~Pechinkin.} 2013. Sta\-tsi\-o\-nar\-nye 
kharakteristiki sistemy obsluzhivaniya s~inversionnym poryadkom obsluzhivaniya, 
veroyatnostnym pri\-o\-ri\-te\-tom i~gisterezisnoy politikoy [Stationary characteristics
of queuing system with an inversion procedure service
probabilistic priority and hysteresis policy]. \textit{Informatika
i~ee Primeneniya~--- Inform. Appl.} 7(1):22--35.

\bibitem{mey-1} %12
\Aue{Meykhanadzhyan, L.\,A.}
 2016. Statsionarnye ve\-ro\-yat\-nosti sostoyaniy v~sisteme obsluzhivaniya konechnoy emkosti s inversionnym poryadkom obsluzhivaniya i obobshchennym veroyatnostnym prioritetom
[Stationary characteristics of the finite capacity queueing 
system with inverse service order and generalized probabilistic priority].
\textit{Informatika i ee Primeneniya~--- Inform. Appl.} 10(62):123--131.

\bibitem{new5-1} %13
\Aue{Afanaseva, L.\,G., and A.\,W.~Tkachenko.} 2019. 
Stability conditions for queueing systems with regenerative flow of interruptions. 
\textit{Theor. Probab. Appl.} 63(4):507--531.

\bibitem{n0-1} %14
\Aue{Razumchik, R.} 2019. Two-priority queueing system with LCFS service, 
probabilistic priority and batch arrivals. \textit{AIP Conf. Proc.} 
2116(1):090011-1--090011-3.


\bibitem{opech-1} %15
\Aue{Pechinkina, O.\,A.} 1995.
Asimptoticheskoe raspredelenie dliny ocheredi v~sisteme $M/G/1$ 
s~in\-ver\-si\-on\-noy ve\-ro\-yat\-nost\-noy distsiplinoy obsluzhivaniya
[Queue size asymptotic distribution for $M/G/1$ system with inverse probabilistic service discipline].
\textit{Vestnik Rossiyskogo Universiteta Druzhby Narodov. Ser. Matematika. 
Informatika. Fizika} [Bulletin of
Peoples' Friendship University of Russia. Ser. Mathematics. Information Sciences.
 Physics] 1:87--100.

\bibitem{Mat-2-1} %16
\Aue{Horv$\acute{\mbox{a}}$th, I., R.~Razumchik, and M.~Telek}. 2019.
The resampling $M/G/1$ non-preemptive LIFO queue and its application 
to systems with uncertain service time. \textit{Perform. Evaluation} 134:102000. 13~p.



\bibitem{limic-1} %17
\Aue{Limic, V.} 2001. A~LIFO queue in heavy traffic. 
\textit{Ann. Appl. Probab.} 11(2):301--331.

\bibitem{asmus-1} %18
\Aue{Asmussen, S., and P.\,W.~Glynn}. 2017. On preemptive-repeat LIFO queues. 
\textit{Queueing Sy.} 87(1-2):1--22.



\bibitem{finch-1} %19
\Aue{Finch, P.\,D.} 1959. 
The output process of the queueing system $M/G/1$. 
\textit{J.~Roy. Stat. Soc.~B Met.} 21(2):375--380.

\bibitem{Mat-1} %20
\Aue{Dell'Amico, M., D.~Carra, and P.~Michiardi.} 2016. 
\mbox{PSBS}: Practical size-based scheduling. 
\textit{IEEE~T. Comput.} 65(7): 2199--2212.


\bibitem{LusMilR-1} %21
\Aue{Milovanova, T.\,A., L.\,A.~Meykhanadzhyan, and R.\,V.~Razumchik.} 2018.
Bounding moments of Sojourn time in M/G/1 
FCFS queue with inaccurate job size information and additive error: Some 
observations from numerical experiments. \textit{CEUR Workshop Proceedings} 
2236:24--30.


\bibitem{new8-2-1} %22
\Aue{Klefsj$\ddot{\mbox{o}}$, B.} 1983.
A~useful ageing property based on the Laplace transform. 
\textit{J. Appl. Probab.} 20(3):615--626.

\bibitem{new7-1} %23
\Aue{Barlow, R., and F. Proschan.} 1965. \textit{Mathematical theory of reliability}.
New York, NY: Wiley. 274~p.

\bibitem{new6-1} %24
\Aue{Stoyan, D.} 1977. \textit{Qualitative Eigenschaften und Abschtzungen 
stochastischer Modelle}. Berlin: Akademie-Verlag. 198~p.




\bibitem{new8-1-1} %25
\Aue{Klefsj$\ddot{\mbox{o}}$, B.} 1982. The hnbue and hnwue 
classes of life distributions. \textit{Nav. Res. Log.} 29(2):331--344. 

\bibitem{Conti-1}
\Aue{Conti, P.\,L.\,J.} 1997. An asymptotic test for a geometric process against 
a~lattice distribution with monotone hazard.  \textit{Ital. Stat. Soc.} 6(3):213--231. 


\bibitem{sx1-1}
\Aue{Artikis, T., A.~Voudouri, and M.~Malliaris.} 1994.
Certain selecting and underreporting processes. \textit{Math. Comput. Model.} 
20(1):103--106.

\bibitem{sx2-1}
\Aue{Korolev, V.\,Yu., A.\,Yu.~Korchagin, and A.\,I.~Zeifman.} 2016.
Teorema Puassona dlya skhemy ispytaniy Bernulli so sluchaynoy veroyatnost'yu 
uspekha i~diskretnyy analog raspredeleniya Veybulla [The Poisson theorem 
for Bernoulli trials with 
a~random probability of success and a discrete analog of the Weibull distribution].
\textit{Informatika i~ee Primeneniya~--- Inform. Appl.} 10(4):11--20.
\end{thebibliography}

 }
 }

\end{multicols}

%\vspace*{-7pt}

\hfill{\small\textit{Received October 15, 2019}}

%\pagebreak

%\vspace*{-22pt}

\Contr

\noindent
\textbf{Meykhanadzhyan Lusine A.} (b.\ 1990)~---
Candidate of Science (PhD) in physics and
mathematics, associate professor,
Department of Data Analysis, Decision-Making and Financial Technology,
Financial University under the Government of the Russian Federation,
49~Leningradsky Prosp., Moscow 125993, Russian Federation;
\mbox{lamejkhanadzhyan@fa.ru}

\vspace*{3pt}

\noindent
\textbf{Razumchik Rostislav V.} (b.\ 1984)~---
Candidate of Science (PhD) in physics and mathematics, leading scientist,
Institute of Informatics Problems, Federal Research Center 
``Computer Science and Control'' of the Russian Academy of Sciences, 
44-2~Vavilov Str., Moscow 119333, Russian Federation; associate professor,
 Peoples' Friendship University of Russia (RUDN University), 
 6~Miklukho-Maklaya Str., Moscow 117198, Russian Federation; \mbox{rrazumchik@ipiran.ru}



\label{end\stat}

\renewcommand{\bibname}{\protect\rm Литература}  