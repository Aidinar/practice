\def\stat{flerov}

\def\tit{ЦИФРОВАЯ МОДЕЛЬ ВЕСОВОГО ПАСПОРТА ЛЕТАТЕЛЬНОГО АППАРАТА}

\def\titkol{Цифровая модель весового паспорта летательного аппарата}

\def\aut{Л.\,Л.~Вышинский$^1$, М.\,К.~Курьянский$^2$, Ю.\,А.~Флеров$^3$}

\def\autkol{Л.\,Л.~Вышинский, М.\,К.~Курьянский, Ю.\,А.~Флеров}

\titel{\tit}{\aut}{\autkol}{\titkol}

\index{Вышинский Л.\,Л.}
\index{Курьянский М.\,К.}
\index{Флеров Ю.\,А.}
\index{Vyshinsky L.\,L.}
\index{Kuryansky M.\,K.}
\index{Flerov Yu.\,A.}


%{\renewcommand{\thefootnote}{\fnsymbol{footnote}} \footnotetext[1]
%{Работа выполнена при частичной поддержке РФФИ (проект 19-07-00187-A).}}


\renewcommand{\thefootnote}{\arabic{footnote}}
\footnotetext[1]{Вычислительный центр им.\ А.\,А.~Дородницына Федерального исследовательского центра <<Информатика 
и~управление>> Российской академии наук, \mbox{wysh@ccas.ru}}
\footnotetext[2]{Научно-технический центр <<Объединенная авиастроительная корпорация>>, 
\mbox{m.kuryanskiy@uacrussia.ru}}
\footnotetext[3]{Вычислительный центр им.\ А.\,А.~Дородницына Федерального исследовательского центра 
<<Информатика и~управление>> Российской академии наук, \mbox{fler@ccas.ru}}

\vspace*{-8pt}
   
 
  

  \Abst{Статья посвящена вопросам создания цифровой модели весового паспорта 
летательных аппаратов (ЛА). Весовой паспорт разрабатывается на этапе проектирования нового 
изделия и~сопровождает его на всех этапах жизненного цикла. Наибольшее значение 
цифровой весовой паспорт приобретает в~процессе эксплуатации готовых изделий. 
Программная реализация весового паспорта служит не только справочным 
эксплуатационным пособием, но и~инструментом для проведения сложных весовых расчетов 
при подготовке полетных заданий, при проведении регламентных и~ремонтных работ. 
Представлена концепция и~программная реализация цифрового весового паспорта 
ЛА.}
  
  \KW{цифровая модель; автоматизация проектирования; летательный аппарат; весовое 
проектирование; весовая модель; дерево конструкции; генератор проектов}

\DOI{10.14357/19922264190401} 
  
\vspace*{-4pt}


\vskip 10pt plus 9pt minus 6pt

\thispagestyle{headings}

\begin{multicols}{2}

\label{st\stat}
  
\section{Понятие весового паспорта летательного аппарата}

\vspace*{-4pt}

  Введение в~обиход нового термина всегда связано с~определенным риском 
появления урод\-ли\-вого мема, допускающего многозначные ин\-тер\-пре\-тации. 
Термин <<весовой паспорт летательного\linebreak аппарата>>, который ранее не 
употреблялся в~самолетостроении, требует своего толкования и~объяснения 
необходимости введения такого понятия. Согласно ГОСТ~2.601-2013 (ЕСКД) 
в~со\-ста\-ве эксплуатационной документации на любую вы\-пус\-ка\-емую продукцию 
должен быть паспорт~--- <<документ, содержащий сведения, удостоверяющие\linebreak 
гарантии изготовителя, значения основных па\-ра\-мет\-ров и~характеристик 
(свойств) изделия, а~также сведения о~сертификации и~утилизации>>. 
  
  Цифровой весовой паспорт ЛА~--- это дополнение к~обычному техническому 
паспорту, которое детализирует такие основные параметры изделия, как вес, 
положение центра масс и~другие мас\-со\-во-инер\-ци\-он\-ные характеристики 
ЛА. Эти па\-ра\-метры напрямую влияют на эффективность 
и~безопас\-ность эксплуатации воздушных судов. Однако информационное 
дополнение к~существующему документу, даже если оно содержит очень 
важную\linebreak информацию, не может служить поводом для введения в~обиход 
нового термина и~нового типа эксплуатационной документации. Разумеется, 
это важное дополнение к~основному документу в~соответствии с~современными 
требованиями должно быть реализовано как специальное компьютерное 
приложение. Но и~это не ново. Отличительная особенность вводимого  
в~данной работе понятия со\-сто\-ит в~характере реализации циф\-ро\-во\-го весового 
паспорта. Цифровой весовой паспорт здесь~--- это не просто электронный 
документ, позволяющий на компьютере хранить и~просматривать техническую 
информацию. Цифровой паспорт должен: (а)~содержать весь необходимый 
объем весовой информации об изделии; (б)~предоставлять ее для анализа 
и~контроля; (в)~обеспечивать поддержание весовой информации об изделии 
в~актуальном со\-сто\-янии в~течение всего жизненного цик\-ла и,~главное, 
(г)~предостав\-лять воз\-мож\-ность проведения необходимых весовых расчетов, в~том 
числе при подготовке изделия к~полету. 
  
  Таким образом, весовой паспорт, с~одной стороны, является 
информационным эксплуата\-ционным документом, а~с~другой стороны, это\linebreak 
программный инструмент, который должен по\-став\-лять\-ся вместе с~готовым 
изделием. Аналогично можно было бы говорить не только о~весовых 
программных эксплуатационных продуктах, но и~о~других программах, 
направленных на различные аспекты эксплуатации ЛА или 
их систем. В~данной работе речь идет о~весовом паспорте, поскольку весовая 
информация охватывает очень широкий круг вопросов, а~ее накопление, 
хранение, использование~--- это сложнейший ор\-га\-ни\-за\-ци\-он\-но-тех\-ни\-че\-ский 
и~ин\-фор\-ма\-ци\-он\-но-вы\-чис\-ли\-тель\-ный процесс. Полная весовая информация об 
изделии составля-\linebreak\vspace*{-12pt}

\pagebreak

\noindent
ет многие мегабайты цифровых данных, которые сложно 
организованы в~многоуровневые иерархические структуры, отражающие 
конструкции тысяч деталей, узлов и~агрегатов. Представлять мегабайты 
и~структуры для визуального контроля, как это обычно бывает 
в~эксплуатационных документах, бессмысленно. Однако эта информация 
необходима для проведения весовых расчетов и~весового контроля в~процессе 
эксплуатации ЛА. Замена узлов и~деталей, вышедших из 
строя (а~таких деталей бывает много), проведение разных форм ремонта, 
различные варианты размещения на борту перевозимой целевой нагрузки, 
процессы заправки и~выработки топлива~--- все эти процессы требуют расчетов 
и~анализа критических па\-ра\-мет\-ров цент\-ров\-ки, полетных  
и~взлет\-но-по\-са\-доч\-ных характеристик. Су\-ще\-ст\-ву\-ющие формы 
эксплуатационной документации не могут в~полном объеме обеспечить 
решение всех перечисленных задач.
  
  Современные информационные технологии и~вычислительные мощности 
компьютеров позволяют держать в~памяти всю необходимую весовую 
информацию в~структурированном упорядоченном виде и~управ\-лять ею для 
скрупулезного весового  контроля при решении всех задач эксплуатации, 
начиная с~испытаний и~сертификации и~до утилизации изделия. В~этом 
и~состоит задача цифрового весового паспорта ЛА.   
  
  Создание информационной базы для весового паспорта и~его комплектация 
как программного продукта должны вестись параллельно с~разработкой самого 
изделия. Это должно быть задачей процесса весового проектирования, одного 
из важнейших процессов создания ЛА.

  \begin{figure*} %fig1
  \vspace*{1pt}
    \begin{center}  
  \mbox{%
 \epsfxsize=163mm 
 \epsfbox{fle-1.eps}
 }
\end{center}
\vspace*{-10pt}
  \Caption{Структура весового паспорта ЛА}
  \vspace*{-3pt}
  \end{figure*}
  
\vspace*{-6pt}
  
\section{Весовое проектирование летательного аппарата}

\vspace*{-2pt}

  На самых начальных этапах проектирования закладывается весовой облик 
ЛА. Под весовым обликом обычно понимают набор тех 
параметров, которые потом указываются во всех эксплуатационных 
документах, справочниках и~энциклопедиях: нормальный и~максимальный 
взлетный вес, максимальный вес перевозимой нагрузки, необходимый запас 
топлива, вес пустого (снаряженного) изделия и~некоторые другие 
немногочисленные величины. Разумеется, на начальной стадии существуют 
лишь некоторые оценки параметров весового облика. Как правило, такие 
оценки вы\-чис\-ля\-ют\-ся на основе анализа возможных прототипов ЛА и/или 
с~применением упрощенных моделей функционирования. На следующем 
этапе, на этапе формирования облика ЛА, когда появляется компоновочная 
схема и~выбираются па\-ра\-мет\-ры основных агрегатов (крыла, фюзеляжа, 
оперения, силовой установки), весовой облик уточняется и~детализируется. 
Одновременно закладываются данные для дальнейшей проработки 
конструкции планера, проектирования бортовых систем, всего комплекса 
бортового обору\-до\-ва\-ния, а~также размещения полезной нагрузки и~топливных 
емкостей.

 Одна из центральных задач в~детальной проработке проекта~--- 
весовые расчеты, весовой анализ и~оптимизация весовых параметров. 
Все эти задачи принято называть весовым проектированием 
ЛА. Термин <<весовое проектирование>> был введен в~книге 
В.\,М.~Шейнина и~В.\,И.~Козловского~[1]. Важность этого аспекта 
проектирования обуслов\-ли\-ва\-ет наличие в~авиационных конструкторских бюро 
специальных весовых бригад, весовых отделов. Сложность задач весового 
проектирования состоит в~экспоненциальном росте числа деталей по мере 
углубления проекта, сетевого разрастания связей между его отдельными 
компонентами, необходимости постоянного мониторинга весовых параметров 
как отдельных агрегатов, включая узлы и~детали, так и~параметров весового 
облика изделия в~целом. Размерность возникающих задач весового 
проектирования постоянно растет. В~современных проектах число деталей, 
узлов и~агрегатов доходит до сотен тысяч. 

Одна из главных задач процесса 
весового проектирования~--- упорядочение и~систематизация всей весовой 
информации, т.\,е.\ построение строгой, формальной, информационной весовой 
модели ЛА, пригодной для использования на всех этапах жизненного цик\-ла 
изделия. Весовая модель ЛА, которая лежит в~основе процессов весового 
проектирования, достаточно подробно описана в~[2]. На базе этой модели 
строится струк\-тур\-но-па\-ра\-мет\-ри\-че\-ская информационная модель весового 
паспорта. Весовую модель ЛА можно разделить на <<условно постоянную>> 
и~<<переменную>> части. По\-сто\-ян\-ная часть весовой модели описывает 
собранную на производстве конструкцию изделия, которую в~весовых 
классификаторах принято называть <<пустым>> изделием. К~переменной же 
части относят те компоненты, которые в~соответствии с~проектом 
и~назначением ЛА размещаются на борту непосредственно перед полетом. Эти 
две части весовой модели играют разную роль в~весовом паспорте. 

Описание 
конструкции пустого изделия на этапе эксплуатации важно при выполнении 
регламентных работ, при контроле выработки ресурса отдельных агрегатов, 
при ремонте и~замене деталей, узлов и~целых агрегатов, например двигателей, 
выработавших свой ресурс. Информация в~весовом паспорте о~переменных 
компонентах ЛА используется регулярно в~процессе подготовки полетных 
заданий. Исходя из этих соображений, строится информационная структура 
весового паспорта.

\vspace{-6pt}
  
\section{Структура весового паспорта летательного аппарата}

\vspace*{-2pt}

  Информационная модель, положенная в~основу весового паспорта, не во 
всем совпадает с~весовой моделью этапа проектирования. На первое место 
в~весовом паспорте выносятся эксплуатационные параметры, которые 
представляют весовой облик изделия. На рис.~1 приведена структура весового 
паспорта.
  

  На первой его экранной форме выводятся общие виды изделия, таблица 
параметров весового облика и~информационные панели основных структур 
весовой модели изделия:
  \begin{itemize}
\item  дерево конструкции пустого изделия;\\[-15pt]
\item варианты снаряжения и~служебной нагрузки;\\[-15pt]
\item варианты целевой загрузки;\\[-15pt]
\item варианты программ заправки и~выработки топлива;\\[-15pt]
\item весовой журнал полетных заданий и~модификаций весового паспорта.
\end{itemize}

  В таблице <<Весовой облик изделия>> даны значения основных весовых 
параметров, в~том числе тех параметров, декларация которых необходима при 
сертификации ЛА\footnote{В~связи с~необходимостью международной сертификации 
вместе с~принятыми в~российской авиационной отрасли обозначениями в~весовом паспорте 
желательно приводить принятые аббревиатуры этих параметров в~англоязычном 
употреблении.}. Как правило, параметры весового облика изделия служат 
ограничителями при подготовке и~выполнении полетных заданий. Набор 
параметров весового облика изделия зависит от назначения ЛА и~может 
варьироваться в~определенных пределах. Параметры весового облика 
детализируются в~основных разделах паспорта.

\begin{figure*} %fig2
\vspace*{1pt}
    \begin{center}  
  \mbox{%
 \epsfxsize=163mm 
 \epsfbox{fle-2.eps}
 }
\end{center}
\vspace*{-10pt}
\Caption{Дерево конструкции пустого изделия}
\vspace*{-3pt}
\end{figure*}
  
  
\vspace*{-7pt}
  
\section{Дерево конструкции пустого изделия }

\vspace*{-4pt}

  В разделе <<Дерево конструкции пустого изделия>> представлена 
постоянная составляющая весовой модели. Конструкция пустого изделия 
представляет собой иерархическую структуру, которая включает собственно 
конструкцию, силовую установку, самолетные системы и~специальное бортовое 
оборудование. Структура дерева конструкции выстраивается в~соответствии 
с~принятым весовым классификатором. Вообще, в~авиационной отрасли нет 
единого весового классификатора. Есть лишь определенные практики, которые 
связаны с~типом и~назначением ЛА, а~также с~традициями 
проектной организации. 

На рис.~2 приведен пример отображения информации 
в~разделе <<Дерево конструкции пустого изделия>>.
В~левой части экранной формы представлена структура конструкции, 
а~в~правой~--- информационная карта выбранного элемента. Надо сказать, что 
на разных стадиях жизненного цик\-ла ЛА структура конструкции может быть 
представлена по-раз\-но\-му, поэтому в~весовом паспорте может потребоваться 
реструктуризация весовой модели, построенной на этапе проектирования. Это 
связано с~некоторыми отличиями логики использования весовой информации 
при проектировании и~при эксплуатации изделий.
  
  Идентификация и~кодификация элементов конструкции~--- важный аспект 
проектирования, от которого во многом зависит удобство работы со 
структурами и~параметрами изделия. В~авиа\-стро\-ении есть ряд отраслевых 
стандартов, определяющих правила присвоения идентификаторов, чертежных 
номеров разным группам элементов конструкции. Но в~целом этот аспект 
разработки весовых моделей находится в~компетенции проектировщиков.
  
  Основными параметрами элементов конструкции в~весовом паспорте служат 
мас\-со\-во-инер\-ци\-он\-ные характеристики~--- масса, координаты центра тяжести, 
моменты инерции. Положение центра тяжести и~моменты инерции задаются 
в~системе координат с~началом в~точке ($X_0, Y_0, Z_0$) и~тремя углами 
поворота ($\alpha$, $\beta$, $\gamma$) относительно указанной по ссылке другой 
системы координат. Таким образом, в~дереве конструкции может быть задано 
несколько вложенных систем координат, что создает удобство по\-стро\-ения 
весовой модели. В~весовом паспорте в~качестве справочного материала могут 
быть заданы другие данные, облегчающие контроль и~поиск  
неисправностей,~--- позиционные параметры сборок, дополнительные 
указатели мест размещения деталей и~пр. Со\-ста\-вы параметров разных 
агрегатов могут отличаться за счет характеристик, опре\-де\-ля\-ющих 
специфические свойства конструкции. Полный набор па\-ра\-мет\-ров элемента 
конструкции определяется типом элемента. Тип служит важным параметром, 
позволяющим осуществлять выборки и~формировать различные сводки 
и~реестры, которые могут быть полезны в~процессе эксплуатации изделий. 
Например, реестры для анализа рас\-по\-ла\-га\-емых ресурсов силовых элементов~--- 
шпангоутов, нервюр, стрингеров, реестры агрегатов для установки 
перевозимых грузов, узлов крепления подвесных топливных баков и~других 
элементов переменной части весовой модели. 

\vspace*{-6pt}

\section{Переменная часть весовой модели летательного аппарата}

\vspace*{-4pt}
 
  К переменной части весовой модели относят снаряжение 
ЛА, служебную и~целевую нагрузку, а~также расходуемое топливо. 
В~соответствующих разделах весового паспорта даются перечни элементов 
нагрузки, которые могут быть размещены на борту данного 
ЛА. Для разных типов ЛА характер перевозимой 
нагрузки имеет принципиальные различия. Для пассажирских воздушных судов 
трудно заранее точно определить состав и~общую массу пассажиров с~багажом. 
Можно лишь приблизительно оценить вес нагрузки в~пассажирских салонах 
и~в~багажных отсеках. Но при подготовке конкретного рейса могут быть 
известны коэффициенты заполнения салонов и~багажных отсеков, на основании 
которых оценивается реальная загрузка. Если же ЛА
пред\-на\-зна\-чен для контейнерных перевозок и/или перевозок крупногабаритных 
грузов, в~весовой паспорт можно включить реестры таких элементов нагрузки 
с~указанием габаритных размеров и~мас\-со\-во-инер\-ци\-он\-ных 
характеристик. Для ЛА военного назначения целевая нагрузка размещается 
либо в~специальных внутренних отсеках, либо на внешних подвесках.
  
  Список допустимых к~установке на борту элементов нагрузки~--- важная 
составляющая весового паспорта. Включение какого-либо элемента 
вооружения или крупногабаритного груза специального назначения в~перечень 
допустимой нагрузки, как правило, требует отдельных расчетов или 
согласования с~конструкторским бюро, несущим ответственность за 
функционирование изделия. Отдельной задачей весовых расчетов, которые 
должны быть реализованы в~весовом паспорте, является анализ динамики 
изменения центровки ЛА в~процессе перемещения на борту крупногабаритных 
грузов или при сбросе нагрузки в~полете. В~основном, это относится 
к~транспортным самолетам, к~самолетам и~вертолетам, используемым при 
пожаротушении, и~т.\,п. Весовой паспорт должен позволять провести все 
необходимые весовые расчеты, связанные с~проверкой весовых 
эксплуатационных ограничений. 
  
  Для удобства работы в~весовом паспорте при подготовке полетных заданий 
данные о допустимых элементах нагрузки должны быть систематизированы, 
классифицированы и~организованы в~специальные каталоги и~реестры. Схема 
построения каталогов элементов нагрузки такая же, как и~для дерева 
конструкции. Каждый элемент нагрузки снабжен информационной картой 
с~присущим данному элементу набором параметров. Так же, как и~в~дереве 
конструкции пустого изделия, основными параметрами служат  
мас\-со\-во-инер\-ци\-он\-ные характеристики, которые при установке на борту 
пересчитываются в~зависимости от места их размещения. Перечни допустимых 
мест размещения конкретных элементов нагрузки являются необходимыми 
атрибутами каталогов нагрузки.
\vspace*{-6pt}

\section{Потребный запас топлива }

\vspace*{-2pt}

  Важнейший аспект подготовки полетных заданий ЛА~--- вычисление 
необходимого запаса расходуемого топлива и~анализ изменения массы 
и~центровки ЛА в~полете. Объем заправляемого топлива 
отражается не только на летных характеристиках, но и~на экономической 
эффективности эксплуатации изделия, поэтому в~весовом паспорте должна 
содержаться вся необходимая информация для расчета потребного запаса 
топлива: 

\vspace*{-2pt}

\noindent
 \begin{multline*}
  M_{\mathrm{потр\_топл}} ={}\\
  {}= M_{\mathrm{рул}} + M_{\mathrm{взл\_пос}} + 
M_{\mathrm{марш}} + M_{\mathrm{комп}} + M_{\mathrm{рез}}, 
\end{multline*}

\vspace*{-2pt}

\noindent
  где $M_{\mathrm{потр\_топл}}$~--- потребный запас топлива для 
выполнения полетного задания;
  $M_{\mathrm{рул}}$~--- количество топлива, расходуемого двигателями на 
земле при прогреве, опробовании и~рулении;
  $M_{\mathrm{взл\_пос}}$~--- количество топлива, расходуемого на взлет 
и~посадку;
  $M_{\mathrm{марш}}$~--- количество топлива, расходуемого в~полете от 
исходного пункта маршрута до конечного пункта или до пункта следующей 
заправки;
  $M_{\mathrm{комп}}$~--- компенсационный запас топлива, который 
учитывает возможные навигационные потери, связанные с~отклонениями от 
маршрута по метеорологическим и~другим обстоятельствам;
  $M_{\mathrm{рез}}$~--- резервный запас для ухода на второй круг или на 
запасной аэродром.
  
  В топливной системе ЛА в~процессе эксплуатации всегда существует 
невырабатываемый и~несливаемый остаток топлива. Масса этого топлива 
согласно весовому классификатору учитывается в~массе снаряженного изделия. 

\begin{figure*} %fig3
\vspace*{1pt}
    \begin{center}  
  \mbox{%
 \epsfxsize=163mm 
 \epsfbox{fle-3.eps}
 }
\end{center}
\vspace*{-9pt}
\Caption{Программа выработки топлива}
\end{figure*}

  
  Расчет всех компонент потребного запаса топлива ведется обычно по 
действующим нормам и~правилам летной эксплуатации воздушных судов 
и~зависит от параметров ЛА и~многих параметров конкретного полетного задания 
и~условий базирования. Для разных типов ЛА алгоритмы расчетов могут 
отличаться. Кроме внешних факторов вся информация для выполнения 
расчетов потребного запаса топлива должна содержаться в~цифровом весовом 
паспорте, а~программные компоненты должны включать все необходимые 
расчетные процедуры, соответствующие регламенту подготовки ЛА к~полету. 
Рассчитанный объем потребного запаса топлива может потребовать включения в~полетную конфигурацию изделия дополнительных подвесных или вкладных 
топливных емкостей. Размещение мобильных топливных емкостей определяет 
динамические массово-инерционные характеристики текущего запаса топлива 
на борту. В~полете происходит автоматическое переключение выработки 
топлива из разных баков. Переключения выработки топлива от бака к~баку, 
а~также перекачка топлива из одного бака в~другой обусловлены 
необходимостью выполнения в~полете ограничений по центровке, т.\,е.\ 
необходимостью сохранения центра тяжести в~определенных границах. 
Программа переключения выработки заложена в~устройства автоматики 
топливной системы или в~бортовой компьютер, управляющий топливной 
системой. В~весовом паспорте программа выработки топлива пред\-став\-ля\-ет\-ся 
в~виде цифровых таблиц послойного расхода топлива из разных баков. На 
рис.~3 приведен пример визуализации хранящейся в~весовом паспорте 
информации по программам выработки топлива. Как уже говорилось, 
программа выработки топлива зависит от полетной конфигурации размещения 
подвесных и~вкладных топливных баков.
{ %\looseness=1

}
  

  Слева на экране визуализируется размещение топливных баков и~приводится 
диаграмма текущих остатков топлива в~каждом из баков. (В~данной 
конфигурации представлены только стационарные баки.) Справа выводится 
таблица последовательной выработки топлива. В~ней указывается, из какого 
бака и~в каком количестве вырабатывается топливо. В~цент\-ре диаграммы дан 
график изменения массы текущего остатка топлива и~положения центра масс 
в~координатах $\langle X_{\mathrm{топл}}, M_{\mathrm{топл}}\rangle$. 
Переломы на графике соответствуют моментам переключения выработки 
топлива с~одного бака на другой. Эта информация используется при анализе 
полетного задания, когда суммируются все весовые данные по пустому 
изделию, снаряжению и~целевой нагрузки.

\vspace*{-6pt}

\section{Весовой журнал полетных заданий}

\vspace*{-2pt}

  Одна из основных задач, в~которых может и~должен использоваться такой 
программный инструмент, как весовой паспорт,~--- это весовой анализ при 
подготовке к~выполнению полетных заданий. Цель предполетного анализа 
состоит в~проверке всех взлетно-посадочных и~полетных ограничений. Такой 
анализ должен проводиться при каждом вылете при конкретных вариантах 
снаряжения, целевой нагрузки и~заправки топливных баков. Если в~полетном 
задании предусмотрена дозаправка в~воздухе или сброс целевой нагрузки, то 
должны быть промоделированы все подобные ситуации полетного задания. 
Актуальной задачей в~планировании и~реализации использования изделий 
является коммерческая оптимизация параметров полетных заданий~--- 
экономия на топливе, на времени подготовки к~вылету, на погрузке-разгрузке. 
Весовой паспорт, его информационные и~вычислительные возможности могут 
с~успехом использоваться при решении таких задач. 

\begin{figure*}[b] %fig4
\vspace*{1pt}
    \begin{center}  
  \mbox{%
 \epsfxsize=128.952mm 
 \epsfbox{fle-4.eps}
 }
\end{center}
\vspace*{-9pt}
\Caption{Весовой паспорт в~жизненном цикле ЛА}
\end{figure*}
  
  В весовом паспорте предусмотрен интерактивный режим для формирования 
условий полетного задания. Последовательность действий в~этой задаче 
состоит из следующих шагов: 
  \begin{itemize}
  \item[(а)] выбор снаряжения из множества допустимых вариантов; 
  \item[(б)] выбор варианта из множества допустимых вариантов целевой 
загрузки; 
  \item[(в)] если такой вариант не найден, то формирование специально под 
данное полетное задание варианта загрузки; 
  \item[(г)] на основании дальности и~режима полета расчет потребного запаса 
топлива с~учетом условий базирования как на аэродроме вылета/прилета, так 
и~промежуточных точках посадки, включая запасные взлет\-но-по\-са\-доч\-ные полосы;
   \item[(д)] выбор конфигурации дополнительных топливных баков для 
обеспечения заправки нужного количества топлива. 
  \end{itemize}
  
  Таким образом будет полностью сформирована полетная конфигурация ЛА. 
В~результате анализа этой конфигурации автоматически пересчитываются  
мас\-со\-во-инер\-ци\-он\-ные характеристики изделия вдоль всей траектории от 
взлета до посадки, с~учетом уборки и~выпуска шасси, со сбросом 
и~передвижением грузов. На экране выводятся результаты расчета, 
автоматически проверяются ограничения по взлету и~посадке, по нагрузке на 
стойки шасси, по передней и~задней центровке. Если заданные ограничения 
нарушены, то об этом сообщается на экране текстом о сути нарушений.
  
  В случае удовлетворения всем ограничениям выпускаются необходимые по 
регламенту документы и~все необходимые данные заносятся в~журнал 
полетных заданий весового паспорта. После выполнения полета в~журнале 
полетных заданий делаются необходимые отметки, для агрегатов конструкции 
с~ограниченным ресурсом проводится его коррекция.

  
  Любые изменения в~конструкции пустого из\-делия, связанные 
с~регламентными работами, %\linebreak 
с~плановым или внеплановым ремонтом, с~заменой 
покуп\-ных изделий, отдельных компонент обору\-до\-ва\-ния, все конструктивные 
изменения должны регистрироваться в~журнале изменений весовой модели, 
с~перерасчетом мас\-со\-во-инер\-ци\-он\-ных параметров и~параметров 
весового облика изделия. Таким образом, весовой паспорт должен отражать все 
процессы, происходящие на этапе эксплуатации~ЛА.

\vspace*{-6pt}

\section{Место цифрового весового паспорта в~жизненном цикле 
летательного аппарата}

\vspace*{-2pt}

  Цифровой весовой паспорт ЛА (так же, как и~весовая модель ЛА) состоит из 
двух частей~--- постоянной час\-ти (программного обеспечения~--- ПО) 
и~переменной час\-ти (базы данных~--- БД), пред\-став\-ля\-ющей весовую модель 
изделия. Программное обеспечение~--- это оболочка весового паспорта, а~БД~--- это его суть. 
Программное обеспечение не 
зависит от назначения и~типа ЛА. Разработка 
ПО весового паспорта~--- дело профессионалов в~об\-ласти прикладных  
ин\-фор\-ма\-ци\-он\-но-вы\-чис\-ли\-тель\-ных сис\-тем. А~вот создание весовой 
модели изделия, наполнение БД паспорта информацией о~конкретном 
проектируемом ЛА~--- это дело профессионалов в~об\-ласти авиастроения. 
  
  Хотелось бы подчеркнуть, что на всех этапах жизненного цикла ЛА, в~том 
числе на этапе эксплуатации, должно осуществляться информационное 
и~программное сопровождение цифрового весового паспорта, для того чтобы 
отслеживать новые разработки, связанные с~данным изделием. 
  
  На рис.~4 схематически показано место весового паспорта в~жизненном 
цик\-ле~ЛА.
  



\vspace{-12pt}

\section{Заключение}

\vspace*{-2pt}

  Цифровая модель весового паспорта ЛА, представленная в~данной статье, 
реализована средствами программного инструментального комплекса 
<<Генератор проектов>>, описанного в~\cite{3-f}.
  %
  Применение к~модели весового паспорта ЛА определения <<цифровая>>~--- 
в~определенной степени дань моде всеобщей <<цифровизации>>. В~то же 
время таким названием хотелось подчеркнуть отличие этого вида технической 
документации от уже привычных <<электронных>> эксплуатационных 
документов, которые в~большинстве своем являются неструктурированными 
текстовыми или графическими файлами. Определение <<цифровой>> 
подчеркивает, что информация в~этом документе построена на числовых 
данных, которые можно не только прочитать, но и~использовать для 
проведения сложных вычислений на этапе подготовки полетных заданий и~во 
всех процессах технического обслуживания изделия.
 % 
  Кроме функций, связанных с~подготовкой полетных заданий, цифровой 
весовой паспорт может использоваться непосредственно в~полете. Он должен 
стать неотъемлемой частью бортового программного обеспечения ЛА. 
В~дальнейшем необходимо будет расширить его функции до сбора 
информации с~различных бортовых цифровых датчиков и~использования этих 
данных в~расчетах текущей полетной массы и~центровки. Создание 
дополнительного канала сбора и~анализа оперативной бортовой информации 
повысит надежность и~безопас\-ность управ\-ле\-ния~ЛА. 

\vspace{-6pt}
  
{\small\frenchspacing
 {%\baselineskip=10.8pt
 \addcontentsline{toc}{section}{References}
 \begin{thebibliography}{9}
\bibitem{1-f}
\Au{Шейнин В.\,М., Козловский~В.\,И.} Весовое проектирование и~эффективность 
пассажирских самолетов. ~--- М.: Машиностроение, 1977.  Т.~1. 343~с.
\bibitem{2-f}
\Au{Вышинский Л.\,Л., Флёров~Ю.\,А., Широков~Н.\,И.} Автоматизированная система 
весового проектирования самолетов~// Информатика и~её применения, 2018. Т.~12. Вып.~1. 
С.~18--30. doi: 10.14357/19922264180103.
\bibitem{3-f}
\Au{Вышинский Л.\,Л., Гринев~И.\,Л., Флеров~Ю.\,А., Широков~А.\,Н., Широков~Н.\,И.}  
Генератор проектов~--- инструментальный комплекс для разработки  
<<кли\-ент-сер\-вер\-ных>> сис\-тем~// Информационные технологии и~вычислительные 
системы, 2003. №\,1-2. С.~6--25.
 \end{thebibliography}

 }
 }

\end{multicols}

\vspace*{-6pt}

\hfill{\small\textit{Поступила в~редакцию 18.03.19}}

\vspace*{6pt}

%\pagebreak

%\newpage

%\vspace*{-28pt}

\hrule

\vspace*{2pt}

\hrule

\vspace*{-6pt}

\def\tit{DIGITAL MODEL OF~THE~AIRCRAFT'S WEIGHT PASSPORT\\[-7pt]}


\def\titkol{Digital model of~the~aircraft's weight passport}

\def\aut{L.\,L.~Vyshinsky$^1$, M.\,K.~Kuryansky$^2$, and~Yu.\,A.~Flerov$^1$\\[-7pt]}

\def\autkol{L.\,L.~Vyshinsky, M.\,K.~Kuryansky, and~Yu.\,A.~Flerov}

\titel{\tit}{\aut}{\autkol}{\titkol}

\vspace*{-18pt}


\noindent
  $^1$A.\,A.~Dorodnicyn Computing Center, Federal Research Center ``Computer Science and 
Control'' of the Russian\linebreak
$\hphantom{^1}$Academy of Sciences, 40~Vavilov Str., Moscow 119333, Russian 
Federation
  
  \noindent
  $^2$Department of Advanced Research-Scientific and Technical Center, United 
Aircraft Corporation, 5B~Pioner-\linebreak
$\hphantom{^1}$skaya Str., Moscow 115054, Russian Federation

\def\leftfootline{\small{\textbf{\thepage}
\hfill INFORMATIKA I EE PRIMENENIYA~--- INFORMATICS AND
APPLICATIONS\ \ \ 2019\ \ \ volume~13\ \ \ issue\ 4}
}%
 \def\rightfootline{\small{INFORMATIKA I EE PRIMENENIYA~---
INFORMATICS AND APPLICATIONS\ \ \ 2019\ \ \ volume~13\ \ \ issue\ 4
\hfill \textbf{\thepage}}}

\vspace*{3pt}   
  
  
    
  
\Abste{The paper is devoted to the problem of the digital modeling of an aircraft's 
weight passport. A~weight passport is developed at the stage of a~new product's 
design and accompanies it during all other stages of the life cycle. A~digital weight 
passport plays the most important role when the released product is being operated. 
A~software implementation of the weight passport serves not only as the reference 
manual, but also as the tool for carrying out complex weight calculations during the 
preparation of flight tasks, maintenance, and repair work. The paper proposes the 
concept and the software implementation of an aircraft's digital weight passport.}

\KWE{digital model; design automation; aircraft; weight design; weighting model; 
design tree; project generator}

\DOI{10.14357/19922264190401} 

%\vspace*{-14pt}

%\Ack
%\noindent



%\vspace*{-6pt}

  \begin{multicols}{2}

\renewcommand{\bibname}{\protect\rmfamily References}
%\renewcommand{\bibname}{\large\protect\rm References}

{\small\frenchspacing
 {%\baselineskip=10.8pt
 \addcontentsline{toc}{section}{References}
 \begin{thebibliography}{9}
\bibitem{1-f-1}
Sheynin, V.\,M., and V.\,I. Kozlovskiy, eds. 1977. \textit{Vesovoe proektirovanie 
i~effektivnost' passazhirskikh samoletov} [Weight design and performance of 
passenger aircraft]. Moscow: Mashinostroenie. Vol.~1. 343~p.
\bibitem{2-f-1}
\Aue{Vyshinskiy, L.\,L., Yu.\,A.~Flerov, and N.\,I.~Shirokov.} 2018. 
Avtomatizirovanaya systema vesovogo proektirovaniya samoletov [Automated 
system of weight design of aircraft]. \textit{Informatika i~ee Primeneniya~--- Inform 
Appl.} 12(1):18--30. doi: 10.14357/19922264180103.
\bibitem{3-f-1}
\Aue{Vyshinskiy, L.\,L., I.\,L.~Grinev, Yu.\,A.~Flerov, A.\,N.~Shirokov, and 
N.\,I.~Shirokov.} 2003. Generator proektov~--- instrumental'nyy kompleks dlya 
razrabotki ``klient-servernykh'' sistem [Project generator~--- a~tool complex for the 
development of ``client--server'' systems]. \textit{Informatsionnyye tekhnologii 
i~vychislitel'nye sistemy} [Information Technologies and Computing Systems] 
1-2:6--25.
\end{thebibliography}

 }
 }

\end{multicols}

\vspace*{-7pt}

\hfill{\small\textit{Received March 18, 2019}}

%\pagebreak

\vspace*{-24pt}

\Contr

\vspace*{-2pt}

\noindent
\textbf{Vyshinsky Leonid L.} (b.\ 1941)~--- Candidate of Science (PhD) in 
physics and mathematics, leading scientist, A.\,A.~Dorodnicyn Computing 
Center, Federal Research Center ``Computer Science and Control'' of the Russian 
Academy of Sciences, 40~Vavilov Str., Moscow 119333, Russian Federation; 
\mbox{wysh@ccas.ru}

%\vspace*{3pt}

\noindent
  \textbf{Kuryansky Mikhael K.}  (b.\ 1955)~--- Deputy Director, Department of 
Advanced Research-Scientific and Technical Center, United Aircraft Corporation, 
5B~Pionerskaya Str., Moscow 115054, Russian Federation; 
\mbox{m.kuryanskiy@uacrussia.ru}

%\vspace*{3pt}

\noindent
\textbf{Flerov Yuri A.} (b.\ 1942)~--- Corresponding Member of the Russian 
Academy of Sciences, Doctor of Science in physics and mathematics, professor, 
principal scientist, A.\,A.~Dorodnicyn Computing Center, Federal Research Center 
``Computer Science and Control'' of the Russian Academy of Sciences, 
40~Vavilov Str., Moscow 119333, Russian Federation; \mbox{fler@ccas.ru}

\label{end\stat}

\renewcommand{\bibname}{\protect\rm Литература}  