\def\stat{zatsman}

\def\tit{КОДИРОВАНИЕ КОНЦЕПТОВ В~ЦИФРОВОЙ СРЕДЕ}

\def\titkol{Кодирование концептов в~цифровой среде}

\def\aut{И.\,М.~Зацман$^1$}

\def\autkol{И.\,М.~Зацман}

\titel{\tit}{\aut}{\autkol}{\titkol}


\index{Зацман И.\,М.}
\index{Zatsman I.\,M.}




%{\renewcommand{\thefootnote}{\fnsymbol{footnote}} \footnotetext[1]
%{Работа выполнена в~Институте проблем информатики ФИЦ ИУ РАН при поддержке РФФИ 
%(проект 18-07-00192).}}


\renewcommand{\thefootnote}{\arabic{footnote}}
\footnotetext[1]{Институт проблем информатики Федерального исследовательского центра <<Информатика 
и~управление>> Российской академии наук, \mbox{izatsman@yandex.ru}}

\vspace*{6pt}



\Abst{Задачи кодирования концептов знания человека в~цифровой среде компьютеров 
и~сетей приобретают особую актуальность в~связи с~широким распространением в~мире 
систем искусственного интеллекта (СИИ). В~процессе расширения сферы их применения 
увеличивается и~спектр категорий кодируемых концептов. Кроме конвенциональных 
концептов, которые имеют устойчивые формы их представления, например словами 
естественных языков (ЕЯ), нередко возникает необходимость кодировать в~цифровой среде 
личностные и~коллективные концепты. При этом иногда необходимо учитывать степень их 
социализации (термин Веж\-биц\-ко\-го--На\-ка\-мо\-ри) и~отражать динамику их изменения 
во времени, а~также этапы их преобразования в~конвенциональные концепты. Во временном 
измерении расширился спектр шкал для описания динамики концептов знания человека. 
Если раньше использовались шкалы с~единицами измерения динамики концептов в~сотни 
и~десятки лет (реже использовались шкалы с~точностью до года и~месяца), то для личностных и~коллективных концептов необходимо использовать шкалу, фиксирующую их динамику 
с~точностью до дней, а~иногда часов и~минут. Цель статьи состоит в~описании проблемы 
асимметрии кодирования в~цифровой среде концептов, которая существенно усложняет 
процессы представления знания человека в~ССИ. Для 
решения этой проблемы предлагается использовать кодирование в~цифровой среде 
одновременно и~концептов перечисленных категорий, и~форм их выражения. Предлагаемый 
подход иллюстрируется примером интеллектуальной словарной системы (ИСС), использующей 
кодирование одновременно и~концептов, и~слов, являющихся вербальными формами 
выражения этих концептов.}

   \KW{кодирование знания; полиадический компьютинг; цифровая среда; искусственный 
интеллект; категории концептов; социализация концептов знания}

\DOI{10.14357/19922264190416} 
  
\vspace*{6pt}


\vskip 10pt plus 9pt minus 6pt

\thispagestyle{headings}

\begin{multicols}{2}

\label{st\stat}
   
\section{Введение}

  Раздел <<Проблематика представления знаний в~7-й Рамочной программе>> 
в~работе~\cite{1-zac} содержит обзор трудов семинара приглашенных 
экспертов <<Knowledge Anywhere Anytime: ``The Social Life of Knowledge''>>, 
который состоялся 29--30~апреля 2004~г.\ в~Брюсселе. Материалы этого 
семинара включают описание перспективных направлений\linebreak исследований по 
проблематике генерации и~представления концептов знания. Это описание 
и~сейчас актуально. В~материалах отмечается, что исследование процессов 
генерации нового знания, \mbox{методов} и~средств влияния на эти процессы является 
актуальной проблемой, которая до сих пор остается во многом нерешенной. 
Участники семинара определили четыре направления исследований в~рамках 
этой проблемы.
  \begin{enumerate}[1.]
  \item  Формирование научного понимания того, как\linebreak знание появляется, каким 
образом на этот процесс и~его результаты влияет совместная деятельность 
участников процесса, как формируется конвенциональное знание, которое 
\mbox{имеет} устойчивые формы представления, например словами 
ЕЯ. Одна из задач этого направления заключается в~том, чтобы создать 
методы и~средства описания различий в~личностном понимании участниками 
совместной деятельности смысла одного и~того же текста, графика, диаграммы, 
карты, изображения и~т.\,д.
  \item  Исследование многообразия форм представления одних и~тех же 
концептов как <<квантов>> ментального знания человека. Кроме форм 
представления конвенциональных кон\-цеп\-тов пред\-ме\-том исследования 
являются формы представления личностных и~коллективных концептов. 
В~рамках этого направления предполагается исследование процессов и~этапов 
формирования конвенциональных концептов на основе личностных 
и~коллективных.



  \item  Создание нового поколения СИИ, 
которые должны отражать эволюцию личностных и~коллективных концептов 
во времени, этапы формирования на их основе конвенционального знания, 
а~также обеспечивать семантическую интероперабельность в~процессе 
совместной работы пользователей этих систем. В~рамках этого направления 
исследуются стадии генерации и~эволюции знания, представленного в~виде 
классификационных систем, словарей, тезаурусов и~других видов онтологий 
с~разной степенью формализации методов представления знания в~СИИ.
  \item  Исследование принципиальных возможностей\linebreak
   и~средств влияния на 
генерацию новых или эволюцию существующих систем знания в~процессе 
совместной деятельности коллективов специалистов. Наиболее актуальные 
вопросы \mbox{этого} направления исследований связаны с~пространственно 
распределенными коллективами специалистов, совместная деятельность 
которых по генерации нового знания обеспечивается сетевыми технологиями.
  \end{enumerate}
  
  Статья по своей тематике относится к~третьему из четырех перечисленных 
направлений. Ее цель состоит в~применении интерфейсов третьего порядка, 
описанных в~работе~\cite{2-zac}, для разработки подхода к~кодированию 
в~цифровой среде СИИ личностных и~коллективных концептов знания 
человека с~отражением их эволюции во времени. Этот подход может быть 
также использован как теоретическая основа решения проблемы асимметрии, 
возникающей при кодировании в~СИИ концептов знания и~форм их 
представления в~разных модальностях (вербальной, пространственной,  
ло\-ги\-ко-ма\-те\-ма\-ти\-че\-ской, вер\-баль\-но-образ\-ной и~т.\,д.). 
Предлагаемый подход иллюстрируется примером ИСС, использующей кодирование одновременно и~концептов, и~слов, 
являющихся вербальными формами выражения этих концептов, на двух языках 
с~учетом как степени социализации знания пользователей сис\-те\-мы (термин 
Веж\-биц\-ко\-го--На\-ка\-мо\-ри~\cite{3-zac, 4-zac}), так и~отражения его 
эволюции на временн$\acute{\mbox{о}}$й шкале.

\begin{figure*} %fig1
  \vspace*{1pt}
    \begin{center}  
  \mbox{%
 \epsfxsize=89.14mm 
 \epsfbox{zac-1.eps}
 }
\end{center}
\vspace*{-9pt}
\Caption{Четыре среды разной природы и~границы между ними}
\end{figure*}

\section{Интерфейсы третьего порядка}

  В работе~\cite{2-zac} было дано описание нового понятия <<интерфейсы  
3-го порядка>> на основе деления на среды разной природы предметной 
области информатики как полиадического компьютинга~\cite{5-zac} 
и~категоризации границ между средами по числу смежных сред. Были 
рассмотрены пять сред: нейросреда (сигналы активности мозга 
в~нейрокоммуникационных технологиях), ментальная (кон\-цеп\-ты в~технологиях 
пред\-став\-ле\-ния знания человека), со\-цио-ин\-фор\-ма\-ци\-он\-ная (тексты на 
ЕЯ, таб\-ли\-цы, рисунки, раст\-ро\-вые и~векторные изоб\-ра\-же\-ния 
и~т.\,д.\ в~технологиях их компьютерного кодирования), ДНК-сре\-да 
(представление информации и~данных с~по\-мощью синтезированных цепочек 
ДНК) и~циф\-ро\-вая среда (компьютерные коды). При этом было отмечено, что 
тео\-ре\-ти\-че\-ское описание интерфейсов в~технологиях будущих поколений может 
по\-тре\-бо\-вать включения в~рас\-смот\-ре\-ние и~сред другой природы, отличной от 
пяти перечисленных.
  
  Основное отличие выделения сред разной природы в~предметной области 
информатики от деления всей науки на отрасли знания~\cite{6-zac} состоит 
в~том, что объекты одной и~той же среды могут изучаться одновременно 
в~разных научных дисциплинах. Например, концепты встречаются в~задачах 
представления знаний в~информатике~\cite{1-zac, 7-zac, 8-zac}, они являются 
объектами исследований в~науках о жизни~\cite{9-zac}  
и~со\-цио-гу\-ма\-ни\-тар\-ных науках~\cite{10-zac}, но все они относятся 
только к~одной ментальной среде знания человека.
  
  Для описания нового понятия <<интерфейсы \mbox{3-го} порядка>> использовались 
четыре из пяти сред, кроме ДНК-сре\-ды (рис.~1), так как этих четырех сред 
достаточно для их описания. На рис.~1 на границах между двумя смежными 
средами кружками условно обозначены интерфейсы 2-го порядка шес\-ти разных 
видов (они пронумерованы от~1 до~6), а~на границах между тремя смежными 
средами обозначены интерфейсы 3-го порядка четырех видов, которые по 
определению обеспечивают связи между объектами трех сред разной природы 
(7--10). В~этой статье интерфейсы 4-го порядка на границе четырех сред не 
рассматриваются.


  Если направление информационных трансформаций на границах сред не 
учитывать, то для четырех сред число видов интерфейсов 2-го порядка равно 
шести. Все они были описаны в~работах~\cite{2-zac, 7-zac}. Число видов 
интерфейсов 3-го порядка равно четырем. Интерфейсы вида №\,7 связывают 
объекты нейросреды, ментальной и~цифровой сред, №\,8~--- ментальной,  
со\-цио-ин\-фор\-ма\-ци\-он\-ной и~цифровой сред, №\,9~--- нейросреды,  
ментальной и~со\-цио-ин\-фор\-ма\-ци\-он\-ной сред, №\,10~--- нейросреды,  
со\-цио-ин\-фор\-ма\-ци\-он\-ной и~цифровой сред. Отметим, что 
в~работе~\cite{2-zac} на рис.~2 интерфейсы третьего порядка видов №\,9 
и~№\,10 не показаны.
  
  Интерфейсы вида №\,7 были описаны в~работе~\cite{2-zac} на примере 
процессов управления роботизированной рукой сигналами головного мозга 
с~использованием интерфейсов <<мозг--ком\-пью\-тер>>~\cite{11-zac, 12-zac}, 
иллюстрирующих возможности интерфейсов 3-го порядка. Данная статья 
посвящена интерфейсам вида №\,8, на основе которого и~предлагается новый 
подход к~кодированию концептов знания человека в~цифровой среде 
с~отражением их эволюции во времени. Интерфейсы 3-го порядка, 
относящиеся к~видам №\,9 и~№\,10, в~статье не рассматриваются.
  
\section{Традиционное кодирование концептов}

  Из шести видов интерфейсов 2-го порядка наиболее известными и~широко используемыми в~информационных технологиях, системах и~средствах информатики являются 
следующие: интерфейсы вида №\,2 на границе ментальной  
и~со\-цио-ин\-фор\-ма\-ци\-он\-ной сред, описывающие взаимосвязи между 
концептами и~формами их выражения, например словами 
ЕЯ~\cite{10-zac}, и~интерфейсы вида №\,3 на границе  
со\-цио-ин\-фор\-ма\-ци\-он\-ной и~цифровой сред, включающие компьютерное 
кодирование растровых или векторных карт и~других изображений, а~также 
символов текстовой информации, например, с~помощью таблиц Unicode 
(см.\ рис.~1).
  
  При разработке СИИ одной из наиболее трудных является проб\-ле\-ма 
представления в~цифровой среде лингвистического  
знания~\cite{13-zac}\footnote{См.\ табл.~1-1 в~\cite{13-zac}, сопоставляющую 
разные виды интеллекта человека согласно подходу к~его категоризации, 
который предложил Говард Гарднер, на лингвистический, пространственный, 
логико-математический и~другие виды~\cite{14-zac}, а также описывающую 
степень сложности компьютерного моделирования каждого из видов 
интеллекта человека.}. Одна из основных причин слож\-ности 
решения этой проб\-ле\-мы заключается в~свойстве асимметрии, которое присуще 
интерфейсам вида №\,2. С~одной стороны, при реализации этих интерфейсов 
необходимо учитывать, что концепт может иметь несколько форм его 
представления (синонимия). С~другой стороны, одна и~та же словоформа 
может выражать разные концепты (омонимия). Свойство асимметрии 
рассмотрим на примере концептов системы лингвистического знания  
о~ло\-ги\-ко-се\-ман\-ти\-че\-ских отношениях (ЛСО) в~текстах, таких как ЛСО 
причины\footnote{В предложениях <<Самая,~--- говорит,~--- дорогая вещь на 
земле~--- это глупость. Потому как за нее всего дороже приходится 
платить$\ldots$>> (А.~Вайнер, Г.~Вайнер. Эра милосердия) ЛСО причины 
выражено с~помощью слов <<Потому как>>.}, альтернативы, несоответствия, 
следствия, условия, генерализации, спецификации  
и~т.\,д.~\cite{15-zac, 16-zac, 17-zac}. Рассмотрим концепты ЛСО в~текстах на 
двух ЕЯ: русском и~французском (рис.~2). На этом 
рисунке показаны две среды, поскольку интерфейсы этого вида описываются 
с~по\-мощью лингвистических знаков как отношения между концептами, 
принадлежащими ментальной среде, и~словоформами (что будем называть 
вербальной модальностью пред\-став\-ле\-ния концептов)  
в~со\-цио-ин\-фор\-ма\-ци\-он\-ной среде. Отметим, что асимметрия 
интерфейсов вида №\,2 встречается и~в тех случаях, когда некоторая сис\-те\-ма 
знания пред\-став\-ле\-на в~формах невербальных модальностей, например 
в~геоязыковой форме~\cite{18-zac}.
  
  Лингвистические знаки, каждый из которых является двухприродной 
сущностью, принадлежат к~границе между ментальной  
и~со\-цио-ин\-фор\-ма\-ци\-он\-ной средами (см. штриховую линию на рис.~2).
  
\begin{figure*} %fig2
\vspace*{1pt}
    \begin{center}  
  \mbox{%
 \epsfxsize=157mm 
 \epsfbox{zac-2.eps}
 }
\end{center}
\vspace*{-9pt}
\Caption{Интерфейсы вида №\,2}
\end{figure*}

  Они условно обозначены на этой границе двумя кружками: один с~двумя 
русскими словами <<а~то>> и~один с~французским словом <<car>>. Они 
выражают ЛСО причины в~текстах на русском и~французском языках 
соответственно. Лингвистические знаки являются наиболее известными 
примерами сущностей, имеющих двойственную природу, так как интегрируют в~себе концепты и~словоформы. Эти знаки и~обеспечивают интерфейсы второго 
порядка в~традиционных технологиях вербального пред\-став\-ле\-ния знания.
  
  В рассматриваемом примере каждому концепту соответствует значение 
только одного ЛСО. Более сложные случаи сочетания двух ЛСО~\cite{19-zac} 
здесь не рассматриваются. Важно отметить, что существуют несколько 
вариантов членения знания об~ЛСО на  
концепты~\cite{20-zac, 21-zac, 22-zac, 23-zac}. На рис.~2 условно обозначены 
только два разных варианта: один для русского языка и~один для французского. 
При этом предполагается, что в~обоих вариантах присутствует концепт ЛСО 
причины, на примере которого проиллюстрируем асимметрию интерфейсов  
2-го порядка вида №\,2. Как будет показано далее, асимметрия в~технологиях 
представления знания может быть снята с~помощью интерфейсов 3-го порядка 
вида~№\,8.
  
  Один и~тот же концепт ЛСО причины в~русском языке может быть выражен 
несколькими коннекторами\footnote{Коннекторами называются лексические 
единицы, основная функция которых состоит в~выражении одного или 
нескольких ЛСО между фрагментами текста~\cite{15-zac, 16-zac}.} 
(\textit{а~то}, \textit{ибо}, \textit{потому что}, \textit{поскольку}, \textit{так 
как} и~т.\,д.). Во французском языке этот концепт также может быть выражен 
несколькими коннекторами (\textit{car}, \textit{parce que}, \textit{puisque}, 
\textit{comme} и~т.\,д.). Иногда один и~тот же коннектор в~разных 
предложениях может выражать разные ЛСО. Например, коннектор 
\textit{а~то} может выражать несколько разных ЛСО, включая три, 
приведенных слева на рис.~2:
  \begin{enumerate}[(1)]
\item \textit{Причина}: <<Ты щетку смочи водой, а то пылишь здорово>> 
(М.\,А.~Булгаков. Белая гвардия).
\item \textit{Альтернатива}: <<Вот отлично! Непременно съезжу к~ним,~--- 
сказал Левин.~--- А~то поедем вместе. Она такая славная. Не правда ли?>> 
(Л.\,Н.~Толстой. Анна Каренина).
\item \textit{Несоответствие}: <<Он женится! Хочешь об заклад, что не 
женится?~--- возразил он.~--- Да ему Захар и~спать-то помогает, а то 
жениться!>> (И.\,А.~Гончаров. Обломов).
  \end{enumerate}
  
  Таким образом, интерфейсы 2-го порядка вида №\,2 являются в~общем случае 
асимметричными и~не обеспечивают однозначного представления концептов 
с~помощью словоформ, что существенно усложняет задачи компьютерного 
кодирования концептов знания в~СИИ. Более того, иногда концепт ЛСО может 
быть имплицирован в~тексте. Например, в~предложении <<Не просите меня 
петь, я~не спою уже больше так$\ldots$>> (И.\,А.~Гончаров. Обломов) концепт 
ЛСО причины не выражен словом или устойчивым словосочетанием. В~этом 
предложении он может быть извлечен только при семантическом анализе 
соотношения содержания двух его частей.

  
  В приведенных выше примерах 1--3 коннектор <<а~то>> можно 
закодировать как последовательность из четырех символов, используя, 
например, шестнадцатеричные коды таблиц Юникода: <<а>>~(0430), <<\ 
>>~(0020), <<т>>~(0442), <<о>>~(043Е). Однако соответствующие 
компьютерные коды представляют только эти символы, а не концепты, для 
кодирования которых используются другие подходы.

  \begin{table*}\small
  \begin{center}
  \tabcolsep=2pt
  \begin{tabular}{|l|l|l|}
  \multicolumn{3}{c}{Семь значений слова \textit{face} по HTE~\cite{25-zac, 26-zac}}\\
  \multicolumn{3}{c}{\ }\\[-6pt]
  \hline
\multicolumn{1}{|c|}{Номер  концепта в~HTE} &
\multicolumn{1}{c|}{Концепт}                 &
\multicolumn{1}{c|}{Слово и~год  появления концепта}\\
\hline
01.02.03.08.01.04                             &
The body :: Face                               &
Matching word(s): face (1290$-$)\\
\hline
01.02.03.08.01.04$\vert$04 &
The body :: Face :: with reference to  beauty&
Matching word(s): face (1591$-$)\\
\hline
01.02.03.08.01.04.01                          &
The body :: Face with  expression/expression&
Matching word(s): face (1330$-$)\\
\hline
01.02.03.08.01.04.01$\vert$01                      &
\tabcolsep=0pt\begin{tabular}{l}The body :: Face with\\      
expression/expression :: grimace/distortion\end{tabular}&
Matching word(s): face (1602$-$)\\
\hline
01.12.05.03.01$\vert$13 &
\tabcolsep=0pt\begin{tabular}{l}Relative position :: Surface :: 
one of several surfaces\\ of a~thing\end{tabular}&
Matching word(s): face (1340$-$)\\
\hline
01.12.05.03.01$\vert$19 &
Relative position :: Surface :: front surface &
Matching word(s): face 
(1611\;+\;1820$-$)\\
\hline
01.12.05.03.01$\vert$19.01 &
\tabcolsep=0pt\begin{tabular}{l}Relative position :: Surface :: front surface  :: 
specifically\\ of a coin/medal/seal/die, etc.\end{tabular} &
Matching word(s): face (1515$-$)\\
\hline
   \multicolumn{3}{p{162mm}}{\footnotesize \hspace*{2mm}\textbf{Примечание.} Если в~третьем столбце 
после <<$-$>> не указан год, то это говорит о том, что это значение встречается в~текстах до настоящего 
времени.}
   \end{tabular}
   \end{center}
   \vspace*{-9pt}
   \end{table*}
  
  Рассмотрим один из подходов к~кодированию концептов, выражаемых 
словами английского языка~\cite{24-zac}. Авторы этого подхода исходили из 
того, что в~ЕЯ используется конечное чис\-ло слов, 
обозначающих концепты, и~они должны выражать значительно большее чис\-ло 
концептов, чем чис\-ло слов в~языке. Отметим, что это является одной из причин 
асимметрии интерфейсов 2-го порядка вида №\,2 при вербальном 
пред\-став\-ле\-нии концептов.
  
  Авторы этого подхода поставили задачу разработать ретроспективную 
модель, описывающую процесс появления в~прошлом концептов, исследуя 
эволюцию английского языка на протяжении нескольких  
столетий~\cite{24-zac}. В~этой модели есть ось времени, на которой указаны 
моменты появления новых значений у исследуемых слов. Эти моменты 
времени определялись по Историческому тезаурусу английского языка (The 
Historical Thesaurus of English~--- HTE), в~котором содержится информация 
о~годах появления новых концептов слов, начиная с~1000~года~\cite{25-zac}.
  
  Ретроспективная модель описывает связи между концептами  
(=\;зна\-че\-ни\-ями) исследуемых слов, а также степень их семантической 
близости, которая определяется на основе иерархического номера, 
присвоенного каждому концепту в~HTE. В~качестве примера приведем семь 
концептов слова \textit{face} по HTE~\cite{26-zac} c указанием их номеров 
и~года появления в~текстах Британского национального корпуса (British National 
Corpus~--- BNC)~\cite{27-zac} (см.\ таблицу). Отметим, что для всех семи концептов 
не указаны годы их исчезновения из текстов, т.\,е.\ они встречаются 
и~в~современных текстах BNC.
  

   
  Если в~этом примере кодировать в~цифровой среде не последовательность из 
четырех символов (face), а сочетание двух параметров (номер концепта в~HTE, 
словоформа <<face>>), то это будет означать применение интерфейсов 3-го 
порядка вида №\,8 в~задачах кодирования. Их использование и~позволяет снять 
асимметрию, свойственную интерфейсам 2-го порядка вида №\,2 (см.\ рис.~2).
  
  В заключение этого раздела приведем пример регистрации 
конвенциональных концептов рубрик Международной патентной 
классификации (МПК) с~указанием их номеров, а~также года и~\textit{месяца}\linebreak 
введения каждой рубрики. Начиная с~января 2007~г.\linebreak
 вновь вводимые или 
изменяемые рубрики рас\-ширенного уровня МПК подготавливаются 
Спе\-циальным подкомитетом Всемирной организации интеллектуальной 
собственности (ВОИС) по пере\-смот\-ру расширенного уровня. Рубрики 
расширенного уровня на этапе рассмотрения этим подкомитетом 
и~согласования их смыслового содержания выражают личностные и/или 
коллективные концепты участников обсуждения. После принятия 
согласованного решения и~его утверждения они приобретают 
конвенциональный характер в~пределах патентной сферы как 
институциональной сис\-те\-мы~\cite{1-zac}.
  
\section{Предлагаемый подход}

  Сформулируем подход к~кодированию концептов знания человека 
в~цифровой среде СИИ, который использует понятие интерфейсов 3-го 
порядка и~обобщает принцип трехкомпонентной кодировки концептов 
геоизображений~\cite{28-zac, 29-zac}. Предполагается, что СИИ включает 
онтологию, которая позволяет описывать концепты и~разнообразие форм их 
представления в~предметной области СИИ.
  
  Основная идея этого подхода, которую опишем на примере вербальной 
модальности форм пред\-став\-ле\-ния знания, состоит в~следующем: для каж\-до\-го 
концепта (ментальная среда) и~$N$ вербаль\-ных\linebreak форм его представления 
(со\-цио-ин\-фор\-ма\-ци\-он\-ная среда) на одном ЕЯ, зарегистрированных в~СИИ 
(указывается дата и~время регистрации), образуются последовательности, 
вклю\-ча\-ющие четыре па\-ра\-мет\-ра: code.date.time.formV1, code.date.time.formV2, 
$\ldots$ ,  code.date.time.formVN, где code~--- номер этого концепта в~тезаурусе 
или словаре СИИ, а~бук\-ва~<<V>> перед~1, 2 и~$N$ обозначает вербальную 
модальность~$N$~форм пред\-став\-ле\-ния этого концепта. Для каждой такой 
по\-сле\-до\-ва\-тель\-ности па\-ра\-мет\-ров предлагается назначать в~СИИ уникальный 
компьютерный код (циф\-ро\-вая среда). В~случае необходимости могут 
добавляться и~другие па\-ра\-мет\-ры, например формы пред\-став\-ле\-ния одного 
и~того же концепта на двух и~большем чис\-ле ЕЯ или в~нескольких 
модальностях, например одновременно в~вербальной и~геоязыковой.
  
  Сама идея одновременного кодирования концептов, их форм представления 
и~временн$\acute{\mbox{ы}}$х параметров не нова. Например, в~неявной 
форме она используется в~проекте анализа процесса возникновения новых 
концептов слов английского языка (вместо даты и~времени указывается только 
год)~\cite{24-zac}, а~также применяется ВОИС с~января 2007~г.\ (указываются 
год и~месяц)~\cite{1-zac}.
  
  Отличительная черта предлагаемого подхода состоит в~добавлении 
параметра степени социализации знания, что дает возможность кодировать 
в~циф\-ро\-вой среде личностные, коллективные и~конвенциональные концепты 
знания человека, а~также отражать этапы их эволюции с~помощью сле\-ду\-ющих 
параметров:
  \begin{itemize}
  \item  код концепта знания человека, принадлежащий ментальной среде;
  \item формы представления этого концепта с~по\-мощью ЕЯ или одновременно в~нескольких 
  модальностях, принадлежащие со\-цио-ин\-фор\-ма\-ци\-он\-ной 
среде;
  \item степень социализации концептов знания (в~случае необходимости 
можно измерять и~степень социализации форм представления концептов);
  \item тезаурусные связи концепта, отражаемые в~СИИ косвенно через его 
код;
  \item момент времени фиксирования в~СИИ дефиниции концепта, его форм 
представления, связей и~степени социализации с~точностью до часов и~минут.
  \end{itemize}
  
  Новизна предлагаемого подхода состоит в~использовании степени 
социализации знания, описанной в~работах~\cite{3-zac, 4-zac}. Вариант 
кодирования степени социализации, детализирующий понятие\linebreak коллективного 
концепта, описан в~работе~\cite{30-zac}. Согласно этому варианту, каждому 
концепту присваивается идентификатор, принимающий значения ноль, единица 
(личностные концепты), целое положительное число $N\hm>1$ (коллективные 
концепты, где $N$~--- число членов коллектива, согласовавших дефиницию 
концепта и~формы его представления) или символ бесконечности~<<$\infty$>> 
(конвенциональные концепты). Нулевое значение идентификатора необходимо 
для обозначения тех <<бывших>> личностных концептов, от которых 
в~некоторый момент времени отказались их авторы. Описанную 
в~работе~\cite{30-zac} шкалу социализации и~предлагается использовать для 
кодирования степени согласованности концептов. Применение этой шкалы дает 
возможность добавить в~перечисленные выше последовательности еще один 
параметр, который обозначим как id: code.id.date.time.form, где id может 
быть равен~0, 1, $N\hm>1$ или~$\infty$.
  
\section{Макет интеллектуальной словарной системы}

  Предлагаемый подход к~кодированию знания на основе интерфейсов 3-го 
порядка вида №\,8, включающий шкалу социализации, был востребован при 
разработке концепции создания макета ИСС\footnote{В~разработке 
и~обсуждении концепции принимали участие  А.\,А.~Гончаров, 
Д.\,О.~Добровольский, А.\,А.~Дурново,  Анна~А.~Зализняк, В.\,И.~Карпов, М.\,Г.~Кружков 
и~А.\,В.~Шарандин.}, включающей лексикографическую базу знаний 
(ЛБЗ)~\cite{31-zac}. Интеллектуальная словарная сис\-те\-ма  является видом информационной системы, 
обеспечивающей формирование и~регулярное обновление двуязычных 
электронных словарей. Лексикографическая база знаний предназначена для систематизации знаний 
о~словах. Именно ЛБЗ обеспечивают работу лексикографов по созданию 
словарных статей и~их обновление в~случае необходимости. Концепция 
создания макета включает перечень из пяти следующих задач, решение 
которых предполагает применение современных методов и~средств 
информатики.
  \begin{enumerate}[1.]
  \item  Систематизировать в~ЛБЗ знания о словах, включая личностные 
и~коллективные знания лексикографов, создающих словари, используя 
лексические единицы двух ЕЯ.
  \item Разработать персонализированный интерфейс пользователя двуязычных 
электронных словарей, сгенерированных с~помощью ЛБЗ. 
  \item Установить в~электронных словарях тезаурусные связи между 
значениями лексических единиц (отношения <<часть--це\-лое>>, родовидовые 
отношения, синонимы и~др.)\ с~возможностью навигации по этим связям.
  \item Реализовать в~двуязычных словарях поиск словарных статей по 
лексическим, грамматическим и~семантическим критериям поиска.
  \item Включить в~словарные статьи необходимые мультимедийные данные.
  \end{enumerate}
  
  Интеллектуальная словарная сис\-те\-ма предоставит расширенные функциональные возможности как 
лексикографам, так и~пользователям словарей. В~ИСС 
предполагается 
реализовать сле\-ду\-ющие фильт\-ры и~виды поиска:
  \begin{itemize}
  \item синтаксический фильтр, когда при поиске могут использоваться 
признаки синтаксических конструкций, в~которые входит искомое слово;
  \item статистический фильтр по частотности слов, что позволит отображать 
или, наоборот, исключать при отображении в~словаре низкочастотные слова 
(частотности слов являются корпусно-ориентированными), т.\,е.\ выше или 
ниже заданного порога;
  \item хронологический фильтр, который позволит использовать датировку 
текстов, т.\,е.\ в~словарной статье могут отображаться примеры переводов 
только в~тех текстах, которые относятся к~заданному пользователем периоду 
времени;
  \item поиск по лексикографически значимым категориям: семантические группы слов 
(времена года, страны света, части речи, дни недели и~др.), функциональные 
группы слов (например, коннекторы, выражающие  
ЛСО в~тексте~\cite{17-zac}), типы 
устойчивых словосочетаний и~т.\,д.
  \end{itemize}
  
\section{Заключение}

  Для демонстрации реализуемости пред\-ла\-га\-емого подхода запланировано 
выполнение серии проек\-тов. Задача первого из них состоит в~разработке 
и~экспериментальной апробации метода\linebreak извлечения лингвистического знания 
из параллельных текстов и~кодирование в~цифровой среде извлеченных, 
в~первую очередь личностных концептов лексикографа, а также вербальных 
форм пред\-став\-ле\-ния таких концептов с~помощью двух ЕЯ: немецкого 
и~русского. Результаты проведенных экспериментов позволяют говорить 
о~реализуемости предлагаемого подхода для форм вербальной модальности 
и~двух ЕЯ~\cite{32-zac, 33-zac, 34-zac, 35-zac}. При реализации этого проекта 
использовался предлагаемый подход к~кодированию концептов 
лингвистического знания, извлекаемых при анализе параллельных текстов, на 
основе интерфейсов 3-го порядка вида №\,8 (см.\ рис.~2). В~процессе их извлечения 
иногда возникала необходимость пополнения СИИ новыми, ранее не 
описанными концептами~\cite{32-zac, 34-zac}. Отметим, что нередко в~текстах 
встречались и~имплицированные концепты, не выраженные отдельными 
словами или устойчивыми словосочетаниями, извле\-ка\-емые только при 
семантическом анализе одного или нескольких предложений~\cite{15-zac}.
  
  В заключение отметим, что более сложные проб\-ле\-мы кодирования знания 
возникают в~тех случаях, когда СИИ должна обеспечивать пред\-став\-ле\-ние 
знания одновременно в~нескольких модальностях. Например, когда 
одновременно используются вербальная и~про\-стран\-ст\-вен\-ная модальности 
и~ставится задача поиска в~СИИ концептов, тогда мо\-даль\-ность их форм 
пред\-став\-ле\-ния при поиске может быть неизвестна~\cite{28-zac}. Для решения 
подобных проб\-лем необходимо разрабатывать мультимодальные онтологии, 
например вер\-баль\-но-образ\-ные геотезаурусы.
  
 {\small\frenchspacing
 {%\baselineskip=10.8pt
 \addcontentsline{toc}{section}{References}
 \begin{thebibliography}{99}
\bibitem{1-zac}
\Au{Зацман И.\,М., Косарик~В.\,В., Курчавова~О.\,А.} Задачи представления личностных 
и~коллективных концептов в~цифровой среде~// Информатика и~её применения, 2008. Т.~2. 
Вып.~3. С.~54--69.
\bibitem{2-zac}
\Au{Зацман И.\,М.} Интерфейсы третьего порядка в~информатике~// Информатика и~её 
применения, 2019. Т.~13. Вып.~3. С.~82--89.
\bibitem{3-zac}
\Au{Wierzbicki A.\,P., Nakamori~Y.} Basic dimensions of creative space~// Creative space: Models 
of creative processes for knowledge civilization age.~--- Berlin--Heidelberg: Springer Verlag, 2006. 
P.~59--90.
\bibitem{4-zac}
\Au{Wierzbicki A.\,P., Nakamori~Y.} Knowledge sciences: Some new developments~// Zeitschrift 
f$\ddot{\mbox{u}}$r Betriebswirtschaft, 2007. Vol.~77. Iss.~3. P.~271--295.
\bibitem{5-zac}
\Au{Rosenbloom P.} On computing: The fourth great scientific domain.~--- Cambridge: MIT Press, 
2013. 308~p.
\bibitem{6-zac}
\Au{Denning P., Rosenbloom~P.} Computing: The fourth great domain of science~// 
Commun. ACM, 2009. Vol.~52. Iss.~9. P.~27--29.

\bibitem{8-zac} %7
\Au{Zatsman I.} Tracing emerging meanings by computer: Semiotic framework~// 13th European 
Conference on Knowledge Management Proceedings.~--- Reading: Academic Publishing 
International Ltd., 2012. Vol.~2. P.~1298--1307.
\bibitem{7-zac} %8
\Au{Зацман И.\,М.} Таблица интерфейсов информатики как  
ин\-фор\-ма\-ци\-он\-но-компью\-тер\-ной науки~// На\-уч\-но-тех\-ни\-че\-ская информация. 
Сер.~1: Организация и~методика информационной работы, 2014. №\,11. С.~1--15.

\bibitem{9-zac}
\Au{Baars B., Gage~N.} Cognition, brain, and consciousness: Introduction to cognitive 
neuroscience.~--- Amsterdam: Academic Press/Elsevier, 2010. 677~p.
\bibitem{10-zac}
\Au{Eco U.} A~theory of semiotics.~--- Bloomington, IN, USA: Indiana University Press, 1976. 356~p.

\bibitem{12-zac} %11
\Au{Schalk G., Leuthardt~E.\,C.} Brain--computer interfaces using electrocorticographic signals~// 
IEEE Rev. Biomed. Eng., 2011. Vol.~4. Iss.~1. P.~140--154.
\bibitem{11-zac} %12
\Au{Sunny T.\,D., Aparna~T., Neethu~P., Venkateswaran~J., Vishnupriya~V., Vyas~P.\,S.} Robotic 
arm with brain--computer interfacing~// Proc. Tech., 2016. Vol.~24. P.~1089--1096.

\bibitem{13-zac}
\Au{Mueller J.\,P., Massaron~L.} Artificial intelligence for dummies.~--- Hoboken,
NJ, USA: John Wiley \& 
Sons, 2018. 316~p. 
\bibitem{14-zac}
\Au{Гарднер Г.} Структура разума: теория множественного интеллекта~/ Пер. с~англ.~--- М.: 
Вильямс, 2007. 512~с. (\Au{Gardner~H.} Frames of mind: The theory of multiple  
intelligences.~--- New York. NY, USA: Basic Books, 2004. 440~p.)
\bibitem{15-zac}
\Au{Гончаров А.\,А., Инькова~О.\,Ю.} Методика поиска имплицитных логико-семантических 
отношений в~тексте~// Информатика и~её применения, 2019. Т.~13. Вып.~3. С.~100--107.
\bibitem{16-zac}
\Au{Гончаров А.\,А., Инькова~О.\,Ю.} Cпособы выражения причинных отношений в~русском 
языке: опыт анализа с~использованием кросслингвистической надкорпусной базы данных~// 
Русская грамматика: активные процессы в~языке и~речи: Сб. научных трудов Междунар. 
научного симпозиума.~--- Ярославль: РИО ЯГПУ, 2019. С.~385--396.
\bibitem{17-zac}
\Au{Инькова О., Манзотти~Э.} Связность текста: мереологические  
ло\-ги\-ко-се\-ман\-ти\-че\-ские отношения.~--- М.: Языки славянских культур, 2019 
(в~печати).
\bibitem{18-zac}
\Au{Лютый А.\,А.} Язык карты: сущность, система, функции.~--- М.: ИГ РАН, 2002. 327~с.
\bibitem{19-zac}
\Au{Инькова О.\,Ю., Кружков~М.\,Г.} Сочетаемость логико-се\-ман\-ти\-че\-ских отношений: 
количественные методы анализа~// Информатика и~её применения, 2019. Т.~13. Вып.~2. 
С.~83--91.
\bibitem{20-zac}
\Au{Mann W.\,C., Thompson~S.\,A.} Rhetorical structure theory: Toward a functional theory of text 
organization~// Text,
%: Interdisciplinary J.~Study of Discourse, 
1988. Vol.~8. Iss.~3. P.~243--281.

\bibitem{23-zac} %21
\Au{Breindl E., Volodina~A., \mbox{Wa{\!\ptb{\!\ss}}\,ner}~U.\,H.} Handbuch der deutschen 
Konnektoren 2.~Semantik der deutschen Satzverkn$\ddot{\mbox{u}}$pfer.~--- Berlin: Walter de 
Gruyter, 2014. 1327~p.

\bibitem{21-zac} %22
\Au{Bunt H., Prasad~R.}
 ISO DR-Core (ISO 24617-8): Core concepts for the annotation of discourse relations~// 12th Joint 
ACL-ISO Workshop on Interoperable Semantic Annotation Proceedings.~--- Portoroz, 2016.  
P.~45--54.
\bibitem{22-zac} %23
\Au{Sanders T.\,J.\,M., Demberg~V., Hoek~J., Scholman~M.\,C.\,J., Torabi~A.\,F., Zufferey~S., 
Evers-Vermeul~J.} Unifying dimensions in coherence relations. How various annotation 
frameworks are related~// Corpus Linguist. Ling., 2018 (ahead of print).  
71~p. {\sf 
https://www.degruyter.com/ view/j/cllt.ahead-of-print/cllt-2016-0078/cllt-2016-0078.xml?rskey=Vr3MhX\&result=1}.

\bibitem{24-zac}
\Au{Ramiro C., Srinivasan~M., Malt~B.\,C., Xu~Y.} Algorithms in the historical emergence of 
word senses~// P.~Natl. Acad. Sci. USA, 2018. Vol.~115. Iss.~10. 
P.~2323--2328.
\bibitem{25-zac}
\Au{Kay C., Roberts~J., Samuels~M., Wotherspoon~I., Alexander~M.} The historical thesaurus of 
English. Version~4.2.~--- Glasgow, U.K.: University of Glasgow, 2015. {\sf 
https://historicalthesaurus.arts.gla.ac.uk}.
\bibitem{26-zac}
\Au{Kay C., Roberts~J., Samuels~M., Wotherspoon~I., Alexander~M.} The historical thesaurus of 
English: Face. {\sf http://historicalthesaurus.arts.gla.ac.uk/category-selection/?qsearch=face}.
\bibitem{27-zac}
The British National Corpus (Oxford University Computing Services). {\sf www.natcorp.ox.ac.uk}.
\bibitem{28-zac}
\Au{Зацман И.\,М.} Концептуальный поиск и~качество информации.~--- М.: Наука, 2003. 
272~с.
\bibitem{29-zac}
\Au{Зацман И.\,М.} Семиотическая модель взаимосвязей концептов, информационных 
объектов и~компьютерных кодов~// Информатика и~её применения, 2009. Т.~3. Вып.~2.  
С.~65--81.
\bibitem{30-zac}
\Au{Зацман И.\,М.} Процессы целенаправленной генерации и~развития кросс-язы\-ко\-вых 
экспертных знаний: семиотические основания моделирования~// Информатика и~её 
применения, 2015. Т.~9. Вып.~3. С.~106--123.
\bibitem{31-zac}
\Au{Вакуленко В.\,В., Зацман~И.\,М.} Принципы создания и~функции интеллектуальной 
словарной системы~// Системы и~средства информатики, 2019. Т.~29. №\,4. С.~96--105.

\bibitem{33-zac} %32
\Au{Зацман И.\,М.} Имплицированные знания: основания и~технологии извлечения~// 
Информатика и~её применения, 2018. Т.~12. Вып.~3. С.~74--82.

\bibitem{32-zac} %33
\Au{Зацман И.\,М.} Стадии целенаправленного извлечения знаний, имплицированных 
в~параллельных текстах~// Системы и~средства информатики, 2018. Т.~28. №\,3.  
С.~175--188.

\bibitem{34-zac}
\Au{Зацман И.\,М.} Целенаправленное развитие систем лингвистических знаний: выявление и~заполнение лакун~// Информатика и~её применения, 2019. Т.~13. Вып.~1. С.~91--98.
\bibitem{35-zac}
\Au{Гончаров А.\,А., Зацман~И.\,М.} Информационные трансформации параллельных текстов 
в~задачах извлечения знаний~// Системы и~средства информатики, 2019. Т.~29. №\,1.  
С.~180--193.
 \end{thebibliography}

 }
 }

\end{multicols}

\vspace*{-6pt}

\hfill{\small\textit{Поступила в~редакцию 01.10.19}}

%\vspace*{8pt}

%\pagebreak

\newpage

\vspace*{-28pt}

%\hrule

%\vspace*{2pt}

%\hrule

%\vspace*{-2pt}

\def\tit{DIGITAL ENCODING OF~CONCEPTS}


\def\titkol{Digital encoding of~concepts}

\def\aut{I.\,M.~Zatsman}

\def\autkol{I.\,M.~Zatsman}

\titel{\tit}{\aut}{\autkol}{\titkol}

\vspace*{-11pt}


 \noindent
   Institute of Informatics Problems, Federal Research Center ``Computer Sciences and 
Control'' of the Russian Academy of Sciences; 44-2~Vavilov Str., Moscow 119133, 
Russian Federation

\def\leftfootline{\small{\textbf{\thepage}
\hfill INFORMATIKA I EE PRIMENENIYA~--- INFORMATICS AND
APPLICATIONS\ \ \ 2019\ \ \ volume~13\ \ \ issue\ 4}
}%
 \def\rightfootline{\small{INFORMATIKA I EE PRIMENENIYA~---
INFORMATICS AND APPLICATIONS\ \ \ 2019\ \ \ volume~13\ \ \ issue\ 4
\hfill \textbf{\thepage}}}

\vspace*{3pt}  

    
    \Abste{The tasks of encoding concepts of human knowledge in the digital 
medium of computers and networks are of particular relevance in connection with the 
widespread use of artificial intelligence systems in the world. In the process of 
expanding the scope of their applications, the range of categories of encoded concepts 
is increasing. In addition to conventional concepts, which have stable forms of their 
presentation, for example, by the words of natural languages, it is often necessary to 
encode personal and collective concepts in the digital medium. Moreover, sometimes, 
it is necessary to take into account the degree of their socialization (the  
Wierzbicki\&Nakamori's term) and reflect the dynamics of their change over time, as 
well as the stages of their transformation into conventional concepts. In the time 
dimension, the spectrum of scales has expanded for describing the dynamics of 
concepts of human knowledge. If earlier scales were used with units of measuring the 
dynamics of concepts in hundreds and tens of years (less often scales with accuracy 
up to a year and a month were used), then for personal and collective concepts, it is 
necessary to use a scale that fixes their dynamics up to days, and sometimes hours 
and minutes. The goal of the paper is to describe the asymmetry problem encountered 
in the encoding process of concepts in the digital medium. The asymmetry 
significantly complicates the processes of representing human knowledge in artificial 
intelligence systems. To solve this problem, it is proposed to use at the same time 
encoding of both concepts of the listed categories and forms of their expression in the 
digital medium. The proposed approach is illustrated by the example of an intelligent 
vocabulary system that uses encoding of both concepts and words, which are verbal 
forms of concept representation.}
    
    
    \KWE{knowledge encoding; polyadic computing; digital medium; artificial 
intelligence; categories of concepts; socialization of knowledge concepts}
    
 \DOI{10.14357/19922264190416} 

%\vspace*{-14pt}

% \Ack
  % \noindent
  


%\vspace*{-6pt}

  \begin{multicols}{2}

\renewcommand{\bibname}{\protect\rmfamily References}
%\renewcommand{\bibname}{\large\protect\rm References}

{\small\frenchspacing
 {%\baselineskip=10.8pt
 \addcontentsline{toc}{section}{References}
 \begin{thebibliography}{99}
\bibitem{1-zac-1}
\Aue{Zatsman, I.\,M., V.\,V.~Kosarik, and O.\,A.~Kurchavova}. 2008. Zadachi 
predstavleniya lichnostnykh i~kollektivnykh kontseptov v~tsifrovoy srede 
[Representation of individual and collective concepts in digital medium]. 
\textit{Informatika 
i~ee Primeneniya~--- Inform. Appl.} 2(3):54--69.
\bibitem{2-zac-1}
\Aue{Zatsman, I.\,M.} 2019. Interfeysy tret'ego poryadka v~informatike [Third-order 
interfaces in informatics]. \textit{Informatika i~ee Primeneniya~--- Inform. Appl.} 
13(3):82--89. 
\bibitem{3-zac-1}
\Aue{Wierzbicki, A.\,P., and Y.~Nakamori.} 2006. Basic dimensions of creative 
space. \textit{Creative space: Models of creative processes for knowledge civilization 
age.} Berlin--Heidelberg: Springer Verlag. 59--90.
\bibitem{4-zac-1}
\Aue{Wierzbicki, A.\,P., and Y.~Nakamori.} 2007. Knowledge sciences: Some new 
developments. \textit{Betriebswirtsch} 77(3):271--295.
\bibitem{5-zac-1}
\Aue{Rosenbloom, P.} 2013. \textit{On computing: The fourth great scientific 
domain.} Cambridge: MIT Press. 308~p.
\bibitem{6-zac-1}
\Aue{Denning, P., and P.~Rosenbloom.} 2009. Computing: The fourth great domain 
of science. \textit{Commun. ACM} 52(9):27--29.

\bibitem{8-zac-1}
\Aue{Zatsman, I.} 2012. Tracing emerging meanings by computer: Semiotic 
framework. \textit{13th European Conference on Knowledge Management 
Proceedings}. Reading: Academic Publishing International Ltd. 2:1298--1307.

\bibitem{7-zac-1}
\Aue{Zatsman, I.} 2014. Table of interfaces of informatics as computer and information 
science. \textit{Sci. Tech. Inf. Proc.} 41(4):233--246.

\bibitem{9-zac-1}
\Aue{Baars, B., and N.~Gage.} 2010. \textit{Cognition, brain, and consciousness: 
Introduction to cognitive neuroscience}. Amsterdam: Academic Press/Elsevier. 
677~p.
\bibitem{10-zac-1}
\Aue{Eco, U.} 1976. \textit{A~theory of semiotics}. Bloomington, IN: Indiana University 
Press. 356~p.


\bibitem{12-zac-1} %11
\Aue{Schalk, G., and E.\,C.~Leuthardt.} 2011. Brain--computer interfaces using 
electrocorticographic signals. \textit{IEEE Rev. Biomed. Eng}. 4(1):140--154.

\bibitem{11-zac-1} %12
\Aue{Sunny, T.\,D., T.~Aparna, P.~Neethu, J.~Venkateswaran, V.~Vishnupriya, and 
P.\,S.~Vyas.} 2016. Robotic arm with brain--computer interfacing. \textit{Proc. 
Tech.} 24:1089--1096.

\bibitem{13-zac-1}
\Aue{Mueller, J.\,P., and L.~Massaron.} 2018. \textit{Artificial intelligence for 
dummies.} Hoboken, NJ: John Wiley \& Sons. 316~p.
\bibitem{14-zac-1}
\Aue{Gardner, H.} 2004. \textit{Frames of mind: The theory of multiple 
intelligences}. New York, NY: Basic Books. 440~p.
\bibitem{15-zac-1}
\Aue{Goncharov, A., and O.~Inkova.} 2019. Metodika poiska implitsitnykh  
logiko-semanticheskikh otnosheniy v~tekste [Methods for identification of implicit 
logical-semantic relations in texts]. \textit{Informatika i~ee Primeneniya~--- Inform. 
Appl}. 13(3):100--107.
\bibitem{16-zac-1}
\Aue{Goncharov, A., and O.~Inkova.} 2019. Sposoby vyrazheniya prichinnykh 
otnosheniy v~russkom yazyke: opyt ana\-li\-za s~ispol'zovaniem krosslingvisticheskoy 
nadkorpusnoy bazy dannykh [Means of expressing causal relations in Russian: 
Analysis using a~cross-linguistic supracorpora database]. \textit{Russkaya 
grammatika: aktivnye protsessy v~yazyke i~rechi: Sb.\ nauchnykh trudov 
Mezhdunar. nauchnogo simpoziuma} [Russian Grammar: Active Processes in Language and Discourse.  
Scientific Symposium (International) Proceedings]. Yaroslavl: RIO YAGPU.  
385--396.
\bibitem{17-zac-1}
\Aue{Inkova, О., and E.~Manzotti.} 2019 (in press). \textit{Svyaznost' teksta: 
mereologicheskie logiko-semanticheskie  otnosheniya}
[Text coherence: Mereological logical semantic relations]. Moscow: LRC Publishing House.  
\bibitem{18-zac-1}
\Aue{Lyutyy, A.\,A.} 2002. \textit{Yazyk karty: sushchnost', sistema, funktsii} [Map 
language: Essence, system, functions]. Moscow: GI RAS. 327~p.
\bibitem{19-zac-1}
\Aue{Inkova, О., and M.\,G.~Kruzhkov.} 2019. So\-che\-ta\-emost'  
logiko-semanticheskikh otnosheniy: kolichestvennye metody analiza [Compatibility 
of logical semantic relations: Methods of quantitative analysis]. \textit{Informatika 
i~ee Primeneniya~--- Inform. Appl}. 13(2):83--91.
\bibitem{20-zac-1}
\Aue{Mann, W.\,C., and S.\,A.~Thompson.} 1988. Rhetorical structure theory: 
Toward a functional theory of text organization. \textit{Text}
%: Interdisciplinary J.~Study of Discourse} 
8(3):243--281.

\bibitem{23-zac-1} %21
\Aue{Breindl, E., A.~Volodina, and U.\,H.~\mbox{Wa{\!\ptb{\!\ss}}ner.}} 2014. 
\textit{Handbuch der deutschen Konnektoren 2.~Semantik der deutschen 
Satzverkn$\ddot{\mbox{u}}$pfer}. Berlin: Walter de Gruyter. 1327~p.

\bibitem{21-zac-1} %22
\Aue{Bunt, H., and R.~Prasad.} 2016. ISO DR-Core (ISO \mbox{24617-8}): Core concepts for 
the annotation of discourse relations. \textit{12th Joint ACL-ISO Workshop on 
Interoperable Semantic Annotation Proceedings}. Portoroz. 45--54.
\bibitem{22-zac-1} %23
\Aue{Sanders, T.\,J.\,M., V.~Demberg, J.~Hoek, M.\,C.\,J.~Scholman, 
A.\,F.~Torabi, S.~Zufferey, and J.~Evers-Vermeul.} 2018. Unifying 
dimensions in coherence relations: How various annotation frameworks are 
related. \textit{Corpus Linguist. Ling.} 71~p. Available at: {\sf 
https://www.degruyter.com/ view/j/cllt.ahead-of-print/cllt-2016-0078/cllt-2016-0078.xml?rskey=Vr3MhX\&result=1} (accessed 
September~30, 2019).

\bibitem{24-zac-1}
\Aue{Ramiro, C., M.~Srinivasan, B.\,C.~Malt, and Y.~Xu.} 2018. Algorithms in the 
historical emergence of word senses. \textit{P.~Natl. Acad. Sci. USA} 115(10): 2323--2328.
\bibitem{25-zac-1}
\Aue{Kay, C., J.~Roberts, M.~Samuels, I.~Wotherspoon, and M.~Alexander.} 2015. 
The historical thesaurus of English. Version~4.2.
Glasgow, U.K.: University of Glasgow.
 Available at: {\sf 
https://historicalthesaurus.arts.gla.ac.uk} (accessed September~30, 2019).
\bibitem{26-zac-1}
\Aue{Kay, C., J.~Roberts, M.~Samuels, I.~Wotherspoon, and M.~Alexander.} 2015. 
The historical thesaurus of English. Version~4.2: Face. Available at: {\sf 
https://\linebreak  historicalthesaurus.arts.gla.ac.uk/category-selection/ ?qsearch=face} (accessed 
September~30, 2019).
\bibitem{27-zac-1}
The British National Corpus (Oxford University Computing Services). Available at: 
{\sf http://www.natcorp.ox.ac.uk} (accessed September~30, 2019).
\bibitem{28-zac-1}
\Aue{Zatsman, I.} 2003. \textit{Kontseptual'nyy poisk i~kachestvo informatsii} 
[Conceptual retrieval and quality of information]. Moscow: Nauka. 272~p.
\bibitem{29-zac-1}
\Aue{Zatsman, I.} 2009. Semioticheskaya model' vzaimosvyazey kontseptov, 
informatsionnykh ob''ektov i~komp'yuternykh kodov [Semiotic model of 
relationships of concepts, information objects, and computer codes]. 
\textit{Informatika i~ee Primeneniya~--- Inform. Appl.} 3(2):65--81.
\bibitem{30-zac-1}
\Aue{Zatsman, I.} 2015. Protsessy tselenapravlennoy generatsii i~razvitiya  
kross-yazykovykh ekspertnykh znaniy: semioticheskie osnovaniya modelirovaniya 
[Goal-oriented processes of cross-lingual expert knowledge creation: Semiotic 
foundations for modeling]. \textit{Informatika i~ee Primeneniya~--- Inform. Appl.} 
9(3):106--123.
\bibitem{31-zac-1}
\Aue{Vakulenko, V., and I.~Zatsman.} 2019. Printsipy sozdaniya i~funktsii 
intellektual'noy slovarnoy sistemy [Framework for the design of intellectual 
dictionary systems and their primary functions]. \textit{Sistemy i~Sredstva 
Informatiki~--- Systems and Means of Informatics} 29(4):96--105.

\bibitem{33-zac-1} %32
\Aue{Zatsman, I.\,M.} 2018. Implitsirovannye znaniya: osnovaniya i~tekhnologii 
izvlecheniya [Implied knowledge: Foundations and technologies of explication]. 
\textit{Informatika i~ee Primeneniya~--- Inform. Appl.} 12(3):74--82.

\bibitem{32-zac-1} %33
\Aue{Zatsman, I.\,M.} 2018. Stadii tselenapravlennogo izvlecheniya znaniy, 
implitsirovannykh v~parallel'nykh tekstakh [Stages of goal-oriented discovery of 
knowledge implied in parallel texts]. \textit{Sistemy i~Sredstva Informatiki~--- 
Systems and Means of Informatics} 28(3):175--188.

\bibitem{34-zac-1}
\Aue{Zatsman, I.\,M.} 2019. Tselenapravlennoe razvitie sistem lingvisticheskikh 
znaniy: vyyavlenie i~zapolnenie lakun [Goal-oriented development of linguistic 
knowledge systems: Identifying and filling of lacunae]. \textit{Informatika i~ee 
Primeneniya~--- Inform. Appl.} 13(1):91--98.
\bibitem{35-zac-1}
\Aue{Goncharov, A.\,A., and I.\,M.~Zatsman.} 2019. Informatsionnye transformatsii 
parallel'nykh tekstov v~zadachakh izvlecheniya znaniy [Information transformations 
of parallel texts in knowledge extraction]. \textit{Sistemy i~Sredstva Informatiki~--- 
Systems and Means of Informatics} 29(1):180--193.
\end{thebibliography}

 }
 }

\end{multicols}

\vspace*{-6pt}

\hfill{\small\textit{Received October 1, 2019}}

%\pagebreak

%\vspace*{-22pt}

\Contrl

\noindent
\textbf{Zatsman Igor M.} (b.\ 1952)~--- Doctor of Science in technology, Head of 
Department, Institute of Informatics Problems, Federal Research Center ``Computer 
Science and Control'' of the Russian Academy of Sciences,  44-2~Vavilov Str., Moscow 
119333, Russian Federation; \mbox{izatsman@yandex.ru}
\label{end\stat}

\renewcommand{\bibname}{\protect\rm Литература}  
    
       