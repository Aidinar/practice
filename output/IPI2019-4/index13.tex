%Информатика и её применения
%Том 13 Выпуск 1-4 Год 2019

\def\stat{cont}
{%\hrule\par
%\vskip 7pt % 7pt
\raggedleft\Large \bf%\baselineskip=3.2ex
А\,В\,Т\,О\,Р\,С\,К\,И\,Й\ \ У\,К\,А\,З\,А\,Т\,Е\,Л\,Ь\ \ З\,А\ \ 2\,0\,1\,9 г. \vskip 17pt
 \hrule
 \par
\vskip 21pt plus 6pt minus 3pt }

\label{st\stat}

\def\tit{\ }

\def\aut{\ }
\def\auf{\ }

\def\leftkol{\ } % ENGLISH ABSTRACTS}

\def\rightkol{\ } %АВТОРСКИЙ УКАЗАТЕЛЬ ЗА 2019 г.} %ENGLISH ABSTRACTS}

\titele{\tit}{\aut}{\auf}{\leftkol}{\rightkol}
\addcontentsline{toc}{subsection}{\textrm\textbf Авторский указатель за 2019 г.}

%\vspace*{-12pt}

\noindent
{\tabcolsep=3pt
\begin{tabular}{p{397pt}cc}
&\textbf{Вып.} & \textbf{Стр.}\\[6pt]
\Avtors{Абгарян~К.\,К., Осипова~В.\,А.} Применение методов поддержки принятия решений для\linebreak
\\[-12pt]
\hspace*{23pt}многокритериальной задачи отбора многомасштабных композиций&2&47--53\\
\Avtors{Агаларов~Я.\,М., Коновалов~М.\,Г.} Доказательство унимодальности целевой функции\linebreak
\\[-12pt]
\hspace*{23pt}в~задаче порогового управления нагрузкой на~сервер&2&2--6\\
\Avtors{Агаларов~Я.\,М., Ушаков~В.\,Г.} Об унимодальности функции дохода системы массового\linebreak
\\[-12pt]
\hspace*{23pt}обслуживания типа $G|M|s$ с~управляемой очередью&1&55--61\\
\Avtors{Агасандян~Г.\,А.} Вычисление показателей оптимальных по CC-VaR портфелей на~рынках\linebreak
\\[-12pt]
\hspace*{23pt}опционов&3&72--81\\
\Avtors{Агасандян~Г.\,А.} Теоретические основы оптимизации по континуальному критерию VaR на совокупности рынков&4&36--41\\
\Avtors{Анашин~В.\,С.} О теоретико-автоматных моделях блокчейн-среды&2&29--36\\
\Avtors{Аникеев~Д.\,А., Пенкин~Г.\,О., Стрижов~В.\,В.} Классификация физической активности\linebreak
\\[-12pt]
\hspace*{23pt}человека с~помощью локальных аппроксимирующих моделей&1&40--48\\
\Avtors{Арутюнов~Е.\,Н., Кудрявцев~А.\,А., Титова~А.\,И.} Байесовские модели баланса факторов, \linebreak
\\[-12pt]
\hspace*{23pt}имеющих априорные распределения Вейбулла и~Накагами&2&71--75\\
\Avtors{Бахтеев~О.\,Ю.} см.\ Грабовой~А.\,В.&&\\
\Avtors{Бондаренко~Н.\,Н.} см.\ Журавлев~Ю.\,И.&&\\
\Avtors{Борисов~А.\,В.} Численные схемы фильтрации марковских скачкообразных процессов по\linebreak
\\[-12pt]
\hspace*{23pt}дискретизованным наблюдениям~I: характеристики точности&4&68--75\\
\Avtors{Босов~А.\,В., Миллер~Г.\,Б.} О развитии концепции условно-минимаксной нелинейной\linebreak
\\[-12pt]
\hspace*{23pt}фильтрации: модифицированный фильтр и~его анализ&2&\hphantom{1}7--15\\
\Avtors{Босов~А.\,В., Мхитарян~Г.\,А., Наумов~А.\,В., Сапунова~А.\,П.} Использование модели гамма-распределения в~задаче формирования ограниченного по времени теста в~системе\linebreak
\\[-12pt]
\hspace*{23pt}дистанционного обучения&4&11--17\\
\Avtors{Босов~А.\,В., Стефанович~А.\,И.} Управление выходом стохастической дифференциальной системы по~квадратичному критерию. II.~Численное решение уравнений динами-\linebreak
\\[-12pt]
\hspace*{23pt}ческого программирования&1&\hphantom{1}9--15\\
\Avtors{Босов~А.\,В., Стефанович~А.\,И.} Управление выходом стохастической дифференциальной системы по квадратичному критерию. III.~Анализ свойств оптимального управ-\linebreak
\\[-12pt]
\hspace*{23pt}ления&3&41--49\\
\Avtors{Бурлуцкий~В.\,В., Якимчук~А.\,В., Мельников~А.\,В., Царегородцев~А.\,Л., Волошин~С.\,В.} Разработка метода формирования признакового пространства и~модели для оценки и~прогнозирования антропогенного влияния на окружающую среду (на примере\linebreak
\\[-12pt]
\hspace*{23pt}лесного фонда нефтедобывающего региона)&3&131--136\\
\Avtors{Вахтанов~Н.\,А.} см.\ Шнурков~П.\,В.&&\\
\Avtors{Вахтанов~Н.\,А.} см.\ Шнурков~П.\,В.&&\\
\Avtors{Виноградов~А.\,П.} см.\ Журавлев~Ю.\,И.&&\\
\Avtors{Волошин~С.\,В.} см.\ Бурлуцкий~В.\,В.&&\\
\Avtors{Вышинский~Л.\,Л., Курьянский~М.\,К., Флеров~Ю.\,А.} Цифровая модель весового паспорта\linebreak
\\[-12pt]
\hspace*{23pt}летательного аппарата&4&\hphantom{1}3--10\\
\Avtors{Гайдамака~А.\,А., Чухно~Н.\,В., Чухно~О.\,В., Самуйлов~К.\,Е., Шоргин~С.\,Я.} Формализация метода ранжирования альтернатив для процесса группового принятия решений при\linebreak
\\[-12pt]
\hspace*{23pt}анализе социальных сетей&3&63--71\\
\Avtors{Гайдамака~Ю.\,В.} см.\ Горбунова~А.\,В.&&\\
\end{tabular}
}

\pagebreak

\def\leftkol{АВТОРСКИЙ УКАЗАТЕЛЬ ЗА 2019 г.} % ENGLISH ABSTRACTS}

\def\rightkol{АВТОРСКИЙ УКАЗАТЕЛЬ ЗА 2019 г.} %ENGLISH ABSTRACTS}

%\thispagestyle{myheadings}
\def\leftfootline{\small{\textbf{\thepage}
\hfill ИНФОРМАТИКА И ЕЁ ПРИМЕНЕНИЯ\ \ \ том~13\ \ \ выпуск~4\ \ \ 2019}
}%
 \def\rightfootline{\small{ИНФОРМАТИКА И ЕЁ ПРИМЕНЕНИЯ\ \ \ том~13\ \ \ выпуск~4\ \ \ 2019
 \hfill \textbf{\thepage}}}


\noindent
{\tabcolsep=3pt
\begin{tabular}{p{394pt}cc}
&\textbf{Вып.} & \textbf{Стр.}\\[3pt]
\Avtors{Гольская~А.\,А.} см.\ Маркова~Е.\,В.&&\\
\Avtors{Гончаров~А.\,А., Зацман~И.\,М., Кружков~М.\,Г.} Темпоральные данные в~лексикографиче-\linebreak
\\[-12pt]
\hspace*{23pt}ских базах знаний&4&90--96\\
\Avtors{Гончаров~А.\,А., Инькова~О.\,Ю.} Методика поиска имплицитных логико-семантических\linebreak
\\[-12pt]
\hspace*{23pt}отношений в~тексте&3&\hphantom{1}97--104\\
\Avtors{Горбунова~А.\,В., Наумов~В.\,А., Гайдамака~Ю.\,В., Самуйлов~К.\,Е.} Ресурсные системы\linebreak
\\[-12pt]
\hspace*{23pt}массового обслуживания с~произвольным обслуживанием&1&\hphantom{1}99--107\\
\Avtors{Горшенин~А.\,К., Кузьмин~В.\,Ю.} Оптимизация гиперпараметров нейронных сетей с~ис-\linebreak
\\[-12pt]
\hspace*{23pt}пользованием высокопроизводительных вычислений для~предсказания осадков&1&75--81\\
\Avtors{Горшенин~А.\,К., Кузьмин~В.\,Ю.} Применение рекуррентных нейронных сетей для\linebreak
\\[-12pt]
\hspace*{23pt}прогнозирования моментов конечных нормальных смесей&3&114--121\\
\Avtors{Горшенин~А.\,К., Мартынов~О.\,П.} Гибридные модели экстремального градиентного\linebreak
\\[-12pt]
\hspace*{23pt}бустинга для восстановления пропущенных значений в~данных об~осадках&3&34--40\\
\Avtors{Грабовой~А.\,В., Бахтеев~О.\,Ю., Стрижов~В.\,В.} Определение релевантности параметров\linebreak
\\[-12pt]
\hspace*{23pt}нейросети&2&62--70\\
\Avtors{Гринченко~С.\,Н.} О генезисе информационного общества: информатико-кибернетиче-\linebreak
\\[-12pt]
\hspace*{23pt}ское модельное представление&2&100--108\\
\Avtors{Грушо~А.\,А., Грушо~Н.\,А., Тимонина~Е.\,Е.} Использование метаданных для реализации\linebreak
\\[-12pt]
\hspace*{23pt}требований политики безопасности MLS&4&85--89\\
\Avtors{Грушо~А.\,А., Грушо~Н.\,А., Тимонина~Е.\,Е.} Методы выявления <<слабых>> признаков\linebreak
\\[-12pt]
\hspace*{23pt}нарушений информационной безопасности&3&3--8\\
\Avtors{Грушо~А.\,А., Забежайло~М.\,И., Грушо~Н.\,А., Тимонина~Е.\,Е.} Архитектурные решения в~задаче выявления мошенничества при анализе информационных потоков\linebreak
\\[-12pt]
\hspace*{23pt}в~цифровой экономике&2&22--28\\
\Avtors{Грушо~А.\,А., Забежайло~М.\,И., Грушо~Н.\,А., Тимонина~Е.\,Е.} Формирование концептов\linebreak
\\[-12pt]
\hspace*{23pt}на основе малых выборок&4&81--84\\
\Avtors{Грушо~Н.\,А.} см.\ Грушо~А.\,А.&&\\
\Avtors{Грушо~Н.\,А.} см.\ Грушо~А.\,А.&&\\
\Avtors{Грушо~Н.\,А.} см.\ Грушо~А.\,А.&&\\
\Avtors{Грушо~Н.\,А.} см.\ Грушо~А.\,А.&&\\
\Avtors{Гудкова~И.\,А.} см.\ Маркова~Е.\,В.&&\\
\Avtors{Дзантиев~И.\,Л.} см.\ Маркова~Е.\,В.&&\\
\Avtors{Докукин~А.\,А.} см.\ Журавлев~Ю.\,И.&&\\
\Avtors{Дулин~С.\,К., Дулина~Н.\,Г., Кожунова~О.\,С.} Синтез геоданных в пространственных\linebreak
\\[-12pt]
\hspace*{23pt}инфраструктурах на~основе связанных данных&1&82--90\\
\Avtors{Дулина~Н.\,Г.} см.\ Дулин~С.\,К.&&\\
\Avtors{Дюкова~Е.\,В., Масляков~Г.\,О., Прокофьев~П.\,А.} О числе максимальных независимых\linebreak
\\[-12pt]
\hspace*{23pt}элементов частичных порядков (случай цепей)&1&25--32\\
\Avtors{Журавлев~Ю.\,И., Сенько~О.\,В., Бондаренко~Н.\,Н., Рязанов~В.\,В., Докукин~А.\,А., Виноградов~А.\,П.} Исследование возможности прогнозирования изменения финансового\linebreak
\\[-12pt]
\hspace*{23pt}состояния кредитной организации на основе публикуемой отчетности&4&30--35\\
\Avtors{Забежайло~М.\,И.} см.\ Грушо~А.\,А.&&\\
\Avtors{Забежайло~М.\,И.} см.\ Грушо~А.\,А.&&\\
\Avtors{Захарова~Т.\,В., Тархов~А.\,А.} Оценка уровня значимости критерия Шуирманна для\linebreak
\\[-12pt]
\hspace*{23pt}проверки гипотезы биоэквивалентности при наличии пропущенных данных&3&58--62\\
\Avtors{Зацаринный~А.\,А., Коротков~В.\,В., Матвеев~М.\,Г.} Моделирование процессов сетевого планирования портфеля проектов с~неоднородными ресурсами в~условиях нечет-\linebreak
\\[-12pt]
\hspace*{23pt}кой информации&2&92--99\\
\Avtors{Зацман~И.\,М.} Интерфейсы третьего порядка в~информатике&3&82--89\\
\Avtors{Зацман~И.\,М.} Кодирование концептов в~цифровой среде&4&\hphantom{1}97--106\\
\Avtors{Зацман~И.\,М.} Целенаправленное развитие систем лингвистических знаний: выявление\linebreak
\\[-12pt]
\hspace*{23pt}и~заполнение лакун&1&91--98\\
\Avtors{Зацман~И.\,М.} см.\ Гончаров~А.\,А.&&\\
\end{tabular}
}

\pagebreak

\def\leftkol{АВТОРСКИЙ УКАЗАТЕЛЬ ЗА 2019 г.} % ENGLISH ABSTRACTS}

\def\rightkol{АВТОРСКИЙ УКАЗАТЕЛЬ ЗА 2019 г.} %ENGLISH ABSTRACTS}

%\thispagestyle{myheadings}
\def\leftfootline{\small{\textbf{\thepage}
\hfill ИНФОРМАТИКА И ЕЁ ПРИМЕНЕНИЯ\ \ \ том~13\ \ \ выпуск~4\ \ \ 2019}
}%
 \def\rightfootline{\small{ИНФОРМАТИКА И ЕЁ ПРИМЕНЕНИЯ\ \ \ том~13\ \ \ выпуск~4\ \ \ 2019
 \hfill \textbf{\thepage}}}


\noindent
{\tabcolsep=3pt
\begin{tabular}{p{394pt}cc}
&\textbf{Вып.} & \textbf{Стр.}\\[3pt]
\Avtors{Зейфман~А.\,И., Сатин~Я.\,А., Киселева~К.\,М.} Об оценках скорости сходимости для некоторых моделей массового обслуживания с~неполно заданными интенсивно-\linebreak
\\[-12pt]
\hspace*{23pt}стями&3&14--19\\
\Avtors{Инькова~О.\,Ю., Кружков~М.\,Г.} Сочетаемость логико-семантических отношений: коли-\linebreak
\\[-12pt]
\hspace*{23pt}чественные методы анализа&2&83--91\\
\Avtors{Инькова~О.\,Ю.} см.\ Гончаров~А.\,А.&&\\
\Avtors{Кириков~И.\,А.} см.\ Румовская~С.\,Б.&&\\
\Avtors{Киселева~К.\,М.} см.\ Зейфман~А.\,И.&&\\
\Avtors{Ковалёв~Д.\,Ю., Тарасов~Е.\,А.} Виртуальные эксперименты в~исследованиях с~интенсив-\linebreak
\\[-12pt]
\hspace*{23pt}ным использованием данных&2&117--125\\
\Avtors{Кожунова~О.\,С.} см.\ Дулин~С.\,К.&&\\
\Avtors{Колесников~А.\,В., Листопад~С.\,В.} Протокол гетерогенного мышления гибридной интеллектуальной многоагентной системы для решения проблемы восстановления\linebreak
\\[-12pt]
\hspace*{23pt}распределительной электросети&2&76--82\\
\Avtors{Коновалов~М.\,Г., Разумчик~Р.\,В.} Комплексное управление в~одном классе систем\linebreak
\\[-12pt]
\hspace*{23pt}с~параллельным обслуживанием&4&54--59\\
\Avtors{Коновалов~М.\,Г.} см.\ Агаларов~Я.\,М.&&\\
\Avtors{Коротков~В.\,В.} см.\ Зацаринный~А.\,А.&&\\
\Avtors{Кривенко~М.\,П.} Выбор модели данных в~задачах медицинской диагностики&4&27--29\\
\Avtors{Кружков~М.\,Г.} см.\ Гончаров~А.\,А.&&\\
\Avtors{Кружков~М.\,Г.} см.\ Инькова~О.\,Ю.&&\\
\Avtors{Кудрявцев~А.\,А.} Априорное обобщенное гамма-распределение в~байесовских моделях\linebreak
\\[-12pt]
\hspace*{23pt}баланса&3&27--33\\
\Avtors{Кудрявцев~А.\,А.} О представлении 
гамма-экспоненциального и~обобщенного отрица-\linebreak
\\[-12pt]
\hspace*{23pt}тельного биномиального распределений&4&76--80\\
\Avtors{Кудрявцев~А.\,А., Палионная~С.\,И., Шоргин~В.\,С.} Априорные Фреше и масштабированное\linebreak
\\[-12pt]
\hspace*{23pt}обратное хи-распределение в~байесовских моделях баланса&1&62--66\\
\Avtors{Кудрявцев~А.\,А.} см.\ Арутюнов~Е.\,Н.&&\\
\Avtors{Кузьмин~В.\,Ю.} см.\ Горшенин~А.\,К.&&\\
\Avtors{Кузьмин~В.\,Ю.} см.\ Горшенин~А.\,К.&&\\
\Avtors{Курьянский~М.\,К.} см.\ Вышинский~Л.\,Л.&&\\
\Avtors{Ланге~M.\,M.} О~сравнительной эффективности схем классификации данных на~ансамбле\linebreak
\\[-12pt]
\hspace*{23pt}источников с~использованием средней взаимной информации&4&18--26\\
\Avtors{Лебедев~А.\,В.} Нетранзитивные триплеты непрерывных случайных величин и~их прило-\linebreak
\\[-12pt]
\hspace*{23pt}жения&3&20--26\\
\Avtors{Листопад~С.\,В.} см.\ Колесников~А.\,В.&&\\
\Avtors{Логачев~О.\,А., Сукаев~А.\,А., Федоров~С.\,Н.} Об одном методе решения систем 
квад\-ра\-тич\-ных булевых уравнений, использующем локальные аффинности булевых\linebreak
\\[-12pt]
\hspace*{23pt}функций&2&37--46\\
\Avtors{Логачев~О.\,А., Сукаев~А.\,А., Федоров~С.\,Н.} Полиномиальные алгоритмы вычисления\linebreak
\\[-12pt]
\hspace*{23pt}локальных аффинностей квадратичных булевых функций&1&67--74\\
\Avtors{Лукашенко~О.\,В., Морозов~Е.\,В., Пагано~М.} Гауссовская аппроксимация процесса\linebreak
\\[-12pt]
\hspace*{23pt}распределенных вычислений&2&109--116\\
\Avtors{Малашенко~Ю.\,Е., Назарова~И.\,А., Новикова~Н.\,М.} Анализ уязвимости многополюсных\linebreak
\\[-12pt]
\hspace*{23pt}сетей при~структурных повреждениях&1&33--39\\
\Avtors{Маркова~Е.\,В., Гольская~А.\,А., Дзантиев~И.\,Л., Гудкова~И.\,А., Шоргин~С.\,Я.} Сравнительный анализ показателей эффективности модели беспроводной сети меж\-ма\-шин\-ного взаимодействия, работающей в~рамках двух политик разделения радиоре-\linebreak
\\[-12pt]
\hspace*{23pt}сурсов&1&108--116\\
\Avtors{Мартынов~О.\,П.} см.\ Горшенин~А.\,К.&&\\
\Avtors{Масляков~Г.\,О.} см.\ Дюкова~Е.\,В.&&\\
\Avtors{Матвеев~М.\,Г.} см.\ Зацаринный~А.\,А.&&\\
\Avtors{Мейханаджян~Л.\,А., Разумчик~Р.\,В.} Система массового обслуживания Geo$/G/1/\infty$\linebreak
\\[-12pt]
\hspace*{23pt}синверсионным порядком обслуживания и~ресамплингом в~дискретном времени&4&60--67\\
\end{tabular}
}

\pagebreak

\def\leftkol{АВТОРСКИЙ УКАЗАТЕЛЬ ЗА 2019 г.} % ENGLISH ABSTRACTS}

\def\rightkol{АВТОРСКИЙ УКАЗАТЕЛЬ ЗА 2019 г.} %ENGLISH ABSTRACTS}

%\thispagestyle{myheadings}
\def\leftfootline{\small{\textbf{\thepage}
\hfill ИНФОРМАТИКА И ЕЁ ПРИМЕНЕНИЯ\ \ \ том~13\ \ \ выпуск~4\ \ \ 2019}
}%
 \def\rightfootline{\small{ИНФОРМАТИКА И ЕЁ ПРИМЕНЕНИЯ\ \ \ том~13\ \ \ выпуск~4\ \ \ 2019
 \hfill \textbf{\thepage}}}


\noindent
{\tabcolsep=3pt
\begin{tabular}{p{394pt}cc}
&\textbf{Вып.} & \textbf{Стр.}\\[3pt]
\Avtors{Мельников~А.\,В.} см.\ Бурлуцкий~В.\,В.&&\\
\Avtors{Миллер~Г.\,Б.} см.\ Босов~А.\,В.&&\\
\Avtors{Морозов~Е.\,В.} см.\ Лукашенко~О.\,В.&&\\
\Avtors{Мхитарян~Г.\,А.} см.\ Босов~А.\,В.&&\\
\Avtors{Назарова~И.\,А.} см.\ Малашенко~Ю.\,Е.&&\\
\Avtors{Наумов~А.\,В.} см.\ Босов~А.\,В.&&\\
\Avtors{Наумов~В.\,А.} см.\ Горбунова~А.\,В.&&\\
\Avtors{Новикова~Н.\,М.} см.\ Малашенко~Ю.\,Е.&&\\
\Avtors{Нуриев~В.\,А.} Архитектура системы нейронного машинного перевода&3&90--96\\
\Avtors{Осипова~В.\,А.} см.\ Абгарян~К.\,К.&&\\
\Avtors{Павлов~Ю.\,Л.} Об асимптотике кластерного коэффициента конфигурационного графа\linebreak
\\[-12pt]
\hspace*{23pt}с~неизвестным распределением степеней вершин&3&\hphantom{1}9--13\\
\Avtors{Пагано~М.} см.\ Лукашенко~О.\,В.&&\\
\Avtors{Палионная~С.\,И.} см.\ Кудрявцев~А.\,А.&&\\
\Avtors{Панов~А.\,И.} см.\ Смирнов~И.\,В.&&\\
\Avtors{Пенкин~Г.\,О.} см.\ Аникеев~Д.\,А.&&\\
\Avtors{Прокофьев~П.\,А.} см.\ Дюкова~Е.\,В.&&\\
\Avtors{Разумчик~Р.\,В.} см.\ Коновалов~М.\,Г.&&\\
\Avtors{Разумчик~Р.\,В.} см.\ Мейханаджян~Л.\,А.&&\\
\Avtors{Румовская~С.\,Б., Кириков~И.\,А.} Методы моделирования и~визуального представления\linebreak
\\[-12pt]
\hspace*{23pt}конфликта в~малом коллективе экспертов, решающих проблемы (обзор)&3&122--130\\
\Avtors{Рыбаков~К.\,А.} Об одном классе задач фильтрации на многообразиях&1&16--24\\
\Avtors{Рязанов~В.\,В.} см.\ Журавлев~Ю.\,И.&&\\
\Avtors{Самуйлов~К.\,Е.} см.\ Гайдамака~А.\,А.&&\\
\Avtors{Самуйлов~К.\,Е.} см.\ Горбунова~А.\,В.&&\\
\Avtors{Сапунова~А.\,П.} см.\ Босов~А.\,В.&&\\
\Avtors{Сатин~Я.\,А.} см.\ Зейфман~А.\,И.&&\\
\Avtors{Сейфуль-Мулюков~Р.\,Б.} Законы информатики и~синергетики в~познании сложных\linebreak
\\[-12pt]
\hspace*{23pt}систем&4&107--113\\
\Avtors{Сенько~О.\,В.} см.\ Журавлев~Ю.\,И.&&\\
\Avtors{Синицын~И.\,Н.} Интерполяционное аналитическое моделирование распределений\linebreak
\\[-12pt]
\hspace*{23pt}в~сложных стохастических системах&1&2--8\\
\Avtors{Скрынник~А.\,А.} см.\ Смирнов~И.\,В.&&\\
\Avtors{Смирнов~И.\,В., Панов~А.\,И., Скрынник~А.\,А., Чистова~Е.\,В.} Персональный когнитивный\linebreak
\\[-12pt]
\hspace*{23pt}ассистент: концепция и~принципы работы&3&105--113\\
\Avtors{Стефанович~А.\,И.} см.\ Босов~А.\,В.&&\\
\Avtors{Стефанович~А.\,И.} см.\ Босов~А.\,В.&&\\
\Avtors{Стрижов~В.\,В.} см.\ Аникеев~Д.\,А.&&\\
\Avtors{Стрижов~В.\,В.} см.\ Грабовой~А.\,В.&&\\
\Avtors{Сукаев~А.\,А.} см.\ Логачев~О.\,А.&&\\
\Avtors{Сукаев~А.\,А.} см.\ Логачев~О.\,А.&&\\
\Avtors{Сучков~А.\,П.} Научный результат как информационный объект в~контексте системы\linebreak
\\[-12pt]
\hspace*{23pt}управления научными сервисами&3&137--144\\
\Avtors{Тарасов~Е.\,А.} см.\ Ковалёв~Д.\,Ю.&&\\
\Avtors{Тархов~А.\,А.} см.\ Захарова~Т.\,В.&&\\
\Avtors{Тимонина~Е.\,Е.} см.\ Грушо~А.\,А.&&\\
\Avtors{Тимонина~Е.\,Е.} см.\ Грушо~А.\,А.&&\\
\Avtors{Тимонина~Е.\,Е.} см.\ Грушо~А.\,А.&&\\
\Avtors{Тимонина~Е.\,Е.} см.\ Грушо~А.\,А.&&\\
\Avtors{Титова~А.\,И.} см.\ Арутюнов~Е.\,Н.&&\\
\Avtors{Ушаков~В.\,Г., Ушаков~Н.\,Г.} Выходящие потоки в~однолинейной системе с~относитель-\linebreak
\\[-12pt]
\hspace*{23pt}ным приоритетом&4&42--47\\
\Avtors{Ушаков~В.\,Г.} см.\ Агаларов~Я.\,М.&&\\
\Avtors{Ушаков~Н.\,Г.} см.\ Ушаков~В.\,Г.&&\\
\end{tabular}
}

\pagebreak

\def\leftkol{АВТОРСКИЙ УКАЗАТЕЛЬ ЗА 2019 г.} % ENGLISH ABSTRACTS}

\def\rightkol{АВТОРСКИЙ УКАЗАТЕЛЬ ЗА 2019 г.} %ENGLISH ABSTRACTS}

%\thispagestyle{myheadings}
\def\leftfootline{\small{\textbf{\thepage}
\hfill ИНФОРМАТИКА И ЕЁ ПРИМЕНЕНИЯ\ \ \ том~13\ \ \ выпуск~4\ \ \ 2019}
}%
 \def\rightfootline{\small{ИНФОРМАТИКА И ЕЁ ПРИМЕНЕНИЯ\ \ \ том~13\ \ \ выпуск~4\ \ \ 2019
 \hfill \textbf{\thepage}}}


\noindent
{\tabcolsep=3pt
\begin{tabular}{p{394pt}cc}
&\textbf{Вып.} & \textbf{Стр.}\\[3pt]
\Avtors{Федоров~С.\,Н.} см.\ Логачев~О.\,А.&&\\
\Avtors{Федоров~С.\,Н.} см.\ Логачев~О.\,А.&&\\
\Avtors{Флеров~Ю.\,А.} см.\ Вышинский~Л.\,Л.&&\\
\Avtors{Царегородцев~А.\,Л.} см.\ Бурлуцкий~В.\,В.&&\\
\Avtors{Чистова~Е.\,В.} см.\ Смирнов~И.\,В.&&\\
\Avtors{Чухно~Н.\,В.} см.\ Гайдамака~А.\,А.&&\\
\Avtors{Чухно~О.\,В.} см.\ Гайдамака~А.\,А.&&\\
\Avtors{Шестаков~О.\,В.} Обращение однородных операторов с помощью стабилизированной\linebreak
\\[-12pt]
\hspace*{23pt}жесткой пороговой обработки при неизвестной дисперсии шума&1&49--54\\
\Avtors{Шестаков~О.\,В.} Свойства вейвлет-оценок сигналов, регистрируемых в~случайные\linebreak
\\[-12pt]
\hspace*{23pt}моменты времени&2&16--21\\
\Avtors{Шестаков~О.\,В.} Среднеквадратичный риск нелинейной регуляризации задачи обраще-\linebreak
\\[-12pt]
\hspace*{23pt}ния линейных однородных операторов при случайном объеме выборки&4&48--53\\
\Avtors{Шнурков~П.\,В., Вахтанов~Н.\,А.} Исследование проблемы оптимального управления запасом дискретного продукта в~стохастической модели регенерации с~непрерывно\linebreak
\\[-12pt]
\hspace*{23pt}происходящим потреблением и~случайной задержкой поставки&2&54--61\\
\Avtors{Шнурков~П.\,В., Вахтанов~Н.\,А.} О~решении проблемы оптимального управления запасом дискретного продукта в~стохастической модели регенерации с непрерывно\linebreak
\\[-12pt]
\hspace*{23pt}происходящим потреблением&3&50--57\\
\Avtors{Шоргин~В.\,С.} см.\ Кудрявцев~А.\,А.&&\\
\Avtors{Шоргин~С.\,Я.} см.\ Гайдамака~А.\,А.&&\\
\Avtors{Шоргин~С.\,Я.} см.\ Маркова~Е.\,В.&&\\
\Avtors{Якимчук~А.\,В.} см.\ Бурлуцкий~В.\,В.&&\\
\end{tabular}
}

%\thispagestyle{myheadings}
\def\leftfootline{\small{\textbf{\thepage}
\hfill ИНФОРМАТИКА И ЕЁ ПРИМЕНЕНИЯ\ \ \ том~13\ \ \ выпуск~4\ \ \ 2019}
}%
 \def\rightfootline{\small{ИНФОРМАТИКА И ЕЁ ПРИМЕНЕНИЯ\ \ \ том~13\ \ \ выпуск~4\ \ \ 2019
 \hfill \textbf{\thepage}}}

 \label{end\stat}

\newpage

\def\stat{cont-e}
{%\hrule\par
%\vskip 7pt % 7pt
\raggedleft\Large \bf%\baselineskip=3.2ex
2\,0\,1\,9\ \ A\,U\,T\,H\,O\,R\ \ I\,N\,D\,E\,X \vskip 17pt
 \hrule
 \par
\vskip 21pt plus 6pt minus 3pt }

\label{st\stat}

\def\tit{\ }

\def\aut{\ }
\def\auf{\ }

\def\leftkol{\ } %2019 AUTHOR INDEX} % ENGLISH ABSTRACTS}

\def\rightkol{\ } %2019 AUTHOR INDEX} %ENGLISH ABSTRACTS}

\titele{\tit}{\aut}{\auf}{\leftkol}{\rightkol}
\addcontentsline{toc}{subsection}{\textrm\textbf 2019 Author Index}

\def\leftfootline{\small{\textbf{\thepage}
\hfill INFORMATIKA I EE PRIMENENIYA~--- INFORMATICS AND APPLICATIONS\ \ \ 2019\
\ \ volume~13\ \ \ issue\ 4}
}%
 \def\rightfootline{\small{INFORMATIKA I EE PRIMENENIYA~--- INFORMATICS AND APPLICATIONS\ \ \ 2019\ \ \ volume~13\ \ \ issue\ 4
\hfill \textbf{\thepage}}}

%\vspace*{-12pt}

\noindent
{\tabcolsep=3pt
\begin{tabular}{p{396pt}cc}
&\textbf{Issue} & \textbf{Page}\\[6pt]
\Avtors{Abgaryan~K.\,K.\ and Osipova~V.\,A.} Application of decision support methods for the multicriterial\linebreak
\\[-12pt]
\hspace*{23pt}selection of multiscale compositions&2&47--53\\
\Avtors{Agalarov~Ya.\,M.\ and Konovalov~M.\,G.} Proof of the unimodality of the objective function in\linebreak
\\[-12pt]
\hspace*{23pt}$M/M/N$ queue with threshold-based congestion control&2&2--6\\
\Avtors{Agalarov~Ya.\,M.\ and Ushakov~V.\,G.} On the unimodality of the~income function of a~type $G|M|s$\linebreak
\\[-12pt]
\hspace*{23pt}queueing system with controlled queue&1&55--61\\
\Avtors{Agasandyan~G.\,A.} Performance estimations for optimal-on-CC-VaR portfolios in option markets&3&72--81\\
\Avtors{Agasandyan~G.\,A.} Theoretical foundations of~continuous VaR criterion optimization in~the~col-\linebreak
\\[-12pt]
\hspace*{23pt}lection of~markets&4&36--41\\
\Avtors{Anashin~V.\,S.} On automata models of blockchain&2&29--36\\
\Avtors{Anikeyev~D.\,A., Penkin~G.\,O., and Strijov~V.\,V.} Local approximation models for~human physical\linebreak
\\[-12pt]
\hspace*{23pt}activity classification&1&40--48\\
\Avtors{Arutyunov~E.\,N., Kudryavtsev~A.\,A., and Titova~A.\,I.} Bayesian models of factors balance with\linebreak
\\[-12pt]
\hspace*{23pt}\textit{a~priori} Weibull and Nakagami distributions&2&71--75\\
\Avtors{Bakhteev~O.\,Yu.} see Grabovoy~A.\,V.&&\\
\Avtors{Bondarenko~N.\,N.} see Zhuravlev~Yu.\,I.&&\\
\Avtors{Borisov~A.\,V.} Numerical schemes of markov jump process filtering given discretized observa-\linebreak
\\[-12pt]
\hspace*{23pt}tions~I:~Accuracy characteristics&4&68--75\\
\Avtors{Bosov~A.\,V.\ and Miller~G.\,B.} On the conditionally minimax nonlinear filtering concept\linebreak
\\[-12pt]
\hspace*{23pt}development: Filter modification and analysis&2&\hphantom{1}7--15\\
\Avtors{Bosov~A.\,V., Naumov~A.\,V., Mkhitaryan~G.\,A., and Sapunova~A.\,P.} Using the model of~gamma\linebreak
\\[-12pt]
\hspace*{23pt}distribution in~the~problem of~forming a~time-limited test in~a~distance learning system&4&11--17\\
\Avtors{Bosov~A.\,V.\ and Stefanovich~A.\,I.} Stochastic differential system output control by~the~quadratic\linebreak
\\[-12pt]
\hspace*{23pt}criterion. II.~Dynamic programming equations numerical solution&1&\hphantom{1}9--15\\
\Avtors{Bosov~A.\,V.\ and Stefanovich~A.\,I.} Stochastic differential system output control by~the~quadratic\linebreak
\\[-12pt]
\hspace*{23pt}criterion. III.~Optimal control properties analysis&3&41--49\\

\Avtors{Burlutskiy~V.\,V., Yakimchuk~A.\,V., Melnikov~A.\,V., Tsaregorodtsev~A.\,L., and Voloshin~S.\,V.} Development of a method for the formation of~attribute space and a~model for~the~assessment and prediction of anthropogenic influence on~the~environment (on~the~example of~the~forest fund of the~oil-producing region)&3& 131--136\\
\Avtors{Chistova~E.\,V.} see Smirnov~I.\,V.&&\\
\Avtors{Chukhno~N.\,V.} see Gaidamaka~A.\,A.&&\\
\Avtors{Chukhno~O.\,V.} see Gaidamaka~A.\,A.&&\\
\Avtors{Djukova~E.\,V., Maslyakov~G.\,O., and Prokofyev~P.\,A.} On the number of maximal independent\linebreak
\\[-12pt]
\hspace*{23pt}elements of~partially ordered sets (the case of~chains)&1&25--32\\
\Avtors{Dokukin~A.\,A.} see Zhuravlev~Yu.\,I.&&\\
\Avtors{Dulin~S.\,K., Dulina~N.\,G., and Kozhunova~O.\,S.} Synthesis of geodata in spatial infrastructures\linebreak
\\[-12pt]
\hspace*{23pt}based on related data&1&82--90\\
\Avtors{Dulina~N.\,G.} see Dulin~S.\,K.&&\\
\Avtors{Dzantiev~I.\,L.} see Markova~E.\,V.&&\\
\Avtors{Fedorov~S.\,N.} see Logachev~O.\,A.&&\\
\Avtors{Fedorov~S.\,N.} see Logachev~O.\,A.&&\\
\end{tabular}
}
\pagebreak

\def\leftfootline{\small{\textbf{\thepage}
\hfill INFORMATIKA I EE PRIMENENIYA~--- INFORMATICS AND APPLICATIONS\ \ \ 2019\
\ \ volume~13\ \ \ issue\ 4}
}%
 \def\rightfootline{\small{INFORMATIKA I EE PRIMENENIYA~---
INFORMATICS AND APPLICATIONS\ \ \ 2019\ \ \ volume~13\ \ \ issue\ 4
\hfill \textbf{\thepage}}}

\def\leftkol{2019 AUTHOR INDEX} % ENGLISH ABSTRACTS}

\def\rightkol{2019 AUTHOR INDEX} %ENGLISH ABSTRACTS}


\noindent
{\tabcolsep=3pt
\begin{tabular}{p{395.48108pt}cc}
&\textbf{Issue} & \textbf{Page}\\[6pt]
\Avtors{Flerov~Yu.\,A.} see Vyshinsky~L.\,L.&&\\
\Avtors{Gaidamaka~A.\,A., Chukhno~N.\,V., Chukhno~O.\,V., Samouylov~K.\,E., and Shorgin~S.\,Ya.} Formalization of the alternatives ranking method for group decision making in social net-\linebreak
\\[-12pt]
\hspace*{23pt}works&3&63--71\\
\Avtors{Gaidamaka~Yu.\,V.} see Gorbunova~A.\,V.&&\\
\Avtors{Golskaia~A.\,A.} see Markova~E.\,V.&&\\
\Avtors{Goncharov~A.\,A.\ and Inkova~O.\,Yu.} Methods for identification of implicit logical-semantic\linebreak
\\[-12pt]
\hspace*{23pt}relations in~texts&3&\hphantom{1}97--104\\
\Avtors{Goncharov~A.\,A., Zatsman~I.\,M., and Kruzhkov~M.\,G.} Temporal data in~lexicographic databases&4&90--96\\
\Avtors{Gorbunova~A.\,V., Naumov~V.\,A., Gaidamaka~Yu.\,V., and Samouylov~K.\,E.} Resource queuing\linebreak
\\[-12pt]
\hspace*{23pt}systems with general service discipline&1&\hphantom{1}99--107\\
\Avtors{Gorshenin~A.\,K.\ and Kuzmin~V.\,Yu.} Application of recurrent neural networks to~forecasting\linebreak
\\[-12pt]
\hspace*{23pt}the~moments of~finite normal mixtures&3&114--121\\
\Avtors{Gorshenin~A.\,K.\ and Kuzmin~V.\,Yu.} Optimization of hyperparameters of neural networks using\linebreak
\\[-12pt]
\hspace*{23pt}high-performance computing for prediction of precipitation&1&75--81\\
\Avtors{Gorshenin~A.\,K.\ and Martynov~O.\,P.} Hybrid extreme gradient boosting models to~impute\linebreak
\\[-12pt]
\hspace*{23pt}the~missing data in~precipitation records&3&34--40\\
\Avtors{Grabovoy~A.\,V., Bakhteev~O.\,Yu., and Strijov~V.\,V.} Estimation of the relevance of the neural\linebreak
\\[-12pt]
\hspace*{23pt}network parameters&2&62--70\\
\Avtors{Grinchenko~S.\,N.} On the genesis of the information society: Informatics-cybernetic model\linebreak
\\[-12pt]
\hspace*{23pt}representation&2&100--108\\
\Avtors{Grusho~A.\,A., Grusho~N.\,A., and Timonina~E.\,E.} Methods of identification of ``weak'' signs of\linebreak
\\[-12pt]
\hspace*{23pt}violations of information security&3&3--8\\
\Avtors{Grusho~A.\,A., Grusho~N.\,A., and Timonina~E.\,E.} Using metadata to~implement multilevel security\linebreak
\\[-12pt]
\hspace*{23pt}policy requirements&4&85--89\\
\Avtors{Grusho~A.\,A., Zabezhailo~M.\,I., Grusho~N.\,A., and Timonina~E.\,E.} Architectural decisions in the problem of identification of~fraud in~the~analysis of~information flows in~digital eco-\linebreak
\\[-12pt]
\hspace*{23pt}nomy&2&22--28\\
\Avtors{Grusho~A.\,A., Zabezhailo~M.\,I., Grusho~N.\,A., and Timonina~E.\,E.} Concepts forming on~the~basis\linebreak
\\[-12pt]
\hspace*{23pt}of~small samples&4&81--84\\
\Avtors{Grusho~N.\,A.} see Grusho~A.\,A.&&\\
\Avtors{Grusho~N.\,A.} see Grusho~A.\,A.&&\\
\Avtors{Grusho~N.\,A.} see Grusho~A.\,A.&&\\
\Avtors{Grusho~N.\,A.} see Grusho~A.\,A.&&\\
\Avtors{Gudkova~I.\,A.} see Markova~E.\,V.&&\\
\Avtors{Inkova~O.\,Yu.\ and Kruzhkov~M.\,G.} Compatibility of logical semantic relations: Methods\linebreak
\\[-12pt]
\hspace*{23pt}of~quantitative analysis&2&83--91\\
\Avtors{Inkova~O.\,Yu.} see Goncharov~A.\,A.&&\\
\Avtors{Kirikov~I.\,A.} see Rumovskaya~S.\,B.&&\\
\Avtors{Kiseleva~K.\,M.} see Zeifman~A.\,I.&&\\
\Avtors{Kolesnikov~A.\,V.\ and Listopad~S.\,V.} Heterogeneous thinking protocol of hybrid intelligent\linebreak
\\[-12pt]
\hspace*{23pt}multiagent system for~solving distributional power grid recovery problem&2&76--82\\
\Avtors{Konovalov~M.\,G.\ and Razumchik~R.\,V.} Mixed policies for~online job allocation in~one class\linebreak
\\[-12pt]
\hspace*{23pt}of~systems with~parallel service&4&54--59\\
\Avtors{Konovalov~M.\,G.} see Agalarov~Ya.\,M.&&\\
\Avtors{Korotkov~V.\,V.} see Zatsarinny~A.\,A.&&\\
\Avtors{Kovalev~D.\,Y.\ and Tarasov~E.\,A.} Virtual experiments in data intensive research&2&117--125\\
\Avtors{Kozhunova~O.\,S.} see Dulin~S.\,K.&&\\
\Avtors{Krivenko~M.\,P.} Data model selection in~medical diagnostic tasks&4&27--29\\
\Avtors{Kruzhkov~M.\,G.} see Goncharov~A.\,A.&&\\
\Avtors{Kruzhkov~M.\,G.} see Inkova~O.\,Yu.&&\\
\Avtors{Kudryavtsev~A.\,A.} \textit{A priori} generalized gamma distribution in Bayesian balance models&3&27--33\\
\Avtors{Kudryavtsev~A.\,A.} On the representation of gamma-exponential and~generalized negative\linebreak
\\[-12pt]
\hspace*{23pt}binomial distributions&4&76--80\\
\end{tabular}
}
\pagebreak

\def\leftfootline{\small{\textbf{\thepage}
\hfill INFORMATIKA I EE PRIMENENIYA~--- INFORMATICS AND APPLICATIONS\ \ \ 2019\
\ \ volume~13\ \ \ issue\ 4}
}%
 \def\rightfootline{\small{INFORMATIKA I EE PRIMENENIYA~---
INFORMATICS AND APPLICATIONS\ \ \ 2019\ \ \ volume~13\ \ \ issue\ 4
\hfill \textbf{\thepage}}}

\def\leftkol{2019 AUTHOR INDEX} % ENGLISH ABSTRACTS}

\def\rightkol{2019 AUTHOR INDEX} %ENGLISH ABSTRACTS}


\noindent
{\tabcolsep=3pt
\begin{tabular}{p{395.48108pt}cc}
&\textbf{Issue} & \textbf{Page}\\[6pt]
\Avtors{Kudryavtsev~A.\,A., Palionnaia~S.\,I., and Shorgin~V.\,S.} \textit{A priori} Frechet and~scaled inverse chi\linebreak
\\[-12pt]
\hspace*{23pt}distribution in~Bayesian balance models&1&62--66\\
\Avtors{Kudryavtsev~A.\,A.} see Arutyunov~E.\,N.&&\\
\Avtors{Kuryansky~M.\,K.} see Vyshinsky~L.\,L.&&\\
\Avtors{Kuzmin~V.\,Yu.} see Gorshenin~A.\,K.&&\\
\Avtors{Kuzmin~V.\,Yu.} see Gorshenin~A.\,K.&&\\
\Avtors{Lange~M.\,M.} On comparative efficiency of classification schemes in an ensemble of data\linebreak
\\[-12pt]
\hspace*{23pt}sources using average mutual information&4&18--26\\
\Avtors{Lebedev~A.\,V.} Nontransitive triplets of continuous random variables and their applications&3&20--26\\
\Avtors{Listopad~S.\,V.} see Kolesnikov~A.\,V.&&\\
\Avtors{Logachev~O.\,A., Sukayev~A.\,A., and Fedorov~S.\,N.} On local affinity based method of solving\linebreak
\\[-12pt]
\hspace*{23pt}systems of quadratic Boolean equations&2&37--46\\
\Avtors{Logachev~O.\,A., Sukayev~A.\,A., and Fedorov~S.\,N.} Polynomial algorithms for~constructing local\linebreak
\\[-12pt]
\hspace*{23pt}affinities of~quadratic Boolean functions&1&67--74\\
\Avtors{Lukashenko~O.\,V., Morozov~E.\,V., and Pagano~M.} A~Gaussian approximation of~the~distributed\linebreak
\\[-12pt]
\hspace*{23pt}computing process&2&109--116\\
\Avtors{Malashenko~Yu.\,E., Nazarova~I.\,A., and Novikova~N.\,M.} Vulnerability analysis of multipolar\linebreak
\\[-12pt]
\hspace*{23pt}networks after structural damages&1&33--39\\
\Avtors{Markova~E.\,V., Golskaia~A.\,A., Dzantiev~I.\,L., Gudkova~I.\,A., and Shorgin~S.\,Ya.} Comparative analysis of performance measures for a wireless machine-to-machine network model\linebreak
\\[-12pt]
\hspace*{23pt}operating within two radio resource management policies&1&108--116\\
\Avtors{Martynov~O.\,P.} see Gorshenin~A.\,K.&&\\
\Avtors{Maslyakov~G.\,O.} see Djukova~E.\,V.&&\\
\Avtors{Matveev~M.\,G.} see Zatsarinny~A.\,A.&&\\
\Avtors{Melnikov~A.\,V.} see Burlutskiy~V.\,V.&&\\
\Avtors{Meykhanadzhyan~L.\,A.\ and Razumchik~R.\,V.} Discrete-time $\mathrm{GEO}/G/1/\infty$ LIFO queue with\linebreak
\\[-12pt]
\hspace*{23pt}resampling policy&4&60--67\\
\Avtors{Miller~G.\,B.} see Bosov~A.\,V.&&\\
\Avtors{Mkhitaryan~G.\,A.} see Bosov~A.\,V.&&\\
\Avtors{Morozov~E.\,V.} see Lukashenko~O.\,V.&&\\
\Avtors{Naumov~A.\,V.} see Bosov~A.\,V.&&\\
\Avtors{Naumov~V.\,A.} see Gorbunova~A.\,V.&&\\
\Avtors{Nazarova~I.\,A.} see Malashenko~Yu.\,E.&&\\
\Avtors{Novikova~N.\,M.} see Malashenko~Yu.\,E.&&\\
\Avtors{Nuriev~V.\,A.} Architecture of a~machine translation system&3&90--96\\
\Avtors{Osipova~V.\,A.} see Abgaryan~K.\,K.&&\\
\Avtors{Pagano~M.} see Lukashenko~O.\,V.&&\\
\Avtors{Palionnaia~S.\,I.} see Kudryavtsev~A.\,A.&&\\
\Avtors{Panov~A.\,I.} see Smirnov~I.\,V.&&\\
\Avtors{Pavlov~Yu.\,L.} On the asymptotics of clustering coefficient in~a~configuration graph with unknown\linebreak
\\[-12pt]
\hspace*{23pt}distribution of~vertex degrees&3&\hphantom{1}9--13\\
\Avtors{Penkin~G.\,O.} see Anikeyev~D.\,A.&&\\
\Avtors{Prokofyev~P.\,A.} see Djukova~E.\,V.&&\\
\Avtors{Razumchik~R.\,V.} see Konovalov~M.\,G.&&\\
\Avtors{Razumchik~R.\,V.} see Meykhanadzhyan~L.\,A.&&\\
\Avtors{Rumovskaya~S.\,B.\ and Kirikov~I.\,A.} Methods of modeling and visual representation of~a~conflict\linebreak
\\[-12pt]
\hspace*{23pt}in~a~small collective of experts solving problems (review)&3&122--130\\
\Avtors{Ryazanov~V.\,V.} see Zhuravlev~Yu.\,I.&&\\
\Avtors{Rybakov~K.\,A.} On a class of filtering problems on~manifolds&1&16--24\\
\Avtors{Samouylov~K.\,E.} see Gaidamaka~A.\,A.&&\\
\Avtors{Samouylov~K.\,E.} see Gorbunova~A.\,V.&&\\
\Avtors{Sapunova~A.\,P.} see Bosov~A.\,V.&&\\
\Avtors{Satin~Y.\,A.} see Zeifman~A.\,I.&&\\
\end{tabular}
}
\pagebreak

\def\leftfootline{\small{\textbf{\thepage}
\hfill INFORMATIKA I EE PRIMENENIYA~--- INFORMATICS AND APPLICATIONS\ \ \ 2019\
\ \ volume~13\ \ \ issue\ 4}
}%
 \def\rightfootline{\small{INFORMATIKA I EE PRIMENENIYA~---
INFORMATICS AND APPLICATIONS\ \ \ 2019\ \ \ volume~13\ \ \ issue\ 4
\hfill \textbf{\thepage}}}

\def\leftkol{2019 AUTHOR INDEX} % ENGLISH ABSTRACTS}

\def\rightkol{2019 AUTHOR INDEX} %ENGLISH ABSTRACTS}


\noindent
{\tabcolsep=3pt
\begin{tabular}{p{395.48108pt}cc}
&\textbf{Issue} & \textbf{Page}\\[6pt]
\Avtors{Sen'ko~O.\,V.} see Zhuravlev~Yu.\,I.&&\\
\Avtors{Seyful-Mulyukov~R.\,B.} Understanding of~complex systems using~the~laws of~synergetics\linebreak
\\[-12pt]
\hspace*{23pt}and~informatics&4&107--113\\
\Avtors{Shestakov~O.\,V.} Inversion of homogeneous operators using stabilized hard thresholding with\linebreak
\\[-12pt]
\hspace*{23pt}unknown noise variance&1&49--54\\
\Avtors{Shestakov~O.\,V.} Properties of wavelet estimates of signals recorded at random time points&2&16--21\\
\Avtors{Shestakov~O.\,V.} The mean square risk of~nonlinear regularization in~the~problem of~inversion\linebreak
\\[-12pt]
\hspace*{23pt}of~linear homogeneous operators with~a~random sample size&4&48--53\\
\Avtors{Shnurkov~P.\,V.\ and Vakhtanov~N.\,A.} On the solution of the optimal control problem of inventory of~a~discrete product in~the~stochastic model of~regeneration with continuously\linebreak
\\[-12pt]
\hspace*{23pt}occuring consumption&3&50--57\\
\Avtors{Shnurkov~P.\,V.\ and Vakhtanov~N.\,A.} Research of the optimal control problem of~inventory of~a~discrete product in~the~stochastic regeneration model with continuously\linebreak
\\[-12pt]
\hspace*{23pt}occuring consumption and random delivery delay&2&54--61\\
\Avtors{Shorgin~S.\,Ya.} see Gaidamaka~A.\,A.&&\\
\Avtors{Shorgin~S.\,Ya.} see Markova~E.\,V.&&\\
\Avtors{Shorgin~V.\,S.} see Kudryavtsev~A.\,A.&&\\
\Avtors{Sinitsyn~I.\,N.} Interpolatonal analytical modeling in~complex stochastic systems&1&2--8\\
\Avtors{Skrynnik~A.\,A.} see Smirnov~I.\,V.&&\\
\Avtors{Smirnov~I.\,V., Panov~A.\,I., Skrynnik~A.\,A., and Chistova~E.\,V.} Personal cognitive assistant: \linebreak
\\[-12pt]
\hspace*{23pt}Concept and key principals&3&105--113\\
\Avtors{Stefanovich~A.\,I.} see Bosov~A.\,V.&&\\
\Avtors{Stefanovich~A.\,I.} see Bosov~A.\,V.&&\\
\Avtors{Strijov~V.\,V.} see Anikeyev~D.\,A.&&\\
\Avtors{Strijov~V.\,V.} see Grabovoy~A.\,V.&&\\
\Avtors{Suchkov~A.\,P.} The scientific result as~the~information object in~the~context of~the~scientific\linebreak
\\[-12pt]
\hspace*{23pt}services system management&3&137--144\\
\Avtors{Sukayev~A.\,A.} see Logachev~O.\,A.&&\\
\Avtors{Sukayev~A.\,A.} see Logachev~O.\,A.&&\\
\Avtors{Tarasov~E.\,A.} see Kovalev~D.\,Y.&&\\
\Avtors{Tarkhov~A.\,A.} see Zakharova~T.\,V.&&\\
\Avtors{Timonina~E.\,E.} see Grusho~A.\,A.&&\\
\Avtors{Timonina~E.\,E.} see Grusho~A.\,A.&&\\
\Avtors{Timonina~E.\,E.} see Grusho~A.\,A.&&\\
\Avtors{Timonina~E.\,E.} see Grusho~A.\,A.&&\\
\Avtors{Titova~A.\,I.} see Arutyunov~E.\,N.&&\\
\Avtors{Tsaregorodtsev~A.\,L.} see Burlutskiy~V.\,V.&&\\
\Avtors{Ushakov~N.\,G.} see Ushakov~V.\,G.&&\\
\Avtors{Ushakov~V.\,G.\ and Ushakov~N.\,G.} The output streams in~the~single server queueing system\linebreak
\\[-12pt]
\hspace*{23pt}with~a~head of~the~line priority&4&42--47\\
\Avtors{Ushakov~V.\,G.} see Agalarov~Ya.\,M.&&\\
\Avtors{Vakhtanov~N.\,A.} see Shnurkov~P.\,V.&&\\
\Avtors{Vakhtanov~N.\,A.} see Shnurkov~P.\,V.&&\\
\Avtors{Vinogradov~A.\,P.} see Zhuravlev~Yu.\,I.&&\\
\Avtors{Voloshin~S.\,V.} see Burlutskiy~V.\,V.&&\\
\Avtors{Vyshinsky~L.\,L., Kuryansky~M.\,K., and Flerov~Yu.\,A.} Digital model of the aircraft's weight\linebreak
\\[-12pt]
\hspace*{23pt}passport&4&\hphantom{1}3--10\\
\Avtors{Yakimchuk~A.\,V.} see Burlutskiy~V.\,V.&&\\
\Avtors{Zabezhailo~M.\,I.} see Grusho~A.\,A.&&\\
\Avtors{Zabezhailo~M.\,I.} see Grusho~A.\,A.&&\\
\Avtors{Zakharova~T.\,V.\ and Tarkhov~A.\,A.} Evaluation of the significance level in schuirmann's test for\linebreak
\\[-12pt]
\hspace*{23pt}checking the~bioequivalence hypothesis in~missing data conditions&3&58--62\\
\end{tabular}
}
\pagebreak

\def\leftfootline{\small{\textbf{\thepage}
\hfill INFORMATIKA I EE PRIMENENIYA~--- INFORMATICS AND APPLICATIONS\ \ \ 2019\
\ \ volume~13\ \ \ issue\ 4}
}%
 \def\rightfootline{\small{INFORMATIKA I EE PRIMENENIYA~---
INFORMATICS AND APPLICATIONS\ \ \ 2019\ \ \ volume~13\ \ \ issue\ 4
\hfill \textbf{\thepage}}}

\def\leftkol{2019 AUTHOR INDEX} % ENGLISH ABSTRACTS}

\def\rightkol{2019 AUTHOR INDEX} %ENGLISH ABSTRACTS}


\noindent
{\tabcolsep=3pt
\begin{tabular}{p{395.48108pt}cc}
&\textbf{Issue} & \textbf{Page}\\[6pt]
\Avtors{Zatsarinny~A.\,A., Korotkov~V.\,V., and Matveev~M.\,G.} Modeling the process of network planning\linebreak
\\[-12pt]
\hspace*{23pt}of~a~portfolio of~projects with heterogeneous resources under fuzziness&2&92--99\\
\Avtors{Zatsman~I.\,M.} Digital encoding of~concepts&4&\hphantom{1}97--106\\
\Avtors{Zatsman~I.\,M.} Goal-oriented development of~linguistic knowledge systems: Identifying and\linebreak
\\[-12pt]
\hspace*{23pt}filling of~lacunae&1&91--98\\
\Avtors{Zatsman~I.\,M.} Third-order interfaces in informatics&3&82--89\\
\Avtors{Zatsman~I.\,M.} see Goncharov~A.\,A.&&\\
\Avtors{Zeifman~A.\,I., Satin~Y.\,A., and Kiseleva~K.\,M.} On the bounds of the rate of convergence for\linebreak
\\[-12pt]
\hspace*{23pt}some queueing models with incompletely defined intensities&3&14--19\\
\Avtors{Zhuravlev~Yu.\,I., Sen'ko~O.\,V., Bondarenko~N.\,N., Ryazanov~V.\,V., Dokukin~A.\,A., and Vinogradov~A.\,P.} Research of~the~possibility to~forecast changes in~financial state of~a~credit\linebreak
\\[-12pt]
\hspace*{23pt}organization on~the~basis of~public financial statements&4&30--35\\
\end{tabular}
}

%\thispagestyle{myheadings}
\def\leftfootline{\small{\textbf{\thepage}
\hfill INFORMATIKA I EE PRIMENENIYA~--- INFORMATICS AND APPLICATIONS\ \ \ 2019\
\ \ volume~13\ \ \ issue\ 4}
}%
 \def\rightfootline{\small{INFORMATIKA I EE PRIMENENIYA~---
INFORMATICS AND APPLICATIONS\ \ \ 2019\ \ \ volume~13\ \ \ issue\ 4
\hfill \textbf{\thepage}}}

 \label{end\stat}

\newpage