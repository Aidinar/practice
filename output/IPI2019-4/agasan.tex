\def\stat{agasan}

\def\tit{ТЕОРЕТИЧЕСКИЕ ОСНОВЫ ОПТИМИЗАЦИИ ПО~КОНТИНУАЛЬНОМУ КРИТЕРИЮ 
VaR\\ НА~СОВОКУПНОСТИ  РЫНКОВ$^*$}

\def\titkol{Теоретические основы оптимизации по
континуальному критерию VaR на совокупности 
рынков}

\def\aut{Г.\,А.~Агасандян$^1$}

\def\autkol{Г.\,А.~Агасандян}

\titel{\tit}{\aut}{\autkol}{\titkol}

\index{Агасандян Г.\,А.}
\index{Agasandyan G.\,A.}


{\renewcommand{\thefootnote}{\fnsymbol{footnote}} \footnotetext[1]
{Работа выполнена при финансовой поддержке РФФИ (проект 17-01-00816).}}


\renewcommand{\thefootnote}{\arabic{footnote}}
\footnotetext[1]{Вычислительный центр им.~А.\,А.~Дородницына Федерального исследовательского 
центра <<Информатика и управление>> Российской академии наук, 
\mbox{agasand17@yandex.ru}}

\vspace*{-12pt}

  
  
  \Abst{Работа продолжает изучение проблем использования континуального критерия VaR 
(CC-VaR) на финансовых рынках. Речь идет о применении CC-VaR на совокупности 
нескольких рынков разных размерностей, связанных между собой базовыми активами. 
В~типовой модели совокупности одного двумерного и двух одномерных теоретических 
рынков рассматривается наиболее общий случай их совместного функционирования. 
Приводится правило построения оптимального по CC-VaR комбинированного портфеля 
с~тремя компонентами. Оно основывается на расхождениях в относительных доходах между 
рынками с~сохранением требований критерия. Оптимальный портфель строится из базисных 
инструментов всех рынков с~ использованием в их конструкциях идей рандомизации. 
Приводятся также его идеалистичная и суррогатная версии, которые могут быть полезными 
при проверке расчетов и для графической иллюстрации платежных функций. Теоретически 
модель без труда распространяется на рынки большей размерности. Возможны и две 
усеченные постановки задачи, в одной из которых исключается один одномерный рынок, 
в~другой~--- двумерный.}
  
  \KW{базовые активы; функция рисковых предпочтений; континуальный критерий VaR; 
стоимостная и прогнозная плотности; функция относительных доходов; процедура  
Ней\-ма\-на--Пир\-со\-на; комбинированный портфель; рандомизация; суррогатный 
портфель; идеалистичный портфель} 

\DOI{10.14357/19922264190406} 
  
\vspace*{-3pt}


\vskip 10pt plus 9pt minus 6pt

\thispagestyle{headings}

\begin{multicols}{2}

\label{st\stat}
  
  \section{Введение}
  
  Работа продолжает исследования по применению введенного автором 
континуального критерия VaR  на финансовых рынках~[1--5] как 
с~одним, так и~с~несколькими базовыми активами~\cite{3-ag}. В~настоящей 
работе изучается оптимальное поведение инвестора, приверженного CC-VaR, 
одновременно на нескольких рынках разных размерностей, связанных между 
собой базовыми активами. В~типовой модели речь идет о совокупности трех 
рынков, один из которых двумерный, а~два других одномерные. Базовые 
активы одномерных рынков образуют пару базовых активов двумерного. 
Такую схему назовем \textit{комбинированным} (и~\textit{тройственным}) 
рынком. Она естественным образом распространяется на совокупности рынков 
больших размерностей, хотя с~их ростом, разумеется, многое в схеме 
технически усложняется.
  
  Для выявления сущности проблемы все рынки рассматриваются 
однопериодными (с~двумя моментами времени~--- началом и концом периода), 
теоретическими (страйки опционов образуют континуальное множество) 
и~идеальными (комиссионные равны нулю, а цены покупателя и продавца 
совпадают). 
  
  Решение ищется в форме совмещения трех портфелей и основывается на 
анализе поточечных расхождений в относительных доходах на рынках, 
обусловленных расхождениями в ценах на разных \mbox{рынках}. 

\vspace*{-6pt}
  
  \section{Исходные теоретические рынки}
  
  \vspace*{-2pt}
  
  Рассматриваются три однопериодных рынка: два одномерных \#X и \#Y 
  с~базовыми активами~$\boldsymbol{X}$ и~$\boldsymbol{Y}$ соответственно 
и~один двумерный \#0 с~парой активов ($\boldsymbol{X}, \boldsymbol{Y}$). 
Цены двух базовых активов обозначаются~$x$ и~$y$, параметры 
инструментов~--- $s$ и~$t$, при этом $x, s \hm\in {\sf X}\hm = [a_1, b_1)$, $y, t 
\hm\in {\sf Y}\hm = [a_2, b_2)$. Рыночная стоимость произвольного 
инструмента~$\boldsymbol{G}$ записывается как~$\vert\boldsymbol{G}\vert$, 
а~средний доход~--- $\|\boldsymbol{G}\|$.
  
  Вкратце напомним обозначения и конструкции рынков с введением 
необходимых дополнений, обусловленных их совместной работой. 
  
  Для рынка \#X заданы \textit{прогнозная} $p_{\mathrm{X}}(x)$ 
и~\textit{стоимостная}~$c_{\mathrm{X}}(x)$, $x\hm\in {\sf X}$, плотности, 
порождающие меры ${\sf C}_{\mathrm{X}}(\cdot)$ и~${\sf 
P}_{\mathrm{X}}(\cdot)$ соответственно. Первая сформирована рынком на 
начало периода, а вторая дает прогноз инвестора на его конец. Важный для 
оптимизации относительный доход $\rho_{\mathrm{X}}(\cdot)\hm=  
p_{\mathrm{X}}(\cdot)/c_{\mathrm{X}}(\cdot)$. На рынке, называемом  
$\delta$-\textit{рын\-ком}, можно торговать любым 
инструментом~$\boldsymbol{G}_{\mathrm{X}}$ с доходом, представимым в виде 
произвольной неотрицательной измеримой функции~$g(x)$, $x\hm\in {\sf X}$. 
Ее называем \textit{платежной функцией} инструмента и обозначаем $\pi(x; 
\boldsymbol{G}_{\mathrm{X}})$, т.\,е.\ $g(x) \hm= \pi(x; 
\boldsymbol{G}_{\mathrm{X}})$, $x\hm\in {\sf X}$, в частности $\pi(x; 
\boldsymbol{X})\hm = x$. 
  
  Базисными на рынке являются инструменты 
$\boldsymbol{D}_{\mathrm{X}}(s)$, $s \hm\in {\sf  X}$, с обобщенной  
$\delta$-функ\-ци\-ей в качестве платежной: $\pi(x; 
\boldsymbol{D}_{\mathrm{X}}(s)) \hm\equiv  \delta(x \hm- s)$. Для них 
  $$\left\vert \boldsymbol{D}_{\mathrm{X}}(s)\right\vert 
=c_{\mathrm{X}}(s)\,,\enskip \left\| 
\boldsymbol{D}_{\mathrm{X}}(s)\right\| =p_{\mathrm{X}}(s)\,,\enskip s\in 
{\sf X}\,.
  $$
  
  Для инструмента $\boldsymbol{G}_{\mathrm{X}}$ с платежной 
функцией~$g(x)$ имеют место соотношения: 
  \begin{align*}
  \boldsymbol{G}_{\mathrm{X}} &= \int\limits_{{\sf X}\times{\sf Y}} 
g_{\mathrm{X}}(s) \boldsymbol{D}_{\mathrm{X}}(s)\,ds\,; \\
  \left\vert \boldsymbol{G}_{\mathrm{X}}\right\vert &=\int\limits_{\sf X} 
g_{\mathrm{X}}(s)c_{\mathrm{X}}(s)\,ds\,;\\
  \left\| \boldsymbol{G}_{\mathrm{X}}\right\| &=\int\limits_{\sf X} 
g_{\mathrm{X}}(s) p_{\mathrm{X}}(s)\,ds\,.
  \end{align*}
  
  Определяются и такие важные для сценарных рынков инструменты, как 
индикаторы множеств $\boldsymbol{H}_{\mathrm{X}}\{M\}$, $M \hm\subset 
{\sf X}$, с~их характеристическими функциями в качестве платежных, а также 
\textit{единичный безрисковый} актив~$\boldsymbol{U}_{\mathrm{X}}$, и~для 
них
  \begin{align*}
  \boldsymbol{H}_{\mathrm{X}}\{M\} &= \int\limits_M 
\boldsymbol{D}_{\mathrm{X}}(s)\,ds\,;\\  
\boldsymbol{U}_{\mathrm{X}} = 
\boldsymbol{H}_{\mathrm{X}}\{{\sf X}\}& =\int\limits_{\sf X} 
\boldsymbol{D}_{\mathrm{X}}(s)\,ds\,;\\
  \left\vert \boldsymbol{H}_{\mathrm{X}}\{M\}\right\vert &= \int\limits_M 
c_{\mathrm{X}}(s)\,ds\,;\\
  \left\vert 
\boldsymbol{U}_{\mathrm{X}}\right\vert
   = \boldsymbol{C}_{\mathrm{X}}\{{\sf X}\} &=\int\limits_{\sf X} 
c_{\mathrm{X}}(s)\,ds\,.
  \end{align*}
  
  Аналогично вводятся агрегаты второго одномерного рынка с очевидной 
заменой $\mathrm{X}\leftrightarrow\mathrm{Y}$, $x\leftrightarrow y$, ${\sf 
X}\leftrightarrow {\sf Y}$, $s\leftrightarrow t$: плотности $p_{\mathrm{Y}}(y)$ 
и~$c_{\mathrm{Y}}(y)$; меры ${\sf P}_{\mathrm{Y}}\{\cdot\}$ и~${\sf 
C}_{\mathrm{Y}}\{\cdot\}$; относительный доход $\rho_{\mathrm{Y}}(y)$; 
инструменты $\boldsymbol{D}_{\mathrm{Y}}(t)$, $y, t \hm\in {\sf Y}$, 
$\boldsymbol{H}_{\mathrm{Y}}\{M\}$, $M \hm\subset {\sf Y}$, 
и~$\boldsymbol{U}_{\mathrm{Y}}$.
  
  Для \textit{двумерного} рынка \#0 задаются двумерные плотности $p(x, y)$ 
и~$c(x, y)$, $x \hm\in {\sf  X}$, $y \hm\in {\sf  Y}$, по\-рож\-да\-ющие меры ${\sf 
P}\{\cdot,\cdot\}$ и~${\sf  C}\{\cdot,\cdot\}$ соответственно, а~относительный 
доход $\rho(\cdot,\cdot)\hm = p(\cdot,\cdot)/c(\cdot,\cdot)$. Базисными 
инструментами служат $\boldsymbol{D}(s, t)$, $s \hm\in{\sf X}$, $t\hm\in{\sf  
Y}$, и для них $\pi(x, y; \boldsymbol{D}(s, t))\hm\equiv \delta(x \hm- s, y \hm- t)$, 
а~также 
  $$
  \left\vert \boldsymbol{D}(s,t)\right\vert =c(s,t)\,,\enskip\!
  \left\| \boldsymbol{D}(s,t)\right\| =p(s,t)\,,\enskip\!
  s\in {\sf X}\,,\ t\in {\sf Y}\,.
  $$
  
  Для инструмента $\boldsymbol{G}$ с платежной функцией $\pi(x, y; 
\boldsymbol{G}) \hm\equiv g(x, y)$ имеем: 

\noindent
  \begin{align*}
  \boldsymbol{G} &= \int\limits_{{\sf X}\times{\sf Y}} g(s,t) 
\boldsymbol{D}(s,t)\,dsdt\,;\\
  \vert \boldsymbol{G}\vert &=\int\limits_{{\sf X}\times {\sf Y}} 
g(s,t)c(s,t)\,dsdt\,;\\
\|\boldsymbol{G}\| &=\int\limits_{{\sf 
X}\times{\sf Y}} g(s,t) p(s,t)\,dsdt\,.
  \end{align*}
  
\vspace*{-2pt}
  
  Для плотностей $p(x, y)$ и $c(x, y)$, $x \hm\in{\sf X}$, $y \hm\in{\sf  Y}$, 
выполняются соотношения:
  $$
  \int\limits_{{\sf X}\times{\sf Y}} p(x,y)\,dxdy=1\,;\enskip
  \int\limits_{{\sf X}\times {\sf Y}} c(x,y)\,dxdy=\fr{1}{r}\,,
  $$
где $r$~--- безрисковый относительный доход за период. Двумерные плотности 
порождают маргинальные плотности $p_1(x)$, $p_2(y)$ и $c_1(x)$, $c_2(y)$, $x 
\hm\in{\sf X}$, $y \hm\in{\sf  Y}$: 

\noindent
\begin{equation}
\left.
\begin{array}{rlrl}
p_1(x) &= \displaystyle\int\limits_{\sf Y} p(x,y)\,dy\,; & p_2(y)&=\displaystyle
 \int\limits_{\sf X} 
p(x,y)\,dx\,;\\[6pt]
     c_1(x) &= \displaystyle\int\limits_{\sf Y} c(x,y)\,dy\,; & c_2(y)&= 
     \displaystyle\int\limits_{\sf X} c(x,y)\,dx\,.
     \end{array}
     \right\}
     \label{e1-ag}
     \end{equation}
     
     \vspace*{-2pt}
  
 \noindent
  При этом
  
  \noindent
  \begin{align*}
  \int\limits_{\sf X} p_1(x)\,dx=\int\limits_{\sf Y} p_2(y)\,dy=\int\limits_{{\sf 
X}\times{\sf Y}} p(x,y)\,dxdy=1\,;\\
  \int\limits_{\sf X} c_1(x)\,dx=\int\limits_{\sf Y} c_2(y)\,dy=\int\limits_{{\sf 
X}\times{\sf Y}} c(x,y)\,dxdy=\fr{1}{r}\,.
  \end{align*}
  
  Вновь, как обычно, без ограничения общности принимаем для простоты 
$r\hm = 1$ для рынка~\#0, что позволяет интерпретировать 
\textit{стоимостную} плотность $c(x, y)$, $x \hm\in{\sf X}$, $y \hm\in{\sf  Y}$, 
как плотность вероятности, порождаемую рынком. 
  
  Однако распространить такое же упрощение на все рынки представленной 
совокупности нельзя, так как на них могут возникать свои безрисковые ставки 
относительного дохода. И появляются новые \textit{параметры} 
$\chi_{\mathrm{X}}$ и~$\chi_{\mathrm{Y}}$~--- ставки безрискового 
относительного дохода на рынках~\#X и \#Y соответственно, дающие только 
интегральные ограничения на одномерные стоимостные плотности: 
  \begin{equation}
  \left.
  \begin{array}{rl}
  \left\vert \boldsymbol{U}_{\mathrm{X}}\right\vert &=\int\limits_{\sf X} 
c_{\mathrm{X}} (x)\,dx=\chi^{-1}_{\mathrm{X}}\,;\\[6pt] 
  \left\vert \boldsymbol{U}_{\mathrm{Y}}\right\vert &=\int\limits_{\sf Y} 
c_{\mathrm{Y}} (y)\,dy=\chi^{-1}_{\mathrm{Y}}\,.
\end{array}
\right\}
  \label{e2-ag}
  \end{equation}
  
  Поскольку в общем случае ценообразование на совместно 
функционирующих трех рынках (как на самостоятельных, хотя и родственных) 
производит-\linebreak\vspace*{-12pt}

\pagebreak

\noindent
ся раздельно, стоимостные плотности $c_{\mathrm{X}}(\cdot)$ 
и~$c_{\mathrm{Y}}(\cdot)$ одномерных рынков~\#X и \#Y не следует 
отождествлять с маргинальными~(1). И,~вообще говоря, 
  $$
  c_{\mathrm{X}}(x)\not= c_1(x)\,,\ c_{\mathrm{Y}}(y)\not= c_2(y)\,,\ x\in {\sf 
X}\,,\ y\in {\sf Y}\,,
  $$
  хотя при этом естественно считать, что 
  $$
  p_{\mathrm{X}}(x)\equiv p_1(x)\,,\ p_{\mathrm{Y}}(y)\equiv p_2(y)\,,\ x\in{\sf 
X}\,,\ y\in{\sf Y}\,,
  $$
так как все прогнозные плотности $p(x, y)$, $p_{\mathrm{X}}(x)$ 
и~$p_{\mathrm{Y}}(y)$ являются предметом единого цельного прогноза инвестора. 

  Для рынка \#0 определяются также инструментальные \textit{индикаторы} 
множеств $\boldsymbol{H}\{M\}$, $M \hm\subset {\sf X}\times{\sf Y}$, 
и~\textit{единичный безрисковый} актив~$\boldsymbol{U}$, и для них 
  \begin{align*}
  \boldsymbol{H}\{M\} &=\int\limits_M \boldsymbol{D}(s,t)\,dsdt\,;\\
  \boldsymbol{U}=\boldsymbol{H}\{{\sf X}\times {\sf Y}\} &=\int\limits_{{\sf 
X}\times{\sf Y}} \boldsymbol{D}(s,t)\,dsdt\,;\\
  \left\vert \boldsymbol{H}\{M\}\right\vert
  & =\int\limits_M c(s,t)\,dsdt\,;\\
     \vert \boldsymbol{U}\vert ={\sf C}\{{\sf X}\times {\sf Y}\}& =\int\limits_{{\sf 
X}\times{\sf Y}} c(s,t)\,dsdt =\fr{1}{r}\,.
  \end{align*}
  
  Наряду с введенными инструментами рынка~\#0 рассматриваются и его 
\textit{маргинальные} инструменты $\boldsymbol{D}_1(\cdot)$, 
$\boldsymbol{D}_2(\cdot)$, $\boldsymbol{U}_1$, $\boldsymbol{U}_2$, 
$\boldsymbol{H}_1\{\cdot\}$ и~$\boldsymbol{H}_2\{\cdot\}$, но каждый из них не 
самостоятелен и обретает смысл лишь в~\textit{произведении}  
с~ка\-ким-ли\-бо инструментом по другой координате.
  
  Наконец, критерий CC-VaR требует, чтобы выполнялись неравенства 
   ${\sf P}\{q\geq\phi(\varepsilon)\}\hm\geq 1\hm-\varepsilon$ сразу для \textit{всех} 
$\varepsilon \hm\in  [0, 1]$,
где $q$~--- доход инвестора; $\phi(\varepsilon)$~--- неотрицательная монотонно 
возрастающая и непрерывная \textit{функция рисковых предпочтений} (ф.р.п.)\ 
инвестора. 

  В связи с соотношениями~(2) следует также иметь в виду проблемы 
взаимодействия рынков.\linebreak Обмен между рынками инструментальными 
средствами не предусмотрен. Это значит, что не допус\-ти\-мо, например, 
расщепление двумерного единичного безрискового актива на два 
компонентных\linebreak инструмента с~целью последующих операций с~ними на двух 
других одномерных рынках. Естественно, что при этом сохраняется 
возможность использования денежных средств, полученных от продажи актива 
на двумерном рынке, для покупки других активов на одномерных рынках. 
Проясним на простейшем примере, как сказываются такие особенности 
многомерных рынков на исходах сделок. 
  
  Рассмотрим последовательность двух рыночных сделок: 
  \begin{enumerate}[(1)]
  \item продажа единицы инструмента $\boldsymbol{U}\hm= 
\boldsymbol{U}_1\times \boldsymbol{U}_2$ на рынке~\#0 по цене $S \hm= 1$; 
  \item приобретение на сумму~$S$ на рынках~\#X и \#Y по отдельности~$u$ 
и~$v$~единиц инструментов $\boldsymbol{U}_{\mathrm{X}}$ 
и~$\boldsymbol{U}_{\mathrm{Y}}$ по ценам $1/\chi_{\mathrm{X}}$ 
и~$1/\chi_{\mathrm{Y}}$~(2) для каждой единицы соответственно. 
  \end{enumerate}
  
  Для определения количеств~$u$ и~$v$ имеем уравнение $S \hm= 
u/\chi_{\mathrm{X}}\hm + v/\chi_{\mathrm{Y}}$. Во вполне приемлемом 
предположении, что $\chi_{\mathrm{X}}\hm = \chi_{\mathrm{Y}}\hm = 1$, одним из 
его решений будет, например, $u \hm= v \hm= 1/2$. Таким образом, в этом 
случае один двумерный инструмент~$\boldsymbol{U}$ эквивалентен по 
стоимости комбинации $\boldsymbol{U}_{\mathrm{X}}/2\hm+ 
\boldsymbol{U}_{\mathrm{Y}}/2$ (а~не $\boldsymbol{U}_{\mathrm{X}}\hm + 
\boldsymbol{U}_{\mathrm{Y}}$!). 
  
  \section{Оптимизация на~тройственном рынке}
  
  Предлагаются алгоритмы построения на совокупности трех теоретических 
рынков оптимального по CC-VaR комбинированного портфеля вместе 
с~некоторыми его версиями. Алгоритмы, как и~в~[1--5], основываются на 
анализе относительных доходов для всех трех исходных рынков 
с~континуальным применением процедуры Ней\-ма\-на--Пир\-со\-на из 
математической статистики~\cite{6-ag}. 
  
  В общей схеме тройственного рынка для целей оптимизации будем 
формировать единую функцию относительного дохода для комбинации трех 
рынков. Это производится путем поточечной замены значений 
функции~$\rho(\cdot,\cdot)$ рынка~\#0 ровно теми значениями 
функций~$\rho_{\mathrm{X}}(\cdot)$ или~$\rho_{\mathrm{Y}}(\cdot)$ для 
рынков~\#X и~\#Y (с~сопоставимыми по вероятностям весами), которые 
оказываются наибольшими из всех трех функций. 
  
  Формально правила замещения задаются разбиением множества ${\sf 
X}\times {\sf Y}$ на подмножества~$M_0$, $M_1$ и~$M_2$, определяемые 
соотношениями эквивалентности: 
  \begin{align}
  (s,t)\in M_0 &\Leftrightarrow \left\{ \rho(s,t)\geq 
\rho_{\mathrm{X}}(s)\&\rho(s,t)\geq \rho_{\mathrm{Y}}(t)\right\};\!\!
  \label{e3-ag}\\
  (s,t)\in M_1 &\Leftrightarrow \left\{ \rho_{\mathrm{X}}(s)> 
\rho(s,t)\&\rho_{\mathrm{X}}(s)\geq \rho_{\mathrm{Y}}(t)\right\};\!\!
  \label{e4-ag}\\
  (s,t)\in M_2 &\Leftrightarrow \left\{ \rho_{\mathrm{Y}}(t)> 
\rho(s,t)\&\rho_{\mathrm{Y}}(t)> \rho_{\mathrm{X}}(s)\right\}.\!\!
  \label{e5-ag}
  \end{align}
  
  Множества $M_0$, $M_1$ и~$M_2$ взаимно не пересекаются, в объединении 
дают полное множество ${\sf X}\times{\sf Y}$ и состоят из тех и только тех пар 
$(s,t) \hm\in {\sf X}\times{\sf Y}$, для которых максимальным является 
относительный доход соответственно $\rho(s, t)$, $\rho_{\mathrm{X}}(s)$ 
и~$\rho_{\mathrm{Y}}(t)$. В~случае равенства этих доходов приоритет 
в~отношении принадлежности множеству устанавливается в порядке 
рынков~\#0, \#X и~\#Y. 
  
  Результат классификации~(\ref{e3-ag})--(\ref{e5-ag}) можно записывать 
посредством принимающей всего три значения \textit{функции замещений} (для 
всех $s \hm\in{\sf X}$, $t \hm\in {\sf Y}$):
  \begin{equation}
  A(s,t)=k\Leftrightarrow (s,t)\in M_k\,,\enskip k=0,1,2\,.
  \label{e6-ag}
  \end{equation}
    Она просто помечает все точки множеств $M_0$, $M_1$ и~$M_2$ их 
индексами~--- цифрами~0, 1 и~2 соответственно. 
  
  Обозначим через $M_{1;s}(\subset {\sf Y})$ и $M_{2;t}(\subset {\sf X})$ 
сечения множеств $M_{1}$~(\ref{e4-ag}) и~$M_2$~(\ref{e5-ag}) для фиксированных 
значений $s\hm\in{\sf X}$ и $t\hm\in {\sf Y}$ соответственно: 
  $$
  M_1=\bigcup\limits_{s\in{\sf X}} M_{1;s}\,;\quad M_2=\bigcup_{t\in {\sf Y}} 
M_{2;t}\,.
  $$
  
  Рассмотрим индикатор $\boldsymbol{M}_1(s)$, $s \hm\in{\sf X}$, рынка~\#0 
как объединение базисных инструментов $\boldsymbol{D}(s, t)$ по $t \hm\in 
M_{1;s}$: 
  \begin{equation}
  \boldsymbol{M}_1(s)=\int\limits_{M_{1;s}} \boldsymbol{D}(s,t)\,dt 
=\boldsymbol{D}_1(s)\times \boldsymbol{H}_2\left\{ M_{1;s}\right\}\,.
  \label{e7-ag}
  \end{equation}
    Это $\delta$-ин\-стру\-мент на~${\sf X}$ и индикатор 
множества~$M_{1;s}$ на~${\sf Y}$, и для него 
  $$
  \left\vert \boldsymbol{M}_1(s)\right\vert =\int\limits_{M_{1;s}} 
c(s,t)\,dt\,;\enskip
  \left\| \boldsymbol{M}_1(s)\right\| =\int\limits_{M_{1;s}} p(s,t)\,dt\,.
  $$
  
  Его двумерную платежную функцию можно представить в виде 
произведения $\delta(x \hm- s)$, $x \hm\in {\sf X}$, на характеристическую 
функцию множества~$M_{1;s}$ по $y \hm\in {\sf  Y}$. Она сингулярна по~$x$ 
и~конечна по~$y$. 
  
  Индикаторы $\boldsymbol{M}_1(s)$~(\ref{e7-ag}) для каждого  $s \hm \in   
{\sf  X}$ являются ровно теми инструментами рынка~\#0, которые согласно 
условиям~(\ref{e4-ag}) для относительных доходов следовало бы заместить 
инструментами~$\boldsymbol{D}_{\mathrm{X}}(s)$ рынка~\#X. Однако действие 
инструмента~$\boldsymbol{D}_{\mathrm{X}}(s)$ распространяется на полное 
множество~${\sf Y}$, а~не только на его подмножество~$M_{1;s}$. Поэтому 
подобное замещение должно быть ограниченным, и инструменты~(\ref{e7-ag}) 
желательно было бы заместить совмещающими рынки~\#0 и~\#X 
<<гибридными>> инструментами: 
  \begin{equation}
  \boldsymbol{M}_{\mathrm{X}}(s)\equiv 
\boldsymbol{D}_{\mathrm{X}}(s)\times \boldsymbol{H}_2\left\{ 
M_{1;s}\right\}\,,\enskip s\in {\sf X}\,.
  \label{e8-ag}
  \end{equation}
  
  Но таких инструментов нет ни на одном из рассматриваемых рынков. Тем не 
менее рыночную реализацию такого замещения можно осуществить, если 
воспользоваться услугами \textit{рандомизации}. Это делается следующим 
образом. 
  
  Вводятся биномиальные случайные величины $\vartheta_{\mathrm{X}}(s)$, 
$s\hm\in {\sf  X}$, с вероятностью успеха (замещения)~$\theta_{\mathrm{X};s}$, 
равной условной вероятности 
  \begin{equation}
  \theta_{\mathrm{X};s} ={\sf P}\left\{ M_{1;s}\vert \mathrm{X}=s\right\} 
=\int\limits_{M_{1;s}} \fr{p(s,t)\,dt}{p_1(s)}\,,\enskip  s\in {\sf X}\,.
  \label{e9-ag}
  \end{equation}
Эти вероятности служат в модели параметрами рандомизации. 
  
  В соответствии с предположениями о вероятностях и ценообразовании для 
инструментов $\boldsymbol{M}_{\mathrm{X}}(s)$ должны были бы выполняться 
равенства: 

\vspace*{2pt}

\noindent
  \begin{equation}
  \left.
  \begin{array}{rl}
  \hspace*{-3mm}\left\vert \boldsymbol{M}_{\mathrm{X}}(s)\right\vert &=\theta_{\mathrm{X};s} 
c_{\mathrm{X}}(s);\\[6pt]
    \hspace*{-3mm}\left\| \boldsymbol{M}_{\mathrm{X}}(s)\right\| 
&=\theta_{\mathrm{X};s} p_{\mathrm{X}}(s)\,,\
  \rho_{\mathrm{X}}(s)=\fr{c_{\mathrm{X}}(s)}{p_{\mathrm{X}}(s)}\,,\ s\in{\sf 
X}.
\end{array}\!
\right\}\!
  \label{e10-ag}
  \end{equation}
  
  В качестве базисных для части~\#X комбинированного портфеля 
предлагается использовать рандомизированные инструменты: 
  \begin{equation}
  \boldsymbol{D}_{\mathrm{X}}^{\mathrm{cmb}}(s) =\vartheta_{\mathrm{X}}(s) 
\boldsymbol{D}_{\mathrm{X}}(s)\,,\enskip s\in{\sf X}\,.
  \label{e11-ag}
  \end{equation}
    Эти инструменты являются случайными, принимающими облик инструмента 
$\boldsymbol{D}_{\mathrm{X}}(s)$ с~ве\-ро\-ят\-ностью~$\theta_{\mathrm{X};s}$ 
и~\textit{нулевого} инструмента~$\boldsymbol{N}_{\mathrm{X}}(s)$\linebreak 
(с~тож\-де\-ст\-вен\-но равным нулю доходом и нулевой сто\-и\-мостью) с вероятностью 
$1\hm- \theta_{\mathrm{X};s}$, $s \hm\in {\sf  X}$. 
  
  Их средние цены и средние доходы (ве\-ро\-ят\-ности) соответственно 
  
  \vspace*{2pt}
  
  \noindent
  \begin{equation}
  \left.
  \begin{array}{rl}
  \left\vert \boldsymbol{D}^{\mathrm{cmb}}_{\mathrm{X}}(s)\right\vert &=
\theta_{\mathrm{X};s} c_{\mathrm{X}}(s)\,;\\[6pt]
  \left\| \boldsymbol{D}^{\mathrm{cmb}}_{\mathrm{X}}(s)\right\| 
&=\displaystyle \theta_{\mathrm{X};s} p_1(s)=\int\limits_{M_{1;s}} p(s,y)\,dy\,.
\end{array}
\right\}
  \label{e12-ag}
  \end{equation}
  
  Выбор~(\ref{e9-ag}) параметров~$\theta_{\mathrm{X};s}$ уравнивает 
вероятности, связанные 
с~инструментами~$\boldsymbol{M}_{\mathrm{X}}(s)$~(\ref{e8-ag}) 
и~$\boldsymbol{M}_1(s)$~(\ref{e7-ag}), поскольку вероятности, с которыми на 
рынке~\#0 инструменты~$\boldsymbol{M}_1(s)$ порождают ненулевой (именно 
единичный) доход, определяются плотностью $p(s, t)$ и~вторым соотношением 
в~(\ref{e10-ag}). 
  
  Свойства~(\ref{e12-ag}) инструментов~(\ref{e11-ag}) позволяют назначить 
их, несмотря на составную структуру, новыми цельными базисными 
инструментами комбинированного рынка, фактически реплицирующими 
инструменты~$\boldsymbol{M}_{\mathrm{X}}(s)$. 
  
  Подобные конструкции, введенные для рынка~\#X, в полной мере 
распространяются на рынок~\#Y. При этом они получаются из 
со\-от\-вет\-ст\-ву\-ющих аналогов рынка~\#X заменой $1\leftrightarrow2$, 
$s\leftrightarrow t$, $i\leftrightarrow j$, $\mathrm{X}\leftrightarrow\mathrm{Y}$. 
Так определяются уже связанные с~множеством $M_2$~(\ref{e5-ag}) замещения 
инструменты~$\boldsymbol{D}_{\mathrm{Y}}(t)$, случайные 
величины~$\vartheta_{\mathrm{Y}}(t)$ с~па\-ра\-мет\-ра\-ми~$\theta_{\mathrm{Y};t}$ 
успеха и~рандомизированные базисные инструменты 

\noindent
  \begin{multline}
  \boldsymbol{D}^{\mathrm{cmb}}_{\mathrm{Y}}(t)=\vartheta_{\mathrm{Y}}(t) 
\boldsymbol{D}_{\mathrm{Y}}(t)\,,\
  \theta_{\mathrm{Y};t} ={\sf P}\left\{ M_{2;t}\vert \mathrm{Y}=t\right\} 
={}\\
{}=\int\limits_{M_{2;t}} \fr{p(s,t)\,dt}{p_2(t)}\,,\enskip t\in{\sf Y}\,.
  \label{e13-ag}
  \end{multline}
  
  Инструменты~(\ref{e11-ag}) и~(\ref{e13-ag}) на рынках~\#X и~\#Y со своими 
ценами и средними доходами~(\ref{e12-ag}) вместе\linebreak\vspace*{-12pt}

\pagebreak

\noindent 
с~инструментами~$\boldsymbol{D}(\cdot,\cdot)$ на рынке~\#0 
с~плотностями~$c(\cdot,\cdot)$ и $p(\cdot,\cdot)$ на множестве~$M_0$ 
образуют полный \textit{комбинированный} базис. 
  
  Для этого базиса формируется единая функция \textit{относительных 
доходов}, и~к~ней применяется общий теоретический алгоритм оптимизации. 
В~результате его работы с новой функцией относительных доходов 
производится новое назначение всех вероятностей и~строится новая весовая 
функция базисных инструментов. Оптимальный \textit{комбинированный} 
портфель вследствие случайности величин $\vartheta_{\mathrm{X}}(s)$ 
и~$\vartheta_{\mathrm{Y}}(t)$ оказывается в итоге случайным и приобретает вид: 

\noindent
  \begin{multline*}
  \boldsymbol{G}^{\mathrm{cmb}}=\int\limits_{M_0} g^{\mathrm{cmb}}(s,t) 
\boldsymbol{D}(s,t)\,ds dt+{}\\
  {}+\int\limits_{\sf X}\! \!g^{\mathrm{cmb}}_{\mathrm{X}}(s) \vartheta_{\mathrm{X}}(s) 
\boldsymbol{D}_{\mathrm{X}}(s)\,ds\,+\!\int\limits_{\sf Y} \!
g^{\mathrm{cmb}}_{\mathrm{Y}}(t) \vartheta_{\mathrm{Y}}(t) 
\boldsymbol{D}_{\mathrm{Y}}(t)\,dt.\hspace*{-7.25882pt}
 % \label{e14-ag}
  \end{multline*}
  
  Нелишне рассмотреть и упрощенную, хотя и нереализуемую на 
тройственном рынке, \textit{идеалистичную} версию портфеля в эквивалентной 
по платежной функции и ценам форме двумерного портфеля с теми же весами: 

\noindent
  \begin{multline}
  \boldsymbol{G}^{\mathrm{idl}}=\int\limits_{M_0} g^{\mathrm{cmb}}(s,t) 
\boldsymbol{D}(s,t)\,dsdt+{}\\
{}+\int\limits_{\sf X} 
g^{\mathrm{cmb}}_{\mathrm{X}}(s)\boldsymbol{M}_{\mathrm{X}}(s)\,ds+
  \int\limits_{\sf Y} 
g^{\mathrm{cmb}}_{\mathrm{Y}}(t)\boldsymbol{M}_{\mathrm{Y}}(t)\,dt\,.
  \label{e15-ag}
  \end{multline}
  
  При всей условности такого представления его можно использовать для 
графической иллюстрации платежной функции в виде единой двумерной 
функции:
  \begin{equation*}
  \pi\left( x,y;\boldsymbol{G}^{\mathrm{idl}}\right)= \max \left( g^{\mathrm{cmb}}(x,y), 
g^{\mathrm{cmb}}_{\mathrm{X}}(x), g^{\mathrm{cmb}}_{\mathrm{Y}}(y)\right)\,.
 % \label{e16-ag}
  \end{equation*}
  
  Наряду с комбинированным можно построить и~портфель, который назовем 
\textit{суррогатным}. Он получается в результате формальной 
\textit{поточечной} замены базисных инструментов $\boldsymbol{D}(s,t)$ 
рынка~\#0 инструментами $\boldsymbol{D}^{\mathrm{srg}}(s, t)$, $s \hm\in {\sf X}$, $t 
\hm\in {\sf Y}$, с~теми же платежными функциями и~вероятностями, но 
с~ценами, скорректированными в~соответствии с~правилами 
замещения~(\ref{e3-ag})--(\ref{e5-ag}) и~с~учетом цен рынков~\#X и~\#Y. Для 
всех $s\hm\in {\sf X}$ и~$t\hm\in {\sf Y}$ и~при $A(s, t)\hm = 0$, 1, 2 
(см.~(\ref{e6-ag})) соответственно 
  $$
  \left\vert \boldsymbol{D}^{\mathrm{srg}}(s,t)\right\vert =c^{\mathrm{srg}} (s,t) =c(s,t), 
\fr{p(s,t)}{p_{\mathrm{X}}(s)}\,, \fr{p(s,t)}{p_{\mathrm{Y}}(t)}\,.
  $$

  
  Далее вновь образуется функция относительных доходов, и на ее основе 
алгоритм находит весовую функцию портфеля $g^{\mathrm{srg}}(s, t)$, $s \hm\in {\sf X}$, 
$t \hm\in {\sf Y}$. И~тогда

\noindent
  $$
  \boldsymbol{G}^{\mathrm{srg}}=\int\limits_{\sf X} \int\limits_{\sf Y} g^{\mathrm{srg}}(s,t) 
\boldsymbol{D}^{\mathrm{srg}}(s,t)\,dsdt\,.
  $$
  %
  Суррогатный портфель, как и портфель~(\ref{e15-ag}), не реализуем на 
рассматриваемом рынке, но ввиду своей простоты вполне может служить 
средством проверки правильности алгоритма в его дискретной версии, тем 
более по графикам доходов.

\vspace*{-6pt}
  
  \section{Заключение}
  
  В работе предложен подход к оптимизации поведения инвестора, 
придерживающегося CC-VaR, на совокупности финансовых рынков разной 
размерности. Изложение ведется для теоретических рынков, на которых 
базисными служат $\delta$-ин\-стру\-мен\-ты. Для целей оптимизации 
приводится правило замещения базисных инструментов двумерного рынка 
более доходными базисными инструментами двух одномерных 
с~использованием механизма рандомизации. Предлагается способ построения 
оптимального комбинированного портфеля из базисных инструментов всех 
рынков вместе с его идеалистичной и суррогатной версиями. Для проверки 
действенности модели и всех ее компонентов необходимо дополнительно 
адаптировать построенные теоретические конструкции к дискретным 
сценарным рынкам, рассмотреть характерные примеры с~проведением 
численных расчетов и демонстрацией результатов на графиках. 

\vspace*{-6pt}
  
{\small\frenchspacing
 {%\baselineskip=10.8pt
 \addcontentsline{toc}{section}{References}
 \begin{thebibliography}{9}
  \bibitem{1-ag}
  \Au{Agasandian G.\,A.} Optimal behavior of an investor in option market~//  
Joint Conference (International) on Neural Networks Proceedings.~--- 
 IEEE, 2002. P.~1859--1864. 
  \bibitem{2-ag}
  \Au{Агасандян Г.\,А.} Применение континуального критерия VaR на 
финансовых рынках.~--- М.: ВЦ РАН, 2011. 299~с. 
  \bibitem{3-ag}
  \Au{Агасандян Г.\,А.} Континуальный критерий VaR на многомерных рынках 
опционов.~--- М.: ВЦ РАН, 2015. 297~с. 
  \bibitem{4-ag}
  \Au{Агасандян Г.\,А.} Континуальный критерий VaR на сценарных рынках~// 
Информатика и её применения, 2018. Т.~12. Вып.~1. С.~32--40. 
  \bibitem{5-ag}
  \Au{Агасандян Г.\,А.} Континуальный критерий VaR и оптимальный 
портфель инвестора~// Управление большими системами, 2018. Вып.~73.  
С.~6--26.
  \bibitem{6-ag}
  \Au{Крамер Г.} Математические методы статистики~/ Пер. с~англ.~--- М.: 
Мир, 1975. 750~с. (\Au{Cramer~H.} Mathematical methods of statistics.~--- 
Princeton, NJ, USA: Princeton University Press, 1946. 575~p.)
 \end{thebibliography}

 }
 }

\end{multicols}

\vspace*{-7pt}

\hfill{\small\textit{Поступила в~редакцию 27.03.19}}

%\vspace*{8pt}

%\pagebreak

\newpage

\vspace*{-28pt}

%\hrule

%\vspace*{2pt}

%\hrule

%\vspace*{-2pt}

\def\tit{THEORETICAL FOUNDATIONS OF~CONTINUOUS VaR CRITERION OPTIMIZATION 
IN~THE~COLLECTION OF~MARKETS}


\def\titkol{Theoretical foundations of~continuous VaR criterion  optimization 
in~the~collection of~markets}

\def\aut{G.\,A.~Agasandyan}

\def\autkol{G.\,A.~Agasandyan}

\titel{\tit}{\aut}{\autkol}{\titkol}

\vspace*{-11pt}


\noindent
  A.\,A.~Dorodnicyn Computing Center, Federal Research Center ``Computer 
Science and Control'' of the Russian Academy of Sciences, 40~Vavilov Str., Moscow 
119333, Russian Federation

\def\leftfootline{\small{\textbf{\thepage}
\hfill INFORMATIKA I EE PRIMENENIYA~--- INFORMATICS AND
APPLICATIONS\ \ \ 2019\ \ \ volume~13\ \ \ issue\ 4}
}%
 \def\rightfootline{\small{INFORMATIKA I EE PRIMENENIYA~---
INFORMATICS AND APPLICATIONS\ \ \ 2019\ \ \ volume~13\ \ \ issue\ 4
\hfill \textbf{\thepage}}}

\vspace*{3pt} 
  
  
   
  \Abste{The work continues studying the problems of using continuous 
  VaR criterion (CC-VaR) in 
financial markets. The application of CC-VaR in a collection of theoretical markets of different 
dimensions that are mutually connected by their underliers is concerned. In a typical model of the 
collection of one two-dimensional market and two one-dimensional markets, the most general case 
of their conjoint functioning is considered. The rule of constructing a combined portfolio optimal on 
CC-VaR in these markets is submitted. This rule is founded on misbalance in returns relative 
between markets with maintaining optimality on CC-VaR. The optimal combined portfolio with 
three components is constructed from basis instruments of all markets and by using ideas of 
randomization in their composition. Also, the idealistic and surrogate versions of this combined 
portfolio, which are useful in testing all algorithmic calculations and in graphic illustrating 
portfolio's payoff functions, are adduced. The model can be extended without academic difficulties 
onto markets of greater dimensions. Also, two truncated variants of problem setting with excluded 
either one of one-dimensional markets or the two-dimensional market are fully justified.} 
  
  \KWE{underliers; risk preferences function; continuous VaR criterion; cost and forecast 
densities; return relative function; Newman--Pearson procedure; combined portfolio; 
randomization; surrogate portfolio; idealistic portfolio}
 
 
  \DOI{10.14357/19922264190406} 

%\vspace*{-14pt}

 \Ack
 \noindent
 The work was supported by the Russian Foundation for Basic 
Research (project 17-01-00816).
 


\vspace*{6pt}

  \begin{multicols}{2}

\renewcommand{\bibname}{\protect\rmfamily References}
%\renewcommand{\bibname}{\large\protect\rm References}

{\small\frenchspacing
 {%\baselineskip=10.8pt
 \addcontentsline{toc}{section}{References}
 \begin{thebibliography}{9}
 
 \vspace*{-18pt}
 
  \bibitem{1-ag-1}
  \Aue{Agasandian, G.\,A.} 2002. Optimal behavior of an investor in option market. 
\textit{Joint Conference (International) on Neural Networks Proceedings.}
 IEEE. 1859--1864. 
  \bibitem{2-ag-1}
  \Aue{Agasandyan, G.\,A.} 2011. \textit{Primenenie kontinual'nogo kriteriya VaR 
na finansovykh rynkakh} [Application of continuous VaR-criterion in financial 
markets]. Moscow: CC RAS. 299 p. 
  \bibitem{3-ag-1}
  \Aue{Agasandyan, G.\,A.} 2015. \textit{Kontinual'nyy kriteriy VaR na 
mnogomernykh rynkakh optsionov} [Continuous VaR-criterion\linebreak\vspace*{-12pt}

\columnbreak

\noindent
 in multidimensional 
option markets]. Moscow: CC RAS. 297~p. 

\vspace*{-2.5pt}

  \bibitem{4-ag-1}
  \Aue{Agasandyan, G.\,A.} 2018. Kontinual'nyy kriteriy VaR na stsenarnykh 
rynkakh [Continuous VaR-criterion in scenario markets]. \textit{Informatika i~ee 
Primeneniya~--- Inform. Appl.} 12(1):32--40. 

\vspace*{-2.5pt}

  \bibitem{5-ag-1}
  \Aue{Agasandyan, G.\,A.} 2018. Kontinual'nyy kriteriy VaR i~optimal'nyy 
portfel' investora [Continuous VaR-criterion and  investor's 
optimal portfolio]. 
\textit{Upravlenie bol'shimi sistemami} [Large-Scale Systems Control] 73:6--26. 

\vspace*{-2.5pt}

  \bibitem{6-ag-1}
  \Aue{Cramer, H.} 1946. \textit{Mathematical methods of statistics}. Princeton, 
NJ: Princeton University Press. 575~p.
 \end{thebibliography}

 }
 }

\end{multicols}

%\vspace*{-7pt}

\hfill{\small\textit{Received March 27, 2019}}

%\pagebreak

%\vspace*{-22pt}

\Contrl

\noindent
\textbf{Agasandyan Gennady A.} (b.\ 1941)~--- Doctor of Science in physics and mathematics, leading 
scientist, A.\,A.~Dorodnicyn Computing Center, Federal Research Center ``Computer Science and Control'' 
of the Russian Academy of Sciences, 40~Vavilov Str., Moscow 119333, Russian Federation; 
\mbox{agasand17@yandex.ru}

   
\label{end\stat}

\renewcommand{\bibname}{\protect\rm Литература}  
   