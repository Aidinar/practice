\def\stat{kudr}

\def\tit{О ПРЕДСТАВЛЕНИИ ГАММА-ЭКСПОНЕНЦИАЛЬНОГО И~ОБОБЩЕННОГО ОТРИЦАТЕЛЬНОГО 
БИНОМИАЛЬНОГО РАСПРЕДЕЛЕНИЙ$^*$}

\def\titkol{О представлении гамма-экспоненциального и~обобщенного отрицательного 
биномиального распределений}

\def\aut{А.\,А.~Кудрявцев$^1$}

\def\autkol{А.\,А.~Кудрявцев}

\titel{\tit}{\aut}{\autkol}{\titkol}

\index{Кудрявцев А.\,А.}
\index{Kudryavtsev A.\,A.}



{\renewcommand{\thefootnote}{\fnsymbol{footnote}} \footnotetext[1]
{Работа выполнена при частичной финансовой 
поддержке РФФИ (проект 17-07-00577).}}


\renewcommand{\thefootnote}{\arabic{footnote}}
\footnotetext[1]{Московский государственный 
университет им.~М.\,В.~Ломоносова, факультет вычислительной математики 
и~кибернетики, \mbox{nubigena@mail.ru}}

%\vspace*{-2pt}





\Abst{Более полутора столетий распределения гамма-типа показывают свою адекватность 
при моделировании реальных процессов и~явлений. С течением времени конструкции, 
использующие распределения из гамма-семейства, все более усложняются с~целью 
улучшения применимости математических моделей к~актуальным аспектам 
жизнедеятельности. В работе приводится ряд результатов как обобщающих, так 
и~упрощающих некоторые классические формы, применяемые при анализе масштабных 
и~структурных смесей обобщенных гамма-законов. Вводится в~рассмотрение 
гамма-экспоненциальное распределение, описываются его характеристики. Приводится явный 
вид для интегральных представлений частных вероятностей обобщенного 
отрицательного биномиального распределения. Результаты формулируются в~терминах 
гамма-экспоненциальной функции. Полученные результаты могут \mbox{найти} широкое 
применение в~моделях, использующих для описания процессов и~явлений масштабные 
и~структурные смеси распределений с~положительным неограниченным носителем.}


\KW{гамма-экспоненциальная функция; обобщенное 
гам\-ма-рас\-пре\-де\-ле\-ние; обобщенное отрицательное биномиальное распределение; 
гам\-ма-экс\-по\-нен\-ци\-аль\-ное распределение; смешанные распределения}

\DOI{10.14357/19922264190412} 
  
%\vspace*{1pt}


\vskip 10pt plus 9pt minus 6pt

\thispagestyle{headings}

\begin{multicols}{2}

\label{st\stat}


\section{Введение}

История применения распределений гам\-ма-ти\-па в~приложениях обширна 
и~разнообразна. Уже во второй половине XIX~в.\ такого рода распределения активно 
использовались в~теории колебаний и~газовой термодинамике. В~1920-х~гг.\  
италь\-ян\-ский экономист Л.~Аморозо описал четырехпараметрическое распределение~\cite{Amoroso1925}, 
плотность которого может быть представлена для 
$m\hm\in\mathbb{R}$, $v,\theta\hm\neq0$, $q\hm>0$ в~виде
\begin{multline}
\label{Amoroso_density}
\hspace*{-6pt}f(x)=\fr{1}{\Gamma(q)}\abs{\frac{v}{\theta}}\left(\frac{x-
m}{\theta}\right)^{vq-1}\!\!\exp\left\{-\left(
\frac{x-m}{\theta}\right)^v\right\},\\
 \begin{cases}
   x>m\,, &\mbox{если } \theta>0\,;\\
   x<m\,, &\mbox{если } \theta<0\,.
 \end{cases}
\end{multline}
Распределение~(\ref{Amoroso_density}) приобрело широкую известность в~своем 
одностороннем несмещенном варианте при $\theta\hm>0$ и~$m\hm=0$:
\begin{equation}
\label{GG_density}
f(x)=\fr{|v| x^{vq -1}e^{-(x/\theta)^v}}{\theta^{vq}\Gamma(q)}, \enskip  v\neq0\,, \ 
 q>0\,,  \ x>0\,,
\end{equation}
под названием обобщенное гамма-рас\-пре\-де\-ле\-ние и~ассоциируется с~именем 
американского исследователя Э.~Стейси~\cite{Stacy1962}, хотя применялось 
и~ранее, например в~работах С.~Крицкого и~М.~Менкеля~\cite{KrMe1946,KrMe1948}, 
и~использовалось в~гид\-ро\-логии.

Гамма-класс распределений достаточно широк и~включает, в~част\-ности:
экспоненциальное распределение;
$\chi^2$-рас\-пре\-де\-ле\-ние;
распределение Эрланга;
гам\-ма-рас\-пре\-де\-ле\-ние;
полунормальное распределение, или
распределение максимума процесса броуновского движения;
распределение Рэлея;
распределение Макс\-вел\-ла--Больц\-ма\-на;
$\chi$-рас\-пре\-де\-ле\-ние;
$m$-рас\-пре\-де\-ле\-ние Накагами;
распределение Виль\-со\-на--Хиль\-фер\-ти;
распределение Вей\-бул\-ла--Гне\-ден\-ко;
обобщенное распределение Вейбулла;
псевдовейбулловское распределение;
распределение Пирсона третьего и~пятого типа;
распределение Леви;
распределение Фреше,
а также их обратные и~масштабированные аналоги.

В настоящее время популярность гам\-ма-клас\-са в~приложениях обусловливается не 
только его гибкостью и~многообразием, но и~возможностью использовать его 
представителей в~качестве адекватных асимптотических аппроксимаций во многих 
предельных схемах, в~частности в~гам\-ма-се\-мей\-ст\-ве присутствуют безгранично 
делимые и~устойчивые законы~\cite{ZaKo2013}.

Дискретный аналог гам\-ма-рас\-пре\-де\-ле\-ния представляет собой смешанное пуассоновское 
распределение со структурным гам\-ма-рас\-пре\-де\-ле\-ни\-ем и~носит название 
отрицательного биномиального распределения, частные вероятности которого при 
$n=0,1,\ldots$ имеют вид:
\begin{multline}
\label{NB_prob}
{\sf P}(N=n)=\int\limits_0^\infty
\fr{\lambda^{n+q -1}e^{-
(1+1/\theta)\lambda}}{\theta^{q}\Gamma(q)n!}\,d\lambda={}\\
{}=
\fr{\Gamma(n+q)}{\Gamma(n+1)\Gamma(q)}\left(\fr{\theta}{\theta+1}\right)^n
\left(\fr{1}{\theta+1}\right)^q.
\end{multline}
Естественным обобщением распределения~(\ref{NB_prob}) служит смешанное 
пуассоновское распределение, структура  которого задается плотностью~(\ref{GG_density}). 
Такие распределения носят названия обобщенных отрицательных 
биномиальных распределений и~широко применяются в~страховании, финансовой 
математике, физике и~других областях~\cite{KoZe2018}.

Важную роль в~приложениях играют масштабные смеси распределений гам\-ма-ти\-па, 
к~частному случаю которых относится обобщенное бе\-та-рас\-пре\-де\-ле\-ние второго рода 
и,~как следствие, распределения Бурра, Сингх--Мад\-да\-ла, Дагума, Ломакса, 
Фи\-ше\-ра--Сне\-де\-ко\-ра и~др.~\cite{Ku2019_1}, применяемые в~эконометрике, статистике, теории 
массового обслуживания, страховании и~пр. В~терминах масштабных смесей 
распределений из гамма-класса описываются модели баланса, в~частности 
байесовские модели повышения надежности и~массового обслуживания~\cite{Ku2018}.

Далее рассматриваются некоторые механизмы обобщения и~упрощения при работе 
с~распределениями из гам\-ма-се\-мей\-ст\-ва и~их структурными и~масштабными смесями.

\section{Гамма-экспоненциальная функция}


Приведем ряд вспомогательных инструментов.

\smallskip

\noindent
\textbf{Определение~1.}
Назовем функцию вида
\begin{multline}
\label{GEF}
{\sf Ge}_{\alpha,\, \beta} (x) = \sum\limits_{k=0}^{\infty}\fr{x^k}{k!}\, \Gamma(\alpha k + 
\beta), \\
 0\le\alpha<1\,, \ \beta> 0\,, \  x\in\mathbb{R}\,,
\end{multline}
гамма-экспоненциальной функцией~\cite{KuTi2017}.

\smallskip

Функция~(\ref{GEF}) обобщает на случай $\beta\neq1$ преобразование, введенное 
Леруа~\cite{LeRoy1900_1} для исследования производящих функций специального 
вида. Кроме того, функцию~(\ref{GEF}) можно рассматривать (при некоторых 
допущениях) как частный случай функции Сри\-ва\-ста\-ва--То\-мов\-ски~\cite{SrTo2009}, 
обобщающей функцию Мит\-таг-Леф\-фле\-ра~\cite{GoKiMaRo2014}.

\smallskip

\noindent
\textbf{Определение~2.}
Интегральной гамма-экс\-по\-нен\-ци\-аль\-ной функцией~\cite{KuPaSh2019_2} 
назовем функцию
$$
{\sf Gi}(r,s,t;\,x)=\fr{1 }{t\Gamma(s)\Gamma(t)}\int\limits_{0}^{x} 
{\sf Ge}_{r,\,tr+s}\left(-z^{1/t}\right) \, dz\,,
$$
где $0\le r<1$, $s,t>0$.

\smallskip


Будем обозначать обобщенное гам\-ма-рас\-пре\-де\-ле\-ние с~плотностью~(\ref{GG_density})
через $\mathrm{GG}(v,q,\theta)$.

В работе~\cite{Ku2019_1} был доказан ряд соотношений, который приведем в~виде 
сле\-ду\-ющих вспомогательных утверждений.

\smallskip

\noindent
\textbf{Лемма~1.}
\textit{Пусть $\alpha,\theta>0$, а $r,u,v\hm\neq0$ имеют один знак. Тогда}
\begin{multline*}
\int\limits_0^{\infty}y^{r-1}e^{-(y/\alpha)^u-(y/\theta)^v} \, dy ={}\\
{}=
 \begin{cases}
   \displaystyle \frac{\theta^{r}}{|v|}{\sf Ge}_{u/v,\,r/v}\left(-
\left(\fr{\theta}{\alpha}\right)^{u}\right), & |v|>|u|\,;\\[9pt]
   \displaystyle \fr{\alpha^{r}}{|u|}{\sf Ge}_{v/u,\,r/u}\left(-
\left(\fr{\alpha}{\theta}\right)^{v}\right), &|u|>|v|\,;\\[9pt]
   \displaystyle \fr{\Gamma(r/u)}{|u|(\alpha^{-u}+\theta^{-u})^{r/u}}, 
&u = v\,.
 \end{cases}
\end{multline*}

%\smallskip

\noindent
\textbf{Лемма~2.}
\textit{Пусть независимые случайные величины $\lambda$ и~$\mu$ имеют соответственно 
распределения $\mathrm{GG}(v,q,\theta)$ и~$\mathrm{GG}(u,p,\alpha)$, причем $uv\hm>0$. Тогда их 
отношение $\rho\hm=\lambda/\mu$ при $x\hm>0$ имеет плотность}
\begin{multline*}
\hspace*{-8pt}f_{\rho}(x) =\!
 \begin{cases}
   \displaystyle \fr{\abs{v}\alpha^{vq}x^{vq-
1}}{\theta^{vq}\Gamma(p)\Gamma(q)}\,{\sf Ge}_{v/u,\, vq/u+p}\left(-\left(\fr{\alpha 
x}{\theta}\right)^{v}\right), &\\
&\hspace*{-20mm}|u|>|v|;\\[3pt]
   \displaystyle \fr{\abs{u}\theta^{up}x^{-up-
1}}{\alpha^{up}\Gamma(p)\Gamma(q)}{\sf Ge}_{u/v,\, up/v+q}\left(-\left(\frac{\alpha 
x}{\theta}\right)^{-u}\right), &\\
&\hspace*{-20mm}|v|>|u|\,,
 \end{cases}\hspace*{-23pt}
\end{multline*}
\textit{функцию распределения}
$$
F_{\rho}(x) =
 \begin{cases}
   \displaystyle {\sf Gi}\left(\fr{v}{u},p,q;\,\left(\fr{\alpha x}{\theta}\right)^{vq}\right), 
& u>v>0;\\[6pt]
   \displaystyle 1-{\sf Gi}\left(\fr{v}{u},p,q;\,\left(\fr{\alpha x}{\theta}\right)^{vq}\right), 
&u<v<0\,;\\[6pt]
   \displaystyle 1-{\sf Gi}\left(\fr{u}{v},q,p;\,\left(\fr{\alpha x}{\theta}\right)^{-up}\right), 
&v>u>0\,;\\[6pt]
   \displaystyle {\sf Gi}\left(\fr{u}{v},q,p;\,\left(\fr{\alpha x}{\theta}\right)^{-up}\right), 
&v<u<0\,,
 \end{cases}
$$
\textit{и при $z\in\mathbb{R}$ имеет место соотношение}
\begin{multline*}
\e\rho^z=\fr{(\theta/\alpha)^{z} \Gamma(q+z/v)\Gamma(p-
z/u)}{\Gamma(q)\Gamma(p)}, \\
   q+\fr{z}{v}>0\,, \enskip  p-\fr{z}{u}>0\,.
\end{multline*}

\section{Гамма-экспоненциальное распределение}

Лемма~2 дает возможность ввести в~рас\-смот\-ре\-ние следующее 
понятие и~сформулировать ряд утверждений.

\smallskip

\noindent
\textbf{Определение~3.}
Будем говорить, что случайная величина~$\zeta$ имеет гам\-ма-экс\-по\-нен\-ци\-аль\-ное (GE) 
распределение с~параметрами $0\hm\le r\hm<1$, $\nu\hm\neq0$, $s,t,\delta\hm>0$, если ее 
плотность при $x\hm>0$ задается соотношением
$$
g_E(x) =
\fr{|\nu|x^{t\nu-1}}{\delta^{t\nu}\Gamma(s)\Gamma(t)}
\,   {\sf Ge}_{r,\, tr+s}\left(-\left( \fr{x}{\delta} \right)^{\nu}\right),
$$
где $E=(r,\nu,s,t,\delta)$.

\smallskip

\noindent
\textbf{Утверждение~1.}
\textit{Пусть независимые случайные величины~$\lambda$ и~$\mu$ имеют соответственно 
распределения $\mathrm{GG}(v,q,\theta)$ и~$\mathrm{GG}(u,p,\alpha)$, $uv\hm>0$. Тогда
распределение $\lambda$ совпадает с~$\mathrm{GE}(0,v,\cdot,q,\theta)$;
распределение $\lambda/\mu$ при $|u|\hm>|v|$ совпадает 
с~$\mathrm{GE}(v/u,v,p,q,\theta/\alpha)$;
распределение $\lambda/\mu$ при $|v|\hm>|u|$ совпадает с~$\mathrm{GE}(u/v,-
u,q,p,\theta/\alpha)$}.


\smallskip

Утверждение~1 демонстрирует, в~частности, что гам\-ма-экс\-по\-нен\-ци\-аль\-ное 
распределение можно рассматривать как обобщение распределения~(\ref{GG_density}).

Лемма~2 также позволяет определить гам\-ма-экс\-по\-нен\-ци\-аль\-ное 
распределение следующим образом.

\smallskip

\noindent
\textbf{Определение~4.}
Случайная величина~$\zeta$ имеет гам\-ма-экс\-по\-нен\-ци\-аль\-ное распределение 
с~параметрами $0\hm\le r\hm<1$, $\nu\hm\neq0$, $s,t,\delta\hm>0$, если ее функция 
распределения при $x\hm>0$ задается соотношением:
$$
G_E(x) =
 \begin{cases}
   \displaystyle {\sf Gi}\left(r,s,t;\,\left(\fr{x}{\delta}\right)^{\nu t}\right), & \nu>0\,;\\[3pt]
   \displaystyle 1-{\sf Gi}\left(r,s,t;\,\left(\fr{x}{\delta}\right)^{\nu t}\right), & \nu<0\,,
 \end{cases}
$$
где $E=(r,\nu,s,t,\delta)$.


\smallskip

Исходя из свойства инвариантности гамма-экс\-по\-нен\-ци\-аль\-ной функции относительно 
интегрального преобразования~\cite{Ku2019_1}, можно упростить при некоторых 
значениях параметров представление функции распределения~$G_E(x)$.

\smallskip

\noindent
\textbf{Утверждение~2.}
\textit{При} $x>0$
\begin{multline*}
G_E(x) ={}\\
{}=
 \begin{cases}
    \displaystyle \fr{rx^{\nu t}}{\delta^{\nu t}\Gamma(t)}\,
{\sf Ge}_{r,\,rt}\left(-\left(\fr{x}{\delta}\right)^{\nu}\right), &\!\ \nu>0\,,\ 
s=1\,;\\[9pt]
   \displaystyle 1-\fr{1}{\Gamma(s)}\,{\sf Ge}_{r,\,s}\left(-
\left(\fr{x}{\delta}\right)^{\nu}\right), &\! \nu<0\,,\ t=1\,.
 \end{cases}
\end{multline*}

Для нахождения моментов распределения $\mathrm{GE}(r,\nu,s,t,\delta)$ воспользуемся 
леммой~2 и~утверждением~1. Заметим, что~$g_E(x)$ 
при $E\hm=(r,\nu,s,t,\delta)$ и~$0\hm<r\hm<1$~--- плотность отношения независимых  
случайных величин с~распределениями $\mathrm{GG}(\nu,t,\delta)$ и~$\mathrm{GG}(\nu/r,s,1)$. Таким 
образом, справедливо следующее утверждение.

\smallskip

\noindent
\textbf{Теорема~1.}
\textit{Пусть случайная величина $\zeta$ имеет распределение $\mathrm{GE}(r,\nu,s,t,\delta)$. 
Тогда}
\begin{multline*}
\e\zeta^z=\fr{\delta^{z} \Gamma(t+z/\nu)\Gamma(s-
zr/\nu)}{\Gamma(t)\Gamma(s)}\,, \\  
 t+\fr{z}{\nu}>0\,, \enskip  s-\fr{zr}{\nu}>0\,.
\end{multline*}

\section{Обобщенное отрицательное биномиальное распределение}


В приложениях смешанные пуассоновские распределения прежде всего используются 
для описания характеристик проекций структурной смеси стандартного 
пуассоновского процесса~$N_1(t)$ и~некоторой неотрицательной случайной величины~$\Lambda$, 
имеющей смысл (при фиксированном элементарном исходе) интенсивности 
процесса. Таким образом, для проекции смешанного пуассоновского процесса~$N(t)$ в~точке $t\hm>0$ справедливо для $n\hm=0,1,\ldots$
\begin{multline}
\label{MPP_distrib}
\p(N(t)=n)=\p\left(N_1(\Lambda t)=n\right)={}\\
{}=
\int\limits_0^\infty e^{-ty}\fr{(ty)^n}{n!}\, 
d\p(\Lambda<y).
\end{multline}
По аналогии с~(\ref{NB_prob}) в~случае, когда независимая от процесса~$N_1(t)$ 
случайная величина~$\Lambda$ имеет распределение $\mathrm{GG}(v,q,\theta)$ 
с~плотностью~(\ref{GG_density}), говорят~\cite{KoZe2018}, что проекции процесса~$N(t)$ 
имеют 
обобщенное отрицательное биномиальное распределение. Везде далее будем 
использовать дополнительный параметр $t\hm>0$, имеющий смысл момента времени. Для 
получения характеристик <<классического>> обобщенного отрицательного 
биномиального распределения достаточно положить $t\hm=1$.

Из леммы~1 и~соотношения~(\ref{MPP_distrib}) следует 
утверж\-де\-ние.

\smallskip

\noindent
\textbf{Теорема~2.}
\textit{Одномерные распределения смешанного пуассоновского процесса со структурным 
распределением $\mathrm{GG}(v,q,\theta)$
при $t\hm>0$ для $n\hm=0,1,\ldots$ задаются вероятностями}
\begin{multline*}
\p (N_1(\Lambda t)=n)=
\fr{vt^n}{\theta^{vq}\Gamma(q)n!}\times{}\\
{}\times
 \begin{cases}
   \left(\fr{1}{t}\right)^{vq+n}{\sf Ge}_{v,\, vq+n}(-(\theta t)^{-v}), &0<v<1\,;\\[12pt]
\fr{\theta^{vq+n}}{v}\,{\sf Ge}_{1/v,\, n/v+q}
\left(-\theta t\right), & v>1\,;\\[3pt]
   (t+\theta^{-1})^{-(n+q)}\Gamma(n+q), & v = 1\,.
 \end{cases}
\end{multline*}


%\smallskip

\noindent
\textbf{Замечание~1.}
В теореме~2 при $v\hm=1$ получаем одномерное распределение процесса 
Пойа, соответствующие отрицательному биномиальному распределению~(\ref{NB_prob}).


\smallskip

Теорема~2 дает возможность ввести в~рассмотрение следующие 
понятия.

\smallskip

\noindent
\textbf{Определение~5.}
Будем говорить, что случайная величина $\xi(t)$, $t\hm>0$, имеет обобщенное 
отрицательное биномиальное распределение первого рода с~параметрами $v\hm>1$, 
$q,\theta\hm>0$, если
$$
\p(\xi(t)=n) =
\fr{(\theta t)^{n}}{\Gamma(q)n!}\,
   {\sf Ge}_{1/v,\, n/v+q}(-\theta t), \enskip n=0,1,\ldots
$$

\smallskip

\noindent
\textbf{Определение~6.}
Будем говорить, что случайная величина $\eta(t)$, $t\hm>0$, имеет обобщенное 
отрицательное биномиальное распределение второго рода с~параметрами $0\hm<v\hm<1$, 
$q,\theta\hm>0$, если
$$
\p(\eta(t)=n) =
\fr{v{\sf Ge}_{v,\, vq+n}(-(\theta t)^{-v})}{(\theta t)^{vq}\Gamma(q)n!}    , \enskip 
n=0,1,\ldots
$$


Данные представления дают возможность \mbox{изучать} характеристики обобщенного 
отрицательного биномиального распределения не только в~интегральном виде, но 
и~посредством гам\-ма-экс\-по\-нен\-ци\-аль\-ной функции.

\section{Заключение}

В работе представлены утверждения и~соотношения, позволяющие обобщить 
и~упростить ряд известных характеристик обобщенных гам\-ма-рас\-пре\-де\-ле\-ний и~их 
структурных и~масштабных смесей за счет использования гам\-ма-экс\-по\-нен\-ци\-аль\-ной 
функции. Результаты работы могут найти применение при исследовании моделей, 
в~которых применяются смешивающие распределения с~неограниченным неотрицательным 
носителем.

{\small\frenchspacing
 {%\baselineskip=10.8pt
 \addcontentsline{toc}{section}{References}
 \begin{thebibliography}{99}
\bibitem{Amoroso1925}
\Au{Amoroso~L.}
Ricerche intorno alla curva dei redditi~// 
Ann. Mat. Pur. Appl. Ser.~4, 1925. Vol.~21. P.~123--159.

\bibitem{Stacy1962}
\Au{Stacy~E.\,W.}
A~generalization of the gamma distribution~// 
Ann. Math. Stat., 1962. Vol.~33. P.~1187--1192.

\bibitem{KrMe1946}
\Au{Крицкий~С.\,Н., Менкель~М.\,Ф.}
О приемах исследования случайных колебаний речного стока~// 
Труды НИУ ГУГМС. Сер.~IV, 1946. Вып.~29. С.~3--32.
%. Труды Гос. гидрологического ин-та, вып. 29, 1946.

\bibitem{KrMe1948}
\Au{Крицкий~С.\,Н., Менкель~М.\,Ф.}
Выбор кривых распределения вероятностей для расчетов речного стока~// 
Известия АН СССР. Отд. техн. наук, 1948. №\,6. С.~15--21.

\bibitem{ZaKo2013}
\Au{Закс~Л.\,М., Королев~В.\,Ю.}
Обобщенные дисперсионные гам\-ма-рас\-пре\-де\-ле\-ния как предельные для случайных сумм~// 
Информатика и~её применения, 2013. Т.~7. Вып.~1. С.~105--115.

\bibitem{KoZe2018}
\Au{Королев~В.\,Ю., Зейфман~А.\,И.}
Generalized negative binomial distributions as mixed geometric laws and related 
limit theorems, 2018.
{\sf https://arxiv.org/pdf/ 1703.07276.pdf}.

\bibitem{Ku2019_1}
\Au{Кудрявцев~А.\,А.}
Априорное обобщенное гам\-ма-рас\-пре\-де\-ле\-ние в~байесовских моделях баланса~// 
Информатика и~её применения, 2019. Т.~13. Вып.~3. С.~20--26.

\bibitem{Ku2018}
\Au{Кудрявцев~А.\,А.}
Байесовские модели баланса~// Информатика и~её применения, 2018. Т.~12. Вып.~3. С.~18--27.

\bibitem{KuTi2017}
\Au{Кудрявцев~А.\,А., Титова~А.\,И.}
Гамма-экс\-по\-нен\-ци\-аль\-ная функция в~байесовских моделях массового обслуживания~// 
Информатика и~её применения, 2017. Т.~11. Вып.~4. С.~104--108.

\bibitem{LeRoy1900_1}
\Au{Le~Roy~$\acute{\mbox{E}}$.}
Sur les s$\acute{\mbox{e}}$ries divergentes et les fonctions d$\acute{\mbox{e}}$finies 
par un d$\acute{\mbox{e}}$veloppement de Taylor~// 
Ann. Facult$\acute{\mbox{e}}$ Sci. Toulouse 2 S$\acute{\mbox{e}}$r., 
1900.  Vol.~2. No.\,3. P.~317--384.

\bibitem{SrTo2009}
\Au{Srivastava~H.\,M., Tomovski~{\ptb{\v{Z}}}.}
Fractional calculus with an integral operator containing a generalized 
Mittag-Leffler function in the kernel~// 
Appl. Math. Comput., 2009. Vol.~211. P.~198--210.

\bibitem{GoKiMaRo2014}
\Au{Gorenlo~R., Kilbas~A.\,A., Mainardi~F., Rogosin~S.\,V.}
Mittag-Leffler functions, related topics and applications.~--- 
Berlin, Heidelberg: Springer-Verlag, 2014. 443~p.

\bibitem{KuPaSh2019_2}
\Au{Кудрявцев~А.\,А., Палионная~С.\,И., Шоргин~В.\,С.}
Априорное обобщенное распределение Фреше в~байесовских моделях баланса~// 
Системы и~средства информатики, 2019. Т.~29. №\,2. С.~39--45.
 \end{thebibliography}

 }
 }

\end{multicols}

\vspace*{-6pt}

\hfill{\small\textit{Поступила в~редакцию 22.09.19}}

%\vspace*{8pt}

%\pagebreak

\newpage

\vspace*{-28pt}

%\hrule

%\vspace*{2pt}

%\hrule

%\vspace*{-2pt}

\def\tit{ON THE REPRESENTATION OF GAMMA-EXPONENTIAL 
AND~GENERALIZED NEGATIVE BINOMIAL DISTRIBUTIONS\\[-7pt]}


\def\titkol{On the representation of gamma-exponential 
and~generalized negative binomial distributions}

\def\aut{A.\,A.~Kudryavtsev\\[-9pt]}

\def\autkol{A.\,A.~Kudryavtsev}

\titel{\tit}{\aut}{\autkol}{\titkol}

\vspace*{-20pt}

\noindent
Department of Mathematical Statistics, Faculty of Computational 
Mathematics and Cybernetics, M.\,V.~Lomonosov Moscow State University, 
1-52~Leninskiye Gory, GSP-1, Moscow 119991, Russian Federation

\def\leftfootline{\small{\textbf{\thepage}
\hfill INFORMATIKA I EE PRIMENENIYA~--- INFORMATICS AND
APPLICATIONS\ \ \ 2019\ \ \ volume~13\ \ \ issue\ 4}
}%
 \def\rightfootline{\small{INFORMATIKA I EE PRIMENENIYA~---
INFORMATICS AND APPLICATIONS\ \ \ 2019\ \ \ volume~13\ \ \ issue\ 4
\hfill \textbf{\thepage}}}

\vspace*{3pt}  




\Abste{For more than a century and a half, gamma-type distributions have shown 
their adequacy in modeling real processes and phenomena. Over time, 
designs using distributions from the gamma family are becoming more 
complex in order to improve the applicability of mathematical models 
to relevant aspects of life. The paper presents a~number of results 
both generalizing and simplifying some classical forms used in the analysis 
of large-scale and structural mixtures of generalized gamma laws. 
The gamma-exponential distribution is introduced and its characteristics 
are described. An explicit form for integral representations of partial 
probabilities of the generalized negative binomial distribution is given. 
The results are formulated in terms of the gamma exponential function. 
The obtained results can be widely used in models that use scale and 
structural mixtures of distributions with positive unrestricted support 
to describe processes and phenomena.}

\KWE{gamma exponential function; generalized gamma distribution; 
generalized negative binomial distribution; gamma-exponential distribution; 
mixed distributions}




\DOI{10.14357/19922264190412} 

\vspace*{-18pt}

 \Ack
 
\vspace*{-5pt}

\noindent
The work was partly supported by the Russian Foundation for Basic Research 
(project 17-07-00577).


\vspace*{-1pt}

  \begin{multicols}{2}

\renewcommand{\bibname}{\protect\rmfamily References}
%\renewcommand{\bibname}{\large\protect\rm References}

{\small\frenchspacing
 {%\baselineskip=10.8pt
 \addcontentsline{toc}{section}{References}
 \begin{thebibliography}{99}
 
 \vspace*{-2pt}

\bibitem{1-ku}
\Aue{Amoroso, L.} 1925. Ricerche intorno alla curva dei redditi. 
\textit{Ann. Mat. Pur. Appl. Ser. 4} 21:123--159.

\bibitem{2-ku}
\Aue{Stacy, E.\,W.} 1962. A~generalization of the gamma distribution. 
\textit{Ann. Math. Stat.} 33:1187--1192.

\bibitem{3-ku}
\Aue{Kritsky, S.\,N., and M.\,F.~Menkel.} 1946. 
O~priemakh issledovaniya sluchaynykh kolebaniy rechnogo stoka 
[Methods of investigation of random fluctuations of river flow]. 
\textit{Trudy NIU GUGMS Ser.~IV} 
[Proceedings of GUGMS research institutions, Ser.~IV] 29:3--32.

\bibitem{4-ku}
\Aue{Kritsky, S.\,N., and M.\,F.~Menkel.} 1948. 
Vybor krivykh raspredeleniya veroyatnostey dlya raschetov rechnogo stoka 
[Selection of probability distribution curves for river flow calculations].
\textit{Izvestiya AN SSSR. Otd. tekhn. nauk}
[Herald of the Russian Academy of Sciences. Technical Sciences] 6:15--21.

\bibitem{5-ku}
\Aue{Zaks, L.\,M., and V.\,Yu.~Korolev. }
2013. Obobshchennye dispersionnye gamma-raspredeleniya kak predel'nye 
dlya sluchaynykh summ [Generalized dispersion gamma distributions 
as limiting for random sums]. 
\textit{Informatika i~ee Primeneniya~--- Inform. Appl.} 7(1):105--115.

\bibitem{6-ku}
\Aue{Korolev, V.\,Yu., and A.\,I.~Zeifman.}
 2018. Generalized negative binomial distributions as mixed geometric 
 laws and related limit theorems. Available at: 
 {\sf https://arxiv.org/pdf/1703.07276.pdf} (accessed September~22, 2019).

\bibitem{7-ku}
\Aue{Kudryavtsev, A.\,A.}
 2019. Apriornoe obobshchennoe gamma-raspredelenie 
 v~bayesovskikh modelyakh balansa [\textit{A~priori} generalized gamma distribution in 
 Bayesian balance models]. \textit{Informatika i~ee Primeneniya~--- Inform. Appl.}
 13(3):20--26.

\bibitem{8-ku}
\Aue{Kudryavtsev, A.\,A.} 2019. 
Bayesovskie modeli balansa [Bayesian balance models]. 
\textit{Informatika i~ee Primeneniya~--- Inform. Appl.} 12(3):18--27.

\bibitem{9-ku}
\Aue{Kudryavtsev, A.\,A., and A.\,I.~Titova}. 
2017. Gamma-eksponentsial'naya funktsiya v~bayesovskikh modelyakh 
massovogo obsluzhivaniya [Gamma-exponential function in Bayesian queuing models]. 
\textit{Informatika i~ee Primeneniya~--- Inform. Appl.} 11(4):104--108.
\bibitem{10-ku}
\Aue{Le Roy,~$\acute{\mbox{E}}$.} 1900. Sur les s$\acute{\mbox{e}}$ries 
divergentes et les fonctions d$\acute{\mbox{e}}$finies par un 
d$\acute{\mbox{e}}$veloppement de Taylor. 
\textit{Ann. Facult$\acute{\mbox{e}}$ Sci. Toulouse 2 S$\acute{\mbox{e}}$r.}
2(3):317--384.
\bibitem{11-ku}
\Aue{Srivastava, H.\,M., and {\ptb{\v{Z}}}.~Tomovski.}
 2009. Fractional calculus with an integral operator containing a generalized 
 Mittag-Leffler function in the kernel. 
 \textit{Appl. Math. Comput.} 211:198--210.
\bibitem{12-ku}
\Aue{Gorenlo, R., A.\,A.~Kilbas, F.~Mainardi, and S.\,V.~Rogosin.}
 2014. \textit{Mittag-Leffler functions, related topics and applications}. 
 Berlin, Heidelberg: Springer-Verlag. 443~p.
\bibitem{13-ku}
\Aue{Kudryavtsev, A.\,A., S.\,I.~Palionnaia, and V.\,S.~Shorgin.}
 2019. Apriornoe obobshchennoe raspredelenie Freshe 
 v~bayesovskikh modelyakh balansa [\textit{A~priori} generalized Frechet distribution 
 in Bayesian balance models]. \textit{Sistemy i~Sredstva Informatiki~--- 
 Systems and Means of Informatics} 29(2):39--45.
\end{thebibliography}

 }
 }

\end{multicols}

\vspace*{-11pt}

\hfill{\small\textit{Received September 22, 2019}}

%\pagebreak

\vspace*{-19pt}

\Contrl

\vspace*{-6pt}

\noindent
\textbf{Kudryavtsev Alexey A.} (b.\ 1978)~--- 
Candidate of Science (PhD) in physics and mathematics, associate professor, 
Department of Mathematical Statistics, Faculty of Computational Mathematics 
and Cybernetics, M.\,V.~Lomonosov Moscow State University, 
1-52~Leninskiye Gory, GSP-1, Moscow 119991, Russian Federation; \mbox{nubigena@mail.ru}
\label{end\stat}

\renewcommand{\bibname}{\protect\rm Литература}  