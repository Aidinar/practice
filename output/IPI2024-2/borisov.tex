\newcommand {\ff}{{\mathcal F}}
\newcommand {\ppp}{{\mathcal P}}
\newcommand {\col}{\mathop{\mathrm{col}}\,}
\newcommand {\qqq}{{\mathcal Q}}
\newcommand{\me}[2]{\mathsf{E}_{ #1 }\left\{ \mathop{#2} \right\} }
\newcommand {\row}{\mathop{\mathrm{row\,}}}


\def\stat{borisov}

\def\tit{РЫНОК С МАРКОВСКОЙ СКАЧКООБРАЗНОЙ ВОЛАТИЛЬНОСТЬЮ V: ПОПОЛНЕНИЕ РЫНКА ДЕРИВАТИВАМИ$^*$}

\def\titkol{Рынок с~марковской скачкообразной волатильностью V: пополнение рынка деривативами}

\def\aut{А.\,В.~Борисов$^1$}

\def\autkol{А.\,В.~Борисов}

\titel{\tit}{\aut}{\autkol}{\titkol}

\index{Борисов А.\,В.}
\index{Borisov A.\,V.}


{\renewcommand{\thefootnote}{\fnsymbol{footnote}} \footnotetext[1]
{Работа выполнялась с~использованием инфраструктуры Центра коллективного пользования <<Высокопроизводительные вы\-чис\-ле\-ния и~большие данные>> 
(ЦКП <<Информатика>>) ФИЦ ИУ РАН (г.~Москва).}}


\renewcommand{\thefootnote}{\arabic{footnote}}
\footnotetext[1]{Федеральный исследовательский центр <<Информатика и~управление>> Российской академии наук, 
\mbox{aborisov@frccsc.ru}}


\vspace*{-6pt}




\Abst{Пятая, заключительная часть цикла посвящена задаче пополнения рынка с~марковской скачкообразной сменой режимов. 
Рынок включает в~себя безрисковый банковский депозит с~известной неслучайной процентной ставкой, 
а~также набор базовых рисковых активов. Мгновенные процентные ставки и~волатильности этих активов представляют собой функции 
скрытого фактора смены режимов, описываемого некоторым марковским скачкообразным процессом (МСП) с~конечным множеством состояний. 
Целью статьи ставится пополнение предложенного рынка. Оно означает добавление некоторого набора новых финансовых инструментов 
таким образом, что выплаты по любому платежному требованию, сформированному на рынке, могут быть воспроизведены 
с~помощью некоторого самофинансируемого портфеля, содержащего исходные и~добавленные инструменты. Доказано, что для пополнения
рынка к~исходному множеству активов достаточно добавить
деривативы европейского типа, построенные на уже представленных на рынке базовых рисковых активах.
При этом число добавляемых деривативов совпадает с~числом режимов рынка. Задача пополнения рынка имеет неединственное решение, 
и~в~статье приведено сравнение предложенного способа с~уже существующим.}


\KW{марковский скачкообразный процесс; портфель ценных бумаг; свойство самофинансирования; полнота рынка}

\DOI{10.14357/19922264240202}{DQFSDO}
  
%\vspace*{-6pt}


\vskip 10pt plus 9pt minus 6pt

\thispagestyle{headings}

\begin{multicols}{2}

\label{st\stat}


\section{Введение}

Предметом исследования заключительной заметки цикла~\cite{B_23_1_IA, B_23_2_IA, B_23_3_IA, B_23_4_IA} 
стала модель рынка, состоящего из безрискового банковского депозита и~набора базовых рисковых активов. 
Мгновенная \mbox{процентная} ставка и~волатильность этих активов зависят от режима рынка, описываемого некоторым 
ненаблюдаемым МСП. Рынки такого типа считаются неполными~\cite{Criens_20}, 
так как содержат <<не\-тор\-гу\-емый>> (\textit{nontradable}) случайный процесс~\cite{Fouque_00}. 
Этот факт усложняет теоретическое и~практическое решение задач определения справедливых цен деривативов, 
хеджирования и~оптимального инвестирования. Одним из на\-прав\-ле\-ний сокращения не\-опре\-де\-лен\-ности в~данных моделях  стало 
оценивание режима рынка по имеющейся статистической информации: журналам заявок на по\-куп\-ку/про\-да\-жу 
финансовых инструментов (базовых и~производных), а~так\-же цен проведенных сделок. Предыду\-щие час\-ти цик\-ла~\cite{B_23_1_IA, B_23_3_IA, B_23_4_IA} 
содержали результаты исследований в~этой об\-ласти.

Другой способ сокращения данной неопределенности~--- 
пополнение рынка. Эта процедура означает добавление к~уже имеющимся на рынке некоторых дополнительных инструментов, несущих информацию 
о~неторгуемых процессах. Пополненный набор инструментов должен давать возможность воспроизводить выплаты по любым платежным 
требованиям, которые можно определить на рынке. Именно задаче пополнения рынка посвящена эта заметка.

Статья организована следующим образом. В~разд.~2 
приведено детальное описание модели исследуемого рынка. Раздел~3 содержит необхо\-димые определения портфельных инвестиций,\linebreak 
а~также %\linebreak 
постановку задачи пополнения рынка. Раздел~4~--- основной в~данной работе: 
он представляет\linebreak набор деривативов европейского типа от су\-щест\-ву\-ющих рисковых активов, пригодный для пополнения рынка. 
В~разд.~4 также приводится сравнение данного набора деривативов со \mbox{скачкообразными} финансовыми инструментами, предложенными 
в~\cite{Courcuera_05, ZhangElliottSiuGuo_11} с~аналогичной целью. Раздел~5 содержит выводы по проведенным исследованиям.


\vspace*{-2pt}

\section{Исследуемая модель рынка}

На вероятностном базисе с~фильтрацией $(\Omega, \ff, \ppp, \{\ff_t\}_{t \in [0,T]})$ рассматривается модель финансового рынка, изначально состоящего
из банковского вклада 

\noindent
$$
B_t = \exp \left( \int\limits_0^t r_u\,d_u\right)
$$ 
с~детерминированной ставкой~$r_t$
и~$N$ базовых рисковых инструментов 
$$
S_t = \col(S_t^1,\ldots,S_t^N).
$$

Цена $S_t $  описывается стохастической дифференциальной сис\-те\-мой (СДС)
  \begin{multline}
  dS_t = \mathrm{diag}\, S_t a(t,Z_t)\,dt + \mathrm{diag}\, S_t \sigma(t,Z_t)\,dw_t,\\
   t \in (0,T], \ S_0 \sim p_0(s),
  \label{eq:S_sys_P}
  \end{multline}
  где $w_t \triangleq \col (w_t^1,\ldots,w_t^N)$~--- $N$-мер\-ный винеровский процесс, 
  а~случайные функции мгновенной процентной ставки~$a$ и~внутренней во\-ла\-тиль\-ности~$\sigma$  имеют вид
  $$
  a(t,Z_t) = \sum\limits_{\ell=1}^L Z_t^{\ell} a^{\ell}(t);\enskip
  \sigma(t,Z_t) = \sum\limits_{\ell=1}^L Z_t^{\ell}  \sigma^{\ell}(t)
  $$
с~набором известных детерминированных функций $\{ a^{\ell}(t)\}_{\ell=\overline{1,L}}$  и~$\{\sigma^{\ell}(t)\}_{\ell=\overline{1,L}}$: 
$$
a^{\ell}(t) \triangleq \col(a_1^{\ell}(t), \ldots,a_N^{\ell}(t) );
\sigma^{\ell}(t) = \|\sigma_{ij}^{\ell}(t)\|_{i,j=\overline{1,N}}.
$$


В функциях $a(\cdot)$ и~$\sigma(\cdot)$  $Z_t \triangleq \col(Z_t^1, \ldots, Z_t^L) \hm\in \{e_1,\ldots, e_L\}$~--- МСП 
с~мат\-ри\-цей интенсивностей переходов~$\Lambda(\cdot)$ и~начальным распределением~$p_0^Z$. 
Марковский скачкообразный процесс~$Z_t$ пред\-став\-ля\-ет собой решение СДС с~мартингалом $M_t \hm= \col(M_t^1, \ldots, M_t^L)$ в~правой части~\cite{ElliottAggounMoore_08}:
\begin{equation}
  dZ_t = \Lambda^{\top}(t)Z_t\,dt + dM_t,\enskip t \in (0,T], \ Z_0 \sim p_0^Z.
  \label{eq:Z_sys}
  \end{equation}

Для вероятностного базиса с~фильтрацией и~модели~(\ref{eq:S_sys_P}), (\ref{eq:Z_sys}) выполнены следующие предположения.
\begin{enumerate}[1.]
\item
С вероятностью 1 траектории процесса~$Z_t$ непрерывны справа и~имеют конечные пределы слева~\cite{liptser2012theory}.
\item
$\ff_t = \sigma \{S_0,\;w_u,\;Z_u: \;0 \leqslant u \hm\leqslant t\}$ для любых $t \hm\in [0,T]$.
\item
Для некоторого $\alpha \hm> 0$ неравенство
$\sigma^{\ell}(t)\left( \sigma^{\ell}(t)\right)^{\top} \hm\geqslant \alpha I$ выполняется для всех $t\hm \in [0,T]$ и~$\ell\hm = \overline{1,L}$.
\item
$\mathcal{P}\{S_0 > 0\} =1$.
\item
$p_0^Z = \col (p_0^{Z,1},\ldots,p_0^{Z,L})$: $\enskip \min_{1 \leqslant \ell \leqslant L}p_0^{Z,\ell} \hm>0$.
\item
Начальные условия $S_0$ и~$Z_0$ независимы.
\end{enumerate}

Несмотря на технический характер, предположения~1--6 необременительны и~имеют очевидный смысл. Предположения~1, 2 и~6 
гарантируют непрерывность справа потока $\{\ff_t\}$~--- стандартное требование для правомерного применения математического аппарата
 стохастического анализа. Предположение~2 также означает, что вся случайность на рынке порождается
только начальным условием~$S_0$, процессами~$w$ и~$Z$. Совместно с~предположением~3 оно также обеспечивает совпадение фильтраций
$$
\ff_t \equiv \sigma\{ S_u, \; Z_u: \; 0 \leqslant u \leqslant t\}, \enskip t \in [0,T].
$$
Предположение~4 с~вероятностью~1 гарантирует положительность цен базовых рисковых активов в~любой момент времени. 
Предположение~5 обеспечивает строгую положительность всех компонентов вектора распределения~$p_t^Z$ МСП~$Z$ при $t \hm\in [0,T]$.

\section{Задача пополнения рынка}

Для корректной постановки задачи пополнения исследуемой модели рынка необходимо ввес\-ти ряд дополнительных определений~\cite{shiryaev1999essentials}. 
Пусть $\mathcal{S}_t \triangleq \sigma \{ S_u: \; 0 \hm\leqslant u \hm\leqslant t \}$~--- естественный поток $\sigma$-ал\-гебр, порожденный 
ценами базовых финансовых инструментов. Он задает структуру информации, доступной трейдерам для выработки торговых стратегий~--- портфелей ценных бумаг.
Заметим, что для любого $t \hm\in [0,T]$ в~исследуемой модели рынка выполняется строгое включение $\mathcal{S}_t \hm\subset \ff_t$, причем поток $\sigma$-ал\-гебр
$\{\mathcal{S}_t\}$ не обладает свойством непрерывности справа в~отличие от~$\{\ff_t\}$. С~прикладной точки зрения это значит, 
что не все случайные процессы, влияющие на рынок, доступны трейдерам в~форме цен тех или иных бумаг: имеется неторгуемый МСП~$Z_t$. 
Данный процесс не является ценой ка\-кой-ли\-бо бумаги, но скрыто влияет на цены~$S_t$ исходных инструментов.

\textit{Портфелем} ценных бумаг называется пара $(\varpi, \pi)$, где
\begin{itemize}
\item
$\varpi \triangleq\{\varpi_t\}_{t \in [0,T]}$, $\varpi_t \in \mathbb{R}$,~--- 
скалярный $\mathcal{S}_t$-пред\-ска\-зу\-емый случайный процесс, определяющий объем средств, вложенных в~банковский депозит;
\item
$\pi \triangleq\{\pi_t\}_{t \in [0,T]}, \; \pi_t \hm\in \mathbb{R}^N$,~--- $N$-мер\-ный $\mathcal{S}_t$-пред\-ска\-зу\-емый случайный 
процесс $\pi_t \hm= \mathrm{row}\,(\pi_t^1, \ldots, \pi_t^N)$, определяющий число рисковых ценных бумаг каждого наименования, 
входящих в~данный момент времени в~портфель.
\end{itemize}
При этом процесс
$
C_t \triangleq \varpi_t B_t + \pi_t S_t
%\label{eq:eq_cap_1}
$
называется \textit{капиталом} портфеля $(\varpi, \pi)$.

Портфель $C$ называется \textit{самофинансируемым}, если равенство
\begin{equation*}
C_t = \varpi_0  B_0 + \pi_0 S_0 + \int\limits_0^t \left(\varpi_u dB_u + \pi_u dS_u\right)
%\label{eq:eq_cap_1}
\end{equation*}
выполняется для любого $t \hm\in [0,T]$.

Рынок называется \textit{полным}, если для любого платеж\-но\-го требования~--- $\ff_T$-из\-ме\-ри\-мой 
неотрицательной квадратично интегрируемой случайной величины~$q_T$~--- существует такой са\-мо\-фи\-нан\-си\-ру\-емый портфель $(\varpi, \pi)$, что
\begin{multline*}
C_T = \varpi_0  B_0 + \pi_0 S_0 + \int\limits_0^T \left(\varpi_u \,dB_u + \pi_u\, dS_u\right) ={}\\
{}= q_T \enskip \ppp\mbox{-{п.~н.}}
%\label{eq:eq_compl_1}
\end{multline*}
Полнота рынка означает, что любое платежное\linebreak средство на нем может быть воспроизведено с~помощью надлежащего выбора уже 
имеющихся финансовых инструментов. Следует также подчеркнуть, что данное определение полноты \mbox{рынка}\linebreak предполагает воз\-мож\-ность 
воспроизведения не только платежного требования <<европейского типа>>, т.\,е.\ $q_T (\omega) \hm= q_T (S_T(\omega))$, 
но и~<<американского>>, <<азиатского>> и~прочих типов~\cite{wilmott_95}:
$$
q_T (\omega) = q_T (S_{[0,T]}(\omega)).
$$

Известно, что модель~(\ref{eq:S_sys_P}), (\ref{eq:Z_sys}) описывает неполный рынок~\cite{ZhangElliottSiuGuo_11}.
\textit{Задача пополнения рынка} заключается в~определении на исследуемом стохастическом базисе дополнительных 
финансовых инструментов таким образом, чтобы новая модель описывала полный рынок, 
а~расширенный набор финансовых инструментов позволял воспроизводить любое платежное требование~$q_T$.

\section{Выбор множества пополняющих деривативов}

Прежде всего отметим, что в~случае $a^{\ell}(t) \hm\equiv a(t)$ и~$\sigma^{\ell}(t) \hm\equiv \sigma(t)$ для 
любых $t \hm\in [0,T]$, $\ell \hm= \overline{1,L}$ модель~(\ref{eq:S_sys_P}), (\ref{eq:Z_sys}) превращается в~классическую модель
 Блэ\-ка--Шо\-ул\-за~\cite{shiryaev1999essentials} и~рынок становится полным. 
 В~этом частном случае и~мгновенная процентная ставка, и~волатильность рисковых активов становятся неслучайными и~известными трейдерам. 
 В~общем же случае модели параметры~$a^{\ell}(\cdot)$ и~$\sigma^{\ell}(\cdot)$ для них скрыты. Идея пополнения рынка заключается
  в~одновременном получении информации об~МСП~$Z$ и~построении дополнительных финансовых инструментов, позволяющих эту информацию использовать: 
  по сути дела, торговать~ею.

Один из вариантов пополнения рынка~(\ref{eq:S_sys_P}), (\ref{eq:Z_sys}) представлен в~работах~\cite{ZhangElliottSiuGuo_11, Courcuera_05}. 
Пусть $J_t^{\ell}$~--- процесс, считающий число скачков МСП~$Z$ в~состояние~$e_{\ell}$, произошедших на отрезке времени $[0,t]$. 
Данный процесс допускает следующее мартингальное разложение:
$$
J_t^{\ell} \hm= \int\limits_0^t \sum\limits_{i:\; i \neq \ell} \Lambda_{i\ell}(u)Z_u^i\, du \hm+
\int\limits_0^t (1-Z_{u-}^{\ell})\,dM_u^{\ell}.
%\label{eq:eq_jump_sec_1}
$$
Для пополнения рынка предлагалось ввести дополнительно~$L$ (по числу возможных режимов рынка) скачкообразных финансовых инструментов, связанных с~процессами~$J_t^{\ell}$. 
Их цены описывались векторным процессом $G_t \triangleq \col(G_t^1,\ldots,G_t^L)$, компоненты которого имеют вид
$$
G_t^{\ell} =  \exp \left( \int\limits_0^t r_u\,du \right)
\int\limits_0^t (1-Z_{u-}^{\ell})\,dM_u^{\ell},\enskip \ell = \overline{1,L}.
$$
%\label{eq:eq_jump_sec_2}
В~\cite{ZhangElliottSiuGuo_11} доказано утверждение, согласно которому для каждого платежного требования~$q_T$ существует самофинансируемый 
$\ff_t$-пред\-ска\-зу\-емый портфель $(\varpi, \pi, \Pi)$ ($\Pi_t \triangleq \row (\Pi_t^{1},\ldots, \Pi_t^{L})$), воспроизводящий~$q_T$:
\begin{multline*}
q_T = \varpi_0  B_0+ \pi_0 S_0 + \Pi_0 G_0 + {}\\
{}+\int\limits_0^T \left(\varpi_u \,dB_u + \pi_u\, dS_u
+ \Pi_u\,dG_u\right) \ \ppp\mbox{-п.~н.}
\end{multline*}
Подобное пополнение рынка представляется искусственным. Во-пер\-вых, введение на рынок скачкообразных инструментов 
предполагает доступность МСП~$Z$ точному наблюдению для всех \mbox{трейдеров}. Во-вто\-рых, 
из определения новых инструментов следует, что цены~$G_t^{\ell}$ могут с~положительной вероятностью принимать отрицательные значения.

В данной работе предлагается провести пополнение рынка более <<естественными>> финансовыми инструментами, представляющими собой 
деривативы от уже имеющихся на рынке. Прежде всего заметим, что рынок~(\ref{eq:S_sys_P}), (\ref{eq:Z_sys})~--- безарбитражный~\cite{Criens_20}. 
Это, в~свою очередь, обеспечивает существование на измеримом пространстве~$(\Omega, \ff)$ \textit{мартингальной} вероятностной меры~$\qqq$ ($\qqq \hm\sim \ppp$),
относительно которой процесс цен $S_t$ описывается решением СДС
 \begin{multline}
  dS_t =r_t S_t \,dt + \mathrm{diag}\, S_t \sigma(t,Z_t)\,dw_t^{\qqq},\\
  t \in (0,T], \ S_0 \sim p_0(s),
  \label{eq:S_sys_Q}
  \end{multline}
  в~которой $w_t^{\qqq} \in \mathbb{R}^N$~--- $\ff_t$-со\-гла\-со\-ван\-ный винеровский процесс относительно~$\qqq$.

  В силу неполноты рынка мартингальная мера~$\qqq$ не является единственной. Будем считать, что $\qqq$~--- одна из мартингальных мер, 
  \textit{преобладающая}, для которой помимо представления~$S_t$ в~форме~(\ref{eq:S_sys_Q}) выполняется следующее условие.
  \begin{enumerate}
  \item[7.] Процесс $M_t$ является мартингалом относительно меры~$\qqq$.
  \end{enumerate}
  Цена $q_t$ платежного требования~$q_T$ в~момент времени $t \hm\in [0,T]$ служит дисконтированным условным средним
 $$
 q_t = e^{-\int\nolimits_t^T r_s\,ds}\me{\qqq}{q_T|\ff_t}.
%\label{eq:F_1}
$$
  Процесс $\mu_t \triangleq \me{\qqq}{q_T|\ff_t}$, будучи $\ff_t$-со\-гла\-со\-ван\-ным $\qqq$-мар\-тин\-га\-лом, допускает разложение~\cite{Elliott_76}:
\begin{equation}
\mu_t = \mu_0 + \int\limits_0^t \xi_u \,dw_u^{\qqq}+  \int\limits_0^t \Xi_u \,dM_u,
\label{eq:mu_1}
\end{equation}
в котором $\xi_t \triangleq
\row (\xi_t^1,\ldots,\xi_t^N)$ и~$\Xi_t \triangleq\linebreak \triangleq
\row (\Xi_t^1,\ldots,\Xi_t^L)$, $t \hm\in [0,T]$,~--- некоторые $\ff_t$-пред\-ска\-зу\-емые интегранды.

Пусть $H(S_T)$~--- некоторое платежное требование, определяющее
дериватив <<европейского типа>>, построенный на имеющихся базовых активах.
Его цена~$F_t$ в~момент времени $t \hm\in [0,T]$ определяется формулой
 $$
F_t = e^{-\int\nolimits_t^T r_s\,ds}\me{\qqq}{H(S_T)|\ff_t}.
%\label{eq:F_1}
$$
В~предположении, что существует детерминированная функция цены $F(t,s,z):[0,T] \times \mathbb{R}^N \times  \mathbb{S}^L$, 
такая что $\overline{F}_t \hm =F(t,S_t(\omega), Z_t(\omega))$ $\ppp$-п.~н., в~\cite{B_23_1_IA} показано, что цена дериватива 
выражается следующим образом:
$$
\overline{F}_t  =\sum\limits_{\ell=1}^L Z_t^{\ell}(\omega) F^{\ell}(t,S_t(\omega)).
$$ 
Здесь век\-тор-фун\-кция $F(t,s)  \triangleq \col (F^1(t,s),\ldots\linebreak \ldots,F^L(t,s))$ пред\-став\-ля\-ет собой 
решение сле\-ду\-ющей сис\-те\-мы дифференциальных уравнений в~частных производных~--- 
обобщения классического уравнения Блэ\-ка--Шо\-ул\-за (зависимость функций от своих аргументов опущена):
\begin{equation}
\left.
\begin{array}{rl}
\!\!\!\hspace*{-1mm}F_t^{\ell} &= \displaystyle rF^{\ell} \! - \!\! \sum\limits_{j=1}^L \!\Lambda_{\ell j}F^{j} \! - \!\! \sum\limits_{n=1}^N \!F_{s^n}^{\ell}s^n(r - a_n^{\ell}) \! - {}\\[6pt]
&\hspace*{-1mm}\!\!\!\hspace*{-5mm}{}-\! \fr{1}{2} \!\sum\limits_{i,j=1}^N  \!\!F_{s^i,s^j}^{\ell}s^is^j b_{ij}^{\ell},\;\;  \ell=\overline{1,L}, \;\;
t \in [0,T];\\[12pt]
\!\!\!\hspace*{-1mm}F^{\ell}(T,s) &= H(s).
\end{array}\!
\right\}\!\!
\label{eq:F_m_eq}
\end{equation}
В данной системе использовано следующее обозначение:
$$
\sigma^{\ell}(t) \left( \sigma^{\ell}(t) \right)^{\top}= b^{\ell}(t) = \|b_{ij}^{\ell}(t)\|_{i,j=\overline{1,L}}.
$$
При этом процесс~$F_t$ допускает следующий стохастический дифференциал относительно мартингальной меры~$\qqq$:
\begin{multline}
dF_t =
r_tF_t\,dt + \sum\limits_{\ell=1}^L F^{\ell}(t,S_t)\,dM_t^{\ell} +{}\\
{}+
\sum\limits_{\ell=1}^L Z_t^{\ell} \nabla_s F^{\ell}(t,S_t) \mathrm{diag}\left(S_t\right)
 \sigma^{\ell}(t)\,dw_t^{\qqq},
 \label{eq:F_m_obs}
\end{multline}
где $
\nabla_s F^{\ell} \triangleq \row \left(
F_{s^1}^{\ell},\ldots, F_{s^N}^{\ell}
\right)$.

Для пополнения рынка введем дополнительно~$L$ деривативов, соответствующих следующему платежному требованию: $H(S_T) 
\triangleq \col (H^1(S_T), \ldots\linebreak \ldots , H^L(S_T))$ ($T$~--- общая дата исполнения всех требований).
Объединим цены в~векторный процесс $\overline{F}_t \triangleq \col (\overline{F}_t^1,\ldots,\overline{F}_t^L)$. 
Относительно $H(\cdot)$ сделано следующее предположение.
\begin{enumerate}
\item[8.] Матрица
\begin{equation*}
\mathbf{F}(t,s) \triangleq
\left[
\begin{array}{cccc}
F^{11}(t,s) & F^{12}(t,s) & \cdots & F^{1L}(t,s) \\
F^{21}(t,s) & F^{22}(t,s) & \cdots & F^{2L}(t,s) \\
\vdots & \vdots & \vdots & \vdots \\
F^{L1}(t,s) & F^{L2}(t,s) & \cdots & F^{LL}(t,s)
\end{array}
\right]
\end{equation*}
невырождена почти везде на множестве \mbox{$[0,T] \times \mathbb{R}_+^{L}$}.
\end{enumerate}

 Заметим, что $k$-я строка матрицы $\mathbf{F}(t,s)$ $(F^{k1}(t,s) , F^{k2}(t,s) , \ldots , F^{kL}(t,s))$ представляет собой решение системы~(\ref{eq:F_m_eq})
 с~терминальным условием $F^{k\ell}(T,s) \hm\equiv H^k(s)$ ($\ell \hm= \overline{1,L}$) и~эта строка полностью задает функцию цены $k$-го дериватива.

Легко видеть, что предположение 8 гарантирует, что МСП
$Z_t$ согласован с~фильтрацией {$\mathcal{S}_t \vee \overline{\mathcal{F}}_t$} 
(здесь {$\overline{\mathcal{F}}_t \triangleq \sigma \{\overline{F}_u: \; 0 \hm\leqslant u\hm \leqslant t\}$}):
\begin{equation}
Z_t = \mathbf{F}^{-1}(t,S_t)\overline{F}_t
\label{eq:Z_through_F}
\end{equation}
и, следовательно, при выполнении предположений~1--8 для любых  $t \hm\in [0,T]$ имеет место совпадение фильтраций
 $\ff_t \hm\equiv \mathcal{S}_t \vee \overline{\mathcal{F}}_t$.
 Поэтому стохастический дифференциал $\qqq$-мар\-тин\-га\-ла~$w^{\qqq}$ принимает вид:
\begin{multline}
dw_t^{\qqq} = \underbrace{
\sum\limits_{\ell=1}^L \left( \mathbf{F}^{-1}(t,S_t)\overline{F}_t\right)^{\ell}
(\sigma^{\ell})^{-1}(t) \mathrm{diag}^{-1} (S_t)}_{\triangleq \gamma_t} \times{}\\
{}\times
\left(dS_t - r_tS_t \,dt \right).
\label{eq:w_Q_through_F}
\end{multline}
Из (\ref{eq:F_m_obs}) и~(\ref{eq:w_Q_through_F}) следует, что
\begin{multline*}
d\overline{F}_t^k = r_t\overline{F}_t^k dt  + \sum\limits_{\ell=1}^L F^{k\ell}(t,S_t)\,dM_t^{\ell} +{} \\ 
{} +
\underbrace{\sum\limits_{\ell=1}^L  \left( \mathbf{F}^{-1}(t,S_t)\overline{F}_t\right)^{\ell} \nabla_s F^{k\ell}(u, S_u)}_{\triangleq \Gamma_t^k} \;
\!\left( dS_t-r_tS_t\,dt
\right), \\
k=\overline{1,L},
%\label{eq:F_through_F}
\end{multline*}
и в~векторном виде эволюция цен деривативов описывается с~помощью следующей СДС (здесь $\Gamma_t\hm = \col(\Gamma_t^1, \ldots, \Gamma_t^L)$):
 \begin{multline*}
d\overline{F}_t = r_t\overline{F}_t\,dt + \mathbf{F}_t \,dM_t +
\Gamma_t\left(
dS_t-r_tS_t\,dt
\right), \\
\overline{F}_0\ \mbox{--- начальное условие.}
%\label{eq:F_through_FF}
\end{multline*}
Таким образом, предположение~8 позволяет выразить мартингал~$M_t$, участвующий в~представлении МСП~$Z_t$:
\begin{equation}
dM_t = \mathbf{F}^{-1}_t \left[ d\overline{F}_t - r_t\overline{F}_t\,dt -\Gamma_t\left( dS_t-r_tS_t\,dt \right)
\right]\!.\!
\label{eq:M_through_SF}
\end{equation}

Вернемся к~мартингалу $\mu_t$~(\ref{eq:mu_1}),
представляющему произвольное платежное требование, и~сконструируем для него воспроизводящий портфель 
$(\pi_t,\Pi_t, \varpi_t)$, обладающий свойством самофинансирования. Векторный процесс $\pi_t 
\triangleq \row (\pi_t^1, \ldots, \pi_t^N)$ описывает часть включенных в~портфель базовых активов, $\Pi_t 
\triangleq \row (\Pi_t^1, \ldots, \Pi_t^L)$ выполняет ту же функцию для предложенных деривативов, 
а~скалярный процесс~$\varpi_t$ определяет долю портфеля, находящуюся на банковском депозите. Выбор доли депозита в~портфеле
\begin{equation}
\varpi_t = \mu_t - B_t^{-1}(\pi_tS_t+\Pi_t\overline{F}_t)
\label{eq:varpi}
\end{equation}
обеспечивает воспроизведение~$\mu_t$. Действительно, если~$C_t$~--- текущий капитал портфеля, то
\begin{multline*}
C_t = \varpi_t B_t +  \pi_tS_t+\Pi_t\overline{F}_t ={}\\
{}= \left( \mu_t \!-\! B_t^{-1}(\pi_tS_t\! +\!\Pi_t\overline{F}_t)\right) B_t
\!+\!  \pi_tS_t \!+\!\Pi_t\overline{F}_t\! =\!  \mu_t  B_t.
\end{multline*}

Рассмотрим прибыль портфеля и~преобразуем его, используя формулы~(\ref{eq:w_Q_through_F})--(\ref{eq:varpi}) и~интегрирование по частям:
\begin{multline*}
\Delta_t  \triangleq \int\limits_0^t \varpi_u r_u B_u\, du + \int\limits_0^t \left( \pi_u dS_u + \Pi_u \,d \overline{F}_u \right) = {}\\
{} =
\int\limits_0^t \left( \mu_u - B_u^{-1}(\pi_uS_u+\Pi_u\overline{F}_u) \right) r_u B_u \,du +{}
\end{multline*}

\noindent
\begin{multline*}
{}+
  \int\limits_0^t \left( \pi_u\, dS_u + \Pi_u\, d \overline{F}_u \right) =
\int\limits_0^t  \mu_u \,dB_u -{}\\
{}- \! \int\limits_0^t \left( \pi_uS_u+\Pi_u\overline{F}_u \right) r_u \,du +
\!\int\limits_0^t \left( \pi_u\, dS_u + \Pi_u \,d\overline{F}_u \right) = {} \\
{} =
\mu_t B_t -\mu_0 - \int\limits_0^t  B_u\, d\mu_u -
\int\limits_0^t \left( \pi_uS_u+\Pi_u\overline{F}_u \right) r_u \,du +{}\\
{}+ 
\int\limits_0^t \left( \pi_u \,dS_u + \Pi_u\, d\overline{F}_u \right)  = 
\mu_t B_t -\mu_0 -{}\\
{}- 
\!\int\limits_0^t \left( \pi_uS_u+\Pi_u\overline{F}_u \right) r_u\, du +
\!\int\limits_0^t \left( \pi_u \,dS_u + \Pi_u \,d\overline{F}_u \right) - {} \\
{} - \int\limits_0^t B_u \left[ \xi_u \gamma_u (dS_u-r_u S_u \,du) + {}\right.\\
\left.{}+ \Xi_u \mathbf{F}^{-1}_u \left(
d\overline{F}_u \!-\!r_u\overline{F}_u\,du \!-\! \Gamma_u(dS_u-r_u S_u \,du) \right) \right] ={} \\
{} =
\mu_t B_t -\mu_0 +
\int\limits_0^t I_u^1 \,du + \int\limits_0^t I_u^2 \,dS_u +
\int\limits_0^t I_u^3 \,d\overline{F}_u,
\end{multline*}
где
\begin{align*}
I_t^1 &\triangleq r_t \left\{ \left[
B_t(\xi_t\gamma_t -\Xi_t\mathbf{F}^{-1}_t \Gamma_t) -\pi_t
\right]S_t + {}\right.\\
&\hspace*{25mm}\left.{}+\left[ B_t \Xi_t\mathbf{F}^{-1}_t - \Pi_t
\right] \overline{F}_t
\right\};
\\
I_t^2 &\triangleq
\pi_t - B_t(\xi_t\gamma_t -\Xi_t\mathbf{F}^{-1}_t \Gamma_t);
\\
I_t^3 &\triangleq \Pi_t - B_t \Xi_t\mathbf{F}^{-1}_t.
\end{align*}
Легко проверить, что выбор долей
$$
\pi_t = B_t(\xi_t\gamma_t -\Xi_t\mathbf{F}^{-1}_t \Gamma_t),
\enskip
\Pi_t = B_t \Xi_t\mathbf{F}^{-1}_t
$$
гарантирует портфелю свойство самофинансирования:
$$
\Delta_t \hm= \mu_t B_t \hm- \mu_0.
$$

Таким образом, доказана


\smallskip

\noindent
\textbf{Теорема~1.}\ 
\textit{Если предположения}~1--8 \textit{верны, то рынок}~(\ref{eq:S_sys_P}), (\ref{eq:Z_sys}) 
\textit{может быть пополнен набором~$L$ производных финансовых инструментов}.

\section{Заключение}

Представленный способ пополнения рынка позволяет сделать следующие выводы.
\begin{enumerate}[1.]\item 
Для пополнения использовались деривативы европейского типа, однако портфели, построенные на них и~имеющихся базовых активах, 
позволяют воспроизводить платежные требования любого типа: американского, азиатского, бермудского и~пр. 
\item Предложенный в~доказательстве портфель воспроизводит не только платежное требование в~момент погашения~$T$, 
но и~всю траекторию цены этого требования на отрезке $[0,T]$. 
\item Предложенные для пополнения инструменты выглядят более естественными, чем предложенные в~\cite{ZhangElliottSiuGuo_11, Courcuera_05}. 
\end{enumerate}

Вообще говоря, и~в~ци\-ти\-ру\-емых работах, и~в~данной заметке используется единый подход. Изначально рынок содержит скрытый случайный процесс, 
недоступный наблюдению для участников рынка. Он порождает дополнительный риск, обладающий некоторой ценой. Далее на основе 
этого процесса строятся некоторые инструменты с~наблюдаемыми ценами, которыми и~предлагается пополнить рынок. 
В~\cite{ZhangElliottSiuGuo_11, Courcuera_05} в~качестве этих активов выступают мартингалы процессов, считающих 
скачки МСП в~то или иное состояние. 

Подобный выбор пред\-став\-ля\-ет\-ся не вполне естественным. Цены могут принимать отрицательные величины, 
МСП является скрытым, и~с~практической точки зрения не вполне понятен источник наблюдений этого процесса. 
Предложенный в~данной работе способ пополнения выглядит более привлекательно: в~качестве пополняющих используются <<естественные>> 
деривативы европейского типа, по\-стро\-ен\-ные на име\-ющих\-ся на рынке базовых активах.
{ %\looseness=1

}

Результаты, представленные в~данной заметке, имеют, на первый взгляд, лишь академический интерес. 
Действительно, при выполнении предположения~8 скрытый режим рынка~$Z_t$ может быть\linebreak
 восстановлен точно с~по\-мощью 
элементарного алгебраического преобразования~(\ref{eq:Z_through_F}) без привлечения методов оптимальной нелинейной фильт\-ра\-ции~\cite{B_21_1_IA}. 
Это означает, что наличие точно \mbox{наблюдаемых} в~непрерывном времени цен деривативов исключает статистическую неопределенность 
процентной ставки и~волатильности. В~\cite{B_23_3_IA}\linebreak приведена аргументация против наличия наблюдений цен базовых активов 
и~деривативов в~непрерывном времени, а~также воз\-мож\-ности наблюдать точные цены деривативов и~в~дискретные 
моменты \mbox{времени}. Эти факты подталкивают к~разработке высокоточных методов оценивания состояний скрытого МСП по разнородным косвенным 
зашумленным наблюдениям. 
Подобные алгоритмы были\linebreak
 предложены в~\cite{B_23_3_IA, B_23_4_IA} для двух разных видов статистической информации. 

Перспективным продолжением исследований в~данной об\-ласти пред\-став\-ля\-ет\-ся решение задачи хеджирования на рынках 
подобного типа с~использованием концепции фильт\-ро\-ван\-но\-го пространства. 

Другое на\-прав\-ле\-ние подразумевает проведение 
аналогичных исследований для моделей Хал\-ла--Уай\-та и~Васичека с~марковскими переключениями.
{ %\looseness=1

}


{\small\frenchspacing
 {\baselineskip=11.5pt
 %\addcontentsline{toc}{section}{References}
 \begin{thebibliography}{99}
   \bibitem{B_23_1_IA}
\Au{Борисов~A.} Рынок с~марковской скачкообразной волатильностью I: мониторинг цены риска как задача оптимальной фильтрации~// 
Информатика и~её применения,~2023. Т.~17. Вып.~2. С.~27--33. doi: 10.14357/19922264230204. EDN: GAXCHQ.

   \bibitem{B_23_2_IA}
\Au{Борисов A.} Рынок с~марковской скачкообразной волатильностью II: алгоритм вычисления справедливой цены деривативов~// 
Информатика и~её применения,~2023. Т.~17. Вып.~3. С.~18--24. doi: 10.14357/ 19922264230303. EDN: DNXJGB.

   \bibitem{B_23_3_IA}
\Au{Борисов A.} Рынок с~марковской скачкообразной волатильностью III:
алгоритм мониторинга цены риска по дискретным наблюдениям цен~// Информатика и~её применения,~2023. Т.~17. Вып.~4. С.~9--16. 
doi: 10.14357/19922264230402. EDN: OFYELT.

   \bibitem{B_23_4_IA}
\Au{Борисов A.} Рынок с~марковской скачкообразной волатильностью IV: алгоритм мониторинга рыночной цены риска 
по потоку высокочастотных наблюдений базовых активов и~деривативов~// Информатика и~её применения,~2024. 
Т.~18. Вып.~1. С.~26--32. doi: 10.14357/19922264240104. EDN: ZRQKIT.

\bibitem{Criens_20}
\Au{Criens D.} No arbitrage in continuous financial markets~//
Math. Financ. Econ.,~2020.~Vol.~14.~P.~461--506.
doi: 10.1007/s11579-020-00262-1.

   \bibitem{Fouque_00}
\Au{Fouque J., Papanicolaou~G., Sircar~K.}
Derivatives in financial markets with stochastic volatility.~--- Cambrige, U.K.: Cambrige University Press,~2000.~218~p.

\bibitem{Courcuera_05}
\Au{Courcuera J., Nualart~D., Schoutens~W.} Completion of a~L$\acute{\mbox{e}}$vy market by power-jump assets~// Financ. 
Stoch., 2005. Vol.~9. Iss.~1. P.~109--127. doi: 10.1007/s00780-004-0139-2.

 \bibitem{ZhangElliottSiuGuo_11}
\Au{Zhang X., Elliott R., Siu~T., Guo~J.}
Markovian regime-switching market completion using additional Markov
  jump assets~//
IMA J.~Manag. Math., 2011. Vol.~23. Iss.~3. P.~283--305.
doi: 10.1093/imaman/dpr018.


\bibitem{ElliottAggounMoore_08}
\Au{Elliott R., Aggoun~L., Moore~J.} Hidden Markov models: Estimation and control.~--- New York,~NY, USA: Springer, 2010. 382~p.

  \bibitem{liptser2012theory}
\Au{Liptser R., Shiryayev~A.}
Theory of martingales; mathematics and its applications.~--- Amsterdam, The Netherlands:   Springer, 2012. 806~p.

\bibitem{shiryaev1999essentials}
\Au{Shiryayev A.} Essentials of stochastic finance: Facts, models, theory.~--- New Jersey, NJ, USA: World Scientific, 1999. 834~p.

  \bibitem{wilmott_95}
\Au{Wilmott P., Howison~S., Dewynne~J.}
The mathematics of financial derivatives: A~student introduction.~--~Cam-\linebreak\vspace*{-12pt}

\pagebreak

\noindent
brige, U.K.:   Cambrige University Press, 1995. 317~p.
 doi: 10.1017/CBO9780511812545.
 
% \pagebreak

\bibitem{Elliott_76}
\Au{Elliott R.}
Double martingales~// Z. Wahrscheinlichkeit., 1976. Vol.~34. P.~17--28. doi: 10.1007/BF00532686.

\columnbreak

     \bibitem{B_21_1_IA}
\Au{Борисов A., Казанчян~Д.} Фильтрация состояний марковских скачкообразных процессов по комплексным наблюдениям I: точное решение задачи~// 
Информатика и~её применения, 2021. Т.~15. Вып.~2. С.~12--19. doi: 10.14357/19922264210202. EDN: NKCTNS.

\end{thebibliography}

 }
 }

\end{multicols}

\vspace*{-6pt}

\hfill{\small\textit{Поступила в~редакцию 05.02.24}}

\vspace*{10pt}

%\pagebreak

%\newpage

%\vspace*{-28pt}

\hrule

\vspace*{2pt}

\hrule



\def\tit{MARKET WITH MARKOV JUMP VOLATILITY V: MARKET~COMPLETION WITH~DERIVATIVES}


\def\titkol{Market with Markov jump volatility V: Market completion with~derivatives}


\def\aut{A.\,V.~Borisov}

\def\autkol{A.\,V.~Borisov}

\titel{\tit}{\aut}{\autkol}{\titkol}

\vspace*{-8pt}


\noindent
Federal Research Center ``Computer Science and Control'' of the Russian Academy of 
Sciences, 44-2~Vavilov Str., Moscow 119333, Russian Federation

\def\leftfootline{\small{\textbf{\thepage}
\hfill INFORMATIKA I EE PRIMENENIYA~--- INFORMATICS AND
APPLICATIONS\ \ \ 2024\ \ \ volume~18\ \ \ issue\ 2}
}%
 \def\rightfootline{\small{INFORMATIKA I EE PRIMENENIYA~---
INFORMATICS AND APPLICATIONS\ \ \ 2024\ \ \ volume~18\ \ \ issue\ 2
\hfill \textbf{\thepage}}}

\vspace*{4pt}



\Abste{The final, fifth, part of the series is devoted to a replenishment procedure of the market with 
a~Markov jump regime change. The market includes a riskless bank deposit with a~known nonrandom interest rate and a~set of underlying risky assets. 
The instantaneous interest rates and volatilities of the assets are the functions of the hidden regime change factor described by some
 finite state Markov jump process. The purpose of this article is to complete the investigated market.
  It means the market enlargement by a set of auxiliary financial instruments. 
  The point is that any contingent claim declared in the market can be replicated with 
  some self-financing portfolio containing the original and auxiliary instruments. For the market completion, 
  it is enough to include European-style derivatives built on underlying risky assets already on the market.
In this case, the number of added derivatives coincides with the number of market modes. The problem of 
replenishing the market has a~nonunique solution and the article compares the proposed replenishment method with the existing one.}

\KWE{Markov jump process; financial portfolio; self-financing property;
market completeness}



\DOI{10.14357/19922264240202}{DQFSDO}

\vspace*{-12pt}

\Ack

\vspace*{-3pt}


     \noindent
     The research was carried out using the infrastructure of the Shared Research Facilities ``High Performance Computing and Big Data'' 
(CKP ``Informatics'') of FRC CSC RAS (Moscow).


  \begin{multicols}{2}

\renewcommand{\bibname}{\protect\rmfamily References}
%\renewcommand{\bibname}{\large\protect\rm References}

{\small\frenchspacing
 {%\baselineskip=10.8pt
 \addcontentsline{toc}{section}{References}
 \begin{thebibliography}{99} 
%1
\bibitem{B_23_1_IA-1} 
\Aue{Borisov, A.} 2023. 
Rynok s~markovskoy skachkoobraznoy volatil'nost'yu I: monitoring tseny riska kak zadacha optimal'noy fil'tratsii 
[Market with Markov jump volatility I: Price of risk monitoring as an optimal filtering problem]. 
\textit{Informatika i~ee Primeneniya~--- Inform. Appl.} 17(2):27--33. 
doi: 10.14357/19922264230204. EDN: GAXCHQ.

%2
\bibitem{B_23_2_IA-1}
\Aue{Borisov, A.} 2023. 
Rynok s`markovskoy skachkoobraznoy volatil'nost'yu II: algoritm vychisleniya spravedlivoy tseny derivativov 
[Market with Markov jump volatility II: Algorithm of derivative fair price calculation].
\textit{Informatika i~ee primeneniya~--- Inform. Appl.} 17(3):18--24. 
doi: 10.14357/19922264230303. EDN: DNXJGB.

%3
\bibitem{B_23_3_IA-1}
\Aue{Borisov, A.} 2023. 
Rynok s~markovskoy skachkoobraznoy volatil'nost'yu III: algoritm monitoringa tseny riska po diskretnym nablyudeniyam tsen aktivov 
[Market with Markov jump volatility III: Price of risk monitoring algorithm given discrete-time observations of asset prices]. 
\textit{Informatika i~ee primeneniya~--- Inform. Appl.} 17(4):9--16. 
doi: 10.14357/19922264230402. EDN: OFYELT.

%4
\bibitem{B_23_4_IA-1}
\Aue{Borisov, A.} 2024. Rynok s~markovskoy skachkoobraznoy volatil'nost'yu IV: algoritm monitoringa rynochnoy tseny riska po potoku 
vysokochastotnykh nablyudeniy bazovykh aktivov i~derivativov [Market with Markov jump volatility IV: 
Price of risk monitoring algorithm given high frequency observation flows of assets prices]. 
\textit{Informatika i~ee primeneniya~--- Inform. Appl.} 18(1):26--32. doi: 10.14357/19922264240104. EDN: ZRQKIT.

%5
\bibitem{Criens_20-1}
\Aue{Criens, D.} 2020. 
No arbitrage in continuous financial markets. \textit{Math. Financ. Econ.} 14:461--506. 
doi: 10.1007/s11579-020-00262-1.

%6
\bibitem{Fouque_00-1}
\Aue{Fouque, J., G.~Papanicolaou, and K.~Sircar.} 2000.
\textit{Derivatives in financial markets with stochastic volatility}. Cambrige, U.K.: Cambrige University Press. 218~p.

%7
\bibitem{Courcuera_05-1}
\Aue{Courcuera, J., D.~Nualart, and W.~Schoutens.} 2005.
Completion of a L$\acute{\mbox{e}}$vy market by power-jump assets. \textit{Financ. Stoch.} 9(1):109--127. doi: 10.1007/s00780-004-0139-2.

%8
\bibitem{ZhangElliottSiuGuo_11-1}
\Aue{Zhang X., R.~Elliott, T.~Siu, and J.~Guo.} 2011.
Markovian regime-switching market completion using additional Markov jump assets.
\textit{IMA J. Manag. Math.} 23(3):283--305.
doi: 10.1093/imaman/dpr018.

%9
\bibitem{ElliottAggounMoore_08-1}
\Aue{Elliott, R., L.~Aggoun, and J.~Moore.} 2010. \textit{Hidden Markov models: Estimation and control}. New York, NY: Springer. 382~p.

%10
\bibitem{liptser2012theory-1}
\Aue{Liptser, R., and A.~Shiryayev.} 2012.
\textit{Theory of martingales. Mathematics and its applications}. Amsterdam, The Netherlands: Springer. 806~p.

%11
\bibitem{shiryaev1999essentials-1}
\Aue{Shiryayev, A.} 1999. 
\textit{Essentials of stochastic finance: Facts, models, theory}. New Jersey, NJ: World Scientific. 834~p.

%12
\bibitem{wilmott_95-1}
\Aue{Wilmott, P., S.~Howison, and J.~Dewynne.} 1995.
\textit{The mathematics of financial derivatives: A~student introduction}.
Cambrige, U.K.: Cambrige University Press. 317~p. doi: 10.1017/CBO9780511812545.

%13
\bibitem{Elliott_76-1}
\Aue{Elliott, R.} 1976.
Double martingales. \textit{Z. Wahrscheinlichkeit.} 34:17--28.
doi: 10.1007/BF00532686.

%14
\bibitem{B_21_1_IA-1}
\Aue{Borisov, A., and D.~Kazanchyan.} 2021. Fil'tratsiya sostoyaniy markovskikh skachkoobraznykh protsessov po kompleksnym nablyudeniyam I: 
tochnoe reshenie zadachi [Filtering of Markov jump processes given composite observations I: Exact solution]. 
\textit{Informatika i~ee primeneniya~--- Inform. Appl.} 15(2):12--19. doi: 10.14357/19922264210202. EDN: NKCTNS.


\end{thebibliography}

 }
 }

\end{multicols}

\vspace*{-6pt}

\hfill{\small\textit{Received February 5, 2024}} 

\vspace*{-18pt}


\Contrl

\vspace*{-3pt}

\noindent
\textbf{Borisov Andrey V.} (b.\ 1965)~--- 
Doctor of Science in physics and mathematics, principal scientist, Federal Research Center ``Computer Science and Control''
 of the Russian Academy of Sciences, 44-2~Vavilov Str., Moscow 119333, Russian Federation; \mbox{aborisov@frccsc.ru}






\label{end\stat}

\renewcommand{\bibname}{\protect\rm Литература} 