\def\stat{vasiliev}

\def\tit{О ФУНКТОРНОМ ПРЕДСТАВЛЕНИИ ОПТИМИЗИРУЕМЫХ ДИНАМИЧЕСКИХ 
МУЛЬТИАГЕНТНЫХ СИСТЕМ}

\def\titkol{О функторном представлении оптимизируемых динамических 
мультиагентных систем}

\def\aut{Н.\,С.~Васильев$^1$}

\def\autkol{Н.\,С.~Васильев}

\titel{\tit}{\aut}{\autkol}{\titkol}

\index{Васильев Н.\,С.}
\index{Vasilyev N.\,S.}


%{\renewcommand{\thefootnote}{\fnsymbol{footnote}} \footnotetext[1]
%{Исследование выполнено за счет гранта Российского научного фонда №\,22-28-00588, {\sf 
%https://rscf.ru/project/22-28-00588/}. Работа проводилась с~использованием инфраструктуры Центра 
%коллективного пользования <<Высокопроизводительные вычисления и~большие данные>> (ЦКП 
%<<Информатика>> ФИЦ ИУ РАН, Москва).}}


\renewcommand{\thefootnote}{\arabic{footnote}}
\footnotetext[1]{Московский государственный технический университет имени Н.\,Э.~Баумана, \mbox{nik8519@yandex.ru}}

\vspace*{2pt}






    \Abst{Топос функторов выбран в~качестве компьютерного инструмента синтеза 
динамических игр многих лиц. Задаваемая шкала упорядочивает объекты, 
отвечающие сопутствующим статическим подыграм. Последние служат 
состояниями динамической мультиагентной сис\-те\-мы (ДМАС). Исходная 
динамическая игра и~все статические подзадачи пред\-став\-ля\-ют\-ся в~моноидальной 
категории бинарных отношений. Под рациональным решением игры понимается 
равновесие. Композициональное строение оп\-ти\-ми\-зи\-ру\-емой ДМАС выражено 
в~форме динамического результирующего отношения (ДРО) игры. Поиску равновесия 
отвечает максимизация ДРО. Это делается методом Беллмана, обобщенным на 
задачи оптимального управления, поставленные в~форме отношений. Программная 
реализация предложенного подхода может быть основана на нейросетевых 
вычислениях ввиду согласованности архитектур применяемых графов отношений и~нейросетей.}
    
    \KW{категория функторов; композициональность; моноидальная категория; 
обратный образ; динамическое отношение игры; статическая подыгра; отношение 
предпочтения; динамическое результирующее отношение; рациональное решение; 
морфизм Беллмана}

\DOI{10.14357/19922264240201}{CLMBXC}
  
%\vspace*{-6pt}


\vskip 10pt plus 9pt minus 6pt

\thispagestyle{headings}

\begin{multicols}{2}

\label{st\stat}
    
\section{Введение}

    В ДМАС агенты принимают 
решения в~каждом состоянии системы в~за\-ви\-си\-мости от рас\-по\-ла\-га\-емой ими 
информации о~поведении других участников конфликта~[1--4]. Воздействие 
стратегий на ДМАС распределено во времени и~связано с~выбором ситуаций 
в~череде со\-пут\-ст\-ву\-ющих статических игровых подзадач. 
    
    Процесс изменения состояний ДМАС традиционно задают с~по\-мощью 
дифференциальных или итерационных уравнений, в~которые входят управ\-ля\-ющие 
воздействия игроков. От этого описания динамики приходится отходить даже 
в~антагонистических дифференциальных играх с~полной ин\-фор\-ми\-ро\-ван\-ностью 
игроков. Управляемую систему представляют дифференциальным уравнением 
в~контингенциях~\cite{3-vas}, т.\,е.\ применяют многозначные отоб\-ра\-же\-ния. 
В~стохастических постановках так\-же возникает дополнительная не\-опре\-де\-лен\-ность, 
связанная с~прогнозированием будущих со\-сто\-яний сис\-те\-мы~\cite{4-vas}. 
    
    Зачастую целесообразно отказаться от функционального описания динамики 
игры и~перейти на\linebreak язык отношений~[5]. Благодаря этому значительно расширяется 
круг приложений~\cite{4-vas, 6-vas, 7-vas, 8-vas, 9-vas}, приобретается свойство 
композициональности моделей ДМАС, удобное для модификации игровых \mbox{задач}. 
Этот подход поддерживается компьютерной алгеброй тео\-рии категорий~\cite{10-vas, 11-vas}. 
Кроме того, графовая структура отношений и~сетевая структура игровой задачи 
допускают эффективную нейросетевую программную реализацию~\cite{6-vas, 7-vas, 12-vas, 13-vas}. 
    
    Категорный подход охватывает общий случай динамических игр многих лиц 
    с~разнообразными классами допустимых стратегий игроков~\cite{5-vas, 11-vas}. 
В~качестве со\-сто\-яний динамической системы рассматриваются со\-пут\-ст\-ву\-ющие 
статические игры, связанные динамическим отношением.
    
    Традиционное моделирование игровой задачи проводится средствами 
категории множеств SET. Естественным обобщением классического подхода 
становится формализация игровой операции, вы\-пол\-ня\-емая на языке моноидальной 
категории бинарных отношений REL. Применяемый аппарат позволяет 
единообразно выражать и~модифицировать интересы агентов, проводить 
эквивалентные преобразования игровой задачи, описывать поиск рационального 
решения~\cite{5-vas}. С~по\-мощью введения ДРО игры исследование ДМАС может быть сведено 
к~последовательной максимизации ДРО.
    
    Итак, вместо многошагового процесса игры предлагается перейти 
    к~представлению динамики задачи с~по\-мощью функтора $\tau: S\hm\to \mathrm{SET}_T$, 
который преобразует вы\-би\-ра\-емую шкалу~$S$ в~категорию $\mathrm{SET}_T$ монады 
$(T,\eta,\psi)$, $T: A\hm\to 2^A$~\cite{11-vas}. Объектами монады служат конечные 
множества, а~морфизмами~--- отношения между ними. Равенство $\mathrm{SET}_T\hm=\mathrm{REL}$ 
означает, что функторная модель ДМАС строится на языке отношений. 
Единообразно выражаются правила игры, интересы участников, классы стратегий 
агентов, динамика задачи и~даже алгоритм ее решения. Строгость функтора 
вложения $\mathrm{SET}\hm\to \mathrm{SET}_T$ обеспечивает пре\-емст\-вен\-ность новой формы 
пред\-став\-ле\-ния задачи и~классической.
    
    Функтор $\tau$ порождает динамическое отношение игры $R\hm= R(\tau)$, 
регламентирующее смену со\-сто\-яний~--- ре\-ша\-емых в~текущий момент времени 
статических подыгр $\Gamma(X)$ с~множеством \mbox{до\-пус\-ти\-мых} ситуаций~$X$. 
Динамическое отношение определяет по\-сле\-до\-ва\-тель\-ность смены со\-сто\-яний 
$\Gamma(X_0), \ldots , \Gamma(X_T)$ и~вводит отношение предшествования подыгр 
$\Gamma(X)\overset{R}{\prec} \Gamma(Y)$. В~соответствии с~ним определена 
преемственность ситуаций $x\overset{R}{\prec} y$, причем выбор любой пары 
ситуаций $x\hm\in X$ и~$y\hm\in Y$, относящихся к~разным состояниям системы, 
возможен, только если выполнено включение $(x,y)\hm\in R$. Таким образом, из 
решений подзадач $\Gamma(X_0), \ldots , \Gamma(X_T)$ строится 
по\-сле\-до\-ва\-тель\-ность $x_0, \ldots , x_T$ вы\-би\-ра\-емых ситуаций подобно тому, как это 
делается в~многошаговых играх с~по\-мощью уравнений. 
    
    Интересы агентов в~задаче заданы в~форме отношений предпочтения для 
терминального со\-сто\-яния сис\-те\-мы~$\Gamma(X^T)$: 
    \begin{equation}
    \tilde{\rho}^{kT} \in \mathrm{REL} \left( X^T, X^T\right)\,,\enskip k\in K\,.
    \label{e1.1-vas}
    \end{equation}
    
    Правила функционирования ДМАС формулируются во всех подыграх 
$\Gamma(X_i)$. Во-пер\-вых, определены цели игроков~--- максимизация 
отношений предпочтения $\rho_i^k\hm\in \mathrm{REL}\left( X_i, X_i\right)$, $k\hm\in 
K(X_i)$, по\-лу\-ча\-емых из заданных~(\ref{e1.1-vas}) с~по\-мощью вы\-чис\-ле\-ния 
обратных образов морфизмов~$\tilde{\rho}^{kT}$ относительно динамического 
отношения~$R$. Во-вто\-рых, агенты выбирают классы допустимых стратегий 
посредством графа коммуникаций и~функции, размечающей его дуги. Тем самым 
задается сетевая структура в~статических подыграх~\cite{12-vas, 13-vas}. Разметка 
графа детализирует информацию, которой обмениваются партнеры при совершении 
ходов и~формировании коалиций~\cite{5-vas}. Стратегии агентов в~исходной 
динамической задаче включают процедуру принятия решений в~каж\-дом состоянии 
сис\-темы. 
    
    Далее изучено композициональное стро\-ение динамического результирующего 
отношения игры, поз\-во\-ля\-ющее искать рациональное решение, т.\,е.\ такое, которое 
оптимизирует функционирование ДМАС. Поиск проводится методом 
динамического программирования, обоб\-щен\-ным на задачи оптимального 
управ\-ле\-ния, по\-став\-лен\-ные в~форме отношений. 

\section{Постановка задачи}

    В качестве шкалы~$S$ выберем категорию 
    \begin{equation}
 S: L \overset{\underset{\mathrm{out}}{\longrightarrow}}{\underset{\overset{\mathrm{in}}{\longrightarrow}}
 {\vphantom{{\mbox{\scriptsize{$\circ$}}}} \ }} M\,.
    \label{e2.1-vas}
    \end{equation}
    
    Динамику мультиагентной системы зададим функтором $\tau: S\hm\to \mathrm{SET}_T$, 
которому отвечает ориентированный граф динамического отношения~$\Gamma_R$, 
$R\hm=R(\tau)$. Граф должен быть ацик\-ли\-че\-ским и~связ\-ным. В~его вершинах 
находятся множества до\-пус\-ти\-мых ситуаций статических подыгр~$\Gamma(X_i)$, 
а~дуги отвечают отношениям $R_{ij}\hm\in \mathrm{REL}\left( X_i, X_j\right)$. 
    
    \smallskip
    
    \noindent
    \textbf{Определение~2.1.}\  Скажем, что ситуация $x_i\hm\in X_i$ 
непосредственно предшествует ситуации $x_j\hm\in X_j$, если и~только если $(x_i, 
x_j)\hm\in R_{ij}$. Отношением предшествования со\-сто\-яний назовем транзитивное 
замыкание $\overline{R}\hm= \overline{R}_{ij}$ динамического отношения. 
    
    \smallskip
    
    Последовательная смена состояний ДМАС происходит в~соответствии 
с~отношением непосредственного предшествования $\Gamma(X_i) 
\overset{R_{ij}}{\prec} \Gamma(X_j)$. Каж\-до\-му со\-сто\-янию соответствует некоторая 
\mbox{статическая} подыгра. Процесс функционирования сис\-те\-мы складывается из 
последовательного выбора игроками ситуаций $x_0, \ldots , x_T$, где  
$x_0$ и~$x_T$~--- решения начальной и~терминальной подыгр. Итак, имеем 
множество всех допустимых ситуаций исходной динамической задачи вида
 \begin{multline*}
    \tilde{X} ={}\\
    {}=\!\left\{\! \tilde{x} =\left( x_0, \ldots , x_T\right): \left( \forall_i x_i\in 
X_i\right) \wedge x_0\overset{R}{\prec} \cdots \overset{R}{\prec} x_T\!\right\}\!.\hspace*{-5.73415pt}
    \end{multline*}
    
    Правила проведения каждой подыгры $\Gamma(X_i)$ задаются сле\-ду\-ющим 
функтором и~функ\-цией: 
    \begin{equation}
    g_i: S\to \mathrm{SET}\,;\enskip h_i: E\to \overline{2}\,.
    \label{e2.2-vas}
    \end{equation}
Игроки стремятся по возможности максимизировать свои отношения предпочтения 
$\rho_l^k: X_l\hm\to X_l$,\linebreak $k\hm\in K_l$. (Иначе рас\-смат\-ри\-ва\-ют\-ся противоположные 
отношения $(\rho_i^k)^{\mathrm{op}}$.) Функторы~(\ref{e2.1-vas}) и~(\ref{e2.2-vas}) 
опре\-де\-ля\-ют сетевую структуру игры. Шкала~$S$ выделяет множество дуг~$E$, по 
которым участники\linebreak $k\hm\in K(X_i)$ статической игры осуществляют 
коммуникацию. Функциями in и~out в~(\ref{e2.1-vas}) вводят порядок ходов. 
Функция~$h_i$ из формулы~(\ref{e2.2-vas}) осуществляет разметку дуг графа 
коммуникаций~$\gamma_i$, определяя данные, которыми обмениваются игроки при 
своих ходах~\cite{5-vas}. Не\-пус\-тые сообщения могут содержать либо выбранную 
стра\-те\-гию-конс\-тан\-ту, либо стра\-те\-гию-функ\-цию~\cite{1-vas, 5-vas}. 
Предполагается так\-же, что никто из агентов не блефует. 
    
    Понятие рационального поведения игроков зависит от сетевой 
структуры~(\ref{e2.2-vas}) статических игр~\cite{5-vas, 12-vas, 13-vas}. Будем 
применять принцип равновесия как к~исходной динамической задаче, так и~ко всем 
по отдельности подыграм~$\Gamma(X_i)$.
    
    \smallskip
    
    \noindent
    \textbf{Пример~2.1.}\ Рассмотрим функтор~(\ref{e2.1-vas}), которому отвечает 
граф
    \begin{equation}
    \Gamma_R: \mbox{\fbox{$X_1$}} \xrightarrow{R_{12}} \mbox{\fbox{$X_2$}} 
\xrightarrow{R_{23}} \cdots \xrightarrow{R_{n-1,n}} \mbox{\fbox{$X_n$}}\,.
    \label{e2.3-vas}
    \end{equation}
    
    Функции out и~in определяют начало $X_i\hm\in M$ и~конец $X_j\hm\in M$ 
каж\-дой дуги, обозначенной стрелкой $R_{i,i+1}\hm\in L$. В~моменты времени 
$i\hm= 1,2,\ldots , n$ решаются игры~$\Gamma(X_i)$.
    
    Схема~(\ref{e2.3-vas}) свойственна многошаговым опе\-ра\-циям.
    
    \smallskip
    
    \noindent
    \textbf{Пример~2.2.}\ Пусть в~позиционной многошаговой игре двух лиц 
известна сле\-ду\-ющая динамическая сис\-те\-ма, начальное условие и~фазовое 
ограничение:
    \begin{gather*}
    \left\{
    \begin{array}{c}
    y_{t+1}^1 =y_t^1+u_t^1\,;\\[6pt]
    y_{t+1}^2 = y_t^2-u_t^2\,;
    \end{array}
    \right.\\[2pt]
    y_0^1 =0\,,\enskip y_0^2 =0\,;\\[2pt]
    \left\vert y_t^1 -y_t^2\right\vert \leq 2\,,\enskip t=\overline{0, T}\,.
    \end{gather*}
Агенты $k=1, 2$ стремятся по возможности максимизировать функции выигрыша
\begin{equation*}
J_1=-\sum\limits_{t=1}^{T-1} y_t u_t^1;\quad J_2= \sum\limits_{t=1}^{T-1} y_t u_t^2,
\end{equation*}
где
$$
y_t=y_t^1 - y_t^2,\quad u_t^k \in U_0\equiv \{ -1, 0, 1\}.
$$
    
    Интересы игроков $\tilde{\rho}^k \hm\in \mathrm{REL}( \tilde{X}, 
\tilde{X})$~(\ref{e1.1-vas}) заданы функционалами~$J_k$. В~статических 
подыг\-рах~$\Gamma(X_t)$, $X_t\hm= \cup y_t X(y_t)$, $X_0\hm= U_0^2$, с~по\-мощью 
функций $f^1\hm= -y_t u_t^1$, $f^2\hm= y_t u_t^2$ введем отношения предпочтения 
игроков~$\rho_t^k$. Видно, что интересы участников подыгр противоположны 
$\rho_t^2\hm= \rho_t^{1\mathrm{op}}$, а~множества допустимых ситуаций расслоены 
в~за\-ви\-си\-мости от позиций $y_t\hm= y_t^1\hm- y_t^2$, $\vert y_t\vert \hm\leq 2$, 
в~которых может пребывать динамическая сис\-те\-ма. Ситуация $(u_t^1, u_t^2)\hm\in 
X(y_t)$ допустима, только если ее выбор не нарушает заданное фазовое 
ограничение. 
    
    Во все моменты времени отношения предпочтения $\rho_t^k(y_t): X(y_t) \hm\to 
X(y_t)$, $k\hm=1,2$,  одинаковы: $\rho_t^k \hm= \rho_1^k$, $X_t\hm= X_1$, $t\hm= 
\overline{1,T-1}$, причем
    $$
    \rho_t^k\triangleq \bigcup\limits_{y_t} \rho_t^k (y_t).
    $$

    
    Вычисления показывают, что при $t\hm= \overline{1, T-1}$ имеем
    
    \vspace*{-8pt}
    
    \noindent
    \begin{multline*}
    X_t=\coprod\limits^2_{y_t=-2} X(y_t),\ X(\pm 2) = \{ (\pm 1,\pm1)\},\\
     X(\pm1) = \{ (\pm1,0), (0,\pm1)\},\\
      X(0)=\{ (1,-1), (0,0), (-1,1)\}\,;
    \end{multline*}
    
    \vspace*{-13pt}
    
    \noindent
    \begin{multline*}
    \rho_t^1(0): (0,-1)\sim (-1,0) \sim (0,0),\\
     \rho_t^1(1)=\{ ((0,-1), (-1,0))\},\\
    \rho^1_t(-1) =\{ ((-1,0), (0,-1))\}\,,\\ 
    \rho_1^1(\pm2) =\{ (\pm 1, \pm1)\}.
    \end{multline*}
    
    \vspace*{-4pt}
    
    Динамическое отношение игры, равное $R_{t,t+1}: X_t\hm\to X_{t+1}$, $t\hm= 
\overline{1, T-1}$, связывает до\-пус\-ти\-мую текущую $u_m\hm\equiv (u_m^1, u_m^2) 
\hm\in X(y_m)$ и~выбираемую сле\-ду\-ющую $u_{m+1}\hm\in X(y_{m+1})$ ситуации 
в~состояниях сис\-те\-мы $\Gamma(X_m)$, $m\hm= t, t\hm+1$. Таким образом, 
отношение непосредственного пред\-шест\-во\-ва\-ния ситуаций определено включением
    $$
    u_t \xrightarrow{R_{tt+1}} u_{t+1},\ u_{t+1} \in X(y_{t+1}),\ y_{t+1} \!=\! y_t+\left( 
u_t^1+ u_t^2\right).
    $$
    
    \vspace*{-2pt}
    
    Пусть сетевая структура~(\ref{e2.2-vas}) такова, что во всех играх 
$\Gamma(X_t)$ один из участников сообщает другому свое управ\-ля\-ющее 
воздействие~$u_t^k$. Тогда рациональное решение игры состоит в~выборе 
ситуации, ста\-би\-ли\-зи\-ру\-ющей траекторию сис\-те\-мы $y_t\hm\equiv 0$, а~многошаговая 
игра имеет седловые точки  $\forall_t u_t^1 \hm= -u_t^2$.
    
    При поиске рационального решения игровых задач применяются 
вспомогательные морфизмы, стро\-ящи\-еся из исходных~(\ref{e1.1-vas}) с~по\-мощью 
операций алгебры отношений $A\hm= \left( \circ, \cup, \cap, {}^{\mathrm{op}}, \times; 
\sigma,\varnothing \right)$~\cite{5-vas, 10-vas, 11-vas}. Получение ка\-ким-ли\-бо 
игроком дополнительной информации уменьшает неопределенность выбора 
подходящей стратегии (см.\ пример~2.2). Выбор сетевой структуры  
игры~(\ref{e2.2-vas}) изменяет отношения предпочтений участников~$\rho$. Игроки 
руководствуются сужениями $\rho\vert_A\hm= \rho \cap A^2$, $A\hm\subset X$, 
исходных отношений~$\rho$ на некоторые, вполне определенные подмножества 
ситуаций. 
    
    При условии сделанных предположений граф динамического отношения 
содержит минимальные и~максимальные элементы~--- вершины $X^0$ и~$X^T$ 
(см.~пример~2.1). Процесс функционирования ДМАС начинается с~решения 
статических \mbox{подыгр}~$\Gamma(X^0)$ и~заканчивается подзадачами~$\Gamma(X^T)$. 
Они названы начальным и~терминальным со\-сто\-яни\-ями сис\-те\-мы соответственно. 
В~отличие от примера~2.2, в~рас\-смат\-ри\-ва\-емой по\-ста\-нов\-ке задаются терминальные 
отношения предпочтения агентов~(\ref{e1.1-vas}).

     \begin{figure*}[b] %fig1
\vspace*{1pt}
\begin{minipage}[t]{79mm}
\begin{center}  
    \mbox{%
\epsfxsize=17.323mm 
\epsfbox{vas-1.eps}
}
\end{center}
\end{minipage}
\hfill
\vspace*{1pt}
\begin{minipage}[t]{79mm}
\begin{center}
     \mbox{%
\epsfxsize=25.462mm 
\epsfbox{vas-2.eps}
}
\end{center}
\end{minipage}
\begin{minipage}[t]{79mm}
\vspace*{-12pt}
\Caption{Образ $r=R\circ r^\prime$ и~обратный образ $r^\prime \hm= R^{\mathrm{op}}\circ r$ 
относительно функтора~$R$}
\end{minipage}
%\end{figure*}
\hfill
% \begin{figure*} %fig3.2
      \begin{minipage}[t]{79mm}
      \vspace*{-12pt}
           \Caption{Граф $\Gamma_R$ динамического отношения~$R$: $R_{ij}\hm\in \mathrm{REL}\,(X_i, X_j)$ }
     \label{f3.2-vas}
     \end{minipage}
%     \vspace*{-24pt}
     \end{figure*} 

    
    Рациональное поведение игроков заключается в~том, что они стремятся 
максимизировать свои 
 предпочтения на множестве всех допустимых ситуаций:
    \begin{equation}
    \forall_k \tilde{\rho}^{kT} \to \mathop{\mathrm{MAX}}\limits_{\tilde{X}}\,.
    \label{e2.4-vas}
    \end{equation}
В~задаче~(\ref{e1.1-vas}), (\ref{e2.1-vas}), (\ref{e2.2-vas}), (\ref{e2.4-vas}) требуется 
найти ситуацию равновесия~\cite{1-vas, 5-vas}. 
    
    Существование равновесия во многом зависит от свойств 
морфизмов~$\rho_i^k$, $k\hm\in K$. Наложим одно из таких требований. Пусть 
партнеры предлагают некоторому игроку~$k$ принять решение в~ситуации~$s$, для 
которой $\rho_i^k\vert_{s_k} \hm= \varnothing$, $s\hm\in X_i$. Предполагается, что 
возникшую неопределенность выбора каж\-дый игрок $k\hm \in K$ разрешит  
с~по\-мощью изменения своего отношения предпочтения, положив 
$\rho_i^k\vert_{s_k} \hm= \{ s\}$. (Игрок соглашается с~выбором ситуации~$s$.) Это 
требование выполняется для рефлексивных отношений.
    
\section{Эквивалентные преобразования динамических~игр}

    Категорное представление допускает применение моноидальных операций для 
преобразования игровой задачи~\cite{11-vas}. Морфизмы $R: \mathrm{REL}\left(Y\right)\hm\to 
\mathrm{REL}\,(X)$ представляют собой функторы, срав\-ни\-ва\-ющие однообъектные 
подкатегории в~категории REL. Образ морфизма $\rho\hm\in \mathrm{REL}\,(Y,Y)$ 
относительно~$R$ будем записывать как композицию $R\circ \rho \hm\in \mathrm{REL}\,(X,X)$.
    
    \smallskip
    
    \noindent
    \textbf{Определение~3.1.} Обратным образом (или кообразом) отношения $r: 
X\hm\to X$ относительно морфизма $R: Y\hm\to X$ назовем стрелку $r^\prime: 
Y\hm\to Y$, превращающую сле\-ду\-ющий квад\-рат в~коммутативный  
(рис.~1).
    
    
     
     
    Ввиду того что морфизм $R^{\mathrm{op}} \hm\in \mathrm{REL}\,(X,Y)$ сохраняет композиции 
стрелок, он сам становится функтором $R^{\mathrm{op}}: \mathrm{REL}\,(X)\hm\to \mathrm{REL}\,(Y)$. Так как 
$\forall_{ij} (x,y) \hm\in R_{ij}^{\mathrm{op}} \hm\Leftrightarrow  (y,x)\hm\in R_{ij}$, то 
назовем~$R^{\mathrm{op}}$ противоположным к~$\{ R_{ij}\}$ динамическим отношением. 
Его можно изобразить, изменив на\-прав\-ле\-ния всех дуг в~графе~$\Gamma_R$ 
(вертикальных стрелок на рис.~1). 




    
    Пусть вершины $X_1$ и~$X_l$ графа динамического отношения~$\Gamma_R$ 
связаны некоторым путем $L\hm= (X_1, \ldots, X_l)$. Взяв композицию $R_L\hm= 
R_{l-1}\circ\cdots\linebreak \cdots \circ R_1$ морфизмов $R_i\hm \in \mathrm{REL}\,(X_i, X_{i+1})$ вдоль цепи
$L\hm= (X_1, \ldots , X_l)$, мож\-но <<опустить>> произвольное бинарное отношение 
$r_l: X_l\hm\to X_l$ с~$X_l$ на множество~$X_1$ и~по\-стро\-ить его обратный образ 
$r_1\hm= R_L^{\mathrm{op}} \circ r_l$.
    
    Пусть теперь множество $\{ L\hm= (X_1, \ldots , X_l)\}$ всех попарно 
различных путей, свя\-зы\-ва\-ющих вершины $X_1$ и~$X_l$, состоит более чем из одного 
элемента. Под обратным образом отношения $\rho_l: X_l\hm\to X_l$, переносимого 
на множество~$X_1$, будем понимать копроизведение
    \begin{multline}
    \rho_1^\prime =R^{\mathrm{op}}\circ \rho_l =\coprod\limits_{\{L\}} R_L^{\mathrm{op}} \circ 
\rho_l;\\
 \rho_1^\prime: \coprod\limits_{\{ L\}} X_1\to \coprod\limits_{\{L\}} X_1\,.
    \label{e3.1-vas}
    \end{multline}
    
    <<Поднятием>> морфизма~$\rho_1$ вдоль путей $\{L\}$ мож\-но по\-стро\-ить 
образ $\rho_L\hm= R\circ \rho_1$. Это отношение $\coprod_{\{L\}} R_L\circ \rho_1$ на 
копроизведении объектов $\coprod_{\{L\}} X_l$ с~числом сомножителей, рав\-ным 
мощ\-ности набора~$\{L\}$.
    
    По формуле~(\ref{e3.1-vas}) заданные в~терминальных со\-сто\-яни\-ях сис\-те\-мы 
морфизмы~(\ref{e1.1-vas}) переносятся во все остальные со\-сто\-яния. Так вводятся 
отношения предпочтения игроков~$\rho_i^k$ в~играх~$\Gamma(X_i)$:
    \begin{equation*}
    \forall_{ik} \rho_i^k =R^{\mathrm{op}}\circ \tilde{\rho}_i^{kT}.
%    \label{e3.2-vas}
    \end{equation*}
    
    \noindent
    \textbf{Замечание~3.1.}\ Общее определение кообраза~(\ref{e3.1-vas}) мож\-но 
заменить двойственной конструкцией, основанной на произведении 
отношений~\cite{11-vas}: 
    \begin{equation}
    \rho^{\prime\prime} \!=\!R^{\mathrm{op}} \circ \rho \!=\!\prod\limits_{\{L\}} R^{\mathrm{op}}_L \circ \rho\,;\ \rho\! =\!R\circ \rho^{\prime\prime}= 
\prod\limits_{\{L\}} R_L\circ \rho^{\prime\prime}\!.\!
    \label{e3.3-vas}
    \end{equation}
Получим морфизмы~(\ref{e3.3-vas}) вида 
$$
\rho: \prod\limits_{\{L\}} X_1\to \prod\limits_{\{L\}} X_1,\ \rho^{\prime\prime}: \prod\limits_{\{L\}} X_l\to \prod\limits _{\{L\}} X_l.
$$

    \smallskip
    
    \noindent
    \textbf{Пример~3.1.}\ Воспользуемся формулой~(\ref{e3.1-vas}) 
применительно к~ДМАС, пред\-став\-лен\-ной на рис.~2. 
    
    % \end{multicols}
     


     
  %   \begin{multicols}{2}
    
     
Кообразом $R^{\mathrm{op}}\circ \rho_4$ отношения $\rho_4: X_4\hm\to X_4$ служит морфизм 
\begin{multline*}
\rho_1^\prime: X_1\coprod X_1 \to X_1\coprod X_1;
\\
\rho_1^\prime =\left( R_{13}^{\mathrm{op}} \circ R_{34}^{\mathrm{op}}\coprod R_{12}^{\mathrm{op}} \circ 
R_{24}^{\mathrm{op}}\right)\circ \rho_4.
\end{multline*}
Опуская~$\rho_4$ на множества $X_2$ и~$X_3$, получим $\rho_2^\prime \hm= 
R_{24}^{\mathrm{op}} \circ \rho_4$, $\rho_3^\prime\hm= R_{34}^{\mathrm{op}} \circ \rho_4$ 
соответственно. Поднятием отношения $\rho_1^\prime: X_1\hm\to X_1$ на 
множество~$X_4$ строится образ  $R\circ \rho_1^\prime$ как
\begin{multline*}
\rho_4: X_4\coprod X_4 \to X_4\coprod X_4\,;\\
 \rho_4=\left( R_{24}\circ R_{12} \coprod 
R_{34} \circ R_{13}\right) \circ \rho_1^\prime.
\end{multline*}
    
    Конструкция~(\ref{e3.3-vas}) дает структуру
\begin{multline*}
    \rho_4: X_4\times X_4 \to X_4\times X_4\,;\\
     \rho_4=\left( R_{24}\circ R_{12} \prod 
R_{34} \circ R_{13}\right) \circ \rho_1^{\prime\prime}.
\end{multline*}
    
    Формулы~(\ref{e3.1-vas}) и~(\ref{e3.3-vas}) позволяют строить эквивалентные 
модели игры. Свойство универсальности объектов $X_2\times X_3$ и~$X_2\coprod 
X_3$, вы\-ра\-жа\-емое по\-средст\-вом коммутативных диаграмм~\cite{11-vas}, однозначно 
определяет новое динамическое отношение с~со\-от\-вет\-ст\-ву\-ющи\-ми морфизмами. 
Например, вместо графа из рис.~2 мож\-но работать с~более прос\-той 
графовой структурой, пред\-став\-лен\-ной в~любой из сле\-ду\-ющих форм:
    \begin{gather}
    \mbox{\fbox{$X_1$}} \xrightarrow{\left\langle R_{12}, R_{13}\right\rangle} 
\mbox{\fbox{$X_2\times X_3$}} \xrightarrow{\left\langle R_{24}^{\mathrm{op}}, 
R_{34}^{\mathrm{op}}\right\rangle^{\mathrm{op}}} \mbox{\fbox{$X_4$}}\,;
    \label{e3.4-vas}\\
    \mbox{\fbox{$X_1$}} \xrightarrow{[ R_{12}, R_{13}]^{\mathrm{op}}} 
\mbox{\fbox{$X_2\coprod X_3$}} \xrightarrow{[ R_{24}, R_{34}]} 
\mbox{\fbox{$X_4$}}\,.
    \label{e3.5-vas}
    \end{gather}
Возможность эквивалентного перехода от произвольного графа~$\Gamma_R$ (см.\ 
рис.~2) к~цепи~(\ref{e2.3-vas}) (см.\ (\ref{e3.4-vas}) и~(\ref{e3.5-vas})) 
обоснована в~тео\-ре\-ме~3.1. 
    
    \smallskip
    
    \noindent
    \textbf{Теорема~3.1.} \textit{Граф всякого динамического отношения 
приводится к~виду}~(\ref{e2.3-vas}).

\vspace*{-6pt}

\section{Результирующее отношение динамической игры}

    Введением результирующего отношения игра сводится к~проблеме 
оптимального управ\-ле\-ния, по\-став\-лен\-ной в~форме отношений. Для ее решения\linebreak 
мож\-но применить обобщенный метод динамического программирования. 
Напомним~\cite{5-vas}, что в~любой статической игре $\Gamma(Y)$ существует 
ре\-зуль\-ти\-ру\-ющее отношение~$P_Y$. Оно выражает \mbox{композициональное} свойство 
игры, учи\-ты\-ва\-ющее интересы всех участников операции, сетевую структуру их 
взаимодействия и~ра\-ци\-о\-наль\-ность поведения. Решение игры $\Gamma(Y)$ сводится
 к~поиску максимальных элементов $P_Y\hm\to \mathrm{MAX}_Y$. Благодаря\linebreak этому, мож\-но 
по-но\-во\-му определить состояния \mbox{исходной} ДМАС. Вместо игр $\Gamma(X_i)$ 
будем рас\-смат\-ри\-вать оптимизационные задачи $(\tilde{X}_i, P_{\tilde{X}_i})$, 
$P_{\tilde{X}_i} \hm\in \mathrm{REL}\,(\tilde{X}_i, \tilde{X}_i)$. Иначе говоря, теперь во всех 
со\-сто\-яни\-ях сис\-те\-мы решение принимает единственный агент. 
    
    %\smallskip
    
    \noindent
    \textbf{Определение~4.1.}\ Пусть $(\tilde{Y}_1, P_{\tilde{Y}_1} ) 
\overset{R}{\prec} (\tilde{Y}_2, P_{\tilde{Y}_2})$; $\tilde{y}_1, \tilde{y}_2 \hm\in 
\tilde{Y}_1$. Ситуация~$\tilde{y}_2$ называется более перспективной по сравнению 
с~$\tilde{y}_1$, если выполнено свойство
    \begin{equation}
    \left( \tilde{y}_1, \tilde{y}_2\right) \in \left( R^{\mathrm{op}}\circ P_{\tilde{Y}_2} \right)\circ 
P_{\tilde{Y}_1}.
    \label{e4.1-vas}
    \end{equation}
    
    В каждом состоянии ДМАС игрокам целесообразно использовать более 
перспективные ситуации и~из них формировать рациональное решение 
задачи~(\ref{e1.1-vas}), (\ref{e2.1-vas}), (\ref{e2.2-vas}), (\ref{e2.4-vas}). В~этом 
заключается прин\-цип Бел\-лмана. 
    
    В случае динамического отношения с~графом~(\ref{e2.3-vas}) (см.\ 
тео\-ре\-му~3.1) рас\-смот\-рим сле\-ду\-ющую итерационную схему по\-стро\-ения 
<<оп\-ти\-ми\-зи\-ру\-ющих>> морфизмов $\{ \tilde{P}_{\tilde{X}_k}\}$:

\vspace*{-4pt}

\noindent
    \begin{multline}
    \tilde{P}_{\tilde{X}_T} =P_{X_T},\ \tilde{P}_{\tilde{X}_{T-k}} =\left( 
R^{\mathrm{op}}\circ \tilde{P}_{\tilde{X}_{T-k+1}}\right) \circ P_{\tilde{X}_{T-k}},\\
    k=\overline{1, T}\,.
    \label{e4.2-vas}
    \end{multline}
    
    \vspace*{-4pt}

\noindent
Назовем~(\ref{e4.2-vas}) уравнениями Беллмана в~форме отношений. 
    
    \smallskip
    
    \noindent
    \textbf{Определение~4.2.} Динамическим ре\-зуль\-ти\-ру\-ющим отношением 
называется семейство морфизмов Беллмана $\{ \tilde{P}_{\tilde{X}_k},\ k\hm= 
\overline{1, T}\}$ из уравнений~(\ref{e4.2-vas}).
    
    \smallskip
    
    \noindent
    \textbf{Теорема~4.1.} \textit{Во всякой ДМАС существует ре\-зу\-ль\-ти\-ру\-ющее 
отношение.}
    
    \smallskip
    
    \noindent
    Д\,о\,к\,а\,з\,а\,т\,е\,л\,ь\,с\,т\,в\,о\,.\ \ Построение ДРО проведем методом 
математической индукции. На начальном шаге, в~терминальном со\-сто\-янии сис\-те\-мы, 
ДРО совпадает с~ре\-зуль\-ти\-ру\-ющим отношением $\tilde{P}_{X^T} \triangleq 
P_{X^T}$, $\tilde{X}^T\hm= X^T$, статической игры~$\Gamma(X^T)$. Опус\-тим 
этот морфизм на все множества $\tilde{X}^{T-1}$, для которых имеет мес\-то 
непосредственное предшествование $\Gamma(\tilde{X}^{T-1}) \overset{R}{\prec} 
\Gamma(\tilde{X}^T)$. Композиция морфизмов~(\ref{e4.1-vas}), $\tilde{Y}_1\hm= 
\tilde{X}^{T-1}$, $\tilde{Y}_2\hm= \tilde{X}^T$, определяет компоненту ДРО 
$\tilde{P}_{\tilde{X}^{T-1}}$, от\-ве\-ча\-ющую со\-сто\-янию сис\-те\-мы $(\tilde{X}^{T-1}, 
P_{\tilde{X}^{T-1}})$. Пусть отношение~$\tilde{P}_{\tilde{X}_i}$ уже построено. 
Тогда опять по формуле~(\ref{e4.1-vas}) его можно продолжить на все со\-сто\-яния 
$\Gamma(\tilde{X}_j) \overset{R_{ij}}{\prec} \Gamma(\tilde{X}_i)$ исходной ДМАС, 
т.\,е.\ построить очередные морфизмы~$\tilde{P}_{\tilde{X}_j}$. Процесс 
завершается в~начальных состояниях сис\-темы. 

\vspace*{-11pt}

\section{Поиск рационального решения игры}

\vspace*{-1pt}

    Для решения задачи~(\ref{e1.1-vas}), (\ref{e2.1-vas}), (\ref{e2.2-vas}),  
(\ref{e2.4-vas}) предложим сле\-ду\-ющее обобщение метода динамического\linebreak 
программирования. Сначала из системы уравнений Беллмана~(\ref{e4.2-vas}) 
найдем ДРО игры. Затем последовательно для моментов времени $1,\ldots, T$ 
вычислим сле\-ду\-ющие множества максимальных \mbox{элементов} отношения Беллмана:

\pagebreak



\noindent
    \begin{multline}
    X_1^*=\mathop{\mathrm{ARGMAX}}\,\tilde{P}_{\tilde{X}_1};\enskip 
    X_2^*=\mathop{\mathrm{ARGMAX}}\limits_{RX_1^*}\, \tilde{P}_{\tilde{X}_2}; \ldots  \\
\ldots ;    X_T^* = \mathop{\mathrm{ARGMAX}}\limits_{RX^*_{T-1}}\,\tilde{P}_{\tilde{X}_T}.
       \label{e5.1-vas}
    \end{multline}
    
\vspace*{-3pt}

    \noindent
    \textbf{Теорема~5.1.}\ \textit{Рациональному решению динамической игровой 
задачи}~(\ref{e1.1-vas})--(\ref{e2.4-vas}) \textit{отвечает ситуация 
$\tilde{x}^*\hm= (x_1^*, \ldots , x_T^*)$; $x_s^*\hm\in X_s^*$, $s\hm= \overline{1, T}$, 
найденная методом Беллмана}~(\ref{e4.2-vas}), (\ref{e5.1-vas}).
    
    \smallskip
    
    \noindent
    \textbf{Пример~5.1.} Графом~$\Gamma_R$~(\ref{e2.3-vas}), $n\hm=2$, 
динамического отношения задана ДМАС 

\vspace*{-4pt}

\noindent
    \begin{multline*}
    R: (x_1, x_2, x_3) \to (x_3, x_1, x_1\vee x_2) \cup (x_2, x_3, x_1\wedge x_2);\\
    x=(x_1, x_2, x_3)\in\underline{8}\simeq \{0,1\}^3.
    \end{multline*}
    
    \vspace*{-3pt}
    
    \noindent
Терминальная задача представляет собой игру Гермейера трех лиц 
$\Gamma^{2,3}$~\cite{1-vas, 5-vas}, где $X^T\hm=\underline{8}$~--- бинарный куб. 
Отношения предпочтений агентов $\tilde{\rho}_2^k \hm\in \mathrm{REL}\,(\underline{8},\underline{8})$ равны:
\begin{align*}
\tilde{\rho}_2^1 &=\{ 01, 05, 40, 34, 32, 76\};\\
\tilde{\rho}_2^2&= \{ 02, 10, 24, 32, 35, 45, 64, 67\};\\
\tilde{\rho}_2^3 &=\{ 20, 40, 35, 31, 64, 76\}.
\end{align*}
Найдем интересы игроков $\rho_1^k\hm= R^{\mathrm{op}}(\tilde{\rho}_2^k)$ в~начальном 
состоянии сис\-те\-мы~$\Gamma^{1,3}$, $X^0\hm= \underline{8}$: 

\vspace*{-4pt}

\noindent
\begin{multline*}
\rho_1^1= \{ 01, 02, 03, 04, 06, 12, 13, 14, 16, 20, 21, 62, 63,\\
 64, 65, 72, 73, 74, 75\}\,;
 \end{multline*}
 
 \vspace*{-14pt}
 
 \noindent
 \begin{multline*}
 \rho_1^2=\{ 03, 04, 05, 13, 14, 15, 20, 21,23, 26, 32, 40, 41,\\
  42, 52, 61, 63, 64, 65, 71, 73, 74, 75, 76\};
  \end{multline*}
  
  \vspace*{-14pt}
 
 \noindent
 \begin{multline*}
\rho_1^3 = \{ 20, 21, 30, 31, 40, 41, 50, 51, 61, 62, 63, 64, 71,\\
 72, 73, 74, 76\}.
\end{multline*}

\vspace*{-5pt}

\noindent
Вычислим результирующие отношения статических подыгр $\Gamma(X^T)$, 
$\Gamma(X^0)$ ~\cite{5-vas}:

\vspace*{-4pt}

\noindent
\begin{multline*}
P_{X^T} =\tilde{\rho}_1^3 \vert_{x_3} \circ \tilde{\rho}_1^2 \circ \left( \tilde{\rho}_1^1 
\cup \tilde{\rho}_1^{2G}\right),\\
\rho_{1}^{2G} =\tilde{\rho}_1^2 \vert_{\mathrm{MIN} \tilde{\rho}_1^2\vert_{{x_1}}}
\Rightarrow P_{X^T} =\{ 40, 05, 42, 35, 30, 70\},\\
P_{X^0} =\rho_1^1\vert_{x_1} \circ \rho_1^2\vert_{x_2}\circ\rho_1^3\vert_{x_3} =\{ 41, 53, 61, 73\}.
\end{multline*}

\vspace*{-4pt}

\noindent
Из уравнений Беллмана~(\ref{e4.2-vas}) найдем искомое ДРО игры:
$$
\tilde{P}_{X^T} =\{ 40, 05, 42, 35, 30, 70\},\ \tilde{P}_{X^0} = \{40, 00, 60, 70\}.
$$

\vspace*{-3pt}

\noindent
По формулам~(\ref{e5.1-vas}) и~тео\-ре\-ме~5.1 рациональное решение игры равно 
$\tilde{x}^*\hm=(0,0)$.

\vspace*{-12pt}

\section{Заключение}

\vspace*{-4pt}

    Функторная модель стала наследником традиционной формы представления 
ДМАС. Обладая композициональной структурой, она создает условия для 
модификации игровой задачи и~выполнения эквивалентных преобразований 
средствами алгебры тео\-рии категорий. Доказано существование ре\-зуль\-ти\-ру\-юще\-го 
отношения динамической игры многих лиц. Для задач оптимального управ\-ле\-ния 
в~форме динамических отношений предложено обобщение метода Беллмана.  
С~по\-мощью ДРО этим методом строится рациональное решение задачи. 
Функторный подход поддерживается компьютерной алгеброй тео\-рии категорий. 
Сетевая архитектура применяемых морфизмов допускает эффективную 
нейросетевую программную реализацию, которую еще только пред\-сто\-ит 
осуществить. 

\vspace*{-12pt}

{\small\frenchspacing
 {\baselineskip=10.6pt
 %\addcontentsline{toc}{section}{References}
 \begin{thebibliography}{99}
 
\vspace*{-3pt}

\bibitem{2-vas}
\Au{Моисеев~Н.\,Н.} Элементы теории оптимальных сис\-тем.~--- М.: Наука, 1974. 526~с.

\bibitem{1-vas} %2
\Au{Гермейер~Ю.\,Б.} Игры с~непротивоположными интересами.~--- М.: Наука, 1976. 326~с.

\bibitem{3-vas}
\Au{Красовский Н.\,Н., Субботин~А.\,И.} Позиционные дифференциальные игры.~--- M.: Наука, 
1976. 456~с.
\bibitem{4-vas}
\Au{Dockner E.\,J., Jorgensen~S., Long~N.\,V., Sorger~G.} Differential games in economics and management
science.~--- Cambridge: Cambridge University 
Press, 2000. 382~p. doi: 10.1017/CBO9780511805127.
\bibitem{5-vas}
\Au{Васильев Н.\,С.} Композициональное представление структуры игры многих лиц 
в~моноидальной категории бинарных отношений~// Информатика и~её применения, 2023. Т.~17. 
Вып.~2. С.~18--26. doi: 10.14357/19922264230203. EDN: GPMZTS.
\bibitem{9-vas}  %6
\Au{Dixit~A.\,K., Natebuff~B.\,J.} The art of strategy.~--- New York; London: W.\,W.~Norton~\&~Co., 
2008. 446~p.

\bibitem{7-vas}
\Au{Shoham~Y., Leyton-Brown~R.} Multiagent systems: Algorithmic, game-theoretic, and logical 
foundations.~--- Cambridge: Cambridge University Press, 2010. 532~p.
\bibitem{6-vas} %8
\Au{Bai~Q., Ren~F., Fujita~K., Znang~M.} Multi-agent and complex systems.~--- Studies in computational 
intelligence ser.~--- Springer Singapore, 2016. Vol.~670. 210~p.
\bibitem{8-vas} %9
\Au{Dixit~A.\,K., Skeath~S., Reiley~W.\,W., Jr.} Games of strategy.~--- New York; London: 
W.\,W.~Norton \&~Co., 2017. 880~p.

\bibitem{10-vas}
\Au{Скорняков Л.\,А.} Элементы общей алгебры.~--- М.: Наука, 1983. 272~с.

%\pagebreak

\bibitem{11-vas}
\Au{Маклейн~С.} Категории для работающего математика~/ Пер.\ с~англ.~--- М.: Физматлит, 2004. 352~с. 
(\Au{Mac Lane~S.} Categories for the working mathematician.~--- 2nd ed.~--- New York, NY, USA: 
Springer, 1998. 318~p.)
\bibitem{12-vas}
\Au{Губко М.\,В.} Управление организационными системами с~сетевым взаимодействием агентов. 
Обзор теории сетевых игр~// Автоматика и~телемеханика, 2004. №\,8. С.~115--132.
\bibitem{13-vas}
Group formation in economics: Networks, clubs, and coalitions~/ Eds.\ G.~Demange, M.~Wooders.~--- 
Cambridge: Cambridge University Press, 2005. 475~p.

\end{thebibliography}

 }
 }

\end{multicols}

\vspace*{-9pt}

\hfill{\small\textit{Поступила в~редакцию 02.02.24}}

%\vspace*{6pt}

\pagebreak

%\newpage

\vspace*{-28pt}

%\hrule

%\vspace*{2pt}

%\hrule



\def\tit{ON FUNCTOR REPRESENTATION OF~OPTIMIZED\\ DYNAMIC~MULTIAGENT~SYSTEMS}


\def\titkol{On functor representation of~optimized dynamic multiagent systems}


\def\aut{N.\,S.~Vasilyev}

\def\autkol{N.\,S.~Vasilyev}

\titel{\tit}{\aut}{\autkol}{\titkol}

\vspace*{-15pt}


\noindent
N.\,E.~Bauman Moscow State Technical University, 5-1~Baumanskaya 2nd Str., Moscow 105005, Russian 
Federation

\def\leftfootline{\small{\textbf{\thepage}
\hfill INFORMATIKA I EE PRIMENENIYA~--- INFORMATICS AND
APPLICATIONS\ \ \ 2024\ \ \ volume~18\ \ \ issue\ 2}
}%
 \def\rightfootline{\small{INFORMATIKA I EE PRIMENENIYA~---
INFORMATICS AND APPLICATIONS\ \ \ 2024\ \ \ volume~18\ \ \ issue\ 2
\hfill \textbf{\thepage}}}

\vspace*{4pt}



\Abste{Functors' topoi is chosen as a computational tool for synthesizing dynamic multiagent systems (DMAS). The scale orders the objects as 
multiagent system states to solve attendant static subgames in them. 
The initial dynamic game and all static subproblems are represented in the monoidal category of binary relations. 
Players' preference relations might be maximized in DMAS. The game rational solution is understood as
 equilibrium. The 
compositional structure of the optimized DMAS can be described in the form of the game dynamic resulting relation (DRR). 
Players' rational behavior search is reduced to DRR subsequent maximization. For this purpose, the Bellman's method 
is generalized to solve control problems stated in the form of relations. 
The program implementation of the approach can be based on neural networks due to the consistency of the architectures 
of the applied relation graphs and neural networks.}

\KWE{functor category; compositionality; monoidal category; opposite image; game dynamic relation; 
static subgame; preference relation; dynamic resulting relation; rational solution; Bellman morphism}

\DOI{10.14357/19922264240201}{CLMBXC}

%\vspace*{-12pt}

%\Ack

%\vspace*{-3pt}

%    \noindent
 

  \begin{multicols}{2}

\renewcommand{\bibname}{\protect\rmfamily References}
%\renewcommand{\bibname}{\large\protect\rm References}

{\small\frenchspacing
 {%\baselineskip=10.8pt
 \addcontentsline{toc}{section}{References}
 \begin{thebibliography}{99} 

\bibitem{2-vas-1}
\Aue{Moiseev, N.\,N.} 1975. \textit{Elementy teorii optimal'nykh sistem} [Elements of optimal systems 
theory]. Moscow: Nauka. 527~p.

\bibitem{1-vas-1}
\Aue{Germeyer, Yu.\,B.} 1976. \textit{Igry s~neprotivopolozhnymi interesami} [Games with  
nonopposing interests]. Moscow: Nauka. 326~p.

\bibitem{3-vas-1}
\Aue{Krasovskiy, N.\,N., and A.\,I.~Subbotin.} 1974. \textit{Pozitsionnye differentsial'nye igry} 
[Positional differential games]. Moscow: Nauka. 456~p.
\bibitem{4-vas-1}
\Aue{Dockner, E.\,J., S.~Jorgensen, N.\,V.~Long, and G.~Sorger.} 2000. \textit{Differential games in 
economics and management science}. Cambridge: Cambridge University Press. 382~p. doi: 
10.1017/CBO9780511805127.
\bibitem{5-vas-1}
\Aue{Vasilyev, N.\,S.} 2023. Kompozitsional'noe predstavlenie struktury igry mnogikh lits 
v~monoidal'noy kategorii binarnykh otnosheniy [Multiplayers' games compositional structure in the 
monoidal category of binary relations]. \textit{Informatika i~ee Primeneniya~--- Inform. Appl.} 
 17(2):18--26. doi: 10.14357/19922264230203. EDN: GPMZTS.
 
 \bibitem{9-vas-1} %6
\Aue{Dixit, A.\,K., and B.\,J.~Nalebuff.} 2008. \textit{The art of strategy}. New York, London: 
W.\,W.~Norton \&~Co. 446~p.


\bibitem{7-vas-1}
\Aue{Shoham, Y., and R.~Leyton-Brown.} 2010. \textit{Multiagent systems: Algorithmic,  
game-theoretic, and logical foundations}. Cambridge University Press. 532~p.

\bibitem{6-vas-1} %8
\Aue{Bai, Q., F.~Ren, K.~Fujita, and M.~Znang.} 2016. \textit{Multi-agent and complex systems}. Studies 
in computational intelligence ser.  Springer Singapore. 210~p.

\bibitem{8-vas-1} %9
\Aue{Dixit, A.\,K., S.~Skeath, and D.\,H.~Reiley, Jr.} 2017. \textit{Games of strategy}. New York, 
London: W.\,W.~Norton \&~Co.\linebreak 880~p.

\bibitem{10-vas-1}
\Aue{Skornyakov, L.\,A.} 1983. \textit{Elementy obshchey algebry} [Elements of general algebra]. 
Moscow: Nauka. 272~p.
\bibitem{11-vas-1}
\Aue{Mac Lane, S.} 1998. \textit{Categories for the working mathematician}. 2nd ed. New York, NY: 
Springer. 318~p. 
\bibitem{12-vas-1}
\Aue{Gubko, M.\,V.} 2004. Control of organizational systems with network interaction of agents. 
II.~Stimulation problems. \textit{Automat. Rem. Contr.} 65(9):1470--1485. doi: 
10.1023/B:AURC.0000041425.34118.7d. EDN: \mbox{LFMUCG}.
\bibitem{13-vas-1}
Demange, G., and M.~Wooders, eds. 2005. \textit{Group formation in economics: Networks, clubs, and 
coalitions}. Cambridge: Cambridge University Press. 475~p. 

\end{thebibliography}

 }
 }

\end{multicols}

\vspace*{-8pt}

\hfill{\small\textit{Received February 2, 2024}} 

\vspace*{-18pt}


\Contrl

\vspace*{-3pt}

\noindent
\textbf{Vasilyev Nikolai S.} (b.\ 1952)~--- Doctor of Science in physics and mathematics, professor, 
N.\,E.~Bauman Moscow State Technical University, 5-1 Baumanskaya 2nd Str., Moscow 105005, Russian 
Federation; \mbox{nik8519@yandex.ru}





\label{end\stat}

\renewcommand{\bibname}{\protect\rm Литература} 