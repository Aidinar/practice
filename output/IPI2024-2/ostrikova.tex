\def\stat{ostrikova}

\def\tit{МЕТОД ОЦЕНКИ ХАРАКТЕРИСТИК СИСТЕМ 5G/6G <<НОВОЕ РАДИО>> 
С~УЧЕТОМ МАКРО- И МИКРОМОБИЛЬНОСТИ ПОЛЬЗОВАТЕЛЕЙ$^*$}

\def\titkol{Метод оценки характеристик систем 5G/6G <<новое радио>> 
с~учетом макро- и~микромобильности пользователей}

\def\aut{Д.\,Ю.~Острикова$^1$, Е.\,С.~Голос$^2$, В.\,А.~Бесчастный$^3$, 
Е.\,А.~Мачнев$^4$, В.\,С.~Шоргин$^5$,\\ Ю.\,В.~Гайдамака$^6$}

\def\autkol{Д.\,Ю.~Острикова, Е.\,С.~Голос, В.\,А.~Бесчастный и~др.}
%$^3$,  Е.\,А.~Мачнев$^4$, В.\,С.~Шоргин$^5$, Ю.\,В.~Гайдамака$^6$}

\titel{\tit}{\aut}{\autkol}{\titkol}

\index{Острикова Д.\,Ю.}
\index{Голос Е.\,С.}
\index{Бесчастный В.\,А.}
\index{Мачнев Е.\,А.}
\index{Шоргин В.\,С.}
\index{Гайдамака Ю.\,В.}
\index{Ostrikova D.\,Yu.}
\index{Golos E.\,S.}
\index{Beschastnyi V.\,A.}
\index{Machnev E.\,A.}
\index{Shorgin V.\,S.}
\index{Gaidamaka Yu.\,V.}


{\renewcommand{\thefootnote}{\fnsymbol{footnote}} \footnotetext[1]
{Исследование выполнено за счет гранта Российского научного фонда №\,23-79-10084, 
{\sf https://rscf.ru/project/23-79-10084}.}}


\renewcommand{\thefootnote}{\arabic{footnote}}
\footnotetext[1]{Российский университет дружбы народов им.\ Патриса Лумумбы, ostrikova-dyu@rudn.ru}
\footnotetext[2]{Российский университет дружбы народов им.\ Патриса Лумумбы, golos-es@rudn.ru}
\footnotetext[3]{Российский университет дружбы народов им.\ Патриса Лумумбы, beschastnyy-va@rudn.ru}
\footnotetext[4]{Российский университет дружбы народов им.\ Патриса Лумумбы, machnev-ea@rudn.ru}
\footnotetext[5]{Федеральный исследовательский центр <<Информатика и~управ\-ле\-ние>> Российской академии наук, 
\mbox{vshorgin@ipiran.ru}}
\footnotetext[6]{Российский университет дружбы народов им.\ Патриса Лумумбы; Федеральный исследовательский 
центр <<Информатика и~управление>> Российской академии наук, \mbox{gaydamaka-yuv@rudn.ru}}

\vspace*{-2pt}


  
      
  
  
  \Abst{Оценка производительности сотовых систем 5G/6G <<новое радио>>, как правило, 
проводится в~предположении о статичном местоположении пользователей и~идеально 
направленных антеннах, которые оправданы для случая регулярной синхронизации между 
пользовательским устройством (ПУ) и~базовой станцией (БС). Однако в~случае работы 
с~имеющими высокую энергоэффективность ПУ с~ограниченным функционалом (RedCap~--- Reduced Capability) 
БС реже получает информацию о качестве принимаемого устройством сигнала, 
которое меняется при перемещении ПУ. Это приводит к~необходимости исследования 
динамики показателей эффективности систем с~ПУ RedCap во времени. В~статье для анализа 
спектральной эффективности в~зависимости от расстояния между БС и~ПУ и~направленности 
антенны ПУ в~произвольный момент времени используются инструменты стохастической 
геометрии и~теории случайных блужданий. Численный эксперимент показал, что 
макромобильность оказывает существенное влияние на спектральную эффективность, влияние 
микромобильности меньше и~проявляется только на коротких промежутках времени, при этом 
размер фазированной антенной решетки (ФАР) на стороне БС практически не влияет на 
полученный результат.} 
  
  \KW{5G <<новое радио>>; mmWave; sub-THz; %5G NR, 
  микромобильность;  макромобильность; спектральная эффективность}

\DOI{10.14357/19922264240205}{JCUFHS}
  
%\vspace*{-6pt}


\vskip 10pt plus 9pt minus 6pt

\thispagestyle{headings}

\begin{multicols}{2}

\label{st\stat}

\section{Введение}

  Ожидается, что современные системы сотовой связи~5G и~будущие~6G, 
работающие в~диапазонах миллиметровых (mmWave, 30--100~ГГц) длин волн 
и~субтерагерцевых (sub-THz, 100--300~ГГц) час\-тот, смогут обеспечить 
исключительно высокую пропускную способность на участке беспроводного 
доступа~[1, 2]. Это позволит внедрить множество новых приложений, включая 
передачу потокового видео в~формате сверхвысокой четкости, приложения 
телемедицины, виртуальной и~дополненной реальности~[3]. Использование 
чрезвычайно высоких частот в~диапазонах mmWave/sub-THz требует применения 
как на стороне БС, так и~на стороне ПУ фазированных антенных решеток, ра\-бо\-та\-ющих в~режиме 
формирования луча, расширяя зону действия БС~[4, 5]. Для своевременной 
компенсации ущерба от потери связи, характерной для сис\-тем с~направленными 
лучами, применяются методы отслеживания луча, синхронизирующие 
направления лучей между ПУ и~БС на малых временн$\acute{\mbox{ы}}$х интервалах~[6]. 
  
  Необходимость экономии ресурса аккумулятора ПУ привела к~появлению 
устройств с~ограниченным функционалом (RedCap)~[7], для которых процедура управления радиоресурсами (\textit{англ}.\ 
Radio\linebreak Resource Management, RRM) подразумевает пропуск блоков сигналов 
синхронизации (\textit{англ}.\ Synchronization Signal Blocks, SSB) для повышения 
энергоэффективности. Следствием более редкой\linebreak синхронизации становится риск 
потери связи с~подвижным ПУ из-за эффектов макро- и~микромобильности~[8, 9] 
на средних и~больших интервалах времени. Под макромобильностью \mbox{понимается} 
перемещение пользователей внут\-ри зоны покрытия, приводящее к~изменению 
расстояния между ПУ и~БС, а~также к~нарушению взаимного выравнивания 
направленных лучей ФАР БС и~ПУ. Под микромобильностью понимается 
незначительное изменение положения ПУ в~пространстве из-за вращения ПУ 
в~руках пользователя, при котором луч ФАР ПУ отклоняется от направления на 
БС, но расстояние между ПУ и~БС не меняется. Таким образом, к~применяемым 
ранее моделям стохастической геометрии, предполагающим методы анализа при 
статическом расположении пользователей~[10--13], необходимо добавить 
инструменты моделирования мо\-биль\-ности~ПУ.
{\looseness=1

} 
  
  В данной работе использован подход к~моделированию макро- 
и~микромобильности ПУ на основе диффузионных процессов, предложенный 
в~[14] при сравнении стратегий энергосбережения для устройств с~ограниченным 
функционалом для приложений промышленной автоматизации. В~отличие 
от~[14], целью данной статьи ставится разработка метода оценки характеристик 
производительности соты 5G/6G <<новое радио>>, в~частности спектральной 
эффективности системы, с~учетом перемещения пользователей внутри зоны 
покрытия при развертывании в~городской среде.

\section{Системная модель}

  Для анализа влияния эффектов макро- и~микромобильности на характеристики 
систем mmWave/sub-THz исследована модель соты сети 5G/6G <<новое 
радио>>~[1], изображенная на рис.~1,\,\textit{б}. Зона покрытия БС, 
расположенной в~точке с~координатами $(0, 0)$, имеет форму квадрата с~длиной 
стороны~$L$~м. Подвижное ПУ, перемещающееся в~зоне покрытия, в~начальный 
момент времени~$t_0$ расположено в~точке $(x_0,y_0)$, в~произвольный момент 
$t\hm>0$~--- в~точке $(x,y)$. Высоты БС и~ПУ постоянны и~равны~$h_A$
  и~$h_U$ соответственно. Макромобильность при перемещении ПУ на средних и~больших временн$\acute{\mbox{ы}}$х интервалах отражается в~динамике мощности 
принимаемого ПУ сигнала из-за изменяющегося расстояния между БС и~ПУ 
и~рассогласования направленности лучей антенн ПУ и~БС (угол~$\beta$ на 
рис.~1,\,\textit{а} и~1,\,\textit{б}). Микромобильность приводит только 
к~рассогласованию на\-прав\-лен\-ности лучей (угол~$\gamma$ на рис.~1,\,\textit{в}), 
при этом эффект наблюдается на малых временн$\acute{\mbox{ы}}$х интервалах.
  


  Для анализа спектральной эффективности в~момент~$t$ используем 
выражение~[15]:
  \begin{equation}
  S_E(t)= \log_2 \left( 1+\fr{P_T G_A(t)G_U(t)L(y,t)}{N_0+I_M}\right).
  \label{e1-ost}
  \end{equation}
где $P_T$~--- мощность излучения антенны БС; $G_A(t)$ и~$G_U(t)$~--- 
коэффициенты усиления антенн БС и~ПУ\linebreak\vspace*{-12pt}

{ \begin{center}  %fig1
 \vspace*{6pt}
    \mbox{%
\epsfxsize=77.964mm 
\epsfbox{ost-1.eps}
}

\end{center}



\noindent
{{\figurename~1}\ \ \small{Эффекты макро- и~микромобильности в~системах <<новое радио>>
}}}

\vspace*{6pt}


\noindent
 в~момент~$t$; $L(y,t)$~--- коэффициент 
потери мощ\-ности принимаемого ПУ сигнала на расстоянии~$y$ от БС 
в~момент~$t$; $N_0$~--- шум; $I_M$~--- интерференция. 
  
  Заметим, что значения функций $L(y,t)$, $G_A(t)$ и~$G_U(t)$ вычисляются 
в~зависимости от трех па\-ра\-мет\-ров, определяемых новым местоположением ПУ 
и~отклонением оси ПУ от направления на БС в~момент~$t$, возникшими 
вследствие перемещения ПУ и~вращения ПУ в~руках пользователя, а~именно: от 
расстояния~$g(t)$ по оси~$y$ и~от угловых отклонений $\beta_H(t)$, 
$\gamma_H(t)$ и~$\beta_V(t)$, $\gamma_V(t)$ оси антенны ПУ от направления на 
БС в~горизонтальной и~вертикальной плоскостях соответственно. Таким образом, 
для оценки значений функций $L(y,t)$, $G_A(t)$ и~$G_U(t)$, меняющихся во 
времени, необходимо выбрать метод моделирования мобильности ПУ.
  
  
  В~[14] для моделирования мак\-ро\-мо\-биль\-ности использованы два независимых 
диффузионных процесса блуждания частицы по осям~$0x$ и~$0y$, ограниченные 
в~$(0,L)$, с~коэффициентами диффузии~$D_x$ и~$D_y$. Для одномерного 
ограниченного диффузионного процесса плотность вероятности $p(x, t\vert 
x_0,t_0)$ найти частицу в~точке~$x$ в~момент~$t$, учитывая, что она находилась в~точке~$x_0$ в~момент~$t_0$, подчиняется частному случаю второго закона 
диффузии Фика и~описывается уравнением Фок\-ке\-ра--План\-ка с~нулевым 
сносом
  \begin{equation}
  \fr{\partial p(x,t\vert x_0, t_0)}{\partial t} = \fr{D\partial^2 p(x,t\vert x_0, 
t_0)}{\partial x^2}
  \label{e2-ost}
  \end{equation}
с начальным условием $p(x,t\vert x_0, t_0)\hm= \delta(x\hm-x_0)$, 
где $D$~--- коэффициент диффузии, характеризующий скорость перемещения 
частицы~[16]. Вид полученного методом разделения переменных решения 
уравнения~(\ref{e2-ost}) в~указанных выше ограничениях показан в~[14]:
\begin{multline*}
\!\!\!p(x,t\vert x_0, t_0) =\fr{1}{L}+
\fr{2}{L}\sum\limits^\infty_{n=1} \left\{
\exp \!\left[ \!-\left( \fr{n\pi}{2}\right)^{\!2}
\fr{t-t_0}{\tau} \right] \times{}\right.\hspace*{-0.34721pt}\\
\left.{}\times \cos \left( \fr{nx\pi}{L}\right) \cos\left(\fr{nx_0\pi}{L}\right)
\vphantom{\left[ \!-\left( \fr{n\pi}{2}\right)^{\!2}
\fr{t-t_0}{\tau} \right]}
\right\},
%\label{e3-ost}
\end{multline*}
где  $\tau=L^2/D$~--- время релаксации.
  
  При условии старта перемещения ПУ в~начальный момент~$t_0$ из точки 
$(x_0,0)$ значения расстояния $y(t)$ и~углов азимута $\beta_H(t)$ 
и~элевации~$\beta_V(t)$ отклонения оси ПУ вследствие макромобильности 
в~произвольный момент~$t$ могут быть вычислены по следующим 
формулам~[14]:
  \begin{align*}
  y(t)  &= \sqrt{\left[ X(t)\right]^2+\left( h_A-h_U\right)^2}\,;
 %\label{e4-ost}
 \\
\beta_V (t) &=\mathrm{arccos}\left[ \fr{X(t)}{Y(t)}\right]\,;
%\label{e5-ost}
\\
\beta_H  (t) &= \mathrm{arctan}\left( \fr{Y(t)}{X(t_0)}\right)\,,
%\label{e6-ost}
\end{align*}
где состояния $X(t_0)$ и~$X(t)$ независимого одномерного диффузионного 
процесса с~коэффициентом диффузии~$D_x$, ограниченного в~$(0,L)$, 
представляют собой координату ПУ на оси~$0x$ в~начальный~$t_0$ 
и~текущий~$t$ моменты времени; состояние $Y(t)$ такого же процесса 
с~коэффициентом диффузии~$D_y$ с~началом движения в~точке~0~--- 
координату ПУ на оси~$0y$ в~момент~$t$, при этом $X(t_0)\hm= x_0$ 
и~$Y(t_0)\hm=0$.
  
  Микромобильность ПУ в~приложениях X-VR моделируется также  
с~по\-мощью двух диффузионных процессов $G_H(t)$ и~$G_V(t)$ 
с~коэффициентами диффузии~$D_H$ и~$D_V$ с~начальными точками 
$G_H(t_0)\hm= G_V(t_0)\hm=0$. Состояния независимых процессов $G_H(t)$ 
и~$G_V(t)$ определяют значения угла азимута~$\gamma_H(t)$ и~угла 
элевации~$\gamma_V(t)$, т.\,е.\ отклонение оси антенны ПУ от направления на 
БС в~момент~$t$ в~горизонтальной и~вертикальной плоскостях соответственно.

 \begin{table*}\small
  \begin{center}
  \begin{tabular}{|c|l|c|}
  \multicolumn{3}{c}{Системные параметры}\\
  \multicolumn{3}{c}{\ }\\[-6pt]
  \hline
  Обозначение&\multicolumn{1}{c|}{Описание}&Значения\\
  \hline
$L$ &Длина стороны квадратной зоны&100 м\\
\hline
$h_U$ &Высота ПУ&1,5 м\\
\hline
$h_A$ &Высота БС NR&4 м\\
\hline
$P_T$ &Излучаемая мощность антенны БС&23 дБ\\
\hline
$f_C$ &Несущая частота&28 ГГц\\
\hline
$\zeta$ &Коэффициент затухания сигнала&2,1\\
\hline
$N_0$ &Шум&$-174$ дБи\\
\hline
$I_M$ &Интерференция&3 дБи\\
\hline
$N_H\times N_V$ &Число антенных элементов ФАР БС/ПУ&$4\times 4$, $8\times 8$, $15\times 15$\\
\hline
$\Delta x$, $\Delta y$ &Средняя скорость перемещения ПУ вдоль осей&0,8 м/с\\
\hline
$\Delta\varphi$, $\Delta\theta$ &
\tabcolsep=0pt\begin{tabular}{l}Средняя скорость отклонения оси ПУ\\ в~горизонтальной 
и~вертикальной плоскостях\end{tabular}&0--4~град/с\\
\hline
\end{tabular}
\end{center}
\end{table*}

\setcounter{figure}{1}
\begin{figure*}[b] %fig2
\vspace*{1pt}
\begin{minipage}[t]{80mm}
      \begin{center}
     \mbox{%
\epsfxsize=79mm 
\epsfbox{ost-2.eps}
}
\end{center}
\vspace*{-9pt}
\Caption{Спектральная эффективность в~зависимости от микромобильности:
\textit{1}~--- расчет по формуле~(1);
\textit{2}~--- видеоприложения;
\textit{3}~--- приложения VR; 
черные кривые~--- $x_0\hm=20$~м; 
серые кривые~---  $x_0\hm=80$~м}
\end{minipage}
%\end{figure*}
\hfill
%\begin{figure*} %fig3
\vspace*{1pt}
\begin{minipage}[t]{80mm}
      \begin{center}
     \mbox{%
\epsfxsize=79mm 
\epsfbox{ost-3.eps}
}
\end{center}
\vspace*{-9pt}
\Caption{Спектральная эффективность в~зависимости от макромобильности: \textit{1}~--- 
$D_M\hm=25$; \textit{2}~--- 50; \textit{3}~--- 100; \textit{4}~--- $D_M\hm=200$}
\end{minipage}
\end{figure*}

 

\section{Численный анализ}

  В работе предполагается наличие планарных симметричных ФАР как на БС, 
так и~на ПУ. Рассматриваются ПУ с~двумя физическими антеннами, 
расположенными на противоположных сторонах устройства, диаграммы 
направленности антенн соответствуют стандарту 3GPP TR 37.977~\cite{17-ost}. 
Модель усиления антенны имеет вид:
  \begin{multline}
  G_U(\beta_H, \beta_V) ={}\\
  {}=\varepsilon \rho_H \left(\beta_H+\gamma_H(t)\right) 
\rho_V \left(\beta_V+\gamma_V(t)\right),
  \label{e7-ost}
  \end{multline}
где $\varepsilon$~--- коэффициент усиления при идеальном 
выравнивании лучей от БС и~от ПУ; $\rho_H(\beta_H)$ и~$\rho_V(\beta_V)$~--- функции направленности для угловых 
отклонений~$\beta_H$ и~$\beta_V$ оси антенны ПУ от направления на БС 
в~горизонтальной и~вертикальной плоскостях при макромобильности, $\beta_H, 
\beta_V\hm\in [0,\pi]$; $\gamma_H$ и~$\gamma_V$~--- дополнительные угловые 
отклонения, воз\-ни\-ка\-ющие при микромобильности в~со\-от\-вет\-ст\-ву\-ющих 
плоскостях,  $\gamma_H, \gamma_V\hm\in [-\pi/2,\pi/2]$.
  
  Заметим, что в~предположении о симметричной конической антенне, т.\,е.\ 
$\beta_H\hm=\beta_V$ и~$\rho_H(\beta_H)\hm= \rho_V(\beta_V) \hm= \rho(\beta)$, 
параметры~$\varepsilon$ и~$\rho(\beta)$ зависят от ширины 2D-лу\-ча~$\alpha$, 
определяемой числом антенных элементов~$N$, и~имеют сле\-ду\-ющий 
вид~\cite{18-ost}:
  \begin{align*}
  \alpha&= 2\arccos \fr{2{,}782}{N\pi}\,;%\label{e8-ost}
  \\
  \varepsilon &= \fr{2}{1-\cos(\alpha/2)}\,;%\label{e9-ost}
  \\
  \rho(\beta) &= \begin{cases}
  1-\fr{\beta}{\alpha}\,, & \beta\leq \alpha\,;\\
  0 & \mbox{в\ противном\ случае.}
  \end{cases}
 % \label{e10-ost}
  \end{align*}
  
  Коэффициент $L(y,t)$ потери мощности принимаемого ПУ сигнала на 
расстоянии~$y$ [м] от БС в~момент~$t$ вычисляется по формуле: 
  \begin{equation*}
  L(y,t) =10^{2\lg f_C+3{,}24} y^{-\zeta}\,,
 % \label{e11-ost}
  \end{equation*}
где $f_C$~--- несущая частота; $\zeta$~--- коэффициент затухания сигнала.
  
  Параметры среды распространения сигнала и~системы по умолчанию 
представлены в~таб\-лице.
  
  
    

  
  На рис.~2 показано влияние мик\-ро\-мо\-биль\-ности на среднюю спектральную 
эффективность, вы-\linebreak чис\-ля\-емую с~помощью~(\ref{e1-ost}), для различных типов 
приложений и~двух точек $(x_0, y_0)$ старта ПУ~--- начальных точек $(20,0)$ 
и~$(80,0)$. Микромобильность, пред\-став\-лен\-ная в~предложенной модели 
процессами изменения угловых отклонений $\gamma_H(t)$ и~$\gamma_V(t)$, 
влияет на усиление антенны ПУ, как показано в~(\ref{e7-ost}). Средняя ско\-рость 
отклонения осей ПУ $\Delta\varphi\hm= \Delta\theta$ со\-став\-ля\-ет 2~град/с для 
видеоприложений и~4~град/с для приложений VR. Сред\-няя ско\-рость 
перемещения ПУ из-за макромобильности составляет $\Delta x \hm= \Delta y 
\hm=0{,}8$~м/с. По умолчанию $L\hm= 100$~м, конфигурация антенны БС~---
   $15\times15$~элементов.
  
  Влияние макромобильности на среднюю спектральную эффективность 
показано на рис.~3 для старта ПУ из начальной точки $x_0\hm=20$, $y_0\hm=0$ 
Средняя скорость перемещения ПУ из-за макромобильности $\Delta x\hm= \Delta 
y$ варьируется в~пределах  0,8--2~м/с, средняя скорость отклонения оси ПУ из-за 
микромобильности $\Delta\varphi \hm= \Delta\theta\hm= 2$~град/с. По 
умолчанию $L\hm=100$~м, конфигурация антенны БС~--- 
$15\times15$~элементов. 
  
  Полученные результаты показывают, что резкие колебания спектральной 
эффективности вследствие микромобильности проявляются на малых 
временн$\acute{\mbox{ы}}$х интервалах в~течение 1--3~с после выравнивания 
луча, тогда как макромобильность влияет на показатели средней спектральной 
эф\-фек\-тив\-ности в~течение более длительного интервала\linebreak\vspace*{-10pt}

{ \begin{center}  %fig4
 \vspace*{-3pt}
    \mbox{%
\epsfxsize=77.21mm 
\epsfbox{ost-4.eps}
}

\end{center}



\noindent
{{\figurename~4}\ \ \small{Усиление на антенне БС в~зависимости от конфигурации антенной решетки: 
\textit{1}~--- $4\times 4$; \textit{2}~--- $8\times 8$; \textit{3}~--- $15\times 15$
}}}

\vspace*{6pt}

 
  
  \noindent
   времени для всего 
диапазона исследованных скоростей перемещения ПУ. Активное перемещение 
ПУ при скорости 2~м/с ($D_M\hm= D_x\hm= D_y\hm=200$) приводит к~более 
быстрому снижению спект\-раль\-ной эф\-фек\-тив\-ности по срав\-не\-нию с~низ\-кой 
ско\-ростью перемещения ПУ 0,8~м/с ($D_M\hm= D_x\hm= D_y\hm= 25$), что 
связано с~раз\-ли\-ча\-ющей\-ся для раз\-ных скоростей ПУ динамикой сред\-не\-го 
рас\-сто\-яния до~БС.
  

  
  Кроме расстояния между БС и~ПУ спектральная эффективность зависит от 
конфигурации антенны, в~частности от ширины угла основного лепестка 
диаграммы направленности антенны БС. Как показано на рис.~4, с~течением 
времени происходит значительное падение коэффициента усиления антенны БС 
из-за рассинхронизации лучей в~результате смещения ПУ от оси антенны БС 
в~сторону границ лепестка, так что коэффициент снижается до усредненных 
значений около~25~дБи. При этом всплески на графике соответствуют 
попаданию ПУ в~боковые лепестки антенны БС. Ожидаемо рез\-че и~быст\-рее 
коэффициент падает для антенны $15\times15$~элементов, фор\-ми\-ру\-ющей более 
узкие лучи с~более ярко выраженными боковыми лепестками. Заметим, что 
дополнительное усиление, по\-лу\-ча\-емое при выравнивании антенны ПУ до 
идеальной направленности на БС, не вносит существенного вклада 
в~спектральную эффективность из-за использования для ее оценки медленно 
возрастающей логарифмической функции. 
  
\section{Заключение}

  Перспективы внедрения новых устройств с~ограниченным функционалом для 
систем 5G <<новое радио>> и~новых механизмов энергосбережения на стороне 
ПУ, позволяющих пропускать циклы синхронизации с~БС, мотивировали авторов 
на разработку аналитического аппарата для оценки спектральной эффективности 
в~беспроводном канале в~условиях сочетания макро- и~мик\-ро\-мо\-биль\-ности\linebreak 
пользователей. Модель учитывает параметры диаграммы направленности антенн 
БС и~ПУ, описанные в~стандартах 3GPP, а~также нарушение выравнивания лучей 
антенн во времени, вызванное \mbox{вращением} ПУ в~руках пользователя при 
одновременном перемещении пользователя в~зоне покрытия БС. Численные 
результаты показали, что разница между спектральной эффективностью, 
\mbox{полученной} в~предположении о~статичном местоположении пользователей 
и~идеально направленных антеннах БС и~ПУ, и~зависящей от времени 
спектральной эффективностью незначительна для покрытия малых сот 
и~находится в~пределах примерно 5\%--10\% для широкого диапазона па\-ра\-мет\-ров 
системы. Это означает, что при необходимости связь может поддерживаться без 
идеального выравнивания антенн БС и~ПУ с~несколько ухудшенным качеством. 
Влияние диаграммы на\-прав\-лен\-ности антенны БС на изменения спект\-раль\-ной 
эф\-фек\-тив\-ности во времени весьма ограничено, что приводит к~разнице до~5\% 
между ФАР $4\times4$~элементов и~$15\times15$~элементов. С~точки зрения 
моделирования стоит отметить, что использование реалистичных диаграмм 
на\-прав\-лен\-ности излучения антенн критически важно при исследовании сис\-тем 
с~макро- и~мик\-ро\-мо\-биль\-ностью.
  
{\small\frenchspacing
 { %\baselineskip=10.6pt
 %\addcontentsline{toc}{section}{References}
 \begin{thebibliography}{99}
\bibitem{1-ost}
\Au{Holma H., Toskala~A., Nakamura~T.} 5G technology: 3GPP new radio.~--- Hoboken, NJ, 
USA: John Wiley \&~Sons, 2020. 536~p.
\bibitem{2-ost}
\Au{Jiang W., Han~B., Habibi~M.\,A., Schotten~H.\,D.} The road towards 6G: A~comprehensive 
survey~// IEEE Open J.~Communications Society, 2021. Vol.~2. P.~334--366. doi: 
10.1109/OJCOMS.2021.3057679.
\bibitem{3-ost}
\Au{Saad W., Bennis~M., Chen~M.} A~vision of 6G wireless systems: Applications, trends, 
technologies, and open research problems~// IEEE Network, 2019. Vol.~34. No.\,3. P.~134--142. 
doi: 10.1109/MNET.001.1900287.
\bibitem{4-ost}
\Au{Zhang J., Ge~X., Li~Q., Guizani~M., Zhang~Y.} 5G millimeter-wave antenna array: Design 
and challenges~// IEEE Wirel. Commun., 2016. Vol.~24. No.\,2. P.~106--112. doi: 
10.1109/MWC.2016.1400374RP.
\bibitem{5-ost}
\Au{Guo Y.\,J., Ziolkowski~R.\,W.} Advanced antenna array engineering for 6G and beyond 
wireless communications.~--- Hoboken, NJ, USA: John Wiley \&~Sons, 2021. 316~p.
\bibitem{6-ost}
3GPP. NR; Physical layer; General description (Release 18). 3GPP TS 38.201 V18.0.0, 
2023.
\bibitem{7-ost}
3GPP. Study on support of reduced capability NR devices (Release 17): Technical Specification 
38.875 V17.0.0,  2021. 136~p. {\sf
https://www.3gpp.org/ftp/Specs/ archive/38\_series/38.875/38875-h00.zip}.
\bibitem{8-ost}
\Au{Stepanov N., Moltchanov~D., Begishev~V., Turlikov~A., Koucheryavy~Y.} Statistical analysis 
and modeling of user micromobility for THz cellular communications~// IEEE T. 
Veh. Technol., 2021. Vol.~71. No.\,1. P.~725--738. doi: 10.1109/TVT.2021.3124870.
\bibitem{9-ost}
\Au{Moltchanov D., Gaidamaka~Y., Ostrikova~D., Beschastnyi~V., Koucheryavy~Y., 
Samouylov~K.} Ergodic outage and capacity of terahertz systems under micromobility and 
blockage impairments~// IEEE T. Wirel. Commun., 2021. Vol.~21. No.\,5. 
P.~3024--3039. doi: 10.1109/ TWC.2021.3117583.
\bibitem{10-ost}
\Au{Haenggi M.} Stochastic geometry for wireless networks.~--- Cambridge, U.K.: Cambridge 
University Press, 2012. 298~p.
\bibitem{11-ost}
\Au{Petrov V., Komarov~M., Moltchanov~D., Jornet~J.\,M., Koucheryavy~Y.} Interference and 
SINR in millimeter wave and terahertz communication systems with blocking and directional 
antennas~// IEEE T. Wirel. Commun., 2017. Vol.~16. No.\,3.  
P.~1791--1808. doi: 10.1109/ TWC.2017.2654351.

\bibitem{13-ost} %12
\Au{Горбунова А.\,В., Наумов~В.\,А., Гайдамака~Ю.\,В., Самуйлов~К.\,Е.} Ресурсные 
системы массового обслуживания как модели беспроводных систем связи~// Информатика и~её применения, 2018. Т.~12. Вып.~3. С. 48--55. doi: 10.14357/19922264180307.
EDN: YAMDIL.

\bibitem{12-ost} %13
\Au{Kovalchukov R., Moltchanov~D., Gaidamaka~Y., Bobrikova~E.} An accurate approximation 
of resource request distributions in millimeter wave 3GPP new radio systems~// Internet of 
things, smart spaces, and next generation networks and systems~/ Eds. O.~Galinina, S.~Andreev, 
S.~Balandin, Y.~Koucheryavy.~--- Lecture notes in computer science ser.~--- Cham: Springer, 
2019. Vol.~11660. P.~572--585. doi: 10.1007/978-3-030-30859-9\_50.

\bibitem{14-ost}
\Au{Бесчастный В.\,А., Голос~Е.\,С., Острикова~Д.\,Ю., Мач\-нев~Е.\,А., Шоргин~В.\,С., 
Гайдамака~Ю.\,В.} Анализ \mbox{совместного} использования стратегий энергосбережения для 
устройств 5G с~ограниченным функционалом~// Сис\-те\-мы и~средства информатики, 2023. 
Т.~33. №\,4. С.~69--81. doi: 10.14357/08696527230407.  EDN: KATMLB.
\bibitem{15-ost}
\Au{Rappaport T.\,S.} Wireless communications: principles and practice.~--- 2nd ed.~--- Cambridge, U.K.: 
Cambridge University Press, 2024. 708~p. doi: 10.1017/9781009489843.
\bibitem{16-ost}
\Au{Risken H.} Fokker--Planck equation.~--- Springer, 1996. 472~p.
\bibitem{17-ost}
3GPP. Universal Terrestrial Radio Access (UTRA) and Evolved Universal Terrestrial Radio 
Access (E-UTRA); Verification of radiated multi-antenna reception performance of User 
Equipment (UE). 3GPP TR 37.977 V17.0, 2022.
\bibitem{18-ost}
\Au{Chukhno O., Chukhno~N., Galinina~O., Andreev~S., Gaidamaka~Y., Samouylov~K., 
Araniti~G.} A~Holistic assessment of directional deafness in mmWave-based distributed 3D 
networks~// IEEE T. Wirel. Commun., 2022. Vol.~21. No.\,9. P.~7491--7505. 
doi: 10.1109/TWC.2022.3159086.

\end{thebibliography}

 }
 }

\end{multicols}

\vspace*{-6pt}

\hfill{\small\textit{Поступила в~редакцию 14.03.24}}

\vspace*{10pt}

%\pagebreak

%\newpage

%\vspace*{-28pt}

\hrule

\vspace*{2pt}

\hrule



\def\tit{ASSESSING THE CHARACTERISTICS OF~5G/6G ``NEW RADIO'' SYSTEMS WITH~USER'S 
MACRO- AND~MICROMOBILITY}


\def\titkol{Assessing the characteristics of~5G/6G ``new radio'' systems with~user's 
macro- and~micromobility}


\def\aut{D.\,Yu.~Ostrikova$^1$, E.\,S.~Golos$^1$, V.\,A.~Beschastnyi$^1$, E.\,A.~Machnev$^1$,  
V.\,S.~Shorgin$^2$,\\ and~Yu.\,V.~Gaidamaka$^{1,2}$}

\def\autkol{D.\,Yu.~Ostrikova, E.\,S.~Golos, V.\,A.~Beschastnyi, et al.}
%E.\,A.~Machnev$^1$,   V.\,S.~Shorgin$^2$, and~Yu.\,V.~Gaidamaka$^{1,2}$}

\titel{\tit}{\aut}{\autkol}{\titkol}

\vspace*{-8pt}


\noindent
$^1$RUDN University, 6~Miklukho-Maklaya Str., Moscow 117198, Russian Federation 

\noindent
$^2$Federal Research Center ``Computer Science and Control'' of the Russian Academy of Sciences,  
44-2~Vavilov\linebreak
$\hphantom{^1}$Str., Moscow 119333, Russian Federation


\def\leftfootline{\small{\textbf{\thepage}
\hfill INFORMATIKA I EE PRIMENENIYA~--- INFORMATICS AND
APPLICATIONS\ \ \ 2024\ \ \ volume~18\ \ \ issue\ 2}
}%
 \def\rightfootline{\small{INFORMATIKA I EE PRIMENENIYA~---
INFORMATICS AND APPLICATIONS\ \ \ 2024\ \ \ volume~18\ \ \ issue\ 2
\hfill \textbf{\thepage}}}

\vspace*{4pt}







\Abste{The performance of 5G/6G cellular ``new radio'' systems is typically evaluated using static 
user location and perfectly directional antennas which are justified in the case of regular 
synchronisation between the user equipment (UE) and the base station (BS). However, in the case of 
high energy-efficient UEs with limited RedCap functionality, BS is less likely to get 
information about the quality of the signal received by the device which changes when the UE moves. 
This leads to the need to investigate the dynamics of the performance indicators of systems with 
RedCap UEs over time. In the paper, tools of stochastic geometry and random walk theory 
are used to analyze the spectral efficiency depending on the distance between the BS and the UE and 
the directionality of the UE antenna at a random moment of time. A~numerical experiment has shown 
that macromobility has a~significant impact on the spectral efficiency, the impact of micromobility is 
smaller and appears only at short time intervals, while the size of the phased antenna array on the BS 
side practically does not affect the obtained result.}

\KWE{5G new radio; mmWave; sub-THz; micromobility; macromobility; spectral efficiency} 



\DOI{10.14357/19922264240205}{JCUFHS}

\vspace*{-12pt}

\Ack

\vspace*{-3pt}

\noindent 
The reported study was funded by the Russian Science Foundation, project No.\,23-79-10084. 
   


  \begin{multicols}{2}

\renewcommand{\bibname}{\protect\rmfamily References}
%\renewcommand{\bibname}{\large\protect\rm References}

{\small\frenchspacing
 {%\baselineskip=10.8pt
 \addcontentsline{toc}{section}{References}
 \begin{thebibliography}{99} 
 \bibitem{1-ost-1}
\Aue{Holma, H., A.~Toskala, and T.~Nakamura.} 2020. \textit{5G technology: 3GPP new radio}. 
New York, NY: John Wiley \&~Sons. 536~p.
\bibitem{2-ost-1}
\Aue{Jiang, W., B.~Han, M.\,A.~Habibi, and H.\,D.~Schotten.} 2021. The road towards 6G: 
A~comprehensive survey. \textit{IEEE Open J. Communications Society} 2:334--366. doi: 
10.1109/OJCOMS.2021.3057679.
\bibitem{3-ost-1}
\Aue{Saad, W., M.~Bennis, and M.~Chen.} 2019. A~vision of 6G wireless systems: Applications, 
trends, technologies, and open research problems. \textit{IEEE Network} 34(3):134--142. doi: 
10.1109/MNET.001.1900287.
\bibitem{4-ost-1}
\Aue{Zhang, J., X.~Ge, Q.~Li, M.~Guizani, and Y.~Zhang.} 2016. 5G millimeter-wave antenna array: 
Design and challenges. \textit{IEEE Wirel. Commun.} 24(2):106--112. doi: 
10.1109/MWC.2016.1400374RP.
\bibitem{5-ost-1}
\Aue{Guo, Y.\,J., and R.\,W.~Ziolkowski.} 2021. \textit{Advanced antenna array engineering for 6G 
and beyond wireless communications}. Hoboken, NJ: John Wiley \&~Sons. 316~p.
\bibitem{6-ost-1}
3GPP. 2023. NR; Physical layer; General description (Release 18). 3GPP TS 38.201 V18.0.0.  
\bibitem{7-ost-1}
3GPP. 2021. Study on support of reduced capability NR devices (Release 17): Technical Specification 
38.875 V17.0.0. 136 p. Available at: {\sf  
https://www.3gpp.org/ftp/\linebreak Specs/archive/38\_series/38.875/38875-h00.zip} (accessed May 17, 2024).
\bibitem{8-ost-1}
\Aue{Stepanov, N.\,V., D.~Moltchanov, V.~Begishev, A.~Turlikov, and Y.~Koucheryavy}. 2021. 
Statistical analysis and modeling of user micromobility for THz cellular communications. \textit{IEEE 
T. Veh. Technol.} 71(1):725--738. doi: 10.1109/TVT.2021.3124870.
\bibitem{9-ost-1}
\Aue{Moltchanov, D., Y.~Gaidamaka, D.~Ostrikova, V.~Beschastnyi, Y.~Koucheryavy, and 
K.~Samouylov.} 2021. Ergodic outage and capacity of terahertz systems under micromobility and 
blockage impairments. \textit{IEEE T. Wirel. Commun.} 21(5):3024--3039. doi:  
10.1109/ TWC.2021.3117583.
\bibitem{10-ost-1}
\Aue{Haenggi, M.} 2012. \textit{Stochastic geometry for wireless networks}. Cambridge, U.K.: Cambridge 
University Press. 298~p.
\bibitem{11-ost-1}
\Aue{Petrov, V., M.~Komarov, D.~Moltchanov, J.\,M.~Jornet, and Y.~Koucheryavy.} 2017. 
Interference and SINR in millimeter wave and terahertz communication systems with blocking and 
directional antennas. \textit{IEEE T. Wirel. Commun.} 16(3):1791--1808. doi: 
10.1109/TWC.2017.2654351.

\bibitem{13-ost-1} %12
\Aue{Gorbunova, A.\,V., V.\,A.~Naumov, Yu.\,V.~Gaydamaka, and K.\,E.~Samuylov.} 2018. 
Resursnye sistemy massovogo obsluzhivaniya kak modeli besprovodnykh sistem svyazi [Resource 
queuing systems as models of wireless communication systems]. \textit{Informatika i~ee 
Primeneniya~--- Inform. Appl.} 12(3):48--55. doi: 10.14357/19922264180307. EDN: YAMDIL.

\bibitem{12-ost-1} %13
\Aue{Kovalchukov, R., D.~Moltchanov, Y.~Gaidamaka, and E.~Bobrikova.} 2019. An accurate 
approximation of resource request distributions in millimeter wave 3GPP new radio systems. 
\textit{Internet of things, smart spaces, and next generation networks and systems}. Eds. O.~Ga\-li\-ni\-na, 
S.~And\-re\-ev, S.~Balandin, and Y.~Koucheryavy. Lecture notes in computer science ser. Cham: 
Springer. 11660:572--585. doi: 10.1007/978-3-030-30859-9-50.
\bibitem{14-ost-1}
\Aue{Beschastnyi, V.\,A., E.\,S.~Golos, D.\,Yu.~Ostrikova, E.\,A.~Machnev, V.\,S.~Shorgin, and 
Yu.\,V.~Gaidamaka.} 2023. Analiz sovmestnogo ispol'zovaniya strategiy energosberezheniya dlya 
ustroystv 5G s ogranichennym funktsionalom [Analysis of joint usage of energy conservation 
strategies for 5G devices with reduced capability]. \textit{Sistemy i~Sredstva Informatiki~--- Systems 
and Means of Informatics} 33(4):69--81. doi: 10.14357/08696527230407. EDN: KATMLB.
\bibitem{15-ost-1}
\Aue{Rappaport, T.\,S.} 2024. \textit{Wireless communications: Principles and practice}. 2nd ed. 
Cambridge, U.K.: Cambridge University Press. 708~p. doi: 10.1017/9781009489843.
\bibitem{16-ost-1}
\Aue{Risken, H.} 1996. \textit{Fokker--Planck equation}. Springer. 472~p.
\bibitem{17-ost-1}
3GPP. 2022. Universal terrestrial radio access (UTRA) and evolved universal terrestrial radio access 
(E-UTRA); Verification of radiated multi-antenna reception performance of User Equipment (UE). 
3GPP TR 37.977 V17.0.0 Release 17.  
\bibitem{18-ost-1}
\Aue{Chukhno, O., N.~Chukhno, O.~Galinina, S.~Andreev, Y.~Gaidamaka, K.~Samouylov, and 
G.~Araniti.} 2022. A~holistic assessment of directional deafness in mmWave-based distributed 3D 
networks. \textit{IEEE T. Wirel. Commun.} 21(9):7491--7505. doi: 10.1109/TWC.2022.3159086.



\end{thebibliography}

 }
 }

\end{multicols}

\vspace*{-6pt}

\hfill{\small\textit{Received March 14, 2024}} 


\vspace*{-12pt}


\Contr

\vspace*{-3pt}

\noindent
\textbf{Ostrikova Daria Yu.} (b.\ 1988)~--- Candidate of Science (PhD) in physics and mathematics, 
associate professor, Department of Probability Theory and Cyber Security, RUDN University, 
6~Miklukho-Maklaya Str., Moscow 117198, Russian Federation; \mbox{ostrikova-dyu@rudn.ru}

\vspace*{3pt}


\noindent
\textbf{Golos Elizaveta S.} (b.\ 1998)~--- PhD student, Department of Probability Theory and Cyber 
Security, RUDN University, 6~Miklukho-Maklaya Str., Moscow 117198, Russian Federation; 
\mbox{1142210130@rudn.ru}

%\vspace*{3pt}

\pagebreak

\noindent
\textbf{Beschastnyi Vitalii A.} (b.\ 1992)~--- Candidate of Science (PhD) in physics and mathematics, 
associate professor, Department of Probability Theory and Cyber Security, RUDN University,  
6~Miklukho-Maklaya Str., Moscow  117198, Russian Federation; \mbox{beschastnyy-va@rudn.ru}

\vspace*{3pt}

\noindent
\textbf{Machnev Egor A.} (b.\ 1996)~--- PhD student, Department of Probability Theory and Cyber 
Security, RUDN University, 6~Miklukho-Maklaya Str., Moscow 117198, Russian Federation; 
\mbox{1042200071@rudn.ru}

\vspace*{3pt}

\noindent
\textbf{Shorgin Vsevolod S.} (b.\ 1978)~--- Candidate of Science (PhD) in technology, senior 
scientist, Federal Research Center ``Computer Science and Control'' of the Russian Academy of 
Sciences, 44-2~Vavilov Str., Moscow 119333, Russian Federation; \mbox{vshorgin@ipiran.ru}

\vspace*{3pt}

\noindent
\textbf{Gaidamaka Yuliya V.} (b.\ 1971)~--- Doctor of Science in physics and mathematics, professor, 
Department of Probability Theory and Cyber Security, RUDN University, 6~Miklukho-Maklaya Str., 
Moscow 117198, Russian Federation; senior scientist, Federal Research Center ``Computer Science 
and Control'' of the Russian Academy of Sciences, 44-2~Vavilov Str., Moscow 119333, Russian 
Federation; \mbox{gaydamaka-yuv@rudn.ru}





\label{end\stat}

\renewcommand{\bibname}{\protect\rm Литература} 