\documentclass[10pt]{book}
\usepackage[utf8]{inputenc}

\usepackage{latexsym,amssymb,amsfonts,amsmath,amsxtra,dsfont,amscd,
indentfirst,shapepar,%fleqn,%
picinpar,shadow,floatflt,enumerate,multicol,colortbl,moreverb,cite,ipi}

\usepackage{rotating}
\usepackage{mathrsfs}
\usepackage[noend]{algorithmic}
\usepackage{ulem}
\usepackage{graphicx}
%\usepackage{algorithm2e}
\usepackage[linesnumbered,boxed,ruled]{algorithm2e}
%\usepackage{xypic}
\usepackage{oldgerm}
\usepackage{epic}
\usepackage{eepic}

\SetAlgorithmName{Algorithm}{алгоритм}{Список алгоритмов}

%из Дюковой

\newcommand{\algKeyword}[1]{{\bf #1}}
\newcommand{\Proc}[1]{\text{\tt #1}}
\def\CALL{\algKeyword{call}~}

\newenvironment{AlgProcedure}[1]
{
\small
\medskip
%    \hrule
\medskip
\algKeyword{PROCEDURE} #1
\begin{algorithmic}[1]}
{\end{algorithmic}
%    \hrule
\bigskip
}

\def\CALL{\algKeyword{call}~}

%конец для Дюковой

%\RequirePackage[ruled]{algorithm}


\input{epsf}

%\nofiles

%\includeonly{avtor}    %pdf
%\includeonly{podgot-rus-site,podgot-eng-site}  
%\includeonly{podgot-rus,podgot-eng}  
%\includeonly{ipi-ind} 
%\includeonly{index-17i}
%\includeonly{toc-rus, toc-en}
%\includeonly{toc-rus}
%\includeonly{toc-en} 
%\includeonly{popravka}



%\includeonly{vasiliev}       %1+pdf+авт+
%\includeonly{borisov}        %2+pdf+авт+повт отпр
%\includeonly{lange}          %3+pdf+авт+
%\includeonly{kochetkova}     %4+pdf+авт+
%\includeonly{ostrikova}      %5+pdf+авт+
%\includeonly{agalarov}       %6+pdf
%\includeonly{torshin}        %7+pdf+авт
%\includeonly{grusho}         %8+pdf+авт+
%\includeonly{kovalenko}      %9+pdf
%\includeonly{bosov}          %10+pdf
%\includeonly{zatsman}        %11+pdf+авт+




%%%%%%%%%%%%%%%%%%%\includeonly{nekrolog-new}



%\includeonly{rekl}




\usepackage{acad}
%\usepackage{courier}
\usepackage{decor}
\usepackage{newton}
\usepackage{pragmatica}
\usepackage{zapfchan}
\usepackage{petrotex}
\usepackage{bm}                     % полужирные греческие буквы
\usepackage{upgreek}                % прямые греческие буквы \upalpha
\usepackage{eufrak}
\usepackage{verbatim}

\renewcommand{\bottomfraction}{0.99}
\renewcommand{\topfraction}{0.99}
\renewcommand{\textfraction}{0.01}

\setcounter{secnumdepth}{1} %здесь - 3 + chapter = 4

\arraycolsep=1.5pt

%\usepackage[pdftex]{graphicx}

%\usepackage{oz}

%NEW COMMANDS



\renewcommand*{\hm}[1]{#1\nobreak\discretionary{}%
            {\hbox{$\mathsurround=0pt #1$}}{}} %% Дублирует знаки операций
                               %при переносе в формуле (перед знаком, который
                               %надо продублировать ставится команда \hm)
                               
                               \newcommand{\PRB}{\begin{picture}(22.5,11)
      \spline(1,8)(4,10)(7,10.5)(10,11)(13,11)(16,10.5)(19,10)(22,8)
               \put(0,0){$P_{i-1}P_{t_{t-1}}$} \end{picture}}

\newcommand{\prb}{\begin{picture}(15.5,9)
      \spline(1,6)(3,8)(5,8.5)(7,9)(9,9)(11,8.5)(13,8)(15,6)
               \put(0,0){$PP_t$} \end{picture}}
               
                 \newcommand{\PRDN}{\begin{picture}(40,11)
      \spline(4,11.5)(7,10.5)(12,10)(16,9)(20,9)(24,10)(29,10.5)(32,11.5)
               \put(0,0){$P_{i-1}P_{t_{t-1}}$} \end{picture}}

\newcommand{\prdn}{\begin{picture}(18,11)
      \spline(3,10.5)(4,10)(6,9)(8,8.5)(10,8.5)(12,9)(14,10)(15,10.5)
               \put(0,0){$PP_t$} \end{picture}}




%\newcommand{\endproof}{\hfill$\Box$}
%\renewcommand{\r}{\mathbb{R}}
%\newcommand{\I}{{\rm I\hspace{-0.7mm}I}}
%\newcommand{\Ikl}{{\tt{1}}\hspace*{-1.44mm}\mathtt{1}}
%\newcommand{\Ik}{\mbox{{\small \tt {1}}\hspace{-1.3mm}{\tt 1}}}
\newcommand{\Ik}{\mbox{{{\tt 1}}\hspace{-1.3mm}{\sf 1}}}
\newcommand{\argmin}{\mathop{\mathrm{arg}\,\mathrm{min}}}
\newcommand{\argmax}{\mathop{\mathrm{arg}\,\mathrm{max}}}
%\newcommand{\capr}{\mathop{\cap\,}}
%\newcommand{\cupr}{\mathop{\cup\,}}
%\def\argmin{\mathop{arg\,min}}

\def\vrp{\varphi}
\def\prt{\partial}
\def\mm{{\sf M}}
\def\modnop#1{\mathop{#1}\limits_{n}}
\def\eam{\mathbin{{\mathop{=}\limits^{\mathrm{def}}}}}
\def\dey#1#2{#1 (#2)}
\def\deyc#1#2{#1 \cdot  #2}
\def\ra#1{\;\mathop{\to}\limits^{#1}\;}
\def\raz#1{\;\mathop{\longrightarrow}\limits^{\!\!\!#1}\;}
\def\ral#1{\;\mathop{\longrightarrow}\limits^{#1}\;}





\newcommand{\il}[2]{\int\limits_{#1}^{#2}}%интеграл с пределами #1 и #2

\def\sm2{\mathop {\sum\limits^{n^\Theta}\sum\limits^{n^\Theta}}}
\def\sss{\sum\limits}
\def\tr{,\,\ldots\,,\,}
\def\rk{\right]}
\def\lk{\left[}
\def\rf{\right\}}
\def\lf{\left\{}
\def\lv{\,\left\vert}
\def\rv{\right\vert\,}
\def\iii{\int\limits}
\def\iin{\int\limits_{-\infty}^\infty}
\def\rrv{\right\vert}


\def\ee{{\cal E}}
\def\ww{{\cal W}}
\def\yy{{\cal Y}}
\def\vv{{\cal V}}

\newcommand{\R}{\mathbb R}
\newcommand{\E}{\mathbb E}
\newcommand{\N}{\mathbb N}
\newcommand{\T}{\mathbb{T}}
\newcommand{\Z}{\mathbb{Z}}

\renewcommand{\P}{\mathbb{P}}

\newcommand{\Nor}{\mathcal{N}}

\newcommand{\h}{{\bf H}}
\newcommand{\p}{{\sf P}}  % вероятность
\newcommand{\e}{{\sf E}}  % мат. ожидание
\newcommand{\D}{{\sf D}}  % дисперсия



\newcommand{\vw}{{\mathbf w}}
\newcommand{\vp}{{\mathbf p}}
\newcommand{\vz}{{\mathbf z}}
\newcommand{\vx}{{\mathbf x}}
\newcommand{\vf}{{\mathbf f}}
\newcommand{\F}{{\mathcal F}}
\def\ap{{\mathrm{ЭР}}}
\newcommand{\ud}{\Delta_n} %uniform ditance
\newcommand{\nud}{\Delta_n(x)}
%\renewcommand{\Re}{\mathrm{Re}\,}

\newcommand{\abs}[1]{\left\vert#1\right\vert}

\newcommand{\norm}[1]{\left\Vert#1\right\Vert}
\def\da{(\Delta_t,A)}

\newcommand{\corr}{\mathrm{corr}}

\newcommand{\cov}{\mathrm{cov}}
\newcommand{\Expect}{\mathbb{E}}

\def\w{\omega}
\def\W{\Omega}


\def\inh{\int\limits_{nh}^{(n+1)h}}

\def\sumin{\sum_{i=1}^N}


\def\bxt{(Y,t)}
\def\xt{(y,t)}

\def\ovth{{\fr{\tau-nh}{h}}}
\def\ov{\overline}
\def\tm{\tilde m}
\def\tl{\tilde\lambda}
\def\tB{\widetilde B}
\def\tb{\tilde b}
\def\ld{\ldots}
\def\cd{\cdots}


\DeclareMathOperator{\sign}{sign}



\newcommand{\g}{\mbox{\textit{g}}}

\renewcommand{\la}{\lambda}
\newcommand{\si}{\sigma}
\newcommand{\eps}{\varepsilon}
\newcommand{\alp}{\alpha}

\newcommand{\pto}{\stackrel{P}{\longrightarrow}} % сходимость по веpоятности

\newcommand{\eqd}{\stackrel{\mathrm{d}}{=}} % равенство по pаспpеделению
\newcommand{\eqdelta}{\stackrel{\triangle}{=}} % равенство по pаспpеделению

\def\be#1{\begin{equation}\label{#1}}
\def\ee{\end{equation}}
\def\re#1{(\ref{#1})}

\def\bn{\begin{enumerate}}
\def\en{\end{enumerate}}
\def\bi{\begin{itemize}}
\def\ei{\end{itemize}}
%\def\i{\item}

%\newcommand{\kp}{\kappa}
%\def\Q{{\cal Q}} \def\H{{\cal H}}
%\newcommand{\bet}{\beta_{2+\delta}}




%\renewcommand{\baselinestretch}{1.2}

%\pagestyle{myheadings}

\setlength{\textwidth}{167mm}      % 122mm
\setlength{\textheight}{658pt}
%\setlength{\textheight}{635.6pt}
\setlength{\columnsep}{4.5mm}

\setcounter{secnumdepth}{4}

%\addtolength{\headheight}{2pt}
%\addtolength{\headsep}{-2mm}

\addtolength{\topmargin}{-7mm}  % for printing


%\hoffset=-30mm  % From Yap
\hoffset=-23mm  % From Acrobat

%\voffset=0mm % From Yap
\voffset=-5mm   % From Acrobat

%\addtolength{\evensidemargin}{-2.5mm} % for printing
%\addtolength{\oddsidemargin}{2.5mm}  % for printing

\addtolength{\evensidemargin}{-12mm} % for printing
\addtolength{\oddsidemargin}{8mm}  % for printing

%\renewcommand{\thefootnote}{\fnsymbol{footnote}}
%\renewcommand{\thefootnote}{\arabic{footnote}}
\renewcommand{\figurename}{\protect\bf Рис.}
\renewcommand{\tablename}{\protect\bf Таблица}

\newcommand{\Caption}[1]{\caption{\protect\small %\baselineskip=2.5ex
#1}}

\renewcommand{\thefigure}{\arabic{figure}}
\renewcommand{\thetable}{\arabic{table}}
\renewcommand{\theequation}{\arabic{equation}}
\renewcommand{\thesection}{\arabic{section}}

\renewcommand{\contentsname}{СОДЕРЖАНИЕ}
\newcommand{\fr}[2]{\displaystyle\frac{\displaystyle #1\mathstrut}{\displaystyle #2\mathstrut}}

%\renewcommand{\thefootnote}{\fnsymbol{footnote}}
%\newcommand{\g}{\mbox{\textit{g}}}

%\newcommand{\Caption}[1]{\caption{\protect\small\baselineskip=2ex #1}}
\newcounter{razdel}
\setcounter{razdel}{0}

\def\god{2024}
\def\tom{18}
\def\vyp{2}


\newcommand{\titel}[4]{%
\

\vspace*{5pt}

\ifodd\therazdel {\raggedright\noindent\Large\textrm\textbf
 \lineskip .75em
  \baselineskip=3.2ex #1 \par}
\vskip 1em {\noindent\large\textrm\textbf #2 \par}
\addcontentsline{toc}{subsection}{{\textrm\textbf #1}\protect\newline #2}
\def\rightheadline{\underline{\noindent\hbox to \textwidth{\hfill\small\textrm{#4}
%\hfill \large\bf\thepage
}}}
\def\leftheadline{\underline{\noindent\parbox{\textwidth}{
%\raggedleft\large\bf\thepage \hfill
\small\textit{#3}\hfill}}}
\def\leftfootline{\small{\textbf{\thepage}
\hfill ИНФОРМАТИКА И ЕЁ ПРИМЕНЕНИЯ\ \ \ том~\tom\ \ \ выпуск~\vyp\ \ \ \god}
}%
 \def\rightfootline{\small{ИНФОРМАТИКА И ЕЁ ПРИМЕНЕНИЯ\ \ \ том~\tom\ \ \ выпуск~\vyp\ \ \ \god
\hfill \textbf{\thepage}}}
\vskip 2em \setcounter{figure}{0}
\setcounter{table}{0}
\setcounter{equation}{0}
\setcounter{section}{0}
\setcounter{subsection}{0}
\setcounter{subsubsection}{0}
\setcounter{footnote}{0}
\setcounter{razdel}{0}
%\end{flushleft}
\else {
 \raggedright\noindent\Large\textrm\textbf
 \lineskip .75em
\baselineskip=3.2ex #1 \par} \vskip 1em
%\begin{flushleft}
{\noindent\large\textrm\textbf #2 \par}
\addcontentsline{toc}{subsection}{{\textrm\textbf #1}\protect\newline #2}
\def\rightheadline{\underline{\noindent\hbox to \textwidth{\hfill\small\textrm{#4}
%\hfill \large\bf\thepage
}}}
\def\leftheadline{\underline{\noindent\parbox{\textwidth}{%\raggedleft\large\bf\thepage \hfill
\small\textit{#3}\hfill}}}
\def\leftfootline{\small{\textbf{\thepage}
\hfill ИНФОРМАТИКА И ЕЁ ПРИМЕНЕНИЯ\ \ \ том~\tom\ \ \ выпуск~\vyp\ \ \ \god}
}%
 \def\rightfootline{\small{ИНФОРМАТИКА И ЕЁ ПРИМЕНЕНИЯ\ \ \ том~18\ \ \ выпуск~\vyp\ \ \ 2024
\hfill \textbf{\thepage}}} \vskip 2em \setcounter{figure}{0}
\setcounter{table}{0} \setcounter{equation}{0} \setcounter{section}{0}
\setcounter{subsection}{0} \setcounter{subsubsection}{0}
\setcounter{footnote}{0}
%\end{flushleft}
\fi}

\newcommand{\titelr}[2]{%
\

\vspace*{5pt}

\ifodd\therazdel {\raggedright\noindent%\Large\textrm\textbf
 \lineskip .75em
  \baselineskip=3.2ex #1 \par}
\vskip 1em {\noindent\normalsize\textrm\textbf #2 \par}
\else {
 \raggedright\noindent\Large\textrm\textbf
 \lineskip .75em
\baselineskip=3.2ex #1 \par} \vskip 1em
%\begin{flushleft}
{\noindent\large\textrm\textbf #2 \par
%\noindent\normalsize\textrm\textbf #2 \par
} \fi}

\newcommand{\titele}[5]{%
\

%\vspace*{5pt}

\ifodd\therazdel {\raggedright\noindent\large
\textrm\textbf
 \lineskip .75em
%  \baselineskip=3.2ex
#1 \par}
\vskip .5em {\noindent\large\textrm\textbf #2 \par}
\vskip .5em
 {\noindent\textrm #3 \par}
\addcontentsline{toc}{subsection}{{\textrm\textbf #1}\protect\newline #2}
\def\rightheadline{\underline{\noindent\hbox to \textwidth{\hfill\small\textrm{#4}
%\hfill \large\bf\thepage
}}}
\def\leftheadline{\underline{\noindent\parbox{\textwidth}{
%\raggedleft\large\bf\thepage \hfill
\small\textrm{#5}\hfill}}}
\def\leftfootline{\small{\textbf{\thepage}
\hfill ИНФОРМАТИКА И ЕЁ ПРИМЕНЕНИЯ\ \ \ том~18\ \ \ выпуск~2\ \ \ 2024}
}%
 \def\rightfootline{\small{ИНФОРМАТИКА И ЕЁ ПРИМЕНЕНИЯ\ \ \ том~18\ \ \ выпуск~2\ \ \ 2024
\hfill \textbf{\thepage}}} \vskip 1em \setcounter{figure}{0}
\setcounter{table}{0} \setcounter{equation}{0} \setcounter{section}{0}
\setcounter{subsection}{0} \setcounter{subsubsection}{0}
\setcounter{footnote}{0} \setcounter{razdel}{0}
%\end{flushleft}
\else {
 \raggedright\noindent\large
 \textrm\textbf
 \lineskip .75em
%\baselineskip=3.2ex
#1 \par} \vskip .5em
%\begin{flushleft}
{\noindent\large\textrm\textbf #2 \par} \vskip .5em
 {\noindent\textrm #3 \par}
\addcontentsline{toc}{subsection}{{\textrm\textbf #1}\protect\newline #2}
\def\rightheadline{\underline{\noindent\hbox to \textwidth{\hfill\small\textrm{#4}
%\hfill \large\bf\thepage
}}}
\def\leftheadline{\underline{\noindent\parbox{\textwidth}{%\raggedleft\large\bf\thepage \hfill
\small\textrm{#5}\hfill}}}
\def\leftfootline{\small{\textbf{\thepage}
\hfill ИНФОРМАТИКА И ЕЁ ПРИМЕНЕНИЯ\ \ \ том~18\ \ \ выпуск~2\ \ \ 2024}
}%
 \def\rightfootline{\small{ИНФОРМАТИКА И ЕЁ ПРИМЕНЕНИЯ\ \ \ том~18\ \ \ выпуск~2\ \ \ 2024
\hfill \textbf{\thepage}}} \vskip 1em \setcounter{figure}{0}
\setcounter{table}{0} \setcounter{equation}{0} \setcounter{section}{0}
\setcounter{subsection}{0} \setcounter{subsubsection}{0}
\setcounter{footnote}{0}
%\end{flushleft}
\fi}

\def\Abst#1{
\begin{center}\small\nwt
\parbox{150mm}{%\baselineskip=2.5ex
\textbf{Аннотация:}\ \
%\hspace*{\parindent}
#1}
\end{center}}
\def\Abste#1{
\begin{center}\small\nwt
\parbox{150mm}{%\baselineskip=2.5ex
\textbf{Abstract:}\ \
%\hspace*{\parindent}
#1}
\end{center}}

%\def\DOI#1{
%\begin{center}\small\nwt
%\parbox{150mm}{%\baselineskip=2.5ex
%\textbf{DOI:}\ \
%%\hspace*{\parindent}
%#1}
%\end{center}}

\def\Abstend#1{
\begin{center}\small\nwt
\parbox{150mm}{%\baselineskip=2.5ex
%\hspace*{\parindent}
#1}
\end{center}}

\newcommand{\DOI}[2]{\begin{center}\small\nwt
\parbox{150mm}{%\baselineskip=2.5ex
\textbf{DOI:}\ \
%\hspace*{\parindent}
#1 \hfill \textbf{EDN:}\ \
#2}
\end{center}}




\def\KW#1{
\begin{center}\small\nwt
\parbox{150mm}{%\baselineskip=2.5ex
\textbf{Ключевые слова:}\ \ #1}
\end{center}}

\def\KWE#1{
\begin{center}\small\nwt
\parbox{150mm}{%\baselineskip=2.5ex
\textbf{Keywords:}\ \ #1}
\end{center}}


\def\KWN#1{
%\begin{center}
%\small
%\parbox{150mm}\end{center}
}

\newcommand{\Avtors}[1]{%\smallskip
%\vspace*{.5pt}
\hangindent=23pt\noindent
%\nwt
{\bfseries#1}\
}


\renewcommand{\thesubsection}{\thesection.\arabic{subsection}\hspace*{-5pt}}
\renewcommand{\thesubsubsection}{\thesubsection\hspace*{5pt}.\arabic{subsubsection}\hspace*{-3pt}}

\newcommand{\Ack}{\section*{\protect\rmfamily Acknowledgments}\noindent}
\newcommand{\Contr}{\section*{\protect\rmfamily Contributors}\noindent}
\newcommand{\Contrl}{\section*{\protect\rmfamily Contributor}\noindent}

\makeindex


\begin{document}
\Rus

\nwt
%\ptb


%\renewcommand{\contentsname}{\protect\Large\bf Содержание}

\setcounter{tocdepth}{2}

%\tableofcontents

\renewcommand{\bibname}{\protect\rmfamily Литература}
  \def\Au#1{{\it #1}}
    \def\Aue#1{{#1}}

%\newcommand{\No}{№}
  \newcommand{\tg}{\,\mathrm{tg}\,}
    \newcommand{\ctg}{\,\mathrm{ctg}\,}
  \newcommand{\arctg}{\,\mathrm{arctg}\,}

\def\forallb{\mathop{\forall}}
\def\cupb{\mathop{\cup}}
\def\existsb{\mathop{\exists}}


\newpage
\addtocounter{razdel}{1}
%\def\razd{РЕГУЛИРУЕМЫЙ ЭЛЕКТРОПРИВОД ДЛЯ ЭЛЕКТРОЭНЕРГЕТИКИ}


\setcounter{page}{2}

%   { %\Large  
   { %\baselineskip=16.6pt
   
   \vspace*{-48pt}
   \begin{center}\LARGE
   \textit{Предисловие}
   \end{center}
   
   %\vspace*{2.5mm}
   
   \vspace*{25mm}
   
   \thispagestyle{empty}
   
   { %\small 

    
Вниманию читателей журнала <<Информатика и её применения>> предлагается 
очередной тематический выпуск <<Вероятностно-статистические методы и 
задачи информатики и информационных технологий>>. Предыдущие тематические 
выпуски журнала по данному направлению вышли в 2008~г.\ (т.~2, вып.~2), 
в 2009~г.\ (т.~3, вып.~3) и в 2010~г.\ (т.~4, вып.~2). 

Статьи, собранные в данном журнале, посвящены разработке новых вероятностно-статистических 
методов, ориентированных на применение к решению конкретных задач информатики и информационных 
технологий, а также~--- в ряде случаев~--- и других прикладных задач. Проблематика, охватываемая 
публикуемыми работами, развивается в рамках научного сотрудничества между Институтом проблем 
информатики Российской академии наук (ИПИ РАН) и Факультетом вычислительной математики и 
кибернетики Московского государственного университета им.\ М.\,В.~Ломоносова в ходе работ 
над совместными научными проектами (в том числе в рамках функционирования 
Научно-образовательного центра <<Вероятностно-статистические методы анализа рисков>>). 
Многие из авторов статей, включенных в данный номер журнала, являются активными участниками 
традиционного международного семинара по проблемам устойчивости стохастических моделей, 
руководимого В.\,М.~Золотаревым и В.\,Ю.~Королевым; регулярные сессии этого семинара 
проводятся под эгидой МГУ и ИПИ РАН (в 2011~г.\ указанный семинар проводится в октябре 
в Калининградской области РФ). 

Наряду с представителями ИПИ РАН и МГУ в число авторов данного выпуска журнала входят 
ученые из Научно-исследовательского института системных исследований РАН, Института 
проблем технологии микроэлектроники и особочистых материалов РАН, Института 
прикладных математических исследований Карельского НЦ РАН, Московского 
авиационного института, Вологодского государственного педагогического университета, 
НИИММ им.\ Н.\,Г.~Чеботарева, Казанского государственного университета, Дебреценского 
университета (Венгрия).

Несколько статей выпуска посвящено разработке и применению стохастических методов и 
информационных технологий для решения различных прикладных задач. В~работе В.\,Г.~Ушакова 
и О.\,В.~Шестакова рассмотрена задача определения вероятностных характеристик случайных 
функций по распределениям интегральных преобразований, возникающих в задачах эмиссионной 
томографии. В~статье Д.\,О.~Яковенко и М.\,А.~Целищева рассмотрены некоторые вопросы 
математической теории риска и предложен новый подход к диверсификации инвестиционных 
портфелей. Работа И.\,А.~Кудрявцевой и А.\,В.~Пантелеева посвящена построению и 
исследованию математической модели, описывающей динамику сильноионизованной плазмы. 
В~статье П.\,П.~Кольцова изучается качество работы ряда алгоритмов сегментации изображений. 
Статья А.\,Н.~Чупрунова и И.~Фазекаша посвящена вероятностному анализу числа без\-оши\-бочных 
блоков при помехоустойчивом кодировании; получены усиленные законы больших чисел для указанных 
величин.

В данном выпуске традиционно присутствует тематика, весьма активно разрабатываемая в течение 
многих лет специалистами ИПИ РАН и МГУ,~--- методы моделирования и управления для 
информационно-телекоммуникационных и вычислительных систем, в частности методы 
теории массового обслуживания. В~статье А.\,И.~Зейфмана с соавторами рассматриваются 
модели обслуживания, описываемые марковскими цепями с непрерывным временем в случае 
наличия катастроф. В~работе М.\,М.~Лери и И.\,А.~Чеплюковой рассматриваются случайные 
графы Интернет-типа, т.\,е.\ графы, степени вершин которых имеют степенные распределения; 
такие задачи находят применение при исследовании глобальных сетей передачи данных. 
Работа Р.\,В.~Разумчика посвящена исследованию систем массового обслуживания специального 
вида~--- с отрицательными заявками и хранением вытесненных заявок.

Ряд статей посвящен развитию перспективных теоретических 
вероятностно-статистических методов, которые находят широкое применение в различных 
задачах информатики и информационных технологий. В~работе В.\,Е.~Бенинга, А.\,К.~Горшенина 
и В.\,Ю.~Королева рассмотрена задача статистической проверки гипотез о числе компонент 
смеси вероятностных распределений, приводится конструкция асимптотически наиболее мощного 
критерия. Результаты этой работы найдут применение в ряде прикладных задач, использующих 
математическую модель смеси вероятностных распределений (в информатике, моделировании 
финансовых рынков, физике турбулентной плазмы и~т.\,д.). В~статье В.\,Ю.~Королева, 
И.\,Г.~Шевцовой и С.\,Я.~Шоргина строится новая, улучшенная оценка точности нормальной 
аппроксимации для пуассоновских случайных сумм; как известно, указанные случайные суммы 
широко используются в качестве моделей многих реальных объектов, в том числе в информатике, 
физике и других прикладных областях. Работа В.\,Г.~Ушакова и Н.\,Г.~Ушакова посвящена 
исследованию ядерной оценки плотности распределения; эти результаты могут применяться, 
в част\-ности, при анализе трафика в телекоммуникационных системах. Серьезные приложения 
в статистике могут получить результаты работы О.\,В.~Шестакова, в которой доказаны оценки 
скорости сходимости распределения выборочного абсолютного медианного отклонения к нормальному 
закону. 

\smallskip

Редакционная коллегия журнала выражает надежду, что данный тематический  выпуск 
будет интересен специалистам в области теории вероятностей и математической статистики 
и их применения к решению задач информатики и информационных технологий.
     
     %\vfill 
     \vspace*{20mm}
     \noindent
     Заместитель главного редактора журнала <<Информатика и её 
применения>>,\\
     директор ИПИ РАН, академик  \hfill
     \textit{И.\,А.~Соколов}\\
     
     \noindent
     Редактор-составитель тематического выпуска,\\
     профессор кафедры математической статистики факультета\\
      вычислительной математики и кибернетики МГУ им.\ М.\,В.~Ломоносова,\\
     ведущий научный сотрудник ИПИ РАН,\\ 
доктор физико-математических наук \hfill
      \textit{В.\,Ю.~Королев}
     
     } }
     }


\def\stat{vasiliev}

\def\tit{О ФУНКТОРНОМ ПРЕДСТАВЛЕНИИ ОПТИМИЗИРУЕМЫХ ДИНАМИЧЕСКИХ 
МУЛЬТИАГЕНТНЫХ СИСТЕМ}

\def\titkol{О функторном представлении оптимизируемых динамических 
мультиагентных систем}

\def\aut{Н.\,С.~Васильев$^1$}

\def\autkol{Н.\,С.~Васильев}

\titel{\tit}{\aut}{\autkol}{\titkol}

\index{Васильев Н.\,С.}
\index{Vasilyev N.\,S.}


%{\renewcommand{\thefootnote}{\fnsymbol{footnote}} \footnotetext[1]
%{Исследование выполнено за счет гранта Российского научного фонда №\,22-28-00588, {\sf 
%https://rscf.ru/project/22-28-00588/}. Работа проводилась с~использованием инфраструктуры Центра 
%коллективного пользования <<Высокопроизводительные вычисления и~большие данные>> (ЦКП 
%<<Информатика>> ФИЦ ИУ РАН, Москва).}}


\renewcommand{\thefootnote}{\arabic{footnote}}
\footnotetext[1]{Московский государственный технический университет имени Н.\,Э.~Баумана, \mbox{nik8519@yandex.ru}}

\vspace*{2pt}






    \Abst{Топос функторов выбран в~качестве компьютерного инструмента синтеза 
динамических игр многих лиц. Задаваемая шкала упорядочивает объекты, 
отвечающие сопутствующим статическим подыграм. Последние служат 
состояниями динамической мультиагентной сис\-те\-мы (ДМАС). Исходная 
динамическая игра и~все статические подзадачи пред\-став\-ля\-ют\-ся в~моноидальной 
категории бинарных отношений. Под рациональным решением игры понимается 
равновесие. Композициональное строение оп\-ти\-ми\-зи\-ру\-емой ДМАС выражено 
в~форме динамического результирующего отношения (ДРО) игры. Поиску равновесия 
отвечает максимизация ДРО. Это делается методом Беллмана, обобщенным на 
задачи оптимального управления, поставленные в~форме отношений. Программная 
реализация предложенного подхода может быть основана на нейросетевых 
вычислениях ввиду согласованности архитектур применяемых графов отношений и~нейросетей.}
    
    \KW{категория функторов; композициональность; моноидальная категория; 
обратный образ; динамическое отношение игры; статическая подыгра; отношение 
предпочтения; динамическое результирующее отношение; рациональное решение; 
морфизм Беллмана}

\DOI{10.14357/19922264240201}{CLMBXC}
  
%\vspace*{-6pt}


\vskip 10pt plus 9pt minus 6pt

\thispagestyle{headings}

\begin{multicols}{2}

\label{st\stat}
    
\section{Введение}

    В ДМАС агенты принимают 
решения в~каждом состоянии системы в~за\-ви\-си\-мости от рас\-по\-ла\-га\-емой ими 
информации о~поведении других участников конфликта~[1--4]. Воздействие 
стратегий на ДМАС распределено во времени и~связано с~выбором ситуаций 
в~череде со\-пут\-ст\-ву\-ющих статических игровых подзадач. 
    
    Процесс изменения состояний ДМАС традиционно задают с~по\-мощью 
дифференциальных или итерационных уравнений, в~которые входят управ\-ля\-ющие 
воздействия игроков. От этого описания динамики приходится отходить даже 
в~антагонистических дифференциальных играх с~полной ин\-фор\-ми\-ро\-ван\-ностью 
игроков. Управляемую систему представляют дифференциальным уравнением 
в~контингенциях~\cite{3-vas}, т.\,е.\ применяют многозначные отоб\-ра\-же\-ния. 
В~стохастических постановках так\-же возникает дополнительная не\-опре\-де\-лен\-ность, 
связанная с~прогнозированием будущих со\-сто\-яний сис\-те\-мы~\cite{4-vas}. 
    
    Зачастую целесообразно отказаться от функционального описания динамики 
игры и~перейти на\linebreak язык отношений~[5]. Благодаря этому значительно расширяется 
круг приложений~\cite{4-vas, 6-vas, 7-vas, 8-vas, 9-vas}, приобретается свойство 
композициональности моделей ДМАС, удобное для модификации игровых \mbox{задач}. 
Этот подход поддерживается компьютерной алгеброй тео\-рии категорий~\cite{10-vas, 11-vas}. 
Кроме того, графовая структура отношений и~сетевая структура игровой задачи 
допускают эффективную нейросетевую программную реализацию~\cite{6-vas, 7-vas, 12-vas, 13-vas}. 
    
    Категорный подход охватывает общий случай динамических игр многих лиц 
    с~разнообразными классами допустимых стратегий игроков~\cite{5-vas, 11-vas}. 
В~качестве со\-сто\-яний динамической системы рассматриваются со\-пут\-ст\-ву\-ющие 
статические игры, связанные динамическим отношением.
    
    Традиционное моделирование игровой задачи проводится средствами 
категории множеств SET. Естественным обобщением классического подхода 
становится формализация игровой операции, вы\-пол\-ня\-емая на языке моноидальной 
категории бинарных отношений REL. Применяемый аппарат позволяет 
единообразно выражать и~модифицировать интересы агентов, проводить 
эквивалентные преобразования игровой задачи, описывать поиск рационального 
решения~\cite{5-vas}. С~по\-мощью введения ДРО игры исследование ДМАС может быть сведено 
к~последовательной максимизации ДРО.
    
    Итак, вместо многошагового процесса игры предлагается перейти 
    к~представлению динамики задачи с~по\-мощью функтора $\tau: S\hm\to \mathrm{SET}_T$, 
который преобразует вы\-би\-ра\-емую шкалу~$S$ в~категорию $\mathrm{SET}_T$ монады 
$(T,\eta,\psi)$, $T: A\hm\to 2^A$~\cite{11-vas}. Объектами монады служат конечные 
множества, а~морфизмами~--- отношения между ними. Равенство $\mathrm{SET}_T\hm=\mathrm{REL}$ 
означает, что функторная модель ДМАС строится на языке отношений. 
Единообразно выражаются правила игры, интересы участников, классы стратегий 
агентов, динамика задачи и~даже алгоритм ее решения. Строгость функтора 
вложения $\mathrm{SET}\hm\to \mathrm{SET}_T$ обеспечивает пре\-емст\-вен\-ность новой формы 
пред\-став\-ле\-ния задачи и~классической.
    
    Функтор $\tau$ порождает динамическое отношение игры $R\hm= R(\tau)$, 
регламентирующее смену со\-сто\-яний~--- ре\-ша\-емых в~текущий момент времени 
статических подыгр $\Gamma(X)$ с~множеством \mbox{до\-пус\-ти\-мых} ситуаций~$X$. 
Динамическое отношение определяет по\-сле\-до\-ва\-тель\-ность смены со\-сто\-яний 
$\Gamma(X_0), \ldots , \Gamma(X_T)$ и~вводит отношение предшествования подыгр 
$\Gamma(X)\overset{R}{\prec} \Gamma(Y)$. В~соответствии с~ним определена 
преемственность ситуаций $x\overset{R}{\prec} y$, причем выбор любой пары 
ситуаций $x\hm\in X$ и~$y\hm\in Y$, относящихся к~разным состояниям системы, 
возможен, только если выполнено включение $(x,y)\hm\in R$. Таким образом, из 
решений подзадач $\Gamma(X_0), \ldots , \Gamma(X_T)$ строится 
по\-сле\-до\-ва\-тель\-ность $x_0, \ldots , x_T$ вы\-би\-ра\-емых ситуаций подобно тому, как это 
делается в~многошаговых играх с~по\-мощью уравнений. 
    
    Интересы агентов в~задаче заданы в~форме отношений предпочтения для 
терминального со\-сто\-яния сис\-те\-мы~$\Gamma(X^T)$: 
    \begin{equation}
    \tilde{\rho}^{kT} \in \mathrm{REL} \left( X^T, X^T\right)\,,\enskip k\in K\,.
    \label{e1.1-vas}
    \end{equation}
    
    Правила функционирования ДМАС формулируются во всех подыграх 
$\Gamma(X_i)$. Во-пер\-вых, определены цели игроков~--- максимизация 
отношений предпочтения $\rho_i^k\hm\in \mathrm{REL}\left( X_i, X_i\right)$, $k\hm\in 
K(X_i)$, по\-лу\-ча\-емых из заданных~(\ref{e1.1-vas}) с~по\-мощью вы\-чис\-ле\-ния 
обратных образов морфизмов~$\tilde{\rho}^{kT}$ относительно динамического 
отношения~$R$. Во-вто\-рых, агенты выбирают классы допустимых стратегий 
посредством графа коммуникаций и~функции, размечающей его дуги. Тем самым 
задается сетевая структура в~статических подыграх~\cite{12-vas, 13-vas}. Разметка 
графа детализирует информацию, которой обмениваются партнеры при совершении 
ходов и~формировании коалиций~\cite{5-vas}. Стратегии агентов в~исходной 
динамической задаче включают процедуру принятия решений в~каж\-дом состоянии 
сис\-темы. 
    
    Далее изучено композициональное стро\-ение динамического результирующего 
отношения игры, поз\-во\-ля\-ющее искать рациональное решение, т.\,е.\ такое, которое 
оптимизирует функционирование ДМАС. Поиск проводится методом 
динамического программирования, обоб\-щен\-ным на задачи оптимального 
управ\-ле\-ния, по\-став\-лен\-ные в~форме отношений. 

\section{Постановка задачи}

    В качестве шкалы~$S$ выберем категорию 
    \begin{equation}
 S: L \overset{\underset{\mathrm{out}}{\longrightarrow}}{\underset{\overset{\mathrm{in}}{\longrightarrow}}
 {\vphantom{{\mbox{\scriptsize{$\circ$}}}} \ }} M\,.
    \label{e2.1-vas}
    \end{equation}
    
    Динамику мультиагентной системы зададим функтором $\tau: S\hm\to \mathrm{SET}_T$, 
которому отвечает ориентированный граф динамического отношения~$\Gamma_R$, 
$R\hm=R(\tau)$. Граф должен быть ацик\-ли\-че\-ским и~связ\-ным. В~его вершинах 
находятся множества до\-пус\-ти\-мых ситуаций статических подыгр~$\Gamma(X_i)$, 
а~дуги отвечают отношениям $R_{ij}\hm\in \mathrm{REL}\left( X_i, X_j\right)$. 
    
    \smallskip
    
    \noindent
    \textbf{Определение~2.1.}\  Скажем, что ситуация $x_i\hm\in X_i$ 
непосредственно предшествует ситуации $x_j\hm\in X_j$, если и~только если $(x_i, 
x_j)\hm\in R_{ij}$. Отношением предшествования со\-сто\-яний назовем транзитивное 
замыкание $\overline{R}\hm= \overline{R}_{ij}$ динамического отношения. 
    
    \smallskip
    
    Последовательная смена состояний ДМАС происходит в~соответствии 
с~отношением непосредственного предшествования $\Gamma(X_i) 
\overset{R_{ij}}{\prec} \Gamma(X_j)$. Каж\-до\-му со\-сто\-янию соответствует некоторая 
\mbox{статическая} подыгра. Процесс функционирования сис\-те\-мы складывается из 
последовательного выбора игроками ситуаций $x_0, \ldots , x_T$, где  
$x_0$ и~$x_T$~--- решения начальной и~терминальной подыгр. Итак, имеем 
множество всех допустимых ситуаций исходной динамической задачи вида
 \begin{multline*}
    \tilde{X} ={}\\
    {}=\!\left\{\! \tilde{x} =\left( x_0, \ldots , x_T\right): \left( \forall_i x_i\in 
X_i\right) \wedge x_0\overset{R}{\prec} \cdots \overset{R}{\prec} x_T\!\right\}\!.\hspace*{-5.73415pt}
    \end{multline*}
    
    Правила проведения каждой подыгры $\Gamma(X_i)$ задаются сле\-ду\-ющим 
функтором и~функ\-цией: 
    \begin{equation}
    g_i: S\to \mathrm{SET}\,;\enskip h_i: E\to \overline{2}\,.
    \label{e2.2-vas}
    \end{equation}
Игроки стремятся по возможности максимизировать свои отношения предпочтения 
$\rho_l^k: X_l\hm\to X_l$,\linebreak $k\hm\in K_l$. (Иначе рас\-смат\-ри\-ва\-ют\-ся противоположные 
отношения $(\rho_i^k)^{\mathrm{op}}$.) Функторы~(\ref{e2.1-vas}) и~(\ref{e2.2-vas}) 
опре\-де\-ля\-ют сетевую структуру игры. Шкала~$S$ выделяет множество дуг~$E$, по 
которым участники\linebreak $k\hm\in K(X_i)$ статической игры осуществляют 
коммуникацию. Функциями in и~out в~(\ref{e2.1-vas}) вводят порядок ходов. 
Функция~$h_i$ из формулы~(\ref{e2.2-vas}) осуществляет разметку дуг графа 
коммуникаций~$\gamma_i$, определяя данные, которыми обмениваются игроки при 
своих ходах~\cite{5-vas}. Не\-пус\-тые сообщения могут содержать либо выбранную 
стра\-те\-гию-конс\-тан\-ту, либо стра\-те\-гию-функ\-цию~\cite{1-vas, 5-vas}. 
Предполагается так\-же, что никто из агентов не блефует. 
    
    Понятие рационального поведения игроков зависит от сетевой 
структуры~(\ref{e2.2-vas}) статических игр~\cite{5-vas, 12-vas, 13-vas}. Будем 
применять принцип равновесия как к~исходной динамической задаче, так и~ко всем 
по отдельности подыграм~$\Gamma(X_i)$.
    
    \smallskip
    
    \noindent
    \textbf{Пример~2.1.}\ Рассмотрим функтор~(\ref{e2.1-vas}), которому отвечает 
граф
    \begin{equation}
    \Gamma_R: \mbox{\fbox{$X_1$}} \xrightarrow{R_{12}} \mbox{\fbox{$X_2$}} 
\xrightarrow{R_{23}} \cdots \xrightarrow{R_{n-1,n}} \mbox{\fbox{$X_n$}}\,.
    \label{e2.3-vas}
    \end{equation}
    
    Функции out и~in определяют начало $X_i\hm\in M$ и~конец $X_j\hm\in M$ 
каж\-дой дуги, обозначенной стрелкой $R_{i,i+1}\hm\in L$. В~моменты времени 
$i\hm= 1,2,\ldots , n$ решаются игры~$\Gamma(X_i)$.
    
    Схема~(\ref{e2.3-vas}) свойственна многошаговым опе\-ра\-циям.
    
    \smallskip
    
    \noindent
    \textbf{Пример~2.2.}\ Пусть в~позиционной многошаговой игре двух лиц 
известна сле\-ду\-ющая динамическая сис\-те\-ма, начальное условие и~фазовое 
ограничение:
    \begin{gather*}
    \left\{
    \begin{array}{c}
    y_{t+1}^1 =y_t^1+u_t^1\,;\\[6pt]
    y_{t+1}^2 = y_t^2-u_t^2\,;
    \end{array}
    \right.\\[2pt]
    y_0^1 =0\,,\enskip y_0^2 =0\,;\\[2pt]
    \left\vert y_t^1 -y_t^2\right\vert \leq 2\,,\enskip t=\overline{0, T}\,.
    \end{gather*}
Агенты $k=1, 2$ стремятся по возможности максимизировать функции выигрыша
\begin{equation*}
J_1=-\sum\limits_{t=1}^{T-1} y_t u_t^1;\quad J_2= \sum\limits_{t=1}^{T-1} y_t u_t^2,
\end{equation*}
где
$$
y_t=y_t^1 - y_t^2,\quad u_t^k \in U_0\equiv \{ -1, 0, 1\}.
$$
    
    Интересы игроков $\tilde{\rho}^k \hm\in \mathrm{REL}( \tilde{X}, 
\tilde{X})$~(\ref{e1.1-vas}) заданы функционалами~$J_k$. В~статических 
подыг\-рах~$\Gamma(X_t)$, $X_t\hm= \cup y_t X(y_t)$, $X_0\hm= U_0^2$, с~по\-мощью 
функций $f^1\hm= -y_t u_t^1$, $f^2\hm= y_t u_t^2$ введем отношения предпочтения 
игроков~$\rho_t^k$. Видно, что интересы участников подыгр противоположны 
$\rho_t^2\hm= \rho_t^{1\mathrm{op}}$, а~множества допустимых ситуаций расслоены 
в~за\-ви\-си\-мости от позиций $y_t\hm= y_t^1\hm- y_t^2$, $\vert y_t\vert \hm\leq 2$, 
в~которых может пребывать динамическая сис\-те\-ма. Ситуация $(u_t^1, u_t^2)\hm\in 
X(y_t)$ допустима, только если ее выбор не нарушает заданное фазовое 
ограничение. 
    
    Во все моменты времени отношения предпочтения $\rho_t^k(y_t): X(y_t) \hm\to 
X(y_t)$, $k\hm=1,2$,  одинаковы: $\rho_t^k \hm= \rho_1^k$, $X_t\hm= X_1$, $t\hm= 
\overline{1,T-1}$, причем
    $$
    \rho_t^k\triangleq \bigcup\limits_{y_t} \rho_t^k (y_t).
    $$

    
    Вычисления показывают, что при $t\hm= \overline{1, T-1}$ имеем
    
    \vspace*{-8pt}
    
    \noindent
    \begin{multline*}
    X_t=\coprod\limits^2_{y_t=-2} X(y_t),\ X(\pm 2) = \{ (\pm 1,\pm1)\},\\
     X(\pm1) = \{ (\pm1,0), (0,\pm1)\},\\
      X(0)=\{ (1,-1), (0,0), (-1,1)\}\,;
    \end{multline*}
    
    \vspace*{-13pt}
    
    \noindent
    \begin{multline*}
    \rho_t^1(0): (0,-1)\sim (-1,0) \sim (0,0),\\
     \rho_t^1(1)=\{ ((0,-1), (-1,0))\},\\
    \rho^1_t(-1) =\{ ((-1,0), (0,-1))\}\,,\\ 
    \rho_1^1(\pm2) =\{ (\pm 1, \pm1)\}.
    \end{multline*}
    
    \vspace*{-4pt}
    
    Динамическое отношение игры, равное $R_{t,t+1}: X_t\hm\to X_{t+1}$, $t\hm= 
\overline{1, T-1}$, связывает до\-пус\-ти\-мую текущую $u_m\hm\equiv (u_m^1, u_m^2) 
\hm\in X(y_m)$ и~выбираемую сле\-ду\-ющую $u_{m+1}\hm\in X(y_{m+1})$ ситуации 
в~состояниях сис\-те\-мы $\Gamma(X_m)$, $m\hm= t, t\hm+1$. Таким образом, 
отношение непосредственного пред\-шест\-во\-ва\-ния ситуаций определено включением
    $$
    u_t \xrightarrow{R_{tt+1}} u_{t+1},\ u_{t+1} \in X(y_{t+1}),\ y_{t+1} \!=\! y_t+\left( 
u_t^1+ u_t^2\right).
    $$
    
    \vspace*{-2pt}
    
    Пусть сетевая структура~(\ref{e2.2-vas}) такова, что во всех играх 
$\Gamma(X_t)$ один из участников сообщает другому свое управ\-ля\-ющее 
воздействие~$u_t^k$. Тогда рациональное решение игры состоит в~выборе 
ситуации, ста\-би\-ли\-зи\-ру\-ющей траекторию сис\-те\-мы $y_t\hm\equiv 0$, а~многошаговая 
игра имеет седловые точки  $\forall_t u_t^1 \hm= -u_t^2$.
    
    При поиске рационального решения игровых задач применяются 
вспомогательные морфизмы, стро\-ящи\-еся из исходных~(\ref{e1.1-vas}) с~по\-мощью 
операций алгебры отношений $A\hm= \left( \circ, \cup, \cap, {}^{\mathrm{op}}, \times; 
\sigma,\varnothing \right)$~\cite{5-vas, 10-vas, 11-vas}. Получение ка\-ким-ли\-бо 
игроком дополнительной информации уменьшает неопределенность выбора 
подходящей стратегии (см.\ пример~2.2). Выбор сетевой структуры  
игры~(\ref{e2.2-vas}) изменяет отношения предпочтений участников~$\rho$. Игроки 
руководствуются сужениями $\rho\vert_A\hm= \rho \cap A^2$, $A\hm\subset X$, 
исходных отношений~$\rho$ на некоторые, вполне определенные подмножества 
ситуаций. 
    
    При условии сделанных предположений граф динамического отношения 
содержит минимальные и~максимальные элементы~--- вершины $X^0$ и~$X^T$ 
(см.~пример~2.1). Процесс функционирования ДМАС начинается с~решения 
статических \mbox{подыгр}~$\Gamma(X^0)$ и~заканчивается подзадачами~$\Gamma(X^T)$. 
Они названы начальным и~терминальным со\-сто\-яни\-ями сис\-те\-мы соответственно. 
В~отличие от примера~2.2, в~рас\-смат\-ри\-ва\-емой по\-ста\-нов\-ке задаются терминальные 
отношения предпочтения агентов~(\ref{e1.1-vas}).

     \begin{figure*}[b] %fig1
\vspace*{1pt}
\begin{minipage}[t]{79mm}
\begin{center}  
    \mbox{%
\epsfxsize=17.323mm 
\epsfbox{vas-1.eps}
}
\end{center}
\end{minipage}
\hfill
\vspace*{1pt}
\begin{minipage}[t]{79mm}
\begin{center}
     \mbox{%
\epsfxsize=25.462mm 
\epsfbox{vas-2.eps}
}
\end{center}
\end{minipage}
\begin{minipage}[t]{79mm}
\vspace*{-12pt}
\Caption{Образ $r=R\circ r^\prime$ и~обратный образ $r^\prime \hm= R^{\mathrm{op}}\circ r$ 
относительно функтора~$R$}
\end{minipage}
%\end{figure*}
\hfill
% \begin{figure*} %fig3.2
      \begin{minipage}[t]{79mm}
      \vspace*{-12pt}
           \Caption{Граф $\Gamma_R$ динамического отношения~$R$: $R_{ij}\hm\in \mathrm{REL}\,(X_i, X_j)$ }
     \label{f3.2-vas}
     \end{minipage}
%     \vspace*{-24pt}
     \end{figure*} 

    
    Рациональное поведение игроков заключается в~том, что они стремятся 
максимизировать свои 
 предпочтения на множестве всех допустимых ситуаций:
    \begin{equation}
    \forall_k \tilde{\rho}^{kT} \to \mathop{\mathrm{MAX}}\limits_{\tilde{X}}\,.
    \label{e2.4-vas}
    \end{equation}
В~задаче~(\ref{e1.1-vas}), (\ref{e2.1-vas}), (\ref{e2.2-vas}), (\ref{e2.4-vas}) требуется 
найти ситуацию равновесия~\cite{1-vas, 5-vas}. 
    
    Существование равновесия во многом зависит от свойств 
морфизмов~$\rho_i^k$, $k\hm\in K$. Наложим одно из таких требований. Пусть 
партнеры предлагают некоторому игроку~$k$ принять решение в~ситуации~$s$, для 
которой $\rho_i^k\vert_{s_k} \hm= \varnothing$, $s\hm\in X_i$. Предполагается, что 
возникшую неопределенность выбора каж\-дый игрок $k\hm \in K$ разрешит  
с~по\-мощью изменения своего отношения предпочтения, положив 
$\rho_i^k\vert_{s_k} \hm= \{ s\}$. (Игрок соглашается с~выбором ситуации~$s$.) Это 
требование выполняется для рефлексивных отношений.
    
\section{Эквивалентные преобразования динамических~игр}

    Категорное представление допускает применение моноидальных операций для 
преобразования игровой задачи~\cite{11-vas}. Морфизмы $R: \mathrm{REL}\left(Y\right)\hm\to 
\mathrm{REL}\,(X)$ представляют собой функторы, срав\-ни\-ва\-ющие однообъектные 
подкатегории в~категории REL. Образ морфизма $\rho\hm\in \mathrm{REL}\,(Y,Y)$ 
относительно~$R$ будем записывать как композицию $R\circ \rho \hm\in \mathrm{REL}\,(X,X)$.
    
    \smallskip
    
    \noindent
    \textbf{Определение~3.1.} Обратным образом (или кообразом) отношения $r: 
X\hm\to X$ относительно морфизма $R: Y\hm\to X$ назовем стрелку $r^\prime: 
Y\hm\to Y$, превращающую сле\-ду\-ющий квад\-рат в~коммутативный  
(рис.~1).
    
    
     
     
    Ввиду того что морфизм $R^{\mathrm{op}} \hm\in \mathrm{REL}\,(X,Y)$ сохраняет композиции 
стрелок, он сам становится функтором $R^{\mathrm{op}}: \mathrm{REL}\,(X)\hm\to \mathrm{REL}\,(Y)$. Так как 
$\forall_{ij} (x,y) \hm\in R_{ij}^{\mathrm{op}} \hm\Leftrightarrow  (y,x)\hm\in R_{ij}$, то 
назовем~$R^{\mathrm{op}}$ противоположным к~$\{ R_{ij}\}$ динамическим отношением. 
Его можно изобразить, изменив на\-прав\-ле\-ния всех дуг в~графе~$\Gamma_R$ 
(вертикальных стрелок на рис.~1). 




    
    Пусть вершины $X_1$ и~$X_l$ графа динамического отношения~$\Gamma_R$ 
связаны некоторым путем $L\hm= (X_1, \ldots, X_l)$. Взяв композицию $R_L\hm= 
R_{l-1}\circ\cdots\linebreak \cdots \circ R_1$ морфизмов $R_i\hm \in \mathrm{REL}\,(X_i, X_{i+1})$ вдоль цепи
$L\hm= (X_1, \ldots , X_l)$, мож\-но <<опустить>> произвольное бинарное отношение 
$r_l: X_l\hm\to X_l$ с~$X_l$ на множество~$X_1$ и~по\-стро\-ить его обратный образ 
$r_1\hm= R_L^{\mathrm{op}} \circ r_l$.
    
    Пусть теперь множество $\{ L\hm= (X_1, \ldots , X_l)\}$ всех попарно 
различных путей, свя\-зы\-ва\-ющих вершины $X_1$ и~$X_l$, состоит более чем из одного 
элемента. Под обратным образом отношения $\rho_l: X_l\hm\to X_l$, переносимого 
на множество~$X_1$, будем понимать копроизведение
    \begin{multline}
    \rho_1^\prime =R^{\mathrm{op}}\circ \rho_l =\coprod\limits_{\{L\}} R_L^{\mathrm{op}} \circ 
\rho_l;\\
 \rho_1^\prime: \coprod\limits_{\{ L\}} X_1\to \coprod\limits_{\{L\}} X_1\,.
    \label{e3.1-vas}
    \end{multline}
    
    <<Поднятием>> морфизма~$\rho_1$ вдоль путей $\{L\}$ мож\-но по\-стро\-ить 
образ $\rho_L\hm= R\circ \rho_1$. Это отношение $\coprod_{\{L\}} R_L\circ \rho_1$ на 
копроизведении объектов $\coprod_{\{L\}} X_l$ с~числом сомножителей, рав\-ным 
мощ\-ности набора~$\{L\}$.
    
    По формуле~(\ref{e3.1-vas}) заданные в~терминальных со\-сто\-яни\-ях сис\-те\-мы 
морфизмы~(\ref{e1.1-vas}) переносятся во все остальные со\-сто\-яния. Так вводятся 
отношения предпочтения игроков~$\rho_i^k$ в~играх~$\Gamma(X_i)$:
    \begin{equation*}
    \forall_{ik} \rho_i^k =R^{\mathrm{op}}\circ \tilde{\rho}_i^{kT}.
%    \label{e3.2-vas}
    \end{equation*}
    
    \noindent
    \textbf{Замечание~3.1.}\ Общее определение кообраза~(\ref{e3.1-vas}) мож\-но 
заменить двойственной конструкцией, основанной на произведении 
отношений~\cite{11-vas}: 
    \begin{equation}
    \rho^{\prime\prime} \!=\!R^{\mathrm{op}} \circ \rho \!=\!\prod\limits_{\{L\}} R^{\mathrm{op}}_L \circ \rho\,;\ \rho\! =\!R\circ \rho^{\prime\prime}= 
\prod\limits_{\{L\}} R_L\circ \rho^{\prime\prime}\!.\!
    \label{e3.3-vas}
    \end{equation}
Получим морфизмы~(\ref{e3.3-vas}) вида 
$$
\rho: \prod\limits_{\{L\}} X_1\to \prod\limits_{\{L\}} X_1,\ \rho^{\prime\prime}: \prod\limits_{\{L\}} X_l\to \prod\limits _{\{L\}} X_l.
$$

    \smallskip
    
    \noindent
    \textbf{Пример~3.1.}\ Воспользуемся формулой~(\ref{e3.1-vas}) 
применительно к~ДМАС, пред\-став\-лен\-ной на рис.~2. 
    
    % \end{multicols}
     


     
  %   \begin{multicols}{2}
    
     
Кообразом $R^{\mathrm{op}}\circ \rho_4$ отношения $\rho_4: X_4\hm\to X_4$ служит морфизм 
\begin{multline*}
\rho_1^\prime: X_1\coprod X_1 \to X_1\coprod X_1;
\\
\rho_1^\prime =\left( R_{13}^{\mathrm{op}} \circ R_{34}^{\mathrm{op}}\coprod R_{12}^{\mathrm{op}} \circ 
R_{24}^{\mathrm{op}}\right)\circ \rho_4.
\end{multline*}
Опуская~$\rho_4$ на множества $X_2$ и~$X_3$, получим $\rho_2^\prime \hm= 
R_{24}^{\mathrm{op}} \circ \rho_4$, $\rho_3^\prime\hm= R_{34}^{\mathrm{op}} \circ \rho_4$ 
соответственно. Поднятием отношения $\rho_1^\prime: X_1\hm\to X_1$ на 
множество~$X_4$ строится образ  $R\circ \rho_1^\prime$ как
\begin{multline*}
\rho_4: X_4\coprod X_4 \to X_4\coprod X_4\,;\\
 \rho_4=\left( R_{24}\circ R_{12} \coprod 
R_{34} \circ R_{13}\right) \circ \rho_1^\prime.
\end{multline*}
    
    Конструкция~(\ref{e3.3-vas}) дает структуру
\begin{multline*}
    \rho_4: X_4\times X_4 \to X_4\times X_4\,;\\
     \rho_4=\left( R_{24}\circ R_{12} \prod 
R_{34} \circ R_{13}\right) \circ \rho_1^{\prime\prime}.
\end{multline*}
    
    Формулы~(\ref{e3.1-vas}) и~(\ref{e3.3-vas}) позволяют строить эквивалентные 
модели игры. Свойство универсальности объектов $X_2\times X_3$ и~$X_2\coprod 
X_3$, вы\-ра\-жа\-емое по\-средст\-вом коммутативных диаграмм~\cite{11-vas}, однозначно 
определяет новое динамическое отношение с~со\-от\-вет\-ст\-ву\-ющи\-ми морфизмами. 
Например, вместо графа из рис.~2 мож\-но работать с~более прос\-той 
графовой структурой, пред\-став\-лен\-ной в~любой из сле\-ду\-ющих форм:
    \begin{gather}
    \mbox{\fbox{$X_1$}} \xrightarrow{\left\langle R_{12}, R_{13}\right\rangle} 
\mbox{\fbox{$X_2\times X_3$}} \xrightarrow{\left\langle R_{24}^{\mathrm{op}}, 
R_{34}^{\mathrm{op}}\right\rangle^{\mathrm{op}}} \mbox{\fbox{$X_4$}}\,;
    \label{e3.4-vas}\\
    \mbox{\fbox{$X_1$}} \xrightarrow{[ R_{12}, R_{13}]^{\mathrm{op}}} 
\mbox{\fbox{$X_2\coprod X_3$}} \xrightarrow{[ R_{24}, R_{34}]} 
\mbox{\fbox{$X_4$}}\,.
    \label{e3.5-vas}
    \end{gather}
Возможность эквивалентного перехода от произвольного графа~$\Gamma_R$ (см.\ 
рис.~2) к~цепи~(\ref{e2.3-vas}) (см.\ (\ref{e3.4-vas}) и~(\ref{e3.5-vas})) 
обоснована в~тео\-ре\-ме~3.1. 
    
    \smallskip
    
    \noindent
    \textbf{Теорема~3.1.} \textit{Граф всякого динамического отношения 
приводится к~виду}~(\ref{e2.3-vas}).

\vspace*{-6pt}

\section{Результирующее отношение динамической игры}

    Введением результирующего отношения игра сводится к~проблеме 
оптимального управ\-ле\-ния, по\-став\-лен\-ной в~форме отношений. Для ее решения\linebreak 
мож\-но применить обобщенный метод динамического программирования. 
Напомним~\cite{5-vas}, что в~любой статической игре $\Gamma(Y)$ существует 
ре\-зуль\-ти\-ру\-ющее отношение~$P_Y$. Оно выражает \mbox{композициональное} свойство 
игры, учи\-ты\-ва\-ющее интересы всех участников операции, сетевую структуру их 
взаимодействия и~ра\-ци\-о\-наль\-ность поведения. Решение игры $\Gamma(Y)$ сводится
 к~поиску максимальных элементов $P_Y\hm\to \mathrm{MAX}_Y$. Благодаря\linebreak этому, мож\-но 
по-но\-во\-му определить состояния \mbox{исходной} ДМАС. Вместо игр $\Gamma(X_i)$ 
будем рас\-смат\-ри\-вать оптимизационные задачи $(\tilde{X}_i, P_{\tilde{X}_i})$, 
$P_{\tilde{X}_i} \hm\in \mathrm{REL}\,(\tilde{X}_i, \tilde{X}_i)$. Иначе говоря, теперь во всех 
со\-сто\-яни\-ях сис\-те\-мы решение принимает единственный агент. 
    
    %\smallskip
    
    \noindent
    \textbf{Определение~4.1.}\ Пусть $(\tilde{Y}_1, P_{\tilde{Y}_1} ) 
\overset{R}{\prec} (\tilde{Y}_2, P_{\tilde{Y}_2})$; $\tilde{y}_1, \tilde{y}_2 \hm\in 
\tilde{Y}_1$. Ситуация~$\tilde{y}_2$ называется более перспективной по сравнению 
с~$\tilde{y}_1$, если выполнено свойство
    \begin{equation}
    \left( \tilde{y}_1, \tilde{y}_2\right) \in \left( R^{\mathrm{op}}\circ P_{\tilde{Y}_2} \right)\circ 
P_{\tilde{Y}_1}.
    \label{e4.1-vas}
    \end{equation}
    
    В каждом состоянии ДМАС игрокам целесообразно использовать более 
перспективные ситуации и~из них формировать рациональное решение 
задачи~(\ref{e1.1-vas}), (\ref{e2.1-vas}), (\ref{e2.2-vas}), (\ref{e2.4-vas}). В~этом 
заключается прин\-цип Бел\-лмана. 
    
    В случае динамического отношения с~графом~(\ref{e2.3-vas}) (см.\ 
тео\-ре\-му~3.1) рас\-смот\-рим сле\-ду\-ющую итерационную схему по\-стро\-ения 
<<оп\-ти\-ми\-зи\-ру\-ющих>> морфизмов $\{ \tilde{P}_{\tilde{X}_k}\}$:

\vspace*{-4pt}

\noindent
    \begin{multline}
    \tilde{P}_{\tilde{X}_T} =P_{X_T},\ \tilde{P}_{\tilde{X}_{T-k}} =\left( 
R^{\mathrm{op}}\circ \tilde{P}_{\tilde{X}_{T-k+1}}\right) \circ P_{\tilde{X}_{T-k}},\\
    k=\overline{1, T}\,.
    \label{e4.2-vas}
    \end{multline}
    
    \vspace*{-4pt}

\noindent
Назовем~(\ref{e4.2-vas}) уравнениями Беллмана в~форме отношений. 
    
    \smallskip
    
    \noindent
    \textbf{Определение~4.2.} Динамическим ре\-зуль\-ти\-ру\-ющим отношением 
называется семейство морфизмов Беллмана $\{ \tilde{P}_{\tilde{X}_k},\ k\hm= 
\overline{1, T}\}$ из уравнений~(\ref{e4.2-vas}).
    
    \smallskip
    
    \noindent
    \textbf{Теорема~4.1.} \textit{Во всякой ДМАС существует ре\-зу\-ль\-ти\-ру\-ющее 
отношение.}
    
    \smallskip
    
    \noindent
    Д\,о\,к\,а\,з\,а\,т\,е\,л\,ь\,с\,т\,в\,о\,.\ \ Построение ДРО проведем методом 
математической индукции. На начальном шаге, в~терминальном со\-сто\-янии сис\-те\-мы, 
ДРО совпадает с~ре\-зуль\-ти\-ру\-ющим отношением $\tilde{P}_{X^T} \triangleq 
P_{X^T}$, $\tilde{X}^T\hm= X^T$, статической игры~$\Gamma(X^T)$. Опус\-тим 
этот морфизм на все множества $\tilde{X}^{T-1}$, для которых имеет мес\-то 
непосредственное предшествование $\Gamma(\tilde{X}^{T-1}) \overset{R}{\prec} 
\Gamma(\tilde{X}^T)$. Композиция морфизмов~(\ref{e4.1-vas}), $\tilde{Y}_1\hm= 
\tilde{X}^{T-1}$, $\tilde{Y}_2\hm= \tilde{X}^T$, определяет компоненту ДРО 
$\tilde{P}_{\tilde{X}^{T-1}}$, от\-ве\-ча\-ющую со\-сто\-янию сис\-те\-мы $(\tilde{X}^{T-1}, 
P_{\tilde{X}^{T-1}})$. Пусть отношение~$\tilde{P}_{\tilde{X}_i}$ уже построено. 
Тогда опять по формуле~(\ref{e4.1-vas}) его можно продолжить на все со\-сто\-яния 
$\Gamma(\tilde{X}_j) \overset{R_{ij}}{\prec} \Gamma(\tilde{X}_i)$ исходной ДМАС, 
т.\,е.\ построить очередные морфизмы~$\tilde{P}_{\tilde{X}_j}$. Процесс 
завершается в~начальных состояниях сис\-темы. 

\vspace*{-11pt}

\section{Поиск рационального решения игры}

\vspace*{-1pt}

    Для решения задачи~(\ref{e1.1-vas}), (\ref{e2.1-vas}), (\ref{e2.2-vas}),  
(\ref{e2.4-vas}) предложим сле\-ду\-ющее обобщение метода динамического\linebreak 
программирования. Сначала из системы уравнений Беллмана~(\ref{e4.2-vas}) 
найдем ДРО игры. Затем последовательно для моментов времени $1,\ldots, T$ 
вычислим сле\-ду\-ющие множества максимальных \mbox{элементов} отношения Беллмана:

\pagebreak



\noindent
    \begin{multline}
    X_1^*=\mathop{\mathrm{ARGMAX}}\,\tilde{P}_{\tilde{X}_1};\enskip 
    X_2^*=\mathop{\mathrm{ARGMAX}}\limits_{RX_1^*}\, \tilde{P}_{\tilde{X}_2}; \ldots  \\
\ldots ;    X_T^* = \mathop{\mathrm{ARGMAX}}\limits_{RX^*_{T-1}}\,\tilde{P}_{\tilde{X}_T}.
       \label{e5.1-vas}
    \end{multline}
    
\vspace*{-3pt}

    \noindent
    \textbf{Теорема~5.1.}\ \textit{Рациональному решению динамической игровой 
задачи}~(\ref{e1.1-vas})--(\ref{e2.4-vas}) \textit{отвечает ситуация 
$\tilde{x}^*\hm= (x_1^*, \ldots , x_T^*)$; $x_s^*\hm\in X_s^*$, $s\hm= \overline{1, T}$, 
найденная методом Беллмана}~(\ref{e4.2-vas}), (\ref{e5.1-vas}).
    
    \smallskip
    
    \noindent
    \textbf{Пример~5.1.} Графом~$\Gamma_R$~(\ref{e2.3-vas}), $n\hm=2$, 
динамического отношения задана ДМАС 

\vspace*{-4pt}

\noindent
    \begin{multline*}
    R: (x_1, x_2, x_3) \to (x_3, x_1, x_1\vee x_2) \cup (x_2, x_3, x_1\wedge x_2);\\
    x=(x_1, x_2, x_3)\in\underline{8}\simeq \{0,1\}^3.
    \end{multline*}
    
    \vspace*{-3pt}
    
    \noindent
Терминальная задача представляет собой игру Гермейера трех лиц 
$\Gamma^{2,3}$~\cite{1-vas, 5-vas}, где $X^T\hm=\underline{8}$~--- бинарный куб. 
Отношения предпочтений агентов $\tilde{\rho}_2^k \hm\in \mathrm{REL}\,(\underline{8},\underline{8})$ равны:
\begin{align*}
\tilde{\rho}_2^1 &=\{ 01, 05, 40, 34, 32, 76\};\\
\tilde{\rho}_2^2&= \{ 02, 10, 24, 32, 35, 45, 64, 67\};\\
\tilde{\rho}_2^3 &=\{ 20, 40, 35, 31, 64, 76\}.
\end{align*}
Найдем интересы игроков $\rho_1^k\hm= R^{\mathrm{op}}(\tilde{\rho}_2^k)$ в~начальном 
состоянии сис\-те\-мы~$\Gamma^{1,3}$, $X^0\hm= \underline{8}$: 

\vspace*{-4pt}

\noindent
\begin{multline*}
\rho_1^1= \{ 01, 02, 03, 04, 06, 12, 13, 14, 16, 20, 21, 62, 63,\\
 64, 65, 72, 73, 74, 75\}\,;
 \end{multline*}
 
 \vspace*{-14pt}
 
 \noindent
 \begin{multline*}
 \rho_1^2=\{ 03, 04, 05, 13, 14, 15, 20, 21,23, 26, 32, 40, 41,\\
  42, 52, 61, 63, 64, 65, 71, 73, 74, 75, 76\};
  \end{multline*}
  
  \vspace*{-14pt}
 
 \noindent
 \begin{multline*}
\rho_1^3 = \{ 20, 21, 30, 31, 40, 41, 50, 51, 61, 62, 63, 64, 71,\\
 72, 73, 74, 76\}.
\end{multline*}

\vspace*{-5pt}

\noindent
Вычислим результирующие отношения статических подыгр $\Gamma(X^T)$, 
$\Gamma(X^0)$ ~\cite{5-vas}:

\vspace*{-4pt}

\noindent
\begin{multline*}
P_{X^T} =\tilde{\rho}_1^3 \vert_{x_3} \circ \tilde{\rho}_1^2 \circ \left( \tilde{\rho}_1^1 
\cup \tilde{\rho}_1^{2G}\right),\\
\rho_{1}^{2G} =\tilde{\rho}_1^2 \vert_{\mathrm{MIN} \tilde{\rho}_1^2\vert_{{x_1}}}
\Rightarrow P_{X^T} =\{ 40, 05, 42, 35, 30, 70\},\\
P_{X^0} =\rho_1^1\vert_{x_1} \circ \rho_1^2\vert_{x_2}\circ\rho_1^3\vert_{x_3} =\{ 41, 53, 61, 73\}.
\end{multline*}

\vspace*{-4pt}

\noindent
Из уравнений Беллмана~(\ref{e4.2-vas}) найдем искомое ДРО игры:
$$
\tilde{P}_{X^T} =\{ 40, 05, 42, 35, 30, 70\},\ \tilde{P}_{X^0} = \{40, 00, 60, 70\}.
$$

\vspace*{-3pt}

\noindent
По формулам~(\ref{e5.1-vas}) и~тео\-ре\-ме~5.1 рациональное решение игры равно 
$\tilde{x}^*\hm=(0,0)$.

\vspace*{-12pt}

\section{Заключение}

\vspace*{-4pt}

    Функторная модель стала наследником традиционной формы представления 
ДМАС. Обладая композициональной структурой, она создает условия для 
модификации игровой задачи и~выполнения эквивалентных преобразований 
средствами алгебры тео\-рии категорий. Доказано существование ре\-зуль\-ти\-ру\-юще\-го 
отношения динамической игры многих лиц. Для задач оптимального управ\-ле\-ния 
в~форме динамических отношений предложено обобщение метода Беллмана.  
С~по\-мощью ДРО этим методом строится рациональное решение задачи. 
Функторный подход поддерживается компьютерной алгеброй тео\-рии категорий. 
Сетевая архитектура применяемых морфизмов допускает эффективную 
нейросетевую программную реализацию, которую еще только пред\-сто\-ит 
осуществить. 

\vspace*{-12pt}

{\small\frenchspacing
 {\baselineskip=10.6pt
 %\addcontentsline{toc}{section}{References}
 \begin{thebibliography}{99}
 
\vspace*{-3pt}

\bibitem{2-vas}
\Au{Моисеев~Н.\,Н.} Элементы теории оптимальных сис\-тем.~--- М.: Наука, 1974. 526~с.

\bibitem{1-vas} %2
\Au{Гермейер~Ю.\,Б.} Игры с~непротивоположными интересами.~--- М.: Наука, 1976. 326~с.

\bibitem{3-vas}
\Au{Красовский Н.\,Н., Субботин~А.\,И.} Позиционные дифференциальные игры.~--- M.: Наука, 
1976. 456~с.
\bibitem{4-vas}
\Au{Dockner E.\,J., Jorgensen~S., Long~N.\,V., Sorger~G.} Differential games in economics and management
science.~--- Cambridge: Cambridge University 
Press, 2000. 382~p. doi: 10.1017/CBO9780511805127.
\bibitem{5-vas}
\Au{Васильев Н.\,С.} Композициональное представление структуры игры многих лиц 
в~моноидальной категории бинарных отношений~// Информатика и~её применения, 2023. Т.~17. 
Вып.~2. С.~18--26. doi: 10.14357/19922264230203. EDN: GPMZTS.
\bibitem{9-vas}  %6
\Au{Dixit~A.\,K., Natebuff~B.\,J.} The art of strategy.~--- New York; London: W.\,W.~Norton~\&~Co., 
2008. 446~p.

\bibitem{7-vas}
\Au{Shoham~Y., Leyton-Brown~R.} Multiagent systems: Algorithmic, game-theoretic, and logical 
foundations.~--- Cambridge: Cambridge University Press, 2010. 532~p.
\bibitem{6-vas} %8
\Au{Bai~Q., Ren~F., Fujita~K., Znang~M.} Multi-agent and complex systems.~--- Studies in computational 
intelligence ser.~--- Springer Singapore, 2016. Vol.~670. 210~p.
\bibitem{8-vas} %9
\Au{Dixit~A.\,K., Skeath~S., Reiley~W.\,W., Jr.} Games of strategy.~--- New York; London: 
W.\,W.~Norton \&~Co., 2017. 880~p.

\bibitem{10-vas}
\Au{Скорняков Л.\,А.} Элементы общей алгебры.~--- М.: Наука, 1983. 272~с.

%\pagebreak

\bibitem{11-vas}
\Au{Маклейн~С.} Категории для работающего математика~/ Пер.\ с~англ.~--- М.: Физматлит, 2004. 352~с. 
(\Au{Mac Lane~S.} Categories for the working mathematician.~--- 2nd ed.~--- New York, NY, USA: 
Springer, 1998. 318~p.)
\bibitem{12-vas}
\Au{Губко М.\,В.} Управление организационными системами с~сетевым взаимодействием агентов. 
Обзор теории сетевых игр~// Автоматика и~телемеханика, 2004. №\,8. С.~115--132.
\bibitem{13-vas}
Group formation in economics: Networks, clubs, and coalitions~/ Eds.\ G.~Demange, M.~Wooders.~--- 
Cambridge: Cambridge University Press, 2005. 475~p.

\end{thebibliography}

 }
 }

\end{multicols}

\vspace*{-9pt}

\hfill{\small\textit{Поступила в~редакцию 02.02.24}}

%\vspace*{6pt}

\pagebreak

%\newpage

\vspace*{-28pt}

%\hrule

%\vspace*{2pt}

%\hrule



\def\tit{ON FUNCTOR REPRESENTATION OF~OPTIMIZED\\ DYNAMIC~MULTIAGENT~SYSTEMS}


\def\titkol{On functor representation of~optimized dynamic multiagent systems}


\def\aut{N.\,S.~Vasilyev}

\def\autkol{N.\,S.~Vasilyev}

\titel{\tit}{\aut}{\autkol}{\titkol}

\vspace*{-15pt}


\noindent
N.\,E.~Bauman Moscow State Technical University, 5-1~Baumanskaya 2nd Str., Moscow 105005, Russian 
Federation

\def\leftfootline{\small{\textbf{\thepage}
\hfill INFORMATIKA I EE PRIMENENIYA~--- INFORMATICS AND
APPLICATIONS\ \ \ 2024\ \ \ volume~18\ \ \ issue\ 2}
}%
 \def\rightfootline{\small{INFORMATIKA I EE PRIMENENIYA~---
INFORMATICS AND APPLICATIONS\ \ \ 2024\ \ \ volume~18\ \ \ issue\ 2
\hfill \textbf{\thepage}}}

\vspace*{4pt}



\Abste{Functors' topoi is chosen as a computational tool for synthesizing dynamic multiagent systems (DMAS). The scale orders the objects as 
multiagent system states to solve attendant static subgames in them. 
The initial dynamic game and all static subproblems are represented in the monoidal category of binary relations. 
Players' preference relations might be maximized in DMAS. The game rational solution is understood as
 equilibrium. The 
compositional structure of the optimized DMAS can be described in the form of the game dynamic resulting relation (DRR). 
Players' rational behavior search is reduced to DRR subsequent maximization. For this purpose, the Bellman's method 
is generalized to solve control problems stated in the form of relations. 
The program implementation of the approach can be based on neural networks due to the consistency of the architectures 
of the applied relation graphs and neural networks.}

\KWE{functor category; compositionality; monoidal category; opposite image; game dynamic relation; 
static subgame; preference relation; dynamic resulting relation; rational solution; Bellman morphism}

\DOI{10.14357/19922264240201}{CLMBXC}

%\vspace*{-12pt}

%\Ack

%\vspace*{-3pt}

%    \noindent
 

  \begin{multicols}{2}

\renewcommand{\bibname}{\protect\rmfamily References}
%\renewcommand{\bibname}{\large\protect\rm References}

{\small\frenchspacing
 {%\baselineskip=10.8pt
 \addcontentsline{toc}{section}{References}
 \begin{thebibliography}{99} 

\bibitem{2-vas-1}
\Aue{Moiseev, N.\,N.} 1975. \textit{Elementy teorii optimal'nykh sistem} [Elements of optimal systems 
theory]. Moscow: Nauka. 527~p.

\bibitem{1-vas-1}
\Aue{Germeyer, Yu.\,B.} 1976. \textit{Igry s~neprotivopolozhnymi interesami} [Games with  
nonopposing interests]. Moscow: Nauka. 326~p.

\bibitem{3-vas-1}
\Aue{Krasovskiy, N.\,N., and A.\,I.~Subbotin.} 1974. \textit{Pozitsionnye differentsial'nye igry} 
[Positional differential games]. Moscow: Nauka. 456~p.
\bibitem{4-vas-1}
\Aue{Dockner, E.\,J., S.~Jorgensen, N.\,V.~Long, and G.~Sorger.} 2000. \textit{Differential games in 
economics and management science}. Cambridge: Cambridge University Press. 382~p. doi: 
10.1017/CBO9780511805127.
\bibitem{5-vas-1}
\Aue{Vasilyev, N.\,S.} 2023. Kompozitsional'noe predstavlenie struktury igry mnogikh lits 
v~monoidal'noy kategorii binarnykh otnosheniy [Multiplayers' games compositional structure in the 
monoidal category of binary relations]. \textit{Informatika i~ee Primeneniya~--- Inform. Appl.} 
 17(2):18--26. doi: 10.14357/19922264230203. EDN: GPMZTS.
 
 \bibitem{9-vas-1} %6
\Aue{Dixit, A.\,K., and B.\,J.~Nalebuff.} 2008. \textit{The art of strategy}. New York, London: 
W.\,W.~Norton \&~Co. 446~p.


\bibitem{7-vas-1}
\Aue{Shoham, Y., and R.~Leyton-Brown.} 2010. \textit{Multiagent systems: Algorithmic,  
game-theoretic, and logical foundations}. Cambridge University Press. 532~p.

\bibitem{6-vas-1} %8
\Aue{Bai, Q., F.~Ren, K.~Fujita, and M.~Znang.} 2016. \textit{Multi-agent and complex systems}. Studies 
in computational intelligence ser.  Springer Singapore. 210~p.

\bibitem{8-vas-1} %9
\Aue{Dixit, A.\,K., S.~Skeath, and D.\,H.~Reiley, Jr.} 2017. \textit{Games of strategy}. New York, 
London: W.\,W.~Norton \&~Co.\linebreak 880~p.

\bibitem{10-vas-1}
\Aue{Skornyakov, L.\,A.} 1983. \textit{Elementy obshchey algebry} [Elements of general algebra]. 
Moscow: Nauka. 272~p.
\bibitem{11-vas-1}
\Aue{Mac Lane, S.} 1998. \textit{Categories for the working mathematician}. 2nd ed. New York, NY: 
Springer. 318~p. 
\bibitem{12-vas-1}
\Aue{Gubko, M.\,V.} 2004. Control of organizational systems with network interaction of agents. 
II.~Stimulation problems. \textit{Automat. Rem. Contr.} 65(9):1470--1485. doi: 
10.1023/B:AURC.0000041425.34118.7d. EDN: \mbox{LFMUCG}.
\bibitem{13-vas-1}
Demange, G., and M.~Wooders, eds. 2005. \textit{Group formation in economics: Networks, clubs, and 
coalitions}. Cambridge: Cambridge University Press. 475~p. 

\end{thebibliography}

 }
 }

\end{multicols}

\vspace*{-8pt}

\hfill{\small\textit{Received February 2, 2024}} 

\vspace*{-18pt}


\Contrl

\vspace*{-3pt}

\noindent
\textbf{Vasilyev Nikolai S.} (b.\ 1952)~--- Doctor of Science in physics and mathematics, professor, 
N.\,E.~Bauman Moscow State Technical University, 5-1 Baumanskaya 2nd Str., Moscow 105005, Russian 
Federation; \mbox{nik8519@yandex.ru}





\label{end\stat}

\renewcommand{\bibname}{\protect\rm Литература}  %1
%\newcommand {\ff}{{\mathcal F}}
\newcommand {\ebd}{\triangleq}
\newcommand{\me}[2]{\mathbf{E}_{ #1 }\left\{ \mathop{#2} \right\} }



\def\stat{borisov}

\def\tit{ФИЛЬТРАЦИЯ СОСТОЯНИЙ МАРКОВСКИХ СКАЧКООБРАЗНЫХ ПРОЦЕССОВ 
ПО~ДИСКРЕТИЗОВАННЫМ НАБЛЮДЕНИЯМ$^*$}

\def\titkol{Фильтрация состояний марковских скачкообразных процессов 
по~дискретизованным наблюдениям}

\def\aut{А.\,В.~Борисов$^1$}

\def\autkol{А.\,В.~Борисов}

\titel{\tit}{\aut}{\autkol}{\titkol}

\index{Борисов А.\,В.}
\index{Borisov A.\,A.}




{\renewcommand{\thefootnote}{\fnsymbol{footnote}} \footnotetext[1]
{Работа выполнена при частичной поддержке РФФИ (проект 16-07-00677).}}


\renewcommand{\thefootnote}{\arabic{footnote}}
\footnotetext[1]{Институт проблем информатики Федерального исследовательского центра <<Информатика 
и~управление>> Российской академии наук,
\mbox{aborisov@frccsc.ru}}

%\vspace*{8pt}



\Abst{Статья посвящена решению задачи оптимальной 
фильтрации состояний однородного марковского скачкообразного процесса (МСП). 
Наблюдения представляют собой приращения случайных процессов~--- интегральных 
преобразований состояний, зашумленные винеровскими процессами, интенсивность 
которых также зависит от оцениваемого состояния. Оптимальная оценка в~моменты 
получения нового наблюдения вычисляется как функция предыдущей оценки и~новых 
наблюдений, а~между моментами наблюдений~--- простейшим прогнозом в~силу системы 
уравнений Колмогорова. Рекуррентная формула пересчета ресурсозатратна, так как 
содержит  интегралы~--- мас\-штаб\-но-сдви\-го\-вые смеси многомерных гауссиан, 
где в~качестве смешивающих выступают распределения времени пребывания 
состояния в~каждом из возможных значений. Предложены более простые аппроксимации, 
основанные на предположении об ограниченности числа скачков состояния за время между 
наблюдениями. Получены универсальные локальная и~глобальная характеристики точности 
аппроксимаций, зависящие от па\-ра\-мет\-ров оцениваемого процесса, величины 
временн$\acute{\mbox{о}}$го шага  между наблюдениями и~максимального числа учитываемых скачков.}

\KW{марковский скачкообразный процесс; оптимальная фильтрация; мультипликативные 
шумы в~наблюдениях; стохастическое дифференциальное уравнение; численная аппроксимация}

\DOI{10.14357/19922264180316}
  
%\vspace*{4pt}


\vskip 10pt plus 9pt minus 6pt

\thispagestyle{headings}

\begin{multicols}{2}

\label{st\stat}



 \section{Введение}
 
 Фильтр Вонэма~\cite{Won_65}~--- один из редких удачных случаев, когда 
 оценка оптимальной фильтрации состо\-яния стохастической системы наблюдения 
 выражается в~виде решения некоторой замк\-ну\-той\linebreak конечномерной сис\-те\-мы 
 стохастических дифференциальных уравнений. 
 
 Алгоритм данного фильт\-ра 
 позволяет вычислить оценку фильт\-ра\-ции со\-сто\-яния \textit{марковского скачкообразного 
 процесса} с~\mbox{конечным} множеством состояний по наблюдениям в~присутствии 
 аддитивных винеровских шумов. Теоретически оптимальная оценка со\-сто\-яния~--- 
 его условное распределение в~текущий момент времени~--- 
 обладает очевидными свойствами неотрицательности и~нормировки. 
 При чис\-лен\-ной реализации данного фильтра классическим методом 
 Эй\-ле\-ра--Ма\-ру\-ямы~\cite{KP_92} данные свойства могут не сохраняться и~процедура 
 вы\-чис\-ле\-ния становится неустойчивой.  В~связи с~этим обстоятельством разрабатывались 
 другие алгоритмы чис\-лен\-но\-го решения уравнения фильтра Вонэма, обладающие 
 требуемыми свойствами устойчивости (см.~\cite{YZL_04, PR_10} и~библиографию в~них). 
 В~час\-ти этих работ доказана лишь слабая сходимость пред\-ла\-га\-емых аппроксимационных 
 схем к~оценке фильт\-ра Вонэма, в~то время как ка\-кая-ли\-бо 
 характеризация точ\-ности этих приближений отсутствует.
 
 В~\cite{B_18} было представлено распространение фильт\-ра Вонэма на случай 
 наблюдений с~мультипликативными шумами. При этом уравнение обобщенного 
 фильт\-ра содержит в~правой части квадратическую характеристику шумов в~наблюдениях. 
 Данный процесс на практике никогда не наблюдается непосредственно, а~является лишь 
 некоторым нелинейным интегральным преобразованием наблюдений. Очевидно, что 
 имеющиеся в~настоящий момент времени алгоритмы приближенного вычисления оценки 
 фильтрации Вонэма для данной системы не подходят. 
 
 Целью предлагаемой работы является ис\-поль\-зование результатов оптимальной 
 фильтрации со\-стояний сис\-тем с~дискретным временем для аппроксимации решения 
 аналогичной задачи для\linebreak стохастических дифференциальных сис\-тем. 
 
 Статья организована следующим образом. Раздел~2 содержит формальную постановку 
 задачи фильт\-ра\-ции со\-сто\-яний однородного МСП с~конечным множеством со\-сто\-яний 
 по наблюдениям, полученным путем временн$\acute{\mbox{о}}$й дискретизации процессов с~непрерывным 
 временем~--- интегральных преобразований со\-сто\-яния сис\-те\-мы в~присутствии 
 мультипликативных винеровских шумов.\linebreak
  В~разд.~3 пред\-став\-ле\-но решение поставленной 
 задачи фильт\-ра\-ции: пересчет оценок со\-сто\-яний в~момент получения новых 
 дискретизованных наблюдений выполняется в~соответствии с~некоторыми\linebreak 
 рекуррентными интегральными соотношениями, в~то время как между 
 моментами наблюдений оценка корректируется в~соответствии с~прогнозом в~силу 
 сис\-те\-мы уравнений Колмогорова. Вы\-чис\-ли\-тель\-ная слож\-ность 
 упомянутых выше интегральных\linebreak 
 соотношений связана с~тем, что в~расчет принимается воз\-мож\-ность того, что между 
 моментами наблюдений оцениваемое со\-сто\-яние может совершить произвольное чис\-ло 
 скачков. В~разд.~4 пред\-став\-лен более простой алгоритм приближенного вы\-чис\-ле\-ния 
 оценки фильт\-ра\-ции, основанный на ограничении возможного числа учитываемых скачков 
 МСП. Доказана тео\-ре\-ма, опре\-де\-ля\-ющая как\linebreak
  локальную (одношаговую), так и~глобальную 
 (многошаговую) характеристики точ\-ности предложенного при\-бли\-же\-ния~--- 
 $\ell_1$-нор\-мы ошибки аппроксимации. Полученные характеристики являются\linebreak 
 универсальными, т.\,е.\ не асимптотическими по шагу дискретизации, и~зависят от характеристик 
 самого МСП, %\linebreak
  шага временн$\acute{\mbox{о}}$й дискретизации и~чис\-ла
  скачков со\-сто\-яния, учи\-ты\-ва\-емых 
 на шаге. Об\-суж\-де\-ние результатов и~заключительные комментарии пред\-став\-ле\-ны 
 в~разд.~5.
 
 \section{Постановка задачи фильтрации}
 
 На полном вероятностном пространстве с~фильт\-ра\-цией 
 $(\Omega,\mathcal{F},\mathcal{P},\{\mathcal{F}_{t}\}_{t \geqslant 0})$ рассматривается система наблюдений
\begin{equation}
 \left.
 \begin{array}{rl}
 \displaystyle X_t &=X_0 +  \displaystyle
 \int\limits_0^t \Lambda^{\top}X_{s}\,ds + \mu_s\,;  \\[6pt]
 \displaystyle Y_k &=  \displaystyle\int\limits_{t_{k-1}}^{t_k}fX_s\,ds+
 \int\limits_{t_{k-1}}^{t_k} 
 \sum\limits_{n=1}^NX_s^ng_n \,dW_s, \\[6pt]
 &\hspace*{10mm}\{t_k\}_{k \geqslant 0}: \; 0 = t_0 < t_1 < t_2\cdots,
 \end{array}
 \right\}
 \label{eq:obsys_1}
 \end{equation}
 где
  \begin{itemize}
  \item
  $X_t \ebd \mathrm{col}\left(X_t^1,\ldots,X_t^N\right) \hm\in \mathbb{S}^N$~--- 
  ненаблюда\-емое состояние системы, являющееся однородным МСП с~конечным 
  множеством состояний $ \mathbb{S}^N \ebd$\linebreak $\ebd \{e_1,\ldots,e_N\}$ ($\mathbb{S}^N$~--- 
  множество единичных векторов евклидова пространства~$\mathbb{R}^N$), 
  матрицей интенсивностей переходов~$\Lambda$ и~начальным распределением~$\pi$;
  \item
  $\mu_t \ebd \mathrm{col}\left(
  \mu_t^1,\ldots,\mu_t^N\right)\hm\in \mathbb{R}^N$~--- 
  ${\mathcal{F}}_t$-со\-гла\-со\-ван\-ный мартингал;
  \item
  $\{Y_k\}_{k \in \mathbb{N}}:\;  Y_k \ebd \mathrm{col}\left(Y_k^1,\ldots,Y_k^M\right) 
  \hm\in \mathbb{R}^M$~--- последовательность дискретизованных наблюдений, 
  доступных в~известные неслучайные  моменты времени~$\{t_k\}_{k \in \mathbb{N}}$,
в~которых $W_t \ebd$\linebreak $\ebd \mathrm{col}\left(W_t^1,\ldots,W_t^M\right) \hm\in \mathbb{R}^M$
 является ${\mathcal{F}}_t$-со\-гла\-со\-ван\-ным стандартным винеровским процессом, 
 определяющим шумы в~наблюдениях,\linebreak  $f$~--- $(M \times N)$-мер\-ная 
 мат\-ри\-ца плана наблюдений, а~набор мат\-риц~$\{g_n\}_{n=\overline{1,N}}$ 
 характеризует интенсивности шумов в~зависимости от текущего состояния~$X_t$.
  \end{itemize}
  
  Введем также в~рассмотрение неубывающие семейства $\sigma$-ал\-гебр 
  $\mathcal{O}_k \ebd \sigma\{ Y_{\ell}: \; 1 \hm\leqslant \ell \hm\leqslant k\}$ 
  и~$\mathcal{O}_t \ebd  \mathcal{O}_{k(t)}$, где 
  $k(t) \ebd \sum\nolimits_{j \in \mathbb{N}}\mathbf{I}(t-t_{j})$; 
  $\mathcal{O}_0 \ebd \{\varnothing,\; \Omega\}$.
  
   \textit{Задача оптимальной фильтрации состояния~$X$ по наблюдениям~$Y$} 
   заключается в~нахождении \textit{условного математического ожидания} (УМО)
  \begin{equation*}
  \widehat{X}_t \ebd {\sf E}\left\{X_t|\mathcal{O}_{t} \right\}\,.
 % \label{eq:fest_1}
  \end{equation*}
  
  Относительно системы~(\ref{eq:obsys_1})  сделаны следующие предположения:
   \begin{itemize}
 \item[(а)]
 ${\mathcal{F}}_t \equiv {\mathcal{F}}_{t}^X \bigvee 
 {\mathcal{F}}_{t}^W $ для любого $t \hm\geqslant 0$;
 \item[(б)]
 шумы в~наблюдениях равномерно невырожденные, т.\,е.\
  $g_ng_n^{\top} \hm\geqslant \alpha I \hm> 0$ для всех $n\hm=\overline{1,N}$ 
  и~некоторого $\alpha\hm>0$.
% \item
 % Верно неравенство
  %\begin{equation}
  %\min_{1\leqslant k \leqslant N}|\lambda_{kk}| > 0.
  %\label{eq:ineq_0}
  % \end{equation}
 %\item
 %Для любого $t \geqslant 0$ все компоненты вектора $p_t \ebd \me{}{X_t}$ строго %положительны. 
 \end{itemize} 

 \section{Уравнения оптимального фильтра} 
 
 Для получения уравнений оптимального фильт\-ра воспользуемся подходом, 
 применяемым для решения аналогичной задачи в~стохастических сис\-те\-мах 
 наблюдения с~дискретным временем~\cite{BSh_85}. 
 Воспользу\-ем\-ся методом математической индукции. 
 
 При $r=0$ 
 \begin{equation}
 \widehat{X}_{t_0}={\sf E}\{X_0|\mathcal{O}_0\}={\sf E}\{X_0\}=\pi\,.
 \label{eq:in_cond}
 \end{equation} 
 
 Пусть для некоторого $ r \hm\geqslant 0$ известна оценка оптимальной 
 фильтрации~$\widehat{X}_{t_r} \hm= {\sf E}{X_{t_r} |\mathcal{O}_r}$. 
 Определим оценку оптимальной фильтрации~$\widehat{X}_{t} $ для $t\hm \in (t_r,t_{r+1}]$. 
 
 Для произвольного момента $t \hm\in (t_r,t_{r+1})$ в~силу мартингального 
 разложения МСП~$X_t$ и~свойств УМО верна следующая цепочка равенств:
 \begin{multline*}
 \widehat{X}_{t} = {\sf E}\left\{X_t | \mathcal{O}_r\right\}={}\\
 {}=
 {\sf E}\left\{{\cal P}^{\top}(t_r,t)X_{t_r}+
 \int\limits_{t_r}^t{\cal P}^{\top}(t_r,s)\,dM_s\big\vert \mathcal{O}_r\right\} = {}
\end{multline*}

\noindent
   \begin{multline}
 \hspace*{-11.66pt}{}=\mathcal{P}^{\top}(t_r,t)\widehat{X}_{t_r} + {\sf E}\hspace*{-2pt}
 \left\{{\sf E}\hspace*{-2pt}\left\{\int\limits_{t_r}^t\hspace*{-2pt}\mathcal{P}^{\top}(t_r,s)\,dM_s |
 {\mathcal{F}}_{t_r}\right\}\!\big\vert 
 \mathcal{O}_r\!\right\} ={}\hspace*{-4.24124pt}\\
 {}=
  \mathcal{P}^{\top}(t_r,t)\widehat{X}_{t_r}\,,
 \label{eq:bw_obs}
 \end{multline}
 где $\mathcal{P}(s,t)$ $(s \hm\leqslant t)$~--- матрица переходной ве\-ро\-ят\-ности МСП 
 на промежутке $[s,t]$, являющаяся решением сис\-те\-мы дифференциальных 
 уравнений Колмогорова
 \begin{equation*}
 \mathcal{P}'_t(s,t) = \mathcal{P}(s,t) \Lambda, \enskip t > s, \enskip \mathcal{P}(s,s) = I.
 \end{equation*}
 В случае однородного МСП $\mathcal{P}(s,t) \hm= e^{(t-s)\Lambda}$.
 
 Далее необходимо определить совместное распределение $(X_{t_{r+1}},Y_{r+1})$ 
 относительно~$ \mathcal{O}_r$. Из модели наблюдений следует, что 
 распределение~$Y_{r+1}$ относительно 
 $\sigma$-ал\-геб\-ры~$\mathcal{F}^X_{t_{r+1}} \vee \mathcal{O}_r$~---
 гауссовское с~параметрами 
 \begin{align*}
{\sf E}\left\{Y_{r+1}|{\mathcal{F}}^X_{t_{r+1}}\right\}& = f \tau_{r+1}\,; \\[6pt]
 \mathrm{cov} \left(Y_{r+1},Y_{r+1}|{\mathcal{F}}^X_{t_{r+1}}\right) &= 
 \displaystyle\sum\limits_{n=1}^N \tau_{r+1}^n g_ng_n^{\top}\,,
% \label{eq:occup_1}
 \end{align*}
 где $\tau_{r+1} \hm= \tau_{r+1}(X(\omega))=
 \mathrm{col}\left(\tau_{r+1}^1,\ldots,\tau_{r+1}^N\right) \ebd$\linebreak
 $\ebd 
 \int\nolimits_{t_r}^{t_{r+1}}X_s\,ds$~--- случайный вектор, $n$-я 
 компонента которого равна времени пребывания процесса~$X$ в~со\-сто\-янии~$e_n$ 
 на  интервале времени $[t_r, t_{r+1}]$. 
 Обозначим через $\mathcal{D}_{r+1} \ebd \{u=\mathrm{col}\,(u^1,\ldots,u^N):\; 
 u_m \hm\geqslant 0,\; \sum\nolimits_{m=1}^Mu_m\hm= t_{r+1}-t_r\}$ $(M-1)$-мер\-ный 
 симплекс в~пространстве~$\mathbb{R}^M$, являющийся носителем распределения 
 вектора~$\tau_{r+1}$. Пусть $\rho^{k,\ell}_{r+1}(du)$~--- 
 распределение вектора $\tau_{r+1} X_{t_{r+1}}^{\ell}$ при условии $X_{t_r}\hm=e_k$, 
 т.\,е.\ 
 для любого $\mathcal{A} \hm\in \mathcal{B}(\mathbb{R}^M)$ верно тождество:
\begin{multline*}
 \mathbf{P}\left\{\omega: \; X_{t_{r+1}}(\omega)=e_{\ell},\right.\\
 \left. 
 \tau_{r+1}(X(\omega)) \in \mathcal{A}\;|\;X_{t_r}=e_k\right\} \equiv
   \rho^{k,\ell}_{r+1}(\mathcal{A})\,.
\end{multline*}
 
Обозначим через
\begin{multline*}
 \mathcal{N}(y,m,K) \ebd (2\pi)^{-M/2} \mathrm{ det}^{-1/2} K\times{}\\
 {}\times\exp
 \left\{ -\fr{1}{2}\left(y-m\right)^{\top}K^{-1}(y-m)\right\}
\end{multline*}
 $M$-мер\-ную плот\-ность гауссовского распределения с~математическим 
 ожиданием~$m$ и~ковариационной матрицей~$K$.
 
 Из марковского свойства  $\{X_{t_{r}},Y_{r})\}_{r \geqslant 0}$ 
 относительно~${\mathcal{F}}_{t_{r}}$~\cite{ZhSh_95} и~теоремы Фубини следует, что 
 для любого  множества $\mathcal{A} \hm\in \mathcal{B}(\mathbb{R}^M)$ 
 верна следующая цепочка равенств:
 \begin{multline*}
 {\sf E}\left\{X_{t_{r+1}}\mathbf{I}_{\mathcal{A}}
 \left(Y_{r+1}\right)\big|\mathcal{O}_r\right\}={}\\
 {}=
{\sf E}\left\{{\sf E}\left\{X_{t_{r+1}}\mathbf{I}_{\mathcal{A}}
\left(Y_{r+1}\right)\big|
\mathcal{F}^X_{t_{r+1}} \vee \mathcal{O}_r\right\}
 \big|\mathcal{O}_r\right\} = {}
\end{multline*}

\noindent
\begin{multline*}
 %{}=
% {\sf E}\left\{{\sf E}\left\{X_{t_{r+1}}\mathbf{I}_{\mathcal{A}}
% \left(Y_{r+1}\right)\vert X_{t_r}\right\}
% \vert\mathcal{O}_r\right\} = {}\\
% {}=
%{\sf E}\left\{\sum\limits_{k=1}^N {\sf E}\left\{X_{t_{r+1}}\mathbf{I}_{\mathcal{A}}
%\left(Y_{r+1}\right)  \big| X_{t_r}=e_k\right\}X_{t_r}^k
% \big|\mathcal{O}_r\right\} = {}\\ 
% {}=
% \sum\limits_{k=1}^N{\sf E}
% \left\{X_{t_{r+1}}\mathbf{I}_{\mathcal{A}}\left(Y_{r+1}\right)\bigl| X_{t_r}=e_k\right\} 
% \widehat{X}_{t_r}^k ={}\\
% {}=\!
% \sum\limits_{k=1}^N{\sf E}
% \left\{{\sf E}\left\{X_{t_{r+1}}\mathbf{I}_{\mathcal{A}}
% \left(Y_{r+1}\right)\!\bigl| {\mathcal{F}}_{t_{r+1}}\right\}\!\bigl| 
% X_{t_r}\!=e_k\right\} \widehat{X}_{t_r}^k ={}\\
% {}=
% \sum\limits_{k=1}^N {\sf E}\left\{
% \vphantom{\int\limits_A\left(\sum\limits_{p=1}^N\right)}
% X_{t_{r+1}} \times{}\right.\\
% {}\times\int\limits_{\mathcal{A}}  
% \mathcal{N}\left(y,f \tau_{r+1}(X),\sum\limits_{p=1}^N \tau_{r+1}^p(X) g_pg_p^{\top}\right)dy
% \Biggl| X_{t_r}={}\\
%\left. {}=e_k
% \vphantom{\int\limits_A\left(\sum\limits_{p=1}^N\right)}
%\right\} \widehat{X}_{t_r}^k = 
% \sum\limits_{k=1}^N \int\limits_{\mathcal{A}}{\sf E}\left\{ 
% \vphantom{\sum\limits_{p=1}^N}
% X_{t_{r+1}} \times{}\right.\\
% {}\times\mathcal{N}\left(y,f \tau_{r+1}(X),\sum\limits_{p=1}^N \tau_{r+1}^p(X) 
% g_p g_p^{\top}\right)
% \Biggl| X_{t_r}={}\\
%\left. {}=e_k
%\vphantom{\sum\limits^N_{p=1}}
%\right\} \widehat{X}_{t_r}^k\, dy
 %={}\\
 {}=
 \sum\limits_{\ell=1}^N e_{\ell} \int\limits_{\mathcal{A}} 
 \left[ \sum\limits_{k=1}^N 
 \int\limits_{\mathcal{D}_{r+1}} 
 \mathcal{N}\left(y,f u,\sum_{p=1}^N u^p g_pg_p^{\top}\right)\times{}\right.\\
\left. {}\times
 \rho^{k,\ell}_{r+1}(du)\widehat{X}_{t_r}^k
 \vphantom{\int\limits_A\sum\limits_{p=1}^N}
 \right] 
 dy,
 \end{multline*}
 из чего следует, что интегранд в~квадратных скобках в~последнем выражении 
 определяет искомое совместное распределение $(X_{t_{r+1}},Y_{r+1})$ 
 относительно~$ \mathcal{O}_r$. Оценка~$\widehat{X}_{t_{r+1}}$ покомпонентно 
 определяется~\cite{BSh_85} с~помощью обобщенного варианта формулы Байеса:
 \begin{multline}
 \widehat{X}_{t_{r+1}}^j = {}\\
 \hspace*{-1mm}{}=
 \fr{\int\nolimits_{\mathcal{D}_{r+1}}\hspace*{-6mm} 
 \mathcal{N}\left(Y_{r+1},f u,\sum\nolimits_{p=1}^N \hspace*{-2mm}
 u^p g_pg_p^{\top}\!\right)\hspace*{-1mm}
 \sum\nolimits_{k=1}^N \hspace*{-2mm}
 \widehat{X}_{t_r}^k
 \rho^{k,j}_{r+1}(du)
 }
 { \int\nolimits_{\mathcal{D}_{r+1}} \hspace*{-6mm}
 \mathcal{N}\left(Y_{r+1},f v,\sum\nolimits_{q=1}^N \hspace*{-2mm}
 v^q g_qg_q^{\top}\!\right)\hspace*{-1mm}
 \sum\nolimits_{i,\ell=1}^N \hspace*{-2mm}
 \widehat{X}_{t_r}^i
 \rho^{i,\ell}_{r+1}(dv)
  },  \\ 
  j = \overline{1,N}\,.
 \label{eq:filt_1}
 \end{multline}
 Таким образом, доказана следующая
 
 %\smallskip
 
 \noindent
 \textbf{Лемма~1.}
\textit{Если для системы наблюдения}~(\ref{eq:obsys_1}) 
\textit{верны условия~(а) и~(б), то оценка~$\widehat{X}_t$ оптимальной фильтрации 
определяется формулой}~(\ref{eq:in_cond}) 
\textit{при $t\hm=0$, рекуррентным соотношением}~(\ref{eq:filt_1})~---
\textit{в~моменты~$t_{r+1}$ получения наблюдений~$Y_{r+1}$ 
и~формулой}~(\ref{eq:bw_obs})~--- 
\textit{в~промежутках времени между моментами получения наблюдений}.


\smallskip
 

 
 Несмотря на компактную запись~(\ref{eq:filt_1}), их прямая численная реализация 
 ресурсозатратна. Во-пер\-вых, в~(\ref{eq:filt_1}) требуется вычислять 
 распределения мас\-штаб\-но-сдви\-го\-вых смесей многомерных нормальных 
 распределений, что является трудоемкой\linebreak процедурой. Во-вто\-рых, 
 распределения~$\rho^{k,j}_{r+1}$ вре-\linebreak мени пребывания представляют собой 
 сумму\linebreak бесконечного ряда, слагаемые которого вычис\-ляются с~помощью 
 некоторой рекуррентной про\-це\-дуры~\cite{S_00}. В-третьих, 
 распределения~$\rho^{k,j}_{r+1}$ не являются абсолютно непрерывными 
 относительно меры Ле\-бега.
 { %\looseness=1
 
 }
 
 Следующий раздел посвящен численной аппроксимации~(\ref{eq:filt_1}) и~исследованию 
 ее точностных характеристик.
 
 \section{Приближенное вычисление оценки фильтрации}
 
 Без ограничения общности будем считать, что сетка~$\{t_r\}_{r \geqslant 0}$ 
 является равномерной с~шагом~$\Delta$, т.\,е.\ $t_r \hm= r \Delta$ 
 и~$\mathcal{D}_r \hm\equiv \mathcal{D}$.
 Обозначим через~$N_{r+1}$ об-\linebreak\vspace*{-12pt}
 
 \pagebreak
 
 \noindent
 щее число скачков процесса~$X_t$, имевших место 
 на промежутке $(t_r,t_{r+1}]$. Тогда из формулы полной вероятности следует, 
 что~(\ref{eq:filt_1}) представима в~виде:
 \begin{multline}
 \widehat{X}_{t_{r+1}}^j =  \left(
 \int\limits_{\mathcal{D}} 
 \mathcal{N}\left(Y_{r+1},f u,\sum\limits_{p=1}^N u^p g_pg_p^{\top}\right)\times{}\right.\\
\left. {}\times
 \sum\limits_{h=0}^{\infty}\sum\limits_{k=1}^N \widehat{X}_{t_r}^k
 \rho^{k,j,h}_{r+1}(du)
 \right)\Bigg/ \\
 \left(
 \vphantom{\sum\limits_{m=0}^{\infty}
 \sum\limits_{i,\ell=1}^N \widehat{X}_{t_r}^i
 \rho^{i,\ell,m}_{r+1}(dv)}
 \int\limits_{\mathcal{D}} 
 \mathcal{N}\left(Y_{r+1},f v,\sum\limits_{q=1}^N v^q g_qg_q^{\top}\right)\times{}\right.\\
\left.{}\times \sum\limits_{m=0}^{\infty}
 \sum\limits_{i,\ell=1}^N \widehat{X}_{t_r}^i
 \rho^{i,\ell,m}_{r+1}(dv)
 \right)
  \,, \enskip j = \overline{1,N}\,,
  \label{eq:filt_1_1}
 \end{multline}
 где 
 $ \rho^{k,j,h}_{r+1}(du)$~--- распределение вектора 
 $\tau_{r+1}X_{t_{r+1}}^{j}\mathbf{I}_{\{h\}}(N_{r+1})$ при 
 условии $X_{t_r}\hm=e_k$, т.\,е.\ 
 для любого $\mathcal{A} \hm\in \mathcal{B}(\mathbb{R}^M)$ верно тождество
\begin{multline*}
 \mathbf{P}\left\{\omega: \; X_{t_{r+1}}(\omega)=e_{j}, \; N_{r+1} = h,\right.\\ 
\left. \tau_{r+1}(X(\omega)) \in \mathcal{A}\;|\;X_{t_r}=e_k\right\} \equiv
  \rho^{k,j,h}_{r+1}(\mathcal{A}).
\end{multline*}
В качестве аппроксимации оценок можно использовать  
 $\overline{X}_{t_{r+1}}^n \ebd 
 \mathrm{col}\,(\overline{X}_{t_{r+1}}^{n,1},\ldots,\overline{X}_{t_{r+1}}^{n,N})$, 
 полученные из~(\ref{eq:filt_1_1}) путем урезания сумм ряда в~числителе и~знаменателе:
 
 \noindent
 \begin{multline}
 \overline{X}_{t_{r+1}}^{n,j} = 
 \left(
 \int\limits_{\mathcal{D}} 
 \mathcal{N}\left(Y_{r+1},f u,\sum\limits_{p=1}^N u^p g_pg_p^{\top}\right)\times{}\right.\\[-1pt]
\left.{}\times \sum\limits_{h=0}^{n}\sum\limits_{k=1}^N \overline{X}_{t_r}^k
 \rho^{k,j,h}_{r+1}(du)
 \right)\Bigg/ \\[-1pt]
 \left(
 \int\limits_{\mathcal{D}} 
 \mathcal{N}\left(Y_{r+1},f v,\sum\limits_{q=1}^N v^q g_qg_q^{\top}\right)\times{}\right.\\[-1pt]
\left. {}\times
 \sum\limits_{m=0}^{n}
 \sum\limits_{i,\ell=1}^N \overline{X}_{t_r}^i
 \rho^{i,\ell,m}_{r+1}(dv)
  \right)\,, \enskip
   j = \overline{1,N}.
  \label{eq:filt_2}
 \end{multline}
 Ниже по формуле полной вероятности получены интегралы из~(\ref{eq:filt_2}) для 
 $h\hm=0,1,2$:
 
\vspace*{-3pt}

 \noindent
  \begin{multline*}
 \int\limits_{\mathcal{D}}  \mathcal{N}
 \left(Y_{r+1},f u,\sum\limits_{p=1}^N u^p g_pg_p^{\top}\right) 
 \rho^{k,j,0}_{r+1}(du) = {}\\[-1pt]
 {}=
 \delta_{kj}\mathcal{N}\left(Y_{r+1},\Delta f^j,\Delta g_jg_j^{\top}\right)
 e^{\lambda_{jj}\Delta};
 %\label{eq:h0}
\\[-1pt]
 \int\limits_{\mathcal{D}}  \mathcal{N}\left(
 Y_{r+1},f u,\sum\limits_{p=1}^N u^p g_pg_p^{\top}\right) 
 \rho^{k,j,1}_{r+1}(du) ={} 
 \end{multline*}
 
 \noindent
 \begin{multline}
 \hspace*{-6.7pt}{}=\left(1-\delta_{kj}\right)\lambda_{kj}e^{\lambda_{jj}\Delta}
\! \int\limits_0^{\Delta}\!
 e^{(\lambda_{kk}-\lambda_{jj})u^k}
 \mathcal{N}\left(Y_{r+1},u^kf^k +{}\right.\hspace*{-0.28818pt}\\[-1pt]
\hspace*{-3mm}\left. {}+ \left(\Delta - u^k\right)f^j, u^k g_kg_k^{\top}+
 \left(\Delta-u^k\right)g_jg_j^{\top}\right)\,du^k;
 \label{eq:h1}
 \end{multline}
 
 \vspace*{-12pt}
 
 \noindent
 \begin{multline}
 \int\limits_D \mathcal{N}\left( 
Y_{r+1},f u,\sum\limits_{p=1}^N u^p g_pg_p^{\top}\right)du ={}\\[-1pt]
{}=
\sum\limits_{\substack{{\ell:\ell \neq k,}\\ {\ell \neq j}}}
 \lambda_{k\ell}\lambda_{\ell j} e^{\lambda_{jj}\Delta}\times {}\\[-1pt] 
 {}\times
 \int\limits_0^{\Delta} \int\limits_0^{\Delta-u^k} \!
e^{(\lambda_{kk}-\lambda_{\ell\ell})u^k+(\lambda_{\ell\ell}-
 \lambda_{jj})u^{\ell}}\times{} \\[-1pt] 
{}  \times
 \mathcal{N}\left(Y_{r+1},u^k f^k+u^{\ell}f^{\ell}+\left(
 \Delta-u^k-u^{\ell} \right)f^j,\right.\\[-1pt]
 \hspace*{-1mm}\left.
 u^k g_kg_k^{\top}+u^{\ell}g_{\ell}g_{\ell}^{\top}+\left(
 \Delta-u^k-u^{\ell} \right)
 g_jg_j^{\top}
 \right) du^{\ell}du^{k}, \!\!
  \label{eq:h2}
 \end{multline} 
 
\vspace*{-2pt}
 
 \noindent
  где  $\delta_{ij}$~--- символ Кронекера. Интегралы для $h\hm>2$ также могут 
  быть получены в~явном виде, однако их сложность резко возрастает.
 

   Так как система~(\ref{eq:obsys_1}) является автономной, то в~качестве локальной 
   характеристики бли\-зости~$\{\overline{X}_{t_r}\}$ 
   к~$\{\widehat{X}_{t_r}\}$ может быть выбрана величина
   
\noindent
 \begin{multline*}
 \overline{\sigma}(\pi) \ebd {\sf E}\left\{
 \|\widehat{X}_{t_{1}}(\pi, Y_{1}) - \overline{X}_{t_{1}}
 \left(\pi,Y_{1}\right)\|_{1}\right\} = {}\\
 {}=
 \sum\limits_{j=1}^N{\sf E}
 \left\{\left\vert \widehat{X}^j_{t_{1}}\left(\pi, Y_{1}\right) - \overline{X}^{n,j}_{t_{1}}
 \left(\pi,Y_{1}\right)\right\vert\right\}.
 %\label{eq:prec_1}
 \end{multline*}
 При этом начальное распределение $\pi \hm\in \mathcal{D}_1 \ebd $\linebreak $\ebd
 \{\mathrm{col}\,(\pi^1,\ldots,\pi^N):\;\pi^j > 0$, 
 $\sum\nolimits_{j=1}^N\pi^j\hm=1\}$ является начальным условием применения 
 одного шага рекурсии~(\ref{eq:filt_1}) или~(\ref{eq:filt_2}) для вычисления 
 оценки~$\widehat{X}_{t_{1}}$
   или~$\overline{X}_{t_{1}}$ соответственно. Фактически, 
 характеристика~$\overline{\sigma}(\pi)$ определяет, насколько сильно 
 рекурсивные схемы~(\ref{eq:filt_1}) и~(\ref{eq:filt_2}) разойдутся за 
 один шаг, стартуя из общей точки~$\pi$.
 
 Рекуррентные схемы~(\ref{eq:filt_1}) и~(\ref{eq:filt_2}), примененные~$r$~раз, 
 позволяют вычислить оценки~$\widehat{X}_{t_r}$ и~$\overline{X}_{t_r}$ 
 в~точке~$t_r$. В~качестве характеристики точности глобальной аппроксимации в~этом 
 случае естественно рассмотреть величину
 
 \vspace*{-2pt}
 
 \noindent
 \begin{equation*}
 \overline{\Sigma}_{t_r}(\pi) \ebd {\sf E}
 \left\{\|\widehat{X}_{t_{r}} - \overline{X}_{t_{r}}\|_{1}\right\} = 
 \!\sum\limits_{j=1}^N\!{\sf E}
 \left\{\left\vert \widehat{X}^j_{t_{r}} - 
 \overline{X}^{n,j}_{t_{r}}\right\vert \right\}.
% \label{eq:prec_2}
 \end{equation*}
 
 Следующее утверждение определяет оценки локальной и~глобальной 
 точности схемы аппроксимации~(\ref{eq:filt_2}).
 
 %\smallskip
 
 \noindent
 \textbf{Теорема~1.}\
\textit{Выполняются неравенства} 

%\vspace*{-2pt}

\noindent
 \begin{equation}
 \sup_{\pi \in \mathcal{D}_1} \overline{\sigma}(\pi) 
 \leqslant 2 \fr{(\overline{\lambda}\Delta)^{n+1}}{(n+1)!}\,;
 \label{eq:prec_loc}
\end{equation}

\noindent
\begin{align}
  \sup\limits_{\pi \in \mathcal{D}_1} \overline{\Sigma}_{t_r}(\pi)
   &\leqslant 2r \fr{(\overline{\lambda}\Delta)^{n+1}}{(n+1)!} +{}\notag\\[-0.5pt]
   &\hspace*{-20mm}{}+
  r(r-1)\left(
  \fr{(\overline{\lambda}\Delta)^{n+1}}{(n+1)!}
  \right)^2
  \left(
  1-\fr{(\overline{\lambda}\Delta)^{n+1}}{(n+1)!}
  \right)^{r-2},
 \label{eq:prec_glob}
 \end{align}
 
 \vspace*{-2pt}
 
 \noindent
 \textit{где} $\overline{\lambda} \ebd \max_{1 \leqslant j \leqslant N}|\lambda_{jj}|$.


%\smallskip

 Доказательство теоремы~1 приведено в~приложении.
 
 Данное утверждение представляет полезные оценки точности. Во-пер\-вых, 
 они являются равномерными по начальному распределению $\pi \hm\in \mathcal{D}_1$. 
 Во-вто\-рых, оценки носят универсальный, а~не асимптотический характер. Это 
 существенно в~практических задачах оценивания по дискретизованным 
 наблюдениям с~физическими или алгоритмическими ограничениями на шаг 
 по времени. Например, в~случае наблюдаемого процесса восстановления в~силу 
 центральной предельной теоремы для процессов восстановления~\cite{B_80} его
  приращения можно рассматривать как гауссовские случайные величины. 
  Однако данная аппроксимация обладает удовлетворительной точностью 
  только в~случае, когда шаг дискретизации по времени достаточно большой. 
 %
 В-третьих, неравенство~(\ref{eq:prec_glob}) позволяет получить порядок 
 аппроксимации при $\Delta \hm\to 0$. Зафиксируем момент времени $t\hm=T$ и~рассмотрим 
 характеристику $\sup\nolimits_{\pi \in \mathcal{D}_1} 
 \overline{\Sigma}_{T}(\pi)$ при $r\hm={T}/{\Delta}$ и~$\Delta \hm\to 0$. 
 Как только~$\Delta$ становится настолько мало, что 
 $\max\left({(\overline{\lambda}\Delta)^{n+1}}/{(n+1)!}, 
 \Delta ({T\lambda^{n+1}}/{(n+1)!})\right)\hm< 1$, из~(\ref{eq:prec_glob}) 
 следует неравенство
  %\begin{equation}
  $\sup\nolimits_{\pi \in \mathcal{D}_1} \overline{\Sigma}_{T}(\pi) 
  \hm\leqslant  ({3\overline{\lambda}^{n+1}}/{(n+1)!}) T\Delta^n.$
 %\label{eq:prec_asympt}
 %\end{equation}
 Это значит, что с~ростом времени~$T$ 
 ошибка аппроксимации копится пропорционально~$T$ и~при этом порядок точности 
 по~$\Delta$ равен~$n$.
 
 %\vspace*{-7pt}
 
  \section{Заключение}
  
  \vspace*{-4pt}
 
  В работе решена задача оценивания состояния однородного МСП по 
  дискретизованным наблюдениям. Получено аналитическое решение и~его 
  чис\-лен\-ные аппроксимации. Локальные и~глобальные показатели точ\-ности этих 
  приближений в~статье так\-же пред\-став\-ле\-ны. Примечательно, что  част\-ный случай 
  аппроксимаций~(\ref{eq:filt_2}) при $n\hm=0$ и~$\Lambda\hm=0$ был ранее 
  пред\-став\-лен в~\cite{B_17_1,B_17_2} для решения задачи байесовской классификации 
  случайного вектора по непрерывным наблюдениям с~мультипликативными шумами. 
 % 
Алгоритм оптимальной фильт\-ра\-ции и~его субоптимальные версии могут 
рас\-смат\-ри\-вать\-ся в~качестве основы чис\-лен\-ной реализации обобщения фильт\-ра 
Вонэма для сис\-тем с~мультипликативными шумами в~наблюдениях. 
Однако для их непосредственного использования необходимо решить 
следующие проб\-ле\-мы. Во-пер\-вых, в~(\ref{eq:h1}) и~(\ref{eq:h2}) присутствуют
 многомерные интегралы. Следует выяснить, какую результирующую погрешность 
 будут вносить ошибки их вы\-чис\-ле\-ния. Во-вто\-рых, представляется интересным 
 определить характеристики точ\-ности оптимальной фильт\-ра\-ции по дискретизованным 
 наблюдениям по отношению к~оптимальной фильт\-ра\-ции по непрерывным наблюдениям: 
 каков порядок точ\-ности по шагу временной дискретизации~$\Delta$? Для случая 
 вы\-чис\-ле\-ния классического фильт\-ра Вонэма с~по\-мощью алгоритма Эй\-ле\-ра--Ма\-ру\-ямы 
 подобный результат известен: порядок глобальной ошибки равен~${1}/{2}$. 
 Перечисленные задачи являются предметом дальнейших исследований.
 
 
  \vspace*{-10pt}
 
{\small
\subsection*{\raggedleft Приложение} 

\vspace*{-2pt}


\noindent
Д\,о\,к\,а\,з\,а\,т\,е\,л\,ь\,с\,т\,в\,о\ \ теоремы~1.\ \ Введем следующие 
обозначения для случайных величин и~мат\-риц, составленных из них:
\begin{align*}
\xi^{ji}(\ell)&\ebd 
\sum\limits_{h=0}^n \int\limits_{\mathcal{D}} 
 \mathcal{N}\left(Y_{\ell},f u,\sum\limits_{p=1}^N u^p g_pg_p^{\top}\right)
 \rho^{j,i,h}_{1}(du)\,; \\
  \theta^{ji}(\ell)&\ebd 
\sum\limits_{h=n+1}^{\infty} \int\limits_{\mathcal{D}} 
 \mathcal{N}\left(Y_{\ell},f u,\sum\limits_{p=1}^N u^p g_pg_p^{\top}\right)
 \rho^{j,i,h}_{1}(du)\,;
\\
 \xi(\ell)&\ebd \|\xi^{ji}(\ell)\|_{j,i=\overline{1,N}}\,,\quad 
 \Xi(r) \ebd \xi(r) \xi(r-1)\cdots \xi(1)\,;
 \\
 \theta(\ell)&\ebd \|\theta^{ji}(\ell)\|_{j,i=\overline{1,N}}\,, \quad 
 \Theta(r) \ebd \theta(r) \theta(r-1)\cdots \theta(1)\,.
%\label{eq:not_1}
\end{align*}
 
 Рекуррентные формулы~(\ref{eq:filt_1}) и~(\ref{eq:filt_2}) можно записать в~явной 
 форме
 
 
\noindent
\begin{align*}
 \widehat{X}_{t_r}& = \left( \mathbf{1}\left(\Xi(r) + 
 \Theta(r)\right)\pi\right)^{-1} \left(\Xi(r) + \Theta(r)\right)\pi\,;
\\
 \overline{X}_{t_r} &= \left( \mathbf{1}\Xi(r)\pi\right)^{-1} \Xi(r) \pi,
\end{align*}

\vspace*{-2pt}

\noindent
где $\mathbf{1} \ebd (1,\ldots,1)$~--- век\-тор-стро\-ка 
подходящей раз\-мер\-ности, составленная из единиц.

%Далее для краткости записи зависимость от~$r$ в~обозначениях~$\Xi(r)$ 
%и~$\Theta(r)$ будет опущена. 
Верна следующая цепочка неравенств:

 \vspace*{-3pt}

\noindent
\begin{multline}
\overline{\Sigma}_{t_r}(\pi)=%
%\me{}{\left\| 
%\widehat{X}_{t_r}(\pi, Y_1,\ldots,Y_r) - \overline{X}_{t_r}(\pi, Y_1,\ldots,Y_r)
%\right\|_1} =\\=
{\sf E}\left\{\left\| 
\fr{1}{\mathbf{1}\left(\Xi(r) + \Theta(r)\right)\pi} \left(\Xi(r) +{}\right.\right.\right.\\[-1pt]
\left.\left.\left.{}+ \Theta(r)\right)\pi
- \fr{1}{\mathbf{1}\Xi(r)\pi}\,\Xi(r) \pi
\right\|_1\right\} ={} \\[-1pt]
{}=
{\sf E}\left\{\fr{1}{\mathbf{1}\left(\Xi(r) + \Theta(r)\right)\pi \mathbf{1}\Xi(r)\pi}
\left\|
 \mathbf{1}\Xi(r) \pi \Theta(r)\pi -{}\right.\right.\\[-1pt]
\left.\left. {}- \mathbf{1}\Theta(r)\pi \Xi(r) \pi
 \right\|_1
 \vphantom{\fr{1}{\mathbf{1}\left(\Xi(r) + \Theta(r)\right)\pi \mathbf{1}\Xi(r)\pi}}
\right\} \leqslant {}\\[-1pt]
{}\leqslant 
{\sf E}\left\{\fr{1}{\mathbf{1}\left(\Xi(r) + \Theta(r)\right)\pi \mathbf{1}\Xi(r)\pi}
\left(
\mathbf{1}\Xi(r)\pi \| \Theta(r)\pi \|_1 +{}\right.\right.\\[-1pt]
\left.\left.{}+ \mathbf{1}\Theta(r)\pi 
\|
\Xi(r) \pi
\|_1
\right)
 \vphantom{\fr{1}{\mathbf{1}\left(\Xi(r) + \Theta(r)\right)\pi \mathbf{1}\Xi(r)\pi}}
\right\} ={}\\[-1pt]
{}=
2\,{\sf E}\left\{\fr{1}{\mathbf{1}\left(\Xi(r) + \Theta(r)\right)\pi}\mathbf{1}\Theta(r)\pi 
\right\}.
\label{eq:ineq_1}
\end{multline}

 
 \noindent
 Рассмотрим случайные события $a_{\ell} \ebd \{\omega \in \Omega: 
 N_{\ell}(\omega) \hm\leqslant n\}$, $\ell \hm= \overline{1,r}$, и~$A_r \ebd \{
 \omega\hm \in \Omega: \max_{1 \leqslant {\ell} \leqslant r}N_{\ell}(\omega) 
 \hm\leqslant n
 \}\hm=\prod\nolimits_{\ell=1}^r a_{\ell}$ и~оценку 
 $
 \widetilde{X}_{t_r}(\pi, Y_1,\ldots,Y_r)\ebd$\linebreak $\ebd
 {\sf E}\left\{X_{t_r}(\omega)\mathbf{I}_{A_r}(\omega)|\mathcal{O}_r\right\}.
 $
 Используя введенные выше обозначе\-ния и~абстрактный вариант формулы Байеса, 
 получаем, что
 
 \noindent
\begin{align}
\widetilde{X}_{t_r}& = \fr{1}{{\mathbf{1}\left(\Xi(r) + 
 \Theta(r)\right)\pi}}\,\Xi(r)\pi\,;\notag
 \\
\widehat{X}_{t_r} - \widetilde{X}_{t_r} &=
{\sf E}\left\{X_{t_r}(\omega)\mathbf{I}_{\overline{A}_r}(\omega)|\mathcal{O}_r\right\} ={}\notag\\[-1pt]
&\hspace*{17mm}{}= 
\fr{1}{\mathbf{1}\left(\Xi(r) + \Theta(r)\right)\pi}\Theta(r)\pi\,. 
\label{eq:eq_2}
 \end{align}
 Из (\ref{eq:ineq_1}) и~(\ref{eq:eq_2}) для $r\hm=1$ следует, что
 
 \vspace*{-4pt}
 
 \noindent
 \begin{multline}
 \overline{\sigma}(\pi) \leqslant 2\,{\sf E}
 \left\{\|{\sf E}\left\{X_{t_1}(\omega)\mathbf{I}_{\overline{a}_1}(\omega)|\mathcal{O}_1
 \right\}\|_1
 \right\} ={}\\[-1.5pt]
 {}=
 2\,{\sf E}\left\{\sum\limits_{n=1}^N {\sf E}
 \left\{X^n_{t_1}(\omega)\mathbf{I}_{\overline{a}_1}
 (\omega)|\mathcal{O}_1\right\}\right\} ={} \\[-2pt] 
 {}=
  2\,{\sf E}\left\{{\sf E}\left\{\mathbf{I}_{\overline{a}_1}(\omega)|\mathcal{O}_1
  \right\}\right\} =
   2 \mathbf{P}\left\{\overline{a}_1(\omega)\right\}.
\label{eq:ineq_3}
\end{multline}

 \vspace*{-2pt}
 
 \noindent
 Процесс $N^X_t$ общего числа скачков состояния~$X_t$ является считающим, и~его
  квадратическая характеристика равна 
  
\vspace*{-2pt}
  
  \noindent
 $$
 \langle N^X, N^X\rangle_t = - \int\limits_0^t \sum\limits_{n=1}^N \lambda_{nn} X_s^n\,ds\,,
 $$
 поэтому искомая вероятность ограничена сверху:
 $$ 
 \mathbf{P}\left\{\overline{a}_1(\omega)\right\} \leqslant 
 e^{-\overline{\lambda}\Delta}\sum\limits_{k=n+1}^{\infty} 
 \fr{(\overline{\lambda}\Delta)^{k}}{k!} <
 \fr{(\overline{\lambda}\Delta)^{n+1}}{(n+1)!}.
 $$
 
  \vspace*{-2pt}
  
  \noindent
 Из последнего неравенства и~(\ref{eq:ineq_3}) следует, что  для любого 
 начального распределения~$\pi$ выполняется неравенство $\overline{\sigma}(\pi)  
 \hm< 2({(\overline{\lambda}\Delta)^{n+1}}/{(n+1)!})$, т.\,е.\ 
 локальная оценка~(\ref{eq:prec_loc}) верна.
 
 С помощью марковского свойства пары $(X_t, N^X_t)$ и~последнего 
 неравенства можно оценить сверху вероятность 
 $\mathbf{P}\left\{\overline{A}_r(\omega)\right\}$:
 
  \vspace*{-2pt}
 
 \noindent
 \begin{multline*}
 \mathbf{P}\left\{\overline{A}_r(\omega)\right\} \leqslant 1 - \left(
 1- \fr{(\overline{\lambda}\Delta)^{n+1}}{(n+1)!}
 \right)^r \leqslant r \fr{(\overline{\lambda}\Delta)^{n+1}}{(n+1)!} + {}\\[-1pt]
 {}+\left|
 \sum\limits_{k=2}^r C_r^k \left(-\fr{(\overline{\lambda}\Delta)^{n+1}}{(n+1)!}
 \right)^k
 \right| \leqslant
 r \fr{(\overline{\lambda}\Delta)^{n+1}}{(n+1)!} +{}\\[-1pt]
 {}+\fr{r(r-1)}{2}
 \left(
 \fr{(\overline{\lambda}\Delta)^{n+1}}{(n+1)!}
 \right)^2
 \left(
 1-\fr{(\overline{\lambda}\Delta)^{n+1}}{(n+1)!}
 \right)^{r-2},
 \end{multline*} 
 из чего следует истинность глобальной оценки~(\ref{eq:prec_glob}).
Теорема~1 доказана.

}

%\vspace*{-12pt}

{\small\frenchspacing
 {%\baselineskip=10.8pt
 \addcontentsline{toc}{section}{References}
 \begin{thebibliography}{99}

\bibitem{Won_65}
\Au{Wonham W.} 
Some applications of stochastic differential equations to optimal
  nonlinear filtering~//
SIAM~J.~Control, 1965. Vol.~2. P.~347--369. 

\bibitem{KP_92}
\Au{Kloeden P., Platen E.} Numerical solution of stochastic
differential equations.~--- Berlin: Springer, 1992.~636~p.

\bibitem{YZL_04}
\Au{Yin G., Zhang Q., Liu Y.} 
Discrete-time approximation of Wonham filters~//
J.~Control Theory Applications, 2004. Iss.~2. P.~1--10.

\bibitem{PR_10}
\Au{Platen E., Rendek R.}
Quasi-exact approximation of hidden Markov chain filters~//
Communicat.~Stoch.~Analys., 2010. Vol.~4. Iss.~1. P.~129--142.

\bibitem{B_18}
\Au{Борисов А.} Фильтрация Вонэма по наблюдениям с~мультипликативными шумами~// 
Автоматика и~телемеханика, 2018.
№~1. C.~52--65. 
 
  \bibitem{BSh_85} %6
\Au{Бертсекас Д., Шрив С.} Стохастическое оптимальное управление. 
Случай дискретного времени~/ Пер. с~англ.~--- М.: Наука, 1985.~280~c.
(\Au{Betsekas~D.\,P., Shreve~S.\,E.} Stochastic optimal control:
The discrete-time case.~--- Orlando, FL, USA:
Academic Press Inc., 1978. 323~p.)

  \bibitem{ZhSh_95} %7
\Au{Жакод Ж., Ширяев А.} Предельные теоремы для случайных процессов,~I.~/
Пер. с~англ.~--- 
М.: Физматлит, 1995.~544~c.
(\Au{Jacod~J., Shiryaev~A.} Limit theorems for stochastic processes.~---
Berlin: Springer, 2003. 664~p.)

\bibitem{S_00}
\Au{Sericola B.} Occupation times in Markov processes~//
Commun. Stat. Stochastic Models, 2000. Vol.~16. Iss.~5. P.~479--510. 

  \bibitem{B_80}
\Au{Боровков А.} Асимптотические методы в~тео\-рии массового обслуживания.~--- 
М.: Физматлит, 1995.~384~c.

  \bibitem{B_17_1}
\Au{Борисов А.} Классификация по непрерывным наблюдениям с~мультипликативными шумами.~I. 
Формулы байесовской оценки~// Информатика и~её применения, 2017. Т.~11. Вып.~1. C.~11--19.
doi: 10.14357/19922264170102.

  \bibitem{B_17_2}
\Au{Борисов А.} Классификация по непрерывным наблюдениям с~мультипликативными 
шумами.~II. Алгоритм численной реализации оценки~// Информатика и~её 
применения, 2017. Т.~11. Вып.~2. C.~33--41.
doi: 10.14357/19922264170204.

 \end{thebibliography}

 }
 }

\end{multicols}

\vspace*{-4pt}

\hfill{\small\textit{Поступила в~редакцию 10.07.18}}

\vspace*{6pt}

%\pagebreak

%\newpage

%\vspace*{-28pt}

\hrule

\vspace*{2pt}

\hrule

%\vspace*{-2pt}

\def\tit{FILTERING OF~MARKOV JUMP PROCESSES\\ BY~DISCRETIZED OBSERVATIONS}

\def\titkol{Filtering of Markov jump processes by discretized observations}

\def\aut{A.\,V.~Borisov}

\def\autkol{A.\,V.~Borisov}

\titel{\tit}{\aut}{\autkol}{\titkol}

\vspace*{-11pt}


\noindent
Institute of Informatics Problems, Federal Research Center ``Computer Science 
and Control'' of the Russian Academy of Sciences, 44-2~Vavilov Str., Moscow 
119333, Russian Federation


\def\leftfootline{\small{\textbf{\thepage}
\hfill INFORMATIKA I EE PRIMENENIYA~--- INFORMATICS AND
APPLICATIONS\ \ \ 2018\ \ \ volume~12\ \ \ issue\ 3}
}%
 \def\rightfootline{\small{INFORMATIKA I EE PRIMENENIYA~---
INFORMATICS AND APPLICATIONS\ \ \ 2018\ \ \ volume~12\ \ \ issue\ 3
\hfill \textbf{\thepage}}}

\vspace*{6pt}



\Abste{The article is devoted to a~solution of the optimal filtering problem 
of a~homogenous Markov
jump process state. The available observations represent 
time increments of the integral transformations of the Markov\linebreak\vspace*{-12pt}}

\Abstend{state corrupted by 
Wiener processes. The noise intensity is also state-dependent. At the instant of 
the consecutive
observation obtaining, the optimal estimate is calculated recursively 
as a~function of previous estimate and the new observation, meanwhile between 
observations the filtering estimate is a simple forecast by virtue of the Kolmogorov 
differential system. The recursion is rather expensive because of  need to calculate 
the integrals, which are the location-scale mixtures of Gaussians. The mixing 
distributions represent the occupation of the state in each of possible values 
during the mid-observation intervals. The paper contains numerically cheaper 
approximations, based on the restriction of the state transitions number between 
the observations. Both the local and global characteristics of approximation 
accuracy are obtained as functions of the dynamics parameters, mid-observation 
interval length, and upper bound of transitions number.}

\KWE{Markov jump process; optimal filtering; multiplicative observation noises; 
stochastic differential equation; numerical approximation}




\DOI{10.14357/19922264180316}

%\vspace*{-14pt}

\Ack
\noindent
The work was supported in part by the Russian Foundation
for Basic Research (Project No.\,16-07-00677).



%\vspace*{6pt}

  \begin{multicols}{2}

\renewcommand{\bibname}{\protect\rmfamily References}
%\renewcommand{\bibname}{\large\protect\rm References}

{\small\frenchspacing
 {%\baselineskip=10.8pt
 \addcontentsline{toc}{section}{References}
 \begin{thebibliography}{99}
\bibitem{Won_65-1}
\Aue{Wonham, W.} 1965.
Some applications of stochastic differential equations to optimal
  nonlinear filtering.
\textit{SIAM~J.~Control} 2:347--369. 

\bibitem{KP_92-1}
\Aue{Kloeden,~P., and E.~Platen.} 1992. \textit{Numerical solution of stochastic
differential equations.} Berlin: Springer. 636~p.

\bibitem{YZL_04-1}
\Aue{Yin,~G., Q.~Zhang, and Y.~Liu.} 2004.
Discrete-time approximation of Wonham filters.
\textit{J.~Control Theory Applications} 2:1--10.

\bibitem{PR_10-1}
\Aue{Platen, E., and R.~Rendek.} 2010.
Quasi-exact approximation of hidden Markov chain filters.
\textit{Communicat. Stoch. Analys.} 4(1):129--142.

\bibitem{B_18-1}
\Aue{Borisov, A.} 2018. Wonham filtering by observations
with multiplicative noises. \textit{Automat.~Rem.~Contr.} 79(1):39--50.  
doi: 10.1134/ S0005117918010046.
 
  \bibitem{BSh_85-1}
\Aue{Bertsekas, D., and S.~Shreve.} 1996.
\textit{Stochastic optimal control: The discrete-time case}.
Nashua, NH: Athena Scientific. 330~p.
  
  \bibitem{ZhSh_95-1}
  \Aue{Jacod,~J., and A.~Shiryaev.} 2003.
\textit{Limit theorems for stochastic processes.}
Berlin: Springer. 664~p.

\bibitem{S_00-1}
\Aue{Sericola, B.}
2000. Occupation times in Markov processes.
\textit{Commun. Stat.} 16(5):479--510. 

  \bibitem{B_80-1}
\Aue{Borovkov, A.} 1984.
 \textit{Asymptotic methods in queueing theory}. 
 Hoboken, NJ: Wiley-Blackwell.~304~p.

  \bibitem{B_17_1-1}
  \Aue{Borisov, A.} 2017. 
  Klassifikatsiya po ne\-pre\-ryv\-nym nablyu\-de\-miyam s~mul'tiplikativnymi shumami. I. 
  Formuly bayesov\-skoy otsenki [Classification by continuous-time observations
in multiplicative noise. I.~Formulae for Bayesian 
estimate]. \textit{Informatika i~ee Primeneniya~--- Inform.~Appl.}
11(1):11--19. doi: 10.14357/19922264170102.

  \bibitem{B_17_2-1}
\Aue{Borisov, A.} 2017. Klassifikatsiya po nepreryvnym nablyudemiyam 
s~mul'tiplikativnymi summami. II.~Formuly bayesovskoy otsenki 
[Classification by continuous-time observations
in multiplicative noise. II.~Numerical algorithm].
\textit{Informatika i~ee Primeneniya~--- Inform.~Appl.}
11(2):33--41. doi: 10.14357/19922264170204.

\end{thebibliography}

 }
 }

\end{multicols}

\vspace*{-6pt}

\hfill{\small\textit{Received July 10, 2018}}

%\pagebreak

%\vspace*{-18pt}

\Contrl

\noindent
\textbf{Borisov Andrey V.} (b.\ 1965)~--- 
Doctor of Science in physics and mathematics, principal scientist, Institute of
Informatics Problems, Federal Research Center ``Computer Science and Control''
 of the Russian Academy of
Sciences, 44-2 Vavilov Str., Moscow 119333, Russian Federation; 
\mbox{aborisov@frccsc.ru}
\label{end\stat}

\renewcommand{\bibname}{\protect\rm Литература}        %2
\renewcommand{\figurename}{\protect\bf Figure}
\renewcommand{\tablename}{\protect\bf Table}

\def\stat{lange}


\def\tit{ON COMPARATIVE EFFICIENCY OF~CLASSIFICATION SCHEMES IN~AN~ENSEMBLE 
OF~DATA SOURCES USING AVERAGE MUTUAL INFORMATION}

\def\titkol{On comparative efficiency of~classification schemes in~an~ensemble 
of~data sources using average mutual information}

\def\autkol{M.\,M.~Lange}

\def\aut{M.\,M.~Lange$^1$}

\titel{\tit}{\aut}{\autkol}{\titkol}

%{\renewcommand{\thefootnote}{\fnsymbol{footnote}}
%\footnotetext[1] {The study was carried out under state order to the Karelian Research 
%Centre of the Russian Academy of Sciences (Institute of Applied Mathematical 
%Research KarRC RAS) and supported by the Russian Foundation for Basic Research, 
%projects 18-07-00187, 18-07-00147, 18-07-00156, 19-07-00303.}}

\renewcommand{\thefootnote}{\arabic{footnote}}
\footnotetext[1]{Federal Research Center ``Computer Science and Control'' of the Russian Academy of Sciences, 
44-2~Vavilov Str., Moscow 119333, Russian Federation; \mbox{lange\_mm@ccas.ru}}


\index{Lange M.\,M.}
\index{Ланге M.\,M.}


\def\leftfootline{\small{\textbf{\thepage}
\hfill INFORMATIKA I EE PRIMENENIYA~--- INFORMATICS AND
APPLICATIONS\ \ \ 2019\ \ \ volume~13\ \ \ issue\ 4}
}%
 \def\rightfootline{\small{INFORMATIKA I EE PRIMENENIYA~---
INFORMATICS AND APPLICATIONS\ \ \ 2019\ \ \ volume~13\ \ \ issue\ 4
\hfill \textbf{\thepage}}}

%\vspace*{-2pt}





%The research is partially supported by the Russian Foundation for Basic Research 
%(grants Nos.\,18-07-01231 and 18-07-01385).




\Abste{Given ensemble of data sources and different fusion schemes, an accuracy of multiclass 
classification of the collections of the source objects is investigated. Using the average mutual 
information between the datasets of the sources and a~set of the classes, a~new approach to 
comparing lower bounds to an error probability in two fusion schemes is developed. The authors 
consider the WMV (Weighted Majority Vote) scheme which uses a~composition of the class 
decisions on the objects of the individual sources and the GDM (General Dissimilarity Measure) 
scheme based on a~composition of metrics in datasets of the sources.  For the above fusion 
schemes, the mean values of the average mutual information per one source are estimated. It is 
proved that the mean in the WMV scheme is less than the similar mean in the GDM scheme. As a~corollary, the lower bound to the error probability in the WMV scheme exceeds the similar 
bound to the error probability in the GDM scheme. This theoretical result is confirmed by 
experimental error rates in face recognition of HSI color images that yield the 
ensemble of H, S, and~I sources.} 

\KWE{multiclass classification; ensemble of sources; fusion scheme; composition of decisions; 
composition of metrics; average mutual information; error probability
}


\DOI{10.14357/19922264190403} 


%\vspace*{8pt}


\vskip 12pt plus 9pt minus 6pt

 \thispagestyle{myheadings}

 \begin{multicols}{2}

 \label{st\stat}

\section{Introduction }

\noindent
There are plenty of multiclass classification schemes that use input data from an 
ensemble of different modality sources. Such ensemble of data  sources produces 
the composite objects as the collections of the same class objects taken by one per 
each source. An example is the ensemble of biometric images such as faces, finger-
prints, signatures, palms, irises, and the like for a~given set of persons or classes. In 
this case, the composite objects are the collections of the same person images taken 
by one per each modality. In any correct classification scheme that makes the 
decisions on the submitted composite objects, an error probability decreases with 
increasing a~number of the sources~[1]. The decisions can be obtained using the 
different fusion schemes and the principal question is: What scheme is better? 

      The classification problem in the ensemble of sources is similar to the source 
coding problem based on quantization~[2].  There are known scalar and vector 
quantization for the continuous values.  The scalar quantization is used for the
individual values while the vector quantization is used for blocks of the values. In 
both cases, the above quantization schemes yield the code vectors for the 
appropriate blocks of the continuous values. 
   
    It should be noted that the optimal vector quantization is constructed with 
covering a~multidimensional space of the values by general spheres whose shape is 
adjusted to a~given dissimilarity measure between any pair of blocks of the 
values~[3]. In scalar quantization, the same multidimensional space is covered by 
cubes whose edge size is an optimal quantization step for any dimension. Thus, the 
code vectors are represented by the centers of the above spheres or cubes. Since
 for the same volume the spheres are more compact than the cubes, the vector 
quantization yields a~smaller error with respect to the scalar quantization. 

Also, for classification in a~given ensemble of the sources, an error probability is 
waited to be smaller in a~scheme of joint classifying each composite object as 
compared to an error probability in the scheme of combining the decisions on the 
objects of the individual sources. The proposed paper is focused on both developing a~theoretical validity of this idea and supporting it by a~computing experiment.  
    
Two fusion schemes that use the different data compositions for 
making the class decisions on the composite objects in the ensemble of the sources
have been investigated. 
They are the traditional WMV scheme by weighted majority voting the decisions 
on the objects of the individual sources~[4] and the original GDM scheme by 
combining the sources with a~general dissimilarity measure between any pair of the 
composite objects~[5]. Notice that WMV scheme is based on a~composition of 
decisions on the objects of individual sources while GDM scheme uses 
a~composition of metrics in datasets of the sources. Thus, ideologically, WMV 
and GDM fusion schemes are similar to the above scalar and vector quantization.

The specified similarity allows one to expect a~smaller error probability in GDM 
scheme as against WMV scheme. Some limits on the majority vote accuracy have 
been obtained in~[6]. Intuitively, it is clear that the minimal error probability of 
any classifier should depend on the average mutual information~[7] between a~set 
of the source objects and a~set of the classes. Moreover, the more average mutual 
information, the less error probability can be attained. So, our goal is to introduce 
the mutual information-based characteristics for WMV and GDM fusion schemes 
and, using these characteristics, to show an advantage of GDM scheme as against 
WMV scheme in the error probability. 

\section{Formalization of~the~Problem}


\subsection{Basic definitions and classification schemes}

\noindent
Let $\Omega=\{\omega_1, \ldots ,\omega_c\}$, $c\hm\geq 2$, be a~set of classes 
of the prior probabilities ${\sf P}(\omega_i)> 0$: $\sum\nolimits^c_{i=1} 
{\sf P}(\omega_i)=1$, and $\mathbf{X}^M= \mathbf{X}_1\cdots \mathbf{X}_M$ be 
an ensemble of sources, where the set $\mathbf{X}_m =\{\mathbf{x}_m= 
(x_{m1}, \ldots , x_{mN_m})\}$, $m=1,\ldots, M$, of $N_m$-dimensional 
vectors gives the $m$th source objects. In the ensemble , the components of 
any vector $\mathbf{x}_m\in \mathbf{X}_m$ take real values in $(-\infty, \infty)$, 
and any composite object $\mathbf{x}^M=(\mathbf{x}_1, \ldots ,
\mathbf{x}_M)\in \mathbf{X}^M$ is produced by a~collection of the vectors by 
one per source belonging to the same class in~$\Omega$.

In each set~$\mathbf{X}_m$, $m=1,\ldots , M$, a~dissimilarity measure between 
any pair of the objects $\mathbf{x}_m\in \mathbf{X}_m$ and 
$\hat{\mathbf{x}}_m\in \mathbf{X}_m$ is defined by
\begin{equation}
d\left( \mathbf{x}_m, \hat{\mathbf{x}}_m\right) =\sum\limits_{n=1}^{N_m} 
\fr{(x_{mn}-\hat{x}_{mn})^2}{\sigma^2_{mn}}
\label{e1-l}
\end{equation}
where $0<\sigma^2_{mn} <\infty$, $n=1,\ldots , N_m$, are unknown parameters. 
Also, for any pair of the composite objects $\mathbf{x}^M\in \mathbf{X}^M$ and 
$\hat{\mathbf{x}}^M\in \mathbf{X}^M$, let us define a~general dissimilarity 
measure as a~weighted composition of the metrics of the form~(1) taken with the 
weights $W=\{w_m>0,\ m=1,\ldots , M\}$ as follows:
\begin{equation}
D\left( \mathbf{x}^M, \hat{\mathbf{x}}^M\right) =\sum\limits^M_{m=1} w_m
d\left( 
\mathbf{x}_m, \hat{\mathbf{x}}_m\right)\,.
\label{e2-l}
\end{equation}

Let
\begin{equation}
\left\{\mathbf{x}_{im},\ i=1,\ldots ,c \right\} \subset \mathbf{X}_m,\enskip 
m=1,\ldots ,M\,,
\label{e3-l}
\end{equation}
be the subsets of the source template objects that represent the classes by one 
object from~$\mathbf{X}_m$ per each class. The subsets~(3) produce the subset 
of the template composite objects 
\begin{equation}
\left\{ \mathbf{x}_i^M=\left( \mathbf{x}_{i1},\ldots , \mathbf{x}_{iM}\right),\ 
i=1,\ldots ,c\right\} \subset \mathbf{X}^M\,.
\label{e4-l}
\end{equation}
Using the dissimilarity measure~(1) and assuming a~compactness of the objects 
in~$\mathbf{X}_m$, $m=1,\ldots , M$, relative to the corresponding template 
objects in~(3), let us define class-conditional densities of the~$m$th source objects 
as follows:
\begin{equation}
p\left(\mathbf{x}_m\vert \omega_i\right) =\fr{e^{-d(\mathbf{x}_m, 
\mathbf{x}_{im})}} {\int\nolimits_{\mathbf{X}_m}\!\!\! e^{-d(\mathbf{x}_m, 
\mathbf{x}_{im})}d\mathbf{x}_m}\,,\enskip i=1, \ldots , c\,.\!\!
\label{e5-l}
\end{equation}
Also, assuming a~compactness of the composite objects in~$\mathbf{X}^M$ 
relative to the corresponding templates in~(\ref{e4-l}) and using the general 
dissimilarity measure~(2), let us define class-conditional densities of the composite 
objects by
\begin{multline}
p_W\left(\mathbf{x}^M\vert \omega_i\right) \fr{e^{-D\left(\mathbf{x}^M, 
\mathbf{x}_i^M\right)}} {\int\nolimits_{\mathbf{X}^M} e^{-D\left(\mathbf{x}^M, 
\mathbf{x}_i^M\right)} d\mathbf{x}^M}\\
{} = \prod\limits^M_{m=1} \fr{e^{-w_m 
d(\mathbf{x}_m, \mathbf{x}_{im})}} {e^{-w_m d(\mathbf{x}_m, 
\mathbf{x}_{im})} d\mathbf{x}_m}\,,\enskip i=1,\ldots , c\,.
\label{e6-l}
\end{multline}
Under the product in~(\ref{e6-l}), there are the weighted class-conditional 
densities 
\begin{multline}
p_{w_m}\left(\mathbf{x}_m\vert \omega_i\right) =\fr{e^{-w_m d(\mathbf{x}_m, 
\mathbf{x}_{im})}} {\int\nolimits_{\mathbf{X}_m} e^{-w_m d(\mathbf{x}_m, 
\mathbf{x}_{im})} d\mathbf{x}_m}\,,\\
 i=1,\ldots ,c\,,
\label{e7-l}
\end{multline}
that give the densities of the form~(\ref{e5-l}) when $w_m=1$. In terms of 
information theory,  the densities~(\ref{e7-l}) define the $m$th source observation 
channel between input set~$\Omega$ and the output set~$\mathbf{X}_m$ as well 
as the  densities~(\ref{e6-l}) yield the observation multichannel 
between~$\Omega$ and~$\mathbf{X}^M$.

\begin{figure*} %fig1
  \vspace*{1pt}
    \begin{center}  
  \mbox{%
 \epsfxsize=107.799mm 
 \epsfbox{lan-1.eps}
 }
\end{center}
\vspace*{-9pt}
\Caption{Schemes of WMV-based~(\textit{a}) and GDM-based~(\textit{b}) classifiers}
\end{figure*}

Let $g_i^d(\mathbf{x}_m)$, $i=1,\ldots , c$, be the discriminant functions that are 
defined in the sets~$\mathbf{X}_m$, $m=1,\ldots , M$, using the dissimilarity 
measure of the form~(1). Then, WMV-based  class label decision on a~composite 
object $\mathbf{x}^M\in \mathbf{X}^M$ is defined by  
\begin{equation}
j^{\mathrm{WMV}}\left(\mathbf{x}^M\right) =\mathrm{arg}\,\max\limits^c_{i=1} 
\sum\limits^M_{m=1} w_m g_i^d\left(\mathbf{x}_m\right)
\label{e8-l}
\end{equation}
where the discriminant functions are independent on the source weights. Similarly, 
using in the ensemble~$\mathbf{X}^M$ the discriminant 
functions~$g_i^D(\mathbf{x}^M)$, $i=1,\ldots , c$, that depend on the 
weights~$W$ of all sources, GDM-based class label decision on the same 
composite object $\mathbf{x}^M\in \mathbf{X}^M$ is the following:
\begin{equation}
j^{\mathrm{GDM}}\left(\mathbf{x}^M\right) =\mathrm{arg}\,\max^c_{i=1} 
g_i^D\left(\mathbf{x}^M\right)\,.
\label{e9-l}
\end{equation}

        The classification schemes by the decision rules~(\ref{e8-l}) and~(\ref{e9-l}) 
are shown in Fig.~1. Here, $\hat{\Omega}=\Omega$ provided that the decisions 
in~$\hat{\Omega}$ can be differed from the real classes in~$\Omega$. The 
appropriate class-conditional densities yield the observation multichannels in 
WMV and GDM fusion schemes, respectively.




\subsection{Information criterion of efficiency for~the~fusion schemes}

\noindent  
Given the prior distribution $\{ P(\omega_i),\ i=1,\ldots\linebreak \ldots ,c\}$ and the weighted 
class-conditional densities $\{ p_{w_m}(\mathbf{x}_m\vert\omega_i), i=1,\ldots 
,c\}$ of the form~(\ref{e7-l}),the average mutual information 
between~$\mathbf{X}_m$ and~$\Omega$ is defined according to~\cite{7-l} by 
\begin{equation}
I_{w_m}\left(\mathbf{X}_m;\Omega\right) =H_{w_m}\left(\mathbf{X}_m\right) 
-H_{w_m} \left(\mathbf{X}_m\vert\Omega\right)\,.
\label{e10-l}
\end{equation}
Here,
\begin{align*}
H_{w_m}\left(\mathbf{X}_m\right) &=-\int\limits_{\mathbf{X}_m} p_{w_m} 
\left(\mathbf{x}_m\right) \ln p_{w_m} 
\left(\mathbf{x}_m\right)\,d\mathbf{x}_m\,;\\
H_{w_m}\left(\mathbf{X}_m\vert \Omega\right) &\\
&\hspace*{-11mm}{}=-\sum\limits^c_{i=1} 
P\left(\omega_i\right) \int\limits_{ \mathbf{X}_m} 
p_{w_m}\left(\mathbf{x}_m\vert \omega_i\right) \ln 
\left(\mathbf{x}_m\vert\omega_i\right)\,d\mathbf{x}_m
\end{align*}
are the differential entropies, and 
$p_{w_m}(\mathbf{x}_m)\linebreak =\sum\nolimits^c_{i=1} P(\omega_i) 
p_{w_m}(\mathbf{x}_m\vert \omega_i)$ is the marginal density 
in~$\mathbf{X}_m$, $m=1,\ldots ,M$. Notice that the average mutual information 
in~(\ref{e10-l}) does not exceed the entropy $H(\Omega)=-\sum\nolimits^c_{i=1} 
P(\omega_i)\ln P(\omega_i)$ of the set of the classes. For $w_m=1$, there is valid 
$p_{w_m}(\mathbf{x}_m\vert \omega_i)=p(\mathbf{x}_m\vert \omega_i)$  that 
yields $I_{w_m}(\mathbf{X}_m;\Omega)=I(\mathbf{X}_m;\Omega)$. 

Taking the means of the values 
$I\left(\mathbf{X}_m; \Omega\right)$  and $I_{w_m}(\mathbf{X}_m;\Omega)$ 
over all $m=1,\ldots , 
M$, one obtains the efficiency characteristics for WMV-based decision~(\ref{e8-l}) 
and GDM-based decision~(\ref{e9-l}), respectively. These means are defined as 
follows: 
\begin{align}
\hspace*{-2mm}I^{\mathrm{WMV}}_{W\_\mathrm{mean}}\left(\mathbf{X}^M;\Omega\right) &= 
\sum\limits^M_{m=1} I\left(\mathbf{X}_m;\Omega\right) 
\fr{w_m}{\sum\nolimits^M_{m=1} w_m}\,;\!\!\label{e11-l}\\
\hspace*{-2mm}I^{\mathrm{GDM}}_{W\_\mathrm{mean}} \left(\mathbf{X}^M;\Omega\right) &= \fr{1}{M} 
\sum\limits^M_{m=1} I_{w_m} \left(\mathbf{X}_m;\Omega\right)\,.
\label{e12-l}
\end{align}
Our goal is to prove the inequality 
\begin{equation}
\max\limits_W I^{\mathrm{WMV}}_{W\_\mathrm{mean}} \left(\mathbf{X}^M;\Omega\right) \leq 
I^{\mathrm{GDM}}_{W^*\_\mathrm{mean}} \left(\mathbf{X}^M;\Omega\right)
\label{e13-l}
\end{equation}
where~$W^*$ is the set of the source weights providing the maximum in the left 
part. 

\begin{figure*} %fig2
 \vspace*{1pt}
    \begin{center}  
  \mbox{%
 \epsfxsize=154.826mm 
 \epsfbox{lan-2.eps}
 }
\end{center}
\vspace*{-9pt}
\Caption{Sketches of the lower bounds to the average mutual information as the function 
of the error probability in WMV and GDM fusion schemes}
\end{figure*}  


\subsection{Average mutual information and~classification error probability}

\noindent
The criterion of the form~(\ref{e13-l}) assumes a~dependence of the average mutual 
information~$I(\mathbf{X}^M;\hat{\Omega})$ between the 
ensemble~$\mathbf{X}^M$ and the set of the class decisions~$\hat{\Omega}$ on 
a~lower bound to the error probability~$\varepsilon$ in the schemes shown in 
Fig.~1. Given observation multichannel, such function has been defined 
in~\cite{8-l} as a~generalization of the rate-distortion function for the source 
coding model with a~noisy observation channel~\cite{9-l}.  According  
to~\cite{8-l}, this function is lower bounded by 
\begin{multline}
R_L(\varepsilon) =I\left(\mathbf{X}^M;\Omega\right) -h\left(\varepsilon-
\varepsilon_{\min} \right)\\
{} -\left( \varepsilon -\varepsilon_{\min} \right) \ln (c-
1)\,,\enskip \varepsilon_{\min}\leq \varepsilon \leq \varepsilon_{\max}\,.
\label{e14-l}
\end{multline}
Here, $h(z) = -z\ln z -(1-z) \ln (1-z)$; 
$R_L(\varepsilon_{\min})\linebreak =I(\mathbf{X}^M;\Omega)$; 
$R_L(\varepsilon_{\max})= 0$; and $I(\mathbf{X}^M;\Omega) 
=H(\mathbf{X}^M)\linebreak - H(\mathbf{X}^M\vert\Omega)$ is the average mutual 
information between the input and the output of the observation multichannel in 
Fig.~1. Function~(\ref{e14-l}) has the largest value 
$I(\mathbf{X}^M;\Omega)$ at the point $\varepsilon=\varepsilon_{\min}$ and 
decreases as $\varepsilon$~increases. It is not difficult to show that the minimal error 
probability~$\varepsilon_{\min}$ is lower estimated by the conditional entropy 
$H(\Omega\vert \mathbf{X}^M)$ and~$\varepsilon_{\min}$ tends to zero 
when $H(\Omega\vert \mathbf{X}^M)$  decreases by increasing the size~$M$ of 
the ensemble. Taking into account the symmetry of the average mutual information 
\begin{multline*}
I(\mathbf{X}^M;\Omega)=H(\mathbf{X}^M)-H(\mathbf{X}^M\vert\Omega)\\
{}= 
H(\Omega)-H(\Omega\vert \mathbf{X}^M),
\end{multline*}
 in case of $\varepsilon_{\min}\to0$,  
function~(\ref{e14-l}) yields the Shannon bound of the form  
$H(\Omega)-h(\varepsilon) -\varepsilon\ln (c-1)$~\cite{7-l}. 

In the bound~(\ref{e14-l}), the average mutual information 
$I(\mathbf{X}^M;\Omega)$ is calculated in the product 
$\Omega*\mathbf{X}^M$ using the prior probabilities of the classes and the 
class-conditional densities of the form~(\ref{e6-l}). According to Fig.~1,  
the class-conditional densities in GDM scheme depend on the source weights and, 
therefore, $I(\mathbf{X}^M;\Omega)=I_W^{\mathrm{GDM}}(\mathbf{X}^M;\Omega)$ is 
the function of~$W$. In WMV scheme, the corresponding average mutual 
information $I(\mathbf{X}^M;\Omega)=I^{\mathrm{WMV}}(\mathbf{X}^M;\Omega)$ is 
equal to $I_W^{\mathrm{GDM}}(\mathbf{X}^M;\Omega)$ taken with the weights 
$w_m=1$, $m=1,\ldots , M$. The values $I^{\mathrm{WMV}}(\mathbf{X}^M;\Omega)$ 
and $I_W^{\mathrm{GDM}}(\mathbf{X}^M;\Omega)$ correspond to the minimal error 
probabilities~$\varepsilon_{\min}^{\mathrm{WMV}}$ and~$\varepsilon_{\min}^{\mathrm{GDM}}$ 
in WMV and GDM fusion schemes, respectively.  These error probabilities are 
achieved by the Bayes decisions of the form~(\ref{e9-l}) when the discriminant 
functions are given by the posterior probabilities of the classes~\cite{10-l}.


In general, the source sets $\mathbf{X}_1,\ldots, \mathbf{X}_M$ are statistically 
dependent on each other and there are valid the relations 
\begin{gather*}
I^{\mathrm{WMV}}_{W\_\mathrm{mean}} 
(\mathbf{X}^M;\Omega) < I^{\mathrm{WMV}}(\mathbf{X}^M;\Omega);\\[6pt]
I_{W\_\mathrm{mean}}^{\mathrm{GDM}} (\mathbf{X}^M;\Omega) 
< I^{\mathrm{GDM}}_{W} 
(\mathbf{X}^M;\Omega).
\end{gather*}

 Thus, for the weights~$W^*$ giving the maximum 
in~(\ref{e13-l}), the means $I^{\mathrm{WMV}}_{W^*\_\mathrm{mean}} 
(\mathbf{X}^M;\Omega)$ 
and $I^{\mathrm{GDM}}_{W^*\_\mathrm{mean}}(\mathbf{X}^M;\Omega)$ yield the error 
probabilities $\varepsilon^{\mathrm{WMV}}\linebreak
>\varepsilon^{\mathrm{WMV}}_{\min}$ and 
$\varepsilon^{\mathrm{GDM}}>\varepsilon^{\mathrm{GDM}}_{\min}$ that belong to the 
corresponding lower bounds of the form~(\ref{e14-l}). Also, taking into account 
that $I^{\mathrm{WMV}}(\mathbf{X}^M;\Omega)\leq 
I^{\mathrm{GDM}}_{W^*}(\mathbf{X}^M;\Omega)$, the inequality~(\ref{e13-l}) 
provides the following relation: $\varepsilon^{\mathrm{WMV}}
\geq \varepsilon^{\mathrm{GDM}}$. 
This fact is illustrated  in Fig.~2. 

\section{Calculation of~the~Average Mutual Information}

\noindent
In this section, an upper estimate of the functional 
$I_{w_m}(\mathbf{X}_m;\Omega)$ given in~(\ref{e10-l}) is obtained as 
a~function of the variable~$w_m^{1/2}$.  At the value $w q_m^{1/2}=1$, this 
function yields the upper estimate for~$I(\mathbf{X}_m;\Omega)$. Using the 
marginal density $p_{w_m}(\mathbf{x}_m)$  and taking into account that $-\ln z$ 
is the convex downwards function of~$z$, it is valid the Jensen 
inequality~\cite{11-l} as follows:
\begin{multline*}
-\ln p_{w_m} \left(\mathbf{x}_m\right) = -\ln \sum\limits^c_{i=1} P(\omega_i) 
p_{w_m} \left(\mathbf{x}_m\vert   \omega_i\right)\\
{} \leq -\sum\limits^c_{i=1} 
P(\omega_i) \ln p_{w_m}\left(\mathbf{x}_m\vert \omega_i\right)\,.
\end{multline*}
Applying this inequality in~(\ref{e10-l}), one obtains the upper estimated 
differential entropy:
\begin{multline}
H_{w_m}\left(\mathbf{X}_m\right) \leq -\sum\limits^c_{i=1} P(\omega_i) 
\sum\limits^c_{j=1} P(\omega_j)\\
{}\times \int\limits_{\mathbf{X}_m} 
p_{w_m}\left(\mathbf{x}_m\vert\omega_i\right) \ln p_{w_m} 
\left(\mathbf{x}_m\vert \omega_j\right)\,d\mathbf{x}_m\,.
\label{e15-l}
\end{multline}

Given the dissimilarity measures~(\ref{e1-l}) and~(\ref{e2-l}), the conditional 
density $p_{w_m}(\mathbf{x}_m\vert\omega_i)$ of the form~(\ref{e7-l}) is the 
Gaussian density of~$N_m$ independent variables that have the 
means~$x_{imn}$ and the variances $\sigma^2_{imn}/(2w_m)$, $n=1,\ldots , 
N_m$, subject to $w_m>0$. It allows us to express the integral in~(\ref{e15-l}) 
over the interval $(-\infty, +\infty)$  as the Euler integral~\cite{12-l}. The 
calculation yields the upper estimated differential entropy:
\begin{multline}
H_{w_m}(\mathbf{X}_m)\leq \fr{1}{2}\ln 
\fr{\pi}{w_m}+\fr{1}{2}\sum\limits^c_{j=1} P(\omega_j)  
\sum\limits_{n=1}^{N_m} \ln \sigma^2_{jmn}\\
{}+w_m \sum\limits^c_{i=1} P(\omega_i) \sum\limits^c_{j=1} P(\omega_j) 
\sum\limits_{n=1}^{N_m} \fr{(x_{imn}-x_{jmn})^2}{\sigma^2_{jmn}}\\
+2\fr{w_m^{1/2}}{\sqrt{\pi}}\sum\limits^c_{i=1} 
P(\omega_i)\sum\limits^c_{j=1} P(\omega_j) \sum\limits_{n=1}^{N_m} 
\fr{\vert x_{imn}-x_{jmn}\vert \sigma_{imn}}{\sigma^2_{jmn}}\\
+\fr{1}{2}\sum\limits^c_{i=1} P(\omega_i) \sum\limits^c_{j=1} P(\omega_j) 
\sum\limits_{n=1}^{N_m} \fr{\sigma^2_{imn}}{\sigma^2_{jmn}}
\label{e16-l}
\end{multline}
and the following conditional differential entropy:
\begin{multline}
H_{w_m}\left(\mathbf{X}_m\vert\Omega\right) \\
{}=\fr{1}{2}\ln \fr{\pi e}{w_m} 
+\fr{1}{2} \sum\limits^c_{i=1} P(\omega_i) \sum\limits_{n=1}^{N_m} \ln 
\sigma^2_{imn}\,.
\label{e17-l}
\end{multline}
The substitutions of the differential entropy and the conditional differential entropy 
in~(\ref{e10-l}) by~(\ref{e16-l}) and~(\ref{e17-l}) yield the upper 
estimated average mutual information:

\noindent
\begin{multline}
I_{w_m}\left(\mathbf{X}_m;\Omega\right) \\
{}\leq w_m \sum\limits^c_{i=1} 
P(\omega_i) \sum\limits^c_{j=1} P(\omega_j) \sum\limits_{n=1}^{N_m} 
\fr{(x_{imn}-x_{jmn})^2}{\sigma^2_{jmn}}\\
+2\fr{w_m^{1/2}}{\sqrt{\pi}} \sum\limits^c_{i=1} P(\omega_i) 
\sum\limits^c_{j=1} P(\omega_j) \sum\limits_{n=1}^{N_m} \fr{\vert x_{imn} -
x_{jmn})^2}{\sigma^2_{jmn}}\\
+\fr{1}{2}\sum\limits^c_{i=1}P(\omega_i) \sum\limits_{j=1}^c P(\omega_j) 
\sum\limits^{N_m}_{n=1} \left( \fr{\sigma^2_{imn}} {\sigma^2_{jmn}}-
1\right)\,.
\label{e18-l}
\end{multline}
The right part in~(\ref{e18-l}) is a~parabolic function 
$a_mw_m\linebreak +b_mw_m^{1/2}+c_m$ of the variable $w_m^{1/2}>0$ for 
$m\linebreak =1,\ldots , M$. Since $a_m>0$, $b_m>0$, and $c_m\geq0$, the parabola 
exceeds the value~$c_m$ and grows when~$w_m^{1/2}$ increases.  For 
$w_m^{1/2}=1$, this function gives the upper estimate $a_m+b_m+c_m$ for 
$I(\mathbf{X}_m;\Omega)$. The weights of interest are defined by the values 
$w_m^{1/2}\geq 1$ that satisfy the condition 
$a_mw_m+b_mw_m^{1/2}+c_m\leq H(\Omega)$, $m=1, \ldots , M$. Setting 
$\delta_m=(a_m+b_m+c_m)/H(\Omega)\leq 1$, we assign the parametric source 
weights 
\begin{equation}
w_m(s)=e^{s\delta_m}\,,\enskip m=1,\ldots , M,
\label{e19-l}
\end{equation}
where $s\geq 0$ is a~free parameter that yields $w_m(s)\geq 1$. In what follows, 
we denote the upper estimates~(\ref{e18-l}) taken with the weights~(\ref{e19-l}) 
by $I_s(\mathbf{X}_m;\Omega)$, $m=1,\ldots , M$.

\section{Main Results}

\noindent
   Using in the right part of the form~(\ref{e11-l}) the estimates 
$I(\mathbf{X}_m;\Omega)\leq a_m+b_m+c_m$, $m=1,\ldots , M$, taken with the 
weights~(\ref{e19-l}), one obtains the upper estimated mean value 
$I^{\mathrm{WMV}}_{s\_\mathrm{mean}} (\mathbf{X}^M;\Omega)$.  Also, the estimates 
$I_{w_m}(\mathbf{X}_m;\Omega)\linebreak \leq a_mw_m+b_m w_m^{1/2}+c_m$, 
$m=1,\ldots , M$, taken with the similar weights in the right part of~(\ref{e12-l}) 
yield the upper estimated mean value $I^{\mathrm{GDM}}_{s\_\mathrm{mean}} 
(\mathbf{X}^M;\Omega)$. Then, for $s\to 0$, we calculate an asymptotic 
maximum $I^{\mathrm{WMV}}_{s^*\_\mathrm{mean}}(\mathbf{X}^M;\Omega)$ at the point~$s^*$ 
and show that this maximum satisfies the inequality 
$I^{\mathrm{WMV}}_{s^*\_\mathrm{mean}}(\mathbf{X}^M;\Omega) \leq 
I^{\mathrm{GDM}}_{s^*\_\mathrm{mean}}(\mathbf{X}^M;\Omega)$.

In subsequent statements, we use the following notations:  
\begin{gather*}
\mu=\fr{1}{M}\sum\limits_{m=1}^M \delta_m\,;\quad 
\Delta_1=\fr{1}{M}\sum\limits^M_{m=1} \delta^2_m-\mu^2\,;\\
\Delta_2=\fr{1}{M}\sum\limits^M_{m=1} \delta_m^3-
\mu\fr{1}{M}\sum\limits^M_{m=1} \delta^2_m\,.
\end{gather*}

\noindent
\textbf{Theorem~1.}\ \textit{For $(2\mu\Delta_1-\Delta_2)>\Delta_1 >0$ and 
$s\to 0$, the value $s^*=\Delta_1/(2\mu \Delta_1-\Delta_2)$ yields}
$$
\max\limits_s I^{\mathrm{WMV}}_{s\_\mathrm{mean}} \left(\mathbf{X}^M;\Omega\right) =
\left( 
\mu+\fr{1}{2}\,\Delta_1 s^*\right) H(\Omega)\,.
$$
\textit{For $\Delta_1=0$, there is valid $I^{\mathrm{WMV}}_{s\_\mathrm{mean}} 
(\mathbf{X}^M;\Omega) =\mu H(\Omega)$ for all $s\geq 0$}.

\smallskip

\noindent
P\,r\,o\,o\,f\,.\ \  Using $q_s(\delta_m) =e^{s\delta_m}/\sum\nolimits^M_{m=1} 
e^{s\delta_m}$,  the upper estimated mean value defined in~(\ref{e11-l}) takes the 
form: 
\begin{equation}
I^{\mathrm{WMV}}_{s\_\mathrm{mean}} (\mathbf{X}^M;\Omega) =H(\Omega) 
\sum\limits^M_{m=1} \delta_m q_s(\delta_m)\,.
\label{e20-l}
\end{equation}
For $s\to 0$, there is valid the asymptotic equation: 
\begin{equation}
\sum\limits^M_{m=1} \delta_m q_s(\delta_m) \approx \mu+ \Delta_1 s-
\fr{1}{2}\left( 2\mu \Delta_1-\Delta_2\right) s^2\,.
\label{e21-l}
\end{equation}
Using the assumption of the theorem, the parabola in the right part of~(\ref{e21-l}) 
takes the maximal value $\mu+\Delta_1 s^*/2$ at the point $s^*=\Delta_1/(2\mu 
\Delta_1-\Delta_2)$. Notice that the same values $\delta_m=\delta$, $m=1,\ldots , 
M$, provide $\Delta_1=0$ and $\Delta_2=0$. In this case, $q_s(\delta_m)=1/M$ 
and the sum in~(\ref{e21-l}) is equal to $\mu=\delta$ for all $s\geq0$. Thus, the 
substitution of the sum in~(\ref{e20-l}) by $\mu+\Delta_1 s^*/2$ in case of 
$\Delta_1>0$ or by~$\mu$ in case of $\Delta_1=0$ completes the proof.


\smallskip

\noindent
\textbf{Theorem~2.}\ \textit{For $\Delta_1>0$ and on condition that $a_m\geq 
c_m$, $m=1,\ldots , M$, there is valid the inequality $I^{\mathrm{WMV}}_{s^*\_\mathrm{mean}} 
(\mathbf{X}^M;\Omega)<I^{\mathrm{GDM}}_{s^*\_\mathrm{mean}} (\mathbf{X}^M;\Omega)$ at 
the optimal point $s^*>0$.  For $\Delta_1=0$ and a~given $s\geq 0$, there is valid 
the inequality $I^{\mathrm{WMV}}_{s\_\mathrm{mean}} (\mathbf{X}^M;\Omega) \leq 
I^{\mathrm{GDM}}_{s\_\mathrm{mean}} (\mathbf{X}^M;\Omega)$ which passes into the equality at 
the point $s=0$.}

\smallskip\

\noindent
P\,r\,o\,o\,f\,.\ The estimates~(\ref{e18-l}) taken with the weights~(\ref{e19-l}) 
give the upper estimated mean value~(\ref{e12-l}) as follows:
\begin{multline}
I^{\mathrm{GDM}}_{s\_\mathrm{mean}} \left(\mathbf{X}^M;\Omega\right)\\ 
{}=\fr{1}{M}\sum\limits^M_{m=1} \left( a_m e^{s\delta_m} +b_m 
s^{s\delta_m/2} +c_m\right)\,.
\label{e22-l}
\end{multline}
Taking the square approximations of the exponential terms in~(\ref{e22-l}), one 
obtains the following inequality: 
\begin{multline}
I^{\mathrm{GDM}}_{s\_\mathrm{mean}} \left(\mathbf{X}^M;\Omega\right) \geq \mu H(\Omega) \\
{}+\left( \fr{1}{M} \sum\limits^M_{m=1} 
a_m\delta_m+\fr{1}{2M}\sum\limits^M_{m=1} b_m\delta_m\right) s\\
{}+ \left( \fr{1}{2M}\sum\limits^M_{m=1} 
a_m\delta_m^2+\fr{1}{4M}\sum\limits^M_{m=1} b_m\delta_m^2\right) s^2\,.
\label{e23-l}
\end{multline}
In case of $\Delta_1>0$, the inequality~(\ref{e23-l}) together with the 
estimates~(\ref{e20-l}) and~(\ref{e21-l}) yield: 
\begin{multline}
I^{\mathrm{GDM}}_{s\_\mathrm{mean}}\left(\mathbf{X}^M;\Omega\right)-
I^{\mathrm{WMV}}_{s\_\mathrm{mean}} 
\left(\mathbf{X}^M;\Omega\right)\\
\geq \left( \fr{1}{M} \sum\limits^M_{m=1}a_m\delta_m 
+\fr{1}{2M}\sum\limits^M_{m=1} b_m\delta_m-\Delta_1H(\Omega)\right)s\\
{}+\left( \fr{1}{2M}\sum\limits^M_{m=1} 
a_m\delta_m^2+\fr{1}{4M}\sum\limits^M_{m=1} 
b_m\delta_m^2\right.\\
\left.{}+\fr{1}{2}\left( 2\mu \Delta_1-\Delta_2\right) 
H(\Omega)
\vphantom{\sum\limits^M_{m=1}}
\right)s^2\,.
\label{e24-l}
\end{multline}
Assuming 
\begin{equation}
\fr{1}{M}\sum\limits^M_{m=1} a_m\delta_m+\fr{1}{2M} 
\sum\limits^M_{m=1} b_m\delta_m\geq \fr{1}{2}\Delta_1 H(\Omega),
\label{e25-l}
\end{equation}
the right part in~(\ref{e24-l}) is lower estimated by the parabola
\begin{multline*}
-\fr{1}{2}\,\Delta_1H(\Omega) s+\left( \fr{1}{2M}\sum\limits^M_{m=1} 
a_m\delta_m^2+\fr{1}{4M} \sum\limits^M_{m=1} b_m 
\delta_m^2\right.\\
\left.{}+\fr{1}{2}\left( 2\mu \Delta_1-\Delta_2\right) H(\Omega) 
\vphantom{\sum\limits^M_{m=1}}
\right) s^2
\end{multline*}
that has a~positive root
\begin{multline*}
s_0=
\Delta_1H(\Omega)\Bigg/
\left(
\vphantom{\sum\limits^M_{m=1}}
(2\mu\Delta_1-\Delta_2)H(\Omega)\right.\\
\left.{} +\fr{1}{M} 
\sum\limits^M_{m=1} a_m\delta_m^2+\fr{1}{2M} \sum\limits^M_{m=1} 
b_m\delta_m^2\right)\\
{}<\fr{\Delta_1}{2\mu\Delta_1-\Delta_2}=s^*\,.
\end{multline*}
Since this parabola is positive for $s>s_0$,  the lower estimate of the right part 
in~(\ref{e24-l}) is positive at the point $s^*>0$ of the maximal value 
$I^{\mathrm{WMV}}_{s^*\_\mathrm{mean}} (\mathbf{X}^M;\Omega)$ that provides the inequality 
$I^{\mathrm{GDM}}_{s^*\_\mathrm{mean}} (\mathbf{X}^M;\Omega)- I^{\mathrm{WMV}}_{s^*\_\mathrm{mean}} 
(\mathbf{X}^M;\Omega) >0$. 

Notice that the assumption of the form~(\ref{e25-l}) is equivalent to the inequality 
$$
\fr{1}{M}\sum\limits^M_{m=1} \left( c_m-a_m\right) \delta_m \leq \mu^2 
H(\Omega)
$$
that is valid under the conditions $a_m\geq c_m$, $m\linebreak =1,\ldots , M$. These 
conditions are held if the templates in different classes are sufficiently distinct 
from each other. Formally, the parameters in~(\ref{e18-l}) should satisfy the 
following relation:
\begin{multline*}
\left( x_{imn}-x_{jmn}\right)^2\geq \fr{1}{2}\left\vert \sigma^2_{imn} -
\sigma^2_{jmn}\right\vert \,,\\
 m=1,\ldots , M\,,\enskip n=1,\ldots , N_m\,.
\end{multline*}
In case of $\Delta_1=0$, one has $I^{\mathrm{WMV}}_{s\_\mathrm{mean}} 
(\mathbf{X}^M;\Omega) =\mu H(\Omega)$ and 
$I^{\mathrm{GDM}}_{s\_\mathrm{mean}}(\mathbf{X}^M;\Omega)\geq \mu H(\Omega)$ for a~given 
$s\geq 0$. So, there is valid the inequality $I^{\mathrm{WMV}}_{s\_\mathrm{mean}} 
(\mathbf{X}^M;\Omega)\linebreak \leq I^{\mathrm{GDM}}_{s\_\mathrm{mean}} (\mathbf{X}^M;\Omega)$ 
which passes into the equality at the point $s=0$. The theorem is proved.

\smallskip


Sketches of the graphics in Fig.~3 interpret the theorems~1 and~2.


\begin{figure*} %fig3
 \vspace*{1pt}
    \begin{center}  
  \mbox{%
 \epsfxsize=162.134mm 
 \epsfbox{lan-3.eps}
 }
\end{center}
\vspace*{-9pt}
\Caption{Graphical interpretation of the results for cases of $\Delta_1>0$~(\textit{a}) and 
$\Delta_1=0$~(\textit{b})}
\end{figure*}

\begin{figure*}[b] %fig4
 \vspace*{1pt}
    \begin{center}  
  \mbox{%
 \epsfxsize=163mm 
 \epsfbox{lan-4.eps}
 }
\end{center}
\vspace*{-9pt}
\Caption{Examples of the 8th level representations for the face HSI images}
\end{figure*}
  



\noindent
\textbf{Corollary.}\ For the optimal value~$s^*$ in the case of $\Delta_1>0$ and any 
$s>0$ in the case of $\Delta_1=0$, the mean values of the average mutual information 
per one source in WMV and GDM fusion schemes provide the lower bounds to the 
error probabilities satisfying the inequality 
$\varepsilon^{\mathrm{WMV}}>\varepsilon^{\mathrm{GDM}}$.

\section{Experimental Results}

\noindent
The efficiency of WMV and GDM fusion schemes is shown by comparative error 
rates for face recognition of HSI color images.The components H, S, and~I produce the 
objects of the individual sources and the ensemble HSI produces the composite 
objects. The color images are taken from~25~persons (classes) per 40~images in 
each class~\cite{13-l}.  The prior probability distribution of the classes is uniform. 
Face recognition has been performed in a~space of multilevel tree-structured 
pattern representations with elliptic primitives~\cite{5-l}. The error rates have 
been obtained for multiclass NN (nearest neighbor) and SVM (support vector 
machine) classifiers that are the collections of elementary ``class-vs-all'' classifiers. 
The experiments have been performed using 100~times, 2~fold cross validation. 

The examples of the tree-structured representations for the face components H, S, and I 
are shown in Fig.~4. The image components correspond to the source numbers 
$m=1, 2, 3$.


Using the above representations, the dissimilarity measure 
$d(\mathbf{x}_m, \hat{\mathbf{x}}_m)\geq 0$  for any pair 
of the objects~$\mathbf{x}_m$ and~$\hat{\mathbf{x}}_m$ has been introduced 
in~\cite{14-l}.
The weighted sum of the above measures taken over the components 
H, S, and I yields the general dissimilarity measure $D(\mathbf{x}^3, 
\hat{\mathbf{x}}^3)$  of the form~(\ref{e2-l}) between the corresponding 
composite objects~$\mathbf{x}^3$ and~$\hat{\mathbf{x}}^3$. 

The dissimilarity 
measures $d(\mathbf{x}_m, \hat{\mathbf{x}}_m)$, $m=1,2,3$, and 
$D(\mathbf{x}^3, \hat{\mathbf{x}}^3)$ have allowed us to construct the 
discriminant functions~$g_i^d(\mathbf{x}_m)$ and~$g_i^D(\mathbf{x}^3)$, 
$i=1, \ldots$\linebreak $\ldots , c$, for making the decisions of the form~(\ref{e8-l}) and~(\ref{e9-l}) 
by the appropriate NN and SVM classifiers.
{\looseness=1

}

\begin{table*}\small
\begin{center}
\tabcolsep=8pt
\begin{tabular}{cccccc}
\multicolumn{6}{c}{Error rates for HSI face recognition by NN and SVM classifiers}\\
\multicolumn{6}{c}{\ }\\[-6pt]
\hline
\multicolumn{1}{c}{\raisebox{-6pt}[0pt][0pt]{Classifier}}&
\multicolumn{3}{c}{Sources} &\multicolumn{2}{c}{Fusion schemes}\\ 
\cline{2-6} 
&H&S&I&\hspace*{2mm}WMV&GDM\\ 
\hline 
NN&0.022&0.017&0.015&\hspace*{2mm}0.009&0.006\\ 
SVM&0.019&0.012&0.011&\hspace*{2mm}0.007&0.003\\ 
\hline 
\end{tabular} 
\end{center} 
\vspace*{-12pt}
\end{table*}
The table summarizes the cross-validation error rates for both the individual sources 
and their ensemble using GDM and WMV fusion schemes. The experimental 
results demonstrate a~decrease of the error rates in the ensemble HSI as against the 
error rates for the sources H, S, and~I. Also, the obtained error rates confirm 
some advantage of GDM scheme as compared with the WMV scheme.
 
\vspace*{-9pt}

\section{Concluding Remarks}

\noindent
To compare the potentially achievable  classification error probabilities for two 
fusion schemes in the ensemble of data sources, the information-based criterion 
has been suggested. The proposed  criterion is based on comparing the mean 
values of the average mutual information between the set of the classes and the 
datasets of the sources. These means  are independent on a~decision algorithm 
and  they are defined in the WMV scheme of fusion of the decisions on the source 
objects and in the GDM scheme of fusion of the metrics in datasets of the sources. 
Taking the above mean values as the points of the appropriate rate distortion 
functions, it has been shown that the  lower bound to GDM-based error probability is 
smaller as compared with the similar WMV-based error probability. The advantage 
in accuracy of the GDM scheme relative to the WMV scheme is confirmed by the error 
rates for NN and SVM decision algorithms in experiments on recognition of HSI 
face images given by the ensemble of the sources Н, S, and I.
In future, we plan to 
extend the ensemble of biometric sources and the set of the decision algorithms.  
For the above fusion schemes and the different decision algorithms, we plan 
to estimate a~redundancy of the error rates relative to the appropriate lower 
bounds. 

\vspace*{-9pt}

\Ack
\noindent
The research is partially supported by the Russian Foundation for Basic Research 
(grants Nos.\,18-07-01231 and 18-07-01385).

\renewcommand{\bibname}{\protect\rmfamily References}


\vspace*{-9pt}

{\small\frenchspacing
{\baselineskip=10.45pt
\begin{thebibliography}{99}
\bibitem{1-l}
\Aue{Kuncheva, L.} 2014. \textit{Combining pattern classifiers, methods and algorithms}. 2nd ed. 
New York, NY: John Wiley and Sons. 384~p.
\bibitem{2-l}
\Aue{Gray, R., and D.~Neuhoff.} 1998. Quantization. 
\textit{IEEE T.~Inform. Theory} 44(6):2325--2383.
\bibitem{3-l}
\Aue{Kolmogorov, A.\,N., and V.\,M.~Tikhomirov.} 1961. 
\mbox{$\varepsilon$-entropy} and  $\varepsilon$-capacity of sets in 
functional spaces. \textit{AMS Transl.} 17(2):277--364.
\bibitem{4-l}
\Aue{Lam, L., and C.~Suen.} 1997. 
Application of majority voting to pattern recognition: An 
analysis of its behavior and performance. \textit{IEEE T.~Syst. Man. Cyb.}
27(5):553--568.
\bibitem{5-l}
\Aue{Lange, M.\,M., and D.\,Y.~Stepanov.} 
2014. Recognition of objects given by collections of 
multichannel images. \textit{Pattern Recogn. Image Anal.} 24(3):431--442.
\bibitem{6-l}
\Aue{Kuncheva, L., C.~Whitaker, C.~Shipp, and R.~Duin.} 2003. Limits on the majority
vote accuracy in classifier fusion. \textit{Pattern Anal. Appl.} 6(1):22--31.
\bibitem{7-l}
\Aue{Gallager, R.} 1968. 
\textit{Information theory and reliable communication}. New York, NY: John Wiley and 
Sons. 608~p.
\bibitem{8-l}
\Aue{Lange, M.\,M., and A.\,M.~Lange.} 2018. 
O~teoretiko-informatsionnoy modeli klassifikatsii
dannykh [On information theoretical model for data classification]. 
\textit{Mashinnoe obuchenie i~analiz dannykh}  [J.~Machine Learning Data Analysis] 
4(3):165--179.
\bibitem{9-l}
\Aue{Dobrushin, R.\,L., and B.\,S.~Tsybakov.} 1962. 
Information transmission with additional noise. 
\textit{IRE T.~Inform. Theor.} 8(5):293--304.
\bibitem{10-l}
\Aue{Duda, R., P. Hart, and D.~Stork.}
 2001. \textit{Pattern classification}. 2nd ed. New York, NY: John Wiley and Sons. 
688~p.
\bibitem{11-l}
\Aue{Beckenbach, E., and R.~Bellman.} 1961. 
\textit{Inequalities}. New York, NY: Springer-Verlag. 55~p.
\bibitem{12-l}
\Aue{Gradshteyn, I.\,S., and I.\,M.~Ryzhik.}
 2007. \textit{Table of integrals, series, and products}. 7th ed. 
Academic Press. 1221~p.
\bibitem{13-l}
Database of face images. Available at:
{\sf http://\linebreak sourceforge.net/projects/colorfaces} (accessed 
October~9, 2019).
\bibitem{14-l}
\Aue{Lange, M.\,M., and S.\,N.~Ganebnykh.} 
2018. On fusion schemes for multiclass object 
classification with reject in a~given ensemble of sources. 
\textit{J.~Phys. Conf. Ser.} 1096:012048. 12~p. Available at: 
{\sf https://\linebreak iopscience.iop.org/article/10.1088/1742-6596/1096/1/ 012048}
 (accessed October~7,  2019).
 \end{thebibliography} } }

\end{multicols}

\vspace*{-9pt}

\hfill{\small\textit{Received July 01, 2019}}

\vspace*{-16pt}

\Contrl

\vspace*{-3pt}

\noindent
\textbf{Lange Mikhail M.} (b.\ 1945)~--- Candidate of Science (PhD) in technology, leading 
scientist, Federal Research Center ``Computer Sciences and Control'' of the Russian Academy of 
Sciences, 44-2~Vavilov Str., Moscow 119333, Russian Federation; 
\mbox{lange\_mm@ccas.ru}

 

\newpage

%\vspace*{8pt}

%\hrule

%\vspace*{2pt}

%\hrule

%\vspace*{-7pt}

%\newpage

\vspace*{-28pt}

\def\tit{О СРАВНИТЕЛЬНОЙ ЭФФЕКТИВНОСТИ СХЕМ КЛАССИФИКАЦИИ ДАННЫХ НА~АНСАМБЛЕ 
ИСТОЧНИКОВ С~ИСПОЛЬЗОВАНИЕМ СРЕДНЕЙ ВЗАИМНОЙ ИНФОРМАЦИИ$^*$}

\def\titkol{О сравнительной эффективности схем классификации данных на~ансамбле 
источников} % с~использованием средней взаимной информации}

\def\aut{M.\,M.~Ланге}

\def\autkol{M.\,M.~Ланге}

{\renewcommand{\thefootnote}{\fnsymbol{footnote}} \footnotetext[1]
{Работа частично поддержана РФФИ (проекты 18-07-01231 и 18-07-01385).}}



\titel{\tit}{\aut}{\autkol}{\titkol}

\vspace*{-11pt}

\noindent
Федеральный исследовательский центр <<Информатика и управление>> Российской академии наук, 
\mbox{lange\_mm@ccas.ru}

\vspace*{1pt}

\def\leftfootline{\small{\textbf{\thepage}
\hfill ИНФОРМАТИКА И ЕЁ ПРИМЕНЕНИЯ\ \ \ том\ 13\ \ \ выпуск\ 4\ \ \ 2019}
}%
 \def\rightfootline{\small{ИНФОРМАТИКА И ЕЁ ПРИМЕНЕНИЯ\ \ \ том\ 13\ \ \ выпуск\ 4\ \ \ 2019
\hfill \textbf{\thepage}}}

\vspace*{-1pt}




\Abst{Исследуется точность многоклассовой классификации наборов объектов от 
ансамбля источников при различных схемах комплексирования данных. Предлагается 
новый подход к~сравнению нижних границ вероятности ошибки для двух схем 
классификации с~использованием средней взаимной информации между данными 
источников и множеством классов. Рассмотрена схема WMV (Weighted Majority Vote) на 
основе композиции решений по объектам источников и~схема GDM (General Dissimilarity 
Measure) на основе композиции метрик на множествах объектов источников. Для 
исследуемых схем получены оценки усредненных значений средней взаимной 
информации на один источник. Доказано, что указанная характеристика схемы WMV не 
превосходит аналогичной характеристики схемы GDM, при этом нижняя граница 
вероятности ошибки в~схеме WMV превосходит нижнюю границу вероятности ошибки 
в~схеме GDM. Полученный теоретический результат подтвержден экспериментальными 
оценками вероятности ошибки распознавания цветных HSI изображений лиц для двух 
схем комплексирования данных от источников H, S и~I.} 

\KW{многоклассовая классификация; ансамбль источников; схема комплексирования; 
композиция решений; композиция метрик; средняя взаимная информация; вероятность 
ошибки}

\DOI{10.14357/19922264190403} 



%\vspace*{-3pt}


 \begin{multicols}{2}

\renewcommand{\bibname}{\protect\rmfamily Литература}
%\renewcommand{\bibname}{\large\protect\rm References}

{\small\frenchspacing
{\baselineskip=10.5pt
\begin{thebibliography}{99}
%\vspace*{-3pt}
\bibitem{1-l-1}
\Au{Kuncheva L.} Combining pattern classifiers, methods and algorithms.~--- 2nd ed.~---
  New York, NY, USA: John Wiley and Sons, 2014. 384~p.
\bibitem{2-l-1}
\Au{Gray R., Neuhoff~D.} Quantization~// IEEE T. Inform. Theory, 1998. 
Vol.~44. Iss.~6. P.~2325--2383.
\bibitem{3-l-1}
\Au{Колмогоров А.\,Н.,  Тихомиров~В.\,М.} 
$\varepsilon$-энтропия и~$\varepsilon$-ем\-кость 
множеств в функциональных пространствах~// УМН, 1959. Т.~14. №\,2(86). С.~3--86.

\bibitem{4-l-1}
\Au{Lam L., Suen~C.} Application of majority voting to pattern recognition: An analysis of its behavior and 
performance~// IEEE T. Syst. Man Cyb., 1997. Vol.~27. Iss.~5. P.~553--568.
\bibitem{5-l-1}
\Au{Lange M.\,M., Stepanov~D.\,Y.}
 Recognition of objects given by collections of multichannel images~// 
Pattern Recogn. Image Anal., 2014. Vol.~24. Iss.~3. P.~431--442.
\bibitem{6-l-1}
\Au{Kuncheva L., Whitaker~C., Shipp~C., Duin~R.}
 Limits on the majority vote accuracy in classifier fusion~// 
Pattern Anal. Appl., 2003. Vol.~6. Iss.~1. P.~22--31.
\bibitem{7-l-1}
\Au{Gallager R.} Information theory and reliable communication.~---
  New York, NY, USA: John Wiley and Sons, 1968. 
608~p.
\bibitem{8-l-1}
\Au{Ланге М.\,М., Ланге~А.\,М.} О~тео\-ре\-ти\-ко-ин\-фор\-ма\-ци\-он\-ной 
модели классификации данных~// 
Машинное обучение и анализ данных, 2018. Т.~4. Вып.~3. С.~165--179.
\bibitem{9-l-1}
\Au{Dobrushin R.\,L., Tsybakov~B.\,S.}
 Information transmission with additional noise~// IRE T. 
Inform. Theor., 1962. Vol.~8. Iss.~5. P.~293--304.
\bibitem{10-l-1}
\Au{Duda R., Hart~P., Stork~D.}
 Pattern classification.~--- 2nd ed.~--- New York, NY, USA: John Wiley and Sons, 2001. 688~p.
\bibitem{11-l-1}
\Au{Beckenbach E., Bellman~R.} Inequalities.~--- New York, NY, USA: Springer-Verlag, 1961. 55~p.
\bibitem{12-l-1}
\Au{Gradshteyn I.\,S., Ryzhik~I.\,M.} Table of integrals, series, and products.~---
7th ed.~--- Academic Press, 
2007. 1221~p.
\bibitem{13-l-1}
Database of face images. {\sf http://sourceforge.net/\linebreak projects/colorfaces}.
\bibitem{14-l-1}
\Au{Lange M.\,M., Ganebnykh~S.\,N.} 
On fusion schemes for multiclass object classification with reject in 
a~given ensemble of sources~// J.~Phys. Conf. Ser., 2018. Vol.~1096.
 Art. ID: 012048.  P.~1--12. 
\end{thebibliography}
} }

\end{multicols}

 \label{end\stat}

 \vspace*{-9pt}

\hfill{\small\textit{Поступила в~редакцию 01.07.2019}}


%\renewcommand{\bibname}{\protect\rm Литература}
\renewcommand{\figurename}{\protect\bf Рис.}
\renewcommand{\tablename}{\protect\bf Таблица}



 
 
%Ланге Михаил Михайлович (р.\ 1945)~--- кандидат технических наук, ведущий научный 
%сотрудник Федерального исследовательского центра <<Информатика и управление>> 
%Российской академии наук

 
 
 
  %3
\def\stat{kochetkova}

\def\tit{ВЕРОЯТНОСТНАЯ МОДЕЛЬ ЗАТУХАНИЯ МОЩНОСТИ СИГНАЛА В СЦЕНАРИЯХ 3GPP TR 38.901 
РАЗВЕРТЫВАНИЯ СЕТИ 5G$^*$}

\def\titkol{Вероятностная модель затухания мощности сигнала в~сценариях 3GPP TR 38.901 
развертывания сети 5G}

\def\aut{Е.\,Д.~Макеева$^1$, И.\,А.~Кочеткова$^2$, С.\,Я.~Шоргин$^3$}

\def\autkol{Е.\,Д.~Макеева, И.\,А.~Кочеткова, С.\,Я.~Шоргин}

\titel{\tit}{\aut}{\autkol}{\titkol}

\index{Макеева Е.\,Д.}
\index{Кочеткова И.\,А.}
\index{Шоргин С.\,Я.}
\index{Makeeva E.\,D.}
\index{Kochetkova I.\,A.}
\index{Shorgin S.\,Ya.}


{\renewcommand{\thefootnote}{\fnsymbol{footnote}} \footnotetext[1]
{Публикация выполнена в~рамках проекта №\,025319-2-000 Системы грантовой 
поддержки научных проектов РУДН.}}


\renewcommand{\thefootnote}{\arabic{footnote}}
\footnotetext[1]{Российский университет дружбы народов имени Патриса Лумумбы; 
Институт проблем управления имени В.\,А.~Трапезникова Российской академии наук, 
\mbox{elena-makeeva-96@mail.ru}}
\footnotetext[2]{Российский университет дружбы народов имени Патриса Лумумбы; 
Федеральный исследовательский центр <<Информатика и~управ\-ле\-ние>> Российской 
академии наук, \mbox{kochetkova-ia@rudn.ru}}
\footnotetext[3]{Федеральный исследовательский центр <<Информатика 
и~управ\-ле\-ние>> Российской академии наук, \mbox{sshorgin@ipiran.ru}}

\vspace*{2pt}






\Abst{Сети пятого (5G) и~последующих поколений будут использовать терагерцевый диапазон 
радиочастот, что обеспечит сверхвысокую скорость передачи данных. Однако при 
этом возможны потери сигнала при прохождении через препятствия. Поэтому 
становится крайне важным моделирование распространения сигнала с~по\-мощью 
стохастической геометрии и~использование актуальных моделей затухания сигнала. 
Модели для описания затухания сигнала для различных сценариев развертывания сети 
5G в~виде эмпирических формул содержатся в~спецификации 3GPP TR 38.901. Тем не 
менее обычно для построения моделей стохастической геометрии используются 
упрощенные виды формул. В~\mbox{статье} представлена функция распределения (ФР) затухания 
мощности сигнала при случайном расположении пользователей в~соответствии со 
сценариями, описанными в~3GPP TR 38.901. На численных примерах показано, что 
разница значений с~упрощенной формулой значительна и~может привести к~занижению 
оценки пропускной способности сети.}

\KW{беспроводная сеть; 5G; 3GPP TR 38.901; мощ\-ность затухания сигнала; прямая 
видимость; непрямая ви\-ди\-мость; сто\-ха\-сти\-че\-ская гео\-метрия}

\DOI{10.14357/19922264240204}{EKLCAP}
  
%\vspace*{-6pt}


\vskip 10pt plus 9pt minus 6pt

\thispagestyle{headings}

\begin{multicols}{2}

\label{st\stat}



\section{Введение}

Сети пятого и~последующих поколений будут использовать терагерцевый 
диапазон радиочастот, чтобы обеспечить сверхвысокую скорость передачи данных и~пропускную способность. Однако использование миллиметровых волн связано со 
сложностями из-за потери сигнала при про\-хож\-де\-нии препятствий. Таким образом, для 
обеспечения производительности сетей~5G становится крайне важным моделирование 
распространения сигнала. Формула Шен\-но\-на--Харт\-ли с~формулой Фрииса задают 
пропускную способность канала
$$
C=B \log_2 \left(1+\fr{P_t G_t G_r}{(N+I) \mathrm{PL}}\right),
$$
 где $B$~--- полоса 
пропускания канала; $P_t$~--- мощ\-ность передающей антенны; $G_t$~--- коэффициент 
усиления передающей антенны; $G_r$~--- коэффициент усиления приемной антенны; $N$~--- мощ\-ность шума; 
$I$~--- мощ\-ность интерференции; $\mathrm{PL}$~--- мощ\-ность затухания 
сигнала (path loss, PL) на расстоянии от передающей антенны до приемной 
антенны~\cite{Moltchanov2022a}.
Пропускная способность канала уже далее используется в~управ\-ле\-нии занятием 
радиоресурсов базовой станции (БС) для соблюдения необходимого качества обслуживания 
пользователей по требуемой ско\-рости передачи данных.

 Ввиду того что пользователи находятся на разных расстояниях от БС, 
значения мощностей затухания сигнала будут случайными. Как показано в~работе~\cite{Hmamouche2021}, для учета влияния на пропускную\linebreak
 способность канала 
случайного положения пользователей в~соте применяется стохастическая гео\-мет\-рия. 
Рассмотреть совместное занятие радиоресурсов и~случайный характер поведения 
пользователей позволяет модель на основе аппарата \mbox{ресурсных} сис\-тем массового 
обслуживания~\cite{Naumov2016, Gorbunova2018}. Такие модели применяются для 
исследования различных сценариев развертывания сетей 
5G~\cite{Moltchanov2022b, Markova2019}, например при анализе совместного 
обслуживания трафика со сверхнизкой задержкой и~широкополосного трафика~\cite{Kochetkova2021}.

\begin{figure*}[b] %fig1
\vspace*{-6pt}
      \begin{center}
     \mbox{%
\epsfxsize=124.62mm 
\epsfbox{koc-1.eps}
}
\end{center}
\vspace*{-9pt}
\Caption{Схема системной модели}
\label{fig1}
\end{figure*}

Модели для описания мощности $\mathrm{PL}$ затухания сигнала для разных сценариев 
отражены в~спецификации 3GPP TR 38.901~\cite{3GPP38901}. И~если при проведении 
имитационного моделирования исследователи по большей части полностью реализуют 
эти модели~\cite{Bolla2023}, то при построении моделей стохастической гео\-мет\-рии 
зачастую применяется упрощенный вид формул.
В~обзоре~\cite{Hmamouche2021} рассмотрены различные виды функциональной 
зависимости затухания мощ\-ности сигнала от расстояния между пользователем 
и~БС, которые применяют исследователи. Например, для простоты 
расчетов в~работе~\cite{Moltchanov2022b} используются упрощенные формулы без 
учета ку\-соч\-но-за\-дан\-но\-го вида функции для прямой видимости и~максимума нескольких 
величин мощностей PL для непрямой видимости при по\-стро\-ении~ФР.

В данной статье получена ФР затухания мощ\-ности сигнала при случайном 
расположении пользователей в~соответствии со сценариями 3GPP TR~38.901 
развертывания сети 5G. Использованы формулы из этой спецификации, где приведены 
зависимости PL от расстояния между пользователем и~БС. В~данной 
статье закон распределения пользователей в~соте взят произвольный, а~для 
численного анализа~--- в~соответствии с~типовыми рекомендованными значениями 
параметров сценариев.



\section{Затухание сигнала как функция от параметров сценария 3GPP} \label{sec2}

При исследовании распространения сигнала необходимо учитывать множество 
параметров сети, таких как частота, основные характеристики местности, высота 
принимающей и~передающей антенн, конфигурация антенн и~другие факторы. Для 
упрощения расчетов мощности PL затухания сигнала стандартом 3GPP TR~38.901~\cite{3GPP38901} 
были выделены основные сценарии развертывания сети~5G: 
мак\-ро\-со\-та в~городе (urban macro, UMa), микросота в~городе (urban micro, UMi), 
мак\-ро\-со\-та в~сельской местности (rural macro, RMa), точка доступа внут\-ри 
помещения (indoor hotspot, InH) и~крытая фабрика (indoor factory, InF),~--- 
и~путем экспериментов были получены эмпирические модели затухания сигнала для них. 
На основе этих моделей и~в~предположении случайного характера поведения 
пользователей в~данном разделе получена ФР мощности затухания сигнала с~учетом 
особенностей, описанных в~данной спецификации.


Рассмотрим общее описание предлагаемых сценариев (рис.~\ref{fig1}). Пусть 
передающая антенна БС расположена на высоте~$h_{\mathrm{BS}}$, 
использует несущую частоту~$f_c$ и~создает покрытие радиуса~$R$. 
Пользовательские устройства (ПУ) находятся на высоте~$h_{\mathrm{UT}}$, а~проекция 
расстоянии от ПУ до БС со\-став\-ля\-ет $d$.

В зависимости от своего расположения ПУ может находиться в~зоне прямой видимости 
(line-of-sight, LOS) с~устойчивым уровнем сигнала или вне этой зоны (non-line-of-sight, NLOS). 
Если ПУ расположено на расстоянии~$d$, то ве\-ро\-ят\-ность того, что 
ПУ находится в~зоне прямой видимости, пред\-став\-ля\-ет собой кусочно-заданную 
функцию:
\begin{equation}
\label{eq1}
{\mathrm{Pr}}_{\mathrm{LOS}}(d)=
\begin{cases}
{\mathrm{Pr}}_1^{\mathrm{LOS}}(d), & 0=r_0 \leq d < r_1; \\
{\mathrm{Pr}}_2^{\mathrm{LOS}}(d), & r_1 \leq d < r_2; \\
\cdots & \cdots \\
{\mathrm{Pr}}_I^{\mathrm{LOS}}(d), & r_{I-1} \leq d \leq r_I=R,
\end{cases}
\end{equation}
где радиусы $R_i$ определяют границы интервалов. 

Тогда мощ\-ность $\mathrm{PL}(d)$ 
затухания сигнала примет вид:
\begin{multline}
\label{eq2}
\mathrm{PL}\,(d)= \mathrm{PL}_{\mathrm{LOS}}(d)  {\mathrm{Pr}}_{\mathrm{LOS}}(d) + {}\\
{}+\mathrm{PL}_{\mathrm{NLOS}}(d)  \left[1-
{\mathrm{Pr}}_{\mathrm{LOS}}(d)\right].
\end{multline}

Мощность затухания сигнала в~условиях прямой видимости LOS описывается ку\-соч\-но-за\-дан\-ной функцией
\begin{multline}
\label{eq3}
\mathrm{PL}^{\mathrm{LOS}}(d)={}\\
{}=
\begin{cases}
\mathrm{PL}_1^{\mathrm{LOS}}(d), & 0=d_0 \leq d < d_1;\\
\mathrm{PL}_2^{\mathrm{LOS}}(d), & d_1 \leq d < d_2;\\
\cdots & \cdots \\
\mathrm{PL}_J^{\mathrm{LOS}}(d), & d_{J-1} \leq d \leq d_J=R,
\end{cases}
\end{multline}
где $d_j$~--- границы интервалов (break point distance), а~в~условиях непрямой 
видимости NLOS пред\-став\-ля\-ет собой максимум
\begin{multline}
\label{eq4}
\mathrm{PL}_{\mathrm{NLOS}}(d) = {}\\
\!\!{}=\!
\max\left(\mathrm{PL}^{\mathrm{LOS}}(d),\mathrm{PL}^{\mathrm{NLOS}}_1(d), \ldots, \mathrm{PL}^{\mathrm{NLOS}}_K(d)\right)\!.\!\!
\end{multline}

Каждая из компонент функций для случаев LOS и~NLOS имеет схожую структуру:
\begin{multline}
 \mathrm{PL}^{l}_m(d)[\mathrm{dB}] = \alpha_m^{l}[\mathrm{dB}]+\beta_m^{l}[\mathrm{dB}]\log_{10}{D(d)},
 \\
 \mathrm{PL}^{l}_m(d) = \alpha_m^{l} \cdot D^{\beta_m^{l}}(d),
\\
 l=\begin{cases}
 \mbox{``}\mathrm{LOS}\mbox{''}, & m=j=\overline{0,J}\,; \\
 \mbox{``}\mathrm{NLOS}\mbox{''},& m=k=\overline{0,K}\,,
 \end{cases}
\label{eq5}
\end{multline}
где $D(d)=\sqrt{d^2+(h_{\mathrm{BS}}\hm-h_{\mathrm{UT}})^2}$~--- расстояние от ПУ до БС в~трехмерном 
пространстве; $\alpha$ и~$\beta$~--- коэффициенты модели затухания сигнала~--- 
константы для каждого из сценариев 3GPP TR~38.901.



\section{Функция распределения затухания сигнала при~случайном расположении 
пользователей} \label{sec3}

Примем теперь, что расстояние между ПУ и~БС~--- случайная величина (СВ)~$\xi_d$ 
со значениями~$d$ и~ФР~$F_{\xi_d}(x)$. Тогда расстояние от ПУ до БС в~трехмерном 
пространстве~$\xi_D$ будет функцией от СВ~$\xi_d$ с~ФР

\noindent
\begin{multline*}
F_{\xi_D}(x)  =
\mathrm{Pr}\,(\xi_D \leq x) ={}\\
{}=
\mathrm{Pr}\left(\sqrt{\xi_d^2+(h_{\mathrm{BS}}- h_{\mathrm{UT}})^2} \leq x \right) ={} \\
{} = \mathrm{Pr}\left(\xi_d \leq \sqrt{x^2-(h_{\mathrm{BS}}- h_{\mathrm{UT}})^2} \right) ={}\\
{}=
F_{\xi_d}\left(\sqrt{x^2-(h_{\mathrm{BS}}- h_{\mathrm{UT}})^2} \right).
\end{multline*}

Случайная величина $\xi_m^l$~--- компонента функции затухания сигнала~--- зависит от СВ $\xi_D$ и~по формуле~(\ref{eq5}) имеет ФР
\begin{multline*}
F_{\xi_m^l}(x)  =
\mathrm{Pr}\,(\xi_m^l \leq x) =
\mathrm{Pr}\left(\alpha_m^{l}  ({\xi}_D)^{\beta_m^{l}} \leq x \right) = {}\\
{}=
\mathrm{Pr}\left(\xi_D \leq \left(\fr{x}{\alpha_m^{l}}\right)^{{1}/{\beta_m^{l}}} \right) = 
F_{\xi_D}\left( \left( \fr{x}{\alpha_m^{l}}\right)^{{1}/{\beta_m^{l}}} 
\right) ={}\\
{}=  F_{\xi_d}\left( \sqrt{ 
\left(\fr{x}{\alpha_m^{l}}\right)^{{2}/{\beta_m^{l}}} - \left(h_\mathrm{BS}-h_\mathrm{UT}\right)^2 }\right), \\
 l=\begin{cases}
 \mbox{``}\mathrm{LOS}\mbox{''}, &  m=j=\overline{0,J}\,; \\
 \mbox{``}\mathrm{NLOS}\mbox{''}, & m=k=\overline{0,K}\,.
 \end{cases}
\end{multline*}

Для затухания сигнала в~условиях прямой видимости ФР СВ~$\xi_{\mathrm{LOS}}$ по 
формуле~(\ref{eq3}) примет вид:
\begin{multline}
F_{\xi_{\mathrm{LOS}}}(x) =
\mathrm{Pr}\,(\xi_{\mathrm{LOS}} \leq x) = {} \\
{}= \sum\limits_{j=1}^J  \mathrm{Pr}\left(\xi_{\mathrm{LOS}} \leq x \mid d_{j-1} \leq \xi_d < d_j\right) \times{}\\
{}\times \mathrm{Pr}\left(d_{j-1} \leq  \xi_d < d_j\right) = {}\\
{} = \sum\limits_{j=1}^J F_{\xi_j^{\mathrm{LOS}}}(x) \left[ F_{\xi_d}(d_j) - 
F_{\xi_d}(d_{j-1}) \right], 
\label{eq6}
\end{multline}
а для непрямой видимости ФР СВ $\xi_{\mathrm{NLOS}}$ по формуле~(\ref{eq4}) 
и~с~учетом~\cite{Ventzel2018} запишем как
\begin{multline}
F_{\xi_{\mathrm{NLOS}}}(x)  =
\mathrm{Pr}\,(\xi_{\mathrm{NLOS}} \leq x) ={}\\
{}=
\mathrm{Pr}\left(\max{\left(\xi_{\mathrm{LOS}}, \; \xi^{\mathrm{NLOS}}_1, \ldots, \xi^{\mathrm{NLOS}}_K \right)} 
\leq x \right) = {} \\
{} = \mathrm{Pr}\left(\xi_{\mathrm{LOS}} \leq x, \; \xi^{\mathrm{NLOS}}_1 \leq x, \ldots, \xi^{\mathrm{NLOS}}_K 
\leq x \right) = {}\\
{} = \mathrm{Pr}\,(\xi_{\mathrm{LOS}} \leq x)  \prod\limits_{k=1}^K{\mathrm{Pr}\left(\xi^{\mathrm{NLOS}}_k \leq x 
\right)} = {}\\
{}=
F_{\xi_{\mathrm{LOS}}}(x)  \prod\limits_{k=1}^K F_{\xi_k^{\mathrm{NLOS}}}(x). 
\label{eq7}
\end{multline}

Наконец, ФР СВ $\xi_{\mathrm{PL}}$ затухания сигнала по формуле~(\ref{eq2}) запишем 
следующим образом:
\begin{multline*}
F_{\xi_{\mathrm{PL}}}(x) =
\mathrm{Pr}\,(\xi_{\mathrm{PL}} \leq x) ={}\\
{}=
\mathrm{Pr}\left(\xi_{\mathrm{LOS}}  \xi_{\mathrm{Pr}_{\mathrm{LOS}}} + \xi_{\mathrm{NLOS}} \left[1-
\xi_{\mathrm{Pr}_{\mathrm{LOS}}}\right] \leq x \right),
\end{multline*}
где $\xi_{\mathrm{Pr}_{\mathrm{LOS}}}$~--- СВ вероятности расположения ПУ в~зоне прямой 
видимости~(\ref{eq1}).
Функция распределения $F_{\xi_{\mathrm{PL}}}(x)$ будет приближенно представлять собой свертку.



\section{Численный анализ}


\begin{figure*}[b]\small
\begin{center}
\tabcolsep=4pt
\begin{tabular}{|c|l|c|c|}

\multicolumn{4}{c}{Коэффициенты модели затухания сигнала для UMa и~UMi}\\
\multicolumn{4}{c}{\ }\\[-6pt]
\hline
 Зона& \multicolumn{1}{c|}{[dB]} & UMa & UMi \\
\hline
&&&\\[-9pt]
 & $\alpha_1^{\mathrm{LOS}}$ & $28+20 \log_{10}{f_c}$ & $32{,}4+20\log_{10}{f_c}$\\
% \cline{2-4}
 LOS& $\beta_1^{\mathrm{LOS}}$ & $22$ & $21$\\
% \cline{2-4}
 & $\alpha_2^{\mathrm{LOS}}$ & $28+20 \log_{10}{f_c}-9\log_{10}{\left(d_1^2+(h_{\mathrm{BS}}-h_{\mathrm{UT}})^2\right)} $ 
 & $32{,}4+20\log_{10}{f_c}- 9{,}5\log_{10}{\left(d_1^2+(h_{\mathrm{BS}}-h_{\mathrm{UT}})^2\right)}$\\
% \cline{2-4}
 & $\beta_2^{\mathrm{LOS}}$ & $40$ & $40$\\
\hline
&&&\\[-9pt]
& $\alpha_1^{\mathrm{NLOS}}$ & $13{,}54+20\log_{10}{f_c}-0{,}6(h_{\mathrm{UT}}-1,5)$ & 
$22{,}4+21{,}3\log_{10}{f_c} - 0{,}3(h_{\mathrm{UT}}-1{,}5)$ \\
 %\cline{2-4}
NLOS  & $\beta_1^{\mathrm{NLOS}}$ & $39{,}08$ & $35{,}3$\\
 %\cline{2-4}
 & $\alpha_{\mathrm{Opt}}$ & $32{,}4+20\log_{10}{f_c}$ & $32{,}4+20\log_{10}{f_c} $\\
 %\cline{2-4}
 & $\beta_{\mathrm{Opt}}$ & $30$ & $31{,}9$\\
 \hline
\end{tabular}
\end{center}
%\end{table*}
%\begin{figure*}[b] %fig2
\setcounter{figure}{1}
\vspace*{12pt}
      \begin{center}
     \mbox{%
\epsfxsize=163mm 
\epsfbox{koc-2.eps}
}
\end{center}
\vspace*{-10pt}
  \Caption{Функции распределения PL для LOS~(\textit{а}) и ~NLOS~(\textit{б}) для сценариев UMa (черные кривые) и~UMi~(серые кривые):
  \textit{1}~--- расчет по формулам~(6) для LOS и~(7) для NLOS; \textit{2}~--- расчет по упрощенным формулам}
 \label{fig:1}
 \end{figure*}

 В спецификации 3GPP TR 38.901 указаны основные диапазоны значений параметров 
для сценариев развертывания сетей~5G. Рассмотрим сценарии макросоты UMa и~микросоты UMi в~городе со следующим набором исходных данных: радиус действия БС 
$R\hm=5000$~м, центральная частота $f_c\hm=6$~ГГц, высота ПУ
$h_{\mathrm{UT}}\hm=1{,}5$~м, высоты БС для UMa $h_{\mathrm{BS}}\hm=25$ м и~для UMi $h_{\mathrm{BS}}\hm=10$~м. 
Предположим, что пользователи распределены равномерно в~об\-ласти действия БС 
радиусом~$R$.

Согласно упрощенным формулам, подобным тем, что описаны в~работе~\cite{Moltchanov2022b}, ФР мощности PL затухания сигнала в~условиях 
прямой и~непрямой видимости могут быть представлены как 
$$
F_{\xi_\mathrm{LOS}}(x) = 
F_{\xi_1^{\mathrm{LOS}}}(x)\,;
$$

\noindent
\begin{multline*}
F_{\xi_\mathrm{NLOS}}(x)=F_{\mathrm{Opt}}(x)={}\\
{}= F_{\xi_d}\left( \sqrt{ 
\left(\fr {x}{\alpha_{\mathrm{Opt}}}\right)^{{2}/{\beta_{\mathrm{Opt}}}} - 
(h_\mathrm{BS}-h_\mathrm{UT})^2 }\,\right).
\end{multline*}

\noindent
 Коэффициенты модели затухания сигнала $\alpha_m^{l}$ 
[dB] и~$\beta_m^{l}$ [dB] для сценариев UMa и~UMi, согласно~\cite{3GPP38901}, 
представлены в~таб\-лице.


 
 \begin{figure*} %fig3
 \vspace*{1pt}
      \begin{center}
     \mbox{%
\epsfxsize=163.204mm 
\epsfbox{koc-3.eps}
}
\end{center}
\vspace*{-15pt}
 \Caption{Разница значений ФР PL для LOS~(\textit{а}) и~NLOS~(\textit{б}):
 \textit{1}~--- UMa; \textit{2}~--- UMi}
 \label{fig:2}
  \vspace*{-3pt}
 \end{figure*}

 
 

В рамках данного численного анализа покажем графики ФР моделей PL затухания 
сигнала для сценариев UMa и~UMi в~условиях прямой и~непрямой видимости по 
формулам~(\ref{eq6}) и~(\ref{eq7}) и~упрощенным формулам. Графики с~полученными 
результатами представлены на рис.~\ref{fig:1} и~3. Во всех случаях 
график ФР по упрощенным формулам идет выше, что представляет собой 
верхнюю оценку. Однако при дальнейших расчетах с~использованием упрощенных 
формул пропускная спо\-соб\-ность и~максимальное число обслуженных пользователей 
в~соте оказываются занижены. Таким образом, авторы рекомендуют при использовании 
модели затухания сигнала как компоненты, например в~ресурсных сис\-те\-мах массового 
обслуживания для анализа беспроводных сетей, использовать формулы (\ref{eq6}) и~(\ref{eq7}).


\section{Заключение}

В статье исследована модель затухания сигнала по формулам сценариев 3GPP TR 
38.901. Была получена функция распределения (ФР) мощности затухания сигнала при 
случайном (произвольный закон) расположении пользователей в~зоне покрытия 
БС. Она учитывает ку\-соч\-но-за\-дан\-ный вид функции для LOS и~максимум 
нескольких величин для NLOS. Проведен численный анализ для данных из 
спецификации 3GPP TR 38.90 для сценариев макро- и~микросот в~городе для 
сравнения ФР, представленных в~данной работе, и~ФР, полученных с~по\-мощью 
упрощенных формул. Результат анализа показал, что ФР по упрощенным формулам дает 
оценку сверху, что может понижать точность расчетов пропускной способности 
канала. Отметим, что авторы статьи не ставили перед собой задачу аналитического 
сравнения двух ФР, а~хотели бы обратить внимание на несложный вид полученных 
формул, которые рекомендуют для использования как компоненту в~ресурсных 
системах массового обслуживания при моделировании беспроводных сетей 5G/6G. 
Задачей дальнейшего исследования станет разработка ресурсной системы массового 
обслуживания с~учетом случайного расположения пользователей в~соте через 
пред\-став\-лен\-ную ФР для оценки схемы приоритетного обслуживания узкополосного 
трафика и~прерывания обслуживания широкополосного трафика в~се\-ти~5G.


\vspace*{-12pt}

{\small\frenchspacing
 {\baselineskip=10.5pt
 %\addcontentsline{toc}{section}{References}
 \begin{thebibliography}{99}
 
 \vspace*{-2pt}
 
\bibitem{Moltchanov2022a}
\Au{Молчанов~Д.\,А., Бегишев~В.\,О., Самуйлов~К.\,Е., Кучерявый~Е.\,А.}
Сети 5G/6G: архитектура, технологии, методы анализа и~расчета.~--- 
М.: РУДН, 2022. 516~с.

\bibitem{Hmamouche2021}
\textit{Hmamouche~Y., Benjillali~M., Saoudi~S., Yanikomeroglu~H., Renzo~M.\,D.}
New trends in stochastic geometry for wireless networks: A~tutorial and survey~//
P.~IEEE, 2021. Vol.~109. No.\,7. P.~1200--1252.
doi: 10.1109/JPROC.2021.3061778.

\bibitem{Naumov2016}
\Au{Наумов~В.\,А., Самуйлов~К.\,Е.}
О~связи ресурсных сис\-тем массового обслуживания с~сетями Эрланга~//
Информатика и~её применения, 2016. Т.~10. Вып.~3. С.~9--14.
doi: 10.14357/19922264160302.

\bibitem{Gorbunova2018} %4
\Au{Горбунова~А.\,В., Наумов~В.\,А., Гайдамака~Ю.\,В., Самуйлов~К.\,Е.}
Ресурсные системы массового обслуживания как модели беспроводных сис\-тем связи~//
Информатика и~её применения, 2018. Т.~12. Вып.~3. С.~48--55.
doi: 10.14357/19922264180307.

\bibitem{Markova2019} %5
\Au{Маркова~Е.\,В., Гольская~А.\,А., Дзантиев~И.\,Л., Гудкова~И.\,А., 
Шоргин~С.\,Я.}
Сравнительный анализ показателей эффективности модели беспроводной сети 
межмашинного взаимодействия, работающей в~рамках двух политик разделения 
радиоресурсов~//
Информатика и~её применения, 2019. Т.~13. Вып.~1. С.~108--116.
doi: 10.14357/19922264190115.

\bibitem{Moltchanov2022b} %6
\Au{Moltchanov~D.\,A., Sopin~E.\,S., Begishev~V.\,O., Sa\-muylov~A.\,K., Koucheryavy~Y.\,A., 
Sa\-mouylov~K.\,E.}
A~tutorial on mathematical modeling of 5G/6G millimeter wave and terahertz 
cellular systems~//
IEEE Commun. Surv.  Tut., 2022. Vol.~24. No.\,2. P.~1072--1116.
doi: 10.1109/ COMST.2022.3156207.



\bibitem{Kochetkova2021}
\Au{Кочеткова~И.\,А., Кущазли~А.\,И., Харин~П.\,А., Шоргин~С.\,Я.}
Модель схемы приоритетного доступа трафика URLLC и~eMBB в~сети пятого поколения в~виде ресурсной сис\-те\-мы массового обслуживания~//
Информатика и~её применения, 2021. Т.~15. Вып.~4. С.~87--92.
doi: 10.14357/19922264210412.

\bibitem{3GPP38901}
3GPP TR 38.901. Study on channel model for frequencies from~0.5 to~100~GHz, 
2024. Release 17.1.0.

\bibitem{Bolla2023}
\Au{Bolla~R., Bruschi~R., Lombardo~C., Mohammadpour~A., Trivisonno~R., 
Poe~W.\,Y.}
A~5G multi-gNodeB simulator for ultra-reliable 0.5--100~GHz communication in 
indoor Industry~4.0 environments~//
Comput. Netw., 2023. Vol.~237. Art. No.\,110103.
doi: 10.1016/j.comnet.2023.11010.

\bibitem{Ventzel2018}
\textit{Вентцель~Е.\,С., Овчаров~Л.\,А.}
Теория вероятностей и~ее инженерные приложения.~--- М.: Юстиция, 
2018. 480~c.

\end{thebibliography}

 }
 }

\end{multicols}

\vspace*{-10pt}

\hfill{\small\textit{Поступила в~редакцию 15.03.24}}

%\vspace*{10pt}

%\pagebreak

\newpage

\vspace*{-28pt}

%\hrule

%\vspace*{2pt}

%\hrule



\def\tit{STOCHASTIC PATH LOSS MODEL IN~5G~NETWORK DEPLOYMENT SCENARIOS: A~STUDY BASED ON~3GPP~TR~38.901}


\def\titkol{Stochastic Path Loss Model in 5G Network Deployment Scenarios: A~Study Based on 3GPP TR 38.901}


\def\aut{E.\,D.~Makeeva$^{1,2}$, I.\,A.~Kochetkova$^{1,3}$, and~S.\,Ya.~Shorgin$^{3}$}

\def\autkol{E.\,D.~Makeeva, I.\,A.~Kochetkova, and~S.\,Ya.~Shorgin}

\titel{\tit}{\aut}{\autkol}{\titkol}

\vspace*{-8pt}


\noindent
$^1$RUDN University, 6 Miklukho-Maklaya Str., Moscow 117198, Russian Federation

\noindent
$^2$V.\,A.~Trapeznikov Institute of Control Science of the Russian Academy of 
Sciences, 65~Profsoyuznaya Str., Moscow\linebreak
$\hphantom{^1}$117997, Russian Federation

\noindent
$^3$Federal Research Center ``Computer Science and Control'' of the Russian 
Academy of Sciences, 44-2~Vavilov\linebreak
$\hphantom{^1}$Str., Moscow 119333, Russian Federation

\def\leftfootline{\small{\textbf{\thepage}
\hfill INFORMATIKA I EE PRIMENENIYA~--- INFORMATICS AND
APPLICATIONS\ \ \ 2024\ \ \ volume~18\ \ \ issue\ 2}
}%
 \def\rightfootline{\small{INFORMATIKA I EE PRIMENENIYA~---
INFORMATICS AND APPLICATIONS\ \ \ 2024\ \ \ volume~18\ \ \ issue\ 2
\hfill \textbf{\thepage}}}

\vspace*{3pt}




\Abste{The fifth-generation (5G) and beyond networks will utilize radio frequencies in the terahertz 
spectrum, enabling extremely high data transmission rates. However, signal loss may occur when signals 
pass through obstacles, making it crucial to simulate signal propagation using stochastic geometry 
and apply up-to-date models for signal attenuation. The 3GPP TR~38.901 specification includes models that describe signal
 attenuation in various 5G~network scenarios using empirical formulas. Nevertheless, simpler formulas are typically employed 
 to create models based on stochastic geometry. The authors present the cumulative distribution function 
 for path loss at random user locations according to the scenarios described in 3GPP TR~38.901. In numerical examples, it is shown that the difference 
in values with the simplified formula can be significant and lead to underestimation of the network's capacity}

\KWE{wireless network; 5G; 3GPP TR 38.901; path loss; line-of-sight (LOS); non-line-of-sight (NLOS); stochastic geometry}



\DOI{10.14357/19922264240204}{EKLCAP}

\vspace*{-12pt}

\Ack

\vspace*{-3pt}

\noindent
The publication has been supported by the RUDN University Scientific Projects 
Grant System, project No.\,025319-2-000.


  \begin{multicols}{2}

\renewcommand{\bibname}{\protect\rmfamily References}
%\renewcommand{\bibname}{\large\protect\rm References}

{\small\frenchspacing
 {%\baselineskip=10.8pt
 \addcontentsline{toc}{section}{References}
 \begin{thebibliography}{99} 
\bibitem{Moltchanov2022a-1}
\Aue{Moltchanov,~D.\,A., V.\,O.~Begishev, K.\,E.~Samouylov, and Y.\,A.~Koucheryavy.}
2022.
\textit{Seti 5G/6G: arkhitektura, tekhnologii, metody analiza i~rascheta}
[The 5G/6G networks: Architecture, technologies, analysis methods, and calculations].
Moscow: RUDN University. 516~p.

\bibitem{Hmamouche2021-a}
\Aue{Hmamouche,~Y., M.~Benjillali, S.~Saoudi, H.~Yanikomeroglu, and 
M.\,D.~Renzo.}
2021.
New trends in stochastic geometry for wireless networks: A~tutorial and survey.
\textit{P.~IEEE}. 109(7):1200--1252.
doi: 10.1109/ JPROC.2021.3061778.

\bibitem{Naumov2016-1}
\Aue{Naumov,~V.\,A., and K.\,E.~Samouylov.}
2016. O~svyazi resursnykh sistem massovogo obsluzhivaniya s~setyami Erlanga
[On relationship between queuing systems with resources and Erlang networks].
\textit{Informatika i~ee Primeneniya~--- Inform Appl.} 10(3):9--14.
doi: 10.14357/ 19922264160302.

\bibitem{Gorbunova2018-1} %4
\Aue{Gorbunova,~A.\,V., V.\,A.~Naumov, Yu.\,V.~Gaidamaka, and K.\,E.~Samouylov.}
2018. Resursnye sistemy massovogo obsluzhivaniya kak modeli besprovodnykh sistem svyazi
[Resource queuing systems as models of wireless communication systems].
\textit{Informatika i~ee Primeneniya~--- Inform. Appl.} 12(3):48--55.
doi: 10.14357/19922264180307.



\bibitem{Markova2019-1} %5
\Aue{Markova,~E.\,V., A.\,A.~Golskaia, I.\,L.~Dzantiev, I.\,A.~Gudkova, and 
S.\,Ya.~Shorgin.}
2019. Sravnitel'nyy analiz pokazateley effektivnosti modeli besprovodnoy seti mezhmashinnogo 
vzaimodeystviya, rabotayushchey v~ramkakh dvukh politik razdeleniya radioresursov
[Comparative analysis of performance measures for a~wireless machine-to-machine 
network model operating within two radio resource management policies].
\textit{Informatika i~ee Primeneniya~--- Inform. Appl}. 13(1):108--116.
doi: 10.14357/ 19922264190115.


\bibitem{Moltchanov2022b-1} %6
\Aue{Moltchanov,~D.\,A., E.\,S.~Sopin, V.\,O.~Begishev, A.\,K.~Sa\-muy\-lov, 
Y.\,A.~Koucheryavy, and K.\,E.~Samouylov.}
2022.
A tutorial on mathematical modeling of 5G/6G millimeter wave and terahertz 
cellular systems.
\textit{IEEE Commun. Surv. Tut.} 24(2):1072--1116.
doi: 10.1109/COMST. 2022.3156207.

\bibitem{Kochetkova2021-1} %7
\Aue{Kochetkova,~I.\,A., A.\,I.~Kushchazli, P.\,A.~Kharin, and S.\,Ya.~Shorgin.}
2021. Model' skhemy prioritetnogo do\-stu\-pa trafika URLLC i~eMBB v~seti pyatogo pokoleniya v~vide resursnoy sistemy massovogo obsluzhivaniya
[Model for analyzing priority admission control of URLLC and eMBB communications 
in 5G networks as a~resource queuing system].
\textit{Informatika i~ee Primeneniya~--- Inform. Appl}. 15(4):87--92.
doi: 10.14357/19922264210412.

\bibitem{3GPP38901-1}
3GPP TR 38.901. 2023. Study on channel model for frequencies from~0.5 to~100~GHz, 
Release 17.1.0.

\bibitem{Bolla2023-1}
\Aue{Bolla,~R., R.~Bruschi, C.~Lombardo, A.~Mohammadpour, R.~Trivisonno, and 
W.\,Y.~Poe.}
2023.
A 5G multi-gNodeB\linebreak\vspace*{-12pt}

\pagebreak

\noindent
 simulator for ultra-reliable 0.5--100~GHz communication in 
indoor Industry 4.0 environments.
\textit{Comput. Netw.} 37:110103. doi: 10.1016/j.comnet.2023.11010.

\bibitem{Ventzel2018-1}
\Aue{Ventzel,~E.\,S. and L.\,A.~Ovcharov.}
2018.
\textit{Teoriya veroyatnostey i~ee inzhenernye prilozheniya}
[Probability theory and its engineering applications].
Moscow: Justice. 480~p.

\end{thebibliography}

 }
 }

\end{multicols}

\vspace*{-6pt}

\hfill{\small\textit{Received March 15, 2024}} 

\vspace*{-12pt}


\Contr

\vspace*{-3pt}

\noindent
\textbf{Makeeva Elena D.} (b.\ 1996)~--- PhD student, Department of Probability 
Theory and Cyber Security, RUDN University, 6~Miklukho-Maklaya Str., Moscow 
117198, Russian Federation; junior scientist, V.\,A.~Trapeznikov Institute of 
Control Science of the Russian Academy of Sciences, 65~Profsoyuznaya Str., 
Moscow 117997, Russian Federation; \mbox{elena-makeeva-96@mail.ru}

\vspace*{3pt}

\noindent
\textbf{Kochetkova Irina A.} (b.\ 1985)~--- Candidate of Science (PhD) in physics 
and mathematics, associate professor, Department of Probability Theory and Cyber 
Security, RUDN University, 6~Miklukho-Maklaya Str., Moscow 117198, Russian 
Federation; senior scientist, Federal Research Center ``Computer Science and 
Control'' of the Russian Academy of Sciences, 44-2~Vavilov Str., Moscow 119333, 
Russian Federation; \mbox{kochetkova-ia@rudn.ru}

\vspace*{3pt}

\noindent
\textbf{Shorgin Sergey Ya.} (b.\ 1952)~--- Doctor of Science in physics and 
mathematics, professor, principal scientist, Federal Research Center ``Computer Science and Control'' of the Russian Academy of 
Sciences, 44-2~Vavilov Str., Moscow 119133, Russian Federation; 
\mbox{sshorgin@ipiran.ru}





\label{end\stat}

\renewcommand{\bibname}{\protect\rm Литература}  %4
\def\stat{ostrikova}

\def\tit{ОБ ОПТИМАЛЬНОМ РАСПОЛОЖЕНИИ АНТЕНН ДЛЯ~V2X-СОЕДИНЕНИЙ 
В~СУБТЕРАГЕРЦЕВОМ ДИАПАЗОНЕ$^*$}

\def\titkol{Об оптимальном расположении антенн для~V2X-соединений 
в~субтерагерцевом диапазоне}

\def\aut{Е.\,А.~Мачнев$^1$, В.\,А.~Бесчастный$^2$, Д.\,Ю.~Острикова$^3$, 
Ю.\,В.~Гайдамака$^4$, С.\,Я.~Шоргин$^5$}

\def\autkol{Е.\,А.~Мачнев, В.\,А.~Бесчастный, Д.\,Ю.~Острикова и~др.}
%$^3$,  Ю.\,В.~Гайдамака$^4$, С.\,Я.~Шоргин$^5$}

\titel{\tit}{\aut}{\autkol}{\titkol}

\index{Мачнев Е.\,А.}
\index{Бесчастный В.\,А.}
\index{Острикова Д.\,Ю.}
\index{Гайдамака Ю.\,В.}
\index{Шоргин С.\,Я.}
\index{Machnev E.\,A.}
\index{Beschastnyi V.\,A.}
\index{Ostrikova D.\,Yu.}
\index{Gaidamaka Yu.\,V.}
\index{Shorgin S.\,Ya.}


{\renewcommand{\thefootnote}{\fnsymbol{footnote}} \footnotetext[1]
{Исследование выполнено за счет гранта Российского научного фонда № 22-29-00694.}}


\renewcommand{\thefootnote}{\arabic{footnote}}
\footnotetext[1]{Российский университет дружбы народов, 1042200071@pfur.ru}
\footnotetext[2]{Российский университет дружбы народов, beschastnyy-va@rudn.ru}
\footnotetext[3]{Российский университет дружбы народов, ostrikova-dyu@rudn.ru}
\footnotetext[4]{Российский университет дружбы народов; Федеральный исследовательский центр <<Информатика и~управ\-ле\-ние>> 
Российской академии наук, \mbox{gaydamaka-yuv@rudn.ru}}
\footnotetext[5]{Федеральный исследовательский центр <<Информатика и~управление>> 
Российской академии наук, sshorgin@ipiran.ru}

\vspace*{-6pt}

  
  \Abst{Субтерагерцевая (суб-ТГц, 100--300~ГГц) связь должна обеспечить огромную  
ско\-рость передачи данных в~сис\-те\-мах~6G. Однако зона покрытия базовых станций 
(БС) ограничена, так как сигнал существенно затухает с~увеличением дистанции, а~также 
легко блокируется различными объектами, встречающимися на пути распространения 
сигнала. Таким образом, БС необходимо располагать достаточно часто, что делает такое 
решение дорогостоящим. Для снижения плотности развертывания БС можно использовать 
механизм ретрансляции сигнала с~помощью транспортных средств (ТС). Данный способ 
в~большой степени зависит от зоны размещения приемо-пе\-ре\-да\-ющей антенны на 
кузове ТС, что ставит вопрос о поиске расположения антенны, при котором механизм 
ретрансляции будет эффективным с~точки зрения скорости передачи данных и~расстояния 
между ТС-источником и~БС. В~данной работе на основе спецификации IEEE 802.15.3d 
и~экспериментальных данных о~распространении сигнала на частоте 300~ГГц 
предложена математическая модель для анализа многозвеньевой системы ретрансляции 
сигнала для трех зон размещения антенны на кузове ТС. Полученные результаты 
показывают, что расположение передатчика в~зоне лобового стекла характеризуется более 
низкой скоростью передачи данных, но при этом гораздо большим покрытием, чем 
расположение в~зонах бампера и~двигателя.}
  
  \KW{5G; <<новое радио>>; V2V; V2X; ретрансляция}
  
  \DOI{10.14357/19922264220407} 
  
%\vspace*{-3pt}


\vskip 10pt plus 9pt minus 6pt

\thispagestyle{headings}

\begin{multicols}{2}

\label{st\stat}

\section{Введение}



     Системы 5G New Radio (NR), \mbox{ра\-бо\-та\-ющие} в~мик\-ро\-вол\-но\-вом 
($\mu$Wave) и~миллиметровом (mmWave) диапазонах, уже вышли на рынок. 
В~то же время исследователи приступили к~изучению субтерагерцевых  
(100--300~ГГц) диапазонов в~контексте сис\-тем~6G~[1, 2]. Однако 
чрезвычайно высокое затухание сигнала, эффекты динамической 
блокировки~[3], а~также микромобильность~[4, 5] ограничивают дальность 
действия таких систем несколькими сотнями метров.
     
     Одна из серьезных проблем для сотовых операторов~--- обеспечение 
расширенного мобильного широкополосного доступа (Enhanced mobile 
broadband, eMBB) для пользователей в~движущихся ТС.\linebreak
 Эта услуга требует не только постоянного подключения, но и~высокой 
ско\-рости передачи данных. Одним из решений могут стать 
высокопроизводительные сис\-те\-мы~5G NR, работающие в~\mbox{диапазоне} 
миллиметровых волн, или 6G в~субтерагерцевом диапазоне. Однако высокое 
затухание в~субтерагерцевом диапазоне и~другие особенности 
распространения сигнала, такие как блокировка прямой видимости, 
ограничивают зону покрытия расположенных по краям дороги БС 
несколькими сотнями метров, что требует высокой плотности развертывания, 
а~значит, ставит вопрос о~рен\-та\-бель\-ности сис\-темы.

\begin{figure*}[b] %fig1
\vspace*{-4pt}
\begin{center}
   \mbox{%
\epsfxsize=163mm
\epsfbox{mac-1.eps}
}
\end{center}
\vspace*{-9pt}
\Caption{Схема моста из одного ТС-ре\-транс\-ля\-то\-ра с~учас\-ти\-ем  
ТС-от\-ра\-жа\-теля}
\end{figure*}

     
     Для обеспечения постоянной связи и~снижения финансовых затрат 
сетевых операторов можно использовать механизм ретрансляции 
сигнала~[6]. Чтобы обеспечить поддержку этого механизма, консорциум 
3GPP недавно стандартизировал технологию интегрированного доступа 
и~транспортной сети (Integrated Access and Backhaul, IAB)~[7]. В~рамках 
этой технологии ТС-ретрансляторы с~высокоскоростными передатчиками 
в~диапазоне субтерагерцевых частот могут организовать так называемые 
<<мосты>>, пересылая по цепочке данные от ТС, которые в~настоящее время 
не имеют прямого подключения к~БС. Эффективность механизма 
многозвеньевой ретрансляции сигнала оценивается на основании ско\-рости 
передачи данных по установленному мосту и~длины моста, т.\,е.\ расстояния 
между ТС-источником сигнала и~БС. При этом расположение  
при\-емо-пе\-ре\-да\-ющей антенны на кузове ТС влияет на оба указанных 
показателя. 
     
     Конечная цель данного исследования~--- выработать рекомендации по 
выбору зоны размещения антенны на ТС, при котором механизм 
ретрансляции обеспечит минимальную плотность развертывания БС для 
заданного набора параметров и~условий дорожного движения, включая 
плотность ТС на дороге и~ско\-рости их движения. Разработанная 
математическая модель основана на недавних исследованиях 
распространения сигнала в~субтерагерцевом диапазоне в~среде~V2V 
(Vehicle-to-Vehicle) и~использует реалистичные параметры связи из стандарта 
IEEE~802.15.3d~[8].
     
\section{Системная модель}

     Рассматривается участок автомагистрали, например городская улица 
или скоростное шоссе, покрытие которого беспроводной связью 
обеспечивается БС~6G, работающими в~субтерагерцевом 
диапазоне. Предполагается, что БС установлены по обеим сторонам дороги 
в~шахматном порядке, например на фонарных столбах, на постоянной 
высоте~$h_A$. Расстояние между БС по одной стороне дороги равно~$d$, 
таким образом, БС образуют равнобедренные треугольники 
с~основанием~$d$, т.\,е.\ БС на противоположных сторонах сдвинуты друг 
относительно друга на расстояние $d/2$. Эти БС служат точками доступа 
в~интернет для пользователей, находящихся в~ТС, например в~автомобилях 
(рис.~1).
     


     Предполагается, что дорога имеет четное чис\-ло~$N_l$~полос, при этом 
возможно движение ТС в~противоположных направлениях. Ширина полос 
постоянна и~равна~$w$. Ско\-рость ТС предполагается равной~$v$. 
Расположение ТС на дорожной полосе определяется пуассоновским 
процессом ин\-тен\-сив\-ности~$\lambda$, согласно которому расположены 
центры ТС. Далее предположим, что автомобили имеют 
одинаковую постоянную длину~$l_v$, а~дорожный просвет равен~$h_C$. 
Дорожное движение предполагается однородным на каждой полосе, т.\,е.\
     $v$, $\lambda$ и~$l_v$ не зависят от рассматриваемой полосы. 
     
     Согласно~[9], новые технологии V2V обеспечат\linebreak эффективный 
и~безопасный контроль над мобильностью ТС при минимально допустимом 
расстоянии между любыми двумя транспортными средствами $d_s\hm= t_s 
v$, где $t_s\hm=0{,}5$~с~--- минимальное\linebreak время, необходимое сис\-те\-мам 
автоматического управ\-ле\-ния для оценки условий движения в~режиме 
реального времени и~принятия превентивных мер.
     
     Для снижения капитальных затрат за счет увеличения расстояния 
между БС предполагается, что доля ТС~$P_E$ оснащена устройствами 
ретрансляции сигнала. Параметр~$P_E$ называется <<сте\-пенью внед\-ре\-ния 
технологии>>. Транспортные средства обору\-до\-ва\-ны двумя при\-емо-пе\-ре\-да\-ющи\-ми антеннами:
одна в~передней и~одна в~задней части ТС. Возможность подключения 
приемопередатчиков к~высокоскоростной внут\-рен\-ней шине со ско\-ростью 
\mbox{передачи} данных, достаточной для обработки {ретранслируемого} трафика, 
под\-тверж\-да\-ет\-ся последними разработками~[10].
     
     Транспортные средства, оснащенные средствами связи, нуждаются 
в~услугах eMBB, предостав\-ля\-емых через БС. Если прямое соединение 
невозможно из-за того, что ТС находится вне зоны действия ближайшей БС, 
они используют ретрансляцию для формирования моста, состоящего из 
одной или нескольких точек ретрансляции, в~качестве которых выступают 
ТС-ре\-транс\-ля\-то\-ры. Длина этого моста и~скорость передачи данных по 
нему существенно зависят от плотности развертывания БС, характеристик 
окружающей среды, влияющих на распространение сигнала, включая 
плотность ТС на дороге и~сценарий дорожного движения, а~также зоны 
размещения антенны.
     
     В работе рассмотрены следующие потенциальные варианты 
расположения антенны: на уровне бампера (0,3--0,4~м), на уровне двигателя 
(0,4--1,0~м) и~на лобовом стекле (1,0--1,5~м). Определим параметр~$\sigma$ 
для обозначения зоны размещения антенны, т.\,е.\ $\sigma\hm\in \{B,E,W\}$ 
соответственно.
     
     Значение отношения уровня сигнала к~уровню шума (Signal to 
Interference plus Noise Ratio, SINR) на ТС на расстоянии~$x$ от БС 
записывается в~виде
     \begin{equation*}
     S(x)= P_T G_A G_U \fr{x^{-\gamma}}{(N_0+I)L_A (f,x) 
L_B}\,,
    % \label{e1-ost}
     \end{equation*}
где $P_T$~--- излучаемая мощность; $G_A$ и~$G_U$~--- коэффициенты 
усиления на стороне приема и~передачи; $\gamma$~--- коэффициент потерь 
на пути сигнала; $N_0$~--- тепловой шум; $I$~--- помехи; $L_A(f,x)$~--- 
коэффициент затухания сигнала; $L_B$~--- потери, вызванные блокировкой 
или отражениями от других объектов.

\section{Математическая модель установления соединения 
для~механизма многозвеньевой ретрансляции сигнала}

     Построение модели начнем с~анализа прямых подключений между  
ТС-ис\-точ\-ни\-ком сигнала и~БС, а именно: с~определения максимальной 
дистанции прямого подключения и~скорости передачи данных. Чтобы 
вычислить данные параметры, используем набор схем модуляции 
и~кодирования, указанных в~стандарте IEEE 802.15.3d~[8].
     
     Следуя~[11], максимальное расстояние подключения  
ав\-то\-мо\-би\-ля-ис\-точ\-ни\-ка к~БС с~допустимой мощ\-ностью  
при\-ни\-ма\-емо\-го сигнала~$S$ можно записать как
     \begin{equation}
     d_\xi(S)= \left( \fr{P_T \sqrt[10]{10^{G_A+G_U}}} 
{\sqrt[10]{10^{N_0+S}}\,10^{2\log_{10} f_c-14{,}86+L_B}} 
\right)^{1/\gamma}\,,
     \label{e2-ost}
     \end{equation}
где $P_T$~--- мощность передачи; $G_A$ и~$G_U$~--- коэффициенты 
усиления на приеме и~передаче; $f_c$~--- 
несущая частота.

     Если следующий автомобиль не поддерживает ретрансляцию, 
расположение антенны на уровне бампера в~дополнение к~прямой видимости 
($L$) может использовать пути отражения под ТС ($U$) и~от соседнего 
автомобиля ($R$). Если антенна расположена на уровне двигателя или 
лобовом стекле, отражение сигнала под автомобилем недоступно ввиду 
геометрических ограничений, при этом отражение от соседнего автомобиля 
доступно. Для расположения на стекле существует также возможность 
прохождения сигнала сквозь стекло следующего автомобиля ($W$), но 
затухание сигнала при этом будет в~несколько раз выше затухания при 
прямой видимости. Для обозначения типов распространения сигнала 
используем параметр $\xi \hm\in \{L,U,R,W\}$.
     
     Следует обратить внимание, что минимальная требуемая мощность 
сигнала приема $S\hm= S_{\min}$ позволяет получить максимальное 
расстояние для связи между соседними автомобилями, далее обозначаемое 
как $d_\xi^{\max}$, $\xi\hm\in \{L,U,R,W\}$.
     
     Функции плотности вероятности расстояния до $i$-го соседнего 
ТС~[12] в~случае распределения цент\-ров ТС согласно процессу Пуассона 
подчиняются распределению Эрланга~[13]:
     \begin{equation*}
     f_i(x)=\fr{2(\pi\lambda)^i}{(i-1)!}\,x^{2i-1} e^{-\pi \lambda x^2} ,\enskip 
x>0\,,\enskip i=1,\ldots , N.
     %\label{e3-ost}
     \end{equation*}
     
     \textbf{Подключение в~зоне прямой видимости.} Связь в~зоне прямой 
видимости на расстоянии~$r$ возможна для всех рассматриваемых зон 
размещения антенны, если расстояние между вза\-и\-мо\-дей\-ст\-ву\-ющи\-ми ТС 
меньше, чем максимальное расстояние связи в~условиях прямой видимости 
$d^{\max}$ из~(\ref{e2-ost}), и~сле\-ду\-ющее ТС оборудовано при\-е\-мо-пе\-ре\-да\-ющей антенной. Таким образом, вероятность успешного подключения в~условиях прямой видимости на расстоянии~$r$ имеет вид:
     \begin{equation*}
     f_{H,\mathrm{LoS}}(r) =I\left( r<d^{\max}\right) P_E f_0(r)\,,
     \label{e4-ost}
     \end{equation*}
где $f_0(r)$ вычисляется как
\begin{equation*}
f_0(r)= 0{,}5 f_1(r) +0{,}5 f_2(r)\,.
%\label{e5-ost}
\end{equation*}
     
     \textbf{Отражение от соседнего автомобиля.} На\-пом\-ним, что боковое 
отражение доступно для всех рас\-смат\-ри\-ва\-емых зон размещения антенны. 
В~случае блокировки прямой видимости, т.\,е.\ когда следующее ТС не 
оборудовано приемо-передающей антенной, предполагаем, что 
расстояние~$d_0$ до блокирующего ТС (см.\ рис.~1) равновероятно 
подчиняется распределению расстояния либо до первого соседа, либо до 
второго. Соответственно, плот\-ность ве\-ро\-ят\-ности расстояния с~обходом 
блокирующего автомобиля определяется как свертка расстояний до двух 
ближайших соседей блокирующего ТС, т.\,е.
     \begin{equation*}
     f_B(r)=\left(f_1*f_2\right) (r)=\int\limits_0^\infty f_1(s) f_2(r-s)\,ds\,.
%     \label{e6-ost}
     \end{equation*}
     
     \begin{figure*}[b] %fig2
\vspace*{-6pt}
\begin{center}
   \mbox{%
\epsfxsize=118.5mm
\epsfbox{mac-2.eps}
}
\end{center}
\vspace*{-9pt}
\Caption{Схема отражения под автомобилем}
\end{figure*}
     
     В модели предполагается, что возможен обход не более одного 
блокирующего ТС через одно боковое отражение (см.\ рис.~1). При этом  
ав\-то\-мо\-биль-<<от\-ра\-жа\-тель>> должен находиться на таком участке 
соседней полосы, где сигнал от <<передатчика>> не будет заблокирован  
ав\-то\-мо\-би\-лем-<<бло\-ки\-ра\-то\-ром>>, находящимся слишком близ\-ко 
либо к~<<передатчику>>, либо к~<<ретранслятору>>. Вероятность 
незаблокированного отражения может быть выражена~как
     \begin{equation*}
     \delta_R(r)=\fr{w_v}{\eta(r)}= \fr{w_v r}{2(w-w_v)}\,,
    % \label{e7-ost}
     \end{equation*}
где $\eta(r)\hm= 2(w\hm- w_v)/r$~--- тангенс угла отклонения луча~$\beta$; 
$w_v$~--- половина ширины ТС.

     Вероятность ${\sf P}_R(r)$ наличия на соседней полосе ТС, 
обеспечивающего незаблокированное отражение сигнала, можно получить, 
используя свойство независимости процесса Пуассона. Предполагая, что 
точка центра отражателя на соседней полосе равномерно распределена на 
отрезке, соответствующем расстоянию между передатчиком 
и~ретранслятором~[14], искомую вероятность можно выразить в~следующем 
виде:
     \begin{equation}
     {\sf P}_R(r)=\begin{cases}
     \left( \fr{N_l}{2}-1\right) 
\fr{\delta_R(r)\lambda}{[l_v+2\Delta_\alpha(r)]^{-1}}\,, & \\
& \hspace*{-20mm}2d_s<r<d_R^{\max}\,;\\
     0 & \hspace*{-20mm}\mbox{в\ других\ случаях},
     \end{cases}
     \label{e8-ost}
     \end{equation}
где $\Delta_\alpha$~--- допустимое смещение соседнего~ТС.
     
     \textbf{Отражение под автомобилем.} В~отличие от зон лобового 
стекла и~двигателя, при размещении антенны в~зоне бампера становится 
возможной передача сигнала под ТС за счет отражения от дорожного 
полотна. Из-за свойства симметричности отраженного пути (рис.~2) 
минимальное расстояние между бампером блокиратора и~бампером 
с~антенной с~обеих сторон одинаково и~определяется выражением
     \begin{equation*}
     \delta_U(r)= \fr{r}{2}\left( 1-\fr{h_C}{h_B}\right)-\Delta_\alpha\,,
    % \label{e9-ost}
     \end{equation*}
где $h_C$~--- дорожный просвет; $h_B$~--- высота размещения антенны; 
$\Delta_\alpha$~--- смещение, допускаемое для бло\-ки\-ру\-юще\-го~ТС.



     Вероятность передачи сигнала под ТС можно найти 
аналогично~(\ref{e8-ost}):

\noindent
     \begin{multline}
     {\sf P}_{B,B}(r)={}\\
     {}= \begin{cases}
     \delta_U(r) \left( \fr{h_C}{h_B}+\fr{\Delta_\alpha}{r}\right), & 2d_s< r< 
d_U^{\max}\,;\\
     0 & \mbox{в\ других\ случаях}.
     \end{cases}
     \label{e10-ost}
     \end{multline} 
     
     \textbf{Прохождение сигнала сквозь стекло.} Наконец, найдем 
вероятность установления соединения через стекла бло\-ки\-ру\-юще\-го~ТС:
     \begin{equation*}
     {\sf P}_{B,W}(r)= \begin{cases}
     1, & 2d_s<r<d_W^{\max}\,;\\
     0 & \mbox{в\ других\ случаях}\,.
     \end{cases}
    % \label{e11-ost}
     \end{equation*}
     
     \textbf{Вероятность успешного многозвеньевого подключения.} 
С~учетом вышеизложенных возможностей установления соединения для 
механизма многозвеньевой ретрансляции сигнала вероятность успешного 
подключения на расстоянии~$r$, $f_{H,\sigma}(r)$, $\sigma\hm\in \{B,E,W\}$, 
определяется как сумма вероятностей двух событий: либо следующее ТС 
оснащено антенной и~можно использовать подключение в~условиях прямой 
видимости, либо следующее ТС на полосе не имеет передающего устройства 
с~ве\-ро\-ят\-ностью $1\hm- {\sf P}_E$ и~единственный вариант~--- обойти его 
с~по\-мощью описанных выше способов. Таким образом, 
расширяя~(\ref{e10-ost}), приходим к~сле\-ду\-ющей формуле для плот\-ности 
вероятности успешного подключения на расстоянии~$r$:
     \begin{multline}
     f_{H,\sigma}(r) = I\left( r< d^{\max}\right) {\sf P}_E f_0(r)+
     \left( {\sf P}_E-{\sf P}_E^2\right)\times{}\\
     {}\times \left( {\sf P}_R(r)+{\sf P}_{B,\sigma}(r) - {\sf P}_R(r) 
{\sf P}_{B,\sigma}(r)\right) f_B(r)\,.
     \label{e12-ost}
     \end{multline}
     
       \begin{table*}[b]\small
       \vspace*{-12pt}
  \begin{center}
  \Caption{Параметры дорожного движения}
  \vspace*{2ex}
  
  \begin{tabular}{|l|c|c|}
  \hline
\multicolumn{1}{|c|}{Сценарий}&\tabcolsep=0pt\begin{tabular}{c}Скорость\\ автомобиля $v$, км/ч \end{tabular}& 
\tabcolsep=0pt\begin{tabular}{c}Расстояние между\\ автомобилями $d_0$, м\end{tabular}\\
\hline
 Пробка& 20 &10\\
Нормальный городской трафик& 60&30\\
Шоссе& 120\hphantom{9} & 60\\
\hline
\end{tabular}
\end{center}
%\end{table*}
%\begin{table*}\small
%\vspace*{4pt}
  \begin{center}
  \Caption{Входные параметры системы}
  \vspace*{2ex}
  
  \begin{tabular}{|c|c|l|}
  \hline
Обозначение&Значение&\multicolumn{1}{c|}{Описание}\\
\hline
$l_v$&4,5~м&Длина автомобиля\\
$\lambda$&0,02~авт/м&Средняя плотность автомобилей\\
$h_A$&3~м&Высота БС\\
$h_B$&0,4/0,7/1,2~м&Высота антенны\\
$h_C$&0,2~м&Дорожный просвет автомобиля\\
$v$&25~м/с&Скорость автомобиля\\
$f_C$&304,2~ГГц&Несущая частота\\
$P_T$&$4{,}2\cdot 10^{-6}$~Вт &Мощность передатчика БС/автомобиля\\
$N_0$&$-$84~дБ&Мощность шума\\
 $S$&$-$56~дБ&Минимальный SINR\\
$G_A$, $G_U$&17,58~дБ&Коэффициенты усиления на приеме и~на передаче\\
$\gamma$&2,1&Коэффициент затухания сигнала\\
\hline
\end{tabular}
\end{center}
\end{table*}
     
     Определив вероятность многозвеньевого подключения  
в~(\ref{e12-ost}), можно перейти к~описанию показателей эффективности. 
В~част\-ности, вероятность того, что мост из~$n$ ТС-ре\-транс\-ля\-то\-ров 
обеспечит передачу данных на расстояние~$r$, может быть получена  
с~по\-мощью $n$-крат\-ной свертки~(\ref{e12-ost}):

\noindent
     \begin{equation*}
     p_{C,\sigma}(n,r) =\int\limits^\infty_{r-d^{\max}} f^{(n)}_{H,\sigma} 
(y)\,dy\,.
    % \label{e13-ost}
     \end{equation*}
     Если допустить возможность бесконечного чис\-ла 
 ТС-ре\-транс\-ля\-то\-ров в~соединении, вероятность того, что длина моста 
окажется не меньше расстояния~$r$ от ТС-ис\-точ\-ни\-ка сигнала до БС, 
может быть получена следующим образом:

\noindent
     \begin{multline*}
     p_{S,\sigma} (r)= \sum\limits^\infty_{n=1} p_{C,\sigma} (n,r) 
\prod\limits_{j=1}^{n-1} \left( 1-p_{C,\sigma}(j,r)\right)\,,\\
     \sigma\in \{B,E,W\}\,.
     %\label{e14-ost}
     \end{multline*}
     
     Скорость передачи по установленному мос\-ту~[14] определяется звеном 
с~наихудшими условиями канала. Тогда, имея пороговые значения~$s_i$, 
$i\hm=1,\ldots , N_C$, чувствительности приемника, соответствующие набору 
показателей качества канала $\{1,\ldots , N_C\}$, можно найти вероятность 
использования схемы кодирования~$i$ в~звене для каждого типа 
распространения сигнала $\xi\hm\in \{L,U,R,W\}$:
     \begin{equation*}
     q_{\xi,i} =\int\limits_{d_\xi(s_i)}^{d_\xi(s_{i+1})} f_{H,\xi}(r)\,dr\,,\enskip 
i=1,\ldots , N_C\,,
    % \label{e15-ost}
     \end{equation*}
где $d_\xi(S_{N_C+1})\hm=\infty$.

     Теперь определим среднюю скорость передачи данных по 
установленному мосту, включающему~$n$ ТС-ре\-транс\-ля\-то\-ров, 
используя биномиальное распределение

\noindent
     \begin{equation*}
     \rho_{\xi,n}=\sum\limits_{i=1}^{N_C} \omega_i \left( 
\sum\limits_{k=i}^{N_C} q_{\xi,k}\right)^n\,,\enskip \xi\in \{L,U,R,W\}\,,
     %\label{e16-ost}
     \end{equation*}
где $\omega_i$~--- спектральная эффективность канала согласно схеме~$i$.

     Тем не менее в~целях контроля качества соединения имеет смысл 
ограничивать максимальное число ТС-ре\-транс\-ля\-то\-ров некоторым 
заданным значением~$N$, $N\hm\geq 1$:
     \begin{multline*}
     \rho_{S,\sigma}(r)={}\\
     {}= \!\!\!\sum\limits_{\xi\in \{L,R,U,W\}}\!\!\!\!\!\!\! {\sf P}_\xi 
\sum\limits_{n=1}^N \rho_{\xi,n} p_{C,\sigma} (n,r) \prod\limits^n_{j=1} \left( 
1-p_{C,\sigma} (j,r)\right)\,,\\
     \sigma\in\{B,E,W\}\,.
   %  \label{e17-ost}
     \end{multline*}
     
     \vspace*{-18pt}

\section{Численный анализ}

\vspace*{-3pt}

     В качестве исходных данных для численного эксперимента 
рассматриваются три сценария дорожного движения: проб\-ка, нормальные 
условия движения в~городе и~скоростное шоссе. Сценарии различаются  
ско\-ростью ТС и~средним расстоянием между ними, как показано в~табл.~1. 
Остальные па\-ра\-мет\-ры сис\-те\-мы приведены в~табл.~2.

\begin{figure*} %fig3
\vspace*{1pt}
\begin{center}
   \mbox{%
\epsfxsize=84.218mm
\epsfbox{mac-3.eps}
}
\end{center}
\vspace*{-11pt}
\Caption{Средняя длина моста (залитые значки) и~скорость передачи (пустые значки) в~зависимости от 
плотности ТС: \textit{1}~--- бампер; \textit{2}~--- стекло; \textit{3}~--- 
двигатель}
\vspace*{-5pt}
\end{figure*}


\begin{figure*}[b] %fig4
  \vspace*{-6pt}
    \begin{minipage}[t]{80mm}
\begin{center}
   \mbox{%
\epsfxsize=78.898mm
\epsfbox{mac-4-a.eps}
}
\end{center}
\vspace*{-9pt}
  \Caption{Вероятность подключения в~зависимости от расстояния между БС и~степени 
внедрения технологии (черные кривые~--- $P_E\hm= 0{,}7$; серые кривые~--- $P_E\hm= 0{,}9$): (\textit{1}~--- бампер; 
  \textit{2}~--- стекло; \textit{3}~--- двигатель}
  \end{minipage}
  %\end{figure*}
   \hfill  
%  \begin{figure*} %fig5
  \vspace*{-6pt}
  \begin{minipage}[t]{80mm}
\begin{center}
   \mbox{%
\epsfxsize=79mm
\epsfbox{mac-4-b.eps}
}
\end{center}
\vspace*{-9pt}
  \Caption{Вероятность подключения в~зависимости от расстояния между БС и~сценариев дорожного движения
  (пунктирные кривые~--- пробка; штриховые~--- нормальный городской трафик; сплошные кривые~--- шоссе): 
  \textit{1}~--- бампер; \textit{2}~--- стекло; \textit{3}~--- двигатель}
    \end{minipage}
  \end{figure*}
  
  

     Начнем с~исследования основных зависимостей между средней длиной 
мос\-та и~ско\-ростью передачи данных, показанных на рис.~3, для различных 
зон размещения антенны в~за\-ви\-си\-мости от плот\-ности ТС
на полосе дорожного движения, где максимальное чис\-ло  
ТС-ре\-транс\-ля\-то\-ров~$N$установлено рав\-ным~10, а~степень внед\-ре\-ния 
технологии $P_E\hm= 0{,}7$.
     


     По результатам эксперимента мож\-но отметить, что средняя длина 
моста уменьшается по мере увеличения плот\-ности ТС на дороге для всех 
рас\-смот\-рен\-ных вариантов расположения антенны. Это объясняется тем, что 
с~увеличением плот\-ности уменьшается среднее расстояние между ТС. 
Обратный эффект наблюдается для ско\-рости передачи данных, и~это связано 
с~более короткими расстояниями меж\-ду автомобилями, а~следовательно, 
лучшим качеством канала.
     
     Анализируя влияние зоны размещения антенны, можно сделать вывод, 
что расположение у~лобового стекла обеспечивает 
б$\acute{\mbox{о}}$льшую сред\-нюю длину мос\-та почти для всех 
рассмотренных плотностей ТС. Однако этот выигрыш достигается за счет 
гораздо меньшей ско\-рости передачи данных. Обоснование данного 
наблюдения заключается в~том, что экраны в~виде заднего и~переднего стекол 
блокирующего автомобиля создают высокие потери при передаче сигнала.
     
     Одной из характеристик, отвечающих за гарантии производительности 
для пользователей коммерческих систем, служит до\-ступ\-ность подключения 
к~БС посредством моста. Рас\-смот\-рим \mbox{до\-ступ\-ность} подключения как 
функцию от рас\-сто\-яния между БС (Inter-site distance, ISD), показанную на 
рис.~4, для различных значений степени внед\-ре\-ния технологии 
и~нормальных условий городского трафика ($\lambda\hm= 1/30$). При 
достижении 1250~м наблюдается резкий спад, ха\-рак\-те\-ри\-зу\-ющий расстояние, 
на котором начинает работать механизм ре\-транс\-ля\-ции сигнала. Из всех 
рас\-смот\-рен\-ных зон размещения антенны лобовое стек\-ло показывает 
наилучшую до\-ступ\-ность~0,95 при степени внед\-ре\-ния технологии~0,9.
     
     Влияние условий дорожного движения на ве\-ро\-ят\-ность подключения 
для различных зон размещения антенны показано на рис.~5. Здесь 
можно заметить, что наихудший возможный сценарий~--- пробки, где не 
только все рас\-смот\-рен\-ные зоны приводят к~одному и~тому же значению 
параметра ISD, но и~связанное с~этим улучшение ISD незначительно. 
Причина в~том, что ТС расположены очень плот\-но, блокируя сразу несколько 
вариантов отражений, в~том чис\-ле под ТС. Тем не менее для нормального 
и~шоссейного сценариев наилучшая зона размещения антенны с~точ\-ки 
зрения длины установленного мос\-та~--- на лобовом стекле.

  
      

  
\section{Заключение}

     В работе предложена математическая модель для оценки 
производительности механизма мно\-го\-звень\-евой ретрансляции сигнала для 
V2X-со\-еди\-не\-ний в~частотных диапазонах субтерагерцевых волн при 
различных условиях распространения сигнала. Рассматриваемые метрики 
включают среднее расстояние подключения к~БС и~скорость передачи 
данных с~учетом возможной многозвеньевой ретрансляции сигнала, а~также 
критический параметр качества обслуживания~--- доступность сети.
     
     Представленные численные результаты для типичных сценариев 
предоставления услуг связи по технологии 6G с~использованием механизма 
многозвеньевой ретрансляции сигнала показывают, что можно 
рекомендовать размещение антенны в~зоне лобового стекла, которое, 
несмотря на более низкую скорость передачи данных, значительно менее 
чувствительно к~степени внедрения технологии и,~как правило, 
характеризуется гораздо б$\acute{\mbox{о}}$льшим покрытием сети.
     
{\small\frenchspacing
 {%\baselineskip=10.8pt
 %\addcontentsline{toc}{section}{References}
 \begin{thebibliography}{99}
\bibitem{1-ost}
\Au{Moltchanov D., Gaidamaka~Y., Ostrikova~D., Beschastnyi~V., Koucheryavy~Y., Samouylov~K.} 
Ergodic outage and capacity of terahertz systems under micromobility and blockage impairments~// 
IEEE T. Wirel. Commun., 2021. Vol.~21. Iss.~5. P.~3024--3039. doi: 
10.1109/ TWC.2021.3117583.
\bibitem{2-ost}
\Au{Moltchanov D., Beschastnyi~V., Ostrikova~D., Gaidamaka~Y., Koucheryavy~Y.} 
Uninterrupted connectivity time in THz systems under user micromobility and blockage~//  
Global Communications Conference Proceedings.~--- Piscataway, NJ, USA: 
IEEE, 2021. Art. 9685384. 6~p. doi: 10.1109/GLOBECOM46510.2021.9685384.
\bibitem{3-ost}
\Au{Gapeyenko M., Samuylov~A., Gerasimenko~M., Moltchanov~D., Singh~S., Akdeniz~M.\,R., 
Aryafar~E., Himayat~N., Andreev~S., Koucheryavy~Y.} On the temporal effects of mobile 
blockers in urban millimeter-wave cellular scenarios~// IEEE T. Veh. Technol., 
2017. Vol.~66. Iss.~11. P.~10124--10138. doi: 10.1109/TVT.2017.2754543.
\bibitem{4-ost}
\Au{Stepanov N.\,V., Moltchanov~D., Begishev~V., Turlikov~A., Koucheryavy~Y.} Statistical 
analysis and modeling of user micromobility for THz cellular communications~// IEEE T. 
Veh. Technol., 2021. Vol.~71. Iss.~1. P.~725--738. doi: 10.1109/TVT.2021.3124870.
\bibitem{5-ost}
\Au{Beschastnyi V., Ostrikova~D., Moltchanov~D., Gaidamaka~Y., Koucheryavy~Y., 
Samouylov~K.} Balancing latency and energy efficiency in mmWave 5G NR systems with 
multiconnectivity~// IEEE Commun. Lett., 2022. Vol.~26. Iss.~8. P.~1952--1956. doi: 
10.1109/LCOMM. 2022.3175663.
\bibitem{6-ost}
\Au{Petrov V., Moltchanov~D., Andreev~S., Heath~R.\,W.} Analysis of intelligent vehicular 
relaying in urban 5G\;+\;millimeter-wave cellular deployments~// Global Communications 
Conference Proceedings.~--- Piscataway, NJ, USA: IEEE, 2019. Art.~05946. 
6~p. doi: 10.48550/arXiv.1908.05946.
\bibitem{7-ost}
Study on integrated access and backhaul (Release 17): Technical Specification 38.874 
V17.0.0. 3GPP, 2020. {\sf  
https://www.3gpp.org/ftp/Specs/archive/38\_series/38.\linebreak 874/38874-g00.zip}.
\bibitem{8-ost}
\Au{Petrov V., Kurner~T., Hosako~I.} IEEE 802.15.3d: First standardization efforts for  
sub-terahertz band communications toward 6G~// IEEE Commun. Mag., 2020. 
Vol.~58. No.\,11. P.~28--33. doi: 10.1109/MCOM.001.2000273.
\bibitem{9-ost}
\Au{Ozpolat M., Bhargava~K., Kampert~E., Higgins~M.\,D.} Multi-lane urban mmWave V2V 
networks: A path loss behavior dependent coverage analysis~// Vehicular Communications, 
2021. Vol.~30. Art.~100348. 11~p. doi: 10.1016/ j.vehcom.2021.100348.
\bibitem{10-ost}
\Au{Wang J., Liu~J., Kato~N.} Networking and communications in autonomous driving: 
A~survey~// IEEE Commun. Surv. Tut., 2018. Vol.~21. No.\,2. P.~1243--1274. doi: 
10.1109/COMST.2018.2888904.
\bibitem{11-ost}
\Au{Eckhardt J.\,M., Petrov~V., Moltchanov~D., Koucheryavy~Y., K$\ddot{\mbox{u}}$rner~T.} 
Channel measurements and modeling for low-terahertz band vehicular communications~// IEEE 
J.~Sel. Area. Comm., 2021. Vol.~39. No.\,6. P.~1590--1603. doi: 
10.1109/JSAC.2021.3071843.
\bibitem{12-ost}
\Au{Moltchanov D.} Distance distributions in random networks~//  Ad Hoc Netw., 
2012. Vol.~10. P.~1146--1166. doi: 10.48550/arXiv.0804.4204. 
\bibitem{13-ost}
\Au{Basharin G., Gaidamaka~Y.\,V., Samouylov~K.\,E.} Mathematical theory of teletraffic and 
its application to the analysis of multiservice communication of next generation networks~// 
Autom. Control Comp.~S., 2013. Vol.~47. No.\,2. P.~62--69. doi: 
10.3103/S0146411613020028.
\bibitem{14-ost}
\Au{Kingman J.\,F.\,C.} Poisson processes.~--- Oxford studies in probability ser.~--- Claredon Press, 1993. 112~p. doi: 
10.1002/0470011815.B2A07042.
\end{thebibliography}

 }
 }

\end{multicols}

\vspace*{-6pt}

\hfill{\small\textit{Поступила в~редакцию 15.10.22}}

%\vspace*{8pt}

%\pagebreak

\newpage

\vspace*{-28pt}

%\hrule

%\vspace*{2pt}

%\hrule

%\vspace*{-2pt}

\def\tit{ON THE OPTIMAL ANTENNA DEPLOYMENT FOR~SUBTERAHERTZ V2X 
COMMUNICATIONS}


\def\titkol{On the optimal antenna deployment for~subterahertz V2X 
communications}


\def\aut{E.\,A.~Machnev$^1$, V.\,A.~Beschastnyi$^1$, D.\,Yu.~Ostrikova$^1$, 
Yu.\,V.~Gaidamaka$^{1,2}$, and~S.\,Ya.~Shorgin$^2$}

\def\autkol{E.\,A.~Machnev, V.\,A.~Beschastnyi, D.\,Yu.~Ostrikova, et al.} 
%Yu.\,V.~Gaidamaka$^{1,2}$, and~S.\,Ya.~Shorgin$^2$}

\titel{\tit}{\aut}{\autkol}{\titkol}

\vspace*{-8pt}


\noindent
    $^1$Peoples' Friendship University of Russia (RUDN University), 6~Miklukho-Maklaya Str., 
Moscow 117198, Russian\linebreak
$\hphantom{^1}$Federation
    
    \noindent
    $^2$Federal Research Center ``Computer Science and Сontrol'' of the Russian Academy of 
Sciences, 44-2~Vavilov\linebreak
$\hphantom{^1}$Str., Moscow 119333, Russian Federation


\def\leftfootline{\small{\textbf{\thepage}
\hfill INFORMATIKA I EE PRIMENENIYA~--- INFORMATICS AND
APPLICATIONS\ \ \ 2022\ \ \ volume~16\ \ \ issue\ 4}
}%
 \def\rightfootline{\small{INFORMATIKA I EE PRIMENENIYA~---
INFORMATICS AND APPLICATIONS\ \ \ 2022\ \ \ volume~16\ \ \ issue\ 4
\hfill \textbf{\thepage}}}

\vspace*{3pt} 
  
  
    \Abste{Subterahertz (sub-THz, 100--300~GHz) communication should provide huge data 
transfer rates in 6G systems. However, the coverage area of base stations (BS) will be very 
limited, since the signal is quite strongly attenuated from the distance and is also easily blocked 
by the presence of any objects in the signal path. Thus, the BS will need to be located too often 
which is a costly process. To reduce the deployment density of the BS, a~mechanism was 
proposed for relaying the signal using vehicles (V2V). This relaying method is characterized by 
various options for the location of the antenna on vehicles which raises the question of finding 
the optimal location. In this work, guided by the IEEE 802.15.3d specification and measurements 
of the signal propagation level at a~frequency of 300~GHz, the authors developed 
a~mathematical model for comparing multihop signal relay systems with different antenna 
locations. The authors consider the following quality of service indicators: coverage, BS 
availability, and data transfer rate. The results show that the windshield transmitter location has 
a~lower data rate but more coverage while the bumper and engine levels show similar 
performance. A~windshield location is recommended as it is less sensitive to the rate of 
technology integration and has a~larger coverage area.}
  
  \KWE{5G; New Radio; V2V; V2X; multihop communications}
  
 
  
  \DOI{10.14357/19922264220407} 

\vspace*{-8pt}

 \Ack
  \noindent
  The reported study was funded by the Russian Science Foundation, project number 22-29-00694 ({\sf 
https://rscf.ru/en/project/22-29-00694}). 


\vspace*{12pt}

  \begin{multicols}{2}

\renewcommand{\bibname}{\protect\rmfamily References}
%\renewcommand{\bibname}{\large\protect\rm References}

{\small\frenchspacing
 {%\baselineskip=10.8pt
 \addcontentsline{toc}{section}{References}
 \begin{thebibliography}{99}
\bibitem{1-ost-1}
  \Aue{Moltchanov, D., Y.~Gaidamaka, D.~Ostrikova, V.~Bes\-chast\-nyi, Y.~Koucheryavy, and 
K.~Samouylov.} 2021. Ergodic outage and capacity of terahertz systems under micromobility 
and blockage impairments. \textit{IEEE T. Wirel. Commun.} 21(5):3024--3039. 
doi: 10.1109/TWC. 2021.3117583.
\bibitem{2-ost-1}
  \Aue{Moltchanov, D., V.~Beschastnyi, D.~Ostrikova, Y.~Gai\-da\-ma\-ka, and Y.~Koucheryavy.} 
2021. Uninterrupted connectivity time in THz systems under user micromobility and blockage. 
\textit{Global Communications Conference Proceedings}. Piscataway, NJ: IEEE. 
9685384. 6~p. doi: 10.1109/GLOBECOM46510.2021.9685384.
\bibitem{3-ost-1}
  \Aue{Gapeyenko, M., A.~Samuylov, M.~Gerasimenko, D.~Mol\-tcha\-nov, S.~Singh, 
M.\,R.~Akdeniz, E.~Aryafar, N.~Himayat, S.~Andreev, and Y.~Koucheryavy.} 2017. On the 
temporal effects of mobile blockers in urban millimeter-wave cellular scenarios. \textit{IEEE 
T. Veh. Technol.} 66(11):10124--10138. doi: 10.1109/TVT.2017.2754543.
\bibitem{4-ost-1}
  \Aue{Stepanov, N.\,V., D.~Moltchanov, V.~Begishev, A.~Turlikov, and Y.~Koucheryavy.} 
2021. Statistical analysis and modeling of user micromobility for THz cellular communications. 
\textit{IEEE T. Veh. Technol.} 71(1):725--738. doi: 10.1109/TVT.2021.3124870.
\bibitem{5-ost-1}
  \Aue{Beschastnyi, V., D.~Ostrikova, D.~Moltchanov, Y.~Gai\-da\-ma\-ka, Y.~Koucheryavy, and 
K.~Samouylov.} 2022. Balancing latency and energy efficiency in mmWave 5G NR systems 
with multiconnectivity. \textit{IEEE Commun. Lett.} 26(8):1952--1956. doi: 
10.1109/LCOMM.2022.3175663.
\bibitem{6-ost-1}
\Aue{Petrov, V., D.~Moltchanov, S.~Andreev, and R.\,W.~Heath.} 2019. Analysis of intelligent 
vehicular relaying in urban 5G\;+\;millimeter-wave cellular deployments. \textit{Global 
Communications Conference Proceedings}. Piscataway, NJ: IEEE. 05946. 6~p. doi: 
10.48550/arXiv.1908.05946.
  
\bibitem{7-ost-1}
  3GPP. 2020. NR. Study on integrated access and backhaul (Release 17): Technical Specification 38.874 
V17.0.0. Available at: 
{\sf https://www.3gpp.org/ftp/Specs/\linebreak archive/38\_series/38.874/38874-g00.zip} (accessed 
November~28, 2022).
\bibitem{8-ost-1}
  \Aue{Petrov, V., T.~Kurner, and I.~Hosako.} 2020. IEEE 802.15.3d: First standardization 
efforts for sub-terahertz band communications toward 6G. \textit{IEEE Commun. 
Mag.} 58(11):28--33. doi: 10.1109/MCOM.001.2000273
\bibitem{9-ost-1}
  \Aue{Ozpolat, M., K.~Bhargava, E.~Kampert, and M.\,D.~Higgins.} 2021. Multi-lane urban 
mmwave V2V networks: A~path loss behavior dependent coverage analysis. \textit{Vehicular 
Communications} 30:100348. 11 p. doi: 10.1016/ j.vehcom.2021.100348.


\bibitem{10-ost-1}
  \Aue{Wang, J., J.~Liu, and N.~Kato.} 2018. Networking and communications in autonomous 
driving: A~survey. \textit{IEEE Commun. Surv. Tut.} 21(2):1243--1274. doi: 
10.1109/ COMST.2018.2888904.
\bibitem{11-ost-1}
\Aue{Eckhardt, J.\,M., V.~Petrov, D.~Moltchanov, Y.~Koucheryavy, and T.~Kurner.} 2021. 
Channel measurements and modeling for low-terahertz band vehicular communications. 
\textit{IEEE J.~Sel. Area. Comm.} 39(6):1590--1603. doi: 
10.1109/JSAC.2021.3071843.
  
\bibitem{12-ost-1}
\Aue{Moltchanov, D.} 2012. Distance distributions in random networks. \textit{Ad Hoc 
Netw.} 10(6):1146--1166. doi: 10.48550/arXiv.0804.4204. 
\bibitem{13-ost-1}
\Aue{Basharin, G.\,P., Yu.\,V.~Gaidamaka, and K.\,E.~Samouylov.} 2013. Mathematical theory 
of teletraffic and its application to the analysis of multiservice communication of next generation 
networks. \textit{Autom. Control Comp.~S.} 47(2):62--69. doi: 10.3103/S0146411613020028.
  
\bibitem{14ost-1}
  \Aue{Kingman, J.\,F.\,C.} 1993. \textit{Poisson processes}. Oxford studies in probability ser. Claredon Press. 112~p. 
doi: 10.1002/0470011815.B2A07042.

\end{thebibliography}

 }
 }

\end{multicols}

\vspace*{-6pt}

\hfill{\small\textit{Received October 15, 2022}}

\vspace*{-12pt}

  
  \Contr
  
  \vspace*{-3pt}
  
  \noindent
  \textbf{Machnev Egor A.} (b.\ 1996)~--- PhD student, Department of Applied Probability and 
Informatics, Peoples' Friendship University of Russia (RUDN University),  
6~Miklukho-Maklaya Str., Moscow 117198, Russian Federation; \mbox{1032143100@rudn.ru}
  
  \vspace*{3pt}
  
  \noindent
  \textbf{Beschastnyi Vitalii A.} (b.\ 1992)~--- Candidate of Science (PhD) in physics and 
mathematics, assistant professor, Department of Applied Probability and Informatics, Peoples' 
Friendship University of Russia (RUDN University), 6~Miklukho-Maklaya Str., Moscow 
117198, Russian Federation; \mbox{beschastnyy-va@rudn.ru}
  
  
  \vspace*{3pt}
  
  \noindent
  \textbf{Ostrikova Daria Yu.} (b.\ 1988)~--- Candidate of Science (PhD) in physics and 
mathematics, associate professor, Department of Applied Probability and Informatics, Peoples' 
Friendship University of Russia (RUDN University), 6~Miklukho-Maklaya Str., Moscow 
117198, Russian Federation; \mbox{ostrikova-dyu@rudn.ru}
  
  
  
  \vspace*{3pt}
  
  \noindent
  \textbf{Gaidamaka Yuliya V.} (b.\ 1971)~--- Doctor of Science in physics and mathematics, 
professor, Department of Applied Probability and Informatics, Peoples' Friendship University of 
Russia (RUDN University), 6~Miklukho-Maklaya Str., Moscow 117198, Russian Federation; 
senior scientist, Institute of Informatics Problems, Federal Research Center ``Computer Science 
and Control'' of the Russian Academy of Sciences, 44-2~Vavilov Str., Moscow 119333, Russian 
Federation; \mbox{gaydamaka-yuv@rudn.ru}
  
  
  \vspace*{3pt}
  
  \noindent
  \textbf{Shorgin Sergey Ya.} (b.\ 1952)~--- Doctor of Science in physics and mathematics, 
professor, principal scientist, Institute of Informatics Problems, Federal Research Center 
``Computer Science and Control'' of the Russian Academy of Sciences, 44-2~Vavilov Str., 
Moscow 119133, Russian Federation; \mbox{sshorgin@ipiran.ru}
  
\label{end\stat}

\renewcommand{\bibname}{\protect\rm Литература}    
    %5
\def\stat{agalarov}


\def\tit{ПРИБЛИЖЕННЫЙ МЕТОД ВЫЧИСЛЕНИЯ ХАРАКТЕРИСТИК УЗЛА 
ТЕЛЕКОММУНИКАЦИОННОЙ СЕТИ С~ПОВТОРНЫМИ ПЕРЕДАЧАМИ}
\def\titkol{Приближенный метод вычисления характеристик узла 
телекоммуникационной сети с~повторными передачами} 

\def\autkol{Я.\,М.~Агаларов}
\def\aut{Я.\,М.~Агаларов$^1$}

\titel{\tit}{\aut}{\autkol}{\titkol}

%{\renewcommand{\thefootnote}{\fnsymbol{footnote}}\footnotetext[1]
%{Работа выполнена при поддержке РФФИ, проекты 08--07--00152 и 08--01--00567.}}

\renewcommand{\thefootnote}{\arabic{footnote}}
\footnotetext[1]{Институт проблем
информатики Российской академии наук, agglar@yandex.ru}

%\vspace*{-6pt}


\Abst{Рассмотрена модель узла коммутации пакетов c повторными передачами для двух 
схем распределения буферной памяти: полнодоступной и полного разделения. Предложен 
приближенный метод вычисления интенсивностей потоков и вероятностей блокировок узла. 
Получены необходимые и достаточные условия существования и единственности решения 
уравнения для потоков в узле при установившемся режиме работы и доказана сходимость 
итерационного метода решения указанного уравнения.}

\KW{узел коммутации пакетов; буферная память; повторные передачи; вероятности 
блокировок; итерационный метод}

      \vskip 18pt plus 9pt minus 6pt

      \thispagestyle{headings}

      \begin{multicols}{2}

      \label{st\stat}


\section{Введение}

    Одной из основных задач предварительного анализа 
телекоммуникационных сетей коммутации пакетов с ограниченной буферной 
памятью является расчет характеристик потоков и вероятностей блокировок в 
узлах связи. Важность указанных характеристик определяется тем, что от их 
значений существенным образом зависят другие основные показатели сети 
(пропускная способность, задержки пакетов и~др.). 

    Существует множество различных моделей узлов коммутации пакетов и 
методов их расчета (см., например,~[1--6]). Для моделей, рассматривающих 
узел с ограниченной буферной памятью как систему массового обслуживания 
(CMO) типа 
$
\begin{matrix}
M \\ \lambda
\end{matrix}
\left |
\begin{matrix}
M \\ \lambda
\end{matrix}
\right |
\overline{m} \vert N
$ или  $\vert PH\vert PH\vert 1\vert r$, в предположении отсутствия повторных 
передач пакетов получены точные методы вычисления характеристик 
узлов~[1, 3, 4, 6]. Приближенные методы расчета узлов, учитывающие повторные 
попытки передачи, используют модели типа $\vert PH\vert PH\vert 1\vert r$ или 
$
\begin{matrix}
M \\ \lambda
\end{matrix}
\left |
\begin{matrix}
M \\ \lambda
\end{matrix}
\right |
1 \vert N
$ и являются 
итерационными~[2, 3, 5, 7]. Для моделей типа 
$BM\!AP\vert PH\vert 1$, $M\vert G\vert 1\vert r$ и $M\!AP\vert 
(PH,PH)\vert 1$ с повторными заявками получены точные методы вычисления 
характеристик (например, в работах~[8--10]), которые также могут быть 
использованы при расчете узлов.

    Ниже будут рассмотрены модели узла коммутации пакетов с повторными 
передачами для двух схем распределения буферной памяти: с 
полнодоступными буферами и с полным разделением буферной памяти. 
Предлагается приближенный метод расчета характеристик, который в качестве 
модели узла использует СМО типа $
\begin{matrix}
M \\ \lambda
\end{matrix}
\left |
\begin{matrix}
M \\ \lambda
\end{matrix}
\right |
\overline{m} \vert N
$ с повторными заявками. Доказаны утверждения о 
достаточных и необходимых условиях существования и единственности 
решения уравнения для вероятности блокировки в установившемся режиме 
работы и сходимости предлагаемого итерационного метода. 

\section{Модель узла}

    Математическая модель узла представляется в виде СМО с ограниченной 
буферной памятью и различными потоками заявок, каждая из которых требует 
обслуживания только на одной из многоканальных линий связи. 

    Пусть $0<N<\infty$~--- число мест хранения в буферной памяти, $u$~--- 
узел связи, $v$~--- линия связи, $\Omega_u^+$~--- множество исходящих из 
узла~$u$ линий, $c_v$~--- канальная емкость линии~$v$. Поток заявок, 
тре\-бу\-ющих обслуживания на линии~$v$, назовем $v$-по\-то\-ком, заявки этого 
потока~--- $v$-за\-яв\-ка\-ми.


    Пусть выполняются следующие предположения: 
\begin{enumerate}[1.]
\item Места в буферной памяти распределяются согласно одной из двух 
схем:
\begin{enumerate}[($i$)]
\item полнодоступная схема~--- каждое свободное место хранения доступно 
любой заявке;
\item схема полного разделения памяти~--- $v$-за\-яв\-кам доступны всего 
$N_v$ мест, где $\sum\limits_{v\in\Omega_u^+} N_v=N$.
\end{enumerate}
\item Если в момент поступления $v$-заявки в буферной памяти есть 
доступное свободное место, то она сразу занимает это место. Если в момент 
поступления $v$-заявки в системе нет свободного доступного места 
хранения, то поступившая заявка через некоторое время повторно поступает 
на систему, оставаясь $v$-заявкой. 
\item Интенсивности первичных потоков $v$-заявок~--- заданные величины 
$0<\Lambda_v<\infty$, $v\in \Omega_u^+$. Суммарные потоки первичных и 
повторных $v$-заявок являются независимыми в совокупности 
пуассоновскими потоками. Для обслуживания $v$-заявки требуется 
одновременно одно место хранения и один канал типа~$v$, $v\in 
\Omega_u^+$.
\item Первичные нагрузки~--- реализуемые, т.\,е.\ в данном случае 
интенсивности входных первичных потоков равны интенсивностям 
выходных потоков выполненных заявок. 
\item Принятые в СМО $v$-заявки обслуживаются линией~$v$ в порядке 
поступления. 
\item Время занятия канала $v$-заявкой~--- экспоненциально 
распределенная случайная величина с параметром $0<\mu_v<\infty$, 
$v\in\Omega_u^+$, независимая от других случайных событий в узле.
\item Выполненная $v$-заявка с вероятностью~$B_v$ повторяется через 
заданное время~$\tau_v$ (тайм-аут) и с вероятностью $1-B_v$ покидает 
систему через время~$t_v$ навсегда, сразу освободив занятый канал и место 
буферной памяти.
\end{enumerate}

   Будем говорить, что узел блокирован для $v$-за\-яв\-ки, если в буферной 
памяти отсутствует доступное место хранения. Ставится задача вычисления 
вероятностей блокировок и интенсивностей потоков в узле.

\section{Вычисление вероятности блокировки и~интенсивностей~потоков} 

   Пусть $\Lambda_v^*$~--- интенсивность суммарного потока внешних 
заявок, требующих передачи по линии~$v$, $\pi_v$~--- вероятность блокировки 
узла для заявок, требующих передачи по исходящей из узла линии~$v$. 

    Пусть в узле используется полнодоступная схема распределения 
буферной памяти. Тогда, как следует из описания модели, $\pi_v 
=\pi_{v^\prime},\,v,\,v^\prime\in \Omega_u^+$, и для 
интенсивностей~$\Lambda_v^*$, $v\in\Omega_u^+$, справедливы соотношения:
\begin{equation*}
\Lambda_v^* = \fr{\Lambda_v}{1-\pi}\,,
%\label{e1aga}
\end{equation*}
    где
    $\pi =\pi_v$, $v\in\Omega_u^+$.

    Пусть 
    $\overline{k} = \{\overline{k}_v$, $v\in\Omega_u^+\}$~--- состояние 
буферной памяти узла, $\overline{k}_v =\left ( k_v,\,k_v^\prime,\,k_v^{\prime\prime}\right )$; 
$k_v$~--- число $v$-заявок в буферной 
памяти, ожидающих выполнения линией~$v$; $k^\prime_v$~--- число 
$v$-заявок в буферной памяти, ожидающих тайм-аут и неуспешно переданных 
в последующий узел; $k_v^{\prime\prime}$~--- число $v$-за\-явок в буферной 
памяти, успешно переданных в последующий узел и ожидающих 
потверждения; 
$A_m = \left \{ \overline{k}:\ \sum\limits_{v\in\Omega_u^+} \left ( 
k_v+k_v^\prime + k_v^{\prime\prime}\right ) =m \right \}$~--- множество различных 
состояний, при которых в памяти узла занято ровно $m$~буферов. Тогда с 
учетом введенных выше обозначений и предположений для ве\-ро\-ят\-ности 
блокировки узла можно написать формулу~\cite{1aga, 2aga}:
\begin{equation}
\pi = \fr{1}{G_N}\sum\limits_{\overline{k}\in A_N} 
p\left (\overline{k},\overline{\rho}^*\right )\,,
\label{e2aga}
\end{equation}
где  
\begin{gather}
p(\overline{k},\overline{\rho}^*) = \prod\limits_{v\in\Omega_u^+} z_v (\pi, 
\rho_v , k_v , k_v^\prime , k_v^{\prime\prime})\,;\\
z_v (\pi, \rho_v , k_v , k_v^\prime , k_v^{\prime\prime}) ={}\notag\\
\!\!{}=
\begin{cases}
 \fr{\rho_v^{\prime *k_v^\prime}}{k_v^{\prime}!}\,
\fr{\rho_v^{\prime\prime * k_v^{\prime\prime}}}{ k_v^{\prime\prime}!}  \,
\fr{\rho_v^{*k_v}}{ k_{v}!} 
&\mbox{при}\ k_v<c_v\,,\\
 \fr{\rho_v^{\prime * k_v^\prime}}{k_v^{\prime}!} \,
\fr{\rho_v^{\prime\prime * k_v^{\prime\prime}}} { k_v^{\prime\prime}!} 
\fr{\rho_v^{*k_v}}{ c_{v}!c_v^{k_v- c_v}} 
& \mbox{при}\ k_v\geq c_v\,;
\end{cases}\\
G_N = \sum\limits_{m=0}^N\sum\limits_{\overline{k}\in A_m}
p(\overline{k},\overline{\rho}^*)\,;\\ 
\overline{\rho}^*=\{\rho_v^*,\,v\in\Omega_u^+\}\,;\\
\rho_v^* = \fr{\rho_v}{1-\pi}\,;\quad \rho_v =\fr{\Lambda_v}{\mu_v(1- B_v)}\,;\\
\rho_v^{\prime *} =\rho_v^*\mu_v\tau_vB_v\,;\quad \rho_v^{\prime\prime *}=
p_v^* \mu_vt_v,\,\quad  v\in \Omega_u^+\,.\label{e3aga}
\end{gather}

Переобозначив $1-\pi$ через $y$, выражение в правой части равенства~(2)~--- через 
$p_{\overline{k}}(\overline{\rho},y)$, выражение в правой части равенства~(4)~--- 
через $g_N(\overline{\rho},y)$, а выражение в правой 
части равенства~(1)~--- через $1-q_N (\overline{\rho},y)$, 
где $\overline{\rho} = (\rho_v,\,v\in \Omega_u^+)$, $\rho_v = \rho_v^*y\;=$\linebreak 
$=\;\Lambda_v/(\mu_v(1-B_v))$, $v\in\Omega_u^+$, получим нелинейное уравнение 
относительно неизвестной переменной~$y$:
\begin{equation}
y=q_N(\overline{\rho},y)\,.
\label{e4aga}
\end{equation}

    Решим уравнение~(8). Как следует из~(2)--(7), верно 
равенство
\begin{equation}
q_N(\overline{\rho},y) = \fr{g_{N-1}(\overline{\rho},y )}{g_N(\overline{\rho},y)}\,.
\label{e5aga}
\end{equation}
Введем функцию  $d_n(\overline{\rho} ,y)$ среднего числа заявок в узле с 
буферной памятью емкости $n\geq 0$:
$$
d_n(\overline{\rho} ,y) = 
\fr{1}{g_n(\overline{\rho},y)}\,\sum\limits_{m=0}^n m\sum\limits_{\overline{k}\in 
A_m} p_{\overline{k}}(\overline{\rho},y)\,.
$$
Заметим, что $g_n$, $d_n$ и $q_n$, 
$n\geq 0$,~--- непрерывно-дифференцируемые функции по $y\in (0,\,1]$. Взяв 
производную функции~$g_n$ по~$y$, из~(2)--(7) получим
\begin{multline}
\fr{\partial g_n(\overline{\rho},y)}{\partial y} ={}\\
{}= -\sum\limits_{m=0}^n m 
\sum\limits_{\overline{k}\in A_m}\fr{\prod\limits_{v\in\Omega_u^+} z_n 
(0,\rho_v, k_v, k_v^\prime , k_v^{\prime\prime})}{y^{m+1}}={}\\
{}= -\fr{1}{y}\,g_n (\overline{\rho},y)d_n(\overline{\rho},y)\,.
\label{e6aga}
\end{multline}
Взяв производную функции $q_N$ по $y$, из~(\ref{e5aga}) и~(\ref{e6aga}) 
получим
\begin{equation}
\fr{\partial q_N(\overline{\rho},y)}{\partial y} = \fr{q_N(\overline{\rho},y)}{y}\left 
[ d_N (\overline{\rho},y)-d_{N-1}(\overline{\rho},y)\right ]\,.
\label{e7aga}
\end{equation}
    Докажем несколько утверждений о свойствах 
функции~$q_N(\overline{\rho},y)$.
\medskip

\noindent
\textbf{Утверждение 1.} \textit{Справедливы неравенства}
\begin{multline}
0<d_{n+1}(\overline{\rho},y)-d_n(\overline{\rho},y) <1\,,\\
\ \ \ \ \ \ \ \ \ \ \ \ \ \ \ \ \ \ \ \ y\in (0,\,1]\,, \ n\geq 0\,.
\label{e8aga}
\end{multline}


\noindent

Д\,о\,к\,а\,з\,а\,т\,е\,л\,ь\,с\,т\,в\,о\,.\ Подставив выражение для функции 
$d_n(\overline{\rho},y)$ и проведя преобразования, получим
\begin{multline*}
d_{n+1}(\overline{\rho},y) -d_n(\overline{\rho},y) = 
\fr{\sum\limits_{m=0}^{n+1}m\sum\limits_{\overline{k}\in A_m} 
p_{\overline{k}}(\overline{\rho},y)}
{\sum\limits_{m=0}^{n+1}
\sum\limits_{\overline{k}\in A_m} p_{\overline{k}}(\overline{\rho},y)} - {}\\
{}-
\fr{\sum\limits_{m=0}^n m \sum\limits_{\overline{k}\in A_m} p_{\overline{k}} 
(\overline{\rho},y)}{\sum\limits_{m=0}^n
\sum\limits_{\overline{k}\in A_m}p_{\overline{k}}(\overline{\rho},y)}={}\\
{}=\fr{\sum\limits_{m=1}^n m \sum\limits_{\overline{k}\in 
A_m}p_{\overline{k}}(\overline{\rho},y)+(n+1)\sum\limits_{\overline{k}\in 
A_{n+1}}  p_{\overline{k}}(\overline{\rho},y)}{\sum\limits_{m=0}^n\sum\limits_{\overline{k
}\in A_m}p_{\overline{k}}(\overline{\rho},y)+\sum\limits_{\overline{k}\in 
A_{n+1}}p_{\overline{k}}(\overline{\rho},y)} -{}
\end{multline*}
\begin{multline}
{}-
\fr{\sum\limits_{m=0}^n m 
\sum\limits_{\overline{k}\in A_m}p_{\overline{k}}(\overline{\rho},y)}
{\sum\limits_{m=0}^n\sum\limits_{\overline{k}\in A_m} 
p_{\overline{k}}(\overline{\rho},y)}={}\\
{}=\fr{(n+1)\sum\limits_{\overline{k}\in 
A_{n+1}}p_{\overline{k}}(\overline{\rho},y)g_n(\overline{\rho},y)}{g_{n+1}(\overline{\rho},y) g_n(\overline{\rho},y)} -{}\\
{}-
\fr{\sum\limits_{\overline{k}\in 
A_{n+1}}p_{\overline{k}}(\overline{\rho},y)\sum\limits_{m=0}^n  m 
\sum\limits_{\overline{k}\in A_m} p_{\overline{k}}(\overline{\rho},y) }
{g_{n+1}(\overline{\rho},y) g_n(\overline{\rho},y)}
={}\\
{}=\left [ 1-q_{n+1}(\overline{\rho},y)\right ] \left [n+1-d_n(\overline{\rho},y)\right ]\,.
\label{e9aga}
\end{multline}


    Докажем утверждение~1 методом индукции. При $n = 0$, как следует 
из~(\ref{e9aga}), имеем
$$
d_2(\overline{\rho},y) - d_1 (\overline{\rho},y) =1-q_1(\overline{\rho},y)\,,
$$
    т.\,е.\ утверждение~1 при $n = 0$ справедливо. 

    Пусть неравенства~(\ref{e8aga}) справедливы для некоторого $n > 0$. 
Докажем, что они справедливы и для $n + 1$. Из~(\ref{e9aga}) получаем
\begin{multline*}
d_{n+1}(\overline{\rho},y)- d_n(\overline{\rho},y)={}\\
{}=\left [ 1-
q_{n+1}(\overline{\rho},y)\right ] \left [n+1-d_n(\overline{\rho},y)\right ] ={}\\
{}= \left [ 1-
1-q_{n+1}(\overline{\rho},y)\right ] \left [ n-{}\right.\\
{}-\left. d_{n-1}(\overline{\rho},y)+d_{n-1}(\overline{\rho},y)-
d_n(\overline{\rho},y)+1\right ] ={}\\
{}=\left [ 1-q_{n+1}(\overline{\rho},y)\right ] 
\left [ n-d_{n-1}(\overline{\rho},y)-{}\right.\\
{}-\left. \left ( d_n(\overline{\rho},y)-d_{n-1}(\overline{\rho},y)\right )+1\right] = {}\\
{}=
\left [ 1-q_{n+1}(\overline{\rho},y)\right ]
\left [ 
\fr{d_n(\overline{\rho},y) -d_{n-1}(\overline{\rho},y)}{1-
q_n(\overline{\rho},y)}\right.-{}\\
{}-\left.
\left ( d_n(\overline{\rho},y)-d_{n-1}(\overline{\rho},y)\right )+1
\vphantom{\fr{d_n(\overline{\rho})}{(q_n)}}
\right ]={}\\
{}=
\left [ 1-q_{n+1}(\overline{\rho},y)\right ]
\left [ 
\vphantom{\fr{d_n(\overline{\rho})}{(q_n)}}
\left ( d_n(\overline{\rho},y\right)\right. -{}\\
 {}-\left.
d_{n-1}\left(\overline{\rho},y)\right )\fr{q_n(\overline{\rho},y)}{1-
q_n(\overline{\rho},y)}+1\right ]\,.
\end{multline*}
Так как по предположению $d_n (\overline{\rho},y) -d_{n-1}(\overline{\rho},y) 
>0$, то правая часть последнего равенства больше нуля; следовательно, 
$d_{n+1}(\overline{\rho},y)-d_n(\overline{\rho},y)>0$. 

    Продолжив преобразование правой части последнего равенства и 
учитывая предположение $d_n(\overline{\rho},y) -d_{n-1}(\overline{\rho},y)<1$, 
получим
\begin{multline*}
d_{n+1}((\overline{\rho},y) -d_n(\overline{\rho},y)<{}\\
{}< \left [ 1-
q_{n+1}(\overline{\rho},y)\right ]
\left ( \fr{q_n(\overline{\rho},y)}{1-q_n(\overline{\rho},y)}+1\right )={}\\
{}=
\fr{1-q_{n+1}(\overline{\rho},y)}{1-q_n(\overline{\rho},y)}<1\,,
\end{multline*}
так как $0< q_n(\overline{\rho},y)<q_{n+1}(\overline{\rho},y)<1$, $n>0$, $y\in 
(0,\,1]$.

Следовательно, утверждение~1 доказано.

\medskip

\noindent
\textbf{Утверждение 2.} $q_N(\overline{\rho},y)$~--- \textit{монотонно-воз\-рас\-та\-ющая 
функция по $y\in (0,\,1]$. При этом $0< q_N(\overline{\rho},y)\;\leq $\linebreak 
$\leq\;q_N(\overline{\rho},1) <1$, $y\in (0,\,1]$,  и $\underset{y\rightarrow 
0}{\mathrm{lim}}\,q_N(\overline{\rho},y) =0$}.

\medskip

\noindent
Д\,о\,к\,а\,з\,а\,т\,е\,л\,ь\,с\,т\,в\,о\,.\  Возрастание функции 
$q_N(\overline{\rho},y)$ следует непосредственно из~(\ref{e7aga}) и 
утверж\-де\-ния~1. Доказательство неравенств в условии утверждения очевидно 
следует из~(\ref{e5aga}) и вида функции $g_n (\overline{\rho},y)$, $n\geq 0$. 
Для предела имеем:
\begin{multline*}
\underset{y\rightarrow 0}{\mathrm{lim}}\,q_N(\overline{\rho},y) 
=\underset{y\rightarrow 0}{\mathrm{lim}}\,\fr{g_{N- 1}(\overline{\rho},y)}{g_N(\overline{\rho},y)} = {}\\
{}= \underset{y\rightarrow 0}{\mathrm{lim}}\,\left (
g_{N-1}(\overline{\rho},y)\Bigg / \left ( 
\vphantom{\prod\limits_{v\in\Omega_u^+}}
g_{N-1}(\overline{\rho},y)\right.\right.+{}\\
{}+\left.\left.\sum\limits_{\overline{k}\in A_N}\prod\limits_{v\in\Omega_u^+} 
\fr{z_v(0,\rho_v,k_v,k^\prime_v,k^{\prime\prime}_v)}{y^N}\right )\right ) = {}\\
{}= \underset{y\rightarrow 0}{\mathrm{lim}}\,\left (
y^N g_{N-1}(\overline{\rho},y)\Bigg / 
\left ( 
\vphantom{\prod\limits_{v\in\Omega_u^+}}
y^N g_{N-1}(\overline{\rho},y)+{}\right.\right.\\
{}+\left.\left.\sum\limits_{\overline{k}\in A_N}
\prod\limits_{v\in\Omega_u^+} z_v(0,\rho_v,k_v,k_v^\prime , k_v^{\prime\prime}) 
\right ) \right )=0\,.
\end{multline*}
    
\medskip

\noindent
\textbf{Утверждение 3.} \textit{Производная функции~$q_N (\overline{\rho},y)$ по 
$y\in (0,\,1]$ удовлетворяет следующим соотношениям}:
\begin{align}
\underset{y\rightarrow 0}{\mathrm{lim}}\fr{\partial q_N(\overline{p},y)}
{\partial  y} &= \fr{\sum\limits_{\overline{k}\in A_{N-1}} 
p_{\overline{k}}(\overline{\rho},1)}{\sum\limits_{\overline{k}\in 
A_N}p_{\overline{k}}(\overline{\rho},1)}\,;\label{e10aga}\\
\fr{\partial q_N(\overline{\rho},y)}{\partial y}\Big |_{y=1}&<1\,.\label{e11aga}
\end{align}

\medskip

\noindent
Д\,о\,к\,а\,з\,а\,т\,е\,л\,ь\,с\,т\,в\,о\,.\ Проведя преобразования 
функции~$q_N(\overline{\rho},y)$, получим:
\begin{multline*}
\underset{y\rightarrow 0}{\mathrm{lim}}\fr{q_N(\overline{\rho},y)}{y} = {}\\
\!\!{}=
\underset{y\rightarrow 0}{\mathrm{lim}}
\fr{\sum\limits_{m=0}^{N-1}\sum\limits_{\overline{k}\in A_m}
\!\!\left (\prod\limits_{v\in\Omega_u^+}\!\! 
z_v(0,\rho_v,k_v,k_v^\prime , k_v^{\prime\prime})\right )\!\!\Bigg /\!\! y^m}
{y\sum\limits_{m=0}^{N}\sum\limits_{\overline{k}\in A_m}
\!\!\left(\prod\limits_{v\in\Omega_u^+}\!\! z_v\left (0,\rho_v,k_v,k_v^\prime , 
k_v^{\prime\prime}\right )\right )\!\!\Bigg /\!\!y^m} = \!\!\!
\end{multline*}
\begin{multline*}
\!\!\!\!\!\!{}=\underset{y\rightarrow 0}{\mathrm{lim}}\,
\fr{\sum\limits_{m=0}^{N-1}\sum\limits_{\overline{k}\in A_m}
y^{N-1-m}\prod\limits_{v\in\Omega_u^+} z_v(0,\rho_v,k_v,k_v^\prime , 
k_v^{\prime\prime})}{\sum\limits_{m=0}^{N}\sum\limits_{\overline{k}
\in A_m} y^{N-m}
\prod\limits_{v\in\Omega_u^+} z_v(0,\rho_v,k_v,k_v^\prime , 
k_v^{\prime\prime})}={}\!\\
{}=\fr{\sum\limits_{\overline{k}\in A_{N-1}} p_{\overline{k}}(\overline{\rho},1)}{ 
\sum\limits_{\overline{k}\in A_{N}} p_{\overline{k}}(\overline{\rho},1)}\,.
\end{multline*}
Очевидно, $\underset{y\rightarrow 0}{\mathrm{lim}} \,[d_N (\overline{\rho},y) -
d_{N-1} (\overline{\rho},y)]=1$, так как $\underset{y\rightarrow 
0}{\mathrm{lim}}\,d_n (\overline{\rho},y)=n$, $n>0$.

Следовательно, учитывая~(\ref{e7aga}), получаем~(\ref{e10aga}). 
Справедливость~(\ref{e11aga}) непосредственно следует из~(\ref{e7aga}) и 
утверждения~1.

\medskip

\noindent
\textbf{Утверждение 4.} \textit{Пусть $y^*\in (0,\,1]$~--- решение 
уравнения}~(\ref{e4aga}). \textit{Тогда}
\begin{equation*}
\fr{\partial q_N(\overline{\rho},y)}{\partial y}\Big |_{y=y^*}<1\,.
%\label{e12aga}
\end{equation*}

\medskip

\noindent
Д\,о\,к\,а\,з\,а\,т\,е\,л\,ь\,с\,т\,в\,о\,.\ \ Доказательство следует из~(\ref{e7aga}), 
так как $q_N(\overline{\rho},y^*)/y^* =1$.
\medskip

\noindent
\textbf{Утверждение 5.} \textit{Уравнение}~(\ref{e4aga}) \textit{имеет решение $y^*\in 
(0,\,1)$ тогда и только тогда, когда} 
\begin{equation}
\fr{\sum\limits_{\overline{k}\in A_{N-1}} p_{\overline{k}}(\overline{\rho},1)}{ 
\sum\limits_{\overline{k}\in A_{N}} p_{\overline{k}}(\overline{\rho},1)} >1\,.
\label{e13aga}
\end{equation}
\textit{Если уравнение}~(\ref{e4aga}) \textit{имеет решение $y^*\in (0,\,1)$, то оно 
единственное положительное решение}.
\medskip

\noindent
Д\,о\,к\,а\,з\,а\,т\,е\,л\,ь\,с\,т\,в\,о\,.\ Пусть выполняется 
неравенство~(\ref{e13aga}). Тогда, как следует из утверждения~3, 
$\underset{y\rightarrow 0}{\mathrm{lim}} (\partial q_N(\overline{\rho},y)/\partial y) 
>1$. Кроме того, как следует из утверждения~2, 
$\underset{y\rightarrow 0}{\mathrm{lim}} q_N(\overline{\rho},y)=0$. Тогда, так 
как $q_N(\overline{\rho},y)$~--- непрерывно-дифференцируемая функция по 
$y\in (0,\,1]$, существует значение $y^\prime \in (0,\,1)$ такое, что 
$q_N(\overline{\rho},y)>y$ для всех $y\in (0,\,y^\prime]$ (следует из теоремы о 
конечном приращении~\cite{11aga}). В то же время, согласно утверждению~2, 
$q_N(\overline{\rho},y)<y$ в окрестности точки $y=1$ (рис.~\ref{f1aga},\,\textit{а}). 
Следовательно, кривая $x=q_N(\overline{\rho},y)$ пересекает прямую $x=y$ 
хотя бы в одной точке $y=y^*\in (0,\,1)$, т.\,е.\ уравнение~(\ref{e4aga}) имеет 
хотя бы одно решение $y^*\in (0,\,1)$.

\begin{figure*}
\vspace*{1pt}
\begin{center}
\vspace*{1pt}
\mbox{%
\epsfxsize=158mm
\epsfbox{aga-1.eps}
}
\end{center}
\vspace*{-9pt}
\Caption{Примеры кривых $x=q_N(\overline{\rho},y)$ и $x=y$ (\textit{а})~при существовании решения 
уравнения~(\ref{e5aga}) и (\textit{б})~при выполнении условий~(17)
\label{f1aga}}
\vspace*{6pt}
\end{figure*}

Пусть уравнение~(\ref{e4aga}) имеет решение $y^*\in (0,\,1)$ и 
\begin{equation}
\fr{\sum\limits_{\overline{k}\in A_{N-1}}p_{\overline{k}}(\overline{\rho},1)}{ 
\sum\limits_{\overline{k}\in A_{N}}p_{\overline{k}}(\overline{\rho},1)}\leq 
1\,.\label{e14aga}
\end{equation}
Тогда из условий утверждений~2 и~3 следует, что 
уравнение~(\ref{e4aga}) в интервале $(0,\,1)$ имеет более одного решения, что 
может быть только при существовании решения $y^\prime \in (0,\,1)$ такого, 
что в окрестности точки $y=y^\prime$ выполняются неравенства: 
$q_N(\overline{\rho},y)<y$ при $y<y^\prime$ и $q_N(\overline{\rho},y)>y$ при 
$y>y^\prime$, где $y$ принадлежит указанной окрест\-ности точки~$y^\prime$ 
(рис.~\ref{f1aga},\,\textit{б}). Тогда в точке $y=y^\prime$ производная функции 
$q_N(\overline{\rho},y)$ по $y$ больше~1, что противоречит утверждению~4. 
Следовательно, неравенство~(\ref{e13aga}) справедливо.


Пусть уравнение~(\ref{e4aga}) имеет более одного положительного 
решения. Рассуждая точно так же, как и выше (в случае выполнения 
условий~(\ref{e14aga})), получим противоречие утверждению~4. 
Следовательно, утверждение~5 справедливо.
\medskip

\noindent
\textbf{Следствие.} \textit{Неравенства}
\begin{gather*}
\fr{\mu_v c_v (1-B_v)}{\Lambda_v}>1\,,\quad \fr{1-B_v}{\Lambda_v \tau_v B_v}>1\,,\\ 
\fr{1-B_v}{\Lambda_v t_v}>1\,,\ v\in\Omega_u^+\,,
\end{gather*}
\textit{являются необходимым условием существования решения 
уравнения}~(\ref{e4aga}).

\medskip
\noindent
Д\,о\,к\,а\,з\,а\,т\,е\,л\,ь\,с\,т\,в\,о\,.\ Пусть $\overline{k}_v$~--- это 
набор~$\overline{k}$, у которого $k_v=0$. Преобразовав левую 
часть~(\ref{e13aga}), получим

\noindent
\begin{multline*}
\fr{\sum\limits_{\overline{k}\in A_{N-1}} p_{\overline{k}} (\overline{\rho},1)}
{ \sum\limits_{\overline{k}\in A_{N}} 
 p_{\overline{k}}(\overline{\rho},1)} 
={}
\\
{}=
\fr{\sum\limits_{\overline{k}\in A_{N-1}}\prod\limits_{v\in \Omega_u^+} 
z_v\left(0,\rho_v,k_v,k_v^\prime , k_v^{\prime\prime}\right)}
{\sum\limits_{\overline{k}\in A_{N}}
\prod\limits_{v\in \Omega_u^+} z_v\left (0,\rho_v,k_v,k_v^\prime , k_v^{\prime\prime}\right )} \leq{}
\\
{}\leq
\left ( 
\vphantom{\prod\limits_{v^\prime\in\Omega_u^+\backslash v}}
\fr{\mu_v c_v(1-B_v)}{\Lambda_v}\right. \times{}\\
{}\times \sum\limits_{k_v=0}^{N-1}\sum\limits_{\overline{k}_v\in A_{N-1-k_v}} z_v\left(0,\rho_v,k_v+1,k_v^\prime , 
k_v^{\prime\prime}\right )\times{}\\
{}\times \left.\prod\limits_{v^\prime\in\Omega_u^+\backslash v} z_v^\prime 
\left(0,\rho_v,k_v,k_v^\prime , k_v^{\prime\prime}\right) \right)
\Bigg /{}\\
\Bigg / \left ( 
\vphantom{\prod\limits_{v^\prime\in\Omega_u^+\backslash v}}
\sum\limits_{k_v=0}^{N-1} \sum\limits_{\overline{k}_v\in A_{N-1-k_v}} z_v 
\left (0,\rho_v,k_v+1,k_v^\prime , 
k_v^{\prime\prime}\right )\right. \times{}\\
{}\times \prod\limits_{v^\prime\in\Omega_u^+\backslash v} 
z_{v^\prime}\left(0,\rho_v,k_v,k^\prime , k_v^{\prime\prime}\right)+{}\\
{}+
\sum\limits_{\overline{k}_v\in A_N} z_v\left (0,\rho_v, 0,k_v^\prime , 
k_v^{\prime\prime}\right)\times{}\\
\left.{}\times \prod\limits_{v^\prime\in\Omega_u^+\backslash v}z_{v^\prime} 
\left(0,\rho_v,k_v,k_v^\prime , k_v^{\prime\prime}\right )\right )\,.
\end{multline*}
Как следует из правой части последнего неравенства, если 
$\mu_v c_v (1-B_v)/\Lambda_v \leq 1$, то она меньше~1. Поэтому, чтобы 
выполнилось условие~(\ref{e13aga}), необходимо выполнение первого 
неравенства в условии следствия для каждого $v\in\Omega_u^+$. Точно так же 
доказывается необходимость выполнения второго и третьего неравенств в 
условии следствия.

    Пусть $y[n]$, $n\geq 0$, последовательность, полученная по формуле 
$y[n+1]=q_N(\overline{\rho},y[n])$, $y[0]=1$.

\medskip

\noindent
\textbf{Утверждение 6.} \textit{Пусть $y^*\in (0,\,1)$~--- решение 
уравнения}~(8). \textit{Тогда последовательность $y[n]$, $n\geq 0$, сходится 
к решению~$y^*$}.

\medskip

\noindent
Д\,о\,к\,а\,з\,а\,т\,е\,л\,ь\,с\,т\,в\,о\,.\ Отметим, что $y[1]<y[0]$ (это следует из 
утверждения~2, так как $y[0]=1$). Пусть для некоторого $n>1$ выполняется 
условие $y[n]<y[n-1]$. Тогда, как следует из утверждения~2, указанное условие 
выполняется и для $n+1$, т.\,е.\ по индукции следует, что последовательность 
$y[n]$, $n\geq 0$, монотонно убывает. 

    Пусть для некоторого $n>0$ $y[n]>y^*$ (существование такого $n$ 
следует из равенства $y[0]=1$). Тогда, как следует из утверждения~2, 
$y[n+1]\;=$\linebreak $=\;q_N(\overline{\rho},y[n])>q_N(\overline{\rho},y^*) =y^*$, т.\,е.\ 
последовательность ограничена снизу величиной~$y^*$. Значит, существует 
$\underset{n\rightarrow \infty}{\mathrm{lim}}\,y[n]=y^0\geq y^*$. Так как 
$q_n(\overline{\rho},y)$~--- непрерывная по~$y$ функция, то можно написать 
$\underset{n\rightarrow 
\infty}{\mathrm{lim}}\,q_N(\overline{\rho},y[n])=q_N(\overline{\rho},y^0)=y^0$, 
т.\,е.\ $y^0$~--- решение уравнения~(\ref{e4aga}). Из единственности 
положительного решения уравнения~(\ref{e4aga}) получаем $y^0=y^*$.

    Пусть в узле используется схема полного разделения буферной памяти. 
Тогда для интенсив\-ностей~$\Lambda_v^*$, $v\in\Omega_u^+$, справедливы 
соотношения:
$$
\Lambda_v^* = \fr{\Lambda_v}{1-\pi_v}\,,
$$
где $v\in\Omega_u^+$.


Фиксируем произвольную линию сети~$v$. Пусть $\overline{k}_v = (k_v, 
k_v^\prime, k_v^{\prime\prime})$~--- состояние буферной памяти линии~$v$; 
$k_v$, $k_v^\prime$, $k_v^{\prime\prime}$ определены выше. Тогда с 
учетом введенных ранее предположений и обозначений для вероятности 
блокировки линии справедлива формула~\cite{4aga}:
\begin{equation}
\pi_v = \fr{1}{G_{N_v}}\sum\limits_{k_v=N_v} 
z_v(\pi_v,\rho_v,\overline{k}_v)\,,
\label{e15aga}
\end{equation}
где 
\begin{multline*}
z_v(\pi_v, \rho_v, \overline{k}_v)={}\\
{}=
\begin{cases}
\fr{\rho_v^{\prime * k_v^\prime}}{k_v^\prime !}\,
 \fr{\rho_v^{\prime\prime * k_v^{\prime\prime}}}{k_v^{\prime\prime}!}\,
 \fr{\rho_v^{*k_v}}{k_v !} & \mbox{при}\ k_v<c_v\,,\\
 \fr{\rho_v^{\prime *k_v^\prime}}{k_v^{\prime }! }
 \fr{\rho_v^{\prime\prime * k_v^{\prime\prime}}}{k_v^{\prime\prime}!}
\fr{\rho_v^{*k_v}}{c_v !c_v^{k_v-c_v}} & \mbox{при}\ k_v\geq c_v\,;
\end{cases}
\end{multline*}
\begin{align*}
G_{N_v} &= \sum\limits_{m=0}^{N_v} z_v (\pi_v ,\rho_v , \overline{k}_v)\,;\\ 
\rho_v^*&=\fr{\rho_v}{1-\pi_v}\,;
\end{align*}
$\rho_v$, $\rho_v^{\prime *}$, 
$\rho_v^{\prime\prime *}$, $v\in\Omega_u^+$ определены выше.
    
Пусть $y_v=1-\pi_v$, а $q_{N_v} (\rho_v, y_v)$~--- выражение в правой 
части~(\ref{e15aga}). Тогда из равенств~(\ref{e15aga}), взяв~$y_v$ в качестве 
неизвестной переменной, получим систему независимых уравнений:
\begin{equation}
y_v = q_{N_v}(\rho_v, y_v)\,, \quad v\in \Omega_u^+\,.
\label{e16aga}
\end{equation}
    
    Заметим, что для фиксированной $v$ и заданных параметров $\Lambda_v$, 
$\mu_v$, $\tau_v$, $t_v$, $N_v$, $v\in\Omega_u^+$, уравнение в~(\ref{e16aga}) 
является частным случаем уравнения~(\ref{e4aga}) и совпадает с последним, 
когда число исходящих линий из узла равно~1. Следовательно, для схемы 
полного разделения памяти также справедливы все приведенные выше 
утверждения~1--6 и следствие. Заметим, что неравенство~(\ref{e13aga}) в 
условии утверждения~5 при $B_v=0$ и $t_v=0$ преобразуется в неравенство 
$\Lambda_v / (\mu_v c_v) >1$, $v\in\Omega_u^+$. Последовательность 
$\overline{y}[n]$, $n\geq 0$, в утверждении~6 определяется по формуле:
    \begin{gather*}
    \overline{y}[n] =\{y_v[n],\ v\in\Omega_u^+\}\,,\
    y_v[n+1]=q_{N_v} (\rho_v,\,y_v[n])\,,\\
    y_v[0] =1\,,\quad n\geq 0\,,\quad v\in \Omega_u^+\,.
    \end{gather*}


\section{Алгоритм расчета} %4

    Для вычисления интенсивностей потоков и вероятностей блокировок в 
узле предлагается следующий алгоритм, описывающий изложенную выше 
итерационную процедуру. Введем обозначения:
$y_u^v$~--- вероятность блокировки узла для заявок, направляемых на 
линию~$v$,
\begin{gather*}
y_u^v  = 
\begin{cases}
y_u & \mbox{для}\ v\in\Omega_u^+\ \mbox{при}\\
&\mbox{полнодоступной схеме},\\
y_v & \mbox{при схеме полного распределения}\\
&\mbox{памяти};
\end{cases}
\\
q_N^v(\overline{\rho}_u^{-v}, y_u^v)  = 
\begin{cases}
q_N(\overline{\rho},y) & \mbox{для}\ v\in\Omega_u^+\ \mbox{при пол-}\\ 
&\mbox{нодоступной схеме},\\
q_{N_v}(\rho_v, y_v) & \mbox{при схеме полного}\\
&\mbox{распределения}\\ 
&\mbox{памяти},  v\in\Omega_u^+\,.
\end{cases}
\end{gather*}



Тогда уравнения~(\ref{e4aga}) и~(\ref{e16aga}) записываются в виде:
$$
y_u^v = q_N^v (\overline{\rho}^v_u, y^v_u)\,,\quad v\in \Omega_u^+\,.
$$
Для значений, вычисляемых на $k$-м шаге алгоритма, к 
обозначениям соответствующих параметров приписывается знак~$[k]$.
\pagebreak

\textbf{Шаг 0.} 
\begin{enumerate}[1.]
\item  \textit{Инициализация}. Вычисление начальных значений 
параметров~$\rho_v$, $v\in\Omega_u^+$: $\Lambda_v[0]=\Lambda_v$, 
$\rho_v[0]=\Lambda_v[0]/(\mu_v(1-B_v))$, $y_u^v[0]=1$.
\item \textit{Проверка условий существования решения}. Если для некоторой 
линии $v\in\Omega_u^+$ выполняется хотя бы одно неравенство $(c_v\mu_v(1-
B_v))/\Lambda_v[0]\;\leq$\linebreak $\leq\;1$, или $(1-B_v)/(\Lambda_v\tau_v B_v) \leq 1$, или 
$(t_v(1\;-$\linebreak $-\;B_v))/\Lambda_v[0] \leq 1$, то алгоритм заканчивает работу с 
результатом <<нагрузка не реализуема>>. Если в узле используется 
полнодоступная схема и $(c_v\mu_v(1-B_v))/\Lambda_v[0] > 1$, $(1-
B_v)/(\Lambda_v\tau_v B_v)\;>$\linebreak $>\;1$, $(t_v(1-B_v))/\Lambda_v[0] > 1$ для всех 
$v\in\Omega_u^+$, то проверяется условие~(\ref{e13aga}) для $\Lambda_v =
\Lambda_v[0]$, $v\in\Omega_u^+$, и при невыполнении этого условия алгоритм 
заканчивает работу с результатом <<нагрузка не реализуема>>.
\end{enumerate}

    При вычислении левой части неравенства~(\ref{e13aga}) рекомендуется 
использовать метод свертки Базена (см.~\cite{12aga}), позволяющий 
производить рекуррентные вычисления (подробно этот метод описан также 
в~[1, 3--6]).



\medskip
\textbf{Шаг~$k$} ($k > 0$):
\begin{enumerate}[1.]
\item \textit{Вычисление вероятностей блокировок}. Используя значения 
параметров $\overline{\rho}_u^v[k-1]$, $y_u^v[k-1]$, $v\in\Omega_u^+$, 
вычисление с помощью формул~(1)--(7) значений 
вероятностей $y[k]=1- \pi [k]$~--- в случае полнодоступной памяти, или 
$y_v[k]=1- \pi_v[k]$, $v\in\Omega_u^+$, с помощью формул~(\ref{e15aga})~--- в 
случае полного разделения памяти. При вычислении этих значений 
рекомендуется использовать метод свертки Базена.
    \item \textit{Проверка условий останова алгоритма}. Если хотя бы для 
одной $v\in\Omega_u^+$ для заданного значения точности   выполняется 
условие
$$
\fr{\vert \Lambda_v^*[k]-\Lambda_v^*[k-1]\vert}{\Lambda_v^*[k]}> \varepsilon\,,
$$
то вычисление параметров $\overline{\rho}_u^v[k]$, $v\in\Omega_u^+$, и 
переход к шагу~$k$, положив $k$ равным $k+1$, иначе алгоритм завершает 
работу. 
\end{enumerate}

    По завершении алгоритма либо выявится, что нагрузка в системе не 
реализуема, либо будут вычислены интенсивности потоков, поступающих на 
линии узла, и стационарные вероятности блокировок для заявок каждого типа. 
    
\section{Примеры расчета}

    Для проверки точности вычисления результатов с помощью 
предложенного выше алгоритма и приемлемости введенных предположений 
были проведены вычислительные эксперименты с использованием 
аналитических и имитационных моделей. Во всех рассмотренных ниже 
примерах потоки внешних заявок считаются пуассоновскими. 
В~табл.~1 приведены значения вероятности блокировок вновь 
поступивших извне заявок, полученные на основании точной формулы, 
приведенной в~\cite{4aga} для СМО типа $M\vert M\vert 1\vert 0$ с повторными 
заявками при экспоненциальном распределении интервала времени между 
повторными попытками (первая строка таблицы), алгоритма из подраздела~5 
настоящей статьи (вторая строка) и имитационной модели при постоянном 
интервале времени между повторными попытками, равном~10 (третья строка). 
Расчет табл.~1 проведен для узла с одной исходящей одноканальной 
линией при интенсивности первичного потока $\Lambda =1$ и емкости 
накопителя $N_v=1$. Таблицы~2 и~3 вычислены с помощью 
алгоритма из подраздела~5 и имитационной модели соответственно при одной 
исходящей линии с числом каналов~10.


    В табл.~\ref{t4aga} и~\ref{t5aga} приведены значения вероятности 
блокировки узла с тремя исходящими линиями канальной емкости~10 каждая 
при $\mu_v =0{,}2$, $v\in\Omega_u^+$,  вычисленные с помощью алгоритма из 
подраздела~5 и имитационной модели с интервалом повторной попытки, 
равным~10, соответственно. В табл.~\ref{t4aga} и~\ref{t5aga} знак <<--->> в 
ячейках означает, что предложенная нагрузка $\Lambda_v$, $v\in\Omega_u^+$, 
не реализуема.



В табл.~\ref{t6aga} отражены вероятности блокировки такого же узла с 
накопителем $N = 35$ при экспоненциальном распределении интервала 
времени между повторными попытками со средним значением~$\tau$. 


Результаты вычислительного эксперимента показывают, что с  увеличением 
длины интервала между повторными попытками  вероятность блокировки 
увеличивается и приближается к значению,\linebreak
вычисленному с помощью 
алгоритма из подраздела~5 (см.\ табл.~\ref{t4aga} и~\ref{t6aga}), т.\,е.\ при 
пуассоновском внешнем потоке заявок предположение, что суммарный 
входной поток заявок  является пуассоновским, вполне приемлемо для 
предварительного анализа характеристик узла (например, при  $\tau c_v\mu_v > 
10$). Как показывают табл.~1--3, вероятность блокировки 
узла существенно зависит от\linebreak 

\vspace*{6pt}
\noindent
%\begin{table*}\small %tabl1
{\small
{{\tablename~1}\ \ \small{Вероятности блокировок при одной исходящей одноканальной линии}}
%\label{t1aga}}
\vspace*{-3pt}

\begin{center}
{\tabcolsep=7.3pt
\begin{tabular}{|c|c|c|c|c|c|}
\hline
&\multicolumn{5}{c|}{$\mu$}\\
\cline{2-6}
\multicolumn{1}{|c|}{\raisebox{4pt}[0pt][0pt]{№}}&1{,}1&1{,}2&2&3&4\\
\hline
1&0,9091&0,8333&0,5000&0,3333&0,2500\\
2&0,9091&0,8333&0,5000&0,3333&0,2500\\
3&0,8867&0,8452&0,4944&0,3167&0,2396\\
\hline
\end{tabular}}
\end{center}
%\vspace*{-6pt}
%\end{table*}
}
%\bigskip
%\medskip
\addtocounter{table}{1}
\pagebreak

\end{multicols}

\renewcommand{\figurename}{\protect\bf Таблица}
%\renewcommand{\tablename}{\protect\bf Рис.}
\begin{figure*}
{\small
\begin{minipage}[t]{76mm}
%\begin{table*}\small %tabl2
\begin{center}
\Caption{Вероятности блокировок при одной исходящей многоканальной линии ($\varepsilon 
=0{,}0001$)
\label{t2aga}}
\vspace*{2ex}

\tabcolsep=6.5pt
\begin{tabular}{|c|c|c|c|c|c|}
\hline
&\multicolumn{5}{c|}{$\mu$}\\
\cline{2-6}
\multicolumn{1}{|c|}{\raisebox{4pt}[0pt][0pt]{$N$}}&0{,}11&0{,}12&0{,}2&0{,}3&0{,}4\\
\hline
10&0,4845&0,2935&0,0204&0,0017&0,0002\\
15&0,1181&0,0545&0,0006&0,0000&0,0000\\
20&0,0489&0,0167&0,0000&0,0000&0,0000\\
\hline
\end{tabular}
\end{center}
%\end{table*}
\end{minipage}
\hfill
\begin{minipage}[t]{76mm}
%\begin{table*}\small %tabl3
\begin{center}
\Caption{Вероятности блокировок при одной исходящей линии
\label{t3aga}}
\vspace*{2ex}

\tabcolsep=6.5pt
\begin{tabular}{|c|c|c|c|c|c|}
\hline
&\multicolumn{5}{c|}{$\mu_v$}\\
\cline{2-6}
\multicolumn{1}{|c|}{\raisebox{4pt}[0pt][0pt]{$N$}}&0{,}11&0{,}12&0{,}2&0{,}3&0{,}4\\
\hline
10&0,5247&0,3238&0,0219&0,0019&0,0001\\
15&0,1726&0,0912&0,0004&0,0001&0,0000\\
20&0,1180&0,0371&0,0000&0,0000&0,0000\\
\hline
\end{tabular}
\end{center}
%\end{table*}
\end{minipage}
}
\vspace*{6pt}
\end{figure*}

\renewcommand{\figurename}{\protect\bf Рис.}
\renewcommand{\tablename}{\protect\bf Таблица}
\addtocounter{table}{2}

\begin{table}\small %tabl4
\begin{center}
\parbox{400pt}{\Caption{Вероятности блокировок при трех исходящих линиях, вычисленные алгоритмом из 
подраздела~5 ($\varepsilon =0{,}0001$)
\label{t4aga}}
}

\vspace*{2ex}

\tabcolsep=8pt
\begin{tabular}{|c|c|c|c|c|c|c|c|c|c|}
\hline
&\multicolumn{9}{c|}{$\Lambda_v$}\\
\cline{2-10}
\multicolumn{1}{|c|}{\raisebox{4pt}[0pt][0pt]{$N$}}&1&1{,}1&1{,}2&1{,}3&1{,}4&1{,}5&1{,}6&1{,}7&1{,}8\\
\hline
20&0,0677&0,1423&0,2975&0,7653&---&---&---&---&---\\
25&0,0065&0,0173&0,0394&0,0827&0.1690&0.3827&---&---&---\\
30&0,0005&0,0019&0,0059&0,0155&0.0361&0.0790&0.1792&0,7259&---\\
35&0,0000&0,0002&0,0008&0,0030&0,0089&0,0234&0,0574&0,1505&---\\
40&0,0000&0,0000&0,0001&0,0005&0,0022&0,0075&0,0220&0,0617&0,2449\\
\hline
\end{tabular}
\end{center}
%\end{table}
\vspace*{6pt}
%\begin{table}\small %tabl5
\begin{center}
\parbox{400pt}{\Caption{Вероятности блокировок при трех исходящих линиях, вычисленные с помощью 
имитационной модели
\label{t5aga}}
}

\vspace*{2ex}

\tabcolsep=8pt
\begin{tabular}{|c|c|c|c|c|c|c|c|c|c|}
\hline
&\multicolumn{9}{c|}{$\Lambda_v$}\\
\cline{2-10}
\multicolumn{1}{|c|}{\raisebox{4pt}[0pt][0pt]{$N$}}&1&1{,}1&1{,}2&1{,}3&1{,}4&1{,}5&1{,}6&1{,}7&1{,}8\\
\hline
20&0,0786&0,1695&0,3549&0,7056&---&---&---&---&---\\
25&0,0069&0,0190&0,0441&0,0998&0,2266&0,4583&---&---&---\\
30&0,0007&0,0024&0,0075&0,0184&0,0462&0,1025&0,2380&0,6931&---\\
35&0,0000&0,0003&0,0007&0,0040&0,0129&0,0307&0,0890&0,2981&---\\
40&0,0000&0,0000&0,0000&0,0011&0,0041&0,0095&0,0346&0,0790&0,3179\\
\hline
\end{tabular}
\end{center}
%\end{table}
\vspace*{6pt}
%\begin{table}\small %tabl6
\begin{center}
\parbox{356pt}{\Caption{Зависимость вероятности блокировки при трех исходящих линиях, вы\-чис\-лен\-ные с 
помощью имитационной модели со случайным интервалом между повторными попытками
\label{t6aga}}
}

\vspace*{2ex}

\tabcolsep=8pt
\begin{tabular}{|c|c|c|c|c|c|c|c|c|}
\hline
&\multicolumn{8}{c|}{$\Lambda_v$}\\
\cline{2-9}
\multicolumn{1}{|c|}{\raisebox{6pt}[0pt][0pt]{$\tau$}}&1&1{,}1&1{,}2&1{,}3&1{,}4&1{,}5&1{,}6&1{,}7\\
\hline
\hphantom{9}1&0.0001&0,0001&0,0017&0,0063&0,0210&0,0733&0,1996&0,4222\\
\hphantom{9}5&0.0000&0,0002&0,0016&0,0036&0,0446&0,0159&0,1360&0,3273\\
10&0.0000&0,0002&0,0011&0,0036&0,0101&0,0430&0,0818&0,2774\\
20&0.0000&0,0003&0,0007&0,0029&0,0089&0,0257&0,0863&0,2045\\
     \hline
\end{tabular}
\end{center}
\end{table}


\begin{multicols}{2}


\noindent
числа каналов в линии при равной суммарной 
производительности. Кроме того, как видно из табл.~\ref{t5aga} и~\ref{t6aga}, 
вероятность блокировки в большей степени зависит от среднего значения 
длины интервала между повторными попытками передачи, чем от закона 
распределения длины интервала. Таким образом, предложенный в работе 
алгоритм позволяет вы\-чис\-лить с достаточной точностью вероятность 
блокировки узла, интенсивности повторных передач и предельную величину 
реализуемой нагрузки. Отметим, что полученные в данной статье результаты 
могут быть использованы для расчета нагрузок в телекоммуникационной сети с 
повторами заявок в предыдущем узле или из источника. 


{\small\frenchspacing
{%\baselineskip=10.8pt
\addcontentsline{toc}{section}{Литература}
\begin{thebibliography}{99}    
\bibitem{1aga}
\Au{Kamoun~F., Kleinrock~L.}
Analysis of shared finite storage in a computer networks node environment under 
general traffic conditions~// IEEE Trans. on Commun., 1980. Vol.~28. No.\,7. 
P.~992--1003.

\bibitem{6aga} %2
\Au{Агаларов~Я.\,М., Шоргин~С.\,Я.}
Рекуррентный метод вычисления параметров сетей связи~// Техника средств 
связи, 1986. Сер. <<Системы связи>>. Вып.~6. С.~42--46.

\bibitem{3aga}
\Au{Башарин Г.\,П., Бочаров~П.\,П., Коган~Я.\,А.}
Анализ очередей в вычислительных сетях.~--- М.: Наука, 1989. 

\bibitem{4aga}
\Au{Бочаров~П.\,П., Печинкин~А.\,В.}
Теория массового обслуживания.~--- М.: Изд-во РУДН, 1995. 

\bibitem{5aga}
\Au{Вишневский~В.\,М.} 
Теоретические основы проектирования компьютерных сетей.~--- М.: 
Техносфера, 2003. 

\bibitem{2aga} %6
\Au{Башарин Г.\,П.}
Лекции по математической теории телетрафика.~--- М.: Изд-во РУДН, 2007. 

\bibitem{7aga}
\Au{Таранцев~А.\,А.}
Инженерные методы теории массового обслуживания.~--- М.: Наука, 2007.

\bibitem{9aga} %8
\Au{D'Apice~C., De~Simone~T., Manzo~R., Rizelian~G.}
$M\vert G\vert 1\vert r$ retrial queueing system with priority service of primary 
customers and a customers-searching server~// Distributed Computer and 
Communication Networks. Stochastic Modelling and Optimization.~--- М.: 
Техносфера, 2003. P.~106--117.

\bibitem{8aga} %9
\Au{Klimenok~V.\,I., Kim~C.\,S.}
$BM\!AP$/$PH$/1 retrial system operating in random environment~// Proceedings of 
the 5th St.-Petersburg Workshop on Simulation, St.-Petersburg, June~26\,--\,July~2, 
2005.~--- St.-Petersburg: NII Chemistry St.-Petersburg University Publs., 
2005. P.~367--372.   

\bibitem{10aga}
\Au{Krishnamoorthy~A., Babu~S.}
$M\!AP\vert (PH,PH)/c$ retrial queue with selegeneration of priorities 
and non-preemptive service~// Proceedings of the 14th International Conference on 
Analytical and Stochastic Modeling Techniques and Applications, June~4--6, 
2007. Prague, Czech Republic.~--- Sbr.-Dudweiler: Digitaldruck Pirrot GmbH, 
2007. P.~70--74.

\bibitem{11aga}
\Au{Корн~Г., Корн~Т.}
Справочник по математике.~--- М.: Наука, 1974.

\label{end\stat}


\bibitem{12aga}
\Au{Buzen~J.\,P.}
Computational algorithm for closed queuing networks with exponential servers~// 
Communications ACM, 1973. Vol.~16. No.\,9. P.~527--531.
 \end{thebibliography}
}
}
\end{multicols}
 
 
   %6
\def\stat{torshin}

\def\tit{О ПОРОЖДЕНИИ СИНТЕТИЧЕСКИХ ПРИЗНАКОВ НА~ОСНОВЕ~ОПОРНЫХ ЦЕПЕЙ 
И~ПРОИЗВОЛЬНЫХ МЕТРИК В~РАМКАХ~ТОПОЛОГИЧЕСКОГО ПОДХОДА 
К~АНАЛИЗУ ДАННЫХ.\\ ЧАСТЬ~2.~ЭКСПЕРИМЕНТАЛЬНАЯ АПРОБАЦИЯ\\ НА~ЗАДАЧАХ ФАРМАКОИНФОРМАТИКИ$^*$}

\def\titkol{О порождении синтетических признаков на основе опорных цепей 
и~произвольных метрик} % в~рамках топологического подхода  к~анализу данных. Часть~2. Экспериментальная апробация на  задачах фармакоинформатики}

\def\aut{И.\,Ю.~Торшин$^1$}

\def\autkol{И.\,Ю.~Торшин}

\titel{\tit}{\aut}{\autkol}{\titkol}

\index{Торшин И.\,Ю.}
\index{Torshin I.\,Yu.}


{\renewcommand{\thefootnote}{\fnsymbol{footnote}} \footnotetext[1]
{Работа выполнена при поддержке гранта РНФ (проект №\,23-21-00154) с~использованием инфраструктуры 
Центра коллективного пользования <<Высокопроизводительные вычисления и~большие данные>> (ЦКП 
<<Информатика>>) ФИЦ ИУ РАН (г.~Москва).}}


\renewcommand{\thefootnote}{\arabic{footnote}}
\footnotetext[1]{Федеральный исследовательский центр <<Информатика и~управление>> Российской академии наук, 
\mbox{tiy135@yahoo.com}}

\vspace*{-12pt}


\Abst{Рассмотрение прецедентных отношений между признаками и~таргетной переменной в~виде наборов элементов булевой решетки указывает на возможность порождения 
синтетических признаков с~использованием метрических функций расстояния. 
Сформулированы подходы к~(1)~оценке релевантности (<<информативности>>) метрик 
по отношению к~решаемым задачам, (2)~порождению и~(3)~отбору синтетических 
признаков, более информативных, чем исходные признаковые описания. Представленные 
результаты топологического анализа 2400~выборок данных  
<<мо\-ле\-ку\-ла--свойство>> из ProteomicsDB позволили получить достаточно 
эффективные алгоритмы прогнозирования свойств молекул (ранговая корреляция  
в~кросс-ва\-ли\-да\-ции~--- $0{,}90\pm0{,}23$). На данной выборке задач установлены 
метрики, которые наиболее часто порождают информативные синтетические признаки: 
максимальное уклонение Колмогорова, <<косое>> расстояние, метрики Lp, Реньи, фон 
Мизеса. Для решения изученного комплекса задач показано преимущество полиномных 
корректоров по сравнению с~нейросетевыми и~с~корректорами типа <<случайный 
лес>>.}

\KW{топологический анализ данных; теория решеток; алгебраический подход 
Ю.\,И.~Жу\-рав\-лё\-ва; фармакоинформатика}

\DOI{10.14357/19922264240207}{OTXCUD}
  
\vspace*{-1pt}


\vskip 10pt plus 9pt minus 6pt

\thispagestyle{headings}

\begin{multicols}{2}

\label{st\stat}

\section{Введение}

     В первой части работы~[1] принимается, что задано регулярное 
множество прецедентов 
$$
\mathbf{Q}\hm= \{\mathrm{D}(x_i)\vert x_i\in 
\mathbf{X}\}
$$ 
на решетке $L(T(\mathbf{X}))$, по\-рож\-ден\-ное на основе 
множества исходных описаний объектов $\mathbf{X}\hm= \{ x_1, \ldots , 
x_{N_0}\}$. Для индивидуального объекта\linebreak $x_i\hm\in \mathbf{X}$ 
прецедентному соотношению между значениями признаками 
$\Gamma_k(x_i)$ и~\mbox{$t$-й} таргетной переменной соответствует множество пар 
$\{(\{\Gamma_k^{-1}(\Gamma_k(x_i)),\linebreak k\hm=\overline{1, ,n}\}, \Gamma_t^{-1}(\Gamma_t(x_i))), i\hm=\overline{1,N_0},\
 k\hm=\overline{1,n},\linebreak t\hm=\overline{n+1, n+l}\}$, где $l$~--- 
число таргетных переменных. В~рамках топологической теории 
распознавания прецедентное соотношение между множествами $\{ 
\Gamma_k^{-1}(\Gamma_k(x_i))\}$ и~$\Gamma_t^{-1}(\Gamma_t(x_i))$ 
моделируется как со\-от\-вет\-ст\-ву\-ющие массивы расстояний, по\-рож\-да\-емые той 
или иной мет\-ри\-кой~$\rho_m$: $L(T(\mathbf{X}))^2\hm\to [0\ldots 1]$, 
$m\hm= \overline{1, m_0}$. В~[1] предложены способы <<встра\-и\-ва\-ния>> 
в~формализм полуэмирических рас\-сто\-яний на множествах $a\hm\in 
L(T(\mathbf{X}))$, векторах $\vec{v}_\alpha [a] \hm= ( v_{\alpha_1}[a], 
v_{\alpha_2}[a], \ldots , v_{\alpha_i}[a],\ldots)$ и~функциях 
$\hat{\phi}(x)\bm{\Gamma}_t(u)$. 
     
     Здесь для практического приложения формализма сформулированы 
подходы к~исследованию свойств~$\rho_m$, способы оценки релевантности 
функций~$\rho_m$ по отношению к~решаемым задачам, способы 
порождения и~отбора синтетических признаков, основанных на~$\rho_m$. 
Представлены результаты экспериментальной апробации на задачах 
фармакоинформатики.
     
\section{Об исследовании свойств функций расстояния~$\rho_m$}

    Рабочая гипотеза настоящего исследования со\-сто\-ит в~том, что для 
порождения более <<информативных>> признаков могут использоваться 
полуэмпирические функционалы расстояния на \mbox{множествах}, векторах, 
функциях~[2]. Метрические свойства ис\-поль\-зу\-емых функций 
расстояния~$\rho_m$ могут исследоваться аналитически или комбинаторно 
с~использованием аксиом метрики~[3]. Для анализа свойств этих 
функционалов в~топологической теории распознавания вводится следующее 
понятие.

\smallskip

\noindent
\textbf{Определение~1.} Обобщенной оценочной функцией расстояния 
будем называть конструкцию вида 
$$
\rho(a,b) = f(g ( v[a\vee b]) - g(v[a\wedge b])),
$$
 в~которой~$f$ и~$g$~--- функции, монотонные на 
соответствующих участках действительной оси; $v:\ L\hm\to R^+$~--- 
изотонная оценка, для которой выполнено условие оценки (\textbf{уО}: $\forall_L 
a,b: v[a]\hm+v[b]\hm= v[a \wedge b]\hm+ v[a\vee b]$) и~изотонности 
(\textbf{уИ}:  $\forall_L a,b: a\supseteq b \hm\Rightarrow v[a]\hm\geq v[b]$). 

\smallskip

\noindent
\textbf{Теорема~1.} \textit{Функция расстояния~$\rho$ считается 
обобщенной оценочной функцией расстояния тогда и~только тогда, когда 
$\rho(a,b)\hm= \rho(a\vee b, a\wedge b)$, а~термы от $a$ и~$b$ в~формуле для 
$\rho(a,b)$ представляют собой композицию монотонной функции 
и~изотонной оценки}. 

\smallskip

Необходимость следует из  $a\vee b\hm= (a\vee b)\vee (a\wedge b)$ и~$a\wedge b \hm= (a\vee b) \wedge (a\wedge b)$  при 
подстановке $a\vee b$ и~$a\wedge b$ вместо $a$ и~$b$ в~определение~1. 
Эквивалентность $\rho(a,b)$ и~$\rho(a\vee b, a\wedge b)$ указывает на то, что 
в~выражение для вычисления~$\rho$ входят тер\-мы-функ\-ци\-о\-на\-лы, 
содержащие выражения $a\vee b$ и~$a\wedge b$, взаимозаменяемые с~$a$ 
и~$b$, т.\,е.\ термы вида $g^\prime (a\vee b)$ и~$g^\prime(a\wedge b)$. По 
условию теоремы эти термы включают монотонную функцию от изотонной 
оценки, т.\,е.~$g^\prime$ монотонна. Так как $\rho$~--- функция расстояния, 
то $g^\prime$-тер\-мы не могут входить в~выражение для~$\rho$ в~виде 
произведения, суммы, отношения, степени или суммы, а~только в~виде 
разности, т.\,е.\
$$
\rho(a,b) = f\left(g^\prime(a\vee b) \hm- g^\prime (a\wedge b)\right),
$$ 
из чего следует достаточность. Теорема доказана.

\smallskip

\noindent
\textbf{Следствие~1.} Для обобщенной оценочной~$\rho$ 
\begin{multline*}
\forall \ell \subseteq L(T(\mathbf{X})): \Delta_{\vee\wedge}(\ell)\equiv 0,\\ 
\Delta_{\vee\wedge}(\ell)=  \sum\limits_{a,b\in \ell} \vert\rho(a,b)- 
\rho(a\vee b, a\wedge b)\vert \fr{2}{\vert\ell\vert/(\vert\ell\vert -1)}\,.
\end{multline*}

\smallskip

\noindent
\textbf{Следствие~2.} Выберем <<опорное>> множество $a\hm\in 
L(T(\mathbf{X}))$ и~обобщенную оценочную~$\rho$. При $f(x)\hm= g(x)\hm= x$ 
$v_{a,\rho}[b]\hm= \rho(a,b)\hm= \rho(a\vee b, a\wedge b)$~--- изотонная 
оценка. 

Следует из того, что любая линейная комбинация изотонных оценок~--- 
изотонная оценка при условии положительной определенности (теорема~2 
в~[4]). Также проверяется прямой подстановкой $v_{a,\rho}[b]$ в~уО и~уИ. 

\smallskip

\noindent
\textbf{Следствие~3.} Расстояния Фре\-ше--Ни\-ко\-ди\-ма, Амана,  
Рэн\-да/Ще\-ка\-нов\-ско\-го, Со\-ка\-ла--Сни\-са (варианты~1, 2 и~3),  
Рас\-се\-ла--Рао, Род\-же\-ра--Та\-ни\-мо\-то, Фейта, Тверского и~Юле 
могут служить обобщенными оценочными функциями расстояния. 

\smallskip

\noindent
\textbf{Следствие~4.} Расстояния Симпсона, Бра\-у\-на--Блан\-ке, 
Андерберга и~Говера-2  не входят в~число обобщенных оценочных функций 
расстояния.

\smallskip

     Теорема~1 со следствиями предоставляет аналитический 
и~комбинаторный инструментарий для исследования свойств 
полуэмпирических функций расстояния. Если заданная~$\rho$ служит 
обобщенной оценочной функцией расстояния, то могут быть получены 
соответствующие аналитические выражения для функций~$f$ и~$g$. 
Например, расстояние Со\-ка\-ла--Сни\-са-2
$$
\rho(a,b) = 1- \fr{\vert a\cap 
b\vert }{\vert a\cup b\vert + \vert a\Delta b\vert}
$$ 
выступает 
обобщенным оценочным расстоянием с~$f(x)\hm= (e^x\hm-1)/(0{,}5e^x\hm-1)$ и~$g(x)\hm=\ln (x)$. При невозможности аналитической проверки 
свойства~$\rho$ как обобщенной оценочной могут быть изучены на 
подмножествах~$\ell$ решетки $L(T(\mathbf{X}))$ посредством вычисления 
значений функционала $\Delta_{\vee\wedge}(\ell)$ (следствие~1). 

\section{О способах оценки релевантности метрик~$\rho_m$ по~отношению к~задаче клас\-сификации/прогнозирования}

     Биекция между множеством прецедентов~$\mathbf{Q}$ и~множеством 
исходных описаний объектов~$\mathbf{X}$, существующая при выполнении 
условия регулярности по Журавлёву ($\forall \mathrm{x}\hm\in \mathbf{X}, 
\mathrm{x}\hm= D^{-1}(D(\mathrm{x}))$, гарантирует однозначность 
соответствия описаний~$x_i$ и~$q_i$. Это делает возможным рассматривать 
прецедентные соотношения, заданные на~$\mathbf{Q}$, в~терминах 
множеств $\{ \Gamma_k^{-1}(\Gamma_k(x_i))\}$ и~$\Gamma_t^{-1}( 
\Gamma_t(x_i))$ с~использованием расстояний~$\rho_m$ на подмножествах 
множества~$\mathbf{X}$~[1].
     
     Пусть таргетный класс объектов $\mathbf{c}_{\bm{\alpha}}$ задан 
посредством $\alpha$-го значения $t$-й переменной $\lambda_{t\alpha}\hm\in 
\mathrm{I}_t$, $t\hm= \overline{n+1,  n+l}$, как $\mathbf{c}_{{\bm 
\alpha}} \hm= \Gamma_t^{-1}(\lambda_{t\alpha})$. В~случае числовой 
переменной за $\mathbf{c}_{\bm{\alpha}}$ может приниматься каждый из 
элементов $u(\lambda_{t\alpha})$ цепи~$A_t$. Так как 
$L(T(\mathbf{X}))$ булева, то дополнение множества 
$\mathbf{c}_{\bm{\alpha}}$, $\overline{\mathbf{c}}_{\bm{\alpha}} \hm= 
\mathbf{X}\backslash \Gamma_t^{-1}(\lambda_{t\alpha})$, определено 
однозначно. Таким образом, выделение класса $\mathbf{c}_{\bm{\alpha}}$ 
порождает задачу классификации $\mathbf{c}_{\bm{\alpha}}/ 
\overline{\mathbf{c}}_{\bm{\alpha}}$. Любая задача числового 
прогнозирования может быть сведена к~последовательности корректно 
решаемых задач $\mathbf{c}_{\bm{\alpha}}/ 
\overline{\mathbf{c}}_{\bm{\alpha}}$~\cite{5-tor}.
     
     Пусть задано подмножество признаков~$p \hm\subseteq [1\ldots n]$ 
     и~элемент решетки $c\in L(T(\mathbf{X}))$. Определим функцию 
$$
\bm{\rho}_{\mathbf{mc}} (x_i, c, {p}) \hm= \{ \rho_m(c, \Gamma_k^{-1}(\Gamma_k (x_i)),\ k\hm\in {p})\}.
$$
 При заданных~$\rho_m$, $p$, 
$\mathbf{c}_{\bm{\alpha}}$ и~$\overline{\mathbf{c}}_{\bm{\alpha}}$ 
для~$x_i$ вычислимы множества расстояний $\bm{\rho}_{\mathbf{mc}}(x_i, 
\mathbf{c}_{\bm{\alpha}}, {p})$ и~$\bm{\rho}_{\mathbf{mc}}(x_i, 
\overline{\mathbf{c}}_{\bm{\alpha}}, {p})$. Обозначим 
\begin{align*}
\bm{\rho}_{\mathbf{m}\bm{\alpha}}(x_i) &=  \bm{\rho}_{\mathbf{mc}} 
(x_i, \mathbf{c}_{\bm{\alpha}}, [1\ldots n]); \\
\bm{\rho}_{\mathbf{m}\overline{\bm{\alpha}}} (x_i) &= 
\bm{\rho}_{\mathbf{mc}}(x_i, \overline{\mathbf{c}}_{\bm{\alpha}} , [1\ldots n]).
\end{align*}
 Для $x_i\hm\in \mathbf{X}$ 
определено множество 
\begin{multline*}
\bm{\rho}_{\mathbf{m}}(x_i,{p})=\left \{ \rho_{mk_1k_2}(x_i, {p}) = {}\right.\\
{}\rho_m\left(\Gamma^{-1}_{k_1}\left(\Gamma_{k_1}(x_i), \Gamma^{-1}_{k_2}\left(\Gamma_{k_2}(x_i)\right)\right)\right),\\
\left. k_1, k_2\hm \in {p},\  k_1\not= k_2\right\},\ \bm{\rho}_{\mathbf{m}}(x_i)=  \bm{\rho}_{\mathbf{m}}(x_i, [1\ldots n]).
\end{multline*}
     
     На основе $\bm{\rho}_{\mathbf{m}{\bm{\alpha}}}(x_i)$ 
и~$\bm{\rho}_{\mathbf{m}\overline{\bm{\alpha}}}(x_i)$ вводятся оценки 
релевантности~$\rho_m$. По отношению к~задаче $\mathbf{c}_{\bm{\alpha}}/ 
\overline{\mathbf{c}}_{\bm{\alpha}}$ более релевантна или 
<<информативна>> такая мет\-ри\-ка~$\rho_m$, которая для всех $x\hm\in 
\mathbf{c}_{\bm{\alpha}}$ минимизирует расстояния в~списке 
$\bm{\rho}_{\mathbf{m}{\bm{\alpha}}}(x)$ и~максимизирует расстояния 
в~списке $\bm{\rho}_{\mathbf{m}\overline{\bm{\alpha}}}(x)$ (т.\,е.\ 
<<приближает>> объекты к~их классам). Выделены два взаимосвязанных 
направления дальнейших исследований: 
\begin{enumerate}[(1)]
\item нахождение подмножеств $p$ 
признаков, <<более информативных>> для~$\rho_m$;  
\item на\-строй\-ка/вы\-бор~$\rho_m$ при фиксированном~$p$.
\end{enumerate}
     
     Для $c^\prime\hm\in L(T(\mathbf{X}))$ определим 
$\vartheta_{\mathbf{mc}}$, операцию слияния списков 
$\bm{\rho}_{\mathbf{mc}}$:
$$
\vartheta_{\mathbf{mc}}(c^\prime, c, 
{p})\hm= \bigcup\limits_{y\in c^\prime} \bm{\rho}_{\mathbf{mc}} (y,c, 
{p}).
$$
 Обозначим 
 $$
 \vartheta_{\mathbf{m}\bm{\alpha}}(\mathbf{c}, 
{p}) \!=\! \vartheta_{\mathbf{mc}}(\mathbf{c}, 
\mathbf{c}_{\bm{\alpha}}, {p});\ 
\vartheta_{\mathbf{m}\bm{\alpha}}(\mathbf{c},{p})\!=\! 
\vartheta_{\mathrm{mc}}(\mathbf{c}, \overline{\mathbf{c}}_{\bm \alpha}, 
{p}),
$$
 вычислим множества $\vartheta_{\mathbf{m}{\bm \alpha}} 
(\mathbf{c}_{\bm \alpha}, {p})$ и~$\vartheta_{\mathbf{m}{\bm \alpha}} 
(\overline{\mathbf{c}}_{\bm \alpha},{p})$ и~сформируем 
эмпирические функции распределения (э.ф.р.)\ $\hat{\phi}(x) 
\vartheta_{\mathbf{m}{\bm \alpha}} (\mathbf{c}_{\bm \alpha}, {p})$ 
и~$\hat{\phi}(x) \vartheta_{\mathbf{m}{\bm \alpha}} 
(\overline{\mathbf{c}}_{\bm \alpha}, {p})$. На пространстве 
однородных монотонно возрастающих функций 
\begin{multline*}
\mathbf{M}^+_{0\ldots1} ={}\\
{}= 
\{f: [0\ldots 1]\hm\to [0\ldots 1],\ x\geq y\hm\Rightarrow f(x)\geq f(y)\}
\end{multline*}
введем 
функционал расстояния $d_f$: $\mathbf{M}^+_{0..1}\hm\to [0\ldots 1]$ 
(максимальное уклонение Колмогорова $D(f(x), g(x))\hm= \mathrm{sup}_x 
\vert f(x)\hm- g(x)\vert$, метрики фон Мизеса, Реньи и~др.). Выбор~$d_f$ 
делает возможной постановку ряда задач топологического анализа данных:
     \begin{enumerate}[(1)]
\item количественные оценки релевантности~$\rho_m$ как 
$d_f(\hat{\phi}(x)\vartheta_{\mathbf{m}{\bm \alpha}}(\mathbf{c}_{\bm \alpha}, 
{p}), \hat{\phi}(x)\vartheta_{\mathbf{m}{\bm 
\alpha}}(\overline{\mathbf{c}}_{\bm \alpha}, {p}))$ для 
разных~$\mathbf{c}_{\bm \alpha}$, $\lambda_{t\alpha} \hm\in \mathrm{I}_t$, 
$\alpha \hm= \overline{1, \vert \mathrm{I}_t\vert}$;
\item задачи оптимизации для увеличения разделения классов 
$\mathbf{c}_{\bm \alpha}/\overline{\mathbf{c}}_{\bm \alpha}$ 
($\argmax_{\rho_m,{p}} d_f(\hat{\phi}\vartheta_{\mathbf{m}{\bm \alpha}}(\overline{\mathbf{c}}_{\bm \alpha},{p}), 
\hat{\phi}\vartheta_{\mathbf{m}{\bm \alpha}}(\mathbf{c}_{\bm \alpha}, {p}))$,
$\argmax_{\rho_m,{p}} d_f(\hat{\phi}\vartheta_{\mathbf{m}\overline{\bm{\alpha}}}, (\overline{\mathbf{c}}_{\bm \alpha}, {p}), 
\hat{\phi}\vartheta_{\mathbf{m}\overline{\bm \alpha}}
(\mathbf{c}_{\bm \alpha}, {p}))$  и~др.);
\item определение $\rho_q$-мет\-рик на пространстве объектов~[2, с.~184--199] 
(например, в~виде $d_f (\hat{\phi}\bm{\rho}_{\mathbf{m}{\bm \alpha}}(x, 
{p}), \hat{\phi}\bm{\rho}_{\mathbf{m}{\bm \alpha}}(y, {p})), 
d_f (\hat{\phi}\bm{\rho}_{\mathbf{m}}(x,{p})$, 
$\hat{\phi}\bm{\rho}_{\mathbf{m}}(y, {p}))$); 
\item оценка близости метрик~$\rho_q$ к~метрике разреза по классам 
$\mathbf{c}_{\bm{\alpha}}/ \overline{\mathbf{c}}_{\bm{\alpha}}$; 
\item формулировка критериев раз\-ре\-ши\-мости/ре\-гу\-ляр\-ности задачи 
$\mathbf{c}_{\bm{\alpha}}/ \overline{\mathbf{c}}_{\bm{\alpha}}$~[6]; 
\item оценки компактности классов $\mathbf{c}_{\bm{\alpha}}$  
и~$\overline{\mathbf{c}}_{\bm{\alpha}}$~[3]. 
\end{enumerate}

\section{О способах порождения и~отбора синтетических 
признаков на~основании функций расстояния}

     Множества $\bm{\rho}_{\mathbf{m}{\bm \alpha}}(x_i,{p})$, 
$\bm{\rho}_{\mathbf{m}{\overline{\bm \alpha}}}(x_i, {p})$ 
и~$\bm{\rho}_{\mathbf{m}}(x_i)$ и~отдельные $\rho_m(\mathbf{c}_{\bm 
\alpha}, \Gamma_k^{-1}(\Gamma_k(x_i))$ используются для формирования 
синтетических числовых признаков $\Gamma_{k^\prime}(x_i)$ 
объекта~$x_i$, $k^\prime\hm= \overline{n+ l+1, n+l+n_S}$. 
Значение синтетического признака~$\Gamma_{k^\prime}(x_i)$ зависит от 
выбора~$\rho_m$, классов $\mathbf{c}_{\bm{\alpha}}$ 
и~$\overline{\mathbf{c}}_{\bm{\alpha}}$  и~от способа его вы\-чис\-ле\-ния: 
\begin{enumerate}[(1)]
\item $\rho_m(\mathbf{c}_{\bm \alpha}, \Gamma_k^{-1}(\Gamma_k(x_i))$; 
\item $\rho_m(\overline{\mathbf{c}}_{\bm \alpha}, \Gamma_k^{-1}(\Gamma_k(x_i))$; 
\item $\rho_m(\mathbf{c}_{\bm \alpha}, \ldots ) \hm- \rho_m(\overline{\mathbf{c}}_{\bm \alpha}, \ldots)$;
\item $1\hm- \rho_m(\mathbf{c}_{\bm \alpha}, \ldots)$;
\item значения э.ф.р.\ 
$\hat{\phi}(x)\bm{\rho}_{\mathbf{m}{\bm \alpha}}(x_i,{p})$ при 
разных~$x$ (например, соответствующих процентилям 
$\hat{\phi}\bm{\rho}_{\mathbf{m}{\bm \alpha}}(x_i,{p})$); 
\item значения $\hat{\phi}(x)\bm{\rho}_{\mathbf{m}\overline{\bm{\alpha}}} 
(x_i, {p})$ при разных~$x$;
\item $\hat{\phi}(x\hm+ \Delta x) 
\bm{\rho}_{\mathbf{m}{\bm \alpha}}(x_i,p) \hm- 
\hat{\phi}(x)\bm{\rho}_{\mathbf{m}{\bm \alpha}} (x_i, {p})$ 
и~$\hat{\phi}(x\hm+ \Delta x) \bm{\rho}_{\mathbf{m}{\overline{\bm \alpha}}} 
(x_i,{p}) \hm- \hat{\phi}(x)\bm{\rho}_{\mathbf{m}\overline{\bm 
\alpha}} (x_i, {p})$, где $\Delta x$~--- шаг.
\end{enumerate}
     
     Кроме того, $\mathbf{c}_{\bm{\alpha}}$ может определяться как 
$\Gamma_t^{-1}(\lambda_{t\alpha})$ или как $u(\lambda_{t\alpha})$; если 
$\mathbf{c}_{\bm \alpha} \hm= \Gamma_t^{-1}(\lambda_{t\alpha})$, то 
$\overline{\mathbf{c}}_{\bm{\alpha}}$ может быть равно $\Gamma^{-1}_t 
(\lambda_{t\alpha+1})$; классы $\mathbf{c}_{\bm{\alpha}}/ 
\overline{\mathbf{c}}_{\bm{\alpha}}$  
$t$-й переменной могут определяться с~использованием раз\-би\-ений на 
различные процентили (которые определяются как подвыборка значений 
$\lambda_{t\alpha} \hm\in \mathrm{I}_t$) и~т.\,д. 
     
     Таким образом, предлагаемые схемы порождают значительное число 
синтетических признаков $\Gamma_{k^\prime}(x_i)$ ($10n$ и~более при $n$ 
исходных признаках $\Gamma_k$), что делает необходимым введение 
процедур отбора признаков. Таргетная переменная $\Gamma_t(x_i)$~--- 
чис\-ло\-вая, и~по\-рож\-да\-емые признаки $\Gamma_{k^\prime}(x_i)$~--- также 
чис\-ло\-вые. Для данного случая в~прикладной математике имеется несколько 
различных подходов к~оценке взаимосвязи $\Gamma_t(x_i)$ 
и~$\Gamma_{k^\prime}(x_i)$: корреляционные оценки (для линейных 
закономерностей), полиномная аппроксимация с~оценкой качества (для 
нелинейных закономерностей) и~методы теории  
ве\-ро\-ят\-но\-стей\,/\,ма\-те\-ма\-ти\-че\-ской статистики, не зависящие от 
вида закономерности (в~том числе на основе <<взаимной 
информации>>~[7]).
{\looseness=1

}
     
     Наиболее фундаментальным представляется тес\-ти\-ро\-ва\-ние взаимосвязи 
двух переменных на осно\-ве <<нулевой гипотезы>> об их независимости. 
Пусть заданы пары тестируемых значений, $(x_i, y_i)$,\linebreak $i\hm= \overline{1,\mathbf{n}_{(\mathrm{x,y})}}$, э.ф.р.~$F_{xy}(x,y)$ характеризует 
совместное распределение~$x$ и~$y$, а~э.ф.р.~$F_{{x}}(x)$ 
и~$F_{{y}}(y)$~--- индивидуальные распределения переменных. 
Эмпирическая функция распределения нулевой \mbox{гипотезы} (независимость~$x$ и~$y$) определяется как 
$F_{{x}}(x)F_{{y}}(y)$. 
     
     Для оценки отличий между $F_{{xy}}(x,y)$\linebreak 
и~$F_{{x}}(x) F_{{y}}(y)$ необходимо ввести расстояние 
меж-\linebreak ду такими функциями (так называемую <<статисти-\linebreak ку>>) и~оценить 
достоверность различий посред\-ст\-вом \mbox{того} или иного статистического\linebreak \mbox{тес\-та}. 
В~качестве расстояния можно использовать функции~$d_f$, адап\-ти\-ро\-ван\-ные 
для 2-мер\-но\-го случая (например, макси\-маль\-ное уклонение 
     $D(\mathrm{F}_{{xy}}(x,y), \mathrm{F}_{{x}}(x) 
\mathrm{F}_{{y}}(y)) \hm= \max ( \vert 
\mathrm{F}_{{xy}}(x_i,y_i) \hm- \mathrm{F}_{{x}}(x_i) 
\mathrm{F}_{{y}}(y_i)\vert )$) и~статистический тест  
Кол\-мо\-го\-ро\-ва--Смир\-но\-ва 
$P_{\mathrm{КС}}$ $(D 
(\mathrm{F}_{{xy}}(x,y), \mathrm{F}_{{x}}(x) 
\mathrm{F}_{{y}}(y)), n_{(x,y)})$. Тогда $1\hm- 
P_{\mathrm{КС}}$ характеризует <<информативность>>~$x$ 
относительно~$y$. 
     
     Более универсальным подходом к~оценке достоверности различий 
между $\mathrm{F}_{{xy}}(x,y)$ и~$\mathrm{F}_{{x}}(x) 
\mathrm{F}_{{y}}(y)$ считается прямое вычисление выбранной 
статистики~$d_f$ на множествах пар значений $(x_i, y_i)$, полученных 
датчиком случайных чисел. 
     
     Пусть \textit{оператор $\hat{\zeta}$, семплирующий} 
множество~$\mathbf{X}$, формирует набор семплов 
$$
\hat{\zeta}\mathbf{X}\hm= \{a_1, a_2, \ldots , a_k, \ldots , 
a_{\vert\hat{\zeta}X\vert}\vert a_k\hm\subset \mathbf{X}\},
$$
 а~процедура 
random~--- датчик случайных чисел (в~диапазоне $[0\ldots 1]$). Для каждого 
семпла~$a_k$ принимается, что ${n}_{({x,y})} \hm= \vert 
a_k\vert$, и~вычисляется множество значений~$d_f$ для случайных 
выборок, 

\noindent
\begin{multline*}
\mathrm{rnd}\,(\hat{\zeta}\mathbf{X}, d_f)= \left\{ 
\vphantom{i=\overline{1,\left\vert \hat{\zeta} X\right\vert }}
d_f\left(
\vphantom{\overline{1, \vert a_i\vert }}
\mathrm{F}_{{xy}}(x_{ij}, y_{ij}), 
\mathrm{F}_{{x}}(x_{ij}) \mathrm{F}_{{y}}(y_{ij}),\right.\right.\\
\left.\left. x_{ij}, 
y_{ij}= \mathrm{random},\  j=\overline{1, \vert a_i\vert }\right),\ i=\overline{1,\left\vert \hat{\zeta} X\right\vert }\right\}.
\end{multline*}

 Для $a\hm\in \hat{\zeta} \mathbf{X}$ значение 
${P}(d_f, \hat{\zeta}\mathbf{X}, a, k^\prime, t)\hm= 1\hm-
\hat{\phi}(d_f(\mathrm{F}_{k^\prime t}(\Gamma_{k^\prime}(z), \Gamma_t(z)), F_{k^\prime}(\Gamma_{k^\prime}(z)) 
\mathrm{F}_t(\Gamma_t(z)))\vert z\hm\in a) \mathrm{rnd}\,(\hat{\zeta}\mathbf{X}, d_f)$~--- статистическая достоверность 
<<зависимости>> $\Gamma_t(z)$ и~$\Gamma_{k^\prime}(z)$ по 
статистике~$d_f$ на семпле~$a$, а~$1\hm- P(d_f, 
\hat{\zeta}\mathbf{X}, a, k^\prime, t)$ количественно оценивает зависимость.
    

При заданном способе оценки зависимости $1\hm- P(d_f, 
\hat{\zeta}\mathbf{X}, a, k^\prime, t)$ задача отбора информативных 
признаков решается посредством так называемого\linebreak  
В-ал\-го\-рит\-ма, исходно разработанного для построения оптимальных 
словарей финальных ин\-фор\-маций (чему и~соответствует литера~<<В>>)~[8]. 
\mbox{Данный} алгоритм, основанный на критерии раз\-ре\-ши\-мости по Журавлёву, 
позволяет выбирать множества финальных информаций на основе 
максимального час\-тич\-но\-го покрытия при минимуме\linebreak элементов покрытия. 
Замена мощности пересечения множеств на $1\hm- P(d_f, 
\hat{\zeta}\mathbf{X}, a, k^\prime, t)$ приведет к~тому, что  
В-ал\-го\-ритм будет выбирать минимум признаков с~максимальной 
<<информативностью>>\linebreak (наиболее информативные признаки, см.\ 
теоремы~1, 7  и~8 работы~[8]).

    Таким образом, в~рамках развиваемого формализма синтез более 
информативных синтетических~$\Gamma_{k^\prime}(x_i)$ осуществляется 
в~5~стадий: 
\begin{enumerate}[(1)]
\item определяется набор исходных (как правило, 
<<низкоинформативных>>) признаков~$\Gamma_k(x_i)$ и~таргетная 
переменная~$\Gamma_t(x_i)$;
\item вводится набор метрик~$\rho_m$, 
оценивается их релевантность $d_f(\hat{\phi}(x)\vartheta_{\mathbf{m}{\bm 
\alpha}}(\mathbf{c}_{\bm \alpha},{p})$,\linebreak 
$\hat{\phi}(x)\vartheta_{\mathbf{m}{\bm \alpha}}(\overline{\mathbf{c}}_{\bm 
\alpha}, {p}))$ для каждого класса~$\mathbf{c}_{\bm \alpha}$ 
значений $t$-й переменной и~отбираются наиболее релевантные~$\rho_m$; 
\item посредством каждой из отобранных~$\rho_m$ по\-рож\-да\-ют\-ся 
синтетические признаки~$\Gamma_{k^\prime}(x_i)$;
\item посредством 
вычислений $1\hm- P(d_f, \hat{\zeta}\mathbf{X}, a, k^\prime, t)$  
и~В-ал\-го\-рит\-ма отбирается минимальное чис\-ло признаков максимальной 
<<ин\-фор\-ма\-тив\-ности>>;
\item применяется алгоритм прогнозирования 
таргетной переменной (корректор по Жу\-рав\-лё\-ву--Ру\-да\-кову). 
\end{enumerate}

\begin{table*}\small
\begin{center}
\begin{tabular}{|l|c|c|}
\multicolumn{3}{p{140mm}}{Ранговые корреляции между экспериментальными 
и~расчетными значениями $EC_{50}$ и~других величин хемокиномного анализа: $r$~--- 
коэффициент ранговой корреляции на обучении; $r_c$~---  на контроле. Усреднение~$r$ 
и~$r_c$ проводилось по 2400~выборкам хемокиномных данных}\\
\multicolumn{3}{c}{\ }\\[-6pt]
\hline
\multicolumn{1}{|c|}{{Эксперимент}}&$r$&$r_c$\\
\hline
{\boldmath $f_{\theta_k}$}\textbf{-алгоритмы, корректор~--- нейросеть}&\boldmath{$0{,}88\pm 
0{,}15$}&\boldmath{$0{,}86\pm0{,}20$}\\
Синтетические $\Gamma_{k^\prime}(x_i)$, корректор~--- нейросеть (2~слоя)&$0{,}45\pm 
0{,}22$&$0{,}22\pm 0{,}21$\\
Синтетические $\Gamma_{k^\prime}(x_i)$, корректор~--- нейросеть 
(10~слоев)&$0{,}52\pm 0{,}25$&$0{,}21\pm 0{,}20$\\
Синтетические $\Gamma_{k^\prime}(x_i)$, корректор~--- <<случайный лес>>, 
вариант~1&$0{,}98\pm 0{,}15$&$0{,}67\pm 0{,}31$\\
Синтетические $\Gamma_{k^\prime}(x_i)$, корректор~--- <<случайный лес>>, 
вариант~2&$0{,}99\pm 0{,}14$&$0{,}71\pm 0{,}35$\\
\textbf{Синтетические {\boldmath $\Gamma_{k^\prime}(x_i)$}, полиномные корректоры, 
вариант~1}&\boldmath{$0{,}93\pm 0{,}11$}&\boldmath{$0{,}90\pm 0{,}23$}\\
\textbf{Синтетические {\boldmath $\Gamma_{k^\prime}(x_i)$}, полиномные корректоры, 
вариант~2}&\boldmath{$0{,}95\pm0{,}08$}&\boldmath{$0{,}86\pm 0{,}27$}\\
\hline
\end{tabular}
\end{center}
\end{table*}

\section{Экспериментальная апробация }

    Формализм апробирован на комплексе задач\linebreak фармакоинформатики: 
получение количественных оценок ингибирования киназ протеома 
перспективными лекарствами (хемокиномный анализ)~[9]. Использованы 
2400~выборок данных <<\mbox{мо\-ле\-ку\-ла}--свой\-ст\-во>> из ProteomicsDB; 
свойства молекул включили константы $EC_{50}$ и~активности для 
концентраций~$(E_j(C_i))$.

     Исходные признаки $\Gamma_k(x_i)$ определялись как булевы 
инварианты над множествами $\chi$-це\-пей и~$\chi$-уз\-лов 
хемографов~$x_i$, как и~в~[9]. Таргетная $\Gamma_t(x_i)$ определялась как 
числовое значение прогнозируемого свойства. В~качестве~$\rho_m$ 
использовались функции расстояния на множествах, векторах и~э.ф.р.\ (всего 
65~функций из справочника~[2]). Классы~$\mathbf{c}_{\bm{\alpha}}$ 
определялись как квартили значений~$\Gamma_t$. Векторы элементов 
$L(T(\mathbf{X}))$ формировались из оценок $v^+_\alpha$, $v^-_\alpha$ 
и~$d_\alpha$~\cite{4-tor} для каждого~$\mathbf{c}_{\bm{\alpha}}$. 
Релевантность~$\rho_m$ по $d_f(\hat{\phi}(x),\vartheta_{\mathbf{m}{\bm 
\alpha}}(\mathbf{c}_{\bm{\alpha}},{p}), 
\hat{\phi}(x)\vartheta_{\mathbf{m}{\bm \alpha}} 
(\overline{\mathbf{c}}_{\bm{\alpha}}, {p}))$ оценивалась для 
каждого~$\mathbf{c}_{\bm{\alpha}}$, $d_f$~--- максимальное уклонение. 
Синтетические признаки~$\Gamma_{k^\prime}(x_i)$ по\-рож\-да\-лись всеми 
перечисленными выше способами; их отбор проводился В-ал\-го\-рит\-мом 
с~использованием $1\hm- {P}(d_f, \hat{\zeta}\mathbf{X}, a, 
     k^\prime, t)$. 
     
     В качестве корректоров использовались нейронные сети с~несколькими 
слоями (от~2 до~10) с~функцией активации softmax, полиномы различных 
конструкций (более 20~формул, в~том числе квазиполиномные модели 
с~элементарными функциями) и~<<случайные леса>> решающих деревьев. 
Оператор семплирования~$\hat{\zeta}$ был реализован как десятикратная  
кросс-ва\-ли\-да\-ция с~делением каждой выборки объектов на 80\% 
(обучение) и~20\% (конт\-роль). Результаты экспериментов суммированы 
в~таблице.
     

     
     Наилучший результат применения нового <<топологического>> 
формализма с~полиномным корректором ($r_c\hm=0{,}90\hm\pm0{,}23$) 
немного превзошел наилучший результат применения \mbox{метода} опорных 
функций (композиций вида $f_{\theta_k} \hm= g(f_1(\sum \omega_k^j x_k), 
\ldots\linebreak \ldots , f_l(\sum \omega_k^j x_k))$, см.~[9]), для которого 
$r_c\hm=0{,}86\hm\pm0{,}20$. Полиномными формулами, наиболее часто 
показывавшими наилучший результат, оказались полиномы 1-й или 2-й 
степеней с~произведениями переменных первой степени, полиномы 5-й 
степени, квазиполиномы 5-й степени с~сигмоидами и~Фурье-по\-ли\-но\-мы  
3-й степени.
     
     Нейросетевые корректоры всех использованных конфигураций 
отличались крайне низкими показателями ($r\hm=0{,}45\hm\pm0{,}22$, 
$r_c\hm=0{,}22\hm\pm0{,}21$), а~<<случайный лес>> приводил 
к~существенному переобучению (см.\ таб\-ли\-цу). При этом в~290 
из~2400~выборок данных (12\%) <<случайный лес>> приводил к~улучшению 
результатов по сравнению с~наилучшими полиномными корректорами, 
а~в~1670 из 2400~выборок данных (70\%)~--- к~ухудшению.
     
     
     Анализ синтетических признаков $\Gamma_{k^\prime}(x_i)$, 
вошедших в~наилучшие полиномные модели, показал, что среди более 
информативных (по оценке $1\hm- P(d_f, \hat{\zeta}\mathbf{X},  
a, k^\prime, t)$) признаков чаще всего встречались признаки, порождаемые 
с~использованием э.ф.р.\ на основе опорных цепей (теорема~1 в~1-й части 
работы~[1]), среди наименее информативных~--- исходные признаки 
$\Gamma_k(x_i)$ и~признаки на основе отдельных расстояний 
$\rho_m(\mathbf{c}_{\bm{\alpha}} , \Gamma_k^{-1}(\Gamma_k(x_i))$. 
Функциями~$\rho_m$, наиболее часто порождающими информативные 
$\Gamma_{k^\prime}(x_i)$ на пространстве э.ф.р., оказались максимальное 
уклонение Колмогорова, <<косое>> расстояние, метрики $\mathrm{Lp}$, 
Реньи, $\chi2$, фон Мизеса, инженерная~\cite{2-tor}. В~среднем по всем 
выборкам данных эти~7~разновидностей~$\rho_m$ порождали более 50\% 
самых информативных признаков~$\Gamma_{k^\prime}(x_i)$, отобранных  
В-ал\-го\-рит\-мом.

\vspace*{-6pt}

\section{Заключение}

\vspace*{-2pt}

    Предлагаемый подход к~порождению информативных синтетических 
признаков подразумевает последовательные трансформации описаний 
объекта:\\[-13pt]
\begin{enumerate}[(1)]
\item исходное множество значений признаков;\\[-13.5pt]
\item множество 
соответствующих элементов решетки;\\[-13.5pt] 
\item ~множество расстояний 
(измеряемых посредством~$\rho_m$) между элементами решетки, 
соответствующими классам и~признакам;\\[-13.5pt]
\item множество э.ф.р.\ расстояний, 
измеренных заданными~$\rho_m$;\\[-13.5pt] 
\item множество синтетических признаков 
объ-\linebreak екта.
\end{enumerate}

\noindent
 Использование многочисленных метрик на стадии порождения 
признаков позволяет рассматривать развиваемый формализм как вариант 
развития идеологии АВО (алгоритмы вычисления \mbox{оценок}) научной школы 
Ю.\,И.~Журавлёва. Экспериментальная апробация предлагаемого подхода на 
2400~однородных задачах фармакоинформатики позволила повысить 
аккуратность и~обобщающую способность алгоритмов. 


{\small\frenchspacing
 {\baselineskip=10.6pt
 %\addcontentsline{toc}{section}{References}
 \begin{thebibliography}{99}
  
  \bibitem{1-tor}
\Au{Торшин И.\,Ю.} О~порождении синтетических признаков на основе опорных цепей 
и~произвольных метрик в~рамках топологического подхода к~анализу данных. Часть~1. 
Включение в~формализм эмпирических функций расстояния~// Информатика и~её 
применения, 2024. Т.~18. Вып.~1. С.~71--77. doi: 10.14357/19922264240110. EDN: 
RIVOXR.
  \bibitem{2-tor}
  \Au{Деза Е.\,И., Деза~М.\,М.} Энциклопедический словарь расстояний~/ Пер. с~англ.~--- М.: Наука, 
2008. 444~с. (\Au{Deza~E.\,I., Deza~M.\,M.} {Dictionary of distances}.~--- North-Holland: 
Elsevier, 2006. 412~p. doi: 10.1016/B978-0-444-52087-6.X5000-8.)
  \bibitem{3-tor}
  \Au{Torshin I.\,Y., Rudakov~K.\,V.} Combinatorial analysis of the solvability properties of 
the problems of recognition and completeness of algorithmic models. Part~2: Metric approach 
within the framework of the theory of classification of feature values~// Pattern Recognition Image 
Analysis, 2017. Vol.~27. No.\,2. P.~184--199. doi: 10.1134/S1054661817020110.
  \bibitem{4-tor}
\Au{Торшин И.\,Ю.} О~формировании множеств прецедентов на основе таблиц 
разнородных признаковых описаний методами топологической теории анализа данных~// 
Информатика и~её применения, 2023. Т.~17. Вып.~3. С.~2--7. doi: 
10.14357/19922264230301. EDN: AQEUYO.
  \bibitem{5-tor}
  \Au{Torshin I.\,Yu., Rudakov~K.\,V.} On the procedures of generation of numerical features 
over partitions of sets of objects in the problem of predicting numerical target variables~// 
Pattern Recognition Image Analysis, 2019. Vol.~29. No.\,4. P.~654--667. doi: 
10.1134/S1054661819040175. 
  \bibitem{6-tor}
  \Au{Torshin I.\,Y., Rudakov~K.\,V.} Combinatorial analysis of the solvability properties of 
the problems of recognition and completeness of algorithmic models. Part~1: Factorization 
approach~// Pattern Recognition Image Analysis, 2017. Vol.~27. No.\,1. P.~16--28. doi: 
10.1134/S1054661817010151.
  \bibitem{7-tor}
  \Au{Sosa-Cabrera G., G$\acute{\mbox{o}}$mez-Guerrero~S.,  
Garc$\acute{\iota}$a-Torres~M., Schaerer~C.\,E.} Feature selection: A~perspective on inter-attribute 
cooperation~// Int. J. Data Science Analytics, 2024. Vol.~17. P.~139--151. doi:  
10.1007/s41060-023-00439-z.
  \bibitem{8-tor}
  \Au{Torshin I.\,Y.} Optimal dictionaries of the final information on the basis of the solvability 
criterion and their applications in bioinformatics~// Pattern Recognition Image Analysis, 2013. 
Vol.~23. No.\,2. P.~319--327. doi: 10.1134/S1054661813020156.
  \bibitem{9-tor}
\Au{Торшин И.\,Ю.} О~задачах оптимизации, воз\-ни\-ка\-ющих при применении 
топологического анализа данных к~поиску алгоритмов прогнозирования 
с~фиксированными корректорами~// Информатика и~её применения, 2023. Т.~17. Вып.~2. 
С.~2--10. doi: 10.14357/19922264230201. EDN: IGSPEW.

\end{thebibliography}

 }
 }

\end{multicols}

\vspace*{-8pt}

\hfill{\small\textit{Поступила в~редакцию 09.04.24}}

\vspace*{6pt}

%\pagebreak

%\newpage

%\vspace*{-28pt}

\hrule

\vspace*{2pt}

\hrule



\def\tit{ON THE GENERATION OF~SYNTHETIC FEATURES BASED~ON~SUPPORT~CHAINS 
AND~ARBITRARY METRICS\\ WITHIN THE~FRAMEWORK OF~A~TOPOLOGICAL 
APPROACH\\ TO~DATA ANALYSIS. PART~2. EXPERIMENTAL TESTING 
ON~PHARMACOINFORMATICS PROBLEMS}


\def\titkol{On the generation of~synthetic features based on~support chains 
and~arbitrary metrics} % within the~framework of~a~topological  approach to~data analysis. Part~2. Experimental testing  on~pharmacoinformatics problems}


\def\aut{I.\,Yu.~Torshin}

\def\autkol{I.\,Yu.~Torshin}

\titel{\tit}{\aut}{\autkol}{\titkol}

\vspace*{-15pt}


\noindent
Federal Research Center ``Computer Science and Control'' of the Russian Academy of 
Sciences, 44-2~Vavilov Str., Moscow 119333, Russian Federation

\def\leftfootline{\small{\textbf{\thepage}
\hfill INFORMATIKA I EE PRIMENENIYA~--- INFORMATICS AND
APPLICATIONS\ \ \ 2024\ \ \ volume~18\ \ \ issue\ 2}
}%
 \def\rightfootline{\small{INFORMATIKA I EE PRIMENENIYA~---
INFORMATICS AND APPLICATIONS\ \ \ 2024\ \ \ volume~18\ \ \ issue\ 2
\hfill \textbf{\thepage}}}

\vspace*{3pt}
  
  


\Abste{Consideration of precedent relationships between features and a target variable in the 
form of sets of Boolean lattice elements indicates the possibility of generating synthetic features 
using metric distance functions. Approaches to ($i$)~assessing the relevance (``informativeness'') 
of metrics in relation to the problems being solved; ($ii$)~generating; and ($iii$)~selecting synthetic 
features that are more informative than the original feature descriptions are formulated. The 
results of topological analysis of~2400~samples of ``molecule--property'' data
from 
ProteomicsDB made it possible to obtain fairly effective algorithms for 
predicting the properties of molecules (rank correlation in cross-validation is~$0.90\pm 0.23$). 
Using this sample of problems, metrics have been established\linebreak\vspace*{-12pt}}

\Abstend{that most often generate 
informative synthetic features: maximum Kolmogorov deviation, ``oblique'' distance, and Lp, Renyi, 
and von Mises metrics. To solve the studied set of problems, the advantage of polynomial 
correctors compared to neural network and random forest correctors is shown.}

\KWE{topological data analysis; lattice theory; algebraic approach of Yu.\,I.~Zhuravlev; 
pharmacoinformatics}




\DOI{10.14357/19922264240207}{OTXCUD}

%\vspace*{-12pt}

\Ack

\vspace*{-3pt}


\noindent
The research was funded by the Russian Science Foundation, project No.\,23-21-00154. The 
research was carried out using the infrastructure of the Shared Research Facilities ``High 
Performance Computing and Big Data'' (CKP ``Informatics'') of FRC CSC RAS (Moscow).
 


  \begin{multicols}{2}

\renewcommand{\bibname}{\protect\rmfamily References}
%\renewcommand{\bibname}{\large\protect\rm References}

{\small\frenchspacing
 {%\baselineskip=10.8pt
 \addcontentsline{toc}{section}{References}
 \begin{thebibliography}{9} 
 
 %\vspace*{-3pt}
  \bibitem{1-tor-1}
\Aue{Torshin, I.\,Yu.} 2024. O~porozhdenii sinteticheskikh priznakov na osno\-ve opor\-nykh 
tsepey i~proizvol'nykh metrik v~ram\-kakh topologicheskogo podkhoda k~analizu dannykh. 
Chast'~1. Vklyuchenie v~formalizm empiricheskikh funktsiy rasstoyaniya [On the generation 
of synthetic features based on support chains and arbitrary metrics within a~topological approach 
to data analysis. Part~1. Inclusion of empirical distance functions into the formalism]. 
\textit{Informatika i~ee Primeneniya~--- Inform Appl.} 18(1):71--77. doi: 
10.14357/19922264240110. EDN: RIVOXR.
  \bibitem{2-tor-1}
\Aue{Deza, E.\,I., and M.\,M.~Deza.} 2006. \textit{Dictionary of distances}. North-Holland: 
Elsevier. 412~p. doi: 10.1016/B978-0-444-52087-6.X5000-8.
  \bibitem{3-tor-1}
\Aue{Torshin, I.\,Yu., and K.\,V.~Rudakov.} 2017. Combinatorial analysis of the solvability 
properties of the problems of recognition and completeness of algorithmic models. Part~2: 
Metric approach within the framework of the theory of classification of feature values. 
\textit{Pattern Recognition Image Analysis} 27(2):184--199. doi: 10.1134/S1054661817020110.
  \bibitem{4-tor-1}
\Aue{Torshin, I.\,Yu.} 2023. O~formirovanii mnozhestv pretsedentov na osnove tablits 
raznorodnykh priznakovykh opisaniy metodami topologicheskoy teorii analiza dannykh [On the 
formation of sets of precedents based on tables of heterogeneous feature descriptions by methods 
of topological theory of data analysis]. \textit{Informatika i~ee Primeneniya~--- Inform Appl.} 
17(3):2--7. doi: 10.14357/19922264230301. EDN: AQEUYO.
  \bibitem{5-tor-1}
\Aue{Torshin, I.\,Yu., and K.\,V.~Rudakov.} 2019. On the procedures of generation of 
numerical features over partitions of sets of objects in the problem of predicting numerical target 
variables. \textit{Pattern Recognition Image Analysis} 29(4):654--667. doi: 
10.1134/S1054661819040175.
  \bibitem{6-tor-1}
\Aue{Torshin, I.\,Y., and K.\,V.~Rudakov.} 2017. Combinatorial analysis of the solvability of 
the problems of recognition, completeness of algorithmic models. Part~1: Factorization 
approach. \textit{Pattern Recognition Image Analysis} 27(1):16--28. doi: 
10.1134/S1054661817010151.
  \bibitem{7-tor-1}
\Aue{Sosa-Cabrera, G., S.~Gуmez-Guerrero, \mbox{M.~Garc$\acute{\!\mbox{{\ptb{\i}}}}$a}-Torres, 
and C.\,E.~Schaerer.} 2024. Feature selection: A~perspective on inter-attribute cooperation. \textit{Int. J. 
Data Science Analytics} 17:139--151. doi: 10.1007/s41060-023-00439-z.
  \bibitem{8-tor-1}
\Aue{Torshin, I.\,Y.} 2013. Optimal dictionaries of the final information on the basis of the 
solvability criterion and their applications in bioinformatics. \textit{Pattern Recognition Image 
Analysis}  23(2):319--327. doi: 10.1134/ S1054661813020156.
  \bibitem{9-tor-1}
\Aue{Torshin, I.\,Yu.} 2023. O~zadachakh optimizatsii, voznikayushchikh pri primenenii 
topologicheskogo analiza dannykh k~poisku algoritmov prognozirovaniya s~fiksirovannymi 
korrektorami [On optimization problems arising from the application of topological data analysis 
to the search for forecasting algorithms with fixed correctors]. \textit{Informatika i~ee 
Primeneniya~--- Inform Appl.} 17(2):2--10. doi: 10.14357/19922264230201. EDN: IGSPEW.

\end{thebibliography}

 }
 }

\end{multicols}

\vspace*{-6pt}

\hfill{\small\textit{Received April 9, 2024}} 

\vspace*{-12pt}


\Contrl

\vspace*{-3pt}

\noindent
\textbf{Torshin Ivan Y.} (b.\ 1972)~--- Candidate of Science (PhD) in physics and mathematics, 
Candidate of Science (PhD) in chemistry, leading scientist, Federal Research Center ``Computer 
Science and Control'' of the Russian Academy of Sciences, 44-2~Vavilov Str, Moscow 119333, 
Russian Federation; \mbox{tiy135@yahoo.com}
  
  



\label{end\stat}

\renewcommand{\bibname}{\protect\rm Литература}           %7
\def\stat{grusho}

\def\tit{АРХИТЕКТУРНЫЕ РЕШЕНИЯ В~ЗАДАЧЕ ВЫЯВЛЕНИЯ МОШЕННИЧЕСТВА ПРИ~АНАЛИЗЕ 
ИНФОРМАЦИОННЫХ ПОТОКОВ В~ЦИФРОВОЙ ЭКОНОМИКЕ$^*$}

\def\titkol{Архитектурные решения в~задаче выявления мошенничества при~анализе 
информационных потоков в
%~цифровой 
экономике}

\def\aut{А.\,А.~Грушо$^1$, М.\,И.~Забежайло$^2$, Н.\,А.~Грушо$^3$, 
Е.\,Е.~Тимонина$^4$}

\def\autkol{А.\,А.~Грушо, М.\,И.~Забежайло, Н.\,А.~Грушо, 
Е.\,Е.~Тимонина}

\titel{\tit}{\aut}{\autkol}{\titkol}

\index{Грушо А.\,А.}
\index{Забежайло М.\,И.}
\index{Грушо Н.\,А.}
\index{Тимонина Е.\,Е.}
\index{Grusho A.\,A.}
\index{Zabezhailo M.\,I.}
\index{Grusho N.\,A.}
\index{Timonina E.\,E.}


{\renewcommand{\thefootnote}{\fnsymbol{footnote}} \footnotetext[1]
{Работа частично поддержана РФФИ (проекты 18-29-03081 и~18-07-00274).}}


\renewcommand{\thefootnote}{\arabic{footnote}}
\footnotetext[1]{Институт проблем информатики Федерального исследовательского центра <<Информатика и~управление>> 
Российской академии наук, grusho@yandex.ru}
\footnotetext[2]{Институт проблем информатики Федерального исследовательского центра <<Информатика и~управление>> 
Российской академии наук, m.zabezhailo@yandex.ru}
\footnotetext[3]{Институт проблем информатики Федерального исследовательского центра <<Информатика и~управление>> 
Российской академии наук, info@itake.ru}
\footnotetext[4]{Институт проблем информатики Федерального исследовательского центра <<Информатика и~управление>> 
Российской академии наук, eltimon@yandex.ru}

\vspace*{-12pt}
   

 
  
  \Abst{Cформулирован подход к~исследованию некоторых видов мошенничества в~цифровой 
экономике с~использованием причинно-следственных связей. Во всех видах рассматриваемых 
мошенничеств должно наблюдаться несоответствие между целями финансовых транзакций 
и~реальной стоимостью достижения этих целей. Данные о транзакциях можно собирать, 
наблюдая информационные потоки, в~которых отражаются эти транзакции. Архитектура сбора 
данных и~их анализа может быть организована с~помощью распределенных реестров 
с~централизованным консенсусом, что позволяет создать аналог электронной бухгалтерской 
книги, фиксирующей финансово-экономическую деятельность субъектов цифровой экономики в~регионе. 
  Рассматриваемые методы выявления мошенничества основаны на противоречиях 
между действиями, описанными в~транзакциях, и~информацией, содержащейся в~планах, 
стандартах, прецедентах и~др. Рассмотрен метод, основанный на некоторой упрощенной схеме 
реализации абстрактного проекта. Для выявления противоречий необходимо проводить анализ 
от следствия к~причине, т.\,е.\ искать аномалии в~информации, описывающей порождение 
наблюдаемых следствий. 
  Показано, как в~реализации проекта можно выделять простые <<необходимые условия>> 
нарушения при\-чин\-но-след\-ст\-вен\-ных связей, т.\,е.\ множество <<необходимых условий>>, 
нарушение которых свидетельствует о наличии мошенничества. Это множество <<необходимых 
условий>> можно назвать метаданными для контроля проекта на выявление мошенничества.} 
 
 
  \KW{цифровая экономика; информационные потоки; при\-чин\-но-след\-ст\-вен\-ные связи; 
выявление мошеннических схем} 

\DOI{10.14357/19922264190204}
  
\vspace*{-4pt}


\vskip 10pt plus 9pt minus 6pt

\thispagestyle{headings}

\begin{multicols}{2}

\label{st\stat}

\section{Введение}

\vspace*{3pt}

  В работе сформулирован подход к~исследованию некоторых видов 
мошенничества в~цифровой экономике с~использованием  
при\-чин\-но-след\-ст\-вен\-ных связей. Рассматриваются три вида мошенничества, 
а именно:
  \begin{enumerate}[(1)]
\item отмыв денег; 
\item обман при выполнении договорных обязательств при реализации 
технических проектов (строительные проекты и~др.); 
\item незаконный вывод денег. 
\end{enumerate}

  Названные виды мошенничества могут быть сведены к~решению одного типа 
задач. Для отмывания денег источник должен заключать фиктивные контракты, 
в~соответствии с~которыми будут переводиться средства за заведомо ненужную 
работу и~материалы. 
  
  Мошенничество, связанное с~невыполнением договорных обязательств, связано 
со снижением качества услуг, качества и~количества закупаемых 
материалов, выполнением работ с~ненадлежащим качеством. 
  
  Вывод денег связан с~переводом средств фир\-мам-од\-но\-днев\-кам, которые 
заведомо не могут выполнить обязательства по контрактам, за которые им 
переводятся средства. 
  
  Таким образом, во всех трех видах рассматриваемых мошенничеств должно 
наблюдаться несоответствие между целями финансовых транзакций и~реальной 
стоимостью достижения этих целей. Данные о транзакциях можно собирать, 
наблюдая информационные потоки, в~которых отражаются эти транзакции. 
  
  Однако для наблюдения таких информационных потоков необходимо создавать 
архитектуру\linebreak телекоммуникационной системы, позволяющей перехватывать 
и~собирать данные о всех транзакциях. Например, такая архитектура может быть 
организована с~помощью распределенных реестров с~централизованным 
консенсусом, т.\,е.\ все информационные потоки, сформированные в~цифровой 
экономике и~несущие информацию о транзакциях, проходят через некоторый 
центральный узел, запоминающий их в~форме распределенного реестра. Такие 
реестры могут дублироваться в~аналогичных центрах различных регионов, что 
позволяет создать аналог электронной бухгалтерской книги, фиксирующей 
фи\-нан\-со\-во-эко\-но\-ми\-че\-скую деятельность субъектов цифровой экономики. Такой 
подход предложено реализовать на базе системы ситуационных центров, что 
отражено в~работах~[1, 2].
  
  Собранная из информационных потоков информация о~транзакциях, т.\,е.\ 
о~контрактах, договорах, платежах, отчетах, закупленных материалах, 
характеристиках исполнителей работ и~др., собирается в~базе данных в~указанном 
центре. Согласно теории интеллектуальных сис\-тем~[3], эту базу данных можно 
называть базой фактов (БФ). Базу фактов можно представить как бинарную мат\-ри\-цу, 
строки которой описывают характеристики, входящие в~транзакции, а столбцы 
нумеруются характеристиками. Строки матрицы будем называть 
\textit{объектами}~[4, 5]. 
  
  Рассматриваемые в~работе методы выявления мошенничества будут основаны 
на противоречиях между действиями, описанными в~транзакциях, и~информацией, 
содержащейся в~планах, стандартах, прецедентах и~др. Для нахождения 
противоречий в~архитектуре центра предусмотрена другая база данных~--- база 
знаний (БЗ)~\cite{3-gr, 6-gr}, которая устроена так же, как БФ. 
  
  Информация в~БЗ собирается на основе положительного опыта или расчетов. 
Используя БЗ, можно выводить факты нарушения при\-чин\-но-след\-ст\-вен\-ных 
связей. Нарушения при\-чин\-но-след\-ст\-вен\-ных связей будем называть 
\textit{аномалиями}. 
  
  Для упрощения дальнейшее изложение будет вестись в~рамках поиска 
противоречий при выполнении некоторого абстрактного проекта. Выявление 
аномалий будет происходить на основе фактов из БФ с~помощью знаний из БЗ 
методами искусственного интеллекта и~интеллектуального анализа 
данных~\cite{6-gr}. 

\vspace*{-10pt}
  
  \section{Модели}
  
  \vspace*{-3pt}
  
  Наиболее сложная из рассмотренных выше задач~--- выявление противоречий, 
т.\,е.\ использование БЗ для получения новых знаний и~выявление аномалий из 
полученных фактов. 
  
  Все способы выявления противоречий основаны на определении 
  причинно-следственных связей. При этом противоречия в~параметрах транзакций по 
отношению к~требуемым в~БЗ составляют сущность аномалий. 
  
   Далее будет рассмотрен метод, основанный на некоторой упрощенной схеме 
реализации абстрактного проекта. 
  
  Каждый проект имеет цель: например, цель представляет собой построение 
некоторой системы. Воспользуемся структурным подходом, который позволяет 
строить проект на основе разбиения системы на подсистемы и~определения 
взаимодействий подсистем~\cite{7-gr}. При этом каждая подсистема также 
представима структурной моделью. 
  
  Как сама система, так и~каждая ее подсистема имеют свой функционал 
и~спецификацию, па\-ра\-мет\-ры настройки и~домены параметров настройки. Кроме 
этих характеристик существует множество характеристик, связанных 
с~<<жизненным циклом>> создания системы. Сюда входят работы, ресурсы, 
сроки выполнения работ по созданию подсистем и~самой системы, стоимости 
компонентов и~материалов, стоимости работ, схемы поставок, договорные 
обязательства и~др. Все характеристики связаны между собой, поэтому можно 
говорить о стоимости и~времени изготовления структурных компонентов системы. 
  
  Одной из важнейших характеристик является смета (система смет для 
подсистем). Смета сопоставляет каждому компоненту системы стоимость его 
изготовления и~настройки. 
  
  Схема построения системы может быть пред\-став\-ле\-на диаграммой, 
изображенной на рис.~1. 

{ \begin{center}  %fig1
 \vspace*{9pt}
   \mbox{%
 \epsfxsize=79mm 
 \epsfbox{gru-1.eps}
 }


\vspace*{9pt}


\noindent
{{\figurename~1}\ \ \small{Диаграмма достижения цели}}
\end{center}
}

\vspace*{9pt}

\addtocounter{figure}{1}
  
  


  Представленная на рис.~1 диаграмма позволяет описать основные классы 
возможных противоречий при достижении цели. Противоречия возникают, когда 
данные БФ не соответствуют требуемым характеристикам. 
  
  
  \section{Потенциальные классы аномалий при~достижении цели}
  
  Выделим четыре потенциальных класса противоречий, которые показывают, 
каким образом нужно искать эти противоречия.
  
 
  Противоречие цели и~проекта (рис.~2) возникает при отсутствии обоснования 
или в~случае логического противоречия между возможностями проектируемого 
функционала и~целью системы. Отметим, что в~проект входят сроки, перечень 
работ, материалы, настройки, которые описываются соответствующими 
параметрами и~допустимыми значениями этих параметров. Проект формируется 
на основе БЗ и~расчетов, исходя из информации, полученной по аналогии 
с~другими проектами и~решениями, которые считаются апробированными. 
  
  Отметим, что цель порождает проект и~в этом смысле является причиной 
проекта. Однако для анализа противоречий необходимо двигаться по штриховой 
стрелке диаграммы (см.\ рис.~2) от проекта к~цели. В~самом деле, любой компонент 
проекта направлен на теоретическое достижение цели. Цель~--- сложный объект, 
поэтому в~проекте могут возникнуть характеристики, противоречащие хотя бы 
некоторым характеристикам цели. Это делает проект противоречивым, но вывод 
об этом может быть сделан только на уровне описания цели. 
  

  Противоречия между проектом и~его реализацией, исключая настройки 
(рис.~3), могут возникать, например, при закупке исполнителем материалов более 
низкого качества по более низким ценам, при попытках достижения требуемых 
сроков работы за счет снижения качества выполнения работ, за счет нахождения 
<<объективных>> причин для увеличения сроков работы и,~следовательно, 
увеличения цены реализации проекта. 


  Для выявления указанных противоречий необходимо двигаться по диаграмме 
(см.\ рис.~3) в~обратную сторону в~соответствии со~штриховыми стрелками. 
Действительно, выявить противоречия между характеристиками закупленных 
материалов и~требуемыми по проекту можно только при обращении к~проекту 
и~его спецификациям. Манипуляции со сроками работы также можно выявить 
только при обращении к~соответствующим расчетам в~проекте. Задержки в~сроках 
работы, связанные с~поставками материалов, можно определить только на 
предыдущем этапе диаграммы (см.\ рис.~3) в~описании проекта. 


  


  Противоречия между реализацией проекта и~его настройкой (рис.~4) возникает, 
когда не удается добиться требуемых значений параметров функционала, не 
удается обеспечить необходимый уровень\linebreak\vspace*{-12pt}

{ \begin{center}  %fig2
 \vspace*{-6pt}
   \mbox{%
 \epsfxsize=16mm 
 \epsfbox{gru-2.eps}
 }


\vspace*{6pt}


\noindent
{{\figurename~2}\ \ \small{Противоречия цели и~проекта}}
\end{center}
}

%\vspace*{9pt}

\addtocounter{figure}{1}

{ \begin{center}  %fig3
 \vspace*{6pt}
    \mbox{%
 \epsfxsize=79mm 
 \epsfbox{gru-3.eps}
 }


\end{center}

\vspace*{-2pt}


\noindent
{{\figurename~3}\ \ \small{Противоречия проекта и~его реализации (без настройки)}}
}

\vspace*{6pt}

\addtocounter{figure}{1}

{ \begin{center}  %fig4
 \vspace*{1pt}
   \mbox{%
 \epsfxsize=54.5mm 
 \epsfbox{gru-4.eps}
 }


\end{center}


\noindent
{{\figurename~4}\ \ \small{Противоречия реализации проекта и~его на\-стройки}}
}

%\vspace*{9pt}

\addtocounter{figure}{1}

{ \begin{center}  %fig5
 \vspace*{5pt}
    \mbox{%
 \epsfxsize=79mm 
 \epsfbox{gru-5.eps}
 }


\end{center}



\noindent
{{\figurename~5}\ \ \small{Противоречия цели и~достигнутой реализации проекта}}
}

\vspace*{6pt}

\addtocounter{figure}{1}

\noindent
 качества реализации проекта. Для 
определения противоречия в~настройках надо опять же двигаться по диаграмме 
(см.\ рис.~4) в~обратную сторону по штриховым стрелкам, так как для выявления 
характеристик результатов работы, которые не дают возможности реализации 
определенного функционала, необходимо иметь информацию о результатах этой 
работы. 


  



  Противоречие между целью и~достигнутой реализацией проекта (рис.~5) 
возникает, когда реализованная система не позволяет достичь цели. В~этом случае 
опять противоречие нужно искать, двигаясь от цели к~реальному достигнутому 
функционалу по штриховой стрелке (см.\ рис.~5).
  
  Суммируя положения, изложенные в~данном разделе, приходим к~выводу, что 
для выявления противоречий необходимо проводить анализ от следствия 
к~причине, т.\,е.\ искать аномалии в~информации, описывающей порождение 
наблюдаемых следствий. 
  
  
  \section{Связь противоречий и~причин}
  
  Прежде чем построить связь между причинами и~противоречиями, кратко 
опишем простейшую модель связи этих понятий. Причины и~противоречия будут 
сформулированы для представления компонентов системы как объектов, 
обладающих наборами известных характеристик~\cite{4-gr, 5-gr}. 
  
  Пусть $U\hm=\{\alpha, \beta, \ldots\}$~--- совокупность характеристик 
(пространство характеристик). Согласно~\cite{4-gr} \textit{объектом}~$O$ 
называется любое подмножество характеристик $O\hm\subseteq U$. Рассмотрим 
последовательность объектов, возможно в~различных пространствах 
характеристик. 
  
  \smallskip
  
  \noindent
  \textbf{Определение~1.}\ Объект~$P$ с~числом характеристик, большим или 
равным~2, является \textit{причиной} объекта (\textit{свойства})~$B$ в~цепочке 
наблюдаемых объектов тогда и~только тогда, когда выполнены следующие 
условия:
  \begin{enumerate}[(1)]
\item для каждого объекта~$C$, если $P\hm\subseteq C$, то $C\mapsto B$, где 
$C\mapsto B$ означает, что объект~$B$ присутствует в~объекте, следующем за 
объектом~$C$;
\item объект~$P$ является минимальным объектом, удовлетворяющим 
условию~1, а~именно: $\forall \alpha\hm\in P$ объект~$P\backslash \{\alpha\}$ 
не является причиной, т.\,е.\ $\exists C:\ \alpha\not\in C$, $P\backslash 
\{\alpha\}\hm\subseteq C$ и~$C\not\mapsto B$, где $C\not\mapsto B$ означает, 
что~$B$ не может содержаться в~объекте, следующем за объектом~$C$. 
\end{enumerate}

  Приведенное определение причины является упрощением причин, 
возникающих в~реальном мире. Например, реальные причины могут возникать\linebreak 
как совокупность характеристик из разных пространств. Одно следствие может 
порождаться разными причинами или возникать из внешних\linebreak и~ненаблюдаемых 
характеристик. Однако пред\-став\-лен\-ная далее формализация позволяет доступно 
изложить при\-чин\-но-след\-ст\-вен\-ные истоки противоречий, которые 
инициируют в~дальнейшем глубокое исследование рассматриваемых процессов.
  
  Будем считать, что для любого интересующего нас свойства~$B$ существует 
причина. Тогда справедлива следующая теорема.
  
  \smallskip
  
  \noindent
  \textbf{Теорема~1.}\ \textit{Для любого свойства~$B$ существует 
единственная причина}. 
  
  \smallskip
  
  \noindent
  Д\,о\,к\,а\,з\,а\,т\,е\,л\,ь\,с\,т\,в\,о\,.\ \ Доказательство будем вести от противного, 
т.\,е.\ предположим, что существуют две причины свойства~$B$: $P$ 
и~$P^\prime$, $P\hm\not= P^\prime$. Тогда существует $\alpha\hm\in U$, которое 
удовлетворяет одному из двух условий:
  \begin{itemize}
\item[(а)] $\alpha\in P$, $\alpha\notin P^\prime$;
\item[(б)] $\alpha\notin P$, $\alpha \in P^\prime$.
\end{itemize}

  Пусть выполняется условие~(б). Тогда $P^\prime\backslash \{\alpha\}$ не 
является причиной по условию~2 определения~1, т.\,е.\ $\exists C$ такое, что 
$\alpha\notin C$, $P^\prime\backslash \{\alpha\}\hm\subseteq C$ и~$C\not\mapsto B$. 
Но если~$B$ произошло и~$P$ его причина, то $C\mapsto B$, что противоречит 
предположению. Теорема~1 доказана.
  
  \smallskip
  
  \noindent
  \textbf{Лемма.} \textit{Если $P$~--- причина появления свойства~$B$, то 
объект~$B$ определяет существование свойства~$P$ в~объекте, 
предшествующем~$B$. }
  
  \smallskip
  
  \noindent
  Д\,о\,к\,а\,з\,а\,т\,е\,л\,ь\,с\,т\,в\,о\,.\ \ Из предположения, что у~каж\-до\-го 
свойства~$B$ есть причина, и~условия, что~$P$ является причиной~$B$, следует, 
что при появлении в~данных свойства~$B$ объект~$C$, предшествующий 
появлению~$B$, содержит как часть объект~$P$. Это следует из теоремы~1 
и~определения причины. 
  
  Докажем принцип <<необходимого условия>>, который, несмотря на простоту 
доказательства, будет играть в~дальнейшем существенную роль.
  
  \smallskip
  
  \noindent
  \textbf{Теорема~2.} \textit{Если~$P$~--- причина появления свойства~$B$ 
и~$A\hm\subseteq P$, то объект~$B$ определяет наличие свойства~$A$ 
в~объекте, предшествующем~$B$}. 
  
  \smallskip
  
  \noindent
  Д\,о\,к\,а\,з\,а\,т\,е\,л\,ь\,с\,т\,в\,о\,.\ \ Пусть в~данных имеется объект~$B$ 
и~$P\mapsto B$, тогда в~силу существования и~единственности причины~$B$ 
в~данных должен существовать объект~$C$, предшествующий~$B$ 
и~содержащий причину~$P$. Поскольку $A\hm\subseteq P$ и~$B$ содержит 
причину~$P$, то $B\mapsto A$. С~учетом леммы теорема~2 доказана.
  
  \smallskip
  
  Пусть даны пространства $U_1, U_2,\ldots$ и~имеется последовательность 
данных (процесс выполнения этапов проекта в~соответствии с~рис.~1) $A, B, 
\ldots$, где каждый объект является подмножеством некоторого 
пространства~$U_i$, $i\hm=1,\ldots$ Тогда в~объекте~$A$ присутствует 
причина~$P$ появления интересующего нас свойства~$C$ в~объекте~$B$. Пусть 
$P\hm\subseteq A$, тогда по теореме~2 $\forall \alpha\hm\in P$:  
$C\mapsto \{\alpha\}$, т.\,е.\ из появления~$C$ следует появление 
характеристики~$\alpha$ в~предшествующем объекте. Это необходимое условие 
того, что~$C$ удовлетворяет причинно-следственным связям развития процесса 
выполнения проекта. Если для~$C$ нет характеристики~$\alpha$, которую можно 
отнести к~причине~$C$, то можно считать, что нарушена  
при\-чин\-но-след\-ст\-вен\-ная связь и~$C$~--- аномальный объект. 
  
  \smallskip
  
  \noindent
  \textbf{Пример.} Если объект~$C$ состоит в~получении суммы~$a$ 
фирмой~$K$, то согласно теореме~2 в~пред\-шест\-ву\-ющем объекте должна 
существовать причина перевода суммы~$a$ на фирму~$K$. Если эта причина 
в~проекте отсутствует, то это можно считать признаком мошеннической схемы. 
Все проекты по предположению собираются из <<кубиков>>, содержащихся в~БЗ. 
Тогда можно сравнить цену объекта~$C$, породившего получение суммы~$a$, 
и~сумму, присутствующую в~смете проекта. Если разница велика, то это либо 
ошибка проекта, либо признак мошеннической схемы.
  
  \section{Поиск противоречий на~основе~принципа <<необходимых~условий>>}
   
  Как было показано в~разд.~3, нахождение противоречий соответствуют 
движению от следствия к~причине. Для каждого объекта в~наблюдаемых данных 
выявление причин его появления является трудоемкой задачей. Кроме того, при 
реализации контроля соблюдения при\-чин\-но-след\-ст\-вен\-ных связей на 
большом множестве участников экономической деятельности задача анализа 
причин становится трудоемкой. Поэтому процедуру контроля необходимо разбить 
на два этапа, где первый этап состоит в~анализе простых <<необходимых 
условий>> проявления мошенничества, когда используется хотя бы одна 
известная характеристика причины. Второй этап (в~режиме офлайн) состоит 
в~выявлении причин, позволяющих провести анализ источников мошеннических 
схем. 
  
  Один из подходов к~выбору <<необходимых условий>> состоит в~построении 
множества подцелей исходной цели проекта (структурный метод построения 
проекта~\cite{7-gr}). Каждая подцель описывается диаграммой на рис.~1, 
и~реализации подцелей должны образовывать полный функционал цели. Это 
является необходимым, но не достаточным условием достижения цели, так как 
при таком подходе отсутствует компонент согласования всех подцелей в~единую 
систему. Однако такой подход значительно упрощает анализ выполнения проекта 
на предмет поиска мошенничества. Если признаки мошенничества будут 
обнаружены в~реализации хотя бы одной из подцелей, то это значит, что 
мошенничество присутствует в~реализации всего проекта. 
  
  Аналогично в~реализации каждого этапа в~любой из подцелей можно выделять 
простые <<необходимые условия>> нарушения при\-чин\-но-след\-ст\-венн\-ых 
связей. 
  
  Таким образом, получается множество <<необходимых условий>>, нарушение 
которых свидетельствует о наличии мошенничества. Это множество 
<<необходимых условий>> можно назвать метаданными~[8, 9] для контроля 
проекта на выявление мошенничества. 
  
  
  \section{Заключение }
  
  В поиске противоречий необходимо от транзакций, соответствующих 
следствиям при\-чин\-но-след\-ст\-вен\-ных связей, переходить к~анализу причин 
наблюдаемых следствий. Это сложная задача, которая связана с~описанием причин 
определенных свойств. 
  
  В работе представлена модель, позволяющая строить множество необходимых 
условий соответствия наблюдаемого следствия вызвавшей его причине. Этот 
подход делает поиск противоречий вполне вычислимой задачей, но не гарантирует 
успех. 
  
  {\small\frenchspacing
 {%\baselineskip=10.8pt
 \addcontentsline{toc}{section}{References}
 \begin{thebibliography}{9}
\bibitem{1-gr}
\Au{Грушо А.\,А., Зацаринный~А.\,А., Тимонина~Е.\,Е.} Блокчейны цифровой экономики на базе 
системы ситуационных центров и~централизованного консенсуса~// Радиолокация, навигация, 
связь: Мат-лы XXV Междунар. научн.-технич. конф.~---
Воронеж: Издательский дом ВГУ, 2019. Т.~6. С.~183--191. 
\bibitem{2-gr}
\Au{Grusho A., Zatsarinny~A., Timonina~E.} A~system approach to information security in 
distributed ledgers on the situational centers platform.~---
Lecture notes in computer science ser.~--- Springer, 2019 
(in press).
\bibitem{3-gr}
\Au{Финн В.\,К.} Искусственный интеллект: Методология, применения, философия.~--- М.: 
Красанд, 2011. 448~с.

\bibitem{5-gr} %4
\Au{Аншаков~О.\,М., Фабрикантова~Е.\,Ф.} ДСМ-ме\-тод автоматического порождения 
гипотез: Логические и~эпистемологические основания.~--- М.: Либроком, 2009. 432~с.

\bibitem{4-gr} %5
\Au{Poelmans J., Elzinga~P., Viaene~S., Dedene~G.} Formal concept analysis in knowledge 
discovery: A~survey~// Conceptual structures: From information to intelligence~/ Eds.\ M.~Croitoru, 
S.~Ferr$\acute{\mbox{e}}$, and D.~Lukose.~--- Lecture notes in computer science 
ser.~--- Berlin--Heidelberg: Springer, 2010. Vol.~6208.  P.~139--153.

\bibitem{6-gr}
\Au{Панкратова~Е.\,С., Финн~В.\,К.} Автоматическое по\-рож\-де\-ние гипотез в~интеллектуальных 
системах.~--- М.: Либроком, 2009. 528~с. 
\bibitem{7-gr}
\Au{Денисов А.\,А., Колесников~Д.\,Н.} Теория больших систем управления.~--- Л.: Энергоиздат, 1982. 488~с.

\bibitem{9-gr}
\Au{Грушо А.\,А., Грушо Н.\,А., Забежайло~М.\,И., Смирнов~Д.\,В., Тимонина~Е.\,Е.} 
Параметризация в~прикладных задачах поиска эмпирических причин~// Информатика и~её 
применения, 2018. Т.~12. Вып.~3. С.~62--66.

\bibitem{8-gr}
\Au{Грушо А.\,А., Грушо Н.\,А., Левыкин~М.\,В., Тимонина~Е.\,Е.} Методы идентификации 
захвата хоста в~распределенной ин\-фор\-ма\-ци\-он\-но-вы\-чис\-ли\-тель\-ной сис\-те\-ме, 
защищенной с~помощью метаданных~// Информатика и~её применения, 2018. Т.~12. Вып.~4. 
С.~41--45.

 \end{thebibliography}

 }
 }

\end{multicols}

\vspace*{-3pt}

\hfill{\small\textit{Поступила в~редакцию 03.04.19}}

%\vspace*{8pt}

%\pagebreak

\newpage

\vspace*{-28pt}

%\hrule

%\vspace*{2pt}

%\hrule

%\vspace*{-2pt}

\def\tit{ARCHITECTURAL DECISIONS IN~THE~PROBLEM 
OF~IDENTIFICATION OF~FRAUD IN~THE~ANALYSIS 
OF~INFORMATION FLOWS IN~DIGITAL ECONOMY\\[-5pt]}


\def\titkol{Architectural decisions in~the~problem 
of~identification of~fraud in~the~analysis 
of~information flows in~digital economy}

\def\aut{A.\,A.~Grusho, M.\,I.~Zabezhailo, N.\,A.~Grusho, and~E.\,E.~Timonina}

\def\autkol{A.\,A.~Grusho, M.\,I.~Zabezhailo, N.\,A.~Grusho, and~E.\,E.~Timonina}

\titel{\tit}{\aut}{\autkol}{\titkol}

\vspace*{-13pt}


 \noindent
   Institute of Informatics Problems, Federal Research Center ``Computer Sciences and 
Control'' of the Russian Academy of Sciences; 44-2~Vavilov Str., Moscow 119133, 
Russian Federation

\def\leftfootline{\small{\textbf{\thepage}
\hfill INFORMATIKA I EE PRIMENENIYA~--- INFORMATICS AND
APPLICATIONS\ \ \ 2019\ \ \ volume~13\ \ \ issue\ 2}
}%
 \def\rightfootline{\small{INFORMATIKA I EE PRIMENENIYA~---
INFORMATICS AND APPLICATIONS\ \ \ 2019\ \ \ volume~13\ \ \ issue\ 2
\hfill \textbf{\thepage}}}

\vspace*{3pt}


   
     
   \Abste{An approach to a~research of some types of fraud in digital economy with the usage of relationships of 
cause and effect is formulated. In all types of the considered frauds, the discrepancy between the 
purposes of financial transactions and actual cost of achievement of these purposes
has to be observed. Data on 
transactions can be collected by observing information flows in which these transactions are reflected. 
The architecture of data collection and their analysis can be organized by means of the distributed 
ledgers with the centralized consensus that allows creating an analog of the electronic account book 
fixing financial and economic activity of subjects of digital economy in the region. 
   The methods of fraud identification considered are based on the contradictions 
between actions described in transactions and information, which is contained in plans, standards, 
precedents, etc. 
   The method based on a~simplified scheme of implementation of the abstract project is considered. 
For identification of contradictions, it is necessary to carry out the analysis from the effect to the cause, 
i.\,e., to look for anomalies in information describing the generation of the observed effects. 
   It is shown how in implementation of the project it is possible to allocate simple ``necessary 
conditions'' of violation of cause and effect relationships, i.\,e., a~set of ``necessary conditions'' 
violation of which demonstrates fraud existence. It is possible to call this set of "necessary conditions" 
by metadata for control of the project for fraud identification.} 
   
   \KWE{digital economy; information flows; relationships of reason and effect; detection of 
fraudulent schemes}
   
  

 \DOI{10.14357/19922264190204}

\vspace*{-20pt}

 \Ack
   \noindent
   The work was partially supported by the Russian Foundation for Basic Research (projects  
18-29-03081 and 18-07-00274).



%\vspace*{6pt}

  \begin{multicols}{2}

\renewcommand{\bibname}{\protect\rmfamily References}
%\renewcommand{\bibname}{\large\protect\rm References}

{\small\frenchspacing
 {\baselineskip=10.5pt
 \addcontentsline{toc}{section}{References}
 \begin{thebibliography}{9}
\bibitem{1-gr-1}
\Aue{Grusho, A.\,A., A.\,A.~Zatsarinny, and E.\,E.~Timonina.} 2019. Blokcheyny tsifrovoy ekonomiki 
na baze sistemy situatsionnykh tsentrov i~tsentralizovannogo konsensusa [Blockchains of digital 
economy on the basis of the system of the situational centres and the centralized consensus]. 
\textit{25th Scientific and Technical Conference (International) ``Radar-Location, Navigation, 
Communication'' Proceedings}. Voronezh: VSU Publs. 6:183--191.
\bibitem{2-gr-1}
\Aue{Grusho, A., A.~Zatsarinny, and E.~Timonina.} 2019 (in press). 
A~system approach to information security 
in distributed ledgers on the situational centers platform. 
Lecture notes in computer science ser. Springer.
\bibitem{3-gr-1}
\Aue{Finn, V.\,K.} 2011. \textit{Iskusstvennyy intellekt: Metodologiya, primeneniya, filosofiya} 
[Artificial intelligence: Methodology, applications, philosophy]. Moscow: KRASAND. 448~p.

\bibitem{5-gr-1}
\Aue{Anshakov, O.\,M., and E.\,F.~Fabrikantova}. 2009. \textit{DSM-metod avtomaticheskogo porozhdeniya gipotez: Logicheskie 
i~epistemologicheskie osnovaniya} [JSM-method of automatic hypothesis generation: Logical and 
epistemological]. Moscow: KD LIBROKOM. 432~p.
\bibitem{4-gr-1} %5
\Aue{Poelmans, J., P.~Elzinga, S.~Viaene, and G.~Dedene.} 2010. Formal concept analysis in 
knowledge discovery: A~survey. \textit{Conceptual structures: From information to intelligence}. 
Eds.\ M.~Croitoru, S.~Ferr$\acute{\mbox{e}}$, and D.~Lukose. Lecture notes in 
computer science ser. Berlin--Heidelberg: Springer. 6208:139--153.

\bibitem{6-gr-1}
\Aue{Pankratov, E.\,S., and V.\,K.~Finn}. 
2009. \textit{Avtomaticheskoe porozhdenie gipotez v~intellektual'nykh 
sistemakh} [Automatic hypotheses generation in intelligent systems]. Moscow: KD 
\mbox{LIBROKOM}.  528~p. 
\bibitem{7-gr-1}
\Aue{Denisov, A.\,A., and D.\,N.~Kolesnikov.} 1982. \textit{Teoriya bol'shikh 
sistem upravleniya} [Theory of big control systems]. Leningrad: Energoizdat. 488~p.

\bibitem{9-gr-1}
\Aue{Grusho, A.\,A., N.\,A.~Grusho, M.\,I.~Zabezhailo, D.\,V.~Smirnov, and 
E.\,E.~Timonina.} 2018. 
Parametrizatsiya v~prikladnykh zadachakh poiska empiricheskikh prichin 
[Parametrization in applied 
problems of search of the empirical reasons]. 
\textit{Informatika i~ee Primeneniya~--- 
Inform. Appl.} 12(3):62--66.

\bibitem{8-gr-1}
\Aue{Grusho, A.\,A., N.\,A.~Grusho, M.\,V.~Levykin, and E.\,E.~Timonina.} 2018. Metody 
identifikatsii zakhvata khosta v~raspredelennoy informatsionno-vychislitel'noy sisteme, 
zashchishchennoy s~pomoshch'yu metadannykh [Methods of identification of host capture 
in the  distributed information system which is protected on the base of meta data].
\textit{Informatika i~ee 
Primeneniya~--- Inform. Appl.} 12(4):41--45.
{ %\looseness=1

}

\end{thebibliography}

 }
 }

\end{multicols}

\vspace*{-12pt}

\hfill{\small\textit{Received April 3, 2019}}

%\pagebreak

%\vspace*{-18pt}

\Contr

\noindent
\textbf{Grusho Alexander A.} (b.\ 1946)~--- Doctor of Science in physics and 
mathematics, professor, principal scientist, Institute of Informatics Problems, 
Federal Research Center ``Computer Sciences and Control'' of the Russian 
Academy of Sciences; 44-2~Vavilov Str., Moscow 119133, Russian Federation; 
\mbox{grusho@yandex.ru} 

\vspace*{3pt}

\noindent
\textbf{Zabezhailo Michael I.} (b.\ 1956)~--- Doctor of Science in physics and 
mathematics, principal scientist, Institute of Informatics Problems, Federal Research 
Center ``Computer Sciences and Control'' of the Russian Academy of Sciences;  
44-2~Vavilov Str., Moscow 119133, Russian Federation; 
\mbox{m.zabezhailo@yandex.ru} 

\vspace*{3pt}


\noindent
\textbf{Grusho Nikolai A.} (b.\ 1982)~--- Candidate of Science (PhD) in physics 
and mathematics, senior scientist, Institute of Informatics Problems, Federal 
Research Center ``Computer Sciences and Control'' of the Russian Academy of 
Sciences; 44-2~Vavilov Str., Moscow 119133, Russian Federation; 
\mbox{info@itake.ru} 

\vspace*{3pt}


\noindent
\textbf{Timonina Elena E.} (b.\ 1952)~--- Doctor of Science in technology, 
professor, leading scientist, Institute of Informatics Problems, Federal Research 
Center ``Computer Sciences and Control'' of the Russian Academy of Sciences;  
44-2~Vavilov Str., Moscow 119133, Russian Federation; 
\mbox{eltimon@yandex.ru} 

\label{end\stat}

\renewcommand{\bibname}{\protect\rm Литература}    %8
\def\stat{kovalenko}

\def\tit{ПРИМЕНЕНИЕ РАЗЛОЖЕНИЯ ИЗОБРАЖЕНИЯ С~ПОМОЩЬЮ~ДИСКРЕТНОГО ВЕЙВЛЕТ-ПРЕОБРАЗОВАНИЯ ДЛЯ~ПОСТРОЕНИЯ АРХИТЕКТУРЫ ШУМОПОДАВЛЯЮЩЕЙ НЕЙРОННОЙ СЕТИ}

\def\titkol{Применение разложения изображения с~помощью дискретного вейвлет-преобразования для построения архитектуры}
% шумоподавляющей нейронной сети}

\def\aut{А.~С.~Коваленко$^1$}

\def\autkol{А.~С.~Коваленко}

\titel{\tit}{\aut}{\autkol}{\titkol}

\index{Коваленко А.~С.}
\index{Kovalenko A.\,S.}


%{\renewcommand{\thefootnote}{\fnsymbol{footnote}} \footnotetext[1]
%{Исследование выполнено за счет гранта Российского научного фонда №\,22-28-00588, {\sf 
%https://rscf.ru/project/22-28-00588/}. Работа проводилась с~использованием инфраструктуры Центра 
%коллективного пользования <<Высокопроизводительные вычисления и~большие данные>> (ЦКП 
%<<Информатика>> ФИЦ ИУ РАН, Москва).}}


\renewcommand{\thefootnote}{\arabic{footnote}}
\footnotetext[1]{Южный федеральный университет, Институт математики, 
механики и~компьютерных наук им.~И.\,И.~Воровича, \mbox{akov@sfedu.ru}}

\vspace*{-12pt}






\Abst{Подавление шума на цифровых изображениях~--- одна 
из самых распространенных задач в~области обработки изображений. На данный 
момент широкое применение имеют подходы подавления шума, основанные на 
применении сверточных нейронных сетей (CNN, convolutional neural network). При этом, как правило, обучение модели 
строится на минимизации функции ошибки между результатом работы сети и~ожидаемым 
эталонным изображением и~дополнительно не используются различные представления 
двумерного сигнала изображения и~их свойства для оптимизации обучения
архитектур шумоподавляющих сетей. Предложен подход к~обуче\-нию 
нейронных сетей подавлять шум. Описанный подход основан на применении N-крат\-но\-го 
быст\-ро\-го вейв\-лет-пре\-обра\-зо\-ва\-ния Хаара (БВПХ). Такое представление дискретного сигнала 
изображения позволяет отказаться от классической архитектуры автоэнкодера 
и~использовать только его часть, кодирующую сигнал, что приводит к~значительному 
сокращению параметров модели и~ускоряет работу сети.}

\KW{нейронные сети; глубокое обучение; 
шумоподавление изображений; методы обработки изображений}

\DOI{10.14357/19922264240209}{UEQSXP}
  
%\vspace*{-6pt}


\vskip 10pt plus 9pt minus 6pt

\thispagestyle{headings}

\begin{multicols}{2}

\label{st\stat}

\section{Введение}



Наилучшие результаты решения задачи по\-дав\-ле\-ния шума на цифровых изображениях 
демонстрируют подходы, основанные на применении глубоких сверточных нейронных 
сетей. Как правило, данные сети имеют архитектуры, схожие с~мо\-делью 
U-Net~\cite{UNET_ORIGINAL}, где сеть представлена в~виде автокодировщика со 
сквозной передачей сигнала между слоями кодировщика входного сигнала и~его 
декодера. Общая схема модели U-Net изображена на рис.~\ref{fig:unet_scheme}. 
Авторы работы~\cite{UNETS_COMPARISON} рассматривают различные модификации 
архитектуры U-Net для подавления шума на входном изображении и~подходы 
к~обуче\-нию таких моделей.

\begin{figure*} %fig1
 \vspace*{1pt}
      \begin{center}
     \mbox{%
\epsfxsize=163mm 
\epsfbox{kov-1.eps}
}
\end{center}
\vspace*{-6pt}
    \Caption{Общая схема архитектуры U-Net: \textit{1}~--- понижение разрешения в~2~раза;
    \textit{2}~--- повышение разрешения в~2~раза;
    \textit{3}~--- конкатенация матриц по каналам; \textit{4}~--- передача выходных значений блока}
    \label{fig:unet_scheme}
\end{figure*}

Изображение, передаваемое в~нейронную сеть, можно рассматривать как сумму 
значений элементов матрицы чистого изображения~$I$ с~матрицей, содержащей шум, 
получаемый из некоторого распределения~$\mathit{P}$, и~может быть записано 
выражением:
\begin{equation*}
\label{eq:input_image_def}
    \tilde{I} = I + \alpha,\enskip \alpha \sim \mathit{P}\,.
\end{equation*}

Поскольку погрешность приема оптического\linebreak сигнала зависит от физических свойств 
КМОП-сен\-со\-ра (КМОП~--- комплементарная структура \mbox{ме\-талл}--ок\-сид--по\-лу\-про\-вод\-ник), 
то для каждой модели существует некоторое уникальное распределение~$\mathit{P}$, 
генерирующее шумовую составляющую сигнала. Также на уровень шума 
будут иметь значительное влияние настройки камеры и~условия 
съемки~\cite{IMAGE_NOISE_CONDITIONS}, при увеличении уровня 
светочувствительности сенсора будет возрастать и~отношение шума к~чистому 
сигналу.

Задача обучения нейронной сети~$f$ заключается в~поиске оптимального набора 
весов слоев сети~$\mathnormal{w}$, при котором будет достигнут минимум 
аппроксимированного эмпирического риска:

\vspace*{-2pt}

\noindent
\begin{equation*}
\label{eq:main_problem}
\tilde{Q}(w, X^{l}) = \sum\limits_{i=1}^{l}\mathcal{L}(I_i, 
f(\tilde{I}_i, w)) \rightarrow \min\limits_{w}\,.
\end{equation*}

\vspace*{-4pt}

В качестве функции ошибки для обучения модели могут быть выбраны меры схожести 
изображений~$L_{1}$ и~$L_{2}$, функция показателя индекса структурного сходства 
(SSIM, structure simularity)~\cite{SSIM} или комбинация нескольких мер сходства изображений, как 
предлагаемая авторами работы~\cite{MIX_LOSS} функция ошибки MIX, которая 
представляет собой взвешенную сумму нормы~$L_{1}$ и~многомасштабной SSIM (MS-SSIM, multiscale SSIM).

Для повышения качества работы U-Net-по\-доб\-ных моделей в~их архитектуру 
интегрируют слои межканального и~пространственного внимания~\cite{CBAMCB}. Эти 
дополнительные слои позволяют модели извлекать не только локальные признаки 
изоб\-ра\-же\-ния, но и~работать с~глобальными особенностями 
изображения~\cite{CBAM_USAGE}. Также для улучшения шумоподавляющих свойств 
обучаемой модели могут использоваться различные преобразования значений скрытого 
пространства слоев сети. Авторы архитектуры Multilevel Wavelet-CNN 
(MWCNN)~\cite{MWCNN} используют дискретное вейв\-лет-пре\-обра\-зо\-ва\-ние Хаара для 
разложения сигнала, передаваемое между слоями модели. Это позволяет улучшить 
качество восстановления высокочастотной компоненты входного изображения и~минимизировать эффект размытия обработанного сетью изображения.

Авторы работы~\cite{WaveletPooling} используют вейв\-лет-пре\-обра\-зо\-ва\-ния 
в~комбинации с~методом объединения признаков, извлеченных разными блоками модели 
для сохранения информации в~восстанавливаемом изоб\-ра\-же\-нии о~текстурах и~границах 
объектов. Повышение разрешения карт признаков путем применения вейв\-лет-пре\-обра\-зо\-ва\-ний также используют авторы работы~\cite{MDIWT}. Подход, схожий 
с~применением вейв\-лет-пре\-обра\-зо\-ва\-ния для обработки признаков входного сигнала, 
применяется в~работе~\cite{WINNet}.

Перечисленные выше подходы используют вейв\-лет-пре\-обра\-зо\-ва\-ния для улучшения 
извлечения скрытыми слоями сети более устойчивых признаков из изображения, но не 
используются в~построении функции ошибки. Если рассматривать\linebreak задачу повышения 
разрешения изображения, то существуют работы, где функция ошибки при обуче\-нии 
модели основывается на применении преобразовании Хаара. Так, в~работе~\cite{DWB} 
сеть\linebreak изучает \mbox{матрицы} коэффициентов, полученные преобразованием Хаара, которые 
необходимо добавить к~вейв\-лет-мат\-ри\-цам входного изображения для уточнения его 
границ. Данный подход уже использует обратное вейв\-лет-пре\-обра\-зо\-ва\-ние для 
получения итогового изображения.

Также существуют подходы, использующие специальные функции ошибок, которые 
учитывают значения скрытых пространств слоев нейронной сети. Для решения задачи 
удаления фона на изображении авторы работы~\cite{IS_NET} разработали собственную 
U-Net-подобную архитектуру~--- IS-Net, а также функцию ошибки для ее обуче\-ния, 
которая требует от глубоких слоев модели строить результирующую матрицу маски 
главного объекта в~разных масштабах, далее результаты работы всех слоев 
объединяются для построения финальной маски объекта.

На основе идей повышения разрешения изоб\-ра\-же\-ния с~помощью предсказываемых матриц 
высокочастотных коэффициентов~\cite{DWB} и~по\-стро\-ения функции ошибки, 
учи\-ты\-ва\-ющей скрытое со\-сто\-яние слоев модели~\cite{IS_NET}, предлагается подход 
к~обуче\-нию архитектур, подобных ResNet~\cite{ResNet}, без не\-об\-хо\-ди\-мости до\-бав\-ле\-ния 
обучаемого декодера для преобразования скрытого пространства модели в~изоб\-ра\-же\-ние.


\section{Предлагаемый подход}

\subsection{Модификация архитектуры классификации}

Классические архитектуры для решения задачи классификации изображений, как 
правило, состоят из блоков, содержащих сверточные слои, слои нормализации 
и~операций объединения~\cite{POOLOPS}. Операции объединения максимумов (Max 
Pooling) вычисляют максимальное значение для предсказанных матриц признаков 
предыдущими сверточными слоями и~используют их для создания матриц признаков 
с~пониженной дискретизацией. Эти операции используются в~архитектуре 
ResNet~\cite{ResNet} и~позволяют извлекать наиболее устойчивые признаки из 
предыдущих предсказаний при уменьшении их размера в~2~раза. Последними слоями 
в~архитектурах для решения задач классификации служат полносвязные слои, которые 
строят распределение вероятностей классов, содержащихся на входном изображении. 
Для построения шумоподавляющей архитектуры сети, рассматриваемой в~данной 
работе, слои классификации не используются и~удаляются из применяемых моделей 
классификации. Удаление последних слоев из модели ResNet делает ее 
полносверточной архитектурой~\cite{FULLYCONV}, и~она становится инвариантной к~размеру входного изоб\-ра\-же\-ния.

Из выбранной архитектуры для задачи классификации рассматриваются промежуточные 
мат\-ри\-цы признаков, полученные после применения каж\-до\-го из сверточных блоков. 
Обозначим набор данных значений $P \hm= \{p_{i}\}_{i=1}^{5}$. К~каждой матрице 
признаков $p_{i}$ применим сверточный слой $\mathrm{Conv}_{i}$ с~размером ядра $\mathrm{dim}\,(K_i) 
\hm= 9 \times C_{i} \times 3 \times 3$ при $i \hm\leq 2$ или $\mathrm{dim}\,(K_i) \hm= 9 \times 
C_{i} \times 1 \times 1$ для остальных уровней блоков, где $C_{i}$~--- число 
каналов у~выходной матрицы соответствующего блока модели. Выбор разного размера 
ядер сверточных слоев обуслов\-лен различной размерностью мат\-риц признаков. 
У~мат\-риц с~более глубоких уровней размерность ниже. После применения сверточных 
слоев к~набору значений~$P$ получается новый набор $F \hm= \{f_{i}\}_{i=1}^5$, где 
$f_i\hm = \mathrm{Conv}_i(p_i)$. Дополнительно к~сверточным слоям $\mathrm{Conv}_{i}$ был добавлен 
механизм канального и~пространственного внимания CBAM (Convolutional Block Attention Module)~\cite{CBAMCB}.  
Модифицированную архитектуру с~параметрами внутренних слоев $\mathnormal{w}$ 
обозначим $\Phi(\tilde{I}, {w})$.

Схема предлагаемой архитектуры нейронной сети~$\Phi$ для предсказания чистого 
изображения~$I$ по входному изображению с~шумом~$\tilde{I}$ изображена на 
рис.~\ref{fig:wpn_scheme}.



\subsection{Вейвлет-преобразование Хаара}

Вейвлет-преобразование Хаара (ВПХ) позволяет декомпозировать сигнал на две 
компоненты: аппроксимацию сигнала и~детализацию сигнала~\cite{WAVELETS_PAPER}. 
Для получения следующего уровня разложения ВПХ применяется к~полученному 
аппроксимационному сигналу. При условии, что первоначальный дискретный сигнал 
представлен массивом их $2^n$~чисел, можно рекурсивно применить дискретное ВПХ~$n$~раз 
к~данному сигналу, получая $n$-крат\-ное применение дискретного вейв\-лет-пре\-обра\-зо\-ва\-ние Хаара.

При рассмотрении одномерного сигнала в~виде набора значений $F \hm= \{ f_i \}_{i = 1}^{N}$ коэффициенты Хаара для аппроксимации~$a_{i}$ и~детализации~$d_{i}$ 
вычисляются по формулам:
\begin{equation*}
\label{eq:raw_haar}
a_{i} = \fr{f_{2i} + f_{2i + 1}}{\sqrt{2}}\,;\enskip d_{i} = \fr{f_{2i} - f_{2i + 
1}}{\sqrt{2}}\,,\enskip  i \in \left[1, \fr{N}{2}\right].
\end{equation*}

В случае использования БВПХ формулы для 
вычисления коэффициентов будут иметь сле\-ду\-ющий вид:
\begin{equation*}
\label{eq:fast_haar}
a_{i} = \fr{f_{2i} + f_{2i + 1}}{2}\,;\ d_{i} = \fr{f_{2i} - f_{2i + 1}}{2}\,,\ 
i \in \left[1, \fr{N}{2}\right].
\end{equation*}
%
Данная модификация преобразования является вычислительно более быстрым по 
сравнению с~оригинальным.

Обратное БВПХ для получения первоначальных значений сигнала из коэффициентов 
БВПХ можно получить по формуле
\begin{equation*}
\label{eq:inv_fast_haar}
f_{i} = a_{\lfloor {i}/{2}\rfloor} + (-1)^{i - 1 \bmod{2}} 
d_{\lfloor {i}/{2}\rfloor}.
\end{equation*}

Так как в~данной работе происходит обработка\linebreak цифровых изображений, то необходимо 
использовать БВПХ для дискретного двумерного сигнала. В~случае двумерного 
сигнала БВПХ \mbox{сначала} применяется к~строкам матрицы изображения~$I$. Обозначим 
матрицы получаемых коэффициентов~$a_j^{\mathrm{row}_i}$ и~$d_{j}^{\mathrm{row}_i}$, а~далее 
повторно применим БВПХ к~столбцам матриц, полученных после первого разложения. 
В~результате данных преобразований будут получены матрицы коэффициентов
$A \hm= \{a_{i,j}\}_{i=1, j=1}^{H, W}$, $H \hm= \{h_{i,j}\}_{i=1, j=1}^{H, W}$,
$V \hm= \{v_{i,j}\}_{i=1, j=1}^{H, W}$, $D \hm= \{d_{i,j}\}_{i=1, j=1}^{H, W}$. 
Матрица~$A$ содержит ин\-фор\-мацию об аппроксимации двумерного сигнала,\linebreak мат\-ри\-цы 
$H$, $V$ и~$D$~--- о~горизонтальных, вертикальных и~диагональных различиях 
значений со\-от\-вет\-ст\-венно.

\subsection{Применение вейвлет-преобразования Хаара для~обучения сети}

При применении $n$-кратного ВПХ к~квадратному изображению размером $N \times N$ 
матрицы коэффициентов будут иметь размер ${N}/2^{n} \times 
{N}/{2^{n}}$, \mbox{поскольку} с~каждым применением ВПХ к~изоб\-ра\-же\-нию или 
последующей аппроксимационной мат\-ри\-це размер будет уменьшаться ровно в~2~раза.\linebreak\vspace*{-12pt}

\pagebreak

\end{multicols}

\begin{figure*} %fig2
\vspace*{1pt}
      \begin{center}
     \mbox{%
\epsfxsize=142.115mm 
\epsfbox{kov-2.eps}
}
\end{center}
\vspace*{-3pt}
    \Caption{Схема предлагаемой архитектуры модели по\-дав\-ле\-ния шума на 
изоб\-ра\-же\-нии: \textit{1}~--- мат\-ри\-цы коэффициентов $n$-кратного ВПХ;
\textit{2}~--- передача выходных значений блока; \textit{3}~--- конкатенация мат\-риц по каналам}
    \label{fig:wpn_scheme}
    \vspace*{3pt}
\end{figure*}

\begin{multicols}{2}

\noindent 
Применяя ВПХ к~многоканальному изображению, коэффициенты надо рассчитывать 
для каждого канала отдельно. Если объединить разложения для всех каналов 
изображения, получится матрица размера $12 \times {N}/{2} \times 
{N}/{2}$, содержащая в~себе конкатенацию матриц~$A$, $H$, $V$ и~$D$, при 
условии, что используемое изображение~$I$ имеет три канала и~размер $N \times 
N$, где~$N$ представляет собой некоторую степень~2, причем эта степень $q \hm\ge 5$.

Во время обучения модель учится предсказывать изоб\-ра\-же\-ние без шума~$I$ по 
входной матрице изоб\-ра\-же\-ния~$\tilde{I}$. Для сокращения параметров сети 
предлагается учить слои сети предсказывать матрицы~$f_i$, равные объединениям 
матриц коэффициентов деталей $(H^i, V^i, D^i)$ $i$-крат\-но\-го БВПХ, примененного к~$I$. 
Данная оптимизация позволяет отказаться от построения слоев декодировки 
результирующего изображения из скрытого состояния сети.

Для построения итогового изображения по высокочастотным коэффициентам необходимо 
применять обратное БВПХ к~соответствующим матрицам коэффициентов. Обозначив 
прямое БВПХ как~DWT, а~обратное~--- IWT, формулы для расчета матриц 
коэффициентов можем записать в~виде

\noindent
\begin{equation*}
A^i, H^i, V^i, D^i = \mathrm{DWT}\left(A^{i-1}\right),\enskip A^0 = I\,,
\end{equation*}
\begin{equation*}
A^i =\mathrm{IWT}\left(A^{i-1}, H^{i-1}, V^{i-1}, D^{i-1}\right).
\end{equation*}

Таким образом, нейронная сеть $\Phi$ будет учиться предсказывать наборы 
коэффициентов БВПХ $\{H^i, V^i, D^i\}_{i=1}^5$, но для восстановления 
изображения $I$ необходима аппроксимация изображения \mbox{5-крат}\-но\-го применения ВПХ~--- 
$A^5$, поскольку она необходима для вычисления~$A^4$. Для получения $A^5$ предлагается применить $5$-кратное БВПХ ко входному изображению с~шумом~$\tilde{I}$ 
и~использовать его коэффициенты приближения изображения~$\tilde{A}^5$. Матрица~$\tilde{A}^5$ будет близка к~матрице приближения чистого 
изображения~$A^5$, поскольку с~каждым применением БВПХ уровень шума в~каждом 
последующем приближении~$\tilde{A}^i$ снижается~\cite{WAVELETS_APPROX}, при этом 
информация о шуме~$\alpha$ будет содержаться в~матрицах $\tilde{H}^i$, 
$\tilde{V}^i$ и~$\tilde{D}^i$~\cite{WAVELETS_NOISE}.

Для сокращения вычислений при расчете мат\-ри\-цы~$\tilde{A}^5$ необходимо применять 
БВПХ только к~коэффициентам приближения~$\{\tilde{A}^q\}_{q=1}^4$. Так, 
коэффициенты матрицы~$\tilde{A}^q$ будут вычисляться по формуле:
\begin{multline*}
\tilde{A}_{i,j}^q = \fr{\tilde{A}_{2i, 2j}^{q-1} + \tilde{A}_{2i, 2j+1}^{q-1} + 
\tilde{A}_{2i+1, 2j}^{q-1} + \tilde{A}_{2i+1, 2j+1}^{q-1}}{4},\\
 i,j \in \left[1, \fr{N}{2^q}\right].
\end{multline*}

Если выразить $\tilde{A}^q$ через значения пикселей изоб\-ра\-же\-ния~$\tilde{I}$, 
формула примет вид:
\begin{equation}
\label{eq:fulltai}
\tilde{A}_{i,j}^q = \sum\limits_{m=1}^{2q}\sum\limits_{s=1}^{2q}\fr{\tilde{I}_{2qi+m, 
2qj+s}}{4q^2},\enskip i,j \in \left[1, \fr{N}{2^q}\right].
\end{equation}

Выражение~(\ref{eq:fulltai}) совпадает с~формулой вычисления пикселя при 
масштабировании изображения с~коэффициентом ${1}/{2^q}$ с~по\-мощью метода 
передискретизации с~использованием отношения площади пикселя. Данный метод 
интерполяции изображения называется {Area} в~терминологии библиотеки 
обработки изображений OpenCV~\cite{OPENCV_LIB}. Матрицу~$\tilde{A}^5$ можно 
напрямую вычислить, применяя данный метод интерполяции к~изображению~$\tilde{I}$.


\subsection{Сравнение с~существующими подходами к~внедрению вейвлет-преобразований в~архитектуры шумоподавляющих моделей}

Архитектура MWCNN, разработанная авторами 
работы~\cite{MWCNN}, основана на применении вейв\-лет-пре\-обра\-зо\-ва\-ний для понижения 
размерности карт признаков при увеличении числа каналов. Это позволило 
отказаться от операций пулинга (Pooling) при проектировании модели. Модель 
построена\linebreak на последовательном применении ВПХ и~сверточных слоев к~коэффициентам 
ВПХ. В~качестве операций обратного пулинга или повышения раз\-мер\-ности карт 
признаков авторами применяется \mbox{обратное} ВПХ. Также в~модели применяется сквозная 
передача признаков из кодирующей части модели в~декодирующую, что делает ее 
схожей в~общем виде с~классической архитектурой U-Net.

Авторы работы~\cite{MDIWT} применяют вейв\-лет-пре\-обра\-зо\-ва\-ния для выделения 
высокочастотной пространственной информации сигнала. К~выделенной 
высокочастотной компоненте применяется блок сверточных слоев для улучшения 
вы\-со\-ко\-час\-тот\-ных признаков сигнала.

На основе идеи разложения сигнала на компоненты с~помощью вейв\-лет-пре\-обра\-зо\-ва\-ний 
основан подход, описанный в~работе~\cite{WINNet}. Авторы \mbox{применяют} 
преобразование с~обучаемым ядром, которое позволяет в~процессе обучения модели 
эффективно разделять сигнал на две компоненты. Данные преобразования по аналогии с~работой~\cite{MDIWT} применяются вместе со сверточными слоями для извлечения 
признаков из сигнала.

В отличие от подходов~\cite{MWCNN, MDIWT, WINNet} пред\-ла\-га\-емая архитектура не 
использует ВПХ для разложения сигнала изображения на компоненты с~\mbox{целью} 
последующей обработки сверточными слоями. Обратное ВПХ используется для 
повышения разрешения изображения по предсказанным коэффициентам ВПХ аналогично 
подходу~\cite{DWB}. Но используется не однократное ВПХ, а~$5$-крат\-ное, что 
позволяет преобразовать изображение размера $32 \times 32$ в~$256 \times 256$. 
Таким образом, по входному зашумленному изображению модель предсказывает 
коэффициенты для каждого уровня ВПХ, позволяя по ним восстанавливать изображение 
исходного размера без шума. Использование описанного способа получения 
результирующего изображения заменяет применение декодирующей части U-Net 
подобных архитектур~\cite{MWCNN, MDIWT}.

\section{Эксперименты по обучению моделей}

\subsection{Обучающий набор данных}

Обучающая выборка состояла из объединения нескольких наборов данных. Для 
обучения на изоб\-ра\-же\-ни\-ях с~шумом, полученных с~реальных сенсоров камер, 
использовался открытый набор данных Smartphone Image Denoising Dataset 
(SIDD)~\cite{SIDD_2018_CVPR}. Набор SIDD предоставляет реальные зашумленные 
изображения и~соответствующие им чистые изображения. Обучающая часть набора 
содержит 320~изображений высокого разрешения, а~проверочная часть содержит 
1280~пар изображений, имеющих размер $256\times 256$~точек. Съемка проводилась авторами на 
5 мобильных устройств с~КМОП-сен\-со\-рами.

Также для обучения использовались изображения из наборов Set5, Set14, Sun-Hays 
80~\cite{FOR_ADD_NOISE_DATASETS} и~DIV2K~\cite{DIV2KDataset}. Данные наборы 
содержат изображения в~двух масштабах и,~как правило, используются для обуче\-ния 
моделей, повышающих разрешение изображения. Так как изображения в~наборах не 
содержат шума~\cite{FOR_ADD_NOISE_DATASETS}, они могут использоваться в~задаче 
обучения шумоподавляющих моделей. Авторы работ~\cite{MWCNN, MDIWT} 
используют перечисленные наборы для обучения моделей подавлять шум на 
изображениях. Для получения изображений с~шумом авторы до\-бав\-ля\-ют к~мат\-ри\-цам 
дополнительный гауссовский шум. Таким образом получаются пары изображений для 
обучения шумоподавляющих нейронных сетей.

В проводимых экспериментах к~чистым изоб\-ра\-же\-ни\-ям из 
наборов~\cite{FOR_ADD_NOISE_DATASETS, DIV2KDataset} попиксельно до\-бав\-лял\-ся 
дополнительный гауссовский шум с~фиксированным параметром математического 
ожидания, равным нулю, и~изменяемым значением среднеквадратичного отклонения 
$\sigma$. При каждой загрузке изображения параметр~$\sigma$ выбирался случайным 
образом из равномерного распределения $R(\sigma|a, b)$ с~диапазоном $a \hm= 0$, $b \hm= 
90$. Шум для каждого пикселя изображения семплировался независимо от остальных 
пикселей. Получение изображения с~шумом приводится в~формуле:
\begin{multline*}
\tilde{I}_{i,j} = I_{i,j} + \alpha_{i,j},\enskip \alpha_{i,j} \sim \mathit{N}(0, \sigma|\mathit{R}(0, 90)),\\ 
i \in [1, H],\ j \in [1, W].
\end{multline*}

Для тестирования работы обученных моделей использовалась валидационная часть 
набора данных SIDD, а~также набор BSD68~\cite{BSD_set} с~заранее наложенным 
шумом для корректного сравнения с~результатами работ~\cite{MWCNN, MDIWT, WINNet}. Дополнительно модель тестировалась на изображениях из набора The 
Darmstadt Noise Dataset (DND)~\cite{DNDSet}.


\subsection{Исследуемые архитектуры}

Предлагаемая архитектура, использующая предсказания коэффициентов матриц ВПХ, 
в~приведенных исследованиях обозначается как Wavelets Prediction Network~(WPNet), 
а~указанная в~скобках архитектура используется в~качестве базовой мо\-дели.

Для сравнения результатов дополнительно строился декодер, использующийся 
в~классической архитектуре U-Net~\cite{UNET_ORIGINAL}. В~оригинальной реализации 
декодера использовались два варианта слоев для повышения разрешения: слои 
деконволюции и~операции билинейной интерполяции с~последующим применением 
сверточных слоев. В~экспериментах обучались оба варианта декодеров. В~качестве 
кодировщика для модели U-Net была выбрана архитектура ResNet10.

Для обучения моделей U-Net использовалась функция ошибки 
$\mathcal{L}_{\mathrm{MIX}}$~\cite{MIX_LOSS}.

\subsection{Параметры обучения}

Архитектура модели и~код обучения реализованы на фреймворке глубокого обучения 
PyTorch~\cite{PYTORCH_LIB}. Реализации архитектур моделей классификации для 
интеграции в~предлагаемый подход использовались из библиотеки 
timm~\cite{TIMMLIB}.

Модели обучались на случайных срезах из изоб\-ра\-же\-ний размером $256 \times 256$ 
пикселей. Вырезанные части изображений конвертировались в~цветовое пространство 
YC$_{\mathrm{r}}$C$_{\mathrm{b}}$, где наибольший вклад в~детализацию изображения вкладывает компонента 
изображения~Y, по которой происходит оценка качества работы моделей 
в~работе~\cite{UFORMER}. Для обучения параметров модели ${w}$ 
применялся метод стохастической оптимизации с~адаптивным параметром скорости 
обучения AdaSmooth~\cite{ADASMOOTH}. Начальное значение параметра ско\-рости 
обучения задавалось равным~0,001. В~качестве функции ошибки использовалась 
$\mathcal{L}_{\mathrm{MIX}}$~\cite{MIX_LOSS}.

Модель запускалась на входном изображении и~на преобразованных изображениях. 
В~качестве преобразований использовались повороты на~90$^\circ$, 180$^\circ$ и~270$^\circ$, 
а~также отражения изображения по вертикали и~горизонтали. После применения модели к~этим изображениям использовались соответствующие обратные преобразования. 
Итоговое изображение получалось после усреднения результатов работы модели на 
преобразованных матрицах.

Для валидации использовалась метрика пикового отношения сигнала к~шуму PSNR (peak signal-to-noise ratio). 
Обучение модели останавливалось при выходе графика результата валидационной 
метрики на плато.

\begin{figure*}[b] %fig3
\vspace*{1pt}
      \begin{center}
     \mbox{%
\epsfxsize=134.1mm 
\epsfbox{kov-3.eps}
}
\end{center}
\vspace*{-9pt}
    \Caption{Пример предсказанных моделью коэффициентов $4$-кратного БВПХ}
    \label{fig:wavelets_prediction}
\end{figure*}

Эксперименты проводились на вычислительной машине с~графическим ускорителем 
Nvidia RTX~4090, процессором Intel i9-10920X и~объемом оперативной памяти 64~ГБ. 
При размере входных изображений $256 \times 256$ в~процессе обучения 
использовались пакеты размером~128 (batch size).

\section{Результаты}

Предлагаемая архитектура, использующая предсказания коэффициентов матриц ВПХ, 
в~приведенных результатах обозначается как WPNet, 
а~указанная в~скобках архитектура используется в~качестве базовой модели.

Для сравнения результатов использовались мет\-ри\-ки пикового отношения сигнала к~шуму (PSNR) и~структурного сходства изображений (SSIM).

Оценка качества подавления шума оценивалась как на реальных изображениях из 
валидационной части набора SIDD~\cite{SIDD_2018_CVPR}, содержащих шум, так и~на 
изображениях из набора BSD68~\cite{BSD_set} с~добавочным гауссовским шумом.

При оценке отклонения предсказываемых моделью коэффициентов ВПХ от эталонных 
значений использовалась мет\-ри\-ка Smooth $L_1$~\cite{SMOOTHL1}. Для оценки 
строились сред\-ние значения и~сред\-не\-квад\-ра\-тич\-ные отклонения по множеству всех 
рас\-смат\-ри\-ва\-емых изображений из набора. Данные, по\-стро\-ен\-ные на валидационной 
части набора SIDD приведены в~табл.~1, все значения 
умножены на~$10^4$. В~табл.~1 не приведены 
коэффициенты аппроксимации~$A$,\linebreak\vspace*{-12pt}

%\begin{table*}\small %tabl1

\vspace*{-3pt}

\begin{center}
\parbox{79mm}{{{\tablename~1}\ \ \small{Сравнение отклонения коэффициентов БВПХ модели WPNet
}}

}

\vspace*{2ex}


{\small \tabcolsep=4pt
\begin{tabular}{ |l r|c|c|c|c|c| }
\hline
 \multicolumn{2}{|c|}{Коэффициенты} & 
\multicolumn{5}{c|}{Уровень модели} \\ 
\cline{3-7}
 \multicolumn{2}{|c|}{БВПХ}  & 1 & 2 & 3 & 4 & 5 \\ 
 \hline
$H$ & $\mathbb{E}$ & 2,105 & 1,132 & 0,589 & 0,176 & 0,0424 \\
 & STD & 2,175 & 1,395 & 0,583 & 0,175 & 0,0426 \\ 
 \hline
 $V$ & $\mathbb{E}$ & 2,108 & 1,393 & 0,586 & 0,174 & 0,0417 \\
 & STD & 2,177 & 1,394 & 0,580 & 0,172 & 0,0421 \\ 
 \hline
$D$ & $\mathbb{E}$ & 1,131 & 0,939 & 0,474 & 0,165 & 0,0446 \\
 & STD & 1,183 & 0,955 & 0,471 & 0,164 & 0,0446 \\ 
 \hline
\end{tabular}
}
\end{center}
%\end{table*}

\noindent
 поскольку они строятся по входному изображению 
в~соответствии с~формулой~(\ref{eq:fulltai}). Коэффициенты на первых уровнях 
модели имеют большее отклонение, чем выходы более глубоких слоев, поскольку 
результат более глубоких уровней модели получен применением большего числа 
слоев. Но в~предлагаемой архитектуре результат напрямую строится из всех уровней 
коэффициентов ВПХ. И~коэффициенты первых применений ВПХ меньше влияют на 
предсказываемое изображение, чем коэффициенты дальнейших применений ВПХ.



Пример предсказанных матриц с~коэффициентами $n$-крат\-но\-го БВПХ приведен на 
рис.~\ref{fig:wavelets_prediction}.

\pagebreak

\end{multicols}


\setcounter{table}{1}
\begin{table*}\small %tabl2
\begin{center}
\parbox{400pt}{\Caption{Сравнение качества и~размера предлагаемой архитектуры с~другими 
моделями на наборе данных BSD68}
}

\label{tab:comparison_BSD68}
\vspace*{2ex}

\begin{tabular}{|l|l|c|l|c|l|c|c|}  %\multicolumn{1}{|c|}{\raisebox{-6pt}[0pt][0pt]{
 \hline
\multicolumn{1}{|c|}{\raisebox{-6pt}[0pt][0pt]{Название модели}} 
& \multicolumn{2}{c|}{$\sigma = 15$} & 
\multicolumn{2}{c|}{$\sigma = 25$} & \multicolumn{2}{c|}{$\sigma = 50$} &
\tabcolsep=0pt\begin{tabular}{c} 
Число\\ параметров \end{tabular} \\ 
\cline{2-7}
 & PSNR & SSIM & PSNR & SSIM & PSNR & SSIM & модели $\times 10^{6}$\\
 \hline
 BM3D~\cite{BM3D} & 31,08 & 0,872 & 28,57 & 0,802 & 25,62 & 0,687 & --- \\
 % \hline
 DnCNN~\cite{DnCNN} & 31,73 & 0,891 & 29,23 & 0,828 & 26,23 & 0,719 & 0,56 \\ 
%\hline
 MWCNN~\cite{MWCNN} & 31,86 & 0,895 & 29,41 & 0,836 & 26,53 & 0,737 & 24,92\hphantom{9} \\ 
%\hline
 MWDCNN~\cite{MDIWT} & 31,77 & --- & 29,28 & --- & 26,29 & --- & 5,24 \\ 
 %\hline
 WINNet~\cite{WINNet} & 31,7 & --- & 29,24 & --- & 26,31 & --- & 0,17 \\ 
% \hline
 U-Net(ResNet10) & 31,26 & 0,898 & 30,29 & 0,872 & 29,06 & 0,83\hphantom{9} & 6,47 \\ 
% \hline
 U-Net(ResNet10) + Bilinear & 32,16 & 0,906 & 30,79 & 0,874 & 29,3 & 0,827 & 7,28 \\ 
% \hline
 \textbf{WPNet(MobileNetV2)} & 31,65 & 0,894 & 30,2 & 0,861 & 28,6 & 0,81\hphantom{9} & 1,83 \\ 
 %\hline
 \textbf{WPNet(ResNet10)} & 31,5 & 0,901 & 30,33 & 0,87\hphantom{9} & 28,87 & 0,824 & 4,99 \\ 
 \hline
\end{tabular}
\end{center}
%\end{table*}
%\begin{table*}\small %tabl3
\begin{center}
\parbox{330pt}{\Caption{Сравнение качества и~размера предлагаемой архитектуры с~другими 
моделями на валидационном наборе SIDD}
}

\label{tab:comparison_sidd}
\vspace*{2ex}

\begin{tabular}{|l|l|l|l|l|c| }
 \hline
\multicolumn{1}{|c|}{\raisebox{-6pt}[0pt][0pt]{Название модели}} & \multicolumn{2}{c|}{SIDD} & 
\multicolumn{2}{c|}{DND} & \tabcolsep=0pt\begin{tabular}{c} 
Число\\ параметров \end{tabular}\\ 
\cline{2-5}
 & {PSNR} & {SSIM} & {PSNR} & {SSIM} & модели $\times 10^{6}$\\
\hline
 BM3D~\cite{BM3D} & 25,65 & 0,685 & 34,51 & 0,851 & --- \\
 % \hline
 DnCNN~\cite{DnCNN} & 35,13 & 0,896 & 37,03 & 0,932 & 0,56 \\ 
 %\hline
 U-Former~\cite{UFORMER} & 39,89 & 0,96 & 39,98 & 0,955 &  26,87\hphantom{9}  \\ 
 %\hline
 U-Net(ResNet10) & 37,65 & 0,897 & 37,62 & 0,947 & 6,47 \\ 
 %\hline
 U-Net(ResNet10)\;+\;Bilinear & 37,79 & 0,891 & 38,03 & 0,946 & 7,28 \\ 
 %\hline
 \textbf{WPNet(MobileNetV2)} & 36,77 & 0,913 & 36,14 & 0,937 & 1,83 \\ 
 %\hline
 \textbf{WPNet(ResNet10)} & 37,28 & 0,927 & 36,6 & 0,944 & 4,99 \\
 \hline
\end{tabular}
\end{center}
\end{table*}

\begin{multicols}{2}

На наборе BSD68 с~добавочным шумом предлагаемая модель сравнивалась 
с~результатами широко распространенных работ BM3D~\cite{BM3D}, DnCNN~\cite{DnCNN}, 
а также с~подходами, использующими вейв\-лет-пре\-обра\-зо\-ва\-ния~\cite{MWCNN, MDIWT, 
WINNet}. Результаты сравнения на изображениях с~разным параметром $\sigma$ 
добавочного шума приведены в~табл.~2, где шум 
генерировался из нормального распределения $\mathit{N}(0, \sigma)$. Для подхода 
BM3D не указывается число параметров, поскольку он не основан на использовании 
обучаемых нейронных сетей.



Для сравнения на наборе SIDD использовались подходы~\cite{BM3D, DnCNN} 
и~современная архитектура, основанная на механизме самовнимания U-
Former~\cite{UFORMER}. Авторы работ~\cite{MWCNN, MDIWT, WINNet} не проводили 
исследований по обучению и~тестированию
моделей на наборе SIDD. Результаты сравнения качества подавления шума на 
изображениях из набора SIDD приведены в~табл.~3.



Также оценивалась производительность предлагаемой архитектуры в~сравнении 
с~другими архитектурами, использующими ВПХ, и~с~U-Net вариантами модели. Замеры 
времени работы моделей\linebreak производились на процессоре Intel i9-10920X. Каж\-дая 
модель запускалась 100~раз на изображении раз-\linebreak\vspace*{-12pt}



%\begin{table*}\small %tabl4
\begin{center}
\parbox{58mm}{{{\tablename~4}\ \ \small{Сравнение скорости работы предлагаемой архитектуры с~другими моделями
}}

}

\label{tab:time_comparison}
\vspace*{2ex}

{\small 
\tabcolsep=12pt
\begin{tabular}{ |l|c|}
\hline
\multicolumn{1}{|c|}{Название модели} & Время, мс \\
\hline
{MWCNN~\cite{MWCNN}} & 109\\
{MWDCNN~\cite{MDIWT}} & 292\\
{WINNet~\cite{WINNet}} & 2297\hphantom{9}\\ 
U-Net (ResNet10) & 72\\
WPNet (ResNet10)& 56\\
\hline
\end{tabular}
}
\end{center}
%\end{table*}

\vspace*{6pt}



\noindent
мером $256 \times 256$, и~время 
запусков усреднялось. Для запуска тестов использовалась версия~2.2.1 фреймворка 
PyTorch и~операционная система Ubuntu 22.04. Результаты сравнения приведены 
в~табл.~4. По результатам сравнения предлагаемая 
архитектура показала меньшее время работы, чем аналогичная U-Net-по\-доб\-ная 
архитектура, и~превзошла по ско\-рости работы модели~\cite{MWCNN, MDIWT, WINNet}.


Полученные модели достигают схожего качества в~сравнении с~современными 
архитектурами\linebreak при меньшем чис\-ле па\-ра\-мет\-ров. При сравнении с~архитектурами, 
использующими вейв\-лет-пре\-обра\-зо\-ва\-ния~\cite{MWCNN, MDIWT, WINNet}, удалось 
добиться лучшего качества
подавления до\-ба\-воч\-но\-го шума. Если рассматривать пред\-ла\-га\-емый подход 
к~декодированию пред\-ска\-зан\-ных признаков в~изоб\-ра\-же\-ние, достигается качество, 
схожее с~использованием де\-ко\-ди\-ру\-ющей час\-ти архитектуры U-Net.

\section{Заключение}

Реализованный подход позволяет строить архитектуру шумоподавляющей сети из 
модифицированной модели для задачи классификации. При таком построении нет 
необходимости использовать слои для декодирования скрытого пространства, 
построенного кодировщиком входного изображения, что позволяет сократить число 
параметров модели. Обученные модели демонстрируют высокую скорость обработки 
изображений при достижении качества подавления шума, сопоставимого с~современными подходами.

Программный код с~реализацией подхода и~запуска обучения предлагаемой модели 
содержится в~GitHub-ре\-по\-зи\-то\-рии по следующей URL-ссыл\-ке: 
{\sf https://github.com/AlexeySrus/WPNet}.

{\small\frenchspacing
 { %\baselineskip=10.6pt
 %\addcontentsline{toc}{section}{References}
 \begin{thebibliography}{99}
    \bibitem{UNET_ORIGINAL}
\Au{Ronneberger~O., Fischer~P., Brox~T.}
    U-Net: Convolutional networks for biomedical image segmentation~// Medical 
Image     Computing and Computer-Assisted Intervention Proceedings.~--- Cham: 
Springer International Publishing, 2015. P.~234--241.

    \bibitem{UNETS_COMPARISON}
    \Au{Komatsu~R., Gonsalves~T.} Comparing U-Net based  models
    for denoising color images~// AI, 2020. Vol.~1. Iss.~4. P.~465--486. doi:  10.3390/ai1040029.

    \bibitem{IMAGE_NOISE_CONDITIONS}
    \Au{Hasinoff~S.\,W.} Saturation (imaging)~// Computer vision: A~reference
guide.~--- Boston, MA, USA: Springer, 2014. P.~699--701. doi: 10.1007/978-0-387-31439-6\_483.
    
    \bibitem{SSIM} %4
    \Au{Wang~Z., Bovic~A., Sheikh~H., Simoncelli~E.}
    Image quality assessment: From error visibility to structural similarity~//
    IEEE T. Image Process., 2004. Vol.~13. Iss.~4. P.~600--612. 
doi: 10.1109/TIP.2003.819861.
    
    \bibitem{MIX_LOSS}
    \Au{Zhao~H., Gallo~O., Frosio~I.,  Kautz~J.}
    Loss functions for image restoration with neural networks~// IEEE
    Transactions Computational Imaging, 2017. Vol.~3. Iss.~1. P.~47--57. 
doi: 10.1109/TCI.2016.2644865.
    
    \bibitem{CBAMCB} %6
    \Au{Woo~S., Jongchan~P., Joon-Young~L., In-So~K.}
    CBAM: Convolutional block attention module.~--- Cornell University, 2018. 
17~p.  arXiv:1807.06521. 
    
    \bibitem{CBAM_USAGE}
    \Au{Jiang~J., Xiangming~H., Zhao~Y.,  Xu~X., Cui~Y.}
    SDAUNet: A~simple dual attention mechanism UNet for mixed noise removal~// 
IET Image Process., 2023. Vol.~17. Iss.~13. P.~3884--3896. doi: 
10.1049/IPR2.12905.
    
    \bibitem{MWCNN} %8
    \Au{Liu~P., Zhang~H., Zhang~K.,  Lin~L., Zuo~W.}
    Multi-level wavelet-CNN for image restoration~// IEEE/CVF Conference 
on Computer Vision and Pattern Recognition Workshops Proceedings.~--- Los Alamitos, 
CA, USA: IEEE Computer Society, 2018. P.~886--895. doi: 
10.1109/CVPRW.2018.00121.
    
    \bibitem{WaveletPooling} %9
    \Au{Batziou~E., Ioannidis~K., Patras~I., Vrochidis~S., Kompatsiaris~I.} Low-light image 
enhancement based on U-Net and Haar wavelet pooling~//  MultiMedia modeling.~--- Cham: Springer Nature Switzerland, 2023. P.~510--522.
 
    \bibitem{MDIWT}
    \Au{Tian~C., Zheng~M., Zuo~W., Zhang~B., Zhang~Y., Zhang~D.}
    Multi-stage image denoising with the wavelet transform~// Pattern
    Recogn., 2023. Vol.~134. Art.~109050. doi: 
10.1016/j.patcog.2022.109050.
  
    \bibitem{WINNet} %11
    \Au{Huang~J.-J., Dragotti~P.\,L.} WINNet: Wavelet-inspired     invertible network for image denoising~//
     IEEE T.  Image Process., 2021. Vol.~31. P.~4377--4392.
    
    \bibitem{DWB} %12
    \Au{Guo~T., Mousavi~H., Vu~T.,  Monga~V.}  Deep wavelet prediction for image super-resolution~// 
 Conference on Computer Vision and Pattern Recognition Workshops 
Proceedings.~--- Los Alamitos, CA, USA: IEEE Computer Society, 2017. P.~1100--1109. 
doi: 10.1109/CVPRW.2017.148.
    
    \bibitem{IS_NET}
    \Au{Qin~X., Dai~H., Hu~X.,
Fan~D.-P., Shao~L., Van~G.}
    Highly accurate dichotomous image segmentation~// Computer vision.~--- Cham: Springer Nature Switzerland, 2022. P.~38--56.
    
    \bibitem{ResNet}
    \Au{He~K., Zhang~H., Ren~S.,  Sun~J.}
    Deep residual learning for image recognition~// Conference on 
Computer Vision and Pattern Recognition Proceedings.~--- Los Alamitos, CA, USA: IEEE 
Computer Society, 2016. P.~770--778. doi: 10.1109/CVPR.2016.90.
    
    \bibitem{POOLOPS}
    \Au{Scherer~D., Muller~A., Behnke~S.}
    Evaluation of pooling operations in convolutional architectures for object 
recognition~// Artificial neural networks.~--- Berlin, Heidelberg: 
Springer,   2010. P.~92--101. doi: 10.1007/978-3-642-15825-4\_10.
 
    \bibitem{FULLYCONV}
  \Au{Макаренко~А.\,В.} Глубокие нейронные сети: за\-рож\-де\-ние, 
становление,     современное состояние~// Проб\-ле\-мы управ\-ле\-ния, 2020. Т.~2. 
С.~3--19. doi: 10.25728/pu.2020.2.1.
  
    \bibitem{WAVELETS_PAPER} %17
    \Au{Буй~Т.\,Т.\,Ч., Спицын В.\,Г.} Разложение 
цифровых     изоб\-ра\-же\-ний с~по\-мощью двумерного дискретного вейв\-лет-пре\-обра\-зо\-ва\-ния и~быст\-ро\-го   преобразования \mbox{Хаара}~// Известия Томского 
политехнического университета, 2011. Т.~318. №\,5. С.~73--76.
    
    \bibitem{WAVELETS_APPROX} %18
    \Au{Павлов~А.\,Н.} Детектирование информационных сигналов на 
основе   реконструкции динамических сис\-тем и~дискретного вейв\-лет-пре\-обра\-зо\-ва\-ния~//\linebreak
 Известия  высших учебных заведений. Прикладная нелинейная 
динамика, 2008. Т.~16. №\,6. С.~3--17.

   \bibitem{WAVELETS_NOISE}
\Au{Пронькин~А.} Оценивание уровня шума в~составе изоб\-ра\-же\-ния с~использованием вейвлетов Хаара~// Труды 22-й Международной 
конференции по  компьютерной графике и~зрению.~---    М.: ИПМ им.\ М.\,В.~Келдыша, 2022.
 Т.~32. С.~442--448. doi: 
10.20948/graphicon-2022-442-448.
   
    \bibitem{OPENCV_LIB}
    \Au{Bradski~G.} The OpenCV Library~// Dr. Dobbs~J., 2000. Vol.~25. Iss.~11. P.~120--125.
    
    \bibitem{SIDD_2018_CVPR}
    \Au{Abdelhamed~A., Lin~S., Brown~M.\,S.} A~high-quality denoising dataset for smartphone cameras~// 
    IEEE/CVF Conference on Computer Vision and Pattern Recognition Proceedings.~--- Los Alamitos, 
CA, USA: IEEE Computer Society, 2018. P.~1692--1700.
    doi: 10.1109/CVPR.2018.00182.
    
    \bibitem{FOR_ADD_NOISE_DATASETS}
    \Au{Huang~J.-B., Singh~A., Ahuja~N.} 
Single     image super-resolution from transformed self-exemplars~// Conference 
on Computer Vision and Pattern Recognition.~--- Los Alamitos, CA, USA: 
IEEE Computer Society, 2015. P.~5197--5206.
    
    \bibitem{DIV2KDataset} %23
    \Au{Agustsson~E., Timofte~R.} NTIRE 2017 challenge 
on single    image super-resolution: Dataset and study~//  Conference 
on Computer Vision and Pattern Recognition Workshops Proceedings.~--- Los Alamitos, 
CA, USA: IEEE Computer Society, 2017. P.~1110--1121. doi: 
10.1109/CVPRW.2017.149.
    
    \bibitem{BSD_set} %24
    \Au{Martin~D., Fowlkes~C., Tal~D.,  Malik~J.}
    A~database of human segmented natural images and its application to 
evaluating
    segmentation algorithms and measuring ecological statistics~//  
8th  Conference (International) on Computer Vision Proceedings.~--- 
Piscataway, NJ, USA: IEEE, 2001. Vol.~2. P.~416--423. doi: 10.1109/ICCV.2001.937655.
    
    \bibitem{DNDSet} %25
    \Au{Plotz~T., Roth~S.} Benchmarking denoising 
algorithms  with     real photographs~// Conference on Computer Vision 
and Pattern Recognition Proceedings.~--- Los Alamitos, CA, USA: IEEE Computer Society, 
2017. P.~2750--2759. doi: 10.1109/CVPR.2017.294.
    
    \bibitem{PYTORCH_LIB} %26
    \Au{Paszke~A., Gross~S., Massa~F.,  Lerer~A.}
    PyTorch: An Imperative style, high-performance deep learning library~//
    Adv. Neur. Inf., 2019. Vol.~32. P.~8024--8035.
   
    \bibitem{TIMMLIB} %27
    \Au{Wightman~R.} PyTorch image models, 2019.
    {\sf  https://github.com/rwightman/pytorch-image-models}.
    
    \bibitem{UFORMER}
    \Au{Wang~Z., Cun~X., Bao~J., Zhou~W., Liu~J., Li~H.}
    Uformer: A~general U-shaped transformer for image restoration~// 
IEEE/CVF Conference on Computer Vision and Pattern Recognition Proceedings.~--- Los 
Alamitos, CA, USA: IEEE Computer Society, 2022. P.~17683--17693. doi: 
10.1109/CVPR52688.2022.01716/
    
    \bibitem{ADASMOOTH}
    \Au{Lu~J.} AdaSmooth: An adaptive learning rate method based on 
effective    ratio~// Sentiment analysis and deep learning.~--- Singapore: 
Springer Nature Singapore,   2023. P.~273--293.
    
    \bibitem{SMOOTHL1}
    \Au{Girshick~R.\,B.} Fast {R-CNN}.~--- Cornell University, 2015. 9~p.
arXiv:1504.08083. 
    
    \bibitem{BM3D} %31
    \Au{Dabov~K., Foi~A., Katkovnik~V.,  Egiazarian~K.}
    Image denoising by sparse 3-D transform-domain collaborative filtering~// 
IEEE T.  Image Process., 2007. Vol.~16. P.~2080--2095.
    
    \bibitem{DnCNN}
    \Au{Zhang~K., Zuo~W., Chen~Y., Meng~D., Zhang~L.}
    Beyond a~Gaussian denoiser: Residual learning of deep CNN for image 
denoising~// IEEE T. Image Process., 2017. Vol.~26. Iss.~7. 
P.~3142--3155.
    doi: 10.1109/TIP.2017.2662206.
\end{thebibliography}

 }
 }

\end{multicols}

\vspace*{-6pt}

\hfill{\small\textit{Поступила в~редакцию 21.12.23}}

\vspace*{10pt}

%\pagebreak

%\newpage

%\vspace*{-28pt}

\hrule

\vspace*{2pt}

\hrule



\def\tit{IMAGE DECOMPOSITION WITH~DISCRETE WAVELET TRANSFORM TO~DESIGN A~DENOISING NEURAL NETWORK}


\def\titkol{Image decomposition with discrete wavelet transform to~design a~denoising neural network}


\def\aut{A.\,S.~Kovalenko}

\def\autkol{A.\,S.~Kovalenko}

\titel{\tit}{\aut}{\autkol}{\titkol}

\vspace*{-8pt}


\noindent 
Institute of Mathematics, Mechanics, and Computer Science named after I.\,I.~Vorovich, 
Southern Federal University, 
105/42 Bolshaya Sadovaya Str.,  Rostov-on-Don 344006, Russian Federation


\def\leftfootline{\small{\textbf{\thepage}
\hfill INFORMATIKA I EE PRIMENENIYA~--- INFORMATICS AND
APPLICATIONS\ \ \ 2024\ \ \ volume~18\ \ \ issue\ 2}
}%
 \def\rightfootline{\small{INFORMATIKA I EE PRIMENENIYA~---
INFORMATICS AND APPLICATIONS\ \ \ 2024\ \ \ volume~18\ \ \ issue\ 2
\hfill \textbf{\thepage}}}

\vspace*{4pt}









\Abste{Reducing noise in digital images is one of the most common tasks in image processing. 
At the moment, noise reduction approaches based on the applying of convolutional neural networks are widely 
used. In this case, as a~rule, model training is based on minimizing the error function between the result 
of the network operation and the expected reference image and, additionally, various representations of 
the two-dimensional image signal and their properties are not used to optimize the training of noise 
reduction network architectures.
The paper proposes an approach to training neural networks to suppress noise. The described approach is 
based on the usage of the N-fold fast Haar wavelet transform. This representation of a discrete image 
signal allows one to discard the classical architecture of the autoencoder and to use only its part that 
encodes the signal which leads to a significant reduction in model parameters and speeds up the network.}

\KWE{neural networks; deep learning; image denoising; image processing}


\DOI{10.14357/19922264240209}{UEQSXP}

%\vspace*{-12pt}

%\Ack

\vspace*{4pt}


    % \noindent
   


  \begin{multicols}{2}

\renewcommand{\bibname}{\protect\rmfamily References}
%\renewcommand{\bibname}{\large\protect\rm References}

{\small\frenchspacing
 {%\baselineskip=10.8pt
 \addcontentsline{toc}{section}{References}
 \begin{thebibliography}{99} 
 
 \vspace*{-6pt}
 
    \bibitem{UNET_ORIGINAL-1} %1
    \Aue{Ronneberger,~O., P.~Fischer, and T.~Brox.} 2015.
    U-Net: Convolutional networks for biomedical image segmentation. \textit{Medical Image
    Computing and Computer-Assisted Intervention Proceedings}.  Cham: Springer
    International Publishing. 234--241.
    
    \bibitem{UNETS_COMPARISON-1} %2
    \Aue{Komatsu, R., and T.~Gonsalves.} 2020. Comparing U-Net based models
    for denoising color images.  \textit{AI} 1(4):465--486. doi: 10.3390/ai1040029.
    
    \bibitem{IMAGE_NOISE_CONDITIONS-1} %3
    \Aue{Hasinoff, S.\,W.} 2014.
    Saturation (imaging). \textit{Computer vision: A~reference
    guide}. Boston, MA: Springer. 699--701. doi: 10.1007/978-0-387-31439-6\_483.
    
    \bibitem{SSIM-1} %4
    \Aue{Wang,~Z., A.~Bovic, H.~Sheikh, and E.~Simoncelli.}  2004.
    Image quality assessment: From error visibility to structural similarity.
    \textit{IEEE T. Image Process.} 13(4):600--612. doi: 10.1109/TIP.2003.819861.
    
    \bibitem{MIX_LOSS-1} %5
    \Aue{Zhao, H., O.~Gallo, I.~Frosio, and J.~Kautz} 2017.
    Loss functions for image restoration with neural networks. \textit{IEEE
    Trans. Computational Imaging} 3(1):47--57. doi: 10.1109/TCI.2016.2644865.
  
    \bibitem{CBAMCB-1} %6
    \Aue{Woo, S., P.~Jongchan, L.~Joon-Young, and K.~In-So.} 2018.
    CBAM: Convolutional block attention module. 17~p. 
    Available at: {\sf  https://arxiv.org/abs/1807.06521} (accessed April~28, 2024).
    
    \bibitem{CBAM_USAGE-1} %7
    \Aue{Jiang, J., H.~Xiangming, Y.~Zhao, X.~Xu,  and Y.~Cui.} 2023.
    SDAUNet: A~simple dual attention mechanism UNet for mixed noise removal. \textit{IET Image Process.}
 17(13):3884--3896. doi: 10.1049/IPR2.12905.
   
    \bibitem{MWCNN-1} %8
    \Aue{Liu, P., H.~Zhang, K.~Zhang, L.~Lin,  and W.~Zuo.} 2018.
    Multi-level wavelet-CNN for image restoration. \textit{IEEE/CVF  Conference on
    Computer Vision and Pattern Recognition Workshops Proceedings}.  Los Alamitos, CA:
    IEEE Computer Society. 886--895. doi: 10.1109/CVPRW. 2018.00121.
    
    \bibitem{WaveletPooling-1} %9
    \Aue{Batziou, E., K.~Ioannidis, I.~Patras, S.~Vrochidis, and I.~Kompatsiaris.} 2023.
    Low-light image enhancement based on U-Net and Haar wavelet pooling. 
    \textit{MultiMedia modeling}.  Cham: Springer Nature Switzerland.  510--522.
  
    \bibitem{MDIWT-1} %10
    \Aue{Tian, C., M.~Zheng, W.~Zuo,  B.~Zhang, Y.~Zhang, and D.~Zhang.} 2023.
    Multi-stage image denoising with the wavelet transform. \textit{Pattern
    Recogn.} 134:109050. doi: 10.1016/j.patcog.2022.109050.
    
    \bibitem{WINNet-1} %11
    \Aue{Huang, J.-J., and P.\,L.~Dragotti.} 2021. WINNet: Wavelet-inspired
    invertible network for image denoising. \textit{IEEE T. Image Process.} 
31:4377--4392.
    
    \bibitem{DWB-1} %12
    \Aue{Guo, T., H.~Mousavi, T.~Vu, and V.~Monga.} 2017.
    Deep wavelet prediction for image super-resolution. \textit{Conference on
    Computer Vision and Pattern Recognition Workshops Proceedings}.  Los Alamitos, CA: IEEE Computer Society. 
    1100--1109. doi: 10.1109/CVPRW.2017.148.
   
    \bibitem{IS_NET-1} %13
    \Aue{Qin, X., H.~Dai, X.~Hu, D.-P.~Fan, L.~Shao, and G.~Van.} 2022.
    Highly accurate dichotomous image segmentation. \textit{Computer vision}.
     Cham: Springer Nature Switzerland.\linebreak 38--56.
    
    \bibitem{ResNet-1} %14
    \Aue{He, K., H.~Zhang, S.~Ren, and J.~Sun.} 2016.
    Deep residual learning for image recognition. \textit{Conference on
    Computer Vision and Pattern Recognition Proceedings}.  Los Alamitos, CA.  770--778.
   
    \bibitem{POOLOPS-1} %15
    \Aue{Scherer, D., A.~Muller, and S.~Behnke.} 2010.
    Evaluation of pooling operations in convolutional architectures for object recognition.
    \textit{Artificial neural networks}.  Berlin, Heidelberg: Springer.  92--101. doi: 10.1007/978-3-642-15825-4\_10.
    
    \bibitem{FULLYCONV-1} %16
    \Aue{Makarenko, A.\,V.} 2020.
    Glubokie neyronnye seti: zarozhdenie, stanovlenie, sovremennoe sostoyanie
    [Deep neural networks: Origins, development, and current status]. 
\textit{Problemy upravleniya} [Control Sciences] 2:3--19. doi: 10.25728/pu.2020.2.1.

\columnbreak
    
    \bibitem{WAVELETS_PAPER-1} %17
    \Aue{Buy, T.\,T.\,C., and V.\,G.~Spitsyn.} 2011. Razlozhenie tsifrovykh 
    izobrazheniy s~po\-moshch'yu dvumernogo diskretnogo veyvlet-preobrazovaniya i~bystrogo preobrazovaniya 
    Khaara
    [Digital image decomposition using two-dimensional discrete wavelet transform and fast Haar transform].
\textit{Izvestiya Tomskogo politekhnicheskogo universiteta} [Bulletin of the Tomsk Polytechnic University] 318(5):73--76.
  
    \bibitem{WAVELETS_APPROX-1} %18
    \Aue{Pavlov, A.\,N.} 2008. Detektirovanie informatsionnykh signalov na osnove rekonstruktsii 
    dinamicheskikh sistem i~diskretnogo veyvlet-preobrazovaniya
    [Detection of information signals based on reconstruction of dynamical systems and discrete wavelet-transform]. 
    \textit{Izvestiya vysshikh uchebnykh zavedeniy. 
    Prikladnaya nelineynaya dinamika} [Bulletin of Higher Educational Institutions. Applied Nonlinear Dynamics] 16(6):3--17.
    
    \bibitem{WAVELETS_NOISE-1}
    \Aue{Pronkin, A.\,V.} 2022. Otsenivanie urovnya shuma v~sostave izobrazheniya s~ispol'zovaniem veyvletov Khaara
    [Noise level estimation in images using Haar wavelets].
\textit{22nd Conference (International) on Computer Graphics and Computer Vision Proceedings}.  Moscow:  IPM im.\ M.\,V.~Keldysha. 32:442--448. 
    doi: 10.20948/graphicon-2022-442-448.
 
    \bibitem{OPENCV_LIB-1}
    \Aue{Bradski, G.} 2000. The OpenCV Library. \textit{Dr. Dobbs J.}  25(11):120--125.
    
    \bibitem{SIDD_2018_CVPR-1} %21
    \Aue{Abdelhamed, A., S.~Lin, and M.\,S.~Brown.} 2018. 
    A~high-quality denoising dataset for smartphone cameras. \textit{IEEE/CVF Conference on Computer Vision and Pattern Recognition Proceedings.} 
    Los Alamitos, CA: IEEE Computer Society. 1692--1700. 
    doi: 10.1109/CVPR. 2018.00182.
   
    \bibitem{FOR_ADD_NOISE_DATASETS-1}
    \Aue{Huang, J.-B., A.~Singh, and N.~Ahuja.} 2015. Single
    image super-resolution from transformed self-exemplars. \textit{Conference on Computer Vision and Pattern Recognition Proceedings}. 
    Los Alamitos, CA: IEEE Computer Society. 5197--5206.
   
    \bibitem{DIV2KDataset-1}
    \Aue{Agustsson, E., and R.~Timofte.} 2017. NTIRE 2017 challenge on single
    image super-resolution: Dataset and study. \textit{Conference on Computer Vision and
    Pattern Recognition Workshops Proceedings}. Los Alamitos, CA: IEEE Computer Society. 1110--1121. doi: 10.1109/ CVPRW.2017.149. 
   
    \bibitem{BSD_set-1} %24
    \Aue{Martin, D., C.~Fowlkes, D.~Tal, and J.~Malik.} 2001.
    A~database of human segmented natural images and its application to evaluating
    segmentation algorithms and measuring ecological statistics. \textit{8th Conference (International) on Computer Vision 
    Proceedings}. Piscataway, NJ: IEEE Computer Society. 2:416--423. doi: 10.1109/ICCV.\linebreak 2001.937655.
    
    
    \bibitem{DNDSet-1} %25
    \Aue{Plotz, T., and S.~Roth.} 2017. Benchmarking denoising algorithms with
    real photographs. 2017. \textit{Conference on Computer Vision and Pattern Recognition Proceedings}. 
Los Alamitos, CA: IEEE Computer Society. 
2750--2759. doi: 10.1109/CVPR.2017.294.
    
    \bibitem{PYTORCH_LIB-1}
    \Aue{Paszke, A., S.~Gross, F.~Massa, and A.~Lerer.} 2019.
    PyTorch: An imperative style, high-performance deep learning library.
\textit{Adv. Neur. Inf.} 32:8024--8035.

\pagebreak
   
    \bibitem{TIMMLIB-1}
    \Aue{Wightman, R.} 2019. PyTorch image models.
    Available at: {\sf https://github.com/rwightman/pytorch-image-models} (accessed April~28, 2024).
   
    \bibitem{UFORMER-1}
    \Aue{Wang, Z., X.~Cun, J.~Bao,
W.~Zhou, J.~Liu, and H.~Li.} 2022.
    Uformer: A~general U-shaped transformer for image restoration. 
\textit{IEEE/CVF Conference on Computer Vision and Pattern Recognition Proceedings}. Los Alamitos, CA:
 IEEE Computer Society. 17683--17693.
    
    \bibitem{ADASMOOTH-1}
    \Aue{Lu, J.} 2023. AdaSmooth: An adaptive learning rate method based on effective
    ratio. \textit{Sentiment analysis and deep learning}.  Singapore: Springer Nature Singapore. 273--293.
    

   
    \bibitem{SMOOTHL1-1}
    \Aue{Girshick, R.\,B.} 2015.
    Fast {R-CNN}. 9~p.  
    Available at: {\sf https://arxiv.org/abs/1504.08083} (accessed April~28, 2024).
    
    \bibitem{BM3D-1}
    \Aue{Dabov, K., A.~Foi, V.~Katkovnik, and K.~Egiazarian.} 2007.
    Image denoising by sparse 3-D transform-domain collaborative filtering. 
    \textit{IEEE T. Image Process.}  16:2080--2095.
    
    \bibitem{DnCNN-1}
    \Aue{Zhang, K., W.~Zuo, Y.~Chen, D.~Meng, and L.~Zhang.} 2017.
    Beyond a~Gaussian denoiser: Residual learning of deep CNN for image denoising. 
\textit{IEEE T. Image Process.}  26(7):3142--3155. 
    doi: 10.1109/TIP.2017.2662206.
\end{thebibliography}

 }
 }

\end{multicols}

\vspace*{-6pt}

\hfill{\small\textit{Received December 21, 2023}} 

\vspace*{-12pt}


\Contrl

\vspace*{-3pt}


\noindent 
\textbf{Kovalenko Alexey S.} (b.\ 1996)~--- assistant, Department of Applied Mathematics and Programming, 
Institute of Mathematics, Mechanics, and Computer Science named after I.\,I.~Vorovich, 
Southern Federal University, 105/42~Bolshaya Sadovaya Str., Rostov-on-Don 344006, Russian Federation; 
\mbox{akov@sfedu.ru}




\label{end\stat}

\renewcommand{\bibname}{\protect\rm Литература}  %9
\def\stat{bosov+stef}

\def\tit{УПРАВЛЕНИЕ ВЫХОДОМ СТОХАСТИЧЕСКОЙ ДИФФЕРЕНЦИАЛЬНОЙ СИСТЕМЫ 
ПО~КВАДРАТИЧНОМУ КРИТЕРИЮ. I.~ОПТИМАЛЬНОЕ РЕШЕНИЕ МЕТОДОМ 
ДИНАМИЧЕСКОГО ПРОГРАММИРОВАНИЯ$^*$}

\def\titkol{Управление выходом стохастической дифференциальной системы 
по~квадратичному критерию. I}
%.~Оптимальное решение методом 
%динамического программирования}

\def\aut{А.\,В.~Босов$^1$, А.\,И.~Стефанович$^2$}

\def\autkol{А.\,В.~Босов, А.\,И.~Стефанович}

\titel{\tit}{\aut}{\autkol}{\titkol}

\index{Босов А.\,В.}
\index{Стефанович А.\,И.}
\index{Bosov A.\,V.}
\index{Stefanovich A.\,I.}




{\renewcommand{\thefootnote}{\fnsymbol{footnote}} \footnotetext[1]
{Работа выполнена при частичной поддержке РФФИ (проект 16-07-00677).}}


\renewcommand{\thefootnote}{\arabic{footnote}}
\footnotetext[1]{Институт проблем информатики Федерального исследовательского центра <<Информатика 
и~управление>> Российской академии наук, \mbox{AVBosov@ipiran.ru}}
\footnotetext[2]{Институт проблем информатики Федерального исследовательского центра <<Информатика 
и~управление>> Российской академии наук, \mbox{AStefanovich@frccsc.ru}}

%\vspace*{8pt}



  
  \Abst{Решается задача оптимального управления для диффузионного процесса 
Ито и~линейного управ\-ля\-емо\-го выхода. Рассматриваемая постановка близка 
к~классической ли\-ней\-но-квад\-ра\-тич\-ной гауссовской задаче управления 
(linear-quadratic Gaussian (LQG) control). Отличия состоят в~том, что состояние описывается нелинейным 
дифференциальным уравнение Ито $dy_t\hm= A_t(y_t) \,dt\hm+ \Sigma_t(y_t)\,dv_t$ 
и~не зависит от управ\-ле\-ния~$u_t$, оптимизации подлежит управ\-ля\-емый 
линейный выход $dz_t\hm= a_t y_t\,dt\hm+ b_t z_t \,dt\hm+ c_t u_t \,dt\hm+ \sigma_t\, 
dw_t$. Дополнительные обобщения внесены в~квад\-ра\-тич\-ный критерий качества 
с~целью воз\-мож\-ности постановки таких задач, как отслеживание выходом 
состояния или управ\-ле\-ни\-ем~--- линейной комбинации состояния и~выхода. Для 
решения используется метод динамического программирования. Функцию 
Беллмана позволяет найти предположение о~ее структуре вида $V_t(y,z)\hm= 
\alpha_t z^2\hm+ \beta_t(y)z \hm+\gamma_t(y)$. Решение дают три 
дифференциальных уравнения для коэффициентов~$\alpha_t$, $\beta_t(y)$ 
и~$\gamma_t(y)$. Эти уравнения со\-став\-ля\-ют оптимальное решение 
рас\-смат\-ри\-ва\-емой задачи.}
  
  \KW{стохастическое дифференциальное уравнение; оптимальное управ\-ле\-ние; 
динамическое программирование; функция Беллмана; уравнение Риккати; 
линейные уравнения параболического типа}

\DOI{10.14357/19922264180314}
  
%\vspace*{4pt}


\vskip 10pt plus 9pt minus 6pt

\thispagestyle{headings}

\begin{multicols}{2}

\label{st\stat}

\section{Введение}

     Ключевые результаты в~области оптимизации стохастических 
динамических систем, со\-став\-ля\-ющие классическую теорию управления, 
получены более~40~лет назад (такова работа~[1] в~отношении задачи 
управ\-ле\-ния ли\-ней\-но-гаус\-сов\-ски\-ми стохастическими сис\-те\-ма\-ми по 
квад\-ра\-тич\-но\-му критерию). К~классической тео\-рии следует относить 
линейные модели стохастических сис\-тем и~квадратичный критерий качества. 
Это исходный базис, на котором основано множество успешно 
исследованных и~решенных задач стохастического управ\-ле\-ния 
и~оптимизации. 

Дальнейшее развитие~--- это новые модели и~критерии, но 
прежде всего это новые методы: от тео\-рии линейных регуляторов, метода 
динамического программирования и~принципа максимума к~адаптивному 
и~минимаксному подходу, импульсному управ\-ле\-нию и~т.\,д. Множество 
инноваций как в~час\-ти моделей, так и~в~час\-ти математического аппарата, 
имевших мес\-то в~по\-сле\-ду\-ющие годы, существенно обогатили тео\-рию 
управ\-ле\-ния. Но и~до настоящего времени линейные модели и~квадратичный 
критерий, несмотря на всю справедливую критику в~отношении их 
аде\-кват\-ности и~гиб\-кости, сохраняют исследовательский интерес и~находят 
современные области приложения.
     
     Не претендуя на сколь\-ко-ни\-будь полное обосно\-ва\-ние последнего 
тезиса, приведем несколько примеров, показавшихся наиболее ин\-те\-рес\-ными. 

Так, в~[2] решается ли\-ней\-но-квад\-ра\-тич\-ная за\-да\-ча в~игровой 
постановке с~запаздыванием. В~близ\-кой по модели работе~[3] задача 
управ\-ле\-ния ставится в~терминах $H_\infty$-ро\-баст\-ности. Точнее \mbox{называть} 
эту тематику $H_2/H_\infty$-управ\-ле\-ни\-ем, и~работ по этой теме очень 
много. Аккуратности ради следует уточнить, что под линейными 
понимаются модели с~мультипликативными по состоянию воз\-му\-ще\-ниями. 

Совсем другой класс моделей, особо популярных в~по\-след\-ние годы, 
составляют скачкообразные процессы. Например, линейные уравнения 
в~сочетании с~пуассоновскими скачками в~[4] используются в~моделях, 
описывающих различные показатели функционирования сетевых протоколов 
передачи данных транспортного уровня. Телекоммуникации представляют 
в~последние годы самый популярный прикладной материал для 
исследований, работ по этой проб\-ле\-ма\-ти\-ке множество, математические 
техники привлекаются самые разные и~самые современные, но и~линейным 
моделям место находится. Еще один любопытный пример исследования 
скачкообразного процесса и~оптимизации на основе квад\-ра\-тич\-но\-го критерия 
можно найти в~[5] применительно к~задаче инвестирования на финансовом 
рынке. Наконец, упомянем еще работу~[6], подводящую итог исследований 
в~отношении классической детерминированной  
ли\-ней\-но-квад\-ра\-тич\-ной задачи с~использованием техники матричных 
неравенств.
     
     В данной работе также эксплуатируются привлекательные свойства 
линейных моделей и~квад\-ра\-тич\-но\-го критерия, причем в~стохастической 
постановке. На\-прав\-ле\-ни\-ем для обобщения \mbox{выбрана} модель динамики 
сис\-те\-мы: основные усилия на\-прав\-ле\-ны на то, чтобы сделать ее нелинейной. 
Кроме того, пред\-став\-лен\-ная постановка может рас\-смат\-ри\-вать\-ся и~как 
обобщение ранее решенной задачи в~дискретном времени~[7, 8] на время 
непрерывное. В~упомянутых работах помимо собственно модельной 
постановки важна еще и~привлекаемая прикладная об\-ласть~--- 
функционирование сложных программных сис\-тем. Результатов, 
ориентированных непосредственно на такие приложения, к~настоящему 
времени пренебрежимо мало, поэтому~[7, 8]~--- это еще и~прикладное 
обоснование рас\-смат\-ри\-ва\-емой далее задачи.
     
     Оптимизируемая динамическая сис\-те\-ма описывается двумя 
уравнениями. Состояние задается нелинейным стохастическим 
дифференциальным уравнением Ито, не содержащим управ\-ля\-емой 
переменной. Возмущение здесь описывается стандартным винеровским 
процессом, накладываются простые условия существования 
и~един\-ст\-вен\-ности решения. Поскольку состояние не управ\-ля\-ет\-ся, то уместно 
его интерпретировать как слож\-ное внешнее возмущение. Вторая 
переменная~--- управ\-ля\-емый выход~--- задается линейным стохастическим 
дифференциальным уравнением. Цель оптимизации выхода формируется 
квадратичным критерием общего вида. Формальная постановка задачи 
приведена в~сле\-ду\-ющем разделе.
     
     Для решения задачи используется метод динамического 
программирования, решается уравнение Беллмана~[9]. Соответственно, 
в~результате получаются аналитические выражения и~для оптимального 
управ\-ле\-ния, и~для значения функционала качества. Технически 
традиционный, стандартный подход к~задаче обременен, пожалуй, 
единственной проблемой~--- поиском верного пред\-став\-ле\-ния структуры 
функции Беллмана. Справиться с~этой проблемой в~большей степени удается 
за счет результата, полученного при решении дискретного по времени 
аналога рассматриваемой постановки~\cite{8-bos}. Конечные соотношения 
для оптимального решения, как и~во всех подобных задачах, включая 
классическую ли\-ней\-но-квад\-ра\-тич\-ную, содержат решения 
определенных дифференциальных уравнений (обыкновенных и~в~частных 
производных). Вывод этих уравнений и~со\-став\-ля\-ет содержание первой час\-ти 
данной работы. Во второй части будет обсуждаться их приближенное 
чис\-лен\-ное решение и~компьютерные эксперименты.
     
     Кратко обозначим основные положения, при\-вле\-ка\-емые далее 
к~решению задачи, следуя в~основном обозначениям 
и~терминологии~\cite{9-bos}, а~именно: будем рассматривать задачу 
оптимального управления в~стохастической динамической сис\-те\-ме по полной 
информации, применяя метод динамического программирования. В~качестве 
целевого функционала, опре\-де\-ля\-юще\-го качество управ\-ле\-ния $U_0^T\hm= \{ 
u_t,\ 0\leq t\leq T\}$, выступает
     \begin{equation}
     J\left(U_0^T\right)={\sf E}\left\{ \int\limits_0^T L_t \left(x_t, u_t\right)\,dt+ 
l\left(x_T\right)\right\}\,.
     \label{e1-bos}
     \end{equation}
Здесь ${\sf E}\{\cdot\}$~--- оператор математического ожидания; $x_t$~--- 
случайный процесс, описываемый стохастическим дифференциальным 
уравнением Ито
     \begin{equation}
     dx_t=m_t\left( x_t, u_t\right) dt+ \sigma_t\left( x_t\right)dW_t\,,\enskip 
x_0=X\,,
     \label{e2-bos}
     \end{equation}
где $W_t$~--- стандартный винеровский процесс подходящей раз\-мер\-ности; 
$X$~--- случайный вектор.

     $U_0^T$ будем выбирать из класса допустимых неупреждающих (по 
отношению к~$W_t$) управлений~\cite{9-bos}. Соответственно, 
относительно функций сноса и~диффузии~$m_t$ и~$\sigma_t$  
в~(\ref{e2-bos}) будем предполагать выполненными ка\-кие-ли\-бо условия 
существования сильного решения для заданного до\-пус\-ти\-мо\-го управ\-ле\-ния. 
Например, для управ\-ле\-ния с~обратной связью $u_t\hm= u_t(x_t)$ будем 
считать, что $m_t(x,u_t(x))$ и~$\sigma_t(x)$ удовлетворяют условию 
линейного рос\-та и~локальному условию Липшица по~$x$ равномерно 
по~$t$ (т.\,е.\ условиям Ито).
     
     Для поиска оптимального управления, минимизирующего $J(U_0^T)$, 
рас\-смат\-ри\-ва\-ет\-ся функция Беллмана
     \begin{equation}
     V_t(x)=\left.\mathop{\mathrm{inf}}\limits_{U_t^T} {\sf E} \left\{ \int\limits_t^T 
L_t \left( x_t, u_t\right)\,dt+l\left( x_T\right) \right\vert \mathcal{F}_t^x\right\}\,,
     \label{e3-bos}
     \end{equation}
где $\mathcal{F}_t^x$~--- $\sigma$-ал\-геб\-ра, по\-рож\-ден\-ная~$x_\tau$, 
$0\hm\leq \tau\hm\leq t$, ${\sf E}\{\cdot\vert \mathcal{F}\}$~--- оператор условного 
математического ожидания относительно~$\mathcal{F}$. Соответственно, 
в~качестве достаточного условия оп\-ти\-маль\-ности воспользуемся уравнением 
динамического программирования
\begin{multline}
\fr{\partial V_t(x)}{\partial t} +\fr{1}{2}\sum\limits^n_{i,j=1} \sigma^2_{t_{ij}}
\fr{\partial^2 V_t(x)}{\partial x_i \partial x_j}+{}\\
{}+\min\limits_u\left[  
\sum\limits^n_{i=1} m_{t_i} \fr{\partial V_t(x)}{\partial x_i} + L_t(x,u)\right] 
=0\,,\\
V_T(x)=l(x)\,,
\label{e4-bos}
\end{multline}
где $m_{t_i}$~--- $i$-й элемент век\-тор-функ\-ции~$m_t(x,u)$; 
$\sigma^2_{t_{ij}} \hm= \sum\nolimits^m_{k=1} 
\sigma_{t_{ik}}\sigma_{t_{ki}}$, $\sigma_{t_{ij}}$~--- $i$-й по строке, $j$-й 
по столб\-цу элемент мат\-рич\-ной функции~$\sigma_t(x)$; $n$ и~$m$~--- 
размерности~$x_t$ и~$W_t$ соответственно.

     Традиционно в~рамках применения метода динамического 
программирования будем предполагать, что функции~$L_t$, $l$, $m_t$ 
и~$\sigma_t$ обеспечивают существование хотя бы одного решения 
уравнения~(\ref{e4-bos}), а~следовательно, и~оптимального 
управления~$u_t^*$, $0\hm\leq t\hm\leq T$, до\-став\-ля\-юще\-го минимум 
целевому функционалу~(\ref{e1-bos}). Задача оптимизации далее получается 
путем указания конкретных выражений для~$L_t$, $l$, $m_t$ и~$\sigma_t$.

\section{Постановка задачи управления выходом}

     Рассматриваемые далее случайные функции будут предполагаться 
скалярными. Такое упрощение позволит разгрузить выкладки и~итоговые 
выражения от не самых существенных деталей.
     
     Рассмотрим стохастическую дифференциальную сис\-те\-му, со\-сто\-яние 
которой представляет диффузи\-он\-ный процесс~$y_t$, описываемый 
нелинейным стохастическим дифференциальным уравнением Ито
     \begin{equation}
     dy_t=A_t\left( y_t\right) dt +\Sigma_t \left( y_t\right) dv_t\,,\enskip 
y_0=Y\,,
     \label{e5-bos}
     \end{equation}
где $v_t$~--- стандартный (одномерный) винеровский процесс; $Y$~--- 
случайная величина с~конечным вторым моментом; функции~$A_t$ 
и~$\Sigma_t$ удовлетворяют условиям Ито:
\begin{equation*}
\left\vert A_t(y)\right\vert +\left\vert \Sigma_t(y)\right\vert \leq C(1+\vert y\vert )\ 
\mbox{для\ всех } 0\leq t\leq T\,;
\end{equation*}

\vspace*{-12pt}

\noindent
\begin{multline*}
\hspace*{-2.10051pt}\left\vert A_t\left(y_1\right) -A_t \left( y_2\right) \right\vert +\left\vert 
\Sigma_t\left( y_1\right) -\Sigma_t \left(y_2\right)\right\vert \leq
C\left\vert y_1-y_2\right\vert\\
 \mbox{для\ всех\ } 0\leq t\leq T\ \mbox{и } 
y_1,y_2\in \mathbb{R}^1\,,
\end{multline*}
обеспечивающим существование единственного сильного (потраекторного) 
решения уравнения.
     
     Будем считать, что~$y_t$ описывает состояние некоторой 
динамической системы. Соответственно, поведение этой сис\-те\-мы опишем 
выходом, линейно связанным с~со\-сто\-янием:
     \begin{equation}
     dz_t=a_t y_t \,dt+ b_t z_t \,dt+ c_t u_t \,dt+\sigma_t \,dw_t\,,\enskip
     z_0=Z\,.
     \label{e6-bos}
     \end{equation}
Здесь $w_t$~--- не зависящий от~$v_t$, $Y$ и~$Z$ стандартный (одномерный) 
винеровский процесс; $Z$~--- случайная величина с~конечным вторым 
моментом; $u_t$~--- допустимое неупреждающее управ\-ле\-ние, качество 
которого определяется целевым функционалом следующего вида:
\begin{multline}
\!\hspace*{-3.98538pt}J\left( U_0^T\right) ={\sf E}\left\{ \int\limits_0^T \!\left( S_t\left( s_ty_t-g_t z_t -h_t 
u_t\right)^2 +G_t z_t^2+{}\right.\right.\\
\left.\left.{}+ H_t u_t^2
\vphantom{S_t\left( s_ty_t-g_t z_t -h_t 
u_t\right)^2}
\right) dt+S_T\left( s_T y_T -g_T 
z_T\right)^2+G_T z_T^2
\vphantom{\int\limits_0^T}\right\}\,,
\label{e7-bos}
\end{multline}
где $S_t$, $G_t$ и~$H_t$~--- неотрицательные функции\linebreak
$0\hm\leq t\hm\leq T$. 
Такой критерий отражает физический смысл задачи распределения ресурсов 
со\-глас\-но аналогичной~(\ref{e5-bos})--(\ref{e7-bos}) задаче для дис\-крет\-но\-го 
времени, рас\-смот\-рен\-ной в~\cite{7-bos}. В~част\-ности,  
функци\-онал~(\ref{e7-bos}) поз\-во\-ля\-ет ставить задачи отслеживания
 выходом 
со\-сто\-яния сис\-те\-мы, используя сла\-га\-емое $(y_t\hm- z_t)^2$, или 
управлением~--- линейной комбинации со\-сто\-яния и~выхода, сла\-га\-емое типа\linebreak 
$(y_t\hm+ z_t\hm- u_t)^2$. Поскольку задача формулируется 
в~предположении наличия пол\-ной информации о~со\-сто\-янии~$y_t$ 
и~выходе~$z_t$ (соответствующую $\sigma$-ал\-геб\-ру 
обозначим~$\mathcal{F}_t^{y,z}$), то допустимое управ\-ле\-ние ищется 
в~классе~$\mathcal{F}_t^{y,z}$-из\-ме\-ри\-мых неупреждающих функций 
(и,~как будет показано далее, оказывается управ\-ле\-ни\-ем с~обратной связью).

     Функции~$a_t$, $b_t$, $c_t$ и~$\sigma_t$ будем предполагать 
ограниченными: $\vert a_t\vert \hm+ \vert b_t\vert \hm+\vert c_t\vert \hm+ \vert 
\sigma_t \vert \hm\leq C$ для всех $0\hm\leq t\hm\leq T$, процесс  
управления~--- допустимым не\-упреж\-да\-ющим~\cite{9-bos}, обеспечивая, 
таким образом, существование сильного решения урав\-не\-ния~(\ref{e6-bos}) 
для любого допустимого управ\-ления.
     
     Задачу составляет поиск~$u_t^*$~--- допустимого управ\-ле\-ния, 
доставляющего минимум квад\-ра\-тич\-но\-му функционалу~$J(U_0^T)$.
      
     Поставленная задача очевидным образом формулируется в~терминах 
введенных выше в~(\ref{e1-bos})--(\ref{e3-bos}) обозначений, а~именно: 
     требуется обозначить
     \begin{gather*}
      x_t=\begin{pmatrix}
     y_t\\ z_t\end{pmatrix};\quad  m_t(x_t, u_t)=\begin{pmatrix}
     A_t(y_t)\\ a_t y_t +b_t z_t +c_t u_t\end{pmatrix};\\
     \sigma_t(x_t)= \begin{pmatrix}
     \Sigma_t(y_t)& 0\\
     0& \sigma_t\end{pmatrix};\quad W_t=\begin{pmatrix}
     v_t \\ w_t\end{pmatrix}
     %     \label{e8-bos}
     \end{gather*}
для записи уравнения со\-сто\-яния типа~(\ref{e2-bos}) и
\begin{align*}
L_t(x,u)&= L_t(y,z,u) ={}\\
&\hspace*{3mm}{}=S_t\left( s_t y-g_t z -h_t u\right)^2 +G_t z^2 +H_t  u^2\,;\\
l(x)&= l(y,z) =S_T \left( S_T y-g_T z\right)^2 +G_T z^2
%\label{e9-bos}
\end{align*}
для записи целевого функционала в~виде~(\ref{e1-bos}).

     Функция Беллмана~(\ref{e3-bos}) принимает вид 
     $V_t(x)\hm= V_t(y,z)$. Для записи со\-от\-вет\-ст\-ву\-юще\-го~(\ref{e4-bos}) 
уравнения Беллмана для~$V_t(y,z)$ заметим, что
     $$
     \left( \sigma^2_{t_{ij}}\right)_{i,j=1,2}= \begin{pmatrix}
     \Sigma_t^2(y) & 0\\
     0 & \sigma_t^2\end{pmatrix}\,.
     $$
     
     С~учетом перечисленных обозначений урав\-не\-ние динамического 
программирования~(\ref{e4-bos}) принимает вид:
     \begin{multline}
     \fr{\partial V_t(y,z)}{\partial t} +\fr{1}{2}\left( \Sigma_t^2(y) \fr{\partial^2 
V_t(y,z)} {\partial y^2}+\sigma_t^2\fr{\partial^2 V_t(y,z)} {\partial 
z^2}\right)+{}\\
    {}+\min\limits_u\! \left[ A_t(y) \fr{\partial V_t(y,z)}{\partial y}+\left( a_t 
y+b_t z+c_t u\right) \fr{\partial V_t(y,z)}{\partial z} +{}\right.\hspace*{-3pt}\\
\left.{}+ S_t\left( s_t y-g_t z-h_t 
u\right)^2+G_t z^2+H_t u^2
     \vphantom{\fr{\partial V_t(y,z)}{\partial y}}\right] =0\,,\\
     V_T(y,z)=S_T\left( s_T y-g_T z\right)^2+G_T z^2\,.
     \label{e10-bos}
     \end{multline}
     Это и~есть то самое уравнение, которое требуется решить: 
существование решения данного урав\-не\-ния суть достаточное условие 
оптимальности; оптимальное управ\-ле\-ние при этом~--- точ\-ка минимума 
со\-от\-вет\-ст\-ву\-юще\-го сла\-га\-емого.
     
\section{Динамическое программирование и~оптимальное 
управление}

     В рассматриваемой постановке линейность\linebreak выхода и~квадратичность 
критерия дают те же преимущества, что и~в~классической  
ли\-ней\-но-квад\-ра\-тич\-ной задаче управ\-ле\-ния~\cite{1-bos}, а~именно: 
позволяют сразу определить вид оптимального управ\-ле\-ния и~фактические 
условия его существования. Действительно, со\-хра\-няя в~(\ref{e10-bos}) под 
знаком $\min\nolimits_u$ только члены, зависящие от~$u$, получаем
     \begin{multline*}
     \fr{\partial V_t(y,z)}{\partial t} +\fr{1}{2}\left( \Sigma_t^2(y) \fr{\partial^2 
V_t(y,z)} {\partial y^2}+\sigma_t^2\fr{\partial^2 V_t(y,z)} {\partial 
z^2}\right)+{}\\
     {}+A_t(y)\fr{\partial V_t(y,z)}{\partial y}+\left( a_t y+b_t z\right) 
\fr{\partial V_t(y,z)}{\partial z}+{}\\
{}+S_t\left( s_t y-g_t z\right)^2 +G_t z^2+{}
\end{multline*}

\noindent
\begin{multline*}
     {}+\min\limits_u \left[ \left( c_t \fr{\partial V_t(y,z)}{\partial z}-2S_t \left( 
s_t y-g_t z\right) h_t\right)u +{}\right.\\
\left.{}+\left( S_t h_t^2+H_t\right) u^2
\vphantom{\fr{\partial V_t(y,z)}{\partial z}}
\right]=0\,,
     %\label{e11-bos}
     \end{multline*}
откуда в~предположении $S_t h_t^2\hm+ H_t\hm>0$ следует, что существует 
оптимальное управ\-ле\-ние, которое определяется равенством
\begin{multline}
u_t^* = u_t^*(y,z)=-\fr{1}{2}\left( S_t h_t^2 +H_t\right)^{-1} \left( c_t 
\fr{\partial V_t(y,z)}{\partial z}-{}\right.\\
\left.{}-2S_t\left( s_t y-g_t z\right) h_t
\vphantom{\fr{\partial V_t(y,z)}{\partial z}}
\right)
\label{e12-bos}
\end{multline}
и доставляет минимум соответствующему сла\-га\-емо\-му в~урав\-не\-нии Беллмана, 
равный
$-\left( S_t h_t^2\hm+\right.$\linebreak
$\left.{}+H_t\right)^{-1} \left( c_t 
{\partial V_t(y,z)}/{\partial 
z}\hm-2S_t\left( s_t y \hm-g_t z\right) h_t \right)^2/4.
$ 
     
     Отметим, что, как и~в~классической ли\-ней\-но-квад\-ра\-тич\-ной 
задаче, управ\-ле\-ние из класса до\-пус\-ти\-мых не\-упреж\-да\-ющих получилось 
управ\-ле\-ни\-ем с~обратной связью.
     
     Таким образом, функция Беллмана описывается сле\-ду\-ющим 
дифференциальным уравнением:
     \begin{multline}
     \fr{\partial V_t(y,z)}{\partial t} +\fr{1}{2}\left( \Sigma_t^2(y) \fr{\partial^2 
V_t(y,z)} {\partial y^2}+\sigma_t^2\fr{\partial^2 V_t(y,z)} {\partial 
z^2}\right)+{}\\
     {}+ A_t(y) \fr{\partial V_t(y,z)}{\partial y}+\left( a_t y+b_t z\right) 
\fr{\partial V_t(y,z)}{\partial z}+{}\\
{}+ S_t \left( s_t y- g_t z\right)^2 +G_t z^2-
 \fr{1}{4}\left( S_t h_t^2+H_t\right)^{-1}\times{}\\
 {}\times \left( c_t \fr{\partial V_t(y,z)} 
{\partial z}-2S_t\left( s_t y -g_t z\right) h_t \right)^2=0\,.
     \label{e13-bos}
     \end{multline}
     
     Возводя в~квадрат по\-след\-нее сла\-га\-емое в~(\ref{e13-bos}), перепишем 
его в~виде:
     \begin{multline}
     \fr{\partial V_t(y,z)}{\partial t} +\fr{1}{2}\left( \Sigma_t^2(y) \fr{\partial^2 
V_t(y,z)} {\partial y^2}+\sigma_t^2\fr{\partial^2 V_t(y,z)} {\partial 
z^2}\!\right)+{}\\
{}+A_t(y) \fr{\partial V_t(y,z)}{\partial y}
+ \left( 
\vphantom{\left( S_t h_t^2 +H_t\right)^{-1}}
a_t y+b_t z+{}\right.\\
\left.{}+\left( S_t h_t^2 +H_t\right)^{-1}
 c_t S_t \left( s_t y-g_t z\right) h_t
\right) 
     \fr{\partial V_t(y,z)}{\partial z}+{}\\
     {}+\left( S_t-\left( S_t h_t^2 +H_t\right)^{-1} S_t^2 h_t^2\right)\left( s_t y -
g_t z\right)^2+{}\\
     \!\!{}+
     G_t z^2 -\fr{1}{4}\left( S_t h_t^2+H_t\right)^{-1}\! c_t^2
     \left(\fr{\partial V_t(y,z)}{\partial z}\right)^{\!2}=0\,.\!\!
     \label{e14-bos}
     \end{multline}
     
     Рассматривая полученное уравнение, заметим, что его решение может 
быть пред\-став\-ле\-но в~виде:
   \begin{equation}
     V_t(y,z)= \alpha_t z^2+\beta_t(y) z +\gamma_t(y)\,,
     \label{e15-bos}
     \end{equation}
т.\,е.\ будем искать решение при дополнительном предположении 
о~квад\-ра\-тич\-ности функции Белл\-ма\-на по переменной~$z$, и~сведем, таким 
образом, поиск оптимального решения к~уравнениям относительно функций 
$\alpha_t$, $\beta_t(y)$ и~$\gamma_t(y)$. Отметим сразу, что явный вид 
функции~$\gamma_t(y)$ для реализации оптимального управ\-ле\-ния не 
требуется, однако далее будет предложен вариант вы\-чис\-ле\-ния и~этой 
функции, что пред\-став\-ля\-ет\-ся небесполезным, поскольку позволит выполнять 
расчет минимума целевого функционала. Источником для 
предложения~(\ref{e15-bos}) является уже упоминавшаяся аналогичная 
задача для случая дис\-крет\-но\-го времени~\cite{7-bos, 8-bos}. В~той задаче 
выражение для функции Беллмана получается формально без 
дополнительных усилий. При этом форма~(\ref{e15-bos}) обнаруживается 
как свойство оптимального решения. В~рассматриваемом случае 
непрерывного времени~(\ref{e15-bos}) постулируется, а~пра\-виль\-ность 
постулата под\-тверж\-да\-ет\-ся далее ре\-зуль\-ти\-ру\-ющи\-ми уравнениями 
для~$\alpha_t$, $\beta_t(y)$ и~$\gamma_t(y)$ Кроме того, данное 
предположение пред\-став\-ля\-ет\-ся вы\-те\-ка\-ющим из линейной структуры задачи 
в~отношении переменной~$z$, в~част\-ности, тем фактом, что такой вид 
функции Беллмана обеспечивает линейность оптимального 
управ\-ле\-ния~(\ref{e12-bos}) по~$z$.

     Граничное условие при выбранном предположении~(\ref{e15-bos}) 
принимает вид:

\noindent
     \begin{multline*}
     V_T(y,z)= S_T\left( s_T y- g_T z\right)^2+G_T z^2 ={}\\[-0.5pt]
     {}=\alpha_T z^2 
+\beta_T(y) z +\gamma_T(y)\,,
    \end{multline*}
т.\,е.

\noindent
\begin{align*}
\alpha_T&= S_T g_T^2 +G_T\,;\\[-0.5pt]
\beta_T(y)&=-2S_T s_T g_T y\,;\\[-0.5pt]
\gamma_T(y)&=S_T s_T^2 y^2\,.
%\label{e16-bos}
\end{align*}
          При этом само оптимальное управ\-ле\-ние, определенное 
выражением~(\ref{e12-bos}), оказывается управ\-ле\-ни\-ем с~обратной связью 
по~$y_t$ и~$z_t$:

\noindent
     \begin{multline}
     u_t^*=u_t^*(y,z) ={}\\[-0.5pt]
     {}=
     -\fr{1}{2}\left( S_t h_t^2 +H_t\right)^{-1}
     \left( c_t \left( 2\alpha_t z +\beta_t(y)\right) +{}\right.\\[-0.5pt]
    \left. {}+2S_t\left( s_t y-g_t z\right) 
h_t\right)\,.
     \label{e17-bos}
     \end{multline}
          Подставляем $V_t(y,z)\hm= \alpha_t z^2 \hm+ \beta_t(y) 
z\hm+\gamma_t(y)$ в~(\ref{e14-bos}):

\noindent
     \begin{multline*}
     \fr{\partial \alpha_t}{\partial t}\, z^2 +
     \fr{\partial \beta_t(y)}{\partial t}\,z +
     \fr{\partial \gamma_t(y)}{\partial t}+{}\\[-0.5pt]
     {}+\fr{1}{2}\left( \Sigma_t^2(y) \left( 
\fr{\partial^2\beta_t(y)}{\partial y^2}\,z +\fr{\partial^2 \gamma_t(y)}{\partial 
y^2}\right) +2\sigma_t^2\alpha_t\right)+{}\\[-0.5pt]
 {}+A_t(y)\left(\fr{\partial \beta_t(y)}{\partial y}\,z + \fr{\partial 
\gamma_t(y)}{\partial y}\right) +{}\\[-0.5pt]
\hspace*{-0.22987pt}{}+\left( a_t y+b_t z+\left( S_t h_t^2 +H_t\right)^{-1} c_t S_t \left( s_t y-
g_t z\right) h_t\right)\times{}
\end{multline*}

\noindent
\begin{multline*}
         {}\times \left( 2\alpha_t z+\beta_t(y)\right)+{}\\
     {}+\left( S_t-\left( S_t h_t^2 +H_t\right)^{-1} S_t^2 h_t^2\right)\left( s_t y-
g_t z\right)^2+{}\\
     {}+ G_t z^2 -\fr{1}{4}\left( S_t h_t^2 +H_t\right)^{-1} c_t^2 \left( 
2\alpha_t z+\beta_t(y)\right)^2=0\,.
     \end{multline*}
          Далее выделяем слагаемые при~$z^2$, $z$ и~$z^0$
          
          \noindent
     \begin{multline*}
     \fr{\partial \alpha_t}{\partial t}\, z^2 +\fr{\partial \beta_t(y)}{\partial t}\,z +
     \fr{\partial \gamma_t(y)}{\partial 
t}+\fr{1}{2}\,\Sigma_t^2(y)\fr{\partial^2\beta_t(y)}{\partial y^2}\,z+ {}\\
{}+
\fr{1}{2}\,\Sigma_t^2(y)\fr{\partial^2\gamma_t(y)}{\partial 
y^2}+\sigma_t^2\alpha_t+A_t(y)\fr{\partial \beta_t(y)}{\partial y}\,z +{}\\
{}+A_t(y) \fr{\partial 
\gamma_t(y)}{\partial y}+{}\\
{}+ 2\alpha_t \left( b_t -\left( S_t h_t^2+H_t\right)^{-1} c_t 
S_t h_t g_t \right)z^2+{}\\
     {}+
     \left( 2\alpha_t\left( \alpha_t+\left( S_t h_t^2+H_t\right)^{-1} c_t S_t h_t 
s_t\right)y +{}\right.\\
\left.{}+\beta_t(y) \left( b_t-\left( S_t h_t^2+H_t\right)^{-1} c_t S_t h_t 
g_t\right) \right) z+{}\\
     {}+\beta_t(y)\left( a_t +\left( S_t h_t^2+H_t\right)^{-1} c_t S_t h_t s_t\right) 
y+{}\\
{}+ \left( S_t -\left( S_t h_t^2+H_t\right)^{-1} S_t^2 h_t^2\right) g_t^2 z^2-{}\\
     {}- 2\left( S_t -\left( S_t h_t^2+H_t\right)^{-1} S_t^2 h_t^2\right) s_t g_t yz 
+{}\\
{}+
     \left( S_t-\left( S_t h_t^2+H_t\right)^{-1} S_t^2 h_t^2\right) s_t^2 y^2+{}\\
     {}+G_t z^2 -\left( S_t h_t^2 +H_t\right)^{-1} c_t^2 \alpha_t^2 z^2 -{}\\
     {}-\left( 
S_t h_t^2+H_t\right)^{-1} c_t^2 \alpha_t \beta_t(y) z-{}\\
{}-
\fr{1}{4}\left( S_t h_t^2+H_t\right)^{-1}  c_t^2 \beta_t^2(y)=0\,,
     \end{multline*}
группируем их и~получаем сле\-ду\-ющие уравнения:
\begin{itemize}
\item  для~$\alpha_t$:

\noindent
\begin{multline}
\fr{\partial\alpha_t}{\partial t}+2\alpha_t\left( b_t-\left( S_t h_t^2+H_t\right)^{-1} c_t 
S_t h_t g_t\right)+{}\\
{}+ \left( S_t- \left( S_t h_t^2+H_t\right)^{-1} S_t^2 h_t^2\right) 
g_t^2+G_t-{}\\
\hspace*{-8mm}{}-\left( S_t h_t^2+H_t\right)^{-1} c_t^2 \alpha_t^2 =0\,,\enskip \alpha_T=S_T 
g_t^2+G_T\,;\!\!
\label{e18-bos}
\end{multline}
\item для $\beta_t$:

\noindent
\begin{multline}
\fr{\partial\beta_t(y)}{\partial 
t}+\fr{1}{2}\,\Sigma_t^2(y)\fr{\partial^2\beta_t(y)}{\partial y^2} 
+A_t(y)\fr{\partial \beta_t(y)}{\partial y}+{}\\
{}+ 2\alpha_t\left( a_t +\left( S_t h_t^2+H_t\right)^{-1} c_t S_t h_t s_t\right) y+{}\\
{}+
\beta_t(y)\left( b_t -\left( S_t h_t^2 +H_t\right)^{-1} c_t S_t h_t g_t\right)-{}\\
{}-2\left( S_t-\left( S_t h_t^2+H_t\right)^{-1} S_t^2 h_t^2\right) s_t g_t y-{}
\\
{}-
\left( S_t h_t^2+H_t\right)^{-1} c_t^2 \alpha_t \beta_t(y)=0\,,\\
\beta_T(y)=-2S_T s_T g_T y\,;
\label{e19-bos}
\end{multline}
\item  для $\gamma_t$:
\begin{multline}
\hspace*{-0.8pt}\fr{\partial \gamma_t(y)}{\partial t}+\fr{1}{2}\,\Sigma_t^2(y)
\fr{\partial^2 \gamma_t(y)}{\partial y^2} +\sigma_t^2 \alpha_t +A_t(y)
\fr{\partial \gamma_t(y)}{\partial y}+{}\\
{}+ \beta_t(y)\left( a_t +\left( S_t h_t^2+H_t\right)^{-1} c_t S_t h_t s_t\right) y+{}\\
{}+
\left( S_t-\left( S_t h_t^2+H_t\right)^{-1} S_t^2 h_t^2\right)  s_t^2 y^2-{}\\
{}-\fr{1}{4}\left( S_t h_t^2+H_t\right)^{-1} c_t^2 \beta_t^2(y) =0\,,\\
\gamma_T(y)=S_T s_T^2 y^2\,.
\label{e20-bos}
\end{multline}
\end{itemize}
     
     Уравнение~(\ref{e18-bos}), легко заметить, является уравнением 
Риккати, которое в~силу сформулированного выше условия   
имеет единственное неотрицательное решение для всех $0\hm\leq t\hm\leq T$. 
Этот факт требует дополнительного комментария. Для получения 
уравнения~(\ref{e18-bos}) рас\-смот\-рим исходную задачу при дополнительных 
условиях $a_t\hm=0$ и~$s_t\hm=0$ для всех $0\hm\leq t\hm\leq T$. Нетрудно 
видеть, что эти условия рассматриваемую по\-ста\-нов\-ку сводят фактически 
к~классической ли\-ней\-но-квад\-ра\-тич\-ной задаче. Имеющуюся 
в~рассматриваемой формулировке чуть более общую форму целевой 
функции (принципиального значения это обобщение, конечно, не имеет) 
сведем к~классической еще одним предположением: $S_t\hm=0$ для всех 
$0\hm\leq t\hm\leq T$. Теперь уравнение для~$\alpha_t$ принимает хорошо 
известный вид:
     \begin{equation}
     \fr{\partial \alpha_t}{\partial t}+2\alpha_t b_t +G_t- H_t^{-1} c_t^2 
\alpha_t^2=0\,,\enskip \alpha_T=G_T\,.
     \label{e21-bos}
     \end{equation}

     В таком случае, как известно~\cite{10-bos}, существует единственное 
оптимальное управление~--- линейное с~обратной связью по выходу~$z_t$, 
с~коэффициентом усиления, опи\-сы\-ва\-емым уравнением  
Риккати~(\ref{e21-bos}). Именно этот результат дают  
уравнения~(\ref{e18-bos})--(\ref{e20-bos}) и~описываемая ими функция 
Беллмана~(\ref{e15-bos}), так как из $a_t\hm=0$ и~$s_t\hm=0$ немедленно 
следует, что $\beta_t(y)\hm=0$, откуда, в~свою очередь, с~учетом 
не\-за\-ви\-си\-мости решения от~$y_t$ следует, что $\gamma_t(y)\hm=\gamma_t$, 
т.\,е.\ не зависит от~$y$ и~задается уравнением: 
     $$
     \fr{\partial \gamma_t(y)}{\partial t} +\sigma^2_t \alpha_t=0\,,\enskip 
\gamma_T=0\,.
     $$ 
     Оптимальное управ\-ле\-ние при этом 
     $$
     u_t^*= -H_t^{-1} c_t \alpha_t z_t\,,
     $$
      т.\,е.\ все полностью совпадает с~известным классическим решением.
     
     С уравнениями~(\ref{e19-bos}) и~(\ref{e20-bos}) ситуация, естественно, 
обстоит сложнее. Это линейные уравнения второго порядка параболического 
типа, поскольку\linebreak
 $\Sigma_t^2(y)\hm>0$. Фактически отсутствуют 
конструктивные условия, гарантирующие существование их\linebreak
 решений 
(требовать, чтобы все фигурирующие в~уравнениях коэффициенты были 
представлены аналитическими функциями на всем пространстве значений, 
вряд ли целесообразно), поэтому далее будем предполагать, что данные 
уравнения имеют на рас\-смат\-ри\-ва\-емом интервале $0\hm\leq t\hm\leq T$ хотя 
бы одно ограниченное решение и~именно эти условия будем рас\-смат\-ри\-вать 
как достаточные условия существования оптимального решения 
рассматриваемой задачи.
     
     Таким образом, доказана следующая тео\-рема.
     
     \smallskip
     
     \noindent
     \textbf{Теорема.}\ \textit{Пусть для диффузионного 
процесса}~(\ref{e5-bos}) \textit{выполнены условия Ито, для 
     процесса}~(\ref{e6-bos})~--- \textit{ограничены коэффициенты, 
уравнения}~(\ref{e18-bos})--(\ref{e20-bos}) \textit{имеют ограниченные 
решения для $0\hm\leq t\hm\leq T$. Тогда минимум  
функционалу}~(\ref{e7-bos}) \textit{доставляет оптимальное 
управ\-ле\-ние}~(\ref{e17-bos}), \textit{где} $y\hm= y_t$; $z\hm=z_t$.
     
\section{Заключение}

     Рассмотренная задача оптимизации в~целом близка и~по модели, и~по 
критерию к~классической ли\-ней\-но-квад\-ра\-тич\-ной постановке. 
Принципиальным отличием является нелинейная модель для описания 
со\-сто\-яния динамической сис\-те\-мы, в~которой отсутствует управ\-ля\-ющее 
воздействие.\linebreak
 Такую модель наряду с~традиционной интер\-пре\-тацией  
<<со\-сто\-яние--вы\-ход>> мож\-но понимать как\linebreak модель неконтролируемого 
слож\-но\-го внешнего воздействия. Небольшое дополнительное отличие дает 
предложенная форма квад\-ра\-тич\-но\-го критерия, поз\-во\-ля\-ющая, в~част\-ности, 
ставить такие задачи, как отслеживание выходом или управ\-ле\-ни\-ем со\-сто\-яния 
сис\-те\-мы или ее выхода.
     
     Поскольку обсуждать возможности точного решения уравнений, 
определяющих оптимальное управ\-ле\-ние, не имеет смыс\-ла, наиболее 
актуальной далее является задача их приближенного чис\-лен\-но\-го решения 
и~анализа воз\-мож\-ности практической реализации. Этому посвящена вторая 
часть данной работы, пла\-ни\-ру\-емая к~выходу в~ближайшее время.

{\small\frenchspacing
 {%\baselineskip=10.8pt
 \addcontentsline{toc}{section}{References}
 \begin{thebibliography}{99}
\bibitem{1-bos}
\Au{Athans M.} Editorial on the LQG problem~// IEEE~T. Automat. Contr., 1971. Vol.~16. 
No.\,6. P.~528--552. doi: 10.1109/TAC.1971.1099845.
\bibitem{2-bos}
\Au{Wu Z.} Forward-backward stochastic differential equations, linear quadratic stochastic 
optimal control and nonzero sum differential games~// J.~Syst. Sci. Complex., 2005. Vol.~18. 
No.\,2. P.~179--192.
\bibitem{3-bos}
\Au{Chen B.\,S., Zhang~W.} Stochastic H2/H1 control with state-dependent noise~// IEEE 
T.~Automat. Contr., 2004. Vol.~49. No.\,1. P.~45--56. doi: 10.1109/TAC.2003.821400.
\bibitem{4-bos}
\Au{Bohacek S.} A~stochastic model of TCP and fair video transmission~// IEEE 
INFOCOM, 2003. Vol.~2. P.~1134--1144. doi: 10.1109/INFCOM.2003.1208950.
\bibitem{5-bos}
\Au{Домбровский В.\,В., Объедко~Т.\,Ю.} Управление с~прогнозированием системами 
с~марковскими скачками при ограничениях и~применение к~оптимизации 
инвестиционного портфеля~// Автомат. телемех., 2011. №\,5. С.~96--112. doi: 
10.1134/S0005117911050079.
\bibitem{6-bos}
\Au{Баландин Д.\,В., Коган~М.\,М.} Оптимальное линейно-квад\-ра\-тич\-ное управление: от 
матричных уравнений к~линейным матричным неравенствам~// Автомат. телемех., 2011. 
№\,11. С.~60--69. doi: 10.1134/ S0005117911110038.
\bibitem{7-bos}
\Au{Босов А.\,В.} Обобщенная задача распределения ресурсов программной системы~// 
Информатика и~её применения, 2014. Т.~8. Вып.~2. С.~39--47. doi: 
10.14357/19922264140204.
\bibitem{8-bos}
\Au{Босов А.\,В.} Управление линейным выходом дискретной стохастической системы по 
квадратичному критерию~// Изв. РАН. Теория и~системы управления, 2016. №\,3.  
С.~19--35. doi: 10.1134/S1064230716030060.
\bibitem{9-bos}
\Au{Флеминг У., Ришел~Р.} Оптимальное управление детерминированными 
и~стохастическими системами~/ Пер. с~англ.~--- М.: Мир, 1978. 316~с. 
(\Au{Fleming~W.\,H., Rishel~R.\,W.} Deterministic and stochastic optimal control.~--- New 
York, NY, USA: Springer-Verlag, 1975. 222~p.)
\bibitem{10-bos}
\Au{Девис М.\,Х.\,А.} Линейное оценивание и~стохастическое управление~/ Пер. с~англ.~--- 
М.: Наука, 1984. 206~с. (\Au{Davis~M.\,H.\,A.} Linear estimation and stochastic control.~--- 
London: Chapman and Hall, 1977. 224~p.)

 \end{thebibliography}

 }
 }

\end{multicols}

\vspace*{-6pt}

\hfill{\small\textit{Поступила в~редакцию 30.03.18}}

\vspace*{4pt}

%\newpage

%\vspace*{-24pt}

\hrule

\vspace*{2pt}

\hrule

\vspace*{-2pt}


\def\tit{STOCHASTIC DIFFERENTIAL SYSTEM OUTPUT CONTROL 
BY~THE~QUADRATIC CRITERION.~I.~DYNAMIC\\ PROGRAMMING 
OPTIMAL SOLUTION}


\def\titkol{Stochastic differential system output control 
by~the~quadratic criterion. I.~Dynamic programming 
optimal solution}

\def\aut{A.\,V.~Bosov and~A.\,I.~Stefanovich}

\def\autkol{A.\,V.~Bosov and~A.\,I.~Stefanovich}

\titel{\tit}{\aut}{\autkol}{\titkol}

\vspace*{-11pt}


\noindent
Institute of Informatics Problems, Federal Research Center ``Computer Science 
and Control'' of the Russian Academy of Sciences, 44-2~Vavilov Str., Moscow 
119333, Russian Federation


\def\leftfootline{\small{\textbf{\thepage}
\hfill INFORMATIKA I EE PRIMENENIYA~--- INFORMATICS AND
APPLICATIONS\ \ \ 2018\ \ \ volume~12\ \ \ issue\ 3}
}%
 \def\rightfootline{\small{INFORMATIKA I EE PRIMENENIYA~---
INFORMATICS AND APPLICATIONS\ \ \ 2018\ \ \ volume~12\ \ \ issue\ 3
\hfill \textbf{\thepage}}}

\vspace*{3pt}



\Abste{The problem of optimal control for the Ito diffusion 
process and a~controlled linear output is solved. The considered 
statement is close to the classical linear-quadratic Gaussian 
control  (LQG control) problem. Differences consist in the fact 
that the state is described by the nonlinear differential Ito equation  $dy_y = A_t(y_t) 
\,dt+\Sigma_t(y_t)\,dv_t$ and does not depend on the control~$u_t$, 
optimization subject is controlled linear output 
 $dz_t=a_ty_t\,dt +b_tz_t\,dt +c_t u_t\,dt +\sigma_t \,dw_t$. 
Additional generalizations are included in the quadratic 
quality criterion for the purpose of statement such problems 
as state tracking by output or a linear combination of state 
and output tracking by control. The method of dynamic programming 
is used for the solution. 
The assumption about Bellman function in the form  $V_t(y,z)= \alpha_t 
z^2+\beta_t(y) z+\gamma_t(y)$ allows one to find it. 
Three differential equations for the coefficients $\alpha_t$,  $\beta_t(y)$,
and $\gamma_t(y)$ give the solution. 
These equations constitute the optimal solution of the problem under consideration.}

\KWE{stochastic differential equation; optimal control; dynamic programming; 
Bellman function; Riccati equation; linear differential equations of parabolic type}


\DOI{10.14357/19922264180314}

\vspace*{-12pt}

\Ack
\noindent
This work was partially supported by the Russian Science Foundation (grant  
16-07-00677).



%\vspace*{6pt}

  \begin{multicols}{2}

\renewcommand{\bibname}{\protect\rmfamily References}
%\renewcommand{\bibname}{\large\protect\rm References}

{\small\frenchspacing
 {%\baselineskip=10.8pt
 \addcontentsline{toc}{section}{References}
 \begin{thebibliography}{99}
\bibitem{1-bos-1}
\Aue{Athans, M.} 1971. Editorial on the LQG problem. \textit{IEEE~T. 
Automat. Contr.} 16(6):528--552. doi: 10.1109/ TAC.1971.1099845.
\bibitem{2-bos-1}
\Aue{Wu, Z.} 2005. Forward-backward stochastic differential equations, linear 
quadratic stochastic optimal control and\linebreak\vspace*{-12pt}

\columnbreak

\noindent
 nonzero sum differential games. 
\textit{J.~Syst. Sci. Complex.} 18(2):179--192.
\bibitem{3-bos-1}
\Aue{Chen, B.\,S. and W.~Zhang.} 2004. Stochastic H2/H1 control with  
state-dependent noise. \textit{IEEE~T. Automat. Contr.} 49(1):45--56.
doi: 10.1109/TAC.2003.821400.
\bibitem{4-bos-1}
\Aue{Bohacek, S.} 2003. A~stochastic model of TCP and fair video 
transmission. \textit{IEEE INFOCOM}. 2:1134--1144.
doi: 10.1109/INFCOM.2003.1208950.
\bibitem{5-bos-1}
\Aue{Dombrovskii, V.\,V., and T.\,Yu.~Ob''edko.} 2011. Predictive control of 
systems with Markovian jumps under constraints and its application to the 
investment portfolio optimization. \textit{Automat. Rem. Contr.}  
72(5):989--1003.
\bibitem{6-bos-1}
\Aue{Balandin, D.\,V., and M.\,M.~Kogan.} 2011. Optimal linear-quadratic 
control: From matrix equations to linear matrix inequalities. \textit{Automat. 
Rem. Contr.} 72(11):2276--2284.
\bibitem{7-bos-1}
\Aue{Bosov, A.\,V.} 2014. Obobshchennaya zadacha raspredeleniya resursov 
programmnoy sistemy [The generalized problem of software system resources 
distribution]. \textit{Informatika i~ee Primeneniya~--- Inform. Appl.}  
8(2):39--47. doi: 
10.14357/19922264140204.
\bibitem{8-bos-1}
\Aue{Bosov, A.\,V.} 2016. Discrete stochastic system linear output control 
with respect to a quadratic criterion. \textit{J.~Comput. Syst. Sc. 
Int.} 55(3):349--364.
\bibitem{9-bos-1}
\Aue{Fleming, W.\,H., and R.\,W.~Rishel.} 1975. \textit{Deterministic and 
stochastic optimal control.} New York, NY: Springer-Verlag. 222~p.
\bibitem{10-bos-1}
\Aue{Davis, M.\,H.\,A.} 1977. \textit{Linear estimation and stochastic 
control.} London: Chapman and Hall. 224~p.
\end{thebibliography}

 }
 }

\end{multicols}

\vspace*{-6pt}

\hfill{\small\textit{Received March 30, 2018}}

%\pagebreak

%\vspace*{-18pt}
     
     \Contr
     
       \noindent
       \textbf{Bosov Alexey V.} (b.\ 1969)~--- Doctor of Science in technology, 
principal scientist, Institute of Informatics Problems, Federal Research 
Center ``Computer Science and Control'' of the Russian Academy of Sciences, 
44-2~Vavilov Str., Moscow 119333, Russian Federation; 
\mbox{AVBosov@ipiran.ru}
       
       \vspace*{3pt}
       
       \noindent
       \textbf{Stefanovich Alexey I.} (b.\ 1983)~--- principal specialist, 
Institute of Informatics Problems, Federal Research Center ``Computer Science 
and Control'' of the Russian Academy of Sciences, 44-2~Vavilov Str., Moscow 
119333, Russian Federation; \mbox{AStefanovich@frccsc.ru}
\label{end\stat}

\renewcommand{\bibname}{\protect\rm Литература}       

        %10
\def\stat{zatsman}

\def\tit{ТРАНСФОРМАЦИИ ОБЪЕКТОВ ПЕРВОГО И~ВТОРОГО ПОРЯДКА 
В~ЛЕКСИКОГРАФИЧЕСКОЙ ИНФОРМАЦИОННОЙ СИСТЕМЕ$^*$}

\def\titkol{Трансформации объектов первого и~второго порядка 
в~лексикографической информационной системе}

\def\aut{И.\,М.~Зацман$^1$}

\def\autkol{И.\,М.~Зацман}

\titel{\tit}{\aut}{\autkol}{\titkol}

\index{Зацман И.\,М.}
\index{Zatsman I.\,M.}


{\renewcommand{\thefootnote}{\fnsymbol{footnote}} \footnotetext[1]
{Исследование выполнено в~ФИЦ ИУ РАН за счет гранта Российского научного фонда №\,24-18-00155, {\sf 
https://rscf.ru/project/24-18-00155}. Работа выполнялась с~использованием инфраструктуры Центра 
коллективного пользования <<Высокопроизводительные вычисления и~большие данные>> (ЦКП 
<<Информатика>>) ФИЦ ИУ РАН (г.\ Москва).}}


\renewcommand{\thefootnote}{\arabic{footnote}}
\footnotetext[1]{ Федеральный исследовательский центр <<Информатика и~управление>> Российской академии наук, 
\mbox{izatsman@yandex.ru}}

\vspace*{-12pt}


  
  \Abst{Рассматриваются теоретические основания проектирования информационных 
технологий (ИТ) интеграции двуязычных словарей и~параллельных корпусов. Дано описание 
первых результатов создания третьего уровня классификации трансформаций объектов 
предметной области информатики, которую предполагается использовать при создании 
концепции лексикографической информационной системы, обеспечивающей интеграцию. 
Все сущности информатики в~статье разделены на два глобальных класса: объекты и~их 
трансформации. Для каждого такого класса конструируется своя классификация. Ранее были 
описаны два верхних уровня классификации трансформаций объектов предметной области. 
В~данной статье рассматривается третий уровень этой классификации. Основанием для 
построения самого верхнего ее уровня служило деление предметной области информатики 
на среды (ментальная, сенсорно воспринимаемая, цифровая и~ряд других сред), каждая из 
которых по определению включает объекты одной природы. Основанием для построения 
второго уровня классификации трансформаций объектов служила типология знаковых  
сис\-тем А.~Соломоника. Цель статьи состоит в~систематизации трансформаций первого 
и~второго порядка объектов предметной области на третьем уровне этой классификации. 
Основанием для систематизации служит средовая версия иерархии Акоффа.}
  
  \KW{объекты предметной области; трансформации объектов; классификация; данные; 
информация; знание; лексикографическая информационная сис\-тема}

\DOI{10.14357/19922264240211}{VZTGVV}
  
\vspace*{3pt}


\vskip 10pt plus 9pt minus 6pt

\thispagestyle{headings}

\begin{multicols}{2}

\label{st\stat}
  
\section{Введение}

\vspace*{-9pt}

  Возникновение параллельных корпусов, в~которых предложениям 
оригинального текста со\-по\-став\-ле\-ны предложения его перевода, обеспечило 
возможность контрастивного лингвистического\linebreak \mbox{анализа} на принципиально 
новом уровне полноты и~точности, недостижимом в~докорпусную эпоху. 
Пионерскими в~этой области стали работы \mbox{1990-х~гг}. Стига Йоханссона  
с~анг\-ло-нор\-веж\-ским корпусом~[1]. В России параллельные корпусы стали 
формироваться в~начале XXI~века в~рамках Национального корпуса русского 
языка~[2].
  
  Создатели двуязычных словарей используют параллельные корпусы для 
сбора материала и~эмпирической проверки своих гипотез, касающихся 
межъязы\-ко\-вой эквивалентности. Ценность параллельных корпусов 
определяется тем, что в~лингвистике этап сбора исходного материала считается 
наиболее трудоемким и~наименее творческим, а~параллельные корпусы 
позволяют значительно сэкономить время и~силы для творческого этапа 
создания словарей~[3].
 % 
  При этом двуязычные словари, создаваемые на основе исходного материала, 
извлеченного из параллельных корпусов, сейчас формируются без связей с~их 
текстами. Другими словами, онлайновые связи созданных словарей 
с~параллельными корпусами, которые служили источниками исходного 
материала, отсутствуют. 

Параллельные корпусы постоянно пополняются 
новыми текстами, в~предложениях которых можно обнаружить новые значения 
слов и~устойчивых словосочетаний. Однако при этом отсутствуют методы 
и~средства оперативного обновления словарей по корпусным данным. 
В~настоящее время проблема установления связей между двуязычными 
словарями и~параллельными корпусами (далее~--- проблема интеграции) 
находится на стадии поиска концептуальных подходов к~их интеграции на 
уровне значений.
  
  Подход к~решению проблемы интеграции, предлагаемый в~статье, учитывает 
  и~появление новых значений слов и~устойчивых словосочетаний, и~динамику 
смысловых значений, которая обусловлена развитием и~пополнением знания 
лингвистов, фиксирующих эти значения в~результате семантического анализа 
пополняемых корпусных данных. Проведенные эксперименты показали, что 
обнаружение нового лингвистического знания обусловливает и~формирование 
дефиниций новых значений, и~пересмотр уже существующих дефиниций~[4, 5].
  
  Например, в~проведенных экспериментах с~использованием ЦКП 
<<Информатика>> ФИЦ ИУ РАН фиксировалась эволюция значений немецких 
модальных глаголов, исходное состояние значений которых было описано 
в~не\-мец\-ко-рус\-ском словаре. В~экспериментальном массиве текстов как 
потенциальных источниках нового знания 16\,268 предложений содержали 
немецкие модальные глаголы и~в~2041 из них встречался глагол sollen. 
В~начале эксперимента в~словаре были описаны~12~значений этого модального 
глагола. По окончании эксперимента лингвисты обнаружили два новых его 
значения, согласовали их дефиниции и~описали эволюцию дефиниций~[6, 7].
  
  Таким образом, для решения проблемы интеграции требуется фиксировать 
новое знание, обнаруженное лингвистами в~текстовых данных параллельных 
корпусов, отслеживать эволюцию знания, представленного в~виде дефиниций 
значений слов и~устойчивых словосочетаний, и,~соответственно, 
актуализировать электронные двуязычные словари. Предлагаемый 
концептуальный подход к~интеграции, который планируется реализовать 
в~процессе проектирования лексикографической информационной сис\-те\-мы, 
фиксирующей эволюцию лингвистического знания, основан на решении 
следующих задач:\\[-14pt]
  \begin{itemize}
  \item категоризация трех базовых понятий информатики, включенных 
  в~иерархию Акоффа~[8] (данные, информация, знание), на объекты 
проектируемой сис\-те\-мы, которая необходима, чтобы фиксировать 
<<кванты>> нового знания и~отслеживать его эволюцию в~этой сис\-теме;\\[-15pt]
  \item  систематизация трансформаций объектов этой сис\-темы.\\[-14pt]
  \end{itemize}
  
  Цель статьи и~состоит в~решении двух задач: категоризации трех базовых 
понятий информатики на объекты лексикографической информационной  
сис\-те\-мы и~сис\-те\-ма\-ти\-за\-ции трансформаций первого и~второго порядка 
ее объектов.
  
  Трансформациями первого порядка, о которых сказано в~формулировке цели 
статьи, называются взаимные преобразования между двумя объектами  
сис\-те\-мы одной природы. Например, перевод в~сис\-те\-ме текста с~русского 
языка на английский относится к~ним. Трансформациями второго порядка 
и~выше называются взаимные преобразования между двумя и~более объектами 
разной природы. Например, кодирование символов текс\-та компьютерными 
кодами и~их декодирование относятся по определению к~трансформациям 
второго порядка.

%\vspace*{-9pt}
  
\section{Процессы трансформаций в~информатике}

%\vspace*{-3pt}

Процессы трансформаций, рассматриваемые в~статье, относятся к~теоретическому ядру информатики, а~не 
только к~проектированию лексикографической информационной сис\-те\-мы. Например, из трех основных 
подходов к~описанию предметной об\-ласти информатики\footnote{В статье предметная область информатики 
трактуется согласно концепции полиадического компьютинга Пола Розенблума~\cite{9-zac}.} (объектный, 
трансформационный и~синтетический) сис\-те\-ма\-ти\-за\-ция трансформаций ближе всего ко второму 
подходу. Примерами первого подхода, в~рамках которого основное внимание уделяется объектам предметной 
области информатики и~в~меньшей степени отношениям\linebreak между ними, могут служить  
работы~\cite{8-zac, 10-zac, 11-zac}; \mbox{примерами} второго подхода, в~рамках которого основное внимание 
уделяется трансформациям и~в~меньшей степени трансформируемым объектам,~---  
работы~\cite{12-zac, 13-zac}; примерами третьего, синтетического подхода, в~котором уделяется внимание 
и~объектам предметной об\-ласти информатики, и~отношениям между ними, могут служить работы~\cite{14-zac, 
15-zac, 16-zac, 17-zac, 18-zac}.

  Таким образом, для описания трансформаций объектов лексикографической 
информационной\linebreak системы предпочтительнее всего трансформационный 
подход, который упоминается и~в определениях информатики. Например, 
в~2009~г.\ П.~Деннинг и~П.~Розенблум сформулировали суть \mbox{информатики} как 
компьютинга следующим образом: <<$\ldots$информатика~--- это не просто 
алгоритмы и~структуры данных; это преобразования [трансформации] 
представлений>>~\cite{12-zac}. Чуть позже, в~контексте краткого описания 
парадигмы информатики как компьютинга, П.~Деннинг и~П.~Фриман изменили 
эту формулировку на такую: <<Центральный объект внимания в~информатике 
можно определить как информационные процессы~--- \textit{естественные или 
искусственные процессы, преобразующие информацию} (курсив мой~--- 
И.\,З.)>>~\cite{13-zac}. Согласно парадигме, предлагаемой авторами этой 
статьи, на начальном этапе проектирования автоматизированных систем 
базовыми элементами моделей их функционирования служат 
\textit{информационные про\-цессы}.
  
  Однако если 15~лет назад в~формулировке из работы~\cite{13-zac} шла речь 
о~процессах, преобразующих информацию, то в~последние~10~лет в~спектр 
процессов трансформаций все чаще стали включать процессы, преобразующие 
не только информацию, но также и~другие объекты автоматизированных 
систем, в~первую очередь данные и~знания~[19--21]. Например, Виктория 
Стодден, позиционируя науку о~данных как одну из дисциплин информатики, 
говорит, что центральный объект исследований в~науке о~данных~--- это 
<<изучение обобщаемого извлечения знания из данных>>~\cite{21-zac}. 
Увеличение и~чис\-ла объектов, и~спект\-ра процессов их трансформаций 
в~автоматизированных сис\-те\-мах обуслов\-ли\-ва\-ет не\-об\-хо\-ди\-мость 
систематизации и~объектов, и~процессов их трансформаций на начальном этапе 
проектирования сис\-тем.
  
  Для создания концепции лексикографической информационной сис\-те\-мы 
и~проектирования ИТ, обеспечивающих интеграцию 
двуязычных словарей и~параллельных корпусов, сначала выполним 
категоризацию на объекты этой сис\-те\-мы трех базовых понятий информатики 
(данные, информация, знание) в~контексте построения классификаций 
сущностей ее предметной об\-ласти.
  
  Необходимость использования классификаций информатики в~процессе 
создания концепции проиллюстрируем, используя иерархию  
Акоффа~\cite{8-zac}. Он использовал принцип их вертикального размещения 
в~иерархии снизу вверх: данные, информация и~знание. Еще в~ней есть термин 
<<мудрость>>, который в~статье не рассматривается. Такое размещение Акофф 
прокомментировал так: <<Каждое из пе\-ре\-чис\-лен\-ных понятий [кроме данных] 
содержит в~себе нижестоящие$\ldots$>>~\cite{8-zac}.
  
  Этому принципу размещения и~комментарию Акоффа свойственны 
недостатки, проанализированные, в~частности, в~работе~\cite{10-zac}. Главный 
вывод, к~которому пришла Роули после изучения иерархии Акоффа, 
заключается в~следующем: <<$\ldots$информация определяется в~терминах 
данных, знание~--- в~терминах информации$\ldots$ но существует меньше 
консенсуса в~описании трансформаций, которые преобразуют сущности, 
расположенные ниже в~иерархии, в~те, которые находятся над ними, что 
приводит к~их терминологической неопределенности>>~\cite{10-zac}. Причина 
этой неопределенности, скорее всего, в~том, что базовые понятия информатики 
включены в~иерархию Акоффа изолированно от общего контекста 
классификаций сущностей ее предметной об\-ласти.

%\vspace*{-9pt}
  
\section{Классификации сущностей информатики}


%\vspace*{-2pt}

  Все сущности предметной области информатики в~работах~[22, 23] 
разделены на два глобальных класса: ее объекты и~их трансформации. Для 
каждого такого класса была предложена своя классификация. 
В~работе~\cite{22-zac} дано описание классификации объектов предметной 
области информатики, первый уровень которой содержит базовые понятия ее 
предметной области (данные, информация, знания и~др.).  
В~работе~\cite{23-zac} дано описание двух верхних уровней классификации 
трансформаций объектов предметной об\-ласти (см.\ рисунок 
в~работе~\cite{23-zac}). Основанием для построения самого верхнего ее уровня послужило деление 
предметной области информатики на среды\footnote{В~работе~\cite{24-zac} дано описание пяти сред 
предметной области информатики (ментальная; сенсорно воспринимаемая, или информационная; 
цифровая; нейро- и~ДНК-среда), каждая из которых по определению включает объекты одной и~той же 
природы.} и~степень разнообразия природы объектов, вовлеченных в~трансформации:
\begin{itemize}
\item  первый класс верхнего уровня классификации включает 
трансформации объектов в~пределах среды только одной природы 
(трансформации первого порядка);
\item  второй класс включает трансформации объектов, относящихся 
к~двум средам разной природы (трансформации второго порядка);
\item третий и~последующие классы включают трансформации объектов, 
относящихся к~трем и~более средам разной природы (трансформации 
третьего и~более высоких порядков).
\end{itemize}

  В работе~\cite{23-zac} были приведены примеры для трех первых классов 
трансформаций, включая пример трансформаций объектов, относящихся 
к~двум средам разной природы (компьютерное кодирование символов текстов 
с~по\-мощью таб\-лиц Unicode).
  
Основанием для построения второго уровня классификации трансформаций объектов послужила типология 
знаковых сис\-тем А.~Соломоника~\cite[c.~131]{25-zac}: естественные знаковые сис\-те\-мы, образные,  
ес\-тест\-вен\-но-язы\-ко\-в$\acute{\mbox{ы}}$е,  
вер\-баль\-но-не\-сло\-вес\-ные сис\-те\-мы записи\footnote{Под системой записи понимается знаковая 
система, сочетающая вербальные знаки с~несловесными (языки нотной записи, карт, таблиц и~др.).} 
и~формализованные знаковые сис\-те\-мы, включая математические. Введем понятие обобщенного текста~--- 
это текст, который может быть создан в~любой из перечисленных знаковых систем. Тогда обобщенные тексты 
могут быть естественными, образными, ес\-тест\-вен\-но-язы\-ко\-в$\acute{\mbox{ы}}$\-ми,  
вер\-баль\-но-не\-сло\-вес\-ны\-ми и~формализованными. Второй уровень классификации трансформаций 
охватывает не все виды объектов предметной  
об\-ласти информатики, а~только перечисленные~5~видов текс\-тов и~их представления, вовлеченные 
в~процессы трансформаций в~одной или более средах вместе с~данными, знанием и~его концептами.

\begin{figure*}[b] %fig1
\vspace*{6pt}
      \begin{center}
     \mbox{%
\epsfxsize=121.191mm 
\epsfbox{zac-1.eps}
}
\end{center}
\vspace*{-6pt}
\Caption{Средовая версия иерархии Акоффа}
\end{figure*}

\section{Классификация трансформаций: построение~третьего 
уровня}

  Основанием для систематизации трансформаций первого и~второго порядка 
на третьем уровне этой классификации служит иерархия Акоффа~\cite{8-zac}, 
на основе которой и~была создана ее средов$\acute{\mbox{а}}$я версия~[26, 
27]. Для создания средов$\acute{\mbox{о}}$й версии была выполнена 
категоризация трех базовых понятий информатики (данные, информация, 
знания) на объекты лексикографической информационной сис\-те\-мы 
в~процессе создания ее концепции\linebreak (рис.~1).
  


  В отличие от классической иерархии Акоффа, в~ее 
средов$\acute{\mbox{о}}$й версии различаются три вида данных: сенсорно 
воспринимаемые, цифровые и~те данные, которые генерируются 
искусственными нейронными сетями (ИНС) в~системах искусственного интеллекта 
(далее~--- ИИ-дан\-ные). Последний вид данных необходим, например, для 
различения входа и~выхода процесса применения обученной 
ИНС в~цифровой модели генерации знания, описанию которой 
посвящена работа~\cite{27-zac}.
  
  Также предлагается различать два вида информации: сенсорно 
воспринимаемая и~цифровая. Кроме знания в~средов$\acute{\mbox{у}}$ю 
версию добавлены концепты и~ментальные образы сенсорно воспринимаемых 
данных. Последние служат промежуточной сущностью между сенсорно 
воспринимаемыми данными и~генерируемым знанием при описании процессов 
извлечения знания из текстовых данных лексикографической информационной 
системы. Описание объектов средов$\acute{\mbox{о}}$й версии иерархии 
Акоффа (см.\ рис.~1) и~отношений между ними дано в~работах~\cite{26-zac, 28-zac}.
  
  В средов$\acute{\mbox{о}}$й версии число объектов равно восьми. Если 
учитывать направления трансформаций, то между восемью объектами на 
рис.~1 она включает~16 их видов (трансформации на границе между сенсорно 
воспринимаемыми данными и~информацией, обозначенные символом~<<?>>, 
в~статье не рас\-смат\-ри\-ва\-ют\-ся). В~будущем число объектов 
в~средов$\acute{\mbox{о}}$й версии, которая выбрана как основание для 
сис\-те\-ма\-ти\-за\-ции трансформаций первого и~второго порядка, может быть 
увеличено. Для построения классификации трансформаций 
важ\-но не возможное увеличение числа объектов 
и~трансформаций между ними, а то, что их виды в~средов$\acute{\mbox{о}}$й 
версии распределены между трансформациями первого и~второго порядка. Из 
16~видов на рис.~1 шесть относятся к~трансформациям первого порядка, это\linebreak 
виды с~номерами~7, 8, 13--16 (далее~--- типология трансформаций первого 
порядка), а~десять~--- к~трансформациям второго порядка, это виды 
с~\mbox{номерами}~1--6 и~9--12 (далее~--- типология трансформаций второго 
порядка). Разместим обе типологии на третьем уровне классификации (см.\ ее 
схему на рис.~2). Перечислим виды трансформаций первой типологии, вводя 
в~скобках их краткие названия, используемые ниже на рис.~3:
  \begin{description}
  \item[\,] 7~--- членение знания на концепты с~помощью одной или нескольких 
знаковых систем (далее~--- членение знания);
  \item[\,] 8~--- формирование знания на основе концептов (формирование 
знания);
  \item[\,] 13~--- обучение ИНС;
  \end{description}
  
  \vspace*{-6pt}
  
  \pagebreak
  
  \end{multicols}
  
  \begin{figure*} %fig2
\vspace*{1pt}
      \begin{center}
     \mbox{%
\epsfxsize=127.513mm 
\epsfbox{zac-2.eps}
}
\end{center}
\vspace*{-9pt}
\Caption{Схема трех верхних уровней классификации трансформаций объектов (объединены 
по три слоя и~для второго, и~для третьего уровней этой классификации)}
\end{figure*}
  
  \begin{multicols}{2}
  
  \noindent
  \begin{description}
  \item[\,] 14~--- восстановление обучающей информации на основе 
содержания обученной ИНС (обращение ИНС);
  \item[\,] 15~--- использование обученной ИНС (использование ИНС);



  \item[\,] 16~--- восстановление исходных данных, соответствующих 
полученным результатам работы обучен\-ной ИНС (восстановление исходных данных 
по результатам ИНС).
  \end{description}
  
  
  Не все виды трансформаций 13--16 поддерживаются в~конкретных системах 
искусственного интеллекта, но с~теоретической точки зрения все их 
предлагается включить в~первую типологию для полноты спектра видов 
трансформаций.
  
  Перечислим виды трансформаций второй типологии:
  \begin{description}
  \item[\,] 1~--- декодирование цифровых данных в~компьютерных системах 
(декодирование данных);
  \item[\,]  2~--- кодирование сенсорно воспринимаемых данных (кодирование 
данных);
  \item[\,] 3~--- ментальное копирование сенсорно воспринимаемых данных 
(ментальное копирование);
  \item[\,] 4~--- восстановление сенсорно воспринимаемых данных по 
ментальным образам (восстановление по образам);
  \item[\,] 5~--- смысловая интерпретация без деления на концепты ментальных 
образов сенсорно воспринимаемых данных (смысловая интерпретация);
  \item[\,] 6~--- восстановление ментальных образов (восстановление образов);
  \item[\,] 9~--- представление концептов в~виде сенсорно воспринимаемой 
информации, например текс\-та\-ми, формулами, таблицами, рисунками и~т.\,д.\ 
(представление концептов);
  \item[\,] 10~--- понимание смысла сенсорно воспринимаемой информации 
(понимание смысла);
  \item[\,] 11~--- кодирование сенсорно воспринимаемой информации 
(кодирование информации);
\end{description}

\vspace*{-6pt}

\pagebreak

\end{multicols}

\begin{figure*} %fig3
\vspace*{1pt}
      \begin{center}
     \mbox{%
\epsfxsize=163mm 
\epsfbox{zac-3.eps}
}
\end{center}
\vspace*{-9pt}
\Caption{Схема частного случая классификации трансформаций объектов (трансформации 
пронумерованы согласно рис.~1)}
\end{figure*}

\begin{multicols}{2}

\noindent
\begin{description}

  \item[\,] 12~--- декодирование цифровой информации (декодирование 
информации).
  \end{description}
  
  Отметим, что в~существующих ИТ
  и~компьютерных системах наиболее часто используются виды 
трансформаций~13 и~15 типологии первого порядка и~1, 2, 11 и~12 типологии 
второго порядка. На рис.~2 в~первом слое третьего уровня классификации 
показаны типологии первого порядка без указания числа трансформаций в~них 
и~без детализации трансформируемых объектов.
  
  Во втором слое третьего уровня классификации условно (без названий) 
показаны типологии второго порядка. Также на рис.~2 в~третьем слое третьего 
уровня классификации условно (также без названий) показаны типологии 
третьего порядка, которые планируется рассмотреть в~отдельной статье. По 
определению они должны включать трансформации между тремя объектами 
разной природы, но средов$\acute{\mbox{а}}$я версия иерархии Акоффа 
включает трансформации только между двумя объектами разной природы. 
Поэтому потребуется другое основание для их систематизации (ранее были 
рассмотрены отдельные примеры трансформаций третьего 
порядка\footnote{Далеко не всегда трансформации третьего и~более высоких порядков можно 
рассматривать как последовательность трансформаций второго порядка. Примером этого могут 
служить трансформации в~процессе обучения пациента пользованию роботизированной рукой, 
охватывающие личностные концепты пациента, релевантные его намерениям, сигналы активности 
мозга как объекты нейросреды и~компьютерные коды~\cite{29-zac}.}~\cite{29-zac}).

\section{Классификация трансформаций: частный~случай}

  Выше было отмечено, что в~будущем число объектов 
в~средов$\acute{\mbox{о}}$й версии иерархии Акоффа может быть увеличено. 
Это означает, что увеличатся и~чис\-ло объектов, и~чис\-ло трансформаций между 
ними в~классификации трансформаций, так как эта средов$\acute{\mbox{а}}$я 
версия служит по определению основанием для систематизации 
трансформаций первого и~второго порядка. Поэтому на третьем уровне рис.~2 
указаны типологии без детализации объектов и~без указания числа 
трансформаций в~каждой из них. С~одной стороны, при таком подходе 
получаем достаточно общий вид этой классификации, так как она не зависит от 
числа объектов в~том или ином варианте средов$\acute{\mbox{о}}$й версии 
(и~это существенно упрощает рис.~2). С~другой стороны, на третьем уровне 
такой общей классификации подразумевается, но не эксплицируется природа 
трансформируемых объектов и~их возможные сочетания в~трансформациях. 

При проектировании лексикографической информационной системы важно 
эксплицировать природу трансформируемых объектов и~их возможные 
сочетания.
  %
  Поэтому в~парадигму информатики~\cite{30-zac} кроме общей 
классификации трансформаций предлагается включать и~ее частные случаи, 
эксплицирующие природу трансформируемых объектов. 

В~этом разделе 
рассмотрим один частный случай, когда используются только естественные 
знаковые сис\-те\-мы из типологии А.~Соломоника~\cite{25-zac} вместе 
с~данными, знанием и~его концептами. Чис\-ло естественных языков при этом не 
ограничено. И~этот частный случай классификации включает только три 
класса природных трансформаций (первого, второго и~третьего порядка, см.\ 
схему классификации на рис.~3).
  
  Первый и~второй уровни схемы общей классификации (см.\ рис.~2) можно 
объединить в~один уровень в~этом частном случае. Ниже этого уровня 
приведено содержание типологий первого и~второго порядка без содержания 
типологий третьего по\-рядка.




  Наполнение типологий первого и~второго порядка соответствует 
средов$\acute{\mbox{о}}$й версии иерархии Акоффа на рис.~1, содержащей 
6~видов трансформаций типологии первого порядка и~10~видов 
трансформаций типологии второго порядка (на рис.~3 стрелки указывают 
направления трансформаций согласно средов$\acute{\mbox{о}}$й версии на рис.~1).
  
  Таким образом, частный случай классификации содержит для этих двух 
типологий 16~теоретически возможных трансформаций, 6 из которых 
в~настоящее время в~существующих ИТ применяются наиболее часто: виды 
трансформаций~1, 2, 11 и~12 типологии второго порядка реализуются 
с~помощью тех или иных методов ко\-ди\-ро\-ва\-ния/де\-ко\-ди\-ро\-ва\-ния 
(например, с~использованием таблиц Unicode), а~виды трансформаций~13 и~15
 в~типологии первого порядка реализуются полностью с~по\-мощью процессов 
цифровой обработки компьютерами.
  
  Остальные виды трансформаций или применяются намного реже (это 
виды~3, 5, 7, 9 и~10), или находятся в~стадии поиска и~разработки (14 и~16) или 
в~настоящее время носят только теоретический характер, обеспечивая полноту 
первой и~второй типологий (4, 6 и~8). Знаком~<<?>> обозначены те виды 
трансформаций, которые по определению не существуют в~используемой 
парадигме информатики~\cite{30-zac}. Однако возможно, что в~других 
будущих подходах к~построению ее парадигмы эти виды трансформаций будут 
существовать.
  
\section{Заключение}

  На сегодняшний день процесс построения классификаций объектов 
предметной области информатики~\cite{22-zac} и~их  
трансформаций~\cite{23-zac} еще не завершен. Однако первые результаты их 
построения уже используются для создания концепции лексикографической 
информационной сис\-те\-мы, обеспечивающей интеграцию двуязычных 
словарей и~параллельных корпусов.
  
  \bigskip
  
  
  Автор признателен рецензентам за помощь в~улучшении статьи.
  
{\small\frenchspacing
 { %\baselineskip=10.6pt
 %\addcontentsline{toc}{section}{References}
 \begin{thebibliography}{99}
\bibitem{1-zac}
\Au{Aijmer K., Altenberg~B.} Advances in corpus-based contrastive linguistics. Studies in honour 
of Stig Johansson.~--- Amsterdam: John Benjamins, 2013. 295~p.  doi: 10.1075/scl.54.
\bibitem{2-zac}
\Au{Добровольский Д.\,О., Кретов~А.\, А., Шаров~С.\,А.} Корпус параллельных текстов~// 
Научная и~техническая информация. Сер.~2: Информационные процессы и~сис\-те\-мы, 2005. 
№\,6. С.~16--27.
\bibitem{3-zac}
\Au{Добровольский Д.\,О.} Корпус параллельных текстов и~сопоставительная 
лексикология~// Труды Института русского языка им.\ В.\,В.~Виноградова, 2015. №\,6. 
С.~413--449. EDN: VJQBHP.
\bibitem{4-zac}
\Au{Гончаров А.\,А., Зацман~И.\,М., Кружков~М.\,Г.} Эволюция классификаций 
в~надкорпусных базах данных~// Информатика и~её применения, 2020. Т.~14. Вып.~4. 
С.~108--116. doi: 10.14357/19922264200415.  
EDN: \mbox{GKWBZT}.
\bibitem{5-zac}
\Au{Гончаров А.\, А., Зацман И. \,М., Кружков~М.\, Г}. Представление новых 
лексикографических знаний в~динамических классификационных сис\-те\-мах~// 
Информатика и~её применения, 2021. Т.~15. Вып.~1. С.~86--93.  doi: 10.14357/19922264210112. EDN: OPEFXW.
\bibitem{6-zac}
\Au{Zatsman I.} Finding and filling lacunas in linguistic typologies~// 15th Forum (International) 
on Knowledge Asset Dynamics Proceedings.~--- Matera, Italy: Institute of Knowledge Asset 
Management, 2020. P.~780--793.
\bibitem{7-zac}
\Au{Zatsman I.} Three-dimensional encoding of emerging meanings in AI-systems~// 21st 
European Conference on Knowledge Management Proceedings.~--- Reading, U.K.: Academic 
Publishing International Ltd., 2020. P.~878--887.
\bibitem{8-zac}
\Au{Ackoff R.} From data to wisdom~// J.~Applied Systems Analysis, 1989. Vol.~16. No.\,1. P.~3--9.
\bibitem{9-zac}
\Au{Rosenbloom P.\,S.} On computing: The fourth great scientific domain.~--- Cambridge, MA, 
USA: MIT Press, 2013. 307~p.
\bibitem{10-zac}
\Au{Rowley J.} The wisdom hierarchy: Representations of the DIKW hierarchy~// J.~Inf. 
Sci., 2007. Vol.~33. Iss.~2. P.~163--180. doi: 10.1177/0165551506070706.
\bibitem{11-zac} 
\Au{Frick$\acute{\mbox{e}}$~M.\,H.} Data--Information--Knowledge--Wisdom (DIKW) pyramid, 
framework, continuum~// Encyclopedia of big data~/ Eds. L.~Schintler, C.~McNeely.~--- Cham: 
Springer, 2018. 4~p. doi: 10.1007/978-3-319-32001-4\_331-1.
\bibitem{12-zac}
\Au{Denning P., Rosenbloom~P.} Computing: The fourth great domain of science~// Commun. 
ACM, 2009. Vol.~52. Iss.~9. P.~27--29.
\bibitem{13-zac}
\Au{Denning P., Freeman~P.} Computing's paradigm~// Commun.  ACM, 2009. Vol.~52. 
Iss.~12. P.~28--30. doi: 10.1145/ 1610252.1610265.
\bibitem{17-zac} %14
\Au{Farradane J.} Knowledge, information, and information science~// J.~Inf. Sci., 
1980. Vol.~2. Iss.~2. P.~75--80. doi: 10.1177/01655515800020020.

\bibitem{15-zac}
\Au{Шрейдер Ю.\,А.} Информация и~знание~// Сис\-тем\-ная концепция информационных 
процессов.~--- М.: ВНИИСИ, 1988. С.~47--52.
\bibitem{16-zac}
\Au{Ingwersen P.} Information and information science~// Enclyclopaedie of library and 
information science~/ Eds. J.\,D.~McDonald, 
M.~Levine-Clark.~--- New York, NY, USA: Marcel Dekker Inc., 1992. Vol.~56. Sup.~19. 
P.~137--174.

\bibitem{14-zac} %17
Информатика как наука об информации: Информационный, документальный, 
технологический, экономический, социальный и~организационный аспекты~/ Под ред. 
Р.\,С.~Гиляревского.~--- М.: Фаир-Пресс, 2006. 592~с.

\bibitem{18-zac}
\Au{Hjorland B.} Library and information science: practice, theory, and philosophical basis~// 
Inform. Process. Manag., 2000. Vol.~36. Iss.~3. P.~501--531. doi:  
10.1016/S0306-\mbox{4573(99)00038-2}.
\bibitem{19-zac}
Deep shift~--- technology tipping points and societal impact.~--- Geneva: WE Forum, 2015. 44~p. 
{\sf http://www3.weforum.org/docs/WEF\_GAC15\_ Technological\_Tipping\_Points\_report\_2015.pdf}.
\bibitem{20-zac}
\Au{Berman F., Rutenbar~R., Hailpern~B., Christensen~H., Davidson~S., Estrin~D., 
Franklin~M., Martonosi~M., Raghavan~P., Stodden~V., Szalay~A.\,S.} Realizing the potential of 
data science~// Commun.  ACM, 2018. Vol.~61. Iss.~4. P.~67--72. doi: 10.1145/3188721.

\bibitem{21-zac}
\Au{Stodden V.} The data science life cycle: A~disciplined approach to advancing data science as 
a~science~// Commun.  ACM, 2020. Vol.~63. Iss.~7. P.~58--66. doi: 10.1145/ 3360646.


\bibitem{23-zac} %22
\Au{Зацман И.\,М.} Научная парадигма информатики: классификация трансформаций 
объектов предметной об\-ласти~// Системы и~средства информатики, 2023. Т.~33. №\,4. 
С.~126--138. doi: 10.14357/08696527230412. EDN: ZIKUWO.

\bibitem{22-zac} %23
\Au{Зацман И.\,М.} Научная парадигма информатики: классификация объектов предметной  
об\-ласти~// Информатика и~её применения, 2023. Т.~17. Вып.~4. С.~96--103. doi: 
10.14357/19922264230413. EDN: FIUQAT.

\bibitem{24-zac}
\Au{Зацман И.\,М.} О~научной парадигме информатики: верхний уровень классификации 
объектов ее предметной об\-ласти~// Информатика и~её применения, 2022. Т.~16. Вып.~4. 
С.~73--79. doi: 10.14357/ 19922264220411. EDN: XZNKVI.

\bibitem{25-zac}
\Au{Соломоник А.\,Б.} Философия знаковых систем и~язык.~--- М.: ЛКИ, 2011. 408~с.
\bibitem{26-zac}
\Au{Зацман И.\,М.} Трансформация иерархии Акоффа в~научной парадигме информатики~// 
Информатика и~её применения, 2023. Т.~17. Вып.~3. С.~107--113. doi: 
10.14357/19922264230315. EDN: UMVRRV.

\bibitem{27-zac}
\Au{Zatsman I.} Building digital spiral models of knowledge generation~// 19th Forum 
(International) on Knowledge Asset Dynamics Proceedings.~--- Matera, Italy: Arts for Business 
Institute, 2024. P.~2185--2196.
\bibitem{28-zac}
\Au{Zatsman I.} Digital spiral model of knowledge creation and encoding its dynamics~// 18th 
Forum (International) on Knowledge Asset Dynamics Proceedings.~--- Matera, Italy: Arts for 
Business Institute, 2023. P.~581--596.
\bibitem{29-zac}
\Au{Зацман И.\,М.} Интерфейсы третьего порядка в~информатике~// Информатика и~её 
применения, 2019. Т.~13. Вып.~3. С.~82--89. doi: 10.14357/19922264190312. EDN: 
EHRQLF.

\bibitem{30-zac}
\Au{Зацман И.\,М.} Научная парадигма информатики как третьей культуры~//  
На\-уч\-но-тех\-ни\-че\-ская информация. Сер.~1: Организация и~методика информационной 
работы, 2023. №\,11. С.~1--14.

\end{thebibliography}

 }
 }

\end{multicols}

\vspace*{-9pt}

\hfill{\small\textit{Поступила в~редакцию 14.04.24}}

\vspace*{4pt}

%\pagebreak

%\newpage

%\vspace*{-28pt}

\hrule

\vspace*{2pt}

\hrule



\def\tit{OBJECT TRANSFORMATIONS OF~THE~FIRST AND~SECOND ORDER
IN~A~LEXICOGRAPHIC INFORMATION SYSTEM\\[-5pt]}


\def\titkol{Object transformations of~the~first and~second order
in~a~lexicographic information system}


\def\aut{I.\,M.~Zatsman}

\def\autkol{I.\,M.~Zatsman}

\titel{\tit}{\aut}{\autkol}{\titkol}

\vspace*{-13pt}


\noindent
Federal Research Center ``Computer Science and Control'' of the Russian Academy of Sciences, 
44-2~Vavilov Str., Moscow 119133, Russian Federation


\def\leftfootline{\small{\textbf{\thepage}
\hfill INFORMATIKA I EE PRIMENENIYA~--- INFORMATICS AND
APPLICATIONS\ \ \ 2024\ \ \ volume~18\ \ \ issue\ 2}
}%
 \def\rightfootline{\small{INFORMATIKA I EE PRIMENENIYA~---
INFORMATICS AND APPLICATIONS\ \ \ 2024\ \ \ volume~18\ \ \ issue\ 2
\hfill \textbf{\thepage}}}

\vspace*{2pt}



\Abste{The theoretical foundations of the design of information technologies used for 
the integration of bilingual dictionaries and parallel corpora are considered. The 
description of the first outcomes of the creation of the third\linebreak\vspace*{-12pt}}

\Abstend{ level of object 
transformations classification in the subject domain of informatics, which is supposed 
to be used
in creating the lexicographic information system providing integration, is 
given. All the entities of informatics are divided into two global classes: objects and 
their transformations. For each such class, its own classification is constructed. 
Previously, the two upper levels of the object transformation classification in the subject 
domain have been described. The present paper discusses the third level of this classification. The 
basis for the construction of its highest level was the division of the subject domain of 
informatics into media (mental, sensory, digital, and a~number of other media), each 
of which by definition includes objects of the same nature. The Solomonick's 
typology of sign systems served as the basis for constructing the second level of the 
object transformation classification. The aim of the paper is to systematize object 
transformations of the first and second orders at the third level of this classification. 
The basis for systematization is the medium version of the Ackoff's hierarchy.}

\KWE{subject domain objects; object transformations; classification; data; 
information; knowledge; lexicographic information system}


\DOI{10.14357/19922264240211}{VZTGVV}

\vspace*{-12pt}

\Ack

\vspace*{-3pt}


\noindent
The reported study was funded by the Russian Science Foundation, project  
No.\,24-18-00155, {\sf 
https://rscf.ru/project/24-18-00155}. The research was carried out using the infrastructure of the Shared 
Research Facilities ``High Performance Computing and Big Data'' (CKP 
``Informatics'') of FRC CSC RAS (Moscow) .
   


  \begin{multicols}{2}

\renewcommand{\bibname}{\protect\rmfamily References}
%\renewcommand{\bibname}{\large\protect\rm References}

{\small\frenchspacing
 {%\baselineskip=10.8pt
 \addcontentsline{toc}{section}{References}
 \begin{thebibliography}{99} 
\bibitem{1-zac-1}
\Aue{Aijmer, K., and B.~Altenberg.} 2013. \textit{Advances in corpus-based 
contrastive linguistics. Studies in honour of Stig Johansson}. Amsterdam: John 
Benjamins. 295~p. doi: 10.1075/scl.54.
\bibitem{2-zac-1}
\Aue{Dobrovolskiy, D.\,O., A.\,A.~Kretov, and S.\,A.~Sharov.} 2005. Korpus 
parallel'nykh tekstov [Corpus of parallel texts]. \textit{Nauchnaya i~tekhnicheskaya 
informatsiya. Ser. 2. Informatsionnye protsessy i~sistemy} [Scientific and Technical 
Information. Ser.~2: Information Processes and Systems] 6:16--27.
\bibitem{3-zac-1}
\Aue{Dobrovolskiy, D.\,O.} 2015. Korpus parallel'nykh tekstov i~sopostavitel'naya 
leksikologiya [The corpus of parallel texts and contrastive lexicology]. \textit{Trudy 
Instituta russkogo yazyka im. V.\,V.~Vinogradova} [Proceedings of the 
V.\,V.~Vinogradov Russian Language Institute] 6:413--449. EDN: VJQBHP.
\bibitem{4-zac-1}
\Aue{Goncharov, A.\,A., I.\,M.~Zatsman, and M.\,G.~Kruzhkov.} 2020. Evolyutsiya 
klassifikatsiy v~nadkorpusnykh ba\-zakh dannykh [Evolution of classifications in 
supracorpora databases]. \textit{Informatika i~ee Primeneniya~--- Inform. \mbox{Appl.}}  
14(4):108--116. doi: 10.14357/19922264200415.  
EDN: GKWBZT.
\bibitem{5-zac-1}
\Aue{Goncharov, A.\,A., I.\,M.~Zatsman, and M.\,G.~Kruzhkov.} 2021. 
Predstavlenie novykh leksikograficheskikh znaniy v~dinamicheskikh 
klassifikatsionnykh sistemakh [Representation of new lexicographical knowledge in 
dynamic classification systems]. \textit{Informatika i~ee Primeneniya~--- Inform. 
Appl.} 15(1):86--93. doi: 10.14357/19922264210112. EDN: OPEFXW.
\bibitem{6-zac-1}
\Aue{Zatsman, I.} 2020. Finding and filling lacunas in linguistic typologies. 
\textit{15th Forum (International) on Knowledge Asset Dynamics Proceedings}. 
Matera, Italy: Institute of Knowledge Asset Management. 780--793.
\bibitem{7-zac-1}
\Aue{Zatsman, I.} 2020. Three-dimensional encoding of emerging meanings in  
AI-systems. \textit{21st European Conference on Knowledge Management 
Proceedings}. Reading, U.K.: Academic Publishing International Ltd. 878--887.
\bibitem{8-zac-1}
\Aue{Ackoff, R.} 1989. From data to wisdom. \textit{J.~Applied Systems Analysis} 
16(1):3--9.
\bibitem{9-zac-1}
\Aue{Rosenbloom, P.\,S.} 2013. \textit{On computing: The fourth great scientific 
domain}. Cambridge, MA: MIT Press. 307~p.
\bibitem{10-zac-1}
\Aue{Rowley, J.} 2007. The wisdom hierarchy: Representations of the DIKW 
hierarchy. \textit{J.~Inf. Sci.} 33(2):163--180. doi: 10.1177/0165551506070706.
\bibitem{11-zac-1}
\Aue{Frick$\acute{\mbox{e}}$, M.\,H.} 2018.  
Data-Information-Knowledge-Wisdom (DIKW) pyramid, framework, continuum. 
\textit{Encyclopedia of big data}. Eds. L.~Schintler and C.~McNeely. Cham: 
Springer. 4~p. doi: 10.1007/978-3-319-32001- 4\_331-1.
\bibitem{12-zac-1}
\Aue{Denning, P., and P.~Rosenbloom.} 2009. Computing: The fourth great domain 
of science. \textit{Commun. ACM} 52(9):27--29.
\bibitem{13-zac-1}
\Aue{Denning, P., and P.~Freeman.} 2009. Computing's paradigm. \textit{Commun. 
ACM} 52(12):28--30. doi: 10.1145/ 1610252.1610265.

\bibitem{17-zac-1} %14
\Aue{Farradane, J.} 1980. Knowledge, information, and information science. 
\textit{J.~Inf. Sci.} 2(2):75--80. doi: 10.1177/ 01655515800020020.

\bibitem{15-zac-1}
\Aue{Shreyder, Yu.\,A.} 1988. Informatsiya i~znanie [Information and knowledge]. 
\textit{Sistemnaya kontseptsiya in\-for\-ma\-tsi\-on\-nykh protsessov} [System concept of 
information processes]. Moscow: VNIISI. 47--52.
\bibitem{16-zac-1}
\Aue{Ingwersen, P.} 1995. Information and information science. 
\textit{Encyclopedia of library and information science}. Eds. J.\,D.~McDonald and 
M.~Levine-Clark. New York, NY: Marcel Dekker Inc. 56(19):137--174.

\bibitem{14-zac-1} %17
Gilyarevskiy, R.\,S., ed. 2006. \textit{Informatika kak nauka ob informatsii: 
informatsionnyy, dokumental'nyy, tekh\-no\-lo\-gi\-che\-skiy, ekonomicheskiy, sotsial'nyy 
i~organizatsionnyy aspekty} [Informatics as information science: Informational, 
documentary, technological, economic, social, and organizational dimensions]. 
Moscow: FAIR-PRESS. 592~p.

\bibitem{18-zac-1}
\Aue{Hjorland, B.} 2000. Library and information science: Practice, theory, and 
philosophical basis. \textit{Inform. Process. Manag.} 36(3):501--531. doi:  
10.1016/S0306-\mbox{4573(99)00038-2}.
\bibitem{19-zac-1}
Deep shift~--- technology tipping points and societal impact. 2015. \textit{World Economic 
Forum}. Geneva. 44~p. Available at: {\sf 
http://www3.weforum.org/docs/WEF\_ GAC15\_Technological\_Tipping\_Points\_report\_2015.pdf} (accessed May~20, 
2024).
\bibitem{20-zac-1}
\Aue{Berman, F., R.~Rutenbar, B.~Hailpern, H.~Christensen, S.~Davidson, 
D.~Estrin, M.~Franklin, M.~Martonosi, P.~Raghavan, V.~Stodden, and 
A.\,S.~Szalay.} 2018. Realizing the potential of data science. \textit{Commun. ACM} 
61(4):67--72. doi: 10.1145/3188721.
\bibitem{21-zac-1}
\Aue{Stodden, V.} 2020. The data science life cycle: A~disciplined approach to 
advancing data science as a~science. \textit{Commun. ACM} 
 63(7):58--66. doi: 10.1145/3360646.

\bibitem{23-zac-1} %22
\Aue{Zatsman, I.\,M.} 2023. Nauchnaya paradigma informatiki: klassifikatsiya 
transformatsiy ob''ektov predmetnoy oblasti [Scientific paradigm of informatics: 
Transformation classification of domain objects]. \textit{Sistemy i~Sredstva 
Informatiki~--- Systems and Means of Informatics} 33(4):126--138. doi: 
10.14357/08696527230412. EDN: ZIKUWO.

\bibitem{22-zac-1} %23
\Aue{Zatsman, I.\,M.} 2023. Nauchnaya paradigma informatiki: klassifikatsiya 
ob''ektov predmetnoy oblasti [Scientific paradigm of informatics: Classification of 
domain objects]. \textit{Informatika i~ee Primeneniya~--- Inform. Appl.} 
 17(4):96--103. doi: 10.14357/19922264230413. EDN: FIUQAT.
 
\bibitem{24-zac-1}
\Aue{   Zatsman, I.\,M.} 2022. O nauchnoy paradigme informatiki: verkhniy uroven' 
klassifikatsii ob''ektov ee predmetnoy oblasti [On the scientific paradigm of 
informatics: The classification high level of its objects]. \textit{Informatika i~ee 
Primeneniya~--- Inform. Appl.} 16(4):73--79. doi: 10.14357/19922264220411. EDN: 
XZNKVI.
\bibitem{25-zac-1}
\Aue{Solomonick, A.\,B.} 2011. \textit{Filosofiya znakovykh system i~yazyk} 
[Philosophy of sign systems and language]. Moscow: LKI. 408~p.
\bibitem{26-zac-1}
\Aue{Zatsman, I.\,M.} 2023. Transformatsiya ierarkhii Akoffa v~nauchnoy 
paradigme informatiki [Transformation of the Ackoff's hierarchy in the scientific 
paradigm of informatics]. \textit{Informatika i~ee Primeneniya~--- Inform. \mbox{Appl.}} 
17(3):107--113. doi: 10.14357/19922264230315. EDN: UMVRRV.
\bibitem{27-zac-1}
\Aue{Zatsman, I.} 2024. Building digital spiral models of knowledge 
generation. \textit{19th Forum (International) on Knowledge Asset Dynamics 
Proceedings}. Matera, Italy: Arts for Business Institute. 2185--2196.
\bibitem{28-zac-1}
\Aue{Zatsman, I.} 2023. Digital spiral model of knowledge creation and encoding its 
dynamics. \textit{18th Forum (International) on Knowledge Asset Dynamics 
Proceedings}. Matera, Italy: Arts for Business Institute. 581--596.
\bibitem{29-zac-1}
\Aue{Zatsman, I.\,M.} 2019. Interfeysy tret'ego poryadka v~informatike 
 [Third-order interfaces in informatics]. \textit{Informatika i~ee Primeneniya~--- 
Inform. Appl.} 13(3):82--89. doi: 10.14357/19922264190312. EDN: EHRQLF.
\bibitem{30-zac-1}
\Aue{Zatsman, I.} 2023. Scientific paradigm of informatics as a~third culture. 
\textit{Scientific Technical Information Processing} 50(4):246--258. doi: 
10.3103/S0147688223040111. EDN: CKHMYS.

\end{thebibliography}

 }
 }

\end{multicols}

\vspace*{-6pt}

\hfill{\small\textit{Received April 14, 2024}} 


\vspace*{-12pt}


\Contrl

\vspace*{-3pt}

\noindent
\textbf{Zatsman Igor M.} (b.\ 1952)~--- Doctor of Science in technology, head of 
department, Federal Research Center ``Computer Science and Control'' of the 
Russian Academy of Sciences, 44-2~Vavilov Str., Moscow 119333, Russian 
Federation; \mbox{izatsman@yandex.ru}





\label{end\stat}

\renewcommand{\bibname}{\protect\rm Литература}  %11


\def\stat{authorsrus}
{%\hrule\par
%\vskip 7pt % 7pt
\raggedleft\Large \bf%\baselineskip=3.2ex
О\,Б\ \ А\,В\,Т\,О\,Р\,А\,Х \vskip 17pt
    \hrule
    \par
\vskip 21pt plus 8pt minus 4pt }


\def\tit{\ }

\def\aut{\ }

\def\auf{\ }

\def\leftkol{\ } % ENGLISH ABSTRACTS}

\def\rightkol{ОБ АВТОРАХ} %ENGLISH ABSTRACTS}

\titele{\tit}{\aut}{\auf}{\leftkol}{\rightkol}
      
            \label{st\stat}



\vspace*{24pt}

\begin{multicols}{2}




\noindent
\textbf{Архипов Олег Петрович} (р.\ 1948)~---
кандидат технических наук, директор Орловского филиала Института проб\-лем информатики
Российской академии наук
%302025, г.Орел, Московское шоссе, д.137

\vspace*{3pt}

\noindent
\textbf{Бирюкова Татьяна Константиновна} (р.\ 1968)~---
кандидат фи\-зи\-ко-ма\-те\-ма\-ти\-че\-ских наук, старший научный сотрудник Института проб\-лем информатики
Российской академии наук

\vspace*{3pt}

\noindent 
\textbf{Бобков  Сергей Геннадьевич} (р.\ 1955)~---
доктор технических наук,  заведующий отделением На\-уч\-но-ис\-сле\-до\-ва\-тель\-ско\-го 
института системных исследований Российской академии наук
%117218, Москва, Нахимовский просп., 36, к.1 

\vspace*{3pt}

\noindent \textbf{Васильев Николай Семенович} (р.\ 1952)~--- доктор 
фи\-зи\-ко-ма\-те\-ма\-ти\-че\-ских наук, профессор, 
МГТУ им.\ Н.\,Э.~Баумана 
%, Москва 105005, 2-я Бауманская ул., д.~5,

\vspace*{3pt}

\noindent
\textbf{Гершкович Максим Михайлович} (р.\ 1968)~---
старший научный сотрудник Института проб\-лем информатики
Российской академии наук

\vspace*{3pt}

\noindent 
\textbf{Дьяченко Юрий Георгиевич} (р.\ 1958)~--- кандидат технических наук, 
старший научный сотрудник Института проб\-лем информатики
Российской академии наук

\vspace*{3pt}

\noindent 
\textbf{Ерошенко Александр Андреевич} (р.\ 1989)~--- аспирант кафедры 
математической статистики факультета вычисли\-тельной математики и кибернетики 
Московского государственного университета им.\ М.\,В.~Ломоносова
%119991, Москва ГСП-1, Ленинские горы, д.\ 1, стр. 52

\vspace*{3pt}
 
\noindent 
\textbf{Захаров Виктор Николаевич} (р.\ 1948)~--- 
доктор технических наук, доцент, ученый секретарь Института проб\-лем информатики
Российской академии наук

\vspace*{3pt}

\noindent
\textbf{Зейфман Александр Израилевич} (р.\ 1954)~---
доктор фи\-зи\-ко-ма\-те\-ма\-ти\-че\-ских наук, профессор, 
заведующий кафедрой Вологодского государственного университета; 
старший научный сотрудник Института проб\-лем информатики
Российской академии наук; главный научный сотрудник ИСЭРТ Российской академии наук

\vspace*{3pt}

\noindent
\textbf{Зыкин Сергей Владимирович} (р.\ 1959)~--- 
доктор технических наук, профессор, заведующий лабораторией Института математики 
им.\ С.\,Л.~Соболева Сибирского отделения Российской академии наук, Новосибирск 
%630090, пр.\ ак.\ Коптюга, 4 

\vspace*{4pt}

\noindent
\textbf{Киреев Владимир Иванович} (р.\ 1938)~---
доктор фи\-зи\-ко-ма\-те\-ма\-ти\-че\-ских наук, профессор Московского 
государственного горного университета
%Адрес: Россия, 119991, г. Москва, Ленинский проспект, д. 6

%\columnbreak

\vspace*{4pt}

\noindent
\textbf{Козеренко Елена Борисовна} (р.\ 1959)~---
кандидат филологических наук, заведующая лабораторией Института проб\-лем информатики
Российской академии наук

\vspace*{4pt}

\noindent
\textbf{Королев Виктор Юрьевич} (р.\ 1954)~--- доктор
фи\-зи\-ко-ма\-те\-ма\-ти\-че\-ских наук, профессор кафедры математической 
статистики факультета вычисли\-тельной математики и кибернетики 
Московского государственного университета; 
ведущий научный сотрудник Института проб\-лем информатики
Российской академии наук

\vspace*{4pt}

\noindent
\textbf{Коротышева Анна Владимировна} (р.\ 1988)~---
старший преподаватель Вологодского государственного университета

\vspace*{4pt}

\noindent 
\textbf{Кун Де Турк} (р.\ 1981)~--- научный сотрудник 
исследовательской группы SMACS факультета телекоммуникаций и обработки информации
Университета Гента, Бельгия
%В-9000 Гент, Бельгия

\vspace*{4pt}

\noindent
\textbf{Лупенцов Олег Сергеевич} (р.\ 1986)~---
аспирант Омского государственного института сервиса
%Омск 644043, ул.\ Певцова 13

\vspace*{4pt}

\noindent
\textbf{Лучко Олег Николаевич} (р.\ 1961)~---
кандидат педагогических наук, профессор, заведующий кафедрой 
Омского государственного института сервиса
%Омск 644043, ул.\ Певцова 13

\vspace*{4pt}

\noindent
\textbf{Малашенко Юрий Евгеньевич} (р.\ 1946)~---
доктор фи\-зи\-ко-ма\-те\-ма\-ти\-че\-ских наук, заведующий сектором 
Вычислительного центра им.\ А.\,А.~Дородницына Российской академии наук
%Адрес: 119333, Москва, ул. Вавилова, 40,

\vspace*{4pt}

\noindent
\textbf{Маньяков Юрий Анатольевич} (р.\ 1984)~---
кандидат технических наук, научный сотрудник Орловского филиала Института проб\-лем информатики
Российской академии наук
%302025, г.Орел, Московское шоссе, д.137

\vspace*{4pt}

\noindent
\textbf{Маренко Валентина Афанасьевна} (р.\ 1951)~---
кандидат технических наук, доцент, старший научный сотрудник 
Института математики им.\ С.\,Л.~Соболева Сибирского отделения Российской академии наук
%Новосибирск 630090, пр. ак. Коптюга, 4 

\vspace*{3pt}

\noindent 
\textbf{Морозов Евсей Викторович} (р.\ 1947)~--- доктор 
фи\-зи\-ко-ма\-те\-ма\-ти\-че\-ских, профессор, ведущий научный сотрудник 
Института прикладных математических исследований Карельского научного центра Российской
академии наук; 
%%185910 Россия, Республика Карелия, г.\ Петрозаводск, ул.\ Пушкинская, 11
профессор Петрозаводского государственного университета, Петрозаводск
%185910 Россия, Республика Карелия, г.\ Петрозаводск, пр.\ Ленина, 33

%\pagebreak

\vspace*{3pt}

\noindent
\textbf{Назарова Ирина Александровна} (р.\ 1966)~---
кандидат фи\-зи\-ко-ма\-те\-ма\-ти\-че\-ских наук, 
научный сотрудник Вычислительного центра им.\ А.\,А.~Дородницына Российской академии наук 
%Адрес: 119333, Москва, ул. Вавилова, 40

\vspace*{3pt}

\noindent
\textbf{Павлов Игорь Валерианович} (р.\ 1945)~--- 
доктор фи\-зи\-ко-ма\-те\-ма\-ти\-че\-ских наук, профессор МГТУ им.\ Н.\,Э.~Баумана 
%Москва 105005, 2-я Бауманская ул., д.~5 

%\pagebreak

\vspace*{3pt}

\noindent 
\textbf{Потахина Любовь Викторовна} (р.\ 1989)~--- аспирантка
Института прикладных математических исследований Карельского научного центра
Российской академии наук; 
%%185910 Россия, Республика Карелия, г.\ Петрозаводск, ул.\ Пушкинская, 11
инженер Петрозаводского государственного университета, Петрозаводск
%185910 Россия, Республика Карелия, г.\ Петрозаводск, пр.\ Ленина, 33

\vspace*{3pt}

\noindent 
\textbf{Рождественский Юрий Владимирович} (р.\ 1952)~--- 
кандидат технических наук, заведующий сектором Института проб\-лем информатики
Российской академии наук

\vspace*{3pt}

\noindent 
\textbf{Синицын Игорь Николаевич} (р.\ 1940)~--- доктор технических наук,
профессор, заслуженный деятель\linebreak\vspace*{-12pt}

\columnbreak

\noindent
 науки РФ, заведующий отделом Института проб\-лем информатики
Российской академии наук

\vspace*{7pt}


\noindent
\textbf{Сиротинин Денис Олегович} (р.\ 1984)~---
кандидат технических наук, научный сотрудник Орловского филиала Института проб\-лем информатики
Российской академии наук
%302025, г.Орел, Московское шоссе, д.137

\vspace*{7pt}

%\columnbreak

\noindent 
\textbf{Соколов  Игорь Анатольевич} (р.\ 1954)~--- академик (действительный член) Российской 
академии наук, доктор технических наук, директор Института проб\-лем информатики
Российской академии наук

\vspace*{7pt}

\noindent
\textbf{Степченков Юрий Афанасьевич} (р.\ 1951)~---
кандидат технических наук, заведующий отделом Института проб\-лем информатики
Российской академии наук

\vspace*{7pt}

\noindent
\textbf{Сурков Алексей Викторович} (р.\ 1978)~--- 
старший научный сотрудник На\-уч\-но-ис\-сле\-до\-ва\-тель\-ско\-го 
института системных исследований Российской академии наук
%117218, Москва, Нахимовский просп., 36, к.1 

\vspace*{7pt}

\noindent 
\textbf{Шестаков Олег Владимирович} (р.\ 1976)~--- доктор 
фи\-зи\-ко-ма\-те\-ма\-ти\-че\-ских, доцент кафедры математической статистики 
факультета вычисли\-тельной математики и кибернетики Московского 
государственного университета им.\ М.\,В.~Ломоносова; 
%119991, Москва ГСП-1, Ленинские горы, д.\ 1, стр. 52
старший научный сотрудник Института проб\-лем информатики
Российской академии наук
%, Москва 119333, ул. Вавилова, д.~44, корп.~2

\vspace*{7pt}

\noindent 
\textbf{Шоргин Сергей Яковлевич} (р.\ 1952.)~--- доктор
фи\-зи\-ко-ма\-те\-ма\-ти\-че\-ских наук, профессор, заместитель директора Института 
проб\-лем информатики Российской академии наук





%%%%%%%%%%%%%%%%%%%%%%%%%%%%%%%%%%%%%%%%%%%%%%%%%%%%%%%%%%%%%%%%%%%%%%%%%%%%%%%




%\def\rightkol{ОБ АВТОРАХ}
%\def\leftkol{ОБ АВТОРАХ}

 \label{end\stat}





%\def\leftfootline{\small{\textbf{\thepage}
%\hfill ИНФОРМАТИКА И ЕЁ ПРИМЕНЕНИЯ\ \ \ том~7\ \ \ выпуск~1\ \ \ 2013}
%}%
% \def\rightfootline{\small{ИНФОРМАТИКА И ЕЁ ПРИМЕНЕНИЯ\ \ \ том~7\ \ \ выпуск~1\ \ \ 2013
%\hfill \textbf{\thepage}}}


%\thispagestyle{myheadings}



\end{multicols}

\newpage  

%\def\stat{cont}
{%\hrule\par
%\vskip 7pt % 7pt
\raggedleft\Large \bf%\baselineskip=3.2ex
А\,В\,Т\,О\,Р\,С\,К\,И\,Й\ \ У\,К\,А\,З\,А\,Т\,Е\,Л\,Ь\ \ З\,А\ \ 2\,0\,0\,7 г. \vskip 17pt
    \hrule
    \par
\vskip 21pt plus 6pt minus 3pt }

\label{st\stat}

\def\tit{\ }

\def\aut{\ }
\def\auf{\ }

\def\leftkol{\ } % ENGLISH ABSTRACTS}

\def\rightkol{\ } %ENGLISH ABSTRACTS}

\titele{\tit}{\aut}{\auf}{\leftkol}{\rightkol}


\contentsline {chapter}{\ }{Выпуск \quad Стр.} 
\contentsline {section}{\textbf{Батракова Д.\,А., Королев В.\,Ю., Шоргин С.\,Я.}\ \ Новый метод вероятностно-ста\-ти\-сти\-че\-ско\-го анализа информационных потоков в\nobreakspace {}телекоммуникационных сетях}{\qquad 1 \qquad 40} 
\contentsline {section}{\textbf{Борисов А.\,В.}\ \ Байесовское оценивание в системах наблюдения с\nobreakspace {}марковскими скачкообразными процессами: игровой подход}{\qquad 2 \qquad 65}
\contentsline {section}{\textbf{Босов А.\,В., Иванов А.\,В.}\ \ Программная инфраструктура информационного Web-пор\-тала}{\qquad 2 \qquad 50}
\contentsline {section}{\textbf{Захаров В.\,Н., Калиниченко Л.\,А., Соколов И.\,А., Ступников С.\,А.}\ \ Конструирование канонических информационных моделей для интегрированных информационных систем}{\qquad 2 \qquad 15}
\contentsline {section}{\textbf{Захаров В.\,Н., Козмидиади В.\,А.}\ \ Средства обеспечения отказоустойчивости при\-ло\-жений}{\qquad 1 \qquad 14} 
\contentsline {section}{\textbf{Иванов А.\,В.}\ \ см. Босов А.\,В.\hfill\hfill\hfill\hfill\hfill\hfill\hfill\hfill\hfill\hfill\hfill\hfill\hfill\hfill\hfill\hfill\hfill\hfill\hfill\hfill\hfill\hfill\hfill\hfill\hfill\hfill\hfill\hfill\hfill\hfill\hfill\hfill\hfill\hfill\hfill}{\ }
\contentsline {section}{\textbf{Ильин В.\,Д., Соколов И.\,А.}\ \ Символьная модель системы знаний информатики в\nobreakspace {}че\-ло\-ве\-ко-автоматной среде}{\qquad 1 \qquad 66} 
\contentsline {section}{\textbf{Калиниченко Л.\,А.}\ \ см. Захаров В.\,Н.\hfill\hfill\hfill\hfill\hfill\hfill\hfill\hfill\hfill\hfill\hfill\hfill\hfill\hfill\hfill\hfill\hfill\hfill\hfill\hfill\hfill\hfill\hfill\hfill\hfill\hfill\hfill\hfill\hfill\hfill\hfill\hfill\hfill\hfill\hfill}{\ }
\contentsline {section}{\textbf{Козеренко Е.\,Б.}\ \ Лингвистическое моделирование для систем машинного перевода и обработки знаний}{\qquad 1 \qquad 54} 
\contentsline {section}{\textbf{Козмидиади В.\,А.}\ \ см. Захаров В.\,Н.\hfill\hfill\hfill\hfill\hfill\hfill\hfill\hfill\hfill\hfill\hfill\hfill\hfill\hfill\hfill\hfill\hfill\hfill\hfill\hfill\hfill\hfill\hfill\hfill\hfill\hfill\hfill\hfill\hfill\hfill\hfill\hfill\hfill\hfill\hfill }{\ } 
\contentsline {section}{\textbf{Королев В.\,Ю.}\ \ см. Батракова Д.\,А.\hfill\hfill\hfill\hfill\hfill\hfill\hfill\hfill\hfill\hfill\hfill\hfill\hfill\hfill\hfill\hfill\hfill\hfill\hfill\hfill\hfill\hfill\hfill\hfill\hfill\hfill\hfill\hfill\hfill\hfill\hfill\hfill\hfill\hfill\hfill}{\ } 
\contentsline {section}{\textbf{Кудрявцев А.\,А., Шоргин С.\,Я.}\ \ Байесовский подход к\nobreakspace {}анализу систем массового обслуживания и\nobreakspace {}показателей надежности}{\qquad 2 \qquad 76}
\contentsline {section}{\textbf{Печинкин А.\,В., Соколов И.\,А., Чаплыгин В.\,В.}\ \ Многолинейная система массового обслуживания с конечным накопителем и ненадежными приборами}{\qquad 1 \qquad 27} 
\contentsline {section}{\textbf{Печинкин А.\,В., Соколов И.\,А., Чаплыгин В.\,В.}\ \ Стационарные характеристики многолинейной\nobreakspace {}системы массового обслуживания с\nobreakspace {}одновременными отказами приборов}{\qquad 2 \qquad 39}
\contentsline {section}{\textbf{Синицын И.\,Н.}\ \ Корреляционные методы построения аналитических информационных моделей флуктуаций полюса Земли по априорным данным}{\qquad 2 \qquad \hphantom{9}2}
\contentsline {section}{\textbf{Синицын И.\,Н.}\ \ Развитие теории фильтров Пугачева для оперативной обработки информации в стохастических системах}{{\qquad 1 \qquad \hphantom{9}3}} 
\contentsline {section}{\textbf{Соколов И.\,А.}\ \ см. Захаров В.\,Н.\hfill\hfill\hfill\hfill\hfill\hfill\hfill\hfill\hfill\hfill\hfill\hfill\hfill\hfill\hfill\hfill\hfill\hfill\hfill\hfill\hfill\hfill\hfill\hfill\hfill\hfill\hfill\hfill\hfill\hfill\hfill\hfill\hfill\hfill\hfill}{\ }
\contentsline {section}{\textbf{Соколов И.\,А.}\ \ см. Ильин В.\,Д.\hfill\hfill\hfill\hfill\hfill\hfill\hfill\hfill\hfill\hfill\hfill\hfill\hfill\hfill\hfill\hfill\hfill\hfill\hfill\hfill\hfill\hfill\hfill\hfill\hfill\hfill\hfill\hfill\hfill\hfill\hfill\hfill\hfill\hfill\hfill}{\ } 
\contentsline {section}{\textbf{Соколов И.\,А.}\ \ см. Печинкин А.\,В.\hfill\hfill\hfill\hfill\hfill\hfill\hfill\hfill\hfill\hfill\hfill\hfill\hfill\hfill\hfill\hfill\hfill\hfill\hfill\hfill\hfill\hfill\hfill\hfill\hfill\hfill\hfill\hfill\hfill\hfill\hfill\hfill\hfill\hfill\hfill}{\ } 
\contentsline {section}{\textbf{Соколов И.\,А.}\ \ см. Печинкин А.\,В.\hfill\hfill\hfill\hfill\hfill\hfill\hfill\hfill\hfill\hfill\hfill\hfill\hfill\hfill\hfill\hfill\hfill\hfill\hfill\hfill\hfill\hfill\hfill\hfill\hfill\hfill\hfill\hfill\hfill\hfill\hfill\hfill\hfill\hfill\hfill}{\ }
\contentsline {section}{\textbf{Ступников С.\,А.}\ \ см. Захаров В.\,Н.\hfill\hfill\hfill\hfill\hfill\hfill\hfill\hfill\hfill\hfill\hfill\hfill\hfill\hfill\hfill\hfill\hfill\hfill\hfill\hfill\hfill\hfill\hfill\hfill\hfill\hfill\hfill\hfill\hfill\hfill\hfill\hfill\hfill\hfill\hfill}{\ }
\contentsline {section}{\textbf{Чаплыгин В.\,В.}\ \ см. Печинкин А.\,В.\hfill\hfill\hfill\hfill\hfill\hfill\hfill\hfill\hfill\hfill\hfill\hfill\hfill\hfill\hfill\hfill\hfill\hfill\hfill\hfill\hfill\hfill\hfill\hfill\hfill\hfill\hfill\hfill\hfill\hfill\hfill\hfill\hfill\hfill\hfill}{\ } 
\contentsline {section}{\textbf{Чаплыгин В.\,В.}\ \ см. Печинкин А.\,В.\hfill\hfill\hfill\hfill\hfill\hfill\hfill\hfill\hfill\hfill\hfill\hfill\hfill\hfill\hfill\hfill\hfill\hfill\hfill\hfill\hfill\hfill\hfill\hfill\hfill\hfill\hfill\hfill\hfill\hfill\hfill\hfill\hfill\hfill\hfill}{\ }
\contentsline {section}{\textbf{Шоргин С.\,Я.}\ \ см. Батракова Д.\,А.\hfill\hfill\hfill\hfill\hfill\hfill\hfill\hfill\hfill\hfill\hfill\hfill\hfill\hfill\hfill\hfill\hfill\hfill\hfill\hfill\hfill\hfill\hfill\hfill\hfill\hfill\hfill\hfill\hfill\hfill\hfill\hfill\hfill\hfill\hfill}{\ } 
\contentsline {section}{\textbf{Шоргин С.\,Я.}\ \ см. Кудрявцев А.\,А.\hfill\hfill\hfill\hfill\hfill\hfill\hfill\hfill\hfill\hfill\hfill\hfill\hfill\hfill\hfill\hfill\hfill\hfill\hfill\hfill\hfill\hfill\hfill\hfill\hfill\hfill\hfill\hfill\hfill\hfill\hfill\hfill\hfill\hfill\hfill}{\ }
%\thispagestyle{myheadings}
\def\leftfootline{\small{\textbf{\thepage}
\hfill ИНФОРМАТИКА И ЕЁ ПРИМЕНЕНИЯ\ \ \ том~1\ \ \ выпуск~2\ \ \ 2007}
}%
 \def\rightfootline{\small{ИНФОРМАТИКА И ЕЁ ПРИМЕНЕНИЯ\ \ \ том~1\ \ \ выпуск~2\ \ \ 2007
 \hfill \textbf{\thepage}}}
 \label{end\stat} 
                     
%\def\stat{cont-e}
{%\hrule\par
%\vskip 7pt % 7pt
\raggedleft\Large \bf%\baselineskip=3.2ex
2\,0\,0\,7\ \ A\,U\,T\,H\,O\,R\ \ I\,N\,D\,E\,X \vskip 17pt
    \hrule
    \par
\vskip 21pt plus 6pt minus 3pt }

\label{st\stat}

\def\tit{\ }

\def\aut{\ }
\def\auf{\ }

\def\leftkol{\ } % ENGLISH ABSTRACTS}

\def\rightkol{\ } %ENGLISH ABSTRACTS}

\titele{\tit}{\aut}{\auf}{\leftkol}{\rightkol}


\contentsline {chapter}{\ }{Issue \quad Page} 
\contentsline {subsection}{\textbf{Batrakova D.\,A., Korolev V.\,Yu., Shorgin S.\,Ya.}\ \ A New Method for the Probabilistic and Statistical Analysis of Information Flows in Telecommunication Networks}{\qquad 1 \qquad 40} 
\contentsline {subsection}{\textbf{Borisov A.\,V.}\ \ Bayesian Estimation in\nobreakspace {}Observation Systems with\nobreakspace {}Markov Jump Processes: Game-Theoretic Approach}{\qquad 2 \qquad 65} 
\contentsline {subsection}{\textbf{Bosov A.\,V., Ivanov A.\,V.}\ \ Linguistic Simulation for Machine Translation and Knowledge Management Systems}{\qquad 2 \qquad 50} 
\contentsline {subsection}{\textbf{Chaplygin V.\,V.} see Pechinkin A.\,V.\hfill\hfill\hfill\hfill\hfill\hfill\hfill\hfill\hfill\hfill\hfill\hfill\hfill\hfill\hfill\hfill\hfill\hfill\hfill\hfill\hfill\hfill\hfill\hfill\hfill\hfill\hfill\hfill\hfill\hfill\hfill\hfill\hfill\hfill\hfill}{\ }
\contentsline {subsection}{\textbf{Chaplygin V.\,V.} see Pechinkin A.\,V.\hfill\hfill\hfill\hfill\hfill\hfill\hfill\hfill\hfill\hfill\hfill\hfill\hfill\hfill\hfill\hfill\hfill\hfill\hfill\hfill\hfill\hfill\hfill\hfill\hfill\hfill\hfill\hfill\hfill\hfill\hfill\hfill\hfill\hfill\hfill}{\ }
\contentsline {subsection}{\textbf{Ilyin V.\,D., Sokolov I.\,A.}\ \ The Symbol Model of Informatics Knowledge System in Human-Automaton Environment}{\qquad 1 \qquad 66} 
\contentsline {subsection}{\textbf{Ivanov A.\,V.} see Bosov A.\,V.\hfill\hfill\hfill\hfill\hfill\hfill\hfill\hfill\hfill\hfill\hfill\hfill\hfill\hfill\hfill\hfill\hfill\hfill\hfill\hfill\hfill\hfill\hfill\hfill\hfill\hfill\hfill\hfill\hfill\hfill\hfill\hfill\hfill\hfill\hfill}{\ }
\contentsline {subsection}{\textbf{Kalinichenko L.\,A.} see Zakharov V.\,N.\hfill\hfill\hfill\hfill\hfill\hfill\hfill\hfill\hfill\hfill\hfill\hfill\hfill\hfill\hfill\hfill\hfill\hfill\hfill\hfill\hfill\hfill\hfill\hfill\hfill\hfill\hfill\hfill\hfill\hfill\hfill\hfill\hfill\hfill\hfill}{\ }
\contentsline {subsection}{\textbf{Korolev V.\,Yu.} see Batrakova D.\,A.\hfill\hfill\hfill\hfill\hfill\hfill\hfill\hfill\hfill\hfill\hfill\hfill\hfill\hfill\hfill\hfill\hfill\hfill\hfill\hfill\hfill\hfill\hfill\hfill\hfill\hfill\hfill\hfill\hfill\hfill\hfill\hfill\hfill\hfill\hfill}{\ }
\contentsline {subsection}{\textbf{Kozerenko E.\,B.}\ \ Linguistic Simulation for Machine Translation and Knowledge Management Systems}{\qquad 1 \qquad 54} 
\contentsline {subsection}{\textbf{Kozmidiady V.\,A.} see Zakharov V.\,N.\hfill\hfill\hfill\hfill\hfill\hfill\hfill\hfill\hfill\hfill\hfill\hfill\hfill\hfill\hfill\hfill\hfill\hfill\hfill\hfill\hfill\hfill\hfill\hfill\hfill\hfill\hfill\hfill\hfill\hfill\hfill\hfill\hfill\hfill\hfill}{\ }
\contentsline {subsection}{\textbf{Kudryavtsev A.\,A., Shorgin S.\,Ya.}\ \ Bayesian Approach to Queueing Systems and Reliability Characteristics}{\qquad 2 \qquad 76} 
\contentsline {subsection}{\textbf{Pechinkin A.\,V., Sokolov I.\,A., Chaplygin V.\,V.}\ \ Multichannel Queuing System with Finite Buffer and Unreliable Servers}{\qquad 1 \qquad 27} 
\contentsline {subsection}{\textbf{Pechinkin A.\,V., Sokolov I.\,A., Chaplygin V.\,V.}\ \ Stationary Characteristics of a Multichannel Queueing System with\nobreakspace {}Simultaneous Refusals of Servers}{\qquad 2 \qquad 39} 
\contentsline {subsection}{\textbf{Shorgin S.\,Ya.} see Batrakova D.\,A.\hfill\hfill\hfill\hfill\hfill\hfill\hfill\hfill\hfill\hfill\hfill\hfill\hfill\hfill\hfill\hfill\hfill\hfill\hfill\hfill\hfill\hfill\hfill\hfill\hfill\hfill\hfill\hfill\hfill\hfill\hfill\hfill\hfill\hfill\hfill}{\ }
\contentsline {subsection}{\textbf{Shorgin S.\,Ya.} see Kudryavtsev A.\,A.\hfill\hfill\hfill\hfill\hfill\hfill\hfill\hfill\hfill\hfill\hfill\hfill\hfill\hfill\hfill\hfill\hfill\hfill\hfill\hfill\hfill\hfill\hfill\hfill\hfill\hfill\hfill\hfill\hfill\hfill\hfill\hfill\hfill\hfill\hfill}{\ }
\contentsline {subsection}{\textbf{Sinitsyn I.\,N.}\ \ Correlational Methods for Analytical Informational Models of the Earth Pole Fluctuations Design Based on a priori Data}{\qquad 2 \qquad \hphantom{9}2}
\contentsline {subsection}{\textbf{Sinitsyn I.\,N.}\ \ Development of Pugachev Filtering for Stochastic Systems}{\qquad 1 \qquad \hphantom{9}3}
\contentsline {subsection}{\textbf{Sokolov I.\,A.} see Ilyin V.\,D.\hfill\hfill\hfill\hfill\hfill\hfill\hfill\hfill\hfill\hfill\hfill\hfill\hfill\hfill\hfill\hfill\hfill\hfill\hfill\hfill\hfill\hfill\hfill\hfill\hfill\hfill\hfill\hfill\hfill\hfill\hfill\hfill\hfill\hfill\hfill}{\ }
\contentsline {subsection}{\textbf{Sokolov I.\,A.} see Pechinkin A.\,V.\hfill\hfill\hfill\hfill\hfill\hfill\hfill\hfill\hfill\hfill\hfill\hfill\hfill\hfill\hfill\hfill\hfill\hfill\hfill\hfill\hfill\hfill\hfill\hfill\hfill\hfill\hfill\hfill\hfill\hfill\hfill\hfill\hfill\hfill\hfill}{\ }
\contentsline {subsection}{\textbf{Sokolov I.\,A.} see Pechinkin A.\,V.\hfill\hfill\hfill\hfill\hfill\hfill\hfill\hfill\hfill\hfill\hfill\hfill\hfill\hfill\hfill\hfill\hfill\hfill\hfill\hfill\hfill\hfill\hfill\hfill\hfill\hfill\hfill\hfill\hfill\hfill\hfill\hfill\hfill\hfill\hfill}{\ }
\contentsline {subsection}{\textbf{Sokolov I.\,A.} see Zakharov V.\,N.\hfill\hfill\hfill\hfill\hfill\hfill\hfill\hfill\hfill\hfill\hfill\hfill\hfill\hfill\hfill\hfill\hfill\hfill\hfill\hfill\hfill\hfill\hfill\hfill\hfill\hfill\hfill\hfill\hfill\hfill\hfill\hfill\hfill\hfill\hfill}{\ }
\contentsline {subsection}{\textbf{Stupnikov S.\,A.} see Zakharov V.\,N.\hfill\hfill\hfill\hfill\hfill\hfill\hfill\hfill\hfill\hfill\hfill\hfill\hfill\hfill\hfill\hfill\hfill\hfill\hfill\hfill\hfill\hfill\hfill\hfill\hfill\hfill\hfill\hfill\hfill\hfill\hfill\hfill\hfill\hfill\hfill}{\ }
\contentsline {subsection}{\textbf{Zakharov V.\,N., Kalinichenko L.\,A., Sokolov I.\,A., Stupnikov S.\,A.}\ \ Development of Canonical Information Models for Integrated Information Systems}{\qquad 2 \qquad 15} 
\contentsline {subsection}{\textbf{Zakharov V.\,N., Kozmidiady V.\,A.}\ \ Means Providing Applications Fault Tolerance}{\qquad 1 \qquad 14} 
\def\leftfootline{\small{\textbf{\thepage}
\hfill ИНФОРМАТИКА И ЕЁ ПРИМЕНЕНИЯ\ \ \ том~1\ \ \ выпуск~2\ \ \ 2007}
}%
 \def\rightfootline{\small{ИНФОРМАТИКА И ЕЁ ПРИМЕНЕНИЯ\ \ \ том~1\ \ \ выпуск~2\ \ \ 2007
 \hfill \textbf{\thepage}}}
 \label{end\stat} 


%\end{document}

%
\def\stat{rekl}
%\label{preobr}

%\def\tit{АКАДЕМИК ПУГАЧЁВ  ВЛАДИМИР СЕМЁНОВИЧ\\
%25.03.1911--25.03.1998}


%   \vspace*{-48pt}
%   \begin{center}\LARGE
%Академик Пугачёв  Владимир Семёнович\\ (25.03.1911--25.03.1998)
%   \end{center}

   %\vspace*{2.5mm}

   \begin{center}

{\prgsh\LARGE
ЮБИЛЕИ}

\end{center}
%\hrule

\vspace*{6pt}


   \vspace*{8mm}

   \thispagestyle{empty}


%\def\stat{emel}


\section*{К 70-летию заместителя директора ИПИ РАН,\\ члена редколлегии журнала
<<Информатика и её применения>>\\ доктора технических наук В.\,И.~Будзко}

\vspace*{18pt}




          \begin{multicols}{2}

%            \label{st\stat}

\begin{center}
\vspace*{1pt}
\mbox{%
\epsfxsize=78mm
\epsfbox{bud-1.eps}
}
\end{center}

\vspace*{12pt}

      14 августа 2014~г.\ исполнилось 70~лет за\-мес\-ти\-те\-лю директора ИПИ РАН по
научной работе доктору технических наук Владимиру Игоревичу Будзко.

      Владимир Игоревич Будзко родился в г.~Москве. Высшее образование получил на факультете
элект\-рон\-но-вы\-чис\-ли\-тель\-ных устройств в Московском
ин\-же\-нер\-но-фи\-зи\-че\-ском институте
(МИФИ), который он окончил в 1968~г., после чего был на\-прав\-лен для прохождения
службы в одну из войс\-ко\-вых частей, где прошел путь от инженера до первого заместителя
командира войсковой части.

      С приходом В.\,И.~Будзко в ИПИ РАН (2001~г.)\ в институте
сформировалось новое научное на\-прав\-ле\-ние теоретических исследований~--- <<Постро\-ение
ин\-фор\-ма\-ци\-он\-но-те\-ле\-ком\-му\-ни\-ка\-ци\-он\-ных\linebreak сис\-тем
высокой до\-ступ\-ности>>. В~рамках этого
направления выполнен широкий круг фундаментальных исследований по поиску подходов и
определению принципов построения средств обеспечения доступности, конфиденциальности
и целостности современных крупномасштабных
ин\-фор\-ма\-ци\-он\-но-те\-ле\-ком\-му\-ни\-ка\-ци\-он\-ных
сис\-тем (ИТС). Разработаны основные сис\-тем\-но-тех\-ни\-че\-ские принципы и базовые
архитектурные решения построения перспективных для условий России ИТС с
централизованной обработкой и хранением информации, сочетающих в себе свойства
высокой доступности, отказо- и катастрофоустойчивости, информационной защищенности.
Определены принципы, методы и математические основы рационального построения и
оптимизации средств восстановления функционирования центров обработки данных (ЦОД)
после возникновения отказов и катастроф, передачи и хранения данных, обеспечения
информационной безопасности при достижении минимальной совокупной стоимости
владения такими системами. Результаты нашли практическое воплощение при реализации
проектов в интересах ряда отечественных государственных и негосударственных
организаций, таких как Банк России (БР), Внешторгбанк, ОАО <<ГМК <<Норильский Никель>>,
<<Газпром>>, Минэкономразвития России, Правительство Москвы, а также ряд силовых
ведомств.

      Под руководством В.\,И.~Будзко начиная с 2001~г.\ выполнен комплекс
      на\-уч\-но-ис\-сле\-до\-ва\-тель\-ских и
      опыт\-но-кон\-ст\-рук\-тор\-ских работ (свыше 100~проектов),
направленных на развитие электронной информационной технологии БР.
Разработаны концепции развития ИТС БР сначала до 2008~г., а затем до 2013~г., которые
были приняты в качестве основы проведения технической политики. За реализацию проекта
<<Катастрофоустойчивая тер\-ри\-то\-ри\-аль\-но-рас\-пре\-де\-лен\-ная
      ин\-фор\-ма\-ци\-он\-но-те\-ле\-ком\-му\-ни\-ка\-ци\-он\-ная сис\-те\-ма централизованной
обработки банковской информации>> В.\,И.~Будзко удостоен Премии Правительства РФ в
области науки и техники за 2010~г.

      В.\,И.~Будзко возглавлял и возглавляет работы по ряду других прикладных проектов,
связанных с созданием, совершенствованием и развитием крупномасштабных ИТС.

      В.\,И.~Будзко~--- генерал-майор, доктор технических наук, член-кор\-рес\-пон\-дент
Академии криптографии РФ, известный ученый в области информатики и применения
информационных технологий при построении территориально распределенных ИТС
различного назначения. Является автором свыше 250~научных работ, опубликованных в
на\-уч\-но-тех\-ни\-че\-ских и специальных изданиях.

    \thispagestyle{empty}

      В.\,И.~Будзко уделяет большое внимание подготовке научных кадров. Под его
руководством защищено 6~диссертаций на соискание ученой степени кандидата
технических наук. Свыше 30~лет он читает лекции в ИКСИ Академии ФСБ, профессор
кафедры НИЯУ МИФИ. Является членом двух диссертационных советов, главным
редактором журнала <<Системы высокой доступности>> и членом редколлегии журнала
<<Информатика и её применения>>.

      \bigskip

      Редакционный совет и Редакционная коллегия журнала <<Информатика и её
применения>> сердечно поздравляют Владимира Игоревича Будзко с 70-ле\-ти\-ем и желают
крепкого здоровья и новых научных достижений.

\end{multicols}


%
\def\stat{cont}
{%\hrule\par
%\vskip 7pt % 7pt
\raggedleft\Large \bf%\baselineskip=3.2ex
А\,В\,Т\,О\,Р\,С\,К\,И\,Й\ \ У\,К\,А\,З\,А\,Т\,Е\,Л\,Ь\ \ З\,А\ \ 2\,0\,2\,3 г. \vskip 17pt
 \hrule
 \par
\vskip 21pt plus 6pt minus 3pt }

\label{st\stat}

\def\tit{\ }

\def\aut{\ }
\def\auf{\ }

\def\leftkol{\ } % ENGLISH ABSTRACTS}

\def\rightkol{\ } %АВТОРСКИЙ УКАЗАТЕЛЬ ЗА 2021 г.} %ENGLISH ABSTRACTS}

\titele{\tit}{\aut}{\auf}{\leftkol}{\rightkol}
\addcontentsline{toc}{subsection}{\textrm\textbf Авторский указатель за 2023 г.}

\vspace*{-24pt}

\noindent
{\tabcolsep=3pt
\begin{tabular}{p{397pt}cc}
&\textbf{Вып.} & \textbf{Стр.}\\[6pt]
\Avtors{Агаларов~Я.\,М.} Об оптимизации работы резервного прибора в~многолинейной 
системе массового обслуживания&\raisebox{-12pt}[0pt][0pt]{1}&\raisebox{-12pt}[0pt][0pt]{89--95}\\
\Avtors{Агаларов~Я.\,М.} Оптимизация схемы распределения буферной памяти узла 
пакетной коммутации&\raisebox{-12pt}[0pt][0pt]{3}&\raisebox{-12pt}[0pt][0pt]{39--48}\\
\Avtors{Агасандян~Г.\,А.} Многомерные баттерфляи в~задачах оптимизации по CC-VaR&1&107--115\\
\Avtors{Аду~К.\,И.\,Б., Маркова~Е.\,В., Гайдамака~Ю.\,В., Шоргин~С.\,Я.} Анализ схемы 
доступа с~прерыванием при нарезке радиоресурсов сети пятого 
поколения&\raisebox{-12pt}[0pt][0pt]{1}&\raisebox{-12pt}[0pt][0pt]{\hphantom{1}96--106}\\
\Avtors{Архипов~П.\,О., Филиппских~С.\,Л., Цуканов~М.\,В.} Разработка новой модели 
ступенчатой сверточной нейронной сети для классификации аномалий на панорамах&\raisebox{-12pt}[0pt][0pt]{1}&\raisebox{-12pt}[0pt][0pt]{50--56}\\
\Avtors{Бегишев~В.\,О.} см.\ Сопин~Э.\,С.&&\\
\Avtors{Берговин~А.\,К., Ушаков~В.\,Г.} Исследование систем обслуживания со 
смешанными приоритетами&\raisebox{-12pt}[0pt][0pt]{2}&\raisebox{-12pt}[0pt][0pt]{57--61}\\
\Avtors{Борисов~А.\,В.} Рынок с~марковской скачкообразной волатильностью 
I:~мониторинг цены риска как задача оптимальной фильтрации&\raisebox{-12pt}[0pt][0pt]{2}&\raisebox{-12pt}[0pt][0pt]{27--33}\\
\Avtors{Борисов~А.\,В.} Рынок с~марковской скачкообразной волатильностью~II: алгоритм 
вы\-чис\-ле\-ния справедливой цены деривативов&\raisebox{-12pt}[0pt][0pt]{3}&\raisebox{-12pt}[0pt][0pt]{18--24}\\
\Avtors{Борисов А.\,В.} Рынок с марковской скачкообразной волатильностью III:  алгоритм 
мониторинга цены риска по дискретным наблюдениям цен активов&\raisebox{-12pt}[0pt][0pt]{4}&\raisebox{-12pt}[0pt][0pt]{\hphantom{9}9--16}\\
\Avtors{Босов~А.\,В.} Исследование робастности численных аппроксимаций фильтра 
Вонэма&2&41--49\\
\Avtors{Босов~А.\,В.} Оптимальная фильтрация состояния нелинейной динамической 
системы по наблюдениям со случайными запаздываниями&\raisebox{-12pt}[0pt][0pt]{3}&\raisebox{-12pt}[0pt][0pt]{\hphantom{1}8--17}\\
\Avtors{Босов~А.\,В., Иванов~А.\,В.} Технология многофакторной классификации 
математического контента электронной системы обучения&\raisebox{-12pt}[0pt][0pt]{4}&\raisebox{-12pt}[0pt][0pt]{32--41}\\
\Avtors{Босов~А.\,В., Игнатов~А.\,Н.} О~задаче оценки и~анализа риска транспортных 
происшествий на рельсовом транспорте&\raisebox{-12pt}[0pt][0pt]{1}&\raisebox{-12pt}[0pt][0pt]{73--82}\\
\Avtors{Вакуленко~В.\,В., Зацман~И.\,М.} Формализованное описание статистической 
обработки информации в~базах данных&\raisebox{-12pt}[0pt][0pt]{3}&\raisebox{-12pt}[0pt][0pt]{93--99}\\
\Avtors{Васильев~Н.\,С.} Композициональное представление структуры игры многих лиц 
в~моноидальной категории бинарных отношений&\raisebox{-12pt}[0pt][0pt]{2}&\raisebox{-12pt}[0pt][0pt]{18--26}\\
\Avtors{Волканов~Д.\,Ю.} см.\ Горшенин~А.\,К.&&\\
\Avtors{Воронцов~М.\,О., Шестаков~О.\,В.} Среднеквадратичный риск FDR-процедуры 
в~условиях слабой зависимости&\raisebox{-12pt}[0pt][0pt]{2}&\raisebox{-12pt}[0pt][0pt]{34--40}\\
\Avtors{Гайдамака~Ю.\,В.} см.\ Аду~К.\,И.\,Б.&&\\
\Avtors{Гайдамака~Ю.\,В.} см.\ Иванова Д.\,В.&&\\
\Avtors{Гайдамака~Ю.\,В.} см.\ Самуйлов~А.\,К.&&\\
\Avtors{Гаримелла~Р.\,М.} см.\ Разумчик~Р.\,В.&&\\
\Avtors{Гончаров~А.\,А.} Аннотирование параллельных корпусов: подходы и направления 
развития&4&81--87\\
\Avtors{Горбунов~С.\,А.} см.\ Горшенин~А.\,К.&&\\
\Avtors{Горшенин~А.\,К., Горбунов~С.\,А., Волканов~Д.\,Ю.} О~кластеризации объектов 
сетевой вы\-чис\-ли\-тель\-ной инфраструктуры на основе анализа статистических аномалий 
в~трафике&\raisebox{-12pt}[0pt][0pt]{3}&\raisebox{-12pt}[0pt][0pt]{76--87}\\
\Avtors{Грушо~А.\,А., Грушо~Н.\,А., Забежайло~М.\,И., Кульченков~В.\,В., 
Тимонина~Е.\,Е., Шоргин~С.\,Я.} Причинно-следственные связи в~задачах 
классификации&\raisebox{-12pt}[0pt][0pt]{1}&\raisebox{-12pt}[0pt][0pt]{43--49}\\
\Avtors{Грушо~А.\,А., Грушо~Н.\,А., Забежайло~М.\,И., Смирнов~Д.\,В., Тимонина~Е.\,Е.} 
Классификация с~помощью причинно-следственных связей&\raisebox{-12pt}[0pt][0pt]{3}&\raisebox{-12pt}[0pt][0pt]{71--75}\\
\Avtors{Грушо~А.\,А., Грушо~Н.\,А., Забежайло~М.\,И., Тимонина~Е.\,Е., Шоргин~С.\,Я.} 
Сложные причинно-следственные связи&\raisebox{-12pt}[0pt][0pt]{2}&\raisebox{-12pt}[0pt][0pt]{84--89}\\
\end{tabular}
}

\pagebreak

\def\leftkol{АВТОРСКИЙ УКАЗАТЕЛЬ ЗА 2023 г.} % ENGLISH ABSTRACTS}

\def\rightkol{АВТОРСКИЙ УКАЗАТЕЛЬ ЗА 2023 г.} %ENGLISH ABSTRACTS}

%\thispagestyle{myheadings}
\def\leftfootline{\small{\textbf{\thepage}
\hfill ИНФОРМАТИКА И ЕЁ ПРИМЕНЕНИЯ\ \ \ том~17\ \ \ выпуск~4\ \ \ 2023}
}%
 \def\rightfootline{\small{ИНФОРМАТИКА И ЕЁ ПРИМЕНЕНИЯ\ \ \ том~17\ \ \ выпуск~4\ \ \ 2023
 \hfill \textbf{\thepage}}}


\noindent
{\tabcolsep=3pt
\begin{tabular}{p{394pt}cc}
&\textbf{Вып.} & \textbf{Стр.}\\[3pt]
\Avtors{Грушо~Н.\,А.} см.\ Грушо~А.\,А.&&\\
\Avtors{Грушо~Н.\,А.} см.\ Грушо~А.\,А.&&\\
\Avtors{Грушо~Н.\,А.} см.\ Грушо~А.\,А.&&\\
\Avtors{Дулин~С.\,К.} см.\ Розенберг~И.\,Н.&&\\
\Avtors{Дулина~Н.\,Г.} см.\ Розенберг~И.\,Н.&&\\
\Avtors{Дюкова~А.\,П.} см.\ Дюкова~Е.\,В.&&\\
\Avtors{Дюкова~Е.\,В., Масляков~Г.\,О., Дюкова~А.\,П.} Логические методы корректной 
классификации данных&\raisebox{-12pt}[0pt][0pt]{3}&\raisebox{-12pt}[0pt][0pt]{64--70}\\
\Avtors{Забежайло~М.\,И.} см.\ Грушо~А.\,А&&\\
\Avtors{Забежайло~М.\,И.} см.\ Грушо~А.\,А.&&\\
\Avtors{Забежайло~М.\,И.} см.\ Грушо~А.\,А.&&\\
\Avtors{Захаров~В.\,Н.} см.\ Сазонтьев В.\,В.&&\\
\Avtors{Захаров В.\,Н.} см.\ Френкель С.\,Л.&&\\
\Avtors{Зацман~И.\,М.} Данные, информация и~знание в~научной парадигме 
информатики&1&116--125\\
\Avtors{Зацман И.\,М.} Научная парадигма информатики: классификация объектов 
предметной области&\raisebox{-12pt}[0pt][0pt]{4}&\raisebox{-12pt}[0pt][0pt]{\hphantom{9}96--103}\\
\Avtors{Зацман~И.\,М.} Трансформация иерархии Акоффа в~научной парадигме 
информатики&3&107--113\\
\Avtors{Зацман~И.\,М.} см.\ Вакуленко~В.\,В.&&\\
\Avtors{Зейфман~А.\,И.} см.\ Усов~И.\,А.&&\\
\Avtors{Иванов~А.\,В.} см.\ Босов~А.\,В.&&\\
\Avtors{Иванова Д.\,В., Маркова Е.\,В., Шоргин~С.\,Я., Гайдамака~Ю.\,В.} Модели 
совместного обслуживания трафика eMBB и URLLC на основе приоритетов в 
промышленных развертываниях 5G NR&\raisebox{-24pt}[0pt][0pt]{4}&\raisebox{-24pt}[0pt][0pt]{64--70}\\
\Avtors{Игнатов~А.\,Н.} см.\ Босов~А.\,В.&&\\
\Avtors{Инькова~О.\,Ю., Кружков~М.\,Г.} Критерии определения семантической близости 
дискурсивных отношений&\raisebox{-12pt}[0pt][0pt]{3}&\raisebox{-12pt}[0pt][0pt]{100--106}\\
\Avtors{Инькова О.\,Ю., Кружков~М.\,Г.} Степень семантической близости дискурсивных 
отношений:  методы и инструменты расчета&\raisebox{-12pt}[0pt][0pt]{4}&\raisebox{-12pt}[0pt][0pt]{88--95}\\
\Avtors{Кабанов~Ю.\,М., Сидоренко~А.\,П.} Аксиоматический взгляд на модели системного 
риска Роджерса--Вераарт и~Судзуки--Эльсингера&\raisebox{-12pt}[0pt][0pt]{1}&\raisebox{-12pt}[0pt][0pt]{11--17}\\
\Avtors{Карпов~В.\,И.} см.\ Нуриев~В.\,А.&&\\
\Avtors{Кириков~И.\,А.} см.\ Листопад~С.\,В.&&\\
\Avtors{Ковалёв~С.\,П.} Монада диаграмм как математическая метамодель системной 
инженерии&2&11--17\\
\Avtors{Королев~Д.\,О., Малеев~О.\,Г.} Исследование эффективности применения бинарных 
нейронных сетей при детектировании объекта на изображении&\raisebox{-12pt}[0pt][0pt]{3}&\raisebox{-12pt}[0pt][0pt]{88--92}\\
\Avtors{Кривенко~М.\,П.} Критерии выбора размерности модели факторизации&2&50--56\\
\Avtors{Кружков~М.\,Г.} см.\ Инькова О.\,Ю.&&\\
\Avtors{Кружков~М.\,Г.} см.\ Инькова~О.\,Ю.&&\\
\Avtors{Кудрявцев~А.\,А., Шестаков~О.\,В.} Метод оценивания параметров 
гамма-экс\-по\-нен\-ци\-аль\-но\-го распределения по выборке со слабо зависимыми компонентами&\raisebox{-12pt}[0pt][0pt]{3}&\raisebox{-12pt}[0pt][0pt]{58--62}\\
\Avtors{Кульченков~В.\,В.} см.\ Грушо~А.\,А.&&\\
\Avtors{Лапко~А.\,В.} см.\ Тубольцев~В.\,П.&&\\
\Avtors{Лапко~В.\,А.} см.\ Тубольцев~В.\,П.&&\\
\Avtors{Лери~М.\,М.} Среднее расстояние в~конфигурационных графах со степенным 
распределением&\raisebox{-12pt}[0pt][0pt]{1}&\raisebox{-12pt}[0pt][0pt]{28--34}\\
\Avtors{Листопад~С.\,В., Кириков~И.\,А.} Метод на основе нечетких правил для 
управления конфликтами агентов в~гибридных интеллектуальных многоагентных 
системах&\raisebox{-12pt}[0pt][0pt]{1}&\raisebox{-12pt}[0pt][0pt]{66--72}\\
\Avtors{Малашенко~Ю.\,Е., Назарова~И.\,А.} Анализ загрузки многопользовательской сети 
при расщеплении потоков по кратчайшим маршрутам&\raisebox{-12pt}[0pt][0pt]{3}&\raisebox{-12pt}[0pt][0pt]{33--38}\\
\Avtors{Малашенко~Ю.\,Е., Назарова~И.\,А.} Оценки распределения ресурсов 
в~многопользовательской сети при равных межузловых нагрузках&\raisebox{-12pt}[0pt][0pt]{1}&\raisebox{-12pt}[0pt][0pt]{83--88}\\
\Avtors{Малеев~О.\,Г.} см.\ Королев~Д.\,О.&&\\
\Avtors{Маркова~Е.\,В.} см.\ Аду~К.\,И.\,Б.&&\\
\Avtors{Маркова Е.\,В.} см.\ Иванова Д.\,В.&&\\
\end{tabular}
}

\pagebreak

\def\leftkol{АВТОРСКИЙ УКАЗАТЕЛЬ ЗА 2023 г.} % ENGLISH ABSTRACTS}

\def\rightkol{АВТОРСКИЙ УКАЗАТЕЛЬ ЗА 2023 г.} %ENGLISH ABSTRACTS}

%\thispagestyle{myheadings}
\def\leftfootline{\small{\textbf{\thepage}
\hfill ИНФОРМАТИКА И ЕЁ ПРИМЕНЕНИЯ\ \ \ том~17\ \ \ выпуск~4\ \ \ 2023}
}%
 \def\rightfootline{\small{ИНФОРМАТИКА И ЕЁ ПРИМЕНЕНИЯ\ \ \ том~17\ \ \ выпуск~4\ \ \ 2023
 \hfill \textbf{\thepage}}}


\noindent
{\tabcolsep=3pt
\begin{tabular}{p{394pt}cc}
&\textbf{Вып.} & \textbf{Стр.}\\[3pt]
\Avtors{Маслов~А.\,Р.} см.\ Сопин~Э.\,С&&\\
\Avtors{Масляков~Г.\,О.} см.\ Дюкова~Е.\,В.&&\\
\Avtors{Мелехин~В.\,Б., Хачумов~В.\,М., Хачумов~М.\,В.} Самообучение автономных 
интеллектуальных роботов в~процессе поисково-исследовательской деятельности&\raisebox{-12pt}[0pt][0pt]{2}&\raisebox{-12pt}[0pt][0pt]{78--83}\\
\Avtors{Назарова~И.\,А.} см.\ Малашенко~Ю.\,Е.&&\\
\Avtors{Назарова~И.\,А.} см.\ Малашенко~Ю.\,Е.&&\\
\Avtors{Нейчев~Р.\,Г., Шибаев~И.\,А., Стрижов~В.\,В.} Восстановление матрицы 
суперпозиции в~задаче символьной регрессии&\raisebox{-12pt}[0pt][0pt]{1}&\raisebox{-12pt}[0pt][0pt]{35--42}\\
\Avtors{Нуриев~В.\,А., Карпов~В.\,И.} Методология корпусно-ориентированного 
исследования в~области контрастивной пунктуации&\raisebox{-12pt}[0pt][0pt]{2}&\raisebox{-12pt}[0pt][0pt]{90--95}\\
\Avtors{Пешкова И.\,В.} Границы незавершенной работы в системе с повторными вызовами 
разных классов и показательным временем обслуживания&\raisebox{-12pt}[0pt][0pt]{4}&\raisebox{-12pt}[0pt][0pt]{57--63}\\
\Avtors{Платонова~А.\,А.} см.\ Самуйлов~А.\,К.&&\\
\Avtors{Рабинович Я.\,И.} Процедура построения множества Парето для дифференцируемых 
критериальных функций&\raisebox{-12pt}[0pt][0pt]{4}&\raisebox{-12pt}[0pt][0pt]{17--22}\\
\Avtors{Разумчик~Р.\,В., Румянцев~А.\,С., Гаримелла~Р.\,М.} Вероятностная модель для 
оценки основных характеристик производительности марковской модели 
суперкомпьютера&\raisebox{-24pt}[0pt][0pt]{2}&\raisebox{-24pt}[0pt][0pt]{62--70}\\
\Avtors{Розенберг~И.\,Н., Дулин~С.\,К., Дулина~Н.\,Г.} Моделирование структуры 
интероперабельности средствами структурной согласованности&\raisebox{-12pt}[0pt][0pt]{1}&\raisebox{-12pt}[0pt][0pt]{57--65}\\
\Avtors{Румовская~С.\,Б.} Подходы к~подбору специалистов при организации 
коллективного решения проблем&\raisebox{-12pt}[0pt][0pt]{2}&\raisebox{-12pt}[0pt][0pt]{\hphantom{1}96--103}\\
\Avtors{Румянцев~А.\,С.} см.\ Разумчик~Р.\,В.&&\\
\Avtors{Сазонтьев В.\,В., Ступников~С.\,А., Захаров~В.\,Н.} Расширяемый подход к слиянию 
данных в распределенных вычислительных средах&\raisebox{-12pt}[0pt][0pt]{4}&\raisebox{-12pt}[0pt][0pt]{42--47}\\
\Avtors{Самуйлов~А.\,К., Платонова~А.\,А., Шоргин~В.\,С., Гайдамака~Ю.\,В.} 
К~моделированию эффектов обслуживания многоадресного трафика в~сетях 5G~NR&\raisebox{-12pt}[0pt][0pt]{2}&\raisebox{-12pt}[0pt][0pt]{71--77}\\
\Avtors{Сатин~Я.\,А.} см.\ Усов~И.\,А.&&\\
\Avtors{Сидоренко~А.\,П.} см.\ Кабанов~Ю.\,М.&&\\
\Avtors{Синицын~И.\,Н.} Аналитическое моделирование распределений с~инвариантной 
мерой в~стохастических системах, не разрешенных относительно 
производных&\raisebox{-12pt}[0pt][0pt]{1}&\raisebox{-12pt}[0pt][0pt]{\hphantom{1}2--10}\\
\Avtors{Смирнов~Д.\,В.} см.\ Грушо~А.\,А.&&\\
\Avtors{Сопин~Э.\,С., Маслов~А.\,Р., Шоргин~В.\,С., Бегишев~В.\,О.} Моделирование 
настойчивого поведения пользователей в~сетях 5G NR с~адаптацией скорости 
и~блокировками&\raisebox{-12pt}[0pt][0pt]{3}&\raisebox{-12pt}[0pt][0pt]{25--32}\\
\Avtors{Степанов~Е.\,П.} см.\ Шестаков~О.\,В.&&\\
\Avtors{Стрижов~В.\,В.} см.\ Нейчев~Р.\,Г.&&\\
\Avtors{Ступников~С.\,А.} см.\ Сазонтьев В.\,В.&&\\
\Avtors{Тимонина~Е.\,Е.} см.\ Грушо~А.\,А.&&\\
\Avtors{Тимонина~Е.\,Е.} см.\ Грушо~А.\,А.&&\\
\Avtors{Тимонина~Е.\,Е.} см.\ Грушо~А.\,А.&&\\
\Avtors{Торшин~И.\,Ю.} О~задачах оптимизации, возникающих при применении 
топологического анализа данных к~поиску алгоритмов прогнозирования с~фиксированными 
корректорами&\raisebox{-24pt}[0pt][0pt]{2}&\raisebox{-24pt}[0pt][0pt]{\hphantom{1}2--10}\\
\Avtors{Торшин~И.\,Ю.} О~формировании множеств прецедентов на основе таблиц 
разнородных признаковых описаний методами топологической теории анализа 
данных&\raisebox{-12pt}[0pt][0pt]{3}&\raisebox{-12pt}[0pt][0pt]{2--7}\\
\Avtors{Тубольцев~В.\,П., Лапко~А.\,В., Лапко~В.\,А.} Непараметрический алгоритм 
автоматической классификации данных дистанционного зондирования&\raisebox{-12pt}[0pt][0pt]{4}&\raisebox{-12pt}[0pt][0pt]{23--31}\\
\Avtors{Усов~И.\,А., Сатин~Я.\,А., Зейфман~А.\,И.} О~скорости сходимости и~предельных 
характеристиках для одного обобщенного процесса рождения и~гибели&\raisebox{-12pt}[0pt][0pt]{3}&\raisebox{-12pt}[0pt][0pt]{49--57}\\
\Avtors{Ушаков~В.\,Г., Ушаков~Н.\,Г.} Критерии нормальности вероятностного 
распределения при округленных данных&\raisebox{-12pt}[0pt][0pt]{1}&\raisebox{-12pt}[0pt][0pt]{18--27}\\
\Avtors{Ушаков~В.\,Г.} см.\ Берговин~А.\,К.&&\\
\Avtors{Ушаков~Н.\,Г.} см.\ Ушаков~В.\,Г.&&\\
\Avtors{Филиппских~С.\,Л.} см.\ Архипов~П.\,О.&&\\
\end{tabular}
}

\pagebreak

\def\leftkol{АВТОРСКИЙ УКАЗАТЕЛЬ ЗА 2023 г.} % ENGLISH ABSTRACTS}

\def\rightkol{АВТОРСКИЙ УКАЗАТЕЛЬ ЗА 2023 г.} %ENGLISH ABSTRACTS}

%\thispagestyle{myheadings}
\def\leftfootline{\small{\textbf{\thepage}
\hfill ИНФОРМАТИКА И ЕЁ ПРИМЕНЕНИЯ\ \ \ том~17\ \ \ выпуск~4\ \ \ 2023}
}%
 \def\rightfootline{\small{ИНФОРМАТИКА И ЕЁ ПРИМЕНЕНИЯ\ \ \ том~17\ \ \ выпуск~4\ \ \ 2023
 \hfill \textbf{\thepage}}}


\noindent
{\tabcolsep=3pt
\begin{tabular}{p{394pt}cc}
&\textbf{Вып.} & \textbf{Стр.}\\[3pt]
\Avtors{Френкель С.\,Л., Захаров В.\,Н.} Модели учета влияния статистических 
характеристик трафика вычислительных сетей на эффективность прогнозирования 
средствами машинного обучения&\raisebox{-24pt}[0pt][0pt]{4}&\raisebox{-24pt}[0pt][0pt]{71--80}\\
\Avtors{Хачумов~В.\,М.} см.\ Мелехин~В.\,Б.&&\\
\Avtors{Хачумов~М.\,В.} см.\ Мелехин~В.\,Б.&&\\
\Avtors{Цуканов~М.\,В.} см.\ Архипов~П.\,О.&&\\
\Avtors{Шестаков~О.\,В., Степанов~Е.\,П.} Нелинейная регуляризация обращения линейных 
однородных операторов с помощью метода блочной пороговой обработки&\raisebox{-12pt}[0pt][0pt]{4}&\raisebox{-12pt}[0pt][0pt]{2--8}\\
\Avtors{Шестаков~О.\,В.} см.\ Воронцов~М.\,О.&&\\
\Avtors{Шестаков~О.\,В.} см.\ Кудрявцев~А.\,А.&&\\
\Avtors{Шибаев~И.\,А.} см.\ Нейчев~Р.\,Г.&&\\
\Avtors{Шнурков П.\,В.} Решение задачи оптимального управления запасом непрерывного 
продукта в стохастической модели регенерации со случайными стоимостными 
характеристиками&\raisebox{-24pt}[0pt][0pt]{4}&\raisebox{-24pt}[0pt][0pt]{48--56}\\
\Avtors{Шоргин~В.\,С.} см.\ Самуйлов~А.\,К.&&\\
\Avtors{Шоргин~В.\,С.} см.\ Сопин~Э.\,С.&&\\
\Avtors{Шоргин~С.\,Я.} см.\ Аду~К.\,И.\,Б.&&\\
\Avtors{Шоргин~С.\,Я.} см.\ Грушо~А.\,А.&&\\
\Avtors{Шоргин~С.\,Я.} см.\ Грушо~А.\,А.&&\\
\Avtors{Шоргин~С.\,Я.} см.\ Иванова Д.\,В.&&\\


\end{tabular}
}

%\thispagestyle{myheadings}
\def\leftfootline{\small{\textbf{\thepage}
\hfill ИНФОРМАТИКА И ЕЁ ПРИМЕНЕНИЯ\ \ \ том~17\ \ \ выпуск~4\ \ \ 2023}
}%
 \def\rightfootline{\small{ИНФОРМАТИКА И ЕЁ ПРИМЕНЕНИЯ\ \ \ том~17\ \ \ выпуск~4\ \ \ 2023
 \hfill \textbf{\thepage}}}

 \label{end\stat}

\newpage

\def\stat{cont-e}
{%\hrule\par
%\vskip 7pt % 7pt
\raggedleft\Large \bf%\baselineskip=3.2ex
2\,0\,2\,3\ \ A\,U\,T\,H\,O\,R\ \ I\,N\,D\,E\,X \vskip 17pt
 \hrule
 \par
\vskip 21pt plus 6pt minus 3pt }

\label{st\stat}

\def\tit{\ }

\def\aut{\ }
\def\auf{\ }

\def\leftkol{\ } %2021 AUTHOR INDEX} % ENGLISH ABSTRACTS}

\def\rightkol{\ } %2021 AUTHOR INDEX} %ENGLISH ABSTRACTS}

\titele{\tit}{\aut}{\auf}{\leftkol}{\rightkol}
\addcontentsline{toc}{subsection}{\textrm\textbf 2023 Author Index}

\def\leftfootline{\small{\textbf{\thepage}
\hfill INFORMATIKA I EE PRIMENENIYA~--- INFORMATICS AND APPLICATIONS\ \ \ 2023\
\ \ volume~17\ \ \ issue\ 4}
}%
 \def\rightfootline{\small{INFORMATIKA I EE PRIMENENIYA~--- INFORMATICS AND APPLICATIONS\ \ \ 2023\ \ \ volume~17\ \ \ issue\ 4
\hfill \textbf{\thepage}}}

\vspace*{-24pt}

\noindent
{\tabcolsep=3pt
\begin{tabular}{p{395.89pt}cc}
&\textbf{Issue} & \textbf{Page}\\[6pt]
\Avtors{Adou~K.\,Y.\,B., Markova~E.\,V., Gaidamaka~Yu.\,V., and~Shorgin~S.\,Ya.} 
Preemption-based prioritization scheme for network resources slicing in 5G 
systems&\raisebox{-12pt}[0pt][0pt]{1}&\raisebox{-12pt}[0pt][0pt]{\hphantom{1}96--106}\\
\Avtors{Agalarov~Ya.\,M.} Optimization of a queue-length dependent additional server in the 
multiserver queue&\raisebox{-12pt}[0pt][0pt]{1}&\raisebox{-12pt}[0pt][0pt]{89--95}\\
\Avtors{Agalarov~Ya.\,M.} Optimization of the buffer memory allocation scheme of the packet 
switching node&\raisebox{-12pt}[0pt][0pt]{3}&\raisebox{-12pt}[0pt][0pt]{39--48}\\
\Avtors{Agasandyan~G.\,A.} Multidimensional butterflies in problems of optimization on CC-VaR&1&107--115\\
\Avtors{Arkhipov~P.\,O., Philippskih~S.\,L., and~Tsukanov~M.\,V.} Development of a~new model 
of a~step convolutional neural network for classification of anomalies on panoramas&\raisebox{-12pt}[0pt][0pt]{1}&\raisebox{-12pt}[0pt][0pt]{50--56}\\
\Avtors{Begishev~V.\,O.} see Sopin~E.\,S.&&\\
\Avtors{Bergovin~A.\,K. and~Ushakov~V.\,G.} Analysis of the queueing systems with mixed 
priorities&2&57--61\\
\Avtors{Borisov~A.\,V.} Market with Markov jump volatility I: Price of risk monitoring as an 
optimal filtering problem&\raisebox{-12pt}[0pt][0pt]{2}&\raisebox{-12pt}[0pt][0pt]{27--33}\\
\Avtors{Borisov~A.\,V.} Market with Markov jump volatility~II: Algorithm of derivative fair price 
calculation&3&18--24\\
\Avtors{Borisov A.\,V.} Market with Markov jump volatility III: Price of risk monitoring algorithm 
given discrete-time observations of asset prices&\raisebox{-12pt}[0pt][0pt]{4}&\raisebox{-12pt}[0pt][0pt]{\hphantom{9}9--16}\\
\Avtors{Bosov~A.\,V.} Nonlinear dynamic system state optimal filtering by observations with 
random delays&\raisebox{-12pt}[0pt][0pt]{3}&\raisebox{-12pt}[0pt][0pt]{\hphantom{1}8--17}\\
\Avtors{Bosov~A.\,V.} Robustness investigation of the numerical approximation of the Wonham 
filter&2&41--49\\
\Avtors{Bosov~A.\,V. and~Ignatov~A.\,N.} On the problem of assessing and analyzing traffic 
accidents risk on the rail transport&\raisebox{-12pt}[0pt][0pt]{1}&\raisebox{-12pt}[0pt][0pt]{73--82}\\
\Avtors{Bosov~A.\,V. and Ivanov~A.\,V.} Multifactor classification technology of mathematical 
content of e-learning system&\raisebox{-12pt}[0pt][0pt]{4}&\raisebox{-12pt}[0pt][0pt]{32--41}\\
\Avtors{Djukova~A.\,P.} see Djukova~E.\,V.&&\\
\Avtors{Djukova~E.\,V., Masliakov~G.\,O., and Djukova~A.\,P.} Logical methods of correct data 
classification&3&64--70\\
\Avtors{Dulin~S.\,K.} see Rozenberg~I.\,N.&&\\
\Avtors{Dulina~N.\,G.} see Rozenberg~I.\,N.&&\\
\Avtors{Frenkel~S.\,L. and Zakharov~V.\,N.} Models for study of the influence of statistical 
characteristics of computer networks traffic on the efficiency of prediction by machine learning 
tools&\raisebox{-12pt}[0pt][0pt]{4}&\raisebox{-12pt}[0pt][0pt]{71--80}\\
\Avtors{Gaidamaka~Yu.\,V.} see Adou~K.\,Y.\,B.&&\\
\Avtors{Gaidamaka~Yu.\,V.} see Ivanova~D.\,V.&&\\
\Avtors{Gaidamaka~Yu.\,V.} see Samouylov~A.\,K.&&\\
\Avtors{Garimella~R.\,M.} see Razumchik~R.\,V.&&\\
\Avtors{Goncharov~A.\,A.} Parallel corpus annotation: Approaches and directions for 
development&4&81--87\\
\Avtors{Gorbunov~S.\,A.} see Gorshenin~A.\,K.&&\\
\Avtors{Gorshenin~A.\,K., Gorbunov~S.\,A., and Volkanov~D.\,Yu.} Toward clustering of 
network computing infrastructure objects based on analysis of statistical anomalies in network 
traffic&\raisebox{-12pt}[0pt][0pt]{3}&\raisebox{-12pt}[0pt][0pt]{76--87}\\
\Avtors{Grusho~A.\,A., Grusho~N.\,A., Zabezhailo~M.\,I., Kulchenkov~V.\,V., Timonina~E.\,E., 
and~Shorgin~S.\,Ya.} Causal relationships in classification problems&\raisebox{-12pt}[0pt][0pt]{1}&\raisebox{-12pt}[0pt][0pt]{43--49}\\
\Avtors{Grusho~A.\,A., Grusho~N.\,A., Zabezhailo~M.\,I., Smirnov~D.\,V., and Timonina~E.\,E.} 
Classification by cause-and-effect relationships&\raisebox{-12pt}[0pt][0pt]{3}&\raisebox{-12pt}[0pt][0pt]{71--75}\\
\Avtors{Grusho~A.\,A., Grusho~N.\,A., Zabezhailo~M.\,I., Timonina~E.\,E., 
and~Shorgin~S.\,Ya.} Complex cause-and-effect relationships&\raisebox{-12pt}[0pt][0pt]{2}&\raisebox{-12pt}[0pt][0pt]{84--89}\\
\Avtors{Grusho~N.\,A.} see Grusho~A.\,A.&&\\
\Avtors{Grusho~N.\,A.} see Grusho~A.\,A.&&\\
\Avtors{Grusho~N.\,A.} see Grusho~A.\,A.&&\\
\end{tabular}
}
\pagebreak

\def\leftfootline{\small{\textbf{\thepage}
\hfill INFORMATIKA I EE PRIMENENIYA~--- INFORMATICS AND APPLICATIONS\ \ \ 2023\
\ \ volume~17\ \ \ issue\ 4}
}%
 \def\rightfootline{\small{INFORMATIKA I EE PRIMENENIYA~---
INFORMATICS AND APPLICATIONS\ \ \ 2023\ \ \ volume~17\ \ \ issue\ 4
\hfill \textbf{\thepage}}}

\def\leftkol{2023 AUTHOR INDEX} % ENGLISH ABSTRACTS}

\def\rightkol{2023 AUTHOR INDEX} %ENGLISH ABSTRACTS}


\noindent
{\tabcolsep=3pt
\begin{tabular}{p{395.5pt}cc}
&\textbf{Issue} & \textbf{Page}\\[6pt]
\Avtors{Ignatov~A.\,N.} see Bosov~A.\,V.&&\\
\Avtors{Inkova O.\,Yu. and Kruzhkov M.\,G.} Evaluating the degree of discourse relations 
semantic affinity: Methods and instruments&\raisebox{-12pt}[0pt][0pt]{4}&\raisebox{-12pt}[0pt][0pt]{88--95}\\
\Avtors{Inkova~O.\,Yu. and Kruzhkov~M.\,G.} Evaluation criteria for discourse relations semantic 
affinity&3&100--106\\
\Avtors{Kruzhkov~M.\,G.} see Inkova~O.\,Yu.&&\\
\Avtors{Ivanov~A.\,V.} see Bosov~A.\,V.&&\\
\Avtors{Ivanova~D.\,V., Markova~E.\,V., Shorgin~S.\,Ya., and~Gaidamaka~Yu.\,V.} Priority-based 
eMBB and URLLC traffic coexistence models in 5G NR industrial deployments&\raisebox{-12pt}[0pt][0pt]{4}&\raisebox{-12pt}[0pt][0pt]{64--70}\\
\Avtors{Kabanov~Yu.\,M. and~Sidorenko~A.\,P.} An axiomatic viewpoint on the Rogers--Veraart 
and Suzuki--Elsinger models of systemic risk&\raisebox{-12pt}[0pt][0pt]{1}&\raisebox{-12pt}[0pt][0pt]{11--17}\\
\Avtors{Karpov~V.\,I.} see Nuriev~V.\,A.&&\\
\Avtors{Khachumov~M.\,V.} see Melekhin~V.\,B.&&\\
\Avtors{Khachumov~V.\,M.} see Melekhin~V.\,B.&&\\
\Avtors{Kirikov~I.\,A.} see Listopad~S.\,V.&&\\
\Avtors{Korolev~D.\,O. and Maleev~O.\,G.} Efficiency of binary neural networks for object 
detection on an image&\raisebox{-12pt}[0pt][0pt]{3}&\raisebox{-12pt}[0pt][0pt]{88--92}\\
\Avtors{Kovalyov~S.\,P.} The monad of diagrams as a mathematical metamodel of systems 
engineering&2&11--17\\
\Avtors{Krivenko~M.\,P.} Criteria for choosing the factorization model dimensionality&2&50--56\\
\Avtors{Kruzhkov M.\,G.} see Inkova O.\,Yu.&&\\
\Avtors{Kudryavtsev~A.\,A. and Shestakov~O.\,V.} A~method for estimating parameters of the 
gamma-exponential distribution from a~sample with weakly dependent components&\raisebox{-12pt}[0pt][0pt]{3}&\raisebox{-12pt}[0pt][0pt]{58--63}\\
\Avtors{Kulchenkov~V.\,V.} see Grusho~A.\,A.&&\\
\Avtors{Lapko~A.\,V.} see Tuboltsev V.\,P.&&\\
\Avtors{Lapko~V.\,A.} see Tuboltsev V.\,P.&&\\
\Avtors{Leri~M.\,M.} An average distance in the power-law configuration graphs&1&28--34\\
\Avtors{Listopad~S.\,V. and~Kirikov~I.\,A.} Fuzzy rules based method for agent conflict 
management in hybrid intelligent multiagent systems&\raisebox{-12pt}[0pt][0pt]{1}&\raisebox{-12pt}[0pt][0pt]{66--72}\\
\Avtors{Malashenko~Yu.\,E. and~Nazarova~I.\,A.} Estimates of the resource distribution in the 
multiuser network with equal internodal loads&\raisebox{-12pt}[0pt][0pt]{1}&\raisebox{-12pt}[0pt][0pt]{83--88}\\
\Avtors{Malashenko~Yu.\,E. and Nazarova~I.\,A.} Multiuser network load analysis by splitting 
flows along the shortest routes&\raisebox{-12pt}[0pt][0pt]{3}&\raisebox{-12pt}[0pt][0pt]{33--38}\\
\Avtors{Maleev~O.\,G.} see Korolev~D.\,O.&&\\
\Avtors{Markova~E.\,V.} see Adou~K.\,Y.\,B.&&\\
\Avtors{Markova~E.\,V.} see Ivanova~D.\,V.&&\\
\Avtors{Masliakov~G.\,O.} see Djukova~E.\,V.&&\\
\Avtors{Maslov~A.\,R.} see Sopin~E.\,S.&&\\
\Avtors{Melekhin~V.\,B., Khachumov~V.\,M., and~Khachumov~M.\,V.} Self-learning of 
autonomous intelligent robots in the process of search and explore activities&\raisebox{-12pt}[0pt][0pt]{2}&\raisebox{-12pt}[0pt][0pt]{78--83}\\
\Avtors{Nazarova~I.\,A.} see Malashenko~Yu.\,E.&&\\
\Avtors{Nazarova~I.\,A.} see Malashenko~Yu.\,E.&&\\
\Avtors{Neychev~R.\,G., Shibaev~I.\,A., and~Strijov~V.\,V.} Optimal spanning tree reconstruction 
in symbolic regression&\raisebox{-12pt}[0pt][0pt]{1}&\raisebox{-12pt}[0pt][0pt]{35--42}\\
\Avtors{Nuriev~V.\,A. and~Karpov~V.\,I.} Methodology of the corpus-based studies in the field of 
contrastive punctuation&\raisebox{-12pt}[0pt][0pt]{2}&\raisebox{-12pt}[0pt][0pt]{90--95}\\
\Avtors{Peshkova~I.\,V.} Bounds of the workload in a~multiclass retrial queue with exponential 
services&\raisebox{-12pt}[0pt][0pt]{4}&\raisebox{-12pt}[0pt][0pt]{57--63}\\
\Avtors{Philippskih~S.\,L.} see Arkhipov~P.\,O.&&\\
\Avtors{Platonova~A.\,A.} see Samouylov~A.\,K.&&\\
\Avtors{Rabinovich Ya.\,I.} Procedure of constructing a~Pareto set for differentiable criteria 
functions&4&17--22\\
\Avtors{Razumchik~R.\,V., Rumyantsev~A.\,S., and~Garimella~R.\,M.} A~queueing system for 
performance evaluation of a~Markovian supercomputer model&\raisebox{-12pt}[0pt][0pt]{2}&\raisebox{-12pt}[0pt][0pt]{62--70}\\
\Avtors{Rozenberg~I.\,N., Dulin~S.\,K., and~Dulina~N.\,G.} Modeling the structure of 
interoperability by means of structural consistency&\raisebox{-12pt}[0pt][0pt]{1}&\raisebox{-12pt}[0pt][0pt]{57--65}\\
\Avtors{Rumovskaya~S.\,B.} Selection of specialists in the organization of collective solving 
problems&2&\hphantom{1}96--103\\
\end{tabular}
}
\pagebreak

\def\leftfootline{\small{\textbf{\thepage}
\hfill INFORMATIKA I EE PRIMENENIYA~--- INFORMATICS AND APPLICATIONS\ \ \ 2023\
\ \ volume~17\ \ \ issue\ 4}
}%
 \def\rightfootline{\small{INFORMATIKA I EE PRIMENENIYA~---
INFORMATICS AND APPLICATIONS\ \ \ 2023\ \ \ volume~17\ \ \ issue\ 4
\hfill \textbf{\thepage}}}

\def\leftkol{2023 AUTHOR INDEX} % ENGLISH ABSTRACTS}

\def\rightkol{2023 AUTHOR INDEX} %ENGLISH ABSTRACTS}


\noindent
{\tabcolsep=3pt
\begin{tabular}{p{395.5pt}cc}
&\textbf{Issue} & \textbf{Page}\\[6pt]
\Avtors{Rumyantsev~A.\,S.} see Razumchik~R.\,V.&&\\
\Avtors{Samouylov~A.\,K., Platonova~A.\,A., Shorgin~V.\,S., and~Gaidamaka~Yu.\,V.} On 
modeling the effects of multicast traffic servicing in 5G NR networks&\raisebox{-12pt}[0pt][0pt]{2}&\raisebox{-12pt}[0pt][0pt]{71--77}\\
\Avtors{Satin~Y.\,A.} see Usov~I.\,A.&&\\
\Avtors{Sazontev V.\,V., Stupnikov~S.\,A., and~Zakharov~V.\,N.} An extensible approach to data 
fusion in~distributed computing environments&\raisebox{-12pt}[0pt][0pt]{4}&\raisebox{-12pt}[0pt][0pt]{42--47}\\
\Avtors{Shestakov~O.\,V. and Stepanov~E.\,P.} Nonlinear regularization of the inversion of linear 
homogeneous operators using the block thresholding method&\raisebox{-12pt}[0pt][0pt]{4}&\raisebox{-12pt}[0pt][0pt]{2--8}\\
\Avtors{Shestakov~O.\,V.} see Kudryavtsev~A.\,A.&&\\
\Avtors{Shestakov~O.\,V.} see Vorontsov~M.\,O.&&\\
\Avtors{Shibaev~I.\,A.} see Neychev~R.\,G.&&\\
\Avtors{Shnurkov P.\,V.} Solution of the problem of~optimal control of~the stock of a~continuous 
product in a~stochastic model of regeneration with random cost characteristics&\raisebox{-12pt}[0pt][0pt]{4}&\raisebox{-12pt}[0pt][0pt]{48--56}\\
\Avtors{Shorgin~S.\,Ya.} see Adou~K.\,Y.\,B.&&\\
\Avtors{Shorgin~S.\,Ya.} see Grusho~A.\,A.&&\\
\Avtors{Shorgin~S.\,Ya.} see Grusho~A.\,A.&&\\
\Avtors{Shorgin~S.\,Ya.} see Ivanova~D.\,V.&&\\
\Avtors{Shorgin~V.\,S.} see Samouylov~A.\,K.&&\\
\Avtors{Shorgin~V.\,S.} see Sopin~E.\,S.&&\\
\Avtors{Sidorenko~A.\,P.} see Kabanov~Yu.\,M.&&\\
\Avtors{Sinitsyn~I.\,N.} Analytical modeling of distributions with invariant measure in stochastic 
systems with unsolved derivatives&\raisebox{-12pt}[0pt][0pt]{1}&\raisebox{-12pt}[0pt][0pt]{\hphantom{1}2--10}\\
\Avtors{Smirnov~D.\,V.} see Grusho~A.\,A.&&\\
\Avtors{Sopin~E.\,S., Maslov~A.\,R., Shorgin~V.\,S., and Begishev~V.\,O.} Modeling insistent 
user behavior in~5G New Radio networks with rate adaptation and blockage&\raisebox{-12pt}[0pt][0pt]{3}&\raisebox{-12pt}[0pt][0pt]{25--32}\\
\Avtors{Stepanov~E.\,P.} see Shestakov~O.\,V.&&\\
\Avtors{Strijov~V.\,V.} see Neychev~R.\,G.&&\\
\Avtors{Stupnikov~S.\,A.} see Sazontev V.\,V.&&\\
\Avtors{Timonina~E.\,E.} see Grusho~A.\,A.&&\\
\Avtors{Timonina~E.\,E.} see Grusho~A.\,A.&&\\
\Avtors{Timonina~E.\,E.} see Grusho~A.\,A.&&\\
\Avtors{Torshin~I.\,Yu.} On optimization problems arising from the application of topological data 
analysis to the search for forecasting algorithms with fixed correctors&\raisebox{-12pt}[0pt][0pt]{2}&\raisebox{-12pt}[0pt][0pt]{\hphantom{1}2--10}\\
\Avtors{Torshin~I.\,Yu.} On the formation of sets of precedents based on tables of heterogeneous 
feature descriptions by methods of topological theory of data analysis&\raisebox{-12pt}[0pt][0pt]{3}&\raisebox{-12pt}[0pt][0pt]{2--7}\\
\Avtors{Tsukanov~M.\,V.} see Arkhipov~P.\,O.&&\\
\Avtors{Tuboltsev V.\,P., Lapko~A.\,V., and~Lapko~V.\,A.} Nonparametric algorithm for 
automatic classification of remote sensing data&\raisebox{-12pt}[0pt][0pt]{4}&\raisebox{-12pt}[0pt][0pt]{23--31}\\
\Avtors{Ushakov~N.\,G.} see Ushakov~V.\,G.&&\\
\Avtors{Ushakov~V.\,G. and~Ushakov~N.\,G.} Tests for normality of the probabilistic distribution 
when data are rounded&\raisebox{-12pt}[0pt][0pt]{1}&\raisebox{-12pt}[0pt][0pt]{18--27}\\
\Avtors{Ushakov~V.\,G.} see Bergovin~A.\,K.&&\\
\Avtors{Usov~I.\,A., Satin~Y.\,A., and Zeifman~A.\,I.} On the rate of convergence and limiting 
characteristics for one quasi-birth--death process&\raisebox{-12pt}[0pt][0pt]{3}&\raisebox{-12pt}[0pt][0pt]{49--57}\\
\Avtors{Vakulenko~V.\,V. and Zatsman~I.\,M.} Formalized description of statistical information 
processing in databases&\raisebox{-12pt}[0pt][0pt]{3}&\raisebox{-12pt}[0pt][0pt]{93--99}\\
\Avtors{Vasilyev~N.\,S.} Multiplayers' games compositional structure in the monoidal category of 
binary relations&\raisebox{-12pt}[0pt][0pt]{2}&\raisebox{-12pt}[0pt][0pt]{18--26}\\
\Avtors{Volkanov~D.\,Yu.} see Gorshenin~A.\,K.&&\\
\Avtors{Vorontsov~M.\,O. and~Shestakov~O.\,V.} Mean-square risk of the FDR procedure under 
weak dependence&\raisebox{-12pt}[0pt][0pt]{2}&\raisebox{-12pt}[0pt][0pt]{34--40}\\
\Avtors{Zabezhailo~M.\,I.} see Grusho~A.\,A.&&\\
\Avtors{Zabezhailo~M.\,I.} see Grusho~A.\,A.&&\\
\end{tabular}
}
\pagebreak

\def\leftfootline{\small{\textbf{\thepage}
\hfill INFORMATIKA I EE PRIMENENIYA~--- INFORMATICS AND APPLICATIONS\ \ \ 2023\
\ \ volume~17\ \ \ issue\ 4}
}%
 \def\rightfootline{\small{INFORMATIKA I EE PRIMENENIYA~---
INFORMATICS AND APPLICATIONS\ \ \ 2023\ \ \ volume~17\ \ \ issue\ 4
\hfill \textbf{\thepage}}}

\def\leftkol{2023 AUTHOR INDEX} % ENGLISH ABSTRACTS}

\def\rightkol{2023 AUTHOR INDEX} %ENGLISH ABSTRACTS}


\noindent
{\tabcolsep=3pt
\begin{tabular}{p{395.5pt}cc}
&\textbf{Issue} & \textbf{Page}\\[6pt]
\Avtors{Zabezhailo~M.\,I.} see Grusho~A.\,A.&&\\
\Avtors{Zakharov~V.\,N.} see Frenkel~S.\,L.&&\\
\Avtors{Zakharov~V.\,N.} see Sazontev V.\,V.&&\\
\Avtors{Zatsman~I.\,M.} On the scientific paradigm of informatics: Data, information, and 
knowledge&1&116--125\\
\Avtors{Zatsman I.\,M.} Scientific paradigm of informatics: Classification of domain 
objects&4&\hphantom{9}96--103\\
\Avtors{Zatsman~I.\,M.} Transformation of the Ackoff's hierarchy in the scientific paradigm of 
informatics&3&107--113\\
\Avtors{Zatsman~I.\,M.} see Vakulenko~V.\,V.&&\\
\Avtors{Zeifman~A.\,I.} see Usov~I.\,A.&&\\
\end{tabular}
}

%\thispagestyle{myheadings}
\def\leftfootline{\small{\textbf{\thepage}
\hfill INFORMATIKA I EE PRIMENENIYA~--- INFORMATICS AND APPLICATIONS\ \ \ 2023\
\ \ volume~17\ \ \ issue\ 4}
}%
 \def\rightfootline{\small{INFORMATIKA I EE PRIMENENIYA~---
INFORMATICS AND APPLICATIONS\ \ \ 2023\ \ \ volume~17\ \ \ issue\ 4
\hfill \textbf{\thepage}}}

 \label{end\stat}

\newpage

%
   \vspace*{-46pt}

\begin{center}
\vspace*{4pt}
\mbox{%

\epsfxsize=55mm %112.705
\epsfbox{zhur-2.eps}
}
%\end{center}

\vspace*{10pt} 


%   \begin{center}
\fbox{\large\textbf{Академик Юрий Иванович Журавлёв}}\\[10pt]
\textbf{\large 14.01.1935--14.01.2022}
   \end{center}


   %\vspace*{2.5mm}

   \vspace*{5mm}

   \thispagestyle{empty}

%\

%\vspace*{-12pt}
       


В январе этого года ушел из жизни главный научный сотрудник Федерального исследовательского 
центра <<Информатика и управление>> РАН, председатель Редакционного совета журнала 
<<Информатика и~её применения>> академик Юрий Иванович Журавлёв. В~его лице мировая 
наука потеряла одного из своих ярчайших представителей~--- выдающегося ученого-исследователя 
и~талантливого ученого-организатора.

Юрий Иванович родился в Воронеже в 1935~г.\ в семье ученого и врача. Среднее образование 
получил в школе №\,6 г.~Фрунзе (ныне Бишкек) Киргизской ССР. В~1952~г.\ поступил на 
ме\-ха\-ни\-ко-ма\-те\-ма\-ти\-че\-ский факультет МГУ им.\ М.\,В.~Ломоносова. В~1957~г.\ Юрий Иванович 
защищает диплом и продолжает обучение в аспирантуре Московского университета на кафедре 
вычислительной математики (возглавляемой тогда академиком С.\,Л.~Соболевым). После 
успешной защиты кандидатской диссертации (к.ф.-м.н., 1959 г., научный руководитель~--- 
А.\,А.~Ляпунов, оппоненты~--- чл.-корр.\ А.\,А.~Марков, к.ф.-м.н.\ О.\,Б.~Лупанов) и~до 
окончательного переезда в Москву в 1969~г.\ работал в Институте математики Сибирского 
отделения АН СССР, занимая в нем последовательно должности младшего научного сотрудника, 
заведующего отделом, заведующего отделением, заместителя директора по научной работе. 
В~этот период (1954--1966~гг.)\ им был опубликован цикл работ по решению задач алгебры и 
математической логики, причем полученные результаты применялись для создания эффективных 
программ для ЭВМ, конструирования схем и сетей для обработки информации. Наиболее значимый 
результат этого периода научной работы~--- обоснование нового направления исследований, 
общей теории локальных алгоритмов. В~ней были окончательно объединены топологические 
принципы и теория алгоритмов. Эта теория и легла в основу докторской диссертации Юрия 
Ивановича (д.ф.-м.н., 1965~г.)\ по еще тогда новой научной специальности <<Математическая 
кибернетика>>. Оппонировали ему как специалисты по кибернетике~--- академик 
В.\,М.~Глушков, член-корреспондент А.\,А.~Ляпунов и О.\,Б.~Лупанов, так и про\-фес\-сор-ал\-геб\-раист А.\,Д.~Тайманов. 

В 1969~г.\ Юрий Иванович переезжает в Москву и возглавляет в Вычислительном центре АН 
СССР лабораторию проблем распознавания. Впоследствии он~--- заместитель директора по 
научной работе. Научные интересы этого периода связаны с проблемами классификации или 
распознавания образов. В~1976--1978~гг.\ Юрий Иванович публикует цикл работ по ставшему 
вскоре знаменитым алгебраическому подходу к проблеме синтеза корректных алгоритмов. Эти 
работы определили современное состояние всей проблематики распознавания и многих смежных 
областей прикладной математики и информатики. В~своих основополагающих работах Юрий 
Иванович показал, что можно в явном виде строить экстремальные по качеству алгоритмы для 
решения очень широких классов плохо формализованных задач. 
{\looseness=-1

}





Научные заслуги Юрия Ивановича получили широкое признание. В~1966~г.\ он совместно с 
О.\,Б.~Лупановым и чле\-ном-кор\-рес\-пон\-ден\-том АН СССР С.\,В.~Яблонским были удостоены 
звания лауреата Ленинской премии в~об\-ласти науки и техники. В~1984~г.\ Юрий Иванович 
был избран членом-корреспондентом АН СССР (по специальности <<Информатика>>), 
а~в~1992~г.~--- академиком РАН (по той же специальности).\linebreak\vspace*{-12pt}

\pagebreak

\

\vspace*{-46pt}

\noindent
\begin{floatingfigure}{48mm}
\begin{center}
%\vspace*{6pt}
\mbox{%

\epsfxsize=46mm %112.705
\epsfbox{zhur-3.eps}
}
\end{center}
\vspace*{6pt}
\end{floatingfigure}

 \thispagestyle{empty}

\noindent
В~1986~г.\ за цикл прикладных 
работ ему и ряду его учеников была при\-суж\-де\-на премия Совета Министров СССР. Он являлся 
членом иностранных академий наук, председателем секции <<Прикладная математика
 и~информатика>> Отделения математических наук РАН, председателем экспертного совета ВАК 
России по управ\-ле\-нию и информатике, заслуженным профессором нескольких университетов, 
председателем Российской ассоциации <<Распознавание образов и обработка изображений>>, 
членом исполкома Международной ассоциации IAPR (распознавание образов и обработка 
изображений). Был награжден 8-ю орденами и медалями СССР и России.

Юрий Иванович проводил большую научно-литературную работу, являясь, в том числе, главным 
редактором международных научных журналов и членом редколлегий ряда рецензируемых 
научных журналов. 


Параллельно с активной научной деятельностью Юрий Иванович вел и преподавательскую 
работу. С~1961 по~1969~гг.~--- в Новосибирском государственном университете на кафедре 
алгебры и математической логики, которую возглавлял в то время академик А.\,И.~Мальцев. 
С~1970~г., будучи уже профессором (1967~г.),~--- в Московском физико-техническом институте 
на кафедре академика Н.\,Н.~Моисеева. В~1997~г.\ по предложению ректора МГУ им.\ 
М.\,В.~Ломоносова академика В.\,А.~Садовничего Юрий Иванович организовал на факультете 
Вычислительной математики и кибернетики новую кафедру <<Математические методы 
прогнозирования>>, которой и руководил до конца жизни. В~2008~г.\ ему была присуждена 
премия Совета Министров РФ в области образования. С~1965~г.\ Юрий Иванович периодически 
читал курсы лекций за рубежом, в университетах США, Франции, Финляндии, Швеции, Австрии, 
Польши, Болгарии, ГДР и других стран. Эта работа в существенной степени обеспечила широкое 
международное признание советской и российской науки в области дискретной математики и~распознавания образов. 

%\begin{floatingfigure}{60mm}
\begin{figure}[b]
\begin{center}
\vspace*{-6pt}
\mbox{%

\epsfxsize=112mm %90mm %112.705
\epsfbox{zhur-1.eps}
}
\end{center}
\end{figure}
%\end{floatingfigure}

Понимая важность вопроса воспитания подрастающего поколения для развития науки в стране, 
Юрий Иванович вскоре после защиты первой диссертации включился в работу по подготовке 
научных кадров. Им создана большая научная школа: под руководством Юрия Ивановича 
защищены более 100~кандидатских диссертаций по всевозможным разделам естествознания 
(математике, информатике, медицине, технике, экономике, геологии), не один десяток докторов 
наук. Он воспитал академиков и членов-корреспондентов РАН и академий государств СНГ. 
С~большим вниманием и участием Юрий Иванович относился к развитию научных школ страны 
в~об\-ласти обработки изображений, распознавания образов и компьютерной оптики. 

Для всех коллег и учеников Юрия Ивановича он останется примером замечательного человека, 
та\-лант\-ли\-во\-го педагога и выдающегося, преданного служению науке ученого. 


%\def\stat{cont}
{%\hrule\par
%\vskip 7pt % 7pt
\raggedleft\Large \bf%\baselineskip=3.2ex
А\,В\,Т\,О\,Р\,С\,К\,И\,Й\ \ У\,К\,А\,З\,А\,Т\,Е\,Л\,Ь\ \ З\,А\ \ 2\,0\,1\,0 г. \vskip 17pt
    \hrule
    \par
\vskip 21pt plus 6pt minus 3pt }

\label{st\stat}

\def\tit{\ }

\def\aut{\ }
\def\auf{\ }

\def\leftkol{\ } % ENGLISH ABSTRACTS}

\def\rightkol{\ } %АВТОРСКИЙ УКАЗАТЕЛЬ ЗА 2010 г.} %ENGLISH ABSTRACTS}

\titele{\tit}{\aut}{\auf}{\leftkol}{\rightkol}

\vspace*{-12pt}

{\tabcolsep=3pt
\begin{tabular}{p{388pt}rr}
&\textbf{Выпуск} & \textbf{Стр.}\\[6pt]
\hangindent=23pt\noindent\textbf{Арутюнян~А.\,Р.} Моделирование влияния деформаций отпечатков пальцев на 
точность\linebreak
\vspace*{-12pt}\\
\hspace*{23pt}дактилоскопической идентификации$\dotfill$&1&51\\
\hangindent=23pt\noindent\textbf{Архипов~О.\,П., Зыкова~З.\,П.} Интеграция гетерогенной информации о цветных 
пикселях\linebreak
\vspace*{-12pt}\\
\hspace*{23pt}и их цветовосприятии$\dotfill$&4&15\\
\hangindent=23pt\noindent\textbf{Баранов~С.\,И., Френкель~С.\,Л., Захаров~В.\,Н.} Полуформальная верификация 
цифрового устройства с конвейером, основанная на использовании алгоритмических машин\linebreak
\vspace*{-12pt}\\
\hspace*{23pt}состояния$\dotfill$&4&49\\
\textbf{Бекетова~И.\,В.} см.~Каратеев~С.\,Л.&&\\
\textbf{Белоусов~В.\,В.} см.~Синицын~И.\,Н.&&\\
\hangindent=23pt\noindent\textbf{Бенинг~В.\,Е., Королев~Р.\,А.} О предельном поведении мощностей критериев в 
случае\linebreak
\vspace*{-12pt}\\
\hspace*{23pt}распределения Лапласа$\dotfill$&2&63\\
\hangindent=23pt\noindent\textbf{Бенинг~В.\,Е., Сипина~А.\,В.} Асимптотическое разложение для мощности 
критерия,\linebreak
\vspace*{-12pt}\\
\hspace*{23pt}основанного на выборочной медиане, в случае распределения Лапласа$\dotfill$&1&18\\
\textbf{Бондаренко~А.\,В.} см.~Каратеев~С.\,Л.&&\\
\hangindent=23pt\noindent\textbf{Бородина~А.\,В., Морозов~Е.\,В.} Об оценивании асимптотики вероятности 
большого\linebreak
\vspace*{-12pt}\\
\hspace*{23pt}уклонения стационарной регенеративной очереди с одним прибором$\dotfill$&3&29\\
\hangindent=23pt\noindent\textbf{Бунтман~Н.\,В., Минель~Ж.-Л., Ле~Пезан~Д., Зацман~И.\,М.} Типология и 
компьютерное\linebreak
\vspace*{-12pt}\\
\hspace*{23pt}моделирование трудностей перевода$\dotfill$&3&77\\
\textbf{Визильтер~Ю.\,В.} см.~Каратеев~С.\,Л.&&\\
\hangindent=23pt\noindent\textbf{Гавриленко~С.\,В.} Оценки скорости сходимости распределений случайных сумм с 
безгранично делимыми индексами к нормальному закону$\dotfill$&4&81\\
\hangindent=23pt\noindent\textbf{Григорьева~М.\,Е., Шевцова~И.\,Г.} Уточнение неравенства 
Каца--Берри--Эссеена$\dotfill$&2&75\\
\hangindent=23pt\noindent\textbf{Грушо~А.\,А., Грушо~Н.\,А., Тимонина~Е.\,Е.} Поиск конфликтов в политиках 
безопасности: модель случайных графов$\dotfill$&3&38\\
\textbf{Грушо~Н.\,А.} см.~Грушо~А.\,А.&&\\
\hangindent=23pt\noindent\textbf{Гудков~В.\,Ю.} Математические модели изображения отпечатка пальца на основе 
описания линий$\dotfill$&1&58\\
\textbf{Гуртов~А.\,В.} см.~Лукьяненко~А.\,С.&&\\
\textbf{Желтов~С.\,Ю.} см.~Каратеев~С.\,Л.&&\\
\hangindent=23pt\noindent\textbf{Захаров~А.\,А., Серебряков~В.\,А.} Система управления электронной библиотекой 
LibMeta$\dotfill$&4&2\\
\textbf{Захаров~В.\,Н.} см.~Баранов~С.\,И.&&\\
\textbf{Захарова~Т.\,В.} см.~Матвеева~С.\,С.&&\\
\hangindent=23pt\noindent\textbf{Зацаринный~А.\,А., Чупраков~К.\,Г.} Некоторые аспекты выбора технологии для 
постро-\linebreak
\vspace*{-12pt}\\
\hspace*{23pt}ения систем отображения информации ситуационного центра$\dotfill$&3&59\\
\textbf{Зацман~И.\,М.} см.~Бунтман~Н.\,В.&&\\
\hangindent=23pt\noindent\textbf{Зейфман~А.\,И., Коротышева~А.\,В., Сатин~Я.\,А., Шоргин~С.\,Я.} Об 
устойчивости нестаци-\linebreak
\vspace*{-12pt}\\
\hspace*{23pt}онарных систем обслуживания с катастрофами$\dotfill$&3&9\\
\textbf{Зыкова~З.\,П.} см.~Архипов~О.\,П.&&\\
\hangindent=23pt\noindent\textbf{Илюшин~Г.\,Я., Соколов~И.\,А.} Организация управляемого доступа пользователей 
к\linebreak
\vspace*{-12pt}\\
\hspace*{23pt}разнородным ведомственным информационным ресурсам$\dotfill$&1&24\\
\hangindent=23pt\noindent\textbf{Кавагучи~Ю., Ульянов~В.\,В., Фуджикоши~Я.} Приближения для статистик, 
описывающих\linebreak
\vspace*{-12pt}\\
\hspace*{23pt}геометрические свойства данных большой размерности, с оценками 
ошибок$\dotfill$&1&12\\
\hangindent=23pt\noindent\textbf{Каратеев~С.\,Л., Бекетова~И.\,В., Ососков~М.\,В., Князь~В.\,А., 
Визильтер~Ю.\,В., Бондаренко~А.\,В., Желтов~С.\,Ю.} Автоматизированный контроль 
качества цифровых\linebreak
\vspace*{-12pt}\\
\hspace*{23pt}изображений для персональных документов$\dotfill$&1&65\\
\end{tabular}
}

\pagebreak

\def\leftkol{АВТОРСКИЙ УКАЗАТЕЛЬ ЗА 2010 г.} % ENGLISH ABSTRACTS}

\def\rightkol{АВТОРСКИЙ УКАЗАТЕЛЬ ЗА 2010 г.} %ENGLISH ABSTRACTS}

{\tabcolsep=3pt
\begin{tabular}{p{388pt}rr}
&\textbf{Выпуск} & \textbf{Стр.}\\[3pt]
\hangindent=23pt\noindent\textbf{Козеренко~Е.\,Б.} Лингвистические фильтры в статистических моделях машинного\linebreak
\vspace*{-12pt}\\
\hspace*{23pt}перевода$\dotfill$&2&83\\
\hangindent=23pt\noindent\textbf{Козеренко~Е.\,Б., Кузнецов~И.\,П.} Когнитивно-лингвистические представления в 
систе-\linebreak
\vspace*{-12pt}\\
\hspace*{23pt}мах обработки текстов$\dotfill$&3&69\\
\textbf{Князь~В.\,А.} см.~Каратеев~С.\,Л.&&\\
\hangindent=23pt\noindent\textbf{Колесников~А.\,В., Солдатов~С.\,А.} Алгоритм координации для гибридной 
интеллектуальной системы решения сложной задачи оперативно-производственного\linebreak
\vspace*{-12pt}\\
\hspace*{23pt}планирования$\dotfill$&4&61\\
\hangindent=23pt\noindent\textbf{Коновалов~М.\,Г.} О планировании потоков в системах вычислительных 
ресурсов$\dotfill$&2&3\\
\textbf{Конушин~А.\,С.} см.~Конушин~В.\,С.&&\\
\hangindent=23pt\noindent\textbf{Конушин~В.\,С., Кривовязь~Г.\,Р., Конушин~А.\,С.} Алгоритм распознавания людей 
в видео-\linebreak
\vspace*{-12pt}\\
\hspace*{23pt}последовательности по одежде$\dotfill$&1&74\\
\textbf{Корепанов~Э.\, Р.} см.~Синицын~И.\,Н.&&\\
\textbf{Королев~В.\,Ю.} см.~Соколов~И.\,А.&&\\
\textbf{Королев~Р.\,А.} см.~Бенинг~В.\,Е.&&\\
\textbf{Коротышева~А.\,В.} см.~Зейфман~А.\,И.&&\\
\hangindent=23pt\noindent\textbf{Кривенко~М.\,П.} Непараметрическое оценивание элементов байесовского 
клас\-си-\linebreak
\vspace*{-12pt}\\
\hspace*{23pt}фикатора$\dotfill$&2&13\\
\textbf{Кривовязь~Г.\,Р.} см.~Конушин~В.\,С.&&\\
\textbf{Крылов~А.\,С.} см.~Павельева~Е.\,А.&&\\
\hangindent=23pt\noindent\textbf{Крылов~В.\,А.} Моделирование и классификация многоканальных дистанционных\linebreak
\vspace*{-12pt}\\
\hspace*{23pt}изображений с использованием копул$\dotfill$&4&34\\
\hangindent=23pt\noindent\textbf{Крючин~О.\,В.} Разработка параллельных эвристических алгоритмов подбора 
весовых\linebreak
\vspace*{-12pt}\\
\hspace*{23pt}коэффициентов искусственной нейтронной сети$\dotfill$&2&53\\
\hangindent=23pt\noindent\textbf{Кудрявцев~А.\,А., Шоргин~С.\,Я.} Байесовские модели массового обслуживания и 
надеж-\linebreak
\vspace*{-12pt}\\
\hspace*{23pt}ности: характеристики среднего числа заявок в системе $M\vert M \vert 1\vert 
\infty$$\dotfill$&3&16\\
\hangindent=23pt\noindent\textbf{Кузнецов~А.\,А.} Связь между временными и структурно-топологическими 
характери-\linebreak
\vspace*{-12pt}\\
\hspace*{23pt}стиками диаграмм ритма сердца здоровых людей$\dotfill$&4&39\\
\textbf{Кузнецов~И.\,П.} см.~Козеренко~Е.\,Б.&&\\
\textbf{Ле~Пезан~Д.} см.~Бунтман~Н.\,В.&&\\
\hangindent=23pt\noindent\textbf{Лукьяненко~А.\,С., Морозов~Е.\,В., Гуртов~А.\,В.} Анализ сетевого протокола с общей 
функ-\linebreak
\vspace*{-12pt}\\
\hspace*{23pt}цией расширения окна передачи сообщения при конфликтах$\dotfill$&2&46\\
\hangindent=23pt\noindent\textbf{Лямин~О.\,О.} О предельном поведении мощностей критериев в случае обобщенного\linebreak
\vspace*{-12pt}\\
\hspace*{23pt}распределения Лапласа$\dotfill$&3&47\\
\hangindent=23pt\noindent\textbf{Маркин~А.\,В., Шестаков~О.\,В.} Асимптотики оценки риска при пороговой 
обработке\linebreak
\vspace*{-12pt}\\
\hspace*{23pt}вейвлет-вейглет коэффициентов в задаче томографии$\dotfill$&2&36\\
\hangindent=23pt\noindent\textbf{Матвеева~С.\,С., Захарова~Т.\,В.} Сети массового обслуживания с наименьшей 
длиной\linebreak
\vspace*{-12pt}\\
\hspace*{23pt}очереди$\dotfill$&3&22\\
\hangindent=23pt\noindent\textbf{Матюшенко~С.\,И.} Стационарные характеристики двухканальной системы 
обслужива-\linebreak
\vspace*{-12pt}\\
\hspace*{23pt}ния с переупорядочиванием заявок и распределениями фазового типа$\dotfill$&4&68\\
\textbf{Минель~Ж.-Л.} см.~Бунтман~Н.\,В.&&\\
\textbf{Морозов~Е.\,В.} см.~Бородина~А.\,В.&&\\
\textbf{Морозов~Е.\,В.} см.~Лукьяненко~А.\,С.&&\\
\textbf{Ососков~М.\,В.} см.~Каратеев~С.\,Л.&&\\
\hangindent=23pt\noindent\textbf{Павельева~Е.\,А., Крылов~А.\,С.} Поиск и анализ ключевых точек радужной 
оболочки\linebreak
\vspace*{-12pt}\\
\hspace*{23pt}глаза методом преобразования Эрмита$\dotfill$&1&79\\
\textbf{Печинкин~А.\,В.} см.~Френкель~С.\,Л.,&&\\
\hangindent=23pt\noindent\textbf{Протасов~В.\,И.} Составление субъективного портрета с использованием 
эволюционно-\linebreak
\vspace*{-12pt}\\
\hspace*{23pt}го морфинга и квалиметрия метода$\dotfill$&1&83\\
\hangindent=23pt\noindent\textbf{Рудаков~К.\,В., Торшин~И.\,Ю.} Вопросы разрешимости задачи распознавания 
вторичной\linebreak
\vspace*{-12pt}\\
\hspace*{23pt}структуры белка$\dotfill$&2&25\\
\textbf{Сатин~Я.\,А.} см.~Зейфман~А.\,И.&&\\
\hangindent=23pt\noindent\textbf{Сейфуль-Мулюков~Р.\,Б.} Нефть как носитель информации о своем 
происхождении,\linebreak
\vspace*{-12pt}\\
\hspace*{23pt}структуре и эволюции$\dotfill$&1&41\\
\end{tabular}
}

{\tabcolsep=3pt
\begin{tabular}{p{388pt}rr}
&\textbf{Выпуск} & \textbf{Стр.}\\[6pt]
\textbf{Семендяев~Н.\,Н.} см.~Синицын~И.\,Н.&&\\
\textbf{Серебряков~В.\,А.} см.~Захаров~А.\,А.&&\\
\textbf{Синицын~В.\,И.} см.~Синицын~И.\,Н.&&\\
\hangindent=23pt\noindent\textbf{Синицын~И.\,Н., Синицын~В.\,И., Корепанов~Э.\, Р., Белоусов~В.\,В., 
Семендяев~Н.\,Н.} Оперативное построение информационных моделей движения полюса 
Земли\linebreak
\vspace*{-12pt}\\
\hspace*{23pt}методами линейных и линеаризованных фильтров$\dotfill$&1&2\\
\textbf{Сипина~А.\,В.} см.~Бенинг~В.\,Е.&&\\
\hangindent=23pt\noindent\textbf{Соколов~И.\,А.} О работах заслуженного деятеля науки Российской Федерации 
И.\,Н.~Синицына в области информационных технологий и автоматизации (к 70-летию\linebreak
\vspace*{-12pt}\\
\hspace*{23pt}со дня рождения)$\dotfill$&3&84\\
\textbf{Соколов~И.\,А.} см.~Илюшин~Г.\,Я.&&\\
\hangindent=23pt\noindent\textbf{Соколов~И.\,А., Королев~В.\,Ю.} Предисловие$\dotfill$&2&2\\
\textbf{Солдатов~С.\,А.} см.~Колесников~А.\,В.&&\\
\hangindent=23pt\noindent\textbf{Степанов~С.\,Ю.} Использование координатного метода фрагментации 
коммутаторной\linebreak
\vspace*{-12pt}\\
\hspace*{23pt}нейронной сети для сокращения трафика$\dotfill$&2&57\\
\textbf{Тимонина~Е.\,Е.} см.~Грушо~А.\,А.&&\\
\textbf{Торшин~И.\,Ю.} см.~Рудаков~К.\,В.&&\\
\textbf{Ульянов~В.\,В.} см.~Кавагучи~Ю.&&\\
\textbf{Фазекаш~И.} см.~Чупрунов~А.\,Н.&&\\
\textbf{Френкель~С.\,Л.} см.~Баранов~С.\,И.&&\\
\hangindent=23pt\noindent\textbf{Френкель~С.\,Л., Печинкин~А.\,В.} Оценка времени самовосстановления в 
цифровых\linebreak
\vspace*{-12pt}\\
\hspace*{23pt}системах после сбоев, вызываемых переходными помехами$\dotfill$&3&2\\
\textbf{Фуджикоши~Я.} см.~Кавагучи~Ю.&&\\
\hangindent=23pt\noindent\textbf{Цискаридзе~А.\,К.} Математическая модель и метод восстановления позы человека 
по\linebreak
\vspace*{-12pt}\\
\hspace*{23pt}стереопаре силуэтных изображений$\dotfill$&4&27\\
\hangindent=23pt\noindent\textbf{Чупраков~К.\,Г.} К вопросу о размещении коллективных средств отображения в 
ситуа-\linebreak
\vspace*{-12pt}\\
\hspace*{23pt}ционном зале с заданными параметрами$\dotfill$&4&89\\
\textbf{Чупраков~К.\,Г.} см.~Зацаринный~А.\,А.&&\\
\hangindent=23pt\noindent\textbf{Чупрунов~А.\,Н., Фазекаш~И.} Законы повторного логарифма для числа 
безошибочных\linebreak
\vspace*{-12pt}\\
\hspace*{23pt}блоков при помехоустойчивом кодировании$\dotfill$&3&42\\
\textbf{Шевцова~И.\,Г.} см.~Григорьева~М.\,Е.&&\\
\hangindent=23pt\noindent\textbf{Шестаков~О.\,В.} Аппроксимация распределения оценки риска пороговой 
обработки вейвлет-коэффициентов нормальным распределением при использовании 
выбо-\linebreak
\vspace*{-12pt}\\
\hspace*{23pt}рочной дисперсии$\dotfill$&4&73\\
\textbf{Шестаков~О.\,В.} см.~Маркин~А.\,В.&&\\
\textbf{Шоргин~С.\,Я.} см.~Зейфман~А.\,И.&&\\
\textbf{Шоргин~С.\,Я.} см.~Кудрявцев~А.\,А.&&\\
\end{tabular}
}

%\thispagestyle{myheadings}
\def\leftfootline{\small{\textbf{\thepage}
\hfill ИНФОРМАТИКА И ЕЁ ПРИМЕНЕНИЯ\ \ \ том~4\ \ \ выпуск~4\ \ \ 2010}
}%
 \def\rightfootline{\small{ИНФОРМАТИКА И ЕЁ ПРИМЕНЕНИЯ\ \ \ том~4\ \ \ выпуск~4\ \ \ 2010
 \hfill \textbf{\thepage}}}
 \label{end\stat}
%
%Том 10 Выпуск 1-4 Год 2016

\def\stat{cont-e}
{%\hrule\par
%\vskip 7pt % 7pt
\raggedleft\Large \bf%\baselineskip=3.2ex
2\,0\,1\,6\ \ A\,U\,T\,H\,O\,R\ \ I\,N\,D\,E\,X \vskip 17pt
 \hrule
 \par
\vskip 21pt plus 6pt minus 3pt }

\label{st\stat}

\def\tit{\ }

\def\aut{\ }
\def\auf{\ }

\def\leftkol{\ } %2016 AUTHOR INDEX} % ENGLISH ABSTRACTS}

\def\rightkol{\ } %2016 AUTHOR INDEX} %ENGLISH ABSTRACTS}

\titele{\tit}{\aut}{\auf}{\leftkol}{\rightkol}

\def\leftfootline{\small{\textbf{\thepage}
\hfill INFORMATIKA I EE PRIMENENIYA~--- INFORMATICS AND APPLICATIONS\ \ \ 2016\
\ \ volume~10\ \ \ issue\ 4}
}%
 \def\rightfootline{\small{INFORMATIKA I EE PRIMENENIYA~--- INFORMATICS AND APPLICATIONS\ \ \ 2016\ \ \ volume~10\ \ \ issue\ 4
\hfill \textbf{\thepage}}}

\vspace*{-12pt}
\vspace*{-18pt}

{\tabcolsep=2.8pt
\begin{tabular}{p{382pt}cc}
&\textbf{Issue} & \textbf{Page}\\[6pt]
\Avtors{Agalarov~M.\,Ya.} see~Agalarov~Ya.\,M.&&\\
\Avtors{Agalarov~Ya.\,M., Agalarov~M.\,Ya., and
Shorgin~V.\,S.} About the optimal threshold of queue\linebreak
\\[-12pt]
\hspace*{23pt}length in a~particular problem of profit maximization
in the $M/G/1$ queuing system&2&70--79\\
\Avtors{Alexeyevsky~D.\,A.} BioNLP ontology extraction from 
a~restricted language corpus with\linebreak
\\[-12pt]
\hspace*{23pt}context-free grammars&1&119--128\\
\Avtors{Andreev~S.\,D.} see~Gaidamaka~Yu.\,V.&&\\
\Avtors{Andreev~S.\,D.} see~Ometov~A.\,Ya.&&\\
\Avtors{Arkhipov~O.\,P., Arkhipov~P.\,O., and Sidorkin~I.\,I.} The
option to create a~local coordinate\linebreak
\\[-12pt]
\hspace*{23pt}system for synchronization of selected images&3&91--97\\
\Avtors{Arkhipov~P.\,O.} see~Arkhipov~O.\,P.&&\\
\Avtors{Belousov~V.\,V.} see~Shnurkov~P.\,V.&&\\
\Avtors{Belousov~V.\,V.} see~Shnurkov~P.\,V.&&\\
\Avtors{Bening~V.\,E.} Calculation of~the~asymptotic deficiency
of~some statistical procedures based\linebreak
\\[-12pt]
\hspace*{23pt}on~samples with~random sizes&4&34--45\\
\Avtors{Borisov~A.\,V., Bosov~A.\,V., and Miller~G.\,B.} Modeling and
monitoring of VoIP connection&2&\hphantom{1}2--13\\
\Avtors{Bosov~A.\,V.} see~Borisov~A.\,V.&&\\
\Avtors{Briukhov~D.\,O.} see~Stupnikov~S.\,A.&&\\
\Avtors{Callaos~N.\,K.\ and Seyful-Mulyukov~R.\,B.} Complexity and
its information content&1&129--139\\
\Avtors{Chertok~A.\,V., Kadaner~A.\,I., Khazeeva~G.\,T., and
Sokolov~I.\,A.} Regime switching detection\linebreak
\\[-12pt]
\hspace*{23pt}for~the~Levy driven
Ornstein--Uhlenbeck process using CUSUM methods&4&46--56\\
\Avtors{Chichagov~V.\,V.} Asymptotic expansions of mean absolute
error of uniformly minimum variance unbiased and maximum likelihood
estimators on the one-parameter exponential\linebreak
\\[-12pt]
\hspace*{23pt}family model of lattice distributions&3&66--76\\
\Avtors{Danishevsky~V.\,I.} see~Kolesnikov A.\,V.&&\\
\Avtors{Fazliev~A.\,Z.} see~Kalinichenko~L.\,A.&&\\
\Avtors{Fedoseev~A.\,A.} What is behind the concept of ``knowledge in
small packages''&3&105--110\\
\Avtors{Gaidamaka~Yu.\,V., Andreev~S.\,D., Sopin~E.\,S.,
Samouylov~K.\,E., and Shorgin~S.\,Ya.} Interference analysis
of~the~device-to-device communications model with~regard to~a~signal\linebreak
\\[-12pt]
\hspace*{23pt}propagation environment&4&\hphantom{1}2--10\\
\Avtors{Gasilov~A.\,V.} see~Yakovlev~O.\,A.&&\\
\Avtors{Goncharov~A.\,V.\ and Strijov~V.\,V.} Metric time series
classification using weighted dynamic\linebreak
\\[-12pt]
\hspace*{23pt}warping relative to centroids of classes&2&36--47\\
\Avtors{Gordov~E.\,P.} see~Kalinichenko~L.\,A.&&\\
\Avtors{Gorshenin~A.\,K.} Concept of online service for stochastic
modeling of real processes&1&72--81\\
\Avtors{Gorshenin~A.\,K.} see~Shnurkov~P.\,V.&&\\
\Avtors{Gorshenin~A.\,K.} see~Shnurkov~P.\,V.&&\\
\Avtors{Grusho~A.\,A., Grusho~N.\,A., Zabezhailo~M.\,I., and
Timonina~E.\,E.} Integration of statistical and\linebreak
\\[-12pt]
\hspace*{23pt}deterministic methods for
analysis of information security&3&2--8\\
\Avtors{Grusho~A.\,A., Zabezhailo~M.\,I., and Zatsarinny~A.\,A.} On
the advanced procedure to reduce\linebreak
\\[-12pt]
\hspace*{23pt}calculation of Galois closures&4&\hphantom{1}96--104\\
\Avtors{Grusho~N.\,A.} see~Grusho~A.\,A.&&\\
\Avtors{Havanskov~V.\,A.} see~Minin~V.\,A.&&\\
\Avtors{Inkova~O.\,Yu.} see~Zatsman~I.\,M.&&\\
\Avtors{Isachenko~R.\,V.\ and Strijov~V.\,V.} Metric learning in
multiclass time series classification\linebreak
\\[-12pt]
\hspace*{23pt}problem&2&48--57\\
\end{tabular}
}
\pagebreak

\def\leftfootline{\small{\textbf{\thepage}
\hfill INFORMATIKA I EE PRIMENENIYA~--- INFORMATICS AND APPLICATIONS\ \ \ 2016\
\ \ volume~10\ \ \ issue\ 4}
}%
 \def\rightfootline{\small{INFORMATIKA I EE PRIMENENIYA~---
INFORMATICS AND APPLICATIONS\ \ \ 2016\ \ \ volume~10\ \ \ issue\ 4
\hfill \textbf{\thepage}}}

\def\leftkol{2016 AUTHOR INDEX} % ENGLISH ABSTRACTS}

\def\rightkol{2016 AUTHOR INDEX} %ENGLISH ABSTRACTS}


{\tabcolsep=2.83pt
\begin{tabular}{p{382pt}cc}
&\textbf{Issue} & \textbf{Page}\\[6pt]
\Avtors{Kadaner~A.\,I.} see~Chertok~A.\,V.&&\\[.255pt]
\Avtors{Kalinichenko~L.\,A., Volnova~A.\,A., Gordov~E.\,P.,
Kiselyova~N.\,N., Kovaleva~D.\,A., Malkov~O.\,Yu., Okladnikov~I.\,G.,
Podkolodnyy~N.\,L., Pozanenko~A.\,S., Ponomareva~N.\,V.,
Stupnikov~S.\,A.,} \textbf{and Fazliev~A.\,Z.} Data access challenges for data
intensive\linebreak
\\[-12pt]
\hspace*{23pt}research in Russia&1& 2--22\\[.255pt]
\Avtors{Karasikov~M.\,E.\ and Strijov~V.\,V.} Feature-based
time-series classification&4&121--131\\[.255pt]
\Avtors{Khazeeva~G.\,T.} see~Chertok~A.\,V.&&\\[.255pt]
\Avtors{Khokhlov~Yu.\,S.} Multivariate fractional Levy motion and its
applications&2&\hphantom{1}98--106\\[.255pt]
\Avtors{Kirikov~I.\,A., Kolesnikov~A.\,V., Listopad~S.\,V., and
Rumovskaya~S.\,B.} Fine-grained hybrid\linebreak
\\[-12pt]
\hspace*{23pt}intelligent systems. Part 2:
Bidirectional hybridization&1&\hphantom{1}96--105\\[.255pt]
\Avtors{Kirikov~I.\,A., Kolesnikov~A.\,V., Listopad~S.\,V., and
Rumovskaya~S.\,B.} ``Virtual council''~---\linebreak
\\[-12pt]
\hspace*{23pt}source environment
supporting complex diagnostic decision making&3&81--90\\[.255pt]
\Avtors{Kiselyova~N.\,N.} see~Kalinichenko~L.\,A.&&\\[.255pt]
\Avtors{Kolesnikov A.\,V., Listopad~S.\,V., Rumovskaya~S.\,B., and
Danishevsky~V.\,I.} Informal axiomatic\linebreak
\\[-12pt]
\hspace*{23pt}theory of~the~role visual models&4&114--120\\[.255pt]
\Avtors{Kolesnikov~A.\,V.} see~Kirikov~I.\,A.&&\\[.255pt]
\Avtors{Kolesnikov~A.\,V.} see~Kirikov~I.\,A.&&\\[.255pt]
\Avtors{Kolin~K.\,K.} Humanitarian aspects of information
security&3&111--121\\[.255pt]
\Avtors{Konovalov~M.\,G.\ and Razumchik~R.\,V.} Dispatching
to~two parallel nonobservable queues using\linebreak
\\[-12pt]
\hspace*{23pt}only static
information&4&57--67\\[.255pt]
\Avtors{Korchagin~A.\,Yu.} see~Korolev~V.\,Yu.&&\\[.255pt]
\Avtors{Korchagin~A.\,Yu.} see~Korolev~V.\,Yu.&&\\[.255pt]
\Avtors{Korepanov~E.\,R.} see~Sinitsyn~I.\,N.&&\\[.255pt]
\Avtors{Korepanov~E.\,R.} see~Sinitsyn~I.\,N.&&\\[.255pt]
\Avtors{Korolev~V.\,Yu., Korchagin~A.\,Yu., and Zeifman~A.\,I.} The
Poisson theorem for Bernoulli trials\linebreak
\\[-12pt]
\hspace*{23pt}with~a~random probability
of~success and~a~discrete analog of~the~Weibull distribution&4&11--20\\[.255pt]
\Avtors{Korolev~V.\,Yu., Zeifman~A.\,I., and Korchagin~A.\,Yu.}
Asymmetric Linnik distributions as~limit\linebreak
\\[-12pt]
\hspace*{23pt}laws for~random sums
of~independent random variables with~finite variances&4&21--33\\[.255pt]
\Avtors{Koucheryavy~E.\,A.} see~Ometov~A.\,Ya.&&\\[.255pt]
\Avtors{Kovaleva~D.\,A.} see~Kalinichenko~L.\,A.&&\\[.255pt]
\Avtors{Kovalyov~S.\,P.} Metaprogramming to increase
manufacturability of large-scale software-\linebreak
\\[-12pt]
\hspace*{23pt}intensive systems&1&56--66\\[.255pt]
\Avtors{Krivenko~M.\,P.} Significance tests of feature selection for
classification&3&32--40\\[.255pt]
\Avtors{Kruzhkov~M.\,G.} see~Zalizniak~Anna~A.&&\\[.255pt]
\Avtors{Kruzhkov~M.\,G.} see~Zatsman~I.\,M.&&\\[.255pt]
\Avtors{Kudryavtsev~A.\,A.} Bayesian queueing and reliability models:
\textit{A~priori} distributions with\linebreak
\\[-12pt]
\hspace*{23pt}compact support&1&67--71\\[.255pt]
\Avtors{Kudryavtsev~A.\,A.} Characteristics dependent on the balance
coefficient in Bayesian models\linebreak
\\[-12pt]
\hspace*{23pt}with compact support of \textit{a priori}
distributions&3&77--80\\[.255pt]
\Avtors{Kudryavtsev~A.\,A.\ and Palionnaia~S.\,I.} Bayesian recurrent
model of reliability growth:\linebreak
\\[-12pt]
\hspace*{23pt}Parabolic distribution of parameters&2&80--83\\[.255pt]
\Avtors{Kudryavtsev~A.\,A.\ and Titova~A.\,I.} Bayesian queuing
and~reliability models: Degenerate-\linebreak
\\[-12pt]
\hspace*{23pt}Weibull case&4&68--71\\[.255pt]
\Avtors{Leontyev~N.\,D.\ and Ushakov~V.\,G.} Analysis of a queueing
system with autoregressive arrivals\linebreak
\\[-12pt]
\hspace*{23pt}and nonpreemptive priority&3&15--22\\[.255pt]
\Avtors{Listopad~S.\,V.} see~Kirikov~I.\,A.&&\\[.255pt]
\Avtors{Listopad~S.\,V.} see~Kirikov~I.\,A.&&\\[.255pt]
\Avtors{Listopad~S.\,V.} see~Kolesnikov A.\,V.&&\\[.255pt]
\Avtors{Malkov~O.\,Yu.} see~Kalinichenko~L.\,A.&&\\[.255pt]
\Avtors{Markov~A.\,S., Monakhov~M.\,M., and
Ulyanov~V.\,V.} Generalized Cornish--Fisher expansions\linebreak
\\[-12pt]
\hspace*{23pt}for distributions of statistics based on samples
of random size&2&84--91\\[.255pt]
\Avtors{Melnikov~A.\,K.\ and Ronzhin~A.\,F.} Generalized statistical
method of~text analysis based\linebreak
\\[-12pt]
\hspace*{23pt}on~calculation of~probability distributions
of~statistical values&4&89--95\\
\end{tabular}
}
\pagebreak

\def\leftfootline{\small{\textbf{\thepage}
\hfill INFORMATIKA I EE PRIMENENIYA~--- INFORMATICS AND APPLICATIONS\ \ \ 2016\
\ \ volume~10\ \ \ issue\ 4}
}%
 \def\rightfootline{\small{INFORMATIKA I EE PRIMENENIYA~---
INFORMATICS AND APPLICATIONS\ \ \ 2016\ \ \ volume~10\ \ \ issue\ 4
\hfill \textbf{\thepage}}}

\def\leftkol{2016 AUTHOR INDEX} % ENGLISH ABSTRACTS}

\def\rightkol{2016 AUTHOR INDEX} %ENGLISH ABSTRACTS}


{\tabcolsep=3pt
\begin{tabular}{p{381pt}cc}
&\textbf{Issue} & \textbf{Page}\\[6pt]
\Avtors{Meykhanadzhyan~L.\,A.} Stationary characteristics of the finite
capacity queueing system with\linebreak
\\[-12pt]
\hspace*{23pt}inverse service order and generalized
probabilistic priority&2&123--131\\[.23pt]
\Avtors{Miller~G.\,B.} see~Borisov~A.\,V.&&\\[.23pt]
\Avtors{Minin~V.\,A., Zatsman~I.\,M., Havanskov~V.\,A., and
Shubnikov~S.\,K.} Intensity of citation of scientific publications in
inventions on information and computer technologies patented\linebreak
\\[-12pt]
\hspace*{23pt}in Russia by domestic and foreign applicants&2&107--122\\[.23pt]
\Avtors{Monakhov~M.\,M.} see~Markov~A.\,S.&&\\[.23pt]
\Avtors{Naumov~V.\,A.\ and Samouylov~K.\,E.} On relationship
between queuing systems with resources\linebreak
\\[-12pt]
\hspace*{23pt}and Erlang networks&3&\hphantom{1}9--14\\[.23pt]
\Avtors{Okladnikov~I.\,G.} see~Kalinichenko~L.\,A.&&\\[.23pt]
\Avtors{Ometov~A.\,Ya., Andreev~S.\,D., Turlikov~A.\,M., and
Koucheryavy~E.\,A.} Performance analysis of\linebreak
\\[-12pt]
\hspace*{23pt}a wireless data
aggregation system with contention for contemporary sensor
networks&3&23--31\\[.23pt]
\Avtors{Palionnaia~S.\,I.} see~Kudryavtsev~A.\,A.&&\\[.23pt]
\Avtors{Podkolodnyy~N.\,L.} see~Kalinichenko~L.\,A.&&\\[.23pt]
\Avtors{Ponomareva~N.\,V.} see~Kalinichenko~L.\,A.&&\\[.23pt]
\Avtors{Popkova~N.\,A.} see~Zatsman~I.\,M.&&\\[.23pt]
\Avtors{Pozanenko~A.\,S.} see~Kalinichenko~L.\,A.&&\\[.23pt]
\Avtors{Razumchik~R.\,V.} see~Konovalov~M.\,G.&&\\[.23pt]
\Avtors{Ronzhin~A.\,F.} see~Melnikov~A.\,K.&&\\[.23pt]
\Avtors{Rumovskaya~S.\,B.} see~Kirikov~I.\,A.&&\\[.23pt]
\Avtors{Rumovskaya~S.\,B.} see~Kirikov~I.\,A.&&\\[.23pt]
\Avtors{Rumovskaya~S.\,B.} see~Kolesnikov A.\,V.&&\\[.23pt]
\Avtors{Samouylov~K.\,E.} see~Gaidamaka~Yu.\,V.&&\\[.23pt]
\Avtors{Samouylov~K.\,E.} see~Naumov~V.\,A.&&\\[.23pt]
\Avtors{Serebryanskii~S.\,M.} see~Tyrsin~A.\,N.&&\\[.23pt]
\Avtors{Seyful-Mulyukov~R.\,B.} see~Callaos~N.\,K.&&\\[.23pt]
\Avtors{Shestakov~O.\,V.} Statistical properties of the denoising method
based on the stabilized hard\linebreak
\\[-12pt]
\hspace*{23pt}thresholding&2&65--69\\[.23pt]
\Avtors{Shestakov~O.\,V.} The strong law of large numbers for the risk
estimate in the problem of\linebreak
\\[-12pt]
\hspace*{23pt}tomographic image reconstruction from
projections with a correlated noise&3&41--45\\[.23pt]
\Avtors{Shestakov~O.\,V.} see~Zakharova~T.\,V.&&\\[.23pt]
\Avtors{Shnurkov~P.\,V., Gorshenin~A.\,K., and Belousov~V.\,V.}
Analytical solution of~the~optimal control\linebreak
\\[-12pt]
\hspace*{23pt}task of~a~semi-Markov
process with~finite set of~states&4&72--88\\[.23pt]
\Avtors{Shnurkov~P.\,V., Zasypko~V.\,V., Belousov~V.\,V., and
Gorshenin~A.\,K.} Development of the algorithm of numerical solution
of the optimal investment control problem\linebreak
\\[-12pt]
\hspace*{23pt}in the closed dynamical model of three-sector economy&1&82--95\\[.23pt]
\Avtors{Shorgin~S.\,Ya.} see~Gaidamaka~Yu.\,V.&&\\[.23pt]
\Avtors{Shorgin~V.\,S.} see~Agalarov~Ya.\,M.&&\\[.23pt]
\Avtors{Shubnikov~S.\,K.} see~Minin~V.\,A.&&\\[.23pt]
\Avtors{Sidorkin~I.\,I.} see~Arkhipov~O.\,P.&&\\[.23pt]
\Avtors{Sinitsyn~I.\,N.} Analytical modeling of processes in stochastic
systems with complex fractional\linebreak
\\[-12pt]
\hspace*{23pt}order Bessel nonlinearities&3&55--65\\[.23pt]
\Avtors{Sinitsyn~I.\,N.} Orthogonal supoptimal filters for nonlinear
stochastic systems on manifolds&1&34--44\\[.23pt]
\Avtors{Sinitsyn~I.\,N.\ and Korepanov~E.\,R.} Normal Pugachev
conditionally-optimal filters and extra-\linebreak
\\[-12pt]
\hspace*{23pt}polators for state linear stochastic systems&2&14--23\\[.23pt]
\Avtors{Sinitsyn~I.\,N.\ and Sinitsyn~V.\,I.} Analytical modeling of
distributions in stochastic systems on\linebreak
\\[-12pt]
\hspace*{23pt}manifolds based on ellipsoidal approximation&1&45--55\\[.23pt]
\Avtors{Sinitsyn~I.\,N., Sinitsyn~V.\,I., and
Korepanov~E.\,R.} Ellipsoidal suboptimal filters for nonlinear\linebreak
\\[-12pt]
\hspace*{23pt}stochastic systems on manifolds&2&24--35\\[.23pt]
\Avtors{Sinitsyn~V.\,I.} see~Sinitsyn~I.\,N.&&\\[.23pt]
\Avtors{Sinitsyn~V.\,I.} see~Sinitsyn~I.\,N.&&\\[.23pt]
\Avtors{Skvortsov~N.\,A.} see~Stupnikov~S.\,A.&&\\[.23pt]
\Avtors{Sokolov~I.\,A.} see~Chertok~A.\,V.&&\\
\end{tabular}
}
\pagebreak

\def\leftfootline{\small{\textbf{\thepage}
\hfill INFORMATIKA I EE PRIMENENIYA~--- INFORMATICS AND APPLICATIONS\ \ \ 2016\
\ \ volume~10\ \ \ issue\ 4}
}%
 \def\rightfootline{\small{INFORMATIKA I EE PRIMENENIYA~---
INFORMATICS AND APPLICATIONS\ \ \ 2016\ \ \ volume~10\ \ \ issue\ 4
\hfill \textbf{\thepage}}}

\def\leftkol{2016 AUTHOR INDEX} % ENGLISH ABSTRACTS}

\def\rightkol{2016 AUTHOR INDEX} %ENGLISH ABSTRACTS}


{\tabcolsep=3pt
\begin{tabular}{p{382pt}cc}
&\textbf{Issue} & \textbf{Page}\\[6pt]
\Avtors{Sopin~E.\,S.} see~Gaidamaka~Yu.\,V.&&\\
\Avtors{Strijov~V.\,V.} see~Goncharov~A.\,V.&&\\
\Avtors{Strijov~V.\,V.} see~Isachenko~R.\,V.&&\\
\Avtors{Strijov~V.\,V.} see~Karasikov~M.\,E.&&\\
\Avtors{Stupnikov~S.\,A., Briukhov~D.\,O., and Skvortsov~N.\,A.}
Co-lending systemic risk analysis over\linebreak
\\[-12pt]
\hspace*{23pt}heterogeneous data collections&1&23--33\\
\Avtors{Stupnikov~S.\,A.} see~Kalinichenko~L.\,A.&&\\
\Avtors{Suchkov~A.\,P.} see~Zatsarinny~A.\,A.&&\\
\Avtors{Timonina~E.\,E.} see~Grusho~A.\,A.&&\\
\Avtors{Titova~A.\,I.} see~Kudryavtsev~A.\,A.&&\\
\Avtors{Turlikov~A.\,M.} see~Ometov~A.\,Ya.&&\\
\Avtors{Tyrsin~A.\,N.\ and Serebryanskii~S.\,M.} Recognition of
dependences on the basis of inverse\linebreak
\\[-12pt]
\hspace*{23pt}mapping&2&58--64\\
\Avtors{Ulyanov~V.\,V.} see~Markov~A.\,S.&&\\
\Avtors{Ushakov~V.\,G.} Queueing system with working vacations and
hyperexponential input stream&2&92--97\\
\Avtors{Ushakov~V.\,G.} see~Leontyev~N.\,D.&&\\
\Avtors{Volnova~A.\,A.} see~Kalinichenko~L.\,A.&&\\
\Avtors{Yakovlev~O.\,A.\ and Gasilov~A.\,V.} Speeded-up stereo
matching using geodesic support weights&3&\hphantom{1}98--104\\
\Avtors{Zabezhailo~M.\,I.} see~Grusho~A.\,A.&&\\
\Avtors{Zabezhailo~M.\,I.} see~Grusho~A.\,A.&&\\
\Avtors{Zakharova~T.\,V.\ and Shestakov~O.\,V.} Precision analysis of
wavelet processing of aerodynamic\linebreak
\\[-12pt]
\hspace*{23pt}flow patterns&3&46--54\\
\Avtors{Zalizniak~Anna~A.\ and Kruzhkov~M.\,G.} Database
of~Russian impersonal verbal constructions&4&132--141\\
\Avtors{Zasypko~V.\,V.} see~Shnurkov~P.\,V.&&\\
\Avtors{Zatsarinny~A.\,A.\ and Suchkov~A.\,P.} Systems engineering
approaches to~the~establishment of\linebreak
\\[-12pt]
\hspace*{23pt}a~system for~decision support based
on~situational analysis&4&105--113\\
\Avtors{Zatsarinny~A.\,A.} see~Grusho~A.\,A.&&\\
\Avtors{Zatsman~I.\,M., Inkova~O.\,Yu., Kruzhkov~M.\,G., and
Popkova~N.\,A.} Representation of cross-\linebreak
\\[-12pt]
\hspace*{23pt}lingual knowledge about
connectors in supracorpora databases&1&106--118\\
\Avtors{Zatsman~I.\,M.} see~Minin~V.\,A.&&\\
\Avtors{Zeifman~A.\,I.} see~Korolev~V.\,Yu.&&\\
\Avtors{Zeifman~A.\,I.} see~Korolev~V.\,Yu.&&\\
\end{tabular}
}

%\thispagestyle{myheadings}
\def\leftfootline{\small{\textbf{\thepage}
\hfill INFORMATIKA I EE PRIMENENIYA~--- INFORMATICS AND APPLICATIONS\ \ \ 2016\
\ \ volume~10\ \ \ issue\ 4}
}%
 \def\rightfootline{\small{INFORMATIKA I EE PRIMENENIYA~---
INFORMATICS AND APPLICATIONS\ \ \ 2016\ \ \ volume~10\ \ \ issue\ 4
\hfill \textbf{\thepage}}}

 \label{end\stat}

\newpage

%\def\stat{rekl}
%\label{preobr}

%\def\tit{АКАДЕМИК ПУГАЧЁВ  ВЛАДИМИР СЕМЁНОВИЧ\\
%25.03.1911--25.03.1998}


%   \vspace*{-48pt}
%   \begin{center}\LARGE
%Академик Пугачёв  Владимир Семёнович\\ (25.03.1911--25.03.1998)
%   \end{center}
   
   %\vspace*{2.5mm}
   
   \begin{center}

{\prgsh\LARGE
ОБЪЯВЛЕНИЯ О КОНФЕРЕНЦИЯХ}

\end{center}
%\hrule

\vspace*{6pt}

   
   \vspace*{10mm}
   
   \thispagestyle{empty}

\noindent
\begin{tabular}{cc}
%\begin{center}
\multicolumn{1}{c}{\raisebox{-40pt}[0pt][0pt]{\mbox{%
\epsfxsize=33mm
\epsfbox{vspu.eps}
}}}
%\end{center}
&
\tabcolsep=0pt\begin{tabular}{c}
{\prg{\Large\textbf{XII Всероссийское совещание}}}\\[6pt]
{\prg{\Large\textbf{по проблемам управления}}}\\[12pt]
{\prg{\large 16--19 июня 2014~г.}}\\[6pt] 
{\prg{\large Институт проблем управления имени В.\,А.~Трапезникова РАН}}\\[6pt]
{\prg{\large Москва, Россия}}
\end{tabular}
\end{tabular}

\vspace*{60pt}

     
 { %\large    
 XII Всероссийское совещание по проблемам управления (ВСПУ XII), посвященное 75-летию 
Института проблем управления (ИПУ) имени В.\,А.~Трапезникова РАН, проводится 16--19~июня 
2014~г.\ 
в ИПУ РАН (г.~Москва, Россия). ВСПУ XII организуется ИПУ РАН при поддержке РФФИ, Отделения 
энергетики, машиностроения, механики и процессов управления Российской академии наук, 
Российского 
национального комитета по автоматическому управлению, Академии навигации и управ\-ле\-ния 
движением, 
Научного совета РАН по комплексным проблемам управления и автоматизации, Совета по 
мехатронике и робототехнике РАН. Официальный язык Совещания~--- русский.

\vspace*{24pt}
     
     \textbf{Направления работы}
     \begin{enumerate}[1.]
\item Теория систем управления
\item Управление подвижными объектами и навигация
\item Интеллектуальные системы управления
\item Управление в промышленности, транспортом и логистикой
\item Управление системами междисциплинарной природы
\item Средства измерения, вычислений и контроля в управлении
\item Системный анализ и принятие решений в задачах управления
\item Информационные технологии в управлении
\item Проблемы образования в области управления: современное содержание и технологии обучения
\end{enumerate}

\vspace*{24pt}

     Подробная информация о Совещании находится на сайте {\sf http://vspu2014.ipu.ru}. Срок 
окончательной подачи докладов через систему подачи докладов на сайте~--- \textbf{30~ноября} 
2013~г.
}


%\end{document}

%\include{nekrolog-rb}



%\include{IPPM-25}

\def\stat{cont-rus}
{%\hrule\par
%\vskip 7pt % 7pt
\vspace*{-24pt}
\raggedleft\Large \bf%\baselineskip=3.2ex
Правила подготовки рукописей  для публикации в журнале
<<Информатика~и~её~применения>> \vskip 8pt
    \hrule
    \par
\vskip 14pt plus 6pt minus 3pt }

\label{st\stat}

\def\tit{\ }

\def\aut{\ }
\def\auf{\ }

\def\leftkol{\ }
% Правила подготовки рукописей  для публикации в журнале
%<<Информатика и её применения>>

\def\rightkol{\ }
%Правила подготовки рукописей  для публикации в журнале
%<<Информатика и её применения>>}


\titele{\tit}{\aut}{\auf}{\leftkol}{\rightkol}


\vspace*{-60pt}
{ %\small

Журнал <<Информатика и её применения>>
публикует теоретические, обзорные и дискуссионные статьи,
посвященные научным исследованиям и разработкам в области
информатики и ее приложений.

Журнал издается на русском языке. По специальному решению
редколлегии отдельные статьи могут печататься на английском языке.

Тематика журнала охватывает следующие направления:
\begin{itemize}
\item теоретические основы информатики;\\[-15pt]
      \item
математические методы исследования сложных систем и процессов;\\[-15pt]
           \item
информационные системы и сети;\\[-15pt]
                \item
информационные технологии;\\[-15pt]
                     \item
архитектура и программное обеспечение вычислительных комплексов и сетей.\\[-15pt]
\end{itemize}


\noindent
\begin{enumerate}[1.]
\item В журнале печатаются статьи, содержащие результаты, ранее не опубликованные и
не предназначенные к одновременной публикации в других изданиях.

%Публикация не должна нарушать закон об авторских правах.
Публикация предоставленной автором(ами) рукописи не должна нарушать 
положений глав~69, 70 раздела~VII части~IV Гражданского кодекса, 
которые определяют права на результаты интеллектуальной деятельности 
и~средства индивидуализации, в~том числе авторские права, в~РФ.

Ответственность за нарушение авторских прав, в~случае предъявления претензий к~редакции журнала,  
несут авторы статей.



Направляя рукопись в редакцию, авторы сохраняют свои права на данную
рукопись и при этом передают учредителям и редколлегии журнала неисключительные права на
издание статьи на русском языке 
(или на языке статьи, если он отличен от рус\-ско\-го) и~на перевод ее на английский
язык, а~также на
ее распространение в России и за рубежом. 
Каждый автор должен представить в~редакцию подписанный 
с~его стороны <<Лицензионный договор о~передаче неисключительных прав 
на использование произведения>>, текст которого размещен по адресу 
{\sf http://www.ipiran.ru/publications/licence.doc}. 
Этот договор может быть пред\-став\-лен в~бумажном (в~2-х экз.)\ 
или в~электронном виде (отсканированная копия заполненного и~подписанного документа).




Редколлегия вправе запросить у авторов экспертное заключение о возможности
пуб\-ли\-ка\-ции пред\-став\-лен\-ной статьи в открытой печати.\\[-13.5pt]

\item К статье прилагаются данные автора (авторов) (см.\ п.~8). При наличии нескольких
авторов указывается фамилия автора, ответственного за переписку с редакцией.\\[-13.5pt]

\item Редакция журнала осуществляет экспертизу присланных статей в соответствии с
принятой в журнале процедурой рецензирования.

Возвращение рукописи на доработку не означает ее принятия к печати.

Доработанный вариант с ответом на замечания рецензента необходимо прислать в
редакцию.\\[-13.5pt]

\item Решение редколлегии о публикации статьи или ее отклонении сообщается авторам.

Редколлегия может также направить авторам текст рецензии на их статью. Дискуссия по
поводу отклоненных статей не ведется.\\[-13.5pt]

%\pagebreak

\item Редактура статей высылается авторам для просмотра. Замечания к редактуре должны
быть присланы авторами в кратчайшие сроки.\\[-13.5pt]

\item Рукопись предоставляется в электронном виде в форматах MS WORD (.doc или
.docx) или \LaTeX\  (.tex), дополнительно~--- в формате .pdf, на дискете, лазерном диске
или электронной почтой. Предоставление бумажной рукописи необязательно.\\[-13.5pt]

\item При подготовке рукописи в MS Word рекомендуется использовать следующие
настройки.

Параметры страницы:
формат~--- А4; ориентация~--- книжная; поля (см): внутри~--- 2,5, снаружи~--- 1,5,
сверху~--- 2, снизу~--- 2, от края до нижнего колонтитула~--- 1,3.

Основной текст: стиль~--- <<Обычный>>, шрифт~--- Times New Roman, размер~---
14~пунк\-тов, абзацный отступ~--- 0,5~см, 1,5~интервала, выравнивание~--- по ширине.

\pagebreak

\def\leftkol{Правила подготовки рукописей  для публикации в журнале
<<Информатика и её применения>>}

\def\rightkol{Правила подготовки рукописей  для публикации в журнале
<<Информатика и её применения>>}



Рекомендуемый объем рукописи~--- не свыше 10~страниц указанного формата.
При превышении указанного объема редколлегия вправе потребовать от 
автора сокращения объема рукописи.


Сокращения слов, помимо стандартных, не допускаются. Допускается минимальное
количество аббревиатур.


Все страницы рукописи нумеруются.

Шаблоны оформления представлены в интернете:

\noindent
 {\sf
http://www.ipiran.ru/journal/template\_iiep\_ssi\_2024.zip}\\[-14pt]

\item Статья должна содержать следующую информацию на {\bfseries\textit{русском и
английском языках}}:\\[-16pt]

\begin{itemize}
\item название статьи;\\[-15pt]
\item Ф.И.О.\ авторов, на английском можно только имя и фамилию;\\[-15pt]
\item место работы, с указанием почтового адреса организации и электронного адреса каждого
автора;\\[-15pt]
\item сведения об авторах, в соответствии с форматом, образцы которого
представлены на страницах:



\def\leftfootline{\small{\textbf{\thepage}
\hfill ИНФОРМАТИКА И ЕЁ ПРИМЕНЕНИЯ\ \ \ том\ 18\ \ \ выпуск\ 3\ \ \ 2024}
}%
 \def\rightfootline{\small{ИНФОРМАТИКА И ЕЁ ПРИМЕНЕНИЯ\ \ \ том\ 18\ \ \ выпуск\ 3\ \ \ 2024
\hfill \textbf{\thepage}}}



{\sf http://www.ipiran.ru/journal/issues/2013\_07\_01/authors.asp} и

{\sf http://www.ipiran.ru/journal/issues/2013\_07\_01\_eng/authors.asp};
\item аннотация (не менее 100~слов на каждом из языков). Аннотация~--- это краткое
резюме работы, которое может публиковаться отдельно. Она является основным
источником информации в~ин\-фор\-ма\-ци\-он\-ных системах и базах данных. Английская
аннотация должна быть оригинальной, может не быть дословным переводом русского
текста и должна быть написана хорошим английским языком. В~аннотации не должно
быть ссылок на литературу и, по возможности, формул;\\[-15pt]
\item ключевые слова~--- желательно из принятых в мировой
на\-уч\-но-тех\-ни\-че\-ской литературе тематических тезаурусов. Предложения не
могут быть ключевыми словами;\\[-15pt]
\item источники финансирования работы (ссылки на гранты, проекты,
поддерживающие организации и~т.\,п.).
\end{itemize}



%\pagebreak

\item  Требования к спискам литературы.\\[-14pt]

Ссылки на литературу в тексте статьи нумеруются (в квадратных скобках) и
располагаются в каждом из списков литературы в порядке  первых упоминаний. Если источник имеет DOI и/или EDN,
то их необходимо указывать.

Списки литературы представляются в двух вариантах:\\[-14pt]


\noindent
\begin{enumerate}[(1)]
\item \textbf{Список литературы к русскоязычной части}. Русские и английские
работы~---  на языке и в алфавите оригинала;\\[-14.5pt]
\item  \textbf{References}. Русские работы и работы на других языках~--- в латинской
транслитерации с переводом на английский язык; английские работы и работы на других
языках~--- на языке оригинала.
\end{enumerate}

Необходимо для составления списка ``References'' пользоваться размещенной на сайте
{\sf http://www. translit.net/ru/bgn/} бесплатной программой транслитерации русского
 текста в~латиницу. %, при этом в~за\-клад\-ке <<варианты\ldots>> следует выбратьопцию BGN.

Список литературы ``References'' приводится полностью отдельным блоком, повторяя все
позиции из списка литературы к русскоязычной части, независимо от того, имеются или
нет в нем иностранные источники. Если в списке литературы к русскоязычной части есть
ссылки на иностранные публикации, набранные латиницей, они полностью повторяются в
списке ``References''.

Ниже приведены примеры ссылок на различные виды публикаций в списке ``References''.

\def\leftfootline{\small{\textbf{\thepage}
\hfill ИНФОРМАТИКА И ЕЁ ПРИМЕНЕНИЯ\ \ \ том\ 18\ \ \ выпуск\ 3\ \ \ 2024}
}%
 \def\rightfootline{\small{ИНФОРМАТИКА И ЕЁ ПРИМЕНЕНИЯ\ \ \ том\ 18\ \ \ выпуск\ 3\ \ \ 2024
\hfill \textbf{\thepage}}}

{\small

\noindent
\textbf{Описание статьи из журнала:}

\Aue{Zagurenko, A.\,G., V.\,A.~Korotovskikh, A.\,A.~Kolesnikov, A.\,V.~Timonov, and D.\,V.~Kardymon}. 2008.
Tekhniko-ekonomicheskaya optimizatsiya dizayna gidrorazryva plasta [Technical and
economic optimization of the design
of hydraulic fracturing]. \textit{Neftyanoe hozyaystvo} [\textit{Oil Industry}] 11:54--57.

\Aue{Zhang, Z., and D.~Zhu}. 2008. Experimental research on the localized
electrochemical micromachining. \textit{Russ. J.~Electrochem.}  44(8):926--930.
{\sf doi:10.1134/S1023193508080077}.

\noindent
\textbf{Описание статьи из электронного журнала:}

\Aue{Swaminathan, V., E.~Lepkoswka-White, and B.\,P.~Rao}. 1999. Browsers or buyers in cyberspace? An
investigation of electronic factors influencing electronic exchange. \textit{JCMC}
5(2). Available at: {\sf http://www.ascusc.org/jcmc/vol5/issue2/} (accessed April~28, 2011).

\def\leftkol{Правила подготовки рукописей  для публикации в журнале
<<Информатика и её применения>>}

\def\rightkol{Правила подготовки рукописей  для публикации в журнале
<<Информатика и её применения>>}


\noindent
\textbf{Описание статьи из продолжающегося издания (сборника трудов):}

\Aue{Astakhov, M.\,V., and T.\,V.~Tagantsev}. 2006. Eksperimental'noe
issledovanie prochnosti soedineniy ``stal'--kompozit'' [Experimental study of
the strength of joints ``steel--composite'']. \textit{Trudy MGTU
``Matematicheskoe modelirovanie slozhnykh tekh\-ni\-che\-skikh sistem''}
[\textit{Bauman MSTU ``Mathematical Modeling of Complex Technical
Systems'' Proceedings}]. 593:125--130.


\pagebreak



\noindent
\textbf{Описание материалов конференций:}

\Aue{Usmanov, T.\,S., A.\,A.~Gusmanov, I.\,Z.~Mullagalin, R.\,Ju.~Muhametshina, A.\,N.~Chervyakova, and
A.\,V.~Sveshnikov}. 2007. Osobennosti proektirovaniya razrabotki mestorozhdeniy
s primeneniem gidrorazryva
plasta [Features of the design of field development with the use of hydraulic fracturing].
\textit{Trudy 6-go
Mezhdu\-na\-rod\-no\-go Simpoziuma ``Novye resursosberegayushchie tekhnologii nedropol'zovaniya i povysheniya
neftegazootdachi''} [\textit{6th  Symposium (International) ``New Energy Saving Subsoil Technologies and
the Increasing of the Oil and Gas Impact'' Proceedings}]. Moscow. 267--272.



\def\leftfootline{\small{\textbf{\thepage}
\hfill ИНФОРМАТИКА И ЕЁ ПРИМЕНЕНИЯ\ \ \ том\ 18\ \ \ выпуск\ 3\ \ \ 2024}
}%
 \def\rightfootline{\small{ИНФОРМАТИКА И ЕЁ ПРИМЕНЕНИЯ\ \ \ том\ 18\ \ \ выпуск\ 3\ \ \ 2024
\hfill \textbf{\thepage}}}



\noindent
\textbf{Описание книги (монографии, сборники):}



Lindorf, L.\,S., and L.\,G.~Mamikoniants, eds. 1972.
\textit{Ekspluatatsiya turbogeneratorov s neposredstvennym
okhlazhdeniem} [\textit{Operation of turbine generators with direct cooling}].
Moscow: Energy Publs. 352~p.


\Aue{Latyshev, V.\,N.} 2009. \textit{Tribologiya rezaniya. Kn.~1: Friktsionnye protsessy
pri rezanii metallov}
[\textit{Tribology of cutting. Vol.~1: Frictional processes in metal cutting}]. Ivanovo: Ivanovskii
State Univ. 108~p.

\def\leftkol{Правила подготовки рукописей  для публикации в журнале
<<Информатика и её применения>>}

\def\rightkol{Правила подготовки рукописей  для публикации в журнале
<<Информатика и её применения>>}

\noindent
\textbf{Описание переводной книги}
(в списке литературы к русскоязычной части необходимо указать:~/ Пер.\ с англ.~---
после названия книги, а в конце ссылки указать оригинал книги в круглых скобках):
\begin{enumerate}[1.]
\item  В русскоязычной части:

\def\leftfootline{\small{\textbf{\thepage}
\hfill ИНФОРМАТИКА И ЕЁ ПРИМЕНЕНИЯ\ \ \ том\ 18\ \ \ выпуск\ 3\ \ \ 2024}
}%
 \def\rightfootline{\small{ИНФОРМАТИКА И ЕЁ ПРИМЕНЕНИЯ\ \ \ том\ 18\ \ \ выпуск\ 3\ \ \ 2024
\hfill \textbf{\thepage}}}

\Au{Тимошенко С.\,П., Янг Д.\,Х., Уивер~У.}
Колебания в инженерном деле~/ Пер.\ с англ.~--- М.: Машиностроение, 1985. 472~с.
(\Au{Timoshenko~S.\,P., Young~D.\,H., Weaver~W.}
Vibration problems in engineering.~--- 4th ed.~--- New York, NY, USA: Wiley, 1974. 521~p.)\\[-13.5pt]
\item  В англоязычной части:

\Aue{Timoshenko, S.\,P., D.\,H.~Young, and W.~Weaver}.
1974. \textit{Vibration problems in engineering}. 4th ed. New York: 
Wiley. 521~p.
\end{enumerate}

\vspace*{-3pt}


\noindent
\textbf{Описание неопубликованного документа:}


\Aue{Latypov, A.\,R., M.\,M.~Khasanov, and V.\,A.~Baikov}.
2004 (unpubl.). Geologiya i~dobycha (NGT GiD) [Geology and production (NGT GiD)]. Certificate on official registration of the computer program
No.\,2004611198. 

\noindent
\textbf{Описание интернет-ресурса:}


Pravila tsitirovaniya istochnikov [Rules for the citing of sources]. Available at: {\sf
http://www.scribd.com/doc/1034528/} (accessed February~7, 2011).

%\pagebreak

\noindent
\textbf{Описание диссертации или автореферата диссертации:}

\Aue{Semenov, V.\,I.}
2003. Matematicheskoe modelirovanie plazmy v sisteme kompaktnyy tor [Mathematical
modeling of the plasma in the compact torus].  Moscow.  D.Sc.\ Diss. 272~p.

\Aue{Kozhunova, O.\,S.} 2009. Tekhnologiya razrabotki semanticheskogo
slovarya informatsionnogo monitoringa [Technology of development of
semantic dictionary of information monitoring system].  Moscow: IPI RAN. PhD Thesis. 23~p.


\noindent
\textbf{Описание ГОСТа:}

GOST 8.586.5-2005. 2007. Metodika vypolneniya izmereniy. Izmerenie raskhoda i~kolichestva zhidkostey i~gazov
s~pomoshch'yu standartnykh suzhayushchikh ustroystv [Method of measurement.
Measurement of flow rate and volume of liquids and gases by means of orifice devices]. Moscow:
Standardinform  Publs. 10~p.

\noindent
\textbf{Описание патента:}

\Aue{Bolshakov, M.\,V., A.\,V.~Kulakov, A.\,N.~Lavrenov, and M.\,V.~Palkin}.
2006. Sposob orientirovaniya po krenu letatel'nogo
apparata s opti\-che\-skoy golovkoy
samonavedeniya [The way to orient on the roll of aircraft with optical homing head].
Patent RF No.\,2280590.
}

\item Присланные в редакцию материалы авторам не возвращаются.\\[-13.5pt]

\item При отправке файлов по электронной почте просим придерживаться следующих
правил:
\begin{itemize}
\item указывать в поле subject (тема) название журнала и фамилию автора;\\[-13.5pt]
\item указывать в тексте письма название статьи, авторов и~журнал, в~который направляется статья;\\[-13.5pt]
\item использовать attach (присоединение);\\[-13.5pt]
\item в состав электронной версии статьи должны входить: файл, содержащий текст
статьи, и файл(ы), содержащий(е) иллюстрации.\\[-13.5pt]
\end{itemize}

\item Журнал <<Информатика и её применения>> является некоммерческим изданием.
Плата за публикацию не взимается, гонорар авторам не выплачивается.
\end{enumerate}



\def\leftfootline{\small{\textbf{\thepage}
\hfill ИНФОРМАТИКА И ЕЁ ПРИМЕНЕНИЯ\ \ \ том\ 18\ \ \ выпуск\ 3\ \ \ 2024}
}%
 \def\rightfootline{\small{ИНФОРМАТИКА И ЕЁ ПРИМЕНЕНИЯ\ \ \ том\ 18\ \ \ выпуск\ 3\ \ \ 2024
\hfill \textbf{\thepage}}}


\vspace*{-1mm}

\begin{center}

\textbf{Адрес редакции журнала <<Информатика и её применения>>:} \\




Москва 119333, ул.~Вавилова, д.~44, корп.~2, ФИЦ ИУ РАН\\[-10pt]

\

Тел.: +7\,(499)\,135-86-92\ \ Факс:  +7\,(495)\,930-45-05\\[-10pt]

 \

e-mail:   {\sf iiep@frccsc.ru} (Стригина Светлана Николаевна)\\[-10pt]

\

{\sf http://www.ipiran.ru/journal/issues/}
\end{center}
}


\def\leftkol{Правила подготовки рукописей  для публикации в журнале
<<Информатика и её применения>>}

\def\rightkol{Правила подготовки рукописей  для публикации в журнале
<<Информатика и её применения>>}


\def\leftfootline{\small{\textbf{\thepage}
\hfill ИНФОРМАТИКА И ЕЁ ПРИМЕНЕНИЯ\ \ \ том\ 18\ \ \ выпуск\ 3\ \ \ 2024}
}%
 \def\rightfootline{\small{ИНФОРМАТИКА И ЕЁ ПРИМЕНЕНИЯ\ \ \ том\ 18\ \ \ выпуск\ 3\ \ \ 2024
\hfill \textbf{\thepage}}} 
\def\stat{podg-e}
{%\hrule\par
%\vskip 7pt % 7pt
\vspace*{-24pt}
\raggedleft\Large \bf%\baselineskip=3.2ex
Requirements for manuscripts submitted to Journal
``Informatics~and~Applications'' \vskip 8pt
    \hrule
    \par
\vskip 21pt plus 6pt minus 3pt }

\label{st\stat}

\def\tit{\ }

\def\aut{\ }
\def\auf{\ }

\def\leftkol{\ }

\def\rightkol{\ }
%Requirements for manuscripts submitted to Journal
%``Informatics~and~Applications''}

\titele{\tit}{\aut}{\auf}{\leftkol}{\rightkol}

\def\leftfootline{\small{\textbf{\thepage}
\hfill INFORMATIKA I EE PRIMENENIYA~--- INFORMATICS AND APPLICATIONS\ \ \ 2019\
\ \ volume~13\ \ \ issue\ 4}
}%
 \def\rightfootline{\small{INFORMATIKA I EE PRIMENENIYA~--- INFORMATICS AND APPLICATIONS\ \ \ 2019\ \ \ volume~13\ \ \ issue\ 4
\hfill \textbf{\thepage}}}

\vspace*{-60pt}

{\small

\noindent
Journal ``Informatics and Applications'' (Inform.\ Appl.)
publishes theoretical, review, and discussion
articles on the research and development in the
field of informatics and its applications.

The journal is published in Russian.
By a special decision of the editorial
board, some articles can be published in English.


The topics covered include the following areas:
\begin{itemize}
               \item
     theoretical fundamentals of informatics; \\[-14pt]
\item
mathematical methods for studying complex systems and processes; \\[-14pt]
\item
information systems and networks;\\[-14pt]
\item
information technologies; and \\[-14pt]
\item
architecture and software of computational complexes and networks. \\[-14pt]
\end{itemize}

\noindent
\begin{enumerate}[1.]
\item The Journal publishes original articles which have not been published before and are not
intended for simultaneous publication in other editions. An article submitted to the Journal must not violate the
Copyright law. Sending the manuscript to the Editorial Board, the authors retain all rights of the
owners of the manuscript and transfer the nonexclusive rights to publish the article in Russian
(or the language of the article, if not Russian) and its distribution in Russia and abroad to the
Founders and the Editorial Board. Authors should submit a letter to the Editorial Board in the
following form:

{\bfseries\textit{Agreement on the transfer of rights to publish:}}

``\textit{We, the undersigned authors of the manuscript ``\ldots'', pass to the
Founder and the Editorial Board of the Journal ``Informatics and Applications''
the nonexclusive right to publish the manuscript of the article in Russian (or
in English) in both print and electronic versions of the Journal. We affirm
that this publication does not violate the Copyright of other persons or
organizations.}

\textit{Author(s) signature(s): (name(s), address(es), date).}

This agreement should be submitted in paper form or in the form of a scanned copy (signed by
the authors).


%The Editorial Board has the right to request from the authors an official expert conclusion that
%the submitted article has no secret data prohibited for publication. \\[-13.5pt]
\item
A submitted article should be attached with \textbf{the data on the author(s)} (see item~8). If
there are several authors, the contact person should be indicated who is responsible for
correspondence with the Editorial Board and other authors about revisions and final approval
of the proofs.\\[-13.5pt]

\item The Editorial Board of the Journal examines the article according to the established
reviewing procedure. If the authors receive their article for correction after reviewing, it does not
mean that the article is approved for publication. The corrected article should be sent to the
Editorial Board for the subsequent review and approval.\\[-13.5pt]

\item The decision on the article publication or its rejection is communicated to the authors. The
Editorial Board may also send the reviews on the submitted articles to the authors. Any
discussion upon the rejected articles is not possible.\\[-13.5pt]

\item The edited articles will be sent to the authors for proofread. The comments of the authors
to the edited text of the article should be sent to the Editorial Board as soon as possible.\\[-13.5pt]

\item The manuscript of the article should be presented electronically in the MS WORD (.doc or
.docx) or \LaTeX\ (.tex) formats, and additionally in the .pdf format. All documents
 may be sent
by e-mail or provided on a CD or diskette. A~hard copy submission is not necessary.\\[-13.5pt]

\item The recommended typesetting instructions for manuscript.

Pages parameters: format A4, portrait orientation, document margins (cm): left~--- 2.5, right~---
1.5, above~--- 2.0, below~--- 2.0, footer 1.3.

Text: font~---Times New Roman, font size~--- 14, paragraph indent~--- 0.5, line spacing~--- 1.5,
justified alignment.

The recommended manuscript size: not more than 15~pages of the specified format.
If the specified size exceeded, the editorial board is entitled to require the author
to reduce the manuscript.

Use only standard abbreviations. Avoid  abbreviations in the title and
abstract. The full term for which an abbreviation stands should precede
its first use in the text unless it is a standard unit of measurement.

All pages of the manuscript should be numbered.

The templates for the manuscript typesetting are presented on site: {\sf
http://www.ipiran.ru/journal/template.doc}.\\[-13.5pt]


%\def\leftkol{Requirements for manuscripts submitted to Journal
%``Informatics~and~Applications''}

\item The articles should enclose data both in \textbf{Russian and English}:
\begin{itemize}
\item title;\\[-13.5pt]
\item author's name and surname;\\[-13.5pt]
\item affiliation~--- organization, its address with ZIP code, city, country, and
official e-mail address;\\[-13.5pt]
\item data on authors according to the format: (see site)

{\sf http://www.ipiran.ru/journal/issues/2013\_07\_01/authors.asp}  and

{\sf  http://www.ipiran.ru/journal/issues/2013\_07\_01\_eng/authors.asp};\\[-13.5pt]

\pagebreak

\def\leftfootline{\small{\textbf{\thepage}
\hfill INFORMATIKA I EE PRIMENENIYA~--- INFORMATICS AND APPLICATIONS\ \ \ 2019\
\ \ volume~13\ \ \ issue\ 4}
}%
 \def\rightfootline{\small{INFORMATIKA I EE PRIMENENIYA~--- INFORMATICS AND APPLICATIONS\ \ \ 2019\ \ \ volume~13\ \ \ issue\ 4
\hfill \textbf{\thepage}}}


%\def\leftkol{Requirements for manuscripts submitted to Journal
%``Informatics~and~Applications''}

%\def\rightkol{Requirements for manuscripts submitted to Journal
%``Informatics~and~Applications''}



\item abstract (not less than 100 words) both in Russian and in English. Abstract is a short
summary of the article that can be published separately. The abstract is the
main source of information on the article and it could be included in leading information
systems and data bases. The abstract in English has to be an original text and should
not be an exact translation of the Russian one. Good English is required.
In abstracts, avoid references and formulae;\\[-13.5pt]
\item indexing is performed on the basis of keywords. The use of keywords from the
internationally accepted thematic Thesauri is recommended.

%\def\leftkol{Requirements for manuscripts submitted to Journal
%``Informatics~and~Applications''}

%\def\rightkol{Requirements for manuscripts submitted to Journal
%``Informatics~and~Applications''}

Important! Keywords must not be sentences;
\item Acknowledgments.
\end{itemize}

\item References. Russian references have to be presented both in English translation and Latin
transliteration (refer {\sf http://www.translit.net/ru/bgn/}).

Please take into account the following examples of Russian references appearance:

\noindent
\textbf{Article in journal:}

\Aue{Zhang, Z., and D.~Zhu}. 2008. Experimental research on the localized electrochemical
micromachining.
\textit{Rus. J.~Electrochem.}  44(8):926--930. {\sf doi:10.1134/S1023193508080077}.


\noindent
\textbf{Journal article in electronic format:}

\Aue{Swaminathan, V., E.~Lepkoswka-White, and B.\,P.~Rao}. 1999. Browsers or buyers in
cyberspace? An
investigation of electronic factors influencing electronic exchange. \textit{JCMC}
5(2). Available at: {\sf http://www.ascusc.org/jcmc/vol5/issue2/} (accessed April~28, 2011).




\noindent
\textbf{Article from the continuing publication (collection of works, proceedings):}

\Aue{Astakhov, M.\,V., and T.\,V.~Tagantsev}. 2006. Eksperimental'noe
issledovanie prochnosti soedineniy ``stal'--kompozit'' [Experimental study of
the strength of joints ``steel--composite'']. \textit{Trudy MGTU
``Matematicheskoe modelirovanie slozhnykh tekh\-ni\-che\-skikh sistem''}
[\textit{Bauman MSTU ``Mathematical Modeling of Complex Technical
Systems'' Proceedings}]. 593:125--130.

\def\leftfootline{\small{\textbf{\thepage}
\hfill INFORMATIKA I EE PRIMENENIYA~--- INFORMATICS AND APPLICATIONS\ \ \ 2019\
\ \ volume~13\ \ \ issue\ 4}
}%
 \def\rightfootline{\small{INFORMATIKA I EE PRIMENENIYA~--- INFORMATICS AND APPLICATIONS\ \ \ 2019\ \ \ volume~13\ \ \ issue\ 4
\hfill \textbf{\thepage}}}

\def\leftkol{Requirements for manuscripts submitted to Journal
``Informatics~and~Applications''}

\def\rightkol{Requirements for manuscripts submitted to Journal
``Informatics~and~Applications''}

\noindent
\textbf{Conference proceedings:}

\Aue{Usmanov, T.\,S., A.\,A.~Gusmanov, I.\,Z.~Mullagalin, R.\,Ju.~Muhametshina,
A.\,N.~Chervyakova, and
A.\,V.~Sveshnikov}. 2007. Osobennosti proektirovaniya razrabotki mestorozhdeniy
s primeneniem gidrorazryva
plasta [Features of the design of field development with the use of hydraulic fracturing].
\textit{Trudy 6-go
Mezhdu\-na\-rod\-no\-go Simpoziuma ``Novye resursosberegayushchie tekhnologii
nedropol'zovaniya i povysheniya
neftegazootdachi''} [\textit{6th  Symposium (International) ``New Energy Saving Subsoil
Technologies and
the Increasing of the Oil and Gas Impact'' Proceedings}]. Moscow. 267--272.


\noindent
\textbf{Books and other monographs:}




Lindorf, L.\,S., and L.\,G.~Mamikoniants, eds. 1972.
\textit{Ekspluatatsiya turbogeneratorov s neposredstvennym
okhlazhdeniem} [\textit{Operation of turbine generators with direct cooling}].
Moscow: Energy Publs. 352~p.


%\Aue{Latyshev, V.\,N.} 2009. \textit{Tribologiya rezaniya. Kn.~1: Frikcionnye prosessy
%pri rezanii metallov}
%[\textit{Tribology of cutting. Vol.~1: Frictional processes in metal cutting}]. Ivanovo: Ivanovskii
%State Univ. 108~p.


%\noindent
%\textbf{Unpublished material:}

%\Aue{Latypov, A.\,R., M.\,M.~Khasanov, and V.\,A.~Baikov}.
%2004. Geology and production (NGT GiD). Certificate on official registration of the computer
%program
%No.\,2004611198. (In Russian, unpubl.)

%\noindent
%\textbf{Internet-source:}

%APA Style. 2011. Available at: {\sf http://www.apastyle.org/apa-style-help.aspx} (accessed
%February~5, 2011).

%Pravila citirovaniya istochnikov [Rules for the citing of sources]. Available at: {\sf
%http://www.scribd.com/doc/1034528/} (accessed February~7, 2011).


\noindent
\textbf{Dissertation and Thesis:}

%\Aue{Semenov, V.\,I.}
%2003. Matematicheskoe modelirovanie plazmy v sisteme kompaktnyy tor. [Mathematical
%modeling of the plasma in the compact torus]. D.Sc.\ Diss. Moscow. 272~p.

\Aue{Kozhunova, O.\,S.} 2009. Tekhnologiya razrabotki semanticheskogo
slovarya informatsionnogo monitoringa [Technology of development of
semantic dictionary of information monitoring system]. PhD Thesis. Moscow: IPI RAN. 23~p.


\noindent
\textbf{State standards and patents:}

GOST 8.586.5-2005. 2007. Metodika vypolneniya izmereniy. Izmerenie raskhoda i~kolichestva
zhidkostey i gazov 
s~pomoshch'yu standartnykh suzhayushchikh ustroystv [Method of measurement.
Measurement of flow rate and volume of liquids and gases by means of orifice devices]. M.:
Standardinform
Publs. 10~p.

%\noindent
%\textbf{Patent:}

\Aue{Bolshakov, M.\,V., A.\,V.~Kulakov, A.\,N.~Lavrenov, and M.\,V.~Palkin}.
2006. Sposob orientirovaniya po krenu letatel'nogo
apparata s opti\-che\-skoy golovkoy
samonavedeniya [The way to orient on the roll of aircraft with optical homing head].
Patent RF No.\,2280590.

References in Latin transcription are presented in the original language.

References in the text are numbered according to the order of their
first appearance; the number is
placed in square brackets. All items from the reference list should be
cited.\\[-13.5pt]

\item Manuscripts and additional materials are not returned to Authors by the Editorial Board.\\[-13.5pt]

\item Submissions of files by e-mail must include:\\[-13.5pt]
\begin{itemize}
\item   the journal title and author's name in the ``Subject'' field; \\[-13.5pt]
\item   an article and additional materials have to be attached using the ``attach'' function;\\[-13.5pt]
\item   an electronic version of the article should contain the file with the text and a separate file
with figures.\\[-13.5pt]
\end{itemize}

\item ``Informatics and Applications'' journal is not a profit publication. There are no
charges for the authors as well as there are no royalties.\\[-13.5pt]
\end{enumerate}

\def\leftfootline{\small{\textbf{\thepage}
\hfill INFORMATIKA I EE PRIMENENIYA~--- INFORMATICS AND APPLICATIONS\ \ \ 2019\
\ \ volume~13\ \ \ issue\ 4}
}%
 \def\rightfootline{\small{INFORMATIKA I EE PRIMENENIYA~--- INFORMATICS AND APPLICATIONS\ \ \ 2019\ \ \ volume~13\ \ \ issue\ 4
\hfill \textbf{\thepage}}}

\def\leftkol{Requirements for manuscripts submitted to Journal
``Informatics~and~Applications''}

\def\rightkol{Requirements for manuscripts submitted to Journal
``Informatics~and~Applications''}


%\vspace*{5mm}


\begin{center}
\textbf{Editorial Board address:} \\

%ABOUT AUTHORS



FRC CSC RAS, 44, block~2, Vavilov Str., Moscow 119333, Russia\\[-10pt]

\

Ph.: +7\,(499)\,135\,86\,92,\ \ Fax: +7\,(495)\,930\,45\,05\\[-10pt]

\

 e-mail: {\sf rust@ipiran.ru} (to Prof.\ Rustem Seyful-Mulyukov)\\[-10pt]

\

 {\sf http://www.ipiran.ru/english/journal.asp}
\end{center}
 }
%\thispagestyle{myheadings}

\def\leftkol{Requirements for manuscripts submitted to Journal
``Informatics~and~Applications''}

\def\rightkol{Requirements for manuscripts submitted to Journal
``Informatics~and~Applications''}

\def\leftfootline{\small{\textbf{\thepage}
\hfill INFORMATIKA I EE PRIMENENIYA~--- INFORMATICS AND APPLICATIONS\ \ \ 2019\
\ \ volume~13\ \ \ issue\ 4}
}%
 \def\rightfootline{\small{INFORMATIKA I EE PRIMENENIYA~--- INFORMATICS AND APPLICATIONS\ \ \ 2019\ \ \ volume~13\ \ \ issue\ 4
\hfill \textbf{\thepage}}}

 \label{end\stat}

\newpage



%\include{ipi-ind}

%\tableofcontents

\end{document}





%%%%%%%%%%%%%%%%%%%%%%

%\newcommand{\Ack}{\subsection*{\protect\large\bf Acknowledgments}}

%\vphantom*{\int\limits_0^T}

{ \begin{center}  %fig1
 \vspace*{6pt}
    \mbox{%
 \epsfxsize=79mm 
 \epsfbox{gru-1.eps}
 }

\end{center}



\noindent
{{\figurename~1}\ \ \small{
}}}

%\vspace*{6pt}

\addtocounter{figure}{1}