\def\stat{agalarov}

\def\tit{ОБ ОДНОПОРОГОВОМ УПРАВЛЕНИИ ОЧЕРЕДЬЮ В~СИСТЕМЕ МАССОВОГО 
ОБСЛУЖИВАНИЯ С~НЕТЕРПЕЛИВЫМИ ЗАЯВКАМИ}

\def\titkol{Об однопороговом управлении очередью в~системе массового 
обслуживания с~нетерпеливыми заявками}

\def\aut{Я.\,М.~Агаларов$^1$}

\def\autkol{Я.\,М.~Агаларов}

\titel{\tit}{\aut}{\autkol}{\titkol}

\index{Агаларов Я.\,М.}
\index{Agalarov Ya.\,M.}


%{\renewcommand{\thefootnote}{\fnsymbol{footnote}} \footnotetext[1]
%{Работа выполнялась с~использованием инфраструктуры Центра коллективного пользования <<Высокопроизводительные вы\-чис\-ле\-ния и~большие данные>> 
%(ЦКП <<Информатика>>) ФИЦ ИУ РАН.}}


\renewcommand{\thefootnote}{\arabic{footnote}}
\footnotetext[1]{Федеральный исследовательский центр <<Информатика и~управление>> Российской академии наук, 
\mbox{agglar@yandex.ru}}


\vspace*{-6pt}

  
  
  \Abst{Изложены результаты теоретического исследования управ\-ля\-емой системы 
массового обслуживания (СМО) типа $M/M/s$ с~нетерпеливыми заявками и~однопороговым 
управлением очередью. Ставится задача оптимизации однопорогового управления очередью, 
суть которой заключается в~вычислении для длины очереди некоторого порогового значения, 
максимизирующего заданную целевую функцию. В~исследуемой сис\-те\-ме заявка покидает 
систему необслуженной, если время ожидания в~очереди (или время обслуживания на приборе) 
превышает некоторый интервал времени случайной длины, распределенной по показательному 
закону с~заданным параметром. В~качестве показателя эф\-фек\-тив\-ности управ\-ле\-ния очередью 
(целевой функции) используется стоимостная функция, учитывающая потери в~единицу 
времени из-за технического обслуживания сис\-те\-мы, отклонения заявок на входе сис\-те\-мы, 
ухода заявок до завершения обслуживания. Предложены метод решения задачи максимизации 
стоимостной целевой функции на множестве однопороговых управ\-ле\-ний очередью 
и~алгоритм гарантированного вычисления оптимального порога. }
  
  
  \KW{система массового обслуживания; нетерпеливые заявки; управ\-ле\-ние очередью}
  
\DOI{10.14357/19922264240206}{JZHAKU}
  
\vspace*{6pt}


\vskip 10pt plus 9pt minus 6pt

\thispagestyle{headings}

\begin{multicols}{2}

\label{st\stat}

  
\section{Введение}

\vspace*{-6pt}


  Настоящая работа служит продолжением исследований, посвященных 
проблеме оптимизации порогового управления очередью в~СМО с~учетом 
стоимостных потерь из-за отклонения и~задержек заявок, а также затрат на 
техническое обслуживание системы. Суть порогового управления очередью 
заключается в~том, что для длины очереди задается одно или несколько 
пороговых значений, по достижении каждого из которых принимается 
соответствующее решение по сбросу нагрузки из очереди с~\mbox{целью} 
повышения эффективности работы сис\-те\-мы~[1]. 
  
  Ниже будем рассматривать оптимизационную задачу управления очередью для 
простейшей СМО, у~которой ограничено время 
пребывания заявки в~очереди или на приборе. Заявка покидает сис\-те\-му 
необслуженной, если время ожидания в~очереди или на приборе превышает 
некоторую случайную\linebreak величину с~заданным средним значением. В~прос\-тей\-шей 
модели системы такого типа предполагают, что заявки покидают очередь через 
случайные интервалы времени, распределенные по \mbox{показательному} закону, т.\,е.\ 
возникает поток уходящих из очереди с~постоянной интенсивностью заявок. 
Таким образом, каж\-дая заявка, на\-хо\-дя\-ща\-яся в~очереди или на приборе, может 
покинуть систему, не дождавшись обслуживания, через случайный интервал 
времени, распределенный по показательному закону. Заявки в~этом случае 
называют <<нетерпеливыми>>, а~СМО~--- сис\-те\-мой с~<<нетерпеливыми>> 
заявками. Такая СМО имеет четыре потока, влия\-ющих на со\-сто\-яние сис\-те\-мы: 
входной поток заявок, поток обслуженных заявок, поток заявок, по\-ки\-да\-ющих 
очередь, не дождавшись начала обслуживания, и~поток уходящих с~приборов 
заявок, не дождавшихся завершения обслуживания. Так как поток уходящих 
заявок пуассоновский, то процесс, протекающий в~сис\-те\-ме под влиянием такого 
потока, будет марковским. 
  
  С увеличением порога длины очереди, с~одной стороны, увеличивается поток 
заявок~--- потенциальных плательщиков за обслуживание, с~другой~--- 
увеличиваются потери сис\-те\-мы (из-за увеличения задержек заявок, ухода 
<<нетерпеливых>> заявок, затрат сис\-те\-мы на хранение и~обслуживание 
заявок). Возникает задача поиска значения порога длины очереди, 
максимизирующего доход сис\-темы.
  
  Результаты теоретических и~экспериментальных исследований по 
рассматриваемой в~данной \mbox{статье} проб\-ле\-ме, изложенные в~ранее 
опубликованных работах, получены для задачи оптимизации порогового 
управления очередью в~одноканальных и~многоканальных СМО с~терпеливыми 
заявками (заявки не покидают сис\-те\-му до завершения обслуживания) (см., 
например,~[2--8]). При исследовании СМО с~управ\-ля\-емы\-ми очередями 
методами математического моделирования, как правило, требуется 
предварительно решить подзадачу расчета  
ве\-ро\-ят\-ност\-но-вре\-мен\-н$\acute{\mbox{ы}}$х характеристик (показателей) исследуемой  
сис\-те\-мы~[9--13] и~использовать данную расчетную модель для разработки 
и~исследования алгоритма управ\-ле\-ния очередью (очередями), что приводит 
к~математической модели с~более сложными функциональными 
взаимозависимостями параметров системы по сравнению с~расчетной моделью. 
В~научных пуб\-ли\-ка\-ци\-ях, посвященных исследованию сис\-тем 
с~нетерпеливыми заявками, отсутствуют результаты по оптимизации  
управ\-ле\-ния очередью, в~основном в~них рас\-смот\-ре\-ны задачи по расчету  
ве\-ро\-ят\-ност\-но-вре\-мен\-н$\acute{\mbox{ы}}$х характеристик и~оптимизации структуры таких 
сис\-тем~\cite{9-ag, 10-ag}.
  
  Ниже приводим метод и~результаты теоретического исследования 
однопорогового управления очередью системы $M/M/s$ с~<<нетерпеливыми>> 
заявками при стоимостном критерии оптимальности. 
  
\section{Постановка задачи }

  Рассматривается многоканальная СМО $M/M/s$ с~управляемой очередью, 
в~которой заявки, не дожидаясь завершения обслуживания, могут покинуть 
систему по истечении некоторого времени пребывания в~очереди или на приборе. 
Предполагается, что время, через которое заявка покидает сис\-те\-му,~--- 
показательно распределенная случайная величина, при этом параметр 
распределения равен~$\alpha_i$ ($\alpha_1\leq \alpha_2\leq\cdots$), если заявка  
\mbox{$i$-я} в~очереди, а~если на приборе, то параметр равен~$\beta$. 
Поступившая извне заявка допускается в~систему, если длина очереди в~системе 
меньше, чем $h\hm\geq 0$~--- некоторая заданная величина (пороговое значение), 
иначе отклоняется и~теряется. Допущенная в~систему заявка занимает любой из 
свободных приборов, если такой есть, иначе становится в~конец очереди. Будем 
считать, что заявки обслуживаются в~порядке поступления. Отметим, что 
поведение такой сис\-те\-мы описывается цепью Маркова, в~которой 
состоянием считается число заявок в~системе~\cite{9-ag}.
  
   Введем обозначения:
  \begin{description}
  \item[\,]  $\lambda$~--- интенсивность входного потока;
  \item[\,]  $\mu$~--- интенсивность обслуживания заявки на приборе;
  \item[\,]  $s$~--- число приборов в~системе;
  \item[\,]  $h+s$~--- объем накопителя;
   \item[\,]  $C_0$~--- плата заявки, принятой в~накопитель сис\-темы; 
  \item[\,]  $C_1$~--- стоимость потерь из-за отклонения заявки на входе системы;
  \item[\,]  $C_2$~--- стоимость потерь из-за ухода $i$-й заявки, находящейся в~очереди;
  \item[\,]  $C_3$~--- стоимость потерь из-за ухода с~прибора заявки, не 
дождавшейся завершения обслуживания;
  \item[\,]  $\pi_i^{(h)}$~--- стационарная вероятность состояния $i$ сис\-те\-мы 
при пороговом значении~$h$;
%  \item[\,]  $\overline{L}^{(h)} =s- \sum\nolimits^S_{i=1} (s-i)\pi_i^{(h)}$~--- среднее 
%значение длины очереди при пороге~$h$;
  \item[\,]  $\overline{S}^{(h)} =s-\sum\nolimits^S_{i=1} (s-i)\pi_i^{(h)}$~--- среднее 
чис\-ло занятых приборов;
  \item[\,]  $Q(h)$~--- доход сис\-те\-мы в~единицу времени при пороговом 
значении~$h$.
  \end{description}
  
  В качестве целевой функции задачи оптимизации порогового управления 
рассматривается предельный средний доход системы в~единицу времени, 
вычисляемый по формуле:
  \begin{multline}
  Q(h)=\lambda C_0\left( 1-\pi^{(h)}_{s+h}\right) -\lambda C_1\pi^{(h)}_{s+h}-{}\\
  {}- d\left( \overline{L}^{(h)}\right) -\beta C_3 \overline{S}^{(h)}.
  \label{e1-ag}
  \end{multline}
  Здесь $\lambda C_0(1\hm- \pi^{(h)}_{s+h})$~--- средняя суммарная плата 
заявок, принимаемых в~накопитель в~единицу времени; $\lambda 
C_1\pi^{(h)}_{s+h}$~--- средние потери в~единицу времени  
из-за отклонения заявок; $ d( \overline{L}^{(h)})$~--- средние потери 
в~единицу времени из-за ухода заявок из очереди:
$$
 d\left(  \overline{L}^{(h)}\right)= C_2\sum\limits^h_{i=1} \sum\limits^i_{j=1} 
\alpha_j \pi_{i+S}^{(h)};
$$
 $\beta C_3\overline{S}^{(h)}$~--- средние потери в~единицу 
времени из-за ухода с~приборов заявок, не дождавшихся завершения 
обслуживания.
  
  Ставится задача оптимизации порогового значения длины очереди, т. \,е.\ 
математическая задача вида
  \begin{equation}
  h^*= \argmax\limits_{0\leq h} Q(h)\,.
  \label{e2-ag}
  \end{equation}

\section{Метод решения и~результаты}

  Для стационарных вероятностей состояний описанной в~предыдущем разделе 
СМО справедливы равенства~\cite{9-ag}:
  \begin{equation}
  \left.
  \begin{array}{rl}
  \pi_l^{(h)} &=\fr{\rho^l}{l!(1+\gamma)^i}\,\pi_0^{(h)}\ \mbox{при}\ l\in \overline{1,s}\,,
  \\[6pt]
  \pi^{(h)}_{s+l} &=\fr{\rho^s}{s!(1+\gamma)^3} \prod\limits^l_{j=1} 
\fr{\rho}{s(1+\gamma) + j\theta_j}\,\pi_0^{(h)}\\[6pt]
 &\hspace*{27mm}\mbox{при}\ l\in \overline{1, h}\,,
 \end{array}
 \right\}
   \label{e3-ag}
  \end{equation}
  где
  
  \vspace*{-6pt}
  
  \noindent
\begin{multline*}
  \pi_0^{(h)} =\left[ 
  \vphantom{\prod\limits^l_{j=1}}
  1+\sum\limits_{m=1}^s \fr{\rho^m}{m!(1+\gamma)^m} + {}\right.\\
\left.   {}+
\fr{\rho^s}{s!(1+\gamma)^s} \sum\limits^h_{l=1} \prod\limits^l_{j=1} 
\fr{\rho}{s(1+\gamma)+j\theta_j}\right]^{-1};
\end{multline*}
  $$
  \rho=\fr{\lambda}{\mu}\,;\quad \gamma= \fr{\beta}{\mu}\,;\quad \theta_j= \fr{\alpha_j}{\mu}\,.
  $$
  
  Покажем, что для стационарных вероятностей справедливы соотношения
  \begin{align}
  \pi_l^{(h+1)} &= \left(1-\pi^{(h+1)}_{s+h+1}\right)\pi_l^{(h)},\ l=\overline{0, s+h}\,;
  \label{e4-ag}\\
  \pi^{(h+1)}_{s+h+1} &={}\notag\\
  &\hspace*{-10mm}{}= \left( 1-\pi_{s+h+1}^{(h+1)} \right) \pi^{(h)}_{s+h}  
\fr{\rho}{s(1+\gamma)+(h+1)\theta_{h+1}}\,.
  \label{e5-ag}
  \end{align}
  
  Из~(\ref{e3-ag}) при $l\hm=\overline{1,  h}$ следует 
  \begin{multline*}
  \pi^{(h)}_{s+l} -\pi_{s+l}^{(h+1)} ={}\\
  {}=\fr{\rho^s}{s!(1+\gamma)^s} 
\prod\limits^l_{j=1} \fr{\rho}{s(1+\gamma)+j\theta_j} \left( \pi_0^{(h)} -
\pi_0^{(h+1)}\right) ={}\\
  {}= \left( \fr{\rho^s}{s!(1+\gamma)^s}\right)^2 \prod\limits^l_{j=1} 
\fr{\rho}{s(1+\gamma)+j\theta_j}\times{}\\
{}\times  \prod\limits^{h+1}_{j=1} 
\fr{\rho}{s(1+\gamma)+j\theta_j}\,\pi_0^{(h)} \pi_0^{(h+1)}= \pi^{(h)}_{s+l} \pi_{s+h+1}^{(h+1)}\,.
  \end{multline*}
  
  Точно так же, использовав~(\ref{e3-ag}) при $l\hm= \overline{0, s}$, получаем 
равенство 
$$
\pi_l^{(h)} - \pi_l^{(h+1)} = \pi_l^{(h)} \pi^{(h+1)}_{s+h+1}.
$$ 
Следовательно, равенства~(\ref{e4-ag}) справедливы. Аналогично, 
использовав~(\ref{e3-ag}) и~(\ref{e4-ag}), находим
  \begin{multline*}
  \pi^{(h+1)}_{s+h+1} =\fr{\rho^s}{s!(1+\gamma)^2} \prod\limits_{j=1}^{h+1} 
\fr{\rho}{s(1+\gamma)+j\theta_j}\,\pi_0^{(h+1)}={}\\
  {}= 
  \fr{\rho^s}{s!(1+\gamma)^s}\,
  \fr{\rho}{s(1+\gamma)+(h+1)\theta_{h+1}} \times{}\\
  {}\times \prod\limits^h_{j=1} \fr{\rho} 
{s(1+\gamma)+j\theta_j} \left( 1-\pi^{(h+1)}_{s+h+1}\right) \pi_0^{(h)},
  \end{multline*}
откуда следует~(\ref{e5-ag}).

  Покажем, что имеет место равенство
  \begin{equation}
  Q(h)-Q(h+1) =\pi_{s+h+1}^{(h+1)}\left[ Q(h)-G(h)\right],
  \label{e6-ag}
  \end{equation}
  
\vspace*{-12pt}
  
  \columnbreak 


\noindent
где



\noindent
\begin{multline}
G(h)=\left( C_0+C_1\right) (\mu+\beta)s +\left( C_0+C_1\right) \alpha_{h+1}(h+1) -
{}\\
{}- \sum\limits_{j=1}^{h+1} \alpha_j C_2 - C_1\lambda -C_3\beta s\,.
\label{e7-ag}
\end{multline}

\vspace*{-6pt}
  
  Использовав~(\ref{e1-ag})--(\ref{e7-ag}), получим:
  
  \vspace*{-6pt}
  
  \noindent
  \begin{multline*}
  Q(h)-Q(h+1) =\lambda C_0\left( 1-\pi^{(h)}_{s+h}\right) -{}\\
  {}- \lambda C_1 
\pi^{(h)}_{s+h} -d\left( \overline{L}^{(h)}\right) -\beta C_3 \overline{S}^{(h)}-{}\\
  {}-
  \lambda C_0\left( 1- \pi^{(h+1)}_{s+h+1}\right) +\lambda C_1 
\pi_{s+h+1}^{(h+1)} +d\left( \overline{L}^{(h+1)}\right) +{}\\
{}+\beta C_3  \overline{S}^{(h+1)}=
  -\lambda C_0\pi^{(h)}_{s+h} -\lambda C_1 \pi^{(h)}_{s+h} -d\left( 
\overline{L}^{(h)}\right) -{}\\
{}- \beta C_3 \overline{S}^{(h)} +\lambda  C_0\pi_{s+h+1}^{(h+1)}+\lambda C_1 \pi^{(h+1)}_{s+h+1}+{}\\
{}+\left( 1-\pi_{s+h+1}^{(h+1)}\right) d\left( \overline{L}^{(h)}\right)+
 C_2\sum\limits_{j=1}^{h+1} \alpha_j \pi_{s+h+1}^{(h+1)} 
+{}\\
{}+\beta s C_3 \pi_{s+h+1}^{(h+1)} +
\beta C_3\left( 1-\pi_{s+h+1}^{(h+1)}\right) \overline{S}^{(h)} ={}\\
{}= -\lambda C_0 
\pi^{(h)}_{s+h} -\lambda C_1\pi^{(h)}_{s+h} +\lambda C_0 \pi^{(h+1)}_{s+h+1} 
+{}\\
  {}+ \lambda C_1\pi_{s+h+1}^{(h+1)} +C_2\sum\limits_{j=1}^{h+1} \alpha_j 
\pi^{(h+1)}_{s+h+1}+\beta s C_3 \pi_{s+h+1}^{(h+1)} -{}\\
{}- \pi_{s+h+1}^{(h+1)} d\left( \overline{L}^{(h)}\right)-
 \beta C_3\pi^{(h+1)}_{s+h+1} \overline{S}^{(h)} ={}\\
 {}=\pi_{s+h+1}^{(h+1)} \Bigg[ 
 %\vphantom{\fr{C_0+C_1}{\pi^{(h+1)}_{s+h+1}}}
\lambda C_0+\lambda C_1 +C_2\sum\limits_{j=1}^{h+1} \alpha_j -d\left( 
\overline{L}^{(h)}\right) - {}\\
  {}- C_3\beta \overline{S}^{(h)} +\beta s C_3 -\lambda 
\fr{C_0+C_1}{\pi^{(h+1)}_{s+h+1}}\,\pi^{(h)}_{s+h}
 %\vphantom{\fr{C_0+C_1}{\pi^{(h+1)}_{s+h+1}}}
 \Bigg]={}\\
  {}=  \pi^{(h+1)}_{s+h+1} \left[
 \vphantom{\fr{C_0+C_1}{\pi^{(h+1)}_{s+h+1}}}
 \lambda C_0 \left( 1-\pi^{(h)}_{s+h}\right) -
\lambda C_1\pi^{(h)}_{s+h} +\lambda C_1 +{}\right.\\
{}+\sum\limits^{h+1}_{j=1} \alpha_j C_2 -
d\left( \overline{L}^{(h)}\right)- C_3\beta\overline{S}^{(h)} +\lambda C_1\pi^{(h)}_{s+h} +{}\\
\left.{}+\lambda 
C_0\pi^{(h)}_{s+h} +\beta s C_3 -
\lambda\fr{C_0+C_1}{\pi^{(h+1)}_{s+h+1}}\,\pi^{(h)}_{s+h}\right]={}\\
  {}= \pi^{(h+1)}_{s+h+1} \left[ 
  Q(h) +\lambda C_1+ \sum\limits_{j=1}^{h+1} 
\alpha_j C_2 +\beta s C_3- {}\right.\\[-2pt]
\left.  {}-  (C_0+C_1)\lambda \fr{s(1+\gamma)+(h+1)\theta}{\rho} 
\vphantom{\sum\limits_{j=1}^{h+1}}
\right]={}\\[-2pt]
{}=  \pi^{(h+1)}_{s+h+1} \Bigg[ Q(h)+\lambda C_1+ C_2 \sum\limits_{j=1}^{h+1} 
\alpha_j +\beta s C_3 -{}\\[-2pt]
{}-(C_0+C_1) [ s(\mu+\beta)+\alpha_{h+1} (h+1)]\Bigg].
  \end{multline*} 
  
\begin{figure*}[b] %fig1
\vspace*{1pt}
\begin{minipage}[t]{80mm}
      \begin{center}
     \mbox{%
\epsfxsize=79mm 
\epsfbox{aga-1.eps}
}
\end{center}
\vspace*{-9pt}
\Caption{Зависимости функций $Q$~(\textit{1}) и~$G$~(\textit{2}) от порогового значения~$h$: $h^*$~--- 
оптимальное пороговое значение длины очереди}
\end{minipage}
%\end{figure*}
\hfill
%\begin{figure*} %fig2
\vspace*{1pt}
\begin{minipage}[t]{80mm}
      \begin{center}
     \mbox{%
\epsfxsize=79mm 
\epsfbox{aga-2.eps}
}
\end{center}
\vspace*{-9pt}
\Caption{Зависимости функций $Q$~(\textit{1}) и~$G$~(\textit{2}) от порогового значения~$h$}
\end{minipage}
\end{figure*}
  
 \vspace*{-6pt}
  
Значит, равенство~(\ref{e6-ag}) имеет место. 

\pagebreak

Так как верно равенство 
$$
\lambda \left(1 - \pi^{(h)}_{s+h}\right) \hm= \sum\limits^h_{i=1} \sum\limits^i_{j=1} 
\alpha_j \pi_i^{(h)} +(\beta+\mu) \overline{S}^{(h)}\,,
$$
то равенства для $Q(h)$ и~$G(h)$ в~(\ref{e1-ag}) и~(\ref{e7-ag}) эквивалентны 
равенствам
\begin{align*}
Q(h) &= \left[ \left( C_0+C_1\right)(\beta+\mu) -\beta C_3\right]\overline{S}^{(h)} 
+{}\\
&{}+\left( C_0 +C_1-C_2\right) \sum\limits^h_{i=1} \sum\limits^i_{j=1} \alpha_j 
\pi_i^{(h)} -C_1\lambda\,;\\
G(h) &= \left[ \left( C_0+C_1\right)(\beta+\mu) -\beta C_3\right]s +{}\\
&{}+\left( 
C_0+C_1\right) \alpha_{h+1} (h+1) -C_2\sum\limits_{j=1}^{h+1} \alpha_j -C_1\lambda\,.
\end{align*}
При $h=0$ последние равенства примут вид:
\begin{align*}
Q(0) &= \left[ \left( C_0+C_1\right((\beta+\mu)-\beta C_3\right] \overline{S}^{(0)} -
C_1\lambda\,;\\
G(0)&= \left[ \left( C_0+C_1\right) (\beta+\mu) -\beta C_3\right] s+{}\\
&\hspace*{18mm}{}+\left( C_0+C_1- C_2\right) \alpha_1 -C_1\lambda\,.
\end{align*}
  
  
  Далее всюду будем предполагать, что при условии $C_0\hm+ C_1\hm- C_2 
\hm<0$ выполняется и~условие $(C_2/(C_0\hm+C_1)-1) 
\alpha_{i+1}/(\alpha_{i+1}\hm-\alpha_i)\hm\geq i$ для всех $i\hm\geq 1$. Обратим 
внимание, что функция $G(h)$ возрастает по~$h$ при $C_0\hm+C_1\hm- C_2\hm 
>0$ и~не возрастает, когда $C_0\hm+C_1\hm- C_2\hm\leq 0$ и~$\alpha_i$ такие, 
что условие $(C_2/(C_0\hm+C_1)-1) \alpha_{i+1}/(\alpha_{i+1}\hm-\alpha_i)\hm\geq 
i$ для всех $i\hm\geq 1$.
  
  Воспользуемся теоремой~1 из работы~\cite{14-ag}. Нетрудно заметить (см.\ 
равенство~(\ref{e6-ag})), что функция $Q(h)$ при $C_0\hm+C_1\hm\leq C_2$ и~функция $-Q(h)$ при $C_0\hm+C_1\hm> C_2$ удовлетворяют условиям 
теоремы~1 из~\cite{14-ag}. Тогда из указанной теоремы непосредственно следует 
справедливость следующего утверж\-де\-ния. 
  
  \smallskip
  
  \noindent
  \textbf{Утверждение.} \textit{При выполнении предположения, введенного 
выше относительно параметров $C_0$, $C_1$, $C_2$ и~$\alpha_i$, $i\hm\geq 1$, 
решение задачи~$(2)$ обладает следующими свойствами}: 
  \begin{enumerate}[(1)]
\item \textit{если $C_0+C_1\hm\leq C_2$, то $Q(h)$~--- унимодальная функция 
(так как $G(h)$ не возрастает по~$h$ и~$G(0)\hm\leq Q(0)$, и~если $[(C_0\hm+ 
C_1)(\beta\hm+ \mu) \hm- \beta C_3] (s\hm- \overline{S}^{(0)} )\hm\leq 
(C_0\hm+C_1\hm-C_2)\alpha_1$, то $h^*\hm=0$, иначе существует $0\hm< h^*\hm< \infty$};
\item \textit{если $C_0+C_1\hm >C_2$ и~$[(C_0\hm+ C_1)(\beta\hm+ \mu) \hm- \beta 
C_3] (s\hm- \overline{S}^{(0)} ) \hm+ (C_0\hm+ C_1\hm- C_2)\alpha_1\hm>0$, то 
$h^*\hm=\infty$ и~при этом $Q(h)$ монотонно возрастает по~$h$ $($так как 
$G(h)$ возрастает по~$h$ и~$G(0)\hm >Q(0))$};
\item \textit{если $C_0+C_1\hm> C_2$ и~$[(C_0\hm+ C_1)(\beta\hm+ \mu) \hm- \beta 
C_3] (s\hm- \overline{S}^{(0)} ) \hm+ (C_0\hm+ C_1\hm- C_2)\alpha_1\hm\leq0$, то 
функция $-Q(h)$ унимодальная (так как удовлетворяет условиям теоремы~$1$ 
из}~\cite{14-ag}) \textit{и~при этом}
$$
h^*=\begin{cases}
0\,, & \mbox{\textit{если}\ \ } Q(\infty) \leq Q(0);\\
\infty\,, & \mbox{\textit{если}\ \ } Q(\infty) > Q(0)
\end{cases}
$$
\textit{$($так как $G(h)$ возрастает по~$h$ 
и}~$G(0)\hm\leq Q(0)$$)$. 
\end{enumerate}

\smallskip

 На рис.~1 и~2 проиллюстрировано поведение функций $Q(h)$ и~$G(h)$ для двух 
наборов значений па\-ра\-мет\-ров рассматриваемой СМО: 
\begin{enumerate}[(1)]
\item рис.~1: $\lambda=8$; $\mu=2$; $\alpha_i=0{,}5$; $i\hm= \overline{1, h}$; 
$\beta\hm= 0{,}25$; $C_0\hm= 20$; $C_1\hm= 5$; $C_2\hm= 40$; $C_3\hm=10$; 
\item рис.~2:
$\lambda=8$; $\mu=1$; $\alpha_i=0{,}25$; $i\hm= \overline{1, h}$; $\beta\hm= 0{,}125$; 
$C_0\hm= 20$; $C_1\hm= 3$; $C_2\hm= 10$; $C_3\hm=15$. 
\end{enumerate}

Заметим, что в~случае, изображенном на рис.~1, целевая функция достигает 
максимума при пороговом значении $h^*\hm=18$, что согласуется 
с~утверж\-де\-ни\-ем пункта~1, а~в~случае рис.~2 оптимальное значение порога 
$h^*\hm=\infty$, что соответствует пункту~2. 



\section{Заключение}

  Практическим результатом проведенных выше исследований стал 
сле\-ду\-ющий прос\-той алгоритм оптимизации однопорогового управ\-ле\-ния 
оче\-редью для рассмотренной выше модели СМО при выполнении условий 
утверж\-де\-ния относительно па\-ра\-мет\-ров~$C_0$, $C_1$, $C_2$ и~$\alpha_i$, 
$i\hm\geq 1$.
  \begin{enumerate}[1.]
\item Если выполняется условие $C_0\hm+ C_1\hm\leq C_2$, то
\begin{enumerate}[(1)]
\item  положить $h\hm=0$;
\item  до тех пор пока выполняется неравенство $Q(h+1)\hm> Q(h)$, полагать 
$h\hm= h+1$;
\item  положить $h^*=h$.
\end{enumerate}
\item Если выполняются неравенства $C_0\hm+C_1\hm >C_2$ и~$[(C_0\hm+ 
C_1)(\beta\hm+ \mu) \hm- \beta C_3] (s\hm- \overline{S}^{(0)} ) \hm+ (C_0\hm+ 
C_1\hm- C_2)\alpha_1\hm>0$, то положить $h^*\hm= \infty$. 
\item Если $C_0\hm+C_1\hm> C_2$ и~$[(C_0\hm+ C_1)(\beta\hm+ \mu) \hm- \beta 
C_3] (s\hm- \overline{S}^{(0)} ) \hm+ (C_0\hm+ C_1\hm- C_2)\alpha_1\hm\leq 0$, то 
положить 
$$
h^*= \begin{cases}
0\,, &\mbox{если\ } Q(\infty)\leq Q(0);\\
\infty &\mbox{иначе.}
\end{cases}
$$
\end{enumerate}
  
  Обратим внимание, что при выполнении условий третьего пункта алгоритма 
справедливы неравенства $Q(0)\hm\leq 0$ и~$G(0)\hm\leq Q(0)$ и~на отрезке 
$[0,h^0]$, где~$h^0$~--- максимальное значение, такое что $Q(h^0)\hm\geq 
G(h^0)$, функция~$Q(h)$ не возрастает, а~при $h\hm\in [h^0,\infty)$ возрастает 
(так как в~случае пункта~3 функция $-Q(h)$ унимодальная). Следовательно, если 
$C_0\hm+C_1\hm\geq C_2$ и~выполняется условие $Q(0)\hm\leq Q(1)$, то 
$h^*\hm=\infty$.

{\small\frenchspacing
 {\baselineskip=11.5pt
 %\addcontentsline{toc}{section}{References}
 \begin{thebibliography}{99}
\bibitem{1-ag}
\Au{Floyd S., Jacobson~V.} Random early detection gateways for congestion avoidance~// 
IEEE ACM T. Network., 1993. Vol.~1. P. 397--413. doi: 10.1109/90.251892.

\bibitem{2-ag}
\Au{Коновалов М.\,Г.} Об одной задаче оптимального управ\-ле\-ния нагрузкой на сервер~// 
Информатика и~её применения, 2013. Т.~7. Вып.~4. С.~34--43. doi: 10.14357/19922264130404. EDN: 
RRROXB.


\bibitem{4-ag} %3
\Au{Konovalov M.\,G., Razumchik~R.\,V.} Comparison of two active queue management schemes 
through the $M/D/1/N$ queue~// Информатика и~её применения, 2018. Т.~12. Вып.~4. С.~9--15. doi: 
10.14357/19922264180402. EDN: VOGJOZ.

\bibitem{3-ag} %4
\Au{Агаларов Я.\,М.} Оптимизация объема буферной памяти узла коммутации при схеме 
полного разделения памяти~// Информатика и~её применения, 2018. Т.~12. Вып.~4. С.~25--32.
doi: 10.14357/19922264180404. EDN: YQHHGP.


\bibitem{5-ag}
\Au{Агаларов Я.\,М., Ушаков~В.\,Г.} Об унимодальности функции дохода системы массового 
обслуживания типа $G/M/s$ с~управ\-ля\-емой очередью~// Информатика и~её применения, 
2019. Т.~13. Вып.~1. С.~55--61. doi: 10.14357/19922264190108. EDN: HYAODW.
\bibitem{6-ag}
\Au{Коновалов М.\,Г., Разумчик~Р.\,В.} Комплексное управ\-ле\-ние в~одном классе систем 
с~параллельным обслуживанием~// Информатика и~её применения, 2019. Т.~13. Вып.~4.  
С.~54--59. doi: 10.14357/19922264190409. EDN: REESRH.
\bibitem{7-ag}
\Au{Агаларов Я.\,М.} Об оптимизации работы резервного прибора в~многолинейной системе 
массового обслуживания~// Информатика и~её применения, 2023. Т.~17. Вып.~1. С.~89--95.  doi: 
10.14357/19922264230112. EDN: FCYDUT.
\bibitem{8-ag}
\Au{Агаларов Я.\, М.} Оптимизация схемы распределения буферной памяти узла пакетной 
коммутации~// Информатика и~её применения, 2023. Т.~17. Вып.~3. С.~39--48. doi: 
10.14357/19922264230306. EDN: \mbox{ХQLXCKV}.

\bibitem{9-ag}
\Au{Кирпичников Ф.\,П., Флакс~Д.\,Б., Галямова~К.\,Н.} Средняя длина очереди в~сис\-те\-ме 
массового обслуживания с~ограниченным средним временем пребывания заявки в~сис\-те\-ме~// 
Вестник Технологического университета, 2017. Т.~20. №\,2. С.~81--84. EDN: 
XVFSTN.

\bibitem{10-ag}
\Au{Савинов Ю.\,Г., Табакова~Е.\,Д., Сафиуллов~И.\,Д.} Оптимизация в~СМО с~нетерпеливыми 
заявками~// Ученые записки УлГУ. Сер. Математика и~информационные технологии, 2019. 
№\,1. С.~92--98. EDN: OWOZYR.

\bibitem{11-ag}
\Au{Мейханаджян Л.\,А., Разумчик~Р.\,В.} Система массового обслуживания 
$\mathrm{Geo}/G/1/\infty$ с~инверсионным порядком обслуживания и~ресамплингом в~дискретном 
времени~// Информатика и~её применения, 2019. Т.~13. Вып.~4. С.~60--67. doi: 
10.14357/19922264190410. EDN: LNIHGC.

\bibitem{12-ag}
\Au{Милованова Т.\,А., Разумчик~Р.\,В.} Однолинейная система массового обслуживания 
с~инверсионным порядком обслуживания с~вероятностным приоритетом, групповым 
пуассоновским потоком и~фоновыми заявками~// Информатика и~её применения, 2020. Т.~14. 
Вып.~3. С.~26--34. doi: 10.14357/19922264200304. EDN: NOMSAM.
\bibitem{13-ag}
\Au{Берговин А.\,К., Ушаков~В.\,Г.} Исследование сис\-тем обслуживания со смешанными 
приоритетами~// Информатика и~её применения, 2023. Т.~17. Вып.~2. С.~57--61. doi: 
10.14357/19922264230208. EDN: \mbox{JULPWS}.



\bibitem{14-ag}
\Au{Агаларов Я.\,М.} Признак унимодальности целочисленной функции одной переменной~// 
Обозрение прикладной и~промышленной математики, 2019. Т.~26. Вып.~1. С.~65--66.


\end{thebibliography}

 }
 }

\end{multicols}

\vspace*{-6pt}

\hfill{\small\textit{Поступила в~редакцию 23.02.24}}

%\vspace*{10pt}

%\pagebreak

\newpage

\vspace*{-28pt}

%\hrule

%\vspace*{2pt}

%\hrule



\def\tit{ON SINGLE-THRESHOLD QUEUE MANAGEMENT\\ IN~A~QUEUING SYSTEM 
WITH~IMPATIENT CUSTOMERS}


\def\titkol{On single-threshold queue management in~a~queuing system 
with~impatient customers}


\def\aut{Ya.\,M.~Agalarov}

\def\autkol{Ya.\,M.~Agalarov}

\titel{\tit}{\aut}{\autkol}{\titkol}

\vspace*{-8pt}


\noindent
Federal Research Center ``Computer Science and Control'' of the Russian Academy of 
Sciences, 44-2~Vavilov Str., Moscow 119333, Russian Federation

\def\leftfootline{\small{\textbf{\thepage}
\hfill INFORMATIKA I EE PRIMENENIYA~--- INFORMATICS AND
APPLICATIONS\ \ \ 2024\ \ \ volume~18\ \ \ issue\ 2}
}%
 \def\rightfootline{\small{INFORMATIKA I EE PRIMENENIYA~---
INFORMATICS AND APPLICATIONS\ \ \ 2024\ \ \ volume~18\ \ \ issue\ 2
\hfill \textbf{\thepage}}}

\vspace*{4pt}




\Abste{The results of a theoretical study of a~managed queuing system of $M/M/k$ type 
with impatient customers and single-threshold queue management are presented. The task of optimizing single-threshold 
queue management is set, the essence of which is to calculate for the queue length a~certain threshold value that 
maximizes a given objective function. In the system under study, a~customer leaves the system unattended if 
the waiting time in the queue (or the service time on the device) exceeds a~certain time interval of random 
length distributed according to an exponential law with a~given parameter. A~cost function is used as an 
indicator of the effectiveness of queue management (objective function) which takes into account the losses per 
unit of time due to system technical maintenance, rejection of customers at the entrance of the system, and 
leaving of customers until the end of the service. A~method for solving the problem of maximizing the cost 
objective function on a~set of single-threshold queue controls and an algorithm for guaranteed calculation of the 
optimal threshold are proposed.}

\KWE{queuing system; impatient customers; queue management}

\DOI{10.14357/19922264240206}{JZHAKU}

%\vspace*{-12pt}

%\Ack

%\vspace*{-3pt}


 %    \noindent
  


  \begin{multicols}{2}

\renewcommand{\bibname}{\protect\rmfamily References}
%\renewcommand{\bibname}{\large\protect\rm References}

{\small\frenchspacing
 {%\baselineskip=10.8pt
 \addcontentsline{toc}{section}{References}
 \begin{thebibliography}{99} 
\bibitem{1-ag-1}
\Aue{Floyd, S., and V.~Jacobson.} 1993. Random early detection gateways for congestion avoidance. 
\textit{IEEE ACM T. Network.} 1:397--413. doi: 10.1109/90.251892.
\bibitem{2-ag-1}
\Aue{Konovalov, M.\,G.} 2013. Ob odnoy zadache optimal'nogo upravleniya nagruzkoy na server [About one 
task of overload control]. \textit{Informatika i~ee Primeneniya~--- Inform. Appl.} 7(4):34--43. doi: 
10.14357/19922264130404. EDN: RRROXB.

\bibitem{4-ag-1}
\Aue{Konovalov, M., and R.~Razumchik}. 2018. Comparison of two active queue management schemes 
through the $M/D/1/N$ queue. \textit{Informatika i~ee Primeneniya~--- Inform. Appl.} 12(4):9--15. doi: 
10.14357/19922264180402. EDN: VOGJOZ.

\bibitem{3-ag-1}
\Aue{Agalarov, Ya.\,M.}  2018. Optimizatsiya ob''ema bufernoy pamyati uzla kommutatsii pri skheme polnogo 
razdeleniya pamyati [Optimization of buffer memory size of switching node in mode of full memory sharing]. 
\textit{Informatika i~ee Primeneniya~--- Inform. Appl.} 12(4):25--32. doi: 10.14357/19922264180404. EDN: 
YQHHGP.

\bibitem{5-ag-1}
\Aue{Agalarov, Ya.\,M., and V.\,G.~Ushakov.} 2019. Ob unimodal'nosti funktsii dokhoda sistemy massovogo 
obsluzhivaniya tipa $G/M/s$ s~upravlyaemoy ochered'yu [On the unimodality of the income function of a~type 
$G/M/s$ queueing system with controlled queue]. \textit{Informatika i~ee Primeneniya~--- Inform. Appl.} 
13(1):55--61. doi: 10.14357/19922264190108. EDN: HYAODW.
\bibitem{6-ag-1}
\Aue{Konovalov, M.\,G., and R.\,V.~Razumchik.} 2019. Komp\-leks\-noe upravlenie v~odnom klasse sistem 
s~parallel'nym obsluzhivaniem [Mixed policies for online job allocation in one class of systems with parallel 
service]. \textit{Informatika i~ee Primeneniya~--- Inform. Appl.} 13(4):54--59. doi: 10.14357/19922264190409. 
EDN: REESRH.
\bibitem{7-ag-1}
\Aue{Agalarov, Ya.\,M.} 2023. Ob optimizatsii raboty rezervnogo pribora v~mnogolineynoy sisteme 
massovogo obsluzhivaniya [Optimization of a~queue-length dependent additional server in the multiserver 
queue]. \textit{Informatika i~ee Primeneniya~--- Inform. Appl.} 17(1):89--95. doi: 10.14357/19922264230112. 
EDN: FCYDUT.
\bibitem{8-ag-1}
\Aue{Agalarov, Ya.\,M.} 2023. Optimizatsiya skhemy raspredeleniya bufernoy pamyati uzla paketnoy 
kommutatsii [Optimization of the buffer memory allocation scheme of the packet switching node]. 
\textit{Informatika i~ee Primeneniya~--- Inform. Appl.}  17(3):39--48. doi: 10.14357/19922264230306. EDN: 
QLXCKV.
\bibitem{9-ag-1}
\Aue{Kirpichnikov, F.\,P., D.\,B.~Flaks, and K.\,N.~Galyamova.} 2017. Srednyaya dlina ocheredi v~sisteme 
massovogo obsluzhivaniya s~ogranichennym srednim vremenem prebyvaniya zayavki v~sisteme [The average 
queue length in a~queuing system with a~limited average time for the request to stay in the system]. 
\textit{Vestnik Tekhnologicheskogo universiteta} [Bulletin of  Technological University] 
20(2):81--84. EDN: XVFSTN.
\bibitem{10-ag-1}
\Aue{Savinov, Yu.\,G., E.\,D.~Tabakova, and I.\,D.~Safiullov.} 2019. Optimizatsiya v~SMO s~neterpelivymi 
zayavkami [Optimization in the queuing system with impatient customers]. \textit{Uchenyye zapiski UlGU. Ser. 
Matematika i~informatsionnye tekhnologii} [Scientific Notes of UlSU. Ser. Mathematics and Information Technology] 1:92--98. EDN: \mbox{OWOZYR}.

\bibitem{11-ag-1}
\Aue{Meykhanadzhyan, L.\,A., and R.\,V.~Razumchik.} 2019. Sistema massovogo obsluzhivaniya 
$\mathrm{Geo}/G/1/\infty$ s~inversionnym poryadkom obsluzhivaniya i~resamplingom v~diskretnom vremeni 
[Discrete-time $\mathrm{Geo}/G/1/\infty$ LIFO queue with resampling policy]. \textit{Informatika i~ee Primeneniya~--- Inform. 
Appl.} 13(4):60--67. doi: 10.14357/ 19922264190410. EDN: LNIHGC.
\bibitem{12-ag-1}
\Aue{Milovanova, T.\,A., and R.\,V.~Razumchik.} 2020. Od\-no\-li\-ney\-naya sistema massovogo obsluzhivaniya 
s~in\-ver\-si\-on\-nym poryadkom obsluzhivaniya s~veroyatnostnym prioritetom, gruppovym puassonovskim 
\mbox{potokom} i~fonovymi zayavkami [A~single-server queueing system with \mbox{LIFO} service, probabilistic priority, 
batch Poisson arrivals, and background customers]. \textit{Informatika i~ee Primeneniya~--- Inform. Appl.} 
14(3):26--34. doi: 10.14357/ 19922264200304. EDN: NOMSAM.
\bibitem{13-ag-1}
\Aue{Bergovin, A.\,K., and V.\,G.~Ushakov.} 2023. Issledovanie sistem obsluzhivaniya so smeshannymi 
prioritetami [Analysis of the queueing systems with mixed priorities]. \textit{Informatika i~ee Primeneniya~--- 
Inform. Appl.} 17(2):57--61. doi: 10.14357/19922264230208. EDN: \mbox{JULPWS}.
{\looseness=1

}
\bibitem{14-ag-1}
\Aue{Agalarov, Ya.\,M.} 2019. Priznak unimodal'nosti tselochislennoy funktsii odnoy peremennoy [A~sign of 
unimodality of an integer function of one variable]. \textit{Obozrenie prikladnoy i~promyshlennoy matematiki} 
[Surveys Applied and Industrial Mathematics] 26(1):65--66.

\end{thebibliography}

 }
 }

\end{multicols}

\vspace*{-6pt}

\hfill{\small\textit{Received February 23, 2024}} 

\vspace*{-18pt}


\Contrl

\vspace*{-3pt}

\noindent
\textbf{Agalarov Yaver M.} (b.\ 1952)~--- Candidate of Science (PhD) in technology, associate professor, 
leading scientist, Federal Research Center ``Computer Science and Control'' of the Russian Academy of 
Sciences, 44-2~Vavilov Str., Moscow 119333, Russian Federation; \mbox{agglar@yandex.ru}




\label{end\stat}

\renewcommand{\bibname}{\protect\rm Литература} 