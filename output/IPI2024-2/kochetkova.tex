\def\stat{kochetkova}

\def\tit{ВЕРОЯТНОСТНАЯ МОДЕЛЬ ЗАТУХАНИЯ МОЩНОСТИ СИГНАЛА В СЦЕНАРИЯХ 3GPP TR 38.901 
РАЗВЕРТЫВАНИЯ СЕТИ 5G$^*$}

\def\titkol{Вероятностная модель затухания мощности сигнала в~сценариях 3GPP TR 38.901 
развертывания сети 5G}

\def\aut{Е.\,Д.~Макеева$^1$, И.\,А.~Кочеткова$^2$, С.\,Я.~Шоргин$^3$}

\def\autkol{Е.\,Д.~Макеева, И.\,А.~Кочеткова, С.\,Я.~Шоргин}

\titel{\tit}{\aut}{\autkol}{\titkol}

\index{Макеева Е.\,Д.}
\index{Кочеткова И.\,А.}
\index{Шоргин С.\,Я.}
\index{Makeeva E.\,D.}
\index{Kochetkova I.\,A.}
\index{Shorgin S.\,Ya.}


{\renewcommand{\thefootnote}{\fnsymbol{footnote}} \footnotetext[1]
{Публикация выполнена в~рамках проекта №\,025319-2-000 Системы грантовой 
поддержки научных проектов РУДН.}}


\renewcommand{\thefootnote}{\arabic{footnote}}
\footnotetext[1]{Российский университет дружбы народов имени Патриса Лумумбы; 
Институт проблем управления имени В.\,А.~Трапезникова Российской академии наук, 
\mbox{elena-makeeva-96@mail.ru}}
\footnotetext[2]{Российский университет дружбы народов имени Патриса Лумумбы; 
Федеральный исследовательский центр <<Информатика и~управ\-ле\-ние>> Российской 
академии наук, \mbox{kochetkova-ia@rudn.ru}}
\footnotetext[3]{Федеральный исследовательский центр <<Информатика 
и~управ\-ле\-ние>> Российской академии наук, \mbox{sshorgin@ipiran.ru}}

\vspace*{2pt}






\Abst{Сети пятого (5G) и~последующих поколений будут использовать терагерцевый диапазон 
радиочастот, что обеспечит сверхвысокую скорость передачи данных. Однако при 
этом возможны потери сигнала при прохождении через препятствия. Поэтому 
становится крайне важным моделирование распространения сигнала с~по\-мощью 
стохастической геометрии и~использование актуальных моделей затухания сигнала. 
Модели для описания затухания сигнала для различных сценариев развертывания сети 
5G в~виде эмпирических формул содержатся в~спецификации 3GPP TR 38.901. Тем не 
менее обычно для построения моделей стохастической геометрии используются 
упрощенные виды формул. В~\mbox{статье} представлена функция распределения (ФР) затухания 
мощности сигнала при случайном расположении пользователей в~соответствии со 
сценариями, описанными в~3GPP TR 38.901. На численных примерах показано, что 
разница значений с~упрощенной формулой значительна и~может привести к~занижению 
оценки пропускной способности сети.}

\KW{беспроводная сеть; 5G; 3GPP TR 38.901; мощ\-ность затухания сигнала; прямая 
видимость; непрямая ви\-ди\-мость; сто\-ха\-сти\-че\-ская гео\-метрия}

\DOI{10.14357/19922264240204}{EKLCAP}
  
%\vspace*{-6pt}


\vskip 10pt plus 9pt minus 6pt

\thispagestyle{headings}

\begin{multicols}{2}

\label{st\stat}



\section{Введение}

Сети пятого и~последующих поколений будут использовать терагерцевый 
диапазон радиочастот, чтобы обеспечить сверхвысокую скорость передачи данных и~пропускную способность. Однако использование миллиметровых волн связано со 
сложностями из-за потери сигнала при про\-хож\-де\-нии препятствий. Таким образом, для 
обеспечения производительности сетей~5G становится крайне важным моделирование 
распространения сигнала. Формула Шен\-но\-на--Харт\-ли с~формулой Фрииса задают 
пропускную способность канала
$$
C=B \log_2 \left(1+\fr{P_t G_t G_r}{(N+I) \mathrm{PL}}\right),
$$
 где $B$~--- полоса 
пропускания канала; $P_t$~--- мощ\-ность передающей антенны; $G_t$~--- коэффициент 
усиления передающей антенны; $G_r$~--- коэффициент усиления приемной антенны; $N$~--- мощ\-ность шума; 
$I$~--- мощ\-ность интерференции; $\mathrm{PL}$~--- мощ\-ность затухания 
сигнала (path loss, PL) на расстоянии от передающей антенны до приемной 
антенны~\cite{Moltchanov2022a}.
Пропускная способность канала уже далее используется в~управ\-ле\-нии занятием 
радиоресурсов базовой станции (БС) для соблюдения необходимого качества обслуживания 
пользователей по требуемой ско\-рости передачи данных.

 Ввиду того что пользователи находятся на разных расстояниях от БС, 
значения мощностей затухания сигнала будут случайными. Как показано в~работе~\cite{Hmamouche2021}, для учета влияния на пропускную\linebreak
 способность канала 
случайного положения пользователей в~соте применяется стохастическая гео\-мет\-рия. 
Рассмотреть совместное занятие радиоресурсов и~случайный характер поведения 
пользователей позволяет модель на основе аппарата \mbox{ресурсных} сис\-тем массового 
обслуживания~\cite{Naumov2016, Gorbunova2018}. Такие модели применяются для 
исследования различных сценариев развертывания сетей 
5G~\cite{Moltchanov2022b, Markova2019}, например при анализе совместного 
обслуживания трафика со сверхнизкой задержкой и~широкополосного трафика~\cite{Kochetkova2021}.

\begin{figure*}[b] %fig1
\vspace*{-6pt}
      \begin{center}
     \mbox{%
\epsfxsize=124.62mm 
\epsfbox{koc-1.eps}
}
\end{center}
\vspace*{-9pt}
\Caption{Схема системной модели}
\label{fig1}
\end{figure*}

Модели для описания мощности $\mathrm{PL}$ затухания сигнала для разных сценариев 
отражены в~спецификации 3GPP TR 38.901~\cite{3GPP38901}. И~если при проведении 
имитационного моделирования исследователи по большей части полностью реализуют 
эти модели~\cite{Bolla2023}, то при построении моделей стохастической гео\-мет\-рии 
зачастую применяется упрощенный вид формул.
В~обзоре~\cite{Hmamouche2021} рассмотрены различные виды функциональной 
зависимости затухания мощ\-ности сигнала от расстояния между пользователем 
и~БС, которые применяют исследователи. Например, для простоты 
расчетов в~работе~\cite{Moltchanov2022b} используются упрощенные формулы без 
учета ку\-соч\-но-за\-дан\-но\-го вида функции для прямой видимости и~максимума нескольких 
величин мощностей PL для непрямой видимости при по\-стро\-ении~ФР.

В данной статье получена ФР затухания мощ\-ности сигнала при случайном 
расположении пользователей в~соответствии со сценариями 3GPP TR~38.901 
развертывания сети 5G. Использованы формулы из этой спецификации, где приведены 
зависимости PL от расстояния между пользователем и~БС. В~данной 
статье закон распределения пользователей в~соте взят произвольный, а~для 
численного анализа~--- в~соответствии с~типовыми рекомендованными значениями 
параметров сценариев.



\section{Затухание сигнала как функция от параметров сценария 3GPP} \label{sec2}

При исследовании распространения сигнала необходимо учитывать множество 
параметров сети, таких как частота, основные характеристики местности, высота 
принимающей и~передающей антенн, конфигурация антенн и~другие факторы. Для 
упрощения расчетов мощности PL затухания сигнала стандартом 3GPP TR~38.901~\cite{3GPP38901} 
были выделены основные сценарии развертывания сети~5G: 
мак\-ро\-со\-та в~городе (urban macro, UMa), микросота в~городе (urban micro, UMi), 
мак\-ро\-со\-та в~сельской местности (rural macro, RMa), точка доступа внут\-ри 
помещения (indoor hotspot, InH) и~крытая фабрика (indoor factory, InF),~--- 
и~путем экспериментов были получены эмпирические модели затухания сигнала для них. 
На основе этих моделей и~в~предположении случайного характера поведения 
пользователей в~данном разделе получена ФР мощности затухания сигнала с~учетом 
особенностей, описанных в~данной спецификации.


Рассмотрим общее описание предлагаемых сценариев (рис.~\ref{fig1}). Пусть 
передающая антенна БС расположена на высоте~$h_{\mathrm{BS}}$, 
использует несущую частоту~$f_c$ и~создает покрытие радиуса~$R$. 
Пользовательские устройства (ПУ) находятся на высоте~$h_{\mathrm{UT}}$, а~проекция 
расстоянии от ПУ до БС со\-став\-ля\-ет $d$.

В зависимости от своего расположения ПУ может находиться в~зоне прямой видимости 
(line-of-sight, LOS) с~устойчивым уровнем сигнала или вне этой зоны (non-line-of-sight, NLOS). 
Если ПУ расположено на расстоянии~$d$, то ве\-ро\-ят\-ность того, что 
ПУ находится в~зоне прямой видимости, пред\-став\-ля\-ет собой кусочно-заданную 
функцию:
\begin{equation}
\label{eq1}
{\mathrm{Pr}}_{\mathrm{LOS}}(d)=
\begin{cases}
{\mathrm{Pr}}_1^{\mathrm{LOS}}(d), & 0=r_0 \leq d < r_1; \\
{\mathrm{Pr}}_2^{\mathrm{LOS}}(d), & r_1 \leq d < r_2; \\
\cdots & \cdots \\
{\mathrm{Pr}}_I^{\mathrm{LOS}}(d), & r_{I-1} \leq d \leq r_I=R,
\end{cases}
\end{equation}
где радиусы $R_i$ определяют границы интервалов. 

Тогда мощ\-ность $\mathrm{PL}(d)$ 
затухания сигнала примет вид:
\begin{multline}
\label{eq2}
\mathrm{PL}\,(d)= \mathrm{PL}_{\mathrm{LOS}}(d)  {\mathrm{Pr}}_{\mathrm{LOS}}(d) + {}\\
{}+\mathrm{PL}_{\mathrm{NLOS}}(d)  \left[1-
{\mathrm{Pr}}_{\mathrm{LOS}}(d)\right].
\end{multline}

Мощность затухания сигнала в~условиях прямой видимости LOS описывается ку\-соч\-но-за\-дан\-ной функцией
\begin{multline}
\label{eq3}
\mathrm{PL}^{\mathrm{LOS}}(d)={}\\
{}=
\begin{cases}
\mathrm{PL}_1^{\mathrm{LOS}}(d), & 0=d_0 \leq d < d_1;\\
\mathrm{PL}_2^{\mathrm{LOS}}(d), & d_1 \leq d < d_2;\\
\cdots & \cdots \\
\mathrm{PL}_J^{\mathrm{LOS}}(d), & d_{J-1} \leq d \leq d_J=R,
\end{cases}
\end{multline}
где $d_j$~--- границы интервалов (break point distance), а~в~условиях непрямой 
видимости NLOS пред\-став\-ля\-ет собой максимум
\begin{multline}
\label{eq4}
\mathrm{PL}_{\mathrm{NLOS}}(d) = {}\\
\!\!{}=\!
\max\left(\mathrm{PL}^{\mathrm{LOS}}(d),\mathrm{PL}^{\mathrm{NLOS}}_1(d), \ldots, \mathrm{PL}^{\mathrm{NLOS}}_K(d)\right)\!.\!\!
\end{multline}

Каждая из компонент функций для случаев LOS и~NLOS имеет схожую структуру:
\begin{multline}
 \mathrm{PL}^{l}_m(d)[\mathrm{dB}] = \alpha_m^{l}[\mathrm{dB}]+\beta_m^{l}[\mathrm{dB}]\log_{10}{D(d)},
 \\
 \mathrm{PL}^{l}_m(d) = \alpha_m^{l} \cdot D^{\beta_m^{l}}(d),
\\
 l=\begin{cases}
 \mbox{``}\mathrm{LOS}\mbox{''}, & m=j=\overline{0,J}\,; \\
 \mbox{``}\mathrm{NLOS}\mbox{''},& m=k=\overline{0,K}\,,
 \end{cases}
\label{eq5}
\end{multline}
где $D(d)=\sqrt{d^2+(h_{\mathrm{BS}}\hm-h_{\mathrm{UT}})^2}$~--- расстояние от ПУ до БС в~трехмерном 
пространстве; $\alpha$ и~$\beta$~--- коэффициенты модели затухания сигнала~--- 
константы для каждого из сценариев 3GPP TR~38.901.



\section{Функция распределения затухания сигнала при~случайном расположении 
пользователей} \label{sec3}

Примем теперь, что расстояние между ПУ и~БС~--- случайная величина (СВ)~$\xi_d$ 
со значениями~$d$ и~ФР~$F_{\xi_d}(x)$. Тогда расстояние от ПУ до БС в~трехмерном 
пространстве~$\xi_D$ будет функцией от СВ~$\xi_d$ с~ФР

\noindent
\begin{multline*}
F_{\xi_D}(x)  =
\mathrm{Pr}\,(\xi_D \leq x) ={}\\
{}=
\mathrm{Pr}\left(\sqrt{\xi_d^2+(h_{\mathrm{BS}}- h_{\mathrm{UT}})^2} \leq x \right) ={} \\
{} = \mathrm{Pr}\left(\xi_d \leq \sqrt{x^2-(h_{\mathrm{BS}}- h_{\mathrm{UT}})^2} \right) ={}\\
{}=
F_{\xi_d}\left(\sqrt{x^2-(h_{\mathrm{BS}}- h_{\mathrm{UT}})^2} \right).
\end{multline*}

Случайная величина $\xi_m^l$~--- компонента функции затухания сигнала~--- зависит от СВ $\xi_D$ и~по формуле~(\ref{eq5}) имеет ФР
\begin{multline*}
F_{\xi_m^l}(x)  =
\mathrm{Pr}\,(\xi_m^l \leq x) =
\mathrm{Pr}\left(\alpha_m^{l}  ({\xi}_D)^{\beta_m^{l}} \leq x \right) = {}\\
{}=
\mathrm{Pr}\left(\xi_D \leq \left(\fr{x}{\alpha_m^{l}}\right)^{{1}/{\beta_m^{l}}} \right) = 
F_{\xi_D}\left( \left( \fr{x}{\alpha_m^{l}}\right)^{{1}/{\beta_m^{l}}} 
\right) ={}\\
{}=  F_{\xi_d}\left( \sqrt{ 
\left(\fr{x}{\alpha_m^{l}}\right)^{{2}/{\beta_m^{l}}} - \left(h_\mathrm{BS}-h_\mathrm{UT}\right)^2 }\right), \\
 l=\begin{cases}
 \mbox{``}\mathrm{LOS}\mbox{''}, &  m=j=\overline{0,J}\,; \\
 \mbox{``}\mathrm{NLOS}\mbox{''}, & m=k=\overline{0,K}\,.
 \end{cases}
\end{multline*}

Для затухания сигнала в~условиях прямой видимости ФР СВ~$\xi_{\mathrm{LOS}}$ по 
формуле~(\ref{eq3}) примет вид:
\begin{multline}
F_{\xi_{\mathrm{LOS}}}(x) =
\mathrm{Pr}\,(\xi_{\mathrm{LOS}} \leq x) = {} \\
{}= \sum\limits_{j=1}^J  \mathrm{Pr}\left(\xi_{\mathrm{LOS}} \leq x \mid d_{j-1} \leq \xi_d < d_j\right) \times{}\\
{}\times \mathrm{Pr}\left(d_{j-1} \leq  \xi_d < d_j\right) = {}\\
{} = \sum\limits_{j=1}^J F_{\xi_j^{\mathrm{LOS}}}(x) \left[ F_{\xi_d}(d_j) - 
F_{\xi_d}(d_{j-1}) \right], 
\label{eq6}
\end{multline}
а для непрямой видимости ФР СВ $\xi_{\mathrm{NLOS}}$ по формуле~(\ref{eq4}) 
и~с~учетом~\cite{Ventzel2018} запишем как
\begin{multline}
F_{\xi_{\mathrm{NLOS}}}(x)  =
\mathrm{Pr}\,(\xi_{\mathrm{NLOS}} \leq x) ={}\\
{}=
\mathrm{Pr}\left(\max{\left(\xi_{\mathrm{LOS}}, \; \xi^{\mathrm{NLOS}}_1, \ldots, \xi^{\mathrm{NLOS}}_K \right)} 
\leq x \right) = {} \\
{} = \mathrm{Pr}\left(\xi_{\mathrm{LOS}} \leq x, \; \xi^{\mathrm{NLOS}}_1 \leq x, \ldots, \xi^{\mathrm{NLOS}}_K 
\leq x \right) = {}\\
{} = \mathrm{Pr}\,(\xi_{\mathrm{LOS}} \leq x)  \prod\limits_{k=1}^K{\mathrm{Pr}\left(\xi^{\mathrm{NLOS}}_k \leq x 
\right)} = {}\\
{}=
F_{\xi_{\mathrm{LOS}}}(x)  \prod\limits_{k=1}^K F_{\xi_k^{\mathrm{NLOS}}}(x). 
\label{eq7}
\end{multline}

Наконец, ФР СВ $\xi_{\mathrm{PL}}$ затухания сигнала по формуле~(\ref{eq2}) запишем 
следующим образом:
\begin{multline*}
F_{\xi_{\mathrm{PL}}}(x) =
\mathrm{Pr}\,(\xi_{\mathrm{PL}} \leq x) ={}\\
{}=
\mathrm{Pr}\left(\xi_{\mathrm{LOS}}  \xi_{\mathrm{Pr}_{\mathrm{LOS}}} + \xi_{\mathrm{NLOS}} \left[1-
\xi_{\mathrm{Pr}_{\mathrm{LOS}}}\right] \leq x \right),
\end{multline*}
где $\xi_{\mathrm{Pr}_{\mathrm{LOS}}}$~--- СВ вероятности расположения ПУ в~зоне прямой 
видимости~(\ref{eq1}).
Функция распределения $F_{\xi_{\mathrm{PL}}}(x)$ будет приближенно представлять собой свертку.



\section{Численный анализ}


\begin{figure*}[b]\small
\begin{center}
\tabcolsep=4pt
\begin{tabular}{|c|l|c|c|}

\multicolumn{4}{c}{Коэффициенты модели затухания сигнала для UMa и~UMi}\\
\multicolumn{4}{c}{\ }\\[-6pt]
\hline
 Зона& \multicolumn{1}{c|}{[dB]} & UMa & UMi \\
\hline
&&&\\[-9pt]
 & $\alpha_1^{\mathrm{LOS}}$ & $28+20 \log_{10}{f_c}$ & $32{,}4+20\log_{10}{f_c}$\\
% \cline{2-4}
 LOS& $\beta_1^{\mathrm{LOS}}$ & $22$ & $21$\\
% \cline{2-4}
 & $\alpha_2^{\mathrm{LOS}}$ & $28+20 \log_{10}{f_c}-9\log_{10}{\left(d_1^2+(h_{\mathrm{BS}}-h_{\mathrm{UT}})^2\right)} $ 
 & $32{,}4+20\log_{10}{f_c}- 9{,}5\log_{10}{\left(d_1^2+(h_{\mathrm{BS}}-h_{\mathrm{UT}})^2\right)}$\\
% \cline{2-4}
 & $\beta_2^{\mathrm{LOS}}$ & $40$ & $40$\\
\hline
&&&\\[-9pt]
& $\alpha_1^{\mathrm{NLOS}}$ & $13{,}54+20\log_{10}{f_c}-0{,}6(h_{\mathrm{UT}}-1,5)$ & 
$22{,}4+21{,}3\log_{10}{f_c} - 0{,}3(h_{\mathrm{UT}}-1{,}5)$ \\
 %\cline{2-4}
NLOS  & $\beta_1^{\mathrm{NLOS}}$ & $39{,}08$ & $35{,}3$\\
 %\cline{2-4}
 & $\alpha_{\mathrm{Opt}}$ & $32{,}4+20\log_{10}{f_c}$ & $32{,}4+20\log_{10}{f_c} $\\
 %\cline{2-4}
 & $\beta_{\mathrm{Opt}}$ & $30$ & $31{,}9$\\
 \hline
\end{tabular}
\end{center}
%\end{table*}
%\begin{figure*}[b] %fig2
\setcounter{figure}{1}
\vspace*{12pt}
      \begin{center}
     \mbox{%
\epsfxsize=163mm 
\epsfbox{koc-2.eps}
}
\end{center}
\vspace*{-10pt}
  \Caption{Функции распределения PL для LOS~(\textit{а}) и ~NLOS~(\textit{б}) для сценариев UMa (черные кривые) и~UMi~(серые кривые):
  \textit{1}~--- расчет по формулам~(6) для LOS и~(7) для NLOS; \textit{2}~--- расчет по упрощенным формулам}
 \label{fig:1}
 \end{figure*}

 В спецификации 3GPP TR 38.901 указаны основные диапазоны значений параметров 
для сценариев развертывания сетей~5G. Рассмотрим сценарии макросоты UMa и~микросоты UMi в~городе со следующим набором исходных данных: радиус действия БС 
$R\hm=5000$~м, центральная частота $f_c\hm=6$~ГГц, высота ПУ
$h_{\mathrm{UT}}\hm=1{,}5$~м, высоты БС для UMa $h_{\mathrm{BS}}\hm=25$ м и~для UMi $h_{\mathrm{BS}}\hm=10$~м. 
Предположим, что пользователи распределены равномерно в~об\-ласти действия БС 
радиусом~$R$.

Согласно упрощенным формулам, подобным тем, что описаны в~работе~\cite{Moltchanov2022b}, ФР мощности PL затухания сигнала в~условиях 
прямой и~непрямой видимости могут быть представлены как 
$$
F_{\xi_\mathrm{LOS}}(x) = 
F_{\xi_1^{\mathrm{LOS}}}(x)\,;
$$

\noindent
\begin{multline*}
F_{\xi_\mathrm{NLOS}}(x)=F_{\mathrm{Opt}}(x)={}\\
{}= F_{\xi_d}\left( \sqrt{ 
\left(\fr {x}{\alpha_{\mathrm{Opt}}}\right)^{{2}/{\beta_{\mathrm{Opt}}}} - 
(h_\mathrm{BS}-h_\mathrm{UT})^2 }\,\right).
\end{multline*}

\noindent
 Коэффициенты модели затухания сигнала $\alpha_m^{l}$ 
[dB] и~$\beta_m^{l}$ [dB] для сценариев UMa и~UMi, согласно~\cite{3GPP38901}, 
представлены в~таб\-лице.


 
 \begin{figure*} %fig3
 \vspace*{1pt}
      \begin{center}
     \mbox{%
\epsfxsize=163.204mm 
\epsfbox{koc-3.eps}
}
\end{center}
\vspace*{-15pt}
 \Caption{Разница значений ФР PL для LOS~(\textit{а}) и~NLOS~(\textit{б}):
 \textit{1}~--- UMa; \textit{2}~--- UMi}
 \label{fig:2}
  \vspace*{-3pt}
 \end{figure*}

 
 

В рамках данного численного анализа покажем графики ФР моделей PL затухания 
сигнала для сценариев UMa и~UMi в~условиях прямой и~непрямой видимости по 
формулам~(\ref{eq6}) и~(\ref{eq7}) и~упрощенным формулам. Графики с~полученными 
результатами представлены на рис.~\ref{fig:1} и~3. Во всех случаях 
график ФР по упрощенным формулам идет выше, что представляет собой 
верхнюю оценку. Однако при дальнейших расчетах с~использованием упрощенных 
формул пропускная спо\-соб\-ность и~максимальное число обслуженных пользователей 
в~соте оказываются занижены. Таким образом, авторы рекомендуют при использовании 
модели затухания сигнала как компоненты, например в~ресурсных сис\-те\-мах массового 
обслуживания для анализа беспроводных сетей, использовать формулы (\ref{eq6}) и~(\ref{eq7}).


\section{Заключение}

В статье исследована модель затухания сигнала по формулам сценариев 3GPP TR 
38.901. Была получена функция распределения (ФР) мощности затухания сигнала при 
случайном (произвольный закон) расположении пользователей в~зоне покрытия 
БС. Она учитывает ку\-соч\-но-за\-дан\-ный вид функции для LOS и~максимум 
нескольких величин для NLOS. Проведен численный анализ для данных из 
спецификации 3GPP TR 38.90 для сценариев макро- и~микросот в~городе для 
сравнения ФР, представленных в~данной работе, и~ФР, полученных с~по\-мощью 
упрощенных формул. Результат анализа показал, что ФР по упрощенным формулам дает 
оценку сверху, что может понижать точность расчетов пропускной способности 
канала. Отметим, что авторы статьи не ставили перед собой задачу аналитического 
сравнения двух ФР, а~хотели бы обратить внимание на несложный вид полученных 
формул, которые рекомендуют для использования как компоненту в~ресурсных 
системах массового обслуживания при моделировании беспроводных сетей 5G/6G. 
Задачей дальнейшего исследования станет разработка ресурсной системы массового 
обслуживания с~учетом случайного расположения пользователей в~соте через 
пред\-став\-лен\-ную ФР для оценки схемы приоритетного обслуживания узкополосного 
трафика и~прерывания обслуживания широкополосного трафика в~се\-ти~5G.


\vspace*{-12pt}

{\small\frenchspacing
 {\baselineskip=10.5pt
 %\addcontentsline{toc}{section}{References}
 \begin{thebibliography}{99}
 
 \vspace*{-2pt}
 
\bibitem{Moltchanov2022a}
\Au{Молчанов~Д.\,А., Бегишев~В.\,О., Самуйлов~К.\,Е., Кучерявый~Е.\,А.}
Сети 5G/6G: архитектура, технологии, методы анализа и~расчета.~--- 
М.: РУДН, 2022. 516~с.

\bibitem{Hmamouche2021}
\textit{Hmamouche~Y., Benjillali~M., Saoudi~S., Yanikomeroglu~H., Renzo~M.\,D.}
New trends in stochastic geometry for wireless networks: A~tutorial and survey~//
P.~IEEE, 2021. Vol.~109. No.\,7. P.~1200--1252.
doi: 10.1109/JPROC.2021.3061778.

\bibitem{Naumov2016}
\Au{Наумов~В.\,А., Самуйлов~К.\,Е.}
О~связи ресурсных сис\-тем массового обслуживания с~сетями Эрланга~//
Информатика и~её применения, 2016. Т.~10. Вып.~3. С.~9--14.
doi: 10.14357/19922264160302.

\bibitem{Gorbunova2018} %4
\Au{Горбунова~А.\,В., Наумов~В.\,А., Гайдамака~Ю.\,В., Самуйлов~К.\,Е.}
Ресурсные системы массового обслуживания как модели беспроводных сис\-тем связи~//
Информатика и~её применения, 2018. Т.~12. Вып.~3. С.~48--55.
doi: 10.14357/19922264180307.

\bibitem{Markova2019} %5
\Au{Маркова~Е.\,В., Гольская~А.\,А., Дзантиев~И.\,Л., Гудкова~И.\,А., 
Шоргин~С.\,Я.}
Сравнительный анализ показателей эффективности модели беспроводной сети 
межмашинного взаимодействия, работающей в~рамках двух политик разделения 
радиоресурсов~//
Информатика и~её применения, 2019. Т.~13. Вып.~1. С.~108--116.
doi: 10.14357/19922264190115.

\bibitem{Moltchanov2022b} %6
\Au{Moltchanov~D.\,A., Sopin~E.\,S., Begishev~V.\,O., Sa\-muylov~A.\,K., Koucheryavy~Y.\,A., 
Sa\-mouylov~K.\,E.}
A~tutorial on mathematical modeling of 5G/6G millimeter wave and terahertz 
cellular systems~//
IEEE Commun. Surv.  Tut., 2022. Vol.~24. No.\,2. P.~1072--1116.
doi: 10.1109/ COMST.2022.3156207.



\bibitem{Kochetkova2021}
\Au{Кочеткова~И.\,А., Кущазли~А.\,И., Харин~П.\,А., Шоргин~С.\,Я.}
Модель схемы приоритетного доступа трафика URLLC и~eMBB в~сети пятого поколения в~виде ресурсной сис\-те\-мы массового обслуживания~//
Информатика и~её применения, 2021. Т.~15. Вып.~4. С.~87--92.
doi: 10.14357/19922264210412.

\bibitem{3GPP38901}
3GPP TR 38.901. Study on channel model for frequencies from~0.5 to~100~GHz, 
2024. Release 17.1.0.

\bibitem{Bolla2023}
\Au{Bolla~R., Bruschi~R., Lombardo~C., Mohammadpour~A., Trivisonno~R., 
Poe~W.\,Y.}
A~5G multi-gNodeB simulator for ultra-reliable 0.5--100~GHz communication in 
indoor Industry~4.0 environments~//
Comput. Netw., 2023. Vol.~237. Art. No.\,110103.
doi: 10.1016/j.comnet.2023.11010.

\bibitem{Ventzel2018}
\textit{Вентцель~Е.\,С., Овчаров~Л.\,А.}
Теория вероятностей и~ее инженерные приложения.~--- М.: Юстиция, 
2018. 480~c.

\end{thebibliography}

 }
 }

\end{multicols}

\vspace*{-10pt}

\hfill{\small\textit{Поступила в~редакцию 15.03.24}}

%\vspace*{10pt}

%\pagebreak

\newpage

\vspace*{-28pt}

%\hrule

%\vspace*{2pt}

%\hrule



\def\tit{STOCHASTIC PATH LOSS MODEL IN~5G~NETWORK DEPLOYMENT SCENARIOS: A~STUDY BASED ON~3GPP~TR~38.901}


\def\titkol{Stochastic Path Loss Model in 5G Network Deployment Scenarios: A~Study Based on 3GPP TR 38.901}


\def\aut{E.\,D.~Makeeva$^{1,2}$, I.\,A.~Kochetkova$^{1,3}$, and~S.\,Ya.~Shorgin$^{3}$}

\def\autkol{E.\,D.~Makeeva, I.\,A.~Kochetkova, and~S.\,Ya.~Shorgin}

\titel{\tit}{\aut}{\autkol}{\titkol}

\vspace*{-8pt}


\noindent
$^1$RUDN University, 6 Miklukho-Maklaya Str., Moscow 117198, Russian Federation

\noindent
$^2$V.\,A.~Trapeznikov Institute of Control Science of the Russian Academy of 
Sciences, 65~Profsoyuznaya Str., Moscow\linebreak
$\hphantom{^1}$117997, Russian Federation

\noindent
$^3$Federal Research Center ``Computer Science and Control'' of the Russian 
Academy of Sciences, 44-2~Vavilov\linebreak
$\hphantom{^1}$Str., Moscow 119333, Russian Federation

\def\leftfootline{\small{\textbf{\thepage}
\hfill INFORMATIKA I EE PRIMENENIYA~--- INFORMATICS AND
APPLICATIONS\ \ \ 2024\ \ \ volume~18\ \ \ issue\ 2}
}%
 \def\rightfootline{\small{INFORMATIKA I EE PRIMENENIYA~---
INFORMATICS AND APPLICATIONS\ \ \ 2024\ \ \ volume~18\ \ \ issue\ 2
\hfill \textbf{\thepage}}}

\vspace*{3pt}




\Abste{The fifth-generation (5G) and beyond networks will utilize radio frequencies in the terahertz 
spectrum, enabling extremely high data transmission rates. However, signal loss may occur when signals 
pass through obstacles, making it crucial to simulate signal propagation using stochastic geometry 
and apply up-to-date models for signal attenuation. The 3GPP TR~38.901 specification includes models that describe signal
 attenuation in various 5G~network scenarios using empirical formulas. Nevertheless, simpler formulas are typically employed 
 to create models based on stochastic geometry. The authors present the cumulative distribution function 
 for path loss at random user locations according to the scenarios described in 3GPP TR~38.901. In numerical examples, it is shown that the difference 
in values with the simplified formula can be significant and lead to underestimation of the network's capacity}

\KWE{wireless network; 5G; 3GPP TR 38.901; path loss; line-of-sight (LOS); non-line-of-sight (NLOS); stochastic geometry}



\DOI{10.14357/19922264240204}{EKLCAP}

\vspace*{-12pt}

\Ack

\vspace*{-3pt}

\noindent
The publication has been supported by the RUDN University Scientific Projects 
Grant System, project No.\,025319-2-000.


  \begin{multicols}{2}

\renewcommand{\bibname}{\protect\rmfamily References}
%\renewcommand{\bibname}{\large\protect\rm References}

{\small\frenchspacing
 {%\baselineskip=10.8pt
 \addcontentsline{toc}{section}{References}
 \begin{thebibliography}{99} 
\bibitem{Moltchanov2022a-1}
\Aue{Moltchanov,~D.\,A., V.\,O.~Begishev, K.\,E.~Samouylov, and Y.\,A.~Koucheryavy.}
2022.
\textit{Seti 5G/6G: arkhitektura, tekhnologii, metody analiza i~rascheta}
[The 5G/6G networks: Architecture, technologies, analysis methods, and calculations].
Moscow: RUDN University. 516~p.

\bibitem{Hmamouche2021-a}
\Aue{Hmamouche,~Y., M.~Benjillali, S.~Saoudi, H.~Yanikomeroglu, and 
M.\,D.~Renzo.}
2021.
New trends in stochastic geometry for wireless networks: A~tutorial and survey.
\textit{P.~IEEE}. 109(7):1200--1252.
doi: 10.1109/ JPROC.2021.3061778.

\bibitem{Naumov2016-1}
\Aue{Naumov,~V.\,A., and K.\,E.~Samouylov.}
2016. O~svyazi resursnykh sistem massovogo obsluzhivaniya s~setyami Erlanga
[On relationship between queuing systems with resources and Erlang networks].
\textit{Informatika i~ee Primeneniya~--- Inform Appl.} 10(3):9--14.
doi: 10.14357/ 19922264160302.

\bibitem{Gorbunova2018-1} %4
\Aue{Gorbunova,~A.\,V., V.\,A.~Naumov, Yu.\,V.~Gaidamaka, and K.\,E.~Samouylov.}
2018. Resursnye sistemy massovogo obsluzhivaniya kak modeli besprovodnykh sistem svyazi
[Resource queuing systems as models of wireless communication systems].
\textit{Informatika i~ee Primeneniya~--- Inform. Appl.} 12(3):48--55.
doi: 10.14357/19922264180307.



\bibitem{Markova2019-1} %5
\Aue{Markova,~E.\,V., A.\,A.~Golskaia, I.\,L.~Dzantiev, I.\,A.~Gudkova, and 
S.\,Ya.~Shorgin.}
2019. Sravnitel'nyy analiz pokazateley effektivnosti modeli besprovodnoy seti mezhmashinnogo 
vzaimodeystviya, rabotayushchey v~ramkakh dvukh politik razdeleniya radioresursov
[Comparative analysis of performance measures for a~wireless machine-to-machine 
network model operating within two radio resource management policies].
\textit{Informatika i~ee Primeneniya~--- Inform. Appl}. 13(1):108--116.
doi: 10.14357/ 19922264190115.


\bibitem{Moltchanov2022b-1} %6
\Aue{Moltchanov,~D.\,A., E.\,S.~Sopin, V.\,O.~Begishev, A.\,K.~Sa\-muy\-lov, 
Y.\,A.~Koucheryavy, and K.\,E.~Samouylov.}
2022.
A tutorial on mathematical modeling of 5G/6G millimeter wave and terahertz 
cellular systems.
\textit{IEEE Commun. Surv. Tut.} 24(2):1072--1116.
doi: 10.1109/COMST. 2022.3156207.

\bibitem{Kochetkova2021-1} %7
\Aue{Kochetkova,~I.\,A., A.\,I.~Kushchazli, P.\,A.~Kharin, and S.\,Ya.~Shorgin.}
2021. Model' skhemy prioritetnogo do\-stu\-pa trafika URLLC i~eMBB v~seti pyatogo pokoleniya v~vide resursnoy sistemy massovogo obsluzhivaniya
[Model for analyzing priority admission control of URLLC and eMBB communications 
in 5G networks as a~resource queuing system].
\textit{Informatika i~ee Primeneniya~--- Inform. Appl}. 15(4):87--92.
doi: 10.14357/19922264210412.

\bibitem{3GPP38901-1}
3GPP TR 38.901. 2023. Study on channel model for frequencies from~0.5 to~100~GHz, 
Release 17.1.0.

\bibitem{Bolla2023-1}
\Aue{Bolla,~R., R.~Bruschi, C.~Lombardo, A.~Mohammadpour, R.~Trivisonno, and 
W.\,Y.~Poe.}
2023.
A 5G multi-gNodeB\linebreak\vspace*{-12pt}

\pagebreak

\noindent
 simulator for ultra-reliable 0.5--100~GHz communication in 
indoor Industry 4.0 environments.
\textit{Comput. Netw.} 37:110103. doi: 10.1016/j.comnet.2023.11010.

\bibitem{Ventzel2018-1}
\Aue{Ventzel,~E.\,S. and L.\,A.~Ovcharov.}
2018.
\textit{Teoriya veroyatnostey i~ee inzhenernye prilozheniya}
[Probability theory and its engineering applications].
Moscow: Justice. 480~p.

\end{thebibliography}

 }
 }

\end{multicols}

\vspace*{-6pt}

\hfill{\small\textit{Received March 15, 2024}} 

\vspace*{-12pt}


\Contr

\vspace*{-3pt}

\noindent
\textbf{Makeeva Elena D.} (b.\ 1996)~--- PhD student, Department of Probability 
Theory and Cyber Security, RUDN University, 6~Miklukho-Maklaya Str., Moscow 
117198, Russian Federation; junior scientist, V.\,A.~Trapeznikov Institute of 
Control Science of the Russian Academy of Sciences, 65~Profsoyuznaya Str., 
Moscow 117997, Russian Federation; \mbox{elena-makeeva-96@mail.ru}

\vspace*{3pt}

\noindent
\textbf{Kochetkova Irina A.} (b.\ 1985)~--- Candidate of Science (PhD) in physics 
and mathematics, associate professor, Department of Probability Theory and Cyber 
Security, RUDN University, 6~Miklukho-Maklaya Str., Moscow 117198, Russian 
Federation; senior scientist, Federal Research Center ``Computer Science and 
Control'' of the Russian Academy of Sciences, 44-2~Vavilov Str., Moscow 119333, 
Russian Federation; \mbox{kochetkova-ia@rudn.ru}

\vspace*{3pt}

\noindent
\textbf{Shorgin Sergey Ya.} (b.\ 1952)~--- Doctor of Science in physics and 
mathematics, professor, principal scientist, Federal Research Center ``Computer Science and Control'' of the Russian Academy of 
Sciences, 44-2~Vavilov Str., Moscow 119133, Russian Federation; 
\mbox{sshorgin@ipiran.ru}





\label{end\stat}

\renewcommand{\bibname}{\protect\rm Литература} 