\def\stat{zol-zats}

\def\tit{МОДЕЛЬ И ТЕХНОЛОГИЯ ИЗВЛЕЧЕНИЯ НОВЫХ ТЕРМИНОВ ИЗ~МЕДИЦИНСКИХ 
ТЕКСТОВ$^*$}

\def\titkol{Модель и~технология извлечения новых терминов из~медицинских 
текстов}

\def\aut{И.\,М.~Зацман$^1$, О.\,В.~Золотарев$^2$, А.\,Х.~Хакимова$^3$, 
Гу Дунсяо$^4$}

\def\autkol{И.\,М.~Зацман, О.\,В.~Золотарев, А.\,Х.~Хакимова, 
Гу Дунсяо}

\titel{\tit}{\aut}{\autkol}{\titkol}

\index{Зацман И.\,М.}
\index{Золотарев О.\,В.}
\index{Хакимова А.\,Х.} 
\index{Дунсяо Гу}
\index{Zatsman I.\,M.}
\index{Zolotarev O.\,V.}
\index{Khakimova A.\,K.}
\index{Dongxiao Gu}


{\renewcommand{\thefootnote}{\fnsymbol{footnote}} \footnotetext[1]
{Исследование выполнено при поддержке РФФИ и~Государственного фонда естественных наук (ГФЕН) 
Китая (проект 21-57-53018).}}


\renewcommand{\thefootnote}{\arabic{footnote}}
\footnotetext[1]{Федеральный исследовательский центр <<Информатика и~управ\-ле\-ние>> Российской академии наук, 
\mbox{izatsman@yandex.ru}}
\footnotetext[2]{Институт информационных систем и~ин\-же\-нер\-но-компью\-тер\-ных технологий Российского нового 
университета, \mbox{ol-zolot@yandex.ru}}
\footnotetext[3]{Институт информационных систем и~ин\-же\-нер\-но-компью\-тер\-ных технологий Российского нового 
университета, \mbox{aida\_khatif@mail.ru}}
\footnotetext[4]{Технологический университет г.~Хэфэй (КНР), dongxiaogu@yeah.net}

\vspace*{-12pt} 
  

  \Abst{Рассматривается модель информационной технологии (ИТ) извлечения новых 
терминов из медицинских текстов, которая относится к~ранее определенному классу 
средовых моделей информатики. В~проведенном эксперименте для 
определения новизны терминов используется словарь MeSH (Medical Subject Headings), 
который создан и~обновляется Национальной медицинской библиотекой США. Появление 
новых терминов обусловлено представлением в~медицинских статьях и~других научных 
текстах нового знания об исследуемых болезнях, методах их лечения и~применяемых 
медикаментах, которое еще не нашло отражения в~медицинских словарях и~тезаурусах. 
В~информационных системах медицинских учреждений и~институтов предлагаемая 
технология позволяет регулярно актуализировать новыми терминами профили 
исследуемых болезней, соответствующих их предметной области. Цель статьи состоит 
в~описании средовой модели ИТ актуализации 
терминологических профилей болезней.}
    

\KW{средовые модели информатики; медицинские тексты; 
терминологический профиль; извлечение новых терминов из текстов}

 \DOI{10.14357/19922264220412} 
  
%\vspace*{-3pt}


\vskip 10pt plus 9pt minus 6pt

\thispagestyle{headings}

\begin{multicols}{2}

\label{st\stat}

\section{Введение}

  Одна из задач российско-ки\-тай\-ско\-го проекта, финансируемого по 
гранту №\,21-57-53018, состоит в~моделировании и~создании 
экспериментальной ИТ регулярного 
и~целенаправленного извлечения\linebreak новых терминов из медицинских текстов 
и~описания экспертами их значений в~терминологических профилях 
болезней. В~целом проект направлен на актуализацию терминологических 
\mbox{профилей} исследуемых болезней\footnote[5]{Терминологический портрет болезни 
включает отобранные экспертами ключевые термины ее описания, дефиниции их значений, 
отношения между значениями терминов в~портрете, контексты использования терминов 
и~аннотации контекстов тех научных статей и~других медицинских текстов, которые 
использовались экспертами при формировании дефиниций значений терминов, а~также степень 
конвенциональности (социализации) значений в~рамках корпоративной базы знаний.} как 
компонентов корпоративной базы знаний медицинского учреждения 
(института) в~его предметной об\-ласти. Речь не идет о терминологическом 
дублировании известных \mbox{медицинских} словарей, тезаурусов и~других 
лингвистических ресурсов~[1--7]. В~рамках проекта решается задача 
извлечения тех новых терминов, которые появляются в~медицинских статьях и~других научных текстах, но при этом еще не вошли в~медицинские 
лингвистические ресурсы (МЛР). Эти термины используются для актуализации 
терминологических профилей исследуемых болезней в~корпоративной базе 
знаний.
  
  Для решения указанной выше задачи проектируется экспериментальная 
ИТ целенаправленного извлечения новых терминов с~использованием 
средов$\acute{\mbox{о}}$й модели актуализации \mbox{терминологических} 
профилей болезней новыми терминами. Цель \mbox{статьи} состоит в~описании этой 
модели, относящейся к~классу средов$\acute{\mbox{ы}}$х моделей 
информатики.

\vspace*{-6pt}
  
\section{Средовая модель актуализации}

\vspace*{-4pt}

  Частные случаи средов$\acute{\mbox{о}}$й модели актуализации, 
рассмотренные в~работах~[8--10], предназначены для описания процесса 
извлечения новых терминов из текстов и~итерационного обновления 
терминологического портрета болезни. В~течение первого года выполнения 
проекта опыт применения этих частных случаев при проектировании 
экспериментальных ИТ и~решении задач проекта показал возможность их 
обобщения. Средов$\acute{\mbox{а}}$я модель актуализации в~предлагаемой 
обобщенной форме даст возможность использовать одну и~ту же модель при 
решении более широкого спектра задач. Ключевой\linebreak\vspace*{-12pt}

\pagebreak

\end{multicols}

\begin{figure*} %fig1
 \vspace*{1pt}
\begin{center}
     \mbox{%
\epsfxsize=163mm
\epsfbox{zac-1.eps}
}

\vspace*{6pt}

{\small Средовая модель актуализации терминологического портрета 
болезни в~корпоративной базе знаний}
\end{center}
\vspace*{-12pt}
\end{figure*}

\begin{multicols}{2}

\noindent
 аспект обобщения состоит 
в~разделении новых терминов на четыре категории с~точки зрения статуса их 
конвенциональности. Далее будут рассмотрены первые итерации обновления и~статусы конвенциональности терминов.

  
  До начала первой итерации обновления терминологического портрета 
исследуемой болезни экспертами формируется его начальная версия на 
основе терминов и~их синонимов, используемых при описании этой болезни, 
а~также их дефиниций в~существующих МЛР~[1--7]. Этим терминам, отобранным для начальной версии 
портрета, присваивается максимальный статус конвенциональности~--- 
\textit{словарный}.
  
  На первой итерации обновления начальная версия терминологического 
портрета используется как критерий новизны при поиске (см.\ стрелку 
с~бук\-вой~$\alpha$ на рисунке). Из электронной коллекции извлекаются статьи 
с~описанием исследуемой болезни (стрелка с~буквой~$\beta$) в~качестве 
источников потенциально новых терминов (ИПНТ)\footnote{Если термин статьи 
с~описанием болезни отсутствует в~начальной версии терминологического портрета, то это 
говорит только о возможной (потенциальной) его новизне, так как при дальнейшем 
семантическом анализе его контекста он может оказаться синонимом словарного термина, 
который отсутствует в~существующих МЛР. В~таком случае речь идет о новом синониме, а~не 
о~новом термине.}. После извлечения найденных статей в~них выделяются 
контексты ИПНТ, которые визуализируются перед операцией их 
семантического анализа (стрелка с~буквой~A).
  
  В процессе семантического анализа контекстов ИПНТ эксперт использует 
текущую (на первой итерации~--- начальную) версию терминологического 
портрета (стрелка с~буквой~$\gamma$). По завершении анализа эксперт 
разделяет найденные новые термины и~новые синонимы словарных терминов 
(ромб <<Новый термин?>>), если он их обнаружил. Значения новых 
терминов аннотируются и~оцифровываются, а дефиниции значений 
и~аннотации контекстов новых терминов (КНТ) с~этими значениями 
загружаются в~терминологический профиль болезни. Новые синонимы 
оцифровываются и~добавляются к~соответствующим словарным терминам. 
После обработки текстов эксперты формируют первую обновленную версию 
терминологического портрета. В~процессе формирования каждому из этих 
терминов присваивается один из четырех используемых в~проекте статусов 
конвенциональности~[10--12]:
  \begin{enumerate}[(1)]
  \item \textit{личностный}, если дефиниция значения (=\;кон\-цеп\-та) термина, 
сформированная одним экспертом, не была согласована им с~другими 
экспертами корпоративной базы знаний;
  \item \textit{коллективный}, если дефиниция значения термина, 
сформированная одним экспертом, была согласована им с~другими 
экспертами корпоративной базы знаний, но не со всеми;
  \item \textit{организационный}, если дефиниция значения термина, 
сформированная одним экспертом, была согласована им со всеми экспертами 
корпоративной базы знаний;
  \item \textit{словарный}, если во время первой итерации в~существующие 
МЛР был добавлен новый термин, включенный в~портрет (если 
в~существующие МЛР были добавлены термины, отсутствующие 
в~обработанных текстах, то они также включаются в~обновленную версию 
портрета в~статусе словарных).
  \end{enumerate}



  На завершающем этапе итерации предусмотрена возможность 
социализации значений новых терминов первых двух статусов 
конвенциональности (личностных и~коллективных) с~использованием их 
контекстов, т.\,е.\ КНТ. На рисунке показан случай социализации только 
\textit{личностных значений терминов}. На этап их социализации передаются 
КНТ (см.\ ромб <<КНТ?>>), а~не все контексты ИПНТ.
  
  На второй и~последующих итерациях используются уже обновленные 
портреты для извлечения новых терминов из новых поступлений текс\-тов\linebreak 
в~коллекцию. В~проекте используются \mbox{новые} поступления текстов в~базу 
данных PubMed, содержащую более 34~млн записей и~обеспечивающую 
развитые возможности поиска и~формирования \mbox{статистических} данных, 
включая получение распределения статей с~заданным термином по годам их 
публикации~\cite{13-zz}.
  
  В процессе построения средов$\acute{\mbox{о}}$й модели актуализации 
использовались теоретические основания их построения, сформулированные 
в~работе~\cite{14-zz}. Далее будут рассмотрены три основных положения 
теоретических оснований построения всего класса 
средов$\acute{\mbox{ы}}$х моделей (выделены ниже курсивом) в~контексте 
их применения в~проекте (см.\ рисунок)~\cite{10-zz, 15-zz}.
  
  \smallskip
  
  1.\ \textit{Выделение в~предметной области информатики тех сред разной 
природы, которые необходимы для моделирования проектируемой ИТ, 
определяется целью ее создания.}
  
  \smallskip
  
  В проекте цель проектирования ИТ сформулирована так: регулярное 
и~целенаправленное извлечение новых терминов из медицинских текстов 
и~описание экспертами их значений в~терминологических профилях 
болезней корпоративной базы знаний. 
  
  Из формулировки этой цели следует, что из пяти\linebreak сред предметной  
об\-ласти информатики (ментальная, информационная, цифровая, нейро-  
и~ДНК-сре\-да), рассмотренных в~работах~\cite{16-zz, 17-zz, 18-zz}, для 
проектируемой ИТ достаточно рассмотреть три \mbox{следующих} среды (см.\ 
рисунок), поскольку объекты нейро- и~ДНК-сред не используются в~этой~ИТ:
  \begin{enumerate}[(1)]
  \item \textit{ментальная среда}~--- это совокупность значений терминов 
как концептов знания экспертов (верхняя часть рисунка);
  \item \textit{информационная среда}~--- это совокупность описаний 
значений терминов, подготовленных экспертами (средняя часть);
  \item \textit{цифровая среда}~--- это совокупность компьютерных кодов 
терминологических профилей болезней корпоративной базы знаний (нижняя 
часть).
  \end{enumerate}
  
  2.\ \textit{Распределение моноприродных этапов ИТ и~полиприродных 
этапов ИТ с~входами и~выходами одной природы по средам согласно их 
природе.}
  
  \smallskip
  
  Рисунок иллюстрирует это достаточно общее теоретическое основание 
построения всего класса средов$\acute{\mbox{ы}}$х моделей на примере 
следующих \textit{шести} моноприродных этапов ИТ, обозначенных 
греческими буквами $\alpha$--$\zeta$:
  \begin{enumerate}[(1)]
  \item чтение в~цифровой среде терминологического портрета для его 
использования в~качестве критерия потенциальной новизны термина при 
поиске (этап~$\alpha$ на рисунке);\\[-15pt]
  \item чтение из электронной коллекции статей с~описанием исследуемой 
болезни как ИПНТ (этап~$\beta$);\\[-15pt]
  \item  чтение терминологического портрета для разделения новых 
терминов и~новых синонимов словарных терминов (этап~$\gamma$);\\[-15pt]
  \item  объединение КНТ с~аннотацией значения нового термина, 
включающей дефиницию этого значения (этап~$\delta$);\\[-15pt]
  \item добавление синонима к~словарному термину (этап~$\varepsilon$, 
стрелка чтения словарного термина на рисунке не показана, чтобы упростить 
его);\\[-15pt]
  \item объединение в~форме аннотации дефиниции концепта нового 
термина с~его контекстом, т.\,е.\ с~КНТ (этап~$\zeta$).
  \end{enumerate}
  
  Кроме шести моноприродных этапов на рисунке показаны \textit{три 
полиприродных этапа} этой ИТ с~входами и~выходами одной природы, но 
с~использованием внутри этапов сущностей другой среды. Распределение 
этих трех этапов по средам (двух этапов в~информационной среде и~одного 
этапа в~ментальной среде) следует из этого же теоретического основания, так 
как: 
  \begin{itemize}
  \item на этапе <<Семантический анализ>> эксперт изуча\-ет контекст ИПНТ 
(информационная среда), принимает решение о новизне термина и~его 
значения на основе своего понимания смысла контекста термина, т.\,е.\ 
своего личностного концепта (ментальная среда), на основе анализа которого 
проставляется метка <<новый термин>> или <<синоним словарного 
термина>> (информационная среда); таким образом, вход и~выход этого 
этапа ИТ принадлежат одной среде~--- \textit{информационной};
  \item на этапе <<Аннотирование нового термина>> эксперт анализирует 
его контекст (информационная среда) и~на основе своего понимания смыс\-ла 
контекста термина, т.\,е.\ своего личностного концепта (ментальная среда), 
формирует аннотацию (информационная среда); таким образом, вход 
и~выход этого этапа ИТ также принадлежат одной среде~--- 
\textit{информационной};
  \item на этапе <<Социализация>> эксперты на основе собственного 
понимания контекстного значения нового термина (ментальная среда) 
предлагают варианты дефиниций (информационная среда) и~пытаются 
сформировать согласованный между ними концепт дефиниции значения 
с~коллективным или организационным статусом (ментальная среда); таким 
образом, вход и~выход этого этапа ИТ принадлежат одной среде~--- 
\textit{ментальной} (возможный случай отсутствия согласования концепта 
экспертами на рисунке не показан).
  \end{itemize}
  
  3.\ \textit{Распределение этапов ИТ по границам между средами, если их 
вход и~выход принадлежат средам разной природы.}
  
  На рисунке показаны \textit{восемь этапов}, входы и~выходы которых 
принадлежат средам разной природы, включая шесть этапов ИТ на границе 
между информационной и~цифровой средами и~два этапа на границе между 
информационной и~ментальной средами:
  \begin{enumerate}[(1)]
  \item этап <<Визуализация ИПНТ>>, компьютерные коды которого 
(циф\-ро\-вая среда) преобразуются в~текст ИПНТ (информационная среда);
  \item этап <<Визуализация [терминологического] портрета>>, 
компьютерные коды которого (циф\-ро\-вая среда) преобразуются в~текст 
портрета (информационная среда);
  \item этап <<Оцифровка личностной аннотации>>, текст которой 
(информационная среда) преобразуется в~компьютерные коды 
терминологического портрета корпоративной базы знаний (циф\-ро\-вая среда);
  \item этап <<Визуализация [личностной] аннотации>>, компьютерные 
коды которой (циф\-ро\-вая среда) преобразуются в~текст (информационная 
среда);
  \item этап <<Оцифровка описания [словарного] термина>>, текст которого 
(информационная среда) преобразуется в~компьютерные коды 
терминологического портрета корпоративной\linebreak базы знаний (циф\-ро\-вая среда);
  \item этап <<Оцифровка согласованной аннотации>>, текст которой 
(информационная среда) преобразуется в~компьютерные коды 
терминологического портрета корпоративной базы знаний (циф\-ро\-вая среда);
  \item этап <<Интернализация>>, на котором у эксперта формируется 
личностное контекстное значение нового термина (ментальная среда) на 
основе аннотации с~дефиницией значения термина и~его контекста 
(информационная среда);
  \item этап <<Экстернализация>>, на котором эксперты формируют 
дефиницию значения термина (информационная среда) на основе 
согласованного ими концепта с~коллективным или организационным 
статусом (ментальная среда).
  \end{enumerate}
  
  Таким образом, согласно средов$\acute{\mbox{о}}$й модели актуализации 
терминологических портретов, 17~перечисленных этапов проектируемой ИТ 
можно распределить по трем категориям. К~первой из них относятся шесть 
моноприродных этапов ИТ, обозначенных греческими буквами  
$\alpha$--$\zeta$; ко второй~--- три этапа с~входами и~выходами одной 
природы, но с~использованием внут\-ри этапов сущностей среды другой 
природы; к~третьей~--- восемь этапов, входы и~выходы которых принадлежат 
средам разной природы. 
  
  На этапах первой категории отсутствуют интерфейсы между сущностями 
разной природы. При этом сами сущности могут быть только цифровыми 
или информационными. Каждый этап второй категории включает пару 
интерфейсов между сущностями разной природы. При этом сущности 
ментальной природы могут быть как на вхо\-де/вы\-хо\-де этапа (см.\ этап 
социализации), так и~внутри этапа (см. этапы семантического анализа 
и~аннотирования). Шесть из восьми этапов третьей категории включают 
традиционный интерфейс между сущностями информационной и~цифровой 
природы на границе между соответствующими средами. Этот интерфейс 
в~проектируемых ИТ может быть реализован, например, с~по\-мощью  
таб\-лиц Unicode. В~процессе проектирования ИТ наибольшую сложность 
представляет разработка этапов второй категории, а~также двух этапов на 
границе между ментальной и~информационной средами, так как они 
включают сущности ментальной природы. Вопрос их компьютеризации 
заслуживает отдельного рассмотрения и~выходит за рамки данной статьи.
  
  Рассмотренная средов$\acute{\mbox{а}}$я модель актуализации позволила 
при выполнении проекта до начала проектирования ИТ определить природу 
сущностей каж\-до\-го этапа и~локализовать этапы с~ментальными сущностями 
на их вхо\-де/вы\-хо\-де или внутри этапов, проб\-ле\-ма компьютеризации 
которых относится в~информатике к~категории проблем \textit{когнитивной 
сложности}~\cite{14-zz, 19-zz}.

\section{Эксперимент}

  В первый год выполнения проекта оставался открытым вопрос о степени 
полноты наборов терминов МЛР~[1--7], т.\,е.\ насколько п$\acute{\mbox{о}}$лно наборы 
терминов, используемых в~научных статьях и~других медицинских текстах 
при описании исследуемых болезней, представлены в~МЛР. Поэтому на 
втором году выполнения проекта был проведен эксперимент, краткое 
описание которого приведено ниже, для проверки гипотезы неполноты 
наборов терминов МЛР и~необходимости регулярной актуализации 
терминологических портретов болезней в~корпоративной базе знаний.
  
  В рамках проведенного эксперимента анализировались термины 
медицинских статей, содержащих описание кластеров 
кальцификации\footnote{Наличие кластеров кальцификации может быть использовано для 
диагностики заболеваний молочных желез.}. В~базе электронных биомедицинских 
публикаций PubMed был осуществлен поиск этих статей по запросу: 
(calcification[Abstract] AND (``neoplasms''[MeSH Terms] OR ``neoplasms''[All 
Fields] OR ``cancer''[All Fields])) AND (``breast''[MeSH Terms] OR ``breast''[All 
Fields]) AND (``2003/01/01''[PubDate]~: ``2021/12/31''[PubDate]). За период 
2003--2021~гг.\ найдены 844~статьи, которые распределились по трем 
временн$\acute{\mbox{ы}}$м периодам следующим образом: 2003--2009~гг.~--- 57~статей; 
2010--2015~гг.~--- 225~статей; 2016--2021~гг.~--- 562~статьи.
  
  При поиске использовались заголовки и~текс\-ты аннотаций статей. После 
автоматического поиска вхождений термина calcification (всего были 
найде\-ны~2658 его вхождений, но не каждое вхождение порождает 
терминоподобное словосочетание) экспертами был выполнен семантический 
анализ их контекстов, чтобы найти терминоподобные словосочетания, 
включающие термин calcification.
  
  В результате семантического анализа 2658~вхож\-де\-ний экспертно были 
выделены~93~терминоподобных словосочетания за период 2003--2021~гг. Из 
них~23 были обнаружены в~статьях 2003--2009~гг.\ (например, Amorphous 
Calcifications, Arterial Calcification, Aortic Calcification), 54~--- в~статьях  
2010--2015~гг., включая~31~терминоподобное словосочетание, которого не 
было в~стать\-ях 2003--2009 гг.\ (например, Arteriosclerotic Calcification),  
и~93~--- в~стать\-ях 2016--2021~гг., включая ранее не 
встре\-чав\-ши\-еся~39~терминов (например, Abnormal Calcification, Arterial 
Media(L) Calcification, Calcification Crystallite).
  
  Для проверки на словарный статус~93~терминоподобных словосочетаний, 
найденных экспертами в~статьях, использовался тезаурус MeSH~\cite{1-zz}. 
В~результате проверки лишь~3 из~93~терминоподобных словосочетаний 
совпали с~терминами тезауруса MeSH, т.\,е.\ имеют словарный статус.

\section{Заключение}

  Проведенный эксперимент подтвердил гипотезу о~существенной 
неполноте наборов терминов МЛР. При этом отсутствующие 
терминологические словосочетания уже используются в~научных стать\-ях 
и~других медицинских текстах при описании исследуемой болезни. 
Проведенный эксперимент показал необходимость регулярной актуализации\linebreak 
терминологических портретов болезней в~корпоративной базе знаний, что, 
в~свою очередь, под\-тверж\-да\-ет важ\-ность решения задачи моделирования 
и~создания ИТ регулярного и~\mbox{целенаправленного}\linebreak извлечения новых 
терминов из медицинских текс\-тов и~описания экспертами их значений 
в~терминологических профилях болезней.
  
{\small\frenchspacing
 {%\baselineskip=10.8pt
 %\addcontentsline{toc}{section}{References}
 \begin{thebibliography}{99}
\bibitem{1-zz}
Medical subject headings (MeSH) Thesaurus. {\sf https:// www.nlm.nih.gov/mesh/meshhome.html.}
\bibitem{2-zz}
Systematized nomenclature of medicine clinical terms (SNOMED CT). {\sf 
https://bioportal.bioontology.org/\linebreak ontologies/SNOMEDCT}.
\bibitem{3-zz}
Unified Medical Language System (UMLS). {\sf https:// www.nlm.nih.gov/research/umls/index.html.}
\bibitem{4-zz}
National Cancer Institute Thesaurus. {\sf 
https://\linebreak ncithesaurus.nci.nih.gov/ncitbrowser/pages/home.jsf?\linebreak version=22.08e.}
\bibitem{5-zz}
Medical dictionary of health terms.~--- Cambridge, MA, USA: Harvard Health Publishing, 2011. 
{\sf https:// www.health.harvard.edu/a-through-c}.
\bibitem{6-zz}
Dorland's illustrated medical dictionary.~--- 33rd ed.~--- Philadelphia, PA, USA: Elsevier 
Saunders, 2019. 2144~p.
\bibitem{7-zz}
MedTerms Medical Dictionary. {\sf  
https://www.\linebreak medicinenet.com/medterms-medical-dictionary/article.\linebreak htm}.
\bibitem{8-zz}
\Au{Зацман И.\,М.} Проб\-лем\-но-ори\-ен\-ти\-ро\-ван\-ная актуализация словарных 
статей двуязычных словарей и~медицинской терминологии: сопоставительный анализ~// 
Информатика и~её применения, 2021. Т.~15. Вып.~1. С.~94--101.
\bibitem{9-zz}
\Au{Зацман И.\,М.} Формы представления нового знания, извлеченного из текс\-тов~// 
Информатика и~её применения, 2021. Т.~15. Вып.~3. С.~83--90.
\bibitem{10-zz}
\Au{Zatsman I., Khakimova~A.} New knowledge discovery for creating terminological profiles 
of diseases~// 22nd European Conference on Knowledge Management Proceedings.~--- 
Reading, U.K.: Academic Publishing International Ltd., 2021. P.~837--846.
\bibitem{11-zz}
\Au{Nissen M.\,E.} Harnessing knowledge dynamics: Principled organisational knowing and 
learning.~--- London: IRM Press, 2006. 287~p.
\bibitem{12-zz}
\Au{Зацман И.\,М.} Компьютерная и~экономическая модели генерации нового знания: 
сопоставительный анализ~// Сис\-те\-мы и~средства информатики, 2021. Т.~31. №\,4.  
С.~84--96.
\bibitem{13-zz}
PubMed. National Library of Medicine. {\sf https://pubmed. ncbi.nlm.nih.gov}.
\bibitem{14-zz}
\Au{Зацман И.\,М.} Средовые модели информационных технологий: теоретические 
основания построения~// Информатика и~её применения, 2022. Т.~16. Вып.~3. С.~59--67.
\bibitem{15-zz}
\Au{Зацман И.\,М. Золотарев~О.\,В., Хакимова~А.\,Х.} Средовые модели извлечения из 
текста новых терминов и~индикаторов настроений~// Информатика и~её применения, 2022. 
Т.~16. Вып.~2. С.~60--67.
\bibitem{16-zz}
\Au{Зацман И.\,М.} Интерфейсы третьего порядка в~информатике~// Информатика и~её 
применения, 2019. Т.~13. Вып.~3. С.~82--89.
\bibitem{17-zz}
\Au{Зацман И.\,М.} Кодирование концептов в~цифровой среде~// Информатика и~её 
применения, 2019. Т.~13. Вып.~4. С.~97--106.
\bibitem{18-zz}
\Au{Зацман И.\,М.} Теоретические основания компьютерного образования: среды 
предметной области информатики как основание классификации ее объектов~//  
Сис\-те\-мы и~средства информатики, 2022. Т.~32. №\,4. С.~77--89.
\bibitem{19-zz}
\Au{Harel D.} Algorithmics~--- the spirit of computing.~--- Reading, MA, USA:  
Addison-Wesley, 1987. 514~p.
\end{thebibliography}

 }
 }

\end{multicols}

\vspace*{-6pt}

\hfill{\small\textit{Поступила в~редакцию 14.10.22}}

\vspace*{8pt}

%\pagebreak

%\newpage

%\vspace*{-28pt}

\hrule

\vspace*{2pt}

\hrule

%\vspace*{-2pt}

\def\tit{MODEL AND TECHNOLOGY FOR~DISCOVERING NEW TERMS IN~MEDICAL TEXTS}


\def\titkol{Model and technology for discovering new terms in medical texts}


\def\aut{I.\,M.~Zatsman$^1$, O.\,V.~Zolotarev$^2$, A.\,K.~Khakimova$^2$, and~Gu~Dongxiao$^3$}

\def\autkol{I.\,M.~Zatsman, O.\,V.~Zolotarev, A.\,K.~Khakimova, and~Gu~Dongxiao}

\titel{\tit}{\aut}{\autkol}{\titkol}

\vspace*{-8pt}


\noindent
$^1$Federal Research Center ``Computer Science and Control'' of the Russian Academy of Sciences,  
44-2~Vavilov\linebreak
$\hphantom{^1}$Str., Moscow 119333, Russian Federation

\noindent
$^2$Russian New University, 22~Radio Str., Moscow 105005, Russian Federation

\noindent
$^3$Hefei University of Technology, 193 Tunxi Road, Hefei, Anhui 230009, P.R.\ China


\def\leftfootline{\small{\textbf{\thepage}
\hfill INFORMATIKA I EE PRIMENENIYA~--- INFORMATICS AND
APPLICATIONS\ \ \ 2022\ \ \ volume~16\ \ \ issue\ 4}
}%
 \def\rightfootline{\small{INFORMATIKA I EE PRIMENENIYA~---
INFORMATICS AND APPLICATIONS\ \ \ 2022\ \ \ volume~16\ \ \ issue\ 4
\hfill \textbf{\thepage}}}

\vspace*{3pt} 
     



\Abste{The model of information technology for discovering new terms in medical texts, which belongs 
to the previously defined class of informatics' medium models, is considered. In the conducted 
experiment, the MeSH (Medical Subject Headers) dictionary is used to determine the novelty of terms 
which was created and updated by the National Library of Medicine (USA). The emergence of new terms 
is due to the representation (in medical papers and other scientific texts) of new knowledge about the 
studied diseases, methods of their treatment, and medicines used which has not yet been reflected in 
medical dictionaries and thesauri. In information systems of medical institutions, the proposed technology 
makes it possible to regularly update the profiles of the studied diseases corresponding to their subject 
domain. The aim of the paper is to describe the medium model of information technology for updating 
terminological profiles of diseases.}

\KWE{medium models in informatics; medical texts; terminological profile; discovering new terms in 
texts}



 \DOI{10.14357/19922264220412} 

\vspace*{-12pt}


 \Ack
\noindent
The reported study was funded by RFBR and NSFC, project number 21-57-53018.


\vspace*{5pt}

  \begin{multicols}{2}

\renewcommand{\bibname}{\protect\rmfamily References}
%\renewcommand{\bibname}{\large\protect\rm References}

{\small\frenchspacing
 {%\baselineskip=10.8pt
 \addcontentsline{toc}{section}{References}
 \begin{thebibliography}{99}
\bibitem{1-zz-1}
MeSH Thesaurus. Available at: {\sf https://www.nlm.nih. gov/mesh/meshhome.html} (accessed 
November~21, 2022). 
\bibitem{2-zz-1}
Systematized nomenclature of medicine clinical terms. Available at: {\sf 
https://bioportal.bioontology.\linebreak org/ontologies/SNOMEDCT} (accessed November~21, 2022).
\bibitem{3-zz-1}
Unified medical language system. Available at: {\sf https:// www.nlm.nih.gov/research/umls/index.html} 
(accessed November~21, 2022).
\bibitem{4-zz-1}
National Cancer Institute Thesaurus. Available at: {\sf https://ncit.nci.nih. gov/ncitbrowser} (accessed November~21, 2022).
\bibitem{5-zz-1}
\textit{Medical dictionary of health terms}. Cambridge, MA: Harvard Health Publishing. Available at: {\sf  
https:// www.health.harvard.edu/a-through-c} (accessed November~21, 2022).
\bibitem{6-zz-1}
\textit{Dorland's illustrated medical dictionary}. 2019. 33rd ed. Philadelphia, PA: Elsevier Saunders. 
2144~p.
\bibitem{7-zz-1}
{MedTerms medical dictionary}. Available at: {\sf  
https:// www.medicinenet.com/medterms-medical-dictionary/ article.htm} (accessed November~21, 2022).
\bibitem{8-zz-1}
\Aue{Zatsman, I.} 2021. Problemno-orientirovannaya ak\-tu\-a\-li\-za\-tsiya slovarnykh statey dvuyazychnykh 
slovarey i~me\-di\-tsin\-skoy terminologii: sopostavitel'nyy analiz [Problem- oriented updating of dictionary 
entries of bilingual \mbox{dictionaries} and medical terminology: Comparative analysis]. \textit{Informatika i~ee 
Primeneniya~--- Inform. Appl.} 15(1):94--101.
\bibitem{9-zz-1}
\Aue{Zatsman, I.} 2021. Formy predstavleniya novogo znaniya, izvlechennogo iz tekstov [Forms 
representing new knowledge discovered in texts]. \textit{Informatika i~ee Primeneniya~--- Inform. Appl.} 
15(3):83--90.
\bibitem{10-zz-1}
\Aue{Zatsman, I., and A.~Khakimova.} 2021. New knowledge discovery for creating terminological 
profiles of diseases. \textit{22nd European Conference on Knowledge Management Proceedings}. 
Reading, U.K.: Academic Publishing International Ltd. 837--846.
\bibitem{11-zz-1}
\Aue{Nissen, M.\,E.} 2006. \textit{Harnessing knowledge dynamics: Principled organizational knowing 
\& learning}. London: IRM Press. 287~p.
\bibitem{12-zz-1}
\Aue{Zatsman, I.} 2021. Komp'yuternaya i~ekonomicheskaya modeli generatsii novogo znaniya: 
sopostavitel'nyy ana\-liz [Computer and economic models of new knowledge generation: A~comparative 
analysis]. \textit{Sistemy i~Sredstva Informatiki~--- Systems and Means of Informatics} 31(4):\mbox{84--96}.
\bibitem{13-zz-1}
National library of medicine. PubMed. Available at: {\sf https://pubmed.ncbi.nlm.nih.gov} (accessed 
November~21, 2022).
\bibitem{14-zz-1}
\Aue{Zatsman, I.} 2022. Sredovye modeli informatsionnykh tekhnologiy: teoreticheskie osnovaniya 
postroeniya [Informatics' medium models of information technology: Theoretical foundations for their 
creating]. \textit{Informatika i~ee Primeneniya~--- Inform. Appl.} 16(3):59--67.
\bibitem{15-zz-1}
\Aue{Zatsman, I., O.~Zolotarev, and A.~Khakimova.} 2022. Sredovye modeli izvlecheniya iz teksta 
novykh terminov i~indikatorov nastroeniy [Medium models for discovering novel terms and sentiments 
from texts]. \textit{Informatika i~ee Primeneniya~--- Inform. Appl.} 16(2):60--67.
\bibitem{16-zz-1}
\Aue{Zatsman, I.\,M.} 2019. Interfeysy tret'ego poryadka v informatike [Third-order interfaces in 
informatics]. \textit{Informatika i~ee Primeneniya~--- Inform. Appl.} 13(3):82--89.
\bibitem{17-zz-1}
\Aue{Zatsman, I.\,M.} 2019. Kodirovanie kontseptov v~tsifrovoy srede [Digital encoding of concepts]. 
\textit{Informatika i~ee Primeneniya~--- Inform. Appl.} 13(4):97--106.
\bibitem{18-zz-1}
\Aue{Zatsman, I.} 2022. Teoreticheskie osnovaniya komp'yu\-ter\-no\-go obrazovaniya: sredy predmetnoy 
oblasti informatiki kak osnovanie klassifikatsii ee ob''ektov [Theoretical foundations of digital 
education: Subject domain media of informatics as the base of its objects' classification]. \textit{Sistemy 
i~Sredstva Informatiki~--- Systems and Means of Informatics} 32(4):77--89.
\bibitem{19-zz-1}
\Aue{Harel, D.} 1987. \textit{Algorithmics~--- the spirit of computing}. Reading, MA: Addison-Wesley. 
514~p.

\end{thebibliography}

 }
 }

\end{multicols}

\vspace*{-6pt}

\hfill{\small\textit{Received October 14, 2022}}

\Contr

\noindent
\textbf{Zatsman Igor M.} (b.\ 1952)~--- Doctor of Science in technology, head of department, Institute 
of Informatics Problems, Federal Research Center ``Computer Science and Control'' of the Russian 
Academy of Sciences, 44-2~Vavilov Str., Moscow 119333, Russian Federation; 
\mbox{izatsman@yandex.ru}

\vspace*{3pt}

\noindent
\textbf{Zolotarev Oleg V.} (b.\ 1959)~--- Candidate of Science (PhD) in technology, head of  
department, Russian New University, 22~Radio Str., Moscow 105005, Russian Federation;  
\mbox{ol-zolot@yandex.ru}

\vspace*{3pt}

\noindent
\textbf{Khakimova Aida Kh.} (b.\ 1963)~--- Candidate of Science (PhD) in biology, leading scientist, 
Russian New University, 22~Radio Str., Moscow 105005, Russian Federation; 
\mbox{aida\_khatif@mail.ru}

\vspace*{3pt}

\noindent
\textbf{Dongxiao Gu} (b.\ 1980)~-- Candidate of Science (PhD) in informatics, professor, Hefei University of Technology, 193 
Tunxi Road, Hefei, Anhui 230009, P.R.\ China; \mbox{dongxiaogu@yeah.net}

     
\label{end\stat}

\renewcommand{\bibname}{\protect\rm Литература}  