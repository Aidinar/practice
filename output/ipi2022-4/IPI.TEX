\documentclass[10pt]{book}
\usepackage[utf8]{inputenc}

\usepackage{latexsym,amssymb,amsfonts,amsmath,amsxtra,dsfont,
indentfirst,shapepar,%fleqn,%
picinpar,shadow,floatflt,enumerate,multicol,colortbl,moreverb,cite,ipi}

\usepackage{rotating}
\usepackage{mathrsfs}
\usepackage[noend]{algorithmic}
\usepackage{ulem}
\usepackage{graphicx}
%\usepackage{algorithm2e}
\usepackage[linesnumbered,boxed,ruled]{algorithm2e}
%\usepackage{xypic}
\usepackage{oldgerm}
\usepackage{epic}
\usepackage{eepic}

\SetAlgorithmName{Algorithm}{алгоритм}{Список алгоритмов}

%из Дюковой

\newcommand{\algKeyword}[1]{{\bf #1}}
\newcommand{\Proc}[1]{\text{\tt #1}}
\def\CALL{\algKeyword{call}~}

\newenvironment{AlgProcedure}[1]
{
\small
\medskip
%    \hrule
\medskip
\algKeyword{PROCEDURE} #1
\begin{algorithmic}[1]}
{\end{algorithmic}
%    \hrule
\bigskip
}

\def\CALL{\algKeyword{call}~}

%конец для Дюковой

%\RequirePackage[ruled]{algorithm}


\input{epsf}

%\nofiles

%\includeonly{avtor}    %pdf
%\includeonly{podgot-rus-site,podgot-eng-site}  
%\includeonly{podgot-rus,podgot-eng}  
%\includeonly{ipi-ind} 
%\includeonly{index-16i}
%\includeonly{toc-rus, toc-en}
%\includeonly{toc-rus}
%\includeonly{toc-en} 
%\includeonly{popravka}



%\includeonly{agalarov}       %pdf+авт+
%\includeonly{bazilevskii}    %pdf+авт+
%\includeonly{hatskevich}     %pdf +авт
%\includeonly{suchkov}        %pdf
%\includeonly{zatsman}        %pdf+авт+
%\includeonly{stupnikov}      %pdf+авт+
%\includeonly{borisov}        %pdf+авт+
%\includeonly{shestakov}      %pdf+авт+
%\includeonly{grusho}         %pdf
%\includeonly{bosov}          %pdf+авт+
%\includeonly{zat-zol}        %pdf+авт+
%\includeonly{peshkova}       %pdf+авт+
%\includeonly{step}           %pdf+авт+
%\includeonly{dukova}         %pdf+авт повт отправила
%\includeonly{ostrikova}      %pdf+авт+




%%%%%%%%%%%%%%%%%%%\includeonly{nekrolog-new}



%\includeonly{rekl}




\usepackage{acad}
%\usepackage{courier}
\usepackage{decor}
\usepackage{newton}
\usepackage{pragmatica}
\usepackage{zapfchan}
\usepackage{petrotex}
\usepackage{bm}                     % полужирные греческие буквы
\usepackage{upgreek}                % прямые греческие буквы \upalpha
\usepackage{eufrak}
\usepackage{verbatim}

\renewcommand{\bottomfraction}{0.99}
\renewcommand{\topfraction}{0.99}
\renewcommand{\textfraction}{0.01}

\setcounter{secnumdepth}{1} %здесь - 3 + chapter = 4

\arraycolsep=1.5pt

%\usepackage[pdftex]{graphicx}

%\usepackage{oz}

%NEW COMMANDS



\renewcommand*{\hm}[1]{#1\nobreak\discretionary{}%
            {\hbox{$\mathsurround=0pt #1$}}{}} %% Дублирует знаки операций
                               %при переносе в формуле (перед знаком, который
                               %надо продублировать ставится команда \hm)
                               
                               \newcommand{\PRB}{\begin{picture}(22.5,11)
      \spline(1,8)(4,10)(7,10.5)(10,11)(13,11)(16,10.5)(19,10)(22,8)
               \put(0,0){$P_{i-1}P_{t_{t-1}}$} \end{picture}}

\newcommand{\prb}{\begin{picture}(15.5,9)
      \spline(1,6)(3,8)(5,8.5)(7,9)(9,9)(11,8.5)(13,8)(15,6)
               \put(0,0){$PP_t$} \end{picture}}
               
                 \newcommand{\PRDN}{\begin{picture}(40,11)
      \spline(4,11.5)(7,10.5)(12,10)(16,9)(20,9)(24,10)(29,10.5)(32,11.5)
               \put(0,0){$P_{i-1}P_{t_{t-1}}$} \end{picture}}

\newcommand{\prdn}{\begin{picture}(18,11)
      \spline(3,10.5)(4,10)(6,9)(8,8.5)(10,8.5)(12,9)(14,10)(15,10.5)
               \put(0,0){$PP_t$} \end{picture}}




%\newcommand{\endproof}{\hfill$\Box$}
%\renewcommand{\r}{\mathbb{R}}
%\newcommand{\I}{{\rm I\hspace{-0.7mm}I}}
%\newcommand{\Ikl}{{\tt{1}}\hspace*{-1.44mm}\mathtt{1}}
\newcommand{\Ik}{\mbox{{\small \tt {1}}\hspace{-1.3mm}{\tt 1}}}
\newcommand{\argmin}{\mathop{\mathrm{arg}\,\mathrm{min}}}
\newcommand{\argmax}{\mathop{\mathrm{arg}\,\mathrm{max}}}
%\newcommand{\capr}{\mathop{\cap\,}}
%\newcommand{\cupr}{\mathop{\cup\,}}
%\def\argmin{\mathop{arg\,min}}

\def\vrp{\varphi}
\def\prt{\partial}
\def\mm{{\sf M}}
\def\modnop#1{\mathop{#1}\limits_{n}}
\def\eam{\mathbin{{\mathop{=}\limits^{\mathrm{def}}}}}
\def\dey#1#2{#1 (#2)}
\def\deyc#1#2{#1 \cdot  #2}
\def\ra#1{\;\mathop{\to}\limits^{#1}\;}
\def\raz#1{\;\mathop{\longrightarrow}\limits^{\!\!\!#1}\;}
\def\ral#1{\;\mathop{\longrightarrow}\limits^{#1}\;}

\newcommand{\Nor}{\mathcal{N}}
\newcommand{\T}{\mathbb{T}}
\newcommand{\Z}{\mathbb{Z}}



\newcommand{\il}[2]{\int\limits_{#1}^{#2}}%интеграл с пределами #1 и #2

\def\sm2{\mathop {\sum\limits^{n^\Theta}\sum\limits^{n^\Theta}}}
\def\sss{\sum\limits}
\def\tr{,\,\ldots\,,\,}
\def\rk{\right]}
\def\lk{\left[}
\def\rf{\right\}}
\def\lf{\left\{}
\def\lv{\,\left\vert}
\def\rv{\right\vert\,}
\def\iii{\int\limits}
\def\iin{\int\limits_{-\infty}^\infty}
\def\rrv{\right\vert}


\def\ee{{\cal E}}
\def\ww{{\cal W}}
\def\yy{{\cal Y}}
\def\vv{{\cal V}}

\newcommand{\R}{\mathbb R}
\newcommand{\E}{\mathbb E}
\newcommand{\N}{\mathbb N}

\renewcommand{\P}{\mathbb{P}}

\newcommand{\h}{{\bf H}}
\newcommand{\p}{{\sf P}}  % вероятность

\newcommand{\e}{{\sf E}}  % мат. ожидание
\newcommand{\D}{{\sf D}}  % дисперсия
\newcommand{\eps}{\varepsilon}
\newcommand{\vw}{{\mathbf w}}
\newcommand{\vp}{{\mathbf p}}
\newcommand{\vz}{{\mathbf z}}
\newcommand{\vx}{{\mathbf x}}
\newcommand{\vf}{{\mathbf f}}
\newcommand{\F}{{\mathcal F}}
\def\ap{{\mathrm{ЭР}}}
\newcommand{\ud}{\Delta_n} %uniform ditance
\newcommand{\nud}{\Delta_n(x)}
%\renewcommand{\Re}{\mathrm{Re}\,}

\newcommand{\abs}[1]{\left\vert#1\right\vert}

\newcommand{\norm}[1]{\left\Vert#1\right\Vert}
\def\da{(\Delta_t,A)}

\newcommand{\corr}{\mathrm{corr}}

\newcommand{\cov}{\mathrm{cov}}
\newcommand{\Expect}{\mathbb{E}}

\def\w{\omega}
\def\W{\Omega}

\def\inh{\int\limits_{nh}^{(n+1)h}}

\def\sumin{\sum_{i=1}^N}


\def\bxt{(Y,t)}
\def\xt{(y,t)}

\def\ovth{{\fr{\tau-nh}{h}}}
\def\ov{\overline}
\def\tm{\tilde m}
\def\tl{\tilde\lambda}
\def\tB{\widetilde B}
\def\tb{\tilde b}
\def\ld{\ldots}
\def\cd{\cdots}


\DeclareMathOperator{\sign}{sign}

%\newcommand{\gr}{{\geqslant}}


\newcommand{\g}{\mbox{\textit{g}}}

\renewcommand{\la}{\lambda}
\newcommand{\si}{\sigma}
\newcommand{\alp}{\alpha}

\newcommand{\pto}{\stackrel{P}{\longrightarrow}} % сходимость по веpоятности

\newcommand{\eqd}{\stackrel{\mathrm{d}}{=}} % равенство по pаспpеделению
\newcommand{\eqdelta}{\stackrel{\triangle}{=}} % равенство по pаспpеделению

\def\be#1{\begin{equation}\label{#1}}
\def\ee{\end{equation}}
\def\re#1{(\ref{#1})}

\def\bn{\begin{enumerate}}
\def\en{\end{enumerate}}
\def\bi{\begin{itemize}}
\def\ei{\end{itemize}}
%\def\i{\item}

%\newcommand{\kp}{\kappa}
%\def\Q{{\cal Q}} \def\H{{\cal H}}
%\newcommand{\bet}{\beta_{2+\delta}}


%\newtheorem{definition}{Определение}
%\renewcommand{\thedefinition}{\arabic{definition}.}
%END NEW COMMANDS

%\renewcommand{\baselinestretch}{1.2}

%\pagestyle{myheadings}

\setlength{\textwidth}{167mm}      % 122mm
\setlength{\textheight}{658pt}
%\setlength{\textheight}{635.6pt}
\setlength{\columnsep}{4.5mm}

\setcounter{secnumdepth}{4}

%\addtolength{\headheight}{2pt}
%\addtolength{\headsep}{-2mm}

\addtolength{\topmargin}{-7mm}  % for printing


%\hoffset=-30mm  % From Yap
\hoffset=-23mm  % From Acrobat

%\voffset=0mm % From Yap
\voffset=-5mm   % From Acrobat

%\addtolength{\evensidemargin}{-2.5mm} % for printing
%\addtolength{\oddsidemargin}{2.5mm}  % for printing

\addtolength{\evensidemargin}{-12mm} % for printing
\addtolength{\oddsidemargin}{8mm}  % for printing

%\renewcommand{\thefootnote}{\fnsymbol{footnote}}
%\renewcommand{\thefootnote}{\arabic{footnote}}
\renewcommand{\figurename}{\protect\bf Рис.}
\renewcommand{\tablename}{\protect\bf Таблица}

\newcommand{\Caption}[1]{\caption{\protect\small %\baselineskip=2.5ex
#1}}

\renewcommand{\thefigure}{\arabic{figure}}
\renewcommand{\thetable}{\arabic{table}}
\renewcommand{\theequation}{\arabic{equation}}
\renewcommand{\thesection}{\arabic{section}}

\renewcommand{\contentsname}{СОДЕРЖАНИЕ}
\newcommand{\fr}[2]{\displaystyle\frac{\displaystyle #1\mathstrut}{\displaystyle #2\mathstrut}}

%\renewcommand{\thefootnote}{\fnsymbol{footnote}}
%\newcommand{\g}{\mbox{\textit{g}}}

%\newcommand{\Caption}[1]{\caption{\protect\small\baselineskip=2ex #1}}
\newcounter{razdel}
\setcounter{razdel}{0}

\def\god{2022}
\def\tom{16}
\def\vyp{4}


\newcommand{\titel}[4]{%
\

\vspace*{5pt}

\ifodd\therazdel {\raggedright\noindent\Large\textrm\textbf
 \lineskip .75em
  \baselineskip=3.2ex #1 \par}
\vskip 1em {\noindent\large\textrm\textbf #2 \par}
\addcontentsline{toc}{subsection}{{\textrm\textbf #1}\protect\newline #2}
\def\rightheadline{\underline{\noindent\hbox to \textwidth{\hfill\small\textrm{#4}
%\hfill \large\bf\thepage
}}}
\def\leftheadline{\underline{\noindent\parbox{\textwidth}{
%\raggedleft\large\bf\thepage \hfill
\small\textit{#3}\hfill}}}
\def\leftfootline{\small{\textbf{\thepage}
\hfill ИНФОРМАТИКА И ЕЁ ПРИМЕНЕНИЯ\ \ \ том~\tom\ \ \ выпуск~\vyp\ \ \ \god}
}%
 \def\rightfootline{\small{ИНФОРМАТИКА И ЕЁ ПРИМЕНЕНИЯ\ \ \ том~\tom\ \ \ выпуск~\vyp\ \ \ \god
\hfill \textbf{\thepage}}}
\vskip 2em \setcounter{figure}{0}
\setcounter{table}{0}
\setcounter{equation}{0}
\setcounter{section}{0}
\setcounter{subsection}{0}
\setcounter{subsubsection}{0}
\setcounter{footnote}{0}
\setcounter{razdel}{0}
%\end{flushleft}
\else {
 \raggedright\noindent\Large\textrm\textbf
 \lineskip .75em
\baselineskip=3.2ex #1 \par} \vskip 1em
%\begin{flushleft}
{\noindent\large\textrm\textbf #2 \par}
\addcontentsline{toc}{subsection}{{\textrm\textbf #1}\protect\newline #2}
\def\rightheadline{\underline{\noindent\hbox to \textwidth{\hfill\small\textrm{#4}
%\hfill \large\bf\thepage
}}}
\def\leftheadline{\underline{\noindent\parbox{\textwidth}{%\raggedleft\large\bf\thepage \hfill
\small\textit{#3}\hfill}}}
\def\leftfootline{\small{\textbf{\thepage}
\hfill ИНФОРМАТИКА И ЕЁ ПРИМЕНЕНИЯ\ \ \ том~\tom\ \ \ выпуск~\vyp\ \ \ \god}
}%
 \def\rightfootline{\small{ИНФОРМАТИКА И ЕЁ ПРИМЕНЕНИЯ\ \ \ том~16\ \ \ выпуск~\vyp\ \ \ 2022
\hfill \textbf{\thepage}}} \vskip 2em \setcounter{figure}{0}
\setcounter{table}{0} \setcounter{equation}{0} \setcounter{section}{0}
\setcounter{subsection}{0} \setcounter{subsubsection}{0}
\setcounter{footnote}{0}
%\end{flushleft}
\fi}

\newcommand{\titelr}[2]{%
\

\vspace*{5pt}

\ifodd\therazdel {\raggedright\noindent%\Large\textrm\textbf
 \lineskip .75em
  \baselineskip=3.2ex #1 \par}
\vskip 1em {\noindent\normalsize\textrm\textbf #2 \par}
\else {
 \raggedright\noindent\Large\textrm\textbf
 \lineskip .75em
\baselineskip=3.2ex #1 \par} \vskip 1em
%\begin{flushleft}
{\noindent\large\textrm\textbf #2 \par
%\noindent\normalsize\textrm\textbf #2 \par
} \fi}

\newcommand{\titele}[5]{%
\

%\vspace*{5pt}

\ifodd\therazdel {\raggedright\noindent\large
\textrm\textbf
 \lineskip .75em
%  \baselineskip=3.2ex
#1 \par}
\vskip .5em {\noindent\large\textrm\textbf #2 \par}
\vskip .5em
 {\noindent\textrm #3 \par}
\addcontentsline{toc}{subsection}{{\textrm\textbf #1}\protect\newline #2}
\def\rightheadline{\underline{\noindent\hbox to \textwidth{\hfill\small\textrm{#4}
%\hfill \large\bf\thepage
}}}
\def\leftheadline{\underline{\noindent\parbox{\textwidth}{
%\raggedleft\large\bf\thepage \hfill
\small\textrm{#5}\hfill}}}
\def\leftfootline{\small{\textbf{\thepage}
\hfill ИНФОРМАТИКА И ЕЁ ПРИМЕНЕНИЯ\ \ \ том~16\ \ \ выпуск~4\ \ \ 2022}
}%
 \def\rightfootline{\small{ИНФОРМАТИКА И ЕЁ ПРИМЕНЕНИЯ\ \ \ том~16\ \ \ выпуск~4\ \ \ 2022
\hfill \textbf{\thepage}}} \vskip 1em \setcounter{figure}{0}
\setcounter{table}{0} \setcounter{equation}{0} \setcounter{section}{0}
\setcounter{subsection}{0} \setcounter{subsubsection}{0}
\setcounter{footnote}{0} \setcounter{razdel}{0}
%\end{flushleft}
\else {
 \raggedright\noindent\large
 \textrm\textbf
 \lineskip .75em
%\baselineskip=3.2ex
#1 \par} \vskip .5em
%\begin{flushleft}
{\noindent\large\textrm\textbf #2 \par} \vskip .5em
 {\noindent\textrm #3 \par}
\addcontentsline{toc}{subsection}{{\textrm\textbf #1}\protect\newline #2}
\def\rightheadline{\underline{\noindent\hbox to \textwidth{\hfill\small\textrm{#4}
%\hfill \large\bf\thepage
}}}
\def\leftheadline{\underline{\noindent\parbox{\textwidth}{%\raggedleft\large\bf\thepage \hfill
\small\textrm{#5}\hfill}}}
\def\leftfootline{\small{\textbf{\thepage}
\hfill ИНФОРМАТИКА И ЕЁ ПРИМЕНЕНИЯ\ \ \ том~16\ \ \ выпуск~4\ \ \ 2022}
}%
 \def\rightfootline{\small{ИНФОРМАТИКА И ЕЁ ПРИМЕНЕНИЯ\ \ \ том~16\ \ \ выпуск~4\ \ \ 2022
\hfill \textbf{\thepage}}} \vskip 1em \setcounter{figure}{0}
\setcounter{table}{0} \setcounter{equation}{0} \setcounter{section}{0}
\setcounter{subsection}{0} \setcounter{subsubsection}{0}
\setcounter{footnote}{0}
%\end{flushleft}
\fi}

\def\Abst#1{
\begin{center}\small\nwt
\parbox{150mm}{%\baselineskip=2.5ex
\textbf{Аннотация:}\ \
%\hspace*{\parindent}
#1}
\end{center}}
\def\Abste#1{
\begin{center}\small\nwt
\parbox{150mm}{%\baselineskip=2.5ex
\textbf{Abstract:}\ \
%\hspace*{\parindent}
#1}
\end{center}}

\def\DOI#1{
\begin{center}\small\nwt
\parbox{150mm}{%\baselineskip=2.5ex
\textbf{DOI:}\ \
%\hspace*{\parindent}
#1}
\end{center}}

\def\Abstend#1{
\begin{center}\small\nwt
\parbox{150mm}{%\baselineskip=2.5ex
%\hspace*{\parindent}
#1}
\end{center}}


\def\KW#1{
\begin{center}\small\nwt
\parbox{150mm}{%\baselineskip=2.5ex
\textbf{Ключевые слова:}\ \ #1}
\end{center}}

\def\KWE#1{
\begin{center}\small\nwt
\parbox{150mm}{%\baselineskip=2.5ex
\textbf{Keywords:}\ \ #1}
\end{center}}


\def\KWN#1{
%\begin{center}
%\small
%\parbox{150mm}\end{center}
}

\newcommand{\Avtors}[1]{%\smallskip
%\vspace*{.5pt}
\hangindent=23pt\noindent
%\nwt
{\bfseries#1}\
}


\renewcommand{\thesubsection}{\thesection.\arabic{subsection}\hspace*{-5pt}}
\renewcommand{\thesubsubsection}{\thesubsection\hspace*{5pt}.\arabic{subsubsection}\hspace*{-3pt}}

\newcommand{\Ack}{\section*{\protect\rmfamily Acknowledgments}\noindent}
\newcommand{\Contr}{\section*{\protect\rmfamily Contributors}\noindent}
\newcommand{\Contrl}{\section*{\protect\rmfamily Contributor}\noindent}

\makeindex


\begin{document}
\Rus

\nwt
%\ptb


%\renewcommand{\contentsname}{\protect\Large\bf Содержание}

\setcounter{tocdepth}{2}

%\tableofcontents

\renewcommand{\bibname}{\protect\rmfamily Литература}
  \def\Au#1{{\it #1}}
    \def\Aue#1{{#1}}

%\newcommand{\No}{№}
  \newcommand{\tg}{\,\mathrm{tg}\,}
    \newcommand{\ctg}{\,\mathrm{ctg}\,}
  \newcommand{\arctg}{\,\mathrm{arctg}\,}

\def\forallb{\mathop{\forall}}
\def\cupb{\mathop{\cup}}
\def\existsb{\mathop{\exists}}


\newpage
\addtocounter{razdel}{1}
%\def\razd{РЕГУЛИРУЕМЫЙ ЭЛЕКТРОПРИВОД ДЛЯ ЭЛЕКТРОЭНЕРГЕТИКИ}


\setcounter{page}{2}

%   { %\Large  
   { %\baselineskip=16.6pt
   
   \vspace*{-48pt}
   \begin{center}\LARGE
   \textit{Предисловие}
   \end{center}
   
   %\vspace*{2.5mm}
   
   \vspace*{25mm}
   
   \thispagestyle{empty}
   
   { %\small 

    
Вниманию читателей журнала <<Информатика и её применения>> предлагается 
очередной тематический выпуск <<Вероятностно-статистические методы и 
задачи информатики и информационных технологий>>. Предыдущие тематические 
выпуски журнала по данному направлению вышли в 2008~г.\ (т.~2, вып.~2), 
в 2009~г.\ (т.~3, вып.~3) и в 2010~г.\ (т.~4, вып.~2). 

Статьи, собранные в данном журнале, посвящены разработке новых вероятностно-статистических 
методов, ориентированных на применение к решению конкретных задач информатики и информационных 
технологий, а также~--- в ряде случаев~--- и других прикладных задач. Проблематика, охватываемая 
публикуемыми работами, развивается в рамках научного сотрудничества между Институтом проблем 
информатики Российской академии наук (ИПИ РАН) и Факультетом вычислительной математики и 
кибернетики Московского государственного университета им.\ М.\,В.~Ломоносова в ходе работ 
над совместными научными проектами (в том числе в рамках функционирования 
Научно-образовательного центра <<Вероятностно-статистические методы анализа рисков>>). 
Многие из авторов статей, включенных в данный номер журнала, являются активными участниками 
традиционного международного семинара по проблемам устойчивости стохастических моделей, 
руководимого В.\,М.~Золотаревым и В.\,Ю.~Королевым; регулярные сессии этого семинара 
проводятся под эгидой МГУ и ИПИ РАН (в 2011~г.\ указанный семинар проводится в октябре 
в Калининградской области РФ). 

Наряду с представителями ИПИ РАН и МГУ в число авторов данного выпуска журнала входят 
ученые из Научно-исследовательского института системных исследований РАН, Института 
проблем технологии микроэлектроники и особочистых материалов РАН, Института 
прикладных математических исследований Карельского НЦ РАН, Московского 
авиационного института, Вологодского государственного педагогического университета, 
НИИММ им.\ Н.\,Г.~Чеботарева, Казанского государственного университета, Дебреценского 
университета (Венгрия).

Несколько статей выпуска посвящено разработке и применению стохастических методов и 
информационных технологий для решения различных прикладных задач. В~работе В.\,Г.~Ушакова 
и О.\,В.~Шестакова рассмотрена задача определения вероятностных характеристик случайных 
функций по распределениям интегральных преобразований, возникающих в задачах эмиссионной 
томографии. В~статье Д.\,О.~Яковенко и М.\,А.~Целищева рассмотрены некоторые вопросы 
математической теории риска и предложен новый подход к диверсификации инвестиционных 
портфелей. Работа И.\,А.~Кудрявцевой и А.\,В.~Пантелеева посвящена построению и 
исследованию математической модели, описывающей динамику сильноионизованной плазмы. 
В~статье П.\,П.~Кольцова изучается качество работы ряда алгоритмов сегментации изображений. 
Статья А.\,Н.~Чупрунова и И.~Фазекаша посвящена вероятностному анализу числа без\-оши\-бочных 
блоков при помехоустойчивом кодировании; получены усиленные законы больших чисел для указанных 
величин.

В данном выпуске традиционно присутствует тематика, весьма активно разрабатываемая в течение 
многих лет специалистами ИПИ РАН и МГУ,~--- методы моделирования и управления для 
информационно-телекоммуникационных и вычислительных систем, в частности методы 
теории массового обслуживания. В~статье А.\,И.~Зейфмана с соавторами рассматриваются 
модели обслуживания, описываемые марковскими цепями с непрерывным временем в случае 
наличия катастроф. В~работе М.\,М.~Лери и И.\,А.~Чеплюковой рассматриваются случайные 
графы Интернет-типа, т.\,е.\ графы, степени вершин которых имеют степенные распределения; 
такие задачи находят применение при исследовании глобальных сетей передачи данных. 
Работа Р.\,В.~Разумчика посвящена исследованию систем массового обслуживания специального 
вида~--- с отрицательными заявками и хранением вытесненных заявок.

Ряд статей посвящен развитию перспективных теоретических 
вероятностно-статистических методов, которые находят широкое применение в различных 
задачах информатики и информационных технологий. В~работе В.\,Е.~Бенинга, А.\,К.~Горшенина 
и В.\,Ю.~Королева рассмотрена задача статистической проверки гипотез о числе компонент 
смеси вероятностных распределений, приводится конструкция асимптотически наиболее мощного 
критерия. Результаты этой работы найдут применение в ряде прикладных задач, использующих 
математическую модель смеси вероятностных распределений (в информатике, моделировании 
финансовых рынков, физике турбулентной плазмы и~т.\,д.). В~статье В.\,Ю.~Королева, 
И.\,Г.~Шевцовой и С.\,Я.~Шоргина строится новая, улучшенная оценка точности нормальной 
аппроксимации для пуассоновских случайных сумм; как известно, указанные случайные суммы 
широко используются в качестве моделей многих реальных объектов, в том числе в информатике, 
физике и других прикладных областях. Работа В.\,Г.~Ушакова и Н.\,Г.~Ушакова посвящена 
исследованию ядерной оценки плотности распределения; эти результаты могут применяться, 
в част\-ности, при анализе трафика в телекоммуникационных системах. Серьезные приложения 
в статистике могут получить результаты работы О.\,В.~Шестакова, в которой доказаны оценки 
скорости сходимости распределения выборочного абсолютного медианного отклонения к нормальному 
закону. 

\smallskip

Редакционная коллегия журнала выражает надежду, что данный тематический  выпуск 
будет интересен специалистам в области теории вероятностей и математической статистики 
и их применения к решению задач информатики и информационных технологий.
     
     %\vfill 
     \vspace*{20mm}
     \noindent
     Заместитель главного редактора журнала <<Информатика и её 
применения>>,\\
     директор ИПИ РАН, академик  \hfill
     \textit{И.\,А.~Соколов}\\
     
     \noindent
     Редактор-составитель тематического выпуска,\\
     профессор кафедры математической статистики факультета\\
      вычислительной математики и кибернетики МГУ им.\ М.\,В.~Ломоносова,\\
     ведущий научный сотрудник ИПИ РАН,\\ 
доктор физико-математических наук \hfill
      \textit{В.\,Ю.~Королев}
     
     } }
     }


\def\stat{stepchenkov}

\def\tit{ОЦЕНКА НАДЕЖНОСТИ СИНХРОННОГО И~САМОСИНХРОННОГО 
КОНВЕЙЕРОВ$^*$}

\def\titkol{Оценка надежности синхронного и~самосинхронного 
конвейеров}

\def\aut{И.\,А.~Соколов$^1$, Ю.\,А.~Степченков$^2$, Ю.\,Г.~Дьяченко$^3$, 
Ю.\,В.~Рождественский$^4$}

\def\autkol{И.\,А.~Соколов, Ю.\,А.~Степченков, Ю.\,Г.~Дьяченко, 
Ю.\,В.~Рождественский}

\titel{\tit}{\aut}{\autkol}{\titkol}

\index{Соколов И.\,А.}
\index{Степченков Ю.\,А.}
\index{Дьяченко Ю.\,Г.}
\index{Рождественский Ю.\,В.}
\index{Sokolov I.\,A.}
\index{Stepchenkov Yu.\,A.}
\index{Diachenko Yu.\,G.}
\index{Rogdestvenski Yu.\,V.}


{\renewcommand{\thefootnote}{\fnsymbol{footnote}} \footnotetext[1]
{Исследование выполнено в~рамках гранта Российского научного фонда (проект 22-19-00237).}}


\renewcommand{\thefootnote}{\arabic{footnote}}
\footnotetext[1]{Федеральный исследовательский центр <<Информатика и~управление>> Российской академии наук, 
\mbox{ISokolov@ipiran.ru}}
\footnotetext[2]{Федеральный исследовательский центр <<Информатика и~управление>> Российской академии наук, 
\mbox{YStepchenkov@ipiran.ru}}
\footnotetext[3]{Федеральный исследовательский центр <<Информатика и~управление>> Российской академии наук, 
\mbox{diaura@mail.ru}}
\footnotetext[4]{Федеральный исследовательский центр <<Информатика и~управление>> Российской академии наук, 
\mbox{YRogdest@ipiran.ru}}


\vspace*{-12pt}




\Abst{Самосинхронная (СС) схемотехника выступает альтернативой 
синхронным схемам. Самосинхронные схе\-мы обладают рядом преимуществ в~сравнении 
с~синхронными аналогами, но аппаратно избыточны. Статья исследует 
иммунность самосинхронных и~синхронных схем к~однократным 
кратковременным логическим сбоям (ЛС) с~учетом аппаратурной избыточности 
СС-схем. Самосинхронные схе\-мы за счет своей неотъемлемой части~--- индикаторной 
подсхемы~--- способны обнаружить ЛС, проявляющийся как 
инверсия состояния выхода логической ячейки схемы, и~приостановить 
функционирование схемы до его исчезновения. Тем самым СС-схе\-мы 
маскируют однократный ЛС и~предотвращают искажение 
данных. Использование модифицированного гистерезисного триггера для 
реализации разряда регистра ступени конвейера маскирует практически все 
ЛС в~комбинационной части (КЧ) ступени конвейера.  
DICE-по\-доб\-ная реализация этого триггера позволяет в~4~раза снизить 
чувствительность СС-ре\-гист\-ра к~ЛС внут\-ри него. 
Количественные оценки сбоеустойчивости показывают явное  
(в~2,5--9,4~раза) преимущество СС-кон\-вей\-ера схемы в~сравнении 
с~синхронным аналогом. }

\KW{самосинхронные схемы; логический сбой; сбоеустойчивость; конвейер; 
индикация; вероятностная оценка}

 \DOI{10.14357/19922264220401} 
  
\vspace*{-3pt}


\vskip 10pt plus 9pt minus 6pt

\thispagestyle{headings}

\begin{multicols}{2}

\label{st\stat}

\section{Введение}

  В современных условиях задача обеспечения\linebreak надежной работы циф\-ро\-вых 
схем выдвигается на первый план. Повышение тактовой частоты 
в~синхронной схемотехнике, все возрастающая функ\-цио\-наль\-ная слож\-ность 
интегральных мик\-ро\-схем создают предпосылки для повышения их 
чув\-ст\-ви\-тель\-ности к~не\-штат\-ным ситуациям~--- ЛС 
и~физическим отказам из-за внеш\-них и~внут\-рен\-них причин. Способ\-ность 
схемы к~маскированию ЛС и~отказов определяет уровень ее на\-деж\-ности. 
Практика показала, что отказы в~циф\-ро\-вых мик\-ро\-схе\-мах встречаются 
гораздо реже, чем ЛС~[1]. 
  
  Логический сбой проявляется как изменение логического уровня сигнала в~цепи, 
приводящее к~искажению результата обработки данных. Методы защиты от 
ЛС в~основном направлены на их маскирование и~используют 
корректирующие коды~[2], методы обнаружения и~изоляции~[3] и~некоторые 
другие подходы~[4, 5]. 
  
  Синхронные безызбыточные схемы не имеют встроенных средств 
контроля корректности переключений. В~отличие от них, асинхронные 
схемы используют зачатки контроля корректности выполняемых 
операций~[6, 7]. Однако их возможности по маскированию ЛС ограничены.
  
  Альтернативой синхронным и~асинхронным схемам выступают 
СС-схе\-мы~[8; 9; 10, p.~61--73]. Они характеризуются двухфазной 
дисциплиной работы и~обязательным под\-тверж\-де\-ни\-ем (индицированием) 
завершения переключения в~каж\-дую фазу. Благодаря этому СС-схе\-мы 
обладают естественной высокой сбоеустойчивостью~[11, 12]. Плата за эти 
преимущества~--- увеличение в~1,5--3,3~раза (в~зависимости от типа схемы) 
сложности реализации в~сравнении с~синхронными аналогами.
  
  При постоянной интенсивности событий, приводящих к~сбоям, число 
сбоев в~схеме обычно рас\-тет с~увеличением сложности схемы~[13]. Поэтому 
решаемая в~статье задача сравнительной количественной оценки 
устойчивости синхронных и~самосинхронных схем к~однократным ЛС с~учетом их 
аппаратной сложности и~особенностей функционирования особенно 
актуальна.

\vspace*{-7pt}
  
\section{Вероятность появления логического сбоя}

\vspace*{-2pt}
  
  В микросхемах, изготовленных по технологии комплементарный  
ме\-талл\,--\,ди\-элект\-рик\,--\,по\-лу\-про\-вод\-ник (КМДП), ЛС выражается 
во временном изменении потенциала некоторой цепи из-за 
индуцирования в~ней избыточных неравновесных носителей заряда. 
В~комбинационных схемах логический уровень сигнала восстанавливается 
спустя ка\-кое-то время~[14]. В~триггерных схемах сбой может запомниться, 
стать критичным. 
  
  При постоянной эксплуатационной плотности потока случайных 
событий~$\lambda_0$ (числа событий в~единицу времени на единицу 
площади), инициирующих однократные сбои, интенсивность сбоев~$\lambda$ в~схеме оценивается как сумма интенсивностей сбоев отдельных компонентов 
схемы~\cite[формула~(3.11)]{13-step}. Обычно в~качестве компонента схемы 
берется КМДП-тран\-зис\-тор~\cite{12-step}:
  $$
  \lambda= N \lambda_0 \alpha\,,
  $$
где $N$~--- число транзисторов в~схеме; $\alpha$~--- усредненная вероятность 
появления сбоя при поражении одного транзистора. Тогда отношение 
интенсивностей сбоев для СС-схе\-мы и~синхронного аналога
$$
K_\lambda= \fr{\lambda_S}{\lambda_{\mathrm{ST}}} =\fr{N_S \lambda_0 \alpha_S}{N_{\mathrm{ST}} 
\lambda_0 \alpha_{\mathrm{ST}}}= \fr{N_S \alpha_S}{N_{\mathrm{ST}} \alpha_{\mathrm{ST}}} =\fr{\alpha_S}{A_R \alpha_{\mathrm{ST}}}\,,
$$
где $\lambda_S$ и~$\lambda_{\mathrm{ST}}$~--- интенсивности сбоев в~синхронной 
и~самосинхронной схе\-мах; $N_S$ и~$N_{\mathrm{ST}}$~--- слож\-ность (чис\-ло  
КМДП-тран\-зис\-то\-ров) синхронной и~самосинхронной схем; $\alpha_S$ и~$\alpha_{\mathrm{ST}}$~--- 
вероятность сбоя при повреждении одного транзистора синхронной  
и~самосинхронной схем; $A_R\hm= N_{\mathrm{ST}}/N_S$~--- коэффициент аппаратной 
избыточности СС-схе\-мы в~сравнении с~синхронным аналогом.

  Типовой эффективный диаметр трека ядерной частицы, одного из 
источников сбоев, достигает 2--2,5~мкм~\cite{15-step}.  
В~КМДП-тех\-но\-ло\-гии с~проектными нормами~65~нм это, например, 
соответствует размеру схемы из нескольких транзисторов. Поэтому 
целесообразно рассматривать вероятность появления сбоя применительно 
к~логическим ячейкам схемы.
  
  Будем рассматривать цифровую схему как совокупность библиотечных 
ячеек, соединенных сигнальными цепями. Для упрощения будем считать, что 
появление сбоя в~любом месте принципиальной схемы ячейки вызывает 
инверсию уровня сигнала на ее выходе с~вероятностью~0,5. Пусть одно 
событие, порождающее сбой, может привести к~ЛС только в~одной ячейке 
схемы. Тогда интенсивность сбоев~$\lambda_i$ в~$i$-й ячейке схемы равна
  $$
  \lambda_i= \fr{S_i}{2S}\,\lambda_0 {\sf P}_{\mathrm{Э}}\,,
  $$
где $S_i$ и~$S$~--- площади топологии $i$-й ячейки и~всей схемы; 
${\sf P}_{\mathrm{Э}}$~--- вероятность индуцирования критического избыточного 
заряда одним сбойным со\-бы\-тием. 
  
  Однако не все ЛС в~ячейках схемы проявляются на ее выходах, поскольку 
при распространении по схеме они могут быть замаскированы. 
Интенсивность ЛС на выходах схемы
  \begin{multline}
  \lambda_{\mathrm{вых}} = \sum\limits_{i=1}^{M_{\mathrm{вых}}} 
\lambda_i +\sum\limits_{i=M_{\mathrm{вых}}+1}^M \left( \lambda_i {\sf P}_{{P},i}\right) ={}\\
  {}= \fr{\lambda_0 {\sf P}_{\mathrm{Э}}}{2S} \left( 
\sum\limits_{i=1}^{M_{\mathrm{вых}}} S_i 
+\sum^M_{i=M_{\mathrm{вых}}+1} \left( S_i {\sf P}_{P,i}\right)\right)\,,
  \label{e1-step}
  \end{multline}
где $M$~--- общее число ячеек в~схеме; $M_{\mathrm{вых}}$~--- чис\-ло 
выходных ячеек в~схеме; ${\sf P}_{P,i}$~--- вероятность того, что сбой 
на выходе $i$-й внутренней ячейки приведет к~сбою на выходе всей схемы. 

Оценим вероятность~${\sf P}_{P,i}$.

\section{Маскирование логического сбоя логикой схемы}

  Введем вероятность ${\sf P}_{P,ij}$ прохода ЛС с~выхода $i$-й 
ячейки на выход~$Y_j$ схемы. Пусть зависимость~$Y_j$ от внутренних и~внешних сигналов схемы описывается монотонной функцией $Y_j\hm = 
F_j(X_1, \ldots , X_M)$. С~учетом разложения~$Y_j$ по~$X_i$,
\begin{multline*}
Y_j = F_{0ij}\left (X_1, \ldots , X_{i-1}, X_{i+1}, \ldots, X_M\right) + {}\\
{}+F_{1ij}\left(X_1, \ldots , X_{i-1}, 
X_{i+1}, \ldots, X_M\right) X_i,
\end{multline*}
вероятность распространения ЛС от $X_i$ до~$Y_j$:
$$
{\sf P}_{P,ij} =\fr{N_{F_0ij} N_{F_1ij}}{2^{M_j-1}}\,,
$$
где $N_{F_0ij}$ и~$ N_{F_1ij}$~--- число комбинаций 
входов схемы, при которых $F_{0ij}\hm=0$ и~$F_{1ij}\hm=1$ соответственно; 
$M_j$~--- реальное число сигналов, от которых зависит~$Y_j$. Если 
$F_{0ij}\hm\equiv0$, то $N_{F_0ij}\hm=1$; если $F_{1ij}\hm\equiv1$, то 
$N_{F_1ij}\hm=1$. Вероятность появления на выходе~$Y_j$ ЛС, 
наблюдаемого на выходе одной из ячеек схемы,
$$
{\sf P}_{P,j}= \sum\limits^M_{i=1} \left( 
\fr{N_{F_0ij} N_{F_1ij}}{2^{M_j-1}} \prod\limits^{i-
1}_{k=1} \left( 1- \fr{N_{F_0kj} N_{F_1kj}}{2^{M_j-1}}\right)\right)\,,
$$
а вероятность появления сбоя, поразившего $i$-ю ячейку, на выходах схемы
\begin{equation}
{\sf P}_{P,i}= \sum\limits_{j=1}^{M_{\mathrm{вых}}} 
\fr{N_{F_0ij} N_{F_1ij}}{2^{M_j-1}}\,.
\label{e2-step}
\end{equation}

\begin{figure*}[b] %fig1
\vspace*{1pt}
\begin{center}
     \mbox{%
\epsfxsize=110.695mm
\epsfbox{ste-1.eps}
}

\vspace*{6pt}

\noindent
\small{Структура типового СС-конвейера}
\end{center}
\end{figure*}
  
  Аналогичные вероятности могут быть рассчитаны для любой пары цепей 
схемы, что позволяет получить оценки чувствительности схемы к~ЛС в~ее 
ячейках уже на этапе ее логического синтеза. 

%  \begin{table*}
{\small
  \begin{center}
  \begin{tabular}{|c|c|c|c|}
  \multicolumn{4}{c}{Парафазное кодирование информационного сигнала}\\
  \multicolumn{4}{c}{\ }\\[-6pt]
  \hline
№&Х&$\{$X, ХВ$\}$&Значение\\
\hline
1&---&00&Нулевой спейсер\\
2&0&01&Бит <<0>>\\
3&1&10&Бит <<1>>\\
4&---&11&Единичный спейсер\\
\hline
\end{tabular}
\end{center}
%\end{table*}
}

\vspace*{9pt}
  
  Сбой выхода СС-схе\-мы часто маскируется СС-дис\-цип\-ли\-ной за счет 
избыточного (парафазного~\cite{9-step}) кодирования данных и~двухфазной 
работы. При парафазном кодировании каждый синхронный сигнал~X 
заменяется парафазным сигналом $\{\mathrm{X, XB}\}$, как показано 
в~таб\-лице. 
  

  
  Парафазный сигнал формируется двумя согласованными логическими 
ячейками. Следовательно, однократный ЛС изменяет одну компоненту 
парафазного сигнала, делая его состояние не соответствующим текущей фазе 
схемы. Индикаторная подсхема обнаруживает это нарушение 
и~оста\-нав\-ли\-ва\-ет функционирование СС-схе\-мы до исчезновения сбоя.
  
  Практические СС-схе\-мы обычно реализуются в~виде конвейера для 
повышения про\-из\-во\-ди\-тель\-ности аналогично синхронным схемам. 
В~типовом СС-кон\-вей\-ере каждая ступень состоит из КЧ 
и~выходного регистра (ВР), как показано на рисунке. Их 
индикаторные подсхемы ИКЧ и~ИВР с~помощью гистерезисного триггера  
(Г-триг\-ге\-ра~\cite{9-step}, на рисунке обозначен буквой~H) разрешают 
предыдущей ступени конвейера переключаться в~следующую фазу работы.
  
  Анализ возможных ситуаций появления ЛС в~ступени конвейера 
показывает, что сбой в~ее КЧ в~худшем случае приводит 
лишь к~приостановке работы конвейера, но не портит обрабатываемые 
данные, если разряды регистра ступени реализованы сбоеустойчивой 
схемой~\cite[Fig.~10]{16-step}. Однако появление ЛС непосредственно 
в~выходном регистре с~вероятностью 0,25 вызывает искажение результата 
обработки данных или <<зависание>> конвейера. Использование в~разряде 
регистра \mbox{Г-триг}\-ге\-ра с~DICE-по\-доб\-ной  
реализацией~\cite[Fig.~12]{16-step} в~4~раза улучшает иммунность регистра.
  
Индикаторные подсхемы КЧ и~регистра ступени  
СС-кон\-вей\-ера вносят незначительный вклад в~чувствительность  
СС-кон\-вей\-ера к~ЛС. Критическая ситуация может возникнуть только в~том 
случае, если ЛС поражает выходной Г-триг\-гер, что в~многоразрядных  
СС-схе\-мах крайне маловероятно.
  

  
  Суммарная вероятность искажения данных\linebreak в~$m$-й ступени  
СС-кон\-вей\-ера с~$n_m$-раз\-ряд\-ным выходным регистром из-за 
ЛС зависит от площадей топологии КЧ
($S_{\mathrm{CP},m}$), Г-триг\-ге\-ров ($S_H$) и~индикаторного элемента ($S_{\mathrm{IE}}$) 
в~разрядах регистра и~индикаторных подсхем КЧ
($S_{\mathrm{CPI},m}$) и~ВР ($S_{\mathrm{ORI},m}$):
  \begin{multline}
  {\sf P}_{\mathrm{ST},m} ={}\\
  \!\!\!{}=\fr{0{,}25 n_m (2S_H+S_{\mathrm{IE}})}{S_{\mathrm{CP},m} +n_m(2S_H+S_{\mathrm{IE}}) 
+S_{\mathrm{CPI},m} +S_{\mathrm{ORI},m}}\,.\!
  \label{e3-step}
  \end{multline}
  
  Пусть КЧ содержит~$M$~ячеек и~ее сложность 
в~$K_{\mathrm{CP},m}$ раз превышает сложность регистра. Тогда  
формула~(\ref{e3-step}) преобразуется к~виду:
  $$
  {\sf P}_{\mathrm{ST},m} \approx \fr{0{,}55}{2{,}2 K_{\mathrm{CP},m}+2{,}7+ 0{,}25M/n_m}\,.
  $$
При реальных значениях $K_{\mathrm{CP},m}\hm = 4$ и~$M\hm = 8n_m$ вероятность 
критического сбоя ${\sf P}_{\mathrm{ST},m}\hm\approx 0{,}041$. При использовании  
DICE-по\-доб\-но\-го Г-триг\-ге\-ра она уменьшается до величины 
${\sf P}_{\mathrm{DICE},m}\hm\approx 0{,}011$, т.\,е.\ почти в~4~раза.

  В синхронном конвейере однократный ЛС, наблюдаемый 
в~любой части $m$-й ступени, не может замаскироваться дисциплиной 
работы. С~учетом формулы~(\ref{e2-step}) вероятность искажения данных 
из-за сбоя в~КЧ $m$-й ступени 
  $$
 {\sf P}_{S,m}=\sum\limits^{M_S}_{i=1} \left( \fr{S_i}{S} 
\sum\limits_{j=1}^{M_{S_{\mathrm{вых}}}} \fr{N_{F_0ij} 
N_{F_1ij}}{2^{M_j-1}}\right)\,,
  $$
где $M_S$~--- число ячеек в~КЧ $m$-й ступени 
синхронного конвейера; $M_{S_{\mathrm{вых}}}$~--- число ее выходов. 
В~первом приближении эту вероятность можно оценить как ${\sf P}_{S,m}\hm =  0{,}25$~\cite{16-step}.

\section{Сравнение сбоеустойчивости синхронных и~самосинхронных схем}

  При заданной эксплуатационной плотности потока сбойных 
событий~$\lambda_0$ интенсивность критических сбоев на выходах $m$-й 
ступени конвейера равна
  $$
  \lambda_{\mathrm{CF,ST}}\approx \lambda_{\mathrm{ST}} P_{\mathrm{ST},m} 
=\fr{0{,}55\lambda_{\mathrm{ST}}}{2{,}2 K_{\mathrm{CP},m} +2{,}7 + 0{,}25M/n_m}
  $$
для СС-конвейера и~$$
\lambda_{\mathrm{CF},S} \approx \lambda_S P_{S,m} =0{,}25 \lambda_S
$$
для синхронного конвейера. Следовательно, отношение интенсивностей 
критических сбоев для синхронного и~самосинхронного кон\-вей\-ера:
\begin{multline}
K_{\mathrm{CF}}= \fr{\lambda_{\mathrm{CF},S}}{\lambda_{\mathrm{CF,ST}}} ={}\\
{}=\fr{0{,}25 \lambda_S \left( 
2{,}2 K_{\mathrm{CP},m} +2{,}7 + 0{,}25M/n_m\right)} {0{,}55 \lambda_{\mathrm{ST}}}\,.
\label{e4-step}
\end{multline}
  
  Поскольку синхронная КЧ в~2~раза проще, чем  
в~СС-кон\-вей\-ере с~парафазным кодированием, и~индикаторной подсхемы 
нет, соотношение площадей топологий синхронного ($S_{S,P}$)  
и~самосинхронного кон\-вей\-еров ($S_{ST,P}$) 
  $$
  K_H= \fr{S_{\mathrm{ST},P}}{S_{S,P}}=2+\fr{4{,}25 M+8{,}5 n_m} {20n_m 
(K_{\mathrm{CP},m}+1)}\,.
  $$

Для реальных значений $K_{\mathrm{CP},m}\hm = 4$, $M\hm = 8n_m$ и~$n_m \hm= 32$ 
получается $K_H\hm = 2{,}4$. Тогда в~соответствии  
с~формулами~(\ref{e1-step}), (\ref{e2-step}) и~(\ref{e4-step}) сбоеустойчивость 
СС-кон\-вей\-ера оказывается лучше сбое\-устой\-чи\-вости синхронного 
конвейера в~2,5--9,4~раза\linebreak в~за\-ви\-си\-мости от схемы реализации разряда 
СС-ре\-гистра.

\section{Заключение}

  Самосинхронные схемы обладают естественной высокой иммунностью к~ЛС
   благодаря избыточному кодированию данных, двухфазной работе 
и~контролю окончания переключения в~каждую фазу. Анализ сбойных 
ситуаций показывает, что наиболее чувствительной к~однократным 
ЛС частью ступени СС-кон\-вей\-ера оказывается ВР. Однако реализация его разрядов на DICE-по\-доб\-ных  
Г-триг\-ге\-рах повышает его сбоеустойчивость в~4~раза.
  
  Реализация цифровой схемы в~виде СС-кон\-вей\-ера гарантирует 
повышение ее устойчивости к~однократным ЛС
в~2,5--9,4~раза в~сравнении с~синхронным конвейером, причем при 
появлении \mbox{критического} сбоя СС-кон\-вей\-ер останавливается и~своими 
индикаторными сигналами локализует место сбоя. Недостатком такой 
реализации является увеличенная в~2,4~раза сложность и,~соответственно, 
площадь схемы в~топологии.

\vspace*{-6pt}
  
{\small\frenchspacing
 {%\baselineskip=10.8pt
 %\addcontentsline{toc}{section}{References}
 \begin{thebibliography}{99}
 \bibitem{1-step}
 \Au{Викторова В.\,C., Лубков Н.\,В., Степанянц~А.\,С.} Анализ надежности 
отказоустойчивых управ\-ля\-ющих вы\-чис\-ли\-тель\-ных сис\-тем.~--- М.: ИПУ РАН, 2016. 117~c.
 \bibitem{2-step}
 \Au{Morelos-Zaragoza R.\,H.} The art of error correcting coding.~--- 2nd ed.~--- Hoboken, NJ, USA: Wiley, 
2006. 269~p.
 \bibitem{3-step}
 \Au{LaFrieda C., Manohar~R.} Fault detection and isolation techniques for quasi  
delay-insensitive circuits~// Conference (International) on Dependable Systems and Networks, 
2004. P.~41--50. doi: 10.1109/DSN.2004.1311875.
 \bibitem{4-step}
 \Au{Monnet Y., Renaudin M., Leveugle~R.} Hardening techniques against transient faults for 
asynchronous circuits~// 11th On-Line Testing Symposium (International) Proceedings.~--- IEEE, 
2005. P.~129--134. 
 \bibitem{5-step}
 \Au{Dug M., Krstic~M., Jokic~D.} Implementation and analysis of methods for error detection 
and correction on FPGA~// IFAC-PapersOnLine, 2018. Vol.~51. No.\,6. P.~348--353.
 
 \bibitem{7-step} %6
 \Au{Lodhi F.\,K., Hasan~S., Hasan~O., Awwad~F.} Low power soft error tolerant macro 
synchronous micro asynchronous (MSMA) pipeline~// Computer Society Annual 
Symposium on VLSI Proceedings.~--- Piscataway, NJ, USA: IEEE, 2014. P.~601--606. doi: 
10.1109/ISVLSI.2014.59.

\bibitem{6-step} %7
 \Au{Gkiokas C., Schoeberl M.\,A.} Fault-tolerant time-predictable processor~// Nordic 
Circuits and Systems Conference: NORCHIP and Symposium (International) of  
System-on-Chip Proceedings.~--- Piscataway, NJ, USA: IEEE, 2019. Art.\ 8906947. 6~p. doi: 
10.1109/NORCHIP. 2019.8906947.

 \bibitem{8-step}
 \Au{Muller D., Bartky~W.} A~theory of asynchronous circuits~// Symposium 
(International) on the Theory of Switching Proceedings.~--- Harvard University Press, 1959. Vol.~29. P.~204--243.
 \bibitem{9-step}
 \Au{Kishinevsky M., Kondratyev~A., Taubin~A., Varshavsky~V.} Concurrent hardware: The 
theory and practice of self-timed design.~--- New York, NY, USA: John Wiley \& Sons, 1994. 368~p.
 \bibitem{10-step}
 \Au{Smith S.\,C., Jia~Di.} Designing asynchronous circuits using NULL convention logic 
(NCL).~--- Synthesis lectures on digital circuits and systems ser.~--- Cham: Springer, 2009. %Vol.~4. No.\,1. P.~61--73.
96~p.
 \bibitem{11-step}
 \Au{Stepchenkov Y.\,A., Kamenskih~A.\,N., Diachenko~Y.\,G., Rogdestvenski~Y.\,V., 
Diachenko~D.\,Y.} Improvement of the natural self-timed circuit tolerance to short-term soft 
errors~// Advances Science Technology Engineering Systems~J., 2020. Vol.~5. No.\,2. P.~44--56. 
 \bibitem{12-step}
 \Au{Соколов И.\,А., Степченков~Ю.\,А., Рождественский~Ю.\,В., Дьяченко~Ю.\,Г.} 
Приближенная оценка эффективности синхронной и~самосинхронной методологий 
в~задачах проектирования сбоеустойчивых вы\-чис\-ли\-тель\-но-управ\-ля\-ющих  
сис\-тем~// Автоматика и~телемеханика, 2022. №\,2. С.~122--132.
 \bibitem{13-step}
 \Au{Dubrova E.} Fault-tolerant design.~--- New York, NY, USA: Springer, 2013. 185~p. doi: 
10.1007/978-1-4614-2113-9.
 \bibitem{14-step}
 \Au{Eaton P., Benedetto~J., Mavis~D., Avery~K., Sibley~M., Gadlage~M., Turflinger~T.} 
Single event transient pulse width measurements using a variable temporal latch technique~// 
IEEE T. Nucl. Sci., 2004. Vol.~51. No.\,6. P.~3365--3368. doi: 10.1109/TNS.2004.840020.
 \bibitem{15-step}
 \Au{Emeliyanov V.\,V., Vatuev A.\,S., Useinov~R.\,G.} Impact of heavy ion energy on charge 
yield in silicon dioxide~// IEEE T. Nucl. Sci., 2018. Vol.~65. No.\,8. P.~1496--1502.
 \bibitem{16-step}
 \Au{Stepchenkov Y., Diachenko~Y., Rogdestvenski~Y., Shikunov~Y., Diachenko~D.}  
Self-timed storage register soft error tolerance improvement~// East--West Design \& 
Test Symposium Proceedings.~--- Piscataway, NJ, USA: IEEE, 2021. P.~145--150.

\end{thebibliography}

 }
 }

\end{multicols}

\vspace*{-6pt}

\hfill{\small\textit{Поступила в~редакцию 20.06.22}}

\vspace*{8pt}

%\pagebreak

%\newpage

%\vspace*{-28pt}

\hrule

\vspace*{2pt}

\hrule

%\vspace*{-2pt}

\def\tit{SYNCHRONOUS AND SELF-TIMED PIPELINE'S RELIABILITY ESTIMATION}


\def\titkol{Synchronous and self-timed pipeline's reliability estimation}


\def\aut{I.\,A.~Sokolov, Yu.\,A.~Stepchenkov, Yu.\,G.~Diachenko, and~Yu.\,V.~Rogdestvenski}

\def\autkol{I.\,A.~Sokolov, Yu.\,A.~Stepchenkov, Yu.\,G.~Diachenko, and~Yu.\,V.~Rogdestvenski}

\titel{\tit}{\aut}{\autkol}{\titkol}

\vspace*{-8pt}


\noindent
Federal Research Center ``Computer Science and Control'' of the Russian Academy of Sciences, 44-2 
Vavilov Str., Moscow 119333, Russian Federation


\def\leftfootline{\small{\textbf{\thepage}
\hfill INFORMATIKA I EE PRIMENENIYA~--- INFORMATICS AND
APPLICATIONS\ \ \ 2022\ \ \ volume~16\ \ \ issue\ 4}
}%
 \def\rightfootline{\small{INFORMATIKA I EE PRIMENENIYA~---
INFORMATICS AND APPLICATIONS\ \ \ 2022\ \ \ volume~16\ \ \ issue\ 4
\hfill \textbf{\thepage}}}

\vspace*{3pt} 
 
 





\Abste{Self-timed (ST) circuitry is an alternative to synchronous circuits. Self-timed circuits have a number of 
advantages over their synchronous counterparts due to their redundant complexity. The article 
investigates the immunity of self-timed and synchronous circuits to single short-term soft error taking 
into account the hardware redundancy of ST circuits. Self-timed circuits, due to their indication subcircuit, are 
able to detect a~soft error which occurs as a~logical cell's output state inversion and suspend the 
operation of the circuit until the soft error disappears. Thus, ST circuits mask a~single soft error and 
prevent distortion of the data processing result. The use of a~modified hysteretic trigger, which prevents 
sticking in the antispacer, to implement a pipeline stage register bit masks almost all soft errors in the 
pipeline stage's combinational part. The DICE-like implementation of this trigger makes it possible to 
reduce the sensitivity of the ST register to the internal soft errors by a~factor of~4. Quantitative 
estimates of failure tolerance show a~clear (by 2.5--9.4~times) advantage of the ST pipeline in 
comparison with the synchronous counterpart.}

\KWE{self-timed circuit; soft error; failure tolerance; pipeline; indication; probabilistic estimate}




 \DOI{10.14357/19922264220401} 

\vspace*{-8pt}

\Ack
\noindent
The research was supported by the Russian Science Foundation (project No.\,22-19-00237).


%\vspace*{5pt}

  \begin{multicols}{2}

\renewcommand{\bibname}{\protect\rmfamily References}
%\renewcommand{\bibname}{\large\protect\rm References}

{\small\frenchspacing
 {%\baselineskip=10.8pt
 \addcontentsline{toc}{section}{References}
 \begin{thebibliography}{99}
\bibitem{1-step-1}
\Aue{Viktorova, V.\,S., N.\,V.~Lubkov, and A.\,S.~Stepanyants.} 2016. \textit{Analiz nadezhnosti 
otkazoustoychivykh upravlyayushchikh vychislitel'nykh sistem} [Analysis of fault-tolerant computing 
systems' reliability]. Moscow: IPU RAN. 117~p.
\bibitem{2-step-1}
\Aue{Morelos-Zaragoza, R.\,H.} 2006. \textit{The art of error correcting coding}. Hoboken, NJ: Wiley. 
269~p.
\bibitem{3-step-1}
\Aue{LaFrieda, C., and R.~Manohar.} 2004. Fault detection and isolation techniques for quasi  
delay-insensitive circuits. \textit{Conference (International) on Dependable Systems and Networks 
Proceedings}. 41--50. doi: 10.1109/ DSN.2004.1311875.
\bibitem{4-step-1}
\Aue{Monnet Y., M.~Renaudin, and R.~Leveugle.} 2005. Hardening techniques against transient faults 
for asynchronous circuits. \textit{11th On-Line Testing Symposium (International) Proceedings}. IEEE. 
129--134.
\bibitem{5-step-1}
\Aue{Dug, M., M.~Krstic, and D.~Jokic.} 2018. Implementation and analysis of methods for error 
detection and correction on FPGA. \textit{IFAC-PapersOnLine} 51(6):348--353.

\bibitem{7-step-1} %6
\Aue{Lodhi, F.\,K., S.~Hasan, O.~Hasan, and F.~Awwad.} 2014. Low power soft error tolerant macro 
synchronous micro asynchronous pipeline. \textit{Computer Society Annual Symposium on VLSI 
Proceedings}. Piscataway, NJ: IEEE. 601--606. doi: 10.1109/ISVLSI.2014.59.

\bibitem{6-step-1} %7
\Aue{Gkiokas, C., and M.\,A.~Schoeberl.} 2019. Fault-tolerant time-predictable processor. \textit{Nordic 
Circuits and Systems Conference: NORCHIP  Symposium (International) of System-on-Chip 
Proceedings}. Piscataway, NJ: IEEE. 8906947. 6~p. doi: 10.1109/NORCHIP.2019.8906947.

\bibitem{8-step-1}
\Aue{Muller, D.\,E., and W.\,C.~Bartky.} 1959. A theory of asynchronous circuits. \textit{Symposium 
(International) on the Theory of Switching Proceedings}. Harvard University Press. 29:204--243.
\bibitem{9-step-1}
\Aue{Kishinevsky, M., A.~Kondratyev, A.~Taubin, and V.~Varshavsky}. 1994. \textit{Concurrent 
hardware: The theory and practice of self-timed design}. New York, NY: John Wiley \& Sons. 368~p.
\bibitem{10-step-1}
\Aue{Smith, S.\,C., and J.~Di.} 2009. \textit{Designing asynchronous circuits using NULL convention logic 
(NCL)}. Synthesis lectures on digital circuits systems ser. Cham: Springer. 96~p.
%4(1):61--73.
\bibitem{11-step-1}
\Aue{Stepchenkov, Y.\,A, A.\,N.~Kamenskih, Y.\,G.~Diachenko, Y.\,V.~Rogdestvenski, and 
D.\,Y.~Diachenko.} 2020. Improvement of the natural self-timed circuit tolerance to shortterm soft errors. 
\textit{Advances Science Technology Engineering Systems~J.} 5(2):44--56.
\bibitem{12-step-1}
\Aue{Sokolov, I.\,A., Yu.\,A.~Stepchenkov, Yu.\,V.~Rozhdestvenskiy, and Yu.\,G.~Diachenko.} 2022. 
An approximate evaluation of the efficiency of synchronous and self-timed methodologies in designing 
failure-tolerant computing and control systems. \textit{Automat. Rem. Contr.} 83(2):264--272.
\bibitem{13-step-1}
\Aue{Dubrova, E.} 2013. \textit{Fault-tolerant design}. New York, NY: Springer. 185~p.
doi: 10.1007/978-1-4614-2113-9.
\bibitem{14-step-1}
\Aue{Eaton, P., J.~Benedetto, D.~Mavis, K.~Avery, M.~Sibley, M.~Gadlage, and T.~Turflinger.} 2004. 
Single event transient pulse width measurements using a~variable temporal latch technique. \textit{IEEE 
T. Nucl. Sci.} 51(6):3365--3368. doi: 10.1109/TNS.2004.840020.
\bibitem{15-step-1}
\Aue{Emeliyanov, V.\,V., A.\,S.~Vatuev, and R.\,G.~Useinov.} 2018. Impact of heavy ion energy on 
charge yield in silicon dioxide. \textit{IEEE T. Nucl. Sci.} 65(8):1496--1502.
\bibitem{16-step-1}
\Aue{Stepchenkov, Y., Y.~Diachenko, Y.~Rogdestvenski, Y.~Shi\-ku\-nov, and D.~Diachenko.} 2021. 
Self-timed storage register soft error tolerance improvement. \textit{East--West Design \& Test 
Symposium Proceedings}. Puscataway, NJ: IEEE. 145--150.
\end{thebibliography}

 }
 }

\end{multicols}

\vspace*{-6pt}

\hfill{\small\textit{Received June 20, 2022}}

\Contr

\noindent
\textbf{Sokolov Igor A.} (b.\ 1954)~--- Doctor of Science in technology, Academician of RAS, director, 
Federal Research Center ``Computer Science and Control'' of the Russian Academy of Sciences,  
44-2~Vavilov Str., Moscow 119133, Russian Federation; \mbox{isokolov@ipiran.ru}

\vspace*{3pt}

\noindent
\textbf{Stepchenkov Yuri A.} (b.\ 1951)~--- Candidate of Science (PhD) in technology, head of 
department, leading scientist, Federal Research Center ``Computer Science and Control'' of the Russian 
Academy of Sciences, 44-2~Vavilov Str., Moscow 119133, Russian Federation; 
\mbox{YStepchenkov@ipiran.ru}

\vspace*{3pt}

\noindent
\textbf{Diachenko Yuri G.} (b.\ 1958)~--- Candidate of Science (PhD) in technology, senior scientist, 
Federal Research Center ``Computer Science and Control'' of the Russian Academy of Sciences,  
44-2~Vavilov Str., Moscow 119133, Russian Federation; \mbox{diaura@mail.ru}

\vspace*{3pt}

\noindent
\textbf{Rogdestvenski Yuri V.} (b.\ 1952)~--- Candidate of Science (PhD) in technology, leading 
scientist, Federal Research Center ``Computer Science and Control''' of the Russian Academy of 
Sciences, 44-2~Vavilov Str., Moscow 119333, Russian Federation; \mbox{YRogdest@ipiran.ru}

 

\label{end\stat}

\renewcommand{\bibname}{\protect\rm Литература}     %1
%\newcommand {\ff}{{\mathcal F}}
\newcommand {\ebd}{\triangleq}
\newcommand{\me}[2]{\mathbf{E}_{ #1 }\left\{ \mathop{#2} \right\} }



\def\stat{borisov}

\def\tit{ФИЛЬТРАЦИЯ СОСТОЯНИЙ МАРКОВСКИХ СКАЧКООБРАЗНЫХ ПРОЦЕССОВ 
ПО~ДИСКРЕТИЗОВАННЫМ НАБЛЮДЕНИЯМ$^*$}

\def\titkol{Фильтрация состояний марковских скачкообразных процессов 
по~дискретизованным наблюдениям}

\def\aut{А.\,В.~Борисов$^1$}

\def\autkol{А.\,В.~Борисов}

\titel{\tit}{\aut}{\autkol}{\titkol}

\index{Борисов А.\,В.}
\index{Borisov A.\,A.}




{\renewcommand{\thefootnote}{\fnsymbol{footnote}} \footnotetext[1]
{Работа выполнена при частичной поддержке РФФИ (проект 16-07-00677).}}


\renewcommand{\thefootnote}{\arabic{footnote}}
\footnotetext[1]{Институт проблем информатики Федерального исследовательского центра <<Информатика 
и~управление>> Российской академии наук,
\mbox{aborisov@frccsc.ru}}

%\vspace*{8pt}



\Abst{Статья посвящена решению задачи оптимальной 
фильтрации состояний однородного марковского скачкообразного процесса (МСП). 
Наблюдения представляют собой приращения случайных процессов~--- интегральных 
преобразований состояний, зашумленные винеровскими процессами, интенсивность 
которых также зависит от оцениваемого состояния. Оптимальная оценка в~моменты 
получения нового наблюдения вычисляется как функция предыдущей оценки и~новых 
наблюдений, а~между моментами наблюдений~--- простейшим прогнозом в~силу системы 
уравнений Колмогорова. Рекуррентная формула пересчета ресурсозатратна, так как 
содержит  интегралы~--- мас\-штаб\-но-сдви\-го\-вые смеси многомерных гауссиан, 
где в~качестве смешивающих выступают распределения времени пребывания 
состояния в~каждом из возможных значений. Предложены более простые аппроксимации, 
основанные на предположении об ограниченности числа скачков состояния за время между 
наблюдениями. Получены универсальные локальная и~глобальная характеристики точности 
аппроксимаций, зависящие от па\-ра\-мет\-ров оцениваемого процесса, величины 
временн$\acute{\mbox{о}}$го шага  между наблюдениями и~максимального числа учитываемых скачков.}

\KW{марковский скачкообразный процесс; оптимальная фильтрация; мультипликативные 
шумы в~наблюдениях; стохастическое дифференциальное уравнение; численная аппроксимация}

\DOI{10.14357/19922264180316}
  
%\vspace*{4pt}


\vskip 10pt plus 9pt minus 6pt

\thispagestyle{headings}

\begin{multicols}{2}

\label{st\stat}



 \section{Введение}
 
 Фильтр Вонэма~\cite{Won_65}~--- один из редких удачных случаев, когда 
 оценка оптимальной фильтрации состо\-яния стохастической системы наблюдения 
 выражается в~виде решения некоторой замк\-ну\-той\linebreak конечномерной сис\-те\-мы 
 стохастических дифференциальных уравнений. 
 
 Алгоритм данного фильт\-ра 
 позволяет вычислить оценку фильт\-ра\-ции со\-сто\-яния \textit{марковского скачкообразного 
 процесса} с~\mbox{конечным} множеством состояний по наблюдениям в~присутствии 
 аддитивных винеровских шумов. Теоретически оптимальная оценка со\-сто\-яния~--- 
 его условное распределение в~текущий момент времени~--- 
 обладает очевидными свойствами неотрицательности и~нормировки. 
 При чис\-лен\-ной реализации данного фильтра классическим методом 
 Эй\-ле\-ра--Ма\-ру\-ямы~\cite{KP_92} данные свойства могут не сохраняться и~процедура 
 вы\-чис\-ле\-ния становится неустойчивой.  В~связи с~этим обстоятельством разрабатывались 
 другие алгоритмы чис\-лен\-но\-го решения уравнения фильтра Вонэма, обладающие 
 требуемыми свойствами устойчивости (см.~\cite{YZL_04, PR_10} и~библиографию в~них). 
 В~час\-ти этих работ доказана лишь слабая сходимость пред\-ла\-га\-емых аппроксимационных 
 схем к~оценке фильт\-ра Вонэма, в~то время как ка\-кая-ли\-бо 
 характеризация точ\-ности этих приближений отсутствует.
 
 В~\cite{B_18} было представлено распространение фильт\-ра Вонэма на случай 
 наблюдений с~мультипликативными шумами. При этом уравнение обобщенного 
 фильт\-ра содержит в~правой части квадратическую характеристику шумов в~наблюдениях. 
 Данный процесс на практике никогда не наблюдается непосредственно, а~является лишь 
 некоторым нелинейным интегральным преобразованием наблюдений. Очевидно, что 
 имеющиеся в~настоящий момент времени алгоритмы приближенного вычисления оценки 
 фильтрации Вонэма для данной системы не подходят. 
 
 Целью предлагаемой работы является ис\-поль\-зование результатов оптимальной 
 фильтрации со\-стояний сис\-тем с~дискретным временем для аппроксимации решения 
 аналогичной задачи для\linebreak стохастических дифференциальных сис\-тем. 
 
 Статья организована следующим образом. Раздел~2 содержит формальную постановку 
 задачи фильт\-ра\-ции со\-сто\-яний однородного МСП с~конечным множеством со\-сто\-яний 
 по наблюдениям, полученным путем временн$\acute{\mbox{о}}$й дискретизации процессов с~непрерывным 
 временем~--- интегральных преобразований со\-сто\-яния сис\-те\-мы в~присутствии 
 мультипликативных винеровских шумов.\linebreak
  В~разд.~3 пред\-став\-ле\-но решение поставленной 
 задачи фильт\-ра\-ции: пересчет оценок со\-сто\-яний в~момент получения новых 
 дискретизованных наблюдений выполняется в~соответствии с~некоторыми\linebreak 
 рекуррентными интегральными соотношениями, в~то время как между 
 моментами наблюдений оценка корректируется в~соответствии с~прогнозом в~силу 
 сис\-те\-мы уравнений Колмогорова. Вы\-чис\-ли\-тель\-ная слож\-ность 
 упомянутых выше интегральных\linebreak 
 соотношений связана с~тем, что в~расчет принимается воз\-мож\-ность того, что между 
 моментами наблюдений оцениваемое со\-сто\-яние может совершить произвольное чис\-ло 
 скачков. В~разд.~4 пред\-став\-лен более простой алгоритм приближенного вы\-чис\-ле\-ния 
 оценки фильт\-ра\-ции, основанный на ограничении возможного числа учитываемых скачков 
 МСП. Доказана тео\-ре\-ма, опре\-де\-ля\-ющая как\linebreak
  локальную (одношаговую), так и~глобальную 
 (многошаговую) характеристики точ\-ности предложенного при\-бли\-же\-ния~--- 
 $\ell_1$-нор\-мы ошибки аппроксимации. Полученные характеристики являются\linebreak 
 универсальными, т.\,е.\ не асимптотическими по шагу дискретизации, и~зависят от характеристик 
 самого МСП, %\linebreak
  шага временн$\acute{\mbox{о}}$й дискретизации и~чис\-ла
  скачков со\-сто\-яния, учи\-ты\-ва\-емых 
 на шаге. Об\-суж\-де\-ние результатов и~заключительные комментарии пред\-став\-ле\-ны 
 в~разд.~5.
 
 \section{Постановка задачи фильтрации}
 
 На полном вероятностном пространстве с~фильт\-ра\-цией 
 $(\Omega,\mathcal{F},\mathcal{P},\{\mathcal{F}_{t}\}_{t \geqslant 0})$ рассматривается система наблюдений
\begin{equation}
 \left.
 \begin{array}{rl}
 \displaystyle X_t &=X_0 +  \displaystyle
 \int\limits_0^t \Lambda^{\top}X_{s}\,ds + \mu_s\,;  \\[6pt]
 \displaystyle Y_k &=  \displaystyle\int\limits_{t_{k-1}}^{t_k}fX_s\,ds+
 \int\limits_{t_{k-1}}^{t_k} 
 \sum\limits_{n=1}^NX_s^ng_n \,dW_s, \\[6pt]
 &\hspace*{10mm}\{t_k\}_{k \geqslant 0}: \; 0 = t_0 < t_1 < t_2\cdots,
 \end{array}
 \right\}
 \label{eq:obsys_1}
 \end{equation}
 где
  \begin{itemize}
  \item
  $X_t \ebd \mathrm{col}\left(X_t^1,\ldots,X_t^N\right) \hm\in \mathbb{S}^N$~--- 
  ненаблюда\-емое состояние системы, являющееся однородным МСП с~конечным 
  множеством состояний $ \mathbb{S}^N \ebd$\linebreak $\ebd \{e_1,\ldots,e_N\}$ ($\mathbb{S}^N$~--- 
  множество единичных векторов евклидова пространства~$\mathbb{R}^N$), 
  матрицей интенсивностей переходов~$\Lambda$ и~начальным распределением~$\pi$;
  \item
  $\mu_t \ebd \mathrm{col}\left(
  \mu_t^1,\ldots,\mu_t^N\right)\hm\in \mathbb{R}^N$~--- 
  ${\mathcal{F}}_t$-со\-гла\-со\-ван\-ный мартингал;
  \item
  $\{Y_k\}_{k \in \mathbb{N}}:\;  Y_k \ebd \mathrm{col}\left(Y_k^1,\ldots,Y_k^M\right) 
  \hm\in \mathbb{R}^M$~--- последовательность дискретизованных наблюдений, 
  доступных в~известные неслучайные  моменты времени~$\{t_k\}_{k \in \mathbb{N}}$,
в~которых $W_t \ebd$\linebreak $\ebd \mathrm{col}\left(W_t^1,\ldots,W_t^M\right) \hm\in \mathbb{R}^M$
 является ${\mathcal{F}}_t$-со\-гла\-со\-ван\-ным стандартным винеровским процессом, 
 определяющим шумы в~наблюдениях,\linebreak  $f$~--- $(M \times N)$-мер\-ная 
 мат\-ри\-ца плана наблюдений, а~набор мат\-риц~$\{g_n\}_{n=\overline{1,N}}$ 
 характеризует интенсивности шумов в~зависимости от текущего состояния~$X_t$.
  \end{itemize}
  
  Введем также в~рассмотрение неубывающие семейства $\sigma$-ал\-гебр 
  $\mathcal{O}_k \ebd \sigma\{ Y_{\ell}: \; 1 \hm\leqslant \ell \hm\leqslant k\}$ 
  и~$\mathcal{O}_t \ebd  \mathcal{O}_{k(t)}$, где 
  $k(t) \ebd \sum\nolimits_{j \in \mathbb{N}}\mathbf{I}(t-t_{j})$; 
  $\mathcal{O}_0 \ebd \{\varnothing,\; \Omega\}$.
  
   \textit{Задача оптимальной фильтрации состояния~$X$ по наблюдениям~$Y$} 
   заключается в~нахождении \textit{условного математического ожидания} (УМО)
  \begin{equation*}
  \widehat{X}_t \ebd {\sf E}\left\{X_t|\mathcal{O}_{t} \right\}\,.
 % \label{eq:fest_1}
  \end{equation*}
  
  Относительно системы~(\ref{eq:obsys_1})  сделаны следующие предположения:
   \begin{itemize}
 \item[(а)]
 ${\mathcal{F}}_t \equiv {\mathcal{F}}_{t}^X \bigvee 
 {\mathcal{F}}_{t}^W $ для любого $t \hm\geqslant 0$;
 \item[(б)]
 шумы в~наблюдениях равномерно невырожденные, т.\,е.\
  $g_ng_n^{\top} \hm\geqslant \alpha I \hm> 0$ для всех $n\hm=\overline{1,N}$ 
  и~некоторого $\alpha\hm>0$.
% \item
 % Верно неравенство
  %\begin{equation}
  %\min_{1\leqslant k \leqslant N}|\lambda_{kk}| > 0.
  %\label{eq:ineq_0}
  % \end{equation}
 %\item
 %Для любого $t \geqslant 0$ все компоненты вектора $p_t \ebd \me{}{X_t}$ строго %положительны. 
 \end{itemize} 

 \section{Уравнения оптимального фильтра} 
 
 Для получения уравнений оптимального фильт\-ра воспользуемся подходом, 
 применяемым для решения аналогичной задачи в~стохастических сис\-те\-мах 
 наблюдения с~дискретным временем~\cite{BSh_85}. 
 Воспользу\-ем\-ся методом математической индукции. 
 
 При $r=0$ 
 \begin{equation}
 \widehat{X}_{t_0}={\sf E}\{X_0|\mathcal{O}_0\}={\sf E}\{X_0\}=\pi\,.
 \label{eq:in_cond}
 \end{equation} 
 
 Пусть для некоторого $ r \hm\geqslant 0$ известна оценка оптимальной 
 фильтрации~$\widehat{X}_{t_r} \hm= {\sf E}{X_{t_r} |\mathcal{O}_r}$. 
 Определим оценку оптимальной фильтрации~$\widehat{X}_{t} $ для $t\hm \in (t_r,t_{r+1}]$. 
 
 Для произвольного момента $t \hm\in (t_r,t_{r+1})$ в~силу мартингального 
 разложения МСП~$X_t$ и~свойств УМО верна следующая цепочка равенств:
 \begin{multline*}
 \widehat{X}_{t} = {\sf E}\left\{X_t | \mathcal{O}_r\right\}={}\\
 {}=
 {\sf E}\left\{{\cal P}^{\top}(t_r,t)X_{t_r}+
 \int\limits_{t_r}^t{\cal P}^{\top}(t_r,s)\,dM_s\big\vert \mathcal{O}_r\right\} = {}
\end{multline*}

\noindent
   \begin{multline}
 \hspace*{-11.66pt}{}=\mathcal{P}^{\top}(t_r,t)\widehat{X}_{t_r} + {\sf E}\hspace*{-2pt}
 \left\{{\sf E}\hspace*{-2pt}\left\{\int\limits_{t_r}^t\hspace*{-2pt}\mathcal{P}^{\top}(t_r,s)\,dM_s |
 {\mathcal{F}}_{t_r}\right\}\!\big\vert 
 \mathcal{O}_r\!\right\} ={}\hspace*{-4.24124pt}\\
 {}=
  \mathcal{P}^{\top}(t_r,t)\widehat{X}_{t_r}\,,
 \label{eq:bw_obs}
 \end{multline}
 где $\mathcal{P}(s,t)$ $(s \hm\leqslant t)$~--- матрица переходной ве\-ро\-ят\-ности МСП 
 на промежутке $[s,t]$, являющаяся решением сис\-те\-мы дифференциальных 
 уравнений Колмогорова
 \begin{equation*}
 \mathcal{P}'_t(s,t) = \mathcal{P}(s,t) \Lambda, \enskip t > s, \enskip \mathcal{P}(s,s) = I.
 \end{equation*}
 В случае однородного МСП $\mathcal{P}(s,t) \hm= e^{(t-s)\Lambda}$.
 
 Далее необходимо определить совместное распределение $(X_{t_{r+1}},Y_{r+1})$ 
 относительно~$ \mathcal{O}_r$. Из модели наблюдений следует, что 
 распределение~$Y_{r+1}$ относительно 
 $\sigma$-ал\-геб\-ры~$\mathcal{F}^X_{t_{r+1}} \vee \mathcal{O}_r$~---
 гауссовское с~параметрами 
 \begin{align*}
{\sf E}\left\{Y_{r+1}|{\mathcal{F}}^X_{t_{r+1}}\right\}& = f \tau_{r+1}\,; \\[6pt]
 \mathrm{cov} \left(Y_{r+1},Y_{r+1}|{\mathcal{F}}^X_{t_{r+1}}\right) &= 
 \displaystyle\sum\limits_{n=1}^N \tau_{r+1}^n g_ng_n^{\top}\,,
% \label{eq:occup_1}
 \end{align*}
 где $\tau_{r+1} \hm= \tau_{r+1}(X(\omega))=
 \mathrm{col}\left(\tau_{r+1}^1,\ldots,\tau_{r+1}^N\right) \ebd$\linebreak
 $\ebd 
 \int\nolimits_{t_r}^{t_{r+1}}X_s\,ds$~--- случайный вектор, $n$-я 
 компонента которого равна времени пребывания процесса~$X$ в~со\-сто\-янии~$e_n$ 
 на  интервале времени $[t_r, t_{r+1}]$. 
 Обозначим через $\mathcal{D}_{r+1} \ebd \{u=\mathrm{col}\,(u^1,\ldots,u^N):\; 
 u_m \hm\geqslant 0,\; \sum\nolimits_{m=1}^Mu_m\hm= t_{r+1}-t_r\}$ $(M-1)$-мер\-ный 
 симплекс в~пространстве~$\mathbb{R}^M$, являющийся носителем распределения 
 вектора~$\tau_{r+1}$. Пусть $\rho^{k,\ell}_{r+1}(du)$~--- 
 распределение вектора $\tau_{r+1} X_{t_{r+1}}^{\ell}$ при условии $X_{t_r}\hm=e_k$, 
 т.\,е.\ 
 для любого $\mathcal{A} \hm\in \mathcal{B}(\mathbb{R}^M)$ верно тождество:
\begin{multline*}
 \mathbf{P}\left\{\omega: \; X_{t_{r+1}}(\omega)=e_{\ell},\right.\\
 \left. 
 \tau_{r+1}(X(\omega)) \in \mathcal{A}\;|\;X_{t_r}=e_k\right\} \equiv
   \rho^{k,\ell}_{r+1}(\mathcal{A})\,.
\end{multline*}
 
Обозначим через
\begin{multline*}
 \mathcal{N}(y,m,K) \ebd (2\pi)^{-M/2} \mathrm{ det}^{-1/2} K\times{}\\
 {}\times\exp
 \left\{ -\fr{1}{2}\left(y-m\right)^{\top}K^{-1}(y-m)\right\}
\end{multline*}
 $M$-мер\-ную плот\-ность гауссовского распределения с~математическим 
 ожиданием~$m$ и~ковариационной матрицей~$K$.
 
 Из марковского свойства  $\{X_{t_{r}},Y_{r})\}_{r \geqslant 0}$ 
 относительно~${\mathcal{F}}_{t_{r}}$~\cite{ZhSh_95} и~теоремы Фубини следует, что 
 для любого  множества $\mathcal{A} \hm\in \mathcal{B}(\mathbb{R}^M)$ 
 верна следующая цепочка равенств:
 \begin{multline*}
 {\sf E}\left\{X_{t_{r+1}}\mathbf{I}_{\mathcal{A}}
 \left(Y_{r+1}\right)\big|\mathcal{O}_r\right\}={}\\
 {}=
{\sf E}\left\{{\sf E}\left\{X_{t_{r+1}}\mathbf{I}_{\mathcal{A}}
\left(Y_{r+1}\right)\big|
\mathcal{F}^X_{t_{r+1}} \vee \mathcal{O}_r\right\}
 \big|\mathcal{O}_r\right\} = {}
\end{multline*}

\noindent
\begin{multline*}
 %{}=
% {\sf E}\left\{{\sf E}\left\{X_{t_{r+1}}\mathbf{I}_{\mathcal{A}}
% \left(Y_{r+1}\right)\vert X_{t_r}\right\}
% \vert\mathcal{O}_r\right\} = {}\\
% {}=
%{\sf E}\left\{\sum\limits_{k=1}^N {\sf E}\left\{X_{t_{r+1}}\mathbf{I}_{\mathcal{A}}
%\left(Y_{r+1}\right)  \big| X_{t_r}=e_k\right\}X_{t_r}^k
% \big|\mathcal{O}_r\right\} = {}\\ 
% {}=
% \sum\limits_{k=1}^N{\sf E}
% \left\{X_{t_{r+1}}\mathbf{I}_{\mathcal{A}}\left(Y_{r+1}\right)\bigl| X_{t_r}=e_k\right\} 
% \widehat{X}_{t_r}^k ={}\\
% {}=\!
% \sum\limits_{k=1}^N{\sf E}
% \left\{{\sf E}\left\{X_{t_{r+1}}\mathbf{I}_{\mathcal{A}}
% \left(Y_{r+1}\right)\!\bigl| {\mathcal{F}}_{t_{r+1}}\right\}\!\bigl| 
% X_{t_r}\!=e_k\right\} \widehat{X}_{t_r}^k ={}\\
% {}=
% \sum\limits_{k=1}^N {\sf E}\left\{
% \vphantom{\int\limits_A\left(\sum\limits_{p=1}^N\right)}
% X_{t_{r+1}} \times{}\right.\\
% {}\times\int\limits_{\mathcal{A}}  
% \mathcal{N}\left(y,f \tau_{r+1}(X),\sum\limits_{p=1}^N \tau_{r+1}^p(X) g_pg_p^{\top}\right)dy
% \Biggl| X_{t_r}={}\\
%\left. {}=e_k
% \vphantom{\int\limits_A\left(\sum\limits_{p=1}^N\right)}
%\right\} \widehat{X}_{t_r}^k = 
% \sum\limits_{k=1}^N \int\limits_{\mathcal{A}}{\sf E}\left\{ 
% \vphantom{\sum\limits_{p=1}^N}
% X_{t_{r+1}} \times{}\right.\\
% {}\times\mathcal{N}\left(y,f \tau_{r+1}(X),\sum\limits_{p=1}^N \tau_{r+1}^p(X) 
% g_p g_p^{\top}\right)
% \Biggl| X_{t_r}={}\\
%\left. {}=e_k
%\vphantom{\sum\limits^N_{p=1}}
%\right\} \widehat{X}_{t_r}^k\, dy
 %={}\\
 {}=
 \sum\limits_{\ell=1}^N e_{\ell} \int\limits_{\mathcal{A}} 
 \left[ \sum\limits_{k=1}^N 
 \int\limits_{\mathcal{D}_{r+1}} 
 \mathcal{N}\left(y,f u,\sum_{p=1}^N u^p g_pg_p^{\top}\right)\times{}\right.\\
\left. {}\times
 \rho^{k,\ell}_{r+1}(du)\widehat{X}_{t_r}^k
 \vphantom{\int\limits_A\sum\limits_{p=1}^N}
 \right] 
 dy,
 \end{multline*}
 из чего следует, что интегранд в~квадратных скобках в~последнем выражении 
 определяет искомое совместное распределение $(X_{t_{r+1}},Y_{r+1})$ 
 относительно~$ \mathcal{O}_r$. Оценка~$\widehat{X}_{t_{r+1}}$ покомпонентно 
 определяется~\cite{BSh_85} с~помощью обобщенного варианта формулы Байеса:
 \begin{multline}
 \widehat{X}_{t_{r+1}}^j = {}\\
 \hspace*{-1mm}{}=
 \fr{\int\nolimits_{\mathcal{D}_{r+1}}\hspace*{-6mm} 
 \mathcal{N}\left(Y_{r+1},f u,\sum\nolimits_{p=1}^N \hspace*{-2mm}
 u^p g_pg_p^{\top}\!\right)\hspace*{-1mm}
 \sum\nolimits_{k=1}^N \hspace*{-2mm}
 \widehat{X}_{t_r}^k
 \rho^{k,j}_{r+1}(du)
 }
 { \int\nolimits_{\mathcal{D}_{r+1}} \hspace*{-6mm}
 \mathcal{N}\left(Y_{r+1},f v,\sum\nolimits_{q=1}^N \hspace*{-2mm}
 v^q g_qg_q^{\top}\!\right)\hspace*{-1mm}
 \sum\nolimits_{i,\ell=1}^N \hspace*{-2mm}
 \widehat{X}_{t_r}^i
 \rho^{i,\ell}_{r+1}(dv)
  },  \\ 
  j = \overline{1,N}\,.
 \label{eq:filt_1}
 \end{multline}
 Таким образом, доказана следующая
 
 %\smallskip
 
 \noindent
 \textbf{Лемма~1.}
\textit{Если для системы наблюдения}~(\ref{eq:obsys_1}) 
\textit{верны условия~(а) и~(б), то оценка~$\widehat{X}_t$ оптимальной фильтрации 
определяется формулой}~(\ref{eq:in_cond}) 
\textit{при $t\hm=0$, рекуррентным соотношением}~(\ref{eq:filt_1})~---
\textit{в~моменты~$t_{r+1}$ получения наблюдений~$Y_{r+1}$ 
и~формулой}~(\ref{eq:bw_obs})~--- 
\textit{в~промежутках времени между моментами получения наблюдений}.


\smallskip
 

 
 Несмотря на компактную запись~(\ref{eq:filt_1}), их прямая численная реализация 
 ресурсозатратна. Во-пер\-вых, в~(\ref{eq:filt_1}) требуется вычислять 
 распределения мас\-штаб\-но-сдви\-го\-вых смесей многомерных нормальных 
 распределений, что является трудоемкой\linebreak процедурой. Во-вто\-рых, 
 распределения~$\rho^{k,j}_{r+1}$ вре-\linebreak мени пребывания представляют собой 
 сумму\linebreak бесконечного ряда, слагаемые которого вычис\-ляются с~помощью 
 некоторой рекуррентной про\-це\-дуры~\cite{S_00}. В-третьих, 
 распределения~$\rho^{k,j}_{r+1}$ не являются абсолютно непрерывными 
 относительно меры Ле\-бега.
 { %\looseness=1
 
 }
 
 Следующий раздел посвящен численной аппроксимации~(\ref{eq:filt_1}) и~исследованию 
 ее точностных характеристик.
 
 \section{Приближенное вычисление оценки фильтрации}
 
 Без ограничения общности будем считать, что сетка~$\{t_r\}_{r \geqslant 0}$ 
 является равномерной с~шагом~$\Delta$, т.\,е.\ $t_r \hm= r \Delta$ 
 и~$\mathcal{D}_r \hm\equiv \mathcal{D}$.
 Обозначим через~$N_{r+1}$ об-\linebreak\vspace*{-12pt}
 
 \pagebreak
 
 \noindent
 щее число скачков процесса~$X_t$, имевших место 
 на промежутке $(t_r,t_{r+1}]$. Тогда из формулы полной вероятности следует, 
 что~(\ref{eq:filt_1}) представима в~виде:
 \begin{multline}
 \widehat{X}_{t_{r+1}}^j =  \left(
 \int\limits_{\mathcal{D}} 
 \mathcal{N}\left(Y_{r+1},f u,\sum\limits_{p=1}^N u^p g_pg_p^{\top}\right)\times{}\right.\\
\left. {}\times
 \sum\limits_{h=0}^{\infty}\sum\limits_{k=1}^N \widehat{X}_{t_r}^k
 \rho^{k,j,h}_{r+1}(du)
 \right)\Bigg/ \\
 \left(
 \vphantom{\sum\limits_{m=0}^{\infty}
 \sum\limits_{i,\ell=1}^N \widehat{X}_{t_r}^i
 \rho^{i,\ell,m}_{r+1}(dv)}
 \int\limits_{\mathcal{D}} 
 \mathcal{N}\left(Y_{r+1},f v,\sum\limits_{q=1}^N v^q g_qg_q^{\top}\right)\times{}\right.\\
\left.{}\times \sum\limits_{m=0}^{\infty}
 \sum\limits_{i,\ell=1}^N \widehat{X}_{t_r}^i
 \rho^{i,\ell,m}_{r+1}(dv)
 \right)
  \,, \enskip j = \overline{1,N}\,,
  \label{eq:filt_1_1}
 \end{multline}
 где 
 $ \rho^{k,j,h}_{r+1}(du)$~--- распределение вектора 
 $\tau_{r+1}X_{t_{r+1}}^{j}\mathbf{I}_{\{h\}}(N_{r+1})$ при 
 условии $X_{t_r}\hm=e_k$, т.\,е.\ 
 для любого $\mathcal{A} \hm\in \mathcal{B}(\mathbb{R}^M)$ верно тождество
\begin{multline*}
 \mathbf{P}\left\{\omega: \; X_{t_{r+1}}(\omega)=e_{j}, \; N_{r+1} = h,\right.\\ 
\left. \tau_{r+1}(X(\omega)) \in \mathcal{A}\;|\;X_{t_r}=e_k\right\} \equiv
  \rho^{k,j,h}_{r+1}(\mathcal{A}).
\end{multline*}
В качестве аппроксимации оценок можно использовать  
 $\overline{X}_{t_{r+1}}^n \ebd 
 \mathrm{col}\,(\overline{X}_{t_{r+1}}^{n,1},\ldots,\overline{X}_{t_{r+1}}^{n,N})$, 
 полученные из~(\ref{eq:filt_1_1}) путем урезания сумм ряда в~числителе и~знаменателе:
 
 \noindent
 \begin{multline}
 \overline{X}_{t_{r+1}}^{n,j} = 
 \left(
 \int\limits_{\mathcal{D}} 
 \mathcal{N}\left(Y_{r+1},f u,\sum\limits_{p=1}^N u^p g_pg_p^{\top}\right)\times{}\right.\\[-1pt]
\left.{}\times \sum\limits_{h=0}^{n}\sum\limits_{k=1}^N \overline{X}_{t_r}^k
 \rho^{k,j,h}_{r+1}(du)
 \right)\Bigg/ \\[-1pt]
 \left(
 \int\limits_{\mathcal{D}} 
 \mathcal{N}\left(Y_{r+1},f v,\sum\limits_{q=1}^N v^q g_qg_q^{\top}\right)\times{}\right.\\[-1pt]
\left. {}\times
 \sum\limits_{m=0}^{n}
 \sum\limits_{i,\ell=1}^N \overline{X}_{t_r}^i
 \rho^{i,\ell,m}_{r+1}(dv)
  \right)\,, \enskip
   j = \overline{1,N}.
  \label{eq:filt_2}
 \end{multline}
 Ниже по формуле полной вероятности получены интегралы из~(\ref{eq:filt_2}) для 
 $h\hm=0,1,2$:
 
\vspace*{-3pt}

 \noindent
  \begin{multline*}
 \int\limits_{\mathcal{D}}  \mathcal{N}
 \left(Y_{r+1},f u,\sum\limits_{p=1}^N u^p g_pg_p^{\top}\right) 
 \rho^{k,j,0}_{r+1}(du) = {}\\[-1pt]
 {}=
 \delta_{kj}\mathcal{N}\left(Y_{r+1},\Delta f^j,\Delta g_jg_j^{\top}\right)
 e^{\lambda_{jj}\Delta};
 %\label{eq:h0}
\\[-1pt]
 \int\limits_{\mathcal{D}}  \mathcal{N}\left(
 Y_{r+1},f u,\sum\limits_{p=1}^N u^p g_pg_p^{\top}\right) 
 \rho^{k,j,1}_{r+1}(du) ={} 
 \end{multline*}
 
 \noindent
 \begin{multline}
 \hspace*{-6.7pt}{}=\left(1-\delta_{kj}\right)\lambda_{kj}e^{\lambda_{jj}\Delta}
\! \int\limits_0^{\Delta}\!
 e^{(\lambda_{kk}-\lambda_{jj})u^k}
 \mathcal{N}\left(Y_{r+1},u^kf^k +{}\right.\hspace*{-0.28818pt}\\[-1pt]
\hspace*{-3mm}\left. {}+ \left(\Delta - u^k\right)f^j, u^k g_kg_k^{\top}+
 \left(\Delta-u^k\right)g_jg_j^{\top}\right)\,du^k;
 \label{eq:h1}
 \end{multline}
 
 \vspace*{-12pt}
 
 \noindent
 \begin{multline}
 \int\limits_D \mathcal{N}\left( 
Y_{r+1},f u,\sum\limits_{p=1}^N u^p g_pg_p^{\top}\right)du ={}\\[-1pt]
{}=
\sum\limits_{\substack{{\ell:\ell \neq k,}\\ {\ell \neq j}}}
 \lambda_{k\ell}\lambda_{\ell j} e^{\lambda_{jj}\Delta}\times {}\\[-1pt] 
 {}\times
 \int\limits_0^{\Delta} \int\limits_0^{\Delta-u^k} \!
e^{(\lambda_{kk}-\lambda_{\ell\ell})u^k+(\lambda_{\ell\ell}-
 \lambda_{jj})u^{\ell}}\times{} \\[-1pt] 
{}  \times
 \mathcal{N}\left(Y_{r+1},u^k f^k+u^{\ell}f^{\ell}+\left(
 \Delta-u^k-u^{\ell} \right)f^j,\right.\\[-1pt]
 \hspace*{-1mm}\left.
 u^k g_kg_k^{\top}+u^{\ell}g_{\ell}g_{\ell}^{\top}+\left(
 \Delta-u^k-u^{\ell} \right)
 g_jg_j^{\top}
 \right) du^{\ell}du^{k}, \!\!
  \label{eq:h2}
 \end{multline} 
 
\vspace*{-2pt}
 
 \noindent
  где  $\delta_{ij}$~--- символ Кронекера. Интегралы для $h\hm>2$ также могут 
  быть получены в~явном виде, однако их сложность резко возрастает.
 

   Так как система~(\ref{eq:obsys_1}) является автономной, то в~качестве локальной 
   характеристики бли\-зости~$\{\overline{X}_{t_r}\}$ 
   к~$\{\widehat{X}_{t_r}\}$ может быть выбрана величина
   
\noindent
 \begin{multline*}
 \overline{\sigma}(\pi) \ebd {\sf E}\left\{
 \|\widehat{X}_{t_{1}}(\pi, Y_{1}) - \overline{X}_{t_{1}}
 \left(\pi,Y_{1}\right)\|_{1}\right\} = {}\\
 {}=
 \sum\limits_{j=1}^N{\sf E}
 \left\{\left\vert \widehat{X}^j_{t_{1}}\left(\pi, Y_{1}\right) - \overline{X}^{n,j}_{t_{1}}
 \left(\pi,Y_{1}\right)\right\vert\right\}.
 %\label{eq:prec_1}
 \end{multline*}
 При этом начальное распределение $\pi \hm\in \mathcal{D}_1 \ebd $\linebreak $\ebd
 \{\mathrm{col}\,(\pi^1,\ldots,\pi^N):\;\pi^j > 0$, 
 $\sum\nolimits_{j=1}^N\pi^j\hm=1\}$ является начальным условием применения 
 одного шага рекурсии~(\ref{eq:filt_1}) или~(\ref{eq:filt_2}) для вычисления 
 оценки~$\widehat{X}_{t_{1}}$
   или~$\overline{X}_{t_{1}}$ соответственно. Фактически, 
 характеристика~$\overline{\sigma}(\pi)$ определяет, насколько сильно 
 рекурсивные схемы~(\ref{eq:filt_1}) и~(\ref{eq:filt_2}) разойдутся за 
 один шаг, стартуя из общей точки~$\pi$.
 
 Рекуррентные схемы~(\ref{eq:filt_1}) и~(\ref{eq:filt_2}), примененные~$r$~раз, 
 позволяют вычислить оценки~$\widehat{X}_{t_r}$ и~$\overline{X}_{t_r}$ 
 в~точке~$t_r$. В~качестве характеристики точности глобальной аппроксимации в~этом 
 случае естественно рассмотреть величину
 
 \vspace*{-2pt}
 
 \noindent
 \begin{equation*}
 \overline{\Sigma}_{t_r}(\pi) \ebd {\sf E}
 \left\{\|\widehat{X}_{t_{r}} - \overline{X}_{t_{r}}\|_{1}\right\} = 
 \!\sum\limits_{j=1}^N\!{\sf E}
 \left\{\left\vert \widehat{X}^j_{t_{r}} - 
 \overline{X}^{n,j}_{t_{r}}\right\vert \right\}.
% \label{eq:prec_2}
 \end{equation*}
 
 Следующее утверждение определяет оценки локальной и~глобальной 
 точности схемы аппроксимации~(\ref{eq:filt_2}).
 
 %\smallskip
 
 \noindent
 \textbf{Теорема~1.}\
\textit{Выполняются неравенства} 

%\vspace*{-2pt}

\noindent
 \begin{equation}
 \sup_{\pi \in \mathcal{D}_1} \overline{\sigma}(\pi) 
 \leqslant 2 \fr{(\overline{\lambda}\Delta)^{n+1}}{(n+1)!}\,;
 \label{eq:prec_loc}
\end{equation}

\noindent
\begin{align}
  \sup\limits_{\pi \in \mathcal{D}_1} \overline{\Sigma}_{t_r}(\pi)
   &\leqslant 2r \fr{(\overline{\lambda}\Delta)^{n+1}}{(n+1)!} +{}\notag\\[-0.5pt]
   &\hspace*{-20mm}{}+
  r(r-1)\left(
  \fr{(\overline{\lambda}\Delta)^{n+1}}{(n+1)!}
  \right)^2
  \left(
  1-\fr{(\overline{\lambda}\Delta)^{n+1}}{(n+1)!}
  \right)^{r-2},
 \label{eq:prec_glob}
 \end{align}
 
 \vspace*{-2pt}
 
 \noindent
 \textit{где} $\overline{\lambda} \ebd \max_{1 \leqslant j \leqslant N}|\lambda_{jj}|$.


%\smallskip

 Доказательство теоремы~1 приведено в~приложении.
 
 Данное утверждение представляет полезные оценки точности. Во-пер\-вых, 
 они являются равномерными по начальному распределению $\pi \hm\in \mathcal{D}_1$. 
 Во-вто\-рых, оценки носят универсальный, а~не асимптотический характер. Это 
 существенно в~практических задачах оценивания по дискретизованным 
 наблюдениям с~физическими или алгоритмическими ограничениями на шаг 
 по времени. Например, в~случае наблюдаемого процесса восстановления в~силу 
 центральной предельной теоремы для процессов восстановления~\cite{B_80} его
  приращения можно рассматривать как гауссовские случайные величины. 
  Однако данная аппроксимация обладает удовлетворительной точностью 
  только в~случае, когда шаг дискретизации по времени достаточно большой. 
 %
 В-третьих, неравенство~(\ref{eq:prec_glob}) позволяет получить порядок 
 аппроксимации при $\Delta \hm\to 0$. Зафиксируем момент времени $t\hm=T$ и~рассмотрим 
 характеристику $\sup\nolimits_{\pi \in \mathcal{D}_1} 
 \overline{\Sigma}_{T}(\pi)$ при $r\hm={T}/{\Delta}$ и~$\Delta \hm\to 0$. 
 Как только~$\Delta$ становится настолько мало, что 
 $\max\left({(\overline{\lambda}\Delta)^{n+1}}/{(n+1)!}, 
 \Delta ({T\lambda^{n+1}}/{(n+1)!})\right)\hm< 1$, из~(\ref{eq:prec_glob}) 
 следует неравенство
  %\begin{equation}
  $\sup\nolimits_{\pi \in \mathcal{D}_1} \overline{\Sigma}_{T}(\pi) 
  \hm\leqslant  ({3\overline{\lambda}^{n+1}}/{(n+1)!}) T\Delta^n.$
 %\label{eq:prec_asympt}
 %\end{equation}
 Это значит, что с~ростом времени~$T$ 
 ошибка аппроксимации копится пропорционально~$T$ и~при этом порядок точности 
 по~$\Delta$ равен~$n$.
 
 %\vspace*{-7pt}
 
  \section{Заключение}
  
  \vspace*{-4pt}
 
  В работе решена задача оценивания состояния однородного МСП по 
  дискретизованным наблюдениям. Получено аналитическое решение и~его 
  чис\-лен\-ные аппроксимации. Локальные и~глобальные показатели точ\-ности этих 
  приближений в~статье так\-же пред\-став\-ле\-ны. Примечательно, что  част\-ный случай 
  аппроксимаций~(\ref{eq:filt_2}) при $n\hm=0$ и~$\Lambda\hm=0$ был ранее 
  пред\-став\-лен в~\cite{B_17_1,B_17_2} для решения задачи байесовской классификации 
  случайного вектора по непрерывным наблюдениям с~мультипликативными шумами. 
 % 
Алгоритм оптимальной фильт\-ра\-ции и~его субоптимальные версии могут 
рас\-смат\-ри\-вать\-ся в~качестве основы чис\-лен\-ной реализации обобщения фильт\-ра 
Вонэма для сис\-тем с~мультипликативными шумами в~наблюдениях. 
Однако для их непосредственного использования необходимо решить 
следующие проб\-ле\-мы. Во-пер\-вых, в~(\ref{eq:h1}) и~(\ref{eq:h2}) присутствуют
 многомерные интегралы. Следует выяснить, какую результирующую погрешность 
 будут вносить ошибки их вы\-чис\-ле\-ния. Во-вто\-рых, представляется интересным 
 определить характеристики точ\-ности оптимальной фильт\-ра\-ции по дискретизованным 
 наблюдениям по отношению к~оптимальной фильт\-ра\-ции по непрерывным наблюдениям: 
 каков порядок точ\-ности по шагу временной дискретизации~$\Delta$? Для случая 
 вы\-чис\-ле\-ния классического фильт\-ра Вонэма с~по\-мощью алгоритма Эй\-ле\-ра--Ма\-ру\-ямы 
 подобный результат известен: порядок глобальной ошибки равен~${1}/{2}$. 
 Перечисленные задачи являются предметом дальнейших исследований.
 
 
  \vspace*{-10pt}
 
{\small
\subsection*{\raggedleft Приложение} 

\vspace*{-2pt}


\noindent
Д\,о\,к\,а\,з\,а\,т\,е\,л\,ь\,с\,т\,в\,о\ \ теоремы~1.\ \ Введем следующие 
обозначения для случайных величин и~мат\-риц, составленных из них:
\begin{align*}
\xi^{ji}(\ell)&\ebd 
\sum\limits_{h=0}^n \int\limits_{\mathcal{D}} 
 \mathcal{N}\left(Y_{\ell},f u,\sum\limits_{p=1}^N u^p g_pg_p^{\top}\right)
 \rho^{j,i,h}_{1}(du)\,; \\
  \theta^{ji}(\ell)&\ebd 
\sum\limits_{h=n+1}^{\infty} \int\limits_{\mathcal{D}} 
 \mathcal{N}\left(Y_{\ell},f u,\sum\limits_{p=1}^N u^p g_pg_p^{\top}\right)
 \rho^{j,i,h}_{1}(du)\,;
\\
 \xi(\ell)&\ebd \|\xi^{ji}(\ell)\|_{j,i=\overline{1,N}}\,,\quad 
 \Xi(r) \ebd \xi(r) \xi(r-1)\cdots \xi(1)\,;
 \\
 \theta(\ell)&\ebd \|\theta^{ji}(\ell)\|_{j,i=\overline{1,N}}\,, \quad 
 \Theta(r) \ebd \theta(r) \theta(r-1)\cdots \theta(1)\,.
%\label{eq:not_1}
\end{align*}
 
 Рекуррентные формулы~(\ref{eq:filt_1}) и~(\ref{eq:filt_2}) можно записать в~явной 
 форме
 
 
\noindent
\begin{align*}
 \widehat{X}_{t_r}& = \left( \mathbf{1}\left(\Xi(r) + 
 \Theta(r)\right)\pi\right)^{-1} \left(\Xi(r) + \Theta(r)\right)\pi\,;
\\
 \overline{X}_{t_r} &= \left( \mathbf{1}\Xi(r)\pi\right)^{-1} \Xi(r) \pi,
\end{align*}

\vspace*{-2pt}

\noindent
где $\mathbf{1} \ebd (1,\ldots,1)$~--- век\-тор-стро\-ка 
подходящей раз\-мер\-ности, составленная из единиц.

%Далее для краткости записи зависимость от~$r$ в~обозначениях~$\Xi(r)$ 
%и~$\Theta(r)$ будет опущена. 
Верна следующая цепочка неравенств:

 \vspace*{-3pt}

\noindent
\begin{multline}
\overline{\Sigma}_{t_r}(\pi)=%
%\me{}{\left\| 
%\widehat{X}_{t_r}(\pi, Y_1,\ldots,Y_r) - \overline{X}_{t_r}(\pi, Y_1,\ldots,Y_r)
%\right\|_1} =\\=
{\sf E}\left\{\left\| 
\fr{1}{\mathbf{1}\left(\Xi(r) + \Theta(r)\right)\pi} \left(\Xi(r) +{}\right.\right.\right.\\[-1pt]
\left.\left.\left.{}+ \Theta(r)\right)\pi
- \fr{1}{\mathbf{1}\Xi(r)\pi}\,\Xi(r) \pi
\right\|_1\right\} ={} \\[-1pt]
{}=
{\sf E}\left\{\fr{1}{\mathbf{1}\left(\Xi(r) + \Theta(r)\right)\pi \mathbf{1}\Xi(r)\pi}
\left\|
 \mathbf{1}\Xi(r) \pi \Theta(r)\pi -{}\right.\right.\\[-1pt]
\left.\left. {}- \mathbf{1}\Theta(r)\pi \Xi(r) \pi
 \right\|_1
 \vphantom{\fr{1}{\mathbf{1}\left(\Xi(r) + \Theta(r)\right)\pi \mathbf{1}\Xi(r)\pi}}
\right\} \leqslant {}\\[-1pt]
{}\leqslant 
{\sf E}\left\{\fr{1}{\mathbf{1}\left(\Xi(r) + \Theta(r)\right)\pi \mathbf{1}\Xi(r)\pi}
\left(
\mathbf{1}\Xi(r)\pi \| \Theta(r)\pi \|_1 +{}\right.\right.\\[-1pt]
\left.\left.{}+ \mathbf{1}\Theta(r)\pi 
\|
\Xi(r) \pi
\|_1
\right)
 \vphantom{\fr{1}{\mathbf{1}\left(\Xi(r) + \Theta(r)\right)\pi \mathbf{1}\Xi(r)\pi}}
\right\} ={}\\[-1pt]
{}=
2\,{\sf E}\left\{\fr{1}{\mathbf{1}\left(\Xi(r) + \Theta(r)\right)\pi}\mathbf{1}\Theta(r)\pi 
\right\}.
\label{eq:ineq_1}
\end{multline}

 
 \noindent
 Рассмотрим случайные события $a_{\ell} \ebd \{\omega \in \Omega: 
 N_{\ell}(\omega) \hm\leqslant n\}$, $\ell \hm= \overline{1,r}$, и~$A_r \ebd \{
 \omega\hm \in \Omega: \max_{1 \leqslant {\ell} \leqslant r}N_{\ell}(\omega) 
 \hm\leqslant n
 \}\hm=\prod\nolimits_{\ell=1}^r a_{\ell}$ и~оценку 
 $
 \widetilde{X}_{t_r}(\pi, Y_1,\ldots,Y_r)\ebd$\linebreak $\ebd
 {\sf E}\left\{X_{t_r}(\omega)\mathbf{I}_{A_r}(\omega)|\mathcal{O}_r\right\}.
 $
 Используя введенные выше обозначе\-ния и~абстрактный вариант формулы Байеса, 
 получаем, что
 
 \noindent
\begin{align}
\widetilde{X}_{t_r}& = \fr{1}{{\mathbf{1}\left(\Xi(r) + 
 \Theta(r)\right)\pi}}\,\Xi(r)\pi\,;\notag
 \\
\widehat{X}_{t_r} - \widetilde{X}_{t_r} &=
{\sf E}\left\{X_{t_r}(\omega)\mathbf{I}_{\overline{A}_r}(\omega)|\mathcal{O}_r\right\} ={}\notag\\[-1pt]
&\hspace*{17mm}{}= 
\fr{1}{\mathbf{1}\left(\Xi(r) + \Theta(r)\right)\pi}\Theta(r)\pi\,. 
\label{eq:eq_2}
 \end{align}
 Из (\ref{eq:ineq_1}) и~(\ref{eq:eq_2}) для $r\hm=1$ следует, что
 
 \vspace*{-4pt}
 
 \noindent
 \begin{multline}
 \overline{\sigma}(\pi) \leqslant 2\,{\sf E}
 \left\{\|{\sf E}\left\{X_{t_1}(\omega)\mathbf{I}_{\overline{a}_1}(\omega)|\mathcal{O}_1
 \right\}\|_1
 \right\} ={}\\[-1.5pt]
 {}=
 2\,{\sf E}\left\{\sum\limits_{n=1}^N {\sf E}
 \left\{X^n_{t_1}(\omega)\mathbf{I}_{\overline{a}_1}
 (\omega)|\mathcal{O}_1\right\}\right\} ={} \\[-2pt] 
 {}=
  2\,{\sf E}\left\{{\sf E}\left\{\mathbf{I}_{\overline{a}_1}(\omega)|\mathcal{O}_1
  \right\}\right\} =
   2 \mathbf{P}\left\{\overline{a}_1(\omega)\right\}.
\label{eq:ineq_3}
\end{multline}

 \vspace*{-2pt}
 
 \noindent
 Процесс $N^X_t$ общего числа скачков состояния~$X_t$ является считающим, и~его
  квадратическая характеристика равна 
  
\vspace*{-2pt}
  
  \noindent
 $$
 \langle N^X, N^X\rangle_t = - \int\limits_0^t \sum\limits_{n=1}^N \lambda_{nn} X_s^n\,ds\,,
 $$
 поэтому искомая вероятность ограничена сверху:
 $$ 
 \mathbf{P}\left\{\overline{a}_1(\omega)\right\} \leqslant 
 e^{-\overline{\lambda}\Delta}\sum\limits_{k=n+1}^{\infty} 
 \fr{(\overline{\lambda}\Delta)^{k}}{k!} <
 \fr{(\overline{\lambda}\Delta)^{n+1}}{(n+1)!}.
 $$
 
  \vspace*{-2pt}
  
  \noindent
 Из последнего неравенства и~(\ref{eq:ineq_3}) следует, что  для любого 
 начального распределения~$\pi$ выполняется неравенство $\overline{\sigma}(\pi)  
 \hm< 2({(\overline{\lambda}\Delta)^{n+1}}/{(n+1)!})$, т.\,е.\ 
 локальная оценка~(\ref{eq:prec_loc}) верна.
 
 С помощью марковского свойства пары $(X_t, N^X_t)$ и~последнего 
 неравенства можно оценить сверху вероятность 
 $\mathbf{P}\left\{\overline{A}_r(\omega)\right\}$:
 
  \vspace*{-2pt}
 
 \noindent
 \begin{multline*}
 \mathbf{P}\left\{\overline{A}_r(\omega)\right\} \leqslant 1 - \left(
 1- \fr{(\overline{\lambda}\Delta)^{n+1}}{(n+1)!}
 \right)^r \leqslant r \fr{(\overline{\lambda}\Delta)^{n+1}}{(n+1)!} + {}\\[-1pt]
 {}+\left|
 \sum\limits_{k=2}^r C_r^k \left(-\fr{(\overline{\lambda}\Delta)^{n+1}}{(n+1)!}
 \right)^k
 \right| \leqslant
 r \fr{(\overline{\lambda}\Delta)^{n+1}}{(n+1)!} +{}\\[-1pt]
 {}+\fr{r(r-1)}{2}
 \left(
 \fr{(\overline{\lambda}\Delta)^{n+1}}{(n+1)!}
 \right)^2
 \left(
 1-\fr{(\overline{\lambda}\Delta)^{n+1}}{(n+1)!}
 \right)^{r-2},
 \end{multline*} 
 из чего следует истинность глобальной оценки~(\ref{eq:prec_glob}).
Теорема~1 доказана.

}

%\vspace*{-12pt}

{\small\frenchspacing
 {%\baselineskip=10.8pt
 \addcontentsline{toc}{section}{References}
 \begin{thebibliography}{99}

\bibitem{Won_65}
\Au{Wonham W.} 
Some applications of stochastic differential equations to optimal
  nonlinear filtering~//
SIAM~J.~Control, 1965. Vol.~2. P.~347--369. 

\bibitem{KP_92}
\Au{Kloeden P., Platen E.} Numerical solution of stochastic
differential equations.~--- Berlin: Springer, 1992.~636~p.

\bibitem{YZL_04}
\Au{Yin G., Zhang Q., Liu Y.} 
Discrete-time approximation of Wonham filters~//
J.~Control Theory Applications, 2004. Iss.~2. P.~1--10.

\bibitem{PR_10}
\Au{Platen E., Rendek R.}
Quasi-exact approximation of hidden Markov chain filters~//
Communicat.~Stoch.~Analys., 2010. Vol.~4. Iss.~1. P.~129--142.

\bibitem{B_18}
\Au{Борисов А.} Фильтрация Вонэма по наблюдениям с~мультипликативными шумами~// 
Автоматика и~телемеханика, 2018.
№~1. C.~52--65. 
 
  \bibitem{BSh_85} %6
\Au{Бертсекас Д., Шрив С.} Стохастическое оптимальное управление. 
Случай дискретного времени~/ Пер. с~англ.~--- М.: Наука, 1985.~280~c.
(\Au{Betsekas~D.\,P., Shreve~S.\,E.} Stochastic optimal control:
The discrete-time case.~--- Orlando, FL, USA:
Academic Press Inc., 1978. 323~p.)

  \bibitem{ZhSh_95} %7
\Au{Жакод Ж., Ширяев А.} Предельные теоремы для случайных процессов,~I.~/
Пер. с~англ.~--- 
М.: Физматлит, 1995.~544~c.
(\Au{Jacod~J., Shiryaev~A.} Limit theorems for stochastic processes.~---
Berlin: Springer, 2003. 664~p.)

\bibitem{S_00}
\Au{Sericola B.} Occupation times in Markov processes~//
Commun. Stat. Stochastic Models, 2000. Vol.~16. Iss.~5. P.~479--510. 

  \bibitem{B_80}
\Au{Боровков А.} Асимптотические методы в~тео\-рии массового обслуживания.~--- 
М.: Физматлит, 1995.~384~c.

  \bibitem{B_17_1}
\Au{Борисов А.} Классификация по непрерывным наблюдениям с~мультипликативными шумами.~I. 
Формулы байесовской оценки~// Информатика и~её применения, 2017. Т.~11. Вып.~1. C.~11--19.
doi: 10.14357/19922264170102.

  \bibitem{B_17_2}
\Au{Борисов А.} Классификация по непрерывным наблюдениям с~мультипликативными 
шумами.~II. Алгоритм численной реализации оценки~// Информатика и~её 
применения, 2017. Т.~11. Вып.~2. C.~33--41.
doi: 10.14357/19922264170204.

 \end{thebibliography}

 }
 }

\end{multicols}

\vspace*{-4pt}

\hfill{\small\textit{Поступила в~редакцию 10.07.18}}

\vspace*{6pt}

%\pagebreak

%\newpage

%\vspace*{-28pt}

\hrule

\vspace*{2pt}

\hrule

%\vspace*{-2pt}

\def\tit{FILTERING OF~MARKOV JUMP PROCESSES\\ BY~DISCRETIZED OBSERVATIONS}

\def\titkol{Filtering of Markov jump processes by discretized observations}

\def\aut{A.\,V.~Borisov}

\def\autkol{A.\,V.~Borisov}

\titel{\tit}{\aut}{\autkol}{\titkol}

\vspace*{-11pt}


\noindent
Institute of Informatics Problems, Federal Research Center ``Computer Science 
and Control'' of the Russian Academy of Sciences, 44-2~Vavilov Str., Moscow 
119333, Russian Federation


\def\leftfootline{\small{\textbf{\thepage}
\hfill INFORMATIKA I EE PRIMENENIYA~--- INFORMATICS AND
APPLICATIONS\ \ \ 2018\ \ \ volume~12\ \ \ issue\ 3}
}%
 \def\rightfootline{\small{INFORMATIKA I EE PRIMENENIYA~---
INFORMATICS AND APPLICATIONS\ \ \ 2018\ \ \ volume~12\ \ \ issue\ 3
\hfill \textbf{\thepage}}}

\vspace*{6pt}



\Abste{The article is devoted to a~solution of the optimal filtering problem 
of a~homogenous Markov
jump process state. The available observations represent 
time increments of the integral transformations of the Markov\linebreak\vspace*{-12pt}}

\Abstend{state corrupted by 
Wiener processes. The noise intensity is also state-dependent. At the instant of 
the consecutive
observation obtaining, the optimal estimate is calculated recursively 
as a~function of previous estimate and the new observation, meanwhile between 
observations the filtering estimate is a simple forecast by virtue of the Kolmogorov 
differential system. The recursion is rather expensive because of  need to calculate 
the integrals, which are the location-scale mixtures of Gaussians. The mixing 
distributions represent the occupation of the state in each of possible values 
during the mid-observation intervals. The paper contains numerically cheaper 
approximations, based on the restriction of the state transitions number between 
the observations. Both the local and global characteristics of approximation 
accuracy are obtained as functions of the dynamics parameters, mid-observation 
interval length, and upper bound of transitions number.}

\KWE{Markov jump process; optimal filtering; multiplicative observation noises; 
stochastic differential equation; numerical approximation}




\DOI{10.14357/19922264180316}

%\vspace*{-14pt}

\Ack
\noindent
The work was supported in part by the Russian Foundation
for Basic Research (Project No.\,16-07-00677).



%\vspace*{6pt}

  \begin{multicols}{2}

\renewcommand{\bibname}{\protect\rmfamily References}
%\renewcommand{\bibname}{\large\protect\rm References}

{\small\frenchspacing
 {%\baselineskip=10.8pt
 \addcontentsline{toc}{section}{References}
 \begin{thebibliography}{99}
\bibitem{Won_65-1}
\Aue{Wonham, W.} 1965.
Some applications of stochastic differential equations to optimal
  nonlinear filtering.
\textit{SIAM~J.~Control} 2:347--369. 

\bibitem{KP_92-1}
\Aue{Kloeden,~P., and E.~Platen.} 1992. \textit{Numerical solution of stochastic
differential equations.} Berlin: Springer. 636~p.

\bibitem{YZL_04-1}
\Aue{Yin,~G., Q.~Zhang, and Y.~Liu.} 2004.
Discrete-time approximation of Wonham filters.
\textit{J.~Control Theory Applications} 2:1--10.

\bibitem{PR_10-1}
\Aue{Platen, E., and R.~Rendek.} 2010.
Quasi-exact approximation of hidden Markov chain filters.
\textit{Communicat. Stoch. Analys.} 4(1):129--142.

\bibitem{B_18-1}
\Aue{Borisov, A.} 2018. Wonham filtering by observations
with multiplicative noises. \textit{Automat.~Rem.~Contr.} 79(1):39--50.  
doi: 10.1134/ S0005117918010046.
 
  \bibitem{BSh_85-1}
\Aue{Bertsekas, D., and S.~Shreve.} 1996.
\textit{Stochastic optimal control: The discrete-time case}.
Nashua, NH: Athena Scientific. 330~p.
  
  \bibitem{ZhSh_95-1}
  \Aue{Jacod,~J., and A.~Shiryaev.} 2003.
\textit{Limit theorems for stochastic processes.}
Berlin: Springer. 664~p.

\bibitem{S_00-1}
\Aue{Sericola, B.}
2000. Occupation times in Markov processes.
\textit{Commun. Stat.} 16(5):479--510. 

  \bibitem{B_80-1}
\Aue{Borovkov, A.} 1984.
 \textit{Asymptotic methods in queueing theory}. 
 Hoboken, NJ: Wiley-Blackwell.~304~p.

  \bibitem{B_17_1-1}
  \Aue{Borisov, A.} 2017. 
  Klassifikatsiya po ne\-pre\-ryv\-nym nablyu\-de\-miyam s~mul'tiplikativnymi shumami. I. 
  Formuly bayesov\-skoy otsenki [Classification by continuous-time observations
in multiplicative noise. I.~Formulae for Bayesian 
estimate]. \textit{Informatika i~ee Primeneniya~--- Inform.~Appl.}
11(1):11--19. doi: 10.14357/19922264170102.

  \bibitem{B_17_2-1}
\Aue{Borisov, A.} 2017. Klassifikatsiya po nepreryvnym nablyudemiyam 
s~mul'tiplikativnymi summami. II.~Formuly bayesovskoy otsenki 
[Classification by continuous-time observations
in multiplicative noise. II.~Numerical algorithm].
\textit{Informatika i~ee Primeneniya~--- Inform.~Appl.}
11(2):33--41. doi: 10.14357/19922264170204.

\end{thebibliography}

 }
 }

\end{multicols}

\vspace*{-6pt}

\hfill{\small\textit{Received July 10, 2018}}

%\pagebreak

%\vspace*{-18pt}

\Contrl

\noindent
\textbf{Borisov Andrey V.} (b.\ 1965)~--- 
Doctor of Science in physics and mathematics, principal scientist, Institute of
Informatics Problems, Federal Research Center ``Computer Science and Control''
 of the Russian Academy of
Sciences, 44-2 Vavilov Str., Moscow 119333, Russian Federation; 
\mbox{aborisov@frccsc.ru}
\label{end\stat}

\renewcommand{\bibname}{\protect\rm Литература}        %2
\def\stat{shestakov+vor}

\def\tit{АСИМПТОТИЧЕСКАЯ НОРМАЛЬНОСТЬ И~СИЛЬНАЯ СОСТОЯТЕЛЬНОСТЬ ОЦЕНКИ РИСКА ПРИ~ИСПОЛЬЗОВАНИИ FDR-ПОРОГА В УСЛОВИЯХ СЛАБОЙ ЗАВИСИМОСТИ}

\def\titkol{Асимптотическая нормальность и~сильная состоятельность оценки риска при~использовании FDR-порога} % в~условиях слабой зависимости}

\def\aut{М.\,О.~Воронцов$^1$, О.\,В.~Шестаков$^2$}

\def\autkol{М.\,О.~Воронцов, О.\,В.~Шестаков}

\titel{\tit}{\aut}{\autkol}{\titkol}

\index{Воронцов М.\,О.}
\index{Шестаков О.\,В.}
\index{Vorontsov M.\,O.}
\index{Shestakov O.\,V.}


%{\renewcommand{\thefootnote}{\fnsymbol{footnote}} \footnotetext[1]
%{Работа 
%выполнена при поддержке Программы развития МГУ, проект №\,23-Ш03-03. При анализе 
%данных использовалась инфраструктура Центра коллективного пользования 
%<<Высокопроизводительные вычисления и~большие данные>> 
%(ЦКП <<Информатика>>) ФИЦ ИУ РАН (г.~Москва)}}


\renewcommand{\thefootnote}{\arabic{footnote}}
\footnotetext[1]{Московский государственный университет 
имени~М.\,В.~Ломоносова, факультет вычислительной математики и~кибернетики;  
Московский центр фундаментальной и~прикладной математики, \mbox{m.vtsov@mail.ru}}
\footnotetext[2]{Московский государственный университет 
имени М.\,В.~Ломоносова, факультет вычислительной математики и~кибернетики; 
Федеральный исследовательский центр <<Информатика и~управление>> Российской 
академии наук; Московский центр фундаментальной и~прикладной математики, 
\mbox{oshestakov@cs.msu.ru}}


\vspace*{-12pt}





\Abst{Рассматривается подход к~решению задачи удаления шума в~большом массиве 
разреженных данных, основанный на методе контроля средней доли ложных отклонений 
гипотез (False Discovery Rate, FDR). Данный подход эквивалентен процедурам 
пороговой обработки, обнуляющим компоненты массива, значения которых не 
превосходят некоторого заданного порога.  Наблюдения в~модели считаются слабо 
зависимыми. Для контроля степени зависимости используются ограничения на 
коэффициент сильного перемешивания и~максимальный коэффициент корреляции. 
В~качестве меры эффективности рассматриваемого подхода используется 
среднеквадратичный риск. Вычислить значение риска можно только на тестовых 
данных, поэтому в~работе рассматривается его статистическая оценка и~исследуются 
ее свойства. Показана асимптотическая нормальность и~сильная состоятельность 
оценки риска при использовании FDR-по\-ро\-га в~условиях слабой зависимости в~данных.}

\KW{пороговая обработка; множественная проверка гипотез; 
оценка риска}

\DOI{10.14357/19922264240309}{ZOQVTO}
  
%\vspace*{-6pt}


\vskip 10pt plus 9pt minus 6pt

\thispagestyle{headings}

\begin{multicols}{2}

\label{st\stat}



\section{Введение}

Во многих прикладных областях возникает задача обработки больших массивов 
зашумленных данных. Примерами служат задачи обработки изоб\-ра\-же\-ний с~высоким 
разрешением~\cite{FDRImage}, задачи множественной проверки гипотез, возникающие 
в~\mbox{исследованиях} в~об\-ласти генетики~\cite{MultipleTesting}, и~другие проб\-ле\-мы. 
В~связи с~этим рас\-смот\-рим модель
$$
x_i = \mu_i + z_i, \enskip i=\overline{1,n}\,,
$$
где $\mu_i\in\mathbb{R}$~--- <<полезные>> данные; $z_i \sim N(0,\sigma^2)$~--- 
шум. Задача заключается в~нахождении оценки неизвестного вектора $\mu \hm= 
(\mu_1,\ldots,\mu_n)$ как функции вектора $x \hm= (x_1,\ldots,x_n)$ и~может 
рассматриваться как задача множественной проверки гипотез о~равенстве нулю 
компонент вектора~$\mu$~\cite{AdaptingFDR}. При этом обычно предполагается, что 
вектор~$\mu$ имеет в~определенном смысле <<разреженную>> структуру, т.\,е.\ для 
<<полезных>> данных используется <<экономное>> представление.



В работе~\cite{AdaptingFDR} для решения рассматриваемой задачи в~условиях 
независимости компонент вектора~$x$ и~разреженности вектора~$\mu$ была 
предложена процедура построения оценки~$\hat{\mu}_F$ вектора~$\mu$, основанная 
на методе контроля средней доли ложных отклонений (FDR) 
гипотез при помощи алгоритма Бен\-жа\-ми\-ни--Хох\-бер\-га,
и~было проведено исследование асимптотики ее среднеквадратичного риска. 
В~работах~\cite{ZasShe17,Mathematics2020} была показана состоятельность 
и~асимптотическая нормальность оценки риска данной процедуры. Аналогичные 
результаты для других методов построения~$\hat{\mu}_F$ получены в~работах~\cite{Shestakov2021-1,Shestakov2021-2,Shestakov2022}.

В то же время в~определенных приложениях, например  при анализе полученных 
в~результате использования ДНК-мик\-ро\-чи\-пов данных~\cite{ResultsOnFDRUnderDependence}, исследовании геофизических процессов 
и~анализе помех\linebreak в~телекоммуникационных каналах, условие незави\-си\-мости компонент 
вектора $x$ может не выполняться. Ранее в~работах~\cite{VorontsovShestakov2023,Vorontsov2024} была \mbox{исследована} асимп\-то\-ти\-ка 
среднеквадратичного риска оценки~$\hat{\mu}_F$ \mbox{в~случае}, когда~$\mu$ принадлежит 
одному из классов разреженности
$$
l_0[\eta] = \left\{\mu\,:\, ||\mu||_0 \leq \eta n\right\}, \enskip \eta \in 
(0,1),
$$

\vspace*{-12pt}

\noindent
\begin{multline*}
m_p[\eta] \equiv{}\\
{}\equiv \left\{\mu \in \mathbb{R}^n : |\mu|_{(k)} \leq \eta n^{1/p} 
k^{-1/p},\ k=\overline{1,n}\right\}, \\
 p\in(0, 2),
\end{multline*}
а компоненты вектора~$x$ слабо зависимы~--- имеют достаточно быстро убывающий 
коэффициент сильного перемешивания~\cite{Bosq}

\noindent
\begin{multline*}
\alpha(k) = \sup\limits_{1\leq m\leq n}\alpha\left(\sigma(x_i, i\leq m), 
\sigma(x_i, i\geq m+k)\right), \\ 
k=\overline{1,n-1}\,,
\end{multline*}
где символом $\sigma(x_i, i\in I)$ обозначена сиг\-ма-ал\-геб\-ра, порожденная 
множеством случайных величин $\{x_i, i \hm\in I\}$, а~мера  $\alpha(\cdot, \cdot)$ 
близости двух сиг\-ма-ал\-гебр определяется как
$$
\alpha(\mathcal{B},\mathcal{C}) = \sup\limits_{B\in\mathcal{B}, 
C\in\mathcal{C}} \left|\p(BC)-\p(B)\p(C)\right|.
$$

В настоящей работе показана асимптотическая нормальность и~сильная 
состоятельность оценки риска при применении FDR-про\-це\-ду\-ры в~случае, когда 
компоненты вектора~$x$ слабо зависимы, а~$\mu$ принадлежит одному из классов 
раз\-ре\-жен\-ности: 
$l_0[\eta]$ или $m_p[\eta]$.


\section{Обработка вектора данных с~помощью FDR-процедуры}

Широким классом методов построения оценки~$\hat{\mu}$ стала пороговая обработка 
вектора~$x$ с~некоторым порогом~$T$. Различают жесткую пороговую обработку, при 
которой полагается
\begin{equation*}
\left(\hat{\mu}\right)_i  = p_H(x_i,T) \equiv
 \begin{cases}
   x_i, & |x_i| > T\,;\\
   0, & |x_i| \leq T\,,
 \end{cases}
\end{equation*}
и мягкую пороговую обработку, для которой
\begin{equation*}
(\hat{\mu})_i  = p_S(x_i,T) \equiv
 \begin{cases}
   x_i-T, & \hphantom{\vert\vert}x_i > T;\\
   x_i+T, & \hphantom{\vert\vert}x_i <- T;\\
   0, & |x_i| \leq T.
 \end{cases}
\end{equation*}
Среднеквадратичный риск подобных процедур определяется как
\begin{equation}
\label{riskDef}
R(T) = {\mathsf E} ||\hat{\mu}-\mu||^2 = \sum\limits_{i=1}^n {\mathsf E} \left((\hat{\mu})_i-
\mu_i\right)^2.
\end{equation}
Обозначим через~$T_m$ наилучшее значение порога:
$$
T_m : \, R(T_m) = \min\limits_{T} R(T).
$$

Предложенная в~\cite{AdaptingFDR} процедура заключается в~жесткой пороговой 
обработке компонент вектора~$x$ с~порогом $\hat{t}_F \hm= \hat{t}_F(x)$, и~ее 
результат~--- оценка $\hat{\mu}_F$ вектора~$\mu$ с~компонентами $(\hat{\mu}_F)_i  
\hm= p_H(x_i,\hat{t}_F)$, где
\begin{multline*}
\hat{t}_F = \sigma z\left(\fr{q \hat{k}_F}{2n}\right), \enskip
\hat{k}_F = \max 
\left\{k \, :\, |x|_{(k)} \geq t_k \right\}, \\
 t_k = \sigma z\left(\fr{q  k}{2n}\right);
\end{multline*}
$z(\alpha)$ --- квантиль уровня $(1\hm-\alpha)$ стандартного нормального 
распределения; $|x|_{(k)}$~--- $k$-й элемент вектора, получаемого в~результате 
упорядочения вектора~$|x|$ по невозрастанию:
$$
|x|_{(1)} \geq |x|_{(2)} \geq \cdots \geq |x|_{(n)};
$$
$q\in(0;1)$~--- управ\-ля\-ющий параметр FDR-ме\-то\-да.
Далее полагается, что $q\hm\equiv q_n$ зависит от~$n$. В~\cite{AdaptingFDR} 
показано, что эта процедура эквивалентна множественной проверке гипотез 
о~равенстве нулю компонент наблюдаемого вектора. Также показано, что с~помощью 
метода штрафных функций данную процедуру можно свести к~другим видам пороговой 
обработки, в~част\-ности к~мягкой пороговой обработке.

В работах~\cite{VorontsovShestakov2023, Vorontsov2024} была исследована 
асимптотика среднеквадратичного риска~$R(\hat{t}_F)$ описанной процедуры 
в~случае, когда компоненты вектора $x$ слабо зависимы, а $\mu$ принадлежит классу 
разреженности~$\Theta_n$, где~$\Theta_n$ есть~$l_0[\eta_n]$ или~$m_p[\eta_n]$. 
Было показано, что~$R(\hat{t}_F)$ асимптотически отличается от минимаксного 
риска
$\inf\nolimits_{\hat{\mu}\hm=\hat{\mu}(x)} \sup\nolimits_{\mu\in \Theta_n} {\mathsf E} 
||\hat{\mu}-\mu||^2$
на множитель не более чем логарифмического по\-рядка.

Отметим, что в~выражении для среднеквадратичного риска~(\ref{riskDef}) 
присутствуют неизвестные величины~$\mu_i$, а~потому вычислить~$R(T_m)$ и~$T_m$ 
не представляется возможным. На практике можно пользоваться, например, следующей 
оценкой среднеквадратичного риска~\cite{Mallat}:
$$
\hat{R}(T) = \sum\limits_{i=1}^n F[x_i, T],
$$
где  
\begin{multline*}
F[x_i, T] = {}\\[3pt]
{}=\!\begin{cases}
\left(x_i^2-\sigma^2\right) \Ik(|x_i|\leq T) + \sigma^2 \Ik\left(|x_i|>T\right) &\\[3pt]
&\hspace*{-53mm}\mbox{для\ жесткой\ пороговой\ обработки};\\[3pt]
\left(x_i^2-\sigma^2\right) \Ik\left(|x_i|\leq T\right) + (\sigma^2+T^2) 
\Ik \left(|x_i|>T\right) \hspace*{-11.21576pt}&\\[3pt]
&\hspace*{-51mm}\mbox{для\ мягкой\ пороговой\ обработки}.
\end{cases}\hspace*{-7.17859pt}
\end{multline*}


\noindent
\textbf{Замечание}.\ При пороговой обработке иногда также используется так 
называемый универсальный порог $T_U\hm = \sigma \sqrt{2\ln n}$, предложенный 
в~работе~\cite{spatialAdaptation}. Исследования в~\cite{AdaptingSURE, ExactRisk} 
показали, что порог~$T_U$ в~определенном смысле максимальный, и~рас\-смат\-ри\-вать 
пороги выше него не имеет смысла. Более того, нетрудно показать, что $t_k \hm< T_U$ 
для всех~$k$ и~всех достаточно больших~$n$, в~связи с~чем всюду далее полагаем, 
что порог~$\hat{t}_F$ выбирается на отрезке $[0; T_U]$.

\section{Вспомогательные утверждения}

Кроме коэффициента сильного перемешивания~$\alpha(\cdot)$ также понадобится 
следующее понятие~\cite{Bosq}.

\smallskip

\noindent
\textbf{Определение.} %\label{defRho}
Максимальным коэффициентом корреляции~$\rho(\cdot)$ компонент вектора~$x$ 
называется
\begin{multline*}
\rho (k) \equiv \rho_n (k) = {}\\
{}=\sup\limits_{1\leq m\leq n}\rho\left(\sigma(x_i, 
i\leq m), \sigma(x_i, i\geq m+k)\right), \\
 k=\overline{1,n-1}\,,
\end{multline*}
где мера $\rho(\cdot, \cdot)$ близости двух сиг\-ма-ал\-гебр определяется как
$$
\rho(\mathcal{B},\mathcal{C}) = \sup\limits_{\substack{\xi 
\in\mathcal{L}^2(\mathcal{B}) \\
 \eta \in\mathcal{L}^2(\mathcal{C})}} 
\left|\mathrm{corr}\,(\xi, \eta)\right|.
$$


Введем обозначения:
$$
T_1 = \sqrt{2\ln \eta_n^{-p}};  \,\gamma_n = \fr{1}{\ln\ln n}; \, \kappa_n 
= \fr{n \eta_n^p T_1^{-p}}{1 - q_n - \gamma_n}; 
$$
$$ 
\kappa_n^0 = \fr{[n \eta_n]}{1 - q_n - \gamma_n} ;\, \rho^\star (k) = 
\sup\limits_{n\geq k+1} \rho(k), k \in \mathbb{N} ;
$$
$$
t_{\kappa_n} = \sigma z\left(\fr{q_n \kappa_n }{2n}\right) , \,\, t_{\kappa_n^0} 
= \sigma z\left(\fr{q_n \kappa_n^0 }{2n}\right).
$$


Следующие два утверждения показывают, что случайный порог~$\hat{t}_F$ в~случае 
$\mu\hm\in m_p[\eta_n]$ (соответственно $\mu\hm\in l_0[\eta_n]$) с~большой 
вероятностью будет не меньше~$t_{\kappa_n}$ (соответственно~$ t_{\kappa_n^0}$). 
Их  доказательства приведены в~работах~\cite{VorontsovShestakov2023, Vorontsov2024}.

\smallskip

\noindent
%\begin{lem}\label{lem5}
\textbf{Лемма~1.}\ \textit{Пусть $n^{-\delta_1} \hm\leq \eta_n^p \hm\leq n^{-\delta_2}$, 
$0\hm<\delta_2\hm<\delta_1<1$, $\mathrm{lim\,inf} q_n \ln n \hm\geq C \hm> 0$, 
$m\hm\in[1;n/2]\cap\mathbb{N}$, а $\alpha(\cdot)$~--- коэффициент сильного 
перемешивания компонент вектора~$x$. Для некоторого $N\hm\in\mathbb{N}$ при $n \hm\geq 
N$ справедливо}
\begin{multline*}
\hspace*{-3pt}\sup\limits_{\mu\in m_p[\eta_n]} \p \left(\hat{k}_F \geq \kappa_n \right) \leq 
4 n \exp\left\{-\fr{m}{256n}  \kappa_n q_n \gamma_n^2    \right\}+{}\\
{}+ 22\left(1+\fr{8n}{\kappa_n q_n \gamma_n}\right)^{1/2} n m 
\alpha\left(\left[\fr{n}{2m}\right]\right).
\end{multline*}



\smallskip

\noindent
\textbf{Лемма 2.}\ 
%\label{lem1}
\textit{Пусть $\eta_n \hm\leq b\hm<1$, $m\in[1;n/2]\cap\mathbb{N}$, а~$\alpha(\cdot)$~--- 
коэффициент сильного перемешивания компонент вектора~$x$. Для некоторого 
$N\hm\in\mathbb{N}$ при $n \hm\geq N$ справедливо}
\begin{multline*}
\sup\limits_{\mu\in l_0[\eta_n]} \p \left(\hat{k}_F \geq \kappa_n^0 \right) 
\leq{}\\
{}\leq 4 n \exp\left\{-\fr{(1-b)m}{64n}\,  \kappa_n^0 q_n \gamma_n^2    
\right\}+{}\\
{}+ 22\left(1+\fr{4n}{(1-b)\kappa_n^0 q_n \gamma_n}\right)^{1/2} n m 
\alpha\left(\left[\fr{n}{2m}\right]\right).
\end{multline*}

Следующие два утверждения доказаны в~\cite{Bosq} и~представляют собой аналоги 
неравенств Хеффдинга и~Бернштейна для слабо зависимых случайных величин.


\smallskip

\noindent
\textbf{Лемма 3.}\
\textit{Пусть для набора действительных случайных величин $X_1, \ldots, X_n$ 
с~коэффициентом сильного перемешивания $\alpha(\cdot)$ выполняется ${\mathsf E} X_i \hm=0$, 
$|X_i|\hm\leq b$, $i\hm=\overline{1,n}$. Тогда для любого целого числа $m\hm\in[1; n/2]$ 
и~любого $\eps\hm>0$ справедливо}
\begin{multline*}
\p\left(\left|\sum\limits_{i=1}^n X_i\right| > n\eps \right) \leq 4 
\exp\left\{-\fr{\eps^2 m}{8 b^2}\right\}+ {}\\
{}+
22\left(1+\fr{4b}{\eps}\right)^{1/2} m\, 
\alpha\left(\left[\fr{n}{2m}\right]\right).
\end{multline*}


\smallskip

\noindent
\textbf{Лемма 4.}\
\textit{Пусть для набора действительных случайных величин $X_1, \ldots, X_k$ 
с~коэффициентом сильного перемешивания $\alpha(\cdot)$ выполняется ${\mathsf E} X_i \hm=0$, 
$|X_i|\hm\leq b$, $i\hm=\overline{1,k}$. Тогда для любого целого числа $m\hm\in[1; k/2]$ 
и~любого $\eps\hm>0$ справедливо}
\begin{multline*}
\p\left(\left|\sum\limits_{i=1}^k X_i\right| > \eps \right) \leq 4 
\exp\left\{-\fr{\eps^2 m}{8 v^2 k^2}\right\}+{}\\
{}+ 22\left(1+\fr{4bk}{\eps}\right)^{1/2} m\, 
\alpha\left(\left[\fr{k}{2m}\right]\right),
\end{multline*}
\textit{где $p = k/(2m)$}:
\begin{multline*}
v^2 =
 \fr{b \eps}{2k} + {}\\
 {}+\fr{2}{p^2} \,  \max\limits_{ j\in[0,\,2m-1]} 
{\mathsf E} \big( ([jp]+1-jp)X_{[jp]+1} + X_{[jp]+2}+{}\\
{}+ \cdots +  X_{[(j+1)p]} + ((j+1)p-[(j+1)p])X_{[(j+1)p+1]}\big)^2.
\end{multline*}

\noindent
\textbf{Замечание.}
Если существует такое число $S \hm> 0$, что сразу для всех $i\hm\in[1;k]$  выполняется 
${\mathsf E} X_i^2 \hm\leq S^2$, то в~качестве~$v^2$ можно взять
$$
v^2 = \fr{b \eps}{2k} + 8 S^2.
$$


Д\,о\,к\,а\,з\,а\,т\,е\,л\,ь\,с\,т\,в\,о\ \ сле\-ду\-юще\-го утверж\-де\-ния приведено в~работе~\cite{AdaptingFDR}.

\smallskip

\noindent
\textbf{Лемма 5.}\ 
\textit{Для $y\leq 0{,}01$ справедливы представления}
\begin{multline}
\label{lem1eq1}
z^2(y) = 2 \ln y^{-1} - \ln \ln y^{-1} - r_2(y), \\
 r_2(y) \in [1{,}8; 3];
\end{multline}

\noindent
\begin{equation}
\label{lem1eq2}
z(y) = \sqrt{2 \ln y^{-1}} - r_1(y), \, \, r_1(y) \in [0; 1{,}5].
\end{equation}


\section{Асимптотическая нормальность оценки риска при~применении FDR-процедуры в~условиях слабой зависимости}

Перейдем к~описанию достаточных условий для асимптотической нормальности оценки 
риска $\hat{R}(\hat{t}_F)$ в~случае $\mu \hm\in m_p[\eta_n]$.

\smallskip

\noindent
\textbf{Теорема~1.}\
\textit{Пусть $\mu \hm\in m_p[\eta_n],$ $\eta_n^p \hm\in[n^{-\delta_1}; n^{-\delta_2}],$ $1/2 \hm< 
\delta_2 \hm< \delta_1<1;$ имеются такие константы $c_1, c_2>0$, что для 
коэффициента сильного перемешивания $\alpha(\cdot)$ компонент вектора $x$ 
справедливо  $\alpha(k) \hm\leq c_1 k^{-1-(5/2)\delta_1/(1-\delta_1)-c_2},$ 
$k\hm=\overline{1,n-1};$ $q_n \hm< c_3 \hm< 1;$ $\mathrm{lim\,inf} q_n \ln n \hm= c_4 \hm> 0;$ и,~кроме того, 
для максимального коэффициента корреляции $\rho(\cdot)$ компонент вектора~$x$ 
справедливо}
$$
\sum\limits_{k = 1}^{\infty} \sup\limits_{n\geq k+1} \rho(k) \equiv 
\sum\limits_{k = 1}^{\infty}  \rho^\star (k) = c_5 < \infty. 
$$
\textit{Тогда при $n \to \infty$}
$$
\fr{\hat{R}(\hat{t}_F) - R(T_m)}{C_\rho \sqrt{2n}} \Rightarrow N(0, 1),
$$
\textit{где}
$$
C_\rho = \sigma^2\sqrt{1 +  \lim\limits_{n\to\infty} \fr{1}{n} \sum\limits_{j\neq i} \mathrm{corr}^2 (x_i, x_j)}.
$$

\noindent
Д\,о\,к\,а\,з\,а\,т\,е\,л\,ь\,с\,т\,в\,о\  \
 приводится для метода мягкой пороговой обработки; в~случае жесткой пороговой 
обработки доказательство аналогично. Обозначим
$$
U(T) = \hat{R}(T) -  \hat{R}(T_m) = \sum \limits_{i=1}^n H_i(T, T_m),
$$
где
$$
H_i(T, T_m) = F[x_i, T] - F[x_i, T_m].
$$
Имеем

\vspace*{-3pt}

\noindent
\begin{multline}
\label{D00}
\hat{R}(\hat{t}_F) - R(T_m) + \hat{R}(T_m) - \hat{R}(T_m) ={}\\
{}= \hat{R}(T_m) - 
R(T_m) + U(\hat{t}_F).
\end{multline}
Покажем, что
\begin{equation}
\label{D0}
\fr{\hat{R}(T_m) - R(T_m)}{C_\rho\sqrt{2n}} \Rightarrow N(0, 1).
\end{equation}


Повторяя рассуждения из~\cite{KuShe2016_1,KuShe2016_2,Jansen}, можно показать, 
что $T_m \hm\geq t_{\kappa_n}$. Учитывая также $T_m\hm \leq T_U$, имеем 
$$
C \sqrt{\ln n} \leq T_m \leq C^\prime \sqrt{\ln n}
$$ 
для некоторых положительных констант $C$ и~$C^\prime$.

\columnbreak

В случае мягкой пороговой обработки $\hat{R}(T_m)$ представляет собой 
несмещенную оценку~$R(T_m)$, а~при жесткой пороговой обработке и~выполнении 
условий теоремы смещение стремится к~нулю при делении на $\sqrt{n}$~\cite{Mallat}.

Для дисперсии числителя~(\ref{D0}) имеем:
\begin{multline*}
{\mathsf D} \left(\hat{R}(T_m) - R(T_m)\right) = \sum\limits_{i=1}^n {\mathsf D} F[x_i, T_m] + {}\\
{}+
\sum\limits_{i=1}^n\sum\limits_{\substack{j=1 \\  j\neq i}}^n \mathrm{cov}\left( F[x_i, T_m], F[x_j, 
T_m] \right).
\end{multline*}

Поскольку $\mu \in m_p[\eta_n]$,
\begin{equation}
\left.
\begin{array}{l}
 \displaystyle\sum\limits_{i: |\mu_i| > 1/T_1} {\mathsf D} F[x_i, T_m]  \leq{}\\
 \hspace*{15mm}{}\leq  4\left(\sigma^2 + T_m^2\right)^2 n \eta_n^p 
T_1^p = o(n);
\\[6pt]
\displaystyle \sum\limits_{\substack{{i,j: \max\{|\mu_i|, |\mu_j|\} > 1/T_1,}\\{j\neq i}}}  \hspace*{-12mm}\mathrm{cov}\,(F[x_i, 
T_m],F[x_j, T_m])  \leq{}\\
\hspace*{10mm}{}\leq 16\left(\sigma^2 + T_m^2\right)^2 n \eta_n^p T_1^p c_5 = o(n). 
\end{array}
\right\}    
\label{D2}
\end{equation}
Далее, учитывая что ${\mathsf D} x_i^2 \hm= 2\sigma^4 \hm+ 4\sigma^2 \mu_i^2$, нетрудно 
убедиться, что
\begin{multline}
\label{D3}
\sum\limits_{i: |\mu_i| \leq 1/T_1}\hspace*{-4mm} {\mathsf D} F[x_i, T_m] ={}\\
{}= \sum\limits_{i: |\mu_i| \leq 1/T_1} \hspace*{-4mm} {\mathsf D} 
x_i^2 + o(n) = 2\sigma^4 n + o(n).
\end{multline}


Введем обозначение 
$$
D_n = \left\{(i,j) : \max\left\{|\mu_i|, |\mu_j|\right\}  \leq \fr{1}{T_1}\,, \enskip j\hm\neq i\right\}.
$$
 Для суммы ковариаций аналогично~(\ref{D3}) получим
\begin{multline*}
\sum\limits_{(i,j)\in D_n} \hspace*{-2mm}\mathrm{cov}\left( F[x_i, T_m], F[x_j, T_m] \right) = {}\\
{}=
\sum\limits_{(i,j)\in D_n} \hspace*{-2mm}\mathrm{cov}\left( x_i^2, x_j^2 \right) + o(n).
\end{multline*}
Воспользуемся тождеством~\cite{Eroshenko}
$$
\mathrm{cov}\left (x_i^2, x_j^2\right) = 4 {\mathsf E} x_i {\mathsf E} x_j \mathrm{cov}\left(x_i, x_j\right) + 2 \mathrm{cov}^2 \left(x_i, x_j\right)
$$
для вектора $(x_i, x_j)$, имеющего двумерное нормальное распределение. Заметим, 
что
\begin{gather*}
 \sum\limits_{(i,j)\in D_n} 4 | {\mathsf E} x_i {\mathsf E} x_j \mathrm{cov}\left(x_i, x_j\right)| \leq 8 T_1^{-2} 
\sigma^2 n c_5 = o(n);
\\
\sum\limits_{(i,j)\in D_n} 2 \mathrm{cov}^2 (x_i, x_j)  = 2\sigma^4 \sum\limits_{(i,j)\in D_n} 
\mathrm{corr}^2 (x_i, x_j). 
\end{gather*}
Более того, поскольку  %< 4 \sigma^2 n c_5.$$
\begin{equation*}
\sum\limits_{\substack{{i,j: \max\{|\mu_i|, |\mu_j|\} > 1/T_1} \\ {j\neq i}}}
\hspace*{-10mm}\mathrm{corr}^2 (x_i, x_j)  
\leq  4 n \eta_n^p T_1^p c_5 =  o(n),
\end{equation*}
имеем
\begin{multline*}
\sum\limits_{(i,j)\in D_n} \mathrm{corr}^2 (x_i, x_j) ={}\\
{}= \sum\limits_{j\neq i} \mathrm{corr}^2 (x_i, x_j) 
+o(n)= c_6 n + o(n),
\end{multline*}
где
$$
c_6 = \lim\limits_{n\to\infty} \fr{1}{n} \sum\limits_{j\neq i} \mathrm{corr}^2 (x_i, x_j) 
\leq 2 c_5.
$$
Полагая $C_\rho \hm= \sigma^2\sqrt{1 + c_6}$, получим, наконец,
\begin{equation}
\label{D1}
{\mathsf D} \left(\hat{R}(T_m) - R(T_m)\right)  =  2 n C_\rho^2 + o(n).
\end{equation}
Заметим, что из~(\ref{D2}), (\ref{D3}) и~(\ref{D1}) следует, что
\begin{equation}
\label{D5}
\sup\limits_{n} \fr{\sum\nolimits_{i=1}^n {\mathsf D} F[x_i, T_m]}{V_n^2} < \infty\,,
\end{equation}
где 
$$
V_n^2 = {\mathsf D} \sum\limits_{i=1}^n \left(F[x_i, T_m] \hm- {\mathsf E} F[x_i, T_m]\right).
$$
Кроме того, поскольку $F[x_i, T_m]$ по модулю ограничены величиной $\sigma^2 \hm+ 
T_m^2$, выполнено условие Линдеберга: для любого $\eps\hm>0$ при $n \hm\to \infty$
\begin{multline}
\label{D6}
\!\!\!\fr{1}{V_n^2}\sum\limits_{i=1}^n {\mathsf E} \left( \!\left( F\left[x_i, T_m\right]\! -\! {\mathsf E} F\left[x_i, T_m\right]\right)^2 
\Ik \left(\vert F\left[x_i, T_m\right] -{}\right.\right.\hspace*{-2.69505pt}\\
\left.\left.{}- {\mathsf E} F\left[x_i, T_m\right]\vert >\eps V_n\right)\!
\vphantom{\left( F\left[x_i, T_m\right]\! -\! {\mathsf E} F\left[x_i, T_m\right]\right)^2}
\right) 
\to  0\,.
\end{multline}
Из~(\ref{D1})--(\ref{D6}), очевидного неравенства
$$ 
\lim\limits_{k\to\infty} \sup\limits_{n\geq k+1}\rho(k) \equiv 
\lim\limits_{k\to\infty} \rho^\star (k)  < 1
$$
 и~центральной предельной теоремы для сильно перемешанных случайных величин~\cite{Peligrad} следует~(\ref{D0}).

Перейдем к~доказательству того, что $U(\hat{t}_F) \, n^{-1/2} \overset{\, \p \, }{\to} 0$.
Всюду далее, не ограничивая общности, полагаем $\sigma=1$. 
Введем обозначения:

\noindent
\begin{align*}
S_1(T) &= \sum\limits_{i: |\mu_i| > 1/T_1} H_i(T, T_m); \\
S_2(T) &= \sum\limits_{i: |\mu_i| \leq 1/T_1} H_i(T, T_m); 
\\
N_1(a, b) &= \sum\limits_{i: |\mu_i| > 1/T_1} \Ik (a<|x_i|\leq b); \\ 
N_2(a, b) &= \sum\limits_{i: |\mu_i| \leq 1/T_1} \Ik (a<|x_i|\leq b);
\end{align*}

\noindent
\begin{align*}
Z_l(T) &= S_l(T) - {\mathsf E} S_l(T),\enskip l = 1,2\,; \\  
d_n &= \fr{T_U -  t_{\kappa_n}}{n};\\
T_j^{\prime} &= t_{\kappa_n}+j d_n,\enskip j = \overline{0,n-1}\,.
\end{align*} 

\vspace*{-3pt}

\noindent
Для произвольного $\eps>0$

\vspace*{-3pt}

\noindent
\begin{multline}
\p \left( \fr{|U(\hat{t}_F)|}{\sqrt{n}}> 4\eps \right) \leq 
\p\left(\hat{t}_F \leq t_{\kappa_n}\right) + {}\\
{}+\p \left(\fr{\sup\nolimits_{T\in 
[t_{\kappa_n}, T_U]} |U(T)|}{\sqrt{n}}>4\eps \right)\leq  {}\\
{}\leq \p\left(\hat{t}_F \leq t_{\kappa_n}\right) + \p\left(\fr{\sup\nolimits_{T\in 
[t_{\kappa_n}, T_U]} |{\mathsf E} U(T)|}{\sqrt{n}}>\eps\right)+{}\\
{}+ \p \left(\sup\limits_{T\in [t_{\kappa_n}, T_U]} |Z_1(T)| > 
\eps\sqrt{n}\right) +{}\\
{}+ \p \left(\sup\limits_{j \in [0, n-1]} |Z_2(T_j^{\prime})| > 
\eps\sqrt{n}\right) +{}\\
{}+ \p \left(\sup\limits_{\substack{j \in [0, n-1] \\
 T\in [T_j^{\prime},T_j^{\prime}+d_n]}} |Z_2(T)-Z_2(T_j^{\prime})| > \eps\sqrt{n}\right).
\label{M1}
\end{multline}
Заметим, что $\gamma_n\hm > \ln^{-1} n$, $\kappa_n\hm > n \eta_n^p \ln ^{-1} n \hm\geq 
n^{1-\delta_1} \ln ^{-1} n$ и~$q_n\hm > c_4 \ln ^{-1} n /2$ для всех достаточно 
больших~$n$.
Для первого слагаемого в~(\ref{M1}) по лемме~1 с~$m \hm= n^{\delta_1} \ln 
^7 n$ для  больших~$n$ имеем

\vspace*{-3pt}

\noindent
\begin{multline}
\label{M1next}
\p\left(\hat{t}_F \leq t_{\kappa_n}\right)  = \p \left(\hat{k}_F \geq \kappa_n 
\right) \leq 4 n e^{-\ln^2 n} + {}\\
{}+n^{1+(3/2)\,\delta_1} \ln^9 n \, 
\alpha\left(\left[\fr{n^{1-\delta_1}}{\ln^{7} n}\right]\right) = o(1)
\end{multline}
при $n\to\infty$. 
Для оценки второго слагаемого в~(\ref{M1}) заметим, что при $T \hm\in 
[t_{\kappa_n}, T_U]$ справедливо
\begin{equation}
\label{M2}
{\mathsf E} H_i(T, T_m) \leq T_U^2 + 1.
\end{equation}
Если же кроме $T \hm\in [t_{\kappa_n}, T_U]$ также выполнено $|\mu_i| \hm\leq T_1^{-1}$, то

\vspace*{-6pt}

\noindent
\begin{multline*}
|{\mathsf E} H_i (T, T_m)| \leq 2 T_U^2 \, \p \left(|x_i| > t_{\kappa_n}\right) \leq {}\\
{}\leq2 
T_U^2 \, \p \left(|x_i-\mu_i| > t_{\kappa_n}-T_1^{-1}\right) \leq{}\\
{}\leq 2 T_U^2  \exp\left\{ -\fr{1}{2} \left(t_{\kappa_n} - T_1^{-
1}\right)^2 \right\}  \leq{}\\
{}\leq
 4 (\ln n)  \exp\left\{ -\fr{1}{2} 
\left(z\left(\fr{q_n\kappa_n}{2n}\right)\right)^2 + t_{\kappa_n} T_1^{-
1}\right\},
\end{multline*}

\vspace*{-2pt}

\noindent
где использовано неравенство 

\noindent
$$
2(1-\Phi(x))\hm \leq \fr{e^{-x^2/2}}{x}
$$

\pagebreak


\noindent
 для $x\hm\geq 0$ 
($\Phi(x)$~--- функция распределения $N(0,1)$). Рас\-смот\-рим выражение 
в~экспоненте. Второе слагаемое не превышает $1\hm+o(1)$ при $n\hm\to\infty$, поскольку 
$t_{\kappa_n} \hm\leq T_1 (1+o(1))$ при $\sigma\hm=1$, что нетрудно получить из 
определения~$t_{\kappa_n}$, пред\-став\-ле\-ния~(\ref{lem1eq2}) и~ограничения на~$q_n$ 
из формулировки тео\-ре\-мы. Для первого слагаемого, используя пред\-став\-ле\-ние~(\ref{lem1eq1}) 
и~ограничения, наложенные на~$q_n$, при больших~$n$ получим
\begin{multline*}
-\fr{1}{2}\left(z\left(\fr{q_n \kappa_n}{2n}\right)\right)^2 \leq - \ln 
\fr{2n (1-q_n-\gamma_n)}{q_n n \eta_n^p T_1^{-p}} + {}\\
{}+\fr{1}{2} \ln 
\left((1+o(1)) \ln \eta_n^{-p}\right) + \fr{3}{2} \leq{}\\
{}\leq \ln \fr{c_3}{1-c_3} + \ln \eta_n^p + \ln T_1^{-p} + \ln T_1 + 
\fr{3}{2}+ o(1).
\end{multline*}
Из приведенных соотношений следует, что с~некоторой константой $c_7 = c_7(c_3, 
p, \delta_1, \delta_2, c_4)$
\begin{equation}\label{M3}
\sup\limits_{\substack{i: |\mu_i| \leq 1/T_1 \\ T\in [t_{\kappa_n}, T_U]}} |{\mathsf E} 
H_i (T, T_m)|  \leq c_7 (\ln n)^{(3-p)/2}\eta_n^p.
\end{equation}
Из (\ref{M2}) и~(\ref{M3}) с~учетом $\delta_2 \hm> 1/2$ следует
\begin{multline*}
\sup\limits_{T\in [t_{\kappa_n}, T_U]} |{\mathsf E} U(T)| \leq{}\\
{}\leq 
 n\eta_n^p T_1^p 
(T_U^2+1) + c_7 (\ln n)^{(3-p)/2} n \eta_n^p = o(\sqrt{n})
\end{multline*}
при $n\to\infty$, а следовательно, для любого $\eps\hm>0$ второе слагаемое в~(\ref{M1}) обращается в~ноль для всех достаточно больших~$n$.

Далее, поскольку при $T \hm\leq T_U$ и~$\sigma\hm=1$
$$
|H_i(T, T_m) - {\mathsf E} H_i(T, T_m)| \leq 2 (T_U^2 +2), \enskip i=\overline{1, n}\,,
$$
а число слагаемых в~$Z_1(T)$ не превосходит $n\eta_n^p T_1^p$, имеем
$$
\sup\limits_{T\in [t_{\kappa_n}, T_U]} |Z_1(T)|  \leq 2 n\eta_n^p T_1^p (T_U^2 
+2) = o(\sqrt{n})
$$
при $n\to\infty$, а следовательно, для любого $\eps\hm>0$ и~третье слагаемое в~(\ref{M1}) обращается в~ноль для всех достаточно больших~$n$.

Перейдем к~оценке четвертого слагаемого в~(\ref{M1}). Аналогично~(\ref{M3}) 
можно получить:
\begin{multline}
\label{M10}
\!\!\sup\limits_{\substack{i: |\mu_i| \leq 1/T_1 \\ T\in [t_{\kappa_n}, T_U]}} \!{\mathsf D} 
H_i (T, T_m)  \leq \!\sup\limits_{\substack{i: |\mu_i| \leq 1/T_1 \\ T\in 
[t_{\kappa_n}, T_U]}} \!{\mathsf E} \left(H_i (T, T_m)\right)^2  \leq{}\\
{}\leq 2 c_7 (\ln n)^{(5-p)/2} \eta_n^p.
\end{multline}
По лемме~4 с~$m \hm= \sqrt{n} (\ln n)^3$ и~$k \hm= n-[n\eta_n^p T_1^p]$ 
для четвертого слагаемого в~(\ref{M1}) имеем:

\noindent
\begin{multline}
\p \left(\sup\limits_{j \in [0, n-1]} |Z_2(T_j^\prime)| > \eps\sqrt{n}\right) 
\leq {}\\
{}\leq \sum\limits_{j \in [0, n-1]} \hspace*{-3mm}\p \left( |Z_2(T_j^\prime)| > \varepsilon\sqrt{n}\right)\leq{}\\
{}\leq 4 n \exp \left\{ - \fr{\eps^2 n^{3/2} (\ln n)^3}{n-[n\eta_n^p T_1^p]}\!\Bigg/\! \big( 8 (T_U^2+2)\eps\sqrt{n} +{}\right.\\
\left.{}+ 128 c_7 (\ln n)^{(5-p)/2} \eta_n^p  (n-
[n\eta_n^p T_1^p])\big) 
\vphantom{ \fr{\eps^2 n^{3/2} (\ln n)^3}{n-[n\eta_n^p T_1^p]}}
\right\} +{}\\
{}
+ 22 \left(1+\fr{8(T_U^2+2) (n-[n\eta_n^p T_1^p])}{\eps 
\sqrt{n}}\right)^{1/2}\times{}\\
{}\times n^{3/2} (\ln n)^3 \alpha\left(\left[\fr{n-[n\eta_n^p 
T_1^p]}{2 (\ln n)^3 \sqrt{n}}\right]\right).
\label{M5}
\end{multline}
Используя ограничения $n^{-\delta_1}\hm\leq \eta_n^p \leq n^{-\delta_2}$ 
и~$1/2\hm<\delta_2\hm<\delta_1\hm<1$, из~(\ref{M5}) получим для любого $\eps\hm>0$
$$
\p \left(\sup\limits_{j \in [0, n-1]} |Z_2(T_j^\prime)| > \eps\sqrt{n}\right) 
\to 0
$$
при $n \to \infty$.

Рассмотрим, наконец, пятое слагаемое в~(\ref{M1})). Заметим, что при $0\hm< a \hm< b$ 
справедливо
$$
|Z_2(b)-Z_2(a)| \leq 2 |N_2(a,b)-{\mathsf E} N_2(a,b)| + n (b^2-a^2).
$$
Полагая $a = T_j^\prime$, $b \hm= T \hm\in [T_j^\prime, T_j^\prime+d_n]$ для 
произвольного $j \hm\in [0, n-1]$ и~учитывая, что
$$
(T^2 - (T_j^\prime )^2) = (T - T_j^\prime)(T+ T_j^\prime ) \leq  2 d_n T_U < 2 
T_U^2 n^{-1}; 
$$

\vspace*{-12pt}

\noindent
\begin{multline*}
\p\left(T_j^\prime < |x_i| \leq T \right) \leq \p\left(T_j^\prime < |x_i| \leq 
T_j^\prime+d_n\right) <{}\\
{}< d_n < T_U n^{-1}, 
\end{multline*}
получим  оценку
$$
|Z_2(T)-Z_2(T_j^\prime)| \leq 2 N_2(T_j^\prime, T) +  3 T_U^2 .
$$
Далее, поскольку $N_2 (T_j^\prime, T) \hm\leq N_2 (T_j^\prime, T_j^\prime+d_n)$ и~${\mathsf E} N_2 (T_j^\prime, T_j^\prime+d_n) \hm< T_U^2$,
имеем
\begin{multline*}
\sup\limits_{T \in [T_j^\prime, T_j^\prime+d_n]} |Z_2(T)-Z_2(T_j^\prime)| \leq {}\\
{}\leq
2 \left|N_2 (T_j^\prime, T_j^\prime+d_n) - {\mathsf E} N_2 (T_j^\prime, 
T_j^\prime+d_n)\right| +  5 T_U^2 .
\end{multline*}
Аналогично~(\ref{M3}) показывается, что
\begin{multline}
\label{M11}
\sup\limits_{\substack{i : |\mu_i| \leq 1/T_1 \\ j \in [0, n-1]}} {\mathsf D} \Ik 
(T_j^\prime < |x_i| \leq T_j^\prime + d_n) <{}\\
{}< c_7 (\ln n)^{(1-p)/2} \eta_n^p.
\end{multline}
Пусть $n > N(\eps)$ настолько, что 
$$
\fr{\eps\sqrt{n} - 5 T_U^2}{2} > \fr{\eps \sqrt{n} }{4}\,.
$$
%
 Тогда для пятого слагаемого в~(\ref{M1}) по лемме~4 с~$m \hm= 
\sqrt{n} (\ln n)^2$ и~$k \hm= n\hm-[n\eta_n^p T_1^p]$ имеем
\begin{multline}
\p \left(\sup\limits_{\substack{j \in [0, n-1] \\ T\in 
[T_j^{\prime},T_j^{\prime}+d_n]}} |Z_2(T)-Z_2(T_j^{\prime})| > 
\eps\sqrt{n}\right) \leq{}\\
{}\leq  \sum\limits_{j \in [0, n-1]} \p \left(  \left|N_2 (T_j^\prime, 
T_j^\prime+d_n) -{}\right.\right.\\
\left.\left.{}- {\mathsf E} N_2 (T_j^\prime, T_j^\prime+d_n)\right| > \fr{\eps\sqrt{n}}{4} 
\right) \leq{}\\
{}\leq  4n \exp \left\{ -  \fr{\eps^2 n^{3/2} (\ln n)^2}{(n-[n\eta_n^p T_1^p])^{-1}}\Bigg/ 
\big( 16 \eps \sqrt{n} +{}\right.\\
\left.{}+ 64 c_7 (\ln n)^{(1-p)/2} \eta_n^p (n-[n\eta_n^p 
T_1^p]) \big) 
\vphantom{\fr{\eps^2 n^{3/2} (\ln n)^2}{(n-[n\eta_n^p T_1^p])^{-1}}}
\right\} +{}\\
{}+ 22 \left(1+\fr{16 (n-[n\eta_n^p T_1^p])}{\eps \sqrt{n}}\right)^{1/2}\times{}\\
{}\times 
n^{3/2} (\ln n)^2 \alpha\left(\left[\fr{n-[n\eta_n^p T_1^p]}{2 (\ln n)^2 
\sqrt{n}}\right]\right).
\label{M6}
\end{multline}
Используя ограничения $n^{-\delta_1}\hm\leq \eta_n^p\hm \leq n^{-\delta_2}$ 
и~$1/2\hm<\delta_2\hm<\delta_1<1$, из~(\ref{M6}) получим для любого $\eps\hm>0$
$$
\p \left(\sup\limits_{\substack{j \in [0, n-1] \\ T\in 
[T_j^{\prime},T_j^{\prime}+d_n]}} |Z_2(T)-Z_2(T_j^{\prime})| > 
\eps\sqrt{n}\right) \to 0
$$
при $n \to \infty$.

Таким образом, показано, что для любого $\eps>0$ все слагаемые в~(\ref{M1}) 
стремятся к~нулю при $n\to\infty$. Следовательно,
$$
\fr{|U(\hat{t}_F)|}{\sqrt{n}}  \overset{\, \p \, }{\to} 0 \,,
$$
что вместе с~(\ref{D0}) завершает доказательство тео\-ремы.~\hfill$\square$

\smallskip

Следующая теорема дает достаточные условия для асимптотической нормальности 
оценки риска $\hat{R}(\hat{t}_F)$ в~случае $\mu \hm\in l_0[\eta_n]$.

\smallskip

\noindent
\textbf{Теорема 2.}\ 
\textit{Пусть $\mu \hm\in l_0[\eta_n]$, $\eta_n\hm\in[n^{-\delta_1}, n^{-\delta_2}]$, $1/2\hm < 
\delta_2\hm < \delta_1\hm<1;$ имеются такие константы $c_1, c_2\hm>0$, что для 
коэффициента сильного перемешивания $\alpha(\cdot)$ компонент вектора~$x$ 
справедливо} 
\begin{gather*}
\alpha(k) \leq c_1 k^{-1-(5/2)\delta_1/(1\hm-\delta_1)\hm-c_2},\enskip 
k=\overline{1,n-1};\\
 q_n < c_3 < 1;\enskip \mathrm{lim\,inf} q_n \ln n = c_4 > 0;
\end{gather*}
\textit{для максимального коэффициента корреляции~$\rho(\cdot)$ компонент вектора~$x$ 
справедливо}
$$
\sum\limits_{k = 1}^{\infty} \sup\limits_{n\geq k+1} \rho(k) \equiv 
\sum\limits_{k = 1}^{\infty}  \rho^\star (k) = c_5 < \infty. 
$$
\textit{Тогда при $n \to \infty$}
$$
\fr{\hat{R}(\hat{t}_F) - R(T_m)}{C_\rho \sqrt{2n}} \Rightarrow N(0, 1),
$$
\textit{где}
$$
C_\rho = \sigma^2\sqrt{1 +   \lim\limits_{n\to\infty} \fr{1}{n} 
\sum\limits_{j\neq i} \mathrm{corr}^2 (x_i, x_j)}\,.
$$

\noindent
Д\,о\,к\,а\,з\,а\,т\,е\,л\,ь\,с\,т\,в\,о\  проводится аналогично доказательству теоремы~1. 
Переменная~$D_n$ теперь определяется как $D_n \hm= \{(i,j) : 
|\mu_i|\hm=|\mu_j|=0$, $j\hm\neq i\}$. Условия вида $|\mu_i|\hm<T_1^{-1}$ (вида 
$|\mu_i|\hm\geq T_1^{-1}$) заменяются условиями  $\mu_i\hm=0$ (соответственно 
$|\mu_i|\hm>0$).
Поскольку $\mu \hm\in l_0[\eta_n]$, количество~$i$ таких, что $|\mu_i|\hm>0$ 
(а~значит, и~число слагаемых в~$Z_1(T)$), не превышает~$[n \eta_n]$.

Для оценки первого слагаемого в~(\ref{M1}) используется лемма~2, 
в~которой можно взять, например, $b\hm=1/2$, а~для~$\kappa_n^0$ использовать оценку 
$\kappa_n^0 \hm> n \eta_n$. Формулы (\ref{M3}),  (\ref{M10}) и~(\ref{M11}) 
принимают вид соответственно
\begin{align*}
\sup\limits_{\substack{i: \mu_i =0 \\ T\in [t_{\kappa_n^0}, T_U]}} |{\mathsf E} H_i (T, 
T_m)| & \leq c_8 (\ln n)^{3/2} \eta_n ;
\\
\sup\limits_{\substack{i: \mu_i =0 \\ T\in [t_{\kappa_n^0}, T_U]}} {\mathsf D} H_i (T, 
T_m)  & \leq 2 c_8 (\ln n)^{5/2} \eta_n;
\\
\sup\limits_{\substack{i : \mu_i =0 \\ j \in [0, n-1]}} {\mathsf D} \Ik (T_j^\prime < 
|x_i| \leq T_j^\prime + d_n) &< c_8 (\ln n)^{1/2} \eta_n,
\end{align*}
где $c_8 = c_8(c_3,\delta_1, \delta_2, c_4)$. В~остальном доказательство 
аналогично.~\hfill$\square$

\section{Сильная состоятельность оценки риска при~применении FDR-процедуры 
в~условиях слабой зависимости}

Следующая теорема дает достаточные условия для сильной состоятельности оценки 
риска $\hat{R}(\hat{t}_F)$ в~случаях $\mu \hm\in m_p[\eta_n]$ и~$\mu \hm\in 
l_0[\eta_n]$.

\smallskip

\noindent
\textbf{Теорема 3.}
\textit{Пусть $\mu\hm \in m_p[\eta_n]$, $\eta_n^p\hm\in[n^{-\delta_1}, n^{-\delta_2}]$ либо 
$\mu \hm\in l_0[\eta_n]$, $\eta_n\hm\in[n^{-\delta_1}, n^{-\delta_2}]$; $0 \hm< \delta_2 
\hm< \delta_1<1$; имеются такие константы $c_1, c_2\hm>0$, что для коэффициента 
сильного перемешивания $\alpha(\cdot)$ компонент вектора~$x$ справедливо}  
$\alpha(k) \hm\leq c_1 k^{-2-(7/2)\delta_1/(1\hm-\delta_1)\hm-c_2}$, $k\hm=\overline{1,n-1}$; 
$q_n \hm< c_3 \hm< 1$; $\mathrm{lim\,inf} q_n \ln n \hm= c_4 \hm> 0$. \textit{Тогда при} $n \hm\to \infty$
$$
\fr{\hat{R}(\hat{t}_F) - R(T_m)}{n} \rightarrow 0 \, \, \,\textit{п.~в.}
$$


\noindent
Д\,о\,к\,а\,з\,а\,т\,е\,л\,ь\,с\,т\,в\,о\,.  Воспользуемся представлением~(\ref{D00}).

Покажем, что $(\hat{R}(T_m)-R(T_m))n^{-1}\hm \to 0$ п.~в.\ при $n\hm\to\infty$. 
При мягкой пороговой обработке ${\mathsf E} \hat{R}(T_m) \hm= R(T_m)$, а~при жесткой 
пороговой обработке
\begin{multline*}
\fr{\hat{R}(T_m)-R(T_m)}{n} = {}\\
{}=\fr{\hat{R}(T_m)-{\mathsf E} \hat{R}(T_m)}{n} 
+\fr{{\mathsf E}\hat{R}(T_m)-R(T_m)}{n}\,,
\end{multline*}
где второе слагаемое стремится к~нулю при $n\to\infty$ \cite{Mallat}. 
Следовательно, достаточно показать, что $(\hat{R}(T_m)\hm-{\mathsf E}\hat{R}(T_m))n^{-1} \hm\to 0$ п.~в.

Полагая в~лемме~3 $X_i \hm= F[x_i, T_m] \hm- {\mathsf E} F[x_i, T_m]$, $b \hm= 
2(\sigma^2\hm+T_m^2)$ и~$m \hm= n^{1/4}$ и~учитывая ограничения на $\alpha(\cdot)$ из 
условия, нетрудно убедиться, что для всех~$n$
$$
\p \left(\left| \fr{\hat{R}(T_m)-{\mathsf E} \hat{R}(T_m)}{n}\right| >\eps \right) 
\leq \fr{c_5}{n^{1+c_6}}\,, 
$$
где константы $c_5$, $c_6$ положительны. Отсюда
$$
\sum\limits_{n=1}^{\infty}\p \left(\left|\fr{\hat{R}(T_m)-{\mathsf E} 
\hat{R}(T_m)}{n}\right| >\eps \right) < \infty,
$$
и по теореме~1.3.4 из~\cite{Serfling2002} 
$$
\left(\hat{R}(T_m)-{\mathsf E}\hat{R}(T_m)\right)n^{-1} \to 0~\mbox{п.~в.}
$$



Покажем теперь, что  $U(\hat{t}_F) \, n^{-1}\hm \to 0$ п.~в. Доказательство 
проведено для $\mu \hm\in m_p[\eta_n]$, в~случае $\mu\hm \in l_0[\eta_n]$ 
доказательство аналогично.
Аналогично формуле~(\ref{M1}), для произвольного $\eps\hm>0$ в~терминах тео\-ре\-мы~1 имеем
\begin{multline*}
\p \left( \fr{|U(\hat{t}_F)|}{n}> 4\eps \right) \leq \p\left(\hat{t}_F 
\leq t_{\kappa_n}\right) +{}\\
{}+ \p\left(\fr{\sup\nolimits_{T\in [t_{\kappa_n}, T_U]} |{\mathsf E} 
U(T)|}{n}>\eps\right)+{}\\
{}+ \p \left(\sup\limits_{T\in [t_{\kappa_n}, T_U]} |Z_1(T)| > \eps n\right) +{}
\end{multline*}

\noindent
\begin{multline}
{}+ \p  \left(\sup\limits_{j \in [0, n-1]} |Z_2(T_j^{\prime})| > \eps n\right) +{}\\
{}+ \p \left(\sup\limits_{\substack{j \in [0, n-1] \\ T\in 
[T_j^{\prime},T_j^{\prime}+d_n]}} |Z_2(T)-Z_2(T_j^{\prime})| > \eps n\right).
\label{M1SC}
\end{multline}
Применяя рассуждения, аналогичные приведенным в~доказательстве теоремы~1, можно показать, что
$$
\sup\limits_{T\in [t_{\kappa_n}, T_U]} |{\mathsf E} U(T)| = o(n); \enskip
\sup\limits_{T\in [t_{\kappa_n}, T_U]} |Z_1(T)|  = o(n),
$$
откуда следует, что второе и~третье слагаемые в~(\ref{M1SC}) обращаются в~ноль 
для всех достаточно больших~$n$.

Для некоторых положительных констант  $c_7$ и~$c_8$ первое, четвертое и~пятое 
слагаемые  в~(\ref{M1SC}) не превышают $c_7 n^{-1-c_8}$ для всех достаточно 
боль\-ших~$n$, что можно показать с~помощью ограничения на $\alpha(\cdot)$ из 
условия и~рассуждений, аналогичных приведенным при выводе соответственно формул~(\ref{M1next}), (\ref{M5}) и~(\ref{M6}), с~тем отличием, что при применении 
леммы~4 полагается $m \hm= (\ln n)^3$.

Из доказанного следует, что
$$
\sum\limits_{n=1}^{\infty}\p \left( \fr{|U(\hat{t}_F)|}{n}> 4\eps \right) 
< \infty,
$$
и по теореме~1.3.4 из~\cite{Serfling2002} $U(\hat{t}_F) \, n^{-1} \to 0$ п.~в., 
что завершает доказательство теоремы.~\hfill$\square$



{\small\frenchspacing
 {\baselineskip=11.5pt
 %\addcontentsline{toc}{section}{References}
 \begin{thebibliography}{99}
\bibitem{FDRImage}
\Au{Krylov V.\,A., Moser~G., Serpico~S.\,B., Zerubia~J.}
False discovery rate approach to unsupervised image change detection~// IEEE 
T. Image Process., 2016. Vol.~25. No.\,10. P.~4704--4718. doi: 10.1109/TIP.2016.2593340.

\bibitem{MultipleTesting} %2
\Au{Menyhart~O., Weltz~B., Gyorffy~B.}
MultipleTesting.com: A~tool for life science researchers for multiple hypothesis 
testing correction~// PLoS One, 2021. Vol.~16. No.\,6. Art.~0245824. doi: 10.1371/journal.pone.0245824.

\bibitem{AdaptingFDR} %3
\Au{Abramovich~F., Benjamini~Y., Donoho~D., Johnstone~I.}
Adapting to unknown sparsity by controlling the false discovery rate~// Ann. Stat., 2006. Vol.~34. No.\,2. P.~584--653.
doi: 10.1214/009053606000000074.

\bibitem{ZasShe17} %4
\Au{Заспа~А.\,Ю., Шестаков~О.\,В.}
Состоятельность оценки риска при множественной проверке гипотез с~FDR-по\-ро\-гом~// 
Вестник ТвГУ. Сер. Прикладная математика, 2017. Вып.~1. С.~5--16.
doi: 10.26456/vtpmk119. EDN: YFYJXT.

\bibitem{Mathematics2020} %5
\Au{Palionnaya~S.\,I., Shestakov~O.\,V.}
Asymptotic properties of MSE estimate for the false discovery rate controlling 
procedures in multiple hypothesis testing // Mathematics, 2020. Vol.~8. No.~11. 
Art.~1913. 11~p. doi: 10.3390/ math8111913.

\bibitem{Shestakov2021-1} %6
\Au{Шестаков~О.\,В.}
Анализ несмещенной оценки среднеквадратичного риска метода блочной пороговой 
обработки~// Информатика и~её применения, 2021. Т.~15. Вып.~2. С.~30--35.
doi: 10.14357/19922264210205. EDN: DSQQAU.

\bibitem{Shestakov2021-2} %7
\Au{Шестаков~О.\,В.}
Пороговые функции в~методах подавления шума, основанных на вейв\-лет-раз\-ло\-же\-нии 
сигнала~// Информатика и~её применения, 2021. Т.~15. Вып.~3. С.~51--56.
doi: 10.14357/19922264210307. EDN: WSEAYG.

\bibitem{Shestakov2022} %8
\Au{Шестаков~О.\,В.}
Несмещенная оценка риска пороговой обработки с~двумя пороговыми значениями~// 
Информатика и~её применения, 2022. Т.~16. Вып.~4. С.~14--19.
doi: 10.14357/19922264220403. EDN: \mbox{DZBVLC}.

\bibitem{ResultsOnFDRUnderDependence} %9
\Au{Farcomeni~A.}
Some results on the control of the false discovery rate under dependence~// 
Scand. J. Stat., 2007. Vol.~34. No.\,2. P.~275--297.
doi: 10.1111/j.1467-9469.2006.00530.x.

\bibitem{VorontsovShestakov2023} %10
\Au{Воронцов~М.\,О., Шестаков~О.\,В.}
Среднеквадратичный риск FDR-про\-це\-ду\-ры в~условиях слабой за\-ви\-си\-мости~// 
Информатика и~её применения, 2023. Т.~17. Вып.~2. С.~34--40.
doi: 10.14357/19922264230205. EDN: AVJZDX.

\bibitem{Vorontsov2024} %11
\Au{Воронцов~М.\,О.}
Анализ среднеквадратичного риска при использовании методов множественной 
проверки гипотез для выбора параметров пороговой обработки в~условиях слабой 
зависимости~// Вестник Московского университета. Сер. 15: Вычислительная 
математика и~кибернетика, 2024. №\,2. С.~18--24.

\bibitem{Bosq} %12
\Au{Bosq~D.}
Nonparametric statistics for stochastic processes: Estimation and prediction.~--- 
Lecture notes in statistics ser.~--- New York, NY, USA: Springer, 1996. Vol.~110. 
188~p.

\bibitem{Mallat} %13
\Au{Mallat~S.}
A wavelet tour of signal processing.~--- New York, NY, USA: Academic Press, 1999. 
857~p.

\bibitem{spatialAdaptation} %14
\Au{Donoho~D., Johnstone~I.}
Ideal spatial adaptation via wavelet shrinkage~// Biometrika, 1994. Vol.~81. 
No.\,3. P.~425--455. doi: 10.1093/biomet/81.3.425.

\bibitem{AdaptingSURE} %15
\Au{Donoho D., Johnstone I.\,M.}
Adapting to unknown smoothness via wavelet shrinkage~// J.~Amer. Stat. Assoc., 
1995. Vol.~90. P.~1200--1224.

\bibitem{ExactRisk} %16
\Au{Marron J.\,S., Adak~S., Johnstone~I.\,M., Neumann~M.\,H., Patil~P.}
Exact risk analysis of wavelet regression~// J.~Comput. Graph. Stat., 1998. 
Vol.~7. P.~278--309. doi: 10.1080/ 10618600.1998.10474777.

\bibitem{Jansen} %17
\Au{Jansen~M.}
Noise reduction by wavelet thresholding.~-- Lecture notes in statistics ser.~--- 
New York, NY, USA: Springer, 2001. Vol.~161. 217~p.

\bibitem{KuShe2016_1} %18
\Au{Кудрявцев~А.\,А., Шестаков~О.\,В.}
Асимптотическое поведение порога, минимизирующего усредненную\linebreak вероятность ошибки 
вычисления вейв\-лет-ко\-эф\-фи\-ци\-ен\-тов~// Докл. Акад. наук, 2016. Т.~468. №\,5. 
С.~487--491.

\bibitem{KuShe2016_2} %19
\Au{Кудрявцев~А.\,А., Шестаков~О.\,В.}
Асимптотически оптимальная пороговая обработка вейв\-лет-ко\-эф\-фи\-ци\-ен\-тов в~моделях с~негауссовым распределением шума~// Докл. Акад. наук, 2016. Т.~471. №\,1. 
С.~11--15.



\bibitem{Eroshenko} %20
\Au{Ерошенко~А.\,А.}
Статистические свойства оценок сигналов и~изображений при пороговой обработке 
коэффициентов в~вейв\-лет-раз\-ло\-же\-ни\-ях: Дис.\ \ldots\ канд. физ.-мат. наук.~--- 
М.: МГУ, 2015. 82~с.

\bibitem{Peligrad} %21
\Au{Peligrad~M.}
On the asymptotic normality of sequences of weak dependent random variables~// 
J. Theor. Probab., 1996. Vol.~9. No.\,3. P.~703--715. doi: 10.1007/BF02214083.

\bibitem{Serfling2002} %22
\Au{Serfling~R.\,J.}
Approximation theorems of mathematical statistics.~--- New York, NY, USA: John Wiley \&~Sons, Inc., 2002. 371~p.

\end{thebibliography}

 }
 }

\end{multicols}

\vspace*{-6pt}

\hfill{\small\textit{Поступила в~редакцию 21.05.24}}

\vspace*{8pt}

%\pagebreak

%\newpage

%\vspace*{-28pt}

\hrule

\vspace*{2pt}

\hrule



\def\tit{ASYMPTOTIC NORMALITY AND STRONG CONSISTENCY\\ OF~RISK ESTIMATE WHEN USING THE~FDR THRESHOLD\\ UNDER WEAK DEPENDENCE CONDITION}


\def\titkol{Asymptotic normality and strong consistency of~risk estimate when using the~FDR threshold under weak dependence condition}


\def\aut{M.\,O.~Vorontsov$^{1,2}$ and~O.\,V.~Shestakov$^{1,2,3}$}

\def\autkol{M.\,O.~Vorontsov and~O.\,V.~Shestakov}

\titel{\tit}{\aut}{\autkol}{\titkol}

\vspace*{-13pt}


\noindent
$^{1}$Department of Mathematical Statistics, Faculty of Computational Mathematics and Cybernetics,
 M.\,V.~Lo\-mo-\linebreak
 $\hphantom{^1}$nosov Moscow State University, 1-52~Leninskie Gory, GSP-1, Moscow 119991, Russian Federation

\noindent
$^{2}$Moscow Center for Fundamental and Applied Mathematics, M.\,V.~Lomonosov Moscow State University,\linebreak
$\hphantom{^1}$1~Leninskie Gory, GSP-1, Moscow 119991, Russian Federation

\noindent
$^{3}$Federal Research Center ``Computer Science and Control'' of the Russian Academy of Sciences, 44-2~Vavilov\linebreak
$\hphantom{^1}$Str., Moscow 119333, Russian Federation


\def\leftfootline{\small{\textbf{\thepage}
\hfill INFORMATIKA I EE PRIMENENIYA~--- INFORMATICS AND
APPLICATIONS\ \ \ 2024\ \ \ volume~18\ \ \ issue\ 3}
}%
 \def\rightfootline{\small{INFORMATIKA I EE PRIMENENIYA~---
INFORMATICS AND APPLICATIONS\ \ \ 2024\ \ \ volume~18\ \ \ issue\ 3
\hfill \textbf{\thepage}}}

\vspace*{2pt}






\Abste{An approach to solving the problem of noise removal in a large array of sparse data is considered
 based on the method of controlling the average proportion of false hypothesis rejections (False Discovery Rate, FDR). 
 This approach is equivalent to threshold processing procedures that remove array components whose values do not exceed 
 some specified threshold. The observations in the model are considered weakly dependent. To control the\linebreak\vspace*{-12pt}}
 
 \Abstend{degree of dependence, 
 restrictions on the strong mixing coefficient and the maximum correlation coefficient are used. The mean-square risk is 
 used as a measure of the effectiveness of the considered approach. It is possible to calculate the risk value only on the test data;
  therefore, its statistical estimate is considered in the work and its properties are investigated. The asymptotic normality and
   strong consistency of the risk estimate are proved when using the FDR threshold under conditions of weak dependence in the data.}

\KWE{thresholding; multiple hypothesis testing; risk estimate}

\DOI{10.14357/19922264240309}{ZOQVTO}

%\vspace*{-12pt}


    
   %   \Ack

%\vspace*{-3pt}
%\noindent



  \begin{multicols}{2}

\renewcommand{\bibname}{\protect\rmfamily References}
%\renewcommand{\bibname}{\large\protect\rm References}

{\small\frenchspacing
 {\baselineskip=10.8pt
 \addcontentsline{toc}{section}{References}
 \begin{thebibliography}{99} 

%1
\bibitem{FDRImage-1}
\Aue{Krylov, V.\,A., G.~Moser, S.\,B.~Serpico, and J.~Zerubia.} 2016. 
False discovery rate approach to unsupervised image change detection. 
\textit{IEEE T. Image Process.} 25(10):4704--4718. doi: 10.1109/TIP.2016.2593340.

%2
\bibitem{MultipleTesting-1}
\Aue{Menyhart, O., B.~Weltz, and B.~Gyorffy.} 2021. 
MultipleTesting.com: A~tool for life science researchers for multiple hypothesis testing correction. 
\textit{PLoS One} 16(6):0245824. 
doi: 10.1371/journal.pone.0245824.

%3
\bibitem{AdaptingFDR-1}
\Aue{Abramovich, F., Y.~Benjamini, D.~Donoho, and I.\,M.~Johnstone.} 2006. 
Adapting to unknown sparsity by controlling the false discovery rate. 
\textit{Ann. Stat.} 34(2):584--653. 
doi: 10.1214/009053606000000074.


%4
\bibitem{ZasShe17-1}
\Aue{Zaspa, A.\,Yu., and O.\,V.~Shestakov.} 2017.
Sostoyatel'nost' otsenki riska pri mnozhestvennoy proverke gipotez s~FDR-porogom
 [Consistency of the risk estimate of the multiple hypothesis testing with the FDR threshold]. 
\textit{Vestnik TvGU. Ser.: Prikladnaya matematika} [Herald of Tver State University. Ser. Applied Mathematics] 1:5--16.
doi: 10.26456/vtpmk119. EDN: YFYJXT.

%5
\bibitem{Mathematics2020-1}
\Aue{Palionnaya, S.\,I., and O.\,V.~Shestakov.} 2020. 
Asymptotic properties of MSE estimate for the false discovery rate controlling procedures in multiple hypothesis testing. 
\textit{Mathematics} 8(11):1913. 11~p.
doi: 10.3390/math8111913.

%6
\bibitem{Shestakov2021-1-1}
\Aue{Shestakov, O.\,V.} 2021.
Analiz nesmeshchennoy otsenki srednekvadratichnogo riska metoda blochnoy po\-ro\-go\-voy obrabotki 
[Analysis of the unbiased mean-square risk estimate of the block thresholding method]. 
\textit{Informatika i~ee Primeneniya~--- Inform. Appl.} 15(2):30--35.
doi: 10.14357/19922264210205. EDN: DSQQAU.

%7
\bibitem{Shestakov2021-2-1}
\Aue{Shestakov, O.\,V.} 2021.
Porogovye funktsii v~metodakh podavleniya shuma, osnovannykh na veyvlet-razlozhenii signala 
[Thresholding functions in the noise suppression methods based on the wavelet expansion of the signal]. 
\textit{Informatika i~ee Primeneniya~--- Inform. Appl.} 15(3):51--56.
doi: 10.14357/19922264210307. EDN: WSEAYG.

%8
\bibitem{Shestakov2022-1}
\Aue{Shestakov, O.\,V.} 2022.
Nesmeshchennaya otsenka riska porogovoy obrabotki s dvumya porogovymi znacheniyami [Unbiased thresholding risk estimate with two threshold values]. 
\textit{Informatika i~ee Primeneniya~--- Inform. Appl.} 16(4):14--19.
doi: 10.14357/19922264220403. EDN: DZBVLC.

%9
\bibitem{ResultsOnFDRUnderDependence-1}
\Aue{Farcomeni, A.} 2007. Some results on the control of the false discovery rate under dependence. 
\textit{Scand. J. Stat.} 34(2):275--297. 
doi: 10.1111/j.1467-9469.2006.00530.x.

%10
\bibitem{VorontsovShestakov2023-1}
\Aue{Vorontsov, M.\,O., and O.\,V.~Shestakov.} 2023.
Sred\-ne\-kvad\-ra\-tich\-nyy risk FDR-protsedury v~usloviyakh slaboy za\-vi\-si\-mosti [Mean-square risk of the FDR procedure under weak dependence]. 
\textit{Informatika i~ee Primeneniya~--- Inform. Appl.} 17(2):34--40.
doi: 10.14357/19922264230205. EDN: AVJZDX.

%11
\bibitem{Vorontsov2024-1}
\Aue{Vorontsov, M.\,O.} 2024. 
RMS risk analysis when using multiple hypothesis testing select parameters of thresholding under conditions of weak dependence. 
\textit{Moscow University Computational Mathematics Cybernetics} 48:91--97. 
doi: 10.3103/S027864192470002X.

%12
\bibitem{Bosq-1}
\Aue{Bosq, D.} 1996. 
\textit{Nonparametric statistics for stochastic processes: Estimation and prediction}. 
Lecture notes in statistics ser. New York, NY: Springer Verlag. Vol.~110. 188~p.

%13
\bibitem{Mallat-1}
\Aue{Mallat, S.} 1999. 
\textit{A wavelet tour of signal processing}. New York, NY: Academic Press. 857~p.

%14
\bibitem{spatialAdaptation-1}
\Aue{Donoho, D., and I.\,M.~Johnstone.} 1994. 
Ideal spatial adaptation via wavelet shrinkage. 
\textit{Biometrika} 81(3):425--455. doi: 10.1093/biomet/81.3.425.

%15
\bibitem{AdaptingSURE-1}
\Aue{Donoho, D., and I.\,M.~Johnstone.} 1995. 
Adapting to unknown smoothness via wavelet shrinkage. 
\textit{J. Am. Stat. Assoc.} 90(432):1200--1224. doi: 10.1080/01621459. 1995.10476626.

%16
\bibitem{ExactRisk-1}
\Aue{Marron, J.\,S., S.~Adak, I.\,M.~Johnstone, M.\,H.~Neumann, and P.~Patil.} 1998. 
Exact risk analysis of wavelet regression. 
\textit{J.~Comput. Graph. Stat.} 7(3):278-309. doi: 10.1080/ 10618600.1998.10474777.

%17
\bibitem{Jansen-1}
\Aue{Jansen, M.} 2001. 
\textit{Noise reduction by wavelet thresholding}. Lecture notes in statistics ser. New York, NY: Springer Verlag. Vol.~161. 217~p.

%18
\bibitem{KuShe2016_1-1}
\Aue{Kudryavtsev, A.\,A., and O.\,V.~Shestakov.} 2016. 
Asymptotic behavior of the threshold minimizing the average probability of error in calculation of wavelet coefficients. 
\textit{Dokl. Math.} 93(3):295--299.
doi: 10.1134/S1064562416030212. EDN: WUMUEV. 

%19
\bibitem{KuShe2016_2-1}
\Aue{Kudryavtsev, A.\,A., and O.\,V.~Shestakov.} 2016. 
Asymptotically optimal wavelet thresholding in the models with non-Gaussian noise distributions. 
\textit{Dokl. Math.} 94(3):615--619.
doi: 10.1134/S1064562416060028. EDN: YUYVUP.




%20
\bibitem{Eroshenko-1}
\Aue{Eroshenko, A.\,A.} 2015. Statisticheskie svoystva otsenok signalov i~izobrazheniy pri porogovoy obrabotke ko\-ef\-fi\-tsi\-en\-tov 
v~veyvlet-razlozheniyakh 
[Statistical properties of signal and image estimates under thresholding of coefficients in wavelet decompositions]. Moscow: MSU. PhD Diss. 82~p.

%21
\bibitem{Peligrad-1}
\Aue{Peligrad, M.} 1996. 
On the asymptotic normality of sequences of weak dependent random variables. 
\textit{J. Theor. Probab.} 9(3):703--715. doi: 10.1007/BF02214083.

%22
\bibitem{Serfling2002-1}
\Aue{Serfling, R.\,J.} 2002. 
\textit{Approximation theorems of mathematical statistics}. New York, NY: John Wiley \&~Sons. 371~p.
\end{thebibliography}

 }
 }

\end{multicols}

\vspace*{-6pt}

\hfill{\small\textit{Received May 21, 2024}} 

%\vspace*{-18pt}

\Contr

\vspace*{-3pt}


\noindent
\textbf{Vorontsov Mikhail O.} (b.\ 1996)~--- PhD student, Department of Mathematical Statistics, 
Faculty of Computational Mathematics and Cybernetics, M.\,V.~Lomonosov Moscow State University, 1-52~Leninskie Gory, GSP-1, Moscow 119991, Russian Federation;  
mathematician, Moscow Center for Fundamental and Applied Mathematics, M.\,V.~Lomonosov Moscow State University, 1~Leninskie Gory, GSP-1, Moscow 119991, Russian Federation;
\mbox{m.vtsov@mail.ru}

\vspace*{6pt}

\noindent
\textbf{Shestakov Oleg V.} (b.\ 1976)~--- Doctor of Science in physics and mathematics, professor, Department of Mathematical Statistics,
 Faculty of Computational Mathematics and Cybernetics, M.\,V.~Lomonosov Moscow State University, 1-52~Leninskie Gory, GSP-1, Moscow 119991, Russian Federation; 
 senior scientist, Federal Research Center ``Computer Science and Control'' of the Russian Academy of Sciences, 44-2~Vavilov Str., Moscow 119333, 
 Russian Federation; leading scientist, Moscow Center for Fundamental and Applied Mathematics, M.\,V.~Lomonosov Moscow State University, 
 1~Leninskie Gory, GSP-1, Moscow 119991, Russian Federation; \mbox{oshestakov@cs.msu.su}


\label{end\stat}

\renewcommand{\bibname}{\protect\rm Литература}  %3
\def\stat{bazilevsky}

\def\tit{МЕТОД ВЫПРЯМЛЕНИЯ\\ ИСКАЖЕННЫХ ИЗ-ЗА~МУЛЬТИКОЛЛИНЕАРНОСТИ\\ 
КОЭФФИЦИЕНТОВ В~РЕГРЕССИОННЫХ МОДЕЛЯХ}

\def\titkol{Метод выпрямления искаженных из-за~мультиколлинеарности 
коэффициентов в~регрессионных моделях}

\def\aut{М.\,П.~Базилевский$^1$}

\def\autkol{М.\,П.~Базилевский}

\titel{\tit}{\aut}{\autkol}{\titkol}

\index{Базилевский М.\,П.}
\index{Bazilevskiy M.\,P.}

%{\renewcommand{\thefootnote}{\fnsymbol{footnote}} \footnotetext[1]
%{Работа выполнена при финансовой поддержке РФФИ (проект 16-29-09458~офи\_м).}}


\renewcommand{\thefootnote}{\arabic{footnote}}
\footnotetext[1]{Иркутский государственный университет путей сообщения, кафедра математики, 
\mbox{mik2178@yandex.ru}}


\vspace*{-12pt}



  \Abst{При построении регрессионной модели из-за сильной мультиколлинеарности 
объясняющих переменных происходит искажение ее коэффициентов, в~частности их знаков, 
что негативно сказывается на интерпретационных качествах такой регрессии. Статья посвящена 
разработке метода выпрямления искаженных из-за мультиколлинеарности коэффициентов. 
В~основе этого метода лежит свойство, которым обладают ранее предложенные автором 
модели полносвязной линейной регрессии. Исследована нелинейная система, по которой 
осуществляется оценивание полносвязных регрессий. Показано, что решение этой системы 
может быть получено численно с~помощью метода простых итераций. Предложен способ 
выбора неизвестных лямбда-параметров в~полносвязной регрессии. Установлено, что 
в~многофакторных полносвязных моделях при сильной корреляции всех факторов знаки 
коэффициентов при переменных во вторичном уравнении совпадают с~соответствующими 
знаками коэффициентов корреляции. Для выпрямления искаженных коэффициентов на основе 
проведенного исследования разработан алгоритм <<Selection~B>>. Разработанный метод 
выпрямления успешно продемонстрирован на примере моделирования валового внутреннего продукта (ВВП) России.}
  
  \KW{регрессионный анализ; модель полносвязной линейной регрессии; 
мультиколлинеарность; интерпретация; численный метод; ВВП России}

\DOI{10.14357/19922264210209}

%\vspace*{-3pt}


\vskip 10pt plus 9pt minus 6pt

\thispagestyle{headings}

\begin{multicols}{2}

\label{st\stat}
  
\section{Введение}

  При оценивании неизвестных параметров регрессионных моделей, например 
с~по\-мощью метода наименьших квадратов (МНК), на практике часто 
приходится сталкиваться с~проблемой мультиколлинеарности~[1, 2]. Это 
негативное явление возникает из-за наличия сильной корреляции между двумя 
или более независимыми переменными. Мультиколлинеарность факторов 
приводит к~искажению коэффициентов в~уравнении регрессии. В~частности, их 
знаки могут противоречить теоретическим предпосылкам решаемой задачи. 
Поэтому построенная при мультиколлинеарности регрессионная модель остается 
годной в~лучшем случае только для прогнозирования, но никак не для 
интерпретации и~принятия ка\-ких-ли\-бо правильных управленческих решений.
  
  Проблема мультиколлинеарности на сегодня еще окончательно не решена. 
Существуют лишь несколько основных подходов к~ее устранению~[3,~4].
{\looseness=-1

}
  
  Во-первых, это метод исключения~\cite{4-baz}. Он заключается в~том, что на 
основе матрицы парных коэффициентов корреляции определяются пары сильно 
коррелированных объясняющих переменных. Затем из каждой пары исключается 
тот фактор, который слабее коррелирует с~зависимой переменной. После чего по 
оставшимся факторам оценивается регрессионная модель. Недостаток данного 
подхода состоит в~том, что в~полученном уравнении из-за исключения нельзя 
изучать совместное влияние всех исходных объясняющих переменных на 
объясняемую.
  
  Во-вторых, метод главных компонент~[5]. С~помощью этого способа 
происходит формирование\linebreak новых и~не коррелирующих между собой 
переменных~--- главных компонент, являющихся линейными комбинациями 
старых переменных. К~сожалению, в~этом случае возникает проблема 
с~\mbox{интерпретацией} главных компонент.
  
  В-третьих, ридж-регрессия~[6]. В~этом случае в~формулу для  
МНК-оце\-ни\-ва\-ния регрессии до\-бав\-ля\-ет\-ся так называемый коэффициент 
регуляризации, который решает проблему мультиколлинеарности. Однако нет 
четких правил для выбора этого коэффициента. И~нет гарантии, что в~полученной 
модели коэффициенты будут удовлетворять содержательному смыслу задачи.
  
  Как справедливо отмечено в~работе~[7], все эти методы ориентированы на 
устранение только вычислительных проб\-лем. Но проблема, связанная 
с~построением интерпретируемых при мультиколлинеарности регрессионных 
моделей, остается нерешенной.
  
  Целью данной работы ставилась разработка метода выпрямления искаженных 
из-за мультиколлинеарности коэффициентов линейных регрессионных моделей. 
Основой для этого метода послужило замеченное автором свойство 
двухфакторных полносвязных регрессий~[8, 9], состоящее в~том, что в~их 
вторичных уравнениях знаки коэффициентов при объясняющих переменных 
совпадают с~соответствующими знаками коэффициентов корреляции. Это же 
свойство может быть справедливо и~для многофакторных моделей полносвязной 
линейной регрессии, впервые предложенных в~работе~[10].
  
\section{Многофакторная модель полносвязной линейной регрессии 
без~выходной переменной}

  Пусть $x_{ij}$, $i\hm=\overline{1,n}$, $j\hm=\overline{1,m}$,~--- наблюдаемые 
значения~$m$~входных переменных $x_1, x_2, \ldots , x_m$. Предположим, что 
существуют их <<истинные>> значения $x^*_{i1}, x^*_{i2},\ldots , x^*_{im}$, 
$i\hm=\overline{1,n}$, связанные с~наблюдаемыми значениями соотношениями
  \begin{equation}
  x_{ij}= x^*_{ij}+\varepsilon_i^{(x_j)}\,,\enskip i=\overline{1,n}\,,\ 
j=\overline{1,m}\,.
  \label{e1-baz}
  \end{equation}
  
  Предположим, что между <<истинными>> переменными $x_1^*, x_2^*, \ldots , 
x_m^*$ имеют место функциональные зависимости
  \begin{equation}
  x_j^*=a_j+b_jx_m^*\,,\enskip j=\overline{1,m-1}\,,
  \label{e2-baz}
  \end{equation}
где $a_j$ и~$b_j$, $j\hm=\overline{1,m-1}$,~--- неизвестные параметры.

  Совокупность уравнений~(\ref{e1-baz}) и~(\ref{e2-baz}) называется 
многофакторной моделью полносвязной линейной регрессии без выходной 
переменной~\cite{10-baz}.
  %
  Для ее оценивания применим взвешенный метод наименьших полных 
квадратов (ВМНПК):
  \begin{multline}
  S=\lambda_1\sum\limits^n_{i=1} \left( x_{i1}-a_1-b_1x^*_{im}\right)^2 
+{}\\
{}+
\lambda_2\sum\limits^n_{i=1} \left( x_{i2}-a_2-b_2x^*_{im}\right)^2+\cdots {}\\
  {}\cdots\ +\lambda_{m-1}\sum\limits^n_{i=1}\left( x_{i,m-1}-a_{m-1}-b_{m-1} 
x^*_{im}\right)^2+{}\\
{}+\sum\limits^n_{i=1} \left( x_{im} -x^*_{im}\right)^2\to \min\,,
  \label{e3-baz}
  \end{multline}
где $\lambda_1, \lambda_2,\ldots , \lambda_{m-1}$~--- положительные весовые 
коэффициенты (лямбда-параметры).
  
  В работе~\cite{10-baz} показано, что если лямб\-да-па\-ра\-мет\-ры известны, то 
решение задачи~(\ref{e3-baz}) осуществляется по следующему алгоритму.
  \begin{enumerate}[1.]
\item Находятся оценки $\tilde{b}_1, \tilde{b}_2,\ldots , \tilde{b}_{m-1}$ 
параметров $b_1, b_2, \ldots, b_{m-1}$. Для этого численно решается нелинейная 
система вида:
\begin{multline} 
b_p\left( D_{x_m}+\sum\limits_{j=1}^{m-1} \lambda^2_j b_j^2 
D_{x_j} +{}\right.\\
{}+2\sum\limits^{m-2}_{j_1=1} \sum\limits^{m-1}_{j_2=j_1+1} 
\lambda_{j_1} \lambda_{j_2} b_{j_1}b_{j_2} 
K_{x_{j_1}x_{j_2}}+{}\\
\left.{}+2\sum\limits_{j=1}^{m-1} \lambda_j b_j 
K_{x_jx_m}\right)= \left( 1+\sum\limits_{j=1}^{m-1} \lambda_j b_j^2\right)\times{}\\
\hspace*{-5mm}{}\times \left( 
\sum\limits_{j=1}^{m-1} \lambda_j b_j K_{x_jx_p}+K_{x_mx_p}\right),\enskip 
  p=\overline{1,m-1}\,.\!\!
  \label{e4-baz}
  \end{multline}
\item Определяются оценки $\tilde{a}_1, \tilde{a}_2, \ldots , \tilde{a}_{m-1}$ 
па\-ра\-мет\-ров $a_1, a_2, \ldots , a_{m-1}$ по формулам:
\begin{equation}
\tilde{a}_j= \overline{x}_j -\tilde{b}_j \overline{x}_m\,,\enskip j\hm=\overline{1,m-
1}\,.
\label{e5-baz}
\end{equation}
\item Вычисляются оценки <<истинных>> значений переменной~$x_m$ по 
формулам:
\begin{multline}
\tilde{x}_{im}^* =\left( 1+\sum\limits^{m-1}_{j=1} \lambda_j \tilde{b}_j^2\right)^{-
1} \left( -\sum\limits_{j=1}^{m-1} \lambda_j \tilde{a}_j \tilde{b}_j 
+{}\right.\\
\left.{}+\sum\limits_{j=1}^{m-1} \lambda_j \tilde{b}_j x_{ij} +x_{im}\right),\enskip
i=\overline{1,n}\,.
\label{e6-baz}
\end{multline}
  \end{enumerate}
  
  Очевидно, что если абсолютные значения парных коэффициентов корреляции 
переменных $x_1, x_2,\ldots , x_m$ равны~1, то при оценивании полносвязной 
регрессии по критерию~(\ref{e3-baz}) все остатки будут равны~0, а ее 
оценки~$\tilde{b}_i$, $i\hm=\overline{1,m-1}$, будут совпадать  
с~МНК-оцен\-ка\-ми соответствующих парных регрессий. А~знаки этих оценок 
согласуются со знаками соответствующих коэффициентов 
корреляции~$r_{x_ix_m}$, $i\hm=\overline{1,m-1}$, т.\,е.\ справедливы условия 
$\tilde{b}_i r_{x_ix_m}\hm>0$, $i\hm=\overline{1,m-1}$. Значит, они будут справедливы и~при сильной 
корреляции факторов.

\section{Численный метод решения нелинейной  
системы~(\ref{e4-baz})}

  Систему~(\ref{e4-baz}) нетрудно привести к~виду:
  \begin{equation}
  H_p b_p^2 +B_p b_p +C_p=0\,,\enskip p=\overline{1,m-1}\,,
  \label{e7-baz}
  \end{equation}
  где 
 \begin{align*}
  H_p&=\lambda_p\!\left( K_{x_mx_p}+\sum\limits_{j\in \{1,\ldots, m-1\}\backslash \{p\}} 
\!\!\!\!\!\!\!\!\!\!\lambda_j b_j K_{x_jx_p}\right)\,;
 \\
  B_p&=D_{x_m}+\sum\limits_{ j\in \{1,\ldots, m-1\}\backslash\{p\}} \!\!\!\!\!\!\!\!\!\!\lambda_j^2 b_j^2 
D_{x_j} + {}\\
  &\hspace*{-20pt}{}+ 2\!\!\!\!\!\!\sum\limits_{ j_1\in \{1,\ldots, m-2\}\backslash \{p\}} \sum\limits_{ j_2\in 
\{j_1+1,\ldots, m-1\}\backslash \{p\}}\!\!\!\!\!\!\!\!\!\!\!\!\!\!\!
 \lambda_{j_1} \lambda_{j_2} b_{j_1} b_{j_2} 
K_{x_{j_1} x_{j_2}}+{}\\
  &{}+2\!\!\! \sum\limits_{ j\in \{1,\ldots, m-1\}\backslash \{p\}} \!\!\!\!\!\!\!\!\!\!\lambda_jb_jK_{x_jx_m}-
\lambda_pD_{x_p} -\lambda_p D_{x_p} \times{}\\
&\hspace*{35mm}{}\times \sum\limits_{ j\in \{1,\ldots, m-1\}\backslash 
\{p\}} \!\!\!\!\!\!\!\!\!\!\lambda_j b_j^2\,;
  \\
  C_p&=-\left( 1+\sum\limits_{ j\in \{1,\ldots, m-1\}\backslash \{p\}}  \!\!\!\!\!\!\!\!\!\!
\lambda_jb_j^2\right)\times{}\\
&\hspace*{15mm}{}\times \left( K_{x_mx_p} +\sum\limits_{ j\in \{1,\ldots, m-1\}\backslash 
\{p\}} \!\!\!\!\!\!\!\!\!\!\lambda_j b_j K_{x_j x_p}\right)\,.
  \end{align*}
    Тогда систему~(\ref{e7-baz}) можно представить в~виде:
  \begin{equation}
  H_p\left( b_p-b^*_{p,1}\right) \left( b_p-b^*_{p,2}\right)=0\,,\enskip 
p=\overline{1,m-1}\,,
  \label{e8-baz}
  \end{equation}
где $b_{p,1}^*\hm= (-B_p-\sqrt{\mathrm{Disc}_p})/(2H_p)$ и~$b^*_{p,2}\hm= 
(-B_p\hm+\sqrt{\mathrm{Disc}_p})/(2H_p)$~--- корни $p$-го квадратного трехчлена 
системы~(\ref{e7-baz}); $\mathrm{Disc}_p\hm= B_p^2\hm- 4H_p C_p$~---  
дискриминанты $p$-го квадратного трехчлена системы~(\ref{e7-baz}), которые, 
как видно, всегда положительны.

  Понятно, что система~(\ref{e8-baz}) равносильна совокупности $2^{m-1}$ 
систем
  \begin{multline}
  \left\{
  \begin{array}{l}
  b_1=b^*_{1,1};\\[3pt]
  b_2=b^*_{2,1};\\[3pt]
   \ldots \\[3pt]
  b_{m-1}=b^*_{m-1,1}\,;
  \end{array}\right.
  \enskip 
  \left\{
  \begin{array}{l}
  b_1=b^*_{1,2};\\[3pt]
  b_2=b^*_{2,1};\\[3pt]
   \ldots \\[3pt]
  b_{m-1}=b^*_{m-1,1}\,;
  \end{array}
  \right.\quad \cdots
\\
\cdots \quad \left\{
  \begin{array}{l}
  b_1=b^*_{1,2};\\[3pt]
  b_2=b^*_{2,2};\\[3pt]
   \ldots \\[3pt]
  b_{m-1}=b^*_{m-1,2}\,.
  \end{array}\right.
  \label{e9-baz}
  \end{multline}
  
  Покажем, что решение задачи~(\ref{e3-baz}) удовлетворяет только системе
  \begin{equation}
  \left\{ 
  \begin{array}{l}
  b_1=b^*_{1,2}\,;\\[3pt]
  b_2=b^*_{2,2}\,;\\[3pt]
  \ldots\\[3pt]
  b_{m-1} =b^*_{m-1,2}\,.
  \end{array}
  \right.
  \label{e10-baz}
  \end{equation}
  
  Вторые частные производные функции~(\ref{e3-baz}) имеют вид:
  
  \noindent
  \begin{multline}
  \fr{\partial^2 S}{\partial b_p^2} =2\lambda_p n\left(  1+\sum\limits_{j=1}^{m-1} 
\lambda_j b_j^2\right)^{-2} \left[
\vphantom{\left( 1+\sum\limits_{j=1}^{m-1} \lambda_j b_j^2\right)}
 2H_p b_p +B_p-{}\right.\\
  \left.{}- 2\fr{b_p}{n}\left( 1+\sum\limits_{j=1}^{m-1} \lambda_j b_j^2\right) 
\fr{\partial S}{\partial b_p}\right]\,,\enskip p=\overline{1,m-1}\,.
  \label{e11-baz}
  \end{multline}
  
  Для того чтобы функция~(\ref{e3-baz}) имела минимум в~некоторой точке, 
матрица Гессе, составленная из частных производных второго порядка, должна 
быть положительно определенной. По критерию Сильвестра для положительно 
определенной мат\-ри\-цы Гессе все ее элементы на главной  
диагонали~(\ref{e11-baz}) должны быть положительными. А~из~$2^{m-1}$ 
систем~(\ref{e9-baz}) это условие выполняется только для случая~(\ref{e10-baz}). 
Поэтому для нахождения оценок полносвязной регрессии вместо 
системы~(\ref{e4-baz}) достаточно решить систему~(\ref{e10-baz}). Если 
$m\hm\geq 3$, то для этого можно воспользоваться методом простых итераций. 
При $m\hm=3$ можно также применить метод подстановки.
  
  Как уже отмечалось, до решения системы~(\ref{e4-baz}) необходимо задать 
значения лямб\-да-па\-ра\-мет\-ров. По мнению автора, рациональным будет 
выбор таких значений этих параметров, при которых суммарное 
аппроксимационное качество полносвязной регрессии~(\ref{e1-baz}),  
(\ref{e2-baz}) будет наилучшим. Для этого введем аддитивный коэффициент 
детерминации
  \begin{equation*}
  R^2_{\mathrm{add}}  =\sum\limits^m_{j=1} R^2_{x_j}\,,
  %\label{e12-baz}
  \end{equation*}
где $R^2_{x_j}$~--- коэффициент детерминации для переменной~$x_j$ 
полносвязной регрессии~(\ref{e1-baz}),~(\ref{e2-baz}).

  Сформулируем следующую оптимизационную задачу:
  \begin{equation*}
  \sum\limits^m_{j=1} R^2_{x_j}\to \max\,,
 % \label{e13-baz}
  \end{equation*}
которая, по определению $R^2_{x_j}$, равносильна задаче
\begin{multline}
\fr{\sum\nolimits^n_{i=1}\left(\varepsilon_i^{(x_1)}\right)^2}{D_{x_1}} +
\fr{\sum\nolimits^n_{i=1}\left(\varepsilon_i^{(x_2)}\right)^2}{D_{x_2}}+\cdots{} \\{}\cdots +
\fr{\sum\nolimits^n_{i=1}\left(\varepsilon_i^{(x_m)}\right)^2}{D_{x_m}}\to  \min\,.
\label{e14-baz}
\end{multline}
    А~задача~(\ref{e14-baz}) эквивалентна задаче~(\ref{e3-baz}) при 
$\lambda_1\hm= D_{x_m}/D_{x_1}, \lambda_2= D_{x_m}/D_{x_2}, \ldots, 
\lambda_m\hm=D_{x_m}/D_{x_{m-1}}$. Таким образом, для полученных 
значений ламб\-да-па\-ра\-мет\-ров аппроксимационное качество многофакторной 
полносвязной регрессии~(\ref{e1-baz}), (\ref{e2-baz}) будет наилучшим.

\section{Многофакторная модель полносвязной линейной 
регрессии с~выходной переменной и~алгоритм <<Straight~B>>}

  Дополним набор входных переменных $x_1, x_2, \ldots$\linebreak $\ldots , x_m$ выходной 
переменной~$y$, которая сильно коррелирует с~ними. Свяжем оцененные 
<<истинные>> значения, например, переменной~$\tilde{x}^*_m$ со значениями 
переменной~$y$ моделью парной линейной регрессии:
  \begin{equation}
  y_i= c_0+c_1\tilde{x}_{im}^* +\varepsilon_i\,,\enskip i=\overline{1,n}\,,
  \label{e15-baz}
  \end{equation}
где $c_0$ и~$c_1$~--- неизвестные параметры, которые находятся с~помощью 
обычного МНК.

  Совокупность уравнений (\ref{e1-baz}), (\ref{e2-baz}), (\ref{e15-baz}) называется 
многофакторной моделью полносвязной линейной регрессии с~выходной 
переменной~$y$~\cite{10-baz}. Если параметры $\lambda_1, \lambda_2,\ldots , 
\lambda_{m-1}$ известны, то ее оценки находятся в~два этапа.
  \begin{enumerate}[1.]
\item С помощью МНПК оценивается полносвязная регрессия без выходной 
переменной~(\ref{e1-baz}), (\ref{e2-baz}).
\item С~по\-мощью МНК оценивается модель парной линейной  
регрессии~(\ref{e15-baz}).
\end{enumerate}

  Пусть оцененная модель (\ref{e1-baz}), (\ref{e2-baz}) , (\ref{e15-baz}) имеет вид:
  \begin{align}
  \tilde{y} &= \tilde{c}_0 +\tilde{c}_1 \tilde{x}_m^*\,;\label{e16-baz}\\
  \tilde{x}_j^* &= \tilde{a}_j +\tilde{b}_j \tilde{x}_m^*\,,\enskip j=\overline{1,m-
1}\,; \notag%\label{e17-baz}
\\
  \tilde{x}_m^*&=A_0 +\sum\limits^m_{j=1} A_j x_j\,,
  \label{e18-baz}
  \end{align}
где
\begin{align*}
A_0&=-\left( 1+\sum\limits^{m-1}_{j=1} \lambda_j \tilde{b}^2_j \right)^{-1} 
\sum\limits_{j=1}^{m-1} \lambda_j \tilde{a}_j \tilde{b}_j\,;\\
A_j&=\lambda_j \tilde{b}_j \left( 1+ \sum\limits^{m-1}_{j=1} \lambda_j 
\tilde{b}^2_j\right)^{-1}\,,\enskip j=\overline{1,m-1}\,;\\
A_m&=\left( 1+\sum\limits^{m-1}_{j=1} \lambda_j \tilde{b}_j^2\right)^{-1}\,.
\end{align*}
  Используя~(\ref{e18-baz}), перепишем уравнение~(\ref{e16-baz}) в~виде:
  \begin{equation}
  \tilde{y}=\theta_0 +\sum\limits^m_{j=1} \theta_j x_{ij}\,,
  \label{e19-baz}
  \end{equation}
где $\theta_0 =\tilde{c}_0 \hm+ \tilde{c}_1A_0$; $\theta_j\hm= \tilde{c}_1A_j$, 
$j\hm=\overline{1,m}$.
  
  Уравнение~(\ref{e19-baz}) называется вторичным уравнением многофакторной 
модели полносвязной линейной регрессии~\cite{10-baz}.
  
  Как уже отмечалось, при сильной корреляции входных переменных $x_1, x_2, 
\ldots , x_m$ коэффициенты уравнения~(\ref{e18-baz}) удовлетворяют условиям
  \begin{equation}
  A_j r_{x_j x_m}>0\,,\enskip j=\overline{1,m-1}\,;\enskip A_m>0\,,
  \label{e20-baz}
  \end{equation}
а при сильной корреляции~$y$ с~этими переменными угловой коэффициент 
уравнения~(\ref{e16-baz})~--- условию
\begin{equation}
\tilde{c}_1 r_{yx_m}>0\,.
\label{e21-baz}
\end{equation}
  
  Перемножив неравенства~(\ref{e20-baz}) на~(\ref{e21-baz}), получим 
$\tilde{c}_1 A_j r_{x_jx_m} r_{yx_m}\hm>0$, $ j\hm=\overline{1,m-1}$, 
и~$\tilde{c}_1 A_m r_{yx_m}\hm>0$.\linebreak Отсюда, учитывая, что знаки 
произведений~$r_{x_jx_m}r_{yx_m}$ совпадают со знаками~$r_{yx_j}$,  
$j\hm=\overline{1,m-1}$, следует, что
  \begin{equation*}
  \theta_j r_{yx_j}>0\,,\enskip    j=\overline{1,m}\,,
  %\label{e22-baz}
  \end{equation*}
т.\,е.\ знаки коэффициентов при объясняющих переменных во вторичном 
уравнении~(\ref{e19-baz}) совпадают с~соответствующими знаками 
коэффициентов корреляции~$r_{yx_j}$, $j\hm=\overline{1,m}$.
  
  На основе проведенных автором исследований разработан следующий 
алгоритм <<Straight~B>>, реализующий метод выпрямления искаженных 
коэффициентов (МВИК) для многофакторных регрессионных моделей.
  \begin{enumerate}[1.]
\item При $\lambda_1=D_{x_m}/D_{x_1}, \lambda_2\hm= D_{x_m}/D_{x_2}, 
\ldots, \lambda_{m-1}\hm= D_{x_m}/D_{x_{m-1}}$ из системы~(\ref{e10-baz}) 
численно находятся оценки $\tilde{b}_1, \tilde{b}_2, \ldots , \tilde{b}_{m-1}$, 
затем по формулам~(\ref{e5-baz})~--- коэффициенты $\tilde{a}_1, \tilde{a}_2, 
\ldots , \tilde{a}_{m-1}$ и, наконец, по формулам~(\ref{e6-baz})~--- оцененные 
<<истинные>> значения переменной~$\tilde{x}_m^*$.
\item С помощью МНК оценивается модель~(\ref{e15-baz}).
\item Путем подстановки~(\ref{e18-baz}) в~равенство~(\ref{e16-baz}) определяется 
искомое уравнение регрессии.
\end{enumerate}

\section{Моделирование валового внутреннего продукта России}

  Для демонстрации МВИК решалась задача моделирования ВВП России. Для 
этого были использованы статистические данные с~сайта Федеральной службы 
государственной статистики ({\sf https://rosstat.gov.ru}) за период с~2000 
по~2018~гг.\ по следующим семи переменным: $y$~--- ВВП России (млрд руб.); $x_1$~--- среднемесячная заработная плата 
в~России (руб.); $x_2$~--- численность безработных в~России (тыс.\ чел.); 
$x_3$~--- потребление электроэнергии в~России (млн кВт$\cdot$ч); $x_4$~--- 
продукция сельского хозяйства России (млрд руб.); $x_5$~--- грузооборот 
железнодорожного транспорта России (млрд т$\cdot$км); $x_6$~--- оборот 
розничной торговли (млн руб.).
  
  Матрица парных коэффициентов корреляции для этих переменных 
представлена в~таблице.
  
  \begin{table*}\small
  \begin{center}
  \begin{tabular}{|c|c|c|c|c|c|c|c|}
  \multicolumn{8}{c}{Матрица парных коэффициентов корреляции}\\
  \multicolumn{8}{c}{\ }\\[-6pt]
  \hline
&$y$&$x_1$&$x_2$&$x_3$&$x_4$&$x_5$&$x_6$\\
\hline
$y$&1&0,997&$-$0,843&0,953&0,985&0,942&0,997\\
$x_1$&0,997&1&$-$0,827&0,95\hphantom{9}&0,988&0,94\hphantom{9}&0,998\\
$x_2$&$-$0,843\hphantom{$-$}&$-$0,827\hphantom{$-$}&1&$-$0,888\hphantom{$-$}&$-$0,805\hphantom{$-$}
&$-$0,922\hphantom{$-$}&$-$0,831\hphantom{$-$}\\
$x_3$&0,953&0,95\hphantom{9}&$-$0,888&1&0,92\hphantom{9}&0,981&0,954\\
$x_4$&0,985&0,988&$-$0,805&0,92\hphantom{9}&1&0,917&0,987\\
$x_5$&0,942&0,94\hphantom{9}&$-$0,922&0,981&0,917&1&0,937\\
$x_6$&0,997&0,998&$-$0,831&0,954&0,987&0,937&1\\
\hline
\end{tabular}
\end{center}
\end{table*}
       
  
  Как видно из таблицы, все объясняющие переменные тесным образом 
коррелируют между собой. Следовательно, в~оцененной по исходным данным 
модели множественной линейной регрессии будет присутствовать эффект 
муль\-ти\-кол\-ли\-не\-ар\-ности. С~помощью МНК была построена следующая 
регрессионная модель:
  \begin{multline}
  \tilde{y}= 24236{,}7+1{,}399x_1-2{,}136x_2-0{,}0067x_3-{}\\
  {}- 0{,}225x_4-3{,}073x_5 +0{,}0013x_6\,.
  \label{e23-baz}
  \end{multline}
  
  Коэффициент детерминации модели~(\ref{e23-baz}) $R^2\hm= 0{,}996$. Как 
и~оказалось, из-за мультиколлинеарности знаки коэффициентов при 
переменных~$x_3$, $x_4$ и~$x_5$ не согласуются с~экономическим смыслом 
задачи. Так, по регрессии~(\ref{e23-baz}) можно сделать абсурдный вывод о~том, 
что для повышения ВВП России требуется снижать объемы производства 
продукции сельского хозяйства.
  
  Для МВИК на основе алгоритма <<Selection~B>> на языке программирования 
hansl эконометрического пакета Gretl была написана соответствующая программа. 
С~помощью этой программы было получено вторичное уравнение полносвязной 
регрессии:
  \begin{multline}
  \tilde{y}=-54754{,}02+0{,}409x_1-4{,}765x_2+0{,}0693x_3+{}\\
  {}+ 3{,}475x_4 +15{,}73x_5+0{,}00054x_6\,.
  \label{e24-baz}
  \end{multline}
  
  Коэффициент детерминации модели~(\ref{e24-baz}) $R^2\hm=0{,}971$. Как 
видно, теперь знаки абсолютно всех коэффициентов согласуются 
с~экономическим смыслом задачи. При этом по отношению  
к~модели~(\ref{e23-baz}) качество регрессии~(\ref{e24-baz}) снизилось 
незначительно, поэтому ее можно использовать не только для интерпретации, но и~для прогнозирования.
  
{\small\frenchspacing
{%\baselineskip=10.8pt
%\addcontentsline{toc}{section}{References}
\begin{thebibliography}{99}
\bibitem{1-baz}
\Au{Gunst R.\,F., Webster~J.\,T.} Regression analysis and problems of multicollinearity~// 
Commun. Stat. Theory, 1975. Vol.~4. P.~277--292.
\bibitem{2-baz}
\Au{Tamura R., Kobayashi~K., Takano~Y., Miyashiro~R., Nakata~K., Matsui~T.} Best subset 
selection for eliminating multicollinearity~// J.~Oper. Res. Soc.  Jpn., 2017. Vol.~60. 
No.\,3. P.~321--336.

\bibitem{4-baz} %3
\Au{Ферстер Э., Ренц Б.} Методы корреляционного и~регрессионного анализа~/
Пер. с~нем.~--- М.: Финансы 
и~статистика, 1983. 304~с.
(\Au{F$\ddot{\mbox{o}}$rster E., R$\ddot{\mbox{o}}$nz B.} Methoden der korrelation und regressionsanalyse.~--- 
Berlin: Verlag Die Wirtschaft, 1979. 369~p.)

\bibitem{3-baz} %4
\Au{Chatterjee S., Hadi A.\,S.} Regression analysis by example.~--- 5th ed.~--- Hoboken, NJ, USA: 
Wiley, 2012. 424~p. 

\bibitem{5-baz}
\Au{Jolliffe I.\,T.} Principal component analysis.~--- New York, NY, USA: Springer-Verlag, 2002. 
488~p.
\bibitem{6-baz}
\Au{Hoerl A.\,E., Kennard~R.\,W.} Ridge regression: Biased estimation for nonorthogonal 
problems~// Technometrics, 1970. Vol.~12. P.~55--67.
\bibitem{7-baz}
\Au{Мокшина С.\,И., Шуршикова~Г.\,В., Щекунских~С.\,С.} Метод построения содержательно 
интерпретируемых регрессионных моделей в~условиях мультиколлинеарности~// Современная 
экономика: проблемы и~решения, 2017. №\,5(89). С.~81--94.
\bibitem{8-baz}
\Au{Базилевский М.\,П.} Синтез модели парной линейной регрессии и~простейшей  
EIV-мо\-де\-ли~// Моделирование, оптимизация и~информационные технологии, 2019. Т.~7. 
№\,1(24). С.~170--182.
\bibitem{9-baz}
\Au{Базилевский М.\,П.} Исследование двухфакторной модели полносвязной линейной 
регрессии~// Моделирование, оптимизация и~информационные технологии, 2019. Т.~7. 
№\,2(25). С.~80--96.
\bibitem{10-baz}
\Au{Базилевский М.\,П.} Многофакторные модели полносвязной линейной регрессии без 
ограничений на соотношения дисперсий ошибок переменных~// Информатика и~её применения, 
2020. Т.~14. Вып.~2. С.~92--97.
 \end{thebibliography}

}
}

\end{multicols}

\vspace*{-3pt}

\hfill{\small\textit{Поступила в~редакцию 21.09.2020}}

%\vspace*{8pt}

%\pagebreak

\newpage

\vspace*{-28pt}

%\hrule

%\vspace*{2pt}

%\hrule

%\vspace*{-2pt}

\def\tit{METHOD OF STRAIGHTENING DISTORTED DUE~TO~MULTICOLLINEARITY 
COEFFICIENTS\\ IN~REGRESSION MODELS}


\def\titkol{Method of straightening distorted due~to~multicollinearity 
coefficients in~regression models}

\def\aut{M.\,P.~Bazilevskiy}

\def\autkol{M.\,P.~Bazilevskiy}

\titel{\tit}{\aut}{\autkol}{\titkol}

\vspace*{-11pt}


\noindent
Department of Mathematics, Irkutsk State Transport University, 15~Chernyshevskogo 
Str., Irkutsk 664074, Russian Federation
 
\def\leftfootline{\small{\textbf{\thepage}
\hfill INFORMATIKA I EE PRIMENENIYA~--- INFORMATICS AND
APPLICATIONS\ \ \ 2021\ \ \ volume~15\ \ \ issue\ 2}
}%
\def\rightfootline{\small{INFORMATIKA I EE PRIMENENIYA~---
INFORMATICS AND APPLICATIONS\ \ \ 2021\ \ \ volume~15\ \ \ issue\ 2
\hfill \textbf{\thepage}}}

\vspace*{3pt}



\Abste{When constructing regression models, due to the strong multicollinearity of the 
explanatory variables, its coefficients are distorted, in particular, their signs, which 
negatively affects the interpretational qualities of such regression. This article is devoted 
to the development of a~method of straightening coefficients distorted due to 
multicollinearity. This method is based on the property of the fully connected linear 
regression models proposed by the author. A~nonlinear system, which is used to 
estimate fully connected regressions, is investigated. It is shown that the solution of this 
system can be obtained numerically using the method of simple iterations. A method for 
choosing unknown lambda-parameters in fully connected regression is proposed. It was 
found that in multivariate fully connected models with a~strong correlation of all 
factors, the signs of the coefficients for the variables in the secondary equation coincide 
with the corresponding signs of the correlation coefficients. To straighten the distorted 
coefficients on the basis of this research, the ``Selection~B'' algorithm was developed. 
The developed method of straightening has been successfully demonstrated by the 
example of modeling Russia's gross domestic product (GDP).}

\KWE{regression analysis; fully connected linear regression model; multicollinearity; 
interpretation; numerical method; GDP of Russia}


\DOI{10.14357/19922264210209}

%\vspace*{-15pt}

% \Ack
%\noindent


%\vspace*{12pt}

  \begin{multicols}{2}

\renewcommand{\bibname}{\protect\rmfamily References}
%\renewcommand{\bibname}{\large\protect\rm References}

{\small\frenchspacing
 {%\baselineskip=10.8pt
 \addcontentsline{toc}{section}{References}
 \begin{thebibliography}{99}
\bibitem{1-baz-1}
\Aue{Gunst, R.\,F., and J.\,T.~Webster.} 1975. Regression analysis and problems of 
multicollinearity. \textit{Commun. Stat. Theory}  
4:277--292.
\bibitem{2-baz-1}
\Aue{Tamura, R., K.~Kobayashi, Y.~Takano, R.~Miyashiro, K.~Nakata, and 
T.~Matsui.} 2017. Best subset selection for eliminating multicollinearity. 
\textit{J.~Oper. Res. Soc. Jpn.} 60(3):321--336.

\bibitem{4-baz-1} %3
\Aue{F$\ddot{\mbox{o}}$rster, E., and B.~R$\ddot{\mbox{o}}$nz.} 1983. 
\textit{Methoden der korrelation und regressionsanalyse}. Berlin: Verlag Die Wirtschaft, 1979. 369~p.
\bibitem{3-baz-1} %4
\Aue{Chatterjee, S., and A.\,S.~Hadi.} 2012. \textit{Regression analysis by example}. 
5th  ed. Hoboken, NJ: Wiley. 424~p.
\bibitem{5-baz-1}
\Aue{Jolliffe, I.\,T.} 2002. \textit{Principal component analysis}. New York, NY: 
Springer-Verlag. 488~p.
\bibitem{6-baz-1}
\Aue{Hoerl, A.\,E., and R.\,W.~Kennard.} 1970. Ridge regression: Biased estimation for 
nonorthogonal problems. \textit{Technometrics} 12:55--67.
\bibitem{7-baz-1}
\Aue{Mokshina, S.\,I., G.\,V.~Shurshikova, and S.\,S.~Shchekunskikh.} 2017. Metod 
postroeniya soderzhatel'no interpretiruemykh regressionnykh modeley v~usloviyakh 
mul'tikollinearnosti [The construction method of meaningful interpreted regression 
models in conditions of multicollinearity]. \textit{Sovremennaya ekonomika: problemy 
i~resheniya} [Modern Economics: Problems and Solutions] 89(5):81--94.
\bibitem{8-baz-1}
\Aue{Bazilevskiy, M.\,P.} 2019. Sintez modeli parnoy lineynoy regressii i~prosteyshey 
EIV-modeli [Synthesis of linear regression model and EIV-model]. 
\textit{Modelirovanie, optimizatsiya i~informatsionnye tekhnologii} [Modeling, 
Optimization and Information Technology] 7(1):170--182.
\bibitem{9-baz-1}
\Aue{Bazilevskiy, M.\,P.} 2019. Issledovanie dvukhfaktornoy modeli polnosvyaznoy 
lineynoy regressii [Investigation of a two-factor fully connected linear regression 
model]. \textit{Modelirovanie, optimizatsiya i~informatsionnye tekhnologii} [Modeling, 
Optimization and Information Technology] 7(2):80--96.
\bibitem{10-baz-1}
\Aue{Bazilevskiy, M.\,P.} 2020. Mnogofaktornye modeli polnosvyaznoy lineynoy 
regressii bez ogranicheniy na sootnosheniya dispersiy oshibok peremennykh 
[Multifactor fully connected linear regression models without constraints to the ratios of 
variables errors variances]. \textit{Informatika i~ee Primeneniya~--- Inform. Appl.} 
14(2):92--97.
\end{thebibliography}

 }
 }

\end{multicols}

\vspace*{-3pt}

  \hfill{\small\textit{Received September~21, 2020}}


%\pagebreak

%\vspace*{-8pt}  

\Contrl

\noindent
\textbf{Bazilevskiy Mikhail P.} (b.\ 1987)~--- Candidate of Science (PhD) in 
technology, associate professor, Irkutsk State Transport University, 
15~Chernyshevkogo Str., Irkutsk 664074, Russian Federation; 
\mbox{mik2178@yandex.ru}

\label{end\stat}

\renewcommand{\bibname}{\protect\rm Литература} %4

%\def\F{{\rm I\!F}}
\def\P{{\rm I\!P}}

\def\stat{peshkova}

\def\tit{ГРАНИЦЫ НЕЗАВЕРШЕННОЙ РАБОТЫ В~СИСТЕМЕ С~ПОВТОРНЫМИ ВЫЗОВАМИ РАЗНЫХ КЛАССОВ 
И~ПОКАЗАТЕЛЬНЫМ ВРЕМЕНЕМ ОБСЛУЖИВАНИЯ$^*$}

\def\titkol{Границы незавершенной работы в~системе с~повторными вызовами разных классов 
и~показательным временем} % обслуживания}

\def\aut{И.\,В.~Пешкова$^1$}

\def\autkol{И.\,В.~Пешкова}

\titel{\tit}{\aut}{\autkol}{\titkol}

\index{Пешкова И.\,В.}
\index{Peshkova I.\,V.}


{\renewcommand{\thefootnote}{\fnsymbol{footnote}} \footnotetext[1]
{Работа выполнена при финансовой поддержке РНФ (проект 21-71-10135).}}


\renewcommand{\thefootnote}{\arabic{footnote}}
\footnotetext[1]{Петрозаводский государственный университет; 
Институт прикладных математических исследований Карельского 
научного центра Российской академии наук, \mbox{iaminova@petrsu.ru}}


%\vspace*{-12pt}



\Abst{Исследуется односерверная система 
с~повторными вызовами и~пуассоновским входным потоком, в~которую поступает~$M$ 
классов заявок.
Для системы с~временами обслуживания, имеющими показательное распределение, 
получены верхняя и~нижняя границы для стационарной незавершенной работы. 
В~качестве нижней границы выступает   стационарная незавершенная работа 
в~классической сис\-те\-ме  $M/H_M/1$ с~гиперэкспоненциальным временем обслуживания. 
Верхней границей служит незавершенная работа в~сис\-те\-ме, в~которой к~времени 
обслуживания добавляется дополнительное время, равное интервалу между попытками 
попасть на сервер с~самой <<медленной орбиты>>. Полученные результаты численного 
моделирования подтверждают теоретические выводы.}


\KW{система с~повторными вызовами; незавершенная 
работа; стохастическая упо\-ря\-до\-чен\-ность} 

\DOI{10.14357/19922264230408}{UOKQRD}
  
%\vspace*{-6pt}


\vskip 10pt plus 9pt minus 6pt

\thispagestyle{headings}

\begin{multicols}{2}

\label{st\stat}

\section{Введение}

Модели систем с~повторными вызовами широко используются для моделирования  
телефонных станций, кол-цент\-ров, сис\-тем связи, телекоммуникационных сетей. 
В~работах~\cite{Ar1, Ar3} \mbox{изложены}  приложения и~математические методы анализа 
сис\-тем c~повторными вызовами, включая сис\-те\-мы с~постоянной интенсивностью 
повторов. В~работе~\cite{F86}   телефонная станция была смоделирована  с~по\-мощью 
такой сис\-те\-мы. Аналогичная модель используется для описания широкого класса 
протоколов множественного доступа~\cite {CSA92, CRP93}. В~част\-ности, в~работе~\cite{BG92} 
показано, что постоянная интенсивность повторных вызовов  снижает 
интенсивность попыток (при незапланированном множественном доступе) обратно 
пропорционально числу задержанных пакетов. В~результате общая ско\-рость повторной 
обработки становится нечувствительной к~виртуальному <<размеру орбиты>> (числу 
отложенных пакетов). Более того, сис\-те\-мы с~повторными вызовами с~постоянной 
интенсивностью вызовов использовались для описания TCP-тра\-фи\-ка с~короткими HTTP-со\-еди\-не\-ни\-ями~\cite{AY08,AY10} 
и~оп\-ти\-ко-элект\-ри\-че\-ской гиб\-рид\-ной схемой разрешения 
конфликтов~\cite{Wongetal09,Yaoetal02}.

 Большинство же современных моделей повторных вызовов имеют сложную природу или 
конфигурацию, и~поэтому для их исследования применяются численные методы или 
имитационное моделирование.

Ранее в~работе~\cite{mathematics2022} была доказана тео\-ре\-ма о~верх\-ней и~ниж\-ней 
границах стационарной незавершенной работы  для исходной сис\-те\-мы с~повторными 
вызовами с~постоянной интенсивностью вызовов  (см.\ тео\-ре\-му~1 ниже). Эта тео\-ре\-ма 
стала основой анализа, развитого в~данной \mbox{статье}.

В данном исследовании рассматривается частный случай  односерверной сис\-те\-мы  
с~повторными вызовами с~пуассоновским входным потоком и~показательным 
распределением времени обслуживания, при этом  время обслуживания  и~время 
нахождения на орбите зависят от класса заявки~$k$.
%
Для такой системы предлагается строить две классические сис\-те\-мы с~неограниченной 
очередью (с~ожиданием) типа $M/G/1$: в~первой (минорантной) сис\-те\-ме время 
обслуживания пред\-став\-ля\-ет собой конечную смесь времен обслуживания заявок всех 
классов (т.\,е.\ имеет гиперэкспоненциальное распределение), во второй 
(мажорантной) сис\-те\-ме ко времени обслуживания первой сис\-те\-мы добавляется 
дополнительное время, равное  интервалу между вызовами  с~самой <<медленной 
орбиты>>.   Более того, для минорантной сис\-те\-мы получено распределение 
стационарного времени ожидания в~явном виде для трех классов ($M\hm=3$).
Сис\-те\-мы, в~которых  время обслуживания задано в~виде конечной смеси 
распределений, обсуждались ранее  в~работах~\cite{pesh-mor2022, pesh2022}.
 
Структура статьи следующая. В~разд.~2 приведено описание модели сис\-те\-мы 
с~повторными вызовами, минорантной и~мажорантной сис\-тем, а~также основная тео\-ре\-ма, 
полученная авторами в~работе~\cite{mathematics2022}.
В~разд.~3 получены коэффициенты загрузки, математические ожидания 
незавершенной нагрузки, а~также преобразования Лап\-ла\-са--Стилть\-еса для 
незавершенной нагрузки в~минорантной и~мажорантной сис\-те\-мах с~показательным 
распределением времени обслуживания. В~разд.~4 приведены результаты численного 
эксперимента для случая трех классов. При этом параметры для минорантной системы 
были использованы такие же, как в~работе~\cite{rego}, в~которой получена  
функция распределения  стационарного времени пребывания. Отметим, что  в~работе~\cite{rego} 
неверно указано, что полученное распределение~--- это распределение 
стационарного времени ожидания. В~статье получена функция распределения  
стационарного времени ожидания для минорантной сис\-те\-мы  в~явном  виде. Для 
исходной сис\-те\-мы с~повторными вызовами и~мажорантной  сис\-те\-мы проведены 
численные эксперименты и~построены эмпирические функции распределения. 
Полученные результаты численного эксперимента иллюстрируют доказанную 
стохастическую упо\-ря\-до\-чен\-ность стационарной незавершенной работы в~рассмотренных 
сис\-те\-мах.

\section{Описание модели}

Рассмотрим односерверную систему с~повторными вызовами~$\Sigma$, в~которой 
обслуживаются~$M$~классов заявок. Заявки $k$-го класса поступают в~сис\-те\-му в~соответствии 
с~пуассоновским потоком с~па\-ра\-мет\-ром~$\lambda_k$, $k\hm=1,\ldots,M$. 
Если заявка застает сервер пустым, то она немедленно обслуживается, в~противном 
случае, если сервер занят,  заявка уходит на $k$-ю орбиту бесконечного объема, 
образуя очередь в~порядке поступления на орбиту, $k\hm=1,\ldots,M$. Первая 
в~очереди на $k$-й орбите заявка производит независимые попытки присоединиться 
к~обслуживанию на сервере через экспоненциальное  время~$\eta_k$.
Интенсивность вызовов с~орбиты не зависит от размера орбиты (т.\,е.\ от числа 
заявок на орбите), но может зависеть от класса орбиты~$k$. Такие сис\-те\-мы 
называют сис\-те\-ма\-ми  с~постоянной ин\-тен\-сив\-ностью вызовов.

Обозначим через $t_n$ момент прихода $n$-й заявки в~общий пуассоновский входной  
поток (образованный суперпозицией~$M$~входных потоков, $t_1\hm=0$),   $S_n{(k)}$~--- 
время обслуживания  $n$-й заявки  $k$-го класса,  $k\hm=1,\ldots,M$, $n\hm\ge1$. Пусть 
последовательность независимых одинаково распределенных (н.\,о.\,р.)\ интервалов 
между приходами заявок  $\{T_n:=t_{n+1}\hm-t_n,\ n\hm\ge 1\}$ и~последовательность 
времен обслуживания  $\{S_n{(k)},\ n\hm\ge1\}$ независимы для каждого  $k$-го 
класса.
Предположим, что интервалы между приходами заявок с~(непустой)  $k$-й орбиты~$\eta_k$ 
распределены показательно и~не зависят от размера орбиты  (числа заявок 
на $k$-й орбите). Время обслуживания заявок $k$-го класса  $S(k)$ имеет 
произвольное распределение  с~функцией распределения (ф.~р.)\ $F_{S(k)}$, 
$k\hm=1,\ldots, M$. (Далее в~обозначениях  опускаем индекс номера заявки.)  Обозначим
\begin{equation*}
%\label{rates}
\lambda=\sum\limits_{k=1}^M\lambda_k ;\ \ \ \rho_k=\lambda_k\mathbb{E} S{(k)}; \enskip 
\rho=\sum\limits_{k=1}^M \rho_k.
\end{equation*}
Пусть $W(t)$ есть незавершенная работа в~момент времени~$t^-$, и~предположим, 
что система в~начальный момент времени пуста: $W(0)\hm=0$. Обозначим 
$W_n=W(t_n)$, $n\hm\ge1$.
Известно~\cite{Morozov2019}, что неравенство
 \begin{equation}
 \label{stability}
 \rho + \max\limits_{k=1,\ldots, M} \fr{\lambda}{\lambda+\eta_k} < 1
 \end{equation}
служит достаточным условием стационарности сис\-те\-мы.  При  таком условии 
существует  предел
$$
W_n \Rightarrow W\,,\quad n\to\infty
$$
(где обозначим $\Rightarrow$ схо\-ди\-мость по распределению). Функция распределения~$F_W$ 
стационарной незавершенной работы~$W$ для исходной сис\-те\-мы~$\Sigma$ неизвестна. 
С~другой стороны, $W$ служит важной характеристикой качества обслуживания 
сис\-те\-мы. Ниже мы построим верхнюю и~нижнюю границы~$F_W$, используя 
классические  $M/G/1$ сис\-те\-мы с~неограниченной оче\-редью, в~которых время 
обслуживания пред\-став\-ля\-ет\-ся конечной смесью распределений.


Рассмотрим две новые системы: \textit{минорантную сис\-те\-му}~$\Sigma^{(1)}$ 
и~\textit{мажорантную сис\-те\-му}~$\Sigma^{(2)}$. (Далее индекс~$(i)$ означает номер 
сис\-те\-мы.)
Входной поток во все три сис\-те\-мы~--- пуассоновский  с~параметром~$\lambda$ (это 
суперпозиция входных потоков, образованных заявками разных классов).

Пусть в~минорантной сис\-те\-ме~$\Sigma^{(1)}$ время обслуживания $S^{(1)}\hm=S$ задано 
конечной смесью распределений вида
\begin{equation}
\label{mixture}
S=\sum\limits_{k=1}^M I(k) S(k), \enskip n\ge1\,,
\end{equation}
где  $I(k)$~--- индикатор, такой что  $\mathbb{E} I(k)\hm=p_k=\lambda_k/\lambda$; $S(k)$~--- время  обслуживания заявки $k$-го класса.
Заметим, что сис\-те\-ма~$\Sigma^{(1)}$ пред\-став\-ля\-ет собой классическую 
(с~дисциплиной обслуживания первым при\-шел\,--\,пер\-вым обслужен) односерверную сис\-те\-му 
типа $M/G/1$ с~неограниченной очередью, в~которой время обслуживания~(\ref{mixture})
является конечной смесью времен обслуживания заявок всех классов исходной 
сис\-те\-мы.

В мажорантной системе~$\Sigma^{(2)}$~--- классической односерверной сис\-те\-ме 
типа  $M/G/1$ с~неограниченной оче\-редью~--- каждая заявка обслуживается на 
сервере в~течение времени~$S$, заданного соотношением~\eqref{mixture},  плюс 
время~$\xi_0$, имеющее показательное распределение с~па\-ра\-мет\-ром  
$$
\mu_0=\min\limits_{1\le k\le M} (\lambda\hm+\eta_k),
$$
 т.\,е.
%\begin{equation*}
%\label{sums2}
$S^{(2)} \hm= S\hm +\xi_0$.
%\end{equation*}
Таким образом, случайная величина (с.\,в.)~$\xi_0$ соответствует самой 
<<медленной>> орбите (с наибольшими интервалами между попытками). Заметим, что 
мажорантная сис\-те\-ма~$\Sigma^{(2)}$ имеет другой коэффициент загрузки,
\begin{equation*}
%\label{rho2def}
 \rho^{(2)}=\lambda \mathbb{E} S+\fr{\lambda}{\mu_0}=\rho+\fr{\lambda}{\mu_0}\,,
\end{equation*}
и условие стационарности~\eqref{stability} для нее принимает вид:
$$
\rho^{(2)}<1.
$$

В работе~\cite{mathematics2022}  доказана сле\-ду\-ющая тео\-ре\-ма, в~которой даны 
верхняя и~нижняя границы незавершенной работы~$W$ в~исходной сис\-те\-ме 
с~повторными вызовами~$\Sigma$.

\smallskip

\noindent
\textbf{Теорема~1.}
\textit{Пусть сис\-те\-мы  $\Sigma^{(1)}$, $\Sigma^{(2)}$ и~$\Sigma$ в~начальный момент времени 
пустые, т.\,е.}
 \begin{equation*}
W_1^{(1)}=W_1=W_1^{(2)}=0\,.
 \end{equation*}
\textit{Тогда при  выполнении условия}~\eqref{stability} \textit{стационарные времена 
незавершенной работы стохастически упорядочены}:
 \begin{equation}
 \label{theor1-1}
 W^{(1)}\underset{\mathrm{st}}\le W \underset{\mathrm{st}}\le W^{(2)},
 \end{equation}
\textit{где $W^{(1)}\le_{\mathrm{st}} W$ означает $\overline F_{W^{(1)}} (x) \hm\le 
\overline F_{W} (x) $ для любого $x\hm\ge 0$, $\overline F_{W^{(1)}}  (x)\hm= 1\hm-  
F_{W^{(1)}} (x)$}.


\smallskip

В следующем разделе применим данный результат для сис\-те\-мы, в~которой~$M$~классов 
заявок, име\-ющих  показательное распределение времени обслуживания.

\section{Границы незавершенной работы~$W$ в~системе с~показательным 
обслуживанием разных классов }

Пусть в~исходной сис\-те\-ме с~повторными вызовами~$\Sigma$ времена обслуживания 
$k$-го класса~$S(k)$ имеют показательное распределение с~ф.\,р.
\begin{equation}
\label{hyperexp}
F_{S(k)}(x)= 1- e^{-\mu_k x}, \enskip x\ge 0\,, \ \mu_k >0\,.
\end{equation}

В качестве минорантной сис\-те\-мы~$\Sigma^{(1)}$ рассмотрим сис\-те\-му 
с~неограниченной  очередью  $M/H_M/1$, в~которой времена обслуживания $S^{(1)}\hm=S$ 
имеют гиперэкспоненциальное распределение (пред\-став\-ля\-ют\-ся $M$-ком\-по\-нент\-ной 
смесью показательно распределенных с.\,в.~$S(k)$) с~ф.\,р.
\begin{multline*}
F_{S^{(1)}}(x) = 1 -  \sum\limits_{k=1}^M p_k e^{-\mu_k x}, \enskip \mu_k > 0\,, \\ 
p_k\ge 0\,,\enskip k=1,\ldots,M, \enskip \sum\limits_{i=k}^M p_k=1\,.
\end{multline*}

Обозначим коэффициент загрузки  $\rho^{(1)} \hm=\sum\nolimits_{k=1}^M \lambda 
p_k/\mu_k$ в~сис\-те\-ме~$\Sigma^{(1)}$. Поскольку
$$
\rho^{(1)} \le \rho + \fr{\lambda}{\mu_0 }\,,
$$
то, если условие стационарности~\eqref{stability} выполнено,    сис\-те\-ма~$\Sigma^{(1)}$ также стационарна.

Рассмотрим преобразование Лап\-ла\-са--Стилть\-еса:
\begin{equation*}
%\label{lstdef}
 \psi_{S_e}(z)=\int\limits_0^\infty e^{-zt} \,dF_{S_e}(t),
\end{equation*}
где $F_{S_e}$ -- так называемый \textit{интегрированный хвост
распределения} с~плот\-ностью
\begin{equation*}
%\label{fequilibr}
f_{S_e}(x)=\fr{1}{\mathbb{E} S}\, \overline F_S(x),\enskip x\ge0\,.
% f_{S_e}(x)=\mu \overline{F}_S(x),\enskip x\ge0\,.
\end{equation*}

Распределение $F_{S_e}$ соответствует распределению стационарного перескока 
процесса вос\-ста\-нов\-ле\-ния, фор\-ми\-ру\-емо\-го по\-сле\-до\-ва\-тель\-ностью н.\,о.\,р.\ времен 
обслуживания~$\{S_n\}$~\cite{Asmus}.

В работе~\cite{mathematics2022} доказано, что преобразование\linebreak Лап\-ла\-са--Стилть\-еса 
стационарной незавершенной работы~$W^{(1)}$ выражается через преобразования 
Лап\-ла\-са--Стилть\-еса компонент смеси времен обслуживания в~сле\-ду\-ющем виде:
\begin{multline}
\label{lstformixture}
\psi_{W^{(1)}}(z)=\fr{1-\rho}{z\left(1-\rho\sum\nolimits_{k=1}^M 
(\rho_k/{\rho}) \psi_{S_e(k)}(z)\right)}={}\\
{}=\fr{1-\rho}{z\left(1-
\sum\nolimits_{k=1}^M \rho_k \psi_{S_e(k)}(z)\right)}.
\end{multline}

Преобразование Лапласа--Стилть\-еса для показательного распределения хорошо 
известно: 
$$
\psi_{S_e(k)}(z) = \fr{\mu_k}{\mu_k +z}\,.
$$ 
Подставляя его в~соотношение~\eqref{lstformixture},  получим
$$
\psi_{W^{(1)}}(z)=\fr{1-\sum\nolimits_{k=1}^M \lambda_k/\mu_k}{z\left(1-
\sum\nolimits_{k=1}^M  {\lambda_k}/(\mu_k +z)   
\right)}\,.
$$

Применяя формулу По\-ла\-чи\-ка--Хин\-чи\-на, получим среднюю стационарную незавершенную 
работу  в~сис\-те\-ме~$\Sigma^{(1)}$ в~виде:
\begin{equation}
\label{ew5}
\mathbb{E}  W^{(1)} = \fr{\lambda \mathbb{E} ( S^{(1)})^2}{2(1-\rho^{(1)})} = 
\fr{\sum\nolimits_{k=1}^M \rho_k^2 +\rho^2}{2\lambda (1-\rho)}\,.
\end{equation}


Рассмотрим теперь мажорантную сис\-те\-му~$\Sigma^{(2)}$,  время обслуживания 
в~которой равно сумме с.\,в.~$S$ с~гиперэкспоненциальным распределением~\eqref{hyperexp} и~с.~в.~$\xi_0$, т.\,е.\
 $\ S^{(2)}\hm=S \hm+ \xi_0$. Обозначим для простоты 
$\mu_0\hm=\min\nolimits_{1\le k\le M} (\lambda\hm+\eta_k)$, и~пусть с.\,в.~$\xi_0$ имеет 
показательное распределение  с~па\-ра\-мет\-ром~$\mu_0$. Условие стационарности 
в~такой сис\-те\-ме совпадает с~\eqref{stability}.

Известно \cite{mathematics2022}, что в~сис\-те\-ме~$\Sigma^{(2)}$ преобразование 
Лап\-ла\-са--Стилть\-еса для стационарной незавершенной работы~$W^{(2)}$ имеет 
сле\-ду\-ющий вид:
\begin{multline}
\label{reslemma2}
 \psi_{W^{(2)}} (z)=
 \left(1-\rho- \fr{\lambda}{\mu_0}\right) \Bigg/
\left(  z\left(
\vphantom{\left(\sum\limits_{k=1}^M\right)}
1-{}\right.\right.\\
\left.\left.{}-\fr{\mu_0}{\mu_0+z}\left(\sum\limits_{k=1}^M \rho_k \psi_{S_e(k)}(z) +
\fr{\lambda}{\mu_0}\right)\right)\right)\,.
\end{multline}
Подставив $\psi_{S_e(k)}(z) = \mu_k/(\mu_k \hm+z)$ в~\eqref{reslemma2}, получим
\begin{multline*}
\psi_{W^{(2)}} (z)=
\left(1-\sum\limits_{k=1}^M \fr{\lambda_k}{\mu_k}-\fr{\lambda}{\mu_0}\right) \Bigg/
\left(z\left(
\vphantom{\left(\sum\limits_{k=1}^M\right)}
1-{}\right.\right.\\
\left.\left.{}-\fr{\mu_0}{\mu_0+z}\left(\sum\limits_{k=1}^M \fr{\lambda_k}{\mu_k+z}  +\fr{\lambda}{\mu_0}\right)\right)\right)\,.
\end{multline*}
Аналогично формуле~\eqref{ew5} можно получить среднюю стационарную незавершенную 
работу
\begin{multline}
\label{ew6}
\mathbb{E}  W^{(2)} = \fr{\lambda \mathbb{E} \left( S^{(2)}\right)^2}{2\left(1-\rho^{(2)}\right)} = {}\\
{}=
\fr{\mu_0^2\left(\rho^2 + \sum\nolimits_{k=1}^M \rho_k^2 \right) +2 \lambda (\lambda +\rho \mu_0)}{2\lambda \mu_0 (\mu_0 -\rho \mu_0 -\lambda)}.
\end{multline}



Из теоремы~1 следует, что
стационарные времена незавершенной работы стохастически упорядочены:
 $$
 W^{(1)}\underset{\mathrm{st}}\le W \underset{\mathrm{st}}\le W^{(2)},
 $$
а следовательно,  их математические ожидания также упорядочены~\cite{Ross}:
$$
 \mathbb{E} W^{(1)}\le \mathbb{E} W \le \mathbb{E} W^{(2)}.
$$

Действительно, легко проверить, что
\begin{multline*}
\mathbb{E}  W^{(2)} = \fr{\lambda \mathbb{E} \left( S^{(2)}\right)^2}{2(1-\rho^{(2)})} = {}\\
{}=
\fr{\mu_0^2 \lambda \mathbb{E} ( S^{(1)})^2 + 2 \lambda (\lambda +\rho \mu_0)} {\mu_0^2 2(1-\rho^{(1)}) - 2\lambda^2 \mu_0 } 
\ge \mathbb{E}  W^{(1)}.
\end{multline*}


\section{Численный эксперимент}

В качестве примера рассмотрим систему с~повторными вызовами~$\Sigma$ с~тремя 
классами заявок ($M\hm=3$), в~которую поступает пуассоновский поток 
с~ин\-тен\-сив\-ностью $\lambda\hm=10$.
Пусть
$p_1\hm=1/2$, $p_2\hm=1/3$, $p_3\hm=1/6$, $\mu_1\hm=10$, $\mu_2\hm=30$ и~$\mu_3\hm=60$.
Будем полагать, что~$\eta_k$ принимают значения
$\eta_1\hm=50$, $\eta_2\hm=100$ и~$\eta_3\hm=150.$
В~этом случае $\mu_0\hm=\lambda\hm+\eta_1\hm=60$,  коэффициенты загрузки \mbox{равны}
$$
\rho^{(1)}=\rho= \sum\limits_{k=1}^3 \fr{\lambda p_k}{\mu_k} = \fr{23}{36}\,; \enskip  
\rho^{(2)} = \rho+\fr{\lambda}{\mu_0} = \fr{29}{36} < 1
$$
и условие стационарности~\eqref{stability} выполнено. По 
формулам~\eqref{ew5}--\eqref{ew6} находим математические ожидания $\mathbb{E} W^{(1)} \hm\approx 0{,}093$ 
и~$\mathbb{E} W^{(2)} \hm\approx 0{,}106$.

Рассмотрим минорантную сис\-те\-му~$\Sigma^{(1)}$, в~которой время обслуживания 
имеет гиперэкспоненциальное распределение
$$
\overline F_S(x)=\sum\limits_{k=1}^3 p_k e^{-\mu_k x}.
$$
%
Для такой классической системы $M/G/1$ в~работе~\cite{rego} получено 
распределение числа клиентов в~сис\-те\-ме в~стационарном режиме~$N^{(1)}$ в~сле\-ду\-ющем виде:
$$
\pi_n^\ast=\mathbb{P}\left\{N^{(1)}=n\right\}=\sum\limits_{k=1}^3 \beta_k (r_k)^n,
$$
где параметры  $\beta_k$ и~$r_k$ вычислены и~равны
\begin{alignat*}{3}
    \beta_1&=0{,}040;&\enskip \beta_2&=0{,}075;&\enskip \beta_3&=0{,}245; \\
    r_1&=0{,}146;&\enskip r_2&=0{,}268;&\enskip r_3&=0{,}711.
\end{alignat*}
В работе~\cite{rego} с~помощью результата~\cite{haji}  получено стационарное 
распределение  \textit{времени пребывания} клиента в~сис\-те\-ме~$V^{(1)}$ с~хвостом 
ф.\,р.\ вида
$$
\overline F_{V^{(1)}}(x)=\sum\limits_{k=1}^3 \gamma_k e^{-\theta_k x},
$$
где коэффициенты $\gamma_k$  и~па\-ра\-мет\-ры~$\theta_k$  для исходных параметров 
\mbox{равны}
\begin{alignat*}{3}
\gamma_1&=0{,}047;&\quad \gamma_2&=0{,}103; &\quad \gamma_3&=0{,}850; \\
\theta_1&=58{,}633; &\quad \theta_2&=27{,}307;&\quad \theta_3&= 4{,}059.
\end{alignat*}
При этом в~утверждении теоремы~3 из работы ~\cite{rego} было  ошибочно указано, 
что $\sum\nolimits_{k=1}^3 \gamma_k\hm=1\hm-\rho^{(1)}$ (что было бы верно, если бы 
распределение~$F_V^{(1)}$ соответствовало  \textit{времени ожидания}). На самом 
деле легко проверить, что $\sum\nolimits_{k=1}^3 \gamma_k\hm=1$. Для исправления этой 
неточности повторим вывод, получив выражение для стационарного времени 
\textit{ожидания} в~сис\-те\-ме (что  соответствует незавершенной работе в~сис\-те\-ме 
в~момент прихода клиента). Заметим, что~$\pi_{n+1}^\ast$ есть стационарная 
вероятность наблюдать~$n$~клиентов в~очереди, т.\,е.
$$
\mathbb{P}\{Q^{(1)}=n\}=\pi_{n+1}^\ast,
$$
где $Q^{(1)}$ есть число клиентов \textit{в очереди} в~стационарном режиме.

Вычислив производящую функцию вероятностей~$\pi (z)$ для~$Q^{(1)}$, получим
\begin{multline*}
    \pi (z)=\sum\limits_{n=0}^\infty z^n \pi_{n+1}^\ast = \sum\limits_{k=1}^3
    \fr{\beta_k }{z}\sum\limits_{n=1}^\infty (r_k z)^n={}\\
    {}=\sum\limits_{k=1}^3 \fr{\beta_k }{z}\left(\fr{1}{1-r_k z}-1\right)=\sum\limits_{k=1}^3 \fr{\beta_k r_k}{1-r_k z}\,.
\end{multline*}

C другой стороны,  производящая функция стационарной очереди~$\pi(z)$ и~преобразование Лап\-ла\-са--Стилть\-еса для стационарного времени 
ожидания~$\psi_{W^{(1)}}(z)$ связаны формулой:
\begin{multline*}
\pi(z)= \sum\limits_{n=0}^{\infty}  \int\limits_{0^-}^{\infty} z^n e^{-\lambda x} \fr{(\lambda x )^n}{n!}\, dF_{W^{(1)}} (x) ={}\\
{}=
\int\limits_{0^-}^{\infty} e^{-(\lambda-\lambda z) x} \, d F_{W^{(1)}} (x) ={}\\
{}=\psi_{W^{(1)}} (\lambda - \lambda z) + \left(1-\rho^{(1)}\right),
\end{multline*}
 где $F_{W^{(1)}}(0) = (1-\rho^{(1)})$~--- скачок  ф.~р.\ в~нуле. Сделав замену 
переменной $s\hm=\lambda\hm-\lambda z$,  получим
$$
\psi_{W^{(1)}} (s)=\sum_{k=1}^3 \fr{\beta_k r_k}{1-r_k(1-
s/\lambda)}=\sum\limits_{k=1}^3 \fr{\beta_k r_k}{1-r_k}\,\fr{\theta_k}{\theta_k+ s}\,,
$$
где, как и~в~работе~\cite{rego},
$$
\theta_k=\fr{\lambda(1-r_k)}{r_k}\,.
$$

{ \begin{center}  %fig1
 \vspace*{-1pt}
     \mbox{%
\epsfxsize=79mm 
\epsfbox{pes-1.eps}
}

\end{center}



\noindent
{\small{Функции распределения в~нижней~$\Sigma^{(1)}$~(\textit{1}), исходной $\Sigma$~(\textit{2}) 
и~верх\-ней~$\Sigma^{(2)}$~(\textit{3}) сис\-те\-мах при $\lambda \hm= 10$, $p_1\hm= 1/2$, $p_2\hm= 1/3$, $p_3\hm=1/6$, 
$\mu_1\hm=10$, $\mu_2\hm=30$, $\mu_3\hm= 60$, $\eta_1\hm=50$, $\eta_2\hm=100$ и~$\eta_3\hm=150$}}}

\vspace*{12pt}

\noindent
Таким образом,~$\psi_{W^{(1)}}$ соответствует взвешенной  сумме показательных 
распределений. Отметим при этом, что, в~отличие от~\cite{rego}, коэффициенты 
смеси имеют вид:
$$
\hat\gamma_k=\fr{\beta_k r_k}{1-r_k}\,.
$$
Таким образом,
\begin{equation}
\label{fwlower}
\overline F_{W^{(1)}}(x)=\sum\limits_{k=1}^3 \hat\gamma_k e^{-\theta_k x} + \left(1- \rho^{(1)}\right),
\end{equation}
где
$\hat\gamma_1=0{,}007$, $\hat\gamma_2\hm=0{,}027$ и~$\hat\gamma_3\hm=0{,}604$. Заметим, что 
$\sum\nolimits_{k=1}^3 \hat\gamma_k\hm=\rho^{(1)}\hm\approx 0{,}638\hm <1.$

Воспользуемся выражением~\eqref{fwlower} для по\-стро\-ения ф.\,р.\ 
в~сис\-те\-ме~$\Sigma^{(1)}$, а~для по\-стро\-ения оценок в~исходной~$\Sigma$ и~верхней~$\Sigma^{(2)}$ 
сис\-те\-мах воспользуемся имитационным моделированием.
Построим графики (эмпирических) ф.~р.\ для незавершенной 
работы в~трех сис\-те\-мах. Как видно на рисунке, стохастический 
порядок~\eqref{theor1-1} для стационарных времен ожидания выполнен, что 
и~следовало ожидать.






\section{Заключение}

В работе показано, что для исходной системы с~повторными вызовами можно 
построить минорантную и~мажорантную системы так, что стационарная незавершенная 
нагрузка во всех трех сис\-те\-мах будет стохастически упорядочена. Численный 
эксперимент для сис\-те\-мы с~показательными временами обслуживания подтверждает 
теоретические выводы. При этом в~качестве примера рас\-смот\-ре\-ны такие па\-ра\-мет\-ры 
(как в~работе~\cite{rego}), для которых получена ф.\,р.\ 
стационарного времени ожидания в~явном виде в~минорантной сис\-теме.


{\small\frenchspacing
 { %\baselineskip=10.6pt
 %\addcontentsline{toc}{section}{References}
 \begin{thebibliography}{99}

\bibitem{Ar1}
\Au{Artalejo J.\,R.} {Accessible bibliography on retrial queues}~// Math. 
Comput. Model., 1999. Vol.~30. Iss.~3-4. P.~1--6. doi: 10.1016/S0895-7177(99)00128-4.


\bibitem{Ar3}
\Au{Artalejo J.,   Gomez-Corral~A.}
{Retrial queueing systems: A~computational approach}.~---   Springer, 2008. 318~p.
doi: 10.1007/978-3-540-78725-9.


\bibitem{F86} 
\Au{Fayolle G.}
A~simple telephone exchange with delayed feedbacks~// 
Seminar (International) on Teletraffic Analysis and Computer Performance 
Evaluation Proceedings.~--- Elseiver Science, 1986. P.~245--253.

\bibitem{CSA92}
\Au{Choi~B.\,D.,  Shin~Y.\,W.,  Ahn~W.\,C.}
Retrial queues with collision arising from unslotted {CSMA/CD} protocol~//
Queueing Syst., 1992.  Vol.~11. P.~335--356. doi: 10.1007/ BF01163860.

\bibitem{CRP93}
\Au{Choi B.\,D., Rhee~K.\,H., Park~K.\,K.} {The $M/G/1$ retrial queue with
retrial rate control policy}~//
Probab.  Eng. Inform. Sc., 1993.  Vol.~7. P.~29--46. doi: 10.1017/ S0269964800002771.

\bibitem{BG92}
\Au{Bertsekas D., Gallager~R.}
{Data networks}.~--- Athena Scientific, 2021.  570~p.

\bibitem{AY08}
\Au{Avrachenkov K., Yechiali~U.}
Retrial networks with finite buffers and their application to Internet data 
traffic~//  Probab. Eng. Inform. Sc., 2008. 
Vol.~22. P.~519--536. doi: 10.1017/S0269964808000314.

\bibitem{AY10} %8
\Au{Avrachenkov K., Yechiali~U.}
{On tandem blocking queues with a~common retrial queue}~// Comput.  
Oper. Res., 2010. Vol.~37. Iss.~7. P.~1174--1180. doi: 10.1016/j.cor.2009. 10.004.



\bibitem{Yaoetal02} %9
\Au{Yao S., Xue~F.,  Mukherjee~B.,  Yoo~S.\,J.\,B., Dixit~S.}
{Electrical ingress buffering and traffic aggregation for optical packet 
switching and their effect on TCP-level performance in optical mesh networks}~//
IEEE Commun. Mag., 2002.
Vol.~40. Iss.~9. P.~66--72. doi: 10.1109/MCOM. 2002.1031831.

\bibitem{Wongetal09} %10
\Au{Wong E.\,W.\,M.,  Andrew L.\,L.\,H.,  Cui~T.,  Moran~B.,  Zalesky~A., Tucker~R.\,S., Zukerman~M.}
{Towards a~bufferless optical internet}~//
J.~Lightwave Technol., 2009. Vol.~27. P.~2817--2833. doi: 10.1109/JLT.2009.2017211.

\bibitem{mathematics2022} %11
\Au{Morozov E.\,V., Peshkova~I.\,V., Rumyantsev~A.\,S.} Bounds and maxima for the 
workload in a~multiclass orbit queue~// Mathematics, 2023. Vol.~11. Iss.~3. 
Art.~564. doi: 10.3390/math11030564.

\bibitem{pesh-mor2022}  %12
\Au{Peshkova I., Morozov~E.} On comparison of 
multiserver systems with multicomponent mixture distributions~// J.~Math. Sci., 2022. Vol.~267. No.\,2. P.~260--272.
doi: 10.1007/ s10958-022-06132-z. 

\bibitem{pesh2022} %13
\Au{Пешкова И.\,В.} 
Границы экстремального индекса времени ожидания в~системе
$M/G/1$ с~распределением времени обслуживания в~виде конечной
смеси~// Информатика и~её применения, 2022.
Т.~16. Вып.~2. С.~26--33. doi: 10.14357/19922264220405. EDN: VFKRKT.



\bibitem{rego} %14
\Au{Rego V.}
Some explicit formulas for mixed exponential service systems~//
Computers Operations Research, 1988. Vol.~15. Iss.~6. P.~509--520. doi: 
{10.1016/0305-0548(88)90047-0}.

\bibitem{Morozov2019}  %15
\Au{Morozov E.\,V.,   Rumyantsev~A.\,S., Dey~S.,  Deepak~T.\,G.}
Performance analysis and stability of multiclass orbit queue with constant 
retrial rates and balking~//
 Perform. Evaluation, 2019.  Vol.~134. Art.~102005. doi: 
10.1016/ J.PEVA.2019.102005.

\bibitem{Asmus} %16
\Au{Asmussen S.} Applied probability and queues. Stochastic modelling and 
applied probability.~--- New York, NY, USA: Springer-Verlag, 2003. 438~p.

\bibitem{Ross} %17
\Au{Ross S., Shanthikumar~J., Zhu~Z.}  On increasing-failure-rate random 
variables~// J.~Appl. Probab., 2005. Vol.~42. P.~797--809. doi: 
10.1239/jap/1127322028.

\bibitem{haji} %18
\Au{Haji R.,  Newell~G.\,F.}  A~relation between stationary queue and waiting 
time distributions~// J.~Appl. Probab., 1971. Vol.~8. P.~617--620. doi: 10.2307/3212186.




\end{thebibliography}

 }
 }

\end{multicols}

\vspace*{-10pt}

\hfill{\small\textit{Поступила в~редакцию 26.08.23}}

\vspace*{8pt}

%\pagebreak

%\newpage

%\vspace*{-28pt}

\hrule

\vspace*{2pt}

\hrule



\def\tit{BOUNDS OF THE WORKLOAD IN~A~MULTICLASS RETRIAL QUEUE WITH~EXPONENTIAL SERVICES}


\def\titkol{Bounds of the workload in~a~multiclass retrial queue with~exponential services}


\def\aut{I.\,V.~Peshkova$^{1,2}$}

\def\autkol{I.\,V.~Peshkova}

\titel{\tit}{\aut}{\autkol}{\titkol}

\vspace*{-10pt}


\noindent 
$^1$Petrozavodsk State University, 33~Lenina Pr., Petrozavodsk 185910, Russian Federation

\noindent 
$^2$Karelian Research Center of
the Russian Academy of Sciences, 11~Pushkinskaya Str., Petrozavodsk 185910,\linebreak
$\hphantom{^1}$Russian Federation 

\def\leftfootline{\small{\textbf{\thepage}
\hfill INFORMATIKA I EE PRIMENENIYA~--- INFORMATICS AND
APPLICATIONS\ \ \ 2023\ \ \ volume~17\ \ \ issue\ 4}
}%
 \def\rightfootline{\small{INFORMATIKA I EE PRIMENENIYA~---
INFORMATICS AND APPLICATIONS\ \ \ 2023\ \ \ volume~17\ \ \ issue\ 4
\hfill \textbf{\thepage}}}

\vspace*{3pt}

 


\Abste{A~multiclass retrial queue with Poisson input and $M$ classes of customers is investigated. 
For the given retrial system with exponential service times, the lower and upper bounds of the workload are derived. 
It is shown that the workload in the classical system $M/H_M/1$ with hyperexponential service times is the lower bound for the workload of the given retrial system. 
The upper bound is the workload in the classical $M/G/1$ system where each customer occupies the server for the given service time and additional
 time corresponding to the inter-retrial time from the ``slowest'' orbit. 
The presented simulation results confirm the theoretical conclusions.}


\KWE{retrial queue; workload; stochastic ordering}  




\DOI{10.14357/19922264230408}{UOKQRD}

\vspace*{-12pt}

\Ack

\vspace*{-4pt}

\noindent
The research has been prepared with the support of the Russian Science Foundation according to
the research project No.\,21-71-10135. 



  \begin{multicols}{2}

\renewcommand{\bibname}{\protect\rmfamily References}
%\renewcommand{\bibname}{\large\protect\rm References}

{\small\frenchspacing
 {%\baselineskip=10.8pt
 \addcontentsline{toc}{section}{References}
 \begin{thebibliography}{99} 
%1
\bibitem{Ar1-1}
\Aue{Artalejo, J.\,R.} 1999. Accessible bibliography on retrial queues. \textit{Math.
Comput. Model.} 30(3-4):1--6. doi: 10.1016/S0895-7177(99)00128-4.
%2
\bibitem{Ar3-1}
\Aue{Artalejo, J., and A.~Gomez-Corral.} 2008. 
\textit{Retrial queueing systems: A computational approach}. Springer. 318~p.
doi: 10.1007/978-3-540-78725-9.
%3
\bibitem{F86-1} 
\Aue{Fayolle, G.} 1986. 
A simple telephone exchange with delayed feedbacks. \textit{Seminar (International) on Teletraffic Analysis and Computer Performance Evaluation Proceedings}.
Elseiver Science. 245--253.
%4
\bibitem{CSA92-1}
\Aue{Choi, B.\,D., Y.\,W.~Shin, and W.\,C.~Ahn.} 1992. 
Retrial queues with collision arising from unslotted \mbox{CSMA}/CD protocol. \textit{Queueing Syst.} 11:335--356.  
doi: 10.1007/ BF01163860.
%5
\bibitem{CRP93-1}
\Aue{Choi, B.\,D., K.\,H.~Rhee, and K.\,K.~Park.} 1993. 
The $M/G/1$ retrial queue with retrial rate control policy.
\textit{Probab. Eng. Inform. Sc.} 7(1):29--46.
doi: 10.1017/ S0269964800002771.
%6
\bibitem{BG92-1}
\Aue{Bertsekas, D., and R.~Gallager.} 2021.
\textit{Data networks}. Athena Scientific. 570~p.
%7
\bibitem{AY08-1}
\Aue{Avrachenkov, K., and U.~Yechiali.} 2008.
Retrial networks with finite buffers and their application to Internet data traffic. \textit{Probab. Eng. Inform. Sc.} 22(4):519--536.
doi: 10.1017/S0269964808000314.
%8
\bibitem{AY10-1} 
\Aue{Avrachenkov, K., and U.~Yechiali.} 2010.
On tandem blocking queues with a~common retrial queue. \textit{Comput. Oper. Res.} 37(7):1174--1180.
doi: 10.1016/j.cor.2009.10.004.

%9
\bibitem{Yaoetal02-1}
\Aue{Yao, S., F.~Xue, B.~Mukherjee, S.\,J.\,B.~Yoo, and S.~Dixit.} 2002.
Electrical ingress buffering and traffic aggregation for optical packet switching and their
effect on TCP-level performance in optical mesh networks.
\textit{IEEE Commun. Mag.} 40(9):66--72. doi: 10.1109/MCOM.2002.1031831.

%10
\bibitem{Wongetal09-1}
\Aue{Wong, E.\,W.\,M., L.\,L.\,H.~Andrew, T.~Cui, B.~Moran, A.~Zalesky, R.\,S.~Tucker, and M.~Zukerman.} 2009.
Towards a~bufferless optical internet.
\textit{J.~Lightwave Technol.} 27(14):2817--2833. doi: 10.1109/JLT.2009.2017211.

%11
\bibitem{mathematics2022-1}
\Aue{Morozov, E.\,V., I.\,V.~Peshkova, and A.\,S.~Rumyantsev.}
 2023. Bounds and maxima for the workload in a~multiclass orbit queue. \textit{Mathematics} 11(3):564. doi: 10.3390/ math11030564.

%12
\bibitem{pesh-mor2022-1} 
\Aue{Peshkova, I., and E.~Morozov.} 2022. On comparison of multiserver systems with multicomponent mixture distributions. 
\textit{J.~Math. Sci.} 267(2):260--272. doi: 10.1007/ s10958-022-06132-z.
%13
\bibitem{pesh2022-1}
\Aue{Peshkova, I.\,V.} 2022. Granitsy ekstremal'nogo in\-dek\-sa vre\-me\-ni ozhi\-da\-niya v~sis\-te\-me $M/G/1$ 
s~raspredeleniem vremeni obsluzhivaniya v~vide konechnoy
smesi [On bounds of the stationary waiting time extremal index in $M/G/1$
system with mixture service times]. \textit{Informatika i~ee Primeneniya~--- Inform. Appl.} 16(4):26--33. doi: 10.14357/19922264220405. EDN: VFKRKT.

%14
\bibitem{rego-1}
\Aue{Rego, V.} 1988. 
Some explicit formulas for mixed exponential service systems. 
\textit{Comput. Oper. Res.} 15(6):509--520. doi: 10.1016/0305-0548(88)90047-0.
%15
\bibitem{Morozov2019-1} 
\Aue{Morozov, E.\,V., A.\,S.~Rumyantsev, S.~Dey, and T.\,G.~Deepak.} 2019.
Performance analysis and stability of multiclass orbit queue with constant retrial rates and balking.
\textit{Perform. Evaluation} 134:102005. doi: 10.1016/ J.PEVA.2019.102005.
%16
\bibitem{Asmus-1}
\Aue{Asmussen, S.} 2003. \textit{Applied probability and queues. Stochastic modelling and 
applied probability.} New York, NY: Springer. 438~p.

%17
\bibitem{Ross-1}
\Aue{Ross, S., J.~Shanthikumar, and Z.~Zhu.}
 2005. On increasing-failure-rate random variables. \textit{J.~Appl. Probab.} 42(3):797--809. doi: 10.1239/jap/1127322028.
 
 %18
\bibitem{haji-1}
\Aue{Haji, R., and G.\,F.~Newell.} 1971. A~relation between stationary queue and waiting time distributions. \textit{J. Appl. Probab.} 8(3):617--620.
doi: 10.2307/3212186.

\end{thebibliography}

 }
 }

\end{multicols}

\vspace*{-6pt}

\hfill{\small\textit{Received August 26, 2023}} 

%\vspace*{-18pt}

\Contrl

\vspace*{-4pt}

\noindent
\textbf{Peshkova Irina V.} (b.\ 1975)~--- 
Candidate of Science (PhD) in physics and mathematics, associate professor, Petrozavodsk State University, 33~Lenina Pr., Petrozavodsk 185910, 
Russian Federation; senior scientist, Karelian Research Center of the Russian Academy of Sciences, 
11~Pushkinskaya Str., Petrozavodsk 185910, Russian Federation; \mbox{iaminova@petrsu.ru}


\label{end\stat}

\renewcommand{\bibname}{\protect\rm Литература}  %5-
\def\stat{agalarov}


\def\tit{ПРИБЛИЖЕННЫЙ МЕТОД ВЫЧИСЛЕНИЯ ХАРАКТЕРИСТИК УЗЛА 
ТЕЛЕКОММУНИКАЦИОННОЙ СЕТИ С~ПОВТОРНЫМИ ПЕРЕДАЧАМИ}
\def\titkol{Приближенный метод вычисления характеристик узла 
телекоммуникационной сети с~повторными передачами} 

\def\autkol{Я.\,М.~Агаларов}
\def\aut{Я.\,М.~Агаларов$^1$}

\titel{\tit}{\aut}{\autkol}{\titkol}

%{\renewcommand{\thefootnote}{\fnsymbol{footnote}}\footnotetext[1]
%{Работа выполнена при поддержке РФФИ, проекты 08--07--00152 и 08--01--00567.}}

\renewcommand{\thefootnote}{\arabic{footnote}}
\footnotetext[1]{Институт проблем
информатики Российской академии наук, agglar@yandex.ru}

%\vspace*{-6pt}


\Abst{Рассмотрена модель узла коммутации пакетов c повторными передачами для двух 
схем распределения буферной памяти: полнодоступной и полного разделения. Предложен 
приближенный метод вычисления интенсивностей потоков и вероятностей блокировок узла. 
Получены необходимые и достаточные условия существования и единственности решения 
уравнения для потоков в узле при установившемся режиме работы и доказана сходимость 
итерационного метода решения указанного уравнения.}

\KW{узел коммутации пакетов; буферная память; повторные передачи; вероятности 
блокировок; итерационный метод}

      \vskip 18pt plus 9pt minus 6pt

      \thispagestyle{headings}

      \begin{multicols}{2}

      \label{st\stat}


\section{Введение}

    Одной из основных задач предварительного анализа 
телекоммуникационных сетей коммутации пакетов с ограниченной буферной 
памятью является расчет характеристик потоков и вероятностей блокировок в 
узлах связи. Важность указанных характеристик определяется тем, что от их 
значений существенным образом зависят другие основные показатели сети 
(пропускная способность, задержки пакетов и~др.). 

    Существует множество различных моделей узлов коммутации пакетов и 
методов их расчета (см., например,~[1--6]). Для моделей, рассматривающих 
узел с ограниченной буферной памятью как систему массового обслуживания 
(CMO) типа 
$
\begin{matrix}
M \\ \lambda
\end{matrix}
\left |
\begin{matrix}
M \\ \lambda
\end{matrix}
\right |
\overline{m} \vert N
$ или  $\vert PH\vert PH\vert 1\vert r$, в предположении отсутствия повторных 
передач пакетов получены точные методы вычисления характеристик 
узлов~[1, 3, 4, 6]. Приближенные методы расчета узлов, учитывающие повторные 
попытки передачи, используют модели типа $\vert PH\vert PH\vert 1\vert r$ или 
$
\begin{matrix}
M \\ \lambda
\end{matrix}
\left |
\begin{matrix}
M \\ \lambda
\end{matrix}
\right |
1 \vert N
$ и являются 
итерационными~[2, 3, 5, 7]. Для моделей типа 
$BM\!AP\vert PH\vert 1$, $M\vert G\vert 1\vert r$ и $M\!AP\vert 
(PH,PH)\vert 1$ с повторными заявками получены точные методы вычисления 
характеристик (например, в работах~[8--10]), которые также могут быть 
использованы при расчете узлов.

    Ниже будут рассмотрены модели узла коммутации пакетов с повторными 
передачами для двух схем распределения буферной памяти: с 
полнодоступными буферами и с полным разделением буферной памяти. 
Предлагается приближенный метод расчета характеристик, который в качестве 
модели узла использует СМО типа $
\begin{matrix}
M \\ \lambda
\end{matrix}
\left |
\begin{matrix}
M \\ \lambda
\end{matrix}
\right |
\overline{m} \vert N
$ с повторными заявками. Доказаны утверждения о 
достаточных и необходимых условиях существования и единственности 
решения уравнения для вероятности блокировки в установившемся режиме 
работы и сходимости предлагаемого итерационного метода. 

\section{Модель узла}

    Математическая модель узла представляется в виде СМО с ограниченной 
буферной памятью и различными потоками заявок, каждая из которых требует 
обслуживания только на одной из многоканальных линий связи. 

    Пусть $0<N<\infty$~--- число мест хранения в буферной памяти, $u$~--- 
узел связи, $v$~--- линия связи, $\Omega_u^+$~--- множество исходящих из 
узла~$u$ линий, $c_v$~--- канальная емкость линии~$v$. Поток заявок, 
тре\-бу\-ющих обслуживания на линии~$v$, назовем $v$-по\-то\-ком, заявки этого 
потока~--- $v$-за\-яв\-ка\-ми.


    Пусть выполняются следующие предположения: 
\begin{enumerate}[1.]
\item Места в буферной памяти распределяются согласно одной из двух 
схем:
\begin{enumerate}[($i$)]
\item полнодоступная схема~--- каждое свободное место хранения доступно 
любой заявке;
\item схема полного разделения памяти~--- $v$-за\-яв\-кам доступны всего 
$N_v$ мест, где $\sum\limits_{v\in\Omega_u^+} N_v=N$.
\end{enumerate}
\item Если в момент поступления $v$-заявки в буферной памяти есть 
доступное свободное место, то она сразу занимает это место. Если в момент 
поступления $v$-заявки в системе нет свободного доступного места 
хранения, то поступившая заявка через некоторое время повторно поступает 
на систему, оставаясь $v$-заявкой. 
\item Интенсивности первичных потоков $v$-заявок~--- заданные величины 
$0<\Lambda_v<\infty$, $v\in \Omega_u^+$. Суммарные потоки первичных и 
повторных $v$-заявок являются независимыми в совокупности 
пуассоновскими потоками. Для обслуживания $v$-заявки требуется 
одновременно одно место хранения и один канал типа~$v$, $v\in 
\Omega_u^+$.
\item Первичные нагрузки~--- реализуемые, т.\,е.\ в данном случае 
интенсивности входных первичных потоков равны интенсивностям 
выходных потоков выполненных заявок. 
\item Принятые в СМО $v$-заявки обслуживаются линией~$v$ в порядке 
поступления. 
\item Время занятия канала $v$-заявкой~--- экспоненциально 
распределенная случайная величина с параметром $0<\mu_v<\infty$, 
$v\in\Omega_u^+$, независимая от других случайных событий в узле.
\item Выполненная $v$-заявка с вероятностью~$B_v$ повторяется через 
заданное время~$\tau_v$ (тайм-аут) и с вероятностью $1-B_v$ покидает 
систему через время~$t_v$ навсегда, сразу освободив занятый канал и место 
буферной памяти.
\end{enumerate}

   Будем говорить, что узел блокирован для $v$-за\-яв\-ки, если в буферной 
памяти отсутствует доступное место хранения. Ставится задача вычисления 
вероятностей блокировок и интенсивностей потоков в узле.

\section{Вычисление вероятности блокировки и~интенсивностей~потоков} 

   Пусть $\Lambda_v^*$~--- интенсивность суммарного потока внешних 
заявок, требующих передачи по линии~$v$, $\pi_v$~--- вероятность блокировки 
узла для заявок, требующих передачи по исходящей из узла линии~$v$. 

    Пусть в узле используется полнодоступная схема распределения 
буферной памяти. Тогда, как следует из описания модели, $\pi_v 
=\pi_{v^\prime},\,v,\,v^\prime\in \Omega_u^+$, и для 
интенсивностей~$\Lambda_v^*$, $v\in\Omega_u^+$, справедливы соотношения:
\begin{equation*}
\Lambda_v^* = \fr{\Lambda_v}{1-\pi}\,,
%\label{e1aga}
\end{equation*}
    где
    $\pi =\pi_v$, $v\in\Omega_u^+$.

    Пусть 
    $\overline{k} = \{\overline{k}_v$, $v\in\Omega_u^+\}$~--- состояние 
буферной памяти узла, $\overline{k}_v =\left ( k_v,\,k_v^\prime,\,k_v^{\prime\prime}\right )$; 
$k_v$~--- число $v$-заявок в буферной 
памяти, ожидающих выполнения линией~$v$; $k^\prime_v$~--- число 
$v$-заявок в буферной памяти, ожидающих тайм-аут и неуспешно переданных 
в последующий узел; $k_v^{\prime\prime}$~--- число $v$-за\-явок в буферной 
памяти, успешно переданных в последующий узел и ожидающих 
потверждения; 
$A_m = \left \{ \overline{k}:\ \sum\limits_{v\in\Omega_u^+} \left ( 
k_v+k_v^\prime + k_v^{\prime\prime}\right ) =m \right \}$~--- множество различных 
состояний, при которых в памяти узла занято ровно $m$~буферов. Тогда с 
учетом введенных выше обозначений и предположений для ве\-ро\-ят\-ности 
блокировки узла можно написать формулу~\cite{1aga, 2aga}:
\begin{equation}
\pi = \fr{1}{G_N}\sum\limits_{\overline{k}\in A_N} 
p\left (\overline{k},\overline{\rho}^*\right )\,,
\label{e2aga}
\end{equation}
где  
\begin{gather}
p(\overline{k},\overline{\rho}^*) = \prod\limits_{v\in\Omega_u^+} z_v (\pi, 
\rho_v , k_v , k_v^\prime , k_v^{\prime\prime})\,;\\
z_v (\pi, \rho_v , k_v , k_v^\prime , k_v^{\prime\prime}) ={}\notag\\
\!\!{}=
\begin{cases}
 \fr{\rho_v^{\prime *k_v^\prime}}{k_v^{\prime}!}\,
\fr{\rho_v^{\prime\prime * k_v^{\prime\prime}}}{ k_v^{\prime\prime}!}  \,
\fr{\rho_v^{*k_v}}{ k_{v}!} 
&\mbox{при}\ k_v<c_v\,,\\
 \fr{\rho_v^{\prime * k_v^\prime}}{k_v^{\prime}!} \,
\fr{\rho_v^{\prime\prime * k_v^{\prime\prime}}} { k_v^{\prime\prime}!} 
\fr{\rho_v^{*k_v}}{ c_{v}!c_v^{k_v- c_v}} 
& \mbox{при}\ k_v\geq c_v\,;
\end{cases}\\
G_N = \sum\limits_{m=0}^N\sum\limits_{\overline{k}\in A_m}
p(\overline{k},\overline{\rho}^*)\,;\\ 
\overline{\rho}^*=\{\rho_v^*,\,v\in\Omega_u^+\}\,;\\
\rho_v^* = \fr{\rho_v}{1-\pi}\,;\quad \rho_v =\fr{\Lambda_v}{\mu_v(1- B_v)}\,;\\
\rho_v^{\prime *} =\rho_v^*\mu_v\tau_vB_v\,;\quad \rho_v^{\prime\prime *}=
p_v^* \mu_vt_v,\,\quad  v\in \Omega_u^+\,.\label{e3aga}
\end{gather}

Переобозначив $1-\pi$ через $y$, выражение в правой части равенства~(2)~--- через 
$p_{\overline{k}}(\overline{\rho},y)$, выражение в правой части равенства~(4)~--- 
через $g_N(\overline{\rho},y)$, а выражение в правой 
части равенства~(1)~--- через $1-q_N (\overline{\rho},y)$, 
где $\overline{\rho} = (\rho_v,\,v\in \Omega_u^+)$, $\rho_v = \rho_v^*y\;=$\linebreak 
$=\;\Lambda_v/(\mu_v(1-B_v))$, $v\in\Omega_u^+$, получим нелинейное уравнение 
относительно неизвестной переменной~$y$:
\begin{equation}
y=q_N(\overline{\rho},y)\,.
\label{e4aga}
\end{equation}

    Решим уравнение~(8). Как следует из~(2)--(7), верно 
равенство
\begin{equation}
q_N(\overline{\rho},y) = \fr{g_{N-1}(\overline{\rho},y )}{g_N(\overline{\rho},y)}\,.
\label{e5aga}
\end{equation}
Введем функцию  $d_n(\overline{\rho} ,y)$ среднего числа заявок в узле с 
буферной памятью емкости $n\geq 0$:
$$
d_n(\overline{\rho} ,y) = 
\fr{1}{g_n(\overline{\rho},y)}\,\sum\limits_{m=0}^n m\sum\limits_{\overline{k}\in 
A_m} p_{\overline{k}}(\overline{\rho},y)\,.
$$
Заметим, что $g_n$, $d_n$ и $q_n$, 
$n\geq 0$,~--- непрерывно-дифференцируемые функции по $y\in (0,\,1]$. Взяв 
производную функции~$g_n$ по~$y$, из~(2)--(7) получим
\begin{multline}
\fr{\partial g_n(\overline{\rho},y)}{\partial y} ={}\\
{}= -\sum\limits_{m=0}^n m 
\sum\limits_{\overline{k}\in A_m}\fr{\prod\limits_{v\in\Omega_u^+} z_n 
(0,\rho_v, k_v, k_v^\prime , k_v^{\prime\prime})}{y^{m+1}}={}\\
{}= -\fr{1}{y}\,g_n (\overline{\rho},y)d_n(\overline{\rho},y)\,.
\label{e6aga}
\end{multline}
Взяв производную функции $q_N$ по $y$, из~(\ref{e5aga}) и~(\ref{e6aga}) 
получим
\begin{equation}
\fr{\partial q_N(\overline{\rho},y)}{\partial y} = \fr{q_N(\overline{\rho},y)}{y}\left 
[ d_N (\overline{\rho},y)-d_{N-1}(\overline{\rho},y)\right ]\,.
\label{e7aga}
\end{equation}
    Докажем несколько утверждений о свойствах 
функции~$q_N(\overline{\rho},y)$.
\medskip

\noindent
\textbf{Утверждение 1.} \textit{Справедливы неравенства}
\begin{multline}
0<d_{n+1}(\overline{\rho},y)-d_n(\overline{\rho},y) <1\,,\\
\ \ \ \ \ \ \ \ \ \ \ \ \ \ \ \ \ \ \ \ y\in (0,\,1]\,, \ n\geq 0\,.
\label{e8aga}
\end{multline}


\noindent

Д\,о\,к\,а\,з\,а\,т\,е\,л\,ь\,с\,т\,в\,о\,.\ Подставив выражение для функции 
$d_n(\overline{\rho},y)$ и проведя преобразования, получим
\begin{multline*}
d_{n+1}(\overline{\rho},y) -d_n(\overline{\rho},y) = 
\fr{\sum\limits_{m=0}^{n+1}m\sum\limits_{\overline{k}\in A_m} 
p_{\overline{k}}(\overline{\rho},y)}
{\sum\limits_{m=0}^{n+1}
\sum\limits_{\overline{k}\in A_m} p_{\overline{k}}(\overline{\rho},y)} - {}\\
{}-
\fr{\sum\limits_{m=0}^n m \sum\limits_{\overline{k}\in A_m} p_{\overline{k}} 
(\overline{\rho},y)}{\sum\limits_{m=0}^n
\sum\limits_{\overline{k}\in A_m}p_{\overline{k}}(\overline{\rho},y)}={}\\
{}=\fr{\sum\limits_{m=1}^n m \sum\limits_{\overline{k}\in 
A_m}p_{\overline{k}}(\overline{\rho},y)+(n+1)\sum\limits_{\overline{k}\in 
A_{n+1}}  p_{\overline{k}}(\overline{\rho},y)}{\sum\limits_{m=0}^n\sum\limits_{\overline{k
}\in A_m}p_{\overline{k}}(\overline{\rho},y)+\sum\limits_{\overline{k}\in 
A_{n+1}}p_{\overline{k}}(\overline{\rho},y)} -{}
\end{multline*}
\begin{multline}
{}-
\fr{\sum\limits_{m=0}^n m 
\sum\limits_{\overline{k}\in A_m}p_{\overline{k}}(\overline{\rho},y)}
{\sum\limits_{m=0}^n\sum\limits_{\overline{k}\in A_m} 
p_{\overline{k}}(\overline{\rho},y)}={}\\
{}=\fr{(n+1)\sum\limits_{\overline{k}\in 
A_{n+1}}p_{\overline{k}}(\overline{\rho},y)g_n(\overline{\rho},y)}{g_{n+1}(\overline{\rho},y) g_n(\overline{\rho},y)} -{}\\
{}-
\fr{\sum\limits_{\overline{k}\in 
A_{n+1}}p_{\overline{k}}(\overline{\rho},y)\sum\limits_{m=0}^n  m 
\sum\limits_{\overline{k}\in A_m} p_{\overline{k}}(\overline{\rho},y) }
{g_{n+1}(\overline{\rho},y) g_n(\overline{\rho},y)}
={}\\
{}=\left [ 1-q_{n+1}(\overline{\rho},y)\right ] \left [n+1-d_n(\overline{\rho},y)\right ]\,.
\label{e9aga}
\end{multline}


    Докажем утверждение~1 методом индукции. При $n = 0$, как следует 
из~(\ref{e9aga}), имеем
$$
d_2(\overline{\rho},y) - d_1 (\overline{\rho},y) =1-q_1(\overline{\rho},y)\,,
$$
    т.\,е.\ утверждение~1 при $n = 0$ справедливо. 

    Пусть неравенства~(\ref{e8aga}) справедливы для некоторого $n > 0$. 
Докажем, что они справедливы и для $n + 1$. Из~(\ref{e9aga}) получаем
\begin{multline*}
d_{n+1}(\overline{\rho},y)- d_n(\overline{\rho},y)={}\\
{}=\left [ 1-
q_{n+1}(\overline{\rho},y)\right ] \left [n+1-d_n(\overline{\rho},y)\right ] ={}\\
{}= \left [ 1-
1-q_{n+1}(\overline{\rho},y)\right ] \left [ n-{}\right.\\
{}-\left. d_{n-1}(\overline{\rho},y)+d_{n-1}(\overline{\rho},y)-
d_n(\overline{\rho},y)+1\right ] ={}\\
{}=\left [ 1-q_{n+1}(\overline{\rho},y)\right ] 
\left [ n-d_{n-1}(\overline{\rho},y)-{}\right.\\
{}-\left. \left ( d_n(\overline{\rho},y)-d_{n-1}(\overline{\rho},y)\right )+1\right] = {}\\
{}=
\left [ 1-q_{n+1}(\overline{\rho},y)\right ]
\left [ 
\fr{d_n(\overline{\rho},y) -d_{n-1}(\overline{\rho},y)}{1-
q_n(\overline{\rho},y)}\right.-{}\\
{}-\left.
\left ( d_n(\overline{\rho},y)-d_{n-1}(\overline{\rho},y)\right )+1
\vphantom{\fr{d_n(\overline{\rho})}{(q_n)}}
\right ]={}\\
{}=
\left [ 1-q_{n+1}(\overline{\rho},y)\right ]
\left [ 
\vphantom{\fr{d_n(\overline{\rho})}{(q_n)}}
\left ( d_n(\overline{\rho},y\right)\right. -{}\\
 {}-\left.
d_{n-1}\left(\overline{\rho},y)\right )\fr{q_n(\overline{\rho},y)}{1-
q_n(\overline{\rho},y)}+1\right ]\,.
\end{multline*}
Так как по предположению $d_n (\overline{\rho},y) -d_{n-1}(\overline{\rho},y) 
>0$, то правая часть последнего равенства больше нуля; следовательно, 
$d_{n+1}(\overline{\rho},y)-d_n(\overline{\rho},y)>0$. 

    Продолжив преобразование правой части последнего равенства и 
учитывая предположение $d_n(\overline{\rho},y) -d_{n-1}(\overline{\rho},y)<1$, 
получим
\begin{multline*}
d_{n+1}((\overline{\rho},y) -d_n(\overline{\rho},y)<{}\\
{}< \left [ 1-
q_{n+1}(\overline{\rho},y)\right ]
\left ( \fr{q_n(\overline{\rho},y)}{1-q_n(\overline{\rho},y)}+1\right )={}\\
{}=
\fr{1-q_{n+1}(\overline{\rho},y)}{1-q_n(\overline{\rho},y)}<1\,,
\end{multline*}
так как $0< q_n(\overline{\rho},y)<q_{n+1}(\overline{\rho},y)<1$, $n>0$, $y\in 
(0,\,1]$.

Следовательно, утверждение~1 доказано.

\medskip

\noindent
\textbf{Утверждение 2.} $q_N(\overline{\rho},y)$~--- \textit{монотонно-воз\-рас\-та\-ющая 
функция по $y\in (0,\,1]$. При этом $0< q_N(\overline{\rho},y)\;\leq $\linebreak 
$\leq\;q_N(\overline{\rho},1) <1$, $y\in (0,\,1]$,  и $\underset{y\rightarrow 
0}{\mathrm{lim}}\,q_N(\overline{\rho},y) =0$}.

\medskip

\noindent
Д\,о\,к\,а\,з\,а\,т\,е\,л\,ь\,с\,т\,в\,о\,.\  Возрастание функции 
$q_N(\overline{\rho},y)$ следует непосредственно из~(\ref{e7aga}) и 
утверж\-де\-ния~1. Доказательство неравенств в условии утверждения очевидно 
следует из~(\ref{e5aga}) и вида функции $g_n (\overline{\rho},y)$, $n\geq 0$. 
Для предела имеем:
\begin{multline*}
\underset{y\rightarrow 0}{\mathrm{lim}}\,q_N(\overline{\rho},y) 
=\underset{y\rightarrow 0}{\mathrm{lim}}\,\fr{g_{N- 1}(\overline{\rho},y)}{g_N(\overline{\rho},y)} = {}\\
{}= \underset{y\rightarrow 0}{\mathrm{lim}}\,\left (
g_{N-1}(\overline{\rho},y)\Bigg / \left ( 
\vphantom{\prod\limits_{v\in\Omega_u^+}}
g_{N-1}(\overline{\rho},y)\right.\right.+{}\\
{}+\left.\left.\sum\limits_{\overline{k}\in A_N}\prod\limits_{v\in\Omega_u^+} 
\fr{z_v(0,\rho_v,k_v,k^\prime_v,k^{\prime\prime}_v)}{y^N}\right )\right ) = {}\\
{}= \underset{y\rightarrow 0}{\mathrm{lim}}\,\left (
y^N g_{N-1}(\overline{\rho},y)\Bigg / 
\left ( 
\vphantom{\prod\limits_{v\in\Omega_u^+}}
y^N g_{N-1}(\overline{\rho},y)+{}\right.\right.\\
{}+\left.\left.\sum\limits_{\overline{k}\in A_N}
\prod\limits_{v\in\Omega_u^+} z_v(0,\rho_v,k_v,k_v^\prime , k_v^{\prime\prime}) 
\right ) \right )=0\,.
\end{multline*}
    
\medskip

\noindent
\textbf{Утверждение 3.} \textit{Производная функции~$q_N (\overline{\rho},y)$ по 
$y\in (0,\,1]$ удовлетворяет следующим соотношениям}:
\begin{align}
\underset{y\rightarrow 0}{\mathrm{lim}}\fr{\partial q_N(\overline{p},y)}
{\partial  y} &= \fr{\sum\limits_{\overline{k}\in A_{N-1}} 
p_{\overline{k}}(\overline{\rho},1)}{\sum\limits_{\overline{k}\in 
A_N}p_{\overline{k}}(\overline{\rho},1)}\,;\label{e10aga}\\
\fr{\partial q_N(\overline{\rho},y)}{\partial y}\Big |_{y=1}&<1\,.\label{e11aga}
\end{align}

\medskip

\noindent
Д\,о\,к\,а\,з\,а\,т\,е\,л\,ь\,с\,т\,в\,о\,.\ Проведя преобразования 
функции~$q_N(\overline{\rho},y)$, получим:
\begin{multline*}
\underset{y\rightarrow 0}{\mathrm{lim}}\fr{q_N(\overline{\rho},y)}{y} = {}\\
\!\!{}=
\underset{y\rightarrow 0}{\mathrm{lim}}
\fr{\sum\limits_{m=0}^{N-1}\sum\limits_{\overline{k}\in A_m}
\!\!\left (\prod\limits_{v\in\Omega_u^+}\!\! 
z_v(0,\rho_v,k_v,k_v^\prime , k_v^{\prime\prime})\right )\!\!\Bigg /\!\! y^m}
{y\sum\limits_{m=0}^{N}\sum\limits_{\overline{k}\in A_m}
\!\!\left(\prod\limits_{v\in\Omega_u^+}\!\! z_v\left (0,\rho_v,k_v,k_v^\prime , 
k_v^{\prime\prime}\right )\right )\!\!\Bigg /\!\!y^m} = \!\!\!
\end{multline*}
\begin{multline*}
\!\!\!\!\!\!{}=\underset{y\rightarrow 0}{\mathrm{lim}}\,
\fr{\sum\limits_{m=0}^{N-1}\sum\limits_{\overline{k}\in A_m}
y^{N-1-m}\prod\limits_{v\in\Omega_u^+} z_v(0,\rho_v,k_v,k_v^\prime , 
k_v^{\prime\prime})}{\sum\limits_{m=0}^{N}\sum\limits_{\overline{k}
\in A_m} y^{N-m}
\prod\limits_{v\in\Omega_u^+} z_v(0,\rho_v,k_v,k_v^\prime , 
k_v^{\prime\prime})}={}\!\\
{}=\fr{\sum\limits_{\overline{k}\in A_{N-1}} p_{\overline{k}}(\overline{\rho},1)}{ 
\sum\limits_{\overline{k}\in A_{N}} p_{\overline{k}}(\overline{\rho},1)}\,.
\end{multline*}
Очевидно, $\underset{y\rightarrow 0}{\mathrm{lim}} \,[d_N (\overline{\rho},y) -
d_{N-1} (\overline{\rho},y)]=1$, так как $\underset{y\rightarrow 
0}{\mathrm{lim}}\,d_n (\overline{\rho},y)=n$, $n>0$.

Следовательно, учитывая~(\ref{e7aga}), получаем~(\ref{e10aga}). 
Справедливость~(\ref{e11aga}) непосредственно следует из~(\ref{e7aga}) и 
утверждения~1.

\medskip

\noindent
\textbf{Утверждение 4.} \textit{Пусть $y^*\in (0,\,1]$~--- решение 
уравнения}~(\ref{e4aga}). \textit{Тогда}
\begin{equation*}
\fr{\partial q_N(\overline{\rho},y)}{\partial y}\Big |_{y=y^*}<1\,.
%\label{e12aga}
\end{equation*}

\medskip

\noindent
Д\,о\,к\,а\,з\,а\,т\,е\,л\,ь\,с\,т\,в\,о\,.\ \ Доказательство следует из~(\ref{e7aga}), 
так как $q_N(\overline{\rho},y^*)/y^* =1$.
\medskip

\noindent
\textbf{Утверждение 5.} \textit{Уравнение}~(\ref{e4aga}) \textit{имеет решение $y^*\in 
(0,\,1)$ тогда и только тогда, когда} 
\begin{equation}
\fr{\sum\limits_{\overline{k}\in A_{N-1}} p_{\overline{k}}(\overline{\rho},1)}{ 
\sum\limits_{\overline{k}\in A_{N}} p_{\overline{k}}(\overline{\rho},1)} >1\,.
\label{e13aga}
\end{equation}
\textit{Если уравнение}~(\ref{e4aga}) \textit{имеет решение $y^*\in (0,\,1)$, то оно 
единственное положительное решение}.
\medskip

\noindent
Д\,о\,к\,а\,з\,а\,т\,е\,л\,ь\,с\,т\,в\,о\,.\ Пусть выполняется 
неравенство~(\ref{e13aga}). Тогда, как следует из утверждения~3, 
$\underset{y\rightarrow 0}{\mathrm{lim}} (\partial q_N(\overline{\rho},y)/\partial y) 
>1$. Кроме того, как следует из утверждения~2, 
$\underset{y\rightarrow 0}{\mathrm{lim}} q_N(\overline{\rho},y)=0$. Тогда, так 
как $q_N(\overline{\rho},y)$~--- непрерывно-дифференцируемая функция по 
$y\in (0,\,1]$, существует значение $y^\prime \in (0,\,1)$ такое, что 
$q_N(\overline{\rho},y)>y$ для всех $y\in (0,\,y^\prime]$ (следует из теоремы о 
конечном приращении~\cite{11aga}). В то же время, согласно утверждению~2, 
$q_N(\overline{\rho},y)<y$ в окрестности точки $y=1$ (рис.~\ref{f1aga},\,\textit{а}). 
Следовательно, кривая $x=q_N(\overline{\rho},y)$ пересекает прямую $x=y$ 
хотя бы в одной точке $y=y^*\in (0,\,1)$, т.\,е.\ уравнение~(\ref{e4aga}) имеет 
хотя бы одно решение $y^*\in (0,\,1)$.

\begin{figure*}
\vspace*{1pt}
\begin{center}
\vspace*{1pt}
\mbox{%
\epsfxsize=158mm
\epsfbox{aga-1.eps}
}
\end{center}
\vspace*{-9pt}
\Caption{Примеры кривых $x=q_N(\overline{\rho},y)$ и $x=y$ (\textit{а})~при существовании решения 
уравнения~(\ref{e5aga}) и (\textit{б})~при выполнении условий~(17)
\label{f1aga}}
\vspace*{6pt}
\end{figure*}

Пусть уравнение~(\ref{e4aga}) имеет решение $y^*\in (0,\,1)$ и 
\begin{equation}
\fr{\sum\limits_{\overline{k}\in A_{N-1}}p_{\overline{k}}(\overline{\rho},1)}{ 
\sum\limits_{\overline{k}\in A_{N}}p_{\overline{k}}(\overline{\rho},1)}\leq 
1\,.\label{e14aga}
\end{equation}
Тогда из условий утверждений~2 и~3 следует, что 
уравнение~(\ref{e4aga}) в интервале $(0,\,1)$ имеет более одного решения, что 
может быть только при существовании решения $y^\prime \in (0,\,1)$ такого, 
что в окрестности точки $y=y^\prime$ выполняются неравенства: 
$q_N(\overline{\rho},y)<y$ при $y<y^\prime$ и $q_N(\overline{\rho},y)>y$ при 
$y>y^\prime$, где $y$ принадлежит указанной окрест\-ности точки~$y^\prime$ 
(рис.~\ref{f1aga},\,\textit{б}). Тогда в точке $y=y^\prime$ производная функции 
$q_N(\overline{\rho},y)$ по $y$ больше~1, что противоречит утверждению~4. 
Следовательно, неравенство~(\ref{e13aga}) справедливо.


Пусть уравнение~(\ref{e4aga}) имеет более одного положительного 
решения. Рассуждая точно так же, как и выше (в случае выполнения 
условий~(\ref{e14aga})), получим противоречие утверждению~4. 
Следовательно, утверждение~5 справедливо.
\medskip

\noindent
\textbf{Следствие.} \textit{Неравенства}
\begin{gather*}
\fr{\mu_v c_v (1-B_v)}{\Lambda_v}>1\,,\quad \fr{1-B_v}{\Lambda_v \tau_v B_v}>1\,,\\ 
\fr{1-B_v}{\Lambda_v t_v}>1\,,\ v\in\Omega_u^+\,,
\end{gather*}
\textit{являются необходимым условием существования решения 
уравнения}~(\ref{e4aga}).

\medskip
\noindent
Д\,о\,к\,а\,з\,а\,т\,е\,л\,ь\,с\,т\,в\,о\,.\ Пусть $\overline{k}_v$~--- это 
набор~$\overline{k}$, у которого $k_v=0$. Преобразовав левую 
часть~(\ref{e13aga}), получим

\noindent
\begin{multline*}
\fr{\sum\limits_{\overline{k}\in A_{N-1}} p_{\overline{k}} (\overline{\rho},1)}
{ \sum\limits_{\overline{k}\in A_{N}} 
 p_{\overline{k}}(\overline{\rho},1)} 
={}
\\
{}=
\fr{\sum\limits_{\overline{k}\in A_{N-1}}\prod\limits_{v\in \Omega_u^+} 
z_v\left(0,\rho_v,k_v,k_v^\prime , k_v^{\prime\prime}\right)}
{\sum\limits_{\overline{k}\in A_{N}}
\prod\limits_{v\in \Omega_u^+} z_v\left (0,\rho_v,k_v,k_v^\prime , k_v^{\prime\prime}\right )} \leq{}
\\
{}\leq
\left ( 
\vphantom{\prod\limits_{v^\prime\in\Omega_u^+\backslash v}}
\fr{\mu_v c_v(1-B_v)}{\Lambda_v}\right. \times{}\\
{}\times \sum\limits_{k_v=0}^{N-1}\sum\limits_{\overline{k}_v\in A_{N-1-k_v}} z_v\left(0,\rho_v,k_v+1,k_v^\prime , 
k_v^{\prime\prime}\right )\times{}\\
{}\times \left.\prod\limits_{v^\prime\in\Omega_u^+\backslash v} z_v^\prime 
\left(0,\rho_v,k_v,k_v^\prime , k_v^{\prime\prime}\right) \right)
\Bigg /{}\\
\Bigg / \left ( 
\vphantom{\prod\limits_{v^\prime\in\Omega_u^+\backslash v}}
\sum\limits_{k_v=0}^{N-1} \sum\limits_{\overline{k}_v\in A_{N-1-k_v}} z_v 
\left (0,\rho_v,k_v+1,k_v^\prime , 
k_v^{\prime\prime}\right )\right. \times{}\\
{}\times \prod\limits_{v^\prime\in\Omega_u^+\backslash v} 
z_{v^\prime}\left(0,\rho_v,k_v,k^\prime , k_v^{\prime\prime}\right)+{}\\
{}+
\sum\limits_{\overline{k}_v\in A_N} z_v\left (0,\rho_v, 0,k_v^\prime , 
k_v^{\prime\prime}\right)\times{}\\
\left.{}\times \prod\limits_{v^\prime\in\Omega_u^+\backslash v}z_{v^\prime} 
\left(0,\rho_v,k_v,k_v^\prime , k_v^{\prime\prime}\right )\right )\,.
\end{multline*}
Как следует из правой части последнего неравенства, если 
$\mu_v c_v (1-B_v)/\Lambda_v \leq 1$, то она меньше~1. Поэтому, чтобы 
выполнилось условие~(\ref{e13aga}), необходимо выполнение первого 
неравенства в условии следствия для каждого $v\in\Omega_u^+$. Точно так же 
доказывается необходимость выполнения второго и третьего неравенств в 
условии следствия.

    Пусть $y[n]$, $n\geq 0$, последовательность, полученная по формуле 
$y[n+1]=q_N(\overline{\rho},y[n])$, $y[0]=1$.

\medskip

\noindent
\textbf{Утверждение 6.} \textit{Пусть $y^*\in (0,\,1)$~--- решение 
уравнения}~(8). \textit{Тогда последовательность $y[n]$, $n\geq 0$, сходится 
к решению~$y^*$}.

\medskip

\noindent
Д\,о\,к\,а\,з\,а\,т\,е\,л\,ь\,с\,т\,в\,о\,.\ Отметим, что $y[1]<y[0]$ (это следует из 
утверждения~2, так как $y[0]=1$). Пусть для некоторого $n>1$ выполняется 
условие $y[n]<y[n-1]$. Тогда, как следует из утверждения~2, указанное условие 
выполняется и для $n+1$, т.\,е.\ по индукции следует, что последовательность 
$y[n]$, $n\geq 0$, монотонно убывает. 

    Пусть для некоторого $n>0$ $y[n]>y^*$ (существование такого $n$ 
следует из равенства $y[0]=1$). Тогда, как следует из утверждения~2, 
$y[n+1]\;=$\linebreak $=\;q_N(\overline{\rho},y[n])>q_N(\overline{\rho},y^*) =y^*$, т.\,е.\ 
последовательность ограничена снизу величиной~$y^*$. Значит, существует 
$\underset{n\rightarrow \infty}{\mathrm{lim}}\,y[n]=y^0\geq y^*$. Так как 
$q_n(\overline{\rho},y)$~--- непрерывная по~$y$ функция, то можно написать 
$\underset{n\rightarrow 
\infty}{\mathrm{lim}}\,q_N(\overline{\rho},y[n])=q_N(\overline{\rho},y^0)=y^0$, 
т.\,е.\ $y^0$~--- решение уравнения~(\ref{e4aga}). Из единственности 
положительного решения уравнения~(\ref{e4aga}) получаем $y^0=y^*$.

    Пусть в узле используется схема полного разделения буферной памяти. 
Тогда для интенсив\-ностей~$\Lambda_v^*$, $v\in\Omega_u^+$, справедливы 
соотношения:
$$
\Lambda_v^* = \fr{\Lambda_v}{1-\pi_v}\,,
$$
где $v\in\Omega_u^+$.


Фиксируем произвольную линию сети~$v$. Пусть $\overline{k}_v = (k_v, 
k_v^\prime, k_v^{\prime\prime})$~--- состояние буферной памяти линии~$v$; 
$k_v$, $k_v^\prime$, $k_v^{\prime\prime}$ определены выше. Тогда с 
учетом введенных ранее предположений и обозначений для вероятности 
блокировки линии справедлива формула~\cite{4aga}:
\begin{equation}
\pi_v = \fr{1}{G_{N_v}}\sum\limits_{k_v=N_v} 
z_v(\pi_v,\rho_v,\overline{k}_v)\,,
\label{e15aga}
\end{equation}
где 
\begin{multline*}
z_v(\pi_v, \rho_v, \overline{k}_v)={}\\
{}=
\begin{cases}
\fr{\rho_v^{\prime * k_v^\prime}}{k_v^\prime !}\,
 \fr{\rho_v^{\prime\prime * k_v^{\prime\prime}}}{k_v^{\prime\prime}!}\,
 \fr{\rho_v^{*k_v}}{k_v !} & \mbox{при}\ k_v<c_v\,,\\
 \fr{\rho_v^{\prime *k_v^\prime}}{k_v^{\prime }! }
 \fr{\rho_v^{\prime\prime * k_v^{\prime\prime}}}{k_v^{\prime\prime}!}
\fr{\rho_v^{*k_v}}{c_v !c_v^{k_v-c_v}} & \mbox{при}\ k_v\geq c_v\,;
\end{cases}
\end{multline*}
\begin{align*}
G_{N_v} &= \sum\limits_{m=0}^{N_v} z_v (\pi_v ,\rho_v , \overline{k}_v)\,;\\ 
\rho_v^*&=\fr{\rho_v}{1-\pi_v}\,;
\end{align*}
$\rho_v$, $\rho_v^{\prime *}$, 
$\rho_v^{\prime\prime *}$, $v\in\Omega_u^+$ определены выше.
    
Пусть $y_v=1-\pi_v$, а $q_{N_v} (\rho_v, y_v)$~--- выражение в правой 
части~(\ref{e15aga}). Тогда из равенств~(\ref{e15aga}), взяв~$y_v$ в качестве 
неизвестной переменной, получим систему независимых уравнений:
\begin{equation}
y_v = q_{N_v}(\rho_v, y_v)\,, \quad v\in \Omega_u^+\,.
\label{e16aga}
\end{equation}
    
    Заметим, что для фиксированной $v$ и заданных параметров $\Lambda_v$, 
$\mu_v$, $\tau_v$, $t_v$, $N_v$, $v\in\Omega_u^+$, уравнение в~(\ref{e16aga}) 
является частным случаем уравнения~(\ref{e4aga}) и совпадает с последним, 
когда число исходящих линий из узла равно~1. Следовательно, для схемы 
полного разделения памяти также справедливы все приведенные выше 
утверждения~1--6 и следствие. Заметим, что неравенство~(\ref{e13aga}) в 
условии утверждения~5 при $B_v=0$ и $t_v=0$ преобразуется в неравенство 
$\Lambda_v / (\mu_v c_v) >1$, $v\in\Omega_u^+$. Последовательность 
$\overline{y}[n]$, $n\geq 0$, в утверждении~6 определяется по формуле:
    \begin{gather*}
    \overline{y}[n] =\{y_v[n],\ v\in\Omega_u^+\}\,,\
    y_v[n+1]=q_{N_v} (\rho_v,\,y_v[n])\,,\\
    y_v[0] =1\,,\quad n\geq 0\,,\quad v\in \Omega_u^+\,.
    \end{gather*}


\section{Алгоритм расчета} %4

    Для вычисления интенсивностей потоков и вероятностей блокировок в 
узле предлагается следующий алгоритм, описывающий изложенную выше 
итерационную процедуру. Введем обозначения:
$y_u^v$~--- вероятность блокировки узла для заявок, направляемых на 
линию~$v$,
\begin{gather*}
y_u^v  = 
\begin{cases}
y_u & \mbox{для}\ v\in\Omega_u^+\ \mbox{при}\\
&\mbox{полнодоступной схеме},\\
y_v & \mbox{при схеме полного распределения}\\
&\mbox{памяти};
\end{cases}
\\
q_N^v(\overline{\rho}_u^{-v}, y_u^v)  = 
\begin{cases}
q_N(\overline{\rho},y) & \mbox{для}\ v\in\Omega_u^+\ \mbox{при пол-}\\ 
&\mbox{нодоступной схеме},\\
q_{N_v}(\rho_v, y_v) & \mbox{при схеме полного}\\
&\mbox{распределения}\\ 
&\mbox{памяти},  v\in\Omega_u^+\,.
\end{cases}
\end{gather*}



Тогда уравнения~(\ref{e4aga}) и~(\ref{e16aga}) записываются в виде:
$$
y_u^v = q_N^v (\overline{\rho}^v_u, y^v_u)\,,\quad v\in \Omega_u^+\,.
$$
Для значений, вычисляемых на $k$-м шаге алгоритма, к 
обозначениям соответствующих параметров приписывается знак~$[k]$.
\pagebreak

\textbf{Шаг 0.} 
\begin{enumerate}[1.]
\item  \textit{Инициализация}. Вычисление начальных значений 
параметров~$\rho_v$, $v\in\Omega_u^+$: $\Lambda_v[0]=\Lambda_v$, 
$\rho_v[0]=\Lambda_v[0]/(\mu_v(1-B_v))$, $y_u^v[0]=1$.
\item \textit{Проверка условий существования решения}. Если для некоторой 
линии $v\in\Omega_u^+$ выполняется хотя бы одно неравенство $(c_v\mu_v(1-
B_v))/\Lambda_v[0]\;\leq$\linebreak $\leq\;1$, или $(1-B_v)/(\Lambda_v\tau_v B_v) \leq 1$, или 
$(t_v(1\;-$\linebreak $-\;B_v))/\Lambda_v[0] \leq 1$, то алгоритм заканчивает работу с 
результатом <<нагрузка не реализуема>>. Если в узле используется 
полнодоступная схема и $(c_v\mu_v(1-B_v))/\Lambda_v[0] > 1$, $(1-
B_v)/(\Lambda_v\tau_v B_v)\;>$\linebreak $>\;1$, $(t_v(1-B_v))/\Lambda_v[0] > 1$ для всех 
$v\in\Omega_u^+$, то проверяется условие~(\ref{e13aga}) для $\Lambda_v =
\Lambda_v[0]$, $v\in\Omega_u^+$, и при невыполнении этого условия алгоритм 
заканчивает работу с результатом <<нагрузка не реализуема>>.
\end{enumerate}

    При вычислении левой части неравенства~(\ref{e13aga}) рекомендуется 
использовать метод свертки Базена (см.~\cite{12aga}), позволяющий 
производить рекуррентные вычисления (подробно этот метод описан также 
в~[1, 3--6]).



\medskip
\textbf{Шаг~$k$} ($k > 0$):
\begin{enumerate}[1.]
\item \textit{Вычисление вероятностей блокировок}. Используя значения 
параметров $\overline{\rho}_u^v[k-1]$, $y_u^v[k-1]$, $v\in\Omega_u^+$, 
вычисление с помощью формул~(1)--(7) значений 
вероятностей $y[k]=1- \pi [k]$~--- в случае полнодоступной памяти, или 
$y_v[k]=1- \pi_v[k]$, $v\in\Omega_u^+$, с помощью формул~(\ref{e15aga})~--- в 
случае полного разделения памяти. При вычислении этих значений 
рекомендуется использовать метод свертки Базена.
    \item \textit{Проверка условий останова алгоритма}. Если хотя бы для 
одной $v\in\Omega_u^+$ для заданного значения точности   выполняется 
условие
$$
\fr{\vert \Lambda_v^*[k]-\Lambda_v^*[k-1]\vert}{\Lambda_v^*[k]}> \varepsilon\,,
$$
то вычисление параметров $\overline{\rho}_u^v[k]$, $v\in\Omega_u^+$, и 
переход к шагу~$k$, положив $k$ равным $k+1$, иначе алгоритм завершает 
работу. 
\end{enumerate}

    По завершении алгоритма либо выявится, что нагрузка в системе не 
реализуема, либо будут вычислены интенсивности потоков, поступающих на 
линии узла, и стационарные вероятности блокировок для заявок каждого типа. 
    
\section{Примеры расчета}

    Для проверки точности вычисления результатов с помощью 
предложенного выше алгоритма и приемлемости введенных предположений 
были проведены вычислительные эксперименты с использованием 
аналитических и имитационных моделей. Во всех рассмотренных ниже 
примерах потоки внешних заявок считаются пуассоновскими. 
В~табл.~1 приведены значения вероятности блокировок вновь 
поступивших извне заявок, полученные на основании точной формулы, 
приведенной в~\cite{4aga} для СМО типа $M\vert M\vert 1\vert 0$ с повторными 
заявками при экспоненциальном распределении интервала времени между 
повторными попытками (первая строка таблицы), алгоритма из подраздела~5 
настоящей статьи (вторая строка) и имитационной модели при постоянном 
интервале времени между повторными попытками, равном~10 (третья строка). 
Расчет табл.~1 проведен для узла с одной исходящей одноканальной 
линией при интенсивности первичного потока $\Lambda =1$ и емкости 
накопителя $N_v=1$. Таблицы~2 и~3 вычислены с помощью 
алгоритма из подраздела~5 и имитационной модели соответственно при одной 
исходящей линии с числом каналов~10.


    В табл.~\ref{t4aga} и~\ref{t5aga} приведены значения вероятности 
блокировки узла с тремя исходящими линиями канальной емкости~10 каждая 
при $\mu_v =0{,}2$, $v\in\Omega_u^+$,  вычисленные с помощью алгоритма из 
подраздела~5 и имитационной модели с интервалом повторной попытки, 
равным~10, соответственно. В табл.~\ref{t4aga} и~\ref{t5aga} знак <<--->> в 
ячейках означает, что предложенная нагрузка $\Lambda_v$, $v\in\Omega_u^+$, 
не реализуема.



В табл.~\ref{t6aga} отражены вероятности блокировки такого же узла с 
накопителем $N = 35$ при экспоненциальном распределении интервала 
времени между повторными попытками со средним значением~$\tau$. 


Результаты вычислительного эксперимента показывают, что с  увеличением 
длины интервала между повторными попытками  вероятность блокировки 
увеличивается и приближается к значению,\linebreak
вычисленному с помощью 
алгоритма из подраздела~5 (см.\ табл.~\ref{t4aga} и~\ref{t6aga}), т.\,е.\ при 
пуассоновском внешнем потоке заявок предположение, что суммарный 
входной поток заявок  является пуассоновским, вполне приемлемо для 
предварительного анализа характеристик узла (например, при  $\tau c_v\mu_v > 
10$). Как показывают табл.~1--3, вероятность блокировки 
узла существенно зависит от\linebreak 

\vspace*{6pt}
\noindent
%\begin{table*}\small %tabl1
{\small
{{\tablename~1}\ \ \small{Вероятности блокировок при одной исходящей одноканальной линии}}
%\label{t1aga}}
\vspace*{-3pt}

\begin{center}
{\tabcolsep=7.3pt
\begin{tabular}{|c|c|c|c|c|c|}
\hline
&\multicolumn{5}{c|}{$\mu$}\\
\cline{2-6}
\multicolumn{1}{|c|}{\raisebox{4pt}[0pt][0pt]{№}}&1{,}1&1{,}2&2&3&4\\
\hline
1&0,9091&0,8333&0,5000&0,3333&0,2500\\
2&0,9091&0,8333&0,5000&0,3333&0,2500\\
3&0,8867&0,8452&0,4944&0,3167&0,2396\\
\hline
\end{tabular}}
\end{center}
%\vspace*{-6pt}
%\end{table*}
}
%\bigskip
%\medskip
\addtocounter{table}{1}
\pagebreak

\end{multicols}

\renewcommand{\figurename}{\protect\bf Таблица}
%\renewcommand{\tablename}{\protect\bf Рис.}
\begin{figure*}
{\small
\begin{minipage}[t]{76mm}
%\begin{table*}\small %tabl2
\begin{center}
\Caption{Вероятности блокировок при одной исходящей многоканальной линии ($\varepsilon 
=0{,}0001$)
\label{t2aga}}
\vspace*{2ex}

\tabcolsep=6.5pt
\begin{tabular}{|c|c|c|c|c|c|}
\hline
&\multicolumn{5}{c|}{$\mu$}\\
\cline{2-6}
\multicolumn{1}{|c|}{\raisebox{4pt}[0pt][0pt]{$N$}}&0{,}11&0{,}12&0{,}2&0{,}3&0{,}4\\
\hline
10&0,4845&0,2935&0,0204&0,0017&0,0002\\
15&0,1181&0,0545&0,0006&0,0000&0,0000\\
20&0,0489&0,0167&0,0000&0,0000&0,0000\\
\hline
\end{tabular}
\end{center}
%\end{table*}
\end{minipage}
\hfill
\begin{minipage}[t]{76mm}
%\begin{table*}\small %tabl3
\begin{center}
\Caption{Вероятности блокировок при одной исходящей линии
\label{t3aga}}
\vspace*{2ex}

\tabcolsep=6.5pt
\begin{tabular}{|c|c|c|c|c|c|}
\hline
&\multicolumn{5}{c|}{$\mu_v$}\\
\cline{2-6}
\multicolumn{1}{|c|}{\raisebox{4pt}[0pt][0pt]{$N$}}&0{,}11&0{,}12&0{,}2&0{,}3&0{,}4\\
\hline
10&0,5247&0,3238&0,0219&0,0019&0,0001\\
15&0,1726&0,0912&0,0004&0,0001&0,0000\\
20&0,1180&0,0371&0,0000&0,0000&0,0000\\
\hline
\end{tabular}
\end{center}
%\end{table*}
\end{minipage}
}
\vspace*{6pt}
\end{figure*}

\renewcommand{\figurename}{\protect\bf Рис.}
\renewcommand{\tablename}{\protect\bf Таблица}
\addtocounter{table}{2}

\begin{table}\small %tabl4
\begin{center}
\parbox{400pt}{\Caption{Вероятности блокировок при трех исходящих линиях, вычисленные алгоритмом из 
подраздела~5 ($\varepsilon =0{,}0001$)
\label{t4aga}}
}

\vspace*{2ex}

\tabcolsep=8pt
\begin{tabular}{|c|c|c|c|c|c|c|c|c|c|}
\hline
&\multicolumn{9}{c|}{$\Lambda_v$}\\
\cline{2-10}
\multicolumn{1}{|c|}{\raisebox{4pt}[0pt][0pt]{$N$}}&1&1{,}1&1{,}2&1{,}3&1{,}4&1{,}5&1{,}6&1{,}7&1{,}8\\
\hline
20&0,0677&0,1423&0,2975&0,7653&---&---&---&---&---\\
25&0,0065&0,0173&0,0394&0,0827&0.1690&0.3827&---&---&---\\
30&0,0005&0,0019&0,0059&0,0155&0.0361&0.0790&0.1792&0,7259&---\\
35&0,0000&0,0002&0,0008&0,0030&0,0089&0,0234&0,0574&0,1505&---\\
40&0,0000&0,0000&0,0001&0,0005&0,0022&0,0075&0,0220&0,0617&0,2449\\
\hline
\end{tabular}
\end{center}
%\end{table}
\vspace*{6pt}
%\begin{table}\small %tabl5
\begin{center}
\parbox{400pt}{\Caption{Вероятности блокировок при трех исходящих линиях, вычисленные с помощью 
имитационной модели
\label{t5aga}}
}

\vspace*{2ex}

\tabcolsep=8pt
\begin{tabular}{|c|c|c|c|c|c|c|c|c|c|}
\hline
&\multicolumn{9}{c|}{$\Lambda_v$}\\
\cline{2-10}
\multicolumn{1}{|c|}{\raisebox{4pt}[0pt][0pt]{$N$}}&1&1{,}1&1{,}2&1{,}3&1{,}4&1{,}5&1{,}6&1{,}7&1{,}8\\
\hline
20&0,0786&0,1695&0,3549&0,7056&---&---&---&---&---\\
25&0,0069&0,0190&0,0441&0,0998&0,2266&0,4583&---&---&---\\
30&0,0007&0,0024&0,0075&0,0184&0,0462&0,1025&0,2380&0,6931&---\\
35&0,0000&0,0003&0,0007&0,0040&0,0129&0,0307&0,0890&0,2981&---\\
40&0,0000&0,0000&0,0000&0,0011&0,0041&0,0095&0,0346&0,0790&0,3179\\
\hline
\end{tabular}
\end{center}
%\end{table}
\vspace*{6pt}
%\begin{table}\small %tabl6
\begin{center}
\parbox{356pt}{\Caption{Зависимость вероятности блокировки при трех исходящих линиях, вы\-чис\-лен\-ные с 
помощью имитационной модели со случайным интервалом между повторными попытками
\label{t6aga}}
}

\vspace*{2ex}

\tabcolsep=8pt
\begin{tabular}{|c|c|c|c|c|c|c|c|c|}
\hline
&\multicolumn{8}{c|}{$\Lambda_v$}\\
\cline{2-9}
\multicolumn{1}{|c|}{\raisebox{6pt}[0pt][0pt]{$\tau$}}&1&1{,}1&1{,}2&1{,}3&1{,}4&1{,}5&1{,}6&1{,}7\\
\hline
\hphantom{9}1&0.0001&0,0001&0,0017&0,0063&0,0210&0,0733&0,1996&0,4222\\
\hphantom{9}5&0.0000&0,0002&0,0016&0,0036&0,0446&0,0159&0,1360&0,3273\\
10&0.0000&0,0002&0,0011&0,0036&0,0101&0,0430&0,0818&0,2774\\
20&0.0000&0,0003&0,0007&0,0029&0,0089&0,0257&0,0863&0,2045\\
     \hline
\end{tabular}
\end{center}
\end{table}


\begin{multicols}{2}


\noindent
числа каналов в линии при равной суммарной 
производительности. Кроме того, как видно из табл.~\ref{t5aga} и~\ref{t6aga}, 
вероятность блокировки в большей степени зависит от среднего значения 
длины интервала между повторными попытками передачи, чем от закона 
распределения длины интервала. Таким образом, предложенный в работе 
алгоритм позволяет вы\-чис\-лить с достаточной точностью вероятность 
блокировки узла, интенсивности повторных передач и предельную величину 
реализуемой нагрузки. Отметим, что полученные в данной статье результаты 
могут быть использованы для расчета нагрузок в телекоммуникационной сети с 
повторами заявок в предыдущем узле или из источника. 


{\small\frenchspacing
{%\baselineskip=10.8pt
\addcontentsline{toc}{section}{Литература}
\begin{thebibliography}{99}    
\bibitem{1aga}
\Au{Kamoun~F., Kleinrock~L.}
Analysis of shared finite storage in a computer networks node environment under 
general traffic conditions~// IEEE Trans. on Commun., 1980. Vol.~28. No.\,7. 
P.~992--1003.

\bibitem{6aga} %2
\Au{Агаларов~Я.\,М., Шоргин~С.\,Я.}
Рекуррентный метод вычисления параметров сетей связи~// Техника средств 
связи, 1986. Сер. <<Системы связи>>. Вып.~6. С.~42--46.

\bibitem{3aga}
\Au{Башарин Г.\,П., Бочаров~П.\,П., Коган~Я.\,А.}
Анализ очередей в вычислительных сетях.~--- М.: Наука, 1989. 

\bibitem{4aga}
\Au{Бочаров~П.\,П., Печинкин~А.\,В.}
Теория массового обслуживания.~--- М.: Изд-во РУДН, 1995. 

\bibitem{5aga}
\Au{Вишневский~В.\,М.} 
Теоретические основы проектирования компьютерных сетей.~--- М.: 
Техносфера, 2003. 

\bibitem{2aga} %6
\Au{Башарин Г.\,П.}
Лекции по математической теории телетрафика.~--- М.: Изд-во РУДН, 2007. 

\bibitem{7aga}
\Au{Таранцев~А.\,А.}
Инженерные методы теории массового обслуживания.~--- М.: Наука, 2007.

\bibitem{9aga} %8
\Au{D'Apice~C., De~Simone~T., Manzo~R., Rizelian~G.}
$M\vert G\vert 1\vert r$ retrial queueing system with priority service of primary 
customers and a customers-searching server~// Distributed Computer and 
Communication Networks. Stochastic Modelling and Optimization.~--- М.: 
Техносфера, 2003. P.~106--117.

\bibitem{8aga} %9
\Au{Klimenok~V.\,I., Kim~C.\,S.}
$BM\!AP$/$PH$/1 retrial system operating in random environment~// Proceedings of 
the 5th St.-Petersburg Workshop on Simulation, St.-Petersburg, June~26\,--\,July~2, 
2005.~--- St.-Petersburg: NII Chemistry St.-Petersburg University Publs., 
2005. P.~367--372.   

\bibitem{10aga}
\Au{Krishnamoorthy~A., Babu~S.}
$M\!AP\vert (PH,PH)/c$ retrial queue with selegeneration of priorities 
and non-preemptive service~// Proceedings of the 14th International Conference on 
Analytical and Stochastic Modeling Techniques and Applications, June~4--6, 
2007. Prague, Czech Republic.~--- Sbr.-Dudweiler: Digitaldruck Pirrot GmbH, 
2007. P.~70--74.

\bibitem{11aga}
\Au{Корн~Г., Корн~Т.}
Справочник по математике.~--- М.: Наука, 1974.

\label{end\stat}


\bibitem{12aga}
\Au{Buzen~J.\,P.}
Computational algorithm for closed queuing networks with exponential servers~// 
Communications ACM, 1973. Vol.~16. No.\,9. P.~527--531.
 \end{thebibliography}
}
}
\end{multicols}
 
 
   %6
\def\stat{ostrikova}

\def\tit{ОБ ОПТИМАЛЬНОМ РАСПОЛОЖЕНИИ АНТЕНН ДЛЯ~V2X-СОЕДИНЕНИЙ 
В~СУБТЕРАГЕРЦЕВОМ ДИАПАЗОНЕ$^*$}

\def\titkol{Об оптимальном расположении антенн для~V2X-соединений 
в~субтерагерцевом диапазоне}

\def\aut{Е.\,А.~Мачнев$^1$, В.\,А.~Бесчастный$^2$, Д.\,Ю.~Острикова$^3$, 
Ю.\,В.~Гайдамака$^4$, С.\,Я.~Шоргин$^5$}

\def\autkol{Е.\,А.~Мачнев, В.\,А.~Бесчастный, Д.\,Ю.~Острикова и~др.}
%$^3$,  Ю.\,В.~Гайдамака$^4$, С.\,Я.~Шоргин$^5$}

\titel{\tit}{\aut}{\autkol}{\titkol}

\index{Мачнев Е.\,А.}
\index{Бесчастный В.\,А.}
\index{Острикова Д.\,Ю.}
\index{Гайдамака Ю.\,В.}
\index{Шоргин С.\,Я.}
\index{Machnev E.\,A.}
\index{Beschastnyi V.\,A.}
\index{Ostrikova D.\,Yu.}
\index{Gaidamaka Yu.\,V.}
\index{Shorgin S.\,Ya.}


{\renewcommand{\thefootnote}{\fnsymbol{footnote}} \footnotetext[1]
{Исследование выполнено за счет гранта Российского научного фонда № 22-29-00694.}}


\renewcommand{\thefootnote}{\arabic{footnote}}
\footnotetext[1]{Российский университет дружбы народов, 1042200071@pfur.ru}
\footnotetext[2]{Российский университет дружбы народов, beschastnyy-va@rudn.ru}
\footnotetext[3]{Российский университет дружбы народов, ostrikova-dyu@rudn.ru}
\footnotetext[4]{Российский университет дружбы народов; Федеральный исследовательский центр <<Информатика и~управ\-ле\-ние>> 
Российской академии наук, \mbox{gaydamaka-yuv@rudn.ru}}
\footnotetext[5]{Федеральный исследовательский центр <<Информатика и~управление>> 
Российской академии наук, sshorgin@ipiran.ru}

\vspace*{-6pt}

  
  \Abst{Субтерагерцевая (суб-ТГц, 100--300~ГГц) связь должна обеспечить огромную  
ско\-рость передачи данных в~сис\-те\-мах~6G. Однако зона покрытия базовых станций 
(БС) ограничена, так как сигнал существенно затухает с~увеличением дистанции, а~также 
легко блокируется различными объектами, встречающимися на пути распространения 
сигнала. Таким образом, БС необходимо располагать достаточно часто, что делает такое 
решение дорогостоящим. Для снижения плотности развертывания БС можно использовать 
механизм ретрансляции сигнала с~помощью транспортных средств (ТС). Данный способ 
в~большой степени зависит от зоны размещения приемо-пе\-ре\-да\-ющей антенны на 
кузове ТС, что ставит вопрос о поиске расположения антенны, при котором механизм 
ретрансляции будет эффективным с~точки зрения скорости передачи данных и~расстояния 
между ТС-источником и~БС. В~данной работе на основе спецификации IEEE 802.15.3d 
и~экспериментальных данных о~распространении сигнала на частоте 300~ГГц 
предложена математическая модель для анализа многозвеньевой системы ретрансляции 
сигнала для трех зон размещения антенны на кузове ТС. Полученные результаты 
показывают, что расположение передатчика в~зоне лобового стекла характеризуется более 
низкой скоростью передачи данных, но при этом гораздо большим покрытием, чем 
расположение в~зонах бампера и~двигателя.}
  
  \KW{5G; <<новое радио>>; V2V; V2X; ретрансляция}
  
  \DOI{10.14357/19922264220407} 
  
%\vspace*{-3pt}


\vskip 10pt plus 9pt minus 6pt

\thispagestyle{headings}

\begin{multicols}{2}

\label{st\stat}

\section{Введение}



     Системы 5G New Radio (NR), \mbox{ра\-бо\-та\-ющие} в~мик\-ро\-вол\-но\-вом 
($\mu$Wave) и~миллиметровом (mmWave) диапазонах, уже вышли на рынок. 
В~то же время исследователи приступили к~изучению субтерагерцевых  
(100--300~ГГц) диапазонов в~контексте сис\-тем~6G~[1, 2]. Однако 
чрезвычайно высокое затухание сигнала, эффекты динамической 
блокировки~[3], а~также микромобильность~[4, 5] ограничивают дальность 
действия таких систем несколькими сотнями метров.
     
     Одна из серьезных проблем для сотовых операторов~--- обеспечение 
расширенного мобильного широкополосного доступа (Enhanced mobile 
broadband, eMBB) для пользователей в~движущихся ТС.\linebreak
 Эта услуга требует не только постоянного подключения, но и~высокой 
ско\-рости передачи данных. Одним из решений могут стать 
высокопроизводительные сис\-те\-мы~5G NR, работающие в~\mbox{диапазоне} 
миллиметровых волн, или 6G в~субтерагерцевом диапазоне. Однако высокое 
затухание в~субтерагерцевом диапазоне и~другие особенности 
распространения сигнала, такие как блокировка прямой видимости, 
ограничивают зону покрытия расположенных по краям дороги БС 
несколькими сотнями метров, что требует высокой плотности развертывания, 
а~значит, ставит вопрос о~рен\-та\-бель\-ности сис\-темы.

\begin{figure*}[b] %fig1
\vspace*{-4pt}
\begin{center}
   \mbox{%
\epsfxsize=163mm
\epsfbox{mac-1.eps}
}
\end{center}
\vspace*{-9pt}
\Caption{Схема моста из одного ТС-ре\-транс\-ля\-то\-ра с~учас\-ти\-ем  
ТС-от\-ра\-жа\-теля}
\end{figure*}

     
     Для обеспечения постоянной связи и~снижения финансовых затрат 
сетевых операторов можно использовать механизм ретрансляции 
сигнала~[6]. Чтобы обеспечить поддержку этого механизма, консорциум 
3GPP недавно стандартизировал технологию интегрированного доступа 
и~транспортной сети (Integrated Access and Backhaul, IAB)~[7]. В~рамках 
этой технологии ТС-ретрансляторы с~высокоскоростными передатчиками 
в~диапазоне субтерагерцевых частот могут организовать так называемые 
<<мосты>>, пересылая по цепочке данные от ТС, которые в~настоящее время 
не имеют прямого подключения к~БС. Эффективность механизма 
многозвеньевой ретрансляции сигнала оценивается на основании ско\-рости 
передачи данных по установленному мосту и~длины моста, т.\,е.\ расстояния 
между ТС-источником сигнала и~БС. При этом расположение  
при\-емо-пе\-ре\-да\-ющей антенны на кузове ТС влияет на оба указанных 
показателя. 
     
     Конечная цель данного исследования~--- выработать рекомендации по 
выбору зоны размещения антенны на ТС, при котором механизм 
ретрансляции обеспечит минимальную плотность развертывания БС для 
заданного набора параметров и~условий дорожного движения, включая 
плотность ТС на дороге и~ско\-рости их движения. Разработанная 
математическая модель основана на недавних исследованиях 
распространения сигнала в~субтерагерцевом диапазоне в~среде~V2V 
(Vehicle-to-Vehicle) и~использует реалистичные параметры связи из стандарта 
IEEE~802.15.3d~[8].
     
\section{Системная модель}

     Рассматривается участок автомагистрали, например городская улица 
или скоростное шоссе, покрытие которого беспроводной связью 
обеспечивается БС~6G, работающими в~субтерагерцевом 
диапазоне. Предполагается, что БС установлены по обеим сторонам дороги 
в~шахматном порядке, например на фонарных столбах, на постоянной 
высоте~$h_A$. Расстояние между БС по одной стороне дороги равно~$d$, 
таким образом, БС образуют равнобедренные треугольники 
с~основанием~$d$, т.\,е.\ БС на противоположных сторонах сдвинуты друг 
относительно друга на расстояние $d/2$. Эти БС служат точками доступа 
в~интернет для пользователей, находящихся в~ТС, например в~автомобилях 
(рис.~1).
     


     Предполагается, что дорога имеет четное чис\-ло~$N_l$~полос, при этом 
возможно движение ТС в~противоположных направлениях. Ширина полос 
постоянна и~равна~$w$. Ско\-рость ТС предполагается равной~$v$. 
Расположение ТС на дорожной полосе определяется пуассоновским 
процессом ин\-тен\-сив\-ности~$\lambda$, согласно которому расположены 
центры ТС. Далее предположим, что автомобили имеют 
одинаковую постоянную длину~$l_v$, а~дорожный просвет равен~$h_C$. 
Дорожное движение предполагается однородным на каждой полосе, т.\,е.\
     $v$, $\lambda$ и~$l_v$ не зависят от рассматриваемой полосы. 
     
     Согласно~[9], новые технологии V2V обеспечат\linebreak эффективный 
и~безопасный контроль над мобильностью ТС при минимально допустимом 
расстоянии между любыми двумя транспортными средствами $d_s\hm= t_s 
v$, где $t_s\hm=0{,}5$~с~--- минимальное\linebreak время, необходимое сис\-те\-мам 
автоматического управ\-ле\-ния для оценки условий движения в~режиме 
реального времени и~принятия превентивных мер.
     
     Для снижения капитальных затрат за счет увеличения расстояния 
между БС предполагается, что доля ТС~$P_E$ оснащена устройствами 
ретрансляции сигнала. Параметр~$P_E$ называется <<сте\-пенью внед\-ре\-ния 
технологии>>. Транспортные средства обору\-до\-ва\-ны двумя при\-емо-пе\-ре\-да\-ющи\-ми антеннами:
одна в~передней и~одна в~задней части ТС. Возможность подключения 
приемопередатчиков к~высокоскоростной внут\-рен\-ней шине со ско\-ростью 
\mbox{передачи} данных, достаточной для обработки {ретранслируемого} трафика, 
под\-тверж\-да\-ет\-ся последними разработками~[10].
     
     Транспортные средства, оснащенные средствами связи, нуждаются 
в~услугах eMBB, предостав\-ля\-емых через БС. Если прямое соединение 
невозможно из-за того, что ТС находится вне зоны действия ближайшей БС, 
они используют ретрансляцию для формирования моста, состоящего из 
одной или нескольких точек ретрансляции, в~качестве которых выступают 
ТС-ре\-транс\-ля\-то\-ры. Длина этого моста и~скорость передачи данных по 
нему существенно зависят от плотности развертывания БС, характеристик 
окружающей среды, влияющих на распространение сигнала, включая 
плотность ТС на дороге и~сценарий дорожного движения, а~также зоны 
размещения антенны.
     
     В работе рассмотрены следующие потенциальные варианты 
расположения антенны: на уровне бампера (0,3--0,4~м), на уровне двигателя 
(0,4--1,0~м) и~на лобовом стекле (1,0--1,5~м). Определим параметр~$\sigma$ 
для обозначения зоны размещения антенны, т.\,е.\ $\sigma\hm\in \{B,E,W\}$ 
соответственно.
     
     Значение отношения уровня сигнала к~уровню шума (Signal to 
Interference plus Noise Ratio, SINR) на ТС на расстоянии~$x$ от БС 
записывается в~виде
     \begin{equation*}
     S(x)= P_T G_A G_U \fr{x^{-\gamma}}{(N_0+I)L_A (f,x) 
L_B}\,,
    % \label{e1-ost}
     \end{equation*}
где $P_T$~--- излучаемая мощность; $G_A$ и~$G_U$~--- коэффициенты 
усиления на стороне приема и~передачи; $\gamma$~--- коэффициент потерь 
на пути сигнала; $N_0$~--- тепловой шум; $I$~--- помехи; $L_A(f,x)$~--- 
коэффициент затухания сигнала; $L_B$~--- потери, вызванные блокировкой 
или отражениями от других объектов.

\section{Математическая модель установления соединения 
для~механизма многозвеньевой ретрансляции сигнала}

     Построение модели начнем с~анализа прямых подключений между  
ТС-ис\-точ\-ни\-ком сигнала и~БС, а именно: с~определения максимальной 
дистанции прямого подключения и~скорости передачи данных. Чтобы 
вычислить данные параметры, используем набор схем модуляции 
и~кодирования, указанных в~стандарте IEEE 802.15.3d~[8].
     
     Следуя~[11], максимальное расстояние подключения  
ав\-то\-мо\-би\-ля-ис\-точ\-ни\-ка к~БС с~допустимой мощ\-ностью  
при\-ни\-ма\-емо\-го сигнала~$S$ можно записать как
     \begin{equation}
     d_\xi(S)= \left( \fr{P_T \sqrt[10]{10^{G_A+G_U}}} 
{\sqrt[10]{10^{N_0+S}}\,10^{2\log_{10} f_c-14{,}86+L_B}} 
\right)^{1/\gamma}\,,
     \label{e2-ost}
     \end{equation}
где $P_T$~--- мощность передачи; $G_A$ и~$G_U$~--- коэффициенты 
усиления на приеме и~передаче; $f_c$~--- 
несущая частота.

     Если следующий автомобиль не поддерживает ретрансляцию, 
расположение антенны на уровне бампера в~дополнение к~прямой видимости 
($L$) может использовать пути отражения под ТС ($U$) и~от соседнего 
автомобиля ($R$). Если антенна расположена на уровне двигателя или 
лобовом стекле, отражение сигнала под автомобилем недоступно ввиду 
геометрических ограничений, при этом отражение от соседнего автомобиля 
доступно. Для расположения на стекле существует также возможность 
прохождения сигнала сквозь стекло следующего автомобиля ($W$), но 
затухание сигнала при этом будет в~несколько раз выше затухания при 
прямой видимости. Для обозначения типов распространения сигнала 
используем параметр $\xi \hm\in \{L,U,R,W\}$.
     
     Следует обратить внимание, что минимальная требуемая мощность 
сигнала приема $S\hm= S_{\min}$ позволяет получить максимальное 
расстояние для связи между соседними автомобилями, далее обозначаемое 
как $d_\xi^{\max}$, $\xi\hm\in \{L,U,R,W\}$.
     
     Функции плотности вероятности расстояния до $i$-го соседнего 
ТС~[12] в~случае распределения цент\-ров ТС согласно процессу Пуассона 
подчиняются распределению Эрланга~[13]:
     \begin{equation*}
     f_i(x)=\fr{2(\pi\lambda)^i}{(i-1)!}\,x^{2i-1} e^{-\pi \lambda x^2} ,\enskip 
x>0\,,\enskip i=1,\ldots , N.
     %\label{e3-ost}
     \end{equation*}
     
     \textbf{Подключение в~зоне прямой видимости.} Связь в~зоне прямой 
видимости на расстоянии~$r$ возможна для всех рассматриваемых зон 
размещения антенны, если расстояние между вза\-и\-мо\-дей\-ст\-ву\-ющи\-ми ТС 
меньше, чем максимальное расстояние связи в~условиях прямой видимости 
$d^{\max}$ из~(\ref{e2-ost}), и~сле\-ду\-ющее ТС оборудовано при\-е\-мо-пе\-ре\-да\-ющей антенной. Таким образом, вероятность успешного подключения в~условиях прямой видимости на расстоянии~$r$ имеет вид:
     \begin{equation*}
     f_{H,\mathrm{LoS}}(r) =I\left( r<d^{\max}\right) P_E f_0(r)\,,
     \label{e4-ost}
     \end{equation*}
где $f_0(r)$ вычисляется как
\begin{equation*}
f_0(r)= 0{,}5 f_1(r) +0{,}5 f_2(r)\,.
%\label{e5-ost}
\end{equation*}
     
     \textbf{Отражение от соседнего автомобиля.} На\-пом\-ним, что боковое 
отражение доступно для всех рас\-смат\-ри\-ва\-емых зон размещения антенны. 
В~случае блокировки прямой видимости, т.\,е.\ когда следующее ТС не 
оборудовано приемо-передающей антенной, предполагаем, что 
расстояние~$d_0$ до блокирующего ТС (см.\ рис.~1) равновероятно 
подчиняется распределению расстояния либо до первого соседа, либо до 
второго. Соответственно, плот\-ность ве\-ро\-ят\-ности расстояния с~обходом 
блокирующего автомобиля определяется как свертка расстояний до двух 
ближайших соседей блокирующего ТС, т.\,е.
     \begin{equation*}
     f_B(r)=\left(f_1*f_2\right) (r)=\int\limits_0^\infty f_1(s) f_2(r-s)\,ds\,.
%     \label{e6-ost}
     \end{equation*}
     
     \begin{figure*}[b] %fig2
\vspace*{-6pt}
\begin{center}
   \mbox{%
\epsfxsize=118.5mm
\epsfbox{mac-2.eps}
}
\end{center}
\vspace*{-9pt}
\Caption{Схема отражения под автомобилем}
\end{figure*}
     
     В модели предполагается, что возможен обход не более одного 
блокирующего ТС через одно боковое отражение (см.\ рис.~1). При этом  
ав\-то\-мо\-биль-<<от\-ра\-жа\-тель>> должен находиться на таком участке 
соседней полосы, где сигнал от <<передатчика>> не будет заблокирован  
ав\-то\-мо\-би\-лем-<<бло\-ки\-ра\-то\-ром>>, находящимся слишком близ\-ко 
либо к~<<передатчику>>, либо к~<<ретранслятору>>. Вероятность 
незаблокированного отражения может быть выражена~как
     \begin{equation*}
     \delta_R(r)=\fr{w_v}{\eta(r)}= \fr{w_v r}{2(w-w_v)}\,,
    % \label{e7-ost}
     \end{equation*}
где $\eta(r)\hm= 2(w\hm- w_v)/r$~--- тангенс угла отклонения луча~$\beta$; 
$w_v$~--- половина ширины ТС.

     Вероятность ${\sf P}_R(r)$ наличия на соседней полосе ТС, 
обеспечивающего незаблокированное отражение сигнала, можно получить, 
используя свойство независимости процесса Пуассона. Предполагая, что 
точка центра отражателя на соседней полосе равномерно распределена на 
отрезке, соответствующем расстоянию между передатчиком 
и~ретранслятором~[14], искомую вероятность можно выразить в~следующем 
виде:
     \begin{equation}
     {\sf P}_R(r)=\begin{cases}
     \left( \fr{N_l}{2}-1\right) 
\fr{\delta_R(r)\lambda}{[l_v+2\Delta_\alpha(r)]^{-1}}\,, & \\
& \hspace*{-20mm}2d_s<r<d_R^{\max}\,;\\
     0 & \hspace*{-20mm}\mbox{в\ других\ случаях},
     \end{cases}
     \label{e8-ost}
     \end{equation}
где $\Delta_\alpha$~--- допустимое смещение соседнего~ТС.
     
     \textbf{Отражение под автомобилем.} В~отличие от зон лобового 
стекла и~двигателя, при размещении антенны в~зоне бампера становится 
возможной передача сигнала под ТС за счет отражения от дорожного 
полотна. Из-за свойства симметричности отраженного пути (рис.~2) 
минимальное расстояние между бампером блокиратора и~бампером 
с~антенной с~обеих сторон одинаково и~определяется выражением
     \begin{equation*}
     \delta_U(r)= \fr{r}{2}\left( 1-\fr{h_C}{h_B}\right)-\Delta_\alpha\,,
    % \label{e9-ost}
     \end{equation*}
где $h_C$~--- дорожный просвет; $h_B$~--- высота размещения антенны; 
$\Delta_\alpha$~--- смещение, допускаемое для бло\-ки\-ру\-юще\-го~ТС.



     Вероятность передачи сигнала под ТС можно найти 
аналогично~(\ref{e8-ost}):

\noindent
     \begin{multline}
     {\sf P}_{B,B}(r)={}\\
     {}= \begin{cases}
     \delta_U(r) \left( \fr{h_C}{h_B}+\fr{\Delta_\alpha}{r}\right), & 2d_s< r< 
d_U^{\max}\,;\\
     0 & \mbox{в\ других\ случаях}.
     \end{cases}
     \label{e10-ost}
     \end{multline} 
     
     \textbf{Прохождение сигнала сквозь стекло.} Наконец, найдем 
вероятность установления соединения через стекла бло\-ки\-ру\-юще\-го~ТС:
     \begin{equation*}
     {\sf P}_{B,W}(r)= \begin{cases}
     1, & 2d_s<r<d_W^{\max}\,;\\
     0 & \mbox{в\ других\ случаях}\,.
     \end{cases}
    % \label{e11-ost}
     \end{equation*}
     
     \textbf{Вероятность успешного многозвеньевого подключения.} 
С~учетом вышеизложенных возможностей установления соединения для 
механизма многозвеньевой ретрансляции сигнала вероятность успешного 
подключения на расстоянии~$r$, $f_{H,\sigma}(r)$, $\sigma\hm\in \{B,E,W\}$, 
определяется как сумма вероятностей двух событий: либо следующее ТС 
оснащено антенной и~можно использовать подключение в~условиях прямой 
видимости, либо следующее ТС на полосе не имеет передающего устройства 
с~ве\-ро\-ят\-ностью $1\hm- {\sf P}_E$ и~единственный вариант~--- обойти его 
с~по\-мощью описанных выше способов. Таким образом, 
расширяя~(\ref{e10-ost}), приходим к~сле\-ду\-ющей формуле для плот\-ности 
вероятности успешного подключения на расстоянии~$r$:
     \begin{multline}
     f_{H,\sigma}(r) = I\left( r< d^{\max}\right) {\sf P}_E f_0(r)+
     \left( {\sf P}_E-{\sf P}_E^2\right)\times{}\\
     {}\times \left( {\sf P}_R(r)+{\sf P}_{B,\sigma}(r) - {\sf P}_R(r) 
{\sf P}_{B,\sigma}(r)\right) f_B(r)\,.
     \label{e12-ost}
     \end{multline}
     
       \begin{table*}[b]\small
       \vspace*{-12pt}
  \begin{center}
  \Caption{Параметры дорожного движения}
  \vspace*{2ex}
  
  \begin{tabular}{|l|c|c|}
  \hline
\multicolumn{1}{|c|}{Сценарий}&\tabcolsep=0pt\begin{tabular}{c}Скорость\\ автомобиля $v$, км/ч \end{tabular}& 
\tabcolsep=0pt\begin{tabular}{c}Расстояние между\\ автомобилями $d_0$, м\end{tabular}\\
\hline
 Пробка& 20 &10\\
Нормальный городской трафик& 60&30\\
Шоссе& 120\hphantom{9} & 60\\
\hline
\end{tabular}
\end{center}
%\end{table*}
%\begin{table*}\small
%\vspace*{4pt}
  \begin{center}
  \Caption{Входные параметры системы}
  \vspace*{2ex}
  
  \begin{tabular}{|c|c|l|}
  \hline
Обозначение&Значение&\multicolumn{1}{c|}{Описание}\\
\hline
$l_v$&4,5~м&Длина автомобиля\\
$\lambda$&0,02~авт/м&Средняя плотность автомобилей\\
$h_A$&3~м&Высота БС\\
$h_B$&0,4/0,7/1,2~м&Высота антенны\\
$h_C$&0,2~м&Дорожный просвет автомобиля\\
$v$&25~м/с&Скорость автомобиля\\
$f_C$&304,2~ГГц&Несущая частота\\
$P_T$&$4{,}2\cdot 10^{-6}$~Вт &Мощность передатчика БС/автомобиля\\
$N_0$&$-$84~дБ&Мощность шума\\
 $S$&$-$56~дБ&Минимальный SINR\\
$G_A$, $G_U$&17,58~дБ&Коэффициенты усиления на приеме и~на передаче\\
$\gamma$&2,1&Коэффициент затухания сигнала\\
\hline
\end{tabular}
\end{center}
\end{table*}
     
     Определив вероятность многозвеньевого подключения  
в~(\ref{e12-ost}), можно перейти к~описанию показателей эффективности. 
В~част\-ности, вероятность того, что мост из~$n$ ТС-ре\-транс\-ля\-то\-ров 
обеспечит передачу данных на расстояние~$r$, может быть получена  
с~по\-мощью $n$-крат\-ной свертки~(\ref{e12-ost}):

\noindent
     \begin{equation*}
     p_{C,\sigma}(n,r) =\int\limits^\infty_{r-d^{\max}} f^{(n)}_{H,\sigma} 
(y)\,dy\,.
    % \label{e13-ost}
     \end{equation*}
     Если допустить возможность бесконечного чис\-ла 
 ТС-ре\-транс\-ля\-то\-ров в~соединении, вероятность того, что длина моста 
окажется не меньше расстояния~$r$ от ТС-ис\-точ\-ни\-ка сигнала до БС, 
может быть получена следующим образом:

\noindent
     \begin{multline*}
     p_{S,\sigma} (r)= \sum\limits^\infty_{n=1} p_{C,\sigma} (n,r) 
\prod\limits_{j=1}^{n-1} \left( 1-p_{C,\sigma}(j,r)\right)\,,\\
     \sigma\in \{B,E,W\}\,.
     %\label{e14-ost}
     \end{multline*}
     
     Скорость передачи по установленному мос\-ту~[14] определяется звеном 
с~наихудшими условиями канала. Тогда, имея пороговые значения~$s_i$, 
$i\hm=1,\ldots , N_C$, чувствительности приемника, соответствующие набору 
показателей качества канала $\{1,\ldots , N_C\}$, можно найти вероятность 
использования схемы кодирования~$i$ в~звене для каждого типа 
распространения сигнала $\xi\hm\in \{L,U,R,W\}$:
     \begin{equation*}
     q_{\xi,i} =\int\limits_{d_\xi(s_i)}^{d_\xi(s_{i+1})} f_{H,\xi}(r)\,dr\,,\enskip 
i=1,\ldots , N_C\,,
    % \label{e15-ost}
     \end{equation*}
где $d_\xi(S_{N_C+1})\hm=\infty$.

     Теперь определим среднюю скорость передачи данных по 
установленному мосту, включающему~$n$ ТС-ре\-транс\-ля\-то\-ров, 
используя биномиальное распределение

\noindent
     \begin{equation*}
     \rho_{\xi,n}=\sum\limits_{i=1}^{N_C} \omega_i \left( 
\sum\limits_{k=i}^{N_C} q_{\xi,k}\right)^n\,,\enskip \xi\in \{L,U,R,W\}\,,
     %\label{e16-ost}
     \end{equation*}
где $\omega_i$~--- спектральная эффективность канала согласно схеме~$i$.

     Тем не менее в~целях контроля качества соединения имеет смысл 
ограничивать максимальное число ТС-ре\-транс\-ля\-то\-ров некоторым 
заданным значением~$N$, $N\hm\geq 1$:
     \begin{multline*}
     \rho_{S,\sigma}(r)={}\\
     {}= \!\!\!\sum\limits_{\xi\in \{L,R,U,W\}}\!\!\!\!\!\!\! {\sf P}_\xi 
\sum\limits_{n=1}^N \rho_{\xi,n} p_{C,\sigma} (n,r) \prod\limits^n_{j=1} \left( 
1-p_{C,\sigma} (j,r)\right)\,,\\
     \sigma\in\{B,E,W\}\,.
   %  \label{e17-ost}
     \end{multline*}
     
     \vspace*{-18pt}

\section{Численный анализ}

\vspace*{-3pt}

     В качестве исходных данных для численного эксперимента 
рассматриваются три сценария дорожного движения: проб\-ка, нормальные 
условия движения в~городе и~скоростное шоссе. Сценарии различаются  
ско\-ростью ТС и~средним расстоянием между ними, как показано в~табл.~1. 
Остальные па\-ра\-мет\-ры сис\-те\-мы приведены в~табл.~2.

\begin{figure*} %fig3
\vspace*{1pt}
\begin{center}
   \mbox{%
\epsfxsize=84.218mm
\epsfbox{mac-3.eps}
}
\end{center}
\vspace*{-11pt}
\Caption{Средняя длина моста (залитые значки) и~скорость передачи (пустые значки) в~зависимости от 
плотности ТС: \textit{1}~--- бампер; \textit{2}~--- стекло; \textit{3}~--- 
двигатель}
\vspace*{-5pt}
\end{figure*}


\begin{figure*}[b] %fig4
  \vspace*{-6pt}
    \begin{minipage}[t]{80mm}
\begin{center}
   \mbox{%
\epsfxsize=78.898mm
\epsfbox{mac-4-a.eps}
}
\end{center}
\vspace*{-9pt}
  \Caption{Вероятность подключения в~зависимости от расстояния между БС и~степени 
внедрения технологии (черные кривые~--- $P_E\hm= 0{,}7$; серые кривые~--- $P_E\hm= 0{,}9$): (\textit{1}~--- бампер; 
  \textit{2}~--- стекло; \textit{3}~--- двигатель}
  \end{minipage}
  %\end{figure*}
   \hfill  
%  \begin{figure*} %fig5
  \vspace*{-6pt}
  \begin{minipage}[t]{80mm}
\begin{center}
   \mbox{%
\epsfxsize=79mm
\epsfbox{mac-4-b.eps}
}
\end{center}
\vspace*{-9pt}
  \Caption{Вероятность подключения в~зависимости от расстояния между БС и~сценариев дорожного движения
  (пунктирные кривые~--- пробка; штриховые~--- нормальный городской трафик; сплошные кривые~--- шоссе): 
  \textit{1}~--- бампер; \textit{2}~--- стекло; \textit{3}~--- двигатель}
    \end{minipage}
  \end{figure*}
  
  

     Начнем с~исследования основных зависимостей между средней длиной 
мос\-та и~ско\-ростью передачи данных, показанных на рис.~3, для различных 
зон размещения антенны в~за\-ви\-си\-мости от плот\-ности ТС
на полосе дорожного движения, где максимальное чис\-ло  
ТС-ре\-транс\-ля\-то\-ров~$N$установлено рав\-ным~10, а~степень внед\-ре\-ния 
технологии $P_E\hm= 0{,}7$.
     


     По результатам эксперимента мож\-но отметить, что средняя длина 
моста уменьшается по мере увеличения плот\-ности ТС на дороге для всех 
рас\-смот\-рен\-ных вариантов расположения антенны. Это объясняется тем, что 
с~увеличением плот\-ности уменьшается среднее расстояние между ТС. 
Обратный эффект наблюдается для ско\-рости передачи данных, и~это связано 
с~более короткими расстояниями меж\-ду автомобилями, а~следовательно, 
лучшим качеством канала.
     
     Анализируя влияние зоны размещения антенны, можно сделать вывод, 
что расположение у~лобового стекла обеспечивает 
б$\acute{\mbox{о}}$льшую сред\-нюю длину мос\-та почти для всех 
рассмотренных плотностей ТС. Однако этот выигрыш достигается за счет 
гораздо меньшей ско\-рости передачи данных. Обоснование данного 
наблюдения заключается в~том, что экраны в~виде заднего и~переднего стекол 
блокирующего автомобиля создают высокие потери при передаче сигнала.
     
     Одной из характеристик, отвечающих за гарантии производительности 
для пользователей коммерческих систем, служит до\-ступ\-ность подключения 
к~БС посредством моста. Рас\-смот\-рим \mbox{до\-ступ\-ность} подключения как 
функцию от рас\-сто\-яния между БС (Inter-site distance, ISD), показанную на 
рис.~4, для различных значений степени внед\-ре\-ния технологии 
и~нормальных условий городского трафика ($\lambda\hm= 1/30$). При 
достижении 1250~м наблюдается резкий спад, ха\-рак\-те\-ри\-зу\-ющий расстояние, 
на котором начинает работать механизм ре\-транс\-ля\-ции сигнала. Из всех 
рас\-смот\-рен\-ных зон размещения антенны лобовое стек\-ло показывает 
наилучшую до\-ступ\-ность~0,95 при степени внед\-ре\-ния технологии~0,9.
     
     Влияние условий дорожного движения на ве\-ро\-ят\-ность подключения 
для различных зон размещения антенны показано на рис.~5. Здесь 
можно заметить, что наихудший возможный сценарий~--- пробки, где не 
только все рас\-смот\-рен\-ные зоны приводят к~одному и~тому же значению 
параметра ISD, но и~связанное с~этим улучшение ISD незначительно. 
Причина в~том, что ТС расположены очень плот\-но, блокируя сразу несколько 
вариантов отражений, в~том чис\-ле под ТС. Тем не менее для нормального 
и~шоссейного сценариев наилучшая зона размещения антенны с~точ\-ки 
зрения длины установленного мос\-та~--- на лобовом стекле.

  
      

  
\section{Заключение}

     В работе предложена математическая модель для оценки 
производительности механизма мно\-го\-звень\-евой ретрансляции сигнала для 
V2X-со\-еди\-не\-ний в~частотных диапазонах субтерагерцевых волн при 
различных условиях распространения сигнала. Рассматриваемые метрики 
включают среднее расстояние подключения к~БС и~скорость передачи 
данных с~учетом возможной многозвеньевой ретрансляции сигнала, а~также 
критический параметр качества обслуживания~--- доступность сети.
     
     Представленные численные результаты для типичных сценариев 
предоставления услуг связи по технологии 6G с~использованием механизма 
многозвеньевой ретрансляции сигнала показывают, что можно 
рекомендовать размещение антенны в~зоне лобового стекла, которое, 
несмотря на более низкую скорость передачи данных, значительно менее 
чувствительно к~степени внедрения технологии и,~как правило, 
характеризуется гораздо б$\acute{\mbox{о}}$льшим покрытием сети.
     
{\small\frenchspacing
 {%\baselineskip=10.8pt
 %\addcontentsline{toc}{section}{References}
 \begin{thebibliography}{99}
\bibitem{1-ost}
\Au{Moltchanov D., Gaidamaka~Y., Ostrikova~D., Beschastnyi~V., Koucheryavy~Y., Samouylov~K.} 
Ergodic outage and capacity of terahertz systems under micromobility and blockage impairments~// 
IEEE T. Wirel. Commun., 2021. Vol.~21. Iss.~5. P.~3024--3039. doi: 
10.1109/ TWC.2021.3117583.
\bibitem{2-ost}
\Au{Moltchanov D., Beschastnyi~V., Ostrikova~D., Gaidamaka~Y., Koucheryavy~Y.} 
Uninterrupted connectivity time in THz systems under user micromobility and blockage~//  
Global Communications Conference Proceedings.~--- Piscataway, NJ, USA: 
IEEE, 2021. Art. 9685384. 6~p. doi: 10.1109/GLOBECOM46510.2021.9685384.
\bibitem{3-ost}
\Au{Gapeyenko M., Samuylov~A., Gerasimenko~M., Moltchanov~D., Singh~S., Akdeniz~M.\,R., 
Aryafar~E., Himayat~N., Andreev~S., Koucheryavy~Y.} On the temporal effects of mobile 
blockers in urban millimeter-wave cellular scenarios~// IEEE T. Veh. Technol., 
2017. Vol.~66. Iss.~11. P.~10124--10138. doi: 10.1109/TVT.2017.2754543.
\bibitem{4-ost}
\Au{Stepanov N.\,V., Moltchanov~D., Begishev~V., Turlikov~A., Koucheryavy~Y.} Statistical 
analysis and modeling of user micromobility for THz cellular communications~// IEEE T. 
Veh. Technol., 2021. Vol.~71. Iss.~1. P.~725--738. doi: 10.1109/TVT.2021.3124870.
\bibitem{5-ost}
\Au{Beschastnyi V., Ostrikova~D., Moltchanov~D., Gaidamaka~Y., Koucheryavy~Y., 
Samouylov~K.} Balancing latency and energy efficiency in mmWave 5G NR systems with 
multiconnectivity~// IEEE Commun. Lett., 2022. Vol.~26. Iss.~8. P.~1952--1956. doi: 
10.1109/LCOMM. 2022.3175663.
\bibitem{6-ost}
\Au{Petrov V., Moltchanov~D., Andreev~S., Heath~R.\,W.} Analysis of intelligent vehicular 
relaying in urban 5G\;+\;millimeter-wave cellular deployments~// Global Communications 
Conference Proceedings.~--- Piscataway, NJ, USA: IEEE, 2019. Art.~05946. 
6~p. doi: 10.48550/arXiv.1908.05946.
\bibitem{7-ost}
Study on integrated access and backhaul (Release 17): Technical Specification 38.874 
V17.0.0. 3GPP, 2020. {\sf  
https://www.3gpp.org/ftp/Specs/archive/38\_series/38.\linebreak 874/38874-g00.zip}.
\bibitem{8-ost}
\Au{Petrov V., Kurner~T., Hosako~I.} IEEE 802.15.3d: First standardization efforts for  
sub-terahertz band communications toward 6G~// IEEE Commun. Mag., 2020. 
Vol.~58. No.\,11. P.~28--33. doi: 10.1109/MCOM.001.2000273.
\bibitem{9-ost}
\Au{Ozpolat M., Bhargava~K., Kampert~E., Higgins~M.\,D.} Multi-lane urban mmWave V2V 
networks: A path loss behavior dependent coverage analysis~// Vehicular Communications, 
2021. Vol.~30. Art.~100348. 11~p. doi: 10.1016/ j.vehcom.2021.100348.
\bibitem{10-ost}
\Au{Wang J., Liu~J., Kato~N.} Networking and communications in autonomous driving: 
A~survey~// IEEE Commun. Surv. Tut., 2018. Vol.~21. No.\,2. P.~1243--1274. doi: 
10.1109/COMST.2018.2888904.
\bibitem{11-ost}
\Au{Eckhardt J.\,M., Petrov~V., Moltchanov~D., Koucheryavy~Y., K$\ddot{\mbox{u}}$rner~T.} 
Channel measurements and modeling for low-terahertz band vehicular communications~// IEEE 
J.~Sel. Area. Comm., 2021. Vol.~39. No.\,6. P.~1590--1603. doi: 
10.1109/JSAC.2021.3071843.
\bibitem{12-ost}
\Au{Moltchanov D.} Distance distributions in random networks~//  Ad Hoc Netw., 
2012. Vol.~10. P.~1146--1166. doi: 10.48550/arXiv.0804.4204. 
\bibitem{13-ost}
\Au{Basharin G., Gaidamaka~Y.\,V., Samouylov~K.\,E.} Mathematical theory of teletraffic and 
its application to the analysis of multiservice communication of next generation networks~// 
Autom. Control Comp.~S., 2013. Vol.~47. No.\,2. P.~62--69. doi: 
10.3103/S0146411613020028.
\bibitem{14-ost}
\Au{Kingman J.\,F.\,C.} Poisson processes.~--- Oxford studies in probability ser.~--- Claredon Press, 1993. 112~p. doi: 
10.1002/0470011815.B2A07042.
\end{thebibliography}

 }
 }

\end{multicols}

\vspace*{-6pt}

\hfill{\small\textit{Поступила в~редакцию 15.10.22}}

%\vspace*{8pt}

%\pagebreak

\newpage

\vspace*{-28pt}

%\hrule

%\vspace*{2pt}

%\hrule

%\vspace*{-2pt}

\def\tit{ON THE OPTIMAL ANTENNA DEPLOYMENT FOR~SUBTERAHERTZ V2X 
COMMUNICATIONS}


\def\titkol{On the optimal antenna deployment for~subterahertz V2X 
communications}


\def\aut{E.\,A.~Machnev$^1$, V.\,A.~Beschastnyi$^1$, D.\,Yu.~Ostrikova$^1$, 
Yu.\,V.~Gaidamaka$^{1,2}$, and~S.\,Ya.~Shorgin$^2$}

\def\autkol{E.\,A.~Machnev, V.\,A.~Beschastnyi, D.\,Yu.~Ostrikova, et al.} 
%Yu.\,V.~Gaidamaka$^{1,2}$, and~S.\,Ya.~Shorgin$^2$}

\titel{\tit}{\aut}{\autkol}{\titkol}

\vspace*{-8pt}


\noindent
    $^1$Peoples' Friendship University of Russia (RUDN University), 6~Miklukho-Maklaya Str., 
Moscow 117198, Russian\linebreak
$\hphantom{^1}$Federation
    
    \noindent
    $^2$Federal Research Center ``Computer Science and Сontrol'' of the Russian Academy of 
Sciences, 44-2~Vavilov\linebreak
$\hphantom{^1}$Str., Moscow 119333, Russian Federation


\def\leftfootline{\small{\textbf{\thepage}
\hfill INFORMATIKA I EE PRIMENENIYA~--- INFORMATICS AND
APPLICATIONS\ \ \ 2022\ \ \ volume~16\ \ \ issue\ 4}
}%
 \def\rightfootline{\small{INFORMATIKA I EE PRIMENENIYA~---
INFORMATICS AND APPLICATIONS\ \ \ 2022\ \ \ volume~16\ \ \ issue\ 4
\hfill \textbf{\thepage}}}

\vspace*{3pt} 
  
  
    \Abste{Subterahertz (sub-THz, 100--300~GHz) communication should provide huge data 
transfer rates in 6G systems. However, the coverage area of base stations (BS) will be very 
limited, since the signal is quite strongly attenuated from the distance and is also easily blocked 
by the presence of any objects in the signal path. Thus, the BS will need to be located too often 
which is a costly process. To reduce the deployment density of the BS, a~mechanism was 
proposed for relaying the signal using vehicles (V2V). This relaying method is characterized by 
various options for the location of the antenna on vehicles which raises the question of finding 
the optimal location. In this work, guided by the IEEE 802.15.3d specification and measurements 
of the signal propagation level at a~frequency of 300~GHz, the authors developed 
a~mathematical model for comparing multihop signal relay systems with different antenna 
locations. The authors consider the following quality of service indicators: coverage, BS 
availability, and data transfer rate. The results show that the windshield transmitter location has 
a~lower data rate but more coverage while the bumper and engine levels show similar 
performance. A~windshield location is recommended as it is less sensitive to the rate of 
technology integration and has a~larger coverage area.}
  
  \KWE{5G; New Radio; V2V; V2X; multihop communications}
  
 
  
  \DOI{10.14357/19922264220407} 

\vspace*{-8pt}

 \Ack
  \noindent
  The reported study was funded by the Russian Science Foundation, project number 22-29-00694 ({\sf 
https://rscf.ru/en/project/22-29-00694}). 


\vspace*{12pt}

  \begin{multicols}{2}

\renewcommand{\bibname}{\protect\rmfamily References}
%\renewcommand{\bibname}{\large\protect\rm References}

{\small\frenchspacing
 {%\baselineskip=10.8pt
 \addcontentsline{toc}{section}{References}
 \begin{thebibliography}{99}
\bibitem{1-ost-1}
  \Aue{Moltchanov, D., Y.~Gaidamaka, D.~Ostrikova, V.~Bes\-chast\-nyi, Y.~Koucheryavy, and 
K.~Samouylov.} 2021. Ergodic outage and capacity of terahertz systems under micromobility 
and blockage impairments. \textit{IEEE T. Wirel. Commun.} 21(5):3024--3039. 
doi: 10.1109/TWC. 2021.3117583.
\bibitem{2-ost-1}
  \Aue{Moltchanov, D., V.~Beschastnyi, D.~Ostrikova, Y.~Gai\-da\-ma\-ka, and Y.~Koucheryavy.} 
2021. Uninterrupted connectivity time in THz systems under user micromobility and blockage. 
\textit{Global Communications Conference Proceedings}. Piscataway, NJ: IEEE. 
9685384. 6~p. doi: 10.1109/GLOBECOM46510.2021.9685384.
\bibitem{3-ost-1}
  \Aue{Gapeyenko, M., A.~Samuylov, M.~Gerasimenko, D.~Mol\-tcha\-nov, S.~Singh, 
M.\,R.~Akdeniz, E.~Aryafar, N.~Himayat, S.~Andreev, and Y.~Koucheryavy.} 2017. On the 
temporal effects of mobile blockers in urban millimeter-wave cellular scenarios. \textit{IEEE 
T. Veh. Technol.} 66(11):10124--10138. doi: 10.1109/TVT.2017.2754543.
\bibitem{4-ost-1}
  \Aue{Stepanov, N.\,V., D.~Moltchanov, V.~Begishev, A.~Turlikov, and Y.~Koucheryavy.} 
2021. Statistical analysis and modeling of user micromobility for THz cellular communications. 
\textit{IEEE T. Veh. Technol.} 71(1):725--738. doi: 10.1109/TVT.2021.3124870.
\bibitem{5-ost-1}
  \Aue{Beschastnyi, V., D.~Ostrikova, D.~Moltchanov, Y.~Gai\-da\-ma\-ka, Y.~Koucheryavy, and 
K.~Samouylov.} 2022. Balancing latency and energy efficiency in mmWave 5G NR systems 
with multiconnectivity. \textit{IEEE Commun. Lett.} 26(8):1952--1956. doi: 
10.1109/LCOMM.2022.3175663.
\bibitem{6-ost-1}
\Aue{Petrov, V., D.~Moltchanov, S.~Andreev, and R.\,W.~Heath.} 2019. Analysis of intelligent 
vehicular relaying in urban 5G\;+\;millimeter-wave cellular deployments. \textit{Global 
Communications Conference Proceedings}. Piscataway, NJ: IEEE. 05946. 6~p. doi: 
10.48550/arXiv.1908.05946.
  
\bibitem{7-ost-1}
  3GPP. 2020. NR. Study on integrated access and backhaul (Release 17): Technical Specification 38.874 
V17.0.0. Available at: 
{\sf https://www.3gpp.org/ftp/Specs/\linebreak archive/38\_series/38.874/38874-g00.zip} (accessed 
November~28, 2022).
\bibitem{8-ost-1}
  \Aue{Petrov, V., T.~Kurner, and I.~Hosako.} 2020. IEEE 802.15.3d: First standardization 
efforts for sub-terahertz band communications toward 6G. \textit{IEEE Commun. 
Mag.} 58(11):28--33. doi: 10.1109/MCOM.001.2000273
\bibitem{9-ost-1}
  \Aue{Ozpolat, M., K.~Bhargava, E.~Kampert, and M.\,D.~Higgins.} 2021. Multi-lane urban 
mmwave V2V networks: A~path loss behavior dependent coverage analysis. \textit{Vehicular 
Communications} 30:100348. 11 p. doi: 10.1016/ j.vehcom.2021.100348.


\bibitem{10-ost-1}
  \Aue{Wang, J., J.~Liu, and N.~Kato.} 2018. Networking and communications in autonomous 
driving: A~survey. \textit{IEEE Commun. Surv. Tut.} 21(2):1243--1274. doi: 
10.1109/ COMST.2018.2888904.
\bibitem{11-ost-1}
\Aue{Eckhardt, J.\,M., V.~Petrov, D.~Moltchanov, Y.~Koucheryavy, and T.~Kurner.} 2021. 
Channel measurements and modeling for low-terahertz band vehicular communications. 
\textit{IEEE J.~Sel. Area. Comm.} 39(6):1590--1603. doi: 
10.1109/JSAC.2021.3071843.
  
\bibitem{12-ost-1}
\Aue{Moltchanov, D.} 2012. Distance distributions in random networks. \textit{Ad Hoc 
Netw.} 10(6):1146--1166. doi: 10.48550/arXiv.0804.4204. 
\bibitem{13-ost-1}
\Aue{Basharin, G.\,P., Yu.\,V.~Gaidamaka, and K.\,E.~Samouylov.} 2013. Mathematical theory 
of teletraffic and its application to the analysis of multiservice communication of next generation 
networks. \textit{Autom. Control Comp.~S.} 47(2):62--69. doi: 10.3103/S0146411613020028.
  
\bibitem{14ost-1}
  \Aue{Kingman, J.\,F.\,C.} 1993. \textit{Poisson processes}. Oxford studies in probability ser. Claredon Press. 112~p. 
doi: 10.1002/0470011815.B2A07042.

\end{thebibliography}

 }
 }

\end{multicols}

\vspace*{-6pt}

\hfill{\small\textit{Received October 15, 2022}}

\vspace*{-12pt}

  
  \Contr
  
  \vspace*{-3pt}
  
  \noindent
  \textbf{Machnev Egor A.} (b.\ 1996)~--- PhD student, Department of Applied Probability and 
Informatics, Peoples' Friendship University of Russia (RUDN University),  
6~Miklukho-Maklaya Str., Moscow 117198, Russian Federation; \mbox{1032143100@rudn.ru}
  
  \vspace*{3pt}
  
  \noindent
  \textbf{Beschastnyi Vitalii A.} (b.\ 1992)~--- Candidate of Science (PhD) in physics and 
mathematics, assistant professor, Department of Applied Probability and Informatics, Peoples' 
Friendship University of Russia (RUDN University), 6~Miklukho-Maklaya Str., Moscow 
117198, Russian Federation; \mbox{beschastnyy-va@rudn.ru}
  
  
  \vspace*{3pt}
  
  \noindent
  \textbf{Ostrikova Daria Yu.} (b.\ 1988)~--- Candidate of Science (PhD) in physics and 
mathematics, associate professor, Department of Applied Probability and Informatics, Peoples' 
Friendship University of Russia (RUDN University), 6~Miklukho-Maklaya Str., Moscow 
117198, Russian Federation; \mbox{ostrikova-dyu@rudn.ru}
  
  
  
  \vspace*{3pt}
  
  \noindent
  \textbf{Gaidamaka Yuliya V.} (b.\ 1971)~--- Doctor of Science in physics and mathematics, 
professor, Department of Applied Probability and Informatics, Peoples' Friendship University of 
Russia (RUDN University), 6~Miklukho-Maklaya Str., Moscow 117198, Russian Federation; 
senior scientist, Institute of Informatics Problems, Federal Research Center ``Computer Science 
and Control'' of the Russian Academy of Sciences, 44-2~Vavilov Str., Moscow 119333, Russian 
Federation; \mbox{gaydamaka-yuv@rudn.ru}
  
  
  \vspace*{3pt}
  
  \noindent
  \textbf{Shorgin Sergey Ya.} (b.\ 1952)~--- Doctor of Science in physics and mathematics, 
professor, principal scientist, Institute of Informatics Problems, Federal Research Center 
``Computer Science and Control'' of the Russian Academy of Sciences, 44-2~Vavilov Str., 
Moscow 119133, Russian Federation; \mbox{sshorgin@ipiran.ru}
  
\label{end\stat}

\renewcommand{\bibname}{\protect\rm Литература}    
   %7
\include{hatskevich} %8
\def\stat{dukova}

\def\tit{О ПОИСКЕ МАКСИМАЛЬНЫХ ЧАСТЫХ И~МИНИМАЛЬНЫХ НЕЧАСТЫХ НАБОРОВ ПРОИЗВЕДЕНИЯ ЧАСТИЧНЫХ ПОРЯДКОВ}

\def\titkol{О поиске максимальных частых и~минимальных нечастых наборов произведения частичных порядков}

\def\aut{Н.\,А.~Драгунов$^1$, Е.\,В.~Дюкова$^2$}

\def\autkol{Н.\,А.~Драгунов, Е.\,В.~Дюкова}

\titel{\tit}{\aut}{\autkol}{\titkol}

\index{Драгунов Н.\,А.}
\index{Дюкова Е.\,В.}
\index{Dragunov N.\,A.}
\index{Djukova E.\,V.}


%{\renewcommand{\thefootnote}{\fnsymbol{footnote}} \footnotetext[1]
%{Работа выполнена при поддержке Министерства науки и~высшего образования Российской Федерации (проект 
%075-15-2020-799).}}


\renewcommand{\thefootnote}{\arabic{footnote}}
\footnotetext[1]{Федеральный исследовательский центр <<Информатика 
и~управ\-ле\-ние>> Российской академии наук, \mbox{nikitadragunovjob@gmail.com}}
\footnotetext[2]{Федеральный исследовательский центр <<Информатика и~управ\-ле\-ние>> 
Российской академии наук, \mbox{edjukova@mail.ru}}

\vspace*{-3pt}




\Abst{Исследованы актуальные вопросы снижения временных затрат, возникающие при 
логическом анализе данных с~элементами из декартова произведения конечных час\-тич\-но 
упорядоченных множеств. Для задачи поиска по базе транзакций максимальных час\-тых и~минимальных 
нечастых наборов произведения час\-тич\-ных порядков предложен оригинальный метод, 
основанный на решении слож\-ной дискретной задачи, называемой дуализацией 
над произведением час\-тич\-ных порядков. Метод представляет собой синтез двух других 
известных методов, один из которых достаточно очевиден, а~другой использует идею 
инкрементального пе\-ре\-чис\-ле\-ния искомых наборов и~поэтому пред\-став\-ля\-ет 
в~основном тео\-ре\-ти\-че\-ский интерес. Проведено экспериментальное исследование предложенного 
подхода к~решению рас\-смат\-ри\-ва\-емой задачи в~случае произведения конечных цепей,
 выявлены условия его эф\-фек\-тив\-ности и~для проводимого анализа данных показана 
 це\-ле\-со\-об\-раз\-ность применения асимптотически оптимальных алгоритмов дуализации 
 над произведением час\-тич\-ных порядков.}

\KW{максимальные час\-тые наборы; минимальные не\-час\-тые наборы; дуализация над 
произведением час\-тич\-ных порядков; асимп\-то\-ти\-чески оптимальный алгоритм дуализации}

\DOI{10.14357/19922264220112}
  
%\vspace*{-4pt}


\vskip 10pt plus 9pt minus 6pt

\thispagestyle{headings}

\begin{multicols}{2}

\label{st\stat}

    \section{Введение}
    
    Рас\-смат\-ри\-ва\-емая задача анализа данных занимает важ\-ное мес\-то в~об\-ласти 
    информационного поиска и~в~случае бинарных данных ставится сле\-ду\-ющим образом~\cite{4}.
    
    Дано некоторое множество элементов~$V$. Подмножества $X \hm\subseteq V$ называются наборами. Пусть~$D$~--- 
    база данных, содержащая некоторые, не обязательно различные, наборы. Наборы, 
    содержащиеся в~$D$, называются транз\-ак\-ци\-ями. Под частотой набора~$\nu(X)$ понимается доля транз\-ак\-ций в~$D$, 
    содержащих~$X$. Если $\nu(X) \hm\geq s$, $s \hm\in \left[0, 1\right]$, то набор~$X$ называется $s$-час\-тым, 
    иначе он называется $s$-не\-час\-тым. Если набор частый и~он не содержится ни в~каком другом 
    час\-том наборе, то такой набор называется максимальным час\-тым. Если набор не\-час\-тый 
    и~при этом он не содержит в~себе никакого другого не\-час\-то\-го набора, то такой набор 
    называется минимальным нечастым. Требуется найти все максимальные час\-тые и~минимальные не\-час\-тые 
    наборы при заданном~$s$.
    
    Рас\-смат\-ри\-ва\-емая задача имеет много важных приложений, одним из которых является 
    нахождение ассоциативных правил в~базах данных. В~случае бинарных данных ассоциативное правило~---
     это упорядоченная пара $ \left( X, Y \right)$ непересекающихся подмножеств множества~$V$, обо\-зна\-ча\-емая 
     $X \hm\Rightarrow Y$. Поддержкой правила $X \hm\Rightarrow Y$ называется час\-то\-та набора $Z\hm = X \cup Y$.
      Достоверностью правила $X\hm \Rightarrow Y$ называется доля транзакций, со\-дер\-жа\-щих~$Y$, 
      среди всех транзакций, содержащих~$X$. Требуется \mbox{найти} все ассоциативные правила, 
      удовле\-тво\-ря\-ющие заданным минимальной поддержке $s\hm \in [0, 1]$ и~минимальной 
      достоверности $c \hm\in [0, 1]$.  Впервые задача нахождения ассоциативных правил
       была поставлена в~\cite{1}, где она формулировалась как задача анализа по\-тре\-би\-тель\-ской корзины.

    В случае небинарных данных каждый элемент из~$V$ имеет некоторое множество чис\-ло\-вых значений 
    и~вместо наборов элементов рас\-смат\-ри\-ва\-ют\-ся наборы их значений.

    Поиск ассоциативных правил осуществляется в~два этапа. 
    На первом этапе находятся частые наборы, на втором этапе из найденных час\-тых 
    наборов формируются ассоциативные правила. При формировании правил на втором 
    этапе фактически возникает задача поиска $t$-не\-час\-тых наборов, где $t\hm > s/c$.
    
    С ростом размерности современных баз данных находить все час\-тые и~не\-час\-тые 
    наборы становится неэффективно как по времени, так и~по памяти в~силу 
    экспоненциального рос\-та чис\-ла таких наборов. Одно из решений данной проблемы 
    заключается в~поиске только максимальных час\-тых наборов и~только минимальных 
    нечастых наборов, что позволяет компактно хранить информацию о~всех час\-тых и~не\-час\-тых 
    наборах соответственно. 
    
    
    В~\cite{9} рас\-смот\-ре\-на задача поиска множеств максимальных час\-тых наборов~$X_{\max}$ 
    и~минимальных не\-час\-тых наборов~$Y_{\min}$ в~данных, пред\-став\-лен\-ных в~виде декартова 
    произведения час\-тич\-но упорядоченных множеств. Показано, что в~этом случае 
    при построении тре\-бу\-емых наборов возникают соответственно задача поиска 
    максимальных независимых элементов час\-тич\-ных порядков и~задача поиска минимальных 
    независимых элементов час\-тич\-ных порядков.  Каж\-дая из этих задач называется 
    дуализацией над произведением час\-тич\-ных порядков~\cite{8}. Обе задачи относятся к~одним 
    из цент\-раль\-ных труд\-но\-ре\-ша\-емых пе\-ре\-чис\-ли\-тель\-ных задач дис\-крет\-ной математики.
    
    Существует достаточно очевидный способ поиска максимальных час\-тых и~минимальных
     не\-час\-тых наборов произведения час\-тич\-ных порядков, основанный на по\-сле\-до\-ва\-тель\-ном 
     по\-стро\-ении указанных множеств. Одно из множеств ищется, например, алгоритмом Apriori~\cite{2},
      второе множество получается путем дуализации первого. 
      В~настоящей работе показано, что метод эффективен только в~случае, когда чис\-ло час\-тых 
      наборов существенно меньше или, наоборот, существенно больше чис\-ла не\-час\-тых наборов. 
      В~\cite{9} предложена идея со\-вмест\-но\-го пе\-ре\-чис\-ле\-ния~$X_{\max}$ и~$Y_{\min}$ с~использованием
       инкрементального алгоритма дуализации из~\cite{14}, которая автором экспериментально 
       не исследована.
    
    Основной результат настоящей работы~--- разработка нового подхода к~решению 
    поставленной задачи, который является синтезом последовательного и~совместного подходов. 
    
    Экспериментальные исследования, проведенные в~настоящей работе для случая
     произведения цепей, свидетельствуют о~том, что предложенный по\-сле\-до\-ва\-тель\-но-со\-вмест\-ный 
     метод наиболее эффективен в~случае, когда мощ\-ность множества час\-тых наборов примерно 
     равна мощ\-ности множества не\-час\-тых наборов.
     
     \vspace*{-6pt}
     
    
    \section{Постановка задачи поиска максимальных частых 
    и~минимальных нечастых наборов произведения частичных порядков}
    
         \vspace*{-2pt}
    
    Пусть $\mathcal{P} = \mathcal{P}_1 \times \dots \times \mathcal{P}_n$~--- 
    де\-кар\-то\-во произведение час\-тич\-но упорядоченных множеств. Элементы~$\mathcal{P}$ называются наборами. 
    На множестве~$\mathcal{P}$ определяется отношение частичного порядка~$\preceq$ сле\-ду\-ющим образом: 
    если $p \hm= (p_1, \dots, p_n) \hm\in \mathcal{P}$ и~$q \hm= (q_1, \dots, q_n)\hm \in \mathcal{P}$, 
    то $ p \hm\preceq q$ в~$ \mathcal{P}\hm \Leftrightarrow p_1 \hm\preceq q_1$ 
    в~$\mathcal{P}_1, \dots, p_n \hm\preceq q_n$ в~$ \mathcal{P}_n$.
    
    Пусть $\mathcal{D} (\mathcal{P})$~--- некоторая со\-во\-куп\-ность
     наборов из~$\mathcal{P}$, называемая базой данных. Наборы, на\-хо\-дя\-щи\-еся в~базе 
     данных $\mathcal{D} (\mathcal{P})$, необязательно по\-пар\-но раз\-лич\-ны и~называются транзакциями. 
     
    Введем обозначения: 
    $\vert \mathcal{D} (\mathcal{P}) \vert$~--- чис\-ло транз\-ак\-ций в~$\mathcal{D} (\mathcal{P})$; 
    $\mathcal{S}_\mathcal{D}(p)$~--- число транз\-ак\-ций в~$\mathcal{D} (\mathcal{P})$, 
    сле\-ду\-ющих за $p \hm\in \mathcal{P}$; $s \hm\in [0, 1]$. 
    
    \smallskip
    
    \noindent
    \textbf{Определение~1.}\
     Набор $p \in \mathcal{P}$ называется $s$-час\-тым, 
     если $\mathcal{S}_\mathcal{D}(p) / \vert \mathcal{D} (\mathcal{P}) \vert \hm\geq s$. Иначе набор~$p$ 
     называется $s$-не\-час\-тым.
    
    \smallskip
    
    \noindent
    \textbf{Определение~2.}\
    Набор $p \in \mathcal{P}$ называется максимальным $s$-час\-тым, если 
    он $s$-час\-тый и~никакой сле\-ду\-ющий за ним набор~$z$, $z\hm \neq p$, не является $s$-час\-тым.

    
    \smallskip
    
    \noindent
    \textbf{Определение~3.}\
    Набор $p \in \mathcal{P}$ называется минимальным $s$-не\-час\-тым, если он $s$-не\-час\-тый 
    и~никакой пред\-шест\-ву\-ющий ему набор~$z$, $z \hm\neq p$, не является $s$-не\-час\-тым.


\smallskip
    
    Далее вместо $s$-частый ($s$-не\-час\-тый) набор будем писать час\-тый (не\-час\-тый) набор. 
    Множество всех максимальных час\-тых наборов будем обозначать как $X_{\max}$, 
    а~множество всех минимальных не\-час\-тых наборов как $Y_{\min}$.
    
    Пусть $R \subset \mathcal{P}$, $R^+\hm = \{ x \in \mathcal{P} \vert \exists\, a \hm\in R, a \hm\preceq x \}$, 
    $R^- \hm= \{ x \hm\in \mathcal{P} \vert \exists\, a \hm\in R, x \hm\preceq a \}$.


    \noindent
    \textbf{Определение~4.}\
     Множество $I(R^+)$, со\-сто\-ящее из всех максимальных элементов множества~$\mathcal{P} \setminus R^+$, 
     называется максимальным независимым от~$R$.

\smallskip


   \noindent
    \textbf{Определение~5.}\
     Множество $I(R^-)$, со\-сто\-ящее из всех минимальных элементов множества~$\mathcal{P} \setminus R^-$, 
     называется минимальным независимым от~$R$.

\smallskip
    
    Каждая из задач построения $I(R^+)$ и~$I(R^-)$ 
    при заданном множестве~$R$ называется задачей дуализации над произведением час\-тич\-ных порядков.
    
    \smallskip

    \noindent
    \textbf{Утверждение~1.}\
    Если $X \hm\subset X_{\max}$, а~$y \hm\in I(X^-)$~--- не\-час\-тый набор, 
    то~$y$~--- минимальный не\-час\-тый набор.

\smallskip    
    
    \noindent
    Д\,о\,к\,а\,з\,а\,т\,е\,л\,ь\,с\,т\,в\,о\,.\  \ 
    Пусть $y \hm\notin I(X_{\max}^-)$. Так как~$y$~--- 
    нечастый набор, то в~$\mathcal{P} \setminus X^{-}_{\max}$ найдется минимальный не\-час\-тый набор~$x$ 
    такой, что $x\hm \neq y$ и~$x \hm\preceq y$. Из того, что $\mathcal{P} \setminus X^{-}_{\max} 
    \hm\subseteq \mathcal{P} \setminus X^-$, следует, что $x\hm \in \mathcal{P} \setminus X^-$, 
    что противоречит условию $y \hm\in I(X^-)$.

\smallskip

\noindent
\textbf{Утверждение~2.}\
    Пусть $X \hm\subseteq X_{\max}$, $Y\hm \subseteq Y_{\min}$. 
    Тогда $I(X^-) \hm= Y$ в~том и~только в~том случае, когда $X \hm= X_{\max}$ и~$Y \hm= Y_{\min}$.


\smallskip


  \noindent
    Д\,о\,к\,а\,з\,а\,т\,е\,л\,ь\,с\,т\,в\,о\,.\  \
    Пусть $X\! \subset\! X_{\max}, x \hm\in X_{\max}\!\setminus\!X$.
     Так как множество~$X_{\max}$~--- антицепь, то $x \hm\notin X^-$. 
     Следовательно, $x \hm\in \mathcal{P} \setminus X^{-}$.
      Но тогда существует элемент $ q \hm\in I(X^-) : q \preceq x$, 
      который является час\-тым. Однако во множестве~$Y$ частых наборов нет; следовательно, $I(X^-) \hm\neq Y$. 
      Если же $X \hm= X_{\max}$, то $I(X^-) \hm= Y_{\min}$. Таким образом, $I(X^-) \hm= Y$ тогда и~только
       тогда, когда $X \hm= X_{\max}$ и~$Y\hm = Y_{\min}$.


    
    \section{Методы построения множеств~$X_{\max}$ и~$Y_{\min}$}

    \subsection{Последовательное перечисление $X_{\max}$~и~$Y_{\min}$}

    Достаточно очевиден поиск~$X_{\max}$ и~$Y_{\min}$ при заданной $\mathcal{D} (\mathcal{P})$ 
    путем последовательного по\-стро\-ения множеств~$X_{\max}$ и~$Y_{\min}$. 
    Данный поиск осуществляется в~два этапа. На первом этапе находятся все максимальные частые 
    наборы~$X_{\max}$, например алгоритмом Apriori~\cite{2}. На втором этапе  используется свойство 
    двойственности $I \left(X_{\max}^- \right)\hm = Y_{\min}$. 
    Минимальные нечастые наборы~$Y_{\min}$ находятся путем дуализации найденного на первом этапе 
    множества~$X_{\max}$. Аналогично можно сначала искать~$Y_{\min}$ алгоритмом Apriori, а~затем 
    искать~$X_{\max}$ путем дуализации~$Y_{\min}$.

    Очевидно, что данный подход будет проявлять себя наилучшим образом в~случаях, когда 
    алгоритм Apriori или его модификации могут найти одно из искомых множеств существенно
     быст\-рее, чем другое множество, например когда мощ\-ность~$X_{\max}$ 
     существенно меньше (больше) мощ\-ности~$Y_{\min}$.
    
    \subsection{Совместное перечисление $X_{\max}$ и~$Y_{\min}$}

    В~\cite{9} предложена идея совместного перечисления множеств~$X_{\max}$ и~$Y_{\min}$. 
    На первом шаге рас\-смат\-ри\-ва\-ет\-ся некоторый случайный набор $q \hm\in \mathcal{P}$. Если $q$~--- 
    час\-тый набор, то ищется максимальный час\-тый набор, сле\-ду\-ющий за~$q$, 
    который пополняет множество $X \hm\subseteq X_{\max}$. Если $q$~---
     не\-час\-тый набор, то ищется минимальный не\-час\-тый набор, пред\-шест\-ву\-ющий~$q$, 
     который пополняет множество $Y \hm\subseteq Y_{\min}$. Пусть на шаге~$i$ ($i\hm \geq 1$) 
     построены множества $X \hm\subseteq X_{\max}$ и~$Y \hm\subseteq Y_{\min}$. Если $X \hm\neq \varnothing$, 
     $Y \hm= \varnothing$, то ищется набор~$q$ такой, что $q \hm\npreceq x, \forall x \hm\in X$. Если 
     $X \hm= \varnothing$, $Y \hm\neq \varnothing$, то ищется набор~$q$ такой, что 
     $q \hm\nsucceq y, \forall y \hm\in Y$. Если же и~$X \hm\neq \varnothing$, и~$Y \hm\neq \varnothing$, 
     то ищется набор~$q$ такой, что $q \hm\npreceq x, \forall x \hm\in X, q \hm\nsucceq y, \forall y \hm\in Y$.
      Затем, аналогично первому шагу, находится максимальный частый или минимальный нечастый набор. 
      Однако в~\cite{9} идея совместного перечисления искомых множеств экспериментально 
      не исследована и~не предложены конкретные указания по воз\-мож\-ной ее реализации.
    
    Алгоритм, основанный на совместном пе\-ре\-чис\-ле\-нии множеств~$X_{\max}$ и~$Y_{\min}$,
     реализован в~на\-сто\-ящей работе. Алгоритм строит две последовательности: $X_1 \hm\subset X_2 
     \subset \dots \subset X_{\max}$, $Y_1\hm \subset Y_2 \subset \dots \subset Y_{\min}$. 
     На первом шаге $X_1 \hm= \{x\}$, $Y_1 \hm= \{y\}$, где~$x$ и~$y$ ищутся алгоритмом Apriori.
      На шаге $i \hm+ 1$ ($i\hm \geq 1$) строится либо~$I(X^{-}_{i})$, либо~$I(Y^{+}_{i})$. Пусть на 
      шаге $i \hm+ 1$ ($i \hm\geq 1$) построено множество~$I(X^{-}_{i})$. 
      Согласно утверждениям~1 и~2, множество~$I(X^{-}_{i})$ либо не содержит час\-тых наборов 
      и~совпадает с~множеством~$Y_{\min}$ (в~этом случае $X_i \hm= X_{\max}$ 
      и~алгоритм заканчивает работу), либо~$I(X^{-}_{i})$ содержит как час\-тые, так и~не\-час\-тые наборы. 
      Каждый нечастый набор из~$I(X^{-}_{i})$ является минимальным не\-час\-тым и~пополняет множество~$Y_{i}$, 
      формируя в~результате множество~$Y_{i+1}$. Для каждого час\-то\-го набора находится один содержащий 
      его максимальный час\-тый набор путем последовательного увеличения текущего 
      частого набора в~лексикографическом порядке, который пополняет множество~$X_{i}$, 
      формируя в~результате множество~$X_{i+1}$.
      
    В~экспериментальной части работы (см.\ разд.~4) рас\-смот\-рен случай произведения цепей. 
    Задача дуализации решается с~помощью асимптотически оптимального алгоритма дуализации
     цепей \mbox{RUNC-M}+~\cite{7}. Асимптотически оптимальные алгоритмы дуализации 
     являются лидерами по ско\-рости счета~\cite{6}.

    Очевидно, что время работы совместного алгоритма в~основном зависит от чис\-ла
     минимальных не\-час\-тых и~максимальных час\-тых наборов. На\linebreak каж\-дой новой 
     итерации происходит дуализация\linebreak все б$\acute{\mbox{о}}$льших по мощ\-ности множеств~$X$ или~$Y$.\linebreak 
     Если число итераций становится достаточно\linebreak большим, то ско\-рость работы совместного 
     перечисления существенно снижается, что делает его практически неприменимым для 
     задач большой раз\-мер\-ности.
     { %\looseness=1
     
     }

    \subsection{Последовательно-совместное перечисление~$X_{\max}$ и~$Y_{\min}$}

    Предлагается следующий итеративный метод, который синтезирует идеи последовательного
     и~совместного методов, описанных выше. Положим $X_0 \hm= \varnothing$. 
     Строится одна по\-сле\-до\-ва\-тель\-ность $X_1 \hm\subset X_2 \hm\subset \dots \subset X_{\max}$. 
     На первом шаге $X_1\hm = \{x\}$, где $x$ ищется алгоритмом Apriori. На шаге $i \hm+ 1$ ($i \hm\geq 1$) 
     решается задача дуализации множества $X_{i} \setminus X_{i-1}$.

    
    
   \setcounter{figure}{1}
    \begin{figure*}[b] %fig2
  \vspace*{12pt}
  \begin{center}  
    \mbox{%
\epsfxsize=163mm
\epsfbox{duk-2.eps}
}

\end{center}
\vspace*{-9pt}
    \Caption{Зависимость времени работы алгоритмов от суммы мощностей множеств~$X_{\max}$ и~$Y_{\min}$ 
    для случая~1~(\textit{а}) и~2~(\textit{б}):
    \textit{1}~--- по\-сле\-до\-ва\-тель\-но-со\-вмест\-ный;
    \textit{2}~--- последовательный; \textit{3}~--- совместный; \textit{4}~--- Apriori}
    \label{12}
    \end{figure*}
     
    Пусть множество~$D$ есть результат дуализации $X_{i} \hm\setminus X_{i-1}$. Согласно утверждению~1, 
    множество~$D$ содержит частые наборы. Для каждого час\-то\-го набора из~$D$ 
    находится один содержащий его максимальный час\-тый набор путем последовательного 
    увеличения текущего час\-то\-го набора в~лексикографическом порядке. Все найденные максимальные
     частые наборы, которых нет в~множестве~$X_{i}$, до\-бав\-ля\-ют\-ся к~$X_{i}$, 
     и~таким образом формируется~$X_{i+1}$. Если же все найденные частые наборы уже содержатся в~$X_{i}$, 
     то решается задача дуализации множества~$X_{i}$. Если в~$I(X^{-}_{i})$ нет частых наборов, 
     то $I(X^{-}_{i})\hm = Y_{\min}$, $X_i \hm= X_{\max}$ и~алгоритм завершает работу. 
     Иначе для каждого частого набора из~$I(X^{-}_{i})$ находится один содержащий его максимальный 
     час\-тый набор, который пополняет множество~$X_{i}$, формируя в~результате множество~$X_{i+1}$.

    \section{Экспериментальное исследование}
    
    Рас\-смат\-ри\-вал\-ся случай данных, пред\-став\-лен\-ных в~виде произведения цепей мощ\-ности~5. 
    Для\linebreak таких данных проводился поиск максимальных час\-тых и~минимальных нечастых 
    наборов сле\-ду\-ющи\-ми методами: алгоритмом Apriori, модифицированным для случая 
    цепей; последовательным \mbox{методом}; совместным методом; по\-сле\-до\-ва\-тель\-но-со\-вмест\-ным методом.
    
    Все методы реализованы на языке Python~3. 
    Задача дуализации решалась алгоритмом дуализации цепей RUNC-M+~\cite{7}. 
    Эксперименты проведены на случайных базах данных различной раз\-мер\-ности. 
    Можно выделить два сле\-ду\-ющих случая соотношения мощностей множеств всех час\-тых и~не\-час\-тых наборов.
    \begin{description}
    \item[Случай 1:] мощ\-ность множества частых наборов примерно рав\-на мощ\-ности множества нечастых наборов.
    \item[Случай 2:] мощ\-ность множества частых наборов существенно меньше (больше) мощ\-ности множества 
    не\-час\-тых наборов.
    \end{description}
    
    Описанные случаи схематично изображены на рис.~1. 

    Графики зависимости времени работы тестируемых методов 
    от мощ\-ности множеств~$X_{\max}$ и~$Y_{\min}$ приведены на рис.~2.
    
    

    

    Нетрудно видеть, что в~случае~1 лучше работает по\-сле\-до\-ва\-тель\-но-со\-вмест\-ный алгоритм: 
    множества час\-тых и~не\-час\-тых наборов имеют примерно одинаковую мощ\-ность, 
    поэтому быст\-рее будет обрабатывать их по\-сле\-до\-ва\-тель\-но-со\-вмест\-ным методом. В~случае~2 
    быст\-рее работает последовательный алгоритм: быст\-рее найти множество максимальных час\-тых наборов, 
    обработав множество час\-тых наборов, и~дуализировать результат. Время поиска множеств~$X_{\max}$ 
    и~$Y_{\min}$ совместным методом и~модифицированным алгоритмом Apriori рас\-тет существенно 
    быст\-рее времени поиска по\-сле\-до\-ва\-тель\-но-со\-вмест\-ным методом в~обоих случаях.
    
    { \begin{center}  %fig1
 \vspace*{9pt}
    \mbox{%
\epsfxsize=67.963mm
\epsfbox{duk-1.eps}
}

\end{center}

\noindent
{{\figurename~1}\ \ \small{
Два случая соотношения мощностей множеств час\-тых и~не\-час\-тых наборов
}}}

%\vspace*{6pt}


    \section{Заключение}
    
Рас\-смот\-ре\-на задача поиска максимальных час\-тых и~минимальных не\-час\-тых наборов в~данных, 
представленных в~виде декартова произведения час\-тич\-ных порядков. Актуальны вопросы 
снижения временн$\acute{\mbox{ы}}$х затрат, возникающих при реализации методов нахождения искомых наборов.
 Разработан новый подход к~по\-стро\-ению множества максимальных частых наборов~$X_{\max}$ и~множества 
 минимальных не\-час\-тых наборов~$Y_{\min}$, пред\-став\-ля\-ющий собой синтез двух ранее известных 
 подходов: последовательного и~со\-вмест\-но\-го (первый достаточно очевиден, идея второго предложена в~\cite{9}). 
 Сложность последовательного, совместного и~пред\-ла\-га\-емо\-го по\-сле\-до\-ва\-тель\-но-со\-вмест\-но\-го поиска 
 обуслов\-ле\-на, в~том чис\-ле, не\-об\-хо\-ди\-мостью рас\-смат\-ри\-вать в~процессе поиска 
 труд\-но\-ре\-ша\-емую пе\-ре\-чис\-ли\-тель\-ную задачу дис\-крет\-ной математики, на\-зы\-ва\-емую дуализацией 
 над произведением час\-тич\-ных порядков.

Для случая, когда данные пред\-став\-ле\-ны в~виде произведения конечных цепей, 
приведены результаты экспериментального срав\-не\-ния названных подходов, а~так\-же независимого 
способа \mbox{по\-стро\-ения} множеств~$X_{\max}$ и~$Y_{\min}$, не тре\-бу\-юще\-го решения задачи дуализации. 
Эксперименты проводились на модельных задачах с~применением асимптотически оптимального
 алгоритма дуализации над произведением конечных цепей \mbox{RUNC-M}+~\cite{7}. 
 Результаты исследования свидетельствуют о~том, что по\-сле\-до\-ва\-тель\-но-со\-вмест\-ный 
 метод наиболее эффективен (требует меньших временн$\acute{\mbox{ы}}$х затрат по сравнению с~другими рас\-смот\-рен\-ны\-ми 
 методами) в~случае, когда мощ\-ность множества час\-тых наборов примерно равна мощ\-ности множества
  нечастых наборов. Иначе выигрывает последовательный поиск. Наихудшие показатели 
  у~независимого пе\-ре\-чис\-ле\-ния множеств~$X_{\max}$ и~$Y_{\min}$ с~использованием в~качестве
   базового алгоритма Apriori~\cite{2}, точ\-нее его модификации на тес\-ти\-ру\-емый случай. 
   Таким образом, показана це\-ле\-со\-об\-раз\-ность применения алгоритмов дуализации для 
   по\-стро\-ения множеств~$X_{\max}$ и~$Y_{\min}$.

  
  {\small\frenchspacing
 {%\baselineskip=10.8pt
 %\addcontentsline{toc}{section}{References}
 \begin{thebibliography}{9}  
    \bibitem{4}
    \Au{Aggarwal C.} 
    Frequent pattern mining.~--- Heidelberg: Springer, 2014. 467~p.
    
    \bibitem{1}
    \Au{Agrawal~R., Imielinski~T., Swami~A.} Mining association rules 
    between sets of items in large databases~// \mbox{SIGMOD} Conference (International) on Management of Data
    Proceedings.~--- New York, NY, USA: ACM, 1993. P.~207--216.
    
    \bibitem{9}
    \Au{Elbassioni K.} On finding minimal infrequent elements in multi-dimensional 
    data defined over partially ordered sets~// arXiv.org, 2014. 30~p. arXiv:1411.2275 [cs.DB].
    
    \bibitem{8}
    \Au{Elbassioni K.} Algorithms for dualization over products of partially 
    ordered sets~// SIAM J.~Discrete Math., 2009. Vol.~23. Iss.~1. P.~487--510.
    
    \bibitem{2}
    \Au{Agrawal R., Srikant~R.} 
    Fast algorithms for mining association rules in large databases~// 
    20th Conference (International) on Very Large Data Bases Proceedings.~--- San Francisco, CA, USA: 
    Morgan Kaufmann Publs. Inc., 1994. P.~487--499.
    
    \bibitem{14}
    \Au{Хачиян Л.\,Г.} Избранные труды.~--- М.: МЦНМО, 2009. 520~с.
    
    \bibitem{7}
    \Au{Дюкова Е.\,В., Масляков~Г.\,О., Прокофьев~П.\,А.} 
    О~дуализации над произведением частичных порядков~// Машинное обучение и~анализ данных, 2017. Т.~3. №\,4.  
    C.~239--249.
    
    \bibitem{6}
    \Au{Дюкова Е.\,В., Прокофьев~П.\,А.} Об асимптотически оптимальных алгоритмах дуализации~// 
    Ж.~вычисл. матем. и~матем. физ., 2015. Т.~55. №\,5. С.~895--910.
    \end{thebibliography}

 }
 }

\end{multicols}

\vspace*{-6pt}

\hfill{\small\textit{Поступила в~редакцию 15.01.21}}

\vspace*{8pt}

%\pagebreak

%\newpage

%\vspace*{-28pt}

\hrule

\vspace*{2pt}

\hrule

%\vspace*{-2pt}

\def\tit{FINDING MAXIMAL FREQUENT AND~MINIMAL INFREQUENT SETS IN~PARTIALLY ORDERED DATA}


\def\titkol{Finding maximal frequent and~minimal infrequent sets in~partially ordered data}


\def\aut{N.\,A.~Dragunov and E.\,V.~Djukova}

\def\autkol{N.\,A.~Dragunov and E.\,V.~Djukova}

\titel{\tit}{\aut}{\autkol}{\titkol}

\vspace*{-11pt}


\noindent
Federal Research Center ``Computer Science and Control'' 
of the Russian Academy of Sciences, 44-2~Vavilov Str., Moscow 119333, Russian Federation

\def\leftfootline{\small{\textbf{\thepage}
\hfill INFORMATIKA I EE PRIMENENIYA~--- INFORMATICS AND
APPLICATIONS\ \ \ 2022\ \ \ volume~16\ \ \ issue\ 1}
}%
 \def\rightfootline{\small{INFORMATIKA I EE PRIMENENIYA~---
INFORMATICS AND APPLICATIONS\ \ \ 2022\ \ \ volume~16\ \ \ issue\ 1
\hfill \textbf{\thepage}}}

\vspace*{3pt} 


\Abste{Relevant issues of time costs reducing in the logical analysis of data with elements 
from the Cartesian product of finite partially ordered sets are investigated. 
An original method based on solving a complex discrete problem called dualization
 over the product of partial orders is proposed for the problem of finding maximal 
 frequent and minimal infrequent sets in the transaction database. The proposed method 
 is a~synthesis of two other known methods, one of which is quite obvious and the other uses 
 the idea of an incremental enumeration of target\linebreak\vspace*{-12pt}}
 
 \Abstend{sets and is, therefore, mainly 
 of theoretical interest. An experimental study of the considered approaches in
  the case of the product of finite chains is carried out and conditions for
   their effectiveness are revealed. The expediency of applying 
asymptotically optimal dualization algorithms over the product of partial orders is shown.}

\KWE{maximal frequent sets; minimal infrequent sets; dualization over the product of 
partial orders; asymptotically optimal dualization algorithm}

\DOI{10.14357/19922264220112}

%\vspace*{-16pt}

%\Ack
%\noindent




%\vspace*{6pt}

  \begin{multicols}{2}

\renewcommand{\bibname}{\protect\rmfamily References}
%\renewcommand{\bibname}{\large\protect\rm References}

{\small\frenchspacing
 {%\baselineskip=10.8pt
 \addcontentsline{toc}{section}{References}
 \begin{thebibliography}{9}
\bibitem{1-dr}
\Aue{Aggarwal, C.} 2014. \textit{Frequent pattern mining}. Heidelberg: Springer. 467~p.
\bibitem{2-dr}
\Aue{Agrawal, R., T.~Imielinski, and A.~Swami.}
 1993. Mining association rules between sets of items in large databases. 
 \textit{SIGMOD  Conference (International) on Management of Data Proceedings}. New York, NY:
 ACM. 207--216. 
\bibitem{3-dr}
\Aue{Elbassioni, K.}
 2014. On finding minimal infrequent elements in multidimensional data defined over partially ordered sets. 
 arXiv.org. 30~p. Available at: 
 {\sf https://arxiv.org/\linebreak pdf/1411.2275.pdf} (accessed January~25, 2022).
\bibitem{4-dr}
\Aue{Elbassioni, K.} 2009. Algorithms for dualization over products of partially ordered sets. 
\textit{SIAM J.~Discrete Math.} 23(1):487--510.
\bibitem{5-dr}
\Aue{Agrawal, R., and R.~Srikant.}
 1994. Fast algorithms for mining association rules in large databases. 
 \textit{20th Conference (International) on Very Large Data Bases Proceedings}.
 San Francisco, CA: 
    Morgan Kaufmann Publs. Inc.  487--499.
\bibitem{6-dr}
\Aue{Khachiyan, L.\,G.} 2009. \textit{Izbrannye trudy} [Selected works]. Moscow: MCCME. 520~p.
\bibitem{7-dr}
\Aue{Djukova, E.\,V., G.\,O.~Maslyakov, and P.\,A.~Prokofyev.} 
2017. O~dualizatsii nad proizvedeniem chastichnykh poryadkov [On dualization over the product of 
partial orders]. \textit{Mashinnoe obuchenie i~analiz dannykh} [J.~Machine Learning Data Analysis] 
3(4):239--249.
\bibitem{8-dr}
\Aue{Djukova, E.\,V., and P.\,A.~Prokofyev.}
 2015. Asymptotically optimal dualization algorithms. \textit{Comp. Math.
 Math. Phys.} 55(5):891--905. 
 
 \end{thebibliography}

 }
 }

\end{multicols}

\vspace*{-6pt}

\hfill{\small\textit{Received January 15, 2021}}

%\pagebreak

%\vspace*{-18pt}

\Contr

\noindent
\textbf{Dragunov Nikita A.} (b.\ 1997)~--- 
PhD student, Federal Research Center ``Computer Science and Control'' 
of the Russian Academy of Sciences, 44-2~Vavilov Str., Moscow 119333, Russian Federation; 
\mbox{nikitadragunovjob@gmail.com}

\vspace*{3pt}

\noindent
\textbf{Djukova Elena V.} (b.\ 1945)~--- 
Doctor of Science in physics and mathematics, principal scientist, Federal Research Center
``Computer Science and Control'' of the Russian Academy of Sciences, 44-2~Vavilov Str., Moscow 119333, 
Russian Federation; \mbox{edjukova@mail.ru}




\label{end\stat}

\renewcommand{\bibname}{\protect\rm Литература}  %9
\def\stat{bosov+stef}

\def\tit{УПРАВЛЕНИЕ ВЫХОДОМ СТОХАСТИЧЕСКОЙ ДИФФЕРЕНЦИАЛЬНОЙ СИСТЕМЫ 
ПО~КВАДРАТИЧНОМУ КРИТЕРИЮ. I.~ОПТИМАЛЬНОЕ РЕШЕНИЕ МЕТОДОМ 
ДИНАМИЧЕСКОГО ПРОГРАММИРОВАНИЯ$^*$}

\def\titkol{Управление выходом стохастической дифференциальной системы 
по~квадратичному критерию. I}
%.~Оптимальное решение методом 
%динамического программирования}

\def\aut{А.\,В.~Босов$^1$, А.\,И.~Стефанович$^2$}

\def\autkol{А.\,В.~Босов, А.\,И.~Стефанович}

\titel{\tit}{\aut}{\autkol}{\titkol}

\index{Босов А.\,В.}
\index{Стефанович А.\,И.}
\index{Bosov A.\,V.}
\index{Stefanovich A.\,I.}




{\renewcommand{\thefootnote}{\fnsymbol{footnote}} \footnotetext[1]
{Работа выполнена при частичной поддержке РФФИ (проект 16-07-00677).}}


\renewcommand{\thefootnote}{\arabic{footnote}}
\footnotetext[1]{Институт проблем информатики Федерального исследовательского центра <<Информатика 
и~управление>> Российской академии наук, \mbox{AVBosov@ipiran.ru}}
\footnotetext[2]{Институт проблем информатики Федерального исследовательского центра <<Информатика 
и~управление>> Российской академии наук, \mbox{AStefanovich@frccsc.ru}}

%\vspace*{8pt}



  
  \Abst{Решается задача оптимального управления для диффузионного процесса 
Ито и~линейного управ\-ля\-емо\-го выхода. Рассматриваемая постановка близка 
к~классической ли\-ней\-но-квад\-ра\-тич\-ной гауссовской задаче управления 
(linear-quadratic Gaussian (LQG) control). Отличия состоят в~том, что состояние описывается нелинейным 
дифференциальным уравнение Ито $dy_t\hm= A_t(y_t) \,dt\hm+ \Sigma_t(y_t)\,dv_t$ 
и~не зависит от управ\-ле\-ния~$u_t$, оптимизации подлежит управ\-ля\-емый 
линейный выход $dz_t\hm= a_t y_t\,dt\hm+ b_t z_t \,dt\hm+ c_t u_t \,dt\hm+ \sigma_t\, 
dw_t$. Дополнительные обобщения внесены в~квад\-ра\-тич\-ный критерий качества 
с~целью воз\-мож\-ности постановки таких задач, как отслеживание выходом 
состояния или управ\-ле\-ни\-ем~--- линейной комбинации состояния и~выхода. Для 
решения используется метод динамического программирования. Функцию 
Беллмана позволяет найти предположение о~ее структуре вида $V_t(y,z)\hm= 
\alpha_t z^2\hm+ \beta_t(y)z \hm+\gamma_t(y)$. Решение дают три 
дифференциальных уравнения для коэффициентов~$\alpha_t$, $\beta_t(y)$ 
и~$\gamma_t(y)$. Эти уравнения со\-став\-ля\-ют оптимальное решение 
рас\-смат\-ри\-ва\-емой задачи.}
  
  \KW{стохастическое дифференциальное уравнение; оптимальное управ\-ле\-ние; 
динамическое программирование; функция Беллмана; уравнение Риккати; 
линейные уравнения параболического типа}

\DOI{10.14357/19922264180314}
  
%\vspace*{4pt}


\vskip 10pt plus 9pt minus 6pt

\thispagestyle{headings}

\begin{multicols}{2}

\label{st\stat}

\section{Введение}

     Ключевые результаты в~области оптимизации стохастических 
динамических систем, со\-став\-ля\-ющие классическую теорию управления, 
получены более~40~лет назад (такова работа~[1] в~отношении задачи 
управ\-ле\-ния ли\-ней\-но-гаус\-сов\-ски\-ми стохастическими сис\-те\-ма\-ми по 
квад\-ра\-тич\-но\-му критерию). К~классической тео\-рии следует относить 
линейные модели стохастических сис\-тем и~квадратичный критерий качества. 
Это исходный базис, на котором основано множество успешно 
исследованных и~решенных задач стохастического управ\-ле\-ния 
и~оптимизации. 

Дальнейшее развитие~--- это новые модели и~критерии, но 
прежде всего это новые методы: от тео\-рии линейных регуляторов, метода 
динамического программирования и~принципа максимума к~адаптивному 
и~минимаксному подходу, импульсному управ\-ле\-нию и~т.\,д. Множество 
инноваций как в~час\-ти моделей, так и~в~час\-ти математического аппарата, 
имевших мес\-то в~по\-сле\-ду\-ющие годы, существенно обогатили тео\-рию 
управ\-ле\-ния. Но и~до настоящего времени линейные модели и~квадратичный 
критерий, несмотря на всю справедливую критику в~отношении их 
аде\-кват\-ности и~гиб\-кости, сохраняют исследовательский интерес и~находят 
современные области приложения.
     
     Не претендуя на сколь\-ко-ни\-будь полное обосно\-ва\-ние последнего 
тезиса, приведем несколько примеров, показавшихся наиболее ин\-те\-рес\-ными. 

Так, в~[2] решается ли\-ней\-но-квад\-ра\-тич\-ная за\-да\-ча в~игровой 
постановке с~запаздыванием. В~близ\-кой по модели работе~[3] задача 
управ\-ле\-ния ставится в~терминах $H_\infty$-ро\-баст\-ности. Точнее \mbox{называть} 
эту тематику $H_2/H_\infty$-управ\-ле\-ни\-ем, и~работ по этой теме очень 
много. Аккуратности ради следует уточнить, что под линейными 
понимаются модели с~мультипликативными по состоянию воз\-му\-ще\-ниями. 

Совсем другой класс моделей, особо популярных в~по\-след\-ние годы, 
составляют скачкообразные процессы. Например, линейные уравнения 
в~сочетании с~пуассоновскими скачками в~[4] используются в~моделях, 
описывающих различные показатели функционирования сетевых протоколов 
передачи данных транспортного уровня. Телекоммуникации представляют 
в~последние годы самый популярный прикладной материал для 
исследований, работ по этой проб\-ле\-ма\-ти\-ке множество, математические 
техники привлекаются самые разные и~самые современные, но и~линейным 
моделям место находится. Еще один любопытный пример исследования 
скачкообразного процесса и~оптимизации на основе квад\-ра\-тич\-но\-го критерия 
можно найти в~[5] применительно к~задаче инвестирования на финансовом 
рынке. Наконец, упомянем еще работу~[6], подводящую итог исследований 
в~отношении классической детерминированной  
ли\-ней\-но-квад\-ра\-тич\-ной задачи с~использованием техники матричных 
неравенств.
     
     В данной работе также эксплуатируются привлекательные свойства 
линейных моделей и~квад\-ра\-тич\-но\-го критерия, причем в~стохастической 
постановке. На\-прав\-ле\-ни\-ем для обобщения \mbox{выбрана} модель динамики 
сис\-те\-мы: основные усилия на\-прав\-ле\-ны на то, чтобы сделать ее нелинейной. 
Кроме того, пред\-став\-лен\-ная постановка может рас\-смат\-ри\-вать\-ся и~как 
обобщение ранее решенной задачи в~дискретном времени~[7, 8] на время 
непрерывное. В~упомянутых работах помимо собственно модельной 
постановки важна еще и~привлекаемая прикладная об\-ласть~--- 
функционирование сложных программных сис\-тем. Результатов, 
ориентированных непосредственно на такие приложения, к~настоящему 
времени пренебрежимо мало, поэтому~[7, 8]~--- это еще и~прикладное 
обоснование рас\-смат\-ри\-ва\-емой далее задачи.
     
     Оптимизируемая динамическая сис\-те\-ма описывается двумя 
уравнениями. Состояние задается нелинейным стохастическим 
дифференциальным уравнением Ито, не содержащим управ\-ля\-емой 
переменной. Возмущение здесь описывается стандартным винеровским 
процессом, накладываются простые условия существования 
и~един\-ст\-вен\-ности решения. Поскольку состояние не управ\-ля\-ет\-ся, то уместно 
его интерпретировать как слож\-ное внешнее возмущение. Вторая 
переменная~--- управ\-ля\-емый выход~--- задается линейным стохастическим 
дифференциальным уравнением. Цель оптимизации выхода формируется 
квадратичным критерием общего вида. Формальная постановка задачи 
приведена в~сле\-ду\-ющем разделе.
     
     Для решения задачи используется метод динамического 
программирования, решается уравнение Беллмана~[9]. Соответственно, 
в~результате получаются аналитические выражения и~для оптимального 
управ\-ле\-ния, и~для значения функционала качества. Технически 
традиционный, стандартный подход к~задаче обременен, пожалуй, 
единственной проблемой~--- поиском верного пред\-став\-ле\-ния структуры 
функции Беллмана. Справиться с~этой проблемой в~большей степени удается 
за счет результата, полученного при решении дискретного по времени 
аналога рассматриваемой постановки~\cite{8-bos}. Конечные соотношения 
для оптимального решения, как и~во всех подобных задачах, включая 
классическую ли\-ней\-но-квад\-ра\-тич\-ную, содержат решения 
определенных дифференциальных уравнений (обыкновенных и~в~частных 
производных). Вывод этих уравнений и~со\-став\-ля\-ет содержание первой час\-ти 
данной работы. Во второй части будет обсуждаться их приближенное 
чис\-лен\-ное решение и~компьютерные эксперименты.
     
     Кратко обозначим основные положения, при\-вле\-ка\-емые далее 
к~решению задачи, следуя в~основном обозначениям 
и~терминологии~\cite{9-bos}, а~именно: будем рассматривать задачу 
оптимального управления в~стохастической динамической сис\-те\-ме по полной 
информации, применяя метод динамического программирования. В~качестве 
целевого функционала, опре\-де\-ля\-юще\-го качество управ\-ле\-ния $U_0^T\hm= \{ 
u_t,\ 0\leq t\leq T\}$, выступает
     \begin{equation}
     J\left(U_0^T\right)={\sf E}\left\{ \int\limits_0^T L_t \left(x_t, u_t\right)\,dt+ 
l\left(x_T\right)\right\}\,.
     \label{e1-bos}
     \end{equation}
Здесь ${\sf E}\{\cdot\}$~--- оператор математического ожидания; $x_t$~--- 
случайный процесс, описываемый стохастическим дифференциальным 
уравнением Ито
     \begin{equation}
     dx_t=m_t\left( x_t, u_t\right) dt+ \sigma_t\left( x_t\right)dW_t\,,\enskip 
x_0=X\,,
     \label{e2-bos}
     \end{equation}
где $W_t$~--- стандартный винеровский процесс подходящей раз\-мер\-ности; 
$X$~--- случайный вектор.

     $U_0^T$ будем выбирать из класса допустимых неупреждающих (по 
отношению к~$W_t$) управлений~\cite{9-bos}. Соответственно, 
относительно функций сноса и~диффузии~$m_t$ и~$\sigma_t$  
в~(\ref{e2-bos}) будем предполагать выполненными ка\-кие-ли\-бо условия 
существования сильного решения для заданного до\-пус\-ти\-мо\-го управ\-ле\-ния. 
Например, для управ\-ле\-ния с~обратной связью $u_t\hm= u_t(x_t)$ будем 
считать, что $m_t(x,u_t(x))$ и~$\sigma_t(x)$ удовлетворяют условию 
линейного рос\-та и~локальному условию Липшица по~$x$ равномерно 
по~$t$ (т.\,е.\ условиям Ито).
     
     Для поиска оптимального управления, минимизирующего $J(U_0^T)$, 
рас\-смат\-ри\-ва\-ет\-ся функция Беллмана
     \begin{equation}
     V_t(x)=\left.\mathop{\mathrm{inf}}\limits_{U_t^T} {\sf E} \left\{ \int\limits_t^T 
L_t \left( x_t, u_t\right)\,dt+l\left( x_T\right) \right\vert \mathcal{F}_t^x\right\}\,,
     \label{e3-bos}
     \end{equation}
где $\mathcal{F}_t^x$~--- $\sigma$-ал\-геб\-ра, по\-рож\-ден\-ная~$x_\tau$, 
$0\hm\leq \tau\hm\leq t$, ${\sf E}\{\cdot\vert \mathcal{F}\}$~--- оператор условного 
математического ожидания относительно~$\mathcal{F}$. Соответственно, 
в~качестве достаточного условия оп\-ти\-маль\-ности воспользуемся уравнением 
динамического программирования
\begin{multline}
\fr{\partial V_t(x)}{\partial t} +\fr{1}{2}\sum\limits^n_{i,j=1} \sigma^2_{t_{ij}}
\fr{\partial^2 V_t(x)}{\partial x_i \partial x_j}+{}\\
{}+\min\limits_u\left[  
\sum\limits^n_{i=1} m_{t_i} \fr{\partial V_t(x)}{\partial x_i} + L_t(x,u)\right] 
=0\,,\\
V_T(x)=l(x)\,,
\label{e4-bos}
\end{multline}
где $m_{t_i}$~--- $i$-й элемент век\-тор-функ\-ции~$m_t(x,u)$; 
$\sigma^2_{t_{ij}} \hm= \sum\nolimits^m_{k=1} 
\sigma_{t_{ik}}\sigma_{t_{ki}}$, $\sigma_{t_{ij}}$~--- $i$-й по строке, $j$-й 
по столб\-цу элемент мат\-рич\-ной функции~$\sigma_t(x)$; $n$ и~$m$~--- 
размерности~$x_t$ и~$W_t$ соответственно.

     Традиционно в~рамках применения метода динамического 
программирования будем предполагать, что функции~$L_t$, $l$, $m_t$ 
и~$\sigma_t$ обеспечивают существование хотя бы одного решения 
уравнения~(\ref{e4-bos}), а~следовательно, и~оптимального 
управления~$u_t^*$, $0\hm\leq t\hm\leq T$, до\-став\-ля\-юще\-го минимум 
целевому функционалу~(\ref{e1-bos}). Задача оптимизации далее получается 
путем указания конкретных выражений для~$L_t$, $l$, $m_t$ и~$\sigma_t$.

\section{Постановка задачи управления выходом}

     Рассматриваемые далее случайные функции будут предполагаться 
скалярными. Такое упрощение позволит разгрузить выкладки и~итоговые 
выражения от не самых существенных деталей.
     
     Рассмотрим стохастическую дифференциальную сис\-те\-му, со\-сто\-яние 
которой представляет диффузи\-он\-ный процесс~$y_t$, описываемый 
нелинейным стохастическим дифференциальным уравнением Ито
     \begin{equation}
     dy_t=A_t\left( y_t\right) dt +\Sigma_t \left( y_t\right) dv_t\,,\enskip 
y_0=Y\,,
     \label{e5-bos}
     \end{equation}
где $v_t$~--- стандартный (одномерный) винеровский процесс; $Y$~--- 
случайная величина с~конечным вторым моментом; функции~$A_t$ 
и~$\Sigma_t$ удовлетворяют условиям Ито:
\begin{equation*}
\left\vert A_t(y)\right\vert +\left\vert \Sigma_t(y)\right\vert \leq C(1+\vert y\vert )\ 
\mbox{для\ всех } 0\leq t\leq T\,;
\end{equation*}

\vspace*{-12pt}

\noindent
\begin{multline*}
\hspace*{-2.10051pt}\left\vert A_t\left(y_1\right) -A_t \left( y_2\right) \right\vert +\left\vert 
\Sigma_t\left( y_1\right) -\Sigma_t \left(y_2\right)\right\vert \leq
C\left\vert y_1-y_2\right\vert\\
 \mbox{для\ всех\ } 0\leq t\leq T\ \mbox{и } 
y_1,y_2\in \mathbb{R}^1\,,
\end{multline*}
обеспечивающим существование единственного сильного (потраекторного) 
решения уравнения.
     
     Будем считать, что~$y_t$ описывает состояние некоторой 
динамической системы. Соответственно, поведение этой сис\-те\-мы опишем 
выходом, линейно связанным с~со\-сто\-янием:
     \begin{equation}
     dz_t=a_t y_t \,dt+ b_t z_t \,dt+ c_t u_t \,dt+\sigma_t \,dw_t\,,\enskip
     z_0=Z\,.
     \label{e6-bos}
     \end{equation}
Здесь $w_t$~--- не зависящий от~$v_t$, $Y$ и~$Z$ стандартный (одномерный) 
винеровский процесс; $Z$~--- случайная величина с~конечным вторым 
моментом; $u_t$~--- допустимое неупреждающее управ\-ле\-ние, качество 
которого определяется целевым функционалом следующего вида:
\begin{multline}
\!\hspace*{-3.98538pt}J\left( U_0^T\right) ={\sf E}\left\{ \int\limits_0^T \!\left( S_t\left( s_ty_t-g_t z_t -h_t 
u_t\right)^2 +G_t z_t^2+{}\right.\right.\\
\left.\left.{}+ H_t u_t^2
\vphantom{S_t\left( s_ty_t-g_t z_t -h_t 
u_t\right)^2}
\right) dt+S_T\left( s_T y_T -g_T 
z_T\right)^2+G_T z_T^2
\vphantom{\int\limits_0^T}\right\}\,,
\label{e7-bos}
\end{multline}
где $S_t$, $G_t$ и~$H_t$~--- неотрицательные функции\linebreak
$0\hm\leq t\hm\leq T$. 
Такой критерий отражает физический смысл задачи распределения ресурсов 
со\-глас\-но аналогичной~(\ref{e5-bos})--(\ref{e7-bos}) задаче для дис\-крет\-но\-го 
времени, рас\-смот\-рен\-ной в~\cite{7-bos}. В~част\-ности,  
функци\-онал~(\ref{e7-bos}) поз\-во\-ля\-ет ставить задачи отслеживания
 выходом 
со\-сто\-яния сис\-те\-мы, используя сла\-га\-емое $(y_t\hm- z_t)^2$, или 
управлением~--- линейной комбинации со\-сто\-яния и~выхода, сла\-га\-емое типа\linebreak 
$(y_t\hm+ z_t\hm- u_t)^2$. Поскольку задача формулируется 
в~предположении наличия пол\-ной информации о~со\-сто\-янии~$y_t$ 
и~выходе~$z_t$ (соответствующую $\sigma$-ал\-геб\-ру 
обозначим~$\mathcal{F}_t^{y,z}$), то допустимое управ\-ле\-ние ищется 
в~классе~$\mathcal{F}_t^{y,z}$-из\-ме\-ри\-мых неупреждающих функций 
(и,~как будет показано далее, оказывается управ\-ле\-ни\-ем с~обратной связью).

     Функции~$a_t$, $b_t$, $c_t$ и~$\sigma_t$ будем предполагать 
ограниченными: $\vert a_t\vert \hm+ \vert b_t\vert \hm+\vert c_t\vert \hm+ \vert 
\sigma_t \vert \hm\leq C$ для всех $0\hm\leq t\hm\leq T$, процесс  
управления~--- допустимым не\-упреж\-да\-ющим~\cite{9-bos}, обеспечивая, 
таким образом, существование сильного решения урав\-не\-ния~(\ref{e6-bos}) 
для любого допустимого управ\-ления.
     
     Задачу составляет поиск~$u_t^*$~--- допустимого управ\-ле\-ния, 
доставляющего минимум квад\-ра\-тич\-но\-му функционалу~$J(U_0^T)$.
      
     Поставленная задача очевидным образом формулируется в~терминах 
введенных выше в~(\ref{e1-bos})--(\ref{e3-bos}) обозначений, а~именно: 
     требуется обозначить
     \begin{gather*}
      x_t=\begin{pmatrix}
     y_t\\ z_t\end{pmatrix};\quad  m_t(x_t, u_t)=\begin{pmatrix}
     A_t(y_t)\\ a_t y_t +b_t z_t +c_t u_t\end{pmatrix};\\
     \sigma_t(x_t)= \begin{pmatrix}
     \Sigma_t(y_t)& 0\\
     0& \sigma_t\end{pmatrix};\quad W_t=\begin{pmatrix}
     v_t \\ w_t\end{pmatrix}
     %     \label{e8-bos}
     \end{gather*}
для записи уравнения со\-сто\-яния типа~(\ref{e2-bos}) и
\begin{align*}
L_t(x,u)&= L_t(y,z,u) ={}\\
&\hspace*{3mm}{}=S_t\left( s_t y-g_t z -h_t u\right)^2 +G_t z^2 +H_t  u^2\,;\\
l(x)&= l(y,z) =S_T \left( S_T y-g_T z\right)^2 +G_T z^2
%\label{e9-bos}
\end{align*}
для записи целевого функционала в~виде~(\ref{e1-bos}).

     Функция Беллмана~(\ref{e3-bos}) принимает вид 
     $V_t(x)\hm= V_t(y,z)$. Для записи со\-от\-вет\-ст\-ву\-юще\-го~(\ref{e4-bos}) 
уравнения Беллмана для~$V_t(y,z)$ заметим, что
     $$
     \left( \sigma^2_{t_{ij}}\right)_{i,j=1,2}= \begin{pmatrix}
     \Sigma_t^2(y) & 0\\
     0 & \sigma_t^2\end{pmatrix}\,.
     $$
     
     С~учетом перечисленных обозначений урав\-не\-ние динамического 
программирования~(\ref{e4-bos}) принимает вид:
     \begin{multline}
     \fr{\partial V_t(y,z)}{\partial t} +\fr{1}{2}\left( \Sigma_t^2(y) \fr{\partial^2 
V_t(y,z)} {\partial y^2}+\sigma_t^2\fr{\partial^2 V_t(y,z)} {\partial 
z^2}\right)+{}\\
    {}+\min\limits_u\! \left[ A_t(y) \fr{\partial V_t(y,z)}{\partial y}+\left( a_t 
y+b_t z+c_t u\right) \fr{\partial V_t(y,z)}{\partial z} +{}\right.\hspace*{-3pt}\\
\left.{}+ S_t\left( s_t y-g_t z-h_t 
u\right)^2+G_t z^2+H_t u^2
     \vphantom{\fr{\partial V_t(y,z)}{\partial y}}\right] =0\,,\\
     V_T(y,z)=S_T\left( s_T y-g_T z\right)^2+G_T z^2\,.
     \label{e10-bos}
     \end{multline}
     Это и~есть то самое уравнение, которое требуется решить: 
существование решения данного урав\-не\-ния суть достаточное условие 
оптимальности; оптимальное управ\-ле\-ние при этом~--- точ\-ка минимума 
со\-от\-вет\-ст\-ву\-юще\-го сла\-га\-емого.
     
\section{Динамическое программирование и~оптимальное 
управление}

     В рассматриваемой постановке линейность\linebreak выхода и~квадратичность 
критерия дают те же преимущества, что и~в~классической  
ли\-ней\-но-квад\-ра\-тич\-ной задаче управ\-ле\-ния~\cite{1-bos}, а~именно: 
позволяют сразу определить вид оптимального управ\-ле\-ния и~фактические 
условия его существования. Действительно, со\-хра\-няя в~(\ref{e10-bos}) под 
знаком $\min\nolimits_u$ только члены, зависящие от~$u$, получаем
     \begin{multline*}
     \fr{\partial V_t(y,z)}{\partial t} +\fr{1}{2}\left( \Sigma_t^2(y) \fr{\partial^2 
V_t(y,z)} {\partial y^2}+\sigma_t^2\fr{\partial^2 V_t(y,z)} {\partial 
z^2}\right)+{}\\
     {}+A_t(y)\fr{\partial V_t(y,z)}{\partial y}+\left( a_t y+b_t z\right) 
\fr{\partial V_t(y,z)}{\partial z}+{}\\
{}+S_t\left( s_t y-g_t z\right)^2 +G_t z^2+{}
\end{multline*}

\noindent
\begin{multline*}
     {}+\min\limits_u \left[ \left( c_t \fr{\partial V_t(y,z)}{\partial z}-2S_t \left( 
s_t y-g_t z\right) h_t\right)u +{}\right.\\
\left.{}+\left( S_t h_t^2+H_t\right) u^2
\vphantom{\fr{\partial V_t(y,z)}{\partial z}}
\right]=0\,,
     %\label{e11-bos}
     \end{multline*}
откуда в~предположении $S_t h_t^2\hm+ H_t\hm>0$ следует, что существует 
оптимальное управ\-ле\-ние, которое определяется равенством
\begin{multline}
u_t^* = u_t^*(y,z)=-\fr{1}{2}\left( S_t h_t^2 +H_t\right)^{-1} \left( c_t 
\fr{\partial V_t(y,z)}{\partial z}-{}\right.\\
\left.{}-2S_t\left( s_t y-g_t z\right) h_t
\vphantom{\fr{\partial V_t(y,z)}{\partial z}}
\right)
\label{e12-bos}
\end{multline}
и доставляет минимум соответствующему сла\-га\-емо\-му в~урав\-не\-нии Беллмана, 
равный
$-\left( S_t h_t^2\hm+\right.$\linebreak
$\left.{}+H_t\right)^{-1} \left( c_t 
{\partial V_t(y,z)}/{\partial 
z}\hm-2S_t\left( s_t y \hm-g_t z\right) h_t \right)^2/4.
$ 
     
     Отметим, что, как и~в~классической ли\-ней\-но-квад\-ра\-тич\-ной 
задаче, управ\-ле\-ние из класса до\-пус\-ти\-мых не\-упреж\-да\-ющих получилось 
управ\-ле\-ни\-ем с~обратной связью.
     
     Таким образом, функция Беллмана описывается сле\-ду\-ющим 
дифференциальным уравнением:
     \begin{multline}
     \fr{\partial V_t(y,z)}{\partial t} +\fr{1}{2}\left( \Sigma_t^2(y) \fr{\partial^2 
V_t(y,z)} {\partial y^2}+\sigma_t^2\fr{\partial^2 V_t(y,z)} {\partial 
z^2}\right)+{}\\
     {}+ A_t(y) \fr{\partial V_t(y,z)}{\partial y}+\left( a_t y+b_t z\right) 
\fr{\partial V_t(y,z)}{\partial z}+{}\\
{}+ S_t \left( s_t y- g_t z\right)^2 +G_t z^2-
 \fr{1}{4}\left( S_t h_t^2+H_t\right)^{-1}\times{}\\
 {}\times \left( c_t \fr{\partial V_t(y,z)} 
{\partial z}-2S_t\left( s_t y -g_t z\right) h_t \right)^2=0\,.
     \label{e13-bos}
     \end{multline}
     
     Возводя в~квадрат по\-след\-нее сла\-га\-емое в~(\ref{e13-bos}), перепишем 
его в~виде:
     \begin{multline}
     \fr{\partial V_t(y,z)}{\partial t} +\fr{1}{2}\left( \Sigma_t^2(y) \fr{\partial^2 
V_t(y,z)} {\partial y^2}+\sigma_t^2\fr{\partial^2 V_t(y,z)} {\partial 
z^2}\!\right)+{}\\
{}+A_t(y) \fr{\partial V_t(y,z)}{\partial y}
+ \left( 
\vphantom{\left( S_t h_t^2 +H_t\right)^{-1}}
a_t y+b_t z+{}\right.\\
\left.{}+\left( S_t h_t^2 +H_t\right)^{-1}
 c_t S_t \left( s_t y-g_t z\right) h_t
\right) 
     \fr{\partial V_t(y,z)}{\partial z}+{}\\
     {}+\left( S_t-\left( S_t h_t^2 +H_t\right)^{-1} S_t^2 h_t^2\right)\left( s_t y -
g_t z\right)^2+{}\\
     \!\!{}+
     G_t z^2 -\fr{1}{4}\left( S_t h_t^2+H_t\right)^{-1}\! c_t^2
     \left(\fr{\partial V_t(y,z)}{\partial z}\right)^{\!2}=0\,.\!\!
     \label{e14-bos}
     \end{multline}
     
     Рассматривая полученное уравнение, заметим, что его решение может 
быть пред\-став\-ле\-но в~виде:
   \begin{equation}
     V_t(y,z)= \alpha_t z^2+\beta_t(y) z +\gamma_t(y)\,,
     \label{e15-bos}
     \end{equation}
т.\,е.\ будем искать решение при дополнительном предположении 
о~квад\-ра\-тич\-ности функции Белл\-ма\-на по переменной~$z$, и~сведем, таким 
образом, поиск оптимального решения к~уравнениям относительно функций 
$\alpha_t$, $\beta_t(y)$ и~$\gamma_t(y)$. Отметим сразу, что явный вид 
функции~$\gamma_t(y)$ для реализации оптимального управ\-ле\-ния не 
требуется, однако далее будет предложен вариант вы\-чис\-ле\-ния и~этой 
функции, что пред\-став\-ля\-ет\-ся небесполезным, поскольку позволит выполнять 
расчет минимума целевого функционала. Источником для 
предложения~(\ref{e15-bos}) является уже упоминавшаяся аналогичная 
задача для случая дис\-крет\-но\-го времени~\cite{7-bos, 8-bos}. В~той задаче 
выражение для функции Беллмана получается формально без 
дополнительных усилий. При этом форма~(\ref{e15-bos}) обнаруживается 
как свойство оптимального решения. В~рассматриваемом случае 
непрерывного времени~(\ref{e15-bos}) постулируется, а~пра\-виль\-ность 
постулата под\-тверж\-да\-ет\-ся далее ре\-зуль\-ти\-ру\-ющи\-ми уравнениями 
для~$\alpha_t$, $\beta_t(y)$ и~$\gamma_t(y)$ Кроме того, данное 
предположение пред\-став\-ля\-ет\-ся вы\-те\-ка\-ющим из линейной структуры задачи 
в~отношении переменной~$z$, в~част\-ности, тем фактом, что такой вид 
функции Беллмана обеспечивает линейность оптимального 
управ\-ле\-ния~(\ref{e12-bos}) по~$z$.

     Граничное условие при выбранном предположении~(\ref{e15-bos}) 
принимает вид:

\noindent
     \begin{multline*}
     V_T(y,z)= S_T\left( s_T y- g_T z\right)^2+G_T z^2 ={}\\[-0.5pt]
     {}=\alpha_T z^2 
+\beta_T(y) z +\gamma_T(y)\,,
    \end{multline*}
т.\,е.

\noindent
\begin{align*}
\alpha_T&= S_T g_T^2 +G_T\,;\\[-0.5pt]
\beta_T(y)&=-2S_T s_T g_T y\,;\\[-0.5pt]
\gamma_T(y)&=S_T s_T^2 y^2\,.
%\label{e16-bos}
\end{align*}
          При этом само оптимальное управ\-ле\-ние, определенное 
выражением~(\ref{e12-bos}), оказывается управ\-ле\-ни\-ем с~обратной связью 
по~$y_t$ и~$z_t$:

\noindent
     \begin{multline}
     u_t^*=u_t^*(y,z) ={}\\[-0.5pt]
     {}=
     -\fr{1}{2}\left( S_t h_t^2 +H_t\right)^{-1}
     \left( c_t \left( 2\alpha_t z +\beta_t(y)\right) +{}\right.\\[-0.5pt]
    \left. {}+2S_t\left( s_t y-g_t z\right) 
h_t\right)\,.
     \label{e17-bos}
     \end{multline}
          Подставляем $V_t(y,z)\hm= \alpha_t z^2 \hm+ \beta_t(y) 
z\hm+\gamma_t(y)$ в~(\ref{e14-bos}):

\noindent
     \begin{multline*}
     \fr{\partial \alpha_t}{\partial t}\, z^2 +
     \fr{\partial \beta_t(y)}{\partial t}\,z +
     \fr{\partial \gamma_t(y)}{\partial t}+{}\\[-0.5pt]
     {}+\fr{1}{2}\left( \Sigma_t^2(y) \left( 
\fr{\partial^2\beta_t(y)}{\partial y^2}\,z +\fr{\partial^2 \gamma_t(y)}{\partial 
y^2}\right) +2\sigma_t^2\alpha_t\right)+{}\\[-0.5pt]
 {}+A_t(y)\left(\fr{\partial \beta_t(y)}{\partial y}\,z + \fr{\partial 
\gamma_t(y)}{\partial y}\right) +{}\\[-0.5pt]
\hspace*{-0.22987pt}{}+\left( a_t y+b_t z+\left( S_t h_t^2 +H_t\right)^{-1} c_t S_t \left( s_t y-
g_t z\right) h_t\right)\times{}
\end{multline*}

\noindent
\begin{multline*}
         {}\times \left( 2\alpha_t z+\beta_t(y)\right)+{}\\
     {}+\left( S_t-\left( S_t h_t^2 +H_t\right)^{-1} S_t^2 h_t^2\right)\left( s_t y-
g_t z\right)^2+{}\\
     {}+ G_t z^2 -\fr{1}{4}\left( S_t h_t^2 +H_t\right)^{-1} c_t^2 \left( 
2\alpha_t z+\beta_t(y)\right)^2=0\,.
     \end{multline*}
          Далее выделяем слагаемые при~$z^2$, $z$ и~$z^0$
          
          \noindent
     \begin{multline*}
     \fr{\partial \alpha_t}{\partial t}\, z^2 +\fr{\partial \beta_t(y)}{\partial t}\,z +
     \fr{\partial \gamma_t(y)}{\partial 
t}+\fr{1}{2}\,\Sigma_t^2(y)\fr{\partial^2\beta_t(y)}{\partial y^2}\,z+ {}\\
{}+
\fr{1}{2}\,\Sigma_t^2(y)\fr{\partial^2\gamma_t(y)}{\partial 
y^2}+\sigma_t^2\alpha_t+A_t(y)\fr{\partial \beta_t(y)}{\partial y}\,z +{}\\
{}+A_t(y) \fr{\partial 
\gamma_t(y)}{\partial y}+{}\\
{}+ 2\alpha_t \left( b_t -\left( S_t h_t^2+H_t\right)^{-1} c_t 
S_t h_t g_t \right)z^2+{}\\
     {}+
     \left( 2\alpha_t\left( \alpha_t+\left( S_t h_t^2+H_t\right)^{-1} c_t S_t h_t 
s_t\right)y +{}\right.\\
\left.{}+\beta_t(y) \left( b_t-\left( S_t h_t^2+H_t\right)^{-1} c_t S_t h_t 
g_t\right) \right) z+{}\\
     {}+\beta_t(y)\left( a_t +\left( S_t h_t^2+H_t\right)^{-1} c_t S_t h_t s_t\right) 
y+{}\\
{}+ \left( S_t -\left( S_t h_t^2+H_t\right)^{-1} S_t^2 h_t^2\right) g_t^2 z^2-{}\\
     {}- 2\left( S_t -\left( S_t h_t^2+H_t\right)^{-1} S_t^2 h_t^2\right) s_t g_t yz 
+{}\\
{}+
     \left( S_t-\left( S_t h_t^2+H_t\right)^{-1} S_t^2 h_t^2\right) s_t^2 y^2+{}\\
     {}+G_t z^2 -\left( S_t h_t^2 +H_t\right)^{-1} c_t^2 \alpha_t^2 z^2 -{}\\
     {}-\left( 
S_t h_t^2+H_t\right)^{-1} c_t^2 \alpha_t \beta_t(y) z-{}\\
{}-
\fr{1}{4}\left( S_t h_t^2+H_t\right)^{-1}  c_t^2 \beta_t^2(y)=0\,,
     \end{multline*}
группируем их и~получаем сле\-ду\-ющие уравнения:
\begin{itemize}
\item  для~$\alpha_t$:

\noindent
\begin{multline}
\fr{\partial\alpha_t}{\partial t}+2\alpha_t\left( b_t-\left( S_t h_t^2+H_t\right)^{-1} c_t 
S_t h_t g_t\right)+{}\\
{}+ \left( S_t- \left( S_t h_t^2+H_t\right)^{-1} S_t^2 h_t^2\right) 
g_t^2+G_t-{}\\
\hspace*{-8mm}{}-\left( S_t h_t^2+H_t\right)^{-1} c_t^2 \alpha_t^2 =0\,,\enskip \alpha_T=S_T 
g_t^2+G_T\,;\!\!
\label{e18-bos}
\end{multline}
\item для $\beta_t$:

\noindent
\begin{multline}
\fr{\partial\beta_t(y)}{\partial 
t}+\fr{1}{2}\,\Sigma_t^2(y)\fr{\partial^2\beta_t(y)}{\partial y^2} 
+A_t(y)\fr{\partial \beta_t(y)}{\partial y}+{}\\
{}+ 2\alpha_t\left( a_t +\left( S_t h_t^2+H_t\right)^{-1} c_t S_t h_t s_t\right) y+{}\\
{}+
\beta_t(y)\left( b_t -\left( S_t h_t^2 +H_t\right)^{-1} c_t S_t h_t g_t\right)-{}\\
{}-2\left( S_t-\left( S_t h_t^2+H_t\right)^{-1} S_t^2 h_t^2\right) s_t g_t y-{}
\\
{}-
\left( S_t h_t^2+H_t\right)^{-1} c_t^2 \alpha_t \beta_t(y)=0\,,\\
\beta_T(y)=-2S_T s_T g_T y\,;
\label{e19-bos}
\end{multline}
\item  для $\gamma_t$:
\begin{multline}
\hspace*{-0.8pt}\fr{\partial \gamma_t(y)}{\partial t}+\fr{1}{2}\,\Sigma_t^2(y)
\fr{\partial^2 \gamma_t(y)}{\partial y^2} +\sigma_t^2 \alpha_t +A_t(y)
\fr{\partial \gamma_t(y)}{\partial y}+{}\\
{}+ \beta_t(y)\left( a_t +\left( S_t h_t^2+H_t\right)^{-1} c_t S_t h_t s_t\right) y+{}\\
{}+
\left( S_t-\left( S_t h_t^2+H_t\right)^{-1} S_t^2 h_t^2\right)  s_t^2 y^2-{}\\
{}-\fr{1}{4}\left( S_t h_t^2+H_t\right)^{-1} c_t^2 \beta_t^2(y) =0\,,\\
\gamma_T(y)=S_T s_T^2 y^2\,.
\label{e20-bos}
\end{multline}
\end{itemize}
     
     Уравнение~(\ref{e18-bos}), легко заметить, является уравнением 
Риккати, которое в~силу сформулированного выше условия   
имеет единственное неотрицательное решение для всех $0\hm\leq t\hm\leq T$. 
Этот факт требует дополнительного комментария. Для получения 
уравнения~(\ref{e18-bos}) рас\-смот\-рим исходную задачу при дополнительных 
условиях $a_t\hm=0$ и~$s_t\hm=0$ для всех $0\hm\leq t\hm\leq T$. Нетрудно 
видеть, что эти условия рассматриваемую по\-ста\-нов\-ку сводят фактически 
к~классической ли\-ней\-но-квад\-ра\-тич\-ной задаче. Имеющуюся 
в~рассматриваемой формулировке чуть более общую форму целевой 
функции (принципиального значения это обобщение, конечно, не имеет) 
сведем к~классической еще одним предположением: $S_t\hm=0$ для всех 
$0\hm\leq t\hm\leq T$. Теперь уравнение для~$\alpha_t$ принимает хорошо 
известный вид:
     \begin{equation}
     \fr{\partial \alpha_t}{\partial t}+2\alpha_t b_t +G_t- H_t^{-1} c_t^2 
\alpha_t^2=0\,,\enskip \alpha_T=G_T\,.
     \label{e21-bos}
     \end{equation}

     В таком случае, как известно~\cite{10-bos}, существует единственное 
оптимальное управление~--- линейное с~обратной связью по выходу~$z_t$, 
с~коэффициентом усиления, опи\-сы\-ва\-емым уравнением  
Риккати~(\ref{e21-bos}). Именно этот результат дают  
уравнения~(\ref{e18-bos})--(\ref{e20-bos}) и~описываемая ими функция 
Беллмана~(\ref{e15-bos}), так как из $a_t\hm=0$ и~$s_t\hm=0$ немедленно 
следует, что $\beta_t(y)\hm=0$, откуда, в~свою очередь, с~учетом 
не\-за\-ви\-си\-мости решения от~$y_t$ следует, что $\gamma_t(y)\hm=\gamma_t$, 
т.\,е.\ не зависит от~$y$ и~задается уравнением: 
     $$
     \fr{\partial \gamma_t(y)}{\partial t} +\sigma^2_t \alpha_t=0\,,\enskip 
\gamma_T=0\,.
     $$ 
     Оптимальное управ\-ле\-ние при этом 
     $$
     u_t^*= -H_t^{-1} c_t \alpha_t z_t\,,
     $$
      т.\,е.\ все полностью совпадает с~известным классическим решением.
     
     С уравнениями~(\ref{e19-bos}) и~(\ref{e20-bos}) ситуация, естественно, 
обстоит сложнее. Это линейные уравнения второго порядка параболического 
типа, поскольку\linebreak
 $\Sigma_t^2(y)\hm>0$. Фактически отсутствуют 
конструктивные условия, гарантирующие существование их\linebreak
 решений 
(требовать, чтобы все фигурирующие в~уравнениях коэффициенты были 
представлены аналитическими функциями на всем пространстве значений, 
вряд ли целесообразно), поэтому далее будем предполагать, что данные 
уравнения имеют на рас\-смат\-ри\-ва\-емом интервале $0\hm\leq t\hm\leq T$ хотя 
бы одно ограниченное решение и~именно эти условия будем рас\-смат\-ри\-вать 
как достаточные условия существования оптимального решения 
рассматриваемой задачи.
     
     Таким образом, доказана следующая тео\-рема.
     
     \smallskip
     
     \noindent
     \textbf{Теорема.}\ \textit{Пусть для диффузионного 
процесса}~(\ref{e5-bos}) \textit{выполнены условия Ито, для 
     процесса}~(\ref{e6-bos})~--- \textit{ограничены коэффициенты, 
уравнения}~(\ref{e18-bos})--(\ref{e20-bos}) \textit{имеют ограниченные 
решения для $0\hm\leq t\hm\leq T$. Тогда минимум  
функционалу}~(\ref{e7-bos}) \textit{доставляет оптимальное 
управ\-ле\-ние}~(\ref{e17-bos}), \textit{где} $y\hm= y_t$; $z\hm=z_t$.
     
\section{Заключение}

     Рассмотренная задача оптимизации в~целом близка и~по модели, и~по 
критерию к~классической ли\-ней\-но-квад\-ра\-тич\-ной постановке. 
Принципиальным отличием является нелинейная модель для описания 
со\-сто\-яния динамической сис\-те\-мы, в~которой отсутствует управ\-ля\-ющее 
воздействие.\linebreak
 Такую модель наряду с~традиционной интер\-пре\-тацией  
<<со\-сто\-яние--вы\-ход>> мож\-но понимать как\linebreak модель неконтролируемого 
слож\-но\-го внешнего воздействия. Небольшое дополнительное отличие дает 
предложенная форма квад\-ра\-тич\-но\-го критерия, поз\-во\-ля\-ющая, в~част\-ности, 
ставить такие задачи, как отслеживание выходом или управ\-ле\-ни\-ем со\-сто\-яния 
сис\-те\-мы или ее выхода.
     
     Поскольку обсуждать возможности точного решения уравнений, 
определяющих оптимальное управ\-ле\-ние, не имеет смыс\-ла, наиболее 
актуальной далее является задача их приближенного чис\-лен\-но\-го решения 
и~анализа воз\-мож\-ности практической реализации. Этому посвящена вторая 
часть данной работы, пла\-ни\-ру\-емая к~выходу в~ближайшее время.

{\small\frenchspacing
 {%\baselineskip=10.8pt
 \addcontentsline{toc}{section}{References}
 \begin{thebibliography}{99}
\bibitem{1-bos}
\Au{Athans M.} Editorial on the LQG problem~// IEEE~T. Automat. Contr., 1971. Vol.~16. 
No.\,6. P.~528--552. doi: 10.1109/TAC.1971.1099845.
\bibitem{2-bos}
\Au{Wu Z.} Forward-backward stochastic differential equations, linear quadratic stochastic 
optimal control and nonzero sum differential games~// J.~Syst. Sci. Complex., 2005. Vol.~18. 
No.\,2. P.~179--192.
\bibitem{3-bos}
\Au{Chen B.\,S., Zhang~W.} Stochastic H2/H1 control with state-dependent noise~// IEEE 
T.~Automat. Contr., 2004. Vol.~49. No.\,1. P.~45--56. doi: 10.1109/TAC.2003.821400.
\bibitem{4-bos}
\Au{Bohacek S.} A~stochastic model of TCP and fair video transmission~// IEEE 
INFOCOM, 2003. Vol.~2. P.~1134--1144. doi: 10.1109/INFCOM.2003.1208950.
\bibitem{5-bos}
\Au{Домбровский В.\,В., Объедко~Т.\,Ю.} Управление с~прогнозированием системами 
с~марковскими скачками при ограничениях и~применение к~оптимизации 
инвестиционного портфеля~// Автомат. телемех., 2011. №\,5. С.~96--112. doi: 
10.1134/S0005117911050079.
\bibitem{6-bos}
\Au{Баландин Д.\,В., Коган~М.\,М.} Оптимальное линейно-квад\-ра\-тич\-ное управление: от 
матричных уравнений к~линейным матричным неравенствам~// Автомат. телемех., 2011. 
№\,11. С.~60--69. doi: 10.1134/ S0005117911110038.
\bibitem{7-bos}
\Au{Босов А.\,В.} Обобщенная задача распределения ресурсов программной системы~// 
Информатика и~её применения, 2014. Т.~8. Вып.~2. С.~39--47. doi: 
10.14357/19922264140204.
\bibitem{8-bos}
\Au{Босов А.\,В.} Управление линейным выходом дискретной стохастической системы по 
квадратичному критерию~// Изв. РАН. Теория и~системы управления, 2016. №\,3.  
С.~19--35. doi: 10.1134/S1064230716030060.
\bibitem{9-bos}
\Au{Флеминг У., Ришел~Р.} Оптимальное управление детерминированными 
и~стохастическими системами~/ Пер. с~англ.~--- М.: Мир, 1978. 316~с. 
(\Au{Fleming~W.\,H., Rishel~R.\,W.} Deterministic and stochastic optimal control.~--- New 
York, NY, USA: Springer-Verlag, 1975. 222~p.)
\bibitem{10-bos}
\Au{Девис М.\,Х.\,А.} Линейное оценивание и~стохастическое управление~/ Пер. с~англ.~--- 
М.: Наука, 1984. 206~с. (\Au{Davis~M.\,H.\,A.} Linear estimation and stochastic control.~--- 
London: Chapman and Hall, 1977. 224~p.)

 \end{thebibliography}

 }
 }

\end{multicols}

\vspace*{-6pt}

\hfill{\small\textit{Поступила в~редакцию 30.03.18}}

\vspace*{4pt}

%\newpage

%\vspace*{-24pt}

\hrule

\vspace*{2pt}

\hrule

\vspace*{-2pt}


\def\tit{STOCHASTIC DIFFERENTIAL SYSTEM OUTPUT CONTROL 
BY~THE~QUADRATIC CRITERION.~I.~DYNAMIC\\ PROGRAMMING 
OPTIMAL SOLUTION}


\def\titkol{Stochastic differential system output control 
by~the~quadratic criterion. I.~Dynamic programming 
optimal solution}

\def\aut{A.\,V.~Bosov and~A.\,I.~Stefanovich}

\def\autkol{A.\,V.~Bosov and~A.\,I.~Stefanovich}

\titel{\tit}{\aut}{\autkol}{\titkol}

\vspace*{-11pt}


\noindent
Institute of Informatics Problems, Federal Research Center ``Computer Science 
and Control'' of the Russian Academy of Sciences, 44-2~Vavilov Str., Moscow 
119333, Russian Federation


\def\leftfootline{\small{\textbf{\thepage}
\hfill INFORMATIKA I EE PRIMENENIYA~--- INFORMATICS AND
APPLICATIONS\ \ \ 2018\ \ \ volume~12\ \ \ issue\ 3}
}%
 \def\rightfootline{\small{INFORMATIKA I EE PRIMENENIYA~---
INFORMATICS AND APPLICATIONS\ \ \ 2018\ \ \ volume~12\ \ \ issue\ 3
\hfill \textbf{\thepage}}}

\vspace*{3pt}



\Abste{The problem of optimal control for the Ito diffusion 
process and a~controlled linear output is solved. The considered 
statement is close to the classical linear-quadratic Gaussian 
control  (LQG control) problem. Differences consist in the fact 
that the state is described by the nonlinear differential Ito equation  $dy_y = A_t(y_t) 
\,dt+\Sigma_t(y_t)\,dv_t$ and does not depend on the control~$u_t$, 
optimization subject is controlled linear output 
 $dz_t=a_ty_t\,dt +b_tz_t\,dt +c_t u_t\,dt +\sigma_t \,dw_t$. 
Additional generalizations are included in the quadratic 
quality criterion for the purpose of statement such problems 
as state tracking by output or a linear combination of state 
and output tracking by control. The method of dynamic programming 
is used for the solution. 
The assumption about Bellman function in the form  $V_t(y,z)= \alpha_t 
z^2+\beta_t(y) z+\gamma_t(y)$ allows one to find it. 
Three differential equations for the coefficients $\alpha_t$,  $\beta_t(y)$,
and $\gamma_t(y)$ give the solution. 
These equations constitute the optimal solution of the problem under consideration.}

\KWE{stochastic differential equation; optimal control; dynamic programming; 
Bellman function; Riccati equation; linear differential equations of parabolic type}


\DOI{10.14357/19922264180314}

\vspace*{-12pt}

\Ack
\noindent
This work was partially supported by the Russian Science Foundation (grant  
16-07-00677).



%\vspace*{6pt}

  \begin{multicols}{2}

\renewcommand{\bibname}{\protect\rmfamily References}
%\renewcommand{\bibname}{\large\protect\rm References}

{\small\frenchspacing
 {%\baselineskip=10.8pt
 \addcontentsline{toc}{section}{References}
 \begin{thebibliography}{99}
\bibitem{1-bos-1}
\Aue{Athans, M.} 1971. Editorial on the LQG problem. \textit{IEEE~T. 
Automat. Contr.} 16(6):528--552. doi: 10.1109/ TAC.1971.1099845.
\bibitem{2-bos-1}
\Aue{Wu, Z.} 2005. Forward-backward stochastic differential equations, linear 
quadratic stochastic optimal control and\linebreak\vspace*{-12pt}

\columnbreak

\noindent
 nonzero sum differential games. 
\textit{J.~Syst. Sci. Complex.} 18(2):179--192.
\bibitem{3-bos-1}
\Aue{Chen, B.\,S. and W.~Zhang.} 2004. Stochastic H2/H1 control with  
state-dependent noise. \textit{IEEE~T. Automat. Contr.} 49(1):45--56.
doi: 10.1109/TAC.2003.821400.
\bibitem{4-bos-1}
\Aue{Bohacek, S.} 2003. A~stochastic model of TCP and fair video 
transmission. \textit{IEEE INFOCOM}. 2:1134--1144.
doi: 10.1109/INFCOM.2003.1208950.
\bibitem{5-bos-1}
\Aue{Dombrovskii, V.\,V., and T.\,Yu.~Ob''edko.} 2011. Predictive control of 
systems with Markovian jumps under constraints and its application to the 
investment portfolio optimization. \textit{Automat. Rem. Contr.}  
72(5):989--1003.
\bibitem{6-bos-1}
\Aue{Balandin, D.\,V., and M.\,M.~Kogan.} 2011. Optimal linear-quadratic 
control: From matrix equations to linear matrix inequalities. \textit{Automat. 
Rem. Contr.} 72(11):2276--2284.
\bibitem{7-bos-1}
\Aue{Bosov, A.\,V.} 2014. Obobshchennaya zadacha raspredeleniya resursov 
programmnoy sistemy [The generalized problem of software system resources 
distribution]. \textit{Informatika i~ee Primeneniya~--- Inform. Appl.}  
8(2):39--47. doi: 
10.14357/19922264140204.
\bibitem{8-bos-1}
\Aue{Bosov, A.\,V.} 2016. Discrete stochastic system linear output control 
with respect to a quadratic criterion. \textit{J.~Comput. Syst. Sc. 
Int.} 55(3):349--364.
\bibitem{9-bos-1}
\Aue{Fleming, W.\,H., and R.\,W.~Rishel.} 1975. \textit{Deterministic and 
stochastic optimal control.} New York, NY: Springer-Verlag. 222~p.
\bibitem{10-bos-1}
\Aue{Davis, M.\,H.\,A.} 1977. \textit{Linear estimation and stochastic 
control.} London: Chapman and Hall. 224~p.
\end{thebibliography}

 }
 }

\end{multicols}

\vspace*{-6pt}

\hfill{\small\textit{Received March 30, 2018}}

%\pagebreak

%\vspace*{-18pt}
     
     \Contr
     
       \noindent
       \textbf{Bosov Alexey V.} (b.\ 1969)~--- Doctor of Science in technology, 
principal scientist, Institute of Informatics Problems, Federal Research 
Center ``Computer Science and Control'' of the Russian Academy of Sciences, 
44-2~Vavilov Str., Moscow 119333, Russian Federation; 
\mbox{AVBosov@ipiran.ru}
       
       \vspace*{3pt}
       
       \noindent
       \textbf{Stefanovich Alexey I.} (b.\ 1983)~--- principal specialist, 
Institute of Informatics Problems, Federal Research Center ``Computer Science 
and Control'' of the Russian Academy of Sciences, 44-2~Vavilov Str., Moscow 
119333, Russian Federation; \mbox{AStefanovich@frccsc.ru}
\label{end\stat}

\renewcommand{\bibname}{\protect\rm Литература}       

       %10-
\def\stat{zatsman}

\def\tit{ТРАНСФОРМАЦИИ ОБЪЕКТОВ ПЕРВОГО И~ВТОРОГО ПОРЯДКА 
В~ЛЕКСИКОГРАФИЧЕСКОЙ ИНФОРМАЦИОННОЙ СИСТЕМЕ$^*$}

\def\titkol{Трансформации объектов первого и~второго порядка 
в~лексикографической информационной системе}

\def\aut{И.\,М.~Зацман$^1$}

\def\autkol{И.\,М.~Зацман}

\titel{\tit}{\aut}{\autkol}{\titkol}

\index{Зацман И.\,М.}
\index{Zatsman I.\,M.}


{\renewcommand{\thefootnote}{\fnsymbol{footnote}} \footnotetext[1]
{Исследование выполнено в~ФИЦ ИУ РАН за счет гранта Российского научного фонда №\,24-18-00155, {\sf 
https://rscf.ru/project/24-18-00155}. Работа выполнялась с~использованием инфраструктуры Центра 
коллективного пользования <<Высокопроизводительные вычисления и~большие данные>> (ЦКП 
<<Информатика>>) ФИЦ ИУ РАН (г.\ Москва).}}


\renewcommand{\thefootnote}{\arabic{footnote}}
\footnotetext[1]{ Федеральный исследовательский центр <<Информатика и~управление>> Российской академии наук, 
\mbox{izatsman@yandex.ru}}

\vspace*{-12pt}


  
  \Abst{Рассматриваются теоретические основания проектирования информационных 
технологий (ИТ) интеграции двуязычных словарей и~параллельных корпусов. Дано описание 
первых результатов создания третьего уровня классификации трансформаций объектов 
предметной области информатики, которую предполагается использовать при создании 
концепции лексикографической информационной системы, обеспечивающей интеграцию. 
Все сущности информатики в~статье разделены на два глобальных класса: объекты и~их 
трансформации. Для каждого такого класса конструируется своя классификация. Ранее были 
описаны два верхних уровня классификации трансформаций объектов предметной области. 
В~данной статье рассматривается третий уровень этой классификации. Основанием для 
построения самого верхнего ее уровня служило деление предметной области информатики 
на среды (ментальная, сенсорно воспринимаемая, цифровая и~ряд других сред), каждая из 
которых по определению включает объекты одной природы. Основанием для построения 
второго уровня классификации трансформаций объектов служила типология знаковых  
сис\-тем А.~Соломоника. Цель статьи состоит в~систематизации трансформаций первого 
и~второго порядка объектов предметной области на третьем уровне этой классификации. 
Основанием для систематизации служит средовая версия иерархии Акоффа.}
  
  \KW{объекты предметной области; трансформации объектов; классификация; данные; 
информация; знание; лексикографическая информационная сис\-тема}

\DOI{10.14357/19922264240211}{VZTGVV}
  
\vspace*{3pt}


\vskip 10pt plus 9pt minus 6pt

\thispagestyle{headings}

\begin{multicols}{2}

\label{st\stat}
  
\section{Введение}

\vspace*{-9pt}

  Возникновение параллельных корпусов, в~которых предложениям 
оригинального текста со\-по\-став\-ле\-ны предложения его перевода, обеспечило 
возможность контрастивного лингвистического\linebreak \mbox{анализа} на принципиально 
новом уровне полноты и~точности, недостижимом в~докорпусную эпоху. 
Пионерскими в~этой области стали работы \mbox{1990-х~гг}. Стига Йоханссона  
с~анг\-ло-нор\-веж\-ским корпусом~[1]. В России параллельные корпусы стали 
формироваться в~начале XXI~века в~рамках Национального корпуса русского 
языка~[2].
  
  Создатели двуязычных словарей используют параллельные корпусы для 
сбора материала и~эмпирической проверки своих гипотез, касающихся 
межъязы\-ко\-вой эквивалентности. Ценность параллельных корпусов 
определяется тем, что в~лингвистике этап сбора исходного материала считается 
наиболее трудоемким и~наименее творческим, а~параллельные корпусы 
позволяют значительно сэкономить время и~силы для творческого этапа 
создания словарей~[3].
 % 
  При этом двуязычные словари, создаваемые на основе исходного материала, 
извлеченного из параллельных корпусов, сейчас формируются без связей с~их 
текстами. Другими словами, онлайновые связи созданных словарей 
с~параллельными корпусами, которые служили источниками исходного 
материала, отсутствуют. 

Параллельные корпусы постоянно пополняются 
новыми текстами, в~предложениях которых можно обнаружить новые значения 
слов и~устойчивых словосочетаний. Однако при этом отсутствуют методы 
и~средства оперативного обновления словарей по корпусным данным. 
В~настоящее время проблема установления связей между двуязычными 
словарями и~параллельными корпусами (далее~--- проблема интеграции) 
находится на стадии поиска концептуальных подходов к~их интеграции на 
уровне значений.
  
  Подход к~решению проблемы интеграции, предлагаемый в~статье, учитывает 
  и~появление новых значений слов и~устойчивых словосочетаний, и~динамику 
смысловых значений, которая обусловлена развитием и~пополнением знания 
лингвистов, фиксирующих эти значения в~результате семантического анализа 
пополняемых корпусных данных. Проведенные эксперименты показали, что 
обнаружение нового лингвистического знания обусловливает и~формирование 
дефиниций новых значений, и~пересмотр уже существующих дефиниций~[4, 5].
  
  Например, в~проведенных экспериментах с~использованием ЦКП 
<<Информатика>> ФИЦ ИУ РАН фиксировалась эволюция значений немецких 
модальных глаголов, исходное состояние значений которых было описано 
в~не\-мец\-ко-рус\-ском словаре. В~экспериментальном массиве текстов как 
потенциальных источниках нового знания 16\,268 предложений содержали 
немецкие модальные глаголы и~в~2041 из них встречался глагол sollen. 
В~начале эксперимента в~словаре были описаны~12~значений этого модального 
глагола. По окончании эксперимента лингвисты обнаружили два новых его 
значения, согласовали их дефиниции и~описали эволюцию дефиниций~[6, 7].
  
  Таким образом, для решения проблемы интеграции требуется фиксировать 
новое знание, обнаруженное лингвистами в~текстовых данных параллельных 
корпусов, отслеживать эволюцию знания, представленного в~виде дефиниций 
значений слов и~устойчивых словосочетаний, и,~соответственно, 
актуализировать электронные двуязычные словари. Предлагаемый 
концептуальный подход к~интеграции, который планируется реализовать 
в~процессе проектирования лексикографической информационной сис\-те\-мы, 
фиксирующей эволюцию лингвистического знания, основан на решении 
следующих задач:\\[-14pt]
  \begin{itemize}
  \item категоризация трех базовых понятий информатики, включенных 
  в~иерархию Акоффа~[8] (данные, информация, знание), на объекты 
проектируемой сис\-те\-мы, которая необходима, чтобы фиксировать 
<<кванты>> нового знания и~отслеживать его эволюцию в~этой сис\-теме;\\[-15pt]
  \item  систематизация трансформаций объектов этой сис\-темы.\\[-14pt]
  \end{itemize}
  
  Цель статьи и~состоит в~решении двух задач: категоризации трех базовых 
понятий информатики на объекты лексикографической информационной  
сис\-те\-мы и~сис\-те\-ма\-ти\-за\-ции трансформаций первого и~второго порядка 
ее объектов.
  
  Трансформациями первого порядка, о которых сказано в~формулировке цели 
статьи, называются взаимные преобразования между двумя объектами  
сис\-те\-мы одной природы. Например, перевод в~сис\-те\-ме текста с~русского 
языка на английский относится к~ним. Трансформациями второго порядка 
и~выше называются взаимные преобразования между двумя и~более объектами 
разной природы. Например, кодирование символов текс\-та компьютерными 
кодами и~их декодирование относятся по определению к~трансформациям 
второго порядка.

%\vspace*{-9pt}
  
\section{Процессы трансформаций в~информатике}

%\vspace*{-3pt}

Процессы трансформаций, рассматриваемые в~статье, относятся к~теоретическому ядру информатики, а~не 
только к~проектированию лексикографической информационной сис\-те\-мы. Например, из трех основных 
подходов к~описанию предметной об\-ласти информатики\footnote{В статье предметная область информатики 
трактуется согласно концепции полиадического компьютинга Пола Розенблума~\cite{9-zac}.} (объектный, 
трансформационный и~синтетический) сис\-те\-ма\-ти\-за\-ция трансформаций ближе всего ко второму 
подходу. Примерами первого подхода, в~рамках которого основное внимание уделяется объектам предметной 
области информатики и~в~меньшей степени отношениям\linebreak между ними, могут служить  
работы~\cite{8-zac, 10-zac, 11-zac}; \mbox{примерами} второго подхода, в~рамках которого основное внимание 
уделяется трансформациям и~в~меньшей степени трансформируемым объектам,~---  
работы~\cite{12-zac, 13-zac}; примерами третьего, синтетического подхода, в~котором уделяется внимание 
и~объектам предметной об\-ласти информатики, и~отношениям между ними, могут служить работы~\cite{14-zac, 
15-zac, 16-zac, 17-zac, 18-zac}.

  Таким образом, для описания трансформаций объектов лексикографической 
информационной\linebreak системы предпочтительнее всего трансформационный 
подход, который упоминается и~в определениях информатики. Например, 
в~2009~г.\ П.~Деннинг и~П.~Розенблум сформулировали суть \mbox{информатики} как 
компьютинга следующим образом: <<$\ldots$информатика~--- это не просто 
алгоритмы и~структуры данных; это преобразования [трансформации] 
представлений>>~\cite{12-zac}. Чуть позже, в~контексте краткого описания 
парадигмы информатики как компьютинга, П.~Деннинг и~П.~Фриман изменили 
эту формулировку на такую: <<Центральный объект внимания в~информатике 
можно определить как информационные процессы~--- \textit{естественные или 
искусственные процессы, преобразующие информацию} (курсив мой~--- 
И.\,З.)>>~\cite{13-zac}. Согласно парадигме, предлагаемой авторами этой 
статьи, на начальном этапе проектирования автоматизированных систем 
базовыми элементами моделей их функционирования служат 
\textit{информационные про\-цессы}.
  
  Однако если 15~лет назад в~формулировке из работы~\cite{13-zac} шла речь 
о~процессах, преобразующих информацию, то в~последние~10~лет в~спектр 
процессов трансформаций все чаще стали включать процессы, преобразующие 
не только информацию, но также и~другие объекты автоматизированных 
систем, в~первую очередь данные и~знания~[19--21]. Например, Виктория 
Стодден, позиционируя науку о~данных как одну из дисциплин информатики, 
говорит, что центральный объект исследований в~науке о~данных~--- это 
<<изучение обобщаемого извлечения знания из данных>>~\cite{21-zac}. 
Увеличение и~чис\-ла объектов, и~спект\-ра процессов их трансформаций 
в~автоматизированных сис\-те\-мах обуслов\-ли\-ва\-ет не\-об\-хо\-ди\-мость 
систематизации и~объектов, и~процессов их трансформаций на начальном этапе 
проектирования сис\-тем.
  
  Для создания концепции лексикографической информационной сис\-те\-мы 
и~проектирования ИТ, обеспечивающих интеграцию 
двуязычных словарей и~параллельных корпусов, сначала выполним 
категоризацию на объекты этой сис\-те\-мы трех базовых понятий информатики 
(данные, информация, знание) в~контексте построения классификаций 
сущностей ее предметной об\-ласти.
  
  Необходимость использования классификаций информатики в~процессе 
создания концепции проиллюстрируем, используя иерархию  
Акоффа~\cite{8-zac}. Он использовал принцип их вертикального размещения 
в~иерархии снизу вверх: данные, информация и~знание. Еще в~ней есть термин 
<<мудрость>>, который в~статье не рассматривается. Такое размещение Акофф 
прокомментировал так: <<Каждое из пе\-ре\-чис\-лен\-ных понятий [кроме данных] 
содержит в~себе нижестоящие$\ldots$>>~\cite{8-zac}.
  
  Этому принципу размещения и~комментарию Акоффа свойственны 
недостатки, проанализированные, в~частности, в~работе~\cite{10-zac}. Главный 
вывод, к~которому пришла Роули после изучения иерархии Акоффа, 
заключается в~следующем: <<$\ldots$информация определяется в~терминах 
данных, знание~--- в~терминах информации$\ldots$ но существует меньше 
консенсуса в~описании трансформаций, которые преобразуют сущности, 
расположенные ниже в~иерархии, в~те, которые находятся над ними, что 
приводит к~их терминологической неопределенности>>~\cite{10-zac}. Причина 
этой неопределенности, скорее всего, в~том, что базовые понятия информатики 
включены в~иерархию Акоффа изолированно от общего контекста 
классификаций сущностей ее предметной об\-ласти.

%\vspace*{-9pt}
  
\section{Классификации сущностей информатики}


%\vspace*{-2pt}

  Все сущности предметной области информатики в~работах~[22, 23] 
разделены на два глобальных класса: ее объекты и~их трансформации. Для 
каждого такого класса была предложена своя классификация. 
В~работе~\cite{22-zac} дано описание классификации объектов предметной 
области информатики, первый уровень которой содержит базовые понятия ее 
предметной области (данные, информация, знания и~др.).  
В~работе~\cite{23-zac} дано описание двух верхних уровней классификации 
трансформаций объектов предметной об\-ласти (см.\ рисунок 
в~работе~\cite{23-zac}). Основанием для построения самого верхнего ее уровня послужило деление 
предметной области информатики на среды\footnote{В~работе~\cite{24-zac} дано описание пяти сред 
предметной области информатики (ментальная; сенсорно воспринимаемая, или информационная; 
цифровая; нейро- и~ДНК-среда), каждая из которых по определению включает объекты одной и~той же 
природы.} и~степень разнообразия природы объектов, вовлеченных в~трансформации:
\begin{itemize}
\item  первый класс верхнего уровня классификации включает 
трансформации объектов в~пределах среды только одной природы 
(трансформации первого порядка);
\item  второй класс включает трансформации объектов, относящихся 
к~двум средам разной природы (трансформации второго порядка);
\item третий и~последующие классы включают трансформации объектов, 
относящихся к~трем и~более средам разной природы (трансформации 
третьего и~более высоких порядков).
\end{itemize}

  В работе~\cite{23-zac} были приведены примеры для трех первых классов 
трансформаций, включая пример трансформаций объектов, относящихся 
к~двум средам разной природы (компьютерное кодирование символов текстов 
с~по\-мощью таб\-лиц Unicode).
  
Основанием для построения второго уровня классификации трансформаций объектов послужила типология 
знаковых сис\-тем А.~Соломоника~\cite[c.~131]{25-zac}: естественные знаковые сис\-те\-мы, образные,  
ес\-тест\-вен\-но-язы\-ко\-в$\acute{\mbox{ы}}$е,  
вер\-баль\-но-не\-сло\-вес\-ные сис\-те\-мы записи\footnote{Под системой записи понимается знаковая 
система, сочетающая вербальные знаки с~несловесными (языки нотной записи, карт, таблиц и~др.).} 
и~формализованные знаковые сис\-те\-мы, включая математические. Введем понятие обобщенного текста~--- 
это текст, который может быть создан в~любой из перечисленных знаковых систем. Тогда обобщенные тексты 
могут быть естественными, образными, ес\-тест\-вен\-но-язы\-ко\-в$\acute{\mbox{ы}}$\-ми,  
вер\-баль\-но-не\-сло\-вес\-ны\-ми и~формализованными. Второй уровень классификации трансформаций 
охватывает не все виды объектов предметной  
об\-ласти информатики, а~только перечисленные~5~видов текс\-тов и~их представления, вовлеченные 
в~процессы трансформаций в~одной или более средах вместе с~данными, знанием и~его концептами.

\begin{figure*}[b] %fig1
\vspace*{6pt}
      \begin{center}
     \mbox{%
\epsfxsize=121.191mm 
\epsfbox{zac-1.eps}
}
\end{center}
\vspace*{-6pt}
\Caption{Средовая версия иерархии Акоффа}
\end{figure*}

\section{Классификация трансформаций: построение~третьего 
уровня}

  Основанием для систематизации трансформаций первого и~второго порядка 
на третьем уровне этой классификации служит иерархия Акоффа~\cite{8-zac}, 
на основе которой и~была создана ее средов$\acute{\mbox{а}}$я версия~[26, 
27]. Для создания средов$\acute{\mbox{о}}$й версии была выполнена 
категоризация трех базовых понятий информатики (данные, информация, 
знания) на объекты лексикографической информационной сис\-те\-мы 
в~процессе создания ее концепции\linebreak (рис.~1).
  


  В отличие от классической иерархии Акоффа, в~ее 
средов$\acute{\mbox{о}}$й версии различаются три вида данных: сенсорно 
воспринимаемые, цифровые и~те данные, которые генерируются 
искусственными нейронными сетями (ИНС) в~системах искусственного интеллекта 
(далее~--- ИИ-дан\-ные). Последний вид данных необходим, например, для 
различения входа и~выхода процесса применения обученной 
ИНС в~цифровой модели генерации знания, описанию которой 
посвящена работа~\cite{27-zac}.
  
  Также предлагается различать два вида информации: сенсорно 
воспринимаемая и~цифровая. Кроме знания в~средов$\acute{\mbox{у}}$ю 
версию добавлены концепты и~ментальные образы сенсорно воспринимаемых 
данных. Последние служат промежуточной сущностью между сенсорно 
воспринимаемыми данными и~генерируемым знанием при описании процессов 
извлечения знания из текстовых данных лексикографической информационной 
системы. Описание объектов средов$\acute{\mbox{о}}$й версии иерархии 
Акоффа (см.\ рис.~1) и~отношений между ними дано в~работах~\cite{26-zac, 28-zac}.
  
  В средов$\acute{\mbox{о}}$й версии число объектов равно восьми. Если 
учитывать направления трансформаций, то между восемью объектами на 
рис.~1 она включает~16 их видов (трансформации на границе между сенсорно 
воспринимаемыми данными и~информацией, обозначенные символом~<<?>>, 
в~статье не рас\-смат\-ри\-ва\-ют\-ся). В~будущем число объектов 
в~средов$\acute{\mbox{о}}$й версии, которая выбрана как основание для 
сис\-те\-ма\-ти\-за\-ции трансформаций первого и~второго порядка, может быть 
увеличено. Для построения классификации трансформаций 
важ\-но не возможное увеличение числа объектов 
и~трансформаций между ними, а то, что их виды в~средов$\acute{\mbox{о}}$й 
версии распределены между трансформациями первого и~второго порядка. Из 
16~видов на рис.~1 шесть относятся к~трансформациям первого порядка, это\linebreak 
виды с~номерами~7, 8, 13--16 (далее~--- типология трансформаций первого 
порядка), а~десять~--- к~трансформациям второго порядка, это виды 
с~\mbox{номерами}~1--6 и~9--12 (далее~--- типология трансформаций второго 
порядка). Разместим обе типологии на третьем уровне классификации (см.\ ее 
схему на рис.~2). Перечислим виды трансформаций первой типологии, вводя 
в~скобках их краткие названия, используемые ниже на рис.~3:
  \begin{description}
  \item[\,] 7~--- членение знания на концепты с~помощью одной или нескольких 
знаковых систем (далее~--- членение знания);
  \item[\,] 8~--- формирование знания на основе концептов (формирование 
знания);
  \item[\,] 13~--- обучение ИНС;
  \end{description}
  
  \vspace*{-6pt}
  
  \pagebreak
  
  \end{multicols}
  
  \begin{figure*} %fig2
\vspace*{1pt}
      \begin{center}
     \mbox{%
\epsfxsize=127.513mm 
\epsfbox{zac-2.eps}
}
\end{center}
\vspace*{-9pt}
\Caption{Схема трех верхних уровней классификации трансформаций объектов (объединены 
по три слоя и~для второго, и~для третьего уровней этой классификации)}
\end{figure*}
  
  \begin{multicols}{2}
  
  \noindent
  \begin{description}
  \item[\,] 14~--- восстановление обучающей информации на основе 
содержания обученной ИНС (обращение ИНС);
  \item[\,] 15~--- использование обученной ИНС (использование ИНС);



  \item[\,] 16~--- восстановление исходных данных, соответствующих 
полученным результатам работы обучен\-ной ИНС (восстановление исходных данных 
по результатам ИНС).
  \end{description}
  
  
  Не все виды трансформаций 13--16 поддерживаются в~конкретных системах 
искусственного интеллекта, но с~теоретической точки зрения все их 
предлагается включить в~первую типологию для полноты спектра видов 
трансформаций.
  
  Перечислим виды трансформаций второй типологии:
  \begin{description}
  \item[\,] 1~--- декодирование цифровых данных в~компьютерных системах 
(декодирование данных);
  \item[\,]  2~--- кодирование сенсорно воспринимаемых данных (кодирование 
данных);
  \item[\,] 3~--- ментальное копирование сенсорно воспринимаемых данных 
(ментальное копирование);
  \item[\,] 4~--- восстановление сенсорно воспринимаемых данных по 
ментальным образам (восстановление по образам);
  \item[\,] 5~--- смысловая интерпретация без деления на концепты ментальных 
образов сенсорно воспринимаемых данных (смысловая интерпретация);
  \item[\,] 6~--- восстановление ментальных образов (восстановление образов);
  \item[\,] 9~--- представление концептов в~виде сенсорно воспринимаемой 
информации, например текс\-та\-ми, формулами, таблицами, рисунками и~т.\,д.\ 
(представление концептов);
  \item[\,] 10~--- понимание смысла сенсорно воспринимаемой информации 
(понимание смысла);
  \item[\,] 11~--- кодирование сенсорно воспринимаемой информации 
(кодирование информации);
\end{description}

\vspace*{-6pt}

\pagebreak

\end{multicols}

\begin{figure*} %fig3
\vspace*{1pt}
      \begin{center}
     \mbox{%
\epsfxsize=163mm 
\epsfbox{zac-3.eps}
}
\end{center}
\vspace*{-9pt}
\Caption{Схема частного случая классификации трансформаций объектов (трансформации 
пронумерованы согласно рис.~1)}
\end{figure*}

\begin{multicols}{2}

\noindent
\begin{description}

  \item[\,] 12~--- декодирование цифровой информации (декодирование 
информации).
  \end{description}
  
  Отметим, что в~существующих ИТ
  и~компьютерных системах наиболее часто используются виды 
трансформаций~13 и~15 типологии первого порядка и~1, 2, 11 и~12 типологии 
второго порядка. На рис.~2 в~первом слое третьего уровня классификации 
показаны типологии первого порядка без указания числа трансформаций в~них 
и~без детализации трансформируемых объектов.
  
  Во втором слое третьего уровня классификации условно (без названий) 
показаны типологии второго порядка. Также на рис.~2 в~третьем слое третьего 
уровня классификации условно (также без названий) показаны типологии 
третьего порядка, которые планируется рассмотреть в~отдельной статье. По 
определению они должны включать трансформации между тремя объектами 
разной природы, но средов$\acute{\mbox{а}}$я версия иерархии Акоффа 
включает трансформации только между двумя объектами разной природы. 
Поэтому потребуется другое основание для их систематизации (ранее были 
рассмотрены отдельные примеры трансформаций третьего 
порядка\footnote{Далеко не всегда трансформации третьего и~более высоких порядков можно 
рассматривать как последовательность трансформаций второго порядка. Примером этого могут 
служить трансформации в~процессе обучения пациента пользованию роботизированной рукой, 
охватывающие личностные концепты пациента, релевантные его намерениям, сигналы активности 
мозга как объекты нейросреды и~компьютерные коды~\cite{29-zac}.}~\cite{29-zac}).

\section{Классификация трансформаций: частный~случай}

  Выше было отмечено, что в~будущем число объектов 
в~средов$\acute{\mbox{о}}$й версии иерархии Акоффа может быть увеличено. 
Это означает, что увеличатся и~чис\-ло объектов, и~чис\-ло трансформаций между 
ними в~классификации трансформаций, так как эта средов$\acute{\mbox{а}}$я 
версия служит по определению основанием для систематизации 
трансформаций первого и~второго порядка. Поэтому на третьем уровне рис.~2 
указаны типологии без детализации объектов и~без указания числа 
трансформаций в~каждой из них. С~одной стороны, при таком подходе 
получаем достаточно общий вид этой классификации, так как она не зависит от 
числа объектов в~том или ином варианте средов$\acute{\mbox{о}}$й версии 
(и~это существенно упрощает рис.~2). С~другой стороны, на третьем уровне 
такой общей классификации подразумевается, но не эксплицируется природа 
трансформируемых объектов и~их возможные сочетания в~трансформациях. 

При проектировании лексикографической информационной системы важно 
эксплицировать природу трансформируемых объектов и~их возможные 
сочетания.
  %
  Поэтому в~парадигму информатики~\cite{30-zac} кроме общей 
классификации трансформаций предлагается включать и~ее частные случаи, 
эксплицирующие природу трансформируемых объектов. 

В~этом разделе 
рассмотрим один частный случай, когда используются только естественные 
знаковые сис\-те\-мы из типологии А.~Соломоника~\cite{25-zac} вместе 
с~данными, знанием и~его концептами. Чис\-ло естественных языков при этом не 
ограничено. И~этот частный случай классификации включает только три 
класса природных трансформаций (первого, второго и~третьего порядка, см.\ 
схему классификации на рис.~3).
  
  Первый и~второй уровни схемы общей классификации (см.\ рис.~2) можно 
объединить в~один уровень в~этом частном случае. Ниже этого уровня 
приведено содержание типологий первого и~второго порядка без содержания 
типологий третьего по\-рядка.




  Наполнение типологий первого и~второго порядка соответствует 
средов$\acute{\mbox{о}}$й версии иерархии Акоффа на рис.~1, содержащей 
6~видов трансформаций типологии первого порядка и~10~видов 
трансформаций типологии второго порядка (на рис.~3 стрелки указывают 
направления трансформаций согласно средов$\acute{\mbox{о}}$й версии на рис.~1).
  
  Таким образом, частный случай классификации содержит для этих двух 
типологий 16~теоретически возможных трансформаций, 6 из которых 
в~настоящее время в~существующих ИТ применяются наиболее часто: виды 
трансформаций~1, 2, 11 и~12 типологии второго порядка реализуются 
с~помощью тех или иных методов ко\-ди\-ро\-ва\-ния/де\-ко\-ди\-ро\-ва\-ния 
(например, с~использованием таблиц Unicode), а~виды трансформаций~13 и~15
 в~типологии первого порядка реализуются полностью с~по\-мощью процессов 
цифровой обработки компьютерами.
  
  Остальные виды трансформаций или применяются намного реже (это 
виды~3, 5, 7, 9 и~10), или находятся в~стадии поиска и~разработки (14 и~16) или 
в~настоящее время носят только теоретический характер, обеспечивая полноту 
первой и~второй типологий (4, 6 и~8). Знаком~<<?>> обозначены те виды 
трансформаций, которые по определению не существуют в~используемой 
парадигме информатики~\cite{30-zac}. Однако возможно, что в~других 
будущих подходах к~построению ее парадигмы эти виды трансформаций будут 
существовать.
  
\section{Заключение}

  На сегодняшний день процесс построения классификаций объектов 
предметной области информатики~\cite{22-zac} и~их  
трансформаций~\cite{23-zac} еще не завершен. Однако первые результаты их 
построения уже используются для создания концепции лексикографической 
информационной сис\-те\-мы, обеспечивающей интеграцию двуязычных 
словарей и~параллельных корпусов.
  
  \bigskip
  
  
  Автор признателен рецензентам за помощь в~улучшении статьи.
  
{\small\frenchspacing
 { %\baselineskip=10.6pt
 %\addcontentsline{toc}{section}{References}
 \begin{thebibliography}{99}
\bibitem{1-zac}
\Au{Aijmer K., Altenberg~B.} Advances in corpus-based contrastive linguistics. Studies in honour 
of Stig Johansson.~--- Amsterdam: John Benjamins, 2013. 295~p.  doi: 10.1075/scl.54.
\bibitem{2-zac}
\Au{Добровольский Д.\,О., Кретов~А.\, А., Шаров~С.\,А.} Корпус параллельных текстов~// 
Научная и~техническая информация. Сер.~2: Информационные процессы и~сис\-те\-мы, 2005. 
№\,6. С.~16--27.
\bibitem{3-zac}
\Au{Добровольский Д.\,О.} Корпус параллельных текстов и~сопоставительная 
лексикология~// Труды Института русского языка им.\ В.\,В.~Виноградова, 2015. №\,6. 
С.~413--449. EDN: VJQBHP.
\bibitem{4-zac}
\Au{Гончаров А.\,А., Зацман~И.\,М., Кружков~М.\,Г.} Эволюция классификаций 
в~надкорпусных базах данных~// Информатика и~её применения, 2020. Т.~14. Вып.~4. 
С.~108--116. doi: 10.14357/19922264200415.  
EDN: \mbox{GKWBZT}.
\bibitem{5-zac}
\Au{Гончаров А.\, А., Зацман И. \,М., Кружков~М.\, Г}. Представление новых 
лексикографических знаний в~динамических классификационных сис\-те\-мах~// 
Информатика и~её применения, 2021. Т.~15. Вып.~1. С.~86--93.  doi: 10.14357/19922264210112. EDN: OPEFXW.
\bibitem{6-zac}
\Au{Zatsman I.} Finding and filling lacunas in linguistic typologies~// 15th Forum (International) 
on Knowledge Asset Dynamics Proceedings.~--- Matera, Italy: Institute of Knowledge Asset 
Management, 2020. P.~780--793.
\bibitem{7-zac}
\Au{Zatsman I.} Three-dimensional encoding of emerging meanings in AI-systems~// 21st 
European Conference on Knowledge Management Proceedings.~--- Reading, U.K.: Academic 
Publishing International Ltd., 2020. P.~878--887.
\bibitem{8-zac}
\Au{Ackoff R.} From data to wisdom~// J.~Applied Systems Analysis, 1989. Vol.~16. No.\,1. P.~3--9.
\bibitem{9-zac}
\Au{Rosenbloom P.\,S.} On computing: The fourth great scientific domain.~--- Cambridge, MA, 
USA: MIT Press, 2013. 307~p.
\bibitem{10-zac}
\Au{Rowley J.} The wisdom hierarchy: Representations of the DIKW hierarchy~// J.~Inf. 
Sci., 2007. Vol.~33. Iss.~2. P.~163--180. doi: 10.1177/0165551506070706.
\bibitem{11-zac} 
\Au{Frick$\acute{\mbox{e}}$~M.\,H.} Data--Information--Knowledge--Wisdom (DIKW) pyramid, 
framework, continuum~// Encyclopedia of big data~/ Eds. L.~Schintler, C.~McNeely.~--- Cham: 
Springer, 2018. 4~p. doi: 10.1007/978-3-319-32001-4\_331-1.
\bibitem{12-zac}
\Au{Denning P., Rosenbloom~P.} Computing: The fourth great domain of science~// Commun. 
ACM, 2009. Vol.~52. Iss.~9. P.~27--29.
\bibitem{13-zac}
\Au{Denning P., Freeman~P.} Computing's paradigm~// Commun.  ACM, 2009. Vol.~52. 
Iss.~12. P.~28--30. doi: 10.1145/ 1610252.1610265.
\bibitem{17-zac} %14
\Au{Farradane J.} Knowledge, information, and information science~// J.~Inf. Sci., 
1980. Vol.~2. Iss.~2. P.~75--80. doi: 10.1177/01655515800020020.

\bibitem{15-zac}
\Au{Шрейдер Ю.\,А.} Информация и~знание~// Сис\-тем\-ная концепция информационных 
процессов.~--- М.: ВНИИСИ, 1988. С.~47--52.
\bibitem{16-zac}
\Au{Ingwersen P.} Information and information science~// Enclyclopaedie of library and 
information science~/ Eds. J.\,D.~McDonald, 
M.~Levine-Clark.~--- New York, NY, USA: Marcel Dekker Inc., 1992. Vol.~56. Sup.~19. 
P.~137--174.

\bibitem{14-zac} %17
Информатика как наука об информации: Информационный, документальный, 
технологический, экономический, социальный и~организационный аспекты~/ Под ред. 
Р.\,С.~Гиляревского.~--- М.: Фаир-Пресс, 2006. 592~с.

\bibitem{18-zac}
\Au{Hjorland B.} Library and information science: practice, theory, and philosophical basis~// 
Inform. Process. Manag., 2000. Vol.~36. Iss.~3. P.~501--531. doi:  
10.1016/S0306-\mbox{4573(99)00038-2}.
\bibitem{19-zac}
Deep shift~--- technology tipping points and societal impact.~--- Geneva: WE Forum, 2015. 44~p. 
{\sf http://www3.weforum.org/docs/WEF\_GAC15\_ Technological\_Tipping\_Points\_report\_2015.pdf}.
\bibitem{20-zac}
\Au{Berman F., Rutenbar~R., Hailpern~B., Christensen~H., Davidson~S., Estrin~D., 
Franklin~M., Martonosi~M., Raghavan~P., Stodden~V., Szalay~A.\,S.} Realizing the potential of 
data science~// Commun.  ACM, 2018. Vol.~61. Iss.~4. P.~67--72. doi: 10.1145/3188721.

\bibitem{21-zac}
\Au{Stodden V.} The data science life cycle: A~disciplined approach to advancing data science as 
a~science~// Commun.  ACM, 2020. Vol.~63. Iss.~7. P.~58--66. doi: 10.1145/ 3360646.


\bibitem{23-zac} %22
\Au{Зацман И.\,М.} Научная парадигма информатики: классификация трансформаций 
объектов предметной об\-ласти~// Системы и~средства информатики, 2023. Т.~33. №\,4. 
С.~126--138. doi: 10.14357/08696527230412. EDN: ZIKUWO.

\bibitem{22-zac} %23
\Au{Зацман И.\,М.} Научная парадигма информатики: классификация объектов предметной  
об\-ласти~// Информатика и~её применения, 2023. Т.~17. Вып.~4. С.~96--103. doi: 
10.14357/19922264230413. EDN: FIUQAT.

\bibitem{24-zac}
\Au{Зацман И.\,М.} О~научной парадигме информатики: верхний уровень классификации 
объектов ее предметной об\-ласти~// Информатика и~её применения, 2022. Т.~16. Вып.~4. 
С.~73--79. doi: 10.14357/ 19922264220411. EDN: XZNKVI.

\bibitem{25-zac}
\Au{Соломоник А.\,Б.} Философия знаковых систем и~язык.~--- М.: ЛКИ, 2011. 408~с.
\bibitem{26-zac}
\Au{Зацман И.\,М.} Трансформация иерархии Акоффа в~научной парадигме информатики~// 
Информатика и~её применения, 2023. Т.~17. Вып.~3. С.~107--113. doi: 
10.14357/19922264230315. EDN: UMVRRV.

\bibitem{27-zac}
\Au{Zatsman I.} Building digital spiral models of knowledge generation~// 19th Forum 
(International) on Knowledge Asset Dynamics Proceedings.~--- Matera, Italy: Arts for Business 
Institute, 2024. P.~2185--2196.
\bibitem{28-zac}
\Au{Zatsman I.} Digital spiral model of knowledge creation and encoding its dynamics~// 18th 
Forum (International) on Knowledge Asset Dynamics Proceedings.~--- Matera, Italy: Arts for 
Business Institute, 2023. P.~581--596.
\bibitem{29-zac}
\Au{Зацман И.\,М.} Интерфейсы третьего порядка в~информатике~// Информатика и~её 
применения, 2019. Т.~13. Вып.~3. С.~82--89. doi: 10.14357/19922264190312. EDN: 
EHRQLF.

\bibitem{30-zac}
\Au{Зацман И.\,М.} Научная парадигма информатики как третьей культуры~//  
На\-уч\-но-тех\-ни\-че\-ская информация. Сер.~1: Организация и~методика информационной 
работы, 2023. №\,11. С.~1--14.

\end{thebibliography}

 }
 }

\end{multicols}

\vspace*{-9pt}

\hfill{\small\textit{Поступила в~редакцию 14.04.24}}

\vspace*{4pt}

%\pagebreak

%\newpage

%\vspace*{-28pt}

\hrule

\vspace*{2pt}

\hrule



\def\tit{OBJECT TRANSFORMATIONS OF~THE~FIRST AND~SECOND ORDER
IN~A~LEXICOGRAPHIC INFORMATION SYSTEM\\[-5pt]}


\def\titkol{Object transformations of~the~first and~second order
in~a~lexicographic information system}


\def\aut{I.\,M.~Zatsman}

\def\autkol{I.\,M.~Zatsman}

\titel{\tit}{\aut}{\autkol}{\titkol}

\vspace*{-13pt}


\noindent
Federal Research Center ``Computer Science and Control'' of the Russian Academy of Sciences, 
44-2~Vavilov Str., Moscow 119133, Russian Federation


\def\leftfootline{\small{\textbf{\thepage}
\hfill INFORMATIKA I EE PRIMENENIYA~--- INFORMATICS AND
APPLICATIONS\ \ \ 2024\ \ \ volume~18\ \ \ issue\ 2}
}%
 \def\rightfootline{\small{INFORMATIKA I EE PRIMENENIYA~---
INFORMATICS AND APPLICATIONS\ \ \ 2024\ \ \ volume~18\ \ \ issue\ 2
\hfill \textbf{\thepage}}}

\vspace*{2pt}



\Abste{The theoretical foundations of the design of information technologies used for 
the integration of bilingual dictionaries and parallel corpora are considered. The 
description of the first outcomes of the creation of the third\linebreak\vspace*{-12pt}}

\Abstend{ level of object 
transformations classification in the subject domain of informatics, which is supposed 
to be used
in creating the lexicographic information system providing integration, is 
given. All the entities of informatics are divided into two global classes: objects and 
their transformations. For each such class, its own classification is constructed. 
Previously, the two upper levels of the object transformation classification in the subject 
domain have been described. The present paper discusses the third level of this classification. The 
basis for the construction of its highest level was the division of the subject domain of 
informatics into media (mental, sensory, digital, and a~number of other media), each 
of which by definition includes objects of the same nature. The Solomonick's 
typology of sign systems served as the basis for constructing the second level of the 
object transformation classification. The aim of the paper is to systematize object 
transformations of the first and second orders at the third level of this classification. 
The basis for systematization is the medium version of the Ackoff's hierarchy.}

\KWE{subject domain objects; object transformations; classification; data; 
information; knowledge; lexicographic information system}


\DOI{10.14357/19922264240211}{VZTGVV}

\vspace*{-12pt}

\Ack

\vspace*{-3pt}


\noindent
The reported study was funded by the Russian Science Foundation, project  
No.\,24-18-00155, {\sf 
https://rscf.ru/project/24-18-00155}. The research was carried out using the infrastructure of the Shared 
Research Facilities ``High Performance Computing and Big Data'' (CKP 
``Informatics'') of FRC CSC RAS (Moscow) .
   


  \begin{multicols}{2}

\renewcommand{\bibname}{\protect\rmfamily References}
%\renewcommand{\bibname}{\large\protect\rm References}

{\small\frenchspacing
 {%\baselineskip=10.8pt
 \addcontentsline{toc}{section}{References}
 \begin{thebibliography}{99} 
\bibitem{1-zac-1}
\Aue{Aijmer, K., and B.~Altenberg.} 2013. \textit{Advances in corpus-based 
contrastive linguistics. Studies in honour of Stig Johansson}. Amsterdam: John 
Benjamins. 295~p. doi: 10.1075/scl.54.
\bibitem{2-zac-1}
\Aue{Dobrovolskiy, D.\,O., A.\,A.~Kretov, and S.\,A.~Sharov.} 2005. Korpus 
parallel'nykh tekstov [Corpus of parallel texts]. \textit{Nauchnaya i~tekhnicheskaya 
informatsiya. Ser. 2. Informatsionnye protsessy i~sistemy} [Scientific and Technical 
Information. Ser.~2: Information Processes and Systems] 6:16--27.
\bibitem{3-zac-1}
\Aue{Dobrovolskiy, D.\,O.} 2015. Korpus parallel'nykh tekstov i~sopostavitel'naya 
leksikologiya [The corpus of parallel texts and contrastive lexicology]. \textit{Trudy 
Instituta russkogo yazyka im. V.\,V.~Vinogradova} [Proceedings of the 
V.\,V.~Vinogradov Russian Language Institute] 6:413--449. EDN: VJQBHP.
\bibitem{4-zac-1}
\Aue{Goncharov, A.\,A., I.\,M.~Zatsman, and M.\,G.~Kruzhkov.} 2020. Evolyutsiya 
klassifikatsiy v~nadkorpusnykh ba\-zakh dannykh [Evolution of classifications in 
supracorpora databases]. \textit{Informatika i~ee Primeneniya~--- Inform. \mbox{Appl.}}  
14(4):108--116. doi: 10.14357/19922264200415.  
EDN: GKWBZT.
\bibitem{5-zac-1}
\Aue{Goncharov, A.\,A., I.\,M.~Zatsman, and M.\,G.~Kruzhkov.} 2021. 
Predstavlenie novykh leksikograficheskikh znaniy v~dinamicheskikh 
klassifikatsionnykh sistemakh [Representation of new lexicographical knowledge in 
dynamic classification systems]. \textit{Informatika i~ee Primeneniya~--- Inform. 
Appl.} 15(1):86--93. doi: 10.14357/19922264210112. EDN: OPEFXW.
\bibitem{6-zac-1}
\Aue{Zatsman, I.} 2020. Finding and filling lacunas in linguistic typologies. 
\textit{15th Forum (International) on Knowledge Asset Dynamics Proceedings}. 
Matera, Italy: Institute of Knowledge Asset Management. 780--793.
\bibitem{7-zac-1}
\Aue{Zatsman, I.} 2020. Three-dimensional encoding of emerging meanings in  
AI-systems. \textit{21st European Conference on Knowledge Management 
Proceedings}. Reading, U.K.: Academic Publishing International Ltd. 878--887.
\bibitem{8-zac-1}
\Aue{Ackoff, R.} 1989. From data to wisdom. \textit{J.~Applied Systems Analysis} 
16(1):3--9.
\bibitem{9-zac-1}
\Aue{Rosenbloom, P.\,S.} 2013. \textit{On computing: The fourth great scientific 
domain}. Cambridge, MA: MIT Press. 307~p.
\bibitem{10-zac-1}
\Aue{Rowley, J.} 2007. The wisdom hierarchy: Representations of the DIKW 
hierarchy. \textit{J.~Inf. Sci.} 33(2):163--180. doi: 10.1177/0165551506070706.
\bibitem{11-zac-1}
\Aue{Frick$\acute{\mbox{e}}$, M.\,H.} 2018.  
Data-Information-Knowledge-Wisdom (DIKW) pyramid, framework, continuum. 
\textit{Encyclopedia of big data}. Eds. L.~Schintler and C.~McNeely. Cham: 
Springer. 4~p. doi: 10.1007/978-3-319-32001- 4\_331-1.
\bibitem{12-zac-1}
\Aue{Denning, P., and P.~Rosenbloom.} 2009. Computing: The fourth great domain 
of science. \textit{Commun. ACM} 52(9):27--29.
\bibitem{13-zac-1}
\Aue{Denning, P., and P.~Freeman.} 2009. Computing's paradigm. \textit{Commun. 
ACM} 52(12):28--30. doi: 10.1145/ 1610252.1610265.

\bibitem{17-zac-1} %14
\Aue{Farradane, J.} 1980. Knowledge, information, and information science. 
\textit{J.~Inf. Sci.} 2(2):75--80. doi: 10.1177/ 01655515800020020.

\bibitem{15-zac-1}
\Aue{Shreyder, Yu.\,A.} 1988. Informatsiya i~znanie [Information and knowledge]. 
\textit{Sistemnaya kontseptsiya in\-for\-ma\-tsi\-on\-nykh protsessov} [System concept of 
information processes]. Moscow: VNIISI. 47--52.
\bibitem{16-zac-1}
\Aue{Ingwersen, P.} 1995. Information and information science. 
\textit{Encyclopedia of library and information science}. Eds. J.\,D.~McDonald and 
M.~Levine-Clark. New York, NY: Marcel Dekker Inc. 56(19):137--174.

\bibitem{14-zac-1} %17
Gilyarevskiy, R.\,S., ed. 2006. \textit{Informatika kak nauka ob informatsii: 
informatsionnyy, dokumental'nyy, tekh\-no\-lo\-gi\-che\-skiy, ekonomicheskiy, sotsial'nyy 
i~organizatsionnyy aspekty} [Informatics as information science: Informational, 
documentary, technological, economic, social, and organizational dimensions]. 
Moscow: FAIR-PRESS. 592~p.

\bibitem{18-zac-1}
\Aue{Hjorland, B.} 2000. Library and information science: Practice, theory, and 
philosophical basis. \textit{Inform. Process. Manag.} 36(3):501--531. doi:  
10.1016/S0306-\mbox{4573(99)00038-2}.
\bibitem{19-zac-1}
Deep shift~--- technology tipping points and societal impact. 2015. \textit{World Economic 
Forum}. Geneva. 44~p. Available at: {\sf 
http://www3.weforum.org/docs/WEF\_ GAC15\_Technological\_Tipping\_Points\_report\_2015.pdf} (accessed May~20, 
2024).
\bibitem{20-zac-1}
\Aue{Berman, F., R.~Rutenbar, B.~Hailpern, H.~Christensen, S.~Davidson, 
D.~Estrin, M.~Franklin, M.~Martonosi, P.~Raghavan, V.~Stodden, and 
A.\,S.~Szalay.} 2018. Realizing the potential of data science. \textit{Commun. ACM} 
61(4):67--72. doi: 10.1145/3188721.
\bibitem{21-zac-1}
\Aue{Stodden, V.} 2020. The data science life cycle: A~disciplined approach to 
advancing data science as a~science. \textit{Commun. ACM} 
 63(7):58--66. doi: 10.1145/3360646.

\bibitem{23-zac-1} %22
\Aue{Zatsman, I.\,M.} 2023. Nauchnaya paradigma informatiki: klassifikatsiya 
transformatsiy ob''ektov predmetnoy oblasti [Scientific paradigm of informatics: 
Transformation classification of domain objects]. \textit{Sistemy i~Sredstva 
Informatiki~--- Systems and Means of Informatics} 33(4):126--138. doi: 
10.14357/08696527230412. EDN: ZIKUWO.

\bibitem{22-zac-1} %23
\Aue{Zatsman, I.\,M.} 2023. Nauchnaya paradigma informatiki: klassifikatsiya 
ob''ektov predmetnoy oblasti [Scientific paradigm of informatics: Classification of 
domain objects]. \textit{Informatika i~ee Primeneniya~--- Inform. Appl.} 
 17(4):96--103. doi: 10.14357/19922264230413. EDN: FIUQAT.
 
\bibitem{24-zac-1}
\Aue{   Zatsman, I.\,M.} 2022. O nauchnoy paradigme informatiki: verkhniy uroven' 
klassifikatsii ob''ektov ee predmetnoy oblasti [On the scientific paradigm of 
informatics: The classification high level of its objects]. \textit{Informatika i~ee 
Primeneniya~--- Inform. Appl.} 16(4):73--79. doi: 10.14357/19922264220411. EDN: 
XZNKVI.
\bibitem{25-zac-1}
\Aue{Solomonick, A.\,B.} 2011. \textit{Filosofiya znakovykh system i~yazyk} 
[Philosophy of sign systems and language]. Moscow: LKI. 408~p.
\bibitem{26-zac-1}
\Aue{Zatsman, I.\,M.} 2023. Transformatsiya ierarkhii Akoffa v~nauchnoy 
paradigme informatiki [Transformation of the Ackoff's hierarchy in the scientific 
paradigm of informatics]. \textit{Informatika i~ee Primeneniya~--- Inform. \mbox{Appl.}} 
17(3):107--113. doi: 10.14357/19922264230315. EDN: UMVRRV.
\bibitem{27-zac-1}
\Aue{Zatsman, I.} 2024. Building digital spiral models of knowledge 
generation. \textit{19th Forum (International) on Knowledge Asset Dynamics 
Proceedings}. Matera, Italy: Arts for Business Institute. 2185--2196.
\bibitem{28-zac-1}
\Aue{Zatsman, I.} 2023. Digital spiral model of knowledge creation and encoding its 
dynamics. \textit{18th Forum (International) on Knowledge Asset Dynamics 
Proceedings}. Matera, Italy: Arts for Business Institute. 581--596.
\bibitem{29-zac-1}
\Aue{Zatsman, I.\,M.} 2019. Interfeysy tret'ego poryadka v~informatike 
 [Third-order interfaces in informatics]. \textit{Informatika i~ee Primeneniya~--- 
Inform. Appl.} 13(3):82--89. doi: 10.14357/19922264190312. EDN: EHRQLF.
\bibitem{30-zac-1}
\Aue{Zatsman, I.} 2023. Scientific paradigm of informatics as a~third culture. 
\textit{Scientific Technical Information Processing} 50(4):246--258. doi: 
10.3103/S0147688223040111. EDN: CKHMYS.

\end{thebibliography}

 }
 }

\end{multicols}

\vspace*{-6pt}

\hfill{\small\textit{Received April 14, 2024}} 


\vspace*{-12pt}


\Contrl

\vspace*{-3pt}

\noindent
\textbf{Zatsman Igor M.} (b.\ 1952)~--- Doctor of Science in technology, head of 
department, Federal Research Center ``Computer Science and Control'' of the 
Russian Academy of Sciences, 44-2~Vavilov Str., Moscow 119333, Russian 
Federation; \mbox{izatsman@yandex.ru}





\label{end\stat}

\renewcommand{\bibname}{\protect\rm Литература}  %11
\def\stat{zol-zats}

\def\tit{МОДЕЛЬ И ТЕХНОЛОГИЯ ИЗВЛЕЧЕНИЯ НОВЫХ ТЕРМИНОВ ИЗ~МЕДИЦИНСКИХ 
ТЕКСТОВ$^*$}

\def\titkol{Модель и~технология извлечения новых терминов из~медицинских 
текстов}

\def\aut{И.\,М.~Зацман$^1$, О.\,В.~Золотарев$^2$, А.\,Х.~Хакимова$^3$, 
Гу Дунсяо$^4$}

\def\autkol{И.\,М.~Зацман, О.\,В.~Золотарев, А.\,Х.~Хакимова, 
Гу Дунсяо}

\titel{\tit}{\aut}{\autkol}{\titkol}

\index{Зацман И.\,М.}
\index{Золотарев О.\,В.}
\index{Хакимова А.\,Х.} 
\index{Дунсяо Гу}
\index{Zatsman I.\,M.}
\index{Zolotarev O.\,V.}
\index{Khakimova A.\,K.}
\index{Dongxiao Gu}


{\renewcommand{\thefootnote}{\fnsymbol{footnote}} \footnotetext[1]
{Исследование выполнено при поддержке РФФИ и~Государственного фонда естественных наук (ГФЕН) 
Китая (проект 21-57-53018).}}


\renewcommand{\thefootnote}{\arabic{footnote}}
\footnotetext[1]{Федеральный исследовательский центр <<Информатика и~управ\-ле\-ние>> Российской академии наук, 
\mbox{izatsman@yandex.ru}}
\footnotetext[2]{Институт информационных систем и~ин\-же\-нер\-но-компью\-тер\-ных технологий Российского нового 
университета, \mbox{ol-zolot@yandex.ru}}
\footnotetext[3]{Институт информационных систем и~ин\-же\-нер\-но-компью\-тер\-ных технологий Российского нового 
университета, \mbox{aida\_khatif@mail.ru}}
\footnotetext[4]{Технологический университет г.~Хэфэй (КНР), dongxiaogu@yeah.net}

\vspace*{-12pt} 
  

  \Abst{Рассматривается модель информационной технологии (ИТ) извлечения новых 
терминов из медицинских текстов, которая относится к~ранее определенному классу 
средовых моделей информатики. В~проведенном эксперименте для 
определения новизны терминов используется словарь MeSH (Medical Subject Headings), 
который создан и~обновляется Национальной медицинской библиотекой США. Появление 
новых терминов обусловлено представлением в~медицинских статьях и~других научных 
текстах нового знания об исследуемых болезнях, методах их лечения и~применяемых 
медикаментах, которое еще не нашло отражения в~медицинских словарях и~тезаурусах. 
В~информационных системах медицинских учреждений и~институтов предлагаемая 
технология позволяет регулярно актуализировать новыми терминами профили 
исследуемых болезней, соответствующих их предметной области. Цель статьи состоит 
в~описании средовой модели ИТ актуализации 
терминологических профилей болезней.}
    

\KW{средовые модели информатики; медицинские тексты; 
терминологический профиль; извлечение новых терминов из текстов}

 \DOI{10.14357/19922264220412} 
  
%\vspace*{-3pt}


\vskip 10pt plus 9pt minus 6pt

\thispagestyle{headings}

\begin{multicols}{2}

\label{st\stat}

\section{Введение}

  Одна из задач российско-ки\-тай\-ско\-го проекта, финансируемого по 
гранту №\,21-57-53018, состоит в~моделировании и~создании 
экспериментальной ИТ регулярного 
и~целенаправленного извлечения\linebreak новых терминов из медицинских текстов 
и~описания экспертами их значений в~терминологических профилях 
болезней. В~целом проект направлен на актуализацию терминологических 
\mbox{профилей} исследуемых болезней\footnote[5]{Терминологический портрет болезни 
включает отобранные экспертами ключевые термины ее описания, дефиниции их значений, 
отношения между значениями терминов в~портрете, контексты использования терминов 
и~аннотации контекстов тех научных статей и~других медицинских текстов, которые 
использовались экспертами при формировании дефиниций значений терминов, а~также степень 
конвенциональности (социализации) значений в~рамках корпоративной базы знаний.} как 
компонентов корпоративной базы знаний медицинского учреждения 
(института) в~его предметной об\-ласти. Речь не идет о терминологическом 
дублировании известных \mbox{медицинских} словарей, тезаурусов и~других 
лингвистических ресурсов~[1--7]. В~рамках проекта решается задача 
извлечения тех новых терминов, которые появляются в~медицинских статьях и~других научных текстах, но при этом еще не вошли в~медицинские 
лингвистические ресурсы (МЛР). Эти термины используются для актуализации 
терминологических профилей исследуемых болезней в~корпоративной базе 
знаний.
  
  Для решения указанной выше задачи проектируется экспериментальная 
ИТ целенаправленного извлечения новых терминов с~использованием 
средов$\acute{\mbox{о}}$й модели актуализации \mbox{терминологических} 
профилей болезней новыми терминами. Цель \mbox{статьи} состоит в~описании этой 
модели, относящейся к~классу средов$\acute{\mbox{ы}}$х моделей 
информатики.

\vspace*{-6pt}
  
\section{Средовая модель актуализации}

\vspace*{-4pt}

  Частные случаи средов$\acute{\mbox{о}}$й модели актуализации, 
рассмотренные в~работах~[8--10], предназначены для описания процесса 
извлечения новых терминов из текстов и~итерационного обновления 
терминологического портрета болезни. В~течение первого года выполнения 
проекта опыт применения этих частных случаев при проектировании 
экспериментальных ИТ и~решении задач проекта показал возможность их 
обобщения. Средов$\acute{\mbox{а}}$я модель актуализации в~предлагаемой 
обобщенной форме даст возможность использовать одну и~ту же модель при 
решении более широкого спектра задач. Ключевой\linebreak\vspace*{-12pt}

\pagebreak

\end{multicols}

\begin{figure*} %fig1
 \vspace*{1pt}
\begin{center}
     \mbox{%
\epsfxsize=163mm
\epsfbox{zac-1.eps}
}

\vspace*{6pt}

{\small Средовая модель актуализации терминологического портрета 
болезни в~корпоративной базе знаний}
\end{center}
\vspace*{-12pt}
\end{figure*}

\begin{multicols}{2}

\noindent
 аспект обобщения состоит 
в~разделении новых терминов на четыре категории с~точки зрения статуса их 
конвенциональности. Далее будут рассмотрены первые итерации обновления и~статусы конвенциональности терминов.

  
  До начала первой итерации обновления терминологического портрета 
исследуемой болезни экспертами формируется его начальная версия на 
основе терминов и~их синонимов, используемых при описании этой болезни, 
а~также их дефиниций в~существующих МЛР~[1--7]. Этим терминам, отобранным для начальной версии 
портрета, присваивается максимальный статус конвенциональности~--- 
\textit{словарный}.
  
  На первой итерации обновления начальная версия терминологического 
портрета используется как критерий новизны при поиске (см.\ стрелку 
с~бук\-вой~$\alpha$ на рисунке). Из электронной коллекции извлекаются статьи 
с~описанием исследуемой болезни (стрелка с~буквой~$\beta$) в~качестве 
источников потенциально новых терминов (ИПНТ)\footnote{Если термин статьи 
с~описанием болезни отсутствует в~начальной версии терминологического портрета, то это 
говорит только о возможной (потенциальной) его новизне, так как при дальнейшем 
семантическом анализе его контекста он может оказаться синонимом словарного термина, 
который отсутствует в~существующих МЛР. В~таком случае речь идет о новом синониме, а~не 
о~новом термине.}. После извлечения найденных статей в~них выделяются 
контексты ИПНТ, которые визуализируются перед операцией их 
семантического анализа (стрелка с~буквой~A).
  
  В процессе семантического анализа контекстов ИПНТ эксперт использует 
текущую (на первой итерации~--- начальную) версию терминологического 
портрета (стрелка с~буквой~$\gamma$). По завершении анализа эксперт 
разделяет найденные новые термины и~новые синонимы словарных терминов 
(ромб <<Новый термин?>>), если он их обнаружил. Значения новых 
терминов аннотируются и~оцифровываются, а дефиниции значений 
и~аннотации контекстов новых терминов (КНТ) с~этими значениями 
загружаются в~терминологический профиль болезни. Новые синонимы 
оцифровываются и~добавляются к~соответствующим словарным терминам. 
После обработки текстов эксперты формируют первую обновленную версию 
терминологического портрета. В~процессе формирования каждому из этих 
терминов присваивается один из четырех используемых в~проекте статусов 
конвенциональности~[10--12]:
  \begin{enumerate}[(1)]
  \item \textit{личностный}, если дефиниция значения (=\;кон\-цеп\-та) термина, 
сформированная одним экспертом, не была согласована им с~другими 
экспертами корпоративной базы знаний;
  \item \textit{коллективный}, если дефиниция значения термина, 
сформированная одним экспертом, была согласована им с~другими 
экспертами корпоративной базы знаний, но не со всеми;
  \item \textit{организационный}, если дефиниция значения термина, 
сформированная одним экспертом, была согласована им со всеми экспертами 
корпоративной базы знаний;
  \item \textit{словарный}, если во время первой итерации в~существующие 
МЛР был добавлен новый термин, включенный в~портрет (если 
в~существующие МЛР были добавлены термины, отсутствующие 
в~обработанных текстах, то они также включаются в~обновленную версию 
портрета в~статусе словарных).
  \end{enumerate}



  На завершающем этапе итерации предусмотрена возможность 
социализации значений новых терминов первых двух статусов 
конвенциональности (личностных и~коллективных) с~использованием их 
контекстов, т.\,е.\ КНТ. На рисунке показан случай социализации только 
\textit{личностных значений терминов}. На этап их социализации передаются 
КНТ (см.\ ромб <<КНТ?>>), а~не все контексты ИПНТ.
  
  На второй и~последующих итерациях используются уже обновленные 
портреты для извлечения новых терминов из новых поступлений текс\-тов\linebreak 
в~коллекцию. В~проекте используются \mbox{новые} поступления текстов в~базу 
данных PubMed, содержащую более 34~млн записей и~обеспечивающую 
развитые возможности поиска и~формирования \mbox{статистических} данных, 
включая получение распределения статей с~заданным термином по годам их 
публикации~\cite{13-zz}.
  
  В процессе построения средов$\acute{\mbox{о}}$й модели актуализации 
использовались теоретические основания их построения, сформулированные 
в~работе~\cite{14-zz}. Далее будут рассмотрены три основных положения 
теоретических оснований построения всего класса 
средов$\acute{\mbox{ы}}$х моделей (выделены ниже курсивом) в~контексте 
их применения в~проекте (см.\ рисунок)~\cite{10-zz, 15-zz}.
  
  \smallskip
  
  1.\ \textit{Выделение в~предметной области информатики тех сред разной 
природы, которые необходимы для моделирования проектируемой ИТ, 
определяется целью ее создания.}
  
  \smallskip
  
  В проекте цель проектирования ИТ сформулирована так: регулярное 
и~целенаправленное извлечение новых терминов из медицинских текстов 
и~описание экспертами их значений в~терминологических профилях 
болезней корпоративной базы знаний. 
  
  Из формулировки этой цели следует, что из пяти\linebreak сред предметной  
об\-ласти информатики (ментальная, информационная, цифровая, нейро-  
и~ДНК-сре\-да), рассмотренных в~работах~\cite{16-zz, 17-zz, 18-zz}, для 
проектируемой ИТ достаточно рассмотреть три \mbox{следующих} среды (см.\ 
рисунок), поскольку объекты нейро- и~ДНК-сред не используются в~этой~ИТ:
  \begin{enumerate}[(1)]
  \item \textit{ментальная среда}~--- это совокупность значений терминов 
как концептов знания экспертов (верхняя часть рисунка);
  \item \textit{информационная среда}~--- это совокупность описаний 
значений терминов, подготовленных экспертами (средняя часть);
  \item \textit{цифровая среда}~--- это совокупность компьютерных кодов 
терминологических профилей болезней корпоративной базы знаний (нижняя 
часть).
  \end{enumerate}
  
  2.\ \textit{Распределение моноприродных этапов ИТ и~полиприродных 
этапов ИТ с~входами и~выходами одной природы по средам согласно их 
природе.}
  
  \smallskip
  
  Рисунок иллюстрирует это достаточно общее теоретическое основание 
построения всего класса средов$\acute{\mbox{ы}}$х моделей на примере 
следующих \textit{шести} моноприродных этапов ИТ, обозначенных 
греческими буквами $\alpha$--$\zeta$:
  \begin{enumerate}[(1)]
  \item чтение в~цифровой среде терминологического портрета для его 
использования в~качестве критерия потенциальной новизны термина при 
поиске (этап~$\alpha$ на рисунке);\\[-15pt]
  \item чтение из электронной коллекции статей с~описанием исследуемой 
болезни как ИПНТ (этап~$\beta$);\\[-15pt]
  \item  чтение терминологического портрета для разделения новых 
терминов и~новых синонимов словарных терминов (этап~$\gamma$);\\[-15pt]
  \item  объединение КНТ с~аннотацией значения нового термина, 
включающей дефиницию этого значения (этап~$\delta$);\\[-15pt]
  \item добавление синонима к~словарному термину (этап~$\varepsilon$, 
стрелка чтения словарного термина на рисунке не показана, чтобы упростить 
его);\\[-15pt]
  \item объединение в~форме аннотации дефиниции концепта нового 
термина с~его контекстом, т.\,е.\ с~КНТ (этап~$\zeta$).
  \end{enumerate}
  
  Кроме шести моноприродных этапов на рисунке показаны \textit{три 
полиприродных этапа} этой ИТ с~входами и~выходами одной природы, но 
с~использованием внутри этапов сущностей другой среды. Распределение 
этих трех этапов по средам (двух этапов в~информационной среде и~одного 
этапа в~ментальной среде) следует из этого же теоретического основания, так 
как: 
  \begin{itemize}
  \item на этапе <<Семантический анализ>> эксперт изуча\-ет контекст ИПНТ 
(информационная среда), принимает решение о новизне термина и~его 
значения на основе своего понимания смысла контекста термина, т.\,е.\ 
своего личностного концепта (ментальная среда), на основе анализа которого 
проставляется метка <<новый термин>> или <<синоним словарного 
термина>> (информационная среда); таким образом, вход и~выход этого 
этапа ИТ принадлежат одной среде~--- \textit{информационной};
  \item на этапе <<Аннотирование нового термина>> эксперт анализирует 
его контекст (информационная среда) и~на основе своего понимания смыс\-ла 
контекста термина, т.\,е.\ своего личностного концепта (ментальная среда), 
формирует аннотацию (информационная среда); таким образом, вход 
и~выход этого этапа ИТ также принадлежат одной среде~--- 
\textit{информационной};
  \item на этапе <<Социализация>> эксперты на основе собственного 
понимания контекстного значения нового термина (ментальная среда) 
предлагают варианты дефиниций (информационная среда) и~пытаются 
сформировать согласованный между ними концепт дефиниции значения 
с~коллективным или организационным статусом (ментальная среда); таким 
образом, вход и~выход этого этапа ИТ принадлежат одной среде~--- 
\textit{ментальной} (возможный случай отсутствия согласования концепта 
экспертами на рисунке не показан).
  \end{itemize}
  
  3.\ \textit{Распределение этапов ИТ по границам между средами, если их 
вход и~выход принадлежат средам разной природы.}
  
  На рисунке показаны \textit{восемь этапов}, входы и~выходы которых 
принадлежат средам разной природы, включая шесть этапов ИТ на границе 
между информационной и~цифровой средами и~два этапа на границе между 
информационной и~ментальной средами:
  \begin{enumerate}[(1)]
  \item этап <<Визуализация ИПНТ>>, компьютерные коды которого 
(циф\-ро\-вая среда) преобразуются в~текст ИПНТ (информационная среда);
  \item этап <<Визуализация [терминологического] портрета>>, 
компьютерные коды которого (циф\-ро\-вая среда) преобразуются в~текст 
портрета (информационная среда);
  \item этап <<Оцифровка личностной аннотации>>, текст которой 
(информационная среда) преобразуется в~компьютерные коды 
терминологического портрета корпоративной базы знаний (циф\-ро\-вая среда);
  \item этап <<Визуализация [личностной] аннотации>>, компьютерные 
коды которой (циф\-ро\-вая среда) преобразуются в~текст (информационная 
среда);
  \item этап <<Оцифровка описания [словарного] термина>>, текст которого 
(информационная среда) преобразуется в~компьютерные коды 
терминологического портрета корпоративной\linebreak базы знаний (циф\-ро\-вая среда);
  \item этап <<Оцифровка согласованной аннотации>>, текст которой 
(информационная среда) преобразуется в~компьютерные коды 
терминологического портрета корпоративной базы знаний (циф\-ро\-вая среда);
  \item этап <<Интернализация>>, на котором у эксперта формируется 
личностное контекстное значение нового термина (ментальная среда) на 
основе аннотации с~дефиницией значения термина и~его контекста 
(информационная среда);
  \item этап <<Экстернализация>>, на котором эксперты формируют 
дефиницию значения термина (информационная среда) на основе 
согласованного ими концепта с~коллективным или организационным 
статусом (ментальная среда).
  \end{enumerate}
  
  Таким образом, согласно средов$\acute{\mbox{о}}$й модели актуализации 
терминологических портретов, 17~перечисленных этапов проектируемой ИТ 
можно распределить по трем категориям. К~первой из них относятся шесть 
моноприродных этапов ИТ, обозначенных греческими буквами  
$\alpha$--$\zeta$; ко второй~--- три этапа с~входами и~выходами одной 
природы, но с~использованием внут\-ри этапов сущностей среды другой 
природы; к~третьей~--- восемь этапов, входы и~выходы которых принадлежат 
средам разной природы. 
  
  На этапах первой категории отсутствуют интерфейсы между сущностями 
разной природы. При этом сами сущности могут быть только цифровыми 
или информационными. Каждый этап второй категории включает пару 
интерфейсов между сущностями разной природы. При этом сущности 
ментальной природы могут быть как на вхо\-де/вы\-хо\-де этапа (см.\ этап 
социализации), так и~внутри этапа (см. этапы семантического анализа 
и~аннотирования). Шесть из восьми этапов третьей категории включают 
традиционный интерфейс между сущностями информационной и~цифровой 
природы на границе между соответствующими средами. Этот интерфейс 
в~проектируемых ИТ может быть реализован, например, с~по\-мощью  
таб\-лиц Unicode. В~процессе проектирования ИТ наибольшую сложность 
представляет разработка этапов второй категории, а~также двух этапов на 
границе между ментальной и~информационной средами, так как они 
включают сущности ментальной природы. Вопрос их компьютеризации 
заслуживает отдельного рассмотрения и~выходит за рамки данной статьи.
  
  Рассмотренная средов$\acute{\mbox{а}}$я модель актуализации позволила 
при выполнении проекта до начала проектирования ИТ определить природу 
сущностей каж\-до\-го этапа и~локализовать этапы с~ментальными сущностями 
на их вхо\-де/вы\-хо\-де или внутри этапов, проб\-ле\-ма компьютеризации 
которых относится в~информатике к~категории проблем \textit{когнитивной 
сложности}~\cite{14-zz, 19-zz}.

\section{Эксперимент}

  В первый год выполнения проекта оставался открытым вопрос о степени 
полноты наборов терминов МЛР~[1--7], т.\,е.\ насколько п$\acute{\mbox{о}}$лно наборы 
терминов, используемых в~научных статьях и~других медицинских текстах 
при описании исследуемых болезней, представлены в~МЛР. Поэтому на 
втором году выполнения проекта был проведен эксперимент, краткое 
описание которого приведено ниже, для проверки гипотезы неполноты 
наборов терминов МЛР и~необходимости регулярной актуализации 
терминологических портретов болезней в~корпоративной базе знаний.
  
  В рамках проведенного эксперимента анализировались термины 
медицинских статей, содержащих описание кластеров 
кальцификации\footnote{Наличие кластеров кальцификации может быть использовано для 
диагностики заболеваний молочных желез.}. В~базе электронных биомедицинских 
публикаций PubMed был осуществлен поиск этих статей по запросу: 
(calcification[Abstract] AND (``neoplasms''[MeSH Terms] OR ``neoplasms''[All 
Fields] OR ``cancer''[All Fields])) AND (``breast''[MeSH Terms] OR ``breast''[All 
Fields]) AND (``2003/01/01''[PubDate]~: ``2021/12/31''[PubDate]). За период 
2003--2021~гг.\ найдены 844~статьи, которые распределились по трем 
временн$\acute{\mbox{ы}}$м периодам следующим образом: 2003--2009~гг.~--- 57~статей; 
2010--2015~гг.~--- 225~статей; 2016--2021~гг.~--- 562~статьи.
  
  При поиске использовались заголовки и~текс\-ты аннотаций статей. После 
автоматического поиска вхождений термина calcification (всего были 
найде\-ны~2658 его вхождений, но не каждое вхождение порождает 
терминоподобное словосочетание) экспертами был выполнен семантический 
анализ их контекстов, чтобы найти терминоподобные словосочетания, 
включающие термин calcification.
  
  В результате семантического анализа 2658~вхож\-де\-ний экспертно были 
выделены~93~терминоподобных словосочетания за период 2003--2021~гг. Из 
них~23 были обнаружены в~статьях 2003--2009~гг.\ (например, Amorphous 
Calcifications, Arterial Calcification, Aortic Calcification), 54~--- в~статьях  
2010--2015~гг., включая~31~терминоподобное словосочетание, которого не 
было в~стать\-ях 2003--2009 гг.\ (например, Arteriosclerotic Calcification),  
и~93~--- в~стать\-ях 2016--2021~гг., включая ранее не 
встре\-чав\-ши\-еся~39~терминов (например, Abnormal Calcification, Arterial 
Media(L) Calcification, Calcification Crystallite).
  
  Для проверки на словарный статус~93~терминоподобных словосочетаний, 
найденных экспертами в~статьях, использовался тезаурус MeSH~\cite{1-zz}. 
В~результате проверки лишь~3 из~93~терминоподобных словосочетаний 
совпали с~терминами тезауруса MeSH, т.\,е.\ имеют словарный статус.

\section{Заключение}

  Проведенный эксперимент подтвердил гипотезу о~существенной 
неполноте наборов терминов МЛР. При этом отсутствующие 
терминологические словосочетания уже используются в~научных стать\-ях 
и~других медицинских текстах при описании исследуемой болезни. 
Проведенный эксперимент показал необходимость регулярной актуализации\linebreak 
терминологических портретов болезней в~корпоративной базе знаний, что, 
в~свою очередь, под\-тверж\-да\-ет важ\-ность решения задачи моделирования 
и~создания ИТ регулярного и~\mbox{целенаправленного}\linebreak извлечения новых 
терминов из медицинских текс\-тов и~описания экспертами их значений 
в~терминологических профилях болезней.
  
{\small\frenchspacing
 {%\baselineskip=10.8pt
 %\addcontentsline{toc}{section}{References}
 \begin{thebibliography}{99}
\bibitem{1-zz}
Medical subject headings (MeSH) Thesaurus. {\sf https:// www.nlm.nih.gov/mesh/meshhome.html.}
\bibitem{2-zz}
Systematized nomenclature of medicine clinical terms (SNOMED CT). {\sf 
https://bioportal.bioontology.org/\linebreak ontologies/SNOMEDCT}.
\bibitem{3-zz}
Unified Medical Language System (UMLS). {\sf https:// www.nlm.nih.gov/research/umls/index.html.}
\bibitem{4-zz}
National Cancer Institute Thesaurus. {\sf 
https://\linebreak ncithesaurus.nci.nih.gov/ncitbrowser/pages/home.jsf?\linebreak version=22.08e.}
\bibitem{5-zz}
Medical dictionary of health terms.~--- Cambridge, MA, USA: Harvard Health Publishing, 2011. 
{\sf https:// www.health.harvard.edu/a-through-c}.
\bibitem{6-zz}
Dorland's illustrated medical dictionary.~--- 33rd ed.~--- Philadelphia, PA, USA: Elsevier 
Saunders, 2019. 2144~p.
\bibitem{7-zz}
MedTerms Medical Dictionary. {\sf  
https://www.\linebreak medicinenet.com/medterms-medical-dictionary/article.\linebreak htm}.
\bibitem{8-zz}
\Au{Зацман И.\,М.} Проб\-лем\-но-ори\-ен\-ти\-ро\-ван\-ная актуализация словарных 
статей двуязычных словарей и~медицинской терминологии: сопоставительный анализ~// 
Информатика и~её применения, 2021. Т.~15. Вып.~1. С.~94--101.
\bibitem{9-zz}
\Au{Зацман И.\,М.} Формы представления нового знания, извлеченного из текс\-тов~// 
Информатика и~её применения, 2021. Т.~15. Вып.~3. С.~83--90.
\bibitem{10-zz}
\Au{Zatsman I., Khakimova~A.} New knowledge discovery for creating terminological profiles 
of diseases~// 22nd European Conference on Knowledge Management Proceedings.~--- 
Reading, U.K.: Academic Publishing International Ltd., 2021. P.~837--846.
\bibitem{11-zz}
\Au{Nissen M.\,E.} Harnessing knowledge dynamics: Principled organisational knowing and 
learning.~--- London: IRM Press, 2006. 287~p.
\bibitem{12-zz}
\Au{Зацман И.\,М.} Компьютерная и~экономическая модели генерации нового знания: 
сопоставительный анализ~// Сис\-те\-мы и~средства информатики, 2021. Т.~31. №\,4.  
С.~84--96.
\bibitem{13-zz}
PubMed. National Library of Medicine. {\sf https://pubmed. ncbi.nlm.nih.gov}.
\bibitem{14-zz}
\Au{Зацман И.\,М.} Средовые модели информационных технологий: теоретические 
основания построения~// Информатика и~её применения, 2022. Т.~16. Вып.~3. С.~59--67.
\bibitem{15-zz}
\Au{Зацман И.\,М. Золотарев~О.\,В., Хакимова~А.\,Х.} Средовые модели извлечения из 
текста новых терминов и~индикаторов настроений~// Информатика и~её применения, 2022. 
Т.~16. Вып.~2. С.~60--67.
\bibitem{16-zz}
\Au{Зацман И.\,М.} Интерфейсы третьего порядка в~информатике~// Информатика и~её 
применения, 2019. Т.~13. Вып.~3. С.~82--89.
\bibitem{17-zz}
\Au{Зацман И.\,М.} Кодирование концептов в~цифровой среде~// Информатика и~её 
применения, 2019. Т.~13. Вып.~4. С.~97--106.
\bibitem{18-zz}
\Au{Зацман И.\,М.} Теоретические основания компьютерного образования: среды 
предметной области информатики как основание классификации ее объектов~//  
Сис\-те\-мы и~средства информатики, 2022. Т.~32. №\,4. С.~77--89.
\bibitem{19-zz}
\Au{Harel D.} Algorithmics~--- the spirit of computing.~--- Reading, MA, USA:  
Addison-Wesley, 1987. 514~p.
\end{thebibliography}

 }
 }

\end{multicols}

\vspace*{-6pt}

\hfill{\small\textit{Поступила в~редакцию 14.10.22}}

\vspace*{8pt}

%\pagebreak

%\newpage

%\vspace*{-28pt}

\hrule

\vspace*{2pt}

\hrule

%\vspace*{-2pt}

\def\tit{MODEL AND TECHNOLOGY FOR~DISCOVERING NEW TERMS IN~MEDICAL TEXTS}


\def\titkol{Model and technology for discovering new terms in medical texts}


\def\aut{I.\,M.~Zatsman$^1$, O.\,V.~Zolotarev$^2$, A.\,K.~Khakimova$^2$, and~Gu~Dongxiao$^3$}

\def\autkol{I.\,M.~Zatsman, O.\,V.~Zolotarev, A.\,K.~Khakimova, and~Gu~Dongxiao}

\titel{\tit}{\aut}{\autkol}{\titkol}

\vspace*{-8pt}


\noindent
$^1$Federal Research Center ``Computer Science and Control'' of the Russian Academy of Sciences,  
44-2~Vavilov\linebreak
$\hphantom{^1}$Str., Moscow 119333, Russian Federation

\noindent
$^2$Russian New University, 22~Radio Str., Moscow 105005, Russian Federation

\noindent
$^3$Hefei University of Technology, 193 Tunxi Road, Hefei, Anhui 230009, P.R.\ China


\def\leftfootline{\small{\textbf{\thepage}
\hfill INFORMATIKA I EE PRIMENENIYA~--- INFORMATICS AND
APPLICATIONS\ \ \ 2022\ \ \ volume~16\ \ \ issue\ 4}
}%
 \def\rightfootline{\small{INFORMATIKA I EE PRIMENENIYA~---
INFORMATICS AND APPLICATIONS\ \ \ 2022\ \ \ volume~16\ \ \ issue\ 4
\hfill \textbf{\thepage}}}

\vspace*{3pt} 
     



\Abste{The model of information technology for discovering new terms in medical texts, which belongs 
to the previously defined class of informatics' medium models, is considered. In the conducted 
experiment, the MeSH (Medical Subject Headers) dictionary is used to determine the novelty of terms 
which was created and updated by the National Library of Medicine (USA). The emergence of new terms 
is due to the representation (in medical papers and other scientific texts) of new knowledge about the 
studied diseases, methods of their treatment, and medicines used which has not yet been reflected in 
medical dictionaries and thesauri. In information systems of medical institutions, the proposed technology 
makes it possible to regularly update the profiles of the studied diseases corresponding to their subject 
domain. The aim of the paper is to describe the medium model of information technology for updating 
terminological profiles of diseases.}

\KWE{medium models in informatics; medical texts; terminological profile; discovering new terms in 
texts}



 \DOI{10.14357/19922264220412} 

\vspace*{-12pt}


 \Ack
\noindent
The reported study was funded by RFBR and NSFC, project number 21-57-53018.


\vspace*{5pt}

  \begin{multicols}{2}

\renewcommand{\bibname}{\protect\rmfamily References}
%\renewcommand{\bibname}{\large\protect\rm References}

{\small\frenchspacing
 {%\baselineskip=10.8pt
 \addcontentsline{toc}{section}{References}
 \begin{thebibliography}{99}
\bibitem{1-zz-1}
MeSH Thesaurus. Available at: {\sf https://www.nlm.nih. gov/mesh/meshhome.html} (accessed 
November~21, 2022). 
\bibitem{2-zz-1}
Systematized nomenclature of medicine clinical terms. Available at: {\sf 
https://bioportal.bioontology.\linebreak org/ontologies/SNOMEDCT} (accessed November~21, 2022).
\bibitem{3-zz-1}
Unified medical language system. Available at: {\sf https:// www.nlm.nih.gov/research/umls/index.html} 
(accessed November~21, 2022).
\bibitem{4-zz-1}
National Cancer Institute Thesaurus. Available at: {\sf https://ncit.nci.nih. gov/ncitbrowser} (accessed November~21, 2022).
\bibitem{5-zz-1}
\textit{Medical dictionary of health terms}. Cambridge, MA: Harvard Health Publishing. Available at: {\sf  
https:// www.health.harvard.edu/a-through-c} (accessed November~21, 2022).
\bibitem{6-zz-1}
\textit{Dorland's illustrated medical dictionary}. 2019. 33rd ed. Philadelphia, PA: Elsevier Saunders. 
2144~p.
\bibitem{7-zz-1}
{MedTerms medical dictionary}. Available at: {\sf  
https:// www.medicinenet.com/medterms-medical-dictionary/ article.htm} (accessed November~21, 2022).
\bibitem{8-zz-1}
\Aue{Zatsman, I.} 2021. Problemno-orientirovannaya ak\-tu\-a\-li\-za\-tsiya slovarnykh statey dvuyazychnykh 
slovarey i~me\-di\-tsin\-skoy terminologii: sopostavitel'nyy analiz [Problem- oriented updating of dictionary 
entries of bilingual \mbox{dictionaries} and medical terminology: Comparative analysis]. \textit{Informatika i~ee 
Primeneniya~--- Inform. Appl.} 15(1):94--101.
\bibitem{9-zz-1}
\Aue{Zatsman, I.} 2021. Formy predstavleniya novogo znaniya, izvlechennogo iz tekstov [Forms 
representing new knowledge discovered in texts]. \textit{Informatika i~ee Primeneniya~--- Inform. Appl.} 
15(3):83--90.
\bibitem{10-zz-1}
\Aue{Zatsman, I., and A.~Khakimova.} 2021. New knowledge discovery for creating terminological 
profiles of diseases. \textit{22nd European Conference on Knowledge Management Proceedings}. 
Reading, U.K.: Academic Publishing International Ltd. 837--846.
\bibitem{11-zz-1}
\Aue{Nissen, M.\,E.} 2006. \textit{Harnessing knowledge dynamics: Principled organizational knowing 
\& learning}. London: IRM Press. 287~p.
\bibitem{12-zz-1}
\Aue{Zatsman, I.} 2021. Komp'yuternaya i~ekonomicheskaya modeli generatsii novogo znaniya: 
sopostavitel'nyy ana\-liz [Computer and economic models of new knowledge generation: A~comparative 
analysis]. \textit{Sistemy i~Sredstva Informatiki~--- Systems and Means of Informatics} 31(4):\mbox{84--96}.
\bibitem{13-zz-1}
National library of medicine. PubMed. Available at: {\sf https://pubmed.ncbi.nlm.nih.gov} (accessed 
November~21, 2022).
\bibitem{14-zz-1}
\Aue{Zatsman, I.} 2022. Sredovye modeli informatsionnykh tekhnologiy: teoreticheskie osnovaniya 
postroeniya [Informatics' medium models of information technology: Theoretical foundations for their 
creating]. \textit{Informatika i~ee Primeneniya~--- Inform. Appl.} 16(3):59--67.
\bibitem{15-zz-1}
\Aue{Zatsman, I., O.~Zolotarev, and A.~Khakimova.} 2022. Sredovye modeli izvlecheniya iz teksta 
novykh terminov i~indikatorov nastroeniy [Medium models for discovering novel terms and sentiments 
from texts]. \textit{Informatika i~ee Primeneniya~--- Inform. Appl.} 16(2):60--67.
\bibitem{16-zz-1}
\Aue{Zatsman, I.\,M.} 2019. Interfeysy tret'ego poryadka v informatike [Third-order interfaces in 
informatics]. \textit{Informatika i~ee Primeneniya~--- Inform. Appl.} 13(3):82--89.
\bibitem{17-zz-1}
\Aue{Zatsman, I.\,M.} 2019. Kodirovanie kontseptov v~tsifrovoy srede [Digital encoding of concepts]. 
\textit{Informatika i~ee Primeneniya~--- Inform. Appl.} 13(4):97--106.
\bibitem{18-zz-1}
\Aue{Zatsman, I.} 2022. Teoreticheskie osnovaniya komp'yu\-ter\-no\-go obrazovaniya: sredy predmetnoy 
oblasti informatiki kak osnovanie klassifikatsii ee ob''ektov [Theoretical foundations of digital 
education: Subject domain media of informatics as the base of its objects' classification]. \textit{Sistemy 
i~Sredstva Informatiki~--- Systems and Means of Informatics} 32(4):77--89.
\bibitem{19-zz-1}
\Aue{Harel, D.} 1987. \textit{Algorithmics~--- the spirit of computing}. Reading, MA: Addison-Wesley. 
514~p.

\end{thebibliography}

 }
 }

\end{multicols}

\vspace*{-6pt}

\hfill{\small\textit{Received October 14, 2022}}

\Contr

\noindent
\textbf{Zatsman Igor M.} (b.\ 1952)~--- Doctor of Science in technology, head of department, Institute 
of Informatics Problems, Federal Research Center ``Computer Science and Control'' of the Russian 
Academy of Sciences, 44-2~Vavilov Str., Moscow 119333, Russian Federation; 
\mbox{izatsman@yandex.ru}

\vspace*{3pt}

\noindent
\textbf{Zolotarev Oleg V.} (b.\ 1959)~--- Candidate of Science (PhD) in technology, head of  
department, Russian New University, 22~Radio Str., Moscow 105005, Russian Federation;  
\mbox{ol-zolot@yandex.ru}

\vspace*{3pt}

\noindent
\textbf{Khakimova Aida Kh.} (b.\ 1963)~--- Candidate of Science (PhD) in biology, leading scientist, 
Russian New University, 22~Radio Str., Moscow 105005, Russian Federation; 
\mbox{aida\_khatif@mail.ru}

\vspace*{3pt}

\noindent
\textbf{Dongxiao Gu} (b.\ 1980)~-- Candidate of Science (PhD) in informatics, professor, Hefei University of Technology, 193 
Tunxi Road, Hefei, Anhui 230009, P.R.\ China; \mbox{dongxiaogu@yeah.net}

     
\label{end\stat}

\renewcommand{\bibname}{\protect\rm Литература}   %12
\def\stat{grusho}

\def\tit{АРХИТЕКТУРНЫЕ РЕШЕНИЯ В~ЗАДАЧЕ ВЫЯВЛЕНИЯ МОШЕННИЧЕСТВА ПРИ~АНАЛИЗЕ 
ИНФОРМАЦИОННЫХ ПОТОКОВ В~ЦИФРОВОЙ ЭКОНОМИКЕ$^*$}

\def\titkol{Архитектурные решения в~задаче выявления мошенничества при~анализе 
информационных потоков в
%~цифровой 
экономике}

\def\aut{А.\,А.~Грушо$^1$, М.\,И.~Забежайло$^2$, Н.\,А.~Грушо$^3$, 
Е.\,Е.~Тимонина$^4$}

\def\autkol{А.\,А.~Грушо, М.\,И.~Забежайло, Н.\,А.~Грушо, 
Е.\,Е.~Тимонина}

\titel{\tit}{\aut}{\autkol}{\titkol}

\index{Грушо А.\,А.}
\index{Забежайло М.\,И.}
\index{Грушо Н.\,А.}
\index{Тимонина Е.\,Е.}
\index{Grusho A.\,A.}
\index{Zabezhailo M.\,I.}
\index{Grusho N.\,A.}
\index{Timonina E.\,E.}


{\renewcommand{\thefootnote}{\fnsymbol{footnote}} \footnotetext[1]
{Работа частично поддержана РФФИ (проекты 18-29-03081 и~18-07-00274).}}


\renewcommand{\thefootnote}{\arabic{footnote}}
\footnotetext[1]{Институт проблем информатики Федерального исследовательского центра <<Информатика и~управление>> 
Российской академии наук, grusho@yandex.ru}
\footnotetext[2]{Институт проблем информатики Федерального исследовательского центра <<Информатика и~управление>> 
Российской академии наук, m.zabezhailo@yandex.ru}
\footnotetext[3]{Институт проблем информатики Федерального исследовательского центра <<Информатика и~управление>> 
Российской академии наук, info@itake.ru}
\footnotetext[4]{Институт проблем информатики Федерального исследовательского центра <<Информатика и~управление>> 
Российской академии наук, eltimon@yandex.ru}

\vspace*{-12pt}
   

 
  
  \Abst{Cформулирован подход к~исследованию некоторых видов мошенничества в~цифровой 
экономике с~использованием причинно-следственных связей. Во всех видах рассматриваемых 
мошенничеств должно наблюдаться несоответствие между целями финансовых транзакций 
и~реальной стоимостью достижения этих целей. Данные о транзакциях можно собирать, 
наблюдая информационные потоки, в~которых отражаются эти транзакции. Архитектура сбора 
данных и~их анализа может быть организована с~помощью распределенных реестров 
с~централизованным консенсусом, что позволяет создать аналог электронной бухгалтерской 
книги, фиксирующей финансово-экономическую деятельность субъектов цифровой экономики в~регионе. 
  Рассматриваемые методы выявления мошенничества основаны на противоречиях 
между действиями, описанными в~транзакциях, и~информацией, содержащейся в~планах, 
стандартах, прецедентах и~др. Рассмотрен метод, основанный на некоторой упрощенной схеме 
реализации абстрактного проекта. Для выявления противоречий необходимо проводить анализ 
от следствия к~причине, т.\,е.\ искать аномалии в~информации, описывающей порождение 
наблюдаемых следствий. 
  Показано, как в~реализации проекта можно выделять простые <<необходимые условия>> 
нарушения при\-чин\-но-след\-ст\-вен\-ных связей, т.\,е.\ множество <<необходимых условий>>, 
нарушение которых свидетельствует о наличии мошенничества. Это множество <<необходимых 
условий>> можно назвать метаданными для контроля проекта на выявление мошенничества.} 
 
 
  \KW{цифровая экономика; информационные потоки; при\-чин\-но-след\-ст\-вен\-ные связи; 
выявление мошеннических схем} 

\DOI{10.14357/19922264190204}
  
\vspace*{-4pt}


\vskip 10pt plus 9pt minus 6pt

\thispagestyle{headings}

\begin{multicols}{2}

\label{st\stat}

\section{Введение}

\vspace*{3pt}

  В работе сформулирован подход к~исследованию некоторых видов 
мошенничества в~цифровой экономике с~использованием  
при\-чин\-но-след\-ст\-вен\-ных связей. Рассматриваются три вида мошенничества, 
а именно:
  \begin{enumerate}[(1)]
\item отмыв денег; 
\item обман при выполнении договорных обязательств при реализации 
технических проектов (строительные проекты и~др.); 
\item незаконный вывод денег. 
\end{enumerate}

  Названные виды мошенничества могут быть сведены к~решению одного типа 
задач. Для отмывания денег источник должен заключать фиктивные контракты, 
в~соответствии с~которыми будут переводиться средства за заведомо ненужную 
работу и~материалы. 
  
  Мошенничество, связанное с~невыполнением договорных обязательств, связано 
со снижением качества услуг, качества и~количества закупаемых 
материалов, выполнением работ с~ненадлежащим качеством. 
  
  Вывод денег связан с~переводом средств фир\-мам-од\-но\-днев\-кам, которые 
заведомо не могут выполнить обязательства по контрактам, за которые им 
переводятся средства. 
  
  Таким образом, во всех трех видах рассматриваемых мошенничеств должно 
наблюдаться несоответствие между целями финансовых транзакций и~реальной 
стоимостью достижения этих целей. Данные о транзакциях можно собирать, 
наблюдая информационные потоки, в~которых отражаются эти транзакции. 
  
  Однако для наблюдения таких информационных потоков необходимо создавать 
архитектуру\linebreak телекоммуникационной системы, позволяющей перехватывать 
и~собирать данные о всех транзакциях. Например, такая архитектура может быть 
организована с~помощью распределенных реестров с~централизованным 
консенсусом, т.\,е.\ все информационные потоки, сформированные в~цифровой 
экономике и~несущие информацию о транзакциях, проходят через некоторый 
центральный узел, запоминающий их в~форме распределенного реестра. Такие 
реестры могут дублироваться в~аналогичных центрах различных регионов, что 
позволяет создать аналог электронной бухгалтерской книги, фиксирующей 
фи\-нан\-со\-во-эко\-но\-ми\-че\-скую деятельность субъектов цифровой экономики. Такой 
подход предложено реализовать на базе системы ситуационных центров, что 
отражено в~работах~[1, 2].
  
  Собранная из информационных потоков информация о~транзакциях, т.\,е.\ 
о~контрактах, договорах, платежах, отчетах, закупленных материалах, 
характеристиках исполнителей работ и~др., собирается в~базе данных в~указанном 
центре. Согласно теории интеллектуальных сис\-тем~[3], эту базу данных можно 
называть базой фактов (БФ). Базу фактов можно представить как бинарную мат\-ри\-цу, 
строки которой описывают характеристики, входящие в~транзакции, а столбцы 
нумеруются характеристиками. Строки матрицы будем называть 
\textit{объектами}~[4, 5]. 
  
  Рассматриваемые в~работе методы выявления мошенничества будут основаны 
на противоречиях между действиями, описанными в~транзакциях, и~информацией, 
содержащейся в~планах, стандартах, прецедентах и~др. Для нахождения 
противоречий в~архитектуре центра предусмотрена другая база данных~--- база 
знаний (БЗ)~\cite{3-gr, 6-gr}, которая устроена так же, как БФ. 
  
  Информация в~БЗ собирается на основе положительного опыта или расчетов. 
Используя БЗ, можно выводить факты нарушения при\-чин\-но-след\-ст\-вен\-ных 
связей. Нарушения при\-чин\-но-след\-ст\-вен\-ных связей будем называть 
\textit{аномалиями}. 
  
  Для упрощения дальнейшее изложение будет вестись в~рамках поиска 
противоречий при выполнении некоторого абстрактного проекта. Выявление 
аномалий будет происходить на основе фактов из БФ с~помощью знаний из БЗ 
методами искусственного интеллекта и~интеллектуального анализа 
данных~\cite{6-gr}. 

\vspace*{-10pt}
  
  \section{Модели}
  
  \vspace*{-3pt}
  
  Наиболее сложная из рассмотренных выше задач~--- выявление противоречий, 
т.\,е.\ использование БЗ для получения новых знаний и~выявление аномалий из 
полученных фактов. 
  
  Все способы выявления противоречий основаны на определении 
  причинно-следственных связей. При этом противоречия в~параметрах транзакций по 
отношению к~требуемым в~БЗ составляют сущность аномалий. 
  
   Далее будет рассмотрен метод, основанный на некоторой упрощенной схеме 
реализации абстрактного проекта. 
  
  Каждый проект имеет цель: например, цель представляет собой построение 
некоторой системы. Воспользуемся структурным подходом, который позволяет 
строить проект на основе разбиения системы на подсистемы и~определения 
взаимодействий подсистем~\cite{7-gr}. При этом каждая подсистема также 
представима структурной моделью. 
  
  Как сама система, так и~каждая ее подсистема имеют свой функционал 
и~спецификацию, па\-ра\-мет\-ры настройки и~домены параметров настройки. Кроме 
этих характеристик существует множество характеристик, связанных 
с~<<жизненным циклом>> создания системы. Сюда входят работы, ресурсы, 
сроки выполнения работ по созданию подсистем и~самой системы, стоимости 
компонентов и~материалов, стоимости работ, схемы поставок, договорные 
обязательства и~др. Все характеристики связаны между собой, поэтому можно 
говорить о стоимости и~времени изготовления структурных компонентов системы. 
  
  Одной из важнейших характеристик является смета (система смет для 
подсистем). Смета сопоставляет каждому компоненту системы стоимость его 
изготовления и~настройки. 
  
  Схема построения системы может быть пред\-став\-ле\-на диаграммой, 
изображенной на рис.~1. 

{ \begin{center}  %fig1
 \vspace*{9pt}
   \mbox{%
 \epsfxsize=79mm 
 \epsfbox{gru-1.eps}
 }


\vspace*{9pt}


\noindent
{{\figurename~1}\ \ \small{Диаграмма достижения цели}}
\end{center}
}

\vspace*{9pt}

\addtocounter{figure}{1}
  
  


  Представленная на рис.~1 диаграмма позволяет описать основные классы 
возможных противоречий при достижении цели. Противоречия возникают, когда 
данные БФ не соответствуют требуемым характеристикам. 
  
  
  \section{Потенциальные классы аномалий при~достижении цели}
  
  Выделим четыре потенциальных класса противоречий, которые показывают, 
каким образом нужно искать эти противоречия.
  
 
  Противоречие цели и~проекта (рис.~2) возникает при отсутствии обоснования 
или в~случае логического противоречия между возможностями проектируемого 
функционала и~целью системы. Отметим, что в~проект входят сроки, перечень 
работ, материалы, настройки, которые описываются соответствующими 
параметрами и~допустимыми значениями этих параметров. Проект формируется 
на основе БЗ и~расчетов, исходя из информации, полученной по аналогии 
с~другими проектами и~решениями, которые считаются апробированными. 
  
  Отметим, что цель порождает проект и~в этом смысле является причиной 
проекта. Однако для анализа противоречий необходимо двигаться по штриховой 
стрелке диаграммы (см.\ рис.~2) от проекта к~цели. В~самом деле, любой компонент 
проекта направлен на теоретическое достижение цели. Цель~--- сложный объект, 
поэтому в~проекте могут возникнуть характеристики, противоречащие хотя бы 
некоторым характеристикам цели. Это делает проект противоречивым, но вывод 
об этом может быть сделан только на уровне описания цели. 
  

  Противоречия между проектом и~его реализацией, исключая настройки 
(рис.~3), могут возникать, например, при закупке исполнителем материалов более 
низкого качества по более низким ценам, при попытках достижения требуемых 
сроков работы за счет снижения качества выполнения работ, за счет нахождения 
<<объективных>> причин для увеличения сроков работы и,~следовательно, 
увеличения цены реализации проекта. 


  Для выявления указанных противоречий необходимо двигаться по диаграмме 
(см.\ рис.~3) в~обратную сторону в~соответствии со~штриховыми стрелками. 
Действительно, выявить противоречия между характеристиками закупленных 
материалов и~требуемыми по проекту можно только при обращении к~проекту 
и~его спецификациям. Манипуляции со сроками работы также можно выявить 
только при обращении к~соответствующим расчетам в~проекте. Задержки в~сроках 
работы, связанные с~поставками материалов, можно определить только на 
предыдущем этапе диаграммы (см.\ рис.~3) в~описании проекта. 


  


  Противоречия между реализацией проекта и~его настройкой (рис.~4) возникает, 
когда не удается добиться требуемых значений параметров функционала, не 
удается обеспечить необходимый уровень\linebreak\vspace*{-12pt}

{ \begin{center}  %fig2
 \vspace*{-6pt}
   \mbox{%
 \epsfxsize=16mm 
 \epsfbox{gru-2.eps}
 }


\vspace*{6pt}


\noindent
{{\figurename~2}\ \ \small{Противоречия цели и~проекта}}
\end{center}
}

%\vspace*{9pt}

\addtocounter{figure}{1}

{ \begin{center}  %fig3
 \vspace*{6pt}
    \mbox{%
 \epsfxsize=79mm 
 \epsfbox{gru-3.eps}
 }


\end{center}

\vspace*{-2pt}


\noindent
{{\figurename~3}\ \ \small{Противоречия проекта и~его реализации (без настройки)}}
}

\vspace*{6pt}

\addtocounter{figure}{1}

{ \begin{center}  %fig4
 \vspace*{1pt}
   \mbox{%
 \epsfxsize=54.5mm 
 \epsfbox{gru-4.eps}
 }


\end{center}


\noindent
{{\figurename~4}\ \ \small{Противоречия реализации проекта и~его на\-стройки}}
}

%\vspace*{9pt}

\addtocounter{figure}{1}

{ \begin{center}  %fig5
 \vspace*{5pt}
    \mbox{%
 \epsfxsize=79mm 
 \epsfbox{gru-5.eps}
 }


\end{center}



\noindent
{{\figurename~5}\ \ \small{Противоречия цели и~достигнутой реализации проекта}}
}

\vspace*{6pt}

\addtocounter{figure}{1}

\noindent
 качества реализации проекта. Для 
определения противоречия в~настройках надо опять же двигаться по диаграмме 
(см.\ рис.~4) в~обратную сторону по штриховым стрелкам, так как для выявления 
характеристик результатов работы, которые не дают возможности реализации 
определенного функционала, необходимо иметь информацию о результатах этой 
работы. 


  



  Противоречие между целью и~достигнутой реализацией проекта (рис.~5) 
возникает, когда реализованная система не позволяет достичь цели. В~этом случае 
опять противоречие нужно искать, двигаясь от цели к~реальному достигнутому 
функционалу по штриховой стрелке (см.\ рис.~5).
  
  Суммируя положения, изложенные в~данном разделе, приходим к~выводу, что 
для выявления противоречий необходимо проводить анализ от следствия 
к~причине, т.\,е.\ искать аномалии в~информации, описывающей порождение 
наблюдаемых следствий. 
  
  
  \section{Связь противоречий и~причин}
  
  Прежде чем построить связь между причинами и~противоречиями, кратко 
опишем простейшую модель связи этих понятий. Причины и~противоречия будут 
сформулированы для представления компонентов системы как объектов, 
обладающих наборами известных характеристик~\cite{4-gr, 5-gr}. 
  
  Пусть $U\hm=\{\alpha, \beta, \ldots\}$~--- совокупность характеристик 
(пространство характеристик). Согласно~\cite{4-gr} \textit{объектом}~$O$ 
называется любое подмножество характеристик $O\hm\subseteq U$. Рассмотрим 
последовательность объектов, возможно в~различных пространствах 
характеристик. 
  
  \smallskip
  
  \noindent
  \textbf{Определение~1.}\ Объект~$P$ с~числом характеристик, большим или 
равным~2, является \textit{причиной} объекта (\textit{свойства})~$B$ в~цепочке 
наблюдаемых объектов тогда и~только тогда, когда выполнены следующие 
условия:
  \begin{enumerate}[(1)]
\item для каждого объекта~$C$, если $P\hm\subseteq C$, то $C\mapsto B$, где 
$C\mapsto B$ означает, что объект~$B$ присутствует в~объекте, следующем за 
объектом~$C$;
\item объект~$P$ является минимальным объектом, удовлетворяющим 
условию~1, а~именно: $\forall \alpha\hm\in P$ объект~$P\backslash \{\alpha\}$ 
не является причиной, т.\,е.\ $\exists C:\ \alpha\not\in C$, $P\backslash 
\{\alpha\}\hm\subseteq C$ и~$C\not\mapsto B$, где $C\not\mapsto B$ означает, 
что~$B$ не может содержаться в~объекте, следующем за объектом~$C$. 
\end{enumerate}

  Приведенное определение причины является упрощением причин, 
возникающих в~реальном мире. Например, реальные причины могут возникать\linebreak 
как совокупность характеристик из разных пространств. Одно следствие может 
порождаться разными причинами или возникать из внешних\linebreak и~ненаблюдаемых 
характеристик. Однако пред\-став\-лен\-ная далее формализация позволяет доступно 
изложить при\-чин\-но-след\-ст\-вен\-ные истоки противоречий, которые 
инициируют в~дальнейшем глубокое исследование рассматриваемых процессов.
  
  Будем считать, что для любого интересующего нас свойства~$B$ существует 
причина. Тогда справедлива следующая теорема.
  
  \smallskip
  
  \noindent
  \textbf{Теорема~1.}\ \textit{Для любого свойства~$B$ существует 
единственная причина}. 
  
  \smallskip
  
  \noindent
  Д\,о\,к\,а\,з\,а\,т\,е\,л\,ь\,с\,т\,в\,о\,.\ \ Доказательство будем вести от противного, 
т.\,е.\ предположим, что существуют две причины свойства~$B$: $P$ 
и~$P^\prime$, $P\hm\not= P^\prime$. Тогда существует $\alpha\hm\in U$, которое 
удовлетворяет одному из двух условий:
  \begin{itemize}
\item[(а)] $\alpha\in P$, $\alpha\notin P^\prime$;
\item[(б)] $\alpha\notin P$, $\alpha \in P^\prime$.
\end{itemize}

  Пусть выполняется условие~(б). Тогда $P^\prime\backslash \{\alpha\}$ не 
является причиной по условию~2 определения~1, т.\,е.\ $\exists C$ такое, что 
$\alpha\notin C$, $P^\prime\backslash \{\alpha\}\hm\subseteq C$ и~$C\not\mapsto B$. 
Но если~$B$ произошло и~$P$ его причина, то $C\mapsto B$, что противоречит 
предположению. Теорема~1 доказана.
  
  \smallskip
  
  \noindent
  \textbf{Лемма.} \textit{Если $P$~--- причина появления свойства~$B$, то 
объект~$B$ определяет существование свойства~$P$ в~объекте, 
предшествующем~$B$. }
  
  \smallskip
  
  \noindent
  Д\,о\,к\,а\,з\,а\,т\,е\,л\,ь\,с\,т\,в\,о\,.\ \ Из предположения, что у~каж\-до\-го 
свойства~$B$ есть причина, и~условия, что~$P$ является причиной~$B$, следует, 
что при появлении в~данных свойства~$B$ объект~$C$, предшествующий 
появлению~$B$, содержит как часть объект~$P$. Это следует из теоремы~1 
и~определения причины. 
  
  Докажем принцип <<необходимого условия>>, который, несмотря на простоту 
доказательства, будет играть в~дальнейшем существенную роль.
  
  \smallskip
  
  \noindent
  \textbf{Теорема~2.} \textit{Если~$P$~--- причина появления свойства~$B$ 
и~$A\hm\subseteq P$, то объект~$B$ определяет наличие свойства~$A$ 
в~объекте, предшествующем~$B$}. 
  
  \smallskip
  
  \noindent
  Д\,о\,к\,а\,з\,а\,т\,е\,л\,ь\,с\,т\,в\,о\,.\ \ Пусть в~данных имеется объект~$B$ 
и~$P\mapsto B$, тогда в~силу существования и~единственности причины~$B$ 
в~данных должен существовать объект~$C$, предшествующий~$B$ 
и~содержащий причину~$P$. Поскольку $A\hm\subseteq P$ и~$B$ содержит 
причину~$P$, то $B\mapsto A$. С~учетом леммы теорема~2 доказана.
  
  \smallskip
  
  Пусть даны пространства $U_1, U_2,\ldots$ и~имеется последовательность 
данных (процесс выполнения этапов проекта в~соответствии с~рис.~1) $A, B, 
\ldots$, где каждый объект является подмножеством некоторого 
пространства~$U_i$, $i\hm=1,\ldots$ Тогда в~объекте~$A$ присутствует 
причина~$P$ появления интересующего нас свойства~$C$ в~объекте~$B$. Пусть 
$P\hm\subseteq A$, тогда по теореме~2 $\forall \alpha\hm\in P$:  
$C\mapsto \{\alpha\}$, т.\,е.\ из появления~$C$ следует появление 
характеристики~$\alpha$ в~предшествующем объекте. Это необходимое условие 
того, что~$C$ удовлетворяет причинно-следственным связям развития процесса 
выполнения проекта. Если для~$C$ нет характеристики~$\alpha$, которую можно 
отнести к~причине~$C$, то можно считать, что нарушена  
при\-чин\-но-след\-ст\-вен\-ная связь и~$C$~--- аномальный объект. 
  
  \smallskip
  
  \noindent
  \textbf{Пример.} Если объект~$C$ состоит в~получении суммы~$a$ 
фирмой~$K$, то согласно теореме~2 в~пред\-шест\-ву\-ющем объекте должна 
существовать причина перевода суммы~$a$ на фирму~$K$. Если эта причина 
в~проекте отсутствует, то это можно считать признаком мошеннической схемы. 
Все проекты по предположению собираются из <<кубиков>>, содержащихся в~БЗ. 
Тогда можно сравнить цену объекта~$C$, породившего получение суммы~$a$, 
и~сумму, присутствующую в~смете проекта. Если разница велика, то это либо 
ошибка проекта, либо признак мошеннической схемы.
  
  \section{Поиск противоречий на~основе~принципа <<необходимых~условий>>}
   
  Как было показано в~разд.~3, нахождение противоречий соответствуют 
движению от следствия к~причине. Для каждого объекта в~наблюдаемых данных 
выявление причин его появления является трудоемкой задачей. Кроме того, при 
реализации контроля соблюдения при\-чин\-но-след\-ст\-вен\-ных связей на 
большом множестве участников экономической деятельности задача анализа 
причин становится трудоемкой. Поэтому процедуру контроля необходимо разбить 
на два этапа, где первый этап состоит в~анализе простых <<необходимых 
условий>> проявления мошенничества, когда используется хотя бы одна 
известная характеристика причины. Второй этап (в~режиме офлайн) состоит 
в~выявлении причин, позволяющих провести анализ источников мошеннических 
схем. 
  
  Один из подходов к~выбору <<необходимых условий>> состоит в~построении 
множества подцелей исходной цели проекта (структурный метод построения 
проекта~\cite{7-gr}). Каждая подцель описывается диаграммой на рис.~1, 
и~реализации подцелей должны образовывать полный функционал цели. Это 
является необходимым, но не достаточным условием достижения цели, так как 
при таком подходе отсутствует компонент согласования всех подцелей в~единую 
систему. Однако такой подход значительно упрощает анализ выполнения проекта 
на предмет поиска мошенничества. Если признаки мошенничества будут 
обнаружены в~реализации хотя бы одной из подцелей, то это значит, что 
мошенничество присутствует в~реализации всего проекта. 
  
  Аналогично в~реализации каждого этапа в~любой из подцелей можно выделять 
простые <<необходимые условия>> нарушения при\-чин\-но-след\-ст\-венн\-ых 
связей. 
  
  Таким образом, получается множество <<необходимых условий>>, нарушение 
которых свидетельствует о наличии мошенничества. Это множество 
<<необходимых условий>> можно назвать метаданными~[8, 9] для контроля 
проекта на выявление мошенничества. 
  
  
  \section{Заключение }
  
  В поиске противоречий необходимо от транзакций, соответствующих 
следствиям при\-чин\-но-след\-ст\-вен\-ных связей, переходить к~анализу причин 
наблюдаемых следствий. Это сложная задача, которая связана с~описанием причин 
определенных свойств. 
  
  В работе представлена модель, позволяющая строить множество необходимых 
условий соответствия наблюдаемого следствия вызвавшей его причине. Этот 
подход делает поиск противоречий вполне вычислимой задачей, но не гарантирует 
успех. 
  
  {\small\frenchspacing
 {%\baselineskip=10.8pt
 \addcontentsline{toc}{section}{References}
 \begin{thebibliography}{9}
\bibitem{1-gr}
\Au{Грушо А.\,А., Зацаринный~А.\,А., Тимонина~Е.\,Е.} Блокчейны цифровой экономики на базе 
системы ситуационных центров и~централизованного консенсуса~// Радиолокация, навигация, 
связь: Мат-лы XXV Междунар. научн.-технич. конф.~---
Воронеж: Издательский дом ВГУ, 2019. Т.~6. С.~183--191. 
\bibitem{2-gr}
\Au{Grusho A., Zatsarinny~A., Timonina~E.} A~system approach to information security in 
distributed ledgers on the situational centers platform.~---
Lecture notes in computer science ser.~--- Springer, 2019 
(in press).
\bibitem{3-gr}
\Au{Финн В.\,К.} Искусственный интеллект: Методология, применения, философия.~--- М.: 
Красанд, 2011. 448~с.

\bibitem{5-gr} %4
\Au{Аншаков~О.\,М., Фабрикантова~Е.\,Ф.} ДСМ-ме\-тод автоматического порождения 
гипотез: Логические и~эпистемологические основания.~--- М.: Либроком, 2009. 432~с.

\bibitem{4-gr} %5
\Au{Poelmans J., Elzinga~P., Viaene~S., Dedene~G.} Formal concept analysis in knowledge 
discovery: A~survey~// Conceptual structures: From information to intelligence~/ Eds.\ M.~Croitoru, 
S.~Ferr$\acute{\mbox{e}}$, and D.~Lukose.~--- Lecture notes in computer science 
ser.~--- Berlin--Heidelberg: Springer, 2010. Vol.~6208.  P.~139--153.

\bibitem{6-gr}
\Au{Панкратова~Е.\,С., Финн~В.\,К.} Автоматическое по\-рож\-де\-ние гипотез в~интеллектуальных 
системах.~--- М.: Либроком, 2009. 528~с. 
\bibitem{7-gr}
\Au{Денисов А.\,А., Колесников~Д.\,Н.} Теория больших систем управления.~--- Л.: Энергоиздат, 1982. 488~с.

\bibitem{9-gr}
\Au{Грушо А.\,А., Грушо Н.\,А., Забежайло~М.\,И., Смирнов~Д.\,В., Тимонина~Е.\,Е.} 
Параметризация в~прикладных задачах поиска эмпирических причин~// Информатика и~её 
применения, 2018. Т.~12. Вып.~3. С.~62--66.

\bibitem{8-gr}
\Au{Грушо А.\,А., Грушо Н.\,А., Левыкин~М.\,В., Тимонина~Е.\,Е.} Методы идентификации 
захвата хоста в~распределенной ин\-фор\-ма\-ци\-он\-но-вы\-чис\-ли\-тель\-ной сис\-те\-ме, 
защищенной с~помощью метаданных~// Информатика и~её применения, 2018. Т.~12. Вып.~4. 
С.~41--45.

 \end{thebibliography}

 }
 }

\end{multicols}

\vspace*{-3pt}

\hfill{\small\textit{Поступила в~редакцию 03.04.19}}

%\vspace*{8pt}

%\pagebreak

\newpage

\vspace*{-28pt}

%\hrule

%\vspace*{2pt}

%\hrule

%\vspace*{-2pt}

\def\tit{ARCHITECTURAL DECISIONS IN~THE~PROBLEM 
OF~IDENTIFICATION OF~FRAUD IN~THE~ANALYSIS 
OF~INFORMATION FLOWS IN~DIGITAL ECONOMY\\[-5pt]}


\def\titkol{Architectural decisions in~the~problem 
of~identification of~fraud in~the~analysis 
of~information flows in~digital economy}

\def\aut{A.\,A.~Grusho, M.\,I.~Zabezhailo, N.\,A.~Grusho, and~E.\,E.~Timonina}

\def\autkol{A.\,A.~Grusho, M.\,I.~Zabezhailo, N.\,A.~Grusho, and~E.\,E.~Timonina}

\titel{\tit}{\aut}{\autkol}{\titkol}

\vspace*{-13pt}


 \noindent
   Institute of Informatics Problems, Federal Research Center ``Computer Sciences and 
Control'' of the Russian Academy of Sciences; 44-2~Vavilov Str., Moscow 119133, 
Russian Federation

\def\leftfootline{\small{\textbf{\thepage}
\hfill INFORMATIKA I EE PRIMENENIYA~--- INFORMATICS AND
APPLICATIONS\ \ \ 2019\ \ \ volume~13\ \ \ issue\ 2}
}%
 \def\rightfootline{\small{INFORMATIKA I EE PRIMENENIYA~---
INFORMATICS AND APPLICATIONS\ \ \ 2019\ \ \ volume~13\ \ \ issue\ 2
\hfill \textbf{\thepage}}}

\vspace*{3pt}


   
     
   \Abste{An approach to a~research of some types of fraud in digital economy with the usage of relationships of 
cause and effect is formulated. In all types of the considered frauds, the discrepancy between the 
purposes of financial transactions and actual cost of achievement of these purposes
has to be observed. Data on 
transactions can be collected by observing information flows in which these transactions are reflected. 
The architecture of data collection and their analysis can be organized by means of the distributed 
ledgers with the centralized consensus that allows creating an analog of the electronic account book 
fixing financial and economic activity of subjects of digital economy in the region. 
   The methods of fraud identification considered are based on the contradictions 
between actions described in transactions and information, which is contained in plans, standards, 
precedents, etc. 
   The method based on a~simplified scheme of implementation of the abstract project is considered. 
For identification of contradictions, it is necessary to carry out the analysis from the effect to the cause, 
i.\,e., to look for anomalies in information describing the generation of the observed effects. 
   It is shown how in implementation of the project it is possible to allocate simple ``necessary 
conditions'' of violation of cause and effect relationships, i.\,e., a~set of ``necessary conditions'' 
violation of which demonstrates fraud existence. It is possible to call this set of "necessary conditions" 
by metadata for control of the project for fraud identification.} 
   
   \KWE{digital economy; information flows; relationships of reason and effect; detection of 
fraudulent schemes}
   
  

 \DOI{10.14357/19922264190204}

\vspace*{-20pt}

 \Ack
   \noindent
   The work was partially supported by the Russian Foundation for Basic Research (projects  
18-29-03081 and 18-07-00274).



%\vspace*{6pt}

  \begin{multicols}{2}

\renewcommand{\bibname}{\protect\rmfamily References}
%\renewcommand{\bibname}{\large\protect\rm References}

{\small\frenchspacing
 {\baselineskip=10.5pt
 \addcontentsline{toc}{section}{References}
 \begin{thebibliography}{9}
\bibitem{1-gr-1}
\Aue{Grusho, A.\,A., A.\,A.~Zatsarinny, and E.\,E.~Timonina.} 2019. Blokcheyny tsifrovoy ekonomiki 
na baze sistemy situatsionnykh tsentrov i~tsentralizovannogo konsensusa [Blockchains of digital 
economy on the basis of the system of the situational centres and the centralized consensus]. 
\textit{25th Scientific and Technical Conference (International) ``Radar-Location, Navigation, 
Communication'' Proceedings}. Voronezh: VSU Publs. 6:183--191.
\bibitem{2-gr-1}
\Aue{Grusho, A., A.~Zatsarinny, and E.~Timonina.} 2019 (in press). 
A~system approach to information security 
in distributed ledgers on the situational centers platform. 
Lecture notes in computer science ser. Springer.
\bibitem{3-gr-1}
\Aue{Finn, V.\,K.} 2011. \textit{Iskusstvennyy intellekt: Metodologiya, primeneniya, filosofiya} 
[Artificial intelligence: Methodology, applications, philosophy]. Moscow: KRASAND. 448~p.

\bibitem{5-gr-1}
\Aue{Anshakov, O.\,M., and E.\,F.~Fabrikantova}. 2009. \textit{DSM-metod avtomaticheskogo porozhdeniya gipotez: Logicheskie 
i~epistemologicheskie osnovaniya} [JSM-method of automatic hypothesis generation: Logical and 
epistemological]. Moscow: KD LIBROKOM. 432~p.
\bibitem{4-gr-1} %5
\Aue{Poelmans, J., P.~Elzinga, S.~Viaene, and G.~Dedene.} 2010. Formal concept analysis in 
knowledge discovery: A~survey. \textit{Conceptual structures: From information to intelligence}. 
Eds.\ M.~Croitoru, S.~Ferr$\acute{\mbox{e}}$, and D.~Lukose. Lecture notes in 
computer science ser. Berlin--Heidelberg: Springer. 6208:139--153.

\bibitem{6-gr-1}
\Aue{Pankratov, E.\,S., and V.\,K.~Finn}. 
2009. \textit{Avtomaticheskoe porozhdenie gipotez v~intellektual'nykh 
sistemakh} [Automatic hypotheses generation in intelligent systems]. Moscow: KD 
\mbox{LIBROKOM}.  528~p. 
\bibitem{7-gr-1}
\Aue{Denisov, A.\,A., and D.\,N.~Kolesnikov.} 1982. \textit{Teoriya bol'shikh 
sistem upravleniya} [Theory of big control systems]. Leningrad: Energoizdat. 488~p.

\bibitem{9-gr-1}
\Aue{Grusho, A.\,A., N.\,A.~Grusho, M.\,I.~Zabezhailo, D.\,V.~Smirnov, and 
E.\,E.~Timonina.} 2018. 
Parametrizatsiya v~prikladnykh zadachakh poiska empiricheskikh prichin 
[Parametrization in applied 
problems of search of the empirical reasons]. 
\textit{Informatika i~ee Primeneniya~--- 
Inform. Appl.} 12(3):62--66.

\bibitem{8-gr-1}
\Aue{Grusho, A.\,A., N.\,A.~Grusho, M.\,V.~Levykin, and E.\,E.~Timonina.} 2018. Metody 
identifikatsii zakhvata khosta v~raspredelennoy informatsionno-vychislitel'noy sisteme, 
zashchishchennoy s~pomoshch'yu metadannykh [Methods of identification of host capture 
in the  distributed information system which is protected on the base of meta data].
\textit{Informatika i~ee 
Primeneniya~--- Inform. Appl.} 12(4):41--45.
{ %\looseness=1

}

\end{thebibliography}

 }
 }

\end{multicols}

\vspace*{-12pt}

\hfill{\small\textit{Received April 3, 2019}}

%\pagebreak

%\vspace*{-18pt}

\Contr

\noindent
\textbf{Grusho Alexander A.} (b.\ 1946)~--- Doctor of Science in physics and 
mathematics, professor, principal scientist, Institute of Informatics Problems, 
Federal Research Center ``Computer Sciences and Control'' of the Russian 
Academy of Sciences; 44-2~Vavilov Str., Moscow 119133, Russian Federation; 
\mbox{grusho@yandex.ru} 

\vspace*{3pt}

\noindent
\textbf{Zabezhailo Michael I.} (b.\ 1956)~--- Doctor of Science in physics and 
mathematics, principal scientist, Institute of Informatics Problems, Federal Research 
Center ``Computer Sciences and Control'' of the Russian Academy of Sciences;  
44-2~Vavilov Str., Moscow 119133, Russian Federation; 
\mbox{m.zabezhailo@yandex.ru} 

\vspace*{3pt}


\noindent
\textbf{Grusho Nikolai A.} (b.\ 1982)~--- Candidate of Science (PhD) in physics 
and mathematics, senior scientist, Institute of Informatics Problems, Federal 
Research Center ``Computer Sciences and Control'' of the Russian Academy of 
Sciences; 44-2~Vavilov Str., Moscow 119133, Russian Federation; 
\mbox{info@itake.ru} 

\vspace*{3pt}


\noindent
\textbf{Timonina Elena E.} (b.\ 1952)~--- Doctor of Science in technology, 
professor, leading scientist, Institute of Informatics Problems, Federal Research 
Center ``Computer Sciences and Control'' of the Russian Academy of Sciences;  
44-2~Vavilov Str., Moscow 119133, Russian Federation; 
\mbox{eltimon@yandex.ru} 

\label{end\stat}

\renewcommand{\bibname}{\protect\rm Литература}   %13
\def\stat{stupnikov}

\def\tit{ВЕРИФИЦИРУЕМОЕ ОТОБРАЖЕНИЕ МОДЕЛИ ДАННЫХ, ОСНОВАННОЙ НА~МНОГОМЕРНЫХ МАССИВАХ, 
В~ОБЪЕКТНУЮ~МОДЕЛЬ ДАННЫХ$^*$}

\def\titkol{Верифицируемое отображение модели данных, основанной на~многомерных массивах, 
в~объектную модель данных}

\def\autkol{С.\,А.~Ступников}

\def\aut{С.\,А.~Ступников$^1$}

\titel{\tit}{\aut}{\autkol}{\titkol}

{\renewcommand{\thefootnote}{\fnsymbol{footnote}}\footnotetext[1] {Работа 
выполнена при поддержке РФФИ (проект 11-07-00402-а). Статья рекомендована к 
публикации в журнале Программным комитетом конференции <<Электронные 
библиотеки: перспективные методы и технологии, электронные коллекции>> 
(RCDL-2012).}}

\renewcommand{\thefootnote}{\arabic{footnote}}
\footnotetext[1]{Институт проблем информатики Российской академии наук, 
ssa@ipi.ac.ru}

\vspace*{-6pt}       

\Abst{Рассматривается отображение модели данных, основанной на 
многомерных мас\-си\-вах (ММ-мо\-де\-ли), в объектную модель данных. Изложены 
общие принципы отображения ММ-мо\-де\-лей в объектные модели данных. 
Рассмотрено отображение конкретной модели~--- Array Data Model (ADM), 
использующейся в системе управления базами данных (СУБД) SciDB, в язык СИНТЕЗ, 
использующийся в качестве канонической модели данных в технологии предметных 
посредников. Проиллюстрирован метод верификации отображения~--- доказательства 
сохранения информации и семантики операций при отображении. Верификация 
осуществляется при помощи формального языка спецификаций AMN. Практической 
целью работы ставилось создание базы для виртуальной или материализованной 
интеграции ресурсов, основанных на многомерных массивах.}

\vspace*{-1pt}

\KW{многомерные массивы; объектная модель данных; отображение моделей 
данных; интеграция баз данных}

\vspace*{-6pt}

\vskip 14pt plus 9pt minus 6pt

      \thispagestyle{headings}

      \begin{multicols}{2}

            \label{st\stat}
            

\section{Введение}

        Развитие науки и промышленности, широкое распространение 
информационных технологий ведет к накоплению огромных объемов данных 
как в науке, так и в бизнесе. Данные могут быть как наблюдательными, 
экспериментальными, так и полученными в ходе компьютерного 
моделирования. Данные таких масштабов (часто измеряемых уже в петабайтах) 
называются <<большими данными>> (Big Data)~\cite{1-stu}. Они плохо 
поддаются обработке и анализу в рамках широко известных технологий баз 
данных, опирающихся в основном на реляционную модель данных.
        
        Именно поэтому развиваются различные модели данных, нацеленные на 
параллельную обработку и анализ данных в распределенных средах~--- гридах 
и облаках. Важными видами таких моделей являются модели данных, 
основанные на многомерных массивах (array-based data models, или ADM) 
и называемые далее ММ-мо\-де\-ля\-ми. Родственны данным моделям 
так называемые <<кубы данных>>, используемые в 
OLAP (online analytical processing) тех\-но\-ло\-гии~[2--4]. 
Исследования ММ-мо\-де\-лей начались достаточно 
давно~\cite{4-stu, 5-stu} и продолжают развиваться. В~данной статье 
рассматривается конкретная модель, а именно модель, используемая в СУБД 
SciDB~\cite{6-stu}.
        
        История SciDB начинается с 2007~г., когда на симпозиуме по 
экстремально большим базам данных (XLDB~--- extremely large data bases) 
представителями науки и 
промышленности был сделан вывод о том, что существующие СУБД не в 
состоянии манипулировать объемами данных, которые появятся в ближайшем 
будущем. Одним из примеров поставщиков таких данных служит строящийся 
телескоп LSST (Large Synoptic Survey Telescope)~\cite{7-stu}. Был также сделан 
вывод о необходимости разработки СУБД нового поколения, которая должна 
удовлетворять, в частности, следующим требованиям~\cite{8-stu}:
        \begin{itemize}
\item модель данных основывается на многомерных массивах, а не на 
кортежах;
\item модель хранения базируется на версионности, а не на обновлении 
значений;
\item масштабируемость до сотен петабайт и высокая отказоустойчивость;
\item СУБД является свободно распространяемым программным 
обеспечением.
\end{itemize}

        Некоторое время спустя был запущен международный проект под 
руководством Майкла Стоунбрейкера, целью которого стало создание новой 
СУБД, получившей название SciDB. В~настоящее 
время свободно распространяется очередная версия системы для операционных
сис\-тем (ОС) Ubuntu и  RedHat.
        
        Целью данной статьи является исследование вопроса о верифицируемом 
отображении ММ-мо\-де\-лей, и в частности ADM~\cite{9-stu}, 
использующейся в системе SciDB, в объектные 
модели данных для виртуальной или материализованной интеграции ресурсов 
при создании федеративных баз данных или хранилищ данных. 
        
        При материализованной интеграции предполагается создание 
хранилища данных (warehouse), в которое загружаются ресурсы, подлежащие 
интеграции. В~процессе загрузки происходит преобразование данных из схемы 
ресурса в общую схему хранилища.
        
        Виртуальная же интеграция рассматривается в статье применительно к 
предметным посредникам~\cite{10-stu}. Предметные посредники представляют 
собой специальный вид программного обеспечения (ПО), образующий 
промежуточный слой между пользователем (приложением) и неоднородными 
информационными ресурсами. При этом данные из ресурсов не 
материализуются в посреднике. Федеративная схема посредника, описывающая 
некоторую предметную область, создается независимо от существующих 
ресурсов. Ресурсы, релевантные предметной области, затем регистрируются в 
посреднике~--- их схемы связываются специальными семантическими 
отображениями с федеративной схемой. Исполнительная среда посредников 
предо\-став\-ля\-ет возможность пользователям (приложениям) задавать запросы 
(программы) к посреднику в терминах федеративной схемы. Эти запросы 
переписываются в частичные запросы над информационными ресурсами, а 
затем исполняются на ресурсах. Результаты частичных запросов объединяются 
и выдаются пользователю также в терминах федеративной схемы.
        
        Важным понятием технологии систем интеграции баз данных является 
каноническая модель, служащая общим языком, унифицирующим 
разнообразные модели ресурсов.
        
        Необходимым предусловием интеграции ресурсов, основанных на 
многомерных массивах, является построение отображения соответствующей\linebreak 
ММ-мо-де\-ли в каноническую модель данных, сохраняющего информацию и 
семантику операций языка манипулирования данными (ЯМД)~\cite{11-stu}. 
Это обусловлено тем, что семантические отображения, связывающие 
федеративную схему и схемы ресурсов, нужно проводить в единой 
(канонической) модели~\cite{12-stu}. Отображение должно быть 
верифицируемым~--- доказуемо правильным. 
        
        В качестве канонической модели в данной работе рассматривается язык 
СИНТЕЗ~\cite{13-stu}~--- комбинированная слабоструктурированная и 
объектная модель данных, нацеленная на разработку предметных посредников 
для решения задач в средах неоднородных ресурсов. Разработан прототип 
программных средств для поддержки среды предметных посредников с языком 
СИНТЕЗ в роли канонической модели~\cite{14-stu}.
        
        С точки зрения предметных посредников СУБД, основанные на 
многомерных массивах, пред\-став\-ля\-ют собой новый вид ресурсов, подлежащих 
интеграции в посредниках вместе с привычными ресурсами~--- реляционными 
и объектными СУБД, веб-сер\-ви\-са\-ми и~т.\,д. 
        
        Нужно отметить, что ADM подвергается некоторой критике со стороны 
исследователей, продолжающих развитие моделей, основанных на 
многомерных массивах. Так, авторы языка SciQL~\cite{15-stu} отмечают, что 
язык ADM является смесью SQL и деревьев алгебраических операций. По их 
мнению, язык для СУБД, основанных на многомерных массивах, должен быть 
интегрирован с синтаксисом и семантикой SQL:2003. Несмотря на эти 
замечания, модель ADM представляет несомненный практический интерес для 
интеграции баз данных. SciDB используется как в научных проектах, связанных 
с LSST (предполагается после запуска телескопа) и физикой высоких энергий, 
так и в коммерческих, связанных с генетикой, страхованием, финансами. 
Сравнительное тестирование SciDB с СУБД Postgres и статистическим ПО R 
показало преимущества SciDB по производительности и масштабируемости.
        
        Статья организована следующим образом. В~разд.~2 рассмотрены и 
проиллюстрированы основные принципы отобра\-же\-ния модели данных ADM в 
язык СИНТЕЗ. Принципы обобщены на случай моделей, отличных от ADM и 
СИНТЕЗ. В~разд.~3 рассмотрен метод доказательства сохранения информации 
и семантики операций при отоб\-ра\-же\-нии моделей с использованием 
формального языка спецификаций AMN~\cite{16-stu}. Метод 
проиллюстрирован на структурах данных и операциях ЯМД моделей SciDB и 
СИНТЕЗ. В~разд.~4 рассмотрены некоторые родственные исследования и 
направления дальнейшей работы.

\vspace*{6pt}

\section{Отображение модели ADM в~язык СИНТЕЗ}

\vspace*{2pt}

        SciDB поддерживает два языка для работы с массивами: AQL (Array 
Query Language) и AFL (Array\linebreak Functional Language). Язык AQL является 
        SQL (Structured Query Language)
        по\-доб\-ным декларативным языком, включающим как операции 
языка описания данных (ЯОД), так и операции ЯМД. Язык AFL представляет собой функциональный язык 
манипулирования массивами, операции которого можно объединять в 
композиции. Допускается использование операций AFL в запросах AQL.
        
        Операции языков и отображение будут иллюстрироваться на 
адаптированных примерах из сценария применения SciDB в области 
оптической астрономии~\cite{17-stu}, а также на простых примерах из 
документации SciDB~\cite{9-stu}.

\subsection{Отображение языка определения данных}

        Отображение ЯОД в данном разделе описывается независимо от вида 
интеграции~--- виртуальной или материализованной.
        
        Основной единицей определения данных в модели ADM является 
массив, имеющий конечное количество {измерений} $d_1, d_2, \ldots , 
d_n$~[9]. Длиной измерения называется количество упорядоченных значений в 
этом измерении. По умолчанию типом измерения являются 64-бит\-ные целые 
числа. Поддерживаются также нецелочисленные измерения, например строки 
или числа с плавающей точкой. Каждая комбинация значений измерений 
соответствует ячейке массива, которая может содержать конечное количество 
значений, называемых \textit{атрибутами}. Типом атрибута может быть один 
из встроенных типов ADM~\cite{9-stu}.
        
        Основная операция ЯОД ADM~--- создание массива~--- выглядит 
следующим образом:
        \begin{verbatim}
CREATE ARRAY source
< ampExposureId: int64 NULL, 
   filterId: int8,
   apMag: double >
[ ra(double), de(double), objectId=0:*];
\end{verbatim}

        Создается массив оптических источников {\sf source}, измерениями 
которого являются координаты {\sf ra} и {\sf de} типа {\sf double} и целочисленный 
идентификатор объекта. Для целочисленного измерения указаны его нижняя (0) 
и верхняя (<<*>>, обозна\-ча\-ющая бесконечность) границы. Ячейка массива 
состоит из трех атрибутов: {\sf ampExposureId}, {\sf filterId}, 
{\sf apMag}. Указано, что 
атрибут {\sf ampExposureId} может принимать неопределенное значение {\sf NULL}. 
В~данном примере приведены только некоторые из реально используемых 
атрибутов и измерений.
        
        В языке СИНТЕЗ создание массива представляется определением 
одноименного класса:
        \begin{verbatim}
{ source; in: class;
  instance_type:{
  double ra;
  ra2long: {in: function; 
            params: {-ret/long}; };
  double de;
  de2long: {in: function; 
            params: {-ret/long}; };
  long objectId; metaslot lower: 0;  
  higher: inf; end
  objectIdBounds: {in: invariant;
    {{all s(source(s) -> s.objectId >= 0)}}
  };
  long ampExposureId;
  short filterId;
  double apMag;
  key: { unique; { ra, de, objectId } };
  definiteness: {obligatory;
    { ra, de, objectId, filterId, apMag } };
  };
}
\end{verbatim}

        Как измерения, так и атрибуты, составляющие ячейку, представляются в 
языке СИНТЕЗ атрибутами типа экземпляров ({\sf instance\_type}) класса. Между 
встроенными типами ADM ({\sf int8}, {\sf int64}, {\sf double} и~др.)\ и встроенными 
типами языка \mbox{СИНТЕЗ} ({\sf short}, {\sf long}, {\sf double}) устанавливается взаимно 
однозначное соответствие. Совокупность атрибутов, со\-от\-вет\-ст\-ву\-ющих 
измерениям, объявляется уникальной (инвариант {\sf key}, выражаемый 
встроенным утверждением {\sf unique}). Объявляется также, что атрибуты, 
соответствующие измерениям и не-{\sf NULL} атрибутам ADM, должны быть 
определены у всех экземпляров класса (инвариант {\sf definiteness}, выражаемый 
встроенным утверждением {\sf obligatory}).
        
        Таким образом обеспечивается сохранение отличи\-тель\-ных свойств 
многомерных массивов (<<кубов данных>>), существенным образом 
раз\-ли\-ча\-ющих измерения и атрибуты, со\-став\-ля\-ющие \mbox{ячейку}:
        \begin{itemize}
\item по набору значений измерений однозначно определяется набор 
значений атрибутов ячейки (уникальность измерений);
\item ячейка массива всегда определяется полным набором значений 
измерений (определенность измерений).
\end{itemize}

        Заметим также, что отсутствие в коллекции объекта с некоторым 
набором значений измерений означает \textit{пустую ячейку} в массиве.
        
        Для нецелочисленных измерений {\sf ra} и {\sf de} в языке СИНТЕЗ кроме 
атрибутов определяются функции {\sf ra2long}, {\sf de2long}, преобразующие 
нецелочисленные значения в целочисленные. Необходимость при\-вне\-се\-ния этих 
функций вызвана следующим. При попытке описать операции, характерные для 
ММ-мо\-де\-лей, в объектной модели (в частности, в языке СИНТЕЗ) 
выясняется необходимость применения принципиально различных механизмов 
работы с целочисленными и нецелочисленными измерениями. Это вызвано 
различием типов измерений, возможной неравномерностью шага измерения 
и~т.\,д.\linebreak Для того чтобы обеспечить возможность единообразного описания 
операций над цело\-чис\-лен\-ными и нецелочисленными измерениями и 
необходимы функции, приводящие нецелочисленные\linebreak измерения к 
целочисленным.
        
        Ограничения, связанные с нижними и верхними границами 
целочисленных измерений, пред\-став\-ля\-ют\-ся в языке СИНТЕЗ, во-пер\-вых, 
мета\-слотом соответствующего атрибута (например,\linebreak {\sf objectId}). В~метаслоте 
хранится метаинформация, связанная с атрибутом как с отдельной сущностью 
языка. В~данном случае метаслот включает два слота {\sf lower} и {\sf higher}, 
отвечающих соответственно верхней и нижней границе измерения. 
        Во-вто\-рых, создается инвариант (например, {\sf objectIdBounds}), 
предикативная спецификация которого устанавливает ограничения на значения 
измерения для каждого из объектов класса, отвечающего массиву. 
Спецификация инварианта имеет вид формулы первого порядка, где {\sf all}~--- 
квантор существования, <<\verb -> >> --- импликация.
        
        Необходимо отметить, что массив представляется в объектной модели 
множеством объектов класса (фактически кортежей значений атрибутов). При 
этом наблюдается некоторое противоречие со стремлением создателей 
        ММ-мо\-де\-лей \mbox{отойти} от моделей, основанных на кортежах. Однако в 
контексте интеграции ресурсов ММ-мо\-де\-ли это лишь один класс из 
большого множества разнообразных классов моделей данных. Представление 
специфических ММ-мо\-де\-лей в объектной модели является методологически 
и технически гораздо более простым и естественным, нежели использование 
многомерных массивов в качестве канонической модели.
        
        Изложенные принципы отображения ЯОД могут быть обобщены на 
случай, когда канонической является объектная или 
        объ\-ект\-но-ре\-ля\-ци\-он\-ная модель, отличная от языка СИНТЕЗ. 
Также не принципиален выбор модели данных, основанной на многомерных 
массивах. В~общем виде принципы отображения ЯОД выглядят следующим 
образом:
        \begin{itemize}
\item массив отображается в коллекцию типизированных объектов (класс) 
объектной модели;
\item измерения и атрибуты, составляющие ячейку массива, отображаются в 
атрибуты типа экземпляров класса;
\item между встроенными типами модели, основанной на многомерных 
массивах, и встроенными типами объектной модели устанавливается 
взаимно однозначное соответствие;
\item совокупность атрибутов, соответствующих измерениям, объявляется 
уникальной (при помощи механизма ключей, утверждений или 
инвариантов);
\item атрибуты, соответствующие измерениям и не-{\sf NULL} атрибутам ячейки 
массива, объявляются определенными (при помощи утверждений или 
инвариантов);
\item для нецелочисленных измерений в типе экземпляров дополнительно 
определяются методы, преобразующие нецелочисленные значения в 
целочисленные;
\item ограничения, связанные с нижними и верхними границами 
целочисленных измерений, отображаются при помощи инвариантов или 
встроенных утверждений о кардинальности соответствующих атрибутов. 
В~случае использования инвариантов при отображении границы измерений 
представляются также метаданными атрибута.
\end{itemize}

\subsection{Отображение языка манипулирования данными}

        При интеграции баз данных для установления семантических 
соотношений между схемами ресурсов и федеративной схемой необходимо 
отображение ЯОД исходной модели в каноническую. Язык манипулирования данными канонической 
модели, напротив, необходимо отображать в ЯМД исходной модели. Это 
связано с тем, что запросы к посреднику в канонической модели необходимо 
отображать в запросы к ресурсам.
        
        Отметим отличие виртуальной и материализованной интеграции. При 
виртуальной интеграции отображение ЯМД обеспечивает возможность 
трансляции программ на языке посредника в запросы на языке ресурсов. 
        
        В случае материализованной интеграции данные извлекаются из ресурса 
и представляются в хранилище в канонической модели. При этом программы 
на языке канонической модели исполняются непосредственно на данных. 
Отоб\-ра\-же\-ние\linebreak ЯМД нужно лишь для того, чтобы убедиться, что отображение 
моделей сохраняет информацию и семантику операций. Семантически 
правильное\linebreak отоб\-ра\-же\-ние служит базой для процесса 
        <<из\-вле\-че\-ния--пре\-образо\-ва\-ния--за\-груз\-ки>> (ETL), 
формирующего из данных ресурса данные хранилища:\linebreak ETL-про\-цесс может 
быть выражен только в терминах канонической модели.
        
        \smallskip
        
        Язык запросов (программ) модели СИНТЕЗ представляет собой 
        Datalog-по\-доб\-ный язык в объектной среде. Программа представляет 
собой набор конъюнктивных запросов (правил) вида 

\noindent
\begin{multline*}
        q(x/T): - C_1(x_1/T_1),\ldots , C_n(x_n/T_n), (X_1,Y_1), 
\ldots \\
\ldots F_m(X_m,Y_m), B\,.
        \end{multline*}
        Тело запроса представляет собой конъюнкцию 
        пре\-ди\-ка\-тов-кол\-лек\-ций, функциональных предикатов и 
ограничения. Здесь $C_i$~--- имена коллекций (классов), $F_i$~--- имена 
функций, $x_i$~--- имена переменных, значения которых пробегают по 
классам, $T_i$~--- типы переменных, $X_j$ и $Y_j$~--- входные и выходные 
параметры функций, $B$~--- ограничение, налагаемое на $x_i$, $X_j$, $Y_j$. 
Предикаты, находящиеся в голове правил, могут быть использованы в телах 
других правил в качестве пре\-ди\-ка\-тов-кол\-лек\-ций. 
        
        В дальнейшем будет часто использоваться запись 
        пре\-ди\-ка\-та-кол\-лек\-ции вида {\sf source([ra, de])}. Неформально это 
означает, что представляют интерес не объекты класса {\sf source} целиком, а 
лишь их атрибуты {\sf ra} и {\sf de}. Формально запись означает сокращение от 
{\sf source(\_/source.inst[ra, de])}. Здесь знак <<{\sf \_}>> обозначает анонимную 
переменную, {\sf source.inst}~--- анонимный тип экземпляров (instance) класса 
{\sf source}, а {\sf ra} и {\sf de}~--- необходимые атрибуты типа экземпляров класса.
        
        Будет также использоваться запись {\sf source([i, j, val1/val])}, означающая 
переименование атрибута {\sf val} в {\sf val1}.
        
        \medskip
        
        При отображении ЯМД будут сначала рассмотрены основные 
конструкции языка программ СИНТЕЗ, соответствующие конструкциям языка 
AQL. Затем будут рассмотрены конструкции \mbox{СИНТЕЗ}, соответствующие 
конструкциям языка AFL.
        
        Начнем рассмотрение с конструкций языка СИНТЕЗ, соответствующих 
конструкциям языка AQL, связанных с {извлечением} данных.
        
%        \smallskip
        
\paragraph*{Предикаты-классы, условия, подзапросы.} Рас\-смот\-рим 
программу, извлекающую координаты ({\sf ra}, {\sf de}) и апертурную звездную 
величину ({\sf apMag}) астрономических источников из класса  {\sf source} с 
условием на фильтр ({\sf filterId}) и апертурную звездную величину, причем 
запрос~{\sf q} использует результаты запроса~{\sf r}:
        \begin{verbatim}
q([ra,de,apMag]) :- r([ra,de,apMag]),
   filterId= #filterId.
r([ra,de,apMag]) :- source([ra,de,apMag]),
   apMag >= #apMag.
\end{verbatim}
Здесь {\sf \#filterId} и {\sf \#apMag}~--- некоторые константы 
соответствующих типов.
        
        Такая программа представляется в AQL сле\-ду\-ющим запросом:
        \begin{verbatim}
SELECT apMag FROM 
  ( SELECT apMag FROM source
    WHERE apMag >= #apMag )
WHERE filterId = #filterId;
\end{verbatim}
        
        Простые условия отображаются в AQL без изменений, рекурсивные 
запросы представляются вложенными запросами. Заметим, что координаты 
{\sf ra} и {\sf de} не указываются в секции {\sf SELECT}~--- они являются измерениями и 
извлекаются по умолчанию.
        
\paragraph*{Соединение классов.} Соединение по определенным атрибутам 
(например, {\sf objectId})
        \begin{verbatim}
q2([ra, de, filterId, uMag]) :- 
    source([ra, de, objectId, fliterId]), 
    objectSummary([objectId, uMag]).
\end{verbatim}
представляется в AQL конструкцией {\sf JOIN-ON}:
\begin{verbatim}
SELECT filterId, uMag INTO q2
FROM source
JOIN objectSummary 
ON source.objectId = objectSummary.objectId;
\end{verbatim}
где массив {\sf objectSummary} имеет вид: 
\begin{verbatim}
CREATE ARRAY objectSummary
<uMag: float NULL,  gMag: float NULL>
[ objectId=0:* ];
\end{verbatim}
        
\paragraph*{Агрегация.} Рассмотрим запрос, возвращающий объекты с 
минимальной звездной величиной {\sf uMag}:
        \begin{verbatim}
q([objectId, uMag]) :-  
  objectSummary(obj/[objectId, uMag]), 
    uMag = min(obj.uMag).
\end{verbatim}

        Запрос представляется в AQL с использованием агрегирующей функции 
того же рода:
        \begin{verbatim}
SELECT uMag
FROM source, 
 (SELECT min(uMag) AS min FROM Source)
WHERE uMag = min;
\end{verbatim}
        
\paragraph*{Группирование.} Рассмотрим запрос, возвра\-ща\-ющий среднее 
значение звездной величины {\sf uMag}, вычисленное на группе по 
идентификатору объекта {\sf filterId}:
        \begin{verbatim}
q([objectId, avgMag]) :- 
    group_by({objectId}, 
       q2(obj/[ra,de,filterId, uMag])),
    avgMag = average(obj.uMag).
\end{verbatim}

        Здесь коллекция {\sf q2}, на которой производится группирование по 
атрибуту {\sf objectId}~--- результат соединения классов {\sf source} и 
{\sf objectSummary}, рассмотренных выше.
        
        Очевидно, в AQL запрос представляется при помощи конструкции 
GROUP BY:
        \begin{verbatim}
SELECT avg(uMag) AS avgMag
FROM q2 GROUP BY objectId;
\end{verbatim}
        
        Рассмотрим конструкции языка СИНТЕЗ, соответствующие 
конструкциям языка AQL и связанные с {изменением} данных.

        
\paragraph*{Обновление.} Рассмотрим запрос, изменяющий значения в 
квадратной матрице (см.\ предыдущий пример) на значения с обратным знаком 
в том случае, если модуль значения больше~5:
        \begin{verbatim}
source(x/[i, j, val]) :- 
    source(x/[i, j, val1/val]), 
       abs(val) > 5, val = -val1.
\end{verbatim}
        
        В AQL данный запрос представляется сле\-ду\-ющим образом:
        \begin{verbatim}
UPDATE source
SET val =  -val WHERE abs(val) > 5;
\end{verbatim}


        
\paragraph*{Удаление.} Рассмотрим программу, удаляющую из базы данных 
класс и все его содержимое:
        \begin{verbatim}
-source(x) :- source(x).; 
delete_frame(source).
\end{verbatim}

        В правилах со знаком минус в голове осуществляется удаление объектов 
из коллекции. В~данном случае из коллекции удаляются все объекты. Функция 
{\sf delete\_frame} удаляет коллекцию как объект из базы данных. Операция <<{\sf ;}>> 
обозначает последовательную композицию программ. В~AQL данный запрос 
представляется при помощи операции {\sf DROP}:
\begin{verbatim}
DROP ARRAY source;
\end{verbatim}

        Рассмотрим принципы отображения конструкций языка СИНТЕЗ, 
соответствующих конструкциям AFL, на примере {расширения элементов 
мас\-си\-ва в подмассивы}. Каждый элемент массива расширя\-ется в подмассив 
определенного размера. Значения всех ячеек подмассива дублируют значение 
оригинальной ячейки. Пример программы, расширяющей каждую ячейку 
матрицы $3\times3$ в подматрицу $2\times2$:
        \begin{verbatim}
q([i,j,val]) :- {x/[i,j,val] | exists y (
  source(y/[i1/i, j1/j, val]) & 
  ( i = i1*2 & j = j1*2 | i = i1*2 +1 & 
  j = j1*2 | i= i1*2 & 
  j= j1*2 + 1 | i= i1*2 +1 & j= j1*2 +1))}.
\end{verbatim}
    Здесь выражение $\{x/T \vert F(x)\}$, где $F$~--- формула со свободной 
переменной~$x$, обозначает конструкцию выделения множества; {\sf exists} 
обозначает квантор существования. 

\columnbreak
        
        В ADM запрос представляется с использованием операции {\sf xgrid}:
        \begin{verbatim}
SELECT * FROM xgrid(source, 2, 2);
\end{verbatim}
        
        Можно заметить, что операция AFL {\sf xgrid} имеет достаточно сложно 
устроенный прообраз в канонической модели (это справедливо и для многих 
других операций). Между тем эти операции являются естественными для 
массивов. Поэтому при интеграции ресурсов, основанных на многомерных 
массивах, в канонической модели возможно использование специального 
класса {\sf array}, инкапсулирующего специфические операции, характерные для 
многомерных массивов:
        \begin{verbatim}
{ array; in: class;
  instance_type: {
  xgrid: { in: function; 
    params: {
     +dimensions/{sequence; 
      type_of_element: string;},
     +scales/{sequence; 
      type_of_element: integer;}};
  };  };
}
\end{verbatim}
        В приведенном примере рассмотрена сигнатура единственной операции 
{\sf xgrid}, параметрами которой являются последовательность имен измерений\linebreak 
{\sf dimensions} и последовательность масштабов увеличения по каждому из 
измерений {\sf scales}. Па\-ра\-мет\-ром операции по умолчанию также считается 
класс\linebreak {\sf array} как коллекция объектов. При отображении ЯОД каждый класс~--- 
образ массива (например, класс {\sf source} из подразд.~2.1) становится подклассом 
класса {\sf array}:
        \begin{verbatim}
{ source; in: class; superclass: array;
  instance_type: { ... };
}
\end{verbatim}

        Аналогично {\sf xgrid}, операциями класса {\sf array} могут быть 
представлены такие операции AFL, как {\sf substitute}, {\sf sort}, 
{\sf multiply} и~т.\,д. 
        
        Заметим, что решение о представлении операций, характерных для 
многомерных массивов, в рамках специального класса канонической модели 
допускает обобщение на объектные канонические модели, отличные от языка 
СИНТЕЗ, и модели, основанные на многомерных массивах, отличные от ADM.
        
        \smallskip
        
        Разработанные отображения ЯОД и ЯМД были частично реализованы на 
языке ATL (ATLAS\linebreak Transformation Language)~\cite{18-stu}. ATL-программы 
пред\-став\-ля\-ют собой де\-кла\-ра\-тив\-но-им\-пе\-ра\-тив\-ные трансформации, 
реализующие отображения произвольных исходных моделей уровня М1 
(согласно классификации MOF~\cite{19-stu}), конформных исходной 
метамодели уровня М2, в целевые модели уровня М1, конформные целевой 
метамодели уровня М2. Модели уровня М1 являются схемами, 
представленными в канонической модели данных или модели ADM; модели 
уровня М2 есть описание абстрактного синтаксиса канонической модели или 
модели ADM. В~качестве метамодели уровня М3, которой конформны 
метамодели уровня M2, рассматривается модель Ecore~\cite{20-stu}. Cинтаксис 
ЯОД и ЯМД ядра канонической информационной модели (языка СИНТЕЗ) и 
модели ADM был представлен в метамодели Ecore. 
        
        Было осуществлено построение ATL-транс\-фор\-ма\-ций, реализующих 
отображения подмножества ЯОД модели ADM в ЯОД канонической модели и 
подмножества ЯМД канонической модели в ЯМД модели ADM. Подмножества 
ЯМД определялись конструкциями ЯОД и ЯМД канонической модели, 
поддерживаемыми в настоящее время в исполнительной среде предметных 
посредников. Поддерживаемый язык запросов канонической модели включает 
правила, в голове которых могут быть пре\-ди\-ка\-ты-кол\-лек\-ции, а в теле~--- 
конъюнкция пре\-ди\-ка\-тов-кол\-лек\-ций, условий соединения коллекций и 
других условий на значения атрибутов типов экземпляров коллекций. 
Условием соединения может быть только равенство атрибутов. 
Поддерживаются основные арифметические предикаты и функции, а также 
термы~--- вызовы функций. 

\section{Сохранение информации и~семантики операций языка манипулирования данными 
при~отображении}
        
        Метод доказательства сохранения информации и семантики операций 
при отображении моделей данных~\cite{21-stu} основывается на представлении 
семантики моделей в формальном языке спецификаций AMN~\cite{16-stu}. 
        
        Язык AMN представляет собой тео\-ре\-ти\-ко-мо\-дель\-ную нотацию, 
основанную на теории множеств и типизированном языке первого порядка. 
Спецификации AMN называются абстрактными машинами. Язык AMN позволяет 
интегрированно рас\-смат\-ри\-вать спецификацию пространства состояний и 
поведения машины (определенного операциями на состояниях). В~AMN 
формализуется специальное отношение между спецификациями~--- 
{уточнение}. Неформально спецификация~$B$ уточняет 
спецификацию~$A$, если пользователь может использовать $B$ вместо~$A$, 
не замечая факта замены~$A$ на~$B$. 
{\looseness=1

}
        
        Идея метода заключается в следующем. Рассмотрим исходную 
модель~$S$ и целевую модель~$T$. Построим отображение~$\theta$ 
модели~$S$ в модель~$T$ (подобно изложенному в предыдущем разделе). 
Выразим семантику моделей в виде абстрактных машин AMN, построив при 
этом машины $M_S$ и $M_T$ соответственно. При этом структуры данных 
моделей~--- классы, массивы~--- представляются переменными машин, 
различные свойства структур данных представляются инвариантами машин, 
характерные операции моделей данных представляются операциями машин. 
Рассматриваемые операции исходной и целевой модели должны быть связаны 
отображением ЯМД. Отображение ЯОД представляется в виде специального 
\textit{склеивающего инварианта}~--- замкнутой формулы, связывающей 
состояния машин~$M_S$ и~$M_T$.
        
        Будем считать отображение~$\theta$ сохраняющим инфор\-ма\-цию и 
семантику операций, если машина~$M_S$, соответствующая исходной модели, 
уточняет машину~$M_T$, соответствующую целевой модели. Уточнение 
доказывается интерактивно при помощи специальных программных 
средств~\cite{22-stu}.
        
        \smallskip
        
        В качестве иллюстрации основных принципов выражения семантики 
моделей ADM и СИНТЕЗ в AMN рассмотрим частичные 
        AMN-спе\-ци\-фи\-ка\-ции, соответствующие данным моделям.
        
        Cпецификация, выражающая семантику объектной модели языка 
СИНТЕЗ, представляется в языке AMN конструкцией {\sf REFINEMENT}:
\begin{verbatim}
REFINEMENT ObjectDM
\end{verbatim}

        Переменные, составляющие пространство состояний объектной модели, 
объявлены в разделе {\sf ABSTRACT\_VARIABLES} машины {\sf ObjectDM} и 
типизируются в разделе {\sf INVARIANT}:
\begin{verbatim}
ABSTRACT_VARIABLES
    typeNames, classNames, attributeNames,
    instanceType, typeAttributes, 
      attributeType,
    unique, obligatory,
    intAttributeLowerBound, 
      intAttributeHigherBound,
    objectIDs, objectType, objectsOfClass,
    integerAttributeValue,
    adtAttributeValue
INVARIANT ...
\end{verbatim}

        Раздел {\sf INVARIANT} содержит формулу, которая состоит из предикатов, 
типизирующих переменные состояния и налагающих различные совместные 
ограничения на переменные. Предикаты соединяются операцией конъюнкции.
        
        Так, имена типов и классов представлены переменными {\sf typeNames} и 
{\sf classNames}, тип которых~--- подмножество множества строк 
({\sf STRING\_Type}):
        \begin{verbatim}
typeNames: POW(STRING_Type) &
classNames: POW(STRING_Type)
\end{verbatim}
        
        \noindent
        Здесь {\sf POW}~--- конструктор множества подмножеств.
        
        Атрибуты (переменная {\sf attributeNames}) пред\-став\-ле\-ны частичной 
функцией (знак <<\verb +-> >>), ставящей в соответствие уникальному идентификатору 
атрибута (натуральному числу из множества {\sf NAT}) имя атрибута (строку):
        \begin{verbatim}
attributeNames: NAT +-> STRING_Type
\end{verbatim}

        Типы экземпляров классов (переменная\linebreak {\sf instanceType}) представлены 
тотальной функцией (знак \verb -> ) из множества имен классов в 
множество имен типов:
        \begin{verbatim}
instanceType: classNames --> typeNames
\end{verbatim}

        Принадлежность атрибутов типам (переменная {\sf typeAttributes}) 
выражена тотальной функцией из множества имен типов в множество 
подмножеств атрибутов:
        \begin{verbatim}
typeAttributes: 
  typeNames --> POW(dom(attributeNames))
\end{verbatim}
        Здесь {\sf dom}~--- операция, возвращающая область определения 
функции, а {\sf dom(attributeNames)}~--- множество имен атрибутов.
        
        Типы значений атрибутов (переменная\linebreak {\sf attributeType}) представлены 
функцией из множества атрибутов в множество идентификаторов встроенных 
типов данных {\sf BuiltInTypes}:
        \begin{verbatim}
attributeType: 
  dom(attributeNames) +-> BuiltInTypes
\end{verbatim}

        Множества уникальных атрибутов типов {\sf unique}\linebreak и множества 
определенных атрибутов типов\linebreak {\sf obligatory} представлены тотальными 
функциями из множества имен типов в множество подмножеств атрибутов:
\begin{verbatim}
unique: 
  typeNames --> POW(dom(attributeNames))&
obligatory: 
  typeNames --> POW(dom(attributeNames))
\end{verbatim}

        Нижние границы целочисленных атрибутов (переменная 
{\sf intAttributeLowerBound}) представлены час\-тич\-ной функцией из множества 
атрибутов в множество целых чисел:
\begin{verbatim}
intAttributeLowerBound: 
  dom(attributeNames) +-> INT
\end{verbatim}

        Аналогично представляются верхние границы.
        
        Идентификаторы объектов (переменная\linebreak {\sf objectIDs}) представлены 
подмножеством натуральных чисел:
        \begin{verbatim}
objectIDs: POW(NAT)
\end{verbatim}

        Типы объектов (переменная {\sf objectType}) представлены тотальной 
функцией из множества объектных идентификаторов в множество имен типов:
\begin{verbatim}
objectType: objectIDs --> typeNames
\end{verbatim}

        Состав классов (переменная {\sf objectsOfClass}) представлен тотальной 
функцией из множества имен классов в множество подмножеств 
идентификаторов объектов:
        \begin{verbatim}
objectsOfClass: 
  classNames --> POW(objectIDs)
\end{verbatim}
        
        Значения атрибутов объектов (переменные\linebreak {\sf integerAttributeValue}, 
{\sf adtAttributeValue} и~др.)\ пред\-став\-ле\-ны функциями из множества атрибутов\linebreak 
в функции из множества идентификаторов объектов в множества значений 
атрибутов. Для простоты рассмотрены лишь функции для целочисленных 
атрибутов и атрибутов со значениями АТД\linebreak (абстрактного типа данных) (объектами):
        \begin{verbatim}
integerAttributeValue: 
 dom(attributeNames) +-> (objectIDs+->INT)& 
adtAttributeValue: 
 dom(attributeNames) +-> (objectIDs+->NAT)
\end{verbatim}
        
        Дополнительные необходимые свойства переменных состояния 
представлены конъюнктивными компонентами инварианта. Так, каждый 
атрибут является атрибутом некоторого типа:
        \begin{verbatim}
        
UNION(tt).(tt:typeNames|typeAttributes(tt))=
    dom(attributeNames)
\end{verbatim}
        Здесь {\sf UNION}~--- родовая операция объединения, в данном случае 
объединяются множества атрибутов {\sf typeAttributes(tt)} по всем именам 
типов~{\sf tt} из множества {\sf typeNames}. 
        
        Никакой атрибут не принадлежит двум типам одновременно:
        \begin{verbatim}
!(t1, t2).(t1: typeNames & t2: typeNames =>
  (typeAttributes(t1) /\ typeAttributes(t2) 
    = {}))
\end{verbatim}
   Здесь <<\verb ! >>~--- знак квантора всеобщности, <<\verb => >>~--- логическая 
импликация, <<\verb /\ >>~--- символ пересечения множеств, <<\verb {} >>~--- пустое 
множество.
        
        Уникальные и определенные атрибуты типа выбираются из множества 
атрибутов типа:
        \begin{verbatim}
!(tt).(tt: dom(unique) => unique(tt) <: 
typeAttributes(tt)) &
!(tt).(tt: dom(obligatory) => 
    obligatory(tt) <: typeAttributes(tt))
\end{verbatim}
        Здесь <<\verb <: >>~--- символ отношения мно\-жес\-во--под\-мно\-жество.
        
        Нижние и верхние границы могут быть определены только для 
целочисленных атрибутов:
        \begin{verbatim}
!(attr).(attr: dom(intAttributeLowerBound)=> 
    attributeType(attr) = Integer) 
\end{verbatim}

        Тип объекта~--- экземпляра класса есть тип экземпляров этого класса:
        \begin{verbatim}
!(cc).(cc: classNames => 
    !(oo).(oo: objectsOfClass(cc) => 
       objectType(oo) = instanceType(cc))) 
\end{verbatim}

        Для каждого атрибута определена ровно одна функция значений:
        \begin{verbatim}
dom(adtAttributeValue) /\ 
  dom(integerAttributeValue) = {} &
dom(adtAttributeValue) \/ 
  dom(integerAttributeValue) = 
    dom(attributeNames)
\end{verbatim}
   Здесь <<\verb \/ >>~--- символ объединения множеств.
        
        Для любого объекта и любого определенного атрибута типа этого 
объекта функция значений атрибута определена на объекте:
        \begin{verbatim}
!(oo, aa).(oo: dom(objectType) & 
  aa: typeAttributes(objectType(oo)) & 
  aa: obligatory(objectType(oo)) =>
      (attributeType(aa) = Integer => 
       oo: dom(integerAttributeValue(aa))) &
      (attributeType(aa) = ADT =>
       oo: dom(adtAttributeValue(aa)))) 
\end{verbatim}

        Для любого объекта и любого целочисленного атрибута типа объекта, 
определенного на объекте и для которого определена нижняя (верхняя) 
граница, значение атрибута не меньше (не больше) нижней (верхней) границы:
        \begin{verbatim}
!(oo, aa).(oo: objectIDs & 
    aa: typeAttributes(objectType(oo)) &
    oo: dom(integerAttributeValue(aa) => 
    (aa: dom(intAttributeLowerBound) =>
        (integerAttributeValue(aa)(oo) >= 
         intAttributeLowerBound(aa))) ) 
\end{verbatim}

        Объект однозначно идентифицируется набором своих уникальных 
атрибутов:
        \begin{verbatim}
!(oo1, oo2).(oo1: objectIDs & 
  oo2: objectIDs &
    objectType(oo1) = objectType(oo2) & 
    unique(objectType(oo1)) /= {} &
    !(aa).(aa: unique(objectType(oo1)) => 
      (attributeType(aa) = Integer =>
        integerAttributeValue(aa)(oo1) =
         integerAttributeValue(aa)(oo2)) &
      (attributeType(aa) = ADT =>
         adtAttributeValue(aa)(oo1) =
          adtAttributeValue(aa)(oo2)) ) => 
    oo1 = oo2 )
\end{verbatim}

        Из всего ЯМД в спецификации рассмотрена единственная операция 
{\sf update} обновления значений атрибута в объектах класса:
        \begin{verbatim}
OPERATIONS
update(cls, attr, exp, cond) =
PRE cls: classNames & 
  attr: typeAttributes(instanceType(cls)) &
  attributeType(attr) = Integer &
  exp: INT --> INT & cond: NAT --> BOOL
THEN
 integerAttributeValue := 
 integerAttributeValue <+ 
 { xx | xx: (NAT*(NAT<->INT)) &
  #(oo, val).( oo: objectsOfClass(cls) & 
  val: INT &
    xx = attr |-> ({oo |-> val}) & 
  (cond(integerAttributeValue(attr)(oo)) 
  = TRUE =>
  val=exp(integerAttributeValue(attr)(oo)))&
  (cond(integerAttributeValue(attr)(oo)) 
  = FALSE => 
  val=integerAttributeValue(attr)(oo)))}
END
\end{verbatim}

        Параметрами операции являются имя класса {\sf cls}, имя целочисленного 
атрибута {\sf attr} типа экземпляров класса, функция {\sf exp}, отвечающая за 
преобразование атрибута, и функция {\sf cond}, отвечающая условию на значение 
атрибута. Пусть {\sf o}~--- некоторый объект класса {\sf cls}, для которого определено 
значение атрибута {\sf attr}, и это значение есть~{\sf v}. Тогда операция {\sf update} 
изменяет значение атрибута на {\sf exp(v)} в случае, если выражение {\sf cond(v)} 
принимает значение <<истина>>, и оставляет значение атрибута без изменений в 
противном случае. Очевидно, такая операция {\sf update} есть обобщение примера 
обновления, рассмотренного в подразд.~2.2. Действительно, для рассмотренного 
примера {\sf cls} есть {\sf source}, {\sf attr} есть {\sf val}, 
{\sf exp(v)}\;=\;-\,{\sf v}, {\sf cond(v)}\;=\;{\sf abs(v)}.
        
        Заметим, что в рассмотренной спецификации для простоты не 
рассмотрены некоторые черты объектной модели, например отношения 
        тип--под\-тип и класс--под\-класс.
        
        \smallskip
        
        Спецификация, выражающая семантику модели ADM, представляется в 
языке AMN конструкцией
        \begin{verbatim}
REFINEMENT ArrayDM
\end{verbatim}

        Переменные, составляющие пространство состояний объектной модели, 
объявлены в разделе {\sf ABSTRACT\_VARIABLES} машины {\sf ArrayDM}:
        \begin{verbatim}
ABSTRACT_VARIABLES
    arrayNames, dimensionNames, 
    cellAttributeNames,
    arrayDimensions, arrayCellAttributes,    
    cellAtrributeType, nullable, 
    dimLowerBound, dimHigherBound,
    cells, dimensionValue, 
    integerCellAttributeValue
\end{verbatim}

        Имена массивов представлены переменной\linebreak 
{\sf arrayNames}; имена измерений~--- переменной\linebreak 
{\sf  dimensionNames}; имена атрибутов ячеек массива~--- переменной 
\mbox{{\sf cellAttributeNames}}; принадлежность измерений массивам~--- переменной 
\mbox{{\sf arrayDimensions}}; принадлежность атрибутов ячеек~--- переменной 
\mbox{{\sf arrayCellAttributes}}; 
тип атрибута ячейки~--- переменной \mbox{{\sf cellAtrributeType}}; 
атрибуты ячеек массивов, которые могут принимать неопределенные 
значения,~--- переменной \mbox{{\sf nullable}}; верхние (нижние) границы измерений~--- 
переменной \mbox{{\sf dimLowerBound}} (\mbox{{\sf dimHigherBound}}); множества 
идентификаторов ячеек массивов~--- переменной 
\mbox{{\sf cells}}, значения измерений в 
ячейках~--- переменной \mbox{{\sf dimensionValue}}; значения атрибутов ячеек~--- 
переменной \mbox{{\sf integerCellAttributeValue}}. Переменные типизируются в разделе 
\mbox{{\sf INVARIANT}} при помощи частичных и тотальных функций аналогично 
переменным, использующимся для придания семантики объектной модели:
        \begin{verbatim}
INVARIANT
   arrayNames: POW(STRING_Type) &
   dimensionNames: NAT +-> STRING_Type &
   cellAttributeNames: NAT +-> STRING_Type &
   arrayDimensions: arrayNames --> 
   POW(dom(dimensionNames)) &
   arrayCellAttributes: arrayNames --> 
     POW(dom(cellAttributeNames)) &
   cellAtrributeType: 
     dom(cellAttributeNames) --> 
       BuiltInTypes &
   nullable: 
     dom(cellAttributeNames) --> BOOL &
   dimLowerBound: 
     dom(dimensionNames) --> INT &
   dimHigherBound: 
     dom(dimensionNames) +-> INT &
   cells: arrayNames --> POW(NAT) & 
   dimensionValue: 
     NAT*dom(dimensionNames) +-> INT  &
   integerCellAttributeValue: 
     NAT*dom(cellAttributeNames) +-> INT &
\end{verbatim}
        Здесь <<\verb * >>~--- знак декартова произведения множеств.
        
        Дополнительные необходимые свойства переменных состояния 
представлены конъюнктивными компонентами инварианта. Так, любая ячейка 
любого массива однозначно идентифицируется набором значений измерений:
        \begin{verbatim}
!(arr, cell1, cell2).(arr: arrayNames & 
  cell1: cells(arr) &  cell2: cells(arr) &
  !(dim).(dim: arrayDimensions(arr) =>
    dimensionValue(cell1, dim) = 
    dimensionValue(cell2, dim)) =>
    cell1 = cell2)
        \end{verbatim}
        
                \vspace*{-6pt}
        
        Для любой ячейки любого массива определены значения всех измерений 
и значение по крайней мере одного атрибута:
        \begin{verbatim}
!(arr, cell).(arr: arrayNames & 
 cell: cells(arr) =>
  !(dim).(dim: arrayDimensions(arr) => 
   (cell |-> dim): dom(dimensionValue)) &
   #(attr).(attr: arrayCellAttributes(arr) & 
    cellAtrributeType(attr) = Integer & 
    (cell, attr): 
      dom(integerCellAttributeValue)) )
        \end{verbatim}
        
        \vspace*{-6pt}
        
        Аналогично объектной модели рассмотрена единственная операция 
ЯМД~--- операция об\-нов\-ле\-ния {\sf update}:
        \begin{verbatim}
OPERATIONS
update(cls, attr, exp, cond) =
PRE cls: arrayNames & 
 attr: arrayCellAttributes(cls) &
  cellAtrributeType(attr) = Integer &
  exp: INT --> INT & cond: NAT --> BOOL
THEN
  integerCellAttributeValue := 
  integerCellAttributeValue <+
  { yy | yy: (NAT*NAT)*INT &
    #(cell, val).(cell: cells(cls) & 
     val: INT & 
    yy = ((cell |-> attr)|-> val) &
    (cond(integerCellAttributeValue(cell, 
     attr)) = TRUE =>
      val = 
       exp(integerCellAttributeValue(cell,
        attr))) &
      (cond(integerCellAttributeValue(cell, 
       attr))= FALSE  =>
    val = 
     integerCellAttributeValue(cell,attr)))}
END   
END
        \end{verbatim}
        
                \vspace*{-6pt}
        
        Сигнатура операции совпадает с сигнатурой операции объектной 
модели. Семантика операции также аналогична: значение~{\sf v} атрибута {\sf attr} 
массива {\sf cls} заменяется на {\sf exp(v)}, если значение {\sf cond(v)} есть 
<<истина>>, и не изменяется в противном случае. 
        
        Заметим, что в данной спецификации для прос\-то\-ты не рассмотрены 
некоторые черты ADM, например нецелочисленные измерения.
        
        \smallskip
        
        Для формального доказательства того, что машина {\sf ArrayDM} уточняет 
машину {\sf ObjectDM}, необходимо построить {инвариант уточнения}, 
связы\-вающий переменные машин, и добавить его к\linebreak инварианту уточняющей 
машины. 
        
        Инвариант формализует принципы отображения ЯОД, изложенные в 
подразд.~2.1, и объединяет их в одну конъюнкцию.
        
        Так, множество имен массивов совпадает с множеством имен классов:
        \begin{verbatim}
classNames = arrayNames
\end{verbatim}

%                \vspace*{-6pt}
        
        Множество идентификаторов и имен измерений и атрибутов ячеек 
совпадает с множеством идентификаторов и имен атрибутов типов экземпляров 
классов:
        \begin{verbatim}
attributeNames = 
  dimensionNames \/ cellAttributeNames
\end{verbatim}

%                \vspace*{-6pt}

        Любому измерению любого массива соответствует атрибут типа 
экземпляра класса, соответствующего этому массиву:
        \begin{verbatim}
!(arr, dim).(arr: arrayNames & 
  dim: arrayDimensions(arr) =>
    #(attr).(attr: 
     typeAttributes(instanceType(arr)) &
          attr = dim & 
          attributeType(attr) = Integer) )s
        \end{verbatim}
        
                        \vspace*{-6pt}
        
        Любому атрибуту ячейки любого массива соответствует атрибут типа 
экземпляра класса, соответствующего этому массиву, и типы атрибутов 
совпадают:
        \begin{verbatim}
!(arr, cattr).(arr: arrayNames & 
   cattr: arrayCellAttributes(arr) =>
    #(attr).(attr: 
      typeAttributes(instanceType(arr)) & 
         attr = cattr & 
         attributeType(attr) = 
           attributeType(cattr)))
        \end{verbatim}
        
                        \vspace*{-9pt}
        
        Атрибут ячейки массива, который может принимать неопределенные 
значения, соответствует определенному ({\sf obligatory}) атрибуту типа:
        \begin{verbatim}
!(arr, cattr).(arr: arrayNames & 
  cattr /: dom(nullable) &
    cattr: arrayCellAttributes(arr) => 
    cattr: obligatory(instanceType(arr)) )
        \end{verbatim}
        
\vspace*{-9pt}

           Здесь знак <<\verb /: >> обозначает отношение непринадлежности элемента 
множеству.
        
        Измерения соответствуют уникальным атрибутам типов:
        \begin{verbatim}
!(arr, dim).(arr: arrayNames & 
    dim: arrayDimensions(arr) => 
      dim: unique(instanceType(arr)) )
        \end{verbatim}
        
                        \vspace*{-6pt}
        
        Верхние (нижние) границы измерений равны верхним (нижним) 
границам соответствующих атрибутов типов:
        \begin{verbatim}
!(dim).(dim: dom(dimLowerBound) =>
    dim: dom(intAttributeLowerBound) & 
    dimLowerBound(dim) = 
      intAttributeLowerBound(dim))
        \end{verbatim}
        
                        \vspace*{-6pt}
        
        Непустые ячейки массивов соответствуют объектам классов:
        \begin{verbatim}
cells = objectsOfClass
\end{verbatim}

%                \vspace*{-6pt}

        Для любой ячейки значения ее измерений и определенных атрибутов 
совпадают со значениями соответствующих атрибутов объекта, 
соответствующего ячейке:
        \begin{verbatim}
!(cell, dim).(cell: NAT & dim: NAT & 
  (cell |-> dim): dom(dimensionValue) =>
  cell: dom(integerAttributeValue(dim)) &
  dimensionValue(cell, dim) = 
    integerAttributeValue(dim)(cell)) &
!(cell, cattr).(cell: NAT & cattr: NAT & 
   (cell |-> cattr): 
   dom(integerCellAttributeValue) =>
   cell: dom(integerAttributeValue(cattr)) &
   integerCellAttributeValue(cell, cattr) =
     integerAttributeValue(cattr)(cell) )
        \end{verbatim}
        
                        \vspace*{-6pt}
        
        Для указания того, что машина {\sf ArrayDM} уточняет машину 
{\sf ObjectDM}, в машину {\sf ArrayDM} была добавлена директива
        \begin{verbatim}
REFINES ObjectDM
\end{verbatim}

%                \vspace*{-6pt}

        Спецификации {\sf ObjectDM} и {\sf ArrayDM} вместе с инвариантом 
уточнения были загружены в инструментальное средство 
        Atelier~B~\cite{22-stu}. Автоматически были сгенерированы теоремы, 
выражающие уточнение спецификаций. В~частности, для операции {\sf update} 
были сгенерированы 10~тео\-рем. Три из них были доказаны автоматически, 
для доказательства остальных необходимо применять интерактивные средства 
доказательства.

\vspace*{-9pt}
  
\section{Родственные исследования и~направления дальнейшей 
работы}

\vspace*{-2pt}

        Родственными данной работе следует считать исследования, связанные с 
отображением моделей, основанных на многомерных массивах, в реляционную 
модель данных. Обычно они нацелены на реализацию многомерных массивов 
при помощи реляционных СУБД. Такие работы начались одновременно с 
исследованиями моделей, основанных на многомерных массивах~\cite{5-stu}, и 
продолжаются в настоящее время~\cite{23-stu}.
        
        Основные особенности данной работы состоят в следующем. 
В~качестве исходной модели при отображении используется специфическая 
модель, основанная на многомерных массивах СУБД SciDB, язык которой 
представляет собой комбинацию декларативного SQL-по\-доб\-но\-го языка и 
функционального языка, включающего специфические\linebreak операции над 
многомерными массивами. В~качестве целевой модели используется объектная 
модель с Datalog-по\-доб\-ным языком запросов (программ)~--- язык СИНТЕЗ. 
Для отображения\linebreak обеспечивается формальное доказательство сохранения 
информации и семантики операций ЯМД.
        
        Отметим, что результаты работы могут быть с легкостью обобщены и 
использованы при интеграции в системах, использующих каноническую 
модель, отличную от языка СИНТЕЗ, например другую объектную (ODMG) 
или объект\-но-ре\-ля\-ци\-он\-ную модель (SQL:2003). Результаты также могут 
быть использованы для интеграции ресурсов, представленных в модели, 
основанной на многомерных массивах, но отличной от ADM.
        
        Некоторые вопросы отображения требуют дальнейших исследований. 
Например, следует ли иметь в канонической модели при интеграции 
        масс\-сив-ори\-ен\-ти\-ро\-ван\-ных моделей данных операции, 
связанные с размером порции (chunk size) данных в БД~\cite{9-stu}?
        
        Дальнейшую работу можно разбить на два этапа:
        \begin{enumerate}[(1)]
\item расширение инструментальных средств поддержки предметных 
посредников для виртуальной интеграции SciDB-ресурсов: 
\begin{itemize}
\item[(а)] расширение средств регистрации ресурсов в посреднике~\cite{10-stu} 
трансформацией ЯОД\ ADM в каноническую модель; 
\item[(б)] создание 
SciDB-адап\-те\-ра~--- специального ПО, связывающего исполнительную 
среду посредников с SciDB-ресурсами (составной частью адаптера является 
разработанная трансформация ЯМД);
\end{itemize}
\item применение технологии предметных посредников для решения 
научных задач в некоторой предметной области над множеством\linebreak 
неоднородных ресурсов, включающим SciDB-ре\-сурсы.
\end{enumerate}

\bigskip
        Автор выражает благодарность Л.\,А.~Калиниченко, П.\,Е.~Велихову и 
А.\,Е.~Вовченко за полезные замечания, высказанные в ходе обсуждения 
данной работы на семинарах ИПИ РАН.

\vspace*{-6pt}

{\small\frenchspacing
{%\baselineskip=10.8pt
\addcontentsline{toc}{section}{Литература}
\begin{thebibliography}{99}

\vspace*{-2pt}

\bibitem{1-stu} %1
Challenges and opportunities with big data~// A~community white paper developed 
by leading researchers across the United States, 2012. {\sf http://cra.org/ccc/docs/ init/bigdatawhitepaper.pdf}. 

\bibitem{1-2-stu} %2
\Au{Abrial J.-R.} The B-Book: Assigning programs to 
meanings.~--- Cambridge: Cambridge University Press, 1996. 

\bibitem{2-stu} %3
\Au{Vassiliadis P., Sellis T.\,K.} A~survey of logical models for OLAP databases~// SIGMOD 
Record, 1999. Vol.~28. No.\,4. P.~64--69. 

\bibitem{3-stu}
\Au{Pedersen T.\,B., Jensen C.\,S.} Multidimensional database technology~// IEEE Computer, 
2001. Vol.~34. No.\,12. P.~40--46. 

\bibitem{4-stu} %5
\Au{Libkin L., Machlin R., Wong~L.} A~query language for multidimensional arrays: Design, 
implementation, and optimization techniques.~--- SIGMOD, 1996. P.~228--239. 
\bibitem{5-stu} %6
\Au{Baumann P.} A~database array algebra for spatio-temporal data and beyond~// Next 
generation information technologies and systems. Lectures notes in computer science ser.
Springer Verlag KG, 1999. Vol.~1649. P.~76--93.
\bibitem{6-stu} %7
Overview of SciDB: Large scale array storage, processing and analysis. The SciDB development 
team.~--- SIGMOD, 2010. 
\bibitem{7-stu}
Large synoptic survey telescope. {\sf http://www.lsst.org}. 
\bibitem{8-stu}
\Au{Becla J., Lim K.-T.} Report from the First Workshop on Extremely Large Databases~// Data 
Sci.~J., 2008. Vol.~7.
\bibitem{9-stu}
SciDB User's Guide. Version~12.3, 2012. {\sf http:// www.scidb.org}.
\bibitem{10-stu}
\Au{Kalinichenko L.\,A., Briukhov D.\,O., Martynov~D.\,O., Skvortsov~N.\,A., Stupnikov~S.\,A.} 
Mediation framework for enterprise information system infrastructures~// Volume Databases and 
Information Systems Integration: 9th Conference (International) on Enterprise Information 
Systems (ICEIS 2007) Proceedings ~--- Funchal, 2007. P.~246--251.
\bibitem{11-stu}
\Au{Захаров В.\,Н., Калиниченко Л.\,А., Соколов~И.\,А., Ступников~С.\,А.} Конструирование 
канонических информационных моделей для интегрированных информационных 
сис\-тем~// Информатика и её применения, 2007. Т.~1. Вып.~2. C.~15--38. 
\bibitem{12-stu}
\Au{Kalinichenko L.\,A., Stupnikov S.\,A.} Heterogeneous information model unification as a 
prerequisite to resource schema mapping~// Information Systems: People, Organizations, 
Institutions, and Technologies: 5th Conference of the Italian Chapter of Association for 
Information Systems itAIS Proceedings.~--- Berlin--Heidelberg: Springer Physica Verlag, 2010. 
P.~373--380. 
\bibitem{13-stu}
\Au{Kalinichenko L.\,A., Stupnikov S.\,A., Martynov~D.\,O.} SYNTHESIS: A~language for 
canonical information modeling and mediator definition for problem solving in heterogeneous 
information resource environments.~--- Moscow: IPI RAN, 2007. 171~p. 
\bibitem{14-stu}
\Au{Брюхов Д.\,О., Вовченко А.\,Е., Захаров~В.\,Н., Желенкова~О.\,П., Калиниченко~Л.\,А., 
Мартынов~Д.\,О., Скворцов~Н.\,А., Ступников~С.\,А.} Архитектура промежуточного слоя 
предметных посредников для решения \mbox{задач} над множеством интегрируемых 
неоднородных распределенных информационных ресурсов в гиб\-рид\-ной 
грид-ин\-фра\-струк\-ту\-ре виртуальных обсерваторий~// Информатика и её применения, 
2008. Т.~2. Вып.~1. С.~2--34. 

\bibitem{15-stu} %16
\Au{Kersten M.\,L., Zhang~Y., Ivanova~M., Nes~N.} SciQL, a query language for science 
applications~// EDBT/ICDT~--- Workshop on Array Databases 2011 Proceedings.~--- Uppsala, 
Sweden, 2011. P.~1--12.

\bibitem{16-stu} %17
\Au{Abrial J.-R.} The B-Book: Assigning programs to meanings.~--- Cambridge: Cambridge 
University Press, 1996.
\bibitem{17-stu} %18
Astronomy in ArrayDB. 
{\sf http://trac.scidb.org/\linebreak raw-attachment/wiki/UseCases/Astronomy\%20in\%20\linebreak
ArrayDB.pdf }
\bibitem{18-stu} %19
ATL Project. {\sf http://www.eclipse.org/m2m/atl}.
\bibitem{19-stu} %20
\Au{Budinsky F., Steinberg D., Ellersick~R., Grose~T.}
Eclipse modeling framework. Ch.~5: Ecore modeling concepts.~--- Addison Wesley 
Professional, 2004.
\bibitem{20-stu} %21
Meta Object Facility (MOF) 2.0 Core Specification, 2003. 
{\sf http://www.omg.org/cgi-bin/apps/doc?ptc/\linebreak 03-10-04.pdf}. 
\bibitem{21-stu} %22
\Au{Kalinichenko L.\,A.} Method for data models integration in the common paradigm~//  1st 
East-European Symposium on Advances in Databases and Information Systems \mbox{ADBIS'97} 
Proceedings.~--- St.-Petersburg: Nevsky Dialect, 1997. Vol.~1: Regular papers. P.~275--284.
\bibitem{22-stu}
Atelier~B: The industrial tool to efficiently deploy the B Method. 
{\sf http://www.atelierb.eu/index-en.php}.

\label{end\stat}

\bibitem{23-stu} %24
\Au{Van Ballegooij A.} RAM: Array database management through relational mapping~// SIKS 
Dissertation ser. No.\,2009-25. {\sf http://oai.cwi.nl/oai/asset/14074/ 14074D.pdf}.
         
\end{thebibliography}
} }

\end{multicols} %14
\def\stat{suchkov}

\def\tit{ИНФОРМАЦИОННО-АНАЛИТИЧЕСКАЯ АВТОМАТИЗИРОВАННАЯ 
СИСТЕМА <<МЕГАЛИТ>> В~ОПТИМИЗАЦИИ ДИАГНОСТИКИ И ЛЕЧЕНИЯ МОЧЕКАМЕННОЙ БОЛЕЗНИ}

\def\titkol{Информационно-аналитическая автоматизированная 
система <<Мегалит>> в~оптимизации диагностики} % и лечения мочекаменной болезни}

\def\autkol{М.\,П.~Кривенко, С.\,А.~Голованов,  П.\,А.~Савченко
 и др.}

\def\aut{М.\,П.~Кривенко$^1$, С.\,А.~Голованов$^2$, П.\,А.~Савченко$^3$, 
А.\,В.~Сивков$^4$,  А.\,П.~Сучков$^5$}

\titel{\tit}{\aut}{\autkol}{\titkol}

%{\renewcommand{\thefootnote}{\fnsymbol{footnote}}\footnotetext[1] {Статья 
%рекомендована к публикации в журнале Программным комитетом конференции 
%<<Электронные библиотеки: перспективные методы и технологии, электронные 
%коллекции>> (RCDL-2012).}}

\renewcommand{\thefootnote}{\arabic{footnote}}
\footnotetext[1]{Институт проблем информатики Российской академии наук, mkrivenko@ipiran.ru} 
\footnotetext[2]{Научно-исследовательский институт урологии, sergeygol124@mail.ru} 
\footnotetext[3]{Институт проблем информатики Российской академии наук, psavchenko@ipiran.ru} 
\footnotetext[4]{Научно-исследовательский институт урологии, uroinfo@yandex.ru} 
\footnotetext[5]{Институт проблем информатики Российской академии наук, asuchkov@ipiran.ru}


\Abst{В статье, первой из предполагаемой серии научных публикаций, рассматриваются 
результаты исследований по автоматизации информационных и аналитических процессов 
обследования, диагностирования и лечения мочекаменной болезни (МКБ). Существенную 
роль в создании систем диагностики МКБ играет разработка информационных технологий 
сбора клинических данных и формирования специализированных баз данных (БД). Изучена 
возможность создания и способы реализации ин\-фор\-ма\-ци\-он\-но-ана\-ли\-ти\-че\-ской 
автоматизированной системы (ИААС) по сбору, хранению и обработке клинических данных 
обследования больных, а также алгоритмизации процессов принятия решений при 
диагностике МКБ и выборе схем лечения и профилактики этого заболевания. Предложенные 
математические методы и алгоритмы могут найти применение при дальнейшем развитии 
фундаментальных научных исследований в области разработки математических методов 
моделирования ме\-ди\-ко-био\-ло\-ги\-че\-ских сис\-тем, а также при создании необходимого 
математического инструментария.}
      
\KW{информационно-аналитическая система; урология; компьютерная диагностика; схема 
лечения; схема профилактики}

\DOI{10.14357/19922264130409}

\vskip 14pt plus 9pt minus 6pt

      \thispagestyle{headings}

      \begin{multicols}{2}

            \label{st\stat}

\section{Введение}

      В настоящее время доля людей, у которых на протяжении их жизни диагностируется 
МКБ, довольно значительна и составляет в странах Западной Европы 5\%--9\%, в Канаде и 
США~--- 7\%--12\%, в странах Азии~--- 1\%--5\%~[1--4]. 
  %    
      Эпидемиологические исследования, проводимые в ряде индустриально развитых стран, 
указывают на сохранение тенденции к росту частоты возникновения МКБ 
среди населения. Так, число  впервые выявленных случаев
МКБ на 100\,000 населения за последние 
десятилетия возросло в США с 58,7 (1950--1954~гг.)\ до 85,1 (2000~г.)~\cite{4-su, 3-su}, 
в Японии~--- с~43,7 
(1965~г.)\ до~134 (2005~г.)~\cite{6-su, 5-su}, в России~--- со 123,3 (2002~г.)\ до~138,6 
(2010~г.)~[7, 8].
      
      По данным исследований~[9] с использованием БД Pediatric Health 
Information System (национальная БД, в которую включены данные об амбулаторных 
визитах, срочных госпитализациях и стационарном лечении детей из 42~детских больниц 
США) по сравнению с общим количеством госпитализированных пациентов число пациентов 
с МКБ увеличилось с 18,4 на 100\,000 населения в 1999~г.\ до 57,0 в 2008~г., годовой прирост 
составил 10,6\% ($p \hm<0{,}0001$). 
      
      В основе развития МКБ лежат характерные нарушения обмена веществ, приводящие к 
образованию камней в мочевых путях. Эти литогенные (камнеобразующие) нарушения обмена 
веществ характеризуются большим многообразием и проявляются различными 
патологическими изменениями биохимического состава крови и мочи пациента.
      
      Необходимым условием для выбора правильной тактики консервативного лечения с 
целью предупреждения повторного камнеобразования является исследование всего комплекса 
метаболических факторов риска (МФР), ответственных за развитие МКБ.
{\looseness=-1

}
      
      В этой связи большое внимание придается изуче\-нию особенностей фи\-зи\-ко-хи\-ми\-че\-ских 
па\-ра\-мет\-ров мочи, во многом определяющих вероятность образования мочевых камней~[10]. Кроме 
того, литогенные нарушения метаболизма зачастую имеют сложный многофакторный 
характер воздействия на процесс формирования камня. Это создает особые трудности для 
врача в полной и объективной оценке всех влияющих литогенных факторов обмена веществ, а 
также в принятии решения по диагностике и выбору лечебной тактики для конкретного 
больного. Отсюда возникает необходимость формирования БД анкетных и 
лабораторных исследований, систем, связанных с диагностикой МКБ и формирования базы 
знаний по профилактике и лечению этого заболевания.
      
\section{Системы компьютерной диагностики в~области урологии}

      Существенную роль в создании систем диагностики МКБ является разработка 
информационных технологий сбора клинических данных и формирования 
специализированных БД. К~ним относится упомянутая Pediatric Health Information 
System. В~ряде медицинских работ упоминается реестр по уролитиазу (БД по 
больным и результатам лечения) Юго-за\-пад\-но\-го медицинского центра Техасского 
университета: <<Retrospective data from the University of Texas Southwestern Medical Center 
\textit{Nephrolithiasis Registry} from 17~studies that dealt with physiologic and physicochemical 
effects of various magnesium and potassium salts were categorized into three groups and 
analyzed\ldots>>~[11]. Однако подробного описания данного реестра не приведено. 
      
      Задачи диагностики, дифференциальной диагностики, прогнозирования, выбора 
стратегии и тактики лечения позволяют решать экспертные медицинские системы~[12].
      
      Ряд работ посвящен использованию в урологии компьютерных диагностических систем 
на основе методов искусственных нейронных сетей (ИНС)~[13].
 Так, в онкоурологии смогли 
прогнозировать 5-лет\-нюю выживаемость пациентов, перенесших радикальную цистэктомию 
по поводу\linebreak
рака мочевого пузы\-ря~[14]. Искусственные ней\-ронные сети применили также для 
автоматизи\-рованного анализа показаний к биопсии предстательной железы~[15]. 
Методика основывалась на\linebreak 
выявлении общего прос\-тат-спе\-ци\-фи\-че\-ско\-го антигена (ПСА) и определении доли 
свободного ПСА. Чувствительность составила 95\%, специфичность~--- 34\%. При 
дополне\-нии нейросети моделью логистической регрессии специфичность возросла до 95\%. 
Искусственная нейронная сеть использовалась для выявле\-ния группы риска рака предстательной железы в сравнении с 
моделью логистической регрессии~[15]. Искусственная нейронная сеть так\-же 
продемонстрировала более точные 
прогностические возможности. Компьютерных систем диагностики именно МКБ по 
литературным данным не выявлено.
{ %\looseness=-1

}
      
      Отсюда ясно, что имеется настоятельная необходимость разработки аналитической 
системы диагностики и лечения больных МКБ в процессе их динами\-ческого наблюдения 
(мониторинге) для пред\-упреж\-де\-ния повторного камнеобразования. Отсутствие подобных 
аналитических систем для мониторинга больных МКБ послужило основанием для разработки 
опытного образца ИААС 
<<Мегалит>>. Создание системы осуществляется ИПИ РАН совместно с НИИ урологии 
Минздравсоцразвития России в рамках серии совместных на\-уч\-но-ис\-сле\-до\-ва\-тель\-ских работ.
      
\subsection*{Основные цели и~задачи создания информационно-аналитической автоматизированной системы
 <<Мегалит>>}

      \noindent
      \begin{enumerate}[1.]
      \item  Создание БД по результатам обследования пациента, включающей:\\[-15pt]
      \begin{itemize}
\item формализованные данные опроса пациента при первом и последующих визитах, 
содержащие информацию о факторах, способных оказывать влияние на возникновение и 
особенности клинического течения МКБ (lifestyle-фак\-то\-ры индивида, факторы среды, 
питания, профессии и проч.);\\[-15pt]
\item данные лабораторного обследования (результатов простого или расширенного 
лабораторного обследования).\\[-15pt] 
\end{itemize}
      \item  Создание аналитической подсистемы, обеспечивающей решение следующих 
задач:\\[-15pt]
      \begin{itemize}
\item на основании данных первичного опроса выявление наличия или отсутствия, а также 
степень риска развития МКБ и определение объема предполагаемого лабораторного 
обследования пациента (простое или расширенное обследование);\\[-15pt]
\item на основе анализа входных данных лабораторного обследования осуществление выбора 
дальнейшей тактики ведения больного~--- дополнительные виды исследования, выбор 
лечебных мероприятий (тип хирургического лечения, схема медикаментозной терапии, 
коррекция диеты и проч.);\\[-15pt]
\item реализация методов оптимального выбора (с учетом показаний и противопоказаний) вида 
хирургического лечения или схемы проведения профилактического лечения (включая прием 
специальных фармпрепаратов, рекомендации по модификации диеты и образа жизни).
\end{itemize}
\end{enumerate}

        При создании опытного образца ИААС <<Мегалит>> учитывались следующие 
требования.
      \begin{enumerate}[1.]
\item Опытный образец аналитической системы <<Мегалит>> должен иметь возможность 
ведения распределенной БД пациентов, содержащей результаты обследований, профили 
МФР и относительный индекс перенасыщенности мочи (ОИП) как 
исходные, так и измененные в результате назначенного лечения, и включать набор подсистем, 
вклю\-ча\-ющих программную реализацию разработанных методов диагностирования и лечения.
\item Данные простого лабораторного обследования пациента должны включать: 
\begin{itemize}
\item исследование химического состава мочевого камня; 
\item биохимическое исследование крови и мочи по различным параметрам; 
\item клинический анализ мочи с посевом на мик\-ро\-флору; 
\item обзорный рентгеновский снимок, сонограмму и другие виды инструментального 
обследования пациента.
\end{itemize}
%\end{enumerate}
       Данные биохимического исследования представлены величинами содержания в крови 
и моче ионов и соединений, способных приводить к образованию мочевых камней. При 
наличии патологических отклонений в биохимических исследования проводится расширенное 
лабораторное обследование.
 \item Данные расширенного лабораторного обследования включают протокол диагностики 
типа гиперкальциурии (ПД-ГКУ) (при выявлении повышен\-ной суточной экскреции кальция у 
пациента). Выполняется поэтапно, с помощью модифицированной по кальцию диеты. 
В~расширенное лабораторное обследование входит также полный диагностический протокол (ПДП)
больного МКБ. 
\item Полный диагностический протокол пред\-став\-ля\-ет собой выраженное в 
графическом виде исходное состояние обмена веществ у пациента с МКБ с выявленными 
МФР и динамику изменения показателей обмена веществ 
в результате проводимого лечения. Графическое отображение МФР и ОИП больного МКБ 
позволяет оценить степень выявленных нарушений и их динамику в процессе 
профилактического лечения и вносить в лечебную схему необходимые коррекции, также 
выбираемые по особому алгорит\-му. Выявленные при первичном обследовании МФР и ОИП 
служат основой для програм\-мно\-го выбора схем коррекции метаболических нарушений и 
предупреждения рецидивов МКБ. Коррекция включает в себя лечебные мероприятия, прием 
специальных фармпрепаратов, рекомендации по модификации диеты и образа жизни.
\item В~аналитической системе <<Мегалит>> пред\-усмат\-ри\-ва\-ет\-ся возможность ее обучения и 
настройки на основе получаемых новых данных о результатах лечения пациента (пациентов) 
на \mbox{каждом} этапе наблюдения.
\item Предусмотреть в программной реализации алгоритмов экспертного модуля: 
\begin{itemize}
\item
алгоритм оценки эффективности выбранной схемы лечения;
\item
алгоритм принятия решения по дальнейшему лечению;
\item
алгоритм поиска и выбора рациональной схемы профилактического лечения.
\end{itemize}
\item Оценить возможности разработки метода корректировки параметров подсистемы 
диагностирования и лечения на основе анализа вновь поступающих данных (обратная связь).
\item Разработанные аналитические методы и алгоритмы, реализованные в составе опытного 
образца аналитической системы <<Мегалит>>, должны пройти апробацию и тестирование в 
реальных клинических условиях. По результатам применения опытного образца должны быть 
сформулированы рекомендации по его совершенствованию и развитию.
\end{enumerate}

\begin{figure*} %fig1
   \vspace*{1pt}
 \begin{center}
 \mbox{%
 \epsfxsize=143.69mm
 \epsfbox{such-1.eps}
 }
 \end{center}
 \vspace*{-6pt}
\Caption{Структура ИААС}
\end{figure*}

\section{Основные подходы к~созданию информационно-аналитической
автоматизированной системы <<Мегалит>> и~их~реализация}
      Основные функции ИААС:
      \begin{itemize}
\item сбор и формализация данных, включая ведение реестра пациентов, системы словарей и 
справочников;
\item поддержка принятия решения по назначению и сбор данных диагностических 
исследований;
\item первичный и ретроспективный анализ тестов;
\item поддержка принятия решения по выбору схемы лечения, оценка эффективности схемы 
лечения;
\item поддержка принятия решения по дальнейшему лечению;
\item поиск и поддержка принятия решения по выбору рациональной схемы 
профилактического лечения.
\end{itemize}

      Информационно-аналитическая автоматизированная система
       <<Мегалит>> включает в себя подсистемы:
      \begin{itemize}
\item администрирования;
\item регистрации пациентов и сбора данных анкет, анамнеза;
\item ведения лингвистического обеспечения;
\item первичного обследования;
\item диагностических исследований;
\item экспертный модуль (поддержки процессов лечения).
\end{itemize}

Структурная схема ИААС <<Мегалит>> пред\-став\-ле\-на на рис.~1.


\begin{figure*}[b] %fig2
   \vspace*{1pt}
 \begin{center}
 \mbox{%
 \epsfxsize=161.589mm
 \epsfbox{such-2.eps}
 }
 \end{center}
 \vspace*{-6pt}
\Caption{Пирамида анализа данных}
\end{figure*}


      Для обеспечения возможности коллективной работы по формированию БД
системы и многопользовательского режима работы с ее аналитическим модулем она 
проектируется в виде веб-сай\-та, доступного авторизованным пользователям в сети Интернет. 
Основные базовые функции информационного сайта должны быть реализованы 
общесистемным функционалом его платформы. 
%
Таким образом, 
в про\-грам\-мно-тех\-но\-ло\-ги\-че\-ской платформе должны быть заложены следующие функции:
      \begin{enumerate}[(1)]
\item выполнение приложений~--- позволяет легко разрабатывать, развертывать различные 
приложения и управлять ими;
\item возможность совместной работы~--- позволяет отдельным пользователям и крупным 
организациям объединить свои ресурсы и работать вместе через Интернет;
\item управление содержимым~--- придает гибкость производству и управлению 
отдельными веб-уз\-ла\-ми, позволяя поставлять конечному пользователю 
приспособленное под него (персонифицированное) содержимое сайта;
\item управление пользователями~--- позволяет организации управлять пользователями, 
ресурсами и безопасностью внутри и вне системы сетевой защиты, а также предоставлять 
канал для внешних связей и проведения электронных транзакций;
\item контроль и управление про\-из\-во\-ди\-тель\-ностью~--- позволяет улучшать качество 
пользовательского интерфейса, обеспечивая:
\begin{itemize}
\item управление знаниями~--- помогает объединять внутреннюю и внешнюю 
информацию и предоставлять информацию, основанную на контекстной 
концепции;
\item поддержку поиска~--- обеспечивает клиента доступом к широкому спектру 
источников информации как внутри, так и вне сайта;
\item безопасность~--- защиту данных, приложений и транзакций;
\item стандартный www-до\-ступ к сайту~--- для технического обеспечения 
функционирования его содержимого.
\end{itemize}
\end{enumerate}
      
      Определяющими характеристиками веб-ре\-шений являются масштабируемость, 
доступность, надежность, защита данных от несанкционированного доступа, транзакционная 
целостность и распространение.
      
      Важной особенностью платформы является то, что она объединяет все необходимые 
модули, которые позволяют выполнять практически любую работу, связанную с созданием и 
обновлением сайта специалистом предметной области. 
      
      В качестве языка программирования выбран один из самых современных языков~--- 
C\#. ASP.NET~--- технология, которая является частью .NET и используется для разработки 
ин\-тер\-нет-ори\-ен\-ти\-ро\-ван\-но\-го программного обеспечения и ин\-тер\-нет-сай\-тов. 

Для работы ин\-тер\-нет-сай\-тов используется связка: операционная система Windows Server 
2008\;+\;ин\-тер\-нет-сер\-вер IIS~7.0\;+\;СУБД Ms SQL Server~2008.

\section{Концепция экспертного модуля системы}

\subsection{Основные подходы к~использованию статистических методов анализа данных 
в~урологии}

      Клинические БД содержат большое количество информации о пациентах и их 
заболеваниях. Скрытые (латентные) связи и структуры в этих данных могут быть источником 
новых медицинских знаний. К~сожалению, немногие из существующих технологий анализа 
данных оказываются непосредственно применимыми и действенными при обнаружении и 
описании этих латентных знаний, но, безусловно, универсальной из них является технология 
на принципах \textit{Data Mining}~--- извлечение скрытой информации из уже накопленных и 
пополняемых сведений об объекте исследования. Эта и ряд других сформировавшихся 
технологий, ориентированных на анализ массивов данных, терминологически пересекаются 
или оказываются взглядом на одном и том же, но с разных точек зрения (краткое освещение 
данного вопроса дано в~[17, разд.~1]). В~первую очередь речь идет о следующих 
подходах:
      \begin{itemize}
\item разведочный анализ данных~--- Exploratory Data Analysis (EDA);
\item извлечение скрытой информации из данных~--- Data Mining 
(DM);
\item обнаружение знаний в данных~--- Knowledge Discovery in Databases (KDD);
\item машинное обучение~--- Machine Learning (ML).
\end{itemize}

      Таким образом, обнаружение в данных ранее не известных, нетривиальных, 
практически полезных и доступных интерпретации знаний, необходимых для поддержки 
принятия решений в различных сферах человеческой деятельности, составляет суть DM. 
Говоря далее об анализе данных, будем понимать при этом цели, задачи, технологии, методы и 
алгоритмы, присущие DM. 
      
      На рис.~2 схематично изображена иерархия содержательной стороны анализа данных. 
Вертикальная стрелка слева показывает направление роста отдельных характеристик задач 
анализа данных в зависимости от их уровня. Примеры постановок практических задач 
приведены справа. Надо понимать, что <<восхождение>> по пирамиде анализа данных 
должно обеспечиваться обязательным существенным ростом объема используемой 
информации (данных и предположений об объектах исследования), а также глубиной 
проработки вопроса о качестве предлагаемых решений.


      
      \textbf{Основные принципы анализа данных.} Среди методов, которые 
использовались при решении проб\-ле\-мы обучения в ML, те, которые представляют\linebreak 
наибольший интерес при анализе данных (снижение размерности, оценивание распределения 
данных, регрессионный анализ, классификация, клас\-те\-ри\-за\-ция), теперь все вместе 
упоминаются как статисти\-ческое обучение. Проблема обучения делится на различные 
категории: две из них, наиболее близкие к статистике, суть контролируемое обучение или 
обучение с <<учителем>> и не\-конт\-ро\-ли\-ру\-емое, без <<учителя>>.
      
      Одна из самых важных задач в анализе данных состоит в том, чтобы оценить качество 
полученных решений, в частности точность предложенного прогноза (например, качество 
построенного классификатора). В~качестве меры точности прогноза обычно используется 
ошибка прогноза. Простейшая оценка ошибки прогноза строится с помощью тех же данных, 
которые используются для построения модели (такой вариант оценки называют самооценкой, 
оценкой переподстановки). Понятно, что в результате сформируется чрезмерно 
оптимистический взгляд на точность прогноза. 
      
      Очевидный способ улучшения состоит в обобщении: оценивать точность прогноза с 
помощью данных, независимых от тех, которые использовались для подгонки модели. 
Получить подобные независимые данные можно путем сбора новых данных. Если это 
невозможно, то имеет смысл разделить исходные данные на части и воспользоваться ими для 
решения самостоятельных задач. Обычная практика заключается в следующем: если набор 
данных достаточно велик, то необходимо использовать случайный механизм для разделения 
данных на два непересекающихся и независимых набора: 
      \begin{enumerate}[(1)]
\item данные для обучения, которые можно использовать для предварительного контроля 
данных, для формирования моделей;
\item тестирующие данные, которые будут использоваться для оценки 
качества построенной модели.
      \end{enumerate}
      
      Альтернативные методы расщепления данных для того, чтобы оценить тестовую 
ошибку, основаны на перепроверке~[16] и бут\-стреп-ме\-то\-де~[17].
     
     Суть вероятностной модели бутстреп-метода в данном случае состоит в следующем. 
Предположим, что по выборке $x\hm=(x_1,\ldots ,x_N)$ данных лабораторных исследований 
из распределения $F(u)$ оценивается значение $\vartheta\hm=\vartheta(F)$ некоторого 
функционала (например, классификатора заболеваний), заданного на семействе~$\mathbf{F}$. 
Качество оценки $\vartheta^*(X)$ характеризуется величиной
     $$
     R(\vartheta^*(X),\vartheta(F))=E_F\{L(\vartheta^*(X),\vartheta(F))\}\,,
     $$
где $L(\vartheta^*(X),\vartheta(F))$~--- потери от принятия оценки $\vartheta^*(X)$ вместо 
неизвестного значения $\vartheta(F)$. Бут\-стреп-ме\-тод позволяет оценить $ 
R(\vartheta^*(X),\vartheta(F))$ с помощью замены распределения~$F$ его некоторой оценкой 
$F^B$ и вычисления статистики $\vartheta^*$ по выборке $x^B$ объемом~$N$ из~$F^B$. 
Совокупность $x^B$ называется бут\-стреп-вы\-бор\-кой, статистика $\vartheta^*(x^B)$~--- 
бут\-стреп-реа\-ли\-за\-ци\-ей~$\vartheta^*$. 
     
     Условное распределение
     \begin{multline*}
     \mathrm{Pr}\left\{ \vartheta^*\left( X^B\right) <u\vert x_1,\ldots , x_N\right\} = {}\\
     {}=
     \int\limits_{\{y:\ \vartheta^*(y)<u\}} dF^B (y_1)\cdots dF^B(y_N)
     \end{multline*}
является бут\-стреп-оцен\-кой функции распределения $\mathrm{Pr}\left\{ 
\vartheta^*(X)<u\right\}$ статистики~$\vartheta^*$. 
     
     Процедура выбора оценки $F^B$ для~$F$ мотивируется наличием априорной 
информации. В~параметрической ситуации, когда $\mathbf{F}\hm= \left\{ F_\lambda, \, 
\lambda\in \Lambda\right\}$, оценка $F^B$ часто оказывается результатом подстановки 
вместо~$\lambda$ некоторой оценки~$\lambda^*$, т.\,е.\ $F^B\hm=F_{\lambda^*}$. 
{\looseness=1

}

Другая  ситуация относится к области непараметрической статистики. Здесь $F^B$ обычно 
оказывается эмпирической функцией распределения, т.\,е.\ каждому наблюденному значению 
(элементу исходной выборки) приписывается вероятность $1/N$. Бут\-стреп-вы\-бор\-ки тогда 
подчиняются условному полиномиальному распределению, сосредоточенному на  $x_1,\ldots , 
x_N$. 
     
     Наиболее трудную часть бутстреп-метода со\-став\-ля\-ет нахождение распределения 
$\vartheta^*(X^B)$, для чего применяются три приема:
    \begin{enumerate}[(1)]
\item прямое теоретическое вычисление;
\item аппроксимация с помощью метода статистических испытаний;
\item аппроксимация с помощью аналитических методов (например, используя разложение в 
ряд Тейлора).
\end{enumerate}

     Прямое теоретическое вычисление распределения $\vartheta^*(X^B)$ может 
осуществляться либо аналитическим путем, либо путем непосредственного перечис\-ле\-ния 
     бут\-стреп-вы\-бо\-рок и подсчета соответствующих вероятностей. Если оба приема 
недоступны (первый из-за аналитических сложностей, второй из-за вычислительных), то 
приходится прибегать к методу статистических испытаний, т.\,е.\ к повторению экспериментов 
по случайному формированию бут\-стреп-вы\-бор\-ки $x^B$ и подсчету значения 
$\vartheta^*(x^B)$.
     
     Следует обратить внимание на реальные возможности бут\-стреп-ме\-то\-да: он не 
позволяет получить новую информацию о наблюдаемых объектах, его назначение~--- 
сформировать объективное представление о свойствах использованных процедур анализа 
данных.
      
      В аналитической системе <<Мегалит>> накапливаются данные следующих типов:
      \begin{enumerate}[1.]
\item Неформализованные (неструктурированные), представленные в виде текста (например, 
текст назначения врача).
\item Формализованные:
\begin{enumerate}[{2.}1.]
\item Качественные:
\begin{enumerate}[{2.1.}1.]
\item Измеренные по шкале наименований (например, пол пациента).
\item Измеренные по порядковой шкале (например, порядковый номер сезона, когда 
обследовался пациент).
\end{enumerate}
\item Количественные:
\begin{enumerate}[{2.2.}1.]
\item Измеренные по одной из соответствующих шкал и при\-ни\-ма\-ющие значения из 
небольшого \mbox{набора} числовых значений (например, дата взятие анализов или 
количество обнаруженных у пациента камней).
\item Измеренные по одной из соответствующих шкал и принимающие значения в виде 
действительных чисел (например, уровень кальция в анализе крови пациента).
\end{enumerate}
\end{enumerate}
\end{enumerate}
      
      Приведенный систематизированный перечень встречающихся типов данных требует 
привлечения разнообразного арсенала средств, таких как лингвистический анализ (п.~1), 
статистический анализ категориальных данных (п.~2.1.1), ранговые процедуры 
(п.~2.1.2), статистический анализ на основе моделей дискретных и непрерывных 
распределений (пп.~2.2.1 и~2.2.2).
      
      Ошибки есть во всех видах БД; к сожалению, встречаются они и в данном 
случае. В~различных прикладных областях накоплен опыт (см., в частности,~[18]), 
позволяющий привести типичный перечень источников ошибок: фальсификация, неполнота, 
несогласованность, дублирование. 
      
      Те ошибки, которые легко обнаружить, вероятнее всего можно найти на стадии 
<<очистки>> данных, более же скрытые, неочевидные могут быть обнаружены только при 
анализе данных. <<Очистка>> данных обычно происходит, когда данные получены и прежде, 
чем они сохраняются в формате только для чтения в хранилище данных. В~частности, должны 
быть исключены ошибки, при которых переменные принимают значения, противоречащие 
естественным ограничениям (например, при описании химического состава камней значения 
отдельных переменных не могут превосходить 100\%). Доля подобных грубых ошибок в 
медицинских исследованиях может превышать~10\%. 
      
      Ошибки недопустимости значений должны быть описаны с помощью логических 
выражений, истинность которых проверяется на этапе <<очистки>>. В~случаях, когда их не 
удается исправить автоматически или автоматизированно, результат должен помечаться 
специальным образом. 
      
      Для данных, уже хранящихся в БД и явля\-ющих\-ся объектом анализа, могут 
быть характерны сле\-ду\-ющие проблемы: 
\begin{itemize}
\item наличие аномальных наблюдений (значения, которые 
существенно отличаются от основной массы наблюдений);
\item пропуски в данных;
\item малочисленность данных (ситуация, когда количество переменных превышает число 
наблюдений).
\end{itemize}
      
      Таким образом, статистический анализ конкретных данных является многоэтапным 
процессом, включающим планирование статистического исследования, организацию сбора 
необходимых статистических данных, первичное описание данных, оценивание характеристик 
данных, проверку статистических гипотез, анализ полученных решений, формулировку 
выводов, составление итоговых документов. 
      
      В основе принципов построения статистического вывода относительно данных лежат 
следующие положения:
      \begin{itemize}
\item при выборе семейства вероятностных распределений, описывающих данные, 
существенную роль играет предварительный анализ данных; последующий итеративный 
процесс уточнения априорных предположений направлен на построение модели, являющейся 
достаточно реалистичной и позволяющей строить содержательные выводы;
\item при построении методов анализа наряду с постановкой задач разработки оптимальных 
процедур и попыткой их решения следует не пренебрегать разумными подходами к созданию 
ка\-ких-ли\-бо процедур с последующим обязательным анализом предлагаемых решений; 
\item завершающим этапом построения методов анализа должен быть количественный или 
качественный анализ влияния на предлагаемые реше\-ния отклонений от априорных 
предположений, при этом исследование качества полученных решений реальнее всего 
проводить с по\-мощью бут\-стреп-ме\-тода.
\end{itemize}

\subsection{Принципиальные возможности создания экспертного модуля системы 
<<Мегалит>>}

      Повседневная деятельность врача требует решения задач интерпретации, диагностики, 
контроля и прогнозирования, т.\,е.\ таких задач, которые могут быть решены с помощью 
систем поддержки принятия решений. Медицина представляет одну из областей человеческой 
деятельности, где знания специалистов трудно формализуемы, однако разработка 
диагностических медицинских систем в настоящее время является актуальной задачей.
      
      При создании экспертного модуля системы <<Мегалит>>, предназначенного для 
поддержки принятия решения по диагностике заболевания, предполагается:
      \begin{itemize}
\item разработать систему представления медицинских знаний (с использованием данных 
анкет, анамнеза, данных инструментальных и лабораторных методов исследования);
\item разработать алгоритм механизма логического вывода (выполнение диагностики типа 
литогенного нарушения обмена веществ; выбора адекватной схемы лечения, оценки 
эффективности заданной схемы лечения с возможностью ее коррекции при дальнейшем 
мониторинге пациента).
\end{itemize}
      
      Система представления медицинских знаний позволяет выделять значимые для 
принятия врачебного решения или постановки медицинского диагноза данные (качественные 
или количественные). Так, анализ качественных данных анамнеза, анкетных данных позволяет 
сделать заключение о силе влияния наследственных, средовых и социальных факторов риска 
развития МКБ; потенциальной активности процесса камнеобразования. Этой же цели служат 
качественные и количественные данные, полученные при инструментальном/лабораторном 
(рентгенологическом, микробиологическом, ультразвуковом или антропометрическом) 
обследовании пациента.
      
      Большой массив количественных данных в виде числовых значений показателей 
получают при биохимическом исследовании. Именно он является основным объектом 
алгоритмизации при разработке экспертного модуля системы <<Мегалит>>. Применение 
этого модуля предназначено для объективной и более точной диагностики метаболического 
литогенного синдрома, выбора адекватной схемы лечения, качественной оценки результатов 
лечения, коррекции лечебной схемы на основе полученных данных в целях выбора 
оптимального лечебного воздействия на нарушенный обмен веществ у пациента.
      
      Учитывая, что указанные процессы являются алгоритмизуемыми, можно полагать, что 
принципиальные возможности создания экспертного модуля для системы <<Мегалит>> 
имеются.

\subsection{Качественное описание алгоритмической базы экспертного~модуля }

      Качественное описание алгоритмической базы включает: постановку задачи, описание 
входных и выходных данных, описание вход\-ных/вы\-ход\-ных форм пользовательского 
интерфейса, описание событий и реакций системы в рамках поддержки процесса оценки 
эффективности.
      
      \subsubsection*{Алгоритм оценки эффективности заданной схемы лечения}
      
      \paragraph*{Постановка задачи.} Качественно и количественно оценить эффективность 
применения выбранной схемы лечения (с выводом о продолжении ее использования в 
лечении; ее модификации в той или иной степени; замены на другую схему лечения).
      
      \paragraph*{Входные данные.} Входными данными служат те количественно измененные 
биохимические показатели, которые на предыдущем этапе про\-грам\-мно\-го анализа были 
определены (диагностированы) как характерные для данного метаболического синдрома. 
      
      \paragraph*{Выходные данные.} Выходными данными служат количественные значения 
биохимических признаков пролеченного метаболического синдрома, полученные при 
лабораторном исследовании после курса лечения. Эти данные должны быть про\-грам\-мно 
проанализированы в сравнении с их исходными (до начала лечения) значениями. 
Используется \textit{алгоритм <<Оценка качества лечебного эффекта>>}, который 
предполагает следующие варианты вывода о качестве лечения:
      \begin{itemize}
\item <<отсутствие эффекта>>;
\item <<слабо выраженный положительный эффект>>;
\item <<выраженный положительный эффект>>;
\item <<слабо выраженный отрицательный эффект>>; 
\item <<выраженный отрицательный эффект>>.
\end{itemize}
      
      \paragraph*{Выходные формы пользовательского интерфейса.} Выходные формы 
пользовательского интерфейса при этом могут быть представлены в виде таб\-ли\-цы со 
значениями биохимических признаков метаболического синдрома до и после лечения. 
Возможна опция отображения исходных данных в виде диаграммы или графика.

      \subsubsection*{Алгоритм принятия решения по дальнейшему лечению}
      
      \paragraph*{Постановка задачи.} Построить алгоритмические правила, позволяющие 
пользователю сделать вывод и принять решение об использовании данной схемы в 
дальнейшем лечении; модификации схемы в той или иной степени; замены данной схемы на 
другую схему лечения.
      
      \paragraph*{Входные данные:}
      \begin{itemize}
\item данные сравнительного анализа численных биохимических величин до и после лечения;
\item данные качественной оценки лечебного эффекта, получаемые в результате обработки 
данных сравнительного анализа численных биохимических величин до и после лечения.
     \end{itemize}
     
     При формировании и сборе данных первого типа выполняется процедура сравнения 
достигнутых в результате лечения величин значимых для данного метаболического синдрома 
показателей с их исходными значениями, диагностированными до начала лечения в ходе 
биохимического лабораторного исследования.
     
     Данные второго типа являются качественными, производными от данных первого типа. 
Эти данные представляют собой возможные варианты вывода о качестве лечения.
     
     \paragraph*{Выходные данные.} Выходными данными алгоритма принятия решения по 
дальнейшему лечению служат установленные типы рекомендаций по применявшейся схеме 
лечения: 
     \begin{itemize}
\item сохранение схемы без изменений и продолжение лечения;
\item модификация схемы трех степеней выраженности 
(незначительная, умеренная, существенная);
\item отказ от применения данной схемы и выбор новой схемы лечения.
\end{itemize}

\section{Перспективы развития информационно-аналитической автоматизированной
системы~<<Мегалит>>}

      Таким образом, разработана методологическая и техническая база для экспертной 
системы комплексной диагностики и профилактического \mbox{лечения} пациентов с МКБ. Учитывая 
не только медицинскую, но и социальную актуальность проб\-ле\-мы МКБ, а также трудности 
принятия врачебного решения в выборе адекватной тактики противорецидивного лечения 
этого заболевания, следует считать целесообразным создание технологий расширения и 
совершенствования функционала экспертного модуля системы <<Мегалит>> на основе 
анализа вновь поступающих данных с использованием принципа обратной связи.
      
      На следующих этапах планируется работа по оптимизации практического применения 
ИААС <<Мегалит>> в 
клинической урологии.
      
      Предложенные математические методы и алгоритмы найдут применение при 
дальнейшем развитии фундаментальных научных исследований в области разработки 
математических методов моделирования медико-биологических систем, а также при создании 
необходимого математического инструментария.
    %  
      В первую очередь это касается постановки развернутого диагноза, наиболее полно 
отражающего особенности метаболического типа конкретного больного (обследуемого), 
специфику функционального состояния почек и мочевых путей пациента. При этом также 
учитывается влияние различных модифицирующих факторов, таких как наличие или 
отсутствие инфекции и степени ее вы\-ра\-жен\-ности, воздействие социальных факторов, 
факторов питания, наследственности и~др. 
      
      Внедрение и практическое использование сис\-те\-мы <<Мегалит>> позволит 
сформировать представительный набор данных, на основе которого разработать методы 
выбора оптимальной лечебной тактики. Таким образом, конечные пользователи получат 
возможность дистанционной диагностики метаболических литогенных синдромов у пациента, 
оценки степени риска развития МКБ, выбора адекватных терапевтических схем лечения МКБ 
и/или профилактики рецидивов камнеобразования, ввода данных о пациенте в единый банк 
данных для последующего мониторинга и~др. 

{\small\frenchspacing
{%\baselineskip=10.8pt
\addcontentsline{toc}{section}{Литература}
\begin{thebibliography}{99}

\bibitem{1-su} %1
\Au{Ramello, A., Vitale C., Marangella D.} Epidemiology of nephrolithiasis~// J.~Nephrol., 2000. 
Vol.~13. Suppl.~3. P.~45--50.

\bibitem{4-su} %2
\Au{Trinchieri A., Coppi F., Montanari~E., Del Nero~A., Zanetti~G., Pisani~E.} Increase in the 
prevalence of symptomatic upper urinary tract stones during the last ten years~// Eur. Urol., 2000. 
Vol.~37. P.~23--25.

\bibitem{2-su} %3
\Au{Pearle M.\,S., Calhoun E.\,A., Curhan~G.\,C.} Urologic diseases in America project: 
Urolithiasis~// J.~Urology, 2005. Vol.~173. P.~848--857. 

\bibitem{3-su} %4
\Au{Lieske J.\,C., Pena de la Vega~L.\,S., Slezak~J.\,M., Bergstralh~E.\,J., Leibson~C.\,L., 
Ho~K.\,L., Gettman~M.\,T.} Renal stone epidemiology in Rochester, Minnesota: An update~// Kidney 
Int., 2006. Vol.~69. No.\,4. P.~760--764.


\bibitem{6-su} %5
\Au{Johnson C.\,M., Wilson D.\,M., O'Fallon~W.\,M., Malek~R.\,S., Kurland~L.\,T.} 
Renal stone epidemiology: A~25-year study in Rochester, Minnesota~// Kidney Int., 1979. Vol.~16. 
No.\,5. P.~624--631.

\bibitem{5-su} %6
\Au{Yasui T., Iguchi M., Suzuki~S., Kohri~K.} Prevalence and epidemiological characteristics of 
urolithiasis in Japan: National trends between 1965 and 2005~// Urology, 2008. Vol.~71. No.\,2. 
P.~209--213.

\bibitem{7-su}
Заболеваемость населения России в 2003~году: Статистические материалы.~--- М., 2004 
(электронная версия МЗ и СР РФ и ЦНИИ организации и информатизации здравоохранения 
МЗ и СР РФ). 
{\sf http:// www.minzdravsoc.ru/docs/mzsr/stat/17}.
\bibitem{8-su}
\Au{Аполихин О.\,И., Сивков А.\,В., Солнцева Т.\,В., Комарова~В.\,А.}
Анализ урологической заболеваемости в Российской Федерации в 2005--2010~годах~//
Экспериментальная и клиническая урология, 2012. №\,2. C.~4--12.
{\sf http://ecuro.ru/article/analiz-urologicheskoi-zabolevaemosti-v-rossiiskoi-federatsii-v-2005-2010-godakh}.
%\bibitem{9-su}
%Заболеваемость населения России в 2007~году: Статистические материалы.~--- М., 2008 
%(электронная версия МЗ и СР РФ и ЦНИИ организации и информатизации здравоохранения 
%МЗ и СР РФ). {\sf http:// www.minzdravsoc.ru/docs/mzsr/stat/27}.
\bibitem{11-su} %9
\Au{Routh J.\,C., Graham D.\,A., Nelson~C.\,P.} Epidemiological trends in pediatric urolithiasis at 
United States freestanding pediatric hospitals~// J.~Urology, 2010. Vol.~184. No.\,3. P.~1100--1104.

\bibitem{10-su} %10
\Au{Голованов С.\,А., Дрожжева В.\,В.}
Кристаллообразующая активность мочи при оксалатном уролитиазе~//
Экспериментальная и клиническая урология, 2010. №\,2. C.~24--29.
{\sf http://ecuro.ru/ article/kristalloobrazuyushchaya-aktivnost-mochi-pri-oksalatnom-urolitiaze}.
\bibitem{12-su} %11
\Au{Bonny O., Rubin A., Huang~Ch.-L., Frawley~W.\,H., Pak~C.\,Y.\,C., Moe~O.\,W.} Mechanism 
of urinary calcium regulation by urinary magnesium and pH~// J.~Am. Soc. Nephrol., 2008. Vol.~19. 
No.\,8. P.~1530--1537.
\bibitem{13-su} %12
\Au{Дюк В.\,А., Эмануэль В.\,Л.} Информационные технологии в 
ме\-ди\-ко-био\-ло\-ги\-че\-ских исследованиях.~--- СПб.: Питер, 2003. 525~с.
\bibitem{14-su} %13
\Au{Liew P.\,L., Lee Y.\,C., Lin~Y.\,C., \textit{et al}.} Comparison of artificial neural networks with 
logistic regression in prediction of gallbladder disease among obese patients~// Digest. Liver Dis., 2007. 
Vol.~39. No.\,4. P.~356--362.
\bibitem{15-su} %14
\Au{Bassi P., Sacco E., De Marco~V., \textit{et al}.} Prognostic accuracy of an artificial neural 
network in patients undergoing radical cystectomy for bladder cancer: A~comparison with logistic 
regression analysis~// BJU Int., 2007. Vol.~99. No.\,5. P.~1007--1012.
\bibitem{16-su} %15
\Au{Stephan C., Xu C., Finne~P., \textit{et al}.} Comparison of two different artificial neural 
networks for prostate biopsy indication in two different patient populations~// J.~Urology, 2007. 
Vol.~70. No.\,3. P.~596--601.
%\bibitem{17-su} 
%\Au{Chun F.\,K., Karakiewicz P.\,I., Briganti~A., \textit{et al}.} A~critical appraisal of logistic 
%regression-based nomograms, artificial neural networks, classification and regression-tree models, 
%look-up tables and risk-group stratification models for prostate cancer ~/ BJU Intern., 2007. Vol.~99. 
%No.\,4. P.~794--800.


\bibitem{19-su} %16
\Au{Stone M.} Cross-validatory choice and assessment of statistical predictions (with discussion)~// 
J.~Roy. Stat. Soc. B, 1974. Vol.~36. P.~111--147.

\bibitem{18-su} %17
\Au{Efron B.} Bootstrap methods: Another look at the jackknife~// Ann. Stat., 1979. Vol.~7. 
P.~1--26.

\bibitem{21-su} %18
\Au{Izenman A.\,J.} Modern multivariate statistical techniques.~--- Springer, 2008. 731~p. 
%\bibitem{20-su} %20
%\Au{Breiman L.} The 1991 census adjustment: Undercount or bad data~// Stat. Sci., 1994. 
%Vol.~9. P.~458--475.


\end{thebibliography} } }

\end{multicols}

\hfill{\small\textit{Поступила в редакцию 17.04.13}}
%\vspace*{12pt}

%\hrule

%\vspace*{2pt}

%\hrule

\newpage

\def\tit{THE INFORMATION-ANALYTICAL COMPUTER SYSTEM ``MEGALITH'' 
IN~OPTIMIZATION OF~THE~DIAGNOSIS AND~TREATMENT OF~UROLITHIASIS}

\def\titkol{The information-analytical computer system ``Megalith'' 
in~the~field of~urology}

\def\aut{M.\,P.~Krivenko$^1$, S.\,A.~Golovanov$^2$, P.\,A.~Savchenko$^1$, A.\,V.~Sivkov$^2$, 
 and~A.\,P.~Suchkov$^1$}
 
 \def\autkol{S.\,A.~Golovanov, M.\,P.~Krivenko, P.\,A.~Savchenko, et al.}


\titel{\tit}{\aut}{\autkol}{\titkol}

\vspace*{-12pt}


\noindent
$^1$Institute of Informatics 
Problems, Russian Academy of Sciences, Moscow 119333, Russian Federation\\
\noindent $^2$Research Institute of Urology, Moscow 105425, Russian Federation

\vspace*{12pt}

\def\leftfootline{\small{\textbf{\thepage}
\hfill INFORMATIKA I EE PRIMENENIYA~--- INFORMATICS AND APPLICATIONS\ \ \ 2013\ \ \ volume~7\ \ \ issue\ 4}
}%
 \def\rightfootline{\small{INFORMATIKA I EE PRIMENENIYA~--- INFORMATICS AND APPLICATIONS\ \ \ 2013\ \ \ volume~7\ \ \ issue\ 4
\hfill \textbf{\thepage}}}

\Abste{In this article, that is the first of an expected series of scientific publications, the results of 
research on automation of the information and analytical processes of the urolithic disease (ULD) 
survey, diagnosis, and treatment are discussed. A significant role in creating the systems of ULD 
diagnostics has the development of information technologies for clinical data collection and 
formation of specialized databases. The possibility of creation and the ways of realization of 
information-analytical computer system of collection, storage, and processing of the clinical data of 
patients examination, as well as programming decision-making processes in the diagnosis ULD and 
the choice of schemes of treatment and prevention of this disease has been studied. The developed 
mathematical methods and algorithms may be applied to the further fundamental scientific researches 
in the field of development of mathematical methods of medical and biological systems modeling; 
besides, they may be applied for necessary mathematical tools creation.}

\KWE{informational-analytical system; urology; computer diagnostics; treatment 
scheme; scheme of prevention} 

\DOI{10.14357/19922264130409}

%\Ack
%\noindent
%?????

\vspace*{3pt}

  \begin{multicols}{2}

\renewcommand{\bibname}{\protect\rmfamily References}
%\renewcommand{\bibname}{\large\protect\rm References}

{\small\frenchspacing
{%\baselineskip=10.8pt
\addcontentsline{toc}{section}{References}
\begin{thebibliography}{99}


\bibitem{1-su-1}
\Aue{Ramello, A., C.~Vitale, and D.~Marangella}. 2000. Epidemiology of nephrolithiasis. 
\textit{J.~Nephrol.} 13(Suppl.~3):45--50.

\bibitem{4-su-1} %2
\Aue{Trinchieri, A., F.~Coppi, E.~Montanari, A.~Del Nero, G.~Zanetti, and E.~Pisani}. 2000. 
Increase in the prevalence of symptomatic upper urinary tract stones during the last ten years. 
\textit{Eur. Urol.} 37:23--25.

\bibitem{2-su-1} %3
\Aue{Pearle, M.\,S., E.\,A.~Calhoun, and G.\,C.~Curhan}. 2005. Urologic diseases in America 
project: Urolithiasis. \textit{J.~Urology} 173:848--857. 
\bibitem{3-su-1} %4
\Aue{Lieske, J.\,C., L.\,S.~Pena de la Vega, J.\,M.~Slezak, E.\,J.~Bergstralh; C.\,L.~Leibson, 
K.\,L.~Ho, and M.\,T.~Gettman}. 2006. Renal stone epidemiology in Rochester, Minnesota: An 
update. \textit{Kidney Int.} 69(4):760--768.


\bibitem{6-su-1} %5
\Aue{Johnson, C.\,M., D.\,M.~Wilson, W.\,M.~O'Fallon, R.\,S.~Malek, and L.\,T.~Kurland}. 
1979. Renal stone epidemiology: A~25-year study in Rochester, Minnesota. \textit{Kidney 
Int.} 16(5):624--631.

\bibitem{5-su-1} %6
\Aue{Yasui, T, M.~Iguchi, S.~Suzuki, and K.~Kohri}. 2008. Prevalence and epidemiological 
characteristics of urolithiasis in Japan: National trends between 1965 and 2005. \textit{Urology}  
 71(2):209--213.

\bibitem{7-su-1} %7
Russian Ministry of Health:
Central Research Institute of Organization and Informatization of Population.
2004. Zabolevaemost' naseleniya Rossii v 2003 godu: Sta\-ti\-sti\-che\-skie materialy [Morbidity of 
population of Russia in 2003: Statistical materials]. Мoscow.  Electronic version.
{\sf http://www.minzdravsoc.ru/docs/mzsr/stat/17}.

\bibitem{8-su-1}
\Aue{Apolikhin,~O.\,I., A.\,V.~Sivkov, T.\,V.~Solntseva, and V.\,A.~Komarova.}
2012. Analysis of urological morbidity in the Russian Federation within the period of 2005--2010.
\textit{Experimental and Clinical Urology} 2:4--12.
{\sf http://ecuro.ru/en/article/analysis-urological-morbidity-russian-federation-within-period-2005-2010}.


\bibitem{11-su-1} %9
\Aue{Routh, J.\,C., D.\,A.~Graham, and C.\,P.~Nelson}. 2100. Epidemiological trends in 
pediatric urolithiasis at United States freestanding pediatric hospitals. \textit{J.~Urology} 
184(3):1100--1104.

%\bibitem{9-su-1}
%Russian Ministry of Health:
%Central Research Institute of Organization and Informatization of Population.
%2008. Zabolevaemost' naseleniya Rossii v 2007 godu: Sta\-ti\-sti\-che\-skie materialy [Morbidity of 
%population of Russia in 2007: Statistical materials]. Мoscow. Electronic version.
%{\sf http://www.minzdravsoc.ru/docs/mzsr/stat/27}.
\bibitem{10-su-1} %10
\Aue{Golovanov, S.\,A., and V.\,V.~Drozhzheva}.
2010. Crystal formation activity of urine in oxalate urolithiasis.
\textit{Experimental and Clinical Urology} 2:24--29.
{\sf http://ecuro.ru/en/article/crystal-formation-activity-urine-oxalate-urolithiasis}.



\bibitem{12-su-1} %12
\Aue{Bonny, O., A.~Rubin, Ch.-L.~Huang, W.\,H.~Frawley, C.\,Y.\,C.~Pak, and 
O.\,W.~Moe}. 2008. Mechanism of urinary calcium regulation by urinary magnesium and $pH$. 
\textit{J.~Am. Soc. Nephrol.} 19(8):1530--1537.
\bibitem{13-su-1}
\Aue{Djuk, V.\,A., and V.\,L.~Jemanujel'}. 2003. \textit{Informatsionnye tekhnologii v 
mediko-biologicheskikh issledovaniyakh} [\textit{Information technologies in medical and 
biological researches}]. St.\ Petersburg, Russia: Piter, 2003. 525~p.
\bibitem{14-su-1}
\Aue{Liew, P.\,L., Y.\,C.~Lee, Y.\,C.~Lin, \textit{et al}.} 2007. Comparison of artificial neural 
networks with logistic regression\linebreak\vspace*{-12pt}

\pagebreak

\noindent
 in prediction of gallbladder disease among obese patients. 
\textit{Digest. Liver Dis.} 39(4):356--362.

%\pagebreak


\bibitem{15-su-1}
\Aue{Bassi, P., E. Sacco, V.~De Marco, \textit{et al}.} 2007. Prognostic accuracy of an 
artificial neural network in patients undergoing radical cystectomy for bladder cancer: 
A~comparison with logistic regression analysis. \textit{BJU Int.}  99(5):1007--1012.
\bibitem{16-su-1}
\Aue{Stephan, C., C.~Xu, P.~Finne, \textit{et al}.} 2007. Comparison of two different artificial 
neural networks for prostate biopsy indication in two different patient populations. 
\textit{J.~Urology}  70(3):596--601.

%\columnbreak

\bibitem{19-su-1} %18
\Aue{Stone, M.} 1974. Cross-validatory choice and assessment of statistical predictions (with 
discussion). \textit{J.~Roy. Stat. Soc. B} 36:111--147.
%\bibitem{17-su-1}
%\Aue{Chun, F.\,K., P.\,I.~Karakiewicz, A.~Briganti, \textit{et al}.} 2007. A~critical appraisal 
%of logistic regression-based nomograms, artificial neural networks, classification and 
%regression-tree models, look-up tables and risk-group stratification models for prostate cancer. 
%\textit{BJU Intern}.  99(4):794--800.



\bibitem{18-su-1} %19
\Aue{Efron, B.} 1979. Bootstrap methods: Another look at the jackknife. \textit{Ann.  
Stat.}  7:1--26.

\bibitem{21-su-1} %17
\Aue{Izenman, A.\,J.} 2008. \textit{Modern multivariate statistical techniques}. Springer. 
731~p.


 
 
 
% \bibitem{20-su-1} %20
%\Aue{Breiman, L.} 1994. The 1991 census adjustment: Undercount or bad data.  
%\textit{Stat. Sci.}  9:458--75.
 

\end{thebibliography}
} }

\end{multicols}

\hfill{\small\textit{Received April 17, 2013}}

\Contr

\noindent
\textbf{Krivenko Michail P.} (b.\ 1946)~--- 
Doctor of Science in technology, principal scientist, Institute of Informatics 
Problems, Russian Academy of Sciences, Moscow 119333, Russian Federation;  mkrivenko@ipiran.ru

\vspace*{3pt}


\noindent\textbf{Golovanov Sergey  A.} (b.\ 1950)~--- Doctor of Science in medicine, Head of 
Laboratory, Research Institute of Urology, Moscow 105425, Russian Federation;
sergeygol124@mail.ru
 

\vspace*{3pt}

\noindent
\textbf{Savchenko Pavel A.} (b.\ 1967)~--- software engineer, Institute of Informatics 
Problems, Russian Academy of Sciences, Moscow 119333, Russian Federation;  
psavchenko@ipiran.ru

\vspace*{3pt}

\noindent
\textbf{Sivkov Andrey V.} (b.\ 1957)~--- Doctor of Science in medicine, Deputy director, 
Research Institute of Urology, Moscow 105425, Russian Federation;  uroinfo@yandex.ru

\vspace*{3pt}

\noindent
\textbf{Suchkov Alexander P.} (b. 1954)~--- Doctor of Science in technology, principal 
scientist, Institute of Informatics Problems, Russian Academy of Sciences, Moscow 119333, Russian Federation;  
asuchkov@ipiran.ru 

\label{end\stat}

\renewcommand{\bibname}{\protect\rm Литература} %15






%%%%%%%%%%%%%%%%%%%%%%%%%%%%%%%%%%%%%%%%

%\def\stat{rez}
{%\hrule\par
%\vskip 7pt % 7pt
\raggedleft\Large \bf%\baselineskip=3.2ex
Р\,Е\,Ц\,Е\,Н\,З\,И\,И \vskip 17pt
    \hrule
    \par
\vskip 6pt plus 6pt minus 3pt }

%\thispagestyle{headings} %с верхним колонтитулом
%\thispagestyle{myheadings} %с нижним колонтитулом, но в верхнем РЕЦЕНЗИИ

\def\tit{НОВАЯ КНИГА И.\,Н.~СИНИЦЫНА, А.\,С.~ШАЛАМОВА <<ЛЕКЦИИ ПО ТЕОРИИ 
ИНТЕГРИРОВАННОЙ ЛОГИСТИЧЕСКОЙ ПОДДЕРЖКИ>> (М.: ТОРУС ПРЕСС, 2012. 624~с.)}

%1
\def\aut{Д.ф.-м.н., профессор С.\,Я.~Шоргин}

\def\auf{\ }

\def\leftkol{\ % РЕЦЕНЗИИ
}

\def\rightkol{ \ } 

%\def\leftkol{\ } % ENGLISH ABSTRACTS}

%\def\rightkol{\ } %ENGLISH ABSTRACTS}

%\def\leftkol{РЕЦЕНЗИИ}

%\def\rightkol{РЕЦЕНЗИИ}

\titele{\tit}{\aut}{\auf}{\leftkol}{\rightkol}
\vspace*{-18pt}


     \label{st\stat}

     \begin{multicols}{2}
     {\small
     {\baselineskip=10.1pt
     

      В книге представлено системное изложение теоретических основ одного из новейших 
направлений в \mbox{об\-ласти} экономики послепродажного обслуживания изделий наукоемкой 
продукции (ИНП) длительного пользования~--- интегрированной логистической поддержки
(ИЛП). 
{\looseness=1

}

Приведены также результаты новых работ, выполненных в Институте проблем информатики 
Российской академии наук в рамках научного направления <<Информационные технологии и 
анализ сложных сис\-тем>>.
 {%\looseness=1

}
     
      Излагаемые в книге научные подходы позво\-ляют карди\-наль\-но реформировать 
существующие системы производства и эксплуатации ИНП путем создания и внед\-ре\-ния 
методов рационального и оптимального управ\-ле\-ния процессами расходования 
вре\-мен\-н$\acute{\mbox{ы}}$х, 
мате\-ри\-аль\-ных, трудовых и других ресурсов на всех стадиях жизненного цикла изделий (ЖЦИ) по 
критериям экономической целесообразности и эф\-фек\-тив\-ности.
  {\looseness=1

}
    
      В книге приведен краткий обзор причин возник\-новения и
      развития CALS-методологии как основы 
современных международных стандартов по созданию и функционированию глобальных 
ин\-фор\-ма\-ци\-он\-но-ком\-му\-ни\-ка\-ци\-он\-ных систем, ее ключевых возможностей и эффективности 
результатов ее использования. 
Авторы %\linebreak 
предлагают ряд научных обоснований для разработки 
единой теории проектирования и управления систем ИЛП для полноценного использования 
преимуществ %\linebreak
 суще\-ст\-ву\-ющей методологии, определяют \mbox{общую} структурную схему 
комплексной системы <<ИНП-СППО>> и необходимость разработки для ее описания 
гибридных стохастических моделей.
{%\looseness=1

}

%\columnbreak
      
      Книга состоит из пяти частей, где последовательно излагается материал по каждой из 
следующих тем: <<Интегрированная логистическая поддержка>>, <<Теория гибридных 
стохастических систем и компьютерная поддержка исследований и разработок>>, <<Основы 
математического моделирования, анализа и синтеза систем послепродажного обслуживания>>, 
<<Определение и анализ показателей экспортного потенциала ИНП при проектировании>>, 
<<Задачи управления поддержкой послепродажного обслуживания>>, а также 
<<Моделирование инвестиционных процессов ИЛП в условиях неравновесных финансовых 
рынков>>. 
   
      В конце каждой главы приведены выводы и даны вопросы и задания для 
самоконтроля. В~приложениях содержатся основные определения по программам работ по 
анализу ИЛП, логистическим базам данных и компьютерным решениям, эквивалентной статистической 
линеаризации нелинейных преобразований ИЛП, справочный материал, а также развернутые 
уравнения для вероятностных характеристик.


      \def\leftkol{РЕЦЕНЗИИ}

\def\rightkol{РЕЦЕНЗИИ} 

      
      Книга заинтересует широкий круг специалистов и может быть использована научными 
проектными организациями в сфере промышленного производства ИНП. Большое количество 
иллюстраций, примеров и вопросов, обращенных к читателю, позволяет использовать книгу 
также в качестве учебного пособия для студентов и аспирантов машиностроительных, 
транспортных и~других специальностей, а также для самостоятельного изучения. 
{%\looseness=-1

}

Книга 
представляет несомненный интерес для специалистов и студентов в области прикладной 
математики и информатики.
    

}

}
\end{multicols}

%\newpage

%\def\stat{popravka}



\def\tit{ПОПРАВКА К СТАТЬЕ О.\,В.~ШЕСТАКОВА 
<<ПОРОГОВЫЕ ФУНКЦИИ В~МЕТОДАХ ПОДАВЛЕНИЯ ШУМА, ОСНОВАННЫХ~НА~ВЕЙВЛЕТ-РАЗЛОЖЕНИИ СИГНАЛА>>\\
(Информатика и её применения, 2021. Т.\ 15.  Вып.\,3. C.\ 51--56)}



\def\titkol{Поправка к статье О.\,В.~Шестакова\\
<<Пороговые функции в~методах подавления шума, основанных
на~вейвлет-разложении сигнала>>}



  \def\aut{\ }

  \def\autkol{\ } 

\titel{\tit}{\aut}{\autkol}{\titkol}

\def\leftfootline{\small{\textbf{\thepage}
\hfill INFORMATIKA I EE PRIMENENIYA~--- INFORMATICS AND
APPLICATIONS\ \ \ 2021\ \ \ volume~15\ \ \ issue\ 4}
}%
 \def\rightfootline{\small{INFORMATIKA I EE PRIMENENIYA~---
INFORMATICS AND APPLICATIONS\ \ \ 2021\ \ \ volume~15\ \ \ issue\ 4
\hfill \textbf{\thepage}}}


 \label{st\stat}

 \thispagestyle{headings}
 
 \vspace*{-24pt}  

\noindent
{\textbf{DOI:} 10.14357/19922264210307}

\vspace*{20pt}

\def\leftfootline{\small{\textbf{\thepage}
\hfill INFORMATIKA I EE PRIMENENIYA~--- INFORMATICS AND
APPLICATIONS\ \ \ 2021\ \ \ volume~15\ \ \ issue\ 4}
}%
 \def\rightfootline{\small{INFORMATIKA I EE PRIMENENIYA~---
INFORMATICS AND APPLICATIONS\ \ \ 2021\ \ \ volume~15\ \ \ issue\ 4
\hfill \textbf{\thepage}}}


%%%%%%%%%

\medskip

\noindent
С.~55, вместо 

\bigskip

\noindent
{\large ANALYSIS OF THE UNBIASED MEAN-SQUARE RISK ESTIMATE\\[6pt]
 OF~THE~BLOCK THRESHOLDING METHOD}

 



\bigskip

\noindent
должно быть

\bigskip

\noindent
{\large THRESHOLDING FUNCTIONS IN~THE~NOISE SUPPRESSION METHODS\\[6pt] 
BASED ON~THE~WAVELET EXPANSION OF~THE~SIGNAL}

 



 
\vskip 10pt plus 9pt minus 6pt

 \thispagestyle{headings}
 
 %\vspace*{-22pt}
  

\label{end\stat}

\renewcommand{\bibname}{\protect\rm Литература} 


\vspace*{8pt}

\hrule

\vspace*{2pt}

\hrule 

\vspace*{12pt}


\def\stat{popravka-1}



\def\tit{ПОПРАВКА К СТАТЬЕ А.\,А.~КУДРЯВЦЕВА, О.\,В.~ШЕСТАКОВА, С.\,Я.~ШОРГИНА
<<МЕТОД ОЦЕНИВАНИЯ ПАРАМЕТРОВ ИЗГИБА, ФОРМЫ И~МАСШТАБА
ГАММА-ЭКСПОНЕНЦИАЛЬНОГО РАСПРЕДЕЛЕНИЯ>>\\
(Информатика и её применения, 2021. Т.\ 15.  Вып.\,3. C.\ 57--62)}



\def\titkol{Поправка к статье А.\,А.~Кудрявцева, О.\,В.~Шестакова, С.\,Я.~Шоргина
<<Метод оценивания параметров изгиба, формы и масштаба
гамма-экспоненциального распределения>>}



  \def\aut{\ }

  \def\autkol{\ } 

\titel{\tit}{\aut}{\autkol}{\titkol}


 \label{st\stat}

 \thispagestyle{headings}
 
 \vspace*{-24pt}  

\noindent
{\textbf{DOI:} 10.14357/19922264210308}

\vspace*{20pt}




%%%%%%%%%

\medskip

\noindent
С.~61, вместо 

\bigskip

\noindent
{\large PROBABILISTIC CHARACTERISTICS OF~BALANCE INDEX
OF~FACTORS\\[6pt] 
WITH~GENERALIZED GAMMA DISTRIBUTION}



 



\bigskip

\noindent
должно быть

\bigskip

\noindent
{\large A METHOD FOR~ESTIMATING BENT, SHAPE AND~SCALE PARAMETERS\\[6pt] 
OF~THE~GAMMA-EXPONENTIAL DISTRIBUTION} 



 



 
\vskip 10pt plus 9pt minus 6pt

 \thispagestyle{headings}
 
 %\vspace*{-22pt}
  

\label{end\stat}

\renewcommand{\bibname}{\protect\rm Литература}  
%\include{popravka-1}

\def\stat{authorsrus}
{%\hrule\par
%\vskip 7pt % 7pt
\raggedleft\Large \bf%\baselineskip=3.2ex
О\,Б\ \ А\,В\,Т\,О\,Р\,А\,Х \vskip 17pt
    \hrule
    \par
\vskip 21pt plus 8pt minus 4pt }


\def\tit{\ }

\def\aut{\ }

\def\auf{\ }

\def\leftkol{\ } % ENGLISH ABSTRACTS}

\def\rightkol{ОБ АВТОРАХ} %ENGLISH ABSTRACTS}

\titele{\tit}{\aut}{\auf}{\leftkol}{\rightkol}
      
            \label{st\stat}



\vspace*{24pt}

\begin{multicols}{2}




\noindent
\textbf{Архипов Олег Петрович} (р.\ 1948)~---
кандидат технических наук, директор Орловского филиала Института проб\-лем информатики
Российской академии наук
%302025, г.Орел, Московское шоссе, д.137

\vspace*{3pt}

\noindent
\textbf{Бирюкова Татьяна Константиновна} (р.\ 1968)~---
кандидат фи\-зи\-ко-ма\-те\-ма\-ти\-че\-ских наук, старший научный сотрудник Института проб\-лем информатики
Российской академии наук

\vspace*{3pt}

\noindent 
\textbf{Бобков  Сергей Геннадьевич} (р.\ 1955)~---
доктор технических наук,  заведующий отделением На\-уч\-но-ис\-сле\-до\-ва\-тель\-ско\-го 
института системных исследований Российской академии наук
%117218, Москва, Нахимовский просп., 36, к.1 

\vspace*{3pt}

\noindent \textbf{Васильев Николай Семенович} (р.\ 1952)~--- доктор 
фи\-зи\-ко-ма\-те\-ма\-ти\-че\-ских наук, профессор, 
МГТУ им.\ Н.\,Э.~Баумана 
%, Москва 105005, 2-я Бауманская ул., д.~5,

\vspace*{3pt}

\noindent
\textbf{Гершкович Максим Михайлович} (р.\ 1968)~---
старший научный сотрудник Института проб\-лем информатики
Российской академии наук

\vspace*{3pt}

\noindent 
\textbf{Дьяченко Юрий Георгиевич} (р.\ 1958)~--- кандидат технических наук, 
старший научный сотрудник Института проб\-лем информатики
Российской академии наук

\vspace*{3pt}

\noindent 
\textbf{Ерошенко Александр Андреевич} (р.\ 1989)~--- аспирант кафедры 
математической статистики факультета вычисли\-тельной математики и кибернетики 
Московского государственного университета им.\ М.\,В.~Ломоносова
%119991, Москва ГСП-1, Ленинские горы, д.\ 1, стр. 52

\vspace*{3pt}
 
\noindent 
\textbf{Захаров Виктор Николаевич} (р.\ 1948)~--- 
доктор технических наук, доцент, ученый секретарь Института проб\-лем информатики
Российской академии наук

\vspace*{3pt}

\noindent
\textbf{Зейфман Александр Израилевич} (р.\ 1954)~---
доктор фи\-зи\-ко-ма\-те\-ма\-ти\-че\-ских наук, профессор, 
заведующий кафедрой Вологодского государственного университета; 
старший научный сотрудник Института проб\-лем информатики
Российской академии наук; главный научный сотрудник ИСЭРТ Российской академии наук

\vspace*{3pt}

\noindent
\textbf{Зыкин Сергей Владимирович} (р.\ 1959)~--- 
доктор технических наук, профессор, заведующий лабораторией Института математики 
им.\ С.\,Л.~Соболева Сибирского отделения Российской академии наук, Новосибирск 
%630090, пр.\ ак.\ Коптюга, 4 

\vspace*{4pt}

\noindent
\textbf{Киреев Владимир Иванович} (р.\ 1938)~---
доктор фи\-зи\-ко-ма\-те\-ма\-ти\-че\-ских наук, профессор Московского 
государственного горного университета
%Адрес: Россия, 119991, г. Москва, Ленинский проспект, д. 6

%\columnbreak

\vspace*{4pt}

\noindent
\textbf{Козеренко Елена Борисовна} (р.\ 1959)~---
кандидат филологических наук, заведующая лабораторией Института проб\-лем информатики
Российской академии наук

\vspace*{4pt}

\noindent
\textbf{Королев Виктор Юрьевич} (р.\ 1954)~--- доктор
фи\-зи\-ко-ма\-те\-ма\-ти\-че\-ских наук, профессор кафедры математической 
статистики факультета вычисли\-тельной математики и кибернетики 
Московского государственного университета; 
ведущий научный сотрудник Института проб\-лем информатики
Российской академии наук

\vspace*{4pt}

\noindent
\textbf{Коротышева Анна Владимировна} (р.\ 1988)~---
старший преподаватель Вологодского государственного университета

\vspace*{4pt}

\noindent 
\textbf{Кун Де Турк} (р.\ 1981)~--- научный сотрудник 
исследовательской группы SMACS факультета телекоммуникаций и обработки информации
Университета Гента, Бельгия
%В-9000 Гент, Бельгия

\vspace*{4pt}

\noindent
\textbf{Лупенцов Олег Сергеевич} (р.\ 1986)~---
аспирант Омского государственного института сервиса
%Омск 644043, ул.\ Певцова 13

\vspace*{4pt}

\noindent
\textbf{Лучко Олег Николаевич} (р.\ 1961)~---
кандидат педагогических наук, профессор, заведующий кафедрой 
Омского государственного института сервиса
%Омск 644043, ул.\ Певцова 13

\vspace*{4pt}

\noindent
\textbf{Малашенко Юрий Евгеньевич} (р.\ 1946)~---
доктор фи\-зи\-ко-ма\-те\-ма\-ти\-че\-ских наук, заведующий сектором 
Вычислительного центра им.\ А.\,А.~Дородницына Российской академии наук
%Адрес: 119333, Москва, ул. Вавилова, 40,

\vspace*{4pt}

\noindent
\textbf{Маньяков Юрий Анатольевич} (р.\ 1984)~---
кандидат технических наук, научный сотрудник Орловского филиала Института проб\-лем информатики
Российской академии наук
%302025, г.Орел, Московское шоссе, д.137

\vspace*{4pt}

\noindent
\textbf{Маренко Валентина Афанасьевна} (р.\ 1951)~---
кандидат технических наук, доцент, старший научный сотрудник 
Института математики им.\ С.\,Л.~Соболева Сибирского отделения Российской академии наук
%Новосибирск 630090, пр. ак. Коптюга, 4 

\vspace*{3pt}

\noindent 
\textbf{Морозов Евсей Викторович} (р.\ 1947)~--- доктор 
фи\-зи\-ко-ма\-те\-ма\-ти\-че\-ских, профессор, ведущий научный сотрудник 
Института прикладных математических исследований Карельского научного центра Российской
академии наук; 
%%185910 Россия, Республика Карелия, г.\ Петрозаводск, ул.\ Пушкинская, 11
профессор Петрозаводского государственного университета, Петрозаводск
%185910 Россия, Республика Карелия, г.\ Петрозаводск, пр.\ Ленина, 33

%\pagebreak

\vspace*{3pt}

\noindent
\textbf{Назарова Ирина Александровна} (р.\ 1966)~---
кандидат фи\-зи\-ко-ма\-те\-ма\-ти\-че\-ских наук, 
научный сотрудник Вычислительного центра им.\ А.\,А.~Дородницына Российской академии наук 
%Адрес: 119333, Москва, ул. Вавилова, 40

\vspace*{3pt}

\noindent
\textbf{Павлов Игорь Валерианович} (р.\ 1945)~--- 
доктор фи\-зи\-ко-ма\-те\-ма\-ти\-че\-ских наук, профессор МГТУ им.\ Н.\,Э.~Баумана 
%Москва 105005, 2-я Бауманская ул., д.~5 

%\pagebreak

\vspace*{3pt}

\noindent 
\textbf{Потахина Любовь Викторовна} (р.\ 1989)~--- аспирантка
Института прикладных математических исследований Карельского научного центра
Российской академии наук; 
%%185910 Россия, Республика Карелия, г.\ Петрозаводск, ул.\ Пушкинская, 11
инженер Петрозаводского государственного университета, Петрозаводск
%185910 Россия, Республика Карелия, г.\ Петрозаводск, пр.\ Ленина, 33

\vspace*{3pt}

\noindent 
\textbf{Рождественский Юрий Владимирович} (р.\ 1952)~--- 
кандидат технических наук, заведующий сектором Института проб\-лем информатики
Российской академии наук

\vspace*{3pt}

\noindent 
\textbf{Синицын Игорь Николаевич} (р.\ 1940)~--- доктор технических наук,
профессор, заслуженный деятель\linebreak\vspace*{-12pt}

\columnbreak

\noindent
 науки РФ, заведующий отделом Института проб\-лем информатики
Российской академии наук

\vspace*{7pt}


\noindent
\textbf{Сиротинин Денис Олегович} (р.\ 1984)~---
кандидат технических наук, научный сотрудник Орловского филиала Института проб\-лем информатики
Российской академии наук
%302025, г.Орел, Московское шоссе, д.137

\vspace*{7pt}

%\columnbreak

\noindent 
\textbf{Соколов  Игорь Анатольевич} (р.\ 1954)~--- академик (действительный член) Российской 
академии наук, доктор технических наук, директор Института проб\-лем информатики
Российской академии наук

\vspace*{7pt}

\noindent
\textbf{Степченков Юрий Афанасьевич} (р.\ 1951)~---
кандидат технических наук, заведующий отделом Института проб\-лем информатики
Российской академии наук

\vspace*{7pt}

\noindent
\textbf{Сурков Алексей Викторович} (р.\ 1978)~--- 
старший научный сотрудник На\-уч\-но-ис\-сле\-до\-ва\-тель\-ско\-го 
института системных исследований Российской академии наук
%117218, Москва, Нахимовский просп., 36, к.1 

\vspace*{7pt}

\noindent 
\textbf{Шестаков Олег Владимирович} (р.\ 1976)~--- доктор 
фи\-зи\-ко-ма\-те\-ма\-ти\-че\-ских, доцент кафедры математической статистики 
факультета вычисли\-тельной математики и кибернетики Московского 
государственного университета им.\ М.\,В.~Ломоносова; 
%119991, Москва ГСП-1, Ленинские горы, д.\ 1, стр. 52
старший научный сотрудник Института проб\-лем информатики
Российской академии наук
%, Москва 119333, ул. Вавилова, д.~44, корп.~2

\vspace*{7pt}

\noindent 
\textbf{Шоргин Сергей Яковлевич} (р.\ 1952.)~--- доктор
фи\-зи\-ко-ма\-те\-ма\-ти\-че\-ских наук, профессор, заместитель директора Института 
проб\-лем информатики Российской академии наук





%%%%%%%%%%%%%%%%%%%%%%%%%%%%%%%%%%%%%%%%%%%%%%%%%%%%%%%%%%%%%%%%%%%%%%%%%%%%%%%




%\def\rightkol{ОБ АВТОРАХ}
%\def\leftkol{ОБ АВТОРАХ}

 \label{end\stat}





%\def\leftfootline{\small{\textbf{\thepage}
%\hfill ИНФОРМАТИКА И ЕЁ ПРИМЕНЕНИЯ\ \ \ том~7\ \ \ выпуск~1\ \ \ 2013}
%}%
% \def\rightfootline{\small{ИНФОРМАТИКА И ЕЁ ПРИМЕНЕНИЯ\ \ \ том~7\ \ \ выпуск~1\ \ \ 2013
%\hfill \textbf{\thepage}}}


%\thispagestyle{myheadings}



\end{multicols}

\newpage  

%\def\stat{cont}
{%\hrule\par
%\vskip 7pt % 7pt
\raggedleft\Large \bf%\baselineskip=3.2ex
А\,В\,Т\,О\,Р\,С\,К\,И\,Й\ \ У\,К\,А\,З\,А\,Т\,Е\,Л\,Ь\ \ З\,А\ \ 2\,0\,0\,7 г. \vskip 17pt
    \hrule
    \par
\vskip 21pt plus 6pt minus 3pt }

\label{st\stat}

\def\tit{\ }

\def\aut{\ }
\def\auf{\ }

\def\leftkol{\ } % ENGLISH ABSTRACTS}

\def\rightkol{\ } %ENGLISH ABSTRACTS}

\titele{\tit}{\aut}{\auf}{\leftkol}{\rightkol}


\contentsline {chapter}{\ }{Выпуск \quad Стр.} 
\contentsline {section}{\textbf{Батракова Д.\,А., Королев В.\,Ю., Шоргин С.\,Я.}\ \ Новый метод вероятностно-ста\-ти\-сти\-че\-ско\-го анализа информационных потоков в\nobreakspace {}телекоммуникационных сетях}{\qquad 1 \qquad 40} 
\contentsline {section}{\textbf{Борисов А.\,В.}\ \ Байесовское оценивание в системах наблюдения с\nobreakspace {}марковскими скачкообразными процессами: игровой подход}{\qquad 2 \qquad 65}
\contentsline {section}{\textbf{Босов А.\,В., Иванов А.\,В.}\ \ Программная инфраструктура информационного Web-пор\-тала}{\qquad 2 \qquad 50}
\contentsline {section}{\textbf{Захаров В.\,Н., Калиниченко Л.\,А., Соколов И.\,А., Ступников С.\,А.}\ \ Конструирование канонических информационных моделей для интегрированных информационных систем}{\qquad 2 \qquad 15}
\contentsline {section}{\textbf{Захаров В.\,Н., Козмидиади В.\,А.}\ \ Средства обеспечения отказоустойчивости при\-ло\-жений}{\qquad 1 \qquad 14} 
\contentsline {section}{\textbf{Иванов А.\,В.}\ \ см. Босов А.\,В.\hfill\hfill\hfill\hfill\hfill\hfill\hfill\hfill\hfill\hfill\hfill\hfill\hfill\hfill\hfill\hfill\hfill\hfill\hfill\hfill\hfill\hfill\hfill\hfill\hfill\hfill\hfill\hfill\hfill\hfill\hfill\hfill\hfill\hfill\hfill}{\ }
\contentsline {section}{\textbf{Ильин В.\,Д., Соколов И.\,А.}\ \ Символьная модель системы знаний информатики в\nobreakspace {}че\-ло\-ве\-ко-автоматной среде}{\qquad 1 \qquad 66} 
\contentsline {section}{\textbf{Калиниченко Л.\,А.}\ \ см. Захаров В.\,Н.\hfill\hfill\hfill\hfill\hfill\hfill\hfill\hfill\hfill\hfill\hfill\hfill\hfill\hfill\hfill\hfill\hfill\hfill\hfill\hfill\hfill\hfill\hfill\hfill\hfill\hfill\hfill\hfill\hfill\hfill\hfill\hfill\hfill\hfill\hfill}{\ }
\contentsline {section}{\textbf{Козеренко Е.\,Б.}\ \ Лингвистическое моделирование для систем машинного перевода и обработки знаний}{\qquad 1 \qquad 54} 
\contentsline {section}{\textbf{Козмидиади В.\,А.}\ \ см. Захаров В.\,Н.\hfill\hfill\hfill\hfill\hfill\hfill\hfill\hfill\hfill\hfill\hfill\hfill\hfill\hfill\hfill\hfill\hfill\hfill\hfill\hfill\hfill\hfill\hfill\hfill\hfill\hfill\hfill\hfill\hfill\hfill\hfill\hfill\hfill\hfill\hfill }{\ } 
\contentsline {section}{\textbf{Королев В.\,Ю.}\ \ см. Батракова Д.\,А.\hfill\hfill\hfill\hfill\hfill\hfill\hfill\hfill\hfill\hfill\hfill\hfill\hfill\hfill\hfill\hfill\hfill\hfill\hfill\hfill\hfill\hfill\hfill\hfill\hfill\hfill\hfill\hfill\hfill\hfill\hfill\hfill\hfill\hfill\hfill}{\ } 
\contentsline {section}{\textbf{Кудрявцев А.\,А., Шоргин С.\,Я.}\ \ Байесовский подход к\nobreakspace {}анализу систем массового обслуживания и\nobreakspace {}показателей надежности}{\qquad 2 \qquad 76}
\contentsline {section}{\textbf{Печинкин А.\,В., Соколов И.\,А., Чаплыгин В.\,В.}\ \ Многолинейная система массового обслуживания с конечным накопителем и ненадежными приборами}{\qquad 1 \qquad 27} 
\contentsline {section}{\textbf{Печинкин А.\,В., Соколов И.\,А., Чаплыгин В.\,В.}\ \ Стационарные характеристики многолинейной\nobreakspace {}системы массового обслуживания с\nobreakspace {}одновременными отказами приборов}{\qquad 2 \qquad 39}
\contentsline {section}{\textbf{Синицын И.\,Н.}\ \ Корреляционные методы построения аналитических информационных моделей флуктуаций полюса Земли по априорным данным}{\qquad 2 \qquad \hphantom{9}2}
\contentsline {section}{\textbf{Синицын И.\,Н.}\ \ Развитие теории фильтров Пугачева для оперативной обработки информации в стохастических системах}{{\qquad 1 \qquad \hphantom{9}3}} 
\contentsline {section}{\textbf{Соколов И.\,А.}\ \ см. Захаров В.\,Н.\hfill\hfill\hfill\hfill\hfill\hfill\hfill\hfill\hfill\hfill\hfill\hfill\hfill\hfill\hfill\hfill\hfill\hfill\hfill\hfill\hfill\hfill\hfill\hfill\hfill\hfill\hfill\hfill\hfill\hfill\hfill\hfill\hfill\hfill\hfill}{\ }
\contentsline {section}{\textbf{Соколов И.\,А.}\ \ см. Ильин В.\,Д.\hfill\hfill\hfill\hfill\hfill\hfill\hfill\hfill\hfill\hfill\hfill\hfill\hfill\hfill\hfill\hfill\hfill\hfill\hfill\hfill\hfill\hfill\hfill\hfill\hfill\hfill\hfill\hfill\hfill\hfill\hfill\hfill\hfill\hfill\hfill}{\ } 
\contentsline {section}{\textbf{Соколов И.\,А.}\ \ см. Печинкин А.\,В.\hfill\hfill\hfill\hfill\hfill\hfill\hfill\hfill\hfill\hfill\hfill\hfill\hfill\hfill\hfill\hfill\hfill\hfill\hfill\hfill\hfill\hfill\hfill\hfill\hfill\hfill\hfill\hfill\hfill\hfill\hfill\hfill\hfill\hfill\hfill}{\ } 
\contentsline {section}{\textbf{Соколов И.\,А.}\ \ см. Печинкин А.\,В.\hfill\hfill\hfill\hfill\hfill\hfill\hfill\hfill\hfill\hfill\hfill\hfill\hfill\hfill\hfill\hfill\hfill\hfill\hfill\hfill\hfill\hfill\hfill\hfill\hfill\hfill\hfill\hfill\hfill\hfill\hfill\hfill\hfill\hfill\hfill}{\ }
\contentsline {section}{\textbf{Ступников С.\,А.}\ \ см. Захаров В.\,Н.\hfill\hfill\hfill\hfill\hfill\hfill\hfill\hfill\hfill\hfill\hfill\hfill\hfill\hfill\hfill\hfill\hfill\hfill\hfill\hfill\hfill\hfill\hfill\hfill\hfill\hfill\hfill\hfill\hfill\hfill\hfill\hfill\hfill\hfill\hfill}{\ }
\contentsline {section}{\textbf{Чаплыгин В.\,В.}\ \ см. Печинкин А.\,В.\hfill\hfill\hfill\hfill\hfill\hfill\hfill\hfill\hfill\hfill\hfill\hfill\hfill\hfill\hfill\hfill\hfill\hfill\hfill\hfill\hfill\hfill\hfill\hfill\hfill\hfill\hfill\hfill\hfill\hfill\hfill\hfill\hfill\hfill\hfill}{\ } 
\contentsline {section}{\textbf{Чаплыгин В.\,В.}\ \ см. Печинкин А.\,В.\hfill\hfill\hfill\hfill\hfill\hfill\hfill\hfill\hfill\hfill\hfill\hfill\hfill\hfill\hfill\hfill\hfill\hfill\hfill\hfill\hfill\hfill\hfill\hfill\hfill\hfill\hfill\hfill\hfill\hfill\hfill\hfill\hfill\hfill\hfill}{\ }
\contentsline {section}{\textbf{Шоргин С.\,Я.}\ \ см. Батракова Д.\,А.\hfill\hfill\hfill\hfill\hfill\hfill\hfill\hfill\hfill\hfill\hfill\hfill\hfill\hfill\hfill\hfill\hfill\hfill\hfill\hfill\hfill\hfill\hfill\hfill\hfill\hfill\hfill\hfill\hfill\hfill\hfill\hfill\hfill\hfill\hfill}{\ } 
\contentsline {section}{\textbf{Шоргин С.\,Я.}\ \ см. Кудрявцев А.\,А.\hfill\hfill\hfill\hfill\hfill\hfill\hfill\hfill\hfill\hfill\hfill\hfill\hfill\hfill\hfill\hfill\hfill\hfill\hfill\hfill\hfill\hfill\hfill\hfill\hfill\hfill\hfill\hfill\hfill\hfill\hfill\hfill\hfill\hfill\hfill}{\ }
%\thispagestyle{myheadings}
\def\leftfootline{\small{\textbf{\thepage}
\hfill ИНФОРМАТИКА И ЕЁ ПРИМЕНЕНИЯ\ \ \ том~1\ \ \ выпуск~2\ \ \ 2007}
}%
 \def\rightfootline{\small{ИНФОРМАТИКА И ЕЁ ПРИМЕНЕНИЯ\ \ \ том~1\ \ \ выпуск~2\ \ \ 2007
 \hfill \textbf{\thepage}}}
 \label{end\stat} 
                     
%\def\stat{cont-e}
{%\hrule\par
%\vskip 7pt % 7pt
\raggedleft\Large \bf%\baselineskip=3.2ex
2\,0\,0\,7\ \ A\,U\,T\,H\,O\,R\ \ I\,N\,D\,E\,X \vskip 17pt
    \hrule
    \par
\vskip 21pt plus 6pt minus 3pt }

\label{st\stat}

\def\tit{\ }

\def\aut{\ }
\def\auf{\ }

\def\leftkol{\ } % ENGLISH ABSTRACTS}

\def\rightkol{\ } %ENGLISH ABSTRACTS}

\titele{\tit}{\aut}{\auf}{\leftkol}{\rightkol}


\contentsline {chapter}{\ }{Issue \quad Page} 
\contentsline {subsection}{\textbf{Batrakova D.\,A., Korolev V.\,Yu., Shorgin S.\,Ya.}\ \ A New Method for the Probabilistic and Statistical Analysis of Information Flows in Telecommunication Networks}{\qquad 1 \qquad 40} 
\contentsline {subsection}{\textbf{Borisov A.\,V.}\ \ Bayesian Estimation in\nobreakspace {}Observation Systems with\nobreakspace {}Markov Jump Processes: Game-Theoretic Approach}{\qquad 2 \qquad 65} 
\contentsline {subsection}{\textbf{Bosov A.\,V., Ivanov A.\,V.}\ \ Linguistic Simulation for Machine Translation and Knowledge Management Systems}{\qquad 2 \qquad 50} 
\contentsline {subsection}{\textbf{Chaplygin V.\,V.} see Pechinkin A.\,V.\hfill\hfill\hfill\hfill\hfill\hfill\hfill\hfill\hfill\hfill\hfill\hfill\hfill\hfill\hfill\hfill\hfill\hfill\hfill\hfill\hfill\hfill\hfill\hfill\hfill\hfill\hfill\hfill\hfill\hfill\hfill\hfill\hfill\hfill\hfill}{\ }
\contentsline {subsection}{\textbf{Chaplygin V.\,V.} see Pechinkin A.\,V.\hfill\hfill\hfill\hfill\hfill\hfill\hfill\hfill\hfill\hfill\hfill\hfill\hfill\hfill\hfill\hfill\hfill\hfill\hfill\hfill\hfill\hfill\hfill\hfill\hfill\hfill\hfill\hfill\hfill\hfill\hfill\hfill\hfill\hfill\hfill}{\ }
\contentsline {subsection}{\textbf{Ilyin V.\,D., Sokolov I.\,A.}\ \ The Symbol Model of Informatics Knowledge System in Human-Automaton Environment}{\qquad 1 \qquad 66} 
\contentsline {subsection}{\textbf{Ivanov A.\,V.} see Bosov A.\,V.\hfill\hfill\hfill\hfill\hfill\hfill\hfill\hfill\hfill\hfill\hfill\hfill\hfill\hfill\hfill\hfill\hfill\hfill\hfill\hfill\hfill\hfill\hfill\hfill\hfill\hfill\hfill\hfill\hfill\hfill\hfill\hfill\hfill\hfill\hfill}{\ }
\contentsline {subsection}{\textbf{Kalinichenko L.\,A.} see Zakharov V.\,N.\hfill\hfill\hfill\hfill\hfill\hfill\hfill\hfill\hfill\hfill\hfill\hfill\hfill\hfill\hfill\hfill\hfill\hfill\hfill\hfill\hfill\hfill\hfill\hfill\hfill\hfill\hfill\hfill\hfill\hfill\hfill\hfill\hfill\hfill\hfill}{\ }
\contentsline {subsection}{\textbf{Korolev V.\,Yu.} see Batrakova D.\,A.\hfill\hfill\hfill\hfill\hfill\hfill\hfill\hfill\hfill\hfill\hfill\hfill\hfill\hfill\hfill\hfill\hfill\hfill\hfill\hfill\hfill\hfill\hfill\hfill\hfill\hfill\hfill\hfill\hfill\hfill\hfill\hfill\hfill\hfill\hfill}{\ }
\contentsline {subsection}{\textbf{Kozerenko E.\,B.}\ \ Linguistic Simulation for Machine Translation and Knowledge Management Systems}{\qquad 1 \qquad 54} 
\contentsline {subsection}{\textbf{Kozmidiady V.\,A.} see Zakharov V.\,N.\hfill\hfill\hfill\hfill\hfill\hfill\hfill\hfill\hfill\hfill\hfill\hfill\hfill\hfill\hfill\hfill\hfill\hfill\hfill\hfill\hfill\hfill\hfill\hfill\hfill\hfill\hfill\hfill\hfill\hfill\hfill\hfill\hfill\hfill\hfill}{\ }
\contentsline {subsection}{\textbf{Kudryavtsev A.\,A., Shorgin S.\,Ya.}\ \ Bayesian Approach to Queueing Systems and Reliability Characteristics}{\qquad 2 \qquad 76} 
\contentsline {subsection}{\textbf{Pechinkin A.\,V., Sokolov I.\,A., Chaplygin V.\,V.}\ \ Multichannel Queuing System with Finite Buffer and Unreliable Servers}{\qquad 1 \qquad 27} 
\contentsline {subsection}{\textbf{Pechinkin A.\,V., Sokolov I.\,A., Chaplygin V.\,V.}\ \ Stationary Characteristics of a Multichannel Queueing System with\nobreakspace {}Simultaneous Refusals of Servers}{\qquad 2 \qquad 39} 
\contentsline {subsection}{\textbf{Shorgin S.\,Ya.} see Batrakova D.\,A.\hfill\hfill\hfill\hfill\hfill\hfill\hfill\hfill\hfill\hfill\hfill\hfill\hfill\hfill\hfill\hfill\hfill\hfill\hfill\hfill\hfill\hfill\hfill\hfill\hfill\hfill\hfill\hfill\hfill\hfill\hfill\hfill\hfill\hfill\hfill}{\ }
\contentsline {subsection}{\textbf{Shorgin S.\,Ya.} see Kudryavtsev A.\,A.\hfill\hfill\hfill\hfill\hfill\hfill\hfill\hfill\hfill\hfill\hfill\hfill\hfill\hfill\hfill\hfill\hfill\hfill\hfill\hfill\hfill\hfill\hfill\hfill\hfill\hfill\hfill\hfill\hfill\hfill\hfill\hfill\hfill\hfill\hfill}{\ }
\contentsline {subsection}{\textbf{Sinitsyn I.\,N.}\ \ Correlational Methods for Analytical Informational Models of the Earth Pole Fluctuations Design Based on a priori Data}{\qquad 2 \qquad \hphantom{9}2}
\contentsline {subsection}{\textbf{Sinitsyn I.\,N.}\ \ Development of Pugachev Filtering for Stochastic Systems}{\qquad 1 \qquad \hphantom{9}3}
\contentsline {subsection}{\textbf{Sokolov I.\,A.} see Ilyin V.\,D.\hfill\hfill\hfill\hfill\hfill\hfill\hfill\hfill\hfill\hfill\hfill\hfill\hfill\hfill\hfill\hfill\hfill\hfill\hfill\hfill\hfill\hfill\hfill\hfill\hfill\hfill\hfill\hfill\hfill\hfill\hfill\hfill\hfill\hfill\hfill}{\ }
\contentsline {subsection}{\textbf{Sokolov I.\,A.} see Pechinkin A.\,V.\hfill\hfill\hfill\hfill\hfill\hfill\hfill\hfill\hfill\hfill\hfill\hfill\hfill\hfill\hfill\hfill\hfill\hfill\hfill\hfill\hfill\hfill\hfill\hfill\hfill\hfill\hfill\hfill\hfill\hfill\hfill\hfill\hfill\hfill\hfill}{\ }
\contentsline {subsection}{\textbf{Sokolov I.\,A.} see Pechinkin A.\,V.\hfill\hfill\hfill\hfill\hfill\hfill\hfill\hfill\hfill\hfill\hfill\hfill\hfill\hfill\hfill\hfill\hfill\hfill\hfill\hfill\hfill\hfill\hfill\hfill\hfill\hfill\hfill\hfill\hfill\hfill\hfill\hfill\hfill\hfill\hfill}{\ }
\contentsline {subsection}{\textbf{Sokolov I.\,A.} see Zakharov V.\,N.\hfill\hfill\hfill\hfill\hfill\hfill\hfill\hfill\hfill\hfill\hfill\hfill\hfill\hfill\hfill\hfill\hfill\hfill\hfill\hfill\hfill\hfill\hfill\hfill\hfill\hfill\hfill\hfill\hfill\hfill\hfill\hfill\hfill\hfill\hfill}{\ }
\contentsline {subsection}{\textbf{Stupnikov S.\,A.} see Zakharov V.\,N.\hfill\hfill\hfill\hfill\hfill\hfill\hfill\hfill\hfill\hfill\hfill\hfill\hfill\hfill\hfill\hfill\hfill\hfill\hfill\hfill\hfill\hfill\hfill\hfill\hfill\hfill\hfill\hfill\hfill\hfill\hfill\hfill\hfill\hfill\hfill}{\ }
\contentsline {subsection}{\textbf{Zakharov V.\,N., Kalinichenko L.\,A., Sokolov I.\,A., Stupnikov S.\,A.}\ \ Development of Canonical Information Models for Integrated Information Systems}{\qquad 2 \qquad 15} 
\contentsline {subsection}{\textbf{Zakharov V.\,N., Kozmidiady V.\,A.}\ \ Means Providing Applications Fault Tolerance}{\qquad 1 \qquad 14} 
\def\leftfootline{\small{\textbf{\thepage}
\hfill ИНФОРМАТИКА И ЕЁ ПРИМЕНЕНИЯ\ \ \ том~1\ \ \ выпуск~2\ \ \ 2007}
}%
 \def\rightfootline{\small{ИНФОРМАТИКА И ЕЁ ПРИМЕНЕНИЯ\ \ \ том~1\ \ \ выпуск~2\ \ \ 2007
 \hfill \textbf{\thepage}}}
 \label{end\stat} 


%\end{document}

%
\def\stat{rekl}
%\label{preobr}

%\def\tit{АКАДЕМИК ПУГАЧЁВ  ВЛАДИМИР СЕМЁНОВИЧ\\
%25.03.1911--25.03.1998}


%   \vspace*{-48pt}
%   \begin{center}\LARGE
%Академик Пугачёв  Владимир Семёнович\\ (25.03.1911--25.03.1998)
%   \end{center}

   %\vspace*{2.5mm}

   \begin{center}

{\prgsh\LARGE
ЮБИЛЕИ}

\end{center}
%\hrule

\vspace*{6pt}


   \vspace*{8mm}

   \thispagestyle{empty}


%\def\stat{emel}


\section*{К 70-летию заместителя директора ИПИ РАН,\\ члена редколлегии журнала
<<Информатика и её применения>>\\ доктора технических наук В.\,И.~Будзко}

\vspace*{18pt}




          \begin{multicols}{2}

%            \label{st\stat}

\begin{center}
\vspace*{1pt}
\mbox{%
\epsfxsize=78mm
\epsfbox{bud-1.eps}
}
\end{center}

\vspace*{12pt}

      14 августа 2014~г.\ исполнилось 70~лет за\-мес\-ти\-те\-лю директора ИПИ РАН по
научной работе доктору технических наук Владимиру Игоревичу Будзко.

      Владимир Игоревич Будзко родился в г.~Москве. Высшее образование получил на факультете
элект\-рон\-но-вы\-чис\-ли\-тель\-ных устройств в Московском
ин\-же\-нер\-но-фи\-зи\-че\-ском институте
(МИФИ), который он окончил в 1968~г., после чего был на\-прав\-лен для прохождения
службы в одну из войс\-ко\-вых частей, где прошел путь от инженера до первого заместителя
командира войсковой части.

      С приходом В.\,И.~Будзко в ИПИ РАН (2001~г.)\ в институте
сформировалось новое научное на\-прав\-ле\-ние теоретических исследований~--- <<Постро\-ение
ин\-фор\-ма\-ци\-он\-но-те\-ле\-ком\-му\-ни\-ка\-ци\-он\-ных\linebreak сис\-тем
высокой до\-ступ\-ности>>. В~рамках этого
направления выполнен широкий круг фундаментальных исследований по поиску подходов и
определению принципов построения средств обеспечения доступности, конфиденциальности
и целостности современных крупномасштабных
ин\-фор\-ма\-ци\-он\-но-те\-ле\-ком\-му\-ни\-ка\-ци\-он\-ных
сис\-тем (ИТС). Разработаны основные сис\-тем\-но-тех\-ни\-че\-ские принципы и базовые
архитектурные решения построения перспективных для условий России ИТС с
централизованной обработкой и хранением информации, сочетающих в себе свойства
высокой доступности, отказо- и катастрофоустойчивости, информационной защищенности.
Определены принципы, методы и математические основы рационального построения и
оптимизации средств восстановления функционирования центров обработки данных (ЦОД)
после возникновения отказов и катастроф, передачи и хранения данных, обеспечения
информационной безопасности при достижении минимальной совокупной стоимости
владения такими системами. Результаты нашли практическое воплощение при реализации
проектов в интересах ряда отечественных государственных и негосударственных
организаций, таких как Банк России (БР), Внешторгбанк, ОАО <<ГМК <<Норильский Никель>>,
<<Газпром>>, Минэкономразвития России, Правительство Москвы, а также ряд силовых
ведомств.

      Под руководством В.\,И.~Будзко начиная с 2001~г.\ выполнен комплекс
      на\-уч\-но-ис\-сле\-до\-ва\-тель\-ских и
      опыт\-но-кон\-ст\-рук\-тор\-ских работ (свыше 100~проектов),
направленных на развитие электронной информационной технологии БР.
Разработаны концепции развития ИТС БР сначала до 2008~г., а затем до 2013~г., которые
были приняты в качестве основы проведения технической политики. За реализацию проекта
<<Катастрофоустойчивая тер\-ри\-то\-ри\-аль\-но-рас\-пре\-де\-лен\-ная
      ин\-фор\-ма\-ци\-он\-но-те\-ле\-ком\-му\-ни\-ка\-ци\-он\-ная сис\-те\-ма централизованной
обработки банковской информации>> В.\,И.~Будзко удостоен Премии Правительства РФ в
области науки и техники за 2010~г.

      В.\,И.~Будзко возглавлял и возглавляет работы по ряду других прикладных проектов,
связанных с созданием, совершенствованием и развитием крупномасштабных ИТС.

      В.\,И.~Будзко~--- генерал-майор, доктор технических наук, член-кор\-рес\-пон\-дент
Академии криптографии РФ, известный ученый в области информатики и применения
информационных технологий при построении территориально распределенных ИТС
различного назначения. Является автором свыше 250~научных работ, опубликованных в
на\-уч\-но-тех\-ни\-че\-ских и специальных изданиях.

    \thispagestyle{empty}

      В.\,И.~Будзко уделяет большое внимание подготовке научных кадров. Под его
руководством защищено 6~диссертаций на соискание ученой степени кандидата
технических наук. Свыше 30~лет он читает лекции в ИКСИ Академии ФСБ, профессор
кафедры НИЯУ МИФИ. Является членом двух диссертационных советов, главным
редактором журнала <<Системы высокой доступности>> и членом редколлегии журнала
<<Информатика и её применения>>.

      \bigskip

      Редакционный совет и Редакционная коллегия журнала <<Информатика и её
применения>> сердечно поздравляют Владимира Игоревича Будзко с 70-ле\-ти\-ем и желают
крепкого здоровья и новых научных достижений.

\end{multicols}


%Информатика Т 16 Год 2022-1\\
\def\stat{cont}
{%\hrule\par
%\vskip 7pt % 7pt
\raggedleft\Large \bf%\baselineskip=3.2ex
А\,В\,Т\,О\,Р\,С\,К\,И\,Й\ \ У\,К\,А\,З\,А\,Т\,Е\,Л\,Ь\ \ З\,А\ \ 2\,0\,2\,2 г. \vskip 17pt
 \hrule
 \par
\vskip 21pt plus 6pt minus 3pt }

\label{st\stat}

\def\tit{\ }

\def\aut{\ }
\def\auf{\ }

\def\leftkol{\ } % ENGLISH ABSTRACTS}

\def\rightkol{\ } %АВТОРСКИЙ УКАЗАТЕЛЬ ЗА 2021 г.} %ENGLISH ABSTRACTS}

\titele{\tit}{\aut}{\auf}{\leftkol}{\rightkol}
\addcontentsline{toc}{subsection}{\textrm\textbf Авторский указатель за 2022 г.}

\vspace*{-24pt}

\noindent
{\tabcolsep=3pt
\begin{tabular}{p{397pt}cc}
&\textbf{Вып.} & \textbf{Стр.}\\[6pt]
\Avtors{Абгарян~К.\,К., Гаврилов~Е.\,С.} Программный комплекс для 
многомасштабного модели-\linebreak
\\[-12pt]
\hspace*{23pt}рования структурных свойств композиционных 
материалов&1&88--97\\
\Avtors{Аблаев~Ф.\,М.} см.\ Андрианов~С.\,Н.&&\\
\Avtors{Агаларов Я.\,М.} Оптимальное управление подключением резервного прибора 
в~системе\linebreak
\\[-12pt]
\hspace*{23pt}массового обслуживания $G/M/1$&4&34--41\\
\Avtors{Агаларов~Я.\,М.} Оптимизация порогового управления переключением 
скорости обслу-\linebreak
\\[-12pt]
\hspace*{23pt}живания в~системе массового обслуживания $G/M/1$&1&73--81\\
\Avtors{Агасандян~Г.\,А.} Многомерные бинарные рынки и~CC-VaR&2&\hphantom{1}2--10\\
\Avtors{Алию~Б., Мачнев~Е.\,А., Мокров~Е.\,В.} Гистерезисное управление нагрузкой 
в~беспроводных\linebreak
\\[-12pt]
\hspace*{23pt}сенсорных сетях&3&83--89\\
\Avtors{Андрианов~С.\,Н., Андрианова~Н.\,С., Аблаев~Ф.\,М., Кочнева~Ю.\,Ю.} 
Контекстный поиск\linebreak
\\[-12pt]
\hspace*{23pt}на фотонах с~использованием тестов Белла&1&20--24\\
\Avtors{Андрианова~Н.\,С.} см.\ Андрианов~С.\,Н.&&\\
\Avtors{Базилевский М.\,П.} Обобщение метода выпрямления искаженных из-за 
мультиколлинеарности коэффициентов для~регрессионных моделей с~различной 
степенью\linebreak
\\[-12pt]
\hspace*{23pt}корреляции объясняющих переменных&4&20--25\\
\Avtors{Бесчастный~В.\,А., Острикова~Д.\,Ю., Шоргин~С.\,Я., Молчанов~Д.\,А., 
Гайдамака~Ю.\,В.} Анализ плотности базовых станций 5G NR для предоставления услуг 
виртуальной\linebreak
\\[-12pt]
\hspace*{23pt}и~дополненной реальности&2&102--108\\
\Avtors{Бесчастный~В.\,А. } см.\ Мачнев Е.\,А.&&\\
\Avtors{Битюков~Ю.\,И.} см.\ Босов~А.\,В.&&\\
\Avtors{Борисов А.\,В.} Общий порядок аппроксимации оценок фильтрации состояний 
марков-\linebreak
\\[-12pt]
\hspace*{23pt}ских скачкообразных процессов по~дискретизованным наблюдениям&4&8--13\\
\Avtors{Босов~А.\,В.} Применение самоорганизующихся нейронных сетей к~процессу 
формиро-\linebreak
\\[-12pt]
\hspace*{23pt}вания индивидуальной траектории обучения&3&\hphantom{1}7--15\\
\Avtors{Босов~А.\,В.} Управление линейным выходом марковской цепи по квадратичному 
крите-\linebreak
\\[-12pt]
\hspace*{23pt}рию. Случай полной информации&2&19--26\\
\Avtors{Босов~А.\,В., Битюков~Ю.\,И., Денискина~Г.\,Ю.} О~поиске оптимальной 
схемы 3D-печати\linebreak
\\[-12pt]
\hspace*{23pt}конструкций из композиционных материалов&1&10--19\\
\Avtors{Босов А.\,В., Иванов А.\,В.} Технология классификации типов контента 
электронного\linebreak
\\[-12pt]
\hspace*{23pt}учебника&4&63--72\\
\Avtors{Брюхов Д.\,О., Ступников~С.\,А.} Логическая реляционная модель структур 
данных для\linebreak
\\[-12pt]
\hspace*{23pt}решения задач в~предметной области управления 
землепользованием&4&93--98\\
\Avtors{Бурцева~С.\,А.} см.\ Власкина~А.\,С.&&\\
\Avtors{Васильев~Н.\,С.} О~достаточных условиях экстремума в~многомерных 
вариационных\linebreak
\\[-12pt]
\hspace*{23pt}задачах&3&39--44\\
\Avtors{Власкина~А.\,С., Бурцева~С.\,А., Кочеткова~И.\,А., Шоргин~С.\,Я.} Управляемая 
система массового обслуживания с~эластичным трафиком и~сигналами для анализа 
нарезки\linebreak
\\[-12pt]
\hspace*{23pt}ресурсов в~сети радиодоступа&3&90--96\\
\Avtors{Гаврилов~Е.\,С.} см.\ Абгарян~К.\,К.&&\\
\Avtors{Гайдамака~Ю.\,В.} см.\ Бесчастный~В.\,А.&&\\
\Avtors{Гайдамака~Ю.\,В.} см.\ Мачнев Е.\,А.&&\\
\Avtors{Горшенин~А.\,К., Гусейнова~Е.\,И.} Повышение доходности торговли на~FOREX 
с~помощью\linebreak
\\[-12pt]
\hspace*{23pt}LSTM-идентификации свечных паттернов и~индикатора тиковых 
объемов&3&26--38\\
\Avtors{Григорьев~О.\,Г.} см.\ Смирнов~И.\,В.&&\\
\end{tabular}
}

\pagebreak

\def\leftkol{АВТОРСКИЙ УКАЗАТЕЛЬ ЗА 2022 г.} % ENGLISH ABSTRACTS}

\def\rightkol{АВТОРСКИЙ УКАЗАТЕЛЬ ЗА 2022 г.} %ENGLISH ABSTRACTS}

%\thispagestyle{myheadings}
\def\leftfootline{\small{\textbf{\thepage}
\hfill ИНФОРМАТИКА И ЕЁ ПРИМЕНЕНИЯ\ \ \ том~16\ \ \ выпуск~4\ \ \ 2022}
}%
 \def\rightfootline{\small{ИНФОРМАТИКА И ЕЁ ПРИМЕНЕНИЯ\ \ \ том~16\ \ \ выпуск~4\ \ \ 2022
 \hfill \textbf{\thepage}}}


\noindent
{\tabcolsep=3pt
\begin{tabular}{p{394pt}cc}
&\textbf{Вып.} & \textbf{Стр.}\\[3pt]
\Avtors{Грушо~А.\,А., Грушо~Н.\,А., Забежайло~М.\,И., Зацаринный~А.\,А., 
Тимонина~Е.\,Е., Шор-}\linebreak
\\[-12pt]
\hspace*{23pt}\textbf{гин~С.\,Я.} Анализ цепочек причинно-следственных связей&2&68--74\\
\Avtors{Грушо А.\,А., Грушо Н.\,А., Забежайло~М.\,И., Смирнов~Д.\,В., Тимонина~Е.\,Е., 
Шоргин~С.\,Я.}\linebreak
\\[-12pt]
\hspace*{23pt}О~безопасной архитектуре вычислительной системы на основе 
микросервисов&4&87--92\\
\Avtors{Грушо~А.\,А., Грушо~Н.\,А., Тимонина~Е.\,Е.} Метаданные в~защищенном 
электронном\linebreak
\\[-12pt]
\hspace*{23pt}документообороте&3&\hphantom{1}97--102\\
\Avtors{Грушо~Н.\,А.} см.\ Грушо~А.\,А.&&\\
\Avtors{Грушо Н.\,А.} см.\ Грушо А.\,А.&&\\
\Avtors{Грушо~Н.\,А.} см.\ Грушо~А.\,А.&&\\
\Avtors{Гусейнова~Е.\,И.} см.\ Горшенин~А.\,К.&&\\
\Avtors{Денискина~Г.\,Ю.} см.\ Босов~А.\,В.&&\\
\Avtors{Драгунов~Н.\,А., Дюкова~Е.\,В.} О~поиске максимальных частых 
и~минимальных нечастых\linebreak
\\[-12pt]
\hspace*{23pt}наборов произведения частичных порядков&1&82--87\\
\Avtors{Дубанов~А.\,А., Нефедова~В.\,А.} Кинематические модели задач преследования 
на~плос-\linebreak
\\[-12pt]
\hspace*{23pt}кости методами параллельного сближения и~погони&3&103--109\\
\Avtors{Дунсяо Гу} см.\ Зацман И.\,М.&&\\
\Avtors{Дурново~А.\,А., Инькова~О.\,Ю., Попкова~Н.\,А.} Принципы описания 
показателей логико-\linebreak
\\[-12pt]
\hspace*{23pt}семантических отношений и~их иерархии&2&52--59\\
\Avtors{Дьяченко~Ю.\,Г.} см.\ Соколов И.\,А.&&\\
\Avtors{Дюкова А.\,П.} см.\ Дюкова Е.\,В.&&\\
\Avtors{Дюкова Е.\,В., Дюкова А.\,П.} О~сложности обучения логических процедур 
классификации&4&57--62\\
\Avtors{Дюкова~Е.\,В.} см.\ Драгунов~Н.\,А.&&\\
\Avtors{Забежайло~М.\,И.} см.\ Грушо А.\,А.&&\\
\Avtors{Забежайло~М.\,И.} см.\ Грушо~А.\,А.&&\\
\Avtors{Зацаринный~А.\,А.} см.\ Грушо~А.\,А.&&\\
\Avtors{Зацман И.\,М.} О~научной парадигме информатики: верхний уровень 
классификации\linebreak
\\[-12pt]
\hspace*{23pt}объектов ее предметной области&4&73--79\\
\Avtors{Зацман~И.\,М.} Средовые модели информационных технологий: теоретические 
основа-\linebreak
\\[-12pt]
\hspace*{23pt}ния построения&3&59--67\\
\Avtors{Зацман~И.\,М., Золотарев~О.\,В., Хакимова~А.\,Х.} Средовые модели извлечения 
из текста\linebreak
\\[-12pt]
\hspace*{23pt}новых терминов и~индикаторов настроений&2&60--67\\
\Avtors{Зацман И.\,М., Золотарев~О.\,В., Хакимова~А.\,Х., Дунсяо~Гу.} Модель 
и~технология\linebreak
\\[-12pt]
\hspace*{23pt}извлечения новых терминов из~медицинских текстов&4&80--86\\
\Avtors{Зейфман~А.\,И.} см.\ Ковалёв~И.\,А.&&\\
\Avtors{Зейфман~А.\,И.} см.\ Сатин~Я.\,А.&&\\
\Avtors{Золотарев~О.\,В.} см.\ Зацман И.\,М.&&\\
\Avtors{Золотарев~О.\,В.} см.\ Зацман~И.\,М.&&\\
\Avtors{Иванов А.\,В.} см.\ Босов А.\,В.&&\\
\Avtors{Инькова~О.\,Ю.} см.\ Дурново~А.\,А.&&\\
\Avtors{Кириков~И.\,А.} см.\ Листопад~С.\,В.&&\\
\Avtors{Кириков~И.\,А.} см.\ Румовская~С.\,Б.&&\\
\Avtors{Киселёв~Г.\,А.} см.\ Смирнов~И.\,В.&&\\
\Avtors{Ковалёв~И.\,А., Сатин~Я.\,А., Синицина~А.\,В., Зейфман~А.\,И.} Об одном 
подходе к~оцениванию скорости сходимости нестационарных марковских моделей систем 
обслужи-\linebreak
\\[-12pt]
\hspace*{23pt}вания&3&75--82\\
\Avtors{Ковалёв~С.\,П.} Алгебраическая спецификация графовых вычислительных 
структур&1&2--9\\
\Avtors{Коновалов~М.\,Г., Разумчик~Р.\,В.} Синтез управления двумерным случайным 
блужданием\linebreak
\\[-12pt]
\hspace*{23pt}с~эталонным стационарным распределением&2&109--117\\
\Avtors{Кочеткова~И.\,А.} см.\ Власкина~А.\,С.&&\\
\Avtors{Кочнева~Ю.\,Ю.} см.\ Андрианов~С.\,Н.&&\\
\Avtors{Кравцова~О.\,А.} Использование критериев стационарности для настройки 
моделей при\linebreak
\\[-12pt]
\hspace*{23pt}прогнозировании временных рядов&2&11--18\\
\Avtors{Кривенко~М.\,П.} Выбор модели при факторизации матрицы данных 
с~пропусками&3&52--58\\
\Avtors{Крюкова~А.\,Л.} см.\ Сатин~Я.\,А.&&\\
\end{tabular}
}

\pagebreak

\def\leftkol{АВТОРСКИЙ УКАЗАТЕЛЬ ЗА 2022 г.} % ENGLISH ABSTRACTS}

\def\rightkol{АВТОРСКИЙ УКАЗАТЕЛЬ ЗА 2022 г.} %ENGLISH ABSTRACTS}

%\thispagestyle{myheadings}
\def\leftfootline{\small{\textbf{\thepage}
\hfill ИНФОРМАТИКА И ЕЁ ПРИМЕНЕНИЯ\ \ \ том~16\ \ \ выпуск~4\ \ \ 2022}
}%
 \def\rightfootline{\small{ИНФОРМАТИКА И ЕЁ ПРИМЕНЕНИЯ\ \ \ том~16\ \ \ выпуск~4\ \ \ 2022
 \hfill \textbf{\thepage}}}


\noindent
{\tabcolsep=3pt
\begin{tabular}{p{394pt}cc}
&\textbf{Вып.} & \textbf{Стр.}\\[3pt]
\Avtors{Курузов~И.\,А.} см.\ Смирнов~И.\,В.&&\\[0.3pt]
\Avtors{Листопад~С.\,В., Кириков~И.\,А.} Разрешение конфликтов в~гибридных 
интеллектуальных\linebreak
\\[-12pt]
\hspace*{23pt}многоагентных системах&1&54--60\\[0.3pt]
\Avtors{Малашенко~Ю.\,Е.} Метрические оценки угловых точек множества достижимых 
межуз-\linebreak
\\[-12pt]
\hspace*{23pt}ловых потоков многопользовательской сети&1&25--31\\[0.3pt]
\Avtors{Малашенко~Ю.\,Е.} Последовательный анализ и~метрические оценки 
предельных рас-\linebreak
\\[-12pt]
\hspace*{23pt}пределений межузловых потоков в~многопользовательской сети&3&45--51\\[0.3pt]
\Avtors{Мачнев Е.\,А., Бесчастный~В.\,А., Острикова~Д.\,Ю., Гайдамака~Ю.\,В., 
Шоргин~С.\,Я.} Об оптимальном расположении антенн для~V2X-соединений 
в~субтерагерцевом диа-\linebreak
\\[-12pt]
\hspace*{23pt}пазоне&4&42--50\\
\Avtors{Мачнев~Е.\,А.} см.\ Алию~Б.&&\\[0.3pt]
\Avtors{Мигуля~М.\,А.} см.\ Шнурков~П.\,В.&&\\[0.3pt]
\Avtors{Мокров~Е.\,В.} см.\ Алию~Б.&&\\[0.3pt]
\Avtors{Молчанов~Д.\,А.} см.\ Бесчастный~В.\,А.&&\\[0.3pt]
\Avtors{Нефедова~В.\,А.} см.\ Дубанов~А.\,А.&&\\[0.3pt]
\Avtors{Нуриев~В.\,А.} Переводческий анализ текста с~применением информационных 
ресурсов:\linebreak
\\[-12pt]
\hspace*{23pt}редуцирование спектра моделей перевода в~надкорпусных базах 
данных&3&68--74\\[0.3pt]
\Avtors{Острикова~Д.\,Ю.} см.\ Бесчастный~В.\,А.&&\\[0.3pt]
\Avtors{Острикова~Д.\,Ю.} см.\ Мачнев Е.\,А.&&\\[0.3pt]
\Avtors{Ошушкова~В.\,С.} см.\ Сатин~Я.\,А.&&\\[0.3pt]
\Avtors{Палионная~С.\,И., Шестаков~О.\,В.} Использование FDR-метода множественной 
провер-\linebreak
\\[-12pt]
\hspace*{23pt}ки гипотез при обращении линейных однородных операторов&2&44--51\\[0.3pt]
\Avtors{Панов~А.\,И.} см.\ Смирнов~И.\,В.&&\\[0.3pt]
\Avtors{Пешкова И.\,В.} Границы экстремального индекса времени ожидания в~системе 
$M/G/1$\linebreak
\\[-12pt]
\hspace*{23pt}с~распределением времени обслуживания в~виде конечной смеси&4&26--33\\[0.3pt]
\Avtors{Пешкова~И.\,В.} Сравнение экстремальных индексов времен ожидания 
в~системах об-\linebreak
\\[-12pt]
\hspace*{23pt}служивания $M/G/1$&1&61--67\\[0.3pt]
\Avtors{Попкова~Н.\,А.} см.\ Дурново~А.\,А.&&\\[0.3pt]
\Avtors{Разумчик~Р.\,В.} см.\ Коновалов~М.\,Г.&&\\[0.3pt]
\Avtors{Рождественский~Ю.\,В.} см.\ Соколов И.\,А.&&\\[0.3pt]
\Avtors{Румовская~С.\,Б., Кириков~И.\,А.} Метод визуализации снижения интенсивности 
и~разре-\linebreak
\\[-12pt]
\hspace*{23pt}шения конфликтов в~гибридных интеллектуальных многоагентных 
системах&2&\hphantom{1}94--101\\[0.3pt]
\Avtors{Сатин~Я.\,А., Крюкова~А.\,Л., Ошушкова~В.\,С., Зейфман~А.\,И.} 
О~монотонности\linebreak
\\[-12pt]
\hspace*{23pt}некоторых классов марковских цепей&2&27--34\\[0.3pt]
\Avtors{Сатин~Я.\,А.} см.\ Ковалёв~И.\,А.&&\\[0.3pt]
\Avtors{Синицина~А.\,В.} см.\ Ковалёв~И.\,А.&&\\[0.3pt]
\Avtors{Синицын~И.\,Н.} Нормализация систем, стохастически не разрешенных 
относительно\linebreak
\\[-12pt]
\hspace*{23pt}производных&1&32--38\\[0.3pt]
\Avtors{Синицын~И.\,Н.} Совместная фильтрация и~распознавание нормальных 
процессов в~сто-\linebreak
\\[-12pt]
\hspace*{23pt}хастических системах, не разрешенных относительно 
производных&2&85--93\\
\Avtors{Смирнов~Д.\,В.} см.\ Грушо А.\,А.&&\\[0.3pt]
\Avtors{Смирнов~И.\,В., Панов~А.\,И., Чуганская~А.\,А., Суворова~М.\,И., 
Киселёв~Г.\,А., Курузов~И.\,А., Григорьев~О.\,Г.} Персональный когнитивный 
ассистент: планирование поведения\linebreak
\\[-12pt]
\hspace*{23pt}на основе сценариев деятельности&1&46--53\\[0.3pt]
\Avtors{Соколов И.\,А., Степченков~Ю.\,А., Дьяченко~Ю.\,Г., Рождественский~Ю.\,В.} 
Оценка надеж-\linebreak
\\[-12pt]
\hspace*{23pt}ности синхронного и~самосинхронного конвейеров&4&2--7\\[0.3pt]
\Avtors{Степченков~Ю.\,А.} см.\ Соколов И.\,А.&&\\[0.3pt]
\Avtors{Ступников~С.\,А.} см.\ Брюхов Д.\,О.&&\\[0.3pt]
\Avtors{Суворова~М.\,И.} см.\ Смирнов~И.\,В.&&\\[0.3pt]
\Avtors{Сучков А.\,П.} Единая модель государственных данных: сценарии 
развития&4&\hphantom{9}99--105\\[0.3pt]
\Avtors{Тимонина~Е.\,Е.} см.\ Грушо А.\,А.&&\\[0.3pt]
\Avtors{Тимонина~Е.\,Е.} см.\ Грушо~А.\,А.&&\\[0.3pt]
\Avtors{Тимонина~Е.\,Е} см.\ Грушо~А.\,А.&&\\
\end{tabular}
}

\pagebreak

\def\leftkol{АВТОРСКИЙ УКАЗАТЕЛЬ ЗА 2022 г.} % ENGLISH ABSTRACTS}

\def\rightkol{АВТОРСКИЙ УКАЗАТЕЛЬ ЗА 2022 г.} %ENGLISH ABSTRACTS}

%\thispagestyle{myheadings}
\def\leftfootline{\small{\textbf{\thepage}
\hfill ИНФОРМАТИКА И ЕЁ ПРИМЕНЕНИЯ\ \ \ том~16\ \ \ выпуск~4\ \ \ 2022}
}%
 \def\rightfootline{\small{ИНФОРМАТИКА И ЕЁ ПРИМЕНЕНИЯ\ \ \ том~16\ \ \ выпуск~4\ \ \ 2022
 \hfill \textbf{\thepage}}}


\noindent
{\tabcolsep=3pt
\begin{tabular}{p{394pt}cc}
&\textbf{Вып.} & \textbf{Стр.}\\[3pt]
\Avtors{Торшин~И.\,Ю.} О~применении топологического подхода к анализу плохо 
формализуемых задач для построения алгоритмов виртуального скрининга кван\-то\-во-ме\-ха\-ни\-че\-ских\linebreak
\\[-12pt]
\hspace*{23pt}свойств органических молекул I:~Основы проблемно ориентированной 
теории&1&39--45\\
\Avtors{Торшин~И.\,Ю.} О~применении топологического подхода к~анализу плохо 
формализуемых задач для построения алгоритмов виртуального скрининга кван\-то\-во-ме\-ха\-ни\-че\-ских 
свойств органических молекул II:~Сопоставление формализма 
с~конструктами\linebreak
\\[-12pt]
\hspace*{23pt}квантовой механики и экспериментальная апробация предложенных 
алгоритмов&2&35--43\\
\Avtors{Хакимова~А.\,Х.} см.\ Зацман И.\,М.&&\\
\Avtors{Хакимова~А.\,Х.} см.\ Зацман~И.\,М.&&\\
\Avtors{Хацкевич В.\,Л.} Нечеткие усредняющие операторы в~задаче агрегирования 
нечеткой\linebreak
\\[-12pt]
\hspace*{23pt}информации&4&51--56\\
\Avtors{Чуганская~А.\,А.} см.\ Смирнов~И.\,В.&&\\
\Avtors{Шведов~А.\,С.} Критерий непустоты эпсилон-ядер для нечетких игр с~нетрансферабель-\linebreak
\\[-12pt]
\hspace*{23pt}ной полезностью и~вычислительные процедуры&3&2--6\\
\Avtors{Шестаков О.\,В.} Несмещенная оценка риска пороговой обработки с~двумя 
пороговыми\linebreak
\\[-12pt]
\hspace*{23pt}значениями&4&14--19\\
\Avtors{Шестаков~О.\,В.} см.\ Палионная~С.\,И.&&\\
\Avtors{Шихиев~Ф.\,Ш.} см.\ Шихиев~Ш.\,Б.&&\\
\Avtors{Шихиев~Ш.\,Б., Шихиев~Ф.\,Ш.} Упрощенный язык зрительных 
образов&1&68--72\\
\Avtors{Шнурков~П.\,В.} Об аналитической структуре некоторых видов целевых 
функционалов,\linebreak
\\[-12pt]
\hspace*{23pt}связанных с~задачами управления полумарковскими случайными 
процессами&2&75--84\\
\Avtors{Шнурков~П.\,В., Мигуля~М.\,А.} Некоторые результаты анализа процесса 
изменения цены\linebreak
\\[-12pt]
\hspace*{23pt}бивалютной корзины на основе методов статистики случайных 
процессов&3&16--25\\
\Avtors{Шоргин~С.\,Я.} см.\ Бесчастный~В.\,А.&&\\
\Avtors{Шоргин~С.\,Я.} см.\ Власкина~А.\,С.&&\\
\Avtors{Шоргин~С.\,Я.} см.\ Грушо А.\,А.&&\\
\Avtors{Шоргин~С.\,Я.} см.\ Грушо~А.\,А.&&\\
\Avtors{Шоргин~С.\,Я.} см.\ Мачнев Е.\,А.&&\\
\end{tabular}
}

%\thispagestyle{myheadings}
\def\leftfootline{\small{\textbf{\thepage}
\hfill ИНФОРМАТИКА И ЕЁ ПРИМЕНЕНИЯ\ \ \ том~16\ \ \ выпуск~4\ \ \ 2022}
}%
 \def\rightfootline{\small{ИНФОРМАТИКА И ЕЁ ПРИМЕНЕНИЯ\ \ \ том~16\ \ \ выпуск~4\ \ \ 2022
 \hfill \textbf{\thepage}}}

 \label{end\stat}

\newpage

\def\stat{cont-e}
{%\hrule\par
%\vskip 7pt % 7pt
\raggedleft\Large \bf%\baselineskip=3.2ex
2\,0\,2\,2\ \ A\,U\,T\,H\,O\,R\ \ I\,N\,D\,E\,X \vskip 17pt
 \hrule
 \par
\vskip 21pt plus 6pt minus 3pt }

\label{st\stat}

\def\tit{\ }

\def\aut{\ }
\def\auf{\ }

\def\leftkol{\ } %2021 AUTHOR INDEX} % ENGLISH ABSTRACTS}

\def\rightkol{\ } %2021 AUTHOR INDEX} %ENGLISH ABSTRACTS}

\titele{\tit}{\aut}{\auf}{\leftkol}{\rightkol}
\addcontentsline{toc}{subsection}{\textrm\textbf 2022 Author Index}

\def\leftfootline{\small{\textbf{\thepage}
\hfill INFORMATIKA I EE PRIMENENIYA~--- INFORMATICS AND APPLICATIONS\ \ \ 2022\
\ \ volume~16\ \ \ issue\ 4}
}%
 \def\rightfootline{\small{INFORMATIKA I EE PRIMENENIYA~--- INFORMATICS AND APPLICATIONS\ \ \ 2022\ \ \ volume~16\ \ \ issue\ 4
\hfill \textbf{\thepage}}}

\vspace*{-24pt}

\noindent
{\tabcolsep=3pt
\begin{tabular}{p{395.89pt}cc}
&\textbf{Issue} & \textbf{Page}\\[6pt]
\Avtors{Abgaryan~K.\,K.\ and Gavrilov~E.\,S.} Software package for multiscale modeling of 
structural\linebreak
\\[-12pt]
\hspace*{23pt}properties of composite materials&1&88--97\\
\Avtors{Ablaev~F.\,M.} see Andrianov~S.\,N.&&\\
\Avtors{Agalarov Ya.\,M.} Optimal control of~a~queue-length dependent additional server 
in~$\mathrm{GI}/M/1$\linebreak
\\[-12pt]
\hspace*{23pt}queue&4&34--41\\
\Avtors{Agalarov~Ya.\,M.} Optimization of the threshold service speed control in the $G/M/1$ 
queue&1&73--81\\
\Avtors{Agasandyan~G.\,A.} Multidimensional binary markets and CC-VaR&2&\hphantom{1}2--10\\
\Avtors{Aliyu~B., Machnev~E.\,A., and Mokrov~E.\,V.} Hysteretic congestion control in 
wireless cloud\linebreak
\\[-12pt]
\hspace*{23pt}sensor networks&3&83--89\\
\Avtors{Andrianov~S.\,N., Andrianova~N.\,S., Ablaev~F.\,M., and Kochneva~Yu.\,Yu.} 
Context query on\linebreak
\\[-12pt]
\hspace*{23pt}photons with the use of Bell tests&1&20--24\\
\Avtors{Andrianova~N.\,S.} see Andrianov~S.\,N.&&\\
\Avtors{Bazilevskiy M.\,P.} Generalization of~a~method for~straightening coefficients 
distorted due~to~mul-\linebreak
\\[-12pt]
\hspace*{23pt}ticollinearity in~regression models with different degrees of~explanatory 
variables correlation&4&20--25\\
\Avtors{Beschastnyi~V.\,A., Ostrikova~D.\,Yu., Shorgin~S.\,Ya., Moltchanov~D.\,A., and 
Gaidamaka~Yu.\,V.}\linebreak
\\[-12pt]
\hspace*{23pt}Density analysis of mmWave NR deployments for delivering scalable 
AR/VR video services&2&102--108\\
\Avtors{Beschastnyi~V.\,A.} see Machnev E.\,A.&&\\
\Avtors{Bityukov~Yu.\,I.} see Bosov~A.\,V.&&\\
\Avtors{Borisov A.\,V.} Total approximation order for~Markov jump process filtering given 
discretized\linebreak
\\[-12pt]
\hspace*{23pt}observations&4&8--13\\
\Avtors{Bosov~A.\,V.} Application of self-organizing neural networks to the process of forming 
an individual\linebreak
\\[-12pt]
\hspace*{23pt}learning path&3&\hphantom{1}7--15\\
\Avtors{Bosov~A.\,V.} Linear output control of Markov chain by square criterion. Complete 
information\linebreak
\\[-12pt]
\hspace*{23pt}case&2&19--26\\
\Avtors{Bosov~A.\,V., Bityukov~Yu.\,I., and Deniskina~G.\,Yu.} About searching for the 
optimal 3D printing\linebreak
\\[-12pt]
\hspace*{23pt}scheme of structures from composite materials&1&10--19\\
\Avtors{Bosov A.\,V. and Ivanov~A.\,V.} Technology for~classification of~content types of~e-textbooks&4&63--72\\
\Avtors{Briukhov D.\,O. and Stupnikov~S.\,A.} Logical relational model of~data structures 
for~problem\linebreak
\\[-12pt]
\hspace*{23pt}solving in~land use management&4&93--98\\
\Avtors{Burtseva~S.\,A.} see Vlaskina~A.\,S.&&\\
\Avtors{Chuganskaya~A.\,A.} see Smirnov~I.\,V&&\\
\Avtors{Deniskina~G.\,Yu.} see Bosov~A.\,V.&&\\
\Avtors{Diachenko~Yu.\,G.} see Sokolov I.\,A.&&\\
\Avtors{Djukova~A.\,P.} see Djukova E.\,V.&&\\
\Avtors{Djukova E.\,V. and Djukova~A.\,P.} On the~complexity of~logical classification 
learning procedures&4&57--62\\
\Avtors{Djukova~E.\,V.} see Dragunov~N.\,A.&&\\
\Avtors{Dongxiao~Gu} see Zatsman I.\,M.&&\\
\Avtors{Dragunov~N.\,A.\ and Djukova~E.\,V.} Finding maximal frequent and minimal 
infrequent sets\linebreak
\\[-12pt]
\hspace*{23pt}in partially ordered data&1&82--87\\
\Avtors{Dubanov~A.\,A.\ and Nefedova~V.\,A.} Kinematic models of pursuit problems on the 
plane\linebreak
\\[-12pt]
\hspace*{23pt}by the methods of parallel approach and pursuit&3&103--109\\
\Avtors{Durnovo~A.\,A., Inkova~O.\,Yu., and Popkova~N.\,A.} Principles of describing 
markers of logical-\linebreak
\\[-12pt]
\hspace*{23pt}semantic relations and their hierarchy&2&52--59\\
\Avtors{Gaidamaka~Yu.\,V.} see Beschastnyi~V.\,A.&&\\
\Avtors{Gaidamaka~Yu.\,V.} see Machnev E.\,A.&&\\
\Avtors{Gavrilov~E.\,S.} see Abgaryan~K.\,K.&&\\

\end{tabular}
}
\pagebreak

\def\leftfootline{\small{\textbf{\thepage}
\hfill INFORMATIKA I EE PRIMENENIYA~--- INFORMATICS AND APPLICATIONS\ \ \ 2022\
\ \ volume~16\ \ \ issue\ 4}
}%
 \def\rightfootline{\small{INFORMATIKA I EE PRIMENENIYA~---
INFORMATICS AND APPLICATIONS\ \ \ 2022\ \ \ volume~16\ \ \ issue\ 4
\hfill \textbf{\thepage}}}

\def\leftkol{2022 AUTHOR INDEX} % ENGLISH ABSTRACTS}

\def\rightkol{2022 AUTHOR INDEX} %ENGLISH ABSTRACTS}


\noindent
{\tabcolsep=3pt
\begin{tabular}{p{395.5pt}cc}
&\textbf{Issue} & \textbf{Page}\\[6pt]
\Avtors{Gorshenin~A.\,K.\ and Guseynova~E.\,I.} Increasing FOREX trading profitability with 
LSTM\linebreak
\\[-12pt]
\hspace*{23pt}candlestick pattern recognition and tick volume indicator&3&26--38\\
\Avtors{Grigoriev~O.\,G.} see Smirnov~I.\,V&&\\[-0.1pt]
\Avtors{Grusho~A.\,A., Grusho~N.\,A., and Timonina~E.\,E.} Metadata in secure electronic 
document\linebreak
\\[-12pt]
\hspace*{23pt}management&3&\hphantom{1}97--102\\[-0.1pt]
\Avtors{Grusho A.\,A., Grusho~N.\,A., Zabezhailo~M.\,I., Smirnov~D.\,V., Timonina~E.\,E., 
and Shorgin~S.\,Ya.}\linebreak
\\[-12pt]
\hspace*{23pt}About the~secure architecture of~a~microservice-based computing 
system&4&87--92\\[-0.1pt]
\Avtors{Grusho~A.\,A., Grusho~N.\,A., Zabezhailo~M.\,I., Zatsarinny~A.\,A., 
Timonina~E.\,E.,}\linebreak
\\[-12pt]
\hspace*{23pt}\textbf{and Shorgin~S.\,Ya.} Cause-and-effect chain analysis&2&68--74\\
\Avtors{Grusho~N.\,A.} see Grusho A.\,A.&&\\[-0.1pt]
\Avtors{Grusho~N.\,A.} see Grusho~A.\,A.&&\\[-0.1pt]
\Avtors{Grusho~N.\,A.} see Grusho~A.\,A.&&\\[-0.1pt]
\Avtors{Guseynova~E.\,I.} see Gorshenin~A.\,K.&&\\
\Avtors{Inkova~O.\,Yu.} see Durnovo~A.\,A.&&\\[-0.1pt]
\Avtors{Ivanov~A.\,V.} see Bosov A.\,V.&&\\[-0.1pt]
\Avtors{Khakimova~A.\,K.} see Zatsman I.\,M.&&\\[-0.1pt]
\Avtors{Khakimova~A.\,K.} see Zatsman~I.\,M.&&\\[-0.1pt]
\Avtors{Khatskevich V.\,L.} Fuzzy averaging operators in~the~problem of~aggregating fuzzy 
information&4&51--56\\[-0.1pt]
\Avtors{Kirikov~I.\,A.} see Listopad~S.\,V.&&\\[-0.1pt]
\Avtors{Kirikov~I.\,A.} see Rumovskaya~S.\,B.&&\\[-0.1pt]
\Avtors{Kiselev~G.\,A.} see Smirnov~I.\,V&&\\[-0.1pt]
\Avtors{Kochetkova~I.\,A.} see Vlaskina~A.\,S.&&\\[-0.1pt]
\Avtors{Kochneva~Yu.\,Yu.} see Andrianov~S.\,N.&&\\[-0.1pt]
\Avtors{Konovalov~M.\,G.\ and Razumchik~R.\,V.} Controlling a bounded two-dimensional 
Markov chain\linebreak
\\[-12pt]
\hspace*{23pt}with a~given invariant measure&2&109--117\\[-0.1pt]
\Avtors{Kovalev~I.\,A., Satin~Y.\,A., Sinitcina~A.\,V., and Zeifman~A.\,I.} On an approach 
for estimating\linebreak
\\[-12pt]
\hspace*{23pt}the rate of convergence for nonstationary Markov models of queueing 
systems&3&75--82\\[-0.1pt]
\Avtors{Kovalyov~S.\,P.} Algebraic specification of graph computational structures&1&2--9\\
\Avtors{Kravtsova~O.\,A.} Model setting using stationarity criteria for time series 
forecasting&2&11--18\\[-0.1pt]
\Avtors{Krivenko~M.\,P.} Model selection for matrix factorization with missing 
components&3&52--58\\[-0.1pt]
\Avtors{Kryukova~A.\,L.} see Satin~Y.\,A.&&\\[-0.1pt]
\Avtors{Kuruzov~I.\,A.} see Smirnov~I.\,V&&\\[-0.1pt]
\Avtors{Listopad~S.\,V.\ and Kirikov~I.\,A.} Conflict resolution in hybrid intelligent multiagent 
systems&1&54--60\\[-0.1pt]
\Avtors{Machnev E.\,A., Beschastnyi~V.\,A., Ostrikova~D.\,Yu., Gaidamaka~Yu.\,V., and 
Shorgin~S.\,Ya.} On\linebreak
\\[-12pt]
\hspace*{23pt}the optimal antenna deployment for~subterahertz V2X 
communications&4&42--50\\[-0.1pt]
\Avtors{Machnev~E.\,A.} see Aliyu~B.&&\\[-0.1pt]
\Avtors{Malashenko~Yu.\,E.} Metric evaluations of the angular points of the set of attainable 
internodal\linebreak
\\[-12pt]
\hspace*{23pt}flows of multiuser network&1&25--31\\[-0.1pt]
\Avtors{Malashenko~Yu.\,E.} Sequential analysis and metric estimates of peak load flows in 
the multiuser\linebreak
\\[-12pt]
\hspace*{23pt}network&3&45--51\\[-0.1pt]
\Avtors{Migulya~M.\,A.} see Shnurkov~P.\,V.&&\\[-0.1pt]
\Avtors{Mokrov~E.\,V.} see Aliyu~B.&&\\[-0.1pt]
\Avtors{Moltchanov~D.\,A.} see Beschastnyi~V.\,A.&&\\[-0.1pt]
\Avtors{Nefedova~V.\,A.} see Dubanov~A.\,A.&&\\[-0.1pt]
\Avtors{Nuriev~V.\,A.} Computer-assisted textual analysis in translation: Reducing the 
spectrum of\linebreak
\\[-12pt]
\hspace*{23pt}translation models in supracorpora databases&3&68--74\\[-0.1pt]
\Avtors{Oshushkova~V.\,S.} see Satin~Y.\,A.&&\\[-0.1pt]
\Avtors{Ostrikova~D.\,Yu.} see Beschastnyi~V.\,A.&&\\[-0.1pt]
\Avtors{Ostrikova~D.\,Yu.} see Machnev E.\,A.&&\\[-0.1pt]
\Avtors{Palionnaya~S.\,I.\ and Shestakov~O.\,V.} The use of the FDR method of multiple 
hypothesis testing\linebreak
\\[-12pt]
\hspace*{23pt}when inverting linear homogeneous operators&2&44--51\\[-0.1pt]
\Avtors{Panov~A.\,I.} see Smirnov~I.\,V&&\\[-0.1pt]
\Avtors{Peshkova I.\,V.} On bounds of~the~stationary waiting time extremal index 
in~$M/G/1$ system\linebreak
\\[-12pt]
\hspace*{23pt}with mixture service times&4&26--33\\[-0.1pt]
\end{tabular}
}
\pagebreak

\def\leftfootline{\small{\textbf{\thepage}
\hfill INFORMATIKA I EE PRIMENENIYA~--- INFORMATICS AND APPLICATIONS\ \ \ 2022\
\ \ volume~16\ \ \ issue\ 4}
}%
 \def\rightfootline{\small{INFORMATIKA I EE PRIMENENIYA~---
INFORMATICS AND APPLICATIONS\ \ \ 2022\ \ \ volume~16\ \ \ issue\ 4
\hfill \textbf{\thepage}}}

\def\leftkol{2022 AUTHOR INDEX} % ENGLISH ABSTRACTS}

\def\rightkol{2022 AUTHOR INDEX} %ENGLISH ABSTRACTS}


\noindent
{\tabcolsep=3pt
\begin{tabular}{p{395.5pt}cc}
&\textbf{Issue} & \textbf{Page}\\[6pt]
\Avtors{Peshkova~I.\,V.} The comparison of waiting time extremal indexes in $M/G/1$ 
queueing systems&1&61--67\\[-0.1pt]
\Avtors{Popkova~N.\,A.} see Durnovo~A.\,A.&&\\[-0.1pt]
\Avtors{Razumchik~R.\,V.} see Konovalov~M.\,G.&&\\[-0.1pt]
\Avtors{Rogdestvenski~Yu.\,V.} see Sokolov I.\,A.&&\\[-0.1pt]
\Avtors{Rumovskaya~S.\,B.\ and Kirikov~I.\,A.} Visual representation of the decrease in 
conflict intensity\linebreak
\\[-12pt]
\hspace*{23pt}and its resolution in hybrid intelligent multiagent 
systems&2&\hphantom{1}94--101\\[-0.1pt]
\Avtors{Satin~Y.\,A., Kryukova~A.\,L., Oshushkova~V.\,S., and Zeifman~A.\,I.} On 
monotonicity of some\linebreak
\\[-12pt]
\hspace*{23pt}classes of Markov chains&2&27--34\\[-0.1pt]
\Avtors{Satin~Y.\,A.} see Kovalev~I.\,A.&&\\[-0.1pt]
\Avtors{Shestakov O.\,V.} Unbiased thresholding risk estimate with two threshold 
values&4&14--19\\[-0.1pt]
\Avtors{Shestakov~O.\,V.} see Palionnaya~S.\,I.&&\\[-0.1pt]
\Avtors{Shihiev~F.\,Sh.} see Shihiev~Sh.\,B.&&\\[-0.1pt]
\Avtors{Shihiev~Sh.\,B.\ and Shihiev~F.\,Sh.} Simplified language for visual images&1&68--72\\[-0.1pt]
\Avtors{Shnurkov~P.\,V.} On the analytical structure of some kinds of target functionals 
associated with\linebreak
\\[-12pt]
\hspace*{23pt}the control problems of semi-Markov stoсhastic processes&2&75--84\\[-0.1pt]
\Avtors{Shnurkov~P.\,V.\ and Migulya~M.\,A.} Some results of the analysis of the process of 
changing\linebreak
\\[-12pt]
\hspace*{23pt}the price of a dual currency basket based on random process statistics 
methods&3&16--25\\[-0.1pt]
\Avtors{Shorgin~S.\,Ya.} see Beschastnyi~V.\,A.&&\\[-0.1pt]
\Avtors{Shorgin~S.\,Ya.} see Grusho A.\,A.&&\\[-0.1pt]
\Avtors{Shorgin~S.\,Ya.} see Grusho~A.\,A.&&\\[-0.1pt]
\Avtors{Shorgin~S.\,Ya.} see Machnev E.\,A.&&\\[-0.1pt]
\Avtors{Shorgin~S.\,Ya.} see Vlaskina~A.\,S.&&\\[-0.1pt]
\Avtors{Shvedov~A.\,S.} A~condition for non-emptiness of the epsilon-core of 
a~nontransferable utility\linebreak
\\[-12pt]
\hspace*{23pt}fuzzy game and computational schemes&3&2--6\\[-0.1pt]
\Avtors{Sinitcina~A.\,V.} see Kovalev~I.\,A.&&\\[-0.1pt]
\Avtors{Sinitsyn~I.\,N.} Joint filtration and recognition of normal proсesses in stochastic 
systems with\linebreak
\\[-12pt]
\hspace*{23pt}unsolved derivatives&2&85--93\\[-0.1pt]
\Avtors{Sinitsyn~I.\,N.} Normalization of systems with stochastically unsolved 
derivatives&1&32--38\\[-0.1pt]
\Avtors{Smirnov~D.\,V.} see Grusho A.\,A.&&\\[-0.1pt]
\Avtors{Smirnov~I.\,V., Panov~A.\,I., Chuganskaya~A.\,A., Suvorova~M.\,I., Kiselev~G.\,A., 
Kuruzov~I.\,A., and}\linebreak
\\[-12pt]
\hspace*{23pt}\textbf{Grigoriev~O.\,G.} Personal cognitive assistant: Planning activity with 
scripts&1&46--53\\[-0.1pt]
\Avtors{Sokolov I.\,A., Stepchenkov Yu.\,A., Diachenko~Yu.\,G., 
and~Rogdestvenski~Yu.\,V.} Synchronous and\linebreak
\\[-12pt]
\hspace*{23pt}self-timed pipeline's reliability 
estimation&4&2--7\\[-0.1pt]
\Avtors{Stepchenkov Yu.\,A.} see Sokolov I.\,A.&&\\[-0.1pt]
\Avtors{Stupnikov~S.\,A.} see Briukhov D.\,O.&&\\[-0.1pt]
\Avtors{Suchkov A.\,P.} Unified model of national data: Development scenarios&4&\hphantom{9}99--105\\
\Avtors{Suvorova~M.\,I.} see Smirnov~I.\,V&&\\[-0.1pt]
\Avtors{Timonina~E.\,E.} see Grusho A.\,A.&&\\[-0.1pt]
\Avtors{Timonina~E.\,E.} see Grusho~A.\,A.&&\\[-0.1pt]
\Avtors{Timonina~E.\,E.} see Grusho~A.\,A.&&\\[-0.1pt]
\Avtors{Torshin~I.\,Yu.} On the application of a~topological approach to analysis of poorly 
formalized problems for constructing algorithms for virtual screening of quantum-mechanical 
properties\linebreak
\\[-12pt]
\hspace*{23pt}of organic molecules I:~The basics of the problem-oriented theory&1&39--45\\[-0.1pt]
\Avtors{Torshin~I.\,Yu.} On the application of a topological approach to analysis of poorly 
formalized problems for constructing algorithms for virtual screening of quantum-mechanical 
properties\linebreak
\\[-12pt]
\hspace*{23pt}of organic molecules II:~Comparison of formalism with constructions of quantum mechan-\linebreak
\\[-12pt]
\hspace*{23pt}ics and experimental approbation of the proposed algorithms&2&35--43\\[-0.1pt]
\Avtors{Vasilyev~N.\,S.} On extremum sufficient conditions in multidimensional variation 
calculus\linebreak
\\[-12pt]
\hspace*{23pt}problems&3&39--44\\[-0.1pt]
\Avtors{Vlaskina~A.\,S., Burtseva~S.\,A., Kochetkova~I.\,A., and Shorgin~S.\,Ya.} 
Controllable queuing system\linebreak
\\[-12pt]
\hspace*{23pt}with elastic traffic and signals for analyzing network 
slicing&3&90--96\\[-0.1pt]
\Avtors{Zabezhailo~M.\,I.} see Grusho A.\,A.&&\\[-0.1pt]
\Avtors{Zabezhailo~M.\,I.} see Grusho~A.\,A.&&\\[-0.1pt]
\Avtors{Zatsarinny~A.\,A.} see Grusho~A.\,A.&&\\[-0.1pt]
\end{tabular}
}
\pagebreak

\def\leftfootline{\small{\textbf{\thepage}
\hfill INFORMATIKA I EE PRIMENENIYA~--- INFORMATICS AND APPLICATIONS\ \ \ 2022\
\ \ volume~16\ \ \ issue\ 4}
}%
 \def\rightfootline{\small{INFORMATIKA I EE PRIMENENIYA~---
INFORMATICS AND APPLICATIONS\ \ \ 2022\ \ \ volume~16\ \ \ issue\ 4
\hfill \textbf{\thepage}}}

\def\leftkol{2022 AUTHOR INDEX} % ENGLISH ABSTRACTS}

\def\rightkol{2022 AUTHOR INDEX} %ENGLISH ABSTRACTS}


\noindent
{\tabcolsep=3pt
\begin{tabular}{p{395.5pt}cc}
&\textbf{Issue} & \textbf{Page}\\[6pt]
\Avtors{Zatsman~I.\,M.} Informatics' medium models of information technology: Theoretical 
foundations\linebreak
\\[-12pt]
\hspace*{23pt}for their creating&3&59--67\\
\Avtors{Zatsman I.\,M.} On the~scientific paradigm of~informatics: The~classification high 
level of~its~objects&4&73--79\\
\Avtors{Zatsman~I.\,M., Zolotarev~O.\,V., and Khakimova~A.\,K.} Medium models for 
discovering novel\linebreak
\\[-12pt]
\hspace*{23pt}terms and sentiments from texts&2&60--67\\
\Avtors{Zatsman I.\,M., Zolotarev~O.\,V., Khakimova~A.\,K., and~Dongxiao~Gu.} Model and 
technology\linebreak
\\[-12pt]
\hspace*{23pt}for discovering new terms in medical texts&4&80--86\\
\Avtors{Zeifman~A.\,I.} see Kovalev~I.\,A.&&\\
\Avtors{Zeifman~A.\,I.} see Satin~Y.\,A.&&\\
\Avtors{Zolotarev~O.\,V.} see Zatsman I.\,M.&&\\
\Avtors{Zolotarev~O.\,V.} see Zatsman~I.\,M.&&\\
\end{tabular}
}

%\thispagestyle{myheadings}
\def\leftfootline{\small{\textbf{\thepage}
\hfill INFORMATIKA I EE PRIMENENIYA~--- INFORMATICS AND APPLICATIONS\ \ \ 2022\
\ \ volume~16\ \ \ issue\ 4}
}%
 \def\rightfootline{\small{INFORMATIKA I EE PRIMENENIYA~---
INFORMATICS AND APPLICATIONS\ \ \ 2022\ \ \ volume~16\ \ \ issue\ 4
\hfill \textbf{\thepage}}}

 \label{end\stat}

\newpage

%
   \vspace*{-46pt}

\begin{center}
\vspace*{4pt}
\mbox{%

\epsfxsize=55mm %112.705
\epsfbox{zhur-2.eps}
}
%\end{center}

\vspace*{10pt} 


%   \begin{center}
\fbox{\large\textbf{Академик Юрий Иванович Журавлёв}}\\[10pt]
\textbf{\large 14.01.1935--14.01.2022}
   \end{center}


   %\vspace*{2.5mm}

   \vspace*{5mm}

   \thispagestyle{empty}

%\

%\vspace*{-12pt}
       


В январе этого года ушел из жизни главный научный сотрудник Федерального исследовательского 
центра <<Информатика и управление>> РАН, председатель Редакционного совета журнала 
<<Информатика и~её применения>> академик Юрий Иванович Журавлёв. В~его лице мировая 
наука потеряла одного из своих ярчайших представителей~--- выдающегося ученого-исследователя 
и~талантливого ученого-организатора.

Юрий Иванович родился в Воронеже в 1935~г.\ в семье ученого и врача. Среднее образование 
получил в школе №\,6 г.~Фрунзе (ныне Бишкек) Киргизской ССР. В~1952~г.\ поступил на 
ме\-ха\-ни\-ко-ма\-те\-ма\-ти\-че\-ский факультет МГУ им.\ М.\,В.~Ломоносова. В~1957~г.\ Юрий Иванович 
защищает диплом и продолжает обучение в аспирантуре Московского университета на кафедре 
вычислительной математики (возглавляемой тогда академиком С.\,Л.~Соболевым). После 
успешной защиты кандидатской диссертации (к.ф.-м.н., 1959 г., научный руководитель~--- 
А.\,А.~Ляпунов, оппоненты~--- чл.-корр.\ А.\,А.~Марков, к.ф.-м.н.\ О.\,Б.~Лупанов) и~до 
окончательного переезда в Москву в 1969~г.\ работал в Институте математики Сибирского 
отделения АН СССР, занимая в нем последовательно должности младшего научного сотрудника, 
заведующего отделом, заведующего отделением, заместителя директора по научной работе. 
В~этот период (1954--1966~гг.)\ им был опубликован цикл работ по решению задач алгебры и 
математической логики, причем полученные результаты применялись для создания эффективных 
программ для ЭВМ, конструирования схем и сетей для обработки информации. Наиболее значимый 
результат этого периода научной работы~--- обоснование нового направления исследований, 
общей теории локальных алгоритмов. В~ней были окончательно объединены топологические 
принципы и теория алгоритмов. Эта теория и легла в основу докторской диссертации Юрия 
Ивановича (д.ф.-м.н., 1965~г.)\ по еще тогда новой научной специальности <<Математическая 
кибернетика>>. Оппонировали ему как специалисты по кибернетике~--- академик 
В.\,М.~Глушков, член-корреспондент А.\,А.~Ляпунов и О.\,Б.~Лупанов, так и про\-фес\-сор-ал\-геб\-раист А.\,Д.~Тайманов. 

В 1969~г.\ Юрий Иванович переезжает в Москву и возглавляет в Вычислительном центре АН 
СССР лабораторию проблем распознавания. Впоследствии он~--- заместитель директора по 
научной работе. Научные интересы этого периода связаны с проблемами классификации или 
распознавания образов. В~1976--1978~гг.\ Юрий Иванович публикует цикл работ по ставшему 
вскоре знаменитым алгебраическому подходу к проблеме синтеза корректных алгоритмов. Эти 
работы определили современное состояние всей проблематики распознавания и многих смежных 
областей прикладной математики и информатики. В~своих основополагающих работах Юрий 
Иванович показал, что можно в явном виде строить экстремальные по качеству алгоритмы для 
решения очень широких классов плохо формализованных задач. 
{\looseness=-1

}





Научные заслуги Юрия Ивановича получили широкое признание. В~1966~г.\ он совместно с 
О.\,Б.~Лупановым и чле\-ном-кор\-рес\-пон\-ден\-том АН СССР С.\,В.~Яблонским были удостоены 
звания лауреата Ленинской премии в~об\-ласти науки и техники. В~1984~г.\ Юрий Иванович 
был избран членом-корреспондентом АН СССР (по специальности <<Информатика>>), 
а~в~1992~г.~--- академиком РАН (по той же специальности).\linebreak\vspace*{-12pt}

\pagebreak

\

\vspace*{-46pt}

\noindent
\begin{floatingfigure}{48mm}
\begin{center}
%\vspace*{6pt}
\mbox{%

\epsfxsize=46mm %112.705
\epsfbox{zhur-3.eps}
}
\end{center}
\vspace*{6pt}
\end{floatingfigure}

 \thispagestyle{empty}

\noindent
В~1986~г.\ за цикл прикладных 
работ ему и ряду его учеников была при\-суж\-де\-на премия Совета Министров СССР. Он являлся 
членом иностранных академий наук, председателем секции <<Прикладная математика
 и~информатика>> Отделения математических наук РАН, председателем экспертного совета ВАК 
России по управ\-ле\-нию и информатике, заслуженным профессором нескольких университетов, 
председателем Российской ассоциации <<Распознавание образов и обработка изображений>>, 
членом исполкома Международной ассоциации IAPR (распознавание образов и обработка 
изображений). Был награжден 8-ю орденами и медалями СССР и России.

Юрий Иванович проводил большую научно-литературную работу, являясь, в том числе, главным 
редактором международных научных журналов и членом редколлегий ряда рецензируемых 
научных журналов. 


Параллельно с активной научной деятельностью Юрий Иванович вел и преподавательскую 
работу. С~1961 по~1969~гг.~--- в Новосибирском государственном университете на кафедре 
алгебры и математической логики, которую возглавлял в то время академик А.\,И.~Мальцев. 
С~1970~г., будучи уже профессором (1967~г.),~--- в Московском физико-техническом институте 
на кафедре академика Н.\,Н.~Моисеева. В~1997~г.\ по предложению ректора МГУ им.\ 
М.\,В.~Ломоносова академика В.\,А.~Садовничего Юрий Иванович организовал на факультете 
Вычислительной математики и кибернетики новую кафедру <<Математические методы 
прогнозирования>>, которой и руководил до конца жизни. В~2008~г.\ ему была присуждена 
премия Совета Министров РФ в области образования. С~1965~г.\ Юрий Иванович периодически 
читал курсы лекций за рубежом, в университетах США, Франции, Финляндии, Швеции, Австрии, 
Польши, Болгарии, ГДР и других стран. Эта работа в существенной степени обеспечила широкое 
международное признание советской и российской науки в области дискретной математики и~распознавания образов. 

%\begin{floatingfigure}{60mm}
\begin{figure}[b]
\begin{center}
\vspace*{-6pt}
\mbox{%

\epsfxsize=112mm %90mm %112.705
\epsfbox{zhur-1.eps}
}
\end{center}
\end{figure}
%\end{floatingfigure}

Понимая важность вопроса воспитания подрастающего поколения для развития науки в стране, 
Юрий Иванович вскоре после защиты первой диссертации включился в работу по подготовке 
научных кадров. Им создана большая научная школа: под руководством Юрия Ивановича 
защищены более 100~кандидатских диссертаций по всевозможным разделам естествознания 
(математике, информатике, медицине, технике, экономике, геологии), не один десяток докторов 
наук. Он воспитал академиков и членов-корреспондентов РАН и академий государств СНГ. 
С~большим вниманием и участием Юрий Иванович относился к развитию научных школ страны 
в~об\-ласти обработки изображений, распознавания образов и компьютерной оптики. 

Для всех коллег и учеников Юрия Ивановича он останется примером замечательного человека, 
та\-лант\-ли\-во\-го педагога и выдающегося, преданного служению науке ученого. 


%\def\stat{cont}
{%\hrule\par
%\vskip 7pt % 7pt
\raggedleft\Large \bf%\baselineskip=3.2ex
А\,В\,Т\,О\,Р\,С\,К\,И\,Й\ \ У\,К\,А\,З\,А\,Т\,Е\,Л\,Ь\ \ З\,А\ \ 2\,0\,1\,0 г. \vskip 17pt
    \hrule
    \par
\vskip 21pt plus 6pt minus 3pt }

\label{st\stat}

\def\tit{\ }

\def\aut{\ }
\def\auf{\ }

\def\leftkol{\ } % ENGLISH ABSTRACTS}

\def\rightkol{\ } %АВТОРСКИЙ УКАЗАТЕЛЬ ЗА 2010 г.} %ENGLISH ABSTRACTS}

\titele{\tit}{\aut}{\auf}{\leftkol}{\rightkol}

\vspace*{-12pt}

{\tabcolsep=3pt
\begin{tabular}{p{388pt}rr}
&\textbf{Выпуск} & \textbf{Стр.}\\[6pt]
\hangindent=23pt\noindent\textbf{Арутюнян~А.\,Р.} Моделирование влияния деформаций отпечатков пальцев на 
точность\linebreak
\vspace*{-12pt}\\
\hspace*{23pt}дактилоскопической идентификации$\dotfill$&1&51\\
\hangindent=23pt\noindent\textbf{Архипов~О.\,П., Зыкова~З.\,П.} Интеграция гетерогенной информации о цветных 
пикселях\linebreak
\vspace*{-12pt}\\
\hspace*{23pt}и их цветовосприятии$\dotfill$&4&15\\
\hangindent=23pt\noindent\textbf{Баранов~С.\,И., Френкель~С.\,Л., Захаров~В.\,Н.} Полуформальная верификация 
цифрового устройства с конвейером, основанная на использовании алгоритмических машин\linebreak
\vspace*{-12pt}\\
\hspace*{23pt}состояния$\dotfill$&4&49\\
\textbf{Бекетова~И.\,В.} см.~Каратеев~С.\,Л.&&\\
\textbf{Белоусов~В.\,В.} см.~Синицын~И.\,Н.&&\\
\hangindent=23pt\noindent\textbf{Бенинг~В.\,Е., Королев~Р.\,А.} О предельном поведении мощностей критериев в 
случае\linebreak
\vspace*{-12pt}\\
\hspace*{23pt}распределения Лапласа$\dotfill$&2&63\\
\hangindent=23pt\noindent\textbf{Бенинг~В.\,Е., Сипина~А.\,В.} Асимптотическое разложение для мощности 
критерия,\linebreak
\vspace*{-12pt}\\
\hspace*{23pt}основанного на выборочной медиане, в случае распределения Лапласа$\dotfill$&1&18\\
\textbf{Бондаренко~А.\,В.} см.~Каратеев~С.\,Л.&&\\
\hangindent=23pt\noindent\textbf{Бородина~А.\,В., Морозов~Е.\,В.} Об оценивании асимптотики вероятности 
большого\linebreak
\vspace*{-12pt}\\
\hspace*{23pt}уклонения стационарной регенеративной очереди с одним прибором$\dotfill$&3&29\\
\hangindent=23pt\noindent\textbf{Бунтман~Н.\,В., Минель~Ж.-Л., Ле~Пезан~Д., Зацман~И.\,М.} Типология и 
компьютерное\linebreak
\vspace*{-12pt}\\
\hspace*{23pt}моделирование трудностей перевода$\dotfill$&3&77\\
\textbf{Визильтер~Ю.\,В.} см.~Каратеев~С.\,Л.&&\\
\hangindent=23pt\noindent\textbf{Гавриленко~С.\,В.} Оценки скорости сходимости распределений случайных сумм с 
безгранично делимыми индексами к нормальному закону$\dotfill$&4&81\\
\hangindent=23pt\noindent\textbf{Григорьева~М.\,Е., Шевцова~И.\,Г.} Уточнение неравенства 
Каца--Берри--Эссеена$\dotfill$&2&75\\
\hangindent=23pt\noindent\textbf{Грушо~А.\,А., Грушо~Н.\,А., Тимонина~Е.\,Е.} Поиск конфликтов в политиках 
безопасности: модель случайных графов$\dotfill$&3&38\\
\textbf{Грушо~Н.\,А.} см.~Грушо~А.\,А.&&\\
\hangindent=23pt\noindent\textbf{Гудков~В.\,Ю.} Математические модели изображения отпечатка пальца на основе 
описания линий$\dotfill$&1&58\\
\textbf{Гуртов~А.\,В.} см.~Лукьяненко~А.\,С.&&\\
\textbf{Желтов~С.\,Ю.} см.~Каратеев~С.\,Л.&&\\
\hangindent=23pt\noindent\textbf{Захаров~А.\,А., Серебряков~В.\,А.} Система управления электронной библиотекой 
LibMeta$\dotfill$&4&2\\
\textbf{Захаров~В.\,Н.} см.~Баранов~С.\,И.&&\\
\textbf{Захарова~Т.\,В.} см.~Матвеева~С.\,С.&&\\
\hangindent=23pt\noindent\textbf{Зацаринный~А.\,А., Чупраков~К.\,Г.} Некоторые аспекты выбора технологии для 
постро-\linebreak
\vspace*{-12pt}\\
\hspace*{23pt}ения систем отображения информации ситуационного центра$\dotfill$&3&59\\
\textbf{Зацман~И.\,М.} см.~Бунтман~Н.\,В.&&\\
\hangindent=23pt\noindent\textbf{Зейфман~А.\,И., Коротышева~А.\,В., Сатин~Я.\,А., Шоргин~С.\,Я.} Об 
устойчивости нестаци-\linebreak
\vspace*{-12pt}\\
\hspace*{23pt}онарных систем обслуживания с катастрофами$\dotfill$&3&9\\
\textbf{Зыкова~З.\,П.} см.~Архипов~О.\,П.&&\\
\hangindent=23pt\noindent\textbf{Илюшин~Г.\,Я., Соколов~И.\,А.} Организация управляемого доступа пользователей 
к\linebreak
\vspace*{-12pt}\\
\hspace*{23pt}разнородным ведомственным информационным ресурсам$\dotfill$&1&24\\
\hangindent=23pt\noindent\textbf{Кавагучи~Ю., Ульянов~В.\,В., Фуджикоши~Я.} Приближения для статистик, 
описывающих\linebreak
\vspace*{-12pt}\\
\hspace*{23pt}геометрические свойства данных большой размерности, с оценками 
ошибок$\dotfill$&1&12\\
\hangindent=23pt\noindent\textbf{Каратеев~С.\,Л., Бекетова~И.\,В., Ососков~М.\,В., Князь~В.\,А., 
Визильтер~Ю.\,В., Бондаренко~А.\,В., Желтов~С.\,Ю.} Автоматизированный контроль 
качества цифровых\linebreak
\vspace*{-12pt}\\
\hspace*{23pt}изображений для персональных документов$\dotfill$&1&65\\
\end{tabular}
}

\pagebreak

\def\leftkol{АВТОРСКИЙ УКАЗАТЕЛЬ ЗА 2010 г.} % ENGLISH ABSTRACTS}

\def\rightkol{АВТОРСКИЙ УКАЗАТЕЛЬ ЗА 2010 г.} %ENGLISH ABSTRACTS}

{\tabcolsep=3pt
\begin{tabular}{p{388pt}rr}
&\textbf{Выпуск} & \textbf{Стр.}\\[3pt]
\hangindent=23pt\noindent\textbf{Козеренко~Е.\,Б.} Лингвистические фильтры в статистических моделях машинного\linebreak
\vspace*{-12pt}\\
\hspace*{23pt}перевода$\dotfill$&2&83\\
\hangindent=23pt\noindent\textbf{Козеренко~Е.\,Б., Кузнецов~И.\,П.} Когнитивно-лингвистические представления в 
систе-\linebreak
\vspace*{-12pt}\\
\hspace*{23pt}мах обработки текстов$\dotfill$&3&69\\
\textbf{Князь~В.\,А.} см.~Каратеев~С.\,Л.&&\\
\hangindent=23pt\noindent\textbf{Колесников~А.\,В., Солдатов~С.\,А.} Алгоритм координации для гибридной 
интеллектуальной системы решения сложной задачи оперативно-производственного\linebreak
\vspace*{-12pt}\\
\hspace*{23pt}планирования$\dotfill$&4&61\\
\hangindent=23pt\noindent\textbf{Коновалов~М.\,Г.} О планировании потоков в системах вычислительных 
ресурсов$\dotfill$&2&3\\
\textbf{Конушин~А.\,С.} см.~Конушин~В.\,С.&&\\
\hangindent=23pt\noindent\textbf{Конушин~В.\,С., Кривовязь~Г.\,Р., Конушин~А.\,С.} Алгоритм распознавания людей 
в видео-\linebreak
\vspace*{-12pt}\\
\hspace*{23pt}последовательности по одежде$\dotfill$&1&74\\
\textbf{Корепанов~Э.\, Р.} см.~Синицын~И.\,Н.&&\\
\textbf{Королев~В.\,Ю.} см.~Соколов~И.\,А.&&\\
\textbf{Королев~Р.\,А.} см.~Бенинг~В.\,Е.&&\\
\textbf{Коротышева~А.\,В.} см.~Зейфман~А.\,И.&&\\
\hangindent=23pt\noindent\textbf{Кривенко~М.\,П.} Непараметрическое оценивание элементов байесовского 
клас\-си-\linebreak
\vspace*{-12pt}\\
\hspace*{23pt}фикатора$\dotfill$&2&13\\
\textbf{Кривовязь~Г.\,Р.} см.~Конушин~В.\,С.&&\\
\textbf{Крылов~А.\,С.} см.~Павельева~Е.\,А.&&\\
\hangindent=23pt\noindent\textbf{Крылов~В.\,А.} Моделирование и классификация многоканальных дистанционных\linebreak
\vspace*{-12pt}\\
\hspace*{23pt}изображений с использованием копул$\dotfill$&4&34\\
\hangindent=23pt\noindent\textbf{Крючин~О.\,В.} Разработка параллельных эвристических алгоритмов подбора 
весовых\linebreak
\vspace*{-12pt}\\
\hspace*{23pt}коэффициентов искусственной нейтронной сети$\dotfill$&2&53\\
\hangindent=23pt\noindent\textbf{Кудрявцев~А.\,А., Шоргин~С.\,Я.} Байесовские модели массового обслуживания и 
надеж-\linebreak
\vspace*{-12pt}\\
\hspace*{23pt}ности: характеристики среднего числа заявок в системе $M\vert M \vert 1\vert 
\infty$$\dotfill$&3&16\\
\hangindent=23pt\noindent\textbf{Кузнецов~А.\,А.} Связь между временными и структурно-топологическими 
характери-\linebreak
\vspace*{-12pt}\\
\hspace*{23pt}стиками диаграмм ритма сердца здоровых людей$\dotfill$&4&39\\
\textbf{Кузнецов~И.\,П.} см.~Козеренко~Е.\,Б.&&\\
\textbf{Ле~Пезан~Д.} см.~Бунтман~Н.\,В.&&\\
\hangindent=23pt\noindent\textbf{Лукьяненко~А.\,С., Морозов~Е.\,В., Гуртов~А.\,В.} Анализ сетевого протокола с общей 
функ-\linebreak
\vspace*{-12pt}\\
\hspace*{23pt}цией расширения окна передачи сообщения при конфликтах$\dotfill$&2&46\\
\hangindent=23pt\noindent\textbf{Лямин~О.\,О.} О предельном поведении мощностей критериев в случае обобщенного\linebreak
\vspace*{-12pt}\\
\hspace*{23pt}распределения Лапласа$\dotfill$&3&47\\
\hangindent=23pt\noindent\textbf{Маркин~А.\,В., Шестаков~О.\,В.} Асимптотики оценки риска при пороговой 
обработке\linebreak
\vspace*{-12pt}\\
\hspace*{23pt}вейвлет-вейглет коэффициентов в задаче томографии$\dotfill$&2&36\\
\hangindent=23pt\noindent\textbf{Матвеева~С.\,С., Захарова~Т.\,В.} Сети массового обслуживания с наименьшей 
длиной\linebreak
\vspace*{-12pt}\\
\hspace*{23pt}очереди$\dotfill$&3&22\\
\hangindent=23pt\noindent\textbf{Матюшенко~С.\,И.} Стационарные характеристики двухканальной системы 
обслужива-\linebreak
\vspace*{-12pt}\\
\hspace*{23pt}ния с переупорядочиванием заявок и распределениями фазового типа$\dotfill$&4&68\\
\textbf{Минель~Ж.-Л.} см.~Бунтман~Н.\,В.&&\\
\textbf{Морозов~Е.\,В.} см.~Бородина~А.\,В.&&\\
\textbf{Морозов~Е.\,В.} см.~Лукьяненко~А.\,С.&&\\
\textbf{Ососков~М.\,В.} см.~Каратеев~С.\,Л.&&\\
\hangindent=23pt\noindent\textbf{Павельева~Е.\,А., Крылов~А.\,С.} Поиск и анализ ключевых точек радужной 
оболочки\linebreak
\vspace*{-12pt}\\
\hspace*{23pt}глаза методом преобразования Эрмита$\dotfill$&1&79\\
\textbf{Печинкин~А.\,В.} см.~Френкель~С.\,Л.,&&\\
\hangindent=23pt\noindent\textbf{Протасов~В.\,И.} Составление субъективного портрета с использованием 
эволюционно-\linebreak
\vspace*{-12pt}\\
\hspace*{23pt}го морфинга и квалиметрия метода$\dotfill$&1&83\\
\hangindent=23pt\noindent\textbf{Рудаков~К.\,В., Торшин~И.\,Ю.} Вопросы разрешимости задачи распознавания 
вторичной\linebreak
\vspace*{-12pt}\\
\hspace*{23pt}структуры белка$\dotfill$&2&25\\
\textbf{Сатин~Я.\,А.} см.~Зейфман~А.\,И.&&\\
\hangindent=23pt\noindent\textbf{Сейфуль-Мулюков~Р.\,Б.} Нефть как носитель информации о своем 
происхождении,\linebreak
\vspace*{-12pt}\\
\hspace*{23pt}структуре и эволюции$\dotfill$&1&41\\
\end{tabular}
}

{\tabcolsep=3pt
\begin{tabular}{p{388pt}rr}
&\textbf{Выпуск} & \textbf{Стр.}\\[6pt]
\textbf{Семендяев~Н.\,Н.} см.~Синицын~И.\,Н.&&\\
\textbf{Серебряков~В.\,А.} см.~Захаров~А.\,А.&&\\
\textbf{Синицын~В.\,И.} см.~Синицын~И.\,Н.&&\\
\hangindent=23pt\noindent\textbf{Синицын~И.\,Н., Синицын~В.\,И., Корепанов~Э.\, Р., Белоусов~В.\,В., 
Семендяев~Н.\,Н.} Оперативное построение информационных моделей движения полюса 
Земли\linebreak
\vspace*{-12pt}\\
\hspace*{23pt}методами линейных и линеаризованных фильтров$\dotfill$&1&2\\
\textbf{Сипина~А.\,В.} см.~Бенинг~В.\,Е.&&\\
\hangindent=23pt\noindent\textbf{Соколов~И.\,А.} О работах заслуженного деятеля науки Российской Федерации 
И.\,Н.~Синицына в области информационных технологий и автоматизации (к 70-летию\linebreak
\vspace*{-12pt}\\
\hspace*{23pt}со дня рождения)$\dotfill$&3&84\\
\textbf{Соколов~И.\,А.} см.~Илюшин~Г.\,Я.&&\\
\hangindent=23pt\noindent\textbf{Соколов~И.\,А., Королев~В.\,Ю.} Предисловие$\dotfill$&2&2\\
\textbf{Солдатов~С.\,А.} см.~Колесников~А.\,В.&&\\
\hangindent=23pt\noindent\textbf{Степанов~С.\,Ю.} Использование координатного метода фрагментации 
коммутаторной\linebreak
\vspace*{-12pt}\\
\hspace*{23pt}нейронной сети для сокращения трафика$\dotfill$&2&57\\
\textbf{Тимонина~Е.\,Е.} см.~Грушо~А.\,А.&&\\
\textbf{Торшин~И.\,Ю.} см.~Рудаков~К.\,В.&&\\
\textbf{Ульянов~В.\,В.} см.~Кавагучи~Ю.&&\\
\textbf{Фазекаш~И.} см.~Чупрунов~А.\,Н.&&\\
\textbf{Френкель~С.\,Л.} см.~Баранов~С.\,И.&&\\
\hangindent=23pt\noindent\textbf{Френкель~С.\,Л., Печинкин~А.\,В.} Оценка времени самовосстановления в 
цифровых\linebreak
\vspace*{-12pt}\\
\hspace*{23pt}системах после сбоев, вызываемых переходными помехами$\dotfill$&3&2\\
\textbf{Фуджикоши~Я.} см.~Кавагучи~Ю.&&\\
\hangindent=23pt\noindent\textbf{Цискаридзе~А.\,К.} Математическая модель и метод восстановления позы человека 
по\linebreak
\vspace*{-12pt}\\
\hspace*{23pt}стереопаре силуэтных изображений$\dotfill$&4&27\\
\hangindent=23pt\noindent\textbf{Чупраков~К.\,Г.} К вопросу о размещении коллективных средств отображения в 
ситуа-\linebreak
\vspace*{-12pt}\\
\hspace*{23pt}ционном зале с заданными параметрами$\dotfill$&4&89\\
\textbf{Чупраков~К.\,Г.} см.~Зацаринный~А.\,А.&&\\
\hangindent=23pt\noindent\textbf{Чупрунов~А.\,Н., Фазекаш~И.} Законы повторного логарифма для числа 
безошибочных\linebreak
\vspace*{-12pt}\\
\hspace*{23pt}блоков при помехоустойчивом кодировании$\dotfill$&3&42\\
\textbf{Шевцова~И.\,Г.} см.~Григорьева~М.\,Е.&&\\
\hangindent=23pt\noindent\textbf{Шестаков~О.\,В.} Аппроксимация распределения оценки риска пороговой 
обработки вейвлет-коэффициентов нормальным распределением при использовании 
выбо-\linebreak
\vspace*{-12pt}\\
\hspace*{23pt}рочной дисперсии$\dotfill$&4&73\\
\textbf{Шестаков~О.\,В.} см.~Маркин~А.\,В.&&\\
\textbf{Шоргин~С.\,Я.} см.~Зейфман~А.\,И.&&\\
\textbf{Шоргин~С.\,Я.} см.~Кудрявцев~А.\,А.&&\\
\end{tabular}
}

%\thispagestyle{myheadings}
\def\leftfootline{\small{\textbf{\thepage}
\hfill ИНФОРМАТИКА И ЕЁ ПРИМЕНЕНИЯ\ \ \ том~4\ \ \ выпуск~4\ \ \ 2010}
}%
 \def\rightfootline{\small{ИНФОРМАТИКА И ЕЁ ПРИМЕНЕНИЯ\ \ \ том~4\ \ \ выпуск~4\ \ \ 2010
 \hfill \textbf{\thepage}}}
 \label{end\stat}
%
%Том 10 Выпуск 1-4 Год 2016

\def\stat{cont-e}
{%\hrule\par
%\vskip 7pt % 7pt
\raggedleft\Large \bf%\baselineskip=3.2ex
2\,0\,1\,6\ \ A\,U\,T\,H\,O\,R\ \ I\,N\,D\,E\,X \vskip 17pt
 \hrule
 \par
\vskip 21pt plus 6pt minus 3pt }

\label{st\stat}

\def\tit{\ }

\def\aut{\ }
\def\auf{\ }

\def\leftkol{\ } %2016 AUTHOR INDEX} % ENGLISH ABSTRACTS}

\def\rightkol{\ } %2016 AUTHOR INDEX} %ENGLISH ABSTRACTS}

\titele{\tit}{\aut}{\auf}{\leftkol}{\rightkol}

\def\leftfootline{\small{\textbf{\thepage}
\hfill INFORMATIKA I EE PRIMENENIYA~--- INFORMATICS AND APPLICATIONS\ \ \ 2016\
\ \ volume~10\ \ \ issue\ 4}
}%
 \def\rightfootline{\small{INFORMATIKA I EE PRIMENENIYA~--- INFORMATICS AND APPLICATIONS\ \ \ 2016\ \ \ volume~10\ \ \ issue\ 4
\hfill \textbf{\thepage}}}

\vspace*{-12pt}
\vspace*{-18pt}

{\tabcolsep=2.8pt
\begin{tabular}{p{382pt}cc}
&\textbf{Issue} & \textbf{Page}\\[6pt]
\Avtors{Agalarov~M.\,Ya.} see~Agalarov~Ya.\,M.&&\\
\Avtors{Agalarov~Ya.\,M., Agalarov~M.\,Ya., and
Shorgin~V.\,S.} About the optimal threshold of queue\linebreak
\\[-12pt]
\hspace*{23pt}length in a~particular problem of profit maximization
in the $M/G/1$ queuing system&2&70--79\\
\Avtors{Alexeyevsky~D.\,A.} BioNLP ontology extraction from 
a~restricted language corpus with\linebreak
\\[-12pt]
\hspace*{23pt}context-free grammars&1&119--128\\
\Avtors{Andreev~S.\,D.} see~Gaidamaka~Yu.\,V.&&\\
\Avtors{Andreev~S.\,D.} see~Ometov~A.\,Ya.&&\\
\Avtors{Arkhipov~O.\,P., Arkhipov~P.\,O., and Sidorkin~I.\,I.} The
option to create a~local coordinate\linebreak
\\[-12pt]
\hspace*{23pt}system for synchronization of selected images&3&91--97\\
\Avtors{Arkhipov~P.\,O.} see~Arkhipov~O.\,P.&&\\
\Avtors{Belousov~V.\,V.} see~Shnurkov~P.\,V.&&\\
\Avtors{Belousov~V.\,V.} see~Shnurkov~P.\,V.&&\\
\Avtors{Bening~V.\,E.} Calculation of~the~asymptotic deficiency
of~some statistical procedures based\linebreak
\\[-12pt]
\hspace*{23pt}on~samples with~random sizes&4&34--45\\
\Avtors{Borisov~A.\,V., Bosov~A.\,V., and Miller~G.\,B.} Modeling and
monitoring of VoIP connection&2&\hphantom{1}2--13\\
\Avtors{Bosov~A.\,V.} see~Borisov~A.\,V.&&\\
\Avtors{Briukhov~D.\,O.} see~Stupnikov~S.\,A.&&\\
\Avtors{Callaos~N.\,K.\ and Seyful-Mulyukov~R.\,B.} Complexity and
its information content&1&129--139\\
\Avtors{Chertok~A.\,V., Kadaner~A.\,I., Khazeeva~G.\,T., and
Sokolov~I.\,A.} Regime switching detection\linebreak
\\[-12pt]
\hspace*{23pt}for~the~Levy driven
Ornstein--Uhlenbeck process using CUSUM methods&4&46--56\\
\Avtors{Chichagov~V.\,V.} Asymptotic expansions of mean absolute
error of uniformly minimum variance unbiased and maximum likelihood
estimators on the one-parameter exponential\linebreak
\\[-12pt]
\hspace*{23pt}family model of lattice distributions&3&66--76\\
\Avtors{Danishevsky~V.\,I.} see~Kolesnikov A.\,V.&&\\
\Avtors{Fazliev~A.\,Z.} see~Kalinichenko~L.\,A.&&\\
\Avtors{Fedoseev~A.\,A.} What is behind the concept of ``knowledge in
small packages''&3&105--110\\
\Avtors{Gaidamaka~Yu.\,V., Andreev~S.\,D., Sopin~E.\,S.,
Samouylov~K.\,E., and Shorgin~S.\,Ya.} Interference analysis
of~the~device-to-device communications model with~regard to~a~signal\linebreak
\\[-12pt]
\hspace*{23pt}propagation environment&4&\hphantom{1}2--10\\
\Avtors{Gasilov~A.\,V.} see~Yakovlev~O.\,A.&&\\
\Avtors{Goncharov~A.\,V.\ and Strijov~V.\,V.} Metric time series
classification using weighted dynamic\linebreak
\\[-12pt]
\hspace*{23pt}warping relative to centroids of classes&2&36--47\\
\Avtors{Gordov~E.\,P.} see~Kalinichenko~L.\,A.&&\\
\Avtors{Gorshenin~A.\,K.} Concept of online service for stochastic
modeling of real processes&1&72--81\\
\Avtors{Gorshenin~A.\,K.} see~Shnurkov~P.\,V.&&\\
\Avtors{Gorshenin~A.\,K.} see~Shnurkov~P.\,V.&&\\
\Avtors{Grusho~A.\,A., Grusho~N.\,A., Zabezhailo~M.\,I., and
Timonina~E.\,E.} Integration of statistical and\linebreak
\\[-12pt]
\hspace*{23pt}deterministic methods for
analysis of information security&3&2--8\\
\Avtors{Grusho~A.\,A., Zabezhailo~M.\,I., and Zatsarinny~A.\,A.} On
the advanced procedure to reduce\linebreak
\\[-12pt]
\hspace*{23pt}calculation of Galois closures&4&\hphantom{1}96--104\\
\Avtors{Grusho~N.\,A.} see~Grusho~A.\,A.&&\\
\Avtors{Havanskov~V.\,A.} see~Minin~V.\,A.&&\\
\Avtors{Inkova~O.\,Yu.} see~Zatsman~I.\,M.&&\\
\Avtors{Isachenko~R.\,V.\ and Strijov~V.\,V.} Metric learning in
multiclass time series classification\linebreak
\\[-12pt]
\hspace*{23pt}problem&2&48--57\\
\end{tabular}
}
\pagebreak

\def\leftfootline{\small{\textbf{\thepage}
\hfill INFORMATIKA I EE PRIMENENIYA~--- INFORMATICS AND APPLICATIONS\ \ \ 2016\
\ \ volume~10\ \ \ issue\ 4}
}%
 \def\rightfootline{\small{INFORMATIKA I EE PRIMENENIYA~---
INFORMATICS AND APPLICATIONS\ \ \ 2016\ \ \ volume~10\ \ \ issue\ 4
\hfill \textbf{\thepage}}}

\def\leftkol{2016 AUTHOR INDEX} % ENGLISH ABSTRACTS}

\def\rightkol{2016 AUTHOR INDEX} %ENGLISH ABSTRACTS}


{\tabcolsep=2.83pt
\begin{tabular}{p{382pt}cc}
&\textbf{Issue} & \textbf{Page}\\[6pt]
\Avtors{Kadaner~A.\,I.} see~Chertok~A.\,V.&&\\[.255pt]
\Avtors{Kalinichenko~L.\,A., Volnova~A.\,A., Gordov~E.\,P.,
Kiselyova~N.\,N., Kovaleva~D.\,A., Malkov~O.\,Yu., Okladnikov~I.\,G.,
Podkolodnyy~N.\,L., Pozanenko~A.\,S., Ponomareva~N.\,V.,
Stupnikov~S.\,A.,} \textbf{and Fazliev~A.\,Z.} Data access challenges for data
intensive\linebreak
\\[-12pt]
\hspace*{23pt}research in Russia&1& 2--22\\[.255pt]
\Avtors{Karasikov~M.\,E.\ and Strijov~V.\,V.} Feature-based
time-series classification&4&121--131\\[.255pt]
\Avtors{Khazeeva~G.\,T.} see~Chertok~A.\,V.&&\\[.255pt]
\Avtors{Khokhlov~Yu.\,S.} Multivariate fractional Levy motion and its
applications&2&\hphantom{1}98--106\\[.255pt]
\Avtors{Kirikov~I.\,A., Kolesnikov~A.\,V., Listopad~S.\,V., and
Rumovskaya~S.\,B.} Fine-grained hybrid\linebreak
\\[-12pt]
\hspace*{23pt}intelligent systems. Part 2:
Bidirectional hybridization&1&\hphantom{1}96--105\\[.255pt]
\Avtors{Kirikov~I.\,A., Kolesnikov~A.\,V., Listopad~S.\,V., and
Rumovskaya~S.\,B.} ``Virtual council''~---\linebreak
\\[-12pt]
\hspace*{23pt}source environment
supporting complex diagnostic decision making&3&81--90\\[.255pt]
\Avtors{Kiselyova~N.\,N.} see~Kalinichenko~L.\,A.&&\\[.255pt]
\Avtors{Kolesnikov A.\,V., Listopad~S.\,V., Rumovskaya~S.\,B., and
Danishevsky~V.\,I.} Informal axiomatic\linebreak
\\[-12pt]
\hspace*{23pt}theory of~the~role visual models&4&114--120\\[.255pt]
\Avtors{Kolesnikov~A.\,V.} see~Kirikov~I.\,A.&&\\[.255pt]
\Avtors{Kolesnikov~A.\,V.} see~Kirikov~I.\,A.&&\\[.255pt]
\Avtors{Kolin~K.\,K.} Humanitarian aspects of information
security&3&111--121\\[.255pt]
\Avtors{Konovalov~M.\,G.\ and Razumchik~R.\,V.} Dispatching
to~two parallel nonobservable queues using\linebreak
\\[-12pt]
\hspace*{23pt}only static
information&4&57--67\\[.255pt]
\Avtors{Korchagin~A.\,Yu.} see~Korolev~V.\,Yu.&&\\[.255pt]
\Avtors{Korchagin~A.\,Yu.} see~Korolev~V.\,Yu.&&\\[.255pt]
\Avtors{Korepanov~E.\,R.} see~Sinitsyn~I.\,N.&&\\[.255pt]
\Avtors{Korepanov~E.\,R.} see~Sinitsyn~I.\,N.&&\\[.255pt]
\Avtors{Korolev~V.\,Yu., Korchagin~A.\,Yu., and Zeifman~A.\,I.} The
Poisson theorem for Bernoulli trials\linebreak
\\[-12pt]
\hspace*{23pt}with~a~random probability
of~success and~a~discrete analog of~the~Weibull distribution&4&11--20\\[.255pt]
\Avtors{Korolev~V.\,Yu., Zeifman~A.\,I., and Korchagin~A.\,Yu.}
Asymmetric Linnik distributions as~limit\linebreak
\\[-12pt]
\hspace*{23pt}laws for~random sums
of~independent random variables with~finite variances&4&21--33\\[.255pt]
\Avtors{Koucheryavy~E.\,A.} see~Ometov~A.\,Ya.&&\\[.255pt]
\Avtors{Kovaleva~D.\,A.} see~Kalinichenko~L.\,A.&&\\[.255pt]
\Avtors{Kovalyov~S.\,P.} Metaprogramming to increase
manufacturability of large-scale software-\linebreak
\\[-12pt]
\hspace*{23pt}intensive systems&1&56--66\\[.255pt]
\Avtors{Krivenko~M.\,P.} Significance tests of feature selection for
classification&3&32--40\\[.255pt]
\Avtors{Kruzhkov~M.\,G.} see~Zalizniak~Anna~A.&&\\[.255pt]
\Avtors{Kruzhkov~M.\,G.} see~Zatsman~I.\,M.&&\\[.255pt]
\Avtors{Kudryavtsev~A.\,A.} Bayesian queueing and reliability models:
\textit{A~priori} distributions with\linebreak
\\[-12pt]
\hspace*{23pt}compact support&1&67--71\\[.255pt]
\Avtors{Kudryavtsev~A.\,A.} Characteristics dependent on the balance
coefficient in Bayesian models\linebreak
\\[-12pt]
\hspace*{23pt}with compact support of \textit{a priori}
distributions&3&77--80\\[.255pt]
\Avtors{Kudryavtsev~A.\,A.\ and Palionnaia~S.\,I.} Bayesian recurrent
model of reliability growth:\linebreak
\\[-12pt]
\hspace*{23pt}Parabolic distribution of parameters&2&80--83\\[.255pt]
\Avtors{Kudryavtsev~A.\,A.\ and Titova~A.\,I.} Bayesian queuing
and~reliability models: Degenerate-\linebreak
\\[-12pt]
\hspace*{23pt}Weibull case&4&68--71\\[.255pt]
\Avtors{Leontyev~N.\,D.\ and Ushakov~V.\,G.} Analysis of a queueing
system with autoregressive arrivals\linebreak
\\[-12pt]
\hspace*{23pt}and nonpreemptive priority&3&15--22\\[.255pt]
\Avtors{Listopad~S.\,V.} see~Kirikov~I.\,A.&&\\[.255pt]
\Avtors{Listopad~S.\,V.} see~Kirikov~I.\,A.&&\\[.255pt]
\Avtors{Listopad~S.\,V.} see~Kolesnikov A.\,V.&&\\[.255pt]
\Avtors{Malkov~O.\,Yu.} see~Kalinichenko~L.\,A.&&\\[.255pt]
\Avtors{Markov~A.\,S., Monakhov~M.\,M., and
Ulyanov~V.\,V.} Generalized Cornish--Fisher expansions\linebreak
\\[-12pt]
\hspace*{23pt}for distributions of statistics based on samples
of random size&2&84--91\\[.255pt]
\Avtors{Melnikov~A.\,K.\ and Ronzhin~A.\,F.} Generalized statistical
method of~text analysis based\linebreak
\\[-12pt]
\hspace*{23pt}on~calculation of~probability distributions
of~statistical values&4&89--95\\
\end{tabular}
}
\pagebreak

\def\leftfootline{\small{\textbf{\thepage}
\hfill INFORMATIKA I EE PRIMENENIYA~--- INFORMATICS AND APPLICATIONS\ \ \ 2016\
\ \ volume~10\ \ \ issue\ 4}
}%
 \def\rightfootline{\small{INFORMATIKA I EE PRIMENENIYA~---
INFORMATICS AND APPLICATIONS\ \ \ 2016\ \ \ volume~10\ \ \ issue\ 4
\hfill \textbf{\thepage}}}

\def\leftkol{2016 AUTHOR INDEX} % ENGLISH ABSTRACTS}

\def\rightkol{2016 AUTHOR INDEX} %ENGLISH ABSTRACTS}


{\tabcolsep=3pt
\begin{tabular}{p{381pt}cc}
&\textbf{Issue} & \textbf{Page}\\[6pt]
\Avtors{Meykhanadzhyan~L.\,A.} Stationary characteristics of the finite
capacity queueing system with\linebreak
\\[-12pt]
\hspace*{23pt}inverse service order and generalized
probabilistic priority&2&123--131\\[.23pt]
\Avtors{Miller~G.\,B.} see~Borisov~A.\,V.&&\\[.23pt]
\Avtors{Minin~V.\,A., Zatsman~I.\,M., Havanskov~V.\,A., and
Shubnikov~S.\,K.} Intensity of citation of scientific publications in
inventions on information and computer technologies patented\linebreak
\\[-12pt]
\hspace*{23pt}in Russia by domestic and foreign applicants&2&107--122\\[.23pt]
\Avtors{Monakhov~M.\,M.} see~Markov~A.\,S.&&\\[.23pt]
\Avtors{Naumov~V.\,A.\ and Samouylov~K.\,E.} On relationship
between queuing systems with resources\linebreak
\\[-12pt]
\hspace*{23pt}and Erlang networks&3&\hphantom{1}9--14\\[.23pt]
\Avtors{Okladnikov~I.\,G.} see~Kalinichenko~L.\,A.&&\\[.23pt]
\Avtors{Ometov~A.\,Ya., Andreev~S.\,D., Turlikov~A.\,M., and
Koucheryavy~E.\,A.} Performance analysis of\linebreak
\\[-12pt]
\hspace*{23pt}a wireless data
aggregation system with contention for contemporary sensor
networks&3&23--31\\[.23pt]
\Avtors{Palionnaia~S.\,I.} see~Kudryavtsev~A.\,A.&&\\[.23pt]
\Avtors{Podkolodnyy~N.\,L.} see~Kalinichenko~L.\,A.&&\\[.23pt]
\Avtors{Ponomareva~N.\,V.} see~Kalinichenko~L.\,A.&&\\[.23pt]
\Avtors{Popkova~N.\,A.} see~Zatsman~I.\,M.&&\\[.23pt]
\Avtors{Pozanenko~A.\,S.} see~Kalinichenko~L.\,A.&&\\[.23pt]
\Avtors{Razumchik~R.\,V.} see~Konovalov~M.\,G.&&\\[.23pt]
\Avtors{Ronzhin~A.\,F.} see~Melnikov~A.\,K.&&\\[.23pt]
\Avtors{Rumovskaya~S.\,B.} see~Kirikov~I.\,A.&&\\[.23pt]
\Avtors{Rumovskaya~S.\,B.} see~Kirikov~I.\,A.&&\\[.23pt]
\Avtors{Rumovskaya~S.\,B.} see~Kolesnikov A.\,V.&&\\[.23pt]
\Avtors{Samouylov~K.\,E.} see~Gaidamaka~Yu.\,V.&&\\[.23pt]
\Avtors{Samouylov~K.\,E.} see~Naumov~V.\,A.&&\\[.23pt]
\Avtors{Serebryanskii~S.\,M.} see~Tyrsin~A.\,N.&&\\[.23pt]
\Avtors{Seyful-Mulyukov~R.\,B.} see~Callaos~N.\,K.&&\\[.23pt]
\Avtors{Shestakov~O.\,V.} Statistical properties of the denoising method
based on the stabilized hard\linebreak
\\[-12pt]
\hspace*{23pt}thresholding&2&65--69\\[.23pt]
\Avtors{Shestakov~O.\,V.} The strong law of large numbers for the risk
estimate in the problem of\linebreak
\\[-12pt]
\hspace*{23pt}tomographic image reconstruction from
projections with a correlated noise&3&41--45\\[.23pt]
\Avtors{Shestakov~O.\,V.} see~Zakharova~T.\,V.&&\\[.23pt]
\Avtors{Shnurkov~P.\,V., Gorshenin~A.\,K., and Belousov~V.\,V.}
Analytical solution of~the~optimal control\linebreak
\\[-12pt]
\hspace*{23pt}task of~a~semi-Markov
process with~finite set of~states&4&72--88\\[.23pt]
\Avtors{Shnurkov~P.\,V., Zasypko~V.\,V., Belousov~V.\,V., and
Gorshenin~A.\,K.} Development of the algorithm of numerical solution
of the optimal investment control problem\linebreak
\\[-12pt]
\hspace*{23pt}in the closed dynamical model of three-sector economy&1&82--95\\[.23pt]
\Avtors{Shorgin~S.\,Ya.} see~Gaidamaka~Yu.\,V.&&\\[.23pt]
\Avtors{Shorgin~V.\,S.} see~Agalarov~Ya.\,M.&&\\[.23pt]
\Avtors{Shubnikov~S.\,K.} see~Minin~V.\,A.&&\\[.23pt]
\Avtors{Sidorkin~I.\,I.} see~Arkhipov~O.\,P.&&\\[.23pt]
\Avtors{Sinitsyn~I.\,N.} Analytical modeling of processes in stochastic
systems with complex fractional\linebreak
\\[-12pt]
\hspace*{23pt}order Bessel nonlinearities&3&55--65\\[.23pt]
\Avtors{Sinitsyn~I.\,N.} Orthogonal supoptimal filters for nonlinear
stochastic systems on manifolds&1&34--44\\[.23pt]
\Avtors{Sinitsyn~I.\,N.\ and Korepanov~E.\,R.} Normal Pugachev
conditionally-optimal filters and extra-\linebreak
\\[-12pt]
\hspace*{23pt}polators for state linear stochastic systems&2&14--23\\[.23pt]
\Avtors{Sinitsyn~I.\,N.\ and Sinitsyn~V.\,I.} Analytical modeling of
distributions in stochastic systems on\linebreak
\\[-12pt]
\hspace*{23pt}manifolds based on ellipsoidal approximation&1&45--55\\[.23pt]
\Avtors{Sinitsyn~I.\,N., Sinitsyn~V.\,I., and
Korepanov~E.\,R.} Ellipsoidal suboptimal filters for nonlinear\linebreak
\\[-12pt]
\hspace*{23pt}stochastic systems on manifolds&2&24--35\\[.23pt]
\Avtors{Sinitsyn~V.\,I.} see~Sinitsyn~I.\,N.&&\\[.23pt]
\Avtors{Sinitsyn~V.\,I.} see~Sinitsyn~I.\,N.&&\\[.23pt]
\Avtors{Skvortsov~N.\,A.} see~Stupnikov~S.\,A.&&\\[.23pt]
\Avtors{Sokolov~I.\,A.} see~Chertok~A.\,V.&&\\
\end{tabular}
}
\pagebreak

\def\leftfootline{\small{\textbf{\thepage}
\hfill INFORMATIKA I EE PRIMENENIYA~--- INFORMATICS AND APPLICATIONS\ \ \ 2016\
\ \ volume~10\ \ \ issue\ 4}
}%
 \def\rightfootline{\small{INFORMATIKA I EE PRIMENENIYA~---
INFORMATICS AND APPLICATIONS\ \ \ 2016\ \ \ volume~10\ \ \ issue\ 4
\hfill \textbf{\thepage}}}

\def\leftkol{2016 AUTHOR INDEX} % ENGLISH ABSTRACTS}

\def\rightkol{2016 AUTHOR INDEX} %ENGLISH ABSTRACTS}


{\tabcolsep=3pt
\begin{tabular}{p{382pt}cc}
&\textbf{Issue} & \textbf{Page}\\[6pt]
\Avtors{Sopin~E.\,S.} see~Gaidamaka~Yu.\,V.&&\\
\Avtors{Strijov~V.\,V.} see~Goncharov~A.\,V.&&\\
\Avtors{Strijov~V.\,V.} see~Isachenko~R.\,V.&&\\
\Avtors{Strijov~V.\,V.} see~Karasikov~M.\,E.&&\\
\Avtors{Stupnikov~S.\,A., Briukhov~D.\,O., and Skvortsov~N.\,A.}
Co-lending systemic risk analysis over\linebreak
\\[-12pt]
\hspace*{23pt}heterogeneous data collections&1&23--33\\
\Avtors{Stupnikov~S.\,A.} see~Kalinichenko~L.\,A.&&\\
\Avtors{Suchkov~A.\,P.} see~Zatsarinny~A.\,A.&&\\
\Avtors{Timonina~E.\,E.} see~Grusho~A.\,A.&&\\
\Avtors{Titova~A.\,I.} see~Kudryavtsev~A.\,A.&&\\
\Avtors{Turlikov~A.\,M.} see~Ometov~A.\,Ya.&&\\
\Avtors{Tyrsin~A.\,N.\ and Serebryanskii~S.\,M.} Recognition of
dependences on the basis of inverse\linebreak
\\[-12pt]
\hspace*{23pt}mapping&2&58--64\\
\Avtors{Ulyanov~V.\,V.} see~Markov~A.\,S.&&\\
\Avtors{Ushakov~V.\,G.} Queueing system with working vacations and
hyperexponential input stream&2&92--97\\
\Avtors{Ushakov~V.\,G.} see~Leontyev~N.\,D.&&\\
\Avtors{Volnova~A.\,A.} see~Kalinichenko~L.\,A.&&\\
\Avtors{Yakovlev~O.\,A.\ and Gasilov~A.\,V.} Speeded-up stereo
matching using geodesic support weights&3&\hphantom{1}98--104\\
\Avtors{Zabezhailo~M.\,I.} see~Grusho~A.\,A.&&\\
\Avtors{Zabezhailo~M.\,I.} see~Grusho~A.\,A.&&\\
\Avtors{Zakharova~T.\,V.\ and Shestakov~O.\,V.} Precision analysis of
wavelet processing of aerodynamic\linebreak
\\[-12pt]
\hspace*{23pt}flow patterns&3&46--54\\
\Avtors{Zalizniak~Anna~A.\ and Kruzhkov~M.\,G.} Database
of~Russian impersonal verbal constructions&4&132--141\\
\Avtors{Zasypko~V.\,V.} see~Shnurkov~P.\,V.&&\\
\Avtors{Zatsarinny~A.\,A.\ and Suchkov~A.\,P.} Systems engineering
approaches to~the~establishment of\linebreak
\\[-12pt]
\hspace*{23pt}a~system for~decision support based
on~situational analysis&4&105--113\\
\Avtors{Zatsarinny~A.\,A.} see~Grusho~A.\,A.&&\\
\Avtors{Zatsman~I.\,M., Inkova~O.\,Yu., Kruzhkov~M.\,G., and
Popkova~N.\,A.} Representation of cross-\linebreak
\\[-12pt]
\hspace*{23pt}lingual knowledge about
connectors in supracorpora databases&1&106--118\\
\Avtors{Zatsman~I.\,M.} see~Minin~V.\,A.&&\\
\Avtors{Zeifman~A.\,I.} see~Korolev~V.\,Yu.&&\\
\Avtors{Zeifman~A.\,I.} see~Korolev~V.\,Yu.&&\\
\end{tabular}
}

%\thispagestyle{myheadings}
\def\leftfootline{\small{\textbf{\thepage}
\hfill INFORMATIKA I EE PRIMENENIYA~--- INFORMATICS AND APPLICATIONS\ \ \ 2016\
\ \ volume~10\ \ \ issue\ 4}
}%
 \def\rightfootline{\small{INFORMATIKA I EE PRIMENENIYA~---
INFORMATICS AND APPLICATIONS\ \ \ 2016\ \ \ volume~10\ \ \ issue\ 4
\hfill \textbf{\thepage}}}

 \label{end\stat}

\newpage

%\def\stat{rekl}
%\label{preobr}

%\def\tit{АКАДЕМИК ПУГАЧЁВ  ВЛАДИМИР СЕМЁНОВИЧ\\
%25.03.1911--25.03.1998}


%   \vspace*{-48pt}
%   \begin{center}\LARGE
%Академик Пугачёв  Владимир Семёнович\\ (25.03.1911--25.03.1998)
%   \end{center}
   
   %\vspace*{2.5mm}
   
   \begin{center}

{\prgsh\LARGE
ОБЪЯВЛЕНИЯ О КОНФЕРЕНЦИЯХ}

\end{center}
%\hrule

\vspace*{6pt}

   
   \vspace*{10mm}
   
   \thispagestyle{empty}

\noindent
\begin{tabular}{cc}
%\begin{center}
\multicolumn{1}{c}{\raisebox{-40pt}[0pt][0pt]{\mbox{%
\epsfxsize=33mm
\epsfbox{vspu.eps}
}}}
%\end{center}
&
\tabcolsep=0pt\begin{tabular}{c}
{\prg{\Large\textbf{XII Всероссийское совещание}}}\\[6pt]
{\prg{\Large\textbf{по проблемам управления}}}\\[12pt]
{\prg{\large 16--19 июня 2014~г.}}\\[6pt] 
{\prg{\large Институт проблем управления имени В.\,А.~Трапезникова РАН}}\\[6pt]
{\prg{\large Москва, Россия}}
\end{tabular}
\end{tabular}

\vspace*{60pt}

     
 { %\large    
 XII Всероссийское совещание по проблемам управления (ВСПУ XII), посвященное 75-летию 
Института проблем управления (ИПУ) имени В.\,А.~Трапезникова РАН, проводится 16--19~июня 
2014~г.\ 
в ИПУ РАН (г.~Москва, Россия). ВСПУ XII организуется ИПУ РАН при поддержке РФФИ, Отделения 
энергетики, машиностроения, механики и процессов управления Российской академии наук, 
Российского 
национального комитета по автоматическому управлению, Академии навигации и управ\-ле\-ния 
движением, 
Научного совета РАН по комплексным проблемам управления и автоматизации, Совета по 
мехатронике и робототехнике РАН. Официальный язык Совещания~--- русский.

\vspace*{24pt}
     
     \textbf{Направления работы}
     \begin{enumerate}[1.]
\item Теория систем управления
\item Управление подвижными объектами и навигация
\item Интеллектуальные системы управления
\item Управление в промышленности, транспортом и логистикой
\item Управление системами междисциплинарной природы
\item Средства измерения, вычислений и контроля в управлении
\item Системный анализ и принятие решений в задачах управления
\item Информационные технологии в управлении
\item Проблемы образования в области управления: современное содержание и технологии обучения
\end{enumerate}

\vspace*{24pt}

     Подробная информация о Совещании находится на сайте {\sf http://vspu2014.ipu.ru}. Срок 
окончательной подачи докладов через систему подачи докладов на сайте~--- \textbf{30~ноября} 
2013~г.
}

%\include{rekl-1}

%\end{document}

%\include{nekrolog-rb}


%\end{document}

%\include{IPPM-25}

\def\stat{cont-rus}
{%\hrule\par
%\vskip 7pt % 7pt
\vspace*{-24pt}
\raggedleft\Large \bf%\baselineskip=3.2ex
Правила подготовки рукописей  для публикации в журнале
<<Информатика~и~её~применения>> \vskip 8pt
    \hrule
    \par
\vskip 14pt plus 6pt minus 3pt }

\label{st\stat}

\def\tit{\ }

\def\aut{\ }
\def\auf{\ }

\def\leftkol{\ }
% Правила подготовки рукописей  для публикации в журнале
%<<Информатика и её применения>>

\def\rightkol{\ }
%Правила подготовки рукописей  для публикации в журнале
%<<Информатика и её применения>>}


\titele{\tit}{\aut}{\auf}{\leftkol}{\rightkol}


\vspace*{-60pt}
{ %\small

Журнал <<Информатика и её применения>>
публикует теоретические, обзорные и дискуссионные статьи,
посвященные научным исследованиям и разработкам в области
информатики и ее приложений.

Журнал издается на русском языке. По специальному решению
редколлегии отдельные статьи могут печататься на английском языке.

Тематика журнала охватывает следующие направления:
\begin{itemize}
\item теоретические основы информатики;\\[-15pt]
      \item
математические методы исследования сложных систем и процессов;\\[-15pt]
           \item
информационные системы и сети;\\[-15pt]
                \item
информационные технологии;\\[-15pt]
                     \item
архитектура и программное обеспечение вычислительных комплексов и сетей.\\[-15pt]
\end{itemize}


\noindent
\begin{enumerate}[1.]
\item В журнале печатаются статьи, содержащие результаты, ранее не опубликованные и
не предназначенные к одновременной публикации в других изданиях.

%Публикация не должна нарушать закон об авторских правах.
Публикация предоставленной автором(ами) рукописи не должна нарушать 
положений глав~69, 70 раздела~VII части~IV Гражданского кодекса, 
которые определяют права на результаты интеллектуальной деятельности 
и~средства индивидуализации, в~том числе авторские права, в~РФ.

Ответственность за нарушение авторских прав, в~случае предъявления претензий к~редакции журнала,  
несут авторы статей.



Направляя рукопись в редакцию, авторы сохраняют свои права на данную
рукопись и при этом передают учредителям и редколлегии журнала неисключительные права на
издание статьи на русском языке 
(или на языке статьи, если он отличен от рус\-ско\-го) и~на перевод ее на английский
язык, а~также на
ее распространение в России и за рубежом. 
Каждый автор должен представить в~редакцию подписанный 
с~его стороны <<Лицензионный договор о~передаче неисключительных прав 
на использование произведения>>, текст которого размещен по адресу 
{\sf http://www.ipiran.ru/publications/licence.doc}. 
Этот договор может быть пред\-став\-лен в~бумажном (в~2-х экз.)\ 
или в~электронном виде (отсканированная копия заполненного и~подписанного документа).




Редколлегия вправе запросить у авторов экспертное заключение о возможности
пуб\-ли\-ка\-ции пред\-став\-лен\-ной статьи в открытой печати.\\[-13.5pt]

\item К статье прилагаются данные автора (авторов) (см.\ п.~8). При наличии нескольких
авторов указывается фамилия автора, ответственного за переписку с редакцией.\\[-13.5pt]

\item Редакция журнала осуществляет экспертизу присланных статей в соответствии с
принятой в журнале процедурой рецензирования.

Возвращение рукописи на доработку не означает ее принятия к печати.

Доработанный вариант с ответом на замечания рецензента необходимо прислать в
редакцию.\\[-13.5pt]

\item Решение редколлегии о публикации статьи или ее отклонении сообщается авторам.

Редколлегия может также направить авторам текст рецензии на их статью. Дискуссия по
поводу отклоненных статей не ведется.\\[-13.5pt]

%\pagebreak

\item Редактура статей высылается авторам для просмотра. Замечания к редактуре должны
быть присланы авторами в кратчайшие сроки.\\[-13.5pt]

\item Рукопись предоставляется в электронном виде в форматах MS WORD (.doc или
.docx) или \LaTeX\  (.tex), дополнительно~--- в формате .pdf, на дискете, лазерном диске
или электронной почтой. Предоставление бумажной рукописи необязательно.\\[-13.5pt]

\item При подготовке рукописи в MS Word рекомендуется использовать следующие
настройки.

Параметры страницы:
формат~--- А4; ориентация~--- книжная; поля (см): внутри~--- 2,5, снаружи~--- 1,5,
сверху~--- 2, снизу~--- 2, от края до нижнего колонтитула~--- 1,3.

Основной текст: стиль~--- <<Обычный>>, шрифт~--- Times New Roman, размер~---
14~пунк\-тов, абзацный отступ~--- 0,5~см, 1,5~интервала, выравнивание~--- по ширине.

\pagebreak

\def\leftkol{Правила подготовки рукописей  для публикации в журнале
<<Информатика и её применения>>}

\def\rightkol{Правила подготовки рукописей  для публикации в журнале
<<Информатика и её применения>>}



Рекомендуемый объем рукописи~--- не свыше 10~страниц указанного формата.
При превышении указанного объема редколлегия вправе потребовать от 
автора сокращения объема рукописи.


Сокращения слов, помимо стандартных, не допускаются. Допускается минимальное
количество аббревиатур.


Все страницы рукописи нумеруются.

Шаблоны оформления представлены в интернете:

\noindent
 {\sf
http://www.ipiran.ru/journal/template\_iiep\_ssi\_2024.zip}\\[-14pt]

\item Статья должна содержать следующую информацию на {\bfseries\textit{русском и
английском языках}}:\\[-16pt]

\begin{itemize}
\item название статьи;\\[-15pt]
\item Ф.И.О.\ авторов, на английском можно только имя и фамилию;\\[-15pt]
\item место работы, с указанием почтового адреса организации и электронного адреса каждого
автора;\\[-15pt]
\item сведения об авторах, в соответствии с форматом, образцы которого
представлены на страницах:



\def\leftfootline{\small{\textbf{\thepage}
\hfill ИНФОРМАТИКА И ЕЁ ПРИМЕНЕНИЯ\ \ \ том\ 18\ \ \ выпуск\ 3\ \ \ 2024}
}%
 \def\rightfootline{\small{ИНФОРМАТИКА И ЕЁ ПРИМЕНЕНИЯ\ \ \ том\ 18\ \ \ выпуск\ 3\ \ \ 2024
\hfill \textbf{\thepage}}}



{\sf http://www.ipiran.ru/journal/issues/2013\_07\_01/authors.asp} и

{\sf http://www.ipiran.ru/journal/issues/2013\_07\_01\_eng/authors.asp};
\item аннотация (не менее 100~слов на каждом из языков). Аннотация~--- это краткое
резюме работы, которое может публиковаться отдельно. Она является основным
источником информации в~ин\-фор\-ма\-ци\-он\-ных системах и базах данных. Английская
аннотация должна быть оригинальной, может не быть дословным переводом русского
текста и должна быть написана хорошим английским языком. В~аннотации не должно
быть ссылок на литературу и, по возможности, формул;\\[-15pt]
\item ключевые слова~--- желательно из принятых в мировой
на\-уч\-но-тех\-ни\-че\-ской литературе тематических тезаурусов. Предложения не
могут быть ключевыми словами;\\[-15pt]
\item источники финансирования работы (ссылки на гранты, проекты,
поддерживающие организации и~т.\,п.).
\end{itemize}



%\pagebreak

\item  Требования к спискам литературы.\\[-14pt]

Ссылки на литературу в тексте статьи нумеруются (в квадратных скобках) и
располагаются в каждом из списков литературы в порядке  первых упоминаний. Если источник имеет DOI и/или EDN,
то их необходимо указывать.

Списки литературы представляются в двух вариантах:\\[-14pt]


\noindent
\begin{enumerate}[(1)]
\item \textbf{Список литературы к русскоязычной части}. Русские и английские
работы~---  на языке и в алфавите оригинала;\\[-14.5pt]
\item  \textbf{References}. Русские работы и работы на других языках~--- в латинской
транслитерации с переводом на английский язык; английские работы и работы на других
языках~--- на языке оригинала.
\end{enumerate}

Необходимо для составления списка ``References'' пользоваться размещенной на сайте
{\sf http://www. translit.net/ru/bgn/} бесплатной программой транслитерации русского
 текста в~латиницу. %, при этом в~за\-клад\-ке <<варианты\ldots>> следует выбратьопцию BGN.

Список литературы ``References'' приводится полностью отдельным блоком, повторяя все
позиции из списка литературы к русскоязычной части, независимо от того, имеются или
нет в нем иностранные источники. Если в списке литературы к русскоязычной части есть
ссылки на иностранные публикации, набранные латиницей, они полностью повторяются в
списке ``References''.

Ниже приведены примеры ссылок на различные виды публикаций в списке ``References''.

\def\leftfootline{\small{\textbf{\thepage}
\hfill ИНФОРМАТИКА И ЕЁ ПРИМЕНЕНИЯ\ \ \ том\ 18\ \ \ выпуск\ 3\ \ \ 2024}
}%
 \def\rightfootline{\small{ИНФОРМАТИКА И ЕЁ ПРИМЕНЕНИЯ\ \ \ том\ 18\ \ \ выпуск\ 3\ \ \ 2024
\hfill \textbf{\thepage}}}

{\small

\noindent
\textbf{Описание статьи из журнала:}

\Aue{Zagurenko, A.\,G., V.\,A.~Korotovskikh, A.\,A.~Kolesnikov, A.\,V.~Timonov, and D.\,V.~Kardymon}. 2008.
Tekhniko-ekonomicheskaya optimizatsiya dizayna gidrorazryva plasta [Technical and
economic optimization of the design
of hydraulic fracturing]. \textit{Neftyanoe hozyaystvo} [\textit{Oil Industry}] 11:54--57.

\Aue{Zhang, Z., and D.~Zhu}. 2008. Experimental research on the localized
electrochemical micromachining. \textit{Russ. J.~Electrochem.}  44(8):926--930.
{\sf doi:10.1134/S1023193508080077}.

\noindent
\textbf{Описание статьи из электронного журнала:}

\Aue{Swaminathan, V., E.~Lepkoswka-White, and B.\,P.~Rao}. 1999. Browsers or buyers in cyberspace? An
investigation of electronic factors influencing electronic exchange. \textit{JCMC}
5(2). Available at: {\sf http://www.ascusc.org/jcmc/vol5/issue2/} (accessed April~28, 2011).

\def\leftkol{Правила подготовки рукописей  для публикации в журнале
<<Информатика и её применения>>}

\def\rightkol{Правила подготовки рукописей  для публикации в журнале
<<Информатика и её применения>>}


\noindent
\textbf{Описание статьи из продолжающегося издания (сборника трудов):}

\Aue{Astakhov, M.\,V., and T.\,V.~Tagantsev}. 2006. Eksperimental'noe
issledovanie prochnosti soedineniy ``stal'--kompozit'' [Experimental study of
the strength of joints ``steel--composite'']. \textit{Trudy MGTU
``Matematicheskoe modelirovanie slozhnykh tekh\-ni\-che\-skikh sistem''}
[\textit{Bauman MSTU ``Mathematical Modeling of Complex Technical
Systems'' Proceedings}]. 593:125--130.


\pagebreak



\noindent
\textbf{Описание материалов конференций:}

\Aue{Usmanov, T.\,S., A.\,A.~Gusmanov, I.\,Z.~Mullagalin, R.\,Ju.~Muhametshina, A.\,N.~Chervyakova, and
A.\,V.~Sveshnikov}. 2007. Osobennosti proektirovaniya razrabotki mestorozhdeniy
s primeneniem gidrorazryva
plasta [Features of the design of field development with the use of hydraulic fracturing].
\textit{Trudy 6-go
Mezhdu\-na\-rod\-no\-go Simpoziuma ``Novye resursosberegayushchie tekhnologii nedropol'zovaniya i povysheniya
neftegazootdachi''} [\textit{6th  Symposium (International) ``New Energy Saving Subsoil Technologies and
the Increasing of the Oil and Gas Impact'' Proceedings}]. Moscow. 267--272.



\def\leftfootline{\small{\textbf{\thepage}
\hfill ИНФОРМАТИКА И ЕЁ ПРИМЕНЕНИЯ\ \ \ том\ 18\ \ \ выпуск\ 3\ \ \ 2024}
}%
 \def\rightfootline{\small{ИНФОРМАТИКА И ЕЁ ПРИМЕНЕНИЯ\ \ \ том\ 18\ \ \ выпуск\ 3\ \ \ 2024
\hfill \textbf{\thepage}}}



\noindent
\textbf{Описание книги (монографии, сборники):}



Lindorf, L.\,S., and L.\,G.~Mamikoniants, eds. 1972.
\textit{Ekspluatatsiya turbogeneratorov s neposredstvennym
okhlazhdeniem} [\textit{Operation of turbine generators with direct cooling}].
Moscow: Energy Publs. 352~p.


\Aue{Latyshev, V.\,N.} 2009. \textit{Tribologiya rezaniya. Kn.~1: Friktsionnye protsessy
pri rezanii metallov}
[\textit{Tribology of cutting. Vol.~1: Frictional processes in metal cutting}]. Ivanovo: Ivanovskii
State Univ. 108~p.

\def\leftkol{Правила подготовки рукописей  для публикации в журнале
<<Информатика и её применения>>}

\def\rightkol{Правила подготовки рукописей  для публикации в журнале
<<Информатика и её применения>>}

\noindent
\textbf{Описание переводной книги}
(в списке литературы к русскоязычной части необходимо указать:~/ Пер.\ с англ.~---
после названия книги, а в конце ссылки указать оригинал книги в круглых скобках):
\begin{enumerate}[1.]
\item  В русскоязычной части:

\def\leftfootline{\small{\textbf{\thepage}
\hfill ИНФОРМАТИКА И ЕЁ ПРИМЕНЕНИЯ\ \ \ том\ 18\ \ \ выпуск\ 3\ \ \ 2024}
}%
 \def\rightfootline{\small{ИНФОРМАТИКА И ЕЁ ПРИМЕНЕНИЯ\ \ \ том\ 18\ \ \ выпуск\ 3\ \ \ 2024
\hfill \textbf{\thepage}}}

\Au{Тимошенко С.\,П., Янг Д.\,Х., Уивер~У.}
Колебания в инженерном деле~/ Пер.\ с англ.~--- М.: Машиностроение, 1985. 472~с.
(\Au{Timoshenko~S.\,P., Young~D.\,H., Weaver~W.}
Vibration problems in engineering.~--- 4th ed.~--- New York, NY, USA: Wiley, 1974. 521~p.)\\[-13.5pt]
\item  В англоязычной части:

\Aue{Timoshenko, S.\,P., D.\,H.~Young, and W.~Weaver}.
1974. \textit{Vibration problems in engineering}. 4th ed. New York: 
Wiley. 521~p.
\end{enumerate}

\vspace*{-3pt}


\noindent
\textbf{Описание неопубликованного документа:}


\Aue{Latypov, A.\,R., M.\,M.~Khasanov, and V.\,A.~Baikov}.
2004 (unpubl.). Geologiya i~dobycha (NGT GiD) [Geology and production (NGT GiD)]. Certificate on official registration of the computer program
No.\,2004611198. 

\noindent
\textbf{Описание интернет-ресурса:}


Pravila tsitirovaniya istochnikov [Rules for the citing of sources]. Available at: {\sf
http://www.scribd.com/doc/1034528/} (accessed February~7, 2011).

%\pagebreak

\noindent
\textbf{Описание диссертации или автореферата диссертации:}

\Aue{Semenov, V.\,I.}
2003. Matematicheskoe modelirovanie plazmy v sisteme kompaktnyy tor [Mathematical
modeling of the plasma in the compact torus].  Moscow.  D.Sc.\ Diss. 272~p.

\Aue{Kozhunova, O.\,S.} 2009. Tekhnologiya razrabotki semanticheskogo
slovarya informatsionnogo monitoringa [Technology of development of
semantic dictionary of information monitoring system].  Moscow: IPI RAN. PhD Thesis. 23~p.


\noindent
\textbf{Описание ГОСТа:}

GOST 8.586.5-2005. 2007. Metodika vypolneniya izmereniy. Izmerenie raskhoda i~kolichestva zhidkostey i~gazov
s~pomoshch'yu standartnykh suzhayushchikh ustroystv [Method of measurement.
Measurement of flow rate and volume of liquids and gases by means of orifice devices]. Moscow:
Standardinform  Publs. 10~p.

\noindent
\textbf{Описание патента:}

\Aue{Bolshakov, M.\,V., A.\,V.~Kulakov, A.\,N.~Lavrenov, and M.\,V.~Palkin}.
2006. Sposob orientirovaniya po krenu letatel'nogo
apparata s opti\-che\-skoy golovkoy
samonavedeniya [The way to orient on the roll of aircraft with optical homing head].
Patent RF No.\,2280590.
}

\item Присланные в редакцию материалы авторам не возвращаются.\\[-13.5pt]

\item При отправке файлов по электронной почте просим придерживаться следующих
правил:
\begin{itemize}
\item указывать в поле subject (тема) название журнала и фамилию автора;\\[-13.5pt]
\item указывать в тексте письма название статьи, авторов и~журнал, в~который направляется статья;\\[-13.5pt]
\item использовать attach (присоединение);\\[-13.5pt]
\item в состав электронной версии статьи должны входить: файл, содержащий текст
статьи, и файл(ы), содержащий(е) иллюстрации.\\[-13.5pt]
\end{itemize}

\item Журнал <<Информатика и её применения>> является некоммерческим изданием.
Плата за публикацию не взимается, гонорар авторам не выплачивается.
\end{enumerate}



\def\leftfootline{\small{\textbf{\thepage}
\hfill ИНФОРМАТИКА И ЕЁ ПРИМЕНЕНИЯ\ \ \ том\ 18\ \ \ выпуск\ 3\ \ \ 2024}
}%
 \def\rightfootline{\small{ИНФОРМАТИКА И ЕЁ ПРИМЕНЕНИЯ\ \ \ том\ 18\ \ \ выпуск\ 3\ \ \ 2024
\hfill \textbf{\thepage}}}


\vspace*{-1mm}

\begin{center}

\textbf{Адрес редакции журнала <<Информатика и её применения>>:} \\




Москва 119333, ул.~Вавилова, д.~44, корп.~2, ФИЦ ИУ РАН\\[-10pt]

\

Тел.: +7\,(499)\,135-86-92\ \ Факс:  +7\,(495)\,930-45-05\\[-10pt]

 \

e-mail:   {\sf iiep@frccsc.ru} (Стригина Светлана Николаевна)\\[-10pt]

\

{\sf http://www.ipiran.ru/journal/issues/}
\end{center}
}


\def\leftkol{Правила подготовки рукописей  для публикации в журнале
<<Информатика и её применения>>}

\def\rightkol{Правила подготовки рукописей  для публикации в журнале
<<Информатика и её применения>>}


\def\leftfootline{\small{\textbf{\thepage}
\hfill ИНФОРМАТИКА И ЕЁ ПРИМЕНЕНИЯ\ \ \ том\ 18\ \ \ выпуск\ 3\ \ \ 2024}
}%
 \def\rightfootline{\small{ИНФОРМАТИКА И ЕЁ ПРИМЕНЕНИЯ\ \ \ том\ 18\ \ \ выпуск\ 3\ \ \ 2024
\hfill \textbf{\thepage}}} 
\def\stat{podg-e}
{%\hrule\par
%\vskip 7pt % 7pt
\vspace*{-24pt}
\raggedleft\Large \bf%\baselineskip=3.2ex
Requirements for manuscripts submitted to Journal
``Informatics~and~Applications'' \vskip 8pt
    \hrule
    \par
\vskip 21pt plus 6pt minus 3pt }

\label{st\stat}

\def\tit{\ }

\def\aut{\ }
\def\auf{\ }

\def\leftkol{\ }

\def\rightkol{\ }
%Requirements for manuscripts submitted to Journal
%``Informatics~and~Applications''}

\titele{\tit}{\aut}{\auf}{\leftkol}{\rightkol}

\def\leftfootline{\small{\textbf{\thepage}
\hfill INFORMATIKA I EE PRIMENENIYA~--- INFORMATICS AND APPLICATIONS\ \ \ 2019\
\ \ volume~13\ \ \ issue\ 4}
}%
 \def\rightfootline{\small{INFORMATIKA I EE PRIMENENIYA~--- INFORMATICS AND APPLICATIONS\ \ \ 2019\ \ \ volume~13\ \ \ issue\ 4
\hfill \textbf{\thepage}}}

\vspace*{-60pt}

{\small

\noindent
Journal ``Informatics and Applications'' (Inform.\ Appl.)
publishes theoretical, review, and discussion
articles on the research and development in the
field of informatics and its applications.

The journal is published in Russian.
By a special decision of the editorial
board, some articles can be published in English.


The topics covered include the following areas:
\begin{itemize}
               \item
     theoretical fundamentals of informatics; \\[-14pt]
\item
mathematical methods for studying complex systems and processes; \\[-14pt]
\item
information systems and networks;\\[-14pt]
\item
information technologies; and \\[-14pt]
\item
architecture and software of computational complexes and networks. \\[-14pt]
\end{itemize}

\noindent
\begin{enumerate}[1.]
\item The Journal publishes original articles which have not been published before and are not
intended for simultaneous publication in other editions. An article submitted to the Journal must not violate the
Copyright law. Sending the manuscript to the Editorial Board, the authors retain all rights of the
owners of the manuscript and transfer the nonexclusive rights to publish the article in Russian
(or the language of the article, if not Russian) and its distribution in Russia and abroad to the
Founders and the Editorial Board. Authors should submit a letter to the Editorial Board in the
following form:

{\bfseries\textit{Agreement on the transfer of rights to publish:}}

``\textit{We, the undersigned authors of the manuscript ``\ldots'', pass to the
Founder and the Editorial Board of the Journal ``Informatics and Applications''
the nonexclusive right to publish the manuscript of the article in Russian (or
in English) in both print and electronic versions of the Journal. We affirm
that this publication does not violate the Copyright of other persons or
organizations.}

\textit{Author(s) signature(s): (name(s), address(es), date).}

This agreement should be submitted in paper form or in the form of a scanned copy (signed by
the authors).


%The Editorial Board has the right to request from the authors an official expert conclusion that
%the submitted article has no secret data prohibited for publication. \\[-13.5pt]
\item
A submitted article should be attached with \textbf{the data on the author(s)} (see item~8). If
there are several authors, the contact person should be indicated who is responsible for
correspondence with the Editorial Board and other authors about revisions and final approval
of the proofs.\\[-13.5pt]

\item The Editorial Board of the Journal examines the article according to the established
reviewing procedure. If the authors receive their article for correction after reviewing, it does not
mean that the article is approved for publication. The corrected article should be sent to the
Editorial Board for the subsequent review and approval.\\[-13.5pt]

\item The decision on the article publication or its rejection is communicated to the authors. The
Editorial Board may also send the reviews on the submitted articles to the authors. Any
discussion upon the rejected articles is not possible.\\[-13.5pt]

\item The edited articles will be sent to the authors for proofread. The comments of the authors
to the edited text of the article should be sent to the Editorial Board as soon as possible.\\[-13.5pt]

\item The manuscript of the article should be presented electronically in the MS WORD (.doc or
.docx) or \LaTeX\ (.tex) formats, and additionally in the .pdf format. All documents
 may be sent
by e-mail or provided on a CD or diskette. A~hard copy submission is not necessary.\\[-13.5pt]

\item The recommended typesetting instructions for manuscript.

Pages parameters: format A4, portrait orientation, document margins (cm): left~--- 2.5, right~---
1.5, above~--- 2.0, below~--- 2.0, footer 1.3.

Text: font~---Times New Roman, font size~--- 14, paragraph indent~--- 0.5, line spacing~--- 1.5,
justified alignment.

The recommended manuscript size: not more than 15~pages of the specified format.
If the specified size exceeded, the editorial board is entitled to require the author
to reduce the manuscript.

Use only standard abbreviations. Avoid  abbreviations in the title and
abstract. The full term for which an abbreviation stands should precede
its first use in the text unless it is a standard unit of measurement.

All pages of the manuscript should be numbered.

The templates for the manuscript typesetting are presented on site: {\sf
http://www.ipiran.ru/journal/template.doc}.\\[-13.5pt]


%\def\leftkol{Requirements for manuscripts submitted to Journal
%``Informatics~and~Applications''}

\item The articles should enclose data both in \textbf{Russian and English}:
\begin{itemize}
\item title;\\[-13.5pt]
\item author's name and surname;\\[-13.5pt]
\item affiliation~--- organization, its address with ZIP code, city, country, and
official e-mail address;\\[-13.5pt]
\item data on authors according to the format: (see site)

{\sf http://www.ipiran.ru/journal/issues/2013\_07\_01/authors.asp}  and

{\sf  http://www.ipiran.ru/journal/issues/2013\_07\_01\_eng/authors.asp};\\[-13.5pt]

\pagebreak

\def\leftfootline{\small{\textbf{\thepage}
\hfill INFORMATIKA I EE PRIMENENIYA~--- INFORMATICS AND APPLICATIONS\ \ \ 2019\
\ \ volume~13\ \ \ issue\ 4}
}%
 \def\rightfootline{\small{INFORMATIKA I EE PRIMENENIYA~--- INFORMATICS AND APPLICATIONS\ \ \ 2019\ \ \ volume~13\ \ \ issue\ 4
\hfill \textbf{\thepage}}}


%\def\leftkol{Requirements for manuscripts submitted to Journal
%``Informatics~and~Applications''}

%\def\rightkol{Requirements for manuscripts submitted to Journal
%``Informatics~and~Applications''}



\item abstract (not less than 100 words) both in Russian and in English. Abstract is a short
summary of the article that can be published separately. The abstract is the
main source of information on the article and it could be included in leading information
systems and data bases. The abstract in English has to be an original text and should
not be an exact translation of the Russian one. Good English is required.
In abstracts, avoid references and formulae;\\[-13.5pt]
\item indexing is performed on the basis of keywords. The use of keywords from the
internationally accepted thematic Thesauri is recommended.

%\def\leftkol{Requirements for manuscripts submitted to Journal
%``Informatics~and~Applications''}

%\def\rightkol{Requirements for manuscripts submitted to Journal
%``Informatics~and~Applications''}

Important! Keywords must not be sentences;
\item Acknowledgments.
\end{itemize}

\item References. Russian references have to be presented both in English translation and Latin
transliteration (refer {\sf http://www.translit.net/ru/bgn/}).

Please take into account the following examples of Russian references appearance:

\noindent
\textbf{Article in journal:}

\Aue{Zhang, Z., and D.~Zhu}. 2008. Experimental research on the localized electrochemical
micromachining.
\textit{Rus. J.~Electrochem.}  44(8):926--930. {\sf doi:10.1134/S1023193508080077}.


\noindent
\textbf{Journal article in electronic format:}

\Aue{Swaminathan, V., E.~Lepkoswka-White, and B.\,P.~Rao}. 1999. Browsers or buyers in
cyberspace? An
investigation of electronic factors influencing electronic exchange. \textit{JCMC}
5(2). Available at: {\sf http://www.ascusc.org/jcmc/vol5/issue2/} (accessed April~28, 2011).




\noindent
\textbf{Article from the continuing publication (collection of works, proceedings):}

\Aue{Astakhov, M.\,V., and T.\,V.~Tagantsev}. 2006. Eksperimental'noe
issledovanie prochnosti soedineniy ``stal'--kompozit'' [Experimental study of
the strength of joints ``steel--composite'']. \textit{Trudy MGTU
``Matematicheskoe modelirovanie slozhnykh tekh\-ni\-che\-skikh sistem''}
[\textit{Bauman MSTU ``Mathematical Modeling of Complex Technical
Systems'' Proceedings}]. 593:125--130.

\def\leftfootline{\small{\textbf{\thepage}
\hfill INFORMATIKA I EE PRIMENENIYA~--- INFORMATICS AND APPLICATIONS\ \ \ 2019\
\ \ volume~13\ \ \ issue\ 4}
}%
 \def\rightfootline{\small{INFORMATIKA I EE PRIMENENIYA~--- INFORMATICS AND APPLICATIONS\ \ \ 2019\ \ \ volume~13\ \ \ issue\ 4
\hfill \textbf{\thepage}}}

\def\leftkol{Requirements for manuscripts submitted to Journal
``Informatics~and~Applications''}

\def\rightkol{Requirements for manuscripts submitted to Journal
``Informatics~and~Applications''}

\noindent
\textbf{Conference proceedings:}

\Aue{Usmanov, T.\,S., A.\,A.~Gusmanov, I.\,Z.~Mullagalin, R.\,Ju.~Muhametshina,
A.\,N.~Chervyakova, and
A.\,V.~Sveshnikov}. 2007. Osobennosti proektirovaniya razrabotki mestorozhdeniy
s primeneniem gidrorazryva
plasta [Features of the design of field development with the use of hydraulic fracturing].
\textit{Trudy 6-go
Mezhdu\-na\-rod\-no\-go Simpoziuma ``Novye resursosberegayushchie tekhnologii
nedropol'zovaniya i povysheniya
neftegazootdachi''} [\textit{6th  Symposium (International) ``New Energy Saving Subsoil
Technologies and
the Increasing of the Oil and Gas Impact'' Proceedings}]. Moscow. 267--272.


\noindent
\textbf{Books and other monographs:}




Lindorf, L.\,S., and L.\,G.~Mamikoniants, eds. 1972.
\textit{Ekspluatatsiya turbogeneratorov s neposredstvennym
okhlazhdeniem} [\textit{Operation of turbine generators with direct cooling}].
Moscow: Energy Publs. 352~p.


%\Aue{Latyshev, V.\,N.} 2009. \textit{Tribologiya rezaniya. Kn.~1: Frikcionnye prosessy
%pri rezanii metallov}
%[\textit{Tribology of cutting. Vol.~1: Frictional processes in metal cutting}]. Ivanovo: Ivanovskii
%State Univ. 108~p.


%\noindent
%\textbf{Unpublished material:}

%\Aue{Latypov, A.\,R., M.\,M.~Khasanov, and V.\,A.~Baikov}.
%2004. Geology and production (NGT GiD). Certificate on official registration of the computer
%program
%No.\,2004611198. (In Russian, unpubl.)

%\noindent
%\textbf{Internet-source:}

%APA Style. 2011. Available at: {\sf http://www.apastyle.org/apa-style-help.aspx} (accessed
%February~5, 2011).

%Pravila citirovaniya istochnikov [Rules for the citing of sources]. Available at: {\sf
%http://www.scribd.com/doc/1034528/} (accessed February~7, 2011).


\noindent
\textbf{Dissertation and Thesis:}

%\Aue{Semenov, V.\,I.}
%2003. Matematicheskoe modelirovanie plazmy v sisteme kompaktnyy tor. [Mathematical
%modeling of the plasma in the compact torus]. D.Sc.\ Diss. Moscow. 272~p.

\Aue{Kozhunova, O.\,S.} 2009. Tekhnologiya razrabotki semanticheskogo
slovarya informatsionnogo monitoringa [Technology of development of
semantic dictionary of information monitoring system]. PhD Thesis. Moscow: IPI RAN. 23~p.


\noindent
\textbf{State standards and patents:}

GOST 8.586.5-2005. 2007. Metodika vypolneniya izmereniy. Izmerenie raskhoda i~kolichestva
zhidkostey i gazov 
s~pomoshch'yu standartnykh suzhayushchikh ustroystv [Method of measurement.
Measurement of flow rate and volume of liquids and gases by means of orifice devices]. M.:
Standardinform
Publs. 10~p.

%\noindent
%\textbf{Patent:}

\Aue{Bolshakov, M.\,V., A.\,V.~Kulakov, A.\,N.~Lavrenov, and M.\,V.~Palkin}.
2006. Sposob orientirovaniya po krenu letatel'nogo
apparata s opti\-che\-skoy golovkoy
samonavedeniya [The way to orient on the roll of aircraft with optical homing head].
Patent RF No.\,2280590.

References in Latin transcription are presented in the original language.

References in the text are numbered according to the order of their
first appearance; the number is
placed in square brackets. All items from the reference list should be
cited.\\[-13.5pt]

\item Manuscripts and additional materials are not returned to Authors by the Editorial Board.\\[-13.5pt]

\item Submissions of files by e-mail must include:\\[-13.5pt]
\begin{itemize}
\item   the journal title and author's name in the ``Subject'' field; \\[-13.5pt]
\item   an article and additional materials have to be attached using the ``attach'' function;\\[-13.5pt]
\item   an electronic version of the article should contain the file with the text and a separate file
with figures.\\[-13.5pt]
\end{itemize}

\item ``Informatics and Applications'' journal is not a profit publication. There are no
charges for the authors as well as there are no royalties.\\[-13.5pt]
\end{enumerate}

\def\leftfootline{\small{\textbf{\thepage}
\hfill INFORMATIKA I EE PRIMENENIYA~--- INFORMATICS AND APPLICATIONS\ \ \ 2019\
\ \ volume~13\ \ \ issue\ 4}
}%
 \def\rightfootline{\small{INFORMATIKA I EE PRIMENENIYA~--- INFORMATICS AND APPLICATIONS\ \ \ 2019\ \ \ volume~13\ \ \ issue\ 4
\hfill \textbf{\thepage}}}

\def\leftkol{Requirements for manuscripts submitted to Journal
``Informatics~and~Applications''}

\def\rightkol{Requirements for manuscripts submitted to Journal
``Informatics~and~Applications''}


%\vspace*{5mm}


\begin{center}
\textbf{Editorial Board address:} \\

%ABOUT AUTHORS



FRC CSC RAS, 44, block~2, Vavilov Str., Moscow 119333, Russia\\[-10pt]

\

Ph.: +7\,(499)\,135\,86\,92,\ \ Fax: +7\,(495)\,930\,45\,05\\[-10pt]

\

 e-mail: {\sf rust@ipiran.ru} (to Prof.\ Rustem Seyful-Mulyukov)\\[-10pt]

\

 {\sf http://www.ipiran.ru/english/journal.asp}
\end{center}
 }
%\thispagestyle{myheadings}

\def\leftkol{Requirements for manuscripts submitted to Journal
``Informatics~and~Applications''}

\def\rightkol{Requirements for manuscripts submitted to Journal
``Informatics~and~Applications''}

\def\leftfootline{\small{\textbf{\thepage}
\hfill INFORMATIKA I EE PRIMENENIYA~--- INFORMATICS AND APPLICATIONS\ \ \ 2019\
\ \ volume~13\ \ \ issue\ 4}
}%
 \def\rightfootline{\small{INFORMATIKA I EE PRIMENENIYA~--- INFORMATICS AND APPLICATIONS\ \ \ 2019\ \ \ volume~13\ \ \ issue\ 4
\hfill \textbf{\thepage}}}

 \label{end\stat}

\newpage

%\vspace*{-60pt} {\small
{\baselineskip=9.1pt
\section*{Правила подготовки рукописей статей для публикации в журнале
<<Информатика и её применения>>}

\thispagestyle{empty}

 Журнал <<Информатика и её применения>> публикует
теоретические, обзорные и дискуссионные статьи, посвященные научным
исследованиям и разработкам в области информатики и ее приложений. Журнал
издается на русском языке. По специальному решению редколлегии отдельные статьи,
в виде исключения, могут печататься на английском языке.
Тематика журнала охватывает следующие направления:
\begin{itemize}
\item теоретические основы информатики; %\\[-13.5pt]
\item математические методы исследования сложных систем и процессов; %\\[-13.5pt]
\item информационные системы и сети; %\\[-13.5pt]
\item информационные технологии; %\\[-13.5pt]
\item архитектура и программное
обеспечение вычислительных комплексов и сетей.
\end{itemize}
\begin{enumerate}
\item В журнале печатаются результаты, ранее не
опубликованные и не предназначенные к одновременной публикации в других
изданиях. Публикация не должна нарушать закон об авторских правах. Направляя
свою рукопись в редакцию, авторы автоматически передают учредителям и
редколлегии неисключительные права на издание данной статьи на русском языке и
на ее распространение в России и за рубежом. При этом за авторами сохраняются
все права как собственников данной рукописи. В связи с этим авторами должно
быть представлено в редакцию письмо в следующей форме:
Соглашение о передаче права на публикацию:

\textit{<<Мы, нижеподписавшиеся, авторы рукописи <<$\qquad\qquad$>>, передаем
учредителям и редколлегии журнала <<Информатика и её применения>>
неисключительное право опубликовать данную рукопись статьи на русском языке как
в печатной, так и в электронной версиях журнала. Мы подтверждаем, что данная
публикация не нарушает авторского права других лиц или организаций. Подписи
авторов: (ф.\,и.\,о., дата, адрес)>>.}

Указанное соглашение может быть представлено 
как в бумажном виде, так и в виде отсканированной копии (с подписями авторов).


Редколлегия вправе запросить у авторов экспертное заключение о возможности
опубликования представленной статьи в открытой печати. %\\[-13.5pt]
\item Статья
подписывается всеми авторами. На отдельном листе представляются данные автора
(или всех авторов): фамилия, полные имя и отчество, телефон, факс, e-mail,
почтовый адрес. Если работа выполнена несколькими авторами, указывается фамилия
одного из них, ответственного за переписку с редакцией. %\\[-13.5pt]
\item Редакция журнала
осуществляет самостоятельную экспертизу присланных статей. Возвращение рукописи
на доработку не означает, что статья уже принята к печати. Доработанный вариант
с ответом на замечания рецензента необходимо прислать в редакцию. %\\[-13.5pt]
\item Решение
редакционной коллегии о принятии статьи к печати или ее отклонении сообщается
авторам. Редколлегия не обязуется направлять рецензию авторам отклоненной
статьи. %\\[-13.5pt]
\item Корректура статей высылается авторам для просмотра. Редакция
просит авторов присылать свои замечания в кратчайшие сроки. %\\[-13.5pt]
\item При
подготовке рукописи в MS Word рекомендуется использовать следующие настройки.
Параметры страницы: формат~--- А4; ориентация~--- книжная; поля (см): внутри~---
2,5, снаружи~--- 1,5, сверху~--- 2, снизу~--- 2, от края до нижнего
колонтитула~--- 1,3. Основной текст: стиль~--- <<Обычный>>: шрифт Times New
Roman, размер 14~пунктов, абзацный отступ~--- 0,5~см, 1,5 интервала,
выравнивание~--- по ширине. Рекомендуемый объем рукописи~--- не свыше
25~страниц указанного формата. Ознакомиться с шаблонами, содержащими примеры
оформления, можно по адресу в Интернете:
\textsf{http://www.ipiran.ru/journal/template.doc}.
\item К рукописи, предоставляемой в 2-х
экземплярах, обязательно прилагается электронная версия статьи (как правило, в
форматах MS WORD (.doc) или \LaTeX\ (.tex), а также~--- дополнительно~--- в
формате .pdf) на дискете, лазерном диске или по электронной почте. Сокращения
слов, кроме стандартных, не применяются. Все страницы рукописи должны быть
пронумерованы. %\\[-13.5pt]
\item Статья должна содержать следующую информацию на русском и
английском языках: название, Ф.И.О. авторов, места работы авторов и их
электронные адреса, подробные сведения об авторах, оформленные в соответствии с форматом, 
определяемым файлами {\sf http://www.ipiran.ru/journal/issues/2011\_05\_01/authors.asp} и 
{\sf http://www.ipiran.ru/journal/issues/2011\_01\_eng/authors.asp},
аннотация (не более 100~слов), ключевые слова. Ссылки на
литературу в тексте статьи нумеруются (в квадратных скобках) и располагаются в
порядке их первого упоминания. В~списке литературы не должно быть позиций, на которые нет ссылки в тексте статьи.
Все фамилии авторов, заглавия статей, названия
книг, конференций и~т.\,п.\ даются на языке оригинала, если этот язык
использует кириллический или латинский алфавит. %\\[-13.5pt]
\item Присланные в редакцию материалы авторам не возвращаются.
\item При отправке файлов по электронной
почте просим придерживаться следующих правил:
\begin{itemize}
\item указывать в поле subject (тема) название журнала и фамилию автора; %\\[-13.5pt]
\item использовать attach (присоединение); %\\[-13.5pt]
\item в случае больших объемов информации возможно
использование общеизвестных архиваторов (ZIP, RAR); %\\[-13.5pt]
\item в состав электронной версии статьи должны входить: файл, содержащий текст статьи, и файл(ы),
содержащий(е) иллюстрации. %\\[-13.5pt]
\end{itemize}
\item Журнал <<Информатика и её применения>> является некоммерческим изданием. 
Плата за публикацию с авторов не взимается, гонорар авторам не выплачивается.
\end{enumerate}
\thispagestyle{empty}
\textbf{Адрес редакции:} Москва 119333,
ул.~Вавилова, д.~44, корп.~2, ИПИ РАН\\
\hphantom{\textbf{Адрес редакции:} }Тел.: +7 (499) 135-86-92\ \
Факс:  +7 (495) 930-45-05\ \  E-mail:   rust@ipiran.ru }
}

%\include{ipi-ind}

%\tableofcontents

\end{document}

%\tableofcontents

%\end{document}

%\tableofcontents


\end{document}

\newcommand{\Ack}{\subsection*{\protect\large\bf Acknowledgments}}

\vphantom*{\int\limits_0^T}

{ \begin{center}  %fig1
 \vspace*{6pt}
    \mbox{%
 \epsfxsize=79mm 
 \epsfbox{gru-1.eps}
 }

\end{center}



\noindent
{{\figurename~1}\ \ \small{
}}}

%\vspace*{6pt}

\addtocounter{figure}{1}

$\acute{\mbox{о}}$

\linebreak