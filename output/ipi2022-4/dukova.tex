
\def\stat{dukova}

\def\tit{О СЛОЖНОСТИ ОБУЧЕНИЯ ЛОГИЧЕСКИХ ПРОЦЕДУР КЛАССИФИКАЦИИ}

\def\titkol{О сложности обучения логических процедур классификации}

\def\aut{Е.\,В.~Дюкова$^1$, А.\,П.~Дюкова$^2$}

\def\autkol{Е.\,В.~Дюкова, А.\,П.~Дюкова}

\titel{\tit}{\aut}{\autkol}{\titkol}

\index{Дюкова Е.\,В.}
\index{Дюкова А.\,П.}
\index{Djukova E.\,V.}
\index{Djukova A.\,P.}


%{\renewcommand{\thefootnote}{\fnsymbol{footnote}} \footnotetext[1]
%{Работа выполнена при поддержке РФФИ (проект 20-07-00804).}}


\renewcommand{\thefootnote}{\arabic{footnote}}
\footnotetext[1]{Федеральный исследовательский центр <<Информатика и~управление>> Российской академии наук, 
\mbox{edjukova@mail.ru}}
\footnotetext[2]{Федеральный исследовательский центр <<Информатика и~управление>. Российской академии 
наук, \mbox{anastasia.d.95@gmail.com}}


\vspace*{-12pt}


  \Abst{Исследуются вопросы сложности логического анализа целочисленных данных. Для 
специальных задач поиска частых и~нечастых элементов в~данных, на решении которых 
базируется обучение логических процедур классификации, приведены асимптотики 
типичного числа решений. Технические основы получения указанных оценок опираются на 
методы получения аналогичных оценок для труднорешаемой дискретной задачи построения 
(перечисления) тупиковых покрытий целочисленной матрицы, формулируемой в~работе как 
задача поиска <<минимальных>> нечастых элементов. Новые результаты в~основном 
касаются изучения метрических (количественных) свойств частых элементов. Полученные 
оценки типичного числа часто встречающихся фрагментов в~описаниях прецедентов 
позволяют сделать вывод о~перспективности применения алгоритмов поиска таких 
фрагментов на этапе обучения логических классификаторов типа <<Кора>>.}
  
  \KW{атрибут; частый элементарный фрагмент; нечастый элементарный фрагмент; 
монотонная дуализация; тупиковое покрытие целочисленной матрицы; классификация по 
прецедентам; классификатор типа <<Кора>>}

\DOI{10.14357/19922264220409} 
  
%\vspace*{-3pt}


\vskip 10pt plus 9pt minus 6pt

\thispagestyle{headings}

\begin{multicols}{2}

\label{st\stat}
  
\section{Введение}

  Рассматриваемые задачи поиска частых и~не\-час\-тых элементов в~данных 
формулируются сле\-ду\-ющим образом.
  
  Исследуется множество объектов~$M$. Каждый объект из~$M$ может быть 
представлен в~виде числового вектора, полученного на основе наблюдения или 
измерения ряда его характеристик. Такие характеристики называют 
атрибутами. Предполагается, что каждый атрибут имеет ограниченное 
множество допустимых значений, которые кодируются целыми числами.
  
  Пусть $X=\{x_1, \ldots , x_n\}$~--- заданное множество атрибутов;   $H$~---
  набор 
из~$r$~различных атрибутов вида $H\hm= \{ x_{j_1}, \ldots , x_{j_r}\}$; 
$\sigma\hm= \left( \sigma_1, \ldots , \sigma_r\right)$~--- набор, 
в~котором~$\sigma_i$~--- допустимое значение признака~$x_{j_i}$, $i\hm= 1,2, 
\ldots, r$. Пару $(\sigma, H)$ назовем элементарным фрагментом (ЭФ) 
длины~$r$. Через $W(M,X)$ обозначим множество всех ЭФ.
  
  Пусть $S=(a_1,\ldots , a_n)$~--- объект из~$M$ (здесь~$a_j$, $j\hm\in \{1, 2 
\ldots ,n\}$,~--- значение атрибута~$x_j$ для объекта~$S$). Будем говорить, 
что~$S$ содержит ЭФ $(\sigma, H)$, если $a_{j_i}\hm= \sigma_i$ при $i\hm= 
1,2,\ldots , r$.
  
  Дана некоторая совокупность объектов~$D$ из~$M$ и~задано число~$p$, 
$0\hm< p\hm\leq 1$. Объекты в~$D$ не обязательно различны. Через $\vert 
D\vert$ обозначается чис\-ло объектов в~$D$. Элементарный фрагмент $(\sigma, 
H)$, $(\sigma,H)\hm\in W(M,X)$, называется $(p,D)$-\textit{час\-тым}, если не 
менее $p\vert D\vert$ объектов %~$S^\prime$ 
из~$D$ содержат $(\sigma, H)$. 
Иначе ЭФ $(\sigma, H)$~--- $(p,D)$-\textit{не\-час\-тый}. Элементарный 
фрагмент $(\sigma, H)$, $(\sigma, H)\hm\in W(M,X)$, называется  
\mbox{$(0,D)$-\textit{не}}\-\textit{час\-тым}, если ни один объект из~$D$ не содержит 
$(\sigma, H)$.
  
  Элементарный фрагмент $(\sigma, H)$, являющийся $(p,D)$-час\-тым 
в~$W(M,X)$, называется \textit{максимальным} $(p,D)$-час\-тым в~$W(M,X)$, 
если любой ЭФ $(\sigma^\prime, H^\prime)$ из~$W(M,X)$, такой что 
$\sigma^\prime \hm\supset \sigma$, $H^\prime \hm\supset H$, не является  
$(p,D)$-час\-тым.
  
  Элементарный фрагмент $(\sigma, H)$ длины~$r$ называется 
\textit{правильным} в~$W(M,X)$, если $(\sigma, H)$~---   
$(r/\vert D\vert, D)$-час\-тый в~$W(M,X)$.
  
  Элементарный фрагмент  $(\sigma, H)$, являющийся $(p,D)$-не\-час\-тым 
в~$W(M,X)$, называется \textit{минимальным} $(p,D)$-не\-час\-тым 
в~$W(M,X)$, если любой ЭФ $(\sigma^\prime, H^\prime)$ из~$W(M,X)$, такой 
что $\sigma^\prime \hm\subset \sigma$, $H^\prime\hm\subset H$, не является 
$(p,D)$-не\-час\-тым. Понятие \textit{минимального} $(0,D)$-не\-час\-то\-го ЭФ 
полностью аналогично введенному понятию минимального  
$(p,D)$-не\-час\-то\-го ЭФ для $p\hm>0$.
  
  Возникают две отдельные задачи: 
  \begin{enumerate}[(1)]
  \item для заданного~$p$, $0\hm< p\hm\leq 1$, 
найти в~$W(M,X)$ все (максимальные) $(p,D)$-час\-тые ЭФ; 
\item для заданного 
$p$, $0\hm\leq p\leq 1$, найти все (минимальные) $(p,D)$-не\-час\-тые ЭФ. 
\end{enumerate}
Иногда требуется совместное перечисление максимальных частых 
и~минимальных не\-час\-тых ЭФ.

  
  Задачи поиска частых и~нечастых элементов в~данных входят в~число 
центральных задач интеллектуального анализа данных и~особенно важны 
в~случае больших данных. Эти задачи актуальны для многих прикладных 
областей, среди которых следует выделить нахождение в~данных 
ассоциативных правил и~машинное обучение.
  
  В первом случае~$D$ называют базой данных, а~каж\-дый объект из~$D$ 
называют транзакцией. Ассоциативное правило (АП) устанавливает 
зависимость между двумя частыми ЭФ, согласно которой один частый ЭФ~$X$ 
(посылка) с~некоторой <<до\-сто\-вер\-ностью>> влечет другой частый 
ЭФ~$Y$. При этом ЭФ~$X$ и~$Y$ порождаются одним общим частым ЭФ, 
обозначаемым $(X,Y)$. Наиболее информативными считаются те АП, которые 
порождаются максимальными частыми ЭФ $(X,Y)$ с~<<минимальной>> 
посылкой~$X$. Вопросы поиска ассоциативных правил возникли в~связи 
с~анализом потребительской корзины и~наиболее изучены для случая 
бинарных данных~[1].
  
  Одна из главных задач машинного обучения~--- задача классификации на 
основе прецедентов. В~этом случае~$D$~--- обучающая выборка (некоторая 
заданная совокупность примеров объектов из~$M$), а каждый объект  
из~$D$~--- обуча\-ющий объект, или прецедент. Подлежащие измерению или 
наблюдению свойства исследуемых объектов называются признаками. 
В~самом простом случае прецеденты делятся на два класса (класс 
положительных и~класс отрицательных примеров). В~общем случае число 
классов может быть больше двух. Требуется по признаковому описанию 
предъявленного объекта, о котором заранее не известно, к~какому классу он 
относится, определить (распознать) этот класс.
  
  Хорошие результаты показывают логические классификаторы, при 
конструировании которых используются основные идеи как алгоритма 
<<Кора>>~[2], так и~алгоритмов вычисления оценок~[3]. Эти классификаторы 
впервые предложены в~[4]. В~дальнейшем подход развивался в~работах~[5--7] 
и~др. В~алгоритмах типа <<Кора>> анализ прецедентной информации 
проводится в~предположении, что признаковые описания любых двух 
обуча\-ющих объектов, принадлежащих разным классам, не совпадают. На этапе 
обуче\-ния для каждого класса~$K$ ищутся так называемые  
$(p,q)$-пред\-ста\-ви\-тель\-ные элементарные классификаторы, 
пред\-став\-ля\-ющие собой специальные ЭФ из~$W(M,X)$.
  
  Пусть $Q(K)$ и~$Q(\overline{K})$~--- множества прецедентов из класса~$K$ 
и~не из класса~$K$ соответственно и~$p\hm>0$, $q\hm<p$. Тогда  
$(p,q)$-пред\-ста\-ви\-тель\-ный элементарный классификатор является 
одновременно (максимальным) $(p,Q(K))$-час\-тым ЭФ и~(минимальным) 
$(q,Q(\overline{K}))$-не\-час\-тым ЭФ в~$W(M,X)$. Как правило, сначала 
строятся минимальные $(q,Q(\overline{K}))$-не\-час\-тые ЭФ, а~затем из них 
отбираются те, которые являются $(p,Q(K))$-час\-ты\-ми. На следующем этапе 
найденные $(p,q)$-пред\-ста\-ви\-тель\-ные элементарные классификаторы 
класса~$K$ участвуют в~процедуре <<голосования>> за отнесение 
распознаваемого объекта к~этому классу. Материал обучения без\-оши\-боч\-но 
классифицируется при $q\hm=0$. Однако нахождение  
$(0,Q(\overline{K}))$-не\-час\-тых ЭФ требует больших вычислительных 
затрат. В~простейшем случае бинарных данных, когда требуется найти все 
минимальные $(0,Q(\overline{K}))$-не\-час\-тые ЭФ вида $(0, \ldots , 0)$, это 
известная труднорешаемая перечислительная задача, называемая монотонной 
дуализацией~[8, 9].
  
  Отметим, что к~задаче совместного перечисления максимальных частых 
  и~минимальных не\-час\-тых ЭФ сводится задача расшифровки монотонной 
функции~[10].
  
  В целях исследования скорости решения рассматриваемых задач в~случае 
больших данных представляет интерес получение асимптотик типичного числа 
частых и~нечастых ЭФ, а~также типичной длины таких ЭФ.  
В~работах~\cite{5-duk, 6-duk} искомые оценки приведены для множества 
минимальных $(0,Q(\overline{K}))$-не\-час\-тых ЭФ, называемых в~этих 
работах тупиковыми покрытиями целочисленной матрицы. В~настоящей 
работе так же, как и~в~\cite{5-duk, 6-duk}, рассмотрен случай большого чис\-ла 
атрибутов. Требуемые оценки получены для правильных ЭФ в~предположении, 
что каж\-дый атрибут имеет~$k$, $k\hm \geq 2$, до\-пус\-ти\-мых значений. При 
доказательстве основной тео\-ре\-мы использовано мат\-рич\-ное пред\-став\-ле\-ние 
данных. Сравнение полученных в~на\-сто\-ящей работе оценок с~оценками  
из~\cite{5-duk, 6-duk} свидетельствует о~перспективности применения методов 
поиска час\-тых ЭФ для построения $(p,0)$-пред\-ста\-ви\-тель\-ных 
элементарных классификаторов.
  
\section{Асимптотика типичного числа правильных элементарных 
фрагментов в~случае большого числа атрибутов}

  Пусть~$L\hm= (a_{ij})$, $i\hm= 1, \ldots , m$, $j\hm= 1, \ldots , n$,~--- 
матрица с~элементами из $\{0,1,\ldots , k-1\}$, $k\hm\geq 2$; $E_k^r$, $r\hm\leq 
n$, $k\hm \geq 2$,~--- множество наборов $(\sigma_1, \ldots , \sigma_r)$, 
$\sigma_i \hm\in \{0,1,\ldots , k-1\}$,$i\hm= 1,2,\ldots , r$; $W_r^n$, $r\hm\leq 
n$,~--- множество всех наборов вида $\{ j_1, \ldots , j_r\}$, где $j_t\hm\in 
\{1,2,\ldots , n\}$ при $t\hm= 1,2,\ldots , r$ и~$j_1< \cdots <j_r$.
  
  Положим $\sigma\hm\in E_k^r$, $\sigma\hm= (\sigma_1, \ldots , \sigma_r)$, 
$H\hm\in W_r^n$, $H\hm= \{ j_1, \ldots , j_r\}$. Будем говорить, что строка 
$(a_{i1}, \ldots , a_{in})$ матрицы~$L$ содержит \textit{фрагмент} $(\sigma, H)$, 
если $a_{ij_t}\hm= \sigma_t$ при $t\hm=1,2,\ldots , r$. Число~$r$ назовем 
\textit{длиной} фрагмента $(\sigma, H)$. Фрагмент $(\sigma,H)$ длины~$r$ 
назовем \textit{правильным} в~$L$, если не менее~$r$~строк в~$L$ содержат 
$(\sigma,H)$.
  
  Набор столбцов матрицы~$L$ с~номерами из~$H$ называется 
\textit{тупиковым}-\textit{покрытием длины}~$r$, если ни одна строка в~$L$ 
не содержит фрагмент $(\sigma, H)$ и~для любого $t\hm\in 1,2, \ldots , r$ в~$L$ найдется строка, 
содержащая фрагмент $(\gamma, H)$, где $\gamma\hm= (\sigma_1, \ldots 
,\sigma_{t-1}, \overline{\sigma}_t, \sigma_{t+1}, \ldots , \sigma_r)$, 
$\overline{\sigma}_t\hm\in \{0,1,\ldots , k-1\}$, $\overline{\sigma}_t \not= 
\sigma_t$.
  
  Введем обозначения: $\mathfrak{M}^k_{mn}$~--- множество всех матриц 
размера $m\times n$ с~элементами из $\{ 0,1,\ldots\linebreak \ldots , k-1\}$, $k\hm\geq 2$; 
$R(L)$~--- множество правильных фрагментов в~матрице~$L$ 
из~$\mathfrak{M}^k_{mn}$; $T(L,\sigma)$~--- множество ту\-пи\-ко\-вых-по\-кры\-тий 
матрицы~$L$ из~$\mathfrak{M}_{mn}^k$; $T(L)\hm= \cup^n_{r=1} 
\cup_{\sigma\in E_k^r} T(L,\sigma)$; $b_n\sim c_n$, $n\hm\to \infty$, означает, 
что $\lim_{n\to \infty} b_n/c_n\hm=1$; $b_n\hm\leq_{\!\!\!n}\ c_n$ означает, что 
$b_n\hm\leq c_n$ при всех достаточно больших~$n$; $\vert N\vert$~--- 
мощность множества~$N$; $\phi_k(m)$~--- интервал
$\left( 0{,}5\log_k mn -0{,}5\log_k \log_k mn -\log_k\log_k\log_k n\right.$,\linebreak
$\left.0{,}5\log_k mn- 0{,}5\log_k \log_k mn+\log_k\log_k\log_k n\right)$.
  
  В настоящей работе получены асимптотики типичных значений величины 
$\vert R(L)\vert$ и~длины фрагмента из~$R(L)$. Выявление типичной ситуации 
связано с~высказыванием типа <<для почти всех матриц~$L$ 
из~$\mathfrak{M}^k_{mn}$ при $n\hm\to \infty$ выполнено $F_1(L)\sim 
F_2(L)$>> (здесь $F_1(L)$ и~$F_2(L)$~--- два функционала, заданные на 
матрицах из~$\mathfrak{M}^k_{mn}$). Данное высказывание означает, что 
существуют две положительные бесконечно убывающие функции~$\alpha(n)$ 
и~$\beta(n)$ такие, что для всех достаточно больших~$n$ имеет место $1\hm- 
\vert \mathfrak{M}\vert/\vert \mathfrak{M}^k_{mn}\vert \hm\leq \alpha(n)$, 
где~$\mathfrak{M}$~--- множество таких матриц~$L$ 
в~$\mathfrak{M}^k_{mn}$, для которых\linebreak $1\hm- \beta(n)\hm< \vert F_1(L)\vert 
/\vert F_2(L)\vert \hm< 1\hm+ \beta(n)$.
  
  Аналогичные результаты для множества $T(L)$ получены  
в~\cite{5-duk, 6-duk} и~приведены ниже (теорема~1) для полноты изложения.
  
  \smallskip
  
  \noindent
  \textbf{Теорема~1} (см.~\cite{5-duk, 6-duk}). \textit{Если $m^a\hm\leq 
n\hm\leq k^m$, $a\hm> 1$, то для почти всех матриц~$L$ 
из~$\mathfrak{M}^k_{mn}$ справедливо}
  $$
  \vert T(L)\vert \sim \sum\limits_{r\in \phi_k(m)} C_n^r C^r_m r! (k-1)^r k^{r-
r^2}\,,\ n\to\infty\,,
  $$
\textit{и длины почти всех тупиковых покрытий из $T(L)$ принадлежат 
интервалу}~$\phi_k(m)$.

\smallskip
  
  \noindent
  \textbf{Теорема~2.} \textit{Если $m^a\hm\leq n\hm\leq k^m$, $a\hm>1$, 
$k\hm\geq 2$, то для почти всех матриц~$L$ из~$\mathfrak{M}^k_{mn}$ 
справедливо}
  $$
  \vert R(L)\vert \sim\sum\limits_{r\in \phi_k(m)} C_n^r C_m^r k^{r-r^2}\,,\enskip 
n\to \infty\,,
  $$
\textit{и длины почти всех фрагментов из~$R(L)$ принадлежат 
интервалу}~$\phi_k(m)$.

\smallskip

  \noindent
  Д\,о\,к\,а\,з\,а\,т\,е\,л\,ь\,с\,т\,в\,о\ тео\-ре\-мы~2 опирается на ряд приводимых 
ниже лемм.
   
  Обозначим через $V_r^m$ $(r\hm\leq m)$ множество всех упорядоченных 
наборов вида $(i_1, \ldots , i_r)$, где $i_t\not= i_l$ при $t, l \hm= 1,2,\ldots , r$.
  
  Пусть $v\hm\in V_r^m$, $H\hm\in W_r^n$, $\sigma \hm\in E_k^r$. Мат\-ри\-ца 
$L\hm\in \mathfrak{M}$ называется $(v,\sigma, H)$-мат\-ри\-цей, если каждая 
строка с~номером~$v$ содержит фрагмент $(\sigma, H)$. Матрица $L\hm\in 
\mathfrak{M}^k_{mn}$, являющаяся $(v,\sigma, H)$-мат\-ри\-цей, называется 
правильной, если из условия $v_1\hm\in V_r^m$, $v_1\not= v$, следует, 
что~$L$ не является $(v_1,\sigma,H)$-мат\-ри\-цей. Обозначим 
через~$N_{(v,\sigma,H)}$ совокупность $(v,\sigma, H)$-мат\-риц 
в~$\mathfrak{M}_{mn}^k$, через~$N^*_{(v,\sigma,H)}$~--- совокупность всех 
правильных $(v,\sigma,H)$-мат\-риц в~$\mathfrak{M}^k_{mn}$.
  
  \smallskip
  
  \noindent
  \textbf{Лемма~1.}\  \textit{Если $v\hm\in V_r^m$, $H\hm\in W_r^n$, 
$\sigma\hm\in E_k^r$, то}
  $$
  \left\vert N_{(v,\sigma,H)} \right\vert =k^{mn-r^2}.
  $$
  
  \noindent
  Д\,о\,к\,а\,з\,а\,т\,е\,л\,ь\,с\,т\,в\,о\,.\ \  Оценим, сколькими способами 
можно построить матрицу~$L$ из~$N_{(v,\sigma,H)}$. Однозначным образом 
определяются те элементы мат\-ри\-цы~$L$, которые расположены на 
пересечении строк с~номерами из~$v$ и~столбцов с~номерами из~$H$. 
Остальные элементы этой матрицы могут быть выбраны произвольным 
образом. Таким образом, элементы матрицы~$L$, расположенные в~столбцах 
с~номерами, не входящими в~$H$, могут быть выбраны $k^{m(n-r)}$ 
способами, а~строки подматрицы матрицы~$L$, образованной столбцами 
из~$H$, можно выбрать $k^{r(m-r)}$ способами. Отсюда получаем требуемую 
оценку. Лемма доказана.
  
  \smallskip
  
  \noindent
  \textbf{Лемма~2.}\ \textit{Если $v\hm\in V_r^m$, $H\hm\in W_r^n$, $\sigma 
\hm\in E_k^r$, то}
  $$
  \left\vert N^*_{(v,\sigma, H)} \right\vert = \left(1-k^{-r}\right)^{m-r} k^{mn-r^2}.
  $$
  
  \noindent
  Д\,о\,к\,а\,з\,а\,т\,е\,л\,ь\,с\,т\,в\,о\,.\ \ Оценим, сколькими способами 
можно построить матрицу~$L$ из $N^*_{(v,\sigma,H)}$. Элементы этой 
мат\-ри\-цы, расположенные в~столбцах с~номерами, не входящими в~$H$, могут 
быть выбраны произвольным образом ($k^{m(n-r)}$ способов). Отсюда, 
учитывая, что строки в~под\-мат\-ри\-це мат\-ри\-цы~$L$, образованной столбцами 
из~$H$, можно выбрать $(k^r\hm-1)^{m-r}$ способами, получаем тре\-бу\-емую 
оценку. Лемма доказана.
  
  \smallskip
  
  \noindent
  \textbf{Лемма~3.}\  \textit{Пусть $v_1\hm\in V_r^m$, $v_2\hm\in 
V_l^m$, $H_1\hm\in W_r^n$, $H_2\hm\in W_l^n$, $\sigma^\prime \hm\in E_k^r$, 
$\sigma^{\prime\prime} \hm\in E_k^l$ и~наборы~$v_1$ и~$v_2$ пересекаются по $a$ $(a\hm\geq 
0)$ элементам, а наборы~$w_1$ и~$w_2$ пересекаются по $b$ $(b\hm\geq 0)$ 
элементам. Тогда}
  $$
  \left\vert N_{(v_1,\sigma^\prime, H_1)} \cap N_{(v_2, \sigma^{\prime\prime}, 
H_2)} \right\vert \leq k^{mn-r^2-l^2+ab}.
  $$
  
  \noindent
  Д\,о\,к\,а\,з\,а\,т\,е\,л\,ь\,с\,т\,в\,о\ леммы~3 не приводится в~силу ее 
очевидности.
  
  \smallskip
  
  \noindent
  \textbf{Лемма~4.} \textit{Если $m^a\hm\leq n\hm\leq k^m$, $a\hm>1$, 
$k\hm\geq 2$, то имеет место}
  $$
  \sum\limits^m_{r=1} C_n^r C_m^r k^{r-r^2}\sim \sum\limits_{r\in \phi_k(m)} 
C_n^r C_m^r k^{r-r^2}\,,\ n\to \infty\,.
  $$
  
  
  \noindent
  Д\,о\,к\,а\,з\,а\,т\,е\,л\,ь\,с\,т\,в\,о\  леммы~4 аналогично 
доказательству леммы~4 из работы~\cite{6-duk}.
  
  \smallskip
  
  \noindent
  \textbf{Лемма~5.}\ \textit{Если $r,l\hm\leq c\log_k n$, $c\hm<1$, то имеет 
место}
  $$
  \sum\limits_{m=0}^{\min(r,l)} k^{lb} C_n^r C_r^b C_{n-r}^{l-b} < C_n^r C_n^l 
(1+\delta(n)),
  $$
\textit{где} $\delta(n)\hm\to 0$ при $n\hm\to \infty$.

\smallskip

  \noindent
  Д\,о\,к\,а\,з\,а\,т\,е\,л\,ь\,с\,т\,в\,о\   леммы~5 аналогично 
доказательству леммы~5 из работы~\cite{6-duk}.
  
  \smallskip
  
  Будем считать $\mathfrak{M}^k_{mn}\hm= \{L\}$ пространством 
элементарных событий, в~котором каждое событие~$L$ происходит 
с~вероятностью $1/\vert \mathfrak{M}^k_{mn}\vert$. Математическое ожидание случайной величины~$X(L)$, 
определенной на множестве~$\mathfrak{M}^k_{mn}$, будем обозначать через 
${\sf M}X(L)$, дисперсию~--- через ${\sf D}X(L)$.
  
  \smallskip
  
  \noindent
  \textbf{Лемма~6} (см.~\cite{11-duk}). \textit{Пусть для случайных величин 
$X_1(L)$ и~$X_2(L)$, определенных на~$\mathfrak{M}^k_{mn}$, выполнено 
$X_1(L)\hm\geq X_2(L)\hm\geq 0$ и~при $n\hm\to \infty$ верно ${\sf M}X_1(L)\sim 
{\sf M}X_2(L)$, ${\sf D}X_2(L)/({\sf M}X_2(L))^2\hm\to 0$. Тогда для почти всех матриц~$L$ из 
$\mathfrak{M}^k_{mn}$ имеет место} $X_1(L)\sim X_2(L)\sim {\sf M}X_2(L)$, 
$n\hm\to \infty$.
  
  \smallskip
  
  Пусть $v\in V_r^m$, $H\hm\in W_r^n$, $\sigma\hm\in E_k^r$. На 
$\mathfrak{M}^k_{mn}\hm= \{L\}$ рассмотрим случайную 
величину~$\zeta_{(\sigma,H)}(L)$, равную~1, если $(\sigma,H)$~--- правильный 
фрагмент в~$L$, и~равную~0 в~противном случае. Положим
  \begin{align*}
  \zeta_1(L)&= \sum\limits_{r=1}^{\min(m,n)} \sum\limits_{\substack{{v\in V_r^m}\\ {H\in  W_r^n}}}
\sum\limits_{\sigma\in E_k^r} \zeta_{(\sigma,H)}(L)\,;\\
  \zeta_2(L) &= \sum\limits_{r\in \phi_k(m)}
   \sum\limits_{\substack{{v\in V_r^m}\\ {H\in  W_r^n}}} \sum\limits_{\sigma\in E_k^r}\zeta_{(\sigma,H)} (L)\,.
  \end{align*}
  %
  Нетрудно видеть, что $\zeta_1(L)\hm= \vert R(L)\vert$, а~$\zeta_2(L)$~--- 
число тех фрагментов в~$R(L)$, длины которых принадлежат 
интервалу~$\phi_k(m)$.
  
  Оценим вероятность события $\zeta_{(\sigma,H)}(L)\hm=1$, обозначаемую 
далее через ${\sf P}(\zeta_{(\sigma,H)}(L)\hm=1)$. Очевидно, в~силу леммы~1
  \begin{equation}
  {\sf P}\left( \zeta_{(\sigma,H)}(L)=1\right) \leq \fr{\left\vert 
N_{(v,\sigma,H)}\right\vert }{\left \vert \mathfrak{M}^k_{mn}\right\vert } =k^{-
r^2}.
  \label{e1-duk}
\end{equation}
 % 
  С другой стороны, в~силу леммы~2 имеем
  $$
  {\sf P}\left( \zeta_{(\sigma,H)}(L)=1\right) \geq \fr{\left\vert 
N^*_{(v,\sigma,H)}\right\vert }{\left\vert \mathfrak{M}^k_{mn}\right\vert } \geq (1-k^{-
r})^{m-r} k^{-r^2}.
  $$
  
  В случае $r\in \phi_k(m)$ получаем 
\begin{align*}
  \left(1- k^{-r}\right)^{m-r}&\geq \left(1- k^{-r}\right)^m \geq 1- mk^{-r};
\\
mk^{-r} &\leq  \fr{c_1 \log_k^2 n}{n^{c_2}},\enskip c_1,  c_2>0.
\end{align*}
 Следовательно,
 
 \noindent
  \begin{equation}
  {\sf P}\left( \zeta_{(\sigma,H)}(L)=1\right)\geq F(n) k^{-r^2},
  \label{e2-duk}
 \end{equation}
 
 \vspace*{-2pt}
 
 \noindent
где $F(n)\hm\to 1$ при $n\hm\to \infty$.
  
  
  \smallskip
  
  \noindent
  \textbf{Лемма~7.}\  \textit{Если $m^a\hm\leq n\hm\leq k^m$, $a\hm>1$, 
$k\hm\geq 2$, то}
  $$
  {\sf M}\zeta_1(L)\sim {\sf M}\zeta_2(L)\sim\sum\limits_{r\in \phi_k(m)} C_n^r C_m^r k^{r-
r^2},\ n\to \infty\,.
  $$
  
  \noindent
  Д\,о\,к\,а\,з\,а\,т\,е\,л\,ь\,с\,т\,в\,о\,.\ \ Имеем
  \begin{align*}
  {\sf M}\zeta_1(L) &= \sum\limits^m_{r=1} \sum\limits_{\substack{{v\in V_r^m}\\ {H\in W_r^n}}} 
\sum\limits_{\sigma\in E_k^r} {\sf P}\left( \zeta_{(\sigma,H)}(L)=1\right);\\
  {\sf M}\zeta_2(L)&= \sum\limits_{r\in \phi_k(m)}
  \sum\limits_{\substack{{v\in V_r^m}\\ {H\in  W_r^n}}} \sum\limits_{\sigma\in E_k^r} {\sf P}\left( \zeta_{(\sigma,H)}(L)=1\right).
  \end{align*}
%  

 \vspace*{-2pt}
 
 \noindent
  Следовательно, в~силу~(\ref{e1-duk})
  \begin{equation}
  {\sf M}\zeta_2(L)\leq {\sf M}\zeta_1(L)\leq \sum\limits^m_{r=1} C_n^r C_m^r k^{r-r^2}.
  \label{e3-duk}
  \end{equation}
  %
  
   \vspace*{-2pt}
 
 \noindent
  В силу~(\ref{e2-duk})
  \begin{equation}
  {\sf M}\zeta_1(L)\geq {\sf M}\zeta_2(L)\geq F(n) \sum\limits_{r\in \phi_k(m)} C_n^r C_m^r 
k^{r-r^2},
  \label{e4-duk}
  \end{equation}
  
   \vspace*{-2pt}
 
 \noindent
где $F(n)\to 1$ при $n\hm\to \infty$.

  Из~(3), (4) и~леммы~4 сразу следует утверждение доказываемой леммы.
  
  \smallskip
  
  \noindent
  \textbf{Лемма~8.}\ \textit{Имеет место}
  $$
  \fr{{\sf D}\zeta_2(L)}{({\sf M}\zeta_2(L))^2} \to 0\,,\enskip n\to \infty\,.
  $$
  
  \noindent
  Д\,о\,к\,а\,з\,а\,т\,е\,л\,ь\,с\,т\,в\,о\,.\ \ Имеем
\begin{equation}
{\sf D}\zeta_2(L) ={\sf M}\left( \zeta_2(L)\right)^2 -\left( {\sf M}\zeta_2(L)\right)^2.
\label{e5-duk}
\end{equation}
  %
  
   \vspace*{-2pt}
 
 \noindent
  Нетрудно видеть, что
$$
{\sf M}\left(\zeta_2(L)\right)^2 \leq \sum\limits_{r,l\in \phi_k(m)} 
\sum\limits_{\substack{{v_1\in V_r^m, v_2\in V_l^m}\\ {H_1\in W_r^n, H_2\in W_l^n}}} 
\sum\limits_{\substack{{\sigma^\prime \in E_k^r}\\
{\sigma^{\prime\prime}\in E_k^l}}} \fr{\vert N\vert}{k^{mn}},
$$

 \vspace*{-2pt}
 
 \noindent
где $N=N_{(v_1,\sigma^\prime, H_1)}\cap N_{(v_2, \sigma^{\prime\prime}, 
H_2)}$. Отсюда, пользуясь леммами~3 и~5, получаем

\noindent
\begin{multline}
{\sf M}\left( \zeta_2(L)\right)^2 \leq{}\\
{}\leq\!\! \!\sum\limits_{r,l\in \phi_k(m)} 
\sum\limits_{b=0}^{\min(r,l)} k^{r+l} k^{-r^2-l^2+lb} C_n^r C_r^b C_{n-r}^{l-b} 
C_m^r C_m^l\leq{}\\
{}\leq \sum\limits_{r,l\in \phi_k(m)} C_n^r C_n^l C_m^r C_m^l k^{r+l} k^{-r^2-
l^2} (1+\delta(n)),
\label{e6-duk}
\end{multline}
где $\delta(n)\to 0$ при $n\hm\to \infty$.
  
  С другой стороны, в~силу леммы~7
  \begin{multline}
  \left( {\sf M}\zeta_2(L)\right)^2 \sim \sum\limits_{r,l\in\phi_k(m)} C_n^r C_n^l C_m^r 
C_m^l k^{r+l} k^{-r^2-l^2}\,,\\ n\to \infty.
  \label{e7-duk}
  \end{multline}
  
  Из~(\ref{e5-duk})--(\ref{e7-duk}) следует утверждение доказываемой леммы.
  
  \smallskip
  
  Утверждение теоремы~2 следует непосредственно из лемм~6--8.
  
  \smallskip
  
  \noindent
  \textbf{Замечание.} В~задаче классификации по прецедентам рассмотрим 
класс~$K$ и~представим множества прецедентов $Q(K)$ и~$Q(\overline{K})$ 
в~виде матриц, имеющих размеры $m_1\times n$ и~$m_2\times n$ соответственно. 
Тогда, заменяя~$m$ на~$m_1$, в~качестве следствия из теоремы~2 получим 
асимптотики типичных значений чис\-ла и~длины правильных ЭФ в~$Q(K)$ 
и,~заменяя~$m$ на~$m_2$, в~качестве следствия из тео\-ре\-мы~1 получим 
аналогичные характеристики для $(0,Q(\overline{K}))$-не\-час\-тых ЭФ. 
Сравнение этих оценок свидетельствует об эффективности в~плане 
вычислительных затрат методов поиска частых ЭФ для построения 
представительных элементарных классификаторов в~алгоритмах типа 
<<Кора>> в~случае, когда~$m_2$ не меньше~$m_1$. Полученные оценки 
согласуются с~приведенными в~\cite{12-duk} результатами 
экспериментальных исследований.
  
\section{Заключение}

  Для множества объектов, описываемых в~виде наборов значений атрибутов 
(измеряемых или наблюдаемых характеристик объектов), исследованы 
метрические (количественные) свойства часто и~нечасто встречающихся частей 
описаний объектов (ЭФ). Рассмотрены специальные 
виды час\-тых и~не\-час\-тых ЭФ и~приведены асимптотики 
типичных значений их чис\-ла и~длины. Подобные оценки ранее были известны 
исключительно для множества решений дискретной перечислительной задачи, 
называемой монотонной дуализацией, а~так\-же некоторых обобщений этой 
задачи. Име\-ют\-ся в~виду задачи построения тупиковых покрытий булевых 
и~целочисленных матриц. Полученные результаты свидетельствуют 
о~це\-ле\-со\-об\-раз\-ности (в~плане сокращения временн$\acute{\mbox{ы}}$х затрат) применения 
методов поиска час\-тых ЭФ при по\-стро\-ении логических 
процедур классификации по прецедентам, ба\-зи\-ру\-ющих\-ся на решении задачи 
монотонной дуализации.
  
{\small\frenchspacing
 {%\baselineskip=10.8pt
 %\addcontentsline{toc}{section}{References}
 \begin{thebibliography}{99}
  \bibitem{1-duk}
  \Au{Aggarwal C.} Frequent pattern mining.~--- Heidelberg:  Springer, 2014. 467~p.
  \bibitem{2-duk}
  \Au{Вайнцвайг М.\,Н.} Алгоритм обучения распознаванию образов <<Кора>>~// 
Алгоритмы обучения распознаванию образов~/ Под ред. В.\,Н.~Вапника.~--- М.: Советское 
радио, 1973. С.~110--116.
  \bibitem{3-duk}
  \Au{Журавлёв Ю.\,И.} Об алгебраическом подходе к~решению задач распознавания 
и~классификации~// Проб\-ле\-мы кибернетики, 1978. Вып.~33. С.~5--68.
  \bibitem{4-duk}
  \Au{Баскакова Л.\,В., Журавлёв~Ю.\,И.} Модель распознающих алгоритмов 
с~представительными наборами и~сис\-те\-ма\-ми опорных множеств~// Ж.~вычисл. матем. 
матем. физ., 1981. Т.~21. №\,5. С.~1264--1275.
  \bibitem{5-duk}
  \Au{Дюкова Е.\,В.} Алгоритмы распознавания типа <<Кора>>: сложность реализации 
  и~метрические свойства~// Распознавание, классификация, прогноз.~--- М.: Наука, 1989. Вып.~2. С.~99--125.
  \bibitem{6-duk}
  \Au{Дюкова Е.\,В., Журавлев~Ю.\,И.} Дискретный анализ признаковых описаний 
в~задачах распознавания большой раз\-мер\-ности~// Ж.~вычисл. матем. матем. физ., 2000. 
Т.~40. №\,8. С.~1264--1278.
  \bibitem{7-duk}
  \Au{Дюкова Е.\,В., Песков Н.\,В.} Поиск информативных фрагментов описаний объектов 
в~дискретных процедурах распознавания~// Ж.~вычисл. матем. матем. физ., 2002. Т.~42. 
№\,5. С.~741--753.
  \bibitem{8-duk}
  \Au{Fredman M.\,L., Khachiyan~L.} On the complexity of dualization of monotone disjunctive 
normal forms~// J.~Algorithm., 1996. Vol.~21. No.\,3. P.~618--628.
  \bibitem{9-duk}
  \Au{Дюкова Е.\,В., Прокофьев~П.\,А.} Об асимптотически оптимальных алгоритмах 
дуализации~// Ж.~вычисл. матем. матем. физ., 2015. Т.~55. №\,5. С.~895--910.
  \bibitem{10-duk}
  \Au{Алексеев В.\,Б.} О~расшифровке некоторых классов монотонных многозначных 
функций~// Ж.~вычисл. матем. матем. физ., 1976. Т.~16. №\,1. С.~189--198.
  \bibitem{11-duk}
  \Au{Носков В.\,Н., Слепян~В.\,А.} О~чис\-ле тупиковых тестов для одного класса  
таб\-лиц~// Кибернетика, 1972. №\,1. С.~60--65.
  \bibitem{12-duk}
  \Au{Dragunov N., Djukova~E., Djukova~A.} Supervised classification and finding frequent 
elements in data~// 8th Conference (International) on Information Technology and 
Nanotechnology Proceedings.~--- Piscataway, NJ, USA: IEEE, 2022. Art. 9848521. 5~p. doi: 
10.1109/ \mbox{ITNT55410}.2022.9848521.

\end{thebibliography}

 }
 }

\end{multicols}

\vspace*{-9pt}

\hfill{\small\textit{Поступила в~редакцию 30.09.22}}

%\vspace*{8pt}

%\pagebreak

\newpage

\vspace*{-28pt}

%\hrule

%\vspace*{2pt}

%\hrule

%\vspace*{-2pt}

\def\tit{ON THE COMPLEXITY OF~LOGICAL CLASSIFICATION LEARNING 
PROCEDURES}


\def\titkol{On the complexity of~logical classification learning 
procedures}


\def\aut{E.\,V.~Djukova and~A.\,P.~Djukova}

\def\autkol{E.\,V.~Djukova and~A.\,P.~Djukova}

\titel{\tit}{\aut}{\autkol}{\titkol}

\vspace*{-8pt}


\noindent
Federal Research Center ``Computer Science and Control'' of the Russian Academy of Sciences, 
44-2~Vavilov Str., Moscow 119333, Russian Federation


\def\leftfootline{\small{\textbf{\thepage}
\hfill INFORMATIKA I EE PRIMENENIYA~--- INFORMATICS AND
APPLICATIONS\ \ \ 2022\ \ \ volume~16\ \ \ issue\ 4}
}%
 \def\rightfootline{\small{INFORMATIKA I EE PRIMENENIYA~---
INFORMATICS AND APPLICATIONS\ \ \ 2022\ \ \ volume~16\ \ \ issue\ 4
\hfill \textbf{\thepage}}}

\vspace*{3pt} 
  


\Abste{The issues of integer data logical analysis complexity are investigated. For 
special tasks of searching in data for frequent and infrequent elements, on the 
solution of which logical supervised classification procedures are based, asymptotics 
of a typical number of solutions are given. The technical foundations for obtaining 
these estimates are based on methods for obtaining similar estimates for intractable 
discrete problem of constructing (enumerating) irredundant coverings of integer 
matrix formulated in the paper as the problem of finding ``minimal'' infrequent 
elements. The new results mainly concern the study of metric (quantitative) 
properties of frequent elements. The obtained estimates for the typical number of 
frequently occurring fragments in precedent descriptions allow one to conclude that the 
use of algorithms for finding such fragments at the stage of training logical classifiers 
of the ``Kora'' type is promising.}

\KWE{attribute; frequent elementary fragment; infrequent elementary fragment; 
monotone dualization; irredundant covering of integer matrix; supervised 
classification; classifier of ``Kora'' type}

\DOI{10.14357/19922264220409} 

\vspace*{-16pt}

% \Ack
 %   \noindent
  

\vspace*{12pt}

  \begin{multicols}{2}

\renewcommand{\bibname}{\protect\rmfamily References}
%\renewcommand{\bibname}{\large\protect\rm References}

{\small\frenchspacing
 {%\baselineskip=10.8pt
 \addcontentsline{toc}{section}{References}
 \begin{thebibliography}{99}
  \bibitem{1-duk-1}
\Aue{Aggarwal, C.} 2014. \textit{Frequent pattern mining}. Heidelberg: Springer. 
467~p.
  \bibitem{2-duk-1}
\Aue{Weinzweig, M.\,N.} 1973. Algoritm obucheniya ras\-po\-zna\-va\-niyu obrazov 
``Kora''  [Algorithm for learning \mbox{pattern} recognition ``Kora'']. \textit{Algoritmy obucheniya raspoznavaniyu obrazov}
[Algorithms 
for learning pattern recognition]. Ed. V.\,N.~Vapnik. Moscow: Sovetskoe radio. 110--116.
  \bibitem{3-duk-1}
\Aue{Zhuravlev, Yu.\,I.} 1978. Ob algebraicheskom podkhode k~resheniyu zadach 
raspoznavaniya i klassifikatsii [On the algebraic approach to solving recognition and 
classification tasks]. \textit{Problemy kibernetiki} [Problems of Cybernetics]  
33:5--68.
  \bibitem{4-duk-1}
\Aue{Baskakova, L., and Yu.~Zhuravlev.} 1981. A~model of recognition algorithms 
with representative samples and systems of supporting sets. \textit{USSR Comp. Math. 
Math.} 21(5):189--199.
  \bibitem{5-duk-1}
\Aue{Djukova, E.\,V.} 1989. Algoritmy raspoznavaniya tipa ``Kora'': slozhnost' 
realizatsii i~metricheskie svoystva [Kora-type recognition algorithms: 
Implementation complexity and metric properties]. \textit{Raspoznavanie, 
klassifikatsiya, prognoz} [Recognition, 
classification, and prediction] . Moscow: Nauka. 
2:99--125.
  \bibitem{6-duk-1}
\Aue{Djukova, E., and Y.~Zhuravlev.} 2000. Discrete analysis of feature descriptions 
in recognition problems of high dimensionality. \textit{Comp. Math. Math. Phys.} 
40(8):1214--1227.
  \bibitem{7-duk-1}
\Aue{Djukova, E., and N.~Peskov.} 2002. Search for informative fragments of object 
descriptions in discrete recognition procedures. \textit{Comp. Math. Math. Phys.} 
42(5):711--723.
  \bibitem{8duk-1}
\Aue{Fredman, M., and L.~Khachiyan.} 1996. On the complexity of dualization of 
monotone disjunctive normal forms. \textit{J.~Algorithm.} 21(3):618--628.
  \bibitem{9-duk-1}
\Aue{Djukova, E., and P.~Prokofyev.} 2015. Asymptotically optimal dualization 
algorithms. \textit{Comp. Math. Math. Phys.} 55(5):891--905. 
  \bibitem{10-duk-1}
\Aue{Alekseev, V.\,B.} 1976. Deciphering of some classes of monotonic 
 many-valued functions. \textit{Comp. Math. Math. Phys.} 16(1):180--189.
  \bibitem{11-duk-1}
\Aue{Noskov, V.\,N., and V.\,A.~Slepyan.} 1972. Number of dead-end tests for a~certain class of tables. 
\textit{Cybern. Syst. Anal.} 8(1):64--71.
  \bibitem{12-duk-1}
\Aue{Dragunov, N., E.~Djukova, and A.~Djukova.} 2022. Supervised classification 
and finding frequent elements in data. \textit{8th Conference (International) on 
Information Technology and Nanotechnology Proceedings}. Piscataway, NJ: IEEE. 9848521. 5~p. doi: 10.1109/\mbox{ITNT55410}. 2022.9848521.9848521. 

\end{thebibliography}

 }
 }

\end{multicols}

\vspace*{-7pt}

\hfill{\small\textit{Received September 30, 2022}}

\vspace*{-16pt}

\Contr

\vspace*{-5pt}

\noindent
\textbf{Djukova Elena V.} (b.\ 1945)~--- Doctor of Science in physics and 
mathematics, principal scientist, Federal Research Center ``Computer Science and 
Control'' of the Russian Academy of Sciences, 44-2~Vavilov Str., Moscow 119333, 
Russian Federation; associate professor, Faculty of Computational Mathematics and 
Cybernetics, M.\,V.~Lomonosov Moscow State University, 1-52~Leninskie Gory, 
GSP-1, Moscow 119991, Russian Federation; \mbox{edjukova@mail.ru}

\vspace*{3pt}

\noindent
\textbf{Djukova Anastasia P.} (b.\ 1995)~--- PhD student, Federal Research Center 
``Computer Science and Control'' of the Russian Academy of Sciences, 44-2~Vavilov 
Str., Moscow 119333, Russian Federation; \mbox{anastasia.d.95@gmail.com}

\label{end\stat}

\renewcommand{\bibname}{\protect\rm Литература}    