\def\stat{peshkova}

\def\tit{ГРАНИЦЫ ЭКСТРЕМАЛЬНОГО ИНДЕКСА ВРЕМЕНИ
ОЖИДАНИЯ В~СИСТЕМЕ $M/G/1$ С~РАСПРЕДЕЛЕНИЕМ ВРЕМЕНИ ОБСЛУЖИВАНИЯ В~ВИДЕ  
КОНЕЧНОЙ СМЕСИ$^*$}

\def\titkol{Границы экстремального индекса времени
ожидания в~системе $M/G/1$ с~распределением времени обслуживания} % в~виде   конечной смеси}

\def\aut{И.\,В.~Пешкова$^1$}

\def\autkol{И.\,В.~Пешкова}

\titel{\tit}{\aut}{\autkol}{\titkol}

\index{Пешкова И.\,В.}
\index{Peshkova I.\,V.}


{\renewcommand{\thefootnote}{\fnsymbol{footnote}} \footnotetext[1]
{Работа выполнена при финансовой поддержке РНФ (проект 21-71-10135).}}


\renewcommand{\thefootnote}{\arabic{footnote}}
\footnotetext[1]{Петрозаводский государственный университет; 
Институт прикладных математических исследований Карельского 
научного центра РАН, \mbox{iaminova@petrsu.ru}}


%\vspace*{-12pt}





\Abst{Доказывается следующая теорема: если исходная  стационарная последовательность имеет распределение 
в~виде $m$-компонентной смеси, компоненты которой стохастически упорядочены, 
и~существуют предельные распределения для максимумов всех компонент, а~также 
упорядочены нормализующие последовательности, то экстремальный индекс исходной 
последовательности находится в~границах экстремальных индексов наименьшей 
и~наибольшей компонент. Этот результат применен для оценки экстремального 
индекса  стационарного  времени ожидания в~сис\-те\-ме обслуживания типа $M/G/1$, 
в~которой время обслуживания задано $m$-ком\-по\-нент\-ной смесью распределений. 
Рас\-смот\-рен  пример системы $M/H_m/1$ с~гиперэкспоненциальным временем 
обслуживания. Методом точного моделирования  получены результаты оценки 
экстремального индекса стационарного времени ожидания в~системе $M/H_2/1$.}


\KW{распределение экстремальных значений; 
экстремальный индекс; система обслуживания; стохастическая упорядоченность} 

 \DOI{10.14357/19922264220405} 
  
\vspace*{9pt}


\vskip 10pt plus 9pt minus 6pt

\thispagestyle{headings}

\begin{multicols}{2}

\label{st\stat}


\section{Введение}

\vspace*{-9pt}

Изучение экстремальных значений находит широкое применение в~различных 
прикладных областях.  Так, превышение высоких пороговых значений может означать 
отказы технических устройств из-за сбоев оборудования, природные катастрофы,\linebreak 
потери данных при передаче и~т.~д. При этом превышения  часто происходят не по 
одному, а~груп\-пируются во времени, образуя так называемые кластеры~\cite{Bertail}.  
Такая кластеризация в~тео\-рии экстремальных значений 
характеризуется с~по\-мощью \textit{экстремального индекса}~\cite{Resnick}, который 
определяет предельное распределение экстремальных значений стационарных 
случайных последовательностей~\cite{embrehts,haan,Leadbetter}.

Экстремальный индекс~$\theta$ отражает кластерную структуру базовой 
последовательности или ее локальную зависимость. Например, для 
последовательности независимых одинаково распределенных величин  $\theta\hm=1$. 
Значение экстремального индекса, равное нулю, $ \theta\hm = 0$, подразумевает общую 
зависимость, которая соответствует <<очень широким>> кластерам превышений 
пороговых значений. Одна из наиболее практически важных интерпретаций 
экстремального индекса состоит в~том, что  $1/\theta $ приблизительно 
соответствует среднему размеру кластера. Это позволяет оценить экстремальный 
индекс непараметрическими методами~\cite{Resnick,novak,weissman94}.

Экстремальные значения характеристик производительности систем массового 
обслуживания изучались в~работах~\cite{Asmus,asmus2,iglehart, Rootzen,Hooghiemstra}. В статье~\cite{razumchik} получены распределения 
кластерных и~межкластерных размеров стационарных времен ожидания в~системе 
$G/G/1$. Предельное распределение и~математическое ожидание времени первого 
превышения порогового значения выведены  в~\cite{markovich}. Выбор порогового 
значения обсуждается в~статье~\cite{rodionov}.


Главная идея данного  исследования состоит\linebreak
 в~по\-стро\-ении границ экстремального 
индекса стационарного времени ожидания в~сис\-те\-ме обслуживания типа $M/G/1$,  
в~которой время обслуживания  задано $m$-ком\-по\-нент\-ной смесью \mbox{распределений}  со  
\textit{стохастически упорядоченными} компонентами. В~качестве примера 
рассматривается  гиперэкспоненциальное распределение, которое пред\-став\-ля\-ет собой 
конечную смесь показательных распределений. Такое распределение играет важ\-ную 
роль в~исследовании современных сис\-тем, поскольку может быть использовано 
в~качестве распределения, ап\-прок\-си\-ми\-ру\-юще\-го распределения с~так на\-зы\-ва\-емы\-ми
<<длинными хвостами>>,  широко при\-ме\-ня\-емы\-ми в~коммуникационных сис\-те\-мах~\cite{feldman}.

Данное исследование опирается на результаты, полученные автором в~предшествующих 
работах~\cite{dccn2021, tomsk2021, peshkova2022, pesh-mor2022}. Например, в~статье~\cite{peshkova2022} 
доказана теорема  о~сравнении экстремальных индексов двух 
стационарных последовательностей, а~в~работе~\cite{pesh-mor2022}
получены результаты стохастической упорядоченности и~упорядоченности по 
интенсивности отказов для $m$-ком\-по\-нент\-ных смесей распределений.

В разд.~2 настоящей статьи
 доказана  тео\-ре\-ма~1, которая  утверж\-да\-ет, что если стационарная 
последовательность задана $m$-ком\-по\-нент\-ной смесью распределений и~ее компоненты 
стохастически упорядочены,  т.\,е.\  $\overline{F}_{Y^1}(x)\hm\le  \overline{F}_{Y^2}(x) 
\le\cdots\le  \overline{F}_{Y^m}(x)$ (где\linebreak
 $\overline F(x)$~--- хвост 
функции распределения), и~соответствующие \textit{нормализующие последовательности} 
также упорядочены,  то экстремальный индекс~$\theta_X$ исходной 
последовательности~$X_n$ удовлетворяет\linebreak неравенству
 $\theta_{Y^1}\hm\ge \theta_X\hm \ge \theta_{Y^m}$.
 В~разд.~3 сформулирована тео\-ре\-ма~2, которая  позволяет построить 
границы экстремального  индекса  стационарного времени ожидания 
в~односерверной сис\-те\-ме обслуживания с~$m$-ком\-по\-нент\-ным распределением времени 
об\-слу\-живания.
В~разд.~4 приведен   пример применения тео\-ре\-мы~2 для сис\-те\-мы 
$M/H_m/1$ с~гиперэкспоненциальным временем обслуживания.


\section{Границы экстремального индекса строго стационарной последовательности, 
заданной $m$-компонентной смесью распределений}
\label{sec2}

Рассмотрим  стационарную в~узком смыс\-ле по\-сле\-до\-ва\-тель\-ность   случайных величин 
(СВ)\  $\{ X_n , \ n\hm\ge 1\}$, удовлетворяющую тождеству
$$
(X_{k+1},\dots X_{k+n})\overset{d}{=}  (X_1,\dots,X_n) \ \mbox{для всех} \  
k,n\ge 1,
$$
где $\overset{d}{=}$ означает  равенство по распределению.
Обозначим~$M_n$ максимум  первых~$n$~членов последовательности, т.\,е.
\begin{equation*}
   M_n=\max \left(X_1, \dots , X_n\right).
\end{equation*}

Введем совместную функцию распределения (ф.~р.)
$$
F_{i_1,\dots i_k} \left(x_1,\dots , x_k\right) = \mathbb{P} \left(X_1 \le x_1, \dots , X_k\le x_k\right).
$$

Для заданной последовательности вещественных чисел $\{ u_n \hm= a_n x\hm+ b_n, \}$, 
$a_n \hm> 0$, будем говорить, что выполнено   условие перемешивания~$D(u_n)$, если 
для любых целых чисел
$$
1\le i_1 <\cdots <i_k<j_1<\cdots <j_k\le n\,,
$$
для которых $j_1-i_k\ge l_n,$
\begin{equation*}
    F_{i_1,\dots ,i_k,j_i,\dots ,j_k} (u_n) -  F_{i_1,\dots ,i_k} (u_n)  
F_{j_i,\dots ,j_k} (u_n) \le v_{n,l}.
\end{equation*}
Здесь $v_{n, l}\to 0$ при $n\hm\to \infty$, $l\hm\to \infty$; $F_{i_1,\dots ,i_k} 
(u_n){:=}F_{i_1,\dots ,i_k} (u_n,\dots ,u_n)$.


Предположим, что
для некоторого числа $\tau\hm>0$ определена такая последовательность $\{ u_n (\tau) \}$, 
что имеет место схо\-ди\-мость
\begin{equation}
\label{glava2-3-max43}
n\overline F(u_n(\tau))\to \tau \ \mbox{при} \ n\to \infty
\end{equation}
и условие $D(u_n(\tau))$ выполнено для каждого такого~$\tau$. Тогда~\cite{Leadbetter}
\begin{equation}
\label{glava2-4-max3}
\mathbb{P} \left(M_n \le u_n (\tau)\right) \to e^{-\theta \tau} \ \mbox{при} \ n\to \infty,
\end{equation}
где параметр  $\theta \hm\in [0, 1]$. Далее будем рассматривать стационарные 
последовательности, для которых выполнено условие перемешивания~$D(u_n(\tau))$.


Заметим, что если $\mathbb{P} (M_n\le u_n(\tau))$ сходится хотя бы для одного значения 
$\tau \hm> 0$, то~(\ref{glava2-4-max3}) выполнено при всех $\tau \hm> 0$ для 
некоторого фиксированного~$\theta$.


Будем говорить, что  стационарная последовательность $\{ X_n , \ n\hm\ge 1\}$  
с~общей ф.~р.~$F$ имеет \textit{экстремальный индекс} $\theta \hm\in
[0, 1]$, если при данном~$\theta$ для каждого $\tau \hm> 0$ существует такая 
последовательность вещественных чисел $\{ u_n\hm=u_n(\tau)\}$, что соотношения~\eqref{glava2-3-max43} 
и~\eqref{glava2-4-max3} выполнены (см.~\cite{embrehts}).


Из определения экстремального индекса следует, что для больших~$n$ можно 
использовать следующую аппроксимацию  распределения максимума~$M_n$ стационарной 
последовательности:
\begin{equation*}
    \mathbb{P}\left(M_n\le u_n\right) \sim F^{n \theta } (u_n) \ \mbox{для больших } \ n,
    \end{equation*}
где $a\sim b$ означает $\lim a/b \hm=1$ при $n\hm\to\infty$.  Следовательно, 
экстремальный индекс $\theta$ служит ключевым параметром для изучения 
асимптотических  распределений максимумов стационарных последовательностей.



Следующая теорема позволяет найти границы  экстремального индекса  строго 
стационарной последовательности $\{ X_n \}$, заданной $m$-ком\-по\-нент\-ной смесью 
распределений.

\smallskip

\noindent
\textbf{Теорема~1.}
\textit{Пусть строго стационарная последовательность~$\{ X_n \}$ задана общей  ф.~р.~$F_X$ вида}
\begin{multline*}
    F_X(x)=p_1 F_{Y^1}(x)+ \dots + p_m F_{Y^m}(x),\\
      \sum\limits_{i=1}^m p_i= 1\,, \enskip p_i\ge 0\,,
\end{multline*}
\textit{т.\,е.\ каждый член последовательности~$X_i$  задан \mbox{$m$-ком}\-по\-нент\-ной смесью 
распределений СВ}\ $Y^1,\dots, Y^m$.

\textit{Пусть существуют такие нор\-ма\-ли\-зу\-ющие последовательности}
$\left\{u_n(x)\hm=a_n x\hm+ b_n\right\}$  \textit{и}~$\left\{u_n^i(x)\hm=a_n^ix\hm+b_n^i\right\}$,
\textit{что $a_n, a_n^i \hm>0$, $n\hm\ge 1$, $\ u_n(x), u_n^i(x) \hm\to \infty$  для каж\-до\-го~$x$ 
при $n\to\infty, i=1,\dots,m$, и~пусть выполнены соотношения}
\begin{align}
       \mathbb{P}\left(M_n^X \le u_n(x)\right) & \to H_X(x), \notag\\  
       &\hspace*{-15mm}\mathbb{P}(M_n^{Y^i} \le u_n^i(x)) \to  H_{Y^i}(x), \enskip  n\to\infty; 
       \label{glava2-5-teor14-1}\\
n\overline F_X(u_n(x)) &\to \tau(x),\  n\overline F_{Y_i} (u_n^i(x)) \to  \tau^i(x),\notag \\
&\hspace*{35mm} n\to \infty;
\label{glava2-5-teor14-2}\\
&\hspace*{-20mm}u_n^1(x) \ge \ u_n(x)\ge u_n^m(x) \ \mbox{\textit{для всех}} \  x, \ n\ge 1. 
\label{glava2-5-teor14-3}
\end{align}
\textit{Пусть также стохастически упорядочены компоненты   смеси,
т.\,е.}
\begin{equation*}
\overline F_{Y^1}(x)\le  \overline F_{Y^2}(x) \le\cdots\le  \overline 
F_{Y^m}(x)\ \mbox{\textit{для всех}} \  x\ge 0 . \, 
%\label{glava2-5-teor3}
\end{equation*}
\textit{Тогда   экстремальные индексы $\theta_X$, $\theta_{Y^1}$ и~$\theta_{Y^m}$  
последовательностей  $\{ X_n \} $, $\{ Y_n^1 \}$  и~$\{ Y_n^m \}$  упорядочены 
сле\-ду\-ющим образом}:
\begin{equation}
    \label{glava2-5-theta_comp}
    \theta_{Y^1}\ge \theta_X \ge \theta_{Y^m}.
\end{equation}


\noindent
Д\,о\,к\,а\,з\,а\,т\,е\,л\,ь\,с\,т\,в\,о\,.\ \ Из  стохастической упорядоченности компонент смеси
\begin{equation*}
%\label{eq1}
Y^1 \underset{\mathrm{st}}{\le} \cdots \underset{\mathrm{st}}{\le} Y^m
\end{equation*}
вытекает соотношение~\cite[теорема~1]{pesh-mor2022}: 
\begin{equation*}
 Y^1  \underset{\mathrm{st}}{\le} X \underset{\mathrm{st}}{\le} Y^m.
\end{equation*}
%
Поэтому, если выполнены условия~(\ref{glava2-5-teor14-1})--(\ref{glava2-5-teor14-3}), 
то  экстремальные индексы, соответствующие этим последовательностям, 
связаны неравенством~(\ref{glava2-5-theta_comp})~\cite[теорема 1]{peshkova2022}. \hfill $\square$


\section{Система $M/G/1$ с~временем обслуживания, заданным конечной смесью 
распределений} %\label{3}

Применим теорему~1 для оценки экстремального индекса стационарного времени 
ожидания\linebreak в~односерверной сис\-те\-ме обслуживания с~временем обслуживания, заданным 
конечной смесью распределений.

Рассмотрим односерверную сис\-те\-му~$\Sigma$ с~пуассоновским входным потоком. 
Дисциплина обслуживания~--- первым пришел, первым обслужен.
 Обозначим
  $S$~--- типичное время обслуживания заявки и~$T$~--- типичный интервал между 
приходами заявок, $\mathbb{E} T\hm=1/\lambda$, $\mathbb{E} S\hm=1/\mu$.  Пусть время обслуживания~$S$
  задано $m$-ком\-по\-нент\-ной смесью распределений  c~ф.~р.\ вида
  \begin{multline}
  \label{mixt1}
      B (x)= \sum\limits_{i=1}^m p_i B_i(x), \\
       \sum\limits_{i=1}^m  p_i =1\,, \enskip p_i\ge 0\,,\enskip i=1,\ldots, m\,.
  \end{multline}
  Предположим, что СВ   $S^{(1)},\dots, S^{(m)}$  независимы 
  и~имеют ф.~р.~$B_i(x)$, $i\hm=1,\dots , m$. Пусть индикаторы~$I_i$
принимают два значения: $I_i\hm= 0$ с~ве\-ро\-ят\-ностью~$p_i$  и~$I_i\hm=1$ с~ве\-ро\-ят\-ностью 
$1\hm-p_i$~--- и~не зависят от $\{S^{(i)}\},$ $i\hm=1,\dots , m$. Тогда СВ
\begin{equation*}
S=I_1\, S^{(1)} + \cdots + I_m S^{(m)}
%\label{2_2}
\end{equation*}
является \textit{$m$-ком\-по\-нент\-ной смесью}  с~ф.~р.~(\ref{mixt1}).
   Математическое ожидание времени обслуживания вычисляется по формуле:
 $$
 \mathbb{E} S=\fr{1}{\mu} = \sum\limits_{i=1}^m \fr{p_i}{\mu_i}\,,
 $$
  где $\mathbb{E} S^{(i)}=1/\mu_i$.
Обозначим
$$
\rho=\lambda \mathbb{E} S =\sum\limits_{i=1}^m \fr{\lambda p_i}{\mu_i} =  
\sum\limits_{i=1}^m p_i \rho_i
$$
коэффициент загрузки  сис\-те\-мы~$\Sigma$, где $\rho_i\hm=\lambda/\mu_i, \ 
i=1,\dots,m$.

Предположим, что компоненты времени обслуживания~$S$ стохастически упорядочены, 
т.\,е.
$$
S^{(1)}\underset{\mathrm{st}}{\le}S^{(2)}\underset{\mathrm{st}}{\le} \cdots \underset{\mathrm{st}}{\le}  
S^{(m)}.
$$

Рассмотрим две  системы обслуживания $M/G/1$:   $\Sigma^{(1)}$ и~$\Sigma^{(m)}$ 
с~таким же входным потоком, как в~исходной системе~$\Sigma$, $\mathbb{E} 
T^{(i)}\hm=1/\lambda_i\hm=1/\lambda,$ $i\hm=1, m$. (Будем снабжать индексом~$i$ 
величины, относящиеся к~$i$-й сис\-те\-ме.) Время обслуживания~$S^{(1)}$ в~сис\-те\-ме~$\Sigma^{(1)}$ 
задано ф.~р.~$B_1$. Время обслуживания~$S^{(m)}$ в~сис\-те\-ме~$\Sigma^{(m)}$ задано ф.~р.~$B_m$.
Будем говорить, что сис\-те\-ма~$\Sigma^{(1)}$ является минорантной, а~сис\-те\-ма~$\Sigma^{(m)}$~--- 
мажорантной для исходной сис\-те\-мы~$\Sigma$.


  Теперь  сравним  в~системах~$\Sigma$,  $\Sigma^{(1)}$ и~$\Sigma^{(m)}$ 
экстремальные индексы стационарных времен ожидания.

Пусть   $\nu_n  (\nu_n^{(i)})$~--- число заявок в~сис\-те\-ме,   $Q_n ( Q_n^{(i)})$~---   
размер очереди и~$W_n (W_n^{(i)})$~---    время ожидания в~очереди в~момент прихода заявки с~номером~$n$ 
в~сис\-те\-му~$\Sigma$ $(\Sigma^{(i)})$ соответственно, $i\hm=\overline{1,m}$.
Обозначим (в~случае их существования) пределы по распределению
\begin{multline}
\label{limits}
Q_n^{(i)} \Rightarrow Q^{(i)};\enskip \nu_n^{(i)} \Rightarrow \nu^{(i)};\enskip
W_n^{(i)} \Rightarrow W^{(i)};\\
 n\to \infty\,.
\end{multline}
Эти  пределы существуют, в~частности,  если времена между приходами заявок~$T$  
являются \textit{нерешетчатыми}  и~$\rho_i\hm<1$,  $i\hm=\overline{1,m}$~\cite{Asmus}.  Заметим, что если соотношения~(\ref{limits}) 
выполнены и~$\rho_i\hm<1$, $i\hm=\overline{1,m}$, то можно показать, что и~для исходной сис\-те\-мы~$\Sigma$ 
слабые пределы~$\nu, Q$ и~$W$  также существуют.


Рассмотрим стационарные последовательности времен ожидания  $\{W_n, n\hm\ge 1 \}$, 
$\{W_n^{(1)}, n\hm\ge 1 \}$ и~$\{W_n^{(m)}, n\hm\ge 1 \}$
в~сис\-те\-мах~$\Sigma$, $\Sigma^{(1)}$ и~$\Sigma^{(m)}$ соответственно.
Обозначим максимумы из~$n$~членов этих последовательностей ($n\hm\ge 1$):
\begin{align*}
W_n^*&=\max\left(W_1,\dots, W_n\right);
\\
W_n^{(1)*}&=\max\left(W_1^{(1)},\dots, W_n^{(1)}\right);\\
W_n^{(m)*}&=\max\left(W_1^{(m)},\dots, W_n^{(m)}\right).
\end{align*}

\noindent
\textbf{Теорема~2.}
\textit{Предположим, что для систем $\Sigma^{(1)}$ и~$\Sigma^{(m)}$  коэффициенты загрузки  $\rho_i\hm<1$, 
$i\hm=\overline{1,m}$, и~выполнены  следующие стохастические соотношения}:
\begin{equation}
\label{ar0:3}
   \nu_1^{(1)}=\nu_1^{(m)}=\nu_1=0\,;\enskip  T\underset{\mathrm{st}}{=}T^{(1)}\underset{\mathrm{st}}{=}T^{(m)}.
\end{equation}
\textit{Пусть компоненты смеси времени обслуживания стохастически упорядочены, т.\,е.}
\begin{equation}
\label{ar0:4}
  \overline B_1(x) \le \dots\le \overline B_m(x).
\end{equation}
\textit{Предположим, что существуют такие нормализующие последовательности 
$\left\{u_n(x)\hm=a_nx\hm+b_n\right\}$ и~$\left\{u_n^{(i)}(x)\hm=a_n^{(i)}x\hm+b_n^{(i)}\right\}$, что $a_n, 
a_n^{(i)} \hm>0$, $n\hm\ge 1$, для каждого~$x$ $u_n(x), u_n^{(i)}(x) \hm\to \infty$.
Пусть}
$$
u_n^{(1)}(x)\ge u_n(x)\ge u_n^{(m)}(x)
$$
\textit{и выполнены соотношения}~(\ref{glava2-5-teor14-1})--(\ref{glava2-5-teor14-2}) \textit{для  
максимумов стационарных времен ожидания~$ W_n^{*}$,  $W_n^{(1)*}$ и~$W_n^{(m)*}$. 
Тогда}
\begin{equation}
\label{extr-wait}
  \theta_{W^{(1)}}\ge  \theta_{W}\ge  \theta_{W^{(m)}},
\end{equation}
\textit{где $\theta_{W}$~--- экстремальный индекс~$W_n^*;$ $\theta_{W^{(i)}}$~--- 
экст\-ре\-маль\-ный индекс}~$W_n^{(i)*}$, $i\hm=1,m.$


\noindent
Д\,о\,к\,а\,з\,а\,т\,е\,л\,ь\,с\,т\,в\,о\,.\ \
 Из неравенств~(\ref{ar0:4}) следует,~что
 $$
  S^{(1)} \underset{\mathrm{st}}{\le} S \underset{\mathrm{st}}{\le} S^{(m)},
  $$
что в~совокупности  с~соотношениями~(\ref{ar0:3}) гарантирует (см.\ \cite[теорема~4]{Whitt}),  что
\begin{equation*}
%\label{ar0:6}
 Q_n^{(1)} \underset{\mathrm{st}}{\le} Q_n \underset{\mathrm{st}}{\le} Q_n^{(m)};\,\,  W_n^{(1)} 
\underset{\mathrm{st}}{\le} W_n \underset{\mathrm{st}}{\le} W_n^{(m)},\,\, n\ge 1.
\end{equation*}
Таким образом, все условия теоремы~1  выполнены, что влечет справедливость  
неравенств~(\ref{extr-wait}) для экстремальных индексов стационарных времен 
ожидания рассматриваемых трех систем. \hfill $\square$

\vspace*{-6pt}

\section{Экстремальный индекс времени ожидания в~системе $M/H_m/1$}
\label{sec4}

\vspace*{-2pt}

В качестве исходной системы~$\Sigma$ рассмотрим сис\-те\-му $M/H_m/1$  с~гиперэкспоненциальным (\mbox{$m$-ком}\-по\-нент\-ным) распределением времени обслуживания 
с~ф.~р.

\vspace*{-3pt}

\noindent
\begin{multline*}
\hspace*{-5.88843pt}B(x)= 1 -  \sum\limits_{i=1}^m p_i e^{-\mu_i x}, \\
\mu_i > 0,\enskip p_i\ge  0,\enskip i=1,\dots,m, \enskip 
\sum\limits_{i=1}^m p_i=1\,.
\end{multline*}
В качестве минорантной и~мажорантной сис\-тем рассмотрим две $M/M/1$ сис\-те\-мы~$\Sigma^{(1)}$ 
и~$\Sigma^{(m)}$, в~которых  времена обслуживания~$S^{(i)}$  
имеют показательный закон распределения с~ф.~р.\ $\overline B_i(x) \hm= e^{-\mu_i  x}$, $i\hm=\overline{1,m}$.
Пусть условия стационарности во всех системах выполнены:
$$
\rho_i=\fr{\lambda}{\mu_i} <1, \enskip i=1,\dots,m\,.
$$
Функция интенсивности отказов для времени обслуживания~$S$ в~исходной системе~$\Sigma$ равна
$$
r(x):= \fr{f_B(x)}{\overline B(x)}= \fr{\sum\nolimits_{i=1}^m p_i \mu_i e^{-\mu_i x} }{\sum\nolimits_{i=1}^m p_ie^{-\mu_i x}}\,,
$$
где $f_B(x)$~--- плотность распределения СВ~$S$.
Функции интенсивности отказов времен обслуживания~$S^{(1)}$ и~$S^{(m)}$ в~сис\-те\-мах~$\Sigma^{(1)}$ 
и~$\Sigma^{(m)}$ соответственно равны $r_1(x)\hm=\mu_1$ и~$\ r_m(x)\hm=\mu_m$.
Нетрудно проверить, что при условии выполнения неравенства
\begin{equation}
    \label{failure}
    \mu_1 \ge \mu_2\ge\cdots \ge \mu_m
\end{equation}
функции интенсивности отказов упорядочены сле\-ду\-ющим образом:
$$
r_1 (x)\ge r(x) \ge r_m(x), \enskip x\ge 0\,,
$$
а следовательно, времена обслуживания в~этих сис\-те\-мах упорядочены по 
интенсивности отказов
$$
S^{(1)} \underset{r}{\le}  S \underset{r}{\le} S^{(m)},
 $$
 
 

\noindent
откуда, в~свою очередь, вытекает их стохастическая упо\-ря\-до\-чен\-ность~\cite{Ross}
$$
S^{(1)} \underset{\mathrm{st}}{\le}  S \underset{\mathrm{st}}{\le} S^{(m)}. $$

{ \begin{center}  %fig1
 \vspace*{-2pt}
     \mbox{%
\epsfxsize=79.029mm
\epsfbox{pes-1.eps}
}

\end{center}



\noindent
{\small Экстремальные индексы в~системах $M/M(\mu_1\hm=4)/1$, 
$M/H_2(\mu_1\hm=4$, $\mu_2\hm=2)/1$ и~$M/M(\mu_2=2)/1$,  $\lambda\hm=\lambda_1\hm=\lambda_2\hm=1$
}}


\vspace*{9pt}

Поскольку $\Sigma^{(1)}$ и~$\Sigma^{(m)}$  относятся к~системам  типа $M/M/1$, 
то для них  известна ф.~р.\ стационарного времени ожидания~$F_{W^{(i)}}$ 
в~явном виде:
\begin{equation*}
F_{W^{(i)}}(x) =1-\rho_i e^{-(\mu_i-\lambda)x}.
\end{equation*}
Выбирая нормализующие последовательности для максимума стационарного времени 
ожидания вида 
\begin{equation*}
 \label{expun}
 u_n^{i}(x) = \fr{x + \log (\rho_i n)}{\mu_i - \lambda}\,, \enskip i=\overline{1,m}\,,
  \end{equation*}
получим  его предельное распределение~\cite{iglehart}
\begin{equation*}
   \lim\limits_{n\to\infty} \mathbb{P} ( W_n^{(i)*}  \le u_n^{i}(x)) = \Lambda(x),  \enskip i=\overline{1,m}\,,
\end{equation*}
где $\Lambda(x) \hm= \exp (- e^{-x})$~--- распределение Гумбеля.
Кроме того, для систем $M/M/1$ известны  выражения для экстремальных индексов  
стационарных времен ожидания в~явном виде~\cite{Hooghiemstra}:
$$
\theta_{W^{(1)}}=(1-\rho_1)^2; \quad  \theta_{W^{(m)}}=(1-\rho_m)^2.
$$
Вернемся к~исходной системе $M/H_m/1$ . Для  сис\-тем $M/G/1$  экстремальный 
индекс  можно вычислить по формуле~\cite{Hooghiemstra}:
\begin{equation*}
    \theta_W^*= \fr{\gamma (1-\rho)}{\gamma + \lambda}\,,
\end{equation*}
где $\gamma$~--- единственное положительное решение уравнения
\begin{equation}
\label{gamma}
\mathbb{E} e^{\gamma(S-T)}=1
\end{equation}
при условии, что $\mathbb{E} [ |S-T|e^{\gamma(S-T)}] \hm< \infty$. (В~случае $m$-ком\-по\-нент\-но\-го 
гиперэкспоненциального распределения
уравнение~(\ref{gamma}) имеет несколько корней). Согласно тео\-ре\-ме~2,
при условии выполнения соотношений~(\ref{failure}) экстремальные индексы в~этих 
сис\-те\-мах связаны неравенством
\begin{equation*}
  \theta_{W^{(1)}}\ge  \theta_{W}\ge  \theta_{W^{(2)}}.
\end{equation*}
На рисунке приведены результаты эксперимента  по  моделированию экстремального 
индекса стационарного времени ожидания системы $M/H_2/1$ c~параметрами 
$\lambda\hm=1$, $\mu_1\hm=4$ и~$\mu_2\hm=2$. Для получения стационарного времени ожидания 
использовался метод точного моделирования~\cite{sigman}. Экстремальный индекс 
в~исходной системе сравнивался с~экстремальными индексами двух систем $M/M/1$ 
с~таким же входным потоком и~временем обслуживания, распределенным по 
показательному закону  с~па\-ра\-мет\-ром $\mu_1\hm=4$ (в~минорантной сис\-те\-ме) 
и~$\mu_2\hm=2$ (в~мажорантной сис\-те\-ме) соответственно. Результаты моделирования 
подтверждают спра\-вед\-ли\-вость формулы~(\ref{extr-wait}).

\vspace*{-9pt}


\section{Заключение}

\vspace*{-4pt}

В работе показано, что если в~исходной  $M/G/1$  системе
время обслуживания задано \mbox{$m$-ком}\-по\-нент\-ной смесью распределений, компоненты 
которой\linebreak стохастически упорядочены, то для нее можно построить две новых сис\-те\-мы: 
минорантную и~мажорантную сис\-те\-мы, экстремальные индексы стационарных времен 
ожидания  которых станут \mbox{границами} для экстремального индекса стационарного 
времени ожидания исходной сис\-те\-мы.
 В~качестве примера рассмотрена сис\-те\-ма $M/H_m/1$ с~гиперэкспоненциальным 
временем обслуживания. Минорантной (мажорантной) сис\-те\-мой послужила сис\-те\-ма 
$M/M/1$ с~таким же входным потоком, время обслуживания которой имеет 
показательное распределение с~наибольшим (наименьшим) па\-ра\-мет\-ром среди всех 
параметров гиперэкспоненциального распределения $\mu_1\hm=\max(\mu_1,\dots ,\mu_m)$ 
($\mu_m\hm=\min(\mu_1,\dots ,\mu_m)$ для мажорантной сис\-те\-мы).  Результаты 
моделирования $M/H_2/1$ показывают устойчивость оценки экстремального индекса 
в~исходной сис\-те\-ме  при разных перцентилях. При этом оценка экстремального индекса 
ограничена значениями экстремальных индексов мажорантной и~минорантной сис\-тем.
 Метод точного моделирования  для оценки экстремального индекса стационарного 
времени ожидания может быть применен  для других распределений времени 
обслуживания, для которых не существует решения уравнения~(\ref{gamma}) в~явном 
виде,  что станет темой будущих исследований.

\vspace*{-9pt}

{\small\frenchspacing
 {%\baselineskip=10.8pt
 %\addcontentsline{toc}{section}{References}
 \begin{thebibliography}{99}
 
 \vspace*{-4pt}

\bibitem{Bertail} %1
\Au{Bertail P., Clemencon~S., Tressou~J.}
Extreme values statistics for Markov chains via the
(pseudo-) regenerative method~// Extremes, 2009.  Vol.~12. Iss.~4. P.~327--360.   
doi: 10.1007/s10687-009-0081-y.

\bibitem{Resnick} %2
\Au{Resnick S.}  Extreme values, regular variation and point 
processes.~--- New York, NY, USA: Springer,  1987. 320~p.

\bibitem{Leadbetter}  %3
\Au{Leadbetter M.\,R., Lindgren~G., Rootzin~H.}  Extremes 
and related properties of random sequences
and processes.~--- New York, NY, USA: Springer,  1983. 336~p.

\bibitem{embrehts} %4
\Au{Embrechts~P., Kluppelberg~C., Mikosch~T.} Modelling extremal events for 
insurance and finance. Applications of mathematics.~--- Berlin, Heidelberg:   
Springer, 1997. 660~p.

\bibitem{haan} %5
\Au{De Haan L., Ferreira~A.}  Extreme value theory:  An introduction.~---   New 
York, NY, USA:  Springer Science\;+\;Business Media LLC, 2006. 491~p.




\bibitem{weissman94} %6
\Au{Smith L., Weissman~I.}
Estimating the extremal index~//
J.~Roy. Stat. Soc.~B Met., 1994.  Vol.~56. Iss.~3. P.~515--528.
doi: 10.1111/J.2517-6161.1994.TB01997.X.

\bibitem{novak} %7
\Au{Weissman I., Novak~S.}
On blocks and runs estimators of extremal index~// J.~Stat. Plan. Infer.,
1998. Vol.~66. Iss.~2. P.~281--288.
doi: 10.1016/S0378-3758(97)00095-5.




\bibitem{iglehart}  %8
\Au{Iglehart D.\,L.}  Extreme values in $\mathrm{GI}/G/1$ queue~// Ann. 
Math. Stat., 1972. Vol.~3. Iss.~2. P.~627--635. doi: 10.1214/ aoms/1177692642.


\bibitem{Rootzen}  %9
\Au{Rootzen H.} Maxima and exceedances of stationary Markov 
chains~// Adv. Appl. Probab., 1988. Vol.~20. Iss.~2. P.~371--390. 
doi: 10.2307/1427395.


\bibitem{Hooghiemstra} %10
\Au{Hooghiemstra G.,  Meester~L.\,E.} Computing the extremal index of special
Markov chains and queues~// Stoch. Proc. Appl., 1996. 
Vol.~65. Iss.~2. P.~171--185. doi: 10.1016/ S0304-4149(96)00111-1.

\bibitem{asmus2} %11
\Au{Asmussen S.} Extreme value theory for queues via cycle maxima~// Extremes, 
1998. Vol.~1. Iss.~2. P.~137--168. doi: 10.1023/A:1009970005784.

\bibitem{Asmus} %12
\Au{Asmussen S.} Applied probability and queues. Stochastic modelling and  applied
probability.~--- New York, NY, USA: Springer-Verlag, 2003. 438~p.


\bibitem{razumchik} %13
\Au{Markovich N., Razumchik~R.} Cluster modeling of Lindley process with 
application to queuing~// Distributed computer and communication networks.~--- Lecture notes in computer 
science ser.~--- Cham, Switzerland: Springer, 2019. Vol.~11965.
P.~330--341. doi:  10.1007/978-3-030-36614-8\_25.

\bibitem{markovich} %14
\Au{Markovich N.} Clustering and hitting times of threshold exceedances and 
applications~// Int. J. Data Analysis Techniques Strategies, 2017.  Vol.~9. Iss.~4. P.~331--347.
doi: 10.1504/IJDATS.2017.10009424.

\bibitem{rodionov} %15
\Au{Rodionov I.} On threshold selection problem for extremal index estimation~// 
Recent developments in stochastic methods and applications~/ Eds. A.\,N. Shiryaev, 
K.\,E. Samouylov, D.\,V. Kozyrev.~--- Springer proceedings in mathematics \& 
statistics ser.~--- Cham, Switzerland: Springer, 2021. Vol.~371. P.~3--16. doi: 
10.1007/978-3-030-83266-7\_1.

\bibitem{feldman}  %16
\Au{Feldman A., Whitt~W.} Fitting mixtures of exponentials to 
long-tail distributions to analyze network performance models~//
Perform. Evaluation, 1998. Vol.~31. Iss.~3-4.  P.~245--279. doi: 
10.1016/S0166-5316(97)00003-5.

\bibitem{dccn2021}  %17
\Au{Peshkova I., Morozov~E., Maltseva~M.} On regenerative 
estimation of extremal index in queueing systems~// Distributed computer and 
communication networks.~--- Lecture notes in computer science ser.~--- Cham, Switzerland: 
Springer, 2021. Vol.~13144. P.~251--264. doi: 10.1007/978-3-030-92507-$9\_21$.

\bibitem{peshkova2022}  %18
\Au{Peshkova I.}
Сравнение экстремальных индексов времен ожидания в~системах обслуживания $M/G/1$~//
Информатика и~её применения, 2022.
Том.~16. Вып.~1. С.~61--67. doi: 10.14357/19922264220109.


\bibitem{pesh-mor2022}  %19
\Au{Peshkova I., Morozov~E.} On comparison of 
multiserver systems with multicomponent mixture distributions~// J.~Math. Sci., 2022. Vol.~267. No.~2. P.~260--272. doi:  
10.1007/s10958-022-06132-z.



\bibitem{tomsk2021} %20
\Au{Peshkova I., Morozov E., Maltseva M.}   On comparison of waiting time 
extremal indexes  in queueing systems with Weibull service times //
{Information technologies and mathematical modelling. Queueing theory and 
applications.}~--- Communications in computer and information science 
ser.~--- Cham, Switzerland: Springer, 2022. Vol.~1605. P.~80--92. doi: 
10.1007/978-3-031-09331-9\_7.

\bibitem{Whitt} %21
\Au{Whitt W.} Comparing counting processes and queues~// Adv. Appl. Probab., 
1981. Vol.~13. P.~207--220.  doi: 10.2307/1426475.


\bibitem{Ross} %22
\Au{Ross S., Shanthikumar~J., Zhu~Z.}  On increasing-failure-rate random 
variables~// J.~Appl. Probab., 2005. Vol.~42. P.~797--809. doi: 
10.1239/jap/1127322028.



\bibitem{sigman} %23
\Au{Sigman K.} Exact simulation of the stationary distribution of the FIFO 
$M/G/c$ queue: The general case for $\rho < c$~// Queueing Syst., 2012. Vol.~70. 
P.~37--43.
doi:  10.1007/ s11134-011-9266-6.
\end{thebibliography}

 }
 }

\end{multicols}

\vspace*{-6pt}

\hfill{\small\textit{Поступила в~редакцию 15.10.22}}

%\vspace*{8pt}

%\pagebreak

\newpage

\vspace*{-28pt}

%\hrule

%\vspace*{2pt}

%\hrule

%\vspace*{-2pt}

\def\tit{ON BOUNDS OF THE  STATIONARY WAITING TIME EXTREMAL INDEX IN~$M/G/1$ SYSTEM WITH~MIXTURE SERVICE TIMES}


\def\titkol{On bounds of the  stationary waiting time extremal index in~$M/G/1$ system with~mixture service times}


\def\aut{I.\,V.~Peshkova$^{1,2}$}

\def\autkol{I.\,V.~Peshkova}

\titel{\tit}{\aut}{\autkol}{\titkol}

\vspace*{-8pt}


\noindent
$^1$Petrozavodsk State University, 33~Lenina Prosp., Petrozavodsk 185910, Russian Federation

\noindent
$^2$Karelian Research Centre of
the Russian Academy of Sciences, 11~Pushkinskaya Str., Petrozavodsk 185910,\linebreak
$\hphantom{^1}$Russian Federation


\def\leftfootline{\small{\textbf{\thepage}
\hfill INFORMATIKA I EE PRIMENENIYA~--- INFORMATICS AND
APPLICATIONS\ \ \ 2022\ \ \ volume~16\ \ \ issue\ 4}
}%
 \def\rightfootline{\small{INFORMATIKA I EE PRIMENENIYA~---
INFORMATICS AND APPLICATIONS\ \ \ 2022\ \ \ volume~16\ \ \ issue\ 4
\hfill \textbf{\thepage}}}

\vspace*{3pt} 


 


\Abste{It is proved that if the original stationary sequence has $m$-component mixture distribution with 
stochastically ordered components,  there are limit distributions for the maxima of all components, and the normalizing 
sequences are ordered, then the extremal index of the original sequence is within the boundaries of the extremal indexes 
of the smallest and largest components. This result is used to estimate the extremal index of the stationary waiting time 
in a~queuing system of type $M/G/1$ in which the queuing time is given by an $m$-component distribution mixture. 
An example of a~system $M/H_m/1$ with hyperexponential service time is considered. Using the exact simulation approach,
 the results of estimating the extremal index of stationary waiting time in the system $M/H_2/1$ are obtained.}


\KWE{extreme value distributions; extremal index; queueing system; stochastic ordering} 




 \DOI{10.14357/19922264220405} 

\vspace*{-12pt}

\Ack

\vspace*{-3pt}

\noindent
The research has been prepared  with the support of the Russian Science Foundation (project No.\,21-71-10135). 


\vspace*{3pt}

  \begin{multicols}{2}

\renewcommand{\bibname}{\protect\rmfamily References}
%\renewcommand{\bibname}{\large\protect\rm References}

{\small\frenchspacing
 {%\baselineskip=10.8pt
 \addcontentsline{toc}{section}{References}
 \begin{thebibliography}{99}

\bibitem{Bertail-1} %1
\Aue{Bertail, P., S.~Clemencon, and J.~Tressou.} 2009.
Extreme values statistics for Markov chains via the \mbox{(pseudo-)} 
regenerative method. \textit{Extremes} 12(4):327--360. doi: 10.1007/s10687-009-0081-y.

\bibitem{Resnick-1}  %2
\Aue{Resnick, S.} 1987.
\textit{Extreme values, regular variation and point processes}. 
New York, NY: Springer. 320~p. 

\bibitem{Leadbetter-1}  %3
\Aue{Leadbetter, M.\,R., G.~Lindgren, and H.~Rootzen.} 1983. 
\Au{Extremes and related properties of random sequences and processes}. New York, NY: Springer. 336~p.

\bibitem{embrehts-1} %4
\Aue{Embrechts, P., C.~Kluppelberg, and T.~Mikosch.}
 1997. \textit{Modelling extremal events for insurance and finance}. Berlin, Heidelberg:   Springer. 660~p.

\bibitem{haan-1} %5
\Aue{De Haan, L., and A.~Ferreira.} 2006.  \textit{Extreme value theory:  An introduction}.
New York, NY:  Springer Science\;+\;Business Media LLC. 491~p.



\bibitem{weissman94-1} %6
\Aue{Smith, L., and I.~Weissman.} 1994.
Estimating the extremal index. \textit{J.~Roy. Stat. Soc.~B Met.} 56(3):515--528.
doi: 10.1111/J.2517-6161.1994.TB01997.X.

\bibitem{novak-1} %7
\Aue{Weissman, I., and S.~Novak.} 1998.
On blocks and runs estimators of extremal index.
\textit{J.~Stat. Plan. Infer.}  66(2):281--288. doi: 10.1016/S0378-3758(97)00095-5.


\bibitem{iglehart-1}  %8
\Aue{Iglehart, D.\,L.} 1972. 
Extreme values in $\mathrm{GI}/G/1$ queue. \textit{Ann. Math. Stat.} 3(2):627--635. doi: 10.1214/aoms/ 1177692642.


\bibitem{Rootzen-1}  %9
\Aue{Rootzen, H.} 1988.
Maxima and exceedances of stationary Markov chains.
\textit{Adv. Appl. Probab.} 20(2):371--390. doi: 10.2307/1427395.


\bibitem{Hooghiemstra-1} %10
\Aue{Hooghiemstra, G., and  L.\,E.~Meester.} 1996. Computing the extremal index of special
Markov chains and queues. \textit{Stoch. Proc. Appl.} 65(2):171--185. doi: 10.1016/S0304-4149(96)00111-1.



\bibitem{asmus2-1} %11
\Aue{Asmussen, S.} 1998. Extreme value theory for queues via cycle maxima. \textit{Extremes} 1(2):137--168. doi: 10.1023/A:1009970005784.

\bibitem{Asmus-1} %12
\Aue{Asmussen, S.} 2003. \textit{Applied probability and queues}. New York, NY: Springer-Verlag. 438~p. 

\bibitem{razumchik-1} %13
\Aue{Markovich, N., and R.~Razumchik.} 2019.  
Cluster modeling of Lindley process with application to queuing. 
\textit{Distributed computer and communication networks.}  Lecture notes in computer science ser. 
Cham, Switzerland: Springer. 11965:330--341. doi:  10.1007/978-3-030-36614-8\_25.

\bibitem{markovich-1} %14
\Aue{Markovich, N.} 2017.  Clustering and hitting times of threshold exceedances and applications.
\textit{Int. J. Data Analysis Techniques Strategies} 9(4):331--347.
doi: 10.1504/\linebreak IJDATS.2017.10009424.


\bibitem{rodionov-1} %15
\Aue{Rodionov, I.} 2021. On threshold selection problem for extremal index estimation. 
\textit{Recent developments in stochastic methods and applications.} Eds. A.\,N.~Shiryaev, K.\,E.~Samouylov, and D.\,V.~Kozyrev. 
Springer proceedings in mathematics \& statistics ser.  Cham, Switzerland: Springer. 371:3--16. doi: 10.1007/978-3-030-83266-7\_1. 

\bibitem{feldman-1}  %16
\Aue{Feldman, A., and   W. Whitt.} 1998. Fitting mixtures of 
exponentials to long-tail distributions
to analyze network performance models. 
\textit{Perform. Evaluation} 31(3--4)245--279. doi: 10.1016/S0166-5316(97)00003-5.

\bibitem{dccn2021-1} %17
\Aue{Peshkova, I., E.~Morozov, and M.~Maltseva.}
 2021. On regenerative estimation of extremal index in queueing systems. \textit{Distributed computer and communication networks}. Eds. V.\,M.~Vishnevskiy,
K.\,E.~Samouylov, and D.\,V.~Kozyrev. Lecture notes in computer science ser. Cham, Switzerland: Springer. 13144:251--264. doi: 10.1007/978-3-030-92507-9\_21.

\bibitem{peshkova2022-1} %18
\Aue{Peshkova, I.} 2022.
Srav\-ne\-niye ekst\-re\-mal'\-nykh in\-dek\-sov vre\-men ozhi\-da\-niya v~sis\-te\-makh ob\-slu\-zhi\-va\-niya ${M/G/1}$
[The comparison of waiting time extremal indexes in $M/G/1$ queueing systems]. \textit{Informatika i~ee Primeneniya~--- Inform. Appl.}
16(1):61--67. doi: 10.14357/ 19922264220109.

\bibitem{pesh-mor2022-1} %19
\Aue{Peshkova, I., and  E.~Morozov.}
 2022.  On comparison of multiserver systems with multicomponent mixture distributions. 
 \textit{J.~Math. Sci.} 267(2):260--272. doi:  10.1007/ s10958-022-06132-z.




\bibitem{tomsk2021-1} %20
Peshkova, I.,  E. Morozov, and M. Maltseva. 2022.  On comparison of waiting time extremal indexes  in queueing systems with Weibull service times.
\textit{Information technologies and mathematical modelling. Queueing theory and 
applications.}
Communications in computer and information science ser. Cham, Switzerland: Springer. 1605:80--92. doi: 10.1007/978-3-031-09331-9\_7.

\bibitem{Whitt-1} %21
\Aue{Whitt, W.} 1981. Comparing counting processes and queues. \textit{Adv. Appl. Probab.} 13:207--220.  doi: 10.2307/ 1426475.
 

\bibitem{Ross-1} %22
 \Aue{Ross, S., J.~Shanthikumar, and Z.~Zhu.} 2005.  On increasing--failure--rate random variables.
 \textit{J.~Appl. Probab.} 42:797--809. doi: 10.1239/jap/1127322028.


\bibitem{sigman-1} %23
\Aue{Sigman, K.} 2012. Exact simulation of the stationary distribution of the FIFO $M/G/c$ queue: 
The general case for $\rho < c$. \textit{Queueing Syst.} 70:~37--43.
doi: 10.1007/s11134-011-9266-6.
\end{thebibliography}

 }
 }

\end{multicols}

\vspace*{-6pt}

\hfill{\small\textit{Received October 15, 2022}}

\Contrl

\noindent
\textbf{Peshkova Irina V.} (b.\ 1975)~--- 
Candidate of Science (PhD) in physics and mathematics, associate professor, Petrozavodsk State University, 33~Lenina Prosp., 
Petrozavodsk 185910, Russian Federation; senior scientist, Karelian Research Centre of the Russian Academy of Sciences, 
11~Pushkinskaya Str., Petrozavodsk 185910, Russian Federation; \mbox{iaminova@petrsu.ru}



\label{end\stat}

\renewcommand{\bibname}{\protect\rm Литература}    
   