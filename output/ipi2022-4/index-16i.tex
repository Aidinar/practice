%Информатика Т 16 Год 2022-1\\
\def\stat{cont}
{%\hrule\par
%\vskip 7pt % 7pt
\raggedleft\Large \bf%\baselineskip=3.2ex
А\,В\,Т\,О\,Р\,С\,К\,И\,Й\ \ У\,К\,А\,З\,А\,Т\,Е\,Л\,Ь\ \ З\,А\ \ 2\,0\,2\,2 г. \vskip 17pt
 \hrule
 \par
\vskip 21pt plus 6pt minus 3pt }

\label{st\stat}

\def\tit{\ }

\def\aut{\ }
\def\auf{\ }

\def\leftkol{\ } % ENGLISH ABSTRACTS}

\def\rightkol{\ } %АВТОРСКИЙ УКАЗАТЕЛЬ ЗА 2021 г.} %ENGLISH ABSTRACTS}

\titele{\tit}{\aut}{\auf}{\leftkol}{\rightkol}
\addcontentsline{toc}{subsection}{\textrm\textbf Авторский указатель за 2022 г.}

\vspace*{-24pt}

\noindent
{\tabcolsep=3pt
\begin{tabular}{p{397pt}cc}
&\textbf{Вып.} & \textbf{Стр.}\\[6pt]
\Avtors{Абгарян~К.\,К., Гаврилов~Е.\,С.} Программный комплекс для 
многомасштабного модели-\linebreak
\\[-12pt]
\hspace*{23pt}рования структурных свойств композиционных 
материалов&1&88--97\\
\Avtors{Аблаев~Ф.\,М.} см.\ Андрианов~С.\,Н.&&\\
\Avtors{Агаларов Я.\,М.} Оптимальное управление подключением резервного прибора 
в~системе\linebreak
\\[-12pt]
\hspace*{23pt}массового обслуживания $G/M/1$&4&34--41\\
\Avtors{Агаларов~Я.\,М.} Оптимизация порогового управления переключением 
скорости обслу-\linebreak
\\[-12pt]
\hspace*{23pt}живания в~системе массового обслуживания $G/M/1$&1&73--81\\
\Avtors{Агасандян~Г.\,А.} Многомерные бинарные рынки и~CC-VaR&2&\hphantom{1}2--10\\
\Avtors{Алию~Б., Мачнев~Е.\,А., Мокров~Е.\,В.} Гистерезисное управление нагрузкой 
в~беспроводных\linebreak
\\[-12pt]
\hspace*{23pt}сенсорных сетях&3&83--89\\
\Avtors{Андрианов~С.\,Н., Андрианова~Н.\,С., Аблаев~Ф.\,М., Кочнева~Ю.\,Ю.} 
Контекстный поиск\linebreak
\\[-12pt]
\hspace*{23pt}на фотонах с~использованием тестов Белла&1&20--24\\
\Avtors{Андрианова~Н.\,С.} см.\ Андрианов~С.\,Н.&&\\
\Avtors{Базилевский М.\,П.} Обобщение метода выпрямления искаженных из-за 
мультиколлинеарности коэффициентов для~регрессионных моделей с~различной 
степенью\linebreak
\\[-12pt]
\hspace*{23pt}корреляции объясняющих переменных&4&20--25\\
\Avtors{Бесчастный~В.\,А., Острикова~Д.\,Ю., Шоргин~С.\,Я., Молчанов~Д.\,А., 
Гайдамака~Ю.\,В.} Анализ плотности базовых станций 5G NR для предоставления услуг 
виртуальной\linebreak
\\[-12pt]
\hspace*{23pt}и~дополненной реальности&2&102--108\\
\Avtors{Бесчастный~В.\,А. } см.\ Мачнев Е.\,А.&&\\
\Avtors{Битюков~Ю.\,И.} см.\ Босов~А.\,В.&&\\
\Avtors{Борисов А.\,В.} Общий порядок аппроксимации оценок фильтрации состояний 
марков-\linebreak
\\[-12pt]
\hspace*{23pt}ских скачкообразных процессов по~дискретизованным наблюдениям&4&8--13\\
\Avtors{Босов~А.\,В.} Применение самоорганизующихся нейронных сетей к~процессу 
формиро-\linebreak
\\[-12pt]
\hspace*{23pt}вания индивидуальной траектории обучения&3&\hphantom{1}7--15\\
\Avtors{Босов~А.\,В.} Управление линейным выходом марковской цепи по квадратичному 
крите-\linebreak
\\[-12pt]
\hspace*{23pt}рию. Случай полной информации&2&19--26\\
\Avtors{Босов~А.\,В., Битюков~Ю.\,И., Денискина~Г.\,Ю.} О~поиске оптимальной 
схемы 3D-печати\linebreak
\\[-12pt]
\hspace*{23pt}конструкций из композиционных материалов&1&10--19\\
\Avtors{Босов А.\,В., Иванов А.\,В.} Технология классификации типов контента 
электронного\linebreak
\\[-12pt]
\hspace*{23pt}учебника&4&63--72\\
\Avtors{Брюхов Д.\,О., Ступников~С.\,А.} Логическая реляционная модель структур 
данных для\linebreak
\\[-12pt]
\hspace*{23pt}решения задач в~предметной области управления 
землепользованием&4&93--98\\
\Avtors{Бурцева~С.\,А.} см.\ Власкина~А.\,С.&&\\
\Avtors{Васильев~Н.\,С.} О~достаточных условиях экстремума в~многомерных 
вариационных\linebreak
\\[-12pt]
\hspace*{23pt}задачах&3&39--44\\
\Avtors{Власкина~А.\,С., Бурцева~С.\,А., Кочеткова~И.\,А., Шоргин~С.\,Я.} Управляемая 
система массового обслуживания с~эластичным трафиком и~сигналами для анализа 
нарезки\linebreak
\\[-12pt]
\hspace*{23pt}ресурсов в~сети радиодоступа&3&90--96\\
\Avtors{Гаврилов~Е.\,С.} см.\ Абгарян~К.\,К.&&\\
\Avtors{Гайдамака~Ю.\,В.} см.\ Бесчастный~В.\,А.&&\\
\Avtors{Гайдамака~Ю.\,В.} см.\ Мачнев Е.\,А.&&\\
\Avtors{Горшенин~А.\,К., Гусейнова~Е.\,И.} Повышение доходности торговли на~FOREX 
с~помощью\linebreak
\\[-12pt]
\hspace*{23pt}LSTM-идентификации свечных паттернов и~индикатора тиковых 
объемов&3&26--38\\
\Avtors{Григорьев~О.\,Г.} см.\ Смирнов~И.\,В.&&\\
\end{tabular}
}

\pagebreak

\def\leftkol{АВТОРСКИЙ УКАЗАТЕЛЬ ЗА 2022 г.} % ENGLISH ABSTRACTS}

\def\rightkol{АВТОРСКИЙ УКАЗАТЕЛЬ ЗА 2022 г.} %ENGLISH ABSTRACTS}

%\thispagestyle{myheadings}
\def\leftfootline{\small{\textbf{\thepage}
\hfill ИНФОРМАТИКА И ЕЁ ПРИМЕНЕНИЯ\ \ \ том~16\ \ \ выпуск~4\ \ \ 2022}
}%
 \def\rightfootline{\small{ИНФОРМАТИКА И ЕЁ ПРИМЕНЕНИЯ\ \ \ том~16\ \ \ выпуск~4\ \ \ 2022
 \hfill \textbf{\thepage}}}


\noindent
{\tabcolsep=3pt
\begin{tabular}{p{394pt}cc}
&\textbf{Вып.} & \textbf{Стр.}\\[3pt]
\Avtors{Грушо~А.\,А., Грушо~Н.\,А., Забежайло~М.\,И., Зацаринный~А.\,А., 
Тимонина~Е.\,Е., Шор-}\linebreak
\\[-12pt]
\hspace*{23pt}\textbf{гин~С.\,Я.} Анализ цепочек причинно-следственных связей&2&68--74\\
\Avtors{Грушо А.\,А., Грушо Н.\,А., Забежайло~М.\,И., Смирнов~Д.\,В., Тимонина~Е.\,Е., 
Шоргин~С.\,Я.}\linebreak
\\[-12pt]
\hspace*{23pt}О~безопасной архитектуре вычислительной системы на основе 
микросервисов&4&87--92\\
\Avtors{Грушо~А.\,А., Грушо~Н.\,А., Тимонина~Е.\,Е.} Метаданные в~защищенном 
электронном\linebreak
\\[-12pt]
\hspace*{23pt}документообороте&3&\hphantom{1}97--102\\
\Avtors{Грушо~Н.\,А.} см.\ Грушо~А.\,А.&&\\
\Avtors{Грушо Н.\,А.} см.\ Грушо А.\,А.&&\\
\Avtors{Грушо~Н.\,А.} см.\ Грушо~А.\,А.&&\\
\Avtors{Гусейнова~Е.\,И.} см.\ Горшенин~А.\,К.&&\\
\Avtors{Денискина~Г.\,Ю.} см.\ Босов~А.\,В.&&\\
\Avtors{Драгунов~Н.\,А., Дюкова~Е.\,В.} О~поиске максимальных частых 
и~минимальных нечастых\linebreak
\\[-12pt]
\hspace*{23pt}наборов произведения частичных порядков&1&82--87\\
\Avtors{Дубанов~А.\,А., Нефедова~В.\,А.} Кинематические модели задач преследования 
на~плос-\linebreak
\\[-12pt]
\hspace*{23pt}кости методами параллельного сближения и~погони&3&103--109\\
\Avtors{Дунсяо Гу} см.\ Зацман И.\,М.&&\\
\Avtors{Дурново~А.\,А., Инькова~О.\,Ю., Попкова~Н.\,А.} Принципы описания 
показателей логико-\linebreak
\\[-12pt]
\hspace*{23pt}семантических отношений и~их иерархии&2&52--59\\
\Avtors{Дьяченко~Ю.\,Г.} см.\ Соколов И.\,А.&&\\
\Avtors{Дюкова А.\,П.} см.\ Дюкова Е.\,В.&&\\
\Avtors{Дюкова Е.\,В., Дюкова А.\,П.} О~сложности обучения логических процедур 
классификации&4&57--62\\
\Avtors{Дюкова~Е.\,В.} см.\ Драгунов~Н.\,А.&&\\
\Avtors{Забежайло~М.\,И.} см.\ Грушо А.\,А.&&\\
\Avtors{Забежайло~М.\,И.} см.\ Грушо~А.\,А.&&\\
\Avtors{Зацаринный~А.\,А.} см.\ Грушо~А.\,А.&&\\
\Avtors{Зацман И.\,М.} О~научной парадигме информатики: верхний уровень 
классификации\linebreak
\\[-12pt]
\hspace*{23pt}объектов ее предметной области&4&73--79\\
\Avtors{Зацман~И.\,М.} Средовые модели информационных технологий: теоретические 
основа-\linebreak
\\[-12pt]
\hspace*{23pt}ния построения&3&59--67\\
\Avtors{Зацман~И.\,М., Золотарев~О.\,В., Хакимова~А.\,Х.} Средовые модели извлечения 
из текста\linebreak
\\[-12pt]
\hspace*{23pt}новых терминов и~индикаторов настроений&2&60--67\\
\Avtors{Зацман И.\,М., Золотарев~О.\,В., Хакимова~А.\,Х., Дунсяо~Гу.} Модель 
и~технология\linebreak
\\[-12pt]
\hspace*{23pt}извлечения новых терминов из~медицинских текстов&4&80--86\\
\Avtors{Зейфман~А.\,И.} см.\ Ковалёв~И.\,А.&&\\
\Avtors{Зейфман~А.\,И.} см.\ Сатин~Я.\,А.&&\\
\Avtors{Золотарев~О.\,В.} см.\ Зацман И.\,М.&&\\
\Avtors{Золотарев~О.\,В.} см.\ Зацман~И.\,М.&&\\
\Avtors{Иванов А.\,В.} см.\ Босов А.\,В.&&\\
\Avtors{Инькова~О.\,Ю.} см.\ Дурново~А.\,А.&&\\
\Avtors{Кириков~И.\,А.} см.\ Листопад~С.\,В.&&\\
\Avtors{Кириков~И.\,А.} см.\ Румовская~С.\,Б.&&\\
\Avtors{Киселёв~Г.\,А.} см.\ Смирнов~И.\,В.&&\\
\Avtors{Ковалёв~И.\,А., Сатин~Я.\,А., Синицина~А.\,В., Зейфман~А.\,И.} Об одном 
подходе к~оцениванию скорости сходимости нестационарных марковских моделей систем 
обслужи-\linebreak
\\[-12pt]
\hspace*{23pt}вания&3&75--82\\
\Avtors{Ковалёв~С.\,П.} Алгебраическая спецификация графовых вычислительных 
структур&1&2--9\\
\Avtors{Коновалов~М.\,Г., Разумчик~Р.\,В.} Синтез управления двумерным случайным 
блужданием\linebreak
\\[-12pt]
\hspace*{23pt}с~эталонным стационарным распределением&2&109--117\\
\Avtors{Кочеткова~И.\,А.} см.\ Власкина~А.\,С.&&\\
\Avtors{Кочнева~Ю.\,Ю.} см.\ Андрианов~С.\,Н.&&\\
\Avtors{Кравцова~О.\,А.} Использование критериев стационарности для настройки 
моделей при\linebreak
\\[-12pt]
\hspace*{23pt}прогнозировании временных рядов&2&11--18\\
\Avtors{Кривенко~М.\,П.} Выбор модели при факторизации матрицы данных 
с~пропусками&3&52--58\\
\Avtors{Крюкова~А.\,Л.} см.\ Сатин~Я.\,А.&&\\
\end{tabular}
}

\pagebreak

\def\leftkol{АВТОРСКИЙ УКАЗАТЕЛЬ ЗА 2022 г.} % ENGLISH ABSTRACTS}

\def\rightkol{АВТОРСКИЙ УКАЗАТЕЛЬ ЗА 2022 г.} %ENGLISH ABSTRACTS}

%\thispagestyle{myheadings}
\def\leftfootline{\small{\textbf{\thepage}
\hfill ИНФОРМАТИКА И ЕЁ ПРИМЕНЕНИЯ\ \ \ том~16\ \ \ выпуск~4\ \ \ 2022}
}%
 \def\rightfootline{\small{ИНФОРМАТИКА И ЕЁ ПРИМЕНЕНИЯ\ \ \ том~16\ \ \ выпуск~4\ \ \ 2022
 \hfill \textbf{\thepage}}}


\noindent
{\tabcolsep=3pt
\begin{tabular}{p{394pt}cc}
&\textbf{Вып.} & \textbf{Стр.}\\[3pt]
\Avtors{Курузов~И.\,А.} см.\ Смирнов~И.\,В.&&\\[0.3pt]
\Avtors{Листопад~С.\,В., Кириков~И.\,А.} Разрешение конфликтов в~гибридных 
интеллектуальных\linebreak
\\[-12pt]
\hspace*{23pt}многоагентных системах&1&54--60\\[0.3pt]
\Avtors{Малашенко~Ю.\,Е.} Метрические оценки угловых точек множества достижимых 
межуз-\linebreak
\\[-12pt]
\hspace*{23pt}ловых потоков многопользовательской сети&1&25--31\\[0.3pt]
\Avtors{Малашенко~Ю.\,Е.} Последовательный анализ и~метрические оценки 
предельных рас-\linebreak
\\[-12pt]
\hspace*{23pt}пределений межузловых потоков в~многопользовательской сети&3&45--51\\[0.3pt]
\Avtors{Мачнев Е.\,А., Бесчастный~В.\,А., Острикова~Д.\,Ю., Гайдамака~Ю.\,В., 
Шоргин~С.\,Я.} Об оптимальном расположении антенн для~V2X-соединений 
в~субтерагерцевом диа-\linebreak
\\[-12pt]
\hspace*{23pt}пазоне&4&42--50\\
\Avtors{Мачнев~Е.\,А.} см.\ Алию~Б.&&\\[0.3pt]
\Avtors{Мигуля~М.\,А.} см.\ Шнурков~П.\,В.&&\\[0.3pt]
\Avtors{Мокров~Е.\,В.} см.\ Алию~Б.&&\\[0.3pt]
\Avtors{Молчанов~Д.\,А.} см.\ Бесчастный~В.\,А.&&\\[0.3pt]
\Avtors{Нефедова~В.\,А.} см.\ Дубанов~А.\,А.&&\\[0.3pt]
\Avtors{Нуриев~В.\,А.} Переводческий анализ текста с~применением информационных 
ресурсов:\linebreak
\\[-12pt]
\hspace*{23pt}редуцирование спектра моделей перевода в~надкорпусных базах 
данных&3&68--74\\[0.3pt]
\Avtors{Острикова~Д.\,Ю.} см.\ Бесчастный~В.\,А.&&\\[0.3pt]
\Avtors{Острикова~Д.\,Ю.} см.\ Мачнев Е.\,А.&&\\[0.3pt]
\Avtors{Ошушкова~В.\,С.} см.\ Сатин~Я.\,А.&&\\[0.3pt]
\Avtors{Палионная~С.\,И., Шестаков~О.\,В.} Использование FDR-метода множественной 
провер-\linebreak
\\[-12pt]
\hspace*{23pt}ки гипотез при обращении линейных однородных операторов&2&44--51\\[0.3pt]
\Avtors{Панов~А.\,И.} см.\ Смирнов~И.\,В.&&\\[0.3pt]
\Avtors{Пешкова И.\,В.} Границы экстремального индекса времени ожидания в~системе 
$M/G/1$\linebreak
\\[-12pt]
\hspace*{23pt}с~распределением времени обслуживания в~виде конечной смеси&4&26--33\\[0.3pt]
\Avtors{Пешкова~И.\,В.} Сравнение экстремальных индексов времен ожидания 
в~системах об-\linebreak
\\[-12pt]
\hspace*{23pt}служивания $M/G/1$&1&61--67\\[0.3pt]
\Avtors{Попкова~Н.\,А.} см.\ Дурново~А.\,А.&&\\[0.3pt]
\Avtors{Разумчик~Р.\,В.} см.\ Коновалов~М.\,Г.&&\\[0.3pt]
\Avtors{Рождественский~Ю.\,В.} см.\ Соколов И.\,А.&&\\[0.3pt]
\Avtors{Румовская~С.\,Б., Кириков~И.\,А.} Метод визуализации снижения интенсивности 
и~разре-\linebreak
\\[-12pt]
\hspace*{23pt}шения конфликтов в~гибридных интеллектуальных многоагентных 
системах&2&\hphantom{1}94--101\\[0.3pt]
\Avtors{Сатин~Я.\,А., Крюкова~А.\,Л., Ошушкова~В.\,С., Зейфман~А.\,И.} 
О~монотонности\linebreak
\\[-12pt]
\hspace*{23pt}некоторых классов марковских цепей&2&27--34\\[0.3pt]
\Avtors{Сатин~Я.\,А.} см.\ Ковалёв~И.\,А.&&\\[0.3pt]
\Avtors{Синицина~А.\,В.} см.\ Ковалёв~И.\,А.&&\\[0.3pt]
\Avtors{Синицын~И.\,Н.} Нормализация систем, стохастически не разрешенных 
относительно\linebreak
\\[-12pt]
\hspace*{23pt}производных&1&32--38\\[0.3pt]
\Avtors{Синицын~И.\,Н.} Совместная фильтрация и~распознавание нормальных 
процессов в~сто-\linebreak
\\[-12pt]
\hspace*{23pt}хастических системах, не разрешенных относительно 
производных&2&85--93\\
\Avtors{Смирнов~Д.\,В.} см.\ Грушо А.\,А.&&\\[0.3pt]
\Avtors{Смирнов~И.\,В., Панов~А.\,И., Чуганская~А.\,А., Суворова~М.\,И., 
Киселёв~Г.\,А., Курузов~И.\,А., Григорьев~О.\,Г.} Персональный когнитивный 
ассистент: планирование поведения\linebreak
\\[-12pt]
\hspace*{23pt}на основе сценариев деятельности&1&46--53\\[0.3pt]
\Avtors{Соколов И.\,А., Степченков~Ю.\,А., Дьяченко~Ю.\,Г., Рождественский~Ю.\,В.} 
Оценка надеж-\linebreak
\\[-12pt]
\hspace*{23pt}ности синхронного и~самосинхронного конвейеров&4&2--7\\[0.3pt]
\Avtors{Степченков~Ю.\,А.} см.\ Соколов И.\,А.&&\\[0.3pt]
\Avtors{Ступников~С.\,А.} см.\ Брюхов Д.\,О.&&\\[0.3pt]
\Avtors{Суворова~М.\,И.} см.\ Смирнов~И.\,В.&&\\[0.3pt]
\Avtors{Сучков А.\,П.} Единая модель государственных данных: сценарии 
развития&4&\hphantom{9}99--105\\[0.3pt]
\Avtors{Тимонина~Е.\,Е.} см.\ Грушо А.\,А.&&\\[0.3pt]
\Avtors{Тимонина~Е.\,Е.} см.\ Грушо~А.\,А.&&\\[0.3pt]
\Avtors{Тимонина~Е.\,Е} см.\ Грушо~А.\,А.&&\\
\end{tabular}
}

\pagebreak

\def\leftkol{АВТОРСКИЙ УКАЗАТЕЛЬ ЗА 2022 г.} % ENGLISH ABSTRACTS}

\def\rightkol{АВТОРСКИЙ УКАЗАТЕЛЬ ЗА 2022 г.} %ENGLISH ABSTRACTS}

%\thispagestyle{myheadings}
\def\leftfootline{\small{\textbf{\thepage}
\hfill ИНФОРМАТИКА И ЕЁ ПРИМЕНЕНИЯ\ \ \ том~16\ \ \ выпуск~4\ \ \ 2022}
}%
 \def\rightfootline{\small{ИНФОРМАТИКА И ЕЁ ПРИМЕНЕНИЯ\ \ \ том~16\ \ \ выпуск~4\ \ \ 2022
 \hfill \textbf{\thepage}}}


\noindent
{\tabcolsep=3pt
\begin{tabular}{p{394pt}cc}
&\textbf{Вып.} & \textbf{Стр.}\\[3pt]
\Avtors{Торшин~И.\,Ю.} О~применении топологического подхода к анализу плохо 
формализуемых задач для построения алгоритмов виртуального скрининга кван\-то\-во-ме\-ха\-ни\-че\-ских\linebreak
\\[-12pt]
\hspace*{23pt}свойств органических молекул I:~Основы проблемно ориентированной 
теории&1&39--45\\
\Avtors{Торшин~И.\,Ю.} О~применении топологического подхода к~анализу плохо 
формализуемых задач для построения алгоритмов виртуального скрининга кван\-то\-во-ме\-ха\-ни\-че\-ских 
свойств органических молекул II:~Сопоставление формализма 
с~конструктами\linebreak
\\[-12pt]
\hspace*{23pt}квантовой механики и экспериментальная апробация предложенных 
алгоритмов&2&35--43\\
\Avtors{Хакимова~А.\,Х.} см.\ Зацман И.\,М.&&\\
\Avtors{Хакимова~А.\,Х.} см.\ Зацман~И.\,М.&&\\
\Avtors{Хацкевич В.\,Л.} Нечеткие усредняющие операторы в~задаче агрегирования 
нечеткой\linebreak
\\[-12pt]
\hspace*{23pt}информации&4&51--56\\
\Avtors{Чуганская~А.\,А.} см.\ Смирнов~И.\,В.&&\\
\Avtors{Шведов~А.\,С.} Критерий непустоты эпсилон-ядер для нечетких игр с~нетрансферабель-\linebreak
\\[-12pt]
\hspace*{23pt}ной полезностью и~вычислительные процедуры&3&2--6\\
\Avtors{Шестаков О.\,В.} Несмещенная оценка риска пороговой обработки с~двумя 
пороговыми\linebreak
\\[-12pt]
\hspace*{23pt}значениями&4&14--19\\
\Avtors{Шестаков~О.\,В.} см.\ Палионная~С.\,И.&&\\
\Avtors{Шихиев~Ф.\,Ш.} см.\ Шихиев~Ш.\,Б.&&\\
\Avtors{Шихиев~Ш.\,Б., Шихиев~Ф.\,Ш.} Упрощенный язык зрительных 
образов&1&68--72\\
\Avtors{Шнурков~П.\,В.} Об аналитической структуре некоторых видов целевых 
функционалов,\linebreak
\\[-12pt]
\hspace*{23pt}связанных с~задачами управления полумарковскими случайными 
процессами&2&75--84\\
\Avtors{Шнурков~П.\,В., Мигуля~М.\,А.} Некоторые результаты анализа процесса 
изменения цены\linebreak
\\[-12pt]
\hspace*{23pt}бивалютной корзины на основе методов статистики случайных 
процессов&3&16--25\\
\Avtors{Шоргин~С.\,Я.} см.\ Бесчастный~В.\,А.&&\\
\Avtors{Шоргин~С.\,Я.} см.\ Власкина~А.\,С.&&\\
\Avtors{Шоргин~С.\,Я.} см.\ Грушо А.\,А.&&\\
\Avtors{Шоргин~С.\,Я.} см.\ Грушо~А.\,А.&&\\
\Avtors{Шоргин~С.\,Я.} см.\ Мачнев Е.\,А.&&\\
\end{tabular}
}

%\thispagestyle{myheadings}
\def\leftfootline{\small{\textbf{\thepage}
\hfill ИНФОРМАТИКА И ЕЁ ПРИМЕНЕНИЯ\ \ \ том~16\ \ \ выпуск~4\ \ \ 2022}
}%
 \def\rightfootline{\small{ИНФОРМАТИКА И ЕЁ ПРИМЕНЕНИЯ\ \ \ том~16\ \ \ выпуск~4\ \ \ 2022
 \hfill \textbf{\thepage}}}

 \label{end\stat}

\newpage

\def\stat{cont-e}
{%\hrule\par
%\vskip 7pt % 7pt
\raggedleft\Large \bf%\baselineskip=3.2ex
2\,0\,2\,2\ \ A\,U\,T\,H\,O\,R\ \ I\,N\,D\,E\,X \vskip 17pt
 \hrule
 \par
\vskip 21pt plus 6pt minus 3pt }

\label{st\stat}

\def\tit{\ }

\def\aut{\ }
\def\auf{\ }

\def\leftkol{\ } %2021 AUTHOR INDEX} % ENGLISH ABSTRACTS}

\def\rightkol{\ } %2021 AUTHOR INDEX} %ENGLISH ABSTRACTS}

\titele{\tit}{\aut}{\auf}{\leftkol}{\rightkol}
\addcontentsline{toc}{subsection}{\textrm\textbf 2022 Author Index}

\def\leftfootline{\small{\textbf{\thepage}
\hfill INFORMATIKA I EE PRIMENENIYA~--- INFORMATICS AND APPLICATIONS\ \ \ 2022\
\ \ volume~16\ \ \ issue\ 4}
}%
 \def\rightfootline{\small{INFORMATIKA I EE PRIMENENIYA~--- INFORMATICS AND APPLICATIONS\ \ \ 2022\ \ \ volume~16\ \ \ issue\ 4
\hfill \textbf{\thepage}}}

\vspace*{-24pt}

\noindent
{\tabcolsep=3pt
\begin{tabular}{p{395.89pt}cc}
&\textbf{Issue} & \textbf{Page}\\[6pt]
\Avtors{Abgaryan~K.\,K.\ and Gavrilov~E.\,S.} Software package for multiscale modeling of 
structural\linebreak
\\[-12pt]
\hspace*{23pt}properties of composite materials&1&88--97\\
\Avtors{Ablaev~F.\,M.} see Andrianov~S.\,N.&&\\
\Avtors{Agalarov Ya.\,M.} Optimal control of~a~queue-length dependent additional server 
in~$\mathrm{GI}/M/1$\linebreak
\\[-12pt]
\hspace*{23pt}queue&4&34--41\\
\Avtors{Agalarov~Ya.\,M.} Optimization of the threshold service speed control in the $G/M/1$ 
queue&1&73--81\\
\Avtors{Agasandyan~G.\,A.} Multidimensional binary markets and CC-VaR&2&\hphantom{1}2--10\\
\Avtors{Aliyu~B., Machnev~E.\,A., and Mokrov~E.\,V.} Hysteretic congestion control in 
wireless cloud\linebreak
\\[-12pt]
\hspace*{23pt}sensor networks&3&83--89\\
\Avtors{Andrianov~S.\,N., Andrianova~N.\,S., Ablaev~F.\,M., and Kochneva~Yu.\,Yu.} 
Context query on\linebreak
\\[-12pt]
\hspace*{23pt}photons with the use of Bell tests&1&20--24\\
\Avtors{Andrianova~N.\,S.} see Andrianov~S.\,N.&&\\
\Avtors{Bazilevskiy M.\,P.} Generalization of~a~method for~straightening coefficients 
distorted due~to~mul-\linebreak
\\[-12pt]
\hspace*{23pt}ticollinearity in~regression models with different degrees of~explanatory 
variables correlation&4&20--25\\
\Avtors{Beschastnyi~V.\,A., Ostrikova~D.\,Yu., Shorgin~S.\,Ya., Moltchanov~D.\,A., and 
Gaidamaka~Yu.\,V.}\linebreak
\\[-12pt]
\hspace*{23pt}Density analysis of mmWave NR deployments for delivering scalable 
AR/VR video services&2&102--108\\
\Avtors{Beschastnyi~V.\,A.} see Machnev E.\,A.&&\\
\Avtors{Bityukov~Yu.\,I.} see Bosov~A.\,V.&&\\
\Avtors{Borisov A.\,V.} Total approximation order for~Markov jump process filtering given 
discretized\linebreak
\\[-12pt]
\hspace*{23pt}observations&4&8--13\\
\Avtors{Bosov~A.\,V.} Application of self-organizing neural networks to the process of forming 
an individual\linebreak
\\[-12pt]
\hspace*{23pt}learning path&3&\hphantom{1}7--15\\
\Avtors{Bosov~A.\,V.} Linear output control of Markov chain by square criterion. Complete 
information\linebreak
\\[-12pt]
\hspace*{23pt}case&2&19--26\\
\Avtors{Bosov~A.\,V., Bityukov~Yu.\,I., and Deniskina~G.\,Yu.} About searching for the 
optimal 3D printing\linebreak
\\[-12pt]
\hspace*{23pt}scheme of structures from composite materials&1&10--19\\
\Avtors{Bosov A.\,V. and Ivanov~A.\,V.} Technology for~classification of~content types of~e-textbooks&4&63--72\\
\Avtors{Briukhov D.\,O. and Stupnikov~S.\,A.} Logical relational model of~data structures 
for~problem\linebreak
\\[-12pt]
\hspace*{23pt}solving in~land use management&4&93--98\\
\Avtors{Burtseva~S.\,A.} see Vlaskina~A.\,S.&&\\
\Avtors{Chuganskaya~A.\,A.} see Smirnov~I.\,V&&\\
\Avtors{Deniskina~G.\,Yu.} see Bosov~A.\,V.&&\\
\Avtors{Diachenko~Yu.\,G.} see Sokolov I.\,A.&&\\
\Avtors{Djukova~A.\,P.} see Djukova E.\,V.&&\\
\Avtors{Djukova E.\,V. and Djukova~A.\,P.} On the~complexity of~logical classification 
learning procedures&4&57--62\\
\Avtors{Djukova~E.\,V.} see Dragunov~N.\,A.&&\\
\Avtors{Dongxiao~Gu} see Zatsman I.\,M.&&\\
\Avtors{Dragunov~N.\,A.\ and Djukova~E.\,V.} Finding maximal frequent and minimal 
infrequent sets\linebreak
\\[-12pt]
\hspace*{23pt}in partially ordered data&1&82--87\\
\Avtors{Dubanov~A.\,A.\ and Nefedova~V.\,A.} Kinematic models of pursuit problems on the 
plane\linebreak
\\[-12pt]
\hspace*{23pt}by the methods of parallel approach and pursuit&3&103--109\\
\Avtors{Durnovo~A.\,A., Inkova~O.\,Yu., and Popkova~N.\,A.} Principles of describing 
markers of logical-\linebreak
\\[-12pt]
\hspace*{23pt}semantic relations and their hierarchy&2&52--59\\
\Avtors{Gaidamaka~Yu.\,V.} see Beschastnyi~V.\,A.&&\\
\Avtors{Gaidamaka~Yu.\,V.} see Machnev E.\,A.&&\\
\Avtors{Gavrilov~E.\,S.} see Abgaryan~K.\,K.&&\\

\end{tabular}
}
\pagebreak

\def\leftfootline{\small{\textbf{\thepage}
\hfill INFORMATIKA I EE PRIMENENIYA~--- INFORMATICS AND APPLICATIONS\ \ \ 2022\
\ \ volume~16\ \ \ issue\ 4}
}%
 \def\rightfootline{\small{INFORMATIKA I EE PRIMENENIYA~---
INFORMATICS AND APPLICATIONS\ \ \ 2022\ \ \ volume~16\ \ \ issue\ 4
\hfill \textbf{\thepage}}}

\def\leftkol{2022 AUTHOR INDEX} % ENGLISH ABSTRACTS}

\def\rightkol{2022 AUTHOR INDEX} %ENGLISH ABSTRACTS}


\noindent
{\tabcolsep=3pt
\begin{tabular}{p{395.5pt}cc}
&\textbf{Issue} & \textbf{Page}\\[6pt]
\Avtors{Gorshenin~A.\,K.\ and Guseynova~E.\,I.} Increasing FOREX trading profitability with 
LSTM\linebreak
\\[-12pt]
\hspace*{23pt}candlestick pattern recognition and tick volume indicator&3&26--38\\
\Avtors{Grigoriev~O.\,G.} see Smirnov~I.\,V&&\\[-0.1pt]
\Avtors{Grusho~A.\,A., Grusho~N.\,A., and Timonina~E.\,E.} Metadata in secure electronic 
document\linebreak
\\[-12pt]
\hspace*{23pt}management&3&\hphantom{1}97--102\\[-0.1pt]
\Avtors{Grusho A.\,A., Grusho~N.\,A., Zabezhailo~M.\,I., Smirnov~D.\,V., Timonina~E.\,E., 
and Shorgin~S.\,Ya.}\linebreak
\\[-12pt]
\hspace*{23pt}About the~secure architecture of~a~microservice-based computing 
system&4&87--92\\[-0.1pt]
\Avtors{Grusho~A.\,A., Grusho~N.\,A., Zabezhailo~M.\,I., Zatsarinny~A.\,A., 
Timonina~E.\,E.,}\linebreak
\\[-12pt]
\hspace*{23pt}\textbf{and Shorgin~S.\,Ya.} Cause-and-effect chain analysis&2&68--74\\
\Avtors{Grusho~N.\,A.} see Grusho A.\,A.&&\\[-0.1pt]
\Avtors{Grusho~N.\,A.} see Grusho~A.\,A.&&\\[-0.1pt]
\Avtors{Grusho~N.\,A.} see Grusho~A.\,A.&&\\[-0.1pt]
\Avtors{Guseynova~E.\,I.} see Gorshenin~A.\,K.&&\\
\Avtors{Inkova~O.\,Yu.} see Durnovo~A.\,A.&&\\[-0.1pt]
\Avtors{Ivanov~A.\,V.} see Bosov A.\,V.&&\\[-0.1pt]
\Avtors{Khakimova~A.\,K.} see Zatsman I.\,M.&&\\[-0.1pt]
\Avtors{Khakimova~A.\,K.} see Zatsman~I.\,M.&&\\[-0.1pt]
\Avtors{Khatskevich V.\,L.} Fuzzy averaging operators in~the~problem of~aggregating fuzzy 
information&4&51--56\\[-0.1pt]
\Avtors{Kirikov~I.\,A.} see Listopad~S.\,V.&&\\[-0.1pt]
\Avtors{Kirikov~I.\,A.} see Rumovskaya~S.\,B.&&\\[-0.1pt]
\Avtors{Kiselev~G.\,A.} see Smirnov~I.\,V&&\\[-0.1pt]
\Avtors{Kochetkova~I.\,A.} see Vlaskina~A.\,S.&&\\[-0.1pt]
\Avtors{Kochneva~Yu.\,Yu.} see Andrianov~S.\,N.&&\\[-0.1pt]
\Avtors{Konovalov~M.\,G.\ and Razumchik~R.\,V.} Controlling a bounded two-dimensional 
Markov chain\linebreak
\\[-12pt]
\hspace*{23pt}with a~given invariant measure&2&109--117\\[-0.1pt]
\Avtors{Kovalev~I.\,A., Satin~Y.\,A., Sinitcina~A.\,V., and Zeifman~A.\,I.} On an approach 
for estimating\linebreak
\\[-12pt]
\hspace*{23pt}the rate of convergence for nonstationary Markov models of queueing 
systems&3&75--82\\[-0.1pt]
\Avtors{Kovalyov~S.\,P.} Algebraic specification of graph computational structures&1&2--9\\
\Avtors{Kravtsova~O.\,A.} Model setting using stationarity criteria for time series 
forecasting&2&11--18\\[-0.1pt]
\Avtors{Krivenko~M.\,P.} Model selection for matrix factorization with missing 
components&3&52--58\\[-0.1pt]
\Avtors{Kryukova~A.\,L.} see Satin~Y.\,A.&&\\[-0.1pt]
\Avtors{Kuruzov~I.\,A.} see Smirnov~I.\,V&&\\[-0.1pt]
\Avtors{Listopad~S.\,V.\ and Kirikov~I.\,A.} Conflict resolution in hybrid intelligent multiagent 
systems&1&54--60\\[-0.1pt]
\Avtors{Machnev E.\,A., Beschastnyi~V.\,A., Ostrikova~D.\,Yu., Gaidamaka~Yu.\,V., and 
Shorgin~S.\,Ya.} On\linebreak
\\[-12pt]
\hspace*{23pt}the optimal antenna deployment for~subterahertz V2X 
communications&4&42--50\\[-0.1pt]
\Avtors{Machnev~E.\,A.} see Aliyu~B.&&\\[-0.1pt]
\Avtors{Malashenko~Yu.\,E.} Metric evaluations of the angular points of the set of attainable 
internodal\linebreak
\\[-12pt]
\hspace*{23pt}flows of multiuser network&1&25--31\\[-0.1pt]
\Avtors{Malashenko~Yu.\,E.} Sequential analysis and metric estimates of peak load flows in 
the multiuser\linebreak
\\[-12pt]
\hspace*{23pt}network&3&45--51\\[-0.1pt]
\Avtors{Migulya~M.\,A.} see Shnurkov~P.\,V.&&\\[-0.1pt]
\Avtors{Mokrov~E.\,V.} see Aliyu~B.&&\\[-0.1pt]
\Avtors{Moltchanov~D.\,A.} see Beschastnyi~V.\,A.&&\\[-0.1pt]
\Avtors{Nefedova~V.\,A.} see Dubanov~A.\,A.&&\\[-0.1pt]
\Avtors{Nuriev~V.\,A.} Computer-assisted textual analysis in translation: Reducing the 
spectrum of\linebreak
\\[-12pt]
\hspace*{23pt}translation models in supracorpora databases&3&68--74\\[-0.1pt]
\Avtors{Oshushkova~V.\,S.} see Satin~Y.\,A.&&\\[-0.1pt]
\Avtors{Ostrikova~D.\,Yu.} see Beschastnyi~V.\,A.&&\\[-0.1pt]
\Avtors{Ostrikova~D.\,Yu.} see Machnev E.\,A.&&\\[-0.1pt]
\Avtors{Palionnaya~S.\,I.\ and Shestakov~O.\,V.} The use of the FDR method of multiple 
hypothesis testing\linebreak
\\[-12pt]
\hspace*{23pt}when inverting linear homogeneous operators&2&44--51\\[-0.1pt]
\Avtors{Panov~A.\,I.} see Smirnov~I.\,V&&\\[-0.1pt]
\Avtors{Peshkova I.\,V.} On bounds of~the~stationary waiting time extremal index 
in~$M/G/1$ system\linebreak
\\[-12pt]
\hspace*{23pt}with mixture service times&4&26--33\\[-0.1pt]
\end{tabular}
}
\pagebreak

\def\leftfootline{\small{\textbf{\thepage}
\hfill INFORMATIKA I EE PRIMENENIYA~--- INFORMATICS AND APPLICATIONS\ \ \ 2022\
\ \ volume~16\ \ \ issue\ 4}
}%
 \def\rightfootline{\small{INFORMATIKA I EE PRIMENENIYA~---
INFORMATICS AND APPLICATIONS\ \ \ 2022\ \ \ volume~16\ \ \ issue\ 4
\hfill \textbf{\thepage}}}

\def\leftkol{2022 AUTHOR INDEX} % ENGLISH ABSTRACTS}

\def\rightkol{2022 AUTHOR INDEX} %ENGLISH ABSTRACTS}


\noindent
{\tabcolsep=3pt
\begin{tabular}{p{395.5pt}cc}
&\textbf{Issue} & \textbf{Page}\\[6pt]
\Avtors{Peshkova~I.\,V.} The comparison of waiting time extremal indexes in $M/G/1$ 
queueing systems&1&61--67\\[-0.1pt]
\Avtors{Popkova~N.\,A.} see Durnovo~A.\,A.&&\\[-0.1pt]
\Avtors{Razumchik~R.\,V.} see Konovalov~M.\,G.&&\\[-0.1pt]
\Avtors{Rogdestvenski~Yu.\,V.} see Sokolov I.\,A.&&\\[-0.1pt]
\Avtors{Rumovskaya~S.\,B.\ and Kirikov~I.\,A.} Visual representation of the decrease in 
conflict intensity\linebreak
\\[-12pt]
\hspace*{23pt}and its resolution in hybrid intelligent multiagent 
systems&2&\hphantom{1}94--101\\[-0.1pt]
\Avtors{Satin~Y.\,A., Kryukova~A.\,L., Oshushkova~V.\,S., and Zeifman~A.\,I.} On 
monotonicity of some\linebreak
\\[-12pt]
\hspace*{23pt}classes of Markov chains&2&27--34\\[-0.1pt]
\Avtors{Satin~Y.\,A.} see Kovalev~I.\,A.&&\\[-0.1pt]
\Avtors{Shestakov O.\,V.} Unbiased thresholding risk estimate with two threshold 
values&4&14--19\\[-0.1pt]
\Avtors{Shestakov~O.\,V.} see Palionnaya~S.\,I.&&\\[-0.1pt]
\Avtors{Shihiev~F.\,Sh.} see Shihiev~Sh.\,B.&&\\[-0.1pt]
\Avtors{Shihiev~Sh.\,B.\ and Shihiev~F.\,Sh.} Simplified language for visual images&1&68--72\\[-0.1pt]
\Avtors{Shnurkov~P.\,V.} On the analytical structure of some kinds of target functionals 
associated with\linebreak
\\[-12pt]
\hspace*{23pt}the control problems of semi-Markov stoсhastic processes&2&75--84\\[-0.1pt]
\Avtors{Shnurkov~P.\,V.\ and Migulya~M.\,A.} Some results of the analysis of the process of 
changing\linebreak
\\[-12pt]
\hspace*{23pt}the price of a dual currency basket based on random process statistics 
methods&3&16--25\\[-0.1pt]
\Avtors{Shorgin~S.\,Ya.} see Beschastnyi~V.\,A.&&\\[-0.1pt]
\Avtors{Shorgin~S.\,Ya.} see Grusho A.\,A.&&\\[-0.1pt]
\Avtors{Shorgin~S.\,Ya.} see Grusho~A.\,A.&&\\[-0.1pt]
\Avtors{Shorgin~S.\,Ya.} see Machnev E.\,A.&&\\[-0.1pt]
\Avtors{Shorgin~S.\,Ya.} see Vlaskina~A.\,S.&&\\[-0.1pt]
\Avtors{Shvedov~A.\,S.} A~condition for non-emptiness of the epsilon-core of 
a~nontransferable utility\linebreak
\\[-12pt]
\hspace*{23pt}fuzzy game and computational schemes&3&2--6\\[-0.1pt]
\Avtors{Sinitcina~A.\,V.} see Kovalev~I.\,A.&&\\[-0.1pt]
\Avtors{Sinitsyn~I.\,N.} Joint filtration and recognition of normal proсesses in stochastic 
systems with\linebreak
\\[-12pt]
\hspace*{23pt}unsolved derivatives&2&85--93\\[-0.1pt]
\Avtors{Sinitsyn~I.\,N.} Normalization of systems with stochastically unsolved 
derivatives&1&32--38\\[-0.1pt]
\Avtors{Smirnov~D.\,V.} see Grusho A.\,A.&&\\[-0.1pt]
\Avtors{Smirnov~I.\,V., Panov~A.\,I., Chuganskaya~A.\,A., Suvorova~M.\,I., Kiselev~G.\,A., 
Kuruzov~I.\,A., and}\linebreak
\\[-12pt]
\hspace*{23pt}\textbf{Grigoriev~O.\,G.} Personal cognitive assistant: Planning activity with 
scripts&1&46--53\\[-0.1pt]
\Avtors{Sokolov I.\,A., Stepchenkov Yu.\,A., Diachenko~Yu.\,G., 
and~Rogdestvenski~Yu.\,V.} Synchronous and\linebreak
\\[-12pt]
\hspace*{23pt}self-timed pipeline's reliability 
estimation&4&2--7\\[-0.1pt]
\Avtors{Stepchenkov Yu.\,A.} see Sokolov I.\,A.&&\\[-0.1pt]
\Avtors{Stupnikov~S.\,A.} see Briukhov D.\,O.&&\\[-0.1pt]
\Avtors{Suchkov A.\,P.} Unified model of national data: Development scenarios&4&\hphantom{9}99--105\\
\Avtors{Suvorova~M.\,I.} see Smirnov~I.\,V&&\\[-0.1pt]
\Avtors{Timonina~E.\,E.} see Grusho A.\,A.&&\\[-0.1pt]
\Avtors{Timonina~E.\,E.} see Grusho~A.\,A.&&\\[-0.1pt]
\Avtors{Timonina~E.\,E.} see Grusho~A.\,A.&&\\[-0.1pt]
\Avtors{Torshin~I.\,Yu.} On the application of a~topological approach to analysis of poorly 
formalized problems for constructing algorithms for virtual screening of quantum-mechanical 
properties\linebreak
\\[-12pt]
\hspace*{23pt}of organic molecules I:~The basics of the problem-oriented theory&1&39--45\\[-0.1pt]
\Avtors{Torshin~I.\,Yu.} On the application of a topological approach to analysis of poorly 
formalized problems for constructing algorithms for virtual screening of quantum-mechanical 
properties\linebreak
\\[-12pt]
\hspace*{23pt}of organic molecules II:~Comparison of formalism with constructions of quantum mechan-\linebreak
\\[-12pt]
\hspace*{23pt}ics and experimental approbation of the proposed algorithms&2&35--43\\[-0.1pt]
\Avtors{Vasilyev~N.\,S.} On extremum sufficient conditions in multidimensional variation 
calculus\linebreak
\\[-12pt]
\hspace*{23pt}problems&3&39--44\\[-0.1pt]
\Avtors{Vlaskina~A.\,S., Burtseva~S.\,A., Kochetkova~I.\,A., and Shorgin~S.\,Ya.} 
Controllable queuing system\linebreak
\\[-12pt]
\hspace*{23pt}with elastic traffic and signals for analyzing network 
slicing&3&90--96\\[-0.1pt]
\Avtors{Zabezhailo~M.\,I.} see Grusho A.\,A.&&\\[-0.1pt]
\Avtors{Zabezhailo~M.\,I.} see Grusho~A.\,A.&&\\[-0.1pt]
\Avtors{Zatsarinny~A.\,A.} see Grusho~A.\,A.&&\\[-0.1pt]
\end{tabular}
}
\pagebreak

\def\leftfootline{\small{\textbf{\thepage}
\hfill INFORMATIKA I EE PRIMENENIYA~--- INFORMATICS AND APPLICATIONS\ \ \ 2022\
\ \ volume~16\ \ \ issue\ 4}
}%
 \def\rightfootline{\small{INFORMATIKA I EE PRIMENENIYA~---
INFORMATICS AND APPLICATIONS\ \ \ 2022\ \ \ volume~16\ \ \ issue\ 4
\hfill \textbf{\thepage}}}

\def\leftkol{2022 AUTHOR INDEX} % ENGLISH ABSTRACTS}

\def\rightkol{2022 AUTHOR INDEX} %ENGLISH ABSTRACTS}


\noindent
{\tabcolsep=3pt
\begin{tabular}{p{395.5pt}cc}
&\textbf{Issue} & \textbf{Page}\\[6pt]
\Avtors{Zatsman~I.\,M.} Informatics' medium models of information technology: Theoretical 
foundations\linebreak
\\[-12pt]
\hspace*{23pt}for their creating&3&59--67\\
\Avtors{Zatsman I.\,M.} On the~scientific paradigm of~informatics: The~classification high 
level of~its~objects&4&73--79\\
\Avtors{Zatsman~I.\,M., Zolotarev~O.\,V., and Khakimova~A.\,K.} Medium models for 
discovering novel\linebreak
\\[-12pt]
\hspace*{23pt}terms and sentiments from texts&2&60--67\\
\Avtors{Zatsman I.\,M., Zolotarev~O.\,V., Khakimova~A.\,K., and~Dongxiao~Gu.} Model and 
technology\linebreak
\\[-12pt]
\hspace*{23pt}for discovering new terms in medical texts&4&80--86\\
\Avtors{Zeifman~A.\,I.} see Kovalev~I.\,A.&&\\
\Avtors{Zeifman~A.\,I.} see Satin~Y.\,A.&&\\
\Avtors{Zolotarev~O.\,V.} see Zatsman I.\,M.&&\\
\Avtors{Zolotarev~O.\,V.} see Zatsman~I.\,M.&&\\
\end{tabular}
}

%\thispagestyle{myheadings}
\def\leftfootline{\small{\textbf{\thepage}
\hfill INFORMATIKA I EE PRIMENENIYA~--- INFORMATICS AND APPLICATIONS\ \ \ 2022\
\ \ volume~16\ \ \ issue\ 4}
}%
 \def\rightfootline{\small{INFORMATIKA I EE PRIMENENIYA~---
INFORMATICS AND APPLICATIONS\ \ \ 2022\ \ \ volume~16\ \ \ issue\ 4
\hfill \textbf{\thepage}}}

 \label{end\stat}

\newpage