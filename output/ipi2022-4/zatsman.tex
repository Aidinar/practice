\def\stat{zatsman}

\def\tit{О~НАУЧНОЙ ПАРАДИГМЕ ИНФОРМАТИКИ: ВЕРХНИЙ 
УРОВЕНЬ КЛАССИФИКАЦИИ ОБЪЕКТОВ ЕЕ~ПРЕДМЕТНОЙ 
ОБЛАСТИ$^*$}

\def\titkol{О~научной парадигме информатики: верхний уровень 
классификации объектов ее предметной области}

\def\aut{И.\,М.~Зацман$^1$}

\def\autkol{И.\,М.~Зацман}

\titel{\tit}{\aut}{\autkol}{\titkol}

\index{Зацман И.\,М.}
\index{Zatsman I.\,M.}

{\renewcommand{\thefootnote}{\fnsymbol{footnote}} \footnotetext[1]
{Исследование выполнено с~использованием ЦКП <<Информатика>> ФИЦ 
ИУ РАН}

\renewcommand{\thefootnote}{\arabic{footnote}}
\footnotetext[1]{Федеральный исследовательский центр <<Информатика 
и~управ\-ле\-ние>> Российской академии наук, \mbox{izatsman@yandex.ru}}

\vspace*{-10pt}

\Abst{Рассматривается подход А.~Соломоника к~структурированию научной 
парадигмы <<зрелой>> науки. Согласно его подходу, описание такой науки 
должно включать четыре составляющих (философские основания; 
аксиоматика; классификация исследуемых объектов; система терминов), 
которые могут разрабатываться отдельно, но объединяются в~единую 
и~цельную конструкцию. В~рамках этого подхода предлагается начать 
описание парадигмы информатики с~уточнения ее позиционирования 
в~системе научного познания и построения верхнего уровня классификации 
исследуемых в~ней объектов. Для позиционирования информатики 
предлагается развитие идеи Деннинга и~Розенблюма о~группировке 
научных дисциплин в~четырех отраслях знания. Для построения верхнего 
уровня классификации используется идея Кристена Нюгора о~различении 
в~предметной области информатики объектов ментальной природы 
(концепты знания человека) и~сенсорно воспринимаемых объектов. 
Цель 
статьи состоит в~попытке начать описание научной парадигмы информатики 
на основе подхода А.~Соломоника и~развития идей Деннинга, Нюгора 
и~Розенблюма с~построения верхнего уровня классификации объектов, 
исследуемых в~информатике.}
  
\KW{научная парадигма; составляющие научной парадигмы; сис\-те\-ма 
научного познания; классификация объектов информатики}

  \DOI{10.14357/19922264220411} 
  
\vspace*{-1pt}

\vskip 10pt plus 9pt minus 6pt

\thispagestyle{headings}

\begin{multicols}{2}

\label{st\stat}

\section{Введение} %1

\vspace*{-4pt}

  В работах Деннинга, Розенберга и~Кари~[1--4] дан анализ основных 
этапов эволюции информатики, на протяжении которых расширялась ее 
предметная область по сравнению с первоначальной трактовкой ее 
содержания\footnote[2]{В~1967~г.\ А.~Ньюэлл, А.\,Дж.~Перлис 
и~Х.\,А.~Саймон написали в~журнале \textit{Science}: <<Объектам и~явлениям 
соответствуют те науки, которые их изучают. Появились компьютеры. 
Следовательно, назначение информатики~--- это изучение 
компьютеров>>~[5]. Такой подход к~определению ее содержания 
доминировал долгие годы в~информатике как компьютерной науке.}. Кроме 
компьютерных кодов и~кодируемых ими информационных объектов, 
например текстов, в~информатике предметом исследования стали 
информационные процессы, происходящие в живой природе. По словам 
Деннинга, <<первоначальное определение информатики [как компьютерной 
науки]$\ldots$ сейчас устарело. Она изучает и естественные, 
и~искусственные информационные процессы>>~\cite{1-zac}. Появление 
работ по изучению влияния информационных систем на социум еще больше 
расширило предметную область информатики~[6], и~эта об\-ласть стала 
охватывать широкий спектр феноменов разной природы, связанных 
с~информационными процессами в~технических, живых и~социальных 
системах.
  
  Проблема описания научной парадигмы информатики усложнялась 
с~ростом многообразия объектов ее расширяющейся предметной об\-ласти. 
Это расширение было во многом обусловлено разработкой и~применением 
технологий, охватывающих информационные процессы в~технических, 
живых и~социальных сис\-те\-мах, что было отмечено в~отчете <<Глубинное 
изменение~--- технологические переломные моменты и~социальное 
воздействие>>~[7]\footnote[3]{Данный отчет был подготовлен под эгидой 
Всемирного экономического форума (Давос, Швейцария). В~подготовке 
материалов, использованных в~отчете, приняли участие около 800~экспертов 
и~руководителей отрасли информационных и~коммуникационных 
технологий. Документ содержит перечень новых информационных 
технологий, которые определяют кардинальный характер преобразования 
общества и~экономики, получившего название <<Четвертая промышленная 
революция>>~[8].}.
  
  В~качестве исходной позиции для создания научной парадигмы 
информатики, учи\-ты\-ва\-ющей расшире\-ние ее предметной об\-ласти, в~\mbox{статье} 
пред\-ла\-га\-ет\-ся начать описание этой парадигмы с~по\-зи\-ционирования 
информатики в~сис\-те\-ме научного познания и~по\-стро\-ения верхнего уровня 
классификации  ис\-сле\-ду\-емых в~ней объектов. Для позиционирования 
информатики предлагается развить\linebreak идею Деннинга и~Розенблюма 
о~группировке научных дисциплин в~четырех отраслях знания. Для\linebreak 
по\-стро\-ения верх\-не\-го уровня классификации используется идея Кристена 
Нюгора о~различении\linebreak в~пред\-мет\-ной об\-ласти информатики объектов 
ментальной природы (концепты знания человека) и~сенсорно 
воспринимаемых объектов. 
%
Цель \mbox{статьи} со\-сто\-ит в~попытке начать описание 
научной парадигмы информатики на основе подхода А.~Соломоника 
к~структурированию научной парадигмы <<зрелой>> науки и~развития идей 
Деннинга, Нюгора и~Розенблюма.
  
\section{Информатика в~системе научного познания} %2
  
  Согласно А.~Соломонику, научная парадигма <<зрелой>> науки состоит 
из сле\-ду\-ющих четырех со\-став\-ля\-ющих, которые могут разрабатываться 
отдельно, но объединяются в~единую и~цельную 
  конструкцию~\cite[с.~23--24]{9-zac}:
  \begin{enumerate}[(1)]
\item философские основания;
\item аксиоматика;
\item классификация исследуемых объектов, процессов и явлений;
\item система терминов.
  \end{enumerate}
  
  Сам термин <<научная парадигма>> трактуется им в~соответствии 
с~тео\-ри\-ей Т.~Куна, которая описывает процесс смены научных 
парадигм~[10]. При этом А.~Соломоник отмечает тот факт, что в~книге Куна 
мы не находим ответа на вопрос: <<Из чего долж\-на со\-сто\-ять парадигма 
любой <<зрелой>> науки?>>~\cite[с.~23]{9-zac}.
  
  О~необходимости смены научной парадигмы информатики в~2011~г.\ 
написал Марк Снир, обосновывая это тем, что быст\-рая эволюция 
информационных технологий~(ИТ) стимулирует периодический пересмотр 
теоретического фундамента их разработки и~что именно сейчас настало 
время провести очередной пересмотр. Он пишет, что <<раньше, когда 
с~компьютерами имел дело ограниченный круг лиц, можно было 
рассчитывать на то, что те, кто профессионально общаются с~компьютерами, 
могут к~ним адаптироваться. В~наши дни, когда значительно большее чис\-ло 
людей взаимодействуют с~циф\-ро\-вы\-ми устройствами и~информационными 
сис\-те\-ма\-ми, ИТ тесно вплетены во многие когнитивные и~социальные 
процессы. В~этих условиях нельзя игнорировать эти процессы при создании 
ИТ>>~[11].
  
  Наиболее радикальная попытка смены парадигмы информатики была 
предложена в~2013~г.\ Полом Розенблюмом~[12]: она основывалась на 
новом позиционировании информатики в~сис\-те\-ме\linebreak научного познания ($\mathrm{united}\ 
\mathrm{science}\hm=\mathrm{science}\bigcup$\linebreak $\bigcup \mathrm{humanities}$), предложенном Деннингом 
и~Розенблюмом в~2009~г.~[13]. В~пред\-ла\-га\-емом ими позиционировании 
информатики рассматриваются только естественные науки (science). Они 
группируются в~три ес\-тест\-вен\-но-на\-уч\-ные отрасли: физические науки (куда 
отнесены химия, геология и~другие науки, изучающие неживую материю), 
науки о~жизни и~социальные науки. Математику Деннинг и~Розенблюм не 
относят к~естественным наукам. Для целей данной \mbox{статьи} добавим математику 
в~первую отрасль и~назовем ее <<фи\-зи\-ко-ма\-те\-ма\-ти\-че\-ские науки>>. Так 
как методы и~средства информатики широко используются 
и~в~гуманитарных науках (humanities), то их добавим в~третью отрасль 
и~назовем ее <<со\-цио\-гу\-ма\-ни\-тар\-ные науки>>.
  
  При таком подходе к~группировке наук в~рамках сис\-те\-мы научного 
познания информатика плохо вписывается в~три пе\-ре\-чис\-лен\-ные отрасли (это 
утверждение остается справедливым и после до\-бав\-ле\-ния в~них математики 
и~гуманитарных наук). Хотя методы и~средства информатики широко 
используются в~сис\-те\-ме научного познания, ни одна из этих отраслей не 
изучает информационные трансформации как таковые: информация не 
является в~общем случае ни фи\-зи\-ко-ма\-те\-ма\-ти\-че\-ской, ни живой, ни  
со\-цио\-гу\-ма\-ни\-тар\-ной сущ\-ностью. Поэтому авторы работы~[13] и~предлагают 
рас\-смат\-ри\-вать информатику как четвертую отрасль, тесно 
взаи\-мо\-дей\-ст\-ву\-ющую с тремя остальными.
  
  Развивая идеи, изложенные в~[13], Розенблюм в~своей монографии~[12] 
предложил обозначить четыре отрасли науки сле\-ду\-ющи\-ми литерами, 
которые будем использовать с~учетом расширения первой и~четвертой 
отраслей: фи\-зи\-ко-ма\-те\-ма\-ти\-че\-ские науки~(P), науки о~жизни~(L),  
со\-цио\-гу\-ма\-ни\-тар\-ные\linebreak науки~(S) и~информатика~(C). Далее он рас\-смат\-ри\-ва\-ет 
информационные трансформации, охва\-ты\-ва\-ющие две, три или четыре 
отрасли, включая\linebreak информатику, обозначая со\-от\-вет\-ст\-ву\-ющие им пред\-мет\-ные 
области сочетаниями этих литер и~давая им сле\-ду\-ющие названия: 
диа\-ди\-че\-ский ком\-пью\-тинг (C\;+\;P, C\;+\;L, C\;+\;S); триадический 
ком\-пью\-тинг (C\;+\;P\;+\;S, C\;+\;P\;+\;L, C\;+\;L\;+\;S);\linebreak 
тет\-ра\-ди\-че\-ский 
компьютинг (P\;+\;S\;+\;L\;+\;C), который в~данной статье трактуется как 
синоним ин\-фор\-ма\-тики.
  
  Фундаментальное расширение предметной об\-ласти информатики, 
предложенное Розенблюмом, включает не только информационные 
трансформации, охватывающие все отрасли науки, но и~феномены разной 
природы, связанные с~этими трансформациями. Эта свя\-зан\-ность 
иллюстрируется им\linebreak на примере интерфейса <<мозг--компь\-ютер>> 
сле\-ду\-ющим образом~[12]: <<Для описания взаимосвязей между мозгом 
и~компьютером, нам необходимо рас\-смот\-реть информационные 
трансформации, \mbox{относящиеся} ко всей предметной об\-ласти тет\-ра\-ди\-че\-ско\-го 
компьютинга (P\;+\;S\;+\;L\;+\;C). Эти интерфейсы включают не только 
трансформации электрических потенциалов, ге\-не\-ри\-ру\-емых мозгом~(L), 
и~формирование компьютерных кодов~(C), но также физические 
устройства~(P), которые обеспечивают преобразование электрических 
потенциалов в~коды компьютера, плюс кон\-цеп\-ты знания человека~(S), 
представленные электрическими потенциалами, которые в конечном итоге 
и~будут использоваться для управ\-ле\-ния компьютером или другими 
устройствами>>.

\vspace*{-4pt}
  
\section{Верхний уровень классификации} %3

\vspace*{-4pt}
  
  Если исходить из идеи Пола Розенблюма о~смене парадигмы информатики 
и~предложенного ее развития на основе деления всей сис\-те\-мы научного 
познания (united science) на четыре отрасли, с~не\-из\-беж\-ностью приходится 
констатировать, что феномены одной и~той же природы изучаются в разных 
научных отраслях и~дисциплинах, а~классификации объектов и~парадигмы 
их исследования могут существенно отличаться~[14]. Например, кон\-цеп\-ты 
знания человека изучаются в~задачах пред\-став\-ле\-ния знания 
в~информатике~[15--17], они являются объектами исследований в~науках 
о~жизни~[18] и~со\-цио\-гу\-ма\-ни\-тар\-ных науках~[19]. При этом они 
относятся к~среде одной и той же природы: ментальной среде знания 
человека.
  %
  Поэтому описание верхнего уровня классификации ис\-сле\-ду\-емых 
в~информатике объектов как феноменов разной природы было предложено 
начать с~выделения в~ее предметной об\-ласти сред, вклю\-ча\-ющих объекты 
одной и~той же природы~[20].

 Отметим, что еще в~1986~г.\ Кристен Нюгор 
в~предметной об\-ласти информатики стал различать объекты ментальной 
природы (концепты знания человека) и~сенсорно воспринимаемые объекты 
(например, текс\-ты на естественных языках)~[21], что мож\-но считать 
прообразом предлагаемого выделения сред из предметной об\-ласти 
информатики как тет\-ра\-ди\-че\-ско\-го компьютинга, где каждая из сред включает 
объекты одной природы (так, ментальная среда включает кон\-цеп\-ты 
знания человека,\linebreak цифровая~--- компьютерные коды, информационная~--- 
сенсорно воспринимаемые объекты, например текс\-ты на естественных 
языках, таб\-ли\-цы,\linebreak графики и рисунки). Кристен Нюгор в~1986~г.\ предложил 
сле\-ду\-ющую дефиницию термина <<информатика>>~[21]:
    <<Информатика~--- это наука, которая имеет своей об\-ластью 
[исследований] информационные процессы и~связанные с~ними феномены 
в~артефактах, обществе и~природе>>. 

Затем Нюгор дал расширенное 
толкование термина <<феномен>>~[21]: <<Важными примерами феноменов 
являются: живые организмы, неодушевленные объекты (включая артефакты, 
например \mbox{машины}), события и~процессы (например, выполнение 
компьютерных программ). Мы также можем говорить о~когнитивных 
феноменах, происходящих в~сознании людей, в~отличие от яв\-ных [сенсорно 
воспринимаемых] феноменов, находящихся вне сознания>>.
  
  В~развитие идеи Нюгора совокупность когнитивных феноменов, 
формируемых в~процессах познания, происходящих в~сознании людей 
(далее~--- кон\-цеп\-ты), предлагается назвать \textit{ментальной} \mbox{\textit{средой}} 
предметной области информатики. Со\-во\-куп\-ность сенсорно вос\-при\-ни\-ма\-емых 
феноменов, находящихся вне сознания, предлагается назвать 
\textit{информационной средой} предметной об\-ласти информатики. Кроме 
того, предлагается выделить еще как минимум три среды:
  \begin{enumerate}[(1)]
  \item \textit{цифровая среда}~--- совокупность компьютерных кодов;
  \item \textit{нейросреда} электрических потенциалов и~магнитных полей, 
ге\-не\-ри\-ру\-емых мозгом, которые используются в~ИТ управ\-л\-ения 
роботизированной рукой~[22] и~других ИТ на основе интерфейсов  
<<мозг--компь\-ютер>>;
  \item \textit{ДНК-среда}~--- совокупность цепочек РНК 
и~ДНК\footnote{Например, модели трансляции естественных ДНК, 
созданные микробиологами, используются при разработке методов записи 
и~хранения данных с~использованием синтезированных цепочек ДНК.}.
  \end{enumerate}
  
  Согласно перечисленным средам, выделенным из предметной об\-ласти 
информатики как тет\-ра\-ди\-че\-ско\-го компьютинга, на верхнем уровне 
классификации объектов ее исследований предлагается указать как минимум 
пять сред, каждая из которых включает объекты одной природы: 
\textit{ментальная, информационная, циф\-ро\-вая, нейро- и ДНК-среда}.
  %
  При этом с~ростом разнообразия природы объектов предметной об\-ласти 
информатики верх\-ний уровень классификации может пополняться новыми 
средами, природа объектов которых отличается от природы сред, ранее 
включенных в~этот уровень классификации, если при проектировании ИТ 
встретятся объекты, которые по своей природе не относятся ни к~одной из 
ранее уже выделенных сред~[17]. В~будущей парадигме информатики его 
предлагается сделать открытым, что обуслов\-ле\-но разнообразием природы ее 
объектов и~возможным включением в~ее предметную об\-ласть объектов 
новой природы (от\-кры\-тость уровня классификации). Отметим, что 
разнообразие природы объектов предметной об\-ласти информатики служит 
объектом пристального изуче\-ния в~сфере компьютерного образования~[23].
  
  Разнообразие природы объектов, сложившееся к~на\-сто\-яще\-му времени, 
существенно усложняет разработку стратегии преподавания информатики на 
всех уровнях образования в~час\-ти описания ее тео\-ре\-ти\-че\-ских оснований. 
Например, в~стратегии Европейского компьютерного образования не 
описана ни ее предметная об\-ласть, ни классификация объектов ее 
исследований~[24]. Отсутствие такого описания мотивируется его большей 
слож\-ностью по срав\-не\-нию с~естественными науками~\cite[с.~7]{25-zac}:
    <<В~то время как естественные науки определяются применительно 
к~миру, в~котором мы живем, информатику как научную дисциплину 
определить слож\-нее; у~нее нет эмпирических основ, как у~естественных 
наук; это нечто большее, чем использование только логических выводов (как 
в~математике); и~это далеко не только сочетание инженерных принципов 
и~технологий>>.
  
  Описание сред разной природы на верхнем уровне классификации 
объектов исследований в~информатике дает воз\-мож\-ность увидеть весь 
спектр тео\-ре\-ти\-че\-ски воз\-мож\-ных интерфейсов меж\-ду объектами разной 
природы и~при проектировании ИТ, и~в~процессе преподавания 
информатики. В~работе~[20] дано описание~11~видов интерфейсов разного 
порядка: второго (6~видов), третьего (4~вида) и~четвертого (1~вид), которые 
могут использоваться в~ИТ, охва\-ты\-ва\-ющих объекты четырех сред: 
ментальной, информационной, циф\-ро\-вой и~нейросреды.
  
  Из 11~видов интерфейсов, которые теоретически могут встретиться при 
проектировании ИТ, охва\-ты\-ва\-ющих сущности этих четырех сред, наиболее 
широко используется интерфейс второго порядка на границе между 
\textit{информационной и~циф\-ро\-вой средами}. В~информационных сис\-те\-мах 
применяются кодовые таб\-ли\-цы, например Unicode, которые служат наиболее 
распространенным способом \textit{сим\-мет\-рич\-ной реализации} этого 
интерфейса, т.\,е.\ каж\-до\-му символу соответствует один компьютерный код 
и~наоборот. Однако есть примеры и~ \textit{асим\-мет\-рич\-ной реализации} 
этого интерфейса~\cite[с.~222--227]{26-zac}.
  
  Теоретически есть еще пять интерфейсов второго порядка на границах 
между следующими сре\-дами:
  \begin{enumerate}[(1)]
  \item ментальной и информационной;
  \item ментальной и цифровой;
  \item ментальной и нейросредой;
  \item информационной и нейросредой;
  \item цифровой и нейросредой.
  \end{enumerate}
  
  Первый из вышеперечисленных интерфейсов (между ментальной 
и~информационной средами) реализуется знаковыми сис\-те\-ма\-ми, которые 
далеко \textit{не всегда обеспечивают сим\-мет\-рич\-ную его реализацию}. 
Известно, что вербальным знаковым сис\-те\-мам естественных языков 
свойственна асим\-мет\-рия, когда одно слово имеет несколько смыс\-ло\-вых 
значений (омонимия и~полисемия), а~одно значение может быть выражено 
разными словами (синонимия)~\cite[с.~47]{27-zac}. Асим\-мет\-рия вербальных 
знаковых сис\-тем существенно услож\-ня\-ет разработку сис\-тем обработки 
текс\-тов на естественном языке и~проектирование ИТ пред\-став\-ле\-ния знания 
в~информационных сис\-те\-мах из-за не\-об\-хо\-ди\-мости разрешения лексической 
не\-од\-но\-знач\-ности~[28--30].
  
\section{Заключение} %4
  
  Согласно А.~Соломонику, описание научной парадигмы <<зрелой>> 
науки долж\-но включать четыре со\-став\-ля\-ющих (философские основания; 
аксиоматика; классификация ис\-сле\-ду\-емых объектов, процессов и~явлений; 
система терминов). В~\mbox{статье} рас\-смот\-ре\-на только одна со\-став\-ля\-ющая 
(классификация) и~дано описание только верх\-не\-го уровня классификации 
объектов информатики. При этом использовалось позиционирование 
информатики в~сис\-те\-ме научного познания, которое развивает идею 
Деннинга и~Розенблюма о~группировке научных дис\-цип\-лин в~четырех 
отраслях знания, так как объединяет математику, естественные 
и~гуманитарные науки.
  
  Предлагаемый вариант формирования верхнего уровня классификации 
объектов информатики рас\-смот\-рен на примере пяти сред разной природы, 
выделенных из ее предметной об\-ласти: ментальной, информационной, 
циф\-ро\-вой, нейро- и~ДНК-сред. Верх\-ний уровень классификации объектов 
исследований в~информатике дает воз\-мож\-ность увидеть весь спектр 
тео\-ре\-ти\-че\-ски воз\-мож\-ных интерфейсов меж\-ду объектами разной природы 
и~при проектировании ИТ, и~в~процессе преподавания информатики. Для 
случая четырех сред разной природы он включает~11~видов интерфейсов 
второго, треть\-его и~четвертого порядка для ИТ, охва\-ты\-ва\-ющих объекты этих 
сред. Если в~будущем появится не\-об\-хо\-ди\-мость в~проектировании ИТ, 
охва\-ты\-ва\-ющих объекты шести или более видов разной природы, то 
пред\-ла\-га\-емый подход к~выделению сред покажет весь спектр интерфейсов, 
которые тео\-ре\-ти\-че\-ски может понадобиться реализовать в~этих ИТ.
  
  В~заключение отметим, что после по\-стро\-ения верхнего уров\-ня 
классификации объектов информатики планируется дать дефиниции тех 
терминов, которые использовались при его описании, и~затем по\-стро\-ить 
сле\-ду\-ющий уровень классификации, выделяя из каж\-дой среды со\-став\-ля\-ющие 
ее домены. Например, для циф\-ро\-вой среды планируется описать четыре 
домена (электрический, магнитный, оптический и~генетический, 
содержащий \textit{синтезированные цепочки ДНК}), а~так\-же интерфейсы 
между ними. Также планируется дать повторное описание верх\-не\-го уровня 
классификации на основе другого подхода к~объединению научных 
дис\-цип\-лин~--- в~три отрасли знания: естественные науки, социальные 
и~гуманитарные науки, формальные науки.
  
  {\small\frenchspacing
   {%\baselineskip=10.8pt
   %\addcontentsline{toc}{section}{References}
   \begin{thebibliography}{99}
  \bibitem{1-zac}
   \Au{Denning P.} Computing is a~natural science~// Commun.  ACM, 2007. Vol.~50. No.\,7. P.~13--18.
  \bibitem{2-zac}
\Au{Rozenberg G.} Computer science, informatics, and natural computing~--- 
personal reflections~// New computational paradigms: Changing conceptions of 
what is computable~/ Eds. S.\,B.~Cooper, B.~L$\ddot{\mbox{o}}$we, 
A.~Sorbi.~--- New York, NY, USA: Springer Science\;+\;Business Media LLC, 
2008. P.~373--379.
  \bibitem{3-zac}
\Au{Kari L., Rozenberg~G.} The many facets of natural computing~// Commun. 
ACM, 2008. Vol.~51. No.\,10. P.~72--83.
  \bibitem{4-zac}
\Au{Denning P.} The science in computer science~// Commun.  ACM, 2013. 
Vol.~56. No.\,5. P.~35--38.
  \bibitem{5-zac}
\Au{Newell A., Perlis~A., Simon~H.} Computer science~// Science, 1967. 
Vol.~157. No.\,3795. P.~1373--1374.
  \bibitem{6-zac}
\Au{Nygaard K., \mbox{Н{\normalsize\!\!{\!\ptb{\r{\!a}}}}ndlykken}~P.} The system development 
process~--- its setting, some problems and needs for methods~// Software 
Engineering Environments Symposium Proceedings.~--- Amsterdam, 1981. P.~157--172.
  \bibitem{7-zac}
Deep shift: Technology tipping points and societal impact.~--- Geneva, 
Switzerland: World Economic Forum, 2015. {\sf 
http://www3.weforum.org/docs/WEF\_GAC15\_ Technological\_Tipping\_Points\_report\_2015.pdf.}
  \bibitem{8-zac}
\Au{Шваб К.} Четвертая промышленная революция~/ Пер. c~англ.~--- М.: 
Эксмо, 2018. 288~с. (\Au{Schwab~K.} The fourth industrial revolution.~--- 
Geneva, Switzerland: World Economic Forum, 2016. 172~p.)
  \bibitem{9-zac}
\Au{Соломоник А.} Парадигма семиотики.~--- Минск: МЕТ, 2006. 335~с.
  \bibitem{10-zac}
\Au{Кун Т.} Структура научных революций~/ Пер. c~англ.~--- М.: Прогресс, 
1977. 302~с. (\Au{Kuhn~T.} The structure of scientific revolutions.~--- Chicago, 
IL, USA: University of Chicago Press, 1962. 264~p.)
  \bibitem{11-zac}
\Au{Snir M.} Computer and information science and engineering: One discipline, 
many specialties~// Commun. ACM, 2011. Vol.~54. No.\,3. P.~38--43.
  \bibitem{12-zac}
\Au{Rosenbloom P.\,S.} On computing: The fourth great scientific domain.~--- 
Cambridge, MA, USA: MIT Press, 2013. 308~p.
  \bibitem{13-zac}
\Au{Denning~P., Rosenbloom~P.} Computing: The fourth great domain of 
science~// Commun. ACM, 2009. Vol.~52. No.\,9. P.~27--29.
  \bibitem{14-zac}
\Au{Tedre M., Pajunen~J.} Grand theories or design guidelines? Perspectives on 
the role of theory in computing education research~// ACM T. Comput. Educ., 
2022 (in press). doi: 10.1145/3487049.
  
  \bibitem{16-zac} %15
\Au{Зацман И.\,М., Косарик~В.\,В., Курчавова~О.\,А.} Задачи пред\-став\-ле\-ния 
личностных и~коллективных кон\-цеп\-тов в~циф\-ро\-вой среде~// Информатика 
и~её применения, 2008. Т.~2. Вып.~3. С.~54--69.
  \bibitem{17-zac} %16
\Au{Zatsman I.} Tracing emerging meanings by computer: Semiotic framework~// 
13th European Conference on Knowledge Management Proceedings.~--- Reading, 
U.K.: Academic Publishing International Ltd., 2012. Vol.~2. P.~1298--1307.

\bibitem{15-zac} %17
\Au{Зацман И.\,М.} Таблица интерфейсов информатики как 
 ин\-фор\-ма\-ци\-он\-но-ком\-пью\-тер\-ной науки~//  
На\-уч\-но-тех\-ни\-че\-ская информация. Сер.~1: Организация и~методика 
информационной работы, 2014. №\,11. С.~\mbox{1--15}.

  \bibitem{18-zac}
\Au{Baars B., Gage~N.} Cognition, brain, and consciousness: Introduction to 
cognitive neuroscience.~--- Amsterdam, Netherlands: Academic Press/Elsevier, 
2010. 677~p.
  \bibitem{19-zac}
\Au{Eco U.} A~theory of semiotics.~--- Bloomington, IL, USA: Indiana University 
Press, 1976. 356~p.
  \bibitem{20-zac}
\Au{Зацман И.\,М.} Тео\-ре\-ти\-че\-ские основания компьютерного образования: 
среды предметной об\-ласти информатики как основание классификации ее 
объектов~// Сис\-те\-мы и~средства информатики, 2022. Т.~32. №\,4.  
С.~77--89.
  \bibitem{21-zac}
\Au{Nygaard K.} Program development as a~social activity~// Information 
Processing: 10th World Computer Congress Proceedings~/ Ed. H.-J.~Kugler.~--- Amsterdam, Netherlands: 
Elsevier Science Publs. B.~V. (North Holland), IFIP, 1986. P.~189--198.
  \bibitem{22-zac}
\Au{Зацман И.\,М.} Интерфейсы треть\-его порядка в~информатике~// 
Информатика и~её применения, 2019. Т.~13. Вып.~3. С.~82--89.
  \bibitem{23-zac}
\Au{Kaya E., Newley~A., Yesilyurt~E., Deniz~H.} Nature of computer science: 
Identification of K-12 accessible nature of computer science tenets and 
development of an open-ended nature of computer science instrument~// 17th  
Conference on International Computing Education Research Proceedings.~---  
New York, NY, USA: ACM, 2021. P.~426. doi: 10.1145/3446871.3469784.
  \bibitem{24-zac}
\Au{Caspersen M.\,E., Gal-Ezer~J., McGettrick~A., Nardelli~E.} Informatics for 
all: The strategy.~--- New York, NY, USA: ACM, 2018. 16~p.
  \bibitem{25-zac}
  Committee on European Computing Education. Informatics education in Europe: Are we all in the same boat?~--- New York, NY, USA: 
ACM, 2017. Technical Report.  251~p. doi: 10.1145/3106077.
  \bibitem{26-zac}
\Au{Зацман И.\,М.} Концептуальный поиск и~качество информации.~--- М.: 
Наука, 2003. 272~с.
  \bibitem{27-zac}
Лингвистический энциклопедический словарь~/ Под ред. В.\,Н.~Ярцевой.~--- 
М.: Советская энциклопедия, 1990. 685~с.
  \bibitem{28-zac}
\Au{Bolshina~A., Loukachevitch~N.} Generating training data for word sense 
disambiguation in Russian~// Компьютерная лингвистика и~интеллектуальные 
технологии:\linebreak\vspace*{-12pt}

\pagebreak

\noindent
 По мат-лам ежегодной Междунар. конф. <<Диалог>>.~--- 
М.: РГГУ, 2020. Вып.~19(26). С.~119--132.
  \bibitem{29-zac}
\Au{Bolshina A., Loukachevitch~N.} All-words word sense disambiguation for 
Russian using automatically generated text collection~// Cybernetics Information 
Technologies, 2020. Vol.~20. No.\,4. C.~90--107.
  \bibitem{30-zac}
\Au{Bolshina~A., Loukachevitch~N.} Automatic labelling of genre-specific 
collections for word sense disambiguation in Russian~// Artificial intelligence~/ Eds.\ S.\,O.~Kuznetsov, A.\,I.~Panov, 
K.\,S.~Yakovlev.~--- Lecture notes in computer science ser.~---  Cham, Switzerland: Springer, 2020. 
Vol.~12412. P.~215--227.

  \end{thebibliography}

 }
 }

\end{multicols}

\vspace*{-6pt}

\hfill{\small\textit{Поступила в~редакцию 14.10.22}}

\vspace*{8pt}

%\pagebreak

%\newpage

%\vspace*{-28pt}

\hrule

\vspace*{2pt}

\hrule

%\vspace*{-2pt}

\def\tit{ON THE SCIENTIFIC PARADIGM OF~INFORMATICS: 
THE~CLASSIFICATION HIGH LEVEL OF~ITS~OBJECTS}

\def\titkol{On the scientific paradigm of informatics: The classification high level 
of its objects}

\def\aut{I.\,M.~Zatsman}

\def\autkol{I.\,M.~Zatsman}

\titel{\tit}{\aut}{\autkol}{\titkol}

\vspace*{-8pt}


\noindent
Federal Research Center ``Computer Science and Control'' of the Russian Academy of Sciences,  
44-2~Vavilov Str., Moscow 119333, Russian Federation

\def\leftfootline{\small{\textbf{\thepage}
\hfill INFORMATIKA I EE PRIMENENIYA~--- INFORMATICS AND
APPLICATIONS\ \ \ 2022\ \ \ volume~16\ \ \ issue\ 4}
}%
 \def\rightfootline{\small{INFORMATIKA I EE PRIMENENIYA~---
INFORMATICS AND APPLICATIONS\ \ \ 2022\ \ \ volume~16\ \ \ issue\ 4
\hfill \textbf{\thepage}}}

\vspace*{3pt}

\Abste{The approach of A.~Solomonik to structuring the scientific paradigm of 
``mature'' science is considered. According to his approach, the description of 
such a~science should include four components (philosophical foundations; 
axiomatics; classification of its objects; and system of terms) which can be developed 
separately but combined into a~single and integral structure. Within the 
framework of this approach, it is proposed to begin the description of the paradigm 
of informatics by clarifying its positioning in united science 
(=\;science\,$\cup$\,humanities) and constructing the classification high level of 
its objects. To position informatics, it is proposed to develop the idea of Denning 
and Rosenbloom about grouping scientific disciplines in Four Great Scientific 
Domains. To build the high level of classification, Kristen Nygaard's idea of 
distinguishing objects of mental nature (concepts of human knowledge) and 
sensory-perceived objects is used. The purpose of the paper is to attempt to begin 
the description of the scientific paradigm of informatics based on the approach of 
A. Solomonik and the development of the ideas of Denning, Nygaard, and 
Rosenbloom with the construction of the high level of classification.}

\KWE{scientific paradigm; scientific paradigm components; united science; 
classification informatics objects}

  \DOI{10.14357/19922264220411} 

%\vspace*{-16pt}

\Ack
\noindent 
The research was carried out using infrastructure of shared research facilities CKP 
``Informatics'' of FRC CSC RAS.

%\vspace*{4pt}

  \begin{multicols}{2}

\renewcommand{\bibname}{\protect\rmfamily References}
%\renewcommand{\bibname}{\large\protect\rm References}

{\small\frenchspacing
 {%\baselineskip=10.8pt
 \addcontentsline{toc}{section}{References}
 \begin{thebibliography}{99}
\bibitem{1-zac-1}  
\Aue{Denning,~P.} 2007.Computing is a~natural science. \textit{Commun. ACM} 
50(7):13--18.
\bibitem{2-zac-1}  
\Aue{Rozenberg,~G.} 2008. 
Computer science, informatics, and natural computing~--- personal reflections. 
\textit{New computational paradigms: Changing conceptions of 
what is computable}. Eds.\ S.\,B.~Cooper, 
B.~L$\ddot{\mbox{o}}$we, and A.~Sorbi. New York, NY: Springer 
Science\;+\;Business Media LLC. 373--379.
\bibitem{3-zac-1}  
\Aue{Kari,~L., and G.~Rozenberg.} 2008. 
The many facets of natural computing. \textit{Commun. ACM} 51(10):72--83.
\bibitem{4-zac-1}  
\Aue{Denning,~P.} 2013. 
The science in computer science. \textit{Commun. ACM} 56(5):35--38.
\bibitem{5-zac-1}  
\Aue{Newell,~A., A.~Perlis, and H.~Simon.} 1967. 
Computer science. \textit{Science} 157(3795):1373--1374.
\bibitem{6-zac-1}  
\Aue{Nygaard,~K., and P.~\mbox{H{\normalsize{\!\!\ptb{\r{\!\!a}}}}ndlykken}.} 1981. The system development 
process~--- its setting, some problems and needs for methods. \textit{Software 
Engineering Environments Symposium Proceedings.} Amsterdam. 157--172.
\bibitem{7-zac-1}  
Deep shift: Technology tipping points and societal impact. 2015. World 
Economic Forum. Geneva, Switzerland. Available at: {\sf 
http://www3.weforum.org/docs/WEF\_ GAC15\_Technological\_Tipping\_Points\_report\_2015.pdf} (accessed 
October~31, 2022).
\bibitem{8-zac-1}  
\Aue{Schwab,~K.} 2016. \textit{The fourth industrial revolution.} Geneva, 
Switzerland: World Economic Forum. 172~p.
\bibitem{9-zac-1}  
\Aue{Solomonik,~A.} 2006. \textit{Paradigma semiotiki} [The paradigm of 
semiotics]. Minsk: MET Publs. 335~p.
\bibitem{10-zac-1}  
\Aue{Kuhn,~T.} 1962. \textit{The structure of scientific revolutions.} Chicago, IL: 
University of Chicago Press. 264~p.
\bibitem{11-zac-1}  
\Aue{Snir,~M.} 2011. Computer and information science and engineering: One 
discipline, many specialties. \textit{Commun. ACM} 54(3):38--43.
\bibitem{12-zac-1}  
\Aue{Rosenbloom,~P.} 2013. \textit{On computing: The fourth great scientific 
domain.} Cambridge, MA: MIT Press. 308~p.
\bibitem{13-zac-1}  
\Aue{Denning,~P., and P.~Rosenbloom.} 2009. Computing: The fourth great 
domain of science. \textit{Commun. ACM} 52(9):\linebreak 27--29.
\bibitem{14-zac-1}  
\Aue{Tedre,~M., and J.~Pajunen.} 2022 (in press). Grand theories or design 
guidelines? Perspectives on the role of theory in computing education research. 
\textit{ACM T. Comput. Educ.} doi: 10.1145/3487049.


\bibitem{16-zac-1}  
\Aue{Zatsman,~I., V.~Kosarik, and O.~Kurchavova.} 2008. Zadachi 
predstavleniya lichnostnykh i~kollektivnykh kon\-tsep\-tov v~tsifrovoy srede 
[Problems of representation of personal and collective concepts in the digital 
medium]. \textit{Informatika i~ee Primeneniya~---  Inform. Appl.} 2(3):54--69.
\bibitem{17-zac-1}  
\Aue{Zatsman,~I.} 2012. Tracing emerging meanings by computer: Semiotic 
framework. \textit{13th European Conference on Knowledge Management 
Proceedings.} Reading, U.K.: Academic Publishing International Ltd.  
2:1298--1307.

\bibitem{15-zac-1}  %17
\Aue{Zatsman,~I.} 2014. Table of interfaces of informatics as computer and 
information science. \textit{Scientific Technical Information Processing} 
 41(4):233--246.
 
\bibitem{18-zac-1}  
\Aue{Baars,~B., and N.~Gage.} 2010. \textit{Cognition, brain, and 
consciousness: Introduction to cognitive neuroscience.} Amsterdam: Academic 
Press/Elsevier. 677~p.
\bibitem{19-zac-1}  
\Aue{Eco,~U.} 1976. \textit{A~theory of semiotics.} Bloomington, IL: Indiana 
University Press. 356~p.
\bibitem{20-zac-1}  
\Aue{Zatsman,~I.} 2022. Teoreticheskie osnovaniya komp'yu\-ter\-no\-go 
obrazovaniya: sredy predmetnoy ob\-lasti informatiki kak osnovanie klassifikatsii 
ee ob''\-ek\-tov [Theoretical foundations of digital education: Subject domain media 
of informatics as the base of its objects' classification]. \textit{Sistemy i~Sredstva 
Informatiki~--- Systems and Means of Informatics} 32(4):77--89.
\bibitem{21-zac-1}  
\Aue{Nygaard,~K.} 1986. Program development as a~social activity.  
\textit{Information Processing: 10th World Computer Congress Proceedings.} Ed. H.-J.~Kugler.  
Amsterdam, Netherlands: Elsevier Science Publs.\ B.~V. (North Holland), IFIP.\linebreak 
189--198.

\columnbreak

\bibitem{22-zac-1}  
\Aue{Zatsman,~I.\,M.} 2019. Interfeysy tret'ego poryadka v~informatike 
[Third-order interfaces in informatics]. \textit{Informatika i~ee Primeneniya~--- Inform. 
Appl.} 13(3):82--89.
\bibitem{23-zac-1}  
\Aue{Kaya,~E., A.~Newley, E.~Yesilyurt, and H.~Deniz.} 2021. Nature of 
computer science: Identification of K-12 accessible nature of computer science 
tenets and development of an open-ended nature of computer science instrument.  
\textit{17th Conference on International Computing Education Research 
Proceedings.} New York, NY: ACM. 426.
doi: 10.1145/3446871.3469784.
\bibitem{24-zac-1}  
\Aue{Caspersen,~M.\,E., J.~Gal-Ezer, A.~McGettrick, and E.~Nardelli.} 2018. 
\textit{Informatics for all: The strategy.} New York, NY: ACM. 16~p.
\bibitem{25-zac-1}  
Committee on European Computing Education. 2017. Informatics education in 
Europe: Are we all in the same boat? New York, NY: ACM. Technical Report. 
251~p. doi: 10.1145/3106077.
\bibitem{26-zac-1}  
\Aue{Zatsman,~I.} 2003. \textit{Kontseptual'nyy poisk i~kachestvo informatsii} 
[Conceptual retrieval and quality of information]. Moscow: Nauka. 272~p.
\bibitem{27-zac-1}  
\Aue{Yartseva,~V.\,N., ed.} 1990. \textit{Lingvisticheskiy entsiklopedicheskiy 
slovar'} [Linguistic encyclopedic dictionary]. Moscow: Soviet Encyclopedia. 
685~p.
\bibitem{28-zac-1}  
\Aue{Bolshina,~A., and N.~Loukachevitch.} 2020. Generating training data for 
word sense disambiguation in Russian. \textit{Computer Linguistic and Intellectual Technologies: Conference (International) 
``Dialog'' Proceedings}. Moscow. 119--132.
\bibitem{29-zac-1}  
\Aue{Bolshina,~A., and N.~Loukachevitch.} 2020. All-words word sense 
disambiguation for Russian using automatically generated text collection. 
\textit{Cybernetics Information Technologies} 20(4):90--107.
\bibitem{30-zac-1}  
\Aue{Bolshina,~A., and N.~Loukachevitch.} 2020. Automatic labelling of  
genre-specific collections for word sense disambiguation in Russian. \textit{Artificial intelligence}. Eds.\ S.\,O.~Kuznetsov, 
A.\,I.~Panov, and K.\,S.~Yakovlev. Lecture notes in computer science ser.
 Cham,  Switzerland: Springer. 12412:215--227.
\end{thebibliography}

 }
 }

\end{multicols}

\vspace*{-6pt}

\hfill{\small\textit{Received October 14, 2022}}


\vspace*{-12pt}

\Contrl

\noindent
\textbf{Zatsman Igor M.} (b.\ 1952)~--- Doctor of Science in technology, head of 
department, Institute of Informatics Problems, Federal Research Center 
``Computer Science and Control'' of the Russian Academy of Sciences,  
44-2~Vavilov Str., Moscow 119333, Russian Federation;  
\mbox{izatsman@yandex.ru}

     
\label{end\stat}

\renewcommand{\bibname}{\protect\rm Литература}    

    