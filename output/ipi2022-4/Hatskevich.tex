\def\stat{hatskevich}

\def\tit{НЕЧЕТКИЕ УСРЕДНЯЮЩИЕ ОПЕРАТОРЫ В~ЗАДАЧЕ АГРЕГИРОВАНИЯ НЕЧЕТКОЙ 
ИНФОРМАЦИИ}

\def\titkol{Нечеткие усредняющие операторы в~задаче агрегирования нечеткой 
информации}

\def\aut{В.\,Л.~Хацкевич$^1$}

\def\autkol{В.\,Л.~Хацкевич}

\titel{\tit}{\aut}{\autkol}{\titkol}

\index{Хацкевич В.\,Л.}
\index{Khatskevich V.\,L.}


%{\renewcommand{\thefootnote}{\fnsymbol{footnote}} \footnotetext[1]
%{Работа выполнена при поддержке РФФИ (проект 20-07-00804).}}


\renewcommand{\thefootnote}{\arabic{footnote}}
\footnotetext[1]{Военно-воздушная академия им.\ проф.\ Н.\,Е.~Жуковского и~Ю.\,А. Гагарина,
\mbox{vlkhats@mail.ru}}


\vspace*{-12pt}
 


\Abst{Рассматривается задача агрегирования нечеткой 
информации посредством построения нечетких усредняющих операторов. Изучены 
взвешенные нечеткие средние систем нечетких чисел и~введен класс нелинейных 
нечетких средних систем нечетких чисел, являющийся модификацией на нечеткие 
чис\-ла общего класса диссипативных чис\-ло\-вых средних. Установлены свойства 
соответствующих усредняющих операторов, которые представляют собой <<нечеткие>> 
аналоги характерных свойств скалярных агрегирующих функций. Это дает обоснование 
использованию введенных нечетких усредняющих операторов в~задаче агрегирования 
нечеткой информации. При этом под результатом агрегирования нечеткой информации, 
заданной набором нечетких чисел, понимается нечеткое чис\-ло, отражающее 
существенные особенности этой совокупности.}

\KW{усредняющие нечеткие операторы; агрегирование нечеткой 
информации}

 \DOI{10.14357/19922264220408} 
  
%\vspace*{-3pt}


\vskip 10pt plus 9pt minus 6pt

\thispagestyle{headings}

\begin{multicols}{2}

\label{st\stat}

\section{Введение}

При разработке и~анализе  информационно-ана\-ли\-ти\-че\-ских сис\-тем возникает задача 
по\-стро\-ения обобщенной оценки состояния сложного объекта с~по\-мощью операции 
агрегирования,  понимаемой как отображение, которое векторной оценке 
объекта $X \hm= (x_1,\ldots, x_n)$ ставит в~соответствие скалярную величину  $A(X) \hm= 
A(x_1,\ldots, x_n)$,  характеризующую этот объект. Компоненты~$x_i$ векторной 
оценки называют частными оценками.

Обозначим через $I$ отрезок расширенной чис\-ло\-вой прямой. Под $n$-мер\-ной 
агрегирующей функцией обычно понимают функцию $A: I^n\hm\rightarrow I$, \mbox{которая} 
обладает следующими свойствами (см., например,~[1--3]).

\begin{description}
\item[Свойство 1.] Идемпотентность: $A(x, \ldots, x) \hm= x$.

\item[Свойство 2.] Монотонность: для любой пары векторных оценок $X \hm= (x_1,\ldots, x_n)$ 
и~$Y \hm= (y_1,\ldots , y_n)$, таких что $x_i\hm\leq y_i$, $i\hm=1,\ldots ,n$, выполняется 
неравенство $A(X)\hm\leq A(Y)$.

\item[Свойство 3.] Непрерывность: функция $A(x_1,\ldots\linebreak \ldots, x_n)$ непрерывна.

\item[Свойство 4.] Граничные условия: 
$$
\inf\limits_{X\in I^n}A(X) = \inf I;\enskip 
\sup\limits_{X\in I^n}A(X)= \sup I\,.
$$
\end{description}


Перечисленные свойства отражают интуитивные требования к~результату 
агрегирования. Идемпотентность  отражает тот факт, что если все частные оценки 
равны между собой, то и~обобщенная оценка должна быть такая же. Монотонность 
означает, что если хотя бы одна частная оценка увеличится (уменьшится), то 
и~обобщенная оценка увеличится (уменьшится). Согласно свойству непрерывности  
малым изменениям частных оценок соответствуют малые изменения обобщенных оценок.

Иногда в~качестве $I$ рассматривают нормированный отрезок $[0, 1]$. Отдельные 
классы агрегирующих функций обладают рядом дополнительных свойств.


Часто в~качестве агрегирующих функций рассматриваются различные средние~[4, 5], 
а~именно: 
\begin{itemize}

\item среднее арифметическое


\noindent
$$
M\left(x_1,\ldots, x_n\right) = \fr{1}{n}\sum\limits_{i=1}^nx_i\,;
$$ 
\item
среднее геометрическое 

\noindent
$$
G\left(x_1,\ldots, x_n\right) = \left(\prod\limits_{i=1}^nx_i\right)^{1/n};
$$ 
\item среднее гармоническое 

\noindent
$$
H\left(x_1,\ldots, x_n\right) = n\left(\sum\limits_{i=1}^n \fr{1}{x_i}\right)^{-1}.
$$  
\end{itemize}
Отметим, что все приведенные средние обладают свойствами~1--4. Кроме того, 
среднее арифметическое $M(x_1,\ldots, x_n)$ обладает  свойствами ад\-ди\-тив\-ности и~од\-но\-род\-ности.


Также в~литературе рассматриваются соответствующие взвешенные средние. 
В~част\-ности,  порядковые операторы взвешенного агрегирования~--- OWA 
(ordered weighted averaging) и~WOWA (weighted OWA)~[6, 7].

С другой стороны, в~последние десятилетия успешно развивается теория нечетких 
множеств, представляющая собой современный аппарат формализации различных видов 
неопределенностей (см., например,~[8, 9]). Основоположником этого направления стал 
Л.~Заде~[10].  Для агрегирования нечеткой информации активно развивается подход, 
связанный с~использованием так называемых треугольных норм и~соответствующих им 
аддитивных генераторов, а~также треугольных конорм (см., например,~[8, гл.~3]).   
Кроме того, в~тео\-рии нечеткого вывода широко распространен метод привлечения 
взвешенного среднего или, в~более общей ситуации, нечеткого интеграла Шоке (см., 
например,~[11--13]). Отметим, что оба этих подхода связаны с~агрегированием 
результирующей функции принадлежности  по заданным агрегируемым функциям 
принадлежности.

В предлагаемой работе рассматривается задача агрегирования нечеткой информации 
посредством построения нечетких усредняющих операторов. При этом под результатом 
агрегирования \mbox{нечеткой} информации, заданной набором нечетких чисел, понимается 
нечеткое чис\-ло, отражающее существенные особенности этой совокупности.

Цель  работы~--- установление свойств нечетких агрегирующих функций, порождаемых 
нечеткими усредняющими операторами, аналогичных свойствам скалярных агрегирующих 
функций. Это служит обоснованием возможности использования рассматриваемых 
нечетких усредняющих операторов для агрегирования нечеткой информации. Такой 
подход представляется новым.


Ниже  $\mathbb{R}$~--- множество действительных чисел. Под нечетким чис\-лом~$\tilde{z}$, 
заданным на универсальном пространстве~$\mathbb{R}$, будем понимать совокупность 
упорядоченных пар $(\mu_{\tilde{z}}(x), x)$, где функция принадлежности 
$\mu_{\tilde{z}}: \mathbb{R}\rightarrow [0,1]$ определяет степень принадлежности $\forall 
\,x\hm\in \mathbb{R}$ множеству~$\tilde{z}$. Носителем нечеткого чис\-ла~$\tilde{z}$ называют 
множество $x\hm\in \mathbb{R}$, для которых $\mu_{\tilde{z}}(x) \hm> 0$, и~обозначают 
$\mathrm{Supp}\,(\tilde{z})$.

Будем дополнительно предполагать выполнение следующих условий (ср.~[9, гл.~2--4]):
\begin{itemize}
\item[(a)] носитель нечеткого чис\-ла~--- замкнутое и~ограниченное (компактное) множество 
действительных чисел;  
\item [(б)] функция принадлежности нечеткого чис\-ла 
$\mu_{\tilde{z}}(x)$ выпукла; 
\item[(в)] функция принадлежности нечеткого чис\-ла 
$\mu_{\tilde{z}}(x)$ нормальна, т.\,е.\ $\sup\nolimits_x\mu_{\tilde{z}}(x) \hm= 1$; 
\item [(г)] функция принадлежности нечеткого чис\-ла $\mu_{\tilde{z}}(x)$ полунепрерывна 
сверху.
\end{itemize}



Совокупность таких нечетких чисел будем обозначать~$J$.

Ниже будет использовано интервальное представление нечетких чисел. Как известно, 
интервал $\alpha$-уров\-ня нечеткого чис\-ла~$\tilde{z}$ с~функцией принадлежности 
$\mu_{\tilde{z}}(x)$ определяется соотношением
$$
Z_{\alpha}  = \left\{x | \mu_{\tilde{z}}(x)\geq \alpha\right\},\ \alpha\in(0, 1],\ Z_0 = \mathrm{Supp}\left(\tilde{z}\right).
$$

Согласно предположениям (a)--(г) на нечеткие чис\-ла все $\alpha$-уров\-ни нечеткого 
чис\-ла~--- замкнутые и~ограниченные интервалы вещественной оси. Обозначим левую  
границу интервала через~$z^{-}(\alpha)$, а~правую~--- через~$z^{+}(\alpha)$. Иногда 
$z^{-}(\alpha)$ и~$z^{+}(\alpha)$ называют соответственно левым и~правым 
индексами нечеткого числа.

Суммой нечетких чисел $\tilde{z}$ и~$\tilde{u}$ с~$\alpha$-ин\-тер\-ва\-ла\-ми $[z^{-}(\alpha)$, 
$z^{+}(\alpha)]$  и~$[u^{-}(\alpha), u^{+}(\alpha)]$ называют нечеткое 
чис\-ло с~$\alpha$-ин\-тер\-ва\-ла\-ми $[z^{-}(\alpha)\hm+u^{-}(\alpha)$, 
$z^{+}(\alpha)\hm+u^{+}(\alpha)]$. Умножение на положительное чис\-ло означает 
умножение индексов на это чис\-ло, а~умножение на отрицательное чис\-ло означает 
умножение индексов на это чис\-ло и~перемену их местами.

Два нечетких чис\-ла равны, если совпадают все их соответствующие $\alpha$-ин\-тер\-валы.

Рассмотрим совокупность~$J^n$ векторов с~нечеткими компонентами (нечетких 
векторов) вида $\tilde{Z} \hm= (\tilde{z}_1,\ldots, \tilde{z}_n)$, где нечеткие чис\-ла 
$\tilde{z}_i$ $(i\hm=1,\ldots,n)$ удовлетворяют условиям~(а)--(г). Для двух нечетких 
векторов~$\tilde{Z}$ и~$\tilde{W}$ их сумму и~умножение на чис\-ла будем понимать 
покоординатно.

Рассмотрим  на множестве~$J$ нечетких чисел мет\-ри\-ку, определяемую для 
$\tilde{z}$, $\tilde{w}\hm\in J$ формулой~[14]:


\noindent
\begin{multline*}
\rho\left(\tilde{z}, \tilde{u}\right) ={}\\
{}= \sup\limits_{\alpha\in[0, 1]}\!\max\left(|z^{-}
(\alpha) \hm- u^{-}(\alpha)|, |z^{+}(\alpha) \hm- u^{+}(\alpha)| \right),
\end{multline*}


\noindent
где $[z^{-}(\alpha), z^{+}(\alpha)]$ и~$[u^{-}(\alpha), u^{+}(\alpha)]$~---
интервалы \mbox{$\alpha$-уров}\-ней чисел~$\tilde{z}$ и~$\tilde{u}$.

На множестве нечетких векторов~$J^n$ для нечетких векторов~$\tilde{Z}$ 
и~$\tilde{W}$ с~компонентами $\tilde{z}_i$ и~$\tilde{w}_i$ $(i\hm=1,\ldots,n)$ 
зададим метрику выражением
$$
\rho_n\left(\tilde{Z}, \tilde{W}\right) = \sum\limits_{i=1}^n\rho\left(\tilde{z}_i \tilde{w}_i\right).
$$


Рассмотрим следующий критерий сравнения нечетких чисел, заданных в~интервальной 
форме~[15, гл.~4, 5]. Будем писать $\tilde{z}\hm\prec\tilde{w}$ для нечетких 
чисел $\tilde{z}$ и~$\tilde{w}$, если одновременно
\begin{equation}
z^{-}(\alpha)\leq w^{-}(\alpha); \ z^{+}(\alpha)\leq 
w^{+}(\alpha)\ (\forall\,\alpha\in (0,1]).
\label{e1-h}
\end{equation}
%
По-существу, это введение частичной упо\-ря\-до\-чен\-ности на множестве нечетких чисел, 
т.\,е.\ введение бинарного отношения, обладающего\linebreak свойствами реф\-лек\-сив\-ности, 
тран\-зи\-тив\-ности и~\mbox{ан\-ти\-сим\-мет\-рич\-ности}. Неравенства нечетких\linebreak\vspace*{-12pt}

\pagebreak

\noindent
 векторов 
$\tilde{Z}\hm\prec\tilde{W}$  будем понимать покоординатно в~смысле определения~(\ref{e1-h}).

%\vspace*{-3pt}

\section{Нечеткие  средние систем нечетких чисел и~усредняющие операторы 
как~агрегаторы нечеткой информации }

%\vspace*{-3pt}

Пусть заданы вещественные чис\-ла
 $\beta_i\hm\in R$ $(i\hm=1,\ldots,n)$ такие, что $\beta_i \hm\geq 0$, 
$\sum\nolimits_{i=1}^n\beta_i\hm = 1$.

Рассмотрим взвешенное нечеткое среднее нечетких чисел   $\tilde{z}_1$,\ldots, 
$\tilde{z}_n$~[16, гл.~7; 17]
\begin{equation}
\tilde{z}_{\mathrm{ср}} = \sum\limits_{i=1}^n\beta_i\tilde{z}_i\,.
\label{e2-h}
\end{equation}


Обозначим через  $z_i^{-}(\alpha)$ и~$z_i^{+}(\alpha)$ левые и~правые индексы 
нечетких чисел~$\tilde{z}_i$, фигурирующих в~формуле~(\ref{e2-h}).

\smallskip

\noindent
\textbf{Лемма 1.} \textit{Левый индекс нечеткого среднего 
$\tilde{z}_{\mathrm{ср}}$, определяемого формулой}~(\ref{e2-h}), 
\textit{равен} 
$$
z^{-}_{\mathrm{ср}}(\alpha)= \sum\limits_{i=1}^n\beta_iz_i^{-}(\alpha),
$$
\textit{а~правый  индекс}~--- 
$$
z^{+}_{\mathrm{ср}}(\alpha)=  \sum\limits_{i=1}^n\beta_iz_i^{+}(\alpha).
$$


\smallskip

Она вытекает  из определения интервального сложения нечетких чисел и~умножения 
на положительное число.

Назовем оператор $A:J^n\hm\longrightarrow J$ нечетким агрегатором, если он обладает 
следующими свойствами:
\begin{enumerate}[(1)]
\item  если $\tilde{Z}\hm=(\tilde{z}, \tilde{z},\ldots, \tilde{z})$, то 
$A(\tilde{Z})\hm=\tilde{z}$ (идемпотентность);
\item
если $\tilde{Z}\prec\tilde{W}$, то $A(\tilde{Z})\prec A(\tilde{W})$ 
(монотонность);
\item
оператор $A:J^n\longrightarrow J$ непрерывен.
\end{enumerate}
Это модификация на случай нечетких чисел характерных свойств скалярного 
оператора агрегатора (см.\ свойства~1--3 из разд.~1).

Рассмотрим усредняющий оператор~$A_{\beta}$, определяемый для заданного 
$\tilde{Z}\hm\in J^n$, $\tilde{Z} \hm= (\tilde{z}_1,\ldots, \tilde{z}_n)$ 
взвешенной суммой~(\ref{e2-h}):
\begin{equation}
A_{\beta}(\tilde{Z}) = \tilde{z}_{\mathrm{ср}} =  \sum\limits_{i=1}^n\beta_i\tilde{z}_i.
\label{e3-h}
\end{equation}

Подчеркнем, что оператор $A_{\beta}(\tilde{Z})$ агрегирует информацию, заданную 
нечетким вектором $\tilde{Z} \hm= (\tilde{z}_1,\ldots,\tilde{z}_n)$ при весовых 
коэффициентах $\beta_i$ $(i\hm=1,\ldots,n)$.

\smallskip


\noindent
\textbf{Теорема~1.}\ \textit{Усредняющий оператор $A_{\beta}$, определяемый 
формулой}~(\ref{e3-h}), \textit{является  нечетким агрегатором}.

\smallskip

\noindent
Д\,о\,к\,а\,з\,а\,т\,е\,л\,ь\,с\,т\,в\,о\,.\ \  Проверим свойство~1~--- идемпотентность. Пусть 
$\tilde{Z}\hm=(\tilde{z}, \tilde{z},\ldots, \tilde{z})$, причем нечеткое чис\-ло 
$\tilde{z}$ имеет индексы~$z^{-}(\alpha)$ и~$z^{+}(\alpha)$. Тогда согласно 
лемме~1 $\tilde{z}_{\mathrm{ср}}$ имеет левый индекс 
$\sum\nolimits_{i=1}^n\beta_iz^{-}(\alpha) \hm= z^{-}(\alpha)$, поскольку 
$\sum\nolimits_{i=1}^n\beta_i\hm  = 1$. И~аналогично правый индекс совпадает 
с~$z^{+}(\alpha)$. Это и~доказывает свойство~1.

Покажем свойство~2~--- монотонность. Пусть $\tilde{Z}, \tilde{W}\in J^n$  и~$\tilde{Z}\hm\prec\tilde{W}$. 
По определению это означает, что для компонент этих 
векторов имеем $\tilde{z}_i\hm\prec\tilde{w}_i$ $(i\hm=1,\ldots,n)$, т.\,е.\ согласно~(\ref{e1-h}) 
для любого $\alpha\hm\in(0, 1]$ и~$i\hm=1,\ldots,n$ для соответствующих индексов 
имеем  $z^{-}_i(\alpha)\hm\leq w^{-}_i(\alpha)$ и~аналогично $z^{+}_i(\alpha)\hm\leq 
w^{+}_i(\alpha)$. Умножая обе части указанных неравенств на~$\beta_i$ и~суммируя 
по $i\hm=1,\ldots,n$, получаем:
\begin{align*}
\sum\limits_{i=1}^n\beta_iz^{-}_i(\alpha)&\leq\sum\limits_{i=1}^n\beta_iw^{-}_i(\alpha)\,;\\ 
\sum\limits_{i=1}^n\beta_iz^{+}_i(\alpha)&\leq\sum\limits_{i=1}^n\beta_iw^{+}_i(\alpha)\,.
\end{align*}
 В этих неравенствах согласно лемме~1 слева стоят $\alpha$-ин\-дек\-сы 
нечеткого чис\-ла $A_{\beta}(\tilde{Z})$, а~справа~---  $A_{\beta}(\tilde{W})$, что 
и~влечет соотношение $A_{\beta}(\tilde{Z})\hm\prec A_{\beta}(\tilde{W})$.

Покажем свойство~3~--- непрерывность. Пусть задано $\varepsilon\hm>0$ и~для 
нечетких векторов $ \tilde{Z}, \tilde{W}\hm\in J^n$ выполнено неравенство 
$\rho_n(\tilde{Z}, \tilde{W})\hm\leq \varepsilon/(2n)$. По определению~$\rho_n$ это 
влечет совокупность неравенств $\rho(\tilde{z}_i, \tilde{w}_i)\hm\leq 
\varepsilon/(2n)$, $i\hm=1,\ldots,n$.

Далее, в~силу~(\ref{e3-h}) и~согласно лемме~1 имеем:
\begin{multline*}
\rho\left(A_{\beta}\left(\tilde{Z}\right), A_{\beta}\left(\tilde{W}\right)\right) = {}\\
{}=\sup\limits_{\alpha\in 
[0,1]}\max\left\{ \left\vert \sum\limits_{i=1}^n\beta_i(z^{-}_i(\alpha)-w^{-}_i(\alpha))\right\vert, \right.\\
\left.
\left\vert \sum\limits_{i=1}^n\beta_i(z^{+}_i(\alpha)-w^{+}_i(\alpha))\right\vert
\right\}.
\end{multline*}
При этом

\noindent
$$
\left\vert \sum\limits_{i=1}^n \beta_i\left(z^{-}_i(\alpha)-w^{-}_i(\alpha)\right)\right\vert \leq 
\sum\limits_{i=1}^n \left\vert \left(z^{-}_i(\alpha)-w^{-}_i(\alpha)\right)\right\vert.
$$
Аналогично для индексов с~плюсом. Так что окончательно имеем:

\noindent
$$
\rho\left(A_{\beta}(\tilde{Z}), A_{\beta}(\tilde{W})\right)\leq 2\sum\limits_{i=1}^n\rho
\left(\tilde{z}_i, \tilde{w}_i\right)\leq\varepsilon\,,
$$
что и~влечет указанную непрерывность.

%\smallskip

\noindent
\textbf{Замечание 1.}\ Граничные условия  для агрегирующих функций  (см.\ условие~4 разд.~1) выполняются для агрегирующей функции $A_{\beta}(\tilde{Z})$
 в~смысле их выполнения для  левых и~правых индексов ре\-зуль\-ти\-ру\-юще\-го нечеткого 
чис\-ла при каждом $\alpha\hm\in[0, 1]$.

Свойство~2 влечет

\smallskip

\noindent
\textbf{Следствие 1.}\ Если $\tilde{Z}$, $\tilde{W}\hm\in J^n$ и~$\tilde{Z}\prec 
\tilde{W}$, то $A_{\beta}(\tilde{Z}\hm + \tilde{V})\hm\prec A_{\beta}(\tilde{W} \hm+ 
\tilde{V})$ для любых $\tilde{V}\hm\in J^n$.

\smallskip

Кроме указанных свойств имеет место

\smallskip

\noindent
\textbf{Теорема~2.}\ \textit{Усредняющий оператор $A_{\beta}: J^n\hm\longrightarrow 
J$, определяемый формулой}~(\ref{e3-h}), \textit{аддитивен и~однороден.}

\smallskip

Действительно, покажем аддитивность~$A_{\beta}$. Пусть заданы нечеткие векторы 
$\tilde{Z}, \tilde{W}\in J^n$.  В~соответствии с~леммой~1 левый и~правый индексы 
$A_{\beta}(\tilde{Z})$ совпадают с~$\sum\nolimits_{i=1}^n\beta_iz^{-}_i(\alpha)$ 
и~$\sum\nolimits_{i=1}^n\beta_iz^{+}_i(\alpha)$ соответственно. Аналогично для 
$A_{\beta}(\tilde{W})$ левый и~правый индексы имеют вид 
$\sum\nolimits_{i=1}^n\beta_iw^{-}_i(\alpha)$ 
и~$\sum\nolimits_{i=1}^n\beta_iw^{+}_i(\alpha)$. Тогда левый индекс суммы 
$A_{\beta}(\tilde{Z})+A_{\beta}(\tilde{W})$ имеет вид 
$\sum\nolimits_{i=1}^n\beta_i(\tilde{z}^{-}(\alpha) + w^{-}(\alpha))$, а~правый 
индекс~--- $\sum\nolimits_{i=1}^n\beta_i(\tilde{z}^{+}_i(\alpha)\hm + 
w^{+}_i(\alpha))$, что совпадает с~левым и~соответственно правым индексом 
нечеткого чис\-ла $A_{\beta}(\tilde{Z}\hm+\tilde{W})$ и~влечет аддитивность.

Однородность есть следствие формулы~(\ref{e3-h}) и~определения интервального умножения 
нечетких чисел на положительное и~отрицательное число.


\section{Нелинейные  нечеткие усредняющие операторы как~агрегаторы нечеткой 
информации}

 Определим понятие функции от нечеткого чис\-ла, используя интервальный подход.  
Пусть задана непрерывная монотонно возрастающая (монотонно убывающая) 
вещественная функция $\phi: R\hm\rightarrow R$. Приведем в~удобном  виде 
формулировку результата из~[18].

\smallskip

\textbf{Лемма~2.} \textit{Если  $\tilde{z}$~--- нечеткое чис\-ло  с~левым и~правым 
индексами $z^{-}(\alpha)$ и~$z^{+}(\alpha)$  и~$\phi: R\hm\rightarrow R$~--- 
непрерывная монотонно возрастающая функция, то $\phi(z^{-}(\alpha))$ 
и~$\phi(z^{+}(\alpha))$ суть соответственно  левый и~правый индексы  нечеткого 
чис\-ла~$\phi(\tilde{z})$. Если~$\phi(x)$~--- непрерывная монотонно убывающая 
функция, то~$\phi(z^{+}(\alpha))$ и~$\phi(z^{-}(\alpha))$~--- левый и~правый 
индексы~$\phi(\tilde{z})$ соответственно.}

\smallskip


Пусть заданы нечеткие чис\-ла $\tilde{z}_1,\ldots, \tilde{z}_n$, а~также 
действительные чис\-ла $\beta_i\hm\in \mathbb{R}$ ($i \hm= 1,\ldots, n$), причем $ \beta_i \hm\geq 
0$, $\sum\nolimits_{i=1}^n\beta_i\hm=1$. Рассмотрим нелинейное нечеткое среднее 
общего вида  для заданной непрерыв-\linebreak\vspace*{-12pt}

\columnbreak

\noindent
ной строго монотонно возрастающей (убывающей) 
функции $\phi:R\hm\rightarrow R$:
\begin{equation}
\tilde{z}_{\phi} = \phi^{-1}\left(\sum\limits_{i=1}^n\beta_i\phi\left(\tilde{z}_i\right)\right).
\label{e4-h}
\end{equation}
 Определение~(\ref{e4-h})~--- аналог вещественного нелинейного ассоциативного среднего~[19, гл.~I]. Функцию~$\phi$ в~этом случае называют определяющей.

В случае определяющей функции $\phi_p(x) \hm= x^p$ $(p\hm>1)$ (или $0\hm<p\hm<1$) в~(\ref{e4-h}) 
получаем аналог взвешенной средней степенной. Если $\phi_G(x) \hm= 
\log_a x$ ($a\hm>1$)~--- это аналог взвешенной средней гео\-мет\-ри\-че\-ской, если 
$\phi_H(x) \hm= {1}/{x}$~--- аналог взвешенной средней гармонической.

\smallskip

\noindent
\textbf{Лемма 3.} \textit{Нечеткое чис\-ло, определяемое формулой}~(\ref{e4-h}), \textit{имеет левый 
индекс $\phi^{-1}(\sum\nolimits_{i=1}^n\beta_i\phi(z^{-}_i(\alpha)))$ и~правый 
индекс  $\phi^{-1}(\sum\nolimits_{i=1}^n\beta_i\phi(z^{+}_i(\alpha)))$, где $z^{-}_i(\alpha)$  
и~$z^{+}_i(\alpha)$~--- левый и~правый индексы нечеткого чис\-ла~$\tilde{z}_i$ соответственно}.

\smallskip

Действительно, проведем рассуждения для левых индексов в~предположении, что 
функция~$\phi$ непрерывна и~монотонно возрастает.
Заметим, что нечеткое чис\-ло $\sum\nolimits_{i=1}^n\beta_i\phi(\tilde{z}_i)$ в~силу 
монотонного возрастания функции~$\phi$ на основании леммы~2 и~по правилу 
интервального сложения нечетких чисел и~умножения их на положительные чис\-ла 
имеет левый индекс $\sum\nolimits_{i=1}^n\beta_i\phi(z_i^{-}(\alpha))$. Так как~$\phi^{-1}$ 
также монотонно возрастающая функция (вмес\-те с~$\phi$), то нечеткое 
чис\-ло $\phi^{-1}(\sum\nolimits_{i=1}^n\beta_i\phi(\tilde{z}_i))\hm = \bar{z}_{\phi}$ 
имеет левый индекс $\phi^{-1}(\sum\nolimits_{i=1}^n\beta_i\phi(z^{-}_i(\alpha)))$.

Аналогично для правых индексов.

Случай монотонного убывания непрерывной функции~$\phi$ в~силу леммы~2 приводит к~такому же результату.

Определим теперь для фиксированного набора чисел $\beta_i\hm\geq 
0$ $(i\hm=1,\ldots,n)$ таких, что $\sum\nolimits_{i=1}^n\beta_i \hm= 1$, и~заданной 
непрерывной строго монотонной функции $\phi:R\hm\longrightarrow R$ нечеткий 
нелинейный усредняющий оператор $F_{\beta,\phi}:J^n\hm\longrightarrow J$ 
равенством
\begin{equation}
F_{\beta,\phi}\left(\tilde{Z}\right) = \phi^{-1}\left(\sum\limits_{i=1}^n\beta_i\phi\left(\tilde{z}_i\right)\right).
\label{e5-h}
\end{equation}

\noindent
\textbf{Теорема 3.} \textit{Для нелинейного усредняющего оператора 
$F_{\beta,\phi}$ выполнены условия}~1--3, \textit{т.\,е.\ он является  нечетким 
агрегатором}.

\smallskip

\noindent
Д\,о\,к\,а\,з\,а\,т\,е\,л\,ь\,с\,т\,в\,о\,.\ \  Пусть, например, функция~$\phi$ непрерывна и~строго 
монотонно возрастает.  Проверим свойство~1~--- идемпотентность. Пусть~$\tilde{Z}$ 
имеет одинаковые компоненты $\tilde{z}_i \hm= \tilde{z}$. Тогда согласно~(\ref{e5-h})
$$
F_{\beta,\phi}(\tilde{Z}) =  \phi^{-1}\left(\sum\limits_{i=1}^n\beta_i\phi\left(\tilde{z}\right)\right) =  \phi^{-1}\left(\phi(\tilde{z})\right) = 
\tilde{z}\,.
$$
Здесь последнее равенство вытекает из леммы~2.

Покажем свойство~2~--- монотонность. Пусть $\tilde{Z}\hm\prec\tilde{W}$. Обозначим 
$\alpha$-ин\-тер\-ва\-лы нечетких чисел~$\tilde{z}_i$ и~$\tilde{w}_i$ (компонент 
векторов~$\tilde{Z}, \tilde{W}$) как $[z^{-}_i(\alpha), z^{+}_i(\alpha)]$ 
и~$[w^{-}_i(\alpha), w^{+}_i(\alpha)]$ соответственно. Так как, по предположению, 
$\phi$~--- строго монотонно  возрастающая функция, то на основании леммы~2 и~по 
правилу сравнения~(\ref{e1-h}) нечетких чисел $\phi(z^{-}_i(\alpha))\hm\leq\phi(w^{-}_i(\alpha))$ 
и,~следовательно, 

\noindent
$$
\sum\limits_{i=1}^n\beta_i\phi\left(\tilde{z}_i^{-}\right)\leq\sum\limits_{i=1}^n\beta_i\phi\left(\tilde{w}_i^{-}\right).
$$ 
Аналогично для правых  индексов.

Так как $\phi^{-1}$ возрастает вместе с~$\phi$, то в~соответствии с~леммой~3 
$F_{\beta,\phi}(\tilde{Z})\hm\prec F_{\beta,\phi}(\tilde{W})$.

Свойство~3~--- непрерывность~--- проверяется рассуждениями, близкими к~доказательству теоремы~1.


\smallskip

В заключение отметим, что взвешенное нечеткое среднее~(\ref{e2-h}) рассматривалось 
в~нечеткой статистике (см., например,~[16, гл.~7; 17]). Однако свойства нечетких 
усредняющих операторов~(\ref{e3-h}), приведенные в~настоящей работе, ранее не отмечались.
%
При этом свойства идемпотентности, аддитивности  и~однородности в~случае 
усредняющего оператора~(\ref{e3-h}) вполне предсказуемы. Непрерывность можно 
рассматривать в~различных метриках на множестве нечетких чисел. Здесь приводится 
один из вариантов.  Что касается свойства монотонности для усредняющих 
операторов~(\ref{e3-h}), то существенным моментом оказался подбор определения 
ранжирования нечетких чисел, при котором это свойство выполняется и~которое 
обеспечивает смысловую нагрузку операции агрегирования.

Нелинейное нечеткое среднее~(\ref{e4-h}) и~свойства соответствующего нечеткого 
нелинейного усред\-ня\-юще\-го оператора~(\ref{e5-h}) ранее, по-ви\-ди\-мо\-му, не рас\-смат\-ри\-ва\-лись.

Совокупность установленных в~настоящей работе свойств для операторов нечеткого 
усреднения~(\ref{e3-h}) и~(\ref{e5-h}) обеспечивает естественность и~адекватность их применения 
для агрегирования нечеткой информации.

\vspace*{-6pt}

{\small\frenchspacing
 {%\baselineskip=10.8pt
 %\addcontentsline{toc}{section}{References}
 \begin{thebibliography}{99}
\bibitem{2}  %1
\Au{Mesiar R., Kolesarova~A., Calvo~T., Komornakova~M.\,A.} 
Review of aggregation functions~// Fuzzy sets and their extensions: 
Representation, aggregation and models~/ Eds. H.~Bustince, F.~Herrera, J.~Montero.~--- 
 Studies in fuzziness and soft computing ser.~--- Springer, 2008. Vol.~220. P.~121--144.

\bibitem{3} 
\Au{Леденева Т.\,М., Подвальный~С.\,Л.} Агрегирование информации в~оценочных системах~// Вестник ВГУ. Сер. Сис\-тем\-ный анализ и~информационные 
технологии, 2016. Vol.~4. P.~155--164.

\bibitem{4} 
\Au{Lopez de Hierro~A.\,F.\,R., Roldin~C., Bustince~H., 
Fernandez~J., Rodriguez~I., Fardoun~H.,  Lafuente~J.} Affine construction 
methodology of aggregation functions~// Fuzzy Set. Syst., 2021. Vol.~414. 
P.~146--164.

\bibitem{5} 
\Au{Dubois D., Prade~H.} On the use of aggregation operations in 
information fusion processes~// Fuzzy Set. Syst., 2014. Vol.~142. P.~143--161.

\bibitem{6} 
\Au{Beliakov G., Bustince~H., Calvo~T.} A~practical guide to 
averaging functions.~--- Cham: Springer, 2016. 352~p.

\bibitem{7} 
\Au{Yager R.} Quantifier guided aggregation using OWA operators~// Int. J.~Intell. Syst., 1996. Vol.~11. Iss.~1. P.~49--73.

\bibitem{8} 
\Au{Torra V.} Andness directedness for operators of the OWA and 
WOWA families~// Fuzzy Set. Syst., 2021. Vol.~144. P.~28--37.



\bibitem{11}  %8
\Au{Дюбуа Д., Прад~А.} Теория возможностей. Приложения к~пред\-став\-ле\-нию знаний в~информатике~/ Пер. с~англ.~--- М.: Радио и~связь, 1990. 288~с.
(\Au{Dubois~D., Prade~H.} {Possibility theory}.~--- New York, NY, USA: Springer, 1988. 280~p.)

\bibitem{10}  %9
\Au{Пегат А.} Нечеткое моделирование и~управ\-ле\-ние~/ Пер. с~англ.~--- М.: Бином,  2015. 786~с.
(\Au{Piegat~A.} {Fuzzy modeling and control}.~--- Springer Science \& Business Media, 2001. 728~p.)

\bibitem{12} 
\Au{Zadeh L.\,A.} Fuzzy sets~// Inform. Control, 1965. Vol.~8. P.~338--353.

\bibitem{13}
\Au{Tahani H., Keller~J.\,M.} Information fusion in computer 
vision using the fuzzy integral~// IEEE T. Syst. Man  Cyb.,  1990. Vol.~20. Iss.~3. P.~733--741.

\bibitem{14} 
\Au{Kwak K., Pedrycz~W.} Face recognition: A~study in 
information fusion using fuzzy integral~// Pattern Recogn. Lett., 2005. Vol.~26.  P.~719--733.

\bibitem{9} 
\Au{Bustince H., Mesiar~R., Fernandez~J., Galar~M., Paternain~D., Altalhi~A., Dimuro G.\,P., Bedregal~B., Takaa~Z.} d-Choquet integrals: 
Choquet integrals based on dissimilarities~// Fuzzy Set. Syst., 2021. Vol.~414. P.~1--27.

\bibitem{15} 
\Au{ Kaleva O., Seikkala~S.} On fuzzy metric spaces~// Fuzzy 
Set. Syst., 1984.  Vol.~12. P.~215--229.

\bibitem{16} 
\Au{Смоляк С.\,А.} Оценки эффективности инвестиционных проектов 
в~условиях риска и~не\-оп\-ре\-де\-лен\-ности.~--- М.: Наука, 2002. 182~с.

\bibitem{18} 
\Au{Nguyen H.\,T., Wu~B.} Fundamentals of statistics with fuzzy  data.~--- Berlin: Springer, 2006. 204~p.

\bibitem{19} 
\Au{De la Rosa de Saa~S., Gil~M.\,A., Gonsalez-Rodrigues~G., 
Lopez~M.\,T., Lubiano~M.\,A.} Fuzzy rating scale-based questionnaires and their 
statistical analysis~// IEEE T. Fuzzy Syst., 2015. Vol.~23. P.~111--126.

\bibitem{20} 
\Au{Nguyen H.\,T.} A Note on the  extension  principle for fuzzy 
sets~// J.~Math. Anal. Appl., 1978. Vol.~79. P.~369--380.

\bibitem{21}  
\Au{Джини К.} Средние величины~/ Пер. с~итал.~--- М.: Статистика, 1970. 447~с.
(\Au{Gini~C.} {Le medie}.~--- Torino: UTET, 1958. 512~p.)

\end{thebibliography}

 }
 }

\end{multicols}

\vspace*{-9pt}

\hfill{\small\textit{Поступила в~редакцию 11.08.21}}

%\vspace*{8pt}

%\pagebreak

\newpage

\vspace*{-28pt}

%\hrule

%\vspace*{2pt}

%\hrule

%\vspace*{-2pt}

\def\tit{FUZZY AVERAGING OPERATORS IN~THE~PROBLEM OF~AGGREGATING FUZZY INFORMATION}


\def\titkol{Fuzzy averaging operators in~the~problem of~aggregating fuzzy information}


\def\aut{V.\,L.~Khatskevich}

\def\autkol{V.\,L.~Khatskevich}

\titel{\tit}{\aut}{\autkol}{\titkol}

\vspace*{-8pt}


\noindent
N.\,E.~Zhukovsky and Y.\,A.~Gagarin Air Force Academy, 54a~Old Bolsheviks Str., 394064 Voronezh, Russian Federation

\def\leftfootline{\small{\textbf{\thepage}
\hfill INFORMATIKA I EE PRIMENENIYA~--- INFORMATICS AND
APPLICATIONS\ \ \ 2022\ \ \ volume~16\ \ \ issue\ 4}
}%
 \def\rightfootline{\small{INFORMATIKA I EE PRIMENENIYA~---
INFORMATICS AND APPLICATIONS\ \ \ 2022\ \ \ volume~16\ \ \ issue\ 4
\hfill \textbf{\thepage}}}

\vspace*{3pt} 




\Abste{The problem of aggregating fuzzy information by constructing fuzzy averaging operators is considered. 
Weighted fuzzy averages of systems of fuzzy numbers are studied and a~class of nonlinear fuzzy averages of systems of 
fuzzy numbers is introduced which is a~modification to fuzzy numbers of the general class of dissipative numerical averages. 
The properties of the corresponding averaging operators which are ``fuzzy'' 
analogues of the characteristic properties of scalar aggregating functions, are established. 
This provides a~justification for the use of the introduced fuzzy averaging operators in the problem of aggregation of fuzzy information. 
At the same time, the result of aggregation of
 fuzzy information given by a~set of fuzzy numbers is understood as a~fuzzy number that reflects the essential features of this set.}

\KWE{averaging fuzzy operators; aggregation of fuzzy information}

 \DOI{10.14357/19922264220408} 

%\vspace*{-16pt}

% \Ack
%   \noindent
 

%\vspace*{4pt}

  \begin{multicols}{2}

\renewcommand{\bibname}{\protect\rmfamily References}
%\renewcommand{\bibname}{\large\protect\rm References}

{\small\frenchspacing
 {%\baselineskip=10.8pt
 \addcontentsline{toc}{section}{References}
 \begin{thebibliography}{99}
\bibitem{1-h}
\Aue{Mesiar, R., A.~Kolesarova, T.~Calvo, and M.~Komornakova.}
 2008. A~review of aggregation functions. 
 \textit{Fuzzy sets and their extensions: Representation, aggregation and models}. Eds. H.~Bustince, F.~Herrera, and J.~Montero. 
 Studies in fuzziness and soft computing ser. Springer. 220:121--144.
\bibitem{2-h}
\Aue{Ledeneva, T.\,M., and S.\,L.~Podval'nyy.}
 2016. Agre\-gi\-ro\-va\-nie informatsii v~otsenochnykh sistemakh [The aggregation of information in the evaluated system]. 
 \textit{Proceedings of Voronezh State University. Ser. Systems Analysis and Information Technologies} 4:155--164.
\bibitem{3-h}
\Aue{Lopez de Hierro, A.\,F.\,R., C.~Roldin, H.~Bustince, J.~Fernandez, I.~Rodriguez, H.~Fardoun, and J.~Lafuente.}
 2021. Affine construction methodology of aggregation functions. \textit{Fuzzy Set. Syst.} 414:146--164.
\bibitem{4-h}
\Aue{Dubois, D., and H.~Prade.} 2004. On the use of aggregation operations in information fusion processes. 
\textit{Fuzzy Set. Syst.} 142:143--161.
\bibitem{5-h}
\Aue{Beliakov, G., H.~Bustince, and T.~Calvo.} 2016. \textit{A~practical guide to averaging functions}. Cham: Springer. 352~p. 
\bibitem{6-h}
\Aue{Yager, R.} 1996. Quantifier guided aggregation using OWA operators. \textit{Int. J.~Intell. Syst.} 11(1):49--73.
\bibitem{7-h}
\Aue{Torra, V.} 2021. Andness directedness for operators of the OWA and WOWA families. \textit{Fuzzy Set. Syst.} 144:28--37.

\bibitem{9-h} %8
\Aue{Dubois, D., and H.~Prade.} 1988. \textit{Possibility theory}. New York, NY: Springer. 280~p.
\bibitem{8-h} %9
\Aue{Piegat, A.} 2001. \textit{Fuzzy modeling and control}. Springer Science \& Business Media. 728~p.

\bibitem{10-h}
\Aue{Zadeh, L.\,A.} 1965. Fuzzy sets. \textit{Inform. Control} 8:338--353.
\bibitem{11-h}
\Aue{Tahani, H., and J.\,M.~Keller.} 1990. Information fusion in computer vision using the fuzzy integral.
\textit{IEEE T. Syst. Man Cyb.} 20(3):733--741.
\bibitem{12-h}
\Aue{Kwak, K., and W.~Pedrycz.} 2005. Face recognition: A~study in information fusion using fuzzy integral. \textit{Pattern Recogn. Lett.} 26:719--733.
\bibitem{13-h}
\Aue{Bustince, H., R.~Mesiar, J.~Fernandez, M.~Galar, D.~Paternain, A.~Altalhi, G.\,P.~Dimuro, B.~Bedregal, and Z.~Takaa.}
 2021. d-Choquet integrals: Choquet integrals based on dissimilarities. \textit{Fuzzy Set. Syst.} 414:1--27.
\bibitem{14-h}
\Aue{Kaleva, O., and S.~Seikkala.} 1984. On fuzzy metric spaces. \textit{Fuzzy Set.  Syst.} 12:215--229.
\bibitem{15-h}
\Aue{Smolyak, S.\,A.} 2002. \textit{Otsenki effektivnosti investitsionnykh proektov v~usloviyakh riska i~neopredelennosti} 
[Evaluation of the effectiveness of investment projects under conditions of risk and uncertainty]. Moscow: Nauka. 182~p. 
\bibitem{16-h}
\Aue{Nguyen, H.\,T., and B.~Wu.} 2006. \textit{Fundamentals of statistics with fuzzy data}. Berlin: Springer. 204~p.
\bibitem{17-h}
\Aue{De la Rosa de Saa, S., M.\,A.~Gil, G.~Gonsalez-Rodrigues, M.\,T.~Lopez, and M.\,A.~Lubiano.}
 2015. Fuzzy rating scale-based questionnaires and their statistical analysis. \textit{IEEE T. Fuzzy Syst.} 23:111--126.
\bibitem{18-h}
\Aue{Nguyen, H.\,T.} 1978. A~note on the extension principle for fuzzy sets. 
\textit{J.~Math. Anal. Appl.} 79:369--380.
\bibitem{19-h}
\Aue{Gini, C.} 1958. \textit{Le medie}. Torino: UTET. 512~p.
\end{thebibliography}

 }
 }

\end{multicols}

\vspace*{-6pt}

\hfill{\small\textit{Received August 11, 2021}} 

\vspace*{-12pt}

\Contrl

\noindent
\textbf{Khatskevich Vladimir L.} (b.\ 1951)~--- 
Doctor of Science in technology, professor, Department of Mathematics, N.\,E.~Zhukovsky and Y.\,A.~Gagarin Air Force Academy, 
54a~Old Bolsheviks Str., 394064 Voronezh, Russian Federation; \mbox{vlkhats@mail.ru}


\label{end\stat}

\renewcommand{\bibname}{\protect\rm Литература}    