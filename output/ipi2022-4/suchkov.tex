\def\stat{suchkov}

\def\tit{ЕДИНАЯ МОДЕЛЬ ГОСУДАРСТВЕННЫХ ДАННЫХ: СЦЕНАРИИ~РАЗВИТИЯ}

\def\titkol{Единая модель государственных данных: сценарии развития}

\def\aut{А.\,П.~Сучков$^1$}

\def\autkol{А.\,П.~Сучков}

\titel{\tit}{\aut}{\autkol}{\titkol}

\index{Сучков А.\,П.}
\index{Suchkov A.\,P.}

%{\renewcommand{\thefootnote}{\fnsymbol{footnote}} \footnotetext[1]
%{Работа выполнена при поддержке РФФИ (проект 20-07-00804).}}

\renewcommand{\thefootnote}{\arabic{footnote}}
\footnotetext[1]{Федеральный исследовательский центр <<Информатика 
и~управление>> Российской академии наук, \mbox{ASuchkov@ipiran.ru}}

\vspace*{-6pt}

\Abst{Обсуждается проблема информационного взаимодействия 
разнородных информационных сис\-тем, которая должна решаться путем 
создания, внедрения и~поддержания единых моделей данных (ЕМД) в~рамках 
основных разделов предметных областей и~в~перспективе в~рамках всей 
предметной об\-ласти информационного взаимодействия в~национальном 
масштабе. На основе онтологического подхода исследуется 
проблема поиска эффективных и~оптимальных путей формирования моделей 
государственных данных. Также рас\-смат\-ри\-ва\-ют\-ся сценарии интеграции 
ведомственных сис\-тем.}

\KW{информационное взаимодействие; единая модель данных; онтология; 
сценарии интеграции}

  \DOI{10.14357/19922264220415} 
  
\vspace*{-6pt}


\vskip 10pt plus 9pt minus 6pt

\thispagestyle{headings}

\begin{multicols}{2}

\label{st\stat}

\section{Введение} %1

Мировой и~отечественный опыт~--- как положительный, так 
и~отрицательный~--- говорит о~том, что проб\-ле\-ма информационного 
взаимодействия должна решаться путем создания, внед\-ре\-ния и~поддержания 
ЕМД в~рамках основных разделов~предметных областей 
и~в~перспективе в~рамках всей предметной области информационного 
взаимодействия в~национальном масштабе~\cite{1-suc}. Когда речь идет 
о~национальном масштабе, основную роль в~этом процессе должно играть 
государство, так как оно заинтересовано в~повышении эф\-фек\-тив\-ности 
создания, сбора и~использования \textit{государственных данных} как для 
предоставления государственных и~муниципальных \textit{услуг} 
и~осуществления государственных и~муниципальных \textit{функций}, так 
и~для обеспечения потребности физических и~юридических лиц в~доступе 
к~информации~\cite{2-suc}. 

Так, с~2006~г.\ развивается National Information 
Exchange Model (NIEM) в~США~\cite{3-suc}, в~2019~г.\ положено начало 
созданию Национальной сис\-те\-мы управ\-ле\-ния данными (НСУД)  
РФ~\cite{2-suc}, в~2020~г.\ Еврокомиссия представила ``A~European strategy 
for data''~\cite{4-suc}.

Текущая версия NIEM~5.0 включает 15~доменов, более 11\,000 
взаимосвязанных информационных объектов (сущностей), для описания 
которых используются свыше 22\,000~атрибутов, и~объединяет более  
70~ин\-фор\-ма\-ци\-он\-но-лин\-г\-вис\-ти\-че\-ских сис\-тем (словарей 
терминов, классификаторов, рубрикаторов и~т.\,п.)\ различных ведомств. 
Однако модель пока не охватывает все разделы национальных данных, 
далеко не все ведомства ее используют.

По проекту НСУД осуществлена разработка и~введение в~опытную 
эксплуатацию информа\-ци\-он\-но-тех\-но\-ло\-ги\-че\-ских компонентов 
НСУД, вклю-\linebreak чая систему защиты информации, необходимых для проведения 
эксперимента по интеграции крупнейших информационных сис\-тем на уровне 
метаданных (Росреестр, Федеральное \mbox{казначейство}, Федеральная налоговая служ\-ба, Пенсионный фонд 
России и~др.). Эксперимент выявил существенные пробелы 
в~нор\-ма\-тив\-но-пра\-во\-вой базе.

Европейская стратегия представляет собой часть инициативы по цифровой 
трансформации\linebreak Европы и~находится на начальной стадии, определены 
10~основных доменов, выделено финансирование (6~млрд евро на три 
года) и~принят <<\mbox{Скользящий} план стандартизации 
ин\-фор\-ма\-ци\-он\-но-ком\-му\-ни\-ка\-ци\-он\-ных технологий 
2020>>~\cite{5-suc}.

Помимо этих проектов можно рассмотреть достаточное чис\-ло 
промежуточных паллиативных решений по создании сис\-тем 
межведомственного взаимодействия в~России~--- это Сис\-те\-ма 
межведомственного электронного взаимодействия (\mbox{СМЭВ})~\cite{6-suc}, 
Сис\-те\-ма межведомственного электронного документооборота федеральных 
органов исполнительной влас\-ти (МЭДО)~\cite{7-suc}, которые входят 
в~инфраструктуру электронного правительства <<Безопасный 
город>>~\cite{8-suc}, и,~конечно, Сис\-те\-ма распределенных ситуационных 
цент\-ров, \mbox{вза\-и\-мо\-дей\-ст\-ву\-ющих} по единому регламенту [9].
Как очень вер\-но утверж\-да\-ет\-ся в~\cite{2-suc}: <<Одна из основных задач 
НСУД~--- установление правил создания \textit{модели государственных 
данных}, основанной на принципах по\-сто\-ян\-но\-го развития, постепенного 
наполнения, кон\-сис\-тент\-ности и~не\-про\-ти\-во\-ре\-чи\-вости, включая разработку 
описаний, взаимосвязей сущностей и~форматов, и,~по сути, является 
онтологией$\ldots$>> Такая постановка проб\-ле\-мы у~нас впервые 
появляется в~официальном нор\-ма\-тив\-но-пра\-во\-вом документе. На основе такого 
онтологического подхода в~\mbox{статье} исследуется проб\-ле\-ма поиска 
эффективных и~оптимальных путей формирования моделей 
государственных данных, тем более что такое исследование все еще 
актуально в~силу того, что отечественные разработки находятся в~начальной 
стадии. Так\-же далее рассматриваются сценарии интеграции ведомственных 
сис\-тем.

\vspace*{-6pt}

\section{Единая модель государственных данных} %2

\vspace*{-4pt}

Информационную модель данных некоторой предметной об\-ласти можно 
трактовать как \textit{онтологию}. Онтология описывается со\-во\-куп\-ностью 
базовых объектов, классов (концептов), атрибутов, отношений и~правил.

Рассматривая обобщенную онтологию всего спект\-ра данных, потенциально 
охва\-ты\-ва\-ющих все виды национального межведомственного 
информационного взаимодействия (МВИВ), можно определить так 
на\-зы\-ва\-емую \textit{верхнюю} онтологию, которая охватывает все доменные 
онтологии различных пред\-мет\-ных областей народного хозяйства. 

Из необходимости интеграции разнородных баз данных возникла задача 
\textit{отображения} онтологий, разработанных независимо друг от друга 
и~име\-ющих, таким образом, свое собственное пред\-став\-ле\-ние базовых 
объектов. Отображение двух онтологий означает, что для каж\-до\-го понятия, 
отношения или объекта одной онтологии подыскиваются со\-от\-вет\-ст\-ву\-ющие 
элементы в~другой онтологии~\cite{10-suc}. Эта проб\-ле\-ма стала важнейшей 
задачей в~об\-ласти искусственного интеллекта.

Основные свойства единой модели государственных данных (ЕМГД): 
\begin{itemize}
\item ЕМГД должна формироваться в~виде онтологии, характеризующейся 
со\-во\-куп\-ностью базовых объектов (сущностей), классов, атрибутов 
и~отношений;
\item ЕМГД должна быть функционально ориентированной на реализацию 
законодательно закрепленных государственных функций и~услуг, а~так\-же 
обеспечивать информационную поддержку процессов управ\-ле\-ния на всех 
уровнях;
\item ЕМГД должна быть независимой от поставщиков операционных 
сис\-тем, носителей информации и~приложений, что может быть достигнуто 
путем применения языков описания XML или JSON;
\item ЕМГД должна быть оснащена методическими, технологическими 
и~организационными средствами поддержания ее в~актуальном со\-сто\-янии.
\end{itemize}

Единая модель данных должна иметь доменную структуру, отра\-жа\-ющую все 
разделы пред\-мет\-ной об\-ласти ЕМГД.

Организационно, по мнению автора, центральный орган ЕМГД должен быть 
государственным и~вневедомственным. Для организации коллективной 
работы, поддержания методологической базы и~обучения экспертов, 
разработчиков и~потребителей необходимо создать информационный портал 
ЕМГД. С~технологической точ\-ки зрения необходимо обеспечить 
оптимальные по соотношению <<це\-на--ка\-чест\-во>> сценарии создания 
и~ведения ЕМГД, к~обоснованию которых и~переходим.

\vspace*{-6pt}

\section{Математическая модель процесса формирования единой модели государственных данных} %3

\vspace*{-4pt}

Отображение онтологий в~процессе интеграции различных информационных 
систем может проходить по разным сценариям. Однако у~всех сценариев 
есть одна общая черта: по мере по\-стро\-ения отображений онтологий 
происходит формирование \textit{ядра интеграционной модели данных}, 
реализующей интегральную онтологию данных. Способы развития этого 
ядра в~процессе межведомственной интеграции и~определяют ее основные 
сценарии. Для того чтобы оценить эффективность процесса формирования 
ЕМГД, необходимо оценить ориентировочную сто\-и\-мость интеграции автоматизированной информационной сис\-те\-мы (АИС) 
при различных сценариях в~рамках формирования ЕМГД.

Означим~$\Omega_{\mathrm{в}}$~--- верхнюю, т.\,е.\ гипотетическую  
все\-охва\-ты\-ва\-ющую онтологию ЕМГД, $\Omega_{\mathrm{и}}$~--- 
текущую онтологию ядра интеграционной модели данных ЕМГД,  
$\Omega_{\mathrm{с}}$~--- онтологию ин\-тег\-ри\-ру\-емой АИС. Сценарии 
развертывания во времени процесса формирования ЕМГД в~виде онтологии 
подразделяются в~за\-ви\-си\-мости от способов пополнения интегрального ядра 
и~по\-лу\-ча\-емо\-го соотношения объемов автоматического 
и~автоматизированного взаимодействия меж\-ду АИС (рис.~1).

\begin{figure*} %fig1
 \vspace*{1pt}
 \begin{center}  
     \mbox{%
\epsfxsize=113.614mm
\epsfbox{suh-1.eps}
}
\end{center}
\vspace*{-3pt}
\Caption{Сценарии формирования ЕМД:
$\Omega_{\mathrm{и}}$~--- интегральная онтология МВИВ;
$\Omega_{\mathrm{с}}$~--- онтология ин\-тег\-ри\-ру\-емой АИС;
$\Omega_{\mathrm{с}}^*$~--- онтология первоначального пула 
ин\-тег\-ри\-ру\-емой АИС;
$\Omega_{\mathrm{с}}^i$~--- онтология $i$-го домена ин\-тег\-ри\-ру\-емой АИС;
\textit{1}~--- пополнение интегральной онтологии МВИВ;
\textit{2}~--- автоматическое взаимодействие АИС;
\textit{3}~--- взаимодействие АИС с~по\-мощью адап\-те\-ров
}
\vspace*{-6pt}
\end{figure*}

Выделим четыре сценария формирования \mbox{ЕМГД}: первые два связаны со 
способами формирования первоначального и~окончательного 
интеграционного ядра; третий и~четвертый~--- с~доменным подходом.
\begin{description}
\item[Сценарий~I.] Осуществляется начальное формирование небольшой 
онтологии интеграционного ядра \mbox{ЕМГД}, содержащей час\-то встре\-ча\-ющи\-еся 
в~$\Omega_{\mathrm{с}}$ базовые сущности. Интеграция АИС 
осуществляется с~учетом существования этого яд\-ра, но без его изменения. 
Автоматическое взаимодействие АИС осуществляется только в~рамках 
онтологии интеграционного ядра относительно небольшого размера. 
Остальные взаимодействия происходят с~по\-мощью адап\-те\-ров. Пример такой 
интеграции~--- \mbox{СМЭВ}, портал госуслуг.
\item[Сценарий~II.] Осуществляется начальное формирование онтологии 
интеграционного ядра \mbox{ЕМГД} на основе представительной выборки 
онтологий из всего со\-ста\-ва АИС (первоначального пула), представляющей 
собой со\-во\-куп\-ность базовых элементов в~$\Omega_{\mathrm{в}}$. 
Интеграция АИС осуществляется с~учетом существования этого ядра, но без 
его изменения. Автоматическое и~адаптерное взаимодействие аналогично 
первому сценарию, но на существенно большем ядре. Примеры~--- GJXDM 
(Global Justice XML Data Model~--- Глобальная XML-мо\-дель данных правоохранительной сферы на базе веб-тех\-но\-ло\-гий, первая версия \mbox{NIEM}).
\item[Сценарий~III.] Формирование ядра \mbox{ЕМГД} осуществляется по мере 
интеграции АИС путем последовательного присоединения онтологий 
$\Omega_{\mathrm{с}}$ к~$\Omega_{\mathrm{и}}$ без создания начального 
интеграционного ядра. Все взаимодействия интегрированных АИС~--- 
автоматизированные. Такой сценарий реализуется при формировании 
доменов NIEM и~в~НСУД (в~час\-ти формирования метаданных).\\[-15pt]
\item[Сценарий~IV.] Формирование ядра \mbox{ЕМГД} осуществляется по треть\-ему 
сценарию, но интегрируются не отдельные АИС, а~целые домены 
предметной об\-ласти, которые, в~свою очередь, формируются также по 
треть\-ему сценарию из АИС каж\-до\-го домена. Этот сценарий реализован 
в~\mbox{NIEM}.\\[-15pt]
\end{description}

Так как построение отображений онтологий в~процессе их интеграции 
происходит поэлементно, затраты на интеграцию базового набора объектов 
онтологии про\-пор\-ци\-о\-наль\-ны их чис\-лу. Введем понятие $\mu$~---
<<мощ\-ности>> онтологии, которую положим рав\-ной чис\-лу со\-дер\-жа\-щих\-ся 
в~ней базовых элементов, и~обозначим  
$\mu_{\mathrm{в}} \hm= \Omega_{\mathrm{в}}$,  
$\mu_{\mathrm{и}} \hm= \Omega_{\mathrm{и}}$  
и~$\mu_{\mathrm{с}} \hm= \Omega_{\mathrm{с}}$.



Мощность верхней онтологии составляет более десятка тысяч базовых 
элементов, поэтому для моделирования процессов интеграции впол\-не 
применимы вероятностные методы. Процесс интеграции представим 
сле\-ду\-ющим образом. В~общем случае онтология $\Omega_{\mathrm{с}}$ 
ин\-тег\-ри\-ру\-емой АИС содержит \mbox{часть}\linebreak объектов онтологии ядра, для которых 
нет на\-доб-\linebreak\vspace*{-12pt}

\pagebreak

\noindent
ности по\-иска соответствий, и~\mbox{часть} объектов верх\-ней онтологии, не 
входящих в~$\Omega_{\mathrm{и}}$. Если предположить, что со\-став~$\Omega_{\mathrm{с}}$ случаен, математическое\linebreak 
\mbox{ожидание} чис\-ла объектов~$\Omega_{\mathrm{с}}$, не при\-над\-ле\-жа\-щих~$\Omega_{\mathrm{и}}$, 
со\-став\-ля\-ет   $\mu_{\mathrm{с}} (\mu_{\mathrm{в}}\hm - \mu_{\mathrm{и}})/\mu_{\mathrm{в}}$.


В~\cite{11-suc} осуществлен анализ соотношения этих характеристик 
и~показано, что среди пер\-вых трех сценариев наилучшим оказался третий. 
Теперь включим в~рас\-смот\-ре\-ние четвертый сценарий. Итак, по первому 
и~второму сценарию для подключения~$N$~АИС за\-тра\-ты пропорциональны 
мощ\-ности по\-пол\-не\-ния ядра и~со\-став\-ля\-ют~\cite{11-suc}:

\noindent
\begin{align*}
C_1 &\hm= \mu_{\mathrm{и}_1} + N  
\mu_{\mathrm{с}}\,\fr{\mu_{\mathrm{в}} - \mu_{\mathrm{и}_1}} 
{\mu_{\mathrm{в}}}\,; \\
C_2 &\hm= \mu_{\mathrm{и}_2} + N 
\mu_{\mathrm{с}}\,\fr{\mu_{\mathrm{в}} - \mu_{\mathrm{и}_{2}}} 
{\mu_{\mathrm{в}}}\,.
\end{align*}

\vspace*{-3pt}

Соотношение затрат по четырем сценариям пред\-став\-ле\-но 
на рис.~2. При $ N \hm= \mu_{\mathrm{в}}/\mu_{\mathrm{с}}$ имеем $C_1 \hm= C_2$ и~с~рос\-том~$N$ $C_1$ превышает~$C_2$.



Для третьего сценария за\-тра\-ты для $N$ АИС со\-став\-ля\-ют~\cite{11-suc}:
$$
C_3 = \mu_{\mathrm{в}}\left( 1 - \left( 1 - \fr{ \mu_{\mathrm{с}}} 
{\mu_{\mathrm{в}}}\right)^{N-1}\right)\,,
$$

\vspace*{-3pt}

\noindent
т.\,е.\
$C_3$ стремится в~гео\-мет\-ри\-че\-ской про\-грес\-сии к~$\mu_{\mathrm{в}}$ 
и~при достаточно большом $N$ $ \mu_{\mathrm{в}} - C_3 <1 $, что означает 
$\Omega_{\mathrm{и}} \hm= \Omega_{\mathrm{в}}$ в~силу дис\-крет\-ности 
процесса. 

Оценим границу $G_3$ значения~$N$, при котором $ \mu_{\mathrm{в}} \hm- C_3 \hm<1 $. Имеем
$$
\mu_{\mathrm{в}} - C_3 = \mu_{\mathrm{в}} - \mu_{\mathrm{в}}\left( 1 - \left( 
1 - \fr{\mu_{\mathrm{с}}}{\mu_{\mathrm{в}}}\right)^{N-1}\right) <1\,.
$$

\vspace*{-3pt}

\noindent
Отсюда следует, что верх\-няя онтология до\-сти\-га\-ет\-ся при 
$$
N > \fr{\ln \mu_{\mathrm{в}}} {\ln ({\mu_{\mathrm{в}}}/({\mu_{\mathrm{в}} - 
\mu_{\mathrm{с}}}))} = G_3\,.
$$

%\vspace*{-3pt}

Покажем, что четвертый сценарий более оптимален, чем третий. Для 
чет\-вер\-то\-го сценария общие за\-тра\-ты суть сумма за\-трат на создание 
и~при\-со\-еди\-не\-ние к~ядру ЕМГД отдельных доменов. Пусть 
$N_i$, $i \hm= 1, \ldots, R$,~--- чис\-ло АИС в~$i$-м домене мощ\-ностью 
$\mu_{\mathrm{в}}^i$, $R$~--- чис\-ло доменов. Тогда
$$
C_4 = \sum\limits_{i\hm=1}^R f_i(N)\,,
$$


\noindent
где 

\vspace*{-3pt}

\noindent
\begin{equation*}
f_1(N) = \begin{cases}
0, & \hspace*{-30mm} N\leq \displaystyle\sum\limits_1^i N_{j-1}\,,\ N_0 = 0\,;\\[12pt]
 \mu_{\mathrm{в}}^i\left( 1 - \left( 1 - 
\fr{\mu_{\mathrm{с}}}{\mu^i_{\mathrm{в}}}\right)^{N-\sum\nolimits_2^i N_{j - 1}-1}\right) , &\\
& \hspace*{-30mm} \displaystyle N > \sum\limits_1^i N_{j -1}\,.
\end{cases}
\end{equation*}

{ \begin{center}  %fig2
 \vspace*{-2pt}
    \mbox{%
\epsfxsize=79mm
\epsfbox{suh-2.eps}
}

\end{center}

\vspace*{-3pt}

\noindent
{{\figurename~2}\ \ \small{Соотношение затрат по четырем сценариям интеграции
}}}

\vspace*{9pt}

\addtocounter{figure}{1}

По четвертому сценарию за\-тра\-ты на формирование доменных онтологий 
уменьшаются по сравнению с~третьим сценарием из-за большей ско\-рости 
по\-пол\-не\-ния ядра доменов на начальных стадиях. 


Оценим границу $G_4$ значения~$N$, при котором $ \mu_{\mathrm{в}}\hm - C_4 <1$. 
Для упрощения вы\-чис\-ле\-ний предположим, что~$\mu^i_{\mathrm{в}}$ 
одинаковы для всех доменов и~рав\-ны~$\mu^d_{\mathrm{в}}$, при этом 
$\mu^d_{\mathrm{в}} \hm= \mu_{\mathrm{в}}/R.$
Имеем для каждого домена

\noindent
$$
\mu^d_{\mathrm{в}} - \mu^d_{\mathrm{в}}\left( 1 - \left( 1 - 
\fr{\mu_{\mathrm{с}}}{\mu^d_{\mathrm{в}}}\right)^{N-1}\right) <1\,.
$$

\vspace*{-3pt}

\noindent
Отсюда

\noindent
$$
N > R\, \fr{\ln (\mu_{\mathrm{в}}/R)}
{\ln (\mu_{\mathrm{в}}/(\mu_{\mathrm{в}} - R\mu_{\mathrm{с}}))} \hm= G_4\,.
$$

\vspace*{-3pt}

Обозначим

\noindent 
$$
G(x) \hm= x\, \fr{\ln (\mu_{\mathrm{в}}/x)}
{\ln (\mu_{\mathrm{в}}/(\mu_{\mathrm{в}} - x\mu_{\mathrm{с}}))}\,, \enskip 
x > \fr{\mu_{\mathrm{в}}}{\mu_{\mathrm{с}}}\,.
$$

\vspace*{-3pt}

\noindent
Очевидно, $G(1) \hm= G_3$, $G(R) \hm= G_4$. Мож\-но показать, что на интервале 
$1\hm<x\hm < \mu_{\mathrm{в}}/\mu_{\mathrm{с}}$ функ\-ция $G(x)$ имеет 
от\-ри\-ца\-тель\-ную производную и,~значит, монотонно убывает (рис.~3) 
(опус\-ка\-ем вы\-чис\-ле\-ния производной в~силу их гро\-мозд\-кости). 
Следовательно, $G(1) \hm= G_3 \hm> G(R) \hm= G_4$. Это означает, что по четвертому 
сценарию, пред\-по\-ла\-га\-юще\-му подоменную интеграцию ЕМГД, верх\-няя 
онтология ЕМГД до\-сти\-га\-ет\-ся существенно быст\-рее, чем по треть\-ему 
сценарию.


Самым плохим вариантом по суммарным и~по\-сто\-ян\-но рас\-ту\-щим затратам 
является пер\-вый сценарий. Во втором сценарии затраты так\-же по\-сто\-ян\-но 
рас\-тут, но медленнее. Третий сценарий обеспечивает меньшие суммарные 
за\-тра\-ты, которые к~тому же после интеграции определенного чис\-ла АИС 
вообще не рас\-тут, так как не требуют дополнительных расходов. Чет\-вер\-тый 
сценарий дает такой же результат, однако существенно улуч\-ша\-ет ско\-рость\linebreak\vspace*{-12pt}

\pagebreak

{ \begin{center}  %fig3
 \vspace*{-2pt}
    \mbox{%
\epsfxsize=79mm
\epsfbox{suh-3.eps}
}

\end{center}

\vspace*{-4pt}

\noindent
{{\figurename~3}\ \ \small{Семейство функций  $G(x)$:
\textit{1}~--- $\mu_{\mathrm{с}} \hm= 50$; 
\textit{2}~--- 75; \textit{3}~--- 100; \textit{4}~--- 150; 
\textit{5}~--- $\mu_{\mathrm{с}} \hm= 200$
}}}

\vspace*{8pt}

\addtocounter{figure}{1}

\noindent
 и~качество интеграции и~уменьшает организационные за\-тра\-ты, так как не 
требует кон\-цент\-ра\-ции универсальных экспертов \textit{всей} верхней 
онтологии \mbox{ЕМГД}. Это показывает, что, насколько бы ни была ма\-ло\-зат\-рат\-ной 
раз\-ра\-бот\-ка отдельного адап\-те\-ра, с~рос\-том чис\-ла под\-клю\-ча\-емых сис\-тем 
расходы на это превысят рас\-хо\-ды на создание единой модели верх\-ней 
онтологии (ЕМГД).



Следует заметить, что оценка за\-трат проведена лишь с~учетом отоб\-ра\-же\-ния 
онтологий без учета рас\-хо\-дов на создание и~под\-дер\-жа\-ние \mbox{ЕМГД}. Сюда 
входят расходы на создание организационной сис\-те\-мы ЕМГД, проведение 
на\-уч\-но-ис\-сле\-до\-ва\-тель\-ских и~опыт\-но-кон\-ст\-рук\-тор\-ских работ
(\mbox{НИОКР}) по формированию структуры и~правил создания единой  
XML-мо\-де\-ли. Однако это не влияет на кор\-рект\-ность сделанных выводов, 
так как эти рас\-хо\-ды не зависят от чис\-ла под\-клю\-ча\-емых сис\-тем.

\vspace*{-6pt}

\section{Оптимальный сценарий интеграции ведомственной системы}

\vspace*{-4pt}

Рассмотрим гипотетическую ведомственную рас\-пре\-де\-лен\-ную сис\-те\-му 
управ\-ле\-ния (\mbox{ВРСУ}) со своей информационной мо\-делью данных. Регламент 
информационного взаимодействия (РИВ) в~\mbox{ВРСУ} мож\-но рас\-смат\-ри\-вать как 
правовой акт и~со\-пут\-ст\-ву\-ющие соглашения, поз\-во\-ля\-ющие осуществлять 
информационное, в~том чис\-ле меж\-ве\-дом\-ст\-вен\-ное, взаимодействие, 
обес\-пе\-чи\-ва\-ющее функциональные по\-треб\-но\-сти \mbox{ВРСУ} для реализации своих 
функ\-ций и~услуг. Если пред\-по\-ло\-жить, что будет взят курс на создание 
\mbox{ЕМГД} в~пол\-ном объеме и~в~национальном мас\-шта\-бе, то каковы могут быть 
пути адап\-та\-ции РИВ в~этом процессе? Несомненно, магистральным 
на\-прав\-ле\-нием развития станет создание ЕМД \mbox{ВРСУ}, со\-вмес\-ти\-мой 
с~национальной сис\-те\-мой \mbox{ЕМГД}. Рас\-смот\-рим воз\-мож\-ные варианты такого 
развития.
\begin{enumerate}[1.]
\item Участие в~создании \mbox{ЕМГД}:
\begin{itemize}
\item[(a)] создание и~ведение \mbox{ЕМГД}~--- интегрального ядра~--- 
и~присоединение доменов;\\[-15pt]
\item[(б)] создание первоначального цент\-раль\-но\-го ядра на основе созданного 
первоначального домена.
\end{itemize}
\item Присоединение к~\mbox{ЕМГД}:
\begin{itemize}
\item[(a)] интеграция в~\mbox{ЕМГД} путем создания адап\-те\-ров;\\[-15pt] 
\item[(б)] использование \mbox{ЕМГД} для создания и~ведения его подмножества~--- 
ЕМД для \mbox{ВРСУ}. 
\end{itemize}
\end{enumerate}

Рассмотрим перечисленные варианты.

\textbf{Вариант 1а.} Полномасштабное создание и~ведение \mbox{ЕМГД} 
предполагает:
\begin{itemize}
\item разработку методологии, структуры и~правил формирования XML-мо\-де\-ли государственных данных;\\[-14.5pt]
\item разработку программного инструментария по созданию, ведению, 
модернизации элементов XML-мо\-де\-ли государственных данных;\\[-14.5pt]
\item тестирование этих элементов на одном из доменов пред\-мет\-ной об\-ласти 
(по аналогии с~\mbox{NIEM}, где первоначально была отработана  GJXDM;\\[-14.5pt]
\item создание и~ведение общедоступного портала для привлечения 
к~об\-суж\-де\-нию, тестированию, обуче\-нию широкого круга экспертов всех 
пред\-мет\-ных областей \mbox{ЕМГД}, а~так\-же инструментария для формирования 
подмоделей для начала работы в~доменах;\\[-14.5pt]
\item организацию сети доменов с~определением со\-ста\-ва экспертных 
цент\-ров в~доменах;\\[-14.5pt]
\item создание технологий и~интеграцию моделей доменов в~общую  
XML-мо\-дель государственных данных.
\end{itemize}

\begin{table*}\small
\begin{center}
%\tabcolsep=6pt
\begin{tabular}{|cl|c|c|c|c|}
\multicolumn{6}{c}{Сравнительная экспертная оценка вариантов развития 
РИВ ВРСУ}\\
\multicolumn{6}c{}{\ }\\[-6pt]
\hline
\multicolumn{2}{|c|}{\tabcolsep=0pt\begin{tabular}{c}Варианты развития РИВ 
ВРСУ\end{tabular}}&$p_1$&$p_2$&$p_3$&$P$\\
\hline
1 &\multicolumn{1}{l|}{Участие в~создании ЕМГД:}&&&&\\
%\hline
(а) &\tabcolsep=0pt\begin{tabular}{l}создание и~ведение ЕМГД~--- 
интегрального ядра и~присоединение доменов\end{tabular}&2&4&4&7\\
%\hline
(б) &\tabcolsep=0pt\begin{tabular}{l}создание первоначального центрального 
ядра на основе созданного домена\end{tabular} &1&3&2&6\\
\hline
2 &\multicolumn{1}{l|}{Присоединение к~ЕМГД:}&&&&\\
%\hline
(а) &\tabcolsep=0pt\begin{tabular}{l}интеграция в~ЕМГД путем создания 
адаптеров\end{tabular}&3&1&1&8\\
%\hline
(б) &\tabcolsep=0pt\begin{tabular}{l}использование ЕМГД для создания и~ведения подмножества ЕМГД для ВРСУ\end{tabular}&4&2&2&9\\
\hline
\end{tabular}
\end{center}
\vspace*{-6pt}
\end{table*}

\textbf{Вариант 1б.} Создание первоначального цент\-раль\-но\-го ядра на основе 
созданного домена предполагает:
\begin{itemize}
\item проведение НИОКР по формированию методологии, структуры 
и~правил формирования и~развития единой XML-мо\-дели;\\[-14.5pt]
\item создание на этой основе первоначального домена;\\[-14.5pt]
\item первоначальное наполнение ядра \mbox{ЕМГД} на основе общей час\-ти 
созданного домена;\\[-14.5pt]
\item передачу наработанных технологий в~цент\-раль\-ный орган ЕМГД для 
внед\-ре\-ния.
\end{itemize}

\textbf{Вариант 2а.} Интеграция в~ЕМГД путем создания адаптеров~--- это 
первый этап, поз\-во\-ля\-ющий подстроиться к~из\-ме\-ня\-ющей\-ся структуре 
данных, по\-сту\-па\-ющих из\-вне от цент\-раль\-но\-го звена ВРСУ. Такие процессы, 
по всей ви\-ди\-мости, апробированы и~отлажены в~на\-сто\-ящее время, так как это 
основа су\-щест\-ву\-ющей технологии запроса и~получения данных из внеш\-них 
источников, однако существенно увеличивается период реализации. 

\textbf{Вариант 2б.} Преобразование информационной модели данных 
\mbox{ВРСУ} в~форматы \mbox{ЕМГД} связано с~формированием подмножества \mbox{ЕМГД} 
для целей функционирования ВРСУ и~ее цент\-раль\-но\-го звена. Как уже 
упоминалось, один из необходимых инструментариев ЕМГД~--- 
формирование подмножеств общей XML-мо\-де\-ли по различным срезам 
предметной об\-ласти, т.\,е.\ формирование ЕМД \mbox{ВРСУ}. После формирования 
ЕМД \mbox{ВРСУ} необходимо провести корректировку структур данных 
информационных сис\-тем цент\-раль\-но\-го звена и~вза\-и\-мо\-дей\-ст\-ву\-ющих сис\-тем 
всех уровней. 



Для обоснования оптимального сценария развития РИВ в~ЕМД \mbox{ВРСУ} 
необходимо провести сравнительный анализ всех выбранных вариантов. Для 
срав\-не\-ния этих вариантов определим показатели, способы их оцен\-ки 
и~критерии для анализа. 

Первый показатель ($p_1$)~--- сравнительное время, необходимое для 
реализации варианта (от~1 до~4): $p_1 \hm\rightarrow \min$. Второй показатель 
($p_2$)~--- сравнительные за\-тра\-ты на реализацию (от~1 до~4): $p_2 
\rightarrow \min$. Третий показатель ($p_3$)~--- качество реализации 
государственных функций и~услуг (от~1 до~4): $p_3 \hm\rightarrow \max$. 
Интегральный показатель $P \hm= p_1 \hm+p_2 \hm+(5\hm- p_3) \hm\rightarrow \min$.

В~таблице приведены экспертные оценки показателей на основе изложенных 
выше соображений. Оценки показателей осуществлялись сле\-ду\-ющим 
образом: для каж\-до\-го показателя варианты ранжировались по возрастанию 
и~затем им присваивались порядковые номера от~1 до~4. Возможны 
и~другие способы оцен\-ки интегрального показателя, например 
с~уста\-нов\-ле\-ни\-ем весов част\-ных показателей. В~соответствии с~выбранной 
методологией оценки оптимальным путем развития РИВ оказался 
вариант~1б~--- создание и~вклю\-че\-ние в~РИВ первоначального цент\-раль\-но\-го 
ядра \mbox{ЕМГД} на основе созданного домена (см.\ таб\-ли\-цу). Конечно, эти оцен\-ки 
относительны и~зависят еще от многих факторов и~обстоятельств, не 
исключая даже ам\-би\-ци\-оз\-ность\linebreak ставящихся руководством ведомства задач. 
Не\-сом\-нен\-но, данный вариант позволяет ведомству наиболее быст\-ро 
подключиться и~даже возглавить процессы решения проб\-ле\-мы 
информационного \mbox{взаимодействия}, вклю\-чая межведомственное в~целом.



Дальнейший ход этой деятельности, по мнению автора, должен 
осуществляться в~рамках созданной отдельной организационной сис\-те\-мы 
ЕМГД федерального уров\-ня при активном учас\-тии и~ведущей роли 
ведомства и~обеспечения эволюции РИВ в~ЕМД ВРСУ.

\vspace*{-6pt}

{\small\frenchspacing
 {\baselineskip=11.5pt
 %\addcontentsline{toc}{section}{References}
 \begin{thebibliography}{99}
 
 \vspace*{-2pt}
 
\bibitem{1-suc}
\Au{Зацаринный А.\,А., Сучков~А.\,П.} Информационное взаимодействие 
в~распределенных сис\-те\-мах ситуационного управ\-ле\-ния.~--- М.: ФИЦ ИУ 
РАН, 2021. 256~с. 
\bibitem{2-suc}
Концепция создания и~функционирования национальной сис\-те\-мы 
управ\-ле\-ния данными: Распоряжение Правительства РФ от 3~июня 2019~г. №\,1189-р.
{\sf 
http://static.government.ru/media/files/jYh27VIwiZs4\linebreak 4qa0IXJlZCa3uu7qqLzl.pdf}.
\bibitem{3-suc}
National Information Exchange Model. {\sf https://www.\linebreak niem.gov}.
\bibitem{4-suc}
A~European strategy for data. {\sf https://www.tadviser.ru/\linebreak index.php}.
\bibitem{5-suc}
Rolling Plan for ICT Standardisation. {\sf  
https://joinup.ec.\linebreak europa.eu/collection/rolling-plan-ict-standardisation/ rolling-plan-2020}.
\bibitem{6-suc}
Технологический портал СМЭВ. {\sf http://smev.\linebreak gosuslugi.ru}.
\bibitem{7-suc}
Система межведомственного электронного документооборота федеральных органов исполнительной влас\-ти. {\sf https://cs-consult.ru/products/delo/medo}.
\bibitem{8-suc}
Концепция построения и~развития ап\-па\-рат\-но-про\-грам\-мно\-го 
комплекса <<Безопас\-ный город>>: Постановление Правительства РФ от 3~декабря 2014~г.  
№\,2446-р.
\bibitem{9-suc}
О~формировании сис\-те\-мы распределенных ситуационных цент\-ров,  
ра\-бо\-та\-ющих по единому регламенту взаимодействия: Указ Президента 
РФ №\,648 от 25.07.2013~г.
\vspace*{-1pt}
\bibitem{10-suc}
\Au{Лапшин В.\,А.} Онтологии в~компьютерных сис\-те\-мах.~--- М.: 
Научный мир, 2010. 224~с.
\bibitem{11-suc}
\Au{Сучков А.\,П.} Анализ процессов межведомственного информационного 
взаимодействия~// Сис\-те\-мы и~средства информатики, 2018. Т.~28. №\,3. 
С.~118--130.
\end{thebibliography}

 }
 }

\end{multicols}

\vspace*{-10pt}

\hfill{\small\textit{Поступила в~редакцию 15.12.21}}

\vspace*{8pt}

%\pagebreak

%\newpage

%\vspace*{-28pt}

\hrule

\vspace*{2pt}

\hrule

\vspace*{-2pt}

\def\tit{UNIFIED MODEL OF NATIONAL DATA: DEVELOPMENT~SCENARIOS\\[-5pt]}

\def\titkol{Unified model of national data: Development scenarios}

\def\aut{A.\,P.~Suchkov}

\def\autkol{A.\,P.~Suchkov}

\titel{\tit}{\aut}{\autkol}{\titkol}

\vspace*{-15pt}


\noindent
Federal Research Center ``Computer Science and Control'' of the Russian 
Academy of Sciences, 
44-2~Vavilov Str., Moscow 119333, Russian Federation

\def\leftfootline{\small{\textbf{\thepage}
\hfill INFORMATIKA I EE PRIMENENIYA~--- INFORMATICS AND
APPLICATIONS\ \ \ 2022\ \ \ volume~16\ \ \ issue\ 4}
}%
 \def\rightfootline{\small{INFORMATIKA I EE PRIMENENIYA~---
INFORMATICS AND APPLICATIONS\ \ \ 2022\ \ \ volume~16\ \ \ issue\ 4
\hfill \textbf{\thepage}}}

\vspace*{2pt} 

\Abste{The problem of information interaction between heterogeneous 
information systems is considered. Such point of view is taken that this problem must be solved by creating, 
implementing, and maintaining unified data models within the main sections of the subject areas. 
In the perspective, the scope must be enlarged so as to encompass the entire subject area of 
information interaction on a~national scale. Based on the ontological approach, the author 
proposes the solution to the problem of finding effective as well as optimal ways to form 
national data models. The scenarios of integration of departmental systems are considered as 
well.}

\KWE{information interaction; unified data model; ontology; integration scenarios}

  \DOI{10.14357/19922264220415} 

%\vspace*{-16pt}

% \Ack
%   \noindent
 

%\vspace*{4pt}


  \begin{multicols}{2}

\renewcommand{\bibname}{\protect\rmfamily References}
%\renewcommand{\bibname}{\large\protect\rm References}

{\small\frenchspacing
 {%\baselineskip=10.8pt
 \addcontentsline{toc}{section}{References}
 \begin{thebibliography}{99}
\bibitem{1-suc-1}
\Aue{Zatsarinny, A.\,A., and A.\,P.~Suchkov}. 2021. \textit{In\-for\-ma\-tsi\-on\-noe 
vza\-i\-mo\-dey\-st\-vie v~ras\-pre\-de\-len\-nykh sis\-te\-makh si\-tu\-a\-tsi\-on\-no\-go uprav\-le\-niya} 
[Information interaction in distributed situational management systems]. Moscow: 
FRC CSC RAS. 256~p. 
\bibitem{2-suc-1}
Kon\-tsep\-tsiya sozdaniya i~funk\-tsio\-ni\-ro\-va\-niya na\-tsi\-o\-nal'\-noy sis\-te\-my uprav\-le\-niya 
dan\-ny\-mi: Ras\-po\-rya\-zhe\-nie Pra\-vi\-tel'\-st\-va RF ot 3~iyulya 2019~g.\ No.\,1189-r [The 
concept of construction and development of the hardware and software complex 
``Safe City'': Government order No.\,1189-r dated 03.06.2019]. Available at: {\sf 
http://static.\linebreak government.ru/media/files/jYh27VIwiZs44qa0IXJlZCa\linebreak 3uu7qqLzl.pdf} (accessed November~1, 2022)
\bibitem{3-suc-1}
National Information Exchange Model. Available at: {\sf 
https://www.niem.gov} (accessed November~1, 2022).
\bibitem{4-suc-1}
A~European strategy for data. Available at: {\sf 
https:// www.tadviser.ru/index.php} (accessed November~1, 2022).
\bibitem{5-suc-1}
Rolling Plan for ICT Standardization. Available at:\linebreak {\sf 
https://joinup.ec.europa.eu/collection/rolling-plan-ict-standardisation/rolling-plan-2020} (accessed November~1, 2022).
\bibitem{6-suc-1}
Tekhnologicheskiy portal SMEV [SMEV Technology Portal]. Available at:  {\sf 
http://smev.gosuslugi.ru} (accessed November~1, 2022).
\bibitem{7-suc-1}
Sis\-te\-ma mezh\-ve\-dom\-st\-ven\-no\-go elekt\-ron\-no\-go do\-ku\-men\-to\-obo\-ro\-ta fe\-de\-ral'\-nykh 
or\-ga\-nov is\-pol\-ni\-tel'\-noy vlas\-ti [System of interdepartmental electronic document 
management of the federal executive authorities]. Available at: {\sf 
https://cs-consult.ru/products/delo/medo} (accessed November~1, 2022).
\bibitem{8-suc-1}
Kon\-tsep\-tsiya po\-stro\-eniya i~raz\-vi\-tiya ap\-parat\-no-\mbox{pro\-gram\-mno\-go} komp\-lek\-sa 
``Bezopas\-nyy go\-rod'': Pos\-ta\-nov\-le\-nie Pra\-vi\-tel'\-s\-tva RF ot 3~dekabrya 2014~g.\ No.\,2446-r 
[The concept of construction and development of the hardware and software 
complex ``Safe City'': Government decree No.\,2446-r dated 03.12.2014]. 
Available at: {\sf http:// static.government.ru/media/files/OapBppc8jyA.pdf} 
(accessed November~1, 2022).
\bibitem{9-suc-1}
O for\-mi\-ro\-va\-nii sis\-te\-my ras\-pre\-de\-len\-nykh si\-tu\-a\-tsi\-on\-nykh tsent\-rov, ra\-bo\-ta\-yushchikh 
po edi\-no\-mu reg\-la\-men\-tu vza\-i\-mo\-deyst\-viya: Ukaz Pre\-zi\-den\-ta RF No.\,648  ot 25.07.2013
[On the formation of a system of distributed situational centers operating under the 
single rules of interaction. Presidential Decree No.\,648 dated 25.07.2013]
\bibitem{10-suc-1}
\Aue{Lapshin, V.\,A.} 2010. \textit{On\-to\-lo\-gii v~komp'\-yuter\-nykh sis\-te\-makh} 
[Ontology in complex systems]. Moscow: Nauchnyy mir. 224~p.
\bibitem{11-suc-1}
\Aue{Suchkov, A.\,P.} 2018. Ana\-liz pro\-tses\-sov mezh\-ve\-domst\-ven\-no\-go 
in\-for\-ma\-tsi\-on\-no\-go vza\-i\-mo\-deyst\-viya [Analysis of processes of interdepartmental 
information interaction].\linebreak \textit{Sistemy i~Sredstva Informatiki~--- Systems and 
Means of Informatics} 28(3):118--130.

\end{thebibliography}

 }
 }

\end{multicols}

\vspace*{-9pt}

\hfill{\small\textit{Received December 15, 2021}}

\vspace*{-16pt}

\Contrl

\vspace*{-3pt}

\noindent
\textbf{Suchkov Alexander P.} (b.\ 1954)~--- Doctor of Science in technology, 
leading scientist, Institute of Informatics Problems, Federal Research Center 
``Computer Science and Control'' of the Russian Academy of Sciences,  
44-2~Vavilov Str., Moscow 119333, Russian Federation;
\mbox{ASuchkov@frccsc.ru}

\label{end\stat}

\renewcommand{\bibname}{\protect\rm Литература}    