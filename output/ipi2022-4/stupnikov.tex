\def\stat{stupnikov}

\def\tit{ЛОГИЧЕСКАЯ РЕЛЯЦИОННАЯ МОДЕЛЬ СТРУКТУР ДАННЫХ ДЛЯ~РЕШЕНИЯ 
ЗАДАЧ В~ПРЕДМЕТНОЙ ОБЛАСТИ УПРАВЛЕНИЯ~ЗЕМЛЕПОЛЬЗОВАНИЕМ$^*$}

\def\titkol{Логическая реляционная модель структур данных для~решения 
задач в %~предметной 
области управления землепользованием}

\def\aut{Д.\,О.~Брюхов$^1$, С.\,А.~Ступников$^2$}

\def\autkol{Д.\,О.~Брюхов, С.\,А.~Ступников}

\titel{\tit}{\aut}{\autkol}{\titkol}

\index{Брюхов Д.\,О.}
\index{Ступников С.\,А.}
\index{Briukhov D.\,O.}
\index{Stupnikov S.\,A.}


{\renewcommand{\thefootnote}{\fnsymbol{footnote}} \footnotetext[1]
{Исследование выполнено при финансовой поддержке Российской Федерации (соглашение с~Минобрнауки 
России №\,075-15-2020-805 от 02~октября 2020~г.).}}


\renewcommand{\thefootnote}{\arabic{footnote}}
\footnotetext[1]{Федеральный исследовательский центр <<Информатика и~управление>> Российской академии наук, 
\mbox{dbriukhov@ipiran.ru}}
\footnotetext[2]{Федеральный исследовательский центр <<Информатика и~управление>> Российской академии наук, 
\mbox{sstupnikov@ipiran.ru}}


\vspace*{-12pt}




\Abst{Статья относится к~теоретическим основам агроинформатики, а~более конкретно~--- 
к~об\-ласти разработки методов и~средств автоматизации управления землепользованием. 
Предлагается логическая модель структур данных для решения задач в~упомянутой 
предметной области. В~виде набора отношений реляционной модели данных и~связей 
между ними формализованы: реестр агроэкологических групп и~видов земель, 
упорядочивающий разнообразие их геоморфологических, почвенных и~агрохимических 
условий; реестр сортов сельскохозяйственных культур и~регистр агротехнологий как 
комплексов технологических операций по управ\-ле\-нию продукционными процессами 
сельскохозяйственных культур. Применение логической модели демонстрируется на 
примерах нескольких обобщенных задач управления землепользованием. Решение каждой 
из задач реализовано в~виде декларативного выражения (запроса) над логической 
моделью. Модель реализована в~реляционной сис\-те\-ме управ\-ле\-ния базами данных (СУБД).}

\KW{реляционная логическая модель; управление землепользованием; решение задач}

 \DOI{10.14357/19922264220414} 
  
%\vspace*{-3pt}


\vskip 10pt plus 9pt minus 6pt

\thispagestyle{headings}

\begin{multicols}{2}

\label{st\stat}

\section{Введение}

    Разработка методов и~средств автоматизации управления 
землепользованием~--- это важное направление современной 
агроинформатики. В~частности, одной из насущных оптимизационных задач 
стала задача планирования севооборота (на каких участках поля и~в~какой 
последовательности следует выращивать сельскохозяйственные культуры). 
Для автоматизации решения этой задачи в~мире активно разрабатываются 
информационные методы и~системы поддержки умного 
землепользования~[1]. 

В~качестве входных данных для проектирования 
севооборотов используются характеристики сортов культур, климатические 
данные, данные о~доступной технике, данные о~полях (включая размеры, 
форму, характеристики почвы)~[2, 3]. В~част\-ности, подход, 
разрабатываемый в~Почвенном институте им.\ В.\,В.~Докучаева, опирается на 
методологию комплексной оценки сельскохозяйственных земель~[4, 5] 
и~предполагает использование данных, накопленных в~реестре 
агроэкологических групп и~видов земель (упорядочивающем разнообразие их 
геоморфологических, почвенных и~агрохимических условий), реестре сортов 
сельскохозяйственных культур (содержащем характеристики урожайности, 
качества и~устойчивости сортов к~различным факторам) и~регистре базовых 
агротехнологий (как комплексов технологических операций по управлению 
продукционными процессами сельскохозяйственных культур). 
    
    Данная статья посвящена теоретическому обосно\-ва\-нию подхода с~точки 
зрения структур данных. 

Предлагается логическая реляционная модель для 
решения задач в~об\-ласти управ\-ле\-ния землепользованием, 
формализующая упомянутые выше реестры в~виде набора отношений 
реляционной модели данных и~связей между ними. Применение логической 
модели демонстрируется на примерах нескольких обобщенных задач 
управления землепользованием. Решение каждой из задач реализовано в~виде 
декларативного выражения (запроса) над логической моделью (в~текс\-те 
статьи приведен один запрос).

\begin{figure*}[b] %fig1
\vspace*{1pt}
\begin{center}
   \mbox{%
\epsfxsize=160mm
\epsfbox{stu-1.eps}
}
\end{center}
\vspace*{-9pt}
\Caption{Фрагмент модели реестра агроэкологических групп и~видов земель}
%\end{figure*}
%\begin{figure*} %fig2
\vspace*{12pt}
\begin{center}
   \mbox{%
\epsfxsize=123mm
\epsfbox{stu-2.eps}
}
\end{center}
\vspace*{-9pt}
\Caption{Фрагмент модели реестра сортов культур}
\end{figure*}

    
    Модель реализована в~СУБД PostgreSQL, а соответствующая база 
данных рассматривается как составная часть системы <<Умное 
землепользование>>, реализуемой в~рамках проекта <<Актуальные научные 
задачи стратегии адаптации потенциала землепользования России в~условиях 
беспрецедентных вызовов (экономический кризис, изменения климата, 
кризис глобальных тенденций природопользования)>> (далее~--- 
Проект). 
    
\section{Логическая реляционная модель для~реестров 
агроэкологических групп земель, сортов, агротехнологий}

\subsection{Логическая модель реестра агроэкологических групп и~видов~земель}

    Реестр агроэкологических групп и~видов земель упорядочивает 
разнообразие их геолого-геоморфологических, гидрогеоморфологических, 
почвенных и~агрохимических условий. 
    
    На рис.~1 представлен фрагмент логической модели реестра 
агроэкологических групп и~видов земель, содержащий структуры данных 
о~группах земель (отношение AE\_GROUP) и~видах земель (\mbox{отношение} 
AE\_TYPE), представленных в~конкретных провинциях (отношение 
PROVINCE) природных зон (отношение ZONE). Модель также содержит 
связи провинций с~конкретными административными областями и~районами 
и~структуру характеристик почв каждого вида земель (отношение 
AE\_TYPE\_PROPERTY). 

\begin{figure*}[b] %fig3
\vspace*{1pt}
\begin{center}
   \mbox{%
\epsfxsize=160mm
\epsfbox{stu-3.eps}
}
\end{center}
\vspace*{-9pt}
\Caption{Фрагмент модели регистра базовых агротехнологий}
%\end{figure*}
%\begin{figure*} %fig4
\vspace*{18pt}
\begin{center}
   \mbox{%
\epsfxsize=160mm
\epsfbox{stu-4.eps}
}
\end{center}
\vspace*{-9pt}
\Caption{Фрагмент модели хо\-зяй\-ст\-вен\-но-эко\-но\-ми\-че\-ских па\-ра\-метров}
\end{figure*}



\subsection{Логическая модель реестра сортов культур}

    На рис.~2 представлен фрагмент логической модели реестра сортов, 
содержащий структуру основных данных о~сортах культур (отношение\linebreak 
CROP\_CULTIVAR), включающую название культуры (crop\_name) и~сорта 
(crop\_cultvar), год районирования (zoning\_year), разновидность (variety) 
и~назначение использования сорта (usage). \mbox{Отношение} 
CROP\_CULTIVAR\_PROPERTY позволяет связывать с~сортом различные 
дополнительные свойства (indicator), такие как длительность периода 
вегетации, морозоустойчивость и~зимостойкость, засухоустойчивость, 
требования к~условиям возделывания и~др., и~указывать их значения 
(value). 
    


\subsection{Логическая модель регистра базовых агротехнологий}

    Современные агротехнологии представляют собой комплексы 
технологических операций по управ\-ле\-нию продукционными процессами 
сельскохозяйственных культур в~агроценозах с~целью достижения 
планируемой урожайности и~качества продукции при обеспечении 
экологической безопасности и~экономической эффективности~[6].
    

    На рис.~3 представлен фрагмент логической модели регистра 
агоротехнологий, содержащий структуру общих данных об агротехнологии 
(отношение AGROTECHNOLOGY), применяемые ресурсы интенсификации 
производства~--- удобрения и~химические средства защиты растений 
(отношение TECHNOLOGY\_INTENSIFICATION\_\linebreak RESOURCES), список 
процессов и~операций агротехнологии (отношение 
TECHNOLOGY\_\linebreak OPERATIONS). Структура общих данных включает название 
выращиваемой культуры (crop\_name) в~провинции (province\_name) зоны 
(zone\_name), применяемый уровень технологии (technology\_\linebreak level), список 
подходящих агроэкологических видов земель, список рекомендуемых сортов, 
показатели качества, гарантированную урожайность и~др. Структура 
данных об операциях включает технологические параметры 
(technology\_parameters), сроки (time), используемую технику (equipment).

\subsection{Логическая модель структуры хозяйственно-экономических 
параметров}

    В отдельную схему модели выделены структуры  
хо\-зяй\-ст\-вен\-но-эко\-но\-ми\-че\-ских параметров использования 
агроэкологических видов земель.
    


    На рис.~4 представлен фрагмент модели  
хо\-зяй\-ст\-вен\-но-эко\-но\-ми\-че\-ских параметров. Отношение 
SOIL\_TYPE\_TECHNOLOGY содержит данные об урожайности (harvest) 
и~себестоимости за тонну (cost) выращивания культур в~за\-ви\-си\-мости от 
провинции, агроэкологического вида земли, уровня ин\-тен\-сив\-ности 
технологий. Отношение PROVINCE\_TECHNOLOGY\_CROP содержит 
данные о~подходящих сортах (crop\_cultvars) для выращивания на землях 
конкретных агроэкологических видов и~в~за\-ви\-си\-мости от уровня технологий. 
Отношение TECHNOLOGY\_LEVEL представляет собой словарь 
су\-щест\-ву\-ющих уровней ин\-тен\-сив\-ности технологий (экстенсивный, 
нормальный, интенсивный, точный)~[6]. 
    
\subsection{Логическая модель данных о~конкретных участках}

    При планировании землепользования в~конкретном хозяйстве одним из 
этапов является разделение территории на участки с~выделением их 
агроэкологических групп и~видов земель. Эти данные выделяются 
в~отдельное отношение FIELD\_AE\_ TYPE (рис.~5), содержащее название 
конкретного хозяйства (agrostation\_name), список его по\-лей/участ\-ков 
(field\_name), агроэкологический вид земель каждого участка (ae\_type) 
и~площадь каждого участка (area). 


    

    Рассмотренная логическая модель базы данных реестров была 
реализована в~СУБД PostgreSQL. В~рамках Проекта были собраны 
и~загружены в~базу данные о~группах и~видах земель для Среднерусской 
провинции юж\-но-та\-еж\-но-лес\-ной зоны, данные о~сор\-тах зерновых 
и~зернобобовых культур и~данные о~базовых агротехнологиях возделывания 
этих культур для Центрального региона Нечерноземной зоны.

{ \begin{center}  %fig5
 \vspace*{7pt}
    \mbox{%
\epsfxsize=48mm
\epsfbox{stu-5.eps}
}

\vspace*{3pt}

\noindent
{{\figurename~5}\ \ \small{
Фрагмент модели данных о конкретных участках
}}
\end{center}
}

%\vspace*{6pt} 


Полная 
реализация модели в~SQL (structured query language), данные, использовавшиеся для экспериментов, 
и~скрипты их загрузки в~базу данных выложены в~репозитории 
github\footnote{{\sf  https://github.com/sstupnikov/Smart-Landuse}.}.





\section{Применение логической модели на~примерах обобщенных 
задач управления землепользованием}

    Ниже приведены некоторые варианты задач, возникающих при 
планировании землепользования и~решаемых с~использованием данных из 
реестров.
    \begin{enumerate}[1.]
\item Для поля, состоящего из $n$ участков с~видами земель $\mathrm{aet}_1, \ldots , 
\mathrm{aet}_n$,  определить минимальный уровень интенсивности технологий, 
при котором культура crop даст среднюю уро\-жай\-ность по полю не менее 
yield.
\item Для участка площадью~$S$ с~видом земли $\mathrm{aet}$ определить 
максимальную по ин\-тен\-сив\-ности технологию для культуры~crop, при 
которой се\-бе\-сто\-и\-мость производства не превышает cost.
\item Для участка с~видом земли $\mathrm{aet}$ определить технологию, 
обеспечивающую максимальную рен\-та\-бель\-ность выращивания культуры 
crop.
\item Для участка с~видом земли $\mathrm{aet}$ определить технологию, 
обес\-пе\-чи\-ва\-ющую максимальный по объему урожай культуры crop.
\end{enumerate}
    
    Реализация решения подобных задач осуществляется с~по\-мощью 
декларативных реляционных запросов над базой данных реестров, 
ре\-а\-ли\-зу\-ющей логическую модель из разд.~3. Ввиду ограниченного объема 
статьи рас\-смот\-рим только запрос, ре\-а\-ли\-зу\-ющий пример 
задачи~2\footnote{Запросы, реализующие примеры для всех четырех задач, 
выложены в~файле {\sf https://github.com/sstupnikov/Smart-Landuse/blob/main/db\_queries\_ed.txt.}}.


    
 Для примера задачи~2 был выбран учас\-ток~1 пло\-щадью 0,792~га 
в~хозяйстве Козлово, находящемся в~среднерусской провинции  
юж\-но-та\-еж\-но-\linebreak лес\-ной зоны (табл.~1). В~качестве культуры вы\-бра\-на
%\noindent
озимая рожь, при этом ограничение на се\-бе\-сто\-и\-мость выращивания 
со\-став\-ля\-ет~20\,000~руб. %\linebreak\vspace*{-6pt}


    

\end{multicols}

\setcounter{table}{0}
\begin{table*}[h]\small %tabl1
\vspace*{-12pt}
    \begin{center}
    \Caption{Данные об участке}
     \vspace*{2ex}
     
     \begin{tabular}{|c|c|c|c|}
     \hline
\textbf{Хозяйство}&\textbf{Участок}&\textbf{Вид земель}&\textbf{Площадь, га}\\
\hline
Козлово&1&1.1. Дерново-подзолистые средне- и~легкосуглинистые культурные&0,792\\
\hline
\end{tabular}
\end{center}
\end{table*}

\pagebreak



    \setcounter{table}{1}
\begin{table*}[h]\small %tabl2
    \begin{center}
    \Caption{Результат запроса}
     \vspace*{2ex}
     
     \begin{tabular}{|c|c|c|c|c|c|c|}
     \hline
\textbf{Хозяйство}&\textbf{Участок}&\textbf{Культура}&\tabcolsep=0pt\begin{tabular}{c}\textbf{Уровень}\\ \textbf{технологий}\end{tabular}&
\tabcolsep=0pt\begin{tabular}{c}\textbf{Себестоимость}\\ \textbf{за  гектар, руб.}\end{tabular}&\textbf{Площадь, га}&\tabcolsep=0pt\begin{tabular}{c}\textbf{Общая}\\ 
\textbf{себестоимость, руб.}\end{tabular}\\
\hline
Козлово&1&Озимая рожь&Нормальный&24\,990&0,792&19\,792,08\\
\hline
\end{tabular}
\end{center}
\end{table*}

 \begin{multicols}{2}  


    
Реляционный запрос на языке SQL, ре\-а\-ли\-зу\-ющий задачу, основан на 
соединении  трех отношений: FIELD\_AE\_TYPE,
SOIL\_TYPE\_\linebreak TECHNOLOGY и~TECHNOLOGY\_LEVEL (см.\ разд.~3). Запрос 
выглядит сле\-ду\-ющим образом:

\vspace*{-6pt}

\noindent
   {\small \begin{verbatim}
select
  d.agrostation_name
  , d.field_name
  , st.crop_name
  , st.technology_level
  , st.harvest*st.cost cost_per_ha
  , d.area
  , st.harvest*st.cost*d.area cost_for_field
from
  economic_parameters."SOIL_TYPE_TECHNOLOGY" st
  , data."FIELD_AE_TYPE" d
  , economic_parameters."TECHNOLOGY_LEVEL" tl
where
  st.zone_name = d.zone_name 
          and st.province_name = d.province_name
  and st.ae_type = d.ae_type
  and st.technology_level = tl .technology_level
  and d.agrostation_name = 'Козлово'
  and d.field_name = '1'
  and st.crop_name = 'Озимая рожь'
  and st.harvest*st.cost*d.area < 20000
order by
  tl.index desc
limit 1
    \end{verbatim}
    }
    
    \vspace*{-6pt}
    
    Себестоимость вычисляется как произведение урожайности (harvest), 
се\-бе\-сто\-и\-мости за тонну (cost) и~площади (area). Максимальный уровень 
ин\-тен\-сив\-ности выбирается при помощи упорядочения (order by) индекса 
уровня технологии (index) по убыванию (desc) и~выбора первого результата 
(limit~1).


    
    
    В табл.~2 представлен результат этого запроса, содержащий данные об 
удовле\-тво\-ря\-ющей заданному бюджету се\-бе\-сто\-и\-мости выращивания на 
участ\-ке озимой ржи для максимального уров\-ня технологии.
    
    

\section{Заключение}

    Направление разработки методов и~средств автоматизации управ\-ле\-ния 
землепользованием~--- одно из важ\-ней\-ших в~современной агроинформатике. 
В~работе рас\-смот\-ре\-на логическая модель структур данных, 
фор\-ма\-ли\-зу\-ющая данные об агроэкологических группах земель, сортах 
и~агротехнологиях. Показано, что эта модель может служить основой для 
решения задач планирования землепользования.
    
    Направление дальнейшей работы~--- встраивание реализованной базы 
данных в~сис\-те\-му <<Умное землепользование>> и~реализация пол\-но\-го 
набора задач планирования землепользования в~виде реляционных запросов.
    
{\small\frenchspacing
 {%\baselineskip=10.8pt
 %\addcontentsline{toc}{section}{References}
 \begin{thebibliography}{9}
\bibitem{1-st}
\Au{Sch$\ddot{\mbox{o}}$ning~J., Wachter~P., Trautz~D.} Crop rotation and management tools 
for every farmer? The current status on crop rotation and management tools for enabling 
sustainable agriculture worldwide~// Smart Agricultural Technology, 2023. Vol.~3. 
Art.\ 100086. 15~p. doi: 10.1016/j. atech.2022.100086.
\bibitem{2-st}
\Au{Pahmeyer~C., Kuhn~T., Britz~W.} ``FruchtFolge'': A~crop rotation decision support system for 
optimizing cropping choices with big data and spatially explicit modeling~// Comput. 
Electron. Agr., 2021. Vol.~181. Art.\ 105948. 11~p.
\bibitem{3-st}
\Au{Kim J.\,S., Kisekka~I.} FARMs: A~geospatial crop modeling and agricultural water 
management system~// ISPRS Int. J.~Geo-Inf., 2021. Vol.~10. No.\,8. 
Art.\ 553. 17~p. doi: 10.3390/ijgi10080553.
\bibitem{4-st}
\Au{Кирюшин В.\,И.} Методология комплексной оцен\-ки сельскохозяйственных земель~// 
Почвоведение, 2020. №\,7. С.~871--879.
\bibitem{5-st}
\Au{Кирюшин В.\,И., Дубачинская~Н.\,Н., Юрова~А.\,Ю.} Комплексная оцен\-ка 
сельскохозяйственных земель на примере Южного Урала~// Почвоведение, 2021. №\,11. 
С.~1363--1375.
\bibitem{6-st}
\Au{Кирюшин В.\,И., Кирюшин~С.\,В.} Агротехнологии.~--- СПб.: Лань, 2022. 464~с. 
\end{thebibliography}

 }
 }

\end{multicols}

\vspace*{-6pt}

\hfill{\small\textit{Поступила в~редакцию 29.08.22}}

%\vspace*{8pt}

%\pagebreak

\newpage

\vspace*{-28pt}

%\hrule

%\vspace*{2pt}

%\hrule

%\vspace*{-2pt}

\def\tit{LOGICAL  RELATIONAL MODEL OF~DATA STRUCTURES FOR~PROBLEM  SOLVING IN~LAND USE MANAGEMENT}


\def\titkol{Logical  relational model of~data structures for~problem  solving in~land use management}


\def\aut{D.\,O.~Briukhov and~S.\,A.~Stupnikov}

\def\autkol{D.\,O.~Briukhov and~S.\,A.~Stupnikov}

\titel{\tit}{\aut}{\autkol}{\titkol}

\vspace*{-8pt}


\noindent
Federal Research Center ``Computer Science and Control'' of the Russian Academy of Sciences, 
44-2~Vavilov Str., Moscow 119333, Russian Federation


\def\leftfootline{\small{\textbf{\thepage}
\hfill INFORMATIKA I EE PRIMENENIYA~--- INFORMATICS AND
APPLICATIONS\ \ \ 2022\ \ \ volume~16\ \ \ issue\ 4}
}%
 \def\rightfootline{\small{INFORMATIKA I EE PRIMENENIYA~---
INFORMATICS AND APPLICATIONS\ \ \ 2022\ \ \ volume~16\ \ \ issue\ 4
\hfill \textbf{\thepage}}}

\vspace*{3pt} 



 \Abste{The paper belongs to theoretical foundations of agroinformatics. To be more precise, it concerns development of 
 methods and tools for automatization of land use management. A~logical model for data structures intended for problem solving in the subject area is proposed. 
 Registry for agroecological groups and types of lands organizing diversity of their geomorphological, soil, and agrochemical conditions, registry for crops 
 and cultivars, and registry for agrotechnologies as complexes of technological operations for management of crop production processes are formalized as sets 
 of relations of the relational data model and relationships between them. Application of the logical model is demonstrated via several generic 
 problems of land use management. Each problem is implemented as 
 a~declarative query expression over the logical model. The model is implemented in a~relational database management system.}

\KWE{logical relational model; land use management; problem solving}


 \DOI{10.14357/19922264220414} 

\vspace*{-12pt}

 \Ack
\noindent
The study was financially supported by the Ministry of Education and Science of 
the Russian Federation (agreement No.\,075-15-2020-805 dated October 2, 2020). 


\vspace*{6pt}

  \begin{multicols}{2}

\renewcommand{\bibname}{\protect\rmfamily References}
%\renewcommand{\bibname}{\large\protect\rm References}

{\small\frenchspacing
 {%\baselineskip=10.8pt
 \addcontentsline{toc}{section}{References}
 \begin{thebibliography}{9}
\bibitem{1-st-1}
\Aue{Sch$\ddot{\mbox{o}}$ning, J., P.~Wachter, and D.~Trautz.} 2023. Crop 
rotation and management tools for every farmer? The current status on crop 
rotation and management tools for enabling sustainable agriculture worldwide. 
\textit{Smart Agricultural Technology} 3:100086. 15~p.  doi: 10.1016/j.atech. 2022.100086.
{\looseness=1

}
\bibitem{2-st-1}
\Aue{Pahmeyer, C., T.~Kuhn, and W.~Britz.} 2021. ``FruchtFolge'': A crop rotation 
decision support system for optimizing cropping choices with big data and 
spatially explicit modeling. \textit{Comput. Electron. Agr.} 181:105948.  11~p.
\bibitem{3-st-1}
\Aue{Kim, J.\,S., and I.~Kisekka.} 2021. FARMs: A~geospatial crop modeling and 
agricultural water management system. \textit{\mbox{ISPRS} Int. J. Geo-Inf.} 10(8):553. 17~p.
doi: 10.3390/\linebreak ijgi10080553.
\bibitem{4-st-1}
\Aue{Kiryushin, V.\,I.} 2020. Methodology for integrated assessment of agricultural 
land. \textit{Eurasian Soil Sci.} 53(7):960--967.
\bibitem{5-st-1}
\Aue{Kiryushin, V.\,I., N.\,N.~Dubachiskaya, and A.\,Yu.~Yurova}. 2021. 
Comprehensive assessment of agricultural land by the exemple of the Southern Urals. \textit{Eurasian Soil Sci.} 54(11):1721--1731.
\bibitem{6-st-1}
\Aue{Kiryushin, V.\,I., and S.\,V.~Kiryushin.} 2022. \textit{Agrotekhnologii} [Agrotechnologies]. 
St.\ Petersburg: Lan'. 464~p.
\end{thebibliography}

 }
 }

\end{multicols}

\vspace*{-6pt}

\hfill{\small\textit{Received August 29, 2022}}

\vspace*{-16pt}
    
    \Contr
    \noindent
    \textbf{Briukhov Dmitry O.} (b.\ 1971)~--- Candidate of Science (PhD) in 
technology, senior scientist, Institute of Informatics Problems, Federal Research 
Center ``Computer Science and Control'' of the Russian Academy of Sciences, \mbox{44-2}~Vavilov Str., Moscow 119333, Russian Federation; 
\mbox{dbriukhov@ipiran.ru}
    
    \vspace*{3pt}
    
    \noindent
    \textbf{Stupnikov Sergey A.} (b.\ 1978)~--- Candidate of Science (PhD) in 
technology, leading scientist, Institute of Informatics Problems, Federal Research 
Center ``Computer Science and Control'' of the Russian Academy of Sciences, 44-2~Vavilov Str., Moscow 119333, Russian Federation; 
\mbox{sstupnikov@ipiran.ru}
  
\label{end\stat}

\renewcommand{\bibname}{\protect\rm Литература}    
   