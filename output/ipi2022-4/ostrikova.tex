\def\stat{ostrikova}

\def\tit{ОБ ОПТИМАЛЬНОМ РАСПОЛОЖЕНИИ АНТЕНН ДЛЯ~V2X-СОЕДИНЕНИЙ 
В~СУБТЕРАГЕРЦЕВОМ ДИАПАЗОНЕ$^*$}

\def\titkol{Об оптимальном расположении антенн для~V2X-соединений 
в~субтерагерцевом диапазоне}

\def\aut{Е.\,А.~Мачнев$^1$, В.\,А.~Бесчастный$^2$, Д.\,Ю.~Острикова$^3$, 
Ю.\,В.~Гайдамака$^4$, С.\,Я.~Шоргин$^5$}

\def\autkol{Е.\,А.~Мачнев, В.\,А.~Бесчастный, Д.\,Ю.~Острикова и~др.}
%$^3$,  Ю.\,В.~Гайдамака$^4$, С.\,Я.~Шоргин$^5$}

\titel{\tit}{\aut}{\autkol}{\titkol}

\index{Мачнев Е.\,А.}
\index{Бесчастный В.\,А.}
\index{Острикова Д.\,Ю.}
\index{Гайдамака Ю.\,В.}
\index{Шоргин С.\,Я.}
\index{Machnev E.\,A.}
\index{Beschastnyi V.\,A.}
\index{Ostrikova D.\,Yu.}
\index{Gaidamaka Yu.\,V.}
\index{Shorgin S.\,Ya.}


{\renewcommand{\thefootnote}{\fnsymbol{footnote}} \footnotetext[1]
{Исследование выполнено за счет гранта Российского научного фонда № 22-29-00694.}}


\renewcommand{\thefootnote}{\arabic{footnote}}
\footnotetext[1]{Российский университет дружбы народов, 1042200071@pfur.ru}
\footnotetext[2]{Российский университет дружбы народов, beschastnyy-va@rudn.ru}
\footnotetext[3]{Российский университет дружбы народов, ostrikova-dyu@rudn.ru}
\footnotetext[4]{Российский университет дружбы народов; Федеральный исследовательский центр <<Информатика и~управ\-ле\-ние>> 
Российской академии наук, \mbox{gaydamaka-yuv@rudn.ru}}
\footnotetext[5]{Федеральный исследовательский центр <<Информатика и~управление>> 
Российской академии наук, sshorgin@ipiran.ru}

\vspace*{-6pt}

  
  \Abst{Субтерагерцевая (суб-ТГц, 100--300~ГГц) связь должна обеспечить огромную  
ско\-рость передачи данных в~сис\-те\-мах~6G. Однако зона покрытия базовых станций 
(БС) ограничена, так как сигнал существенно затухает с~увеличением дистанции, а~также 
легко блокируется различными объектами, встречающимися на пути распространения 
сигнала. Таким образом, БС необходимо располагать достаточно часто, что делает такое 
решение дорогостоящим. Для снижения плотности развертывания БС можно использовать 
механизм ретрансляции сигнала с~помощью транспортных средств (ТС). Данный способ 
в~большой степени зависит от зоны размещения приемо-пе\-ре\-да\-ющей антенны на 
кузове ТС, что ставит вопрос о поиске расположения антенны, при котором механизм 
ретрансляции будет эффективным с~точки зрения скорости передачи данных и~расстояния 
между ТС-источником и~БС. В~данной работе на основе спецификации IEEE 802.15.3d 
и~экспериментальных данных о~распространении сигнала на частоте 300~ГГц 
предложена математическая модель для анализа многозвеньевой системы ретрансляции 
сигнала для трех зон размещения антенны на кузове ТС. Полученные результаты 
показывают, что расположение передатчика в~зоне лобового стекла характеризуется более 
низкой скоростью передачи данных, но при этом гораздо большим покрытием, чем 
расположение в~зонах бампера и~двигателя.}
  
  \KW{5G; <<новое радио>>; V2V; V2X; ретрансляция}
  
  \DOI{10.14357/19922264220407} 
  
%\vspace*{-3pt}


\vskip 10pt plus 9pt minus 6pt

\thispagestyle{headings}

\begin{multicols}{2}

\label{st\stat}

\section{Введение}



     Системы 5G New Radio (NR), \mbox{ра\-бо\-та\-ющие} в~мик\-ро\-вол\-но\-вом 
($\mu$Wave) и~миллиметровом (mmWave) диапазонах, уже вышли на рынок. 
В~то же время исследователи приступили к~изучению субтерагерцевых  
(100--300~ГГц) диапазонов в~контексте сис\-тем~6G~[1, 2]. Однако 
чрезвычайно высокое затухание сигнала, эффекты динамической 
блокировки~[3], а~также микромобильность~[4, 5] ограничивают дальность 
действия таких систем несколькими сотнями метров.
     
     Одна из серьезных проблем для сотовых операторов~--- обеспечение 
расширенного мобильного широкополосного доступа (Enhanced mobile 
broadband, eMBB) для пользователей в~движущихся ТС.\linebreak
 Эта услуга требует не только постоянного подключения, но и~высокой 
ско\-рости передачи данных. Одним из решений могут стать 
высокопроизводительные сис\-те\-мы~5G NR, работающие в~\mbox{диапазоне} 
миллиметровых волн, или 6G в~субтерагерцевом диапазоне. Однако высокое 
затухание в~субтерагерцевом диапазоне и~другие особенности 
распространения сигнала, такие как блокировка прямой видимости, 
ограничивают зону покрытия расположенных по краям дороги БС 
несколькими сотнями метров, что требует высокой плотности развертывания, 
а~значит, ставит вопрос о~рен\-та\-бель\-ности сис\-темы.

\begin{figure*}[b] %fig1
\vspace*{-4pt}
\begin{center}
   \mbox{%
\epsfxsize=163mm
\epsfbox{mac-1.eps}
}
\end{center}
\vspace*{-9pt}
\Caption{Схема моста из одного ТС-ре\-транс\-ля\-то\-ра с~учас\-ти\-ем  
ТС-от\-ра\-жа\-теля}
\end{figure*}

     
     Для обеспечения постоянной связи и~снижения финансовых затрат 
сетевых операторов можно использовать механизм ретрансляции 
сигнала~[6]. Чтобы обеспечить поддержку этого механизма, консорциум 
3GPP недавно стандартизировал технологию интегрированного доступа 
и~транспортной сети (Integrated Access and Backhaul, IAB)~[7]. В~рамках 
этой технологии ТС-ретрансляторы с~высокоскоростными передатчиками 
в~диапазоне субтерагерцевых частот могут организовать так называемые 
<<мосты>>, пересылая по цепочке данные от ТС, которые в~настоящее время 
не имеют прямого подключения к~БС. Эффективность механизма 
многозвеньевой ретрансляции сигнала оценивается на основании ско\-рости 
передачи данных по установленному мосту и~длины моста, т.\,е.\ расстояния 
между ТС-источником сигнала и~БС. При этом расположение  
при\-емо-пе\-ре\-да\-ющей антенны на кузове ТС влияет на оба указанных 
показателя. 
     
     Конечная цель данного исследования~--- выработать рекомендации по 
выбору зоны размещения антенны на ТС, при котором механизм 
ретрансляции обеспечит минимальную плотность развертывания БС для 
заданного набора параметров и~условий дорожного движения, включая 
плотность ТС на дороге и~ско\-рости их движения. Разработанная 
математическая модель основана на недавних исследованиях 
распространения сигнала в~субтерагерцевом диапазоне в~среде~V2V 
(Vehicle-to-Vehicle) и~использует реалистичные параметры связи из стандарта 
IEEE~802.15.3d~[8].
     
\section{Системная модель}

     Рассматривается участок автомагистрали, например городская улица 
или скоростное шоссе, покрытие которого беспроводной связью 
обеспечивается БС~6G, работающими в~субтерагерцевом 
диапазоне. Предполагается, что БС установлены по обеим сторонам дороги 
в~шахматном порядке, например на фонарных столбах, на постоянной 
высоте~$h_A$. Расстояние между БС по одной стороне дороги равно~$d$, 
таким образом, БС образуют равнобедренные треугольники 
с~основанием~$d$, т.\,е.\ БС на противоположных сторонах сдвинуты друг 
относительно друга на расстояние $d/2$. Эти БС служат точками доступа 
в~интернет для пользователей, находящихся в~ТС, например в~автомобилях 
(рис.~1).
     


     Предполагается, что дорога имеет четное чис\-ло~$N_l$~полос, при этом 
возможно движение ТС в~противоположных направлениях. Ширина полос 
постоянна и~равна~$w$. Ско\-рость ТС предполагается равной~$v$. 
Расположение ТС на дорожной полосе определяется пуассоновским 
процессом ин\-тен\-сив\-ности~$\lambda$, согласно которому расположены 
центры ТС. Далее предположим, что автомобили имеют 
одинаковую постоянную длину~$l_v$, а~дорожный просвет равен~$h_C$. 
Дорожное движение предполагается однородным на каждой полосе, т.\,е.\
     $v$, $\lambda$ и~$l_v$ не зависят от рассматриваемой полосы. 
     
     Согласно~[9], новые технологии V2V обеспечат\linebreak эффективный 
и~безопасный контроль над мобильностью ТС при минимально допустимом 
расстоянии между любыми двумя транспортными средствами $d_s\hm= t_s 
v$, где $t_s\hm=0{,}5$~с~--- минимальное\linebreak время, необходимое сис\-те\-мам 
автоматического управ\-ле\-ния для оценки условий движения в~режиме 
реального времени и~принятия превентивных мер.
     
     Для снижения капитальных затрат за счет увеличения расстояния 
между БС предполагается, что доля ТС~$P_E$ оснащена устройствами 
ретрансляции сигнала. Параметр~$P_E$ называется <<сте\-пенью внед\-ре\-ния 
технологии>>. Транспортные средства обору\-до\-ва\-ны двумя при\-емо-пе\-ре\-да\-ющи\-ми антеннами:
одна в~передней и~одна в~задней части ТС. Возможность подключения 
приемопередатчиков к~высокоскоростной внут\-рен\-ней шине со ско\-ростью 
\mbox{передачи} данных, достаточной для обработки {ретранслируемого} трафика, 
под\-тверж\-да\-ет\-ся последними разработками~[10].
     
     Транспортные средства, оснащенные средствами связи, нуждаются 
в~услугах eMBB, предостав\-ля\-емых через БС. Если прямое соединение 
невозможно из-за того, что ТС находится вне зоны действия ближайшей БС, 
они используют ретрансляцию для формирования моста, состоящего из 
одной или нескольких точек ретрансляции, в~качестве которых выступают 
ТС-ре\-транс\-ля\-то\-ры. Длина этого моста и~скорость передачи данных по 
нему существенно зависят от плотности развертывания БС, характеристик 
окружающей среды, влияющих на распространение сигнала, включая 
плотность ТС на дороге и~сценарий дорожного движения, а~также зоны 
размещения антенны.
     
     В работе рассмотрены следующие потенциальные варианты 
расположения антенны: на уровне бампера (0,3--0,4~м), на уровне двигателя 
(0,4--1,0~м) и~на лобовом стекле (1,0--1,5~м). Определим параметр~$\sigma$ 
для обозначения зоны размещения антенны, т.\,е.\ $\sigma\hm\in \{B,E,W\}$ 
соответственно.
     
     Значение отношения уровня сигнала к~уровню шума (Signal to 
Interference plus Noise Ratio, SINR) на ТС на расстоянии~$x$ от БС 
записывается в~виде
     \begin{equation*}
     S(x)= P_T G_A G_U \fr{x^{-\gamma}}{(N_0+I)L_A (f,x) 
L_B}\,,
    % \label{e1-ost}
     \end{equation*}
где $P_T$~--- излучаемая мощность; $G_A$ и~$G_U$~--- коэффициенты 
усиления на стороне приема и~передачи; $\gamma$~--- коэффициент потерь 
на пути сигнала; $N_0$~--- тепловой шум; $I$~--- помехи; $L_A(f,x)$~--- 
коэффициент затухания сигнала; $L_B$~--- потери, вызванные блокировкой 
или отражениями от других объектов.

\section{Математическая модель установления соединения 
для~механизма многозвеньевой ретрансляции сигнала}

     Построение модели начнем с~анализа прямых подключений между  
ТС-ис\-точ\-ни\-ком сигнала и~БС, а именно: с~определения максимальной 
дистанции прямого подключения и~скорости передачи данных. Чтобы 
вычислить данные параметры, используем набор схем модуляции 
и~кодирования, указанных в~стандарте IEEE 802.15.3d~[8].
     
     Следуя~[11], максимальное расстояние подключения  
ав\-то\-мо\-би\-ля-ис\-точ\-ни\-ка к~БС с~допустимой мощ\-ностью  
при\-ни\-ма\-емо\-го сигнала~$S$ можно записать как
     \begin{equation}
     d_\xi(S)= \left( \fr{P_T \sqrt[10]{10^{G_A+G_U}}} 
{\sqrt[10]{10^{N_0+S}}\,10^{2\log_{10} f_c-14{,}86+L_B}} 
\right)^{1/\gamma}\,,
     \label{e2-ost}
     \end{equation}
где $P_T$~--- мощность передачи; $G_A$ и~$G_U$~--- коэффициенты 
усиления на приеме и~передаче; $f_c$~--- 
несущая частота.

     Если следующий автомобиль не поддерживает ретрансляцию, 
расположение антенны на уровне бампера в~дополнение к~прямой видимости 
($L$) может использовать пути отражения под ТС ($U$) и~от соседнего 
автомобиля ($R$). Если антенна расположена на уровне двигателя или 
лобовом стекле, отражение сигнала под автомобилем недоступно ввиду 
геометрических ограничений, при этом отражение от соседнего автомобиля 
доступно. Для расположения на стекле существует также возможность 
прохождения сигнала сквозь стекло следующего автомобиля ($W$), но 
затухание сигнала при этом будет в~несколько раз выше затухания при 
прямой видимости. Для обозначения типов распространения сигнала 
используем параметр $\xi \hm\in \{L,U,R,W\}$.
     
     Следует обратить внимание, что минимальная требуемая мощность 
сигнала приема $S\hm= S_{\min}$ позволяет получить максимальное 
расстояние для связи между соседними автомобилями, далее обозначаемое 
как $d_\xi^{\max}$, $\xi\hm\in \{L,U,R,W\}$.
     
     Функции плотности вероятности расстояния до $i$-го соседнего 
ТС~[12] в~случае распределения цент\-ров ТС согласно процессу Пуассона 
подчиняются распределению Эрланга~[13]:
     \begin{equation*}
     f_i(x)=\fr{2(\pi\lambda)^i}{(i-1)!}\,x^{2i-1} e^{-\pi \lambda x^2} ,\enskip 
x>0\,,\enskip i=1,\ldots , N.
     %\label{e3-ost}
     \end{equation*}
     
     \textbf{Подключение в~зоне прямой видимости.} Связь в~зоне прямой 
видимости на расстоянии~$r$ возможна для всех рассматриваемых зон 
размещения антенны, если расстояние между вза\-и\-мо\-дей\-ст\-ву\-ющи\-ми ТС 
меньше, чем максимальное расстояние связи в~условиях прямой видимости 
$d^{\max}$ из~(\ref{e2-ost}), и~сле\-ду\-ющее ТС оборудовано при\-е\-мо-пе\-ре\-да\-ющей антенной. Таким образом, вероятность успешного подключения в~условиях прямой видимости на расстоянии~$r$ имеет вид:
     \begin{equation*}
     f_{H,\mathrm{LoS}}(r) =I\left( r<d^{\max}\right) P_E f_0(r)\,,
     \label{e4-ost}
     \end{equation*}
где $f_0(r)$ вычисляется как
\begin{equation*}
f_0(r)= 0{,}5 f_1(r) +0{,}5 f_2(r)\,.
%\label{e5-ost}
\end{equation*}
     
     \textbf{Отражение от соседнего автомобиля.} На\-пом\-ним, что боковое 
отражение доступно для всех рас\-смат\-ри\-ва\-емых зон размещения антенны. 
В~случае блокировки прямой видимости, т.\,е.\ когда следующее ТС не 
оборудовано приемо-передающей антенной, предполагаем, что 
расстояние~$d_0$ до блокирующего ТС (см.\ рис.~1) равновероятно 
подчиняется распределению расстояния либо до первого соседа, либо до 
второго. Соответственно, плот\-ность ве\-ро\-ят\-ности расстояния с~обходом 
блокирующего автомобиля определяется как свертка расстояний до двух 
ближайших соседей блокирующего ТС, т.\,е.
     \begin{equation*}
     f_B(r)=\left(f_1*f_2\right) (r)=\int\limits_0^\infty f_1(s) f_2(r-s)\,ds\,.
%     \label{e6-ost}
     \end{equation*}
     
     \begin{figure*}[b] %fig2
\vspace*{-6pt}
\begin{center}
   \mbox{%
\epsfxsize=118.5mm
\epsfbox{mac-2.eps}
}
\end{center}
\vspace*{-9pt}
\Caption{Схема отражения под автомобилем}
\end{figure*}
     
     В модели предполагается, что возможен обход не более одного 
блокирующего ТС через одно боковое отражение (см.\ рис.~1). При этом  
ав\-то\-мо\-биль-<<от\-ра\-жа\-тель>> должен находиться на таком участке 
соседней полосы, где сигнал от <<передатчика>> не будет заблокирован  
ав\-то\-мо\-би\-лем-<<бло\-ки\-ра\-то\-ром>>, находящимся слишком близ\-ко 
либо к~<<передатчику>>, либо к~<<ретранслятору>>. Вероятность 
незаблокированного отражения может быть выражена~как
     \begin{equation*}
     \delta_R(r)=\fr{w_v}{\eta(r)}= \fr{w_v r}{2(w-w_v)}\,,
    % \label{e7-ost}
     \end{equation*}
где $\eta(r)\hm= 2(w\hm- w_v)/r$~--- тангенс угла отклонения луча~$\beta$; 
$w_v$~--- половина ширины ТС.

     Вероятность ${\sf P}_R(r)$ наличия на соседней полосе ТС, 
обеспечивающего незаблокированное отражение сигнала, можно получить, 
используя свойство независимости процесса Пуассона. Предполагая, что 
точка центра отражателя на соседней полосе равномерно распределена на 
отрезке, соответствующем расстоянию между передатчиком 
и~ретранслятором~[14], искомую вероятность можно выразить в~следующем 
виде:
     \begin{equation}
     {\sf P}_R(r)=\begin{cases}
     \left( \fr{N_l}{2}-1\right) 
\fr{\delta_R(r)\lambda}{[l_v+2\Delta_\alpha(r)]^{-1}}\,, & \\
& \hspace*{-20mm}2d_s<r<d_R^{\max}\,;\\
     0 & \hspace*{-20mm}\mbox{в\ других\ случаях},
     \end{cases}
     \label{e8-ost}
     \end{equation}
где $\Delta_\alpha$~--- допустимое смещение соседнего~ТС.
     
     \textbf{Отражение под автомобилем.} В~отличие от зон лобового 
стекла и~двигателя, при размещении антенны в~зоне бампера становится 
возможной передача сигнала под ТС за счет отражения от дорожного 
полотна. Из-за свойства симметричности отраженного пути (рис.~2) 
минимальное расстояние между бампером блокиратора и~бампером 
с~антенной с~обеих сторон одинаково и~определяется выражением
     \begin{equation*}
     \delta_U(r)= \fr{r}{2}\left( 1-\fr{h_C}{h_B}\right)-\Delta_\alpha\,,
    % \label{e9-ost}
     \end{equation*}
где $h_C$~--- дорожный просвет; $h_B$~--- высота размещения антенны; 
$\Delta_\alpha$~--- смещение, допускаемое для бло\-ки\-ру\-юще\-го~ТС.



     Вероятность передачи сигнала под ТС можно найти 
аналогично~(\ref{e8-ost}):

\noindent
     \begin{multline}
     {\sf P}_{B,B}(r)={}\\
     {}= \begin{cases}
     \delta_U(r) \left( \fr{h_C}{h_B}+\fr{\Delta_\alpha}{r}\right), & 2d_s< r< 
d_U^{\max}\,;\\
     0 & \mbox{в\ других\ случаях}.
     \end{cases}
     \label{e10-ost}
     \end{multline} 
     
     \textbf{Прохождение сигнала сквозь стекло.} Наконец, найдем 
вероятность установления соединения через стекла бло\-ки\-ру\-юще\-го~ТС:
     \begin{equation*}
     {\sf P}_{B,W}(r)= \begin{cases}
     1, & 2d_s<r<d_W^{\max}\,;\\
     0 & \mbox{в\ других\ случаях}\,.
     \end{cases}
    % \label{e11-ost}
     \end{equation*}
     
     \textbf{Вероятность успешного многозвеньевого подключения.} 
С~учетом вышеизложенных возможностей установления соединения для 
механизма многозвеньевой ретрансляции сигнала вероятность успешного 
подключения на расстоянии~$r$, $f_{H,\sigma}(r)$, $\sigma\hm\in \{B,E,W\}$, 
определяется как сумма вероятностей двух событий: либо следующее ТС 
оснащено антенной и~можно использовать подключение в~условиях прямой 
видимости, либо следующее ТС на полосе не имеет передающего устройства 
с~ве\-ро\-ят\-ностью $1\hm- {\sf P}_E$ и~единственный вариант~--- обойти его 
с~по\-мощью описанных выше способов. Таким образом, 
расширяя~(\ref{e10-ost}), приходим к~сле\-ду\-ющей формуле для плот\-ности 
вероятности успешного подключения на расстоянии~$r$:
     \begin{multline}
     f_{H,\sigma}(r) = I\left( r< d^{\max}\right) {\sf P}_E f_0(r)+
     \left( {\sf P}_E-{\sf P}_E^2\right)\times{}\\
     {}\times \left( {\sf P}_R(r)+{\sf P}_{B,\sigma}(r) - {\sf P}_R(r) 
{\sf P}_{B,\sigma}(r)\right) f_B(r)\,.
     \label{e12-ost}
     \end{multline}
     
       \begin{table*}[b]\small
       \vspace*{-12pt}
  \begin{center}
  \Caption{Параметры дорожного движения}
  \vspace*{2ex}
  
  \begin{tabular}{|l|c|c|}
  \hline
\multicolumn{1}{|c|}{Сценарий}&\tabcolsep=0pt\begin{tabular}{c}Скорость\\ автомобиля $v$, км/ч \end{tabular}& 
\tabcolsep=0pt\begin{tabular}{c}Расстояние между\\ автомобилями $d_0$, м\end{tabular}\\
\hline
 Пробка& 20 &10\\
Нормальный городской трафик& 60&30\\
Шоссе& 120\hphantom{9} & 60\\
\hline
\end{tabular}
\end{center}
%\end{table*}
%\begin{table*}\small
%\vspace*{4pt}
  \begin{center}
  \Caption{Входные параметры системы}
  \vspace*{2ex}
  
  \begin{tabular}{|c|c|l|}
  \hline
Обозначение&Значение&\multicolumn{1}{c|}{Описание}\\
\hline
$l_v$&4,5~м&Длина автомобиля\\
$\lambda$&0,02~авт/м&Средняя плотность автомобилей\\
$h_A$&3~м&Высота БС\\
$h_B$&0,4/0,7/1,2~м&Высота антенны\\
$h_C$&0,2~м&Дорожный просвет автомобиля\\
$v$&25~м/с&Скорость автомобиля\\
$f_C$&304,2~ГГц&Несущая частота\\
$P_T$&$4{,}2\cdot 10^{-6}$~Вт &Мощность передатчика БС/автомобиля\\
$N_0$&$-$84~дБ&Мощность шума\\
 $S$&$-$56~дБ&Минимальный SINR\\
$G_A$, $G_U$&17,58~дБ&Коэффициенты усиления на приеме и~на передаче\\
$\gamma$&2,1&Коэффициент затухания сигнала\\
\hline
\end{tabular}
\end{center}
\end{table*}
     
     Определив вероятность многозвеньевого подключения  
в~(\ref{e12-ost}), можно перейти к~описанию показателей эффективности. 
В~част\-ности, вероятность того, что мост из~$n$ ТС-ре\-транс\-ля\-то\-ров 
обеспечит передачу данных на расстояние~$r$, может быть получена  
с~по\-мощью $n$-крат\-ной свертки~(\ref{e12-ost}):

\noindent
     \begin{equation*}
     p_{C,\sigma}(n,r) =\int\limits^\infty_{r-d^{\max}} f^{(n)}_{H,\sigma} 
(y)\,dy\,.
    % \label{e13-ost}
     \end{equation*}
     Если допустить возможность бесконечного чис\-ла 
 ТС-ре\-транс\-ля\-то\-ров в~соединении, вероятность того, что длина моста 
окажется не меньше расстояния~$r$ от ТС-ис\-точ\-ни\-ка сигнала до БС, 
может быть получена следующим образом:

\noindent
     \begin{multline*}
     p_{S,\sigma} (r)= \sum\limits^\infty_{n=1} p_{C,\sigma} (n,r) 
\prod\limits_{j=1}^{n-1} \left( 1-p_{C,\sigma}(j,r)\right)\,,\\
     \sigma\in \{B,E,W\}\,.
     %\label{e14-ost}
     \end{multline*}
     
     Скорость передачи по установленному мос\-ту~[14] определяется звеном 
с~наихудшими условиями канала. Тогда, имея пороговые значения~$s_i$, 
$i\hm=1,\ldots , N_C$, чувствительности приемника, соответствующие набору 
показателей качества канала $\{1,\ldots , N_C\}$, можно найти вероятность 
использования схемы кодирования~$i$ в~звене для каждого типа 
распространения сигнала $\xi\hm\in \{L,U,R,W\}$:
     \begin{equation*}
     q_{\xi,i} =\int\limits_{d_\xi(s_i)}^{d_\xi(s_{i+1})} f_{H,\xi}(r)\,dr\,,\enskip 
i=1,\ldots , N_C\,,
    % \label{e15-ost}
     \end{equation*}
где $d_\xi(S_{N_C+1})\hm=\infty$.

     Теперь определим среднюю скорость передачи данных по 
установленному мосту, включающему~$n$ ТС-ре\-транс\-ля\-то\-ров, 
используя биномиальное распределение

\noindent
     \begin{equation*}
     \rho_{\xi,n}=\sum\limits_{i=1}^{N_C} \omega_i \left( 
\sum\limits_{k=i}^{N_C} q_{\xi,k}\right)^n\,,\enskip \xi\in \{L,U,R,W\}\,,
     %\label{e16-ost}
     \end{equation*}
где $\omega_i$~--- спектральная эффективность канала согласно схеме~$i$.

     Тем не менее в~целях контроля качества соединения имеет смысл 
ограничивать максимальное число ТС-ре\-транс\-ля\-то\-ров некоторым 
заданным значением~$N$, $N\hm\geq 1$:
     \begin{multline*}
     \rho_{S,\sigma}(r)={}\\
     {}= \!\!\!\sum\limits_{\xi\in \{L,R,U,W\}}\!\!\!\!\!\!\! {\sf P}_\xi 
\sum\limits_{n=1}^N \rho_{\xi,n} p_{C,\sigma} (n,r) \prod\limits^n_{j=1} \left( 
1-p_{C,\sigma} (j,r)\right)\,,\\
     \sigma\in\{B,E,W\}\,.
   %  \label{e17-ost}
     \end{multline*}
     
     \vspace*{-18pt}

\section{Численный анализ}

\vspace*{-3pt}

     В качестве исходных данных для численного эксперимента 
рассматриваются три сценария дорожного движения: проб\-ка, нормальные 
условия движения в~городе и~скоростное шоссе. Сценарии различаются  
ско\-ростью ТС и~средним расстоянием между ними, как показано в~табл.~1. 
Остальные па\-ра\-мет\-ры сис\-те\-мы приведены в~табл.~2.

\begin{figure*} %fig3
\vspace*{1pt}
\begin{center}
   \mbox{%
\epsfxsize=84.218mm
\epsfbox{mac-3.eps}
}
\end{center}
\vspace*{-11pt}
\Caption{Средняя длина моста (залитые значки) и~скорость передачи (пустые значки) в~зависимости от 
плотности ТС: \textit{1}~--- бампер; \textit{2}~--- стекло; \textit{3}~--- 
двигатель}
\vspace*{-5pt}
\end{figure*}


\begin{figure*}[b] %fig4
  \vspace*{-6pt}
    \begin{minipage}[t]{80mm}
\begin{center}
   \mbox{%
\epsfxsize=78.898mm
\epsfbox{mac-4-a.eps}
}
\end{center}
\vspace*{-9pt}
  \Caption{Вероятность подключения в~зависимости от расстояния между БС и~степени 
внедрения технологии (черные кривые~--- $P_E\hm= 0{,}7$; серые кривые~--- $P_E\hm= 0{,}9$): (\textit{1}~--- бампер; 
  \textit{2}~--- стекло; \textit{3}~--- двигатель}
  \end{minipage}
  %\end{figure*}
   \hfill  
%  \begin{figure*} %fig5
  \vspace*{-6pt}
  \begin{minipage}[t]{80mm}
\begin{center}
   \mbox{%
\epsfxsize=79mm
\epsfbox{mac-4-b.eps}
}
\end{center}
\vspace*{-9pt}
  \Caption{Вероятность подключения в~зависимости от расстояния между БС и~сценариев дорожного движения
  (пунктирные кривые~--- пробка; штриховые~--- нормальный городской трафик; сплошные кривые~--- шоссе): 
  \textit{1}~--- бампер; \textit{2}~--- стекло; \textit{3}~--- двигатель}
    \end{minipage}
  \end{figure*}
  
  

     Начнем с~исследования основных зависимостей между средней длиной 
мос\-та и~ско\-ростью передачи данных, показанных на рис.~3, для различных 
зон размещения антенны в~за\-ви\-си\-мости от плот\-ности ТС
на полосе дорожного движения, где максимальное чис\-ло  
ТС-ре\-транс\-ля\-то\-ров~$N$установлено рав\-ным~10, а~степень внед\-ре\-ния 
технологии $P_E\hm= 0{,}7$.
     


     По результатам эксперимента мож\-но отметить, что средняя длина 
моста уменьшается по мере увеличения плот\-ности ТС на дороге для всех 
рас\-смот\-рен\-ных вариантов расположения антенны. Это объясняется тем, что 
с~увеличением плот\-ности уменьшается среднее расстояние между ТС. 
Обратный эффект наблюдается для ско\-рости передачи данных, и~это связано 
с~более короткими расстояниями меж\-ду автомобилями, а~следовательно, 
лучшим качеством канала.
     
     Анализируя влияние зоны размещения антенны, можно сделать вывод, 
что расположение у~лобового стекла обеспечивает 
б$\acute{\mbox{о}}$льшую сред\-нюю длину мос\-та почти для всех 
рассмотренных плотностей ТС. Однако этот выигрыш достигается за счет 
гораздо меньшей ско\-рости передачи данных. Обоснование данного 
наблюдения заключается в~том, что экраны в~виде заднего и~переднего стекол 
блокирующего автомобиля создают высокие потери при передаче сигнала.
     
     Одной из характеристик, отвечающих за гарантии производительности 
для пользователей коммерческих систем, служит до\-ступ\-ность подключения 
к~БС посредством моста. Рас\-смот\-рим \mbox{до\-ступ\-ность} подключения как 
функцию от рас\-сто\-яния между БС (Inter-site distance, ISD), показанную на 
рис.~4, для различных значений степени внед\-ре\-ния технологии 
и~нормальных условий городского трафика ($\lambda\hm= 1/30$). При 
достижении 1250~м наблюдается резкий спад, ха\-рак\-те\-ри\-зу\-ющий расстояние, 
на котором начинает работать механизм ре\-транс\-ля\-ции сигнала. Из всех 
рас\-смот\-рен\-ных зон размещения антенны лобовое стек\-ло показывает 
наилучшую до\-ступ\-ность~0,95 при степени внед\-ре\-ния технологии~0,9.
     
     Влияние условий дорожного движения на ве\-ро\-ят\-ность подключения 
для различных зон размещения антенны показано на рис.~5. Здесь 
можно заметить, что наихудший возможный сценарий~--- пробки, где не 
только все рас\-смот\-рен\-ные зоны приводят к~одному и~тому же значению 
параметра ISD, но и~связанное с~этим улучшение ISD незначительно. 
Причина в~том, что ТС расположены очень плот\-но, блокируя сразу несколько 
вариантов отражений, в~том чис\-ле под ТС. Тем не менее для нормального 
и~шоссейного сценариев наилучшая зона размещения антенны с~точ\-ки 
зрения длины установленного мос\-та~--- на лобовом стекле.

  
      

  
\section{Заключение}

     В работе предложена математическая модель для оценки 
производительности механизма мно\-го\-звень\-евой ретрансляции сигнала для 
V2X-со\-еди\-не\-ний в~частотных диапазонах субтерагерцевых волн при 
различных условиях распространения сигнала. Рассматриваемые метрики 
включают среднее расстояние подключения к~БС и~скорость передачи 
данных с~учетом возможной многозвеньевой ретрансляции сигнала, а~также 
критический параметр качества обслуживания~--- доступность сети.
     
     Представленные численные результаты для типичных сценариев 
предоставления услуг связи по технологии 6G с~использованием механизма 
многозвеньевой ретрансляции сигнала показывают, что можно 
рекомендовать размещение антенны в~зоне лобового стекла, которое, 
несмотря на более низкую скорость передачи данных, значительно менее 
чувствительно к~степени внедрения технологии и,~как правило, 
характеризуется гораздо б$\acute{\mbox{о}}$льшим покрытием сети.
     
{\small\frenchspacing
 {%\baselineskip=10.8pt
 %\addcontentsline{toc}{section}{References}
 \begin{thebibliography}{99}
\bibitem{1-ost}
\Au{Moltchanov D., Gaidamaka~Y., Ostrikova~D., Beschastnyi~V., Koucheryavy~Y., Samouylov~K.} 
Ergodic outage and capacity of terahertz systems under micromobility and blockage impairments~// 
IEEE T. Wirel. Commun., 2021. Vol.~21. Iss.~5. P.~3024--3039. doi: 
10.1109/ TWC.2021.3117583.
\bibitem{2-ost}
\Au{Moltchanov D., Beschastnyi~V., Ostrikova~D., Gaidamaka~Y., Koucheryavy~Y.} 
Uninterrupted connectivity time in THz systems under user micromobility and blockage~//  
Global Communications Conference Proceedings.~--- Piscataway, NJ, USA: 
IEEE, 2021. Art. 9685384. 6~p. doi: 10.1109/GLOBECOM46510.2021.9685384.
\bibitem{3-ost}
\Au{Gapeyenko M., Samuylov~A., Gerasimenko~M., Moltchanov~D., Singh~S., Akdeniz~M.\,R., 
Aryafar~E., Himayat~N., Andreev~S., Koucheryavy~Y.} On the temporal effects of mobile 
blockers in urban millimeter-wave cellular scenarios~// IEEE T. Veh. Technol., 
2017. Vol.~66. Iss.~11. P.~10124--10138. doi: 10.1109/TVT.2017.2754543.
\bibitem{4-ost}
\Au{Stepanov N.\,V., Moltchanov~D., Begishev~V., Turlikov~A., Koucheryavy~Y.} Statistical 
analysis and modeling of user micromobility for THz cellular communications~// IEEE T. 
Veh. Technol., 2021. Vol.~71. Iss.~1. P.~725--738. doi: 10.1109/TVT.2021.3124870.
\bibitem{5-ost}
\Au{Beschastnyi V., Ostrikova~D., Moltchanov~D., Gaidamaka~Y., Koucheryavy~Y., 
Samouylov~K.} Balancing latency and energy efficiency in mmWave 5G NR systems with 
multiconnectivity~// IEEE Commun. Lett., 2022. Vol.~26. Iss.~8. P.~1952--1956. doi: 
10.1109/LCOMM. 2022.3175663.
\bibitem{6-ost}
\Au{Petrov V., Moltchanov~D., Andreev~S., Heath~R.\,W.} Analysis of intelligent vehicular 
relaying in urban 5G\;+\;millimeter-wave cellular deployments~// Global Communications 
Conference Proceedings.~--- Piscataway, NJ, USA: IEEE, 2019. Art.~05946. 
6~p. doi: 10.48550/arXiv.1908.05946.
\bibitem{7-ost}
Study on integrated access and backhaul (Release 17): Technical Specification 38.874 
V17.0.0. 3GPP, 2020. {\sf  
https://www.3gpp.org/ftp/Specs/archive/38\_series/38.\linebreak 874/38874-g00.zip}.
\bibitem{8-ost}
\Au{Petrov V., Kurner~T., Hosako~I.} IEEE 802.15.3d: First standardization efforts for  
sub-terahertz band communications toward 6G~// IEEE Commun. Mag., 2020. 
Vol.~58. No.\,11. P.~28--33. doi: 10.1109/MCOM.001.2000273.
\bibitem{9-ost}
\Au{Ozpolat M., Bhargava~K., Kampert~E., Higgins~M.\,D.} Multi-lane urban mmWave V2V 
networks: A path loss behavior dependent coverage analysis~// Vehicular Communications, 
2021. Vol.~30. Art.~100348. 11~p. doi: 10.1016/ j.vehcom.2021.100348.
\bibitem{10-ost}
\Au{Wang J., Liu~J., Kato~N.} Networking and communications in autonomous driving: 
A~survey~// IEEE Commun. Surv. Tut., 2018. Vol.~21. No.\,2. P.~1243--1274. doi: 
10.1109/COMST.2018.2888904.
\bibitem{11-ost}
\Au{Eckhardt J.\,M., Petrov~V., Moltchanov~D., Koucheryavy~Y., K$\ddot{\mbox{u}}$rner~T.} 
Channel measurements and modeling for low-terahertz band vehicular communications~// IEEE 
J.~Sel. Area. Comm., 2021. Vol.~39. No.\,6. P.~1590--1603. doi: 
10.1109/JSAC.2021.3071843.
\bibitem{12-ost}
\Au{Moltchanov D.} Distance distributions in random networks~//  Ad Hoc Netw., 
2012. Vol.~10. P.~1146--1166. doi: 10.48550/arXiv.0804.4204. 
\bibitem{13-ost}
\Au{Basharin G., Gaidamaka~Y.\,V., Samouylov~K.\,E.} Mathematical theory of teletraffic and 
its application to the analysis of multiservice communication of next generation networks~// 
Autom. Control Comp.~S., 2013. Vol.~47. No.\,2. P.~62--69. doi: 
10.3103/S0146411613020028.
\bibitem{14-ost}
\Au{Kingman J.\,F.\,C.} Poisson processes.~--- Oxford studies in probability ser.~--- Claredon Press, 1993. 112~p. doi: 
10.1002/0470011815.B2A07042.
\end{thebibliography}

 }
 }

\end{multicols}

\vspace*{-6pt}

\hfill{\small\textit{Поступила в~редакцию 15.10.22}}

%\vspace*{8pt}

%\pagebreak

\newpage

\vspace*{-28pt}

%\hrule

%\vspace*{2pt}

%\hrule

%\vspace*{-2pt}

\def\tit{ON THE OPTIMAL ANTENNA DEPLOYMENT FOR~SUBTERAHERTZ V2X 
COMMUNICATIONS}


\def\titkol{On the optimal antenna deployment for~subterahertz V2X 
communications}


\def\aut{E.\,A.~Machnev$^1$, V.\,A.~Beschastnyi$^1$, D.\,Yu.~Ostrikova$^1$, 
Yu.\,V.~Gaidamaka$^{1,2}$, and~S.\,Ya.~Shorgin$^2$}

\def\autkol{E.\,A.~Machnev, V.\,A.~Beschastnyi, D.\,Yu.~Ostrikova, et al.} 
%Yu.\,V.~Gaidamaka$^{1,2}$, and~S.\,Ya.~Shorgin$^2$}

\titel{\tit}{\aut}{\autkol}{\titkol}

\vspace*{-8pt}


\noindent
    $^1$Peoples' Friendship University of Russia (RUDN University), 6~Miklukho-Maklaya Str., 
Moscow 117198, Russian\linebreak
$\hphantom{^1}$Federation
    
    \noindent
    $^2$Federal Research Center ``Computer Science and Сontrol'' of the Russian Academy of 
Sciences, 44-2~Vavilov\linebreak
$\hphantom{^1}$Str., Moscow 119333, Russian Federation


\def\leftfootline{\small{\textbf{\thepage}
\hfill INFORMATIKA I EE PRIMENENIYA~--- INFORMATICS AND
APPLICATIONS\ \ \ 2022\ \ \ volume~16\ \ \ issue\ 4}
}%
 \def\rightfootline{\small{INFORMATIKA I EE PRIMENENIYA~---
INFORMATICS AND APPLICATIONS\ \ \ 2022\ \ \ volume~16\ \ \ issue\ 4
\hfill \textbf{\thepage}}}

\vspace*{3pt} 
  
  
    \Abste{Subterahertz (sub-THz, 100--300~GHz) communication should provide huge data 
transfer rates in 6G systems. However, the coverage area of base stations (BS) will be very 
limited, since the signal is quite strongly attenuated from the distance and is also easily blocked 
by the presence of any objects in the signal path. Thus, the BS will need to be located too often 
which is a costly process. To reduce the deployment density of the BS, a~mechanism was 
proposed for relaying the signal using vehicles (V2V). This relaying method is characterized by 
various options for the location of the antenna on vehicles which raises the question of finding 
the optimal location. In this work, guided by the IEEE 802.15.3d specification and measurements 
of the signal propagation level at a~frequency of 300~GHz, the authors developed 
a~mathematical model for comparing multihop signal relay systems with different antenna 
locations. The authors consider the following quality of service indicators: coverage, BS 
availability, and data transfer rate. The results show that the windshield transmitter location has 
a~lower data rate but more coverage while the bumper and engine levels show similar 
performance. A~windshield location is recommended as it is less sensitive to the rate of 
technology integration and has a~larger coverage area.}
  
  \KWE{5G; New Radio; V2V; V2X; multihop communications}
  
 
  
  \DOI{10.14357/19922264220407} 

\vspace*{-8pt}

 \Ack
  \noindent
  The reported study was funded by the Russian Science Foundation, project number 22-29-00694 ({\sf 
https://rscf.ru/en/project/22-29-00694}). 


\vspace*{12pt}

  \begin{multicols}{2}

\renewcommand{\bibname}{\protect\rmfamily References}
%\renewcommand{\bibname}{\large\protect\rm References}

{\small\frenchspacing
 {%\baselineskip=10.8pt
 \addcontentsline{toc}{section}{References}
 \begin{thebibliography}{99}
\bibitem{1-ost-1}
  \Aue{Moltchanov, D., Y.~Gaidamaka, D.~Ostrikova, V.~Bes\-chast\-nyi, Y.~Koucheryavy, and 
K.~Samouylov.} 2021. Ergodic outage and capacity of terahertz systems under micromobility 
and blockage impairments. \textit{IEEE T. Wirel. Commun.} 21(5):3024--3039. 
doi: 10.1109/TWC. 2021.3117583.
\bibitem{2-ost-1}
  \Aue{Moltchanov, D., V.~Beschastnyi, D.~Ostrikova, Y.~Gai\-da\-ma\-ka, and Y.~Koucheryavy.} 
2021. Uninterrupted connectivity time in THz systems under user micromobility and blockage. 
\textit{Global Communications Conference Proceedings}. Piscataway, NJ: IEEE. 
9685384. 6~p. doi: 10.1109/GLOBECOM46510.2021.9685384.
\bibitem{3-ost-1}
  \Aue{Gapeyenko, M., A.~Samuylov, M.~Gerasimenko, D.~Mol\-tcha\-nov, S.~Singh, 
M.\,R.~Akdeniz, E.~Aryafar, N.~Himayat, S.~Andreev, and Y.~Koucheryavy.} 2017. On the 
temporal effects of mobile blockers in urban millimeter-wave cellular scenarios. \textit{IEEE 
T. Veh. Technol.} 66(11):10124--10138. doi: 10.1109/TVT.2017.2754543.
\bibitem{4-ost-1}
  \Aue{Stepanov, N.\,V., D.~Moltchanov, V.~Begishev, A.~Turlikov, and Y.~Koucheryavy.} 
2021. Statistical analysis and modeling of user micromobility for THz cellular communications. 
\textit{IEEE T. Veh. Technol.} 71(1):725--738. doi: 10.1109/TVT.2021.3124870.
\bibitem{5-ost-1}
  \Aue{Beschastnyi, V., D.~Ostrikova, D.~Moltchanov, Y.~Gai\-da\-ma\-ka, Y.~Koucheryavy, and 
K.~Samouylov.} 2022. Balancing latency and energy efficiency in mmWave 5G NR systems 
with multiconnectivity. \textit{IEEE Commun. Lett.} 26(8):1952--1956. doi: 
10.1109/LCOMM.2022.3175663.
\bibitem{6-ost-1}
\Aue{Petrov, V., D.~Moltchanov, S.~Andreev, and R.\,W.~Heath.} 2019. Analysis of intelligent 
vehicular relaying in urban 5G\;+\;millimeter-wave cellular deployments. \textit{Global 
Communications Conference Proceedings}. Piscataway, NJ: IEEE. 05946. 6~p. doi: 
10.48550/arXiv.1908.05946.
  
\bibitem{7-ost-1}
  3GPP. 2020. NR. Study on integrated access and backhaul (Release 17): Technical Specification 38.874 
V17.0.0. Available at: 
{\sf https://www.3gpp.org/ftp/Specs/\linebreak archive/38\_series/38.874/38874-g00.zip} (accessed 
November~28, 2022).
\bibitem{8-ost-1}
  \Aue{Petrov, V., T.~Kurner, and I.~Hosako.} 2020. IEEE 802.15.3d: First standardization 
efforts for sub-terahertz band communications toward 6G. \textit{IEEE Commun. 
Mag.} 58(11):28--33. doi: 10.1109/MCOM.001.2000273
\bibitem{9-ost-1}
  \Aue{Ozpolat, M., K.~Bhargava, E.~Kampert, and M.\,D.~Higgins.} 2021. Multi-lane urban 
mmwave V2V networks: A~path loss behavior dependent coverage analysis. \textit{Vehicular 
Communications} 30:100348. 11 p. doi: 10.1016/ j.vehcom.2021.100348.


\bibitem{10-ost-1}
  \Aue{Wang, J., J.~Liu, and N.~Kato.} 2018. Networking and communications in autonomous 
driving: A~survey. \textit{IEEE Commun. Surv. Tut.} 21(2):1243--1274. doi: 
10.1109/ COMST.2018.2888904.
\bibitem{11-ost-1}
\Aue{Eckhardt, J.\,M., V.~Petrov, D.~Moltchanov, Y.~Koucheryavy, and T.~Kurner.} 2021. 
Channel measurements and modeling for low-terahertz band vehicular communications. 
\textit{IEEE J.~Sel. Area. Comm.} 39(6):1590--1603. doi: 
10.1109/JSAC.2021.3071843.
  
\bibitem{12-ost-1}
\Aue{Moltchanov, D.} 2012. Distance distributions in random networks. \textit{Ad Hoc 
Netw.} 10(6):1146--1166. doi: 10.48550/arXiv.0804.4204. 
\bibitem{13-ost-1}
\Aue{Basharin, G.\,P., Yu.\,V.~Gaidamaka, and K.\,E.~Samouylov.} 2013. Mathematical theory 
of teletraffic and its application to the analysis of multiservice communication of next generation 
networks. \textit{Autom. Control Comp.~S.} 47(2):62--69. doi: 10.3103/S0146411613020028.
  
\bibitem{14ost-1}
  \Aue{Kingman, J.\,F.\,C.} 1993. \textit{Poisson processes}. Oxford studies in probability ser. Claredon Press. 112~p. 
doi: 10.1002/0470011815.B2A07042.

\end{thebibliography}

 }
 }

\end{multicols}

\vspace*{-6pt}

\hfill{\small\textit{Received October 15, 2022}}

\vspace*{-12pt}

  
  \Contr
  
  \vspace*{-3pt}
  
  \noindent
  \textbf{Machnev Egor A.} (b.\ 1996)~--- PhD student, Department of Applied Probability and 
Informatics, Peoples' Friendship University of Russia (RUDN University),  
6~Miklukho-Maklaya Str., Moscow 117198, Russian Federation; \mbox{1032143100@rudn.ru}
  
  \vspace*{3pt}
  
  \noindent
  \textbf{Beschastnyi Vitalii A.} (b.\ 1992)~--- Candidate of Science (PhD) in physics and 
mathematics, assistant professor, Department of Applied Probability and Informatics, Peoples' 
Friendship University of Russia (RUDN University), 6~Miklukho-Maklaya Str., Moscow 
117198, Russian Federation; \mbox{beschastnyy-va@rudn.ru}
  
  
  \vspace*{3pt}
  
  \noindent
  \textbf{Ostrikova Daria Yu.} (b.\ 1988)~--- Candidate of Science (PhD) in physics and 
mathematics, associate professor, Department of Applied Probability and Informatics, Peoples' 
Friendship University of Russia (RUDN University), 6~Miklukho-Maklaya Str., Moscow 
117198, Russian Federation; \mbox{ostrikova-dyu@rudn.ru}
  
  
  
  \vspace*{3pt}
  
  \noindent
  \textbf{Gaidamaka Yuliya V.} (b.\ 1971)~--- Doctor of Science in physics and mathematics, 
professor, Department of Applied Probability and Informatics, Peoples' Friendship University of 
Russia (RUDN University), 6~Miklukho-Maklaya Str., Moscow 117198, Russian Federation; 
senior scientist, Institute of Informatics Problems, Federal Research Center ``Computer Science 
and Control'' of the Russian Academy of Sciences, 44-2~Vavilov Str., Moscow 119333, Russian 
Federation; \mbox{gaydamaka-yuv@rudn.ru}
  
  
  \vspace*{3pt}
  
  \noindent
  \textbf{Shorgin Sergey Ya.} (b.\ 1952)~--- Doctor of Science in physics and mathematics, 
professor, principal scientist, Institute of Informatics Problems, Federal Research Center 
``Computer Science and Control'' of the Russian Academy of Sciences, 44-2~Vavilov Str., 
Moscow 119133, Russian Federation; \mbox{sshorgin@ipiran.ru}
  
\label{end\stat}

\renewcommand{\bibname}{\protect\rm Литература}    
  