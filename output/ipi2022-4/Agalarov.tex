\def\stat{agalarov}

\def\tit{ОПТИМАЛЬНОЕ УПРАВЛЕНИЕ ПОДКЛЮЧЕНИЕМ РЕЗЕРВНОГО ПРИБОРА 
В~СИСТЕМЕ МАССОВОГО ОБСЛУЖИВАНИЯ $G/M/1$$^*$}

\def\titkol{Оптимальное управление подключением резервного прибора 
в~системе массового обслуживания $G/M/1$}

\def\aut{Я.\,М.~Агаларов$^1$}

\def\autkol{Я.\,М.~Агаларов}

\titel{\tit}{\aut}{\autkol}{\titkol}

\index{Агаларов Я.\,М.}
\index{Agalarov Ya.\,M.}


{\renewcommand{\thefootnote}{\fnsymbol{footnote}} \footnotetext[1]
{Работа выполнена при поддержке РФФИ (проект 20-07-00804).}}


\renewcommand{\thefootnote}{\arabic{footnote}}
\footnotetext[1]{Федеральный исследовательский центр <<Информатика и~управление>> Российской академии наук,
\mbox{agglar@yandex.ru}}


%\vspace*{-12pt}




  \Abst{Рассматривается задача оптимизации режима подключения и~отключения 
резервного прибора в~сис\-те\-ме массового обслуживания (СМО) $G/M/1$ в~зависимости от длины 
очереди заявок на обслуживание. В~качестве целевой функции используется суммарный 
предельный средний доход сис\-те\-мы в~единицу времени, рассчитываемый как разность платы 
за обслуживание заявок и~суммарных потерь за задержку заявки в~очереди и~амортизацию 
резервного прибора. Рассмотрена также аналогичная задача для СМО
$G/M/1/r$ с~целевой функцией, учитывающей дополнительные потери дохода 
из-за отклонения заявок. Доказано свойство унимодальности функции дохода сис\-тем 
$G/M/1/r$ и~$G/M/1$ и~сформулированы необходимые и~достаточные условия 
существования конечного оптимального порога длины очереди, который служит критерием 
при подключении и~отключении резервного прибора. Предложен простой алгоритм 
оптимального управления порогом, требующий только наблюдения за величиной дохода 
сис\-темы.}
  
  \KW{система массового обслуживания; резервирование; управление; оптимизация}
  
  \DOI{10.14357/19922264220406} 
  
\vspace*{-3pt}


\vskip 10pt plus 9pt minus 6pt

\thispagestyle{headings}

\begin{multicols}{2}

\label{st\stat}

\section{Введение}

  Одним из способов повышения надежности и~эффективности сис\-те\-мы 
служит включение в~ее\linebreak состав (на стадии проектирования или эксплуатации) 
резервных элементов. Очень часто при исследовании реальных сис\-тем 
с~резервными элементами (как и~многих других сис\-тем), \mbox{предназначенных} для 
обслуживания случайных потоков заявок, на практике появляется 
необходимость применения математических моделей типа 
СМО, и~в~ряде пуб\-ли\-ка\-ций, посвященных этим моделям, 
рассмотрена задача управ\-ле\-ния включением и~отключением резервного 
прибора с~целью повышения качества обслуживания заявок при 
одновременном снижении затрат на ресурсы (минимизации потерь)  
СМО~[1--6]. Задача в~них сформулирована как задача поиска\linebreak оптимального 
порогового управления подключением или отключением резервного прибора 
в~зависимости от достижения длиной очереди или временем пребывания 
заявки соответствующих \mbox{пороговых} значений. Подобные задачи относятся\linebreak 
к~классу управ\-ля\-емых СМО, в~которых допускается 
воздействие на поведение сис\-те\-мы (например, путем введения управ\-ле\-ния 
некоторыми па\-ра\-мет\-ра\-ми) в~за\-ви\-си\-мости от текущего со\-сто\-яния параметров 
сис\-те\-мы с~\mbox{целью} повышения качества работы сис\-те\-мы. В~пуб\-ли\-ка\-ци\-ях, 
посвященных этому классу сис\-тем, основное внимание уделялось задачам 
поиска оптимальных дисциплин обслуживания заявок, и~мало встречается 
работ, по\-ста\-вив\-ших задачу по управ\-ле\-нию ин\-тен\-сив\-ностью обслуживания, 
в~том числе по принятию оптимальных решений при подключении резервных 
приборов. В~известных автору пуб\-ли\-ка\-ци\-ях задача по выбору оптимального 
режима подключения резервного прибора решена в~предположении 
пуассоновского входящего потока, пред\-ла\-га\-емые методы решения являются 
численными и~сложны для построения простых процедур оптимального  
управ\-ле\-ния. 
  
  Ниже приведены результаты исследования задачи оптимизации простой 
пороговой стратегии подключения и~отключения резервного прибора СМО 
типа $G/M/1$. Данная работа дополняет результаты, полученные в~\cite{6-aga}, 
и~использует метод решения, приведенный в~работах~\cite{7-aga, 8-aga}.

\vspace*{-12pt}

\section{Описание системы и~постановка задачи}

\vspace*{-4pt}

  Рассматривается однолинейная СМО с~неограниченной очередью 
и~резервным прибором, на которую поступает рекуррентный поток заявок 
с~функцией распределения вероятностей~$U(t)$ со средним значением $0\hm< 
\overline{v}\hm<\infty$. Заявки становятся в~очередь на обслуживание в~порядке 
поступления и~покидают сис\-те\-му только при завершении обслуживания, 
освободив одновременно прибор и~накопитель, а~на освободившийся прибор 
(если он не отключен) поступает первая в~очереди заявка из накопителя (если 
таковая есть). При подключении резервный прибор сразу начинает обслуживать 
первую в~очереди заявку. Подключение и~отключение прибора происходит 
в~зависимости от чис\-ла заявок в~сис\-те\-ме сле\-ду\-ющим образом: 
резервный прибор подключается к~обслуживанию (если он еще не подключен) в~момент поступления заявки в~сис\-те\-му, если чис\-ло заявок в~сис\-те\-ме 
(с~учетом поступившей) не меньше заранее заданного порогового значения 
$h_1\hm \geq 2$, и~резервный прибор отключается (если он подключен) от  
сис\-те\-мы, если в~момент освобождения любого из приборов в~сис\-те\-ме 
остается не более $(h_1\hm-1)$ заявок. При этом предполагается, что если 
уходит заявка с~основного прибора, то заявка с~резервного прибора переходит 
на основной. 

Время обслуживания заявки на основном и~резервном приборах 
распределено по экспоненциальному закону с~па\-ра\-мет\-ра\-ми  
$\mu_1\hm> 0$ и~$\mu_2\hm>0$ соответственно, $\overline{v}\hm< 
(\mu_1\hm+\mu_2)^{-1}$. В~дальнейшем в~каждом состоянии заявки в~очереди 
будем считать помеченными условными номерами 1, 2, \ldots\ в~порядке 
поступления, не меняющимися до перехода в~новое со\-сто\-яние. 
  
  Ниже в~качестве целевой функции для этой сис\-те\-мы используется 
интегральный показатель (который будем называть функцией дохода), 
учи\-ты\-ва\-ющий сле\-ду\-ющие стоимостные па\-ра\-метры: 
  \begin{description}
  \item[\,] $C_0\geq0$~--- плата, получаемая сис\-те\-мой, если поступившая 
заявка обслужена сис\-те\-мой (принята в~накопитель); 
  \item[\,] $C_1\geq 0$~--- потери в~единицу времени за ожидание заявки 
в~сис\-те\-ме;
  \item[\,] $C_2\geq 0$~--- потери на амортизацию резервного прибора 
в~единицу времени, когда он подключен к~сис\-те\-ме и~занят заявкой;
  \item[\,] $C_3\geq 0$~--- потери на амортизацию резервного прибора 
в~единицу времени, когда он простаивает ($C_3\hm\leq C_2$);
  \item[\,] $C_4\geq 0$~--- потери из-за отклонения поступившей заявки при 
переполнении накопителя. 
  \end{description}
  
  Обозначим:
  \begin{description}
  \item[\,] $D^{h_1}$~--- предельное среднее значение дохода сис\-те\-мы 
в~единицу времени при пороге~$h_1$;
  \item[\,] $g^{h_1}(a)$~--- предельное среднее значение суммарного дохода  
сис\-те\-мы, усредненного по чис\-лу поступивших заявок при пороге~$h_1$.
  \end{description}
  
  Ставится задача максимизации функции~$D^{h_1}$ по $h_1\hm>1$, которая 
эквивалентна задаче (так как $D^{h_1}\hm= g^{h_1}/\overline{v}$) найти 
оптимальный порог $h_1^*\hm> 1$ такой, что
  \begin{equation}
  \max\limits_{h_1>1} g^{h_1}=g^{h_1^*}\,.
  \label{e1-aga}
  \end{equation}
  
\section{Метод решения}

\vspace*{2pt}

  Рассмотрим задачу~(1) для СМО с~накопителем емкостью $h_2\hm= h_1\hm+ 
a$, $h_1\hm\geq 2$, $a\hm= const \hm\geq 0$, которая отличается от исходной 
только тем, что заявка, заставшая сис\-те\-му в~состоянии~$h_2$, теряется. 
  
  Отметим, что процесс обслуживания заявок в~данной сис\-те\-ме описывается 
цепью Mаркова, где переходы цепи определяются моментами поступления 
заявок, а~со\-сто\-яние цепи~--- это чис\-ло заявок, находящихся в~сис\-те\-ме 
в~момент поступления~\cite{9-aga}. 
  
  Введем обозначения:
  \begin{description}
  \item[\,] $\{\pi_i^{h_1},\ 0\leq i\leq h_2\}$~--- стационарное распределение 
вероятностей цепи при пороге~$h_1$ ($\pi_i^{h_1}$~--- стационарная 
вероятность того, что цепь находится в~со\-сто\-я\-нии~$i$);
  \item[\,] $d_i^{h_1}$~---  средний доход, получаемый сис\-те\-мой  
в~со\-сто\-я\-нии~$i$ при пороге~$h_1$;
  \item[\,] $g^{h_1}(a)$~--- предельное среднее значение суммарного дохода  
сис\-темы.
\end{description}

Из определения вложенной цепи Маркова следует
\begin{equation}
g^{h_1}(a)= \sum\limits_{i=0}^{h_2} \pi_i^{h_1} d_i^{h_1}\,.
\label{e2-aga}
\end{equation}
  
  Выпишем выражения для стационарных вероятностей~$\pi_i^{h_1}$, $i\hm= 
0,\ldots , h_2$.
   
  Обозначим: 
\begin{description}
\item
  $r^{h_1}_{i,m}$~--- вероятность того, что в~состоянии~$i$ будут обслужены 
ровно~$m$~заявок при условии, что в~этом состоянии чис\-ло подключенных 
приборов не изменится; 
\item $a^{h_1}_{i,m}$~--- ве\-ро\-ят\-ность того, что 
в~состоянии~$i$ будут обслужены ров\-но~$m$~заявок при условии, что 
в~этом состоянии чис\-ло подключенных приборов изменится.
\end{description}
  
  Справедливы формулы:
  \begin{align*}
  r^{h_1}_{i,m} &= \int\limits_0^\infty \fr{(\mu_i^{h_1} v)^m}{m!}\,e^{-\mu_i^{h_1}t}dU(v)\,;\\
  a^{h_1}_{i,m} &= \int\limits_0^\infty\! \int\limits_0^v 
\fr{\mu_i^{h_1}(\mu_i^{h_1} t)^{i-h_1+1}}{(i-h_1+1)!}\,e^{-\mu_i^{h_1}t}\times{}\\
&{}\times  
\fr{[\mu_1(v-t)]^{m-i+h_1-2}} {(m-i+h_1-2)!}\,e^{-\mu_1(v-t)}dtdU(v)\,,
\end{align*}
где
$$
  \mu_i^{h_1} = \begin{cases}
  \mu_1 &\ \mbox{при } 0\leq i\leq h_1-2\,;\\
  \mu_1+\mu_2 &\ \mbox{при } h_1-1\leq i\leq h_2\,.
  \end{cases}
 $$
  
  


\pagebreak

Составив сис\-те\-му уравнений равновесия для неизвестных стационарных 
вероятностей со\-сто\-яний
 рас\-смат\-ри\-ва\-емой цепи~$\pi_i^{h_1}$, $0\hm\leq 
i\hm\leq h_2$, и~заменив в~каждом уравнения~$\pi_j^{h_1}$ 
на~$R_j^{h_1}\pi_{h_2}^{h_1}$ и~исключив последовательно в~каж\-дом 
уравнении~$R_j^{h_1}$ с~наименьшим индексом, получим рекуррентные 
формулы для вы\-чис\-ле\-ния стационарных вероятностей состояний 
$\pi_j^{h_1}$, $0\hm\leq j\hm\leq h_2$:
  \begin{equation}
  \pi_j^{h_1}=\fr{R_j^{h_1}}{\sum\nolimits_{i=0}^{h_2} R_i^{h_1}}\,,\enskip 
0\leq j\leq h_2\,,
\label{e3-aga}
  \end{equation}
где
\begin{equation*}
%\left.
\begin{array}{l}
\qquad \qquad R_{h_2}^{h_1} =1\,;\qquad R^{h_1}_{h_2-1} =\fr{1-r^{h_1}_{h_2,0}}{r^{h_1}_{h_2,0}}\,;\\[6pt]
R^{h_1}_{j-1} =\fr{R_j^{h_1}(1-r^{h_1}_{j,1})}{r^{h_1}_{j-1,0}}-{}\\[12pt]
\hspace*{15mm}{}= 
\fr{\sum\nolimits^{h_2-1}_{i=j+1} R_i^{h_1} r^{h_1}_{j,i+1-j} +R^{h_1}_{h_2} r^{h_1}_{h_2,h_2-j}} {r^{h_1}_{j-1,0}}\,,\\[12pt]
 \hspace*{50mm}h_1\leq j\leq h_2-1\,;\\[6pt]
  R^{h_1}_{h_1-2} = \fr{R^{h_1}_{h_1-1} (1-r_{h_1-1,1})}{r^{h_1}_{h_1-2,0}} - {}\\[12pt]
\hspace*{11mm}{}-\fr{\sum\nolimits_{i=h_1}^{h_2-1} \! R_i^{h_1} r_{i,i+2-h_1} +R^{h_1}_{h_2} r_{h_2, h_2-h_1+1}} {r^{h_1}_{h_1-2,0}}\,;\\[12pt]
  R^{h_1}_{j-1} =\fr{1}{r^{h_1}_{j-1,0}} \left(
   %\fr   {
   \displaystyle R_j^{h_1}(1-r_{j,1}) - \sum\limits^{h_1-2}_{i=j+1} \! R_i^{h_1} r_{i,i+1-j} - {}\right.\\[12pt]
\hspace*{20mm}\left.  \displaystyle {}-\!\!\!\sum\limits^{h_2-1}_{i=h_1-1}\!\! R_i^{h_1} a_{i,i+1-j}\right)%}
- \fr{R^{h_1}_{h_2} a_{h_2, h_2-j}}{r^{h_1}_{j-1,0}}\,,\\[12pt]
 \hspace{55mm}1\leq j\leq h_1-2\,.
   \end{array}
%\!   \right\}\!
  % 
   \end{equation*}
  
  Из~(\ref{e3-aga}) следует соотношение:
  \begin{equation}
  \pi^{h_1+1}_{j+1} =\left( 1-\pi_0^{h_1+1}\right) \pi_j^{h_1}\,,\enskip 0\leq j\leq 
h_2\,.
  \label{e4-aga}
  \end{equation}
  
  В дальнейшем для краткости изложения будем пользоваться обозначениями:
  \begin{align*}
  r^{h_1}_{i,m}(v)&=\fr{\left(\mu_i^{h_1}v\right)^m}{m!}\,e^{-\mu_i^{h_1}t}\,;\\
  z_i^{h_1}(v,t,m) &=\fr{\mu_i^{h_1}\left(\mu_i^{h_1}t\right)^{i-h_1+1}}{(i-h_1+1)!}\,  
e^{-\mu_i^{h_1} t} \times{}\\
&\times \fr{[\mu_1(v-t)]^{m-i+h_1-2}}{(m-i+h_1-2)!}  \,e^{-\mu_1(v-t)}.
  \end{align*}
Получим явные выражения для параметров~$d_i^{h_1}$  
в~формуле~(\ref{e2-aga}) для~$g^{h_1}(a)$. 

\smallskip

\noindent
  \textbf{Лемма~1.} \textit{Среднее значение дохода, получаемого сис\-те\-мой 
при пороге~$h_1$ в~состоянии~$i$, равно}
  \begin{equation}
  d_i^{h_1}= \begin{cases}
  C_0-C_1 \overline{T}_i^{h_1} -C_2\left( \overline{v} -
\overline{T}^{h_1}_{\mathrm{пр1},i}\right) -{}\\[3pt]
& \hspace*{-40mm}{}- C_3\overline{T}^{h_1}_{\mathrm{пр1},i}\,,\ \ 0\leq i\leq h_2-1\,;\\[3pt]
  d^{h_1}_{h_2-1}-C_0 -C_4\,, & \hspace*{-12.2mm}i=h_2\,,
  \end{cases}
  \label{e5-aga}
  \end{equation}
\textit{где $\overline{T}_i^{h_1}$~--- среднее суммарное время задержки заявок в~очереди за время нахождения сис\-те\-мы в~состоянии}~$i$:

\noindent
\begin{multline}
\overline{T}_i^{h_1}={}\\[4pt]
\!\!{}=\!\begin{cases}
\displaystyle \fr{1}{\mu_i^{h_1}}\left[ i \sum\limits_{m=1}^{i+1} 
mr_{i,m}^{h_1}-\fr{1}{2}\sum\limits_{m=1}^{i+1} (m-1)mr^{h_1}_{i,m} +{}\right.\\[6pt]
\hspace*{5mm}\left.{}+\fr{1}{2}\,i(i+1) \displaystyle\sum\limits^\infty_{m=i+2} r^{h_1}_{i,m}\right], \ \ \ i\leq h_1-2\,;\\[6pt]
\displaystyle\fr{1}{\mu_i^{h_1}} \left[ i \sum\limits_{m=1}^{i-h_1+3} mr^{h_1}_{i,m} -{}\right.\\[6pt]
\hspace*{10mm}\left. {}-
\fr{1}{2}\sum\limits_{m=1}^{i-h_1+3}(m-1)mr^{h_1}_{i,m}\right]+{}\\[6pt]
\hspace*{10mm}\displaystyle{}+
   \sum\limits^i_{m=i-h_1+3} \int\limits_0^\infty\!\int\limits_0^v \left[ \fr{i-h_1+1}{2}\,t +{}\right.\\[6pt]
  \hspace*{15mm} {}+(m-i+h_1-1)t+{}\\[9pt]
\hspace*{15mm}\left.   {}+  \fr{(m-i+h_1-2)(v-t)}{2}\right]\times{}\\[6pt]
\hspace*{20mm}{}\times  x_i^{h_1}(v,t,m)\,dtdU(v)+{}\\[6pt]
   \hspace*{15mm}{}+
   \displaystyle\!\!\!\!\!\! \sum\limits^{i-1}_{m=i-h_1+3} \int\limits_0^\infty\!\!\int\limits_0^v (i-m)v\times{}\\[6pt]
   \hspace*{20mm}{}\times z_i^{h_1}(v,t,m)\,dtdU(v)+{}\\[6pt]
{}+
   \displaystyle\sum\limits^\infty_{m=i+1} \int\limits_0^\infty\!\int\limits_0^v \left[ 
   \fr{i-h_1+1}{2}\,t +(h_1-1)t +{}\right.\\[6pt]
\hspace*{10mm}\left.   {}+\fr{(h_1-2)(h_1-1)(v-t)} {2(m-i+h_1-1)}\right]\times{}\\[6pt]
\hspace*{10mm}{}\times z_i^{h_1}(v,t,m)\,dtdU(v),  \ \ \ i\geq h_1-1;
\end{cases}\!\!\!\!
\label{e6-aga}
   \end{multline}

\noindent
\textit{$\overline{T}^{h_1}_{\mathrm{пр1},i}$~--- среднее время простоя резервного 
прибора в~состоянии~$i$}:
\begin{multline}
\overline{T}^{h_1}_{\mathrm{пр1},i} ={}\\[3pt]
{}=
\begin{cases}
\overline{v}\,, &\hspace*{-20mm} i\leq h_1-2\,;\\
\displaystyle \overline{v}-\!\int\limits_0^\infty \!\!\int\limits_0^v \! t 
\fr{\mu_i^{h_1}(\mu_i^{h_1}t)^{i-h_1+1}} {(i-h_1+1)!}\,e^{-\mu_i^{h_1}t} dt 
dU(v)\,, &\\
&\hspace*{-20mm}i\geq h_1-1\,.
\end{cases}\!\!\!
\label{e7-aga}
\end{multline}
  
  \textit{Среднее значение времени простоя основного прибора 
$\overline{T}^{h_1}_{\mathrm{пр2},i}$ при пороге~$h_1$ в~состоянии~$i$ 
равно} 

\noindent
  \begin{multline}
  \overline{T}^{h_1}_{\mathrm{пр2},i} ={}\\[3pt]
  {}= \begin{cases}
 \displaystyle \sum\limits^\infty_{m=i+1} \int\limits_0^\infty \!\int\limits_0^v \left[ v-t-\fr{(h_1-1)(v-t)}{m-i+h_1-1}\right]\times{}&\\
&\hspace*{-55.5mm}{} \times   z_i^{h_1}(v,t,m)\,dtdU(v)\,, \ \  \ i\geq h_1-1\,;\\[3pt]
 \displaystyle \fr{1}{\mu_1}\sum\limits^\infty_{m=i+2} (m-i-1) r^{h_1}_{i,m}\,, &\hspace*{-17mm} i\leq  
h_1-2\,.
  \end{cases}
  \label{e8-aga}
  \end{multline}
  
  \noindent
  Д\,о\,к\,а\,з\,а\,т\,е\,л\,ь\,с\,т\,в\,о\,.\ \ Справедливость 
формулы~(\ref{e5-aga}) следует непосредственно из определений $d_i^{h_1}$, 
$\overline{T}^{h_1}_{\mathrm{пр1},i}$ и~из того, что при $0\hm\leq i\hm\leq 
h_2\hm-1$ сис\-те\-ма получает плату~$C_0$ и~нет потерь~$C_4$, а~при 
$i\hm=h_2$ сис\-те\-ма из-за отклонения поступившей заявки не получает плату 
и~несет потери~$C_4$. 
  
  Докажем равенство~(\ref{e6-aga}). Фиксируем состояние~$i$ и~$v$~--- 
время нахождения сис\-те\-мы в~состоянии~$i$. Найдем сначала выражения для 
суммарного среднего времени ожидания всех заявок в~очереди и~средние 
времена простоя резервного и~основного приборов за время~$v$.
  
  Пусть $T_l$, $l\geq 1$,~--- время ожидания $l$-й по номеру заявки за время 
нахождения сис\-те\-мы в~состоянии~$i$, если ее обслуживание на приборе 
завершилось или началось; $A_{m,t}$~--- событие, состоящее в~том, что за 
время~$t$ завершилось обслуживание ровно~$m$~заявок и~за это время число 
подключенных приборов не изменилось; $E_{m,t}$~--- событие, состоящее 
в~том, что за время~$v$ нахождения в~состоянии $i\hm\geq h_1\hm-1$ 
завершилось обслуживание ровно $m\hm> i\hm-h_1\hm+2$ заявок 
и~обслуживание $(i\hm- h_1\hm+2)$-й заявки завершилось в~момент $t\hm\in 
(0,v]$.
  
  Отметим, что математическое ожидание случайной величины (СВ)~$T_l$ при условии 
выполнения события~$A_{m,t}$ равно $lt/(m+1)$, $l\hm\leq m$ (см., 
например,~\cite{6-aga, 9-aga}). Использовав эту формулу, получим следующие 
выражения условных математических ожиданий для СВ~$T_l$:
  \begin{itemize}
  \item среднее СВ~$T_l$ при условии выполнения события~$A_{m,v}$ 
и~$l\hm\leq m$ равно
  \begin{multline*}
  \overline{T}_{l/m}=\fr{l}{m+1}\,v\ \mbox{при } i\leq h_1-2\\ \mbox{или } i\geq 
h_1-1,\ l\leq i-h_1+1\,;
\end{multline*}
  \item среднее СВ~$T_l$ при $l\hm\geq i\hm- h_1\hm+2\hm\geq 1$ 
и~условии, что произошло событие~$E_{m,t}$, равно
  $$
  \overline{T}_{l/m,t}=t+\fr{(l-i+h_1-2)(v-t)}{m-i+h_1-1}\,.
  $$
\end{itemize}
Следовательно, суммарное среднее время ожидания всех заявок в~сис\-те\-ме, 
обслуживание которых завершилось или началось в~состоянии~$i$, равно:
$$
\overline{T}_{\mathrm{обсл}/m}= 
\begin{cases}
\fr{mv}{2}\,, & \hspace*{-9mm}m\leq i,\ i\leq h_1-2 \ \mbox{или}\\[3pt]
& \hspace*{-9mm}i\geq h_1-1,\ m\leq i-h_1+2\,;\\[3pt]
\fr{i(i+1)v}{2(m+1)}\,, & \!\! m>i,\ i\leq h_1-2\,,
\end{cases}
$$
при условии что произошло~$A_{m,v}$;
\begin{multline*}
\overline{T}_{\mathrm{обсл}/m,t} = {}\\
{}=
\begin{cases}
mt-\fr{i-h_1+1}{2}\,t +\fr{(m-i+h_1-2)(v-t)}{2}\,, &\\
&\hspace*{-48mm} i\geq m> i-h_1+2,\ i\geq h_1\,;\\
\fr{i-h_1+1}{2}\,t +(h_1-1) t +{}&\\
&\hspace*{-72mm}{}+\fr{(h_1-2)(h_1-1)(v-t)}{2(m-i+h_1-1)}\,, \ \  m>i\,,\ i\geq h_1\,,
\end{cases}\hspace*{-7.47255pt}
\end{multline*}
при условии что произошло~$E_{m,t}$.
  
  Суммарное среднее время ожидания заявок в~сис\-те\-ме, обслуживание 
которых не началось в~состоянии~$i$, при условии что число уже 
обслуженных заявок $m\hm< i$, равно $\overline{T}_{\mathrm{необсл}/m} 
\hm= (i\hm- m)v$.
  
  Для $\overline{T}_i^{h_1}$, использовав полученные выше выражения для 
$\overline{T}_{\mathrm{обсл}/m}$, $\overline{T}_{\mathrm{обсл}/m,t}$, 
$\overline{T}_{\mathrm{необсл}/m}$ и~формулу полной вероятности на 
соответствующей группе несовместных событий, получим  
формулу~(\ref{e6-aga}).
  
  Рассуждая аналогично выводу формулы~(\ref{e6-aga}), получим 
формулы~(\ref{e7-aga}) и~(\ref{e8-aga}). 
  
  Приведем еще три леммы.
  
  \smallskip
  
  \noindent
  \textbf{Лемма~2.}\ \textit{Справедливы равенства}
  \begin{multline}
   d^{h_1+1}_{i+1} -d_i^{h_1} ={}\\[6pt]
  {}=\begin{cases}
\displaystyle  -\fr{C_1}{\mu_i^{h_1}}  \sum\limits_{m=1}^{i+1} mr^{h_1}_{i,m} -\fr{C_1(i+1)}{\mu_i^{h_1}} 
\sum\limits^\infty_{m=i+2} r^{h_1}_{i,m} \\[6pt]
  \hspace*{39mm}\mbox{\textit{при} } i\leq h_1-2\,;\\[6pt]
\displaystyle -C_1 \int\limits_0^\infty v \Bigg [ 
\sum\limits_{m=0}^{i-h_1+2} r^{h_1}_{i,m}(v) +{}\\[6pt]
\hspace*{4mm}\displaystyle {}+\!\!\!\! \sum\limits^i_{m=i-h_1+3} 
\int\limits_0^v z_i^{h_1}(v,t,m)\,dt \Bigg] dU(v)-{}\\[6pt]
 \hspace*{1mm}\displaystyle {}-C_1\!\! \sum\limits^\infty_{m=i+1} \int\limits^\infty_0 \!\int\limits_0^v \left[ t+ 
\fr{(h_1-1)(v-t)}{m-i+h_1-1}\right]\times{}\\
\hspace*{3mm}{}\times z_i^{h_1}(v,t,m)\,dt dU(v)\quad \mbox{\textit{при} } i\geq h_1-1.
 \end{cases}\!\!\!
  \label{e9-aga}
  \end{multline}


\smallskip

\noindent
  Д\,о\,к\,а\,з\,а\,т\,е\,л\,ь\,с\,т\,в\,о\,.\ \ Равенства~(\ref{e9-aga}) 
непосредственно следуют из~(\ref{e5-aga}), (\ref{e6-aga}) и~(\ref{e7-aga}) после 
несложных преобразований. 
  
  Введем обозначения:
  \begin{multline}
  w(h_1)= \sum\limits_{i=0}^{h_1-2} \pi_i^{h_1} \fr{1}{\mu_i^{h_1}} 
\sum\limits^\infty_{m=i+2} (m-i-1) r^{h_1}_{i,m}+{}\\
  {}+ 
\sum\limits_{i=h_1-1}^{h_2-1} \!\!\pi_i^{h_1}\! \sum\limits^\infty_{m=i+1} \int\limits_0^\infty\! \Bigg[ v- \!
\int\limits_0^v \! \left[ t+ \fr{(h_1-1)(v-t)}{m-i+h_1-1}\right]\times{}\\
{}\times  z_i^{h_1}(v,t,m)\,dt\Bigg] dU(v)+{}\\
{}+
\pi^{h_1}_{h_2}\sum\limits^\infty_{m=h_2} \int\limits_0^\infty \Bigg[ v-\!\int\limits_0^v \left[ t+\fr{(h_1-
1)(v-t)}{m-h_2+h_1}\right] \times{}\\
{}\times z_i^{h_1}(v,t,m)\,dt\Bigg] dU(v)\,;
\label{e10-aga}
\end{multline}

\vspace*{-12pt}

\noindent
\begin{multline}
f(h_1,a) =C_0-C_3\overline{v} -{}\\
{}- C_1\fr{1-\pi_0^{h_1+1}}{\pi_0^{h_1+1}}\left[ \overline{v} - w(h_1,a)\right].
\label{e11-aga}
\end{multline}

  \noindent
  \textbf{Лемма~3.}\ \textit{Справедливо соотношение}
  \begin{equation}
  g^{h_1}(a)- g^{h_1+1}(a) =\pi_0^{h_1+1} \left[ g^{h_1}(a)-f\left( 
h_1,a\right)\right].
  \label{e12-aga}
  \end{equation}

\noindent
Д\,о\,к\,а\,з\,а\,т\,е\,л\,ь\,с\,т\,в\,о\,.\ \  Использовав~(\ref{e2-aga}) 
и~(\ref{e4-aga}), находим:
\begin{multline*}
g^{h_1}(a) -g^{h_1+1}(a)=\sum\limits_{i=0}^{h_2-1} \pi_i^{h_1} d_i^{h_1} 
+\pi_{h_2}^{h_1} d_{h_2}^{h_1} -{}\\
{}- 
   \sum\limits_{i=0}^{h_2} \!\pi_i^{h_1+1} d_i^{h_1+1} -\pi_{h_2+1}^{h_1+1} d_{h_2+1}^{h_1+1} = \!\!
\sum\limits_{i=0}^{h_2-1} \!\!\pi_i^{h_1} d_i^{h_1} +\pi_{h_2}^{h_1} d_{h_2}^{h_1}-{}\\
   {}-
   \left( 1-\pi_0^{h_1+1}\right) \Bigg(  \sum\limits_{i=0}^{h_2-1} \pi_i^{h_1} d_{i+1}^{h_1+1} 
+\pi_{h_2}^{h_1} d_{h_2+1}^{h_1+1}\Bigg) -{}\\
{}- \pi_0^{h_1+1} d_0^{h_1+1}=\!
   \sum\limits_{i=0}^{h_2-1} \!\pi_i^{h_1} d_i^{h_1} +\pi^{h_1}_{h_2} d^{h_1}_{h_2} -\left( 1-
\pi_0^{h_1+1}\right) \times{}\\
   {}\times
   \Bigg(  \sum\limits_{i=0}^{h_2-1} \pi_i^{h_1} \left( d_i^{h_1} +d_{i+1}^{h_1+1} - d_i^{h_1}\right) + {}\\
   {}+
   \pi^{h_1}_{h_2} \left( q^{h_1}_{h_2} +d^{h_1+1}_{h_2} -d^{h_1}_{h_2-1}\right)\Bigg)-
   \pi_0^{h_1+1} d_0^{h_1+1} ={}\\
   {}= \pi_0^{h_1+1}\Bigg\{ 
   g^{h_1} -\fr{1-\pi_0^{h_1+1}}{\pi_0^{h_1+1}} \Bigg[ \sum\limits_{i=0}^{h_2-1} \pi_i^{h_1} \left( 
d_{i+1}^{h_1+1} -d_i^{h_1}\right)+{}\\
   {}+ \pi_{h_2}^{h_1} \left( d_{h_2}^{h_1+1} -d^{h_1}_{h_2-1} \right)\Bigg] -d_0^{h_1+1}\Bigg\}.
   \end{multline*}
   
    
  
  Подставив в~правую часть последнего равенства вместо $\left( 
d_{i+1}^{h_1+1}\hm- d_i^{h_1}\right)$ их выражения из  
формулы~(\ref{e9-aga}), после несложных преобразований получим

\noindent
  \begin{multline*}
  g^{h_1}(a)-g^{h_1+1}(a)=
   \pi_0^{h_1+1} \Bigg\{ 
   g^{h_1} +C_1\fr{1-\pi_0^{h_1+1}}{ \pi_0^{h_1+1}}\times{}\\[0.9pt]
   {}\times \Bigg[ \sum\limits_{i=0}^{h_1-2} \pi_i^{h_1} \Bigg[ 
   \overline{v}-\fr{1}{\mu_i^{h_1}} \sum\limits_{m=i+2}^\infty (m-i-1) r^{h_1}_{i,m}\Bigg]-{}\\[0.9pt]
   {}-
   \sum\limits_{i=h_1-1}^{h_2-1} \!\! \pi_i^{h_1} \Bigg[ \sum\limits^\infty_{m=i+1} \!\int\limits_0^\infty \Bigg[ 
 v- \!\int\limits_0^v \Bigg[ t+ \fr{(h_1-1)(v-t)}{m-i+h_1-1}\Bigg] \times{}\\[0.9pt]
 {}\times
 z_i^{h_1}(v,t,m)\,dt\Bigg] dU(v)\Bigg]-
 \pi^{h_1}_{h_2} \Bigg[ \sum\limits^\infty_{m=h_2} \int\limits_0^\infty \Bigg[ v-{}\\[0.9pt]
 {}- \int\limits_0^v \Bigg [ t+ 
\fr{(h_1-1)(v-t)}{m-i+h_1-1}\Bigg] z_i^{h_1}(v,t,m)\,dt\Bigg] dU(v)\Bigg]+{}\\[0.9pt]
   {}+
   \sum\limits_{i=h_1-1}^{h_2-1} \pi_i^{h_1} \int\limits_0^\infty v \Bigg[ \sum\limits_{m=0}^{i-h_1+2} 
r_{i,m}^{h_1}(v) +{}\\[0.9pt]
{}+
\sum\limits^i_{m=i-h_1+3} \int\limits_0^v z_i^{h_1}(v,t,m)\,dt+{}\\[0.9pt]
{}+
   \sum\limits^\infty_{m=i+1} \int\limits_0^v z_i^{h_1}(v,t,m)\,dt \Bigg] dU(v) +{}\\[0.9pt]
   {}+\pi^{h_1}_{h_2} 
\int\limits_0^\infty v \Bigg[ \sum\limits_{m=0}^{h_2-h_1+2} r^{h_1}_{i,m}(v)+{}\\[0.9pt]
   {}+
   \sum\limits_{m=h_2-h_1+3}^{h_2-1} \int\limits_0^v z_i^{h_1}(v,t,m)\,dt +{}\\[0.9pt]
   {}+
   \sum\limits^\infty_{m=h_2} \int\limits_0^v z_i^{h_1}(v,t,m)\,dt\Bigg] dU(v)\Bigg] - 
d_0^{h_1+1}\Bigg\}={}\\[0.9pt]
   {}=
   \pi_0^{h_1+1} \Bigg\{ g^{h_1} +C_1\fr{1-\pi_0^{h_1+1}}{\pi_0^{h_1+1}} \times{}\\[0.9pt]
   {}\times \Bigg[ \sum\limits_{i=0}^{h_1-2} 
\pi_i^{h_1}\Bigg[ \overline{v} -\fr{1}{\mu_i^{h_1}} \sum\limits_{m=i+2}^\infty (m-i-1) r^{h_1}_{i,m}\Bigg]+{}\\[0.9pt]
   {}+
   \sum\limits_{i=h_1-1}^{h_2-1} \pi_i^{h_1} \Bigg[ \overline{v} - \sum\limits^\infty_{m=i+1} 
\int\limits_0^\infty \!\int\limits_0^v \Bigg[ v-t-\fr{(h_1-1)(v-t)}{m-i+h_1-1}\Bigg]\times{}\\[0.9pt]
{}\times z_i^{h_1}(v,t,m)\,dt  dU(v)\Bigg] +{}\\[0.9pt]
   {}+
  \pi^{h_1}_{h_2} \Bigg[ \overline{v} - \sum\limits_{m=h_2}^\infty 
\int\limits_0^\infty\! \int\limits_0^v \Bigg[ v-t-\fr{(h_1-1)(v-t)}{m-i+h_1-1}\Bigg] \times{}\\[0.9pt]
{}\times 
z_i^{h_1}(v,t,m)\,dt dU(v)\Bigg] \Bigg]-  d_0^{h_1+1}\Bigg\}.
  \end{multline*}
  
\pagebreak
  
В обозначениях~(\ref{e10-aga}) и~(\ref{e11-aga}) полученное выше для 
разности $\left( g^{h_1}(a)\hm- g^{h_1+1}(a)\right)$ последнее равенство после 
замены $d_0^{h_1+1} \hm= C_0\hm- C_3\overline{v}$ примет вид~(\ref{e12-aga}).

\smallskip

\noindent
  \textbf{Лемма~4.}\ \textit{Функция $f(h_1,a)$ не возрастает по} 
$h_1\hm>0$.
  
  \smallskip
  
  \noindent
  Д\,о\,к\,а\,з\,а\,т\,е\,л\,ь\,с\,т\,в\,о\,.\ \  Как следует из~(\ref{e8-aga}) 
и~(\ref{e10-aga}), $w(h_1,a)$~--- среднее время простоя основного прибора. 
Функция $w(h_1,a)$ и~вероятность~$\pi_0^{h_1}$ убывают по~$h_1$ 
(доказательство аналогично их доказательству в~лемме~4 работы~\cite{6-aga}). 
Тогда из~(\ref{e11-aga}) следует доказательство леммы~4.
  
  Пусть $h_1^*$~--- решение задачи~(1), $f(h_1)\hm= \lim\nolimits_{a\to\infty} 
f(h_1,a)$ и~$g^{h_1}\hm= \lim\nolimits_{a\to\infty} g^{h_1}(a)$ (существование 
пределов следует из ограниченности функций). Справедлива следующая 
теорема. 



  
  \smallskip
  
  \noindent
  \textbf{Теорема~1.} \textit{Справедливы утверждения}: 
  \begin{enumerate}[(1)]
  \item \textit{существует 
порог $h_1^*\hm<\infty$, если $\mathrm{inf}_{h_1>1} f(h_1)\hm< \mathrm{sup}_{h_1>1} g^{h_1}$ 
и}~$C_1\hm>0$;
\item $h_1^*\hm=\infty$, \textit{если $\mathrm{inf}_{h_1>1} f(h_1) \hm\geq 
\mathrm{sup}_{h_1>1} g^{h_1}$ и~$C_1\hm>0$ или  $g^{h_1}\hm> f(h_1)$ при 
$h_1\hm=2$ и~$C_1\hm=0$}; 
\item $h_1^*\hm=2$, \textit{если $g^{h_1}\hm\leq f(h_1)$ 
при $h_1\hm=2$}; 
\item \textit{для существования $1\hm< h_1^* \hm<\infty$ необходимо 
и~достаточно выполнение условий $g^{h_1^*-1} \hm< g^{h_1^*}$ 
и}~$g^{h_1^*+1}\hm\leq g^{h_1^*}$.
\end{enumerate}
  
  \smallskip
  
  \noindent
  Д\,о\,к\,а\,з\,а\,т\,е\,л\,ь\,с\,т\,в\,о\,.\ \  Доказательство теоремы следует 
из справедливости утверждений: 
\begin{enumerate}[(1)]
\item существует порог $h_1^* \hm< \infty$, если 
$\mathrm{inf}_{h_1>1} f(h_1,a)\hm< \mathrm{sup}_{h_1>1} g^{h_1}(a)$ и~$C_1\hm>0$; 
\item $h_1^*\hm=\infty$, если $\mathrm{inf}_{h_1>1} f(h_1,a)\hm\geq \mathrm{sup}_{h_1>1} 
g^{h_1}(a)$ и~$C_1\hm>0$ или $g^{h_1}(a)\hm> f(h_1,a)$ при $h_1\hm=2$  
и~$C_1\hm=0$; 
\item $h_1^*\hm=2$, если $g^{h_1}(a)\hm\leq f(h_1,a)$ при 
$h_1\hm=2$; 
\item для существования $1\hm< h_1^*\hm< \infty$ необходимо 
и~достаточно выполнение условий $g^{h_1^*-1}(a)\hm< g^{h_1^*}(a)$ 
и~$g^{h_1^*+1}(a)\hm\leq g^{h_1^*}(a)$. 
\end{enumerate}
  
  Доказательство последних утверждений для краткости изложения не 
приводим, их доказательство полностью следует из лемм~2--4 и~теоремы из 
работы~\cite{10-aga} и~совпадает с~доказательством аналогичных утверждений 
теоремы~1 в~\cite{6-aga}. 
  
  На рисунке проиллюстрировано поведение функций $g^{h_1}(a)$, $f(h_1,a)$ 
при изменении значения порога~$h_1$.
Графики вычислены для следующих исходных 
данных: $U(t)\hm= f_1(1\hm- e^{-\lambda_1t})\hm+ f_2(1-e^{-\lambda_2t})$, $f_i > 
0$, $\lambda_i>0$, $i\hm= 1,2$; $f_1\hm+f_2\hm=1$; $f_1\hm= 0{,}3$; $f_2\hm= 
0{,}7$; $\lambda_1\hm= 2$; $\lambda_2\hm= 3$; $a\hm=20$; $C_0\hm=5$; 
$C_1\hm= 0{,}1$; $C_2\hm= 10$; $C_3\hm= 2$; $C_4\hm= 0$; $\mu_1\hm= 
2{,}5$; $\mu_2\hm= 1{,}25$. 
  
 { \begin{center}  %fig1
 \vspace*{-5pt}
     \mbox{%
\epsfxsize=79mm
\epsfbox{aga-1.eps}
}

\end{center}



\noindent
{\small Зависимости функций $g^{h_1}(a)/\overline{v}$~(\textit{1}) 
и~$f(h_1,a)/\overline{v}$~(\textit{2}) от порогового значения~$h_1$;
\textit{3}~--- оптимальное значение~$h_1^*$
}}

%\vspace*{6pt}
  
 
  
\section{Заключение}

  Отметим следующие основные выводы, вытекающие из полученных выше 
результатов:
  \begin{itemize}
\item целевая функция~(\ref{e2-aga}) при $a\hm<\infty$ для любых $0\hm< 
\mu_1\hm+\mu_2\hm<\infty$ и~при $a\hm=\infty$ для $0\hm< 
(\mu_1\hm+\mu_2)/\overline{v} \hm<1$ унимодальна по~$h_1$;
\item для существования конечного оптимального порога подключения 
резервного прибора необходимо и~достаточно, чтобы выполнялись условия: 
либо  $\mathrm{inf}_{h_1>1} f(h_1)\hm< \mathrm{sup}_{h_1>1} g^{h_1}$ и~$C_1\hm>0$, либо 
$g^2\hm\leq f(2)$;
\item если $\mathrm{inf}_{h_1>1} f(h_1)\hm\geq \mathrm{sup}_{h_1>1} g^{h_1}$ и~$C_1\hm>0$ 
или $g^2\hm> f(2)$ и~$C_1\hm=0$, то подключение резервного прибора 
уменьшит предельный доход СМО при любом пороге;
\item для поиска оптимального конечного порога~$h_1^*$ достаточно найти 
порог~$h_1$, при котором выполняются условия $g^{h_1-1}\hm< g^{h_1}$ 
и~$g^{h_1+1}\hm\leq g^{h_1}$;
\item на практике при устойчивой работе сис\-те\-мы для оптимизации режима 
использования резервного прибора следует применить сле\-ду\-ющее простое 
правило: если при увеличении порога среднее значение дохода 
увеличивается (или при уменьшении порога уменьшается доход), то значение 
порога необходимо увеличить, а~в~противном случае значение порога 
необходимо уменьшить. 
\end{itemize}

  Результаты данной работы могут быть использованы на этапах 
проектирования и~эксплуатации с~целю повышения экономической 
эффективности реальных сис\-тем, для которых в~качестве модели применима 
СМО типа $G/M/1$ ($G/M/1/r$) с~резервированием. 
  
{\small\frenchspacing
 {%\baselineskip=10.8pt
 %\addcontentsline{toc}{section}{References}
 \begin{thebibliography}{99}
\bibitem{1-aga}
\Au{Горцев А.\,М.} Сис\-те\-ма массового обслуживания с~произвольным 
числом резервных каналов и~гистерезисным управ\-ле\-ни\-ем включением 
и~выключением резервных каналов~// Автоматика и~телемеханика, 1977. 
Вып.~10. С.~30--37.
\bibitem{2-aga}
\Au{Дудин А.\,Н.} О~задаче оптимального управления многоскоростной 
сис\-те\-мой массового обслуживания~// Автоматика и~телемеханика, 1980. 
Вып.~9. С.~43--51. 
\bibitem{3-aga}
\Au{Рыков В.\,В.} Об условиях монотонности оптимальных политик 
управ\-ле\-ния сис\-те\-ма\-ми массового обслуживания~// Автоматика 
и~телемеханика, 1999. Вып.~9. С.~92--106.
\bibitem{4-aga}
\Au{Самочернова Е.\,С., Петров~Л.\,И.} Оптимизация сис\-те\-мы массового 
обслуживания с~однотипным резервным прибором~// Известия Томского 
политехнического университета, 2010. Т.~317. №\,5. С.~28--31.
\bibitem{5-aga}
\Au{Клименок В.\,И.} Многолинейная сис\-те\-ма массового обслуживания 
с~резервными приборами~// Ж.~Белорусского государственного 
университета. Математика. Информатика, 2019. №\,3. С.~57--70.
\bibitem{6-aga}
\Au{Агаларов Я.\,М.} Оптимизация порогового управ\-ле\-ния переключением 
ско\-рости обслуживания в~сис\-те\-ме массового обслуживания $G/M/1$~// 
Информатика и~её применения, 2022. Т.~16. Вып.~1. С.~73--81.
\bibitem{7-aga}
\Au{Агаларов Я.\,М., Ушаков~В.\,Г.} Об унимодальности функции дохода 
системы массового обслуживания типа $G/M/s$ с~управ\-ля\-емой очередью~// Информатика и~её 
применения, 2019. Т.~13. Вып.~1. С.~55--61.
\bibitem{8-aga}
\Au{Агаларов Я.\,М., Коновалов~М.\,Г.} Доказательство уни\-мо\-даль\-ности 
целевой функции в~задаче порогового управ\-ле\-ния нагрузкой на сервер~// 
Информатика и~её применения, 2019. Т.~13. Вып.~2. С.~2--6. 
\bibitem{9-aga}
\Au{Карлин С.} Основы тео\-рии случайных процессов~/ Пер. с~англ.~--- М.: 
Мир, 1971.  537~с. (\Au{Karlin~S.} A~first course in stochastic processes.~--- New York; 
London: Academic Press, 1968. 502~p.)
\bibitem{10-aga}
\Au{Агаларов Я.\,М.} Признак унимодальности це\-ло\-чис\-лен\-ной функции 
одной переменной~// Обозрение при\-клад\-ной и~промышленной математики, 
2019. Т.~26. Вып.~1. С.~65--66.
\end{thebibliography}

 }
 }

\end{multicols}

\vspace*{-6pt}

\hfill{\small\textit{Поступила в~редакцию 10.02.22}}

\vspace*{8pt}

%\pagebreak

%\newpage

%\vspace*{-28pt}

\hrule

\vspace*{2pt}

\hrule

%\vspace*{-2pt}

\def\tit{OPTIMAL CONTROL OF~A~QUEUE-LENGTH DEPENDENT ADDITIONAL SERVER 
IN~$\mathrm{GI}/M/1$ QUEUE}


\def\titkol{Optimal control of~a~queue-length dependent additional server 
in~$\mathrm{GI}/M/1$ queue}


\def\aut{Ya.\,M.~Agalarov}

\def\autkol{Ya.\,M.~Agalarov}

\titel{\tit}{\aut}{\autkol}{\titkol}

\vspace*{-8pt}


\noindent
Federal Research Center ``Computer Science and Control'' of the Russian Academy of Sciences, 
44-2~Vavilov Str., Moscow 119333, Russian Federation


\def\leftfootline{\small{\textbf{\thepage}
\hfill INFORMATIKA I EE PRIMENENIYA~--- INFORMATICS AND
APPLICATIONS\ \ \ 2022\ \ \ volume~16\ \ \ issue\ 4}
}%
 \def\rightfootline{\small{INFORMATIKA I EE PRIMENENIYA~---
INFORMATICS AND APPLICATIONS\ \ \ 2022\ \ \ volume~16\ \ \ issue\ 4
\hfill \textbf{\thepage}}}

\vspace*{3pt} 
  
   
   
   \Abste{Consideration is given to a~$\mathrm{GI}/M/1$ queue in which there is an additional server available for serving customers from the queue.
    The additional server can be turned on and off depending on the current queue length. 
    The long-run total cost per unit time, equal to the difference between the paid amount for service and the losses due to the waiting
     of customers and additional server depreciation, is being optimized. The case of finite queue capacity is also considered 
     in which the losses also account for lost customers. It is proved that the cost function considered is unimodal. 
     Necessary and sufficient conditions are given for the existence of the decision point (queue length) at which 
     application of the additional server is optimal. 
   A~simple algorithm for controlling the decision point, requiring only observations of the cost function value, is provided.}
   
   \KWE{queuing system; redundancy; management; optimization}
   
  
   
  \DOI{10.14357/19922264220406} 

\vspace*{-16pt}

 \Ack
 
 \vspace*{-2pt}
 
   \noindent
   The reported study was partly funded by RFBR, project number 20-07-00804.


%\vspace*{5pt}

  \begin{multicols}{2}

\renewcommand{\bibname}{\protect\rmfamily References}
%\renewcommand{\bibname}{\large\protect\rm References}

{\small\frenchspacing
 {%\baselineskip=10.8pt
 \addcontentsline{toc}{section}{References}
 \begin{thebibliography}{99}
\bibitem{1-aga-1}
  \Aue{Gortsev, A.\,M.} 1978. A~queueing 
system with an arbitrary number of stand-by channels and hysteresis control of their connection and 
disconnection. \textit{Automat. Rem. Contr.} 38(10):1451--1457.
\bibitem{2-aga-1}
  \Aue{Dudin, A.\,N.} 1981. On optimal control of a~multi-rate service system. \textit{Automat. Rem. Contr.} 41(9):1221--1228.
\bibitem{3-aga-1}
  \Aue{Rykov, V.\,V.} 1999. On monotonicity conditions for optimal policies for the control of queueing 
systems. \textit{Automat. Rem. Contr.} 60(9):1290--1301.
\bibitem{4-aga-1}
  \Aue{Samochernova, E.\,S., and L.\,I.~Petrov.} 2010. Op\-ti\-mi\-za\-tsiya sis\-te\-my mas\-so\-vo\-go ob\-slu\-zhi\-va\-niya 
  s~odnotipnym re\-zerv\-nym pri\-bo\-rom [Optimization of the queuing system with the same type of backup device]. 
\textit{Bulletin Tomsk Polytechnic University} 317(5):28--31.
\bibitem{5-aga-1}
  \Aue{Klimenok, V.\,I.} 2019. Mnogolineynaya sis\-te\-ma mas\-so\-vo\-go ob\-slu\-zhi\-va\-niya s~re\-zerv\-ny\-mi pri\-bo\-ra\-mi 
[Multi-server queueing system with reserve servers]. \textit{J.~Belarusian State University. Mathematics Informatics} 3:57--70.
\bibitem{6-aga-1}
  \Aue{Agalarov, Ya.\,M.} 2022. Op\-ti\-mi\-za\-tsiya po\-ro\-go\-vo\-go uprav\-le\-niya pe\-re\-klyu\-che\-ni\-em sko\-rosti 
ob\-slu\-zhi\-va\-niya v~sis\-te\-me massovogo ob\-slu\-zhi\-va\-niya $G/M/1$ [Optimization of the threshold service speed 
control in the $G/M/1$ queue]. \textit{Informatika i~ee primeneniya~--- Inform. Appl.} 16(1):73--81.

\bibitem{7-aga-1}
  \Aue{Agalarov, Ya.\,M., and V.\,G.~Ushakov.} 2019. Ob uni\-mo\-dal'\-nosti funk\-tsii do\-kho\-da 
  sis\-te\-my mas\-so\-vo\-go   obslu\-zhi\-va\-niya ti\-pa $G/M/s$ 
  s~uprav\-lya\-emoy oche\-red'\-yu [On the unimodality of the income function of a~type $G/M/s$ queueing system with 
controlled queue]. \textit{Informatika i~ee primeneniya~--- Inform. Appl.} 13(1):55--61. 
\bibitem{8-aga-1}
  \Aue{Agalarov, Ya.\,M., and M.\,G.~Konovalov.} 2019. Do\-ka\-za\-tel'\-st\-vo uni\-mo\-dal'\-nosti tse\-le\-voy funk\-tsii 
  v~za\-da\-che po\-ro\-go\-vo\-go uprav\-le\-niya na\-gruz\-koy na ser\-ver [Proof of the unimodality of the objective function in 
$M/M/N$ queue with threshold-based congestion control]. \textit{Informatika i~ee primeneniya~--- Inform. Appl.} 
13(2):\linebreak 2--6.
\bibitem{9-aga-1}
  \Aue{Karlin, S.} 1968. \textit{A~first course in stochastic processes}. New York, NY: Academic Press. 502~p.
\bibitem{10-aga-1}
  \Aue{Agalarov, Ya.\,M.} 2019. Pri\-znak uni\-mo\-dal'\-nosti tse\-lo\-chis\-len\-noy funk\-tsii od\-noy pe\-re\-men\-noy [A~sign of 
unimodality of an integer function of one variable]. \textit{Obozrenie pri\-klad\-noy i~promyshlennoy matematiki} 
[Surveys Applied Industrial Mathematics] 26(1):65--66.

\end{thebibliography}

 }
 }

\end{multicols}

\vspace*{-6pt}

\hfill{\small\textit{Received February 10, 2022}}
   
   \Contrl
   
   \noindent
   \textbf{Agalarov Yaver M.} (b.\ 1952)~--- Candidate of Science (PhD) in technology, associate 
professor, leading scientist, Institute of Informatics Problems, Federal Research Center ``Computer Science 
and Control'' of the Russian Academy of Sciences, 44-2~Vavilov Str., Moscow 119333, Russian Federation; 
\mbox{agglar@yandex.ru}
  

\label{end\stat}

\renewcommand{\bibname}{\protect\rm Литература}    
   