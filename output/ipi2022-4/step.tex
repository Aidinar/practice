\def\stat{stepchenkov}

\def\tit{ОЦЕНКА НАДЕЖНОСТИ СИНХРОННОГО И~САМОСИНХРОННОГО 
КОНВЕЙЕРОВ$^*$}

\def\titkol{Оценка надежности синхронного и~самосинхронного 
конвейеров}

\def\aut{И.\,А.~Соколов$^1$, Ю.\,А.~Степченков$^2$, Ю.\,Г.~Дьяченко$^3$, 
Ю.\,В.~Рождественский$^4$}

\def\autkol{И.\,А.~Соколов, Ю.\,А.~Степченков, Ю.\,Г.~Дьяченко, 
Ю.\,В.~Рождественский}

\titel{\tit}{\aut}{\autkol}{\titkol}

\index{Соколов И.\,А.}
\index{Степченков Ю.\,А.}
\index{Дьяченко Ю.\,Г.}
\index{Рождественский Ю.\,В.}
\index{Sokolov I.\,A.}
\index{Stepchenkov Yu.\,A.}
\index{Diachenko Yu.\,G.}
\index{Rogdestvenski Yu.\,V.}


{\renewcommand{\thefootnote}{\fnsymbol{footnote}} \footnotetext[1]
{Исследование выполнено в~рамках гранта Российского научного фонда (проект 22-19-00237).}}


\renewcommand{\thefootnote}{\arabic{footnote}}
\footnotetext[1]{Федеральный исследовательский центр <<Информатика и~управление>> Российской академии наук, 
\mbox{ISokolov@ipiran.ru}}
\footnotetext[2]{Федеральный исследовательский центр <<Информатика и~управление>> Российской академии наук, 
\mbox{YStepchenkov@ipiran.ru}}
\footnotetext[3]{Федеральный исследовательский центр <<Информатика и~управление>> Российской академии наук, 
\mbox{diaura@mail.ru}}
\footnotetext[4]{Федеральный исследовательский центр <<Информатика и~управление>> Российской академии наук, 
\mbox{YRogdest@ipiran.ru}}


\vspace*{-12pt}




\Abst{Самосинхронная (СС) схемотехника выступает альтернативой 
синхронным схемам. Самосинхронные схе\-мы обладают рядом преимуществ в~сравнении 
с~синхронными аналогами, но аппаратно избыточны. Статья исследует 
иммунность самосинхронных и~синхронных схем к~однократным 
кратковременным логическим сбоям (ЛС) с~учетом аппаратурной избыточности 
СС-схем. Самосинхронные схе\-мы за счет своей неотъемлемой части~--- индикаторной 
подсхемы~--- способны обнаружить ЛС, проявляющийся как 
инверсия состояния выхода логической ячейки схемы, и~приостановить 
функционирование схемы до его исчезновения. Тем самым СС-схе\-мы 
маскируют однократный ЛС и~предотвращают искажение 
данных. Использование модифицированного гистерезисного триггера для 
реализации разряда регистра ступени конвейера маскирует практически все 
ЛС в~комбинационной части (КЧ) ступени конвейера.  
DICE-по\-доб\-ная реализация этого триггера позволяет в~4~раза снизить 
чувствительность СС-ре\-гист\-ра к~ЛС внут\-ри него. 
Количественные оценки сбоеустойчивости показывают явное  
(в~2,5--9,4~раза) преимущество СС-кон\-вей\-ера схемы в~сравнении 
с~синхронным аналогом. }

\KW{самосинхронные схемы; логический сбой; сбоеустойчивость; конвейер; 
индикация; вероятностная оценка}

 \DOI{10.14357/19922264220401} 
  
\vspace*{-3pt}


\vskip 10pt plus 9pt minus 6pt

\thispagestyle{headings}

\begin{multicols}{2}

\label{st\stat}

\section{Введение}

  В современных условиях задача обеспечения\linebreak надежной работы циф\-ро\-вых 
схем выдвигается на первый план. Повышение тактовой частоты 
в~синхронной схемотехнике, все возрастающая функ\-цио\-наль\-ная слож\-ность 
интегральных мик\-ро\-схем создают предпосылки для повышения их 
чув\-ст\-ви\-тель\-ности к~не\-штат\-ным ситуациям~--- ЛС 
и~физическим отказам из-за внеш\-них и~внут\-рен\-них причин. Способ\-ность 
схемы к~маскированию ЛС и~отказов определяет уровень ее на\-деж\-ности. 
Практика показала, что отказы в~циф\-ро\-вых мик\-ро\-схе\-мах встречаются 
гораздо реже, чем ЛС~[1]. 
  
  Логический сбой проявляется как изменение логического уровня сигнала в~цепи, 
приводящее к~искажению результата обработки данных. Методы защиты от 
ЛС в~основном направлены на их маскирование и~используют 
корректирующие коды~[2], методы обнаружения и~изоляции~[3] и~некоторые 
другие подходы~[4, 5]. 
  
  Синхронные безызбыточные схемы не имеют встроенных средств 
контроля корректности переключений. В~отличие от них, асинхронные 
схемы используют зачатки контроля корректности выполняемых 
операций~[6, 7]. Однако их возможности по маскированию ЛС ограничены.
  
  Альтернативой синхронным и~асинхронным схемам выступают 
СС-схе\-мы~[8; 9; 10, p.~61--73]. Они характеризуются двухфазной 
дисциплиной работы и~обязательным под\-тверж\-де\-ни\-ем (индицированием) 
завершения переключения в~каж\-дую фазу. Благодаря этому СС-схе\-мы 
обладают естественной высокой сбоеустойчивостью~[11, 12]. Плата за эти 
преимущества~--- увеличение в~1,5--3,3~раза (в~зависимости от типа схемы) 
сложности реализации в~сравнении с~синхронными аналогами.
  
  При постоянной интенсивности событий, приводящих к~сбоям, число 
сбоев в~схеме обычно рас\-тет с~увеличением сложности схемы~[13]. Поэтому 
решаемая в~статье задача сравнительной количественной оценки 
устойчивости синхронных и~самосинхронных схем к~однократным ЛС с~учетом их 
аппаратной сложности и~особенностей функционирования особенно 
актуальна.

\vspace*{-7pt}
  
\section{Вероятность появления логического сбоя}

\vspace*{-2pt}
  
  В микросхемах, изготовленных по технологии комплементарный  
ме\-талл\,--\,ди\-элект\-рик\,--\,по\-лу\-про\-вод\-ник (КМДП), ЛС выражается 
во временном изменении потенциала некоторой цепи из-за 
индуцирования в~ней избыточных неравновесных носителей заряда. 
В~комбинационных схемах логический уровень сигнала восстанавливается 
спустя ка\-кое-то время~[14]. В~триггерных схемах сбой может запомниться, 
стать критичным. 
  
  При постоянной эксплуатационной плотности потока случайных 
событий~$\lambda_0$ (числа событий в~единицу времени на единицу 
площади), инициирующих однократные сбои, интенсивность сбоев~$\lambda$ в~схеме оценивается как сумма интенсивностей сбоев отдельных компонентов 
схемы~\cite[формула~(3.11)]{13-step}. Обычно в~качестве компонента схемы 
берется КМДП-тран\-зис\-тор~\cite{12-step}:
  $$
  \lambda= N \lambda_0 \alpha\,,
  $$
где $N$~--- число транзисторов в~схеме; $\alpha$~--- усредненная вероятность 
появления сбоя при поражении одного транзистора. Тогда отношение 
интенсивностей сбоев для СС-схе\-мы и~синхронного аналога
$$
K_\lambda= \fr{\lambda_S}{\lambda_{\mathrm{ST}}} =\fr{N_S \lambda_0 \alpha_S}{N_{\mathrm{ST}} 
\lambda_0 \alpha_{\mathrm{ST}}}= \fr{N_S \alpha_S}{N_{\mathrm{ST}} \alpha_{\mathrm{ST}}} =\fr{\alpha_S}{A_R \alpha_{\mathrm{ST}}}\,,
$$
где $\lambda_S$ и~$\lambda_{\mathrm{ST}}$~--- интенсивности сбоев в~синхронной 
и~самосинхронной схе\-мах; $N_S$ и~$N_{\mathrm{ST}}$~--- слож\-ность (чис\-ло  
КМДП-тран\-зис\-то\-ров) синхронной и~самосинхронной схем; $\alpha_S$ и~$\alpha_{\mathrm{ST}}$~--- 
вероятность сбоя при повреждении одного транзистора синхронной  
и~самосинхронной схем; $A_R\hm= N_{\mathrm{ST}}/N_S$~--- коэффициент аппаратной 
избыточности СС-схе\-мы в~сравнении с~синхронным аналогом.

  Типовой эффективный диаметр трека ядерной частицы, одного из 
источников сбоев, достигает 2--2,5~мкм~\cite{15-step}.  
В~КМДП-тех\-но\-ло\-гии с~проектными нормами~65~нм это, например, 
соответствует размеру схемы из нескольких транзисторов. Поэтому 
целесообразно рассматривать вероятность появления сбоя применительно 
к~логическим ячейкам схемы.
  
  Будем рассматривать цифровую схему как совокупность библиотечных 
ячеек, соединенных сигнальными цепями. Для упрощения будем считать, что 
появление сбоя в~любом месте принципиальной схемы ячейки вызывает 
инверсию уровня сигнала на ее выходе с~вероятностью~0,5. Пусть одно 
событие, порождающее сбой, может привести к~ЛС только в~одной ячейке 
схемы. Тогда интенсивность сбоев~$\lambda_i$ в~$i$-й ячейке схемы равна
  $$
  \lambda_i= \fr{S_i}{2S}\,\lambda_0 {\sf P}_{\mathrm{Э}}\,,
  $$
где $S_i$ и~$S$~--- площади топологии $i$-й ячейки и~всей схемы; 
${\sf P}_{\mathrm{Э}}$~--- вероятность индуцирования критического избыточного 
заряда одним сбойным со\-бы\-тием. 
  
  Однако не все ЛС в~ячейках схемы проявляются на ее выходах, поскольку 
при распространении по схеме они могут быть замаскированы. 
Интенсивность ЛС на выходах схемы
  \begin{multline}
  \lambda_{\mathrm{вых}} = \sum\limits_{i=1}^{M_{\mathrm{вых}}} 
\lambda_i +\sum\limits_{i=M_{\mathrm{вых}}+1}^M \left( \lambda_i {\sf P}_{{P},i}\right) ={}\\
  {}= \fr{\lambda_0 {\sf P}_{\mathrm{Э}}}{2S} \left( 
\sum\limits_{i=1}^{M_{\mathrm{вых}}} S_i 
+\sum^M_{i=M_{\mathrm{вых}}+1} \left( S_i {\sf P}_{P,i}\right)\right)\,,
  \label{e1-step}
  \end{multline}
где $M$~--- общее число ячеек в~схеме; $M_{\mathrm{вых}}$~--- чис\-ло 
выходных ячеек в~схеме; ${\sf P}_{P,i}$~--- вероятность того, что сбой 
на выходе $i$-й внутренней ячейки приведет к~сбою на выходе всей схемы. 

Оценим вероятность~${\sf P}_{P,i}$.

\section{Маскирование логического сбоя логикой схемы}

  Введем вероятность ${\sf P}_{P,ij}$ прохода ЛС с~выхода $i$-й 
ячейки на выход~$Y_j$ схемы. Пусть зависимость~$Y_j$ от внутренних и~внешних сигналов схемы описывается монотонной функцией $Y_j\hm = 
F_j(X_1, \ldots , X_M)$. С~учетом разложения~$Y_j$ по~$X_i$,
\begin{multline*}
Y_j = F_{0ij}\left (X_1, \ldots , X_{i-1}, X_{i+1}, \ldots, X_M\right) + {}\\
{}+F_{1ij}\left(X_1, \ldots , X_{i-1}, 
X_{i+1}, \ldots, X_M\right) X_i,
\end{multline*}
вероятность распространения ЛС от $X_i$ до~$Y_j$:
$$
{\sf P}_{P,ij} =\fr{N_{F_0ij} N_{F_1ij}}{2^{M_j-1}}\,,
$$
где $N_{F_0ij}$ и~$ N_{F_1ij}$~--- число комбинаций 
входов схемы, при которых $F_{0ij}\hm=0$ и~$F_{1ij}\hm=1$ соответственно; 
$M_j$~--- реальное число сигналов, от которых зависит~$Y_j$. Если 
$F_{0ij}\hm\equiv0$, то $N_{F_0ij}\hm=1$; если $F_{1ij}\hm\equiv1$, то 
$N_{F_1ij}\hm=1$. Вероятность появления на выходе~$Y_j$ ЛС, 
наблюдаемого на выходе одной из ячеек схемы,
$$
{\sf P}_{P,j}= \sum\limits^M_{i=1} \left( 
\fr{N_{F_0ij} N_{F_1ij}}{2^{M_j-1}} \prod\limits^{i-
1}_{k=1} \left( 1- \fr{N_{F_0kj} N_{F_1kj}}{2^{M_j-1}}\right)\right)\,,
$$
а вероятность появления сбоя, поразившего $i$-ю ячейку, на выходах схемы
\begin{equation}
{\sf P}_{P,i}= \sum\limits_{j=1}^{M_{\mathrm{вых}}} 
\fr{N_{F_0ij} N_{F_1ij}}{2^{M_j-1}}\,.
\label{e2-step}
\end{equation}

\begin{figure*}[b] %fig1
\vspace*{1pt}
\begin{center}
     \mbox{%
\epsfxsize=110.695mm
\epsfbox{ste-1.eps}
}

\vspace*{6pt}

\noindent
\small{Структура типового СС-конвейера}
\end{center}
\end{figure*}
  
  Аналогичные вероятности могут быть рассчитаны для любой пары цепей 
схемы, что позволяет получить оценки чувствительности схемы к~ЛС в~ее 
ячейках уже на этапе ее логического синтеза. 

%  \begin{table*}
{\small
  \begin{center}
  \begin{tabular}{|c|c|c|c|}
  \multicolumn{4}{c}{Парафазное кодирование информационного сигнала}\\
  \multicolumn{4}{c}{\ }\\[-6pt]
  \hline
№&Х&$\{$X, ХВ$\}$&Значение\\
\hline
1&---&00&Нулевой спейсер\\
2&0&01&Бит <<0>>\\
3&1&10&Бит <<1>>\\
4&---&11&Единичный спейсер\\
\hline
\end{tabular}
\end{center}
%\end{table*}
}

\vspace*{9pt}
  
  Сбой выхода СС-схе\-мы часто маскируется СС-дис\-цип\-ли\-ной за счет 
избыточного (парафазного~\cite{9-step}) кодирования данных и~двухфазной 
работы. При парафазном кодировании каждый синхронный сигнал~X 
заменяется парафазным сигналом $\{\mathrm{X, XB}\}$, как показано 
в~таб\-лице. 
  

  
  Парафазный сигнал формируется двумя согласованными логическими 
ячейками. Следовательно, однократный ЛС изменяет одну компоненту 
парафазного сигнала, делая его состояние не соответствующим текущей фазе 
схемы. Индикаторная подсхема обнаруживает это нарушение 
и~оста\-нав\-ли\-ва\-ет функционирование СС-схе\-мы до исчезновения сбоя.
  
  Практические СС-схе\-мы обычно реализуются в~виде конвейера для 
повышения про\-из\-во\-ди\-тель\-ности аналогично синхронным схемам. 
В~типовом СС-кон\-вей\-ере каждая ступень состоит из КЧ 
и~выходного регистра (ВР), как показано на рисунке. Их 
индикаторные подсхемы ИКЧ и~ИВР с~помощью гистерезисного триггера  
(Г-триг\-ге\-ра~\cite{9-step}, на рисунке обозначен буквой~H) разрешают 
предыдущей ступени конвейера переключаться в~следующую фазу работы.
  
  Анализ возможных ситуаций появления ЛС в~ступени конвейера 
показывает, что сбой в~ее КЧ в~худшем случае приводит 
лишь к~приостановке работы конвейера, но не портит обрабатываемые 
данные, если разряды регистра ступени реализованы сбоеустойчивой 
схемой~\cite[Fig.~10]{16-step}. Однако появление ЛС непосредственно 
в~выходном регистре с~вероятностью 0,25 вызывает искажение результата 
обработки данных или <<зависание>> конвейера. Использование в~разряде 
регистра \mbox{Г-триг}\-ге\-ра с~DICE-по\-доб\-ной  
реализацией~\cite[Fig.~12]{16-step} в~4~раза улучшает иммунность регистра.
  
Индикаторные подсхемы КЧ и~регистра ступени  
СС-кон\-вей\-ера вносят незначительный вклад в~чувствительность  
СС-кон\-вей\-ера к~ЛС. Критическая ситуация может возникнуть только в~том 
случае, если ЛС поражает выходной Г-триг\-гер, что в~многоразрядных  
СС-схе\-мах крайне маловероятно.
  

  
  Суммарная вероятность искажения данных\linebreak в~$m$-й ступени  
СС-кон\-вей\-ера с~$n_m$-раз\-ряд\-ным выходным регистром из-за 
ЛС зависит от площадей топологии КЧ
($S_{\mathrm{CP},m}$), Г-триг\-ге\-ров ($S_H$) и~индикаторного элемента ($S_{\mathrm{IE}}$) 
в~разрядах регистра и~индикаторных подсхем КЧ
($S_{\mathrm{CPI},m}$) и~ВР ($S_{\mathrm{ORI},m}$):
  \begin{multline}
  {\sf P}_{\mathrm{ST},m} ={}\\
  \!\!\!{}=\fr{0{,}25 n_m (2S_H+S_{\mathrm{IE}})}{S_{\mathrm{CP},m} +n_m(2S_H+S_{\mathrm{IE}}) 
+S_{\mathrm{CPI},m} +S_{\mathrm{ORI},m}}\,.\!
  \label{e3-step}
  \end{multline}
  
  Пусть КЧ содержит~$M$~ячеек и~ее сложность 
в~$K_{\mathrm{CP},m}$ раз превышает сложность регистра. Тогда  
формула~(\ref{e3-step}) преобразуется к~виду:
  $$
  {\sf P}_{\mathrm{ST},m} \approx \fr{0{,}55}{2{,}2 K_{\mathrm{CP},m}+2{,}7+ 0{,}25M/n_m}\,.
  $$
При реальных значениях $K_{\mathrm{CP},m}\hm = 4$ и~$M\hm = 8n_m$ вероятность 
критического сбоя ${\sf P}_{\mathrm{ST},m}\hm\approx 0{,}041$. При использовании  
DICE-по\-доб\-но\-го Г-триг\-ге\-ра она уменьшается до величины 
${\sf P}_{\mathrm{DICE},m}\hm\approx 0{,}011$, т.\,е.\ почти в~4~раза.

  В синхронном конвейере однократный ЛС, наблюдаемый 
в~любой части $m$-й ступени, не может замаскироваться дисциплиной 
работы. С~учетом формулы~(\ref{e2-step}) вероятность искажения данных 
из-за сбоя в~КЧ $m$-й ступени 
  $$
 {\sf P}_{S,m}=\sum\limits^{M_S}_{i=1} \left( \fr{S_i}{S} 
\sum\limits_{j=1}^{M_{S_{\mathrm{вых}}}} \fr{N_{F_0ij} 
N_{F_1ij}}{2^{M_j-1}}\right)\,,
  $$
где $M_S$~--- число ячеек в~КЧ $m$-й ступени 
синхронного конвейера; $M_{S_{\mathrm{вых}}}$~--- число ее выходов. 
В~первом приближении эту вероятность можно оценить как ${\sf P}_{S,m}\hm =  0{,}25$~\cite{16-step}.

\section{Сравнение сбоеустойчивости синхронных и~самосинхронных схем}

  При заданной эксплуатационной плотности потока сбойных 
событий~$\lambda_0$ интенсивность критических сбоев на выходах $m$-й 
ступени конвейера равна
  $$
  \lambda_{\mathrm{CF,ST}}\approx \lambda_{\mathrm{ST}} P_{\mathrm{ST},m} 
=\fr{0{,}55\lambda_{\mathrm{ST}}}{2{,}2 K_{\mathrm{CP},m} +2{,}7 + 0{,}25M/n_m}
  $$
для СС-конвейера и~$$
\lambda_{\mathrm{CF},S} \approx \lambda_S P_{S,m} =0{,}25 \lambda_S
$$
для синхронного конвейера. Следовательно, отношение интенсивностей 
критических сбоев для синхронного и~самосинхронного кон\-вей\-ера:
\begin{multline}
K_{\mathrm{CF}}= \fr{\lambda_{\mathrm{CF},S}}{\lambda_{\mathrm{CF,ST}}} ={}\\
{}=\fr{0{,}25 \lambda_S \left( 
2{,}2 K_{\mathrm{CP},m} +2{,}7 + 0{,}25M/n_m\right)} {0{,}55 \lambda_{\mathrm{ST}}}\,.
\label{e4-step}
\end{multline}
  
  Поскольку синхронная КЧ в~2~раза проще, чем  
в~СС-кон\-вей\-ере с~парафазным кодированием, и~индикаторной подсхемы 
нет, соотношение площадей топологий синхронного ($S_{S,P}$)  
и~самосинхронного кон\-вей\-еров ($S_{ST,P}$) 
  $$
  K_H= \fr{S_{\mathrm{ST},P}}{S_{S,P}}=2+\fr{4{,}25 M+8{,}5 n_m} {20n_m 
(K_{\mathrm{CP},m}+1)}\,.
  $$

Для реальных значений $K_{\mathrm{CP},m}\hm = 4$, $M\hm = 8n_m$ и~$n_m \hm= 32$ 
получается $K_H\hm = 2{,}4$. Тогда в~соответствии  
с~формулами~(\ref{e1-step}), (\ref{e2-step}) и~(\ref{e4-step}) сбоеустойчивость 
СС-кон\-вей\-ера оказывается лучше сбое\-устой\-чи\-вости синхронного 
конвейера в~2,5--9,4~раза\linebreak в~за\-ви\-си\-мости от схемы реализации разряда 
СС-ре\-гистра.

\section{Заключение}

  Самосинхронные схемы обладают естественной высокой иммунностью к~ЛС
   благодаря избыточному кодированию данных, двухфазной работе 
и~контролю окончания переключения в~каждую фазу. Анализ сбойных 
ситуаций показывает, что наиболее чувствительной к~однократным 
ЛС частью ступени СС-кон\-вей\-ера оказывается ВР. Однако реализация его разрядов на DICE-по\-доб\-ных  
Г-триг\-ге\-рах повышает его сбоеустойчивость в~4~раза.
  
  Реализация цифровой схемы в~виде СС-кон\-вей\-ера гарантирует 
повышение ее устойчивости к~однократным ЛС
в~2,5--9,4~раза в~сравнении с~синхронным конвейером, причем при 
появлении \mbox{критического} сбоя СС-кон\-вей\-ер останавливается и~своими 
индикаторными сигналами локализует место сбоя. Недостатком такой 
реализации является увеличенная в~2,4~раза сложность и,~соответственно, 
площадь схемы в~топологии.

\vspace*{-6pt}
  
{\small\frenchspacing
 {%\baselineskip=10.8pt
 %\addcontentsline{toc}{section}{References}
 \begin{thebibliography}{99}
 \bibitem{1-step}
 \Au{Викторова В.\,C., Лубков Н.\,В., Степанянц~А.\,С.} Анализ надежности 
отказоустойчивых управ\-ля\-ющих вы\-чис\-ли\-тель\-ных сис\-тем.~--- М.: ИПУ РАН, 2016. 117~c.
 \bibitem{2-step}
 \Au{Morelos-Zaragoza R.\,H.} The art of error correcting coding.~--- 2nd ed.~--- Hoboken, NJ, USA: Wiley, 
2006. 269~p.
 \bibitem{3-step}
 \Au{LaFrieda C., Manohar~R.} Fault detection and isolation techniques for quasi  
delay-insensitive circuits~// Conference (International) on Dependable Systems and Networks, 
2004. P.~41--50. doi: 10.1109/DSN.2004.1311875.
 \bibitem{4-step}
 \Au{Monnet Y., Renaudin M., Leveugle~R.} Hardening techniques against transient faults for 
asynchronous circuits~// 11th On-Line Testing Symposium (International) Proceedings.~--- IEEE, 
2005. P.~129--134. 
 \bibitem{5-step}
 \Au{Dug M., Krstic~M., Jokic~D.} Implementation and analysis of methods for error detection 
and correction on FPGA~// IFAC-PapersOnLine, 2018. Vol.~51. No.\,6. P.~348--353.
 
 \bibitem{7-step} %6
 \Au{Lodhi F.\,K., Hasan~S., Hasan~O., Awwad~F.} Low power soft error tolerant macro 
synchronous micro asynchronous (MSMA) pipeline~// Computer Society Annual 
Symposium on VLSI Proceedings.~--- Piscataway, NJ, USA: IEEE, 2014. P.~601--606. doi: 
10.1109/ISVLSI.2014.59.

\bibitem{6-step} %7
 \Au{Gkiokas C., Schoeberl M.\,A.} Fault-tolerant time-predictable processor~// Nordic 
Circuits and Systems Conference: NORCHIP and Symposium (International) of  
System-on-Chip Proceedings.~--- Piscataway, NJ, USA: IEEE, 2019. Art.\ 8906947. 6~p. doi: 
10.1109/NORCHIP. 2019.8906947.

 \bibitem{8-step}
 \Au{Muller D., Bartky~W.} A~theory of asynchronous circuits~// Symposium 
(International) on the Theory of Switching Proceedings.~--- Harvard University Press, 1959. Vol.~29. P.~204--243.
 \bibitem{9-step}
 \Au{Kishinevsky M., Kondratyev~A., Taubin~A., Varshavsky~V.} Concurrent hardware: The 
theory and practice of self-timed design.~--- New York, NY, USA: John Wiley \& Sons, 1994. 368~p.
 \bibitem{10-step}
 \Au{Smith S.\,C., Jia~Di.} Designing asynchronous circuits using NULL convention logic 
(NCL).~--- Synthesis lectures on digital circuits and systems ser.~--- Cham: Springer, 2009. %Vol.~4. No.\,1. P.~61--73.
96~p.
 \bibitem{11-step}
 \Au{Stepchenkov Y.\,A., Kamenskih~A.\,N., Diachenko~Y.\,G., Rogdestvenski~Y.\,V., 
Diachenko~D.\,Y.} Improvement of the natural self-timed circuit tolerance to short-term soft 
errors~// Advances Science Technology Engineering Systems~J., 2020. Vol.~5. No.\,2. P.~44--56. 
 \bibitem{12-step}
 \Au{Соколов И.\,А., Степченков~Ю.\,А., Рождественский~Ю.\,В., Дьяченко~Ю.\,Г.} 
Приближенная оценка эффективности синхронной и~самосинхронной методологий 
в~задачах проектирования сбоеустойчивых вы\-чис\-ли\-тель\-но-управ\-ля\-ющих  
сис\-тем~// Автоматика и~телемеханика, 2022. №\,2. С.~122--132.
 \bibitem{13-step}
 \Au{Dubrova E.} Fault-tolerant design.~--- New York, NY, USA: Springer, 2013. 185~p. doi: 
10.1007/978-1-4614-2113-9.
 \bibitem{14-step}
 \Au{Eaton P., Benedetto~J., Mavis~D., Avery~K., Sibley~M., Gadlage~M., Turflinger~T.} 
Single event transient pulse width measurements using a variable temporal latch technique~// 
IEEE T. Nucl. Sci., 2004. Vol.~51. No.\,6. P.~3365--3368. doi: 10.1109/TNS.2004.840020.
 \bibitem{15-step}
 \Au{Emeliyanov V.\,V., Vatuev A.\,S., Useinov~R.\,G.} Impact of heavy ion energy on charge 
yield in silicon dioxide~// IEEE T. Nucl. Sci., 2018. Vol.~65. No.\,8. P.~1496--1502.
 \bibitem{16-step}
 \Au{Stepchenkov Y., Diachenko~Y., Rogdestvenski~Y., Shikunov~Y., Diachenko~D.}  
Self-timed storage register soft error tolerance improvement~// East--West Design \& 
Test Symposium Proceedings.~--- Piscataway, NJ, USA: IEEE, 2021. P.~145--150.

\end{thebibliography}

 }
 }

\end{multicols}

\vspace*{-6pt}

\hfill{\small\textit{Поступила в~редакцию 20.06.22}}

\vspace*{8pt}

%\pagebreak

%\newpage

%\vspace*{-28pt}

\hrule

\vspace*{2pt}

\hrule

%\vspace*{-2pt}

\def\tit{SYNCHRONOUS AND SELF-TIMED PIPELINE'S RELIABILITY ESTIMATION}


\def\titkol{Synchronous and self-timed pipeline's reliability estimation}


\def\aut{I.\,A.~Sokolov, Yu.\,A.~Stepchenkov, Yu.\,G.~Diachenko, and~Yu.\,V.~Rogdestvenski}

\def\autkol{I.\,A.~Sokolov, Yu.\,A.~Stepchenkov, Yu.\,G.~Diachenko, and~Yu.\,V.~Rogdestvenski}

\titel{\tit}{\aut}{\autkol}{\titkol}

\vspace*{-8pt}


\noindent
Federal Research Center ``Computer Science and Control'' of the Russian Academy of Sciences, 44-2 
Vavilov Str., Moscow 119333, Russian Federation


\def\leftfootline{\small{\textbf{\thepage}
\hfill INFORMATIKA I EE PRIMENENIYA~--- INFORMATICS AND
APPLICATIONS\ \ \ 2022\ \ \ volume~16\ \ \ issue\ 4}
}%
 \def\rightfootline{\small{INFORMATIKA I EE PRIMENENIYA~---
INFORMATICS AND APPLICATIONS\ \ \ 2022\ \ \ volume~16\ \ \ issue\ 4
\hfill \textbf{\thepage}}}

\vspace*{3pt} 
 
 





\Abste{Self-timed (ST) circuitry is an alternative to synchronous circuits. Self-timed circuits have a number of 
advantages over their synchronous counterparts due to their redundant complexity. The article 
investigates the immunity of self-timed and synchronous circuits to single short-term soft error taking 
into account the hardware redundancy of ST circuits. Self-timed circuits, due to their indication subcircuit, are 
able to detect a~soft error which occurs as a~logical cell's output state inversion and suspend the 
operation of the circuit until the soft error disappears. Thus, ST circuits mask a~single soft error and 
prevent distortion of the data processing result. The use of a~modified hysteretic trigger, which prevents 
sticking in the antispacer, to implement a pipeline stage register bit masks almost all soft errors in the 
pipeline stage's combinational part. The DICE-like implementation of this trigger makes it possible to 
reduce the sensitivity of the ST register to the internal soft errors by a~factor of~4. Quantitative 
estimates of failure tolerance show a~clear (by 2.5--9.4~times) advantage of the ST pipeline in 
comparison with the synchronous counterpart.}

\KWE{self-timed circuit; soft error; failure tolerance; pipeline; indication; probabilistic estimate}




 \DOI{10.14357/19922264220401} 

\vspace*{-8pt}

\Ack
\noindent
The research was supported by the Russian Science Foundation (project No.\,22-19-00237).


%\vspace*{5pt}

  \begin{multicols}{2}

\renewcommand{\bibname}{\protect\rmfamily References}
%\renewcommand{\bibname}{\large\protect\rm References}

{\small\frenchspacing
 {%\baselineskip=10.8pt
 \addcontentsline{toc}{section}{References}
 \begin{thebibliography}{99}
\bibitem{1-step-1}
\Aue{Viktorova, V.\,S., N.\,V.~Lubkov, and A.\,S.~Stepanyants.} 2016. \textit{Analiz nadezhnosti 
otkazoustoychivykh upravlyayushchikh vychislitel'nykh sistem} [Analysis of fault-tolerant computing 
systems' reliability]. Moscow: IPU RAN. 117~p.
\bibitem{2-step-1}
\Aue{Morelos-Zaragoza, R.\,H.} 2006. \textit{The art of error correcting coding}. Hoboken, NJ: Wiley. 
269~p.
\bibitem{3-step-1}
\Aue{LaFrieda, C., and R.~Manohar.} 2004. Fault detection and isolation techniques for quasi  
delay-insensitive circuits. \textit{Conference (International) on Dependable Systems and Networks 
Proceedings}. 41--50. doi: 10.1109/ DSN.2004.1311875.
\bibitem{4-step-1}
\Aue{Monnet Y., M.~Renaudin, and R.~Leveugle.} 2005. Hardening techniques against transient faults 
for asynchronous circuits. \textit{11th On-Line Testing Symposium (International) Proceedings}. IEEE. 
129--134.
\bibitem{5-step-1}
\Aue{Dug, M., M.~Krstic, and D.~Jokic.} 2018. Implementation and analysis of methods for error 
detection and correction on FPGA. \textit{IFAC-PapersOnLine} 51(6):348--353.

\bibitem{7-step-1} %6
\Aue{Lodhi, F.\,K., S.~Hasan, O.~Hasan, and F.~Awwad.} 2014. Low power soft error tolerant macro 
synchronous micro asynchronous pipeline. \textit{Computer Society Annual Symposium on VLSI 
Proceedings}. Piscataway, NJ: IEEE. 601--606. doi: 10.1109/ISVLSI.2014.59.

\bibitem{6-step-1} %7
\Aue{Gkiokas, C., and M.\,A.~Schoeberl.} 2019. Fault-tolerant time-predictable processor. \textit{Nordic 
Circuits and Systems Conference: NORCHIP  Symposium (International) of System-on-Chip 
Proceedings}. Piscataway, NJ: IEEE. 8906947. 6~p. doi: 10.1109/NORCHIP.2019.8906947.

\bibitem{8-step-1}
\Aue{Muller, D.\,E., and W.\,C.~Bartky.} 1959. A theory of asynchronous circuits. \textit{Symposium 
(International) on the Theory of Switching Proceedings}. Harvard University Press. 29:204--243.
\bibitem{9-step-1}
\Aue{Kishinevsky, M., A.~Kondratyev, A.~Taubin, and V.~Varshavsky}. 1994. \textit{Concurrent 
hardware: The theory and practice of self-timed design}. New York, NY: John Wiley \& Sons. 368~p.
\bibitem{10-step-1}
\Aue{Smith, S.\,C., and J.~Di.} 2009. \textit{Designing asynchronous circuits using NULL convention logic 
(NCL)}. Synthesis lectures on digital circuits systems ser. Cham: Springer. 96~p.
%4(1):61--73.
\bibitem{11-step-1}
\Aue{Stepchenkov, Y.\,A, A.\,N.~Kamenskih, Y.\,G.~Diachenko, Y.\,V.~Rogdestvenski, and 
D.\,Y.~Diachenko.} 2020. Improvement of the natural self-timed circuit tolerance to shortterm soft errors. 
\textit{Advances Science Technology Engineering Systems~J.} 5(2):44--56.
\bibitem{12-step-1}
\Aue{Sokolov, I.\,A., Yu.\,A.~Stepchenkov, Yu.\,V.~Rozhdestvenskiy, and Yu.\,G.~Diachenko.} 2022. 
An approximate evaluation of the efficiency of synchronous and self-timed methodologies in designing 
failure-tolerant computing and control systems. \textit{Automat. Rem. Contr.} 83(2):264--272.
\bibitem{13-step-1}
\Aue{Dubrova, E.} 2013. \textit{Fault-tolerant design}. New York, NY: Springer. 185~p.
doi: 10.1007/978-1-4614-2113-9.
\bibitem{14-step-1}
\Aue{Eaton, P., J.~Benedetto, D.~Mavis, K.~Avery, M.~Sibley, M.~Gadlage, and T.~Turflinger.} 2004. 
Single event transient pulse width measurements using a~variable temporal latch technique. \textit{IEEE 
T. Nucl. Sci.} 51(6):3365--3368. doi: 10.1109/TNS.2004.840020.
\bibitem{15-step-1}
\Aue{Emeliyanov, V.\,V., A.\,S.~Vatuev, and R.\,G.~Useinov.} 2018. Impact of heavy ion energy on 
charge yield in silicon dioxide. \textit{IEEE T. Nucl. Sci.} 65(8):1496--1502.
\bibitem{16-step-1}
\Aue{Stepchenkov, Y., Y.~Diachenko, Y.~Rogdestvenski, Y.~Shi\-ku\-nov, and D.~Diachenko.} 2021. 
Self-timed storage register soft error tolerance improvement. \textit{East--West Design \& Test 
Symposium Proceedings}. Puscataway, NJ: IEEE. 145--150.
\end{thebibliography}

 }
 }

\end{multicols}

\vspace*{-6pt}

\hfill{\small\textit{Received June 20, 2022}}

\Contr

\noindent
\textbf{Sokolov Igor A.} (b.\ 1954)~--- Doctor of Science in technology, Academician of RAS, director, 
Federal Research Center ``Computer Science and Control'' of the Russian Academy of Sciences,  
44-2~Vavilov Str., Moscow 119133, Russian Federation; \mbox{isokolov@ipiran.ru}

\vspace*{3pt}

\noindent
\textbf{Stepchenkov Yuri A.} (b.\ 1951)~--- Candidate of Science (PhD) in technology, head of 
department, leading scientist, Federal Research Center ``Computer Science and Control'' of the Russian 
Academy of Sciences, 44-2~Vavilov Str., Moscow 119133, Russian Federation; 
\mbox{YStepchenkov@ipiran.ru}

\vspace*{3pt}

\noindent
\textbf{Diachenko Yuri G.} (b.\ 1958)~--- Candidate of Science (PhD) in technology, senior scientist, 
Federal Research Center ``Computer Science and Control'' of the Russian Academy of Sciences,  
44-2~Vavilov Str., Moscow 119133, Russian Federation; \mbox{diaura@mail.ru}

\vspace*{3pt}

\noindent
\textbf{Rogdestvenski Yuri V.} (b.\ 1952)~--- Candidate of Science (PhD) in technology, leading 
scientist, Federal Research Center ``Computer Science and Control''' of the Russian Academy of 
Sciences, 44-2~Vavilov Str., Moscow 119333, Russian Federation; \mbox{YRogdest@ipiran.ru}

 

\label{end\stat}

\renewcommand{\bibname}{\protect\rm Литература}    