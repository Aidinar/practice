\def\stat{grusho}

\def\tit{О БЕЗОПАСНОЙ АРХИТЕКТУРЕ ВЫЧИСЛИТЕЛЬНОЙ СИСТЕМЫ НА~ОСНОВЕ 
МИКРОСЕРВИСОВ}

\def\titkol{О безопасной архитектуре вычислительной системы на~основе 
микросервисов}

\def\aut{А.\,А.~Грушо$^1$, Н.\,А.~Грушо$^2$, М.\,И.~Забежайло$^3$, 
Д.\,В.~Смирнов$^4$,  Е.\,Е.~Тимонина$^5$,\\ С.\,Я.~Шоргин$^6$}

\def\autkol{А.\,А.~Грушо, Н.\,А.~Грушо, М.\,И.~Забежайло и~др.}
%$^3$,  Д.\,В.~Смирнов$^4$,  Е.\,Е.~Тимонина$^5$, С.\,Я.~Шоргин$^6$}

\titel{\tit}{\aut}{\autkol}{\titkol}

\index{Грушо А.\,А.}
\index{Грушо Н.\,А.}
\index{Забежайло М.\,И.}
\index{Смирнов Д.\,В.}
\index{Тимонина Е.\,Е.}
\index{Шоргин С.\,Я.}
\index{Grusho A.\,A.}
\index{Grusho N.\,A.}
\index{Zabezhailo M.\,I.}
\index{Smirnov D.\,V.}
\index{Timonina E.\,E.}
\index{Shorgin S.\,Ya.}


%{\renewcommand{\thefootnote}{\fnsymbol{footnote}} \footnotetext[1]
%{Работа выполнена при поддержке РФФИ (проект 20-07-00804).}}


\renewcommand{\thefootnote}{\arabic{footnote}}
\footnotetext[1]{Федеральный исследовательский центр <<Информатика и~управление>> Российской академии наук,
\mbox{grusho@yandex.ru}}
\footnotetext[2]{Федеральный исследовательский центр <<Информатика и~управление>> Российской академии наук,
\mbox{info@itake.ru}}
\footnotetext[3]{Федеральный исследовательский центр <<Информатика и~управление>> Российской академии наук,
\mbox{m.zabezhailo@yandex.ru}}
\footnotetext[4]{ПАО Сбербанк России, \mbox{dvlsmirnov@sberbank.ru}}
\footnotetext[5]{Федеральный исследовательский центр <<Информатика и~управление>> Российской академии наук,
\mbox{eltimon@yandex.ru}}
\footnotetext[6]{Федеральный исследовательский центр <<Информатика и~управление>> Российской академии наук,
\mbox{sshorgin@ipiran.ru}}


\vspace*{-10pt}




  \Abst{Рассматривается сетецентрическая система микросервисной архитектуры, в~которой 
для простоты все микросервисные вычислители (МВ) одинаковы. В~каждом 
МВ может произойти сбой или он может получить вредоносный код (ВК). Максимальное 
негативное воздействие на МВ заключается в~ошибке вычислений 
и~предоставлении потребителю неправильного результата. 
  Рассмотрены задачи выявления сбойных МВ и~МВ с~ВК. В~решении поставленных задач 
используются элементы обучения. Правильно решенные задачи (условия, исходные данные 
и~правильные ответы) накапливаются в~памяти управляющей сис\-те\-мы. Это означает, что 
есть возможность повторного запуска любой задачи с~уже известным правильным 
результатом. При этом в~статье используются идеи и~результаты авторов по обеспечению 
информационной безопас\-ности с~по\-мощью метаданных. 
  В~зависимости от предположений о~возможных действиях ВК построены два 
класса алгоритмов безопасных вычислений в~условиях его возможного воздействия на 
промежуточные результаты в~потоке решаемых задач. Второй класс алгоритмов работает 
в~предположении, что ВК может с~вероятностью~$p$ правильно вычислять 
решение текущей задачи и~с~вероятностью $1\hm-p$ вносить в~результат искажение. 
Рассмотрены три вида искажений, которые может вносить ВК и~которые 
позволяют находить истинное решение точно или с~некоторой, сколь угодно малой 
вероятностью ошибки.}
  
  \KW{информационная безопасность; безопасные вычисления в~условиях вредоносного кода; 
микросервисная архитектура}

 \DOI{10.14357/19922264220413} 
  
%\vspace*{-3pt}


\vskip 10pt plus 9pt minus 6pt

\thispagestyle{headings}

\begin{multicols}{2}

\label{st\stat}
   
  \section{Введение }
  
  \vspace*{-2pt}
  
  Безопасным архитектурам вычислительных сис\-тем посвящено много 
исследований. Основную часть научной литературы в~об\-ласти безопасных 
архитектур составляют руководства по внедрению различных наборов средств 
и~механизмов защиты в~распределенных информационных системах. Сюда 
относится большинство книг и~статей, таких как фундаментальные труды~[1--3] 
и~др. Один из основных принципов в~создании безопасных архитектур~--- 
изоляция бизнес-функций и~связанная с~ней изоляция информационной 
поддержки этих функций~[4, 5].
  
  В последнее время наблюдается изменение архитектур вычислительных систем, 
позволяющее полностью или частично изолировать поддержку различных 
функций информационных технологий (ИТ). Здесь необходимо выделить два 
базовых направления: виртуализация и~контейнеризация.
  
  Виртуализация базируется на создании множества виртуальных компьютеров на 
базе одного вычислительного устройства. Создание виртуальных компьютеров 
основано на гипервизоре, который, опираясь на основную операционную систему 
(ОС) вычислительного устройства, реализует виртуальные компьютеры и~их 
функционирование на том же сервере~[6, 7]. Виртуальные машины с~различными 
ОС работают на одном физическом сервере.

  Контейнеризация~--- это технология, которая позволяет запускать приложения 
изолированно от основной ОС. Приложение упаковывается в~специальный 
контейнер, внутри которого развернута среда, необходимая для работы. 
Контейнеры устанавливаются на физический сервер и~его ОС. Вместе с~тем 
проверена возможность удаленной контейнеризации, когда необходимые 
дополнения для создания облегченной ОС, необходимой для работы конкретного 
приложения, достаточно быст\-ро передаются по сети и~обеспечивают полную 
изоляцию работы контейнера и~приложения в~нем.
  
  Одной из перспективных тенденций пред\-став\-ля\-ет\-ся введение микросервисных 
архитектур, поз\-во\-ля\-ющих диверсифицировать и~ускорить решения многих задач 
и~ИТ~[8--10]. Среда для выполнения 
микросервисов состоит из сис\-те\-мы контейнеризованных вычислителей. В~этом 
случае каждый из\linebreak микросервисов, как правило, изолируется в~отдельный 
контейнер или небольшую группу контейнеров, до\-ступ\-ную по сети, 
контролируется и~управ\-ля\-ет\-ся сис\-те\-мой оркестрации или центром \mbox{управления}.
  
  В работе рассматривается сетецентрическая сис\-те\-ма микросервисной 
архитектуры, в~которой для простоты все МВ 
одинаковы. 
  
  Центр управления состоит из двух блоков~$I_1$ и~$I_2$. Блок~$I_1$ имеет 
прямые соединения со всеми МВ, случайно выбирает свободный МВ и~направляет 
ему задачу с~исходными данными. Каждый МВ может быть занят решением 
только одной задачи и~освобождается только после передачи блоку~$I_1$ 
результатов решения своей очередной задачи. После этого ему может быть 
назначена следующая задача. Блок~$I_2$ содержит метаданные о~порядке 
решения задач и~использовании полученных от~$I_1$ результатов их 
решения~[11--14]. 
  
  В каждом МВ может произойти сбой или МВ может получить ВК. Предположим, что максимальное негативное воздействие на МВ состоит 
в~ошибке вычислений и~предоставлении~$I_1$ неправильного результата. Задача 
состоит в~выявлении сбойных МВ и~МВ с~ВК. 
  
  Пусть правильно решенные задачи накапливаются в~памяти~$I_2$ (условия, 
исходные данные и~правильные ответы). Это означает, что имеется возможность 
повторного запуска любой задачи с~уже известным правильным результатом.
  
  \section{Алгоритм $A_1$ поиска сбойного микросервисного 
вычислителя}
  
  Рассмотрим следующую модель. Пусть имеются~$M$~МВ
 $\{ \beta_1, \beta_2, \ldots , \beta_M\}$, из которых~$m$~МВ 
оказываются сбойными или пораженными ВК. Система решает задачи $\alpha_1, 
\alpha_2, \ldots , \alpha_N$ в~потоке.
  
  Рассмотрим решение задачи в~ситуации, когда сбой МВ неустраним, а~ВК 
портит решение каждой задачи. Тогда можно воспользоваться результатами 
правильно решенных задач из памяти~$I_2$ сле\-ду\-ющим образом. 
  
  Алгоритм~$A_1$ берет любую правильно решенную задачу и~последовательно 
пробует решать ее каждым МВ. При этом считается, что сложность решения 
любой задачи равна~1. Если сбой или поражение ВК происходит только в~одном 
МВ, то среднее число шагов поиска сбоя равно $(M\hm-1)/2$. Если число сбойных 
МВ или МВ, содержащих ВК, неизвестно, то число шагов до выявления их всех 
алгоритмом~$A_1$ равно~$M$. После фильтрации всех сбойных МВ или МВ, 
зараженных ВК, система будет решать все задачи правильно.
  
  Предположим, что ВК способен маскироваться в~том смысле, что любую 
решаемую задачу он решает правильно с~вероятностью~$p$ независимо от других 
задач и~с~вероятностью $(1\hm-p)$ решает ее заведомо неправильно. Рассмотрим 
возможности применения алгоритма~$A_1$ для выявления ВК. 
  
  Для фиксированного ВМ пробуем решить известную задачу, но получение 
правильного ответа в~этом случае не гарантирует отсутствие ВК в~данном МВ. 
Если предполагать, что проведены~$n$~испытаний одного МВ, то вероятность 
невыявления ВК равна~$p^n$. Это означает, что вероятность выявления ВК 
за~$n$~шагов равна $1\hm- p^n$. 
  
  Эта вероятность может быть сделана близкой к~1. Однако может возникнуть 
эффект больших уклонений за счет потока задач. Вероятность выявления ВК на 
данной задаче равна $1\hm-p^n$, но вероятность выявления алгоритмом~$A_1$ 
всех МВ с~ВК равна\linebreak $(1\hm- p^n)^M$. Тогда вероятность пропустить хотя бы одну 
МВ с~ВК равна $1\hm- (1\hm-p^n)^M$, что может оказаться не совсем малой 
величиной, т.\,е.\ алгоритм~$A_1$ не гарантирует правильного решения всех 
задач в~потоке.
  
  \section{Алгоритм $A_2$ обеспечения защиты от~вредоносного кода 
в~микросервисном вычислителе}
  
  Можно построить алгоритм~$A_2$, гарантирующий правильное решение всех 
задач и~выявление ВК в~сформулированных выше предположениях при 
допущении случайного включения повреждения результата с~помощью ВК. Для 
этого введем понятие <<отказ>>, которое означает, что алгоритм~$A_2$ не может 
точно указать наличие ВК в~трех выбранных МВ. 
  
  Алгоритм~$A_2$ строится следующим образом. Каж\-дый раз, когда~$I_2$ 
передает для~$I_1$ очередную задачу, $I_1$ случайно и~независимо выбирает не 
одну МВ, а~сразу три различных МВ. Далее~$I_1$ передает всем трем МВ одну 
и~ту же задачу, которую он получил от~$I_2$. В~результате с~учетом 
воз\-мож\-ности наличия ВК получаем три результата решения одной и~той же 
задачи. Если все три результата одинаковые, то считаем, что получено правильное 
решение, а все три МВ не портят результат, даже если в~них есть~ВК. 
  
  Предположим, что на одном МВ получен ответ, не совпадающий с~двумя 
одинаковыми ответами  двух других МВ. Это означает, что правильный ответ дают 
два МВ, так как случайное совпадение двух ложных ответов, полученных 
независимо друг от друга, считаем невозможным событием. Одновременно 
выявляется МВ, в~котором сработал механизм ВК, порождающий неправильный 
ответ. 
{\looseness=1

}
  
  Рассмотрим случай, когда все три ответа на одну и~ту же поставленную задачу 
различны. Этот случай соответствует понятию отказ, так как в~этом случае 
невозможно выделить правильный ответ, даже если он существует в~полученных 
результатах. Возможен также случай, когда сработал ВК во всех трех МВ. 
В~случае получения отказа алгоритм~$A_2$ повторно запускает независимый 
выбор трех новых МВ и~поручает им заново решать ту же задачу. Если опять 
получены разные результаты, то $A_2$ сравнивает их с~предыдущей тройкой. 
Если получены два одинаковых результата, по одному в~первой и~второй тройке, то 
этот результат считается правильным, а~все несовпадающие с~ним определяют 
МВ, зараженные ВК. Если одинаковых результатов нет, то опять выбирается новая 
тройка МВ и~т.\,д.
{\looseness=1

}
  
  Отметим, что в~алгоритме~$A_2$ выявление МВ с~ВК происходит 
непосредственно в~процессе решения потока задач и~допускает безопасную работу 
в~условиях наличия небезопасных МВ. При этом не ставится задача выявления 
всех зараженных~МВ.
{\looseness=1

}
  
  \section{Вероятностный анализ алгоритма~$A_2$}
  
  Еще раз отметим, что в~основе эффективной работы алгоритма~$A_2$ лежит 
предположение о~малой вероятности случайного совпадения неправильных 
ответов, порожденных ВК, между собой и~между неправильными и~настоящими 
ответами.
  
  В сделанных предположениях найдем оценки отказов. Случайность выбора МВ 
для решения конкретной задачи позволяет допустить гипергеометрическое 
распределение вариантов появления МВ. Для двух МВ, содержащих ВК, которые 
породили две ошибки, и~одном правильном решении получаем следующую 
вероятность отказа:
  $$
{\sf  P}_0(2)=\left(1-p\right)^2 \fr{(M-m)\begin{pmatrix} m\\ 2\end{pmatrix} }{\begin{pmatrix} 
M\\ 3\end{pmatrix}}+p(1-p)^2 \fr{\begin{pmatrix} m \\ 3\end{pmatrix}} 
{\begin{pmatrix} M\\ 3\end{pmatrix}}\,.
  $$
    Если все три МВ дали неправильные результаты, то вероятность этого события 
равна
  $$
  {\sf P}_0(3)=(1-p)^3 \fr{\begin{pmatrix} m\\ 3\end{pmatrix}}{\begin{pmatrix} M\\ 
3\end{pmatrix}}\,.
  $$
   Тогда суммарная вероятность отказа при решении данной задачи составит
  $$
  {\sf P}_0={\sf P}_0(2)+{\sf P}_0(3)\,.
  $$
    Среднее число отказов в~потоке задач длины~$N$ равно $N{\sf P}_0$.
  
  Оказывается, что алгоритм~$A_2$ с~большой вероятностью не сможет находить 
правильное решение одной задачи, если число зараженных МВ велико. В~самом 
деле, если $m\hm=M$, то
  $$
  {\sf P}_0=(1-p)^2\,.
  $$
    Тогда вероятность получить правильное решение данной задачи равна $1\hm- 
(1\hm- p)^2$ и~среднее число правильно решенных задач в~потоке длины~$N$ 
равно $N(1\hm- (1\hm-p)^2)$. 
  
  \section{Работа алгоритма~$A_2$ в~различных предположениях 
о~вредоносном коде}
  
  Рассмотрим наихудший случай, когда в~результате решения МВ очередной 
задачи получается чис\-ло\-вой вектор, а~ВК осуществляет одинаковую замену 
некоторого значения в~этом векторе на \mbox{фиксированное} неизвестное 
значение~$\gamma$. Пусть все остальные предположения, рассмотренные ранее, 
остаются в~силе. В~этом случае алгоритм~$A_2$ перестает работать. В~самом 
деле, рассмотрим различные сочетания результатов применения 
алгоритма~$A_2$.
  
  Если получены все три одинаковых результата решения одной задачи тремя 
МВ, то все три из них могут оказаться одинаково испорченными с~вероятностью 
$(1\hm-p)^3$. Два одинаковых результата и~один отличный от них создают 
неопределенность: либо два ложных решения, либо одно ложное. Три различных 
результата в~указанных условиях возникнуть не могут.
  
  В то же время наличие базы правильно решенных задач позволяет найти 
неизвестное значение~$\gamma$. Для этого надо запустить несколько ранее 
правильно решенных задач и~дождаться нескольких различий правильных 
и~неправильных ответов. Тогда значение~$\gamma$ восстанавливается 
однозначно.
  
   При известном значении~$\gamma$ сочетания ответов по алгоритму~$A_2$ по 
новой, ранее не решенной задаче оцениваются по-дру\-гому.
  
  Если получены все три одинаковых ответа без значения~$\gamma$ в~них, то 
полученное решение~--- правильное. Если получены два ответа без 
значения~$\gamma$ в~них и~один ответ со значением~$\gamma$ в~нем, то два 
одинаковых ответа определяют истинное решение. Если получены два ответа со 
значением~$\gamma$ в~них и~один ответ без значения~$\gamma$, то этот один 
ответ определяет истинное решение. Если все три ответа со значением~$\gamma$ 
одинаковые, то получаем отказ, так как как все три могут быть получены 
с~по\-мощью~ВК. Вероятность такого события равна
  $$
  {\sf P}_0(3)=(1-p)^3\fr{\begin{pmatrix} m\\ 3\end{pmatrix}} {\begin{pmatrix} M\\ 
3\end{pmatrix}}\,.
  $$
    В этом случае необходимо повторять решение этой же задачи до тех пор, пока 
на появится ответ без значения~$\gamma$. Поскольку вероятность ошибки 
известна, то, аппроксимируя повторные испытания гео\-мет\-ри\-че\-ским 
распределением, придется принять решение с~присутствием значения~$\gamma$ с~заданной вероятностью ошибки.
  
  Пусть ВК может использовать несколько значений $\Gamma\hm= \{ \gamma_2, 
\gamma_2, \ldots , \gamma_k\}$ для нарушения правильности ответа. Тогда, используя базу 
данных правильных решений различных задач, можно \mbox{восстановить}~$\Gamma$. 
При этом все три решения с~по\-мощью алгоритма~$A_2$, не содержащие 
элементов~$\Gamma$, определяют правильное решение текущей задачи. 
  
  Если одно решение не содержит элементов~$\Gamma$, а~два содержат, то 
истинным считается решение без элементов~$\Gamma$. Если два одинаковых 
решения не содержат элементов~$\Gamma$, то это истинное решение. Если все 
три содержат элементы~$\Gamma$, то это отказ и~надо повторять процедуру 
алгоритма~$A_2$. Если при повторах возникают другие элементы~$\Gamma$, то 
их можно сразу отсеять. Постоянно повторяющееся значение элемента 
из~$\Gamma$ определяет истинное решение с~заданной вероятностью ошибки.

%\vspace*{-5pt}
  
  \section{Заключение }
  
 % \vspace*{-3pt}
  
  В работе рассмотрены все варианты решения задачи преодоления ошибки 
вычисления, порожденного ВК, в~сетецентрической архитектуре вы\-чис\-ли\-тель\-ной 
системы. Построены два класса \mbox{алгоритмов} решения задачи поиска сбоя и~ВК 
в~такой архитектуре. В~отличие от алгоритма~$A_1$, алгоритм~$A_2$ 
предоставляет больше примеров однозначного выявления истинного решения 
каждой текущей задачи. В~тех случаях когда истинное решение определяется 
с~ненулевой вероятностью ошибки, рассчитаны все необходимые вероятности.
  
  Построен пример архитектуры, позволяющей обеспечивать безопас\-ные 
решения задач в~условиях наличия небезопасных компонентов в~вычислительной 
сис\-теме.

\vspace*{-5pt}
  
{\small\frenchspacing
 {%\baselineskip=10.8pt
 %\addcontentsline{toc}{section}{References}
 \begin{thebibliography}{99}
 
 \vspace*{-2pt}
 
\bibitem{1-gr}
\Au{Ramachandran J.} Designing security architecture solutions.~--- John Wiley \& Sons Singapore 
Pte. Ltd., 2002. 452~p.

\bibitem{3-gr}
\Au{Sherwood J., Clark~A., Lynas~D.} Enterprise security architecture, 2009. 25~p. {\sf 
https://sabsacourses.com/wp-content/uploads/2021/02/TSI-W100-SABSA-White-Paper.pdf}.

\bibitem{2-gr}
\Au{Wang S.,  Ledley~R.\,S.} Computer architecture and security fundamentals of designing secure 
computer systems.~--- John Wiley \& Sons Singapore Pte. Ltd., 2013. 342~p.

\bibitem{4-gr}
\Au{Тимонина Е.\,Е.} Анализ угроз скрытых каналов и~методы построения гарантированно 
защищенных распределенных автоматизированных систем: Дисс.\ \ldots\  докт. техн. наук.~--- 
М.: РГГУ, 2004. 204~с.
\bibitem{5-gr}
\Au{Грушо А.\,А., Грушо~Н.\,А., Тимонина~Е.\,Е., Шоргин~С.\,Я.} Возможности построения 
безопасной архитектуры для динамически изменяющейся информационной сис\-те\-мы~//  
Сис\-те\-мы и~средства информатики, 2015. Т.~25. №\,3. С.~78--93. 
\bibitem{6-gr}
\Au{Lacoste M.} Architecture for secure computation infrastructure and self-management of VM 
security~// Zenodo, 2015. 97~p. doi: 10.5281/zenodo.49743.
\bibitem{7-gr}
\Au{Grusho A., Nikolaev~A., Piskovski~V., Sentchilo~V., Timonina~E.} Endpoint cloud terminal as an 
approach to secure the use of an enterprise private cloud~// Scientific and Technical Conference 
(International) on Modern Computer Network Technologies Proceedings.~--- Piscataway, NJ, USA: 
IEEE, 2020. Art.\ 9258244. 4~p. doi: 10.1109/MoNeTeC49726.2020.9258244.
\bibitem{8-gr}
\Au{D$\ddot{\mbox{u}}$llmann T.\,F.} Performance anomaly detection in microservice architectures 
under continuous change: Master's Thesis, 2017. {\sf 
https://elib.uni-stuttgart.de/\linebreak bitstream/11682/9083/1/MScThesis-TFDuellmann.pdf}.
\bibitem{9-gr}
\Au{Mayer B., Weinreich~R.} Dashboard for microservice monitoring and management //  
Conference (International) on Software Architecture Workshops.~--- Piscataway, NJ, USA: IEEE, 
2017. Р.~66--69. doi: 10.1109/\linebreak ICSAW.2017.44.
\bibitem{10-gr}
\Au{Brand$\acute{\mbox{o}}$n~$\acute{\mbox{A}}$., Sol$\acute{\mbox{e}}$~M., 
Hu$\acute{\mbox{e}}$lamo~A., Solans~D., P$\acute{\mbox{e}}$rez~M.\,S.,  
Munt$\acute{\mbox{e}}$s-Mulero~V.} Graph-based root cause analysis for service-oriented and 
microservice architectures~// J.~Syst. Software, 2020. Vol.~159. Art.\ 110432. 17~p.

\bibitem{12-gr} %11
\Au{Grusho A.\,A., Timonina~E.\,E., Shorgin~S.\,Ya.} Modelling for ensuring information security of 
the distributed information systems~// 31st European Conference on Modelling and Simulation 
Proceedings.~--- Dudweiler, Germany: Digitaldruck Pirrot GmbH, 2017. P.~656--660.
\bibitem{13-gr} %12
\Au{Grusho A., Grusho~N., Zabezhailo~M., Zatsarinny~A., Timonina~E.} Information security of 
SDN on the basis of meta data~// Computer network security~/
Eds. J.~Rak, J.~Bay, I.~V. Kotenko, \textit{et al}.~--- Lecture notes in computer science ser.~--- Springer, 2017. Vol.~10446. P.~339--347.

\bibitem{11-gr} %13
\Au{Грушо А.\,А., Забежайло~М.\,И., Грушо~Н.\,А., Тимонина~Е.\,Е.} Информационная 
безопасность на основе метаданных в~ком\-по\-нент\-но-ин\-тег\-ра\-ци\-он\-ных архитектурах 
информационных сис\-тем~// Сис\-те\-мы и~средства информатики, 2018. Т.~28. №\,2. С.~34--41.

\bibitem{14-gr}
\Au{Grusho~A., Grusho~N., Zabezhailo~M., Timonina~E.} Generation of metadata for network 
control~// Distributed computer and communication networks~/
Eds. V.\,M.~Vishnevskiy, K.\,E.~Samouylov, D.\,V.~Kozyrev. ~--- Lecture notes in 
computer science ser.~--- Cham: Springer, 2020. Vol.~12563. P.~723--735. doi: 
 10.1007/978-3-030-66471-8\_55. 
 
 \end{thebibliography}

 }
 }

\end{multicols}

\vspace*{-6pt}

\hfill{\small\textit{Поступила в~редакцию 11.10.22}}

\vspace*{8pt}

%\pagebreak

%\newpage

%\vspace*{-28pt}

\hrule

\vspace*{2pt}

\hrule

%\vspace*{-2pt}

\def\tit{ABOUT THE SECURE ARCHITECTURE OF~A~MICROSERVICE-BASED 
COMPUTING SYSTEM}


\def\titkol{About the secure architecture of~a~microservice-based 
computing system}


\def\aut{A.\,A.~Grusho$^1$, N.\,A.~Grusho$^1$, M.\,I.~Zabezhailo$^1$, D.\,V.~Smirnov$^2$, 
E.\,E.~Timonina$^1$,\\ and~S.\,Ya.~Shorgin$^1$}

\def\autkol{A.\,A.~Grusho, N.\,A.~Grusho, M.\,I.~Zabezhailo, et al.}
%D.\,V.~Smirnov$^2$,  E.\,E.~Timonina$^1$, and~S.\,Ya.~Shorgin$^1$}

\titel{\tit}{\aut}{\autkol}{\titkol}

\vspace*{-8pt}


\noindent
$^1$Federal Research Center ``Computer Science and Control'' of the Russian Academy of Sciences, 
44-2~Vavilov\linebreak
$\hphantom{^1}$Str., Moscow 119333, Russian Federation

\noindent
$^2$Sberbank of Russia, 19~Vavilov Str., Moscow 117999, Russian Federation


\def\leftfootline{\small{\textbf{\thepage}
\hfill INFORMATIKA I EE PRIMENENIYA~--- INFORMATICS AND
APPLICATIONS\ \ \ 2022\ \ \ volume~16\ \ \ issue\ 4}
}%
 \def\rightfootline{\small{INFORMATIKA I EE PRIMENENIYA~---
INFORMATICS AND APPLICATIONS\ \ \ 2022\ \ \ volume~16\ \ \ issue\ 4
\hfill \textbf{\thepage}}}

\vspace*{3pt} 




\Abste{The paper discusses a network-centric microservice architecture system in 
which all microservice computers are the same for simplicity. Each microservice 
computer may fail or receive malicious code. The maximum negative impact on the 
microservice computer is a calculation error and providing the consumer with the wrong 
result. The tasks of detecting failed microservice computers and detecting microservice 
computers with malicious code are considered. In solving the set tasks, elements of 
training are used. Correctly solved problems (conditions, source data,  and correct 
answers) are accumulated in the memory of the control system. This means that one can 
restart any task with an already known correct result. At the same time, the article uses 
the ideas and results of the present authors to ensure information security while using metadata. 
Depending on the assumptions about the possible actions of malicious code, two classes 
of secure computing algorithms are built in the context of its possible impact on 
intermediate results in the flow of solved problems. The second class of algorithms 
works in the assumption that malicious code can correctly calculate the solution to the 
current problem with probability~$p$ and introduce distortion into the result with 
probability $1-p$. The authors consider three types of distortions that malicious code can 
introduce and which allow one to either find the true solution accurately or with low 
probability of error.}

\KWE{information security; secure computing under malicious code conditions; 
microservice architecture}

 \DOI{10.14357/19922264220413} 

%\vspace*{-16pt}

% \Ack
 %  \noindent
  


%\vspace*{5pt}

  \begin{multicols}{2}

\renewcommand{\bibname}{\protect\rmfamily References}
%\renewcommand{\bibname}{\large\protect\rm References}

{\small\frenchspacing
 {%\baselineskip=10.8pt
 \addcontentsline{toc}{section}{References}
 \begin{thebibliography}{99}
\bibitem{1-gr-1}
\Aue{Ramachandran, J.} 2002. \textit{Designing security architecture solutions}. John 
Wiley \& Sons Singapore Pte. Ltd. 452~p.

\bibitem{3-gr-1}
\Aue{Sherwood, J., A.~Clark, and D.~Lynas.} 2009. Enterprise security architecture. 
25~p. Available at: {\sf  
https:// sabsacourses.com/wp-content/uploads/2021/02/TSI-W100-SABSA-White-Paper.pdf} (accessed 
November~16, 2022).

\bibitem{2-gr-1}
\Aue{Wang, S., and R.\,S.~Ledley.} 2013. \textit{Computer architecture and security 
fundamentals of designing secure computer systems.} John Wiley \& Sons Singapore 
Pte. Ltd. 342~p.
\bibitem{4-gr-1}
\Aue{Timonina, E.\,E.} 2004. Analiz ugroz skrytykh kanalov i~metody postroeniya 
garantirovanno zashchishchennykh raspredelennykh avtomatizirovannykh sistem [The 
analysis of threats of covert channels and methods of creation of guaranteed protected 
distributed automated systems]. Moscow: Russian State University for the Humanities. 
D.Sc. Diss. 204~p. 
\bibitem{5-gr-1}
\Aue{Grusho, A.\,A., N.\,A.~Grusho, E.\,E.~Timonina, and S.\,Ya.~Shorgin.} 2015. 
Vozmozhnosti postroeniya bezopasnoy arkhitektury dlya dinamicheski 
izmenyayushcheysya informatsionnoy sistemy [Regarding capabilities of secured 
architecture creation for dynamic changing information system]. \textit{Sistemy 
i~Sredstva Informatiki~--- Systems and Means of Informatics} 25(3):78--93. 
\bibitem{6-gr-1}
Lacoste, M., ed. 2015. \textit{Architecture for secure computation infrastructure and 
self-management of VM security}. Zenodo. 97~p. doi: 10.5281/zenodo.49743.

\bibitem{7-gr-1}
\Aue{Grusho, A., A.~Nikolaev, V.~Piskovski, V.~Sentchilo, and E.~Timonina}. 2020. 
Endpoint cloud terminal as an approach to secure the use of an enterprise private cloud. 
\textit{Scientific and Technical Conference (International) on Modern Computer 
Network Technologies Proceedings}. Piscataway, NJ: IEEE. 9258244. 4~p. doi: 
10.1109/\linebreak MoNeTeC49726.2020.9258244.
\bibitem{8-gr-1}
\Aue{D$\ddot{\mbox{u}}$llmann, T.\,F.} 2017. Performance anomaly detection in 
microservice architectures under continuous change. Master's Thesis. Available at: {\sf  
https://elib.\linebreak  uni-stuttgart.de/bitstream/11682/9083/1/MScThesis-TFDuellmann.pdf} 
(accessed November~16, 2022).
\bibitem{9-gr-1}
\Aue{Mayer, B., and R.~Weinreich.} 2017. Dashboard for microservice monitoring and 
management. \textit{Conference (International) on Software Architecture Workshops 
Proceedings}. Piscataway, NJ: IEEE. 66--69.  doi: 10.1109/ \mbox{ICSAW}.2017.44.

\bibitem{10-gr-1}
\Aue{Brandon, $\acute{\mbox{A}}$., M.~Sol$\acute{\mbox{e}}$, A.~Hu$\acute{\mbox{e}}$lamo, D.~Solans, M.\,S.~P$\acute{\mbox{e}}$rez, and  
V.~Munt$\acute{\mbox{e}}$s-Mulero.} 2020. Graph-based root cause analysis for service-oriented and 
microservice architectures. \textit{J.~Syst. Software} 159:110432. 17~p. 

\bibitem{12-gr-1} %11
\Aue{Grusho, A.\,A., E.\,E.~Timonina, and S.\,Ya.~Shorgin.} 2017. Modelling for 
ensuring information security of the distributed information systems. \textit{31st 
European Conference on Modelling and Simulation Proceedings}. Dudweiler, 
Germany: Digitaldruck Pirrot GmbHP. 656--660.
\bibitem{13-gr-1} %12
\Aue{Grusho, A., N.~Grusho, M.~Zabezhailo, A.~Zatsarinny, and E.~Timonina.} 2017. 
Information security of SDN on the basis of meta data. \textit{Computer network security}. Eds. 
J.~Rak, J.~Bay, I.\,V.~Kotenko, \textit{et al}. Lecture notes in computer science ser. Springer. 
10446:339--347.
\bibitem{11-gr-1} %13
\Aue{Grusho,~A.\,A., M.\,I.~Zabezhailo, N.\,A.~Grusho, and E.\,E.~Timonina}. 2018.
Informatsionnaya bezopasnost' na osnove metadannykh  
v~komponentno-integratsionnykh arkhitekturakh informatsionnykh sistem [Information 
security on the basis of meta data in enterprise application integration architecture of 
information systems]. \textit{Sistemy i~Sredstva Informatiki~--- Systems and Means of 
Informatics} 28(2):34--41.
\bibitem{14-gr-1}
\Aue{Grusho, A., N.~Grusho, M.~Zabezhailo, and E.~Timonina.} 2020. Generation of 
metadata for network control. \textit{Distributed computer and communication 
networks}. Eds. V.\,M.~Vishnevskiy, K.\,E.~Samouylov, and D.\,V.~Kozyrev. Lecture 
notes in computer science ser. Cham: Springer. 12563:723--735. 
doi: 10.1007/978-3-030-66471-8\_55. 
\end{thebibliography}

 }
 }

\end{multicols}

\vspace*{-6pt}

\hfill{\small\textit{Received October 11, 2022}}

\Contr


\noindent
\textbf{Grusho Alexander A.} (b.\ 1946)~--- Doctor of Science in physics and 
mathematics, professor, principal scientist, Institute of Informatics Problems, Federal 
Research Center ``Computer Science and Control'' of the Russian Academy of Sciences, 
44-2~Vavilov Str., Moscow 119333, Russian Federation; \mbox{grusho@yandex.ru}

\vspace*{3pt}

\noindent
\textbf{Grusho Nikolai A.} (b.\ 1982)~--- Candidate of Science (PhD) in physics and 
mathematics, senior scientist, Institute of Informatics Problems, Federal Research 
Center ``Computer Science and Control'' of the Russian Academy of Sciences,  
44-2~Vavilov Str., Moscow 119133, Russian Federation; \mbox{info@itake.ru}

\vspace*{3pt}

\noindent
\textbf{Zabezhailo Michael I.} (b.\ 1956)~--- Doctor of Science in physics and 
mathematics, principal scientist, A.\,A.~Dorodnicyn Computing Center, Federal 
Research Center ``Computer Science and Control'' of the Russian Academy of Sciences, 
40~Vavilov Str., Moscow 119333, Russian Federation; 
\mbox{m.zabezhailo@yandex.ru}

\vspace*{3pt}

\noindent
\textbf{Timonina Elena E.} (b.\ 1952)~--- Doctor of Science in technology, professor, 
leading scientist, Institute of Informatics Problems, Federal Research Center 
``Computer Science and Control'' of the Russian Academy of Sciences, 44-2~Vavilov 
Str., Moscow 119133, Russian Federation; \mbox{eltimon@yandex.ru}

\vspace*{3pt}

\noindent
\textbf{Smirnov Dmitry V.} (b.\ 1984)~--- business partner for IT security department, 
Sberbank of Russia, 19~Vavilov Str., Moscow 117999, Russian Federation; 
dvlsmirnov@sberbank.ru

\vspace*{3pt}

\noindent
\textbf{Shorgin Sergey Ya.} (b.\ 1952)~--- Doctor of Science in physics and 
mathematics, professor, principal scientist, Institute of Informatics Problems, Federal 
Research Center ``Computer Science and Control'' of the Russian Academy of Sciences, 
44-2~Vavilov Str., Moscow 119133, Russian Federation; \mbox{sshorgin@ipiran.ru}


\label{end\stat}

\renewcommand{\bibname}{\protect\rm Литература}    
   