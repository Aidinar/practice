\newcommand{\me}[2]{\mathbf{E}_{ #1 }\left\{ \mathop{#2} \right\} }
\newcommand {\ff}{{\mathcal F}}
\newcommand {\ppp}{{\mathcal P}}
\newcommand{\pp}[1]{\mathbf{P}\left\{ #1 \right\}}
\newcommand {\col}{\mathop{\mathrm{col}}}
\newcommand {\ebd}{\triangleq}




\def\stat{borisov}

\def\tit{ОБЩИЙ ПОРЯДОК АППРОКСИМАЦИИ ОЦЕНОК ФИЛЬТРАЦИИ СОСТОЯНИЙ МАРКОВСКИХ 
СКАЧКООБРАЗНЫХ ПРОЦЕССОВ ПО ДИСКРЕТИЗОВАННЫМ НАБЛЮДЕНИЯМ$^*$}

\def\titkol{Общий порядок аппроксимации оценок фильтрации состояний МСП
%марковских  скачкообразных процессов 
по дискретизованным наблюдениям}

\def\aut{А.\,В.~Борисов$^1$}

\def\autkol{А.\,В.~Борисов}

\titel{\tit}{\aut}{\autkol}{\titkol}

\index{А.\,В.~Борисов$^1$}
\index{A.\,V.~Borisov}


{\renewcommand{\thefootnote}{\fnsymbol{footnote}} \footnotetext[1]
{Работа выполнялась с~использованием инфраструктуры Центра коллективного пользования 
<<Высокопроизводительные вы\-чис\-ле\-ния и~большие данные>> (ЦКП <<Информатика>>) ФИЦ ИУ  РАН.}}


\renewcommand{\thefootnote}{\arabic{footnote}}
\footnotetext[1]{Федеральный исследовательский центр <<Информатика и~управление>> Российской академии наук,
\mbox{aborisov@frccsc.ru}}


\vspace*{-8pt}



\Abst{Заметка продолжает исследования, по\-свя\-щен\-ные чис\-лен\-ной 
аппроксимации оценок фильт\-ра\-ции со\-сто\-яний марковских скачкообразных процессов (МСП) по 
счи\-та\-ющим и~диффузионным наблюдениям с~мультипликативными шумами. Оценки 
аппроксимируются с~использованием наблюдений, дискретизованных по времени. 
В~отличие от ранних алгоритмов, огра\-ни\-чи\-ва\-ющих воз\-мож\-ное чис\-ло скачков со\-сто\-яния 
на интервале дискретизации, новые оценки предлагается вы\-чис\-лять
без по\-доб\-ных ограничений с~по\-мощью единой ре\-кур\-рент\-ной схемы. В~заметке получена 
верх\-няя граница показателя точ\-ности аппроксимации как функция па\-ра\-мет\-ров сис\-те\-мы 
наблюдения, ис\-поль\-зу\-емой схемы чис\-лен\-но\-го интегрирования, шага дискретизации 
и~момента оценивания. Чис\-лен\-ный пример иллюстрирует  сублинейное поведение данного 
показателя в~за\-ви\-си\-мости от последнего аргумента.}


\KW{марковский скачкообразный процесс; оптимальная фильт\-ра\-ция; 
диффузионные и~счи\-та\-ющие наблюдения; мультипликативные шумы в~наблюдениях; 
точ\-ность чис\-лен\-ной аппроксимации}

 \DOI{10.14357/19922264220402} 
  
\vspace*{4pt}


\vskip 10pt plus 9pt minus 6pt

\thispagestyle{headings}

\begin{multicols}{2}

\label{st\stat}


 \section{Введение}
 
\vspace*{-5pt}
 
 Данная заметка продолжает исследования в~об\-ласти разработки чис\-лен\-ных методов 
фильт\-ра\-ции со\-сто\-яний МСП по со\-во\-куп\-ности 
диффузионных и~счи\-та\-ющих наблюдений~\cite{B_20_1_ARC, B_20_2_ARC, B_21_1_IA, B_21_2_IA}. 
Ин\-тен\-сив\-ность шумов в~диффузионных наблюдениях зависит от 
оце\-ни\-ва\-емо\-го со\-сто\-яния.

 В~\cite{B_20_1_ARC, B_20_2_ARC} была решена задача фильт\-ра\-ции при наличии 
только диффузионных наблюдений. Помимо аналитического решения был предложен ряд 
чис\-лен\-ных алгоритмов, основанных на обработке наблюдений, дискретизованных по 
времени. Алгоритмы строились, исходя из предположения, что на интервале 
дискретизации оце\-ни\-ва\-емый МСП может совершить чис\-ло скачков, ограниченное 
некоторой величиной. Был получен показатель точ\-ности таких аппроксимаций, как 
функция па\-ра\-мет\-ров сис\-те\-мы наблюдения, максимального чис\-ла скачков, величины 
шага временн$\acute{\mbox{о}}$й дискретизации и~точ\-ности при\-ме\-ня\-емой схемы чис\-лен\-но\-го 
интегрирования.

 В работе~\cite{B_21_1_IA} пред\-став\-ле\-но решение задачи фильт\-ра\-ции по 
совокупности диффузионных и~счи\-та\-ющих наблюдений, а~в~\cite{B_21_2_IA}  
предложен метод при\-бли\-жен\-но\-го вы\-чис\-ле\-ния данной оценки. Следует отметить, что не 
все схемы чис\-лен\-но\-го интегрирования предполагают ограничение чис\-ла воз\-мож\-ных 
скачков МСП на интервале дискретизации: например, метод квадратур Гаусса может 
быть использован без данного условия. В~\cite{B_21_2_IA} предложен \textit{общий 
показатель} точ\-ности в~фор\-ме средней $\mathcal{L}_1$-нор\-мы ошибки аппроксимации 
оценки фильт\-ра\-ции и~доказано утверж\-де\-ние о~его локальной за\-ви\-си\-мости от шага 
временн$\acute{\mbox{о}}$й дискретизации и~схемы чис\-лен\-но\-го интегрирования.

 Целью данной работы ставится определение глобального поведения показателя 
точ\-ности аппроксимации, т.\,е.\ нахождение за\-ви\-си\-мости данной величины от шага 
временн$\acute{\mbox{о}}$й дискретизации~$\delta$ и~момента оценивания~$T$.

 Статья организована следующим образом. Раздел~2 содержит формальную постановку 
задачи фильт\-ра\-ции со\-сто\-яний МСП по диффузионным и~счи\-та\-ющим наблюдениям, 
дискретизованным по времени с~шагом $\delta$. Аналитическое решение данной 
задачи пред\-став\-ле\-но в~разд.~3 в~виде формулы Байеса, содержащей интегралы по 
абстрактной вероятностной мере в~чис\-ли\-те\-ле и~знаменателе. Для ее практической 
реализации интегралы предлагается заменить некоторыми интегральными суммами. 
Точ\-ность аппроксимации так\-же описывается в~форме сред\-ней $\mathcal{L}_1$-нор\-мы 
раз\-ности оценки фильт\-ра\-ции и~ее при\-бли\-же\-ния. Основным результатом работы 
является утверж\-де\-ние,
пред\-став\-ля\-ющее различные границы сверху этого показателя точ\-ности. Показано, что
эти границы пред\-став\-ля\-ют собой функции шага временн$\acute{\mbox{о}}$й дискретизации~$\delta$ 
и~момента оценивания~$T$. В~за\-ви\-си\-мости от схемы чис\-лен\-но\-го интегрирования, 
при\-ме\-ня\-емо\-го на одном шаге, верх\-няя граница может быть как линейной, так и~\textit{сублинейной} функцией времени~$T$.
Раздел~4 содержит чис\-лен\-ный пример, ил\-лю\-ст\-ри\-ру\-ющий различие предложенных 
в~работе границ точ\-ности. В~разд.~5 пред\-став\-ле\-ны выводы и~заключительные 
замечания.

  \section{Постановка задачи фильтрации состояния марковских скачкообразных процессов по~дискретизованным 
наблюдениям}

 На полном вероятностном пространстве с~фильт\-ра\-ци\-ей 
$(\Omega,\ff,\ppp,\{\ff_t\}_{t \geqslant 0})$ рас\-смат\-ри\-ва\-ет\-ся не\-пре\-рыв\-но-дис\-крет\-ная сто\-ха\-сти\-че\-ская
 динамическая сис\-те\-ма наблюдения
\begin{equation}
  X_t =X_0 + \int\limits_0^t \Lambda^{\top}X_{s}\,ds + \mu_t^X,
 \label{eq:stat_1}
 \end{equation}
где
   $X_t \in \mathbb{S}^N$~--- ненаблюдаемое со\-сто\-яние сис\-те\-мы~--- однородный МСП 
  с~множеством со\-сто\-яний $ \mathbb{S}^N \hm\ebd \{e_1,\ldots,e_N\}$ ($\mathbb{S}^N$~--- 
множество единичных век\-то\-ров про\-стран\-ст\-ва~$\mathbb{R}^N$), мат\-ри\-цей 
интенсивностей переходов~$\Lambda$ и~начальным распределением~$\pi$;
  $\mu_t^X \hm\in \mathbb{R}^N$~--- $\ff_t$-со\-гла\-со\-ван\-ный мартингал;
 \begin{multline}
  \mathcal{Y}_r = \int\limits_{t_{r-1}}^{t_r}fX_s\,ds+\int\limits_{t_{r-1}}^{t_r} 
\sum\limits_{n=1}^NX_s^ng_n^{{1}/{2}}\, dW_s, \\
r \in \mathbb{N}, \enskip t_r = r \delta, 
 \label{eq:cobs_1}
\end{multline}
где
  $ \mathcal{Y}_r \in \mathbb{R}^M$~--- диффузионные наблюдения
  $Y_t \hm= \int\nolimits_{0}^{t}fX_s\,ds\hm+\int\nolimits_{0}^{t} \sum\nolimits_{n=1}^NX_s^ng_n^{{1}/{2}}\,dW_s$,
   дискретизованные по времени с~шагом~$\delta$,
за\-шум\-лен\-ные $\ff_t$-со\-гла\-со\-ван\-ным стандартным винеровским процессом $W_t \hm\in 
\mathbb{R}^M$;  $(M \times N)$-мер\-ная мат\-ри\-ца~$f$ характеризует план 
наблюдений, а~набор $(M\times M)$-мер\-ных сим\-мет\-рич\-ных мат\-риц
  $\{g_n\}_{n=\overline{1,N}}$ определяет интенсивности шумов в~за\-ви\-си\-мости от 
текущего со\-сто\-яния~$X_t$;
\begin{equation}
\mathcal{Z}_r =\!\!\!\!\int\limits_{t_{r-1}}^{t_r} \! \! hX_s\,ds+\left(\mu_{t_r}^Z- \mu_{t_{r-1}}^Z\right)\!,\enskip
r \in \mathbb{N}, \ \  t_r = r \delta,
 \label{eq:sobs_1}
 \end{equation}
где 
  $ \mathcal{Z}_r \in \mathbb{R}^K$~--- считающие наблюдения
   $Z_t \hm=\int\nolimits_{0}^{t}hX_s\,ds\hm+\mu_{t}^Z$, 
  дискретизованные по времени с~шагом~$\delta$: элементы $(K \times N)$-мер\-ной 
мат\-ри\-цы~$h$ определяют ин\-тен\-сив\-ность скач\-ков отдельных компонент в~за\-ви\-си\-мости 
от текущего со\-сто\-яния~$X_t$; $\mu_t^Z \hm\in \mathbb{R}^K$~---  $\ff_t$-со\-гла\-со\-ван\-ный мар\-тин\-гал.
 

  Обозначим через $\mathfrak{O}_r \ebd \left\{ \mathcal{Y}_q, \mathcal{Z}_q: \   q=\overline{1,r}  \right\}$ 
  по\-сле\-до\-ва\-тель\-ность $\sigma$-ал\-гебр всех дискретизованных наблюдений,
полученных на отрезке времени $[0,t_r]$, $r \hm\in \mathbb{N}$, $\mathfrak{O}_0 \hm \ebd \{\varnothing,\Omega\}$.


\textit{Задача оптимальной фильтрации состояния МСП}~$X_t$ по дискретизованным 
наблюдениям заключается в~по\-стро\-ении услов\-но\-го математического ожидания  
$\widehat{\mathcal{X}}_r \hm\ebd \me{}{X_{t_r}|\mathfrak{O}_r}$, $r \hm\in 
\mathbb{N}$.

Система наблюдения~(\ref{eq:stat_1})--(\ref{eq:sobs_1}) удовлетворяет сле\-ду\-ющим 
ограничениям.
\begin{itemize}
   \item[А.]
  Исследуемый вероятностный базис с~фильт\-ра\-ци\-ей является пространством Ви\-не\-ра--Пу\-ас\-со\-на~\cite{IK_06}.
  \item[Б.]
  Шумы в~$\mathcal{Y}$ равномерно не\-вы\-рож\-ден\-ны~\cite{LSh_1_01}, т.\,е.\
  $\min\nolimits_{1 \leqslant n \leqslant N }g_n > 0$.
  \item[В.] Компоненты мартингала~$\mu^{Z}$ в~наблюдениях~$\mathcal{Z}$~(\ref{eq:sobs_1}) ортогональны друг другу, т.\,е.
    $\langle  \mu^Z   \rangle_t \hm= \int\nolimits_0^t \mathrm{diag}\left( h X_s\right)ds.$
  Мартингалы в~уравнении\linebreak со\-сто\-яния~$\mu^{X}$ и~в~наблюдениях~$\mu^{Z}$ так\-же 
ортогональны   $\langle \mu^X,\mu^Z \rangle_t \hm\equiv 0.$
\end{itemize}

\section{Аналитическое решение задачи и~численная аппроксимация}

Ниже в~статье будут использоваться сле\-ду\-ющие обозначения.
\begin{enumerate}[1.]
 \item
  $\|\alpha\|^2_{Q} \ebd \alpha^{\top}Q\alpha$.
 
  \item Для $X = \col\,(x_1,\ldots,x_N)\hm \in \mathbb{R}^N$  
  $$
  \|X\|_1 \hm\ebd 
\sum\nolimits_{n=1}^N |x_n|\,;
$$
 для $A\hm=\|a_{ij}\|_{i,j=\overline{1,N}} \hm\in 
\mathbb{R}^{N\times N}$  
$$
\|A\|_1 \hm\ebd \max_{1 \leqslant j \leqslant N} 
\sum\nolimits_{i=1}^N |a_{ij}|\,.
$$
  \item
  Единичная матрица подходящей раз\-мер\-ности:~$I$;
  нулевая мат\-ри\-ца подходящей раз\-мер\-ности:~$\mathbf{0}$;
  век\-тор-стро\-ка подходящей раз\-мер\-ности, со\-став\-лен\-ная из единиц:~$\mathbf{1}$.
  \item
   Характеристическая функция множества~$\mathcal{A}$: $\mathbf{I}_{\mathcal{A}}(x)$.
 \item
 Вектор, компоненты которого равны 
случайному времени пребывания процесса~$X$ в~каж\-дом из воз\-мож\-ных со\-сто\-яний на  
отрезке $[t_{r-1}, t_{r}]$:
$$
\tau_r \ebd \int\limits_{t_{r-1}}^{t_{r}} X_s\,ds\,.
$$
  \item
  $(N-1)$-мер\-ный симплекс в~пространстве 
$\mathbb{R}^N$, носитель распределения век\-то\-ра~$\tau_r$:

 \vspace*{-5pt}
 
 \noindent
 \begin{multline*}
 \mathcal{D} \ebd{}\\
 \hspace*{-9.43114pt}{}\ebd \left\{u=\col\,\left(u^1,\ldots,u^N\right):\; u^n \geqslant 0, %\right.\\ 
%\left.
\sum\limits_{n=1}^Nu^n= \delta \right\}\!.
\end{multline*}

\item  <<Вероятностный симплекс>>, множество воз\-мож\-ных начальных распределений~$\pi$:

 \vspace*{-5pt}
 
 \noindent
 \begin{multline*}
 \Pi \ebd{}\\
\hspace*{-6.22176pt} {}\ebd \left\{\pi=\col\,(\pi^1,\ldots,\pi^N):\ \pi^n > 0,\enskip \sum\limits_{n=1}^N\pi^n = 1 
\right\}\!.\hspace*{-6.55559pt}
\end{multline*} 

 \item
 Распределение вектора $ X_{t_{r}}^{\ell}\tau_{r}$ 
при условии $X_{t_{r-1}}\hm=e_k$:
$\mathbf{p}^{k \ell}(du)$, т.\,е.\
 для любого $\mathcal{G} \hm\in \mathcal{B}(\mathbb{R}^N)$ вер\-но равенство
 $$
  \me{}{\mathbf{I}_{\mathcal{G}}(\tau_r)X_{t_r}^{\ell}|X_{t_{r-1}}=e_k} =  \int\limits_{\mathcal{G}} \mathbf{p}^{k \ell}(du).
  $$
 \item
 $M$-мер\-ная плот\-ность гауссовского распределения с~математическим 
ожиданием~$m$ и~не\-вы\-рож\-ден\-ной ковариационной мат\-ри\-цей~$Q$:

 \vspace*{-5pt}
 
 \noindent
\begin{multline*}
%\hspace*{-9pt}
\mathcal{N}(y,m,Q) \ebd{}\\
{}\ebd (2\pi)^{-{M}/{2}} \mathrm{det}^{-{1}/{2}} Q 
\exp\left\{ - \fr{1}{2}\left\|y- 
%\right.\right.\\
%\left.\left.{}-
m\right\|^2_{Q^{-1}}
%\vphantom{\fr{1}{2}}
\right\}\!.
\end{multline*}
 \item
  Распределение $K$-мер\-но\-го 
случайного вектора с~независимыми компонентами, 
име\-ющи\-ми пуассоновское распределение с~па\-ра\-мет\-ра\-ми~$\alpha^k$, 
$k\hm=\overline{1,K}$:
$$
\mathcal{P}(z,\alpha) \ebd e^{-\sum\nolimits_{k=1}^k \alpha^k}
 \prod\limits_{k=1}^K \fr{(\alpha^k)^{z^k}}{(z^k)!},
 $$
 где
 
 \vspace*{-6pt}
 
 \noindent
\begin{multline*}
z = \col\,(z^1,\ldots,z^K) \hm\in 
\mathbb{Z}_+^K;\\
\alpha \hm= \col\,(\alpha^1,\ldots,\alpha^K) \hm\in \mathbb{R}_+^K.
\end{multline*}
 
 \item Функции
 
  \vspace*{-5pt}
  
  \noindent
 \begin{multline*}
 \theta^{kj} = \theta^{kj}(y,z) \ebd{}\\
 {}\ebd \int\limits_{\mathcal{D}} 
\mathcal{N}\left(y,fu,\sum\limits_{p=1}^N u^p g_p\right)
\mathcal{P}(z,hu)\mathbf{p}^{k j}(du).
\end{multline*}
 %\label{eq:int_1}
 %\end{equation}
 Мат\-ри\-цы, со\-став\-лен\-ные из этих функций:
$$
\theta =\theta(y,z) \ebd \|\theta^{kj}(y,z)\|_{k,j=\overline{1,N}}\,.
$$ 

Матричнозначные функции случайных величин~--- дискретизованных наблюдений, 
полученных в~момент времени~$t_r$:
$$
\theta_r \ebd \theta\left(\mathcal{Y}_r,\mathcal{Z}_r\right).
$$

\vspace*{-12pt}

\columnbreak


\noindent
Произведения мат\-риц~$\{\theta_i\}$ с~на\-рас\-та\-ющим итогом:
\begin{multline*}
\mathbf{\Theta}_r = \mathbf{\Theta}\left(y_1,\ldots,y_r, z_1,\ldots,z_r\right) \ebd{}\\
{}\ebd
\theta\left(y_1,z_1\right)\theta\left(y_2,z_2\right)\cdots \theta\left(y_r,z_r\right).
\end{multline*}

Матричнозначные функции  дискретизованных наблюдений, полученных на 
промежутке времени $[0, t_r]$:

%\vspace*{-5pt}

\noindent
$$
\Theta_r \ebd \theta_1\theta_2 \cdots\theta_r.
$$
\end{enumerate}

Используя введенные выше определения, оптимальную оценку 
$\widehat{\mathcal{X}}_r$~\cite{B_21_2_IA} по дискретизованным наблюдениям можно 
записать в~явном виде:

\noindent
\begin{equation*}
\widehat{\mathcal{X}}_r =  \fr{1}{\mathbf{1}\,\Theta_r^{\top} \pi}\Theta_r^{\top} \pi\,.
%\label{eq:expl_2}
\end{equation*}

\vspace*{-3pt}

\noindent
Матричнозначные функции~$\theta$ и~$\mathbf{\Theta}_r$ обладают сле\-ду\-ющи\-ми 
очевидными свойствами:
\begin{enumerate}[(1)]
\item
$ %\displaystyle
\sum\nolimits_{z \in \mathbb{Z}_+^K} \int_{\mathbb{R}^M} \theta^{kj}(y,z)\, dy\hm = \mathbf{P}\left\{X_h = e_j \vert  X_0=\right.$\linebreak
$\left.{}= e_k\right\} = \left(\exp\{h\Lambda\}\right)^{kj};
$
\item
$ %\displaystyle
\sum\nolimits_{z \in \mathbb{Z}_+^K} \int\nolimits_{\mathbb{R}^M}\left \|\theta^{\top}(y,z)\right\|_1 dy = 1;
$
\item
$ \sum\nolimits_{(z_1,\ldots, z_r) \in \mathbb{Z}_+^{rK}} \int\nolimits_{\mathbb{R}^{rM}}
\left\|\mathbf{\Theta}_r^{\top}\left(y_1,\ldots, y_r,z_1,\ldots\right.\right.$\linebreak $\left.\left.\ldots , z_r\right)\right\|_1 dy_1\cdots dy_r = 1.
$
\end{enumerate}
Интегралы в~$\theta^{kj}$ не могут быть вы\-чис\-ле\-ны аналитически, поэтому их 
предлагается аппроксимировать интегральными суммами общего вида. Для этого будем 
использовать сле\-ду\-ющие функции и~мат\-ри\-цы, со\-став\-лен\-ные из них:
\begin{itemize}
\item аппроксимация~$\theta^{kj}$

\noindent
$$
\psi^{kj} \!\ebd\! \sum\nolimits_{\ell=1}^L
\mathcal{N}\left(y,fu_{\ell},\sum\limits_{p=1}^N u_{\ell}^p g_p\right)
\mathcal{P}\left(z,hu_{\ell}\right)\rho_{\ell}^{k j};
$$
\item аппроксимация $\theta$

\noindent
$$
\psi =\psi(y,z) \ebd \left\|\psi^{kj}(y,z)\right\|_{k,j=\overline{1,N}}\,;
$$
\item аппроксимация~$\theta_r$

\noindent
$$
\psi_r \ebd \psi(\mathcal{Y}_r,\mathcal{Z}_r);
$$ 
\item  аппроксимация~$\mathbf{\Theta}_r$ 
\vspace*{-5pt}

\noindent
\begin{multline*}
\mathbf{\Psi}_r = \mathbf{\Psi}\left(y_1,\ldots,y_r, z_1,\ldots,z_r\right) \ebd{}\\
{}\ebd
\psi\left(y_1,z_1\right)\psi\left(y_2,z_2\right)\cdots \psi\left(y_r,z_r\right);
\end{multline*}
\item аппроксимация~$\Theta_r$

\noindent
$$
\Psi_r \ebd \psi_1\psi_2 \cdots \psi_r.
$$
\end{itemize}

В определенных выше интегральных суммах $\{u_{\ell}\}_{\ell=\overline{1,L}} 
\hm\subset \mathcal{D}$ ($u_{\ell}\hm=u_{\ell}(\delta)$)~--- набор точек; 
$\{\rho_{\ell}^{kj}\}_{\ell,k,j} \hm\subset \mathbb{R}_+$
($\rho_{\ell}^{kj} \hm=\rho_{\ell}^{kj}(\delta)$)~--- набор весов. Будем использовать только аппроксимации, обла\-да\-ющие сле\-ду\-ющим 
свойством:
\begin{enumerate}[(1)]
\setcounter{enumi}{3}
\item
$ %\displaystyle
\max_{1 \leqslant k \leqslant N} \sum\nolimits_{j=1}^N \sum\nolimits_{\ell=1}^L 
\rho_{\ell}^{kj}(\delta) \hm\ebd \mathfrak{W}(\delta)\hm \leqslant 1.$
\end{enumerate}
Условие~4 обеспечивает аппроксимациям выполнение свойств, близ\-ких по смыс\-лу~1--3:
\begin{enumerate}[(1)]
\setcounter{enumi}{4}
\item
$ \sum\nolimits_{z \in \mathbb{Z}_+^K} \int\nolimits_{\mathbb{R}^M} \psi^{kj}(y,z)\, dy\hm = 
\sum\nolimits_{\ell=1}^L \rho_{\ell}^{kj} \leqslant \mathfrak{W}(\delta) \hm\leqslant 1;$
\item
$ \sum\nolimits_{z \in \mathbb{Z}_+^K} \int\nolimits_{\mathbb{R}^M} \left\|\psi^{\top}(y,z)\right\|_1 \,dy \leqslant \mathfrak{W}(\delta) \hm\leqslant 1;$
\item
$\sum\nolimits_{(z_1,\ldots, z_r) \in \mathbb{Z}_+^{rK}} \int\nolimits_{\mathbb{R}^{rM}}
\left\|\mathbf{\Psi}_r^{\top}\left(y_1,\ldots, y_r,z_1,\ldots\right.\right.$\linebreak
$\left.\left.\ldots , z_r\right)\right\|_1\, dy_1\cdots dy_r 
\hm\leqslant \mathfrak{W}^r(\delta) \hm\leqslant 1.$
\end{enumerate}
Таким образом, вместо~$\widehat{\mathcal{X}}_r$ мож\-но вы\-чис\-лить ее  
аппроксимацию

\noindent
\begin{equation*}
\overline{\mathcal{X}}_r = \fr{1}{\mathbf{1}\Psi_r^{\top}  \pi}\,\Psi_r^{\top} \pi\,.
%\label{eq:expl_approx}
\end{equation*}

Определим глобальную характеристику бли\-зости оценки~$\{\widehat{\mathcal{X}}\}$  
и~ее аппроксимации $\{\overline{\mathcal{X}}\}$ сле\-ду\-ющим образом. Пусть  $T \hm> 0$~---  фиксированный момент времени,   
$r \hm\ebd {T}/{\delta}$ (без 
ограничения общ\-ности считаем, что~$T$ делится на~$\delta$ нацело). Показатель 
точ\-ности опре\-де\-ля\-ет\-ся сле\-ду\-ющим образом:
 \begin{equation*}
\Sigma(T,\delta) \ebd \sup\limits_{\pi \in \Pi}\me{\pi}{\left\|\widehat{\mathcal{X}}_{{T}/{\delta}} - 
\overline{\mathcal{X}}_{{T}/{\delta}}\right\|_{1}}.
% \label{eq:prec_glob_1}
 \end{equation*}

\noindent
\textbf{Теорема 1.}\ 
\textit{Пусть для аппроксимации $\psi$ выполнено условие}

\noindent
 \begin{equation}
 \sum\limits_{z \in \mathbb{Z}_+^K} \int\limits_{\mathbb{R}^M} \left\|
 (\psi(y,z)-\theta(y,z))^{\top}\right\|_1 \,dy \leqslant C\delta^{1+\alpha}
 \label{eq:cond_eps}
 \end{equation}
 \textit{при некоторых па\-ра\-мет\-рах $C,\;\alpha\hm > 0$. Тогда}
 
 \noindent
 \begin{alignat}{2}
\!\!\!\! \Sigma(T,\delta) &\leqslant 2C   \fr{1-
\mathfrak{W}^{{T}/{\delta}}(\delta)}{1-\mathfrak{W}(\delta)}\, 
\delta^{1+\alpha}, &\ \mbox{если} \ \mathfrak{W}(\delta) &< 1\,;
 \label{eq:gl_err_1}
\\
\!\!\!\! \Sigma(T,\delta) &\leqslant 2 TC  \delta^{\alpha} , &\ \mbox{если} \  
\mathfrak{W}(\delta) &= 1\,.\!\!
 \label{eq:gl_err_2}
 \end{alignat}
 %если $\mathfrak{W}(\delta) = 1$.

 Доказательство теоремы~1 приведено в~приложении.

 Суммарный вес $\mathfrak{W}$ зависит не только от шага дискретизации~$\delta$, 
но и~от мат\-ри\-цы~$\Lambda$ в~уравнении динамики~(\ref{eq:stat_1}). В~то же время 
значение~$C$ зависит от па\-ра\-мет\-ров~$f$, $\{g_n\}$ и~$h$ модели наблюдения~(\ref{eq:cobs_1}) и~(\ref{eq:sobs_1}).

 \section{Численный пример}
 
 Проиллюстрируем различия верхних границ~(\ref{eq:gl_err_1}) 
 и~(\ref{eq:gl_err_2}) на конкретном примере. В~статье~\cite{B_20_2_ARC} была 
предложена со\-став\-ная схема <<сред\-них>> прямоугольников для фильт\-ра\-ции со\-сто\-яний 
МСП по 
  наблюдениям с~мультипликативными шумами при наличии только диффузионных 
наблюдений. При наличии дополнительных счи\-та\-ющих наблюдений эта схема принимает 
вид

\vspace*{-5pt}

\noindent
 \begin{multline*}
 \psi^{kj} \ebd \delta_{kj}e^{\lambda_{jj} \delta} \mathcal{N}(y,\delta f^j, 
\delta g_j)  \mathcal{P}(y, \delta f^j)  +{}\\
 {}+
  (1-\delta_{kj})\lambda_{kj}  \delta^{1+\alpha} e^{\lambda_{jj}\delta} \times{}\\
  {}\times
 \sum\limits_{\ell=1}^{\left[\delta^{-\alpha}\right]}
 e^{(\lambda_{kk}-\lambda_{jj})u}
 \mathcal{N}\left(y,uf^k+(\delta-u) f^j, ug_k+{}\right.\\
 \left. {}+(\delta-u) g_j\right)
 \mathcal{P}\left(y, uh^k+(\delta-u) h^j\right)
 \Bigl|_{u=(\ell-{1}/{2})\delta^{1+\alpha}},\hspace*{-1.02296pt}
% \label{eq: mid_rec}
 \end{multline*}
 
 \vspace*{-3pt}
 
 \noindent
 где $f^j$ и~$h^j$ -- $j$-е столбцы матриц $f$ и~$h$.

 Действуя аналогично~\cite{B_20_2_ARC}, можно показать, что для данной схемы 
вы\-чис\-ле\-ния коэффициентов~$\psi^{kj}$ выполнено условие~(\ref{eq:cond_eps}), 
в~котором величина
$C>0$ зависит от па\-ра\-мет\-ров модели наблюдения, а~параметр $\alpha\hm>0$  описывает 
степень <<дополнительного дроб\-ле\-ния>> отрезка дискретизации~$\delta$.

 Весовые коэффициенты в~данной схеме выражаются сле\-ду\-ющим образом:
 
\vspace*{-5pt}

\noindent
\begin{multline*}
 \varrho^{kj}_{\ell} = \delta_{kj}e^{\lambda_{jj} \delta} +{}\\
 {}+
 \left(1-\delta_{kj}\right)\lambda_{kj}  \delta^{1+\alpha}
 \exp \left(  \delta  \left(  \lambda_{kk}\left(\ell-\fr{1}{2}\right)\delta^{\alpha} +{}\right.\right.\\
\left.\left.{}+ \lambda_{jj}\left(1-\left(\ell-\fr{1}{2}\right)\delta^{\alpha}\right) \right)  \right).
\end{multline*}

\vspace*{-5pt}

 Используя второй замечательный предел, мож\-но показать, что $\mathfrak{W}(\delta) 
\hm< 1$ для до\-ста\-точ\-но малого $\delta\hm > 0$.


 Численный эксперимент был проведен для значений па\-ра\-мет\-ров $N \hm= 10$,  
$\lambda_{ij} \hm\equiv 5$ при $i \hm\neq j$, $T \hm\in [0;2]$. Показатель точ\-ности~$\Sigma (T,\delta)$ 
прямо пропорционален па\-ра\-мет\-ру~$C$, поэтому нормируем его на~$C$ и~ис\-сле\-ду\-ем част\-ное ${\Sigma (T,\delta)}/{C}$.




{ \begin{center}  %fig1
% \vspace*{6pt}
   \mbox{%
\epsfxsize=76.664mm
\epsfbox{bor-1.eps}
}

\end{center}

\vspace*{-6pt}

\noindent
{\small Зависимость ${\Sigma (T,\delta)}/{C}$ от~$T$, вы\-чис\-лен\-ная по 
формулам~(\ref{eq:gl_err_1}) (черный кривые) и~(\ref{eq:gl_err_2}) (серые кривые) при различных па\-ра\-мет\-рах~$\delta$:
\textit{1}~--- $\delta\hm= 10^{-2}$; \textit{2}~--- 10$^{-3}$; \textit{3}~--- 10$^{-4}$; 
\textit{4}~--- 10$^{-5}$; \textit{5}~--- $\delta\hm= 10^{-6}$
}
}

%\vspace*{6pt}




 На рисунке представлена за\-ви\-си\-мость от~$T$  функции ${\Sigma 
(T,\delta)}/{C}$ при различных значениях шага дискретизации~$\delta$.



 
Анализируя графики, легко прийти к~сле\-ду\-юще\-му заключению.  Если удается 
по\-стро\-ить схемы числен\-но\-го интегрирования, для которых выполняется условие  
$\mathfrak{W}(\delta) \hm< 1$, то пессимистичные линейные оценки~(\ref{eq:gl_err_2}) 
верх\-ней границы $\Sigma (T,\delta)$ могут быть существенно 
уточ\-не\-ны с~использованием формулы~(\ref{eq:gl_err_1}).



 \section{Заключение}
 
 Формулы (\ref{eq:gl_err_1}) и~(\ref{eq:gl_err_2}), описывающие оценки сверху 
для средней $\mathcal{L}_1$-нор\-мы ошибки аппроксимации как функции шага 
временн$\acute{\mbox{о}}$й дискретизации~$\delta$ и~момента по\-стро\-ения оценки фильт\-ра\-ции~$T$, 
являются главными результатами данной работы. Тот факт, что в~(\ref{eq:gl_err_2}) порядок глобальной точ\-ности аппроксимации 
на~$1$  меньше 
порядка локальной точ\-ности, вполне ожидаем: потеря порядка точ\-ности на единицу 
наблюдается для чис\-лен\-ных методов решения как обыкновенных, так и~сто\-ха\-сти\-че\-ских 
дифференциальных уравнений~\cite{KP_92}. Линейный рост величины ошиб\-ки, 
опре\-де\-ля\-емой формулой~(\ref{eq:gl_err_2}), с~рос\-том момента оценивания~$T$ так\-же 
представляется достаточно рутинным и~следует из дискретного варианта неравенства 
Гро\-ну\-ол\-ла--Бел\-лма\-на и~свойств~1--3.

Вместе с~тем вариант верхней оценки глобального показателя точ\-ности, опи\-сы\-ва\-емый 
(\ref{eq:gl_err_1}), обладает несколькими необычными качествами. Во-пер\-вых, 
данная оцен\-ка демонстрирует не линейный, но \textit{сублинейный} рост по~$T$. 
Во-вто\-рых, порядок глобальной точ\-ности понизился меньше чем на единицу. Эти 
результаты получены благодаря тому, что матричнозначные функции 
$\mathbf{\Theta}_r$ с~единичной нормой удалось при\-бли\-зить функциями 
$\mathbf{\Psi}_r$, име\-ющи\-ми норму, меньшую~1. Именно это качество позволяет 
описать эволюцию во времени верх\-ней границы с~по\-мощью устойчивой рекуррентной 
зависимости, до\-став\-ля\-ющей в~итоге менее консервативную оцен\-ку показателю 
точ\-ности.

\vspace*{-6pt}

{\small 
\subsection*{\raggedleft Приложение}

%\renewcommand{\theequation}{П.\arabic{equation}}


\noindent
Д\,о\,к\,а\,з\,а\,т\,е\,л\,ь\,с\,т\,в\,о\ \ тео\-ре\-мы~1.
Преобразуем ошиб\-ку аппроксимации сле\-ду\-ющим образом:

\noindent
\begin{multline*}
\Delta_r = \overline{\mathcal{X}}_r - \widehat{\mathcal{X}}_r ={}\\
{}= 
\fr{1}{\mathbf{1}\Psi_r^{\top}\pi \mathbf{1}\Theta_r^{\top}\pi}
\left(\mathbf{1}\Theta_r^{\top}\pi \Psi_r^{\top}\pi - \mathbf{1}\Psi_r^{\top}\pi 
\Theta_r^{\top}\pi\right) ={}\\
{}=
\fr{1}{\mathbf{1}\Psi_r^{\top}\pi \mathbf{1}\Theta_r^{\top}\pi}
\left(\Psi_r^{\top}\pi\mathbf{1} \Theta_r^{\top}\pi - \mathbf{1}\Psi_r^{\top}\pi 
\Theta_r^{\top}\pi \right) ={} 
\end{multline*}

\noindent
\begin{multline*}
{} = 
\fr{1}{\mathbf{1}\Psi_r^{\top}\pi \mathbf{1}\Theta_r^{\top}\pi} 
\left(\Psi_r^{\top}\pi\mathbf{1} - \mathbf{1}\Psi_r^{\top}\pi I\right) \Theta_r^{\top}\pi ={}\\
{}=
\fr{1}{\mathbf{1}\Psi_r^{\top}\pi \mathbf{1}\Theta_r^{\top}\pi} 
\left(\Psi_r^{\top}\pi\mathbf{1} - \mathbf{1}\Psi_r^{\top}\pi I\right) 
\left(\Theta_r^{\top}\pi - \Psi_r^{\top}\pi\right) ={} \\
{} = 
\left(\overline{\mathcal{X}}_r \mathbf{1} - I\right) \fr{1}{\mathbf{1}\Theta_r^{\top}\pi}
\left(\Theta_r - \Psi_r\right)^{\top}\pi\,.
\end{multline*}
Из свойств нормы $\|\cdot\|_1$ следует, что $\|\pi\|_1\hm \equiv 1$ и~$\| 
\overline{\mathcal{X}}_r \mathbf{1} \hm- I\|_1 \hm\leqslant 2$, поэтому
\begin{equation}
\left\|\Delta_r\right\|_1 \leqslant 2\left\| \fr{1}{\mathbf{1}\Theta_r^{\top}\pi}
\left(\Theta_r - \Psi_r\right)^{\top} \right\|_1\,.
\label{eq:ineq_1}
\end{equation}
Используя обозначения
$\gamma_r \ebd \psi_r\hm - \theta_r$ и~$\Gamma_r \hm\ebd \Psi_r \hm- \Theta_r$,
мож\-но записать рекурсию
\begin{equation}
\Gamma_r =  \Gamma_{r-1}\psi_r+\Theta_{r-1} \gamma_r,  \enskip  r > 1, \enskip
\Gamma_1 = \gamma_1.
\label{eq:rec_1}
\end{equation}
Согласно лемме~3~\cite{B_20_1_ARC} для любой интегрируемой функции 
$a\left(y_1,\ldots,y_r, z_1,\ldots,z_r\right)$ имеет мес\-то равенство:
\begin{multline*} 
\me{\pi}{\fr{a\left(\mathcal{Y}_1,\ldots,\mathcal{Y}_r, 
\mathcal{Z}_1,\ldots,\mathcal{Z}_r\right)}{\mathbf{1}\Theta_r^{\top}\left(\mathcal{Y}_1,\ldots,\mathcal{Y}_r, 
\mathcal{Z}_1,\ldots,\mathcal{Z}_r\right)\pi}} ={}\\
{}=
\sum\limits_{(z_1,\ldots, z_r) \in \mathbb{Z}_+^{rK}} \int\limits_{\mathbb{R}^{rM}}
a\left(y_1,\ldots,y_r, z_1,\ldots,z_r\right) \,dy_1\cdots dy_r.\hspace*{-0.70177pt}
% \label{eq:int_1}
\end{multline*}
Взяв математическое ожидание левой и~правой час\-тей~(\ref{eq:ineq_1}), благодаря 
утверж\-де\-нию данной леммы получаем
\begin{multline}
\me{\pi}{\left\|\Delta_r\right\|_1} \leqslant  2 \!\!\!\!\!\!\!\!\!\!\sum\limits_{(z_1,\ldots, z_r) \in 
\mathbb{Z}_+^{rK}} \int\limits_{\,\mathbb{R}^{rM}}
\Big\|\Gamma^{\top}\left(y_1,\ldots,y_r,\right.\\
\left. z_1,\ldots,z_r\right)
\Big\|_1 \,dy_1\cdots dy_r.
 \label{eq:ineq_2}
\end{multline}
Обозначим 
$$
J_r \ebd \! \!\!\!\!\!\!\!\!\!\!\!\!\!\sum\limits_{(z_1,\ldots, z_r) \in \mathbb{Z}_+^{rK}} 
\int\limits_{\,\mathbb{R}^{rM}} \!\!\left\|\Gamma^{\top}\!\left(y_1,\ldots,y_r, z_1,\ldots,z_r\right)\right\|_1 dy_1\cdots dy_r.
$$ 
Тогда, вычисляя интеграл нормы $\|\cdot\|_1$ от левой и~правой час\-тей~(\ref{eq:rec_1}) и~учитывая свойства~3, 6 и~7, для $r\hm>1$ мож\-но 
получить  рекуррентное неравенство
 $ J_r \hm\leqslant \mathfrak{W}(\delta) J_{r-1}\hm+ C\delta^{1+ \alpha}$ и~начальное 
условие $J_1\hm = C\delta^{1+ \alpha}$,
 откуда по свойству убы\-ва\-ющей гео\-мет\-ри\-че\-ской прогрессии следует, что
$$
 J_r\hm \leqslant
 \begin{cases}
  C\left(\fr{1-\mathfrak{W}^{{T}/{\delta}}(\delta)}{1-
\mathfrak{W}(\delta)}\right) \delta^{1+\alpha}, & \mbox{если } \mathfrak{W}(\delta)\hm < 1; \\
 C\left(\fr{T}{\delta}\right)\, \delta^{1+\alpha} = CT  \delta^{\alpha}, & \mbox{если } \mathfrak{W}(\delta) = 1.
 \end{cases}
 $$
  Неравенства для~$J_r $ не зависят от 
начального распределения~$\pi$. Утверж\-де\-ние тео\-ре\-мы следует из подстановки 
неравенств в~правую часть~(\ref{eq:ineq_2}) и~по\-сле\-ду\-юще\-го взятия от обеих 
час\-тей точ\-ной верхней грани по $\pi \hm\in \Pi$.
Тео\-ре\-ма~1 доказана.

}


{\small\frenchspacing
 {\baselineskip=10.8pt
 %\addcontentsline{toc}{section}{References}
 \begin{thebibliography}{9}
\bibitem{B_20_1_ARC}
\Au{Борисов~A.} $\mathcal{L}_1$-оп\-ти\-маль\-ная фильт\-ра\-ция марковских 
скачкообразных процессов I: точ\-ное решение и~чис\-лен\-ные схемы реализации~// 
Автоматика и~телемеханика, 2020. Вып.~11. С.~11--31.


\bibitem{B_20_2_ARC}
\Au{Борисов A.} $\mathcal{L}_1$-оп\-ти\-маль\-ная фильт\-ра\-ция марковских 
скачкообразных процессов II: численный анализ конкретных схем~// Автоматика и~телемеханика, 2020. Вып.~12. С.~24--49.

     \bibitem{B_21_1_IA}
  \Au{Борисов А., Казанчян~Д.} Фильтрация состояний марковских скачкообразных 
процессов по комплексным наблюдениям I: точное решение задачи~// Информатика и~её применения, 2021. Т.~15. Вып.~2.\linebreak C.~11--18.

       \bibitem{B_21_2_IA}
  \Au{Борисов А., Казанчян~Д.} Фильтрация состояний марковских скачкообразных 
процессов по комплексным наблюдениям II: численный алгоритм~// Информатика и~её 
применения, 2021. Т.~15. Вып.~3. C.~9--15.

    \bibitem{IK_06}
\Au{Ishikawa Y., Kunita~H.} Malliavin calculus on the Wiener--Poisson space 
and its application to canonical SDE with jumps~// Stoch. Proc. 
Appl., 2006. Vol.~116. P.~1743--1769.

 \bibitem{LSh_1_01}
  \Au{Liptser R., Shiryaev~A.} Statistics of random processes~I: General 
theory.~--- Berlin/Heidelberg:~Springer, 2001. 427~p.

  \bibitem{KP_92}
\Au{Kloeden P., Platen~E.} Numerical solution of stochastic differential 
equations.~--- Berlin: Springer, 1992. 636~p.
\end{thebibliography}

 }
 }

\end{multicols}

\vspace*{-9pt}

\hfill{\small\textit{Поступила в~редакцию 27.12.21}}

\vspace*{6pt}

%\pagebreak

%\newpage

%\vspace*{-28pt}

\hrule

\vspace*{2pt}

\hrule

\vspace*{-4pt}

\def\tit{TOTAL APPROXIMATION ORDER FOR~MARKOV JUMP PROCESS FILTERING GIVEN DISCRETIZED OBSERVATIONS\\[-7pt]}


\def\titkol{Total approximation order for~Markov jump process filtering given discretized observations}


\def\aut{A.\,V.~Borisov}

\def\autkol{A.\,V.~Borisov}

\titel{\tit}{\aut}{\autkol}{\titkol}

\vspace*{-17pt}


\noindent
Federal Research Center ``Computer Science and Control'' of the Russian Academy of Sciences, 
44-2~Vavilov Str., Moscow 119333, Russian Federation


\def\leftfootline{\small{\textbf{\thepage}
\hfill INFORMATIKA I EE PRIMENENIYA~--- INFORMATICS AND
APPLICATIONS\ \ \ 2022\ \ \ volume~16\ \ \ issue\ 4}
}%
 \def\rightfootline{\small{INFORMATIKA I EE PRIMENENIYA~---
INFORMATICS AND APPLICATIONS\ \ \ 2022\ \ \ volume~16\ \ \ issue\ 4
\hfill \textbf{\thepage}}}

\vspace*{1pt} 




\Abste{The note proceeds the investigation devoted to the numerical approximation of the Markov jump process filtering given 
both the counting and diffusion observations with the multiplicative noise. The filtering estimates are approximated using the 
observations, previously discretized by time. By contrast with the previous algorithms which limit the number of the 
Markov state transitions that occurred during the time discretization interval, the new estimates are free of these restrictions 
and constructed via a~unified scheme. The note presents an upper bound for the approximation accuracy as a~function of the observation
 system parameters, applied scheme of the numerical integration, the time discretization step, and the filtering moment. 
 A~numerical example illustrates a~sublinear character of the bound towards the latter argument.}

\KWE{Markov jump process; optimal filtering; diffusion and counting observations; multiplicative observation noise; numerical approximation accuracy}



 \DOI{10.14357/19922264220402} 

\vspace*{-20pt}


   \Ack
   
   \vspace*{-6pt}
   
\noindent
The research was carried out using the infrastructure of shared research facilities CKP ``Informatics'' of FRC CSC RAS.


\vspace*{-1pt}

  \begin{multicols}{2}

\renewcommand{\bibname}{\protect\rmfamily References}
%\renewcommand{\bibname}{\large\protect\rm References}

{\small\frenchspacing
 {\baselineskip=10.6pt
 \addcontentsline{toc}{section}{References}
 \begin{thebibliography}{9}
\bibitem{1-bor}
\Aue{Borisov, A.} 2020. $\mathcal{L}_1$-optimal filtering of Markov jump processes. I. Exact solution and numerical implementation schemes. 
\textit{Automat. Rem. Contr.} 81(11):1945--1962.
\bibitem{2-bor}
\Aue{Borisov, A.} 2020. $\mathcal{L}_1$-optimal filtering of Markov jump processes. II. Numerical analysis of particular realizations schemes. 
\textit{Automat. Rem. Contr.} 81(12):2160--2180. 
\bibitem{3-bor}
\Aue{Borisov, A., and D.~Kazanchyan.} 2021. Fil't\-ra\-tsiya so\-sto\-yaniy mar\-kov\-skikh skach\-ko\-ob\-raz\-nykh 
pro\-tses\-sov po komp\-leks\-nym nab\-lyu\-de\-ni\-yam~I: Toch\-noe re\-she\-nie za\-da\-chi 
[\mbox{Filtering} of Markov jump processes given composite observation~I: Exact solution]. 
\textit{Informatika i~ee Primeneniya~--- Inform. Appl.} 15(2):11--18.
\bibitem{4-bor}
\Aue{Borisov, A., and D.~Kazanchyan.}
 2021. Fil'tra\-tsiya so\-sto\-yaniy mar\-kov\-skikh skach\-ko\-ob\-raz\-nykh pro\-tses\-sov po komp\-leks\-nym nab\-lyu\-de\-ni\-yam II: Chis\-len\-nyy al\-go\-ritm 
 [Filtering of Markov jump processes given composite observations II: Numerical algorithm]. 
 \textit{Informatika i~ee Primeneniya~--- Inform. Appl.} 15(3):9--15.
\bibitem{5-bor}
\Aue{Ishikawa, Y., and H.~Kunita.} 2006. Malliavin calculus on the Wiener--Poisson space and 
its application to canonical SDE with jumps. \textit{Stoch. Proc. Appl.} 116:1743--1769.
\bibitem{6-bor}
\Aue{Liptser, R., and A.~Shiryaev.}
 2001. \textit{Statistics of random processes I: General theory}. Berlin/Heidelberg: Springer. 427~p.
\bibitem{7-bor}
\Aue{Kloeden, P., and E.~Platen.} 1992. 
\textit{Numerical solution of stochastic differential equations}. Berlin: Springer. 636~p.
 \end{thebibliography}

 }
 }

\end{multicols}

\vspace*{-8pt}

\hfill{\small\textit{Received December 27, 2021}}

\vspace*{-22pt}

\Contrl

\vspace*{-4pt}

\noindent
\textbf{Borisov Andrey V.} (b.\ 1965)~--- 
Doctor of Science in physics and mathematics, principal scientist, Federal Research Center ``Computer Science and Control'' 
of the Russian Academy of Sciences, 44-2~Vavilov Str., Moscow 119333, Russian Federation; \mbox{aborisov@frccsc.ru}

\label{end\stat}

\renewcommand{\bibname}{\protect\rm Литература}  