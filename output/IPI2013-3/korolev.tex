
%\newcommand{\norm}[1]{\left\Vert#1\right\Vert}
%\newcommand{\abs}[1]{\left\vert#1\right\vert}
%\newcommand{\eps}{\varepsilon}
%\renewcommand{\r}{\mathbb R}
%\newcommand{\N}{\mathbb N}
%\renewcommand{\P}{{\sf P}}
%\newcommand{\E}{{\sf E}}
%\newcommand{\D}{{\sf D}}
%\newcommand{\sign}{{\rm sign}}
%\renewcommand{\le}{\leqslant}
%\renewcommand{\ge}{\geqslant}
%\newcommand{\I}{\mathbb{I}}
%\newcommand{\betm}{{\beta_{m+1+\delta}}}
%\newcommand{\bet}{\beta_{2+\delta}}
%\renewcommand{\endproof}{\hfill$\Box$}
%\renewcommand{\phi}{\varphi}
%\newcommand{\la}{\lambda}
%\newcommand{\si}{{\rm Si}\:}
%\renewcommand{\Re}{{\rm Re}\:}
%\newcommand{\eqd}{\stackrel{d}{=}}

\def\stat{korolev}

\def\tit{ОЦЕНКИ СКОРОСТИ СХОДИМОСТИ РАСПРЕДЕЛЕНИЙ НЕКОТОРЫХ СЛУЧАЙНЫХ СУММ К~УСТОЙЧИВЫМ 
ЗАКОНАМ$^*$}

\def\titkol{Оценки скорости сходимости распределений некоторых случайных сумм к устойчивым 
законам}

\def\autkol{В.\,Ю.~Королев,  Л.\,М.~Закс}

\def\aut{В.\,Ю.~Королев$^1$,  Л.\,М.~Закс$^2$}

\titel{\tit}{\aut}{\autkol}{\titkol}

{\renewcommand{\thefootnote}{\fnsymbol{footnote}}
\footnotetext[1] {Работа поддержана Российским фондом фундаментальных исследований (проекты 
12-07-00115а, 12-07-00109а,  11-01-00515а и 11-07-00112а).}}

\renewcommand{\thefootnote}{\arabic{footnote}}
\footnotetext[1]{Факультет вычислительной математики и кибернетики 
Московского государственного университета им.\ М.\,В.~Ломоносова; Институт 
проблем информатики РАН, vkorolev@cs.msu.su}
\footnotetext[2]{Альфа-банк, 
отдел моделирования и математической статистики, lily.zaks@gmail.com}

\vspace*{6pt}

\Abst{Приведены оценки скорости сходимости распределений
специальных сумм случайного числа независимых одинаково
распределенных случайных величин с конечными дисперсиями к
симметричным строго устойчивым законам. Предполагается, что
случайный индекс имеет смешанное пуассоновское распределение, в
котором смешивающее распределение является устойчивым законом,
сосредоточенным на положительной полуоси. Абсолютные константы
выписаны в явном виде.}

\vspace*{1pt}

\KW{устойчивое распределение; неравенство
Бер\-ри--Эс\-се\-ена; случайная сумма; дважды стохастический пуассоновский
процесс (процесс Кокса); смешанное пуассоновское распределение}


\vspace*{4pt}

 \vskip 14pt plus 9pt minus 6pt

      \thispagestyle{headings}

      \begin{multicols}{2}

            \label{st\stat}



Функцию распределения и плотность строго устойчивого распределения с
характеристическим показателем $\alpha$ и параметром~$\theta$,
задаваемого характеристической функцией
\begin{multline}
\mathfrak{g}_{\alpha,\theta}(t)={}\\
{}=\exp\left\{-|t|^{\alpha}\exp\left\{
-\fr{i\pi\theta\alpha}{2}\mathrm{sign}t\right\}\right\}\,,\enskip
t\in\r\,,\label{e1-kor}
\end{multline}
где $0<\alpha\hm\le2$,
$|\theta|\hm\le\theta_{\alpha}\hm=\min\{1,{2}/{\alpha}-1\}$, будем обозначать 
соответственно $G_{\alpha,\theta}(x)$ и $g_{\alpha,\theta}(x)$. Симметричным 
строго устойчивым распределениям соответствует значение $\theta\hm=0$. 
Односторонним устойчивым распределениям соответствуют значения $\theta\hm=1$ и 
$0\hm<\alpha\hm\le1$. Функцию распределения и плотность стандартного 
нормального закона ($\alpha\hm=2$, $\theta\hm=0$) будем обозначать 
соответственно $\Phi(x)$ и~$\phi(x)$:
$$
\phi(x)=\fr{1}{\sqrt{2\pi}}\,e^{-x^2/2}\,;\quad
\Phi(x)=\int\limits_{-\infty}^x\phi(z)\,dz\,.
$$

Рассмотрим последовательность независимых одинаково распределенных
случайных величин $X_1,X_2,\ldots$, заданных на некотором
вероятностном пространстве $(\Omega,\, \mathfrak{A},\,{\sf P})$.
Будем предполагать, что
\begin{equation*}
{\sf E}X_1=0\,, \enskip 0<\sigma^2={\sf D}X_1<\infty\,. %\label{e2-kor}
\end{equation*}
Для натурального $n\hm\ge1$ положим
$$
S_n=X_1+\cdots+X_n\,.
$$
Пусть $N_1,N_2,\ldots$~--- последовательность це\-ло\-чис\-лен\-ных 
неотрицательных случайных величин, заданных на том же самом вероятностном 
пространстве так, что при каждом $n\hm\ge1$ случайная величина $N_n$ независима 
от последовательности $X_1,X_2,\ldots$ Всюду далее для определенности будем 
считать, что $\sum\limits_{j=1}^0\hm=0$.

Принято считать, что случайная последовательность $N_1,N_2,\ldots$
неограниченно возрастает ($N_n\hm\longrightarrow\infty$) по
вероятности, если для любого $m\hm\in(0,\infty)$ ${\sf P}(N_n\hm\le
m)\longrightarrow 0$ при $n\hm\to\infty$. Всюду далее символы
$\Longrightarrow$ и $\eqd$ обозначают соответственно сходимость по
распределению и совпадение распределений.

В статье~\cite{Korolev1997} доказан следующий критерий схо\-димости
сумм случайного числа независимых одинако\-во распределенных случайных
величин \textit{с конечными дисперсиями} к симметричным строго
устойчивым законам.

\smallskip

\noindent
\textbf{Лемма 1.} \textit{Предположим, что случайные величины
$X_1,X_2,\ldots$ и $N_1,N_2,\ldots$ удовлетворяют указанным выше
условиям, причем $N_n\longrightarrow\infty$ по вероятности при
$n\hm\to\infty$. Для того чтобы при} $n\hm\to\infty$
$$
{\sf P}\left(\fr{S_{N_n}}{\sigma\sqrt{n}}<x\right) \Longrightarrow
G_{\alpha,0}(x)\,,
$$
\textit{необходимо и достаточно, чтобы}
$$
{\sf P}(N_n<nx)\Longrightarrow G_{\alpha/2,1}(x)\,.
$$



В лемме~1 главным условием является сходимость распределений
нормированных индексов $N_n$ к одностороннему строго устойчивому
распределению $G_{\alpha/2,1}(x)$. Далее будет рассматриваться
довольно полезная с точки зрения практических приложе\-ний специальная
ситуация, в которой это условие выполнено.

В книге~\cite{GnedenkoKorolev1996} предложено моделировать эволюцию\linebreak
неоднородных хаотических стохастических процессов, в частности
динамику цен финансовых активов, с помощью обобщенных дважды
стохастических пуассоновских процессов (обобщенных\linebreak процессов Кокса).
Этот подход получил дополнительное обоснование и развитие в книгах~[3--6]. 
В~книгах~\cite{Korolev2011, KorolevSkvortsova2006} этот
подход успешно применен к моделированию процессов плазменной
турбулентности. В~соответствии с указанным подходом поток
информативных событий, в результате каждого из которых появляется
очередное <<наблюденное>> значение рассматриваемой характеристики,
описывается с помощью точечного случайного процесса вида
$M(\Lambda(t))$, где $M(t)$, $t\hm\geq0$,~--- однородный пуассоновский
процесс с единичной ин\-тен\-сив\-ностью, а $\Lambda(t)$, $t\hm\geq0$,~---
независимый от $M(t)$ случайный процесс, обладающий следующими
свойствами: $\Lambda(0)\hm=0$, ${\sf P}(\Lambda(t)\hm<\infty)\hm=1$ для
любого $t\hm>0$, траектории $\Lambda(t)$ не убывают и непрерывны
справа. Процесс $M(\Lambda(t))$, $t\hm\geq0$, называется дважды
стохастическим пуассоновским процессом (процессом Кокса). 
В~частности, если процесс $\Lambda(t)$ допускает представление
$$
\Lambda(t)=\int\limits_{0}^{t}\lambda(\tau)\,d\tau\,,\enskip t\ge0\,,
$$
в котором $\lambda(t)$~--- положительный случайный процесс с
интегрируемыми траекториями, то $\lambda(t)$ можно интерпретировать
как мгновенную стохастическую интенсивность процесса Кокса.

В соответствии с такой моделью в каждый момент времени~$t$
распределение случайной величины $M(\Lambda(t))$ является смешанным
пуассоновским. Для большей наглядности рассмотрим случай, когда в
рассматриваемой модели время~$t$ остается фиксированным, а
$\Lambda(t)\hm=nU_{\alpha/2,1}$, где $n$~--- вспомогательный параметр,
$U_{\alpha/2,1}$~--- случайная величина c функцией распределения
$G_{\alpha/2,1}(x)$, независимая от стандартного пуассоновского
процесса $M(t)$, $t\hm\ge0$. При этом асимптотика $n\hm\to\infty$ может
интерпретироваться как то, что (случайная) интенсивность потока
информативных событий считается очень большой. Для каждого
натурального~$n$ положим
$$
N_n=M(nU_{\alpha/2,1})\,.
$$
Очевидно, что так определенная случайная величина $N_n$ имеет
смешанное пуассоновское распределение:

\noindent
\begin{multline}
{\sf P}(N_n=k)={\sf P}
\left(M(nU_{\alpha/2,1})=k\right)={}\\
{}=
\int\limits_0^{\infty}e^{-nz}\fr{(nz)^k}{k!}\,g_{\alpha/2,1}(z)\,dz\,,\enskip
k=0,1,\ldots
\label{e3-kor}
\end{multline}
Случайная величина $N_n$ может быть интер\-пре\-ти\-рована как число
событий, зарегистрированных к моменту времени~$n$ в пуассоновском
процессе со случайной интенсивностью, имеющей строго устойчивую
плотность $g_{\alpha/2,1}(z)$. Высокая адекватность устойчивых
распределений как моделей статистических закономерностей динамики
цен финансовых активов отмечается во многих работах (см., например,~\cite{McCulloch1996}).

Предположим, что случайная величина $U_{\alpha/2,1}$ и пуассоновский
процесс $M(t)$ независимы от последовательности $X_1,X_2,\ldots$
Тогда, очевидно, при каждом~$n$ случайная величина~$N_n$ также будет
независима от этой последовательности.

Обозначим $A_n(z)={\sf P}(N_n<nz)$, $z\hm\ge0$ ($A_n(z)\hm=0$ при $z\hm<0$).
Несложно видеть, что
$$
A_n(x)\Longrightarrow G_{\alpha/2,1}(x)\enskip (n\to\infty)\,.
$$
Действительно, как известно, если $\Pi(x;\ell)$~--- функция
распределения Пуассона с параметром $\ell\hm>0$ и $E(x;c)$~--- функция
распределения с единственным единичным скачком в точке $c\hm\in\r$, то
$$
\Pi(\ell x;\ell)\Longrightarrow E(x;1)\enskip (\ell\to\infty)\,.
$$
Так как для $x\in\r$
$$
A_n(x)=\int\limits_{0}^{\infty}\Pi(n x; n z)\,dG_{\alpha/2,1}(z)\,,
$$
то по теореме Лебега о мажорируемой сходимости при $n\hm\to\infty$
\begin{multline*}
A_n(x)\Longrightarrow\int\limits_{0}^{\infty}E(x/z;1)\,dG_{\alpha/2,1}(z)={}\\
{}=
\int\limits_{0}^{x}\,\,dG_{\alpha/2,1}(z)=G_{\alpha/2,1}(x)\,,
\end{multline*}
т.\,е.\ так определенные случайные величины $N_n$\linebreak удовлетворяют
условию, фигурирующему в леммe~1.

В дополнение к сформулированным выше условиям на случайные величины
$X_1,X_2,\ldots$ предположим, что
\begin{equation}
\beta^3={\sf E}|X_1|^3<\infty\,.\label{e4-kor}
\end{equation}
Обозначим
$$
D_{n,\alpha}=\sup_x\left\vert {\sf
P}\left(S_{N_n}<x\sigma\sqrt{n}\right)-G_{\alpha,0}(x)\right\vert\,.
$$


\smallskip

\noindent
\textbf{Теорема~1.} \textit{Пусть выполнены условия}~(\ref{e3-kor}) и~(\ref{e4-kor}). \textit{Для
любого $n\ge1$ справедлива оценка}
$$
D_{n,\alpha}\le0{,}2428
\fr{\Gamma\left({1}/{\alpha}\right)\beta^3}{\alpha\sigma^3\sqrt{n}}\,.
$$

\smallskip

\noindent
Д\,о\,к\,а\,з\,а\,т\,е\,л\,ь\,с\,т\,в\,о\,.\ \  Распределение случайной величины $N_n$
является смешанным пуассоновским (см.~(\ref{e3-kor})). Следовательно, по теореме
Фубини
\begin{multline}
{\sf P}\left(S_{N_n}<x\sigma\sqrt{n}\right)=
{\sf P}\left(S_{M(nU_{\alpha/2,1})}<x\sigma\sqrt{n}\right)={}\\
{}=\int\limits_0^{\infty}{\sf P}\left(S_{M(nz)}<x\sigma\sqrt{n}\right)
g_{\alpha/2,1}(z)\,dz\,.\label{e5-kor}
\end{multline}
Далее, как известно, симметричное строго устойчивое распределение с
параметром~$\alpha$ является масштабной смесью нормальных законов, в
которой смешивающим распределением является односторонний устойчивый
закон ($\theta\hm=1$) с параметром $\alpha/2$:
\begin{equation}
G_{\alpha,0}(x)=\int\limits_{0}^{\infty}\Phi\left(\fr{x}{\sqrt{z}}\right)\,dG_{\alpha/2,1}(z)\,,\enskip
x\in\r\label{e6-kor}
\end{equation}
(см., например, теорему~3.3.1 в~\cite{Zolotarev1983}). Из~(\ref{e5-kor}) и~(\ref{e6-kor})
следует, что
\begin{multline}
D_{n,\alpha}\le{}\\
\!{}\le\int\limits_0^{\infty}\!\!\sup\limits_x\left\vert{\sf P}\!
\left(\fr{S_{M(nz)}}{\sigma\sqrt{n}}<x\right)-
\Phi\left(\fr{x}{\sqrt{z}}\right)\right\vert\,dG_{\alpha/2,1}(z)={}\\
\!\!\!{}= \int\limits_0^{\infty}\sup\limits_x \left\vert{\sf P}
\left(\fr{S_{M(nz)}}{\sigma\sqrt{nz}}<x\right)-\Phi(x)\right\vert\,dG_{\alpha/2,1}(z).\!\!
\label{e7-kor}
\end{multline}
Подынтегральное выражение в~(\ref{e7-kor}) оценим с по\-мощью следующего аналога
неравенства Бер\-ри--Эс\-се\-ена для пуассоновских случайных сумм.

\medskip

\noindent
\textbf{Лемма 2.} \textit{Пусть случайные величины $X_1,X_2,\ldots$
одинаково распределены, причем ${\sf E}X_1\hm=0$ и ${\sf E}|X_1|^3\hm<\infty$. 
Пусть $N_{\lambda}$~--- пуассоновская случайная
величина с параметром $\lambda\hm>0$. Предположим, что случайные
величины $N_{\lambda},X_1,X_2,\ldots$ независимы в совокупности.
Обозначим}
$$
Z_{\lambda}=X_1+\cdots+X_{N_{\lambda}}\,.
$$
\textit{Тогда}
$$
\sup\limits_x\left\vert {\sf P}\left(\fr{Z_{\lambda}}{\sqrt{{\sf D}Z_{\lambda}}}<x \right)-
\Phi(x)\right\vert\le\fr{0{,}3041}{\sqrt{\lambda}}\,\fr{{\sf E}|X_1|^3}{({\sf E}X_1^2)^{3/2}}\,.
$$

\smallskip

\noindent
Д\,о\,к\,а\,з\,а\,т\,е\,л\,ь\,с\,т\,в\,о\ этого утверждения приведено 
в~\cite{KorolevShevtsova2010}, также см.\ теорему~2.4.3 в~\cite{KorolevBeningShorgin2011}.

\smallskip

Далее понадобится следующее утверждение, позволяющее вычислить ${\sf E}U_{\alpha/2,1}^{-1/2}$, 
несмотря на то что плот\-ность
$g_{\alpha/2,1}(z)$, вообще говоря, нельзя выписать в явном виде в
терминах элементарных функций.

\medskip

\noindent
\textbf{Лемма 3.}
\begin{equation*}
{\sf E}U_{\alpha/2,1}^{-1/2}=\fr{\sqrt{2}\Gamma\left({1}/{\alpha}\right)}{\alpha\sqrt{\pi}}\,.
%\label{e8-kor}
\end{equation*}

\smallskip

\noindent
Д\,о\,к\,а\,з\,а\,т\,е\,л\,ь\,с\,т\,в\,о\,.\ Из~(\ref{e1-kor}) вытекает, что характеристическая
функция симметричного ($\theta\hm=0$) строго устойчивого распределения
имеет вид:
\begin{equation}
\mathfrak{f}_{\alpha,0}(t)=e^{-|t|^{\alpha}}\,,\enskip t\in\r\,. \label{e9-kor}
\end{equation}
С другой стороны, записав соотношение~(\ref{e6-kor}) в терминах
характеристических функций с учетом~(\ref{e9-kor}), получим
\begin{equation}
e^{-|t|^{\alpha}}=\int\limits_0^{\infty}\exp\left\{-\fr{t^2z}{2}\right\}
g_{\alpha/2,1}(z)\,dz\,.\label{e10-kor}
\end{equation}
Обозначим
$$
h_{\alpha/2}(z)=\fr{\alpha}{\Gamma({1}/{\alpha})}\sqrt{\fr{\pi}{2}}\,
\fr{g_{\alpha/2,1}(z)}{\sqrt{z}}\,,\enskip
z\ge0\,.
$$
Обобщенным распределением Лапласа принято называть абсолютно
непрерывное распределение вероятностей, задаваемое плотностью
$$
\ell_{\alpha}(x)=\fr{\alpha}{2\Gamma({1}/{\alpha})} \,
e^{-|x|^{\alpha}}\,,\enskip -\infty< x<\infty\,.
$$
Переобозначив аргумент $t\mapsto x$ и выполнив несколько формальных
тождественных преобразований равенства~(\ref{e10-kor}), будем иметь:
\begin{multline}
\ell_{\alpha}(x)=\fr{\alpha}{2\Gamma({1}/{\alpha})}e^{-|x|^{\alpha}}={}\\
{}=
\fr{\alpha}{\Gamma({1}/{\alpha})}\sqrt{\fr{\pi}{2}}\,\int\limits_0^{\infty}
\fr{\sqrt{z}}{\sqrt{2\pi}}\,\exp\left\{-\fr{x^2z}{2}\right\}
\fr{g_{\alpha/2,1}(z)}{\sqrt{z}}\,dz={}\\
{}=
\int\limits_0^{\infty}\sqrt{z}\phi(x\sqrt{z})h_{\alpha/2}(z)\,dz\,.\label{e11-kor}
\end{multline}
Можно убедиться, что $h_{\alpha/2}(z)$~--- плотность распределения
неотрицательной случайной величины. Действительно, при каждом $z\hm>0$
$$
\int\limits_{-\infty}^{\infty}\sqrt{z}\phi(x\sqrt{z})\,dx=1\,.
$$
Поэтому из~(\ref{e11-kor}) вытекает, что
\begin{multline*}
1=\int\limits_{-\infty}^{\infty}\!\ell_{\alpha}(x)\,dx=
\int\limits_{-\infty}^{\infty}\!\int\limits_{0}^{\infty}\!\sqrt{z}\phi(x\sqrt{z})
h_{\alpha/2}(z)\,dz dx={}
\\
{}=\int\limits_{0}^{\infty}\!h_{\alpha/2}(z)\left(\,
\int\limits_{-\infty}^{\infty}\sqrt{z}\phi(x\sqrt{z})\,dx\right)\,dz={}\\
{}=
\int\limits_{0}^{\infty}\!h_{\alpha/2}(z)\,dz.
\end{multline*}
Лемма доказана.

\smallskip

Продолжив~(\ref{e7-kor}) с учетом лемм~2 и~3, получим
\begin{multline*}
D_{n,\alpha}\le0{,}3041\,\fr{\beta^3}{\sigma^3\sqrt{n}}\,
{\sf E}U_{\alpha/2,1}^{-1/2}={}\\
{}=0{,}3041
\fr{\sqrt{2}\Gamma\left({1}/{\alpha}\right)}{\alpha\sqrt{\pi}}\,
\fr{\beta^3}{\sigma^3\sqrt{n}}\,.
\end{multline*}
Теорема доказана.

\vspace*{-6pt}

{\small\frenchspacing
{%\baselineskip=10.8pt
\addcontentsline{toc}{section}{Литература}
\begin{thebibliography}{99}

\bibitem{Korolev1997} 
\Au{Королев В.\,Ю.} О~сходимости pаспpеделений случайных сумм независимых
случайных величин к устойчивым законам~// Теоpия веpоятностей и ее
пpименения, 1997. Т.~42. Вып.~4. С.~818--820.

\bibitem{GnedenkoKorolev1996} 
\Au{Gnedenko B.\,V., Korolev~V.\,Yu.} Random summation:
Limit theorems and applications.~--- Boca Raton: CRC Press, 1996.

\bibitem{BeningKorolev2002} 
\Au{Bening V., Korolev V.} Generalized Poisson models and their applications in
insurance and finance.~--- Utrecht: VSP, 2002. 434~p.

\bibitem{KorolevSokolov2008} 
\Au{Королев В.\,Ю., Соколов И.\,А.} Математические модели
неоднородных потоков экстремальных событий.~--- М.: ТОРУС ПРЕСС,
2008.

\bibitem{KorolevBeningShorgin2011} 
\Au{Королев В.\,Ю., Бенинг В.\,Е., Шоргин~С.\,Я.}
Математические основы теории риска.~--- 2-е изд., перераб. и доп.~--- М.:
Физматлит, 2011. 620~с.

\bibitem{Korolev2011} 
\Au{Королев В.\,Ю.} Вероят\-но\-ст\-но-ста\-ти\-сти\-че\-ские методы
декомпозиции волатильности хаотических процессов.~--- М.: Изд-во
Московского ун-та, 2011. 510~с.

\bibitem{KorolevSkvortsova2006} 
Stochastic models of structural plasma turbulence~/
Eds. V.~Korolev, N.~Skvortsova.~--- Utrecht: VSP, 2006. 400~p.

\bibitem{McCulloch1996} 
\Au{McCulloch J.\,H.} Financial applications of stable
distributions~// Handbook of statistics.~--- Amsterdam:
Elsevier Science, 1996.  Vol.~14. P.~393--425.

\bibitem{Zolotarev1983} 
\Au{Золотарев В.\,М.} Одномерные устойчивые
распределения.~--- М.: Наука, 1983.

\label{end\stat}

\bibitem{KorolevShevtsova2010} 
\Au{Korolev V., Shevtsova~I.} An improvement of the Berry--Esseen
inequality with applications to Poisson and mixed Poisson random
sums~// Scandinavian Actuarial~J., 2012. Vol.~2012. No.\,2.
P.~81--105. Available online since June~4, 2010.

\end{thebibliography} } }

\end{multicols}