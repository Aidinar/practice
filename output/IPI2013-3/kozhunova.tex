\def\stat{kozhunova}

\def\tit{КОГНИТИВНАЯ ИНТЕРОПЕРАБЕЛЬНОСТЬ ЭКСПЕРТНОГО ВЗАИМОДЕЙСТВИЯ В~ЗАДАЧЕ ОБРАБОТКИ 
РУССКО-ФРАНЦУЗСКИХ ПАРАЛЛЕЛЬНЫХ ТЕКСТОВ: ЛИНГВОКОГНИТИВНЫЕ АСПЕКТЫ}

\def\titkol{Когнитивная интероперабельность экспертного взаимодействия в~задаче обработки
%русско-французских 
параллельных текстов} %: лингвокогнитивные аспекты}

\def\autkol{О.\,С.~Кожунова}

\def\aut{О.\,С.~Кожунова$^1$}

\titel{\tit}{\aut}{\autkol}{\titkol}

%{\renewcommand{\thefootnote}{\fnsymbol{footnote}}\footnotetext[1] {Статья 
%рекомендована к публикации в журнале Программным комитетом конференции 
%<<Электронные библиотеки: перспективные методы и технологии, электронные 
%коллекции>> (RCDL-2012).}}

\renewcommand{\thefootnote}{\arabic{footnote}}
\footnotetext[1]{Институт проблем информатики Российской академии наук, 
kozhunovka@mail.ru}

 

\Abst{Обсуждаются ресурсы информационно-коммуникационных технологий (ИКТ)
<<По\-пол\-ня\-емая база лингвистических данных по 
трудностям перевода>> и <<Специальный тезаурус рус\-ско-фран\-цуз\-ских 
параллельных текстов>>, которые находятся на стадии проектирования и будут 
разработаны одновременно с созданием параллельного корпуса 
рус\-ско-фран\-цуз\-ских художественных текстов. Помимо их функциональности 
рассматриваются лингвокогнитивные аспекты взаимодействия экспертов различных 
областей, решающих задачу обработки рус\-ско-фран\-цуз\-ских параллельных 
текстов совместными усилиями.}

\KW{когнитивная интероперабельность; задача обработки естественного языка; 
рус\-ско-фран\-цуз\-ские параллельные тексты}

\vskip 14pt plus 9pt minus 6pt

      \thispagestyle{headings}

      \begin{multicols}{2}

            \label{st\stat}

\section{Введение}

     Сегодня задача автоматической обработки параллельных текстов и 
создания соответствующего инструментария в помощь филологу, который 
занимается сравнительным языкознанием, переводоведением и другими 
аспектами анализа переводных текстов, находится на пике своей 
актуальности. Это происходит, поскольку 
ИКТ достигли уровня развития, позволяющего моделировать и частично замещать 
деятельность экспертов. Особенно ресурсы ИКТ востребованы в задачах 
машинного перевода, сопоставления параллельных текстов на разных 
языках, сопоставления языковых структур различного уровня для проведения 
лингвистического анализа текстов, формирования выводов об интересующем 
исследователя языковом явлении, например о поведении грамматической 
конструкции, использования выразительных средств в языке и~т.\,п.
     
     За рубежом активные попытки создания необходимого инструментария 
предпринимались уже в 1990-х~гг.\ и в настоящее время 
приобрели более сфокусированный характер, т.\,е.\ нацелены на решение 
специфических проблем и задач корпусных исследований~[1--4]. 

Отечественные филологи и специалисты по компьютерной лингвистике 
также заинтересованы в решении узкоспециализированных задач корпусной 
лингвистики, но помимо этого предпринимают попытки к созданию 
универсальных ресурсов для обработки и выравнивания параллельных 
текстов~[5, 6].
     
     Обсуждаемые в настоящей работе ИКТ-ре\-сур\-сы <<Пополняемая 
база лингвистических данных по трудностям перевода>> и <<Специальный 
тезаурус рус\-ско-фран\-цуз\-ских параллельных текстов>> находятся на 
стадии проектирования и будут разработаны одновременно с созданием 
параллельного корпуса рус\-ско-фран\-цуз\-ских художественных текстов. 
Эти ресурсы позволят не только выйти на качественно иной уровень работы 
с параллельными текстами, в том числе в составе лингвистического корпуса, 
но и решить те актуальные задачи, которые сегодня стоят перед филологами, 
работающими в областях сравнительного языкознания и переводоведения, а 
именно: выявление и\linebreak фикса\-ция трудностей рус\-ско-фран\-цуз\-ско\-го 
перевода, сис\-тематизация и типология таких трудностей с пред\-ложениями по 
их разрешению и примерами из параллель\-ных текстов, составление тезауруса 
переводных рус\-ско-фран\-цуз\-ских терминов определенной 
стилистической направленности, установ\-ле\-ние необходимых связей и 
отношений между ними и~т.\,д.
     
     Кроме научно-исследовательского и практического применения 
вышеупомянутых лингвистических ресурсов предполагается также 
использовать их для проведения дистанционных корпусных исследований 
студентами, аспирантами и докторантами и пополнить их для этих целей 
учебными программами выравнивания параллельных художественных 
текстов.
     
     В работе уделяется особое внимание понятию когнитивной 
интероперабельности\footnote{Интероперабельность (\textit{англ}.\ 
{interoperability})~--- способность к взаимодействию.} экспертного 
взаимодействия в рамках вышеупомянутой задачи в силу ее 
междисциплинарности и сложности в установлении такого взаимодействия 
между экспертами из разных предметных областей. 

\section{Параллельный корпус и~лингвистические 
ресурсы информационно-коммуникационных технологий}
     
     Параллельный корпус рус\-ско-фран\-цуз\-ских художественных 
текстов необходим по многим причинам. Во-пер\-вых, в виду активного 
создания и редактирования <<Национального корпуса русского языка>>~[7] 
соответствующие инициативы, связанные с созданием параллельных 
корпусов, являются ожидаемыми и востребованными, в том числе и в рамках 
международного на\-уч\-но-ис\-сле\-до\-ва\-тель\-ско\-го сотрудничества в 
этой области. Во-вто\-рых, на материале художественных текстов языковые 
несоответствия русского и французского языков и сложности перевода с 
одного языка на другой выступают наиболее ярко, поэтому предлагается в 
первую очередь создавать соответствующий корпус именно художественных 
параллельных текстов.
     
     Построение пополняемой базы лингвистических данных по трудностям 
перевода является первым шагом в создании необходимых ИКТ-ре\-сур\-сов, 
сопровождающих корпус параллельных русско-фран\-цуз\-ских текстов. 
Эта база лингвистических данных предназначена для формализации 
типологии трудностей перевода, установления 
     при\-чин\-но-след\-ст\-вен\-ных и иных отношений между отдельными 
трудностями и целыми классами, сопоставления отдельных трудностей с 
примерами из текстов и даже для размещения вариантов разрешения 
обозначенных трудностей перевода. Такая база данных предоставит богатый 
материал не только для филологических исследований в области 
переводоведения и сравнительного языкознания, но и позволит сделать 
качественный рывок в практическом рус\-ско-фран\-цуз\-ском переводе~[8].
     
     Формирование учебных программ выравнивания параллельных 
художественных текстов позволит вовлекать в решение важных проблем 
корпусной лингвистики, сравнительного языкознания и разделов филологии, 
связанных с переводоведением, студентов, аспирантов и молодых 
специалистов, развивать у них навыки практических научных работ во 
многих разделах филологии, а также позволит готовить новые кадры для 
будущих исследований в этой области.
     
     Помимо вышеперечисленных ресурсов необходимо подчеркнуть 
важность создания специального тезауруса рус\-ско-фран\-цуз\-ских 
параллельных текстов. Поскольку в рамках построения параллельных 
     рус\-ско-фран\-цуз\-ских корпусов возникает множество 
филологических и лингвистических задач, связанных как с извлечением 
данных и знаний из текста, сопоставлением структур разного уровня анализа, 
так и с представлением знаний, описанных в тексте, классификацией 
терминов и связей между ними, задача построения тезауруса 
     рус\-ско-фран\-цуз\-ских терминов является актуальной. Кроме того, 
решение этой задачи дополняет процесс создания параллельных корпусов и 
релевантные исследования. Аккумуляция рус\-ско-фран\-цуз\-ской лексики 
определенной стилистики с одновременной фиксацией семантических, 
родовидовых и других тезаурусных отношений позволит качественно 
изменить подход к анализу рус\-ско-фран\-цуз\-ских текстов и содержащихся 
в них языковых структур и отношений.
     
     Идея построения специального тезауруса и его применения в задаче 
обработки параллельных рус\-ско-фран\-цуз\-ских текстов с последующим 
формированием соответствующего корпуса основана на успешном опыте 
масштабных проектов по созданию многоязычных тезаурусов и 
тезаурусоподобных лингвистических ресурсов. Одним из наиболее 
распространенных типов таких ресурсов являются автоматизированные 
словари, построенные по модели WordNet~[9--12].
     
     Проект по разработке словаря Princeton WordNet (PWN) английского 
языка в Принстонском университете (США) стартовал в первой половине 
1980-х~гг.\ и продолжается по сей день. Сейчас уже доступна версия 
2.0 WordNet. Существующая версия охватывает более 120~тыс.\ слов 
общеупотребительной лексики современного английского 
     языка~\cite{12-ko}.
     
     \begin{figure*} %fig1
\vspace*{1pt}
 \begin{center}
 \mbox{%
 \epsfxsize=162.463mm
 \epsfbox{koz-1.eps}
 }
 \end{center}
 \vspace*{-3pt}
\Caption{Фрагмент архитектуры базы данных EuroWordNet для английского и 
испанского языков}
\end{figure*}
     
     Этот словарь~--- базу данных тезаурусного типа~--- можно использовать 
для различных лингвистических задач. В~частности, при проведении 
информационного поиска wordnet-сло\-ва\-ри применяются для расширения 
запроса пользователя за счет парадигматически и синтагматически 
связанных слов, например компонентов синсета (множества синонимов, 
объединенных в набор) вместе с его гипонимами и согипонимами или связей 
типа <<гла\-гол--ак\-тант>>, которые дают возможность осуществлять 
контекстный поиск. Данные о синтагматических отношениях слов позволяют 
применять wordnet-сло\-ва\-ри для решения задачи снятия неоднозначности 
смысла слова. Wordnet можно использовать для вычисления смысловой 
близости текстов на основе гиперонимических отношений. 
     Wordnet-сло\-ва\-ри могут служить лексиконом для формальных 
грамматик. Формат wordnet является удобным формализмом для 
представления состава и структуры лексики специальных подъязыков 
(например, медицинских, экономических терминов). Wordnet-сло\-ва\-ри 
являются удобным инструментом для проведения исследований в области 
лексической семантики, например гипонимические отношения в 
     wordnet-сло\-ва\-рях позволяют определять направление 
метонимических переносов и прогнозировать появление новых 
     лек\-си\-ко-се\-ман\-ти\-че\-ских вариантов~\cite{13-ko}.

     За период с марта 1996~г.\ по сентябрь 1999~г.\ при финансировании 
Европейской комиссии был создан многоязычный вариант WordNet~--- 
EuroWordNet~\cite{14-ko}, что стало новым этапом в эволюции word\-net-сло\-ва\-рей. 
В~рамках европейского проекта было создано не только несколько 
тезаурусов для европейских языков (голландского, испан\-ского, италь\-ян\-ско\-го, 
немецкого, французского, чешско\-го и эстонского), но и впервые была 
реализована идея объединения отдельных wordnet-пред\-став\-ле\-ний в 
общую систему. Все компоненты EuroWordNet были построены по единой 
модели, что, \mbox{однако}, не предполагало прямого перевода анг\-лий\-ско\-го 
варианта WordNet~1.5. Перед разработчиками стояла задача~--- отразить все 
особенности лексических систем национальных языков. 

Со\-вмес\-ти\-мость 
компонентов EuroWordNet была обеспечена \mbox{единством} принципов и 
заданным набором общих понятий (Basic Concepts), на основе которых 
определялась система межъязыковых отсылок (Inter-Lingual-Index), дающих 
возможность переходить от лексикализованных значений одного языка к 
сходным, но не обязательно тождественным значениям в другом языке. 
Данный индекс позволяет использовать EuroWordNet не только для 
информационного поиска в рамках одного языка, но и для многоязычного 
поиска (рис.~1).
     
     
     В рамках проекта EuroWordNet первоначальная структура словаря 
претерпела серьезные изменения. Был расширен набор семантических 
отношений за счет парадигматических отношений, связывающих слова 
разных час\-тей речи (например, XPOS\_NEAR\_SYNONYMY: dead--death; 
XPOS\_HYPERONYMY: to love\,--\,emotion; XPOS\_ANTONYMY: to live\,--\,dead) и 
синтагматических отношений между глаголами и ак\-тан\-та\-ми-су\-ще\-ст\-ви\-тель\-ны\-ми 
(например, ROLE\_INSTRUMENT: to write\,--\,pencil). 
Был сформирован новый подход к построению wordnet-сло\-ва\-рей: с опорой 
на использование лексикографических источников (толковых, переводных и 
синонимических словарей) и результатов обработки корпусов современных 
текстов. 
     
     Успешное завершение проекта EuroWordNet послужило толчком к 
созданию большого числа wordnet-пред\-став\-ле\-ний для языков разных 
типов (например, венгерского, турецкого, арабского, тамильского, 
китайского и~пр.), а также многоязычных ресурсов типа EuroWordNet 
(например, проект BalkaNet нацелен на объединение греческого, румынского, 
болгарского, сербского, турецкого и чешского wordnet-сло\-ва\-рей). 
В~2001~г.\ была создана Всемирная Ассоциация WordNet (Global WordNet 
Association), целью которой является объединение уже существующих и 
только развивающихся национальных ресурсов этого типа, 
усовершенствование системы межъязыковых индексов и разработка общих 
стандартов, позволяющих использовать модель WordNet для языков разных 
типов~\cite{6-ko}.
     
     С 1999~г.\ на кафедре математической лингвистики СПбГУ 
исследовательская группа под руководством И.\,В.~Азаровой 
(Азарова~И.\,В., Митрофанова~О.\,А., Синопальникова~А.\,А.\ и~др.)\ ведет 
работы по проекту RussNet~--- созданию русской версии компьютерного 
словаря типа WordNet~\cite{15-ko}. В~задачи проекта входит построение 
лек\-си\-ко-се\-ман\-ти\-че\-ско\-го ресурса для отражения организации 
лексической системы русского языка в целом, для\linebreak представления ядра его 
общеупотребительной лексики и фиксации семантических, 
     се\-ман\-ти\-ко-грам\-ма\-ти\-че\-ских и 
     се\-ман\-ти\-ко-де\-ри\-ва\-ци\-он\-ных отношений русского языка. 
Кроме того, в настоящее \mbox{время} в Петербургском государственном 
университете путей сообщения разрабатывается проект русской\linebreak версии 
WordNet под руководством С.\,А.~Яблонского и 
     А.\,М.~Сухоногова~\cite{12-ko}.
     %
     Поэтому предполагается, что построение специального тезауруса\linebreak 
     рус\-ско-фран\-цуз\-ских параллельных текстов с учетом 
меж\-ду\-на\-род\-ного опыта формирования многоязычных тезаурусов и 
особенностей меж\-дис\-цип-\linebreak ли\-нар\-ной задачи создания и обработки корпуса\linebreak 
параллельных текстов позволит проводить филологические исследования 
большего масштаба и глубины.
     
     Поскольку сама задача создания корпуса параллельных текстов 
является междисциплинарной, то методы и подходы, задействованные в ее 
решении, так же разнообразны, а именно: методы системного анализа, 
искусственного интеллекта, компьютерной лингвистики, психолингвистики, 
когнитивного моделирования и~т.\,п. В~частности, используются методы 
извлечения данных из текстов, подходы к представлению знаний, 
классификации терминов и методы построения и обработки запросов на 
поиск в слабоструктурированных полнотекстовых документах, методы 
моделирования человеческого восприятия информации различного уровня 
сложности и~т.\,п. Тем самым предполагается вовлечение в работу экспертов 
из нескольких областей: компьютерной лингвистики, текстологии, 
переводоведения, искусственного интеллекта, когнитивной науки, 
психологии и~др. Такое многообразие специалистов, безусловно, 
повлечет за собой ряд затруднений в решении поставленной задачи, 
поскольку их картины мира, акценты и ракурсы взгляда на одни и те же 
     дан\-ные\,/\,си\-ту\-а\-тив\-ные кон\-текс\-ты\,/\,проб\-ле\-мы изначально 
отличаются друг от друга. Предложенные способы согласования 
восприятия информации и соответствующих концептуальных связях будут 
описаны в разд.~3.

\vspace*{-6pt}

\section{Опыт междисциплинарных задач}

\vspace*{-2pt}

     Ранее было сказано о междисциплинарности задачи построения 
     рус\-ско-фран\-цуз\-ско\-го корпуса и соответствующих 
лингвистических ресурсов. Действительно, для решения такой 
многоплановой задачи необходимо обратиться к экспертным знаниям и 
компетенциям из разных областей. Однако и сама идея вовлечения экспертов 
с разными подходами к мировосприятию и 
     на\-уч\-но-ис\-сле\-до\-ва\-тель\-ски\-ми парадигмами таит в себе еще 
одну научную проблему: обеспечение взаимодействия экспертов из разных 
областей для выполнения одной задачи и согласование их картин мира, 
образов ситуации, ракурсов и подходов к решению возникающих вопросов. 

     \begin{figure*}[b] 
%          \vspace*{-12pt} %fig2
%\noindent
\begin{center}
   \begin{fmpage}{155mm}
    Показатели
     \begin{center}
     \begin{enumerate}[1.]
     \item Индикаторы\\[-13pt]
     \begin{enumerate}[{1.}1]
\item Индикаторы результатов фундаментальных 
научных исследований (научные результаты)\\[-13pt]
\begin{enumerate}[{1.1.}1]
\item
Непосредственные результаты\\[-13pt]
\item Целевые результаты\\[-13pt]
\item Индикаторы взаимосвязей и влияния 
научных результатов\\[-13pt]
\begin{enumerate}[{1.1.3.}1]
\item 
Взаимосвязи и влияние на здравоохранение\\[-13pt]
\item 
Взаимосвязи и влияние на развитие сферы 
науки\\[-13pt]
\item Взаимосвязи и влияние на развитие 
технологий\\

%\begin{enumerate}[1.1.3.3.}1]
1.1.3.3.1.~Индексы самоцитирования в патентах
%\end{enumerate}
\item Взаимосвязи и влияние на образование\\[-13pt]
\end{enumerate}
\end{enumerate}
\end{enumerate}
\item Критерии\\[-13pt]
\item  Параметры\\[-13pt]
\item  Экспертные оценки
\end{enumerate}

%\vspace*{3pt}
      \end{center}
       \end{fmpage}
     \Caption{Классификационная схема, использованная для построения структуры 
семантического словаря}
\end{center}
      \end{figure*}

     
     Несмотря на новизну этой проблемы, у автора уже имеется опыт 
работы с междисциплинарными задачами и подходы к согласованию 
действий и понятий экспертов в разных областях знаний. В~первую очередь, 
опыт получен в рамках подготовки диссертационной работы~\cite{16-ko}, 
посвященной созданию и исследованию технологии разработки 
семантического словаря показателей для сис\-тем информационного 
мониторинга. Одной из проблем, решаемой в рамках этой работы, была 
проблема различия в понимании экспертами смыс\-ла индикаторов, поскольку 
это является серьезным препятствием в реализации всех трех основных 
процедур, необходимых для оценивания про\-грам\-мной деятельности в сфере 
науки: информационный мониторинг, анализ, получение количественных и 
экспертных оценок ее результатов, эффективности и результативности. Это 
вызвало необходимость решения задачи согласования понимания 
индикаторов разными экспертами. Было отмечено, что в силу особенностей 
формирования терминов мониторинга возникает также задача частной 
референции, когда одно название индикатора может обозначать целый класс 
индикаторов (например, индексы цитирования, смысл которых зависит от 
учета самоцитирования, а также цитирования соавторами и~т.\,п.). Поэтому 
был предложен и разработан семантический словарь, который позволил 
отобразить информацию и связи, необходимые для решения задачи. Новизна 
предложенного семантического словаря состоит в том, что он содержит 
ссылки на алгоритмические и информационные ресурсы системы 
информационного мониторинга, а также нормативные документы как 
источники терминов рассматриваемой предметной области. Построенная 
формализация процесса извлечения терминов мониторинга и их определений 
из массива текстов (нормативные документы, научные статьи и~т.\,д.)\ 
позволила выделить необходимые индикаторы, связи между ними и их 
определения. Это облегчило дальнейшую интеграцию индикаторов и других 
показателей мониторинга в классификационную схему семантического 
словаря (рис.~2 и~3)~\cite{16-ko}.
     

\begin{figure*} %fig3
\vspace*{1pt}
 \begin{center}
 \mbox{%
 \epsfxsize=150.718mm
 \epsfbox{koz-3.eps}
 }
 \end{center}
 \vspace*{-6pt}
\Caption{Связь статей семантического словаря с ресурсами ИТСМ РАН (ИТСМ~--- 
Ин\-фор\-ма\-ци\-он\-но-тех\-но\-ло\-ги\-че\-ская система мониторинга, разрабатываемая в 
 в отделе~16 ИПИ РАН)}
\end{figure*}

     
     После завершения диссертационного исследования была начата 
     на\-уч\-но-ис\-сле\-до\-ва\-тель\-ская работа 
     <<Лек\-си\-ко-се\-ман\-ти\-че\-ские методы создания 
     проблемно-ориентированных лингвистических ресурсов информационных систем>>, 
выполняемая в Институте проблем информатики РАН в 2011--2013~гг. Само 
название и поставленные задачи уже говорят о междисциплинарности этого 
исследования:
\columnbreak

\noindent
     \begin{enumerate}[(1)]
\item семантическое моделирование предметных областей в рамках 
научно-ис\-сле\-до\-ва\-тель\-ской работы (НИР) как основа создания лингвистического обеспечения 
информационных систем;
\item модификация лингвистических ресурсов сис\-тем 
информационного мониторинга с применением 
лек\-си\-ко-се\-ман\-ти\-че\-ских методов в процессе создания систем 
информационного мониторин\-га научной деятельности;
\item формализация и компьютерное моделирование трудностей 
перевода.
\end{enumerate}

     Из перечисленных задач реализованы и успешно сданы подзадачи 
задач~1--2, задача~3 находится в процессе выполнения. В~ходе выполнения 
НИР был получен опыт и разработаны подходы к извлечению знаний из 
слабоструктурированных текстов, решению задач компьютерной 
лингвистики применительно к области информационного мониторинга и их 
формализации и моделированию в рамках разрабатываемой 
информационной сис\-те\-мы, проектированию и созданию макета 
лингвистических ресурсов информационной сис\-те\-мы (семантический и 
проективный словари), компонентному анализу лексических единиц на 
немецком языке и\linebreak установлению лек\-си\-ко-смыс\-ло\-вых связей с 
семантическими единицами русского языка (на примере 
     рус\-ско-не\-мец\-ких языковых пар патентных \mbox{текстов}), выявлению 
различных языковых трансформаций и их классификации, а также 
разработка метапредставлений для компонентного 
     лек\-си\-ко-се\-ман\-ти\-че\-ско\-го анализа~[18--21].
     
     Разработанные в рамках НИР подходы были успешно применены в 
междисциплинарных задачах информационного мониторинга и 
со\-по\-став\-ле\-ния параллельных патентных текстов~\cite{21-ko, 22-ko}. Поэто\-му 
при взаимодействии с экспертами геоинформатики, задачи которой также 
подразумевают вовлечение специалистов из нескольких предметных 
областей, были применены разработанные подходы, дополненные 
возможностями понятий когнитивной интероперабельности, когнитивного 
пространства и ряда междисциплинарных подходов и методов, образованных 
на стыке прикладной лингвистики, когнитивной психологии и 
искусственного интеллекта. Предлагаемый подход был успешно применен в 
междисциплинарных задачах геоинформатики~\cite{23-ko}.
     
     В рамках данного подхода были предложены и частично реализованы 
следующие задачи: 
     \begin{itemize}
\item разработка лингвистических методов и моделей обеспечения 
когнитивной интероперабельности экспертной 
ин\-фор\-ма\-ци\-он\-но-ана\-ли\-ти\-че\-ской деятельности на основе 
геоинформационных описаний;
\item формирование корпуса параллельных тематических текстов 
различных предметных областей на нескольких языках. Параллельные 
многоязычные корпуса в данном случае предназначены для накопления 
эмпирических данных по проблематике предметных областей, в 
частности в вопросах полноты и согласованности терминосистем, 
экспертного ин\-фор\-ма\-ци\-он\-но-ана\-ли\-ти\-че\-ско\-го 
взаимодействия и~т.\,д., а так\-же для апробации созданного корпуса и 
ког\-ни\-тив\-но-линг\-ви\-сти\-че\-ских методов на массиве реальных 
кросс-язы\-ко\-вых данных;
\item разработка программных приложений для апробации 
предлагаемого подхода к моделированию когнитивного пространства 
взаимодействия экспертов в рамках заданной предметной области. 
Предполагается также апробация приложений в междисциплинарных 
задачах с целью разрешения терминологических разногласий между 
экспертами смежных областей, восстановления адекватных 
при\-чин\-но-след\-ст\-вен\-ных связей между понятиями предметной 
области, обеспечения возможности принятия решений в случаях 
недоопределенных терминосистем или их отсутствия и других задачах;
\item создание информационных технологий (ИТ) нового поколения, учитывающих особенности 
взаимодействия экспертов разного уровня как внутри предметных 
областей, так и в междисциплинарных задачах.
\end{itemize}

     Предлагаемый подход был апробирован на практике в ходе 
координации экс\-пер\-тов-ана\-ли\-ти\-ков, принимающих согласованные 
решения на основе анализа геоинформационных описаний (рис.~4).
     
     Существующие методы лингвистического анализа, применяемые к 
задачам формирования и\linebreak\vspace*{-12pt}

\pagebreak

\end{multicols}

            \begin{figure} %fig4
      \vspace*{1pt}
 \begin{center}
 \mbox{%
 \epsfxsize=161.258mm
 \epsfbox{koz-4.eps}
 }
 \end{center}
 \vspace*{-6pt}
\Caption{Взаимодействие экспертов из разных предметных областей в 
геоинформационной системе}
\end{figure}

\begin{multicols}{2}

\noindent
 сопровождения интеллектуальных 
геоинформационных сис\-тем, потребовали привлечения методов,\linebreak которые 
позволили выделить необходимые единицы в соответствующих 
информационных структурах, фиксировать связи между ними, а также 
выявить и отобразить их семантику и ситуативный контекст~\cite{24-ko}. 

Среди исследованных методов были выделены методы 
     лек\-си\-ко-се\-ман\-ти\-че\-ско\-го моделирования когнитивных 
структур знаний, которые позволяют учесть особенности 
геоинформационной среды~\cite{23-ko}.
     В работе~\cite{23-ko}, в частности, говорилось о том, что структура 
интеллектуального анализа геоинформационных данных основана на 
гибридных формах представления знаний, а именно иерархической 
классификации с возможностью установления и корректировки нескольких 
видов те\-зау\-рус\-но-се\-ман\-ти\-че\-ских отношений (собственно 
семантические, отношения часть--це\-лое, ги\-по\-ним--ги\-пе\-ро\-ним 
и~т.\,д.). В~данной структуре, в отличие от онтологий, акцент смещен в 
сторону когнитивных аспектов описания и семантического наполнения 
терминов, которыми оперируют в работе эксперты различных областей и 
уровней компетенции, а не на описание самой предметной области. 
     
     Структура была спроектирована и представлена на основе xsd-схем, 
заполнение которых объектами отрасли и их связями было реализовано в 
отдельных xml-фай\-лах. Анализ и валидация предложенной структуры 
осуществлены на основе программных модулей на C++. Затем был 
разработан подход на основе лек\-си\-ко-се\-ман\-ти\-че\-ско\-го 
моделирования к интеграции лингвистического обеспечения тезаурусного 
типа, совместимого с заданной структурой, определенными в ней объектами 
и отношениями. В~результате были получены программные модули на C++, 
разделенные по функциональности: поддержка структуры интеллектуального 
анализа данных в сис\-те\-ме, сопровождение и наполнение тезауруса 
геоинформационной сис\-те\-мы, обеспечение согласованной интеграции 
тезауруса и сопряженного лингвистического обеспечения в 
геоинформационную сис\-те\-му~\cite{23-ko}.
     
     Следующим ключевым понятием, задействованным в разработке 
междисциплинарного подхода к координации экс\-пер\-тов-ана\-ли\-ти\-ков, 
принимающих согласованные решения на основе анализа 
геоинформационных описаний, было понятие когнитивного пространства. 
Когнитивное пространство с учетом многофакторности взаимодействия 
экспертов на основе геоинформационных описаний расширяет понятие 
единого геоинформационного пространства~\cite{25-ko}. Были 
сформулированы требования к разрабатываемой архитектуре единого 
геоинформационного пространства на основе сравнительного анализа 
существующих метасхем баз геоданных с учетом семантики, заложенной в 
пространственные онтологии. Проведенный анализ исследований показал 
необходимость разработки модели, учитывающей латентные атрибутивные 
связи между онтологиями разнородных геоданных,\linebreak что должно повысить 
качество отображаемых связей в концептуальной схеме базы геоданных. 
Предлагаемая архитектура представляет собой многоуровневую структуру 
базы геоданных, при этом \mbox{каждый} из уровней ориентирован на решение 
определенного класса задач и содержит определенный набор 
пользовательских и библиотечных процедур управления процессами 
обработки геоданных для создания инструментария управления 
про\-грам\-мным комплексом~\cite{23-ko}.
     
     Следует отметить, что впервые инструментарий и возможности 
     лек\-си\-ко-се\-ман\-ти\-че\-ских методов и моделирования были 
применены в задачах гео\-информатики. Интеграция лингвистического 
обеспе\-чения тезаурусного типа на основе лек\-си\-ко-се\-ман\-ти\-че\-ско\-го 
моделирования позволила сгруппировать разнородные геоданные и 
структуры в едином геоинформационном пространстве в рамках 
разработанной гибридной формы представления знаний~--- иерархической 
классификации с элементами лек\-си\-ко-се\-ман\-ти\-че\-ских отношений, а 
также выявить объекты и понятия данной предметной области и фиксировать 
те\-зау\-рус\-но-се\-ман\-ти\-че\-ские отношения между ними~\cite{23-ko}.
     
     Методы и подходы лексико-семантического модели\-рования, которые в 
настоящее время широко применяются в различных предметных об\-лас\-тях, 
впервые были использованы при проектировании архитектуры 
лингвистического обеспечения геоинформационного пространства, 
форми\-ровании структуры интеллектуального анализа геодан\-ных в 
геоинформационных сис\-те\-мах и интеграции лингвистического 
обеспечения тезаурусного типа в единое геоинформационное пространство. 
Лек\-си\-ко-се\-ман\-ти\-че\-ское моделирование применительно к задачам и 
проблемам геоинформатики позволяет использовать средства анализа 
глубинных структур языка для извлечения геоданных, их связей и 
отношений, встраивания их в заданную геоинформационную структуру и 
верификации их семантики в разрезе формирования единого 
геоинформа\-ционного пространства. Инструменты глубинного\linebreak представ\-ле\-ния 
языковых структур способствуют разрешению многочисленных 
неоднозначностей геоданных и связанной с этим рассогласованности при 
принятии важных решений экс\-пер\-та\-ми-ана\-ли\-ти\-ка\-ми, а также 
формировать единое геоинформационное пространство, позволяющее 
отображать адекватную информационную структуру геоданных~\cite{23-ko}.
     
     В качестве возможных приложений предложенного подхода 
рассматриваются следующие задачи: координация работы экспертов из 
различных предмет\-ных областей, в том числе в междисциплинарных задачах, 
в информационных сис\-те\-мах в ав\-то\-ма\-ти\-че\-ском/по\-лу\-ав\-то\-ма\-ти\-че\-ском 
режиме; междисциплинарные научные исследования различных предметных 
областей с целью разработки проб\-лем\-но-ори\-ен\-ти\-ро\-ван\-ных 
программных приложений и установления взаимодействия экспертов 
различного профиля и уровня подготовки; изучение методов и подходов 
компьютерной лингвистики и их адап\-тив\-ных приложений для создания 
нового поколения~ИТ.

\section{Когнитивная интероперабельность и~предлагаемый 
подход}
     
     Вышеописанная структура представления знаний хорошо себя показала 
применительно к задачам информационного мониторинга и геоинформатики. 
Но будет ли она так же универсальна, если эксперты столкнутся с 
ситуациями неопреде\-лен\-ности ключевых понятий, с нечеткими 
постановками задач, с лексической полисемией и другими препятствиями к 
однозначной и эксплицитной трактовке терминов, их смысловых связей и их 
восприятия специалистами из разных областей (например, психологии, 
лингвистики, информатики и~др.)?
     
     Определенно, необходимо усовершенствовать предложенный подход с 
учетом указанных ког\-ни\-тив\-но-линг\-ви\-сти\-че\-ских механизмов 
восприятия информации человеком. По этому поводу было подготовлено 
много содержательных работ. Так, в~\cite{26-ko} рассматриваются 
механизмы образования лексической полисемии и даже предлагается 
     кон\-цеп\-ту\-аль\-но-смыс\-ло\-вая модель ее образования. 
Моделирование взаимодействия смысловых значений слова и изучение 
истоков его многозначности существенно расширит возможности любого 
тезауруса предметной области. В~частности, использование автором 
работы~\cite{26-ko} понятия базового концепта~\cite{27-ko} и дихотомии 
<<основное vs.\ производное значение слова>>~\cite{28-ko} позволит 
моделировать потенциальные девиации смыслов терминов и 
позиционировать новые смыслы среди традиционных, а также выделять 
основные, <<ядерные>> понятия предметной области и выстраивать вокруг 
них такую ие\-рар\-хи\-чески-се\-те\-вую структуру представления знаний, 
которая наиболее полно отражает терминологический портрет изучаемой 
     об\-ласти/за\-да\-чи/проб\-ле\-мы. Последние соображения связаны с 
понятием базового концепта, который определяется как обобщенный 
(концептуальный) объект, пред\-став\-ля\-ющий собой сложную когнитивную 
единицу~--- совокупность Формы, Действий и Интенций~\cite{26-ko}:
     \begin{multline*}
\mbox{Концепт} = \mbox{концептуальный Объект} ={}\\
{}= (\mbox{Форма, Действия, 
Интенции})\,.
\end{multline*}

     Под Формой здесь понимается структура элементарных 
пространственных объемов, под Интенциями~--- содержательные 
характеристики Формы (желания, цели, намерения, потребности, функции 
и~т.\,п.), а Действия представляют собой типичные физические действия 
Формы, посредством которых реализуются ее Интенции~\cite{26-ko}. 
Интересно, что такой концептуальный Объект задает свою категорию 
конкретных объектов, схожих с ним по всем трем характеристикам. 
В~процессе когнитивного развития ребенка именно эта пара~--- 
концептуальный Объект и задаваемая им категория~--- позволяет познавать 
мир и расширять границы уже познанного. Это означает что, поскольку такое 
представление понятий не зависит от родного языка носителя и хранится в 
его долговременной памяти, механизм фиксации концептуальных Объектов и 
их категорий чрезвычайно важен при отображении знаний экспертов о 
     ка\-кой-ли\-бо об\-ласти и позволяет выработать обобщенный подход к 
формализации понятий и их связей.
     %
     Что касается основных и производных значений~\cite{28-ko}, то, как 
было отмечено выше, их фиксация и встраивание в 
     иерар\-хи\-чески-се\-те\-вую структуру знаний некоторой области 
позволит не только моделировать потенциальные девиации смыслов 
терминов, но и позиционировать новые смыслы среди традиционных. 

Такой 
подход к моделированию смыслов играет важную роль при взаимодействии 
экспертов из разных областей, так как производные смыслы не хранятся в 
памяти носителя языка, а основные смыс\-лы одних и тех же понятий у разных 
людей могут существенно отличаться друг от друга (и даже у одних и тех же 
людей, но в разных языковых ситуациях) (см.\ пример~1). Это позволит 
обеспечить когнитивную интероперабельность~\cite{29-ko} в их работе и 
достичь поставленных целей при решении ими междисциплинарных 
     за\-дач/проб\-лем.

\medskip

\noindent
\textbf{Пример 1.}

{\small
\textit{По полю рыжей стрелой летела лиса} (основное значение).

\textit{Ну и лиса эта Ваша Маша!} (производное значение).

\textit{Вон смотри, какая обезьяна!} (в зоопарке, про животное~--- основное значение).

\textit{Фу, какая обезьяна!} (про уродливого человека или волосатого мужчину).
}

\medskip

     Приведенные выше соображения находят подтверждения и в других 
публикациях, в том числе в~\cite{30-ko}. Авторы провели серию 
     ког\-ни\-тив\-но-линг\-ви\-сти\-че\-ских экспериментов с носителями 
языка, что позволило им выявить интересные закономерности восприятия 
людьми лексических единиц языка и их смысловых связей. Например, при 
прослушивании устной речи неоднозначными оказываются единицы, 
совпадающие только по звучанию, но различающиеся написанием (омофоны: 
плот--плод, по\-рог--по\-рок и~т.\,д.), а при восприятии письменной речи, 
напротив,~--- совпадающие по написанию, но различающиеся звучанием 
(омографы: зам\textbf{о}к~--- з\textbf{а}мок, м\textbf{у}ка~--- мук\textbf{а} 
и~т.\,п.)~\cite{30-ko}. Кроме того, Черниговская и соавторы подчеркивают, 
что выбор значения слова в первую очередь происходит на основании его 
структуры, а затем уже учитывается его контекст. Любопытно, что в одном 
из экспериментов, описанном ими, оценивалась роль частотности при 
интерпретации лексически неоднозначного фрагмента речи при восприятии 
омофонов. В~результате была подтверждена первостепенная роль 
частотности словоформ при осуществлении выбора между омофонами.

\begin{figure*}[b] %fig5
%\vspace*{9pt}
 \begin{center}
 \mbox{%
 \epsfxsize=109.851mm
 \epsfbox{koz-5.eps}
 }
 \end{center}
 \vspace*{-6pt}
\Caption{Специальный тезаурус русско-французских параллельных текстов и его связи с 
другими лингвистическими ресурсами}
\end{figure*}
     
     Отсюда можно сделать вывод о том, что для успешного 
взаимодействия экспертов в рамках такой междисциплинарной задачи, как 
построение и\linebreak ведение базы данных трудностей рус\-ско-фран\-цуз\-ско\-го 
перевода и обеспечения когнитивной ин\-тер\-опе\-ра\-бель\-ности их деятельности 
(т.\,е.\ в двух различных системах эксперты видят согласованные \mbox{образы} 
представляемой информации), действительно необходим такой ресурс, как 
специальный тезаурус рус\-ско-фран\-цуз\-ских параллельных текстов. Этот 
тезаурус позволит не просто зафиксировать специальные термины и понятия 
из различных областей и семантические связи между ними, но и отобразит 
информацию об особенностях восприятия этих терминов разными 
специалистами в разных речевых ситуациях и контекстах.
     
     Такого рода информация справедливо отнесена к области 
<<когнитивной семантики>> автором работы~\cite{31-ko}, который, как и 
автор~\cite{26-ko}, апеллирует к подходам Д.~Лакоффа~\cite{32-ko}. 
Кузнецов подчеркивает, что значения слов возникают раньше, чем 
концептуальные структуры (из доконцептуального телесного опыта)~[31]. Под 
доконцептуальными структурами здесь подразумеваются гештальты\linebreak (единые 
ментальные образы) и об\-раз\-но-схе\-ма\-ти\-че\-ские структуры: 
вместилище, верх--низ, часть--це\-лое и~т.\,д. Причем связанные с ними 
кон\-цеп\-ты непосредственно значимы, что влияет на непосредствен\-ное, 
однозначное восприятие предложения или фразы. Поэтому понимание, 
согласно Лакоффу, есть не что иное, как способность соотносить концепты 
со своим опытом, включая доконцептуальный. Следовательно, чтобы 
обеспечить возможность понимания экспертами необходимых 
     тер\-ми\-нов/кон\-цеп\-тов/си\-ту\-а\-ций и прочего в едином ключе, 
необходимо прежде всего облегчить восприятие ими заданной информации 
на максимально доступном уровне с точки зрения общечеловеческих 
когнитивных способностей такого рода. В~частности, использовать 
     об\-раз\-но-схе\-ма\-ти\-че\-ские структуры, которые дают возможность 
людям рассуждать быстрее машин~\cite{31-ko}.
     
     За понятием образно-схе\-ма\-ти\-че\-ской струк\-ту\-ры скрывается 
идея о том, что существуют определенные схемы, которые человек 
изначально накладывает на воспринимаемый мир. Например, людям 
свойственно представлять большие фрагменты своего повседневного опыта в 
терминах вместилища, в английском языке характеризуемого в первую 
очередь через предлоги <<in>> и <<out>>. Так, мы выходим из (out) 
полусонного состояния, забытья, смотрим в (in) зеркало и~т.\,п. Лакофф 
провел специальное лингвистическое исследование, в котором на материале 
600\,(!) глаголов английского языка демонстрируется категоризация по схеме 
<<вместилище>>~\cite{33-ko}.
     
     Таким образом, адекватная с точки зрения человеческого восприятия 
форма представления знаний в специальном тезаурусе, в том числе 
доконцептуальных структур, может существенно облегчить восприятие 
информации, необходимой для взаимодействия экспертов, причем 
независимо от их изначальной профессиональной специализации.

\medskip

\noindent
\textbf{Пример 2.}


{\small Фрагмент из повести А.\,П.~Чехова <<Скучная история>> и ее художественный 
перевод на французский язык:

\textit{В детстве и в юности я почему-то питал страх к швейцарам и к театральным 
капельдинерам}\ldots

\textit{Dans mon enfance et mon adolescence, j'avais, je ne sais pourquoi, peur}  
{\bfseries\textit{des Suisses}} \textit{et des ouvreurs de theater}\ldots

Здесь {\bfseries\textit{des Suisses}} дословно переводится с французского языка как 
<<швейцарцы>> (национальность).
}

\medskip
     
     Возможная информация в специальном тезаурусе по поводу этого 
случая (пример~2):

\medskip
     
\hangindent=5mm{\small \hspace*{\parindent}Образно-схематическая структура повестей А.\,П.~Чехова 
<<Прислуга~--- швейцары, капельдинеры, горничные>> сопоставляется с 
аналогичной французской структурой <<portiers, concierges, ouvreurs de loges, 
femmes de chambre>>. В~базе данных трудностей перевода этот случай 
фиксируется с отсылкой к тезаурусу для согласованного восприятия и 
отображения экспертами параллельных текстов с пометой возможной 
подмены французскими лингвистами термина <<швейцар>> термином 
<<швейцарец>>.}

\smallskip
     
     Получившаяся концептуальная схема специального тезауруса 
     рус\-ско-фран\-цуз\-ских параллельных текстов, разработанная с 
учетом вышеизложенных дополнений по усовершенствованию подхода к 
обеспечению когнитивной интероперабельности\linebreak\vspace*{-12pt}

\pagebreak

\end{multicols}

\begin{figure} %fig6
\vspace*{1pt}
 \begin{center}
 \mbox{%
 \epsfxsize=114.703mm
 \epsfbox{koz-6.eps}
 }
 \end{center}
 \vspace*{-6pt}
\Caption{Концептуальная схема усовершенствованного специального тезауруса}
\vspace*{9pt}
\end{figure}

\begin{multicols}{2}

\noindent
 экспертов различных 
областей, которые работают над созданием и обработкой параллельного 
корпуса рус\-ско-фран\-цуз\-ских художественных текстов, представлена 
на рис.~5 и~6.


\vspace*{-6pt}     


\section{Заключение}

     Создание таких актуальных лингвистических ресурсов, как корпус 
рус\-ско-фран\-цуз\-ских параллельных художественных текстов, 
пополняемая \mbox{база} лингвистических данных по трудностям перевода, учебные 
программы выравнивания параллельных художественных текстов, а также 
специальный тезаурус рус\-ско-фран\-цуз\-ских параллельных текстов, влечет 
за собой необходимость совмещения разнообразных междисциплинарных 
подходов и, как следствие этого процесса, вызывает потребность в 
установлении взаимодействия и согласования базовых понятий у экспертов 
из разных об\-ластей.
     
     В данной работе был приведен опыт работы с междисциплинарными 
задачами и краткое описание сформированных в результате подходов. 

Далее 
были рассмотрены лингвистические аспекты когнитивного восприятия 
информации экспертами разных областей, включая механизмы выделения 
базовых концептов, дихотомию <<основные vs.\ производные значения>>, 
фиксацию неоднозначных лексических единиц и ситуаций (омографы, 
омофоны, лексическая полисемия), учет час\-тот\-ности словоформ и 
ситуативного контекста в трактовке смысла терминов, отображение 
     об\-раз\-но-схе\-ма\-ти\-че\-ских струк\-тур для определенных 
языковых ситуаций и~т.\,д.

\pagebreak
     
     Рассмотренные аспекты позволили принять во внимание 
соответствующие особенности ког\-ни\-тив\-но-линг\-ви\-сти\-че\-ских 
механизмов восприятия информации и картины мира людьми~--- 
экспертами, а также усовершенствовать подход к обеспечению когнитивной 
интероперабельности экспертной деятельности на примере 
междисциплинарной\linebreak задачи по созданию и обработке корпуса 
     рус\-ско-фран\-цуз\-ских параллельных художественных текстов и 
сопутствующих ресурсов (пополняемая \mbox{база} лингвистических данных по 
трудностям перевода; учебные программы выравнивания параллельных 
художественных текстов; специальный тезаурус рус\-ско-фран\-цуз\-ских 
параллельных текстов). Все это позволило улучшить концептуальную схему 
специального тезауруса, тем самым подготовив почву к его разработке и 
апробации.

{\small\frenchspacing
{%\baselineskip=10.8pt
\addcontentsline{toc}{section}{Литература}
\begin{thebibliography}{99}

\bibitem{1-ko}
\Au{Sinclair J.} The automatic analysis of corpora~// Directions in Corpus Linguistics: Nobel 
Symposium 82 Proceedings.~--- Berlin: Mouton de Gruyter, 1992.
\bibitem{2-ko}
\Au{Wallis S., Nelson G.} Knowledge discovery in grammatically analysed corpora~// Data 
Mining and Knowledge Discovery, 2001. Vol.~5. P.~307--340. 
\bibitem{3-ko}
\Au{Dukes K., Atwell E., Habash~N.} Supervised collaboration for syntactic annotation of 
quranic arabic~// Language Resources and Evaluation~J., Special Issue on Collaboratively 
Constructed Language Resources, 2011.
\bibitem{4-ko}
\Au{McCarthy D.} Exploiting distributional similarity for lexical acquisition~// Компьютерная 
лингвистика и интеллектуальные технологии: По мат-лам ежегодной международной 
конф. <<Диалог'2011>>.~--- М.: РГГУ, 2011. Вып.~10(17). C.~19--31.

\bibitem{6-ko}
\Au{Азарова И.\,В., Синопальникова А.\,А., Яворская~М.\,В.} Принципы построения 
wordnet-те\-зау\-ру\-са RussNet~// Компьютерная лингвистика и интеллектуальные 
технологии: Труды Междунар. конф. Диалог'2004.~--- М., 2004. С.~542--547.

\bibitem{5-ko}
\Au{Ляшевская О.\,Н., Кузнецова Ю.\,Л.} Русский Фреймнет: к задаче создания корпусного 
словаря конструкций~// Компьютерная лингвистика и интеллектуальные технологии: По 
мат-лам ежегодной Междунар. конф. <<Диалог'2009>>.~---  М.: РГГУ, 2009. 
Вып.~8(15). C.~306--312.

\bibitem{7-ko}
Национальный корпус русского языка: Сайт проекта {\sf http://www.ruscorpora.ru}.


\bibitem{8-ko}
\Au{Бунтман Н.\,В., Зацман И.\,М.} О~проекте создания\linebreak компьютерного ресурса 
трудностей перевода: замет-\linebreak ки на полях~// Маргиналии-2010: границы культуры и текста: 
Тезисы II Междунар. конф.~--- М.: МГУ, 2010. 
С.~41--43. {\sf http://uni-persona.srcc.msu.su/site/ conf/marginalii-2010/thesis.htm}.


\bibitem{11-ko-1} %9
\Au{Miller~G.\,A.} 
Five papers on WordNet. CSL-Report.~--- Princeton: Princeton University, 1990.  Vol.~43.

\bibitem{10-ko} %10
\Au{Fellbaum C.} WordNet: An electronic lexical database.~--- Cambridge, 1998. 



\bibitem{11-ko} %11
\Au{Кожунова О.\,С.} Семантический словарь системы информационного мониторинга в 
сфере науки и ресурс Eurowordnet: структура, задачи и функции~// Сис\-те\-мы и средства 
информатики.~---  М.: Наука, 2008. Вып.~18. С.~156--170.

\bibitem{9-ko} %12
\Au{Кожунова О.\,С.} Подходы к лек\-си\-ко-се\-ман\-ти\-че\-ско\-му моделированию и 
лингвистические ресурсы информационных сис\-тем~// Сис\-те\-мы и средства 
информатики.~--- М: ИПИ РАН, 2011. С.~139--161.

\bibitem{12-ko} %13
\Au{Сухоногов А.\,М., Яблонский С.\,А.} Словари типа WordNet в технологиях Semantic 
Web~// Конф. по искусственному интеллекту (КИИ-2004): Тр. 9-й Национальной 
конф. по искусственному интеллекту с международным участием.~--- В~3-х~т.~--- М.: 
Физ\-мат\-лит, 2004. Т.~2. С.~557--564.
\bibitem{13-ko} %14
\Au{Азарова И.\,В., Митрофанова О.\,А., Синопальникова~А.\,А., Ушакова~А.\,А., 
Яворская~М.\,В.} Разработка компьютерного тезауруса русского языка типа WordNet~// 
Корпусная лингвистика и лингвистические базы данных: Докл. науч. конф.~/ Под 
ред. А.\,С.~Герда.~--- СПб.: СПбГУ, 2002. С.~6--18.
\bibitem{14-ko} %15
\Au{Vossen P.} Introduction to EuroWordNet~// Computers  Humanities, 1998. Vol.~32. 
No.\,2--3. P.~73--89.
\bibitem{15-ko} %16
\Au{Азарова И.\,В., Митрофанова О.\,А., Синопальникова~А.\,А.} Компьютерный тезаурус 
русского языка типа WordNet~// Мат-лы конф. <<Диалог-2003>>.~--- М., 2003.
\bibitem{16-ko} %17
\Au{Кожунова О.\,С.} Технология разработки семантического словаря системы 
информационного мониторинга: Автореф. дисс.\ \ldots\  канд. техн. наук.~--- М.: ИПИ 
РАН, 2009. 2~ с.
\bibitem{17-ko} %18
\Au{Зацман И.\,М., Дурново А.\,А.} Моделирование процессов формирования экспертных 
знаний для мониторинга про\-грам\-мно-це\-ле\-вой деятельности~// Информатика и её 
применения, 2011. Т.~5. Вып.~4. С.~84--98.
\bibitem{18-ko} %19
\Au{Kozhunova O.} Lexical and semantic methods in design of the problem-oriented linguistic 
resources~// WORLDCOMP'11:  2011 World Congress in Computer Science, Computer 
Engineering and Applied Computing Proceedings.~--- Las Vegas: CSREA Press, 2011. Vol.~II. 
P.~618--624.
\bibitem{19-ko} %20
\Au{Kozhunova O.} Cross-disciplinary approach to expert activity cognitive interoperability 
support~//  5th Conference (International) on Cognitive Science Proceedings.~--- Kaliningrad, 
 2012. С.~91--92.
\bibitem{20-ko} %21
\Au{Кожунова О.\,С.} Моделирование лексической семантики в задачах компьютерной 
лингвистики~//  Сис\-те\-мы и средства информатики, 2012. Т.~22. №\,1. С.~86--109.
\bibitem{21-ko} %22
\Au{Kozhunova O.} Detection of nominalized structures in parallel patent texts in Russian and in 
German~// WORLDCOMP'09:  2009 World Congress in Computer Science, Computer 
Engineering and Applied Computing Proceedings.~--- Las Vegas: CSREA Press, 2009. Vol.~I. 
P.~479--485.
\bibitem{22-ko} %23
\Au{Кожунова О.\,С.} Выявление номинализованных конструкций в параллельных 
текстах патентных документов на русском и немецком языках~// Компьютерная 
лингвистика и интеллектуальные технологии: По мат-лам ежегодной Междунар. 
конф. <<Диалог'2009>>.~--- М.: РГГУ, 2009. Вып.~8(15). C.~185--191.
\bibitem{23-ko} %24
\Au{Дулин С.\,К., Дулина Н.\,Г., Кожунова~О.\,С.} Когнитивная интероперабельность 
экспертной деятельности и ее приложение в геоинформатике~// Конф. по искусственному 
интеллекту (КИИ-2012): Труды 13-й Национальной конф. по искусственному интеллекту с 
международным участием.~--- Белгород: БГТУ им. В.\,Г.~Шухова, 2012. С.~351--357.
\bibitem{24-ko} %25
\Au{Дулин С.\,К., Розенберг И.\,Н.} О~развитии методологических основ и концепций 
геоинформатики~// Сис\-те\-мы и средства информатики. Спец. вып.: 
На\-уч\-но-ме\-то\-до\-ло\-ги\-че\-ские проблемы информатики.~--- М.: ИПИ РАН, 2006. С.~201--256.
\bibitem{25-ko} %26
\Au{Цветков В.\,Я.} Информатизация, инновационные процессы и геоинформационные 
технологии~// Геодезия и аэрофотосъемка, 2006. №\,4. С.~112--118.
\bibitem{26-ko} %27
\Au{Кошелев А.\,Д.} Кон\-цеп\-ту\-аль\-но-смыс\-ло\-вая модель образования лексической 
полисемии~// 5-я Междунар. конф.\ по когнитивной науке: Тезисы докладов.~--- 
Калининград, 2012. Т.~2. С.~464--465.
\bibitem{27-ko} %28
\Au{Norvig P., Lakoff G.} Taking: A~study in lexical network theory~//  13th Berkeley 
Linguistics Society Annual Meeting Proceedings: BLS, 1987. P.~195--206.
\bibitem{28-ko} %29
\Au{Виноградов В.\,В.} Основные типы лексических значений слова~// 
Избранные труды. Лексикология и лексикография.~--- М., 1977. С.~162--189.
\bibitem{29-ko} %30
\Au{Buddenberg R.} Toward an interoperability reference model, 2006. {\sf 
http://web1.nps.navy.mil/?budden/\linebreak lecture.notes/interop RM.html}.
\bibitem{30-ko} %31
\Au{Черниговская Т.\,В., Дубасова~А.\,В., Риехакайнен~Е.\,И.} Лексическая 
неоднозначность и организация ментального лексикона~// 5-я Междунар. конф. по 
когнитивной науке: Тезисы докладов.~--- Калининград, 2012. Т.~2. С.~698--700.
\bibitem{31-ko} %32
\Au{Кузнецов О.\,П.} О~возможности организации знаний на основе когнитивной 
семантики~// 5-я Междунар. конф. по когнитивной науке: Тезисы докладов.~--- 
Калининград, 2012. Т.~2. С.~806--807.
\bibitem{32-ko} %33
\Au{Lakoff J.} Women, fire, and dangerous things: What categories reveal about the mind.~--- 
University of Chicago Press, 1987.

\label{end\stat}

\bibitem{33-ko} %34
\Au{Зайцев Д.} Язык как зеркало мышления: Рецензия на книгу Джорджа Лакоффа 
<<Женщины, огонь и опасные вещи: что категории языка говорят нам о мышлении>>~/ 
Пер. с англ. И.\,Б.~Шатуновского.~--- М.: Языки славянской культуры, 2004. 792~с.~// 
Отечественные записки, 2004.
\end{thebibliography} } }

\end{multicols}