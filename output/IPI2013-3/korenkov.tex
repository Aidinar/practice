\def\stat{korenkov}

\def\tit{РАЗРАБОТКА ИМИТАЦИОННОЙ МОДЕЛИ СБОРА И~ОБРАБОТКИ ДАННЫХ ЭКСПЕРИМЕНТОВ НА 
УСКОРИТЕЛЬНОМ КОМПЛЕКСЕ НИКА$^*$}

\def\titkol{Разработка имитационной модели сбора и~обработки данных экспериментов на 
ускорительном комплексе НИКА}

\def\autkol{В.\,В.~Кореньков, А.\,В.~Нечаевский, В.\,В.~Трофимов}

\def\aut{В.\,В.~Кореньков$^1$, А.\,В.~Нечаевский$^2$, В.\,В.~Трофимов$^3$}

\titel{\tit}{\aut}{\autkol}{\titkol}

{\renewcommand{\thefootnote}{\fnsymbol{footnote}}\footnotetext[1] {Работа 
частично выполнена в рамках ФЦП <<Исследования и разработки по приоритетным 
направлениям развития научно-тех\-но\-ло\-ги\-че\-ско\-го комплекса России на 
2007--2013 годы>> (гос.\ контракт №\,07.524.12.4008).}}

\renewcommand{\thefootnote}{\arabic{footnote}}
\footnotetext[1]{Объединенный институт ядерных исследований, korenkov@cv.jinr.ru} 
\footnotetext[2]{Объединенный институт ядерных исследований, Andrey.Nechaevskiy@gmail.com}
\footnotetext[3]{Объединенный институт ядерных исследований, trofimov@jinr.ru}
  
\vspace*{24pt}

\Abst{В работе обоснована необходимость создания имитационной модели системы 
хранения и обработки данных ускорительного комплекса НИКА. В~качестве платформы для 
создания модели выбрана система GridSim. В~работе описан подход к моделированию 
системы хранения данных dCache и каналов передачи. На простом примере показаны 
возможности использования модели. }

\vspace*{6pt}

\KW{грид-технологии; грид-инфраструктуры; сис\-те\-ма хранения данных; оптимизация; 
моделирование; исследование; разработки; dCache; Tier1; НИКА; грид}

\vspace*{24pt}

\vskip 14pt plus 9pt minus 6pt

      \thispagestyle{headings}

      \begin{multicols}{2}

            \label{st\stat}


\section{Введение}

   В настоящее время в Объединенном институте ядерных исследований 
создается ускорительный комплекс НИКА. 

Комплекс НИКА представляет 
собой ускоритель тяжелых ионов НИКА и установку MPD (multipurpose 
detector), объединяющую детекторы для изучения ядерной материи в горячем и 
плотном состоянии, которое возникает при столкновении ускоренных тяжелых 
ионов. Установка MPD является источником данных с интенсивностью потока десятки 
петабайт в год. 
   
   Ожидаемая интенсивность потока данных настолько велика, что массивы 
данных характеризуются как сверхбольшие. Для обработки таких потоков 
данных используются распределенные системы коллективного пользования, 
построенные на грид-тех\-но\-ло\-гиях.
   
   Для оптимизации структуры будущего комплекса обработки данных 
необходимо определить его основные параметры, структуру и проверить 
предлагаемые технические решения с помощью моделирования. Для этих целей 
на базе пакета моделирования GridSim создана имитационная модель грид-сайта.

\vspace*{-6pt}

\section{Система обработки данных ускорительного комплекса 
НИКА}

\vspace*{-2pt}

   Хранение и использование экспериментальных\linebreak данных в современных 
исследованиях в об\-ласти физики высоких энергий является актуальной 
проб\-ле\-мой. Объем получаемых и обрабатываемых данных исключает 
возможность их хранения и использования не только на одном кластере, но и в 
пределах одной организации, поэтому на первый план выходит создание 
распределенной системы хранения и обработки данных.
   
   Для эксперимента MPD на НИКА предполагается, что поток данных будет 
иметь следующие параметры:
   \begin{itemize}
\item высокая скорость набора событий (до 6~кГц);
\item в центральном столкновении Au-Au при энергиях НИКА 
образуется до 1000~заряженных час\-тиц;
\item размер файла с первоначальной моделируемой информацией с 
детекторов для одного события занимает около 0,45~MБ.
\end{itemize}

\end{multicols}

\begin{figure} %fig1
\vspace*{9pt}
 \begin{center}
 \mbox{%
 \epsfxsize=157.686mm
 \epsfbox{kor-1.eps}
 }
 \end{center}
 \vspace*{-6pt}
\Caption{Схема обработки физических данных ускорительного комплекса НИКА}
\vspace*{6pt}
\end{figure}

\begin{multicols}{2}

   Схема получения и обработки данных пред\-став\-ле\-на на рис.~1. 
   
   Данные, 
идущие от персональных компьютеров поддетекторов MPD, накапливаются 
специально предназначенными для сборки событий программами (Event 
Builder) компьютерной фермы в режиме онлайн. После формирования события 
в режиме офлайн через специально предназначенную для этой цели 
во\-ло\-кон\-но-оп\-ти\-че\-скую линию связи с пропускной способностью 10~ГБ/с данные 
записываются на диск.
   
   После триггера высокого уровня отобранные события записываются в 
   RAW-фай\-лы (скорость записи один файл в 1~минуту сбора данных) и затем 
полностью восстанавливаются. 
   
   Прогнозируемое число обрабатываемых событий при этом составляет 
приблизительно $19\cdot 10^9$. Принимая скорость передачи данных от 
датчиков равной 4,7~ГБ/с, общий объем исходных данных можно оценить в 
30~ПБ ежегодно, или 8,4~ПБ после сжатия. Эти оценки основаны на 
особенностях DAQ (data acquisition) и подобных оценках, выполненных для эксперимента 
ALICE~[1].
   


   В качестве системы обработки физической информации в эксперименте 
НИКА предполагается использование грид. Грид (название по аналогии с 
электрическими сетями~--- electric power grid)~--- это компьютерная 
инфраструктура нового типа, обеспечивающая глобальную интеграцию 
информационных и вычислительных ресурсов. Суть инициативы грид состоит 
в создании набора стандартизированных служб для обеспечения надежного, 
совместимого, дешевого и безопасного доступа\linebreak к географически 
распределенным высокотехно\-логичным информационным и вычислительным 
ресурсам~--- отдельным компьютерам, кластерам и\linebreak суперкомпьютерным 
центрам, хранилищам информации, сетям, научному инструментарию 
и~т.\,д.~[2]. 

   \begin{table*}\small
   \begin{center}
   \Caption{Уровни иерархической модели и их функции~\cite{4-kor}}
   \vspace*{2ex}
   
   \begin{tabular}{|l|l|}
   \hline
\multicolumn{1}{|c|}{Уровень} & \multicolumn{1}{c|}{Функции}\\
\hline
Tier0&Первичная реконструкция событий, калибровка, хранение копий полных баз данных\\
\hline
Tier1&\tabcolsep=0pt\begin{tabular}{l}Полная реконструкция событий, хранение 
актуальных баз данных по событиям, 
создание\\ и хранение наборов анализируемых событий, моделирование, анализ\end{tabular}\\
\hline
Tier2&Репликация и хранение наборов анализируемых событий, моделирование, анализ\\
\hline
\end{tabular}
\end{center}
\vspace*{-9pt}
\end{table*}
   \begin{table*}\small
   \begin{center}
   \Caption{Функции и свойства симуляторов грид~\cite{7-kor}}
   \vspace*{2ex}
   
   \begin{tabular}{|l|c|c|c|c|c|c|}
   \hline
\multicolumn{1}{|c|}{Функция}&GridSim&OptorSim&Monarc&ChicSim&SimGrid&MicroGrid\\
\hline
Репликация данных&Да&Да&Да&Да&Нет&Нет\\
\hline
Издержки записи/чтения диска&Да&Нет&Да&Нет&Нет&Да\\
\hline
\tabcolsep=0pt\begin{tabular}{l}Комплексное фильтрование\\ или запросы данных\end{tabular}&Да&Нет&Нет&Нет&Нет&Нет\\
\hline
\tabcolsep=0pt\begin{tabular}{l}Планировка пользовательских\\ задач\end{tabular}&Да&Нет&Да&Да&Да&Да\\
\hline
\tabcolsep=0pt\begin{tabular}{l}Резервирование центрального\\ процессорного устройства\end{tabular}&Да&Нет&Нет&Нет&Нет&Нет\\
\hline
Симуляция нагрузки&Да&Нет&Нет&Да&Нет&Нет\\
\hline
\tabcolsep=0pt\begin{tabular}{l}Дифференцированное качество\\ обслуживания сети\end{tabular}&Да&Нет&Нет&Нет&Нет&Нет\\
\hline
\tabcolsep=0pt\begin{tabular}{l}Генерация фонового сетевого трафика\end{tabular}&Да&Да&Нет&Нет&Да&Да\\
\hline
\end{tabular}
\end{center}
\end{table*}
   
   Эксперименты, в которых для обработки данных используется 
   грид-ин\-фра\-струк\-ту\-ра или облачные вычисления, имеют некоторые 
общие черты: большие потоки данных, длительный цикл проектирования и 
строительства, длительный период эксплуатации. Так, компьютерная 
инфраструктура для эксперимента ALICE представляет собой иерархическую 
грид-структуру с компьютерными центрами класса Tier 0/1/2. Функциональные 
различия уровней иерархической модели представлены в табл.~1. Для хранения 
и обработки данных в эксперименте PANDA~\cite{3-kor} также предполагается 
использование грид. 
   



   Проектирование грид-структур больших мас\-шта\-бов подразумевает не только 
привлечение специалистов, обладающих уникальными навыками,\linebreak но и 
применение инструментов для {моделирования}. При создании распределенной 
системы требуется принять решения по архитектуре инфраструктуры, 
количеству ресурсных центров, объему \mbox{требуемых} ресурсов. Кроме того, 
необходимо обеспечить достаточную пропускную способность, решить 
проблемы сохранности данных, обеспечить распределение ресурсов между 
различными группами пользователей, выбрать алгоритмы обработки и запуска 
задач и многое другое. Для решения этих вопросов, а также обоснования 
решений требуется создание имитационной модели обработки данных 
эксперимента. Возникает необходимость создания имитационной модели, 
которая бы удовлетворяла всем условиям. 
   
   Актуальность темы обусловливается тем, что на основе модели в дальнейшем 
могут быть обоснованы рекомендации и техническое задание на разработку 
компьютерной инфраструктуры, рас\-смот\-ре\-ны различные варианты 
организации хранения данных эксперимента. 

\section{Выбор пакета моделирования}

   На сегодняшний день существуют различные инструменты моделирования 
грид-сис\-тем~\cite{5-kor}. Проект GridSim разрабатывается группой 
исследователей в лаборатории по изучению облачных и распределенных 
вычислений отдела информатики и компьютерных вычислений в Университете 
Мельбурна, Австралия. Пакет моделирования GridSim неоднократно 
применялся~\cite{6-kor} для моделирования грид-струк\-тур и планировщиков.
   
   GridSim~--- это библиотека классов, предназначенных для построения 
модели грид-системы. Она, в свою очередь, построена на стандартной 
библиотеке SimJava, с помощью которой можно моделировать поток 
дискретных событий во времени. Приложение создается расширением классов 
GridSim и объединением их в программу, которая моделирует обработку потока 
заданий грид-струк\-ту\-рой, обладающей определенными ресурсами и с 
заданной дисциплиной их резервирования и использования. В~сравнении с 
другими пакетами моделирования грид GridSim обладает рядом преимуществ. 
Основные преимущества представлены в табл.~2. 

С~по\-мощью GridSim можно 
проводить воспроизводимые эксперименты, которые сложно реализовать в 
настоящем окружении динамических грид-сис\-тем.
   


   После анализа целого ряда систем для разработки имитационной модели 
была выбрана платформа GridSim. 
  
\section{Моделирование сайта уровня Т1 грид-структуры}

   В качестве примера грид-струк\-ту\-ры уровня T1 будет рассмотрен 
   Оф\-лайн-уро\-вень обработки физических данных ускорительного 
комплекса \mbox{НИКА}. Для эффективной работы грид-сай\-та, проведения 
исследований по оптимизации нагрузки, разработки и тестирования новых 
алгоритмов с точки зрения скорости достижения результата необходимо 
использовать средства моделирования грид-сис\-тем. При создании модели 
предполагается, что основой для построения системы хранения данных будет 
dCache~\cite{8-kor}. Модель сайта Т1 строится на следующем алгоритме 
обработки данных (см.\ рис.~1): 
   \begin{enumerate}[(1)]
\item данные появляются с заданной частотой и записываются на 
локальные диски компьютеров. После перемещения данных на второй 
уровень диск очищается;
\item данные перемещаются автоматически на второй уровень по 
каналам. В качестве носителей второго уровня используются пулы 
системы dCache, рассматриваемые в модели как единая память. При 
обработке данных предполагается, что вначале данные попадают в 
дисковый пул системы хранения, а затем по локальному протоколу 
передается на узлы обработки. Непосредственное монтирование 
директории на рабочих узлах не используется;
\item для долгосрочного хранения данных используется ленточный 
робот. Копии файлов автоматически создаются на лентах, после чего 
файлы удаляются с дисковых пулов. 
\end{enumerate}

   Отличительная особенность конфигурации dCache~--- наличие не менее двух 
уровней хранения: жесткие диски и ленточный накопитель. Под ленточным 
накопителем подразумеваются автоматизированные биб\-лио\-те\-ки, 
оснащенные роботизированным загрузочным механизмом и стойкой на 
несколько картриджей (лент). Объем такой биб\-лио\-те\-ки~$Q$ можно определить 
простейшими вы\-чис\-ле\-ни\-ями, исходными данными для которых будут 
производительность установки~$p$, время ее работы~$T$ и емкость 
накопителей~$c$:
   $$
   Q=\fr{pT}{c}\,.
   $$
   
   Другие вопросы создания грид-сай\-та требуют более тщательного анализа и 
выбора приемлемого варианта. Таким образом, перед разработчиками системы 
встают следующие вопросы:
   \begin{itemize}
   \item определение необходимого количества драйвов;
\item способы группировки файлов на лентах;
\item политика записи файлов.
\end{itemize}

   Стоит отметить ряд ключевых особенностей GridSim, которые потребовали 
доработки из-за несоответствия требованиям модели: 
   \begin{itemize}
   \item создавать файлы может только пользователь;
\item все объекты моделирования объединены в сеть при помощи 
каналов передачи данных;
\item пользователь может копировать (создавать) только один файл 
единовременно. 
\end{itemize}
   
   Для решения этих вопросов потребовалось расширение существующих 
классов и добавление новых объектов. Так, в сис\-те\-му добавлены следующие 
объекты (рис.~2):
   \begin{itemize}
\item Drive~--- драйв магнитофона;
\item Arm~--- рука робота;
\item Reel archive~--- архив картриджей;
\item Reel~--- картридж.
\end{itemize}




   Набор этих классов позволяет моделировать все процессы, происходящие с 
копией файла на лентах: загрузку и выгрузку ленты манипулятором, 
монтирование на драйве, поиск файла на ленте и его чте\-ние/за\-пись.
   
   Задача моделирования сетевой инфраструктуры в библиотеке GridSim 
решена с помощью классов Router, Link, NetPacket и некоторых других. Этот 
набор средств позволяет моделировать прохождение пакетов по сети. 
Пользователю предоставляется возможность встраивать свои планировщики 
пакетов в исходную модель. Такой подход обеспечивает высокую точность 
моделирования. Его недостатком применительно к задаче моделирования Т1 
является избыточность~--- вопросы маршрутизации, столкновений пакетов, 
влияния фоновой загрузки каналов в данной модели не рассматриваются и, 
следовательно, уровень детализации до пакета представляется избыточным. 
В~рас\-смат\-ри\-ва\-емом случае интерес представляет только изменение 
нагрузки на отдельные компоненты сети. 


\pagebreak

\end{multicols}

\begin{figure} %fig2
\vspace*{1pt}
 \begin{center}
 \mbox{%
 \epsfxsize=160.035mm
 \epsfbox{kor-2.eps}
 }
 \end{center}
 \vspace*{-6pt}
\Caption{Описание новых классов в модели }
%\vspace*{6pt}
\end{figure}

\begin{multicols}{2}


   
   Исходя из вышеизложенного требуется дополнить GridSim следующим 
механизмом. 

Вводится понятие \textit{операции передачи данных}. Под этим 
подразумевается за\-пись/чте\-ние час\-ти или целого файла экспериментальных 
данных. В~этом случае ввод и вывод служебной и диагностической 
информации считается пренебрежимо малым. Операция 
   рас\-смат\-ри\-ва\-ет\-ся как атомарная, т.\,е.\ начинается методом <<начать 
операцию>>. Параметрами метода являются: устройство~1~--- источник 
данных,
 устройство~2~--- получатель и список всех промежуточных устройств, 
которые необходимо пройти от источника до получателя. Элемент сети в 
системе описывается классом Stange. Взаимодействие классов, описывающих 
сеть, отражено на рис.~3.

\begin{figure*} %fig3
\vspace*{9pt}
 \begin{center}
 \mbox{%
 \epsfxsize=97.571mm
 \epsfbox{kor-3.eps}
 }
 \end{center}
 \vspace*{-6pt}
\Caption{Взаимодействие классов, описывающих сеть}
\end{figure*}

   Результаты моделирования доступны пользователю в виде таблиц и 
графиков. Для этой цели используются классы генератора лога и визуального 
отображения результатов:
%   \begin{itemize}
%\item 
Info~--- описание вы\-чис\-ли\-тель\-ной структуры и потока заданий;
%\item 
Reporter~--- генератор лога;
%\item 
парсер лога;
%\item 
объект визуального отображения результатов и~др.
%\end{itemize}

   Ниже приведем пример задачи, которая возникает при проектировании 
системы сбора и хранения данных. 

\section{Пример использования модели}

   С помощью системы моделирования можно исследовать прохождение набора 
заданий и передачу файлов через грид-струк\-ту\-ру с заданной пользователем 
топологией и параметрами центров обработки. Модель позволяет получить 
оценку временн$\acute{\mbox{ы}}$х параметров обработки потока заданий при заданной 
пользователем дисциплине распределения ресурсов между заданиями и 
структурой очередей к центрам обработки.
   
   Моделирование дает ответы на вопросы:
   \begin{itemize}
   \item какие вычислительные ресурсы требуются для обработки данных;
   \item как должны быть связаны между собой центры обработки;
   \item каким должен быть уровень сжатия данных;
   \item какой должна быть конфигурация роботизированной библиотеки;
   \item хватит ли ресурсов на обработку потока данных и предоставление 
данных пользователям. 
   \end{itemize}
   
   Проиллюстрировать применение упомянутых выше классов можно на 
примере моделирования процесса обработки данных с одновременной \mbox{записью} 
на ленты. Задача проектировщика~--- определить необходимое количество 
драйвов библиотеки. При этом исследуются два вопроса: какое количество 
драйвов библиотеки необходимо для того, чтобы записать весь поток 
<<сырых>> (RAW) данных с детекторов эксперимента, и насколько при этом 
процесс обработки данных (поток заданий от пользователей) будет мешать 
записи, если обработка потребует загрузки файлов с лент на диски.
   
   Допустим, что имеется в распоряжении биб\-ли\-о\-тека, количество драйвов в 
библиотеке фик\-си\-ровано и равно пяти. Это существенно меньше 
необходимого, но достаточно для иллюстрации возмож\-но\-стей модели. Когда 
для обслуживания поступающих на сайт заданий и записи RAW-дан\-ных 
используются одни и те же пулы (драйвы), процесс начинает вести себя 
хаотично, многократно монтируя и размонтируя ленты для записи даже при 
незначительных загрузках. Для того чтобы избежать этой ситуации, в 
рассматриваемой модели пулы лент разделены на принимающие данные 
(RAW) и обслуживающие поток заданий (DLT~--- digital linear tape). Возникает вопрос: каким 
образом распределить драйвы между двумя пулами при фиксированных 
параметрах потока заданий? Предполагаем, что файлы запрашиваются 
случайным образом. 
   
   Моделируемая система~--- двухуровневая. На первом уровне находится 
дисковый массив, на втором уровне~--- ленточный накопитель. 
В~существующей модели скорость записи и чтения с дискового массива не 
зависит от загрузки. Параметры драйвов и робота соответствуют параметрам 
планируемых к установке устройств (табл.~3). Количество драйвов в роботе 
фиксировано, и есть только одна <<рука>>, загружающая файлы в драйв.
   
   \begin{table*}\small %tabl3
   \begin{center}
   \Caption{Параметры для моделирования ленточной библиотеки}
   \vspace*{2ex}
   
   \tabcolsep=10pt
   \begin{tabular}{|l|c|}
   \hline
\multicolumn{1}{|c|}{Параметр}&Значение\\
\hline
Время монтирования/размонтирования, с&\hphantom{99}22\\
Скорость поиска, с&\hphantom{9}300\\
Скорость чтения/записи, c&\hphantom{9}120\\
Скорость перемотки, с&1000\\
Время загрузки/разгрузки картриджа в драйв, с&\hphantom{9}100\\
Размер файла, МБ&6000\\
\hline
\end{tabular}
\end{center}
\vspace*{-9pt}
\end{table*}

   Результаты моделирования приведены в табл.~4. С~помощью модели 
исследовались следующие характеристики: 
   \begin{itemize}
\item время выполнения~--- астрономическое время выполнения потока 
заданий, которое из общих соображений будет уменьшаться с 
увеличением количества драйвов;
\item длина очереди~--- максимальная длина очереди на запись 
RAW-дан\-ных на ленту. 
\end{itemize}

\begin{table*}\small %tabl4
\begin{center}
\Caption{Результаты моделирования}
\vspace*{2ex}

\begin{tabular}{|c|c|c|c|c|}
\hline
Эксперимент&Драйвов RAW&Драйвов DLT&Время выполнения, с&Длина очереди\\
\hline
1&1&4&28\,959&13\hphantom{9}\\
2&2&3&28\,703&1\\
3&3&2&28\,814&1\\
4&4&1&59\,275&1\\
\hline
\end{tabular}
\end{center}
\vspace*{-6pt}
\end{table*}
   
   Моделирование показало, что при заданном темпе сбора данных для записи 
должно быть выделено не менее двух драйвов. С~другой стороны, для 
обработки потока заданий должно быть выделено не менее двух драйвов для 
чтения накопленной информации. Если за критерий оп\-ти\-маль\-ности принять 
минимальное астрономическое время выполнения потока заданий, то 
оптимальным можно считать распределение драйвов по варианту №\,2.
   
   Этот пример иллюстрирует один из вариантов использования программы. 
Такие исследования могут быть проведены с использованием аналитических 
моделей теории массового обслуживания, однако добавление простейших 
условий группировки заданий и файлов значительно усложняет аналитические 
модели, тогда как для имитационной модели изменения сводятся к нескольким 
строчками программного кода. 

\section{Заключение}

   Созданная система моделирования позволяет проводить разнообразные 
эксперименты с исследуемым объектом, не прибегая к физической реализации. 
Это позволяет предсказать и предотвратить большое число неожиданных 
ситуаций в процессе эксплуатации, которые могли бы при\-вес\-ти к 
неоправданным затратам, потере данных, а возможно, и к по\-вреж\-де\-нию 
дорогостоящего обору\-до\-ва\-ния. В~процессе моделирования можно подобрать 
минимально необходимое оборудование, обеспечивающее потребности 
передачи, обработки и хранения данных, оценить необходимый запас 
производительности оборудования, обеспе\-чи\-ва\-юще\-го возможное увеличение 
производственных потребностей, выбрать несколько вариантов оборудования с 
учетом текущих потребностей и перспективы развития в будущем, провести 
проверку работы сис\-те\-мы, выявить ее <<узкие>> места и~т.\,д.
   
   Применение системы моделирования позволит определить параметры 
системы обработки данных ускорительного комплекса НИКА на этапе 
технического проектирования. 
   
   Дальнейшее развитие системы предполагает внесение дополнений с целью 
создания модели грид-сай\-та уровня Т1 с использованием двух и трех уровней 
dCache. Для моделирования предполагается использовать оригинальный 
алгоритм назначения пулов dCache и оригинальные данные по потокам. Также 
необходимо провести полномасштабные испытания модели с целью выявления 
ошибок и создания базы сценариев моделирования. 
   
   Важное значение разработанной системы моделирования связано с 
созданием в Объединенном институте ядерных исследований 
автоматизированной системы обработки и хранения данных (АСОД) уровня T1 
для эксперимента CMS (Compact Muon Solenoid) на Большом адронном коллайдере и предназначенной 
для работы в составе глобальной грид-сис\-те\-мы для обработки данных 
(WLCG~--- Worldwide LHC Computing Grid). Автоматитзированная система обработки и хранения данных нацелена на проведение полного цикла обработки физической 
информации, получаемой в ходе проведения эксперимента, обеспечения работ 
по моделированию физических процессов, защищенного хранения и 
при\-ема/пе\-ре\-да\-чи данных в другие центры WLCG. Основной системой хранения 
данных в АСОД является dCache. Очевидно, что в процессе длительного 
(10~лет и более) функционирования центра будет необходимо оперативно 
масштабировать систему хранения и повышать эффективность использования 
ленточного робота в системе dCache без остановки работы всего комплекса. 
В~этом процессе предварительное моделирование работы системы хранения 
станет необходимым инструментом.
   
   Результаты работы могут быть рекомендованы для использования при 
проектировании грид-сис\-те\-мы для сбора, передачи, обработки и хранения 
данных с мегаустановок или других аналогичных установок, генерирующих 
большие объемы данных.

{\small\frenchspacing
{%\baselineskip=10.8pt
\addcontentsline{toc}{section}{Литература}
\begin{thebibliography}{9}
  
\bibitem{1-kor}
\Au{Cortese P., Carminati F., Fabjan C.\,W., \textit{et al.}} ALICE Technical Design Report of the Computing~// 
CERN/LHCC 2005-018, ALICE TDR~12, 2005.
\bibitem{2-kor}
\Au{Кореньков~В.\,В.} Грид-тех\-но\-ло\-гии: статус и перспективы~// Вест\-ник 
Международной академии наук. Русская секция, 2010. №\,1. C.~41--44.

\bibitem{4-kor} %3
\Au{Ильин~В.\,А., Кореньков~В.\,В., Солдатов~А.\,А.} Российский сегмент 
глобальной инфраструктуры LCG~// Открытые системы, 2003. №\,1. C.~56--60.

\bibitem{3-kor} %4
Веб-пор\-тал проекта PANDA. {\sf http://www-panda.gsi.de}.

\bibitem{5-kor}
\Au{Нечаевский~А.\,В., Кореньков~В.\,В.} Пакеты моделирования DataGrid~// 
Сис\-тем\-ный анализ в науке и образовании: Электронный журнал, 2009. №\,1.
\bibitem{6-kor}
Веб-пор\-тал проекта GridSim. {\sf http://www.gridbus.org/ gridsim}.

\bibitem{7-kor}
\Au{Sulistio~A., Cibej~U., Venugopal~S., Robic~B., Buyya~R.} A~toolkit for 
modelling and simulating data grids: An extension to GridSim~// Concurrency  
Computation Practice Experience (CCPE), 2008. Vol.~20. No.\,13. 
P.~1591--1609.


\label{end\stat}

\bibitem{8-kor}
Веб-пор\-тал проекта dCache. {\sf http://www.dcache.org}.


\end{thebibliography}
} }

\end{multicols}