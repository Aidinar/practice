\def\stat{listopad}

\def\tit{МОДЕЛИРОВАНИЕ СИСТЕМ ПОДДЕРЖКИ ПРИНЯТИЯ РЕШЕНИЙ  
СИНЕРГЕТИЧЕСКИМ ИСКУССТВЕННЫМ ИНТЕЛЛЕКТОМ}

\def\titkol{Моделирование систем поддержки принятия решений  
синергетическим искусственным интеллектом}

\def\autkol{И.\,А.~Кириков, А.\,В.~Колесников, С.\,В.~Листопад}

\def\aut{И.\,А.~Кириков$^1$, А.\,В.~Колесников$^2$, С.\,В.~Листопад$^3$}

\titel{\tit}{\aut}{\autkol}{\titkol}

%{\renewcommand{\thefootnote}{\fnsymbol{footnote}}\footnotetext[1] {Статья рекомендована к публикации в журнале Программным комитетом конференции <<Электронные 
%библиотеки: перспективные методы и технологии, электронные коллекции>> (RCDL-2012).}}

\renewcommand{\thefootnote}{\arabic{footnote}}
\footnotetext[1]{Калининградский филиал Института проблем информатики Российской академии наук,  
baltbipiran@mail.ru}
\footnotetext[2]{Калининградский филиал Института проблем информатики Российской академии наук,  
avkolesnikov@yandex.ru}
\footnotetext[3]{Калининградский филиал Института проблем информатики Российской академии наук,  
ser-list-post@yandex.ru}

  
  \Abst{Рассматривается подход к моделированию коллективных эффектов систем 
поддержки принятия решений в рамках синергетической парадигмы искусственного 
интеллекта. Приведена модель и функциональная структура гибридной 
интеллектуальной многоагентной системы (ГиИМАС) для моделирования систем поддержки 
принятия решений (СППР). Представлены результаты вычислительных экспериментов, 
демонстрирующие положительное влияние эффекта самоорганизации на качество 
коллективных решений.}
  
  \KW{компьютерная система поддержки принятия решений; гибридная интеллектуальная 
многоагентная система с самоорганизацией}

\vskip 14pt plus 9pt minus 6pt

      \thispagestyle{headings}

      \begin{multicols}{2}

            \label{st\stat}
  
  
\section{Введение}

  Люди ежедневно сталкиваются с принятием решений. Одни решения даются 
легко, часто без обдумывания, в некоторых случаях обращаются к коллегам, 
справочной литературе и другим источникам информации. Это и называется 
поддержкой принятия решений~[1]. В~наиболее сложных случаях нужно 
учитывать противоречивые требования, оценивая множество альтернатив по 
нескольким критериям. Тогда создается СППР~--- 
коллектив экспертов под управлением лица, принимающе-\linebreak

\begin{center}  %fig1
 \mbox{%
 \epsfxsize=77.058mm
 \epsfbox{lis-1.eps}
 }
 \end{center}
 \vspace*{6pt}
 
\noindent{{\figurename~1}\ \ \small{Концептуальная модель СППР: \textit{1}~--- участники СППР;
  \textit{2}~--- отношения между участниками \mbox{СППР};
  \textit{3}~--- возможность расширения количества экспертов \mbox{СППР} и установления отношений между ними
  и другими участниками СППР}}


%\pagebreak

%\vspace*{15pt}

\addtocounter{figure}{1}


 

\noindent
го 
решения (ЛПР). Концептуальная модель СППР показана на рис.~1~[2]. 
  
  Согласно данному определению СППР, процесс принятия решений в ней~--- 
частный случай коллективного принятия решений в малых группах, изучаемых 
со второй четверти XX~в.\ социальной психологией и социологией, где 
получены результаты по увеличению эффективности работы индивидов в 
группе, а также выявлению возникающих при этом коллективных 
  эффектов~[3, 4]. 
  

  
  В то время как люди уже давно решают сложные задачи только коллективно, 
в СППР создаваемые интеллектуальные информационные системы все еще не 
релевантны этим целям~[5]. В~этой связи актуально научить ЭВМ работать в 
условиях сложных задач не хуже коллектива специалистов~[6], для чего важны 
исследования коллективных эффектов, построение их моделей и создание 
компьютерных СППР (КСППР). 

\begin{table*}[b]\small
\vspace*{-9pt}
\begin{center}
\Caption{Коллективные эффекты в СППР}
\vspace*{2ex}

\begin{tabular}{|p{19mm}|p{40mm}|p{44mm}|p{46mm}|}
\hline

\multicolumn{1}{|c|}{Эффект}&\multicolumn{1}{c|}{Краткое описание}&\multicolumn{1}{c|}{Позитивное 
влияние}&\multicolumn{1}{c|}{Негативное влияние}\\
\hline
1.~Бумеранг&При недоверии к информации возникает мнение, обратное содержащемуся в ней&Информация не 
воспринимается либо считается заведомо ложной&Достоверная информация от недостоверного источника 
может быть расценена как ложная\\
\hline
2.~Волна&Распространение идей в СППР, отвечающих интересам ее членов&Коллективная доработка идей 
&Длительная работа экспертов над бесперспективными идеями\\
\hline
3.~Групповой эгоизм&Цели коллектива важнее целей ее члена и целей общества& &Эффективная деятельность 
коллектива может вредить обществу\\
\hline
4.~Групсинк и конформизм&Общее мнение~--- истина, мнение отдельного человека~--- ничто& &Препятствует 
возникновению новых подходов к решению проб\-ле\-мы\\
\hline
5.~Мода (подражание)&Добровольное принятие установившейся в коллективе точки зрения на проб\-ле\-му&Основа 
самообучения членов коллектива; способствует адап\-та\-ции людей друг к другу&Снижает вероятность 
появления оригинальных взглядов и подходов к решению проб\-лемы\\
\hline
6.~Ореол&В условиях дефицита времени объекты могут восприниматься на основе стереотипов&Быстрое 
формирование мнения (модели) о некоторой сущности&Формируемые модели неточны и могут приводить к 
ошибочным решениям\\
\hline
7.~При\-над\-леж\-ность к\newline группе&Члены коллектива реагируют на других людей с позиций коллектива, а не с 
позиций личности&Способствует сближению моделей внешнего мира у членов группы и упрощает 
взаимодействие между ними&Гиперболизация эффекта ведет коллектив к переоценке возможностей, отрыву от 
других коллективов\\
\hline
8.~Пульсар&Изменение активности коллектива в зависимости от стимулов&Эффективное решение срочных задач 
(рост скорости при том же качестве), снижение активности в другое время&На отдельных этапах решения 
задачи возникает перенапряжение членов коллектива\\
\hline
9.~Рингельмана&С ростом численности группы снижается индивидуальный вклад в общую работу&Снижается 
нагрузка на отдельных участников СППР&Снижается мотивация экспертов к эффективной совместной работе\\
\hline
10.~Само\-ор\-га\-ни\-за\-ция&Отношения между экспертами динамичны и изменяются в процессе работы&
Адаптация к 
внешней среде: каж\-дый раз вырабатывать новый метод, релевантный задаче. Появление оригинальных 
подходов к решению задачи и синергии&Затрудняет анализ работы и внешнее управление коллективом\\
\hline
11.~Синергия&Получение общего результата, который не могут получить эксперты индивидуально&Получение 
эмерджентного качественно лучшего результата&Возможен эффект отрицательной синергии (дисергии) \\
\hline
12.~Социальная фасилитация &Усиление доминантных реакций в присутствии других людей &Ускоряет решение 
простых задач, на которые индивид знает ответ&В сложных задачах повышает вероятность ошибочного 
ответа\\
\hline
\end{tabular}
\end{center}
\vspace*{-12pt}
\end{table*}
   
\section{Коллективные эффекты в~системах поддержки 
принятия~решений}
  
  Коллективные эффекты~--- механизмы функционирования коллектива 
людей, за счет которых осуществляются коллективные процессы и достигаются 
коллективные состояния; средства, интегрирующие индивидуальные действия 
в коллективной работе и общении~[7] (ниже названы макро\-уровневыми 
процессами). В~результате анализа литературы по социальной 
  психологии~[7--10] и синергетическому искусственному 
  интеллекту~\cite{5-lis, 11-lis, 12-lis} выявлено 12~основных коллективных 
эффектов (табл.~1).
  
  Как показывает анализ табл.~1, все эффекты в той или иной степени 
сказываются на качестве принимаемых в СППР решений. Особую и 
основополагающую роль среди них играет самоорганизация~--- предпосылка 
возникновения других коллективных эффектов, положительно влияющих на 
результат работы СППР, например синергии~\cite{5-lis}.



  Как показал анализ подходов к моделированию СППР с использованием 
методов синергетического искусственного интеллекта~\cite{5-lis}, среди 
которых гибридные сис\-те\-мы~\cite{13-lis}, интегрированные экспертные 
системы~\cite{14-lis}, гибридные интеллектуальные системы (ГиИС)~[2], 
многоагентные системы (МАС)~\cite{11-lis}, все они позволяют моделировать 
те или иные понятия, используемые при описании СППР. Тем не менее только 
МАС релевантны моделированию коллективных эффектов СППР, однако в 
МАС не оговаривается, какие интеллектуальные технологии должны 
использовать агенты МАС. В~случае гомогенных агентов МАС будет 
неспособна решать сложные неоднородные задачи, так как многие аспекты 
сложной задачи выпадут из <<поля зрения>> всех ее агентов. В~связи с этим в 
данной работе в качестве модели СППР исследуется новый класс 
интеллектуальных сис\-тем~--- ГиИМАС. Они объединяют преимущества МАС и ГиИС, с одной 
стороны, имитируя коллективные эффекты в СППР, а с другой~--- интегрируя 
разнородные технологии моделирования интеллектуальной деятельности 
человека. 
  
\section{Модель самоорганизации в~системах поддержки принятия~решений} 
  
  Самоорганизация в СППР рассматривается как изменение ЛПР отношений 
между экспертами на основе анализа их целей:
  \begin{multline*}
  \mathrm{so}^{\mathrm{goa}} = r_2^{\mathrm{res\,act}} (\mathrm{dss}, \mathrm{ACT}^{\mathrm{sen}})\circ \\
  \circ
r_1^{\mathrm{act\,res}}(\mathrm{ACT}^{\mathrm{sen}},\mathrm{env})\circ 
R_1^{\mathrm{\mathrm{res\,res}}}(\widetilde{\mathrm{DSS}},\widetilde{\mathrm{DSS}})\circ\\
  \circ  r_3^{\mathrm{res\,res}}(\mathrm{dss},\mathrm{prt}^{\mathrm{dm}})\circ 
  r_2^{\mathrm{res\,act}}(\mathrm{prt}^{\mathrm{dm}},\mathrm{act}_{\mathrm{ia}})\circ \\
  \circ
r_1^{\mathrm{act\,res}}(\mathrm{act}_{\mathrm{ia}},\widetilde{\mathrm{dss}}_{\mathrm{cur}})\circ\\
  \circ r_2^{\mathrm{res\,act}} (\mathrm{prt}^{\mathrm{dm}},\mathrm{act}_{\mathrm{ac}})\circ 
  r_1^{\mathrm{act\,res}}(\mathrm{act}_{\mathrm{ac}},\widetilde{\mathrm{DSS}})\circ \\
  \circ
r_2^{\mathrm{act\,res}} (\mathrm{act}_{\mathrm{ac}},\widetilde{\mathrm{dss}}_{\mathrm{des}})\,,
  \end{multline*}
где $\mathrm{so}^{\mathrm{goa}}$~--- концептуальная модель самоорганизации СППР на основе анализа 
целей экспертов; $\mathrm{dss}$~--- концептуальная модель СППР; $\mathrm{ACT}^{\mathrm{sen}}$~--- 
множество действий по восприятию внешней среды;   env~--- концептуальная 
модель внешней среды; $\widetilde{\mathrm{DSS}}$~--- множество возможных в данной СППР 
ситуаций коллективного решения; prt$^{\mathrm{dm}}$~--- ЛПР; $\mathrm{act}_{\mathrm{ia}}$~--- действие ЛПР 
<<идентификация текущей ситуации коллективного решения>>; 
$\widetilde{\mathrm{dss}}_{\mathrm{cur}}$~--- текущая ситуация коллективного решения; 
act$_{\mathrm{ac}}$~--- действие ЛПР <<выбор желаемой ситуации коллективного решения из 
множества возможных>>; $\widetilde{\mathrm{dss}}_{\mathrm{des}}$~--- желаемая ЛПР с точки зрения 
параметров задачи и его знаний об эффективности той или ситуации из $\mathrm{DSS}$; 
$r_2^{\mathrm{res\,act}}$~--- отношение <<выполнять>> для субъекта и выполняемого им 
действия; $r_1^{\mathrm{res\,act}}$~--- отношение <<иметь объектом>> для действия и его 
ресурса; $R_1^{\mathrm{res\,res}}$~--- множество отношений <<следовать за>> класса 
<<ре\-сурс--ре\-сурс>>; $r_3^{\mathrm{res\,res}}$~--- отношение <<включать>>; 
$r_2^{\mathrm{act\,res}}$~--- отношение <<иметь результатом>> для действия и его 
результата.
  
  Процесс самоорганизации в СППР моделируется с использованием  
ГиИМАС, определяемой следующим образом:
  \begin{gather}
  \mathrm{himas}= \langle \mathrm{AG}^*, \mathrm{env}, \mathrm{INT}^*, \mathrm{ORG}, 
  \{\mathrm{so}^{\mathrm{goa}}\}\rangle\,;\label{e1-lis}\\
  \mathrm{AG}^*=\{ \mathrm{ag}_1, \ldots , \mathrm{ag}_n, \mathrm{ag}^{\mathrm{dm}}\}\,;\label{e2-lis}\\
  \mathrm{INT}^*=\{ \mathrm{prot}, \mathrm{lang}, \mathrm{ont}, \mathrm{rcl}\}\,; \label{e3-lis}
  \end{gather}
  \begin{equation}
  \left.
  \begin{array}{c}
  \mathrm{ORG}=\mathrm{ORG}_{\mathrm{coop}}\cup \mathrm{ORG}_{\mathrm{neut}}\cup \mathrm{ORG}_{\mathrm{comp}}\,;\\[9pt]
  \mathrm{ORG}_{\mathrm{coop}}\cap \mathrm{ORG}_{\mathrm{neut}}=\varnothing\,;\\[9pt]
  \mathrm{ORG}_{\mathrm{coop}}\cap \mathrm{ORG}_{\mathrm{comp}}=\varnothing\,;\\[9pt]
  \mathrm{ORG}_{\mathrm{comp}}\cap \mathrm{ORG}_{\mathrm{neut}}=\varnothing\,;
  \end{array}
  \right\}
  \label{e4-lis}
  \end{equation} 
  
  \vspace*{-12pt}
  
  \noindent
  \begin{multline}
  \mathrm{act}_{\mathrm{himas}}={}\\
  {}=\left( \bigcup\limits_{\mathrm{ag}\in \mathrm{AG}^*} \mathrm{act}_{\mathrm{ag}}\right) \cup
\mathrm{act}_{\mathrm{ia}}\cup \mathrm{act}_{\mathrm{ac}}\cup \mathrm{act}_{\mathrm{col}}\,;\label{e5-lis}
\end{multline}

\vspace*{-12pt}

\noindent
\begin{multline}
  \mathrm{act}_{\mathrm{ag}}= (\mathrm{MET}_{\mathrm{ag}},\mathrm{IT}_{\mathrm{ag}})\,,\\
   \mathrm{ag}\in \mathrm{AG}^*\,,\enskip 
   \left\vert \bigcup\limits_{\mathrm{ag}\in  \mathrm{AG}^*} \mathrm{IT}_{\mathrm{ag}}\right\vert \geq 2\,;
   \label{e6-lis}
\end{multline}
\begin{equation}  
\mathrm{ag}=\mathrm{ag}\vee \mathrm{himas}\,,\label{e7-lis}
  \end{equation}
где AG$^*$~--- множество агентов~ag (моделей экспертов), включающее 
агента, принимающего решения (АПР)~--- ag$^{\mathrm{dm}}$, $n$~--- чис\-ло 
аген\-тов-экс\-пер\-тов; env~--- концептуальная модель внешней среды 
ГиИМАС; INT$^*$~--- элементы структурирования взаимодействий агентов: 
prot~--- протокол взаимодействия; lang~--- язык передачи сообщений; 
ont~--- модель предметной области; rcl~--- классификатор отношений 
агентов (классифицирует отношения между агентами в зависимости от их 
целей на классы нейтралитета, конкуренции и сотрудничества); ORG~--- 
множество архитектур ГиИМАС (ORG$_{\mathrm{coop}}$~--- с сотрудничающими; 
ORG$_{\mathrm{neut}}$~--- с нейтральными и ORG$_{\mathrm{comp}}$~--- с конкурирующими 
агентами); $\{\mathrm{so}^{\mathrm{goa}}\}$~--- множество концептуальных моделей 
макроуровневых процессов в ГиИМАС: $\mathrm{so}^{\mathrm{goa}}$~--- модель эффекта 
самоорганизации на основе анализа целей; act$_{\mathrm{himas}}$~--- функция ГиИМАС 
в целом; act$_{\mathrm{ag}}$~--- функция агента из множества~AG$^*$; act$_{\mathrm{ia}}$~--- 
функция <<анализ взаимодействий>> АПР ag$^{\mathrm{dm}}$ (модель действия ЛПР 
<<идентификация текущей ситуации коллективного решения>> act$_{\mathrm{ia}}$); 
act$_{\mathrm{ac}}$~--- функция <<выбор архитектуры>> АПР ag$^{\mathrm{dm}}$ (модель 
действия ЛПР <<выбор желаемой ситуации коллективного решения из 
множества возможных>> act$_{\mathrm{ac}}$); act$_{\mathrm{col}}$~--- коллективная функция 
ГиИМАС на межагентных отношениях $R^{\mathrm{res\,res}}$ (см.~(\ref{e8-lis})) 
и определяемая текущей архитектурой~org, 
конструируемая динамически в процессе функционирования системы; 
$\mathrm{met}_{\mathrm{ag}}$~--- метод решения задачи; it$_{\mathrm{ag}}$~--- интеллектуальная 
технология, в рамках которой реализован метод met$_{\mathrm{ag}}$.
  
  Для реализации функции АПР <<анализ взаимодействий>> ag$^{\mathrm{dm}}$ 
вводится формализованное понятие нечеткой цели агента~pr~--- нечеткое 
множество с функцией принадлежности $\mu(\mathrm{st})$, заданное на множестве 
состояний~ST объекта управления (ОУ). Состояние $\mathrm{st}\hm\in \mathrm{ST}$ 
ОУ описывается набором его свойств $\mathrm{PR}\hm= \{\mathrm{pr}_1, \ldots , 
\mathrm{pr}_{N_{\mathrm{pr}}}\}$, т.\,е.\ $\mu(\mathrm{st})\hm= \mu
(\mathrm{pr}_1, \ldots , \mathrm{pr}_{N_{\mathrm{pr}}})$. Значение 
нечеткой цели определяется подстановкой в $\mu(\mathrm{pr}_1, \ldots , \mathrm{pr}_{N_{\mathrm{pr}}})$ 
значений свойств ОУ для данного состояния из множества $\mathrm{VAL}\hm= \{\mathrm{val}_1, 
\ldots , \mathrm{val}_{N_{\mathrm{val}}}\}$, т.\,е.\ 
$\mu(\mathrm{val}_1, \ldots , \mathrm{val}_{N_{\mathrm{val}}})$. Вводится мера сходства 
нечетких целей~$A$ и~$B$ для одномерного случая~[5]:
  \begin{multline*}
  s(A,B) ={}\\
  {}=\fr{1}{2}\left(\,  \int\limits_{\mathrm{val}_{\min}}^{\mathrm{val}_{\max}} \!\!\!\mu_{A\cap B 
}(\mathrm{pr})\, d\,\mathrm{pr}\!\right) \!
\left( \!\!\left( \, \int\limits_{\mathrm{val}_{\min}}^{\mathrm{val}_{\max}} 
\!\!\!\mu_A(\mathrm{pr})\,d\,\mathrm{pr}\!\right)^{\!-1}\!\!+{}\right.\\
  \left.{}+\left(\, \int\limits_{\mathrm{val}_{\min}}^{\mathrm{val}_{\max}} 
 \!\!\! \mu_B(\mathrm{pr})\,d\,\mathrm{pr}\right)^{-1}\right)\,.
  \end{multline*}
  
  На основе значения меры сходства нечетких \mbox{целей} отношения между 
агентами могут быть классифицированы на отношения конкуренции, 
нейт\-ра\-ли\-те\-та и сотрудничества, т.\,е.\ определена согласованность 
взаимодействия агентов. Представим класс отношения по согласованности 
взаимодействия нечеткими множествами на универсуме значений меры 
сходства целей~$s$ (множестве действительных чисел в интервале [0; 1]): 
\begin{align*}
\mu_{\mathrm{конк}}(s)&= (1+(3s)^8)^{-1}\,; \\
\mu_{\mathrm{нейтр}} (s) &= (1\hm+(6(s\hm-0{,}5))^8)^{-1}\,;\\ 
\mu_{\mathrm{сотр}}(s)&=(1\hm+(3(s\hm-1))^8)^{-1}\,.
\end{align*}
  
  Класс отношения между агентами ГиИМАС по согласованности их 
взаимодействия представим лингвистической переменной
  $$
\mathrm{cl}=\langle \beta, T,U,G,M\rangle\,,
  $$
где $\beta$\;= <<{класс отношений}>>~--- наименование 
лингвистической переменной; $T$\;=\;$\{$<<{конкуренция}>>; 
<<{нейтралитет}>>; <<{сотрудничество}>>$\}$~--- 
терм-мно\-же\-ст\-во ее значений, каж\-дое из которых~--- название нечеткой 
переменной; $U\hm=[0;1]$~--- универсум нечетких переменных; 
$G\hm=\varnothing$~--- процедура образования из элементов множества~$T$ 
новых термов; $M\hm= \{\mu_{\mathrm{конк}}(s), 
\mu_{\mathrm{нейтр}}(s), \mu_{\mathrm{сотр}}(s)\}$~--- 
процедура, ставящая в соответствие каждому терму множества~$T$ 
осмысленное содержание путем формирования соответствующего нечеткого 
множества.
  
  Когда для каждой пары агентов ГиИМАС определено значение~cl и 
составлена матрица \textbf{CL} (матрица классов отношений), она 
анализируется, чтобы идентифицировать текущую архитектуру ГиИМАС: с 
сотрудничающими, нейтральными или конкурирующими агентами~[5]. 
В~зависимости от па\-ра\-мет\-ров задачи АПР стремится установить одну из них, 
чтобы повысить эффективность работы ГиИМАС. 
  
  В результате имитации процессов самоорганизации определяется 
архитектура ГиИМАС: 
  \begin{equation}
\mathrm{org}=R^{\mathrm{res\,res}} (\mathrm{AG}^*,\mathrm{env}) \circ 
R^{\mathrm{res\,res}} (\mathrm{AG}^*, \mathrm{AG}^*)\,,
  \label{e8-lis}
  \end{equation}
где $R^{\mathrm{res\,res}}$~--- множество отношений <<ре\-сурс--ре\-сурс>>. 
  
  Учитывая, что функция act$_{\mathrm{ag}}$ агента $\mathrm{ag}\hm\in \mathrm{AG}^*$ выполняется 
множеством методов MET$_{\mathrm{ag}}$, концептуальная модель ГиИМАС как 
метода решения сложной задачи представляется выражением:
\begin{multline*}
  \mathrm{MET}_{\mathrm{himas}} =R^{\mathrm{met\,met}} (\mathrm{MET}_{\mathrm{ag}_i},
   \mathrm{MET}_{\mathrm{ag}_j})\,,\\ 
  \mathrm{ag}_i, \mathrm{ag}_j \in \mathrm{AG}^*\,,\enskip \mathrm{ag}_i\not=\mathrm{ag}_j\,,
  \end{multline*}
где метод MET$_{\mathrm{himas}}$, вырабатываемый ГиИМАС при решении сложной 
задачи, ~--- взаимосвязанная совокупность реализуемых агентами методов 
MET$_{\mathrm{ag}}$. При решении каждой задачи АПР анализирует взаимодействия 
act$_{\mathrm{ia}}$ между агентами, выбирает архитектуру act$_{\mathrm{ac}}$, определяя 
интенсивность и согласованность отношений $R^{\mathrm{met\,met}}$ между моделями 
знаний агентов, что рассматривается как выработка нового метода, 
релевантного ситуации решения сложной задачи. 

\begin{figure*}[b] %fig2
  \vspace*{12pt}
 \begin{center}
 \mbox{%
 \epsfxsize=152.951mm
 \epsfbox{lis-2.eps}
 }
 \end{center}
 \vspace*{-6pt}
\Caption{Универсальная функциональная структура ГиИМАС с самоорганизацией:
\textit{1}~--- взаимоотношения агентов (запросы информации, передача результатов
их решения);
\textit{2}~--- взаимоотношения агентов (запросы помощи в решении подзадач);
\textit{3}~--- взаимодействие (получение сведений из модели,
обновление модели) агентов с моделью предметной области}
\end{figure*}
  
  Для реализации функции АПР <<выбор архитектуры>> ag$^{\mathrm{dm}}$ 
необходима база знаний о ре\-ле\-вант\-ности архитектуры ГиИМАС ситуации 
решения за\-дачи:
  $$
  \mathrm{act}_{\mathrm{ac}}=r_1^{\mathrm{act\,mod}}(\mathrm{act}_{\mathrm{ac}}, 
  \mathrm{mod}_{\mathrm{ac}}) \circ r_1^{\mathrm{act\,alg}} 
(\mathrm{act}_{\mathrm{ac}},\mathrm{alg}_{\mathrm{ac}})\,,
  $$
где mod$_{\mathrm{ac}}$~--- модель нечеткого вывода; alg$_{\mathrm{ac}}$~--- алгоритм функции 
<<выбор архитектуры>>; $r_1^{\mathrm{act\,mod}}$~---\linebreak отношение между действием и 
его моделью; $r_1^{\mathrm{act\,alg}}$~--- отношение действия и его алгоритма. При 
разработке модели нечеткого вывода использован нечеткий вывод Мамдани~[1] 
с самообучением, в результате которого определяются четкие оценки степеней 
уверенности АПР в выборе архитектуры ГиИМАС, релевантной решаемой 
задаче. Используемый алгоритм самообучения модели нечеткого вывода 
Мамдани основан на методе обратного распространения ошибки~[1], 
применяющемся при обучении нейронных сетей.
  
  В итоге суть метода моделирования самоорганизации с использованием 
ГиИМАС состоит в выполнении логически упорядоченной последо\-ва\-тель\-ности 
нижеперечисленных действий: 
  \begin{enumerate}[(1)]
  \item анализ взаимодействий агентов; 
  \item выбор архитектуры: 
  \begin{enumerate}[({2}.1)]
  \item получение исходных данных; 
  \item вычисление по алгоритму нечеткого вывода Мамдани значений 
степеней уверенности в выборе архитектуры; 
  \item выбор архитектуры с вероятностью, пропорциональной степеням 
уверенности из~(2.2); 
  \item имитационный процесс решения задачи на архитектуре из~(2.3); 
  \item вычисление значения абсолютной ошибки нечеткого вывода в~(2.2); 
  \item корректировка функций принадлежности нечетких переменных по 
ошибке. 
  \end{enumerate}
   \end{enumerate}
   
\section{Гибридная интеллектуальная многоагентная система 
для~моделирования самоорганизации в~системах поддержки 
принятия решений}
  
  Для компьютерной реализации модели~(\ref{e1-lis})--(\ref{e7-lis}) 
разработана универсальная функциональная структура ГиИМАС (рис.~2). Она 
может быть использо\-вана для широкого круга задач, поскольку 
\begin{enumerate}[(1)]
\item использована общая многоагентная модель действительности; 
\item перечень 
аген\-тов-ре\-ша\-те\-лей охватывает пять классов методов из шести, 
используемых в ГиИС~[2]; 
\item порядок взаимодействия агентов определяется 
моделью предметной области и конкретными реализуемыми агентами 
алгоритмами, не специфицированными данной архитектурой, которые 
реализуются разработчиком автоматизированной системы для решения 
поставленной задачи.
\end{enumerate}
  
\begin{table*}[b]\small
\begin{center}
\Caption{Количественные параметры тестируемых задач}
\vspace*{2ex}

\begin{tabular}{|c|c|c|c|c|c|}
\hline
Задача&\tabcolsep=0pt\begin{tabular}{c}Количество\\ клиентов\end{tabular}&
\tabcolsep=0pt\begin{tabular}{c}Количество\\ дорог\end{tabular}&
\tabcolsep=0pt\begin{tabular}{c}Количество\\ водителей\end{tabular}&
\tabcolsep=0pt\begin{tabular}{c}Количество\\ грузчиков\end{tabular}&
\tabcolsep=0pt\begin{tabular}{c}Количество\\ транспортных\\ средств\end{tabular}\\
\hline
З\_10&10&\hphantom{9}75&3&3&3\\
З\_15&15&240&5&5&5\\
З\_20&20&420&5&5&5\\
З\_25&25&650&9&9&9\\
З\_30&30&377&6&6&6\\
\hline
\end{tabular}
\end{center}
\end{table*}


  Рассмотрим назначение ее агентов: 
  \begin{itemize}
\item интерфейсный агент запрашивает входные данные и выдает результат;
\item АПР рассылает агентам поиска решения (поисковикам) условия задачи, 
определяет порядок их взаимодействия с помощью функций <<анализ 
взаимодействия>> и <<выбор архитектуры>>. Когда последние решили задачу, 
он выбирает альтернативу и передает интерфейсному агенту или запускает 
новую итерацию решения задачи, рассылая решение остальным агентам 
поиска;
\item агенты поиска решения имеют знания о предметной области и используют 
муравьиный алгоритм~[5] для решения подзадач;
\item агент-посредник отслеживает имена, модели и возможности 
зарегистрированных агентов интеллектуальных технологий (решателей). 
Агенты обращаются к нему, чтобы узнать, какой из решателей может помочь в 
решении поставленной перед ними подзадачи;
\item решатели из верхней части рис.~2 вместе с 
аген\-том-пре\-обра\-зо\-ва\-те\-лем реализуют гибридную составляющую 
ГиИМАС, комбинируя разнородные знания, и предоставляют <<услуги>> 
агентам с использованием следующих моделей и алгоритмов: алгебраических 
уравнений для описания при\-чин\-но-след\-ст\-вен\-ных связей концептов 
предметной области; метода Мон\-те-Кар\-ло; продукционной экспертной 
системы с рассуждениями в прямом направлении; алгоритма нечеткого вывода 
Мамдани;
\item модель предметной области~--- семантическая сеть, основа 
взаимодействия агентов, по\-стро\-ена по концептуальной модели решаемой 
задачи. Агенты интерпретируют смысл получаемых сообщений на этой модели. 
  \end{itemize}

\section{Результаты экспериментов }

  Для оценки влияния эффекта самоорганизации на качество решений 
ГиИМАС проведены серии экспериментов, в которых требовалось решить 
сложную транс\-порт\-но-ло\-ги\-сти\-че\-скую задачу, т.\,е.\ найти для 
нескольких транспортных средств совокупность маршрутов, оптимальную по 
четырем критериям:
\begin{enumerate}[(1)]
\item суммарная стоимость маршрута; 
\item общая длительность 
поездок для всех транспортных средств; 
\item вероятность опоздания хотя бы к 
одному клиенту; 
\item надежность (мерой надежности будем считать математическое 
ожидание увеличения стоимости совокупности маршрутов)~[5]. 
\end{enumerate}

Учитывались 
такие стохастические факторы, как вероятность возникновения дорожных 
пробок и вероятность опоздания к клиенту, потери от боя груза и~др.
  
  Исходные данные:
  \begin{enumerate}[(1)]
  \item запросы клиентов на доставку грузов: наименование, 
количество товара, временной интервал его доставки; 
\item сведения о дорогах к 
клиентам: протяженность, загруженность, качество;
\item паспортные данные 
транспортных средств: расход го\-рю\-че-сма\-зоч\-ных материалов, 
грузоподъемность и~т.\,п.; 
\item сведения о графиках работы и заработной плате 
персонала: водителей и грузчиков; 
\item информация о грузе: вес, габариты, 
хрупкость и~т.\,п. 
\end{enumerate}
  
  Выходные данные: совокупность маршрутов доставки грузов (по одному на 
транспортное средство) и ее параметры: стоимость, длительность, надежность и 
вероятность опоздания, сводный критерий качества маршрута (среднее 
значение нечеткой цели АПР). Для тестирования разработаны пять задач, 
количественные параметры которых приведены в табл.~2.
  
\begin{figure*} %fig3
  \vspace*{9pt}
 \begin{center}
 \mbox{%
 \epsfxsize=113.371mm
 \epsfbox{lis-3.eps}
 }
 \end{center}
 \vspace*{-6pt}
\Caption{Зависимость числа ситуаций, когда коллективное решение лучше любого 
индивидуального, от числа клиентов в задаче}
\end{figure*}

  Исследовались три архитектуры ГиИМАС: с нейтральными, 
сотрудничающими и конкурирующими агентами. В ГиИМАС с нейтральными 
агентами каждый из четырех агентов поиска решения минимизирует значение 
<<своего>> критерия оценки решения. В ГиИМАС с сотрудничающими 
агентами все четыре аген\-та-по\-иско\-ви\-ка минимизируют все четыре 
критерия оценки решения (аналогично АПР). В ГиИМАС с конкурирующими 
агентами один агент минимизирует стоимость и максимизирует длительность, 
второй~--- максимизирует стоимость и минимизирует длительность, третий~--- 
минимизирует вероятность опоздания и максимизирует надежность, а 
четвертый~--- максимизирует вероятность опоздания и минимизирует 
надежность. 
  
  Качество тестовых решений оценивалось по объективным показателям и 
субъективно экспертами. Для пяти задач и каждой архитектуры ГиИМАС 
проведено по 100~вычислительных экспериментов. По всем задачам и 
архитектурам ГиИМАС, а также для архитектуры ГиИМАС без взаимодействия 
(аген\-ты-по\-иско\-ви\-ки не обменивались индивидуальными решениями) 
построены графические зависимости среднего значения цели агента, 
принимающего решения, средних значений стоимости, длительности, 
надежности, вероятности опоздания для маршрутов от числа клиентов, анализ 
которых показал высокое качество маршрутов, ре\-ко\-мен\-ду\-емых ГиИМАС. 

Одна из графических зависимостей, отображающая при\-чин\-но-след\-ст\-вен\-ную 
связь чис\-ла ситуаций (в процентах), в которых возникает синергетический 
эффект и, как следствие, коллективное решение (решение ЛПР) оказывается 
лучшим,\linebreak чем любое индивидуальное решении эксперта, от количества 
клиентов, дана на рис.~3. 
 Видно, что качество принимаемых решений ГиИМАС с 
нейт\-раль\-ны\-ми агентами выше, чем ГиИМАС других\linebreak архитектур. Это прямое 
следствие того, что в \mbox{ГиИМАС} с нейт\-раль\-ны\-ми агентами вероятность 
возникновения синергетического эффекта выше. Однако чем меньше 
размерность задачи, тем менее влияет этот эффект на качество решения. 

Следует отметить, что эффективность \mbox{ГиИМАС} при любом взаимодействии 
агентов выше, чем без такового.
  
  Анализ технико-эксплу\-а\-та\-ци\-он\-ных па\-ра\-мет\-ров показывает, что в 
условиях эксперимента время решения сложной 
  транс\-порт\-но-ло\-ги\-сти\-че\-ской задачи составило 2--30~мин, а качество 
решений подтверждено экспериментально и экспертами. По итогам 
практического использования программного продукта, реализующего ГиИМАС 
на двух объектах, средняя суммарная себестоимость доставки грузов в день 
сократилась на 7,2\%, средняя суммарная длительность доставки в день~--- на 
12,13\%, среднее время построения маршрутов в день уменьшилось на 23,14\%. 

\section{Заключение}

  В рамках парадигмы синергетического искусственного интеллекта 
рассмотрен подход к со\-зданию КСППР, моделирующей СППР и коллективные 
эффекты в них, приведена ее модель и функциональ\-ная структура. Система 
динамически перестраивает алгоритм своего функционирования, каждый раз 
при работе над сложной задачей вырабатывая релевантный ей метод решения. 
Метод моделирования самоорганизации СППР на основе анализа целей 
агентов, по сути, позволяет разнообразию условий и ситуаций решения 
сложных задач коллективом сопоставить не один-един\-ст\-вен\-ный 
инструмент, а множество динамично синтезируемых и изменяемых 
интегрированных моделей. Это открывает компьютерным сис\-те\-мам 
  управ\-ле\-ния перспективу для решения сложных задач, с которыми сейчас 
могут справиться только коллективы экспертов. 
  
  Вычислительные эксперименты на ГиИМАС с самоорганизацией сделали 
контрастными новые знания о релевантности архитектур ГиИМАС различным 
ситуациям решения задачи в СППР. 

Сравнение \mbox{ГиИМАС} и аналогичной 
гибридной интеллектуальной системы~[5] для решения сложной транс\-порт\-но-ло\-ги\-сти\-че\-ской 
задачи показало, что моделирование эффекта самоорганизации 
позволило динамически синтезировать метод решения сложных задач на основе 
базовых методов~--- знаний агентов: 
\begin{enumerate}[(1)]
\item при решении сложной транс\-порт\-но-ло\-ги\-сти\-че\-ской 
задачи вырабатывается релевантный ей метод, что повышает 
качество решения; 
\item стало возможным решать задачи с более высокими 
оценками сложности моделирования~[5]; 
\item сокращается трудоемкость 
проекта за счет отказа от настройки межагентных связей; 
\item хотя, по оценке, 
сложность рассмотренной тес\-то\-вой задачи выше сложности задачи, решаемой 
\mbox{ГиИС}, скорость получения результата \mbox{ГиИМАС} выше примерно в 
  2--4~раза. 
  \end{enumerate}
  
  {\small\frenchspacing
{%\baselineskip=10.8pt
\addcontentsline{toc}{section}{Литература}
\begin{thebibliography}{99}

\bibitem{1-lis}
\Au{Трахтенгерц Э.\,А.} Компьютерная поддержка принятия решений.~--- М.: Синтег, 1998.
\bibitem{2-lis}
\Au{Колесников А.\,В.} Гибридные интеллектуальные сис\-те\-мы. Теория и технология 
разработки~/ Под ред. А.\,М.~Яшина.~--- СПб.: СПбГТУ, 2001.
\bibitem{3-lis}
\Au{Freud S.} Group psychology and the analysis of the Ego.~--- New York: Liveright Publishing, 
1922. 
\bibitem{4-lis}
\Au{Lewin K.} Resolving social conflicts: Selected papers on group dynamics.~--- New York: 
Harper \& Row, 1948.
\bibitem{5-lis}
\Au{Колесников А.\,В. Кириков И.\,А., Листопад~С.\,В., Румовская~С.\,Б., 
Доманицкий~А.\,А.} Решение сложных задач коммивояжера методами функциональных 
гибридных интеллектуальных сис\-тем~/ Под ред. А.\,В.~Колесникова.~--- М.: ИПИ РАН, 
2011.
\bibitem{6-lis}
\Au{Поспелов Д.\,А.} Десять <<горячих точек>> в исследованиях по искусственному 
интеллекту~// Интеллектуальные системы, 1996. Т.~1. Вып.~1-4. C.~47--56.
\bibitem{7-lis}
\Au{Почебут Л.\,Г., Чикер В.\,А.} Организационная социальная психология: Уч.\ 
пособие.~--- СПб.: Речь, 2002. 
\bibitem{8-lis}
\Au{Майерс Д.} Социальная психология.~--- СПб.: Питер, 1997.
\bibitem{9-lis}
\Au{Кричевский Р.\,Л., Дубовская Е.\,М.} Социальная психология малой группы.~--- М.: 
Аспект Пресс, 2001.
\bibitem{10-lis}
Социальная психология~/ Под ред. А.\,Л.~Журавлева.~--- М.: ПЕР СЭ, 2002.
\bibitem{11-lis}
\Au{Тарасов В.\,Б.} От многоагентных систем к интеллектуальным организациям: 
философия, психология, информатика.~--- М.: Эдиториал УРСС, 2002. 
\bibitem{12-lis}
\Au{Листопад С.\,В.} Интеллектуальная система моделирования коллективного принятия 
решений для сложной транс\-порт\-но-ло\-ги\-сти\-че\-ской задачи: Дисс.\ \ldots\ канд. техн. 
наук.~--- М.: ИПИ РАН, 2012. 151~с.
\bibitem{13-lis}
\Au{Brockett R.\,W.} Hybrid models for motion control systems. Technical Report 
CICS-P-364.~--- Massachusetts Institute of Technology, Center for Intelligent Control Systems, 
1993.

\label{end\stat}

\bibitem{14-lis}
\Au{Рыбина Г.\,В.} Проектирование систем, основанных на знаниях.~--- М.: МИФИ, 
1997. 
\end{thebibliography} } }

\end{multicols}