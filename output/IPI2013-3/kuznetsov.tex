
\def\stat{kuznetsov}

\def\tit{УНИВЕРСАЛЬНЫЙ МЕТРИЧЕСКИЙ ТЕЗАУРУС\\ РУССКОГО ЯЗЫКА}

\def\titkol{Универсальный метрический тезаурус русского языка}

\def\autkol{Л.\,А.~Кузнецов, В.\,Ф.~Кузнецова, А.\,В.~Капнин}

\def\aut{Л.\,А.~Кузнецов$^1$, В.\,Ф.~Кузнецова$^1$, А.\,В.~Капнин$^2$}

\titel{\tit}{\aut}{\autkol}{\titkol}

%{\renewcommand{\thefootnote}{\fnsymbol{footnote}}\footnotetext[1] {Статья рекомендована к публикации в журнале Программным комитетом конференции <<Электронные 
%библиотеки: перспективные методы и технологии, электронные коллекции>> (RCDL-2012).}}

\renewcommand{\thefootnote}{\arabic{footnote}}
\footnotetext[1]{Российская академия народного хозяйства и государственной службы при Президенте РФ (Липецкий филиал), 
kuznetsov.leonid48@gmail.com}
%\footnotetext[2]{Российская академия народного хозяйства и государственной службы при Президенте РФ (Липецкий филиал), 
%kuznetsov.leonid48@gmail.com}
\footnotetext[2]{Липецкий государственный технический университет, gert@inbox.ru}


\Abst{Известные тезаурусы русского языка составлены группами экспертов. 
В~статье предлагается вариант разработки инструментов для автоматизированного 
формирования тезауруса на основе формального представления текстов, поясняющих 
семантику слов, и количественной оценки семантического расстояния между словами 
как меры их близости. Предлагаемые решения позволяют ориентироваться на 
фор\-маль\-но-ма\-те\-ма\-ти\-че\-ские представления, минимизирующие элемент 
субъективности в оценке бли\-зости слов. Они открывают возможность синтеза 
автоматических систем оценки семантической близости слов и решения иных задач в 
области обработки текстов. }

\KW{компьютерная лингвистика; универсальный тезаурус; метрический тезаурус; 
семантическая оценка близости; семантическое расстояние; теория информации}

 \vskip 14pt plus 9pt minus 6pt

\vskip 14pt plus 3pt minus 2pt

      \thispagestyle{headings}

      \begin{multicols}{2}

            \label{st\stat}

\section{Введение и~постановка задачи}
     
     Исследование возможностей создания систем автоматической оценки 
степени семантического подобия текстов, представленных на естественном 
(русском) языке, выдвинуло на первый план задачу оценки семантической 
близости двух любых слов, в том числе и разных частей речи. Решение задачи 
может базироваться на современных представлениях автоматизированной 
обработки текстовой информации, обеспечиваемых базами данных, 
способными быстро обрабатывать и хранить большие объемы слов. В~общем 
случае такие базы данных называются тезаурусами. Тезаурус (от греч.\ 
сокровище) в современной лингвистике~--- особая разновидность словарей 
общей или специальной лексики, в которых указаны семантические отношения 
(синонимы, антонимы, паронимы, гипонимы, гиперонимы и~т.\,п.)\ между 
лексическими единицами~[1]. 
    {\looseness=1
    
    }
     
     Известные тезаурусы русского языка (РуТез, RussNet, Лингвокультурный 
тезаурус русского языка и~др.)\ со\-став\-ле\-ны группами экспертов на основе 
их представлений о степени семантической бли\-зости отдельных слов. Такие 
тезаурусы вследствие неавтоматизированной методики их формирования, 
опирающейся на субъективные представления экспертов, как правило, имеют 
небольшой объем (количество статей и отношений между словами), не могут 
быть адаптированы к изменениям словарной базы. В~них не могут быть 
введены отношения между словами различных частей речи, изменения 
словарной базы влекут их полную переработку. Связи в существующих 
тезаурусах выражены с помощью логических отношений, что мешает 
применению математического аппарата при создании и совершенствовании 
инструментов компьютерной лингвистики~[2--4]. Мнения экспертов 
субъективны, поэтому даже однотипные семантические отношения (синонимы, 
антонимы, паронимы, гипонимы и~т.\,п.) определяются различным образом 
представителями разных школ. Этот результат согласуется с теорией 
информации, в соответствии с которой вербальное представление информации 
всегда, с одной стороны, избыточно, а с другой~--- неоднозначно. 
     
     В математике используются метрические пространства, в которых 
определена метрика: расстояние между элементами этого пространства~--- 
точками. Простой пример~--- обычная комната (трехмерное пространство), в 
которой можно измерить расстояние между любыми точками, например 
электрической лампочкой, свисающей с потолка, и стаканом, стоящим на столе. 
     
     Представляется полезным и для оценки бли\-зости слов разработать 
некоторую формальную меру <<расстояния>> между словами, в соответствии с 
которой между словами, близкими по смыслу, расстояние было бы небольшим, 
а между словами, далекими по смыслу,~--- большим, т.\,е.\ чтобы расстояние 
отражало степень семантического подобия слов. При этом важно, чтобы мера 
имела объективную фор\-маль\-но-ма\-те\-ма\-ти\-че\-скую основу,\linebreak\vspace*{-12pt}


\pagebreak



%\vspace*{-12pt}

\begin{center}  %fig2
\vspace*{-3pt}
\mbox{%
 \epsfxsize=73.981mm
 \epsfbox{kuz-1.eps}
 }
 \end{center}
% \vspace*{6pt}
{\small{Возможные варианты близости семантического содержания слов}}


%\pagebreak

\vspace*{15pt}

%\addtocounter{figure}{1}



\noindent
 а  следовательно, допускала разработку автоматизированных процедур ее 
вычисления на основе имеющихся словарей русского языка без привлечения 
субъективных представлений экспертов. 
     
     Наглядная иллюстрация развиваемой далее идеи может быть получена на 
основании пред\-став\-ле\-ний теории множеств. Допустим, исследуется близость 
слов $V_1$ и~$V_2$, содержание каждого из которых разъясняется в 
соответствующих словарных статьях словаря. Рисунок показывает возможное 
соотношение между содержанием словарных статей~$V_1$ и~$V_2$, которое 
интуитивно используется экспертами при оценке близости сопоставляемых 
слов. Вариант~1 соответствует случаю слов, име\-ющих совершенно различное 
семантическое содержание, при втором варианте имеет место частичная 
эквивалентность семантического содержания слов, а в третьем варианте 
слова~$V_1$ и~$V_2$ семантически эквивалентны. 
     


     Семантика слов полностью определяется соответствующими словарными 
статьями, на основании содержания которых эксперты приходят к выбору 
одного варианта из приведенных на рисунке. Содержательная сущность задачи 
оценки семантической близости слов показывает, что для ее автоматизации 
необходимо формализовать, во-пер\-вых, представление словарного описания 
содержания сопоставляемых слов $V_1$ и~$V_2$, т.\,е.\ представить его в виде 
некоторого математического объекта, допускающего количественную оценку. 
Во-вто\-рых, необходимо ввести некоторую формальную меру для оценки 
близости семантического содержания слов на основании определяющих их 
словарных статей, представленных в виде математических объектов. 
Положительное решение этих вопросов позволит разработать автоматическую 
процедуру оценки <<семантического расстояния>> между словами, опираясь 
исключительно на сведения существующих словарных описаний слов~$V_1$ 
и~$V_2$. 
     
     В настоящей статье предлагается вариант 
     фор\-маль\-но-ма\-те\-ма\-ти\-че\-ско\-го представления текстов, 
поясняющих семантику слов, и методики оценки семантического расстояния 
между словами как меры их близости. Отличительной особен\-ностью 
предлагаемых решений является ориентация на формальные представления, 
открывающие возможность синтеза на их основе автоматических сис\-тем 
оценки семантической близости слов и решения иных задач в об\-ласти 
обработки текстов. 
     
\section{Формирование образа слова}

      Исследования показали, что для решения задачи оценки семантической 
близости слов может быть привлечен математический аппарат теории 
информации, который позволяет синтезировать характеристику, интуитивно 
представляющуюся адекватной мере <<семантического расстояния>> между 
словами. В~работе~[5] показано, что для оценки семантической близости 
вербально представленных объектов могут быть использованы элементы 
теории информации. Дальнейшие исследования позволили выявить 
возможность формирования на ее основе количественной меры 
<<семантического расстояния>> между словами.
     
     Теория информации позволяет синтезировать математическую модель 
описания содержания слов для оценки меры их семантической близости. 
Получаемые на ее основе представления, с одной стороны, полностью 
согласуются с интуитивными представлениями о близости смысла слов, а с 
другой~--- позволяют ввести меру в виде количества информации. Семантика 
слов может быть извлечена исключительно из словарей соответствующего 
языка. В~дальнейшем для определенности имеется в виду русский язык, хотя 
разрабатываемая мера близости применима и к другим языкам, которые в 
рассматриваемом смысле представляются более простыми из-за имеющихся в 
них дополнительных структурных ограничений. 
     
     Авторы разработали и реализовали оригинальную технологию 
определения численной меры семантической близости между всеми словами, 
на основе которой разработан \textbf{универсальный специфицированный} 
метрический тезаурус русского языка. Тезаурус наполняется автоматически по 
разработанной технологии. Объем тезауруса ограничен лишь объемом 
имеющихся в наличии словарей и может пополняться автоматически при 
появлении новых словарных ресурсов.
     
     Полученный тезаурус решает не только поставленную задачу нахождения 
численной оценки смысловой близости слов, но и представляет наиболее 
полную базу отношений между словами русского языка, которые позволяют 
автоматизировать кластеризацию слов (построение словарей) на основании 
логических отношений типа: синонимы и антонимы, гиперонимы и гипонимы 
и~др., используя формальные методы, исключающие субъективизм экспертов.
     
     Словари дают, по возможности, лаконичное, с минимальным 
количеством необязательных для отражения смыслового содержания слов 
разъяснение смысла. Составляются они специалистами, хорошо знающими 
язык и опирающимися на имеющийся набор предшествующих изданий. 
Поэтому использование в них замысловатых, экзотических или редких слов 
маловероятно. 
     
     В основу концепции положены две гипотезы:
     \begin{enumerate}[(1)]
\item словарные статьи, определяющие смысл семантически близких 
слов, содержат больше одинаковых слов (в нормальной форме), чем 
семантически удаленных;
\item чем больше словарных статей используется для определения каждого 
из слов, тем точнее можно оценить количественно степень их близости.
\end{enumerate}

     Первая гипотеза базируется на том, что со\-по\-став\-ля\-емые слова относятся 
к определенным час\-тям речи. Различные час\-ти речи отражают вполне 
определенные семантические пред\-став\-ле\-ния: объекты, состояния объектов, 
свойства объектов и~т.\,п. Для пояснения близких семантических 
пред\-став\-ле\-ний используются однородные по морфологической и 
синтаксической структуре и словарному со\-ста\-ву тексты. Поэтому логично 
ожидать, и это подтверждается дальнейшими исследованиями, что сходная 
семантика представляется близкими наборами слов. Чем дальше семантика 
слов, тем дальше структура и словарное представление текстов, поясняющих 
слова.
     
     Вторая гипотеза обосновывается тем, что, например, в теории 
информации доказывается факт, что дополнительная информация об объекте не 
увеличивает его неопределенность. В~контексте рассматриваемого вопроса это 
означает, что любая дополнительная информация о семантике слова не 
увеличивает ее неопределенность. Существуют различные словари русского 
языка: общие, предметно или профессионально ориентированные, специальные 
и пр., в каждом из которых можно найти несколько отличающиеся определения 
слов (словарных статей). Их объединение в базе данных не уменьшает 
информацию о слове, а, наоборот, увеличивает ее. При этом словарные статьи из 
использованных словарей могут быть структурированы и снабжены 
соответствующими пометами. В~результате формируется 
\textit{мультисловарь}, который сам по себе представляет культурную и 
научную ценность.

{\small
\begin{center}
{{\normalsize \tablename~1}\ \ \small{Значимые части речи}}

\vspace*{6pt}

\begin{tabular}{|c|l|}
\hline
Случайное событие&\multicolumn{1}{c|}{Тип}\\
\hline
$A_1$&Существительное\\
$A_2$&Глагол\\
$A_3$&Прилагательное\\
$A_4$&Наречие\\
$A_5$&Числительное\\
$A_6$&Местоимение\\
$A_7$&Предикатив (можно, пора)\\
$A_8$&Причастие\\
$A_9$&Деепричастие\\
\hline
\end{tabular}
\end{center}
}
%\end{table*}

\vspace*{12pt}

\addtocounter{table}{1}

     
     В соответствии со второй гипотезой формирование тезауруса начинается 
с объединения словарных статей из разных словарей для каждого слова в 
единое описание. Описание дополняется определяемым словом и 
трансформируется в набор слов. Набор слов фильтруется с помощью 
синтаксического анализатора~[6]: остаются только значимые слова~--- это 
слова, часть речи которых свидетельствует о сильной смысловой нагрузке 
(табл.~1). Остав\-ши\-еся слова в наборе приводятся к нормальной форме (с 
помощью синтаксического анализатора~[6]) и сортируются в алфавитном 
порядке. Далее из набора исключаются дубликаты, а полученные слова $w_1, 
w_2,\ldots$ сохраняются в базе данных через запятую в качестве {образа} 
$O=\{w_1,w_2,\ldots\}$ данного слова~$V$. 
     
     Существует целое множество методов оценки семантических расстояний 
между словами~[7].\linebreak Большинство из них применимо к сетевым струк-\linebreak турам, 
когда между словами установлены логические отношения. Как было отмечено в 
начале \mbox{статьи}, допускающих компьютерную лингвистическую обработку 
тезаурусов не найдено. Иных методов формального представления описания 
семантического содержания слов, адаптированных для русского языка и 
допускающих введение количественных мер, обнаружено не было. Для 
адекватной оценки семантической бли\-зости слов на основании име\-ющих\-ся их 
словарных описаний важно обеспечить возможность учета (отражения) 
морфологической и синтаксической структуры описаний слов. Структура 
описаний связана с грамматической принадлежностью описываемого слова; 
кроме того, различные слова, используемые в словарных статьях для 
разъяснения смысла определяемого слова, могут содержать различное 
количество информации и вносить различный вклад в меру бли\-зости слов. 
Структуризация позволяет учесть это при формировании количественной 
оценки меры бли\-зости. В~работе~[5] введено представление текста в виде 
математической модели случайного объекта: 
     \begin{equation}
     M_\omega =\{\Omega, \aleph, P(A_j)\}\,,
     \label{e1-kuz}
     \end{equation}
где $\Omega=\{\omega_1,\omega_2, \ldots ,\omega_n\}$~--- пространство 
элементарных исходов;
$\aleph=\{A_1,A_2, \ldots , A_j, \varnothing, \Omega\}$~--- \mbox{алгебра} событий~$A_j$, 
составленных с помощью операций логического сложения, умножения и 
отрицания из элементарных событий~$\omega_i$ и дополненная 
невозможным~$\varnothing$ и достоверным~$\Omega$ событиями;\linebreak
$P(A_j)$~--- вероятности событий, составляющих алгебру, которые 
рассчитываются по вероятностям элементарных исходов $p(\omega_i)$, 
$i \hm= 1, 2, \ldots , n_\omega$, со\-став\-ля\-ющих это событие:

\vspace*{4pt}

\noindent
\begin{equation}
P(A_j)=\sum\limits_{\omega\in A_j} p(\omega_i)\,.
\label{e2-kuz}
\end{equation}

\vspace*{-2pt}

%\begin{table*}




     В данном контексте $\omega_i$, $i\hm = 1, 2, \ldots , n$,~--- это слова, 
составляющие текст, а $n$~--- их количество. Эмпирические вероятности 
$p(\omega_i)$, находящиеся под знаком суммы в~(\ref{e2-kuz}), представляют 
относительные частоты, которые равны отношению числа появлений слова~$i$ 
в тексте, к общему количеству слов в тексте, т.\,е.\ $p(\omega_i) \hm= 
n_\omega/n$. 
     
     Алгебра событий $\aleph$~--- это сис\-те\-ма, комбинируемая из 
используемых частей речи~$A_j$. Введение алгебры определяет используемую 
структуризацию языка. Может быть использована морфологическая 
структуризация языка, в которой случайные события~$A_j$, $j \hm= 1, 2, \ldots , 
J$, отождествляются с частями речи, приведенными, например, в табл.~1. 
     


     
     Могут использоваться не все части речи, а, например, только 
знаменательные. 
     
     На основании вероятностей~(\ref{e2-kuz}) событий~$A_j$, $j \hm= 1, 2, 
\ldots , J$, определяется энтропия текста, представленного в виде 
модели~(\ref{e1-kuz}):

\vspace*{4pt}

\noindent
     \begin{equation}
     H_\omega=-\sum\limits_{A_j\in \aleph} P(A_j) \ln P(A_j)\,.
     \label{e3-kuz}
     \end{equation}
     
     \vspace*{-2pt}
     
     Модель текста, определяющего семантическое содержание конкретного 
слова, далее называется образом слова и обозначается~$O$. После определения 
алгебры, по которой распределяются слова пояснительного текста, 
образ~(\ref{e1-kuz}) конкретного слова содержит всю информацию о нем, 
включая и собственно количество информации, определяемое 
выражением~(\ref{e3-kuz}). 

\vspace*{-6pt}
     
     
\section{Определение семантического расстояния между словами}
     
     Оценка близости двух слов $V_1$ и~$V_2$, как отмечалось выше, может 
базироваться на введении некоторой меры расстояния между их 
образами~$O_1$ и~$O_2$. Введение меры расстояния может быть 
осуществлено в виде соотношения, характеризующего отношение общего 
количества информации, содержащейся в двух образах~$O_1$ и~$O_2$, к 
количеству взаимной информации~[5], т.\,е.\ количеству информации, 
принадлежащей одновременно и~$O_1$, и~$O_2$. Такая мера может быть 
представлена в виде: 
     \begin{equation}
     \rho_{1,2}=\fr{H_1^2+H_2^1}{H^{1,2}}\,,
     \label{e4-kuz}
     \end{equation}
где $H_{1,2}$, $H_1^2$, $H_2^1$~--- энтропии, определяемые по образам 
сопоставляемых слов:
\begin{align*}
H_2^1 &=-\sum\limits_{A_j\in \aleph} P\left(A^1_{2j}\right)\ln P\left(A^1_{2j}\right)\,;\\
H_1^2 &=-\sum\limits_{A_j\in \aleph} P\left(A^2_{1j}\right)\ln P\left(A^2_{1j}\right)\,;\\
H^{1,2} &=-\sum\limits_{A_j\in \aleph} P\left(A^{1,2}_{j}\right)\ln P\left(A^{1,2}_{j}\right)\,.
\end{align*}

     Случайные события определяются следующим образом:
     \begin{align}
     A_j^{1,2} &= \{\omega_{ik}\in \Omega_1+\Omega_2\vert \omega_{ik}\in 
A_j^1\cap \omega_{ik}\in A_j^2\}\,;\label{e6-1-kuz}\\
     A_{2j}^{1} &= \{\omega_{ik}\in \Omega_1+\Omega_2\vert \omega_{ik}\in 
A_j^1\cap \omega_{ik}\notin A_j^2\}\,;\label{e6-2-kuz}\\
     A_{1j}^{2} &= \{\omega_{ik}\in \Omega_1+\Omega_2\vert 
\omega_{ik}\notin A_j^1\cap \omega_{ik}\in A_j^2\}\,,\label{e6-3-kuz}
     \end{align}
где верхним индексом отмечен объект, в который входит рассматриваемое 
слово, а нижним~--- объект, в который слово не входит.
 
     Событие~(\ref{e6-1-kuz}) составляется из слов, входящих в 
событие~$A_j$ объектов~1 и~2; (\ref{e6-2-kuz}) объединяет слова, входящие в 
объект~1, но отсутствующие в объекте~2; (\ref{e6-3-kuz}) объединяет слова, не 
входящие в объект~1, но входящие в объект~2. Других вариантов для слов, 
присутствующих в двух сопоставляемых текстах, нет. 
     
     Так как энтропии вычисляются по вероятностям, то они являются 
безразмерными абстрактными величинами. Содержательный смысл им 
при\-пи\-сы\-ва\-ет\-ся в соответствии с контекстом ре\-шаемых задач. 
В~рассматриваемом случае семантическое расстояние может быть 
безразмерной величи\-ной, принципиальным является диапазон из\-ме\-не\-ния этого 
расстояния и непрерывность меры. Непрерывность меры~(\ref{e4-kuz}) следует 
из непрерывности логарифмической функции. Диапазон изменения меры с 
учетом ее непрерывности может быть оценен по предельным (граничным) 
значениям расстояния, определяемого с ее помощью. 
     
     Допустим, что рассматриваются два объекта (слова из словаря). 
Обозначим их $V_1$ и~$V_2$. Каждое из них сопровождается образом 
$O_i\leftrightarrow \{\omega_1,\omega_2, \ldots ,\omega_n\}$, где $n$~--- 
количество слов текста, определяющего смысл слова. Количество слов в 
определяющем тексте может быть различным: $n$, $m$, $l$ и~т.\,п. Из 
содержательного представления вопроса о предельных значениях 
меры~(\ref{e4-kuz}) понятно, что предельными являются такие варианты: 
\begin{enumerate}[(1)]
\item образы слов полностью совпадают (вариант~3 на рисунке); 
\item образы слов 
полностью не совпадают (вариант~1 на рисунке). 
\end{enumerate}
     
     В первом варианте $\{\omega_1,\omega_2, \ldots ,\omega_n\}_1\hm= 
\{\omega_1,\omega_2, \ldots ,\omega_n\}_2$; следовательно, будут совпадать 
распределения слов по событиям~$A_j$, $j \hm= 1, 2, \ldots , J$. В~принципе не 
имеет значения, как распределятся слова по существительным, 
прилагательным, глаголам и~т.\,д. Важно, что эти распределения будут 
полностью совпадать для первого и второго слова. События типа $A^1_{2j}$ 
и~$A^2_{1j}$, $j \hm= 1, 2, \ldots , J$, не будут содержать элементов (слов), 
поэтому вероятности $P(A^1_{2j})\hm=0$ и $P(A^2_{1j})\hm=0$, $j \hm= 1, 2, 
\ldots , J$. Из этого следует, что и энтропии $H_2^1\hm=H_1^2\hm=0$. 
Подставив эти значения в~(\ref{e4-kuz}), можно получить величину расстояния 
между сопоставляемыми словами для случая полного совпадения поясняющих 
их текстов:
     $$
     \rho_{1,2}=\fr{H_1^2+H_2^1}{H^{1,2}}=\fr{0}{H^{1,2}}=0\,.
     $$
     
     По второму варианту $\{\omega_1,\omega_2, \ldots ,\omega_n\}_1$ 
пол\-ностью отличается от $\{\omega_1,\omega_2, \ldots ,\omega_n\}_2$, 
     т.\,е.\ $\omega_{i1}\not=$\linebreak $\not=\;\omega_{i2}$ для всех $i \hm= 1, 2, \ldots , n$. 
Следовательно, в этом случае не будет ни одного слова, которое встретилось 
бы одновременно в объяснении семантики и объекта~$V_1$, и объекта~$V_2$, 
поэтому случайные события системы~(\ref{e6-1-kuz})~$A_j^{1,2}$, $j \hm= 1, 
2, \ldots , J$, не будут содержать реализаций и все вероятности $P(A_j^{1,2})$, 
$j \hm= 1, 2, \ldots , J$, будут равны нулю. В теории информации принято 
считать $0\ln0 \hm= 0$. На основании этого и равенства нулю всех совместных 
вероятностей $P(A_j^{1,2})$ следует равенство нулю энтропии 
$H^{1,2}\hm=0$. Тогда из~(\ref{e4-kuz}) при произвольных значениях 
энтропий объектов~1, 2 следует:
     $$
     \rho_{1,2}=\fr{H_1^2+H_2^1}{H^{1,2}}=\fr{H_1^2+H_2^1}{0}=\infty\,.
     $$
     
     Таким образом, значение меры~(\ref{e4-kuz}) семантического расстояния 
между словами изменяется от нуля при тождественном совпадении 
содержательного описания слов до бесконечности при полном отличии их 
описаний. Диапазон изменений и непрерывность меры~(\ref{e4-kuz}) вполне 
соответствуют интуитивным представлениям о расстоянии вообще и 
семантическом расстоянии между словами в частности. 
     
     Расстояние~(\ref{e4-kuz}) представляет мощный инструмент, который 
может быть использован в качестве количественной характеристики близости 
слов в различного рода системах, обрабатывающих текстовые источники. 
     

\section{Пример формирования образов слов и~оценки 
семантического расстояния}

\vspace*{2pt}

     Демонстрация практического применения мето\-ди\-ки осуществляется 
оценкой меры семантической близости между словом <<Красивый>> и 
словами <<Прекрасный>>, <<Великолепный>>, <<Мечтательный>>. Слова 
выбраны так, что на основании общих представлений понятно, что 
семантическое расстояние между этими словами должно возрастать в порядке 
перечисления, а семантическое расстояние между одним и тем же словом, 
например <<Красивый>>, должно быть нулевым.
     
     В табл.~2 представлены структурированные по частям речи образы слов, 
являющиеся частью реальных образов. В~примере алгебра состоит из трех 
событий $A_1$, $A_2$, $A_3$ (см.\ табл.~1), что является упрощением для 
наглядного представления технологии.
     
\begin{table*}\small
\begin{center}
\Caption{Модели текстов, определяющих семантику сопоставляемых слов (образы слов)}
\vspace*{2ex}

\tabcolsep=1.8pt
\begin{tabular}{|c|c|l|l|l|l|}
\hline
\multicolumn{2}{|c|}{Слово}&\multicolumn{1}{c|}{$V_1$\;=\;<<Красивый>>}&
\multicolumn{1}{c|}{$V_2$\;=\;<<Прекрасный>>}&
\multicolumn{1}{c|}{$V_3$\;=\;<<Великолепный>>}&
\multicolumn{1}{c|}{$V_4$\;=\;<<Мечтательный>>}\\
\hline
\multicolumn{1}{|c|}{\raisebox{-76pt}[0pt][0pt]{Алгебра}}&\tabcolsep=0pt\begin{tabular}{c}Событие $A_1$\\ (существительные)\end{tabular}&
\begin{tabular}{l}Вид\\
Женщина\\ Линия\\ Очертание\\ Прелесть\\ Стан\\ Тон\\ Щеголь\end{tabular}&
\begin{tabular}{l}\sout{Женщина} \\ День\\ Прекрасница\\ \sout{Стан} \\
\sout{Тон}\\ Улыбка\\ \sout{Щеголь}\end{tabular}&
\begin{tabular}{l}Архитектура\\ \sout{Вид} \\ Восклицание\\ Одобрение\\
\sout{Очертание}\\ Радость\end{tabular}&
\begin{tabular}{l}Взгляд\\ \sout{Вид}\\ Мечтательность\\ Настроение\end{tabular}\\
\cline{2-6}
&\tabcolsep=0pt\begin{tabular}{c}Событие $A_2$\\ (глаголы)\end{tabular}&
\begin{tabular}{l}Блестеть\\ Выглядеть\\ Казаться\\ Привлекать\\ Рисовать\end{tabular}&
\begin{tabular}{l}\sout{Выглядеть}\\ \sout{Казаться}\\ \sout{Рисовать}\\ Создавать\\
Украшать\end{tabular}&
\begin{tabular}{l}\sout{Блестеть}\\ Выражать\\ Выспаться\\ \sout{Рисовать}\end{tabular}&
\begin{tabular}{l}Выражать\\ \sout{Смотреть}\\ Создать\\ Стремиться\end{tabular}\\
\cline{2-6}
&\tabcolsep=0pt\begin{tabular}{c}Событие $A_3$\\ (прилагательные)\end{tabular}&
\begin{tabular}{l}Благовидный\\ Восхитительный\\ Красивый\\ Обворожительный\\
Прекрасный\\ Хороший\end{tabular}&
\begin{tabular}{l}\sout{Благовидный}\\ Великолепный\\ \sout{Восхитительный}\\
Замечательный\\ \sout{Красивый}\\ Необыкновенный\\ Отличный\\
\sout{Прекрасный}\\ \sout{Хороший}\end{tabular}&
\begin{tabular}{l}Величественный\\ Живописный\\ Картинный\\ \sout{Красивый}\\
\sout{Прекрасный}\\ Пышный\\ Роскошный\end{tabular}&
\begin{tabular}{l}Возвышенный\\ Восхитительный\\ Мечтательный\\ Склонный\end{tabular}\\
\hline 
\multicolumn{6}{p{160mm}}{\footnotesize \textbf{Примечание.} Зачеркнуты 
слова, которые присутствует в образе $V_1$\;=\;<<Красивый>>, с которым 
сравниваются другие слова.}
\end{tabular}
\end{center}
\vspace*{-6pt}
\end{table*}

     Покажем подробно процесс вычисления семантического расстояния 
между $V_1\hm =$\;<<Красивый>> и $V_2 =$\;<<Прекрасный>>. 
     
     Объединенный образ для слов $V_1$ и $V_2$ выглядит так: 
$\{$\textit{Вид, Женщина, Линия, Очертание, Прелесть, Стан, Тон, Щеголь, 
Блестеть, Выглядеть, Казаться, Привлекать, Рисовать, Благовидный, 
Восхитительный, Красивый, Обворожительный, Прекрасный, Хороший,} 
{\bfseries\textit{Женщина, День, Прекрасница, Стан, Тон, Улыбка, Щеголь, 
Выглядеть, Казаться, Рисовать, Создавать, Украшать, Благовидный, 
Великолепный, Восхитительный, Замечательный, Красивый, 
Необыкновенный, Отличный, Прекрасный, Хороший}}$\}$. Количество слов 
в объединенном образе~--- 40.

\vspace*{6pt}
     
     События группы $A_1$ (существительные):
     \begin{align*}
        A_{21}^1 &= \{\mbox{\textit{Вид, Линия, Очертания, Прелесть}}\}\,;\\
        A_{11}^2 &= \{\mbox{\textit{День, Прекрасница, Улыбка}}\}\,;\\
        A_{1}^{1,2} &=\{\mbox{\textit{Женщина, Стан, Тон, Щеголь, Женщина,}}\\
        & \hspace*{38mm}\mbox{\textit{Стан,  Тон, Щеголь}}\}\,.
     \end{align*}.
     
     События группы $A_2$ (глаголы):
     \begin{align*}
        A_{22}^1 &=\{ \mbox{\textit{Блестеть, Привлекать}}\}\,;\\
        A_{12}^2 &=\{\mbox{\textit{Создавать, Украшать}}\}\,;\\
        A_2^{1,2} &= \{\mbox{\textit{Выглядеть, Казаться, Рисовать,}}\\ 
&\hspace{16mm}\mbox{\textit{Выглядеть, Казаться, Рисовать}}\}\,.
     \end{align*}
     
     События группы $A_3$ (прилагательные):
     \begin{align*}
        A^1_{23} &= \{\mbox{\textit{Обворожительный}}\}\,;\\
        A_{13}^2 &= \{\mbox{\textit{Великолепный, Замечательный;}}\\
        &\hspace*{20mm}\mbox{\textit{Необыкновенный,  Отличный}}\}\,;
                \end{align*}
        
        \noindent
        \begin{align*}
        A_3^{1,2} &= \{\mbox{\textit{Благовидный, Восхитительный, Красивый,}}\\
        &\hspace*{7mm}\mbox{\textit{Прекрасный, Хороший, Благовидный,}}\\
&\hspace*{8mm}\mbox{\textit{Восхитительный, Красивый, Прекрасный,}}\\
&\hspace*{54mm}\mbox{\textit{Хороший}}\}\,.
     \end{align*}
     
     Получаются энтропии:
     \begin{align*}
     H_2^1 &= -\fr{4}{40}\ln\left( \fr{4}{40}\right) -\fr{2}{40}\ln\left( 
\fr{2}{40}\right)-\fr{1}{40}\ln\left( \fr{1}{40}\right) =\\
&\hspace*{59mm}{}=0{,}472\,;\\
     H_1^2 &= -\fr{3}{40}\ln\left( \fr{3}{40}\right) -\fr{2}{40}\ln\left( 
\fr{2}{40}\right)-\fr{4}{40}\ln\left( \fr{4}{40}\right) =\\
&\hspace*{59mm}{}=0{,}574\,;\\
     H^{1,2} &= -\fr{8}{40}\ln\left( \fr{8}{40}\right) -\fr{6}{40}\ln\left( 
\fr{6}{40}\right)-\fr{10}{40}\ln\left( \fr{10}{40}\right) =\\
&\hspace*{59mm}{}=0{,}953\,.
     \end{align*}
     
     Семантическое расстояние между словами $V_1$ и~$V_2$:
     $$
     \rho_{1,2}=\fr{0{,}472+0{,}574}{0{,}953}=1{,}098\,.
     $$
     
     Результаты промежуточных расчетов и семантические расстояния 
представлены в табл.~3.
     

        
     Для сравнения найдем семантические расстояние   известным методом 
сопоставления свойств~[7] (табл.~4), основанным на тео\-ре\-ти\-ко-мно\-же\-ст\-вен\-ном 
подходе Тверски, при котором слова 
образа понимаются как свойства:

\noindent
     \begin{equation*}
     \mu_{12} = \fr{\vert O_1\vert +\vert O_2\vert}{2\vert A^{1,2}\vert}-1\,,
%     \label{e8-kuz}
     \end{equation*}
где $\vert O\vert$~--- количество слов образа (мощность множества слов 
образа);
$\vert A^{1,2}\vert$~--- количество совпадающих слов в образах (мощность 
множества совпадающих слов образов).


     
     
     
     Полученные в примере расстояния позволяют семантически 
позиционировать слова относительно друг друга. Предложенный метод, 
основанный на применении энтропии, показывает более содержательный 
результат, что закономерно объясняется использованием структуризации с 
помощью частей речи. Расстояние~$\rho$ отражает объективную смысловую 
близость слов, а не просто выстраивает слова в порядке смысловой 
удаленности.

\vspace*{-6pt}
     
\section{Краткая характеристика разработанного тезауруса}

     Отличительной особенностью разработанного тезауруса является 
универсальность, которая следует из того, что в нем ассимилированы сведения 
из 8~словарей четырех типов: 
     \begin{enumerate}[(1)]
\item словарь синонимов~[8];
\item словарь иностранных слов~[9];
\item толковые словари~[10--14];
\item энциклопедический словарь~[15].
\end{enumerate}

\end{multicols}

     \begin{table*}\small
     \begin{center}
     \Caption{Семантические расстояния между словами, определенные с помощью 
энтропии}
      \vspace*{2ex}
      
      \tabcolsep=3.3pt
      \begin{tabular}{|l|l|c|c|c|c|}
      \hline
\multicolumn{2}{|c|}{Параметры}&\tabcolsep=0pt\begin{tabular}{c}Для $\rho_{11}$\\
с $V_1$\;=\\ \;=\;<<Красивый>>\end{tabular}&
 \tabcolsep=0pt\begin{tabular}{c}Для $\rho_{12}$\\
с $V_2$\;=\\
=\;<<Прекрасный>>\end{tabular}&
\tabcolsep=0pt\begin{tabular}{c}Для $\rho_{13}$\\ с $V_3$\;=\\ =\;<<Великолепный>>\end{tabular}&
\tabcolsep=0pt\begin{tabular}{c}Для $\rho_{14}$ \\ с $V_4$\;=\\ =\;<<Мечтательный>>\end{tabular}\\
\hline
&Расстояние  $\rho_{1i}$&0&1,098&1,951&4,621\\
&Всего слов&38&40&36&31\\
\cline{2-6}
&&&&&\\[-9pt]
Общие&Составляющая $H_2^1$&\hphantom{9}0&\hphantom{999,9}0,472&\hphantom{999,9}0,750&\hphantom{999,9}0,918\\
&Составляющая $H_1^2$&\hphantom{9}0&\hphantom{999,9}0,574&\hphantom{999,9}0,679&\hphantom{999,9}0,716\\
&Составляющая $H^{1,2}$ &\hphantom{999,9}0,487&\hphantom{999,9}0,953&\hphantom{999,9}0,732&\hphantom{999,9}0,354\\
\hline
&&&&&\\[-9pt]
&Количество событий $A^1_{i1}$&\hphantom{9}0&\hphantom{9}4&\hphantom{9}6&\hphantom{9}7\\
\multicolumn{1}{|l|}{\raisebox{6pt}[0pt][0pt]{Существи-}}&Количество событий $A^i_{11}$&\hphantom{9}0&\hphantom{9}3&\hphantom{9}4&\hphantom{9}3\\
\multicolumn{1}{|l|}{\raisebox{6pt}[0pt][0pt]{тельные}}&Количество событий $A_1^{1,i}$&16&\hphantom{9}8&\hphantom{9}4&\hphantom{9}2\\
\hline
&&&&&\\[-9pt]
&Количество событий $A^1_{i2}$&\hphantom{9}0&\hphantom{9}2&\hphantom{9}3&\hphantom{9}4\\
Глаголы&Количество событий  $A^i_{12}$&\hphantom{9}0&\hphantom{9}2&\hphantom{9}2&\hphantom{9}3\\
&Количество событий $A_2^{1,i}$&10&\hphantom{9}6&\hphantom{9}4&\hphantom{9}2\\
\hline
&&&&&\\[-9pt]
&Количество событий $A^1_{i3}$&\hphantom{9}0&\hphantom{9}1&\hphantom{9}4&\hphantom{9}6\\
\multicolumn{1}{|l|}{\raisebox{6pt}[0pt][0pt]{Прилага-}}&Количество событий $A^i_{13}$&\hphantom{9}0&\hphantom{9}4&\hphantom{9}5&\hphantom{9}4\\
\multicolumn{1}{|l|}{\raisebox{6pt}[0pt][0pt]{тельные}}&Количество событий $A_3^{1,i}$&12&10&\hphantom{9}4&\hphantom{9}0\\
\hline
\end{tabular}
\end{center}
\end{table*}

\begin{table*}\small
\begin{center}
\Caption{Семантические расстояния между словами, определенные с помощью метода 
сопоставления свойств}
\vspace*{2ex}

\tabcolsep=3.1pt
\begin{tabular}{|l|c|c|c|c|}
\hline
&\tabcolsep=0pt\begin{tabular}{c}Для $\mu_{11}$\\ с $V_1$\;=\;<<Красивый>>\end{tabular}&
\tabcolsep=0pt\begin{tabular}{c}Для $\mu_{12}$\\ с $V_2$\;=\;<<Прекрасный>>\end{tabular}&
\tabcolsep=0pt\begin{tabular}{c}Для $\mu_{13}$\\ с $V_3$\;=\;<<Великолепный>>\end{tabular}&
\tabcolsep=0pt\begin{tabular}{c}Для $\mu_{14}$\\ с $V_4$\;=\;<<Мечтательный>>\end{tabular}\\
\hline
\tabcolsep=0pt\begin{tabular}{l}Количество слов\\ образа, 
$\vert O\vert$\end{tabular}&19&21&17&12\\
\hline
\tabcolsep=0pt\begin{tabular}{l}Количество\\ совпадающих\\ слов в образах, 
$\vert A^{1,2}\vert$\end{tabular}&19&12&\hphantom{9}6&\hphantom{9}2\\
\hline
\tabcolsep=0pt\begin{tabular}{l}Семантическое \\ расстояние,
$\mu_{1i}$\end{tabular}&\hphantom{9}0&\hphantom{999,9}0,375&\hphantom{999,9}0,917&\hphantom{99,}2,5\\
\hline
\end{tabular}
\end{center}
\end{table*}

\begin{multicols}{2}
     
      Он содержит более 200\,000~слов в нормальной форме и более 
400\,000~словарных статей, в которых отражаются различные семантические, 
морфологические и синтаксические характеристики слов.
     
     Важнейшей сущностью тезауруса является интеллектуальная 
составляющая в виде отношений\linebreak
между словами, помещенными в тезаурус. 
Под отношением понимается метрическая оценка семантической близости 
слов. Интеллектуальная со\-став\-ля\-ющая включает около 4\,000\,000\,000 
значимых\linebreak отношений, численно выражающих семантическую близость слов. 
     
     При разработке актуальной версии тезауруса были использованы 
некоторые допущения, что неизбежно привело к соответствующим 
недостаткам:
     \begin{itemize}
\item замена буквы <<ё>> на <<е>> в определяемых словах в ряде 
случаев приводит к трансформации в иное по смыслу слово, уже 
существующее в тезаурусе (например, нёбо\;$\rightarrow$\;небо). 
Словарные статьи в этом случае объединяются, а данные слова 
становятся омонимами. Так как словарные статьи объединяются, то связи 
исходных <<ё>>-слов не будет утеряны, но численные оценки будут 
значительно занижены. Если учесть, что таких слов оказалось менее 
0,1\%, такое занижение оценок близости выглядит несущественным;
\item при формировании образов тире заменяется на пробел, что 
вызывает и замену на пробел дефиса в силу того, что многие словари 
были оформлены ненадлежащим образом. С~другой стороны, эта замена 
не должна привести к значимым колебаниям оценки, так как количество 
слов с дефисом не столь велико (менее 0,5\%), а замена дефиса на пробел 
в большинстве случаев не меняет, а семантически расширяет образ слова; 
\item аббревиатуры и сокращения не расшифровывались; следовательно, 
синтаксический анализатор не мог корректно анализировать такие слова. 
Это приводило к семантическому искажению образов. Текст словарных 
статей преимущественно описывается без аббревиатур и малым 
количеством сокращений слов (б$\acute{\mbox{о}}$льшая часть которых не несет яркой 
смысловой нагрузки и будет исключена анализатором), что позволяет 
сделать вывод о небольшой зна\-чи\-мости искажений;
\item в силу ненадлежащего оформления текстов словарей возможны 
некоторые орфографические ошибки. Вручную проверить такой объем 
информации не представлялось возможным, поэтому текст подвергся 
автоматизированным проверкам орфографии.
\end{itemize}
     
     Формирование тезауруса осуществлялось раз\-ра\-бо\-тан\-ной для этой цели 
автоматизированной \mbox{системой}. Хотя процесс и требует больших 
вычислительных ресурсов (высокопроизводительной вычислительной системы 
с кли\-ент-сер\-вер\-ной архитектурой и мощной дисковой подсистемой), но\linebreak 
благодаря автоматизации можно обойти принципиальные препятствия на пути 
дальнейшего совершенствования тезауруса, направленного на исключение 
негативных последствий принятых\linebreak допущений и недостатков, следующих из 
оформления текстов словарей. Спектр используемых в качестве основы 
тезауруса словарей может быть расширен и реализована более детальная 
синтаксическая и орфографическая проверка исходных текстов.
     
     Несмотря на принятые допущения, уже разра\-ботанный тезаурус 
открывает широкое поле для\linebreak работы в области компьютерной лингвистики. 
Чис\-лен\-ная оценка близости слов дает возможность применить математический 
аппарат в на\-прав\-ле\-ниях:
     \begin{itemize}
\item кластеризации слов и формирования тезауруса на основе логических 
отношений;
\item формирования словарей синонимов;
\item построения мощных поисковых машин;
\item сравнения и установления авторства текстов;
\item разработки автоматических интерактивных сис\-тем консультирования и 
помощи, а также в других направлениях и областях обработки текстовой 
информации. 
\end{itemize}
     

\section{Заключение}

     Разработана методика представления словарных статей, разъясняющих 
содержание слов, в виде формальных объектов, имеющих количественную 
меру оценки содержащейся в них информации. Разработана мера оценки 
семантического расстояния между словами, позволяющая устанавливать 
степень их семантической близости. На их основе создана автоматизированная 
система формирования тезауруса русского языка на основании имеющихся 
словарей и разработан универсальный тезаурус, снабженный инструментом 
интеллектуальных отношений между словами, позволяющим оценивать их 
семантическую близость. 

{\small\frenchspacing
{%\baselineskip=10.8pt
\addcontentsline{toc}{section}{Литература}
\begin{thebibliography}{99}
      
\bibitem{1-kuz} %1
\Au{Лесников С.\,В.} Тезаурус как отражение системности языка~// Вестник 
челябинского государственного ун-та, 2011. №\,28 (243). Филология. 
Искусствоведение. Вып.~59. С.~52--61.

\bibitem{3-kuz} %2
\Au{Караулов Ю.\,Н.} Лингвистическое конструирование и тезаурус 
литературного языка.~--- М.: Наука, 1981. 367~с.

\bibitem{2-kuz} %3
\Au{Добров Б.\,В., Иванов В.\,В., Лукашевич~Н.\,В., Соловьев~В.\,Д.} Онтологии 
и тезаурусы: Учеб\-но-ме\-то\-ди\-че\-ское пособие.~--- Казань: КГУ, 2006. 190~с.

\bibitem{4-kuz}
\Au{Тарасов С.\,Д.} Подход к реализации автоматизированной системы 
построения тезауруса~// Электронные библиотеки: перспективные методы и 
технологии, электронные коллекции: Труды IX Всеросс. научной конф. 
RCDL'2007.~---  Переславль-Залесский: Ун-т\linebreak г.~Переславля, 2007. Т.~2. 
С.~63--66.
\bibitem{5-kuz}
\Au{Кузнецов Л.\,А.} Ве\-ро\-ят\-но\-ст\-но-ста\-ти\-сти\-че\-ская оценка 
адекватности информационных объектов~// Информатика и её применения, 
2011. Т.~5. Вып.~4. С.~64--75.
\bibitem{6-kuz}
\Au{Антонова А.\,А., Мисюрев А.\,В.} Об использовании синтаксического 
анализатора Cognitive Dwarf 2.0~// Труды Института системного анализа РАН, 
2008. №\,38. С.~91--107. 
\bibitem{7-kuz}
\Au{Крюков К.\,В., Панкова Л.\,А., Пронина~В.\,А., Суховеров~В.\,С., 
Шипилина~Л.\,Б.} Меры семантической близости в онтологиях~// Проблемы 
управления, 2010. №\,5. С.~2--14.
\bibitem{8-kuz}
\Au{Абрамов Н.} Словарь русских синонимов и сходных по смыслу 
выражений.~--- М.: Русские словари, 1999. 431~с. 
\bibitem{9-kuz}
\Au{Комлев Н.\,Г.} Словарь иностранных слов.~--- М.: ЭКС\-МО-Пресс, 2000. 
1308~с. 
\bibitem{10-kuz}
\Au{Винокур Г.\,О., Ларин Б.\,А., Ожегов~С.\,И., Томашевский~Б.\,В., 
Ушаков~Д.\,Н.} Толковый словарь русского языка: в 4~т.~--- М.: Советская 
энциклопедия; ОГИЗ, 1935--1940. 5529~с.
\bibitem{11-kuz}
Современный толковый словарь.~--- М.: Большая Советская энциклопедия, 
1997. 6110~с.

\bibitem{13-kuz}
\Au{Даль В.\,И.} Толковый словарь живого великорусского языка.~--- М.: 
Цитадель, 1998. 4249~с. 

\bibitem{12-kuz} %13
\Au{Ефремова Т.\,Ф.} Современный толковый словарь русского языка.~--- М.: 
АСТ, 2006. 3312~с.

\bibitem{14-kuz}
\Au{Ожегов С.\,И.} Толковый словарь русского языка.~--- М.: Оникс, 2008. 
976~с.

\label{end\stat}

\bibitem{15-kuz}
Современный энциклопедический словарь.~--- М.: Большая Советская 
энциклопедия, 1997. 1382~с.
\end{thebibliography}
} }

\end{multicols}