
\def\stat{kalenov}

\def\tit{ПРОБЛЕМЫ СЕТЕВОГО ДОСТУПА К НАУЧНЫМ ЖУРНАЛАМ}

\def\titkol{Проблемы сетевого доступа к научным журналам}

\def\autkol{А.\,В.~Глушановский, Н.\,Е.~Калёнов}

\def\aut{А.\,В.~Глушановский$^1$, Н.\,Е.~Калёнов$^2$}

\titel{\tit}{\aut}{\autkol}{\titkol}

%{\renewcommand{\thefootnote}{\fnsymbol{footnote}}\footnotetext[1] {Статья 
%рекомендована к публикации в журнале Программным комитетом конференции 
%<<Электронные библиотеки: перспективные методы и технологии, электронные 
%коллекции>> (RCDL-2012).}}

\renewcommand{\thefootnote}{\arabic{footnote}}
\footnotetext[1]{Библиотека по естественным наукам Российской академии наук, 
avglush@benran.ru} 
\footnotetext[2]{Библиотека по естественным наукам 
Российской академии наук, nek@benran.ru}



\Abst{Рассматриваются проблемы организации сетевого доступа российских ученых к 
научным журналам и базам данных. В соответствии с мировой практикой организацию 
такого доступа осуществляют научные биб\-лио\-те\-ки, объединяющиеся в 
консорциумы для получения выгодных финансовых условий. Описывается существующая 
в России практика организации доступа к зарубежным научным ресурсам через 
посредство Российского
фонда фундаментальных исследований (РФФИ) и <<Национального электронно-информационного 
консорциума>> (НЭИКОН). Приведена статистика востребованности пользователями 
Российской академии наук (РАН) научных журналов, предоставляемых через НЭИКОН. 
Предложены организационные действия для решения задачи оптимизации доступа к 
коммерческим сетевым научным ресурсам в условиях существующих в РАН финансовых 
ограничений.}


\KW{научные журналы; информация; Интернет; удаленный доступ; библиотеки; 
консорциум}

\vskip 14pt plus 9pt minus 6pt

      \thispagestyle{headings}

      \begin{multicols}{2}

            \label{st\stat}

     Анализ информационных потребностей ученых РАН показывает, что 
по-прежнему одним из важнейших источников научной информации для них 
остаются научные журналы (в первую очередь~--- иностранные). 
В~настоящее время наряду с традиционной печатной формой все более 
широкое распространение получил доступ к научным журналам через сеть 
Интернет.
     
     Технически такой доступ не представляет затруднений, что создает 
впечатление легкой доступности полных текстов статей. На самом деле все 
обстоит несколько сложнее. Большинство ведущих зарубежных издательств 
и научных обществ, таких как Elsevier, Springer, American Physical Society, 
American Chemical Society и других, представляют в свободном доступе 
только биб\-лио\-гра\-фи\-че\-скую информацию (описание статей) и (в лучшем 
случае) их рефераты. Доступ к полным текстам является платным и требует 
заключения соответствующего договора с издательством, причем суммы 
таких договоров, во-пер\-вых, весьма значительны, а во-вто\-рых, весьма 
заметно варьируются в зависимости от числа пользователей в организации, 
количества подключаемых компьютеров и ряда других па\-ра\-мет\-ров.
     
     Организация сетевого доступа к коммерческим\linebreak
      источникам научной 
информации требует значительной по объему и сложности специфической\linebreak 
работы, связанной с выбором нужных ресурсов, проведением переговоров с 
поставщиками, согласованием условий предоставления ресурсов, 
заключением контрактов и оформлением лицензионных соглашений, 
предоставлением IP-ад\-ре\-сов и контролем выполнения договорных 
обязательств. Для научных сотрудников такая деятельность не является 
характерной, поэтому сложившаяся мировая практика организации сетевого 
доступа к научной информации состоит в том, что ею занимаются 
биб\-лио\-те\-ки университетов, научных центров и других научных и учебных 
организаций. Биб\-лио\-те\-ки в силу специфики своей деятельности лучше 
знакомы с издательским миром, имеют опыт взаимодействия как с 
издательствами, так и с пользователями информации, и работа по 
информационному обеспечению научных исследований является их прямой 
обязанностью 
     
     Подобная практика сложилась и в России, в частности в РАН. 
Центральные академические биб\-лио\-те\-ки (такие как Библиотека Российской 
академии наук (БАН) в Санкт-Пе\-тер\-бур\-ге, Биб\-лио\-те\-ка по естественным наукам 
РАН (БЕН РАН) в Москве, Государственная публичная научнотехническая биб\-лио\-те\-ка Сибирского 
отделения РАН (\mbox{ГПНТБ} СО РАН)
 в Новосибирске, Центральная научная биб\-лио\-те\-ка 
Уральского отделения РАН (ЦНБ УрО РАН) в Екатеринбурге, Центральная научная 
биб\-лио\-тека Дальневос\-точ\-ного отделения РАН (ЦНБ ДвО РАН) во Вла\-ди\-востоке), 
обеспечивающие информационные потребности многих институтов РАН, 
тематика исследований которых в значительной мере пересекается, могут 
получить\linebreak значительно более выгодные условия доступа к \mbox{научным} журналам 
и базам данных (БД), нежели отдельные институты, заключающие 
самостоятельные договора с поставщиками. Кроме того, биб\-лио\-те\-ки (и/или 
их объединения~--- консорциумы) берут на себя организационные вопросы 
(переговоры с издательствами, заключение и оплата договоров, оформление 
лицензионных соглашений, сбор IP-адресов и организацию их подключения 
и~т.\,д.). Библиотеки также ведут анализ фактического использования 
доступа и оптимизируют подписку (в условиях жестких финансовых 
ограничений) для своих систем в целом.
 %    
     Как принято в мировой практике, для оптимизации финансовых 
условий доступа биб\-лио\-те\-ки объединяются в консорциумы, выступающие 
как единое юридическое лицо в отношениях с из\-да\-ющи\-ми организациями 
(или поставщиками ресурсов).
     
     Обычно в консорциумы объединяются биб\-лио\-те\-ки исходя из двух 
положений~--- либо предоставить узкотематическую информацию как можно 
более широкому кругу пользователей (объединяются организации с 
близкими научными интересами) и получить скидки (в расчете на одного 
участника консорциума) за счет значительного числа пользователей данного 
ресурса, либо предоставить пользователям консорциума как можно более 
широкий спектр информационных ресурсов (объединяются организации с 
различными тематическими интересами) и получить скидки (в расчете на 
одного участника консорциума) за счет увеличения объема предоставляемых 
ресурсов.
     
     В мире существует значительное число различных биб\-лио\-теч\-ных 
консорциумов. Международное объединение биб\-лио\-теч\-ных консорциумов 
(The International Coalition of Library Consortia~--- ICOLC)~[1] объединяет 
более 200~биб\-лио\-теч\-ных консорциумов, созданных на основе коалиций 
биб\-лио\-тек по тематическому или территориальному принципу. 
Консорциумы бывают разной величины и типа. Например, в Финляндии 
практически все университетские биб\-лио\-те\-ки, биб\-лио\-те\-ки научных 
учреждений и пуб\-лич\-ные биб\-лио\-те\-ки объединены в FinELib~[2]~--- 
национальный консорциум, по\-став\-ля\-ющий более 70\% всей электронной 
информации~[3]. Различные типы европейских биб\-лио\-теч\-ных консорциумов 
описаны в~[4].

В России в 1990--2000-е~гг.\ сложилась аналогичная практика 
организации доступа к научным журналам~[5]. С~1997~г.\ такой доступ 
предоставлялся в рамках консорциума, созданного по инициативе БЕН РАН 
и включавшего \mbox{РФФИ} и 
14~крупнейших научных биб\-лио\-тек. Финансирование консорциума 
осуществлял \mbox{РФФИ} в рамках принятой в конце 1996~г.\ <<Программы 
поддержки российских научных биб\-лио\-тек>>. В~рамках этой программы 
была создана научная электронная биб\-лио\-те\-ка (НЭБ). В~соответствии с 
принципами ее организации электронные версии журналов поступали из 
издательств в \mbox{РФФИ} и загружались на специальный сервер НЭБ и его 
зеркала в Казани и Новосибирске. Доступ предоставлялся всем 
пользователям биб\-лио\-тек, входящих в консорциум, а поскольку в 
консорциум входили все центральные академические биб\-лио\-те\-ки (БАН, БЕН 
РАН, \mbox{ГПНТБ} СО РАН, ЦНБ УрО РАН и ЦНБ ДвО РАН), любой сотрудник 
Академии наук мог читать основные научные журналы мира. К~началу 
2002~г.\ на серверы НЭБ было загружено около 2000~наименований (около 
75\,000~выпусков) журналов наиболее значимых научных издательств 
мира~[6]. Научная электронная библиотека пользовалась большой популярностью у специалистов~--- за 
год в начале \mbox{2000-х}~гг.\ из нее выгружалось около четверти миллиона 
статей. 
     
     Соглашение о консорциуме НЭБ, подписанное РФФИ и ведущими 
биб\-лио\-те\-ка\-ми, сопровождалось рядом условий, выдвинутых издательствами 
и направленных, в частности, на сохранение перечня приобретаемых 
биб\-лио\-те\-ка\-ми печатных версий журналов (по условиям участия в 
консорциуме организация должна была выписать для себя в печатном виде 
не менее 5~журналов, не входящих в подписку консорциума~[7]). Имелся (и 
сохраняется до сих пор во всех подобного рода консорциумах) ряд 
ограничений на выгрузку и распространение полученных текстов 
(запрещается сплошное копирование номера журнала, распространение 
полученных материалов за пределами ор\-га\-ни\-за\-ции-участ\-ника).
     
     К сожалению, в 2004~г.\ НЭБ РФФИ прекратила свое существование в 
том виде, который преду\-смат\-ри\-вал\-ся соглашениями 1996~г. Причинами 
этого стали несколько факторов, в частности проверка РФФИ со стороны 
Счетной палаты. Проверка вы\-яви\-ла нарушения Устава РФФИ, согласно 
которому последний не имеет права финансировать что-ли\-бо без 
проведения конкурсов. Это по\-влек\-ло за собой проблемы финансирования 
поддержки технологии функционирования НЭБ (обработка и загрузка 
массивов данных, поддержка серверов). 

С~вступлением в силу 94-го 
Федерального закона о закупках фактически были ликвидированы 
механизмы координированной работы биб\-лио\-тек по приобретению научных 
ресурсов. Из-за распада консорциума наиболее значимые научные 
издательства отказались передавать журналы российской стороне. 
В~результате уже загруженные на сервер журналы НЭБ были юридически 
переданы ООО <<Научная электронная биб\-лио\-те\-ка>> с условием 
бесплатного предоставления на ее сервере ({\sf http://www.elibrary.ru}), чем в 
настоящее время могут пользоваться российские ученые.
     
     Российский фонд фундаментальных исследований заключил новые договора с рядом зарубежных издательств о 
доступе к их журналам, но уже в режиме онлайн и только для своих 
грантодержателей.
     
     С этого периода и по настоящее время в России существуют два 
основных централизованных канала сетевого доступа учреждений РАН к 
зарубежной\linebreak научной информации~--- за счет \mbox{РФФИ} при посредстве 
Внешнеэкономического
объединения <<Академинторг>> и за счет средств Мин\-обр\-на\-у\-ки при посредстве 
\mbox{НЭИКОН}. 
Кроме централизованных\linebreak источников подписки в масштабах страны доступ 
к зарубежным научным журналам и БД приобретают 
вышеперечисленные центральные биб\-лио\-те\-ки РАН за счет средств, 
выделяемых Президиумом РАН и руководством ее региональных отделений, 
а также некоторые академические институты за счет своих средств. Однако 
количество ресурсов, приобретаемых академическими организациями, в 
десятки раз меньше количества ресурсов, приобретаемых РФФИ и НЭИКОН. 
     
     В настоящее время РФФИ финансирует своим грантодержателям (на 
уровне организаций, через которые осуществляется оплата средств по 
грантам) доступ к журналам шести издательств: Wiley (1600~журналов), The 
American Mathematical Society (предоставляется реферативная БД MathSciNet
(MSN), включающая около двух миллионов описаний статей), American 
Physical Society (9~журналов), Institute of Physics~(49 журналов), The Royal 
Society of Chemistry (6~журналов), Elsevier (Freedom Collection~--- около 
1700~журналов). До 2011~г.\ предо\-став\-лял\-ся также доступ к журналам 
издательства Springer, но в 2012~г.\ \mbox{РФФИ} отказался от этой подписки, 
мотивируя это решение нехваткой финансовых средств (одновременно было 
сокращено число доступных журналов The Royal Society of Chemistry с 23 
до~6). С~2011~г.\ грантодержателям \mbox{РФФИ} стали доступны журналы одной 
из коллекций издательства Elsevier (Freedom Collection~--- более 
1700~журналов).
     
     За счет средств РФФИ также организован доступ пяти крупнейших 
академических биб\-лио\-тек к известной БД Web of Knowledge, которая широко 
используется для определения публикационной активности и уровня 
цитирования научных публикаций.
     
     Следует заметить, что, лишившись в 2012~г.\ доступа к текущим 
журналам издательства Springer, пользователи РФФИ лишились и доступа к 
журналам предыдущих лет издания, подписка на которые была ранее 
оплачена. Согласно условиям контракта, при прекращении подписки для 
доступа к ранее оплаченным журналам каждая организация должна 
заплатить поставщику определенную сумму в качестве компенсации затрат 
на поддержку его серверов. 
     
     Как указывалось выше, каждый поставщик в зависимости от суммы 
контракта формулирует свои условия предоставления доступа к своим 
ресурсам. В~частности, ограничивает число пользователей, IP-ад\-ре\-сов или 
количество доступных журналов. Это обусловливает ограничения для 
грантодержателей \mbox{РФФИ} в получении доступа к сетевым ресурсам. Каж\-дый 
грантодержатель в начале 2012~г.\ должен был выбрать из предложенного 
\mbox{РФФИ} списка от одного до четырех издательств, журналы которых ему 
необходимы, и сообщить о своем выборе \mbox{РФФИ}. Последний, в зависимости 
от возможностей, диктуемых контрактами, принимал окончательное 
решение, кому и какие ресурсы предоставить. 
     
     <<Национальный электронно-информационный консорциум>>, 
     включающий в свой состав несколько сот организаций 
науки и образования, предо\-став\-ля\-ет им в 2012~г.\ за счет средств 
Министерства образования и науки доступ к полным текстам журналов 
следующих издательств, представляющих интерес для РАН: American 
Chemical Society (ACS~--- 38~журналов), American Institute of Physics 
     (AIP~--- 10~журналов), Annual Reviews Sciences Collection (AR~--- 
37~журналов), Business Source Complete (BSC~--- около 3500~журналов), 
Computers \& Applied Sciences Complete (CASC~--- около 950~журналов), 
Nature Publishing Group (NPG~--- 8~журналов), Oxford University Press 
(OUP~--- более 200~журналов), Optical Society of America (OSA~--- 
14~журналов), Sage STM (Science, Technology \& Medicine~--- более 
100~журналов), SPIE~--- International Society for Optics and Photonics 
(6~журналов и материалы конференций), Taylor \& Francis (T\&F~--- более 
1000~журналов), Georg Thieme Verlag KG (Thieme~--- 5~журналов) The 
American Association for the Advancement of Science (AAAS~--- журнал 
Science).
     
     Для сравнения~--- БЕН РАН на средства, выделенные ей Президиумом 
РАН в рамках целевого финансирования на приобретение научной 
литературы в 2011~г., смогла приобрести права сетевого доступа на 2012~г.\ 
лишь к 142~наименованиям журналов, отсутствующих в списках \mbox{РФФИ} и 
\mbox{НЭИКОН} (по соглашениям с поставщиками доступ предоставляется не 
только из центрального здания БЕН РАН, но и из ее отделов, расположенных 
в научных учреждениях РАН).
     
<<Национальный электронно-информационный консорциум>>, 
работая по контракту с Минобрнауки, уделяет серьезное 
внимание анализу использования ресурсов, предоставляемых научным 
организациям. Эту работу, касающуюся академических учреждений, с 
2010~г.\ по договору с НЭИКОН проводит БЕН РАН. В~ходе проводимого 
анализа был получен ряд интересных предварительных (работы 
заканчиваются в 2013~г.)\  результатов, которые приведены ниже.



 \begin{figure*}[b]
     \vspace*{1pt}
 \begin{center}
 \mbox{%
 \epsfxsize=99mm
 \epsfbox{glu-1.eps}
 }
 \end{center}
 \vspace*{-6pt}
\begin{center}
{\small Распределение числа выгрузок по издательствам}
\end{center}
     \end{figure*}

     
     По 14 издательствам, используемым в учреждениях РАН в 2010~г., в 
среднем в месяц выгружалось\linebreak\vspace*{-12pt}

\pagebreak

\begin{center}
 \vspace*{-6pt}
{{\tablename~1}\ \ \small{Активность использования ресурсов}}

\vspace*{6pt}

      \begin{tabular}{|l|c|c|}
      \hline
\multicolumn{1}{|c|}{Ресурс}&\tabcolsep=0pt\begin{tabular}{c}Количество\\ журналов\end{tabular} &
\tabcolsep=0pt\begin{tabular}{c}Количество\\ выгрузок\\ в месяц\end{tabular}\\
\hline
ACS &\hphantom{9}38&23\,806\hphantom{9}\\
AIP &\hphantom{9}10 &12\,930\hphantom{9}\\
NPG & \hphantom{99}8 &6378\\
T\&F &1547\hphantom{9}&3236\\
AAAS (Science) &\hphantom{99}1 &3015\\
OSA &\hphantom{9}14&2505\\
OUP\_Full &217&2106\\
Thieme &\hphantom{99}5&2035\\
SPIE &\hphantom{99}6&1568\\
Cell &\hphantom{9}15 &1286\\
Annual Review &\hphantom{9}37&\hphantom{9}402\\
SAGE &382&\hphantom{9}240\\
ACM &420&\hphantom{99}91\\
BSC &3345\hphantom{9}&\hphantom{99}42\\
\hline
\end{tabular}
\end{center}

%\pagebreak

\vspace*{12pt}

   


\addtocounter{table}{1}
\setcounter{figure}{0}

\noindent
 59\,642~статьи. Ресурсы использовались 
186~организациями РАН (журнал Science~--- 111~организаций, журналы 
AIP~--- 106~организаций, журналы ACS~--- 
82~организации, журналы группы Nature (NPG~--- Nature Publishing Group)~--- 
70~организаций и~т.\,д.). В целом, в 2010~г.\ активность учреж\-де\-ний РАН, 
измеряемая числом выгрузок полных текстов статей в месяц, выглядит 
следующим образом (табл.~1).
     

     
     По данным табл.~1 представлен график (см.\ рисунок).
     
    
     Данный график имеет две точки перегиба (после NPG 
и после группы журналов Cell). Суммарное число выгрузок до первой точки 
перегиба составляет 72\% от общего количества статей, выгруженных РАН, а 
до второй~--- 97\%.
     
      Наибольшим спросом у ученых РАН пользуются журналы ACS, AIP и NPG. 
Наименьшим спросом~--- журналы издательств Business Source Complete, 
Association for Computing Machinery (ACM), Sage и Annual Review. В средней 
части таблицы~--- пользующиеся, тем не менее, заметным спросом журналы 
издательств T\&F, AAAS (Science), OSA, 
OUP, Thieme, SPIE и Cell. 

\begin{table*}[b]\small
\vspace*{-6pt}
\begin{center}
\Caption{Использование ресурсов ОНИТ РАН}
\vspace*{2ex}

\begin{tabular}{|l|c|}
\hline
\multicolumn{1}{|c|}{Ресурс}&Число выгрузок
в месяц\\
\hline
American Institute of Physics (AIP)&413\hphantom{9}\\
Optical Society of America (OSA)&331\hphantom{9}\\
Society of Photographic Instrumentation Engineers (SPIE) &94\\
American Chemical Society (ACS)&42\\
AAAS (Science)&36\\
Nature&31\\
Sage &18\\
Nature Physics &17\\
Nature Nanotechnology&13\\
Association for Computing Machinery (ACM)&10\\
Nature Photonics &\hphantom{9}8\\
Nature Materials&\hphantom{9}7\\
Nature Chemistry &\hphantom{9999}0,25\\
Nature Methods &\hphantom{9999}0,17\\
Business Source Complete &\hphantom{9}0\\
Cell&\hphantom{9}0\\
Nature Biotechnology &\hphantom{9}0\\
Oxford University Press. Mathematics \& Computing &\hphantom{9}0\\
Oxford University Press BioMed &\hphantom{9}0\\
Oxford University Press Life &\hphantom{9}0\\
Oxford University Press Med &\hphantom{9}0\\
Oxford University Press STM &\hphantom{9}0\\
Taylor \& Francis Bio &\hphantom{9}0\\
Taylor \& Francis Chem &\hphantom{9}0\\
Taylor \& Francis Earth &\hphantom{9}0\\
Taylor \& Francis. Natural Sciences &\hphantom{9}0\\
Taylor \& Francis Med &\hphantom{9}0\\
Taylor \& Francis. Other &\hphantom{9}0\\
Taylor \& Francis. Physics \& Mathematics &\hphantom{9}0\\
Taylor \& Francis. Technique&\hphantom{9}0\\
\hline
\end{tabular}
\end{center}
\end{table*}
     
     До настоящего времени доступ к журналам, выписываемым через 
НЭИКОН, предоставлялся бесплатно. Со второй половины 2012~г.\ 
НЭИКОН планирует (по указанию Минобрнауки) взимать часть стоимости 
ресурсов с получателей. В~этой ситуации станет актуальным вопрос, не 
дешевле ли будет вместо оплаты доступа к базам малоспрашиваемых 
журналов заказывать электронные копии отдельных статей с 
<<постатейной>> оплатой. Этот вид сервиса достаточно хорошо развит за 
рубежом, им пользуются как отдельные ученые, так и научные биб\-лио\-те\-ки 
по заказам своих пользователей. Как правило, к оглавлениям и аннотациям 
статей из научных журналов предоставляется свободный доступ. Функции 
<<посредника>>, осуществляющего прием заказов на статьи от ученых РАН, 
контакты с поставщиками, оплату заказов в валюте могли бы взять на себя 
центральные академические биб\-лио\-те\-ки. По данным БЕН РАН, стоимость 
электронной копии статьи объемом до 20~страниц в биб\-лио\-те\-ках 
континентальной Европы составляет в среднем около 10~евро, что при 
годовых объемах 50--60~статей будет существенно дешевле, чем оплата 
доступа ко всем журналам издательства, не являющегося приоритетным для 
РАН. Очевидно, что такой подход потребует некоторого перераспределения 
средств и организационной перестройки биб\-лио\-теч\-ных служб, но он может 
оказаться достаточно эффективным.
     
     <<Национальный электронно-информационный консорциум>> достаточно оперативно реагирует на изменения 
спроса на журналы. Так, по \mbox{результа\-там} проведенного анализа с 2011~г.\ 
была прекращена подписка на журналы ACM. 
Что касается журналов издательств Business Source Complete, они 
пользуются значительным спросом у второй большой группы 
     ор\-га\-ни\-за\-ций--поль\-зо\-ва\-те\-лей \mbox{НЭИКОН}: российских университетов. 
Издательство Sage, которое также оказалось в нижней части рейтинговой 
таб\-ли\-цы, в 2010~г.\ было пред\-став\-ле\-но в \mbox{НЭИКОН} только журналами по 
гуманитарным и социальным наукам, хотя выпускает оно и другую научную 
литературу. В~настоящее время НЭИКОН предполагает дополнить этот 
ресурс журналами группы Sage STM (Science, Technology \& Medicine), и, 
возможно, результаты его востребованности академическими организациями 
изменятся.
     
     Получив полные данные о спросе на журналы НЭИКОН, авторы статьи 
провели анализ их использования сотрудниками отделений РАН. В~табл.~2 
представлены показатели Отделения нанотехнологий и информационных 
технологий (ОНИТ). 


     
     В табл.~2 более подробно, чем в предыдущей, раскрыты журналы 
группы Nature, а также журналы издательств Taylor \& Francis и Oxford 
University Press, поэтому таблица формально включает 30~ресурсов, но 
фактически это те же 14~ресурсов, раскрытых более подробно.
     
     Как видно из табл.~2, наибольший интерес для сотрудников ОНИТ 
     представляют журналы  AIP и OSA. Далее со 
значительным отрывом следуют\linebreak журна\-лы Society of Photographic 
Instrumentation Engineers. Среднюю группу (30--40~обращений к полным 
текстам в месяц) составляют журналы American Chemical Society, AAAS 
(Science) и основной журнал группы Nature. Значительно меньшим спросом 
пользуются остальные журналы группы Nature, журналы же остальных 
издательств не представляют для ОНИТ никакого интереса. 
     
     Россиский фонд фундаментальных исследований, в отличие от \mbox{НЭИКОН}, не 
     пред\-остав\-ля\-ет пользователям 
статистики использования пред\-став\-ля\-емых Фондом ресурсов (хотя БЕН РАН 
\mbox{обращалась} по этому поводу к руководству \mbox{РФФИ} и получила 
принципиальное согласие, но данные пока не получила), поэтому 
аналогичный анализ по этим ресурсам пока невозможен. Однако, по 
наблюдениям авторов статьи, журналы, предлагаемые через РФФИ, 
пользовались также весьма заметным спросом. Так, статьи из журналов 
издательства Springer в 2011~г.\ только в БЕН РАН выгружались в среднем 
158~раз в месяц, в ГПНТБ СО РАН~--- 489~раз; статьи издательства Elsevier 
в 2011~г.\ выгружались в БЕН РАН в среднем 1536~раз в месяц.
     
     Таким образом, в настоящее время в России созда\-на действующая 
система доступа к полным текстам нескольких тысяч зарубежных научных 
журналов. Эта система охватывает подавляющее большинство научных 
организаций РАН и пользуется значительной популярностью. Однако она не 
охватывает (в первую очередь в силу недостаточного финансирования) весь 
необходимый объем научной информации, требуемый для эффективного 
функционирования научных институтов и центров РАН. В~существующих 
условиях, к сожалению, не представляется возможным обеспечить доступ с 
каждого рабочего места сотрудника РАН ко всем необходимым ему 
журналам. Оптимизировать систему возможно за счет серьезного анализа 
фактического спроса, создания ранжированного списка наиболее 
востребованных журналов, выявления необходимых издательств и 
централизованного (на базе существующих или вновь создаваемых 
консорциумов) заключения договоров с этими из\-да\-тель\-ст\-вами.
     
     Другим фактором оптимизации системы является сокращение числа 
пользователей (IP-ад\-ре\-сов) каждого научного института (научного 
центра), получающих доступ к тому или иному ресурсу, что позволит 
снизить стоимость договоров с поставщиками. В~этом плане представляется 
целесообразным подход, реализованный в БЕН РАН, которая при 
заключении договоров оговаривает права доступа к ресурсам не только из 
здания Центральной биб\-лио\-те\-ки, но и из ее отделов (биб\-лио\-тек) в 
на\-уч\-но-ис\-сле\-до\-ва\-тель\-ских учреждениях (НИУ) РАН. 
При этом поставщикам официально сообщаются IP-ад\-ре\-са биб\-лио\-теч\-ных 
компьютеров, с которых сотрудники институтов могут читать журналы. 
Такая схема работает уже несколько лет и позволяет без существенных 
затрат обеспечивать важнейшей информацией (хотя и не с каждого 
компьютера института) сотрудников более 40~институтов и научных центров 
Москвы и Московского региона. Существенным ограничением этой схемы 
является требование поставщиков, чтобы отделы БЕН в НИУ РАН имели 
компьютеры с выделенными IP-ад\-ре\-са\-ми, поэтому биб\-лио\-те\-кам, 
работающим через про\-кси-сер\-ве\-ры институтов, доступ к ресурсам 
предоставлен быть не может.
     
     Необходимо отметить, что в России и, в част\-ности, в РАН доля 
финансирования, выделяемая на информационное обеспечение науки, 
существенно меньше принятой в развитых и развивающихся странах. 
Согласно мировой практике эта доля составляет от~8\% до~12\% от 
ассигнований на научные исследования. У~нас она не достигает и~1\%.
     
     В существующих условиях, когда биб\-лио\-те\-кам катастрофически не 
хватает централизованно выделяемых РАН средств на приобретение 
информационных ресурсов, необходимо и впредь развивать идеи создания 
академических и межведомственных консорциумов по доступу к научной 
информации, интеграции финансов, выделяемых биб\-лио\-те\-кам, и 
собственных финансов НИУ, перехода на новые системы информационного 
обслуживания пользователей.


{\small\frenchspacing
{%\baselineskip=10.8pt
\addcontentsline{toc}{section}{Литература}
\begin{thebibliography}{9}
     
     
\bibitem{1-g}
The International Coalition of Library Consortia (ICOLC). {\sf 
http://www.library.yale.edu/consortia}.
\bibitem{2-g}
FinELib, the National Electronic Library. The National Library of Finland. {\sf 
http://www.nationallibrary.fi/\linebreak libraries/finelib/finelibconsortium.html}.
\bibitem{3-g}
\Au{H$\ddot{\mbox{o}}$kli E.} Libraries in Finland establish consortia~// Liber 
Quarterly: The J.~European Research Libraries, 2001. Vol.~11. No.\,1. P.~53--59.
\bibitem{4-g}
\Au{Hormia-Poutanen K., Xenidou-Dervou~C., Kupryte~R., Stange~K., 
Kuznetsov~A., Woodward~H.} Consortia in Europe: Describing the various 
solutions through four country examples~// Library Trends, 2006. Vol.~54. No.\,3. 
{\sf https://dspace.lib.cranfield.ac.uk/handle/1826/1014}.
\bibitem{5-g}
\Au{Литвинова Н.\,Н.} Электронные документы: отбор, использование и 
хранение~// Библиотека, 2005. №\,6. С.~6--9.


\label{end\stat}

\bibitem{7-g}
\Au{Никаньшин Д.\,П., Туриянский~И.\,Е., Астафьев~М.\,Н.} О~развитии 
зеркального сервера научной электронной биб\-лио\-те\-ки РФФИ~// 
Исследования по информатике, 2003. Вып.~5. С.~133--142.

\bibitem{6-g}
\Au{Хельферих П., Красикова~О.\,Л.} Научная информация для российских 
биб\-лио\-тек~// Библиотеки и ассоциации в меняющемся мире: новые 
технологии и новые формы сотрудничества: Мат-лы 7-й Междунар. 
конф.~--- Судак, Крым, Украина, 2000.~--- Т.~2. С.~127--128.


\end{thebibliography} } }

\end{multicols}