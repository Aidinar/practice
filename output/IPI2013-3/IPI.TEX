\documentclass[10pt]{book}
\usepackage[utf8]{inputenc}

\usepackage{latexsym,amssymb,amsfonts,amsmath,indentfirst,shapepar,%fleqn,%
picinpar,shadow,floatflt,enumerate,multicol,colortbl,ipi}

\usepackage{rotating}
\usepackage{mathrsfs}
\usepackage[noend]{algorithmic}
\usepackage{ulem}

\input{epsf}

%\nofiles

%\includeonly{kuznetsov,obchak,avtor,avtor-eng} %+pdf
%\includeonly{obchak,avtor}
%\includeonly{pred}      %+pdf
%\includeonly{podgot-rus,podgot-eng}  %+
%\includeonly{ocherk} %+

%\includeonly{andreev}    %+1+pdf+авт
%\includeonly{klimenkov}  %+2+pdf+
%\includeonly{stupnikov}  %+3+pdf+авт
%\includeonly{shkotin}    %+4+pdf+авт
%\includeonly{mashechkin} %+5+pdf+авт???
%\includeonly{kalenov}    %+6+pdf+авт
%\includeonly{listopad}   %+7+pdf+авт
%\includeonly{kovalev}    %+8pdf+авт
%\includeonly{kozhunova}  %9+pdf
%\includeonly{korenkov}   %+10+pdf+
%\includeonly{korolev}    %+11+pdf
%\includeonly{kuznetsov}  %+12+pdf+
                                              %\includeonly{belyaev}    %+13
%\includeonly{belyaev_updated}    %+13+pdf


%\includeonly{rekl}       %+


%\includeonly{toc-rus, toc-en}
%\includeonly{obchak} %,toc-en}

%\includeonly{obchak}
%\includeonly{reshal}  %pdf
%\includeonly{eng-index}
%\includeonly{cover3}

\usepackage{acad}
%\usepackage{courier}
\usepackage{decor}
\usepackage{newton}
\usepackage{pragmatica}
\usepackage{zapfchan}
\usepackage{petrotex}
\usepackage{bm}                     % полужирные греческие буквы
\usepackage{upgreek}                % прямые греческие буквы
\usepackage{eufrak}
\usepackage{verbatim}

\renewcommand{\bottomfraction}{0.99}
\renewcommand{\topfraction}{0.99}
\renewcommand{\textfraction}{0.01}

\setcounter{secnumdepth}{1} %здесь - 3 + chapter = 4

\arraycolsep=1.5pt

%\usepackage[pdftex]{graphicx}

%\usepackage{oz}

%NEW COMMANDS
\newsavebox{\fmbox}
\newenvironment{fmpage}[1]
{\begin{lrbox}{\fmbox}\begin{minipage}{#1}}
{\end{minipage}\end{lrbox}\fbox{\usebox{\fmbox}}}

\renewcommand*{\hm}[1]{#1\nobreak\discretionary{}%
            {\hbox{$\mathsurround=0pt #1$}}{}} %% Дублирует знаки операций
                               %при переносе в формуле (перед знаком, который 
                               %надо продублировать ставится команда \hm)

%\newcommand{\endproof}{\hfill$\Box$}
\renewcommand{\r}{\mathbb{R}}
\newcommand{\I}{{\rm I\hspace{-0.7mm}I}}
%\newcommand{\Ikl}{{\tt{1}}\hspace*{-1.44mm}\mathtt{1}}
\newcommand{\Ik}{\mbox{{\small \tt {1}}\hspace{-1.5mm}{\tt 1}}}
\newcommand{\argmin}{\mathop{\mathrm{arg}\,\mathrm{min}}}
\newcommand{\argmax}{\mathop{\mathrm{arg}\,\mathrm{max}}}
%\newcommand{\capr}{\mathop{\cap\,}}
%\newcommand{\cupr}{\mathop{\cup\,}}
%\def\argmin{\mathop{arg\,min}}

\def\vrp{\varphi}
\def\prt{\partial}
\def\mm{{\rm M}}

\newcommand{\il}[2]{\int\limits_{#1}^{#2}}%интеграл с пределами #1 и #2

\def\ss2{\mathop {\sum\limits^p\sum\limits^p}}
\def\sss{\sum\limits}
\def\tr{,\,\ldots\,,\,}
\def\rk{\right]}
\def\lk{\left[}
\def\rf{\right\}}
\def\lf{\left\{}
\def\lv{\,\left\vert}
\def\rv{\right\vert\,}


\def\ee{{\cal E}}
\def\ww{{\cal W}}
\def\yy{{\cal Y}}
\def\vv{{\cal V}}

\newcommand{\R}{\mathbb R}
\newcommand{\E}{\mathbb E}
\newcommand{\N}{\mathbb N}

\newcommand{\h}{{\bf H}}
\newcommand{\p}{{\sf P}}  % вероятность

\newcommand{\e}{{\sf E}}  % мат. ожидание
\newcommand{\D}{{\sf D}}  % дисперсия
\newcommand{\eps}{\varepsilon}
\newcommand{\vp}{{\mathbf p}}
\newcommand{\vz}{{\mathbf z}}
\newcommand{\vx}{{\mathbf x}}
\newcommand{\vf}{{\mathbf f}}
\newcommand{\F}{{\mathcal F}}
\def\ap{{\mathrm{ЭР}}}
\newcommand{\ud}{\Delta_n} %uniform ditance
\newcommand{\nud}{\Delta_n(x)}

\newcommand{\abs}[1]{\left\vert#1\right\vert}
\newcommand{\norm}[1]{\left\Vert#1\right\Vert}
\def\da{(\Delta_t,A)}

\def\w{\omega}
\def\W{\Omega}
\def\iii{\int\limits}
\def\inh{\int\limits_{nh}^{(n+1)h}}
\def\iin{\int\limits_{-\infty}^\infty}
\def\sumin{\sum_{i=1}^N}


\def\bxt{(Y,t)}
\def\xt{(y,t)}

\def\ovth{{\fr{\tau-nh}{h}}}


\DeclareMathOperator{\sign}{sign}

%\newcommand{\gr}{{\geqslant}}


\newcommand{\g}{\mbox{\textit{g}}}

\renewcommand{\la}{\lambda}
\newcommand{\si}{\sigma}
\newcommand{\alp}{\alpha}

%\newcommand{\pto}{\stackrel{P}{\longrightarrow}} % сходимость по веpоятности

\newcommand{\eqd}{\stackrel{d}{=}} % равенство по pаспpеделению

%\newcommand{\kp}{\kappa}
%\def\Q{{\cal Q}} \def\H{{\cal H}}
%\newcommand{\bet}{\beta_{2+\delta}}


%\newtheorem{definition}{Определение}
%\renewcommand{\thedefinition}{\arabic{definition}.}
%END NEW COMMANDS

%\renewcommand{\baselinestretch}{1.2}

%\pagestyle{myheadings}

\setlength{\textwidth}{167mm}      % 122mm
\setlength{\textheight}{658pt}
%\setlength{\textheight}{635.6pt}
\setlength{\columnsep}{4.5mm}

\setcounter{secnumdepth}{4}

%\addtolength{\headheight}{2pt}
%\addtolength{\headsep}{-2mm}

%\addtolength{\topmargin}{-20mm}  % for printing


\hoffset=-30mm  % From Yap
%\hoffset=-20mm  % From Acrobat

%\voffset=0mm % From Yap
%\voffset=-15mm   % From Acrobat

\addtolength{\evensidemargin}{-9.5mm} % for printing
\addtolength{\oddsidemargin}{9.5mm}  % for printing

%\renewcommand{\thefootnote}{\fnsymbol{footnote}}
%\renewcommand{\thefootnote}{\arabic{footnote}}
\renewcommand{\figurename}{\protect\bf Рис.}
\renewcommand{\tablename}{\protect\bf Таблица}

\newcommand{\Caption}[1]{\caption{\protect\small %\baselineskip=2.5ex
#1}}

\renewcommand{\thefigure}{\arabic{figure}}
\renewcommand{\thetable}{\arabic{table}}
\renewcommand{\theequation}{\arabic{equation}}
\renewcommand{\thesection}{\arabic{section}}

\renewcommand{\contentsname}{СОДЕРЖАНИЕ}
\newcommand{\fr}[2]{\displaystyle\frac{\displaystyle #1\mathstrut}{\displaystyle #2\mathstrut}}

%\renewcommand{\thefootnote}{\fnsymbol{footnote}}
%\newcommand{\g}{\mbox{\textit{g}}}

%\newcommand{\Caption}[1]{\caption{\protect\small\baselineskip=2ex #1}}
\newcounter{razdel}
\setcounter{razdel}{0}


\newcommand{\titel}[4]{%
\

\vspace*{5pt}

\ifodd\therazdel {\raggedright\noindent\Large\textrm\textbf
 \lineskip .75em
  \baselineskip=3.2ex #1 \par}
\vskip 1em {\noindent\large\textrm\textbf #2 \par}
\addcontentsline{toc}{subsection}{{\textrm\textbf #3}\protect\newline #1}
\def\rightheadline{\underline{\noindent\hbox to \textwidth{\hfill\small\textrm{#4}
%\hfill \large\bf\thepage
}}}
\def\leftheadline{\underline{\noindent\parbox{\textwidth}{
%\raggedleft\large\bf\thepage \hfill
\small\textit{#3}\hfill}}}
\def\leftfootline{\small{\textbf{\thepage}
\hfill ИНФОРМАТИКА И ЕЁ ПРИМЕНЕНИЯ\ \ \ том~7\ \ \ выпуск 3\ \ \ 2013}
}%
 \def\rightfootline{\small{ИНФОРМАТИКА И ЕЁ ПРИМЕНЕНИЯ\ \ \ том~7\ \ \ выпуск~3\ \ \ 2013
\hfill \textbf{\thepage}}} 
\vskip 2em \setcounter{figure}{0}
\setcounter{table}{0} 
\setcounter{equation}{0} 
\setcounter{section}{0}
\setcounter{subsection}{0} 
\setcounter{subsubsection}{0}
\setcounter{footnote}{0} 
\setcounter{razdel}{0}
%\end{flushleft}
\else {
 \raggedright\noindent\Large\textrm\textbf
 \lineskip .75em
\baselineskip=3.2ex #1 \par} \vskip 1em
%\begin{flushleft}
{\noindent\large\textrm\textbf #2 \par}
\addcontentsline{toc}{subsection}{{\textrm\textbf #3}\protect\newline #1}
\def\rightheadline{\underline{\noindent\hbox to \textwidth{\hfill\small\textrm{#4}
%\hfill \large\bf\thepage
}}}
\def\leftheadline{\underline{\noindent\parbox{\textwidth}{%\raggedleft\large\bf\thepage \hfill
\small\textit{#3}\hfill}}}
\def\leftfootline{\small{\textbf{\thepage}
\hfill ИНФОРМАТИКА И ЕЁ ПРИМЕНЕНИЯ\ \ \ том~7\ \ \ выпуск~3\ \ \ 2013}
}%
 \def\rightfootline{\small{ИНФОРМАТИКА И ЕЁ ПРИМЕНЕНИЯ\ \ \ том~7\ \ \ выпуск~3\ \ \ 2013
\hfill \textbf{\thepage}}} \vskip 2em \setcounter{figure}{0}
\setcounter{table}{0} \setcounter{equation}{0} \setcounter{section}{0}
\setcounter{subsection}{0} \setcounter{subsubsection}{0}
\setcounter{footnote}{0}
%\end{flushleft}
\fi}

\newcommand{\titelr}[2]{%
\

\vspace*{5pt}

\ifodd\therazdel {\raggedright\noindent\large\textrm\textbf
 \lineskip .75em
  \baselineskip=3.2ex #1 \par}
\vskip 1em {\noindent\normalsize\textrm\textbf #2 \par}
\else {
 \raggedright\noindent\large\textrm\textbf
 \lineskip .75em
\baselineskip=3.2ex #1 \par} \vskip 1em
%\begin{flushleft}
{\noindent\normalsize\textrm\textbf #2 \par}
\fi}

\newcommand{\titele}[5]{%
\

%\vspace*{5pt}

\ifodd\therazdel {\raggedright\noindent%\large
\textrm\textbf
 \lineskip .75em
%  \baselineskip=3.2ex
#1 \par}
\vskip .5em {\noindent\large\textrm\textbf #2 \par}
\vskip .5em
 {\noindent\textrm #3 \par}
\addcontentsline{toc}{subsection}{{\textrm\textbf #1}\protect\newline #2}
\def\rightheadline{\underline{\noindent\hbox to \textwidth{\hfill\small\textrm{#4}
%\hfill \large\bf\thepage
}}}
\def\leftheadline{\underline{\noindent\parbox{\textwidth}{
%\raggedleft\large\bf\thepage \hfill
\small\textrm{#5}\hfill}}}
\def\leftfootline{\small{\textbf{\thepage}
\hfill ИНФОРМАТИКА И ЕЁ ПРИМЕНЕНИЯ\ \ \ том~7\ \ \ выпуск~3\ \ \ 2013}
}%
 \def\rightfootline{\small{ИНФОРМАТИКА И ЕЁ ПРИМЕНЕНИЯ\ \ \ том~7\ \ \ выпуск~3\ \ \ 2013
\hfill \textbf{\thepage}}} \vskip 1em \setcounter{figure}{0}
\setcounter{table}{0} \setcounter{equation}{0} \setcounter{section}{0}
\setcounter{subsection}{0} \setcounter{subsubsection}{0}
\setcounter{footnote}{0} \setcounter{razdel}{0}
%\end{flushleft}
\else {
 \raggedright\noindent%\large
 \textrm\textbf
 \lineskip .75em
%\baselineskip=3.2ex
#1 \par} \vskip .5em
%\begin{flushleft}
{\noindent\large\textrm\textbf #2 \par} \vskip .5em
 {\noindent\textrm #3 \par}
\addcontentsline{toc}{subsection}{{\textrm\textbf #1}\protect\newline #2}
\def\rightheadline{\underline{\noindent\hbox to \textwidth{\hfill\small\textrm{#4}
%\hfill \large\bf\thepage
}}}
\def\leftheadline{\underline{\noindent\parbox{\textwidth}{%\raggedleft\large\bf\thepage \hfill
\small\textrm{#5}\hfill}}}
\def\leftfootline{\small{\textbf{\thepage}
\hfill ИНФОРМАТИКА И ЕЁ ПРИМЕНЕНИЯ\ \ \ том~7\ \ \ выпуск~3\ \ \ 2013}
}%
 \def\rightfootline{\small{ИНФОРМАТИКА И ЕЁ ПРИМЕНЕНИЯ\ \ \ том~7\ \ \ выпуск~3\ \ \ 2013
\hfill \textbf{\thepage}}} \vskip 1em \setcounter{figure}{0}
\setcounter{table}{0} \setcounter{equation}{0} \setcounter{section}{0}
\setcounter{subsection}{0} \setcounter{subsubsection}{0}
\setcounter{footnote}{0}
%\end{flushleft}
\fi}

\def\Abst#1{
\begin{center}\small\nwt
\parbox{150mm}{%\baselineskip=2.5ex
\textbf{Аннотация:}\ \
%\hspace*{\parindent}
#1}
\end{center}}
\def\Abste#1{
\begin{center}\small\nwt
\parbox{150mm}{%\baselineskip=2.5ex
\textbf{Abstract:}\ \
%\hspace*{\parindent}
#1}
\end{center}}

\def\KW#1{
\begin{center}\small\nwt
\parbox{150mm}{%\baselineskip=2.5ex
\textbf{Ключевые слова:}\ \ #1}
\end{center}}

\def\KWE#1{
\begin{center}\small\nwt
\parbox{150mm}{%\baselineskip=2.5ex
\textbf{Keywords:}\ \ #1}
\end{center}}


\def\KWN#1{
%\begin{center}
%\small
%\parbox{150mm}\end{center}
}

\renewcommand{\thesubsection}{\thesection.\arabic{subsection}\hspace*{-5pt}}
\renewcommand{\thesubsubsection}{\thesubsection\hspace*{5pt}.\arabic{subsubsection}\hspace*{-3pt}}

\newcommand{\Ack}{\subsection*{\protect\large Acknowledgments}\noindent}


\begin{document}
\Rus

\nwt
%\ptb

%\renewcommand{\contentsname}{\protect\Large\bf Содержание}

\setcounter{tocdepth}{2}

%\tableofcontents

\renewcommand{\bibname}{\protect\rmfamily Литература}
  \def\Au#1{{\it #1}}

%\newcommand{\No}{№}
  \newcommand{\tg}{\,\mathrm{tg}\,}
    \newcommand{\ctg}{\,\mathrm{ctg}\,}
  \newcommand{\arctg}{\,\mathrm{arctg}\,}
  
\def\forallb{\mathop{\forall}}
\def\cupb{\mathop{\cup}}
\def\existsb{\mathop{\exists}}

\setcounter{page}{1}

\newpage
\addtocounter{razdel}{1}
%\def\razd{РЕГУЛИРУЕМЫЙ ЭЛЕКТРОПРИВОД ДЛЯ ЭЛЕКТРОЭНЕРГЕТИКИ}


\setcounter{page}{2}


%   { %\Large  
   { %\baselineskip=16.6pt
   
   \vspace*{-48pt}
   \begin{center}\LARGE
   \textit{Предисловие}
   \end{center}
   
   %\vspace*{2.5mm}
   
   \vspace*{25mm}
   
   \thispagestyle{empty}
   
   { %\small 

    
Вниманию читателей журнала <<Информатика и её применения>> предлагается 
очередной тематический выпуск <<Вероятностно-статистические методы и 
задачи информатики и информационных технологий>>. Предыдущие тематические 
выпуски журнала по данному направлению вышли в 2008~г.\ (т.~2, вып.~2), 
в 2009~г.\ (т.~3, вып.~3) и в 2010~г.\ (т.~4, вып.~2). 

Статьи, собранные в данном журнале, посвящены разработке новых вероятностно-статистических 
методов, ориентированных на применение к решению конкретных задач информатики и информационных 
технологий, а также~--- в ряде случаев~--- и других прикладных задач. Проблематика, охватываемая 
публикуемыми работами, развивается в рамках научного сотрудничества между Институтом проблем 
информатики Российской академии наук (ИПИ РАН) и Факультетом вычислительной математики и 
кибернетики Московского государственного университета им.\ М.\,В.~Ломоносова в ходе работ 
над совместными научными проектами (в том числе в рамках функционирования 
Научно-образовательного центра <<Вероятностно-статистические методы анализа рисков>>). 
Многие из авторов статей, включенных в данный номер журнала, являются активными участниками 
традиционного международного семинара по проблемам устойчивости стохастических моделей, 
руководимого В.\,М.~Золотаревым и В.\,Ю.~Королевым; регулярные сессии этого семинара 
проводятся под эгидой МГУ и ИПИ РАН (в 2011~г.\ указанный семинар проводится в октябре 
в Калининградской области РФ). 

Наряду с представителями ИПИ РАН и МГУ в число авторов данного выпуска журнала входят 
ученые из Научно-исследовательского института системных исследований РАН, Института 
проблем технологии микроэлектроники и особочистых материалов РАН, Института 
прикладных математических исследований Карельского НЦ РАН, Московского 
авиационного института, Вологодского государственного педагогического университета, 
НИИММ им.\ Н.\,Г.~Чеботарева, Казанского государственного университета, Дебреценского 
университета (Венгрия).

Несколько статей выпуска посвящено разработке и применению стохастических методов и 
информационных технологий для решения различных прикладных задач. В~работе В.\,Г.~Ушакова 
и О.\,В.~Шестакова рассмотрена задача определения вероятностных характеристик случайных 
функций по распределениям интегральных преобразований, возникающих в задачах эмиссионной 
томографии. В~статье Д.\,О.~Яковенко и М.\,А.~Целищева рассмотрены некоторые вопросы 
математической теории риска и предложен новый подход к диверсификации инвестиционных 
портфелей. Работа И.\,А.~Кудрявцевой и А.\,В.~Пантелеева посвящена построению и 
исследованию математической модели, описывающей динамику сильноионизованной плазмы. 
В~статье П.\,П.~Кольцова изучается качество работы ряда алгоритмов сегментации изображений. 
Статья А.\,Н.~Чупрунова и И.~Фазекаша посвящена вероятностному анализу числа без\-оши\-бочных 
блоков при помехоустойчивом кодировании; получены усиленные законы больших чисел для указанных 
величин.

В данном выпуске традиционно присутствует тематика, весьма активно разрабатываемая в течение 
многих лет специалистами ИПИ РАН и МГУ,~--- методы моделирования и управления для 
информационно-телекоммуникационных и вычислительных систем, в частности методы 
теории массового обслуживания. В~статье А.\,И.~Зейфмана с соавторами рассматриваются 
модели обслуживания, описываемые марковскими цепями с непрерывным временем в случае 
наличия катастроф. В~работе М.\,М.~Лери и И.\,А.~Чеплюковой рассматриваются случайные 
графы Интернет-типа, т.\,е.\ графы, степени вершин которых имеют степенные распределения; 
такие задачи находят применение при исследовании глобальных сетей передачи данных. 
Работа Р.\,В.~Разумчика посвящена исследованию систем массового обслуживания специального 
вида~--- с отрицательными заявками и хранением вытесненных заявок.

Ряд статей посвящен развитию перспективных теоретических 
вероятностно-статистических методов, которые находят широкое применение в различных 
задачах информатики и информационных технологий. В~работе В.\,Е.~Бенинга, А.\,К.~Горшенина 
и В.\,Ю.~Королева рассмотрена задача статистической проверки гипотез о числе компонент 
смеси вероятностных распределений, приводится конструкция асимптотически наиболее мощного 
критерия. Результаты этой работы найдут применение в ряде прикладных задач, использующих 
математическую модель смеси вероятностных распределений (в информатике, моделировании 
финансовых рынков, физике турбулентной плазмы и~т.\,д.). В~статье В.\,Ю.~Королева, 
И.\,Г.~Шевцовой и С.\,Я.~Шоргина строится новая, улучшенная оценка точности нормальной 
аппроксимации для пуассоновских случайных сумм; как известно, указанные случайные суммы 
широко используются в качестве моделей многих реальных объектов, в том числе в информатике, 
физике и других прикладных областях. Работа В.\,Г.~Ушакова и Н.\,Г.~Ушакова посвящена 
исследованию ядерной оценки плотности распределения; эти результаты могут применяться, 
в част\-ности, при анализе трафика в телекоммуникационных системах. Серьезные приложения 
в статистике могут получить результаты работы О.\,В.~Шестакова, в которой доказаны оценки 
скорости сходимости распределения выборочного абсолютного медианного отклонения к нормальному 
закону. 

\smallskip

Редакционная коллегия журнала выражает надежду, что данный тематический  выпуск 
будет интересен специалистам в области теории вероятностей и математической статистики 
и их применения к решению задач информатики и информационных технологий.
     
     %\vfill 
     \vspace*{20mm}
     \noindent
     Заместитель главного редактора журнала <<Информатика и её 
применения>>,\\
     директор ИПИ РАН, академик  \hfill
     \textit{И.\,А.~Соколов}\\
     
     \noindent
     Редактор-составитель тематического выпуска,\\
     профессор кафедры математической статистики факультета\\
      вычислительной математики и кибернетики МГУ им.\ М.\,В.~Ломоносова,\\
     ведущий научный сотрудник ИПИ РАН,\\ 
доктор физико-математических наук \hfill
      \textit{В.\,Ю.~Королев}
     
     } }
     }




\def\stat{andreev}

\def\tit{ПОДХОД К АВТОМАТИЗИРОВАННОМУ КОНТРОЛЮ РАБОТЫ СИСТЕМЫ 
ИЗВЛЕЧЕНИЯ ДАННЫХ С~ВЕБ-САЙТОВ$^*$}

\def\titkol{Подход к автоматизированному контролю работы системы 
извлечения данных с~веб-сайтов}

\def\autkol{А.\,М.~Андреев, Д.\,В.~Березкин, И.\,А.~Козлов, 
К.\,В.~Симаков}

\def\aut{А.\,М.~Андреев$^1$, Д.\,В.~Березкин$^2$, И.\,А.~Козлов$^3$, 
К.\,В.~Симаков$^4$}

\titel{\tit}{\aut}{\autkol}{\titkol}

{\renewcommand{\thefootnote}{\fnsymbol{footnote}}\footnotetext[1] {Статья рекомендована 
к публикации в журнале Программным комитетом конференции <<Электронные 
библиотеки: перспективные методы и технологии, электронные коллекции>> (RCDL-2012).}}

\renewcommand{\thefootnote}{\arabic{footnote}}
\footnotetext[1]{Московский государственный технический университет им.\ Н.\,Э.~Баумана, arkandreev@gmail.com}
\footnotetext[2]{Московский государственный технический университет им.\ Н.\,Э.~Баумана, dmitryb2007@yandex.ru}
\footnotetext[3]{Московский государственный технический университет им.\ Н.\,Э.~Баумана, kozlovilya89@gmail.com}
\footnotetext[4]{Московский государственный технический университет им.\ Н.\,Э.~Баумана, skv@ixlab.ru}


\Abst{Системы извлечения данных с веб-сайтов используют информацию о разметке 
HTML-стра\-ниц. Для обеспечения бесперебойной работы таких систем необходимо 
решить проблему своевременного обнаружения изменений структуры веб-сай\-тов. 
В~статье предложен подход к решению этой проблемы, предполагающий наличие двух 
этапов детектирования изменений верстки: оперативного и отложенного. В~основе 
первого из них лежит кластеризация, при этом HTML-до\-ку\-мент рассматривается как 
вектор некоторых характеристик. Второй этап основан на сравнении распределений этих 
характеристик для эталонного и тестового наборов документов. Проведена 
экспериментальная оценка предложенного подхода, демонстрирующая его практическую 
применимость.}

\KW{сбор текстовой информации; парсинг веб-сайтов; кластеризация; статистический 
анализ HTML-верстки} 

\vskip 14pt plus 9pt minus 6pt

      \thispagestyle{headings}

      \begin{multicols}{2}

            \label{st\stat}
  

\section{Введение}
  
    При разработке промышленных систем интеллектуальной обработки текстов 
класса Text Mining приходится сталкиваться с задачами сбора текстовой 
информации из открытых ин\-тер\-нет-ис\-точ\-ни\-ков, ее унификации и 
накопления. Методы автоматической обработки текстов (кластеризация, 
полнотекстовый поиск, выявление скрытых зависимостей) могут эффективно 
использоваться лишь при наличии актуальной, регулярно пополняющейся базы 
документов.


  
    В данной статье рассматривается решение задачи качественного сбора 
информации с новостных веб-сай\-тов. Эта информация включает в себя текст 
новости, а также сопутствующие метаданные: название, дату публикации, 
автора новости и~др. Под качественным сбором в первую очередь 
подразумевается очистка текста новости от окружающей его служебной 
информации: меню сайта, рекламных баннеров, блоков социальных сетей, 
комментариев пользователей и~т.\,д. 
  
    Основное внимание в данной работе уделено проблеме своевременного 
обнаружения изменения структуры опрашиваемых веб-сай\-тов. Пред\-ла\-га\-емый 
подход может быть использован как для обработки новостных сайтов, так и для 
сбора сообщений из электронных библиотек, блогов, форумов и социальных сетей.

\section{Постановка задачи}

\vspace*{-3pt}
  
\subsection{Функционирование системы сбора}

\vspace*{-3pt}
  
    Существует множество подходов к организации сбора открытых текстовых 
материалов с веб-сай\-тов. Как правило, система сбора использует информацию 
об HTML-раз\-мет\-ке целевых страниц для поиска в них нужной 
информации~[1]. Эта информация используется правилами распознавания, 
записываемыми на принятом в системе формальном языке. 

Распространение 
получили как ручной способ описания правил, когда правила распознавания 
формирует программист~[2], так и автоматизированный способ, когда правила 
формируются автоматически на основе обучающей выборки, подготовленной 
оператором~[3--5]. Имея набор правил, сис\-те\-ма сбора выполняет 
периодический опрос веб-сай\-тов в поисках новых материалов. 
  
    В данной работе рассматривается система, выполняющая сбор информации 
на основе правил, заданных вручную программистом. Основные 
функциональные элементы системы сбора представлены на рис.~1.
\begin{figure*} %fig1
\vspace*{9pt}
 \begin{center}
 \mbox{%
 \epsfxsize=164.84mm
 \epsfbox{and-1.eps}
 }
 \end{center}
 \vspace*{-6pt}
  \Caption{Функционирование системы сбора}
  \end{figure*}
  
    В рамках системы осуществляется периодический опрос открытых 
  ин\-тер\-нет-ис\-точ\-ни\-ков и получение новых текстовых материалов. 
Система сбора выполняет чтение RSS (Rich Site Summary)  (или HTML) ленты сайта, откуда 
извлекаются метаданные о каждом документе: название, аннотация, время 
публикации и URL текста. Далее по полученному URL (Uniform Research Locator)
осуществляется чтение 
страницы с текстом документа, выполняется построение объектной мо\-де\-ли (DOM~---
Document Object Model) этой 
страницы, откуда и выполняется извлечение чистого текста на основе 
имеющихся XPath-пра\-вил. Результат сбора представляет собой чистый текст 
документа и XML-файл с метаданными. Далее эта информация заносится в базу 
данных, где осуществляется ее накопление и аналитическая обработка. Кроме 
этого, система выполняет постоянную регистрацию и накопление 
статистической информации о состоянии и структуре опрашиваемого 
  веб-сайта.

\subsection{Задача обнаружения сбоев}
  
    Все методы сбора информации с веб-сай\-тов, использующих особенности 
разметки страниц, объединяет то, что при изменении верстки сайта воз\-никает 
необходимость перенастраивать правила распознавания. При выполнении 
круглосуточного опроса целевых сайтов своевременность обнаружения 
изменения верстки является весьма актуальной задачей, поскольку система 
сбора фактически перестает работать до тех пор, пока оператор не 
откорректирует набор правил распознавания.
  
    В простейшем случае при существенном изменении структуры сайта 
система сбора станет выдавать в качестве результата пустые текстовые 
документы. Однако существуют достаточно сложные ситуации, когда при 
изменении верстки система сбора начинает извлекать тексты не полностью 
либо фрагменты из других участков сайта, например комментарии 
пользователей. Именно выявлению таких нетривиальных ситуаций посвящена 
данная статья.

\subsection{Существующие подходы к~решению задачи}
  
    В работах, посвященных теме выявления сбоев систем извлечения 
данных~[6--8], представлено несколько подходов к решению вышеуказанной 
задачи. Большинство из них основано на оценке статистических характеристик 
документов, извлекаемых системой. При этом оценке может подвергаться как 
отдельно взятый документ~\cite{6-and} (в этом случае вычисляется вероятность 
его корректности, которая затем сравнивается с задаваемым пользователем 
пороговым значением), так и их набор~\cite{8-and} (оценке подвергается 
схожесть законов распределения случайных величин, соответствующих 
характеристикам документов из обучающей и тестовой выборок. Для сравнения 
используется критерий согласия Пирсона~\cite{9-and}).

\begin{figure*}[b] %fig2
\vspace*{9pt}
 \begin{center}
 \mbox{%
 \epsfxsize=112.962mm
 \epsfbox{and-2.eps}
 }
 \end{center}
 \vspace*{-6pt}
\Caption{Предложенный подход}
\end{figure*}
  
  В~\cite{8-and} также представлен подход, основанный на использовании 
методов machine learning для обучения системы обнаружения сбоев на наборах 
корректных документов для последующего определения правильности ее 
работы на новых данных. В~качест\-ве таких методов используется, в частности, 
одноклассовая классификация (выявление аномалий)~\cite{10-and}.
  
  Кроме того, анализ и выявление изменений в процессе сбора информации 
осуществляется на основе статистических данных и логов, накопленных 
системой сбора, а потому рассматриваемая задача может быть отнесена к 
направлению Process Mining~\cite{11-and}. Эта дисциплина, находящаяся на 
стыке Data Mining и моделирования процессов, предлагает ряд подходов к 
анализу процессов на основе знаний, извлеченных из логов 
  событий~\cite{12-and}.

\section{Принцип обнаружения сбоев}
  
    Для распознавания сбоев, связанных с изменением верстки, в систему сбора 
встраивается подсистема, осуществляющая контроль корректности 
поступающих документов и выявляющая сбои в их верстке. Возможны два 
следующих подхода к обнаружению сбоев.
  \begin{enumerate}[1.]
\item \textbf{Анализ одной загруженной веб-стра\-ни\-цы.} Суть данного подхода 
заключается в использовании классификатора, который определяет 
принадлежность веб-стра\-ни\-цы к классу корректных или некорректных 
страниц. В~своей работе классификатор использует набор выделяемых из 
веб-стра\-ни\-цы признаков. Обучение классификатора осуществляется на 
предопределенных наборах документов обоих классов. Преимуществом 
такого подхода является высокая скорость реакции детектора на сбой: 
<<плохой>> документ будет выявлен непосредственно после его 
поступления. Однако этот метод имеет и серьезный недостаток. Документы, 
подвергающиеся анализу, могут сильно отличаться друг от друга. Так, 
иногда на вход детектора поступают <<хорошие>>, но нетипичные для 
данного источника веб-стра\-ни\-цы. Если подобных документов не было в 
обучающей выборке классификатора, они не могут быть корректно 
распознаны, и в результате происходит ложное срабатывание. При 
накапливании корректных документов и увеличении обучающей выборки 
частота возникновения таких ошибок постепенно уменьшается, но они 
продолжают периодически возникать. 
\item \textbf{Анализ контрольной серии из нескольких последних загруженных 
веб-стра\-ниц.} Данный подход позволяет избавиться от ложных 
срабатываний. Даже если в контрольную серию попало несколько 
подозрительных документов, то усредненные характеристики этой 
коллекции останутся близкими к характеристикам эталонной обучающей 
выборки. Если же сомнительные документы будут поступать от источника 
регулярно, то через некоторое время, когда в контрольной серии их будет 
накоплено достаточное количество, они будут составлять значительную 
долю анализируемого набора. В~результате характеристики контрольной 
серии изменятся и можно будет обнаружить сбой. Такой подход к фиксации 
сбоев более надежен. Причем качество проверки будет возрастать с 
увеличением числа документов в контрольной серии. Но это приведет к 
возникновению значительной задержки между моментом, в который 
произошел сбой, и временем его обнаружения. 
\end{enumerate}

    Предложенный в данной работе метод сочетает преимущества двух 
вышеописанных подходов (рис.~2): быструю реакцию на сбой и высокое 
качество проверки. 



  Данный метод положен в основу подсистемы контроля корректности 
загружаемых документов. Подсистема представляет собой двухступенчатый 
детектор сбоев. Один из его компонентов~--- <<оперативный детектор>>~--- 
проверяет документы непосредственно в момент их поступления и делает 
предварительный вывод о вероятности сбоя. Если вероятность высока, 
выполняется проверка <<отложенным детектором>>, уточняющая этот 
результат.

\section{Предложенные модели документов}
  
    В основе системы обнаружения сбоев лежит модель анализируемых данных. 
Два основных компонента системы работают с разными входными данными и 
анализируют различные характеристики, поэтому для каждого из них 
предложена своя модель: модель документа, подвергающаяся обработке 
<<оперативным детектором>>, и модель набора документов, анализируемая 
<<отложенным детектором>>.

\subsection{Модель документа}
  
    Под моделью документа понимается совокупность его характеристик, 
учитываемых <<оперативным детектором>> при его обработке.   

При создании 
детектора для системы сбора выбор па\-ра\-мет\-ров производился с учетом 
некоторых особенностей функционирования системы. Текст на целевых 
  веб-стра\-ни\-цах обычно разбит на параграфы (HTML-эле\-мент $\langle 
p\rangle$). Также внутри текстовых параграфов могут встречаться стилевые 
элементы разметки. С~учетом этих факторов для \mbox{оценки} корректности 
документов были выбраны следующие характеристики:
  \begin{itemize}
\item объем веб-страницы, содержащей статью ($P$);
\item суммарный размер параграфов документа ($S$). Учитывается только 
текст, без HTML-эле\-мен\-тов;
\item число параграфов в статье ($N$);
\item дисперсия размера параграфа в рамках документа ($V$);
\item количество HTML-эле\-мен\-тов различных типов, включенных в текст 
документа. Для сокращения типов HTML-элементов они были 
сгруппированы по нескольким категориям. Были выделены классы наиболее 
часто встречающихся элементов: <<Гиперссылки ($H$)>> (в этот класс 
попал элемент href), <<Текстовые блоки ($B$)>> (br, div, span), 
<<Форматирование текста ($S$)>> ($i$, $b$, $u$, em, strong), 
<<Изображения ($I$)>> (img). Остальные теги попали в класс <<Прочее 
($O$)>>. Для каждой категории был введен параметр (соответственно 
$T_H$, $T_B$, $T_S$, $T_I$ и $T_O$), значение которого равно числу 
элементов соответствующего класса, включенных в текст документа.
\end{itemize}

    Таким образом, каждый документ характеризуется рядом параметров (в 
данном случае~--- де\-вятью), поэтому с точки зрения детектора документ 
представлен девятимерным случайным вектором, элементами которого 
являются значения пе\-ре\-чис\-лен\-ных характеристик:
  \begin{equation}
  X=(P,S,N,V,T_H,T_B,T_S,T_I,T_O)\,.
  \label{e1-and}
  \end{equation}

\subsection{Модель набора документов}
  
  Для описания модели набора из нескольких документов заметим следующее. 
Группы характеристик $(P, S, N, V)$ и $(T_H, T_B, T_S, T_I, T_O)$ имеют 
разную природу. Характеристики первой группы описывают свойства текста 
документа, тогда как характеристики второй группы отражают свойства его 
разметки. Для описания свойств набора из нескольких документов будем 
рассматривать эти группы характеристик отдельно.
  
  Случайные величины группы $(P, S, N, V)$ имеют разнородные области 
значений. Так, величина $N$ обычно принимает значения в диапазоне от~1 
до~100, величина~$V$ непрерывна, а значения дискретной величины~$P$ 
могут достигать 10$^5$. В~связи с этим для последующего анализа удобно все 
величины при\-вес\-ти к дискретному виду, а области их значений отоб\-ра\-зить на 
множество фиксированной мощности. Для этого необходимо разбить область 
значений каж\-дой величины группы $(P, S, N, V)$ на фиксированное количество 
интервалов равной длины. Число таких интервалов~$m$ выбирается в 
зависимости от объема выборки. Одним из наиболее распространенных 
способов определения оптимального числа интервалов является формула 
Стерджесса $m\hm=1\hm+\log_2n$, где $n$~--- количество документов в 
наборе~\cite{13-and}.
  
  Для снижения вычислительной сложности алгоритмов, использующих 
предлагаемую модель, в контексте набора из нескольких документов будем 
рассматривать величины $(P, S, N, V)$ независимо друг от друга. Поэтому с 
точки зрения величин $(P, S, N, V)$ модель для набора документов будет 
представлять собой следующие четыре статистических ряда:
  \begin{equation}
  \left.
  \begin{array}{rl}
  P^n & =(P_1, \ldots , P_m)\,;\quad S^n=(S_1, \ldots , S_m)\,;\\[9pt]
  N^n & =(N_1, \ldots , N_m)\,; \quad V^n=(V_1, \ldots , V_m)\,,
  \end{array}
  \right\}
  \label{e2-and}
  \end{equation}
  где $P_i$, $S_i$, $N_i$ и $V_i$~--- частота попадания в $i$-й интервал 
значения величин~$P$, $S$, $N$ и~$V$ соответственно на выборке из $n$ 
документов.
  
    Для учета в модели~(\ref{e2-and}) величин ($T_H$, $T_B$, $T_S$, $T_I$, $T_O$) 
рассмотрим другой подход к представлению информации о HTML-эле\-мен\-тах. 
В~$i$-м документе выборки встречается определенное количество тегов 
каждой из выделенных ранее пяти категорий $H$, $B$, $S$, $I$ и~$O$. 
Обозначим эти количества $T_H^i$, $T_B^i$, $T_S^i$, $T_I^i$, $T_O^i$ соответственно. Просуммируем их по всем 
документам выборки и получим следующие значения: 
  \begin{gather*}
  T_H=\sum\limits_{i=1}^n T_H^i\,;\enskip  T_B=\sum\limits_{i=1}^n T_B^i\,;\enskip 
T_S=\sum\limits_{i=1}^n T_S^i\,;\\
T_I=\sum\limits_{i=1}^n T_I^i\,; \enskip
T_O=\sum\limits_{i=1}^n T_O^i\,,
\end{gather*}
которые образуют пятиэлементный 
статистический ряд $T^n\hm=(T_H, T_B, T_S, T_I, T_O)$. Этот ряд будем 
рассматривать в качестве модели набора документов с точки зрения частоты 
встречаемости в нем тегов из пяти выделенных категорий.
  
    Таким образом, модель набора документов представляет собой совокупность 
следующих пяти статистических рядов:
  \begin{equation}
  \left.
  \begin{array}{c}
  P^n=(P_1, \ldots , P_m)\,;\enskip S^n=(S_1, \ldots , S_m)\,;\\[9pt]
  N^n=(N_1, \ldots , N_m)\,;\enskip V^n=(V_1, \ldots ,V_m)\,;\\[9pt]
  T^{n} =\left(T_{H}, T_{B}, T_{S}, 
T_{I}, T_{O}\right)\,.
  \end{array}
  \right\}
  \label{e3-and}
  \end{equation}

\section{Оперативный детектор}

\subsection{Принцип работы оперативного детектора}
  
    Быстродействующий компонент детек\-ти\-ру\-ющей системы представляет 
собой бинарный классификатор, который на основании значений параметров 
документа делает вывод о его корректности или некорректности. При выборе 
метода классификации нужно учитывать, что при обучении оперативного 
детектора в большинстве случаев количество <<хороших>> документов 
намного больше числа <<плохих>>. В~некоторых случаях в обучающей 
выборке может вообще не содержаться некорректных документов. Поэтому 
было решено проводить обучение классификатора на позитивных примерах, но 
при этом его работа была организована следующим образом: в режиме 
проверки документов детектор должен считать корректными лишь статьи, 
похожие на элементы обучающей выборки. Определим эту схожесть в 
терминах выбранной модели документа.

\begin{center}  %fig3
\vspace*{-3pt}
\mbox{%
 \epsfxsize=78.877mm
 \epsfbox{and-3.eps}
 }
 \vspace*{6pt}
{{\figurename~3}\ \ \small{Распределение значений параметров~$N$ и~$P$}}
 \end{center}

%\pagebreak

\vspace*{6pt}

\addtocounter{figure}{1}

 
    Каждый документ представлен девятимерным вектором. Для примера 
рассмотрим двумерную проекцию множества таких векторов, 
соответствующего набору новостей с сайта {\sf kp.ru}, на плоскость, задаваемую 
параметрами~$N$ (количество параграфов в статье) и~$P$ (объем 
  веб-стра\-ни\-цы, содержащей статью) (рис.~3).
  

  
    Точки не распределены в пространстве равномерно, они сгруппированы в 
некоторых областях. Новый документ, поступающий на проверку, можно 
считать корректным, если соответствующая ему точка попадает в одну из таких 
областей. Если же точка находится в отдалении от этих зон (как, например, три 
точки в правой верхней части рисунка, помеченные крестиком), то 
соответствующая статья является подозрительной.
  
    Таким образом, обучение оперативного детектора сводится к выделению 
таких областей, а классификация статей на корректные и некорректные~--- к 
определению, попадает ли документ в одну из выделенных областей.
  
     Рисунок~3 демонстрирует применение предложенного подхода для 
определения корректности объектов с двумерными векторами характеристик, 
но аналогичным образом может осуществляться классификация и в случае 
большей размерности векторов. Однако с ростом размерности для 
формирования плотных областей требуется существенно увеличивать 
обучающую выборку. Учитывая предполагаемые объемы наборов документов 
(десятки тысяч), при использовании девяти характеристик добиться высокой 
плотности при сохранении небольшого количества выделяемых зон 
невозможно.
  
    Таким образом, описанная в~(\ref{e1-and}) модель документа в виде 
  9-мер\-но\-го вектора оказывается неудобной для непосредственного 
использования оперативным детектором, поэтому в нее были внесены 
изменения. Заменим девятимерный вектор~$X$ на набор векторов меньшей 
размерности ($Y_1, Y_2, \ldots , Y_k$), каждый из которых содержит некоторое 
подмножество элементов~$X$. Будем выбирать этот набор векторов исходя из 
следующих соображений:
  \begin{enumerate}[(1)]
\item нужно по возможности использовать векторы наименьшей 
размерности (двумерные) для получения максимальной плотности 
кластеров;
\item нужно избегать использования векторов, которые могут оказаться 
бесполезными для некоторых источников.
\end{enumerate}

    Второй пункт относится прежде всего к характеристикам, отражающим 
количество HTML-эле\-мен\-тов, включенных в текст документа. Сайты 
обычно применяют для оформления текста лишь небольшой набор тегов, при 
этом некоторые группы HTML-эле\-мен\-тов могут не использоваться вовсе. 
Поэтому только некоторые из параметров ($T_H$, $T_B$, $T_S$, $T_I$, 
$T_O$) будут принимать ненулевые значения. Каждый сайт использует 
собственный подход к оформлению и выбору набора тегов, что не позволяет 
определить универсальный критерий по\-лез\-ности каждой из этих характеристик 
и их совокупностей. Поэтому было решено все перечисленные величины 
включить в пятимерный вектор~$Y_1$.
  
    Каждый из оставшихся четырех параметров является важной 
характеристикой структуры документа, поэтому в качестве элементов 
остальных векторов использовались все попарные сочетания величин~$P$, $S$, 
$N$ и~$V$. Так были получены 6~двумерных векторов $Y_2, \ldots , Y_7$.
  
    Таким образом, модель документа, под\-вер\-га\-юща\-я\-ся обработке 
<<оперативным детектором>>, представляет собой совокупность из следующих 
семи случайных векторов:
  \begin{equation}
  \left.
  \begin{array}{c}
  Y_1 = (T_H, T_B, T_S, T_I, T_O)\,; \enskip Y_2=(P,S)\,;\\[9pt]
  Y_3=(P,N)\,;\enskip Y_4=(P,V)\,;\enskip Y_5=(S,N)\,;\\[9pt]
  Y_6=(S,V)\,;\enskip Y_7=(N,V)\,.
  \end{array}
  \right\}
  \label{e4-and}
  \end{equation}

\subsection{Кластеризация документов}
  
    Выделение областей необходимо производить таким образом, чтобы 
максимально облегчить последующую проверку принадлежности точек этим 
областям. Поэтому нет смысла выбирать зоны сложной формы~--- более 
эффективным решением является нахождение плотных групп точек и 
построение простых ограничивающих поверхностей для этих групп. Для 
разбиения всего множества документов из обучающей выборки на группы 
нужно решить задачу кластеризации. Существует множество подходов к 
кластерному анализу, и применение\linebreak различных алгоритмов к одним и тем же 
входным данным может дать совершенно разные результаты~[14, 15]. 
Основным требованием, опре\-де\-ля\-ющим пригодность метода для кластеризации 
документов, является простая, гиперсферическая\linebreak
 форма кластеров, 
позволяющая получить с по\-мощью простых ограничивающих поверхностей 
плотные области без разреженных участков. Среди популярных методов 
кластеризации (k-means~[16, 17], иерархические методы~[18]) наилучшим 
образом отвечает требованиям к виду формируемых клас\-те\-ров иерархический 
метод средней связи~[19]. Однако он имеет серьезный недостаток, характерный 
для всех иерархических методов~--- высокую вычислительную сложность 
(O(n2)). Тем не менее в данной работе за основу был взят этот метод, в который 
были внесены следующие модификации.
  
    Ограничим число элементов, подвергающихся кластеризации методом 
средней связи, числом~$n$. Тогда кластеризация $N$ элементов ($N\hm>n$) 
будет осуществляться следующим образом.
  \begin{enumerate}[1.]
\item Выбрать из множества документов $n$~элементов.
\item Произвести кластеризацию этих элементов методом средней связи.
\item Найти центроиды кластеров.
\item Поместить центроиды в множество точек в качестве новых элементов.
\item Повторять п.~1--4 пока в множестве не останется необходимое 
число элементов.
\item Определить принадлежность исходных элементов найденным 
кластерам.
\end{enumerate}

    Результат кластеризации, произведенной описанным способом при 
$n\hm = 20$, приведен на рис.~4.



  В качестве ограничивающих поверхностей для областей рассматривались 
гиперпараллелепипед, гиперсфера и гиперэллипсоид.  Выбор
 был сделан\linebreak\vspace*{-12pt}
\begin{center}  %fig4
\vspace*{12pt}
\mbox{%
 \epsfxsize=78.877mm
 \epsfbox{and-4.eps}
 }
 \vspace*{6pt}
{{\figurename~4}\ \ \small{Ограничивающие поверхности кластеров}}

 \end{center}


%\pagebreak

%\vspace*{15pt}

\addtocounter{figure}{1}


\noindent
 в пользу наиболее простых в построении гиперпараллелепипедов, показавших 
хорошие результаты при оценке плотности точек. Таким образом,\linebreak каж\-дый 
кластер задается набором пар ($z_{\min}, z_{\max}$), определяющих 
граничные значения со\-от\-вет\-ст\-ву\-ющего гиперпараллелепипеда по 
па\-ра\-мет\-ру~$Z$. \mbox{Элемент} принадлежит кластеру, если для каж\-до\-го 
па\-ра\-мет\-ра~$Z$ выполняется $z_{\min}\hm\leq z\hm\leq z_{\max}$, где $z$~--- 
значение параметра~$Z$ для рассматриваемого элемента. При классификации 
документ считается подозрительным, если он не попадает ни в один из 
кластеров.
  
    Кластеризация и построение ограничивающих поверхностей и последующая 
классификация загружаемых документов производятся отдельно для каждого 
из семи выделенных векторов~(\ref{e4-and}). Таким образом, результатом 
классификации является набор из семи двоичных значений. Решение о 
корректности документа принимается на основании этого набора: статья 
считается корректной, если она успешно прошла проверку по каждому из семи 
критериев. 
  
\section{Отложенный детектор}
  
    Второй компонент системы обнаружения сбоев осуществляет оценку набора 
документов. Оценка осуществляется на основе статистических 
  рядов~(\ref{e3-and}), которые можно рассматривать как приближения к 
функциям вероятности соответствующих случайных величин. Идея, лежащая в 
основе функционирования отложенного детектора, заключается в следующем: 
рассматриваемые случайные величины, составляющие вектор~(\ref{e1-and}), 
подчиняются некоторым законам распределения, которые при отсутствии сбоя 
остаются неизменными. Изменение же верстки с высокой вероятностью 
повлияет на эти законы распределения. Следовательно, две разных выборки, 
состоящие из корректных документов, будут обладать высокой степенью 
сходства. Если же одна из них будет содержать <<плохие>> статьи, то различие 
между выборками будет значительно сильнее. Таким образом, задача детектора 
заключается в определении степени сходства проверяемой выборки и выборки, 
состоящей из гарантированно корректных статей, сформированной в процессе 
обучения (назовем ее эталонной). На основе полученного результата 
принимается решение о наличии/отсутствии сбоя. 
  
    Для примера рассмотрим три выборки случайной величины~$S$ 
(суммарный размер параграфов документа), соответствующие наборам 
новостей с сайта {\sf lenta.ru}: эталонную~(\textit{а}); тестовую выборку, состоящую из 
<<хороших>> документов~(\textit{б}) и тесто-\linebreak\vspace*{-12pt}
\begin{center}  %fig5
\vspace*{1pt}
 \mbox{%
 \epsfxsize=71.266mm
 \epsfbox{and-5.eps}
 }
 \vspace*{4pt}
{{\figurename~5}\ \ \small{Гистограммы выборок}}
 \end{center}


%\pagebreak

\vspace*{6pt}

\addtocounter{figure}{1}

\noindent
вую выборку, содержащую некорректные 
статьи~(\textit{в}). В~качестве последних использовались новости с сайта {\sf cnews.ru}. 
  
  На рис.~5 показаны гистограммы, соответствующие этим выборкам. Первые 
две из них обладают высокой степенью сходства, в то время как третья 
значительно от них отличается.

    \begin{figure*}[b]
  \begin{minipage}[t]{82mm}
    \vspace*{1pt}
  \begin{center}  %fig6
\mbox{%
 \epsfxsize=78.994mm
 \epsfbox{and-6.eps}
 }
 \end{center}
 \vspace*{-9pt}
\Caption{Зависимость максимального значения KLIC от мощности набора}
%\end{figure}
\end{minipage}
\hfill
\begin{minipage}[t]{82mm}
\vspace*{1pt}
\begin{center}
 \mbox{%
 \epsfxsize=78.994mm
 \epsfbox{and-7.eps}
 }
 \end{center}
 \vspace*{-9pt}
 \Caption{График пороговой функции}
 \end{minipage}
\end{figure*}

  Для оценивания сходства выборок используется относительная энтропия 
(расстояние Куль\-ба\-ка--Лейб\-ле\-ра, KLIC~\cite{20-and}). Для дискретных 
случайных величин с функциями вероятности~$p$ и~$q$, принимающих 
значения в одном множестве $\mathcal{M}\subset \mathbb{R}$, это расстояние 
задается формулой 
  $$
  D_{KL}(p,q) =\sum\limits_{x\subset \mathcal{M}} p(x) \ln \fr{p(x)}{q(x)}\,.
  $$
  

  
    Вместо функций вероятности используются час\-то\-ты рядов~(\ref{e3-and}). 
При этом $p(x)$ соответствует эталонной выборке, а $q(x)$~--- проверяемой. 
  
    Результатом расчета KLIC для рядов~(\ref{e3-and}) являются значения 
$D_P$, $D_S$, $D_N$, $D_V$ и $D_N$ соответственно.
  
    После расчета расстояния Куль\-ба\-ка--Лейб\-ле\-ра встает вопрос: как по 
найденному значению определить, произошел сбой или нет? Необходимо 
задать некоторое пороговое значение~$K$, такое что наличие сбоя можно 
определить как
  $$
  f(D_{\mathrm{KL}} ) = 
  \begin{cases}
  0\,, &\ D_{\mathrm{KL}}\leq K (\mbox{сбоя нет});\\
  1\,, &\ D_{\mathrm{KL}}>K (\mbox{произошел сбой}).
  \end{cases}
  $$
  
  Данный порог не является фиксированной величиной, его значение зависит 
от числа документов в тестовой выборке. Поясним это утверждение на 
примере. Выберем множество $\mathbf{A}=\{A_i\}$ наборов документов $A_i$ 
различной мощности и вычислим для каждого из них расстояние 
  Куль\-ба\-ка--Лейб\-ле\-ра ~$d_i$ 
  от эталонного закона распределения. Сопоставим натуральным числам~$j$, 
соответствующим мощностям наборов из множества $\mathbf{A}\hm=\{A_i\}$, 
числа~$K_j$, определяемые как
  $$
  K_j=\max\limits_{A_i\in \mathbf{A}}\{ D_i:\vert A_i\vert=j\}\,.
  $$
  
  Рассмотрим зависимость максимального расстояния 
  Куль\-ба\-ка--Лейб\-ле\-ра от мощности набора. На рис.~6 приведена такая 
зависимость для новостей с {\sf kp.ru}. При этом использовалась оценка 
характеристики~$P$, отражающей объем веб-стра\-ни\-цы, но аналогичная 
зависимость имеет место и для других характеристик. 
  


    Такой вид зависимости легко объясним: чем больше выборка, тем меньше на 
нее влияют локальные колебания значений параметров. Таким образом, при 
выборе порогового значения необходимо учитывать мощность анализируемого 
набора. Для этого необходимо определить пороговую функцию $K\hm = h(x)$, 
устанавливающую соответствие между количеством документов в наборе и 
пороговым значением для этого набора. 
  
    Анализ рис.~6 ведет к предположению об обратно пропорциональной 
зависимости значения~$K_j$ от~$j$ и целесообразности использования 
аппроксимирующей функции вида $h(x)\hm=a/x^b$. Однако проведение 
подобного исследования для других источников и параметров показывает, что 
такая функция не всегда дает приемлемый результат: в некоторых случаях 
зависимость имеет более сложный характер. Чтобы сделать метод определения 
пороговой функции пригодным для различных случаев и при этом учесть 
общую закономерность (постепенное уменьшение значения функции при 
воз\-рас\-та\-нии
 аргумента), было решено использовать для аппроксимации 
функцию $h(x)\hm= \sum\limits_{i=0}^k a_i/x^i$. Коэффициенты~$a_i$ 
определяются в процессе обучения (с по\-мощью метода наименьших 
квадратов (МНК)~\cite{21-and}), а значение $k\hm = 7$ было выбрано на основе 
исследования зависимостей, характерных для различных источников. Таким 
образом, пороговая функция принимает вид:
  $$
  h(x)=\sum\limits_{i=0}^7 \fr{a_i}{x^i}\,.
  $$
  




  
  Для минимизации числа ложных срабатываний было решено подвергнуть 
функцию преобразованию, которое бы обеспечило выполнение условия 
$h_j\hm\geq K_j$ для всех узлов аппроксимации. Для этого коэффициент~$a_0$ 
необходимо увеличить на величину $\Delta\hm= \max\limits_j(K_j-h_j)$. На 
рис.~7 приведены графики пороговой функции до (пунктирная линия) и после 
(сплошная линия) коррекции.
  
  \begin{figure*}[b] %fig8
%  \vspace*{9pt}
 \begin{center}
 \mbox{%
 \epsfxsize=113.506mm
 \epsfbox{and-8.eps}
 }
 \end{center}
 \vspace*{-6pt}
  \Caption{Этапы обнаружения сбоев}
  \end{figure*}
  
    С помощью приведенной пороговой функции на основании показателей 
$D_P$, $D_S$, $D_N$, $D_V$ и $D_T$ получим набор из пяти двоичных 
значений: ($F_P$, $F_S$, $F_N$, $F_V$, $F_T$). В~зависимости от количества 
единиц в этом наборе и от того, какие именно критерии приняли единичное 
значение, делается заключение о вероятности сбоя. В~разработанной системе 
используется следующий подход:
  \begin{itemize}
\item количество единиц в наборе равно~0 или~1~--- низкая вероятность (сбоя 
нет);
\item 2 или 3~--- средняя вероятность (нельзя с уверенностью судить о наличии 
или отсутствии сбоя);
\item 4 или~5~--- высокая вероятность (произошел сбой).
\end{itemize}

\section{Взаимодействие детекторов}
  
    Отдельной задачей является организация взаимодействия двух детекторов с 
целью достижения максимально эффективного функционирования системы 
отслеживания сбоев. Поскольку отложен\-ный детектор осуществляет более 
качественный анализ и менее склонен к ложным срабатываниям, он 
используется для контроля работы оперативного классификатора. Этот 
контроль подразумевает две основные функции:
  \begin{enumerate}[(1)]
\item проверку правильности результатов, полученных классификатором 
оперативного детектора; 
\item обучение классификатора. Если оперативный детектор обнаружил 
подозрительный документ, а отложенный детектор в результате проверки 
установил отсутствие сбоя, значит, произошло ложное срабатывание. Это 
свидетельствует о недостаточной обученности оперативного детектора. 
Поэтому необходимо произвести его переобучение с использованием 
документов, определенных им в категорию подозрительных.
\end{enumerate}

    В некоторых случаях ложные срабатывания могут быть обнаружены без 
участия отложенного детектора. Для этого оперативный классификатор был 
оснащен функцией самопроверки. Он способен самостоятельно отличить 
единичный выброс от массового поступления некорректных статей путем 
анализа частоты появления таких статей среди последних скачанных 
документов. Если эта частота меньше заданного порогового значения 
(например, 50\%), делается вывод о ложном срабатывании и запускается 
переобучение. В~качестве анализируемого набора при самопроверке 
используется группа документов, полученных в рамках последней транзакции, 
т.\,е.\ при последней загрузке документов с сайта.
  
    Рассмотрим итоговый метод обнаружения изменений структуры 
  веб-сай\-тов, реализованный в работе подсистемы обнаружения сбоев с 
учетом выбранного подхода к реализации взаимодействия детекторов. Этапы 
функционирования подсистемы приведены на рис.~8. 
  

  
  На этапе классификации оперативный детектор проверяет поступающие 
статьи. Документы классифицируются на корректные и подозрительные. 
Необходимые для классификации данные о кластерах и ограничивающих 
поверхностях извлекаются из базы данных.
  
    После поступления от источника группы документов оперативный детектор 
выполняет самопроверку: вычисляется частота детектирования подозрительных 
статей в пределах текущей транзакции. Если она ниже порогового значения, но 
не равна нулю, делается заключение о ложном срабатывании и выполняется 
переход к блоку переобучения. Если частота выше порогового значения~--- к 
блоку отложенной проверки.
  
        Работа блока отложенной проверки начинается с оповещения отложенного 
детектора о необходимости выполнения анализа. Выполнение проверки 
непосредственно после получения оповещения не имеет смысла, поскольку 
сбой может быть зафиксирован только после накапливания достаточного числа 
некорректных статей. После поступления необходимого числа документов 
отложенный детектор выполняет проверку этого набора. Для ее проведения из 
базы данных извлекаются статистические ряды эталонных выборок и 
коэффициенты~$a_i$ пороговой функции. Результат проверки передается блоку 
принятия решения. 
  
    Блок принятия решения определяет дальнейшие действия подсистемы в 
зависимости от результата отложенной проверки. Если она показала высокую 
вероятность сбоя, администратор системы оповещается о необходимости 
корректировки системы сбора документов. Если вероятность сбоя низка, 
делается заключение о ложном срабатывании оперативного классификатора и 
выполняется переход к блоку переобучения. Если же результат анализа не 
позволяет с высокой долей уверенности судить о наличии или отсутствии сбоя, 
выполняется повторная отложенная проверка.
  
    На этапе переобучения для оперативного детектора заново определяются 
кластеры и строятся ограничивающие поверхности с использованием нового, 
дополненного набора данных. Количество кластеров и граничные значения 
гиперпараллелепипедов заносятся в базу данных.

\vspace*{-9pt}
  
\section{Экспериментальная проверка системы}

\vspace*{-2pt}
  
    В рамках данной работы были проведены эксперименты, направленные на 
анализ качества работы разработанной системы обнаружения сбоев. 
Эксперименты проводились на ПЭВМ со следующими основными 
параметрами: процессор Intel Core~2 Duo 1,8~ГГц, объем ОЗУ 2~ГБ.
  
    Для проведения экспериментов использовалась коллекция новостей, 
извлеченных со следующих сайтов: {\sf mail.ru}, {\sf itar-tass.com}, {\sf kp.ru}, {\sf 
rbc.ru}, \mbox{\sf kommersant}.{\sf ru}, {\sf ria.ru}, {\sf rambler.ru} 
(табл.~1). Для обучения 
использовалось в общей сложности 72\,888 корректных документов. При 
обучении оперативного детектора формировалось 10~кластеров.
  
    При самопроверке оперативного детектора было использовано пороговое 
значение, равное 10\%. Накопленные за время тестирования документы 
использовались в качестве тестовой выборки для отложенного детектора.
  
    Целью первого эксперимента была оценка работы системы на корректных 
данных. В~качестве входных данных использовались гарантированно 
корректные статьи, полученные с использованием
правильных настроек 
системы сбора. Для проведения эксперимента использовалось в общей 
сложности 5169~документов. 

  В рамках эксперимента проверке были подвергнуты 5169 корректных статей. 
При первичной классификации 65 из них (1,26\%) были определены как\linebreak\vspace*{-12pt}

\vspace*{6pt}
% \begin{table*}

  \noindent{{\normalsize\tablename~1}\ \ \small{Ложные срабатывания оперативного детектора}}

{\small  \begin{center} 
  \tabcolsep=5pt
  \begin{tabular}{|l|c|c|c|c|c|}
  \hline
\multicolumn{1}{|c|}{Источник}&$M_L$&$M_T$&$M_S$&$N_D$&$N_S$\\
\hline
{\sf mail.ru}&25\,296&2631\hphantom{9}&20&14&0\\
{\sf itar-tass.com}&11\,548&560&76&\hphantom{9}0&0\\
{\sf kp.ru}&\hphantom{9}7\,220&218&24&\hphantom{9}4&1\\
{\sf rbc.ru}&\hphantom{9}3\,517&227&25&14&5\\
{\sf kommersant.ru}&\hphantom{9}5\,288&260&47&\hphantom{9}4&0\\
{\sf ria.ru}&16\,519&1115&29&12&5\\
{\sf rambler.ru}&\hphantom{9}3\,500&\hphantom{9}158&15&17&13\hphantom{9}\\
\hline
Всего&72\,888&5169&34&65&24\hphantom{9}\\
\hline
\multicolumn{6}{p{220pt}}{\footnotesize \textbf{Примечания:} $M_L$~--- размер обучающей 
выборки; $M_T$~--- размер тестовой выборки; $M_S$~--- средний размер 
анализируемого набора документов при самопроверке; $N_D$~--- количество 
подозрительных статей; $N_S$~--- количество подозрительных статей после 
самопроверки.}
\end{tabular}
\end{center}
}
%\end{table*} 

\addtocounter{table}{1}



\setcounter{table}{1}
\begin{table*}\small %tabl2
\begin{center}
\Caption{Ложные срабатывания отложенного детектора}
\vspace*{2ex}

\begin{tabular}{|l|c|c|c|c|c|c|c|c|c|}
\hline
\multicolumn{1}{|c|}{Источник}&$M_L$&$M_T$&$F_P$&$F_S$&$F_N$&$F_V$&$F_T$&$
N_F$&$P_F$\\
\hline
{\sf mail.ru}&25\,296&2631&0&0&0&0&0&0 из 5&$L$\\
{\sf itar-tass.com}&11\,548&\hphantom{9}560&0&0&0&0&0&0 из 5&$L$\\
{\sf kp.ru}&\hphantom{9}7\,220&\hphantom{9}218&1&0&0&0&0&1 из 5&$L$\\
{\sf rbc.ru}&\hphantom{9}3\,517&\hphantom{9}227&0&0&0&0&0&0 из 5&$L$\\
{\sf kommersant.ru}&\hphantom{9}5\,288&\hphantom{9}260&0&0&0&0&0&0 из 5&$L$\\
{\sf ria.ru}&16\,519&1115&0&0&0&0&0&0 из 5&$L$\\
{\sf rambler.ru}&\hphantom{9}3\,500&\hphantom{9}158&0&0&0&0&0&0 из 5&$L$\\
\hline
Всего&72\,888&5169&1&0&0&0&0&\hphantom{9}1 из 35&\\
\hline
\multicolumn{10}{p{320pt}}{\footnotesize \textbf{Примечания:} $N_F$~--- количество критериев, 
показавших наличие сбоя; $P_F$~--- заключение детектора: вероятность сбоя 
($L$~--- низкая, $M$~--- средняя, $H$~--- высокая).}
\end{tabular}
\end{center}
\end{table*}

\setcounter{table}{3}  
\begin{table*}[b]\small %tabl4
\vspace*{-12pt}
\begin{center}
\Caption{Оценка пропуска сбоев отложенным детектором}
\vspace*{2ex}

\begin{tabular}{|l|c|c|c|c|c|c|c|c|c|}
\hline
\multicolumn{1}{|c|}{Источник}&$M_L$&$M_T$&$F_P$&$F_S$&$F_N$&$F_V$&$F_T$&$
N_F$&$P_F$\\
\hline
{\sf mail.ru}&25\,296&356&1&1&1&0&0&3 из 5&$M$\\
{\sf itar-tass.com}&11\,548&356&1&1&1&0&0&3 из 5&$M$\\
{\sf kp.ru}&\hphantom{9}7\,220&356&1&0&1&0&1&3 из 5&$M$\\
{\sf rbc.ru}&\hphantom{9}3\,517&356&1&1&1&0&1&4 из 5&$H$\\
{\sf kommersant.ru}&\hphantom{9}5\,288&356&1&1&1&1&1&5 из 5&$H$\\
{\sf ria.ru}&16\,519&356&1&0&1&1&1&4 из 5&$H$\\
rambler.ru&\hphantom{9}3\,500&356&1&1&1&1&1&5 из 5&$H$\\
\hline
Всего&72\,888&2492\hphantom{9}&7&5&7&3&5&27 из 35&\\
\hline
\end{tabular}
\end{center}
\vspace*{-12pt}
\end{table*}



%\begin{table*}
%tabl3
{\small
  \noindent{{\normalsize\tablename~3}\ \ \small{Оценка пропуска сбоев оперативным детектором}}
  
  \begin{center}

\tabcolsep=5pt
\begin{tabular}{|l|c|c|c|c|c|}
\hline
\multicolumn{1}{|c|}{Источник}&$M_L$&$M_T$&$M_S$&$N_D$&$N_S$\\
\hline
{\sf mail.ru}&25\,296&356&25&356&356\\
{\sf itar-tass.com}&\hphantom{9}3\,500&356&25&356&356\\
{\sf kp.ru}&11\,548&356&25&356&356\\
{\sf rbc.ru}&\hphantom{9}7\,220&356&25&356&356\\
{\sf kommersant.ru}&16\,519&356&25&356&356\\
{\sf ria.ru}&\hphantom{9}3\,517&356&25&356&356\\
{\sf rambler.ru}&\hphantom{9}5\,288&356&25&356&356\\
\hline
Всего&72\,888&2492\hphantom{9}&25&2492\hphantom{9}&2492\hphantom{9}\\
\hline
\end{tabular}
\end{center}
}
%\end{table*} 

\addtocounter{table}{1}
  
\noindent
подозрительные. В~результате самопроверки 41 из них был переведен в 
категорию корректных. Оставшиеся 24 (0,46\% от общего числа) были 
ошибочно признаны некорректными. 
  

  Отложенный детектор показал правильный результат при проверке тестовой 
выборки каждого сайта (табл.~2). Ошибочное значение критерия было 
зафиксировано лишь в 1~случае из~35 (2,86\%). 
  
  В рамках второго эксперимента (см.\ табл.~3) оценивалась способность 
системы обнаруживать сбои. Ввиду отсутствия для многих сайтов достаточного 
числа негативных примеров тестовые наборы были созданы искусственно: в 
качестве <<плохих>> документов использовались комментарии к новостям, 
полученные с сайта {\sf championat.com}. Такой выбор тестовых данных 
обусловлен тем, что возможным последствием изменения верстки является 
извлечение из веб-стра\-ниц не новостей, а текстов с других участков сайта, в 
частности комментариев. Для проведения эксперимента использовалось 
356~документов (для всех источников использовался одинаковый тестовый 
набор).
  
  В рамках эксперимента проверке были подвергнуты 356~некорректных 
статей. При первичной классификации все они были определены как 
подозрительные для каждого из семи источников. В~результате самопроверки 
никаких изменений произведено не было. 
  
  Отложенный детектор показал правильный результат для 4~источников из~7. 
Для оставшихся 3 источников он не смог сделать вывод о наличии или 
отсутствии сбоя. В~8~случаях из~35 (22,85\%) значение критериев было 
неверным. Данным ситуациям соответствуют значения~0 соответствующего 
критерия в табл.~4.
  
  Если в ходе первого эксперимента система обнаружения сбоев 
продемонстрировала свою работоспособность при выполнении как 
оперативной, так и отложенной проверки корректных данных, то с задачей 
обнаружения сбоев она справилась значительно хуже. Возможной причиной 
низкого качества работы системы при анализе некорректных документов 
является неудачный подход к определению результата проверки. Анализ 
результатов экспериментов показывает необходимость понижения порога 
фиксации сбоя. Кроме того, при проведении второго эксперимента 
критерии~$F_P$, $F_S$, $F_N$, $F_V$ и~$F_T$ были приняты равнозначными, 
однако оказалось, что некоторые из них показывают наличие сбоя значительно 
точнее, чем другие. Так, критерии~$F_P$ и~$F_N$ приняли верное значение в 
7~случаях из~7, а $F_V$~--- лишь в~3. Чтобы учесть различную значимость 
критериев, для каждого из них может быть установлен весовой коэффициент, 
определяющий влияние значения соответствующего критерия на результат 
проверки.





\vspace*{-6pt}

\section{Заключение}

\vspace*{-2pt}

    В работе предложен подход к автоматизированному контролю работы 
системы извлечения данных с веб-сай\-тов. В~его основе лежит двухуровневая 
проверка корректности веб-стра\-ниц, обеспе\-чи\-ва\-ющая быстроту реакции и 
высокое качество оценки документов.
  
    В основе первичной классификации лежит проверка схожести документа с 
элементами обуча-\linebreak ющей выборки. Это позволяет системе адекватно реа\-ги\-ровать 
на любые нетипичные для сайта веб-стра\-ни\-цы. Простота выполнения такой 
проверки достигается с помощью предложенного метода клас\-те\-ри\-за\-ции. Он 
относится к иерархическим методам, но имеет меньшую вычислительную 
сложность по сравнению с другими алгоритмами этого класса.
  
    Отложенная проверка корректности основана на сравнении законов 
распределения. Для правильной интерпретации полученного результата 
используется пороговая функция, полученная путем аппроксимации  
МНК. Такой подход обеспечивает высокую точность проверки вне зависимости 
от размера оцениваемой выборки.
  
    Проведенные эксперименты показали эффективность совместного 
использования двух детекторов. Предложенный подход был реализован в виде 
подсистемы отслеживания сбоев в системе сбора новостной информации. 
Данная система успешно внедрена в Совете Федерации Федерального 
Собрания РФ в рамках комплекса <<Обзор СМИ>>, решающего задачу сбора, 
накопления и классификации новостей об\-ще\-ст\-вен\-но-по\-ли\-ти\-че\-ской 
тематики. 

\vspace*{-6pt}

{\small\frenchspacing
{%\baselineskip=10.8pt
\addcontentsline{toc}{section}{Литература}
\begin{thebibliography}{99}

\vspace*{-2pt}
  
\bibitem{1-and}
\Au{Nikovski D., Esenther A., Baba~A.} Semi-supervised information extraction from 
variable-length web-page lists~// ICEIS 2009: 11th Conference (International) on 
Enterprise Information Systems Proceedings.~--- Milan, Italy, 2009. P.~261--266.
\bibitem{2-and}
\Au{Oro E., Ruffolo M., Staab~S.} SXPath~--- Extending XPath towards spatial 
querying on web documents~// VLDB Endowment Proceedings, 2011. Vol.~4. 
No.\,2. P.~129--140.

\bibitem{4-and} %3
\Au{Chidlovskii B., Ragetli J., de Rijke~M.} Wrapper generation by reversible 
grammar induction~// Machine learning~--- ECML 2000:  11th European Conference 
on Machine Learning Proceedings (Barcelona,  2000). 
Lecture notes in computer sci. ser. Vol.~1810.~--- Springer, 2000. P.~96--108.

\bibitem{5-and} %4
\Au{Kushmerick N.} Wrapper induction: Efficiency and expressiveness~// Artificial 
Intelligence, 2000. No.\,118. P.~15--68.

\bibitem{3-and} %5
\Au{Tobias A.} XPath-Wrapper Induction by generalizing tree traversal patterns~// 
Workshopwoche der GI-Fachgruppen/Arbeitskreise.~--- GI-Fachgruppen ABIS, 
AKKD, FGML, 2005. P.~126--133.


\bibitem{6-and}
\Au{Kushmerick N., Weld D.\,S., Doorenbos~R.\,B.} Wrapper induction for 
information extraction~//  IJCAI 97: 15th Joint Conference (International)  on 
Artificial Intelligence Proceedings.~--- Nagoya, Japan, 1997. Vol.~1. P.~729--737.
\bibitem{7-and}
\Au{Kushmerick N.} Wrapper verification~// World Wide Web~J., 2000. Vol.~3. 
No.\,2. P.~79--94.
\bibitem{8-and}
\Au{Lerman K., Minton S, Knoblock~C.} Wrapper maintenance: A~machine learning 
approach~// J.~Artificial Intelligence Research, 2003. Vol.~18. P.~149--181.
\bibitem{9-and}
\Au{Кендалл М., Стьюарт А.} Статистические выводы и связи.~--- М.: Наука, 
1973.
\bibitem{10-and}
\Au{Kriegel H.-P., Kr$\ddot{\mbox{o}}$ger~P., Zimek~A.} Outlier detection 
techniques~// PAKDD 2009: 13th Pacific-Asia Conference on Knowledge Discovery 
and Data Mining Proceedings.~--- Bangkok, Thailand, 2009.
\bibitem{11-and}
Process Mining. {\sf http://www.processmining.org}.
\bibitem{12-and}
\Au{Van der Aalst W.\,M.\,P.} Process mining: Discovery, conformance and 
enhancement of business processes.~--- Springer-Verlag, 2011.
\bibitem{13-and}
\Au{Sturges H.} The choice of a class-interval~// J.~Amer. Statistical Association, 
1926. Vol.~21. No.\,153. P.~65--66.

\bibitem{15-and} %14
\Au{Дюран Б., Оделл П.} Кластерный анализ.~--- М.: Статистика, 1977. 128~с. 

\bibitem{14-and} %15
\Au{Мандель И.\,Д.} Кластерный анализ.~--- М.: Финансы и статистика, 1988. 
176~с.

\bibitem{16-and}
\Au{Jain A., Dubs R.} Clustering methods and algorithms.~--- Prentice-Hall, 1988.
\bibitem{17-and}
\Au{Андреев А.\,М., Березкин Д.\,В., Морозов~В.\,В., Симаков~К.\,В.} Метод 
кластеризации документов текстовых коллекций и синтеза аннотаций кластеров~// 
Электронные библиотеки: перспективные методы и технологии, электронные 
коллекции (RCDL'2008): Труды 10-й Всеросс. научной конф.~--- Дубна, 2008. 
С.~220--229.
\bibitem{18-and}
\Au{Жамбю М.} Иерархический класс\-тер-ана\-лиз и соответствия.~--- М.: 
Финансы и статистика, 1988. 342~с. 
\bibitem{19-and}
\Au{Бериков В.\,Б., Лбов Г.\,С.} Современные тенденции в кластерном 
анализе.~--- Новосибирск: Институт математики им.\ С.\,Л.~Соболева, 2008. 
26~с.
\bibitem{20-and}
\Au{Kullback S., Leibler R.\,A.} On information and sufficiency~// The Annals of 
Math. Stat., 1951. Vol.~22. No.\,1. P.~79--86.

\label{end\stat}

\bibitem{21-and}
Аппроксимация методом наименьших квадратов (МНК). {\sf 
http://alglib.sources.ru/interpolation/\linebreak linearleastsquares.php}.

\end{thebibliography} } }

\end{multicols}

   %1
\def\stat{klimenkov}

\def\tit{ПОСТРОЕНИЕ НОВОСТНОГО РЕКОМЕНДАТЕЛЬНОГО СЕРВИСА РЕАЛЬНОГО 
ВРЕМЕНИ С ИСПОЛЬЗОВАНИЕМ NoSQL СУБД$^*$}

\def\titkol{Построение новостного рекомендательного сервиса реального 
времени с использованием NoSQL СУБД}

\def\autkol{П.\,А.~Клеменков}

\def\aut{П.\,А.~Клеменков$^1$}

\titel{\tit}{\aut}{\autkol}{\titkol}

{\renewcommand{\thefootnote}{\fnsymbol{footnote}}\footnotetext[1] {Статья рекомендована к публикации в журнале Программным комитетом конференции <<Электронные 
библиотеки: перспективные методы и технологии, электронные коллекции>> (RCDL-2012).}}

\renewcommand{\thefootnote}{\arabic{footnote}}
\footnotetext[1]{Московский государственный университет им.\ М.\,В.~Ломоносова, parser@cs.msu.su}


\Abst{Обсуждаются вопросы анализа взаимодействия пользователя с 
веб-при\-ло\-же\-ни\-ем, методы проведения подобного анализа и их 
недостатки. Приведена реализация новостного рекомендательного сервиса с 
использованием существующих подходов. Описан новый подход к 
построению рекомендательных сис\-тем, работающих в режиме, близком к 
режиму реального времени, с использованием NoSQL (not only structured query language)
сис\-те\-мы управ\-ле\-ния базами данных (СУБД).}

\KW{рекомендательные системы; minhash; mapreduce; NoSQL}

 \vskip 14pt plus 9pt minus 6pt

      \thispagestyle{headings}

      \begin{multicols}{2}

            \label{st\stat}


\section{Введение}

   Основным отличием приложений Веб~2.0 от их более старых аналогов 
является анализ взаимодействия пользователя с приложением и использование 
результатов этого анализа для модификации контента и его представления. 
Темпы развития сети Интернет диктуют создателям современных 
   веб-при\-ло\-же\-ний необходимость очень быстро адап\-ти\-ро\-вать контент под 
предпочтения пользователей. Наиболее востребованным решением этой задачи 
стали рекомендательные сис\-те\-мы, способные анализировать поведение 
пользователя, его склон\-ности и предлагать наиболее интересное наполнение. 
Проблема с подобными системами заключается в том, что они недостаточно 
быстро реагируют на постоянно изменяющийся поток входных данных. 
Особенно подвержены этому новостные ресурсы. Такое поведение связано не 
столько с алгоритмами, применяемыми для выявления пользовательских 
предпочтений, сколько с архитектурными особенностями той инфраструктуры 
и библиотек, которые широко используются для подобного анализа. В~данной 
статье представлен подход к организации новостного рекомендательного 
сервиса, призванного максимально устранить задержки в пересчете 
рекомендаций и обеспечить работу в режиме, близком к режиму реального 
времени.
   
\section{Методы веб-анализа}

   Сегодня для анализа взаимодействия пользователя с веб-при\-ло\-же\-ни\-ем 
применяются два основных подхода:
   \begin{enumerate}[(1)]
\item аналитика в реальном времени;
\item отложенная пакетная обработка логов доступа к веб-при\-ло\-же\-нию.
   \end{enumerate}
   
   У каждого из этих подходов есть преимущества и недостатки, на которых 
стоит остановиться подробнее.

\subsection{Аналитика в~реальном времени}

   Суть подхода заключается в том, что в ответ на взаимодействие пользователя 
с веб-при\-ло\-же\-ни\-ем специально установленный фрагмент кода (счетчик) 
генерирует определенные события, обрабатываемые при\-ло\-же\-ни\-ем-ана\-ли\-за\-то\-ром 
в реальном времени. Очевидно, что основным преимуществом 
подобной парадигмы является немедленное получение результатов и их 
обновление. Однако методы, применяемые при анализе данных в реальном 
времени, наиболее подходят для различных статистических расчетов (CTR, 
Churn Rate). При этом целые классы приложений не могут быть реализованы 
предложенными средствами.

\begin{figure*}[b] %fig1
\vspace*{1pt}
\begin{center}
 \mbox{%
 \epsfxsize=133.566mm
 \epsfbox{kle-1.eps}
 }
 \end{center}
 \vspace*{-6pt}
 \Caption{Схема работы Hadoop-реализации на первом~(\textit{а}) и на втором~(\textit{б}) этапах}
\end{figure*}
   
\subsection{Отложенная пакетная обработка логов доступа к веб-приложению}

   Этот подход строится на сборе логов доступа к веб-при\-ло\-же\-нию и их 
последующей обработке большими частями. Имея срез данных о 
взаимодействии пользователей с приложением за определенный период, 
возможно строить сложные модели поведения и применять их, например, для 
выдачи рекомендаций. Современные фреймворки (например, Apache Hadoop) 
обеспечивают высокую производительность, реализуя потоковую обработку 
больших объемов данных с использованием метода параллельных вычислений 
MapReduce~[1, 2].

\section{Рекомендательный сервис проекта Рамблер-новости}

   Рекомендательный сервис проекта Рамблер-но\-вости основывается на 
объединении пользователей в группы по схожести интересов и вычислении 
наиболее популярных среди групп новостей в заданном временном окне.

\subsection{Реализация сервиса}

   Суть алгоритма заключается в том, что все пользователи идентифицируются 
уникальными идентификаторами. Эти идентификаторы связываются
 с каждым 
HTTP-за\-про\-сом к новостным ресурсам (если, конечно, запрос содержал 
заголовок Cookie с корректным значением). Таким образом, поведение 
пользователя на сайте характеризуется подмножеством логов доступа к 
   веб-сер\-ве\-рам. Подсчитав схожесть каждого подмножества со всеми 
другими, можно объединить пользователей в группы с похожими 
предпочтениями.


   
   В качестве меры схожести множеств естественно использовать коэффициент 
Жаккарда. Однако проблема заключается в том, что время работы алгоритма 
подсчета этого коэффициента на нескольких миллионах множеств с сотнями и 
тысячами элементов являeтся неприемлемо большим. В~качестве оптимизации 
широко применяется вероятностный\linebreak алгоритм MinHash~[3]. Основная идея 
этого алгоритма заключается в вычислении вероятности равенства 
минимальных значений хеш-функ\-ций элементов множеств. Очевидно, что чем 
больше одинако\-вых элементов в двух срав\-ни\-ва\-емых множествах, тем выше 
указанная вероятность. А~так как вычисление сигнатуры множества 
(минимумов используемых хеш-функ\-ций) происходит только один
 раз, а 
размер сигнатуры фиксирован, то вычислительная сложность решаемой задачи 
резко снижается.


   
   Для вычисления новостных рекомендаций было принято решение 
производить обработку логов доступа веб-сер\-ве\-ров Рамб\-лер-но\-во\-стей во 
временн$\acute{\mbox{о}}$м окне 5~дней. Средний объем логов за указанный период составляет 
примерно 7~ГБ. Для реализации алгоритма был выбран фреймворк Hadoop, 
являющийся де-фак\-то стандартом для потоковой обработки больших объемов 
данных.



   Алгоритм вычисления рекомендаций был реализован в виде 
последовательности MapReduce-за\-дач, разделенных на два этапа: подсчет групп 
пользователей во временн$\acute{\mbox{о}}$м окне 5~сут.\ и подсчет рекоменда\-ций для групп 
во временн$\acute{\mbox{о}}$м окне 5~ч. Первый этап составляют следующие ступени (рис.~1,\,\textit{а}).
   \begin{enumerate}[1.]
\item Фильтрация логов во временн$\acute{\mbox{о}}$м окне 5~сут.\ и генерация множества 
уникальных URL (uniform resource locator)
для каж\-до\-го идентификатора пользователя.
\item Подсчет значений хеш-функ\-ций для всех уникальных URL каждого 
пользователя и вычисление минимума, который становится 
идентификатором группы.
\item Подсчет численности групп и отсечение \mbox{$\sim100$}~групп с 
наибольшей численностью. Необходимо заметить, что порог отсечения 
вычисляется статистически, поэтому имеет место небольшая дисперсия 
числа групп. Однако на производительность и поведение алгоритма это 
влияет незначительно.
\end{enumerate}


   Также необходимо отметить, что первоначальная реализация алгоритма 
имела еще один шаг, который позволял строго отсечь необходимое число 
групп, но ради сокращения вычислений им было решено пренебречь.
   
   Второй этап разделен на следующие ступени (рис.~1,\,\textit{б}).
   \begin{enumerate}[1.]
\item Фильтрация логов во временн$\acute{\mbox{о}}$м окне 5~ч и генерация множества 
уникальных URL для каж\-до\-го идентификатора пользователя.
\item Подсчет кликабельности новостей во всех группах.
\item Отсечение заданного числа наиболее популярных новостей в каж\-дой 
группе.
\end{enumerate}

   Получающиеся в результате отображения идентификаторов в группы и групп 
в популярные ново\-сти загружаются в хранилище Redis, поз\-во\-ля\-ющее 
запрашивать список рекомендаций для данного пользователя в реальном 
времени.

\subsection{Производительность}

   Приведенная реализация алгоритма использовалась в продуктивном 
окружении проекта Рамб\-лер-но\-во\-сти более полугода, показывая 
приемлемое время работы. На Hadoop-клас\-те\-ре из 8~узлов первая ступень 
обсчитывалась примерно 7~мин, а вторая~--- 3,5--4~мин при условии, что 
другие задачи не выполнялись параллельно. Необходимо отметить, что важным 
фактором производительности MapReduce-за\-дач является верный выбор 
количества мапперов и редьюсеров. Выбор количества мапперов производился 
автоматически. Экспериментальным путем было выяснено, что оптимальное 
число редьюсеров в данной конфигурации~---~16.

\subsection{Проблемы}

   Внимательно изучив получившуюся архитектуру и приняв во внимание 
проблемы, возникшие при реализации рекомендательного сервиса, можно 
отметить следующие аспекты.
   \begin{enumerate}[1.]
\item Загрузка логов в HDFS (Hadoop Distributed File System) и их 
обработка~--- две не связанные задачи. В~данном случае синхронизация 
логов выполнялась с помощью утилиты {\sf rsync}, а вы\-чис\-ле\-ние разности между 
файлами в директории синхронизации и файлами в HDFS, а также загрузка 
новых файлов~--- с помощью специально написанного Makefile и 
shell-скрип\-тов.
\item В Hadoop отсутствует возможность получать \mbox{данные} из разных 
источников. В частности, резуль\-та\-ты работы первого этапа алгоритма 
приходилось передавать в окружение второй ступени второго этапа в виде 
файла в кеше Hadoop. Вследствие того что этот файл может иметь весьма 
внушительный размер, MapReduce-за\-да\-чи на всех узлах могут 
столкнуться с проблемой нехватки памяти.
\item Задачи подсчета рекомендаций и их использования также не являются 
связанными и выполняются разными инструментами. В~данном случае~--- 
Hadoop и Redis.
\item Ну и самое главное~--- пакетный потоковый режим работы Hadoop не 
позволяет хоть сколь\-ко-ни\-будь приблизиться к реальному времени 
пересчета результатов.
   \end{enumerate}
   
   Отсюда возникает вопрос: можно ли решить все вышеперечисленные 
проблемы, воспользовавшись другим подходом? В~следующей части статьи 
будет описана архитектура подобного решения с применением 
   NoSQL-хра\-ни\-лищ данных.
   
\section{Введение в~NoSQL}

   Термин NoSQL впервые был использован в 1998~г.\ для описания 
реляционной базы данных, не использовавшей SQL. Он был вновь подхвачен в 
2009~г.\ и использован на конференциях приверженцами нереляционных баз 
данных. Основной движущей силой развития NoSQL-хра\-ни\-лищ стали 
   веб-стартапы, для которых важнейшей задачей является поддержание 
постоянной высокой пропускной способности хранилища при неограниченном 
увеличении объема данных. Рассмотрим основные особенности 
   NoSQL-под\-хо\-да, делающие его таким привлекательным для 
высоконагруженных веб-про\-ек\-тов~[4,~5].
   \begin{enumerate}[1.]
\item \textbf{Исключение излишнего усложнения.} Реляционные базы 
данных выполняют огромное множество различных функций и 
обеспечивают строгую консистентность данных. Однако для многих 
приложений подобный набор функций, а также удовлетворение требованиям 
ACID (atomicity, consistency, isolation, durability)
являются излишними.
\item \textbf{Высокая пропускная способность.} Многие 
NoSQL-ре\-ше\-ния обеспечивают гораздо более высокую пропускную 
способность данных, нежели традиционные СУБД. Например, колоночное 
хранилище Hypertable, реализующее подход Google Bigtable, позволяет 
поисковому движку Zvent сохранять около миллиарда записей в день. 
В~качестве другого примера можно привести саму Bigtable~[6], способную 
обработать 20~ПБ информации в день.
\item \textbf{Неограниченное горизонтальное масштабирование.} 
В~противовес реляционным СУБД, NoSQL-ре\-ше\-ния проектируются для 
неограниченного горизонтального масштабирования. При этом добавление и 
удаление узлов в кластере никак не сказывается на работоспособности и 
производительности всей системы. Дополнительным преимуществом 
подобной архитектуры является то, что NoSQL-клас\-тер может быть 
развернут на обычном аппаратном обеспечении, существенно снижая 
стоимость всей системы.
\item \textbf{Консистентность в угоду производительности.} При 
описании подхода NoSQL нельзя не упомянуть теорему CAP (consistency,
availability, partition tolerance). Согласно этой 
теореме, многие NoSQL-хра\-ни\-ли\-ща реализуют доступность данных 
(availability) и устойчивость к разделению (partition tolerance), жертвуя 
консистентностью в угоду высокой производительности. И~действительно, 
для многих классов приложений строгая консистентность данных~--- это то, 
от чего вполне можно отказаться.
\end{enumerate}

\section{Классификация NoSQL-хранилищ}

   На сегодняшний день создано большое число NoSQL-хра\-ни\-лищ. Все они 
основываются на четырех принципах из предыдущего раздела, но могут 
довольно сильно отличаться друг от друга. Многие теоретики и практики 
создавали свои собственные классификации, но наиболее простой и 
общеупотребительной можно считать сис\-те\-му, основанную на используемой 
модели данных, предложенную Риком Кейтелем (см.\ табл.)~[7].

\begin{center}  %табл
{\small{Классификация NoSQL-хранилищ по модели данных}}

\vspace*{6pt}

{\small
\tabcolsep=11pt
\begin{tabular}{|l|l|}
\hline
\multicolumn{1}{|c|}{Тип}&\multicolumn{1}{c|}{Примеры}\\
\hline
Хранилища ключ--значение&\tabcolsep=0pt\begin{tabular}{l}Redis\\
Scalaris\\
Tokio Tyrant\\
Voldemort\\
Riak\end{tabular}\\
\hline
\tabcolsep=0pt\begin{tabular}{l}Документно-ориентированные\\ хранилища\end{tabular}&
\tabcolsep=0pt\begin{tabular}{l}SimpleDB\\
CouchDB\\
MongoDB
\end{tabular}\\
\hline
Колоночные хранилища&\tabcolsep=0pt\begin{tabular}{l}Bigtable\\
HBase\\
HyperTable\\
Cassandra\end{tabular}\\
\hline
Хранилища на графах&Neo4j\\
\hline
\end{tabular}}
\end{center}

\noindent
   \begin{enumerate}[1.]
\item \textbf{Хранилища ключ--значение.} Отличительной особенностью 
является простая модель данных~--- ассоциативный массив или словарь, 
поз\-во\-ля\-ющий работать с данными по ключу. \mbox{Основная} \mbox{задача} подобных 
хранилищ~--- максимальная производительность, поэтому никакая 
информации о структуре значений не сохраняется.
\item \textbf{Документно-ориен\-ти\-ро\-ван\-ные хранилища.} Модель 
данных подобных хранилищ позволяет объединять множество пар 
ключ--зна\-че\-ние в абстракцию, называемую <<документ>>. Документы 
могут иметь вложенную структуру и объединяться в коллекции. Однако это 
скорее удобный способ логического объединения, так как никакой жесткой 
схемы у документов нет и множества пар ключ--зна\-че\-ние даже в рамках 
одной коллекции могут быть абсолютно произвольными. Работа с 
документами производится по ключу, однако существуют решения, 
позволяющие осуществлять запросы по значениям атрибутов.
\item \textbf{Колоночные хранилища.} Этот тип кажется наиболее схожим 
с традиционными реляционными СУБД. Модель данных в хранилищах 
подобного типа подразумевает хранение значений как неинтерпретируемых 
байтовых массивов, адресуемых кортежами $\langle$ключ строки, ключ 
столбца, метка времени$\rangle$. Основой модели данных служит колонка, 
число колонок для одной таблицы может быть неограниченным. Колонки по 
ключам объединяются в семейства, обладающие определенным набором 
свойств.
\item \textbf{Хранилища на графах.} Подобные хранилища применяются 
для работы с данными, которые естественным образом представляются 
графами (например, социальная сеть). Модель данных состоит из вершин, 
ребер и свойств. Работа с данными осуществляется путем обхода графа по 
ребрам с заданными свойствами.
\end{enumerate}





\section{Построение рекомендательного сервиса Рамблер-новостей 
с~помощью NoSQL}
   
   Вспоминая недостатки реализации рекомендательного сервиса на 
фреймворке Hadoop, можно отметить, что NoSQL-хра\-ни\-ли\-ща кажутся 
приемлемым вариантом их устранения. NoSQL-хра\-ни\-ли\-ща обеспечивают 
высокую пропускную способность данных как при чтении, так и при записи. Из 
этого следует, что логи доступа к веб-при\-ло\-же\-нию можно записывать 
непосредственно в базу данных. Важно также отметить, что при использовании 
до\-ку\-мент\-но-ориен\-ти\-ро\-ван\-ных решений логам можно придавать 
произвольный вид, не создавая жесткую схему. Это позволяет решать довольно 
интересную задачу хранения и обработки структурированных логов. К~тому же 
механизм выборки документов по значениям атрибутов позволяет решать 
множество аналитических задач.
   
   Большинство современных NoSQL-ре\-ше\-ний ре\-а\-ли\-зу\-ют парадигму 
вычислений MapReduсe. Наря\-ду с фундаментальным свойством 
горизон\-таль\-но\-го масштабирования это дает \mbox{возможность} переносить 
алгоритмы, предназначенные для фреймворков типа Hadoop, на хранилища 
NoSQL, получая все дополнительные преимущества.
   
   Учитывая высокую пропускную способность операций чтения, задачи 
подсчета рекомендаций и их использования можно не разделять. 
Следовательно, обновленные рекомендации будут тут же доступны 
потребителям, что приближает сервис к требованиям реального времени.
   
   Далее следовало определиться с конкретным продуктом, который можно 
было бы использовать для реализации сервиса. Среди 
   до\-ку\-мент\-но-ориен\-ти\-ро\-ван\-ных баз данных первоначальный выбор 
пал на проект Apache CouchDB~\cite{8-kli}. CouchDB работает с документами, 
представленными в формате JSON (JavaScript Object Notation). Для работы с документами предоставляется 
REST API (REpresentation State Transfer Application Programming Interface). 
Для построения запросов к документам CouchDB и их анализа 
применяются так называемые <<представления>>. По сути представление 
является обычной MapReduce-за\-да\-чей, которая может сохранять результаты 
выполнения в базе. Интересной особенностью модели данных CouchDB 
является то, что для индексации документов и представлений используются 
модифицированные B-де\-ревья. Сохраняя все особенности и преимущества 
стандартного B-де\-ре\-ва, B-де\-ревья CouchDB реализуют режим <<только 
добавление>>. Это означает, что любые операции вставки, модификации и 
изменения записываются в конец файла, представляющего B-де\-ре\-во на 
диске. Такая архитектура дает два основных преимущества: высокую ско\-рость 
записи и возможность исполнять MapReduce-за\-да\-чи только на 
изменившихся данных. Однако при всех своих преимуществах CouchDB не 
подходила для решения поставленной задачи. Во-пер\-вых, проект не 
поддерживает никакого языка запросов, что сильно затрудняет выборку 
документов по определенным критериям. Во-вто\-рых, важным критерием 
выбора была поддержка ссылок на другие документы. Подобная возможность 
есть в CouchDB, но работает она только на этапе эмиссии документа из 
   map-за\-да\-чи. К~тому же нет возможности создания ссылок на документы 
из других баз. В-третьих, неоптимизированное JSON-представление 
документов приводит к увеличению трафика между клиентом и хранилищем, 
чего хотелось избежать.
 Окончательный выбор пал на проект MongoDB~\cite{9-kli}. Обладая всеми 
преимуществами CouchDB, это хранилище устраняет перечисленные 
недостатки и предоставляет дополнительные удобные возможности. Они будут 
упомянуты в следующем разделе, опи\-сы\-ва\-ющем реализацию 
рекомендательного сервиса.

\section{Реализация рекомендательного сервиса Рамблер-новости}
   
   Первая задача, которую предстояло решить,~--- это запись логов в базу 
данных MongoDB. Первым делом требовалось определить, какое число 
операций записи в секунду обеспечивала выбранная конфигурация. Стоит 
отметить, что тестовая конфигурация представляла собой кластер из двух 
узлов, на каждом их которых был запущен демон {\sf mongod} без репликации. На 
одном их хостов запускался демон mongos, обеспечивавший шардинг 
документов. Для определения скорости записи был разработан простой скрипт, 
производивший загрузку суточных логов новостей в базу MongoDB. Лог 
состоял из 2\,770\,695~записей. Среднее время записи составило 18~мин 
30~с. Таким образом, средняя скорость записи в представленной 
конфигурации~--- 2496~операций/с. Шардинг документов осуществлялся по 
атрибуту {\sf ruid} (уникальный идентификатор пользователя). Подобный результат 
более чем достаточен для рассматриваемого сервиса, так как среднее 
количество запросов в секунду к веб-сай\-ту Рамб\-лер-но\-вости существенно 
меньше. Однако загрузка логов из ротированных лог-фай\-лов разработчика не 
устраивала. Для удовлетворения требования реального времени необходимо 
было обеспечить загрузку логов в базу сразу после обработки запроса 
   веб-сер\-ве\-ром. Для этого с помощью библиотеки ZeroMQ был разработан 
специализированный демон, агрегировавший логи с нескольких фронт-эндов 
новостей в хранилище MongoDB. 

Необходимо отметить, что загрузчик логов не 
только производил их фильтрацию, представление в формате BSON (Binary JavaScript
Object Notation) и запись в 
базу, но и подсчет значений хеш-функ\-ций для каждого URL. Это было 
обусловлено двумя факторами: снижением времени вычислений и отсутствием 
приемлемых реализаций быстрого хеширования в языке JavaScript (на нем 
реализуются MapReduce задачи в MongoDB).
   
   После того как задача загрузки логов была решена, необходимо было 
перенести реализацию алгоритма подсчета рекомендаций с Hadoop на 
MongoDB. Возвращаясь к реализации первого этапа в под\-разд.~3.1, можно 
отметить, что задачи фильтрации логов и подсчета значений хеш-функ\-ций для 
них реализуются загрузчиком. Поэтому оставалось перенести только подсчет 
минимальных значений хешей и отсечение групп с заданной численностью 
(рис.~2).

   Стоит обратить внимание на то, что из новой реализации пропал этап 
отсечения групп по численности. В первоначальной реализации отсечение 
делалось главным образом для сокращения времени загрузки отображения 
$\langle$идентификатор\ пользователя\;$\rightarrow$\;группа$\rangle$ в Redis. При использовании NoSQL-хра\-ни\-ли\-ща 
подобной проблемы не возникало.
   
   Возвращаясь к цифрам, отметим, что задача подсчета минимального хеша 
для суточных логов (2\,770\,695~записей) заняла примерно 3~мин 
10~с. Это не сильно отличается от времени выполнения той же задачи на 
Hadoop-клас\-те\-ре, и почему это\linebreak\vspace*{-12pt}

\begin{center}  %fig2
\vspace*{-3pt}
 \mbox{%
 \epsfxsize=80mm %1.356mm
 \epsfbox{kle-3.eps}
 }
 \end{center}
% \vspace*{6pt}
{{\figurename~2}\ \ \small{Схема работы NoSQL-реализации на первом~(\textit{а}) и на втором~(\textit{б}) этапах}}


%\pagebreak

\vspace*{15pt}

\addtocounter{figure}{1}

\noindent
 происходит, вполне очевидно. Однако здесь 
на помощь приходит вся мощь MongoDB. Во-пер\-вых, результаты 
   MapReduce-за\-дач сохраняются в отдельной коллекции. Последующие 
вычисления можно производить только на добавленных с прошлого запуска 
логах, выполняя rereduce на получившихся результатах. Во-вто\-рых, мощный 
язык запросов MongoDB позволяет осуществить выборку логов, добавленных с 
момента последнего запуска задачи. В~предлагаемой архитектуре задача 
сохранения времени последнего выполнения и перезапуск вы\-чис\-ле\-ний 
возложена на загрузчик логов. Важно отметить, что высокая 
производительность библиотеки ZeroMQ позволила не масштабировать 
загрузчик логов, поэтому проблем с синхронизацией времени не возникало. 
   В-третьих, MongoDB поддерживает создание и поддержание индексов на 
атрибутах документов, что существенно ускоряет выборки. На основании всего 
вышесказанного было принято решение перезапускать задачу подсчета 
минимального хеша после записи одной тысячи новых логов с выборкой по 
атрибуту {\sf timestamp} документа. Данная задача без индекса завершалась в 
среднем через 3~с, а с индексом~--- через 400--500~мс, что уже существенно 
приблизило разработку к требованиям реального времени.
   
   Теперь перейдем ко второму этапу алгоритма~--- выработке рекомендаций
   (рис.~2,\,\textit{б}). 
Здесь возникают три основные проблемы: выборка логов в заданном 
временн$\acute{\mbox{о}}$м окне, дополнительная фильтрация и ввод данных из нескольких 
источников. Выборку логов в заданном временном окне можно, как и на первом 
этапе, осуществлять запросом по атрибуту {\sf timestamp}. Стоит отметить, что 
MongoDB реализует capped collections. Это коллекции с заранее определенным 
объемом. Если объем коллекции достиг заданного порога, то новые значения 
затирают старые. Это интересный подход к ротации, но для рассматриваемой 
задачи он не подходит, так как количество логов может меняться
день ото дня. Дополнительная фильтрация осуществляется регулярными 
выражениями JavaScript, здесь нет никаких сложностей. Проблема ввода 
данных из нескольких источников решается с помощью механизма DBRef 
MongoDB. Он позволяет создавать ссылки на связанные документы в виде 
вложенных документов и получать к ним доступ при выполнении map-за\-дач. 
Удобная особенность DBRef следует из отсутствия схемы документов и других 
ограничений~--- ссылаться можно на несуществующие документы и коллекции. 
Этим фактом пользуется загрузчик логов, создавая ссылки на группы, которых 
еще нет. 

Таким образом, первые две ступени второго \mbox{этапа} удалось объединить 
в одну: map-за\-да\-ча фильтрует выборку логов во временн$\acute{\mbox{о}}$м окне 5~ч и 
возвращает пару $\langle\mathrm{group}\_\mathrm{id:url}, \mathrm{1}\rangle$, а 
reduce-за\-да\-ча подсчитывает количество кликов по каждой новости всех 
групп. Среднее время выполнения этой ступени~--- 350~мс на той же тысяче 
логов. Третья ступень была прос\-то адаптирована для исполнения MongoDB. 
Надо, правда, отметить, что отсечение заданного количества популярных 
новостей не производится. Эту задачу с целью сокращения объема вычислений 
было решено возложить на потребителя. Также следует сказать, что на 
последней ступени используется функция {\sf finalize}, позволяющая видоизменить 
результаты reduce-за\-да\-чи. В~данном случае функция {\sf finalize} производит 
сортировку новостей в группах по числу кликов.


\section{Проблемы, возникшие при~реализации сервиса 
рекомендаций}

   Естественно, при реализации сервиса возник определенный набор 
трудностей, о которых важно упомянуть. Первая трудность~--- ротирование 
логов. Так как в MongoDB отсутствует механизм времени жизни ключей, 
задачу ротирования логов приходится решать периодическим запуском 
отдельной MapReduce-за\-да\-чи. К~тому же во всех документах, тре\-бу\-ющих 
удаления, приходится явно хранить метку времени жизни. Вторая труд\-ность 
заключается в том, что формат возвращаемых map-задачей значений должен 
совпадать с форматом значений, возвращаемых reduce-за\-да\-чей. Из-за этого 
приходится создавать довольно сложные структуры, чего хотелось бы 
избежать. Третья труд\-ность~--- это специфическое устройство шардинга в 
MongoDB. Ключи распределяются по узлам не равномерно, а группами. Из-за 
этого некоторые MapReduce-за\-да\-чи на небольшом числе документов 
выполняются на одном узле, содержащем все ключи.

\section{Заключение}

   В результате проведенного эксперимента удалось создать рекомендательный 
сервис, время пересчета рекомендаций в котором на каждую тысячу новых 
логов составляет 1,5--2~с. Для проекта Рамб\-лер-но\-вос\-ти подобный 
результат является удовлетворительным, так как 1000~новых запросов к сайту 
делается за чуть большее время. Стоит отметить, что алгоритм MinHash как 
таковой не предназначен для подсчета рекомендаций в режиме реального 
времени. Более того, эффективность новой реализации рекомендательного 
сервиса может оказаться ниже, чем предыдущая реализация с по\-мощью 
фреймворка Hadoop. Однако целью данной работы было показать 
целесообразность применения NoSQL-под\-хо\-да к построению сис\-тем 
анализа данных в режиме, близком к режиму реального времени. Сделанные 
выводы позволят реализовать на описанной платформе более подходящие 
рекомендательные алгоритмы, например Covisitation~\cite{5-kli}. Важным 
свойством приведенной реализации является то, что задачи хранения и анализа 
данных удалось объединить с задачей предоставления доступа к результатам в 
единой системе, избежав накладных расходов на перемещение данных из 
одного источника в другой и улучшив общую производительность сервиса. 
Кроме того, предложенный подход упрощает решение повседневных задач 
сбора статистики о взаимодействии пользователя с веб-при\-ло\-же\-ни\-ем 
путем анализа структурированных логов мощным языком запросов СУБД 
MongoDB. Можно утверждать, что применение NoSQL-под\-хо\-да к решению 
подобного класса задач весьма перспективно и может быть использовано в 
продуктивном окружении высоконагруженных веб-при\-ло\-жений.

{\small\frenchspacing
{%\baselineskip=10.8pt
\addcontentsline{toc}{section}{Литература}
\begin{thebibliography}{9}

\bibitem{1-kli}
\Au{Dean J., Ghemawat S.} MapReduce: Simplified data processing on large 
clusters~// OSDI'04:  6th Symposium on Operating System Design and 
Implementation Proceedings.~--- Berkeley, CA, USA: USENIX Association, 
2004. P.~137--149.
\bibitem{2-kli}
\Au{Venner J.} Pro Hadoop.~--- N.Y.: Apress, 2009.
\bibitem{3-kli}
\Au{Das A.\,S., Datar M., Garg~A., Rajaram~Sh.} Google news personalization: 
Scalable online collaborative filtering~// 16th Conference (International) on World 
Wide Web Proceedings, 2007. P.~271--280.
\bibitem{4-kli}
\Au{Pokorny J.} NoSQL databases: A~step to database scalability in web 
environment~//  13th Conference (International) on Information Integration and 
Web-Based Applications and Services Proceedings, 2011. P.~278--283.
\bibitem{5-kli}
\Au{Strauch C.} NoSQL databases. {\sf http://www.christof-strauch.de/nosqldbs.pdf}.
\bibitem{6-kli}
\Au{Chang F., Dean J., Ghemawat~S., Hsieh~W.\,C., Wallach~D.\,A., 
Burrows~M., Chandra~T., Fikes~A., Gruber~R.\,E.} Bigtable: A~distributed 
storage system for structured data~//  7th USENIX Symposium on Operating 
Systems Design and Implementation Proceedings.~--- Berkeley, CA, USA: 
USENIX Association, 2006. Vol.~7. P.~205--218.
\bibitem{7-kli}
\Au{Cattel R.} Scalable SQL and NoSQL data stores~// ACM SIGMOD Record, 
2010. Vol.~39. No.\,4. P.~12--27.


\bibitem{8-kli}
\Au{Anderson J.\,C., Lehnardt~J., Slater~N.} CouchDB: The definitive guide.~--- 
Sebastopol: O'Reilly Media, 2010.

\label{end\stat}


\bibitem{9-kli}
\Au{Chodorow K., Dirolf~M.} MongoDB: The definitive guide.~--- Sebastopol: 
O'Reilly Media, 2010. 
\end{thebibliography}
} }

\end{multicols} %2
\def\stat{stupnikov}

\def\tit{ВЕРИФИЦИРУЕМОЕ ОТОБРАЖЕНИЕ МОДЕЛИ ДАННЫХ, ОСНОВАННОЙ НА~МНОГОМЕРНЫХ МАССИВАХ, 
В~ОБЪЕКТНУЮ~МОДЕЛЬ ДАННЫХ$^*$}

\def\titkol{Верифицируемое отображение модели данных, основанной на~многомерных массивах, 
в~объектную модель данных}

\def\autkol{С.\,А.~Ступников}

\def\aut{С.\,А.~Ступников$^1$}

\titel{\tit}{\aut}{\autkol}{\titkol}

{\renewcommand{\thefootnote}{\fnsymbol{footnote}}\footnotetext[1] {Работа 
выполнена при поддержке РФФИ (проект 11-07-00402-а). Статья рекомендована к 
публикации в журнале Программным комитетом конференции <<Электронные 
библиотеки: перспективные методы и технологии, электронные коллекции>> 
(RCDL-2012).}}

\renewcommand{\thefootnote}{\arabic{footnote}}
\footnotetext[1]{Институт проблем информатики Российской академии наук, 
ssa@ipi.ac.ru}

\vspace*{-6pt}       

\Abst{Рассматривается отображение модели данных, основанной на 
многомерных мас\-си\-вах (ММ-мо\-де\-ли), в объектную модель данных. Изложены 
общие принципы отображения ММ-мо\-де\-лей в объектные модели данных. 
Рассмотрено отображение конкретной модели~--- Array Data Model (ADM), 
использующейся в системе управления базами данных (СУБД) SciDB, в язык СИНТЕЗ, 
использующийся в качестве канонической модели данных в технологии предметных 
посредников. Проиллюстрирован метод верификации отображения~--- доказательства 
сохранения информации и семантики операций при отображении. Верификация 
осуществляется при помощи формального языка спецификаций AMN. Практической 
целью работы ставилось создание базы для виртуальной или материализованной 
интеграции ресурсов, основанных на многомерных массивах.}

\vspace*{-1pt}

\KW{многомерные массивы; объектная модель данных; отображение моделей 
данных; интеграция баз данных}

\vspace*{-6pt}

\vskip 14pt plus 9pt minus 6pt

      \thispagestyle{headings}

      \begin{multicols}{2}

            \label{st\stat}
            

\section{Введение}

        Развитие науки и промышленности, широкое распространение 
информационных технологий ведет к накоплению огромных объемов данных 
как в науке, так и в бизнесе. Данные могут быть как наблюдательными, 
экспериментальными, так и полученными в ходе компьютерного 
моделирования. Данные таких масштабов (часто измеряемых уже в петабайтах) 
называются <<большими данными>> (Big Data)~\cite{1-stu}. Они плохо 
поддаются обработке и анализу в рамках широко известных технологий баз 
данных, опирающихся в основном на реляционную модель данных.
        
        Именно поэтому развиваются различные модели данных, нацеленные на 
параллельную обработку и анализ данных в распределенных средах~--- гридах 
и облаках. Важными видами таких моделей являются модели данных, 
основанные на многомерных массивах (array-based data models, или ADM) 
и называемые далее ММ-мо\-де\-ля\-ми. Родственны данным моделям 
так называемые <<кубы данных>>, используемые в 
OLAP (online analytical processing) тех\-но\-ло\-гии~[2--4]. 
Исследования ММ-мо\-де\-лей начались достаточно 
давно~\cite{4-stu, 5-stu} и продолжают развиваться. В~данной статье 
рассматривается конкретная модель, а именно модель, используемая в СУБД 
SciDB~\cite{6-stu}.
        
        История SciDB начинается с 2007~г., когда на симпозиуме по 
экстремально большим базам данных (XLDB~--- extremely large data bases) 
представителями науки и 
промышленности был сделан вывод о том, что существующие СУБД не в 
состоянии манипулировать объемами данных, которые появятся в ближайшем 
будущем. Одним из примеров поставщиков таких данных служит строящийся 
телескоп LSST (Large Synoptic Survey Telescope)~\cite{7-stu}. Был также сделан 
вывод о необходимости разработки СУБД нового поколения, которая должна 
удовлетворять, в частности, следующим требованиям~\cite{8-stu}:
        \begin{itemize}
\item модель данных основывается на многомерных массивах, а не на 
кортежах;
\item модель хранения базируется на версионности, а не на обновлении 
значений;
\item масштабируемость до сотен петабайт и высокая отказоустойчивость;
\item СУБД является свободно распространяемым программным 
обеспечением.
\end{itemize}

        Некоторое время спустя был запущен международный проект под 
руководством Майкла Стоунбрейкера, целью которого стало создание новой 
СУБД, получившей название SciDB. В~настоящее 
время свободно распространяется очередная версия системы для операционных
сис\-тем (ОС) Ubuntu и  RedHat.
        
        Целью данной статьи является исследование вопроса о верифицируемом 
отображении ММ-мо\-де\-лей, и в частности ADM~\cite{9-stu}, 
использующейся в системе SciDB, в объектные 
модели данных для виртуальной или материализованной интеграции ресурсов 
при создании федеративных баз данных или хранилищ данных. 
        
        При материализованной интеграции предполагается создание 
хранилища данных (warehouse), в которое загружаются ресурсы, подлежащие 
интеграции. В~процессе загрузки происходит преобразование данных из схемы 
ресурса в общую схему хранилища.
        
        Виртуальная же интеграция рассматривается в статье применительно к 
предметным посредникам~\cite{10-stu}. Предметные посредники представляют 
собой специальный вид программного обеспечения (ПО), образующий 
промежуточный слой между пользователем (приложением) и неоднородными 
информационными ресурсами. При этом данные из ресурсов не 
материализуются в посреднике. Федеративная схема посредника, описывающая 
некоторую предметную область, создается независимо от существующих 
ресурсов. Ресурсы, релевантные предметной области, затем регистрируются в 
посреднике~--- их схемы связываются специальными семантическими 
отображениями с федеративной схемой. Исполнительная среда посредников 
предо\-став\-ля\-ет возможность пользователям (приложениям) задавать запросы 
(программы) к посреднику в терминах федеративной схемы. Эти запросы 
переписываются в частичные запросы над информационными ресурсами, а 
затем исполняются на ресурсах. Результаты частичных запросов объединяются 
и выдаются пользователю также в терминах федеративной схемы.
        
        Важным понятием технологии систем интеграции баз данных является 
каноническая модель, служащая общим языком, унифицирующим 
разнообразные модели ресурсов.
        
        Необходимым предусловием интеграции ресурсов, основанных на 
многомерных массивах, является построение отображения соответствующей\linebreak 
ММ-мо-де\-ли в каноническую модель данных, сохраняющего информацию и 
семантику операций языка манипулирования данными (ЯМД)~\cite{11-stu}. 
Это обусловлено тем, что семантические отображения, связывающие 
федеративную схему и схемы ресурсов, нужно проводить в единой 
(канонической) модели~\cite{12-stu}. Отображение должно быть 
верифицируемым~--- доказуемо правильным. 
        
        В качестве канонической модели в данной работе рассматривается язык 
СИНТЕЗ~\cite{13-stu}~--- комбинированная слабоструктурированная и 
объектная модель данных, нацеленная на разработку предметных посредников 
для решения задач в средах неоднородных ресурсов. Разработан прототип 
программных средств для поддержки среды предметных посредников с языком 
СИНТЕЗ в роли канонической модели~\cite{14-stu}.
        
        С точки зрения предметных посредников СУБД, основанные на 
многомерных массивах, пред\-став\-ля\-ют собой новый вид ресурсов, подлежащих 
интеграции в посредниках вместе с привычными ресурсами~--- реляционными 
и объектными СУБД, веб-сер\-ви\-са\-ми и~т.\,д. 
        
        Нужно отметить, что ADM подвергается некоторой критике со стороны 
исследователей, продолжающих развитие моделей, основанных на 
многомерных массивах. Так, авторы языка SciQL~\cite{15-stu} отмечают, что 
язык ADM является смесью SQL и деревьев алгебраических операций. По их 
мнению, язык для СУБД, основанных на многомерных массивах, должен быть 
интегрирован с синтаксисом и семантикой SQL:2003. Несмотря на эти 
замечания, модель ADM представляет несомненный практический интерес для 
интеграции баз данных. SciDB используется как в научных проектах, связанных 
с LSST (предполагается после запуска телескопа) и физикой высоких энергий, 
так и в коммерческих, связанных с генетикой, страхованием, финансами. 
Сравнительное тестирование SciDB с СУБД Postgres и статистическим ПО R 
показало преимущества SciDB по производительности и масштабируемости.
        
        Статья организована следующим образом. В~разд.~2 рассмотрены и 
проиллюстрированы основные принципы отобра\-же\-ния модели данных ADM в 
язык СИНТЕЗ. Принципы обобщены на случай моделей, отличных от ADM и 
СИНТЕЗ. В~разд.~3 рассмотрен метод доказательства сохранения информации 
и семантики операций при отоб\-ра\-же\-нии моделей с использованием 
формального языка спецификаций AMN~\cite{16-stu}. Метод 
проиллюстрирован на структурах данных и операциях ЯМД моделей SciDB и 
СИНТЕЗ. В~разд.~4 рассмотрены некоторые родственные исследования и 
направления дальнейшей работы.

\vspace*{6pt}

\section{Отображение модели ADM в~язык СИНТЕЗ}

\vspace*{2pt}

        SciDB поддерживает два языка для работы с массивами: AQL (Array 
Query Language) и AFL (Array\linebreak Functional Language). Язык AQL является 
        SQL (Structured Query Language)
        по\-доб\-ным декларативным языком, включающим как операции 
языка описания данных (ЯОД), так и операции ЯМД. Язык AFL представляет собой функциональный язык 
манипулирования массивами, операции которого можно объединять в 
композиции. Допускается использование операций AFL в запросах AQL.
        
        Операции языков и отображение будут иллюстрироваться на 
адаптированных примерах из сценария применения SciDB в области 
оптической астрономии~\cite{17-stu}, а также на простых примерах из 
документации SciDB~\cite{9-stu}.

\subsection{Отображение языка определения данных}

        Отображение ЯОД в данном разделе описывается независимо от вида 
интеграции~--- виртуальной или материализованной.
        
        Основной единицей определения данных в модели ADM является 
массив, имеющий конечное количество {измерений} $d_1, d_2, \ldots , 
d_n$~[9]. Длиной измерения называется количество упорядоченных значений в 
этом измерении. По умолчанию типом измерения являются 64-бит\-ные целые 
числа. Поддерживаются также нецелочисленные измерения, например строки 
или числа с плавающей точкой. Каждая комбинация значений измерений 
соответствует ячейке массива, которая может содержать конечное количество 
значений, называемых \textit{атрибутами}. Типом атрибута может быть один 
из встроенных типов ADM~\cite{9-stu}.
        
        Основная операция ЯОД ADM~--- создание массива~--- выглядит 
следующим образом:
        \begin{verbatim}
CREATE ARRAY source
< ampExposureId: int64 NULL, 
   filterId: int8,
   apMag: double >
[ ra(double), de(double), objectId=0:*];
\end{verbatim}

        Создается массив оптических источников {\sf source}, измерениями 
которого являются координаты {\sf ra} и {\sf de} типа {\sf double} и целочисленный 
идентификатор объекта. Для целочисленного измерения указаны его нижняя (0) 
и верхняя (<<*>>, обозна\-ча\-ющая бесконечность) границы. Ячейка массива 
состоит из трех атрибутов: {\sf ampExposureId}, {\sf filterId}, 
{\sf apMag}. Указано, что 
атрибут {\sf ampExposureId} может принимать неопределенное значение {\sf NULL}. 
В~данном примере приведены только некоторые из реально используемых 
атрибутов и измерений.
        
        В языке СИНТЕЗ создание массива представляется определением 
одноименного класса:
        \begin{verbatim}
{ source; in: class;
  instance_type:{
  double ra;
  ra2long: {in: function; 
            params: {-ret/long}; };
  double de;
  de2long: {in: function; 
            params: {-ret/long}; };
  long objectId; metaslot lower: 0;  
  higher: inf; end
  objectIdBounds: {in: invariant;
    {{all s(source(s) -> s.objectId >= 0)}}
  };
  long ampExposureId;
  short filterId;
  double apMag;
  key: { unique; { ra, de, objectId } };
  definiteness: {obligatory;
    { ra, de, objectId, filterId, apMag } };
  };
}
\end{verbatim}

        Как измерения, так и атрибуты, составляющие ячейку, представляются в 
языке СИНТЕЗ атрибутами типа экземпляров ({\sf instance\_type}) класса. Между 
встроенными типами ADM ({\sf int8}, {\sf int64}, {\sf double} и~др.)\ и встроенными 
типами языка \mbox{СИНТЕЗ} ({\sf short}, {\sf long}, {\sf double}) устанавливается взаимно 
однозначное соответствие. Совокупность атрибутов, со\-от\-вет\-ст\-ву\-ющих 
измерениям, объявляется уникальной (инвариант {\sf key}, выражаемый 
встроенным утверждением {\sf unique}). Объявляется также, что атрибуты, 
соответствующие измерениям и не-{\sf NULL} атрибутам ADM, должны быть 
определены у всех экземпляров класса (инвариант {\sf definiteness}, выражаемый 
встроенным утверждением {\sf obligatory}).
        
        Таким образом обеспечивается сохранение отличи\-тель\-ных свойств 
многомерных массивов (<<кубов данных>>), существенным образом 
раз\-ли\-ча\-ющих измерения и атрибуты, со\-став\-ля\-ющие \mbox{ячейку}:
        \begin{itemize}
\item по набору значений измерений однозначно определяется набор 
значений атрибутов ячейки (уникальность измерений);
\item ячейка массива всегда определяется полным набором значений 
измерений (определенность измерений).
\end{itemize}

        Заметим также, что отсутствие в коллекции объекта с некоторым 
набором значений измерений означает \textit{пустую ячейку} в массиве.
        
        Для нецелочисленных измерений {\sf ra} и {\sf de} в языке СИНТЕЗ кроме 
атрибутов определяются функции {\sf ra2long}, {\sf de2long}, преобразующие 
нецелочисленные значения в целочисленные. Необходимость при\-вне\-се\-ния этих 
функций вызвана следующим. При попытке описать операции, характерные для 
ММ-мо\-де\-лей, в объектной модели (в частности, в языке СИНТЕЗ) 
выясняется необходимость применения принципиально различных механизмов 
работы с целочисленными и нецелочисленными измерениями. Это вызвано 
различием типов измерений, возможной неравномерностью шага измерения 
и~т.\,д.\linebreak Для того чтобы обеспечить возможность единообразного описания 
операций над цело\-чис\-лен\-ными и нецелочисленными измерениями и 
необходимы функции, приводящие нецелочисленные\linebreak измерения к 
целочисленным.
        
        Ограничения, связанные с нижними и верхними границами 
целочисленных измерений, пред\-став\-ля\-ют\-ся в языке СИНТЕЗ, во-пер\-вых, 
мета\-слотом соответствующего атрибута (например,\linebreak {\sf objectId}). В~метаслоте 
хранится метаинформация, связанная с атрибутом как с отдельной сущностью 
языка. В~данном случае метаслот включает два слота {\sf lower} и {\sf higher}, 
отвечающих соответственно верхней и нижней границе измерения. 
        Во-вто\-рых, создается инвариант (например, {\sf objectIdBounds}), 
предикативная спецификация которого устанавливает ограничения на значения 
измерения для каждого из объектов класса, отвечающего массиву. 
Спецификация инварианта имеет вид формулы первого порядка, где {\sf all}~--- 
квантор существования, <<\verb -> >> --- импликация.
        
        Необходимо отметить, что массив представляется в объектной модели 
множеством объектов класса (фактически кортежей значений атрибутов). При 
этом наблюдается некоторое противоречие со стремлением создателей 
        ММ-мо\-де\-лей \mbox{отойти} от моделей, основанных на кортежах. Однако в 
контексте интеграции ресурсов ММ-мо\-де\-ли это лишь один класс из 
большого множества разнообразных классов моделей данных. Представление 
специфических ММ-мо\-де\-лей в объектной модели является методологически 
и технически гораздо более простым и естественным, нежели использование 
многомерных массивов в качестве канонической модели.
        
        Изложенные принципы отображения ЯОД могут быть обобщены на 
случай, когда канонической является объектная или 
        объ\-ект\-но-ре\-ля\-ци\-он\-ная модель, отличная от языка СИНТЕЗ. 
Также не принципиален выбор модели данных, основанной на многомерных 
массивах. В~общем виде принципы отображения ЯОД выглядят следующим 
образом:
        \begin{itemize}
\item массив отображается в коллекцию типизированных объектов (класс) 
объектной модели;
\item измерения и атрибуты, составляющие ячейку массива, отображаются в 
атрибуты типа экземпляров класса;
\item между встроенными типами модели, основанной на многомерных 
массивах, и встроенными типами объектной модели устанавливается 
взаимно однозначное соответствие;
\item совокупность атрибутов, соответствующих измерениям, объявляется 
уникальной (при помощи механизма ключей, утверждений или 
инвариантов);
\item атрибуты, соответствующие измерениям и не-{\sf NULL} атрибутам ячейки 
массива, объявляются определенными (при помощи утверждений или 
инвариантов);
\item для нецелочисленных измерений в типе экземпляров дополнительно 
определяются методы, преобразующие нецелочисленные значения в 
целочисленные;
\item ограничения, связанные с нижними и верхними границами 
целочисленных измерений, отображаются при помощи инвариантов или 
встроенных утверждений о кардинальности соответствующих атрибутов. 
В~случае использования инвариантов при отображении границы измерений 
представляются также метаданными атрибута.
\end{itemize}

\subsection{Отображение языка манипулирования данными}

        При интеграции баз данных для установления семантических 
соотношений между схемами ресурсов и федеративной схемой необходимо 
отображение ЯОД исходной модели в каноническую. Язык манипулирования данными канонической 
модели, напротив, необходимо отображать в ЯМД исходной модели. Это 
связано с тем, что запросы к посреднику в канонической модели необходимо 
отображать в запросы к ресурсам.
        
        Отметим отличие виртуальной и материализованной интеграции. При 
виртуальной интеграции отображение ЯМД обеспечивает возможность 
трансляции программ на языке посредника в запросы на языке ресурсов. 
        
        В случае материализованной интеграции данные извлекаются из ресурса 
и представляются в хранилище в канонической модели. При этом программы 
на языке канонической модели исполняются непосредственно на данных. 
Отоб\-ра\-же\-ние\linebreak ЯМД нужно лишь для того, чтобы убедиться, что отображение 
моделей сохраняет информацию и семантику операций. Семантически 
правильное\linebreak отоб\-ра\-же\-ние служит базой для процесса 
        <<из\-вле\-че\-ния--пре\-образо\-ва\-ния--за\-груз\-ки>> (ETL), 
формирующего из данных ресурса данные хранилища:\linebreak ETL-про\-цесс может 
быть выражен только в терминах канонической модели.
        
        \smallskip
        
        Язык запросов (программ) модели СИНТЕЗ представляет собой 
        Datalog-по\-доб\-ный язык в объектной среде. Программа представляет 
собой набор конъюнктивных запросов (правил) вида 

\noindent
\begin{multline*}
        q(x/T): - C_1(x_1/T_1),\ldots , C_n(x_n/T_n), (X_1,Y_1), 
\ldots \\
\ldots F_m(X_m,Y_m), B\,.
        \end{multline*}
        Тело запроса представляет собой конъюнкцию 
        пре\-ди\-ка\-тов-кол\-лек\-ций, функциональных предикатов и 
ограничения. Здесь $C_i$~--- имена коллекций (классов), $F_i$~--- имена 
функций, $x_i$~--- имена переменных, значения которых пробегают по 
классам, $T_i$~--- типы переменных, $X_j$ и $Y_j$~--- входные и выходные 
параметры функций, $B$~--- ограничение, налагаемое на $x_i$, $X_j$, $Y_j$. 
Предикаты, находящиеся в голове правил, могут быть использованы в телах 
других правил в качестве пре\-ди\-ка\-тов-кол\-лек\-ций. 
        
        В дальнейшем будет часто использоваться запись 
        пре\-ди\-ка\-та-кол\-лек\-ции вида {\sf source([ra, de])}. Неформально это 
означает, что представляют интерес не объекты класса {\sf source} целиком, а 
лишь их атрибуты {\sf ra} и {\sf de}. Формально запись означает сокращение от 
{\sf source(\_/source.inst[ra, de])}. Здесь знак <<{\sf \_}>> обозначает анонимную 
переменную, {\sf source.inst}~--- анонимный тип экземпляров (instance) класса 
{\sf source}, а {\sf ra} и {\sf de}~--- необходимые атрибуты типа экземпляров класса.
        
        Будет также использоваться запись {\sf source([i, j, val1/val])}, означающая 
переименование атрибута {\sf val} в {\sf val1}.
        
        \medskip
        
        При отображении ЯМД будут сначала рассмотрены основные 
конструкции языка программ СИНТЕЗ, соответствующие конструкциям языка 
AQL. Затем будут рассмотрены конструкции \mbox{СИНТЕЗ}, соответствующие 
конструкциям языка AFL.
        
        Начнем рассмотрение с конструкций языка СИНТЕЗ, соответствующих 
конструкциям языка AQL, связанных с {извлечением} данных.
        
%        \smallskip
        
\paragraph*{Предикаты-классы, условия, подзапросы.} Рас\-смот\-рим 
программу, извлекающую координаты ({\sf ra}, {\sf de}) и апертурную звездную 
величину ({\sf apMag}) астрономических источников из класса  {\sf source} с 
условием на фильтр ({\sf filterId}) и апертурную звездную величину, причем 
запрос~{\sf q} использует результаты запроса~{\sf r}:
        \begin{verbatim}
q([ra,de,apMag]) :- r([ra,de,apMag]),
   filterId= #filterId.
r([ra,de,apMag]) :- source([ra,de,apMag]),
   apMag >= #apMag.
\end{verbatim}
Здесь {\sf \#filterId} и {\sf \#apMag}~--- некоторые константы 
соответствующих типов.
        
        Такая программа представляется в AQL сле\-ду\-ющим запросом:
        \begin{verbatim}
SELECT apMag FROM 
  ( SELECT apMag FROM source
    WHERE apMag >= #apMag )
WHERE filterId = #filterId;
\end{verbatim}
        
        Простые условия отображаются в AQL без изменений, рекурсивные 
запросы представляются вложенными запросами. Заметим, что координаты 
{\sf ra} и {\sf de} не указываются в секции {\sf SELECT}~--- они являются измерениями и 
извлекаются по умолчанию.
        
\paragraph*{Соединение классов.} Соединение по определенным атрибутам 
(например, {\sf objectId})
        \begin{verbatim}
q2([ra, de, filterId, uMag]) :- 
    source([ra, de, objectId, fliterId]), 
    objectSummary([objectId, uMag]).
\end{verbatim}
представляется в AQL конструкцией {\sf JOIN-ON}:
\begin{verbatim}
SELECT filterId, uMag INTO q2
FROM source
JOIN objectSummary 
ON source.objectId = objectSummary.objectId;
\end{verbatim}
где массив {\sf objectSummary} имеет вид: 
\begin{verbatim}
CREATE ARRAY objectSummary
<uMag: float NULL,  gMag: float NULL>
[ objectId=0:* ];
\end{verbatim}
        
\paragraph*{Агрегация.} Рассмотрим запрос, возвращающий объекты с 
минимальной звездной величиной {\sf uMag}:
        \begin{verbatim}
q([objectId, uMag]) :-  
  objectSummary(obj/[objectId, uMag]), 
    uMag = min(obj.uMag).
\end{verbatim}

        Запрос представляется в AQL с использованием агрегирующей функции 
того же рода:
        \begin{verbatim}
SELECT uMag
FROM source, 
 (SELECT min(uMag) AS min FROM Source)
WHERE uMag = min;
\end{verbatim}
        
\paragraph*{Группирование.} Рассмотрим запрос, возвра\-ща\-ющий среднее 
значение звездной величины {\sf uMag}, вычисленное на группе по 
идентификатору объекта {\sf filterId}:
        \begin{verbatim}
q([objectId, avgMag]) :- 
    group_by({objectId}, 
       q2(obj/[ra,de,filterId, uMag])),
    avgMag = average(obj.uMag).
\end{verbatim}

        Здесь коллекция {\sf q2}, на которой производится группирование по 
атрибуту {\sf objectId}~--- результат соединения классов {\sf source} и 
{\sf objectSummary}, рассмотренных выше.
        
        Очевидно, в AQL запрос представляется при помощи конструкции 
GROUP BY:
        \begin{verbatim}
SELECT avg(uMag) AS avgMag
FROM q2 GROUP BY objectId;
\end{verbatim}
        
        Рассмотрим конструкции языка СИНТЕЗ, соответствующие 
конструкциям языка AQL и связанные с {изменением} данных.

        
\paragraph*{Обновление.} Рассмотрим запрос, изменяющий значения в 
квадратной матрице (см.\ предыдущий пример) на значения с обратным знаком 
в том случае, если модуль значения больше~5:
        \begin{verbatim}
source(x/[i, j, val]) :- 
    source(x/[i, j, val1/val]), 
       abs(val) > 5, val = -val1.
\end{verbatim}
        
        В AQL данный запрос представляется сле\-ду\-ющим образом:
        \begin{verbatim}
UPDATE source
SET val =  -val WHERE abs(val) > 5;
\end{verbatim}


        
\paragraph*{Удаление.} Рассмотрим программу, удаляющую из базы данных 
класс и все его содержимое:
        \begin{verbatim}
-source(x) :- source(x).; 
delete_frame(source).
\end{verbatim}

        В правилах со знаком минус в голове осуществляется удаление объектов 
из коллекции. В~данном случае из коллекции удаляются все объекты. Функция 
{\sf delete\_frame} удаляет коллекцию как объект из базы данных. Операция <<{\sf ;}>> 
обозначает последовательную композицию программ. В~AQL данный запрос 
представляется при помощи операции {\sf DROP}:
\begin{verbatim}
DROP ARRAY source;
\end{verbatim}

        Рассмотрим принципы отображения конструкций языка СИНТЕЗ, 
соответствующих конструкциям AFL, на примере {расширения элементов 
мас\-си\-ва в подмассивы}. Каждый элемент массива расширя\-ется в подмассив 
определенного размера. Значения всех ячеек подмассива дублируют значение 
оригинальной ячейки. Пример программы, расширяющей каждую ячейку 
матрицы $3\times3$ в подматрицу $2\times2$:
        \begin{verbatim}
q([i,j,val]) :- {x/[i,j,val] | exists y (
  source(y/[i1/i, j1/j, val]) & 
  ( i = i1*2 & j = j1*2 | i = i1*2 +1 & 
  j = j1*2 | i= i1*2 & 
  j= j1*2 + 1 | i= i1*2 +1 & j= j1*2 +1))}.
\end{verbatim}
    Здесь выражение $\{x/T \vert F(x)\}$, где $F$~--- формула со свободной 
переменной~$x$, обозначает конструкцию выделения множества; {\sf exists} 
обозначает квантор существования. 

\columnbreak
        
        В ADM запрос представляется с использованием операции {\sf xgrid}:
        \begin{verbatim}
SELECT * FROM xgrid(source, 2, 2);
\end{verbatim}
        
        Можно заметить, что операция AFL {\sf xgrid} имеет достаточно сложно 
устроенный прообраз в канонической модели (это справедливо и для многих 
других операций). Между тем эти операции являются естественными для 
массивов. Поэтому при интеграции ресурсов, основанных на многомерных 
массивах, в канонической модели возможно использование специального 
класса {\sf array}, инкапсулирующего специфические операции, характерные для 
многомерных массивов:
        \begin{verbatim}
{ array; in: class;
  instance_type: {
  xgrid: { in: function; 
    params: {
     +dimensions/{sequence; 
      type_of_element: string;},
     +scales/{sequence; 
      type_of_element: integer;}};
  };  };
}
\end{verbatim}
        В приведенном примере рассмотрена сигнатура единственной операции 
{\sf xgrid}, параметрами которой являются последовательность имен измерений\linebreak 
{\sf dimensions} и последовательность масштабов увеличения по каждому из 
измерений {\sf scales}. Па\-ра\-мет\-ром операции по умолчанию также считается 
класс\linebreak {\sf array} как коллекция объектов. При отображении ЯОД каждый класс~--- 
образ массива (например, класс {\sf source} из подразд.~2.1) становится подклассом 
класса {\sf array}:
        \begin{verbatim}
{ source; in: class; superclass: array;
  instance_type: { ... };
}
\end{verbatim}

        Аналогично {\sf xgrid}, операциями класса {\sf array} могут быть 
представлены такие операции AFL, как {\sf substitute}, {\sf sort}, 
{\sf multiply} и~т.\,д. 
        
        Заметим, что решение о представлении операций, характерных для 
многомерных массивов, в рамках специального класса канонической модели 
допускает обобщение на объектные канонические модели, отличные от языка 
СИНТЕЗ, и модели, основанные на многомерных массивах, отличные от ADM.
        
        \smallskip
        
        Разработанные отображения ЯОД и ЯМД были частично реализованы на 
языке ATL (ATLAS\linebreak Transformation Language)~\cite{18-stu}. ATL-программы 
пред\-став\-ля\-ют собой де\-кла\-ра\-тив\-но-им\-пе\-ра\-тив\-ные трансформации, 
реализующие отображения произвольных исходных моделей уровня М1 
(согласно классификации MOF~\cite{19-stu}), конформных исходной 
метамодели уровня М2, в целевые модели уровня М1, конформные целевой 
метамодели уровня М2. Модели уровня М1 являются схемами, 
представленными в канонической модели данных или модели ADM; модели 
уровня М2 есть описание абстрактного синтаксиса канонической модели или 
модели ADM. В~качестве метамодели уровня М3, которой конформны 
метамодели уровня M2, рассматривается модель Ecore~\cite{20-stu}. Cинтаксис 
ЯОД и ЯМД ядра канонической информационной модели (языка СИНТЕЗ) и 
модели ADM был представлен в метамодели Ecore. 
        
        Было осуществлено построение ATL-транс\-фор\-ма\-ций, реализующих 
отображения подмножества ЯОД модели ADM в ЯОД канонической модели и 
подмножества ЯМД канонической модели в ЯМД модели ADM. Подмножества 
ЯМД определялись конструкциями ЯОД и ЯМД канонической модели, 
поддерживаемыми в настоящее время в исполнительной среде предметных 
посредников. Поддерживаемый язык запросов канонической модели включает 
правила, в голове которых могут быть пре\-ди\-ка\-ты-кол\-лек\-ции, а в теле~--- 
конъюнкция пре\-ди\-ка\-тов-кол\-лек\-ций, условий соединения коллекций и 
других условий на значения атрибутов типов экземпляров коллекций. 
Условием соединения может быть только равенство атрибутов. 
Поддерживаются основные арифметические предикаты и функции, а также 
термы~--- вызовы функций. 

\section{Сохранение информации и~семантики операций языка манипулирования данными 
при~отображении}
        
        Метод доказательства сохранения информации и семантики операций 
при отображении моделей данных~\cite{21-stu} основывается на представлении 
семантики моделей в формальном языке спецификаций AMN~\cite{16-stu}. 
        
        Язык AMN представляет собой тео\-ре\-ти\-ко-мо\-дель\-ную нотацию, 
основанную на теории множеств и типизированном языке первого порядка. 
Спецификации AMN называются абстрактными машинами. Язык AMN позволяет 
интегрированно рас\-смат\-ри\-вать спецификацию пространства состояний и 
поведения машины (определенного операциями на состояниях). В~AMN 
формализуется специальное отношение между спецификациями~--- 
{уточнение}. Неформально спецификация~$B$ уточняет 
спецификацию~$A$, если пользователь может использовать $B$ вместо~$A$, 
не замечая факта замены~$A$ на~$B$. 
{\looseness=1

}
        
        Идея метода заключается в следующем. Рассмотрим исходную 
модель~$S$ и целевую модель~$T$. Построим отображение~$\theta$ 
модели~$S$ в модель~$T$ (подобно изложенному в предыдущем разделе). 
Выразим семантику моделей в виде абстрактных машин AMN, построив при 
этом машины $M_S$ и $M_T$ соответственно. При этом структуры данных 
моделей~--- классы, массивы~--- представляются переменными машин, 
различные свойства структур данных представляются инвариантами машин, 
характерные операции моделей данных представляются операциями машин. 
Рассматриваемые операции исходной и целевой модели должны быть связаны 
отображением ЯМД. Отображение ЯОД представляется в виде специального 
\textit{склеивающего инварианта}~--- замкнутой формулы, связывающей 
состояния машин~$M_S$ и~$M_T$.
        
        Будем считать отображение~$\theta$ сохраняющим инфор\-ма\-цию и 
семантику операций, если машина~$M_S$, соответствующая исходной модели, 
уточняет машину~$M_T$, соответствующую целевой модели. Уточнение 
доказывается интерактивно при помощи специальных программных 
средств~\cite{22-stu}.
        
        \smallskip
        
        В качестве иллюстрации основных принципов выражения семантики 
моделей ADM и СИНТЕЗ в AMN рассмотрим частичные 
        AMN-спе\-ци\-фи\-ка\-ции, соответствующие данным моделям.
        
        Cпецификация, выражающая семантику объектной модели языка 
СИНТЕЗ, представляется в языке AMN конструкцией {\sf REFINEMENT}:
\begin{verbatim}
REFINEMENT ObjectDM
\end{verbatim}

        Переменные, составляющие пространство состояний объектной модели, 
объявлены в разделе {\sf ABSTRACT\_VARIABLES} машины {\sf ObjectDM} и 
типизируются в разделе {\sf INVARIANT}:
\begin{verbatim}
ABSTRACT_VARIABLES
    typeNames, classNames, attributeNames,
    instanceType, typeAttributes, 
      attributeType,
    unique, obligatory,
    intAttributeLowerBound, 
      intAttributeHigherBound,
    objectIDs, objectType, objectsOfClass,
    integerAttributeValue,
    adtAttributeValue
INVARIANT ...
\end{verbatim}

        Раздел {\sf INVARIANT} содержит формулу, которая состоит из предикатов, 
типизирующих переменные состояния и налагающих различные совместные 
ограничения на переменные. Предикаты соединяются операцией конъюнкции.
        
        Так, имена типов и классов представлены переменными {\sf typeNames} и 
{\sf classNames}, тип которых~--- подмножество множества строк 
({\sf STRING\_Type}):
        \begin{verbatim}
typeNames: POW(STRING_Type) &
classNames: POW(STRING_Type)
\end{verbatim}
        
        \noindent
        Здесь {\sf POW}~--- конструктор множества подмножеств.
        
        Атрибуты (переменная {\sf attributeNames}) пред\-став\-ле\-ны частичной 
функцией (знак <<\verb +-> >>), ставящей в соответствие уникальному идентификатору 
атрибута (натуральному числу из множества {\sf NAT}) имя атрибута (строку):
        \begin{verbatim}
attributeNames: NAT +-> STRING_Type
\end{verbatim}

        Типы экземпляров классов (переменная\linebreak {\sf instanceType}) представлены 
тотальной функцией (знак \verb -> ) из множества имен классов в 
множество имен типов:
        \begin{verbatim}
instanceType: classNames --> typeNames
\end{verbatim}

        Принадлежность атрибутов типам (переменная {\sf typeAttributes}) 
выражена тотальной функцией из множества имен типов в множество 
подмножеств атрибутов:
        \begin{verbatim}
typeAttributes: 
  typeNames --> POW(dom(attributeNames))
\end{verbatim}
        Здесь {\sf dom}~--- операция, возвращающая область определения 
функции, а {\sf dom(attributeNames)}~--- множество имен атрибутов.
        
        Типы значений атрибутов (переменная\linebreak {\sf attributeType}) представлены 
функцией из множества атрибутов в множество идентификаторов встроенных 
типов данных {\sf BuiltInTypes}:
        \begin{verbatim}
attributeType: 
  dom(attributeNames) +-> BuiltInTypes
\end{verbatim}

        Множества уникальных атрибутов типов {\sf unique}\linebreak и множества 
определенных атрибутов типов\linebreak {\sf obligatory} представлены тотальными 
функциями из множества имен типов в множество подмножеств атрибутов:
\begin{verbatim}
unique: 
  typeNames --> POW(dom(attributeNames))&
obligatory: 
  typeNames --> POW(dom(attributeNames))
\end{verbatim}

        Нижние границы целочисленных атрибутов (переменная 
{\sf intAttributeLowerBound}) представлены час\-тич\-ной функцией из множества 
атрибутов в множество целых чисел:
\begin{verbatim}
intAttributeLowerBound: 
  dom(attributeNames) +-> INT
\end{verbatim}

        Аналогично представляются верхние границы.
        
        Идентификаторы объектов (переменная\linebreak {\sf objectIDs}) представлены 
подмножеством натуральных чисел:
        \begin{verbatim}
objectIDs: POW(NAT)
\end{verbatim}

        Типы объектов (переменная {\sf objectType}) представлены тотальной 
функцией из множества объектных идентификаторов в множество имен типов:
\begin{verbatim}
objectType: objectIDs --> typeNames
\end{verbatim}

        Состав классов (переменная {\sf objectsOfClass}) представлен тотальной 
функцией из множества имен классов в множество подмножеств 
идентификаторов объектов:
        \begin{verbatim}
objectsOfClass: 
  classNames --> POW(objectIDs)
\end{verbatim}
        
        Значения атрибутов объектов (переменные\linebreak {\sf integerAttributeValue}, 
{\sf adtAttributeValue} и~др.)\ пред\-став\-ле\-ны функциями из множества атрибутов\linebreak 
в функции из множества идентификаторов объектов в множества значений 
атрибутов. Для простоты рассмотрены лишь функции для целочисленных 
атрибутов и атрибутов со значениями АТД\linebreak (абстрактного типа данных) (объектами):
        \begin{verbatim}
integerAttributeValue: 
 dom(attributeNames) +-> (objectIDs+->INT)& 
adtAttributeValue: 
 dom(attributeNames) +-> (objectIDs+->NAT)
\end{verbatim}
        
        Дополнительные необходимые свойства переменных состояния 
представлены конъюнктивными компонентами инварианта. Так, каждый 
атрибут является атрибутом некоторого типа:
        \begin{verbatim}
        
UNION(tt).(tt:typeNames|typeAttributes(tt))=
    dom(attributeNames)
\end{verbatim}
        Здесь {\sf UNION}~--- родовая операция объединения, в данном случае 
объединяются множества атрибутов {\sf typeAttributes(tt)} по всем именам 
типов~{\sf tt} из множества {\sf typeNames}. 
        
        Никакой атрибут не принадлежит двум типам одновременно:
        \begin{verbatim}
!(t1, t2).(t1: typeNames & t2: typeNames =>
  (typeAttributes(t1) /\ typeAttributes(t2) 
    = {}))
\end{verbatim}
   Здесь <<\verb ! >>~--- знак квантора всеобщности, <<\verb => >>~--- логическая 
импликация, <<\verb /\ >>~--- символ пересечения множеств, <<\verb {} >>~--- пустое 
множество.
        
        Уникальные и определенные атрибуты типа выбираются из множества 
атрибутов типа:
        \begin{verbatim}
!(tt).(tt: dom(unique) => unique(tt) <: 
typeAttributes(tt)) &
!(tt).(tt: dom(obligatory) => 
    obligatory(tt) <: typeAttributes(tt))
\end{verbatim}
        Здесь <<\verb <: >>~--- символ отношения мно\-жес\-во--под\-мно\-жество.
        
        Нижние и верхние границы могут быть определены только для 
целочисленных атрибутов:
        \begin{verbatim}
!(attr).(attr: dom(intAttributeLowerBound)=> 
    attributeType(attr) = Integer) 
\end{verbatim}

        Тип объекта~--- экземпляра класса есть тип экземпляров этого класса:
        \begin{verbatim}
!(cc).(cc: classNames => 
    !(oo).(oo: objectsOfClass(cc) => 
       objectType(oo) = instanceType(cc))) 
\end{verbatim}

        Для каждого атрибута определена ровно одна функция значений:
        \begin{verbatim}
dom(adtAttributeValue) /\ 
  dom(integerAttributeValue) = {} &
dom(adtAttributeValue) \/ 
  dom(integerAttributeValue) = 
    dom(attributeNames)
\end{verbatim}
   Здесь <<\verb \/ >>~--- символ объединения множеств.
        
        Для любого объекта и любого определенного атрибута типа этого 
объекта функция значений атрибута определена на объекте:
        \begin{verbatim}
!(oo, aa).(oo: dom(objectType) & 
  aa: typeAttributes(objectType(oo)) & 
  aa: obligatory(objectType(oo)) =>
      (attributeType(aa) = Integer => 
       oo: dom(integerAttributeValue(aa))) &
      (attributeType(aa) = ADT =>
       oo: dom(adtAttributeValue(aa)))) 
\end{verbatim}

        Для любого объекта и любого целочисленного атрибута типа объекта, 
определенного на объекте и для которого определена нижняя (верхняя) 
граница, значение атрибута не меньше (не больше) нижней (верхней) границы:
        \begin{verbatim}
!(oo, aa).(oo: objectIDs & 
    aa: typeAttributes(objectType(oo)) &
    oo: dom(integerAttributeValue(aa) => 
    (aa: dom(intAttributeLowerBound) =>
        (integerAttributeValue(aa)(oo) >= 
         intAttributeLowerBound(aa))) ) 
\end{verbatim}

        Объект однозначно идентифицируется набором своих уникальных 
атрибутов:
        \begin{verbatim}
!(oo1, oo2).(oo1: objectIDs & 
  oo2: objectIDs &
    objectType(oo1) = objectType(oo2) & 
    unique(objectType(oo1)) /= {} &
    !(aa).(aa: unique(objectType(oo1)) => 
      (attributeType(aa) = Integer =>
        integerAttributeValue(aa)(oo1) =
         integerAttributeValue(aa)(oo2)) &
      (attributeType(aa) = ADT =>
         adtAttributeValue(aa)(oo1) =
          adtAttributeValue(aa)(oo2)) ) => 
    oo1 = oo2 )
\end{verbatim}

        Из всего ЯМД в спецификации рассмотрена единственная операция 
{\sf update} обновления значений атрибута в объектах класса:
        \begin{verbatim}
OPERATIONS
update(cls, attr, exp, cond) =
PRE cls: classNames & 
  attr: typeAttributes(instanceType(cls)) &
  attributeType(attr) = Integer &
  exp: INT --> INT & cond: NAT --> BOOL
THEN
 integerAttributeValue := 
 integerAttributeValue <+ 
 { xx | xx: (NAT*(NAT<->INT)) &
  #(oo, val).( oo: objectsOfClass(cls) & 
  val: INT &
    xx = attr |-> ({oo |-> val}) & 
  (cond(integerAttributeValue(attr)(oo)) 
  = TRUE =>
  val=exp(integerAttributeValue(attr)(oo)))&
  (cond(integerAttributeValue(attr)(oo)) 
  = FALSE => 
  val=integerAttributeValue(attr)(oo)))}
END
\end{verbatim}

        Параметрами операции являются имя класса {\sf cls}, имя целочисленного 
атрибута {\sf attr} типа экземпляров класса, функция {\sf exp}, отвечающая за 
преобразование атрибута, и функция {\sf cond}, отвечающая условию на значение 
атрибута. Пусть {\sf o}~--- некоторый объект класса {\sf cls}, для которого определено 
значение атрибута {\sf attr}, и это значение есть~{\sf v}. Тогда операция {\sf update} 
изменяет значение атрибута на {\sf exp(v)} в случае, если выражение {\sf cond(v)} 
принимает значение <<истина>>, и оставляет значение атрибута без изменений в 
противном случае. Очевидно, такая операция {\sf update} есть обобщение примера 
обновления, рассмотренного в подразд.~2.2. Действительно, для рассмотренного 
примера {\sf cls} есть {\sf source}, {\sf attr} есть {\sf val}, 
{\sf exp(v)}\;=\;-\,{\sf v}, {\sf cond(v)}\;=\;{\sf abs(v)}.
        
        Заметим, что в рассмотренной спецификации для простоты не 
рассмотрены некоторые черты объектной модели, например отношения 
        тип--под\-тип и класс--под\-класс.
        
        \smallskip
        
        Спецификация, выражающая семантику модели ADM, представляется в 
языке AMN конструкцией
        \begin{verbatim}
REFINEMENT ArrayDM
\end{verbatim}

        Переменные, составляющие пространство состояний объектной модели, 
объявлены в разделе {\sf ABSTRACT\_VARIABLES} машины {\sf ArrayDM}:
        \begin{verbatim}
ABSTRACT_VARIABLES
    arrayNames, dimensionNames, 
    cellAttributeNames,
    arrayDimensions, arrayCellAttributes,    
    cellAtrributeType, nullable, 
    dimLowerBound, dimHigherBound,
    cells, dimensionValue, 
    integerCellAttributeValue
\end{verbatim}

        Имена массивов представлены переменной\linebreak 
{\sf arrayNames}; имена измерений~--- переменной\linebreak 
{\sf  dimensionNames}; имена атрибутов ячеек массива~--- переменной 
\mbox{{\sf cellAttributeNames}}; принадлежность измерений массивам~--- переменной 
\mbox{{\sf arrayDimensions}}; принадлежность атрибутов ячеек~--- переменной 
\mbox{{\sf arrayCellAttributes}}; 
тип атрибута ячейки~--- переменной \mbox{{\sf cellAtrributeType}}; 
атрибуты ячеек массивов, которые могут принимать неопределенные 
значения,~--- переменной \mbox{{\sf nullable}}; верхние (нижние) границы измерений~--- 
переменной \mbox{{\sf dimLowerBound}} (\mbox{{\sf dimHigherBound}}); множества 
идентификаторов ячеек массивов~--- переменной 
\mbox{{\sf cells}}, значения измерений в 
ячейках~--- переменной \mbox{{\sf dimensionValue}}; значения атрибутов ячеек~--- 
переменной \mbox{{\sf integerCellAttributeValue}}. Переменные типизируются в разделе 
\mbox{{\sf INVARIANT}} при помощи частичных и тотальных функций аналогично 
переменным, использующимся для придания семантики объектной модели:
        \begin{verbatim}
INVARIANT
   arrayNames: POW(STRING_Type) &
   dimensionNames: NAT +-> STRING_Type &
   cellAttributeNames: NAT +-> STRING_Type &
   arrayDimensions: arrayNames --> 
   POW(dom(dimensionNames)) &
   arrayCellAttributes: arrayNames --> 
     POW(dom(cellAttributeNames)) &
   cellAtrributeType: 
     dom(cellAttributeNames) --> 
       BuiltInTypes &
   nullable: 
     dom(cellAttributeNames) --> BOOL &
   dimLowerBound: 
     dom(dimensionNames) --> INT &
   dimHigherBound: 
     dom(dimensionNames) +-> INT &
   cells: arrayNames --> POW(NAT) & 
   dimensionValue: 
     NAT*dom(dimensionNames) +-> INT  &
   integerCellAttributeValue: 
     NAT*dom(cellAttributeNames) +-> INT &
\end{verbatim}
        Здесь <<\verb * >>~--- знак декартова произведения множеств.
        
        Дополнительные необходимые свойства переменных состояния 
представлены конъюнктивными компонентами инварианта. Так, любая ячейка 
любого массива однозначно идентифицируется набором значений измерений:
        \begin{verbatim}
!(arr, cell1, cell2).(arr: arrayNames & 
  cell1: cells(arr) &  cell2: cells(arr) &
  !(dim).(dim: arrayDimensions(arr) =>
    dimensionValue(cell1, dim) = 
    dimensionValue(cell2, dim)) =>
    cell1 = cell2)
        \end{verbatim}
        
                \vspace*{-6pt}
        
        Для любой ячейки любого массива определены значения всех измерений 
и значение по крайней мере одного атрибута:
        \begin{verbatim}
!(arr, cell).(arr: arrayNames & 
 cell: cells(arr) =>
  !(dim).(dim: arrayDimensions(arr) => 
   (cell |-> dim): dom(dimensionValue)) &
   #(attr).(attr: arrayCellAttributes(arr) & 
    cellAtrributeType(attr) = Integer & 
    (cell, attr): 
      dom(integerCellAttributeValue)) )
        \end{verbatim}
        
        \vspace*{-6pt}
        
        Аналогично объектной модели рассмотрена единственная операция 
ЯМД~--- операция об\-нов\-ле\-ния {\sf update}:
        \begin{verbatim}
OPERATIONS
update(cls, attr, exp, cond) =
PRE cls: arrayNames & 
 attr: arrayCellAttributes(cls) &
  cellAtrributeType(attr) = Integer &
  exp: INT --> INT & cond: NAT --> BOOL
THEN
  integerCellAttributeValue := 
  integerCellAttributeValue <+
  { yy | yy: (NAT*NAT)*INT &
    #(cell, val).(cell: cells(cls) & 
     val: INT & 
    yy = ((cell |-> attr)|-> val) &
    (cond(integerCellAttributeValue(cell, 
     attr)) = TRUE =>
      val = 
       exp(integerCellAttributeValue(cell,
        attr))) &
      (cond(integerCellAttributeValue(cell, 
       attr))= FALSE  =>
    val = 
     integerCellAttributeValue(cell,attr)))}
END   
END
        \end{verbatim}
        
                \vspace*{-6pt}
        
        Сигнатура операции совпадает с сигнатурой операции объектной 
модели. Семантика операции также аналогична: значение~{\sf v} атрибута {\sf attr} 
массива {\sf cls} заменяется на {\sf exp(v)}, если значение {\sf cond(v)} есть 
<<истина>>, и не изменяется в противном случае. 
        
        Заметим, что в данной спецификации для прос\-то\-ты не рассмотрены 
некоторые черты ADM, например нецелочисленные измерения.
        
        \smallskip
        
        Для формального доказательства того, что машина {\sf ArrayDM} уточняет 
машину {\sf ObjectDM}, необходимо построить {инвариант уточнения}, 
связы\-вающий переменные машин, и добавить его к\linebreak инварианту уточняющей 
машины. 
        
        Инвариант формализует принципы отображения ЯОД, изложенные в 
подразд.~2.1, и объединяет их в одну конъюнкцию.
        
        Так, множество имен массивов совпадает с множеством имен классов:
        \begin{verbatim}
classNames = arrayNames
\end{verbatim}

%                \vspace*{-6pt}
        
        Множество идентификаторов и имен измерений и атрибутов ячеек 
совпадает с множеством идентификаторов и имен атрибутов типов экземпляров 
классов:
        \begin{verbatim}
attributeNames = 
  dimensionNames \/ cellAttributeNames
\end{verbatim}

%                \vspace*{-6pt}

        Любому измерению любого массива соответствует атрибут типа 
экземпляра класса, соответствующего этому массиву:
        \begin{verbatim}
!(arr, dim).(arr: arrayNames & 
  dim: arrayDimensions(arr) =>
    #(attr).(attr: 
     typeAttributes(instanceType(arr)) &
          attr = dim & 
          attributeType(attr) = Integer) )s
        \end{verbatim}
        
                        \vspace*{-6pt}
        
        Любому атрибуту ячейки любого массива соответствует атрибут типа 
экземпляра класса, соответствующего этому массиву, и типы атрибутов 
совпадают:
        \begin{verbatim}
!(arr, cattr).(arr: arrayNames & 
   cattr: arrayCellAttributes(arr) =>
    #(attr).(attr: 
      typeAttributes(instanceType(arr)) & 
         attr = cattr & 
         attributeType(attr) = 
           attributeType(cattr)))
        \end{verbatim}
        
                        \vspace*{-9pt}
        
        Атрибут ячейки массива, который может принимать неопределенные 
значения, соответствует определенному ({\sf obligatory}) атрибуту типа:
        \begin{verbatim}
!(arr, cattr).(arr: arrayNames & 
  cattr /: dom(nullable) &
    cattr: arrayCellAttributes(arr) => 
    cattr: obligatory(instanceType(arr)) )
        \end{verbatim}
        
\vspace*{-9pt}

           Здесь знак <<\verb /: >> обозначает отношение непринадлежности элемента 
множеству.
        
        Измерения соответствуют уникальным атрибутам типов:
        \begin{verbatim}
!(arr, dim).(arr: arrayNames & 
    dim: arrayDimensions(arr) => 
      dim: unique(instanceType(arr)) )
        \end{verbatim}
        
                        \vspace*{-6pt}
        
        Верхние (нижние) границы измерений равны верхним (нижним) 
границам соответствующих атрибутов типов:
        \begin{verbatim}
!(dim).(dim: dom(dimLowerBound) =>
    dim: dom(intAttributeLowerBound) & 
    dimLowerBound(dim) = 
      intAttributeLowerBound(dim))
        \end{verbatim}
        
                        \vspace*{-6pt}
        
        Непустые ячейки массивов соответствуют объектам классов:
        \begin{verbatim}
cells = objectsOfClass
\end{verbatim}

%                \vspace*{-6pt}

        Для любой ячейки значения ее измерений и определенных атрибутов 
совпадают со значениями соответствующих атрибутов объекта, 
соответствующего ячейке:
        \begin{verbatim}
!(cell, dim).(cell: NAT & dim: NAT & 
  (cell |-> dim): dom(dimensionValue) =>
  cell: dom(integerAttributeValue(dim)) &
  dimensionValue(cell, dim) = 
    integerAttributeValue(dim)(cell)) &
!(cell, cattr).(cell: NAT & cattr: NAT & 
   (cell |-> cattr): 
   dom(integerCellAttributeValue) =>
   cell: dom(integerAttributeValue(cattr)) &
   integerCellAttributeValue(cell, cattr) =
     integerAttributeValue(cattr)(cell) )
        \end{verbatim}
        
                        \vspace*{-6pt}
        
        Для указания того, что машина {\sf ArrayDM} уточняет машину 
{\sf ObjectDM}, в машину {\sf ArrayDM} была добавлена директива
        \begin{verbatim}
REFINES ObjectDM
\end{verbatim}

%                \vspace*{-6pt}

        Спецификации {\sf ObjectDM} и {\sf ArrayDM} вместе с инвариантом 
уточнения были загружены в инструментальное средство 
        Atelier~B~\cite{22-stu}. Автоматически были сгенерированы теоремы, 
выражающие уточнение спецификаций. В~частности, для операции {\sf update} 
были сгенерированы 10~тео\-рем. Три из них были доказаны автоматически, 
для доказательства остальных необходимо применять интерактивные средства 
доказательства.

\vspace*{-9pt}
  
\section{Родственные исследования и~направления дальнейшей 
работы}

\vspace*{-2pt}

        Родственными данной работе следует считать исследования, связанные с 
отображением моделей, основанных на многомерных массивах, в реляционную 
модель данных. Обычно они нацелены на реализацию многомерных массивов 
при помощи реляционных СУБД. Такие работы начались одновременно с 
исследованиями моделей, основанных на многомерных массивах~\cite{5-stu}, и 
продолжаются в настоящее время~\cite{23-stu}.
        
        Основные особенности данной работы состоят в следующем. 
В~качестве исходной модели при отображении используется специфическая 
модель, основанная на многомерных массивах СУБД SciDB, язык которой 
представляет собой комбинацию декларативного SQL-по\-доб\-но\-го языка и 
функционального языка, включающего специфические\linebreak операции над 
многомерными массивами. В~качестве целевой модели используется объектная 
модель с Datalog-по\-доб\-ным языком запросов (программ)~--- язык СИНТЕЗ. 
Для отображения\linebreak обеспечивается формальное доказательство сохранения 
информации и семантики операций ЯМД.
        
        Отметим, что результаты работы могут быть с легкостью обобщены и 
использованы при интеграции в системах, использующих каноническую 
модель, отличную от языка СИНТЕЗ, например другую объектную (ODMG) 
или объект\-но-ре\-ля\-ци\-он\-ную модель (SQL:2003). Результаты также могут 
быть использованы для интеграции ресурсов, представленных в модели, 
основанной на многомерных массивах, но отличной от ADM.
        
        Некоторые вопросы отображения требуют дальнейших исследований. 
Например, следует ли иметь в канонической модели при интеграции 
        масс\-сив-ори\-ен\-ти\-ро\-ван\-ных моделей данных операции, 
связанные с размером порции (chunk size) данных в БД~\cite{9-stu}?
        
        Дальнейшую работу можно разбить на два этапа:
        \begin{enumerate}[(1)]
\item расширение инструментальных средств поддержки предметных 
посредников для виртуальной интеграции SciDB-ресурсов: 
\begin{itemize}
\item[(а)] расширение средств регистрации ресурсов в посреднике~\cite{10-stu} 
трансформацией ЯОД\ ADM в каноническую модель; 
\item[(б)] создание 
SciDB-адап\-те\-ра~--- специального ПО, связывающего исполнительную 
среду посредников с SciDB-ресурсами (составной частью адаптера является 
разработанная трансформация ЯМД);
\end{itemize}
\item применение технологии предметных посредников для решения 
научных задач в некоторой предметной области над множеством\linebreak 
неоднородных ресурсов, включающим SciDB-ре\-сурсы.
\end{enumerate}

\bigskip
        Автор выражает благодарность Л.\,А.~Калиниченко, П.\,Е.~Велихову и 
А.\,Е.~Вовченко за полезные замечания, высказанные в ходе обсуждения 
данной работы на семинарах ИПИ РАН.

\vspace*{-6pt}

{\small\frenchspacing
{%\baselineskip=10.8pt
\addcontentsline{toc}{section}{Литература}
\begin{thebibliography}{99}

\vspace*{-2pt}

\bibitem{1-stu} %1
Challenges and opportunities with big data~// A~community white paper developed 
by leading researchers across the United States, 2012. {\sf http://cra.org/ccc/docs/ init/bigdatawhitepaper.pdf}. 

\bibitem{1-2-stu} %2
\Au{Abrial J.-R.} The B-Book: Assigning programs to 
meanings.~--- Cambridge: Cambridge University Press, 1996. 

\bibitem{2-stu} %3
\Au{Vassiliadis P., Sellis T.\,K.} A~survey of logical models for OLAP databases~// SIGMOD 
Record, 1999. Vol.~28. No.\,4. P.~64--69. 

\bibitem{3-stu}
\Au{Pedersen T.\,B., Jensen C.\,S.} Multidimensional database technology~// IEEE Computer, 
2001. Vol.~34. No.\,12. P.~40--46. 

\bibitem{4-stu} %5
\Au{Libkin L., Machlin R., Wong~L.} A~query language for multidimensional arrays: Design, 
implementation, and optimization techniques.~--- SIGMOD, 1996. P.~228--239. 
\bibitem{5-stu} %6
\Au{Baumann P.} A~database array algebra for spatio-temporal data and beyond~// Next 
generation information technologies and systems. Lectures notes in computer science ser.
Springer Verlag KG, 1999. Vol.~1649. P.~76--93.
\bibitem{6-stu} %7
Overview of SciDB: Large scale array storage, processing and analysis. The SciDB development 
team.~--- SIGMOD, 2010. 
\bibitem{7-stu}
Large synoptic survey telescope. {\sf http://www.lsst.org}. 
\bibitem{8-stu}
\Au{Becla J., Lim K.-T.} Report from the First Workshop on Extremely Large Databases~// Data 
Sci.~J., 2008. Vol.~7.
\bibitem{9-stu}
SciDB User's Guide. Version~12.3, 2012. {\sf http:// www.scidb.org}.
\bibitem{10-stu}
\Au{Kalinichenko L.\,A., Briukhov D.\,O., Martynov~D.\,O., Skvortsov~N.\,A., Stupnikov~S.\,A.} 
Mediation framework for enterprise information system infrastructures~// Volume Databases and 
Information Systems Integration: 9th Conference (International) on Enterprise Information 
Systems (ICEIS 2007) Proceedings ~--- Funchal, 2007. P.~246--251.
\bibitem{11-stu}
\Au{Захаров В.\,Н., Калиниченко Л.\,А., Соколов~И.\,А., Ступников~С.\,А.} Конструирование 
канонических информационных моделей для интегрированных информационных 
сис\-тем~// Информатика и её применения, 2007. Т.~1. Вып.~2. C.~15--38. 
\bibitem{12-stu}
\Au{Kalinichenko L.\,A., Stupnikov S.\,A.} Heterogeneous information model unification as a 
prerequisite to resource schema mapping~// Information Systems: People, Organizations, 
Institutions, and Technologies: 5th Conference of the Italian Chapter of Association for 
Information Systems itAIS Proceedings.~--- Berlin--Heidelberg: Springer Physica Verlag, 2010. 
P.~373--380. 
\bibitem{13-stu}
\Au{Kalinichenko L.\,A., Stupnikov S.\,A., Martynov~D.\,O.} SYNTHESIS: A~language for 
canonical information modeling and mediator definition for problem solving in heterogeneous 
information resource environments.~--- Moscow: IPI RAN, 2007. 171~p. 
\bibitem{14-stu}
\Au{Брюхов Д.\,О., Вовченко А.\,Е., Захаров~В.\,Н., Желенкова~О.\,П., Калиниченко~Л.\,А., 
Мартынов~Д.\,О., Скворцов~Н.\,А., Ступников~С.\,А.} Архитектура промежуточного слоя 
предметных посредников для решения \mbox{задач} над множеством интегрируемых 
неоднородных распределенных информационных ресурсов в гиб\-рид\-ной 
грид-ин\-фра\-струк\-ту\-ре виртуальных обсерваторий~// Информатика и её применения, 
2008. Т.~2. Вып.~1. С.~2--34. 

\bibitem{15-stu} %16
\Au{Kersten M.\,L., Zhang~Y., Ivanova~M., Nes~N.} SciQL, a query language for science 
applications~// EDBT/ICDT~--- Workshop on Array Databases 2011 Proceedings.~--- Uppsala, 
Sweden, 2011. P.~1--12.

\bibitem{16-stu} %17
\Au{Abrial J.-R.} The B-Book: Assigning programs to meanings.~--- Cambridge: Cambridge 
University Press, 1996.
\bibitem{17-stu} %18
Astronomy in ArrayDB. 
{\sf http://trac.scidb.org/\linebreak raw-attachment/wiki/UseCases/Astronomy\%20in\%20\linebreak
ArrayDB.pdf }
\bibitem{18-stu} %19
ATL Project. {\sf http://www.eclipse.org/m2m/atl}.
\bibitem{19-stu} %20
\Au{Budinsky F., Steinberg D., Ellersick~R., Grose~T.}
Eclipse modeling framework. Ch.~5: Ecore modeling concepts.~--- Addison Wesley 
Professional, 2004.
\bibitem{20-stu} %21
Meta Object Facility (MOF) 2.0 Core Specification, 2003. 
{\sf http://www.omg.org/cgi-bin/apps/doc?ptc/\linebreak 03-10-04.pdf}. 
\bibitem{21-stu} %22
\Au{Kalinichenko L.\,A.} Method for data models integration in the common paradigm~//  1st 
East-European Symposium on Advances in Databases and Information Systems \mbox{ADBIS'97} 
Proceedings.~--- St.-Petersburg: Nevsky Dialect, 1997. Vol.~1: Regular papers. P.~275--284.
\bibitem{22-stu}
Atelier~B: The industrial tool to efficiently deploy the B Method. 
{\sf http://www.atelierb.eu/index-en.php}.

\label{end\stat}

\bibitem{23-stu} %24
\Au{Van Ballegooij A.} RAM: Array database management through relational mapping~// SIKS 
Dissertation ser. No.\,2009-25. {\sf http://oai.cwi.nl/oai/asset/14074/ 14074D.pdf}.
         
\end{thebibliography}
} }

\end{multicols} %3
\def\stat{shkotin}

\def\tit{ИССЛЕДОВАНИЕ ГРАФА КАТЕГОРИЙ АНГЛИЙСКОЙ ВЕРСИИ ВИКИПЕДИИ$^*$}

\def\titkol{Исследование графа категорий английской версии Википедии}

\def\autkol{А.\,В.~Шкотин}

\def\aut{А.\,В.~Шкотин$^1$}

\titel{\tit}{\aut}{\autkol}{\titkol}

{\renewcommand{\thefootnote}{\fnsymbol{footnote}}\footnotetext[1] {Статья 
рекомендована к публикации в журнале Программным комитетом конференции 
<<Электронные библиотеки: перспективные методы и технологии, электронные 
коллекции>> (RCDL-2012).}}

\renewcommand{\thefootnote}{\arabic{footnote}}
\footnotetext[1]{Государственный геологический музей им.\ В.\,И.~Вернадского 
Российской академии наук, отдел ГИС, ashkotin@acm.org}

\vspace*{6pt}


\Abst{Википедия является выдающимся проектом по накоплению знаний как общего 
пользования, так и различных областей специализации. Проверка качества этих знаний, 
особенно автоматическая, чрезвычайно важна. В работе представлены результаты изучения 
строения английской версии ГКВ (орграфа категориальных статей Википедии). Являясь по 
своей идее системой тем, он поддерживает систематизацию знаний, и представляет интерес, 
из чего эта систематизация состоит и как она устроена. Показано, что в графе есть 
неприемлемые логические нарушения, и обсуждаются организационные и технические 
методы их устранения.}

\vspace*{4pt}

\KW{Википедия; орграф; связные компоненты; логический анализ}

\vspace*{4pt}


\vskip 14pt plus 9pt minus 6pt

      \thispagestyle{headings}

      \begin{multicols}{2}

            \label{st\stat}
            
\section{Введение}

Орграф категориальных статей Википедии~[1] есть подграф графа, в котором статьи 
Википедии приписаны категориальным статьям. Выделение ГКВ из этого полного 
графа есть первая техническая задача. Важно, что далее изучается строение ГКВ 
на некоторый момент времени и в нем есть незавершенная, <<строящаяся>> часть. 
Поэтому выводы надо делать с осторожностью. Естественно ввести термин <<точка 
роста>>, когда натыкаешься в ГКВ на часть, которая еще не завершена. Дамп 
полного графа получен из ИСП РАН в виде двух текстовых файлов: файла 
отображения номера страницы Википедии в номер категориальной страницы, что 
приписывает страницу категории, и файла, в котором номеру страницы Википедии 
приписано ее наименование. Математически ГКВ есть орграф, каждый узел которого 
взаимно однозначно соответствует категориальной странице и помечен ее номером. 
Стрелка (дуга) из узла~$N_1$ в узел~$N_2$ идет тогда и только тогда, когда 
страница с номером~$N_1$ есть подкатегория страницы с номером~$N_2$. Всего 
таких стрелок 1\,221\,133.
   
   Множество узлов ГКВ (593\,796), как и любого произвольного графа, распадается 
на два подмножества: изолированные узлы (26\,272) и узлы, связанные стрелками (567\,24). 
Изолированная категория~--- это, скорее всего, <<точка роста>>: на момент снятия 
дампа она уже есть, но стрелок еще нет.
   
   Далее анализируется только <<граф стрелок>>, т.\,е.\ все характеристики даны без учета 
изолированных узлов. Состав изолированных узлов можно по\-смот\-реть в отчете~[2] 
   (далее~--- отчет) в таблице, указанной во введении. Состав и характеристики узлов со 
стрелками можно посмотреть в таблице, указанной там же, равно как и граф стрелок. 
Важный вопрос~--- количество связных компонент графа, так как в дальнейшем их строение 
можно изучать отдельно. Таких компонент оказалось~1987. Изолированные узлы при этом 
учитываются отдельно. Алгоритм разбиения описан в отчете~[3]. Впрочем, проще 
воспользоваться пакетом программ, например Pajek~[4], который умеет разбивать узлы 
графа на слабо связные компоненты.
   
   Первые 10 самых больших компонент указаны в табл.~1,
где $C_n$~--- уникальный номер компоненты, присвоенный при разбиении. Конечно, в 
случае с Википедией малые компоненты~--- это точки роста. Петель ($N_1\hm\rightarrow 
N_1$) в графе нет. 
   
  \begin{center}  %tabl1
% \vspace*{6pt}
\parbox{100pt}{{\tablename~1}\ \ \small{Объем связных компонент}}

\vspace*{6pt}

{\small   
%\tabcolsep=12pt
\begin{tabular}{|r|c|}
   \hline
\multicolumn{1}{|c|}{$C_n$}&\multicolumn{1}{c|}{Количество}\\
\hline
1&561\,636\hphantom{9999}\\
21\,727&210\hphantom{9}\\
14\,332&36\\
\hphantom{9}2\,863&29\\
20\,842&27\\
\hphantom{9}6\,680&20\\
19\,212&19\\
20\,868&19\\
13\,325&17\\
13\,287&16\\
\hline
\end{tabular}
}
\end{center}

%\vspace*{15pt}

\addtocounter{table}{1}

   
   



\begin{figure*}[b] %fig1
\vspace*{-6pt}
\vspace*{9pt}
 \begin{center}
 \mbox{%
 \epsfxsize=124.212mm
 \epsfbox{shk-ch-L.eps}
 }
 \end{center}
 \vspace*{-6pt}
 \Caption{Состав используемых символов }
 \end{figure*}
   
   Источников (узлов, в которые нет входящих стрелок)~--- 345\,597. Это категории нижнего 
уровня. Стоков (узлов, из которых нет исходящих стрелок)~--- 11\,767. Это 
категории верхнего уровня дампа и, скорее всего, точки роста. Промежуточных 
узлов, соответственно, 210\,160.
   
   Максимальное количество исходящих из одного узла стрелок~--- 85. Речь идет о 
промежуточном узле №\,690451 с заголовком Category:World War~II, т.\,е.\ эта категория 
приписана 85~надкатегориям. Максимальное количество входящих стрелок (12\,625) имеет 
промежуточный узел №\,692309 с про\-яс\-ня\-ющим заголовком Category:Albums by artist.

\vspace*{-6pt}

\section{Анализ заголовков}

\vspace*{-2pt}

 Заголовки всех узлов категорий (включая изолированные) можно посмотреть в отчете в 
таблице, указанной в разделе <<Анализ заголовков>>. Таблица содержит 584\,606~узлов. 
Следовательно, 9190~узлов ГКВ не имеют заголовков. Они ждут своего исследователя. 
Анализ текстов заголовков, даже безотносительно к их подчинению,~--- отдельная 
увлекательная задача. Но начать надо с использованного состава букв.

\vspace*{-6pt}

\subsection{Алфавит}

   Рассмотрим состав букв (characters), упо\-треб\-лен\-ных при именовании категориальных 
статей. Текстовый файл (UTF-8), содержащий состав алфавита можно посмотреть в 
прикреплении {\sf cat2title.abc0.txt}  к отчету. Как разделитель букв используется знак 
<<$\vert$>>. %Вот этот алфавит:
Этот алфавит представлен на рис.~1.
В приложении {\sf id2title.abc.txt} к отчету можно найти впечатляющее разнообразие 
букв заголовков всех статей Википедии.

\vspace*{-6pt}

\subsection{Термины в~заголовках}

   Это может стать отдельным важным исследованием. Например, количество заголовков, в 
которых встречается слово album,~--- 17\,591. 

\vspace*{-6pt}

\section{Стоки}

\vspace*{-2pt}

   В приложении~1 к отчету можно посмотреть начало таблицы стоков с самым большим 
количеством входящих стрелок.
   
   В приложении~2 к отчету можно посмотреть путь-рекордсмен, предоставленный Антоном 
Коршуновым из ИСП РАН.
   
   Самый длинный путь~--- 294~вершины. Его начальная категория~--- №\,5760285 
Category:Anastacia songs, а конечная~--- №\,691484 Category:Music. 

\section{Строение орграфа категориальных статей Википедии в~целом}

   В работе~[5, с.~9] указывается, что в ГКВ есть циклы. По идее, циклы~--- это аномалии на 
графе подчинения категорий, они долж\-ны занимать малую его часть. Назовем для краткости 
объединение орциклов графа и орпутей между циклами~--- \textit{ядром}, а дополнительную 
часть графа~--- \textit{мантией}. Стрелки же между мантией и ядром назовем 
\textit{связующими}. Таким образом, в целом граф состоит из ядра, мантии и связующих 
стрелок, часть из которых идет из ядра в мантию, а часть~--- из мантии в ядро. Чтобы 
выделить ядро, был применен следующий алгоритм: 
   \begin{enumerate}[(1)]
   \item находим в графе источники и стоки и удаляем их; 
   \item если в графе не осталось узлов, то стоп (ядра нет);
   \item если есть источники или стоки, то идти на~(1). Если нет, то
   стоп (в графе осталось только ядро).
   \end{enumerate}
   
\section{Ядро орграфа категориальных статей Википедии}
   
   В строении ядра важно, что пути между циклами, сами не входящие в циклы, составляют 
самостоятельную интересную часть ядра. 

\subsection{Состав ядра}

   Количество стрелок в ядре~--- 38\,538. Узлов же~--- 13\,545. Граф ядра опубликован в 
таблице, указанной в разделе <<Состав ядра>> отчета. Далее было традиционно выполнено 
<<расщепление>> ядра на связные компоненты. Оказалось, что имеются одна большая 
компонента~--- 13\,507~узлов и~еще 19~пар узлов. Характеристики узлов ядра, включая 
разбиение на связные компоненты, можно посмотреть в таблице, указанной в разделе 
<<Состав ядра>> отчета.
   
   Рассмотрим компоненту №\,764 ядра. Это пример пары, которая является даже связной 
компонентой не только ядра, а самого ГКВ. В компоненте два узла:
\begin{enumerate}[(1)]   
\item
№\,28736601, Category:Wikipedia sockpuppets of ShantanuSingh198 ;
\item
№\,28736686, Category:Suspected Wikipedia sockpuppets of ShantanuSingh19.
\end{enumerate}
   
      В Википедии они также ссылаются только друг на друга.
      
      \vspace*{12pt}
   
   \textbf{Анализ.} Что бы ни обозначало <<Wikipedia sockpuppets of ShantanuSingh198>>, 
очевидно, что нечто, под него подпадающее (как под понятие), не может быть 
одновременно лишь <<подозреваемым>> на подпадание. Равно как и наоборот, 
т.\,е.\ логически эти две категории не пересекаются и~обе стрелки должны быть 
удалены. Отношения же между ними, например, на OWL~2~\cite{6-shk} должно было 
бы быть таким:\\

\noindent
 {\sf 
DisjointClasses(wcg:Wikipedia\_sockpuppets\_of\_}

\noindent{\sf ShantanuSingh198}, 

\noindent
{\sf wcg:Suspected\_Wikipedia\_sockpuppets\_of\_}

\noindent
{\sf ShantanuSingh198)}\\

   При этом правильнее ссылаться в обеих статьях друг на друга через тег Википедии <<See 
also>>.

\subsection{Сильно связные компоненты ядра}

   В ядре представляет интерес зацикливание отношения подкатегория--над\-ка\-те\-го\-рия. Тут 
есть два подхода:
   \begin{enumerate}[(1)]
   \item общий~--- применить алгоритм поиска сильно связных компонент (ССК);
   \item частный~--- найти так называемые <<линзы>>~--- два узла, ссылающихся друг на 
друга (как под\-ка\-те\-го\-рия--надкатегория). 
   \end{enumerate}
   
   Второй путь вполне приемлем для ГКВ, так как, по идее, в нем вообще не должно быть 
циклов. Впрочем, как в отношении линз, так и в отношении циклов большей длины следует 
заметить, что они математически утверждают эквивалентность соответствующих терминов, 
т.\,е.\ синонимию, что в принципе возможно. Но конкретно в Википедии возможна реализация
через {\sf redirect}. Интуитивно же в большинстве случаев обнаруживается ошибка, 
т.\,е.\ ка\-кие-то стрелки цикла ошибочны.
   
   Чтобы получить состав сильно связных компонент ядра, была использована программа 
Pajek~[4]. Заметим, что петель в ГКВ нет, а поэтому узлы ядра, не попавшие в ССК,~--- это 
узлы на путях между циклами (см.\ выше).
   
   Сильно связных компонент оказалось 457. Узлов, не входящих в ССК, так сказать, 
   связующих ядра,~--- 7646. 
Есть одна гигантская по сравнению с остальными ССК~--- в ней 3967 узлов.
   
   В отчете в разделе <<Сильно связные компоненты ядра>> приведена таблица самых 
больших ССК. Рассмотрим для примера компоненту №\,41, у которой всего 9~узлов 
(табл.~2, рис.~2).

   
   Если номер накладывается на стрелку, то под ним наконечника (треугольничка) нет. Это 
важно, так как Pajek рисует <<линзы>> ($\mathrm{У}_1\rightarrow \mathrm{У}_2\rightarrow 
\mathrm{У}_1$) как одну стрелку с наконечниками на обоих окончани-\linebreak\vspace*{-12pt}

\vspace*{6pt}


  \begin{center}  %tabl2
% \vspace*{6pt}
{\tablename~2}\ \ \small{Заголовки узлов ССК №\,41}

\vspace*{6pt}

{\small   
\tabcolsep=3.5pt
\begin{tabular}{|c|l|}
   \hline
Номер узла&\multicolumn{1}{c|}{Заголовок}\\
\hline
717\,227&Category:Orthodox rabbis\\
717\,302&Category:Talmud rabbis\\
799\,461&Category:Mishnah\\
799\,587&Category:Talmud\\
6\,110\,893\hphantom{\,9}&Category:Talmudists\\
8\,398\,752\hphantom{\,9}&Category:Talmud people\\
11\,334\,178\hphantom{\,99}&Category:Rabbinic literature\\
15\,249\,105\hphantom{\,99}&Category:Talmud concepts and terminology\\
26\,795\,615\hphantom{\,99}&Category:Chazal\\
\hline
\end{tabular}
}
\end{center}

\vspace*{3pt}

\addtocounter{table}{1}

 \begin{center}  %fig2
 \mbox{%
 \epsfxsize=67.079mm
 \epsfbox{shk-1.eps}
 }
 
 \vspace*{6pt}
{\figurename~2}\ \ \small{Рисунок графа ССК №\,41}
\end{center}


\addtocounter{figure}{1}

\noindent
ях. В~данной ССК 
(см.\ рис.~2) линза всего одна~--- слева внизу вертикально.
   
 
\subsection{Линзы}

   Линза~--- это два узла таких, что $\mathrm{У}_1\hm\rightarrow \mathrm{У}_2$ и 
$\mathrm{У}_2 \hm\rightarrow \mathrm{У}_1$. Она может быть отдельной ССК, а может 
входить в ССК как часть.
   
   В ядре оказалось 1269~линз. Из них 1260 имеют заголовки для обоих узлов. Их можно 
посмотреть в таблице, указанной в разделе <<Линзы>> отчета.

\section{Мантия~--- ациклическая часть орграфа категориальных статей Википедии}

   Чтобы получить мантию, удалим из ГКВ ядро. При этом оказывается, что часть 
источников и стоков стали изолированными. В~первом случае все исходящие из них стрелки 
попали в ядро, во втором~--- все входящие в них стрелки шли из ядра. Изолировавшихся 
источников~--- 14\,421, сто-\linebreak ков~---~60.
{ %\looseness=-1

}
   
   Кроме того, в мантии появляются ложные вершины (пики). Это те ее узлы, которые стали 
стоками после удаления ядра, а вообще-то имели исходящие стрелки, которые все попадали 
в ядро. Таких вершин 18\,157. Причем максимальная высота~--- 28. Для сравнения, стоков 
ГКВ, получивших уровень, т.\,е.\ неизолированных~--- 11\,707, максимальная высота~---~24.
   
   Ложная вершина-рекордсмен (высоты~28) имеет №\,15\,715\,670, а заголовок~--- 
Category:Creation myths. 
   
   \medskip
   
   \noindent
   \textbf{Замечание.} Конечно, ГКВ можно представить и в виде 
   <<гал\-сту\-ка-ба\-боч\-ки>>, как в работе~\cite{7-shk}, где орграф был использован для 
представления схемы связей между транснациональными корпорациями. Но в данном случае 
нагляднее сравнение с горами~--- вверх к более обширным темам, горами, в которых есть 
ядро из 20~связных компонент, одна из которых большая, а 19~--- линзы.
   
   Число узлов на уровнях показано в табл.~3 и оправдывает сравнение с горами.
   В  строке NULL указано количество изолированных узлов мантии, 
а в строке~0~--- количество узлов в ядре.

\subsection*{Связующие стрелки}

   Между мантией и ядром есть стрелки~--- связу\-ющие. Стрелок из ядра в мантию~--- 591. 
Стрелок из мантии в ядро~--- 210\,514.
   
     \begin{center}  %tabl2
% \vspace*{6pt}
\parbox{120pt}{{\tablename~3}\ \ \small{Распределение узлов по уровням}}

\vspace*{6pt}

{\small   
\tabcolsep=3.5pt
   \begin{tabular}{|c|c|}
   \hline
   Уровень & Количество узлов\\
   \hline
   NULL & 14\,481\hphantom{99}\\
   28&\hphantom{99}1\\
   27 &\hphantom{99}2\\
   26&\hphantom{99}3\\
   25&\hphantom{99}3\\
   24&\hphantom{99}5\\
   23&\hphantom{99}7\\
   22&\hphantom{9}12\\
   21&\hphantom{9}16\\
   20&\hphantom{9}20\\
   19&\hphantom{9}30\\
   18&\hphantom{9}50\\
   17&\hphantom{9}57\\
   16&\hphantom{9}71\\
   15&100\\
   14&149\\
   13&226\\
   12&425\\
   11&697\\
   10&1\,187\hphantom{9}\\
      \hphantom{9}9&1\,915\hphantom{9}\\
      \hphantom{9}8&3\,103\hphantom{9}\\
      \hphantom{9}7&4\,858\hphantom{9}\\
      \hphantom{9}6&7\,754\hphantom{9}\\
      \hphantom{9}5&13\,019\hphantom{99}\\
      \hphantom{9}4&23\,302\hphantom{99}\\
      \hphantom{9}3&45\,323\hphantom{99}\\
      \hphantom{9}2&105\,958\hphantom{999}\\
      \hphantom{9}1& 331\,205\hphantom{999}\\
   \hphantom{9}0&13\,545\hphantom{99}\\
   \hline
   \end{tabular}
}
\end{center}

%\vspace*{15pt}

\addtocounter{table}{1}
   


\section{Обсуждение}

   \noindent
   \begin{enumerate}[1.]
   \item Влияние ядра на строение мантии оказывается существенным. Так, ложный сток 
имеет высоту~28 при максимальной высоте настоящего стока 24. Такое может случиться, 
только если ядро находится на <<вершине>> мантии. Кроме того, появилось 29~ложных 
источников~--- в них шли стрелки только из ядра. Количество ложных стоков~--- 18\,157. 
При этом ядро состоит всего лишь из одной большой связной компоненты и 19~линз.
   \item Хотя <<физически>> отдельное изучении мантии оправдано, для совокупного 
строения графа лучше не удалять ядро из ГКВ, а свернуть его ССК в <<тяжелые>> узлы, 
пометив их количеством узлов в ССК. Такой тяжелый узел наследует все внешние ССК 
стрелки, а образовавшиеся петли внутренних стрелок стоит удалить. Тогда получится 
ациклический граф, так как ССК не могут образовывать цикл. Уровни такого графа с 
указанием распределения по ним тяжелых узлов дадут более реалистичную картину ГКВ. 
При этом есть все основания полагать, что по крайней мере одна из тяжелых вершин будет 
стоком.
   \item В растущем графе, которым по преимуществу является ГКВ, <<точки роста>> 
(изолированные узлы; все связные компоненты, кроме главной; стоки вне главной 
связной компоненты) не представляют 
особого интереса. Поэтому стоит сразу выделить главную связную компоненту 
и изучать только ее.
\end{enumerate}

\section{Другие способы исследования}

   Можно напрямую изучать {\sf http://dbpedia.org}\linebreak через точку входа для SPARQL: 
      {\sf 
http://dbpedia.org/ sparql}. Привязка к категории идет через свойство {\sf 
http://purl.org/dc/terms/subject}.
   
   Вот пример запроса, который начинает выдавать полный граф связи страниц и категорий:\\[-8pt]
   
   
{\sf select ?x ?z where \{?x dcterms:subject ?z\}}\\[-8pt]
   
   Надо только поставить timeout, например, 1000.
   
   Запрос, выдающий отношение <<{\sf x} is a sub-category of {\sf z}>> (см.\ с.~5 
<<Categories>>~\cite{5-shk}):\\[-8pt]

{\sf select ?x ?z where \{?x skos:broader ?z\}}\\[-8pt]
   
   А вот запрос, выдающий <<линзы>>:\\[-8pt]

{\sf select ?x ?z where \{?x skos:broader ?z. ?z }

{\sf skos:broader ?x.\}}\\[-8pt]
   
   Вот узлы первой:\\[-8pt] 

{\sf http://dbpedia.org/resource/Category:Political\_}

{\sf philosophers;}


   
{\sf http://dbpedia.org/resource/Category:Political\_}

{\sf theorists.}\\[-8pt]
   
   Она действительно есть в Википедии.
   
   А всего запрос выдает 2000 линз, что, наверное, не предел.
   
\section{Заключение}

   Естественно считать, что ГКВ должен быть ацик\-ли\-че\-ским графом. Таким образом, 
исследование показало, что аномалии значительны.
   
   Можно создать средства, которые, обнаруживая аномалию, например линзу, будут 
размещать на соответствующих страницах в Discussion уведомление о возможном 
логическом противоречии.
   
   Основных вопросов два.
   \begin{enumerate}[1.]
   \item Как к такому подходу отнесутся авторы страниц категорий? Это можно проверить 
экспериментально.
   \item Как к логическим противоречиям относятся идеологи Википедии? Те, кто задает 
правила классификации? Судя по всему, индифферентно.
   \end{enumerate}
   
   Общая рекомендация: многие отношения между категориями, попавшие в 
   <<sub-category of>>, следует перенести в <<See also>>.
   
   Оценить предстоящую работу можно так: для начала надо разобраться с 1269~линзами. 
Они значительно убавят размер ССК.
   
   Только если это нужно википедистам, можно было бы продолжить работу в следующих 
на\-прав\-ле\-ни\-ях:
   \begin{itemize}
   \item исследовать длинные пути;
   \item попытаться представить архитектуру графа в целом (например, применить 
   трехмерную визуализацию);
   \item проанализировать состав и логику связи заголовков (особенно ССК).
   \end{itemize}
   
   Особняком стоит задача получить и проанализировать русский ГКВ. В~проекте DBpedia 
можно получить дамп русской версии, надо только перекодировать буквы с rdf-кодов 
(например, $\backslash$u0432) в UTF-8.
   
{\small\frenchspacing
{%\baselineskip=10.8pt
\addcontentsline{toc}{section}{Литература}
\begin{thebibliography}{9}

\bibitem{1-shk}
\Au{Korshunov A., Turdakov D., Jeong~J., Lee~M., Moon~Ch.} A~category-driven approach to 
deriving domain specific subset of Wikipedia~// SYRCoDIS'11: 7th Spring Researchers 
Colloquium on Databases and Information Systems Proceedings, 2011. P.~43--53.
\bibitem{2-shk}
\Au{Шкотин А.} Исследование графа категорий английской версии Wikipedia. 
Сообщение о результатах первого этапа. 2011. {\sf http://sites.google.com/site/ alex0shkotin/grafy/wikipedia-category-graph}.
\bibitem{3-shk}
\Au{Шкотин А.} Разбиение графа на связные компоненты: Алгоритм и программа. 2011. 
{\sf http://sites.google. com/site/alex0shkotin/grafy/svaznye-komponenty}.
\bibitem{4-shk}
\Au{Batagelj V., Mrvar A.} Pajek: Program for analysis and visualization of large networks: 
Reference manual.~--- Ljubljana: University,  2012.
\bibitem{5-shk}
\Au{Bizer C., Lehmann J., Kobilarov~G., Auer~S., Becker~C., Cyganiak~R., Hellmann~S.}
DBpedia~--- a~crystallization point for the Web of Data~// J.~Web Semantics, 2009. Vol.~7. 
No.\,3. P.~154--165.
\bibitem{6-shk}
OWL 2 Web Ontology Language: Structural specification and functional-style 
syntax: W3C recommendation~/ Eds. B.~Motik, P.\,F.~Patel-Schneider, 
B.~Parsia. 2009. {\sf http:// www.w3.org/TR/owl2-syntax}. 

\label{end\stat}

\bibitem{7-shk}
\Au{Vitali S., Glattfelder J.\,B., Battiston~S.} The network of global corporate control~// 
Cornell University Library (submitted on July~28, 2011 (v1), last revised  Sep.~19, 2011 (this 
version, v2)). {\sf http://arxiv.org/abs/1107.5728}.
\end{thebibliography}
} }

\end{multicols} %4
\def\stat{mashechkin}

\def\tit{МЕТОДЫ АКТИВНОЙ АУТЕНТИФИКАЦИИ НА ОСНОВЕ АНАЛИЗА ДИНАМИКИ РАБОТЫ ПОЛЬЗОВАТЕЛЕЙ 
С~КЛАВИАТУРОЙ}

\def\titkol{Методы активной аутентификации на основе анализа динамики работы пользователей 
с клавиатурой}

\def\autkol{В.\,Ю.~Каганов, А.\,К.~Королёв, М.\,Н.~Крылов и др.}

\def\aut{В.\,Ю.~Каганов$^1$,  А.\,К.~Королёв$^2$, М.\,Н.~Крылов$^3$,
    И.\,В.~Машечкин$^4$,  М.\,И.~Петровский$^5$}

\titel{\tit}{\aut}{\autkol}{\titkol}

%{\renewcommand{\thefootnote}{\fnsymbol{footnote}}\footnotetext[1] {Работа 
%выполнена при поддержке РФФИ (проект 11-07-00402-а). Статья рекомендована к 
%публикации в журнале Программным комитетом конференции <<Электронные 
%библиотеки: перспективные методы и технологии, электронные коллекции>> 
%(RCDL-2012).}}

\renewcommand{\thefootnote}{\arabic{footnote}}

\footnotetext[1]{Московский государственный 
университет им.\ М.\,В.~Ломоносова, факультет вычислительной математики и 
кибернетики, vladhid@mlab.cs.msu.su} 
\footnotetext[2]{Московский 
государственный университет им.\ М.\,В.~Ломоносова, факультет вычислительной 
математики и кибернетики, akorolev@mlab.cs.msu.su} 
\footnotetext[3]{Московский государственный университет им.\ М.\,В.~Ломоносова, 
факультет вычислительной математики и кибернетики, 
krylovm@mlab.cs.msu.su} 
\footnotetext[4]{Московский государственный 
университет им.\ М.\,В.~Ломоносова, факультет вычислительной математики и 
кибернетики, mash@cs.msu.su} 
\footnotetext[5]{Московский 
государственный университет им.\ М.\,В.~Ломоносова, факультет вычислительной 
математики и кибернетики, michael@cs.msu.su}

\vspace*{-6pt}

\Abst{Проведен обзор некоторых эффективных методов аутентификации на основе 
поведенческих моделей пользователей, построенных с использованием данных, 
полученных при анализе работы пользователя с клавиатурой. Также предложен новый 
подход к представлению собираемых данных, проведены эксперименты с 
использованием этого представления и различных алгоритмов машинного обучения.}

\vspace*{-2pt}


\KW{аутентификация; машинное обучение; деревья
решений; клавиатура; потенциальные функции; поведенческий анализ}

            \vspace*{-5pt}


\vskip 14pt plus 9pt minus 6pt

      \thispagestyle{headings}

      \begin{multicols}{2}

            \label{st\stat}
            

            
\section{Введение}

В большинстве современных информационных систем одной из важнейших
задач, помимо сохранения и обработки данных, является задача
разграничения доступа к ресурсам. Это необходимо как для
предотвращения несанкционированного доступа к сис\-те\-мам извне, так и
для разграничения прав сотрудников, работающих с информационной
сис\-те\-мой внутри организации. Поэтому задача аутентификации, т.\,е.\
проверки подлинности пользователя, желающего получить доступ к
системе, является одной из ключевых.

В настоящее время эта задача может быть решена множеством различных
способов. Пожалуй, самый популярный способ из тех, что используются
в современных информационных системах,~--- это аутентификация по
паролю~--- специальной последовательности символов, не известной
никому, кроме пользователя, которому разрешен доступ к системе.
Помимо очевидных преимуществ, таких как простота реализации и
использования, а также распространенность, у нее есть и существенные
недостатки: сама кодовая фраза может быть забыта, передана другим
лицам, а при недостаточной длине или сложности --- подобрана простым
перебором или перебором по словарю, что ставит под сомнение
возможность их использования в системах, требующих высшего уровня
безопасности.

От последнего недостатка свободны системы электронно-цифровых
подписей (ЭЦП), также применяемых для аутентификации. Однако
проб\-ле\-му безопасного хранения закрытых ключей таким образом решить
невозможно.

В связи с вышесказанным значительная часть систем, обеспечивающих
эффективную безопасность, использует биометрию для определения
личности пользователя. Биометрические комплексы могут быть разделены
на две категории:
\begin{enumerate}[(1)]
    \item системы, которые используют различные в силу естественных причин особенности 
    человека, такие как отпечатки пальцев, сетчатка глаза, голос, тепловая карта тела 
    и~т.~п. Они эффективны, но довольно дороги, так как требуют установки специального оборудования;
    \item cистемы, которые анализируют поведение пользователя и основаны на 
    опыте или особых навыках. Они не требуют какого-либо специального оборудования и 
    просты для внедрения. Один из таких подходов -- анализ динамики нажатий клавиш.
\end{enumerate}

В этой статье будут рассмотрены подходы к аутентификации
пользователя по различным моделям, построенным на информации о
нажатиях клавиш.

\section{Постановка задачи}

\subsection{Способы аутентификации}
В рамках проблемы рассматриваются сле\-ду\-ющие методы анализа динамики нажатий клавиш пользователем:
\begin{itemize}
    \item \textbf{статическая аутентификация при входе.} Анализ основывается на известном шаблоне, слове или другом заранее предопределенном тексте. Набираемые пользователем при входе данные собираются (например, при вводе пароля) и сравниваются с предшествующими удачными попытками входа. Данный подход рассматривается как расширение стандартного метода аутентификации при входе с использованием логина/пароля (т.~е. при входе в систему проверяется не только \textit{что} набрал пользователь, но и \textit{как} он это сделал). Стоит отметить следующие особенности статической аутентификации:
    \begin{itemize}\label{goal}
        \item {\bf небольшое количество входных данных.} Как правило, статическая аутентификация работает в паре с аутентификацией по паролю, а использование чрезвычайно длинных паролей, которые пользователь набирал бы вручную (более 100 символов), практически исключено;
        \item {\bf однообразие данных.} Один и тот же пароль, как правило, используется 
        для входа в сис\-те\-му множество раз, и из набора этого пароля можно извлечь лишь небольшое количество признаков. Таким образом, метод должен быть оптимизирован для распознавания пользователя по небольшому множеству параметров;
         \item {\bf высокая скорость работы.} В~случае статической аутентификации нет возможности проводить обработку данных для аутентификации параллельно с работой пользователя. До тех пор пока аутентификация не завершится успешно, пользователь не будет допущен к работе с системой. Поэтому необходимо сделать задержку между вводом пароля и входом в систему как можно меньшей.
    \end{itemize}

    \item \textbf{периодическая динамическая аутентификация.} В~динамических методах 
    происходит аутентификация пользователя по его работе с клавиатурой \textit{во время} 
    сессии работы с системой. В~данном случае процесс проверки пользователя\linebreak \mbox{может} 
    запускаться по какому-либо событию, например по времени или при обнаружении 
    потенциально подозрительной активности. Собранные в рамках сессии данные сравниваются с поведением пользователя в предшествующих сессиях для выявления аномалий. Данный метод имеет ряд преимуществ перед статическим анализом. Во-первых, набираемый текст может быть произвольным -- анализируется только то, \textit{как} пользователь его набирает. Во-вторых, такие данные значительно проще собрать: в общем случае они собираются в фоновом режиме при обычной работе пользователя. Таким образом обучить систему значительно проще и, кроме того, на большом массиве данных даже у неопытного пользователя проще выделить характерные признаки;
    \item \textbf{непрерывная динамическая аутентификация.} Является дополнением 
    периодической аутентификации в том, что проверка подлинности запускается в 
    фоновом режиме и выполняется постоянно (с точностью до минимальной порции данных, 
    на которой возможна корректная работа алгоритма классификации).
\end{itemize}

\subsection{Формальная постановка задачи}
Для всех перечисленных способов формальная постановка задачи звучит одинаково.

Пусть задано некоторое множество {\it пользователей} $ \mathfrak{U}
\hm= \{U_1, U_2, \ldots , U_n\}$. Также введем понятие {\it действия}
пользователя $A$, которым может считаться, например, нажатие одной
клавиши или набор пароля. Задача обучения в этом случае заключается
в том, чтобы  каждому пользователю $U_i\hm \in \mathfrak{U} $
сопоставить некоторую функцию (модель) $F_i$, которая может служить
мерой аномальности действия $A^{\mathrm{next}}_i$ для пользователя
$U_i$. В таком случае задача аутентификации~--- это вычисление
$F_i(A^{\mathrm{next}}_i)$ и обработка полученного значения, по
которому принимается решение об успешности аутентификации
пользователя $U_i$.

\section{Обзор существующих методов статической аутентификации}

Интересным представляется множество подходов, описанных в статье~\cite{lao}. 
Из них стоит выделить 4~эксперимента, проведенных
авторами статьи, которые были направлены на оценку качества тех или
иных моделей представления данных.

\subsection{Модель, основанная на~измерении времени удержания}\label{hold}

Одним из наиболее существенных признаков, извлекаемых из данных о
динамике работы пользователя с клавиатурой, является время удержания
пользователем различных кнопок в нажатом состоянии. Априори
предполагается, что у разных пользователей эти времена будут заметно
различаться, и по величине отклонения этого времени можно будет
установить аномальность действия $A_i^{\mathrm{next}}$.

В этом методе в качестве модели берется вектор~$X$ из $n$ элементов,
каждый из которых соответствует одной кнопке на клавиатуре и
является парой $(M_k, D_k)$: элемент $M_k$~--- среднее время
удержания клавиши $k$, а $D_k$~--- величина стандартного отклонения
для клавиши $k$.

Таким образом, действие пользователя $A^{\mathrm{next}}_i$ --- время
удержания некоторой клавиши~$K$ -- признается аномальным, если оно
отличается от среднего $M_K$ на величину, б$\acute{\mbox{о}}$льшую~$D_K$. Для модели
задается процент допустимых аномальных действий, и при превышении
этого порога аутентификация признается неуспешной.

\subsubsection{Постановка эксперимента}
Для проверки пригодности вышеописанной модели был поставлен
эксперимент. В~нем участвовало 15~человек, каждый из которых 10 раз
набирал английскую панграмму (фразу, содержащую все буквы алфавита)
<<{\sf The quick brown fox jumps over the lazy dog}>>. На этих
данных были построены модели, и для достижения наилучших результатов
была произведена вариация порогового значения допустимых аномальных
действий.

\subsubsection{Оценка результата}\label{shortres}
Для оценки результатов в этом и других экспериментах использовалось
вычисление величин:
\begin{itemize}
    \item False Rejection Rate (FRR)~--- процент нажатий пользователя, на  котором система обучена, воспринятые системой как нажатия другого пользователя (ошибка 1-го рода);
    \item  False Acceptance Rate (FAR)~--- процент нажатий другого пользователя, которые система определила как нажатия пользователя, на котором обучена (ошибка 2-го рода).
\end{itemize}



На эти признаки влияют два параметра:
\begin{enumerate}[(1)]
\item требуемая доля неаномальных нажатий для признания попытки авторизации успешной;
\item % и
допустимое для признания нажатия неаномальным количество стандартных отклонений продолжительности нажатия 
в модели.
\end{enumerate}
 Зависимость величин FRR и FAR от них отражают табл.~1 и~2.
Легко увидеть, что наилучшие результаты этим методом
достигаются при требовании попадания 75\%--80\% нажатий в границы двух
стан-\linebreak\vspace*{-12pt}
\begin{center}  %tabl1-2
% \vspace*{6pt}
{{\tablename~1}\ \ \small{False Acceptance Rate}}

\vspace*{6pt}

%\pagebreak


%\begin{table*}
{\small
%\begin{center}
\tabcolsep=4.1pt
\begin{tabular}{|c|c|c|c|c|}
\hline
        Доля неаномальных   &\multicolumn{4}{c|}{ Продолжительность нажатия}\\
         \cline{2-5}
 нажатий& \ \ \ 1,0\ \ \  & \ \ \ 1,5 \ \ \  & \ \ \ 2,0 \ \ \   & 2,5     \\
                \hline
                0,75 & 0\% & 0,61\% & 8,26\% & 25,0\%\hphantom{9}  \\
    %            \hline
                    0,80 & 0\% & 0\%\hphantom{9,9}    & 3,45\% & 16,79\% \\
     %           \hline
                0,85 & 0\% & 0\%\hphantom{9,9}    & 1,01\% & \hphantom{9}7,39\%  \\
      %          \hline
                0,90 & 0\% & 0\%\hphantom{9,9}    & 0\%\hphantom{,99}    & \hphantom{9}2,01\%  \\
                                \hline
                \end{tabular}
 %           \end{center}
%        \end{table*}

}

\vspace*{6pt}

{{\tablename~2}\ \ \small{False Rejection Rate}}

\vspace*{6pt}
        

%        \begin{table*}\small
 
{\small                \tabcolsep=1.5pt
                \begin{tabular}{|c|c|c|c|c|}
                               \hline
        Доля неаномальных   &\multicolumn{4}{c|}{ Продолжительность нажатия}\\
         \cline{2-5}
 нажатий& \ \ \ 1,0\ \ \  & \ \ \ 1,5 \ \ \  & \ \ \ 2,0 \ \ \   & 2,5     \\
                \hline
                    0,75 & 96,35\% & 31,41\% & \hphantom{9}1,25\%  & 0\%\hphantom{,99}  \\
 %                   \hline
                    0,80 & 99,43\% & 52,47\% & \hphantom{9}8,95\%  & 0\%\hphantom{,99} \\
  %                  \hline
                    0,85 & 100\%\hphantom{,999}   & 84,85\% & 24,16\% & 4,03\%  \\
   %                 \hline
                    0,90 & 100\%\hphantom{,999}   & 94,97\% & 64,01\% & 19,59\%\hphantom{9} \\
                                    \hline
                \end{tabular}
                }
            \end{center}
%\end{table*}

        \addtocounter{table}{2}
        
        \vspace*{6pt}


\noindent
дартных отклонений от среднего значения, что позволяет получить
результат 3,45\% FAR и 8,95\% FRR. Для такого сравнительно
несложного подхода их следует признать достаточно высокими.

\vspace*{-4pt}

\subsection {Наблюдение за порядком нажатия и~отпускания кнопок}

\vspace*{-2pt}

\subsubsection{Модель и метод}

Назовем {\bf <<обменом>> (<<swap>>)} такую па\-ру нажатий и отпусканий, 
которые происходят в непрямом порядке:

\vspace*{-18pt}

   \hspace*{-10pt}{ \begin{picture}(224, 100)
        \put(0,10){\vector(1,0){220}} \put(100,20){<<обмен>>}
        \put(30,10){\circle*{2}} \put(20,0){$\downarrow a$}
        \put(90,10){\circle*{2}} \put(80,0){$\downarrow b$}
        \put(150,10){\circle*{2}} \put(140,0){$\uparrow a$}
        \put(210,10){\circle*{2}} \put(200,0){$\uparrow b$}
        \put(0,60){\vector(1,0){220}} \put(60, 70){обычный порядок нажатий}
        \put(30,60){\circle*{2}} \put(20,50){$\downarrow a$}
        \put(90,60){\circle*{2}} \put(80,50){$\uparrow a$}
        \put(150,60){\circle*{2}} \put(140,50){$\downarrow b$}
        \put(210,60){\circle*{2}} \put(200,50){$\uparrow b$}
    \end{picture}}
    
    \vspace*{6pt}
    
\noindent
Так, при обычном порядке нажатий сначала нажимается кнопка~$a$,
затем она отпускается, затем нажимается кнопка~$b$, затем $b$
отпускается. Было замечено, что некоторые люди при наборе слов
непроизвольно меняют этот порядок, нажимая кнопку~$a$, затем, не
отпуская ее, кнопку~$b$, затем отпускают обе.

Идея метода, описываемого в этой части, заключается в учете подобных
<<обменов>> для определения аномальности действий пользователя.

Пусть задана последовательность нажатий и отпусканий~$S$,
представляющая собой действия пользователя для аутентификации. Для
нее можно вычислить количество таких <<обменов>>~$x_S$ и ввести
расстояние между двумя последовательностями~$S_1$ и~$S_2$ как
$|x_{S_1}\hm - x_{S_2}|$.

Для каждого пользователя вычислим расстояния между каждой парой его
наборов, посчитаем среднее и стандартное отклонение. Эти два числа
будут служить моделью для пользователя.

\subsubsection{Эксперимент}

Для эксперимента использовались те же данные, что и в подразд.~3.1,
однако не учитывались клавиши {\sf Shift} и~{\sf Delete}, для
того чтобы не вносить излишний шум в результаты. Для обучения была
использована треть собранных данных, для тестирования --- две трети.

\subsubsection{Результаты}

Для того чтобы получить оценку аномаль\-ности, дистанция между
тестируемым и пользовательским набором сравнивалась со средней
дистанцией между наборами одного и того же пользователя, и при выходе
значения дистанции за пределы одного стандартного отклонения попытка
авторизации признавалась неудачной.

Было выяснено, что значения FAR и FRR для результатов применения
этого метода сильно зависят от пар пользователей, в экспериментах
проявлялись доли ошибок от 0\% до 70\%, что не позволяет признать
этот метод подходящим для использования в одиночку. Полагается, что
длина текста недостаточна для такого подхода, и предлагается
усовершенствование метода путем учета того, на каких парах клавиш
происходят <<обмены>>.

\subsection{Относительная скорость печати}

Существует предположение, что на разных текстах абсолютная скорость
печати пользователя варь\-и\-ру\-ет\-ся в широких пределах (к примеру,
осмыс\-лен\-ный текст человек будет набирать быстрее, чем случайные
символы), однако для каждой пары кнопок скорость нажатий остается
примерно одинаковой. Поэтому высказывается предложение замерять
скорости набора пар кнопок и использовать их в качестве модели для
пользователя.

Пусть $S$ и $S'$~--- векторы пар кнопок, упорядоченных по скорости
набора. Пусть $S[i]$ и $S'[i]$~--- положения пары кнопок~$i$ в
векторах $S$ и $S'$ соответственно. Тогда расстояние между этими
двумя векторами вводится следующим образом: $\sum\limits_i |S[i] \hm-
S'[i]|$. Для нормировки предлагается делить расстояние на количество
элементов в векторах, тогда расстояния между более полными
(содержащими большее количество пар) векторами не будут резко
отличаться от расстояний между векторами, содержащими меньшее число
пар.

Следующие результаты были получены вариацией порогового значения
расстояния между тес\-ти\-ру\-емым и известным значениями вектора,
необходимого для признания аутентификации удачной, на данных
эксперимента, описанного в подразд.~3.1:
\begin{itemize}
    \item расстояния между двумя векторами одного и того же пользователя составили в 
    среднем 0,3192, расстояние между векторами разных пользователей~--- в среднем~0,529;
    \item стандартные отклонения расстояний между двумя векторами отличались незначительно.
\end{itemize}

Однако точно судить об успешности аутентификации пользователя можно
только при различии в векторах, меньшем 0,3, а о неуспешности~--- при
различии, большем 0,6.

\subsection{Использование правой и левой клавиш {\sf Shift}}\label{shift}

Гипотетически предполагается, что при наборе текста разные люди
используют правую и левую клавиши {\sf Shift} по-разному.
Вероятно, это можно использовать для аутентификации.

Для проверки этой гипотезы предлагается модель пользователя,
состоящая из пар вида $(x, L), (y, L), (z, R)$, где первый элемент
каждой пары~--- нажатая клавиша, а второй -- элемент из множества
$\{L,R\}$, соответствующий правой или левой кнопке {\sf Shift}, с
которым была нажата клавиша, соответствующая первому элементу.

Для эксперимента использовалась фраза, набранная каждым
пользователем по 5~раз и содержащая все заглавные буквы английского
алфавита: <<{\sf Another Quick Brown Fox Jumps Over The Lazy Dog
Yet Round Cats Eat Plain Goldfish Heartily In Maine Not Kansas Under
Some Vain Zealous Xena Warrior}>>.

По собранным данным 15 пользователей были разделены на 4~класса: 8 из них
 пользовались только левой клавишей {\sf Shift}; четверо~---
только правой; двое использовали левую клавишу чаще правой; еще
один~--- правую чаще левой.

Очевидно, только класса пользователя недостаточно, чтобы признать
аутентификацию успешной. Однако попадание в чужой класс дает весомое
основание отвергнуть попытку аутентификации. (Здесь не берется в
расчет тот факт, что клавиатура пользователя может выйти из строя.)




\subsection{Об одном из методов для~коротких буквенных или~цифровых паролей}

\subsubsection{Предлагаемый метод и~модель данных}

Одной из особенностей cтатической аутентификации, о которой было
упомянуто в подразд.~2.1, являлся небольшой объем входных данных. 
В~случае использования учета динамики нажатий клавиш совместно с
паролем это несложно обосновать: чем короче пароль, тем легче он
набирается пользователем; с увеличением длины пароля растет и
вероятность опечаток, ведущих к необходимости повторного набора.
Также было упомянуто небольшое число признаков: как правило, в
пароле содержится небольшое подмножество символов, бук\-вен\-но-циф\-ро\-вых
(в случае пароля для входа в сис\-те\-му) либо только цифровых (таким
образом устроены PIN-ко\-ды в банкоматах). Из этого следует, что
необходимо разрабатывать способы аутентификации с учетом ограничений
по длине и разнообразию символов.

Интересным в этом свете представляется подход, описанный в статье~\cite{saggio}. 
В~качестве модели авторами были взяты замеры
продолжительности нажатий клавиш, однако вычислялись они тремя
разными способами:
\begin{enumerate}[(1)]
    \item {\bf абсолютное время нажатия}. 
    В~качестве элементов вектора берутся времена удержаний клавиш и времена, 
    когда не нажата ни одна клавиша по отдельности;
    \item {\bf кумулятивное время удержания}. Аналогично предыдущему, однако 
    замеряемые времена аккумулируются, что позволяет сгладить вы\-бросы;
    \item {\bf использование отношения задержек}. В~векторе в качестве значения элемента, 
    соответствующего нажатию, берется отношение времени удержания к продолжительности 
    последующего за ним промежутка, когда не нажата ни одна клавиша.
\end{enumerate}
В качестве алгоритма, с помощью которого осуществлялось обучение,
был взят мультиклассовый линейный SVM (support vector machine)~\cite{svm}, так как он
демонстрирует высокие результаты на данных несложной структуры.

\subsubsection{Постановка эксперимента
}
Для участия в эксперименте были приглашены 16~человек, 8 из которых
были {\it информированы} о проводимом эксперименте, а другие 8~---
нет. Во избежание нежелательного искажения результатов
непреднамеренной тренировкой 300 попыток были разбиты на 10~дней, по
30~попыток набора ежед\-невно.

Одна попытка представляла собой набор слова <<{\sf special}>> в
случае буквенного пароля и число <<{\sf 12057}>> в случае
цифрового, что соответствовало средней длине пароля пользователя при
входе в систему и длине PIN-ко\-да. При опечатке в наборе попытка не
засчитывалась, это обосновано тем, что, как правило, при наборе
пароли и PIN-коды не видны пользователю и опечатка приведет к ошибке
при аутентификации.

\subsubsection{Результаты}

Для оценки результатов применялись те же способы, что и в
п.\,3.1.2. Значения параметров FAR и FRR приведены в табл.~3
и~4.


       
В строках таблиц отражен метод замера времени, в столбцах~--- 
результаты по данным, полученным
от информированных и неинформированных пользователей.

Несложно заметить, что процент ошибок сильно (в разы) зависит от
информированности пользователей и метод замера отношений несколько
более эффективен, нежели два прочих.

\vspace*{6pt}

\begin{center}  %tabl3-4
% \vspace*{6pt}
{{\tablename~3}\ \ \small{Буквенный пароль}}

\vspace*{6pt}


{\small
%\begin{center}
\begin{tabular}{|c|c|c|c|c|}  
\hline
\multicolumn{1}{|c|}{\raisebox{-6pt}[0pt][0pt]
{\tabcolsep=0pt\begin{tabular}{c} Метод\\ замера\end{tabular}}} &\multicolumn{2}{c|}{\tabcolsep=0pt\begin{tabular}{c}
Неинформи-\\рованные\\ пользователи\end{tabular}}&
\multicolumn{2}{c|}{\tabcolsep=0pt\begin{tabular}{c}Информи-\\ рованные\\
пользователи\end{tabular}}\\
\cline{2-5}
 & FAR & FRR & FAR & FRR \\
\hline
Абсолютный   & 6,82\%   & 12,30\% & 1,59\%  & 3,82\% \\
%                \hline
Кумулятивный & 5,92\%   & 11,43\% & 1,32\%  & 3,15\% \\
%               \hline
Отношений      & 6,73\%   & 11,69\% & 0,91\%  & 2,31\%\\
\hline
\end{tabular}

}

\vspace*{12pt}

{{\tablename~4}\ \ \small{Цифровой пароль}}

\vspace*{6pt}
        
{\small         
\begin{tabular}{|c|c|c|c|c|}
\hline
\multicolumn{1}{|c|}{\raisebox{-6pt}[0pt][0pt]
{\tabcolsep=0pt\begin{tabular}{c}Метод\\ замера\end{tabular}}} &\multicolumn{2}{c|}{\tabcolsep=0pt\begin{tabular}{c}
Неинформи-\\рованные\\ пользователи\end{tabular}}&
\multicolumn{2}{c|}{\tabcolsep=0pt\begin{tabular}{c}Информи-\\ рованные\\
пользователи\end{tabular}}\\
\cline{2-5}
& FAR & FRR & FAR & FRR \\
\hline
Абсолютный  & 5,67\%   & 10,36\% & 1,75\%  & 3,21\% \\
%                \hline
Кумулятивный  & 4,93\%   & \hphantom{9}9,69\% & 1,31\%  & 2,58\% \\
%               \hline
Отношений      & 5,10\%   & \hphantom{9}9,92\% & 0,99\%  & 1,92\%\\
\hline
\end{tabular}
   }
\end{center}
            
\vspace*{6pt}

        \addtocounter{table}{2}

 \section{Обзор существующих методов динамической аутентификации}

\subsection{Слияние классификаторов}

В качестве события $A$  будем рассматривать некоторую порцию
данных, например набор одного абзаца или строки. Назовем ее сессией
(не путать с сессией работы с системой). Весь процесс аутентификации
будет рассмотрен именно для сессий, а не для отдельных нажатий
клавиш.

\subsubsection{Формат данных}

В данном методе в качестве регистрируемых данных рассматриваются
задержки между событиями, производимыми клавиатурой. Событие
пред\-став\-ля\-ет собой нажатие или отпускание клавиши. Очевидно, что для
одной и той же пары нажатий можно использовать различные интервалы
времени:
\begin{itemize}
    \item PP (Press--Press): время между двумя последовательными нажатиями клавиш;
    \item PR (Press--Release): время, в течение которого клавиша была нажата;
    \item RP (Release--Press): время между отпусканием предыдущей и нажатием следующей клавиши;
    \item RR (Release--Release): время между двумя последовательными отпусканиями клавиш.
\end{itemize}
Все остальные характеристики, такие как сила нажатия на клавишу 
и~пр., в данном исследовании не рассматриваются, так как могут быть
получены только при помощи дополнительного оборудования. Для анализа
будем использовать совокупность временн$\acute{\mbox{ы}}$х значений PP, PR, RP, RR
для вводимой с клавиатуры последовательности.

\vspace*{-4pt}

\subsubsection{Определение по расстоянию до среднего}

В случае аутентификации часто используется метод~\cite{legett}. Для
формирования профиля пользователя рассчитывается среднее значение~$\mu$ 
и стандартное отклонение~$\sigma$ для всех типов событий. Одно
событие считается корректным, если его значение отличается от
среднего не более чем на половину отклонения. Иными словами, $|t\hm-
\mu| \hm< \alpha  \sigma$. Сессия считается корректной, если доля
корректных событий в ней не менее некоторого порога~$\beta$.

Можно доработать этот метод, адаптируя параметры под каждого
конкретного пользователя~\cite{hocquet}. Например, изначально
установить порог $\beta \hm= 1$ и далее понижать его на заданное
небольшое число, пока считаются корректными все события из
тренировочного набора.

Также вместо того, чтобы непосредственно сравнивать время со
средним, можно использовать взвешенную оценку. Для этого каждому
событию сопоставим оценку (лежащую в интервале $[0,1]$)\linebreak $\mathrm{score} \hm=
\exp ({|t-\mu|/\sigma})$. В~качестве оценки для сессии возьмем
среднее по оценкам, входящим в нее. Это среднее далее и будем
сравнивать с порогом~$\beta$.

\vspace*{-4pt}

\subsubsection{Определение по ритму набора}

Как и в музыке, где ритм определяется как относительная длительность
нот (половина ноты, чет\-верть ноты и~т.\,д.), предлагается рассмотреть
ритм, с которым пользователь набирает текст~\cite{hocquet}. Основная
идея здесь заключается в том, чтобы разделить численные временн$\acute{\mbox{ы}}$е
значения на несколько дискретных классов. Для этого можно
использовать пороговые значения:
\begin{itemize}
    \item $t > 200$: класс 1;\\[-14pt]
    \item $100 <t < 200$: класс 2;\\[-14pt]
    \item $70<t < 100$: класс 3;\\[-14pt]
    \item $30< t < 70$: класс 4;\\[-14pt]
    \item $t>30$: класс 5.
\end{itemize}
Недостаток такого подхода в том, что при использовании статических
порогов не учитывается средняя скорость набора текста пользователем.
Для решения этой проблемы можно классифицировать время внутри одной
сессии в сравнении с остальными интервалами в ней.
\begin{itemize}
    \item ${1}/(10)$ самых медленных: класс~1;
    \item ${1}/{3}$ самых медленных: класс~2;
    \item ${2/}{3}$ самых медленных: класс~3;
    \item ${3}/{4}$ самых медленных: класс~4;
    \item ${1}/{4}$ самых быстрых: класс~5.
\end{itemize}
В качестве профиля пользователя каждому нажатию сопоставляем вектор
классов, к которым отнесены соответствующие временн$\acute{\mbox{ы}}$е интервалы.
Далее над этими векторами вводим дистанцию как сумму разниц между
номерами классов в векторах. Далее, сравнивая эту сумму с порогом,
получаем критерий корректности пользователя.

\subsubsection{Определение по ранжированию времен}

Данный метод используется только для статической аутентификации~\cite{bergadano}. 
Для каждого события считается его ранг (порядковый
номер в отсортированной последовательности тех же значений). 
В~качестве профиля пользователя вычисляется среднее по рангам для
каждого из нажатий (по тестовой выборке). Для того чтобы оценить
сессию, используется коэффициент Спирмана, который считается по
формуле:
$$r_{\mathrm{Sp}} = 1 - \fr{6 \sum\limits_{i=1}^n\left(r_i^1 - r_i^2\right)}{n  (n^2 -1)}\,,
$$
где $r_i^1$ -- ранг $i$-го нажатия в сессии $1$, а $n$~--- количество нажатий в сессии.

\subsubsection{Слияние методов}
Пусть для каждого пользователя имеется три классификатора,
работающих на одних и тех же данных, но оценивающих их различные
характеристики. Для комбинирования их в одном подходе воспользуемся
правилами слияния, описанными в~\cite{kittler}.

Итак, имеется три классификатора, которые решают проблему о
принадлежности к одному из двух классов (корректный
пользователь\,/\,на\-ру\-ши\-тель), возвращающих оценку от~0 до~1. Проблема в
том, что оценки каждого из классификаторов не похожи и по-раз\-но\-му
распределены для одних и тех же наборов данных. В~связи с этим
неэффективно было бы применять любую схему голосования (<<хотя бы
один>>, <<большинство>>, <<все>>). Для этого существует несколько
методов нормализации:
\begin{itemize}
    \item нормализация по максимуму: 
    $$\mathrm{score}'=\fr{\mathrm{score}}{\mathrm{scoreMax}}\,;
    $$
    \item нормализация по максимальной разнице: 
    $$
    \mathrm{score}' = \fr{\mathrm{score - scoreMin}}{\mathrm{scoreMax - scoreMin}}\,;
    $$
    \item Z-мера: 
    $$
    \mathrm{score}'=\fr{|\mathrm{score} - \overline{\mathrm{score}}|}{\sigma_{\mathrm{score}}}\,.
    $$
\end{itemize}

Стоит отметить, что первые два метода нормализации могут привести к
получению значений больше 1 или меньше~0, однако в данном случае это
будет обозначать векторы, крайне близкие или отдаленные от профиля
пользователя.

Все эти оценки требуют предварительного знания о самой выборке, а
также дополнительного расчета (особенно в случае Z-ме\-ры), поэтому
использовать их в реальном времени весьма затруднительно. Можно
использовать методы слияния, описанные в~\cite{kittler}. Для начала
сведем задачу к бинарной классификации с ограничением 
$$
P(\mathrm{user}) = 1 - P(\mathrm{impostor})\,.
$$
Так как нет информации о вероятности, положим
$$
P(\mathrm{user}|i) = \mathrm{Score}^i\,,$$
 где $\mathrm{Score}^i$~--- оценка $i$-го
классификатора. Далее можно использовать следующие операторы
слияния:
\begin{itemize}
    \item максимум и минимум: 
    $$
    \mathrm{Score}= \max\limits_i (\mathrm{Score}^i)\,;
    $$
    \item медиана: 
    $$\mathrm{Score}= \mathrm{MEDIAN}(\mathrm{Score}^i)\,;
    $$
    \item правило произведения: 
    $$
    \mathrm{Score}= \prod\limits_i \mathrm{Score}^i\,;
    $$
    \item правило суммы: 
    $$
    \mathrm{Score}= \sum\limits_i \mathrm{Score}^i\,.
    $$
\end{itemize}

\subsubsection{Результаты}

Для сравнения методов использовались три меры оценки:
\begin{enumerate}[(1)]
    \item FAR~--- процент некорректных пользователей, 
    которые были приняты классификатором (ошибка 2-го рода);
    \item FRR~--- процент корректных пользователей, 
    которые были отвергнуты классификатором (ошибка 1-го рода);
    \item EER (Equal Error Rate)~--- точка, в которой FAR\;=\;FRR.
\end{enumerate}

\begin{center}  %tabl5-7
% \vspace*{6pt}
{{\tablename~5}\ \ \small{Средние ошибки}}

\vspace*{6pt}


{\small
\tabcolsep=5pt
\begin{tabular}{|c|c|c|c|}
\hline
Метод  & FAR      & FRR      & ERR\\
\hline
Расстояние до среднего & 4,39\% & 4,81\% &4,60\%\\
\hline
\tabcolsep=0pt\begin{tabular}{c}Расстояние до среднего\\ (взвешенное)\end{tabular} & 
3,62\% & 3,61\% &3,62\%\\
\hline
Ранжирование по Спирману & 3,56\% & 3,62\% &3,59\%\\
\hline
Ритм набора (по порогам) & 3,47\% & 3,39\% &3,43\%\\
\hline
\tabcolsep=0pt\begin{tabular}{c}Ритм набора\\ (пропорциональный) \end{tabular}& 3,55\% & 4,02\% &3,79\%\\
\hline
\end{tabular}}
 
 \vspace*{9pt}
 
 {{\tablename~6}\ \ \small{Применение нормализации}}

\vspace*{6pt}

{\small
\begin{tabular}{|c|c|c|c|}
\hline
Нормализация  & FAR      & FRR      & ERR\\
\hline
По максимуму & 1,75\% & 2,46\% &2,11\%\\
По максимальной разнице & 2,00\% & 2,00\% &2,00\%\\
Z-мера & 1,81\% & 1,69\% &1,75\%\\
\hline
\end{tabular}
}

\vspace*{6pt}
 
 {{\tablename~7}\ \ \small{Результаты слияния}}

\vspace*{6pt}

{\small
\tabcolsep=4pt
\begin{tabular}{|c|c|c|c|}
\hline
Метод  & FAR      & FRR      & ERR\\
\hline
Голосование (<<все>>) & 1,17\% & 7,92\% &4,55\%\\
Голосование (<<большинство>>) & 2,39\% & 2,15\% &2,27\%\\
Голосование (<<хотя бы один>>) & 7,60\% & 0,54\% &4,07\%\\
Правило произведения & 2,00\% & 2,00\% &2,00\%\\
Правило максимума & 3,62\% & 3,61\% &3,62\%\\
Правило минимума & 3,62\% & 3,62\% &3,62\%\\
Правило медианы & 3,34\% & 3,39\% &3,37\%\\
Правило суммы & 1,81\% & 1,69\% &1,75\%\\
\hline
\end{tabular}
 }
\end{center}

\vspace*{6pt}

Значения этих мер оценки на разных методах приведены в
табл.~5 и~6. Как видно из
табл.~7, наилучший результат для данного метода
достигается при применении правила суммы для результатов трех
исходных классификаторов.



\addtocounter{table}{3}

\subsection{Кластеризация методом Partition Around Medoids}

Сначала рассмотрим алгоритм классификации пользователей, основанный
на кластеризации и представленный в~\cite{bert}.

\subsubsection{Сбор данных и модель представления}

В данной статье был использован набор данных {Si6}~\cite{si6}, 
который состоял из 66~сессий набора 62~различных
пользователей.

Каждая сессия набора состояла из 15~предложений. Для каждой сессии
фиксировалось на\-би\-ра\-емое предложение, время нажатия (или отпускания)
клавиши с точностью до миллисекунд, сама клавиша и тип действия
(нажатие или отпускание).

В эксперименте учитывались только завершенные сессии. Для трех
пользователей, которые набрали несколько сессий, была выбрана только
одна. Также не учитывались ошибочные предложения. После исключения
осталось 54~сессии, каждая из которых состояла из 15 предложений.

Для каждого предложения каждой сессии каж\-до\-го пользователя был
вычислен вектор времени диграфов (время между двумя
последовательными нажатиями на клавиши). Диграфы, содержащие
непечатные символы, были исключены. Рас\-смат\-ри\-ва\-лись оставшиеся
88~946 времен диграфов, соответствующих 411~уникальным
последовательностям из двух клавиш.

Были проведены три различных эксперимента с разной 
группировкой предложений пользователя в подсессии:
%\renewcommand{\labelenumi}{\arabic{enumi})}
\begin{enumerate}[(1)]
    \item 3 подсессии, каждая состояла из 5~предложений;
    \item 5 подсессий, каждая состояла из 3~предложений;
    \item 15 подсессий, каждая состояла из 1~предложения.
\end{enumerate}
Каждая подсессия характеризовалась вектором средних значений всех
различных диграфов, отсортированных по убыванию числа появлений у
всех пользователей.

Количество кластеров для классификации (параметр $k$) было выбрано по алгоритму 
\textit{силуэта}, описанному в~\cite{sil}.

\subsubsection{Фильтрация}

Фильтрация играет значимую роль в процессе интеллектуального анализа данных. 
Были использованы два уровня фильтрации:
\begin{enumerate}[(1)]
    \item при грубой фильтрации были исключены:
    \begin{itemize}
        \item[(а)] диграфы со временем менее 10 и более 750~мс;
        \item[(б)] диграфы, время которых отличалось от среднего времени всех диграфов 
        больше чем на стандартное отклонение времени всех диграфов, т.\,е.\ 
        $140\hm\pm 113~$мс\;$\approx$\;(25~мс, 250~мс).
    \end{itemize}
     Таким образом было исключено около 17\% диграфов;
    \item на этапе более тонкой фильтрации исключались:
    \begin{itemize}
        \item[(а)] диграфы, встречающиеся только один раз;
        \item[(б)] диграфы, у которых стандартное отклонение больше 
        удвоенного среднего стандартного отклонения всех диграфов данной подсессии;
        \item[(в)] из одинаковых диграфов в подсессии исключались те, которые имели 
        время, отличающееся от среднего в данной выборке больше чем на удвоенное 
        стандартное отклонение.
    \end{itemize}
\end{enumerate}

\subsubsection{Результаты}

При оценке качества определение в один клас\-тер двух подсессий,
принадлежащих одному пользователю, являлось верной классификацией\linebreak
(TP~--- true positive). Ложно-положительным (FP~--- false positive)
срабатыванием считалось определение двух подсессий различных
пользователей в один кластер. Лож\-но-от\-ри\-ца\-тель\-ным (FN~--- false
negative)~--- определение двух подсессий одного пользователя в разные
кластеры.

Значение полноты (Precision) было принято равным $\mathrm{TP/(TP +
FP)}$; точности (Recall)~--- $\mathrm{TP/(TP + FN)}$; $F$-ме\-ра: 
$$
F= 2  \fr{\text{Presicion} \cdot
\text{Recall}}{\text{Presicion} + \text{Recall}}\,.
$$

Лучшие значения $F$-меры составили: для разделения сессий на 3~подсессии~--- 0,9, 
на 5~подсессий~--- 0,78, на 15~подсессий~--- 0,26.

\subsection{Относительная скорость печати}

Теперь рассмотрим метод, описанный в~\cite{gunet}. Центральным
понятием здесь является $n$-граф~--- время между первым и последним
нажатием серии из $n$~клавиш.

В нем описывается использование как относительных метрик различия
между сессиями печати, которые учитывают только то, какие $n$-гра\-фы
набираются быстрее других, так и абсолютных, сравнивающих время
набора.

\subsubsection{Относительные метрики ($R$-метрики)}

Для массива $V$ из $k$ чисел можно вычислить степень
неупорядоченности, которая является суммой расстояний между
элементом~$V$ и соответ\-ст\-ву\-ющим элементом упорядоченного массива
$V'$. Так, для массива $A \hm= [2,3,1,4,5]$ степень неупорядоченности
равна $1 \hm+ 1 \hm+ 2 \hm+ 0 \hm+ 0 \hm= 4$.

Получим максимальную степень неупорядоченности массива из $k$ чисел:
    $$
    D_k =
    \begin{cases}
        \fr{k^2}{2}\,,  &\ k \mbox{ --- четное}\,; \\
        \fr{k^2 - 1}{2}\,, &\ k \mbox{ --- нечетное}\,.
    \end{cases}
    $$
Это позволит нормировать неупорядоченность. Так, для массива~$A$ из
примера нормированная неупорядоченность будет равна: 
$$
\fr{1 + 1 + 2+ 0 + 0}{(5^2 - 1) / 2} \hm= \fr{8}{12} = 0{,}6666\,.
$$

Образец печати пользователя $S$ представляет собой массив средних
времен набора $n$-гра\-фов, встречающихся в тексте, отсортированный по
возрастанию. Таким образом, вводится расстояние между двумя
образцами печати $S_1$ и $S_2$. Если у $S_1$ и $S_2$ $k$ общих
$n$-графов, то расстояние $R_n(S_1, S_2)$ равно сумме расстояний
между общими $n$-графами, нормированному на $D_k$. Те $n$-гра\-фы,
которые не принадлежат пересечению $S_1$ и $S_2$, просто
игнорируются.

Заметим, что для небольших наборов данных у двух образцов может не
оказаться общих $n$-гра\-фов, что сделает невозможным вычисление~$R_n$.

Можно предположить, что вычисление относительной метрики между двумя
образцами для диграфов, триграфов или иных $n$-гра\-фов способно
предоставить различные аспекты информации о различии ритмов печати.

Если у образцов $S_1$ и $S_2$ $N$ общих $n$-графов и $M$ общих $m$-гра\-фов ($N \hm> M$), 
вводится кумулятивная относительная метрика:
$$
R_{n, m} = R_n(S_1, S_2) + R_m(S_1, S_2)  \fr{M}{N}\,.
$$
Аналогично можно расширить метрику для применения к большему количеству 
различных $n$-гра\-фов:
$$
R_{n, m} = R_n(S_1, S_2) + R_m(S_1, S_2)  
\fr{M}{N} + R_p(S_1, S_2)  \fr{P}{N}\,.
$$

\subsubsection{Абсолютные метрики ($A$-метрики)}

К сожалению, у $R$-мет\-рик есть существенный недостаток. Если среднее
время каждого $n$-гра\-фа в образце $S_2$ в два раза больше, чем в
образце $S_1$, то $R_n(S_1, S_2) \hm= 0$. Таким образом, относительная
метрика не может отличить двух пользователей, имеющих очень похожие
ритмы печати, пусть даже один из них печатает намного быстрее
другого.

В отличие от $R$-метрик, $A$-мет\-ри\-ки учитывают абсолютное время
набора $n$-графов. Пусть $G_{S_1, d_1}$ и $G_{S_2, d_2}$~--- это один
и тот же $n$-граф, присутствующий в $S_1$ со временем $d_1$ и в
$S_2$ со временем $d_2$. \textit{Похожими} называются $n$-гра\-фы, для
которых выполняется соотношение
$$
1 < \fr{\max(d_1, d_2)}{\min(d_1,
d_2)} \le t\,,
$$
где $t$~--- некоторая константа, большая единицы.

Введем метрику
$$
A^t_n(S_1, S_2) = 1 - \fr{\mbox{количество\ \textit{похожих}\ $n$-графов}}
{\mbox{общее\ количество\ $n$-графов}}\,.
$$

Таким образом, расстояние между образцами, у которых все $n$-гра\-фы
похожи, будет равно~0. Между образцами, у которых нет общих
$n$-гра\-фов,~---~1. Кумулятивные метрики вводятся аналогично
$R$-мет\-рике:
\begin{align*}
A^t_{n, m} &= A^t_n(S_1, S_2) + A^t_m(S_1, S_2)  \fr{M}{N}\,;
\\
A^t_{n, m} &= A^t_n(S_1, S_2) + A^t_m(S_1, S_2)  \fr{M}{N} + A^t_p(S_1, S_2) 
\fr{P}{N}\,.
\end{align*}

Стоит заметить, что во всех ниже описываемых экспериментах принималось $t \hm= 1{,}25$.

\subsubsection{Сбор данных}

Сорок человек предоставили по 15 сессий печати. Они выполняли роль
легитимных пользователей системы. Еще 165~человек предоставили по
одной сессии. Эти образцы использовались для имитации атаки на
систему.

Перерыв между сессиями был не менее одного дня. Каждая сессия
представляла собой свободный текст длиной около 700--900~символов.

У всех участников эксперимента итальянский является родным языком.

\begin{table*}[b]\small %tabl8
\begin{center}
    \Caption{Одиночные метрики}
        \label{res_sing}
\vspace*{2ex}

%    \tabcolsep=0,05cm
    \begin{tabular}{|c|c|c|c|c|}
                    \hline
                    Метрики& FP count& FN count & FP rate, \% & FN rate, \%\\
                    \hline
 $R_2$ & 563& 50& 0,13& 8,33\\
 $R_{2,3}$ &  324& 32& 0,07& 5,33\\
 $R_{2,4}$ & 259& 41& 0,06& 6,83\\
 $R_{2,3,4}$ &  199& 41& 0,04& 6,83\\
 $A_2$ &  590& 92& 0,13& 15,3\\
 $A_{2,3}$ &  335& 80& 0,07& 13,3\\
 $A_{2,4}$ &  366& 84& 0,08 & 14,0 \\
 $A_{2,3,4}$ & 331& 79& 0,07& 13,2\\
        \hline
          \end{tabular}
    \end{center}
%\end{table*}
%\begin{table*}\small
\begin{center}
 \Caption{Сумма метрик}
    \label{res_sum}
    \vspace*{2ex}
    
%    \tabcolsep=0,05cm
  \begin{tabular}{|c|c|c|c|c|}
                    \hline
                    Метрики& FP count& FN count & FP rate, \% & FN rate, \%\\
                    \hline
$R_2 + A_2$ & 360& 36& 0,08\hphantom{9}& 6,0\hphantom{99}\\
$R_2 + A_{2,3}$ & 272& 34& 0,06\hphantom{9}& 5,667\\
 $R_2 + A_{2,4}$ &  260& 37& 0,057& 6,167\\
 $R_2 + A_{2,3,4}$ & 237& 41& 0,052& 6,833\\
$R_{2,3,4} + A_2$ & 124& 19& 0,028& 3,167\\
$R_{2,3,4} + A_{2,3}$ & \hphantom{9}78 & 23& 0,017& 3,833\\ 
$R_{2,3,4} + A_{2,4}$ &  \hphantom{9}95& 22 & 0,021& 3,667\\ 
$R_{2,3,4} + A_{2,3,4}$ & 131& 23& 0,029& 3,833\\
        \hline
    \end{tabular}
    \end{center}
%\end{table*}

%\begin{table*}\small %tabl10
\begin{center}
    \Caption{Последовательные метрики}
    \label{res_seq}
    \vspace*{2ex}
    
%    \tabcolsep=0,05cm
    \begin{tabular}{|c|c|c|c|c|}
                    \hline
                    Метрики& FP count& FN count & FP rate, \% & FN rate, \%\\
                    \hline
$\{R_2 + A_2,\ R_2+A_{2,3}, \ R_2+A_{2,4}.\ R_2+A_{2,3,4}\}$ & 83 & 55 & 0,018& 9,167\\
$\{R_{2,3} + A_2,\ R_{2,3}+A_{2,3},\ R_{2,3}+A_{2,4},\ R_{2,3}+A_{2,3,4}\}$ &
74& 38& 0,016& 6,333\\
 $\{R_{2,3,4} + A_2,\ R_{2,3,4} + A_{2,3},\ R_{2,3,4} + A_{2,4},\
R_{2,3,4} + A_{2,3,4}\}$ & 22& 29& 0,005& 4,833\\
        \hline
    \end{tabular}
    \end{center}
\end{table*}

\subsubsection{Аутентификация}

В оригинальной статье рассматриваются эксперименты по классификации,
идентификации и аутентификации. В~рамках данного обзора будет описан
только эксперимент по аутентификации.

%\pagebreak

Пусть рассматриваются пользователи $A$, $B$, $C, \ldots$, которые
представлены образцами печати $A_1, \ldots, A_n$, $B_1, \ldots, B_k$
и~т.\,д. Средним расстоянием неизвестного образца $X$ от пользователя
$A$ назо-\linebreak вем
\begin{multline*}
\mathrm{md}(A, X) ={}\\
{}= \fr{1}{n} \left(d(A_1, X) + d(A_2, X) + \cdots + d(A_n, X)\right)\,,
\end{multline*}
где $d(A_i, X)$~--- расстояние между двумя образцами.

Среднее расстояние между образцами пользователя~$A$ обозначим как $m(A)$.

Неизвестный образец $X$ считается принадлежащим 
шаблону пользователя~$A$ в случае выполнения следующих условий:
\begin{itemize}
    \item $\mathrm{md}(A,X) < \mathrm{md}(B,X)$ для любого другого легитимного пользователя~$B$;
    \item $\mathrm{md}(A,X) < m(A)$ {или} $\mathrm{md}(A,X) \hm- m(A) \hm< 
    \mathrm{md}(B,X) \hm- \mathrm{md}(A,X)$ 
    для любого другого легитимного пользователя~$B$.
\end{itemize}

Каждый образец $S$ легитимного пользователя $U$ был использован как
новый и неизвестный образец. При корректной работе классификатора он
должен быть отнесен к пользователю~$U$, представленному оставшимися
14~образцами. Также $S$ использовался для попытки аутентификации под
видом каждого из оставшихся 39~пользователей. Когда образец~$S$
пользователя~$U$ используется для аутентификации под видом
пользователя~$U'$, пользователь~$U$ временно изымается из модели.
Таким образом, нарушитель всегда был неизвестен.

Каждый из 165 образцов нарушителей был использован для попытки
аутентификации под видом каждого легитимного пользователя.
Получается 600 попыток аутентификации правильным пользователем и
450\,000 атак $40 \hm+ 165 \hm= 205$~нарушителями ($600 \cdot 39 \cdot 15 \hm+
165 \cdot 40 \cdot 15 \hm= 450\,000$).

\subsubsection {Результаты}

В табл.~\ref{res_sing} представлены результаты первого эксперимента.
В нем были использованы одиночные метрики, перечисленные в первом
столбце. В~табл.~8 FP count~--- количество аутентифицированных атак (из
450\,000); FN count~--- количество отвергнутых легитимных
аутентификаций (из 600); FP rate и FN rate~--- отношение FP count и
FN rate к 450\,000 и 600 соответственно.

Для второго эксперимента, результаты которого отражены в
табл.~\ref{res_sum}, метриками служили суммы известных метрик.

В третьем эксперименте (табл.~\ref{res_seq}) образец считался
аутентифицированным, если он признавался аутентифицированным каждой
метрикой последовательности.

\subsection{Метод наказания-поощрения (\textit{penalty-reward})}

Интересный метод был предложен в~\cite{nisk}. Он основан на
динамическом изменении уровня недоверия к пользователю, что
позволяет оперативно реагировать на подмену пользователя.

\subsubsection{Функция наказания и поощрения}

Во время набора пользователь характеризуется уровнем доверия~$C$. 
В~начале сессии он принимается равным~0. При каждом нажатии $C$
изменяется в зависимости от информации в шаблоне пользователя. Если
ритм нажатия хорошо подходит к шаб\-ло\-ну, пользователь поощряется
уменьшением~$C$. В~противном случае~--- наказывается увеличением.

Пока $C$ остается ниже некоторого порога, есть уверенность в том,
что это тот пользователь, за которого он себя выдает. Если же
значение~$C$ велико, то, скорее всего, системе нужно предпринять
действия по уточнению личности пользователя.

Значение $C$ не должно становиться отрицательным. Если этого не
учитывать, то во время работы легитимного пользователя $C$ станет
очень маленьким, что даст нарушителю много времени до превышения~$C$
заданного порога. Нужно решить, как должно меняться значение уровня
доверия в случае, если очередная клавиша (комбинация клавиш) не
входят в шаблон пользователя. Авторами предлагается увеличивать~$C$
на небольшую заданную константу.

Таким образом, можно ввести следующую функцию:
$$ C =\begin{cases}
    0\,, &\ \mbox{начало\ сессии}\,; \\
    \max(C - R, 0)\,, &\ d \le T\,; \\
    C + d - T\,, &\ d > T\,; \\
    C + \alpha\,, &\ \mbox{клавиша\ не\ из\ шаблона}\,.
\end{cases}
$$
Здесь $R$ -- величина поощрения пользователя за нажатие,
соответствующее шаблону; $d$~--- расстояние от очередного нажатия до
шаблона пользователя; $T$~--- доверительный порог расстояния;
$\alpha$~--- наказание за клавишу, не принадлежащую шаблону.

\subsubsection{Сбор данных и модель представления}

В эксперименте участвовало 25~пользователей, предоставляя информацию
о нажатиях в течение 6--15~дней своей обычной работы за компьютером.

Время между нажатием и отпусканием клавиши назовем
\textit{удержанием}. Время между отпусканием одной клавиши и
нажатием следующей назовем \textit{задержкой}.

Шаблон пользователя представляет собой математическое ожидание~($\mu$) 
и стандартное отклонение~($\sigma$) удержания и задержки
нажатых клавиш и комбинаций клавиш. Решение о включении клавиши
(комбинации клавиш) в шаблон принималось на основе числа вхождений~$N$ 
в наборе и отношения~$\mu$ и~$\sigma$: $N$ должно быть выше
некоторого порога, ${\mu}/{\sigma}$~--- ниже.

Расстояние между новой клавишей и шаблоном введено как
$$
d = d((\mu, \sigma), t) = \left|\fr{t - \mu}{\sigma}\right|\,,
$$
где $t$~--- время удержания или задержки.

Расстояние между комбинацией клавиш $k_1k_2$ и шаблоном:
$$
d = \fr{1}{3}  \left( \left|\fr{t_{k_1} - \mu_{k_1}}{\sigma_{k_1}}\right| + 
\left|\fr{t_{k_1 k_2} - \mu_{k_1 k_2}}{\sigma_{k_1 k_2}}\right| + 
\left|\fr{t_{k_2} - \mu_{k_2}}{\sigma_{k_2}}\right| \right)\!.
$$

Необходимо определить максимальное значение $C$, после которого
можно считать пользователя нарушителем~--- $T_{\mathrm{action}}$. Это
значение будет своим для каждого пользователя. Оно будет равно
максимальному значению~$C$, полученному при применении шаблона
пользователя к обучающим данным. Это позволит свести к минимуму
лож\-но-от\-ри\-ца\-тель\-ные срабатывания.

\subsubsection{Результаты}

Для каждой пары различных пользователей $i$ и~$j$ шаблон~$i$
применялся к данным~$j$. Измерялось среднее количество нажатий,
после которого значение~$C$ превышало порог $T^i_{\mathrm{action}}$.

Требовалось от 79 до 348~нажатий для определения нарушителя. 
В~среднем нарушитель определялся после 265~нажатий.

\section{Предлагаемый подход}

\subsection{Представление данных}

Среди рассмотренных выше способов лишь подход, использующий
$n$-гра\-фы, учитывает в единице данных события, происходившие на
протяжении некоторого времени. Тем не менее очевидно, что параметры,
относящиеся только к одному нажатию, несут в себе меньше информации,
нежели параметры, описывающие их серию. Поэтому предлагается новый
подход, позволяющий эффективно сохранять недавнюю активность
пользователя при работе с клавиатурой.

Предлагаемое представление данных основывается на отображении
$$
\varphi: (A, t_d, t_h) \rightarrow H, A \in \Omega\,,
$$
где $\Omega$ -- некоторый конечный алфавит действий пользователя;
$t_d$~--- время, в которое была нажата клавиша; $t_h$~--- время, в
течение которого клавиша удерживалась нажатой; $H$~--- вектор
признаков. Это отображение должно оказывать большее влияние на
анализ события $A^{\mathrm{next}}$ в зависимости от  сле\-ду\-ющих
факторов:
\begin{itemize}
  \item недавняя активность пользователя;
  \item частота совершения действия;
  \item протяженность действия по времени.
\end{itemize}

Для построения такого отображения была использована теория
потенциальных функций~\cite{aizerman}. Пусть каждое возможное
событие $A_i \hm\in \Omega$ имеет свой потенциал в момент~$t$. Он
убывает в за\-ви\-си\-мости от времени, прошедшего с момента совершения
действия. Этот процесс характеризуется функцией $\mathrm{Pf}: \mathrm{Time} \times
\mathrm{Time} \hm\rightarrow \mathbb R$. Если последовательность содержит два
или более событий~$A_i$, то их потенциалы суммируются.

Таким образом, можно определить отображение последовательности
действий пользователя в $L$-мер\-ный вещественный вектор, где $L \hm=
|\Omega|$,  в виде:
$$
  \varphi(H(U), t) = \left(
  \sum\limits_{\substack{(A, t_m) \in H(U) \\ t > t_m}} P\!f(t, t_m) \right)_{A \in \Omega}\,.
$$
Согласно этой формуле, активность пользователя в каждый конкретный
момент времени~$t$ может быть определена как множество из $L \hm=
|\Omega|$ потенциалов $\varphi_A(H(U), t)$. Такой подход учитывает
как частоту предыдущих действий, так и время, в которое текущее
действие было совершено.

Был выбран класс ра\-ди\-аль\-но-ба\-зис\-ных функций, поскольку эти функции
обладают необходимым в данном случае свойством: они зависят от
интервалов между действиями, но не зависят от абсолютного времени их
совершения. Кроме того, эти функции легко параметризовать для того,
чтобы добиться эффективности в задачах  аутентификации и
идентификации.

В качестве потенциальной ра\-ди\-аль\-но-ба\-зис\-ной функции была выбрана
экспоненциальная функция $\mathrm{Pf}(x, y) \hm= e^{-\sigma||x-y||}$, где
$\sigma$~--- коэффициент затухания, отвечающий за то, как быстро
потенциал будет убывать. Кроме того, для конечной последовательности
действий эта функция может быть рассчитана рекурсивно (полагая
$\varphi_A(0) \hm= 0$):
$$
\varphi_A(t_n)  = 
\begin{cases}
  \varphi_A(t_{n \! - \! 1})   e^{-\sigma||t_n-t_{n -  1}||}\,, &\ A \ne A_n; \\[3pt]
     e^{- \alpha}\,, &\ A = A_n\,.
\end{cases}
$$
Для достижения максимальной эффективности коэффициенты $\sigma$
 и~$\alpha$ могут быть подобраны, учитывая специфику конкретной задачи.
Коэффициент~$\alpha$ отвечает за влияние времени нажатия на значение
потенциала.

Для того чтобы избежать зашумленности данных низкими значениями,
можно ввести константный порог~$\varepsilon$ следующим образом:
$$
\label{functor_threshold}
\hat{\varphi}_A(t_n) =
\begin{cases}
  \varphi_A(t_n)\,, &\varphi_A(t_n) \ge \varepsilon\,; \\[3pt]
  0\,, &\varphi_A(t_n) < \varepsilon\,.
\end{cases}
$$

Текущая активность пользователя описывается последовательностью
$n$-мер\-ных векторов (где $n \hm= |\Omega|$), содержащих потенциалы
$\hat\varphi_A$ для каждого действия~$A$.

В качестве примера рассмотрим набор слова {\sf Hello} и
получаемые при этом векторы. Возьмем упрощенный случай: $\Omega$
ограничим буквами {\sf H, E, L, O}. В~таком случае получаемые
векторы будут четырехмерными и могут иметь вид, отраженный в
табл.~11.

На рис.~1 видно, что элементы вектора
представляют собой значения потенциальной функции соответствующего
действия в моменты нажатий.

Таким образом, один вектор содержит достаточное количество
информации, чтобы быть правильно распознанным с высокой долей
ве\-ро\-ят\-ности с помощью алгоритмов машинного обучения, и при этом
генерируется на каждое нажатие, что позволяет увеличить объем данных
для обучения и распознавания.


 \vspace*{14pt} %tabl11
 
\noindent
{{\tablename~11}\ \ \small{Получаемые векторы при наборе слова {\sf Hello}}}

%\vspace*{1pt}


\begin{center} 
%\begin{table*}
{\small
\tabcolsep=7.5pt
\begin{tabular}{|c|c|c|c|c|}
\hline
\multicolumn{1}{|c|}{\raisebox{-6pt}[0pt][0pt]{Событие}} & \multicolumn{4}{c|}{Состояние вектора} \\
\cline{2-5}
& <<{\sf H}>> & <<{\sf E}>> & {<<{\sf L}>>} & {<<{\sf O}>>} \\
\hline
Начальное состояние & 0\hphantom{,99} & 0\hphantom{,99} & 0\hphantom{,99} & 0 \\
%            \hline
Нажата <<{\sf H}>> & 1\hphantom{,99} & 0\hphantom{,99} & 0\hphantom{,99} & 0 \\
%           \hline
Нажата <<{\sf E}>> & 0,82 & 0\hphantom{,99} & 0\hphantom{,99} & 0 \\
%          \hline
Нажата {<<{\sf L}>>} & 0,53 & 0,64 & 1\hphantom{,99} & 0 \\
%         \hline
Нажата {<<{\sf L}>>} & 0,34 & 0,41 & 1\hphantom{,99} & 0 \\
%        \hline
Нажата <<{\sf O}>> & 0,23 & 0,30 & 0,80 & 1 \\
\hline
\end{tabular}}
\end{center}

\setcounter{figure}{1}
\begin{figure*}[b] %fig2
\vspace*{9pt}
 \begin{center}
 \mbox{%
 \epsfxsize=163.641mm
 \epsfbox{mas-2.eps}
 }
 \end{center}
 \vspace*{-6pt}
    \Caption{ROC-кривые (TPR~--- true positive rate; FPR~--- false positive
    rate) для набора данных Si6~(\textit{а}) и для данных, собранных авторами~(\textit{б}):
    \textit{1}~--- случайный лес; \textit{2}~--- дерево решений (алгоритм С5.0)}
    \label{roc_si6}
\end{figure*}

\addtocounter{table}{1}

\noindent
\begin{center}  %fig1
\vspace*{3pt}
  \mbox{%
 \epsfxsize=77.374mm
 \epsfbox{mas-1.eps}
 }
 \vspace*{6pt}
{{\figurename~1}\ \ \small{Набор слова <<{\sf Hello}>>}}
 \end{center}

\addtocounter{figure}{1}

\subsection{Сбор данных}

Для проверки качества предлагаемой модели представления был
использован уже описанный выше набор данных Si6~\cite{si6},
состоящий из 55~пользователей.

Кроме этого, в рамках исследовательской работы был проведен
независимый сбор данных. Сорока испытуемым было предложено установить
про\-грам\-му-агент, которая в фоновом режиме собирала информацию о
нажатиях во время их обычной работы (преимущественно с текстовым
редактором и веб-брау\-зе\-ром). Весь эксперимент был  разделен на 2
дня, в каждый из которых условия или оборудование для  испытуемого
оставались неизменными. Таким образом, для пользователя
регистрировались две сессии длительностью 3~ч в первый день и 6~ч 
во второй. В~среднем на одного испытуемого приходилось порядка
24~тыс.\ нажатий.

\subsection{Статическая аутентификация с~предлагаемой моделью поведения}

Для проверки нового представления данных был проведен эксперимент по
статической аутентификации. Все исходные данные обрабатывались
потенциальным функтором, и для анализа использовались только векторы
значений потенциальной функции, полученные в результате.

Для каждого пользователя все данные в наборе разделялись на <<свои>>
и <<чужие>> (по принадлежности вектора этому пользователю). 
В~качестве тренировочного набора бралось по 1500~векторов, полученных
случайной выборкой из <<своих>> и <<чужих>> данных (всего 
3000~векторов), в качестве тес\-то\-вой выборки из каждого набора бралось по
1000 векторов (всего 2000).

На тренировочном наборе было обучено 2~классификатора:
\begin{enumerate}[(1)]
\item \textbf{дерево решений.} В качестве реализации был выбран алгоритм 
See5/C5.0 \cite{c50};
\item \textbf{случайный лес} (ансамбль деревьев решений). Была 
взята реализация алгоритма из~\cite{rf}. В~данном эксперименте использовался 
ансамбль из 500~деревьев решений.
\end{enumerate}

Далее каждому классификатору на вход подавались векторы из тестового
набора, для каждого из которых он выдавал вероятность принадлежности
данного события классу <<свой>>.

\setcounter{figure}{2}
\begin{figure*} %fig3
\vspace*{9pt}
 \begin{center}
 \mbox{%
 \epsfxsize=162.085mm
 \epsfbox{mas-4.eps}
 }
 \end{center}
 \vspace*{-6pt}
    \Caption{Распределение EER для данных Si6~(\textit{а}) и
    для данных, собранных авторами~(\textit{б}) }
    \label{eer_si6}
\end{figure*}

На основании этого алгоритма было проведено 2~эксперимента:
\begin{enumerate}[(1)]
\item первый эксперимент был проведен на наборе 
данных Si6~\cite{si6}. ROC-кри\-вые (Receiver Operating Characteristics)
для этого набора представлены на рис.~\ref{roc_si6},\,\textit{а}. 
Как видно из распределений на рис.~3,\,\textit{а}, наилучший EER достигался при 
использовании ансамбля деревьев решений и составлял в среднем 6\%.  В~этом 
подходе $F$-ме\-ра оказалась равной 0,94;
\item второй эксперимент был проведен на данных, собранных авторами статьи 
(рис.~2,\,\textit{б}). Здесь показатели несколько хуже и, как видно из распределений 
на рис.~3,\,\textit{б}, EER в среднем составляет 13\%.
\end{enumerate}

\subsection{Непрерывная динамическая аутентификация с~предлагаемой моделью представления}

Для проверки качества представления данных был проведен эксперимент
по непрерывной динамической аутентификации, использующий функцию
на\-ка\-за\-ния-поощре\-ния.

В качестве алгоритма интеллектуального анализа данных были
использованы деревья решений в реализации C5.0~\cite{c50}.

Функция наказания-поощрения, записанная выше, применительно к
данному представлению (отсутствие данных о нажатии какой-либо
клавиши не является особым случаем, который надо рас\-смат\-ри\-вать
отдельно) принимает следующий вид:
$$ C =
\begin{cases}
    0\,, &\ \mbox{начало\ сессии}\,; \\
    \max(C - R, 0)\,, &\ d \le T\,; \\
    C + d - T\,, &\ d > T\,,
\end{cases}
$$
где $d$~--- вероятность того, что очередной вектор соответствует шаблону 
пользователя, выданная классификатором.

\subsubsection{Постановка эксперимента}

В рамках эксперимента моделировалась ситуация, при которой каждый
пользователь делал попытку аутентификации под всеми другими
пользователями.

В каждом случае классификатор обучался на 2000~векторов
пользователя, под видом которого происходила аутентификация, и на
2000 случайных векторов всех других пользователей (кроме того,
который производил попытку аутентификации). Таким образом, атакующий
пользователь никогда не принимал участия в обучении классификатора.

Валидационный набор представлял собой 1000~дополнительных векторов
атакуемого пользователя. Порог уровня доверия, при превышении
которого пользователь считался нарушителем, был взят в 10~раз выше,
чем максимальный уровень доверия на валидационном наборе.

На рис.~\ref{conf_friend} показаны уровни доверия
для вали\-дационного набора легитимного пользователя и тес\-то\-во\-го
набора атакующего пользователя соответственно. Обратите внимание на
различие в еди\-ни\-цах измерения порога. В~данном примере атака была
определена на 212-м нажатии. Заметим, что уровень доверия
легитимного пользователя не превосходит 1,4 на протяжении 
1000~нажатий. Доверие тестового набора с определенного момента только
увеличивается.

Таким образом, для каждого из 55~пользователей было получено 
54~номера нажатия, на которых уровень доверия превышал указанный порог.
Для оценки качества алгоритма был взят средний номер нажатия в
каждой выборке.

\begin{figure*} %fig4
\vspace*{9pt}
 \begin{center}
 \mbox{%
 \epsfxsize=161.908mm
 \epsfbox{mas-7.eps}
 }
 \end{center}
 \vspace*{-6pt}
    \Caption{График уровня доверия для легитимного пользователя~(\textit{а}) и 
    для атакующего набора~(\textit{б})}
    \label{conf_friend}
    \vspace*{6pt}
\end{figure*}

Аналогичным образом был проведен эксперимент и на данных, собранных
авторами. Результаты показаны на рис.~5. В~половине случаев
наруши-\linebreak

\vspace*{-1pt}

\begin{center}  %fig5
  \mbox{%
 \epsfxsize=75.76mm
 \epsfbox{mas-6.eps}
 }
 \end{center}
 \vspace*{3pt}
{{\figurename~5}\ \ \small{Количество нажатий, необходимое для обнаружения нарушителя}}
% \end{center}

\vspace*{12pt}

\noindent
тель был определен уже после 50-го и 100-го нажатия (для Si6 и
данных, собранных авторами, соответственно).

\vspace*{-2pt}

\section{Заключение}

В статье проведен обзор методов статической и динамической
аутентификации с учетом динамики нажатий клавиш. Среди методов
статической аутентификации рассмотрены подходы, учитывающие время
удержания клавиш, порядок нажатия и отпускания кнопок; методы,
принимающие решение об успешности аутентификации на основе
относительной скорости печати и использовании левой и правой клавиш
{\sf Shift}. Кроме того, рассмотрен подход, дающий высокие
результаты на коротких паролях, и оценено влияние информированности
пользователя на результаты.

В рамках задачи динамической аутентификации рассматривался метод,
основанный на кластеризации, метод, который использует комбинацию
метрик расстояния между очередным нажатием и шаблоном пользователя,
показавший хорошие результаты. Также рассматривался подходящий для
задачи непрерывного контроля пользователя алгоритм, применяющий
функцию на\-ка\-за\-ния-по\-ощре\-ния.

Как альтернатива существующим методам была предложена модель
представления данных, основанная на потенциальных функциях вместе с
алгоритмами, на которых ее применение дает высокие результаты
(алгоритмы деревьев решений C5.0 и случайный лес). Для задачи
динамической аутентификации был использован метод поощрения и
наказания. Для проверки эффективности сочетаний модели и
перечисленных алгоритмов были проведены эксперименты на данных из
набора Si6 и данных, собранных авторами в рамках этой
исследовательской работы.

Проведенные эксперименты показали применимость предложенного подхода
для динамической аутентификации. Таким образом, авторами статьи было
признано целесообразным создание экспериментальной системы
аутентификации, основанной на механизме, предложенном в ней.

{\small\frenchspacing
{%\baselineskip=10.8pt
\addcontentsline{toc}{section}{Литература}
\begin{thebibliography}{99}



\bibitem{lao}
\Au{Lau~E., Liu~X., Xiao~C., Yu~X.} Enhanced user authentication
through keystroke biometrics.~--- Massachusetts Institute of
Technology, 2004.

\bibitem{saggio}
\Au{Saggio~G., Costantini~G., Todisco~M.} Cumulative and ratio
time evaluations in keystroke dynamics to improve the password
security mechanism~// J.~Computer Information
Technol., 2011.  Vol.~1. No.\,2. P.~4--11.

\bibitem{svm}
\Au{Sung~K.~S., Cho~S.} GA SVM  wrapper ensemble for keystroke
dynamics authentication~//  Conference (International) on
Biometrics Proceedings.~--- Hong Kong: ICB, 2004. P.~654--660.

\bibitem{legett}
\Au{Leggett~J., Williams~G.} Verifying identity via keystroke
characteristics~// Int. J.~Man-Machine Studies,
1988. Vol.~28. No.\,1. P.~67--76.

\bibitem{hocquet}
\Au{Hocquet~S., Ramel~J.-Yv., Cardot~H.} Fusion of methods for
keystroke dynamic authentication~// Automatic Identification
Advanced Technologies: 4th IEEE Workshop Proceedings.~--- Buffalo, 2005.
P.~224--229.

\bibitem{bergadano} %6
\Au{Bergadano~F., Gunetti~D., Picardi~C.} User authentication
through keystroke dynamics~// ACM Trans. Information 
Syst. Security (TISSEC), 2002. Vol.~5. No.\,4. P.~367--397.

\bibitem{kittler}
\Au{Kittler~J., Hatef~M., Duin R.~P.~W., Matas.~J.} On combining
classifiers~// Pattern Analysis and Machine Intelligence, IEEE
Trans., 1998. Vol.~20. No.\,3. P.~226--239.

\bibitem{bert}
\Au{Bertacchini~M., Benitez~C., Fierens~P.\,I.} User clustering
based on keystroke dynamics~// XVI Congreso Argentino de Ciencias de
la Computaci$\acute{\mbox{o}}$n. -- Mor$\acute{\mbox{o}}$n, 2010. P.~832--841.

\bibitem{si6}
\Au{Bello~L., Benitez~C., Bertacchini~M., Pizzoni~J.\,C.,
Cipriano~M.} Collection and publication of a fixed text keystroke
dynamics dataset~// XVI Congreso Argentino de Ciencias de la
Computaci$\acute{\mbox{o}}$n. -- Mor$\acute{\mbox{o}}$n, 2010. P.~822--831.

\bibitem{sil}
\Au{Rousseeuw~P.\,J.} Silhouettes: A~graphical aid to the
interpretation and validation of cluster analysis~// J.~Comput. Appl. Math., 
1987. Vol.~20. P.~53--65.

\bibitem{gunet}
\Au{Gunetti~D., Picardi~C.} Keystroke analysis of free text~//
ACM Trans. Information Syst. Security (TISSEC), 2005.
Vol.~8. No.\,3. P.~312--347.

\bibitem{nisk}
\Au{Monrose~F., Rubin~A.\,D.} Keystroke dynamics as a biometric
for authentication~// Future Generation Computer Syst., 2000.
Vol.~16. No.\,4. P.~351--359.

\bibitem{aizerman}
\Au{Айзерман~М.\,А., Браверман~Е.\,М., Розоноэр~Л.\,И.} Метод потенциальных
функций в теории обучения машин.~--- M.: Наука, 1970.

\bibitem{c50}
\Au{Quinlan~J.\,R.} C4.5: Programs for machine learning.~--- Morgan Kaufmann, 1993.

\label{end\stat}

\bibitem{rf}
\Au{Breiman~L.} Random forests~// Machine Learning, 2001.
Vol.~45. No.\,1. P.~5--32.
\end{thebibliography}
} }

\end{multicols} %5  

\def\stat{kalenov}

\def\tit{ПРОБЛЕМЫ СЕТЕВОГО ДОСТУПА К НАУЧНЫМ ЖУРНАЛАМ}

\def\titkol{Проблемы сетевого доступа к научным журналам}

\def\autkol{А.\,В.~Глушановский, Н.\,Е.~Калёнов}

\def\aut{А.\,В.~Глушановский$^1$, Н.\,Е.~Калёнов$^2$}

\titel{\tit}{\aut}{\autkol}{\titkol}

%{\renewcommand{\thefootnote}{\fnsymbol{footnote}}\footnotetext[1] {Статья 
%рекомендована к публикации в журнале Программным комитетом конференции 
%<<Электронные библиотеки: перспективные методы и технологии, электронные 
%коллекции>> (RCDL-2012).}}

\renewcommand{\thefootnote}{\arabic{footnote}}
\footnotetext[1]{Библиотека по естественным наукам Российской академии наук, 
avglush@benran.ru} 
\footnotetext[2]{Библиотека по естественным наукам 
Российской академии наук, nek@benran.ru}



\Abst{Рассматриваются проблемы организации сетевого доступа российских ученых к 
научным журналам и базам данных. В соответствии с мировой практикой организацию 
такого доступа осуществляют научные биб\-лио\-те\-ки, объединяющиеся в 
консорциумы для получения выгодных финансовых условий. Описывается существующая 
в России практика организации доступа к зарубежным научным ресурсам через 
посредство Российского
фонда фундаментальных исследований (РФФИ) и <<Национального электронно-информационного 
консорциума>> (НЭИКОН). Приведена статистика востребованности пользователями 
Российской академии наук (РАН) научных журналов, предоставляемых через НЭИКОН. 
Предложены организационные действия для решения задачи оптимизации доступа к 
коммерческим сетевым научным ресурсам в условиях существующих в РАН финансовых 
ограничений.}


\KW{научные журналы; информация; Интернет; удаленный доступ; библиотеки; 
консорциум}

\vskip 14pt plus 9pt minus 6pt

      \thispagestyle{headings}

      \begin{multicols}{2}

            \label{st\stat}

     Анализ информационных потребностей ученых РАН показывает, что 
по-прежнему одним из важнейших источников научной информации для них 
остаются научные журналы (в первую очередь~--- иностранные). 
В~настоящее время наряду с традиционной печатной формой все более 
широкое распространение получил доступ к научным журналам через сеть 
Интернет.
     
     Технически такой доступ не представляет затруднений, что создает 
впечатление легкой доступности полных текстов статей. На самом деле все 
обстоит несколько сложнее. Большинство ведущих зарубежных издательств 
и научных обществ, таких как Elsevier, Springer, American Physical Society, 
American Chemical Society и других, представляют в свободном доступе 
только биб\-лио\-гра\-фи\-че\-скую информацию (описание статей) и (в лучшем 
случае) их рефераты. Доступ к полным текстам является платным и требует 
заключения соответствующего договора с издательством, причем суммы 
таких договоров, во-пер\-вых, весьма значительны, а во-вто\-рых, весьма 
заметно варьируются в зависимости от числа пользователей в организации, 
количества подключаемых компьютеров и ряда других па\-ра\-мет\-ров.
     
     Организация сетевого доступа к коммерческим\linebreak
      источникам научной 
информации требует значительной по объему и сложности специфической\linebreak 
работы, связанной с выбором нужных ресурсов, проведением переговоров с 
поставщиками, согласованием условий предоставления ресурсов, 
заключением контрактов и оформлением лицензионных соглашений, 
предоставлением IP-ад\-ре\-сов и контролем выполнения договорных 
обязательств. Для научных сотрудников такая деятельность не является 
характерной, поэтому сложившаяся мировая практика организации сетевого 
доступа к научной информации состоит в том, что ею занимаются 
биб\-лио\-те\-ки университетов, научных центров и других научных и учебных 
организаций. Биб\-лио\-те\-ки в силу специфики своей деятельности лучше 
знакомы с издательским миром, имеют опыт взаимодействия как с 
издательствами, так и с пользователями информации, и работа по 
информационному обеспечению научных исследований является их прямой 
обязанностью 
     
     Подобная практика сложилась и в России, в частности в РАН. 
Центральные академические биб\-лио\-те\-ки (такие как Библиотека Российской 
академии наук (БАН) в Санкт-Пе\-тер\-бур\-ге, Биб\-лио\-те\-ка по естественным наукам 
РАН (БЕН РАН) в Москве, Государственная публичная научнотехническая биб\-лио\-те\-ка Сибирского 
отделения РАН (\mbox{ГПНТБ} СО РАН)
 в Новосибирске, Центральная научная биб\-лио\-те\-ка 
Уральского отделения РАН (ЦНБ УрО РАН) в Екатеринбурге, Центральная научная 
биб\-лио\-тека Дальневос\-точ\-ного отделения РАН (ЦНБ ДвО РАН) во Вла\-ди\-востоке), 
обеспечивающие информационные потребности многих институтов РАН, 
тематика исследований которых в значительной мере пересекается, могут 
получить\linebreak значительно более выгодные условия доступа к \mbox{научным} журналам 
и базам данных (БД), нежели отдельные институты, заключающие 
самостоятельные договора с поставщиками. Кроме того, биб\-лио\-те\-ки (и/или 
их объединения~--- консорциумы) берут на себя организационные вопросы 
(переговоры с издательствами, заключение и оплата договоров, оформление 
лицензионных соглашений, сбор IP-адресов и организацию их подключения 
и~т.\,д.). Библиотеки также ведут анализ фактического использования 
доступа и оптимизируют подписку (в условиях жестких финансовых 
ограничений) для своих систем в целом.
 %    
     Как принято в мировой практике, для оптимизации финансовых 
условий доступа биб\-лио\-те\-ки объединяются в консорциумы, выступающие 
как единое юридическое лицо в отношениях с из\-да\-ющи\-ми организациями 
(или поставщиками ресурсов).
     
     Обычно в консорциумы объединяются биб\-лио\-те\-ки исходя из двух 
положений~--- либо предоставить узкотематическую информацию как можно 
более широкому кругу пользователей (объединяются организации с 
близкими научными интересами) и получить скидки (в расчете на одного 
участника консорциума) за счет значительного числа пользователей данного 
ресурса, либо предоставить пользователям консорциума как можно более 
широкий спектр информационных ресурсов (объединяются организации с 
различными тематическими интересами) и получить скидки (в расчете на 
одного участника консорциума) за счет увеличения объема предоставляемых 
ресурсов.
     
     В мире существует значительное число различных биб\-лио\-теч\-ных 
консорциумов. Международное объединение биб\-лио\-теч\-ных консорциумов 
(The International Coalition of Library Consortia~--- ICOLC)~[1] объединяет 
более 200~биб\-лио\-теч\-ных консорциумов, созданных на основе коалиций 
биб\-лио\-тек по тематическому или территориальному принципу. 
Консорциумы бывают разной величины и типа. Например, в Финляндии 
практически все университетские биб\-лио\-те\-ки, биб\-лио\-те\-ки научных 
учреждений и пуб\-лич\-ные биб\-лио\-те\-ки объединены в FinELib~[2]~--- 
национальный консорциум, по\-став\-ля\-ющий более 70\% всей электронной 
информации~[3]. Различные типы европейских биб\-лио\-теч\-ных консорциумов 
описаны в~[4].

В России в 1990--2000-е~гг.\ сложилась аналогичная практика 
организации доступа к научным журналам~[5]. С~1997~г.\ такой доступ 
предоставлялся в рамках консорциума, созданного по инициативе БЕН РАН 
и включавшего \mbox{РФФИ} и 
14~крупнейших научных биб\-лио\-тек. Финансирование консорциума 
осуществлял \mbox{РФФИ} в рамках принятой в конце 1996~г.\ <<Программы 
поддержки российских научных биб\-лио\-тек>>. В~рамках этой программы 
была создана научная электронная биб\-лио\-те\-ка (НЭБ). В~соответствии с 
принципами ее организации электронные версии журналов поступали из 
издательств в \mbox{РФФИ} и загружались на специальный сервер НЭБ и его 
зеркала в Казани и Новосибирске. Доступ предоставлялся всем 
пользователям биб\-лио\-тек, входящих в консорциум, а поскольку в 
консорциум входили все центральные академические биб\-лио\-те\-ки (БАН, БЕН 
РАН, \mbox{ГПНТБ} СО РАН, ЦНБ УрО РАН и ЦНБ ДвО РАН), любой сотрудник 
Академии наук мог читать основные научные журналы мира. К~началу 
2002~г.\ на серверы НЭБ было загружено около 2000~наименований (около 
75\,000~выпусков) журналов наиболее значимых научных издательств 
мира~[6]. Научная электронная библиотека пользовалась большой популярностью у специалистов~--- за 
год в начале \mbox{2000-х}~гг.\ из нее выгружалось около четверти миллиона 
статей. 
     
     Соглашение о консорциуме НЭБ, подписанное РФФИ и ведущими 
биб\-лио\-те\-ка\-ми, сопровождалось рядом условий, выдвинутых издательствами 
и направленных, в частности, на сохранение перечня приобретаемых 
биб\-лио\-те\-ка\-ми печатных версий журналов (по условиям участия в 
консорциуме организация должна была выписать для себя в печатном виде 
не менее 5~журналов, не входящих в подписку консорциума~[7]). Имелся (и 
сохраняется до сих пор во всех подобного рода консорциумах) ряд 
ограничений на выгрузку и распространение полученных текстов 
(запрещается сплошное копирование номера журнала, распространение 
полученных материалов за пределами ор\-га\-ни\-за\-ции-участ\-ника).
     
     К сожалению, в 2004~г.\ НЭБ РФФИ прекратила свое существование в 
том виде, который преду\-смат\-ри\-вал\-ся соглашениями 1996~г. Причинами 
этого стали несколько факторов, в частности проверка РФФИ со стороны 
Счетной палаты. Проверка вы\-яви\-ла нарушения Устава РФФИ, согласно 
которому последний не имеет права финансировать что-ли\-бо без 
проведения конкурсов. Это по\-влек\-ло за собой проблемы финансирования 
поддержки технологии функционирования НЭБ (обработка и загрузка 
массивов данных, поддержка серверов). 

С~вступлением в силу 94-го 
Федерального закона о закупках фактически были ликвидированы 
механизмы координированной работы биб\-лио\-тек по приобретению научных 
ресурсов. Из-за распада консорциума наиболее значимые научные 
издательства отказались передавать журналы российской стороне. 
В~результате уже загруженные на сервер журналы НЭБ были юридически 
переданы ООО <<Научная электронная биб\-лио\-те\-ка>> с условием 
бесплатного предоставления на ее сервере ({\sf http://www.elibrary.ru}), чем в 
настоящее время могут пользоваться российские ученые.
     
     Российский фонд фундаментальных исследований заключил новые договора с рядом зарубежных издательств о 
доступе к их журналам, но уже в режиме онлайн и только для своих 
грантодержателей.
     
     С этого периода и по настоящее время в России существуют два 
основных централизованных канала сетевого доступа учреждений РАН к 
зарубежной\linebreak научной информации~--- за счет \mbox{РФФИ} при посредстве 
Внешнеэкономического
объединения <<Академинторг>> и за счет средств Мин\-обр\-на\-у\-ки при посредстве 
\mbox{НЭИКОН}. 
Кроме централизованных\linebreak источников подписки в масштабах страны доступ 
к зарубежным научным журналам и БД приобретают 
вышеперечисленные центральные биб\-лио\-те\-ки РАН за счет средств, 
выделяемых Президиумом РАН и руководством ее региональных отделений, 
а также некоторые академические институты за счет своих средств. Однако 
количество ресурсов, приобретаемых академическими организациями, в 
десятки раз меньше количества ресурсов, приобретаемых РФФИ и НЭИКОН. 
     
     В настоящее время РФФИ финансирует своим грантодержателям (на 
уровне организаций, через которые осуществляется оплата средств по 
грантам) доступ к журналам шести издательств: Wiley (1600~журналов), The 
American Mathematical Society (предоставляется реферативная БД MathSciNet
(MSN), включающая около двух миллионов описаний статей), American 
Physical Society (9~журналов), Institute of Physics~(49 журналов), The Royal 
Society of Chemistry (6~журналов), Elsevier (Freedom Collection~--- около 
1700~журналов). До 2011~г.\ предо\-став\-лял\-ся также доступ к журналам 
издательства Springer, но в 2012~г.\ \mbox{РФФИ} отказался от этой подписки, 
мотивируя это решение нехваткой финансовых средств (одновременно было 
сокращено число доступных журналов The Royal Society of Chemistry с 23 
до~6). С~2011~г.\ грантодержателям \mbox{РФФИ} стали доступны журналы одной 
из коллекций издательства Elsevier (Freedom Collection~--- более 
1700~журналов).
     
     За счет средств РФФИ также организован доступ пяти крупнейших 
академических биб\-лио\-тек к известной БД Web of Knowledge, которая широко 
используется для определения публикационной активности и уровня 
цитирования научных публикаций.
     
     Следует заметить, что, лишившись в 2012~г.\ доступа к текущим 
журналам издательства Springer, пользователи РФФИ лишились и доступа к 
журналам предыдущих лет издания, подписка на которые была ранее 
оплачена. Согласно условиям контракта, при прекращении подписки для 
доступа к ранее оплаченным журналам каждая организация должна 
заплатить поставщику определенную сумму в качестве компенсации затрат 
на поддержку его серверов. 
     
     Как указывалось выше, каждый поставщик в зависимости от суммы 
контракта формулирует свои условия предоставления доступа к своим 
ресурсам. В~частности, ограничивает число пользователей, IP-ад\-ре\-сов или 
количество доступных журналов. Это обусловливает ограничения для 
грантодержателей \mbox{РФФИ} в получении доступа к сетевым ресурсам. Каж\-дый 
грантодержатель в начале 2012~г.\ должен был выбрать из предложенного 
\mbox{РФФИ} списка от одного до четырех издательств, журналы которых ему 
необходимы, и сообщить о своем выборе \mbox{РФФИ}. Последний, в зависимости 
от возможностей, диктуемых контрактами, принимал окончательное 
решение, кому и какие ресурсы предоставить. 
     
     <<Национальный электронно-информационный консорциум>>, 
     включающий в свой состав несколько сот организаций 
науки и образования, предо\-став\-ля\-ет им в 2012~г.\ за счет средств 
Министерства образования и науки доступ к полным текстам журналов 
следующих издательств, представляющих интерес для РАН: American 
Chemical Society (ACS~--- 38~журналов), American Institute of Physics 
     (AIP~--- 10~журналов), Annual Reviews Sciences Collection (AR~--- 
37~журналов), Business Source Complete (BSC~--- около 3500~журналов), 
Computers \& Applied Sciences Complete (CASC~--- около 950~журналов), 
Nature Publishing Group (NPG~--- 8~журналов), Oxford University Press 
(OUP~--- более 200~журналов), Optical Society of America (OSA~--- 
14~журналов), Sage STM (Science, Technology \& Medicine~--- более 
100~журналов), SPIE~--- International Society for Optics and Photonics 
(6~журналов и материалы конференций), Taylor \& Francis (T\&F~--- более 
1000~журналов), Georg Thieme Verlag KG (Thieme~--- 5~журналов) The 
American Association for the Advancement of Science (AAAS~--- журнал 
Science).
     
     Для сравнения~--- БЕН РАН на средства, выделенные ей Президиумом 
РАН в рамках целевого финансирования на приобретение научной 
литературы в 2011~г., смогла приобрести права сетевого доступа на 2012~г.\ 
лишь к 142~наименованиям журналов, отсутствующих в списках \mbox{РФФИ} и 
\mbox{НЭИКОН} (по соглашениям с поставщиками доступ предоставляется не 
только из центрального здания БЕН РАН, но и из ее отделов, расположенных 
в научных учреждениях РАН).
     
<<Национальный электронно-информационный консорциум>>, 
работая по контракту с Минобрнауки, уделяет серьезное 
внимание анализу использования ресурсов, предоставляемых научным 
организациям. Эту работу, касающуюся академических учреждений, с 
2010~г.\ по договору с НЭИКОН проводит БЕН РАН. В~ходе проводимого 
анализа был получен ряд интересных предварительных (работы 
заканчиваются в 2013~г.)\  результатов, которые приведены ниже.



 \begin{figure*}[b]
     \vspace*{1pt}
 \begin{center}
 \mbox{%
 \epsfxsize=99mm
 \epsfbox{glu-1.eps}
 }
 \end{center}
 \vspace*{-6pt}
\begin{center}
{\small Распределение числа выгрузок по издательствам}
\end{center}
     \end{figure*}

     
     По 14 издательствам, используемым в учреждениях РАН в 2010~г., в 
среднем в месяц выгружалось\linebreak\vspace*{-12pt}

\pagebreak

\begin{center}
 \vspace*{-6pt}
{{\tablename~1}\ \ \small{Активность использования ресурсов}}

\vspace*{6pt}

      \begin{tabular}{|l|c|c|}
      \hline
\multicolumn{1}{|c|}{Ресурс}&\tabcolsep=0pt\begin{tabular}{c}Количество\\ журналов\end{tabular} &
\tabcolsep=0pt\begin{tabular}{c}Количество\\ выгрузок\\ в месяц\end{tabular}\\
\hline
ACS &\hphantom{9}38&23\,806\hphantom{9}\\
AIP &\hphantom{9}10 &12\,930\hphantom{9}\\
NPG & \hphantom{99}8 &6378\\
T\&F &1547\hphantom{9}&3236\\
AAAS (Science) &\hphantom{99}1 &3015\\
OSA &\hphantom{9}14&2505\\
OUP\_Full &217&2106\\
Thieme &\hphantom{99}5&2035\\
SPIE &\hphantom{99}6&1568\\
Cell &\hphantom{9}15 &1286\\
Annual Review &\hphantom{9}37&\hphantom{9}402\\
SAGE &382&\hphantom{9}240\\
ACM &420&\hphantom{99}91\\
BSC &3345\hphantom{9}&\hphantom{99}42\\
\hline
\end{tabular}
\end{center}

%\pagebreak

\vspace*{12pt}

   


\addtocounter{table}{1}
\setcounter{figure}{0}

\noindent
 59\,642~статьи. Ресурсы использовались 
186~организациями РАН (журнал Science~--- 111~организаций, журналы 
AIP~--- 106~организаций, журналы ACS~--- 
82~организации, журналы группы Nature (NPG~--- Nature Publishing Group)~--- 
70~организаций и~т.\,д.). В целом, в 2010~г.\ активность учреж\-де\-ний РАН, 
измеряемая числом выгрузок полных текстов статей в месяц, выглядит 
следующим образом (табл.~1).
     

     
     По данным табл.~1 представлен график (см.\ рисунок).
     
    
     Данный график имеет две точки перегиба (после NPG 
и после группы журналов Cell). Суммарное число выгрузок до первой точки 
перегиба составляет 72\% от общего количества статей, выгруженных РАН, а 
до второй~--- 97\%.
     
      Наибольшим спросом у ученых РАН пользуются журналы ACS, AIP и NPG. 
Наименьшим спросом~--- журналы издательств Business Source Complete, 
Association for Computing Machinery (ACM), Sage и Annual Review. В средней 
части таблицы~--- пользующиеся, тем не менее, заметным спросом журналы 
издательств T\&F, AAAS (Science), OSA, 
OUP, Thieme, SPIE и Cell. 

\begin{table*}[b]\small
\vspace*{-6pt}
\begin{center}
\Caption{Использование ресурсов ОНИТ РАН}
\vspace*{2ex}

\begin{tabular}{|l|c|}
\hline
\multicolumn{1}{|c|}{Ресурс}&Число выгрузок
в месяц\\
\hline
American Institute of Physics (AIP)&413\hphantom{9}\\
Optical Society of America (OSA)&331\hphantom{9}\\
Society of Photographic Instrumentation Engineers (SPIE) &94\\
American Chemical Society (ACS)&42\\
AAAS (Science)&36\\
Nature&31\\
Sage &18\\
Nature Physics &17\\
Nature Nanotechnology&13\\
Association for Computing Machinery (ACM)&10\\
Nature Photonics &\hphantom{9}8\\
Nature Materials&\hphantom{9}7\\
Nature Chemistry &\hphantom{9999}0,25\\
Nature Methods &\hphantom{9999}0,17\\
Business Source Complete &\hphantom{9}0\\
Cell&\hphantom{9}0\\
Nature Biotechnology &\hphantom{9}0\\
Oxford University Press. Mathematics \& Computing &\hphantom{9}0\\
Oxford University Press BioMed &\hphantom{9}0\\
Oxford University Press Life &\hphantom{9}0\\
Oxford University Press Med &\hphantom{9}0\\
Oxford University Press STM &\hphantom{9}0\\
Taylor \& Francis Bio &\hphantom{9}0\\
Taylor \& Francis Chem &\hphantom{9}0\\
Taylor \& Francis Earth &\hphantom{9}0\\
Taylor \& Francis. Natural Sciences &\hphantom{9}0\\
Taylor \& Francis Med &\hphantom{9}0\\
Taylor \& Francis. Other &\hphantom{9}0\\
Taylor \& Francis. Physics \& Mathematics &\hphantom{9}0\\
Taylor \& Francis. Technique&\hphantom{9}0\\
\hline
\end{tabular}
\end{center}
\end{table*}
     
     До настоящего времени доступ к журналам, выписываемым через 
НЭИКОН, предоставлялся бесплатно. Со второй половины 2012~г.\ 
НЭИКОН планирует (по указанию Минобрнауки) взимать часть стоимости 
ресурсов с получателей. В~этой ситуации станет актуальным вопрос, не 
дешевле ли будет вместо оплаты доступа к базам малоспрашиваемых 
журналов заказывать электронные копии отдельных статей с 
<<постатейной>> оплатой. Этот вид сервиса достаточно хорошо развит за 
рубежом, им пользуются как отдельные ученые, так и научные биб\-лио\-те\-ки 
по заказам своих пользователей. Как правило, к оглавлениям и аннотациям 
статей из научных журналов предоставляется свободный доступ. Функции 
<<посредника>>, осуществляющего прием заказов на статьи от ученых РАН, 
контакты с поставщиками, оплату заказов в валюте могли бы взять на себя 
центральные академические биб\-лио\-те\-ки. По данным БЕН РАН, стоимость 
электронной копии статьи объемом до 20~страниц в биб\-лио\-те\-ках 
континентальной Европы составляет в среднем около 10~евро, что при 
годовых объемах 50--60~статей будет существенно дешевле, чем оплата 
доступа ко всем журналам издательства, не являющегося приоритетным для 
РАН. Очевидно, что такой подход потребует некоторого перераспределения 
средств и организационной перестройки биб\-лио\-теч\-ных служб, но он может 
оказаться достаточно эффективным.
     
     <<Национальный электронно-информационный консорциум>> достаточно оперативно реагирует на изменения 
спроса на журналы. Так, по \mbox{результа\-там} проведенного анализа с 2011~г.\ 
была прекращена подписка на журналы ACM. 
Что касается журналов издательств Business Source Complete, они 
пользуются значительным спросом у второй большой группы 
     ор\-га\-ни\-за\-ций--поль\-зо\-ва\-те\-лей \mbox{НЭИКОН}: российских университетов. 
Издательство Sage, которое также оказалось в нижней части рейтинговой 
таб\-ли\-цы, в 2010~г.\ было пред\-став\-ле\-но в \mbox{НЭИКОН} только журналами по 
гуманитарным и социальным наукам, хотя выпускает оно и другую научную 
литературу. В~настоящее время НЭИКОН предполагает дополнить этот 
ресурс журналами группы Sage STM (Science, Technology \& Medicine), и, 
возможно, результаты его востребованности академическими организациями 
изменятся.
     
     Получив полные данные о спросе на журналы НЭИКОН, авторы статьи 
провели анализ их использования сотрудниками отделений РАН. В~табл.~2 
представлены показатели Отделения нанотехнологий и информационных 
технологий (ОНИТ). 


     
     В табл.~2 более подробно, чем в предыдущей, раскрыты журналы 
группы Nature, а также журналы издательств Taylor \& Francis и Oxford 
University Press, поэтому таблица формально включает 30~ресурсов, но 
фактически это те же 14~ресурсов, раскрытых более подробно.
     
     Как видно из табл.~2, наибольший интерес для сотрудников ОНИТ 
     представляют журналы  AIP и OSA. Далее со 
значительным отрывом следуют\linebreak журна\-лы Society of Photographic 
Instrumentation Engineers. Среднюю группу (30--40~обращений к полным 
текстам в месяц) составляют журналы American Chemical Society, AAAS 
(Science) и основной журнал группы Nature. Значительно меньшим спросом 
пользуются остальные журналы группы Nature, журналы же остальных 
издательств не представляют для ОНИТ никакого интереса. 
     
     Россиский фонд фундаментальных исследований, в отличие от \mbox{НЭИКОН}, не 
     пред\-остав\-ля\-ет пользователям 
статистики использования пред\-став\-ля\-емых Фондом ресурсов (хотя БЕН РАН 
\mbox{обращалась} по этому поводу к руководству \mbox{РФФИ} и получила 
принципиальное согласие, но данные пока не получила), поэтому 
аналогичный анализ по этим ресурсам пока невозможен. Однако, по 
наблюдениям авторов статьи, журналы, предлагаемые через РФФИ, 
пользовались также весьма заметным спросом. Так, статьи из журналов 
издательства Springer в 2011~г.\ только в БЕН РАН выгружались в среднем 
158~раз в месяц, в ГПНТБ СО РАН~--- 489~раз; статьи издательства Elsevier 
в 2011~г.\ выгружались в БЕН РАН в среднем 1536~раз в месяц.
     
     Таким образом, в настоящее время в России созда\-на действующая 
система доступа к полным текстам нескольких тысяч зарубежных научных 
журналов. Эта система охватывает подавляющее большинство научных 
организаций РАН и пользуется значительной популярностью. Однако она не 
охватывает (в первую очередь в силу недостаточного финансирования) весь 
необходимый объем научной информации, требуемый для эффективного 
функционирования научных институтов и центров РАН. В~существующих 
условиях, к сожалению, не представляется возможным обеспечить доступ с 
каждого рабочего места сотрудника РАН ко всем необходимым ему 
журналам. Оптимизировать систему возможно за счет серьезного анализа 
фактического спроса, создания ранжированного списка наиболее 
востребованных журналов, выявления необходимых издательств и 
централизованного (на базе существующих или вновь создаваемых 
консорциумов) заключения договоров с этими из\-да\-тель\-ст\-вами.
     
     Другим фактором оптимизации системы является сокращение числа 
пользователей (IP-ад\-ре\-сов) каждого научного института (научного 
центра), получающих доступ к тому или иному ресурсу, что позволит 
снизить стоимость договоров с поставщиками. В~этом плане представляется 
целесообразным подход, реализованный в БЕН РАН, которая при 
заключении договоров оговаривает права доступа к ресурсам не только из 
здания Центральной биб\-лио\-те\-ки, но и из ее отделов (биб\-лио\-тек) в 
на\-уч\-но-ис\-сле\-до\-ва\-тель\-ских учреждениях (НИУ) РАН. 
При этом поставщикам официально сообщаются IP-ад\-ре\-са биб\-лио\-теч\-ных 
компьютеров, с которых сотрудники институтов могут читать журналы. 
Такая схема работает уже несколько лет и позволяет без существенных 
затрат обеспечивать важнейшей информацией (хотя и не с каждого 
компьютера института) сотрудников более 40~институтов и научных центров 
Москвы и Московского региона. Существенным ограничением этой схемы 
является требование поставщиков, чтобы отделы БЕН в НИУ РАН имели 
компьютеры с выделенными IP-ад\-ре\-са\-ми, поэтому биб\-лио\-те\-кам, 
работающим через про\-кси-сер\-ве\-ры институтов, доступ к ресурсам 
предоставлен быть не может.
     
     Необходимо отметить, что в России и, в част\-ности, в РАН доля 
финансирования, выделяемая на информационное обеспечение науки, 
существенно меньше принятой в развитых и развивающихся странах. 
Согласно мировой практике эта доля составляет от~8\% до~12\% от 
ассигнований на научные исследования. У~нас она не достигает и~1\%.
     
     В существующих условиях, когда биб\-лио\-те\-кам катастрофически не 
хватает централизованно выделяемых РАН средств на приобретение 
информационных ресурсов, необходимо и впредь развивать идеи создания 
академических и межведомственных консорциумов по доступу к научной 
информации, интеграции финансов, выделяемых биб\-лио\-те\-кам, и 
собственных финансов НИУ, перехода на новые системы информационного 
обслуживания пользователей.


{\small\frenchspacing
{%\baselineskip=10.8pt
\addcontentsline{toc}{section}{Литература}
\begin{thebibliography}{9}
     
     
\bibitem{1-g}
The International Coalition of Library Consortia (ICOLC). {\sf 
http://www.library.yale.edu/consortia}.
\bibitem{2-g}
FinELib, the National Electronic Library. The National Library of Finland. {\sf 
http://www.nationallibrary.fi/\linebreak libraries/finelib/finelibconsortium.html}.
\bibitem{3-g}
\Au{H$\ddot{\mbox{o}}$kli E.} Libraries in Finland establish consortia~// Liber 
Quarterly: The J.~European Research Libraries, 2001. Vol.~11. No.\,1. P.~53--59.
\bibitem{4-g}
\Au{Hormia-Poutanen K., Xenidou-Dervou~C., Kupryte~R., Stange~K., 
Kuznetsov~A., Woodward~H.} Consortia in Europe: Describing the various 
solutions through four country examples~// Library Trends, 2006. Vol.~54. No.\,3. 
{\sf https://dspace.lib.cranfield.ac.uk/handle/1826/1014}.
\bibitem{5-g}
\Au{Литвинова Н.\,Н.} Электронные документы: отбор, использование и 
хранение~// Библиотека, 2005. №\,6. С.~6--9.


\label{end\stat}

\bibitem{7-g}
\Au{Никаньшин Д.\,П., Туриянский~И.\,Е., Астафьев~М.\,Н.} О~развитии 
зеркального сервера научной электронной биб\-лио\-те\-ки РФФИ~// 
Исследования по информатике, 2003. Вып.~5. С.~133--142.

\bibitem{6-g}
\Au{Хельферих П., Красикова~О.\,Л.} Научная информация для российских 
биб\-лио\-тек~// Библиотеки и ассоциации в меняющемся мире: новые 
технологии и новые формы сотрудничества: Мат-лы 7-й Междунар. 
конф.~--- Судак, Крым, Украина, 2000.~--- Т.~2. С.~127--128.


\end{thebibliography} } }

\end{multicols} %6
\def\stat{listopad}

\def\tit{ЖИЗНЕННЫЙ ЦИКЛ МЕТОДОЛОГИИ ПОСТРОЕНИЯ РЕФЛЕКСИВНО-АКТИВНЫХ 
СИСТЕМ ИСКУССТВЕННЫХ ГЕТЕРОГЕННЫХ ИНТЕЛЛЕКТУАЛЬНЫХ АГЕНТОВ$^*$}

\def\titkol{Жизненный цикл методологии построения РАСИГИА} %рефлексивно-активных систем искусственных гетерогенных интеллектуальных агентов}

\def\aut{С.\,В.~Листопад$^1$}

\def\autkol{С.\,В.~Листопад}

\titel{\tit}{\aut}{\autkol}{\titkol}

\index{Листопад С.\,В.}
\index{Listopad S.\,V.}


{\renewcommand{\thefootnote}{\fnsymbol{footnote}} \footnotetext[1]
{Исследование выполнено за счет гранта Российского научного фонда №\,23-21-00218, 
{\sf https://rscf.ru/project/23-21-00218/}.}}


\renewcommand{\thefootnote}{\arabic{footnote}}
\footnotetext[1]{Федеральный исследовательский центр <<Информатика и~управ\-ле\-ние>> Российской академии наук, 
\mbox{ser-list-post@yandex.ru}}

%\vspace*{-12pt}

  
  

  \Abst{Представлена темпоральная структура (жизненный цикл) методологии построения 
рефлексивно-активных систем искусственных гетерогенных интеллектуальных агентов (\mbox{РАСИГИА}). 
Такие системы предназначены для компьютерного моделирования процессов и~эффектов, 
возникающих при решении практических проблем коллективами специалистов под 
руководством лица, принимающего решения. Искусственные гетерогенные 
интеллектуальные агенты реф\-лек\-сив\-но-ак\-тив\-ных сис\-тем~--- активные субъекты, 
способные к~рассуждениям, коммуникации и~рефлексии как умению моделировать 
рассуждения других агентов системы и~себя самих. Моделирование рефлексивных процессов 
обеспечивает выработку агентами согласованного представления об объекте управ\-ле\-ния, 
цели коллективной работы и~нормах взаимодействия, позволяя системе в~ходе 
самоорганизации генерировать заново релевантный гибридный интеллектуальный метод 
решения очередной проб\-лемы.} 
  
  \KW{рефлексия; методология; рефлексивно-активная сис\-те\-ма искусственных 
гетерогенных интеллектуальных агентов; гибридная интеллектуальная многоагентная 
система; коллектив специалистов}

\DOI{10.14357/19922264240112}{GUAMVE}
  
%\vspace*{-6pt}


\vskip 10pt plus 9pt minus 6pt

\thispagestyle{headings}

\begin{multicols}{2}

\label{st\stat}

\section{Введение}

  Компьютерное моделирование процессов и~эффектов, возникающих при 
решении практических проблем коллективами специалистов, каждый из 
которых обладает собственным опытом, знаниями и~пониманием предметной 
области,~--- перспективное на\-прав\-ле\-ние научных разработок, которое 
Д.\,А.~Поспелов выделял как одну из десяти горячих точек в~исследованиях по 
искусственному\linebreak интеллекту~[1]. Для компьютерного моделирования 
рас\-суж\-де\-ний коллективов специалистов предлагается создание \mbox{РАСИГИА} 
в~рамках многоагентного подхода~[2] на основе модели 
\mbox{ги\-брид\-ных} интеллектуальных многоагентных сис\-тем~[3]. Агенты 
\mbox{РАСИГИА}~--- активные программные сущности, способные 
рас\-суж\-дать, взаимодействовать и~рефлексировать. Рефлексивное 
моделирование агентами друг друга обеспечивает выработку согласованного 
пред\-став\-ле\-ния об объекте управ\-ле\-ния, \mbox{цели} коллективной работы и~нормах 
взаимодействия, а~также эволюцию \mbox{РАСИГИА} в~ходе 
самоорганизации в~сильном смыс\-ле. В~на\-сто\-ящей работе рас\-смат\-ри\-ва\-ют\-ся 
вопросы создания методологии разработки сис\-тем такого класса, которая 
понимается как учение об организации продуктивной де\-я\-тель\-ности 
в~це\-лост\-ную сис\-те\-му с~чет\-ко определенными характеристиками, логической 
структурой и~процессом ее осуществления (темпоральной структурой)~[4]. 
Характеристики (особенности и~принципы) и~логическая структура (субъект, 
объект, предмет, методы, средства, результат) методологии разработки 
\mbox{РАСИГИА} рас\-смот\-ре\-ны в~[5]. Данная работа по\-свя\-ще\-на разработке 
жизненного цик\-ла (темпоральной структуры) предлагаемой методологии.

\begin{figure*} %fig1
\vspace*{1pt}
      \begin{center}
     \mbox{%
\epsfxsize=148.855mm 
\epsfbox{lis-1.eps}
}
\end{center}
%\vspace*{-9pt}

{\small Темпоральная структура методологии построения РАСИГИА: \textit{1}~--- этап методологии; \textit{2}~--- стадия методологии;
\textit{3}~--- граница фазы методологии; \textit{4}~--- отношение следования при нормальном завершении этапа;  
\textit{5}~--- возврат к~предыдущим этапам при выявлении допущенных на них недочетов}
\end{figure*}

\vspace*{-6pt}
  
\section{Темпоральная структура методологии}

\vspace*{-6pt}

  Укрупненно в~жизненном цикле методологии построения \mbox{РАСИГИА}, 
показанном на рисунке, могут быть выделены проектная, технологическая 
и~рефлексивная фазы, которые со\-сто\-ят из стадий и~этапов. Как видно, 
последовательное выполнение этапов методологии приводит к~же\-ла\-емо\-му 
результату лишь в~идеальном случае, когда проектировщик сразу получает всю 
необходимую достоверную информацию, имеет необходимый арсенал методов, 
не совершает ошибок ни на одном из этапов и,~по сути, заранее знает, какой 
должна быть разрабатываемая \mbox{РАСИГИА}. В~реальности на каждом из 
этапов могут обнаруживаться ранее допущенные недочеты, требующие 
возврата к~соответствующему этапу, их исправления и~повторного выполнения 
проделанной работы с~новыми исходными данными. В~определенном смыс\-ле 
такой подход представляет собой метод проб и~ошибок, и~чем слож\-нее 
проблема, для которой проектируется \mbox{РАСИГИА}, с~точ\-ки зрения 
конкретного коллектива разработчиков, тем больше будет возвратов в~ходе 
проектирования системы~[6]. Рас\-смот\-рим по\-дроб\-нее каждую из фаз 
методологии.



\section{Проектная фаза}

  Проектная фаза включает в~себя стадии концептуального описания проб\-ле\-мы и~моделирования, выполняемые сис\-тем\-ны\-ми аналитиками из коллектива 
разработчиков. В~рамках первой стадии фазы на доформальном, 
содержательном уровне рас\-смат\-ри\-ва\-ет\-ся проб\-ле\-ма как отрицательное 
отношение субъекта к~реальности~[6] и~проблемная ситуация как объективное 
стечение обстоятельств, обуслов\-ли\-ва\-ющее проб\-ле\-му. Данная стадия со\-сто\-ит из 
сле\-ду\-ющих этапов:
  \begin{itemize}
\item формулирование проб\-ле\-мы, ее предварительное описание в~ходе 
интервьюирования лица, при\-ни\-ма\-юще\-го решение, его советников и~активных 
групп на естественном языке с~использованием привычных для них 
определений и~формулировок~[7];
  \item определение проб\-ле\-ма\-ти\-ки, т.\,е.\ комплекса проб\-лем, связанных 
с~рас\-смат\-ри\-ва\-емой~[4], чтобы учесть создаваемые ее решением последствия 
для каж\-дой из них. Необходимо охватить весь круг участников проб\-лем\-ной 
ситуации (стейкхолдеров, заинтересованных лиц): непосредственных 
участников ситуации, пред\-ста\-ви\-те\-лей проб\-ле\-мо\-раз\-ре\-ша\-ющих 
и~проб\-ле\-мо\-со\-дер\-жа\-щих сис\-тем, же\-ла\-емых помощников или союзников, 
субъектов, связанных с~ситуацией юридически, лиц с~возможным негативным 
отношением к~решению проб\-ле\-мы~[6]. Для по\-стро\-ения проб\-ле\-ма\-ти\-ки может 
быть использована, например, технология Дж.~Уор\-фил\-да, подходы 
с~использованием метафор организации, взгляда на проблему стейкхолдером 
с~раз\-ных точек зрения, рас\-смот\-ре\-ния проб\-ле\-мы в~рамках различных парадигм 
(функциональной, объяснительной, освободительной, пост\-мо\-дер\-нист\-ской)~[4, 6]. 
Формируется древовидная или сетевая структура в~виде диаграммы связей, 
концептуальной кар\-ты или аналогичных инструментов;
  \item определение целей проектирования \mbox{РАСИГИА}, 
пред\-по\-ла\-га\-ющее проведение собеседований с~каж\-дым стейк\-хол\-де\-ром, 
выяснение их целей и~пожеланий, формирование и~структурирование 
множества целей в~виде дерева или сетевидной структуры и~его 
визуализация~[4, 6]. Выделяются следующие уровни целей: ожи\-да\-емые 
в~плановом периоде результаты; задачи, которые не будут решены 
в~рас\-смат\-ри\-ва\-емом периоде, но будет достигнут существенный прогресс на 
пути к~ним; не\-до\-сти\-жи\-мые идеалы, к~которым следует стремиться~[8];
  \item выбор критериев, т.\,е.\ до\-ступ\-ных для наблюдения и~измерения 
характеристик, опи\-сы\-ва\-ющих важ\-ные особенности объектов или процессов 
и~поз\-во\-ля\-ющих сравнивать \mbox{пред\-ла\-га\-емые} альтернативы, контролировать 
процесс решения~[6]. Со\-во\-куп\-ность критериев долж\-на быть релевантной 
количественной моделью выделенных ранее качественных целей. Отдельно 
выделяются ограничения, фик\-си\-ру\-ющие условия, которые не могут нарушаться 
при до\-сти\-же\-нии цели;
  \item оценка концептуального описания проб\-ле\-мы в~ходе специально 
спланированного эксперимента. Если существует коллектив специалистов, 
решающий на практике по\-став\-ле\-нную или схожие проб\-ле\-мы, он выступает 
образцом, прототипом создаваемой сис\-те\-мы агентов. В~этом случае 
выполняется наблюдение за работой такого коллектива в~рамках решения 
реальных или тренировочных проб\-лем и~оценка релевантности 
зафиксированной информации сведениям, полученным в~ходе предыду\-щих 
этапов. Если выявлено существенное рас\-хож\-де\-ние, выполняется возврат 
к~этапу, в~рамках которого были получены некорректные сведения. Сведения 
о~составе участников коллектива, вы\-де\-ля\-емых ими под\-проб\-ле\-мах, методах их 
решения используются на по\-сле\-ду\-ющих этапах проектирования 
\mbox{РАСИГИА} при по\-стро\-ении со\-от\-вет\-ст\-ву\-ющих моделей проб\-ле\-мы 
и~сис\-те\-мы <<как есть сейчас>>. Данные о~качестве принятых решений 
и~дли\-тель\-ности их выработки используются в~дальнейшем как показатель 
эффекта от разработки и~внед\-ре\-ния \mbox{РАСИГИА}. Если подобных 
коллективов нет или не\-воз\-мож\-но реализовать со\-от\-вет\-ст\-ву\-ющий эксперимент, 
данный этап отсутствует.
  \end{itemize}
  
  Стадия моделирования предполагает разработку формализованного описания 
проб\-ле\-мы, коллектива специалистов, ре\-ша\-юще\-го проб\-ле\-му на момент 
разработки \mbox{РАСИГИА}, если он существует, и~самой 
\mbox{РАСИГИА}. Модели строятся с~использованием визуального 
метаязыка~[9], что позволяет наглядно их изобразить, а~так\-же поз\-во\-ля\-ет 
с~использованием заранее заданных соответствий однозначно отоб\-ра\-зить 
графическое пред\-став\-ле\-ние моделей в~формальное символьное пред\-став\-ле\-ние, 
пригодное для компьютерной интерпретации. Данная стадия со\-сто\-ит из 
сле\-ду\-ющих этапов:
  \begin{itemize}
  \item моделирование проб\-ле\-мы, которое обеспечивает ее формальное 
пред\-став\-ле\-ние на макро- и~мик\-ро\-уров\-не. Мак\-ро\-уров\-не\-вая модель описывает 
проб\-ле\-му как <<чер\-ный ящик>>, отражая ее место в~ме\-та\-проб\-ле\-ме (проб\-ле\-ме 
более высокого уровня), свойства как целого и~связи с~другими проб\-ле\-ма\-ми 
ме\-та\-проб\-ле\-мы. Атрибуты проблемы на макроуровне~--- цели, критерии 
(включая ограничения), исходные данные и~идентификатор. Мик\-ро\-уров\-не\-вая 
модель раскрывает со\-став и~структуру проб\-ле\-мы, описывает ее под\-проб\-ле\-мы 
и~связи между ними. Для каждой под\-проб\-ле\-мы специфицируются цели, 
критерии, исходные данные и~идентификатор, выполняется поиск релевантных 
методов решения. Если такие методы найдены, дальнейшая декомпозиция 
под\-проб\-ле\-мы не требуется, иначе выполняется по\-стро\-ение ее мик\-ро\-уров\-не\-вой 
модели, т.\,е.\ модели более глубокого уров\-ня иерархии. Таким образом, 
формируется многоуровневая иерархическая структура по\-став\-лен\-ной 
проб\-лемы;
  \item моделирование коллектива, которое отражает ситуацию решения 
проб\-ле\-мы <<как есть сейчас>> со всеми ее преимуществами и~недостатками. 
Модель коллектива~--- основа, образец для проектирования \mbox{РАСИГИА} и~оценки эф\-фек\-тив\-ности 
альтернативных конфигураций \mbox{РАСИГИА}. 
При моделировании коллектива специалистов фиксируется его со\-став в~виде 
множества ролей участников, час\-ти проб\-ле\-мы, ре\-ша\-емые каж\-дым из 
участников с~определенной ролью, знания и~методы, ис\-поль\-зу\-емые 
участниками для решения своей части проб\-ле\-мы, а~так\-же порядок и~нормы 
взаимодействия участников коллектива; 
  \item моделирование \mbox{РАСИГИА}, фор\-ми\-ру\-ющее идеализированное 
пред\-став\-ле\-ние <<как должно стать>> о~коллективе интеллектуальных агентов, 
ре\-ша\-ющих по\-став\-лен\-ную проб\-ле\-му. В~ходе моделирования \mbox{РАСИГИА} 
должны быть специфицированы со\-став и~иерархия ролей агентов, множество 
агентов, ис\-поль\-зу\-емые протоколы взаимодействия, под\-дер\-жи\-ва\-емые языки 
передачи сообщений, базовая онтология как осно\-ва для интерпретации 
семантики пе\-ре\-да\-ва\-емых сообщений, модель окру\-жа\-ющей среды, содержащая 
в~том чис\-ле пул, из которого агенты могут привлекаться сис\-те\-мой по мере 
не\-об\-хо\-ди\-мости и~в~который попадают ис\-клю\-ча\-емые из нее агенты, множество 
моделей архитектур \mbox{РАСИГИА}, множество необходимых моделей 
мак\-ро\-уров\-не\-вых эффектов. В~множестве агентов должны присутствовать 
агенты, пред\-став\-ля\-ющие стейк\-хол\-де\-ров с~их целями, критериями достижения 
цели и~ограничениями. Если на предыду\-щем этапе была по\-стро\-ена модель 
коллектива, то одна из архитектур \mbox{РАСИГИА} долж\-на соответствовать 
данной модели. 
  \end{itemize}
  
\section{Технологическая фаза}

  Технологическая фаза включает в~себя разработку эскизного проекта 
\mbox{РАСИГИА}, ее технического проекта и~программной реализации. 
Стадия разработки эскизного проекта \mbox{РАСИГИА} обеспечивает 
пред\-став\-ле\-ние создаваемой сис\-те\-мы и~ее внеш\-ней среды в~виде 
взаимосвязанных мо\-ду\-лей-бло\-ков в~соответствии с~моделью 
\mbox{РАСИГИА}, по\-стро\-ен\-ной на стадии проектирования. Данная стадия 
со\-сто\-ит из сле\-ду\-ющих этапов:
  \begin{itemize}
  \item разработка функциональной структуры, в~ходе которой строится 
множество взаимосвязанных схем-диа\-грамм, определяющих под\-сис\-те\-мы 
РАСИГИА, распределение агентов по ним, функционал агентов, до\-пус\-ти\-мые 
языки передачи сообщений и~протоколы взаимодействия для каж\-дой пары или 
группы ролей агентов, технологические элементы сис\-те\-мы, потоки 
информации и~управ\-ле\-ния, а~также отношения, воз\-ни\-ка\-ющие между агентами 
в~процессе решения проб\-лем. Для каждой роли указывается множество 
релевантных ей уже существующих (разработанных ранее для других сис\-тем) 
агентов, если таковые имеются. В~случае отсутствия релевантных агентов они 
должны быть разработаны на сле\-ду\-ющих этапах. Кроме того, 
специфицируются функциональные мо\-ду\-ли-бло\-ки, от\-ве\-ча\-ющие за организацию 
макроуровневых эффектов в~\mbox{РАСИГИА};
  \item разработка структуры внешней среды по аналогии с~разработкой 
функциональной структуры \mbox{РАСИГИА} предполагает построение схем-диа\-грамм, описывающих виртуальную внеш\-нюю среду, ее под\-сис\-те\-мы, роли 
агентов и~способы взаимодействия \mbox{РАСИГИА} с~ними, т.\,е.\ языки 
передачи сообщений и~протоколы взаимодействия, отношения, потоки 
информации и~управ\-ле\-ния. Для каж\-дой роли указываются су\-щест\-ву\-ющие 
релевантные ей агенты, если они имеются;
  \item разработка архитектур агентов выполняется для тех ролей 
в~функциональной структуре и~структуре внеш\-ней среды, для которых не 
найдено релевантных реализованных агентов. Архитектура агента~--- схема, 
описывающая со\-став, структуру и~взаимосвязь функ\-ций-бло\-ков, 
ре\-а\-ли\-зу\-емых агентом, обеспечивающая выполнение им своего предназначения. 
Для каждой функ\-ции-бло\-ка указывается метод или алгоритм, с~по\-мощью 
которого она реализуется, в~случае если таковые отсутствуют, они долж\-ны 
быть разработаны в~рамках сле\-ду\-ющей стадии.
  \end{itemize}
  
  Стадия разработки технического проекта \mbox{РАСИГИА} обеспечивает 
создание недостающих блоков для ее агентов или технологических элементов. 
При этом может по\-тре\-бо\-вать\-ся разработка методов решения под\-проб\-лем, 
алгоритмов на основе метода, баз данных, онтологий и~др. Порядок их 
разработки не регламентируется на\-сто\-ящей методологией в~связи 
с~существенным разнообразием и~не\-воз\-мож\-ностью совместного рас\-смот\-ре\-ния. 
На данной стадии должен быть сформирован технический проект, 
опи\-сы\-ва\-ющий для каждого блока со\-став, структуру и~форму пред\-став\-ле\-ния 
входных и~выходных данных, алгоритм его функционирования, спецификацию 
необходимых технических средств~[10].
  
  Стадия программной реализации и~отладки предполагает разработку 
программного кода \mbox{РАСИГИА} и~его тестирование на предмет 
корректной работы с~\mbox{целью} формирования полноценного программного 
продукта, а~так\-же разработку программной документации. Данная стадия 
со\-сто\-ит из сле\-ду\-ющих этапов:
  \begin{itemize}
  \item программная реализация и~разработка документации выполняется 
с~использованием платформы JaCaMo~[11], объединяющей технологию Jason 
для программирования автономных агентов, Cartago для программирования 
элементов внеш\-ней среды и~Moise для программирования многоагентных 
организаций. Кроме того, применяется язык Java для программирования 
отдельных элементов сис\-те\-мы и~тонкой настройки механизмов 
платформы~[12];
  \item тестирование и~отладка обеспечивают выявление и~устранение 
основных дефектов в~сис\-те\-ме. Ввиду того что полное тестирование  
сколь\-ко-ни\-будь слож\-ной программы не\-воз\-мож\-но~[13], выполняется 
выборочное тестирование в~сле\-ду\-ющем порядке: отдельные функ\-ции и~блоки 
из состава аген\-тов и~технологических элементов, межмодульные связи, агенты 
и~технологические элементы в~целом, протоколы взаимодействия агентов, 
\mbox{РАСИГИА} в~целом. В~тес\-ти\-ро\-ва\-нии принимают участие 
представители всех ролей команды разработчиков, так как каж\-дый из них 
выполняет поиск ошибок разного рода~[14]. При этом выделяется отдельная 
роль тестировщика, опре\-де\-ля\-юще\-го стратегию тес\-ти\-ро\-ва\-ния,  
тест-тре\-бо\-ва\-ния и~тест-пла\-ны для каждой из фаз проекта; он выполняет 
тестирование сис\-те\-мы, собирает и~анализирует отчеты о~про\-хож\-де\-нии 
тестирования. 
\end{itemize}

\section{Рефлексивная фаза}

  Рефлексивная фаза предназначена для оценки показателей реализованной 
\mbox{РАСИГИА} и~процесса ее разработки, выявления ее недостатков и~при 
не\-об\-хо\-ди\-мости до\-ра\-бот\-ки как сис\-те\-мы, так и~методологии ее построения. 
Стадия оценки эф\-фек\-тив\-ности \mbox{РАСИГИА} предполагает сбор 
показателей работы сис\-те\-мы и~их сравнение с~целевыми значениями. Если 
выявляется их несоответствие, выполняется анализ причин отклонений, 
переход к~этапу методологии, вызвавшему их, и~повторное выполнение данного и~по\-сле\-ду\-ющих этапов с~учетом тре\-бу\-емых корректировок. Кроме того, на этой 
стадии продолжается отладка сис\-те\-мы. Данная стадия выполняется в~три этапа:
  \begin{enumerate}[(1)]
  \item оценка в~лабораторных условиях командой разработчиков, когда 
система работает в~виртуальной внешней среде, решая тестовые проб\-ле\-мы. На 
данной стадии оценка сис\-те\-мы выполняется вычислительными моделями 
стейкхолдеров, реализованными со\-от\-вет\-ст\-ву\-ющи\-ми агентами виртуальной 
внеш\-ней среды; 
  \item оценка по результатам тестовой эксплуатации, когда \mbox{РАСИГИА} 
функционирует в~реальной внеш\-ней среде параллельно с~традиционным 
методом решения проб\-ле\-мы и~выполняется сравнение их эф\-фек\-тив\-ности 
пользователями и~реальными стейк\-хол\-де\-ра\-ми. Первоначально 
у~\mbox{РАСИГИА} должна быть отключена воз\-мож\-ность оказывать ка\-кое-ли\-бо воздействие на реальную внеш\-нюю среду, а~результатом ее\linebreak
 работы 
долж\-ны стать рекомендации по оказанию таких воздействий. После 
удовлетворительной оценки пользователей и~стейк\-хол\-де\-ров \mbox{РАСИГИА} 
может быть переведена в~\mbox{автоматический} режим взаимодействия со средой, 
а~традиционный метод решения проб\-ле\-мы используется в~качестве резервного 
для проверки ее работы еще в~течение некоторого времени. Длительности 
каждого из этих периодов долж\-ны определяться заказчиками сис\-те\-мы для 
решения конкретной проб\-ле\-мы совместно с~коллективом разработчиков; 
  \item сопровождение после внед\-ре\-ния поз\-во\-ля\-ет собирать жалобы, замечания и~предложения в~процессе эксплуатации \mbox{РАСИГИА}, в~том чис\-ле от 
людей, которые ошибочно не были включены в~со\-став стейкхолдеров.\\[-13pt] 
  \end{enumerate}
  
  Стадия оценки и~корректировки методологии в~определенном смысле длится 
на протяжении всего проекта, так как для ее реализации долж\-ны вес\-тись 
протоколы де\-я\-тель\-ности разработчиков,\linebreak в~которых отмечается дли\-тель\-ность 
реализации каж\-до\-го этапа, выполненные возвраты и~их причины. Однако 
именно по завершении проекта выполняется рефлексия проделанной работы, 
когда разработчики долж\-ны проанализировать удачные и~провальные решения, 
причины рас\-хож\-де\-ния результатов с~планами, возвратов к~предыду\-щим этапам 
и~фазам разработки \mbox{РАСИГИА}, затягивания отдельных этапов 
разработки, из\-бы\-точ\-ность или, наоборот, не\-ин\-фор\-ма\-тив\-ность 
по\-стро\-ений~\cite{4-lis}. По результатам анализа в~методологию вносятся 
изменения в~статусе <<предложение>>, которые после под\-тверж\-де\-ния 
эф\-фек\-тив\-ности в~новых проектах закрепляются в~новой версии методологии.

\vspace*{-9pt}

\section{Заключение}

\vspace*{-3pt}

  В работе представлена темпоральная структура (жизненный цикл) 
разработки \mbox{РАСИГИА}, опи\-сы\-ва\-ющая процессы сис\-тем\-но\-го анализа 
проб\-ле-\linebreak мы, моделирования, эскизного и~технического \mbox{проектирования} сис\-те\-мы, 
ее программной реализации, отладки и~тестирования. 
Основной результат 
организации работ в~соответствии с~предложенной методологией~--- 
программная реализация \mbox{РАСИГИА}, релевантно моделирующая 
коллектив специалистов, со\-вмест\-но ре\-ша\-ющих по\-став\-лен\-ную проб\-ле\-му 
с~учетом ее слабой формализации, не\-од\-но\-род\-ности, сетевого характера условий 
и~целей, не\-опре\-де\-лен\-ности и~ди\-на\-мич\-ности~\cite{5-lis}. Кроме того, в~результате 
рефлексивной стадии методологии формируется ее новая версия или 
подтверждается эф\-фек\-тив\-ность су\-щест\-ву\-ющей, что представляется\linebreak 
дополнительным результатом работ. Таким образом, методология предполагает 
свое развитие, потенциально обеспечивающее ее ре\-ле\-вант\-ность \mbox{актуальным}
подходам к~проектированию и~реализации интеллектуальных информационных 
сис\-тем.

\vspace*{-9pt}
  
{\small\frenchspacing
 { %\baselineskip=10.6pt
 %\addcontentsline{toc}{section}{References}
 \begin{thebibliography}{99}
 
 \vspace*{-3pt}
 
  \bibitem{1-lis}
   \Au{Поспелов Д.\,А.} Десять <<горячих точек>> в~исследованиях по искусственному 
интеллекту~// Искусственный\linebreak\vspace*{-12pt}

\columnbreak

\noindent
 интеллект и~принятие решений, 2019. №\,4. С.~3--9. doi: 
10.14357/20718594190401. EDN: BAUHFV.
  
  \bibitem{2-lis}
\Au{Тарасов В.\,Б.} От многоагентных сис\-тем к~интеллектуальным организациям: 
философия, психология, информатика.~--- М.: Эдиториал УРСС, 2002. 348~с.
  \bibitem{3-lis}
  \Au{Колесников А.\,В., Кириков~И.\,А., Листопад~С.\,В.} Ги\-брид\-ные интеллектуальные 
сис\-те\-мы с~самоорганизацией: координация, со\-гла\-со\-ван\-ность, спор.~--- М.: ИПИ РАН, 2014. 
189~с.
  \bibitem{4-lis}
  \Au{Новиков А.\,М., Новиков~Д.\,А.} Методология.~--- М.: Синтег, 2007. 668~с.
  \bibitem{5-lis}
  \Au{Листопад С.\,В.} Характеристики и~логическая структура методологии по\-стро\-ения  
реф\-лек\-сив\-но-ак\-тив\-ных сис\-тем искусственных гетерогенных интеллектуальных 
агентов~// Сис\-те\-мы и~средства \mbox{информатики}, 2023. Т.~33. №\,4. С.~16--27. doi: 
10.14357/ 08696527230402. EDN: TRTHEI.
  \bibitem{6-lis}
  \Au{Тарасенко Ф.\,П.} Прикладной сис\-те\-мный анализ.~--- М.: 
КНОРУС, 2010. 224~с.
  \bibitem{7-lis}
  \Au{Ларичев О.\,И.} Вербальный анализ решений.~--- М.: Наука, 2006. 181~с.
  \bibitem{8-lis}
  \Au{Акофф Р.} Акофф о менеджменте~/ Пер.\ с~англ.~--- СПб.: Питер, 2002. 448~с.
  (\Au{Akoff~R.\,L.} Ackoff's best: His classic writings on management.~--- New 
York, NY, USA: Wiley, 1999. 368~p.)
  \bibitem{9-lis}
  \Au{Колесников А.\,В., Листопад~С.\,В., Румовская~С.\,Б., Данишевский~В.\,И.} 
Неформальная аксиоматическая тео\-рия ролевых визуальных моделей~// Информатика и~её 
применения, 2016. Т.~10. Вып.~4. С.~114--120.  doi: 10.14357/19922264160412. EDN: XGSIVN.
  \bibitem{10-lis}
  \Au{Черушева Т.\,В.} Проектирование программного обеспечения.~--- Пенза: ПГУ, 2014. 
172~с.
  \bibitem{11-lis}
  \Au{Boissier O., Bordini~R.\,H., Hubnerand~J., Ricci~A.} Multi-agent oriented programming: 
Programming multi-agent systems using JaCaMo.~--- Intelligent robotics and autonomous agents 
series.~--- Cambridge: The MIT Press, 2020. 264~p.
  \bibitem{12-lis}
  \Au{Смирнов С.\,С., Смольянинова~В.\,А.} Введение в~разработку многоагентных сис\-тем 
в~среде Jason. Основы программирования на языке AgentSpeak.~--- М.: \mbox{МИРЭА}, 2009. 136~с.
  \bibitem{13-lis}
  \Au{Канер~С., Фолк~Д., Нгуен~Е.\,К.} Тестирование про\-грам\-мно\-го обеспечения. 
Фундаментальные концепции менеджмента биз\-нес-при\-ло\-же\-ний~/
Пер. с~англ.~--- Киев: ДиаСофт, 
2001. 544~с. (\Au{Kaner~С., Falk~J., Nguyen~H.\,Q.} {Testing computer software}.~--- 
International Thomson Computer Press,  1999. 496~p.)
  \bibitem{14-lis}
  \Au{Романькова Т.\,Л.} Тестирование программного обеспечения. {\sf 
https://elib.gstu.by/bitstream/handle/220612/ 9860/416.pdf}.

\end{thebibliography}

 }
 }

\end{multicols}

\vspace*{-6pt}

\hfill{\small\textit{Поступила в~редакцию 25.11.23}}

%\vspace*{8pt}

%\pagebreak

\newpage

\vspace*{-28pt}

%\hrule

%\vspace*{2pt}

%\hrule



\def\tit{LIFE CYCLE OF METHODOLOGY FOR~CONSTRUCTING REFLEXIVE-ACTIVE SYSTEMS OF~ARTIFICIAL HETEROGENEOUS INTELLIGENT AGENTS}


\def\titkol{Life cycle of methodology for~constructing reflexive-active systems of~artificial heterogeneous intelligent agents}


\def\aut{S.\,V.~Listopad}

\def\autkol{S.\,V.~Listopad}

\titel{\tit}{\aut}{\autkol}{\titkol}

\vspace*{-8pt}


\noindent
Federal Research Center ``Computer Science and Control'' of the Russian Academy of 
Sciences, 44-2~Vavilov Str., Moscow 119333, Russian Federation

\def\leftfootline{\small{\textbf{\thepage}
\hfill INFORMATIKA I EE PRIMENENIYA~--- INFORMATICS AND
APPLICATIONS\ \ \ 2024\ \ \ volume~18\ \ \ issue\ 1}
}%
 \def\rightfootline{\small{INFORMATIKA I EE PRIMENENIYA~---
INFORMATICS AND APPLICATIONS\ \ \ 2024\ \ \ volume~18\ \ \ issue\ 1
\hfill \textbf{\thepage}}}

\vspace*{4pt}
  
  
   
   \Abste{The paper presents the temporal structure (life cycle) of the methodology for 
constructing reflexive-active systems of artificial heterogeneous intelligent agents. These systems 
are designed for computer modeling of processes and effects that arise when solving practical 
problems by teams of specialists under the guidance of a~decision maker. Artificial heterogeneous 
intelligent agents of reflexive-active systems are active subjects capable of reasoning, 
communication, and reflection as the ability to model the reasoning of other agents of the system 
and themselves. Modeling of reflexive processes ensures the development by agents of a~consistent 
understanding of the control object, the purpose of collective work, and the norms of interaction 
allowing the system to self-organize and re-develop a relevant hybrid intelligent method for solving 
the next problem.}
   
   \KWE{reflection; methodology; reflexive-active system of artificial heterogeneous intelligent 
agents; hybrid intelligent multiagent system; team of specialists}
   
 
   
\DOI{10.14357/19922264240112}{GUAMVE}

\vspace*{-8pt}

\Ack

\vspace*{-1pt}


     \noindent
     This work was supported by the Russian Science Foundation, project No.\,23-21-00218.


\vspace*{6pt}

  \begin{multicols}{2}

\renewcommand{\bibname}{\protect\rmfamily References}
%\renewcommand{\bibname}{\large\protect\rm References}

{\small\frenchspacing
 {\baselineskip=11.5pt
 \addcontentsline{toc}{section}{References}
 \begin{thebibliography}{99} 
  \bibitem{1-lis-1}
   \Aue{Pospelov, D.\,A.} 2019. Desyat' ``goryachikh tochek'' v~issledovaniyakh po 
iskusstvennomu intellektu [Ten hot topics in AI research]. \textit{Is\-kus\-stven\-nyy in\-tel\-lekt 
i~pri\-nya\-tie re\-she\-niy} [Artificial Intelligence and Decision Making] 4:3--9. doi: 
10.14357/20718594190401. EDN: BAUHFV.
  \bibitem{2-lis-1}
   \Aue{Tarasov, V.\,B.} 2002. \textit{Ot mnogoagentnykh sis\-tem k~in\-tel\-lek\-tu\-al'\-nym 
or\-ga\-ni\-za\-tsi\-yam: fi\-lo\-so\-fiya, psi\-kho\-lo\-giya, in\-for\-ma\-ti\-ka} [From multiagent systems to intelligent 
organizations: Philosophy, psychology, and computer science]. Moscow: Editorial URSS. 348~p.
  \bibitem{3-lis-1}
   \Aue{Kolesnikov, A.\,V., I.\,A.~Kirikov, and S.\,V.~Listopad.} 2014. \textit{Gib\-rid\-nye 
in\-tel\-lek\-tu\-al'\-nye sis\-te\-my s~sa\-mo\-or\-ga\-ni\-za\-tsiey: ko\-or\-di\-na\-tsiya, so\-gla\-so\-van\-nost', spor} [Hybrid 
intelligent systems with self-organization: Coordination, consistency, and dispute]. Moscow: IPI 
RAN. 189~p.
  \bibitem{4-lis-1}
   \Aue{Novikov, A.\,M., and D.\,A.~Novikov.} 2007. \textit{Me\-to\-do\-lo\-giya} [Methodology]. 
Moscow: SINTEG. 668~p.
  \bibitem{5-lis-1}
   \Aue{Listopad, S.\,V.} 2023. Kharakteristiki i~logicheskaya struk\-tu\-ra me\-to\-do\-lo\-gii po\-stro\-eniya 
refleksivno-aktivnykh sis\-tem is\-kus\-stven\-nykh ge\-te\-ro\-gen\-nykh in\-tel\-lek\-tu\-al'\-nykh agen\-tov 
[Characteristics and logical structure of the methodology for constructing reflexive-active systems 
of artificial heterogeneous intelligent agents]. \textit{Sistemy i~Sredstva Informatiki~--- Systems 
and Means of Informatics} 33(4):16--27. doi: 10.14357/08696527230402. EDN: TRTHEI.
  \bibitem{6-lis-1}
   \Aue{Tarasenko, F.\,P.} 2010. \textit{Pri\-klad\-noy sis\-tem\-nyy ana\-liz} 
[Applied systems analysis]. Moscow: KNORUS. 224~p.
  \bibitem{7-lis-1}
   \Aue{Larichev, O.\,I.} 2006. \textit{Ver\-bal'\-nyy ana\-liz re\-she\-niy} [Verbal analysis of decisions]. 
Moscow: Nauka. 181~p.
  \bibitem{8-lis-1}
   \Aue{Akoff, R.\,L.} 1999. \textit{Ackoff's best: His classic writings on management}. New 
York, NY: Wiley. 368~p.
  \bibitem{9-lis-1}
   \Aue{Kolesnikov, A.\,V., S.\,V.~Listopad, S.\,B.~Rumovskaya, and V.\,I.~Danishevskiy.} 
2016. Ne\-for\-mal'\-naya ak\-sio\-ma\-ti\-che\-skaya teo\-riya ro\-le\-vykh vi\-zu\-al'\-nykh mo\-de\-ley [Informal axiomatic 
theory of the role visual models]. \textit{Informatika i~ee Primeneniya~--- Inform. Appl.} 
10(4):114--120. doi: 10.14357/19922264160412. EDN: XGSIVN.
  \bibitem{10-lis-1}
   \Aue{Cherusheva, T.\,V.} 2014. \textit{Pro\-ek\-ti\-ro\-va\-nie pro\-gram\-mno\-go obes\-pe\-che\-niya} 
[Software design]. Penza: PGU. 172~p.
  \bibitem{11-lis-1}
   \Aue{Boissier, O., R.\,H.~Bordini, J.~Hubnerand, and A.~Ricci}. 2020. \textit{Multi-agent 
oriented programming: Programming multi-agent systems using JaCaMo}. Intelligent robotics and 
autonomous agents ser. Cambridge: The MIT Press. 264~p.
  \bibitem{12-lis-1}
   \Aue{Smirnov, S.\,S., and V.\,A.~Smol'yaninova}. 2009. \textit{Vve\-de\-nie v~raz\-ra\-bot\-ku 
mno\-go\-agent\-nykh sis\-tem v~sre\-de Jason. Osno\-vy pro\-gram\-mi\-ro\-va\-niya na yazy\-ke AgentSpeak} 
[Introduction to the development of multiagent systems in the Jason environment. Fundamentals of 
programming in the AgentSpeak language]. Moscow: MIREA. 136~p.
  \bibitem{13-lis-1}
   \Aue{Kaner, С., J.~Falk, and H.\,Q.~Nguyen}. 1999. \textit{Testing computer software}. 
International Thomson Computer Press. 496~p.
  \bibitem{14-lis-1}
   \Aue{Romankova, T.\,L.} 2014. Tes\-ti\-ro\-va\-nie pro\-gram\-mno\-go obes\-pe\-che\-niya [Software testing]. 
Available at: {\sf https://}\linebreak\vspace*{-12pt}

\columnbreak

\noindent
 {\sf elib.gstu.by/bitstream/handle/220612/9860/416.pdf} (accessed January~16, 
2024).
   
  \end{thebibliography}

 }
 }

\end{multicols}

\vspace*{-6pt}

\hfill{\small\textit{Received November 25, 2023}} 

%\vspace*{-18pt}
     
     \Contrl
     
 %    \vspace*{-3pt}
   
   \noindent
   \textbf{Listopad Sergey V.} (b.\ 1984)~--- Candidate of Science (PhD) in technology, senior 
scientist, Federal Research Center ``Computer Science and Control'' of the Russian Academy of 
Sciences, 44-2~Vavilov Str., Moscow 119133, Russian Federation;  
\mbox{ser-list-post@yandex.ru}
   
    
\label{end\stat}

\renewcommand{\bibname}{\protect\rm Литература}  %7
\def\stat{kovalev}

\def\tit{МЕТОДЫ ТЕОРИИ КАТЕГОРИЙ В~МОДЕЛЬНО-ОРИЕНТИРОВАННОЙ СИСТЕМНОЙ 
ИНЖЕНЕРИИ}

\def\titkol{Методы теории категорий в~модельно-ориентированной системной 
инженерии}

\def\aut{С.\,П.~Ковалёв$^1$}

\def\autkol{С.\,П.~Ковалёв}

\titel{\tit}{\aut}{\autkol}{\titkol}

\index{Ковалёв С.\,П.}
\index{Kovalyov S.\,P.}


%{\renewcommand{\thefootnote}{\fnsymbol{footnote}} \footnotetext[1]
%{Исследование выполнено при финансовой поддержке Российского научного фонда (проект 16-11-10227).}}


\renewcommand{\thefootnote}{\arabic{footnote}}
\footnotetext[1]{Институт проблем управления им.\ В.\,А.~Трапезникова 
Российской академии наук,  \mbox{kovalyov@nm.ru}}

%\vspace*{-18pt}

\Abst{Предложен математический аппарат на базе теории категорий, который позволяет 
формально описывать и~строго исследовать процедуры применения моделей в~инженерной 
деятельности, составляющие сущность мо\-дель\-но-ори\-ен\-ти\-ро\-ван\-ной системной 
инженерии (Model-Based Systems Engineering, MBSE). В~основе аппарата лежит 
математическое представление сборочных чертежей (мегамоделей сис\-тем) диаграммами 
в~категориях, объектами которых служат модели, а~морфизмы представляют действия по 
сборке моделей сис\-тем из моделей компонентов. Адекватность аппарата обоснована исходя 
из требований стандартов, регламентирующих описание структуры систем, в~том числе 
IEC~81346. Предложены и~исследованы тео\-ре\-ти\-ко-ка\-те\-гор\-ные методы решения ряда 
практических задач сборки систем. Приведены примеры решения таких задач в~категориях, 
представляющих две ключевые области применения MBSE: гео\-мет\-ри\-че\-ское моделирование 
изделий сложной формы и~дис\-крет\-но-со\-бы\-тий\-ное имитационное моделирование 
поведения технических систем.}

\KW{модельно-ориентированная системная инженерия; мегамодель; теория категорий; 
копредел}



\DOI{10.14357/19922264170305} 


\vspace*{6pt}

\vskip 10pt plus 9pt minus 6pt

\thispagestyle{headings}

\begin{multicols}{2}

\label{st\stat}

\section{Введение}

   Модельно-ориентированная системная инженерия состоит в~формализованном применении моделирования в~
поддержке жизненного цикла сис\-тем, включая сбор требований, 
проектирование, проверку и~приемку, другие стадии~[1]. Модели, 
разрабатываемые в~ходе процедур MBSE, пригодны к~автоматической 
обработке на компьютерах. Это позволяет сначала задавать, верифицировать 
и~оптимизировать проектные решения на моделях <<в циф\-ре~и только потом 
воплощать <<в железе>>, снижая затраты на организацию жизненного цикла 
изделий и~сокращая сроки выполнения работ~[2].
   
   И все же внедрение технологий MBSE в~инженерную деятельность 
происходит медленно. Это связано во многом с~нехваткой единой 
концептуальной базы инженерного моделирования: предлагается много 
частных языков и~технологий, слабо совместимых друг с~другом и~плохо 
приспособленных для совместной разработки моделей большими 
мультидисциплинарными коллективами~[3]. Тем самым затрудняется переход 
от набора электронных чертежей к~полноценному электронно-цифровому 
макету (digital mock-up) промышленного изделия.
   
   Естественный, хотя и~<<трудный>>, подход к~получению результатов 
общего характера, унифи\-ци\-ру\-ющих разнородные технологии, состоит в~том, 
чтобы как можно более строго формализовать процедуры моделирования. 
Формализация позволит совершенствовать процедуры MBSE и~передавать их 
на исполнение компьютеру без пробелов и~искажений. Самый высокий уровень 
строгости достигается при привлечении математического аппарата, поскольку 
математика позволяет надежно доказывать или опровергать утверждения, 
ха\-рак\-те\-ри\-зу\-ющие корректность и~эффективность процедур.
   
   В настоящей работе предложен аппарат, основанный на математическом 
представлении сборочных чертежей (<<мегамоделей>> систем) 
ориенти-\linebreak рованными графами (диаграммами). Узлы такого\linebreak графа помечаются 
обозначениями моделей час\-тей, а~реб\-ра помечаются обозначениями действий\linebreak 
(activities), посредством которых части собираются в~систему. Представление 
структуры систем графами регламентируется, в~частности, стандартом 
IEC~81346~[4]. Естественным источником математических методов 
конструирования и~анализа мегамоделей служит теория категорий (см., 
например,~[5, 6]). Модели рассматриваются как объекты подходящих 
категорий, а~действия формально описываются морфизмами. Строятся 
и~исследу-\linebreak ются тео\-ре\-ти\-ко-ка\-те\-гор\-ные конструкции, опи\-сы\-ва\-ющие процедуры 
MBSE на абстрактном кон-\linebreak цептуальном уровне. Определенный опыт такого\linebreak 
исследования был накоплен в~инженерии программного обеспечения~[7] 
и~теперь может быть обобщен для системной инженерии в~целом. Например, 
сборке системы согласно некоторой мегамодели отвечает построение 
копредела диаграммы~--- универсальной конструкции~\cite{5-kov}.
   
   Статья построена следующим образом. В~разд.~2 приведен обзор 
принципов описания структуры сис\-тем согласно стандарту IEC~81346. 
Раздел~3 посвящен практическим проб\-ле\-мам мегамоделирования и~сборке 
сис\-тем. В~разд.~4 вводятся конструкции тео\-рии категорий, позволяющие 
формально решать задачи мегамоделирования. В~заключении приводятся 
выводы и~намечаются направления дальнейших исследований.

\section{Структура систем и~стандарт~IEC~81346}

   Важной проблемой MBSE, отмеченной во введении, является слабая 
совместимость языков и~инструмен\-тов моделирования от разных поставщиков. 
Основным подходом к~достижению совместимости является стандартизация~--- 
принятие обязывающих документов, устанавливающих требования и~принципы 
взаимозаменяемости инструментов. Многие стандарты определяют конкретные 
форматы машиночитаемой записи моделей, нейтральные относительно 
разработчиков инструментов MBSE. Примером служит формат описания 
твердотельных геометрических моделей STEP, стандартизованный семейством 
ISO~10303. Однако для формализации MBSE в~целом интерес представляют 
в~первую очередь стандарты более общего плана, унифицирующие принципы 
и~методы применения моделей в~жизненном цикле систем независимо от 
способа записи моделей. С~этой точки зрения внимания заслуживает 
международный стандарт IEC 81346-1:2009 <<Промышленные системы, 
установки и~обору\-до\-ва\-ние~--- принципы структурирования и~ссылочные 
обозначения~--- часть~1: основные правила>> (<<Industrial Systems, 
Installations and Equipment and Industrial Products~--- Structuring Principles and 
Reference Designations~--- Part~1: Basic Rules>>)~\cite{4-kov}. Стандарт не 
принят в~России, однако ряду его положений в~области структуры систем 
соответствует российский ГОСТ~2.053-2013 <<ЕСКД. Электронная структура 
изделия. Общие положения>>.
   
   В стандарте IEC~81346 рассматривается ряд вопросов моделирования 
структуры систем и~идентификации отдельных единиц в~составе систем. 
Системная единица названа в~стандарте объектом, причем принципиально не 
проводится различие между объектами реального мира, составляющими 
реально существующие системы, и~объектами мыслительной деятельности~--- 
моделями единиц, составляющими модели систем. Таким образом, стандарт 
выходит за рамки MBSE и~рассматривает ряд вопросов системной инженерии 
вообще. Иерар\-хи\-че\-ская структура системы (холархия~\cite{3-kov}) 
изображается деревом, узлы которого помечены обозначениями объектов. 
Важным достижением стандарта является выявление того факта, что одна и~та 
же система задается не одной, а несколькими в~общем случае различными 
иерархическими структурами, возникающими в~результате декомпозиции 
согласно различным принципам (аспектам). В~их числе:
   \begin{itemize}
\item функциональная (function-oriented) структура, отвечающая разделению 
системных единиц по выполняемым ими функциям в~составе сис\-темы;
\item продуктовая (product-oriented), или модульная, структура, отражающая 
сборочную (технологическую) конфигурацию сис\-темы;
\item структура размещения (location-oriented), в~соответствии с~которой 
единицы располагаются в~физическом пространстве.
\end{itemize}

   Ясно, что один и~тот же объект может входить в~несколько структур и~при 
этом находиться на различных уровнях. В~то же время в~некоторых аспектах 
объект может никак не проявлять себя и~вследствие этого отсутствовать 
в~соответствующих структурах. Полное идентифицирующее ссылочное 
обозначение объекта (reference designation) конструируется путем 
последовательного перечисления всех объектов, находящихся на пути от корня 
дерева рассматриваемой структуры до дан\-ного объекта включительно. 
Наименование каж\-до\-го объекта в~этом перечислении составляется из 
символьного обозначения аспекта, буквенного обозначения класса (типа), 
к~которому относится  объект, и~порядкового номера объекта среди 
экземпляров своего класса. Таким путем обеспечивается\linebreak  уникальность 
наименования любой единицы\linebreak
 в~пределах системы. Например, функциональная 
структура обозначается символом <<=>>, а~функциональный класс 
переключателей потоков ресурсов обозначается буквами QA, так что первая по 
порядку единица, выполняющая функцию переключения, называется =QA1, 
а~ее полное ссылочное обозначение может выглядеть как =WP1=WC1=QA1. 
Если объект присутствует в~нескольких структурах, то он может иметь 
несколько ссылочных обозначений, как показано на рис.~1~\cite{4-kov}.

\begin{figure*} %fig1
    \vspace*{1pt}
\begin{center}
\mbox{%
\epsfxsize=165mm
\epsfbox{kov-1.eps}
}
\end{center}
\vspace*{-9pt}
\Caption{Пример ссылочных обозначений структурных единиц системы}
\vspace*{9pt}
\end{figure*}

   С~точки зрения практики системной инженерии большой интерес 
представляет описание эволюции структурного представления системы по ходу 
жизненного цикла, приведенное в~приложении~B к~стандарту IEC~81346. 
<<Строительный материал>> для структур имеет вид (виртуального) 
справочника или каталога объектов, из которого выбираются объекты для 
включения в~структуру. 

В~начале жизненного цикла системы на основе 
исходных требований к~ней конструктор строит ее функциональную структуру. 
Затем определяется пространственное положение функциональных объектов, 
в~результате чего создается структура размещения. На следующей стадии 
формируются закупочные спецификации, образующие продуктовую структуру. 
В~ходе последующих стадий жизненного цикла эти структуры могут 
трансформироваться. На каждой стадии могут происходить замена, слияние 
и~расщепление объектов. Таким образом, объекты разных структур системы 
связаны отношением вида <<многие ко многим>>, вдоль которого 
прослеживаются (трассируются) исходные требования.
   
   В то же время стандарт не предусматривает указа\-ние способов, какими 
объекты собраны в~сис\-те\-мы. Поэтому структуру сис\-те\-мы можно рас\-смат\-ри\-вать 
как эскизный проект, в~котором отражены лишь факты вхождения системных 
единиц более низкого уровня иерархии в~единицы более высокого уровня. 


Проект такого рода поступает на вход технологу, который определяет 
конкретные операции сборки каждой единицы каждого уровня иерархии. При 
необходимости технолог вносит изменения в~конструкцию объектов (такие как 
нарезка резьбы) и~добавляет связующие интерфейсные объекты (такие как 
клей, трансформатор и~др.). В~результате для каждого составного объекта 
формируется сборочный чертеж, на котором указаны все со\-став\-ля\-ющие 
объекты и~действия по их соединению в~целях получения сис\-те\-мы. 
Технологическая проработка требуется на всех стадиях жизненного цикла, на 
которых формируется либо изменяется ка\-кая-ли\-бо из структур системы.

%\vspace*{-6pt}

\section{Мегамоделирование и~сборка~систем}

   В MBSE объекты, образующие 
структуры\linebreak
 сис\-тем, описываются формализованными ком\-пьютерными моделями 
различных видов: геометрическими фигурами и~телами, численными 
аппроксимациями дифференциальных уравнений, оснащенными графами и~
т.\,д. При этом, как свидетель\-ст\-ву\-ют стандарты типа IEC~81346, для анализа 
структуры систем и~организации сборки необходимо знать не столько 
внутреннюю структуру моделей, сколько ассортимент их возможностей 
соединяться с~другими моделями в~целях формирования моделей составных 
объектов. Иными словами, модели рассматриваются как <<черные ящики>> 
с~известным поведением по отношению к~другим моделям. Каталог объектов, 
упоминавшийся в~предыду\-щем разделе, в~условиях применения \mbox{MBSE} 
составляется из моделей и~описаний действий по их соединению.
   
   Структуры систем и~сборочные чертежи представляют собой частные 
случаи мегамоделей (mega\-mod\-el)~--- моделей, состоящих из моделей и~связей 
между ними~\cite{8-kov}. Мегамодель, в~которой связи описывают соединение 
моделей, образующих некоторую сис\-те\-му, называется конфигурацией этой 
сис\-те\-мы~\cite{5-kov}. Существуют и~другие виды мегамоделей, 
предназначенные для описания других процедур \mbox{MBSE}, таких как 
формирование модели согласно заданной метамодели  
(instantiating)~\cite{9-kov}. Но в~настоящей работе сосредоточимся на 
конфигурациях и~сборке систем.
   
   Например, в~моделировании механических сис\-тем, состоящих из твердых 
тел, моделями деталей и~сборочных единиц служат геометрические тела, 
которые могут быть представлены для компьютерной обработки различными 
способами: конструктивным, воксельным, граничным~\cite{10-kov}. Объекты, 
составляющие механические системы, т.\,е.\ представления экземпляров тел, 
получаются из моделей путем аффинных изометрий и~растяжений. Так, из 
набора цилиндров разных размеров составляется модель штанги (спортивного 
снаряда). В~функциональной структуре штанги по IEC~81346 цилиндры 
представлены разными объектами, поскольку они выполняют разные функции, 
хотя порождаются одной и~той же геометрической моделью. Соответственно, 
в~каталоге моделей содержится тело в~форме цилиндра, допускающее 
несколько разных действий по включению в~состав штанги.
   
   В качестве еще одного примера рассмотрим дис\-крет\-но-со\-бы\-тий\-ное 
имитационное моделирование, поддержка которого относится к~числу 
важнейших достижений MBSE~\cite{1-kov}. Здесь модель имеет вид 
сценария~--- фрагмента предполагаемой истории поведения моделируемой 
системы, пред\-став\-лен\-но\-го потоком дискретных событий различных видов. 
Некоторые события могут вызывать либо запрещать возникновение других 
событий. Описания действий по сборке сценариев поведения систем отражают 
вклад сценариев поведения составляющих. Так, сценарий работы цеха 
составляется из сценариев работы станков, связанных друг с~другом согласно 
маршрутным картам~\cite{11-kov}.
   
   Сформулируем задачу мегамоделирования сборки систем в~общем виде 
следующим образом. По мегамодели, представляющей конфигура\-цию 
некоторой системы, требуется сконструировать модель системы как целого 
и~рассчитать для нее моделируемые параметры, в~том числе эмерджентные~--- 
не присущие никакой из со\-став\-ля\-ющих единиц в~отдельности. Принцип 
конструирования модели системы легко усмотреть из организации 
структур-\linebreak\vspace*{-12pt}

\columnbreak

 { \begin{center}  %fig1
 \vspace*{1pt}
\mbox{%
\epsfxsize=57.246mm
\epsfbox{kov-2.eps}
}


\vspace*{12pt}


\noindent
{{\figurename~2}\ \ \small{Схема склеивания}}
\end{center}
}

\vspace*{18pt}

\addtocounter{figure}{1}

\noindent
ного представления: система должна находиться на иерархическом 
уровне, располагающемся непосредственно над уровнем со\-став\-ля\-ющих ее 
объектов. Иными словами, модель системы должна включать в~себя модели 
всех составляющих с~учетом их конфигурационных связей и~в~то же время 
включаться в~любые модели, включающие в~себя модели всех составляющих 
конфигурации.
   
   Поясним этот принцип на простом примере. Предположим, что нужно 
объединить в~систему два объекта~$P$ и~$S$ и~что технолог решил сделать это 
с~по\-мощью клея~--- третьего объекта~$G$, который может быть соединен 
и~с~$P$, и~с~$S$. Действие клея описывается конфигурацией следующего 
вида: объекты~$G$ и~$P$ порождают в~результате соединения известный 
промежуточный комплексный объект~$P_G$, содержащий их, а~объекты~$G$ 
и~$S$ порождают объект~$S_G$. Система~$R$, полученная путем 
склеивания~$P$ с~$S$ при помощи~$G$, отбирается среди объектов, 
содержащих~$P_G$ и~$S_G$, по следующему структурному критерию: 
объект~$R$ должен содержаться в~любом объекте~$T$, содержащем~$P_G$ 
и~$S_G$. Схематически этот критерий изображен на рис.~2.


   Если объект $R$, удовлетворяющий указанному структурному критерию, 
существует, то он действительно отвечает системе, которая собрана из~$S$ 
и~$P$ путем склеивания посредством~$G$ (и~не содержит ничего 
<<лишнего>>). Более того, легко видеть, что такой объект~$R$ определяется, 
по существу, однозначно в~том смысле, что любые два объекта~$R$ 
и~$R^\prime$, удовлетворяющие структурному критерию, содержатся друг 
в~друге. Если же нужного объекта~$R$ не существует, то делается вывод, что 
технолог ошибся: клей~$G$ не способен соединить объекты~$P$ и~$S$.
   
   В структурное представление, выполненное по стандарту IEC~81346 либо по 
ГОСТу 2.053-2013, входят только объекты~$P$, $S$ и~$R$ и~две композитные 
стрелки: $P\hm\to R$, проходящая через~$P_G$, и~$S\hm\to R$, проходящая 
через~$S_G$ (так что мегамодель склеивания~--- это часть схемы, ограниченная 
треугольником~$PSR$). Кроме того, стрелки на схеме склеивания, в~отличие от 
структуры, представляют не просто факты включения объектов друг в~друга, 
а~конкретные действия по их соединению. При этом соблюдается следующее 
естественное условие структурной корректности: если из одного объекта 
можно прийти в~другой разными путями по схеме, то эти пути задают одно и~то 
же композитное действие. Например, клей~$G$ включается в~состав 
системы~$R$ единственным способом, несмотря на наличие двух путей $G 
\hm\to  P_G \hm\to R$ и~$G \hm\to S_G \hm\to R$: в~действительности не имеет 
значения, через какой промежуточный объект <<прослеживается>> включение 
клея в~систему. Таким образом, мегамодель сборки содержит больше 
информации, чем иерархическая структура системы.
   
   Если модели содержат значения тех или иных параметров, а описание 
действий по их соединению позволяет выявить правила преобразования 
значений, то по мегамодели сборки можно вы\-чис\-лить значения параметров для 
системы. Известны примеры вычислений такого рода в~области разработки 
новых композиционных материалов~\cite{12-kov}. Осредненные 
(эффективные) физические характеристики композитов, такие как модуль Юнга и~коэффициент Пуассона, сложным образом зависят от характеристик 
компонентов и~способов изготовления композита из них. При помощи методов 
теории упру\-гости эти зависимости задаются в~форме линеаризованных 
матричных соотношений, которые приписываются к~стрелкам мегамоделей, 
пред\-став\-ля\-ющим включение компонентов в~композиты. Появляется 
возможность рассчитывать на компьютере свойства композитов по базе данных 
компонентов, без проведения дорогостоящих физических экспериментов.
   
   В заключение раздела отметим, что хотя прямой расчет системы по 
конфигурации имеет большое значение, в~MBSE он играет вспомогательную 
роль. Согласно стандарту IEC~81346 и~практикам системной инженерии, 
система обычно проектируется сверху вниз~--- от корня структурной иерархии 
к~составляющим~\cite{13-kov}. Это означает, что технолог в~основном решает 
не прямую, а~обратную задачу: модель системы, которую нужно собрать, 
известна, а~нужно построить (восстановить) конфигурацию, из которой такая 
система может быть получена путем сборки, с~учетом различных ограничений. 
Формальные математические постановки и~методы решения обратных задач 
мегамоделирования представляют собой крупную перспективную тему 
исследований, выходящую за рамки настоящей статьи.

\section{Теория категорий в~мегамоделировании}

   Как указывалось во введении, естественным источни\-ком математических 
методов кон\-стру\-ирова\-ния и~анализа мегамоделей служит теория категорий. 
Категорией называется коллекция абстрактных объектов, попарно связанных 
морфизмами (стрелками). Точное определение занимает буквально несколько 
строк~\cite{14-kov}: категория~$C$ состоит из совокупности 
объектов~$\mathrm{Ob}\,C$ и~совокупности морфизмов~$\mathrm{Mor}\,C$, 
на которых заданы следующие операции:
\begin{enumerate}[(1)]
\item каждому морфизму~$f$ 
сопоставляется два объекта: область $\mathrm{dom}\,f$ и~кообласть 
$\mathrm{codom}\,f$ (соотношения вида $\mathrm{dom}\,f \hm= A$ и~
$\mathrm{codom}\,f \hm= B$ наглядно записываются в~форме стрелки~$f$: 
$A\hm\to B$, а множество всех морфизмов, удовлетворяющих этим 
соотношениям, обозначается через $\mathrm{Mor}(A, B))$;
\item для 
любой пары морфизмов~$f, g$, удовлетворяющей условию 
$\mathrm{codom}\,f\hm = \mathrm{dom}\,g$, определена композиция~--- 
морфизм $g \circ f : \mathrm{dom}\,f \hm\to  \mathrm{codom}\,g$, причем она 
ассоциативна: для любой тройки морфизмов~$f, g, h$, удовлетворяющей 
условиям $\mathrm{codom}\,f \hm= \mathrm{dom}\,g$ и~$\mathrm{codom}\,g 
\hm= \mathrm{dom}\,h$, выполняется соотношение $h \circ (g \circ f) \hm= (h 
\circ g) \circ f$;
\item любой объект~$A$ обладает тождественным 
морфизмом~$1_A : A \to A$ таким, что для любого морфизма~$f : A\hm\to B$ 
выполняется соотношение $f \circ 1_A \hm= 1_B \circ  f \hm= f$.
\end{enumerate}

Классическим 
примером категории служит $\mathbf{Set}$, состоящая из всех множеств и~всех 
их отображений: закон композиции отображений задается стандартной 
подстановкой, а тождественным морфизмом произвольного множества служит 
его тождественное отображение на себя.
   
   Вместе с~категорией вводится понятие функтора~--- отображения категорий, 
сохраняющего структуру. Функтор $\mathrm{fun}\,: C \hm\to D$, действующий из 
категории~$C$ в~$D$,~--- это пара одноименных отображений $\mathrm{fun}\,: 
\mathrm{Ob}\,C \hm\to \mathrm{Ob}\,D$, $\mathrm{fun}\,: \mathrm{Mor}\,C \hm\to 
\mathrm{Mor}\,D$, удовлетворяющая следующим условиям (для произвольных 
$C$-мор\-физ\-мов~$f, g$ и~$C$-объ\-ек\-та~$A$): 
\begin{enumerate}[(1)]
\item $\mathrm{fun}\,(\mathrm{dom}\,f) 
\hm= \mathrm{dom}\,\mathrm{fun}\,(f), \mathrm{fun}\,(\mathrm{codom}\,f)\hm = 
\mathrm{codom}\,\mathrm{fun}\,(f)$;  
\item $\mathrm{fun}\,(g \circ f) = \mathrm{fun}\,(g) \circ \mathrm{fun}\,(f)$, 
если композиция $g \circ f$ определена; 
\item $\mathrm{fun}\,(1_A) \hm= 1_{\mathrm{fun}\,(A)}$.
\end{enumerate}
 Все категории и~все функторы образуют 
(формальную) категорию~$\mathbf{CAT}$. Чтобы исследовать взаимосвязь 
между функторами, вводится следующее понятие: естественным 
преобразованием~$\varepsilon$ функтора $\mathrm{fun}\, : C\hm\to D$ в~$\mathrm{fun}^\prime\, : C 
\hm\to D$ называется любое семейство $D$-мор\-физ\-мов~$\varepsilon_A : 
\mathrm{fun}\,(A) \hm\to \mathrm{fun}^\prime (A)$, $A \hm\in \mathrm{Ob}\,C$, 
такое что для любого 
\mbox{$C$-мор}\-физ\-ма $f : A\hm\to B$ выполняется соотношение $\varepsilon_B \circ 
\mathrm{fun}\,(f) \hm= \mathrm{fun}^\prime(f) \circ \varepsilon_A$:

%\begin{figure*} %рис
\vspace*{1pt}
\begin{center}
\mbox{%
\epsfxsize=54.473mm
\epsfbox{kov-3.eps}
}
\end{center}
%\vspace*{-9pt}
%\end{figure*}

   Эффективность применения теории категорий в~качестве математического 
аппарата \mbox{MBSE} обуслов\-ле\-на тем, что любой каталог моделей представляет 
собой не что иное, как категорию. Действительно, любая цепочка действий по 
соединению моделей порождает композитное действие (процесс) и, кроме того, 
любая модель допускает пустое действие над самой собою, не 
подразумевающее никаких изменений (процедура <<ничегонеделания>>). 
Например, в~твердотельном моделировании механических систем объектами 
категории\linebreak моделей выступают тела~--- подмножества в~$\mathbb{R}^3$, 
которые являются ограниченными, регулярными\linebreak
 (совпадают с~замыканием 
своей внутренности) и~полуаналитическими (допускают представление 
конечными булевыми комбинациями множеств вида $\{(x, y, z) \vert  F_i(x, y, 
z)\hm\leq 0\}$, где~$F_i : \mathbb{R}^3\hm\to \mathbb{R}$ является 
вещественной аналитической функцией для всех~$i$)~\cite{10-kov}. Чтобы 
было возможно задавать процедуры типа склеивания участков поверхности тел, в~категорию геометрических моделей добавляются ограниченные регулярные 
полуаналитические подмножества в~$\mathbb{R}^n$, $0 \hm\leq n \hm\leq 2$, 
при помощи стандартного вложения~$\mathbb{R}^n$ в~$\mathbb{R}^3$. Далее 
выполняется факторизация: отождествляются друг с~другом все множества, 
переходящие друг в~друга под действием аффинных изометрий. Морфизмы 
таких классов эквивалентности, описывающие действия по сборке составных 
механических сис\-тем, порождаются изометрическими вложениями множеств 
и~растяжениями. Получается подкатегория в~\textbf{Set}, которую будем обозначать 
через $\mathbf{MBS}$ (от Multibody Systems).
   
   Для многих известных технологий MBSE формальное описание каталогов 
поддерживаемых моделей приводит к~категориям множеств со структурой~--- 
алгебраических систем, топологических пространств, графов и~т.\,д. 
Морфизмами в~таких категориях служат отображения множеств, со\-вмес\-ти\-мые 
со структурой. На любой такой категории действует канонический функтор 
в~$\mathbf{Set}$, <<забывающий>> структуру. 

В~качестве примера приведем  
дис\-крет\-но-со\-бы\-тий\-ное моделирование, в~котором математической 
моделью сценария служит множество событий, час-\linebreak тич\-но упорядоченное  
при\-чин\-но-след\-ст\-вен\-ны\-ми зависимостями и~размеченное видами 
событий~\cite{15-kov}. Действия по сборке сложных сценариев задаются 
монотонными отображениями, сохраняющими разметку, поскольку ни 
события, ни зависимости, ни метки не могут быть <<потеряны>> при 
соединении сценариев поведения компонентов в~сценарии поведения систем. 
Получается категория~$\mathbf{Pomset}$, состоящая из всех помеченных 
частично упорядоченных множеств и~всех их монотонных отображений, 
сохраняющих разметку. Имеется функтор $\vert \mbox{--} \vert : 
\mathbf{Pomset}\hm\to \mathbf{Set} : S \mapsto \vert S\vert$, <<забывающий>> 
порядок и~разметку.
   
   Зафиксируем произвольную категорию~$C$, представляющую некоторый 
каталог моделей. Как и~для любой алгебраической системы, определена 
конструкция подкатегории в~$C$~--- это пара, состоящая из подкласса 
в~$\mathrm{Ob}\,C$ и~подкласса в~$\mathrm{Mor}\,C$, замкнутых 
относительно унаследованных из~$C$ операций. Подкатегория в~$C$ 
называется полной, если любой \mbox{$C$-мор}\-физм, область и~кообласть которого 
содержатся в~ней, сам содержится в~ней. Например, подкатегориями 
описываются различные аспекты структурного представления систем согласно 
стандарту IEC~81346. Действительно, композиция двух морфизмов, 
представляющих действия по формированию некоторого аспекта структуры, 
также должна входить в~этот аспект, поскольку стандарт предписывает строить 
цепочки для идентификации объектов в~структуре системы. Кроме того, если 
объект присутствует в~аспекте, то его тождественный морфизм формально 
должен быть включен в~этот аспект. В~то же время подкатегории, 
опи\-сы\-ва\-ющие все аспекты, не обязаны образовывать в~совокупности разбиение 
категории~$C$: как показывает рис.~1, возможны как действия, входящие 
в~несколько аспектов одновременно, так и~композитные действия с~переходом 
между структурами, не входящие ни в~один аспект. Требуется лишь, чтобы 
объединение классов объектов всех этих подкатегорий совпадало 
с~$\mathrm{Ob}\,C$, поскольку не имеет смысла вводить модели, не входящие 
ни в~одну структуру.
   
   Категории можно получать из графов: любой ориентированный мультиграф 
порождает категорию, объектами в~которой служат все узлы, а морфизмами~--- 
все пути. Областью и~кообластью морфизма являются соответственно начало 
и~конец пути, композиция морфизмов действует как конкатенация путей, 
а~тождественным морфизмом узла~$a$ является пустой путь из~$a$ в~$a$, не 
содержащий ни одного ребра. Отсюда получается фундаментальное понятие  
$C$-диа\-грам\-мы~--- это функтор вида~$\Delta : X \hm\to C$, где~$X$~--- 
категория, порожденная некоторым графом и~называемая схемой диаграммы. 
Все $C$-диа\-грам\-мы образуют категорию~$\mathbf{D}C$ (ковариантная 
категория <<сверхзапятой>>~\cite{14-kov}), в~которой морфизмом 
диаграммы~$\Delta : X \hm\to C$ в~$\Xi : Y \hm\to C$ служит любая пара 
вида $\langle\gamma, fd\rangle$, состоящая из функтора~$fd : X\hm\to Y$ 
и~естественного преобразования~$\gamma : \Delta\hm\to \Xi \circ fd$; закон 
композиции морфизмов диаграмм имеет вид:
$$
\langle \gamma, fd\rangle \circ 
\langle \varphi, gd\rangle \hm = \langle \gamma_{gd(-)} \circ \varphi, fd \circ 
gd\rangle\,.
$$ 
В~тео\-рии категорий накоплен богатый арсенал алгебраических 
методов конструирования и~анализа диаграмм.
   
   Любая мегамодель задается $C$-диа\-грам\-мой, так что категорное 
представление каталогов моделей позволяет формально решать задачи 
мегамоделирования. Морфизмы диаграмм описывают структурные 
преобразования мегамоделей, выполняемые при помощи инструментов MBSE. 
Покажем, как решаются средствами теории категорий прямые задачи 
мегамоделирования. Здесь применяется одна из основных  
тео\-ре\-ти\-ко-ка\-те\-гор\-ных конструкций~--- копредел  
диаграммы~\cite{5-kov}, который строится следующим образом. Обозначим 
через~$\mathbf{1}$ категорию,\linebreak состоящую из одного объекта~0 и~одного 
морфизма~$1_0$. Из любой категории~$X$ имеется в~точ\-ности один 
функтор~$!_X : X \hm\to \mathbf{1}$, сопоставляющий объект~0  
любому~$X$-объ\-ек\-ту (иными словами, $\mathbf{1}$ является терминальным 
$\mathbf{CAT}$-объ\-ек\-том). Имеется вложение (инъективный функтор) 
$\ulcorner \mbox{--}\urcorner : C \hookrightarrow \mathbf{D}C$, сопоставляющее 
произвольному $C$-объ\-ек\-ту $Q$~точку~--- диаграмму $\ulcorner Q\urcorner : 
\mathbf{1}\hm\to  C : 0 \mapsto Q$. Коконусом (cocone) называется 
$\mathbf{D}C$-мор\-физм, имеющий точку в~качестве кообласти. Можно 
изобразить коконус $\langle \sigma, !_X\rangle : \Delta\hm\to \ulcorner 
Q\urcorner$ над диаграммой $\Delta : X\hm\to C$ в~виде диаграммы, 
<<пририсовав>> к~$\Delta$ дополнительную вершину, помеченную 
объектом~$Q$, и~набор ребер~--- стрелок, по одной для каждого узла $I\hm\in 
\mathrm{Ob}\,X$, направленной из~$I$ в~вершину и~помеченной морфизмом 
$\sigma_I : \Delta (I) \hm\to Q$. Копределом (colimit) диаграммы~$\Delta$ 
называется коконус $\mathrm{colim}\,\Delta : \Delta\hm\to \ulcorner R\urcorner$, 
универсальный в~том смысле, что для любых \mbox{$C$-объ}\-ек\-та~$T$ 
и~коконуса~$\delta : \Delta\hm\to\ulcorner T\urcorner$ существует единственный 
$C$-мор\-физм~$w : R \hm\to T$ такой, что $\delta\hm= \ulcorner w\urcorner \circ  
\mathrm{colim}\,\Delta$. Легко видеть, что это условие универсальности 
представляет собой в~точности структурный критерий из разд.~3. Таким 
образом, конструирование копредела конфигурации~$\Delta$ описывает на 
строгом математическом языке сборку системы, которой отвечает 
вершина~$R$. В~категориях типа $\mathbf{MBS}$ и~$\mathbf{Pomset}$ 
построение копредела сводится к~факторизации раздельных объединений 
объектов, представляющих компоненты системы, по отношениям 
эквивалентности, индуцированным моделями клея и~других средств сборки.
   
   Копредел любой диаграммы, если он сущест\-вует, определяется однозначно 
   с~точностью до изомор\-физма. Более того, можно описать сборку сис\-тем из 
конфигураций в~виде функтора. Пусть $Cd$~--- некоторый класс  
$C$-диа\-грамм, имеющих копределы. Он порождает полную подкатегорию 
в~$\mathbf{D}C$, из которой в~$C$ действует функтор копредела $\mathrm{colim}$, 
сопоставляя каждой диаграмме из~$Cd$~вершину некоторого ее копредела, а 
каждому \mbox{$\mathbf{D}C$-мор}\-физ\-му~$\theta : \Delta\hm\to \Xi$, 
где~$\Delta, \Xi\hm\in Cd$~--- стрелку копредела $\mathrm{colim}\,(\theta)$ такую, что 
$\mathrm{colim}\,\Xi \circ \theta \hm= \ulcorner \mathrm{colim}\,(\theta)\urcorner \circ 
\mathrm{colim}\,\Delta$.

%\begin{figure*}
\vspace*{1pt}
\begin{center}
\mbox{%
\epsfxsize=56.127mm
\epsfbox{kov-4.eps}
}
\end{center}
%\vspace*{-9pt}
%\end{figure*}

   Например, в~категории \textbf{Set} любая диаграмма имеет 
копредел~\cite[упражнение~5.1.8]{14-kov}, поэтому имеется функтор $\mathrm{colim}\, : 
\mathbf{D}(\mathbf{Set})\hm\to \mathbf{Set}$. Примечательно, что этот функтор 
является рефлектором: он сопряжен слева с~вложением $\ulcorner \mbox{--}\urcorner : 
\mathbf{Set}\hookrightarrow \mathbf{D}(\mathbf{Set})$, причем 
единица рефлексии состоит из $\mathbf{D}(\mathbf{Set})$-мор\-физ\-мов 
$\mathrm{colim}\,\Delta : \Delta\hm\to \ulcorner\mathrm{colim}\,(\Delta)\urcorner$, 
$\Delta\hm\in \mathrm{Ob}\ \mathbf{D}(\mathbf{Set})$. Напомним, что единица 
рефлексии~--- это естественное преобразование тождественного функтора 
в~композицию рефлектора и~вложения (в~данном случае, естественное 
преобразование функтора $1_{\mathbf{D}(\mathbf{Set})}$ в~$\ulcorner \mathrm{colim}\,(  
\mbox{--})\urcorner)$, состоящее из универсальных  
стрелок~\cite[разд.~4.3]{14-kov}. И~для произвольного класса~$Cd$, 
содержащего достаточное количество одноточечных диаграмм, функтор 
$\mathrm{colim}$ сопряжен слева с~ограничением  
вложения~$\ulcorner \mbox{--}\urcorner$ на подходящую полную подкатегорию 
в~$C$. А~поскольку сопряженный функтор задается однозначно с~точностью 
до изоморфизма~\cite[разд.~4.1]{14-kov}, можно сделать вывод, что сборка 
систем в~некотором смысле <<зашифрована>> в~процедуре построения 
одноточечных диаграмм~--- моделей систем как целого без раскрытия 
струк\-туры. 

Так наглядно проявляется двойственность прямых и~обратных задач 
мегамоделирования.

\section{Заключение}

   Аппарат теории категорий обладает большим потенциалом в~области 
повышения полезной отдачи от MBSE, в~том числе путем математически 
строгого решения задач мегамоделирования. Так, базовая процедура системной 
инженерии~--- сборка\linebreak
 системы из заданной конфигурации взаимо\-свя\-занных 
компонентов~--- формально описывается тео\-ретико-ка\-те\-гор\-ной 
конструкцией копредела диа\-граммы. Более сложные конструкции отвечают\linebreak 
сложным процедурам сборки, таким как связывание (weaving) общесистемных 
функций, рассеянных по всем компонентам (crosscutting concerns), например 
мониторинговых или защитных~\cite{16-kov}. Математического представления 
требуют и~другие процедуры MBSE, в~частности коллективная модификация 
мегамоделей и~составляющих моделей, восстановление конфигурации заданной 
системы, оценка взаимозаменяемости компонентов. 

Актуальны вопросы 
внедрения аппарата теории категорий в~практику, в~том числе путем развития 
программных инструментов моделирования и~мегамоделирования. Здесь 
открывается широкий спектр направлений для дальнейших исследований.
   
{\small\frenchspacing
 {%\baselineskip=10.8pt
 \addcontentsline{toc}{section}{References}
 \begin{thebibliography}{99}
\bibitem{1-kov}
Modeling and simulation-based systems engineering handbook~/
Eds.\ D.~Gianni,  A.~D'Ambrogio, A.~Tolk.~--- London: CRC Press, 2014. 513~p.
\bibitem{2-kov}
\Au{Ковалёв С.\,П., Толок~А.\,В.} Применение модельно-ори\-ен\-ти\-ро\-ван\-но\-го подхода 
в~управ\-ле\-нии жизненным циклом технических изделий~// Информационные технологии 
в~проектировании и~производстве, 2015. №\,2. С.~3--9.
\bibitem{3-kov}
\Au{Левенчук А.\,И.} Системноинженерное мышление.~--- М.: TechInvestLab, 2015. 305~с.
\bibitem{4-kov}
IEC 81346-1:2009. Industrial Systems, Installations and Equipment and Industrial Products~--- 
Structuring Principles and Reference Designations~--- Part~1: Basic Rules.~--- Geneva: ISO, 2009. 
168~p.
\bibitem{5-kov}
\Au{Ginali S., Goguen~J.} A~categorical approach to general systems~// 
 Conference (International) on Applied General Systems 
Research Proceedings~/
Ed. G.\,J.~Klir.~--- NATO conference series.~--- New York, NY, USA: Plenum 
Press, 1978. Vol.~5. P.~257--270.
\bibitem{6-kov}
\Au{Mabrok M.\,A., Ryan M.\,J.} Category theory as a~formal mathematical foundation for  
model-based systems engineering~// Appl. Math. Inform. Sci., 2017. Vol.~11. No.\,1. P.~43--51.
\bibitem{7-kov}
\Au{Ковалёв С.\,П.} Тео\-ре\-ти\-ко-ка\-те\-гор\-ный подход к~проектированию программных 
сис\-тем~// Фундаментальная и~прикладная математика, 2014. Т.~19. Вып.~3. С.~111--170.
\bibitem{8-kov}
\Au{B$\acute{\mbox{e}}$zivin J., Jouault~F., Rosenthal~P., Valduriez~P.} Modeling in the large 
and modeling in the small~// Model Driven Architecture: European MDA Workshops on 
Foundations and Applications Proceedings~/
Eds.\ U.~A{\!\ptb{\ss}}mann, M.~Aksit,  A.~Rensink.~--- 
Lecture notes in computer science ser.~--- Springer, 2005. Vol.~3599. 
P.~33--46.
\bibitem{9-kov}
\Au{Diskin Z., Kokaly~S., Maibaum~T.} Mapping-aware mega\-mod\-eling: Design patterns and 
laws~// Software Language Engineering: 6th Conference (International) Proceedings~/
Eds.\ M.~Erwig, R.\,F.~Paige, E.~Van Wyk.~--- 
Lecture notes  in computer science ser.~--- Springer, 2013. Vol.~8225. P.~322--343.
\bibitem{10-kov}
\Au{Requicha A.\,G.} Representations for rigid solids: Theory, methods, and systems~// 
ACM  Comput. Surv., 1980. Vol.~12. Iss.~4. P.~437--464.
\bibitem{11-kov}
\Au{K$\acute{\mbox{a}}$d$\acute{\mbox{a}}$r B., Pfeiffer~A., Monostori~L.} Discrete event 
simulation for supporting production planning and scheduling decisions in digital
 factories~//  37th 
CIRP Seminar (International) on Manufacturing Systems Proceedings.~--- Budapest, 2004.  
P.~444--448.
\bibitem{12-kov}
\Au{Giesa T., Spivak D.\,I., Buehler~M.\,J.} Category theory based solution for the building block 
replacement problem in materials design~// Adv. Eng. Mater., 2012. Vol.~14. 
Iss.~9. P.~810--817.
\bibitem{13-kov}
\Au{Косяков А., Свит У., Сеймур~С., Бимер~С.} Системная инженерия. Принципы 
и~практика~/ Пер. с~англ.~--- М.: ДМК-Пресс, 2014. 636~с. (\Au{Kossiakoff~A., Sweet~W.\,N., 
Seymour~S., Biemer~S.\,M.} Systems engineering principles and practice.~--- 2nd ed.~--- New 
York, NY, USA: John Wiley, 2011. 560~p.)
\bibitem{14-kov}
\Au{Маклейн С.} Категории для работающего математика~/ Пер. с~англ.~--- М.: Физматлит, 
2004. 352~с. (\Au{Mac Lane~S.} Categories for the working mathematician.~--- New York, NY, 
USA: Springer, 1978. 317~p.)
\bibitem{15-kov}
\Au{Pratt V.\,R.} Modeling concurrency with partial orders~// Int. J.~Parallel 
Prog., 1986. Vol.~15. No.\,1. P.~33--71.
\bibitem{16-kov}
\Au{Ковалёв С.\,П.} Семантика ас\-пект\-но-ори\-ен\-ти\-ро\-ван\-но\-го моделирования 
данных и~процессов~// Информатика и~её применения, 2013. Т.~7. Вып.~3. С.~70--80.
 \end{thebibliography}

 }
 }

\end{multicols}

\vspace*{-3pt}

\hfill{\small\textit{Поступила в~редакцию 16.01.17}}

%\vspace*{8pt}

\newpage

\vspace*{-30pt}

%\hrule

%\vspace*{2pt}

%\hrule

%\vspace*{8pt}


\def\tit{METHODS OF CATEGORY THEORY IN~MODEL-BASED SYSTEMS ENGINEERING\\[-7pt]}

\def\titkol{Methods of category theory in~model-based systems engineering}

\def\aut{S.\,P.~Kovalyov\\[-12pt]}

\def\autkol{S.\,P.~Kovalyov}

\titel{\tit}{\aut}{\autkol}{\titkol}

\vspace*{-14pt}


\noindent
Institute of Control Sciences, Russian Academy of Sciences, 65~Profsoyuznaya Str., 
Moscow 117997, Russian Federation



\def\leftfootline{\small{\textbf{\thepage}
\hfill INFORMATIKA I EE PRIMENENIYA~--- INFORMATICS AND
APPLICATIONS\ \ \ 2017\ \ \ volume~11\ \ \ issue\ 3}
}%
 \def\rightfootline{\small{INFORMATIKA I EE PRIMENENIYA~---
INFORMATICS AND APPLICATIONS\ \ \ 2017\ \ \ volume~11\ \ \ issue\ 3
\hfill \textbf{\thepage}}}

\vspace*{1pt}

 

\Abste{A mathematical device based on the category theory is proposed to formally describe and 
rigorously explore procedures of employing models in engineering that constitute the contents of 
model-based systems engineering (MBSE). The essence of the device consists in mathematical 
representation of assembly drawings (megamodels of systems) as diagrams in categories whose 
objects are models, and morphisms represent actions associated with assembling system models 
from component models. The soundness of the device is justified on the basis of standards that 
govern description of the systems' structure such as IEC~81346. Category-theoretical methods for 
solving a number of practical problems of assembling systems are proposed and explored. 
Examples of solving such problems are provided in categories that represent two key application 
areas for MBSE: geometric modeling of complex shapes and discrete-event simulation of the 
behavior of industrial systems.}

\KWE{ model-based systems engineering; megamodel; category theory; colimit}

\DOI{10.14357/19922264170305} 

%\vspace*{-18pt}

%\Ack
%\noindent




\vspace*{-7pt}

  \begin{multicols}{2}

\renewcommand{\bibname}{\protect\rmfamily References}
%\renewcommand{\bibname}{\large\protect\rm References}

{\small\frenchspacing
 {%\baselineskip=10.8pt
 \addcontentsline{toc}{section}{References}
 \begin{thebibliography}{99}
\bibitem{1-kov-1}
Gianni, D., A.~D'Ambrogio, and A.~Tolk, eds. 2014. \textit{Modeling and simulation-based 
systems engineering handbook}. London: CRC Press. 513~p.
\bibitem{2-kov-1}
\Aue{Kovalyov, S.\,P., and A.\,V.~Tolok.} 2015. Primenenie model'no-orientirovannogo podkhoda 
v~upravlenii zhiznennym tsiklom tekhnicheskikh izdeliy [Applying model-based approach 
to product lifecycle management].\linebreak \textit{Informatsionnye tekhnologii v~proektirovanii 
i~proizvod\-st\-ve} [Information Technologies in Design and Industry] 2(158):3--9.
\bibitem{3-kov-1}
\Aue{Levenchuk A.\,I.} 2015. 
\textit{Sistemnoinzhenernoe myshlenie} [Systems engineering thinking]. 
Moscow: TechInvestLab. 305~p.
\bibitem{4-kov-1}
IEC 81346-1:2009. 2009. 
Industrial Systems, Installations and Equipment and Industrial 
Products~--- Structuring Principles and Reference Designations~--- 
Part~1: Basic Rules. Geneva:  ISO. 168~p.
\bibitem{5-kov-1}
\Aue{Ginali, S., and J.~Goguen.} 1978. 
A~categorical approach to general systems. \textit{Conference 
(International) on Applied General Systems Research Proceedings}. Ed.\
 G.\,J.~Klir. \mbox{NATO}  conference ser. Plenum Press. 5:257--270.
\bibitem{6-kov-1}
\Aue{Mabrok, M.\,A., and M.\,J.~Ryan}. 
2017. Category theory as a~formal mathematical foundation for 
model-based systems engineering. \textit{Appl. Math.  Inform. Sci.} 11(1):43--51.
\bibitem{7-kov-1}
\Aue{Kovalyov, S.\,P.} 2016. 
Category-theoretic approach to software systems design. \textit{J.~Math. Sci.} 
214(6):814--853.
\bibitem{8-kov-1}
\Aue{B$\acute{\mbox{e}}$zivin, J., F.~Jouault, P.~Rosenthal, and P.~Valduriez.}
 2005. Modeling in 
the large and modeling in the small. 
\textit{Model Driven Architecture: European MDA Workshops on 
Foundations and Applications Proceedings.} 
Eds.\ U.~\mbox{A{\!\ptb{\ss}}mann}, M.~Aksit, and A.~Rensink. 
Lecture notes in computer science ser. Springer. 3599:33--46.
\bibitem{9-kov-1}
\Aue{Diskin, Z., S.~Kokaly, and T.~Maibaum.} 2013. 
Mapping-aware megamodeling: Design patterns 
and laws. \textit{6th Conference (International) on Software Language Engineering 
Proceedings}. Eds.\ M.~Erwig, R.\,F.~Paige, and E.~Van Wyk. 
Lecture notes in computer science ser. Springer. 
8225:322--343.
\bibitem{10-kov-1}
\Aue{Requicha, A.\,G.} 1980. Representations for rigid solids: 
Theory, methods, and systems. \textit{ACM 
Comput. Surv.} 12(4):437--464.
\bibitem{11-kov-1}
\Aue{K$\acute{\mbox{a}}$d$\acute{\mbox{a}}$r,~B., A.~Pfeiffer, and L.~Monostori.}
2004. Discrete 
event simulation for supporting production planning and scheduling decisions in 
digital factories. \textit{37th CIRP Seminar (International) on Manufacturing 
Systems Proceedings}. Budapest.  444--448.
\bibitem{12-kov-1}
\Aue{Giesa, T., D.\,I.~Spivak, and M.\,J.~Buehler.} 2012. 
Category theory based solution for the building 
block replacement problem in materials design. 
\textit{Adv. Eng. Mater.} 14(9):810--817.
\bibitem{13-kov-1}
\Aue{Kossiakoff, A., W.\,N.~Sweet, S.~Seymour, and S.\,M.~Bie\-mer.}
2011. \textit{Systems engineering 
principles and practice}. 2nd ed. New York, NY: John Wiley. 560~p.
\bibitem{14-kov-1}
\Aue{Mac Lane, S.} 1978. \textit{Categories for the working mathematician}. 
New York, NY: Springer. 317~p.
\bibitem{15-kov-1}
\Aue{Pratt, V.\,R.} 1986. Modeling concurrency with partial orders. 
\textit{Int. J.~Parallel Prog.} 15(1):33--71.
\bibitem{16-kov-1}
\Aue{Kovalyov, S.\,P.} 2013. 
Semantika aspektno-ori\-en\-ti\-ro\-van\-no\-go modelirovaniya dannykh 
i~protsessov [Semantics of aspect-oriented modeling of data and processes]. 
\textit{Informatika i~ee  Primeneniya~--- Inform. Appl.} 7(3):70--80.
\end{thebibliography}

 }
 }

\end{multicols}

\vspace*{-9pt}

\hfill{\small\textit{Received January 16, 2017}}

\vspace*{-18pt}

\Contrl

\noindent
\textbf{Kovalyov Sergey P.} (b.\ 1972)~--- Doctor of Science in physics and 
mathematics, leading scientist, Institute of Control Problems, Russian 
Academy of Sciences, 65~Profsoyuznaya Str., Moscow 117997, Russian 
Federation Federation; \mbox{kovalyov@nm.ru} 

\label{end\stat}


\renewcommand{\bibname}{\protect\rm Литература}  %8
\def\stat{kozhunova}

\def\tit{КОГНИТИВНАЯ ИНТЕРОПЕРАБЕЛЬНОСТЬ ЭКСПЕРТНОГО ВЗАИМОДЕЙСТВИЯ В~ЗАДАЧЕ ОБРАБОТКИ 
РУССКО-ФРАНЦУЗСКИХ ПАРАЛЛЕЛЬНЫХ ТЕКСТОВ: ЛИНГВОКОГНИТИВНЫЕ АСПЕКТЫ}

\def\titkol{Когнитивная интероперабельность экспертного взаимодействия в~задаче обработки
%русско-французских 
параллельных текстов} %: лингвокогнитивные аспекты}

\def\autkol{О.\,С.~Кожунова}

\def\aut{О.\,С.~Кожунова$^1$}

\titel{\tit}{\aut}{\autkol}{\titkol}

%{\renewcommand{\thefootnote}{\fnsymbol{footnote}}\footnotetext[1] {Статья 
%рекомендована к публикации в журнале Программным комитетом конференции 
%<<Электронные библиотеки: перспективные методы и технологии, электронные 
%коллекции>> (RCDL-2012).}}

\renewcommand{\thefootnote}{\arabic{footnote}}
\footnotetext[1]{Институт проблем информатики Российской академии наук, 
kozhunovka@mail.ru}

 

\Abst{Обсуждаются ресурсы информационно-коммуникационных технологий (ИКТ)
<<По\-пол\-ня\-емая база лингвистических данных по 
трудностям перевода>> и <<Специальный тезаурус рус\-ско-фран\-цуз\-ских 
параллельных текстов>>, которые находятся на стадии проектирования и будут 
разработаны одновременно с созданием параллельного корпуса 
рус\-ско-фран\-цуз\-ских художественных текстов. Помимо их функциональности 
рассматриваются лингвокогнитивные аспекты взаимодействия экспертов различных 
областей, решающих задачу обработки рус\-ско-фран\-цуз\-ских параллельных 
текстов совместными усилиями.}

\KW{когнитивная интероперабельность; задача обработки естественного языка; 
рус\-ско-фран\-цуз\-ские параллельные тексты}

\vskip 14pt plus 9pt minus 6pt

      \thispagestyle{headings}

      \begin{multicols}{2}

            \label{st\stat}

\section{Введение}

     Сегодня задача автоматической обработки параллельных текстов и 
создания соответствующего инструментария в помощь филологу, который 
занимается сравнительным языкознанием, переводоведением и другими 
аспектами анализа переводных текстов, находится на пике своей 
актуальности. Это происходит, поскольку 
ИКТ достигли уровня развития, позволяющего моделировать и частично замещать 
деятельность экспертов. Особенно ресурсы ИКТ востребованы в задачах 
машинного перевода, сопоставления параллельных текстов на разных 
языках, сопоставления языковых структур различного уровня для проведения 
лингвистического анализа текстов, формирования выводов об интересующем 
исследователя языковом явлении, например о поведении грамматической 
конструкции, использования выразительных средств в языке и~т.\,п.
     
     За рубежом активные попытки создания необходимого инструментария 
предпринимались уже в 1990-х~гг.\ и в настоящее время 
приобрели более сфокусированный характер, т.\,е.\ нацелены на решение 
специфических проблем и задач корпусных исследований~[1--4]. 

Отечественные филологи и специалисты по компьютерной лингвистике 
также заинтересованы в решении узкоспециализированных задач корпусной 
лингвистики, но помимо этого предпринимают попытки к созданию 
универсальных ресурсов для обработки и выравнивания параллельных 
текстов~[5, 6].
     
     Обсуждаемые в настоящей работе ИКТ-ре\-сур\-сы <<Пополняемая 
база лингвистических данных по трудностям перевода>> и <<Специальный 
тезаурус рус\-ско-фран\-цуз\-ских параллельных текстов>> находятся на 
стадии проектирования и будут разработаны одновременно с созданием 
параллельного корпуса рус\-ско-фран\-цуз\-ских художественных текстов. 
Эти ресурсы позволят не только выйти на качественно иной уровень работы 
с параллельными текстами, в том числе в составе лингвистического корпуса, 
но и решить те актуальные задачи, которые сегодня стоят перед филологами, 
работающими в областях сравнительного языкознания и переводоведения, а 
именно: выявление и\linebreak фикса\-ция трудностей рус\-ско-фран\-цуз\-ско\-го 
перевода, сис\-тематизация и типология таких трудностей с пред\-ложениями по 
их разрешению и примерами из параллель\-ных текстов, составление тезауруса 
переводных рус\-ско-фран\-цуз\-ских терминов определенной 
стилистической направленности, установ\-ле\-ние необходимых связей и 
отношений между ними и~т.\,д.
     
     Кроме научно-исследовательского и практического применения 
вышеупомянутых лингвистических ресурсов предполагается также 
использовать их для проведения дистанционных корпусных исследований 
студентами, аспирантами и докторантами и пополнить их для этих целей 
учебными программами выравнивания параллельных художественных 
текстов.
     
     В работе уделяется особое внимание понятию когнитивной 
интероперабельности\footnote{Интероперабельность (\textit{англ}.\ 
{interoperability})~--- способность к взаимодействию.} экспертного 
взаимодействия в рамках вышеупомянутой задачи в силу ее 
междисциплинарности и сложности в установлении такого взаимодействия 
между экспертами из разных предметных областей. 

\section{Параллельный корпус и~лингвистические 
ресурсы информационно-коммуникационных технологий}
     
     Параллельный корпус рус\-ско-фран\-цуз\-ских художественных 
текстов необходим по многим причинам. Во-пер\-вых, в виду активного 
создания и редактирования <<Национального корпуса русского языка>>~[7] 
соответствующие инициативы, связанные с созданием параллельных 
корпусов, являются ожидаемыми и востребованными, в том числе и в рамках 
международного на\-уч\-но-ис\-сле\-до\-ва\-тель\-ско\-го сотрудничества в 
этой области. Во-вто\-рых, на материале художественных текстов языковые 
несоответствия русского и французского языков и сложности перевода с 
одного языка на другой выступают наиболее ярко, поэтому предлагается в 
первую очередь создавать соответствующий корпус именно художественных 
параллельных текстов.
     
     Построение пополняемой базы лингвистических данных по трудностям 
перевода является первым шагом в создании необходимых ИКТ-ре\-сур\-сов, 
сопровождающих корпус параллельных русско-фран\-цуз\-ских текстов. 
Эта база лингвистических данных предназначена для формализации 
типологии трудностей перевода, установления 
     при\-чин\-но-след\-ст\-вен\-ных и иных отношений между отдельными 
трудностями и целыми классами, сопоставления отдельных трудностей с 
примерами из текстов и даже для размещения вариантов разрешения 
обозначенных трудностей перевода. Такая база данных предоставит богатый 
материал не только для филологических исследований в области 
переводоведения и сравнительного языкознания, но и позволит сделать 
качественный рывок в практическом рус\-ско-фран\-цуз\-ском переводе~[8].
     
     Формирование учебных программ выравнивания параллельных 
художественных текстов позволит вовлекать в решение важных проблем 
корпусной лингвистики, сравнительного языкознания и разделов филологии, 
связанных с переводоведением, студентов, аспирантов и молодых 
специалистов, развивать у них навыки практических научных работ во 
многих разделах филологии, а также позволит готовить новые кадры для 
будущих исследований в этой области.
     
     Помимо вышеперечисленных ресурсов необходимо подчеркнуть 
важность создания специального тезауруса рус\-ско-фран\-цуз\-ских 
параллельных текстов. Поскольку в рамках построения параллельных 
     рус\-ско-фран\-цуз\-ских корпусов возникает множество 
филологических и лингвистических задач, связанных как с извлечением 
данных и знаний из текста, сопоставлением структур разного уровня анализа, 
так и с представлением знаний, описанных в тексте, классификацией 
терминов и связей между ними, задача построения тезауруса 
     рус\-ско-фран\-цуз\-ских терминов является актуальной. Кроме того, 
решение этой задачи дополняет процесс создания параллельных корпусов и 
релевантные исследования. Аккумуляция рус\-ско-фран\-цуз\-ской лексики 
определенной стилистики с одновременной фиксацией семантических, 
родовидовых и других тезаурусных отношений позволит качественно 
изменить подход к анализу рус\-ско-фран\-цуз\-ских текстов и содержащихся 
в них языковых структур и отношений.
     
     Идея построения специального тезауруса и его применения в задаче 
обработки параллельных рус\-ско-фран\-цуз\-ских текстов с последующим 
формированием соответствующего корпуса основана на успешном опыте 
масштабных проектов по созданию многоязычных тезаурусов и 
тезаурусоподобных лингвистических ресурсов. Одним из наиболее 
распространенных типов таких ресурсов являются автоматизированные 
словари, построенные по модели WordNet~[9--12].
     
     Проект по разработке словаря Princeton WordNet (PWN) английского 
языка в Принстонском университете (США) стартовал в первой половине 
1980-х~гг.\ и продолжается по сей день. Сейчас уже доступна версия 
2.0 WordNet. Существующая версия охватывает более 120~тыс.\ слов 
общеупотребительной лексики современного английского 
     языка~\cite{12-ko}.
     
     \begin{figure*} %fig1
\vspace*{1pt}
 \begin{center}
 \mbox{%
 \epsfxsize=162.463mm
 \epsfbox{koz-1.eps}
 }
 \end{center}
 \vspace*{-3pt}
\Caption{Фрагмент архитектуры базы данных EuroWordNet для английского и 
испанского языков}
\end{figure*}
     
     Этот словарь~--- базу данных тезаурусного типа~--- можно использовать 
для различных лингвистических задач. В~частности, при проведении 
информационного поиска wordnet-сло\-ва\-ри применяются для расширения 
запроса пользователя за счет парадигматически и синтагматически 
связанных слов, например компонентов синсета (множества синонимов, 
объединенных в набор) вместе с его гипонимами и согипонимами или связей 
типа <<гла\-гол--ак\-тант>>, которые дают возможность осуществлять 
контекстный поиск. Данные о синтагматических отношениях слов позволяют 
применять wordnet-сло\-ва\-ри для решения задачи снятия неоднозначности 
смысла слова. Wordnet можно использовать для вычисления смысловой 
близости текстов на основе гиперонимических отношений. 
     Wordnet-сло\-ва\-ри могут служить лексиконом для формальных 
грамматик. Формат wordnet является удобным формализмом для 
представления состава и структуры лексики специальных подъязыков 
(например, медицинских, экономических терминов). Wordnet-сло\-ва\-ри 
являются удобным инструментом для проведения исследований в области 
лексической семантики, например гипонимические отношения в 
     wordnet-сло\-ва\-рях позволяют определять направление 
метонимических переносов и прогнозировать появление новых 
     лек\-си\-ко-се\-ман\-ти\-че\-ских вариантов~\cite{13-ko}.

     За период с марта 1996~г.\ по сентябрь 1999~г.\ при финансировании 
Европейской комиссии был создан многоязычный вариант WordNet~--- 
EuroWordNet~\cite{14-ko}, что стало новым этапом в эволюции word\-net-сло\-ва\-рей. 
В~рамках европейского проекта было создано не только несколько 
тезаурусов для европейских языков (голландского, испан\-ского, италь\-ян\-ско\-го, 
немецкого, французского, чешско\-го и эстонского), но и впервые была 
реализована идея объединения отдельных wordnet-пред\-став\-ле\-ний в 
общую систему. Все компоненты EuroWordNet были построены по единой 
модели, что, \mbox{однако}, не предполагало прямого перевода анг\-лий\-ско\-го 
варианта WordNet~1.5. Перед разработчиками стояла задача~--- отразить все 
особенности лексических систем национальных языков. 

Со\-вмес\-ти\-мость 
компонентов EuroWordNet была обеспечена \mbox{единством} принципов и 
заданным набором общих понятий (Basic Concepts), на основе которых 
определялась система межъязыковых отсылок (Inter-Lingual-Index), дающих 
возможность переходить от лексикализованных значений одного языка к 
сходным, но не обязательно тождественным значениям в другом языке. 
Данный индекс позволяет использовать EuroWordNet не только для 
информационного поиска в рамках одного языка, но и для многоязычного 
поиска (рис.~1).
     
     
     В рамках проекта EuroWordNet первоначальная структура словаря 
претерпела серьезные изменения. Был расширен набор семантических 
отношений за счет парадигматических отношений, связывающих слова 
разных час\-тей речи (например, XPOS\_NEAR\_SYNONYMY: dead--death; 
XPOS\_HYPERONYMY: to love\,--\,emotion; XPOS\_ANTONYMY: to live\,--\,dead) и 
синтагматических отношений между глаголами и ак\-тан\-та\-ми-су\-ще\-ст\-ви\-тель\-ны\-ми 
(например, ROLE\_INSTRUMENT: to write\,--\,pencil). 
Был сформирован новый подход к построению wordnet-сло\-ва\-рей: с опорой 
на использование лексикографических источников (толковых, переводных и 
синонимических словарей) и результатов обработки корпусов современных 
текстов. 
     
     Успешное завершение проекта EuroWordNet послужило толчком к 
созданию большого числа wordnet-пред\-став\-ле\-ний для языков разных 
типов (например, венгерского, турецкого, арабского, тамильского, 
китайского и~пр.), а также многоязычных ресурсов типа EuroWordNet 
(например, проект BalkaNet нацелен на объединение греческого, румынского, 
болгарского, сербского, турецкого и чешского wordnet-сло\-ва\-рей). 
В~2001~г.\ была создана Всемирная Ассоциация WordNet (Global WordNet 
Association), целью которой является объединение уже существующих и 
только развивающихся национальных ресурсов этого типа, 
усовершенствование системы межъязыковых индексов и разработка общих 
стандартов, позволяющих использовать модель WordNet для языков разных 
типов~\cite{6-ko}.
     
     С 1999~г.\ на кафедре математической лингвистики СПбГУ 
исследовательская группа под руководством И.\,В.~Азаровой 
(Азарова~И.\,В., Митрофанова~О.\,А., Синопальникова~А.\,А.\ и~др.)\ ведет 
работы по проекту RussNet~--- созданию русской версии компьютерного 
словаря типа WordNet~\cite{15-ko}. В~задачи проекта входит построение 
лек\-си\-ко-се\-ман\-ти\-че\-ско\-го ресурса для отражения организации 
лексической системы русского языка в целом, для\linebreak представления ядра его 
общеупотребительной лексики и фиксации семантических, 
     се\-ман\-ти\-ко-грам\-ма\-ти\-че\-ских и 
     се\-ман\-ти\-ко-де\-ри\-ва\-ци\-он\-ных отношений русского языка. 
Кроме того, в настоящее \mbox{время} в Петербургском государственном 
университете путей сообщения разрабатывается проект русской\linebreak версии 
WordNet под руководством С.\,А.~Яблонского и 
     А.\,М.~Сухоногова~\cite{12-ko}.
     %
     Поэтому предполагается, что построение специального тезауруса\linebreak 
     рус\-ско-фран\-цуз\-ских параллельных текстов с учетом 
меж\-ду\-на\-род\-ного опыта формирования многоязычных тезаурусов и 
особенностей меж\-дис\-цип-\linebreak ли\-нар\-ной задачи создания и обработки корпуса\linebreak 
параллельных текстов позволит проводить филологические исследования 
большего масштаба и глубины.
     
     Поскольку сама задача создания корпуса параллельных текстов 
является междисциплинарной, то методы и подходы, задействованные в ее 
решении, так же разнообразны, а именно: методы системного анализа, 
искусственного интеллекта, компьютерной лингвистики, психолингвистики, 
когнитивного моделирования и~т.\,п. В~частности, используются методы 
извлечения данных из текстов, подходы к представлению знаний, 
классификации терминов и методы построения и обработки запросов на 
поиск в слабоструктурированных полнотекстовых документах, методы 
моделирования человеческого восприятия информации различного уровня 
сложности и~т.\,п. Тем самым предполагается вовлечение в работу экспертов 
из нескольких областей: компьютерной лингвистики, текстологии, 
переводоведения, искусственного интеллекта, когнитивной науки, 
психологии и~др. Такое многообразие специалистов, безусловно, 
повлечет за собой ряд затруднений в решении поставленной задачи, 
поскольку их картины мира, акценты и ракурсы взгляда на одни и те же 
     дан\-ные\,/\,си\-ту\-а\-тив\-ные кон\-текс\-ты\,/\,проб\-ле\-мы изначально 
отличаются друг от друга. Предложенные способы согласования 
восприятия информации и соответствующих концептуальных связях будут 
описаны в разд.~3.

\vspace*{-6pt}

\section{Опыт междисциплинарных задач}

\vspace*{-2pt}

     Ранее было сказано о междисциплинарности задачи построения 
     рус\-ско-фран\-цуз\-ско\-го корпуса и соответствующих 
лингвистических ресурсов. Действительно, для решения такой 
многоплановой задачи необходимо обратиться к экспертным знаниям и 
компетенциям из разных областей. Однако и сама идея вовлечения экспертов 
с разными подходами к мировосприятию и 
     на\-уч\-но-ис\-сле\-до\-ва\-тель\-ски\-ми парадигмами таит в себе еще 
одну научную проблему: обеспечение взаимодействия экспертов из разных 
областей для выполнения одной задачи и согласование их картин мира, 
образов ситуации, ракурсов и подходов к решению возникающих вопросов. 

     \begin{figure*}[b] 
%          \vspace*{-12pt} %fig2
%\noindent
\begin{center}
   \begin{fmpage}{155mm}
    Показатели
     \begin{center}
     \begin{enumerate}[1.]
     \item Индикаторы\\[-13pt]
     \begin{enumerate}[{1.}1]
\item Индикаторы результатов фундаментальных 
научных исследований (научные результаты)\\[-13pt]
\begin{enumerate}[{1.1.}1]
\item
Непосредственные результаты\\[-13pt]
\item Целевые результаты\\[-13pt]
\item Индикаторы взаимосвязей и влияния 
научных результатов\\[-13pt]
\begin{enumerate}[{1.1.3.}1]
\item 
Взаимосвязи и влияние на здравоохранение\\[-13pt]
\item 
Взаимосвязи и влияние на развитие сферы 
науки\\[-13pt]
\item Взаимосвязи и влияние на развитие 
технологий\\

%\begin{enumerate}[1.1.3.3.}1]
1.1.3.3.1.~Индексы самоцитирования в патентах
%\end{enumerate}
\item Взаимосвязи и влияние на образование\\[-13pt]
\end{enumerate}
\end{enumerate}
\end{enumerate}
\item Критерии\\[-13pt]
\item  Параметры\\[-13pt]
\item  Экспертные оценки
\end{enumerate}

%\vspace*{3pt}
      \end{center}
       \end{fmpage}
     \Caption{Классификационная схема, использованная для построения структуры 
семантического словаря}
\end{center}
      \end{figure*}

     
     Несмотря на новизну этой проблемы, у автора уже имеется опыт 
работы с междисциплинарными задачами и подходы к согласованию 
действий и понятий экспертов в разных областях знаний. В~первую очередь, 
опыт получен в рамках подготовки диссертационной работы~\cite{16-ko}, 
посвященной созданию и исследованию технологии разработки 
семантического словаря показателей для сис\-тем информационного 
мониторинга. Одной из проблем, решаемой в рамках этой работы, была 
проблема различия в понимании экспертами смыс\-ла индикаторов, поскольку 
это является серьезным препятствием в реализации всех трех основных 
процедур, необходимых для оценивания про\-грам\-мной деятельности в сфере 
науки: информационный мониторинг, анализ, получение количественных и 
экспертных оценок ее результатов, эффективности и результативности. Это 
вызвало необходимость решения задачи согласования понимания 
индикаторов разными экспертами. Было отмечено, что в силу особенностей 
формирования терминов мониторинга возникает также задача частной 
референции, когда одно название индикатора может обозначать целый класс 
индикаторов (например, индексы цитирования, смысл которых зависит от 
учета самоцитирования, а также цитирования соавторами и~т.\,п.). Поэтому 
был предложен и разработан семантический словарь, который позволил 
отобразить информацию и связи, необходимые для решения задачи. Новизна 
предложенного семантического словаря состоит в том, что он содержит 
ссылки на алгоритмические и информационные ресурсы системы 
информационного мониторинга, а также нормативные документы как 
источники терминов рассматриваемой предметной области. Построенная 
формализация процесса извлечения терминов мониторинга и их определений 
из массива текстов (нормативные документы, научные статьи и~т.\,д.)\ 
позволила выделить необходимые индикаторы, связи между ними и их 
определения. Это облегчило дальнейшую интеграцию индикаторов и других 
показателей мониторинга в классификационную схему семантического 
словаря (рис.~2 и~3)~\cite{16-ko}.
     

\begin{figure*} %fig3
\vspace*{1pt}
 \begin{center}
 \mbox{%
 \epsfxsize=150.718mm
 \epsfbox{koz-3.eps}
 }
 \end{center}
 \vspace*{-6pt}
\Caption{Связь статей семантического словаря с ресурсами ИТСМ РАН (ИТСМ~--- 
Ин\-фор\-ма\-ци\-он\-но-тех\-но\-ло\-ги\-че\-ская система мониторинга, разрабатываемая в 
 в отделе~16 ИПИ РАН)}
\end{figure*}

     
     После завершения диссертационного исследования была начата 
     на\-уч\-но-ис\-сле\-до\-ва\-тель\-ская работа 
     <<Лек\-си\-ко-се\-ман\-ти\-че\-ские методы создания 
     проблемно-ориентированных лингвистических ресурсов информационных систем>>, 
выполняемая в Институте проблем информатики РАН в 2011--2013~гг. Само 
название и поставленные задачи уже говорят о междисциплинарности этого 
исследования:
\columnbreak

\noindent
     \begin{enumerate}[(1)]
\item семантическое моделирование предметных областей в рамках 
научно-ис\-сле\-до\-ва\-тель\-ской работы (НИР) как основа создания лингвистического обеспечения 
информационных систем;
\item модификация лингвистических ресурсов сис\-тем 
информационного мониторинга с применением 
лек\-си\-ко-се\-ман\-ти\-че\-ских методов в процессе создания систем 
информационного мониторин\-га научной деятельности;
\item формализация и компьютерное моделирование трудностей 
перевода.
\end{enumerate}

     Из перечисленных задач реализованы и успешно сданы подзадачи 
задач~1--2, задача~3 находится в процессе выполнения. В~ходе выполнения 
НИР был получен опыт и разработаны подходы к извлечению знаний из 
слабоструктурированных текстов, решению задач компьютерной 
лингвистики применительно к области информационного мониторинга и их 
формализации и моделированию в рамках разрабатываемой 
информационной сис\-те\-мы, проектированию и созданию макета 
лингвистических ресурсов информационной сис\-те\-мы (семантический и 
проективный словари), компонентному анализу лексических единиц на 
немецком языке и\linebreak установлению лек\-си\-ко-смыс\-ло\-вых связей с 
семантическими единицами русского языка (на примере 
     рус\-ско-не\-мец\-ких языковых пар патентных \mbox{текстов}), выявлению 
различных языковых трансформаций и их классификации, а также 
разработка метапредставлений для компонентного 
     лек\-си\-ко-се\-ман\-ти\-че\-ско\-го анализа~[18--21].
     
     Разработанные в рамках НИР подходы были успешно применены в 
междисциплинарных задачах информационного мониторинга и 
со\-по\-став\-ле\-ния параллельных патентных текстов~\cite{21-ko, 22-ko}. Поэто\-му 
при взаимодействии с экспертами геоинформатики, задачи которой также 
подразумевают вовлечение специалистов из нескольких предметных 
областей, были применены разработанные подходы, дополненные 
возможностями понятий когнитивной интероперабельности, когнитивного 
пространства и ряда междисциплинарных подходов и методов, образованных 
на стыке прикладной лингвистики, когнитивной психологии и 
искусственного интеллекта. Предлагаемый подход был успешно применен в 
междисциплинарных задачах геоинформатики~\cite{23-ko}.
     
     В рамках данного подхода были предложены и частично реализованы 
следующие задачи: 
     \begin{itemize}
\item разработка лингвистических методов и моделей обеспечения 
когнитивной интероперабельности экспертной 
ин\-фор\-ма\-ци\-он\-но-ана\-ли\-ти\-че\-ской деятельности на основе 
геоинформационных описаний;
\item формирование корпуса параллельных тематических текстов 
различных предметных областей на нескольких языках. Параллельные 
многоязычные корпуса в данном случае предназначены для накопления 
эмпирических данных по проблематике предметных областей, в 
частности в вопросах полноты и согласованности терминосистем, 
экспертного ин\-фор\-ма\-ци\-он\-но-ана\-ли\-ти\-че\-ско\-го 
взаимодействия и~т.\,д., а так\-же для апробации созданного корпуса и 
ког\-ни\-тив\-но-линг\-ви\-сти\-че\-ских методов на массиве реальных 
кросс-язы\-ко\-вых данных;
\item разработка программных приложений для апробации 
предлагаемого подхода к моделированию когнитивного пространства 
взаимодействия экспертов в рамках заданной предметной области. 
Предполагается также апробация приложений в междисциплинарных 
задачах с целью разрешения терминологических разногласий между 
экспертами смежных областей, восстановления адекватных 
при\-чин\-но-след\-ст\-вен\-ных связей между понятиями предметной 
области, обеспечения возможности принятия решений в случаях 
недоопределенных терминосистем или их отсутствия и других задачах;
\item создание информационных технологий (ИТ) нового поколения, учитывающих особенности 
взаимодействия экспертов разного уровня как внутри предметных 
областей, так и в междисциплинарных задачах.
\end{itemize}

     Предлагаемый подход был апробирован на практике в ходе 
координации экс\-пер\-тов-ана\-ли\-ти\-ков, принимающих согласованные 
решения на основе анализа геоинформационных описаний (рис.~4).
     
     Существующие методы лингвистического анализа, применяемые к 
задачам формирования и\linebreak\vspace*{-12pt}

\pagebreak

\end{multicols}

            \begin{figure} %fig4
      \vspace*{1pt}
 \begin{center}
 \mbox{%
 \epsfxsize=161.258mm
 \epsfbox{koz-4.eps}
 }
 \end{center}
 \vspace*{-6pt}
\Caption{Взаимодействие экспертов из разных предметных областей в 
геоинформационной системе}
\end{figure}

\begin{multicols}{2}

\noindent
 сопровождения интеллектуальных 
геоинформационных сис\-тем, потребовали привлечения методов,\linebreak которые 
позволили выделить необходимые единицы в соответствующих 
информационных структурах, фиксировать связи между ними, а также 
выявить и отобразить их семантику и ситуативный контекст~\cite{24-ko}. 

Среди исследованных методов были выделены методы 
     лек\-си\-ко-се\-ман\-ти\-че\-ско\-го моделирования когнитивных 
структур знаний, которые позволяют учесть особенности 
геоинформационной среды~\cite{23-ko}.
     В работе~\cite{23-ko}, в частности, говорилось о том, что структура 
интеллектуального анализа геоинформационных данных основана на 
гибридных формах представления знаний, а именно иерархической 
классификации с возможностью установления и корректировки нескольких 
видов те\-зау\-рус\-но-се\-ман\-ти\-че\-ских отношений (собственно 
семантические, отношения часть--це\-лое, ги\-по\-ним--ги\-пе\-ро\-ним 
и~т.\,д.). В~данной структуре, в отличие от онтологий, акцент смещен в 
сторону когнитивных аспектов описания и семантического наполнения 
терминов, которыми оперируют в работе эксперты различных областей и 
уровней компетенции, а не на описание самой предметной области. 
     
     Структура была спроектирована и представлена на основе xsd-схем, 
заполнение которых объектами отрасли и их связями было реализовано в 
отдельных xml-фай\-лах. Анализ и валидация предложенной структуры 
осуществлены на основе программных модулей на C++. Затем был 
разработан подход на основе лек\-си\-ко-се\-ман\-ти\-че\-ско\-го 
моделирования к интеграции лингвистического обеспечения тезаурусного 
типа, совместимого с заданной структурой, определенными в ней объектами 
и отношениями. В~результате были получены программные модули на C++, 
разделенные по функциональности: поддержка структуры интеллектуального 
анализа данных в сис\-те\-ме, сопровождение и наполнение тезауруса 
геоинформационной сис\-те\-мы, обеспечение согласованной интеграции 
тезауруса и сопряженного лингвистического обеспечения в 
геоинформационную сис\-те\-му~\cite{23-ko}.
     
     Следующим ключевым понятием, задействованным в разработке 
междисциплинарного подхода к координации экс\-пер\-тов-ана\-ли\-ти\-ков, 
принимающих согласованные решения на основе анализа 
геоинформационных описаний, было понятие когнитивного пространства. 
Когнитивное пространство с учетом многофакторности взаимодействия 
экспертов на основе геоинформационных описаний расширяет понятие 
единого геоинформационного пространства~\cite{25-ko}. Были 
сформулированы требования к разрабатываемой архитектуре единого 
геоинформационного пространства на основе сравнительного анализа 
существующих метасхем баз геоданных с учетом семантики, заложенной в 
пространственные онтологии. Проведенный анализ исследований показал 
необходимость разработки модели, учитывающей латентные атрибутивные 
связи между онтологиями разнородных геоданных,\linebreak что должно повысить 
качество отображаемых связей в концептуальной схеме базы геоданных. 
Предлагаемая архитектура представляет собой многоуровневую структуру 
базы геоданных, при этом \mbox{каждый} из уровней ориентирован на решение 
определенного класса задач и содержит определенный набор 
пользовательских и библиотечных процедур управления процессами 
обработки геоданных для создания инструментария управления 
про\-грам\-мным комплексом~\cite{23-ko}.
     
     Следует отметить, что впервые инструментарий и возможности 
     лек\-си\-ко-се\-ман\-ти\-че\-ских методов и моделирования были 
применены в задачах гео\-информатики. Интеграция лингвистического 
обеспе\-чения тезаурусного типа на основе лек\-си\-ко-се\-ман\-ти\-че\-ско\-го 
моделирования позволила сгруппировать разнородные геоданные и 
структуры в едином геоинформационном пространстве в рамках 
разработанной гибридной формы представления знаний~--- иерархической 
классификации с элементами лек\-си\-ко-се\-ман\-ти\-че\-ских отношений, а 
также выявить объекты и понятия данной предметной области и фиксировать 
те\-зау\-рус\-но-се\-ман\-ти\-че\-ские отношения между ними~\cite{23-ko}.
     
     Методы и подходы лексико-семантического модели\-рования, которые в 
настоящее время широко применяются в различных предметных об\-лас\-тях, 
впервые были использованы при проектировании архитектуры 
лингвистического обеспечения геоинформационного пространства, 
форми\-ровании структуры интеллектуального анализа геодан\-ных в 
геоинформационных сис\-те\-мах и интеграции лингвистического 
обеспечения тезаурусного типа в единое геоинформационное пространство. 
Лек\-си\-ко-се\-ман\-ти\-че\-ское моделирование применительно к задачам и 
проблемам геоинформатики позволяет использовать средства анализа 
глубинных структур языка для извлечения геоданных, их связей и 
отношений, встраивания их в заданную геоинформационную структуру и 
верификации их семантики в разрезе формирования единого 
геоинформа\-ционного пространства. Инструменты глубинного\linebreak представ\-ле\-ния 
языковых структур способствуют разрешению многочисленных 
неоднозначностей геоданных и связанной с этим рассогласованности при 
принятии важных решений экс\-пер\-та\-ми-ана\-ли\-ти\-ка\-ми, а также 
формировать единое геоинформационное пространство, позволяющее 
отображать адекватную информационную структуру геоданных~\cite{23-ko}.
     
     В качестве возможных приложений предложенного подхода 
рассматриваются следующие задачи: координация работы экспертов из 
различных предмет\-ных областей, в том числе в междисциплинарных задачах, 
в информационных сис\-те\-мах в ав\-то\-ма\-ти\-че\-ском/по\-лу\-ав\-то\-ма\-ти\-че\-ском 
режиме; междисциплинарные научные исследования различных предметных 
областей с целью разработки проб\-лем\-но-ори\-ен\-ти\-ро\-ван\-ных 
программных приложений и установления взаимодействия экспертов 
различного профиля и уровня подготовки; изучение методов и подходов 
компьютерной лингвистики и их адап\-тив\-ных приложений для создания 
нового поколения~ИТ.

\section{Когнитивная интероперабельность и~предлагаемый 
подход}
     
     Вышеописанная структура представления знаний хорошо себя показала 
применительно к задачам информационного мониторинга и геоинформатики. 
Но будет ли она так же универсальна, если эксперты столкнутся с 
ситуациями неопреде\-лен\-ности ключевых понятий, с нечеткими 
постановками задач, с лексической полисемией и другими препятствиями к 
однозначной и эксплицитной трактовке терминов, их смысловых связей и их 
восприятия специалистами из разных областей (например, психологии, 
лингвистики, информатики и~др.)?
     
     Определенно, необходимо усовершенствовать предложенный подход с 
учетом указанных ког\-ни\-тив\-но-линг\-ви\-сти\-че\-ских механизмов 
восприятия информации человеком. По этому поводу было подготовлено 
много содержательных работ. Так, в~\cite{26-ko} рассматриваются 
механизмы образования лексической полисемии и даже предлагается 
     кон\-цеп\-ту\-аль\-но-смыс\-ло\-вая модель ее образования. 
Моделирование взаимодействия смысловых значений слова и изучение 
истоков его многозначности существенно расширит возможности любого 
тезауруса предметной области. В~частности, использование автором 
работы~\cite{26-ko} понятия базового концепта~\cite{27-ko} и дихотомии 
<<основное vs.\ производное значение слова>>~\cite{28-ko} позволит 
моделировать потенциальные девиации смыслов терминов и 
позиционировать новые смыслы среди традиционных, а также выделять 
основные, <<ядерные>> понятия предметной области и выстраивать вокруг 
них такую ие\-рар\-хи\-чески-се\-те\-вую структуру представления знаний, 
которая наиболее полно отражает терминологический портрет изучаемой 
     об\-ласти/за\-да\-чи/проб\-ле\-мы. Последние соображения связаны с 
понятием базового концепта, который определяется как обобщенный 
(концептуальный) объект, пред\-став\-ля\-ющий собой сложную когнитивную 
единицу~--- совокупность Формы, Действий и Интенций~\cite{26-ko}:
     \begin{multline*}
\mbox{Концепт} = \mbox{концептуальный Объект} ={}\\
{}= (\mbox{Форма, Действия, 
Интенции})\,.
\end{multline*}

     Под Формой здесь понимается структура элементарных 
пространственных объемов, под Интенциями~--- содержательные 
характеристики Формы (желания, цели, намерения, потребности, функции 
и~т.\,п.), а Действия представляют собой типичные физические действия 
Формы, посредством которых реализуются ее Интенции~\cite{26-ko}. 
Интересно, что такой концептуальный Объект задает свою категорию 
конкретных объектов, схожих с ним по всем трем характеристикам. 
В~процессе когнитивного развития ребенка именно эта пара~--- 
концептуальный Объект и задаваемая им категория~--- позволяет познавать 
мир и расширять границы уже познанного. Это означает что, поскольку такое 
представление понятий не зависит от родного языка носителя и хранится в 
его долговременной памяти, механизм фиксации концептуальных Объектов и 
их категорий чрезвычайно важен при отображении знаний экспертов о 
     ка\-кой-ли\-бо об\-ласти и позволяет выработать обобщенный подход к 
формализации понятий и их связей.
     %
     Что касается основных и производных значений~\cite{28-ko}, то, как 
было отмечено выше, их фиксация и встраивание в 
     иерар\-хи\-чески-се\-те\-вую структуру знаний некоторой области 
позволит не только моделировать потенциальные девиации смыслов 
терминов, но и позиционировать новые смыслы среди традиционных. 

Такой 
подход к моделированию смыслов играет важную роль при взаимодействии 
экспертов из разных областей, так как производные смыслы не хранятся в 
памяти носителя языка, а основные смыс\-лы одних и тех же понятий у разных 
людей могут существенно отличаться друг от друга (и даже у одних и тех же 
людей, но в разных языковых ситуациях) (см.\ пример~1). Это позволит 
обеспечить когнитивную интероперабельность~\cite{29-ko} в их работе и 
достичь поставленных целей при решении ими междисциплинарных 
     за\-дач/проб\-лем.

\medskip

\noindent
\textbf{Пример 1.}

{\small
\textit{По полю рыжей стрелой летела лиса} (основное значение).

\textit{Ну и лиса эта Ваша Маша!} (производное значение).

\textit{Вон смотри, какая обезьяна!} (в зоопарке, про животное~--- основное значение).

\textit{Фу, какая обезьяна!} (про уродливого человека или волосатого мужчину).
}

\medskip

     Приведенные выше соображения находят подтверждения и в других 
публикациях, в том числе в~\cite{30-ko}. Авторы провели серию 
     ког\-ни\-тив\-но-линг\-ви\-сти\-че\-ских экспериментов с носителями 
языка, что позволило им выявить интересные закономерности восприятия 
людьми лексических единиц языка и их смысловых связей. Например, при 
прослушивании устной речи неоднозначными оказываются единицы, 
совпадающие только по звучанию, но различающиеся написанием (омофоны: 
плот--плод, по\-рог--по\-рок и~т.\,д.), а при восприятии письменной речи, 
напротив,~--- совпадающие по написанию, но различающиеся звучанием 
(омографы: зам\textbf{о}к~--- з\textbf{а}мок, м\textbf{у}ка~--- мук\textbf{а} 
и~т.\,п.)~\cite{30-ko}. Кроме того, Черниговская и соавторы подчеркивают, 
что выбор значения слова в первую очередь происходит на основании его 
структуры, а затем уже учитывается его контекст. Любопытно, что в одном 
из экспериментов, описанном ими, оценивалась роль частотности при 
интерпретации лексически неоднозначного фрагмента речи при восприятии 
омофонов. В~результате была подтверждена первостепенная роль 
частотности словоформ при осуществлении выбора между омофонами.

\begin{figure*}[b] %fig5
%\vspace*{9pt}
 \begin{center}
 \mbox{%
 \epsfxsize=109.851mm
 \epsfbox{koz-5.eps}
 }
 \end{center}
 \vspace*{-6pt}
\Caption{Специальный тезаурус русско-французских параллельных текстов и его связи с 
другими лингвистическими ресурсами}
\end{figure*}
     
     Отсюда можно сделать вывод о том, что для успешного 
взаимодействия экспертов в рамках такой междисциплинарной задачи, как 
построение и\linebreak ведение базы данных трудностей рус\-ско-фран\-цуз\-ско\-го 
перевода и обеспечения когнитивной ин\-тер\-опе\-ра\-бель\-ности их деятельности 
(т.\,е.\ в двух различных системах эксперты видят согласованные \mbox{образы} 
представляемой информации), действительно необходим такой ресурс, как 
специальный тезаурус рус\-ско-фран\-цуз\-ских параллельных текстов. Этот 
тезаурус позволит не просто зафиксировать специальные термины и понятия 
из различных областей и семантические связи между ними, но и отобразит 
информацию об особенностях восприятия этих терминов разными 
специалистами в разных речевых ситуациях и контекстах.
     
     Такого рода информация справедливо отнесена к области 
<<когнитивной семантики>> автором работы~\cite{31-ko}, который, как и 
автор~\cite{26-ko}, апеллирует к подходам Д.~Лакоффа~\cite{32-ko}. 
Кузнецов подчеркивает, что значения слов возникают раньше, чем 
концептуальные структуры (из доконцептуального телесного опыта)~[31]. Под 
доконцептуальными структурами здесь подразумеваются гештальты\linebreak (единые 
ментальные образы) и об\-раз\-но-схе\-ма\-ти\-че\-ские структуры: 
вместилище, верх--низ, часть--це\-лое и~т.\,д. Причем связанные с ними 
кон\-цеп\-ты непосредственно значимы, что влияет на непосредствен\-ное, 
однозначное восприятие предложения или фразы. Поэтому понимание, 
согласно Лакоффу, есть не что иное, как способность соотносить концепты 
со своим опытом, включая доконцептуальный. Следовательно, чтобы 
обеспечить возможность понимания экспертами необходимых 
     тер\-ми\-нов/кон\-цеп\-тов/си\-ту\-а\-ций и прочего в едином ключе, 
необходимо прежде всего облегчить восприятие ими заданной информации 
на максимально доступном уровне с точки зрения общечеловеческих 
когнитивных способностей такого рода. В~частности, использовать 
     об\-раз\-но-схе\-ма\-ти\-че\-ские структуры, которые дают возможность 
людям рассуждать быстрее машин~\cite{31-ko}.
     
     За понятием образно-схе\-ма\-ти\-че\-ской струк\-ту\-ры скрывается 
идея о том, что существуют определенные схемы, которые человек 
изначально накладывает на воспринимаемый мир. Например, людям 
свойственно представлять большие фрагменты своего повседневного опыта в 
терминах вместилища, в английском языке характеризуемого в первую 
очередь через предлоги <<in>> и <<out>>. Так, мы выходим из (out) 
полусонного состояния, забытья, смотрим в (in) зеркало и~т.\,п. Лакофф 
провел специальное лингвистическое исследование, в котором на материале 
600\,(!) глаголов английского языка демонстрируется категоризация по схеме 
<<вместилище>>~\cite{33-ko}.
     
     Таким образом, адекватная с точки зрения человеческого восприятия 
форма представления знаний в специальном тезаурусе, в том числе 
доконцептуальных структур, может существенно облегчить восприятие 
информации, необходимой для взаимодействия экспертов, причем 
независимо от их изначальной профессиональной специализации.

\medskip

\noindent
\textbf{Пример 2.}


{\small Фрагмент из повести А.\,П.~Чехова <<Скучная история>> и ее художественный 
перевод на французский язык:

\textit{В детстве и в юности я почему-то питал страх к швейцарам и к театральным 
капельдинерам}\ldots

\textit{Dans mon enfance et mon adolescence, j'avais, je ne sais pourquoi, peur}  
{\bfseries\textit{des Suisses}} \textit{et des ouvreurs de theater}\ldots

Здесь {\bfseries\textit{des Suisses}} дословно переводится с французского языка как 
<<швейцарцы>> (национальность).
}

\medskip
     
     Возможная информация в специальном тезаурусе по поводу этого 
случая (пример~2):

\medskip
     
\hangindent=5mm{\small \hspace*{\parindent}Образно-схематическая структура повестей А.\,П.~Чехова 
<<Прислуга~--- швейцары, капельдинеры, горничные>> сопоставляется с 
аналогичной французской структурой <<portiers, concierges, ouvreurs de loges, 
femmes de chambre>>. В~базе данных трудностей перевода этот случай 
фиксируется с отсылкой к тезаурусу для согласованного восприятия и 
отображения экспертами параллельных текстов с пометой возможной 
подмены французскими лингвистами термина <<швейцар>> термином 
<<швейцарец>>.}

\smallskip
     
     Получившаяся концептуальная схема специального тезауруса 
     рус\-ско-фран\-цуз\-ских параллельных текстов, разработанная с 
учетом вышеизложенных дополнений по усовершенствованию подхода к 
обеспечению когнитивной интероперабельности\linebreak\vspace*{-12pt}

\pagebreak

\end{multicols}

\begin{figure} %fig6
\vspace*{1pt}
 \begin{center}
 \mbox{%
 \epsfxsize=114.703mm
 \epsfbox{koz-6.eps}
 }
 \end{center}
 \vspace*{-6pt}
\Caption{Концептуальная схема усовершенствованного специального тезауруса}
\vspace*{9pt}
\end{figure}

\begin{multicols}{2}

\noindent
 экспертов различных 
областей, которые работают над созданием и обработкой параллельного 
корпуса рус\-ско-фран\-цуз\-ских художественных текстов, представлена 
на рис.~5 и~6.


\vspace*{-6pt}     


\section{Заключение}

     Создание таких актуальных лингвистических ресурсов, как корпус 
рус\-ско-фран\-цуз\-ских параллельных художественных текстов, 
пополняемая \mbox{база} лингвистических данных по трудностям перевода, учебные 
программы выравнивания параллельных художественных текстов, а также 
специальный тезаурус рус\-ско-фран\-цуз\-ских параллельных текстов, влечет 
за собой необходимость совмещения разнообразных междисциплинарных 
подходов и, как следствие этого процесса, вызывает потребность в 
установлении взаимодействия и согласования базовых понятий у экспертов 
из разных об\-ластей.
     
     В данной работе был приведен опыт работы с междисциплинарными 
задачами и краткое описание сформированных в результате подходов. 

Далее 
были рассмотрены лингвистические аспекты когнитивного восприятия 
информации экспертами разных областей, включая механизмы выделения 
базовых концептов, дихотомию <<основные vs.\ производные значения>>, 
фиксацию неоднозначных лексических единиц и ситуаций (омографы, 
омофоны, лексическая полисемия), учет час\-тот\-ности словоформ и 
ситуативного контекста в трактовке смысла терминов, отображение 
     об\-раз\-но-схе\-ма\-ти\-че\-ских струк\-тур для определенных 
языковых ситуаций и~т.\,д.

\pagebreak
     
     Рассмотренные аспекты позволили принять во внимание 
соответствующие особенности ког\-ни\-тив\-но-линг\-ви\-сти\-че\-ских 
механизмов восприятия информации и картины мира людьми~--- 
экспертами, а также усовершенствовать подход к обеспечению когнитивной 
интероперабельности экспертной деятельности на примере 
междисциплинарной\linebreak задачи по созданию и обработке корпуса 
     рус\-ско-фран\-цуз\-ских параллельных художественных текстов и 
сопутствующих ресурсов (пополняемая \mbox{база} лингвистических данных по 
трудностям перевода; учебные программы выравнивания параллельных 
художественных текстов; специальный тезаурус рус\-ско-фран\-цуз\-ских 
параллельных текстов). Все это позволило улучшить концептуальную схему 
специального тезауруса, тем самым подготовив почву к его разработке и 
апробации.

{\small\frenchspacing
{%\baselineskip=10.8pt
\addcontentsline{toc}{section}{Литература}
\begin{thebibliography}{99}

\bibitem{1-ko}
\Au{Sinclair J.} The automatic analysis of corpora~// Directions in Corpus Linguistics: Nobel 
Symposium 82 Proceedings.~--- Berlin: Mouton de Gruyter, 1992.
\bibitem{2-ko}
\Au{Wallis S., Nelson G.} Knowledge discovery in grammatically analysed corpora~// Data 
Mining and Knowledge Discovery, 2001. Vol.~5. P.~307--340. 
\bibitem{3-ko}
\Au{Dukes K., Atwell E., Habash~N.} Supervised collaboration for syntactic annotation of 
quranic arabic~// Language Resources and Evaluation~J., Special Issue on Collaboratively 
Constructed Language Resources, 2011.
\bibitem{4-ko}
\Au{McCarthy D.} Exploiting distributional similarity for lexical acquisition~// Компьютерная 
лингвистика и интеллектуальные технологии: По мат-лам ежегодной международной 
конф. <<Диалог'2011>>.~--- М.: РГГУ, 2011. Вып.~10(17). C.~19--31.

\bibitem{6-ko}
\Au{Азарова И.\,В., Синопальникова А.\,А., Яворская~М.\,В.} Принципы построения 
wordnet-те\-зау\-ру\-са RussNet~// Компьютерная лингвистика и интеллектуальные 
технологии: Труды Междунар. конф. Диалог'2004.~--- М., 2004. С.~542--547.

\bibitem{5-ko}
\Au{Ляшевская О.\,Н., Кузнецова Ю.\,Л.} Русский Фреймнет: к задаче создания корпусного 
словаря конструкций~// Компьютерная лингвистика и интеллектуальные технологии: По 
мат-лам ежегодной Междунар. конф. <<Диалог'2009>>.~---  М.: РГГУ, 2009. 
Вып.~8(15). C.~306--312.

\bibitem{7-ko}
Национальный корпус русского языка: Сайт проекта {\sf http://www.ruscorpora.ru}.


\bibitem{8-ko}
\Au{Бунтман Н.\,В., Зацман И.\,М.} О~проекте создания\linebreak компьютерного ресурса 
трудностей перевода: замет-\linebreak ки на полях~// Маргиналии-2010: границы культуры и текста: 
Тезисы II Междунар. конф.~--- М.: МГУ, 2010. 
С.~41--43. {\sf http://uni-persona.srcc.msu.su/site/ conf/marginalii-2010/thesis.htm}.


\bibitem{11-ko-1} %9
\Au{Miller~G.\,A.} 
Five papers on WordNet. CSL-Report.~--- Princeton: Princeton University, 1990.  Vol.~43.

\bibitem{10-ko} %10
\Au{Fellbaum C.} WordNet: An electronic lexical database.~--- Cambridge, 1998. 



\bibitem{11-ko} %11
\Au{Кожунова О.\,С.} Семантический словарь системы информационного мониторинга в 
сфере науки и ресурс Eurowordnet: структура, задачи и функции~// Сис\-те\-мы и средства 
информатики.~---  М.: Наука, 2008. Вып.~18. С.~156--170.

\bibitem{9-ko} %12
\Au{Кожунова О.\,С.} Подходы к лек\-си\-ко-се\-ман\-ти\-че\-ско\-му моделированию и 
лингвистические ресурсы информационных сис\-тем~// Сис\-те\-мы и средства 
информатики.~--- М: ИПИ РАН, 2011. С.~139--161.

\bibitem{12-ko} %13
\Au{Сухоногов А.\,М., Яблонский С.\,А.} Словари типа WordNet в технологиях Semantic 
Web~// Конф. по искусственному интеллекту (КИИ-2004): Тр. 9-й Национальной 
конф. по искусственному интеллекту с международным участием.~--- В~3-х~т.~--- М.: 
Физ\-мат\-лит, 2004. Т.~2. С.~557--564.
\bibitem{13-ko} %14
\Au{Азарова И.\,В., Митрофанова О.\,А., Синопальникова~А.\,А., Ушакова~А.\,А., 
Яворская~М.\,В.} Разработка компьютерного тезауруса русского языка типа WordNet~// 
Корпусная лингвистика и лингвистические базы данных: Докл. науч. конф.~/ Под 
ред. А.\,С.~Герда.~--- СПб.: СПбГУ, 2002. С.~6--18.
\bibitem{14-ko} %15
\Au{Vossen P.} Introduction to EuroWordNet~// Computers  Humanities, 1998. Vol.~32. 
No.\,2--3. P.~73--89.
\bibitem{15-ko} %16
\Au{Азарова И.\,В., Митрофанова О.\,А., Синопальникова~А.\,А.} Компьютерный тезаурус 
русского языка типа WordNet~// Мат-лы конф. <<Диалог-2003>>.~--- М., 2003.
\bibitem{16-ko} %17
\Au{Кожунова О.\,С.} Технология разработки семантического словаря системы 
информационного мониторинга: Автореф. дисс.\ \ldots\  канд. техн. наук.~--- М.: ИПИ 
РАН, 2009. 2~ с.
\bibitem{17-ko} %18
\Au{Зацман И.\,М., Дурново А.\,А.} Моделирование процессов формирования экспертных 
знаний для мониторинга про\-грам\-мно-це\-ле\-вой деятельности~// Информатика и её 
применения, 2011. Т.~5. Вып.~4. С.~84--98.
\bibitem{18-ko} %19
\Au{Kozhunova O.} Lexical and semantic methods in design of the problem-oriented linguistic 
resources~// WORLDCOMP'11:  2011 World Congress in Computer Science, Computer 
Engineering and Applied Computing Proceedings.~--- Las Vegas: CSREA Press, 2011. Vol.~II. 
P.~618--624.
\bibitem{19-ko} %20
\Au{Kozhunova O.} Cross-disciplinary approach to expert activity cognitive interoperability 
support~//  5th Conference (International) on Cognitive Science Proceedings.~--- Kaliningrad, 
 2012. С.~91--92.
\bibitem{20-ko} %21
\Au{Кожунова О.\,С.} Моделирование лексической семантики в задачах компьютерной 
лингвистики~//  Сис\-те\-мы и средства информатики, 2012. Т.~22. №\,1. С.~86--109.
\bibitem{21-ko} %22
\Au{Kozhunova O.} Detection of nominalized structures in parallel patent texts in Russian and in 
German~// WORLDCOMP'09:  2009 World Congress in Computer Science, Computer 
Engineering and Applied Computing Proceedings.~--- Las Vegas: CSREA Press, 2009. Vol.~I. 
P.~479--485.
\bibitem{22-ko} %23
\Au{Кожунова О.\,С.} Выявление номинализованных конструкций в параллельных 
текстах патентных документов на русском и немецком языках~// Компьютерная 
лингвистика и интеллектуальные технологии: По мат-лам ежегодной Междунар. 
конф. <<Диалог'2009>>.~--- М.: РГГУ, 2009. Вып.~8(15). C.~185--191.
\bibitem{23-ko} %24
\Au{Дулин С.\,К., Дулина Н.\,Г., Кожунова~О.\,С.} Когнитивная интероперабельность 
экспертной деятельности и ее приложение в геоинформатике~// Конф. по искусственному 
интеллекту (КИИ-2012): Труды 13-й Национальной конф. по искусственному интеллекту с 
международным участием.~--- Белгород: БГТУ им. В.\,Г.~Шухова, 2012. С.~351--357.
\bibitem{24-ko} %25
\Au{Дулин С.\,К., Розенберг И.\,Н.} О~развитии методологических основ и концепций 
геоинформатики~// Сис\-те\-мы и средства информатики. Спец. вып.: 
На\-уч\-но-ме\-то\-до\-ло\-ги\-че\-ские проблемы информатики.~--- М.: ИПИ РАН, 2006. С.~201--256.
\bibitem{25-ko} %26
\Au{Цветков В.\,Я.} Информатизация, инновационные процессы и геоинформационные 
технологии~// Геодезия и аэрофотосъемка, 2006. №\,4. С.~112--118.
\bibitem{26-ko} %27
\Au{Кошелев А.\,Д.} Кон\-цеп\-ту\-аль\-но-смыс\-ло\-вая модель образования лексической 
полисемии~// 5-я Междунар. конф.\ по когнитивной науке: Тезисы докладов.~--- 
Калининград, 2012. Т.~2. С.~464--465.
\bibitem{27-ko} %28
\Au{Norvig P., Lakoff G.} Taking: A~study in lexical network theory~//  13th Berkeley 
Linguistics Society Annual Meeting Proceedings: BLS, 1987. P.~195--206.
\bibitem{28-ko} %29
\Au{Виноградов В.\,В.} Основные типы лексических значений слова~// 
Избранные труды. Лексикология и лексикография.~--- М., 1977. С.~162--189.
\bibitem{29-ko} %30
\Au{Buddenberg R.} Toward an interoperability reference model, 2006. {\sf 
http://web1.nps.navy.mil/?budden/\linebreak lecture.notes/interop RM.html}.
\bibitem{30-ko} %31
\Au{Черниговская Т.\,В., Дубасова~А.\,В., Риехакайнен~Е.\,И.} Лексическая 
неоднозначность и организация ментального лексикона~// 5-я Междунар. конф. по 
когнитивной науке: Тезисы докладов.~--- Калининград, 2012. Т.~2. С.~698--700.
\bibitem{31-ko} %32
\Au{Кузнецов О.\,П.} О~возможности организации знаний на основе когнитивной 
семантики~// 5-я Междунар. конф. по когнитивной науке: Тезисы докладов.~--- 
Калининград, 2012. Т.~2. С.~806--807.
\bibitem{32-ko} %33
\Au{Lakoff J.} Women, fire, and dangerous things: What categories reveal about the mind.~--- 
University of Chicago Press, 1987.

\label{end\stat}

\bibitem{33-ko} %34
\Au{Зайцев Д.} Язык как зеркало мышления: Рецензия на книгу Джорджа Лакоффа 
<<Женщины, огонь и опасные вещи: что категории языка говорят нам о мышлении>>~/ 
Пер. с англ. И.\,Б.~Шатуновского.~--- М.: Языки славянской культуры, 2004. 792~с.~// 
Отечественные записки, 2004.
\end{thebibliography} } }

\end{multicols}  %9
\def\stat{korenkov}

\def\tit{РАЗРАБОТКА ИМИТАЦИОННОЙ МОДЕЛИ СБОРА И~ОБРАБОТКИ ДАННЫХ ЭКСПЕРИМЕНТОВ НА 
УСКОРИТЕЛЬНОМ КОМПЛЕКСЕ НИКА$^*$}

\def\titkol{Разработка имитационной модели сбора и~обработки данных экспериментов на 
ускорительном комплексе НИКА}

\def\autkol{В.\,В.~Кореньков, А.\,В.~Нечаевский, В.\,В.~Трофимов}

\def\aut{В.\,В.~Кореньков$^1$, А.\,В.~Нечаевский$^2$, В.\,В.~Трофимов$^3$}

\titel{\tit}{\aut}{\autkol}{\titkol}

{\renewcommand{\thefootnote}{\fnsymbol{footnote}}\footnotetext[1] {Работа 
частично выполнена в рамках ФЦП <<Исследования и разработки по приоритетным 
направлениям развития научно-тех\-но\-ло\-ги\-че\-ско\-го комплекса России на 
2007--2013 годы>> (гос.\ контракт №\,07.524.12.4008).}}

\renewcommand{\thefootnote}{\arabic{footnote}}
\footnotetext[1]{Объединенный институт ядерных исследований, korenkov@cv.jinr.ru} 
\footnotetext[2]{Объединенный институт ядерных исследований, Andrey.Nechaevskiy@gmail.com}
\footnotetext[3]{Объединенный институт ядерных исследований, trofimov@jinr.ru}
  
\vspace*{24pt}

\Abst{В работе обоснована необходимость создания имитационной модели системы 
хранения и обработки данных ускорительного комплекса НИКА. В~качестве платформы для 
создания модели выбрана система GridSim. В~работе описан подход к моделированию 
системы хранения данных dCache и каналов передачи. На простом примере показаны 
возможности использования модели. }

\vspace*{6pt}

\KW{грид-технологии; грид-инфраструктуры; сис\-те\-ма хранения данных; оптимизация; 
моделирование; исследование; разработки; dCache; Tier1; НИКА; грид}

\vspace*{24pt}

\vskip 14pt plus 9pt minus 6pt

      \thispagestyle{headings}

      \begin{multicols}{2}

            \label{st\stat}


\section{Введение}

   В настоящее время в Объединенном институте ядерных исследований 
создается ускорительный комплекс НИКА. 

Комплекс НИКА представляет 
собой ускоритель тяжелых ионов НИКА и установку MPD (multipurpose 
detector), объединяющую детекторы для изучения ядерной материи в горячем и 
плотном состоянии, которое возникает при столкновении ускоренных тяжелых 
ионов. Установка MPD является источником данных с интенсивностью потока десятки 
петабайт в год. 
   
   Ожидаемая интенсивность потока данных настолько велика, что массивы 
данных характеризуются как сверхбольшие. Для обработки таких потоков 
данных используются распределенные системы коллективного пользования, 
построенные на грид-тех\-но\-ло\-гиях.
   
   Для оптимизации структуры будущего комплекса обработки данных 
необходимо определить его основные параметры, структуру и проверить 
предлагаемые технические решения с помощью моделирования. Для этих целей 
на базе пакета моделирования GridSim создана имитационная модель грид-сайта.

\vspace*{-6pt}

\section{Система обработки данных ускорительного комплекса 
НИКА}

\vspace*{-2pt}

   Хранение и использование экспериментальных\linebreak данных в современных 
исследованиях в об\-ласти физики высоких энергий является актуальной 
проб\-ле\-мой. Объем получаемых и обрабатываемых данных исключает 
возможность их хранения и использования не только на одном кластере, но и в 
пределах одной организации, поэтому на первый план выходит создание 
распределенной системы хранения и обработки данных.
   
   Для эксперимента MPD на НИКА предполагается, что поток данных будет 
иметь следующие параметры:
   \begin{itemize}
\item высокая скорость набора событий (до 6~кГц);
\item в центральном столкновении Au-Au при энергиях НИКА 
образуется до 1000~заряженных час\-тиц;
\item размер файла с первоначальной моделируемой информацией с 
детекторов для одного события занимает около 0,45~MБ.
\end{itemize}

\end{multicols}

\begin{figure} %fig1
\vspace*{9pt}
 \begin{center}
 \mbox{%
 \epsfxsize=157.686mm
 \epsfbox{kor-1.eps}
 }
 \end{center}
 \vspace*{-6pt}
\Caption{Схема обработки физических данных ускорительного комплекса НИКА}
\vspace*{6pt}
\end{figure}

\begin{multicols}{2}

   Схема получения и обработки данных пред\-став\-ле\-на на рис.~1. 
   
   Данные, 
идущие от персональных компьютеров поддетекторов MPD, накапливаются 
специально предназначенными для сборки событий программами (Event 
Builder) компьютерной фермы в режиме онлайн. После формирования события 
в режиме офлайн через специально предназначенную для этой цели 
во\-ло\-кон\-но-оп\-ти\-че\-скую линию связи с пропускной способностью 10~ГБ/с данные 
записываются на диск.
   
   После триггера высокого уровня отобранные события записываются в 
   RAW-фай\-лы (скорость записи один файл в 1~минуту сбора данных) и затем 
полностью восстанавливаются. 
   
   Прогнозируемое число обрабатываемых событий при этом составляет 
приблизительно $19\cdot 10^9$. Принимая скорость передачи данных от 
датчиков равной 4,7~ГБ/с, общий объем исходных данных можно оценить в 
30~ПБ ежегодно, или 8,4~ПБ после сжатия. Эти оценки основаны на 
особенностях DAQ (data acquisition) и подобных оценках, выполненных для эксперимента 
ALICE~[1].
   


   В качестве системы обработки физической информации в эксперименте 
НИКА предполагается использование грид. Грид (название по аналогии с 
электрическими сетями~--- electric power grid)~--- это компьютерная 
инфраструктура нового типа, обеспечивающая глобальную интеграцию 
информационных и вычислительных ресурсов. Суть инициативы грид состоит 
в создании набора стандартизированных служб для обеспечения надежного, 
совместимого, дешевого и безопасного доступа\linebreak к географически 
распределенным высокотехно\-логичным информационным и вычислительным 
ресурсам~--- отдельным компьютерам, кластерам и\linebreak суперкомпьютерным 
центрам, хранилищам информации, сетям, научному инструментарию 
и~т.\,д.~[2]. 

   \begin{table*}\small
   \begin{center}
   \Caption{Уровни иерархической модели и их функции~\cite{4-kor}}
   \vspace*{2ex}
   
   \begin{tabular}{|l|l|}
   \hline
\multicolumn{1}{|c|}{Уровень} & \multicolumn{1}{c|}{Функции}\\
\hline
Tier0&Первичная реконструкция событий, калибровка, хранение копий полных баз данных\\
\hline
Tier1&\tabcolsep=0pt\begin{tabular}{l}Полная реконструкция событий, хранение 
актуальных баз данных по событиям, 
создание\\ и хранение наборов анализируемых событий, моделирование, анализ\end{tabular}\\
\hline
Tier2&Репликация и хранение наборов анализируемых событий, моделирование, анализ\\
\hline
\end{tabular}
\end{center}
\vspace*{-9pt}
\end{table*}
   \begin{table*}\small
   \begin{center}
   \Caption{Функции и свойства симуляторов грид~\cite{7-kor}}
   \vspace*{2ex}
   
   \begin{tabular}{|l|c|c|c|c|c|c|}
   \hline
\multicolumn{1}{|c|}{Функция}&GridSim&OptorSim&Monarc&ChicSim&SimGrid&MicroGrid\\
\hline
Репликация данных&Да&Да&Да&Да&Нет&Нет\\
\hline
Издержки записи/чтения диска&Да&Нет&Да&Нет&Нет&Да\\
\hline
\tabcolsep=0pt\begin{tabular}{l}Комплексное фильтрование\\ или запросы данных\end{tabular}&Да&Нет&Нет&Нет&Нет&Нет\\
\hline
\tabcolsep=0pt\begin{tabular}{l}Планировка пользовательских\\ задач\end{tabular}&Да&Нет&Да&Да&Да&Да\\
\hline
\tabcolsep=0pt\begin{tabular}{l}Резервирование центрального\\ процессорного устройства\end{tabular}&Да&Нет&Нет&Нет&Нет&Нет\\
\hline
Симуляция нагрузки&Да&Нет&Нет&Да&Нет&Нет\\
\hline
\tabcolsep=0pt\begin{tabular}{l}Дифференцированное качество\\ обслуживания сети\end{tabular}&Да&Нет&Нет&Нет&Нет&Нет\\
\hline
\tabcolsep=0pt\begin{tabular}{l}Генерация фонового сетевого трафика\end{tabular}&Да&Да&Нет&Нет&Да&Да\\
\hline
\end{tabular}
\end{center}
\end{table*}
   
   Эксперименты, в которых для обработки данных используется 
   грид-ин\-фра\-струк\-ту\-ра или облачные вычисления, имеют некоторые 
общие черты: большие потоки данных, длительный цикл проектирования и 
строительства, длительный период эксплуатации. Так, компьютерная 
инфраструктура для эксперимента ALICE представляет собой иерархическую 
грид-структуру с компьютерными центрами класса Tier 0/1/2. Функциональные 
различия уровней иерархической модели представлены в табл.~1. Для хранения 
и обработки данных в эксперименте PANDA~\cite{3-kor} также предполагается 
использование грид. 
   



   Проектирование грид-структур больших мас\-шта\-бов подразумевает не только 
привлечение специалистов, обладающих уникальными навыками,\linebreak но и 
применение инструментов для {моделирования}. При создании распределенной 
системы требуется принять решения по архитектуре инфраструктуры, 
количеству ресурсных центров, объему \mbox{требуемых} ресурсов. Кроме того, 
необходимо обеспечить достаточную пропускную способность, решить 
проблемы сохранности данных, обеспечить распределение ресурсов между 
различными группами пользователей, выбрать алгоритмы обработки и запуска 
задач и многое другое. Для решения этих вопросов, а также обоснования 
решений требуется создание имитационной модели обработки данных 
эксперимента. Возникает необходимость создания имитационной модели, 
которая бы удовлетворяла всем условиям. 
   
   Актуальность темы обусловливается тем, что на основе модели в дальнейшем 
могут быть обоснованы рекомендации и техническое задание на разработку 
компьютерной инфраструктуры, рас\-смот\-ре\-ны различные варианты 
организации хранения данных эксперимента. 

\section{Выбор пакета моделирования}

   На сегодняшний день существуют различные инструменты моделирования 
грид-сис\-тем~\cite{5-kor}. Проект GridSim разрабатывается группой 
исследователей в лаборатории по изучению облачных и распределенных 
вычислений отдела информатики и компьютерных вычислений в Университете 
Мельбурна, Австралия. Пакет моделирования GridSim неоднократно 
применялся~\cite{6-kor} для моделирования грид-струк\-тур и планировщиков.
   
   GridSim~--- это библиотека классов, предназначенных для построения 
модели грид-системы. Она, в свою очередь, построена на стандартной 
библиотеке SimJava, с помощью которой можно моделировать поток 
дискретных событий во времени. Приложение создается расширением классов 
GridSim и объединением их в программу, которая моделирует обработку потока 
заданий грид-струк\-ту\-рой, обладающей определенными ресурсами и с 
заданной дисциплиной их резервирования и использования. В~сравнении с 
другими пакетами моделирования грид GridSim обладает рядом преимуществ. 
Основные преимущества представлены в табл.~2. 

С~по\-мощью GridSim можно 
проводить воспроизводимые эксперименты, которые сложно реализовать в 
настоящем окружении динамических грид-сис\-тем.
   


   После анализа целого ряда систем для разработки имитационной модели 
была выбрана платформа GridSim. 
  
\section{Моделирование сайта уровня Т1 грид-структуры}

   В качестве примера грид-струк\-ту\-ры уровня T1 будет рассмотрен 
   Оф\-лайн-уро\-вень обработки физических данных ускорительного 
комплекса \mbox{НИКА}. Для эффективной работы грид-сай\-та, проведения 
исследований по оптимизации нагрузки, разработки и тестирования новых 
алгоритмов с точки зрения скорости достижения результата необходимо 
использовать средства моделирования грид-сис\-тем. При создании модели 
предполагается, что основой для построения системы хранения данных будет 
dCache~\cite{8-kor}. Модель сайта Т1 строится на следующем алгоритме 
обработки данных (см.\ рис.~1): 
   \begin{enumerate}[(1)]
\item данные появляются с заданной частотой и записываются на 
локальные диски компьютеров. После перемещения данных на второй 
уровень диск очищается;
\item данные перемещаются автоматически на второй уровень по 
каналам. В качестве носителей второго уровня используются пулы 
системы dCache, рассматриваемые в модели как единая память. При 
обработке данных предполагается, что вначале данные попадают в 
дисковый пул системы хранения, а затем по локальному протоколу 
передается на узлы обработки. Непосредственное монтирование 
директории на рабочих узлах не используется;
\item для долгосрочного хранения данных используется ленточный 
робот. Копии файлов автоматически создаются на лентах, после чего 
файлы удаляются с дисковых пулов. 
\end{enumerate}

   Отличительная особенность конфигурации dCache~--- наличие не менее двух 
уровней хранения: жесткие диски и ленточный накопитель. Под ленточным 
накопителем подразумеваются автоматизированные биб\-лио\-те\-ки, 
оснащенные роботизированным загрузочным механизмом и стойкой на 
несколько картриджей (лент). Объем такой биб\-лио\-те\-ки~$Q$ можно определить 
простейшими вы\-чис\-ле\-ни\-ями, исходными данными для которых будут 
производительность установки~$p$, время ее работы~$T$ и емкость 
накопителей~$c$:
   $$
   Q=\fr{pT}{c}\,.
   $$
   
   Другие вопросы создания грид-сай\-та требуют более тщательного анализа и 
выбора приемлемого варианта. Таким образом, перед разработчиками системы 
встают следующие вопросы:
   \begin{itemize}
   \item определение необходимого количества драйвов;
\item способы группировки файлов на лентах;
\item политика записи файлов.
\end{itemize}

   Стоит отметить ряд ключевых особенностей GridSim, которые потребовали 
доработки из-за несоответствия требованиям модели: 
   \begin{itemize}
   \item создавать файлы может только пользователь;
\item все объекты моделирования объединены в сеть при помощи 
каналов передачи данных;
\item пользователь может копировать (создавать) только один файл 
единовременно. 
\end{itemize}
   
   Для решения этих вопросов потребовалось расширение существующих 
классов и добавление новых объектов. Так, в сис\-те\-му добавлены следующие 
объекты (рис.~2):
   \begin{itemize}
\item Drive~--- драйв магнитофона;
\item Arm~--- рука робота;
\item Reel archive~--- архив картриджей;
\item Reel~--- картридж.
\end{itemize}




   Набор этих классов позволяет моделировать все процессы, происходящие с 
копией файла на лентах: загрузку и выгрузку ленты манипулятором, 
монтирование на драйве, поиск файла на ленте и его чте\-ние/за\-пись.
   
   Задача моделирования сетевой инфраструктуры в библиотеке GridSim 
решена с помощью классов Router, Link, NetPacket и некоторых других. Этот 
набор средств позволяет моделировать прохождение пакетов по сети. 
Пользователю предоставляется возможность встраивать свои планировщики 
пакетов в исходную модель. Такой подход обеспечивает высокую точность 
моделирования. Его недостатком применительно к задаче моделирования Т1 
является избыточность~--- вопросы маршрутизации, столкновений пакетов, 
влияния фоновой загрузки каналов в данной модели не рассматриваются и, 
следовательно, уровень детализации до пакета представляется избыточным. 
В~рас\-смат\-ри\-ва\-емом случае интерес представляет только изменение 
нагрузки на отдельные компоненты сети. 


\pagebreak

\end{multicols}

\begin{figure} %fig2
\vspace*{1pt}
 \begin{center}
 \mbox{%
 \epsfxsize=160.035mm
 \epsfbox{kor-2.eps}
 }
 \end{center}
 \vspace*{-6pt}
\Caption{Описание новых классов в модели }
%\vspace*{6pt}
\end{figure}

\begin{multicols}{2}


   
   Исходя из вышеизложенного требуется дополнить GridSim следующим 
механизмом. 

Вводится понятие \textit{операции передачи данных}. Под этим 
подразумевается за\-пись/чте\-ние час\-ти или целого файла экспериментальных 
данных. В~этом случае ввод и вывод служебной и диагностической 
информации считается пренебрежимо малым. Операция 
   рас\-смат\-ри\-ва\-ет\-ся как атомарная, т.\,е.\ начинается методом <<начать 
операцию>>. Параметрами метода являются: устройство~1~--- источник 
данных,
 устройство~2~--- получатель и список всех промежуточных устройств, 
которые необходимо пройти от источника до получателя. Элемент сети в 
системе описывается классом Stange. Взаимодействие классов, описывающих 
сеть, отражено на рис.~3.

\begin{figure*} %fig3
\vspace*{9pt}
 \begin{center}
 \mbox{%
 \epsfxsize=97.571mm
 \epsfbox{kor-3.eps}
 }
 \end{center}
 \vspace*{-6pt}
\Caption{Взаимодействие классов, описывающих сеть}
\end{figure*}

   Результаты моделирования доступны пользователю в виде таблиц и 
графиков. Для этой цели используются классы генератора лога и визуального 
отображения результатов:
%   \begin{itemize}
%\item 
Info~--- описание вы\-чис\-ли\-тель\-ной структуры и потока заданий;
%\item 
Reporter~--- генератор лога;
%\item 
парсер лога;
%\item 
объект визуального отображения результатов и~др.
%\end{itemize}

   Ниже приведем пример задачи, которая возникает при проектировании 
системы сбора и хранения данных. 

\section{Пример использования модели}

   С помощью системы моделирования можно исследовать прохождение набора 
заданий и передачу файлов через грид-струк\-ту\-ру с заданной пользователем 
топологией и параметрами центров обработки. Модель позволяет получить 
оценку временн$\acute{\mbox{ы}}$х параметров обработки потока заданий при заданной 
пользователем дисциплине распределения ресурсов между заданиями и 
структурой очередей к центрам обработки.
   
   Моделирование дает ответы на вопросы:
   \begin{itemize}
   \item какие вычислительные ресурсы требуются для обработки данных;
   \item как должны быть связаны между собой центры обработки;
   \item каким должен быть уровень сжатия данных;
   \item какой должна быть конфигурация роботизированной библиотеки;
   \item хватит ли ресурсов на обработку потока данных и предоставление 
данных пользователям. 
   \end{itemize}
   
   Проиллюстрировать применение упомянутых выше классов можно на 
примере моделирования процесса обработки данных с одновременной \mbox{записью} 
на ленты. Задача проектировщика~--- определить необходимое количество 
драйвов библиотеки. При этом исследуются два вопроса: какое количество 
драйвов библиотеки необходимо для того, чтобы записать весь поток 
<<сырых>> (RAW) данных с детекторов эксперимента, и насколько при этом 
процесс обработки данных (поток заданий от пользователей) будет мешать 
записи, если обработка потребует загрузки файлов с лент на диски.
   
   Допустим, что имеется в распоряжении биб\-ли\-о\-тека, количество драйвов в 
библиотеке фик\-си\-ровано и равно пяти. Это существенно меньше 
необходимого, но достаточно для иллюстрации возмож\-но\-стей модели. Когда 
для обслуживания поступающих на сайт заданий и записи RAW-дан\-ных 
используются одни и те же пулы (драйвы), процесс начинает вести себя 
хаотично, многократно монтируя и размонтируя ленты для записи даже при 
незначительных загрузках. Для того чтобы избежать этой ситуации, в 
рассматриваемой модели пулы лент разделены на принимающие данные 
(RAW) и обслуживающие поток заданий (DLT~--- digital linear tape). Возникает вопрос: каким 
образом распределить драйвы между двумя пулами при фиксированных 
параметрах потока заданий? Предполагаем, что файлы запрашиваются 
случайным образом. 
   
   Моделируемая система~--- двухуровневая. На первом уровне находится 
дисковый массив, на втором уровне~--- ленточный накопитель. 
В~существующей модели скорость записи и чтения с дискового массива не 
зависит от загрузки. Параметры драйвов и робота соответствуют параметрам 
планируемых к установке устройств (табл.~3). Количество драйвов в роботе 
фиксировано, и есть только одна <<рука>>, загружающая файлы в драйв.
   
   \begin{table*}\small %tabl3
   \begin{center}
   \Caption{Параметры для моделирования ленточной библиотеки}
   \vspace*{2ex}
   
   \tabcolsep=10pt
   \begin{tabular}{|l|c|}
   \hline
\multicolumn{1}{|c|}{Параметр}&Значение\\
\hline
Время монтирования/размонтирования, с&\hphantom{99}22\\
Скорость поиска, с&\hphantom{9}300\\
Скорость чтения/записи, c&\hphantom{9}120\\
Скорость перемотки, с&1000\\
Время загрузки/разгрузки картриджа в драйв, с&\hphantom{9}100\\
Размер файла, МБ&6000\\
\hline
\end{tabular}
\end{center}
\vspace*{-9pt}
\end{table*}

   Результаты моделирования приведены в табл.~4. С~помощью модели 
исследовались следующие характеристики: 
   \begin{itemize}
\item время выполнения~--- астрономическое время выполнения потока 
заданий, которое из общих соображений будет уменьшаться с 
увеличением количества драйвов;
\item длина очереди~--- максимальная длина очереди на запись 
RAW-дан\-ных на ленту. 
\end{itemize}

\begin{table*}\small %tabl4
\begin{center}
\Caption{Результаты моделирования}
\vspace*{2ex}

\begin{tabular}{|c|c|c|c|c|}
\hline
Эксперимент&Драйвов RAW&Драйвов DLT&Время выполнения, с&Длина очереди\\
\hline
1&1&4&28\,959&13\hphantom{9}\\
2&2&3&28\,703&1\\
3&3&2&28\,814&1\\
4&4&1&59\,275&1\\
\hline
\end{tabular}
\end{center}
\vspace*{-6pt}
\end{table*}
   
   Моделирование показало, что при заданном темпе сбора данных для записи 
должно быть выделено не менее двух драйвов. С~другой стороны, для 
обработки потока заданий должно быть выделено не менее двух драйвов для 
чтения накопленной информации. Если за критерий оп\-ти\-маль\-ности принять 
минимальное астрономическое время выполнения потока заданий, то 
оптимальным можно считать распределение драйвов по варианту №\,2.
   
   Этот пример иллюстрирует один из вариантов использования программы. 
Такие исследования могут быть проведены с использованием аналитических 
моделей теории массового обслуживания, однако добавление простейших 
условий группировки заданий и файлов значительно усложняет аналитические 
модели, тогда как для имитационной модели изменения сводятся к нескольким 
строчками программного кода. 

\section{Заключение}

   Созданная система моделирования позволяет проводить разнообразные 
эксперименты с исследуемым объектом, не прибегая к физической реализации. 
Это позволяет предсказать и предотвратить большое число неожиданных 
ситуаций в процессе эксплуатации, которые могли бы при\-вес\-ти к 
неоправданным затратам, потере данных, а возможно, и к по\-вреж\-де\-нию 
дорогостоящего обору\-до\-ва\-ния. В~процессе моделирования можно подобрать 
минимально необходимое оборудование, обеспечивающее потребности 
передачи, обработки и хранения данных, оценить необходимый запас 
производительности оборудования, обеспе\-чи\-ва\-юще\-го возможное увеличение 
производственных потребностей, выбрать несколько вариантов оборудования с 
учетом текущих потребностей и перспективы развития в будущем, провести 
проверку работы сис\-те\-мы, выявить ее <<узкие>> места и~т.\,д.
   
   Применение системы моделирования позволит определить параметры 
системы обработки данных ускорительного комплекса НИКА на этапе 
технического проектирования. 
   
   Дальнейшее развитие системы предполагает внесение дополнений с целью 
создания модели грид-сай\-та уровня Т1 с использованием двух и трех уровней 
dCache. Для моделирования предполагается использовать оригинальный 
алгоритм назначения пулов dCache и оригинальные данные по потокам. Также 
необходимо провести полномасштабные испытания модели с целью выявления 
ошибок и создания базы сценариев моделирования. 
   
   Важное значение разработанной системы моделирования связано с 
созданием в Объединенном институте ядерных исследований 
автоматизированной системы обработки и хранения данных (АСОД) уровня T1 
для эксперимента CMS (Compact Muon Solenoid) на Большом адронном коллайдере и предназначенной 
для работы в составе глобальной грид-сис\-те\-мы для обработки данных 
(WLCG~--- Worldwide LHC Computing Grid). Автоматитзированная система обработки и хранения данных нацелена на проведение полного цикла обработки физической 
информации, получаемой в ходе проведения эксперимента, обеспечения работ 
по моделированию физических процессов, защищенного хранения и 
при\-ема/пе\-ре\-да\-чи данных в другие центры WLCG. Основной системой хранения 
данных в АСОД является dCache. Очевидно, что в процессе длительного 
(10~лет и более) функционирования центра будет необходимо оперативно 
масштабировать систему хранения и повышать эффективность использования 
ленточного робота в системе dCache без остановки работы всего комплекса. 
В~этом процессе предварительное моделирование работы системы хранения 
станет необходимым инструментом.
   
   Результаты работы могут быть рекомендованы для использования при 
проектировании грид-сис\-те\-мы для сбора, передачи, обработки и хранения 
данных с мегаустановок или других аналогичных установок, генерирующих 
большие объемы данных.

{\small\frenchspacing
{%\baselineskip=10.8pt
\addcontentsline{toc}{section}{Литература}
\begin{thebibliography}{9}
  
\bibitem{1-kor}
\Au{Cortese P., Carminati F., Fabjan C.\,W., \textit{et al.}} ALICE Technical Design Report of the Computing~// 
CERN/LHCC 2005-018, ALICE TDR~12, 2005.
\bibitem{2-kor}
\Au{Кореньков~В.\,В.} Грид-тех\-но\-ло\-гии: статус и перспективы~// Вест\-ник 
Международной академии наук. Русская секция, 2010. №\,1. C.~41--44.

\bibitem{4-kor} %3
\Au{Ильин~В.\,А., Кореньков~В.\,В., Солдатов~А.\,А.} Российский сегмент 
глобальной инфраструктуры LCG~// Открытые системы, 2003. №\,1. C.~56--60.

\bibitem{3-kor} %4
Веб-пор\-тал проекта PANDA. {\sf http://www-panda.gsi.de}.

\bibitem{5-kor}
\Au{Нечаевский~А.\,В., Кореньков~В.\,В.} Пакеты моделирования DataGrid~// 
Сис\-тем\-ный анализ в науке и образовании: Электронный журнал, 2009. №\,1.
\bibitem{6-kor}
Веб-пор\-тал проекта GridSim. {\sf http://www.gridbus.org/ gridsim}.

\bibitem{7-kor}
\Au{Sulistio~A., Cibej~U., Venugopal~S., Robic~B., Buyya~R.} A~toolkit for 
modelling and simulating data grids: An extension to GridSim~// Concurrency  
Computation Practice Experience (CCPE), 2008. Vol.~20. No.\,13. 
P.~1591--1609.


\label{end\stat}

\bibitem{8-kor}
Веб-пор\-тал проекта dCache. {\sf http://www.dcache.org}.


\end{thebibliography}
} }

\end{multicols} %10 
\def\stat{kor-kor}



\def\tit{МОДИФИЦИРОВАННЫЙ СЕТОЧНЫЙ МЕТОД РАЗДЕЛЕНИЯ ДИСПЕРСИОННО-СДВИГОВЫХ
СМЕСЕЙ НОРМАЛЬНЫХ ЗАКОНОВ$^*$}



\def\titkol{Модифицированный сеточный метод разделения дисперсионно-сдвиговых
смесей нормальных законов}

\def\aut{В.\,Ю.~Королев$^1$,  А.\,Ю.~Корчагин$^2$}

\def\autkol{В.\,Ю.~Королев,  А.\,Ю.~Корчагин}

\titel{\tit}{\aut}{\autkol}{\titkol}

{\renewcommand{\thefootnote}{\fnsymbol{footnote}} \footnotetext[1]
{Работа поддержана Российским научным фондом (проект 14-11-00364).}}


\renewcommand{\thefootnote}{\arabic{footnote}}
\footnotetext[1]{Факультет
вычислительной математики и кибернетики Московского государственного
университета им.\ М.\,В.~Ломоносова; Институт проблем информатики
Российской академии наук; victoryukorolev@yandex.ru}
\footnotetext[2]{Факультет вычислительной математики и кибернетики
Московского государственного университета им.\ М.\,В.~Ломоносова;
sasha.korchagin@gmail.com}

%\vspace*{2pt}



\Abst{Описывается модифицированный двухэтапный
сеточный метод разделения дис\-пер\-си\-он\-но-сдви\-го\-вых смесей нормальных
законов, представляющий собой альтернативу чистому ЕМ (expectation-maximization)
ал\-го\-рит\-му. На
первом этапе этого алгоритма строится дискретная аппроксимация для
смешивающего распределения, на втором этапе подбирается абсолютно
непрерывное распределение из заранее заданного семейства, например,
обобщенных обратных гауссовских законов, ближайшее к~дискретному
распределению, полученному на первом этапе. Обсуждаются вопросы
сходимости этого двухэтапного алгоритма. Доказана монотонность
сеточного итерационного метода, используемого на первом этапе.
Подробно обсуждается вопрос оптимального выбора параметров метода,
прежде всего сетки, накидываемой на носитель смешивающего
распределения. С~этой целью предложены статистические оценки
квантилей смешивающего распределения. Эффективность метода
иллюстрируется примерами конкретных вычислений оценок параметров
обобщенных гиперболических распределений.}

\KW{смесь распределений вероятностей;
дис\-пер\-си\-он\-но-сдви\-го\-вая смесь нормальных законов; обобщенное
гиперболическое распределение; ЕМ-ал\-го\-ритм; сеточный метод
разделения смесей}

\vspace*{1pt}

%\vspace*{2pt}

\DOI{10.14357/19922264140402}


\vskip 12pt plus 9pt minus 6pt

\thispagestyle{headings}

\begin{multicols}{2}

\label{st\stat}

\section{Введение}

При {\it практическом} решении задачи моделирования и исследования
волатильности (изменчивости) хаотических стохастических процессов
ключевым этапом является статистическое разделение смесей
вероятностных распределений. Задача разделения смесей~---
статистического оценивания параметров смесей вероятностных
распределений~--- в~деталях разобрана, например, в~книге~\cite{k2011}.

Для решения задачи разделения смесей вероятностных распределений
традиционно используются итерационные процедуры типа ЕМ-ал\-го\-рит\-ма.
К~сожалению, классический ЕМ-ал\-го\-ритм обладает рядом серьезных
недостатков при его применении к~смесям нормальных законов, а~именно:
он демонстрирует крайнюю неустойчивость по отношению к~исходным
данным и~начальным приближениям.

Для преодоления этих недостатков
предложено много модификаций ЕМ-ал\-го\-рит\-ма (см., например,~\cite{k2011}).
Вместе с тем в~указанной книге предложен и~исследован
принципиально новый~--- сеточный~--- метод приближенного решения
задачи разделения смесей. В~работе~\cite{n2013} подробно исследованы
вопросы сходимости сеточных методов разделения смесей.

В соответствии с подходом к~статистическому анализу хаотических
стохастических процессов, в~частности к~решению задачи декомпозиции
волатильности таких процессов, развитом в~книге~\cite{k2011},
в~общем случае на практике приходится решать задачу разделения
конечных смесей нормальных законов с~произвольно большим числом
неизвестных параметров (параметров компонент и~их весов).
И~хотя в~большинстве приложений возникают смеси не более чем с~пятью--семью
компонентами, даже при использовании таких смесей, скажем, в~задачах
анализа и~прогнозирования финансовых рисков приходится моделировать
траекторию движения точки в~пространствах, размерность которых
соответственно лежит в~пределах от~14 (для пятикомпонентных смесей)
до~20 (для семикомпонентных смесей), что существенно увеличивает
вычислительные и~временн$\acute{\mbox{ы}}$е ресурсы, необходимые для практического
решения указанных задач.

Поскольку во многих ситуациях (например,
при прогнозировании на основе высокочастотных данных) эти задачи
необходимо решать в~режиме, близком к~реальному времени, для
создания эффективных методов статистического анализа на основе
смешанных моделей на первый план выходит проб\-ле\-ма снижения
размерности решаемой задачи, т.\,е.\ параметрического пространства.

Одним из возможных подходов к~снижению размерности является
априорное сужение классов допусти\-мых смесей. К~примеру, при решении
многих задач, связанных с~анализом процессов атмосферной или
плазменной турбулентности, а~так\-же процессов, описывающих эволюцию
различных финансовых индексов, высочайшую адекватность
продемонстрировали модели, основанные на дис\-пер\-си\-он\-но-сдви\-го\-вых
смесях нормальных законов. Класс таких смесей очень обширен
и,~в~част\-ности, включает в~себя обобщенные гиперболические распределения,
которые были введены О.-Е.~Барн\-дорфф-Ниль\-се\-ном в~1977--1978~гг.\ как
класс специальных сдвиг-мас\-штаб\-ных смесей нормальных законов~\cite{BN1977, BN1978}.
Пусть $\alpha\hm\in\r$, $\beta\hm\in\r$. Если
функцию распределения обобщенного гиперболического закона
с~параметрами~$\alpha$, $\beta$, $\nu$, $\mu$, $\lambda$ обозначить
$P_{GH}(x;\alpha,\beta,\nu,\mu,\lambda)$, то по определению
\begin{multline}
P_{GH}(x;\alpha,\beta,\nu,\mu,\lambda)={}\\
{}=
\int\limits_{0}^{\infty}\Phi\left(\fr{x-\beta-\alpha
z}{\sqrt{z}}\right)\,p_{GIG}(z;\nu,\mu,\lambda)\,dz\,,\\
x\in\r\,,
\label{e1-kor}
\end{multline}
где $\Phi(x)$~--- стандартная нормальная функция распределения:
$$
\Phi(x)=\int\limits_{-\infty}^{x}\varphi(z)\,dz\,,\enskip
\varphi(x)=\fr{1}{\sqrt{2\pi}}e^{-x^2/2}\,,\enskip  x\in\mathbb{R}\,;
$$
$p_{GIG}(x;\nu,\mu,\lambda)$~--- плот\-ность обобщенного обратного
гауссовского распределения:
\begin{multline*}
p_{GIG}(x;\nu,\mu,\lambda)={}\\
{}=\fr{\lambda^{\nu/2}}{2\mu^{\nu/2}
K_{\nu}\left(\sqrt{\mu\lambda}\right)}\,
x^{\nu-1}\exp\left\{-\fr{1}{2}\left(\fr{\mu}{x}+\lambda
x\right)\right\}\,,\\ x>0\,.
\end{multline*}
Здесь $\nu\in\r$;
$$
\begin{array}{lll}
\mu>0\,, & \lambda\geqslant0\,, & \mbox{если }\nu<0\,;\\[6pt]
\mu>0\,, & \lambda>0\,, & \mbox{если }\nu=0\,;\\[6pt]
\mu\geqslant0\,, & \lambda>0\,, & \mbox{если }\nu>0\,;
\end{array}
$$
$K_{\nu}(z)$~--- модифицированная бесселева функция третьего рода
порядка~$\nu$:

\noindent
\begin{multline*}
K_{\nu}(z)=\fr{1}{2}\int\limits_{0}^{\infty}y^{\nu-1}\exp
\left\{-\fr{z}{2}\left(y+\fr{1}{y}\right)\right\}\,dy\,,\\
z\in\mathbb{C}\,,\enskip \mathrm{Re}\,z>0\,.
\end{multline*}
Обратим внимание, что в~(1) смешивание происходит одновременно и~по
параметру сдвига, и~по параметру масштаба, но так как эти параметры
в~(1)  связаны жесткой зависимостью, так что параметр сдвига
смешиваемого распределения пропорционален его дисперсии, то
фактически смесь~(1) является {\it однопараметрической} и~поэтому
называется {\it дис\-пер\-си\-он\-но-сдви\-го\-вой} (см., например,~\cite{BN1982}).

Другим примером дис\-пер\-си\-он\-но-сдви\-го\-вых смесей нормальных законов
являются обобщенные дисперсионные гам\-ма-рас\-пре\-де\-ле\-ния, в~которых
смешивающими являются обобщенные гам\-ма-рас\-пре\-де\-ле\-ния~\cite{ks2012, zk2013}.

В указанных семействах смесей число неизвестных параметров равно
пяти или шести (если\linebreak учитывать неслучайный сдвиг). Вместе
с~тем у~подоб\-ных моделей имеются довольно серьезные тео\-ре\-ти\-че\-ские
обоснования: в~работах~\cite{zk2013, k2013} показано, что указанные
модели являются асимптотическими аппроксимациями в~простой
предельной схеме случайного суммирования и~потому могут успешно
применяться для анализа процессов типа остановленных случайных
блужданий. Эти выводы подтверждены статистическим анализом
вы\-со\-ко\-час\-тот\-ных финансовых данных, в~результате которого выявлен
синхронизированный характер изменения интенсивностей потоков заявок
в~сис\-те\-мах электронных торгов, что естественно приводит к~синхронизированному
поведению па\-ра\-мет\-ров сдвига и~диффузии в~соответствующих моделях вида смесей
нормальных законов~\cite{kckg2013}.

\section{Описание моди\-фи\-ци\-ро\-ван\-но\-го
сеточного ме\-то\-да разделения дисперсионно-сдвиговых смесей
нормальных законов и~его свойства}

Оказывается, что сеточные методы разделения смесей довольно
эффективны не только при разделении конечных смесей нормальных
законов, но и~при разделении произвольных дис\-пер\-си\-он\-но-сдви\-го\-вых
смесей нормальных законов. Поясним сказанное на примере задачи
оценивания па\-ра\-мет\-ров обобщенных гиперболических распределений.

Для решения задачи оценивания параметров обобщенных гиперболических
распределений традиционно используется метод, предложенный в~статье~\cite{p2004}
и~по сути являющийся классическим ЕМ-ал\-го\-рит\-мом,
приспособленным к~конкретной задаче, и,~соответственно, наследующий
присущие ЕМ-ал\-го\-рит\-мам недостатки.

Рассмотрим следующий альтернативный двухэтапный метод. На первом
этапе на поло\-жи\-тельной полупрямой выделим основную часть носителя
смешивающего распределения, т.\,е.\ \mbox{ограниченный} интервал,
вероятность которого, вычисленная в~соответствии со смешивающим
распределением, практически равна единице. На этот интервал накинем
конечную сетку, содержащую, возможно, очень много {\it известных}
узлов $u_1,\ldots,u_K$. Считая параметр сдвига~$\beta$ равным нулю,
приблизим искомое обобщенное гиперболическое распределение конечной
смесью нормальных законов:

\noindent
\begin{multline}
P_{GH}(x;\,\alpha,0,\nu,\mu,\lambda)\approx{}\\
{}\approx \sum\limits_{i=1}^K
p_i\Phi\left(\fr{x-\alpha u_i}{\sqrt{u_i}}\right)\,,\enskip
x\in\mathbb{R}\,.\label{e2-kor}
\end{multline}
В смеси, стоящей в~правой части соотношения~(2), неизвестными
являются только параметры $p_1,\ldots,p_{K-1}$ и~$\alpha$. Пусть
$x_1,\ldots,x_n$~--- анализируемая выборка значений случайной
величины с~оцениваемым обобщенным гиперболическим распределением.
Итерационный процесс, определяющий сеточный ЕМ-ал\-го\-ритм для данной
задачи, задается следующим образом. Пусть
$p_1^{(m)},\ldots,p_{K-1}^{(m)}$ и~$\alpha^{(m)}$~--- оценки параметров
$p_1,\ldots,p_{K-1}$ и~$\alpha$ на $m$-й итерации,
$p_K^{(m)}\hm=1\hm-p_1^{(m)}-\cdots-p_{K-1}^{(m)}$. Обозначим

\noindent
\begin{align*}
\varphi_{ij}^{(m)}&=\fr{1}{\sqrt{u_i}}\varphi\left(\fr{x_j-\alpha^{(m)}u_i}{\sqrt{u_i}}\right)\,;
\\
g_{ij}^{(m)}&=\fr{p_i^{(m)}\varphi_{ij}^{(m)}}{\sum\limits_{r=1}^K
p_r^{(m)}\varphi_{rj}^{(m)}}\,,\\
&\hspace*{14mm}i=1,\ldots,K\,;\enskip j=1,\ldots,n\,.
\end{align*}
Тогда, используя стандартные рассуждения, определяющие
вычислительные формулы EM-ал\-го\-рит\-ма для параметров конечной смеси
нормальных законов (см, например,~[1, разд.~5.3.7--5.3.8]),
следует положить

\noindent
\begin{equation}
p_i^{(m+1)}=\fr{1}{n}\sum\limits_{j=1}^n g_{ij}^{(m)}\,, \enskip
i=1,\ldots,K\,.\label{e3-kor}
\end{equation}
Обозначим $\overline{x}=(1/n)\sum\limits_{j=1}^nx_j$. Используя
соотношение~(5.3.24) в~\cite{k2011}, с~учетом очевидного равенства
$\sum\limits_{i=1}^K g_{ij}^{(m)}\hm=1$ можно заметить, что уточненная
оценка параметра~$\alpha$ имеет вид:

\columnbreak

\noindent
\begin{equation}
\alpha^{(m+1)}=\fr{\overline{x}}{\sum\limits_{i=1}^K u_ip_i^{(m+1)}}\,,
\label{e4-kor}
\end{equation}
т.\,е.\ равна отношению генерального выборочного среднего и~текущего
эмпирического среднего смешивающего распределения, что вполне
согласуется с~тем, что в~соответствии с~приводимым ниже соотношением~(\ref{e5-kor})
в~данном случае ${\sf E}X\hm=\alpha{\sf E}U$.

В силу монотонности классического ЕМ-ал\-го\-рит\-ма справедливо следующее
утверждение.

\smallskip

\noindent
\textbf{Теорема~1.} {\it Пусть узлы $u_1,\ldots,u_K$ сетки различны,
неотрицательны и~известны. Тогда итерационный процесс $(3)$--$(4)$
является монотонным, т.\,е.\ каждая его итерация не уменьшает
целевую сеточную функцию правдоподобия}
\begin{multline*}
L(p_1,\ldots,p_K,\alpha;x_1,\ldots,x_n)={}\\
{}=
\prod\nolimits_{j=1}^n\left[\sum\nolimits_{i=1}^K
\fr{p_i}{\sqrt{u_i}}\,\varphi\left(\fr{x_j-\alpha^{(m)}u_i}{\sqrt{u_i}}\right)\right].
\end{multline*}

\smallskip

\noindent
\textbf{Замечание~1.} В~разд.~5.7.4 книги~\cite{k2011} показано, что
при каждом фиксированном значении параметра~$\alpha$ сеточная
функция правдоподобия\linebreak
$L(p_1,\ldots,p_{K-1},\alpha;\,x_1,\ldots,x_n)$ вогнута по
аргументам $p_1,\ldots,p_{K-1}$. Поэтому на каждом шаге
итерационного процесса вместо соотношения~(3) можно\linebreak использо\-вать
любой более быстрый алгоритм максимизации функции
$L(p_1,\ldots,p_{K-1},\alpha^{(m)};\,x_1,\ldots$\linebreak $\ldots,x_n)$ по переменным
$p_1,\ldots,p_{K-1}$. Например, оценки весов $p_1,\ldots,p_K$ можно
искать методом условного градиента~\cite{k2011, kn2010}.

\smallskip

Таким образом, на первом этапе получаются оценки параметра~$\alpha$
и~весов всех узлов~$u_i$ конечной сетки, накинутой на носитель
смешивающего обобщенного обратного гауссовского распределения
$P_{\mathrm{GIG}}(z;\,\nu,\mu,\lambda)$.

На втором этапе остается применить ка\-кой-ли\-бо стандартный метод
подгонки обобщенного обратного гауссовского распределения
$P_{\mathrm{GIG}}(z;\,\nu,\mu,\lambda)$ к~эмпирическим данным типа
гистограммы $(u_1, p_1),\ldots, (u_K, p_K)$. Например, параметры~$\nu$,
$\mu$ и~$\lambda$ можно оценить, минимизируя соответствующую
статистику хи-квад\-рат. Или же, например, можно решить задачу
наименьших квад\-ратов:
\begin{multline*}
(\nu^*,\mu^*,\lambda^*)={}\\
{}=\arg\min\limits_{\nu,\mu,\lambda}\sum\limits_{i=1}^K
\left[p_i- \!\!\!\!\!
\int\limits_{(1/2)\left(u_{i-1}+u_i\right)}^{(1/2)(u_i+u_{i+1})}\!\!\!\!\!\!\!\!\!\!\!\!\!\!\!
p_{GIG}(u;\,\nu,\mu,\lambda)\,du\right]^2,
\end{multline*}
где $u_0=0$; $u_{K+1}\hm=\infty$.

На практике хорошие результаты показал подход с решением задачи
наименьших квадратов. Для поиска параметров использовался алгоритм
ns2sol, описанный в~книге~\cite{DSch1983}. Указанный алгоритм
доступен во многих статистических пакетах, отличается высоким
быстродействием и~возможностью при желании задавать разумные
интервалы для поиска параметров.

%\vspace*{-9pt}

\section{О практическом выборе сетки
на~первом этапе моди\-фи\-ци\-ро\-ван\-но\-го
сеточного метода разделения дисперсионно-сдвиговых смесей нормальных
законов}

Естественно, что при использовании указанного двухэтапного метода
в~динамическом режиме крайне важным становится вопрос о~выборе
наиболее эффективных и~быстродействующих численных процедур и~их
параметров. В~частности, исключительную важность приобретает
правильный выбор сетки на первом этапе. Рассмотрим этот вопрос
подробнее.

Формально рассматриваемая задача выглядит так: по наблюдаемым
значениям $x_1,\ldots,x_n$ требуется построить статистическую оценку
верхней границы квантилей заданного порядка сме\-ши\-ва\-юще\-го закона так,
чтобы как можно точнее оценить носитель смешивающего распределения.

В дальнейшем будем считать, что $x_1,\ldots,x_n$~--- независимые
реализации случайной величины $X\hm=Y\sqrt{U}+\alpha U$, где $Y$~---
случайная величина со стандартным нормальным распределением, а~$U$~---
независимая от нее случайная величина с~обобщенным обратным
гауссовским распределением. Тогда, очевидно, распределение случайной
величины~$X$ имеет вид~(1). Предположим, что у~случайной величины~$U$
существуют моменты первых двух порядков. Тогда, как несложно видеть,
\begin{equation}
{\sf E}X={\sf E}Y\cdot{\sf E}\sqrt{U}+\alpha{\sf E}U=\alpha{\sf
E}U\,.\label{e5-kor}
\end{equation}
При этом по усиленному закону больших чисел с~вероятностью единица
$\overline x\hm\longrightarrow {\sf E}X$ $(n\hm\to\infty)$, так что при
больших~$n$ справедливо приближенное равенство ${\sf E}X\hm\approx\overline x$
и~с учетом~(\ref{e5-kor})
\begin{equation}
{\sf E}U\approx\fr{\overline x}{\alpha}\,.\label{e6-kor}
\end{equation}
Далее, очевидно,

\columnbreak

\noindent
\begin{multline}
{\sf E}X^2={\sf E}Y^2\cdot{\sf E}U+2\alpha{\sf E}X\cdot{\sf E}U^{3/2}+{}\\
{}+
\alpha^2{\sf E}U^2={\sf E}U+\alpha^2{\sf E}U^2\,.
\label{e7-kor}
\end{multline}

\noindent
Поэтому, обозначив
$$
m^2=\fr{1}{n}\sum\limits_{i=1}^nx_i^2\,,
$$
получаем приближенное равенство ${\sf E}X^2\hm\approx m^2$, так что
с~учетом~(\ref{e6-kor}) и~(\ref{e7-kor}) имеем:
\begin{equation}
{\sf E}U^2\approx\fr{1}{\alpha^2}\left(m^2-\fr{\overline
x}{\alpha}\right)\,.\label{e8-kor}
\end{equation}
Если параметр~$\alpha$ известен, то для определения верхней границы~$u^*$
сетки, накидываемой на носитель распределения случайной
величины~$U$, можно задать малое положительное число~$\varepsilon$
и~воспользоваться требованием
\begin{equation}
{\sf P}(U\geqslant u^*)\leqslant\varepsilon\,.\label{e9-kor}
\end{equation}
А~для гарантированного выполнения требования~(\ref{e9-kor}) можно использовать
неравенство Маркова:
$$
{\sf P}(U\geqslant u^*)\leqslant\fr{{\sf E}U^2}{(u^*)^2}\leqslant \varepsilon\,,
$$
откуда с учетом~(\ref{e8-kor})
$$
(u^*)^2\geqslant\fr{{\sf E}U^2}{\varepsilon}\approx
\fr{1}{\alpha^2\varepsilon}\left( m^2-\fr{\overline x}{\alpha}\right)
$$
или
\begin{equation}
u^*\approx\fr{1}{\alpha\sqrt{\varepsilon}}\sqrt{m^2-
\fr{\overline x}{\alpha}}\,.\label{e10-kor}
\end{equation}

\begin{figure*}[b] %fig1
\vspace*{1pt}
 \begin{center}
 \mbox{%
 \epsfxsize=161.718mm
 \epsfbox{kor-1.eps}
 }
 \end{center}
 \vspace*{-9pt}
\Caption{Примеры применения модифицированного двухэтапного сеточного
ЕМ-ал\-го\-рит\-ма для подгонки обобщенного гиперболического распределения
к искусственным данным, $\beta\hm=0$: (\textit{a})~$n\hm=1000$, $\alpha\hm=0{,}3$,
$\nu\hm=1{,}3$, $\mu\hm=1{,}6$, $\lambda\hm=0{,}2$;
(\textit{б})~$n\hm=1000$, $\alpha\hm=0{,}5$, $\nu\hm=1$, $\mu\hm=1$,
$\lambda\hm=3$;
(\textit{в})~$n\hm=1000$, $\alpha\hm=3$,
 $\nu\hm=1{,}3$, $\mu\hm=1{,}6$, $\lambda\hm=2$;
(\textit{г})~$n\hm=10\,000$,
$\alpha\hm=0{,}3$, $\nu\hm=1{,}3$, $\mu\hm=1{,}6$, $\lambda\hm=0{,}2$}
\end{figure*}


Если же параметр~$\alpha$, определяющий асим\-мет\-рию распределения
случайной величины~$X$, неизвестен, то можно воспользоваться
следующими рассуждениями. Обозначим
$$
q_n=\fr{1}{n}\sum\limits_{i=1}^n{\bf 1}(x_i<0)\,,
$$
где ${\bf 1}(A)$~--- индикаторная функция множества (события)~$A$.
При этом по усиленному закону больших чисел с~вероятностью единица
$q_n\hm\longrightarrow {\sf P}(X\hm<0)$ $(n\hm\to\infty)$, так что при
больших~$n$ справедливо приближенное равенство
\begin{equation}
q_n\approx{\sf P}(X<0)\,.\label{e11-kor}
\end{equation}
Но
\begin{multline}
{\sf P}(X<0)=\int\limits_{0}^{\infty}\Phi
\left(-\alpha\sqrt{u}\right) p_{\mathrm{GIG}}(u;\nu,\mu,\lambda)\,du={}\\
{}=
{\sf E}\Phi\left(-\alpha\sqrt{U}\right)\,.\label{e12-kor}
\end{multline}

\pagebreak

\noindent
Предположим сначала, что $q_n\hm<1/2$. Если~$n$ достаточно велико,
то можно с~большой степенью
 уверенности утверж\-дать, что тогда
$\overline x\hm>0$ и~$-\alpha\hm<0$, т.\,е.
 $\alpha\hm>0$ и,~стало быть, на
положительной полуоси значений аргумента~$u$ функция $\Phi(\alpha u)$
вогнута, т.\,е.\ выпукла вверх. Тогда из~(\ref{e11-kor}) и~(\ref{e12-kor}), дважды
применяя неравенство Иенсена, в~силу монотонности функции~$\Phi$
получаем:
\begin{multline}
1-q_n\approx 1-{\sf E}\Phi\left(-\alpha\sqrt{U}\right)=
          {\sf E}\Phi\left(\alpha\sqrt{U}\right)\leqslant{}\\
          {}\leqslant\Phi
          \left(\alpha{\sf E}\sqrt{U}\right)\leqslant
          \Phi\left(\alpha\sqrt{{\sf E}U}\right)\,.\label{e13-kor}
\end{multline}
Если теперь для $t\hm\in(0,1)$ символом~$v_t$ обозначить $t$-кван\-тиль
стандартного нормального закона, то из~(\ref{e13-kor}) и~(\ref{e6-kor}) вытекает
<<приближенное неравенство>>
$$
v_{1-q_n}\hm\leqslant \alpha\sqrt{{\sf E}U}\,,
$$
т.\,е.
$$
\alpha\geqslant\fr{v_{1-q_n}}{\sqrt{{\sf E}U}}\approx
\fr{v_{1-q_n}\sqrt{\alpha}}{\sqrt{\overline x}}\,,
$$
откуда получаем, что при достаточно больших~$n$
\begin{equation}
\alpha\geqslant\fr{v_{1-q_n}^2}{\overline x}\,.\label{e14-kor}
\end{equation}
Если теперь задать малое положительное число~$\varepsilon$, то
для определения верхней границы~$u^*$ сетки, накидываемой на
носитель распределения случайной величины~$U$, можно воспользоваться
требованием~(\ref{e9-kor}), для гарантированного выполнения которого
с~учетом~(\ref{e6-kor}) и~(\ref{e14-kor}) можно использовать неравенство Маркова:
$$
{\sf P}(U\geqslant u^*)\leqslant \fr{{\sf E}U}{u^*}\approx\fr{\overline
x}{\alpha u^*}\leqslant \fr{(\overline x)^2}{v_{1-q_n}^2 u^*}\leqslant
\varepsilon\,,
$$
откуда окончательно вытекает оценка
\begin{equation}
u^*\approx\fr{(\overline x)^2}{v_{1-q_n}^2 \varepsilon}\,.\label{e15-kor}
\end{equation}

\begin{figure*}[b] %fig2
\vspace*{18pt}
 \begin{center}
 \mbox{%
 \epsfxsize=162.433mm
 \epsfbox{kor-3.eps}
 }
 \end{center}
 \vspace*{-9pt}
\Caption{Примеры применения модифицированного двухэтапного
сеточного ЕМ-ал\-го\-рит\-ма для подгонки обобщенного гиперболического
распределения к~искусственным данным, $n=10\,000$, $\beta\hm=0$:
(\textit{а})~$\alpha\hm=0{,}3$,
$\nu\hm=2$, $\mu\hm=2$, $\lambda\hm=2{,}5$;
(\textit{б})~$\alpha\hm=0{,}5$,  $\nu\hm=1$, $\mu\hm=1$, $\lambda\hm=3$;
(\textit{в})~$\alpha\hm=0{,}8$,
$\nu\hm=1{,}3$, $\mu\hm=1{,}6$, $\lambda\hm=2$;
(\textit{г})~$\alpha\hm=1{,}3$, $\nu\hm=2$, $\mu\hm=2$, $\lambda\hm=2{,}5$}
\end{figure*}



В случае $q_n\hm\geqslant1/2$, если $n$ достаточно велико, то можно
с~большой степенью уверенности утверж\-дать, что $\overline x\hm\leqslant 0$
и~$-\alpha\hm\geqslant 0$, т.\,е.\ на положительной\linebreak\vspace*{-12pt}

\pagebreak

%\end{multicols}


%\begin{multicols}{2}

\noindent
 полуоси значений аргумента~$u$
функция $\Phi(-\alpha u)$ вогнута, т.\,е.\ выпукла вверх. Тогда
из~(\ref{e11-kor}) и~(\ref{e12-kor}), дважды применяя неравенство Иенсена, в~силу
монотонности функции~$\Phi$ получаем
$$
q_n\approx {\sf E}\Phi\left(-\alpha\sqrt{U}\right)\leqslant
\Phi\left(-\alpha\sqrt{{\sf E}U}\right)\,,
$$
откуда вытекает <<приближенное неравенство>> $v_{q_n}\hm \leqslant
-\alpha\sqrt{{\sf E}U}$,
т.\,е.
$$
-\alpha\geqslant\fr{v_{q_n}}{\sqrt{{\sf E}U}}\approx
\fr{v_{q_n}\sqrt{|\alpha|}}{\sqrt{|\overline x|}}
$$
и при достаточно больших~$n$
\begin{equation}
|\alpha|\geqslant\fr{v_{q_n}^2}{|\overline x|}\,.\label{e16-kor}
\end{equation}
Для определения верхней границы~$u^*$ сетки, накидываемой на
носитель распределения случайной величины~$U$, снова зададим малое
положительное число~$\varepsilon$ и~потребуем, чтобы было
справедливо условие~(\ref{e9-kor}), для гарантированного выполнения которого
с~учетом~(\ref{e6-kor}) и~(\ref{e16-kor}) используем неравенство Маркова и~тот факт, что
$\mathrm{sign}\, \overline x\hm=\mathrm{sign}\,\alpha$ при достаточно
больших~$n$:
\begin{multline}
{\sf P}(U\geqslant u^*)\leqslant \fr{{\sf E}U}{u^*}\approx
\fr{\overline x}{\alpha u^*}=
\fr{|\overline x|}{|\alpha| u^*} \leqslant{}\\
{}\leqslant
\fr{(\overline x)^2}{v_{q_n}^2 u^*}\leqslant
\varepsilon\,.\label{e17-kor}
\end{multline}
В силу симметричности нормального распределения $v_{t}\hm=-v_{1-t}$ для
любого $t\hm\in(0,1)$, поэтому $v_{q_n}^2\hm=v_{1-q_n}^2$ и~в~случае
$q_n\hm\geqslant1/2$ соотношение~(\ref{e17-kor}) снова приводит к~оценке~(\ref{e15-kor}).

Справедливости ради необходимо отметить, что оценки~(\ref{e10-kor}) и~(\ref{e15-kor})
являются завышенными, но они гарантируют, что
$(1-\varepsilon)$-почти-весь носитель распределения случайной
величины~$U$ будет лежать внутри интервала $[0, u^*]$.

\section{Результаты численных экспериментов}

Приводимые в~данном разделе графики иллюстрируют качество работы
модифицированного сеточного метода разделения дис\-пер\-си\-он\-но-сдви\-го\-вых
смесей нормальных законов на примере его\linebreak применения к~оцениванию
параметров обоб\-щенных гиперболических распределений с~ис\-поль\-зованием
указанного алгоритма выбора сетки\linebreak с~умеренным чис\-лом узлов $K\hm=40$.
Для вы\-чис\-ле\-ний использовались искусственно сгенерированные выборки
объемов $n\hm=1000$ и~$n\hm=10\,000$ с~разными наборами параметров, значения
которых указаны на рисунках. На рис.~1 и~2 изображены гистограммы
(серые столбики) и~графики
истинной плот\-ности (штриховые линии), промежуточной
оценки, полученной сеточным ЕМ-ал\-го\-рит\-мом (пунктирные линии)
и~итоговой оценки (непрерывные линии). На рис.~1 и~2 так\-же указаны
значения полученных оценок параметров. Как видно из приводимых
рисунков, параметры~$\alpha$ оцениваются очень точно. Точность
оценок остальных параметров удовлетворительная и~может быть повышена
за счет использования более частых сеток и~более чувствительных
критериев остановки ЕМ-ал\-го\-рит\-ма на первом этапе. Следует отметить,
что даже в~тех случаях, в~которых наблюдаются заметные расхождения
оценок параметров и~их точных значений, оценки самих плотностей
довольно \mbox{точны}.




{\small\frenchspacing
 {%\baselineskip=10.8pt
 \addcontentsline{toc}{section}{References}
 \begin{thebibliography}{99}
\bibitem{k2011}
\Au{Королев В.\,Ю.} Ве\-ро\-ят\-но\-ст\-но-ста\-ти\-сти\-че\-ские методы
декомпозиции волатильности хаотических процессов.~--- М.: Изд-во
Московского ун-та, 2011.

\bibitem{n2013}
\Au{Назаров А.\,Л.} Приближенные методы разделения смесей
вероятностных распределений: Дисс.\ \ldots\  канд. физ.-мат. наук.~--- М.:
МГУ им.\ М.\,В.~Ломоносова, 2013.

\bibitem{BN1977}
\Au{Barndorff-Nielsen~O.-E.} Exponentially decreasing distributions
for the logarithm of particle size~// Proc. Roy. Soc. Lond.~A,
1977. Vol.~353. P.~401--419.

\bibitem{BN1978}
\Au{Barndorff-Nielsen~O.-E.} Hyperbolic distributions and
distributions of hyperbolae~// Scand. J. Statist., 1978. Vol.~5.
P.~151--157.

\bibitem{BN1982}
\Au{Barndorff-Nielsen~O.-E., Kent~J., S\!{\!\ptb{\!\o}}\,rensen~M.} Normal
variance-mean mixtures and $z$-distributions~// Int. Statist. Rev.,
1982. Vol.~50. No.\,2. P.~145--159.

\bibitem{ks2012}
\Aue{Королев В.\,Ю., Соколов И.\,А.} Скошенные распределения
Стьюдента, дисперсионные гам\-ма-рас\-пре\-де\-ле\-ния и~их обобщения как
асимптотические аппроксимации~// Информатика и~её применения, 2012.
Т.~6. Вып.~1. С.~2--10.

\bibitem{zk2013}
\Au{Закс Л.\,М., Королев В.\,Ю.} Обобщенные дисперсионные
гам\-ма-рас\-пре\-де\-ле\-ния как предельные для случайных сумм~// Информатика
и её применения, 2013. Т.~7. Вып.~1. С.~105--115.

\bibitem{k2013}
\Au{Королев В.\,Ю.} Обобщенные гиперболические
распределения как предельные для случайных сумм~// Тео\-рия
вероятностей и~ее применения, 2013. Т.~58. Вып.~1. С.~117--132.

\bibitem{kckg2013}
\Au{Королев В.\,Ю., Черток А.\,В., Корчагин~А.\,Ю.,
Горшенин~А.\,К.} Ве\-ро\-ят\-но\-ст\-но-ста\-ти\-сти\-че\-ское моделирование
информационных потоков в~сложных финансовых системах на основе
высокочастотных данных~// Информатика и~её применения, 2013. Т.~7.
Вып.~1. С.~12--21.

\bibitem{p2004}
\Au{Protassov R.\,S.} EM-based maximum likelihood parameter
estimation for a~multivariate generalized hyperbolic distribution
with fixed~$\lambda$~// Statistics Computing, 2004. Vol.~14.
P.~67--77.

\bibitem{kn2010}
\Au{Королев В.\,Ю., Назаров А.\,Л.} Разделение смесей
вероятностных распределений при помощи сеточных методов моментов и~максимального правдоподобия~//
Автоматика и~телемеханика, 2010. Вып.~3. С.~98--116.

\bibitem{DSch1983}
\Au{Dennis J.\,E., Schnabel R.\,B.} Numerical methods for
unconstrained optimization and nonlinear equations.~--- Englewood
Cliffs: Prentice-Hall, 1983. 378~p.
 \end{thebibliography}

 }
 }

\end{multicols}

\vspace*{-6pt}

\hfill{\small\textit{Поступила в редакцию 01.10.14}}

\newpage

%\vspace*{12pt}

%\hrule

%\vspace*{2pt}

%\hrule

%\vspace*{12pt}

\def\tit{A MODIFIED GRID METHOD FOR~STATISTICAL SEPARATION
OF~NORMAL VARIANCE-MEAN MIXTURES}

\def\titkol{A modified grid method for statistical separation
of~normal variance-mean mixtures}

\def\aut{V.\,Yu.~Korolev$^{1,2}$ and~A.\,Yu.~Korchagin$^1$}

\def\autkol{V.\,Yu.~Korolev and~A.\,Yu.~Korchagin}

\titel{\tit}{\aut}{\autkol}{\titkol}

\vspace*{-9pt}


\noindent
$^1$Faculty of Computational Mathematics and Cybernetics,
M.\,V.~Lomonosov Moscow State University,\linebreak
$\hphantom{^1}$1-52 Leninskiye Gory, GSP-1, Moscow 119991, Russian Federation


\noindent
$^2$Institute of Informatics Problems, Russian Academy of Sciences,
44-2~Vavilov Str., Moscow 119333, Russian\linebreak
$\hphantom{^1}$Federation

\def\leftfootline{\small{\textbf{\thepage}
\hfill INFORMATIKA I EE PRIMENENIYA~--- INFORMATICS AND
APPLICATIONS\ \ \ 2014\ \ \ volume~8\ \ \ issue\ 4}
}%
 \def\rightfootline{\small{INFORMATIKA I EE PRIMENENIYA~---
INFORMATICS AND APPLICATIONS\ \ \ 2014\ \ \ volume~8\ \ \ issue\ 4
\hfill \textbf{\thepage}}}

\vspace*{3pt}

\Abste{A~modified two-stage grid method for
statistical separation of normal variance-mean mixtures is described
as an alternative to a pure EM (expectation-maximization) algorithm.
At the first stage of this
algorithm, a~discrete approximation is constructed to the mixing
distribution. At the second stage, the obtained discrete
distribution is approximated by an absolutely continuous
distribution from a~predetermined family, say, by a generalized
inverse Gaussian distribution. The convergence of this two-stage
procedure is discussed. The monotonicity of the grid procedure used
at the first stage is proved. The problem of the optimal choice of
the parameters of the method is discussed in detail. First of all,
the problem of the optimal choice of the grid thrown on the support
of the mixing distribution is considered. Statistical estimators are
proposed for the quantiles of the mixing law. The efficiency of the
method is illustrated by examples of its application to the
estimation of the parameters of generalized hyperbolic
distributions.}

\smallskip

\KWE{mixture of probability distributions; normal
variance-mean mixture; generalized hyperbolic distribution;
EM-algorithm; grid method of separation of mixtures}

\DOI{10.14357/19922264140402}

\Ack
\noindent
The research was supported by the Russian Science Foundation (project 14-11-00364).

%\vspace*{3pt}

  \begin{multicols}{2}

\renewcommand{\bibname}{\protect\rmfamily References}
%\renewcommand{\bibname}{\large\protect\rm References}



{\small\frenchspacing
 {%\baselineskip=10.8pt
 \addcontentsline{toc}{section}{References}
 \begin{thebibliography}{99}
 \bibitem{k2011eng}
 \Aue{Korolev, V.\,Yu.} 2011.
\textit{Veroyatnostno-statisticheskie metody dekompozitsii
volatil'nosti khaoticheskikh protsessov}
[Probabilistic and statistical methods for the decomposition of volatility
of chaotic processes].
Moscow: Moscow University Press. 510~p.

\bibitem{n2013eng}
\Aue{Nazarov, A.\,L.} 2013.
{Priblizhennye metody razdeleniya smesey veroyatnostnykh raspredeleniy}
[Approximate methods for the decomposition of volatility of chaotic processes].
Ph.D. Thesis. Moscow: Moscow State University.

\bibitem{BN1977eng}
\Aue{Barndorff-Nielsen, O.\,E.} 1977.
Exponentially decreasing distributions for the logarithm of particle size.
\textit{Proc. Roy. Soc. Lond. A} 353:401--419.

\bibitem{BN1978eng}
\Aue{Barndorff-Nielsen, O.\,E.} 1978.
Hyperbolic distributions and distributions of hyperbolae.
\textit{Scand. J. Statist.} 5:151--157.

\bibitem{BN1982eng}
\Aue{Barndorff-Nielsen, O.\,E., J.~Kent, and M.~S\!{\ptb{\o}}rensen}. 1982.
Normal variance-mean mixtures and $z$-distributions.
\textit{Int. Statist. Rev.} 50(2):145--159.

\bibitem{ks2012eng}
\Aue{Korolev, V.\,Yu., and I.\,A. Sokolov}. 2012.
{Skoshennye raspredeleniya St'yudenta, dispersionnye
gam\-ma-ras\-pre\-de\-le\-niya i~ikh obobshcheniya kak asimptoticheskie
approksimatsii}
[Skewed Student's distributions, variance gamma distributions, and their
generalizations as asymptotic approximations].
\textit{Informatika i ee Primeneniya}~--- \textit{Inform. Appl.} 6(1):2--10.

\bibitem{zk2013eng}
\Aue{Korolev, V.\,Yu., and L.\,M.~Zaks}. 2013.
{Obobshchennye dispersionnye gam\-ma-ras\-pre\-de\-le\-niya kak
predel'nye dlya sluchaynykh summ}
[Generalized variance gamma distributions as limiting for random sums].
\textit{Informatika i ee Primeneniya}~--- \textit{Inform. Appl.} 7(1):105--115.

\bibitem{k2013eng} \Aue{Korolev, V.\,Yu.} 2013.
{Obobshchennye giperbolicheskie raspredeleniya kak predel'nye dlya sluchaynykh summ}
[Generalized hyperbolic distributions as limiting for random sums]
\textit{Theory Probab. Appl.} 58(1):117--132.

\bibitem{kckg2013eng}
\Aue{Korolev, V.\,Yu., A.\,V. Chertok, A.\,Yu.~Korchagin, and A.\,K.~Gorshenin}.
2013. {Ve\-ro\-yat\-no\-st\-no-sta\-ti\-sti\-che\-skoe
mo\-de\-li\-ro\-va\-nie informatsionnykh potokov v~slozhnykh finansovykh sistemakh
na osnove vysokochastotnykh dannykh}
[Probability and statistical modeling of information flows in complex
financial systems from high-frequency data].
\textit{Informatika i~ee Primeneniya}~--- \textit{Inform.  Appl.} 7(1):12--21.

\bibitem{p2004eng-1}
\Aue{Protassov, R.\,S.} 2004.
EM-based maximum likelihood parameter estimation for a multivariate
generalized hyperbolic distribution with fixed~$\lambda$.
\textit{Statistics Computing} 14:67--77.

\bibitem{kn2010eng-1}
\Aue{Korolev, V.\,Yu., and A.\,L.~Nazarov}. 2010.
{Razdelenie smesey veroyatnostnykh raspredeleniy pri pomoshchi
setochnykh metodov momentov i~maksimal'nogo pravdopodobiya}
[Separation of mixtures using grid moment-based methods and maximum likelihood].
\textit{Avtomatika i~Telemekhanika} [Automatics and Telemechanics] 3:98--116.

\bibitem{DSch1983eng}
\Aue{Dennis, J.\,E., and R.\,B.~Schnabel}. 1983.
\textit{Numerical methods for unconstrained optimization and nonlinear equations}.
Englewood Cliffs: Prentice-Hall. 378~p.


\end{thebibliography}

 }
 }

\end{multicols}

\vspace*{-6pt}

\hfill{\small\textit{Received October 01, 2014}}

\vspace*{-18pt}

\Contr

\noindent
\textbf{Korolev Victor Yu.} (b.\ 1954)~---
Doctor of Science in physics and mathematics, professor,
Department of Mathematical Statistics, Faculty of Computational Mathematics
and Cybernetics, M.\,V.~Lomonosov Moscow State University,
1-52 Leninskiye Gory, GSP-1, Moscow 119991, Russian Federation;
leading scientist, Institute of Informatics Problems,
Russian Academy of Sciences, 44-2~Vavilov Str., Moscow 119333, Russian
Federation; victoryukorolev@yandex.ru

\vspace*{3pt}

\noindent
\textbf{Korchagin Alexander Yu.} (b.\ 1989)~---
PhD student, Faculty of Computational Mathematics and Cybernetics,
M.\,V.~Lomonosov Moscow State University,
1-52 Leninskiye Gory, GSP-1, Moscow 119991, Russian Federation;
sasha.korchagin@gmail.com


\label{end\stat}

\renewcommand{\bibname}{\protect\rm Литература}  %+11
\def\stat{kuznetsov}

\def\tit{УНИВЕРСАЛЬНАЯ ТЕХНОЛОГИЯ ОЦЕНКИ БЛИЗОСТИ ИНФОРМАЦИОННЫХ
ОБЪЕКТОВ}

\def\titkol{Универсальная технология оценки близости информационных
объектов}

\def\autkol{Л.\,А.~Кузнецов}

\def\aut{Л.\,А.~Кузнецов$^1$}

\titel{\tit}{\aut}{\autkol}{\titkol}

%{\renewcommand{\thefootnote}{\fnsymbol{footnote}} \footnotetext[1]{Работа
%выполнена при финансовой поддержке РФФИ (проект 11-01-00515а).}}

\renewcommand{\thefootnote}{\arabic{footnote}}
\footnotetext[1]{Российская академия народного хозяйства и государственной службы при Президенте Российской Федерации
(Липецкий филиал), Kuznetsov.Leonid48@gmail.com}

  \Abst{Изложена технология определения степени подобия информационных объектов,
которые представлены текстами или графическими изображениями. Объекты
формализуются вероятностными моделями. Структура модели задается алгеброй на
минимальном наборе изобразительных компонентов объекта. Количественными
характеристиками структуры объектов являются распределения вероятностей на заданной
алгебре. Количество информации в объектах оценивается энтропией. На энтропиях задается
мера информационного подобия сопоставляемых объектов. Показана методика
формирования оценки для текстовых и графических объектов. Приведены примеры
реализации алгоритмов оценки и показана более высокая эффективность разработанных
методов по сравнению с методами, описанными в литературе. Технология формирования
образов информационных объектов и сравнения их семантического содержания является
универсальной. Показаны возможности адаптации разработанной технологии к
содержательным характеристикам исследуемых объектов.}

  \KW{информационный объект; текст; изображение; вероятностная модель; семантическое
подобие; энтропия; мера подобия}

\DOI{10.14357/19922264140213}

\vskip 14pt plus 9pt minus 6pt

      \thispagestyle{headings}

      \begin{multicols}{2}

            \label{st\stat}


\section{Введение}

  Оценка близости информационных объектов является наиболее
распространенным компонентом информационных технологий. На компоненте
оценки информационного подобия объектов базируются технологии поиска
информации, сравне\-ния проектов, оценки оригинальности научных результа\-тов
и~т.\,п. На этом компоненте в дальнейшем будут разработаны автоматические
про\-цедуры дифференцированной оценки знаний, вклю\-ча\-ющие оценку уровня
близости текстов на естественных и формальных языках, структури\-рованных и
неструктурированных графических объектов, т.\,е.\ всех компонентов
представления знаний, к определению уровня усвоения которых сводится
проверка качества обучения. Совокупность процедур оценки близости
информационных объектов, представленных на разных <<языках>>, позволит
создавать автоматизированные системы проверки качества на всех уровнях
обучения. Такие системы позволят исключить тестовый самообман, обеспечить
объективность полноценной оценки качества подготовки, устранить
возможность коррупции и разнообразных подтасовок в сфере образования.

  Ниже излагается универсальная технология формальной оценки уровня
подобия информационных объектов, имеющих однотипное представление на
естественном языке (тексты), на формальном языке (формулы математические,
химические и~др.), графическое представление (схемы, чертежи, картины).

  Информационные объекты обычно представляют собой композицию
количественных данных и неформализованных сведений, которые могут быть\linebreak
представлены текстовой и графической информацией. В~зависимости от
содержания и предназначения объекта доля текстовой и графической
информа\-ционных составляющих может иметь определяющее значение.
Поэтому разработка моделей формального описания и оценки подобия
объектов, представленных текстовой и графической информацией, является
важной задачей, решению которой посвящена работа.

  В информационно-поисковых системах при классификации текстов, при
проверке текстов на плагиат~[1] применяются статистические подходы на
основе век\-тор\-но-про\-стран\-ст\-вен\-ной модели текста, предложенный
Солтоном с соавторами в 1975~г.~[2]. В~ней текст представляется вектором частот
входящих в него слов, а оценка близости текстов равна косинусу угла между
векторами текстов.

  Более эффективные инструменты оценки близости информационных
объектов могут быть синтезированы на основе их представления в виде
однотипных вероятностных моделей, допускающих структуризацию и
количественное сопоставление содержащейся в них информации. Исследования
показали, что вероятностная модель позволяет формально и с произвольной
глубиной детализации описывать объекты, представленные текстами на
естественных и формальных языках~[3] или структурированными~[4] и
неструктурированными графическими изображениями~[5]. Вероятностная
модель позволяет формализовать объекты в виде системы классов различной
информационной значимости. Система классов может адаптироваться к
содержательной специфике сопоставляемых объектов и алфавитам их
представления. Оценка уровня близости объектов производится на системе
классов с учетом их ин\-фор\-ма\-ци\-он\-но-се\-ман\-ти\-че\-ской значимости.
Процедуры формирования системы классов на заданном алфавите и уровня их
значимости могут быть реализованы в виде самонастраивающихся по принципу
обратной связи систем.

\section{Вероятностная модель}

  Сопоставляемые информационные объекты формализуются в виде
вероятностных моделей. Абстрактная вероятностная модель эксперимента с
конечным числом исходов, или просто вероятностная модель, вводится в
теории вероятностей для формального представления результатов
произвольного эксперимента. Модель представляет собой совокупность
трех составляющих~[6]:
  \begin{equation}
  M=\{ \Omega, \aleph, P(A_i)\}\,,
  \label{e1-kuz}
  \end{equation}
  где $\Omega = \{\omega_1, \omega_2, \ldots , \omega_n\}$~--- множество
элементарных событий (исходов или реализаций исследу\-емой случайной
величины);
  $\aleph\hm= \{A_1, A_2, \ldots , A_m\}$~--- множество (система) случайных
событий (алгебра);
  $P(A_i)$, $i \hm= 1, 2, \ldots , m$,~--- вероятности случайных событий.

  В контексте оценки близости информационных объектов под случайными
величинами могут пониматься величины или элементы, совокупностью
которых представляется объект. Случайной величиной или элементом
информационного объекта, представленного текстом на естественном языке,\linebreak
является слово. Для структурированного графи\-ческого объекта,
представленного, например, электри\-ческой схемой, случайной величиной
является стандартизованное обозначение элементов электрических схем. Для
неструктурированного графического объекта, представленного, например,
картиной, случайными величинами являются цвет и координаты пикселов,
отражающих ее на мони\-торе.

  В соответствии с содержанием информационных объектов случайные
величины, используемые для их представления, принимают свои значения~---
элементарные события~--- из конечных множеств, соответствующих
содержанию. Под элементарными событиями понимаются неделимые при
сравнении элементы~$\omega$, из которых формируются объекты. Случайная
величина <<слово>> (для русского текста) может принимать значения всех
слов, имеющихся в словарях русского языка. Случайная величина <<элемент
электрических схем>> может принимать значения элементов из
соответствующего ГОСТа~\cite{7-kuz}. Случайная величина <<пиксел>>
принимает значения из палитры цветов, представимых на мониторе, и
возможных координат его положения.

  Любые объекты представляют собой множества реализаций, или, по
терминологии теории вероятностей, элементарных исходов соответствующих
случайных величин. Текст представляется множеством различных слов,
которые являются реализациями случайной величины <<слово>>,
электрическая схема представляется набором значений величины <<элемент
электрических схем>>, картина~--- множеством реализаций величины
<<пиксел>>. Современные возможности информационных технологий
позволяют достаточно просто проверить полную идентичность
информационных объектов, представленных в электронном виде. Но в
большинстве задач обработки информации требуется не установка
идентичности, а оценка степени бли\-зости информационных объектов,
отраженная некоторой количественной мерой.

  Такая задача может быть решена с использова\-нием вероятностной модели
информационных\linebreak объектов, позволяющей структурировать множество
реализаций случайной величины $\Omega\hm= \{\omega_1,\omega_2, \ldots
,\omega_n\}$ введением системы классов $\aleph\hm=\{A_1, A_2, \ldots , A_m\}$.
Классы~$A_j$, $j\hm=1, 2, \ldots , m$,\linebreak позволяют отразить разнообразные
содержательные особенности и информационную значимость отдельных
совокупностей реализаций случайной величины. Классами может быть
формально пред\-став\-ле\-на семантика информационных объектов, являющаяся
определяющей при оценке их подобия.

  Случайные события $A_i$ конструируются на множестве реализаций
случайной величины $\Omega\hm= \{\omega_1,\omega_2, \ldots ,\omega_n\}$ с
помощью операций $\cup$~--- сложения случайных событий (объединения
множеств),\linebreak $\cap$~--- произведения случайных событий (пересечения
множеств), $\overline{\ \  \vphantom{A} }$~--- отрицания событий (дополнения множеств).
Случайные события~$A_i$ представляют подмножества множества~$\Omega$.
Случайные\linebreak события, получаемые из~$A_i$ с помощью перечисленных
операций, также принадлежат алгебре~$\aleph$. В~ал\-геб\-ру~$\aleph$ входят
невозможное событие (пустое множество~$\emptyset$) и достоверное событие
(множество~$\Omega$).

  Целью практического применения вероятностной модели~(1) является
структуризация информации, содержащейся в множестве~$\Omega$, и
выявление некоторых неслучайных ее характеристик, скрытых во множестве
случайных реализаций $\{\omega_1,\omega_2, \ldots ,\omega_n\}$. Именно эти
содержательные характеристики исследуемой случайной величины и
отражаются в модели~(1) случайными событиями~$A_i$, составляющими
алгебру~$\aleph$.

  Введение алгебры случайных событий $\aleph\hm= \{A_1, A_2, \ldots , A_m\}$
задает систему содержательных (качественных) характеристик, позволяющих
разделить множество элементарных событий $\Omega\hm=
\{\omega_1,\omega_2, \ldots ,\omega_n\}$ на классы~$A_1, A_2, \ldots , A_m$.
Принципиальным в данном контексте свойством ал\-геб\-ры является возможность
конструирования новых случайных событий из множества уже имеющихся,
потому что новые случайные события, полученные объединением,
пересечением и отрицанием принадлежащих алгебре событий, также
принадлежат алгебре, или точнее:

\vspace*{-2pt}

\noindent
  \begin{multline}
\mbox{из}\  A_i,\ A_j \in\aleph\  \mbox{ следует }\ A_i \cap A_j\in \aleph,\\
 A_i\cup a_j \in \aleph\
\mbox{ и }\ \overline{A}_i \in \aleph,\ \  i, j\in [1, m]\,.
\label{e2-kuz}
\end{multline}

  На основании~(2) алгебра на каждом новом этапе формирования (эволюции)
модели~(1) может неограниченно расширяться и изменяться применением
операций сложения, перемножения и отрицания к системе случайных событий
предшест\-ву\-юще\-го этапа.

  Количественной мерой, определяющей соотношение случайных событий,
образующих алгебру, является третий элемент модели~(1)~--- вероятности
случайных событий $P(A_i)$, которые находятся по вероятностям
элементарных событий:

\vspace*{2pt}

\noindent
  \begin{equation}
  P(A_i)= \sum\limits_{\omega_j\in A_i} p(\omega_j)\,,
  \label{e3-kuz}
  \end{equation}
где $p(\omega_j)$~--- вероятность элементарного события~$\omega_j$.

  Можно видеть, что вероятностная модель~(1) позволяет отразить всю
возможную информацию о случайной величине, которая может быть
формально извлечена из множества ее реализаций~$\Omega$. Система
случайных событий~$\aleph$ обеспечивает возможность выявления
неоднородности элементов множества~$\Omega$, а вероятности случайных
событий $A_i\hm\in \aleph$, $i\hm =1, 2, \ldots , m$, позволяют количественно
оценить степень неоднородности. Возможность модификации алгебры~(2)
позволяет осуществлять адаптацию структуры вероятностной модели~(1),
направление и результаты которой могут определяться некоторыми
функционалами, заданными на распределении вероятностей~(3).

\section{Вероятностная модель текстовых информационных
объектов}

  Оценка близости информационных объектов может осуществляться
сопоставлением их вероятностных моделей. Современные информационные
технологии позволяют представить в электронном виде информацию,
полностью характеризующую объекты самого разного содержания. При этом
информация на электронных носителях обычно представляет собой
комбинацию текстов на естественном языке и графических изображений.
Поэтому для формального представления содержательных информационных
объектов в виде вероятностных моделей~(1)--(3) необходимы инструменты
трансформации текстов и графических изображений в такие абстрактные
модели.

  Пусть информационный объект представлен текстом на естественном языке.
Будем иметь в виду структурированные языки, наделенные морфологией и
синтаксисом. Рассмотрим погружение информационного объекта в
вероятностную модель~(1), или, что одно и то же, формирование по тексту,
представленному на естественном языке, вероятностной модели~(1).
Вероятностная модель позволяет осуществить на необходимом уровне
детализации разложение моделируемых текстов на морфологические и
синтаксические компоненты, которые в обобщенном виде отражают
однотипные семантические представления.

  Уровень детализации структуры текстов определяется из условия конечного
предназначения вероят\-ностной модели~--- оценки степени семантического
подобия объектов (текстов). При разработке технологии можно исходить из
общепринятой практики неавтоматизированной экспертной оценки
семантической близости информационных объектов, представленных текстами
на естественном языке. Эксперт в содержании текстов выделяет и сопоставляет
определяющие семантические аспекты: (1)~объекты, т.\,е.\ о чем или о ком
сообщается в текстах; (2)~действия объекта или с объектом; (3)~образ и условия
действия (как, при каких условиях, где, когда действует объект или
осуществляются действия с ним); (4)~результат действий и~т.\,п.

  Отражению всех содержательных аспектов в структурированных языках
соответствуют определенные синтаксические конструкции, опирающиеся на
морфологию. В~результате разложения сопоставляемых текстов по алгебрам
единой структуры тексты трансформируются в обобщенные предложения, в
которых роль отдельных членов играют введенные компоненты алгебры~---
случайные события~$A_i$, $i\hm=1, 2, \ldots , m$, представляющие обобщенные
подлежащие, сказуемые, дополнения, обстоятельства и~т.\,п. Оценка близости
текстов на множестве компонентов алгебры, с учетом семантической ценности
слов и конструкций, позволяет принципиально изменить качество оценки
подобия текстов по сравнению с отмеченным методом простого подсчета
одинаковых слов в текстах (пересечения списков слов).

  Для иллюстрации формирования вероятностной модели текста~[8] ниже
используется простой объект, представленный на естественном языке
нижеследующим абзацем текста статьи, выделенным курсивом.

  \textit{Адаптация абстрактной модели~$(1)$ к описанию информационных
объектов, представленных текстовой и графической информацией,
достигается конкретизацией случайных величин и алгебры к содержательной
специфике объекта.}

  Содержание этого предложения может быть выражено бесконечным
множеством семантически равноценных фраз, в которых могут использоваться
наборы других слов. Из этого следует, что с
  ин\-фор\-ма\-ци\-он\-но-се\-ман\-ти\-че\-ских позиций слово может
трактоваться случайной величиной, а конкретные слова, использованные в этом
(и любом другом) предложении~--- реализациями, элементарными исходами
случайной величины <<слово>>. Поэтому выделенное предложение может
трактоваться как результат эксперимента, в котором случайная величина
<<слово>> получила следующие элементарные исходы (множество
реализаций):

\vspace*{-2pt}

\noindent
  \begin{multline}
  \Omega_1 =\{\omega_1 = \mbox{адаптация},\ \omega_2 = \mbox{абстрактной},\\
\omega_3 = \mbox{модели},\
  \omega_4 = 1,\ \omega_5 = \mbox{к},\  \omega_6 = \mbox{описанию},\\
  \omega_7 =
\mbox{информационных},\ \omega_8 = \mbox{объектов},\\
\omega_9 = \mbox{представленных},\
  \omega_{10} = \mbox{текстовой},\\
  \omega_{11} = \mbox{и},\ \omega_{12} =
\mbox{графической},\ \omega_{13} = \mbox{информацией},\\
  \omega_{14} = \mbox{достигается},\ \omega_{15}= \mbox{конкретизацией},\\
  \omega_{16} = \mbox{случайных},\ \omega_{17} = \mbox{величин},\ \omega_{18} =
\mbox{и},\\
\omega_{19} = \mbox{алгебры},\ \omega_{20} = \mbox{к},\
  \omega_{21} = \mbox{содержательной},\\
  \omega_{22}= \mbox{специфике},\ \omega_{23} = \mbox{объекта}\}\,.
  \label{e4-kuz}
  \end{multline}

  \vspace*{-2pt}

  Называя элементарные исходы, как оговорено выше, просто элементами,
можно сказать, что исследуемый текст состоит из элементов~(4). Различ\-ные
слова, образующие~$\Omega_1$, несут существенно отличающуюся
семантическую нагрузку, которая\linebreak связа\-на с их морфологической и
синтаксической принадлежностью. Поэтому дифференциация слов~---
элементарных исходов~$\Omega_1$ с учетом их семантической значимости
позволяет значительно повысить уровень адекватности формальной моде-\linebreak\vspace*{-12pt}
\columnbreak

\noindent
ли,
отражающей информационные объекты. При дифференциации слов
целесообразно использовать естественную структуру языка, регламентируемую
морфологией и синтаксисом.

  В структурированных языках морфология определяет принадлежность слов к
частям речи, их грамматические категории и формы. Семантический вес слов
определяется их принадлежностью к определенным частям речи. Синтаксис
регламентирует строй языка, место и роль в предложении отдельных слов,
которые отражают их семантическую значимость. Роль и место слов в
предложении, регули\-ру\-емые синтаксисом, тесно связаны с их морфологией,
определяющей принадлежность слов к\linebreak частям речи и форму представления.
Морфология и синтаксис дополняют друг друга и позволяют синтезировать
алгебру, достаточно полно отражающую семантическую нагрузку слов в тексте.

  Алгебра может синтезироваться на морфологической основе. При этом
система случайных \mbox{событий}, по которым распределяются слова, синтезируется
на частях речи. Для большинства структурированных языков это
существительные, прилагательные, числительные, местоимения и глаголы.

  Синтаксис также может быть принят за основу системы случайных событий,
по которым распределяются слова, составляющие текст. В~этом случае система
событий будет представлена наборами подлежащих, сказуемых, определений,
дополнений и других членов предложения. Система случайных событий, по
которым раскладываются тексты, может конструироваться на комплексной
основе морфологии и синтаксиса.

  Используя для иллюстрации простейшую морфологическую алгебру,
отражающую знаменательные части речи: существительные, прилагательные,
числительные, местоимения, наречия и глаголы (предлоги и союзы
игнорируются), получаем следующую систему случайных событий:

\vspace*{-2pt}

\noindent
  \begin{multline}
\aleph_1 = \{A_1~\mbox{--- существительное},\\
A_2~\mbox{--- прилагательное},\ A_3~\mbox{---
числительное},\\
A_4~\mbox{--- глагол}\}\,,
\label{e5-kuz}
\end{multline}
где

\vspace*{-2pt}

\noindent
\begin{multline*}
A_1=\{\omega_1 = \mbox{адаптация},\ \omega_3= \mbox{модели},\\
\omega_6=
\mbox{описанию},\ \omega_8= \mbox{объектов},\\
\omega_{13}= \mbox{информацией},\ \omega_{15}\ = \mbox{конкретизацией},\\
\omega_{17} = \mbox{величин},\ \omega_{19} = \mbox{алгебры},\\
\omega_{22} =
\mbox{специфике},\ \omega_{23} = \mbox{объекта}\}\,;
\end{multline*}

\vspace*{-12pt}

\noindent
\begin{multline*}
    A_2 = \{\omega_2 = \mbox{абстрактной},\\
     \omega_7= \mbox{информационных},\\
          \omega_9 = \mbox{представленных},\ \omega_{10} =
\mbox{текстовой},
     \end{multline*}

     \noindent
     \begin{multline*}
\omega_{12} = \mbox{графической},\ \omega_{16} = \mbox{случайных},\\
\omega_{21} = \mbox{содержательной}\}\,;
    \end{multline*}

    \noindent
    $$
    A_3=\{\omega_4= 1\}\,;
    $$
    $$
    A_4 =\{\omega_{14}= \mbox{достигается}\}\,.
    $$

  Четыре слова (к, и, и, к), содержащиеся в~$\Omega_1$~(4), исключены из
дальнейшего рассмотрения, так что общее число сохраненных элементарных
событий~--- количество исходов~--- 19. Вероятность каж\-до\-го из них
$p(\omega_j) \hm= 1/19$. Вероятности случайных событий будут равны:
  \begin{gather*}
  P(A_1) = \fr{10}{19}\,;\enskip  P(A_2) = \fr{7}{19}\,;\\
  P(A_3) = \fr{1}{19}\,;\enskip  P(A_4) = \fr{1}{19}\,.
%  \label{e6-kuz}
  \end{gather*}
  При этом выполняется условие нормированности вероятности:
$$
P(A_1) + P(A_2) + P(A_3) + P(A_4) = 1\,.
$$

  Формальное представление выделенного текста в виде вероятностной
модели~(1) имеет вид:
  \begin{equation}
  M_1 =\{ \Omega_1, \aleph_1, P(A_i)\}\,,
  \label{e7-kuz}
  \end{equation}
где $\Omega_1$~--- множество~(\ref{e4-kuz}) элементарных исходов~--- слов,
принадлежащих знаменательным частям речи;
  $\aleph_1$~--- система случайных событий~(\ref{e5-kuz}), ассоциируемая (в
примере) со знаменательными частями речи;
  $P(A_i)$, $i \hm= 1, 2, 3, 4$,~--- вероятности случайных событий.

  Индекс~1 в модели~(\ref{e7-kuz}) подчеркивает, что это модель конкретного
информационного объекта, определенного множеством элементарных
исходов~$\Omega_1$. При сопоставлении информационных объектов,
представленных текстами, они формализуются моделями~$M_1$ и~$M_2$
вида~(\ref{e7-kuz}), которые отражают их вероятностную структуру на
введенной алгебре. Уровень подобия объектов оценивается по близости
распределений вероятностей по единой для обоих объектов алгебре.
Количественной мерой подобия является взаимная информация в объектах,
определение которой рассматривается ниже.

\section{Вероятностная модель графических информационных
объектов}

  Графические информационные объекты, оценку содержательной близости
которых требуется производить при решении различных практических задач,
формализуются в виде вероятностной модели~(1). Построение формального
образа~(1) графического объекта зависит от наличия или отсутствия некоторого
конечного <<алфавита>> в его исходном представлении. Под <<алфавитом>> в
данном случае понимается конечный набор определенных графических
структур~--- элементов, из которых компонуются сопоставляемые в процессе
оценки близости графические информационные объекты.

  Наличие <<алфавитов>> характерно для изображений большинства
искусственно создаваемых объектов: сооружений и схем различного
содержания и предназначения. Такие объекты можно назвать
структурированными. Отличными от них являются неструктурированные
изображения, например объекты искусства или визуальные изображения
некоторых фаз, картин протекания технологических процессов и состояний в
них некоторого континуума.

  Построение формального образа~(1) структурированного объекта может
осуществляться по изложенной в предыдущем пункте методике. Наличие
конечного алфавита или конечного набора элементов позволяет все элементы
моделируемого объекта однозначно связать с конечным множеством $\aleph$
классов~$A_j$ элементов. Классы~$A_j$ могут конструироваться из
элементов~$\omega_i$, $i\hm= 1, 2, \ldots , n$, в соответствии с~(2) и отражать
любые сложные\linebreak
композиции из элементов и классов, соответствующие
содержательным представлениям в предметной области. По близости
однотипных классов оценивает\-ся степень отличия или подобия сопоставляемых
объектов. Эти классы, подобно частям речи в предыдущем примере,
объявляются случайными событиями~$A_k$, $k\hm=1, 2, \ldots , K$, множество
которых образует алгебру~$\aleph$. В~результате сопоставляемые графические
объекты, отражающие проекты здания или электрические схемы,
формализуются в виде вероятностных моделей~(1), структура которых
детализируется по существенным для оценки степени их подобия компонентам.
Количественная мера близости объектов определяется по распределению
вероятностей на единой для них алгебре~$\aleph$.

  Неструктурированные графические объекты представляют собой некоторые
непрерывные изоб\-ра\-же\-ния, степень близости которых требуется оценить.
Близкая задача решается в биометрических системах, обеспечивающих оценку
сходства контролируемого изображения с заданным его эталоном на
определенном уровне доверительной вероятности~\cite{9-kuz}. При этом
системы настраиваются на определенный класс объектов (эталонов).
В~рассматриваемом подходе предполагается, что объекты могут иметь
произвольную палитру и очертания. В~этом случае для представления объектов
в виде вероятностных моделей~(1) их необходимо структурировать.

  Любое изображение в электронном виде представляется множеством
  точек~--- пик\-се\-лов. Каждый пиксел содержит в себе закодированную
информацию о цвете, формат представления которой зависит от используемой
цветовой модели. Цветовая модель с помощью современных графических
редакторов может быть трансформирована в любую другую цветовую модель,
которая по тем или иным причинам более удобна для исследователя.

  Наиболее распространенной в современных графических форматах является
цветовая модель $RGB$ (см., например,~\cite{11-kuz, 10-kuz}). Модель $RGB$
является аддитивной, в ней новые цвета образуются путем добавления
основных цветов к базовому черному цвету. Как следует из названия самой
модели, основными цветами являются красный, зеленый и синий. Каждая точка
несет в себе информацию об интенсивностях этих трех основных цветов, из
которых образуется все множество видимых человеком цветовых оттенков. На
представление интенсивности каждого цвета отводится 8~бит (или один октет),
так что цвет может иметь 256~уровней интенсивности (от~0 до~255).
Сочетание цветовых интенсивностей $(0, 0, 0)$ соответствует черному цвету,
$(255, 255, 255)$~--- белому цвету. Суммарное количество цветов, которое
можно представить данной моделью, равно $256\times256\times256 \hm=
16\,777\,216$. Цветовое пространство $RGB$ можно представить в виде
трехмерного куба с осями~$R$, $G$ и~$B$. Сторона куба имеет длину
256~единиц с координатами начала отрезка~0 и конца отрезка~255.

  При переходе к вероятностной модели~(1) случайными величинами
считаются интенсивности цветов в точке. В~существующих алгоритмах
сравнения изображений (например, в методе цветовых
  гистограмм~\cite{11-kuz, 10-kuz}) $RGB$-про\-стран\-ст\-во делится на
несколько непересекающихся подпространств, или областей. При обработке
изображений подсчитывается количество пикселов, попадающих в каж\-дую из
выделенных областей пространства $RGB$. В~результате получается
гистограмма распределения час\-тот по подпространствам. Сравнением
гистограммы исследуемого объекта с гистограммой эталона оценивается
степень их близости. При этом вследствие предопределенности эталона
исключается необходимость отражения геометрии объекта.

  Оценка близости произвольных графических объектов требует
сопоставления не только цветовой палитры, но и привязки геометрии объектов
к сис\-те\-ме координат для сопоставления конфигурации. Разработана
модификация метода цветовых гистограмм, позволяющая отразить
конфигурацию сопоставляемых объектов~\cite{12-kuz}. Модификация состоит
в добавлении к традиционному $RGB$-про\-стран\-ст\-ву координатных
осей~$X$ и~$Y$. Комбинированное с координатными осями
  $RGBXY$-про\-стран\-ст\-во позволяет хранить информацию о цветовой
интенсивности и пространственном расположении точек изображения.

  Пятимерное $RGBXY$-пространство может быть разбито на систему
непересекающихся подпространств, которые позволяют определить сис\-те\-му
случайных событий~$A_i$, $i\hm=1, 2, \ldots , m$, образующих
алгебру~$\aleph$. Сетка разбиения необязательно равномерная, что позволяет
детализировать образы изображений. Распределение точек изображения по
подпространствам формирует распределение вероятностей для образов
сопоставляемых объектов. В~результате не структурированные исходно
графические объекты формализуются в виде вероятностных моделей~(1).

\section{Технология оценки близости объектов}

  Технология оценки степени близости со\-по\-став\-ляемых объектов,
представленных их образами в виде вероятностных моделей~(1), может
базироваться на представлениях теории ин\-формации. В~тео\-рии информации
вводится мера количества информации, содержащегося в случайной величине,\linebreak
которая позволяет определить количественные меры соотношения информации
в случайных объектах, характеризующие уровень близости объектов.

  Информационные объекты исходно пред\-став\-ле\-ны различными наборами
элементарных событий $\Omega_1\hm= (\omega^1_1, \omega^1_2,
\ldots , \omega^1_{N_1})$,  $\Omega_2\hm= (\omega_1^2,
\omega_2^2, \ldots$\linebreak $\ldots , \omega^2_{N_2})$~--- реализаций случайных
величин.\linebreak В~данном контексте под случайными величинами понимаются
минимальные неделимые (атомарные) элементы, наборами которых
пред\-став\-ля\-ют\-ся информационные объекты: слова, элементы изображения
формул, схем, пикселы и~т.\,п. Элементарными событиями~--- значениями
случайных величин (реализациями) будут конкретные слова исследуемого
текста, элементы изображения конкретных формул, схем,
  $RGBXY$-зна\-че\-ния пикселов и~т.\,п.

  Для представления объектов в виде вероятностных моделей (образов)
вводится единая алгебра\linebreak $\aleph\hm= \{A_1, A_2, \ldots , A_m\}$, структура
которой должна максимально полно отражать принципиальные\linebreak компоненты
информационной сущности со\-по\-став\-ля\-емых объектов. Содержание
информационных объектов $\Omega_1\hm= (\omega_1^1, \omega_2^1,
\ldots , \omega^1_{N_1})$, $\Omega_2\hm= (\omega_1^2,
\omega_2^2,  \ldots$\linebreak $\ldots , \omega^2_{N_2}\}$ раскладывается по системе
событий $\aleph \hm= \{ A_1, A_2, \ldots , A_m\}$. В~результате информационные
объекты представляются в виде систем случайных событий:
  \begin{equation}
  \label{e8-kuz}
\left.
\begin{array}{rl}
\hspace*{-10mm}\Omega_1&= (\omega_1^1, \omega_2^1,\ldots , \omega^1_{N_1})
\Rightarrow{}\\
&\hspace*{15mm}{}\Rightarrow \aleph = \{ A_1^1, A_2^1, \ldots , A_m^1\}\,; %\label{e8a-kuz}
\\[9pt]
  \hspace*{-10mm}\Omega_2&= (\omega_1^2, \omega_2^2,\ldots , \omega^2_{N_2})
\Rightarrow {}\\
&\hspace*{15mm}{}\Rightarrow\aleph = \{ A_1^2, A_2^2, \ldots , A_m^2\}\,. %\label{e8b-kuz}
\end{array}
\right\}
\end{equation}

  На основании свойства~(2) из случайных событий с помощью операций над
множествами могут быть синтезированы новые случайные события,
принадлежащие алгебре $\aleph \hm= \{A_1, A_2, \ldots$\linebreak $\ldots , A_m\}$.  Количественной
характеристикой распределения реализаций $\Omega_1\hm= (\omega_1^1,
\omega_2^1, \ldots  , \omega^1_{N_1})$, $\Omega_2\hm=
(\omega_1^2, \omega_2^2, \ldots , \omega^2_{N_2})$, составляющих
объекты, по системе случайных событий $\aleph\hm=\{A_1, A_2, \ldots , A_m\}$
служат, в соответствии с~(\ref{e3-kuz}), эмпирические вероятности
  \begin{equation}
  P(A_j^1) =\sum\limits_{w_i^1\in A_j^1} p(\omega_i^1)\,;\enskip  P(A_j^2)
=\sum\limits_{\omega^2_i\in A_j^2} p(\omega_i^2)\,.
  \label{e9-kuz}
  \end{equation}

  В результате информационные объекты $M_1$ и~$M_2$, заданные
множествами реализаций $\Omega_1\hm= (\omega_1^1, \omega_2^1,
 \ldots , \omega^1_{N_1})$, $\Omega_2\hm= (\omega_1^2,
\omega_2^2, \ldots , \omega^2_{N_2})$, представляются в виде
вероятностных моделей
  \begin{equation}
  \label{e10-kuz}
  \left.
  \begin{array}{rl}
  M_1 &=\{ \Omega_1,\aleph, P(A_j^1)\}\,; %\label{e10a-kuz}
  \\[9pt]
  M_2 &= \{\Omega_2, \aleph, P(A_j^2)\}\,. %\label{e10b-kuz}
  \end{array}
  \right\}
\end{equation}

  Назовем формализованное представление информационных
объектов~(\ref{e10-kuz}) в виде вероятностных моделей~(1) образами объектов. Задача состоит
в оценке подобия объектов, имеющих образы~(\ref{e10-kuz}).

  Формальная оценка степени подобия объектов может базироваться на
количественных характеристиках их образов. Адекватной основой для синтеза
таких характеристик представляется теория информации, основоположником
которой является К.~Шеннон~\cite{13-kuz}. В~теории информации количество
информации, содержащееся в случайной величине, может оцениваться энтропией,
которая определяется по распределению вероятностей случайной величины.
Энтропия вероятностной модели, отражающая количество информации в ней,
может определяться в виде
  \begin{equation}
  H=-\sum\limits_{A_j\in \aleph} P(A_j) \ln P(A_j)\,,
  \label{e11-kuz}
  \end{equation}
где $P(A_j)$~--- вероятности случайных событий~(\ref{e9-kuz}).

  На основании энтропии могут быть синтезированы различные меры близости
информационных объектов. Учитывая аналогию терминов и пред\-став\-ле\-ний
теории вероятностей и теории множеств, воспользуемся графической
иллюстрацией,\linebreak\vspace*{-12pt}


\noindent
\begin{center}  %fig1
 \mbox{%
 \epsfxsize=76.556mm
 \epsfbox{kuz-1.eps}
 }
  \end{center}
%  \vspace*{6pt}
{{\figurename~1}\ \ \small{Геометрическая интерпретация  операций с количествами информации,
содержащимися в одноименных случайных событиях системы~$\aleph$ двух объектов}}

\vspace*{12pt}

\addtocounter{figure}{1}


\noindent
 представленной на рис.~1, для пояснения существа
предлагаемых мер близости информационных объектов.


  Площадь эллипсов $A_j^1$ и~$A_j^2$ на рис.~1 условно отображает
вероятности случайных событий $A_j^1$ и~$A_j^2$ (множества
реализаций~$\omega$, принадлежащих этим событиям). Для отдельных
реализаций и сформированных из них случайных событий вероятности могут
быть определены по~(\ref{e9-kuz}). Но для оценки близости объектов
необходимо рассматривать объект, представляющий собой объединение
исходных. Множество его реализаций получается объединением
множеств~(\ref{e8-kuz}) $\Omega\hm=
(\Omega_1\hm+\Omega_2)$, и количество элементов в объединенном объекте
равно сумме $N\hm = N_1 \hm+ N_2$. Поэтому вероятности комбинированных
случайных событий, показанных на рис.~1, определяются следующим образом:
  \begin{align*}
  p_i(\omega_i^l)& =\fr{n(\omega_i^l)}{N_1+N_2}\,;\\
  P_j^l(A_j^l) &=\sum\limits_{\omega_i^l\in A_j^l} p_i(\omega_i^l)\,;\enskip l=1,2\,,
%  \label{e12-kuz}
\end{align*}
где $n(\omega_i^l)$~--- количество реализаций~$\omega_i^l$ в множестве
реализаций~$\Omega^l$.

  Для оценки близости могут быть использованы комбинированные случайные
события, определенные для объединенного объекта $\Omega\hm=
(\Omega_1+\Omega_2)$. Они получаются (см.\ рис.~1) композицией случайных
событий первого~$A_j^1$, $j\hm=1, 2, \ldots , m$, и второго~$A_j^2$, $j\hm=1,
2, \ldots , m$, объектов с помощью отмеченных выше операций для всех
компонентов системы~$\aleph$. Это следующие комбинированные случайные
события:
  \begin{enumerate}[(1)]
\item сумма (объединение) случайных событий $A_j^1$ и $A_j^2$, которая на
рис.~1 представляется пло\-щадью обоих эллипсов и определя\-ет\-ся в виде
\renewcommand{\theequation}{\arabic{equation}a}
\begin{equation}
\hspace*{-3mm}A_j^{1+2} =A_j^1+A_j^2 =\{\omega\in \Omega \vert \omega\in [A_j^1+A_j^2]\}\,;
\label{e13a-kuz}
\end{equation}
\item произведение (пересечение) случайных событий~$A_j^1$ и~$A_j^2$,
включающее реализации, принадлежащие одновременно $A_j^1$
и~$A_j^2$, и опреде\-ля\-емое в виде
\setcounter{equation}{10}
\renewcommand{\theequation}{\arabic{equation}{б}}
\begin{equation}
\hspace*{-8mm}A_j^{1\&2} =A_j^1\&A_j^2 =\{ \omega\in\Omega \vert\omega \in A_j^1\ \&\
\omega \in A_j^2\}\,,
\label{e13b-kuz}
\end{equation}
на рис.~1 ему соответствует площадь с вертикальной штриховкой;
\item разность случайных событий $A_j^1$ и~$A_j^2$ (событие, включающее
элементы $\omega\hm\in \Omega$, принадлежащие $A_j^1$ и не
принадлежащие~$A_j^2$), которая определяется в виде
\setcounter{equation}{10}
\renewcommand{\theequation}{\arabic{equation}в}
\begin{equation}
\hspace*{-8mm}A_j^{1-2} =A_j^1-A_j^2 =\{ \omega\in \Omega \vert \omega \in
A_j^1\ \&\ \omega\not\in A_j^2\}\,,
\label{e13c-kuz}
\end{equation}
на рис.~1 ей соответствует площадь с горизонтальной штриховкой;
\item разность случайных событий~$A_j^2$ и~$A_j^1$, отражаемая на
рис.~1 областью без штриховки и определяемая в виде
\setcounter{equation}{10}
\renewcommand{\theequation}{\arabic{equation}г}
\begin{equation}
\hspace*{-8mm}A_j^{2-1} =A_j^2 -A_j^1 = \{ \omega \in \Omega \vert \omega\in A_j^2\ \&\
\omega\not\in A_j^1\}\,.
\label{e13d-kuz}
\end{equation}
  \end{enumerate}

  \setcounter{equation}{11}

  События~(11) позволяют отразить уровень совпадения или различия
множеств реализаций, входящих в однотипные случайные события системы
$\aleph\hm=\{A_1, A_2, \ldots , A_m\}$, по которой раскладываются образы
сопоставляемых объектов. В~зависимости от сущности объекта его реализации
представляют собой слова естественного или искусственного (формального)
языка, компоненты структурированных или неструктурированных графических
объектов.

  Нетрудно видеть, что пересечение~(\ref{e13b-kuz}) отражает множество
общих для обоих объектов реализаций $j$-го типа, которое и определяет
степень близости объектов: чем больше значение~(\ref{e13b-kuz}), тем больше
степень их подобия. Наоборот, разности~(\ref{e13c-kuz}), (\ref{e13d-kuz}),
независимо от знака, отражают отличие объектов, которое возрастает с
увеличением разности.

  Сумма (\ref{e13a-kuz}) представляет объединение реализаций (элементарных
событий) обоих объектов. Сопоставляемые объекты известны в виде
множеств~$\Omega_1$ и~$\Omega_2$ элементарных событий, так что их
объединение представляет общее множество элементарных событий
  \begin{multline}
  \Omega = (\Omega_1+\Omega_2) ={}\\
  {}=(\omega_1^1, \omega_2^1, \ldots
, \omega^1_{N_1}, \omega_1^2, \omega_2^2,\ldots ,
\omega^2_{N_2})\,.
  \label{e14-kuz}
  \end{multline}

  На множестве~$\Omega$ могут быть определены вероятности всех типов
комбинированных случайных событий~(11) на всех компонентах алгебры
$\aleph \hm=\{A_1, A_2, \ldots , A_m\}$ в виде суммы вероятностей $p(\omega_i)$
реализаций~$\omega_i^1$ и~$\omega_i^2$:
  \begin{multline*}
  P(A_j^V) =\sum\limits_{\omega_i\in A_j^V} p(\omega_i)
=\sum\limits_{\omega_i\in A_j^V}\fr{ n(\omega_i)}{N_1+N_2}\,,\\ j=1,2,\ldots ,
m\,,
%  \label{e15-kuz}
  \end{multline*}
где $n(\omega_i)$~--- количество исходов $\omega_i$ в множестве~$\Omega$,
соответствующих указанной верхним индексом~$V$ операции из набора~(11).

  Количество информации в случайных событиях определяется
энтропией~(\ref{e11-kuz}), которая может быть вычислена по распределению
вероятностей событий любого типа из~(11) по системе~$\aleph$. Собственно
энтропия является абстрактной величиной и не может характеризовать степень
близости объектов. Но если использовать отношение энтропий,
характеризующих количество информации в комбинированных случайных
событиях, определенных на множестве~(\ref{e14-kuz}), то можно получить
меры бли\-зости, достаточно адекватные задаче оценки подобия объектов.

  Например, система случайных событий~(\ref{e13b-kuz}) отражает объем
общей для сопоставляемых объектов информации, который количественно
может оцениваться энтропией, вычисляемой по вероятностям
системы~(\ref{e13b-kuz}):

\noindent
  \begin{equation}
  H(1\&2) =-\sum\limits_{j=1}^m P(A_j^{1\&2}) \ln P(A_j^{1\&2})\,.
  \label{e16-kuz}
  \end{equation}

  Информация, отличающая объекты~1 и~2, отражается случайными
событиями~(\ref{e13c-kuz}). Количественно объем этой информации
определяется энтропией, вычисляемой по вероятностям
  событий~(\ref{e13c-kuz}):

  \noindent
  \begin{equation}
  H(1-2) =-\sum\limits_{j=1}^m P(A_j^{1-2}) \ln P(A_j^{1-2})\,.
  \label{e17-kuz}
  \end{equation}

  Меры степени отличия объектов могут содержательно соответствовать
интуитивным представлениям о близости как о расстоянии между объектами.
Отношение количества различающей объекты информации~(\ref{e17-kuz}) к
количеству информации, общей для обоих объектов~(\ref{e16-kuz}),
представляет одну из таких безразмерных величин

\noindent
  \begin{equation}
  \rho1 =\fr{ H(1-2) }{H(1\&2)}\,.
  \label{e18-kuz}
  \end{equation}

  Для оценки диапазона изменения величины~(\ref{e18-kuz}) можно
рассмотреть два предельных варианта: (1)~объекты идентичны и (2)~объекты не
имеют общих элементов. Значение~(\ref{e18-kuz}) в первом варианте
получается из условий идентичности множеств элементарных событий,
образующих объекты: $\Omega_1\hm= \Omega_2\hm\to N_2\hm=N_1$ и
$(\omega_1, \omega_2, \ldots , \omega_{N_1})\hm=(\omega_1, \omega_2, \ldots ,
\omega_{N_2})$. Отсюда следует совпадение определенных на~$\Omega_1$
и~$\Omega_2$ случайных событий~$A_j^1$, $A_j^2$, $j\hm=1, 2, \ldots , m$, и
равенство распределений ве-\linebreak\vspace*{-12pt}

\pagebreak

\noindent
роятностей $P_1(A_j^1)\hm= P_2(A_j^2)$. Поэтому
разность вероятностей равна нулю: $P(A_j^{1-2})\hm= P(A_j^1\hm-
A_j^2)\hm=P(0)$, $j \hm= 1,2, \ldots , m$. В~теории информации принято
соглашение $P(0) \ln P(0) \hm= 0$; следовательно, $H(1\mbox{--}2)\hm=0$.

  Сопоставляемые объекты предполагаются независимыми. Пересечение
множеств $A_j^1\hm= A_j^2$ равно $A_j^{1\&2} \hm= A_j^1\&A_j^2\hm=
2A_j^1\hm= 2A_j^2$ и $P(A_j^{1\&2})\hm= P(A_j^1)\times P(A_j^2)$, $j\hm=1,
2, \ldots , m$; следовательно, $H(1\&2)_j \hm= H(A_j^1)\hm+ H(A_j^2)$ и общая
энтропия~(\ref{e16-kuz}) будет отражать количество информации в обоих
объектах $H(1\&2)$, отличное от нуля. Поэтому в первом варианте
идентичности объектов значение~(\ref{e18-kuz}) $\rho1\hm= 0 / H(1\&2) \hm=
0$.

  Второй вариант, когда объекты не имеют общих элементов, дает разность,
равную содержанию уменьшаемого, а пересечение~--- равное нулю. Поэтому
получается $\rho1\hm=\infty$. Так что мера~(\ref{e18-kuz}) изменяется от нуля
для совпадающих объектов до бес\-ко\-неч\-ности для объектов, не имеющих общих
элементов, что соответствует представлениям о расстоянии, как мере близости.

  Могут быть использованы и другие, близкие по смыслу~$\rho1$, меры.
Например, в числителе~(\ref{e18-kuz}) энтропию, отражающую вероятности
событий~(\ref{e13c-kuz}), можно заменить энтропией, отражающей общее
количество информации в объектах, вычисляемой по вероятностям
событий~(\ref{e13a-kuz}). В~этом случае мера получается в виде
  \begin{equation}
  \rho2 =\fr{H(1+2)}{H(1\&2)}\,,
  \label{e19-kuz}
  \end{equation}
где $H(1+2)$ вычисляется по~(\ref{e17-kuz}) заменой событий~(\ref{e13c-kuz})
событиями~(\ref{e13a-kuz}).

  Мера~(\ref{e19-kuz}) также может интерпретироваться расстоянием между
объектами, которое для идентичных объектов будет равно единице, а для
абсолютно разных~--- бесконечности. Для придания содержательного
соответствия технологии оценки близости объектов существу объектов и
решаемым задачам меры подобия типа~(\ref{e18-kuz}) или~(\ref{e19-kuz}) могут
градуироваться в соответствующих единицах. Разработанная технология
позволяет дифференцировать информационные объекты по системе случайных
событий~--- алгебре~$\aleph$, отражающей их семантическую значимость.

  При разработке своего подхода к оценке бли\-зости информационных
объектов автор сознательно сделал упор на использование табличного их\linebreak
представления по следующей причине. Информационные объекты разного типа
являются композициями некоторых атомарных (неделимых при исследова\-нии)
элементов. На этих атомарных элементах конструируется алгебра~$\aleph$
вероятностной модели~(1). После определения алгебры информационный
объект раскладывается по системе случайных событий и представляется
таблицей, столбцы которой именуются случайными событиями.

  В реляционных базах данных таблица именуется отношением, а столбцы~---
атрибутами. Дело, конечно, не в названиях, а в том, что табличное
представление информационного объекта превращает его в обычное отношение
структуры данных. Для работы с отношением (в данном контексте~--- с
пред\-став\-ле\-ни\-ем вероятностного образа объекта) могут быть использованы
операции реляционной ал\-геб\-ры, которые позволяют из исходных отношений
конструировать и автоматически формировать новые отношения, наращивая
уровень их сложности композицией предшествующих атрибутов. Поэтому
табличное представление образов информационных объектов открывает
возможности использовать реляционную алгебру для автоматизации процедур
их исследования и детализации представления объектов.

  Например, синтаксис определяет построение из слов~--- атомарных
компонентов языка~--- различных конструкций, несущих определенную
семантическую нагрузку. Поэтому, начиная с алгебры, представленной
атомарными компонентами, могут вводиться композиционные конструкции
атомарных компонентов с постепенным наращиванием сложности из
компонентов предшествующих уровней. Выполняться все эти операции могут
стандартными средствами реляционных баз данных.

  На основе математического аппарата реляционных баз данных может быть
реализована автоматизированная рекурсивная процедура формирования
структуры вероятностных образов (алгебры) сравниваемых объектов.
Начальное приближение $\aleph (0)$ системы случайных
  событий~(\ref{e8-kuz}) задается на уровне атомарных событий, формируются
вероятностные модели $M_1(0)$ и $M_2(0)$~(\ref{e10-kuz}), определяется мера
их близости. Затем на основе~$\aleph (0)$ начального приближения,
свойств~(\ref{e2-kuz}) алгебры, синтезируются новые случайные события и их
атрибуты~--- композиции атрибутов из $\aleph (0)$. В~результате получается
алгебра $\aleph (1)$, по атрибутам которой формируются новые отношения
(таблицы) $M_1(1)$ и $M_2(1)$, определяется мера близости объектов и
сравнивается с полученной на $M_1(0)$ и~$M_2(0)$. На основании сравнения
значений меры выбирается продолжение рекурсии или прекращение процесса
уточнения оценки близости объектов.

  Усиление дифференциации семантической значимости отдельных
компонентов алгебры~$\aleph$ в конкретных задачах достигается введением
весовых коэффици\-ен\-тов. Для настройки технологии на конкретные объекты и
задачи может применяться детализация алгебры, адаптация весов и
градуировки меры подобия объектов на основании эмпирической информации.
Эти возможности позволяют создавать инструменты эффективной оценки
близости содержательной сущности информационных объектов.

\section{Иллюстрация применения технологии }

  Тестирование разработанной технологии оценки близости информационных
объектов было осуществлено на текстовых и графических объектах.\linebreak Результаты
ее применения изложены в ряде работ. В~\cite{8-kuz} приведены результаты
экспериментальной проверки возможности применения раз\-ра\-ботанной
технологии для автоматизированной \mbox{оценки} знаний студентов. В~штатном
режиме контроля знаний группа из 22~студентов написала изложение на
английском языке. Изложения были проверены и оценены по 100-балль\-ной
шкале преподавателем по стандартной методике. Затем тексты работ студентов
и исходный текст, прочитанный преподавателем, были введены в систему, в
которой работы студентов представлялись в виде об\-ра\-зов-ко\-пий $M_1(S)$,
$S\hm=1, 2, \ldots , 22$, а исходный текст принимался за эталон~$M_2$.

     Близость ответов эталону оценивалась по изложенной выше методике.
Для представления объектов в эксперименте использовалась сле\-ду\-ющая
система событий~(\ref{e8-kuz}): $A_1 = \mbox{существительное}$; $A_2 =
\mbox{гла\-гол}$; $A_3 = \mbox{прилагательное}$; $A_4 = \mbox{наречие}$;
$A_5 = \mbox{числительное}$; $A_6 = \mbox{неопределенное слово}$.

     Количество взаимной информации $H^{\mathrm{Э}\&\mathrm{О}}$ было
вычислено по~(\ref{e16-kuz}) для всех ответов. Для придания абстрактной
энтропии содержательного смыс\-ла она с использованием оценок,
выставленных преподавателем, была проградуирована в единицах 100-балль\-ной
системы оценок, принятой в вузе. Параметры модели градуировки
определялись методом наименьших квадратов в двух вариантах: первый в виде
$y\hm=a \hm+ b H^{\mathrm{Э}\&\mathrm{О}}$ и второй в виде $y_1 \hm= a_0 +
\sum\limits_i a_i H_i^{\mathrm{Э}\&\mathrm{О}}$, где $H_i^{\mathrm{Э}\&\mathrm{О}}$~---
энтропии случайных событий~$A_i^{\mathrm{Э}\&\mathrm{О}}$, $i \hm= 1, 2, \ldots ,
6$. Фактически во втором варианте параметры~$a_i$, $i \hm =1, 2, \ldots , 6$,
отражают различный <<вклад>> семантических компонентов в соответствие
между эталоном и ответом и иллюстрируют возможности адаптации
градуировки к содержательным особенностям проверяемых дисциплин, к
методикам оценки и~т.\,п. Более подробно детали структуризации текстов,
определения взаимной информации и методики ее градуировки изложены
в~\cite{8-kuz}.

\noindent
\begin{center}  %fig2
 \mbox{%
 \epsfxsize=80.161mm
 \epsfbox{kuz-2.eps}
 }
  \end{center}
%  \vspace*{6pt}
{{\figurename~2}\ \ \small{Сопоставление оценок преподавателя и оценок системы:
\textit{1}~--- при градуировке без взвешивания частей речи; \textit{2}~--- при взвешивании
частей речи}}

\vspace*{12pt}

\addtocounter{figure}{1}

     На рис.~2 приводится графическая иллюстрация результата
     из~\cite{8-kuz}, отражающая уравнения градуировки и отличие оценки,
определяемой автоматически по близости ответа эталону, от оценки,
выставленной преподавателем.


     Среднеквадратичное отклонение оценок, выставленных преподавателем,
от оценок по количеству информации~$y$ при градуировке по первому варианту
составляет 11,04~балла, коэффициент корреляции равен~0,847, при
градуировке по второму варианту уклонение составило 6,324~балла, а
множественный коэффициент корреляции достиг~0,955.

  При детальном анализе выяснилось, что две оценки, выставленные
преподавателем и существенно выпадавшие из общего ряда, были не в полной
мере адекватны содержанию изложений. Исключение двух этих точек и расчет
по оставшимся 20~точкам принципиально изменяет результат:
среднеквадратичная ошибка оценки~$y$ становится равной 1,593~балла, а
оценки~$y_1$~--- 1,248~балла.

  Регрессия, полученная для этого эмпирического материала, следуя
  век\-тор\-но-про\-стран\-ст\-вен\-ной модели текста, предложенной
Г.~Солтоном~[2], для всех 22~изложений дает среднеквадратичное отклонение
прогноза оценки от оценки преподавателя 13,27~балла. После удаления из
массива двух выпадавших точек ошибка составила 2,155~балла, т.\,е.\ почти в
два раза выше, чем при использовании информационной модели 1, 248~балла.

\begin{table*}\small
\begin{center}
\Caption{Алгебра событий и характеристики сопоставления схем}
\vspace*{2ex}

%\tabcolsep=1pt
\begin{tabular}{|c|c|c|c|}
\hline
\tabcolsep=0pt\begin{tabular}{c}Случайные\\ события\end{tabular} & Перечень& Количество &Вероятность\\
\hline
\multicolumn{1}{|l|}{$A_1$~--- элементы}& R4, L1, \ldots & $n_1$ & $n_1/N$\\
\multicolumn{1}{|l|}{$A_2$~---ветви}& L1C1, L2C2,\ldots & $n_2$ & $n_2/N$\\
\multicolumn{1}{|l|}{$A_3$~--- узлы} & T1BC1R4,\ldots & $N_3$ & $N_3/N$\\
\hline
\end{tabular}
\end{center}
\end{table*}

  Разработанная технология формального сопоставления информационного
содержания текстов и синтеза количественной меры семантической\linebreak
 близости
слов языка позволяет автоматизировать исследования проблем в об\-ласти
языкознания. Например, семантические отношения между словами (синонимы,
антонимы, паронимы, гипонимы и~т.\,п.)\ определяются в настоящее время
субъективно представителями разных школ. Субъективные представления не
формализуются, не могут быть упорядочены или сопоставлены.

  Между тем понятно, что в основе определения семантических отношений
лежат различные композиции множеств оттенков значений со\-по\-став\-ля\-емых
слов~(11), которые представлены в существующих словарях. Изложенный
подход позволяет формализовать и, следовательно, автоматизировать
процедуры количественной оценки меры <<информационного расстояния>>
между словами. На этой основе могут быть введены объективные,
количественные меры синонимичности, антонимичности и~т.\,п. Интересные
результаты в этой области могут быть получены с использованием нечетких
отношений вместо четких, показанных на рис.~1.

  В работе~\cite{14-kuz} изложена технология разработки универсального
метрического тезауруса языка на примере русского языка. Технология
базируется на формировании для каждого слова языка его содержательного
образа в виде вероятностной модели. Существуют специализированные
словари, ориентированные на определенные сферы деятельности и знаний.
Обычно эти словари составляются экспертами в области языкознания на
основании их субъективных представлений о семантике слов.

  Технология представления текстов вероятностными моделями позволяет
автоматизировать со\-здание тезауруса языка в виде обобщенных
со\-держательных образов отдельных слов языка. Обобщен\-ный образ
формируется в виде структурированной на выбранной алгебре~$\aleph$ суммы
отражающих смысл слова словарных статей из всех доступных словарей.
Тезаурус формируется автоматически по введенным электронным версиям
имеющихся словарей и представляет словарь, в котором с каждым словом
связан его максимально полный содержательный образ.

  Разработанный универсальный метрический тезаурус позволяет формально,
опираясь на всю имеющуюся информацию о значении слов, и в этом смысле
объективно, решать проблему синонимов при оценке близости текстов.
Метрическая оценка семантической близости слов производится автоматически
по изложенной технологии со\-по\-став\-ле\-ни\-ем их образов. Мерой близости
является расстояние, определяемое в виде~(\ref{e18-kuz}) или~(\ref{e19-kuz})
по обобщенным образам слов. При появлении новых версий электронных
словарей они могут вводиться в систему и автоматически ассимилироваться ею.
Технология создания тезауруса с использованием образов слов в виде
вероятностных моделей~(1) инвариантна по отношению к структурированным
языкам и может быть использована в любом из них. Универсальный
электронный тезаурус~\cite{14-kuz} является мощным инструментом для
исследований в сфере филологии и языкознания.

  В работе~\cite{4-kuz} показана возможность применения информационной
технологии и вероятностной модели~(1) для оценки близости схем. Схемы
используются в процессе изучения многих дисциплин технического
направления и поэтому широко представлены в обучающих системах. Для
примера использовались электрические схемы, множества элементарных
событий при представлении которых формируются элементами,
регламентированными стандартом~\cite{7-kuz}.

  В табл.~1 показана система случайных событий, составляющих алгебру в
этом случае, и количественные характеристики схем, при этом~$n_i$, $i\hm= 1,
2, 3$,~--- количество компонентов (реализаций) $i$-го типа, $N\hm=n_1 \hm+
n_2\hm + n_3$~--- общее количество компонентов в схеме.



  Проверка осуществлялась на схеме, содержащей компоненты $n_1\hm=41$,
$n_2\hm=34$, $n_3\hm=18$, $N\hm=93$. Ошибки экзаменуемых моделировались
устранением компонентов из схем-от\-ве\-тов. На рис.~3 показаны результаты
увеличения информационного расстояния между эталоном и ответами по мере
нарастания их ошибочности.



  Полученные результаты показали~\cite{4-kuz} возможность синтеза
автоматизированных процедур для оценки уровня соответствия схем их
эталонному образу. Такие процедуры позволяют, в частности, разрабатывать в
обучающих системах модули проверки ответов обучаемых в этой области.

  \begin{table*}[b]\small %tabl2
%  \vspace*{-3pt}
  \begin{center}
  \Caption{Результаты оценки близости объектов разными методами}
  \vspace*{2ex}

\tabcolsep=4pt
  \begin{tabular}{|c|c|p{29mm}|p{29mm}|p{29mm}|l|}
  \hline
&\multicolumn{1}{|c|}{\raisebox{-6pt}[0pt][0pt]{Метод}}&\multicolumn{4}{c|}{Характеристики для образов}\\
\cline{3-6}
&&\multicolumn{1}{c|}{Образ 1}&\multicolumn{1}{c|}{Образ 2}&\multicolumn{1}{c|}{Образ 3}&Совместная\\
\cline{2-6}
&&&&&\\[-15pt]
\multicolumn{1}{|c|}{\raisebox{-48pt}[0pt][0pt]{Тип}}&
\multicolumn{1}{|c|}
{\raisebox{-48pt}[0pt][0pt]{\tabcolsep=0pt\begin{tabular}{c}Графическое\\
представление\end{tabular}}}&
\begin{center}
 \mbox{%
 \epsfxsize=28mm
 \epsfbox{kuz-t-1.eps}
 }
 \end{center}&
 \begin{center}
 \mbox{%
 \epsfxsize=28mm
 \epsfbox{kuz-t-2.eps}
 }
 \end{center}&
 \begin{center}
 \mbox{%
 \epsfxsize=28mm
 \epsfbox{kuz-t-3.eps}
 }
 \end{center}&\multicolumn{1}{|c|}{\raisebox{-48pt}[0pt][0pt]{---}}\\
 &&&&&\\[-15pt]
\hline
\multicolumn{1}{|l|}{\raisebox{-12pt}[0pt][0pt]
{\tabcolsep=0pt\begin{tabular}{l}Класси-\\ ческие\end{tabular}}}&
\tabcolsep=0pt\begin{tabular}{c}Разбиение\\ $RGB$-осей\end{tabular}&
\multicolumn{1}{c|}{---}&\multicolumn{1}{c|}{---}&\multicolumn{1}{c|}{---}&
\tabcolsep=0pt\begin{tabular}{c}$R_{12} = 0{,}897$\\
$R_{13} = 0{,}735$\end{tabular}\\
\cline{2-6}
&\tabcolsep=0pt\begin{tabular}{c}Разбиение\\ $RGB$-пространства\end{tabular}&
\multicolumn{1}{c|}{---}&\multicolumn{1}{c|}{---}&\multicolumn{1}{c|}{---}&
\tabcolsep=0pt\begin{tabular}{l}$R_{12} =
0{,}7$\\
$R_{13} = 0{,}746$\end{tabular}\\
\hline
\multicolumn{1}{|l|}{\raisebox{-32pt}[0pt][0pt]
{\tabcolsep=0pt\begin{tabular}{l}Разрабо-\\ танные\end{tabular}}}&\tabcolsep=0pt\begin{tabular}{c}Модифицированный\\ метод\end{tabular}&
\multicolumn{1}{c|}{---}&\multicolumn{1}{c|}{---}&\multicolumn{1}{c|}{---}
&\tabcolsep=0pt\begin{tabular}{l}$R_{12} = 0{,}324$\\
$R_{13} = 0{,}913$\end{tabular}\\
\cline{2-6}
&\tabcolsep=0pt\begin{tabular}{c}Интегральный\\ метод\end{tabular}&
\multicolumn{1}{c|}{$E_1 = 461{,}811$}&\multicolumn{1}{c|}{$E_2 =
463{,}793$}&\multicolumn{1}{c|}{$E_3 = 462{,}152$}&
\tabcolsep=0pt\begin{tabular}{l}$R_{12}= 0{,}055$\\
$R_{13} = 0{,}922$\end{tabular}\\
\cline{2-6}
&\tabcolsep=0pt\begin{tabular}{c}Дифференциальный\\ метод\end{tabular}&\multicolumn{1}{c|}{$E_1 =
1134{,}189$}&\multicolumn{1}{c|}{$E_2 = 1280{,}136$}&\multicolumn{1}{c|}{$E_3 =
1234{,}701$}&\tabcolsep=0pt\begin{tabular}{l}$R_{12} = 0{,}069$\\
$R_{13} = 0{,}938$\end{tabular}\\
\cline{2-6}
&&&&&\\[-9pt]
&\tabcolsep=0pt\begin{tabular}{c}Энтропийное\\ расстояние\end{tabular}&\multicolumn{1}{c|}{\tabcolsep=0pt\begin{tabular}{c}$H_{12}(\cup)
= 971{,}602$\\
$H_{13}(\cup) = 1227{,}317$\end{tabular}}&\multicolumn{1}{c|}{$H_{12}(\cap) =
110{,}171$}&\multicolumn{1}{c|}{$H_{13}(\cap) =
740{,}361$}&\tabcolsep=0pt\begin{tabular}{l}$\rho2^{12}= 4{,}41$\\
$\rho2^{13}= 0{,}829$\end{tabular}\\
\hline
\end{tabular}
\end{center}
\end{table*}

  Применение разработанной технологии для оценки близости
неструктурированных графических объектов~\cite{5-kuz} показало
возможность на ее\linebreak\vspace*{-12pt}

\pagebreak

\noindent
\begin{center}  %fig2
 \mbox{%
 \epsfxsize=76.837mm
 \epsfbox{kuz-3.eps}
 }
  \end{center}
%  \vspace*{6pt}
{{\figurename~3}\ \ \small{График роста информационного расстояния при последовательном увеличении
отклонений схем-от\-ве\-тов от эталона}}

\vspace*{12pt}

\addtocounter{figure}{1}


\noindent
 основе существенного повышения уровня до\-сто\-вер\-ности
оценки. Этот вывод следует непосредственно из данных воспроизводимой
табл.~2 из~\cite{5-kuz}, в которой показаны сопоставляемые объекты и меры их
близости, полученные по разным технологиям. Объекты выбраны так, чтобы их
подобие и отличие были очевидны: уровень подобия объектов~1 и~3 намного
выше, чем объектов~1 и~2, или 2 и~3.

  В работе~\cite{5-kuz} приведены результаты исследования различных мер
для оценки степени близости объектов. Для исследования были реализованы
стандартные, используемые на практике методы и оригинальные, опирающиеся
на сопоставление количеств информации в объектах.




  Для стандартного метода цветовых гистограмм~[9--11], в
основу которого положена $RGB$-мо\-дель, были реализованы алгоритмы,
ис\-поль\-зу\-ющие модификации разбиения $RGB$-про\-стран\-ст\-ва на
подпространства и $RGB$-осей~--- на интервалы. Цветовые модели объектов,
как это принято в биометрических системах, представлялись в виде цветовых
гистограмм. Оценка близости объектов производилась по корреляции их
цветовых гистограмм.

  Затем были разработаны алгоритмы, реализу\-ющие оригинальную
технологию представления объектов в виде вероятностных моделей,
син\-те\-зи\-ру\-емых разбиением пятимерного $RGBXY$-про\-стран\-ст\-ва на
сис\-те\-му непересека\-ющих\-ся подпространств, образующих алгебру
$\aleph\hm=\{A_1, A_2, \ldots , A_m\}$.

  На алгебре $\aleph\hm=\{A_1, A_2, \ldots , A_m\}$ определялись
комбинированные события~(\ref{e13a-kuz}) и~(\ref{e13b-kuz}) и
соответствующие им распределения вероятностей для объектов~1, 2, 3,
показанных в табл.~2. На распределениях вероятностей на системе введенных
классов~$\aleph$ вычислялись соответствующие энтропии.
В~исследовании~\cite{5-kuz} близость объектов в информационном масштабе
оценивалась энтропийным расстоянием между ними, определяемым в
виде~(\ref{e19-kuz}), которое изменяется от~1 для совпадающих объектов
до~$\infty$ для объектов, не имеющих общих элементов.

  Оценка близости графических объектов, показанных в табл.~2, была
произведена по су\-ще\-ст\-ву\-ющим и разработанным технологиям. При
использовании стандартного метода цветовых гистограмм оценка
осуществлялась коэффициентом корреляции цветовых гистограмм объектов,
которые обозначены~$R_{ij}$, где $i, j$~--- номера объектов в первой строке
табл.~2.  %Полученные результаты приведены в табл.~2.

  В алгоритмах биометрической идентификации~\cite{9-kuz} пороговое
значение степени соответствия устанавливается обычно на уровне~0,65. Из
табл.~2 следует, что значения коэффициента подобия, полученные по
классическим алгоритмам разбиения $RGB$-осей на интервалы и
  $RGB$-про\-странств на прямоугольные параллелепипеды, превышают
пороговое значение для пар~1,~2 и~1,~3. Это свидетельствует об идентичности
всех трех объектов. Оценки близости всех трех объектов практически
совпадают, хотя отличие объекта~2 от~1 и~3 очевидно.

  Введение в оценку близости объектов системы координат
(модифицированный метод $RGBXY$) существенно изменяет результат.
Коэффициент подобия образов~1 и~2 при использовании модифицированного
алгоритма принимает значение $R \hm= 0{,}324$, которое более чем в 2~раза
ниже коэффициентов~0,897 и~0,7, полученных с использованием классических
алгоритмов. Одновременно коэффициент подобия объектов~1 и~3 увеличился.
Сопоставление <<модифицированных про\-стран\-ст\-вен\-но-цве\-то\-вых
гистограмм>> показывает, что объекты~1 и~3 примерно в 3~раза <<ближе>>,
чем объекты~1 и~2.

  Информационный метод дает адекватную сопоставляемым изображениям
меру соответствия в виде расстояния между ними. Расстояние $\rho2$
пропорционально различию объектов и тем больше, чем меньше степень их
совпадения. Расстояние $\rho2_{1,2}\hm = 4{,}41$ между объектами~1 и~2 более
чем в пять раз превышает расстояние $\rho2_{1,3}\hm = 0{,}829$ между
объектами~1 и~3. Можно видеть, что реально следующая из рисунков,
включенных в табл.~2, близость объектов~1 и~3 отражается близостью
значения $\rho2_{1,3}\hm = 0{,}829$ к минимально возможному, равному~1, для
совпадающих объектов.

  Градуировка шкалы меры $\rho2$ (и иных, полу\-ча\-емых информационным
методом) может быть осуществлена на основании реального эмпирического
материала. Технология вычисления оценки инвариантна по отношению к
разбиению объектов на подмножества. Разбиение объектов может
производиться с переменным шагом по осям координат, что позволяет
акцентировать внимание на областях с высокой плотностью информации
использованием мелкого шага. Процедура оценки близости объектов может
быть итеративной с использованием на каждом следующем шаге более
детальных образов объектов, получаемых детализацией алгебры~$\aleph$,
использованной на предыдущем этапе оценки. И~сетка разбиения, и веса
областей могут автоматически адаптироваться в процессе функционирования
сис\-темы.

\section{Заключение }

  Формальная оценка семантического подобия информационных объектов
может базироваться\linebreak на количественной мере сопоставления их содержания.
Случайность элементарных компонентов\linebreak описания информационных объектов
и произвольность формы представления объектов в информационных
технологиях выдвигает актуальную задачу разработки формальной технологии
и меры оценки уровня их семантического подобия. Объекты представляются
множествами случайных реализаций набора элементарных изобразительных
компонентов.

  Элементарные компоненты различны для разных форм представления, но
являются общими для сопоставляемых объектов. Единый подход к содержанию
информационных объектов как множеству\linebreak реализаций случайных величин
позволяет разра\-ботать единую технологию оценки их бли\-зости. Различия
представления объектов разного типа сводятся лишь к разным наборам
элементарных изобразительных компонентов.

  Информационные объекты разного типа формализуются вероятностными
моделями, в которых реализации группируются в случайные события в
соответствии с их информационной ценностью. Количественными
характеристиками разложения информационных объектов по системам
случайных событий являются распределения вероятностей. Распределения
вероятностей определяют энтропии объектов. Энтропии отражают количества
информации в сопоставляемых информационных объектах и позволяют
синтезировать количественную меру их близости, подобную метрической.

  Технология позволяет разработать универсальные эффективные процедуры
оценки подобия информа\-ционных объектов, представленных графически и
текстами на естественном или искусственном языке. Процедуры могут
использоваться в сис\-те\-мах поиска информации, оценки близости текстовых и
графических объектов, автоматизированной проверки уровня усвоения знаний.

{\small\frenchspacing
{%\baselineskip=10.8pt
\addcontentsline{toc}{section}{References}
\begin{thebibliography}{99}
\bibitem{1-kuz}
\Au{Manning Ch.\,D., Raghavan P., Sch$\ddot{\mbox{u}}$tz~H}. An introduction to
information retrieval.~--- Cambridge: University Press, 2009. 569~p.
\bibitem{2-kuz}
\Au{Salton G., Wong A., Yang~C.\,S.} A~vector space model for automatic
indexing~// Comm. ACM, 1975. Vol.~18. No.\,11. P.~613--620.
\bibitem{3-kuz}
\Au{Кузнецов Л.\,А.} Веро\-ят\-ност\-но-ста\-ти\-сти\-че\-ская оценка
адекватности информационных объектов~// Информатика и её применения,
2011. Т.~5. Вып.~4. С.~39--50.
\bibitem{4-kuz}
\Au{Кузнецов Л.\,А., Кузнецова В.\,Ф., Антонов~Д.\,И.} Оценка близости
графических объектов на примере электрических схем с помощью
информационного критерия~// Открытое и дистанционное образование, 2013.
№\,2(50). С.~35--43.
\bibitem{5-kuz}
\Au{Кузнецов Л.\,А., Бугаков Д.\,А.} Разработка меры оценки информационного
расстояния между графическими объектами~//
Ин\-фор\-ма\-ци\-он\-но-управ\-ля\-ющие сис\-те\-мы, 2013. №\,1. С.~74--79.
\bibitem{6-kuz}
\Au{Гнеденко Б.\,В.} Курс теории вероятностей.~--- 9-е изд., испр.~---
М.: ЛКИ, 2007. 448~с.
\bibitem{7-kuz}
ГОСТ 2.743-91 ЕСКД. Обозначения условные графические в схемах. Элементы
цифровой техники.~--- М.: Госстандарт, 1991. 75~с.
\bibitem{8-kuz}
\Au{Кузнецов Л.\,А., Кузнецова В.\,Ф.} Оценка семантической адекватности
текстов информационным методом~// Информатика и её применения, 2013.
Т.~7. Вып.~1. С.~19--29.
\bibitem{9-kuz}
\Au{Гаспарян А.\,В., Киракосян А.\,А.} Система сравнения отпечатков пальцев
по локальным признакам~// Вестник РАУ. Сер.
Фи\-зи\-ко-ма\-те\-ма\-ти\-че\-ские и естественные науки, 2006. №\,2. С.~85--91.

\bibitem{11-kuz}
\Au{Swain M.\,J., Ballard D.\,H.} Color indexing~// Int. J.~Computer Vision, 1991.
Vol.~7. No.\,1. P.~11--32.

\bibitem{10-kuz}
\Au{Sticker M., Orengo M.} Similarity of color images~// SPIE Conference
Proceedings, 1995. Vol.~2420. P.~381--392.

\bibitem{12-kuz}
\Au{Кузнецов Л.\,А., Бугаков Д.\,А.} Развитие метода сравнения и
классификации графических объектов~// Вестник компьютерных и
информационных технологий, 2013. №\,2(104). С.~11--16.
\bibitem{13-kuz}
\Au{Шеннон К.} Работы по теории информации и кибернетике.~--- М.: Изд-во
ИЛ, 1963. 833~с.
\bibitem{14-kuz}
\Au{Кузнецов Л.\,А., Кузнецова В.\,Ф., Капнин~А.\,В.} Универсальный
метрический тезаурус русского языка~// Информатика и её применения,
2013. Т.~7. Вып.~3.\linebreak С.~27--35.

\end{thebibliography}
} }

\end{multicols}

\vspace*{-9pt}

\hfill{\small\textit{Поступила в редакцию 10.12.13}}

%\newpage


\vspace*{10pt}

\hrule

\vspace*{2pt}

\hrule


\def\tit{UNIVERSAL TECHNOLOGY OF INFORMATION OBJECTS PROXIMITY ASSESSMENT}

\def\titkol{Universal technology of information objects proximity assessment}

\def\aut{L.\,A.~Kuznetsov}
\def\autkol{L.\,A.~Kuznetsov}


\titel{\tit}{\aut}{\autkol}{\titkol}

\vspace*{-9pt}

\noindent
Russian Presidential Academy of National Economy and Public Administration
(Lipetsk Branch), 3 Internatsional'naya Str., Lipetskaya oblast, Lipetsk 398050,
Russian Federation



\def\leftfootline{\small{\textbf{\thepage}
\hfill INFORMATIKA I EE PRIMENENIYA~--- INFORMATICS AND APPLICATIONS\ \ \ 2014\ \ \ volume~8\ \ \ issue\ 2}
}%
 \def\rightfootline{\small{INFORMATIKA I EE PRIMENENIYA~--- INFORMATICS AND APPLICATIONS\ \ \ 2014\ \ \ volume~8\ \ \ issue\ 2
\hfill \textbf{\thepage}}}

\vspace*{6pt}


\Abste{The paper outlines the technology used to determine the degree of similarity
of information objects, which are represented by text or graphic images. Objects are
formalized by probabilistic models. The structure of the model is set by an algebra on
a minimum set of graphic components of an object. Quantitative characteristics of
the structure of objects are the probability distributions on the algebra. The amount of
information in objects is estimated by entropy. The similarity measure of information
objects is based on entropy. The paper describes the method of estimating the
proximity of text and graphic objects. The paper provides several examples of
estimation algorithms implementation. It is shown that the developed method is more
efficient compared to the methods described in the literature. The technology used to
form images of information objects and to compare their semantic content is universal.
It is possible to adapt the technology to the meaningful characteristics of objects
being analyzed.}

  \KWE{information object; text; image; probabilistic model; semantic similarity;
entropy; measure of similarity}

\DOI{10.14357/19922264140213}

%\Ack
%\noindent


  \begin{multicols}{2}

\renewcommand{\bibname}{\protect\rmfamily References}
%\renewcommand{\bibname}{\large\protect\rm References}

{\small\frenchspacing
{%\baselineskip=10.8pt
\addcontentsline{toc}{section}{References}
\begin{thebibliography}{99}


\bibitem{1-kuz-1}
\Aue{Manning, Ch.\,D., P. Raghavan, and H.~Sch$\ddot{\mbox{u}}$tz}.
2009. \textit{An introduction to information retrieval}.
Cambridge: University Press. 569~p.
\bibitem{2-kuz-1}
\Aue{Salton, G., A. Wong, and C.\,S.~Yang}. 1975.
A~vector space model for automatic indexing.
\textit{Comm. ACM}.  11:613--620.
\bibitem{3-kuz-1}
\Aue{Kuznetsov, L.\,A.} 2011. Veroyatnostno-statisticheskaya
otsen\-ka adekvatnosti informatsionnykh ob"ektov
[Probabilistic and statistical evaluation of the adequacy of information objects].
\textit{Informatika i ee Primeneniya}~---
\textit{Inform. Appl.} 5(4):39--50.
\bibitem{4-kuz-1}
\Aue{Kuznetsov, L.\,A., V.\,F.~Kuznetsova, and D.\,I.~Antonov}.
2013. Otsenka blizosti graficheskikh ob"ektov na primere elektricheskikh
skhem s pomoshch'yu informatsionnogo kriteriya
[Estimation of the distance graphical objects on the example of electrical
circuits using information criterion].
\textit{Otkrytoe i Distantsionnoe Obrazovanie} [Open and Distance Education]
2:35--43.
\bibitem{5-kuz-1}
\Aue{Kuznetsov, L.\,A., and D.\,A.~Bugakov}. 2013.
Razrabotka mery otsenki informatsionnogo rasstoyaniya mezhdu graficheskimi
ob"ektami [Development of measures assessing the information distance between
graphic objects].
\textit{Informatsionno-Upravlyayushchie Sistemy}
[Information and Control Systems] 1:74--79.
\bibitem{6-kuz-1}
\Aue{Gnedenko, B.\,V.} 2007. \textit{Kurs teorii veroyatnostey}
[Course of probability theory].  Moscow: LKI Publs. 448~p.
\bibitem{7-kuz-1}
GOST 2.743-91 ESKD. Oboznacheniya uslovnye gra\-fi\-che\-skie v skhemakh.
Elementy tsifrovoy tekhniki [State Standard 2.743-91 ESKD.
Graphic symbols in schemes. Elements of digital technology].
M.: Gosstandart, 1991. 75~p.
\bibitem{8-kuz-1}
\Aue{Kuznetsov, L.\,A., and V.\,F.~Kuznetsova}. 2013. Otsen\-ka semanticheskoy
adekvatnosti tekstov informatsionnym metodom
[Evaluation of the semantic adequacy of texts by information method].
\textit{Informatika i ee Primeneniya}~--- \textit{Inform. Appl.} 7(1):19--29.
\bibitem{9-kuz-1}
\Aue{Gasparyan, A.\,V., and A.\,A.~Kirakosyan}. 2006.
Sistema sravneniya otpechatkov pal'tsev po lokal'nym priznakam
[Fingerprint comparisons on local characteristics].
\textit{Vestnik RAU. Ser. Fiziko-Matematicheskie i Estestvennye Nauki}
[Herald of RAU. Physics, Mathematics, and Natural Sciences ser.] 2:85--91.

\bibitem{11-kuz-1}
\Aue{Swain, M.\,J., and D.\,H.~Ballard}. 1991. Color indexing.
\textit{Int. J.~Computer Vision} 7(1):11--32.

\bibitem{10-kuz-1}
\Aue{Sticker, M., and M.~Orengo}. 1995. Similarity of color images.
\textit{SPIE Conference Proceedings} 2420:381--392.

\bibitem{12-kuz-1}
\Aue{Kuznetsov, L.\,A., and D.\,A.~Bugakov}. 2013. Razvitie metoda sravneniya i
klassifikatsii graficheskikh ob"ektov
[Development of the method of comparison and classification].
\textit{Vestnik Komp'yuternykh i Informatsionnykh Tekhnologiy}
[Computer and Information Bulletin Technology] 2(104):11--16.
\bibitem{13-kuz-1}
\Aue{Shennon, K.} 1948. A~mathematical theory of communication. Pt.~I, II.
\textit{Bell. Syst. Techn. J.} 27(3):379--423; 27(4):623--656.
\bibitem{14-kuz-1}
\Aue{Kuznetsov, L.\,A., V.\,F.~Kuznetsova, and A.\,V.~Kapnin}.
2013. Universal'nyy metricheskiy tezaurus russkogo yazyka
[Universal Russian language thesaurus metric].
\textit{Informatika i ee Primeneniya}~--- \textit{Inform. Appl.} 7(3):27--35.

\end{thebibliography}
} }


\end{multicols}

\vspace*{-6pt}

\hfill{\small\textit{Received December 10, 2013}}

\vspace*{-18pt}



\Contrl

\noindent
\textbf{Kuznetsov Leonid A.} (b.\ 1942)~--- Doctor of Science in technology,
professor, Honored Scientist of Russian Federation,
Head of Department, Russian Presidential Academy of National
Economy and Public Administration (Lipetsk Branch),
3 Internatsional'naya Str., Lipetskaya oblast, Lipetsk 398050,
Russian Federation; Kuznetsov.Leonid48@gmail.com




 \label{end\stat}

\renewcommand{\bibname}{\protect\rm Литература}
  %12
 \renewcommand{\figurename}{\protect\bf Figure}
\renewcommand{\tablename}{\protect\bf Table}
\renewcommand{\bibname}{\protect\rmfamily References}

\def\stat{belyev}

\def\tit{APPROXIMATION OF A~MULTIDIMENSIONAL DEPENDENCY BASED ON A LINEAR EXPANSION IN A 
DICTIONARY OF~PARAMETRIC FUNCTIONS$^*$}

\def\titkol{Approximation of a multidimensional dependency based on linear expansion in a 
dictionary of parametric functions}

\def\autkol{M.\,G.~Belyaev and E.\,V.~Burnaev}

\def\aut{M.\,G.~Belyaev$^1$ and E.\,V.~Burnaev$^2$}

\titel{\tit}{\aut}{\autkol}{\titkol}

{\renewcommand{\thefootnote}{\fnsymbol{footnote}}\footnotetext[1] {The authors are partially supported by Laboratory for Structural Methods of 
Data Analysis in Predictive Modeling, MIPT, RF government grant, ag.\ 
11.G34.31.0073; RFBR grant 13-01-00521. Results, described in this work, were 
obtained in the framework of ``COPTI-X: Surrogate Model Construction for 
Structure Approximation and Optimization'' joint project with Airbus Operations 
SAS.}}

\renewcommand{\thefootnote}{\arabic{footnote}}
\footnotetext[1]{Institute for Information Transmission Problems RAS, Moscow 
Institute of Physics and Technology, Datadvance LLC, belyaev@iitp.ru}
\footnotetext[2]{Institute for Information Transmission Problems RAS, Moscow 
Institute of Physics and Technology, Datadvance LLC, burnaev@iitp.ru}

\Abste{The problem of a multidimensional function approximation using a finite 
set of pairs ``point''\,--\,``function value at this point'' is considered. As  
a model for the function, an expansion in a dictionary containing nonlinear 
parametric functions has been used. Several subproblems should be solved when 
constructing an approximation based on such model: extraction of a validation 
sample, initialization of parameters of the functions from the dictionary, and 
tuning of these parameters. Efficient methods for solving these subproblems 
have been suggested. Efficiency of the proposed approach is demonstrated on 
some problems of engineering design.}

\KWE{nonlinear approximation; parametric dictionaries}

\vskip 14pt plus 9pt minus 6pt

      \thispagestyle{headings}

      \begin{multicols}{2}

            \label{st\stat}

\section{Introduction}

\noindent For engineering design, it is necessary to model complex physical 
phenomena. Typically used models are represented by complex systems of 
differential equations. Such systems do not have analytical solutions; so, 
computationally heavy numerical methods are used. One approach to solving 
problems of engineering design,
   actively developing in recent years, is the surrogate modeling~\cite{surrogateModeling, kuleshov2}.
In this approach, a complex physical phenomenon is described by a simplified 
(surrogate) model constructed using data mining techniques and a set of 
examples,
   representing results of a detailed physical modeling and/or real experiments.
The problem of approximation of a multidimensional function using a finite set 
of pairs ``point''\,--\,``value of the function at this point'' is one of the 
main problems to be solved in the construction of the surrogate model.  
This problem  will be considered in the following formulation:

\medskip

\noindent \textbf{Problem 1.}\ \textit{Let $f\left( {\vec x} \right) \in R^1$ 
be some continuous function on a compact set 
$
\mathrm{D} \subset R^d$ with known 
output values in a finite set of input points. The set 
 \begin{equation*}
 S_{\mathrm{learning}} = \{ {\vec x}_i, y_i \}_{i=1}^{N_{\mathrm{learning}}}\,,\ \ \ 
{\vec x}_i \in \mathrm{D}\,,\enskip 
y_i = f\left( {\vec x}_i\right)\,,
\end{equation*} 
forms the learning sample.
The approximation problem is to construct an approximation      
$\hat f\left( {\vec x} \right)$ (approximator) of the function $f\left( {\vec x} \right)$
    using the given data sample $S_{\mathrm{learning}}$ such that}
    $f\left( {\vec x} \right) \approx \hat f\left( {\vec x} \right)$
for all $\vec x\in\mathrm{D}$.


\medskip

\noindent \textbf{Remark~1.}\ In general case, the values of the function $f\left( 
{\vec x} \right)$ are known only for the finite set of input points; so, the 
proximity $f\left( {\vec x} \right) \approx \hat f\left( {\vec x} \right)$ is 
usually measured be the mean square error
$$
Q\left( S_{\mathrm{test}}, \hat f \right) = \left(\fr{1}{N_{\mathrm{test}}}\right) 
    \sum\limits_{i=1}^{N_{\mathrm{test}}} \left( y_i - \hat f \left( 
    \vec x_i\right) \right)^2
    $$ 
    calculated using an independent test data sample
      $$
      S_{\mathrm{test}} = \{ {\vec x}_i, y_i \}_{i=1}^{N_{\mathrm{test}}}\,,\enskip  
{\vec x}_i \in \mathrm{D}\,,\ 
\ y_i = f\left( {\vec x}_i\right)\,.
$$ 

The criterion $Q\left( S_{\mathrm{test}}, \hat f \right)$ 
of approximation quality makes sense if input vectors from the samples $S_{\mathrm{learning}}$ 
and $S_{\mathrm{test}}$ are generated by the same distribution and cover the design space 
$\mathrm{D}$ sufficiently densely.
The function $Q\left( S, \hat f \right)$ for some set~$S$ 
of pairs ``point''\,--\,``value of the function at this point'' is called an 
error function on the set~$S$.

\medskip

Due to requirements of surrogate modeling problems (in particular, the need to build quickly 
computable global approximation model and to work with large data samples),
   the most common method of solving approximation problems is based on Artificial Neural Networks (ANN)~\cite{elementsOfStatLearning}. 
An approximation based on the ANN model provides high-speed calculations of output predictions. The ANN model can be easily ``extended'' 
by increasing number of layers and/or their sizes as the learning sample size increases while the computational complexity of the 
approximation model construction grows only linearly.

A typical scheme of approximation construction based on the ANN model can be 
divided into two phases:
\begin{enumerate}[(1)]
\item an initialization phase (setting the initial values of the 
model parameters;  and \\[-9pt]
\item extraction of the validation sample) and a training phase (tuning the ANN 
 parameters to fit the data sample).
 \end{enumerate}
  Usually, random methods are used during the 
 initialization phase. Such methods lead to a large scatter of the approximation accuracy.
 Surrogate models are constructed to replace computationally 
 heavy objective functions (and/or constraints) in optimization problems~\cite{copti}. 
 Engineering design based on surrogate models is iterative:
 \begin{enumerate}[(1)]
 \item  find the optimum of the 
 objective function (surrogate model);\\[-9pt]
 \item  generate additional pairs
   ``point''\,--\,``value of the function at the point'' in the neighbourhood of the optimum;\\[-9pt] 
\item reconstruct the surrogate model and optimize it,\linebreak etc.
\end{enumerate}
 Therefore, 
  unpredictable significant changes in the surrogate model structure and significant variations of 
  the approximation accuracy deteriorate surrogate based optimization.

This paper investigates methods for constructing approximations based on the 
linear expansion in nonlinear functions from the parametric dictionary (ANN model 
with one hidden layer).
Here, the methods for initialization and training that reduce the average approximation 
error and its variations are suggested. Let describe the structure of the paper.

The model based on a linear expansion in a dictionary of parametric functions is described in 
subsection~2.1. The main steps of the approximation construction based on
  this model are described in subsection~2.2.
Subsection~2.3 is devoted to various subproblems that arise when 
constructing an approximation using the proposed algorithm.
In section~3, the subproblems of the initialization step
 of the approximation construction algorithm are considered. 
In subsection~3.1, a new algorithm is described for the validation sample 
extraction, such that points from the validation sample are distributed as uniformly
 as possible among the remaining points of the learning sample. 
 A~computationally efficient deterministic algorithm for the validation sample extraction 
 based on the greedy optimization of some uniformity criterion.
In subsection~3.3, a new algorithm is described for the initialization of functions 
from the parametric dictionary. 
In section~4, a new training algorithm is considered. 
In subsection~4.1, a special form of the error function that takes 
into account the structure of the approximation error dependence on different groups 
of parameters and includes adaptive regularization is suggested.
In subsection~4.2, a new method for the regularization 
parameter selection is described.
Both sections~\ref{stage1} and~\ref{stage2} contain results of the computational experiments on artificial functions.
Experimental results for some engineering design problems are described 
in section~\ref{experiments}.
{\looseness=1

}

\section{Algorithm for~Approximation Construction}

\subsection{Model based on a linear expansion in~a~dictionary of~parametric functions}
\label{model}

\noindent
The approximation $\hat f\left( {\vec x} \right)$ is modeled
by the linear expansion in a 
dictionary of parametric functions, i.\,e.,
$$ 
\hat f\left({\vec x}\right) = \sum\limits_{j = 1}^p \alpha_j 
\psi_j\left( {\vec \theta_j},  {\vec x} \right) + \alpha_0\,.
$$
Let rewrite this equality in matrix form
$$
\hat f \left( {\vec x} \right) = {\vec \psi} \left( {\Theta}, {\vec x} \right) {\vec \alpha}
$$ 
where the vectors are denoted by caps with vector signs and matrices are denoted by caps 
$({\vec \alpha} = \{ \alpha_j \}_{j = 0}^p\,,\enskip \Theta$\linebreak $ = \{ \vec \theta_j \}_{j = 1}^p)$.
The row-vector ${\vec \psi} \left( {\Theta}, {\vec x} \right)$ consists of the dictionary functions 
values at the point ${\vec x}$ ($\psi_0 \equiv 1$ corresponds to $\alpha_0$).
Therefore, the approximator $\hat f \left( {\vec x}\right)$ is defined by the 
matrix~$\Theta$ of dictionary functions parameters and by the vector
  ${\vec \alpha}$ of the linear combination coefficients.
In order to form the dictionary, sigmoid functions (sigmoids) are used:
\begin{gather*}
  \psi_j\left( {\vec \theta_j}, {\vec x} \right) =
        \sigma\left({\vec x}^{\mathrm{T}}\theta_j + \theta_j^0\right)\,;\\
        {\vec\theta}_j = (\theta_j,\theta_j^0)\,,\enskip 
        \theta_j \in R^d\,, \enskip 
        \theta_j^0\in R^1\,,
        \end{gather*}
        where
        $$
     \sigma\left(z\right) = \fr{e^z - e^{-z}}{e^z + e^{-z}}\,,\ z\in R^1\,.
$$
The dictionary consisting of such functions can be used for approximation of rather wide class 
of  functions~$f$ (see the results in papers~\cite{petrushev, sigmoidApprox1}).
For example, in \cite{sigmoidApprox1}, it is shown that the approximator
 $\hat f$, composed of $p$ sigmoid functions, allows to get the approximation 
 accuracy of the order $O \left( 1/p^{{1}/{d}}\right)$. 
However, if in the dictionary 
are  not included arbitrary sigmoid functions but  they are selected
depending on the approximated function~$f$ 
  (``tune'' the dictionary functions),
then the approximation accuracy has the order $O\left(1/p\right)$.

\subsection{Structure of the Approximation Algorithm}
\label{approxAlgo}

\noindent
In order to construct the dictionary, it is necessary to select type and  
number of functions in the dictionary and initialize their parameters.
It is impossible to determine the dictionary functions parameters explicitly.
The standard approach approximation construction has been applied that uses partitioning 
of the original sample into two parts $S_{\mathrm{learning}} = 
S_{\mathrm{train}} \cup S_{\mathrm{validation}}$ 
in order to prevent overtraining~\cite{vapnik}.

\bigskip

\noindent
\textbf{Algorithm~1} (main steps of the approximation construction algorithm) % \label{alg:1}
\noindent
\begin{enumerate} 
    \item {Model Selection Step (in the considered case, the model structure 
    is defined by the number $p$ of the dictionary functions)}.\\[-9pt]
    \item Initialization Step:\\[-9pt] 
    \begin{itemize}
    \item[(a)] divide the sample $S_{\mathrm{learning}}$ 
    into subsamples $S_{\mathrm{train}}$ and $S_{\mathrm{validation}}$; and \\[-9pt]
\item[(b)] set the initial values of the dictionary functions parameters~$\Theta$ 
and expansion coefficients~$\vec \alpha$.\\[-9pt]
\end{itemize}
    \item \label{train} Training Step:\\[-9pt] 
    \begin{itemize}
\item[(a)] iteratively minimize $Q\left(S_{\mathrm{train}}, \hat f \right)$ with 
    respect to the parameters $\Theta$ and $\vec \alpha$; and\\[-9pt] 
\item[(b)] stop minimization if the error $Q\left( S_{\mathrm{validation}}, \hat f \right)$ 
    begin to increase.
    \end{itemize}
  \end{enumerate}

\medskip

Each of the steps of Algorithm~1 is a separate subproblem. 
There exists a lot of methods for solution of these subproblems. 
In this paper, more efficient methods are proposed
(except solution for the subproblem of the dictionary size~$p$ selection).
As for the choice of the dictionary size~$p$, usually, 
some upper bound on its value is defined depending on the 
learning sample size and the exact value of~$p$ is selected using cross-validation.

\subsection{Optimization Algorithm}

\noindent
Usually, in order to optimize the error function (see step~3 of Algorithm~1) 
with respect to parameters of complex regression models (for example, multilayer 
neural networks),
gradient methods are used due to very high dimensionality of the parametric space.
Since in the considered case the number of parameters is relatively small,  
second-order optimization methods can be used.
The Gauss--Newton method~\cite{Nocedal} is the most common method for minimizing 
functions of the form
$$
Q\left({S_{\mathrm{train}}}, \hat f\right) = \left( \vec y - \hat f({X}) \right)^{\mathrm{T}} 
 \left( \vec y - \hat f({ X}) \right)
 $$
 

 
 \noindent
 where $X$ is a matrix of all points from 
 $S_{\mathrm{train}}$ and $\vec y = f(X)$ is a vector of the function $f$ values at 
 these points.  
In fact, the main difference between this method and the Newton method consists in
how the matrix of second-order derivatives of the error function is calculated.
Let denote by $\Omega = \{\Theta,\vec \alpha\}$ the set of all parameters of the model
 (parameters of the dictionary functions and the corresponding expansion coefficients); 
 then, assuming that a residual vector components $\vec e = \hat f({ X}) - \vec y$ are 
 small, one gets
\begin{equation}
  \label{nls}
  Q_{\Omega \Omega} = \hat f_{\Omega}^{\mathrm{T}} \hat f_{\Omega} + 
  \sum\limits_{i=1}^{N_{\mathrm{train}}} e_i \hat f_{\Omega \Omega} 
  \left(\vec x_i\right) \approx \hat f_{\Omega}^{\mathrm{T}} 
  \hat f_{\Omega} = {J}^{\mathrm{T}} { J}
\end{equation}
where ${J} = \hat f_{\Omega}$ is the matrix of the model~$\hat f$ derivatives with respect 
to $\Omega$ at the points $S_{\mathrm{train}}$.
Since $Q_{\Omega \Omega} \approx { J}^{\mathrm{T}} { J} \succeq 0$, then the 
approximate matrix of the second derivatives, calculated according to formula~(\ref{nls}), 
is always nonnegative definite. This partially solves degeneracy problem of the Newton method being applied to 
nonconvex functions.

Nevertheless, let note that during the minimization of the error function approximation of 
the Hessian,
 $Q_{\Omega \Omega} \approx { J}^{\mathrm{T}} { J}$ can become degenerate and noninvertible.
 {\looseness=1
 
 }
 
The Levenberg--Marquardt algorithm \cite{LM} was developed to solve this degeneracy problem. 
The main idea is to add identity matrix with regularization multiplier to the approximate 
Hessian matrix when searching the step size according to the Gauss--Newton method, i.\,e.,
{\looseness=1

}
\begin{equation}
  \label{lm}
  \Omega^k = \Omega^{k-1} - \left( { J}^{\mathrm{T}} { J} + \mu \mathrm{I} \right)^{-1} Q_\Omega\,.
\end{equation}
The parameter $\mu$ defines behavior of the algorithm:
for small~$\mu$, the step size is close to the step size of the Gauss--Newton 
method; for big~$\mu$, the step is done along the antigradient with the size 
approximately equal to ${1}/{\mu}$.
Therefore, one can always find such value of the parameter~$\mu$ which provides decrease of the error function. 
When training the model, the Levenberg--Marquardt algorithm will be used (this method is 
also realized in MatLab for ANN training).

\section{Initialization Step}
\label{stage1}

\noindent
In this section, the algorithms for solving two subproblems of the initialization step
are proposed:
extraction of the validation sample and initialization of the model parameters.
Also, in subsection~3.3,  a methodology for an experimental comparison 
of approximation algorithms is described and the results of this comparison
are provided.

\subsection{Extraction of the Validation Sample}
\label{divideSample}

\noindent
Let consider the first subproblem of the initialization step, i.\,e., 
decomposition of the initial sample 
$S_{\mathrm{learning}}$ into two parts $S_{\mathrm{train}}$ and $S_{\mathrm{validation}}$ where
$S_{\mathrm{train}}$ is used for an iterative tuning of the model parameters and $S_{\mathrm{validation}}$ is used 
to estimate a generalization ability (an estimate of the proximity between the original function 
and its approximation) and to stop the iterative tuning process when the overtraining 
appears.

In order to estimate the generalization, the set of points
$X_{\mathrm{validation}} = \{\vec x \in S_{\mathrm{validation}}\}$ should be ``uniformly'' 
distributed among other points of the learning sample 
$X_{\mathrm{learning}} = \{\vec x \in S_{\mathrm{learning}}\}$.
Standard algorithms for approximation construction perform decomposition into the validation
 $S_{\mathrm{validation}}$ and the training $S_{\mathrm{train}}$ samples randomly, which often results in 
 inconsistent decomposition.

Let estimate the uniformity of $X_{\mathrm{learning}}$ decomposition into the 
sets $X_{\mathrm{train}}$ and $X_{\mathrm{validation}}$ using the following criterion:
\begin{multline}
  \label{GurMorMit}
   U(X_{\mathrm{train}}, X_{\mathrm{validation}}) = 
   \fr{1}{\# r^{1}}\sum\limits_{ \{\vec x_i, \vec x_j\} \in r^{1}} 
   \fr{1}{\| \vec x_i - \vec x_j\|}\\
   {} - \fr{1}{\# r^{2}}
   \sum\limits_{ \{\vec x_i, \vec x_j\} \in r^{2}} \fr{1}{\|\vec x_i - \vec x_j\|}
\end{multline}
where $r^{1}$ is the set of pairs of points such that each of points belongs either to
 $X_{\mathrm{train}}$ or to $X_{\mathrm{validation}}$;
  $r^{2}$ is the set of pairs of points such that one of them belongs to
 $X_{\mathrm{train}}$ and another belongs to $X_{\mathrm{validation}}$; and
  $\# r^{i}$ is the cardinality of $r^{i}$, $i = 1, 2$.
  When minimizing  $U(X_{\mathrm{train}}, X_{\mathrm{validation}})$ 
  (with respect to different decompositions of $X_{\mathrm{learning}}$), 
  the distance between points from one class is maximized and the distance 
  between points from different classes is minimized that fully meets all objectives.

Optimization of the criterion~\eqref{GurMorMit} is an $NP$-hard combinatorial problem
that cannot be solved for reasonable time when the sample size $N_{\mathrm{learning}} \gg 1$. 
On the other hand, for the case $N_{\mathrm{learning}} \sim 10$, the optimization 
can be performed by 
the full search. Therefore, let consider the simplification of this optimization problem:
divide all the design domain into rather small hypercubes and optimize the criterion locally by relocating points from one class
 ($S_{\mathrm{train}}$) to another  ($S_{\mathrm{validation}}$) only within each of the hypercubes.
 
In order to divide the design domain, Classification and Regression Trees~\cite{CART} have been used. 
This method is based on the sequence of simple cuts of the design domain with respect to 
the input vector components. In each of the hypercubes, obtained during the previous iteration, 
a constant approximation was constructed by averaging the output values of the points belonging to this 
hypercube.
The input component and the location of the next cut are selected optimally in the 
sense of the mean square error of the corresponding piecewise constant approximation.
There are a lot of criteria for stopping the tree construction process. These criteria are 
based on the generalization ability estimation of the tree~\cite{elementsOfStatLearning}. 
Here, as a stopping criterion, an upper bound on the number of points belonging 
to each leaf of the tree is used since the optimization complexity of the criterion~\eqref{GurMorMit} 
depends on this upper bound.

\medskip

\noindent
\textbf{Algorithm~2} (sample $S_{\mathrm{learning}}$ decomposition) %\label{alg:2}
\ 
\noindent
  \begin{enumerate}
    \item {Construct a regression tree~\cite{CART} with an upper 
    bound on the number of points belonging to the tree leaves (the 
    number of points should be bigger than $\mathrm{leaf}_{\min} = 8$ and smaller than 
    $\mathrm{leaf}_{\max}=16$) approximating the function $f$ with piecewise constant approximation}. 
{Let the number of leaves of the constructed tree be equal to some $K$}. 
    \item {Let $s_k$ be the set of points 
    $\vec x \in X_{\mathrm{learning}}$ belonging to the leaf with the number~$k$}.
    \item {For all $k = 1, \dots, K$ using greedy algorithm (local optimization 
    in each separate hypercube),  decompose the set $s_k$ into subsets $s_k^{\mathrm{val}}$ and 
    $s_k^{\mathrm{tr}}$ such that the criterion  $U(s_k^{\mathrm{val}}, s_k^{\mathrm{tr}})$ takes its minimal 
    value under the restriction 
    that the fraction of the validation sample size is not smaller than $\mathrm{val}_{\mathrm{part}} = 0.2$}.
    \item {Construct $S_{\mathrm{validation}}$ and} $S_{\mathrm{train}}$
\begin{align*}
S_{\mathrm{validation}} &= \left\{ \{\vec x, y = f(\vec x)\}: \vec x \in 
\bigcup\limits_{k=1}^K  s_k^{\mathrm{val}} \right \}\,; \\
S_{\mathrm{train}} &= \left\{ \{\vec x, y = f(\vec x)\}: \vec x \in 
\bigcup\limits_{k=1}^K  s_k^{\mathrm{tr}} \right \}\,.
\end{align*}
  \end{enumerate}


\medskip

The proposed algorithm contains two computationally expensive steps: 
construction of the regression tree (complexity is equal to 
$O(N_{\mathrm{learning}}\log(N_{\mathrm{learning}}))$)
and local optimization of the criterion $U(s_k^{\mathrm{val}}, s_k^{\mathrm{tr}})$ 
(complexity is equal to $O(K) = O(N_{\mathrm{learning}})$).
For the second step, the constant in
 $O(K)$ depends on the number of ways to decompose a leaf with 
 $\mathrm{leaf}_{\mathrm{num}}$\linebreak $\in [\mathrm{leaf}_{\min}, \mathrm{leaf}_{\max}]$ points into two parts. 
 Due to the restriction on the proportion
 between the sizes of the validation and the training samples,
  this number is equal to
 $C_{\mathrm{leaf}_{\mathrm{num}}}^{[\mathrm{leaf}_{\mathrm{num}}
 \cdot \mathrm{val}_{\mathrm{part}}]}$, i.\,e., it is bigger than~28 
 (for $\mathrm{leaf}_{\mathrm{num}} = \mathrm{leaf}_{\min} = 8$) and smaller than 560 (for 
  $\mathrm{leaf}_{\mathrm{num}} = \mathrm{leaf}_{\min} = 16$). 
  
\subsection{Initialization of the Model Parameters}
\label{init}

\noindent
Initialization of the dictionary functions parameters~$\Theta$ significantly 
influences the approximation construction process and the final approximation accuracy.
The random methods Nguyen--Widrow (NW)~\cite{initNW} and SCAWI
(statistically controlled activation weight initialization)~\cite{initSCAWI}
are the most widely used algorithm for $\Theta$ initialization.
These methods use some matrix of independent random variables,
   multiplied by a fixed factor defining smoothness of the dictionary functions.
   {\looseness=1
   
   }
   
Note that the random vectors generation in high-dimensional spaces
   leads to their clustering thus giving rise to clustering of the directions 
   $\left\{\theta_j\right\}_{j = 1}^p$.
Here, an initialization algorithm, which
generates a rich functional dictionary    
(in terms of a uniform distribution of the directions) is proposed.
The directions $\left\{\theta_j\right\}_{j = 1}^p $ are generated
uniformly on the unit 
sphere using the methods from~\cite{SpherRnd}. 
This method is based on a normalization of points generated by a multivariate normal distribution.
The method uses the invariance property of a normal density relative 
to an arbitrary rotation and efficiently generates points with uniform 
distribution on the unit sphere.
   {\looseness=1
   
   }

Let now consider how to select norms values (defined by some scaling multipliers) 
of the vectors
$\left\{\theta_j\right\}_{j=1}^p$.
The value of the norm influences the smoothness of the corresponding sigmoid:
for a sufficiently small value, the sigmoid is almost linear on the compact~D;  
for a big value, the sigmoid behaves like the step function.
If one fixed value is used for all norms, then all sigmoids will have equal smoothness 
and the dictionary will not be ``rich'' enough.
Therefore, in order to define the norms values, the multipliers generated 
by the uniform distribution in some range have been used. Thus, the vectors 
$\left\{\theta_j\right\}_{j=1}^p$ have different norms and, finally, the 
dictionary contains sigmoids with different smoothness.
{\looseness=1

}

\bigskip

\noindent
\textbf{Algorithm~3} (initialization of the parameters $\Theta$) %\label{alg:3}

\noindent
 \begin{enumerate}
    \item {Construct a matrix $S$ of size $p \times d$ containing $p$ vectors, generated by the uniform 
    distribution on a \mbox{$d$-dimensional} sphere of unit radius}. 
    \item {Let
      $r = \sqrt{d}\, p^{1 / N_{\mathrm{train}}}$ and generate values of scaling multipliers $\xi_j, j=1,\dots,p$, 
      uniformly randomly on $[0, r]$. Such definition of $r$ guarantees that the number of points belonging to the 
      domain of a sigmoid saturation is small~\cite{initNW}.
    \item {Let define the sigmoids direction vectors} 
    $\left\{\theta_j \right\}_{j=1}^p$ according to the formula
    $\theta_j = \xi_j S_j$, $j=1,\dots,p$, where ${S}_j$ 
    is the $j$th line of the matrix~S. The offset values 
    $\left\{\theta_j^0\right\}_{j=1}^p$ are defined in the same way as in~\cite{initNW}, 
    i.\,e., according to the uniform grid on the interval} [$-r,r$].
  \end{enumerate}

\medskip

In~\cite{itas11_Yerofeyev}, written by the present
authors, comparison of popular random initialization 
methods with the proposed method and several specially developed deterministic initialization methods 
can be found.

\subsection{Experimental Results}
\label{exp1}

\noindent
The proposed algorithms for the initialization step should be compared 
with the standard approaches in terms of the average error of approximation and its 
variations. For generation of test problems, a set of artificial multidimensional 
functions has been used for testing optimization algorithms.
This choice is due to the fact that in the framework of 
surrogate modeling, constructed approximation is often used as the 
objective function instead of original function.
Let distinguish two types of test problems:
\begin{enumerate}[(1)]
  \item test functions with fixed dimension of the input vector $\vec x$~--- 
  allinit, beale, hartmann, ishigami, wbd; and
  \item test functions with dimension of the input vector~$\vec x$ that can be 
  set to some predefined value~--- gSobol, michalewicz, rosenbrock, whitley, zdt3. 
  In experiments, the input dimension was varied from~3 to~10.
\end{enumerate}
Exact formulas for the test functions can be found in~\cite{data1, data2}.

As a relative approximation error, the root-mean-square error normed by the analogous 
error for the constant approximation was used:
\begin{equation}
  E\left(S_{\mathrm{test}}, \hat f\right) = 
  \sqrt{\fr{\sum\nolimits_{i=1}^{N_{\mathrm{test}}} 
  \left( y_i - \hat f \left( \vec x_i\right) \right)^2}{\sum\nolimits_{i=1}^{N_{\mathrm{test}}} 
  \left( y_i - \bar{y} \right)^2}}
    \label{errorDef}
  \end{equation}
  where
  \begin{equation*}
  \bar{y} = \fr{1}{N_{\mathrm{test}}} \sum\limits_{i=1}^{N_{\mathrm{test}}} y_i\,.
\end{equation*}
The error comparable with $1$ corresponds to a very inaccurate approximation 
and the error comparable with $10^{-3}$ corresponds to a very accurate approximation.
In order to calculate the error, a separate test set $S_{\mathrm{test}}$ has been used.

\begin{figure*} %fig1
\vspace*{1pt}
 \begin{center}
 \mbox{%
 \epsfxsize=162.923mm
 \epsfbox{bel-1.eps}
 }
 \end{center}
 \vspace*{-6pt}
  \Caption{Methods for solving problems of the initialization step (errors for some tests):
  (\textit{a})~Whitley, $d=3$, $N=350$; (\textit{b})~gSobol, $d=3$, $N=500$;
  (\textit{c})~wbd, $d=3$, $N=75$; 
  (\textit{d})~Rosenbrock, $d=3$, $N=350$;
  (\textit{e})~Hartmann, $d=3$, $N=500$; and
  (\textit{f})~Allinit, $d=3$, $N=75$}
  \label{boxplot_init}
\end{figure*}

For each function (or pair ``function-input dimension'' for test problems of 
the second kind),  the learning sample size was selected such that the approximation error~(\ref{errorDef}), 
obtained using a standard approximation method, belonged to the interval
0.01--0.2 since this range is the most interesting for practice (usually, smaller 
values are not required in practice).
As a standard approximation method, the realization of ANN from MatLab such that
NW algorithm is used for the initialization, the validation sample is extracted 
randomly, and the training is performed by the Levenberg--Marquardt algorithm. In order to select 
the optimal dictionary size (the number of functions $p$) in each experiment, the brute 
force algorithm has been used with the criterion defined by the approximation error, estimated using 
cross-validation. Each separate experiment (the test function, the input dimension, and the
learning sample size are fixed) has the following setup:
10~random learning sample $S_{\mathrm{learning}}$ are generated; for each sample,
10~approximators are constructed. Let compare the following algorithms:
\begin{itemize}
\item the standard ANN algorithm (the methods are denoted on plots by number~1); 
\item the standard ANN with the proposed method for the validation sample extraction
(method~2); 
\item the standard ANN with the proposed
method for parameters initialization (method~3); and 
\item the standard ANN with both proposed
methods for the initialization step (method~4).
\end{itemize}


In most cases, the proposed algorithms for solution the problems of the initialization 
step improve the final quality of the approximation (applied either independently or simultaneously). 
It is interesting that combination of the proposed algorithms (method~4) 
in most cases makes it possible to obtain the model with the superior accuracy 
compared with the accuracy of the model obtained using these algorithms independently.
Therefore, new algorithm is more effective, see examples of diagrams 
in Fig.~\ref{boxplot_init} and also, Dolan--More curves in section~\ref{experiments}.

\section{Approximation Training}
\label{stage2}
 
\subsection{Separability of Variables}
\label{separability}

\noindent
Let consider the error function on the training sample as a function of the parameters $\Theta$ and $\vec \alpha$:
\begin{multline*}
  Q\left( S_{\mathrm{train}}, \hat f\right) =
  Q\left( \Theta, {\vec \alpha} \right)\\
  {}=   \left( \vec y - \hat f \left( X, \Theta, \vec\alpha \right) \right)^{\mathrm{T}}
  \left( \vec y - \hat f \left( X, \Theta, \vec\alpha \right) \right)\,.
\end{multline*}

Note that the dependence of the error function~$Q$ on the dictionary 
functions parameters~$\Theta$ is nonlinear and very complex.
At the same time, the dependence of~$Q$ on expansion coefficients $\vec \alpha$ is quadratic.
For the fixed~$\Theta$, the optimal values of~$\vec\alpha$ can be found by the least squares 
method:
\begin{equation}
\label{alphaLsq}
\vec\alpha \left( \Theta \right) =
 \left( \Psi\left( \Theta \right)^{\mathrm{T}} \Psi\left( \Theta \right) \right)^{-1}
 \Psi\left( \Theta \right)^{\mathrm{T}} \vec y
 \end{equation}
  where  
  $$
  \Psi\left( \Theta \right) = \left\{{\vec \psi} \left( {\Theta}, {\vec x}_i \right),
  i = 1,\ldots,N_{\mathrm{train}}\right\}\,.
$$

Let take this fact into account when tuning parameters of the approximator. 
For this, let consider the objective function
  $R\left( \Theta \right) = Q\left( \Theta, \vec\alpha\left(\Theta\right) \right)$
  where $\vec\alpha\left(\Theta\right)$ is calculated according to~(\ref{alphaLsq}).
When calculating the expansion coefficient according to~(\ref{alphaLsq}), a 
nonlinear dependence 
$\vec\alpha =  \left( \Theta \right)$ should be taken into account when calculating 
the derivatives $R_\Theta$ and~$R_{\Theta\Theta}$.
An algorithm for such calculations was proposed in~\cite{golub} and its 
theoretical properties were investigated in~\cite{ruhe}.

However, this algorithm has significant shortcoming: in~(\ref{alphaLsq}), 
inversion of matrix $\Psi^{\mathrm{T}} \Psi$  which, in general, can be degenerated, was used. 
In such case, the inverse matrix does not exist and it is not reasonable to use the error 
function $R\left( \Theta \right)$.
This situation was investigated in details in the framework of the linear regression methods~\cite{demidenko}.
It can be shown that even if the matrix
$\Psi^{\mathrm{T}} \Psi$ does not degenerate but is ill-conditioned, then estimates of the coefficients
 $\vec \alpha$ are unstable (in statistical terms, this means that the estimate of
 $\vec \alpha$ has very big variance).
In such case, the calculated values of the gradient and the Hessian of the error function 
are also unstable: even with small variations of the parameters~$\Theta$, 
the first and the second derivatives of the error function change significantly 
resulting in a very low training speed.
 
Let consider a classical approach for regularization in linear regression 
problems, namely,  ridge regression~\cite{demidenko}.
Let add to the matrix $\Psi^{\mathrm{T}}\Psi$ a positively definite matrix $\lambda I_p$, 
where
 $I_p$ is a unit matrix with sizes $p\times p$ and $\lambda > 0$ is a some constant.
In such case, formula~(\ref{alphaLsq}) takes the form
\begin{equation}
\label{alphaRidge}
\vec\alpha \left( \Theta \right) =
 \left( \Psi\left( \Theta \right)^{\mathrm{T}} \Psi\left( \Theta \right) + 
 \lambda I_p\right)^{-1}
 \Psi\left( \Theta \right)^{\mathrm{T}} \vec y\,.
\end{equation}
It is obvious that any level of matrix 
  $\left( \Psi\left( \Theta \right)^{\mathrm{T}} \Psi\left( \Theta \right)\right.$\linebreak 
  $ \left.+\;\lambda I_p\right)$ 
  conditionality can be reached by increasing the value of $\lambda$.
Use of the ridge regression is equivalent to redefinition of the 
error function in the following way:
\begin{multline*}
%  \label{ridgeR}
  \widetilde R\left( \Theta \right) = 
  \widetilde Q\left( \Theta, \vec \alpha\left(\Theta\right) \right)\\ =
  \left( \vec y - \hat f \left( X, \Theta, \vec\alpha\left(\Theta\right) \right) \right)^{\mathrm{T}}
  \left( \vec y - \hat f \left( X, \Theta, \vec\alpha\left(\Theta\right) \right) \right)\\
{}  + \lambda \vec \alpha\left(\Theta\right)^{\mathrm{T}} \vec \alpha\left(\Theta\right)\,.
\end{multline*}

\noindent
\textbf{Statement 1.} %\label{derivCalc}
\textit{The gradient and the Hessian of the error function
 $\widetilde R$ can be calculated according to the following formulas}:
  \begin{align}
  \widetilde R_\Theta &= \widetilde Q_\Theta = \vec e^{\mathrm{T}} \mathrm{J}\,;\notag\\ 
    \widetilde R_{\Theta\Theta} &=  
      \mathrm{J}^{\mathrm{T}}\mathrm{J} + \vec e \odot \hat f_{\Theta \Theta}-
      \left({\vec e}^{\mathrm{T}} \odot \Psi_\Theta + \mathrm{J}^{\mathrm{T}}  \Psi\right)\notag\\
      &{}\times  \left( \Psi^{\mathrm{T}} \Psi  + \lambda \mathrm{I} \right)^{-1}
      \left({\vec e}^{\mathrm{T}} \odot \Psi_\Theta + \mathrm{J}^{\mathrm{T}}  
      \Psi\right)^{\mathrm{T}}\,.    \label{rHessBeforeNls}
  \end{align}

\noindent
P\,r\,o\,o\,f.\ \ 
Let find the first derivatives of the function~$\widetilde R\left( \Theta \right)$:
  \begin{equation}
    \label{gradR}
    \widetilde R_\Theta = \widetilde Q_\Theta + \widetilde Q_\alpha \alpha_\Theta = \widetilde Q_\Theta + \vec0 \alpha_\Theta = \widetilde Q_\Theta\,.
  \end{equation}
   Since coefficients 
  $\vec\alpha \left( \Theta \right)$, obtained using~(\ref{alphaRidge}), 
  are the minimizer of the function
 $\widetilde Q\left( \Theta, \vec \alpha\left(\Theta\right) \right)$,
 $\widetilde Q_\alpha = \vec 0$.
  Let  differentiate equality~(\ref{gradR}) with respect to $\Theta$:
  \begin{equation}
    \label{HessianR}
    \widetilde R_{\Theta\Theta} = \widetilde Q_{\Theta\Theta} + \widetilde Q_{\Theta\alpha} \alpha_\Theta\,.
  \end{equation}

Contrary to~(\ref{gradR}), the second summand is not equal to~0. 
Therefore, it is necessary to calculate $\alpha_\Theta$.
Taking into account the equality $\widetilde Q_\alpha = \vec 0$ as a consequence, 
one gets that
\begin{equation*}
    d{\widetilde Q}_\alpha = \widetilde Q_{\alpha \alpha}d\alpha  + 
    \widetilde Q_{\alpha \Theta}d\Theta  = 0\,.
\end{equation*}
Consequently,
$$
    \alpha_\Theta = \fr{d\alpha}{d\Theta} = - 
    \widetilde Q_{\alpha \alpha}^{-1} \widetilde Q_{\alpha \Theta}\,.
$$
Then formula~(\ref{HessianR}) takes the form
  $$
  \widetilde R_{\Theta\Theta} =  \widetilde Q_{\Theta\Theta} - \widetilde Q_{\Theta\alpha} \widetilde Q_{\alpha \alpha}^{-1} \widetilde Q_{\alpha \Theta}\,.
  $$

Now let write explicitly expressions for the derivatives of the function~$\widetilde Q$.
Let denote by $\mathrm{J} = {\hat f_\Theta(X)}$ the matrix of the derivatives 
of the model~$\hat f$ with respect to~$\Theta$ at the points~$S_{\mathrm{train}}$.
Therefore,
  \begin{equation}
    \left.
    \begin{array}{rl}
    \widetilde Q_{\Theta} &= {\vec e}^{\mathrm{T}} \mathrm{J}\,; \\[9pt]
    \widetilde Q_{\Theta \Theta} &= \mathrm{J}^{\mathrm{T}}\mathrm{J} + 
    \vec e \odot {\hat f_{\Theta \Theta}}\,; \\[9pt]
    \widetilde Q_{\alpha \alpha} &= \Psi^{\mathrm{T}} \Psi + \lambda \mathrm{I}\,;\\[9pt]
    \widetilde Q_{\Theta\alpha} &= {\vec e}^{\mathrm{T}} \odot \Psi_\Theta + 
    \mathrm{J}^{\mathrm{T}}  \Psi\,.
    \end{array}
    \right\}
        \label{qDeriv}
  \end{equation}

Final formula~(\ref{rHessBeforeNls}) can be obtained directly from 
  $$
  \widetilde R_{\Theta\Theta} =  \widetilde Q_{\Theta\Theta} - 
  \widetilde Q_{\Theta\alpha} \widetilde Q_{\alpha \alpha}^{-1} 
  \widetilde Q_{\alpha \Theta}
  $$ 
  using expressions for the derivatives from~(\ref{qDeriv}).

\medskip

\noindent
Assuming the residual vector 
${\vec e}$ to be small,   all terms with ${\vec e}$ in~(\ref{rHessBeforeNls}) can be neglected:
\begin{equation}
  \label{HessianRfinal}
  \widetilde R_{\Theta\Theta} \approx \mathrm{J}^{\mathrm{T}}\mathrm{J}  - 
  \left(\mathrm{J}^{\mathrm{T}}  \Psi\right)
    \left( \Psi^{\mathrm{T}} \Psi + \lambda \mathrm{I}\right)^{-1} \left( \mathrm{J}^{\mathrm{T}} 
    \Psi\right)^{\mathrm{T}}\,.
\end{equation}
Therefore, one gets explicit formulas for calculation of the gradient and 
the approximate Hessian matrix of the modified error function $\widetilde R(\Theta)$ 
that are necessary for optimization based on the Levenberg--Marquardt algorithm 
(see formulas~(\ref{nls}) and~(\ref{lm})).

Let note that the product
$\mathrm{J}^{\mathrm{T}} \mathrm{J} \approx \widetilde Q_{\Theta\Theta}$ is 
nonnegatively definite.
Let show that the Hessian $\widetilde R_{\Theta\Theta}$ also has this property. 
In such case, one can construct accurate local-quadratic approximations of the error function during its optimization.

\medskip

\noindent
\textbf{Statement~2.} %\label{nonNegDef}
\textit{The matrix 
$$
\mathrm{H} \mathrel{\stackrel{\mathrm{def}}=} \mathrm{J}^{\mathrm{T}}
\mathrm{J}  - \left(\mathrm{J}^{\mathrm{T}}  \Psi\right)
    \left( \Psi^{\mathrm{T}} \Psi + \lambda \mathrm{I}\right)^{-1} 
    \left( \mathrm{J}^{\mathrm{T}} 
    \Psi\right)^{\mathrm{T}}$$ 
    is nonnegatively definite for any} $\lambda \ge 0$.

\medskip

\noindent
P\,r\,o\,o\,f\,.\ \ 
  Let rewrite matrix~H in the following form:
\begin{multline*}
   \mathrm{H} = \mathrm{J}^{\mathrm{T}}\mathrm{J}  - \left(\mathrm{J}^{\mathrm{T}} \Psi\right)
     \left( \Psi^{\mathrm{T}} \Psi + \lambda \mathrm{I}_p \right)^{-1} 
     \left( \mathrm{J}^{\mathrm{T}}  \Psi\right)^{\mathrm{T}}\\
      {}=
     \mathrm{J}^{\mathrm{T}}\left( \mathrm{I}_{N} - \Psi \left( \Psi^{\mathrm{T}} \Psi + 
     \lambda \mathrm{I}_p\right)^{-1} \Psi^{\mathrm{T}}  \right)\mathrm{J}\,.
  \end{multline*}
Let $\Psi = \mathrm{V} \mathrm{Q} \mathrm{U}^{\mathrm{T}}$ be a singular value 
decomposition for the matrix of regressors, then
\begin{multline*}
 \Psi \left( \Psi^{\mathrm{T}} \Psi + \lambda \mathrm{I}_p\right)^{-1} \Psi^{\mathrm{T}} \\
{}=  \mathrm{V} \mathrm{Q} \mathrm{U}^{\mathrm{T}} \left( \mathrm{U} 
\mathrm{Q}^2 \mathrm{U}^{\mathrm{T}} + \lambda \mathrm{U} \mathrm{U}^{\mathrm{T}}  
\right)^{-1} \mathrm{U} \mathrm{Q} \mathrm{V}^{\mathrm{T}} \\
{}   = \mathrm{V} \mathrm{Q} \mathrm{U}^{\mathrm{T}} 
\mathrm{U} \left( \mathrm{Q}^2  + \lambda \mathrm{I}_p  \right)^{-1} \mathrm{U}^{\mathrm{T}} 
\mathrm{U} \mathrm{Q} \mathrm{V}^{\mathrm{T}}\\
{} =
\mathrm{V} \mathrm{Q}^2 \left( \mathrm{Q}^2  + \lambda \mathrm{I}_p  
\right)^{-1} \mathrm{V}^{\mathrm{T}}\,.
\end{multline*}

Let $\tilde q_j$ be eigenvalues of the matrix $\mathrm{V} \mathrm{Q}^2 \left( \mathrm{Q}^2  + 
\lambda \mathrm{I}_p  \right)^{-1} \mathrm{V}^{\mathrm{T}}$, then
  $\tilde q_j = {q^2_j}/({q^2_j + \lambda})$ where $q_j$ are the elements of the matrix~$Q$.
The eigenvalues of the matrix $\mathrm{P} = \mathrm{I}_{N} - \Psi \left( 
\Psi^{\mathrm{T}} \Psi + \lambda \mathrm{I}_p\right)^{-1} \Psi^{\mathrm{T}}$ 
are not smaller than the difference of the minimal eigenvalue of
  $\mathrm{I}_{N}$ and the maximal eigenvalue from the set $\tilde q_j$.
  It is obvious that this difference is nonnegative since 
  $$
  1 - \max_j \tilde q_j =  1 - \max\limits_j \left( \fr{q^2_j}{{q^2_j + \lambda}}\right) 
  \ge 0
  $$ 
  for $\lambda \ge 0$.
  Therefore, the matrix $\mathrm{P}$ is nonnegatively definite and the matrix 
  $\mathrm{H} = \mathrm{J}^{\mathrm{T}} \mathrm{P} \mathrm{J}$ is also 
  nonnegatively definite.


\medskip

Now let estimate the computational complexity of formula~(\ref{HessianRfinal}) 
for the Hessian calculation $\widetilde R_{\Theta\Theta}$.
The size of the Jacobian matrix
 $\mathrm{J}$ is equal to $N_{\mathrm{train}} \times p(d+1)$, the size of 
 the matrix with regressors $\Psi$ is equal to $N_{\mathrm{train}} \times p$.
It is necessary to perform $O(N_{\mathrm{train}}p^2d^2)$ operations in order 
to calculate the main term $\mathrm{J}^{\mathrm{T}}\mathrm{J}$ , 
which is needed also for calculation of the standard error function.
At the same time, additional summand
 $\left(\mathrm{J}^{\mathrm{T}}  \Psi\right)
    \left( \Psi^{\mathrm{T}} \Psi + \lambda \mathrm{I}\right)^{-1} \left( 
    \mathrm{J}^{\mathrm{T}}  \Psi\right)^{\mathrm{T}}$ can be calculated for 
    $$
    O(N_{\mathrm{train}} p^2 d + p^3 + p^3d + p^3d^2) = O(N_{\mathrm{train}} p^2 d + p^3d^2)
    $$ 
    operations. Since in the considered class of approximation problems
$N_{\mathrm{train}} \gg d$, $N_{\mathrm{train}} \gg p$, then calculation 
of the additional summand requires significantly smaller number of operations 
compared to the number of operations for calculation of the main summand.

The proposed error function $\widetilde R\left( \Theta \right)$ does not only 
increase approximation accuracy but also decreases training time 
compared to the training time when using the standard error function
 $Q\left( \Theta, \vec \alpha \right)$~\cite{itas11_Belyaev}.
This advantage can be explained by two reasons. Some part of the parameters 
are estimated optimally on each iteration of the training algorithm and adaptive 
regularization increases numerical stability of the training process. In this work,
it is assumed that the output $y$ dimension is equal to~1, but for many applied problems, 
the output~$y$ can be multidimensional and its dimension~$d_y$ can even be higher 
than the input dimension $d$. In such case, the number of parameters of the standard 
error function is $({d+d_y})/{d}$ times higher than that of the modified error function.
Thus, separability of the variables can significantly decrease the number of optimized 
parameters in some problems.
{ %\looseness=1

}

\subsection{Adaptive Regularization}
\label{regularization}

\noindent
Standard approaches for regularization of models with the structure
 $\hat f \left( {\vec x} \right) = {\vec \psi} \left( {\Theta}, {\vec x} \right) 
 {\vec \alpha}$ use the 
  $L_2$ penalty on all parameters of the model~\cite{elementsOfStatLearning} and 
  a regularization coefficient is determined experimentally using some additionally 
  extracted validation samples and multiple training of the surrogate model.
In~\cite{bayesianLearning}, some Bayesian approach is proposed for the regularization 
parameter selection and, again, the $L_2$ penalty on all parameters of the model are used.
This method has two key shortcomings: 
\begin{enumerate}[(1)]
\item selection of the regularization parameter does 
not take into account the error of approximation; and\\[-9pt]
\item  the penalty incorporates norms of 
the parameters $\Theta$ and $\vec\alpha$ with equal weights and does not take into 
account essentially different nature of these parameters.
\end{enumerate}

In the proposed approach,  only the expansion coefficients~$\vec \alpha$  have been penalized
and regularization parameter~$\lambda$ has been selected optimally (in some sense) with 
taking into account the error of approximation.
{ %\looseness=1

}

When tuning the regularization parameter $\lambda$ during the training process, 
the functional dependency  $\lambda = \lambda \left( \Theta \right)$ appears. In general 
case, one should take into account this dependency when calculating the derivatives 
of the error function with respect to~$\Theta$.
However, in the framework of the considered approximation problem, optimization of 
the error function on the set $S_{\mathrm{train}}$ can be considered as the process for 
generation of different models with the final aim to obtain the model with the smallest 
error on the independent test set $S_{\mathrm{test}}$ rather than for finding the minimum of the 
error function on the train set $S_{\mathrm{train}}$.
Due to this remark, the change of the regularization parameter value during the training process 
is allowed and  these changes can be neglected
when calculating the derivatives of the error  function.
{\looseness=1

}

\begin{figure*}[b] %fig2
\vspace*{6pt}
 \begin{center}
 \mbox{%
 \epsfxsize=162.923mm
 \epsfbox{bel-2.eps}
 }
 \end{center}
 \vspace*{-6pt}
    \Caption{Methods for solving problems of the training step (errors for some tests):
  (\textit{a})~Whitley, $d=4$, $N=750$; (\textit{b})~wbd, $d=4$, $N=125$;
  (\textit{c})~Beale, $d=3$, $N=100$; 
  (\textit{d})~Michalewicz, $d=5$, $N=250$;
  (\textit{e})~zdt, $d=3$, $N=500$; and
  (\textit{f})~Ishigami, $d=3$, $N=250$}
  \label{boxplot_train}
\end{figure*}

In order to select the regularization parameter~$\lambda$ on each iteration of 
the training algorithm, let minimize the GCV  (Generalized Cross 
Validation) criterion~\cite{elementsOfStatLearning} estimating the approximation error
on the test sample in linear regression problems (${\vec \alpha }\left( \lambda \right)$ 
is calculated according to~(\ref{alphaRidge})):

\noindent
\begin{multline*}
%\label{GCV}
  GCV(\Psi, \lambda) = %\left( {\Lambda} \right) =
    \fr{\sum\limits_{i=1}^{N_{\mathrm{train}}} \left( y_i - \hat f(x_i)\right)^2 }
    {\left( 1 - ({1}/N_{\mathrm{train}}) \mathrm{tr}\left( \mathrm{L}  \right) \right)^2  }\\
   = %{N_{train}
    \fr{\left(\vec y - \Psi \vec \alpha(\lambda)\right)^{\mathrm{T}} 
    \left(\vec y - \Psi \vec \alpha(\lambda)\right)}
    {\left( 1 - ({1}/{N_{\mathrm{train}}})\mathrm{tr}\left(  
     \left( \vec{\Psi}^{\mathrm{T}} \vec{\Psi} + \lambda \mathrm{I} \right)^{-1} 
     \vec{\Psi}^{\mathrm{T}} \vec{\Psi}  \right) \right)^2  }\,.
\end{multline*}
Let  note that when minimizing the criterion
GCV with respect to $\lambda$ it is necessary to control condition 
number of the matrix $\Psi^{\mathrm{T}}\Psi + \lambda I$ in order the 
training process to be stable~\cite{oldRegul}.
It is proposed to impose a lower bound on the value of $\lambda$ such that provides
necessary level of the conditionality ($10^{12}$ in the present realization of the algorithm).



\subsection{Experimental Results}
\label{exp2}

\noindent
Let use the testing methodology, described in subsection~\ref{exp1}.
Let compare the following methods: the standard method (method~1), the 
method incorporating algorithms from subsections~3.1 and~3.2 (method~4), 
the method incorporating only the modified error function (method~5),  
and the method incorporating all the proposed algorithms  (method~6).



The most indicative results are given in Fig.~\ref{boxplot_train}.
The proposed modification of the error function allows to significantly improve 
approximation quality for some problems (for example, function michalewicz).
If method~6, which incorporates all the proposed algorithms,   is considered, then one 
can see that this method does not only additionally increase approximation accuracy 
compared to the best of two methods~4 and~5, but also significantly decreases 
the approximation error in some cases (for example, zdt3 function).

\section{Experimental Results}
\label{experiments}

\noindent
In this section, the results of full experiments on artificial 
functions will be shown and the proposed approach will be compared with other similar approaches 
on some applied problems of surrogate modeling.
Let use Dolan--More curves $P_k(a)$~\cite{DolanMore} for visualization of the results.
The quantity $P_k(a)$ shows on which fraction of the problems the errors of the 
considered approximation method~$k$ are not $a\geq 1$ higher than the minimal (among 
all considered methods) approximation error for the corresponding approximation problems.
{ %\looseness=1

}

\begin{figure*} %fig3
\vspace*{1pt}
 \begin{center}
 \mbox{%
 \epsfxsize=162.066mm
 \epsfbox{bel-3.eps}
 }
 \end{center}
 \vspace*{-9pt}
  \Caption{Results for artificial problems: (\textit{a})~median values (accuracy)
  and (\textit{b})~standard deviation (scatter) 
  (\textit{1}~--- standard algorithm;
  \textit{2}~--- extraction of Svalidation; \textit{3}~--- initialization of $\theta$;
  \textit{4}~--- extraction\;+\;initialization; \textit{5}~--- training algorithm;
and  \textit{6}~--- all proposed algorithms)}
  \label{dm_artificial_median}
  \vspace*{3pt}
\end{figure*}

\begin{table*}[b]\small
\vspace*{-6pt}
\begin{center}
  \Caption{Relative approximation error
  \label{sampleTable}}
  \vspace*{2ex}
  
     \begin{tabular}{cccccc}
      \hline
      \multicolumn{2}{c}{Problem} & Composite structure & Wing characteristics & Sand  & Concrete \\
      \hline
      \multicolumn{2}{c}{Dimension of $\vec x$}   &16 &78 &3  &8\\ 
      \hline
      \multicolumn{2}{c}{Sample size} &50000  &65000  &10000  &1030\\ \hline
              \multicolumn{2}{c}{Proposed approach}
              &{0.092}  &{0.159}    &{0.091} & 0.320\\ 
              \hline
      & Linear regression & 0.616   &0.330      &0.698  &  \hphantom{9}0.6391\\ 
%      \cline{2-6}
  \multicolumn{1}{c}{\raisebox{-6pt}[0pt][0pt]{MatLab}}     & Quadratic regression
                &0.390      &---      &0.608& 0.485  \\ 
 %               \cline{2-6}
      &ANN      &0.194  &0.258      & 0.132     & 0.350  \\ 
 %     \cline{2-6}
      &Radial basis function (RBF)  &0.336  &0.628      &0.464&     0.371   \\ 
      \hline
      &$k$-nearest neighbors     &0.597      &1.115      &0.470& 0.529   \\ 
%      \cline{2-6}
      &Anisotronic kriging
        &0.528  & --- & 0.813& 2.521  \\ 
       % \cline{2-6}
  mode    &Kriging      &0.382 &1.031   &0.937& 0.889  \\ 
  %    \cline{2-6}
Frontier     &RBF          &0.363&9.392        &0.494  &0.367  \\ 
  %\cline{2-6}
      &ANN
                &0.299      &1.424& 0.356& 0.730  \\ 
   %             \cline{2-6}
      &Gaussian process   
                &0.807      &---      &0.561 & 2.088  \\ 
                \hline
    \end{tabular}
  \end{center}
\end{table*}

\subsection{Artificial Functions}

\noindent
Using Dolan--More curves, let compare the standard algorithm with all algorithms from 
subsections~3.3 and~4.3 on all problems used for testing.
As a ``separate problem,'' let consider all independently generated samples, i.\,e., 
10~problems correspond to each function. Let 
use the following criteria of approximation quality: median of the errors (accuracy) 
and standard deviation of the errors (scatter of the errors). For each problem,
 let run each method $10$ times (the error for each run is estimated using 
 formula~(\ref{errorDef})) and estimate the accuracy and the scatter using results of these 
 runs.
 { %\looseness=1
 
 }



The results of comparison are given in 
Fig.~\ref{dm_artificial_median}.
One can see that the main contribution into accuracy is obtained 
due to the modified error function with the adaptive regularization, 
but the algorithms of the initialization step 
also significantly improve the accuracy of the methods (see curves for 
the methods~5 and~6).

\subsection{Real Applied Problems}

\noindent
Let compare approximation quality of the proposed method (combining 
all the proposed algorithms)
with widely-used methods for approximation construction.
There was used realization of such methods in MatLab and modeFrontier.
This software systems are used by many of industrial companies.
There are a number of methods for approximation construction, implemented in MatLab and 
modeFrontier,
namely, ANN, regression based on Gaussian Process, etc.
Note that many of these methods have serious 
restrictions on possible characteristics of the sample (input dimension~$d$ and sample 
size $N_{\mathrm{learning}}$) which, in turn, limits the applicability of the methods and,
as a consequence, reduces the accuracy of approximation. In most of the cases, such 
restrictions are due to algorithmic peculiarities of the corresponding realizations.
{\looseness=1

}

Let consider results on some indicative problems covering a wide range 
of sample sizes and input dimensions:
\begin{itemize}
  \item the strength of the composite structure of the aircraft fuselage~\cite{copti};
  \item aerodynamic characteristics of the aircraft wing~\cite{quadpal};
  \item structure of the sand in the field~\cite{icfault}; and
  \item concrete compressive strength~\cite{concrete}.
\end{itemize}
Reference results (of approximation algorithms from MatLab and modeFrontier)
were obtained in 2010 during the work on the PhD thesis.
The approximation quality was measured using error~(\ref{errorDef}), 
calculated using the independent test set~$S_{\mathrm{test}}$.
The results of the comparison are given in Table~\ref{sampleTable} 
  (few methods do not work for the problem of approximation of the aircraft
  wing aerodynamic characteristics due to high input dimensionality).
  {\looseness=-1
  
  }



\section{Concluding Remarks}

\noindent
The problem of approximation of a multidimensional dependency has been considered
based on a
linear expansion in a dictionary of parametric functions.
The new methods have been proposed for solving subproblems that arise in the framework 
of approximation construction problem.
Each of these methods as well as their combination significantly increases the
approximation quality compared to the approximation quality obtained using standard methods.
Using the proposed algorithm, it was possible to solve important applied problems, 
(see, for example,~\cite{copti}).

{\small\frenchspacing
{%\baselineskip=10.8pt
\addcontentsline{toc}{section}{Литература}
\begin{thebibliography}{99}


\bibitem{surrogateModeling} %1
\Au{Forrester A., Sobester A., Keane~A.}
    Engineering design via surrogate modelling. A~practical guide.~---
    Wiley, 2008.

\bibitem{kuleshov2} %2
\Au{Kuleshov A., Bernstein A.}
  Cognitive technologies in adaptive models of complex plants~//
  Keynote Papers of 13th IFAC Symposium on Information Control Problems in Manufacturing (INCOM'09),
   2009. P.~70--81.

\bibitem{elementsOfStatLearning} %3
\Au{Hastie T., Tibshirani R., Friedman~J.}
  The elements of statistical learning: Data mining, inference, and prediction.~---
  Springer, 2008.

\bibitem{copti} %4
\Au{Grihon S., Alestra~S., Burnaev~E., Prikhodko~P.}
 Optimization of composite structure based on surrogate modelling of
   buckling analysis~//
Information Technologies and Systems Conference Proceedings, 2012. P.~41--47.

\bibitem{petrushev} %5
\Au{Petrushev~P.} 
{Approximation by ridge functions and neural networks}~// 
SIAM J.~Math. Anal. 30, 1998. P.~155--189.

\bibitem{sigmoidApprox1} %6
\Au{Pinkus A.}
  Approximation theory of the MLP model in neural networks~//
  Acta Numerica, 1999. Vol.~8. P.~143--195.
  

\bibitem{vapnik} %7
\Au{Vapnik V.\,N., Chervonenkis A.\,Ja.} {Ordered risk minimization (I and~II)}~// 
Autom. Remote Control, 1974. Vol.~34. P.~1226--1235; 1403--1412.

\bibitem{Nocedal} %8
\Au{Nocedal J., Wright~S.}
Numerical optimization.~--- 2nd ed.~---
  Springer, 2006. P.~664.

\bibitem{LM} %9
  \Au{Marquardt D.\,W.}
  An algorithm for least-squares estimation of nonlinear parameters~//
  J.~SIAM, 1963. Vol.~11. No.\,2. P.~431--441.

\bibitem{CART} %10
\Au{Breiman L.}
  Classification and regression trees.~--- Wadsworth, 1984.

\bibitem{initNW} %11
\Au{Nguyen D., Widrow B.}
  Improving the learning speed of 2-layer neural networks by choosing
      initial values of the adaptive weights~//
  IJCNN  Joint Conference (International), 1990. P.~21--26.

\bibitem{initSCAWI} %12
\Au{Drago G., Ridella S.}
  Statistically controlled activation weight initialization (SCAWI)~//
  Trans. Neur. Netw., IEEE Press, 1992. Vol.~3. No.\,4. P.~627--631.

\bibitem{SpherRnd} %13
\Au{Rubinstein R.\,Y.}
  Generating random vectors uniformly distributed inside and on the surface of different regions~//
  Eur. J.~Oper. Res., 1982. Vol.~10. No.\,2. P.~205--209.
  
  \bibitem{itas11_Yerofeyev} %14
\Au{Belyaev M.\,G., Burnaev E.\,V., Erofeev~P.\,D., Prikhodko~P.\,V.}
  Comparison of the efficiency of the initialization methods for non-linear regression models~//
  Information Technologies and Systems Conference Proceedings, 2011. P.~315--320.

\bibitem{data1} %15
\Au{Hedar A.-R.}
    Global optimization test problems~//
    {\sf http://www-optima.amp.i.kyoto-u.ac.jp/member/\linebreak student/hedar/Hedar\_files/TestGO.htm}.

\bibitem{data2} %16
\Au{Molga M., Smutnicki~C.}
    {Test functions for optimization needs}~//
    {\sf www.zsd.ict.pwr.wroc.pl/files/docs/\linebreak functions.pdf}.


\bibitem{golub} %17
\Au{Golub G.\,H., Pereyra  V.}
  The differentiation of pseudo-inverses and nonlinear least squares
  problems whose variables separate~//
  SIAM J.~Numer. Anal., 1973. Vol.~10. P.~413--432.

\bibitem{ruhe}
\Au{Ruhe A., Wedin P.\,A.}
  Algorithms for separable nonlinear least squares problems~//
  SIAM Review, 1980. Vol.~22. No.\,3. P.~318--337.

\bibitem{demidenko}
\Au{Demidenko E.\,Z. }
  Linear and non-linear regression.~---
  Finance and stochastics. 1981.

\bibitem{itas11_Belyaev}
\Au{Belyaev M.\,G., Lyubin A.\,D.}
Peculiarities of the optimization problem, which arises when constructing approximation of multidimensional function~//
Information Technologies and Systems Conference Proceedings, 2011. P.~415--422.

\bibitem{bayesianLearning}
\Au{Foresee D., Hagan M.}
  Gauss-Newton approximation to Bayesian learning~//
Conference (International) on Neural Networks Proceedings, 1997. Vol.~3. P.~1930--1935.

\bibitem{oldRegul}
\Au{Belyaev M.\,G., Burnaev E.\,V.}
  Adaptive regularization in the problem of multidimensional functions approximation~//
Information Technologies and Systems Conference Proceedings, 2009. P.~431--435.

\bibitem{DolanMore}
\Au{Dolan~E., More~J.}
  {Benchmarking optimization software with performance profiles}~//
    Math. Programming, Ser.~A, 2002. Vol.~91. P.~201--213.
  
  \bibitem{quadpal}
\Au{Chervonenkis A.\,Ya., Chernova S.\,S., Zykova~T.\,V.}
Applications of kernel ridge estimation to the problem of computing 
the aerodynamical characteristics of a passenger plane (in comparison 
with results obtained with artificial neural networks)~//  Automation Remote Control,
2011. Vol.~72. Iss.~5. P.~1061-1067.

  \bibitem{icfault}
  IC Fault dataset. 
{\sf  imperial.ac.uk/earthscienceand\linebreak engineering/research/perm/icfaultmodel}.


  \bibitem{concrete}
  Concrete Compressive Strength dataset.
{\sf   archive.ics.uci. edu/ml/datasets}.

\end{thebibliography} } }

\end{multicols}

%\vspace*{6pt}

%\hrule

%\vspace*{6pt}

\def\tit{АППРОКСИМАЦИЯ МНОГОМЕРНЫХ ЗАВИСИМОСТЕЙ НА ОСНОВЕ РАЗЛОЖЕНИЯ ПО СЛОВАРЮ ПАРАМЕТРИЧЕСКИХ
ФУНКЦИЙ}

\def\aut{М.\,Г.~Беляев$^1$, Е.\,В.~Бурнаев$^2$}

\titelr{\tit}{\aut}

\vspace*{6pt}

\noindent
$^1$Институт проблем передачи информации РАН, Московский физико-технический институт\\
$\hphantom{^1}$(государственный университет); ООО Датадванс, belyaev@iitp.ru\\[1pt]
$^2$Институт проблем передачи информации РАН, Московский физико-технический институт\\
$\hphantom{^1}$(государственный университет); ООО Датадванс, burnaev@iitp.ru


\vspace*{6pt}


\Abst{Рассматривается задача аппроксимации многомерной зависимости по конечному множеству
пар <<точка>>\,--\,<<значение функции в точке>>. Для решения этой задачи используется модель зависимости,
представляющая собой разложение по словарю нелинейных параметрических функций.
Построение аппроксимации, основанной на этой модели, может быть разбито на несколько подзадач:
выделение валидационной подвыборки, инициализация параметров функций словаря,
последующая настройка параметров функций словаря. Предложены эффективные методы решения этих
подзадач. Описанный подход демонстрирует высокое качество работы на ряде задач инженерного проектирования
и успешно применяется в реальных приложениях.} 

\label{end\stat}

\vspace*{-3pt}

\KW{нелинейная аппроксимация; словарь параметрических функций} 



\renewcommand{\figurename}{\protect\bf Рис.}
\renewcommand{\tablename}{\protect\bf Таблица}
\renewcommand{\bibname}{\protect\rmfamily Литература} %13


%\end{document}


%%%%%%%%%%%%%%%%%%%%%%%%%%%%%%%%%%%%%%%%%%%%%%%


%\def\stat{rez}
{%\hrule\par
%\vskip 7pt % 7pt
\raggedleft\Large \bf%\baselineskip=3.2ex
Р\,Е\,Ц\,Е\,Н\,З\,И\,И \vskip 17pt
    \hrule
    \par
\vskip 6pt plus 6pt minus 3pt }

%\thispagestyle{headings} %с верхним колонтитулом
%\thispagestyle{myheadings} %с нижним колонтитулом, но в верхнем РЕЦЕНЗИИ

\def\tit{НОВАЯ КНИГА И.\,Н.~СИНИЦЫНА, А.\,С.~ШАЛАМОВА <<ЛЕКЦИИ ПО ТЕОРИИ 
ИНТЕГРИРОВАННОЙ ЛОГИСТИЧЕСКОЙ ПОДДЕРЖКИ>> (М.: ТОРУС ПРЕСС, 2012. 624~с.)}

%1
\def\aut{Д.ф.-м.н., профессор С.\,Я.~Шоргин}

\def\auf{\ }

\def\leftkol{\ % РЕЦЕНЗИИ
}

\def\rightkol{ \ } 

%\def\leftkol{\ } % ENGLISH ABSTRACTS}

%\def\rightkol{\ } %ENGLISH ABSTRACTS}

%\def\leftkol{РЕЦЕНЗИИ}

%\def\rightkol{РЕЦЕНЗИИ}

\titele{\tit}{\aut}{\auf}{\leftkol}{\rightkol}
\vspace*{-18pt}


     \label{st\stat}

     \begin{multicols}{2}
     {\small
     {\baselineskip=10.1pt
     

      В книге представлено системное изложение теоретических основ одного из новейших 
направлений в \mbox{об\-ласти} экономики послепродажного обслуживания изделий наукоемкой 
продукции (ИНП) длительного пользования~--- интегрированной логистической поддержки
(ИЛП). 
{\looseness=1

}

Приведены также результаты новых работ, выполненных в Институте проблем информатики 
Российской академии наук в рамках научного направления <<Информационные технологии и 
анализ сложных сис\-тем>>.
 {%\looseness=1

}
     
      Излагаемые в книге научные подходы позво\-ляют карди\-наль\-но реформировать 
существующие системы производства и эксплуатации ИНП путем создания и внед\-ре\-ния 
методов рационального и оптимального управ\-ле\-ния процессами расходования 
вре\-мен\-н$\acute{\mbox{ы}}$х, 
мате\-ри\-аль\-ных, трудовых и других ресурсов на всех стадиях жизненного цикла изделий (ЖЦИ) по 
критериям экономической целесообразности и эф\-фек\-тив\-ности.
  {\looseness=1

}
    
      В книге приведен краткий обзор причин возник\-новения и
      развития CALS-методологии как основы 
современных международных стандартов по созданию и функционированию глобальных 
ин\-фор\-ма\-ци\-он\-но-ком\-му\-ни\-ка\-ци\-он\-ных систем, ее ключевых возможностей и эффективности 
результатов ее использования. 
Авторы %\linebreak 
предлагают ряд научных обоснований для разработки 
единой теории проектирования и управления систем ИЛП для полноценного использования 
преимуществ %\linebreak
 суще\-ст\-ву\-ющей методологии, определяют \mbox{общую} структурную схему 
комплексной системы <<ИНП-СППО>> и необходимость разработки для ее описания 
гибридных стохастических моделей.
{%\looseness=1

}

%\columnbreak
      
      Книга состоит из пяти частей, где последовательно излагается материал по каждой из 
следующих тем: <<Интегрированная логистическая поддержка>>, <<Теория гибридных 
стохастических систем и компьютерная поддержка исследований и разработок>>, <<Основы 
математического моделирования, анализа и синтеза систем послепродажного обслуживания>>, 
<<Определение и анализ показателей экспортного потенциала ИНП при проектировании>>, 
<<Задачи управления поддержкой послепродажного обслуживания>>, а также 
<<Моделирование инвестиционных процессов ИЛП в условиях неравновесных финансовых 
рынков>>. 
   
      В конце каждой главы приведены выводы и даны вопросы и задания для 
самоконтроля. В~приложениях содержатся основные определения по программам работ по 
анализу ИЛП, логистическим базам данных и компьютерным решениям, эквивалентной статистической 
линеаризации нелинейных преобразований ИЛП, справочный материал, а также развернутые 
уравнения для вероятностных характеристик.


      \def\leftkol{РЕЦЕНЗИИ}

\def\rightkol{РЕЦЕНЗИИ} 

      
      Книга заинтересует широкий круг специалистов и может быть использована научными 
проектными организациями в сфере промышленного производства ИНП. Большое количество 
иллюстраций, примеров и вопросов, обращенных к читателю, позволяет использовать книгу 
также в качестве учебного пособия для студентов и аспирантов машиностроительных, 
транспортных и~других специальностей, а также для самостоятельного изучения. 
{%\looseness=-1

}

Книга 
представляет несомненный интерес для специалистов и студентов в области прикладной 
математики и информатики.
    

}

}
\end{multicols}

%\newpage

%\end{document}

\include{obchak}

%\end{document}


\def\stat{authorsrus}
{%\hrule\par
%\vskip 7pt % 7pt
\raggedleft\Large \bf%\baselineskip=3.2ex
О\,Б\ \ А\,В\,Т\,О\,Р\,А\,Х \vskip 17pt
    \hrule
    \par
\vskip 21pt plus 8pt minus 4pt }


\def\tit{\ }

\def\aut{\ }

\def\auf{\ }

\def\leftkol{\ } % ENGLISH ABSTRACTS}

\def\rightkol{ОБ АВТОРАХ} %ENGLISH ABSTRACTS}

\titele{\tit}{\aut}{\auf}{\leftkol}{\rightkol}
      
            \label{st\stat}



\vspace*{24pt}

\begin{multicols}{2}




\noindent
\textbf{Архипов Олег Петрович} (р.\ 1948)~---
кандидат технических наук, директор Орловского филиала Института проб\-лем информатики
Российской академии наук
%302025, г.Орел, Московское шоссе, д.137

\vspace*{3pt}

\noindent
\textbf{Бирюкова Татьяна Константиновна} (р.\ 1968)~---
кандидат фи\-зи\-ко-ма\-те\-ма\-ти\-че\-ских наук, старший научный сотрудник Института проб\-лем информатики
Российской академии наук

\vspace*{3pt}

\noindent 
\textbf{Бобков  Сергей Геннадьевич} (р.\ 1955)~---
доктор технических наук,  заведующий отделением На\-уч\-но-ис\-сле\-до\-ва\-тель\-ско\-го 
института системных исследований Российской академии наук
%117218, Москва, Нахимовский просп., 36, к.1 

\vspace*{3pt}

\noindent \textbf{Васильев Николай Семенович} (р.\ 1952)~--- доктор 
фи\-зи\-ко-ма\-те\-ма\-ти\-че\-ских наук, профессор, 
МГТУ им.\ Н.\,Э.~Баумана 
%, Москва 105005, 2-я Бауманская ул., д.~5,

\vspace*{3pt}

\noindent
\textbf{Гершкович Максим Михайлович} (р.\ 1968)~---
старший научный сотрудник Института проб\-лем информатики
Российской академии наук

\vspace*{3pt}

\noindent 
\textbf{Дьяченко Юрий Георгиевич} (р.\ 1958)~--- кандидат технических наук, 
старший научный сотрудник Института проб\-лем информатики
Российской академии наук

\vspace*{3pt}

\noindent 
\textbf{Ерошенко Александр Андреевич} (р.\ 1989)~--- аспирант кафедры 
математической статистики факультета вычисли\-тельной математики и кибернетики 
Московского государственного университета им.\ М.\,В.~Ломоносова
%119991, Москва ГСП-1, Ленинские горы, д.\ 1, стр. 52

\vspace*{3pt}
 
\noindent 
\textbf{Захаров Виктор Николаевич} (р.\ 1948)~--- 
доктор технических наук, доцент, ученый секретарь Института проб\-лем информатики
Российской академии наук

\vspace*{3pt}

\noindent
\textbf{Зейфман Александр Израилевич} (р.\ 1954)~---
доктор фи\-зи\-ко-ма\-те\-ма\-ти\-че\-ских наук, профессор, 
заведующий кафедрой Вологодского государственного университета; 
старший научный сотрудник Института проб\-лем информатики
Российской академии наук; главный научный сотрудник ИСЭРТ Российской академии наук

\vspace*{3pt}

\noindent
\textbf{Зыкин Сергей Владимирович} (р.\ 1959)~--- 
доктор технических наук, профессор, заведующий лабораторией Института математики 
им.\ С.\,Л.~Соболева Сибирского отделения Российской академии наук, Новосибирск 
%630090, пр.\ ак.\ Коптюга, 4 

\vspace*{4pt}

\noindent
\textbf{Киреев Владимир Иванович} (р.\ 1938)~---
доктор фи\-зи\-ко-ма\-те\-ма\-ти\-че\-ских наук, профессор Московского 
государственного горного университета
%Адрес: Россия, 119991, г. Москва, Ленинский проспект, д. 6

%\columnbreak

\vspace*{4pt}

\noindent
\textbf{Козеренко Елена Борисовна} (р.\ 1959)~---
кандидат филологических наук, заведующая лабораторией Института проб\-лем информатики
Российской академии наук

\vspace*{4pt}

\noindent
\textbf{Королев Виктор Юрьевич} (р.\ 1954)~--- доктор
фи\-зи\-ко-ма\-те\-ма\-ти\-че\-ских наук, профессор кафедры математической 
статистики факультета вычисли\-тельной математики и кибернетики 
Московского государственного университета; 
ведущий научный сотрудник Института проб\-лем информатики
Российской академии наук

\vspace*{4pt}

\noindent
\textbf{Коротышева Анна Владимировна} (р.\ 1988)~---
старший преподаватель Вологодского государственного университета

\vspace*{4pt}

\noindent 
\textbf{Кун Де Турк} (р.\ 1981)~--- научный сотрудник 
исследовательской группы SMACS факультета телекоммуникаций и обработки информации
Университета Гента, Бельгия
%В-9000 Гент, Бельгия

\vspace*{4pt}

\noindent
\textbf{Лупенцов Олег Сергеевич} (р.\ 1986)~---
аспирант Омского государственного института сервиса
%Омск 644043, ул.\ Певцова 13

\vspace*{4pt}

\noindent
\textbf{Лучко Олег Николаевич} (р.\ 1961)~---
кандидат педагогических наук, профессор, заведующий кафедрой 
Омского государственного института сервиса
%Омск 644043, ул.\ Певцова 13

\vspace*{4pt}

\noindent
\textbf{Малашенко Юрий Евгеньевич} (р.\ 1946)~---
доктор фи\-зи\-ко-ма\-те\-ма\-ти\-че\-ских наук, заведующий сектором 
Вычислительного центра им.\ А.\,А.~Дородницына Российской академии наук
%Адрес: 119333, Москва, ул. Вавилова, 40,

\vspace*{4pt}

\noindent
\textbf{Маньяков Юрий Анатольевич} (р.\ 1984)~---
кандидат технических наук, научный сотрудник Орловского филиала Института проб\-лем информатики
Российской академии наук
%302025, г.Орел, Московское шоссе, д.137

\vspace*{4pt}

\noindent
\textbf{Маренко Валентина Афанасьевна} (р.\ 1951)~---
кандидат технических наук, доцент, старший научный сотрудник 
Института математики им.\ С.\,Л.~Соболева Сибирского отделения Российской академии наук
%Новосибирск 630090, пр. ак. Коптюга, 4 

\vspace*{3pt}

\noindent 
\textbf{Морозов Евсей Викторович} (р.\ 1947)~--- доктор 
фи\-зи\-ко-ма\-те\-ма\-ти\-че\-ских, профессор, ведущий научный сотрудник 
Института прикладных математических исследований Карельского научного центра Российской
академии наук; 
%%185910 Россия, Республика Карелия, г.\ Петрозаводск, ул.\ Пушкинская, 11
профессор Петрозаводского государственного университета, Петрозаводск
%185910 Россия, Республика Карелия, г.\ Петрозаводск, пр.\ Ленина, 33

%\pagebreak

\vspace*{3pt}

\noindent
\textbf{Назарова Ирина Александровна} (р.\ 1966)~---
кандидат фи\-зи\-ко-ма\-те\-ма\-ти\-че\-ских наук, 
научный сотрудник Вычислительного центра им.\ А.\,А.~Дородницына Российской академии наук 
%Адрес: 119333, Москва, ул. Вавилова, 40

\vspace*{3pt}

\noindent
\textbf{Павлов Игорь Валерианович} (р.\ 1945)~--- 
доктор фи\-зи\-ко-ма\-те\-ма\-ти\-че\-ских наук, профессор МГТУ им.\ Н.\,Э.~Баумана 
%Москва 105005, 2-я Бауманская ул., д.~5 

%\pagebreak

\vspace*{3pt}

\noindent 
\textbf{Потахина Любовь Викторовна} (р.\ 1989)~--- аспирантка
Института прикладных математических исследований Карельского научного центра
Российской академии наук; 
%%185910 Россия, Республика Карелия, г.\ Петрозаводск, ул.\ Пушкинская, 11
инженер Петрозаводского государственного университета, Петрозаводск
%185910 Россия, Республика Карелия, г.\ Петрозаводск, пр.\ Ленина, 33

\vspace*{3pt}

\noindent 
\textbf{Рождественский Юрий Владимирович} (р.\ 1952)~--- 
кандидат технических наук, заведующий сектором Института проб\-лем информатики
Российской академии наук

\vspace*{3pt}

\noindent 
\textbf{Синицын Игорь Николаевич} (р.\ 1940)~--- доктор технических наук,
профессор, заслуженный деятель\linebreak\vspace*{-12pt}

\columnbreak

\noindent
 науки РФ, заведующий отделом Института проб\-лем информатики
Российской академии наук

\vspace*{7pt}


\noindent
\textbf{Сиротинин Денис Олегович} (р.\ 1984)~---
кандидат технических наук, научный сотрудник Орловского филиала Института проб\-лем информатики
Российской академии наук
%302025, г.Орел, Московское шоссе, д.137

\vspace*{7pt}

%\columnbreak

\noindent 
\textbf{Соколов  Игорь Анатольевич} (р.\ 1954)~--- академик (действительный член) Российской 
академии наук, доктор технических наук, директор Института проб\-лем информатики
Российской академии наук

\vspace*{7pt}

\noindent
\textbf{Степченков Юрий Афанасьевич} (р.\ 1951)~---
кандидат технических наук, заведующий отделом Института проб\-лем информатики
Российской академии наук

\vspace*{7pt}

\noindent
\textbf{Сурков Алексей Викторович} (р.\ 1978)~--- 
старший научный сотрудник На\-уч\-но-ис\-сле\-до\-ва\-тель\-ско\-го 
института системных исследований Российской академии наук
%117218, Москва, Нахимовский просп., 36, к.1 

\vspace*{7pt}

\noindent 
\textbf{Шестаков Олег Владимирович} (р.\ 1976)~--- доктор 
фи\-зи\-ко-ма\-те\-ма\-ти\-че\-ских, доцент кафедры математической статистики 
факультета вычисли\-тельной математики и кибернетики Московского 
государственного университета им.\ М.\,В.~Ломоносова; 
%119991, Москва ГСП-1, Ленинские горы, д.\ 1, стр. 52
старший научный сотрудник Института проб\-лем информатики
Российской академии наук
%, Москва 119333, ул. Вавилова, д.~44, корп.~2

\vspace*{7pt}

\noindent 
\textbf{Шоргин Сергей Яковлевич} (р.\ 1952.)~--- доктор
фи\-зи\-ко-ма\-те\-ма\-ти\-че\-ских наук, профессор, заместитель директора Института 
проб\-лем информатики Российской академии наук





%%%%%%%%%%%%%%%%%%%%%%%%%%%%%%%%%%%%%%%%%%%%%%%%%%%%%%%%%%%%%%%%%%%%%%%%%%%%%%%




%\def\rightkol{ОБ АВТОРАХ}
%\def\leftkol{ОБ АВТОРАХ}

 \label{end\stat}





%\def\leftfootline{\small{\textbf{\thepage}
%\hfill ИНФОРМАТИКА И ЕЁ ПРИМЕНЕНИЯ\ \ \ том~7\ \ \ выпуск~1\ \ \ 2013}
%}%
% \def\rightfootline{\small{ИНФОРМАТИКА И ЕЁ ПРИМЕНЕНИЯ\ \ \ том~7\ \ \ выпуск~1\ \ \ 2013
%\hfill \textbf{\thepage}}}


%\thispagestyle{myheadings}



\end{multicols}

\newpage


%\vspace*{-48pt}
\begin{center}\LARGE
\textit{About Authors}
\end{center}

\thispagestyle{empty}
\def\tit{\ }

\def\aut{\ }

\def\auf{\ }


\def\leftkol{ABOUT AUTHORS}

\def\rightkol{ABOUT AUTHORS}

\vspace*{-18pt}

\titele{\tit}{\aut}{\auf}{\leftkol}{\rightkol}

%\vspace*{36pt}

\def\rightmark{{\noindent\hbox to \textwidth{\hfill\small ABOUT AUTHORS
%\hfill \large\bf\thepage
}}}
\def\leftmark{{\noindent\parbox{\textwidth}{
%\raggedleft\large\bf\thepage \hfill
\small\textrm{ABOUT AUTHORS}\hfill}}}


\def\leftfootline{\small{\textbf{\thepage}
\hfill ИНФОРМАТИКА И ЕЁ ПРИМЕНЕНИЯ\ \ \ том~6\ \ \ выпуск~2\ \ \ 2012}
}%
 \def\rightfootline{\small{ИНФОРМАТИКА И ЕЁ ПРИМЕНЕНИЯ\ \ \ том~6\ \ \ выпуск~2\ \ \ 2012
\hfill \textbf{\thepage}}}


\begin{multicols}{2}

\noindent
\textbf{Agalarov Yaver M.} (b.\ 1952)~--- Candidate of Science (PhD)
in technology, 
leading scientist, Institute of Informatics Problems, Russian Academy of Sciences

\vspace*{5pt}


  \noindent
\textbf{Bosov Alexey V.} (b.\ 1969)~--- Doctor of Science in technology, Head of
Laboratory, Institute of Informatics Problems, Russian Academy of Sciences

\vspace*{5pt}


\noindent
\textbf{Dulin Sergey K.} (b.\ 1950)~--- Doctor of Science in technology, 
professor, senior scientist, Institute of Informatics Problems, Russian Academy of Sciences

\vspace*{5pt}

\noindent
\textbf{Gorshenin Andrey K.}~--- (b.\ 1986)~--- Candidate of Science (PhD)
in physics and mathematics,
senior scientist, Institute of Informatics Problems, Russian Academy of Sciences

\vspace*{5pt}

\noindent
\textbf{Kalenov Nikolay E.}  (b.\ 1945)~--- Doctor of Science in technology,
professor, Director, Library for Natural Sciences,  Russian Academy of Sciences 

\vspace*{5pt}

\noindent
\textbf{Kalinichenko Leonid A.} (b.\ 1937)~--- Doctor of Science in physics and mathematics, 
professor, Honored scientist of RF, 
Head of Laboratory, Institute of Informatics Problems, Russian Academy of Sciences 

\vspace*{5pt}

\noindent
\textbf{Karpov Alexey A.} (b.\ 1978)~--- Candidate of Science (PhD) in technology, 
senior scientist, St.\ Petersburg Institute for
Informatics and Automation,  Russian Academy of Sciences

\vspace*{5pt}

\noindent
\textbf{Kuznetsov Igor P.} (b.\ 1938)~--- Doctor of Science in technology, 
professor, principal scientist, Institute of Informatics Problems, Russian Academy of Sciences

\vspace*{5pt}


\noindent
\textbf{Markova Natalia A.} (b.\ 1950)~--- Candidate of Science (PhD) in
physics and mathematics, leading scientist,  
Institute of Informatics Problems, Russian Academy of Sciences

\vspace*{5pt}

\noindent
\textbf{Nikolaev Andrey V.} (b.\ 1985)~--- Candidate of Science (PhD) in technology, 
senior lecturer, Tchaikovsky Technological Institute, Branch of the Izhevsk State Technical 
University

\vspace*{6pt}

\noindent
\textbf{Pavlov Igor V.} (b.\ 1945)~---  Doctor of Science in physics and mathematics,
professor, Bauman Moscow State Technical University

\vspace*{6pt}

%\columnbreak

\noindent
\textbf{Rozenberg Igor N.} (b.\ 1965)~--- Doctor of Science in technology, 
First Deputy Director General, Research \& Design Institute for Information 
Technology, Signalling and Telecommunications on Railway Transport (JSC NIIAS)

\vspace*{6pt}


\noindent
\textbf{Semenov Konstantin K.} (b.\ 1986)~--- MPhil, 
associate professor, St.\ Petersburg State Polytechnical University

\vspace*{6pt}

\noindent
\textbf{Sharnin Mikhail M.} (b.\ 1959)~--- Candidate of Science (PhD) 
in technology, senior scientist, Institute of Informatics Problems, Russian Academy of Sciences

\vspace*{6pt}

\noindent 
\textbf{Shestakov Oleg V.} (b.\ 1976)~--- Candidate of Science (PhD) in physics and mathematics,
associate professor, Department of Mathematical Statistics, Faculty of Computational Mathematics and Cybernetics,
M.\,V.~Lomonosov Moscow State University; senior scientist, Institute of Informatics Problems, 
Russian Academy of Sciences

\vspace*{6pt}

\noindent
\textbf{Stupnikov Sergey A.} (b.\ 1978)~--- Candidate of Science (PhD) in technology, 
senior scientist, Institute of Informatics Problems, Russian Academy of Sciences 

\vspace*{6pt}

\noindent
\textbf{Umansky Vladimir I.} (b.\ 1954)~--- Candidate of Science (PhD) in technology, 
Director General, ``IntechGeoTrans'' Closed Joint Stock Company

\vspace*{6pt}

\noindent
\textbf{Zhevnerchuk Dmitry V.} (b.\ 1978)~--- Candidate of Science (PhD) in technology, 
associate professor, Tchaikovsky Technological Institute, Branch of the Izhevsk State 
Technical University

%\vspace*{6pt}

\def\leftfootline{\small{\textbf{\thepage}
\hfill ИНФОРМАТИКА И ЕЁ ПРИМЕНЕНИЯ\ \ \ том~6\ \ \ выпуск~2\ \ \ 2012}
}%
 \def\rightfootline{\small{ИНФОРМАТИКА И ЕЁ ПРИМЕНЕНИЯ\ \ \ том~6\ \ \ выпуск~2\ \ \ 2012
\hfill \textbf{\thepage}}}



%\thispagestyle{myheadings}

\end{multicols}
\newpage

\def\stat{rekl}
%\label{preobr}

%\def\tit{АКАДЕМИК ПУГАЧЁВ  ВЛАДИМИР СЕМЁНОВИЧ\\
%25.03.1911--25.03.1998}


%   \vspace*{-48pt}
%   \begin{center}\LARGE
%Академик Пугачёв  Владимир Семёнович\\ (25.03.1911--25.03.1998)
%   \end{center}
   
   %\vspace*{2.5mm}
   
   \begin{center}

{\prgsh\LARGE
ОБЪЯВЛЕНИЯ О КОНФЕРЕНЦИЯХ}

\end{center}
%\hrule

\vspace*{6pt}

   
   \vspace*{10mm}
   
   \thispagestyle{empty}

\noindent
\begin{tabular}{cc}
%\begin{center}
\multicolumn{1}{c}{\raisebox{-40pt}[0pt][0pt]{\mbox{%
\epsfxsize=33mm
\epsfbox{vspu.eps}
}}}
%\end{center}
&
\tabcolsep=0pt\begin{tabular}{c}
{\prg{\Large\textbf{XII Всероссийское совещание}}}\\[6pt]
{\prg{\Large\textbf{по проблемам управления}}}\\[12pt]
{\prg{\large 16--19 июня 2014~г.}}\\[6pt] 
{\prg{\large Институт проблем управления имени В.\,А.~Трапезникова РАН}}\\[6pt]
{\prg{\large Москва, Россия}}
\end{tabular}
\end{tabular}

\vspace*{60pt}

     
 { %\large    
 XII Всероссийское совещание по проблемам управления (ВСПУ XII), посвященное 75-летию 
Института проблем управления (ИПУ) имени В.\,А.~Трапезникова РАН, проводится 16--19~июня 
2014~г.\ 
в ИПУ РАН (г.~Москва, Россия). ВСПУ XII организуется ИПУ РАН при поддержке РФФИ, Отделения 
энергетики, машиностроения, механики и процессов управления Российской академии наук, 
Российского 
национального комитета по автоматическому управлению, Академии навигации и управ\-ле\-ния 
движением, 
Научного совета РАН по комплексным проблемам управления и автоматизации, Совета по 
мехатронике и робототехнике РАН. Официальный язык Совещания~--- русский.

\vspace*{24pt}
     
     \textbf{Направления работы}
     \begin{enumerate}[1.]
\item Теория систем управления
\item Управление подвижными объектами и навигация
\item Интеллектуальные системы управления
\item Управление в промышленности, транспортом и логистикой
\item Управление системами междисциплинарной природы
\item Средства измерения, вычислений и контроля в управлении
\item Системный анализ и принятие решений в задачах управления
\item Информационные технологии в управлении
\item Проблемы образования в области управления: современное содержание и технологии обучения
\end{enumerate}

\vspace*{24pt}

     Подробная информация о Совещании находится на сайте {\sf http://vspu2014.ipu.ru}. Срок 
окончательной подачи докладов через систему подачи докладов на сайте~--- \textbf{30~ноября} 
2013~г.
}

%\end{document}

%   \vspace*{-48pt}

\begin{center}
\vspace*{6pt}
\mbox{%
\epsfxsize=53.502mm
\epsfbox{foto-1.eps}
}
\end{center}

\vspace*{6pt} %Академик


   \begin{center}
\fbox{\Large\textbf{Профессор Игорь Алексеевич Ушаков}}\\[12pt]
\textbf{\large 22.01.1935--27.02.2015}
   \end{center}


   %\vspace*{2.5mm}

   \vspace*{5mm}

   \thispagestyle{empty}

%\

%\vspace*{-12pt}


Редакционный совет и редакционная коллегия журнала <<Информатика и~её применения>> с~глубоким прискорбием извещают, что 27~февраля 2015~г.\ после тяжелой
и~продолжительной болезни скончался Игорь Алексеевич Ушаков~--- доктор технических наук, профессор, член редколлегии журнала <<Информатика и ее применения>>.

Игорь Алексеевич Ушаков окончил Московский авиационный институт, в~1963~г.\ защитил кандидатскую, а~в~1968~г.~--- докторскую диссертацию. С~1958 по 1989~гг.\ работал в~ряде научно-исследовательских организаций СССР, в~том числе руководил отделами в~НИИ АА и~ВЦ АН СССР; с 1969 по 1989 гг. преподавал в~МФТИ (был профессором, а~затем заведующим кафедрой) и~в~МЭИ. С~1989~г.~---- в~США: являлся профессором университета Дж.\ Вашингтона, университета Дж.\ Мэйсона и~Калифорнийского университета, сотрудником компаний MCI, Qualcomm и Hughes.

И.\,А.~Ушаков с момента основания журнала <<Надежность и~контроль качества>> был заместителем ответственного редактора, а~затем на протяжении многих лет членом редколлегии. В~2006~г.\ основал электронный международный журнал ``Reliability: Theory \& Application'', главным редактором которого оставался до конца жизни.

Учебниками и справочниками по теории надежности, написанными И.\,А.~Ушаковым, пользовались и~пользуются несколько поколений ученых и~специалистов в~разных странах мира.

Игорь Алексеевич всегда уделял огромное внимание работе с~молодежью; более~50 его учеников защитили докторские и~кандидатские диссертации.

И.\,А.~Ушаков вел активную научно-про\-све\-ти\-тель\-скую деятельность. В~частности, он был одним из организаторов и~руководителей Московского кабинета качества и~надежности при Политехническом музее (целью этого Кабинета было оказание консультаций работникам промышленных предприятий и~чтение курсов лекций для инженеров, занимающихся проблемой надежности). Находясь в~США, И.\,А.~Ушаков создал международный ин\-тер\-нет-фо\-рум им.\ Б.\,В.~Гнеденко, объединивший около~400~видных специалистов по приложениям теории вероятностей и~математической статистики, преимущественно в~об\-ласти теории надежности и~анализа риска, из десятков стран мира; коллективным членов этого Форума является и~наш журнал. Цели Форума~--- содействие контактам между специалистами из разных стран, организация обмена профессиональными 
новостями и~информацией (новые публикации, предстоящие события и~др.). Также необходимо отметить большое число на\-уч\-но-по\-пу\-ляр\-ных работ, опубликованных И.\,А.~Ушаковым.

И.\,А.~Ушаков обладал большим личным обаянием, имел широкий круг интересов. Все знавшие И.\,А.~Ушакова всегда будут помнить его как замечательного ученого и~прекрасного человека.

\bigskip

Редакционный совет и редакционная коллегия журнала <<Информатика и~её применения>> 
выражают глубокие соболезнования родным и близким покойного, всем, кто его знал и~работал с~ним.


\def\stat{cont-rus}
{%\hrule\par
%\vskip 7pt % 7pt
\vspace*{-24pt}
\raggedleft\Large \bf%\baselineskip=3.2ex
Правила подготовки рукописей  для публикации в журнале
<<Информатика~и~её~применения>> \vskip 8pt
    \hrule
    \par
\vskip 14pt plus 6pt minus 3pt }

\label{st\stat}

\def\tit{\ }

\def\aut{\ }
\def\auf{\ }

\def\leftkol{\ }
% Правила подготовки рукописей  для публикации в журнале
%<<Информатика и её применения>>

\def\rightkol{\ }
%Правила подготовки рукописей  для публикации в журнале
%<<Информатика и её применения>>}


\titele{\tit}{\aut}{\auf}{\leftkol}{\rightkol}


\vspace*{-60pt}
{ %\small

Журнал <<Информатика и её применения>>
публикует теоретические, обзорные и дискуссионные статьи,
посвященные научным исследованиям и разработкам в области
информатики и ее приложений.

Журнал издается на русском языке. По специальному решению
редколлегии отдельные статьи могут печататься на английском языке.

Тематика журнала охватывает следующие направления:
\begin{itemize}
\item теоретические основы информатики;\\[-15pt]
      \item
математические методы исследования сложных систем и процессов;\\[-15pt]
           \item
информационные системы и сети;\\[-15pt]
                \item
информационные технологии;\\[-15pt]
                     \item
архитектура и программное обеспечение вычислительных комплексов и сетей.\\[-15pt]
\end{itemize}


\noindent
\begin{enumerate}[1.]
\item В журнале печатаются статьи, содержащие результаты, ранее не опубликованные и
не предназначенные к одновременной публикации в других изданиях.

%Публикация не должна нарушать закон об авторских правах.
Публикация предоставленной автором(ами) рукописи не должна нарушать 
положений глав~69, 70 раздела~VII части~IV Гражданского кодекса, 
которые определяют права на результаты интеллектуальной деятельности 
и~средства индивидуализации, в~том числе авторские права, в~РФ.

Ответственность за нарушение авторских прав, в~случае предъявления претензий к~редакции журнала,  
несут авторы статей.



Направляя рукопись в редакцию, авторы сохраняют свои права на данную
рукопись и при этом передают учредителям и редколлегии журнала неисключительные права на
издание статьи на русском языке 
(или на языке статьи, если он отличен от рус\-ско\-го) и~на перевод ее на английский
язык, а~также на
ее распространение в России и за рубежом. 
Каждый автор должен представить в~редакцию подписанный 
с~его стороны <<Лицензионный договор о~передаче неисключительных прав 
на использование произведения>>, текст которого размещен по адресу 
{\sf http://www.ipiran.ru/publications/licence.doc}. 
Этот договор может быть пред\-став\-лен в~бумажном (в~2-х экз.)\ 
или в~электронном виде (отсканированная копия заполненного и~подписанного документа).




Редколлегия вправе запросить у авторов экспертное заключение о возможности
пуб\-ли\-ка\-ции пред\-став\-лен\-ной статьи в открытой печати.\\[-13.5pt]

\item К статье прилагаются данные автора (авторов) (см.\ п.~8). При наличии нескольких
авторов указывается фамилия автора, ответственного за переписку с редакцией.\\[-13.5pt]

\item Редакция журнала осуществляет экспертизу присланных статей в соответствии с
принятой в журнале процедурой рецензирования.

Возвращение рукописи на доработку не означает ее принятия к печати.

Доработанный вариант с ответом на замечания рецензента необходимо прислать в
редакцию.\\[-13.5pt]

\item Решение редколлегии о публикации статьи или ее отклонении сообщается авторам.

Редколлегия может также направить авторам текст рецензии на их статью. Дискуссия по
поводу отклоненных статей не ведется.\\[-13.5pt]

%\pagebreak

\item Редактура статей высылается авторам для просмотра. Замечания к редактуре должны
быть присланы авторами в кратчайшие сроки.\\[-13.5pt]

\item Рукопись предоставляется в электронном виде в форматах MS WORD (.doc или
.docx) или \LaTeX\  (.tex), дополнительно~--- в формате .pdf, на дискете, лазерном диске
или электронной почтой. Предоставление бумажной рукописи необязательно.\\[-13.5pt]

\item При подготовке рукописи в MS Word рекомендуется использовать следующие
настройки.

Параметры страницы:
формат~--- А4; ориентация~--- книжная; поля (см): внутри~--- 2,5, снаружи~--- 1,5,
сверху~--- 2, снизу~--- 2, от края до нижнего колонтитула~--- 1,3.

Основной текст: стиль~--- <<Обычный>>, шрифт~--- Times New Roman, размер~---
14~пунк\-тов, абзацный отступ~--- 0,5~см, 1,5~интервала, выравнивание~--- по ширине.

\pagebreak

\def\leftkol{Правила подготовки рукописей  для публикации в журнале
<<Информатика и её применения>>}

\def\rightkol{Правила подготовки рукописей  для публикации в журнале
<<Информатика и её применения>>}



Рекомендуемый объем рукописи~--- не свыше 10~страниц указанного формата.
При превышении указанного объема редколлегия вправе потребовать от 
автора сокращения объема рукописи.


Сокращения слов, помимо стандартных, не допускаются. Допускается минимальное
количество аббревиатур.


Все страницы рукописи нумеруются.

Шаблоны оформления представлены в интернете:

\noindent
 {\sf
http://www.ipiran.ru/journal/template\_iiep\_ssi\_2024.zip}\\[-14pt]

\item Статья должна содержать следующую информацию на {\bfseries\textit{русском и
английском языках}}:\\[-16pt]

\begin{itemize}
\item название статьи;\\[-15pt]
\item Ф.И.О.\ авторов, на английском можно только имя и фамилию;\\[-15pt]
\item место работы, с указанием почтового адреса организации и электронного адреса каждого
автора;\\[-15pt]
\item сведения об авторах, в соответствии с форматом, образцы которого
представлены на страницах:



\def\leftfootline{\small{\textbf{\thepage}
\hfill ИНФОРМАТИКА И ЕЁ ПРИМЕНЕНИЯ\ \ \ том\ 18\ \ \ выпуск\ 3\ \ \ 2024}
}%
 \def\rightfootline{\small{ИНФОРМАТИКА И ЕЁ ПРИМЕНЕНИЯ\ \ \ том\ 18\ \ \ выпуск\ 3\ \ \ 2024
\hfill \textbf{\thepage}}}



{\sf http://www.ipiran.ru/journal/issues/2013\_07\_01/authors.asp} и

{\sf http://www.ipiran.ru/journal/issues/2013\_07\_01\_eng/authors.asp};
\item аннотация (не менее 100~слов на каждом из языков). Аннотация~--- это краткое
резюме работы, которое может публиковаться отдельно. Она является основным
источником информации в~ин\-фор\-ма\-ци\-он\-ных системах и базах данных. Английская
аннотация должна быть оригинальной, может не быть дословным переводом русского
текста и должна быть написана хорошим английским языком. В~аннотации не должно
быть ссылок на литературу и, по возможности, формул;\\[-15pt]
\item ключевые слова~--- желательно из принятых в мировой
на\-уч\-но-тех\-ни\-че\-ской литературе тематических тезаурусов. Предложения не
могут быть ключевыми словами;\\[-15pt]
\item источники финансирования работы (ссылки на гранты, проекты,
поддерживающие организации и~т.\,п.).
\end{itemize}



%\pagebreak

\item  Требования к спискам литературы.\\[-14pt]

Ссылки на литературу в тексте статьи нумеруются (в квадратных скобках) и
располагаются в каждом из списков литературы в порядке  первых упоминаний. Если источник имеет DOI и/или EDN,
то их необходимо указывать.

Списки литературы представляются в двух вариантах:\\[-14pt]


\noindent
\begin{enumerate}[(1)]
\item \textbf{Список литературы к русскоязычной части}. Русские и английские
работы~---  на языке и в алфавите оригинала;\\[-14.5pt]
\item  \textbf{References}. Русские работы и работы на других языках~--- в латинской
транслитерации с переводом на английский язык; английские работы и работы на других
языках~--- на языке оригинала.
\end{enumerate}

Необходимо для составления списка ``References'' пользоваться размещенной на сайте
{\sf http://www. translit.net/ru/bgn/} бесплатной программой транслитерации русского
 текста в~латиницу. %, при этом в~за\-клад\-ке <<варианты\ldots>> следует выбратьопцию BGN.

Список литературы ``References'' приводится полностью отдельным блоком, повторяя все
позиции из списка литературы к русскоязычной части, независимо от того, имеются или
нет в нем иностранные источники. Если в списке литературы к русскоязычной части есть
ссылки на иностранные публикации, набранные латиницей, они полностью повторяются в
списке ``References''.

Ниже приведены примеры ссылок на различные виды публикаций в списке ``References''.

\def\leftfootline{\small{\textbf{\thepage}
\hfill ИНФОРМАТИКА И ЕЁ ПРИМЕНЕНИЯ\ \ \ том\ 18\ \ \ выпуск\ 3\ \ \ 2024}
}%
 \def\rightfootline{\small{ИНФОРМАТИКА И ЕЁ ПРИМЕНЕНИЯ\ \ \ том\ 18\ \ \ выпуск\ 3\ \ \ 2024
\hfill \textbf{\thepage}}}

{\small

\noindent
\textbf{Описание статьи из журнала:}

\Aue{Zagurenko, A.\,G., V.\,A.~Korotovskikh, A.\,A.~Kolesnikov, A.\,V.~Timonov, and D.\,V.~Kardymon}. 2008.
Tekhniko-ekonomicheskaya optimizatsiya dizayna gidrorazryva plasta [Technical and
economic optimization of the design
of hydraulic fracturing]. \textit{Neftyanoe hozyaystvo} [\textit{Oil Industry}] 11:54--57.

\Aue{Zhang, Z., and D.~Zhu}. 2008. Experimental research on the localized
electrochemical micromachining. \textit{Russ. J.~Electrochem.}  44(8):926--930.
{\sf doi:10.1134/S1023193508080077}.

\noindent
\textbf{Описание статьи из электронного журнала:}

\Aue{Swaminathan, V., E.~Lepkoswka-White, and B.\,P.~Rao}. 1999. Browsers or buyers in cyberspace? An
investigation of electronic factors influencing electronic exchange. \textit{JCMC}
5(2). Available at: {\sf http://www.ascusc.org/jcmc/vol5/issue2/} (accessed April~28, 2011).

\def\leftkol{Правила подготовки рукописей  для публикации в журнале
<<Информатика и её применения>>}

\def\rightkol{Правила подготовки рукописей  для публикации в журнале
<<Информатика и её применения>>}


\noindent
\textbf{Описание статьи из продолжающегося издания (сборника трудов):}

\Aue{Astakhov, M.\,V., and T.\,V.~Tagantsev}. 2006. Eksperimental'noe
issledovanie prochnosti soedineniy ``stal'--kompozit'' [Experimental study of
the strength of joints ``steel--composite'']. \textit{Trudy MGTU
``Matematicheskoe modelirovanie slozhnykh tekh\-ni\-che\-skikh sistem''}
[\textit{Bauman MSTU ``Mathematical Modeling of Complex Technical
Systems'' Proceedings}]. 593:125--130.


\pagebreak



\noindent
\textbf{Описание материалов конференций:}

\Aue{Usmanov, T.\,S., A.\,A.~Gusmanov, I.\,Z.~Mullagalin, R.\,Ju.~Muhametshina, A.\,N.~Chervyakova, and
A.\,V.~Sveshnikov}. 2007. Osobennosti proektirovaniya razrabotki mestorozhdeniy
s primeneniem gidrorazryva
plasta [Features of the design of field development with the use of hydraulic fracturing].
\textit{Trudy 6-go
Mezhdu\-na\-rod\-no\-go Simpoziuma ``Novye resursosberegayushchie tekhnologii nedropol'zovaniya i povysheniya
neftegazootdachi''} [\textit{6th  Symposium (International) ``New Energy Saving Subsoil Technologies and
the Increasing of the Oil and Gas Impact'' Proceedings}]. Moscow. 267--272.



\def\leftfootline{\small{\textbf{\thepage}
\hfill ИНФОРМАТИКА И ЕЁ ПРИМЕНЕНИЯ\ \ \ том\ 18\ \ \ выпуск\ 3\ \ \ 2024}
}%
 \def\rightfootline{\small{ИНФОРМАТИКА И ЕЁ ПРИМЕНЕНИЯ\ \ \ том\ 18\ \ \ выпуск\ 3\ \ \ 2024
\hfill \textbf{\thepage}}}



\noindent
\textbf{Описание книги (монографии, сборники):}



Lindorf, L.\,S., and L.\,G.~Mamikoniants, eds. 1972.
\textit{Ekspluatatsiya turbogeneratorov s neposredstvennym
okhlazhdeniem} [\textit{Operation of turbine generators with direct cooling}].
Moscow: Energy Publs. 352~p.


\Aue{Latyshev, V.\,N.} 2009. \textit{Tribologiya rezaniya. Kn.~1: Friktsionnye protsessy
pri rezanii metallov}
[\textit{Tribology of cutting. Vol.~1: Frictional processes in metal cutting}]. Ivanovo: Ivanovskii
State Univ. 108~p.

\def\leftkol{Правила подготовки рукописей  для публикации в журнале
<<Информатика и её применения>>}

\def\rightkol{Правила подготовки рукописей  для публикации в журнале
<<Информатика и её применения>>}

\noindent
\textbf{Описание переводной книги}
(в списке литературы к русскоязычной части необходимо указать:~/ Пер.\ с англ.~---
после названия книги, а в конце ссылки указать оригинал книги в круглых скобках):
\begin{enumerate}[1.]
\item  В русскоязычной части:

\def\leftfootline{\small{\textbf{\thepage}
\hfill ИНФОРМАТИКА И ЕЁ ПРИМЕНЕНИЯ\ \ \ том\ 18\ \ \ выпуск\ 3\ \ \ 2024}
}%
 \def\rightfootline{\small{ИНФОРМАТИКА И ЕЁ ПРИМЕНЕНИЯ\ \ \ том\ 18\ \ \ выпуск\ 3\ \ \ 2024
\hfill \textbf{\thepage}}}

\Au{Тимошенко С.\,П., Янг Д.\,Х., Уивер~У.}
Колебания в инженерном деле~/ Пер.\ с англ.~--- М.: Машиностроение, 1985. 472~с.
(\Au{Timoshenko~S.\,P., Young~D.\,H., Weaver~W.}
Vibration problems in engineering.~--- 4th ed.~--- New York, NY, USA: Wiley, 1974. 521~p.)\\[-13.5pt]
\item  В англоязычной части:

\Aue{Timoshenko, S.\,P., D.\,H.~Young, and W.~Weaver}.
1974. \textit{Vibration problems in engineering}. 4th ed. New York: 
Wiley. 521~p.
\end{enumerate}

\vspace*{-3pt}


\noindent
\textbf{Описание неопубликованного документа:}


\Aue{Latypov, A.\,R., M.\,M.~Khasanov, and V.\,A.~Baikov}.
2004 (unpubl.). Geologiya i~dobycha (NGT GiD) [Geology and production (NGT GiD)]. Certificate on official registration of the computer program
No.\,2004611198. 

\noindent
\textbf{Описание интернет-ресурса:}


Pravila tsitirovaniya istochnikov [Rules for the citing of sources]. Available at: {\sf
http://www.scribd.com/doc/1034528/} (accessed February~7, 2011).

%\pagebreak

\noindent
\textbf{Описание диссертации или автореферата диссертации:}

\Aue{Semenov, V.\,I.}
2003. Matematicheskoe modelirovanie plazmy v sisteme kompaktnyy tor [Mathematical
modeling of the plasma in the compact torus].  Moscow.  D.Sc.\ Diss. 272~p.

\Aue{Kozhunova, O.\,S.} 2009. Tekhnologiya razrabotki semanticheskogo
slovarya informatsionnogo monitoringa [Technology of development of
semantic dictionary of information monitoring system].  Moscow: IPI RAN. PhD Thesis. 23~p.


\noindent
\textbf{Описание ГОСТа:}

GOST 8.586.5-2005. 2007. Metodika vypolneniya izmereniy. Izmerenie raskhoda i~kolichestva zhidkostey i~gazov
s~pomoshch'yu standartnykh suzhayushchikh ustroystv [Method of measurement.
Measurement of flow rate and volume of liquids and gases by means of orifice devices]. Moscow:
Standardinform  Publs. 10~p.

\noindent
\textbf{Описание патента:}

\Aue{Bolshakov, M.\,V., A.\,V.~Kulakov, A.\,N.~Lavrenov, and M.\,V.~Palkin}.
2006. Sposob orientirovaniya po krenu letatel'nogo
apparata s opti\-che\-skoy golovkoy
samonavedeniya [The way to orient on the roll of aircraft with optical homing head].
Patent RF No.\,2280590.
}

\item Присланные в редакцию материалы авторам не возвращаются.\\[-13.5pt]

\item При отправке файлов по электронной почте просим придерживаться следующих
правил:
\begin{itemize}
\item указывать в поле subject (тема) название журнала и фамилию автора;\\[-13.5pt]
\item указывать в тексте письма название статьи, авторов и~журнал, в~который направляется статья;\\[-13.5pt]
\item использовать attach (присоединение);\\[-13.5pt]
\item в состав электронной версии статьи должны входить: файл, содержащий текст
статьи, и файл(ы), содержащий(е) иллюстрации.\\[-13.5pt]
\end{itemize}

\item Журнал <<Информатика и её применения>> является некоммерческим изданием.
Плата за публикацию не взимается, гонорар авторам не выплачивается.
\end{enumerate}



\def\leftfootline{\small{\textbf{\thepage}
\hfill ИНФОРМАТИКА И ЕЁ ПРИМЕНЕНИЯ\ \ \ том\ 18\ \ \ выпуск\ 3\ \ \ 2024}
}%
 \def\rightfootline{\small{ИНФОРМАТИКА И ЕЁ ПРИМЕНЕНИЯ\ \ \ том\ 18\ \ \ выпуск\ 3\ \ \ 2024
\hfill \textbf{\thepage}}}


\vspace*{-1mm}

\begin{center}

\textbf{Адрес редакции журнала <<Информатика и её применения>>:} \\




Москва 119333, ул.~Вавилова, д.~44, корп.~2, ФИЦ ИУ РАН\\[-10pt]

\

Тел.: +7\,(499)\,135-86-92\ \ Факс:  +7\,(495)\,930-45-05\\[-10pt]

 \

e-mail:   {\sf iiep@frccsc.ru} (Стригина Светлана Николаевна)\\[-10pt]

\

{\sf http://www.ipiran.ru/journal/issues/}
\end{center}
}


\def\leftkol{Правила подготовки рукописей  для публикации в журнале
<<Информатика и её применения>>}

\def\rightkol{Правила подготовки рукописей  для публикации в журнале
<<Информатика и её применения>>}


\def\leftfootline{\small{\textbf{\thepage}
\hfill ИНФОРМАТИКА И ЕЁ ПРИМЕНЕНИЯ\ \ \ том\ 18\ \ \ выпуск\ 3\ \ \ 2024}
}%
 \def\rightfootline{\small{ИНФОРМАТИКА И ЕЁ ПРИМЕНЕНИЯ\ \ \ том\ 18\ \ \ выпуск\ 3\ \ \ 2024
\hfill \textbf{\thepage}}} 
\def\stat{podg-e}
{%\hrule\par
%\vskip 7pt % 7pt
\vspace*{-24pt}
\raggedleft\Large \bf%\baselineskip=3.2ex
Requirements for manuscripts submitted to Journal
``Informatics~and~Applications'' \vskip 8pt
    \hrule
    \par
\vskip 21pt plus 6pt minus 3pt }

\label{st\stat}

\def\tit{\ }

\def\aut{\ }
\def\auf{\ }

\def\leftkol{\ }

\def\rightkol{\ }
%Requirements for manuscripts submitted to Journal
%``Informatics~and~Applications''}

\titele{\tit}{\aut}{\auf}{\leftkol}{\rightkol}

\def\leftfootline{\small{\textbf{\thepage}
\hfill INFORMATIKA I EE PRIMENENIYA~--- INFORMATICS AND APPLICATIONS\ \ \ 2019\
\ \ volume~13\ \ \ issue\ 4}
}%
 \def\rightfootline{\small{INFORMATIKA I EE PRIMENENIYA~--- INFORMATICS AND APPLICATIONS\ \ \ 2019\ \ \ volume~13\ \ \ issue\ 4
\hfill \textbf{\thepage}}}

\vspace*{-60pt}

{\small

\noindent
Journal ``Informatics and Applications'' (Inform.\ Appl.)
publishes theoretical, review, and discussion
articles on the research and development in the
field of informatics and its applications.

The journal is published in Russian.
By a special decision of the editorial
board, some articles can be published in English.


The topics covered include the following areas:
\begin{itemize}
               \item
     theoretical fundamentals of informatics; \\[-14pt]
\item
mathematical methods for studying complex systems and processes; \\[-14pt]
\item
information systems and networks;\\[-14pt]
\item
information technologies; and \\[-14pt]
\item
architecture and software of computational complexes and networks. \\[-14pt]
\end{itemize}

\noindent
\begin{enumerate}[1.]
\item The Journal publishes original articles which have not been published before and are not
intended for simultaneous publication in other editions. An article submitted to the Journal must not violate the
Copyright law. Sending the manuscript to the Editorial Board, the authors retain all rights of the
owners of the manuscript and transfer the nonexclusive rights to publish the article in Russian
(or the language of the article, if not Russian) and its distribution in Russia and abroad to the
Founders and the Editorial Board. Authors should submit a letter to the Editorial Board in the
following form:

{\bfseries\textit{Agreement on the transfer of rights to publish:}}

``\textit{We, the undersigned authors of the manuscript ``\ldots'', pass to the
Founder and the Editorial Board of the Journal ``Informatics and Applications''
the nonexclusive right to publish the manuscript of the article in Russian (or
in English) in both print and electronic versions of the Journal. We affirm
that this publication does not violate the Copyright of other persons or
organizations.}

\textit{Author(s) signature(s): (name(s), address(es), date).}

This agreement should be submitted in paper form or in the form of a scanned copy (signed by
the authors).


%The Editorial Board has the right to request from the authors an official expert conclusion that
%the submitted article has no secret data prohibited for publication. \\[-13.5pt]
\item
A submitted article should be attached with \textbf{the data on the author(s)} (see item~8). If
there are several authors, the contact person should be indicated who is responsible for
correspondence with the Editorial Board and other authors about revisions and final approval
of the proofs.\\[-13.5pt]

\item The Editorial Board of the Journal examines the article according to the established
reviewing procedure. If the authors receive their article for correction after reviewing, it does not
mean that the article is approved for publication. The corrected article should be sent to the
Editorial Board for the subsequent review and approval.\\[-13.5pt]

\item The decision on the article publication or its rejection is communicated to the authors. The
Editorial Board may also send the reviews on the submitted articles to the authors. Any
discussion upon the rejected articles is not possible.\\[-13.5pt]

\item The edited articles will be sent to the authors for proofread. The comments of the authors
to the edited text of the article should be sent to the Editorial Board as soon as possible.\\[-13.5pt]

\item The manuscript of the article should be presented electronically in the MS WORD (.doc or
.docx) or \LaTeX\ (.tex) formats, and additionally in the .pdf format. All documents
 may be sent
by e-mail or provided on a CD or diskette. A~hard copy submission is not necessary.\\[-13.5pt]

\item The recommended typesetting instructions for manuscript.

Pages parameters: format A4, portrait orientation, document margins (cm): left~--- 2.5, right~---
1.5, above~--- 2.0, below~--- 2.0, footer 1.3.

Text: font~---Times New Roman, font size~--- 14, paragraph indent~--- 0.5, line spacing~--- 1.5,
justified alignment.

The recommended manuscript size: not more than 15~pages of the specified format.
If the specified size exceeded, the editorial board is entitled to require the author
to reduce the manuscript.

Use only standard abbreviations. Avoid  abbreviations in the title and
abstract. The full term for which an abbreviation stands should precede
its first use in the text unless it is a standard unit of measurement.

All pages of the manuscript should be numbered.

The templates for the manuscript typesetting are presented on site: {\sf
http://www.ipiran.ru/journal/template.doc}.\\[-13.5pt]


%\def\leftkol{Requirements for manuscripts submitted to Journal
%``Informatics~and~Applications''}

\item The articles should enclose data both in \textbf{Russian and English}:
\begin{itemize}
\item title;\\[-13.5pt]
\item author's name and surname;\\[-13.5pt]
\item affiliation~--- organization, its address with ZIP code, city, country, and
official e-mail address;\\[-13.5pt]
\item data on authors according to the format: (see site)

{\sf http://www.ipiran.ru/journal/issues/2013\_07\_01/authors.asp}  and

{\sf  http://www.ipiran.ru/journal/issues/2013\_07\_01\_eng/authors.asp};\\[-13.5pt]

\pagebreak

\def\leftfootline{\small{\textbf{\thepage}
\hfill INFORMATIKA I EE PRIMENENIYA~--- INFORMATICS AND APPLICATIONS\ \ \ 2019\
\ \ volume~13\ \ \ issue\ 4}
}%
 \def\rightfootline{\small{INFORMATIKA I EE PRIMENENIYA~--- INFORMATICS AND APPLICATIONS\ \ \ 2019\ \ \ volume~13\ \ \ issue\ 4
\hfill \textbf{\thepage}}}


%\def\leftkol{Requirements for manuscripts submitted to Journal
%``Informatics~and~Applications''}

%\def\rightkol{Requirements for manuscripts submitted to Journal
%``Informatics~and~Applications''}



\item abstract (not less than 100 words) both in Russian and in English. Abstract is a short
summary of the article that can be published separately. The abstract is the
main source of information on the article and it could be included in leading information
systems and data bases. The abstract in English has to be an original text and should
not be an exact translation of the Russian one. Good English is required.
In abstracts, avoid references and formulae;\\[-13.5pt]
\item indexing is performed on the basis of keywords. The use of keywords from the
internationally accepted thematic Thesauri is recommended.

%\def\leftkol{Requirements for manuscripts submitted to Journal
%``Informatics~and~Applications''}

%\def\rightkol{Requirements for manuscripts submitted to Journal
%``Informatics~and~Applications''}

Important! Keywords must not be sentences;
\item Acknowledgments.
\end{itemize}

\item References. Russian references have to be presented both in English translation and Latin
transliteration (refer {\sf http://www.translit.net/ru/bgn/}).

Please take into account the following examples of Russian references appearance:

\noindent
\textbf{Article in journal:}

\Aue{Zhang, Z., and D.~Zhu}. 2008. Experimental research on the localized electrochemical
micromachining.
\textit{Rus. J.~Electrochem.}  44(8):926--930. {\sf doi:10.1134/S1023193508080077}.


\noindent
\textbf{Journal article in electronic format:}

\Aue{Swaminathan, V., E.~Lepkoswka-White, and B.\,P.~Rao}. 1999. Browsers or buyers in
cyberspace? An
investigation of electronic factors influencing electronic exchange. \textit{JCMC}
5(2). Available at: {\sf http://www.ascusc.org/jcmc/vol5/issue2/} (accessed April~28, 2011).




\noindent
\textbf{Article from the continuing publication (collection of works, proceedings):}

\Aue{Astakhov, M.\,V., and T.\,V.~Tagantsev}. 2006. Eksperimental'noe
issledovanie prochnosti soedineniy ``stal'--kompozit'' [Experimental study of
the strength of joints ``steel--composite'']. \textit{Trudy MGTU
``Matematicheskoe modelirovanie slozhnykh tekh\-ni\-che\-skikh sistem''}
[\textit{Bauman MSTU ``Mathematical Modeling of Complex Technical
Systems'' Proceedings}]. 593:125--130.

\def\leftfootline{\small{\textbf{\thepage}
\hfill INFORMATIKA I EE PRIMENENIYA~--- INFORMATICS AND APPLICATIONS\ \ \ 2019\
\ \ volume~13\ \ \ issue\ 4}
}%
 \def\rightfootline{\small{INFORMATIKA I EE PRIMENENIYA~--- INFORMATICS AND APPLICATIONS\ \ \ 2019\ \ \ volume~13\ \ \ issue\ 4
\hfill \textbf{\thepage}}}

\def\leftkol{Requirements for manuscripts submitted to Journal
``Informatics~and~Applications''}

\def\rightkol{Requirements for manuscripts submitted to Journal
``Informatics~and~Applications''}

\noindent
\textbf{Conference proceedings:}

\Aue{Usmanov, T.\,S., A.\,A.~Gusmanov, I.\,Z.~Mullagalin, R.\,Ju.~Muhametshina,
A.\,N.~Chervyakova, and
A.\,V.~Sveshnikov}. 2007. Osobennosti proektirovaniya razrabotki mestorozhdeniy
s primeneniem gidrorazryva
plasta [Features of the design of field development with the use of hydraulic fracturing].
\textit{Trudy 6-go
Mezhdu\-na\-rod\-no\-go Simpoziuma ``Novye resursosberegayushchie tekhnologii
nedropol'zovaniya i povysheniya
neftegazootdachi''} [\textit{6th  Symposium (International) ``New Energy Saving Subsoil
Technologies and
the Increasing of the Oil and Gas Impact'' Proceedings}]. Moscow. 267--272.


\noindent
\textbf{Books and other monographs:}




Lindorf, L.\,S., and L.\,G.~Mamikoniants, eds. 1972.
\textit{Ekspluatatsiya turbogeneratorov s neposredstvennym
okhlazhdeniem} [\textit{Operation of turbine generators with direct cooling}].
Moscow: Energy Publs. 352~p.


%\Aue{Latyshev, V.\,N.} 2009. \textit{Tribologiya rezaniya. Kn.~1: Frikcionnye prosessy
%pri rezanii metallov}
%[\textit{Tribology of cutting. Vol.~1: Frictional processes in metal cutting}]. Ivanovo: Ivanovskii
%State Univ. 108~p.


%\noindent
%\textbf{Unpublished material:}

%\Aue{Latypov, A.\,R., M.\,M.~Khasanov, and V.\,A.~Baikov}.
%2004. Geology and production (NGT GiD). Certificate on official registration of the computer
%program
%No.\,2004611198. (In Russian, unpubl.)

%\noindent
%\textbf{Internet-source:}

%APA Style. 2011. Available at: {\sf http://www.apastyle.org/apa-style-help.aspx} (accessed
%February~5, 2011).

%Pravila citirovaniya istochnikov [Rules for the citing of sources]. Available at: {\sf
%http://www.scribd.com/doc/1034528/} (accessed February~7, 2011).


\noindent
\textbf{Dissertation and Thesis:}

%\Aue{Semenov, V.\,I.}
%2003. Matematicheskoe modelirovanie plazmy v sisteme kompaktnyy tor. [Mathematical
%modeling of the plasma in the compact torus]. D.Sc.\ Diss. Moscow. 272~p.

\Aue{Kozhunova, O.\,S.} 2009. Tekhnologiya razrabotki semanticheskogo
slovarya informatsionnogo monitoringa [Technology of development of
semantic dictionary of information monitoring system]. PhD Thesis. Moscow: IPI RAN. 23~p.


\noindent
\textbf{State standards and patents:}

GOST 8.586.5-2005. 2007. Metodika vypolneniya izmereniy. Izmerenie raskhoda i~kolichestva
zhidkostey i gazov 
s~pomoshch'yu standartnykh suzhayushchikh ustroystv [Method of measurement.
Measurement of flow rate and volume of liquids and gases by means of orifice devices]. M.:
Standardinform
Publs. 10~p.

%\noindent
%\textbf{Patent:}

\Aue{Bolshakov, M.\,V., A.\,V.~Kulakov, A.\,N.~Lavrenov, and M.\,V.~Palkin}.
2006. Sposob orientirovaniya po krenu letatel'nogo
apparata s opti\-che\-skoy golovkoy
samonavedeniya [The way to orient on the roll of aircraft with optical homing head].
Patent RF No.\,2280590.

References in Latin transcription are presented in the original language.

References in the text are numbered according to the order of their
first appearance; the number is
placed in square brackets. All items from the reference list should be
cited.\\[-13.5pt]

\item Manuscripts and additional materials are not returned to Authors by the Editorial Board.\\[-13.5pt]

\item Submissions of files by e-mail must include:\\[-13.5pt]
\begin{itemize}
\item   the journal title and author's name in the ``Subject'' field; \\[-13.5pt]
\item   an article and additional materials have to be attached using the ``attach'' function;\\[-13.5pt]
\item   an electronic version of the article should contain the file with the text and a separate file
with figures.\\[-13.5pt]
\end{itemize}

\item ``Informatics and Applications'' journal is not a profit publication. There are no
charges for the authors as well as there are no royalties.\\[-13.5pt]
\end{enumerate}

\def\leftfootline{\small{\textbf{\thepage}
\hfill INFORMATIKA I EE PRIMENENIYA~--- INFORMATICS AND APPLICATIONS\ \ \ 2019\
\ \ volume~13\ \ \ issue\ 4}
}%
 \def\rightfootline{\small{INFORMATIKA I EE PRIMENENIYA~--- INFORMATICS AND APPLICATIONS\ \ \ 2019\ \ \ volume~13\ \ \ issue\ 4
\hfill \textbf{\thepage}}}

\def\leftkol{Requirements for manuscripts submitted to Journal
``Informatics~and~Applications''}

\def\rightkol{Requirements for manuscripts submitted to Journal
``Informatics~and~Applications''}


%\vspace*{5mm}


\begin{center}
\textbf{Editorial Board address:} \\

%ABOUT AUTHORS



FRC CSC RAS, 44, block~2, Vavilov Str., Moscow 119333, Russia\\[-10pt]

\

Ph.: +7\,(499)\,135\,86\,92,\ \ Fax: +7\,(495)\,930\,45\,05\\[-10pt]

\

 e-mail: {\sf rust@ipiran.ru} (to Prof.\ Rustem Seyful-Mulyukov)\\[-10pt]

\

 {\sf http://www.ipiran.ru/english/journal.asp}
\end{center}
 }
%\thispagestyle{myheadings}

\def\leftkol{Requirements for manuscripts submitted to Journal
``Informatics~and~Applications''}

\def\rightkol{Requirements for manuscripts submitted to Journal
``Informatics~and~Applications''}

\def\leftfootline{\small{\textbf{\thepage}
\hfill INFORMATIKA I EE PRIMENENIYA~--- INFORMATICS AND APPLICATIONS\ \ \ 2019\
\ \ volume~13\ \ \ issue\ 4}
}%
 \def\rightfootline{\small{INFORMATIKA I EE PRIMENENIYA~--- INFORMATICS AND APPLICATIONS\ \ \ 2019\ \ \ volume~13\ \ \ issue\ 4
\hfill \textbf{\thepage}}}

 \label{end\stat}

\newpage


%\end{document}

%\include{IPPM-25}

%\def\stat{cont}
{%\hrule\par
%\vskip 7pt % 7pt
\raggedleft\Large \bf%\baselineskip=3.2ex
А\,В\,Т\,О\,Р\,С\,К\,И\,Й\ \ У\,К\,А\,З\,А\,Т\,Е\,Л\,Ь\ \ З\,А\ \ 2\,0\,1\,0 г. \vskip 17pt
    \hrule
    \par
\vskip 21pt plus 6pt minus 3pt }

\label{st\stat}

\def\tit{\ }

\def\aut{\ }
\def\auf{\ }

\def\leftkol{\ } % ENGLISH ABSTRACTS}

\def\rightkol{\ } %АВТОРСКИЙ УКАЗАТЕЛЬ ЗА 2010 г.} %ENGLISH ABSTRACTS}

\titele{\tit}{\aut}{\auf}{\leftkol}{\rightkol}

\vspace*{-12pt}

{\tabcolsep=3pt
\begin{tabular}{p{388pt}rr}
&\textbf{Выпуск} & \textbf{Стр.}\\[6pt]
\hangindent=23pt\noindent\textbf{Арутюнян~А.\,Р.} Моделирование влияния деформаций отпечатков пальцев на 
точность\linebreak
\vspace*{-12pt}\\
\hspace*{23pt}дактилоскопической идентификации$\dotfill$&1&51\\
\hangindent=23pt\noindent\textbf{Архипов~О.\,П., Зыкова~З.\,П.} Интеграция гетерогенной информации о цветных 
пикселях\linebreak
\vspace*{-12pt}\\
\hspace*{23pt}и их цветовосприятии$\dotfill$&4&15\\
\hangindent=23pt\noindent\textbf{Баранов~С.\,И., Френкель~С.\,Л., Захаров~В.\,Н.} Полуформальная верификация 
цифрового устройства с конвейером, основанная на использовании алгоритмических машин\linebreak
\vspace*{-12pt}\\
\hspace*{23pt}состояния$\dotfill$&4&49\\
\textbf{Бекетова~И.\,В.} см.~Каратеев~С.\,Л.&&\\
\textbf{Белоусов~В.\,В.} см.~Синицын~И.\,Н.&&\\
\hangindent=23pt\noindent\textbf{Бенинг~В.\,Е., Королев~Р.\,А.} О предельном поведении мощностей критериев в 
случае\linebreak
\vspace*{-12pt}\\
\hspace*{23pt}распределения Лапласа$\dotfill$&2&63\\
\hangindent=23pt\noindent\textbf{Бенинг~В.\,Е., Сипина~А.\,В.} Асимптотическое разложение для мощности 
критерия,\linebreak
\vspace*{-12pt}\\
\hspace*{23pt}основанного на выборочной медиане, в случае распределения Лапласа$\dotfill$&1&18\\
\textbf{Бондаренко~А.\,В.} см.~Каратеев~С.\,Л.&&\\
\hangindent=23pt\noindent\textbf{Бородина~А.\,В., Морозов~Е.\,В.} Об оценивании асимптотики вероятности 
большого\linebreak
\vspace*{-12pt}\\
\hspace*{23pt}уклонения стационарной регенеративной очереди с одним прибором$\dotfill$&3&29\\
\hangindent=23pt\noindent\textbf{Бунтман~Н.\,В., Минель~Ж.-Л., Ле~Пезан~Д., Зацман~И.\,М.} Типология и 
компьютерное\linebreak
\vspace*{-12pt}\\
\hspace*{23pt}моделирование трудностей перевода$\dotfill$&3&77\\
\textbf{Визильтер~Ю.\,В.} см.~Каратеев~С.\,Л.&&\\
\hangindent=23pt\noindent\textbf{Гавриленко~С.\,В.} Оценки скорости сходимости распределений случайных сумм с 
безгранично делимыми индексами к нормальному закону$\dotfill$&4&81\\
\hangindent=23pt\noindent\textbf{Григорьева~М.\,Е., Шевцова~И.\,Г.} Уточнение неравенства 
Каца--Берри--Эссеена$\dotfill$&2&75\\
\hangindent=23pt\noindent\textbf{Грушо~А.\,А., Грушо~Н.\,А., Тимонина~Е.\,Е.} Поиск конфликтов в политиках 
безопасности: модель случайных графов$\dotfill$&3&38\\
\textbf{Грушо~Н.\,А.} см.~Грушо~А.\,А.&&\\
\hangindent=23pt\noindent\textbf{Гудков~В.\,Ю.} Математические модели изображения отпечатка пальца на основе 
описания линий$\dotfill$&1&58\\
\textbf{Гуртов~А.\,В.} см.~Лукьяненко~А.\,С.&&\\
\textbf{Желтов~С.\,Ю.} см.~Каратеев~С.\,Л.&&\\
\hangindent=23pt\noindent\textbf{Захаров~А.\,А., Серебряков~В.\,А.} Система управления электронной библиотекой 
LibMeta$\dotfill$&4&2\\
\textbf{Захаров~В.\,Н.} см.~Баранов~С.\,И.&&\\
\textbf{Захарова~Т.\,В.} см.~Матвеева~С.\,С.&&\\
\hangindent=23pt\noindent\textbf{Зацаринный~А.\,А., Чупраков~К.\,Г.} Некоторые аспекты выбора технологии для 
постро-\linebreak
\vspace*{-12pt}\\
\hspace*{23pt}ения систем отображения информации ситуационного центра$\dotfill$&3&59\\
\textbf{Зацман~И.\,М.} см.~Бунтман~Н.\,В.&&\\
\hangindent=23pt\noindent\textbf{Зейфман~А.\,И., Коротышева~А.\,В., Сатин~Я.\,А., Шоргин~С.\,Я.} Об 
устойчивости нестаци-\linebreak
\vspace*{-12pt}\\
\hspace*{23pt}онарных систем обслуживания с катастрофами$\dotfill$&3&9\\
\textbf{Зыкова~З.\,П.} см.~Архипов~О.\,П.&&\\
\hangindent=23pt\noindent\textbf{Илюшин~Г.\,Я., Соколов~И.\,А.} Организация управляемого доступа пользователей 
к\linebreak
\vspace*{-12pt}\\
\hspace*{23pt}разнородным ведомственным информационным ресурсам$\dotfill$&1&24\\
\hangindent=23pt\noindent\textbf{Кавагучи~Ю., Ульянов~В.\,В., Фуджикоши~Я.} Приближения для статистик, 
описывающих\linebreak
\vspace*{-12pt}\\
\hspace*{23pt}геометрические свойства данных большой размерности, с оценками 
ошибок$\dotfill$&1&12\\
\hangindent=23pt\noindent\textbf{Каратеев~С.\,Л., Бекетова~И.\,В., Ососков~М.\,В., Князь~В.\,А., 
Визильтер~Ю.\,В., Бондаренко~А.\,В., Желтов~С.\,Ю.} Автоматизированный контроль 
качества цифровых\linebreak
\vspace*{-12pt}\\
\hspace*{23pt}изображений для персональных документов$\dotfill$&1&65\\
\end{tabular}
}

\pagebreak

\def\leftkol{АВТОРСКИЙ УКАЗАТЕЛЬ ЗА 2010 г.} % ENGLISH ABSTRACTS}

\def\rightkol{АВТОРСКИЙ УКАЗАТЕЛЬ ЗА 2010 г.} %ENGLISH ABSTRACTS}

{\tabcolsep=3pt
\begin{tabular}{p{388pt}rr}
&\textbf{Выпуск} & \textbf{Стр.}\\[3pt]
\hangindent=23pt\noindent\textbf{Козеренко~Е.\,Б.} Лингвистические фильтры в статистических моделях машинного\linebreak
\vspace*{-12pt}\\
\hspace*{23pt}перевода$\dotfill$&2&83\\
\hangindent=23pt\noindent\textbf{Козеренко~Е.\,Б., Кузнецов~И.\,П.} Когнитивно-лингвистические представления в 
систе-\linebreak
\vspace*{-12pt}\\
\hspace*{23pt}мах обработки текстов$\dotfill$&3&69\\
\textbf{Князь~В.\,А.} см.~Каратеев~С.\,Л.&&\\
\hangindent=23pt\noindent\textbf{Колесников~А.\,В., Солдатов~С.\,А.} Алгоритм координации для гибридной 
интеллектуальной системы решения сложной задачи оперативно-производственного\linebreak
\vspace*{-12pt}\\
\hspace*{23pt}планирования$\dotfill$&4&61\\
\hangindent=23pt\noindent\textbf{Коновалов~М.\,Г.} О планировании потоков в системах вычислительных 
ресурсов$\dotfill$&2&3\\
\textbf{Конушин~А.\,С.} см.~Конушин~В.\,С.&&\\
\hangindent=23pt\noindent\textbf{Конушин~В.\,С., Кривовязь~Г.\,Р., Конушин~А.\,С.} Алгоритм распознавания людей 
в видео-\linebreak
\vspace*{-12pt}\\
\hspace*{23pt}последовательности по одежде$\dotfill$&1&74\\
\textbf{Корепанов~Э.\, Р.} см.~Синицын~И.\,Н.&&\\
\textbf{Королев~В.\,Ю.} см.~Соколов~И.\,А.&&\\
\textbf{Королев~Р.\,А.} см.~Бенинг~В.\,Е.&&\\
\textbf{Коротышева~А.\,В.} см.~Зейфман~А.\,И.&&\\
\hangindent=23pt\noindent\textbf{Кривенко~М.\,П.} Непараметрическое оценивание элементов байесовского 
клас\-си-\linebreak
\vspace*{-12pt}\\
\hspace*{23pt}фикатора$\dotfill$&2&13\\
\textbf{Кривовязь~Г.\,Р.} см.~Конушин~В.\,С.&&\\
\textbf{Крылов~А.\,С.} см.~Павельева~Е.\,А.&&\\
\hangindent=23pt\noindent\textbf{Крылов~В.\,А.} Моделирование и классификация многоканальных дистанционных\linebreak
\vspace*{-12pt}\\
\hspace*{23pt}изображений с использованием копул$\dotfill$&4&34\\
\hangindent=23pt\noindent\textbf{Крючин~О.\,В.} Разработка параллельных эвристических алгоритмов подбора 
весовых\linebreak
\vspace*{-12pt}\\
\hspace*{23pt}коэффициентов искусственной нейтронной сети$\dotfill$&2&53\\
\hangindent=23pt\noindent\textbf{Кудрявцев~А.\,А., Шоргин~С.\,Я.} Байесовские модели массового обслуживания и 
надеж-\linebreak
\vspace*{-12pt}\\
\hspace*{23pt}ности: характеристики среднего числа заявок в системе $M\vert M \vert 1\vert 
\infty$$\dotfill$&3&16\\
\hangindent=23pt\noindent\textbf{Кузнецов~А.\,А.} Связь между временными и структурно-топологическими 
характери-\linebreak
\vspace*{-12pt}\\
\hspace*{23pt}стиками диаграмм ритма сердца здоровых людей$\dotfill$&4&39\\
\textbf{Кузнецов~И.\,П.} см.~Козеренко~Е.\,Б.&&\\
\textbf{Ле~Пезан~Д.} см.~Бунтман~Н.\,В.&&\\
\hangindent=23pt\noindent\textbf{Лукьяненко~А.\,С., Морозов~Е.\,В., Гуртов~А.\,В.} Анализ сетевого протокола с общей 
функ-\linebreak
\vspace*{-12pt}\\
\hspace*{23pt}цией расширения окна передачи сообщения при конфликтах$\dotfill$&2&46\\
\hangindent=23pt\noindent\textbf{Лямин~О.\,О.} О предельном поведении мощностей критериев в случае обобщенного\linebreak
\vspace*{-12pt}\\
\hspace*{23pt}распределения Лапласа$\dotfill$&3&47\\
\hangindent=23pt\noindent\textbf{Маркин~А.\,В., Шестаков~О.\,В.} Асимптотики оценки риска при пороговой 
обработке\linebreak
\vspace*{-12pt}\\
\hspace*{23pt}вейвлет-вейглет коэффициентов в задаче томографии$\dotfill$&2&36\\
\hangindent=23pt\noindent\textbf{Матвеева~С.\,С., Захарова~Т.\,В.} Сети массового обслуживания с наименьшей 
длиной\linebreak
\vspace*{-12pt}\\
\hspace*{23pt}очереди$\dotfill$&3&22\\
\hangindent=23pt\noindent\textbf{Матюшенко~С.\,И.} Стационарные характеристики двухканальной системы 
обслужива-\linebreak
\vspace*{-12pt}\\
\hspace*{23pt}ния с переупорядочиванием заявок и распределениями фазового типа$\dotfill$&4&68\\
\textbf{Минель~Ж.-Л.} см.~Бунтман~Н.\,В.&&\\
\textbf{Морозов~Е.\,В.} см.~Бородина~А.\,В.&&\\
\textbf{Морозов~Е.\,В.} см.~Лукьяненко~А.\,С.&&\\
\textbf{Ососков~М.\,В.} см.~Каратеев~С.\,Л.&&\\
\hangindent=23pt\noindent\textbf{Павельева~Е.\,А., Крылов~А.\,С.} Поиск и анализ ключевых точек радужной 
оболочки\linebreak
\vspace*{-12pt}\\
\hspace*{23pt}глаза методом преобразования Эрмита$\dotfill$&1&79\\
\textbf{Печинкин~А.\,В.} см.~Френкель~С.\,Л.,&&\\
\hangindent=23pt\noindent\textbf{Протасов~В.\,И.} Составление субъективного портрета с использованием 
эволюционно-\linebreak
\vspace*{-12pt}\\
\hspace*{23pt}го морфинга и квалиметрия метода$\dotfill$&1&83\\
\hangindent=23pt\noindent\textbf{Рудаков~К.\,В., Торшин~И.\,Ю.} Вопросы разрешимости задачи распознавания 
вторичной\linebreak
\vspace*{-12pt}\\
\hspace*{23pt}структуры белка$\dotfill$&2&25\\
\textbf{Сатин~Я.\,А.} см.~Зейфман~А.\,И.&&\\
\hangindent=23pt\noindent\textbf{Сейфуль-Мулюков~Р.\,Б.} Нефть как носитель информации о своем 
происхождении,\linebreak
\vspace*{-12pt}\\
\hspace*{23pt}структуре и эволюции$\dotfill$&1&41\\
\end{tabular}
}

{\tabcolsep=3pt
\begin{tabular}{p{388pt}rr}
&\textbf{Выпуск} & \textbf{Стр.}\\[6pt]
\textbf{Семендяев~Н.\,Н.} см.~Синицын~И.\,Н.&&\\
\textbf{Серебряков~В.\,А.} см.~Захаров~А.\,А.&&\\
\textbf{Синицын~В.\,И.} см.~Синицын~И.\,Н.&&\\
\hangindent=23pt\noindent\textbf{Синицын~И.\,Н., Синицын~В.\,И., Корепанов~Э.\, Р., Белоусов~В.\,В., 
Семендяев~Н.\,Н.} Оперативное построение информационных моделей движения полюса 
Земли\linebreak
\vspace*{-12pt}\\
\hspace*{23pt}методами линейных и линеаризованных фильтров$\dotfill$&1&2\\
\textbf{Сипина~А.\,В.} см.~Бенинг~В.\,Е.&&\\
\hangindent=23pt\noindent\textbf{Соколов~И.\,А.} О работах заслуженного деятеля науки Российской Федерации 
И.\,Н.~Синицына в области информационных технологий и автоматизации (к 70-летию\linebreak
\vspace*{-12pt}\\
\hspace*{23pt}со дня рождения)$\dotfill$&3&84\\
\textbf{Соколов~И.\,А.} см.~Илюшин~Г.\,Я.&&\\
\hangindent=23pt\noindent\textbf{Соколов~И.\,А., Королев~В.\,Ю.} Предисловие$\dotfill$&2&2\\
\textbf{Солдатов~С.\,А.} см.~Колесников~А.\,В.&&\\
\hangindent=23pt\noindent\textbf{Степанов~С.\,Ю.} Использование координатного метода фрагментации 
коммутаторной\linebreak
\vspace*{-12pt}\\
\hspace*{23pt}нейронной сети для сокращения трафика$\dotfill$&2&57\\
\textbf{Тимонина~Е.\,Е.} см.~Грушо~А.\,А.&&\\
\textbf{Торшин~И.\,Ю.} см.~Рудаков~К.\,В.&&\\
\textbf{Ульянов~В.\,В.} см.~Кавагучи~Ю.&&\\
\textbf{Фазекаш~И.} см.~Чупрунов~А.\,Н.&&\\
\textbf{Френкель~С.\,Л.} см.~Баранов~С.\,И.&&\\
\hangindent=23pt\noindent\textbf{Френкель~С.\,Л., Печинкин~А.\,В.} Оценка времени самовосстановления в 
цифровых\linebreak
\vspace*{-12pt}\\
\hspace*{23pt}системах после сбоев, вызываемых переходными помехами$\dotfill$&3&2\\
\textbf{Фуджикоши~Я.} см.~Кавагучи~Ю.&&\\
\hangindent=23pt\noindent\textbf{Цискаридзе~А.\,К.} Математическая модель и метод восстановления позы человека 
по\linebreak
\vspace*{-12pt}\\
\hspace*{23pt}стереопаре силуэтных изображений$\dotfill$&4&27\\
\hangindent=23pt\noindent\textbf{Чупраков~К.\,Г.} К вопросу о размещении коллективных средств отображения в 
ситуа-\linebreak
\vspace*{-12pt}\\
\hspace*{23pt}ционном зале с заданными параметрами$\dotfill$&4&89\\
\textbf{Чупраков~К.\,Г.} см.~Зацаринный~А.\,А.&&\\
\hangindent=23pt\noindent\textbf{Чупрунов~А.\,Н., Фазекаш~И.} Законы повторного логарифма для числа 
безошибочных\linebreak
\vspace*{-12pt}\\
\hspace*{23pt}блоков при помехоустойчивом кодировании$\dotfill$&3&42\\
\textbf{Шевцова~И.\,Г.} см.~Григорьева~М.\,Е.&&\\
\hangindent=23pt\noindent\textbf{Шестаков~О.\,В.} Аппроксимация распределения оценки риска пороговой 
обработки вейвлет-коэффициентов нормальным распределением при использовании 
выбо-\linebreak
\vspace*{-12pt}\\
\hspace*{23pt}рочной дисперсии$\dotfill$&4&73\\
\textbf{Шестаков~О.\,В.} см.~Маркин~А.\,В.&&\\
\textbf{Шоргин~С.\,Я.} см.~Зейфман~А.\,И.&&\\
\textbf{Шоргин~С.\,Я.} см.~Кудрявцев~А.\,А.&&\\
\end{tabular}
}

%\thispagestyle{myheadings}
\def\leftfootline{\small{\textbf{\thepage}
\hfill ИНФОРМАТИКА И ЕЁ ПРИМЕНЕНИЯ\ \ \ том~4\ \ \ выпуск~4\ \ \ 2010}
}%
 \def\rightfootline{\small{ИНФОРМАТИКА И ЕЁ ПРИМЕНЕНИЯ\ \ \ том~4\ \ \ выпуск~4\ \ \ 2010
 \hfill \textbf{\thepage}}}
 \label{end\stat}

%
%Том 10 Выпуск 1-4 Год 2016

\def\stat{cont-e}
{%\hrule\par
%\vskip 7pt % 7pt
\raggedleft\Large \bf%\baselineskip=3.2ex
2\,0\,1\,6\ \ A\,U\,T\,H\,O\,R\ \ I\,N\,D\,E\,X \vskip 17pt
 \hrule
 \par
\vskip 21pt plus 6pt minus 3pt }

\label{st\stat}

\def\tit{\ }

\def\aut{\ }
\def\auf{\ }

\def\leftkol{\ } %2016 AUTHOR INDEX} % ENGLISH ABSTRACTS}

\def\rightkol{\ } %2016 AUTHOR INDEX} %ENGLISH ABSTRACTS}

\titele{\tit}{\aut}{\auf}{\leftkol}{\rightkol}

\def\leftfootline{\small{\textbf{\thepage}
\hfill INFORMATIKA I EE PRIMENENIYA~--- INFORMATICS AND APPLICATIONS\ \ \ 2016\
\ \ volume~10\ \ \ issue\ 4}
}%
 \def\rightfootline{\small{INFORMATIKA I EE PRIMENENIYA~--- INFORMATICS AND APPLICATIONS\ \ \ 2016\ \ \ volume~10\ \ \ issue\ 4
\hfill \textbf{\thepage}}}

\vspace*{-12pt}
\vspace*{-18pt}

{\tabcolsep=2.8pt
\begin{tabular}{p{382pt}cc}
&\textbf{Issue} & \textbf{Page}\\[6pt]
\Avtors{Agalarov~M.\,Ya.} see~Agalarov~Ya.\,M.&&\\
\Avtors{Agalarov~Ya.\,M., Agalarov~M.\,Ya., and
Shorgin~V.\,S.} About the optimal threshold of queue\linebreak
\\[-12pt]
\hspace*{23pt}length in a~particular problem of profit maximization
in the $M/G/1$ queuing system&2&70--79\\
\Avtors{Alexeyevsky~D.\,A.} BioNLP ontology extraction from 
a~restricted language corpus with\linebreak
\\[-12pt]
\hspace*{23pt}context-free grammars&1&119--128\\
\Avtors{Andreev~S.\,D.} see~Gaidamaka~Yu.\,V.&&\\
\Avtors{Andreev~S.\,D.} see~Ometov~A.\,Ya.&&\\
\Avtors{Arkhipov~O.\,P., Arkhipov~P.\,O., and Sidorkin~I.\,I.} The
option to create a~local coordinate\linebreak
\\[-12pt]
\hspace*{23pt}system for synchronization of selected images&3&91--97\\
\Avtors{Arkhipov~P.\,O.} see~Arkhipov~O.\,P.&&\\
\Avtors{Belousov~V.\,V.} see~Shnurkov~P.\,V.&&\\
\Avtors{Belousov~V.\,V.} see~Shnurkov~P.\,V.&&\\
\Avtors{Bening~V.\,E.} Calculation of~the~asymptotic deficiency
of~some statistical procedures based\linebreak
\\[-12pt]
\hspace*{23pt}on~samples with~random sizes&4&34--45\\
\Avtors{Borisov~A.\,V., Bosov~A.\,V., and Miller~G.\,B.} Modeling and
monitoring of VoIP connection&2&\hphantom{1}2--13\\
\Avtors{Bosov~A.\,V.} see~Borisov~A.\,V.&&\\
\Avtors{Briukhov~D.\,O.} see~Stupnikov~S.\,A.&&\\
\Avtors{Callaos~N.\,K.\ and Seyful-Mulyukov~R.\,B.} Complexity and
its information content&1&129--139\\
\Avtors{Chertok~A.\,V., Kadaner~A.\,I., Khazeeva~G.\,T., and
Sokolov~I.\,A.} Regime switching detection\linebreak
\\[-12pt]
\hspace*{23pt}for~the~Levy driven
Ornstein--Uhlenbeck process using CUSUM methods&4&46--56\\
\Avtors{Chichagov~V.\,V.} Asymptotic expansions of mean absolute
error of uniformly minimum variance unbiased and maximum likelihood
estimators on the one-parameter exponential\linebreak
\\[-12pt]
\hspace*{23pt}family model of lattice distributions&3&66--76\\
\Avtors{Danishevsky~V.\,I.} see~Kolesnikov A.\,V.&&\\
\Avtors{Fazliev~A.\,Z.} see~Kalinichenko~L.\,A.&&\\
\Avtors{Fedoseev~A.\,A.} What is behind the concept of ``knowledge in
small packages''&3&105--110\\
\Avtors{Gaidamaka~Yu.\,V., Andreev~S.\,D., Sopin~E.\,S.,
Samouylov~K.\,E., and Shorgin~S.\,Ya.} Interference analysis
of~the~device-to-device communications model with~regard to~a~signal\linebreak
\\[-12pt]
\hspace*{23pt}propagation environment&4&\hphantom{1}2--10\\
\Avtors{Gasilov~A.\,V.} see~Yakovlev~O.\,A.&&\\
\Avtors{Goncharov~A.\,V.\ and Strijov~V.\,V.} Metric time series
classification using weighted dynamic\linebreak
\\[-12pt]
\hspace*{23pt}warping relative to centroids of classes&2&36--47\\
\Avtors{Gordov~E.\,P.} see~Kalinichenko~L.\,A.&&\\
\Avtors{Gorshenin~A.\,K.} Concept of online service for stochastic
modeling of real processes&1&72--81\\
\Avtors{Gorshenin~A.\,K.} see~Shnurkov~P.\,V.&&\\
\Avtors{Gorshenin~A.\,K.} see~Shnurkov~P.\,V.&&\\
\Avtors{Grusho~A.\,A., Grusho~N.\,A., Zabezhailo~M.\,I., and
Timonina~E.\,E.} Integration of statistical and\linebreak
\\[-12pt]
\hspace*{23pt}deterministic methods for
analysis of information security&3&2--8\\
\Avtors{Grusho~A.\,A., Zabezhailo~M.\,I., and Zatsarinny~A.\,A.} On
the advanced procedure to reduce\linebreak
\\[-12pt]
\hspace*{23pt}calculation of Galois closures&4&\hphantom{1}96--104\\
\Avtors{Grusho~N.\,A.} see~Grusho~A.\,A.&&\\
\Avtors{Havanskov~V.\,A.} see~Minin~V.\,A.&&\\
\Avtors{Inkova~O.\,Yu.} see~Zatsman~I.\,M.&&\\
\Avtors{Isachenko~R.\,V.\ and Strijov~V.\,V.} Metric learning in
multiclass time series classification\linebreak
\\[-12pt]
\hspace*{23pt}problem&2&48--57\\
\end{tabular}
}
\pagebreak

\def\leftfootline{\small{\textbf{\thepage}
\hfill INFORMATIKA I EE PRIMENENIYA~--- INFORMATICS AND APPLICATIONS\ \ \ 2016\
\ \ volume~10\ \ \ issue\ 4}
}%
 \def\rightfootline{\small{INFORMATIKA I EE PRIMENENIYA~---
INFORMATICS AND APPLICATIONS\ \ \ 2016\ \ \ volume~10\ \ \ issue\ 4
\hfill \textbf{\thepage}}}

\def\leftkol{2016 AUTHOR INDEX} % ENGLISH ABSTRACTS}

\def\rightkol{2016 AUTHOR INDEX} %ENGLISH ABSTRACTS}


{\tabcolsep=2.83pt
\begin{tabular}{p{382pt}cc}
&\textbf{Issue} & \textbf{Page}\\[6pt]
\Avtors{Kadaner~A.\,I.} see~Chertok~A.\,V.&&\\[.255pt]
\Avtors{Kalinichenko~L.\,A., Volnova~A.\,A., Gordov~E.\,P.,
Kiselyova~N.\,N., Kovaleva~D.\,A., Malkov~O.\,Yu., Okladnikov~I.\,G.,
Podkolodnyy~N.\,L., Pozanenko~A.\,S., Ponomareva~N.\,V.,
Stupnikov~S.\,A.,} \textbf{and Fazliev~A.\,Z.} Data access challenges for data
intensive\linebreak
\\[-12pt]
\hspace*{23pt}research in Russia&1& 2--22\\[.255pt]
\Avtors{Karasikov~M.\,E.\ and Strijov~V.\,V.} Feature-based
time-series classification&4&121--131\\[.255pt]
\Avtors{Khazeeva~G.\,T.} see~Chertok~A.\,V.&&\\[.255pt]
\Avtors{Khokhlov~Yu.\,S.} Multivariate fractional Levy motion and its
applications&2&\hphantom{1}98--106\\[.255pt]
\Avtors{Kirikov~I.\,A., Kolesnikov~A.\,V., Listopad~S.\,V., and
Rumovskaya~S.\,B.} Fine-grained hybrid\linebreak
\\[-12pt]
\hspace*{23pt}intelligent systems. Part 2:
Bidirectional hybridization&1&\hphantom{1}96--105\\[.255pt]
\Avtors{Kirikov~I.\,A., Kolesnikov~A.\,V., Listopad~S.\,V., and
Rumovskaya~S.\,B.} ``Virtual council''~---\linebreak
\\[-12pt]
\hspace*{23pt}source environment
supporting complex diagnostic decision making&3&81--90\\[.255pt]
\Avtors{Kiselyova~N.\,N.} see~Kalinichenko~L.\,A.&&\\[.255pt]
\Avtors{Kolesnikov A.\,V., Listopad~S.\,V., Rumovskaya~S.\,B., and
Danishevsky~V.\,I.} Informal axiomatic\linebreak
\\[-12pt]
\hspace*{23pt}theory of~the~role visual models&4&114--120\\[.255pt]
\Avtors{Kolesnikov~A.\,V.} see~Kirikov~I.\,A.&&\\[.255pt]
\Avtors{Kolesnikov~A.\,V.} see~Kirikov~I.\,A.&&\\[.255pt]
\Avtors{Kolin~K.\,K.} Humanitarian aspects of information
security&3&111--121\\[.255pt]
\Avtors{Konovalov~M.\,G.\ and Razumchik~R.\,V.} Dispatching
to~two parallel nonobservable queues using\linebreak
\\[-12pt]
\hspace*{23pt}only static
information&4&57--67\\[.255pt]
\Avtors{Korchagin~A.\,Yu.} see~Korolev~V.\,Yu.&&\\[.255pt]
\Avtors{Korchagin~A.\,Yu.} see~Korolev~V.\,Yu.&&\\[.255pt]
\Avtors{Korepanov~E.\,R.} see~Sinitsyn~I.\,N.&&\\[.255pt]
\Avtors{Korepanov~E.\,R.} see~Sinitsyn~I.\,N.&&\\[.255pt]
\Avtors{Korolev~V.\,Yu., Korchagin~A.\,Yu., and Zeifman~A.\,I.} The
Poisson theorem for Bernoulli trials\linebreak
\\[-12pt]
\hspace*{23pt}with~a~random probability
of~success and~a~discrete analog of~the~Weibull distribution&4&11--20\\[.255pt]
\Avtors{Korolev~V.\,Yu., Zeifman~A.\,I., and Korchagin~A.\,Yu.}
Asymmetric Linnik distributions as~limit\linebreak
\\[-12pt]
\hspace*{23pt}laws for~random sums
of~independent random variables with~finite variances&4&21--33\\[.255pt]
\Avtors{Koucheryavy~E.\,A.} see~Ometov~A.\,Ya.&&\\[.255pt]
\Avtors{Kovaleva~D.\,A.} see~Kalinichenko~L.\,A.&&\\[.255pt]
\Avtors{Kovalyov~S.\,P.} Metaprogramming to increase
manufacturability of large-scale software-\linebreak
\\[-12pt]
\hspace*{23pt}intensive systems&1&56--66\\[.255pt]
\Avtors{Krivenko~M.\,P.} Significance tests of feature selection for
classification&3&32--40\\[.255pt]
\Avtors{Kruzhkov~M.\,G.} see~Zalizniak~Anna~A.&&\\[.255pt]
\Avtors{Kruzhkov~M.\,G.} see~Zatsman~I.\,M.&&\\[.255pt]
\Avtors{Kudryavtsev~A.\,A.} Bayesian queueing and reliability models:
\textit{A~priori} distributions with\linebreak
\\[-12pt]
\hspace*{23pt}compact support&1&67--71\\[.255pt]
\Avtors{Kudryavtsev~A.\,A.} Characteristics dependent on the balance
coefficient in Bayesian models\linebreak
\\[-12pt]
\hspace*{23pt}with compact support of \textit{a priori}
distributions&3&77--80\\[.255pt]
\Avtors{Kudryavtsev~A.\,A.\ and Palionnaia~S.\,I.} Bayesian recurrent
model of reliability growth:\linebreak
\\[-12pt]
\hspace*{23pt}Parabolic distribution of parameters&2&80--83\\[.255pt]
\Avtors{Kudryavtsev~A.\,A.\ and Titova~A.\,I.} Bayesian queuing
and~reliability models: Degenerate-\linebreak
\\[-12pt]
\hspace*{23pt}Weibull case&4&68--71\\[.255pt]
\Avtors{Leontyev~N.\,D.\ and Ushakov~V.\,G.} Analysis of a queueing
system with autoregressive arrivals\linebreak
\\[-12pt]
\hspace*{23pt}and nonpreemptive priority&3&15--22\\[.255pt]
\Avtors{Listopad~S.\,V.} see~Kirikov~I.\,A.&&\\[.255pt]
\Avtors{Listopad~S.\,V.} see~Kirikov~I.\,A.&&\\[.255pt]
\Avtors{Listopad~S.\,V.} see~Kolesnikov A.\,V.&&\\[.255pt]
\Avtors{Malkov~O.\,Yu.} see~Kalinichenko~L.\,A.&&\\[.255pt]
\Avtors{Markov~A.\,S., Monakhov~M.\,M., and
Ulyanov~V.\,V.} Generalized Cornish--Fisher expansions\linebreak
\\[-12pt]
\hspace*{23pt}for distributions of statistics based on samples
of random size&2&84--91\\[.255pt]
\Avtors{Melnikov~A.\,K.\ and Ronzhin~A.\,F.} Generalized statistical
method of~text analysis based\linebreak
\\[-12pt]
\hspace*{23pt}on~calculation of~probability distributions
of~statistical values&4&89--95\\
\end{tabular}
}
\pagebreak

\def\leftfootline{\small{\textbf{\thepage}
\hfill INFORMATIKA I EE PRIMENENIYA~--- INFORMATICS AND APPLICATIONS\ \ \ 2016\
\ \ volume~10\ \ \ issue\ 4}
}%
 \def\rightfootline{\small{INFORMATIKA I EE PRIMENENIYA~---
INFORMATICS AND APPLICATIONS\ \ \ 2016\ \ \ volume~10\ \ \ issue\ 4
\hfill \textbf{\thepage}}}

\def\leftkol{2016 AUTHOR INDEX} % ENGLISH ABSTRACTS}

\def\rightkol{2016 AUTHOR INDEX} %ENGLISH ABSTRACTS}


{\tabcolsep=3pt
\begin{tabular}{p{381pt}cc}
&\textbf{Issue} & \textbf{Page}\\[6pt]
\Avtors{Meykhanadzhyan~L.\,A.} Stationary characteristics of the finite
capacity queueing system with\linebreak
\\[-12pt]
\hspace*{23pt}inverse service order and generalized
probabilistic priority&2&123--131\\[.23pt]
\Avtors{Miller~G.\,B.} see~Borisov~A.\,V.&&\\[.23pt]
\Avtors{Minin~V.\,A., Zatsman~I.\,M., Havanskov~V.\,A., and
Shubnikov~S.\,K.} Intensity of citation of scientific publications in
inventions on information and computer technologies patented\linebreak
\\[-12pt]
\hspace*{23pt}in Russia by domestic and foreign applicants&2&107--122\\[.23pt]
\Avtors{Monakhov~M.\,M.} see~Markov~A.\,S.&&\\[.23pt]
\Avtors{Naumov~V.\,A.\ and Samouylov~K.\,E.} On relationship
between queuing systems with resources\linebreak
\\[-12pt]
\hspace*{23pt}and Erlang networks&3&\hphantom{1}9--14\\[.23pt]
\Avtors{Okladnikov~I.\,G.} see~Kalinichenko~L.\,A.&&\\[.23pt]
\Avtors{Ometov~A.\,Ya., Andreev~S.\,D., Turlikov~A.\,M., and
Koucheryavy~E.\,A.} Performance analysis of\linebreak
\\[-12pt]
\hspace*{23pt}a wireless data
aggregation system with contention for contemporary sensor
networks&3&23--31\\[.23pt]
\Avtors{Palionnaia~S.\,I.} see~Kudryavtsev~A.\,A.&&\\[.23pt]
\Avtors{Podkolodnyy~N.\,L.} see~Kalinichenko~L.\,A.&&\\[.23pt]
\Avtors{Ponomareva~N.\,V.} see~Kalinichenko~L.\,A.&&\\[.23pt]
\Avtors{Popkova~N.\,A.} see~Zatsman~I.\,M.&&\\[.23pt]
\Avtors{Pozanenko~A.\,S.} see~Kalinichenko~L.\,A.&&\\[.23pt]
\Avtors{Razumchik~R.\,V.} see~Konovalov~M.\,G.&&\\[.23pt]
\Avtors{Ronzhin~A.\,F.} see~Melnikov~A.\,K.&&\\[.23pt]
\Avtors{Rumovskaya~S.\,B.} see~Kirikov~I.\,A.&&\\[.23pt]
\Avtors{Rumovskaya~S.\,B.} see~Kirikov~I.\,A.&&\\[.23pt]
\Avtors{Rumovskaya~S.\,B.} see~Kolesnikov A.\,V.&&\\[.23pt]
\Avtors{Samouylov~K.\,E.} see~Gaidamaka~Yu.\,V.&&\\[.23pt]
\Avtors{Samouylov~K.\,E.} see~Naumov~V.\,A.&&\\[.23pt]
\Avtors{Serebryanskii~S.\,M.} see~Tyrsin~A.\,N.&&\\[.23pt]
\Avtors{Seyful-Mulyukov~R.\,B.} see~Callaos~N.\,K.&&\\[.23pt]
\Avtors{Shestakov~O.\,V.} Statistical properties of the denoising method
based on the stabilized hard\linebreak
\\[-12pt]
\hspace*{23pt}thresholding&2&65--69\\[.23pt]
\Avtors{Shestakov~O.\,V.} The strong law of large numbers for the risk
estimate in the problem of\linebreak
\\[-12pt]
\hspace*{23pt}tomographic image reconstruction from
projections with a correlated noise&3&41--45\\[.23pt]
\Avtors{Shestakov~O.\,V.} see~Zakharova~T.\,V.&&\\[.23pt]
\Avtors{Shnurkov~P.\,V., Gorshenin~A.\,K., and Belousov~V.\,V.}
Analytical solution of~the~optimal control\linebreak
\\[-12pt]
\hspace*{23pt}task of~a~semi-Markov
process with~finite set of~states&4&72--88\\[.23pt]
\Avtors{Shnurkov~P.\,V., Zasypko~V.\,V., Belousov~V.\,V., and
Gorshenin~A.\,K.} Development of the algorithm of numerical solution
of the optimal investment control problem\linebreak
\\[-12pt]
\hspace*{23pt}in the closed dynamical model of three-sector economy&1&82--95\\[.23pt]
\Avtors{Shorgin~S.\,Ya.} see~Gaidamaka~Yu.\,V.&&\\[.23pt]
\Avtors{Shorgin~V.\,S.} see~Agalarov~Ya.\,M.&&\\[.23pt]
\Avtors{Shubnikov~S.\,K.} see~Minin~V.\,A.&&\\[.23pt]
\Avtors{Sidorkin~I.\,I.} see~Arkhipov~O.\,P.&&\\[.23pt]
\Avtors{Sinitsyn~I.\,N.} Analytical modeling of processes in stochastic
systems with complex fractional\linebreak
\\[-12pt]
\hspace*{23pt}order Bessel nonlinearities&3&55--65\\[.23pt]
\Avtors{Sinitsyn~I.\,N.} Orthogonal supoptimal filters for nonlinear
stochastic systems on manifolds&1&34--44\\[.23pt]
\Avtors{Sinitsyn~I.\,N.\ and Korepanov~E.\,R.} Normal Pugachev
conditionally-optimal filters and extra-\linebreak
\\[-12pt]
\hspace*{23pt}polators for state linear stochastic systems&2&14--23\\[.23pt]
\Avtors{Sinitsyn~I.\,N.\ and Sinitsyn~V.\,I.} Analytical modeling of
distributions in stochastic systems on\linebreak
\\[-12pt]
\hspace*{23pt}manifolds based on ellipsoidal approximation&1&45--55\\[.23pt]
\Avtors{Sinitsyn~I.\,N., Sinitsyn~V.\,I., and
Korepanov~E.\,R.} Ellipsoidal suboptimal filters for nonlinear\linebreak
\\[-12pt]
\hspace*{23pt}stochastic systems on manifolds&2&24--35\\[.23pt]
\Avtors{Sinitsyn~V.\,I.} see~Sinitsyn~I.\,N.&&\\[.23pt]
\Avtors{Sinitsyn~V.\,I.} see~Sinitsyn~I.\,N.&&\\[.23pt]
\Avtors{Skvortsov~N.\,A.} see~Stupnikov~S.\,A.&&\\[.23pt]
\Avtors{Sokolov~I.\,A.} see~Chertok~A.\,V.&&\\
\end{tabular}
}
\pagebreak

\def\leftfootline{\small{\textbf{\thepage}
\hfill INFORMATIKA I EE PRIMENENIYA~--- INFORMATICS AND APPLICATIONS\ \ \ 2016\
\ \ volume~10\ \ \ issue\ 4}
}%
 \def\rightfootline{\small{INFORMATIKA I EE PRIMENENIYA~---
INFORMATICS AND APPLICATIONS\ \ \ 2016\ \ \ volume~10\ \ \ issue\ 4
\hfill \textbf{\thepage}}}

\def\leftkol{2016 AUTHOR INDEX} % ENGLISH ABSTRACTS}

\def\rightkol{2016 AUTHOR INDEX} %ENGLISH ABSTRACTS}


{\tabcolsep=3pt
\begin{tabular}{p{382pt}cc}
&\textbf{Issue} & \textbf{Page}\\[6pt]
\Avtors{Sopin~E.\,S.} see~Gaidamaka~Yu.\,V.&&\\
\Avtors{Strijov~V.\,V.} see~Goncharov~A.\,V.&&\\
\Avtors{Strijov~V.\,V.} see~Isachenko~R.\,V.&&\\
\Avtors{Strijov~V.\,V.} see~Karasikov~M.\,E.&&\\
\Avtors{Stupnikov~S.\,A., Briukhov~D.\,O., and Skvortsov~N.\,A.}
Co-lending systemic risk analysis over\linebreak
\\[-12pt]
\hspace*{23pt}heterogeneous data collections&1&23--33\\
\Avtors{Stupnikov~S.\,A.} see~Kalinichenko~L.\,A.&&\\
\Avtors{Suchkov~A.\,P.} see~Zatsarinny~A.\,A.&&\\
\Avtors{Timonina~E.\,E.} see~Grusho~A.\,A.&&\\
\Avtors{Titova~A.\,I.} see~Kudryavtsev~A.\,A.&&\\
\Avtors{Turlikov~A.\,M.} see~Ometov~A.\,Ya.&&\\
\Avtors{Tyrsin~A.\,N.\ and Serebryanskii~S.\,M.} Recognition of
dependences on the basis of inverse\linebreak
\\[-12pt]
\hspace*{23pt}mapping&2&58--64\\
\Avtors{Ulyanov~V.\,V.} see~Markov~A.\,S.&&\\
\Avtors{Ushakov~V.\,G.} Queueing system with working vacations and
hyperexponential input stream&2&92--97\\
\Avtors{Ushakov~V.\,G.} see~Leontyev~N.\,D.&&\\
\Avtors{Volnova~A.\,A.} see~Kalinichenko~L.\,A.&&\\
\Avtors{Yakovlev~O.\,A.\ and Gasilov~A.\,V.} Speeded-up stereo
matching using geodesic support weights&3&\hphantom{1}98--104\\
\Avtors{Zabezhailo~M.\,I.} see~Grusho~A.\,A.&&\\
\Avtors{Zabezhailo~M.\,I.} see~Grusho~A.\,A.&&\\
\Avtors{Zakharova~T.\,V.\ and Shestakov~O.\,V.} Precision analysis of
wavelet processing of aerodynamic\linebreak
\\[-12pt]
\hspace*{23pt}flow patterns&3&46--54\\
\Avtors{Zalizniak~Anna~A.\ and Kruzhkov~M.\,G.} Database
of~Russian impersonal verbal constructions&4&132--141\\
\Avtors{Zasypko~V.\,V.} see~Shnurkov~P.\,V.&&\\
\Avtors{Zatsarinny~A.\,A.\ and Suchkov~A.\,P.} Systems engineering
approaches to~the~establishment of\linebreak
\\[-12pt]
\hspace*{23pt}a~system for~decision support based
on~situational analysis&4&105--113\\
\Avtors{Zatsarinny~A.\,A.} see~Grusho~A.\,A.&&\\
\Avtors{Zatsman~I.\,M., Inkova~O.\,Yu., Kruzhkov~M.\,G., and
Popkova~N.\,A.} Representation of cross-\linebreak
\\[-12pt]
\hspace*{23pt}lingual knowledge about
connectors in supracorpora databases&1&106--118\\
\Avtors{Zatsman~I.\,M.} see~Minin~V.\,A.&&\\
\Avtors{Zeifman~A.\,I.} see~Korolev~V.\,Yu.&&\\
\Avtors{Zeifman~A.\,I.} see~Korolev~V.\,Yu.&&\\
\end{tabular}
}

%\thispagestyle{myheadings}
\def\leftfootline{\small{\textbf{\thepage}
\hfill INFORMATIKA I EE PRIMENENIYA~--- INFORMATICS AND APPLICATIONS\ \ \ 2016\
\ \ volume~10\ \ \ issue\ 4}
}%
 \def\rightfootline{\small{INFORMATIKA I EE PRIMENENIYA~---
INFORMATICS AND APPLICATIONS\ \ \ 2016\ \ \ volume~10\ \ \ issue\ 4
\hfill \textbf{\thepage}}}

 \label{end\stat}

\newpage


%\vspace*{-60pt} {\small
{\baselineskip=9.1pt
\section*{Правила подготовки рукописей статей для публикации в журнале
<<Информатика и её применения>>}

\thispagestyle{empty}

 Журнал <<Информатика и её применения>> публикует
теоретические, обзорные и дискуссионные статьи, посвященные научным
исследованиям и разработкам в области информатики и ее приложений. Журнал
издается на русском языке. По специальному решению редколлегии отдельные статьи,
в виде исключения, могут печататься на английском языке.
Тематика журнала охватывает следующие направления:
\begin{itemize}
\item теоретические основы информатики; %\\[-13.5pt]
\item математические методы исследования сложных систем и процессов; %\\[-13.5pt]
\item информационные системы и сети; %\\[-13.5pt]
\item информационные технологии; %\\[-13.5pt]
\item архитектура и программное
обеспечение вычислительных комплексов и сетей.
\end{itemize}
\begin{enumerate}
\item В журнале печатаются результаты, ранее не
опубликованные и не предназначенные к одновременной публикации в других
изданиях. Публикация не должна нарушать закон об авторских правах. Направляя
свою рукопись в редакцию, авторы автоматически передают учредителям и
редколлегии неисключительные права на издание данной статьи на русском языке и
на ее распространение в России и за рубежом. При этом за авторами сохраняются
все права как собственников данной рукописи. В связи с этим авторами должно
быть представлено в редакцию письмо в следующей форме:
Соглашение о передаче права на публикацию:

\textit{<<Мы, нижеподписавшиеся, авторы рукописи <<$\qquad\qquad$>>, передаем
учредителям и редколлегии журнала <<Информатика и её применения>>
неисключительное право опубликовать данную рукопись статьи на русском языке как
в печатной, так и в электронной версиях журнала. Мы подтверждаем, что данная
публикация не нарушает авторского права других лиц или организаций. Подписи
авторов: (ф.\,и.\,о., дата, адрес)>>.}

Указанное соглашение может быть представлено 
как в бумажном виде, так и в виде отсканированной копии (с подписями авторов).


Редколлегия вправе запросить у авторов экспертное заключение о возможности
опубликования представленной статьи в открытой печати. %\\[-13.5pt]
\item Статья
подписывается всеми авторами. На отдельном листе представляются данные автора
(или всех авторов): фамилия, полные имя и отчество, телефон, факс, e-mail,
почтовый адрес. Если работа выполнена несколькими авторами, указывается фамилия
одного из них, ответственного за переписку с редакцией. %\\[-13.5pt]
\item Редакция журнала
осуществляет самостоятельную экспертизу присланных статей. Возвращение рукописи
на доработку не означает, что статья уже принята к печати. Доработанный вариант
с ответом на замечания рецензента необходимо прислать в редакцию. %\\[-13.5pt]
\item Решение
редакционной коллегии о принятии статьи к печати или ее отклонении сообщается
авторам. Редколлегия не обязуется направлять рецензию авторам отклоненной
статьи. %\\[-13.5pt]
\item Корректура статей высылается авторам для просмотра. Редакция
просит авторов присылать свои замечания в кратчайшие сроки. %\\[-13.5pt]
\item При
подготовке рукописи в MS Word рекомендуется использовать следующие настройки.
Параметры страницы: формат~--- А4; ориентация~--- книжная; поля (см): внутри~---
2,5, снаружи~--- 1,5, сверху~--- 2, снизу~--- 2, от края до нижнего
колонтитула~--- 1,3. Основной текст: стиль~--- <<Обычный>>: шрифт Times New
Roman, размер 14~пунктов, абзацный отступ~--- 0,5~см, 1,5 интервала,
выравнивание~--- по ширине. Рекомендуемый объем рукописи~--- не свыше
25~страниц указанного формата. Ознакомиться с шаблонами, содержащими примеры
оформления, можно по адресу в Интернете:
\textsf{http://www.ipiran.ru/journal/template.doc}.
\item К рукописи, предоставляемой в 2-х
экземплярах, обязательно прилагается электронная версия статьи (как правило, в
форматах MS WORD (.doc) или \LaTeX\ (.tex), а также~--- дополнительно~--- в
формате .pdf) на дискете, лазерном диске или по электронной почте. Сокращения
слов, кроме стандартных, не применяются. Все страницы рукописи должны быть
пронумерованы. %\\[-13.5pt]
\item Статья должна содержать следующую информацию на русском и
английском языках: название, Ф.И.О. авторов, места работы авторов и их
электронные адреса, подробные сведения об авторах, оформленные в соответствии с форматом, 
определяемым файлами {\sf http://www.ipiran.ru/journal/issues/2011\_05\_01/authors.asp} и 
{\sf http://www.ipiran.ru/journal/issues/2011\_01\_eng/authors.asp},
аннотация (не более 100~слов), ключевые слова. Ссылки на
литературу в тексте статьи нумеруются (в квадратных скобках) и располагаются в
порядке их первого упоминания. В~списке литературы не должно быть позиций, на которые нет ссылки в тексте статьи.
Все фамилии авторов, заглавия статей, названия
книг, конференций и~т.\,п.\ даются на языке оригинала, если этот язык
использует кириллический или латинский алфавит. %\\[-13.5pt]
\item Присланные в редакцию материалы авторам не возвращаются.
\item При отправке файлов по электронной
почте просим придерживаться следующих правил:
\begin{itemize}
\item указывать в поле subject (тема) название журнала и фамилию автора; %\\[-13.5pt]
\item использовать attach (присоединение); %\\[-13.5pt]
\item в случае больших объемов информации возможно
использование общеизвестных архиваторов (ZIP, RAR); %\\[-13.5pt]
\item в состав электронной версии статьи должны входить: файл, содержащий текст статьи, и файл(ы),
содержащий(е) иллюстрации. %\\[-13.5pt]
\end{itemize}
\item Журнал <<Информатика и её применения>> является некоммерческим изданием. 
Плата за публикацию с авторов не взимается, гонорар авторам не выплачивается.
\end{enumerate}
\thispagestyle{empty}
\textbf{Адрес редакции:} Москва 119333,
ул.~Вавилова, д.~44, корп.~2, ИПИ РАН\\
\hphantom{\textbf{Адрес редакции:} }Тел.: +7 (499) 135-86-92\ \
Факс:  +7 (495) 930-45-05\ \  E-mail:   rust@ipiran.ru }
}

\end{document}


%\tableofcontents

%\end{document}





%\def\stat{cont}
{%\hrule\par
%\vskip 7pt % 7pt
\raggedleft\Large \bf%\baselineskip=3.2ex
А\,В\,Т\,О\,Р\,С\,К\,И\,Й\ \ У\,К\,А\,З\,А\,Т\,Е\,Л\,Ь\ \ З\,А\ \ 2\,0\,0\,7 г. \vskip 17pt
    \hrule
    \par
\vskip 21pt plus 6pt minus 3pt }

\label{st\stat}

\def\tit{\ }

\def\aut{\ }
\def\auf{\ }

\def\leftkol{\ } % ENGLISH ABSTRACTS}

\def\rightkol{\ } %ENGLISH ABSTRACTS}

\titele{\tit}{\aut}{\auf}{\leftkol}{\rightkol}


\contentsline {chapter}{\ }{Выпуск \quad Стр.} 
\contentsline {section}{\textbf{Батракова Д.\,А., Королев В.\,Ю., Шоргин С.\,Я.}\ \ Новый метод вероятностно-ста\-ти\-сти\-че\-ско\-го анализа информационных потоков в\nobreakspace {}телекоммуникационных сетях}{\qquad 1 \qquad 40} 
\contentsline {section}{\textbf{Борисов А.\,В.}\ \ Байесовское оценивание в системах наблюдения с\nobreakspace {}марковскими скачкообразными процессами: игровой подход}{\qquad 2 \qquad 65}
\contentsline {section}{\textbf{Босов А.\,В., Иванов А.\,В.}\ \ Программная инфраструктура информационного Web-пор\-тала}{\qquad 2 \qquad 50}
\contentsline {section}{\textbf{Захаров В.\,Н., Калиниченко Л.\,А., Соколов И.\,А., Ступников С.\,А.}\ \ Конструирование канонических информационных моделей для интегрированных информационных систем}{\qquad 2 \qquad 15}
\contentsline {section}{\textbf{Захаров В.\,Н., Козмидиади В.\,А.}\ \ Средства обеспечения отказоустойчивости при\-ло\-жений}{\qquad 1 \qquad 14} 
\contentsline {section}{\textbf{Иванов А.\,В.}\ \ см. Босов А.\,В.\hfill\hfill\hfill\hfill\hfill\hfill\hfill\hfill\hfill\hfill\hfill\hfill\hfill\hfill\hfill\hfill\hfill\hfill\hfill\hfill\hfill\hfill\hfill\hfill\hfill\hfill\hfill\hfill\hfill\hfill\hfill\hfill\hfill\hfill\hfill}{\ }
\contentsline {section}{\textbf{Ильин В.\,Д., Соколов И.\,А.}\ \ Символьная модель системы знаний информатики в\nobreakspace {}че\-ло\-ве\-ко-автоматной среде}{\qquad 1 \qquad 66} 
\contentsline {section}{\textbf{Калиниченко Л.\,А.}\ \ см. Захаров В.\,Н.\hfill\hfill\hfill\hfill\hfill\hfill\hfill\hfill\hfill\hfill\hfill\hfill\hfill\hfill\hfill\hfill\hfill\hfill\hfill\hfill\hfill\hfill\hfill\hfill\hfill\hfill\hfill\hfill\hfill\hfill\hfill\hfill\hfill\hfill\hfill}{\ }
\contentsline {section}{\textbf{Козеренко Е.\,Б.}\ \ Лингвистическое моделирование для систем машинного перевода и обработки знаний}{\qquad 1 \qquad 54} 
\contentsline {section}{\textbf{Козмидиади В.\,А.}\ \ см. Захаров В.\,Н.\hfill\hfill\hfill\hfill\hfill\hfill\hfill\hfill\hfill\hfill\hfill\hfill\hfill\hfill\hfill\hfill\hfill\hfill\hfill\hfill\hfill\hfill\hfill\hfill\hfill\hfill\hfill\hfill\hfill\hfill\hfill\hfill\hfill\hfill\hfill }{\ } 
\contentsline {section}{\textbf{Королев В.\,Ю.}\ \ см. Батракова Д.\,А.\hfill\hfill\hfill\hfill\hfill\hfill\hfill\hfill\hfill\hfill\hfill\hfill\hfill\hfill\hfill\hfill\hfill\hfill\hfill\hfill\hfill\hfill\hfill\hfill\hfill\hfill\hfill\hfill\hfill\hfill\hfill\hfill\hfill\hfill\hfill}{\ } 
\contentsline {section}{\textbf{Кудрявцев А.\,А., Шоргин С.\,Я.}\ \ Байесовский подход к\nobreakspace {}анализу систем массового обслуживания и\nobreakspace {}показателей надежности}{\qquad 2 \qquad 76}
\contentsline {section}{\textbf{Печинкин А.\,В., Соколов И.\,А., Чаплыгин В.\,В.}\ \ Многолинейная система массового обслуживания с конечным накопителем и ненадежными приборами}{\qquad 1 \qquad 27} 
\contentsline {section}{\textbf{Печинкин А.\,В., Соколов И.\,А., Чаплыгин В.\,В.}\ \ Стационарные характеристики многолинейной\nobreakspace {}системы массового обслуживания с\nobreakspace {}одновременными отказами приборов}{\qquad 2 \qquad 39}
\contentsline {section}{\textbf{Синицын И.\,Н.}\ \ Корреляционные методы построения аналитических информационных моделей флуктуаций полюса Земли по априорным данным}{\qquad 2 \qquad \hphantom{9}2}
\contentsline {section}{\textbf{Синицын И.\,Н.}\ \ Развитие теории фильтров Пугачева для оперативной обработки информации в стохастических системах}{{\qquad 1 \qquad \hphantom{9}3}} 
\contentsline {section}{\textbf{Соколов И.\,А.}\ \ см. Захаров В.\,Н.\hfill\hfill\hfill\hfill\hfill\hfill\hfill\hfill\hfill\hfill\hfill\hfill\hfill\hfill\hfill\hfill\hfill\hfill\hfill\hfill\hfill\hfill\hfill\hfill\hfill\hfill\hfill\hfill\hfill\hfill\hfill\hfill\hfill\hfill\hfill}{\ }
\contentsline {section}{\textbf{Соколов И.\,А.}\ \ см. Ильин В.\,Д.\hfill\hfill\hfill\hfill\hfill\hfill\hfill\hfill\hfill\hfill\hfill\hfill\hfill\hfill\hfill\hfill\hfill\hfill\hfill\hfill\hfill\hfill\hfill\hfill\hfill\hfill\hfill\hfill\hfill\hfill\hfill\hfill\hfill\hfill\hfill}{\ } 
\contentsline {section}{\textbf{Соколов И.\,А.}\ \ см. Печинкин А.\,В.\hfill\hfill\hfill\hfill\hfill\hfill\hfill\hfill\hfill\hfill\hfill\hfill\hfill\hfill\hfill\hfill\hfill\hfill\hfill\hfill\hfill\hfill\hfill\hfill\hfill\hfill\hfill\hfill\hfill\hfill\hfill\hfill\hfill\hfill\hfill}{\ } 
\contentsline {section}{\textbf{Соколов И.\,А.}\ \ см. Печинкин А.\,В.\hfill\hfill\hfill\hfill\hfill\hfill\hfill\hfill\hfill\hfill\hfill\hfill\hfill\hfill\hfill\hfill\hfill\hfill\hfill\hfill\hfill\hfill\hfill\hfill\hfill\hfill\hfill\hfill\hfill\hfill\hfill\hfill\hfill\hfill\hfill}{\ }
\contentsline {section}{\textbf{Ступников С.\,А.}\ \ см. Захаров В.\,Н.\hfill\hfill\hfill\hfill\hfill\hfill\hfill\hfill\hfill\hfill\hfill\hfill\hfill\hfill\hfill\hfill\hfill\hfill\hfill\hfill\hfill\hfill\hfill\hfill\hfill\hfill\hfill\hfill\hfill\hfill\hfill\hfill\hfill\hfill\hfill}{\ }
\contentsline {section}{\textbf{Чаплыгин В.\,В.}\ \ см. Печинкин А.\,В.\hfill\hfill\hfill\hfill\hfill\hfill\hfill\hfill\hfill\hfill\hfill\hfill\hfill\hfill\hfill\hfill\hfill\hfill\hfill\hfill\hfill\hfill\hfill\hfill\hfill\hfill\hfill\hfill\hfill\hfill\hfill\hfill\hfill\hfill\hfill}{\ } 
\contentsline {section}{\textbf{Чаплыгин В.\,В.}\ \ см. Печинкин А.\,В.\hfill\hfill\hfill\hfill\hfill\hfill\hfill\hfill\hfill\hfill\hfill\hfill\hfill\hfill\hfill\hfill\hfill\hfill\hfill\hfill\hfill\hfill\hfill\hfill\hfill\hfill\hfill\hfill\hfill\hfill\hfill\hfill\hfill\hfill\hfill}{\ }
\contentsline {section}{\textbf{Шоргин С.\,Я.}\ \ см. Батракова Д.\,А.\hfill\hfill\hfill\hfill\hfill\hfill\hfill\hfill\hfill\hfill\hfill\hfill\hfill\hfill\hfill\hfill\hfill\hfill\hfill\hfill\hfill\hfill\hfill\hfill\hfill\hfill\hfill\hfill\hfill\hfill\hfill\hfill\hfill\hfill\hfill}{\ } 
\contentsline {section}{\textbf{Шоргин С.\,Я.}\ \ см. Кудрявцев А.\,А.\hfill\hfill\hfill\hfill\hfill\hfill\hfill\hfill\hfill\hfill\hfill\hfill\hfill\hfill\hfill\hfill\hfill\hfill\hfill\hfill\hfill\hfill\hfill\hfill\hfill\hfill\hfill\hfill\hfill\hfill\hfill\hfill\hfill\hfill\hfill}{\ }
%\thispagestyle{myheadings}
\def\leftfootline{\small{\textbf{\thepage}
\hfill ИНФОРМАТИКА И ЕЁ ПРИМЕНЕНИЯ\ \ \ том~1\ \ \ выпуск~2\ \ \ 2007}
}%
 \def\rightfootline{\small{ИНФОРМАТИКА И ЕЁ ПРИМЕНЕНИЯ\ \ \ том~1\ \ \ выпуск~2\ \ \ 2007
 \hfill \textbf{\thepage}}}
 \label{end\stat}

%\def\stat{cont-e}
{%\hrule\par
%\vskip 7pt % 7pt
\raggedleft\Large \bf%\baselineskip=3.2ex
2\,0\,0\,7\ \ A\,U\,T\,H\,O\,R\ \ I\,N\,D\,E\,X \vskip 17pt
    \hrule
    \par
\vskip 21pt plus 6pt minus 3pt }

\label{st\stat}

\def\tit{\ }

\def\aut{\ }
\def\auf{\ }

\def\leftkol{\ } % ENGLISH ABSTRACTS}

\def\rightkol{\ } %ENGLISH ABSTRACTS}

\titele{\tit}{\aut}{\auf}{\leftkol}{\rightkol}


\contentsline {chapter}{\ }{Issue \quad Page} 
\contentsline {subsection}{\textbf{Batrakova D.\,A., Korolev V.\,Yu., Shorgin S.\,Ya.}\ \ A New Method for the Probabilistic and Statistical Analysis of Information Flows in Telecommunication Networks}{\qquad 1 \qquad 40} 
\contentsline {subsection}{\textbf{Borisov A.\,V.}\ \ Bayesian Estimation in\nobreakspace {}Observation Systems with\nobreakspace {}Markov Jump Processes: Game-Theoretic Approach}{\qquad 2 \qquad 65} 
\contentsline {subsection}{\textbf{Bosov A.\,V., Ivanov A.\,V.}\ \ Linguistic Simulation for Machine Translation and Knowledge Management Systems}{\qquad 2 \qquad 50} 
\contentsline {subsection}{\textbf{Chaplygin V.\,V.} see Pechinkin A.\,V.\hfill\hfill\hfill\hfill\hfill\hfill\hfill\hfill\hfill\hfill\hfill\hfill\hfill\hfill\hfill\hfill\hfill\hfill\hfill\hfill\hfill\hfill\hfill\hfill\hfill\hfill\hfill\hfill\hfill\hfill\hfill\hfill\hfill\hfill\hfill}{\ }
\contentsline {subsection}{\textbf{Chaplygin V.\,V.} see Pechinkin A.\,V.\hfill\hfill\hfill\hfill\hfill\hfill\hfill\hfill\hfill\hfill\hfill\hfill\hfill\hfill\hfill\hfill\hfill\hfill\hfill\hfill\hfill\hfill\hfill\hfill\hfill\hfill\hfill\hfill\hfill\hfill\hfill\hfill\hfill\hfill\hfill}{\ }
\contentsline {subsection}{\textbf{Ilyin V.\,D., Sokolov I.\,A.}\ \ The Symbol Model of Informatics Knowledge System in Human-Automaton Environment}{\qquad 1 \qquad 66} 
\contentsline {subsection}{\textbf{Ivanov A.\,V.} see Bosov A.\,V.\hfill\hfill\hfill\hfill\hfill\hfill\hfill\hfill\hfill\hfill\hfill\hfill\hfill\hfill\hfill\hfill\hfill\hfill\hfill\hfill\hfill\hfill\hfill\hfill\hfill\hfill\hfill\hfill\hfill\hfill\hfill\hfill\hfill\hfill\hfill}{\ }
\contentsline {subsection}{\textbf{Kalinichenko L.\,A.} see Zakharov V.\,N.\hfill\hfill\hfill\hfill\hfill\hfill\hfill\hfill\hfill\hfill\hfill\hfill\hfill\hfill\hfill\hfill\hfill\hfill\hfill\hfill\hfill\hfill\hfill\hfill\hfill\hfill\hfill\hfill\hfill\hfill\hfill\hfill\hfill\hfill\hfill}{\ }
\contentsline {subsection}{\textbf{Korolev V.\,Yu.} see Batrakova D.\,A.\hfill\hfill\hfill\hfill\hfill\hfill\hfill\hfill\hfill\hfill\hfill\hfill\hfill\hfill\hfill\hfill\hfill\hfill\hfill\hfill\hfill\hfill\hfill\hfill\hfill\hfill\hfill\hfill\hfill\hfill\hfill\hfill\hfill\hfill\hfill}{\ }
\contentsline {subsection}{\textbf{Kozerenko E.\,B.}\ \ Linguistic Simulation for Machine Translation and Knowledge Management Systems}{\qquad 1 \qquad 54} 
\contentsline {subsection}{\textbf{Kozmidiady V.\,A.} see Zakharov V.\,N.\hfill\hfill\hfill\hfill\hfill\hfill\hfill\hfill\hfill\hfill\hfill\hfill\hfill\hfill\hfill\hfill\hfill\hfill\hfill\hfill\hfill\hfill\hfill\hfill\hfill\hfill\hfill\hfill\hfill\hfill\hfill\hfill\hfill\hfill\hfill}{\ }
\contentsline {subsection}{\textbf{Kudryavtsev A.\,A., Shorgin S.\,Ya.}\ \ Bayesian Approach to Queueing Systems and Reliability Characteristics}{\qquad 2 \qquad 76} 
\contentsline {subsection}{\textbf{Pechinkin A.\,V., Sokolov I.\,A., Chaplygin V.\,V.}\ \ Multichannel Queuing System with Finite Buffer and Unreliable Servers}{\qquad 1 \qquad 27} 
\contentsline {subsection}{\textbf{Pechinkin A.\,V., Sokolov I.\,A., Chaplygin V.\,V.}\ \ Stationary Characteristics of a Multichannel Queueing System with\nobreakspace {}Simultaneous Refusals of Servers}{\qquad 2 \qquad 39} 
\contentsline {subsection}{\textbf{Shorgin S.\,Ya.} see Batrakova D.\,A.\hfill\hfill\hfill\hfill\hfill\hfill\hfill\hfill\hfill\hfill\hfill\hfill\hfill\hfill\hfill\hfill\hfill\hfill\hfill\hfill\hfill\hfill\hfill\hfill\hfill\hfill\hfill\hfill\hfill\hfill\hfill\hfill\hfill\hfill\hfill}{\ }
\contentsline {subsection}{\textbf{Shorgin S.\,Ya.} see Kudryavtsev A.\,A.\hfill\hfill\hfill\hfill\hfill\hfill\hfill\hfill\hfill\hfill\hfill\hfill\hfill\hfill\hfill\hfill\hfill\hfill\hfill\hfill\hfill\hfill\hfill\hfill\hfill\hfill\hfill\hfill\hfill\hfill\hfill\hfill\hfill\hfill\hfill}{\ }
\contentsline {subsection}{\textbf{Sinitsyn I.\,N.}\ \ Correlational Methods for Analytical Informational Models of the Earth Pole Fluctuations Design Based on a priori Data}{\qquad 2 \qquad \hphantom{9}2}
\contentsline {subsection}{\textbf{Sinitsyn I.\,N.}\ \ Development of Pugachev Filtering for Stochastic Systems}{\qquad 1 \qquad \hphantom{9}3}
\contentsline {subsection}{\textbf{Sokolov I.\,A.} see Ilyin V.\,D.\hfill\hfill\hfill\hfill\hfill\hfill\hfill\hfill\hfill\hfill\hfill\hfill\hfill\hfill\hfill\hfill\hfill\hfill\hfill\hfill\hfill\hfill\hfill\hfill\hfill\hfill\hfill\hfill\hfill\hfill\hfill\hfill\hfill\hfill\hfill}{\ }
\contentsline {subsection}{\textbf{Sokolov I.\,A.} see Pechinkin A.\,V.\hfill\hfill\hfill\hfill\hfill\hfill\hfill\hfill\hfill\hfill\hfill\hfill\hfill\hfill\hfill\hfill\hfill\hfill\hfill\hfill\hfill\hfill\hfill\hfill\hfill\hfill\hfill\hfill\hfill\hfill\hfill\hfill\hfill\hfill\hfill}{\ }
\contentsline {subsection}{\textbf{Sokolov I.\,A.} see Pechinkin A.\,V.\hfill\hfill\hfill\hfill\hfill\hfill\hfill\hfill\hfill\hfill\hfill\hfill\hfill\hfill\hfill\hfill\hfill\hfill\hfill\hfill\hfill\hfill\hfill\hfill\hfill\hfill\hfill\hfill\hfill\hfill\hfill\hfill\hfill\hfill\hfill}{\ }
\contentsline {subsection}{\textbf{Sokolov I.\,A.} see Zakharov V.\,N.\hfill\hfill\hfill\hfill\hfill\hfill\hfill\hfill\hfill\hfill\hfill\hfill\hfill\hfill\hfill\hfill\hfill\hfill\hfill\hfill\hfill\hfill\hfill\hfill\hfill\hfill\hfill\hfill\hfill\hfill\hfill\hfill\hfill\hfill\hfill}{\ }
\contentsline {subsection}{\textbf{Stupnikov S.\,A.} see Zakharov V.\,N.\hfill\hfill\hfill\hfill\hfill\hfill\hfill\hfill\hfill\hfill\hfill\hfill\hfill\hfill\hfill\hfill\hfill\hfill\hfill\hfill\hfill\hfill\hfill\hfill\hfill\hfill\hfill\hfill\hfill\hfill\hfill\hfill\hfill\hfill\hfill}{\ }
\contentsline {subsection}{\textbf{Zakharov V.\,N., Kalinichenko L.\,A., Sokolov I.\,A., Stupnikov S.\,A.}\ \ Development of Canonical Information Models for Integrated Information Systems}{\qquad 2 \qquad 15} 
\contentsline {subsection}{\textbf{Zakharov V.\,N., Kozmidiady V.\,A.}\ \ Means Providing Applications Fault Tolerance}{\qquad 1 \qquad 14} 
\def\leftfootline{\small{\textbf{\thepage}
\hfill ИНФОРМАТИКА И ЕЁ ПРИМЕНЕНИЯ\ \ \ том~1\ \ \ выпуск~2\ \ \ 2007}
}%
 \def\rightfootline{\small{ИНФОРМАТИКА И ЕЁ ПРИМЕНЕНИЯ\ \ \ том~1\ \ \ выпуск~2\ \ \ 2007
 \hfill \textbf{\thepage}}}
 \label{end\stat}


%\tableofcontents


\end{document}

\newcommand{\Ack}{\subsection*{\protect\large\bf Acknowledgments}}