\def\stat{abstr}
{%\hrule\par
%\vskip 7pt % 7pt
\raggedleft\Large \bf%\baselineskip=3.2ex
A\,B\,S\,T\,R\,A\,C\,T\,S \vskip 17pt
    \hrule
    \par
\vskip 21pt plus 6pt minus 3pt }

\label{st\stat}


%\def\rightkol{ENGLISH ABSTRACTS}
%\def\leftkol{\ }  %ENGLISH ABSTRACTS}


%1
\def\tit{UNSUPERVISED APPROACH TO~WEB WRAPPER MAINTENANCE}

\def\aut{A.\,M.~Andreev$^1$, D.\,V.~Berezkin$^2$, I.\,A.~Kozlov$^3$, and~K.\,V.~Simakov$^4$}

\def\auf{$^1$Bauman Moscow State Technical University, arkandreev@gmail.com\\[1pt]
  $^2$Bauman Moscow State Technical University, dmitryb2007@yandex.ru\\[1pt]
  $^3$Bauman Moscow State Technical University, kozlovilya89@gmail.com\\[1pt]
  $^4$Bauman Moscow State Technical University, skv@ixlab.ru}

\def\leftkol{\ } % ENGLISH ABSTRACTS}
\def\rightkol{\ } %ENGLISH ABSTRACTS}

\titele{\tit}{\aut}{\auf}{\leftkol}{\rightkol}

\vspace*{-2pt}

\noindent 
HTML-wrapper applications rely on formatting regularities of targeted 
websites. Therefore, maintenance of such applications is connected with the 
problem of detecting markup changes of web pages. This article describes the 
unsupervised approach to this problem. The proposed method of detection 
consists of two parts: the real-time one based on clustering considering HTML-document 
as a vector of some features and the time-lagged one based on comparison of 
distributions of such features for learning and testing sets of HTML-documents. 
There have been carried out several experiments with data obtained from real 
wrapper. The results reveal feasibility of the suggested approach.


\vspace*{-5pt}

\KWN{wrapper maintenance; web-site parsing; clustering; HTML-markup statistical 
processing}

%\thispagestyle{myheadings}



\vskip 8pt plus 6pt minus 3pt

%\pagebreak


%2
\def\tit{BUILDING REAL-TIME NEWS RECOMMENDATION SERVICE USING NoSQL DBMS}

\def\aut{P.\,A.~Klemenkov}


\def\auf{M.\,V.~Lomonosov Moscow State University, parser@cs.msu.su}


\def\leftkol{\ } % ENGLISH ABSTRACTS}

\def\rightkol{\ } %ENGLISH ABSTRACTS}

\titele{\tit}{\aut}{\auf}{\leftkol}{\rightkol}

\vspace*{-2pt}

\noindent The analysis of user interaction with a Web application, the methods 
of conducting such an analysis, and their shortcomings are discussed. An 
implementation of the news recommendation service using existing approaches is 
described. A~new NoSQL approach to building recommendation systems that operate 
in near real time is suggested.

\vspace*{-5pt}

\KWN{recommendation systems; minhash; mapreduce; NoSQL}


%\pagebreak

 \vskip 8pt plus 6pt minus 3pt

%\pagebreak

\def\leftkol{\ } % ENGLISH ABSTRACTS}
\def\rightkol{\ } %ENGLISH ABSTRACTS}

 %3
\def\tit{A VERIFIABLE MAPPING OF A MULTIDIMENSIONAL ARRAY DATA MODEL\\ INTO AN 
OBJECT DATA MODEL}

\def\aut{S.\,A.~Stupnikov}

\def\auf{IPI RAN, ssa@ipi.ac.ru}


\titele{\tit}{\aut}{\auf}{\leftkol}{\rightkol}

\vspace*{-2pt}
 
\noindent The paper considers a mapping of a multidimensional array data model 
into an object data model. General principles of mappings of array data models 
into object data models are formulated. A~mapping of concrete models is also 
considered. The source model is the Array Data Model used in the SciDB DBMS. 
The target model is the SYNTHESIS language used as the canonical data model in 
the subject mediation technology. A~method for verification of the mapping is 
considered. Verification means a formal proof that the mapping preserves 
information and semantics of the operations. Verification is realized using the 
AMN formal specification language. A~practical aim of the paper is to provide a 
basis for virtual or materialized integration of array-based information 
resources.


\vspace*{-5pt}

\KWN{multidimensional arrays; object data model; data model mapping; database 
integration}

 \vskip 8pt plus 6pt minus 3pt

%4
\def\tit{STUDY OF THE WIKIPEDIA(EN) CATEGORIES GRAPH}

\def\aut{A.\,V.~Shkotin}

\def\auf{GIS department,  State 
Geological Museum of the Russian Academy of Sciences, ashkotin@acm.org}


%\def\leftkol{ENGLISH ABSTRACTS}
%\def\rightkol{ENGLISH ABSTRACTS}

\titele{\tit}{\aut}{\auf}{\leftkol}{\rightkol}

%\vspace*{-4pt}


\noindent Wikipedia is the outstanding project of knowledge accumulation 
both of general using and different areas of 
specialization. Quality check of this knowledge, especially automatic, is very 
important. In this paper, the results\linebreak\vspace*{-12pt}

\pagebreak

\noindent of studying a structure of the English 
version of WCG (Wikipedia Categories Graph) as a whole are presented. The WCG is a 
system that supports structure of knowledge and it is interesting to know what WCG 
includes and how it is arranged. It is shown that in graph, there are 
unacceptable logic violations and organizational and technical methods for 
elimination are discussed.


\vspace*{-5pt}

\KWN{Wikipedia; digraph; connected components; logical analysis}

\def\leftkol{ENGLISH ABSTRACTS}

\def\rightkol{ENGLISH ABSTRACTS}

 \vskip 10pt plus 6pt minus 3pt

%5
\def\tit{ACTIVE AUTHENTICATION METHODS USING KEYSTROKE DYNAMICS}

\def\aut{V.\,Yu.~Kaganov$^1$, A.\,K.~Korolyov$^2$, M.\,N.~Krylov$^3$, 
I.\,V.~Mashechkin$^4$, and~M.\,I.~Petrovskiy$^5$}

\def\auf{$^1$Faculty of Computational Mathematics and Cybernetics, M.\,V.~Lomonosov Moscow 
State University,\\ 
$\hphantom{^1}$vladhid@mlab.cs.msu.su\\[1pt]
$^2$Faculty of Computational Mathematics and Cybernetics, M.\,V.~Lomonosov Moscow State University,\\ 
$\hphantom{^1}$akorolev@mlab.cs.msu.su\\[1pt]
$^3$Faculty of Computational Mathematics and Cybernetics, M.\,V.~Lomonosov Moscow State University,\\ 
$\hphantom{^1}$krylovm@mlab.cs.msu.su\\[1pt]
$^4$Faculty of Computational Mathematics and Cybernetics, M.\,V.~Lomonosov Moscow State University,\\ 
$\hphantom{^1}$mash@cs.msu.su\\[1pt]
$^5$Faculty of Computational Mathematics and Cybernetics, M.\,V.~Lomonosov Moscow State University,\\
$\hphantom{^1}$michael@cs.msu.su}

%\def\leftkol{ENGLISH ABSTRACTS}
%\def\rightkol{ENGLISH ABSTRACTS}

\titele{\tit}{\aut}{\auf}{\leftkol}{\rightkol}

\vspace*{-2pt}

\def\leftkol{ENGLISH ABSTRACTS}

\def\rightkol{ENGLISH ABSTRACTS}

\noindent 
An overview of some effective methods of 
authentication using behavior models, created from keystroke dynamics data is presented. 
Also, a new data representation model was proposed, a number of experiments 
conducted using this model, and various algorithms of machine learing.


\vspace*{-5pt}

\KWN{wavelets; thresholding; risk estimate; normal distribution; rate of convergence}

 \vskip 10pt plus 6pt minus 3pt

%6
\def\tit{PROBLEMS OF THE ONLINE ACCESS TO SCIENTIFIC JOURNALS}


\def\aut{A.\,V.~Glushanovskii$^1$ and~N.\,E.~Kalenov$^2$}

\def\auf{$^1$Library for Natural Sciences, Russian Academy of Sciences, avglush@benran.ru\\[1pt]
$^2$Library for Natural Sciences, Russian Academy of Sciences, nek@benran.ru}


%\def\leftkol{ENGLISH ABSTRACTS}
%\def\rightkol{ENGLISH ABSTRACTS}

\titele{\tit}{\aut}{\auf}{\leftkol}{\rightkol}

\vspace*{-2pt}

\def\leftkol{ENGLISH ABSTRACTS}

\def\rightkol{ENGLISH ABSTRACTS}

\noindent
The problems of supplying with full-text scientific information 
access via Internet for the institutions of the Russian Academy of Sciences (RAS)
are considered. 
According to world practice, this task is resolved by the scientific 
libraries and libraries consortia for the best financial conditions. 
The practice of such access organization in Russia via Russian Foundation 
for Basic Research and National Electronic-information Consortia (NEICON) 
is described. The statistics of using NEICON provided online journals by 
RAS staff is considered. Organizational proposals for optimal decision of the 
task of online access to scientific information in the situation of financial 
limits in RAS are suggested.
 

\vspace*{-5pt}

\KWN{scientific journals; full texts; Internet; remote access; libraries; consortia}

%\pagebreak

 \vskip 14pt plus 6pt minus 3pt

%7
\def\tit{DECISION SUPPORT SYSTEMS MODELING
WITH SYNERGETIC ARTIFICIAL INTELLIGENCE}

\def\aut{I.\,A.~Kirikov$^1$, A.\,V.~Kolesnikov$^2$, and~S.\,V.~Listopad$^3$} 


\def\auf{$^1$Immanuel Kant Baltic Federal University (в статье другое);
Kaliningrad Branch of Institute of Informatics\\
$\hphantom{^1}$Problems, Russian Academy of Sciences, baltbipiran@mail.ru\\[1pt]
$^2$Kaliningrad Branch of Institute of Informatics Problems, Russian Academy of Sciences, avkolesnikov@yandex.ru\\[1pt]
$^3$Kaliningrad Branch of Institute of Informatics Problems, Russian Academy of Sciences, ser-list-post@yandex.ru}

%\def\leftkol{ENGLISH ABSTRACTS}
%\def\rightkol{ENGLISH ABSTRACTS}

\titele{\tit}{\aut}{\auf}{\leftkol}{\rightkol}

%\vspace*{-2pt}

\def\leftkol{ENGLISH ABSTRACTS}

\def\rightkol{ENGLISH ABSTRACTS}

\noindent
The approach to modeling collective effects of decision support 
systems within the paradigm of synergetic artificial intelligence is 
considered. The model and the functional structure of the hybrid intelligent 
multiagent system for modeling decision support systems are proposed. The 
results of computational experiments that demonstrate a positive impact of 
the self-organization effect on the quality of collective decisions are presented.



%\vspace*{-2pt}

\KWN{decision support computer system; 
hybrid intelligent multiagent system with self-organization}


 \vskip 14pt plus 6pt minus 3pt


%8
\def\tit{SEMANTICS OF ASPECT-ORIENTED MODELING OF~DATA AND~PROCESSES}

\def\aut{S.\,P.~Kovalyov} 

\def\auf{Institute of Control Problems, Russian Academy of Sciences, kovalyov@nm.ru}

%\def\leftkol{ENGLISH ABSTRACTS}
%\def\rightkol{ENGLISH ABSTRACTS}

\titele{\tit}{\aut}{\auf}{\leftkol}{\rightkol}

%\vspace*{-2pt}

\def\leftkol{ENGLISH ABSTRACTS}

\def\rightkol{ENGLISH ABSTRACTS}

%\vspace*{-2pt}

\noindent
An approach to semantic unification of aspect-oriented programming (AOP) 
technologies based on formalization by means of category theory is presented. 
Aspect-oriented programming technology is represented as a category of formal models of aspect-oriented 
programs and their interconnections equipped with functor of taking 
aspectual structure (labeling of models by concerns). Weaving of aspect-oriented 
programs is formalized as certain universal construction in this category. Formal 
AOP technologies applicable for reducing costs at modeling data and process scenarios 
are defined and considered. Weaving existence condition for scenario models is stated 
and justified.

%\vspace*{-2pt}

\KWN{aspect-oriented programming; category theory; aspect weaving}

\vskip 14pt plus 6pt minus 3pt



%9
\def\tit{COGNITIVE INTEROPERABILITY OF~EXPERT COLLABORATION IN~THE~TASK 
OF~THE~RUSSIAN-FRENCH PARALLEL
TEXTS PROCESSING:\\ LINGUISTIC AND~COGNITIVE ASPECTS}

\def\aut{O.\,S.~Kozhunova}

\def\auf{IPI RAN, kozhunovka@mail.ru}


\def\leftkol{ENGLISH ABSTRACTS}

\def\rightkol{ENGLISH ABSTRACTS}

\titele{\tit}{\aut}{\auf}{\leftkol}{\rightkol}

%\vspace*{-2pt}

\noindent
The resources of information and communication technologies ``Refillable linguistic data base 
on translation difficulties'' 
and ``Subject-oriented thesaurus of Russian-French parallel texts'' 
are discussed. The resources are at the design stage and to be 
implemented simultaneously with the Russian-French parallel corpus of 
belles-lettres. Apart from the functionality, linguistic and cognitive 
aspects of expert interaction within the task of the Russian-French 
parallel texts processing through cooperative efforts are considered.

%\vspace*{-2pt}

\KWN{cognitive interoperability; task of natural language processing; 
Russian-French parallel texts}
%\pagebreak

\vskip 14pt plus 6pt minus 3pt

%10
\def\tit{DATA ACQUISITION SIMULATION FOR NICA EXPERIMENT}

\def\aut{V.\,V.~Korenkov$^1$, A.\,V.~Nechaevskiy$^2$, and~V.\,V.~Trofimov$^3$}

\def\auf{$^1$Joint Institute for Nuclear Research, Laboratory of Information 
Technologies Dubna, korenkov@cv.jinr.ru\\[1pt]
$^2$Joint Institute for Nuclear Research, Laboratory of Information Technologies Dubna,\\
$\hphantom{^1}$Andrey.Nechaevskiy@gmail.com\\[1pt]
$^3$Joint Institute for Nuclear Research, Laboratory of Information Technologies Dubna, 
trofimov@jinr.ru}


\def\leftkol{ENGLISH ABSTRACTS}

\def\rightkol{ENGLISH ABSTRACTS}

\titele{\tit}{\aut}{\auf}{\leftkol}{\rightkol}

%\vspace*{-2pt}

\noindent
The need for simulation model of data storage and processing for NICA 
accelerator complex is shown. The base of the simulation model is GridSim. 
This paper describes an approach to simulation the dCache and network. 
A~simple example shows the case of the model use.


 
%\vspace*{-2pt}

\KWN{grid technologies; grid infrastructures; data storage systems; optimization; 
simulation; research; development; dCache; Tier1; NICA; Grid}

\pagebreak

%\vskip 14pt plus 6pt minus 3pt

% \vskip 12pt plus 6pt minus 3pt

%11
\def\tit{ESTIMATES OF THE RATE OF CONVERGENCE OF~THE~DISTRIBUTIONS 
OF~SOME RANDOM SUMS TO~STABLE LAWS}

\def\aut{V.\,Yu.~Korolev$^1$ and L.\,M.~Zaks$^2$}

\def\auf{$^1$Faculty of Computational Mathematics and Cybernetics, 
   M.\,V.~Lomonosov Moscow State University;  IPI RAN,\\
$\hphantom{^1}$vkorolev@cs.msu.su\\[1pt]
$^2$Department of Modeling and Mathematical Statistics, Alpha-Bank, lily.zaks@gmail.com}




\def\leftkol{ENGLISH ABSTRACTS}

\def\rightkol{ENGLISH ABSTRACTS}

\titele{\tit}{\aut}{\auf}{\leftkol}{\rightkol}

%\vspace*{-2pt}

\noindent 
Estimates are presented for the rate of convergence of the distributions 
of special sums of independent identically distributed random variables 
with finite variances to symmetric strictly stable laws. The distribution 
of the random index is assumed to be mixed Poisson in which the mixing 
distribution is a stable law concentrated on the positive half-line. 
The absolute constants are written out explicitly.


%\vspace*{-2pt}

\KWN{stable distribution; Berry--Esseen inequality; random sum; 
doubly stochastic Poisson process (Cox process); mixed Poisson distribution}

\vskip 14pt plus 6pt minus 3pt


%12
\def\tit{UNIVERSAL METRIC THESAURUS OF~RUSSIAN LANGUAGE}

\def\aut{L.\,A.~Kuznetsov$^1$, V.\,F.~Kuznetsova, and~A.\,V.~Kapnin$^3$}

\def\auf{$^1$Russian Presidential Academy of National 
Economy and Public Administration  (Lipetsk Branch),\\
$\hphantom{^1}$kuznetsov.leonid48@gmail.com\\[1pt]
$^2$Russian Presidential Academy of National Economy and Public Administration  
(Lipetsk Branch),\\
$\hphantom{^1}$kuznetsov.leonid48@gmail.com\\[1pt]
$^3$Lipetsk State Technical University, gert@inbox.ru}
   
 \def\leftkol{ENGLISH ABSTRACTS}

\def\rightkol{ENGLISH ABSTRACTS}

\titele{\tit}{\aut}{\auf}{\leftkol}{\rightkol}

%\vspace*{-2pt}

\noindent 
All Russian language available thesauri are compiled by expert groups. 
In the paper, the tools for automatic generating of a thesaurus are presented. 
The tools are based on a formal presentation of the texts explaining semantics of 
the words and a quantify assessment of the semantic distance between the 
words as a measure of their proximity. The proposed solutions allow to use 
the formal mathematical presentations that minimize subjectivity in 
assessing the proximity of the words. The solutions give an opportunity to 
synthesize automatic systems for evaluating the semantic proximity of the 
words and to solve other problems in the area of texts processing.



%\vspace*{-2pt}

\KWN{computational linguistics; universal thesaurus; metric thesaurus; 
semantic proximity assessment; semantic distance; information theory}


\vskip 14pt plus 6pt minus 3pt

%13
\def\tit{APPROXIMATION OF A~MULTIDIMENSIONAL DEPENDENCY BASED ON LINEAR EXPANSION\\ IN A 
DICTIONARY OF PARAMETRIC FUNCTIONS}

\def\aut{M.\,G.~Belyaev$^1$ and E.\,V.~Bunaev$^2$}

\def\auf{$^1$Institute for Information Transmission Problems RAS, Moscow 
Institute of Physics and Technology, Datadvance LLC, belyaev@iitp.ru\\[1pt]
$^2$Institute for Information Transmission Problems RAS, Moscow 
Institute of Physics and Technology, Datadvance LLC, burnaev@iitp.ru}


\def\leftkol{ENGLISH ABSTRACTS} % ENGLISH ABSTRACTS}

\def\rightkol{ENGLISH ABSTRACTS}

\titele{\tit}{\aut}{\auf}{\leftkol}{\rightkol}

%\vspace*{12pt}

\noindent
The problem of a multidimensional function approximation 
using a finite set of pairs ``point''\,--\,``function value at this point'' is considered.
As  a model for the function, an expansion in a dictionary containing nonlinear 
parametric functions has been used. Several subproblems should be solved when constructing an 
approximation based on such model: extraction of a validation sample, 
initialization of parameters of the functions from the dictionary, and tuning of 
these parameters. Efficient methods for solving these subproblems have been suggested. 
Efficiency of the proposed approach is demonstrated on some problems of 
engineering design.

%\vspace*{-5pt}
 \label{end\stat}


\KWN{nonlinear approximation; parametric dictionaries}


\newpage