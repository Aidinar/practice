\def\stat{shkotin}

\def\tit{ИССЛЕДОВАНИЕ ГРАФА КАТЕГОРИЙ АНГЛИЙСКОЙ ВЕРСИИ ВИКИПЕДИИ$^*$}

\def\titkol{Исследование графа категорий английской версии Википедии}

\def\autkol{А.\,В.~Шкотин}

\def\aut{А.\,В.~Шкотин$^1$}

\titel{\tit}{\aut}{\autkol}{\titkol}

{\renewcommand{\thefootnote}{\fnsymbol{footnote}}\footnotetext[1] {Статья 
рекомендована к публикации в журнале Программным комитетом конференции 
<<Электронные библиотеки: перспективные методы и технологии, электронные 
коллекции>> (RCDL-2012).}}

\renewcommand{\thefootnote}{\arabic{footnote}}
\footnotetext[1]{Государственный геологический музей им.\ В.\,И.~Вернадского 
Российской академии наук, отдел ГИС, ashkotin@acm.org}

\vspace*{6pt}


\Abst{Википедия является выдающимся проектом по накоплению знаний как общего 
пользования, так и различных областей специализации. Проверка качества этих знаний, 
особенно автоматическая, чрезвычайно важна. В работе представлены результаты изучения 
строения английской версии ГКВ (орграфа категориальных статей Википедии). Являясь по 
своей идее системой тем, он поддерживает систематизацию знаний, и представляет интерес, 
из чего эта систематизация состоит и как она устроена. Показано, что в графе есть 
неприемлемые логические нарушения, и обсуждаются организационные и технические 
методы их устранения.}

\vspace*{4pt}

\KW{Википедия; орграф; связные компоненты; логический анализ}

\vspace*{4pt}


\vskip 14pt plus 9pt minus 6pt

      \thispagestyle{headings}

      \begin{multicols}{2}

            \label{st\stat}
            
\section{Введение}

Орграф категориальных статей Википедии~[1] есть подграф графа, в котором статьи 
Википедии приписаны категориальным статьям. Выделение ГКВ из этого полного 
графа есть первая техническая задача. Важно, что далее изучается строение ГКВ 
на некоторый момент времени и в нем есть незавершенная, <<строящаяся>> часть. 
Поэтому выводы надо делать с осторожностью. Естественно ввести термин <<точка 
роста>>, когда натыкаешься в ГКВ на часть, которая еще не завершена. Дамп 
полного графа получен из ИСП РАН в виде двух текстовых файлов: файла 
отображения номера страницы Википедии в номер категориальной страницы, что 
приписывает страницу категории, и файла, в котором номеру страницы Википедии 
приписано ее наименование. Математически ГКВ есть орграф, каждый узел которого 
взаимно однозначно соответствует категориальной странице и помечен ее номером. 
Стрелка (дуга) из узла~$N_1$ в узел~$N_2$ идет тогда и только тогда, когда 
страница с номером~$N_1$ есть подкатегория страницы с номером~$N_2$. Всего 
таких стрелок 1\,221\,133.
   
   Множество узлов ГКВ (593\,796), как и любого произвольного графа, распадается 
на два подмножества: изолированные узлы (26\,272) и узлы, связанные стрелками (567\,24). 
Изолированная категория~--- это, скорее всего, <<точка роста>>: на момент снятия 
дампа она уже есть, но стрелок еще нет.
   
   Далее анализируется только <<граф стрелок>>, т.\,е.\ все характеристики даны без учета 
изолированных узлов. Состав изолированных узлов можно по\-смот\-реть в отчете~[2] 
   (далее~--- отчет) в таблице, указанной во введении. Состав и характеристики узлов со 
стрелками можно посмотреть в таблице, указанной там же, равно как и граф стрелок. 
Важный вопрос~--- количество связных компонент графа, так как в дальнейшем их строение 
можно изучать отдельно. Таких компонент оказалось~1987. Изолированные узлы при этом 
учитываются отдельно. Алгоритм разбиения описан в отчете~[3]. Впрочем, проще 
воспользоваться пакетом программ, например Pajek~[4], который умеет разбивать узлы 
графа на слабо связные компоненты.
   
   Первые 10 самых больших компонент указаны в табл.~1,
где $C_n$~--- уникальный номер компоненты, присвоенный при разбиении. Конечно, в 
случае с Википедией малые компоненты~--- это точки роста. Петель ($N_1\hm\rightarrow 
N_1$) в графе нет. 
   
  \begin{center}  %tabl1
% \vspace*{6pt}
\parbox{100pt}{{\tablename~1}\ \ \small{Объем связных компонент}}

\vspace*{6pt}

{\small   
%\tabcolsep=12pt
\begin{tabular}{|r|c|}
   \hline
\multicolumn{1}{|c|}{$C_n$}&\multicolumn{1}{c|}{Количество}\\
\hline
1&561\,636\hphantom{9999}\\
21\,727&210\hphantom{9}\\
14\,332&36\\
\hphantom{9}2\,863&29\\
20\,842&27\\
\hphantom{9}6\,680&20\\
19\,212&19\\
20\,868&19\\
13\,325&17\\
13\,287&16\\
\hline
\end{tabular}
}
\end{center}

%\vspace*{15pt}

\addtocounter{table}{1}

   
   



\begin{figure*}[b] %fig1
\vspace*{-6pt}
\vspace*{9pt}
 \begin{center}
 \mbox{%
 \epsfxsize=124.212mm
 \epsfbox{shk-ch-L.eps}
 }
 \end{center}
 \vspace*{-6pt}
 \Caption{Состав используемых символов }
 \end{figure*}
   
   Источников (узлов, в которые нет входящих стрелок)~--- 345\,597. Это категории нижнего 
уровня. Стоков (узлов, из которых нет исходящих стрелок)~--- 11\,767. Это 
категории верхнего уровня дампа и, скорее всего, точки роста. Промежуточных 
узлов, соответственно, 210\,160.
   
   Максимальное количество исходящих из одного узла стрелок~--- 85. Речь идет о 
промежуточном узле №\,690451 с заголовком Category:World War~II, т.\,е.\ эта категория 
приписана 85~надкатегориям. Максимальное количество входящих стрелок (12\,625) имеет 
промежуточный узел №\,692309 с про\-яс\-ня\-ющим заголовком Category:Albums by artist.

\vspace*{-6pt}

\section{Анализ заголовков}

\vspace*{-2pt}

 Заголовки всех узлов категорий (включая изолированные) можно посмотреть в отчете в 
таблице, указанной в разделе <<Анализ заголовков>>. Таблица содержит 584\,606~узлов. 
Следовательно, 9190~узлов ГКВ не имеют заголовков. Они ждут своего исследователя. 
Анализ текстов заголовков, даже безотносительно к их подчинению,~--- отдельная 
увлекательная задача. Но начать надо с использованного состава букв.

\vspace*{-6pt}

\subsection{Алфавит}

   Рассмотрим состав букв (characters), упо\-треб\-лен\-ных при именовании категориальных 
статей. Текстовый файл (UTF-8), содержащий состав алфавита можно посмотреть в 
прикреплении {\sf cat2title.abc0.txt}  к отчету. Как разделитель букв используется знак 
<<$\vert$>>. %Вот этот алфавит:
Этот алфавит представлен на рис.~1.
В приложении {\sf id2title.abc.txt} к отчету можно найти впечатляющее разнообразие 
букв заголовков всех статей Википедии.

\vspace*{-6pt}

\subsection{Термины в~заголовках}

   Это может стать отдельным важным исследованием. Например, количество заголовков, в 
которых встречается слово album,~--- 17\,591. 

\vspace*{-6pt}

\section{Стоки}

\vspace*{-2pt}

   В приложении~1 к отчету можно посмотреть начало таблицы стоков с самым большим 
количеством входящих стрелок.
   
   В приложении~2 к отчету можно посмотреть путь-рекордсмен, предоставленный Антоном 
Коршуновым из ИСП РАН.
   
   Самый длинный путь~--- 294~вершины. Его начальная категория~--- №\,5760285 
Category:Anastacia songs, а конечная~--- №\,691484 Category:Music. 

\section{Строение орграфа категориальных статей Википедии в~целом}

   В работе~[5, с.~9] указывается, что в ГКВ есть циклы. По идее, циклы~--- это аномалии на 
графе подчинения категорий, они долж\-ны занимать малую его часть. Назовем для краткости 
объединение орциклов графа и орпутей между циклами~--- \textit{ядром}, а дополнительную 
часть графа~--- \textit{мантией}. Стрелки же между мантией и ядром назовем 
\textit{связующими}. Таким образом, в целом граф состоит из ядра, мантии и связующих 
стрелок, часть из которых идет из ядра в мантию, а часть~--- из мантии в ядро. Чтобы 
выделить ядро, был применен следующий алгоритм: 
   \begin{enumerate}[(1)]
   \item находим в графе источники и стоки и удаляем их; 
   \item если в графе не осталось узлов, то стоп (ядра нет);
   \item если есть источники или стоки, то идти на~(1). Если нет, то
   стоп (в графе осталось только ядро).
   \end{enumerate}
   
\section{Ядро орграфа категориальных статей Википедии}
   
   В строении ядра важно, что пути между циклами, сами не входящие в циклы, составляют 
самостоятельную интересную часть ядра. 

\subsection{Состав ядра}

   Количество стрелок в ядре~--- 38\,538. Узлов же~--- 13\,545. Граф ядра опубликован в 
таблице, указанной в разделе <<Состав ядра>> отчета. Далее было традиционно выполнено 
<<расщепление>> ядра на связные компоненты. Оказалось, что имеются одна большая 
компонента~--- 13\,507~узлов и~еще 19~пар узлов. Характеристики узлов ядра, включая 
разбиение на связные компоненты, можно посмотреть в таблице, указанной в разделе 
<<Состав ядра>> отчета.
   
   Рассмотрим компоненту №\,764 ядра. Это пример пары, которая является даже связной 
компонентой не только ядра, а самого ГКВ. В компоненте два узла:
\begin{enumerate}[(1)]   
\item
№\,28736601, Category:Wikipedia sockpuppets of ShantanuSingh198 ;
\item
№\,28736686, Category:Suspected Wikipedia sockpuppets of ShantanuSingh19.
\end{enumerate}
   
      В Википедии они также ссылаются только друг на друга.
      
      \vspace*{12pt}
   
   \textbf{Анализ.} Что бы ни обозначало <<Wikipedia sockpuppets of ShantanuSingh198>>, 
очевидно, что нечто, под него подпадающее (как под понятие), не может быть 
одновременно лишь <<подозреваемым>> на подпадание. Равно как и наоборот, 
т.\,е.\ логически эти две категории не пересекаются и~обе стрелки должны быть 
удалены. Отношения же между ними, например, на OWL~2~\cite{6-shk} должно было 
бы быть таким:\\

\noindent
 {\sf 
DisjointClasses(wcg:Wikipedia\_sockpuppets\_of\_}

\noindent{\sf ShantanuSingh198}, 

\noindent
{\sf wcg:Suspected\_Wikipedia\_sockpuppets\_of\_}

\noindent
{\sf ShantanuSingh198)}\\

   При этом правильнее ссылаться в обеих статьях друг на друга через тег Википедии <<See 
also>>.

\subsection{Сильно связные компоненты ядра}

   В ядре представляет интерес зацикливание отношения подкатегория--над\-ка\-те\-го\-рия. Тут 
есть два подхода:
   \begin{enumerate}[(1)]
   \item общий~--- применить алгоритм поиска сильно связных компонент (ССК);
   \item частный~--- найти так называемые <<линзы>>~--- два узла, ссылающихся друг на 
друга (как под\-ка\-те\-го\-рия--надкатегория). 
   \end{enumerate}
   
   Второй путь вполне приемлем для ГКВ, так как, по идее, в нем вообще не должно быть 
циклов. Впрочем, как в отношении линз, так и в отношении циклов большей длины следует 
заметить, что они математически утверждают эквивалентность соответствующих терминов, 
т.\,е.\ синонимию, что в принципе возможно. Но конкретно в Википедии возможна реализация
через {\sf redirect}. Интуитивно же в большинстве случаев обнаруживается ошибка, 
т.\,е.\ ка\-кие-то стрелки цикла ошибочны.
   
   Чтобы получить состав сильно связных компонент ядра, была использована программа 
Pajek~[4]. Заметим, что петель в ГКВ нет, а поэтому узлы ядра, не попавшие в ССК,~--- это 
узлы на путях между циклами (см.\ выше).
   
   Сильно связных компонент оказалось 457. Узлов, не входящих в ССК, так сказать, 
   связующих ядра,~--- 7646. 
Есть одна гигантская по сравнению с остальными ССК~--- в ней 3967 узлов.
   
   В отчете в разделе <<Сильно связные компоненты ядра>> приведена таблица самых 
больших ССК. Рассмотрим для примера компоненту №\,41, у которой всего 9~узлов 
(табл.~2, рис.~2).

   
   Если номер накладывается на стрелку, то под ним наконечника (треугольничка) нет. Это 
важно, так как Pajek рисует <<линзы>> ($\mathrm{У}_1\rightarrow \mathrm{У}_2\rightarrow 
\mathrm{У}_1$) как одну стрелку с наконечниками на обоих окончани-\linebreak\vspace*{-12pt}

\vspace*{6pt}


  \begin{center}  %tabl2
% \vspace*{6pt}
{\tablename~2}\ \ \small{Заголовки узлов ССК №\,41}

\vspace*{6pt}

{\small   
\tabcolsep=3.5pt
\begin{tabular}{|c|l|}
   \hline
Номер узла&\multicolumn{1}{c|}{Заголовок}\\
\hline
717\,227&Category:Orthodox rabbis\\
717\,302&Category:Talmud rabbis\\
799\,461&Category:Mishnah\\
799\,587&Category:Talmud\\
6\,110\,893\hphantom{\,9}&Category:Talmudists\\
8\,398\,752\hphantom{\,9}&Category:Talmud people\\
11\,334\,178\hphantom{\,99}&Category:Rabbinic literature\\
15\,249\,105\hphantom{\,99}&Category:Talmud concepts and terminology\\
26\,795\,615\hphantom{\,99}&Category:Chazal\\
\hline
\end{tabular}
}
\end{center}

\vspace*{3pt}

\addtocounter{table}{1}

 \begin{center}  %fig2
 \mbox{%
 \epsfxsize=67.079mm
 \epsfbox{shk-1.eps}
 }
 
 \vspace*{6pt}
{\figurename~2}\ \ \small{Рисунок графа ССК №\,41}
\end{center}


\addtocounter{figure}{1}

\noindent
ях. В~данной ССК 
(см.\ рис.~2) линза всего одна~--- слева внизу вертикально.
   
 
\subsection{Линзы}

   Линза~--- это два узла таких, что $\mathrm{У}_1\hm\rightarrow \mathrm{У}_2$ и 
$\mathrm{У}_2 \hm\rightarrow \mathrm{У}_1$. Она может быть отдельной ССК, а может 
входить в ССК как часть.
   
   В ядре оказалось 1269~линз. Из них 1260 имеют заголовки для обоих узлов. Их можно 
посмотреть в таблице, указанной в разделе <<Линзы>> отчета.

\section{Мантия~--- ациклическая часть орграфа категориальных статей Википедии}

   Чтобы получить мантию, удалим из ГКВ ядро. При этом оказывается, что часть 
источников и стоков стали изолированными. В~первом случае все исходящие из них стрелки 
попали в ядро, во втором~--- все входящие в них стрелки шли из ядра. Изолировавшихся 
источников~--- 14\,421, сто-\linebreak ков~---~60.
{ %\looseness=-1

}
   
   Кроме того, в мантии появляются ложные вершины (пики). Это те ее узлы, которые стали 
стоками после удаления ядра, а вообще-то имели исходящие стрелки, которые все попадали 
в ядро. Таких вершин 18\,157. Причем максимальная высота~--- 28. Для сравнения, стоков 
ГКВ, получивших уровень, т.\,е.\ неизолированных~--- 11\,707, максимальная высота~---~24.
   
   Ложная вершина-рекордсмен (высоты~28) имеет №\,15\,715\,670, а заголовок~--- 
Category:Creation myths. 
   
   \medskip
   
   \noindent
   \textbf{Замечание.} Конечно, ГКВ можно представить и в виде 
   <<гал\-сту\-ка-ба\-боч\-ки>>, как в работе~\cite{7-shk}, где орграф был использован для 
представления схемы связей между транснациональными корпорациями. Но в данном случае 
нагляднее сравнение с горами~--- вверх к более обширным темам, горами, в которых есть 
ядро из 20~связных компонент, одна из которых большая, а 19~--- линзы.
   
   Число узлов на уровнях показано в табл.~3 и оправдывает сравнение с горами.
   В  строке NULL указано количество изолированных узлов мантии, 
а в строке~0~--- количество узлов в ядре.

\subsection*{Связующие стрелки}

   Между мантией и ядром есть стрелки~--- связу\-ющие. Стрелок из ядра в мантию~--- 591. 
Стрелок из мантии в ядро~--- 210\,514.
   
     \begin{center}  %tabl2
% \vspace*{6pt}
\parbox{120pt}{{\tablename~3}\ \ \small{Распределение узлов по уровням}}

\vspace*{6pt}

{\small   
\tabcolsep=3.5pt
   \begin{tabular}{|c|c|}
   \hline
   Уровень & Количество узлов\\
   \hline
   NULL & 14\,481\hphantom{99}\\
   28&\hphantom{99}1\\
   27 &\hphantom{99}2\\
   26&\hphantom{99}3\\
   25&\hphantom{99}3\\
   24&\hphantom{99}5\\
   23&\hphantom{99}7\\
   22&\hphantom{9}12\\
   21&\hphantom{9}16\\
   20&\hphantom{9}20\\
   19&\hphantom{9}30\\
   18&\hphantom{9}50\\
   17&\hphantom{9}57\\
   16&\hphantom{9}71\\
   15&100\\
   14&149\\
   13&226\\
   12&425\\
   11&697\\
   10&1\,187\hphantom{9}\\
      \hphantom{9}9&1\,915\hphantom{9}\\
      \hphantom{9}8&3\,103\hphantom{9}\\
      \hphantom{9}7&4\,858\hphantom{9}\\
      \hphantom{9}6&7\,754\hphantom{9}\\
      \hphantom{9}5&13\,019\hphantom{99}\\
      \hphantom{9}4&23\,302\hphantom{99}\\
      \hphantom{9}3&45\,323\hphantom{99}\\
      \hphantom{9}2&105\,958\hphantom{999}\\
      \hphantom{9}1& 331\,205\hphantom{999}\\
   \hphantom{9}0&13\,545\hphantom{99}\\
   \hline
   \end{tabular}
}
\end{center}

%\vspace*{15pt}

\addtocounter{table}{1}
   


\section{Обсуждение}

   \noindent
   \begin{enumerate}[1.]
   \item Влияние ядра на строение мантии оказывается существенным. Так, ложный сток 
имеет высоту~28 при максимальной высоте настоящего стока 24. Такое может случиться, 
только если ядро находится на <<вершине>> мантии. Кроме того, появилось 29~ложных 
источников~--- в них шли стрелки только из ядра. Количество ложных стоков~--- 18\,157. 
При этом ядро состоит всего лишь из одной большой связной компоненты и 19~линз.
   \item Хотя <<физически>> отдельное изучении мантии оправдано, для совокупного 
строения графа лучше не удалять ядро из ГКВ, а свернуть его ССК в <<тяжелые>> узлы, 
пометив их количеством узлов в ССК. Такой тяжелый узел наследует все внешние ССК 
стрелки, а образовавшиеся петли внутренних стрелок стоит удалить. Тогда получится 
ациклический граф, так как ССК не могут образовывать цикл. Уровни такого графа с 
указанием распределения по ним тяжелых узлов дадут более реалистичную картину ГКВ. 
При этом есть все основания полагать, что по крайней мере одна из тяжелых вершин будет 
стоком.
   \item В растущем графе, которым по преимуществу является ГКВ, <<точки роста>> 
(изолированные узлы; все связные компоненты, кроме главной; стоки вне главной 
связной компоненты) не представляют 
особого интереса. Поэтому стоит сразу выделить главную связную компоненту 
и изучать только ее.
\end{enumerate}

\section{Другие способы исследования}

   Можно напрямую изучать {\sf http://dbpedia.org}\linebreak через точку входа для SPARQL: 
      {\sf 
http://dbpedia.org/ sparql}. Привязка к категории идет через свойство {\sf 
http://purl.org/dc/terms/subject}.
   
   Вот пример запроса, который начинает выдавать полный граф связи страниц и категорий:\\[-8pt]
   
   
{\sf select ?x ?z where \{?x dcterms:subject ?z\}}\\[-8pt]
   
   Надо только поставить timeout, например, 1000.
   
   Запрос, выдающий отношение <<{\sf x} is a sub-category of {\sf z}>> (см.\ с.~5 
<<Categories>>~\cite{5-shk}):\\[-8pt]

{\sf select ?x ?z where \{?x skos:broader ?z\}}\\[-8pt]
   
   А вот запрос, выдающий <<линзы>>:\\[-8pt]

{\sf select ?x ?z where \{?x skos:broader ?z. ?z }

{\sf skos:broader ?x.\}}\\[-8pt]
   
   Вот узлы первой:\\[-8pt] 

{\sf http://dbpedia.org/resource/Category:Political\_}

{\sf philosophers;}


   
{\sf http://dbpedia.org/resource/Category:Political\_}

{\sf theorists.}\\[-8pt]
   
   Она действительно есть в Википедии.
   
   А всего запрос выдает 2000 линз, что, наверное, не предел.
   
\section{Заключение}

   Естественно считать, что ГКВ должен быть ацик\-ли\-че\-ским графом. Таким образом, 
исследование показало, что аномалии значительны.
   
   Можно создать средства, которые, обнаруживая аномалию, например линзу, будут 
размещать на соответствующих страницах в Discussion уведомление о возможном 
логическом противоречии.
   
   Основных вопросов два.
   \begin{enumerate}[1.]
   \item Как к такому подходу отнесутся авторы страниц категорий? Это можно проверить 
экспериментально.
   \item Как к логическим противоречиям относятся идеологи Википедии? Те, кто задает 
правила классификации? Судя по всему, индифферентно.
   \end{enumerate}
   
   Общая рекомендация: многие отношения между категориями, попавшие в 
   <<sub-category of>>, следует перенести в <<See also>>.
   
   Оценить предстоящую работу можно так: для начала надо разобраться с 1269~линзами. 
Они значительно убавят размер ССК.
   
   Только если это нужно википедистам, можно было бы продолжить работу в следующих 
на\-прав\-ле\-ни\-ях:
   \begin{itemize}
   \item исследовать длинные пути;
   \item попытаться представить архитектуру графа в целом (например, применить 
   трехмерную визуализацию);
   \item проанализировать состав и логику связи заголовков (особенно ССК).
   \end{itemize}
   
   Особняком стоит задача получить и проанализировать русский ГКВ. В~проекте DBpedia 
можно получить дамп русской версии, надо только перекодировать буквы с rdf-кодов 
(например, $\backslash$u0432) в UTF-8.
   
{\small\frenchspacing
{%\baselineskip=10.8pt
\addcontentsline{toc}{section}{Литература}
\begin{thebibliography}{9}

\bibitem{1-shk}
\Au{Korshunov A., Turdakov D., Jeong~J., Lee~M., Moon~Ch.} A~category-driven approach to 
deriving domain specific subset of Wikipedia~// SYRCoDIS'11: 7th Spring Researchers 
Colloquium on Databases and Information Systems Proceedings, 2011. P.~43--53.
\bibitem{2-shk}
\Au{Шкотин А.} Исследование графа категорий английской версии Wikipedia. 
Сообщение о результатах первого этапа. 2011. {\sf http://sites.google.com/site/ alex0shkotin/grafy/wikipedia-category-graph}.
\bibitem{3-shk}
\Au{Шкотин А.} Разбиение графа на связные компоненты: Алгоритм и программа. 2011. 
{\sf http://sites.google. com/site/alex0shkotin/grafy/svaznye-komponenty}.
\bibitem{4-shk}
\Au{Batagelj V., Mrvar A.} Pajek: Program for analysis and visualization of large networks: 
Reference manual.~--- Ljubljana: University,  2012.
\bibitem{5-shk}
\Au{Bizer C., Lehmann J., Kobilarov~G., Auer~S., Becker~C., Cyganiak~R., Hellmann~S.}
DBpedia~--- a~crystallization point for the Web of Data~// J.~Web Semantics, 2009. Vol.~7. 
No.\,3. P.~154--165.
\bibitem{6-shk}
OWL 2 Web Ontology Language: Structural specification and functional-style 
syntax: W3C recommendation~/ Eds. B.~Motik, P.\,F.~Patel-Schneider, 
B.~Parsia. 2009. {\sf http:// www.w3.org/TR/owl2-syntax}. 

\label{end\stat}

\bibitem{7-shk}
\Au{Vitali S., Glattfelder J.\,B., Battiston~S.} The network of global corporate control~// 
Cornell University Library (submitted on July~28, 2011 (v1), last revised  Sep.~19, 2011 (this 
version, v2)). {\sf http://arxiv.org/abs/1107.5728}.
\end{thebibliography}
} }

\end{multicols}