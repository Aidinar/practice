\vspace*{-60pt} %{ %small
{ %\baselineskip=9.1pt
\section*{Правила подготовки рукописей  для публикации в журнале
<<Информатика и её применения>>}

\thispagestyle{empty}

\noindent
\begin{enumerate}[1.]
\item В журнале печатаются статьи, содержащие результаты, ранее не опубликованные и 
не предназначенные к одновременной публикации в других изданиях. 
 
Публикация не должна нарушать закон об авторских правах. 
 
Направляя рукопись в редакцию, авторы сохраняют все права собственников данной 
рукописи и при этом передают учредителям и редколлегии неисключительные права на 
издание статьи на русском языке (или на языке статьи, если он отличен от русского) и на 
ее распространение в России и за рубежом. Авторы должны представить в редакцию 
письмо в следующей форме: 


{\bfseries\textit{Соглашение о передаче права на публикацию:}}

<<\textit{Мы, нижеподписавшиеся, авторы рукописи} <<\ldots>>, 
\textit{передаем учредителям и редколлегии журнала <<Информатика и её 
применения>> неисключительное право опубликовать данную рукопись 
статьи на русском языке как в печатной, так и в электронной версиях 
журнала. Мы подтверждаем, что данная публикация не нарушает 
авторского права других лиц или организаций. }
 
\textit{Подписи авторов: (ф. и. о., дата, адрес)}>>.  
 
Это соглашение может быть представлено в бумажном виде или в виде 
отсканированной копии (с подписями авторов).  
 
Редколлегия вправе запросить у авторов экспертное заключение о 
возможности публикации пред\-став\-лен\-ной статьи в открытой печати. 

\item К статье прилагаются данные автора (авторов) (см.\ п.~8). При наличии нескольких 
авторов указывается фамилия автора, ответственного за переписку с редакцией. 

\item Редакция журнала осуществляет экспертизу присланных статей в соответствии с 
принятой в журнале процедурой рецензирования.

Возвращение рукописи на доработку не означает ее принятия к печати.  

Доработанный вариант с ответом на замечания рецензента необходимо прислать в 
редакцию. 

\item Решение редколлегии о публикации статьи или ее отклонении сообщается авторам.  
Редколлегия может также направить авторам текст рецензии на их статью. Дискуссия по 
поводу отклоненных статей не ведется. 

\item Редактура статей высылается авторам для просмотра. Замечания к редактуре должны 
быть присланы авторами в кратчайшие сроки. 

\item Рукопись предоставляется в электронном виде в форматах MS WORD (.doc или 
.docx) или \LaTeX\ (.tex), дополнительно~--- в формате .pdf, на дискете, лазерном диске 
или электронной почтой. Предоставление бумажной рукописи необязательно.

\item При подготовке рукописи в MS Word рекомендуется использовать следующие 
настройки.

Параметры страницы:  
формат~--- А4; ориентация~--- книжная; поля (см): внутри~--- 2,5, снаружи~--- 1,5, 
сверху~--- 2, снизу~--- 2, от края до нижнего колонтитула~--- 1,3.  

Основной текст: стиль~--- <<Обычный>>, шрифт ~--- Times New Roman, размер~--- 
14~пунктов, абзацный отступ~--- 0,5~см, 1,5~интервала, выравнивание~--- по ширине.  
 
Рекомендуемый объем рукописи~--- не свыше 20 страниц указанного формата.  

Сокращения слов, помимо стандартных, не допускаются. Допускается минимальное 
количество аббревиатур. 

Все страницы рукописи нумеруются. 

Шаблоны примеров оформления, представлены в Интернете: 

{\sf http://www.ipiran.ru/journal/template.doc}.

\item Статья должна содержать следующую информацию на {\bfseries\textit{русском и 
английском языках:}} 
\begin{itemize}
\item название статьи; 
\item Ф.И.О.\ авторов, на английском можно только имя и фамилию;
\item место работы, с указанием города и страны и электронного адреса каждого 
автора; 

\pagebreak  

\thispagestyle{empty}


\vspace*{-36pt}


\item сведения об авторах, в соответствии с форматом, образцы которого 
представлены на страницах: 

{\sf http://www.ipiran.ru/journal/issues/2013\_07\_01/authors.asp}  и

 {\sf 
http://www.ipiran.ru/journal/issues/2013\_07\_01\_eng/authors.asp};

\item аннотация (не менее 100 слов на каждом из языков). Аннотация~--- это 
краткое резюме работы, которое может публиковаться отдельно. Она является 
основным источником информации в информационных системах и базах данных; 
Английская аннотация должна быть оригинальной, может не быть дословным 
переводом русского текста и должна быть написана хорошим английским языком.
\item ключевые слова, желательно из принятых в мировой научно-технической 
литературе тематических тезаурусов. Предложения не могут быть ключевыми 
словами.
\end{itemize}

\item Литература. По включенным в список литературы работам на русском языке 
информация в списке представляется как в кириллице, так и с использованием латинской 
транслитерации, а по работам, написанным латиницей,~--- на языке оригинала. 

Ссылки на литературу в тексте статьи нумеруются (в квадратных скобках) и 
располагаются в списке литературы в порядке упоминания.  

В списке литературы не должно быть позиций, на которые нет ссылки в тексте статьи.  

\item Присланные в редакцию материалы авторам не возвращаются. 

\item При отправке файлов по электронной почте просим придерживаться следующих 
правил: 
\begin{itemize}
\item указывать в поле subject (тема) название журнала и фамилию автора; 
\item использовать attach (присоединение); 
\item в состав электронной версии статьи должны входить: файл, содержащий 
текст статьи, и файл(ы), содержащий(е) иллюстрации. 
\end{itemize}
\item  Журнал <<Информатика и её применения>> является некоммерческим изданием. 
Плата за публикацию не взимается, гонорар авторам не выплачивается. 
\end{enumerate}


\thispagestyle{empty}

\noindent
\textbf{Адрес редакции:} Москва 119333, ул.~Вавилова, д.~44, корп.~2, ИПИ РАН

\noindent
\hphantom{\textbf{Адрес редакции: }}Тел.: +7 (499) 135-86-92\ \ Факс:  +7 (495) 930-45-05  

\noindent
\hphantom{\textbf{Адрес редакции: }}e-mail:   {\sf rust@ipiran.ru} (Сейфуль-Мулюков Рустем Бадриевич)
}
%}


%\vfill
%\begin{center}


%Технический редактор Л. Кокушкина\\
%%Выпускающий редактор Т. Торжкова\\
%Художественный редактор М. Седакова\\
%Сдано в набор 03.04.13. Подписано в печать 14.06.13. Формат 60 х 84 / 8\\
%Бумага офсетная. Печать цифровая. Усл.-печ.\ л.\ ??,?. Уч.-изд.\ л.\ ??,?. Тираж 100 экз.\\
%\ \\
%Заказ №\,\\
%\ \\
%Издательство <<ТОРУС ПРЕСС>>, Москва 121614, ул. Крылатская, 29-1-43\\
%torus@torus-press.ru; http://www.torus-press.ru\\
%\ \\
%Отпечатано в Академиздатцентре <<Наука>> РАН с готовых файлов\\
%Москва 121099, Шубинский пер., д.~6\\
%\end{center}