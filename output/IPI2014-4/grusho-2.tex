\def\stat{grusho-2}



\def\tit{ВКЛЮЧЕНИЕ НОВЫХ ЗАПРЕТОВ В~СЛУЧАЙНЫЕ
ПОСЛЕДОВАТЕЛЬНОСТИ$^*$}



\def\titkol{Включение новых запретов в~случайные
последовательности}

\def\aut{А.\,А. Грушо$^1$, Н.\,А. Грушо$^2$, Е.\,Е. Тимонина$^3$}

\def\autkol{А.\,А. Грушо, Н.\,А. Грушо, Е.\,Е. Тимонина}

\titel{\tit}{\aut}{\autkol}{\titkol}

{\renewcommand{\thefootnote}{\fnsymbol{footnote}} \footnotetext[1]
{Работа частично поддержана РФФИ (проект 13-01-00215).}}


\renewcommand{\thefootnote}{\arabic{footnote}}
\footnotetext[1]{Институт проблем информатики Российской академии наук;
факультет вычислительной математики
и~кибернетики Московского государственного университета им.\ М.\,В. Ломоносова,
grusho@yandex.ru}
\footnotetext[2]{Институт проблем информатики Российской академии наук, info@itake.ru}
\footnotetext[3]{Институт проблем информатики Российской академии наук, eltimon@yandex.ru}





\Abst{Рассматривается задача порождения одних вероятностных мер на пространстве
бесконечных последовательностей над конечными алфавитами с~$\sigma$-ал\-геб\-рой,
порожденной цилиндрическими множествами, из других вероятностных мер на этом
пространстве. При этом новая вероятностная мера устроена так, чтобы определенным
образом сокращать множество допустимых траекторий случайных последовательностей.
Недопустимость траекторий определяется в~терминах спецификаций наименьших запретов.}

\KW{случайные последовательности; запреты вероятностных мер; порождение
вероятностных мер; статистические задачи на случайных последовательностях}

\DOI{10.14357/19922264140406}
%\vspace*{5pt}


\vskip 12pt plus 9pt minus 6pt

\thispagestyle{headings}

\begin{multicols}{2}

\label{st\stat}

\section{Введение}

    В системах контроля информационной безопасности и~при поиске
скрытых каналов часто возникает задача поиска аномалий в~наблюдаемой
случайной последовательности. Одним из основных инструментов в~решении
таких задач является математическая статистика. Однако при ненулевой ошибке
принятия решения об аномалии наблюдение процесса порождает большое
число ложных тревог~[1], которые затрудняют или делают невозможным анализ
причин аномалий. Для случайных процессов с~дискретным временем
и~конечным множеством состояний найден подход для проверки
последовательности гипотез о~распределении проекций вероятностных мер на
пространстве бесконечных последовательностей, при котором вероятность
ложной тревоги всегда равна нулю. При этом с~ростом размерностей
вероятность правильного решения о наличии аномалий стремится к~единице.
Этот подход основан на запретах вероятностных мер~[2, 3].

    В предыдущих исследованиях~[2, 3] было введено определение запрета
для вероятностной меры на конечном пространстве. Запрет означает
последовательность, имеющую нулевую вероятность на конечном пространстве.
Было показано, что введение понятия запрета полезно для решения указанных
выше задач. Запреты позволяют определять критические множества
статистических критериев прос\-тей\-шим для вычисления способом~\cite{2-grs}.
Были доказаны необходимые и~достаточные условия существования
состоятельной последовательности критериев, в~которых все критические
множества статистических критериев определяются с~по\-мощью
запретов~\cite{3-grs}.

    Статистические критерии, основанные на запретах, обладают важными
особенностями. Вероятности ложных решений равны нулю. Так как
критические множества статистических критериев определяются только
запретами, то эти критерии одинаковы для вероятностных мер, име\-ющих
одинаковые множества запретов (т.\,е.\ это робастные критерии). В~случае
проверки гипотез состоятельность определяется тем условием, что вероятность
появления хотя бы одного запрета при альтернативе стремится к единице.

    Рассмотрим задачу внесения запретов в~случайную последовательность.

Приведем следующий пример~[4]. Для поиска скрытых каналов в~случайную
последовательность вносится запрет. При этом организаторы скрытого канала
не знают о~внесенном запрете, поэтому, когда в~наблюдаемой
последовательности наблюдается запрет, контролер определяет, что
функционирует скрытый канал.

    В работе~[4] было показано, что другие статистические методы, отличные
от методов, основанных на запретах, очень чувствительны к~изменениям
вероятностных моделей, поэтому статистические\linebreak процедуры по\-ис\-ка скрытых
каналов необходимо строить на робастных процедурах, учитывающих
гетерогенный характер последовательностей передаваемых сообщений.
Условиям робастности удовле\-тво\-ря\-ют статистические методы, основанные на
запретах.

    Пусть исходная случайная последовательность определена с~помощью
вероятностной меры~$P$ на пространстве бесконечных последовательностей
с~$\sigma$-ал\-геб\-рой, порожденной цилиндрическими множествами.
Проекции этой меры удовлетворяют условию согласованности, т.\,е.\
распределение очередного ($n\hm+1$)-го знака в~последовательности
выражается через условное распределение появления этого знака при условии
появления предыду\-щих~$n$~знаков последовательности и~безусловного
распределения этих $n$ знаков. Внесение запрета не должно нарушать условие
согласованности, иначе мера на пространстве бесконечных
последовательностей может определяться некорректно. В~рассматриваемом
методе условных распределений для внесения запрета необходимо оперировать
вероятностями всех начальных участков случайной последовательности.

    В работе рассматривается другой способ внесения запрета в~случайную
последовательность. Этот метод основан на корректном определении некоторых
множеств функций и~учитывает спе\-ци\-фи\-кации наименьших запретов, которые
должны присутствовать в~новой построенной мере. В~предлагаемом методе
сохраняются условия со\-гла\-со\-ван\-ности, что позволяет использовать теорему
Каратеодори~[5] об однозначном продолжении меры.

  Статья имеет следующую структуру. В~разд.~2 приводятся определения
и~предыдущие результаты. Раздел~3 определяет условия для случая, когда
вероятностная мера генерируется согласно заданной спецификации
наименьших запретов. В~разд.~4 приводится пример использования
доказанных условий. В~разд.~5 кратко анализируются условия существования
состоятельной последовательности критериев для мер, построенных в~примере.

\vspace*{-10pt}

\section{Основные определения и~предыдущие результаты}

\vspace*{-2pt}

    Пусть $X_i$, $i=1,2,\ldots,$~--- конечные множества, $\prod\limits_{i=1}^n
X_i$~--- декартово произведение~$X_i$, $i\hm=1,2,\ldots, n$, $X^\infty$~---
множество всех последовательностей, где $i$-й элемент принадлежит~$X_i$.
Пусть $\mathcal{A}$~--- это $\sigma$-ал\-геб\-ра на~$X^\infty$, порожденная
цилиндрическими множествами; $\mathcal{A}$~также является борелевской
$\sigma$-ал\-геб\-рой в~тихоновском произведении~$X^\infty$, где $X_i$ имеют
дискретную топологию~[6, 7].

    На ($X^\infty, \mathcal{A}$) определена вероятностная мера~$P$.
Предположим, что $P_n$ является проекцией меры~$P$ на пространство
конечномерных векторов, по\-рож\-ден\-ных первыми $n$ координатами
последовательностей из~$X^\infty$. Обозначим $X_n^\infty \hm=
\prod\limits_{i=n+1}^\infty X_i$. Ясно, что для каждого $B_n\hm\subseteq
\prod\limits_{i=1}^n X_i$
    $$
    P_n(B_n) =P\left( B_n\times X_n^\infty\right)\,.
    $$

    Пусть $D_n$~--- носитель меры~$P_n$:
    $$
    D_n=\left\{ \vec{x}_n\in \prod\limits_{i=1}^n X_i\,\ \ P_n\left(\vec{x}_n\right)
>0\right\}\,.
    $$

    Обозначим $\Delta_n=D_n\times X^\infty$. Последовательность~$\Delta_n$,
$n\hm= 1,2,\ldots$, невозрастающая и
    $$
    \Delta_P=\lim\limits_{n\to\infty} \Delta_n = \mathop{\bigcap}\limits_{n=1}^\infty \Delta_n\,.
    $$

    Множество $\Delta_P$ замкнуто в~топологии тихоновского произведения
и~является носителем меры~$P$. Если $\overline{\omega}^{(k)} \hm\in
\prod\limits_{i=1}^k X_i$, то $\tilde{\omega}^{(k-1)}$ получается
из~$\overline{\omega}^{(k)}$ отбрасыванием последней координаты.

    \smallskip

    \noindent
    \textbf{Определение 1.}\ Запретом в~мере~$P_n$ называется вектор
$\overline{\omega}^{(k)} \hm\in \prod\limits_{i=1}^k X_i$, $k\hm\leq n$ , такой
что
    $$
    P_n \left( \overline{\omega}^{(k)} \times \prod\limits_{i=k+1}^n X_i\right)
=0\,.
    $$

    Если $P_{k-1}\left( \tilde{\omega}^{(k-1)}\right)\hm>0$, то
$\overline{\omega}^{(k)}$ называется наименьшим запретом.

    Если $\overline{\omega}^{(k)}$ является запретом в~мере~$P_n$, тогда для
любых $k\hm\leq s\hm\leq n$ и~любых
последовательностей~$\overline{\omega}^{(s)}$, начинающихся
с~последовательности~$\overline{\omega}^{(k)}$, имеем:
    $$
    P_s\left( \overline{\omega}^{(s)}\right)=0\,.
    $$

    Действительно, если $P_k\left(\overline{\omega}^{(k)}\right)\hm=0$, то
\begin{align*}
    P\left( \overline{\omega}^{(k)} \times X_k^\infty \right)&=0\,;
\\
P\left( \overline{\omega}^{(k)} \times \prod\limits_{i=k+1}^s X_i\times X_s^\infty
     \right)&=0\,.
    \end{align*}
Из этого следует, что
\begin{multline*}
P_s\left(\overline{\omega}^{(s)}\right) = P\left( \overline{\omega}^{(s)}\times
X_s^\infty\right) \leq{}\\
{}\leq P\left( \overline{\omega}^{(k)} \times
\prod\limits_{i=k+1}^s X_i\times X_s^\infty\right)=0\,.
\end{multline*}

    Если существует  $\overline{\omega}^{(n)} \in \prod\limits_{i=1}^n X_i$
    такое, что $P_n\left(\overline{\omega}^{(n)}\right)\hm=0$, то существует
наименьший запрет.

    Пусть для всех $n$ носители мер~$P_n$ совпадают с~$\prod\limits_{i=1}^n
X_i$. Тогда носитель меры~$P$ совпадает с~$X^\infty$. Предположим, что
задана спецификация исходных наименьших запретов
$\nu^\prime\hm= \{\nu^\prime_n, n\hm= 1,2,\ldots\}$, где
$\nu^\prime_n$~--- число наименьших запретов длины~$n$ в~мере~$P$. Пусть задана новая
спецификация, определяемая дополнительными ограничениями
$\nu\hm= \{\nu_n, \nu_n\hm\geq \nu_n^\prime, n\hm= 1,2,\ldots\}$.
Задача состоит в~том, чтобы, используя меру~$P$
и~спецификации~$\nu$   и~$\nu^\prime$, построить вероятностную меру~$Q$ на пространстве
$(X^\infty, \mathcal{A})$, у~которой множество наименьших запретов обладает
спецификацией~$\nu$. Для построения~$Q$ сначала построим согласованную
систему вероятностных мер~$Q_n$, $n\hm= 1,2,\ldots$, на пространствах,
определяемых проекциями~$X^\infty$ на первые $n$ координат. Эти меры
определяют аддитивную меру на алгебре цилиндрических множеств, которая по
теореме Каратеодори будет однозначно определять меру~$Q$ на $(X^\infty,
\mathcal{A})$. Далее будем обозначать через~$D^\prime_n$, $n\hm=1, 2,\ldots$,
$D_n^\prime\hm\subseteq \prod\limits_{i=1}^n X_i$, носители мер~$P_n$, через
$d_n^\prime$~--- мощности этих носителей, а через~$D_n$, $n\hm=1, 2,\ldots$,
$D_n\hm\subseteq \prod\limits_{i=1}^n X_i$, носители мер~$Q_n$ и~через
$d_n$~--- мощности этих носителей.

    В работе~\cite{3-grs} доказано, что числа~$d_n$, $n\hm= 1, 2,\ldots$,
однозначно связаны со спецификацией~$\nu$ сле\-ду\-ющи\-ми соотношениями:
    \begin{equation}
    \nu_1 \prod\limits_{i=2}^n m_i +\cdots + \nu_{n-1} m_n +\nu_n +d_n
=\prod\limits_{i=1}^n m_i\,.
    \label{e1-grs}
    \end{equation}
для всех $n = 1, 2,\ldots$

  Таким образом, необходимо построить согласованное семейство
вероятностных мер $\{Q_n\}$, мощности носителей которых однозначно
определены соотношениями~(1).

\vspace*{-4pt}

\section{Порождение вероятностных мер с~заданной спецификацией
наименьших запретов}

\vspace*{-2pt}

    Пусть $\{D_n$, $D_n\hm\subseteq D_n^\prime$, $D_n\hm\subseteq
\prod\limits_{i=1}^n X_i$, $n\hm=1,2,\ldots\}$~--- некоторое семейство множеств,
удовле\-тво\-ря\-ющих~(1), $\overline{x}_n$~--- произвольный элемент
$\prod\limits_{i=1}^n X_i$. Для любого~$n$, $n\hm=1, 2,\ldots$, определим
функцию
    $$
    g_{n+1}:\ D_{n+1} \to \prod_{i=1}^n X_i
    $$
следующим образом. Для любых $\overline{x}_{n+1}\hm\in D_{n+1}$,
$\overline{x}_{n+1}\hm= \overline{x}_nx$, где
$\overline{x}_n \hm\in \prod\limits_{i=1}^n X_i$, $x\hm\in X_{n+1}$, определяем
$$
g_{n+1}\left( \overline{x}_{n+1}\right) =\overline{x}_n\,.
$$

    Кроме~(1) на множества~$\{D_n\}$ наложим сле\-ду\-ющие два ограничения,
связанных с~функциями~$g_n$. Для любого~$n$, $n\hm= 1, 2,\ldots$, и~любых
$\overline{x}_{n+1}\hm\in D_{n+1}$
    \begin{gather}
    g_{n+1}\left(\overline{x}_{n+1}\right) \in D_n\,;\label{e2-grs}\\
    g_{n+1}:\ D_{n+1}\stackrel{\mbox{\tiny на}}{\to} D_n\,.\label{e3-grs}
    \end{gather}

    По аналогии с~функциями~$g_n$ определяются функции~$h_n$ для
последовательности множеств~$D_n^\prime$ так, что для любого~$n$,
$n\hm= 1, 2,\ldots$, и~любых~$\overline{x}_n x\hm\in D_{n+1}^\prime$
    $$
    h_{n+1}\left( \overline{x}_n x\right) =\overline{x}_n\,.
    $$

    \smallskip

    \noindent
    \textbf{Лемма.}\ \textit{Пусть~$P$~--- вероятностная мера на $(X^\infty,
\mathcal{A})$, $\nu^\prime$~--- спецификация наименьших запретов,
$\{D_n^\prime\}$~--- носители мер~$P_n$, $d_n^\prime \hm= \vert
D_n^\prime\vert$,  $n\hm=1,2,\ldots$ Тогда для~$\nu^\prime$, $\{D_n^\prime\}$,
$\{d_n^\prime\}$, $\{h_n\}$ выполняются соотношения}~(1)--(3).

    \smallskip

    \noindent
    Д\,о\,к\,а\,з\,а\,т\,е\,л\,ь\,с\,т\,в\,о\,.\ \  Выполнение~(1) доказано в~[3]. По
определению $\forall\ \overline{x}_{n+1}\hm= \overline{x}_nx\hm\in
D_{n+1}^\prime$
    $$
    h_{n+1}\left(\overline{x}_{n+1}\right) =\overline{x}_n\,.
    $$
По определению $D_{n+1}^\prime$ имеем:
 $$
 P_n\left( \overline{x}_{n+1}\right) >0\,;
 $$

 \vspace*{-14pt}

 \noindent
\begin{multline*}
0<P_{n+1}\left(\overline{x}_{n+1}\right) = P\left( \overline{x}_{n+1}\times
X^\infty_{n+1}\right) = {}\\[2pt]
{}=P\left(\overline{x}_n \times \{x\} \times
X^\infty_{n+1}\right) \leq {}\\[2pt]
{}\leq P\left(\overline{x}_n \times X_n \times X^\infty_{n+1}\right) =
P\left(\overline{x}_n \times X_n^\infty \right) =P_n\left(\overline{x}_n\right).\hspace*{-5.73422pt}
\end{multline*}
Отсюда следует, что $\overline{x}_n\hm\in D_n^\prime$. Это доказывает
соотношение~(2).

    Если $\overline{x}_n \in D_n^\prime$ и~$\forall x\hm\in X_{n+1}$
выполняется
    $$
    P_{n+1}\left(\overline{x}_nx\right)=0\,,
    $$
то $P_{n+1}\left(\overline{x}_n, X_{n+1}\right)\hm=0$, что противоречит
согласованности мер~$P_n$ и~$P_{n+1}$:
$$
P_{n+1}\left(\overline{x}_n, X_{n+1}\right) =P_n\left(\overline{x}_n\right)
$$
и~предположению, что $P_n\left(\overline{x}_n\right)\hm>0$. Значит,
существует такое~$x$, что $P_{n+1}\left(\overline{x}_n x\right) \hm>0$, т.\,е.\
$\overline{x}_n x\hm\in D^\prime_{n+1}$. Это доказывает соотношение~(3).

    Возьмем произвольную последовательность\linebreak сюрьективных функций
    $f_n:\ D_n^\prime \hm\to D_n$, $n\hm=1, 2,\ldots$

    Каждая такая функция порождает на~$D_n$ и,~следовательно, на
$\prod\limits_{i=1}^n X_i$ вероятностную меру~$Q_n$ с~носителем~$D_n$.

    Тогда для всех~$n$ функции~$g_{n+1}$ и~меры~$Q_{n+1}$ порождают
на $\prod\limits_{i=1}^n X_i$ вероятностные меры~$Q_n^\prime$
с~носителями~$D_n$ (это следует из сюрьективности~$f_{n+1}$ и~$g_{n+1}$).

    \smallskip

    \noindent
    \textbf{Теорема~1.}\ \textit{Пусть задано произвольное семейство
вероятных мер $\{Q_n\}$ с~носителями $\{D_n\}$ и~семейство функций
$\{g_n\}$, удовлетворяющих условиям~$(2)$ и~$(3)$. Семейство вероятностных
мер $\{Q_n\}$ является согласованным семейством тогда и~только тогда,
когда для всех~$n$ выполняются равенства} $Q_n\hm= Q_n^\prime$.

    \smallskip

    \noindent
    Д\,о\,к\,а\,з\,а\,т\,е\,л\,ь\,с\,т\,в\,о\,.\ \ Докажем достаточность. Для
согласованности мер достаточно, чтобы для любых $\overline{x}\hm\in
\prod\limits_{i=1}^n X_i$

\noindent
    $$
    Q_n\left(\overline{x}_n\right) = Q_{n+1}\left(\overline{x}_n, X_{n+1}\right)\,.
    $$

    Из конечности вероятностных схем
    $$
    Q_{n+1}\left(\overline{x}_n, X_{n+1}\right) = \sum\limits_{x\in X_{n+1}}
    \!\!\!\!\!Q_{n+1} \left (\overline{x}_n x\right)\,.
    $$

    По определению
    \begin{multline*}
    Q_n^\prime\left(\overline{x}_n\right) =Q_{n+1}\left( g_{n+1}^{-1}
    \left(\overline{x}_n\right)\right) = %{}\\
%{}=
    \!\!\!\!\!\!\sum\limits_{(\overline{x}_nx)\in D_{n+1}}\!\!\!\!\!\!\!\!\!\!Q_{n+1} \left(\overline{x}_n
x\right) ={}\\
    {}=    \!\!\!\!\!\!\sum\limits_{x\in X_{n+1}}\!\!\!\!\! Q_{n+1} \left(\overline{x}_n x\right) =
Q_{n+1} \left(\overline{x}_n, X_{n+1}\right)\,.
    \end{multline*}

    По условию теоремы
    $$
    Q_n^\prime \left (\overline{x}_n\right) =Q_n \left(\overline{x}_n\right)
    $$
    для любых
    $\overline{x}_n \hm\in \prod\limits_{i=1}^n X_i$. Отсюда следует, что для любых
$\overline{x}_n\hm\in \prod\limits_{i=1}^n X_i$
    $$
Q_n\left(\overline{x}_n\right) = Q_{n+1}\left(\overline{x}_n, X_{n+1}\right)\,.
$$

    Достаточность доказана. Докажем необходимость. Если $\{Q_n\}$~---
согласованное семейство вероятностных мер, то для любых
$\overline{x}_n\hm\in \prod\limits_{i=1}^n X_i$
    $$
    Q_{n+1}\left(\overline{x}_n, X_{n+1}\right) =Q_n \left(\overline{x}_n\right)\,.
    $$
Кроме того, для любых $\overline{x}_n \hm\in \prod\limits_{i=1}^n X_i$
\begin{multline*}
Q_n^\prime \left(\overline{x}_n\right) =  Q_{n+1} \left( g^{-
1}_{n+1}\left(\overline{x}_n\right)\right) ={}\\
{}=Q_{n+1}\left(\overline{x}_n,
X_{n+1}\right) =Q_n \left(\overline{x}_n\right)\,.
\end{multline*}

    Теорема доказана.
    \smallskip

    Семейство функций $\{f_n\}$ и~вероятностная мера~$P$ порождают
семейство вероятностных мер $\{Q_n\}$ с~носителями $\{D_n\}$. Пусть
функции $\{g_n\}$ удовле-\linebreak\vspace*{-12pt}

\columnbreak

\noindent
творяют условия~(2) и~(3). Тогда справедливо
следующее утверждение.

    \smallskip

    \noindent
    \textbf{Следствие 1.}\ Семейство функций $\{f_n\}$ и~вероятностная
мера~$P$ порождают единственную вероятностную меру~$Q$ тогда и~только
тогда, когда для всех $n \hm=1, 2,\ldots$ выполняется равенство $Q_n\hm=
Q_n^\prime$.

    \smallskip

    \noindent
    \textbf{Теорема 2.}\ \textit{Для согласованности множества
вероятностных мер $\{Q_n\}$, порожденных функциями $\{f_n\}$
и~проекциями меры~$P$, достаточно, чтобы функции $\{g_n\}$
удовлетворяли соотношениям~$(2)$ и~$(3)$ и~для всех~$n$ были
коммутативны следующие диаграммы}:
    \begin{equation}
    \begin{array}{ccccc}
     & D_n^\prime & \stackrel{h_{n+1}}{\longleftarrow} & D_{n+1}^\prime & \\
    f_n & \!\!\downarrow & & \downarrow &\!\! f_{n+1}\\
    & D_n & \stackrel{g_{n+1}}{\longleftarrow} & D_{n+1} &
    \end{array}
    \label{e4-grs}
    \end{equation}

    \noindent
    Д\,о\,к\,а\,з\,а\,т\,е\,л\,ь\,с\,т\,в\,о\,.\ \ Каждая функция~$f_n$
и~мера~$P_n$ на~$D_n^\prime$ порождают на~$D_n$ вероятностное
распределение~$Q_n$. В~силу согласованности проекций меры~$P$ каждая
функция~$h_{n+1}$ порождает меру~$P_n$ из меры~$P_{n+1}$. Поэтому
можно считать, что мера~$Q_n$ порождена из меры~$P_{n+1}$ с~помощью
композиции отображений $(f_n * h_{n+1})$.

  В свою очередь, функция~$f_{n+1}$ и~мера~$P_{n+1}$ по\-рож\-да\-ют
распределение вероятностей $Q_{n+1}$ на~$D_{n+1}$. Эта мера
и~функция~$g_{n+1}$ порождают меру~$Q_n^\prime$ на~$D_n$, т.\,е.\
мера~$Q_n^\prime$ на~$D_n$ порождена из меры~$P_{n+1}$ с~помощью
композиции функций $(g_{n+1}*f_{n+1})$. По условию~(4) функции
$(f_n*h_{n+1})$) и~$(g_{n+1}*f_{n+1})$ совпадают. Следовательно, эти функции
и~мера~$P_{n+1}$ порождают на~$D_n$ одну и~ту же меру, т.\,е.\ $Q_n\hm=
Q_n^\prime$. Отсюда и~из теоремы~1 следует согласованность семейства
вероятностных мер $\{Q_n\}$. Теорема доказана.

\section{Пример порождения вероятностных мер с~заданной~спецификацией
наименьших~запретов}

    Пусть $\nu\hm= \{ \nu_i\hm=1, i\hm=1,2,\ldots\}$, $X_m\hm=
\{0,1,\ldots , m-1\}$, $m\hm>2$, и~$P$~--- равномерная мера на~$X^\infty$.
Пусть элементы $\prod\limits_{i=1}^n X_i$ лексикографически упорядочены.

    Приведем пример построения меры~$Q$ со спецификацией наименьших
запретов~$\nu$ с~помощью подхода, описанного в~разд.~3.

    В каждом множестве $B_n\hm\subseteq \prod\limits_{i=1}^n X_i$ есть
наименьший вектор\ $\overline{x}_n$ с~точки зрения лексикографическо-\linebreak го
порядка. Требуемую меру будем строить индуктивно. В~$D_1$ наименьшим
запретом будем считать~0. Функция~$f_1$ отображает~$X_1$ в~$X_1\backslash
\{0\}$. Предположим, что определены~$D_n$ и~$f_n$. Определим~$D_{n+1}$.
Пусть $\overline{x}_n^0$~--- наименьший элемент в~$D_n$. В~множестве
$D_n\times X_{n+1}$ определим наименьший запрет~--- $\left (\overline{x}_n^0
0\right)$. Положим
    $$
    D_{n+1}= \left( D_n\times X_{n+1}\right) \backslash \{ (\overline{x}_n^0
0)\}\,.
    $$

    Построим сюрьективную функцию
    $$
    f_{n+1}:\ \prod\limits_{i=1}^{n+1} X_i \stackrel{\mbox{\tiny на}}{\rightarrow}
D_{n+1}\,.
    $$

    Для любых $\left(\overline{x}_n x\right) \hm\in \prod\limits_{i=1}^{n+1}
X_i$, кроме тех, у~которых $f_n\left(\overline{x}_n\right)\hm= \overline{x}_n^0$
и~$x\hm=0$, положим
    $$
    f_{n+1}\left(\overline{x}_n x\right) =\left( f_n\left(\overline{x}_n\right),x\right)
\in D_n\times X_{n+1}\,.
    $$

    Обозначим $\overline{y}_n^{(i)}$,  $i \hm= 1,\ldots ,k$, все элементы
множества $f_n^{-1}\left(\overline{x}_n^0\right)$. Определим
    $$
    f_{n+1}\left( \overline{y}_n^{(i)},0\right) = \left(\overline{x}_n^0,1\right) \in
D_n\times X_{n+1}\,.
    $$

    Отметим, что $\left(\overline{x}_n^0,1\right) $~--- наименьший элемент
в~$D_{n+1}$. По определению~$f_n$~--- это отображение $\prod\limits_{i=1}^n
X_i$ на~$D_n$. Поэтому $f_{n+1}$ отображает~$X^{n+1}$ на $D_{n+1}\hm=
(D_n\times X_{n+1}) \backslash \{ \overline{x}_n^0,0\}$. По
построению~$D_{n+1}$ из~$D_n$ и~из~(1) следует, что $\nu_{n+1}\hm=1$
и~этот элемент равен $\left( \overline{x}_n^0,0\right)$.

    Докажем коммутативность диаграмм~(4). По построению~$f_n$
и~$h_{n+1}$ для любых $x\hm\in X_{n+1}$
    $$
    h_{n+1}\left( \overline{y}_n^{(i)},x\right) =\overline{y}_n^{(i)}\,,
    $$
поэтому
$$
\left( f_n * h_{n+1}\right) \left( \overline{y}_n^{(i)},x\right) =\overline{x}_n^0\,.
$$
При $\overline{x}_n \not= \overline{y}_n^{(i)}$, $i \hm= 1,\ldots , k$,
$$
\left( f_n * h_{n+1}\right) \left( \overline{x}_n,x\right)
=f_n\left( \overline{x}_n\right)\,.
$$
Далее
\begin{gather*}
f_{n+1}\left( \overline{y}_n^{(i)},0\right) =\left( \overline{x}_n^0,1\right)\in
D_{n+1}\,,\enskip i=1,\ldots ,k\,;\\
\left( f_{n+1}* g_{n+1}\right) \left( \overline{y}_n^{(i)},0\right) =f_n
\left( \overline{y}_n^{(i)}\right) =\overline{x}_n^0\,.
\end{gather*}
Для элементов $\left( \overline{y}_n^{(i)}, x\right)$, $x\not=0$, $i \hm= 1,\ldots ,
k$,
$$
f_{n+1} \left( \overline{y}_n^{(i)},x\right) =\left( f_n
\left( \overline{y}_n^{(i)}\right),x\right) = \left(\overline{x}_n^0,x\right) \in
D_{n+1}\,.
$$
Поэтому при $x\not=0$
$$
\left( f_{n+1}*g_{n+1}\right) \left( \overline{y}_n^{(i)},x\right)
=\overline{x}_n^0\,.
$$
При  $\overline{x}_n\not= \overline{y}_n^{(i)}$, $i \hm= 1,\ldots , k$, по
определению~$f_{n+1}$ имеем, что для любых $x\hm\in X_{n+1}$
$$
\left( f_{n+1}\right) \left(\overline{x}_n,x\right)
=\left( f_n\left(\overline{x}_n\right)x\right) \in D_{n+1}\,.
$$
Тогда по построению
$$
\left( g_{n+1}*f_{n+1}\right) =\left( f_n* h_{n+1}\right)\,.
$$

    Коммутативность диаграмм~(4) доказана.

    \smallskip

  Отсюда следует существование меры~$Q$ на $(X^\infty, \mathcal{A})$ со
спецификацией наименьших запретов $\nu\hm= \{ \nu_i \hm=1,\ i\hm=
1,2,\ldots\}$.

\section{Применение к анализу состоятельности}

    Из соотношений~(1) получаем соотношения:
    \begin{equation}
    d_{n+1}-m_{n+1}d_n +\nu_{n+1} =0\,,\ n=1,2,\ldots
    \label{e5-grs}
    \end{equation}

    Пусть $P$~--- равномерная мера на $(X^\infty, \mathcal{A})$. Тогда
отношение $d_n\big/\prod\limits_{i=1}^n m_i$ есть вероятность
множества~$D_n$ в~мере~$P_n$.

    Для спецификации $\nu\hm=\{\nu_i\hm=1,\ i\hm= 1,2,\ldots\}$
из~(\ref{e5-grs}) для некоторого положительного~$\varepsilon$ получаем
следующие соотношения при $m_n\hm\geq3$, $n\hm=1, 2,\ldots$:
    $$
    \fr{d_n}{\prod\limits_{i=1}^n m_i}>\varepsilon>0\,.
    $$

    При $n\to\infty$ предел этой вероятности равен вероятности~$P$
носителя~$\Delta_Q$ меры~$Q$
    $$
    P\left( \Delta_Q\right) \geq \varepsilon >0\,.
    $$

  Из необходимых и~достаточных условий~[2] существования состоятельных
последовательностей критериев, определяемых запретами, для проверки
гипотез $H_{0,n}:Q_n$ против $H_{1,n}:P_n$ следует, что таких
последовательностей критериев нет.

\section{Заключение}

Получены условия корректного построения дополни\-тельных
ограничений на случайную последовательность с~помощью спецификации
наименьших запретов. Корректное построение стохастических моделей
позволяет использовать в~анализе\linebreak статистических данных хорошо
разработанный аппарат теории случайных последовательностей и~процессов.

{\small\frenchspacing
 {%\baselineskip=10.8pt
 \addcontentsline{toc}{section}{References}
 \begin{thebibliography}{9}
\bibitem{1-grs}
    \Au{Axelson S.} The base-rate fallacy and its implications for the difficulty of
intrusion detection~// 6th ACM Conference on Computer and Communications
Security Proceedings.~--- New York: ASM, 1999. P.~1--7.
\bibitem{2-grs}
    \Au{Грушо А., Тимонина Е.} Запреты в~дискретных
    ве\-ро\-ят\-но\-ст\-но-ста\-ти\-сти\-че\-ских задачах~// Дискретная
математика, 2011. Т.~23. Вып.~2. С.~53--58.
\bibitem{3-grs}
    \Au{Grusho A., Grusho N., Timonina~E.} Consistent sequences of tests defined
by bans~// Springer proceedings in mathematics \& statistics, optimization theory,
decision making, and operation research applications.~---
    New York\,--\,Heidelberg\,--\,Dordrecht\,--\,London: Springer, 2013.
    \mbox{P.~281--291.}
\bibitem{4-grs}
    \Au{Grusho A., Grusho N., Timonina~E.} Problems of modeling in the analysis
of covert channels~// Computer network security, 2010. Lecture notes in computer
science ser. Vol.~6258. P.~118--124. doi: 10.1007/978-3-642-14706-7\_9.
\bibitem{5-grs}
    \Au{Неве Ж.} Математические основы теории вероятностей~/
    Пер. с англ.~--- М.: Мир,
1969. 309~с.
(\Au{Neveu~J.} {Bases mathematiques du calcul des probabilites}.
Paris: Masson, 1964. 203~p.)
\bibitem{6-grs}
    \Au{Бурбаки Н.} Общая топология. Основные структуры~/ Пер.
     с~франц.~--- М.: Наука, 1968. 272~с. (\Au{Bourbaki~N.} Topologie
G$\acute{\mbox{e}}$n$\acute{\mbox{e}}$rale. Chapitre~1: Structures
topologiques. Chapitre~2: Structures uniformes.~--- Paris: Hermann, 1940. 129~p.)
\bibitem{7-grs}
    \Au{Прохоров Ю.\,В., Розанов Ю.\,А.} Теория вероятностей.~--- М.: Наука,
1993. 496~с.



 \end{thebibliography}

 }
 }

\end{multicols}

\vspace*{-10pt}

\hfill{\small\textit{Поступила в редакцию 31.10.14}}

%\newpage

\vspace*{8pt}

\hrule

\vspace*{2pt}

\hrule

\vspace*{-2pt}

\def\tit{SWITCHING ON OF NEW BANS IN RANDOM SEQUENCES}

\def\titkol{Switching on of new bans in random sequences}

\def\aut{A.\,A.~Grusho$^{1,2}$, N.\,A.~Grusho$^1$, and E.\,E.~Timonina$^1$}

\def\autkol{A.\,A.~Grusho, N.\,A.~Grusho, and E.\,E.~Timonina}

\titel{\tit}{\aut}{\autkol}{\titkol}

\vspace*{-9pt}


\noindent
$^1$Institute of Informatics Problems, Russian Academy of Sciences,
44-2~Vavilov Str., Moscow 119333, Russian\linebreak
$\hphantom{^1}$Federation

\noindent
$^2$Faculty of Computational Mathematics and Cybernetics,
M.\,V.~Lomonosov Moscow State University,\linebreak
$\hphantom{^1}$1-52~Leninskiye Gory, GSP-1, Moscow 119991, Russian Federation


\def\leftfootline{\small{\textbf{\thepage}
\hfill INFORMATIKA I EE PRIMENENIYA~--- INFORMATICS AND
APPLICATIONS\ \ \ 2014\ \ \ volume~8\ \ \ issue\ 4}
}%
 \def\rightfootline{\small{INFORMATIKA I EE PRIMENENIYA~---
INFORMATICS AND APPLICATIONS\ \ \ 2014\ \ \ volume~8\ \ \ issue\ 4
\hfill \textbf{\thepage}}}

\vspace*{3pt}

\Abste{The problem of generating one probability measure on space
of the infinite sequences on finite alphabets with  $\sigma$-algebra generated
by cylindrical sets out of another probability measure on this space is
considered. A~new probability measure is arranged
to reduce the set of admissible trajectories of random sequences definitely. Inadmissibility
of trajectories is defined in terms of specifications of the smallest bans.
If a~specification of the smallest bans is given, then the powers of support
of projections of the new measure can be determined. It gives conditions to
construct several sets of functions. These functions and projections of the
initial measure define a set of measures on finite spaces which define
 the only probability measure on the space of infinite sequences.}


\KWE{random sequences; bans of probability measures;
generation of probability measures; statistical problems on random sequences}


\DOI{10.14357/19922264140406}

\vspace*{-20pt}

\Ack

\vspace*{-2pt}

\noindent
The paper was partially supported by the
Russian Foundation for Basic Research  (project 13-01-00215).

%\vspace*{3pt}

  \begin{multicols}{2}

\renewcommand{\bibname}{\protect\rmfamily References}
%\renewcommand{\bibname}{\large\protect\rm References}



{\small\frenchspacing
 {%\baselineskip=10.8pt
 \addcontentsline{toc}{section}{References}
 \begin{thebibliography}{9}
\bibitem{1-grs-1}
\Aue{Axelsson, S.} 1999. The base-rate fallacy and its implications for the
difficulty of intrusion detection.
\textit{6th Conference on Computer and Communications Security Proceedings}.
New York: ASM. 1--7.
\bibitem{2-grs-1}
\Aue{Grusho, A., and E. Timonina.} 2011.
Prohibitions in discrete probabilistic statistical problems.
\textit{Discrete Mathematics Applications} 21(3):275--281.
\bibitem{3-grs-1}
\Aue{Grusho, A., N. Grusho, and E.~Timonina}. 2013. Consistent
sequences of tests defined by bans.
\textit{Springer proceedings in mathematics \& statistics, optimization theory,
decision making, and operation research applications}.
New York\,--\,Heidelberg\,--\,Dordrecht\,--\,London: Springer. 281--291.
\bibitem{4-grs-1}
\Aue{Grusho, A., N.~Grusho, and E.~Timonina}.
2010. Problems of modeling in the analysis of covert channels.
\textit{Computer network security}.
Lecture notes in computer science ser. 6258:118--124.
doi: 10.1007/978-3-642-14706-7\_9.
\bibitem{5-grs-1}
\Aue{Neveu, J.} 1964. \textit{Bases mathematiques du calcul des probabilites}.
Paris: Masson. 203~p.
\bibitem{6-grs-1}
\Aue{Bourbaki, N.} 1940. \textit{Topologie G$\acute{\mbox{e}}$n$\acute{\mbox{e}}$rale}.
Chapitre~1: Structures topologiques. Chapitre~2: Structures uniformes.
Paris: Hermann, 1940. 129~p.
\bibitem{7-grs-1}
\Aue{Prokhorov, Yu.\,V., and Yu.\,A.~Rozanov}. 1993. \textit{Teoriya veroyatnostey}
[Theory of probabilities].  Moscow: Nauka. 496~p.

\end{thebibliography}

 }
 }

\end{multicols}

\vspace*{-10pt}

\hfill{\small\textit{Received October 31, 2014}}

\pagebreak

%\vspace*{-18pt}


    \Contr

\noindent
\textbf{Grusho Alexander A.} (b.\ 1946)~---
Doctor of Science in physics and mathematics,
Corresponding member of the Russian Academy
of Cryptography; leading scientist, Institute of Informatics Problems,
Russian Academy of
Sciences, 44-2 Vavilov Str., Moscow 119333, Russian Federation;
professor, Faculty of
Computational Mathematics and Cybernetics, M.\,V.~Lomonosov Moscow State University,
1-52~Leninskiye Gory, GSP-1, Moscow 119991, Russian Federation;
grusho@yandex.ru

\vspace*{3pt}

\noindent
\textbf{Grusho Nikolai A.}\ (b.\ 1982)~---
Candidate of Science (PhD) in physics and mathematics, senior scientist, Institute of Informatics
Problems, Russian Academy of Sciences, 44-2 Vavilov Str., Moscow 119333, Russian
Federation; info@itake.ru


\vspace*{3pt}

\noindent
\textbf{Timonina Elena E.}\ (b.\ 1952)~---
Doctor of Science in technology, professor, leading scientist, Institute of Informatics Problems,
Russian Academy of Sciences, 44-2 Vavilov Str., Moscow 119333, Russian Federation;
eltimon@yandex.ru

\label{end\stat}

\renewcommand{\bibname}{\protect\rm Литература}