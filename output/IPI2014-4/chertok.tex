\newcommand{\Esf}{{\sf E}}
\newcommand{\Psf}{{\sf P}}



\def\stat{chertok}



\def\tit{О ФОРМАЛИЗАЦИИ ПОНЯТИЯ ТОКСИЧНОСТИ ПОТОКА ЗАЯВОК НА ФИНАНСОВЫХ РЫНКАХ$^*$}

\def\titkol{О формализации понятия токсичности потока заявок на финансовых рынках}

\def\aut{А.\,В.~Черток$^1$}

\def\autkol{А.\,В.~Черток}

\titel{\tit}{\aut}{\autkol}{\titkol}

{\renewcommand{\thefootnote}{\fnsymbol{footnote}} \footnotetext[1]
{Работа выполнена при частичной поддержке РФФИ (проект 14-07-00041а).}}


\renewcommand{\thefootnote}{\arabic{footnote}}
\footnotetext[1]{Факультет вычислительной математики и кибернетики Московского
государственного университета им.\ М.\,В.~Ломоносова; Euphoria Group LLC;
a.v.chertok@gmail.com}

\vspace*{6pt}

\Abst{Рассматривается микроструктурная модель
потоков заявок на финансовых рынках. В~качестве интегрального
индикатора текущего состояния книги заявок используется дисбаланс
потока заявок. Для анализа свойств текущего состояния книги заявок
используется модель дисбаланса потока заявок, имеющая вид
двустороннего процесса риска, известного в~актуарной математике как
процесс риска со случайными премиями. Исследуется понятие
токсичности потока заявок на финансовых рынках. Понятие токсичности
потока заявок на
финансовых рынках формализуется с~по\-мощью вероятностей пересечения
процессом дисбаланса потоков заявок фиксированных уровней. Вводятся
понятия мгновенного профиля токсичности и байесовского
и~квантильного показателей токсичности. Эти показатели рассчитываются
для двух модельных типов потоков заявок, в~первом из которых заявки
имеют единичный объем, во втором~--- объем заявок является случайным
и~имеющим показательное распределение.}

%\vspace*{3pt}

\KW{финансовые рынки; книга заявок; поток заявок;
дисбаланс потока заявок; неблагоприятный отбор; токсичность; пуассоновский процесс;
обобщенный пуассоновский процесс;
двусторонний процесс риска; процесс риска со случайными премиями;
вероятность разорения}

\vspace*{3pt}

\DOI{10.14357/19922264140403}

\vspace*{6pt}


\vskip 14pt plus 9pt minus 6pt

\thispagestyle{headings}

\begin{multicols}{2}

\label{st\stat}

\section{Введение}

Активное развитие электронной торговли на финансовых рынках выявило
необходимость анализа биржевых высокочастотных данных для \mbox{более}
глубокого понимания рыночной микроструктуры, на которую оказали
огромное влияние компании, занимающиеся автоматизированным
высо\-кочастотным трейдингом (они формируют\linebreak до 70\%--80\% дневного
оборота на ведущих мировых площадках). Эти высокочастотные системы,\linebreak
как правило, являются мар\-кет-мей\-ке\-ра\-ми~--- поставщиками ликвидности
посредством размещения пассивных (лимитных) заявок на различных
уровнях электронной книги заявок. Поставщик ликвидности, выставивший
пассивную заявку, не имеет возможности влиять на время ее исполнения
(разуме\-ет\-ся, кроме как снять заявку). Мар\-кет-мей\-ке\-ры
зачастую не прогнозируют в~явном виде динамику рынка, а~используют
шумовую составляющую рыночных движений.

Степень эффективности
деятельности мар\-кет-мей\-ке\-ров
связана с~контролем риска оказаться с~большим количеством
купленных или проданных контрактов, что напрямую
зависит от их способности контролировать эффект неблагоприятного
отбора (adverse selection) в~отношении пассивных заявок.

Практики, как правило, описывают принцип неблагоприятного отбора как
<<естественную тенденцию слишком быстрого исполнения пассивных
заявок в~тех ситуациях, когда они должны исполняться медленно,
и~наоборот: исполняться слишком медленно в~тех ситуациях, когда они
должны исполниться быст\-ро>>~\cite{Jeria2008}. Эта интуитивная
формулировка согласуется с~ранними микроструктурными моделями
рынка~[2--4], в~которых информированные
трейдеры получают преимущество над
неинформированными участниками рынка. Поток заявок считается
токсичным, когда происходит эффект неблагоприятного отбора
мар\-кет-мей\-ке\-ров, поставляющих ликвидность.

В работе~\cite{Easley2012} предложена эмпирическая процедура оценки
токсичности потока заявок на основе анализа информации о~\textit{сделках}.
В~предлагаемой статье рассматривается более точный подход
к~измерению токсичности рынка, использующий всю доступную информацию
о~\textit{потоке заявок} (не только сами сделки, но также
и~по\-ста\-нов\-ки/сня\-тия заявок) на основе аналитической модели процесса
дисбаланса потока заявок, рассмотренной ранее в~работах~\cite{Korolev_2013, Chertok2014}.

\section{Модель потока заявок}

\subsection{Терминология}

На электронных рынках биржевая цена финансового инструмента в~ее
классическом понимании является результирующей, интегральной
характеристикой системы торгов, которая описывается динамикой так
называемой {\it книги заявок $($limit order book$)$}, представляющей
собой информацию о~всех актуальных на данный момент предложениях
о~покупке и продаже инструмента по различным ценам (рис.~1).

\begin{center}  %fig1
\vspace*{8pt}
\mbox{%
\epsfxsize=79.096mm
\epsfbox{che-1.eps}
}
\end{center}

\noindent
{{\figurename~1}\ \ \small{Книга заявок в~некоторый момент времени. Высота столбиков равна
  суммарному объему лимитных заявок на соответствующем ценовом уровне:
\textit{1}~--- покупки; \textit{2}~--- продажи}}


\vspace*{9pt}


\addtocounter{figure}{1}



Динамика книги заявок определяется тремя типами
заявок, которые участники рынка могут отправить на рынок:
\begin{enumerate}[(1)]
\item {\it лимитная} заявка обозначает желание купить (продать)
заданное количество акций по цене не выше (не ниже) заданной, при
этом такая заявка немедленно добавляется в~книгу заявок;
\item {\it рыночная} заявка обозначает желание купить или продать
заданное количество акций по лучшей цене, представленной в~книге
заявок, после чего немедленно происходит сделка;
\item заявка {\it на отмену} обозначает намерение отменить
существующую лимитную заявку, после чего она удаляется из книги заявок.
\end{enumerate}



Более формально, в~каждый момент времени информация о~первых $d \hm= 5$
уровнях книги заявок представляет собой массив
\begin{equation*}
\mathrm{book} = \left( b_1, a_1, v^b_1, v^b_2, \ldots, v^b_{10}, v^a_1, v^a_2,
\ldots, v^a_{10} \right)\,, %\label{e1-che}
\end{equation*}
где
$b_1$~--- лучшая цена на покупку (бид) на текущий момент
(кратная минимальному шагу цены~$\delta$);
$a_1$~--- лучшая цена на продажу (аск) на текущий момент
(кратная минимальному шагу цены~$\delta$);
$v^b_i \hm\geqslant 0$~--- суммарный объем заявок по цене~$b_i$ (при этом
автоматически $b_i \hm= b_1\hm - (i\hm - 1) \delta$);
$v^a_i \hm\geqslant 0$~--- суммарный объем заявок по цене~$a_i$ (при этом
автоматически $a_i \hm= a_1 \hm+ (i \hm- 1) \delta$).

Всегда выполняется условие $b_1 \hm< a_1$, так как иначе
соответствующие заявки должны быть сведены в~сделку, величина $p \hm=
(1/2)(b_1 \hm+ a_1)$ обычно называется {\it мидпрайсом}, а~величина
$s \hm= a_1 \hm- b_1$ называется {\it спредом}.

\subsection{Динамика книги заявок}

Потоки заявок моделируются с~использованием независимых
пуассоновских процессов~--- процессов восстановления
с~экспоненциальными распределе\-ниями интервалов между
восстановлениями (как это сделано, например,
в~работах~\cite{ContRamaStoikov2010b, ContLarrard2011}):
\begin{itemize}
\item лимитные заявки на покупку (продажу) приходят на ценовой уровень,
расположенный на\linebreak
 расстоянии~$i$ от лучшей котировки противоположного
типа, в~независимые моменты времени, имеющие экспоненциальное
распределение с~параметром $\lambda_i^{+} (\lambda_i^{-})$
(эмпирические исследова\-ния~\cite{Bouchaud2002, ZovkoFarmer2002} показывают, что степенный закон
$
\lambda_i^{\pm} = k/i^\alpha$
является хорошей аппроксимацией);
\item рыночные заявки на покупку (продажу) приходят в~независимые
моменты времени, име\-ющие экспоненциальное распределение с~параметром
$\mu^{+} (\mu^{-})$;
\item заявки на отмену лимитного ордера на покупку (продажу), находящегося
на дистанции~$i$ от лучшей котировки того же типа, приходят с~час\-то\-той $\theta_i^{+} (\theta_i^{-})$.
\end{itemize}

Рассмотрим два пуассоновских процесса $N^+(t)$ и $N^-(t)$
с~интенсивностями соответственно
\begin{align*}
\lambda^+ &= \mu^{+} + \sum\limits_{i} \lambda_i^{+} + \sum\limits_{i} \theta_i^{-}\,;
\\
\lambda^- &= \mu^{-} + \sum\limits_i \lambda_i^{-} + \sum\limits_i \theta_i^{+}\,.
\end{align*}
По своей сути процессы $N^+(t)$ и $N^-(t)$ соответствуют информации
о~числе заявок от покупателей и~продавцов соответственно, пришедших
к~моменту времени~$t$. Будем также считать, что объемы заявок от
покупателей и продавцов~--- независимые одинаково распределенные
величины $X_i^+$ и~$X_i^-$ с~функциями распределения $G(t)$ и~$F(t)$
соответственно и~не зависят от процессов $N^+(t)$ и~$N^-(t)$.

\begin{figure*}[b] %fig2
\vspace*{1pt}
 \begin{center}
 \mbox{%
 \epsfxsize=141.02mm
 \epsfbox{che-2.eps}
 }
 \end{center}
 \vspace*{-9pt}
  \Caption{Динамика лучшей цены покупки~(\textit{1}),
  лучшей цены продажи~(\textit{2}) и~процесса
  дисбаланса потока заявок $Q(t)$ в~течение 1~с~(\textit{3}) с~момента
  10:00:12,730 01.07.2014 (фьючерс на индекс РТС)
  \label{fig:ofibidask_mono}}
\end{figure*}

\section{Процесс дисбаланса потока заявок и~его связь с~ценой}

Понятие дисбаланса потока заявок введено в~работе~\cite{Cont2011},
окончательный вариант которой~\cite{Cont2014} опуб\-ли\-ко\-ван в~2014~г.\
В~работах~\cite{Korolev_2013, Chertok2014} независимо этот же
процесс исследовался под названием {\it процесс обобщенной цены}.

В работах~\cite{Korolev_2013, Chertok2014} в~качестве
математической модели эволюции процесса дисбаланса потока \mbox{заявок}
было предложено использовать двусторонний процесс риска~---
специальный обобщенный (compound) пуассоновский процесс. Следуя
этому подходу, зафиксируем малый интервал времени~$[0; T]$,\linebreak в~течение
которого параметры распределений, описывающих объемы заявок,
и~интенсивности потоков заявок одного типа остаются постоянными и~известными.
Для $t\hm\in[0,T]$ пусть $N^+(t)$ и~$N^-(t)$~---\linebreak количества
заявок, пришедших от покупателей и~продавцов соответственно
в~течение интервала времени $[0,t]$~--- независимые пуассоновские
процессы с~интенсивностями $\lambda^+ \hm> 0$ и~$\lambda^- \hm> 0$ ($\Esf
N^+(t) = \lambda^+ t$, $N^+(0) = 0$, $\Esf N^-(t) = \lambda^- t$,
$N^-(0) = 0$). Пусть $X^+_i$ и~$X^-_i$, $i\hm=1,2,\ldots, $~--- объемы
заявок, поступающих от покупателей и~продавцов соответственно -- две
независимые последовательности независимых и~одинаково в~каждой
последовательности распределенных случайных величин с~функциями
распределения $G(x)$ и~$F(x)$ соответственно, независимых от
пуассоновских процессов $N^+(t)$ и~$N^-(t)$. Положим
$$
Q^+(t)=\sum\limits_{i=1}^{N^+(t)}X_i^+\,;\enskip
Q^-(t)=\sum\limits_{j=1}^{N^-(t)}X_j^-
$$
и определим процесс {\it дисбаланса потока заявок} $Q(t)$ как
$$
Q(t)=Q^+(t) - Q^-(t)\,.
$$

Этот процесс является намного более чувствительным индикатором
(показателем) текущего состо\-яния книги заявок, поскольку интервалы
времени между последовательными изменениями со\-сто\-яний книги заявок
обычно так малы, что изменения цены (мидпрайса) по сравнению с~ними
являются редкими событиями. Поэтому процесс цены является намного
более грубым показателем, характеризующим книгу заявок, и~дает грубое
и~весьма ограниченное описание динамики рынка. Вместе с~тем процесс
дисбаланса потоков заявок учитывает не только текущие значения
наилучших цен покупки и~продажи, но и~влияние событий <<в глубине>>
книги заявок и~потому меняется существенно быстрее и~позволяет
интерполировать динамику рынка между изменениями цены, в~част\-ности
отслеживать ситуации, связанные с~токсичностью потоков заявок, т.\,е.\
чреватые необоснованными трендами в~поведении цены (рис.~\ref{fig:ofibidask_mono}).


В работе~\cite{Cont2014} с~помощью линейной модели
$$
\fr{S(t + \Delta) - S(t)}{\delta} = c \fr{Q(t, t + \Delta)}{D(t)} + \epsilon(t)
$$
было показано, что процесс дисбаланса потока заявок $Q(t)$ имеет
сильную связь с~высокочастотными изменениями цены финансового актива
$S(t)$, построенной по ценам сделок, где~$\delta$~--- минимальный шаг
цены (тик цены), $\epsilon(t)$~--- белый шум и~$D(t)$~--- мера глубины
книги заявок (количество заявок на лучшем биде/аске). Эмпирический
анализ высокочастотных данных для американских акций подтверждает
наличие линейной связи: коэффициент~$c$ оценивается между~0,1 и~1
и~оказывается статистически значим в~98\% случаев. Наличие
такого рода связи позволяет напрямую исследовать свойства процесса
дисбаланса потока заявок и~соотносить их со свойствами процесса цены
$S(t)$.

В работах~\cite{Korolev_2013, Chertok2014, Korolev_2014}
с~помощью предельных \mbox{теорем} для двусторонних
процессов риска были получены асимптотические аппроксимации
для процесса дисбаланса потока заявок $Q(t)$ и~его распределений.

\section{Профиль мгновенной токсичности потока заявок}

Как уже было сказано выше, поток заявок считается токсичным, когда
он оказывается неблагоприятным для мар\-кет-мей\-ке\-ров, предоставляющих
ликвидность в~книге заявок. В~работе~\cite{Easley2012} предложена
процедура оценки токсичности потока заявок на основе анализа
информации об интенсивности и~направлении {\it сделок} (направление
сделки определяется в~зависимости от того, кто являлся ее
инициатором~--- покупатель или продавец).
В~данной работе будет рассмотрен более точный подход
к~измерению токсичности потока
заявок, использующий всю доступную информацию о~заявках (не только
сами сделки, но также и~по\-ста\-нов\-ки/сня\-тия заявок).

Чтобы формализовать понятие токсичности потока заявок, для начала
рассмотрим процесс дисбаланса потока заявок $Q(t)\hm=Q^+(t) \hm- Q^-(t)$
в предположении, что $\Esf Q(t) > 0$, т.\,е.\
$$
\lambda^+ \Esf X_1^+ > \lambda^- \Esf X_1^-,
$$
что означает преимущество покупателей над продавцами в~рамках
интервала $[0, T]$. Предположим, что $Q(0)\hm=0$.

Для $u > 0$ рассмотрим вероятность
$$
\varphi_{\pm}(u, T) = \Psf\left(\inf\limits_{0 < t \leqslant T} Q(t) \geqslant -u\right)\,,
$$
т.\,е.\ вероятность того, что траектория процесса $Q(t)$ в~течение
интервала времени $[0, T]$ целиком будет находиться не ниже уровня
$-u$, а~также аналогичную предельную вероятность на бесконечном
интервале времени:
\begin{multline*}
\varphi_{\pm}(u) = \Psf\left(\inf\limits_{t > 0} \left( Q^+(t) - Q^-(t) \right) \geqslant -u\right)
= {}\\
{}=\lim\limits_{T \to \infty} \varphi_{\pm}(u, T)\,.
\end{multline*}
Вероятность $\varphi_{\pm}(u)$ описывает вероятность того, что при {\it
положительном} тренде процесс дисбаланса никогда не достигнет {\it
отрицательного} уровня $-u$ при условии, что параметры потока заявок
($\lambda^+$, $\lambda^-$, $G(x)$ и~$F(x)$) остаются неизменными.


\smallskip

\noindent
\textbf{Определение 1.}\
Функцию $\varphi_{\pm}(u)$ будем называть \textit{профилем мгновенной токсичности}
потока заявок.

\smallskip

Введенная таким образом характеристика~--- профиль мгновенной
токсичности потока заявок~--- формально совпадает с~{\it вероятностью
неразорения} в~классической модели коллективного риска со случайными
премиями, рассматривавшейся, например, в~работах~[15--17].
В~некоторых источниках (см., в~частности,~\cite{KorolevBeningShorgin2011})
справедливо отмечено, что
интерпретация этого показателя именно как вероятности физического
разорения страховой компании некорректна, поскольку изначальное
предположение о~неизменности основных параметров потоков страховых
премий и~страховых выплат в~течение бесконечного интервала времени
заведомо не выполняется. Тем не менее эта характеристика является
удобным показателем текущей эффективности работы страховой компании
и~имеет смысл некоей оценки качества текущего состояния параметров
страховой деятельности. Точно так же профиль мгновенной токсичности
потока заявок является удобно интерпретируемым показателем
неустойчивости текущего состояния потоков заявок.

Из работ~\cite{Boykov2002, Boykov2003} следует

\smallskip

\noindent
\textbf{Лемма~1.}\ \textit{Функция профиля мгновенной токсичности потока заявок
$\varphi_{\pm}(u)$ удовлетворяет интегральному уравнению
\begin{multline*}
\left(\lambda^+ + \lambda^-\right) \varphi_{\pm}(u) =
\lambda^- \int\limits_0^u \varphi_{\pm}(u - v)\,d F(v) + {}\\
{}+\lambda^+
\int\limits_0^{\infty} \varphi_{\pm}(u + v) \,d G(v)\,.
\end{multline*}
Если $R$~--- решение характеристического уравнения
$$
\lambda^+ \left(\Esf e^{-R X^+_1} - 1\right) + \lambda^-
\left(\Esf e^{R X^-_1} - 1\right) = 0\,,
$$
то
$$
\varphi_{\pm}(u) = \fr{e^{-R u}}{\Esf \{\left. e^{-R Q(t)}\right\vert \tau < \infty\}}\,,
$$
при этом} $\varphi_{\pm}(u) \hm\geqslant 1 - e^{-Ru}$.


\section{Токсичность потока заявок}

Профиль токсичности представляет собой {\it функцию}, аргументом
которой является уровень~$u$. Это затрудняет сравнение токсичности
потоков заявок на разных участков рынка, поскольку, вообще говоря,
в~множестве функций
нельзя ввести отношение порядка. Поэтому хотелось бы иметь некий
интегральный показатель токсичности, выражаемый одним числом. Для
построения такого показателя можно воспользоваться одним из двух подходов.

\subsection{Байесовский подход}

Выделим некий <<характеристический>> уровень~$u_0$, пересечение
которого может иметь серьезные последствия. Пусть $w(x)$~---
некоторая плотность распределения вероятностей, обла\-да\-ющая
свойствами
\begin{equation}
\label{eq:evcond}
\int\limits_{0}^{\infty}w(x)\,dx=1\,;\enskip
\int\limits_{0}^{\infty}xw(x)\,dx=u_0\,.
\end{equation}

\noindent
\textbf{Определение~2.}\
Байесовским показателем
\textit{мгновенной токсичности} потока заявок $\theta_{\pm}^{(w)}$
называется величина
$$
\theta_{\pm}^{(w)}=\theta_{\pm}^{(w)}(u_0)=
\int\limits_{0}^{\infty}\varphi_{\pm}(u)w(u)\,du\,.
$$


По сути показатель мгновенной токсичности потока заявок~$\theta_{\pm}$
есть математическое ожидание <<случайного>> профиля
мгновенной токсичности $\varphi_{\pm}(U)$, где~$U$~--- неотрицательная
случайная величина с~плотностью распределения $w(x)$ и~имеющая
математическое ожидание~$u_0$.

В случае, когда $\Esf Q(t) \hm< 0$, т.\,е.\ $\lambda^+ \Esf X_1^+ \hm< \lambda^-
\Esf X_1^-$, что означает преимущество продавцов над покупателями на
интервале $[0, T]$, вместо $\varphi_{\pm}(u)$ будем рассматривать
вероятность
\begin{multline*}
\varphi_{\mp}(u) = \Psf(\sup\limits_{t > 0} \left( Q^+(t) - Q^-(t) \right) \leqslant u)
= {}\\
{}=\lim\limits_{T \to \infty} \varphi_{\mp}(u, T)\,,
\end{multline*}
которая описывает вероятность того, что при {\it отрицательном}
тренде траектория процесса $Q(t)$ не превысит {\it положительный}
уровень~$u$ при условии, что параметры потока заявок ($\lambda^+$,
$\lambda^-$, $G(x)$ и~$F(x)$) остаются неизменными.

В таком случае в~качестве байесовского показателя \textbf{мгновенной
токсичности} потока заявок~$\theta$ возьмем величину
$$
\theta_{\mp}^{(w)}=\theta_{\mp}^{(w)}(u_0)=\int\limits_{0}^{\infty}\varphi_{\mp}(u)w(u)\,du\,.
$$

\subsection{Квантильный подход}

При условии $\Esf Q(t) > 0$ на промежутке $[0, T]$ зафиксируем
некоторое $0 \hm< \alpha\hm < 1$.

\smallskip

\noindent
\textbf{Определение~3.}\
Квантильным {$\alpha$-по\-ка\-за\-те\-лем мгновенной токсичности} потока заявок
называется такое минимальное значение~$q_{\pm}$, при котором
$\varphi_{\pm}(q_{\pm}) \geqslant \alpha$.


Таким образом, при наличии положительного тренда у~процесса $Q(t)$
квантильный $\alpha$-по\-ка\-за\-тель мгновенной токсичности~---
это настолько минимальное значение~$q_{\pm}$, что вероятность того,
что траектория
процесса $Q(t)$ на интервале $[0, T]$ целиком пройдет выше уровня~$-q_{\pm}$,
больше или равна~$\alpha$. Чем больше значение~$q_{\pm}$, тем более
токсичен поток заявок от покупателей.

По аналогии с~предыдущим пунктом при наличии у процесса $Q(t)$
отрицательного тренда (т.\,е.\ при условии $\lambda^+ \Esf X_1^+ \hm<
\lambda^- \Esf X_1^-$) $\alpha$-кван\-тиль\-ный показатель мгновенной
токсичности~$q_{\mp}$ определяется из уравнения
$\varphi_{\mp}(q_{\mp}) \geqslant \alpha.$
Чем больше значение~$q_{\mp}$, тем более токсичен поток заявок от
продавцов на интервале $[0, T$].

\section{Модели потоков заявок}

В некоторых случаях удается напрямую вы\-чис\-лить профиль мгновенной
токсичности потока \mbox{заявок}. Аналоги моделей, приведенных ниже,
рассмотрены в~работе~\cite{Boykov2002} в~рамках модели
Кра\-ме\-ра--Лунд\-бер\-га со стохастическими премиями.

\subsection{Модель рынка с~заявками единичного объема}

Рассмотрим простейшую модель рынка, где потоки заявок имеют
единичный объем, т.\,е.\
$$
\Psf (X_i^+ = 1) = \Psf (X_i^- = 1) = 1\,.
$$
В этом случае
$$
Q(t) = \sum\limits_{i=1}^{N^+(t)} 1 - \sum\limits_{i=1}^{N^-(t)} 1 = N^+(t) - N^-(t)\,.
$$
Несмотря на очевидно идеальный характер такого примера, он имеет
реальный практический смысл, поскольку при этом становится возможным
учитывать чистые интенсивности потоков заявок и~отслеживать влияние
их соотношения (дисбаланса интенсивностей потоков заявок) на
токсичность ситуации. Более того, в~таком случае рассматриваемый
процесс дисбаланса потоков заявок $Q(t)$ является простейшим
процессом рождения и~гибели, различные характеристики которого можно
исследовать специально разработанными для этого методами.

Если $\lambda_+ > \lambda_-$, то покупатели преобладают над
продавцами и~характеристическое уравнение имеет вид:
$$
\lambda^+ \left[ e^{-R} - 1 \right] + \lambda^- \left[ e^{R} - 1 \right] = 0\,,
$$
откуда $e^R = {\lambda^+}/{\lambda^-}$ или $e^R \hm= 1$. По лемме~1
для $u \hm> 0$ имеем
$$
\varphi_{\pm}(u) \geqslant 1 - \left( \fr{\lambda^-}{\lambda^+} \right)^ {u}\,;\enskip
\varphi_{\pm}(\infty) = 1\,.
$$
Равенство $\varphi_{\pm}(u) \hm= \varphi_{\pm}([u])$ очевидно. Для целых~$u$
интегральное уравнение переходит в~разностное:
\begin{equation}
\lambda_1 \varphi_{\pm}(u + 1) - \left(\lambda^- + \lambda^+\right) \varphi_{\pm}(u)
+ \lambda \varphi_{\pm}(u - 1) = 0\,,
\label{eq:phidiff}
\end{equation}
откуда
$$
\varphi_{\pm}(u) = C_1 + C_2 \left (
\fr{\lambda^-}{\lambda^+} \right )^u\,,\enskip C_1 \hm= \varphi(\infty) = 1\,.
$$
Константу~$C_2$ найдем при подстановке в~уравнение~(\ref{eq:phidiff}) $u \hm= 0$:
$$
\left( \lambda^- + \lambda^+ \right) \varphi(0) = \lambda \varphi(1)\,,\enskip
C_2 = -\fr{\lambda^-}{\lambda^+}\,,
$$
откуда получаем, что для $u \hm> 0$ профиль мгновенной токсичности имеет вид:
$$
\varphi_{\pm}(u) = \varphi_{\pm}([u]) = 1 -  \left( \fr{\lambda^-}{\lambda^+} \right)^{[u] + 1}\,.
$$

\subsubsection{Байесовский показатель мгновенной токсичности}



Коль скоро исследователь может сам назначать уровень~$u_0$,
относительно которого будут рассчитываться характеристики
токсичности потока заявок, будем рассматривать~$u_0$ на множестве
натуральных чисел, а~в~качестве функции $w(u)$ можно выбрать функцию
плотности вероятности распределения Пуассона (относительно считающей
меры), также определенную на множестве натуральных чисел.
Для удобства обозначим $r \hm= {\lambda^-}/{\lambda^+}$, при этом
$r \hm< 1$ (рис.~3). В~таком случае


\begin{center}  %fig3
\vspace*{1pt}
 \mbox{%
 \epsfxsize=77.205mm
 \epsfbox{che-3.eps}
 }
\end{center}

\noindent
{{\figurename~3}\ \ \small{Функция профиля токсичности потока заявок $\varphi_{\pm}(u)$
в~модели рынка с~единичными потоками заявок для разных значений
$r \hm= {\lambda^-}/{\lambda^+}$: \textit{1}~--- $r=2/5$;
\textit{2}~--- 1/2; \textit{3}~--- $r=2/3$.
Штрихпунктирная кривая: функция $w(u)$~---
    плотность (относительно считающей меры) пуассоновского
    распределения со средним $u_0 \hm= 3$}}


\vspace*{9pt}


\addtocounter{figure}{1}



\noindent
\begin{multline*}
\theta_{\pm}^{(w)}(u_0) = \int\limits_{\mathbb{N}} \varphi_{\pm}(u) w(u)\, du ={}\\
{}= \sum\limits_{k=0}^{\infty} (1 - r^{k + 1}) \fr{u_0^k e^{-u_0}}{k!} =
1 - r e^{-u_0} \sum\limits_{k=0}^{\infty} \fr{(ru_0)^k}{k!} = {}\\
{}=1 - r e^{u_0(r - 1)}\,.
\end{multline*}
На рис.~\ref{fig:toxunitsize},\,\textit{a} изображен
график токсичности в~зависимости от значений $r \hm= {\lambda^-}/{\lambda^+}$ для
фиксированного значения $u_0 \hm= 3$ в~условиях положительного тренда
($\lambda^+ \hm> \lambda^-$). Чем меньше значение~$r$, тем токсичнее
рынок. И~напротив: рынок нетоксичен, когда $\lambda^+ \hm= \lambda^-$,
т.\,е.\ наблюдается баланс между покупателями и~продавцами.

\subsubsection{Квантильный показатель мгновенной токсичности}

Для заданного $\alpha \hm\in (0, 1)$ квантильный показатель мгновенной
токсичности~--- это такое минимальное $q_{\pm}\hm \in \mathbb{N}$, при котором
$$
\varphi_{\pm}(q_{\pm}) = 1 - r^{q_{\pm} + 1} \geqslant \alpha\,,
$$
откуда
$$
q_{\pm}(\alpha) = \left \lceil \fr{\ln(1 - \alpha)}{\ln r} - 1 \right \rceil\,.
$$
Заметим, что при $r \hm= 1$ $\alpha$-кван\-тиль\-ный показатель мгновенной
токсичности не определен и~в~таком случае полагается равным нулю.

\begin{figure*} %fig4
\vspace*{1pt}
 \begin{center}
 \mbox{%
 \epsfxsize=160.51mm
 \epsfbox{che-4.eps}
 }
 \end{center}
 \vspace*{-9pt}
\Caption{Графики показателей токсичности в~зависимости от значения
$r \hm= {\lambda^-}/{\lambda^+}$: (\textit{а})~байесовский подход;
(\textit{б})~квантильный подход}
    \label{fig:toxunitsize}
\end{figure*}

На рис.~\ref{fig:toxunitsize},\,\textit{б} на график нанесены различные значения
квантильного показателя мгновенной токсичности потока заявок в~зависимости от
значения $r \hm= {\lambda^-}/{\lambda^+}$ на
промежутке $[0, T]$. Токсичность покупателей максимальна при малых
значениях~$r$ и~близка к нулю при наличии баланса между покупателями
и~продавцами.
Монотонность обоих графиков по $r$ подтверждает обоснованность
использования $\theta_{\pm}$ и~$q_{\pm}$ в~качестве показателей
токсичности потока заявок в~случае модели рынка с~заявками
единичного объема.

\vspace*{-3pt}

\subsection{Модель рынка с~экспоненциальными объемами заявок}

Пусть объемы заявок покупателей и~продавцов имеют экспоненциальное распределение
(рис.~5),
т.\,е.\
$$
G(t) = 1 - e^{-bt}\,; \quad F(t) = 1 - e^{-at}\,.
$$

\begin{center}  %fig5
\vspace*{6pt}
  \mbox{%
 \epsfxsize=78.264mm
 \epsfbox{che-5.eps}
 }
\end{center}

\vspace*{-5pt}

\noindent
{{\figurename~5}\ \ \small{Функция профиля токсичности $\varphi_{\pm}(u)$ в~модели рынка
    с~экспоненциальными объемами заявок для разных наборов
    $(\lambda^+, \lambda^-, b, a)$:
    \textit{1}~--- (3, 1, 2, 1); \textit{2}~--- (5, 2, 2, 1);
    \textit{3}~--- (1, 1, 1, 3).
Штрихпунктирная кривая: весовая функция $w(u)$~---
    плот\-ность гам\-ма-рас\-пре\-де\-ле\-ния $\Gamma(u_0^2, u_0^{-1})$ при
    $u_0 \hm= 2$ }}

    \columnbreak


%\vspace*{9pt}




\addtocounter{figure}{1}



\noindent
В случае, когда покупатели преобладают над продавцами, т.\,е.\
$\lambda^+ / b \hm> \lambda^- / a$, характеристическое уравнение имеет
вид:
$$
\lambda^+ \left[ \fr{b}{b + R} - 1 \right] + \lambda^- \left[ \fr{a}{a - R} -
1 \right] = 0\,,
$$
откуда $R = (\lambda^+ a \hm- \lambda^- b) / (\lambda^+ \hm+ \lambda^-)$ или~0,
а~профиль мгновенной токсичности потока заявок~\cite{Boykov2002}
$$
\varphi_{\pm}(u) = \fr{(a + b) \lambda^-}{(\lambda^+ + \lambda^-) a} \exp
\left( -\fr{\lambda^+ a - \lambda^- b}{\lambda^+ + \lambda^-}\, u \right)\,.
$$
В~случае, когда продавцы преобладают над покупателями,
$$
\varphi_{\mp}(u) = \fr{(a + b) \lambda^+}{(\lambda^+ + \lambda^-) b}
\exp \left( -\fr{\lambda^- b - \lambda^+ a}{\lambda^+ + \lambda^-}
\,u \right)\,.
$$







\subsubsection{Байесовский показатель мгновенной токсичности}

На множестве функций $w(u)$, удовле\-тво\-ря\-ющих условиям~(\ref{eq:evcond}),
рассмотрим функции, удовлетво\-ря\-ющие также условию
\begin{equation}
\label{eq:stdcond}
\int\limits_{0}^{\infty}x^2w(x)\,dx - \left(
\int\limits_{0}^{\infty}xw(x)\,dx \right)^2 = 1\,,
\end{equation}
т.\,е.\ обеспечивающие единичную дисперсию соответствующей случайной
величины, имеющей функцию $w(u)$ в~качестве плотности своего
распределения.

Для вычисления байесовского показателя мгновенной токсичности
возьмем в~качестве $w(u)$ плотность гам\-ма-рас\-пре\-де\-ле\-ния

\pagebreak

\noindent
$$
w(u) = u^{k-1} \fr{e^{-u / \theta}}{\theta^k \, \Gamma(k)}\,,
$$
где $\Gamma(k)$~--- гам\-ма-функ\-ция Эйлера:
$$
\Gamma(k) = \int\limits_0^{+\infty} t^{k-1}e^{-t}\,dt\,.
$$
Поскольку математическое ожидание и~дисперсия случайной величины~$U$,
имеющей гам\-ма-рас\-пре\-де\-ле\-ние, равны $k \theta$ и~$k \theta^2$
соответственно, то с~учетом условий~(\ref{eq:evcond}) и~(\ref{eq:stdcond})
значения~$k$ и~$\theta$ определяются из уравнений
$k \theta \hm= u_0$ и~$k \theta^2 \hm= 1$, откуда $k \hm= u_0^2$ и~$\theta \hm=
u_0^{-1}$.

\begin{figure*}[b] %fig6
\vspace*{8pt}
 \begin{center}
 \mbox{%
 \epsfxsize=164.366mm
 \epsfbox{che-6.eps}
 }
 \end{center}
 \vspace*{-9pt}
  \Caption{Оценка параметров $\lambda^+$, $\lambda^-$, $b$ и~$a$ в~режиме реального
  времени (серый цвет~--- интервалы доверия),
  ось~$x$~--- номер соответствующего $\tau$-ин\-тер\-ва\-ла
  (фьючерс на индекс РТС, дневная сессия 01.07.2014)}
  \label{fig:muhat}
%  \vspace*{-2pt}
\end{figure*}


Для удобства обозначим
\begin{equation}
\label{eq:alphabeta}
\beta =  \fr{(a + b) \lambda^-}{(\lambda^+ + \lambda^-) a} \mbox{ и~}
\gamma = \fr{\lambda^+ a - \lambda^- b}{\lambda^+ + \lambda^-} > 0\,.
\end{equation}
Байесовский показатель мгновенной токсичности равен
\begin{multline*}
\theta_{\pm}^{(w)}(u_0) = \int\limits_{0}^{\infty}\varphi_{\pm}(u)w(u)\,du ={}\\
{}=
\int\limits_{0}^{\infty} \left(1 - \beta e^{-\gamma u}\right) u^{k-1}
\fr{e^{-u / \theta}}{\theta^k \, \Gamma(k)}\, du = {}\\
{}    = 1 - \fr{\beta}{\theta^k \Gamma(k)} \int\limits_0^\infty
e^{-(\gamma + \theta^{-1}) u} u^{k - 1}\, du ={}\\
{}=
\left[ t = \fr{\theta \gamma + 1}{\theta} u \right] =
 1 -{}\\
 {}- \fr{\beta}{(\theta \gamma + 1)^k \Gamma(k)}
\int\limits_0^\infty e^{-t} \fr{\theta^{k - 1}}{(\theta \gamma + 1)^{k - 1}}\,
t^{k - 1} \fr{\theta}{\theta \gamma + 1}\, dt ={} \\
{}    = 1 - \fr{\beta}{(\theta \gamma + 1)^k}\,.
\end{multline*}
После подстановки~$k$ и~$\theta$ получаем значение показателя
$$
\theta_{\pm}(u_0) = 1 - \fr{\beta}{ \left( \gamma u_0^{-1} + 1 \right)^{u_0^2}}\,.
$$

\subsubsection{Квантильный показатель мгновенной токсичности}

Для заданного $\alpha \hm\in (0, 1)$ квантильный показатель мгновенной
токсичности~--- это такое минимальное~$q_{\pm}$, при котором
$$
\varphi_{\pm}(q_{\pm}) = 1 - \beta e^{-\gamma q_{\pm}} \geqslant \alpha\,.
$$
Так как функция~$\varphi_{\pm}$ является непрерывной по~$q_{\pm}$, то
данное неравенство может быть обращено в~равенство, откуда получаем
$$
q_{\pm}(\alpha) = \fr{\ln \beta - \ln (1 - \alpha)}{\gamma}\,.
$$

Заметим, что байесовский и~квантильный показатели токсичности
являются монотонными по каждой из величин~$\beta$ и~$\gamma$.

\section{Реальные данные}

В данном разделе описывается структура данных о~потоках заявок, на
базе которых можно провести валидацию модели,
предложенной в~параграфе~6.2.
{\looseness=1

}

Далее оценим параметры потока заявок
$\lambda^+$, $\lambda^-$, $b$ и~$a$ в~режиме скользящего окна и~рассчитаем
функции профиля мгновенной токсичности, а~так\-же показатели
мгновенной токсичности потока \mbox{заявок} $\theta(t)$ и~$q(t)$ в~режиме
реального времени и~проанализируем адекватность полученных
характеристик.
{\looseness=1

}

\begin{table*}\small
\begin{center}


\tabcolsep=3.5pt
\begin{tabular}{|l|*{15}{c|}r|}
\multicolumn{17}{c}{Пример данных о потоке заявок для фьючерса на индекс РТС}\\[6pt]
\hline
&&&&&&&&&&&&&&&&\\[-9pt]
\multicolumn{1}{|c|}{Время} & Тип & Опрерация & Цена, у.е. & Объем & $b_1$, у.е. & $a_1$, у.е. & $v^b_1$ & $v^a_1$ & $v^b_2$ & $v^a_2$ & $v^b_3$ & $v^a_3$ & $v^b_4$ & $v^a_4$ & $v^b_5$ & $v^a_5$  \\
\hline
10:02:36,444 & L & Покупка & 130\,020 & 2 & 130040 & 130050 & 2 & 4 & 22 & 23 & {\bf 54} & 22 & 81 & 31 & 759 & 20 \\
10:02:36,445 & L & Продажа & 130\,070 & 1 & 130040 & 130050 & 2 & 4 & 22 & 23 & {\bf 55} & 22 & 81 & 31 & 759 & 20 \\
10:02:36,465 & C & Покупка & 130\,040 & 1 & 130040 & 130050 & {\bf 1} & 4 & 22 & 23 & 55 & 22 & 81 & 31 & 759 & 20 \\
10:02:36,473 & L & Покупка & 130\,050 & 3 & 130040 & 130050 & 1 & {\bf 1} & 22 & 23 & 55 & 22 & 81 & 31 & 759 & 20 \\
\hline
\end{tabular}
\end{center}
\end{table*}

\subsection{Описание данных}

Рассматриваются данные о~потоках всех заявок
(лимитных, рыночных и~заявок на отмену) на первые $d \hm= 5$
уровней книги заявок фьючерса на
индекс РТС (Российской торговой системы)
за период с~1 по~30~июля 2014~г. Эти данные дают доступ
к~самой детальной информации о~рыночных торгах в~отличие от данных
о~сделках и~котировках (TAQ, Trades and Quotes), которые часто
используются для анализа высокочастотных данных и~состоящих из цен
и~объемов сделок (что соответствует только рыночным заявкам в~потоке
всех заявок), а~так\-же информации о~цене и~объеме лучших котировок на
покупку и~продажу (т.\,е.\ только первый уровень книги заявок)
с~проставленными моментами времени.

В таблице приведен пример данных о~потоке заявок для фьючерса на
индекс РТС и~о~том, как выглядел срез книги заявок после прихода
соответствующей заявки. Заметим, что на рынке FORTS присутствует
всего два типа заявок: лимитные (L) и~заявки на отмену (C),
а~механизм рыночных заявок участники рынка реализуют самостоятельно
(отправляя лимитные заявки с~ценами, гарантирующими их моментальное
исполнение). Тем не менее имеется возможность оценить параметры
потоков рыночных заявок в~рамках предлагаемой модели, рассматривая для
этого потоки лимитных заявок, которые приводили к сделкам.

%\vspace*{-7pt}

\subsection{Процедура оценки параметров}



Разобьем один из рассматриваемых торговых дней (1~июля 2014~г.)\ на
временн$\acute{\mbox{ы}}$е интервалы с~шагом $\tau \hm= 15$~с.
При этом исключим интервалы времени в~5~мин торгов (с~10:00
до~10:05), а~также в~последние 5~мин торгов (с~18:40 до~18:45),
поскольку они характеризуются аномальными всплесками волатильности,
слабо поддающейся анализу в~рамках представленной модели. Внутри
каждого $\tau$-ин\-тер\-ва\-ла проведем оценку параметров $\lambda^+,
\lambda^-, b$ и~$a$ согласно модели рынка с~экспоненциальными
объемами заявок, предложенной в~параграфе~6.2. Результат оценки
параметров в~режиме реального времени изображен на рис.~\ref{fig:muhat}.

%\vspace*{-7pt}

\subsection{Показатели токсичности}



На основе оценок для $\lambda^+$, $\lambda^-$, $b$ и~$a$ можно вы\-чис\-лить~$\beta$
и~$\gamma$, а~затем построить графики показателей мгновенной токсичности
потока заявок
$\theta(u_0)$ и~$q(\alpha)$ для фиксированных~$u_0$ и~$\alpha$
в~режиме реального времени (рис.~7)
и~идентифицировать участки, на которых деятельность покупателей или
продавцов была токсичной. Прикладные исследования демонстрируют
достаточную значимость данного показателя для своевременной
идентификации участков неблагоприятного отбора мар\-кет-мей\-керов.

%\vspace*{-7pt}

\section{Заключение}

В~данной работе рассмотрена микроструктурная модель рынка,
в~которой потоки заявок моделируются пуассоновскими процессами
с~постоянными интенсивностями (такая аппроксимация возможна на
небольших временн$\acute{\mbox{ы}}$х интервалах). В~качестве интегрального индикатора
текущего состояния книги заявок применялся дисбаланс потока заявок
(order flow imbalance), который использует не только текущие значения
наилучших цен покупки и~продажи, но и~влияние событий <<в~глубине>>
книги заявок и~потому меняется существенно быст\-рее и~позволяет
интерполировать динамику рынка между изменениями цены, в~частности
отслеживать ситуации, связанные с~токсичностью потока заявок.
В~рамках рассмотренной модели были введены такие понятия,
как мгновенный профиль токсичности, а~так\-же байесовский и~квантильний
показатели токсичности, рассчитываемые на основе параметров, описывающих
потоки всех заявок. Эти показатели рассчитываются для двух модельных типов
потоков заявок, в~первом из которых заявки имеют единичный объем, во втором~---
объем заявок является случайным и~имеющим показательное распределение. Для
последней из двух моделей была проведена валидация на реальных данных
(фьючерс на индекс РТС) и~были построены показатели токсичности в~режиме
реального\linebreak\vspace*{-12pt}
\begin{center}  %fig7
\vspace*{1pt}
 \mbox{%
 \epsfxsize=79.735mm
 \epsfbox{che-7.eps}
 }
\end{center}

%\vspace*{-3pt}

\noindent
{{\figurename~7}\ \ \small{Графики $\beta$ и $\gamma$, рассчитанных по формулам~(\ref{eq:alphabeta}),
 байесовского и~квантильного показателей токсичности в~режиме реального времени,
 ось~$x$~--- номер сооветствующего $\tau$-ин\-тер\-ва\-ла
 (фьючерс на индекс РТС, дневная сессия 01.07.2014)}}

 \vspace*{13pt}




\noindent
 времени. Предложенная методика расчета показателей токсичности
на основе информации о~потоках всех заявок является перспективной и~может
быть распространена на модели рынка с~неоднородными интенсивностями потоков
заявок.


\smallskip

Автор статьи выражает огромную благодарность своему научному
руководителю профессору Виктору Юрьевичу Королеву за ценные идеи и~замечания,
а~также студентам факультета вычислительной
математики и~кибернетики МГУ
им.\ М.\,В.~Ломоносова Дарье Николайчук и~Гелане
Хазеевой за помощь в~подготовке материала статьи.

%\vspace*{-9pt}

{\small\frenchspacing
 {%\baselineskip=10.8pt
 \addcontentsline{toc}{section}{References}
 \begin{thebibliography}{99}
\bibitem{Jeria2008} %1
\Au{Jeria, D., Sofianos~G.}
Passive orders and natural adverse selection~//
Street Smart, September~4, 2008. No.\,33.

\bibitem{Glosten1985} %2
\Au{Glosten L.\,R., Milgrom~P.} Bid, ask and transaction prices in a specialist market
with heterogeneously informed traders~// J.~Financ. Econ., 1985.
Vol.~14. P.~71--100.

\bibitem{Kyle1985} %3
\Au{Kyle A.\,S.} Continuous auctions and insider trading~// Econometrica, 1985.
Vol.~53. P.~1315--1335.

\bibitem{Easley1992} %4
\Au{Easley D.,  O'Hara~M}. Time and the process of security price adjustment~//
J.~Financ., 1992. Vol.~47. P.~576--605.

\bibitem{Easley2012} %5
\Au{Easley D., Lopez de Prado~M., O'Hara~M.}
Flow toxicity and liquidity in a high frequency world.
\textit{Rev. Financ. Stud.}, 2012. Vol.~25. No.\,5. P.~1457--1493.

\bibitem{Korolev_2013} %6
\Au{Королев В.\,Ю., Черток А.\,В.,  Корчагин~А.\,Ю., Горшенин~А.\,К.}
Ве\-ро\-ят\-ност\-но-ста\-ти\-сти\-че\-ское моделирование информационных потоков
   в~сложных финансовых системах на основе высокочастотных данных~//
  Информатика и её применения, 2013. Т.~7. Вып.~1. С.~12--21.

\bibitem{Chertok2014} %7
\Au{Chertok A., Korolev V., Korchagin~A., Shorgin~S.}
Modeling high-frequency non-homogeneous order flows by compound Cox processes.
January~14, 2014. %Available at SSRN:
{\sf http://ssrn.com/ abstract=2378975}.


\bibitem{ContRamaStoikov2010b} %8
\Au{Cont R., Stoikov~S., Talreja~R.} A~stochastic model for order book dynamics~//
Oper. Res., 2010. Vol.~58. No.\,3. P.~549--563.

\bibitem{ContLarrard2011} %9
\Au{Cont R., de Larrard~A.} Price dynamics in a~Mar\-ko\-vian limit order market.
Working paper.
{\sf http://ssrn.com/\linebreak abstract=1735338}.

\bibitem{Bouchaud2002} %10
\Au{Bouchaud J.-P.,  Mezard M., Potters~M.}  Statistical properties of
stock order books: Empirical results and models~// Quant. Financ., 2002.
Vol.~2. P.~251--256.

\bibitem{ZovkoFarmer2002} %11
\Au{Zovko I., Farmer~J.\,D.}  The power of patience; A behavioral regularity
in limit order placement~// Quant. Financ., 2002. Vol.~2. P.~387--392.



\bibitem{Cont2011} %12
\Au{Cont R.,  Kukanov A., Stoikov~S.}
The price impact of order book
events. March 01, 2011.
{\sf http://ssrn.com/ abstract=1712822}.

\bibitem{Cont2014}
\Au{Cont R., Kukanov A., Stoikov~S.} The price impact of order book
events~// J.~Financ. Economet., 2014. Vol.~12. No.\,1. P.~47--88.

\bibitem{Korolev_2014}
\Au{Korolev V., Chertok A., Zeifman~A.}
Functional limit theorems for order flow imbalance process.
{\sf http://ssrn. com/abstract=1735338}.


\bibitem{Boykov2002}
\Au{Boykov A.} Cramer--Lundberg model with stochastic premiums~//
Theor. Probab. Appl., 2002. Vol.~47. No.\,3. P.~549--553.

\bibitem{Boykov2003}
\Au{Бойков А.\,В.} Стохастические модели капитала страховой
компании и~оценивание вероятности неразорения.  Дисс.\ \ldots\ канд.
физ.-мат. наук.~--- М.: Математический институт им.\ В.\,А.~Стеклова
РАН, 2003. 83~с.

\bibitem{Temnov2004} %17
\Au{Темнов Г.\,О.} Математические модели риска и~случайного
притока взносов в~страховании. Дисс.\ \ldots\ канд.\
 физ.-мат. наук.~---
С.-Пе\-тер\-бург: Санкт-Пе\-тер\-бург\-ский государственный
ар\-хи\-тек\-тур\-но-стро\-и\-тель\-ный университет, 2004. 102~с.

\bibitem{KorolevBeningShorgin2011}
\Au{Королев В.\,Ю., Бенинг В.\,Е., Шоргин~С.\,Я.}
Математические основы теории риска.~--- 2-е изд., перераб. и~доп.~---
М.: Физматлит, 2011. 620~с.
 \end{thebibliography}

 }
 }

\end{multicols}

\vspace*{-6pt}

\hfill{\small\textit{Поступила в редакцию 08.10.14}}

%\newpage

\vspace*{12pt}

\hrule

\vspace*{2pt}

\hrule

%\vspace*{12pt}

\def\tit{ON THE FORMALIZATION OF~ORDER FLOW TOXICITY ON~FINANCIAL MARKETS}

\def\titkol{On the formalization of order flow toxicity on financial markets}

\def\aut{A.\,V.~Chertok$^{1,2}$}

\def\autkol{A.\,V.~Chertok}

\titel{\tit}{\aut}{\autkol}{\titkol}

\vspace*{-9pt}

 \noindent
 $^1$Faculty of Computational Mathematics and Cybernetics,
M.\,V.~Lomonosov Moscow State University;\linebreak
$\hphantom{^1}$1-52  Leninskiye Gory, GSP-1, Moscow 119991, Russian Federation

\noindent
$^2$Euphoria Group LLC,
 9, bld.~1, of.~6 Arkhangelsky Lane, Moscow 101000, Russian Federation




\def\leftfootline{\small{\textbf{\thepage}
\hfill INFORMATIKA I EE PRIMENENIYA~--- INFORMATICS AND
APPLICATIONS\ \ \ 2014\ \ \ volume~8\ \ \ issue\ 4}
}%
 \def\rightfootline{\small{INFORMATIKA I EE PRIMENENIYA~---
INFORMATICS AND APPLICATIONS\ \ \ 2014\ \ \ volume~8\ \ \ issue\ 4
\hfill \textbf{\thepage}}}

\vspace*{3pt}


\Abste{The paper considers the microstructural order flow model for
financial markets. The order flow imbalance process is used as an integral
indicator of the current state of the limit-order book. The model of order flow
imbalance is used to analyze the properties of the current limit-order
book state, which is considered as two-sided risk process with stochastic premiums.
The concept of order flow toxicity on financial markets is studied.
This concept is formalized with probabilities of crossing fixed levels by the
order flow imbalance process. The paper introduces the concepts of the
instantaneous toxicity profile and Bayesian and quantile indicators of toxicity.
These indicators are calculated for two model types of order flows: the first
one has unit volume orders and the second one consists of orders with random volume
which has exponential distribution.}


\KWE{financial markets; limit-order book; order flow;
order flow imbalance; adverse selection; order flow toxicity;
Poisson process; compound Poisson process; two-side risk process;
risk process with stochastic premiums; ruin probability}

  \DOI{10.14357/19922264140403}


\Ack
\noindent
The research was partly financially supported by the Russian Foundation
for Basic Research (project 14-07-00041а).



%\vspace*{3pt}

  \begin{multicols}{2}

\renewcommand{\bibname}{\protect\rmfamily References}
%\renewcommand{\bibname}{\large\protect\rm References}



{\small\frenchspacing
 {%\baselineskip=10.8pt
 \addcontentsline{toc}{section}{References}
 \begin{thebibliography}{99}

\bibitem{Jeria2008-1}
\Aue{Jeria, D., and G. Sofianos}. September~4, 2008.
Passive orders and natural adverse selection. {\it Street Smart} 33.

\bibitem{Glosten1985-1}
\Aue{Glosten L.\,R., and P. Milgrom}. 1985.
Bid, ask and transaction prices in a specialist market
with heterogeneously informed traders.
{\it J.~Financ. Econ.} 14:71--100.

\bibitem{Kyle1985-1}
\Aue{Kyle, A.\,S.} 1985.
Continuous auctions and insider trading. {\it Econometrica} 53:1315--1335.

\bibitem{Easley1992-1}
\Aue{Easley, D., and M.~O'Hara}. 1992.
Time and the process of security price adjustment.
{\it J.~Financ.} 47:576--605.

\bibitem{Easley2012-1}
\Aue{Easley, D., M.~Lopez de~Prado, and M.~O'Hara}. 2012.
Flow toxicity and liquidity in a high frequency world.
{\it Rev. Financ. Stud.} 25(5):1457--1493.

\bibitem{Korolev_2013-1}
\Aue{Korolev, V., A. Chertok, A.~Korchagin, and A.~Gorshenin}. 2013.
Veroyatnostno-statisticheskoe modelirovanie informatsionnykh
potokov v~slozhnykh finansovykh sistemakh na osnove vysokochastotnykh dannykh
[Probability and statistical modeling of information flows in complex financial
systems from high-frequency data]. \textit{Informatika i ee Primeneniya}~---
\textit{Inform. Appl.} 7(1):12--21.

\bibitem{Chertok2014-1}
\Aue{Chertok, A., V. Korolev , A.~Korchagin, and S.~Shorgin}. 2014.
Modeling high-frequency non-homogeneous order flows by compound Cox processes.
Available at: {\sf http://ssrn.com/abstract=2378975} (accessed January~14, 2014).

\bibitem{ContRamaStoikov2010b-1}
\Aue{Cont, R., S. Stoikov, and R.~Talreja}. 2010.
A~stochastic model for order book dynamics. {\it Oper. Res.} 58(3):549--563.

\bibitem{ContLarrard2011-1}
\Aue{Cont, R., and A.~de~Larrard}.
Price dynamics in a Markovian limit order market.
Working paper. Available at: {\sf http://ssrn.com/abstract=1735338}
 (accessed February 2012).

 \bibitem{Bouchaud2002-1}
\Aue{Bouchaud, J.-P., M. Mezard, and M.~Potters}. 2002.
Statistical properties of stock order books: Empirical results and models.
{\it Quant. Financ.} 2:251--256.

\bibitem{ZovkoFarmer2002-1}
\Aue{Zovko, I., and J.\,D.~Farmer}. 2002. The power of patience;
a~behavioral regularity in limit order placement. {\it Quant. Financ.} 2:387--392.



\bibitem{Cont2011-1}
\Aue{Cont, R., A. Kukanov, and S.~Stoikov}.
The price impact of order book events. Available at:
{\sf http:// ssrn.com/ abstract=1735338} (accessed March~01, 2011).

\bibitem{Cont2014-1}
\Aue{Cont, R., A. Kukanov, and S.~Stoikov}.
2014. The price impact of order book events. {\it J.~Financ. Economet.} 12(1):47--88.

\bibitem{Korolev_2014-1}
\Aue{Korolev, V., A. Chertok, and A.~Zeifman}.
Functional limit theorems for order flow imbalance process.
Available at:
{\sf http://ssrn.com/abstract=1735338} (accessed October~6, 2014).

\bibitem{Boykov2002-1}
\Aue{Boykov, A.} 2002. Cramer--Lundberg model with stochastic premiums.
{\it Theor. Probab.  Appl.} 47(3):549--553.

\bibitem{Boykov2003-1}
\Aue{Boykov, A.} 2003. Stokhasticheskie modeli kapitala strakhovoy
kompanii i~otsenivanie veroyatnosti nerazoreniya [Stochastic
models of the capital of the insurance company and the evaluation of the
 probability of non-bankruptcy]. Ph.D. Diss. Moscow. 83~p.

\bibitem{Temnov2004-1}
\Aue{Temnov, G.\,O.}
2004. Matematicheskie modeli riska i~sluchaynogo pritoka vznosov
v~strakhovanii [Mathematical models of risk and random inflow of
contributions to insurance]. Ph.D. Diss. St.\ Petersburg. 102~p.

\bibitem{KorolevBeningShorgin2011-1}
\Aue{Korolev, V.\,Yu., V.\,E.~Bening, and S.\,Ya.~Shorgin}.
2011. {\it Matematicheskie osnovy teorii riska} [Mathematical
foundations of the risk theory]. Moscow: Fizmatlit. 620~p.

\end{thebibliography}

 }
 }

\end{multicols}

\vspace*{-6pt}

\hfill{\small\textit{Received October 8, 2014}}

\vspace*{-18pt}

\Contrl

\noindent
\textbf{Chertok Andrey V.} (b.\ 1987)~---
junior scientist, Faculty of Computational Mathematics and Cybernetics,
M.\,V.~Lomonosov Moscow State University;
1-52  Leninskiye Gory, GSP-1, Moscow 119991, Russian Federation;
Director General,  Euphoria Group LLC,
 9, bld.~1, of.~6 Arkhangelsky Lane, Moscow 101000,
 Russian Federation;  a.v.chertok@gmail.com
\label{end\stat}

\renewcommand{\bibname}{\protect\rm Литература}
