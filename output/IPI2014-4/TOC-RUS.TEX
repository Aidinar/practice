%Том 8   Выпуск 1-4   Год 2014

\def\stat{cont}
{%\hrule\par
%\vskip 7pt % 7pt
\raggedleft\Large \bf%\baselineskip=3.2ex
А\,В\,Т\,О\,Р\,С\,К\,И\,Й\ \ У\,К\,А\,З\,А\,Т\,Е\,Л\,Ь\ \ З\,А\ \ 2\,0\,1\,4 г. \vskip 17pt
    \hrule
    \par
\vskip 21pt plus 6pt minus 3pt }

\label{st\stat}

\def\tit{\ }

\def\aut{\ }
\def\auf{\ }

\def\leftkol{\ } % ENGLISH ABSTRACTS}

\def\rightkol{\ } %АВТОРСКИЙ УКАЗАТЕЛЬ ЗА 2014 г.} %ENGLISH ABSTRACTS}

\titele{\tit}{\aut}{\auf}{\leftkol}{\rightkol}

\vspace*{-12pt}
\vspace*{-12pt}

{\tabcolsep=3pt
\begin{tabular}{p{388pt}rr}
&\textbf{Выпуск} & \textbf{Стр.}\\[6pt]

\textbf{Агаларов~Я.\,М.} Модели для сравнительного анализа методов классификации в
неко-\linebreak
\vspace*{-12pt}\\
\hspace*{23pt}торых распределенных системах распознавания образов\dotfill&3&45\\
\textbf{Адигеев~М.\,Г.} О полиномиальной разрешимости ультраметрических версий
некоторых\linebreak
\vspace*{-12pt}\\
\hspace*{23pt}NP-трудных задач\dotfill&2&70\\
\textbf{Архипов~О.\,П., Зыкова~З.\,П.} Применение полутоновых представлений при
анализе\linebreak
\vspace*{-12pt}\\
\hspace*{23pt}изменений цветных изображений\dotfill&3&90\\
\textbf{Архипов~О.\,П., Маньяков~Ю.\,А., Сиротинин~Д.\,О.} Информационная модель
технологии\linebreak
\vspace*{-12pt}\\
\hspace*{23pt}представления натурного объекта и изменения его пространственного
положения\dotfill&1&71\\
\textbf{Архипов~О.\,П., Маньяков~Ю.\,А.} Текстурирование воксельных моделей на
основе цве-\linebreak
\vspace*{-12pt}\\
\hspace*{23pt}товой информации об опорных точках\dotfill&3&100\\
\hangindent=23pt\noindent\textbf{Бенинг~В.\,Е., Драницына~М.\,А., Захарова~Т.\,В., Карпов~П.\,И.} Решение
обратной задачи в многодипольной модели источников магнитоэнцефалограмм методом
незави-\linebreak
\vspace*{-12pt}\\
\hspace*{23pt}симых компонент\dotfill&2&77\\
\textbf{Бирюкова~Т.\,К.} см.\ Киреев~В.\,И.&&\\
\textbf{Бобков~С.\,Г.} см.\ Соколов~И.\,А.&&\\
\textbf{Борисов~А.\,В.} Применение алгоритмов оптимальной фильтрации для решения
задачи\linebreak
\vspace*{-12pt}\\
\hspace*{23pt}мониторинга доступности удаленного сервера\dotfill&3&53\\
\textbf{Босов~А.\,В.} Обобщенная задача распределения ресурсов программной
системы\dotfill&2&39\\
\hangindent=23pt\noindent\textbf{Бронштейн~Е.\,М., Зелёв~П.\,А.}
Об оптимальной доставке грузов транспортным средством с учетом зависимости
стоимости перевозок от загрузки транспортных средств\linebreak
\vspace*{-12pt}\\
\hspace*{23pt}по нескольким циклическим маршрутам\dotfill&4&53\\
\hangindent=23pt\noindent\textbf{Бунтман~Н.\,В., Зализняк~Анна~A., Зацман~И.\,M., Кружков~М.\,Г.,
Лощилова~Е.\,Ю., Сичинава~Д.\,В.} Информационные технологии корпусных
исследований: принципы\linebreak
\vspace*{-12pt}\\
\hspace*{23pt}построения кросслингвистических баз данных\dotfill&2&98\\
\textbf{Васильев~Н.\,С.} Использование принципа равновесия для управления
маршрутизацией\linebreak
\vspace*{-12pt}\\
\hspace*{23pt}в транспортных сетях\dotfill&1&28\\
\textbf{Вовченко~А.\,Е.} см.\ Калиниченко~Л.\,А.&&\\
\textbf{Вовченко~А.\,Е., Калиниченко~Л.\,А., Ковалев~Д.\,Ю.} Методы разрешения
сущностей\linebreak
\vspace*{-12pt}\\
\hspace*{23pt}и сли\-яния данных в ETL-процессе и их реализация в среде Hadoop\dotfill&4&94\\
\textbf{Галина~И.\,В.} см.\ Михеев~М.\,Ю.&&\\
\textbf{Гершкович~М.\,М.} см.\ Киреев~В.\,И.&&\\
\textbf{Горшенин~А.\,К.} Визуализация результатов для метода скользящего разделения
смесей %\dotfill
&4&78\\
\textbf{Грушо~А.\,А., Грушо~Н.\,А., Тимонина~Е.\,Е.} Анализ меток в скрытых
каналах\dotfill&4&41\\
\textbf{Грушо~А.\,А., Грушо~Н.\,А., Тимонина~Е.\,Е.} Включение новых запретов в
случайные по-\linebreak
\vspace*{-12pt}\\
\hspace*{23pt}следовательности\dotfill&4&46\\
\textbf{Грушо~Н.\,А.} см.\ Грушо~А.\,А.&&\\
\textbf{Грушо~Н.\,А.} см.\ Грушо~А.\,А.&&\\
\textbf{Де~Турк~K.} см.\ Морозов~Е.\,В.&&\\
\textbf{Драницына~М.\,А.} см.\ Бенинг~В.\,Е.&&\\
\textbf{Дьяченко~Ю.\,Г.} см.\ Соколов~И.\,А.&&\\
\hangindent=23pt\noindent\textbf{Ерошенко~А.\,А., Шестаков~О.\,В.} Асимптотические свойства оценки риска в
задаче восстановления изображения с коррелированным  шумом при обращении
пре-\linebreak
\vspace*{-12pt}\\
\hspace*{23pt}образования Радона\dotfill&4&32\\
\textbf{Ерошенко~А.\,А., Шестаков~О.\,В.} Асимптотические свойства оценки риска при
порого-\linebreak
\vspace*{-12pt}\\
\hspace*{23pt}вой обработке вейвлет-коэффициентов в модели с коррелированным
шумом\dotfill&1&36\\
\end{tabular}
}

\pagebreak

\def\leftkol{АВТОРСКИЙ УКАЗАТЕЛЬ ЗА 2014 г.} % ENGLISH ABSTRACTS}

\def\rightkol{АВТОРСКИЙ УКАЗАТЕЛЬ ЗА 2014 г.} %ENGLISH ABSTRACTS}

%\thispagestyle{myheadings}
\def\leftfootline{\small{\textbf{\thepage}
\hfill ИНФОРМАТИКА И ЕЁ ПРИМЕНЕНИЯ\ \ \ том~8\ \ \ выпуск~4\ \ \ 2014}
}%
 \def\rightfootline{\small{ИНФОРМАТИКА И ЕЁ ПРИМЕНЕНИЯ\ \ \ том~8\ \ \ выпуск~4\ \ \ 2014
 \hfill \textbf{\thepage}}}


{\tabcolsep=3pt
\begin{tabular}{p{388pt}rr}
&\textbf{Выпуск} & \textbf{Стр.}\\[3pt]
\textbf{Жаворонкова~Ю.\,В., Кудрявцев~А.\,А., Шоргин~С.\,Я.} Байесовская рекуррентная
модель\linebreak
\vspace*{-12pt}\\
\hspace*{23pt}роста надежности: бета-распределение параметров\dotfill&2&48\\
\textbf{Зализняк~Анна~A.} см.\ Бунтман~Н.\,В.&&\\
\textbf{Захаров~В.\,Н.} см.\ Соколов~И.\,А.&&\\
\textbf{Захарова~Т.\,В.} см.\ Бенинг~В.\,Е.&&\\
\textbf{Зацаринный~А.\,А., Чупраков~К.\,Г.} Об эргономических зависимостях между
параметрами\linebreak
\vspace*{-12pt}\\
\hspace*{23pt}ситуационного зала с использованием изогнутого коллективного
экрана\dotfill&4&85\\
\textbf{Зацаринный~А.\,А., Шабанов~А.\,П.} Аналитические аспекты оценки
эффективности\linebreak
\vspace*{-12pt}\\
\hspace*{23pt}в технологии поддержки деятельности организационной системы\dotfill&3&126\\
\textbf{Зацман~И.\,M.} см.\ Бунтман~Н.\,В.&&\\
\textbf{Зацман~И.\,М.} см.\ Минин~В.\,А.&&\\
\textbf{Зейфман~А.\,И., Королев~В.\,Ю., Коротышева~А.\,В., Шоргин~С.\,Я.} Общие
оценки устой-\linebreak
\vspace*{-12pt}\\
\hspace*{23pt}чивости для нестационарных марковских цепей с непрерывным
временем\dotfill&1&106\\
\hangindent=23pt\noindent\textbf{Зейфман~А.\,И., Коротышева~А.\,В., Киселева~К.\,М., Королев~В.\,Ю.,
Шоргин~С.\,Я.}\linebreak Об~оценках скорости сходимости и устойчивости для некоторых моделей
мас\-со-\linebreak
\vspace*{-12pt}\\
\hspace*{23pt}во\-го обслуживания\dotfill&3&19\\
\textbf{Зелёв~П.\,А.} см.\ Бронштейн~Е.\,М.&&\\
\textbf{Золотарев~О.\,В.} см.\ Михеев~М.\,Ю.&&\\
\textbf{Зыкин~С.\,В.} Динамические контексты базы данных реляционного типа\dotfill&1&77\\
\textbf{Зыкова~З.\,П.} см.\ Архипов~О.\,П.&&\\
\textbf{Калиниченко~Л.\,А.} см.\ Вовченко~А.\,Е.&&\\
\textbf{Калиниченко~Л.\,А., Ступников~С.\,А., Вовченко~А.\,Е., Ковалев~Д.\,Ю.}
Концептуальное\linebreak
\vspace*{-12pt}\\
\hspace*{23pt}моделирование мультидиалектных потоков работ\dotfill&4&110\\
\textbf{Кантор~О.\,Г., Спивак~С.\,И.} Построение моделей системной динамики в
условиях\linebreak
\vspace*{-12pt}\\
\hspace*{23pt}ограниченной экспертной информации\dotfill&2&111\\
\textbf{Карпов~П.\,И.} см.\ Бенинг~В.\,Е.&&\\
\hangindent=23pt\noindent\textbf{Киреев~В.\,И., Гершкович~М.\,М., Бирюкова~Т.\,К.} Об аппроксимации и
сходимости од-\linebreak
\vspace*{-12pt}\\
\hspace*{23pt}номерных параболических интегродифференциальных многочленов и
сплайнов %\dotfill
&1&118\\
\textbf{Киселева~К.\,М.} см.\ Зейфман~А.\,И.&&\\
\textbf{Ковалев~Д.\,Ю.} см.\ Вовченко~А.\,Е.&&\\
\textbf{Ковалев~Д.\,Ю.} см.\ Калиниченко~Л.\,А.&&\\
\textbf{Козеренко~Е.\,Б.} Интегральное моделирование языковых структур в
лингвистических\linebreak
\vspace*{-12pt}\\
\hspace*{23pt}процессорах систем обработки знаний и машинного перевода\dotfill&1&89\\
\textbf{Козеренко~Е.\,Б.} см.\ Михеев~М.\,Ю.&&\\
\textbf{Королев~В.\,Ю.} см.\ Зейфман~А.\,И.&&\\
\textbf{Королев~В.\,Ю.} см.\ Зейфман~А.\,И.&&\\
\textbf{Королев~В.\,Ю., Корчагин~А.\,Ю.} Модифицированный сеточный метод
разделения дис-\linebreak
\vspace*{-12pt}\\
\hspace*{23pt}пер\-си\-он\-но-сдвиговых смесей нормальных законов\dotfill&4&11\\
\textbf{Королев~В.\,Ю., Соколов~И.\,А.} Об условиях сходимости распределений
экстремальных\linebreak
\vspace*{-12pt}\\
\hspace*{23pt}порядковых статистик к распределению Вейбулла\dotfill&3&3\\
\textbf{Коротышева~А.\,В.} см.\ Зейфман~А.\,И.&&\\
\textbf{Коротышева~А.\,В.} см.\ Зейфман~А.\,И.&&\\
\textbf{Корчагин~А.\,Ю.} см.\ Королев~В.\,Ю.&&\\
\textbf{Кривенко~М.\,П.} Сравнительный анализ процедур регрессионного
анализа\dotfill&3&70\\
\textbf{Кружков~М.\,Г.} см.\ Бунтман~Н.\,В.&&\\
\textbf{Кудрявцев~А.\,А.} см.\ Жаворонкова~Ю.\,В.&&\\
\textbf{Кузнецов~Л.\,А.} Универсальная технология оценки близости информационных
объектов %\dotfill
&2&130\\
\textbf{Кульберг~Н.\,С.} см.\ Яковлева~Т.\,В.&&\\
\textbf{Леонтьев~Н.\,Д., Ушаков~В.\,Г.} Анализ системы обслуживания с входящим
потоком\linebreak
\vspace*{-12pt}\\
\hspace*{23pt}ав\-то\-ре\-грессионного типа\dotfill&3&39\\
\textbf{Лощилова~Е.\,Ю.} см.\ Бунтман~Н.\,В.&&\\
\textbf{Лукашенко~О.\,В., Морозов~Е.\,В., Пагано М.} Об асимптотике вероятности
переполнения\linebreak
\vspace*{-12pt}\\
\hspace*{23pt}гауссовской очереди\dotfill&2&28\\
\textbf{Лупенцов~О.\,С.} см.\ Маренко~В.\,А.&&\\
\end{tabular}
}

\pagebreak

\def\leftkol{АВТОРСКИЙ УКАЗАТЕЛЬ ЗА 2014 г.} % ENGLISH ABSTRACTS}

\def\rightkol{АВТОРСКИЙ УКАЗАТЕЛЬ ЗА 2014 г.} %ENGLISH ABSTRACTS}

%\thispagestyle{myheadings}
\def\leftfootline{\small{\textbf{\thepage}
\hfill ИНФОРМАТИКА И ЕЁ ПРИМЕНЕНИЯ\ \ \ том~8\ \ \ выпуск~4\ \ \ 2014}
}%
 \def\rightfootline{\small{ИНФОРМАТИКА И ЕЁ ПРИМЕНЕНИЯ\ \ \ том~8\ \ \ выпуск~4\ \ \ 2014
 \hfill \textbf{\thepage}}}


{\tabcolsep=3pt
\begin{tabular}{p{388pt}rr}
&\textbf{Выпуск} & \textbf{Стр.}\\[3pt]
\textbf{Лучко~О.\,Н.} см.\ Маренко~В.\,А.&&\\
\textbf{Малашенко~Ю.\,Е., Назарова~И.\,А.} Анализ задержек при диспетчеризации
однородных\linebreak
\vspace*{-12pt}\\
\hspace*{23pt}заданий в условиях неопределенности\dotfill&1&12\\
\textbf{Маньяков~Ю.\,А.} см.\ Архипов~О.\,П.&&\\
\textbf{Маньяков~Ю.\,А.} см.\ Архипов~О.\,П.&&\\
\textbf{Маренко~В.\,А., Лучко~О.\,Н., Лупенцов~О.\,С.} Разработка модели управления
процессом\linebreak
\vspace*{-12pt}\\
\hspace*{23pt}обучения с использованием когнитивных технологий\dotfill&1&99\\
\textbf{Мацкевич~А.\,Г.} Декларативные структуры знаний в проблемно-ориентированных
сис-\linebreak
\vspace*{-12pt}\\
\hspace*{23pt}те\-мах искусственного интеллекта\dotfill&2&122\\
\hangindent=23pt\noindent\textbf{Мейханаджян~Л.\,А., Милованова~Т.\,А., Печинкин~А.\,В., Разумчик~Р.\,В.}
Стационарные вероятности состояний в системе обслуживания с инверсионным порядком
об-\linebreak
\vspace*{-12pt}\\
\hspace*{23pt}слу\-жи\-ва\-ния и обобщенным вероятностным приоритетом\dotfill&3&28\\
\textbf{Милованова~Т.\,А.} см.\ Мейханаджян~Л.\,А.&&\\
\textbf{Минин~В.\,А., Зацман~И.\,М., Хавансков~В.\,А., Шубников~С.\,К.} Индикаторы
тематических\linebreak
\vspace*{-12pt}\\
\hspace*{23pt}взаимосвязей науки и технологий: от текста к числам\dotfill&3&114\\
\textbf{Миронов~А.\,М.} О сходимости распределений случайных сумм к скошенным
экс\-по\-нен-\linebreak
\vspace*{-12pt}\\
\hspace*{23pt}ци\-аль\-но-степенным законам\dotfill&2&55\\
\textbf{Миронов~А.\,М., Френкель~С.\,Л.} Метод повышения эффективности решения
задач\linebreak
\vspace*{-12pt}\\
\hspace*{23pt}вероятностной верификации вычислительных и телекоммуникационных
систем\dotfill&4&58\\
\hangindent=23pt\noindent\textbf{Михеев~М.\,Ю., Сомин~Н.\,В., Галина~И.\,В., 
Золотарев~О.\,В., Козеренко~Е.\,Б., Морозова~Ю.\,И., Шарнин~М.\,М.} Фальштексты: 
классификация и методы опознания\linebreak
\vspace*{-12pt}\\
\hspace*{23pt}текс\-то\-вых имитаций и документов с подменой авторства\dotfill&4&70\\
\textbf{Морозов~Е.\,В.} см.\ Лукашенко~О.\,В.&&\\
\textbf{Морозов~Е.\,В., Потахина~Л.\,В., Де~Турк~K.} Анализ устойчивости системы
передачи\linebreak
\vspace*{-12pt}\\
\hspace*{23pt}данных с~оптическими линиями задержки случайной длины\dotfill&1&127\\
\textbf{Морозова~Ю.\,И.} см.\ Михеев~М.\,Ю.&&\\
\textbf{Мотренко~А.\,П., Стрижов~В.\,В.} Построение агрегированных прогнозов объемов
желез-\linebreak
\vspace*{-12pt}\\
\hspace*{23pt}нодорожных грузоперевозок c использованием расстояния Кульбака--Лейблера %\dotfill
&2&86\\
\textbf{Назарова~И.\,А.} см.\ Малашенко~Ю.\,Е.&&\\
\textbf{Павлов~И.\,В.} Оценка надежности сложных систем с восстановлением по
результатам\linebreak
\vspace*{-12pt}\\
\hspace*{23pt}испытаний элементов\dotfill&1&21\\
\textbf{Пагано М.} см.\ Лукашенко~О.\,В.&&\\
\textbf{Печинкин~А.\,В.} см.\ Мейханаджян~Л.\,А.&&\\
\textbf{Печинкин~А.\,В., Разумчик~Р.\,В.} Система Geo/Geo/1/$R$ с гистерезисной
политикой\dotfill&2&15\\
\hangindent=23pt\noindent\textbf{Печинкин~А.\,В., Разумчик~Р.\,В.} Совместное стационарное распределение числа
заявок в накопителе и в бункере переупорядочения в многоканальной системе
об\-слу\-жи-\linebreak
\vspace*{-12pt}\\
\hspace*{23pt}ва\-ния с переупорядочением заявок\dotfill&4&3\\
\textbf{Плеханов~Л.\,П.} Проектирование самосинхронных схем: структурные методы в
иерар-\linebreak
\vspace*{-12pt}\\
\hspace*{23pt}хическом анализе\dotfill&3&105\\
\textbf{Потахина~Л.\,В.} см.\ Морозов~Е.\,В.&&\\
\textbf{Разумчик~Р.\,В.} см.\ Мейханаджян~Л.\,А.&&\\
\textbf{Разумчик~Р.\,В.} см.\ Печинкин~А.\,В.&&\\
\textbf{Разумчик~Р.\,В.} см.\ Печинкин~А.\,В.&&\\
\textbf{Рождественский~Ю.\,В.} см.\ Соколов~И.\,А.&&\\
\textbf{Синицын~В.\,И.} см.\ Синицын~И.\,Н.&&\\
\textbf{Синицын~И.\,Н.} Анализ и моделирование распределений в эредитарных
стохастических\linebreak
\vspace*{-12pt}\\
\hspace*{23pt}системах\dotfill&1&2\\
\hangindent=23pt\noindent\textbf{Синицын~И.\,Н.} Аналитическое моделирование распределений с инвариантной
мерой в~негауссовских дифференциальных и приводимых к ним эредитарных
стохасти-\linebreak
\vspace*{-12pt}\\
\hspace*{23pt}ческих системах\dotfill&2&2\\
\textbf{Синицын~И.\,Н., Синицын~В.\,И.} Аналитическое моделирование нормальных
процессов\linebreak
\vspace*{-12pt}\\
\hspace*{23pt}в стохастических системах со сложными нелинейностями\dotfill&3&12\\
\textbf{Сиротинин~Д.\,О.} см.\ Архипов~О.\,П.&&\\
\textbf{Сичинава~Д.\,В.} см.\ Бунтман~Н.\,В.&&\\
\end{tabular}
}

\pagebreak

\def\leftkol{АВТОРСКИЙ УКАЗАТЕЛЬ ЗА 2014 г.} % ENGLISH ABSTRACTS}

\def\rightkol{АВТОРСКИЙ УКАЗАТЕЛЬ ЗА 2014 г.} %ENGLISH ABSTRACTS}

%\thispagestyle{myheadings}
\def\leftfootline{\small{\textbf{\thepage}
\hfill ИНФОРМАТИКА И ЕЁ ПРИМЕНЕНИЯ\ \ \ том~8\ \ \ выпуск~4\ \ \ 2014}
}%
 \def\rightfootline{\small{ИНФОРМАТИКА И ЕЁ ПРИМЕНЕНИЯ\ \ \ том~8\ \ \ выпуск~4\ \ \ 2014
 \hfill \textbf{\thepage}}}


{\tabcolsep=3pt
\begin{tabular}{p{388pt}rr}
&\textbf{Выпуск} & \textbf{Стр.}\\[3pt]
\textbf{Соколов~И.\,А.} см.\ Королев~В.\,Ю.&&\\
\hangindent=23pt\noindent\textbf{Соколов~И.\,А., Степченков~Ю.\,А., Бобков~С.\,Г.,
Захаров~В.\,Н., Дьяченко~Ю.\,Г., Рож\-де\-ст\-вен\-ский~Ю.\,В., Сурков~А.\,В.}
Базис реализации супер-ЭВМ эксафлопсного\linebreak
\vspace*{-12pt}\\
\hspace*{23pt}класса\dotfill&1&45\\
\textbf{Сомин~Н.\,В.} см.\ Михеев~М.\,Ю.&&\\
\textbf{Сорокин~А.\,В.} Автоматизация за пределами WEB 2.0\dotfill&4&125\\
\textbf{Спивак~С.\,И.} см.\ Кантор~О.\,Г.&&\\
\textbf{Степченков~Ю.\,А.} см.\ Соколов~И.\,А.&&\\
\textbf{Стрижов~В.\,В.} см.\ Мотренко~А.\,П.&&\\
\textbf{Ступников~С.\,А.} см.\ Калиниченко~Л.\,А.&&\\
\textbf{Сурков~А.\,В.} см.\ Соколов~И.\,А.&&\\
\textbf{Тимонина~Е.\,Е.} см.\ Грушо~А.\,А.&&\\
\textbf{Тимонина~Е.\,Е.} см.\ Грушо~А.\,А.&&\\
\textbf{Ушаков~В.\,Г.} см.\ Леонтьев~Н.\,Д.&&\\
\textbf{Френкель~С.\,Л.} см.\ Миронов~А.\,М.&&\\
\textbf{Хавансков~В.\,А.} см.\ Минин~В.\,А.&&\\
\textbf{Черток~А.\,В.} О формализации понятия токсичности потока заявок на
финансовых\linebreak
\vspace*{-12pt}\\
\hspace*{23pt}рынках\dotfill&4&20\\
\textbf{Чупраков~К.\,Г.} см.\ Зацаринный~А.\,А.&&\\
\textbf{Шабанов~А.\,П.} см.\ Зацаринный~А.\,А.&&\\
\textbf{Шарнин~М.\,М.} см.\ Михеев~М.\,Ю.&&\\
\textbf{Шестаков~О.\,В.} см.\ Ерошенко~А.\,А.&&\\
\textbf{Шестаков~О.\,В.} см.\ Ерошенко~А.\,А.&&\\
\textbf{Шоргин~С.\,Я.} см.\ Жаворонкова~Ю.\,В.&&\\
\textbf{Шоргин~С.\,Я.} см.\ Зейфман~А.\,И.&&\\
\textbf{Шоргин~С.\,Я.} см.\ Зейфман~А.\,И.&&\\
\textbf{Шубников~С.\,К.} см.\ Минин~В.\,А.&&\\
\textbf{Яковлева~Т.\,В., Кульберг~Н.\,С.} Методы математической статистики как
инструмент\linebreak
\vspace*{-12pt}\\
\hspace*{23pt}двухпараметрического анализа магнитно-резонансного изображения\dotfill&3&79\\
\end{tabular}
}

%\thispagestyle{myheadings}
\def\leftfootline{\small{\textbf{\thepage}
\hfill ИНФОРМАТИКА И ЕЁ ПРИМЕНЕНИЯ\ \ \ том~8\ \ \ выпуск~4\ \ \ 2014}
}%
 \def\rightfootline{\small{ИНФОРМАТИКА И ЕЁ ПРИМЕНЕНИЯ\ \ \ том~8\ \ \ выпуск~4\ \ \ 2014
 \hfill \textbf{\thepage}}}

 \label{end\stat}