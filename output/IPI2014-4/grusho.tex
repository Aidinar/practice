\def\stat{grusho}



\def\tit{АНАЛИЗ МЕТОК В~СКРЫТЫХ КАНАЛАХ$^*$}



\def\titkol{Анализ меток в~скрытых каналах}

\def\aut{А.\,А. Грушо$^1$, Н.\,А. Грушо$^2$, Е.\,Е. Тимонина$^3$}

\def\autkol{А.\,А. Грушо, Н.\,А. Грушо, Е.\,Е. Тимонина}

\titel{\tit}{\aut}{\autkol}{\titkol}

{\renewcommand{\thefootnote}{\fnsymbol{footnote}} \footnotetext[1]
{Работа частично поддержана РФФИ (проект 13-01-00215).}}


\renewcommand{\thefootnote}{\arabic{footnote}}
\footnotetext[1]{Институт проблем информатики Российской академии наук;
факультет вычислительной математики
и~кибернетики Московского государственного университета им.\ М.\,В. Ломоносова,
grusho@yandex.ru}
\footnotetext[2]{Институт проблем информатики Российской академии наук, info@itake.ru}
\footnotetext[3]{Институт проблем информатики Российской академии наук, eltimon@yandex.ru}


  \Abst{Рассматривается класс скрытых каналов, построенных на
основе меток. Предполагается, что выявление скрытого канала ведется
контролером исключительно статистическими методами. Это значит, что редко
встречающиеся и~часто встречающиеся лингвистические конструкции для
контролера неразличимы. Для него важно, чтобы передаваемая по каналу
последовательность символов не содержала запретов, не соответствующих
вероятностной модели легальных сообщений. Основная проблема при
обеспечении невидимости таких каналов состоит в~том, что при встраивании
меток могут возникать запреты вероятностной меры, описывающей легальную
передачу. В~работе предложен метод построения меток, которые не могут
выявляться контролером. Благодаря такому построению меток скрытый канал
невидим.}

\KW{скрытые каналы; информационная безопасность; метки,
порождающие скрытый канал; невидимость меток; математические модели
скрытых каналов}

\DOI{10.14357/19922264140405}


\vskip 12pt plus 9pt minus 6pt

\thispagestyle{headings}

\begin{multicols}{2}

\label{st\stat}

\section{Введение}

   В~работе рассматривается класс скрытых каналов, порожденных метками~[1,~2].
   В~дальнейшем этот класс скрытых каналов будем называть классом~$A$.

  Скрытые каналы класса~$A$ устроены сле\-ду\-ющим образом. На приемном
и~передающем концах скрытого канала существуют синхронно ра\-бо\-та\-ющие
счетчики, которые фиксируют условное время. Чаще всего это время
вычисляется в~длинах сообщений между метками. Каждое сообщение
определяется двумя метками~--- началом и~концом сообщения, а~длина
сообщения между метками, или условное время, является кодом передаваемого
сообщения. Метки должны однозначно идентифицироваться на приемном конце
скрытого канала, в~противном случае может возникнуть шум, т.\,е.\ сбой при
приеме сообщения. С~другой стороны, хорошо заметные метки являются
признаком работы скрытого канала. Поэтому метки должны быть сделаны
максимально невидимыми для контролирующего передачу субъекта.

  Таким образом, каналы класса~$A$ предъявляют противоречивые требования
к~своему формированию: метки должны быть однозначно опре\-де\-ля\-емы\-ми
и~невидимыми. В~работах~[1, 2] рассмотрены примеры разрешения
указанного противоречия. Так, в~работе [1] предлагается метод передачи меток
по ненаблюдаемому, синхронно работающему каналу. Если само сообщение
представляет собой последовательность битов, то метки являются изменением
формы передаваемого сигнала, которое в~реально работающих технических
устройствах не наблюдается. Однако при обработке сигналов в~процессоре, где
находится агент (про\-грам\-мно-ап\-па\-рат\-ная сущность, присутствующая
в~микропроцессоре и~являющаяся отправителем или получателем в~работе
скрытого канала) получателя сообщения, изменение формы сигнала легко
обнаруживается на физическом уровне.

  В работе~\cite{2-gr} приведены примеры других способов построения
скрытых каналов класса~$A$ в~глобальной сети, основанных на скрытой
синхронизации легальных каналов передачи сообщений. \mbox{Кроме}~того, для
сообщений, которые хорошо моделируются случайной равновероятной
последовательностью, в~этой работе описан метод построения невидимых
скрытых меток.

  В данной работе проводится исследование возможности построения меток
для скрытых каналов, использующих модели сообщений, которые описываются
случайными процессами, не являющимися случайными равновероятными
последовательностями.

  Поясним постановку задачи. В~работе~[3] показано, что даже при
моделировании метки случайным процессом, полностью соответствующим
модели передаваемого легального сообщения, возникают запреты~\cite{4-gr, 5-gr},
однозначно выявляющие скрытый канал. При этом метки, определяющие
сообщение из скрытого канала класса~$A$, являются вставками в~легальное
сообщение. Запреты возникают в~силу наличия зависимостей между
элементами случайной последовательности, моделирующей легальное
сообщение. Метка в~форме вставки в~случайную последовательность может
нарушить эти зависимости, порождая запрет в~случайной последовательности,
который не может присутствовать в~реальной реализации легального
сообщения.

  Основной результат работы заключается в~демонстрации того, как
согласовать статистические характеристики случайного процесса,
мо\-де\-ли\-ру\-юще\-го легальное сообщение, и~свойства метки, которая является
ключом, известным на приемном и~передающем концах канала. Показано, что
контролер, который реализует поиск запретов в~случайной последовательности,
находится в~худшем положении, чем организатор скрытого канала, по
двум причинам. Первая причина состоит в~том, что организатор скрытого
канала традиционно предполагается знающим алгоритмы контролера. Вторая
причина кроется в~том, что контролер для выявления скрытого канала
вынужден использовать более глубокие зависимости легальных сообщений, чем
использует организатор скрытого канала.

%\vspace*{-6pt}

\section{Математическая модель скрытого канала}

  Пусть $X=\{x_1, \ldots ,x_r\}$~--- конечное множество, $X^n$~--- декартово
произведение~$X$,  $X^\infty$~--- множество всех последовательностей
с~элементами из~$X$. Пусть $\mathcal{A}$~--- это $\sigma$-ал\-геб\-ра
на~$X^\infty$, порожденная цилиндрическими множествами; $\mathcal{A}$
также является борелевской $\sigma$-ал\-геб\-рой в~тихоновском
произведении~$X^\infty$, где $X$ имеет дискретную топологию~\cite{6-gr, 7-gr}.

  На $(X^\infty, \mathcal{A})$ определена вероятностная мера~$P$.
Предположим, что $P_n$ является проекцией меры~$P$ на первые $n$
координат последовательностей из~$X^\infty$.

  \textit{Запретом}~\cite{4-gr, 5-gr} в~мере~$P_n$ называется вектор
$\overline{x}_l\hm\in X^l$, $l\hm\leq n$, такой что
  $$
  P_n\left( \overline{x}_l\times X^{n-l}\right)=0\,.
  $$

  Пусть $\overline{x}_l\in X^l$~--- запрет, а $\tilde{x}_{l-1}$ получена
из~$\overline{x}_l$ отбрасыванием последней координаты. Если $P_{l-
1}\left(\tilde{x}_{l-1}\right)\hm>0$, то~$\overline{x}_l$ называется
\textit{наименьшим запретом}~\cite{4-gr, 5-gr}.

  Если~$\overline{x}_l$ является запретом в~мере~$P_n$, то для любых
$l\hm\leq s\hm\leq n$ и~для любых векторов~$\overline{x}_s$, начинающихся
с~$\overline{x}_l$, имеем
  $$
  P_s\left( \overline{x}_s\right)=0\,.
  $$

  Пусть $\overline{x}_s\hm\in X^s$ и~для любого $n\hm\geq s$ и~любого
$\overline{x}_n\hm\in X^n$, если $\overline{x}_s$ является
частью~$\overline{x}_n$, $P_n(\overline{x}_n)\hm=0$.

   Вектор $\overline{x}_s$ называется \textit{минимальным запретом}~[8], если
любой вектор~$\overline{x}_n$, $P_n(\overline{x}_n)\hm>0$, после
приписывания в~конце недостающих знаков из вектора~$\overline{x}_s$
становится наименьшим запретом.

  Меру~$P$ будем называть моделью легального сообщения. Далее будем
предполагать, что все запреты меры~$P$ определяются минимальными
запретами и~количество минимальных запретов конечно.

%\begin{figure*} %fig1
\vspace*{6pt}
 \begin{center}
 \mbox{%
 \epsfxsize=74.148mm
 \epsfbox{gr1-1.eps}
 }

\vspace*{3pt}

 {\small{Модель скрытого канала}}
  \end{center}

  \vspace*{9pt}

%\end{figure*}

  Рассмотрим следующую модель скрытого канала (см.\ рисунок).
Компьютеры~A и~B связаны однонаправленным каналом от~A к~B. Выходная
последовательность компьютера~А проходит через \mbox{агента}~A$^\prime$
  (про\-грам\-мно-ап\-па\-рат\-ную сущность, функционирующую
в~компьютерной среде незаметно для пользователей и~средств защиты),
который строит скрытый канал и~передает по нему вместе с~легальной
информацией имеющуюся у~него скрываемую информацию. На приемном
конце информация из канала прежде всего попадает агенту~B$^\prime$,
который распознает скрытый канал, считывает скрытую информацию,
уничтожает следы скрытого канала и~передает компьютеру~B легальную
информацию. U~--- контролер передачи в~канале.



  Будем считать, что как противник, так и~контролер, ищущий скрытые каналы,
могут изучать свойства~$P$ статистическими методами.
Противник\,--\,от\-пра\-ви\-тель сообщения и~про\-тив\-ник\,--\,по\-лу\-ча\-тель сообщения имеют общие
ключи в~виде множества меток и~некоторых ограничений на места их
появления. Метки представляют собой набор векторов в~алфавите~$X$,
имеющих для простоты одну длину~$k$. Сообщения передаются кодом,
который представляется длинами легального сообщения между двумя метками.
Метка начала скрытого сообщения вставляется в~легальное сообщение на одно
из допустимых мест, определяемых ключом. Метка окончания скрытого
сообщения ставится в~фиксированном месте легального сообщения на
расстоянии от последнего символа входной метки так, чтобы расстояние между
метками соответствовало коду.

  Допускается, что векторы, соответствующие меткам, могут появляться
в~легальном сообщении случайно. Это может привести к сбою скрытой
передачи. В~связи с этим на метки накладываются следующие ограничения:
  \begin{itemize}
\item вероятность случайного появления метки должна быть как можно
меньше;\\[-14pt]
\item метки должны выбираться таким образом, чтобы не допустить
появления запретов вероятностной меры~$P$.
\end{itemize}

\vspace*{-10pt}

\section{Построение меток для~невидимого скрытого канала}

  Предположим, что все скрываемые сообщения имеют конечную длину не
более~$N$. Это значит, что начиная с~некоторой длины $N_0\hm>N$ (с~\mbox{учетом}
ограничений на начало скрываемого сообщения) \mbox{случайная} последовательность,
соответствующая легальному сообщению, будет строиться в~соответствии
с~вероятностным распределением, аналогичным исходному процессу~$P$.

  Однако это не означает, что скрытый канал невозможно выявить. Если
вставки порождают запреты вероятностной меры~$P$~[4, 5], то скрытый канал
выявляется однозначно. Значит, вероятность выявления скрытого канала равна
вероятности появления запрета меры~$P_{N_0}$. Вероятность случайного
появления метки длины~$k$ не меньше вероятности появления данного вектора
в~мере~$P_N$. За исключением вырожденных случаев, можно считать, что
с~помощью выбора~$k$ эта вероятность может быть сделана как угодно малой
и~отвечает требованиям помехозащищенности скрытого канала.

  Для простоты рассмотрим возможность скрытия одной метки. Обозначим ее
через вектор $\overline{b}\hm= (b_1, \ldots , b_k)$, а~место ее вставки~--- $t$.
Пусть минимальные запреты меры~$P$ представимы следующими векторами:
  \begin{equation}
  \left.
  \begin{array}{l}
  \overline{a}^{(1)} = \left( a^{(1)}_{i_1},\ldots a^{(1)}_{i_{m_1}}\right)\,;\\
  \cdots\cdots\cdots\cdots\cdots\cdots\cdots\\
  \overline{a}^{(s)} = \left( a^{(s)}_{i_1},\ldots a^{(s)}_{i_{m_s}}\right)\,,
  \end{array}
  \right\}
  \label{e1-gr}
  \end{equation}
где все $a_{i_j}\in X$. Для простоты далее положим $m_i\hm=m$, $i\hm=
1,\ldots , s$.

  Определим возможности контролера~U. Контролер~U знает
распределение~$P$, знает все запреты меры~$P$ и~может их отслеживать
в~реальном вре\-мени. Контролер не знает вставку~$\overline{b}$ и~место ее\linebreak
расположения~$t$. Контролер~U не анализирует\linebreak семантику, поэтому для него
маловероятные и~высоковероятные последовательности одинаково являются
допустимыми. Исходя из этих предположений, контролер~U выявляет скрытый
канал с~по\-мощью нахождения минимальных запретов в~передаваемой
последовательности.

  Пусть $\overline{x}=x(1),x(2),\ldots, x(t-1),x(t),\ldots)$~--- легальная
последовательность, а~$\overline{x}^\prime\hm= (x(1),x(2),\ldots$\linebreak
$\ldots, x(t- 1),b_1,\ldots , b_k, x(t),\ldots)$~--- последовательность со вставкой. Возможность
по\-яв\-ле\-ния запрета в~последовательности~$\overline{x}^\prime$ определяется
тем, что хотя бы один из векторов
  \begin{equation}
  \left.
  \begin{array}{l}
  x(t-k+1), x(t-k+2),\ldots , x(t-1),b_1\,;\\[-1pt]
  x(t-k+2), x(t-k+3),\ldots , x(t-1),b_1, b_2\,;\\[-1pt]
\cdots\cdots\cdots\cdots\cdots\cdots\cdots\cdots\cdots\cdots\cdots\cdots\cdots\cdots\\[-1pt]
  b_1,b_2,\ldots , b_k\,;\\[-1pt]
\cdots\cdots\cdots\cdots\cdots\cdots\cdots\cdots\cdots\cdots\cdots\cdots\cdots\cdots\\[-1pt]
  b_k,x(t),x(t+1),\ldots , x(t+k-1)
  \end{array}
  \right\}\!
  \label{e2-gr}
  \end{equation}

  \vspace*{-2pt}

  \noindent
принадлежит~(1). В~других местах последовательности~$\overline{x}^\prime$
появление запретов невозможно по определению меры~$P$. Если
в~последовательности~$\overline{x}^\prime$ не появилось запретов, то любой
начальный учас\-ток этой последовательности является допустимым вектором
и~его вероятность больше~0 в~мере~$P_N$. Поэтому контролер~U не может
выявить скрытый канал. Отсюда следует следующее

  \smallskip

  \noindent
  \textbf{Утверждение~1.} Контролер~U не выявляет скрытый канал тогда
и~только тогда, когда ни один из векторов~(2) не содержится в~множестве
векторов~(1).

  \smallskip

  Таким образом, для организации невидимого скрытого канала
метка~$\overline{b}$ должна быть выбрана так, чтобы выполнялось
достаточное условие утверждения~1 при любом допустимом~$\overline{x}$.
При этом организатор скрытого канала может определить минимальные
запреты только статистически, наблюдая за передаваемыми
последовательностями. Для этого мера~$P$ должна удовлетворять
дополнительным ограничениям. Например, случайная
последовательность~$\overline{x}$ должна быть эргодической.

  Но даже если организатор скрытого канала знает минимальные запреты,
построение метки, удовле\-тво\-ря\-ющей утверждению~1 для всех~$\overline{x}$,
является сложной задачей.

  Однако можно предложить упрощенный метод решения поставленной задачи
построения метки для невидимого скрытого канала. Пусть организатор
скрытого канала нашел минимальные запреты. Рассмотрим множество~$C$
биграмм вида
  \begin{multline*}
  C=\left\{ \left( a^{(i)}_{i_{m-1}}, a^{(j)}_{i_m}\right),\
  j=\overline{1,s}\right\}\bigcup{}\\
  {}\bigcup \left\{ \left( a^{(j)}_{i_1}, a^{(j)}_{i_2}\right),\
  j=\overline{1,s}\right\}\,.
%  \label{e3-gr}
  \end{multline*}

  Тогда справедливо следующее

  \smallskip

  \noindent
  \textbf{Утверждение~2.} Если для любой
последовательности~$\overline{x}$ в~векторе $(x(t-1),b_1,b_2,\ldots ,b_k,x(t))$
нет биграмм из множества~$C$, то скрытый канал не выявляем.

  \smallskip

  \noindent
  Д\,о\,к\,а\,з\,а\,т\,е\,л\,ь\,с\,т\,в\,о\,.\ \ Если в~векторе $(x(t\hm-
1),b_1,b_2,\ldots ,b_k,x(t))$ не встречаются биграммы из множества~$C$, то
в~последовательности~$\overline{x}^\prime$ не встречаются запреты из
множества~(1). В~самом деле, появление запрета влечет появление биграмм из
множества~$C$. Если вставка не порождает запретов, то согласно
утверждению~1 контролер не выявляет скрытого канала. Утверждение~2
доказано.

  \smallskip

  Однако создание метки, исключающей появление биграмм из множества~$C$
для всех последовательностей~$\overline{x}$, может оказаться невозможным.
В~этом случае метку можно сделать сложной, а~именно: метка~$\overline{b}$
не должна содержать запрещенных биграмм из множества~$C$. Для
недопущения появления запрещенной биграммы на концах вставки можно
ввести дополнительные символы до и~после метки, которые не несут
содержательной информации, но исключают возможность появления
запрещенных биграмм на концах вставки. Эти незначащие символы
выбираются для каждой конкретной последовательности~$\overline{x}$.

  Предположим, что организатор скрытого канала заранее скрытно выбрал
момент~$t$. Затем он статистически нашел все наименьшие запреты
в~окрестности этого момента времени, используя различные реализации
передаваемых сообщений. Тогда, используя описанный метод перехода
к~биграммам, можно построить метку, которая не порождает наименьших
запретов, связанных со вставкой метки в~момент~$t$. Такой канал будет
невидим для~U.

\vspace*{-6pt}

\section{Заключение}

  Интенсивное использование многоагентных систем для скрытого контроля
компьютерной среды требует разработки невидимых скрытых каналов для
организации коллективной деятельности агентов. Уязвимость простых
протоколов общения агентов может привести к полной нейтрализации систем
контроля. Поэтому необходимо строить простые, доказано невидимые скрытые
каналы. Рассмотренные в~статье скрытые каналы просты, но могут легко
выявляться при неправильной организации меток. Предложенный в~работе
метод позволяет делать метки невидимыми, когда их поиск ведется
статистическими методами.

\vspace*{-6pt}

{\small\frenchspacing
 {%\baselineskip=10.8pt
 \addcontentsline{toc}{section}{References}
 \begin{thebibliography}{9}
    \bibitem{1-gr}
    \Au{Грушо Н.} Скрытые каналы, основанные на метках~// Системы
и~средства информатики, 2013. Т.~23. №\,1. С.~7--13.
    \bibitem{2-gr}
    \Au{Грушо А., Грушо Н., Тимонина~Е.} Скрытые каналы, порожденные
метками, в~дейтаграммах~// Системы и~средства информатики, 2013. Т.~23.
№\,2. С.~3--18.
    \bibitem{3-gr}
    \Au{Grusho A., Grusho N., Timonina~E.} Problems of modeling in the analysis
of covert channels~// Computer network security~/
Eds.\ I.~Kotenko, V.~Skormin. Lecture notes in computer
science ser.~--- Berlin--Heidelberg: Springer-Verlag, 2010. Vol.~6258. P.~118--124. doi: 10.1007/978-3-642-14706-7\_9.
    \bibitem{4-gr}
    \Au{Грушо А., Тимонина Е.} Запреты в~дискретных ве\-ро\-ят\-но\-ст\-но-ста\-ти\-сти\-че\-ских задачах~// Дискретная математика, 2011. Т.~23. Вып.~2.
    С.~53--58.
    \bibitem{5-gr}
    \Au{Grusho A., Grusho N., Timonina~E.} Consistent sequences of tests defined
by bans~// Springer proceedings in mathematics \& statistics, optimization theory,
decision making, and operation research applications.~---
New York\,--\,Heidelberg\,--\,Dordrecht\,--\,London: Springer, 2013. \mbox{P.~281--291}.
    \bibitem{6-gr}
    \Au{Бурбаки Н.} Общая топология. Основные структуры~/ Пер.
    с~франц.~--- М.: Наука, 1968. 272~с. (\Au{Bourbaki N.} Topologie
G$\acute{\mbox{e}}$n$\acute{\mbox{e}}$rale. Chapitre~1: Structures
topologiques. Chapitre~2: Structures uniformes.~--- Paris: Hermann, 1940. 129~p.)
    \bibitem{7-gr}
    \Au{Прохоров Ю.\,В., Розанов Ю.\,А.} Теория вероятностей.~--- 2-е изд.~---
    М.: Наука, 1973. 494~с.
    \bibitem{8-gr}
    \Au{Грушо А., Грушо Н., Тимонина~Е.} Статистические методы
определения запретов вероятностных мер на дискретных пространствах~//
Информатика и~её применения, 2013. Т.~7. Вып.~1. С.~54--57.

 \end{thebibliography}

 }
 }

\end{multicols}

\vspace*{-12pt}

\hfill{\small\textit{Поступила в редакцию 28.09.14}}

%\newpage

\vspace*{10pt}

\hrule

\vspace*{2pt}

\hrule

%\vspace*{12pt}

\def\tit{THE ANALYSIS OF~TAGS IN~COVERT CHANNELS}

\def\titkol{The analysis of~tags in~covert channels}

\def\aut{A.\,A.~Grusho$^{1,2}$, N.\,A.~Grusho$^1$, and E.\,E.~Timonina$^1$}

\def\autkol{A.\,A.~Grusho, N.\,A.~Grusho, and E.\,E.~Timonina}

\titel{\tit}{\aut}{\autkol}{\titkol}

\vspace*{-9pt}


\noindent
$^1$Institute of Informatics Problems, Russian Academy of Sciences,
44-2~Vavilov Str., Moscow 119333, Russian\linebreak
$\hphantom{^1}$Federation

\noindent
$^2$Faculty of Computational Mathematics and Cybernetics,
M.\,V.~Lomonosov Moscow State University,\linebreak
$\hphantom{^1}$1-52 Leninskiye Gory, GSP-1, Moscow 119991, Russian Federation


\def\leftfootline{\small{\textbf{\thepage}
\hfill INFORMATIKA I EE PRIMENENIYA~--- INFORMATICS AND
APPLICATIONS\ \ \ 2014\ \ \ volume~8\ \ \ issue\ 4}
}%
 \def\rightfootline{\small{INFORMATIKA I EE PRIMENENIYA~---
INFORMATICS AND APPLICATIONS\ \ \ 2014\ \ \ volume~8\ \ \ issue\ 4
\hfill \textbf{\thepage}}}

\vspace*{3pt}



\Abste{The class of covert channels constructed
on the basis of tags is considered.  It is supposed that
a~covert channel is detected by the control subject using exclusively statistical
techniques. It means that linguistic constructions\linebreak\vspace*{-12pt}}

\Abstend{seldom or often met
are indiscernible for the control subject.  For control subject, it is important
that the sequence transmitted through the channel does not contain the bans which
do not correspond to the~probability model of legal messages.  The main problem
for supporting invisibility of such channels consists in the fact that
there can be bans of the probability measure describing a~legal transmission
when embedding
tags.
 In the paper, the method for creating tags which cannot be detected by the
 control subject becomes suggested.
Thanks to such creation of tags, the covert channel becomes invisible.}

\KWE{covert channels; information security; covert channel generated by tags;
``invisibility''  of tags; mathematical models of covert channels}

\DOI{10.14357/19922264140405}

\Ack
\noindent
The paper was partly supported by the Russian Foundation
for Basic Research  (project 13-01-00215).

%\vspace*{3pt}

  \begin{multicols}{2}

\renewcommand{\bibname}{\protect\rmfamily References}
%\renewcommand{\bibname}{\large\protect\rm References}



{\small\frenchspacing
 {%\baselineskip=10.8pt
 \addcontentsline{toc}{section}{References}
 \begin{thebibliography}{9}
\bibitem{1-gr-1}
\Aue{Grusho, N.} 2013. Skrytye kanaly, osnovannye na metkakh
[Covert channels generated by tags].  \textit{Sistemy i~Sredstva Informatiki}~---
\textit{Systems and Means of Informatics} 23(1):7--13.
\bibitem{2-gr-1}
\Aue{Grusho, A., N.~Grusho, and E.~Timonina}.  2013.
Skrytye kanaly, porozhdennye metkami, v~deytagrammakh [Covert channels
generated by tags in datagrams]. \textit{Sistemy i~Sredstva Informatiki}~---
\textit{Systems and Means of Informatics} 23(2):3--18.
\bibitem{3-gr-1}
\Aue{Grusho,~A., N.~Grusho, and E.~Timonina}. 2010. Problems of modeling
in the analysis of covert channels.
\textit{Computer network security}. Eds. I.~Kotenko and V.~Skormin.
Lecture notes in computer science ser.
Berlin--Heidelberg: Springer-Verlag. 6258:118--124.
doi: 10.1007/978-3-642-14706-7\_9.
\bibitem{4-gr-1}
\Aue{Grusho, A., and E.~Timonina}. 2011. Prohibitions in discrete
probabilistic statistical problems.
\textit{Discrete Mathematics Application} 21(3):275--281.
\bibitem{5-gr-1}
\Aue{Grusho, A., N.~Grusho, and E.~Timonina}. 2013. Consistent sequences
of tests defined by bans.
\textit{Springer proceedings in mathematics \& statistics, optimization theory,
decision making, and operation research applications}.
New York\,--\,Heidelberg\,--\,Dordrecht\,--\,London: Springer. 281--291.

\smallskip

\bibitem{6-gr-1}
\Aue{Bourbaki, N.}
1940. \textit{Topologie G$\acute{\mbox{e}}$n$\acute{\mbox{e}}$rale}.
Chapitre~1: Structures topologiques. Chapitre~2: Structures uniformes.
Paris: Hermann. 129~p.

\smallskip

\bibitem{7-gr-1}
\Aue{Prokhorov, U.\,V., and U.\,A.~Rozanov}. 1973. Teoriya veroyatnostey
[Theory of probabilities]. 2nd ed. Moscow: Nauka. 494~p.

\smallskip

\bibitem{8-gr-1}
\Aue{Grusho, A., N.~Grusho, and E.~Timonina}.  2013. Sta\-ti\-sti\-che\-skie
metody opredeleniya zapretov veroyatnostnykh mer na diskretnykh
prostranstvakh [Statistical techniques of bans determination
of probability measures in discrete spaces].
\textit{Informatika i ee Primeneniya}~--- \textit{Inform. Appl.} 7(1):\linebreak 54--57.
\end{thebibliography}

 }
 }

\end{multicols}

\vspace*{-6pt}

\hfill{\small\textit{Received September 28, 2014}}

\vspace*{-18pt}

    \Contr

\noindent
\textbf{Grusho Alexander A.} (b.\ 1946)~---
Doctor of Science in physics and mathematics, Corresponding member of the Russian Academy
of Cryptography; leading scientist, Institute of Informatics Problems, Russian Academy of
Sciences, 44-2 Vavilov Str., Moscow 119333, Russian Federation; professor, Faculty of
Computational Mathematics and Cybernetics, M.\,V.~Lomonosov Moscow State University,
1-52~Leninskiye Gory, GSP-1, Moscow 119991, Russian Federation; grusho@yandex.ru

\vspace*{3pt}

\noindent
\textbf{Grusho Nikolai A.}\ (b.\ 1982)~---
Candidate of Science (PhD) in physics and mathematics, senior scientist, Institute of Informatics
Problems, Russian Academy of Sciences, 44-2 Vavilov Str., Moscow 119333, Russian
Federation; info@itake.ru


\vspace*{3pt}

\noindent
\textbf{Timonina Elena E.}\ (b.\ 1952)~---
Doctor of Science in technology, professor, leading scientist, Institute of Informatics Problems,
Russian Academy of Sciences, 44-2 Vavilov Str., Moscow 119333, Russian Federation;
eltimon@yandex.ru

\label{end\stat}

\renewcommand{\bibname}{\protect\rm Литература}
