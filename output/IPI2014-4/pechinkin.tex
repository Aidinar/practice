\def\stat{pechinkin}


\def\tit{СОВМЕСТНОЕ СТАЦИОНАРНОЕ РАСПРЕДЕЛЕНИЕ
ЧИСЛА ЗАЯВОК В~НАКОПИТЕЛЕ И~В~БУНКЕРЕ
ПЕРЕУПОРЯДОЧЕНИЯ В~МНОГОКАНАЛЬНОЙ СИСТЕМЕ
ОБСЛУЖИВАНИЯ С~ПЕРЕУПОРЯДОЧЕНИЕМ
ЗАЯВОК$^*$}


\def\titkol{Совместное стационарное распределение
числа заявок в~накопителе и~в~бункере
переупорядочения} % в~многоканальной системе обслуживания с~переупорядочением заявок}

\def\aut{\fbox{А.\,В.\~Печинкин}$^1$, Р.\,В.~Разумчик$^2$}

\def\autkol{А.\,В.\~Печинкин, Р.\,В.~Разумчик}

\titel{\tit}{\aut}{\autkol}{\titkol}

{\renewcommand{\thefootnote}{\fnsymbol{footnote}} \footnotetext[1]
{Работа выполнена при частичной поддержке РФФИ (проект 13-07-00223).}}


\renewcommand{\thefootnote}{\arabic{footnote}}
\footnotetext[1]{Институт проблем информатики Российской академии наук}
\footnotetext[2]{Институт проблем информатики Российской академии наук; Российский
университет дружбы народов, rrazumchik@ieee.org}

%\vspace*{3pt}

\Abst{Рассматривается функционирующая в~непрерывном времени
многоканальная система обслуживания с~накопителем
бесконечной емкости и переупорядочением заявок.
В~систему поступает пуассоновский поток заявок, время
обслуживания каждым прибором распределено по
экспоненциальному закону с~одним и~тем же параметром.
При поступлении в~систему всем заявкам  присваивается
порядковый номер. На выходе из системы сохраняется
порядок между заявками, установленный при входе в~нее.
Заявки, завершившие обслуживание и~нарушившие установленный порядок,
накапливаются на выходе системы
в~бункере переупорядочения (БП), который также имеет неограниченную емкость.
Найдено совместное стационарное распределение
числа заявок в~накопителе и~суммарного числа
заявок в~БП в~терминах
вычислительных алгоритмов и~производящих функций (ПФ).
Приведены примеры расчетов по полученным
соотношениям.}

\KW{многолинейная система массового обслуживания;
переупорядочение; стационарное распределение
числа заявок}

\DOI{10.14357/19922264140401}


%\vspace*{3pt}

\vskip 12pt plus 9pt minus 6pt

\thispagestyle{headings}

\begin{multicols}{2}

\label{st\stat}


\section{Введение}

Для функционирования ряда
ин\-фор\-ма\-ци\-он\-но-те\-ле\-ком\-му\-ни\-ка\-ци\-он\-ных сис\-тем
и для предоставления на их основе услуг
необходимо соблюдение\linebreak требования сохранения порядка в~потоке передаваемых сообщений.
Различные действия, необходимые для этого, можно объединить
в~одно понятие~--- переупорядочение.
Для изучения влияния\linebreak \mbox{переупорядочения} на качество
функционирования ин\-фор\-ма\-ци\-он\-но-те\-ле\-ком\-му\-ни\-ка\-ци\-он\-ных
сис\-тем к~настоящему времени предложено множество
моделей, которые в~своей основе используют методы
и~модели теории массового обслуживания.
Исследуемая сис\-те\-ма обычно представляется в~виде
системы или сети массового обслуживания с одним\linebreak или
несколькими входящими потоками сообщений.
Эффект переупорядочения часто моделируется с~помощью
дополнительной очереди (БП),
в~которую попадают сообщения, обработанные\linebreak в~системе,
и~ожидают там до тех пор, пока порядок следования сообщений
нельзя будет восстановить.
Некоторый обзор работ в~этом направлении можно найти
в~\cite{a1, a2},
а~некоторые последние результаты~--- в~[3--8].

Настоящая работа является развитием \cite{a8}, в~которой
рассматривается система массового обслуживания (СМО)
с~переупорядочением в~виде марковской многоканальной
системы обслуживания неограниченной емкости и~бункером
переупорядочения, также имеющим неограниченную
емкость.
В~\cite{a8} была получена система уравнений равновесия для
совместного стационарного распределения чис\-ла заявок в~системе
и~бункере переупорядочения и~приведены некоторые результаты
численных расчетов.
Однако несомненный интерес представляют
две задачи, не освещенные в~\cite{a8}, которые и~являются
предметом данной статьи, а~именно:
разработка рекуррентного алгоритма расчета вышеупомянутого
совместного стационарного распределения и~нахождение
этого распределения в~терминах ПФ.

Статья организована таким образом.
В~разд.~2 приводится подробное описание
системы.
В~разд.~3 дается рекуррентный алгоритм расчета
совместного стационарного распределения, а~в~разд.~4
показано, как совместное стационарное распределение
можно найти в~терминах ПФ.
Примеры расчетов, проведенных по формулам разд.~4,
представлены в~разд.~5.
В~заключении сформулированы основные результаты работы.

\section{Описание системы}

Рассмотрим функционирующую в~непрерывном времени
$N$-ли\-ней\-ную ($N\hm\ge 2$) СМО с накопителем
неограниченной емкости, входящим пуассоновским
потоком заявок интенсивности~$\lambda$ \mbox{и~экспоненциальным}
распределением времени
обслуживания заявки каждым прибором с~па\-ра\-мет\-ром~$\mu$.


При поступлении в~систему всем заявкам  присваивается
порядковый номер.
На выходе из СМО сохраняется порядок между заявками,
установленный при входе в~нее.
Заявки, завершившие обслуживание и~нарушившие
установленный порядок, накапливаются на выходе
системы в~БП и~покидают СМО только
после того, как закончится обслуживание всех заявок с~меньшими номерами.
Такая СМО носит название системы с переупорядочением
заявок.

Предполагается также выполнение необходимого и~достаточного условия
существования стационарного режима функционирования СМО
$$\tilde {\rho}\hm=\fr{\rho}{N}<1\,,
$$
 где $\rho\hm=\lambda/\mu$.

\vspace*{-9pt}

\section{Алгоритм нахождения совместного стационарного распределения}

Предположим, что на приборах находится $n$, $n\hm=\overline{1,N}$, заявок.
Тогда заявкой первого уровня будем называть ту из них,
которая в~систему поступила последней, второго уровня~--- предпоследней,
$\ldots,$ $n$-го уровня~--- первой. При этом если $n\hm=N$ (все приборы
заняты), то находящиеся в~БП заявки, поступившие между заявками
второго и~первого уровней, будем называть заявками первой очереди,
заявки, поступившие между заявками третьего и~второго уровней,~---
заявками второй очереди, $\ldots,$ заявки, поступившие между
заявками $N$-го и~$(N-1)$-го уровней,~--- заявками $(N-1)$-й
очереди. Если же $n<N$, то  заявками первой очереди будем называть
заявки из БП, поступившие после заявки первого уровня, заявками
второй очереди~--- заявки, поступившие между заявками второго и~первого уровней,
и~т.\,д.

При $n\ge N$ обозначим через
$p^{(m)}_{n;i}$, ${m\hm=\overline{1,N-1}}$, ${i\hm\ge 0}$,
стационарную вероятность того, что в~системе на
приборах и~в накопителе находится~$n$~заявок,
а~в~БП имеется в~сумме~$i$~заявок первой,
второй, $\ldots,$ $m$-й очереди.
Через
$p^{(m)}_{n;i}$, ${m\hm=\overline{1,n}}$, ${i\hm\ge 0}$,
обозначим аналогичную стационарную вероятность
при $n\hm=\overline{1,N-1}$.
Через~$p_n$, $n\hm\ge 0$, обозначим
стационарную вероятность того, что в~системе на
приборах и~в накопителе (без учета числа заявок в~БП) находится~$n$~заявок.
Очевидно, что стационарные вероятности~$p_n$
определяются теми же самыми формулами, что и~в~обычной
марковской СМО $M/M/N/\infty$
(см., например,~\cite{boch}):
\begin{align}
p_{0} &= \left( \sum\limits_{i=0}^{N-1} \fr{\rho^i}{i!} +
\fr{\rho^N}{(N-1)! (N-\rho)}
\right)^{-1} \,;\label{3-1}
\\
p_{i} &= \begin{cases}
\fr{\rho^i }{i!} p_{0}\,, &\ i=\overline{1,N}\,,
\\
%\label{3-3}
\fr{\rho^i}{N!\, N^{i-N}} p_{0}
= \tilde \rho^{i-N} p_{N}\,, &\ i\ge N+1\,.
\end{cases}
\label{3-2}
\end{align}

Наконец, через $p_{n;i}$, ${n\hm\ge 1}$, ${i\hm\ge 0}$, обозначим
стационарную вероятность того, что в~системе на
приборах и~в накопителе находится~$n$~заявок,
а~в~БП~--- $i$~заявок.

Используя принцип глобального баланса, можно выписать систему уравнений для
вероятностей~$p^{(m)}_{n;i}$.
Для вероятностей $p^{(1)}_{n;i}$, $n\hm\ge N$,
$i \hm\ge 0$, справедливы уравнения:
\begin{align}
\hspace*{-2.8mm}p^{(1)}_{n;0} (\lambda+N\mu) &= p^{(1)}_{n-1;0} \lambda +
p_{n+1} (N-1) \mu \,,\ n\ge N;
\!\!\label{eq-1-1}
\\
\hspace*{-2.8mm}p^{(1)}_{n;i} (\lambda+N\mu) &= p^{(1)}_{n-1;i} \lambda +
p^{(1)}_{n+1;i-1} \mu \,,\notag\\
&\hspace*{25mm} n\ge N\,,\enskip i \ge 1\,.
\label{eq-1-2}\!\!
\end{align}
%%%%%%%%%%%%%%%%%%%%%%%
%%%%%%%%%%%%%%%%%%%%%%%
Для вероятностей $p^{(1)}_{N-1;i}$,\ \ $i \ge 0$,
справедливы уравнения:
%%%%%%%%%%%%%%%%%%%
\begin{align}
\label{eq-1-3}
p^{(1)}_{N-1;0} [\lambda+(N-1)\mu] &=
p_{N-2} \lambda + p_{N} (N-1)\mu\,;
\\
\label{eq-1-4}
p^{(1)}_{N-1;i} [\lambda+(N-1)\mu] &= p^{(1)}_{N;i-1} \mu\,,\enskip i \ge 1\,.
\end{align}
Для вероятностей
$p^{(1)}_{n;i}$, $n\hm=\overline{1,N-2}$, $i \hm\ge 0$,
справедливы уравнения
\begin{align}
\label{eq-1-5}
\hspace*{-2mm}p^{(1)}_{n;0} (\lambda+n\mu) &= p_{n-1} \lambda +
p^{(1)}_{n+1;0} n\mu ,\  n=\overline{1,N-2};
\\
\label{eq-1-6}
\hspace*{-2mm}p^{(1)}_{n;i} (\lambda+n\mu) &= p^{(1)}_{n+1;i} n\mu
+ p^{(2)}_{n+1;i-1} \mu \,,\notag\\
&\hspace*{15mm}n=\overline{1,N-2},\ \ i \ge 1.
\end{align}


Для остальных вероятностей
$p^{(m)}_{n;i}$, $m\hm=\overline{2,N-1}$, справедливы формулы:
\begin{align}
p^{(m)}_{n;0} (\lambda+N\mu) &= p^{(m)}_{n-1;0} \lambda +
p^{(m-1)}_{n+1;0} (N-m) \mu\,,\notag\\
& \hspace*{30mm}n\ge N\,; \label{bat-1}
\\
p^{(m)}_{n;i} (\lambda+N\mu) &= p^{(m)}_{n-1;i} \lambda +
p^{(m-1)}_{n+1;i} (N-m) \mu +{}\notag\\
&\hspace*{-10mm}{}+p^{(m)}_{n+1;i-1} m \mu \,,\enskip
n\ge N\,,\ \ i\ge 1\,;
\label{bat-2}
\end{align}

\noindent
\begin{align}
p^{(m)}_{N-1;0} [\lambda+(N-1)\mu] &={}\notag\\
{}=p^{(m-1)}_{N-2;0} \lambda
&{}=+ p^{(m-1)}_{N;0} (N-m) \mu \,;
\label{bat-3}
\end{align}

\noindent
\begin{multline}
p^{(m)}_{N-1;i} [\lambda+(N-1)\mu] =p^{(m-1)}_{N-2;i} \lambda+{}\\
{}+
p^{(m-1)}_{N;i} (N-m) \mu +p^{(m)}_{N;i-1} m \mu\,,\enskip i\ge 1\,;
\label{bat-4}
\end{multline}

\vspace*{-12pt}



\noindent
\begin{multline}
\label{bat-5}
p^{(m)}_{n;0} (\lambda+n\mu) = p^{(m-1)}_{n-1;0} \lambda+
p^{(m)}_{n+1;0} (n-m+1) \mu \,,\\
 n=\overline{m,N-2}\,;
\end{multline}

\noindent
\begin{multline}
\label{bat-6}
p^{(m)}_{n;i} (\lambda+n\mu) = p^{(m-1)}_{n-1;i} \lambda
+
p^{(m)}_{n+1;i} (n-m+1) \mu +{}\\
{}+ p^{(m+1)}_{n+1;i-1} m \mu\,,\enskip
 n=\overline{m,N-2}\,,\ \ i\ge 1\,.
\end{multline}

Решение данной системы уравнений позволяет
найти совместное стационарное распределение
$p_{n;i}$ числа заявок на приборах и~в
накопителе и~суммарного числа заявок в~БП в~виде следующих ра\-венств:
\begin{alignat*}{2}
%\label{bat-7}
p_{n;i} &= p^{(N-1)}_{n;i}\,, &\quad  n&\ge N\,,\ \ i\ge 0\,,
\\
%\label{bat-8}
p_{n;i} &= p^{(n)}_{n;i} \,, &\quad n&=\overline{1,N-1}\,,\ \ i\ge 0\,.
\end{alignat*}

Анализ системы~\eqref{eq-1-1}--\eqref{bat-6}
показал, что вычисление стационарных
вероятностей $p^{(m)}_{n;i}$ можно проводить
рекуррентным образом по следующему алгоритму.

\bigskip

\noindent
А\,л\,г\,о\,р\,и\,т\,м~1\ (\textbf{Алгоритм решения системы уравнений равновесия}).

\noindent
\textit{Задать} $\lambda$, $\mu$ и $n$.

\noindent
\textit{Для $n\ge 0$ рассчитать $p_{n}$ по
формулам}~\eqref{3-1} и~\eqref{3-2}.

\noindent
\textit{Рассчитать $p^{(1)}_{N-1;0}$ по формуле}~\eqref{eq-1-3}.

\noindent
\textit{Для $n\ge N$ рассчитать $p^{(1)}_{n;0}$ по
формуле}~\eqref{eq-1-1}.

\noindent
\textit{Для $i\ge1$}


\textit{рассчитать $p^{(1)}_{N-1;i}$ по формуле}~\eqref{eq-1-4}.

\textit{для $n\ge N$ рассчитать $p^{(1)}_{n;i}$ по формуле}~\eqref{eq-1-2}.

\noindent
\textit{Для $n=\overline{N-2,1}$ рассчитать $p^{(1)}_{n;0}$
по формуле}~\eqref{eq-1-5}.

\noindent
\textit{Для $m=\overline{2,N-1}$}

\textit{рассчитать $p^{(m)}_{N-1;0}$ по формуле}~\eqref{bat-3}.


\textit{для $n\ge N$ рассчитать $p^{(m)}_{n;0}$
   по формуле}~\eqref{bat-1};

\textit{для} $i\hm\ge1$

    \hspace*{9pt}\textit{рассчитать $p^{(1)}_{N-m;i}$ по
    формуле}~\eqref{eq-1-6};


    \hspace*{9pt}\textit{если $m \ne 2$, для}  $j\hm=\overline{2,m-1}$ \textit{рассчитать}\linebreak\vspace*{-12pt}

 \hspace*{9pt}\textit{$p^{(j)}_{N-m+j-1;i}$ по формуле}~\eqref{bat-6};

\hspace*{9pt}\textit{рассчитать $p^{(m)}_{N-1;i}$ по формуле}~\eqref{bat-4};

\hspace*{9pt}\textit{для $n\ge N$ рассчитать $p^{(m)}_{n;i}$
    по формуле}~\eqref{bat-2};

\textit{если {$m \ne N-1$}, для $m\hm=\overline{N-2,m}$
   рассчитать}\linebreak

   \textit{$p^{(m)}_{n;0}$ по формуле}~\eqref{bat-5}.

\bigskip

В~связи с~тем, что вычисление моментов после расчета
вероятностей по представленному алгоритму
может давать погрешности, в~следующем разделе
находятся формулы для совместного стационарного
распределения в~терминах ПФ.


\section{Использование производящих функций}

Система уравнений~\eqref{eq-1-1}--\eqref{bat-6}
допускает также решение с~помощью ПФ.
Для нахождения этого решения положим
\begin{equation*}
\label{f-m}
f_m(u,z) = \lambda u^2 - (\lambda + N\mu) u + m \mu z\,,\
 m=\overline{1,N-1}\,.
\end{equation*}

Обозначим через $u_m\hm=u_m(z)$, $m\hm=\overline{1,N-1}$,
минимальное решение уравнения
$$
f_m(u,z) = 0\,,
$$
т.\,е.
\begin{equation*}
%\label{sqrt}
u_m = \fr{\lambda + N\mu - \sqrt{(\lambda + N\mu)^2 - 4 m \lambda \mu z}}
{2 \lambda }\,.
\end{equation*}


Введем ПФ
\begin{multline*}
P^{(m)}_{n}(z) = \sum\limits_{i=0}^{\infty}
z^{i} p^{(m)}_{n;i}\,, \\
0<z<1\,, \ \ n\ge1\,,\ \
m=\overline{1,\min(n,N-1)} \,;
\end{multline*}

\vspace*{-12pt}


\noindent
\begin{multline*}
P^{(m)}(u,z) = \sum\limits_{n=N}^{\infty} u^{n-N} P^{(m)}_{n}(z)\,, \\
0<u,z<1\,, \ \ m=\overline{1,N-1}\,,
\end{multline*}
и, кроме того, положим
$$
P(u) = \sum\limits_{n=N}^{\infty} u^{n-N} p_{n}
= \fr{1}{1 - \tilde{\rho} u}\, p_N \,.
$$

Тогда, умножая~\eqref{eq-1-1} и~\eqref{eq-1-2}
на~$z^i$ и~суммируя по всем~$i$ от нуля до
бесконечности, получаем:
\begin{multline*}
%\label{eq-z-1}
(\lambda+N\mu) P^{(1)}_{n}(z) =
\lambda P^{(1)}_{n-1}(z) +
(N-1) \mu p_{n+1}
+ {}\\
{}+\mu z P^{(1)}_{n+1}(z)\,,\enskip n\ge N\,.
\end{multline*}
Умножая последнее выражение на $u^{n-N}$ и~суммируя по всем значениям $n\hm\ge N$,
после приведения подобных слагаемых имеем:
\begin{multline}
\label{eq-z-2}
f_1(u,z) P^{(1)}(u,z) =
\mu z P^{(1)}_{N}(z) -{}\\
{}- \lambda u P^{(1)}_{N-1}(z) -
(N-1) \mu [P(u) - p_{N}] \,.
\end{multline}


Теперь умножим \eqref{bat-1} и~\eqref{bat-2}
на~$z^i$ и~просуммируем по всем значениям $i\hm\ge0$.
В~результате приходим к~выражению:
\begin{multline*}
%\label{bat-2*}
(\lambda+N\mu) P^{(m)}_{n}(z) = \lambda P^{(m)}_{n-1}(z)
+{}\\
{}+(N-m) \mu P^{(m-1)}_{n+1}(z) +
m \mu z P^{(m)}_{n+1}(z) \,,\enskip n\ge N\,.
\end{multline*}
Умножая последнее выражение на $u^{n-N}$, после
суммирования по всем $n\hm\ge N$ получаем:

\pagebreak

\noindent
\begin{multline}
\label{bat-2*}
f_m(u,z) P^{(m)}(u,z) = m \mu z P^{(m)}_{N}(z)
- \lambda u P^{(m)}_{N-1}(z) -{}\\
{}-
(N-m) \mu [P^{(m-1)}(u,z) - P^{(m-1)}_{N}(z)]\,,\\ m=\overline{2,N-1}\,.
\end{multline}

Из уравнений~\eqref{eq-1-3} и~\eqref{eq-1-4}
после умножения на~$z^i$ и~суммирования по
всем значениям $i \hm\ge 0$ находим:
\begin{multline}
\label{eq-z-3}
P^{(1)}_{N-1}(z)=\fr{\lambda p_{N-2} + (N-1)\mu p_{N}}
{\lambda+(N-1)\mu }+{}\\
{}+ \fr{\mu z}{\lambda+(N-1)\mu} \,P^{(1)}_N(z)\,.
\end{multline}

Действуя аналогичным образом
с~уравнениями~\eqref{bat-3} и~\eqref{bat-4}, как и~с~уравнениями~\eqref{eq-1-3}
и~\eqref{eq-1-4}, приходим к выражению:
\begin{multline}
\label{bat-4*}
P^{(m)}_{N-1}(z) = \fr{ \lambda P^{(m-1)}_{N-2}(z) + (N-m) \mu P^{(m-1)}_{N}(z)
}{\lambda+(N-1)\mu }+{}\\
{}+\fr{m \mu z}{\lambda+(N-1)\mu}\,P^{(m)}_{N}(z) \,,\enskip m=\overline{2,N-1}\,.
\end{multline}


Домножая уравнения~\eqref{eq-1-5} и~\eqref{eq-1-6}
на~$z^i$, после суммирования по всем
значениям $i \hm\ge 0$ имеем:
\begin{multline}
\label{eq-z-4}
P^{(1)}_{n}(z)= \fr{ \lambda p_{n-1} + n \mu P^{(1)}_{n+1}(z) }{
\lambda+n\mu }+ \fr{\mu z}{\lambda+n\mu}\,P^{(2)}_{n+1}(z) \,,\\
n=\overline{1,N-2}\,.
\end{multline}

Наконец, производя аналогичные преобразования
с~уравнениями~\eqref{bat-5} и~\eqref{bat-6}, получаем:
\begin{multline}
\label{bat-6*}
P^{(m)}_n(z)= \fr {\lambda P^{(m-1)}_{n-1}(z) +
(n-m+1) \mu P^{(m)}_{n+1}(z)} {\lambda+n\mu}
+{}
\\
{}+
\fr{m \mu z}{\lambda+n\mu} P^{(m+1)}_{n+1}(z)\,,\enskip
m=\overline{2,N-2}\,,\\
n=\overline{m,N-2}\,.
\end{multline}

Уравнения~\eqref{eq-z-2}--\eqref{bat-6*} позволяют
находить выражения для всех
ПФ $P^{(m)}_{n}(z)$, $m\hm=\overline{1,N-1}$,
$n\hm=\overline{1,N-1}$, а~так\-же совместное
стационарное распределение рекуррентным образом.
Подставляя выражение для $P^{(1)}_{N-1}(z)$ из
формулы~\eqref{eq-z-3} в~формулу~\eqref{eq-z-2}, получаем:
\begin{multline}
P^{(1)}(u,z) = \left(
\left[
\mu z - \fr{\lambda \mu z u}{\lambda+(N-1)\mu}
\right] P^{(1)}_N(z) -{}\right.\\
{}-
\left[
\lambda u \fr{\lambda p_{N-2} + (N-1)\mu p_{N}}{\lambda+(N-1)\mu}+{}\right.\\
\left.\left.{}+
 (N-1) \mu [P(u) - p_{N}]
\vphantom{\fr{\lambda p_{N-2} + (N-1)\mu p_{N}}{\lambda+(N-1)\mu}}\right]
\right)
\Bigg /
f_1(u,z)\,,
\label{m25}
\end{multline}
откуда из равенства нулю в~точке $u_1(z)$ числителя и~знаменателя
правой части формулы~\eqref{m25} следует:
\columnbreak


%%%%%%%%%%%%%%%%%%%%%%%%%%%
\noindent
\begin{multline*}
%\label{r1}
P^{(1)}_N(z)= \left(
\lambda u_1(z) [\lambda p_{N-2} + (N-1)\mu p_{N}]
+{}\right.\\
{}+
\left.(\lambda+(N-1)\mu)(N-1) \mu \left[P(u_1(z)) - p_{N}\right]\right)\!\!\Big/\!\!
\left(\mu z \left[\lambda+{}\right.\right.\\
\left.\left.{}+(N-1)\mu  - \lambda u_1(z)\right]\right)\,.
\end{multline*}
%%%%%%%%%%%%%%%%%%%%%%%%%%%%%%%%%%%%%%%%%%
%%%%%%%%%%%%%%%%%%%%%%%%%%%%%%%%%%%%%%%%
Теперь, возвращаясь к~формуле~\eqref{eq-z-3},
получаем выражение для $P^{(1)}_{N-1}(z)$:
\begin{multline*}
%\label{r2}
P^{(1)}_{N-1}(z)=
\left([\lambda p_{N-2} + (N-1)\mu p_{N}]+{}\right.\\
\left.{}
+ (N-1) \mu [P(u_1(z)) - p_{N}]\right)\Big /
\left(\lambda+(N-1)\mu  - {}\right.\\
\left.{}-\lambda u_1(z)\right)\,.
\end{multline*}

Далее из равенства~\eqref{eq-z-4} выражаем $P^{(1)}_{N-2}(z)$ через
$P^{(2)}_{N-1}(z)$. Из равенства~\eqref{bat-4*} выражаем
$P^{(2)}_{N-1}(z)$ через $P^{(2)}_{N}(z)$. Подставляя полученное
выражение для $P^{(2)}_{N-1}(z)$ в~формулу~\eqref{bat-2*}, из
равенства нулю в~точке~$u_2$ левой и~правой части получившегося
равенства находим $P^{(2)}(u,z)$. Затем из равенства~\eqref{eq-z-4}
выражаем $P^{(1)}_{N-3}(z)$ через $P^{(2)}_{N-2}(z)$ и~т.\,д.

Продолжая эту процедуру, можно найти
соотношения для вычисления всех
ПФ $P^{(m)}_{n}(z)$, $m\hm=\overline{1,N-1}$, $n\hm=\overline{1,N-1}$.

С каждым шагом выражение для очередной ПФ становится все сложнее,
и~в итоге при большом числе приборов выписать явный вид всех ПФ не
удается. Тем не менее нахождение значений ПФ в~каждой точке $z \hm\ne
0$ можно свести к последовательному решению систем линейных
уравнений. Для этого обозначим через $A_n(z)$, $n\hm =\overline{2,N-1}$,
мат\-ри\-цы размера $(n+1)\times (n+1)$, име\-ющие
следующую структуру:
\begin{gather*}
\setcounter{MaxMatrixCols}{3}
A_2(z)=
\begin{pmatrix}
 2 \mu z   & 0  & -2 \mu z         \\
 - \lambda u_2(z) & - \mu z   &  \lambda +(N-1) \mu       \\
0  & \lambda +(N-2)\mu &    - \lambda
\end{pmatrix}\,;
\end{gather*}

\vspace*{-12pt}

\noindent
{ %\scriptsize
\begin{multline*}
\setcounter{MaxMatrixCols}{7}
A_n(z)=\left(
\begin{matrix}
 n \mu z   & 0  & - n \mu z &      \!\cdots\!          \\
 - \lambda u_n(z) \! & 0  & \! \lambda +(N-1) \mu \! & \!\cdots\!  \\
  \vdots   & \vdots & \vdots &  \!\cdots\! \\
 0   & 0 & 0&  \!\cdots\!  \\
 0      & 0 & 0    &     \!\cdots\! \\
 0   & - \mu z  &0   &  \cdots\! \\
0 & \!\lambda +(N-n)\mu \!&0  &  \!\cdots\!
\end{matrix}\right.\\
\left.\begin{matrix}
    \cdots\!     & 0    & 0       \\
    \cdots\!  & 0 & 0 \\
    \cdots\! & \vdots     & \vdots  \\
    \cdots\!  & - 3 \mu z     & 0   \\
    \cdots\! & \! \lambda +(N-n+2) &-2\mu z\\
    \cdots\! & - \lambda  & \! \lambda+(N-n+1)\mu\\
    \cdots\!  & 0  & - \lambda
\end{matrix}\right)\,,
\\ n =\overline{3,N-1}\,.
\end{multline*}
}

\noindent
Определим вектор-стр$\acute{\mbox{о}}$\-ки $\vec{a}_n(z)$ и~$\vec{b}_n(z)$
длины $(n+1)$ следующим образом:
\begin{multline*}
\vec{a}_n(z) = \left (
P^{(n)}_{N}(z), P^{(n)}_{N-1}(z), \dots\right.\\
\left.\dots,  P^{(2)}_{N-n+1}(z), P^{(1)}_{N-n}(z)
\right )\,,\enskip
n =\overline{2,N-1}\,;
\end{multline*}

\vspace*{-12pt}
\noindent
\begin{multline*}
\vec{b}_2(z) = \left (
(N-2) \mu [P^{(1)}(u_2,z) - P^{(1)}_{N}(z)] ,
\lambda p_{N-3}+{}\right.\\
\left.{}+ (N-2) \mu P^{(1)}_{N-1}(z),
(N-2) \mu P^{(1)}_{N}(z) \right)\,;
\end{multline*}

\vspace*{-12pt}

\noindent
\begin{multline*}
\vec{b}_n(z) = \left (
(N-n) \mu
[P^{(n-1)}(u_n,z) - P^{(n-1)}_{N}(z)],\right.
\\
\lambda p_{N-1-n}+ (N-n) \mu P^{(1)}_{N-1-(n-2)}(z),\\
(N-n) \mu P^{(n-1)}_{N}(z), (N-n)\mu  P^{(n-1)}_{N-1}(z),
\dots ,
\\
\left.
(N-n)\mu  P^{(3)}_{N-n+3}(z), (N-n)\mu  P^{(2)}_{N-n+2}(z)
\right )\,,\\
n =\overline{3,N-1}\,.
\end{multline*}
Тогда алгоритм нахождения ПФ состоит в~последовательном начиная с~$n\hm=2$ решении
системы линейных уравнений
$$
\vec{a}_n(z) A_n(z) = \vec{b}_n(z) \,.
$$
Из структуры матрицы $A_n(z)$, $n \hm=\overline{3,N-1}$, видно, что
она неприводима и~обладает свойством диагонального преобладания
т.\,е.\ перестановкой строк и~столбцов можно добиться того,
что в~каждой строке модуль диагонального элемента будет либо строго
больше, либо не меньше суммы модулей всех остальных элементов в~строке.
Покажем это. Если определить матрицы перестановки~$P^L_n$ и~$P^R_n$
размера $(n+1)\times (n+1)$ при $n \hm=\overline{3,N-1}$
следующим образом:
\begin{gather*}
\setcounter{MaxMatrixCols}{5}
P^L_n=
\begin{pmatrix}
 0   & 0  & \cdots & 0& 1 \\
 1   & 0  & \cdots & 0& 0 \\
  \vdots   &  \vdots  & \cdots &  \vdots &  \vdots \\
  0   & 0  & \cdots & 0& 0 \\
   0   & 0  & \cdots & 1& 0
\end{pmatrix}\,;
\enskip
\setcounter{MaxMatrixCols}{5}
P^R_n=
\begin{pmatrix}
 0   & 1  & \cdots & 0& 0         \\
 1   & 0  & \cdots & 0& 0 \\
   \vdots   &  \vdots  & \cdots &  \vdots &  \vdots \\
  0   & 0  & \cdots & 1& 0 \\
   0   & 0  & \cdots & 0& 1
\end{pmatrix}\,,
\end{gather*}
то матрица $P^L_n A_n(z)P^R_n$, $n \hm=\overline{3,N-1}$,
примет вид:
\begin{multline*}
\setcounter{MaxMatrixCols}{7}
P^L_n A_n(z)P^R_n={}\\
{}=\left(
\begin{matrix}
 \lambda +(N-n)\mu &0 & 0  & \cdots\\
 0  &  n \mu z   & - n \mu z &  \cdots       \\
  0  & - \lambda u_n(z) & \lambda +(N-1) \mu  & \cdots  \\
  \vdots   & \vdots & \vdots & \cdots   \\
 0   & 0 & 0& \cdots \\
 0      & 0 & 0    & \cdots  \\
 - \mu z  & 0   &0   & \cdots
\end{matrix}\right.
\end{multline*}

\noindent
\begin{equation*}
\hspace*{15mm}\left.\begin{matrix}
\cdots  & 0  & - \lambda\\
\cdots        & 0    & 0       \\
\cdots    & 0    & 0      \\
\cdots   & \vdots     & \vdots       \\
\cdots  & - 3 \mu z     & 0       \\
\cdots   & \lambda +(N-n+2) \mu & - 2 \mu z       \\
\cdots      & - \lambda  & \lambda +(N-n+1)\mu
\end{matrix}\right).
\end{equation*}
Легко видеть, что в~каждой строке модуль диагонального
элемента либо строго больше, либо не меньше суммы
модулей всех остальных элементов в~строке.
Тогда, как вытекает из следствия~6.2.27 в~\cite{horn},
у~матрицы $A_n(z)$ существует обратная
и,~значит, система $\vec{a}_n(z) A_n(z) \hm= \vec{b}_n(z)$
при $z\hm\neq 0$ имеет единственное решение.

\vspace*{-4pt}

\section{Примеры расчетов}

На основе полученных в~разд.~4 результатов {были} проведены расчеты
среднего и~дисперсии чис\-ла заявок в~БП,
а~также коэффициента корреляции числа заявок в~накопителе и~числа
заявок в~БП для различного чис\-ла
приборов~$N$~и~значений загрузки системы $\rho/N$. \mbox{Напомним}, что аналогичные
показатели были рассчитаны в~\cite{a8} по определению, на основе
стационарных вероятностей, рассчитанных по приведенному выше
алгоритму. Далее можно видеть, что результаты, полученные с~по\-мощью
ПФ, как и~ожидалось, полностью совпадают с~результатами,
представленными в~\cite{a8}.

На рис.~1 отражено поведение значения среднего числа заявок
в~БП в~зависимости от загрузки системы $\rho/N$.
Отметим, что полученные в~предыдущих  разделах результаты позволяют
рассчитывать такие
 характеристики, как среднее число заявок только
в~первой очереди в~БП, в~сумме в~первой и~во второй очередях в~БП
(когда обе очереди существуют), в~сумме в~первой, второй,\ldots ,
$(N-1)$-й очере-\linebreak\vspace*{-12pt}

\vspace*{6pt}

\begin{center}  %fig1
\vspace*{2pt}
\mbox{%
 \epsfxsize=75.145mm
 \epsfbox{pec-1.eps}
 }
\end{center}

\noindent
{{\figurename~1}\ \ \small{Поведение
 среднего числа заявок в~БП в~зависимости от загрузки
системы  $\rho/N$: \textit{1}~--- $N\hm=4$; \textit{2}~--- 7;
\textit{3}~--- $N=9$}}

%\vspace*{9pt}


\addtocounter{figure}{1}


\begin{center}  %fig2
\vspace*{2pt}
 \mbox{%
 \epsfxsize=75.027mm
 \epsfbox{pec-2.eps}
 }
 \end{center}

\noindent
{{\figurename~2}\ \ \small{Поведение среднего числа заявок в~первой
очереди в~БП~(\textit{1}), в~сумме в~первой и~во второй очередях в~БП~(\textit{2}),
в~сумме в~первой, второй и~третьей очередях в~БП~(\textit{3})
в~зависимости от загрузки системы $\rho/N$. Число
приборов $N\hm=4$}}

\vspace*{18pt}


\begin{center}  %fig3
\vspace*{2pt}
 \mbox{%
 \epsfxsize=74.929mm
 \epsfbox{pec-3.eps}
 }
 \end{center}

\noindent
{{\figurename~3}\ \ \small{Поведение
 дисперсии числа заявок в~БП в~зависимости от загрузки
системы  $\rho/N$: \textit{1}~--- $N\hm=4$; \textit{2}~--- 7; \textit{3}~--- $N=9$}}

\vspace*{18pt}

\begin{center}  %fig4
\vspace*{2pt}
 \mbox{%
 \epsfxsize=75.192mm
 \epsfbox{pec-4.eps}
 }
 \end{center}

\noindent
{{\figurename~4}\ \ \small{Поведение
 коэффициента корреляции числа заявок в~накопителе и~числа
заявок в~БП в~зависимости от загрузки системы  $\rho/N$:
\textit{1}~--- $N\hm=4$; \textit{2}~--- 7; \textit{3}~--- $N=9$}}


%\vspace*{9pt}


\noindent
дях в~БП (когда каждая из очередей существует).
Поведение данных характеристик в~зависимости от загрузки системы
$\rho/N$ для случая $N\hm=4$ пред\-став\-ле\-но на рис.~2.

На рис.~3 и~4 изображено поведение дисперсии числа
заявок в~БП и~поведение
коэффициента корреляции числа заявок в~накопителе и~числа
заявок в~БП соответственно.

Во всех расчетах интенсивность обслуживания заявок~$\mu$ принималась
равной~1.

%\addtocounter{figure}{1}
%%%%%%%%%%%%%%%%%%%%%%%%%%%%%%%%%%%%%%%%%%%%%%%%%%%%%

Анализируя графики на рис.~1--4, стоит отметить два момента. Среднее
число заявок в~БП не уходит в~бесконечность с ростом загрузки
(и~даже при загрузке больше единицы), что следует из формулы Литтла.
Число заявок в~накопителе и~число заявок в~БП весьма слабо
коррелированы, и~с~рос\-том числа приборов коэффициент корреляции
уменьшается.

\section{Заключение}

В настоящей работе рассмотрена функционирующая в~непрерывном времени
многоканальная система обслуживания с~накопителем бесконечной емкости
и~переупорядочением заявок.
В~систему поступает пуассоновский поток заявок, время
обслуживания каждым прибором распределено по
экспоненциальному закону с~одним и~тем же параметром.
Для нахождения совместного стационарного распределения
числа заявок в~накопителе и~суммарного числа
заявок в~БП получен рекуррентный алгоритм.
Также показано, как можно находить совместное распределение
в~терминах ПФ, которые облегчают расчет его моментов.

{\small\frenchspacing
 {%\baselineskip=10.8pt
 \addcontentsline{toc}{section}{References}
 \begin{thebibliography}{99}
 \bibitem{a1} %1
\Au{Boxma O., Koole G., Liu~Z.}
Queueing-theoretic solution methods for
models of parallel and distributed systems~//
Performance Evaluation of Parallel and Distributed Systems Solution
Methods, 1994. CWI Tract~105 and~106. P.~1--24.

\bibitem{a2} %2
\Au{Dimitrov B.}
Queues with resequencing. A~survey and recent results~//
{2nd World Congress on Nonlinear Analysis,
Theory, Methods, Applications Proceedings}, 1997. Vol.~30. No.\,8. P.~5447--5456.

\bibitem{a3} %3
\Au{Huisman T., Boucherie R.\,J.}
The sojourn time distribution in an infinite server
resequencing queue with dependent interarrival and
service times~// J.~Appl. Probab., 2002.
Vol.~39. No.\,3. P.~590--603.

\bibitem{a5} %4
\Au{Xia Y., Tse D.\,N.\,C.}
On the large deviations of resequencing
queue size: 2-$M$/$M$/1 сase~// IEEE Trans. Inform. Theory, 2008.
Vol.~54. No.\,9. P.~4107--4118.

\bibitem{a4} %5
\Au{Leung K., Li V.\,O.\,K.}
A~resequencing model for high-speed packet-switching networks~//
J.~Comput. Commun., 2010.
Vol.~33. No.\,4. P.~443--453.

\bibitem{a7} %6
\Au{Матюшенко С.\,И.} Стационарные характеристики двухканальной
системы обслуживания с~переупорядочением заявок и~распределениями
фазового типа~// Информатика и~её применения, 2010. Т.~4. Вып.~4.
С.~67--71.

\bibitem{a6} %7
\Au{De Nicola C., Pechinkin A.\,V., Razumchik~R.\,V.}
Stationary characteristics of homogenous Geo/Geo/2
queue with resequencing in discrete time~//
27th European Conference on Modelling and
Simulation Proceedings.~---- Aalesund, 2013. P.~594--600.

\bibitem{a7+} %8
\Au{Pechinkin A.\,V., Caraccio~I., Razumchik~R.\,V.}
Joint stationary distribution of queues in
homogenous $M\vert M\vert$3 queue with resequencing~//
28th European Conference on
Modelling and Simulation Proceedings.~--- Brescia, 2014. P.~558--564.

\bibitem{a8}
\Au{Pechinkin A.\,V., Caraccio~I., Razumchik~R.\,V.}
On joint stationary distribution in exponential
multiserver reordering queue~// 12th  Conference (International) on
Numerical Analysis and Applied Mathematics Proceedings, 2014 (in press).

\bibitem{boch}
\Au{Bocharov P.\,P., D'Apice C., Pechinkin~A.\,V., Salerno~S.}
Queueing theory.~--- Urecht, Boston: VSP, 2004. 446~p.

\bibitem{horn}
\Au{Horn R.\,A., Johnson C.\,R.}
Matrix analysis.~--- 2nd ed.~--- Cambridge: Cambridge University Press, 2013.
662~p.
 \end{thebibliography}

 }
 }

\end{multicols}

\vspace*{-9pt}

\hfill{\small\textit{Поступила в редакцию 28.10.14}}

%\newpage

\vspace*{12pt}

\hrule

\vspace*{2pt}

\hrule

%\vspace*{12pt}

\def\tit{JOINT STATIONARY DISTRIBUTION OF~THE~NUMBER OF~CUSTOMERS IN~THE~SYSTEM
AND REORDERING BUFFER IN~THE~MULTISERVER REORDERING QUEUE}

\def\titkol{Joint stationary distribution of~the~number of~customers in~the~system
and reordering buffer in~the~multiserver reordering queue}



\def\aut{\fbox{A.\,V.~Pechinkin}$^1$ and R.\,V.~Razumchik$^{1,2}$}

\def\autkol{A.\,V.~Pechinkin and R.\,V.~Razumchik}

\titel{\tit}{\aut}{\autkol}{\titkol}

\vspace*{-9pt}

\noindent
$^1$Institute of Informatics Problems, Russian Academy of Sciences,
44-2 Vavilov Str., Moscow 119333, Russian\\
$\hphantom{^1}$Federation


\noindent
$^2$Peoples' Friendship University of Russia,
6~Miklukho-Maklaya Str., Moscow 117198, Russian Federation



\def\leftfootline{\small{\textbf{\thepage}
\hfill INFORMATIKA I EE PRIMENENIYA~--- INFORMATICS AND
APPLICATIONS\ \ \ 2014\ \ \ volume~8\ \ \ issue\ 4}
}%
 \def\rightfootline{\small{INFORMATIKA I EE PRIMENENIYA~---
INFORMATICS AND APPLICATIONS\ \ \ 2014\ \ \ volume~8\ \ \ issue\ 4
\hfill \textbf{\thepage}}}

\vspace*{3pt}



\Abste{The paper considers a continuous-time multiserver queueing
system with buffer on infinite capacity and reordering. The Poisson
flow of customers arrives at the system. Service times of customers at
each server are exponentially distributed with the same parameter.
Each customer obtains a~sequential number upon arrival. The order of
customers upon arrival should be preserved upon departure from the system.
Customers whose service finished but which violated the order are kept in
the reordering buffer of infinite capacity. A~joint stationary distribution
of the number of customers in the buffer, servers, and
reordering buffer is obtained in terms of a~computational algorithm and
a~generating function. A~numerical example is provided.}


\KWE{queueing system; reordering; infinite capacity; joint distribution}

\DOI{10.14357/19922264140401}

%\vspace*{3pt}

\Ack
\noindent
The research was partially financially supported by the Russian Foundation for
Basic Research (project 13-07-00223).


  \begin{multicols}{2}

\renewcommand{\bibname}{\protect\rmfamily References}
%\renewcommand{\bibname}{\large\protect\rm References}



{\small\frenchspacing
 {%\baselineskip=10.8pt
 \addcontentsline{toc}{section}{References}
 \begin{thebibliography}{99}


 \bibitem{a1-1}
\Aue{Boxma O., G. Koole, and Z.~Liu}. 1994.
Queueing-theoretic solution methods for
models of parallel and distributed systems.
\textit{Performance Evaluation of Parallel and
Distributed Systems Solution Methods}.  CWI Tract 105
and 106:1--24.

\bibitem{a2-1}
\Aue{Dimitrov, B.} 1997.
Queues with resequencing. A~survey and recent results.
\textit{2nd World Congress on Nonlinear
Analysis, Theory, Methods, Applications Proceedings}. 30(8):5447--5456.

\bibitem{a3-1}
\Aue{Huisman, T., and R.\,J.~Boucherie}. 2002.
The sojourn time distribution in an infinite server
resequencing queue with dependent interarrival and service times.
\textit{J.~Appl. Probab}. 39(3):590--603.

\bibitem{a5-1}
\Aue{Xia, Y., and D.\,N.\,C.~Tse}. 2008.
On the large deviations of resequencing
queue size: 2-$M$/$M$/1 case.
\textit{IEEE Trans. Inform. Theory} 54(9):4107--4118.

\bibitem{a4-1} %5
\Aue{Leung, K., and V.\,O.\,K.~Li}. 2010.
A~resequencing model for high-speed
packet-switching networks.
\textit{J.~ Comput. Commun.} 33(4):443--453.

\bibitem{a7-1} %6
\Aue{Matyushenko, S.\,I.} 2010.
 Statsionarnye kharakteristiki
dvukh\-ka\-nal'\-noy sistemy obsluzhivaniya s~pe\-re\-upo\-rya\-do\-chi\-va\-ni\-em zayavok
i~raspredeleniyami
fazovogo tipa [Stationary characteristics of the two-channel
queueing system with reordering customers and distributions of phase type].
\textit{Informatika i ee Primemeniya}~--- \textit{Inform. Appl.}
4(4):67--71.

\bibitem{a6-1} %7
\Aue{De Nicola, C., A.\,V.~Pechinkin, and R.\,V.~Razumchik}. 2013.
Stationary characteristics of homogenous Geo/Geo/2
queue with resequencing in discrete time.
\textit{27th European Conference
on Modelling and Simulation Proceedings}. Aalesund. 594--600.

\bibitem{a7+-1}
\Aue{Pechinkin, A.\,V., I.~Caraccio, and R.\,V.~Razumchik}. 2014.
joint stationary distribution of queues
in homogenous $M \vert M \vert3$ queue with resequencing.
\textit{28th European Conference
on Modelling and Simulation Proceedings}. Brescia. 558--564.

\bibitem{a8-1}
\Aue{Pechinkin, A.\,V., I.~Caraccio, and R.\,V.~Razumchik}. 2014 (in press).
On joint stationary distribution in exponential
multiserver reordering queue.
\textit{12th  Conference (International) on
Numerical Analysis and Applied Mathematics Proceedings}.

\bibitem{boch-1}
\Aue{Bocharov,  P.\,P., C.~D'Apice, A.\,V.~Pechinkin, and S.~Salerno}. 2004.
\textit{Queueing theory}. Urecht, Boston: VSP. 446~p.

\bibitem{horn-1}
\Aue{Horn, R.\,A., and C.\,R.~Johnson}. 2013.
\textit{Matrix analysis}. Cambridge: Cambridge University Press. 662~p.
\end{thebibliography}

 }
 }

\end{multicols}

\vspace*{-6pt}

\hfill{\small\textit{Received October 28, 2014}}

\vspace*{-18pt}

\Contr

\noindent
\textbf{Pechinkin Alexander V.} (1946--2014)~--- Doctor
of Science in physics and mathematics; principal
scientist, Institute of Informatics Problems of
the Russian Academy of Sciences, 44-2 Vavilov Str.,
Moscow 119333, Russian Federation


\vspace*{3pt}

\noindent
\textbf{Razumchik Rostislav V.} (b.\ 1984)~--- Candidate
of Science (PhD) in physics and mathematics,
senior scientist, Institute of Informatics
Problems of the Russian Academy of Sciences, 44-2 Vavilov Str.,
Moscow 119333, Russian Federation;
associate professor,
Peoples' Friendship University of Russia,
6~Miklukho-Maklaya Str., Moscow 117198, Russian Federation;
rrazumchik@ieee.org


\label{end\stat}

\renewcommand{\bibname}{\protect\rm Литература}