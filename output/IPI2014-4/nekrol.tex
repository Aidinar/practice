   \vspace*{-36pt}

\begin{center}
\vspace*{6pt}
\mbox{%
\epsfxsize=56mm %76.2mm
\epsfbox{foto.eps}
}
\end{center}

\vspace*{6pt} %Академик


   \begin{center}
\fbox{\Large\textbf{Профессор Александр Владимирович Печинкин}}\\[12pt]
\textbf{\large 7.10.1946--4.12.2014}
   \end{center}

   %\vspace*{2.5mm}

   \vspace*{3mm}

   \thispagestyle{empty}

%\

%\vspace*{-12pt}



Институт проблем информатики Российской академии наук, редакционный совет
и~редакционная коллегия журнала <<Информатика и её применения>> с~глубоким
прискорбием извещают,  что 4~декабря 2014~г.\ после продолжительной
и~тяжелой болезни скончался Александр Владимирович Печинкин~--- лауреат
премии Правительства РФ в~области науки и~техники, доктор
физико-математических наук, профессор, главный научный сотрудник ИПИ РАН,
член редколлегии журнала <<Информатика и~её применения>>.

А.\,В. Печинкин родился в 1946~г.\ в~Москве. Еще до окончания в~1968~г.\
механико-математического факультета  МГУ им.\ М.\,В.~Ломоносова А.\,В.~Печинкин
начал вести научную и~педагогическую деятельность, которую затем
продолжил в~различных на\-уч\-но-ис\-сле\-до\-ва\-тель\-ских учреждениях
и~высших учебных заведениях столицы (НИИ ССУ, МИЭМ, МГТУ им.\ Н.\,Э.~Баумана,
РУДН). С~2000~г.\ его работа была неразрывно связана с~ИПИ РАН.
Выдающийся ученый, получивший признание в~России и~за рубежом, внесший
существенный научный вклад в развитие теории массового обслуживания,
А.\,В.~Печинкин основал крупную научную школу, из которой вышло большое
число молодых ученых. Научные работы А.\,В.~Печинкина главным образом
относятся к~теории вероятностей и~ее приложениям. Он автор свыше
200~фундаментальных трудов по прикладной теории вероятностей и~теории массового обслуживания. А.\,В.~Печинкин являлся членом различных диссертационных советов, редколлегий научных журналов, программных комитетов международных научных конференций.

А.\,В. Печинкин был выдающимся преподавателем. Его педагогический талант нашел
свое отражение в~учебниках <<Теория вероятностей и математическая статистика>>
и~<<Теория массового обслуживания>>, которые были переведены на английский язык
и~по которым училось несколько поколений сту\-ден\-тов-ма\-те\-ма\-ти\-ков в~России и~за
рубежом. За серию учебников <<Математика в~техническом университете>>
он был удостоен премии Правительства РФ в~области науки и техники.

Для научного творчества А.\,В.~Печинкина была характерна любовь к~конкретно
поставленным вопросам любой важности~--- от занимательных задач для школьников
и~студентов до сложных вопросов чистой и~прикладной математики.
Лекции А.\,В.~Печинкина стимулировали в~каждом заинтересованном
слушателе представления о~существовании замечательных связей между
разнородными математическими объектами, на первый взгляд совершенно различными.
Он с~одинаково большим вниманием и~участием относился и~к~своим молодым ученикам,
и~к~уже состоявшимся ученым. А.\,В.~Печинкин обладал большим личным обаянием,
имел широкий круг интересов.

Все знавшие А.\,В.~Печинкина всегда будут помнить его как замечательного ученого
и~прекрасного товарища.

\smallskip
Институт проблем информатики Российской академии наук, редакционный совет
и~редакционная коллегия журнала <<Информатика и~её применения>>
выражают глубокое соболезнование родным и~близким покойного.