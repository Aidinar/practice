\def\stat{kozerenko}



\def\tit{ФАЛЬШТЕКСТЫ: КЛАССИФИКАЦИЯ И~МЕТОДЫ ОПОЗНАНИЯ
ТЕКСТОВЫХ ИМИТАЦИЙ И~ДОКУМЕНТОВ С~ПОДМЕНОЙ АВТОРСТВА$^*$}



\def\titkol{Фальштексты: классификация и~методы опознания
текстовых имитаций и~документов с~подменой авторства}

\def\aut{М.\,Ю.~Михеев$^1$, Н.\,В.~Сомин$^2$, И.\,В.~Галина$^3$,
О.\,В.~Золотарев$^4$, Е.\,Б.~Козеренко$^5$, Ю.\,И.~Морозова$^6$,
М.\,М.~Шарнин$^7$}

\def\autkol{М.\,Ю.~Михеев, Н.\,В.~Сомин, И.\,В.~Галина и др.}

\titel{\tit}{\aut}{\autkol}{\titkol}

{\renewcommand{\thefootnote}{\fnsymbol{footnote}} \footnotetext[1]
{Работа выполнена при финансовой поддержке РФФИ (проект 13-06-00402).}}


\renewcommand{\thefootnote}{\arabic{footnote}}
\footnotetext[1]{Научно-исследовательский вычислительный центр Московского
государственного университета им.\ М.\,В.~Ломоносова;
Институт проблем информатики Российской академии наук, m-miheev@rambler.ru}
\footnotetext[2]{Институт проблем информатики Российской академии наук, somin@post.ru}
\footnotetext[3]{Институт проблем информатики Российской академии наук, irn\_gl@mail.ru}
\footnotetext[4]{Российский новый университет, ol-zolot@yandex.ru}
\footnotetext[5]{Институт проблем информатики Российской академии наук, kozerenko@mail.ru}
\footnotetext[6]{Институт проблем информатики Российской академии наук, miss-yulia-morozova@yandex.ru}
\footnotetext[7]{Институт проблем информатики Российской академии наук, mc@keywen.com}

   \Abst{Современное текстовое пространство, включая Интернет, огромно и~постоянно
пополняется новыми текстами. Все текстовые документы можно разбить на две большие
группы: <<добросовестные тексты>> и~то, что можно назвать <<фальштекстами>>.
К~настоящему времени индустрия фальштекстов приобрела столь массовое
распространение, что возникает настоятельная потребность изучения этого явления
и~разработки действенных методов обнаружения подобных текстовых документов. Цель
на\-сто\-ящей статьи состоит в~том, чтобы дать адекватное описание понятия фальштекста
как информационного и~лингвистического феномена и~предложить некоторые подходы к
опознанию таких текстов.}

   \KW{порождение текста; обработка естественного языка; статистический анализ
языковых объектов; плагиат; типология фальштекстов}

\DOI{10.14357/19922264140409}


\vskip 12pt plus 9pt minus 6pt

\thispagestyle{headings}

\begin{multicols}{2}

\label{st\stat}

\section{Что такое фальштекст?}

   Задача исследования, представленного в~данной статье, состоит в~том,
чтобы найти под\-да\-ющи\-еся проверке признаки фальштекстов и~по ним
сделать выводы о степени их легитимности. Для этого постараемся выяснить
смысл понятий <<доб\-ро\-со\-вест\-ный текст>> и~<<фальштекст>>. Под
добросовестным текстом (текстовым документом) будем понимать текст,
несущий ка\-кую-либо содержательную информацию. К~добросовестным
текстовым документам относятся тексты различных жанров:
художественная, научная, инженерная литература, материалы в~средствах
массовой информации,
сообщения на форумах Интернета, в~социальных сетях и~прочие тексты,
име\-ющие хождение в~современном потоке документов и~относимые к точно
определенному автору или источнику.

Создание документа, имеющего
достаточно высокую степень признания в~обществе, требует значительных
трудовых и~финансовых затрат. Поэтому у~недобросовестных авторов
возникает соблазн добиться аналогичного социального эффекта с~по\-мощью
текстов, порождение которых требует гораздо меньших усилий. Подобные
тексты и~будем называть фальштекстами.

   Материалом исследований, проводимых авторами, являются только
   ес\-те\-ст\-вен\-но-язы\-ко\-вые текс\-ты, которые рассчитаны на прочтение
их людьми (а не программами). Текс\-ты на искусственных языках (например,
на языках программирования) не рассматриваются. Термин
<<добросовестный текст>> относится к способу порождения текста, а~не
к~информации, содержащейся в~том или ином высказывании, и~ее истинности.

Таким образом, постановка проблемы фальш\-текстов не выходит за рамки
анализа ес\-те\-ст\-вен\-но-язы\-ко\-во\-го текстового пространства.

   Для опознания фальштекстов применяются уже наработанные научные
методы, в~частности методы статистического и~структурного анализа
лингвистических объектов, теории информации.

   Итак, фальштекст~--- это текстовый документ, который маскируется его
создателем под добросовестный текстовый документ, при этом делаются
попытки изменить текст для получения нужного составителю социального
и~экономического эффекта, однако его <<фальшивость>> можно определить
путем текстологического анализа.

   Использование фальштекстов может быть об\-услов\-лено разными целями:
это похищение чужого текста, или плагиат; борьба за высокие рейтинги
   ин\-тер\-нет-сай\-тов (а~значит, высокую посещаемость сайтов
и,~соответственно, лучшую рекламу представленных на них товаров). Могут
быть и~более сложные случаи, например фальсификация, т.\,е.\ выдача
своего собственного текста за чужой, принадлежащий знаменитому автору.

   Следует подчеркнуть, что какие бы цели ни ставились, фальштексты,
безусловно, весьма нежелательны. В~ряде случаев запрет на использование
фальштекстов носит правовой характер, например в~случае плагиата или
контрафакта, которые нарушают или ущемляют авторские права. Но и~в тех
случаях, когда правовых запретов нет, фальштексты существенно ухудшают
информационную эффективность, засоряя информационное пространство
текстами низкого качества и~дезориентируя поисковые системы.

   К настоящему времени индустрия фальштекстов приобрела столь
массовое распространение, что возникает настоятельная потребность
изучения этого явления и~разработки действенных методов обнаружения
подобных текстовых документов. Фактически мы становимся свидетелями
возникновения новой отрасли информатики (пока еще не имеющей
общепринятого названия), предметом изучения которой являются
фальштексты.

   В настоящей статье рассматриваются некоторые первоначальные аспекты
проблематики фальштекстов. Объектом рассмотрения являются любые
естественно-языковые тексты, доступные в~электронном виде. Предметом
является исключительно информационный аспект проблемы~--- изучение
фальштекстов как информационного и~лингвистического феномена.

\section{Типология фальштекстов}

   Типологию фальштекстов начнем с~наиболее распространенного типа~---
плагиата, т.\,е.\ присваивания авторства на чужой текст. Обычная цель
плагиатора~--- незаслуженно присвоить себе определенную квалификацию
или популярность. Круг плагиаторов широк и~разнообразен: тут
и~копирование (полное или частичное) чужих художественных или научных
произведений, тут и~(возникающее как следствие первого) незаслуженное
присвоение ученых степеней и~званий, тут и~изначально не замечаемое
некритичное <<списывание>> студенческих курсовых работ. Плагиат
помимо того, что обесценивает квалификационные звания, резко ухудшает
эффективность учебного процесса, просто является обманом,
разновидностью кражи и~потому не\-до\-пус\-тим.

   Плагиат~--- достаточно сложное явление, и~поэтому сейчас в~обществе
нет четкого понимания, что считать плагиатом. Жестко установить границу
плагиата невозможно, и~на этот счет существуют разные мнения. Это
понятие обязательно будет <<плавать>> в~зависимости от моральных
установок общества и~его технических возможностей. Кроме того,
невооруженным глазом видно, что степени плагиата различны: от дословного
копирования до заимствования только нюансов самой идеи. Виды и~степени
плагиата, конечно, требуют тщательного изучения и~более подробной их
кодификации. Но назрела необходимость введения и~различения
по крайней мере двух уровней плагиата: <<жесткого>> плагиата, под
которым понимается грубое копирование достаточно больших фрагментов
текста, и~<<мягкого>> плагиата, когда воруется идея посредством тем или
иным способом скорректированного (и отличного от оригинала) текста.

   Плагиат следует отличать от контрафакта~--- несанкционированного
копирования чужих текстов (без присвоения авторства). Текстовый
контрафакт сейчас очень распространен, поскольку он может существенно
повысить рейтинг ин\-тер\-нет-ре\-сур\-са. Поэтому его выявление и~борьба
с~ним также являются весьма актуальной задачей.

   Другой большой класс фальштекстов~--- <<ге\-неранты>>, т.\,е.\ тексты,
практически полностью сгенерированные компьютером. Обычная цель
введения генерантов~--- повышение рейтинга \mbox{в~поисковых} системах.
Генеранты иногда предназначают для прочтения людьми~--- в~этом случае
они должны представлять собой осмысленные тексты. Однако по большей
части их назначение~--- ввести в~заблуждение программную систему. Для
этого в~соответствии с~алгоритмами поисковых машин применяется либо
генерация многочисленных текстов с~определенными ключевыми словами
(это так называемые <<дорвеи>>), либо встраивание таких блоков ключевых
слов в~посещаемые страницы. Аналогичные манипуляции выполняются
и~со ссылками на сайты. Если первый тип генерантов (для
   лю\-дей-по\-тре\-би\-те\-лей контента) можно назвать <<осмысленными
генерантами>>, то второй тип получил название <<сгенерированный шум>>,
поскольку подобные тексты сильно зашумляют информационное
пространство. Отметим, что все приемы веб-спа\-ма или <<сгенерированного
шума>>: <<скрытый текст>>, <<клоакинг>>, <<ссылочный спам>>,
<<дорвей>>, <<редирект>>, <<свопинг>> и~др.~[1--4],~--- хорошо
известны программистам, занимающимся продвижением сайтов, как
<<черные>> и~<<серые>> методы их раскрутки. К~сожалению, столь
некорректные методы сейчас приобретают все большее и~большее
распространение.

   Отметим, что методики машинной генерации фальштекстов часто
сочетаются с~плагиатом (контрафактом). С~одной стороны, сам плагиат
(а~точнее, его сокрытие) часто реализуется путем автомати\-че\-ской замены
некоторых слов на синонимы (с~\mbox{целью} затруднить обнаружение плагиата).
А~с~другой стороны, осмысленные генеранты часто используют фразы или
целые предложения из имеющихся в~Интернете источников. Поэтому имеет
смысл рассматривать еще один тип текстов: <<гибриды>>, сочетающие
в~себе характеристики плагиатов и~генерантов. Отметим, что гибриды, если
их качество достаточно высоко, могут приносить пользу и~не являться
фальштекстами. Например, к~таким гиб\-ри\-дам, генерирующим вполне
осмысленные тексты, принадлежит разработанная одним из авторов
настоящей статьи система автоматического построения энциклопедий
Keywen~[5].

\section{Пространство оценки фальштекстов}

   Для опознавания фальштекстов и~установле-\linebreak ния их типов необходима
разработка критериев и~характеристик, допускающих их объективное\linebreak
определение. Явление фальштекстов настолько нетривиально,
что выбрать единственный универсальный критерий для оценки таких
текстов невозможно. Поэтому предлагается оценивать фальштексты
   с~по\-мощью системы критериев, куда \mbox{входят}: оригинальность,
натуральность, лингвистичность и~информативность (сокращенно \mbox{ОНЛИ}).
Рассмотрим координаты этой системы и~взаимоотношение ее компонент.

   Два первых критерия~--- оригинальность и~натуральность~--- образуют
двумерную систему координат.

   Оригинальность есть свойство текстового документа текстуально или
идейно не зависеть от других имеющихся в~информационном пространстве
текс\-то\-вых документов. Наиболее важной тут является оригинальность
информации, новизна содержания, которые и~должны выражаться
в~текстуальной уникальности.

   Оригинальность может быть представлена в~виде непрерывной
<<шкалы>>, измеряемой в~долях или процентах, поскольку подавляющее
большинство текстов оригинально лишь в~ка\-кой-то степени и~неизбежно
содержит определенную долю заимствованных фрагментов. Причем под
фрагментами понимаются не отдельные слова, а~достаточно длинные части
текста.

   Разумеется, далеко не каждое заимствование текста является плагиатом
или контрафактом. В~огромном числе случаев копирование чужого текс\-та не
влечет негативных последствий или даже желательно, так как подтверждает
актуальность исходного текс\-та. Поэтому в~критерий оригинальности
помимо процента заимствований входят признаки ситуации,
сигнализирующие о~наличии или отсутствии плагиата или контрафакта.
Например, факт публикации чужого текста под именем и~фамилией
публикатора без ссылки на источник является наиболее очевидным
признаком наличия плагиата. Поскольку эти признаки влекут правовые
последствия, то их идентификация является одной из важных задач изучения
фальштекстов. С~другой стороны, как уже указывалось, у~плагиата нет
четкой границы, и~потому сформулировать их раз и~навсегда не
представляется возможным.

   Натуральность выражает качество текста в~смыс\-ле его приближения
к~естественному, написанному человеком художественному, научному или
техническому тексту. Натуральность про\-ти\-во\-по\-став\-ля\-ет\-ся некорректности,
искусственности текста, что связано с~недавно появившимися и~ныне
широко применяющимися методами автоматической компьютерной
генерации текстов.

   По мнению авторов, натуральность является важнейшей характеристикой
именно в~контексте рассмотрения фальштекстов. Она позволяет
идентифицировать и~отсеивать многие виды некорректных генерантов. Дело
в~том, что качественное автоматическое порождение текста является одной
из труднейших научных проблем и~требует для своей реализации больших
ресурсов, в~частности финансовых. Но, как утверждается
   в~работе~\cite{1-koz}, порождение генерантов выгодно только в~случае
затраты на них незначительных средств. Поэтому натуральность генерантов
неизбежно является невысокой (хотя, вероятно, постепенно будет
увеличиваться), что и~может служить их идентифицирующим признаком.
Отметим, что натуральность, как и~оригинальность, представляется
непрерывной шкалой.
{\looseness=1

}

   Вводя понятие натуральности, авторы преследовали цель дать
интегральный критерий качества\linebreak текста. Анализируя этот критерий, можно
заметить, что натуральность~--- сложное понятие, имеющее по крайней мере
две составляющие: лингвистическую
   и~ин\-фор\-ма\-тив\-но-со\-дер\-жа\-тель\-ную. \mbox{Поэтому} натуральность
имеет смысл представлять как суперпозицию двух частных критериев~---
лингвистичности и~информативности.

   Лингвистичность~--- это языковая правильность текста, соответствие
законам языка, включая лексику, морфологию, синтаксис, семантику,
прагматику. Степень нарушения лингвистичности может быть разная:
начиная с~отсутствия общей темы и~потери связности сообщения при
сохранении смысла отдельных предложений и~вплоть до текста,
содержащего бессвязный набор лексем.

   Информативность есть свойство текста представлять собой осмысленное
для человека сообщение. Отметим, что в~понятие информативности входят
далеко не все содержательные параметры сообщения. В~част\-ности,
сообщение может быть истинным или ложным, положительным или
негативным, провокационным или нейтральным~--- все это в~понятие
информативности не входит, а~входит лишь информационная насыщенность
текста. Разбирая понятие информативности, многие авторы указывают на его
субъективный характер~--- информативность текста зависит от уровня
знаний воспринимающего~[6]. Безусловно, это так. Но этот факт не умаляет
интегральных оценок информационной насыщенности текста, таких как
индекс цитируемости.

   Таким образом, система критериев ОНЛИ, на основании которых
предполагается изучение фальш\-текс\-тов, выглядит так, как это показано на
рис.~1.

\begin{center}  %fig1
\vspace*{12pt}
 \mbox{%
 \epsfxsize=73.279mm
 \epsfbox{koz-1.eps}
 }

 \vspace*{11pt}


{{\figurename~1}\ \ \small{Система критериев оценки фальштекстов ОНЛИ}}
 \end{center}

\vspace*{12pt}

\addtocounter{figure}{1}


   Следует отметить, что система ОНЛИ обладает значительной гибкостью
при ее приложении к~различным частным задачам исследования
фальш\-текстов.
Так, применяя отдельные критерии, \mbox{мож\-но} независимо
исследовать: плагиат и~контрафакт (оригинальность), общую
приближенность текста к <<нормальному, человеческому>> (натураль-\linebreak ность),
языковую правильность текста (лингви\-стичность) и~информационную
ценность текста (информативность). Можно рассматривать текс\-ты
в~двумерном пространстве ори\-ги\-наль\-ность--на\-ту\-раль\-ность,
   ори\-ги\-наль\-ность--лингви\-стич\-ность,\linebreak\vspace*{-12pt}

\columnbreak

\begin{center}  %fig2
\vspace*{1pt}
 \mbox{%
 \epsfxsize=76.684mm
 \epsfbox{koz-2.eps}
 }

 \end{center}

 \vspace*{-9pt}


\noindent
{{\figurename~2}\ \ \small{Основные типы фальштекстов в~пространстве
   оригинальность--натуральность}}


\vspace*{14pt}



\noindent
   оригиналь\-ность--ин\-фор\-ма\-тив\-ность. Наконец,
можно
абстрагироваться от проблем копирования
 и~плагиата, исследовать текст
в~более привычных для лингвиста координатах
   лингви\-стич\-ность--ин\-фор\-ма\-тив\-ность. Например, рассмотрение
проблемы в~координатах ори\-ги\-наль\-ность--на\-ту\-раль\-ность позволяет
наглядно представить основные типы фальштекстов (рис.~2.)
   При этом каждый тип текстов занимает свою нишу.


   Добросовестные
тексты характеризуются и~высокой степенью оригинальности, и~достаточно\linebreak
высокой натуральностью. Некорректные сгенерированные тексты
(<<сгенерированный шум>>) отличаются от добросовестных текстов гораздо
меньшей степенью натуральности. И~наконец, плагиаты (<<жесткие>>)
отличаются от добросовестных текстов гораздо меньшей степенью
оригинальности.

   Отметим, что границы между типами нельзя считать жесткими~--- типы
отчасти перекрывают друг друга и~плавно переходят один в~другой. Так,
между фальштекстами и~добросовестными текстами лежит <<нейтральная
полоса>>~--- тексты, занимающие промежуточное положение. Эту полосу,
с~одной стороны, занимают <<мягкие плагиаты>>, а~с~другой~---
<<осмысленные генеранты>>. Поскольку для создания плагиатов часто
используется про\-грам\-мная генерация текстов, то возникают всевозможные
гибриды.

   Необходимо указать, что все предложенные критерии, хотя и~являются
интуитивно ясными, но эксплицировать их и~привязать к определенным
легко вычисляемым параметрам документа оказывается непростой задачей.
Поэтому необходима разработка конкретных методов и~систем оценки этих
критериев. В~двух последующих разделах настоящей статьи предлагаются
некоторые соображения на этот счет.

\section{Методы оценки оригинальности текстов}

   Прежде всего, отметим, что оригинальность текс\-та выявляет любая
система обнаружения плагиата (СОП), так что в~настоящий момент оценка
оригинальности текстов неразрывно связана с~созданием СОП. Рассмотрим
несколько примеров.

   Распространенная у нас в~России система <<eTXT Антиплагиат>>~[7]
оценивает оригинальность в~процентах. Принцип работы системы~---
сравнение текс\-та с~базой данных оригинальных текс\-тов. Этот\linebreak принцип далек
от совершенства, поскольку поддерживать в~актуальном состоянии такую
базу текс\-тов~--- задача трудновыполнимая. Хотя, как утверждается, системы,
основанные на принципе сопоставле\-ния с~базой данных, успешно работают.
Примером может служить ин\-тер\-нет-сер\-вис Turnitin, разработанный
в~Университете Роберта Гордона из Абердина (Шотландия). Общий объем
базы текстов~--- 25~ТБ, причем база пополняется со скоростью 40~млн
новых страниц в~день~[8].

   Имеются системы, которые осуществляют поиск текс\-тов-ана\-ло\-гов
прямо в~Интернете, например: Advego Plagiatus ({\sf
http://advego.ru/ plagiatus}), Praide Unique Content Analyser~II ({\sf
http://www.nado.su/downloads.html}), он\-лайн-сер\-ви\-сы {\sf
www.miratools.ru, www.istio.com}.

   Отметим, что параллельно с~этим существует также и~несколько
<<антисистем>>, помо\-га\-ющих недобросовестным авторам преодолеть СОП.\linebreak
К~таким программам относятся, например, программы Smartrewriter pro ({\sf
http://www.райхан.рф/ raskrutka-bloga/predlagayu-vam-programmu-dlya-xoro
shego-rerajta/}) и~Анти Плагиат Killer ({\sf http:// antiplagius.ru/}),
нацеленная на <<обман>> системы <<Антиплагиат>>. Чтобы замаскировать
плагиат под оригинальный текст, производятся следующие преобразования
текстовых фрагментов:
   \begin{itemize}
\item замена слова на синонимичное ему слово;
\item добавление новых слов;
\item удаление слов;
\item разбиение одного предложения на два;
\item объединение двух предложений в~одно.
\end{itemize}

   В целом можно сделать вывод, что сейчас ве\-дется нешуточная война
между разработчиками\linebreak сис\-тем антиплагиата и~плагиаторами, которые
изобретают все более изощренные способы <<уникализации>> текста. Пока
<<плагиатчикам>> удается работать с~опережением.

   Поскольку работа по выявлению плагиата связана с~множеством тонких
моментов, требующих вмешательства специалиста высокой квалификации,
то имеет смысл говорить о~рабочем месте <<антиплагиатора>>, т.\,е.\
о~комплексе программных средств, объединенных в~единую систему,
которая давала бы возможность специалисту воспользоваться теми или
иными инструментами для наиболее точного определения степени
оригинальности документа~[9].


\section{Методы определения лингвистичности
и~информативности текстов}

   Как уже указывалось выше, натуральность текста является обобщающим
критерием, который разбивается на два более специальных: лингвистичность
и~информативность.

   Насколько известно авторам, ка\-ких-то завершенных реально
действующих систем, оценива\-ющих лингвистичность текста, не существует.
Однако методы лингвистики активно используются для выявления
некорректных генерантов. Так,\linebreak в~работе~[10] используется ряд
лингвистических методов для обнаружения <<дорвеев>>~--- текстов,
имитирующих натуральные тексты, но являющихся искусственными
и~содержащих большое количество ключевых слов для повышения
поискового рейтинга определенной рекламной страницы.

   Оценка информативности текста является весьма сложной задачей. Для ее
решения может быть привлечена следующая гипотеза: документ имеет тем
более высокую информативность, чем больше имеется ссылок на него или
чем чаще он подвергается корректному цитированию. Конечно, эта гипотеза
не гарантирует выявления всех информативных документов. Однако их
основную массу можно определить с~ее помощью и,~более того, можно
оценить степень информативности конкретного документа. На основе этой
гипотезы одним из авторов настоящей работы был разработан метод оценки
информативности, реализованный в~автоматизированной системе
построения текстовых энциклопедий Keywen~[11]. Суть метода~---
в~отыскании в~среде Интернета множества текстовых документов,
связанных с~определенной предметной областью, и~вычислении их
рейтингов на основе корректного их цитирования. Этот рейтинг
и~принимается за оценку информативности текста. Такой метод сродни
установлению рейтинга сайтов в~поисковых системах с~использованием
индекса цитирования, но касается он не сайтов, а~отдельных документов.
Несомненно, такой подход может дать в~большинстве случаев вполне
приемлемый результат, поскольку публикация научных статей в~открытом
доступе в~Интернете приобретает все б$\acute{\mbox{о}}$льшую популярность. Отчасти это
связано с~тем, что, как показал исследователь из Вильнюса Альгирдас
Аушра~[8], такие работы цитируются намного активнее, чем доступные по
платной подписке или в~библиотеке.

   Следует, однако, отметить, что цитируемость документа далеко не всегда
идентична его информационной ценности. Например, только что
опуб\-ли\-ко\-ван\-ная статья, содержащая новые научные идеи, может по этой
методике получить низкий рейтинг. Априори ценность этих новых идей
неизвестна, и~поэтому должно пройти определенное время, за которое
научное сообщество оценит их и~<<проголосует>> либо за них~---
увеличением индекса цитируемости, либо против них, оставив статью без
ссылок. Отсюда видна ограниченность рейтингового подхода, который
оценивает информативность в~статике, не учитывая динамики
информационных процессов. В~связи с~этим встает проблема определения
времени <<отстоя>> документа, т.\,е.\ временн$\acute{\mbox{о}}$го промежутка, за который
научное сообщество <<проголосует>> за него своими ссылками.

\vspace*{-6pt}

\section{Перспективы дальнейших исследований}

   Можно указать на несколько направлений дальнейших исследований
фальштекстов.
\begin{enumerate}[1.]
\item  Необходимо продолжить исследование их типологии. Сегодня
информационное сообщество имеет дело только с~описанными в~статье
разновидностями фальштекстов. Но не исклю\-чено, что в~будущем появятся
другие разновидности~--- ведь ввести в~заблуждение человека (или систему)
и~получить от этого определенный выигрыш можно самыми разными
способами.
\item
Увеличение числа типов фальш-до\-ку\-мен\-тов потребует
введения новых критериев оценки в~дополнение к уже предложенным.
\item
   Необходимо разработать эффективные методы оценки уже
выявленных критериев \mbox{ОНЛИ}. В~частности, тут могут быть использованы
методы компьютерного семантического анализа~[5].

   \item Дальнейшей задачей могло бы стать лингвистическое
и~информационное описание методов, которыми пользуются <<заказчики>>
и~<<исполнители>> фальш\-текс\-тов, и~выработка рекомендаций для их
эффективного выявления и~устранения.

\item Необходима разработка практически удобных
и~эффективных программных средств обнаружения и~обезвреживания
фальш\-текс\-тов. В~будущем можно ждать появления как систем, работающих
автоматически, так и~<<рабочих мест антиплагиатора>>, предоставляющих
набор инструментов, способствующих выявлению фальш\-текстов.
\end{enumerate}

   Проблема фальшдо\-ку\-мен\-тов только ставится на повестку дня. Будем
надеяться, что, как в~свое время с~компьютерными вирусами, эта проблема
будет со временем удовлетворительно решена.

{\small\frenchspacing
 {%\baselineskip=10.8pt
 \addcontentsline{toc}{section}{References}
 \begin{thebibliography}{99}

\bibitem{1-koz}
\Au{Дёнди З., Гарсия-Молина Г.} Таксономия веб-спа\-ма. {\sf
http://wseob.ru/seo/web-spam-taxonomy}.

\bibitem{3-koz} %2
\Au{Hall S.} On postmodernism and articulation: An interview with Stuart Hall~//
J.~Communication Inquiry, 1986. No.\,5. P.~35--60.
doi: 10.1177/019685998601000204.


\bibitem{4-koz} %3
\Au{Baggaley J., Spencer B.} The mind of a plagiarist~// Learning Media Technol.,
2005. Vol.~30. No.\,1. P.~55--62.
\bibitem{2-koz} %4
\Au{Selber S.\,A., Johnson-Eilola J.} Plagiarism, originality, assemblage~//
Comput. Composition, 2007. Vol.~24. No.\,4. P.~375--403.
\bibitem{5-koz}
Keywen: Encyclopedia of keywords, key-phrases \& key ideas. {\sf  www.keywen.com}.
\bibitem{6-koz}
\Au{Шрейдер Ю.\,А.} Об одной модели семантической информации~// Проблемы
кибернетики.~---  М.: Наука, 1965. Вып.~13. С.~233--240.
\bibitem{7-koz}
eTXT Антиплагиат: Система проверки уникальности текста. {\sf http://www.etxt.ru/antiplagiat}.
\bibitem{8-koz}
\Au{Аушра А.} Научная электронная библиотека как средство борьбы с~плагиатом~//
J.~Educ. Technol.  Soc., 2006. Vol.~9. Iss.~3. P.~270--276.
\bibitem{9-koz}
\Au{Дягилев В.\,В., Цхай А.\,А., Бутаков~С.\,В.} Архитектура сервиса определения
плагиата, исключающая возможность нарушения авторских прав~// Вестник НГУ. Сер.
Информационные технологии, 2011. Т.~9. Вып.~3. С.~23--29.
\bibitem{10koz}
\Au{Павлов А.\,С., Добров Б.\,В.} Метод определения массово порождаемых
неестественных текстов~// Компьютерная лингвистика и~интеллектуальные технологии:
По мат-лам ежегодной Междунар. конф. <<Диалог>>.~--- М.:
РГГУ, 2010. Вып.~9(16). С.~368--374.
\bibitem{11-koz}
\Au{Кузнецов И.\,П., Шарнин М.\,М., Мацкевич~А.\,Г.} Интеллектуальные механизмы
семантического поиска в~сети Интернет~// Системы и~средства информатики, 2012.
Т.~22. №\,2. С.~129--145.
 \end{thebibliography}

 }
 }

\end{multicols}

\vspace*{-9pt}

\hfill{\small\textit{Поступила в редакцию 01.11.14}}

\newpage

%\vspace*{12pt}

%\hrule

%\vspace*{2pt}

%\hrule

%\vspace*{12pt}

\def\tit{FALSE TEXTS: CLASSIFICATION AND~METHODS OF~IDENTIFICATION
OF~TEXT DOCUMENTS WITH~IMITATIONS~AND~SUBSTITUTION OF~AUTHORSHIP\\[-4pt]}

\def\titkol{False texts: Classification and methods of identification of text documents with
imitations  and substitution of authorship}

\def\aut{M.\,Yu.~Mikheev$^{1,2}$, N.\,V.~Somin$^2$,
I.\,V.~Galina$^2$, O.\,V.~Zolotaryev$^3$,
E.\,B.~Kozerenko$^2$, Yu.\,I.~Morozova$^2$,
and~M.\,M.~Charnine$^2$\\[-8pt]}

\def\autkol{M.\,Yu.~Mikheev, N.\,V.~Somin,
I.\,V.~Galina, et al.}

\titel{\tit}{\aut}{\autkol}{\titkol}

\vspace*{-14pt}

\noindent
$^1$Research Computer Center, M.\,V.~Lomonosov Moscow State
Uviversity (MGU NIVC),  1-52  Leninskiye Gory,\linebreak
$\hphantom{^1}$GSP-1, Moscow 119991, Russian Federation


\noindent
$^2$Institute of Informatics Problems, Russian Academy of Sciences,
44-2~Vavilov Str., Moscow 119333, Russian\linebreak
$\hphantom{^1}$Federation

\noindent
$^3$Russian New University, 22 Radio Str., Moscow
105005, Russian Federation


\def\leftfootline{\small{\textbf{\thepage}
\hfill INFORMATIKA I EE PRIMENENIYA~--- INFORMATICS AND
APPLICATIONS\ \ \ 2014\ \ \ volume~8\ \ \ issue\ 4}
}%
 \def\rightfootline{\small{INFORMATIKA I EE PRIMENENIYA~---
INFORMATICS AND APPLICATIONS\ \ \ 2014\ \ \ volume~8\ \ \ issue\ 4
\hfill \textbf{\thepage}}}

\vspace*{2pt}



\Abste{Modern textual space, including the Internet, is enormous and is constantly
updated with new texts. All text documents can be divided into two large groups:
``good texts'' and that might be called ``false texts.'' So far, the industry of false
texts flow production has become so massive that there is an urgent need to study
this phenomenon and to develop effective methods of detection of such text
documents. The purpose of the paper is to give an adequate description of the
concept of false text as information and linguistic phenomenon and suggest some
approaches to the identification of such texts.}

\vspace*{-9pt}

\KWE{text generation; natural language processing; statistical analysis of
language objects; plagiarism; typology of false texts}

\DOI{10.14357/19922264140409}

\vspace*{-24pt}

\Ack

\vspace*{-4pt}

\noindent
The research was financially supported by the Russian Foundation for Basic Research
(project 13-06-00402).


\vspace*{-2pt}

  \begin{multicols}{2}



\renewcommand{\bibname}{\protect\rmfamily References}
%\renewcommand{\bibname}{\large\protect\rm References}



{\small\frenchspacing
 {\baselineskip=10.8pt
 \addcontentsline{toc}{section}{References}
 \begin{thebibliography}{99}

\bibitem{1-koz-1}
\Aue{Gyongyi, Z., and H. Garcia-Molina}. 2007. Taksonomiya web-spama [The
taxonomy of web spam]. Available at: {\sf
http://wseob.ru/seo/web-spam-taxonomy} (accessed November~01, 2014).

\bibitem{3-koz-1} %2
\Aue{Hall, S., and L.~Grossberg}. 1986. On postmodernism and articulation: An
interview with Stuart Hall. \textit{J.~Communication Inquiry} 5:35--60. doi:
10.1177/019685998601000204.
\bibitem{4-koz-1} %3
\Aue{Baggaley, J., and B.~Spencer}. 2005. The mind of a plagiarist.
\textit{Learning Media Technol.} 30(1):55--62.
\bibitem{2-koz-1} %4
\Aue{Selber, S.\,A., and J.~Johnson-Eilola}. 2007. Plagiarism, originality,
assemblage. \textit{Comput. Composition} 24(4):375--403.
\bibitem{5-koz-1}
Keywen: Encyclopedia of keywords, key-phrases \& key ideas. Available at: {\sf
www.keywen.com} (accessed November~01, 2014).
\bibitem{6-koz-1}
\Aue{Shrader, J.\,A.} 1965. Ob odnoy modeli semanticheskoy informatsii [On one
model of semantic information]. \textit{Problemy kibernetiki} [Problems of
cybernetics]. Moscow: Nauka.  13:233--240.
\bibitem{7-koz-1}
eTXT Antiplagiat: Sistema proverki unikal'nosti teksta [eTXT Antiplagiarism: The
system to verify the uniqueness of the text]. Available at: {\sf
http://www.etxt.ru/antiplagiat} (accessed November~01, 2014).
\bibitem{8-koz-1}
\Aue{Aushra, A.} 2006.  Nauchnaya elektronnaya biblioteka kak sredstvo bor'by s
plagiatom [Scientific electronic library as a means of combating plagiarism].
\textit{J.~Educ. Technol. Soc.} 9(3):270--276.
\bibitem{9-koz-1}
\Aue{Diaghilev, V.\,V., A.\,A.Tskhai, and  S.\,V.~Butakov}. 2011. Arkhitektura
servisa opredeleniya plagiata isklyuchayuschaya vozmozhnost' narusheniya
avtorskikh prav [Architecture of the service for detection of plagiarism which
excludes the possibility of copyright infringement]. \textit{Vestnik NGU. Ser.
Informatsionnye Tekhnologii} [Herald of the NSU. Information
technology ser.]. 9(3):23--29.
\bibitem{10-koz-1}
\Aue{Pavlov, A.\,S., and B.\,V.~Dobrov}. 2010. Metod opredeleniya massovo
porozhdaemykh neestestvennykh tekstov [Mass generation unnatural texts
detection method]. \textit{Komp'uternaya lingvistika i~intellektual'nye
tekhnologii: Po mat-lam Ezhegodnoi Mezhdunar. konf. ``Dialog''}
[Computational Linguistics and Intelligent Technologies:
based on the Proceedings of the ``Dialog'' Annual Conference]. Moscow: RSUH. 9(16):368--374.
\bibitem{11-koz-1}
\Aue{Kuznetsov, I.\,P., M.\,M.Sharnin, and A.\,G.~Matskevich}. 2012.
Intellektual'nye mekhanizmy semanticheskogo poiska v~seti Internet [Intellectual
mechanisms for semantic searching in Internet]. \textit{Sistemy i~Sredstva
Informatiki}~--- \textit{Systems and Means of Informatics} 22(2):129--145.

\end{thebibliography}

 }
 }

\end{multicols}

\vspace*{-12pt}

\hfill{\small\textit{Received November 01, 2014}}

%\pagebreak

%\vspace*{-18pt}


\Contr

\noindent
\textbf{Mikheev Michael Yu.} (b.\ 1957)~--- Doctor of Sciences in philology,
leading scientist, Research Computer Center, M.\,V.~Lomonosov Moscow State
Uviversity (MGU NIVC),  1-52  Leninskiye Gory, GSP-1, Moscow 119991, Russian Federation;
leading scientist, Institute of Informatics Problems, Russian
Academy of Sciences, 44-2~Vavilov Str., Moscow 119333, Russian Federation;
m-miheev@rambler.ru


\vspace*{3pt}

\noindent
\textbf{Somin Nicolay V.} (b.\ 1947)~--- Candidate of Science (PhD) in physics
and mathematics, leading scientist, Institute of Informatics Problems, Russian
Academy of Sciences, 44-2~Vavilov Str., Moscow 119333, Russian Federation;
somin@post.ru

\vspace*{3pt}

\noindent
\textbf{Galina Irina V.} (b.\ 1967)~--- Senior scientist, Institute of Informatics
Problems, Russian Academy of Sciences, 44-2~Vavilov Str., Moscow 119333,
Russian Federation; irn\_gl@mail.ru

\vspace*{3pt}

\noindent
\textbf{Zolotaryev Oleg V.} (b.\ 1959)~--- Candidate of Science (PhD) in
technology, associate professor, Russian New University, 22 Radio Str., Moscow
105005, Russian Federation; ol-zolot@yandex.ru

\vspace*{3pt}

\noindent
\textbf{Kozerenko Elena B.} (b.\ 1959)~--- Candidate of Science (PhD) in
linguistics, Head of Laboratory, Institute of Informatics Problems, Russian
Academy of Sciences, 44-2~Vavilov Str., Moscow 119333, Russian Federation;
kozerenko@mail.ru

\vspace*{3pt}

\noindent
\textbf{Morozova Yulia I.} (b.\ 1984)~--- scientist, Institute of Informatics
Problems, Russian Academy of Sciences, 44-2~Vavilov Str., Moscow
119333, Russian Federation; miss-yulia-morozova@yandex.ru

\vspace*{3pt}

\noindent
\textbf{Charnine Mikhail M.} (b.\ 1959)~--- Candidate of Science (PhD) in
technology, senior scientist, Institute of Informatics Problems, Russian Academy
of Sciences, 44-2~Vavilov Str., Moscow 119333, Russian Federation;
mc@keywen.com


\label{end\stat}

\renewcommand{\bibname}{\protect\rm Литература}