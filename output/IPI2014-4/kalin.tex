\renewcommand{\figurename}{\protect\bf Figure}
\renewcommand{\tablename}{\protect\bf Table}

\def\stat{kalinich}


\def\tit{CONCEPTUAL MODELING OF~MULTIDIALECT WORKFLOWS}

\def\titkol{Conceptual modeling of~multidialect workflows}

\def\autkol{L.~Kalinichenko, S.~Stupnikov, A.~Vovchenko,
and~D.~Kovalev}

\def\aut{L.~Kalinichenko$^{1,2}$, S.~Stupnikov$^1$, A.~Vovchenko$^1$,
and~D.~Kovalev$^1$}

\titel{\tit}{\aut}{\autkol}{\titkol}

%{\renewcommand{\thefootnote}{\fnsymbol{footnote}}
%\footnotetext[1] {}}

\renewcommand{\thefootnote}{\arabic{footnote}}
\footnotetext[1]{Institute of Informatics Problems, Russian Academy of Sciences,
44-2 Vavilov Str., Moscow 119333, Russian Federation}
\footnotetext[2]{Faculty of
Computational Mathematics and Cybernetics, M.\,V.~Lomonosov Moscow State University,
1-52~Leninskiye Gory, GSP-1, Moscow 119991, Russian Federation}


%\vspace*{6pt}

\def\leftfootline{\small{\textbf{\thepage}
\hfill INFORMATIKA I EE PRIMENENIYA~--- INFORMATICS AND APPLICATIONS\ \ \ 2014\ \ \ volume~8\ \ \ issue\ 4}
}%
 \def\rightfootline{\small{INFORMATIKA I EE PRIMENENIYA~--- INFORMATICS AND APPLICATIONS\ \ \ 2014\ \ \ volume~8\ \ \ issue\ 4
\hfill \textbf{\thepage}}}

%\vspace*{6pt}


\Abste{This paper contributes to the techniques for conceptual representation of
data analysis algorithms and data integration facilities as well as processes to
specify data and behavior semantics in one paradigm. An investigation of a~novel
approach for applying a~combination of semantically different
platform-independent rule-based languages (dialects) for interoperable conceptual
specifications over various rule-based systems (RSs) relying on the rule-based
program transformation technique recommended by the W3C Rule Interchange
Format (RIF) is extended here. Such approach is combined with the facilities
aimed at the semantic rule-based mediation intended for the heterogeneous data
base integration. This paper extends a~previous research of the authors in the
direction of workflow modeling for definition of compositions of algorithmic
modules in a~process structure. A~capability of the multidialect workflow
support specifying the tasks in semantically different languages mostly suited to
the task orientation is presented. A~practical workflow use case, the
interoperating tasks of which are specified in several rule-based languages
(RIF-CASPD, RIF-BLD, RIF-PRD), is introduced. In addition, OWL~2 is used
for the conceptual schema definition, RIF-PRD is used also for the workflow
orchestration. The use case implementation infrastructure includes a~production
rule-based system (IBM ILOG), a~logic rule-based system (DLV), and
a~mediation system.}

\KWE{conceptual specification; workflow; RIF; production rule languages;
database integration; mediators; PRD; multidialect infrastructure}

\DOI{10.14357/19922264140413}

%\vspace*{6pt}


\vskip 12pt plus 9pt minus 6pt

      \thispagestyle{myheadings}

      \begin{multicols}{2}

                  \label{st\stat}

\section{Introduction}

  \noindent
  This work keeps on the intention of developing the facilities for conceptual
declarative problem specification and solving in data intensive domains (DID). In [1]
it was claimed that conceptual data semantics alone (e.\,g., formalized in ontology
languages based on description logic) are insufficient, so that conceptual
representation of data analysis algorithms as well as processes for problem solving are
required to specify data and behavior semantics in one paradigm.

The results presented in this paper\footnote[3]{This paper is an extended for the journal version of
the results presented in the ``Multidialect Workflows'' report at the ADBIS'2014
Conference.} extend the research~[1] aimed at the definition and implementation of the
facilities for conceptually-driven problems specification and solving in DID aiming at
ensuring eventually the following capabilities for expressing the specifications:
\begin{enumerate}[(1)]
\item an ability to provide complete and precise specification of the abstract
structure and behavior of the domain entities, their consistency, relationship, and
interaction;
\item well-grounded diversity of semantics of the modeling facilities providing for
the best attainable expressiveness, compactness, and precision of the definition of
the problem solving algorithm specifications;
\item arrangements for the extensions of the modeling facilities satisfying the
changing technological and practical needs;
\item specification independence from implementation platforms (languages,
systems);
\item specification independence from concrete information resources (IRs)
(databases,
services, ontologies, etc.)\ combined with facilities for their semantic
integration and interoperability; and
\item built-in methodologies for creation of unifying specification languages
providing for construction of semantics-preserving mappings of conceptual
specifications into their implementations in specific platforms.
\end{enumerate}

  The research reported in~[1] investigated the conceptual modeling facilities for
DID applying rule-based declarative logic languages possessing different,
complementary semantics and capabilities combined with the methods and languages
for heterogeneous data mediation and integration. Two fundamental techniques were
combined: ($i$)~constructing of the unifying extensible language providing for
semantics-preserving mapping into it of various IR
specification languages (e.\,g., such as data definition (DDL) and
data manipulation (DML) languages for databases); and ($ii$)~creation of
the unified extensible family of rule-based languages (dialects) and a~model of
interoperability of the programs expressed in such dialects with different semantics.

  The first technique is based on the experience obtained in course of the
SYNTHESIS language development~[2]. The kernel of the SYNTHESIS language is
based on the object-frame data model used together with the declarative rule-based
facilities in the logic language similar to a~stratified Datalog with functions and
negation. The extensions of the kernel are constructed in such a~way that each
extension together with the kernel is a~result of semantic preserving mapping of some
IR language into the SYNTHESIS~[2]. The canonical information model is
constructed as a~union of the kernel with such extensions defined for various resource
languages. Canonical model is used for development of \textit{mediators} positioned
between the users, conceptually formulating problems in terms of the mediators, and
distributed resources. A~schema of a~subject mediator for a~class of problems
includes the specification of the domain concepts defined by the respective
ontologies.
{ %\looseness=1

}

  Another, multidialect technique for rule-based programs interoperability applied is
based on the RIF standard [3] of W3C. The RIF standard introduces a~unified family of rule-based
languages together with a~methodology for constructing of semantic preserving
mappings of specific languages used in various RSs into RIF
dialects. Examples of RSs include {SILK}, {OntoBroker}, {DLV},
{IBM Websphere ILOG JRules}, {RIF4J}\;+\;{IRIS}, and others
(more examples can be found at {\sf
http:// www.w3.org/2005/rules/wiki/Implementations}). From the RIF point of view, an
IR is a~program developed in a~specific language of some RS.

  In [1], the first results obtained were presented including the description of an
approach and an infrastructure supporting:
  \begin{itemize}
\item application domain conceptual specification and problem solving algorithms
definitions based on the combination of the heterogeneous database mediation
technique and the rule-based multidialect facilities;
\item interoperability of distributed multidialect rule-based programs and mediators
integrating heterogeneous databases; and
\item rule delegation approach for the peer interactions in the multidialect
environment.
\end{itemize}

  The proof-of-concept prototype of the infrastructure based on the SYNTHESIS
environment and RIF standards has been implemented. The approach for multidialect
conceptualization of a~problem domain, rule delegation, rule-based programs, and
mediators interoperability were explained in detail and illustrated on an use-case in
the finance domain~[1]. For the conceptual definition of the use-case problem, the
OWL was used for the domain concepts definition and two RIF logic dialects
RIF-BLD~[4] and RIF-CASPD~[5] were used and mapped for implementation into the
SYNTHESIS formula language and the ASP (answer set programming)
based DLV~[6] language, respectively.

  The results obtained so far are quite encouraging for future work: they show that
the mentioned in the beginning capabilities~(1)--(6) sought for conceptual modeling
become feasible. This paper reports the results of extending the research in the
direction of modeling of the processes for the problem solving following the approach
briefly outlined above. These results include extensions of the infrastructure and
specification languages considered in~[1] to the workflow level keeping the same
approach and paradigm as well as aiming at the capabilities of the
conceptualization~(1)--(6) that were stated in~[1] and mentioned in the beginning of
the introduction.

  For investigation of such extension with respect to the choice of rule-based languages, it was
decided not to go outside the limits of the existing set of the published RIF dialects.
Such decision would allow to retain well-defined semantics of the conceptual
rule-based languages with a~possibility to check preservation of their semantics by various
languages of the implementing systems.

  The production rule dialect RIF PRD~[7] has been chosen as the language for the
workflow modeling in such a~way that the tasks of the workflow can have
multidialect rule-based representation (as defined in~[1]). This paper reporting the results
of such investigation is structured as follows. To make the paper self-contained, the
next section provides a~brief overview of the infrastructure supporting multidialect
programming defined in details in~[1]. Here, it is stressed
 that this infrastructure is
suitable for the workflow tasks specification. Workflow-oriented extension of the
multidialect infrastructureis considered in section~3. Use case implementation in
the proof-of-concept prototype is given in section~4. Related works are reviewed
in section~5. Concluding Remarks summarize contributions of the research.

%\vspace*{-24pt}

\begin{figure*} %fig1
\vspace*{1pt}
 \begin{center}
 \mbox{%
 \epsfxsize=164.734mm
 \epsfbox{kal-1.eps}
 }
 \end{center}
 \vspace*{-9pt}
\Caption{Conceptual schema and peer specifications }
\vspace*{-2pt}
  \end{figure*}

\section{Basic Principles of the Workflow Tasks Representation
in~the~Multidialect Infrastructure}

  \noindent
  Each workflow task (besides those that for pragmatic reasons are defined as
externally specified functions) is assumed to be represented in the novel infrastructure
defined in details in~[1]. Conceptual programming of tasks is performed using the
RIF dialects (now not only logic but also PRDs can be
used).

Conceptual tasks are implemented by their transformation into the rule-based
programs of the respective RSs and mediation systems (MSs). \textit{Conceptual
specification of a~task} is defined in the context of a~subject domain and consists of
a~set of RIF-documents (document is a~specification unit of RIF). The
\textit{conceptual schema} of the domain is defined using OWL 2~\cite{8-kal}
ontologies. Such usage of ontology is analogous to~\cite{22-kal}; however, it is
specifically important in the multidialect environment due to the formally defined
compatibility between RIF and OWL. The ontologies contain entities of the domain
and their relationships (Fig.~1, right-hand part). Conceptual specification of a~task is
defined over conceptual schema. Ontologies are imported into the RIF-documents
specifying an import profile, for instance, {OWL Direct}. Documents
\textit{import} other documents having the same semantics (the \textit{Import}
directive), \textit{link} documents defined using other dialects and having different
semantics (remote module directive \textit{Module}) or \textit{refer} to entities
contained in other documents using \textit{external terms}.
{\looseness=1

}

  Semantics of a~conceptual task definition in such setting becomes a~multidialect
one. The specification modules of a~task are treated as peers. Mediation modules are
assumed to be defined in RIF-BLD for representation of the mediator rules (to be
interpreted in SYNTHESIS) supporting schema mapping and semantic integration of
the IRs. Multidialect task is implemented by means of
transformation of conceptual specifications into modular, component-based
peer-to-peer (P2P)
program represented in the languages of the MSs and RSs
with the respective semantics. Interoperability of logic rule components of such
distributed program is carried out by means of the delegation technique [1,
section~3.3]. Production rule components are considered as external functions,
interoperability is achieved through the mechanism of external terms.

  A schema $S_R$ of a~peer~$R$ is a~set of entities (classes or relations and their
attributes) corresponding to extensional and intensional predicates of the resource
implementing the peer~$R$.

  The RS or the MS of each peer~$R$ should be
a~conformant~$D_R$ consumer where~$D_R$ is the~respective RIF dialect (Fig.~1,
left-hand part). Conformance is formally defined using formula entailment and
language mappings~[3].

  The peer $R$ is relevant to a~RIF-document~$d$ of a~conceptual specification of a~problem
  (Fig.~1, right-hand part) if ($i$)~$D_R$ is a~subdialect of the document~$d$
dialect (subdialect is a~language obtained from some dialect by removing certain
syntactic constructsand imposing respective restrictions on its semantics~[4]; each
program that conforms with the subdialect also conforms with the dialect) and
($ii$)~entities of the peer schema~$S_R$ (if they exist) are \textit{ontologically
relevant} to entities of the conceptual schema the names of which are used
in~$d$~for extensional predicates.

  The schema of a~relevant peer is mapped into the conceptual schema. The mapping
establishes the correspondence of the conceptual entities referred in the
document~$d$ to their expressions in terms of entities of the schema~$S_R$ using
rules of the $D_R$ dialect. These schema mapping rules constitute separate
  RIF-document (Fig.~1, middle part).

  Peers communicate using a~technique for distributed execution of the rule-based
programs. The basic notion of the technique is delegation-transferring facts and
rules from one peer to another. A~peer is installed on a~node of the multidialect
infrastructure. A~node is a~combination of a~wrapper, an RS or an MS, and a~peer
(for the details, refer~[1, Fig.~3]). A~wrapper
transforms programs and facts from the specific RIF dialect into the language of the
RS or MS and \textit{vice versa}. A~wrapper also implements the delegation mechanism.
Transferring facts and rules among peers is performed in the RIF dialects.

  A~special component (\textit{Supervisor}) of the architecture defined in~[1] stores
shared information of the environment, i.\,e., conceptual specifications related to the
domain and to the problem, a~list of the relevant resources, RIF-documents combining
rules for the conceptual specification and a~resource schema mapping.

  Implementation of the conceptual specification includes the following steps:
  \begin{enumerate}[(1)]
\item rewriting of the conceptual documents into the RIF-programs of the peers
performed by the \textit{Supervisor}. The rewriting includes also ($i$)~replacing the
document identifiers (used to mark predicates) by peer identifiers and ($ii$)~adding
schema mapping rules to programs (Fig.~1, middle part); %\\[-14pt]
\item a transfer of the rewritten programs to nodes containing peers relevant to the
respective conceptual documents. The transfer is performed by the \textit{Supervisor}
by calling the method \textit{loadRules} of the respective node wrappers; %\\[-14pt]
\item a transformation of the RIF-programs into the concrete RS or MS languages.
The transformation is performed by the \textit{NodeWrapper} or by the RS or MS
itself (if the RS or MS supports the respective RIF dialect); and %\\[-14pt]
\item an execution of the produced programs in P2P environment.
\end{enumerate}

  During the process of rewriting of the conceptual schema into the resource
programs, the relationships between RIF-documents of the conceptual schema defined
by remote or imported terms are replaced by relationships between peers also defined
by remote or imported terms. To implement remote and imported terms, a~\textit{rule
delegation} mechanism is used to transfer facts and rules from one peer to another.
The details of rule delegation approach including description of the related algorithms
are provided in~[1].

\vspace*{-7pt}

\section{Workflow-Oriented Extension of~the~Multidialect
Infrastructure}

\vspace*{-2pt}

  \noindent
  The aim of the infrastructure proposed is a~conceptual programming of problems in
the RIF-dialects and an implementation of conceptual specifications using rule-based
languages of the RSs and MSs. One of the objectives of this particular paper is to
introduce an extension of the existing multidialect infrastructure~[1] aiming at the
conceptual specification of rule-based workflows.

  Conceptual specification of a~problem (class of problems) is defined in the context
of a~subject domain and consists of a~set of
  RIF-documents. Besides the documents expressed in the logic dialects of RIF, the
documents expressed in the production rule dialect (RIF-PRD) also can be a~part of
conceptual specification of a~problem. In particular, these documents are aimed to
express a~process of solving the problem as the production rule-based workflow.

\vspace*{-4pt}

\subsection{Specification of~workflow orchestration}

  A workflow consists of a~set of tasks orchestrated by specific constructs
(\textit{workflow patterns}~\cite{9-kal}, for instance, \textit{sequence}, \textit{split},
\textit{join}) defining the order of tasks execution. The specification of such
orchestration is called here a~\textit{workflow skeleton}. A~skeleton is defined using
RIF-PRD production rules. Workflows and workflow patterns can be represented
using production rules in various ways, e.\,g., as in~\cite{9-kal, 17-kal}. The
approach applied in this paper to represent workflows requires the extension of
  RIF-PRD dialect by several built-in predicates (they are considered to be a~part of
\textit{wkfl} namespace referenced by
  {\sf http://www.w3.org/2014/rif-workflow-predicate\#} URI similarly to
\textit{func} and \textit{pred} namespaces defined in~\cite{21-kal} for built-in
functions and predicates of RIF):
  \begin{itemize}
\item predicate {\sf wkfl:end-of-task(?arg)} where \textit{\sf ?arg} is an identifier of a~task. The value space of
{\sf ?arg} is the XML-Schema built-in data type
{\sf xsd:Name} representing XML names. The predicate turns into true if a~task
\textit{?arg} has been completed;
\item predicate {\sf wkfl:variable-definition(?arg1\, ?arg2)} where {\sf ?arg1}
is the identifier of a~variable and {\sf ?arg2} is the identifier of a~type of the
variable.
The value space for both arguments is {\sf xsd:Name}. Turning the predicate into
true means that a~variable {\sf ?arg1} of type {\sf ?arg2} is defined in the
context of a~workflow;
\item predicate {\sf wkfl:variable-value(?arg1\,?arg2)} where {\sf ?arg1} is the
identifier of a~variable and {\sf ?arg2} is the value of the variable. The value space for
the first argument is {\sf xsd:Name}, the value space for the second argument is the
union of value spaces of all RIF built-in datatypes. Turning the predicate into true
means that a~variable {\sf ?arg1} has the value {\sf ?arg2};
\item predicate {\sf wkfl:parameter-definition(?arg1\,?arg2\,
?arg3)} where
{\sf ?arg1} is the identifier of a~workflow parameter; {\sf ?arg2} is the identifier
of a~type of the parameter; and {\sf ?arg3} is the direction of the parameter. The value
space for the first and for the second arguments is {\sf xsd:Name}. The value space
for the third argument is {\sf \{IN, OUT, IN\_OUT\}}
(\textit{input}, \textit{output}, or
\textit{input--output} parameter).
Turning the predicate into true means that a~parameter {\sf ?arg1} of
type {\sf ?arg2}, and direction {\sf ?arg3} is defined
for a~workflow; and
\item predicate {\sf wkfl:parameter-value(?arg1\,?arg2)} defines values of
workflow parameters in the same way as {\sf wkfl:variable-value} defines values
of workflow variables.
  \end{itemize}

  Predicates {\sf wkfl:variable-definition} and
  {\sf wkfl:}\linebreak {\sf variable-value} allow
to specify workflow variables and their values and thus to organize the data flow
within a~workflow. Predicates {\sf wkfl:parameter-definition} and
{\sf wkfl:parameter-value} allow to specify workflow parameters and their values
and thus to define the interface of a~workflow in terms of input and output parameters.
Using of workflow parameters and variables is illustrated in the Appendix.

  The predicate {\sf wkfl:end-of-task(?arg)} allows to orchestrate the order of
execution of workflows tasks using conditions and actions of production rules. In this
section, the template rules intended for representation of several basic workflow
patterns (Fig.~2) are provided.

\begin{center}  %fig2
\vspace*{6pt}
\mbox{%
 \epsfxsize=76.913mm
 \epsfbox{kal-2.eps}
 }
  \vspace*{2pt}

{{\figurename~2}\ \ \small{Basic workflow patterns}}
  \end{center}

\vspace*{6pt}


\addtocounter{figure}{1}

  Three well-known workflow patterns are considered below: {\sf Sequence},
{\sf AND-Split}, and {\sf AND-Join}.

  The \textit{AND-Split}\footnote{In this paper, the simplified \textit{presentation
syntax}~\cite{7-kal} is used.} workflow pattern is represented in RIF-PRD by the
following production ruletemplate using {\sf wkfl:end-of-task} predicate:
  \begin{verbatim}
If Not(External(wkfl:end-of-task(A)))
Then Do (Act(A)
 Assert(External(wkfl:end-of-task(A))))
If And(Not(External(wkfl:end-of-task(B)))
 External(wkfl:end-of-task(A)))
Then Do (Act(B)
 Assert(External(wkfl:end-of-task(B))))
If And(Not(External(wkfl:end-of-task(C)))
 External(wkfl:end-of-task(A)))
Then Do (Act(C)
 Assert(External(wkfl:end-of-task(C))))
\end{verbatim}

  The template includes three rules for tasks~$A$, $B$, and~$C$, respectively.
${\sf Act}(A)$, ${\sf Act}(B)$, and ${\sf Act}(C)$ denote \textit{actions} associated with tasks~$A$,
$B$, and $C$. Orchestration (tasks~$B$ and~$C$ are executed concurrently right after
task~$A$ is completed) is specified using {\sf wkfl:end-of-task} predicate in
conditions and {\sf Assert} actions of rules.

  Similarly, the {\sf AND-Split} pattern is represented in RIF-PRD by the
following production rule template:

\vspace*{-1.5pt}

\noindent
  \begin{verbatim}
If Not(External(wkfl:end-of-task(A)))
Then Do (Act(A)
 Assert(External(wkfl:end-of-task(A))))
If And(Not(External(wkfl:end-of-task(B)))
 External(wkfl:end-of-task(A)))
Then Do (Act(B)
 Assert(External(wkfl:end-of-task(B))))
If And(Not(External(wkfl:end-of-task(C)))
 External(wkfl:end-of-task(A)))
Then Do (Act(C)
 Assert(External(wkfl:end-of-task(C))))
\end{verbatim}

\vspace*{-1.5pt}

  The {\sf Sequence} pattern is represented in RIF-PRD by the following
production rule template:

\vspace*{-1.5pt}

\noindent
  \begin{verbatim}
If Not(External(wkfl:end-of-task(A)))
Then Do (Act(A)
 Assert(External(wkfl:end-of-task(A))))
If And(Not(External(wkfl:end-of-task(B)))
 External(wkfl:end-of-task(A)))
Then Do (Act(B)
 Assert(External(wkfl:end-of-task(B))))
\end{verbatim}

\vspace*{-1.5pt}

  More complicated patterns like OR-, XOR- splits and joins, structured loops,
subflows, and others are represented in RIF-PRD similarly.

\vspace*{-6pt}

\subsection{Workflow tasks specification}

  \noindent
  Workflow taskscan be specified as:
  \begin{itemize}
\item separate RIF-documents in various logic RIF-dialects (this is the way how
multidialect infrastructure~[1] is extended with workflow capabilities);\\[-15pt]
\item separate RIF-documents in the RIF-PRD dialect;\\[-15pt]
\item set of production rules embedded into the workflow skeleton; and\\[-15pt]
\item external functions treated as ``black boxes.''
\end{itemize}

  Semantics of tasks specified as multidialect logic programs are defined in
accordance with the RIF-FLD~[3] standard and standards for the respective
  RIF-dialects (BLD, CASPD, etc.). Semantics of tasks specified as production
rule programs are defined in accordance with the RIF-PRD standard. Semantics of
external functions ``are assumed to be specified externally in some document''~[3].

  All kinds of tasks (except those that are embedded into a~workflow skeleton) are
referenced in the workflow skeleton as \textit{external terms}~[3] like
${\sf External}\left({\sf t}\right)$
where term~{\sf t} is defined by an external resource identified by internationalized
resource identifier (IRI)~[3].

\begin{figure*} %fig3
\vspace*{1pt}
 \begin{center}
 \mbox{%
 \epsfxsize=163.675mm
 \epsfbox{kal-3.eps}
 }
 \end{center}
 \vspace*{-9pt}
\Caption{Extended multidialect infrastructure}
\end{figure*}


\vspace*{-6pt}

\subsection{Workflow implementation infrastructure}

  \noindent
  Workflows defined in the conceptual specification are implemented in the
environment shown in Fig.~3. Peer-\linebreak\vspace*{-12pt}

\pagebreak

\noindent
to-peer environment~[1] intended to implement logic
programs is extended with a~production rule-based system (PRS) (for
instance, a~production system compliant with the OMG Production Rule
Representation~\cite{16-kal}) and with external functions, implemented as
  web-services. Implementation of the conceptual specification includes the
following steps:
  \begin{enumerate}[(1)]
\item transfer of the conceptual RIF-documents constituting a~workflow skeleton to
the production rule-based system node (performed by the \textit{Supervisor} component);\\[-14pt]
\item transformation of the conceptual RIF-documents constituting a~workflow
skeleton into the language of the production rule-based system (performed by the PRS
Wrapper component);\\[-14pt]
\item transferring RIF logic programs related to tasks to the relevant nodes of the
environment and transformation of the RIF-programs into the concrete RS or MS
languages~[1]; and\\[-14pt]
\item execution of the workflow.
\end{enumerate}


  The interface of the \textit{Supervisor} includes methods for submitting and
executing a~workflow represented as a~set of RIF-documents, and for getting the
result of the workflow execution.

  To provide a~proof of the multidialect infrastructure concept, a~use case in the
financial domain has been implemented. The problem to be solved in the use case is
called the \textit{investment portfolio diversification problem}. The detailed
description of the use case is included in the Appendix.

\vspace*{-9pt}

\section{Related Work}

  \noindent
  Two types of workflow models, namely, abstract and concrete, were
identified~\cite{15-kal}. In the abstract model, a~workflow is described in an abstract
form, without re-\linebreak\vspace*{-12pt}
\columnbreak

\noindent
ferring to specific resources. In this paper, workflow
representation in abstract and platform-independent  form is suggested.

  A classification model for scientific workflow characteristics~\cite{9-kal}
contributes to better understanding of scientific workflow requirements. The list of
structural patterns discovered during this analysis (including sequential, parallel,
parallel-split, parallel-merge, and mesh) influenced the choice of the required workflow
patterns.

  The OMG standard~\cite{16-kal} reflects an attitude to production rules from the
industrial side providing an OMG MDA (model-driven architecture)
platform-independent model (PIM)  with a~high probability of
support at the PSM (platform-specific model) level from the rule engine vendors.
Similar capabilities though formally defined are used as the basis for the
RIF-PRD~\cite{7-kal}.

  Some vendors of such production rule engines have extended their languages with
the workflow specification capabilities. IBM has extended ILOG to provide the
ruleflow capability. Microsoft supports Windows Workflow Foundation as a~platform
providing the workflow and rules capabilities. The examples of specific formalisms for
PIM rule-based process specifications are also provided in~\cite{17-kal}.

  Comparing to the known variants of the PIM production rule representations,
  selection of the RIF-PRD is considered to be well grounded:
  \begin{enumerate}[(1)]
\item the RIF-PRD is formally defined;
\item RIF ensures support of interoperability of modules written in different
rule-based dialects with different semantics;
\item RIF provides foundations for PIM to PSM semantic preserving
transformation; and
\item RIF also provides ability for specification of the concepts in application
domain terms combining rule-based specifications with the OWL ontologies.
\end{enumerate}

  Importance of providing the interdialect interoperation is advocated
  in~\cite{18-kal} for combining the functionalities of production systems and logic
programs for abductive logic programming (ALP). The ALP framework gives a~model-theoretic semantics to both kinds of rules and provides them with powerful
proof procedures, combining backward and forward reasoning.

  Papers related to RIF-PRD experimentations are focused mainly on the issue of the
PRD programs transformation to an implementation system. In~\cite{19-kal}, a~case
study of bridging the ILOG Rule Language (IRL) to RIF-PRD and vice versa is
considered. In~\cite{20-kal}, implementation of RIF-PRD in three different
paradigms: Answer Set Programming, Production Rules, and Logic Programming
(XSB) is investigated.

  The contribution of this paper with regard to previous works of the authors~[1] consists in
extensions of the infrastructure and specification languages considered in~[1] to the
workflow level.

\section{Concluding Remarks}

  \noindent
  Progress in the investigation of the infrastructure~[1] for the conceptual
multidialect interoperable programming in the abstract, rule-based,
platform-independent notations is reported. An extension of the coherent
combination of the multidialect rule-based programming technique recommended by
the W3C RIF with the approach for unifying modeling of heterogeneous data bases
for their semantic mediation is presented. The extension of the infrastructure and specification
languages considered in~[1] in the direction of the workflow modeling is presented.

  Sticking to the limits of the existing set of the published RIF dialects,
   a~capability of the multidialect workflow support
   is presented with the tasks specified in
semantically different languages mostly suited to the task orientation.
Also, a~realistic problem solving use case containing the interoperating tasks
specified in
several platform-independent rule-based languages: RIF-CASPD, RIF-BLD,
  RIF-PRD, is presented. In addition, OWL~2 is used for the conceptual schema definition,
  RIF-PRD is applied for the workflow orchestration. The platforms selected for
implementation of the tasks include: DLV, SYNTHESIS, IBM ILOG. Such approach
retains well-defined semantics of the platform-independent rule-based languages with
a possibility to check preservation of their semantics by various languages of the
implementing systems. The principle of independence of tasks from the specific IRs is
carried out by the heterogeneous database mediation facilitates contributing to the
  reuse of tasks and workflows. Alongside with the further extension of the
approach, in the future work, the authors plan to apply the conceptual multidialect
programming philosophy for support of the experiments in data intensive sciences. In
particular, they plan to investigate modeling hypotheses in astronomy representing them
as a~set of rules applying the multiplicity of the dialects required.


%\setcounter{equation}{0}

\vspace*{12pt}

{{\hfill \textbf{APPENDIX A}}}

\vspace*{-12pt}

\subsection*{MULTIDIALECT WORKFLOW USE CASE}


{\small


 %\section*{\raggedleft Appendix~A.\\ Multidialect Workflow Use-Case}


%\renewcommand{\thesection}{A\arabic{equation}}

\subsection*{A.1\ Investment portfolio diversification\\
\hspace{20pt}problem extended}

  \noindent
  Motivation of the use case that illustrates the proposed approach comes from the
finance area. The use case extends the \textit{investment portfolio diversification
problem} defined in~[1, Appendix] by adding workflow orchestration applying the
RIF-PRD. The idea of the portfolio diversification problem is
as follows. The portfolio is a~collection of securities of companies, and its size is the
number of securities in the portfolio. The problem is to build a~diversified portfolio of
maximum size. Diversification means that the prices of the securities in portfolio
should be almost independent of each other. If the price of one security falls, it will
not significantly affect the prices of others. Thus, the risk of a~portfolio sharp decrease
is reduced.

  The input data for the problem is a~set of securities and respective time series of
indicators of the security price for each security. Time series for each security is a~set
of pairs $(d, v)$ where $d$ is a~date and~$v$ is an indicator of the security price (for
instance, closing price). The financial services \textit{Google Finance} ({\sf
https://www.google.com/finance}) and \textit{Yahoo! Finance} ({\sf
http://finance.yahoo.com/}) are considered. They include various indicators of the
security price for all trading days of the last decades. For the diversified portfolio, the
securities having noncorrelated time series should be used. Noncorrelation of the
time series means that their correlation is less than some predetermined price
correlation value. The output data for the problem is a~set of subsets of securities of
the maximum size, for which the pair wise correlation will be less than the
predetermined one.

  The maximum satisfying subset of securities is calculated in the following way.
Let~$G$ be a~graph where the vertices are the securities. An edge between two
securities exists if absolute value of their correlation is less than a~specified number.
So, any two securities connected by an edge are considered as noncorrelated. In such
case, the problem of finding the portfolio of the maximum size is exactly the problem
of finding a~maximum clique in an undirected graph. A~maximal clique is a~maximal
portfolio. Note that several different maximal portfolios can be found.

  The conceptual specification of the use case~[1] used two RIF-dialects: RIF-BLD
and RIF-CASPD. The use case was implemented in the environment containing a~mediation system used as a~platform for RIF-BLD~[4] and ASP-based DLV
system~\cite{6-kal}~--- a~platform for RIF-CASPD. The RIF-BLD was used to
specify the problem of data integration, and RIF-CASPD~--- the problem of finding
a~maximum clique in an undirected graph.

In this work, the portfolio use case is extended in the following way. The goal is
not only to build a~set of diversified portfolios, but
also to choose the ``best'' of them
according to some criteria. There are several approaches to choose the most
appropriate portfolio.

  The most recognized one is based on the Markovitz portfolio theory~\cite{10-kal}.
The idea is to choose the portfolio, which has the maximum risk/return ratio. The
most well-known metric to operate with risk/return is Sharpe-ratio~\cite{11-kal}:
  $(r_p -r_f)/\sigma^2$. Here, $r_p$ denotes the expected return of the portfolio;
$r_f$ denotes the~risk free rate; and $\sigma^2$ denotes the~portfolio standard deviation (risk).
The more the Sharpe-ratio is, the better the investment is.

  Another approach is based on an idea that with the advent of social networks, it
became possible to monitor ideas, sentiments, actions of people and lots of available
information has to do with the markets and investments. In~\cite{12-kal}, Bollen
\textit{et al.}\ draw the connection between the mood of investor tweets and the move
of Dow Jones Index, stating that correlation between them is more than 80\%. The
idea of using tweets to assess market movements has been implemented in several
hedge funds.

  Combining these two strategies could provide benefits of both of them, which
leads to the following problem statement: having S\&P500 (a stock market index
maintained by the Standard\,\&\,Poor's, comprising 500~large-cap American
companies) list of companies, compute the diversified portfolio of maximum size
with the best risk/return and sentiment ratios.

%\vspace*{-6pt}

\subsection*{A.2\	Conceptual specification\\
\hspace*{20pt}of~the~application domain\\
\hspace*{20pt}and~the~problem}

%\vspace*{-2pt}

  \noindent
  Conceptual schema (ontology) of the application domain of historical prices of
securities is written in the simplified OWL functional syntax~\cite{8-kal}
({\sf Declaration} keyword is omitted; {\sf property}, {\sf domain}, and {\sf range}
declarations are combined).
  \begin{verbatim}
Ontology(<http://synthesis.ipi.ac.ru/portfolio/
    ontology>
 Class(Portfolio)
  ObjectProperty(securities domain(Portfolio)
   range(Portfolio))
  DataProperty(expected_return domain(Portfolio)
   range(xsd:double))
  DataExactCardinality(1 expected_return
   Portfolio)
  DataProperty(std_dev domain(Portfolio)
   range(xsd:double))
  DataExactCardinality(1 std_dev Portfolio)
  DataProperty(sharpe_ratio domain(Portfolio)
   range(xsd:double))
  DataExactCardinality(1 sharpe_ratio Portfolio)
  DataProperty(twitter_positive_ratio
   domain(Portfolio) range(xsd:double))
  DataExactCardinality(1 twitter_positive_ratio
   Portfolio)
  DataProperty(risk_free_rate domain(Portfolio)
   range(xsd:double))
  DataExactCardinality(1 risk_free_rate
   Portfolio)
  DataProperty(recommended domain(Portfolio)
   range(xsd:boolean))
  DataExactCardinality(1 recommended Portfolio)

 Class(Security)
  DataProperty(ticker  domain(Security)
   range(xsd:string))
  DataExactCardinality(1 ticker Security)
  DataProperty(rates  domain(Security)
   range(StockRate))
  DataProperty(positive_tweets domain(Security)
   range(xsd:double))
  DataExactCardinality(1 positive_tweets
   Security)
  DataProperty(sec_expected_return
   domain(Security) range(xsd:double))
  DataExactCardinality(1 sec_expected_return
   Security)
  DataProperty(sec_std_dev domain(Security)
   range(xsd:double))
  DataExactCardinality(1 sec_std_dev Security)

 Class(StockRate)
  DataProperty(date domain(StockRate)
   range(xsd:date))
  DataExactCardinality(1 date StockRate)
  DataProperty(price domain(StockRate)
   range(xsd:double))
  DataExactCardinality(1 price StockRate)
)
  \end{verbatim}

  \vspace*{-6pt}

  A~portfolio (the {\sf Portfolio} class) is characterized by a~set of securities
({\sf securities} attribute) contained in the portfolio, by several metrics: expected
return ({\sf expected\_return} attribute), standard deviation ({\sf std\_dev}
attribute), Sharpe ratio ({\sf sharpe\_ratio attribute}), risk free rate
({\sf risk\_free\_rate} attribute), and
ratio of positive tweets mentioning securities of
the portfolio ({\sf twitter\_positive\_ratio} attribute).

  A security (the {\sf Security} class) is characterized by identifier ({\sf ticker}
attribute), time series of historical prices (attribute {\sf
rates}), ratio of positive
tweets mentioning the security ({\sf positive\_tweets} attribute), expected return
({\sf sec\_expected\_return} attribute), and standard deviation ({\sf sec\_std\_dev}
attribute).

\begin{figure*} %fig4
\vspace*{1pt}
 \begin{center}
 \mbox{%
 \epsfxsize=126.24mm
 \epsfbox{kal-4.eps}
 }
 \end{center}
 \vspace*{-11pt}
\Caption{Portfolio workflow}
\vspace*{-6pt}
  \end{figure*}

The workflow of the extended portfolio problem is demonstrated in Fig. 4. The workflow
contains six tasks\footnote{To save space, specifications are provided only for
{\sf getPortfolios}, {\sf getPositiveTweetRatio}, and
{\sf computePortfolioTwitterMetrics} tasks.}:
\begin{enumerate}[(1)]
\item {\sf getPortfolios}. A~set of diversified portfolio candidates is computed. The
multidialect task specification consists of two RIF-documents in BLD and CASPD
dialects~[1, Appendix]. Portfolios received as a~result contain only security tickers,
they have to be augmented by financial and sentiments ratios;
\item {\sf getPositiveTweetRatio}. This task is responsible for computing a~sentiment ratio of tweets for every security. Every tweet is assessed
to be positive,
negative, or neutral. The task is specified as a~call of external function;
\item {\sf computePortfolioTwitterMetrics}. The portfolio sentiment ratio is
computed as the average of its securities sentiment ratio. The task is specified using
RIF-PRD;
{\looseness=1

}
\item {\sf getSecurityFinancialMetrics}. For every security in a~portfolio the
financial rates (the {\sf expected return} and the {\sf standard deviation}) are
calculated on the basis of historical rates of securities specified as an OWL~2 class of
the ontology of the application domain. The task is specified using RIF-BLD dialect;
\item {\sf computePortfolioFinancialMetrics}. The computation of the portfolio
expected return, risk, and Sharpe-ratio is done within this task. The task is specified
using RIF-PRD dialect; and
\item {\sf choosePortfolio}. The best portfolio is chosen according to maximizing
the (\textit{Sharpe ratio * sentiment ratio}) coefficient. The task is specified using
RIF-PRD dialect.
  \end{enumerate}

  Workflow skeleton is specified as a~RIF-PRD document importing the ontology of
the application domain:
  \begin{verbatim}
Document( Dialect(RIF-PRD)
 Base(<http://synthesis.ipi.ac.ru/portfolio/
  workflow#>)
 Import(<http://synthesis.ipi.ac.ru/portfolio/
  ontology#>
 <http://www.w3.org/ns/entailment/OWL-Direct>)
Prefix(ont<http://synthesis.ipi.ac.ru/portfolio/
 ontology#>)
Prefix(ofws<http://synthesis.ipi.ac.ru/
 synthesis/projects/RuleInt/OpinionFinderWS#>)
Prefix(mws<http://synthesis.ipi.ac.ru/
 synthesis/projects/RuleInt/MediatorWS#>)

Group 2 (
 Do(
  Assert(External(wkfl:parameter-definition(
   startDatexsd:string IN)))
  Assert(External(wkfl:parameter-definition(
   endDatexsd:string IN)))
  Assert(External(wkfl:parameter-definition(
   bestPortfolioont:Portfolio OUT)))
  Assert(External(wkfl:variable-definition(
   ps  List<ont:Portfolio> IN)))
  Assert(External(wkfl:
   variable-value(ps List())))
 )
)
\end{verbatim}

\noindent
\begin{verbatim}

Group 1 (
 Forall ?sd ?ed such that (
  External(wkfl:parameter-value(startDate ?sd))
  External(wkfl:parameter-value(endDate ?ed))  )
( If Not(External(wkfl:
   end-of-task(getPortfolios)))
  Then
   Do( Modify(External(wkfl:variable-value(ps
    External(mws:getPortfolios(?sd ?ed) )))
   Assert(External(wkfl:
    end-of-task(getPortfolios))) )
 )

 Forall ?ps ?p ?scs ?s ?t such that (
  External(wkfl:variable-value(ps ?ps))
  ?p#?ps  ?p[securities->?scs]
   ?s#?scs ?s[ticker->?t] )
( If And( Not(External(wkfl:
     end-of-task(getTweets)))
   External(wkfl:end-of-task(getPortfolios)))
  Then
  Do( Modify(?s[positive_tweets->
   External(ofws:computeSecPosTweets(?t))] )
   Assert(External(wkfl:
    end-of-task(getTweets))) )
)

Forall ?ps ?p such that (
 External(wkfl:variable-value(ps  ?ps))
  ?p#?ps)
( If And(Not(External(wkfl:
   end-of-task(countTwitterMetrics)))
   External(wkfl:end-of-task(getTweets)) )
  Then Do(
   Modify(?p[twitter_positive_ratio->
    External(func:numeric-divide(
    Sum{?pt | Exists
     ?scs ?s(?p[securities->?scs]
     ?s#?scs  ?s[positive_tweets->?pt])}
    External(func:count(?ps))))])
   Assert(External(wkfl:
    end-of-task(countTwitterMetrics)))
)	)) )
\end{verbatim}

\begin{figure*}[b] %fig5
\vspace*{-4pt}
 \begin{center}
 \mbox{%
 \epsfxsize=162.319mm
 \epsfbox{kal-5.eps}
 }
 \end{center}
 \vspace*{-9pt}
\Caption{Portfolio problem implementation infrastructure}
  \end{figure*}

  Production rules of the document are divided into two groups. The first group with
priority~2 contains rules defining workflow parameters and variable. Input parameters
are \textit{start date} and \textit{end date} of historical rates used for calculation of
\textit{portfolio metrics}. Workflow variable {\sf ps} denotes a~set containing
\textit{portfolio candidates}.

  The second group with priority~1 contains the orchestration rules~--- workflow
skeleton. The only orchestration rule provided in the example above corresponds to
the task {\sf getPortfolios}. The external function {\sf getPortfolios}
encapsulates a~multidialect logic program calculating portfolio candidates~[1,
Appendix]. A~{\sf Modify} action is used to call the function and to put the
returned result into the {\sf ps} variable.

\vspace*{-6pt}

\subsection*{A.3\	Revised portfolio problem infrastructure}

  \noindent
  The implementation structure of the use case is shown in Fig.~5.




  The RIF-PRD workflow skeleton was transformed into a~program (rule set) in the
ILOG~\cite{13-kal} language combining production rules and workflow facilities
(like {\sf fork} and {\sf sequence}). The ILOG program was executed in the
{IBM Operational Decision Manager} tool~\cite{24-kal}. In order to execute
ILOG programs, the underlying execution model (XOM)~\cite{25-kal}
was defined as a~set of
Java classes: {\sf Portfolio}, {\sf Security}, and {\sf StockRate}. The
{\sf Portfolio class} corresponds to a~financial portfolio and contains as attributes a~set of
securities in it, its expected return, standard deviation, Sharpe ratio, and twitter
positive ratio. Code of this class is provided below:
  \begin{verbatim}
public class Portfolio {
 private Collection<Security> securities;
 private double expected_return;
 private double std_dev;
 private double sharpe_ratio;
 private double twitter_positive_ratio;
 // as of 05.04.14 US 5-year treasuries
 private static double risk_free_rate = 0.0169;
 private boolean recommended;
}
\end{verbatim}

  Class {\sf Security} corresponds to real world financial securities. The class
contains as attributes a~ticker, ratio of positive tweet number to the sum of positive
and negative tweets, a~set of stock rates, security's standard deviation, and expected
return. These attributes are set as responses to corresponding web services queries:

\vspace*{-2pt}

\noindent
  \begin{verbatim}
public class Security {
 public String ticker;
 public double positive_tweets;
 public Collection<StockRate> rates;
 public double std_dev;
 public double expected_return;
 public static int number_of_periods = 5;
}	
\end{verbatim}

\vspace*{-2pt}

{\sf StockRate} is a~simple class and contains just two attributes~--- price and date:

\vspace*{-2pt}

\noindent
  \begin{verbatim}
public class StockRate {
 public float price;
 public String date;
}
\end{verbatim}

\vspace*{-2pt}

  It is easy to see that the one-to-one mapping exists between conceptual schema
entities and execution model entities.

  Parameters of RIF-PRD workflow skeleton ({\sf startDate}, {\sf endDate}, and
{\sf bestPortfolio}) are mapped into the respective parameters of ILOG rule set
(Fig.~6).

\begin{figure*} %fig6
\vspace*{1pt}
 \begin{center}
 \mbox{%
 \epsfxsize=115mm
 \epsfbox{kal-6.eps}
 }
 \end{center}
 \vspace*{-9pt}
\Caption{Rule set parameters}
  \end{figure*}

  The variable of RIF-PRD workflow skeleton ({\sf ps}) is mapped into a~local variable
of the rule set. Specification of the variable looks as follows:

\vspace*{-2pt}

\noindent
  \begin{verbatim}
<?xml version="1.0" encoding="UTF-8"?>
<ilog.rules.studio.model.base:VariableSetxmi:
  version="2.0"
xmlns:xmi="http://www.omg.org/XMI"
  xmlns:ilog.rules.studio.model.base =
"http://ilog.rules.studio/model/base.ecore">
 <name>local_vars</name>
 <variables name="ps" type="java.util.ArrayList"
   initialValue=""verbalization="ps"/>
</ilog.rules.studio.model.base:VariableSet>
\end{verbatim}

  Rules of the RIF-PRD workflow skeleton are mapped into ILOG
\textit{ruleflow}~\cite{25-kal}:

\vspace*{-6pt}

\noindent
  \begin{verbatim}
flowtask portfolio$_$flow {
 property mainflowtask = true;
 property ilog.rules.business_name =
  "portfolio_flow";
 body {
  portfolio$_$flow#getPortfolios;
  fork {
   portfolio$_$flow#getRates;
   portfolio$_$flow
   #computePortfolioFinancialMetrics;} &&
  { portfolio$_$flow#getTweets;
   portfolio$_$flow#
    computePortfolioTwitterMetrics;}
  portfolio$_$flow#choosePortfolio;
 }
};

ruletask portfolio$_$flow#getPortfolios {
 property ilog.rules.business_name =
   "portfolio_flow>getPortfolios";
 body { getPortfolios.*}
};

ruletask portfolio$_$flow#
   computePortfolioTwitterMetrics {
 propertyilog.rules.business_name =
  "portfolio_flow>
   computePortfolioTwitterMetrics";
 body { computePortfolioTwitterMetrics.* }
};

ruletask portfolio$_$flow#getTweets {
 property ilog.rules.business_name =
  "portfolio_flow>getTweets";
 property ilog.rules.package_name = "";
 body {getTweets.*}
};
\end{verbatim}

  The {\sf computePortfolioTwitterMetrics},
{\sf computePortfolioFinancialMetrics}, and {\sf choosePortfolio} tasks are
implemented as production rules in ILOG:

\vspace*{-6pt}

\noindent
  \begin{verbatim}
package computePortfolioTwitterMetrics {
 use ps;
 import portfolio.*;

 rule computePortfolioTwitterMetrics {
  property status = "new";
  when {	IlrContext() from ?context;	}
  then {
   foreach (Portfolio p in ps) {
    double ?twitter_metrics = 0;
    int ?length = 0;
     foreach (Security security
       in p.securities) {
      ?twitter_metrics= ?twitter_metrics +
       security.positive_tweets;
      ?length = ?length + 1; }
     p.twitter_positive_ratio=
      ?twitter_metrics / ?length;
}}}}
\end{verbatim}

  The {\sf getPortfolios} and {\sf computeSecurityFinancialMetrics} tasks are
implemented by the following production rules in ILOG:


\noindent
  \begin{verbatim}
package getPortfolios {
 use ps;
 import portfolio.*;

 rule getPortfolios {
  when { IlrContext() from ?context; }
  then {
   ps = Supervisor.getPortfolios(startDate,
    endDate);
} } }
\end{verbatim}

\begin{table*}\small
\begin{center}
\Caption{Metrics for the securities}
  \vspace*{2ex}

  \begin{tabular}{cccc}
  \hline
Security identifier&Expected return&Standard deviation&Positive tweet ratio\\
\hline
COG&0.163&0.201&0.507\\
DO&0.015&0.019&0.651\\
EQR&0.150&0.022&0.846\\
FOSL&0.513&0.030&0.579\\
SCG&0.050&0.010&0.622\\
\hline
\end{tabular}
\end{center}
%\end{table*}
%\begin{table*}\small
\begin{center}
\Caption{Metrics for the portfolio candidates}
  \vspace*{2ex}

  \begin{tabular}{lcccccc}
  \hline
\multicolumn{1}{c}{\tabcolsep=0pt\begin{tabular}{c}Portfolio\\ identifier\end{tabular}}&
\tabcolsep=0pt\begin{tabular}{c}Expected\\ return\end{tabular}&
\tabcolsep=0pt\begin{tabular}{c}Standard\\ deviation\end{tabular}&
\tabcolsep=0pt\begin{tabular}{c}Risk free\\ rate\end{tabular}&
\tabcolsep=0pt\begin{tabular}{c}Sharpe\\ ratio\end{tabular}&
\tabcolsep=0pt\begin{tabular}{c}Positive\\ tweet ratio\end{tabular}&
\tabcolsep=0pt\begin{tabular}{c}Sharpe ratio\\$\times$\;Positive tweet ratio\end{tabular}\\
\hline
1&0.111&0.008&0.0169&11.755&0.660&7.758\\
2&2.400&0.507&0.0169&\hphantom{9}4.701&0.508&2.388\\
3&2.381&0.508&0.0169&\hphantom{9}4.662&0.557&2.597\\
4&2.347&0.505&0.0169&\hphantom{9}4.606&0.708&3.261\\
5 (best)&0.178&0.011&0.0169&14.227&0.641&9.120\\
6&0.147&0.008&0.0169&15.577&0.521&8.166\\
\hline
\end{tabular}
\end{center}
\vspace*{-3pt}
\end{table*}



\noindent
  Here, the {\sf Supervisor} is the~Java class wrapping execution of logic programs
in multidialect infrastructure including two nodes~[1]. The nodes correspond to the
mediation system (which integrates \textit{Google Finance} and the \textit{Yahoo!
Finance} services) and to a~rule-based programming system DLV.

  The {\sf getSecurityFinancialMetrics} task uses the same instance of the
mediation system as the {\sf getPortfolios} task. The reason is that financial
metrics are calculated using the historical rates of the securities. This is exactly the
information that is extracted by the mediation system from {Google Finance}
and {Yahoo! Finance}. The difference between two tasks is that the
{\sf getPortfolios} is implemented as a~submission of a~query to the DLV node, but
the {\sf getSecurityFinancialMetrics} is implemented as a~submission of a~different
query to the Mediation Node.

  The {\sf getPositiveTweetRatio} task is implemented by the following
production rule in ILOG:

%\pagebreak

\noindent
  \begin{verbatim}
package getTweets {
 use ps;
 import portfolio.*;

 rule getTweets {
  when { IlrContext() from ?context; }
  then {
   foreach (Portfolio p in ps) {
    foreach (Security s in p.securities) {
     s.positive_tweets =
      WebServices.computeSecPosTweets(s.ticker);
} } } } }
\end{verbatim}




\noindent
Here, {\sf WebServices} is the~Java-class wrapping invocation of a~web-service.
The WSDL specification of the web-service can be found at {\sf
http://synthesis.ipi.ac.ru/synthesis/ projects/RuleInt/OpinionFinderWS}. The
  web-service, in its turn, encapsulates a~Java-program. The program first collects
tweets using the {\sf Twitter Streaming API}. After that, a~sentiment analysis is
done by the {\sf Polarity Classifier} of the {\sf OpinionFinder}
  tool~\cite{14-kal} which assesses if tweet is positive, negative, or neutral. Finally,
the sentiment ratio for every security in a~portfolio is calculated and returned as the
result.

\vspace*{-6pt}

\subsection*{A.4\	Result of~the~use case workflow execution}

  \noindent
  The results obtained by one of the use case runs are as follows. The task
{\sf getPortfolios} computes portfolio candidates on the basis of historical rates of
daily closing prices of securities from S\&P500 list for the 2011--2013. Six portfolios
of size~5 were calculated. Each portfolio is a~set of identifiers (tickers) of
companies:
  \begin{verbatim}
Candidate 1: { ALXN, BF.B, EW, POM, VNO }
Candidate 2: { BMC, JBL, LUK, MNST, POM }
Candidate 3: { AVP, BMC, JPL, MNST, POM }
Candidate 4: { ALTR, BF.B, BMC, DGX, PEG }
Candidate 5: { COG, DO, EQR, FOSL, SCG }
Candidate 6: { ADSK, GILD, INTC, POM, TJX }
\end{verbatim}

  The task {\sf getSecurityFinancialMetrics} computes the expected return and
the standard deviation for every security mentioned in portfolio candidates.
 The task
{\sf getPositiveTweetRatio} computes positive sentiment ratios for every security
mentioned in portfolio candidates (500~latest tweets for every security were used for
the computation). Financial and twitter metrics for several securities are provided in
Table~1.



  The task {\sf computePortfolioFinancialMetrics} computes financial metrics for
every portfolio candidate on the basis of respective metrics for
securities in a~portfolio. The task {\sf computePortfolioTwitterMetrics} computes sentiment
metrics for every portfolio candidate on the basis of sentiment metrics for securities in
a~portfolio. Financial and twitter metrics for portfolio candidates are provided in
Table~2. The task {\sf choosePortfolio} identifies the best portfolio by maximum
value of the products of Sharpe ratio and positive tweet ratio obtained for every
portfolio (see Table~2).


}

\vspace*{-9pt}

\Ack
\noindent
This research has been done under the support of the \mbox{RFBR} (projects13-07-00579,
14-07-00548) and the Program for Basic Research of the Presidium of RAS.

\renewcommand{\bibname}{\protect\rmfamily References}

\vspace*{-9pt}

{\small\frenchspacing
{%\baselineskip=10.8pt
\begin{thebibliography}{99}

\bibitem{1-kal}
\Aue{Kalinichenko, L.\,A., S.\,A.~Stupnikov, A.\,E.~Vovchenko, and D.\,Y.~Kovalev}.
2013. Conceptual declarative problem specification and solving in data intensive domains.
\textit{Informatics and Applications}~--- \textit{Inform \mbox{Appl.}} 7(4):112--139.
Available at: {\sf http://synthesis.ipi.ac.\linebreak ru/synthesis/publications/13ia-multidialect}
 (accessed December~9, 2014).
\bibitem{2-kal}
\Aue{Kalinichenko, L.\,A., S.\,A.~Stupnikov, and D.\,O.~Martynov}. 2007.
\textit{SYNTHESIS: A~language for canonical information modeling and mediator
definition for problem solving in heterogeneous information resource environments}.
Moscow: IPIRAN. 171~p.
\bibitem{3-kal}
Boley, H., and M.~Kifer, eds. 2013. {RIF framework for logic dialects. W3C
recommendation}. 2nd ed. Available at:
{\sf http://www.w3.org/TR/2013/REC-rif-fld-20130205/}
(accessed December~9, 2014).

\bibitem{4-kal}
Boley, H., and M.~Kifer, eds. 2013. {RIF basic logic dialect. W3C Recommendation}.
2nd ed. Available at:
{\sf http://www.w3.org/TR/2013/REC-rif-bld-20130205/}
(accessed December~9, 2014).


\bibitem{5-kal}
Heymans, S., and M.~Kifer, eds. 2009. {RIF core answer set programming dialect}.
Available at: {\sf http:// ruleml.org/rif/RIF-CASPD.html} (accessed November~5, 2014).
\bibitem{6-kal}
\Aue{Leone, N., G.~Pfeifer, W.~Faber, T.~Eiter,  G.~Gottlob, S.~Perri, and F.~Scarcello}.
2006. The DLV system for knowledge representation and reasoning. \textit{ACM Trans.
Comput. Logic} 7(3):499--562.
\bibitem{7-kal}
DeSante, M.\,C., G.~Hallmark, and A.~Paschke, eds. 2013. {RIF production rule
dialect. W3C Recommendation}. 2nd ed.
Available at: {\sf http://www.w3.org/TR/2013/REC-rif-prd-20130205/}
(accessed December~9, 2014).

\bibitem{8-kal}
Motik, B., P.\,F.~Patel-Schneider, and B.~Parsia, eds.
2012. {OWL~2 Web Ontology Language structural
specification and functional-style syntax. W3C Recommendation}. 2nd ed.
Available at:
{\sf http://www.w3.org/TR/owl2-syntax/} (accessed November~5, 2014).
\bibitem{22-kal} %9
\Aue{Calvanese, D., G.~De Giacomo, D.~Lembo, M.~Lenzerini, A.~Poggi, and
R.~Rosati}.
 2007. Ontology-based database access. \textit{15th
Italian Symposium on Advanced Database Systems Proceedings}. 324--331.

\bibitem{9-kal} %10
\Aue{Ramakrishnan, L., and B.~Plale}. 2010. A~multi-dimensional classification model for
scientific workflow\linebreak characteristics. \textit{1st Workshop (International) on Workflow
Approaches to New Data-Centric Science Proceedings}. New York: ACM.
Article No.\,4. 12~p.
Available at: {\sf http://dl.acm.org/citation.cfm?id=1833402}\linebreak
(accessed December~9, 2014).

\bibitem{17-kal} %11
\Aue{Boukhebouze, M, Y.~Amghar, A.-N.~Benharkat,  and Z.~Maamar}. 2011.
A~rule-based approach to model and verify flexible business processes. \textit{Int.
J.~Business Process Integration Management} 5(4):287--307.

\bibitem{21-kal} %12
Polleres, A., H.~Boley, and M.~Kifer, eds. 2013. {RIF datatypes and Built-Ins~1.0
W3C Recommendation.} 2nd ed.
Available at: {\sf http://www.w3.org/TR/2013/REC-rif-dtb-20130205/}
(accessed December~9, 2014).




\bibitem{16-kal} %13
Production Rule Representation (PRR), Version 1.0. OMG Document Number:
formal/2009-12-01. Available at: {\sf http://www.omg.org/spec/PRR/1.0} (accessed
November~5, 2014).

\bibitem{15-kal} %14
\Aue{Yu,~J., and R.~Buyya}. 2005. A~taxonomy of scientific workflow systems for grid
computing. \textit{ACM SIGMOD Records} 34(3):44--49.

\bibitem{18-kal} %15
\Aue{Kowalski, R., and F.~Sadri}. 2009. Integrating logic programming and production
systems in abductive logic programming agents.
\textit{Web reasoning and rule systems}. Eds. A.~Polleres and T.~Swift.
Lecture notes in computer science ser. Springer.
5837:1--23.

\bibitem{19-kal} %16
\Aue{Cosentino, V., M.\,D.~Del Fabro, and A.~El Ghali}. 2012. A~model driven approach
for bridging ILOG rule language and RIF. \textit{6th Symposium (International) on Rules
RuleML Proceedings}. CEUR-WS.org. 874:96--102.
\bibitem{20-kal} %17
\Aue{Veiga, F.\,D.\,J.} 2011. Implementation of the RIF-PRD.  Universidade
Nova de Lisboa. Master Thesis. Available at: {\sf
http://run.unl.pt/bitstream/10362/6310/1/Veiga\_\linebreak 2011.pdf} (accessed November~5, 2014).

\bibitem{10-kal} %18
\Aue{Markowitz, H.\,M.} 1991. \textit{Portfolio selection: Efficient diversification of
investments}. Wiley. 402~p.
\bibitem{11-kal} %19
\Aue{Sharpe, W.\,F.} 1966. Mutual fund performance. \textit{J.~Business}
39(S1):119--138.
\bibitem{12-kal} %20
\Aue{Bollen,~J., H.~Maoa, and X.~Zeng}. 2011. Twitter mood predicts the stockmarket.
\textit{J.~Comput. Sci.} 2(1):1--8.
\bibitem{13-kal} %21
IBM WebSphere ILOG JRules Version~7.0. Online documentation. Available at: {\sf
http://pic.dhe.ibm.com/\linebreak infocenter/brjrules/v7r0/index.jsp} (accessed November~5, 2014).

\bibitem{24-kal} %22
IBM Operational Decision Manager. Available at:
{\sf http:// www-03.ibm.com/software/products/en/odm} (accessed November~5, 2014).
\bibitem{25-kal} %23
IBM Operational Decision Manager Version~8.5 Information Center. Available at: {\sf
http://pic.dhe.ibm.com/\linebreak infocenter/dmanager/v8r5/index.jsp} (accessed November~5, 2014).


\bibitem{14-kal} %24
\Aue{Wilson, T., J.~Wiebe, and P.~Hoffmann}. 2005.
Recognizing contextual polarity in phrase-level sentiment Analysis. \textit{Conference on
Human Language Technology and Empirical Methods in Natural Language Processing
Proceedings}. Stroudsburg: Association for Computational Linguistics. 347--354.


%\bibitem{23-kal}
%Bock, C., \textit{et. al.}, eds. 2012. \textit{OWL~2 Web Ontology Language Structural
%Specification and Functional-Style Syntax. W3C Recommendation}. 2nd ed.


\end{thebibliography} } }

\end{multicols}

\vspace*{-9pt}

\hfill{\small\textit{Received November 3, 2014}}

\vspace*{-18pt}

\Contr

\noindent
\textbf{Kalinichenko Leonid A.} (b.\ 1937)~---
 Doctor of Science in physics and mathematics, professor;
 Head of Laboratory, Institute of Informatics Problems, 44-2 Vavilov Str.,
 Moscow 119333, Russian Federation; professor,
 Faculty of Computational Mathematics and Cybernetics, M.\,V.~Lomonosov Moscow
 State University, 1-52 Leninskiye Gory, GSP-1, Moscow 119991,
 Russian Federation; leonidandk@gmail.com

 \vspace*{3pt}

 \noindent
 \textbf{Stupnikov Sergey A.} (b.\ 1978)~---
 Candidate of Science (PhD) in technology, senior scientist,
 Institute of Informatics Problems, Russian Academy of Sciences,
 44-2 Vavilov Str.,
 Moscow 119333, Russian Federation; ssa@ipi.ac.ru

 \vspace*{3pt}

 \noindent
 \textbf{Vovchenko Alexey E.} (b.\ 1984)~---
 Candidate of Science (PhD) in technology, senior scientist,
 Institute of Informatics Problems, Russian Academy of Sciences,
 44-2 Vavilov Str.,
 Moscow 119333, Russian Federation; itsnein@gmail.com

 \vspace*{3pt}

 \noindent
 \textbf{Kovalev Dmitry Yu.} (b.\ 1988)~---
 junior scientist, Institute of Informatics Problems, Russian Academy of Sciences,
 44-2 Vavilov Str.,
 Moscow 119333, Russian Federation; dm.kovalev@gmail.com


%\vspace*{24pt}

%\hrule

%\vspace*{2pt}

%\hrule

%\vspace*{-6pt}

\newpage


\def\tit{КОНЦЕПТУАЛЬНОЕ МОДЕЛИРОВАНИЕ МУЛЬТИДИАЛЕКТНЫХ ПОТОКОВ РАБОТ$^*$}

\def\aut{Л.\,А.~Калиниченко$^{1,2}$, С.~Ступников$^1$, А.~Вовченко$^1$, Д.~Ковалев$^1$}


\def\titkol{Концептуальное моделирование мультидиалектных потоков работ}

\def\autkol{Л.\,А.~Калиниченко, С. Ступников, А. Вовченко, Д. Ковалев}

{\renewcommand{\thefootnote}{\fnsymbol{footnote}}
\footnotetext[1]{Работа выполнена при поддержке РФФИ (проекты
13-07-00579, 14-07-00548) и~Программы фундаментальных исследований Президиума РАН.}}


\titel{\tit}{\aut}{\autkol}{\titkol}

\vspace*{-12pt}

\noindent
$^1$Институт проблем информатики Российской академии наук

\noindent
$^2$Московский государственный университет им.\ М.\,В.~Ломоносова, факультет вычислительной
матема-\linebreak
$\hphantom{^1}$тики и~кибернетики

\vspace*{6pt}

\def\leftfootline{\small{\textbf{\thepage}
\hfill ИНФОРМАТИКА И ЕЁ ПРИМЕНЕНИЯ\ \ \ том\ 8\ \ \ выпуск\ 4\ \ \ 2014}
}%
 \def\rightfootline{\small{ИНФОРМАТИКА И ЕЁ ПРИМЕНЕНИЯ\ \ \ том\ 8\ \ \ выпуск\ 4\ \ \ 2014
\hfill \textbf{\thepage}}}


\Abst{Рассматриваются методы концептуального представления
алгоритмов анализа данных, средств интеграции данных, а~также процессов,
направленных на спецификацию семантики данных и~поведения в~единой парадигме.
Расширяется новый подход к~применению комбинации семантически различных
плат\-фор\-мо\-не\-за\-ви\-си\-мых языков на правилах (диалектов) для создания
интероперабельных концептуальных спецификаций над различными системами на правилах.
Подход опирается на методику трансформации программ на правилах, рекомендованную
стандартом W3C Rule Interchange Format (RIF). Подход, предлагаемый в~стандарте RIF,
сочетается со технологией семантической интеграции неоднородных баз данных
в~предметных посредниках. Статья расширяет предыдущие исследования авторов
в~направлении моделирования потоков работ для определения композиций
алгоритмических модулей в~процессной структуре. Рассмотрены возможности
спецификации задач в~мультидиалектных потоках работ с~применением семантически
различных языков, наиболее подходящих для конкретных задач. Приведен практический
пример потока работ, задачи которого специфицированы с~использованием нескольких
 языков на правилах (RIF-CASPD, RIF-BLD, RIF-PRD). Для определения концептуальной
 схемы использован язык OWL~2, для оркестровки потока работ использован язык
 RIF-PRD. Инфраструктура реализации примера включает систему на продукционных
 правилах (IBM ILOG), систему на логических правилах (DLV) и~предметный посредник.}

\KW{концептуальные спецификации; потоки работ; RIF; языки продукционных правил;
интеграция баз данных; посредники; PRD; мультидиалектная инфраструктура}

\DOI{10.14357/19922264140413}

\vspace*{6pt}


 \begin{multicols}{2}

\renewcommand{\bibname}{\protect\rmfamily Литература}
%\renewcommand{\bibname}{\large\protect\rm References}

{\small\frenchspacing
{%\baselineskip=10.8pt
\begin{thebibliography}{99}

\bibitem{1-kal-1}
\Au{Kalinichenko L.\,A., Stupnikov S.\,A.. Vovchenko~A.\,E.,
Kovalev~D.\,Y.}
Conceptual declarative problem specification and solving in data intensive domains~//
Информатика и~её применения, 2013. Т.~7. Вып.~4. С.~112--139.
{\sf http://synthesis.ipi.ac.ru/synthesis/publications/13ia-multidialect}.
\bibitem{2-kal-1}
\Au{Kalinichenko L.\,A., Stupnikov~S.\,A., Martynov~D.\,O.}
 SYNTHESIS: A~language for canonical information modeling and mediator definition
 for problem solving in heterogeneous information resource environments.~---
 Moscow: IPI RAN, 2007. 171~p.
\bibitem{3-kal-1}
RIF framework for logic dialects. W3C Recommendation~/
Eds. H.~Boley, M.~Kifer. 2nd ed.
{\sf http:// www.w3.org/TR/2013/REC-rif-fld-20130205/}.
\bibitem{4-kal-1}
RIF basic logic dialect. W3C Recommendation~/
Eds. H.~Boley, M.~Kifer. 2nd ed.
{\sf http://www.w3.org/ TR/2013/REC-rif-bld-20130205/}.
\bibitem{5-kal-1}
RIF core answer set programming dialect~/
Eds.\ S.~Heymans, M.~Kifer, 2009. {\sf  http://ruleml.org/rif/RIF-CASPD.html}.
\bibitem{6-kal-1}
\Au{Leone N., Pfeifer G., Faber~W., Eiter~T., Gottlob~G., Perri~S., Scarcello~F.}
The DLV system for knowledge representation and reasoning~//
 ACM Trans. Comput. Logic, 2006. Vol.~7. No.\,3. P.~499--562.
\bibitem{7-kal-1}
RIF production rule dialect. W3C Recommendation~/
Eds.\ De Sante Marie~C., Hallmark~G., A.~Paschke.~ 2nd ed.
{\sf http://www.w3.org/TR/2013/REC-rif-prd-20130205/}.
\bibitem{8-kal-1}
OWL~2 Web Ontology Language Structural Specification and Functional-Style Syntax.
W3C Recommendation~/ Eds. B.~Motik, P.\,F.~Patel-Schneider, B.~Parsia. 2nd ed.
{\sf http://www.w3.org/TR/owl2-syntax/}.

\bibitem{22-kal-1} %9
\Au{Calvanese, D., De Giacomo~G., Lembo~D., Lenzerini~M., Poggi~A.,
Rosati~R.}
Ontology-based database access~// 15th Italian Symposium on Advanced
Database Systems Proceedings, 2007. P.~324--331.

\bibitem{9-kal-1} %10
\Au{Ramakrishnan L., Plale~B.}
A~multi-dimensional classification model for scientific workflow characteristics~//
1st  Workshop (International) on Workflow Approaches to New Data-Centric Science
Proceedings.  New York: ACM, 2010. Aricle No.\,4. 12~p.
{\sf http://dl. acm.org/citation.cfm?id=1833402}.
\pagebreak

\bibitem{17-kal-1} %11
\Au{Boukhebouze M., Amghar~Y., Benharkat~A.-N., Maamar~Z.}
A~rule-based approach to model and verify flexible business processes~//
Int. J.~Business Process Integration Management, 2011. Vol.~5. No.\,4. P.~287--307.

\bibitem{21-kal-1} %12
RIF Datatypes and Built-Ins 1.0. W3C Recommendation~/
Eds.\ A.~Polleres, H.~Boley, M.~Kifer. 2nd ed.
{\sf http://www.w3.org/TR/2013/REC-rif-dtb-20130205/}.



%\pagebreak

\bibitem{16-kal-1} %13
Production Rule Representation (PRR), Version~1.0.
OMG Document Number: formal/2009-12-01. {\sf http:// www.omg.org/spec/PRR/1.0}.

\bibitem{15-kal-1} %14
\Au{Yu J., Buyya~R.}
 A~taxonomy of scientific workflow systems for grid computing~//
 ACM SIGMOD Records, 2005. Vol.~34. No.\,3. P.~44--49.

 \bibitem{18-kal-1} %15
\Au{Kowalski R., Sadri~F.}
Integrating logic programming and production systems in abductive logic programming
agents~//
Web reasoning and rule systems~/ Eds. A.~Polleres, T.~Swift.
Lecture notes in computer science ser.~--- Springer, 2009. Vol.~5837. P.~1--23.
\bibitem{19-kal-1} %16
\Au{Cosentino V., Del Fabro~M.\,D., El Ghali~A.}
 A~model driven approach for bridging ILOG rule language and RIF~//
 6th  Symposium (International) on Rules, RuleML 2012 Proceedings.  2012.
 CEUR-WS.org. Vol.~874. P.~96--102.
\bibitem{20-kal-1} %17
\Au{Veiga F.\,D.\,J.}
Implementation of the RIF-PRD. Universidade Nova de Lisboa, 2011. Master Thesis.


\bibitem{10-kal-1} %18
\Au{Markowitz H.\,M.}
Portfolio selection: Efficient diversification of investments. Wiley, 1991.
402~p.
\bibitem{11-kal-1} %19
\Au{Sharpe, W.\,F.}
Mutual fund performance~// J.~Business, 1966. Vol.~39(S1). P.~119--138.

\bibitem{12-kal-1} %20
\Au{Bollen J., Maoa H., Zeng~X.}
Twitter mood predicts the stock market~// J.~Comput. Sci., 2011. Vol.~2. No.\,1.
P.~1--8.

\bibitem{13-kal-1} %21
IBM WebSphere ILOG JRules Version 7.0. Online documentation.
{\sf http://pic.dhe.ibm.com/infocenter/\linebreak brjrules/v7r0/index.jsp}.

 \bibitem{24-kal-1} %22
 IBM Operational Decision Manager.
 {\sf http://www-03.\linebreak ibm.com/software/products/en/odm}.
\bibitem{25-kal-1} %23
IBM Operational Decision Manager Version~8.5 Information Center.
{\sf http://pic.dhe.ibm.com/infocenter/\linebreak dmanager/v8r5/index.jsp}.

\bibitem{14-kal-1} %24
\Au{Wilson T., Wiebe~J., Hoffmann~P.}
Recognizing contextual polarity in phrase-level sentiment analysis.
\textit{Conference on Human Language Technology and Empirical Methods in Natural
Language Processing Proceedinhgs}.
Stroudsburg: Association for Computational Linguistics, 2005. P.~347--354.

\end{thebibliography}
} }

\end{multicols}

 \label{end\stat}

 \vspace*{-3pt}

\hfill{\small\textit{Поступила в редакцию 03.11.2014}}
%\renewcommand{\bibname}{\protect\rm Литература}
\renewcommand{\figurename}{\protect\bf Рис.}