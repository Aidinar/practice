\def\stat{bronshtein}



\def\tit{ОБ ОПТИМАЛЬНОЙ ДОСТАВКЕ ГРУЗОВ ТРАНСПОРТНЫМ
СРЕДСТВОМ С~УЧЕТОМ ЗАВИСИМОСТИ СТОИМОСТИ
ПЕРЕВОЗОК ОТ~ЗАГРУЗКИ ТРАНСПОРТНЫХ СРЕДСТВ
ПО~НЕСКОЛЬКИМ ЦИКЛИЧЕСКИМ МАРШРУТАМ$^*$}

\def\titkol{Об оптимальной доставке грузов транспортным
средством с~учетом зависимости стоимости
перевозок от~загрузки} % транспортных средств по~нескольким циклическим маршрутам}

\def\aut{Е.\,М. Бронштейн$^1$, П.\,А.~Зелёв$^2$}

\def\autkol{Е.\,М. Бронштейн, П.\,А.~Зелёв}

\titel{\tit}{\aut}{\autkol}{\titkol}

{\renewcommand{\thefootnote}{\fnsymbol{footnote}} \footnotetext[1]
{Работа выполнена при поддержке РФФИ (проект 13-01-00005).}}


\renewcommand{\thefootnote}{\arabic{footnote}}
\footnotetext[1]{Уфимский государственный авиационный технический университет, bro-efim@yandex.ru}
\footnotetext[2]{Уфимский государственный авиационный технический университет, pz1988@yandex.ru}

  \Abst{Рассматривается задача построения маршрута доставки грузов
потребителям от одного производителя (базы, склада) транспортным средством
(ТС) с~минимальными затратами на перевозки. При этом учитывается
зависимость стоимости транспортировки от загрузки ТС и~качества дороги.
Предполагается, что ТС может возвращаться на базу для дозагрузки. Построена
соответствующая математическая модель. Для случая линейной зависимости
стоимости проезда от загрузки получена линейная целочисленная модель. Для
решения поставленной задачи наряду с~точным алгоритмом предложена
модификация известного эвристического алгоритма Клар\-ка--Рай\-та.
Проведен вычислительный эксперимент.}

  \KW{эвристический алгоритм; построение маршрута; транспортировка;
задача маршрутизации}

\DOI{10.14357/19922264140407}


\vskip 12pt plus 9pt minus 6pt

\thispagestyle{headings}

\begin{multicols}{2}

\label{st\stat}

\section{Введение}

  Систематическое изучение оптимизационных задач транспортной логистики
(Vehicle Routing Problem, VRP) началось с~работы~[1]. За последние полвека
поставлено множество задач этого типа, развиты как точные, так
и~эвристические методы решения. Частным случаем подобных задач является
известная задача коммивояжера.
  Классификация оптимизационных задач транспортной логистики приведена,
например, в~[2].

  Задача, рассматриваемая в~данной работе, перекликается с~CVRP
(Capacitated VRP), в~этих задачах учитываются ограничения на вместимость
ТС (см., например,~[3]). В~большинстве случаев при постановке задач
маршрутизации стоимость транспортировки является функцией, зависящей
только от расстояния между городами.

  В работе~[4] предлагается при постановке задач маршрутизации учитывать
ряд дополнительных факторов, часть из которых рассматривается в~настоящей
работе. В~частности, предполагается, что стоимость транспортировки зависит
от загрузки ТС и~состояния дороги.

  Рассматриваемая задача является NP-труд\-ной, как обобщение задачи
коммивояжера, в~связи с~чем точные методы применимы лишь при малых
размерностях задач (менее 10), а~значит, актуальной задачей является
разработка эвристических методов ее решения.

\section{Постановка задачи}

  Груз следует доставить потребителям из пункта производства
ТС грузоподъемностью~$Q$. Известны значения $q_i$~--- потребности
в~грузе $i$-го пункта потребления ($i\hm=1, \ldots, n$), $n$~--- число пунктов
потребления. Для единообразия будем считать пункт производства (базу, склад)
нулевым пунктом. Предполагается, что ТС, доставив груз в~некоторые пункты,
возвращается на базу и~загружается для доставки другим потребителям.

  Дороги между пунктами характеризуются двумя показателями:
расстоянием~$l_{ij}$ ($i, j\hm=0, \ldots , n$) и~коэффициен\-том слож\-ности
дороги~$k_{ij}$ между пунктами~$i$ и~$j$. Расстояние~$l_{ij}$ может зависеть от
направления движения (например, есть дороги с~односторонним движением),
коэффициент слож\-ности дороги влияет на расход топлива, его величина также
может зависеть от направления маршрута (пример: подъем в~гору или спуск
с~горы)~[5].

  Каждый пункт посещается в~точности один раз (не допускается split
delivery~[6]). Очевидно, что задача имеет решение тогда и~только тогда, когда
вместимость ТС не меньше потребности в~грузе в~каждом из пунктов
потребления. Число циклов, за которые происходит доставка, не
превосходит~$n$. Будем считать, что оно равно~$n$, но среди них могут быть
пустые.

  Предполагается известной зависимость $f(q)$ расхода топлива на единицу
пути дороги стандартного качества.

  Функция расхода топлива в~зависимости от веса перевозимого груза
является возрастающей и~(как правило) вогнутой на всем интервале области
определения функции  $[0;\,Q]$.

  Введем булевы переменные~$X_{ij}^t$ ($i,j,\,t\hm=0, \ldots , n$), равные~1
тогда и~только тогда, когда следующим после $i$-го пункта на пути
следования ТС в~$t$-м цикле является $j$-й пункт. Должны выполняться
следующие условия:
  \begin{align}
  \sum\limits_{i=0}^n X^t_{i0} &= \sum\limits_{j=0}^n X^t_{0j} =1\,,\enskip
t=1,\ldots ,n\,;\label{e1-br}\\
  \sum\limits_{t=1}^n \sum\limits_{i=0}^n X_{ij}^t &=1\,,\enskip j=1,\ldots,
n\,;\label{e2-br}\\
  \sum\limits_{t=1}^n \sum\limits_{i=0}^n X_{ij}^t &=1\,,\enskip i=1,\ldots, n\,.
  \label{e3-br}
  \end{align}

  Ограничение~(1) означает наличие в~каждом цик\-ле одного пункта
потребления (или базы для нулевого маршрута), из которого ТС попадает
непосредственно на базу, и~ровно одного пункта потребления (или базы для
нулевого маршрута), в~который ТС выезжает с~базы; ограничения~(2) и~(3)
описывают условие однократного посещения каж\-до\-го пункта потребления.

  Введем целочисленные переменные  $v_i^t$ ($i,t\hm=1,\ldots ,n)$), имеющие
смысл номеров потребителей в~порядке прохождения в~$t$-м цикле,
содержащем все пункты, за исключением начального:
  \begin{gather}
  1\leq v_i^t \leq \sum\limits_{i=0}^n \sum\limits_{j=0}^n X_{ij}^t\,, \enskip
i,t=1,\ldots, n\,;\label{e4-br}\\
  \left( v_i^t -v_j^t\right) +nX^t_{ij} \leq n-1\,,\enskip i,j,t=1,\ldots, n\,.
  \label{e5-br}
  \end{gather}

  Из условий~(4) и~(5) следует отсутствие под\-цик\-лов.

  Введем, наконец, булевы переменные $Z^t_{is}$, равные~1 тогда и~только
тогда, когда $v_s^t\hm>v_i^t$ ($i,s,t\hm=1,\ldots ,n$). Ограничения на введенные
переменные имеют следующий вид:
  \begin{equation}
  nZ_{is}^t \geq v_s^t -v_i^t\,,\enskip i,s,t=1,\ldots, n\,.
  \label{e6-br}
  \end{equation}


  Это условие обеспечивает выполнение равенства $Z_{is}^t\hm= 1$ при
$v_s^t\hm>v_i^t$.

  Для обеспечения равенства $Z_{is}^t\hm=0$ при $v_s^t\hm\leq v_i^t$
заметим, что число величин~$v_i^t$, меньших~$v_s^t$, равно $v_s^t\hm-1$.
Таким образом, выполнение нужного свойства обеспечивается равенствами:
  \begin{equation}
  \sum\limits_{i=1}^n Z_{is}^t =v_s^t -1\,, s\,,t=1,\ldots, n\,.
  \label{e7-br}
  \end{equation}

     Ограничение на вместимость:
     \begin{equation}
     \sum\limits_{i=0}^n \sum\limits_{j=0}^n q_i X_{ij}^t \leq Q\,,\enskip
t=1,\ldots, n\,.
     \label{e8-br}
     \end{equation}

  Целью является минимизация расходов на транспортировку:
  \begin{multline}
  R(X) = {}\\
  {}=\sum\limits_{t=1}^n \sum\limits_{j=0}^n \sum\limits_{i=0}^n X_{ij}^t f
\left ( \sum\limits_{s=0}^n Z^t_{is} q_i\right) l_{ij} k_{ij} \to \min\,.
  \label{e9-br}
  \end{multline}

\section{Линейная целочисленная модель}

  Предположим, что стоимость транспортировки является линейной функцией
от массы перевозимого груза:
$$
f(q)=vq+w\,.
$$
%
 Тогда при сохранении
условий~(1)--(8) целевая функция~(\ref{e9-br}) примет вид:
  \begin{multline}
  R(X) =v \sum\limits_{t=1}^n \sum\limits_{s=0}^n \sum\limits_{j=0}^n
\sum\limits_{i=0}^n X_{ij}^t Z_{is}^t q_s l_{ij} k_{ij} +{}\\
 {}+w \sum\limits_{t=1}^n
\sum\limits_{j=0}^n \sum\limits_{i=0}^n X_{ij}^t l_{ij} k_{ij} \to \min\,.
  \label{e10-br}
  \end{multline}

  Для приведения целевой функции к линейной форме введем булевы
переменные $P^t_{ijs} \hm= X_{ij}^t Z_{is}^t$ ($i,j,s,t\hm= 0,\ldots ,n$). Это
условие равносильно выполнению следующей системы неравенств:
  \begin{alignat}{2}
  P^t_{ijs} &\geq X_{ij}^t +Z_{is}^t -1\,, &\enskip i,j,s,t&=0,\ldots, n\,;\label{e11-br}\\
  P_{ijs}^t &\leq X_{ij}^t\,, &\enskip i,j,s,t&=0,\ldots ,n\,;\label{e12-br}\\
  P_{ijs}^t&\leq Z_{is}^t\,, &\enskip i,j,s,t&=0,\ldots, n\,.\label{e13-br}
  \end{alignat}
%
  Тогда целевая функция~(10) примет вид:
  \begin{multline}
  R(X) =v\sum\limits_{t=1}^n \sum\limits_{s=0}^n \sum\limits_{j=0}^n
\sum\limits_{i=0}^n P^t_{ijs} q_s l_{ij} k_{ij} +{}\\
{}+w \sum\limits_{t=1}^n
\sum\limits_{j=0}^n \sum\limits_{i=0}^n X_{ij}^t l_{ij} k_{ij}\to \min\,.
  \label{e14-br}
  \end{multline}

  Как число переменных задачи~(1)--(8), (11)--(14), так и~число ограничений
имеют порядок $O(n^4)$.

\section{Модифицированный алгоритм Кларка--Райта}

  Для решения задачи был модифицирован эвристический алгоритм
  Клар\-ка--Рай\-та~[7, 8]. Достоинствами метода являются его простота,
надежность и~гибкость.

  Опишем основные идеи алгоритма.

  Метод Кларка--Райта является итерационным, причем на каждой итерации
осуществляется попытка сращивания двух циклических маршрутов по
определенным правилам. Модификация метода Клар\-ка--Рай\-та заключается
в~возможности варьирования глубины сращивания маршрутов~$p$, а~также
в~учете зависимости стоимости перевозки от направления прохождения цикла.
В~классическом алгоритме Клар\-ка--Рай\-та глубина $p\hm=0$.

  Изначально генерируются~$n$~цик\-лов вида 0--$i$--0
($i\hm=\overline{1,n}$). На каждом из последующих шагов рассматриваются
всевозможные допустимые пары циклов (т.\,е.\ такие, для которых суммарная
загрузка не превосходит~$Q$), для каждой из которых строится не более
8~цик\-лов при $p\hm =0$ и~не более $16p$~цик\-лов при $p\hm\geq 1$
следующим образом: каждый из новых циклов состоит из начального или
конечного отрезка (длиной не более~$p$) одного из циклов, затем проходится
второй цикл (без нулевой вершины), затем продолжается движение по первому
циклу. При этом учитывается возможность изменения направления движения
по каждому из циклов на противоположное, поскольку задача не
предполагается симметричной и~загрузка ТС зависит от порядка прохождения
пунктов.

  Например, сращивание циклов $a_1^1,a_2^1,\ldots, a^1_{k_1}$ и~$a_1^2,
a_2^2,\ldots, a^2_{k_2}$ (с исключенной нулевой вершиной) при единичных
начальном и~конечном отрезках дает следующие циклы:
  \begin{gather*}
  0, a_1^1, a_1^2, a_2^2,\ldots, a^2_{k_2}, a_2^1,\ldots, a^1_{k_1}, 0;\\
   0, a_1^1, a^2_{k_2}, a^2_{k_2-1},\ldots , a_1^2, a_2^1,\ldots, a^1_{k_1},0;\\
  0,a_{k_1}^1, a^1_{k_1-1}, \ldots, a_2^1, a_1^2, a_2^2,\ldots, a^2_{k_2},
a_1^1,0;\\
  0,a^1_{k_1}, a^1_{k_1-1},\ldots , a_2^1, a^2_{k_2}, a^2_{k_2-1},\ldots, a_1^2,
a_1^1,0\,.
  \end{gather*}

  Еще четыре цикла возникнут при добавлении части второго цикла между
пунктами $a^1_{k_1}$ и~$a^1_{k_1-1}$. Затем первый и~второй циклы можно
поменять мес\-та\-ми, тогда получим еще 8~циклов.

  В каждом случае вычисляется разность между суммой расходов на доставку
грузов по отдельным циклам и~расходов на доставку грузов по их
объединению и~выбирается объединенный цикл, для которого эта разность
максимальная. При этом в~модификации алгоритма Клар\-ка--Рай\-та для
каждой пары циклов перебираются все возможные варианты их сращивания
при всех значениях глубины сращивания от~0 до заданного пользователем
значения~$p$. Процесс прекращается при невозможности дальнейшего
сращивания циклов или отсутствии пары циклов, сращивание которых
выгодно~[7,~8].

\vspace*{-9pt}

\section{Вычислительный эксперимент}

\vspace*{-2pt}

  Задача (1)--(8), (11)--(14) решалась двумя способами: точным методом
и~модифицированным алгоритмом Клар\-ка--Рай\-та, которые были
реализованы в~среде Scilab-4.1.2. Поскольку на задачах с~размерностями
$n\hm\geq 8$ время решения линейной целочисленной задачи оказалось весьма
продолжительным (в~некоторых случаях более 45~ч), сравнение
алгоритмов производилось для случайно сгенерированных задач при
размерностях $n\hm=\overline{4,8}$. Для каж\-дой размерности было решено по
20~задач каж\-дым из алгоритмов.

  Генерация тестовых примеров осуществлялась для значений
грузоподъемности ТС $Q\hm\in [1{,}5,\,5]$ (т); элементы матрицы~$L$
протяженности дорог между пунктами генерировались из диапазона
$L_{ij}\hm\in [1,\,100]$, $i,j\hm=\overline{1,n}$ (км); элементы матрицы
коэффициентов сложности дорог между пунктами генерировались целыми из
диапазона $[1,\,5]$; вектор потребностей пунктов в~грузе генерировался из
диапазона $q_i\hm\in [300,\,1500]$ (кг). Функция расхода топлива для
используемого типа ТС задана полиномом первой степени от величины
загрузки ТС в~виде $f(q)\hm=15\hm+3q$.

  Все сгенерированные задачи были решены при помощи точного алгоритма,
а~также с~помощью модифицированного алгоритма Клар\-ка--Рай\-та при
значении параметра $p\hm=3$. Для каждого значения $n\hm= \overline{4,8}$
в~таблице приведены результаты численного эксперимента~--- средние
значения отклонения\linebreak от оптимального значения расхода топлива, среднее время
работы точного алгоритма и~модифицированного алгоритма Клар\-ка--Рай\-та.
Также приведены сведения о доле задач, решение которых алгорит\-мом
  Клар\-ка--Рай\-та совпало с~оптимальным.

\begin{table*}\small
\begin{center}
\begin{tabular}{|c|c|c|c|c|}
\multicolumn{5}{c}{Результаты численного эксперимента}\\[6pt]
\hline
&Точный алгоритм&\multicolumn{3}{c|}{Модифицированный алгоритм
Кларка--Райта}\\
\cline{2-5}
\tabcolsep=0pt\begin{tabular}{c}$n$\\ \ \end{tabular}&
\tabcolsep=0pt\begin{tabular}{c}Среднее время\\ работы алгоритма, с\end{tabular}&
\tabcolsep=0pt\begin{tabular}{c}Среднее время\\ работы алгоритма, с\end{tabular}&
\tabcolsep=0pt\begin{tabular}{c}Среднее отклонение\\ расхода топлива, \%\end{tabular}
&\tabcolsep=0pt\begin{tabular}{c}Совпадения результатов\\ с~оптимумом, \%\end{tabular}\\
\hline
4&2,70&17,57&3,08&85\\
5&13,74&20,82&4,47&75\\
6&213,85&24,36&3,04&65\\
7&6050,91&46,40&9,15&50\\
8&100\,084&54,60&14,26\hphantom{9}&25\\
\hline
\end{tabular}
\end{center}
\end{table*}



  Как видно из таблицы, модифицированный алгоритм Клар\-ка--Рай\-та на
небольших размерностях показывает хорошие результаты. Вычислительное
время, затрачиваемое точным алгоритмом, оказывается неприемлемо велико
(более 27~ч уже при $n\hm=8$), в~то время как вычислительное время,
затрачиваемое модифицированным алгоритмом Клар\-ка--Рай\-та, остается
приемлемым при существенном увеличении размерности

  Рисунок иллюстрирует зависимость времени работы эвристического
алгоритма от размерности за-\linebreak\vspace*{-12pt}
 \begin{center}
 \mbox{%
 \epsfxsize=77.428mm
 \epsfbox{bro-1.eps}
 }
 \end{center}
% \vspace*{-9pt}

\noindent
{\small Зависимость вычислительного времени работы модификации алгоритма
Клар\-ка--Рай\-та от размерности задач}

\vspace*{9pt}


\noindent
дачи. По горизонтали указаны значения~$n$~---
размерности задачи, по вертикали~--- среднее время вычисления (в секундах).
Как видно, при увеличении размерности время работы модифицированного
алгоритма Клар\-ка--Рай\-та возрастает, однако точный алгоритм на
идентичных размерностях работает несравнимо дольше.



\section{Заключение}

  Рассматриваемая проблема является актуальной прикладной задачей.
Полученные результаты позволяют сделать вывод о~целесообразности
использования модификации алгоритма Клар\-ка--Рай\-та для решения задачи
транспортировки грузов с~учетом загрузки ТС для
снижения отклонения получаемых результатов от точно найденного оптимума.

{\small\frenchspacing
 {%\baselineskip=10.8pt
 \addcontentsline{toc}{section}{References}
 \begin{thebibliography}{9}
\bibitem{1-br}
\Au{Dantzig G.\,B., Ramser J.\,H.} The truck dispatching problem~// Management
Sci., 1959. No.\,1. P.~80--91.
\bibitem{2-br}
\Au{Бронштейн Е.\,М., Заико Т.\,А}. Детерминированные оптимизационные
задачи транспортной логистики~// Автоматика и~телемеханика, 2010. №\,10.
С.~133--147.
\bibitem{3-br}
\Au{Ralphs T.\,K., Kopman L., Pulleyblank~W.\ R., Trotter~L.\,E., Jr.} On the
capacitated vehicle routing problem~// Math. Program. Ser. B, 2003. Vol.~94.
P.~343--359.
\bibitem{4-br}
\Au{Kara I., Kara B.\,Y., Kadri Yetis~M.} Energy minimizing vehicle routing
problem~// Combinatorial optimization and applications~/
Eds. Dress~A.\,W.\,M., Xu~Y., Zhu~B.
Lecture notes in computer science ser.~--- Springer, 2007. Vol.~4616.
P.~62--71.
\bibitem{5-br}
\Au{Зелёв П.\,А., Бронштейн Е.\,М.} Задача транспортной логистики с~учетом
зависимости расходов на транспортировку от загрузки транспортного
средства~// Логистика и~управление цепями поставок, 2010. №\,4. С.~39--45.
\bibitem{6-br}
\Au{Dror M., Laporte G., Trudeau~P.} Vehicle routing with split deliveries~//
Discrete Appl. Math., 1994. Vol.~50. P.~239--254.

\bibitem{8-br}
\Au{Clarke G., Right J.\,W.} Scheduling of vehicles from a central depot to a~number
of delivery points~// Oper. Res., 1963. No.\,11. P.~568--581.

\bibitem{7-br} %8
\Au{Бронштейн Е.\,М., Зелёв П.\,А.} Задача маршрутизации транспортного
средства с~учетом зависимости стоимости перевозок от загрузки~//
Информационные технологии, 2014. №\,4. С.~33--37.

 \end{thebibliography}

 }
 }

\end{multicols}

\vspace*{-3pt}

\hfill{\small\textit{Поступила в редакцию 06.02.14}}

\newpage

%\vspace*{12pt}

%\hrule

%\vspace*{2pt}

%\hrule

%\vspace*{12pt}

\def\tit{ABOUT OPTIMUM DELIVERY OF FREIGHTS BY~THE~VEHICLE TAKING INTO ACCOUNT DEPENDENCE
OF~COST OF~TRANSPORTATIONS ON~LOADING OF~VEHICLES ON~SEVERAL CYCLIC ROUTES}

\def\titkol{About optimum delivery of freights by~the vehicle
taking into account dependence of~cost
of~transportations on~loading of~vehicles} % on several cyclic routes}

\def\aut{E.\,M.~Bronshtein and P.\,A.~Zelyov}

\def\autkol{E.\,M.~Bronshtein and P.\,A.~Zelyov}

\titel{\tit}{\aut}{\autkol}{\titkol}

\vspace*{-9pt}

\noindent
Ufa State Aviation Technical University,
12 K.~Marx Str., Ufa 450000, Russian Federation


\def\leftfootline{\small{\textbf{\thepage}
\hfill INFORMATIKA I EE PRIMENENIYA~--- INFORMATICS AND
APPLICATIONS\ \ \ 2014\ \ \ volume~8\ \ \ issue\ 4}
}%
 \def\rightfootline{\small{INFORMATIKA I EE PRIMENENIYA~---
INFORMATICS AND APPLICATIONS\ \ \ 2014\ \ \ volume~8\ \ \ issue\ 4
\hfill \textbf{\thepage}}}

\vspace*{3pt}




  \Abste{The problem of creation of a route of freights delivery from
  one producer (base, a warehouse) to consumers by the vehicle with
  the minimum costs of transportations is considered. Dependence of cost
  of transportation on loading of the vehicle and quality of the road is
  thus considered. It is supposed that the vehicle can come back to the base
  for additional charge. The corresponding mathematical model is constructed;
  for a case of linear dependence of fare from loading, the linear integer model
  is received. For the solution of an objective along with the exact algorithm,
  modification
  of the known heuristic algorithm of Clark and Right is suggested. Computing experiment
  has been made.}

  \KWE{heuristic algorithm; creation of a route; transportation; problem of routing}


  \DOI{10.14357/19922264140407}


\Ack
\noindent
The work was financially supported by the Russian  Foundation for
Basic Research (project 13-01-00005).



%\vspace*{3pt}

  \begin{multicols}{2}

\renewcommand{\bibname}{\protect\rmfamily References}
%\renewcommand{\bibname}{\large\protect\rm References}



{\small\frenchspacing
 {%\baselineskip=10.8pt
 \addcontentsline{toc}{section}{References}
 \begin{thebibliography}{9}
\bibitem{1-br-1}
\Aue{Dantzig, G.\,B., and J.\,H.~Ramser}. 1959. The truck dispatching problem.
\textit{Management Sci.} 1:80--91.
\bibitem{2-br-1}
\Aue{Bronshtein, E.\,M., and T.\,A.~Zaiko}. 2010. Deterministic optimizational
problems of transportation logistics. \textit{Automation Remote Control}
10:2132--2144.
\bibitem{3-br-1}
\Aue{Ralphs, T.\,K., L. Kopman, W.\,R.~Pulleyblank, and L.\,E.~Trotter, Jr.} 2003.
On the capacitated vehicle routing problem. \textit{Math. Program. Ser. B}.
94:343--359.
\bibitem{4-br-1}
\Au{Kara, I., B.\,Y. Kara, and M.~Kadri Yetis}. 2007. Energy minimizing vehicle
routing problem. \textit{Combinatorial optimization and applications}.
Eds.\ A.\,W.\,M.~Dress, Y.~Xu, and B.~Zhu.
Lecture notes in computer science ser. 4616:62--71.
\bibitem{5-br-1}
\Aue{Zelev, P.\,A., and E.\,M. Bronshteyn}. 2010. Zadacha transportnoy logistiki s~uchetom zavisimosti raskhodov na transportirovku ot zagruzki transportnogo sredstva
[The problem of transportation logistics, taking into account transportation costs,
depending on the load of the vehicle]. \textit{Logistika i~Upravlenie Tsepyami
Postavok} [Logistics and Supply Chain Management] 4:39--45.
\bibitem{6-br-1}
\Aue{Dror, M., G. Laporte, and P.~Trudeau}. 1994. Vehicle routing with split
deliveries. \textit{Discrete Appl. Math.} 50:239--254.
\bibitem{7-br-1} %8
\Aue{Bronshteyn, E.\,M., and P.\,A.~Zelev}. 2014. Zadacha marsh\-ru\-ti\-za\-tsii
transportnogo sredstva s~uchetom zavisimosti stoimosti perevozok ot zagruzki
[Vehicle routing problem, taking into account the cost of transportаtion, depending
on the load]. \textit{Informatsionnye Tekhnologii} [Information Technologies]
4:33--37.
\bibitem{8-br-1} %7
\Aue{Clarke, G., and J.\,W.~Right}. 1963. Scheduling of vehicles from a~central
depot to a~number of delivery points. \textit{Oper. Res.} 11:568--581.
\end{thebibliography}

 }
 }

\end{multicols}

\vspace*{-6pt}

\hfill{\small\textit{Received February 6, 2014}}

\vspace*{-18pt}

\Contr

\noindent
\textbf{Bronshtein Efim M.}\ (b.\ 1946)~--- Doctor of Science in physics and mathematics; professor,
Ufa State Aviation Technical University, 12 K.~Marx Street, Ufa 450000, Russian Federation; bro-efim@yandex.ru

\vspace*{3pt}

\noindent
\textbf{Zelyov Pavel A.}\ (b.\ 1988)~---
PhD student,  Ufa State Aviation Technical University, 12 K.~Marx Street, Ufa 450000, Russian Federation;
pz1988@yandex.ru
\label{end\stat}

\renewcommand{\bibname}{\protect\rm Литература}
		
