\def\stat{gorshenin}



\def\tit{ВИЗУАЛИЗАЦИЯ РЕЗУЛЬТАТОВ ДЛЯ МЕТОДА СКОЛЬЗЯЩЕГО РАЗДЕЛЕНИЯ СМЕСЕЙ$^*$}



\def\titkol{Визуализация результатов для метода скользящего разделения смесей}

\def\aut{А.\,К.~Горшенин$^1$}

\def\autkol{А.\,К.~Горшенин}

\titel{\tit}{\aut}{\autkol}{\titkol}

{\renewcommand{\thefootnote}{\fnsymbol{footnote}} \footnotetext[1]
{Работа
выполнена при частичной финансовой поддержке гранта Президента
Российской Федерации МК-4103.2014.9.}}


\renewcommand{\thefootnote}{\arabic{footnote}}
\footnotetext[1]{Институт проблем
информатики Российской академии наук; Московский государственный
технический университет радиотехники, электроники и~автоматики;
agorshenin@ipiran.ru}


\Abst{Метод скользящего разделения смесей (СРС-метод) представляет собой
мощный инструмент
анализа стохастических процессов различной природы. Именно на основании
экспертной оценки
результатов, полученных в~ходе работы итерационных процедур СРС-ме\-то\-да,
был получен ряд важных
результатов в~физике турбулентной плазмы, произведено уточнение математических
моделей
функционирования финансовых рынков. Зачастую каждая группа исследователей готовит
результаты в~удобном для себя формате, что затрудняет экспертам сравнение
и~интерпретацию результатов, особенно
если речь идет о~тестировании одной модели на принципиально разнородных выборках
из отличных между
собой предметных областей. В~на\-сто\-ящей работе представлено удобное
для ис\-сле\-до\-ва\-те\-ля-экс\-пер\-та
средство визуального отображения оценок параметров моделей, не зависящее от
используемых для расчетов
методов.}

\KW{метод скользящего разделение смесей;
пользовательский интерфейс; смеси нормальных распределений; вероятностные модели; интеллектуальный анализ данных}

\DOI{10.14357/19922264140410}


\vskip 10pt plus 9pt minus 6pt

\thispagestyle{headings}

\begin{multicols}{2}

\label{st\stat}

\section{Введение}

Для обнаружения и~отслеживания изменений во времени в~структуре формирующих
стохастических процессов успешно используется так называемый 
СРС-ме\-тод~\cite{Korolev2011}.
Применение \mbox{данного} мето\-да для проведения исследований в~области физики
турбулентной плазмы (см., например, работу~\cite{Skvortsova2006}), данных
с~финансовых рынков (см.,
например, \mbox{статью}~\cite{Gorshenin2008}) позволило получить ряд \mbox{важных}, принципиально
новых результатов
(в~част\-ности, впервые удалось определить число процессов, которые формируют
ион\-но-зву\-ко\-вую
турбулентность~\cite{Gorshenin2011MM1}). Получе\-ние подобных результатов на основании
анализа порт\-ре\-тов волатильности~\cite{Korolev2011} было бы невозможным без экспертных оценок
специалистов из областей~--- поставщиков данных. В~то же время при реализации общих принципов
обработки с~помощью СРС-метода каждый исследователь придерживается своих представлений об удобстве
отображения результатов, что, безусловно, не способствует универсальности графического вывода для
различных предметных областей. В~на\-сто\-ящей работе предлагается новое средство визуализации,
учитывающее все особенности и~тонкости СРС-ме\-то\-да, предлагающее максимальную наглядность и~упрощающее
интерпретацию результатов для широкого круга специалистов в~области интеллектуального анализа данных.

\section{Основы метода скользящего разделения смесей}

В рамках СРС-метода традиционно используется многомерная интерпретация
во времени такого важного
параметра стохастических процессов, как волатильность. Предполагается, что
волатильность может быть разложена на так называемые динамическую и~диффузионную
компоненты, которые позволяют отслеживать появление различных по своей природе
эффектов в~модели
(например, трендовую составляющую и~совокупное влияние значительного числа случайных факторов на
основной процесс). С~помощью СРС-метода каждая из указанных составляющих может
быть представлена в~виде совокупности
различных компонент, обычно соответствующих определенным реальным про\-цессам.

Предполагается, что моделирование стохастических процессов осуществляется
с~помощью обобщенных
процессов Кокса со скачками, имеющими конечную дисперсию, так как данные модели являются в~определенном смысле наилучшими для аппроксимации неоднородных хаотических потоков. Таким образом,
задача статистической реконструкции распределений процессов сводится к~оценке параметров неизвестного
смешивающего распределения. Для корректности решения данной задачи,
а~также с~целью упрощения вида
конечной модели, проводится аппроксимация конечными
сдвиг-мас\-штаб\-ны\-ми смесями нормальных законов. При этом такое приближение неизвестного распределения
естественным образом приводит к многомерной интерпретации волатильности (подробнее об этом~---
в~книге~\cite{Korolev2011}).

Итак, в~рамках СРС-метода предполагается, что неизвестное распределение
некоторого процесса~$Z$ можно представить в~виде:
\begin{equation}
F_Z(x)=\sum\limits_{i=1}^kp_j F_i(x)\,, \label{Mixture}
\end{equation}
где
\begin{align}
\hspace*{-5mm}F_i(x)&=\fr{1}{\sigma_i\sqrt{2\pi}}\int\limits_{-\infty}^{x}\exp
\left\{-\fr{\left(t-a_i\right)^2}{2\sigma_i^2}\right\}\,dt, \notag\\
& \hspace*{25mm}x\in\R\,,\ a_i\in\R\,,  \sigma_i>0\,; \label{NormMixt}\\
\sum\limits_{i=1}^{k}p_{i}&=1\,,\enskip p_{i}\geqslant 0\,.
\label{Weights}
\end{align}

Модель вида~\eqref{Mixture} называется конечной смесью распределений $F_i(x)$.
В~частности, если
$F_i(x)$ определяются формулами вида~\eqref{NormMixt}, то говорят о~{конечных смесях нормальных
законов}. Параметры  $p_{1},\ldots,p_{k}$ называются {весами} компонент
$F_{1}(x),\ldots$\linebreak $\ldots,F_{k}(x)$, при этом предполагается справедливость условия~\eqref{Weights}. Параметр
$k$ в~приведенных выше формулах~--- количество компонент смеси.

Правую часть~\eqref{Mixture} с~учетом~\eqref{NormMixt} и~обозначения $\Phi(x)$ для стандартной
нормальной функция распределения можно переписать в~виде:
\begin{multline*}
F_Z(x)={\mathbb P}(Z<x)=\sum_{i=1}^k
p_i\Phi\left(\fr{x-a_i}{\sigma_j}\right)={}\\
{}=\E\Phi\left(\fr{x-V}{U}\right)\,,
\end{multline*}
где пара случайных величин $U,V$ имеет дискретное
распределение
\begin{equation*}
{\mathbb P}((U,V)=(\sigma_i,a_i))=p_i,\quad i=1,\ldots,k\,.
\end{equation*}
Волатильность естественно отождествить с~величиной
\begin{equation}
{\mathbb D} Z={\mathbb D} V+\E U^2 \label{DZ}
\end{equation}
(или $\sqrt{{\mathbb D} Z}$; о~справедливости самого пред\-став\-ления~\eqref{DZ}
подробнее см.\ книгу~\cite{Korolev2011}). При этом величина ${\mathbb D} V$ 
в формуле~\eqref{DZ} зависит
только от весов~$p_i$ и~параметров сдвига~$a_i$
компонент, а~потому характеризует ту часть волатильности, которая
обуслов\-ле\-на наличием локальных трендов, т.\,е.\ {динамическую}
компоненту волатильности. Величина $\E U^2$ в~\eqref{DZ} зависит только от весов~$p_i$ и~параметров
масштаба (<<коэффициентов диффузии>>) $\sigma_i$ компонент и~потому характеризует
{диффузионную}
компоненту волатильности.

Статистические закономерности поведения стохастических процессов зачастую изменяются во времени
нерегулярным образом, результатом чего является отсутствие универсального смешивающего
закона. Таким образом, чтобы изучить динамику изменения статистических закономерностей в~поведении
исследуемого хаотического процесса, задача статистического разделения конечных
смесей нормальных законов должна быть последовательно решена на интервалах времени, постоянно
сдвигающихся в~направлении <<астрономического>> времени. Такие интервалы в~рамках СРС-метода принято
называть {окнами} и~проводить оценивание неизвестных параметров на каждом из положений окна.
Обычно размер окна (т.\,е.\ количество со\-став\-ля\-ющих элементов) выбирается заранее и~не изменяется в~процессе работы. Для получения наиболее точной картины окно на каждом шаге <<сдвигается>> только на
один элемент выборки. Это позволяет отследить момент формирования или исчезновения той или иной
компоненты.

Несомненным достоинством СРС-ме\-то\-да является выявление объективно
существующих диффузионных и~динамических компонент во\-ла\-тиль\-ности.
Таким образом, сначала автоматически выделяют\-ся
компоненты, формирующие стохастический процесс (со статистической оценкой их параметров), и~лишь
затем возникает задача поиска соответствия выделенных компонент предметной области. Это позволяет
избегать предположения о~малой значимости (или наоборот~--- значительном вкладе) того или иного
явления на этапе априорного анализа.

\begin{figure*}[b] %fig1
\vspace*{1pt}
 \begin{center}
 \mbox{%
 \epsfxsize=134.5mm
 \epsfbox{gor-1.eps}
 }
 \end{center}
 \vspace*{-9pt}
\Caption{Вид начального экрана приложения}\label{Init}
\end{figure*}


\section{Требования к графическому выводу метода скользящего разделения смесей}

При практической реализации СРС-ме\-то\-да возникает задача
отображения изменяющихся во времени
динамической и~диффузионной компонент. Помимо численного значения соответствующих параметров на
каждом шаге каждая оценка имеет свой вес, который не связан с~абсолютной величиной отображаемого
параметра, но который также необходимо продемонстрировать на графике.
При работе с~СРС-методом
на первый план зачастую выходит экспертная оценка результатов,
полученных в~автоматическом режиме, а~потому качество графического вывода приобретает первостепенное
значение.

Для отображения результатов традиционно~\cite{Korolev2011} используются двумерные графики, в~которых
каж\-дой точке по оси абсцисс соответствует текущее положение окна, а~по оси ординат
откладываются значения оценок, полученных на данном шаге. Для изображения весов на графиках
используется цветовая шкала с~плавной градацией от тем\-но-си\-не\-го до
тем\-но-крас\-но\-го, при этом по
специальному правилу каждому весу из сегмента $[0,1]$ ставится в~соответствие цвет по шкале
\verb"RGB". Однако при работе с~реальными данными часто возникает множество оценок с~небольшими
весами, которые скорее относятся к <<шумам>>, т.\,е.\ погрешностям вычислений, и~не несут
значительной смысловой нагрузки, но затрудняют комфортное восприятие основных компонент.
%
Для
преодоления такой сложности можно использовать различные способы. Например, осуществлять отбрасывание
компонент с~некоторым весом при выводе графиков. Но такая <<фильтрация>> вывода, очевидно, может
существенно изменить сами результаты, так как далеко не во всех ситуациях можно корректно установить
порог отсечения для весов. Возможно при рисовании точек использовать такой параметр, как
прозрачность, однако при типографской печати таких графиков могут возникать определенные сложности,
также ухудшающие восприятие информации. В~на\-сто\-ящей статье предложено решение проблемы,
основанное на определении размера выводимой точки в~зависимости от веса соответствующего параметра,
а~именно: размер каждой точки при рисовании задается формулой $\lceil p_i^{(m)}
\cdot\mathrm{Size}_{\max}\rceil$, где $p_i^{(m)}$ обозначает вес компоненты с~номером~$i$ на $m$-м итерационном
шаге, а~Size$_{\max}$~--- некоторое заранее заданное максимальное значение размера выводимой точки.
Кроме того, максимальный размер Size$_{\max}$ варьируется для разных размеров выборок: для более
длинных рядов разумно использовать меньшее значение, чтобы отдельные точки не сливались друг с~другом
(особенно это важно при рисовании объектов с~близкими значениями оценок параметров и~весов).

\begin{figure*}[b] %fig2
\vspace*{-3pt}
 \begin{center}
 \mbox{%
 \epsfxsize=160mm
 \epsfbox{gor-2.eps}
 }
 \end{center}
 \vspace*{-9pt}
\Caption{Пример вывода диффузионной компоненты для некоторого ряда Params}\label{Diff}
\end{figure*}

\vspace*{-4pt}

\section{Описание функциональных возможностей программного продукта}


Перейдем к описанию возможностей средства визуализации результатов для СРС-ме\-то\-да. Начальный экран
при запуске приложения пред\-став\-лен на рис.~\ref{Init}.


Область <<\verb"График оценок параметров">> (<<\verb"Fig-" \verb"ure">>)
предназначена для непосредственной\linebreak
визуализации оценок, полученных с~по\-мощью ка\-ко\-го-ли\-бо метода.
В~начале работы обе оси промаркированы
в~диапазоне от~0 до~1 с~шагом~0,1. Однако при отрисовке актуального
графика обозначения будут
автоматически выбраны в~соответствии с~данными.

Кнопка с~надписью <<\verb"Русский">> (<<\verb"English">>) поз\-во\-ляет выбирать язык интерфейса (по
умолчанию установлено отображение на русском языке, нажатие на кнопку изменяет все надписи на
англоязычные варианты), содержимое остальных полей при переключении не изменяется, область вывода
графика не перерисовывается, не происходит повторного вывода значений оценок.

Блок <<\verb"Диапазон элементов">> (<<\verb"Gap for win-" \verb"dows">>) задает область вывода оценок по
временн$\acute{\mbox{о}}$й оси, соответствующей количеству сдвигов окна в~СРС-ме\-то\-де. Например, если есть
необходимость рассмотреть крупнее отдельную область или анализируемый ряд слишком большой (оценки
сливаются), то можно отобразить только часть данных.
В~качестве значений по умолчанию используется
диапазон от первого до сотого элемента.

Блок <<\verb"Диапазон вывода параметров">> (<<\verb"Param-" \verb"eter gap">>) задает область вывода значений
оценок. Для разных данных получаются разные по порядку оценки, кроме того, динамическая компонента
может принимать и~отрицательные значения. Для удобства масштабирования и~корректности вывода
параметров и~предназначен данный блок. В~качестве значений по умолчанию используется диапазон от~0
до~1.

Таблица <<\verb"Оценки параметров">>
(<<\verb"Estimations" \verb"of parameters">>) отображает полный набор
оцененных параметров из блока <<\verb"Диапазон" \verb"элементов">>, который хранится в~файле, адрес (имя)
которого задается в~поле <<\verb"Имя файла">> (<<\verb"Filename">>).

По умолчанию предполагается, что отоб\-ра\-жа\-ет\-ся диффузионная компонента
(<<\verb"Diffusive" \verb"component">>), однако в~блоке <<\verb"Выбор"
\verb"типа" \verb"графика">>
(<<\verb"Type" \verb"of" \verb"figure">>) это можно изменить, выбрав динамическую компоненту
(<<\verb"Dynamic" \verb"component">>). Нажатие кнопки <<\verb"Refresh">> осуществит корректное обновление
(либо первичное изображение) графика в~соответствии с~выбранными настройками с~по\-мощью специально
разработанного для СРС-ме\-то\-да алгоритма рисования.

\begin{figure*} %fig3
\vspace*{1pt}
 \begin{center}
 \mbox{%
 \epsfxsize=160mm
 \epsfbox{gor-3.eps}
 }
 \end{center}
 \vspace*{-10pt}
\Caption{Пример вывода динамической компоненты для некоторого ряда P1
\label{Dyn}
(англоязычный вариант интерфейса)}
\vspace*{-3pt}
\end{figure*}

\vspace*{-6pt}

\section{Примеры}

\vspace*{-2pt}

В настоящем разделе рассмотрим примеры применения разработанного средства визуализации для реальных
данных. В~данном случае не станем уделять внимание предметным областям, которые выступали в~роли
поставщиков выборок, а~сосредоточимся исключительно на демонстрации описанных выше возможностей
программы.

На рис.~\ref{Diff} изображена диффузионная компонента для некоторого ряда с~названием \verb"Params".
Вывод осуществляется для положений окон от~4000 до~9000,
при этом границы параметров установлены
от~0 до~$1{,}1\cdot10^{-4}$.



В данной ситуации расчеты проводились с~высокой точностью
(с~пороговым значением $10^{-12}$ для
критерия останова), для максимального количества компонент в~смеси использовалась величина $k=2$.
Визуально четко выделяются две фор\-ми\-ру\-ющие компоненты~---
тем\-но-крас\-ная и~составленная из точек разных
цветов (голубой, зеленый, желтый, оранжевый). Шумовые компоненты с~очень маленькими весами
практически отсутствуют, например можно выделить всего несколько небольших (напомним, что вес и~размер точки для рисования в~разработанном средстве визуализации тесно связаны) отметок между
делениями 5500 и~6000 на оси абсцисс. В~такой ситуации какая-либо фильтрация данных для улучшения
визуального восприятия не требуется. В~окне <<Оценки параметров>> отображается табличное
представление параметров, где первые две строки соответствуют значениям для динамической компоненты,
третья и~четвертая~--- диффузионной, а~пятая и~шестая строки содержат изменяющиеся во времени значения
весов.

На рис.~\ref{Dyn} изображена динамическая компонента волатильности для другого ряда с~именем
\verb"P1". Вывод осуществляется от первого положения окна до значения 4800, при этом границы
параметров определяются диапазоном от $-0{,}08$ до~0,08. Также отметим, что рис.~\ref{Dyn}
демонстрирует вариант вывода с~англоязычными надписями.



Для данного ряда в~качестве максимального количества компонент использовалось значение
$k\hm=6$, а~вычисления
проводились с~точ\-ностью $10^{-5}$. На графике присутствуют от двух до трех основных компонент, при
этом общая картина существенно дополняется шумовыми составляющими. Однако использование точек
небольшого размера для значений параметров с~малыми весами позволяет анализировать приведенный график
без проведения какой-либо фильтрации достаточно уверенно. Таблица в~окне <<Оценки параметров>>
построена по тому же принципу, как и~для предшествующего графика, но теперь диапазон строк для
динамической компоненты~--- 1--6, для диффузионной~--- 7--12, а~строки~13--18
соответствуют весам.

\section{Заключение}

Средство визуализации создано с~помощью встроенного языка программирования системы \verb"MATLAB" и~ориентировано на работу с~файлами формата \verb"MAT". Однако  не представляет каких-либо сложностей
экспортировать данные и~из текстовых файлов \verb"TXT", таблиц \verb"CSV".
Это позволяет осуществлять
расчеты с~помощью методов, реализованных с~помощью произвольных языков программирования (например,
специализированных, достаточно низкоуровневых~--- для повышения скорости вычислений), а~затем работать
с~удобным средством визуализации.

Это особенно важно для групп исследователей, осуществляющих
обработку различными (причем и~с~программной, и~с~математической точки зрения) способами, так как
общие итоговые графики будут унифицированы за счет единого интерфейса.

Предусмотрена возможность автоматического сохранения области графика в~файл \verb"PNG", что удобно
для представления результатов в~рамках презентаций, журнальных публикаций. Таким образом,
разработанное средство визуализации можно использовать не только в~качестве решения для проведения
интеллектуального анализа данных, но и~вспомогательного инструмента подготовки отчетов по научным и~практическим исследованиям. Кроме того, описанное в~работе средство визуализации обладает интуитивно
понятным визуальным интерфейсом, а~потому подойдет как для продвинутых пользователей, так и~для тех,
кто только начинает изучение возможностей СРС-метода.

Для разработанного программного кода в~Роспатенте получено свидетельство государственной регистрации
программ для ЭВМ  <<Средство визуализации результатов для метода скользящего разделения смесей>>
(автор Горшенин~А.\,К., №\,2014661369 от 29.10.2014).

В качестве возможного пути дальнейшего развития созданного продукта можно отметить возможность
использования интерфейса в~качестве базового модуля для построения информационной технологии,
интегрирующей методы визуализации с~инструментарием оценивания неизвестных параметров модели.
Разработка такого пакетного решения представляет собой совершенно самостоятельную и~в~достаточной
степени трудоемкую задачу.

{\small\frenchspacing
 {%\baselineskip=10.8pt
 \addcontentsline{toc}{section}{References}
 \begin{thebibliography}{9}
\bibitem{Korolev2011}
\Au{Королев~В.\,Ю.}
Ве\-ро\-ят\-но\-ст\-но-ста\-ти\-сти\-че\-ские методы декомпозиции
волатильности хаотических процессов.~--- М.: Изд-во Моск. ун-та, 2011. 512~с.

\bibitem{Skvortsova2006}
\Au{Skvortsova N.\,N., Batanov~G.\,M., Malakhov~D.\,V.,
Petrov~A.\,E., Saenko~V.\,V., Sarksyan~K.\,A., Kharchev~N.\,K.,
Kholnov~Yu.\,V., Korolev~V.\,Yu., Zhukov~Yu.\,V., Rey~M., Merkulov~A.\,S.,
Shatalin~S.\,V., Lashkul~S.\,I., Vekshina~E.\,O., Popov~A.\,Yu.}
Estimation of dynamic and diffusive
components in edge turbulent particle fluxes in the L-2M stellarator and the FT-2
tokamak~//
21st IAEA Fusion Energy Conference. Chengdu, 2006. IAEA-CN-149, PD/P6-3.

\bibitem{Gorshenin2008} \Au{Горшенин~А.\,К., Королев~В.\,Ю., Турсунбаев~А.\,М.}
Медианные
модификации EM- и~SEM-ал\-го\-рит\-мов для разделения смесей вероятностных
распределений и~их применение к~декомпозиции волатильности финансовых временных
рядов~// Информатика и~её применения, 2008. Т.~2. Вып.~4. C.~12--47.

\bibitem{Gorshenin2011MM1}
\Au{Батанов~Г.\,М., Горшенин~А.\,К., Королев~В.\,Ю.,
Малахов~Д.\,В., Скворцова~Н.\,Н.}
Эволюция вероятностных характеристик низкочастотной турбулентности плазмы
в~микроволновом поле~// Математическое моделирование, 2011. Т.~23. №\,5. C.~35--55.
 \end{thebibliography}

 }
 }

\end{multicols}

\vspace*{-9pt}

\hfill{\small\textit{Поступила в редакцию 13.11.14}}

%\newpage

\vspace*{12pt}

\hrule

\vspace*{2pt}

\hrule

%\vspace*{12pt}

\def\tit{A~VISUALIZATION OF~ESTIMATORS IN~THE~METHOD OF~MOVING SEPARATION OF~MIXTURES}

\def\titkol{A~visualization of~estimators in~the~method of~moving separation of~mixtures}

\def\aut{A.\,K.~Gorshenin$^{1,2}$}

\def\autkol{A.\,K.~Gorshenin}

\titel{\tit}{\aut}{\autkol}{\titkol}

\vspace*{-9pt}

\noindent
$^1$Institute of Informatics Problems, Russian Academy of Sciences,
44-2~Vavilov Str., Moscow 119333, Russian\linebreak
$\hphantom{^1}$Federation

\noindent
$^2$Moscow Institute of Radio, Electronics,
and Automation (MIREA), 78 Prosp.\ Vernadskogo, Moscow 119454,\linebreak
$\hphantom{^1}$Russian Federation


\def\leftfootline{\small{\textbf{\thepage}
\hfill INFORMATIKA I EE PRIMENENIYA~--- INFORMATICS AND
APPLICATIONS\ \ \ 2014\ \ \ volume~8\ \ \ issue\ 4}
}%
 \def\rightfootline{\small{INFORMATIKA I EE PRIMENENIYA~---
INFORMATICS AND APPLICATIONS\ \ \ 2014\ \ \ volume~8\ \ \ issue\ 4
\hfill \textbf{\thepage}}}

\vspace*{3pt}

\Abste{The method of moving separation of mixtures (MSM method) is a~powerful
tool for analyzing different stochastic processes. Using the MSM method within
the experts' conclusions for the results obtained by iterative numerical procedures,
 a~number of important results were achieved in the physics of turbulent plasma,
 a~few mathematical models for the functioning of financial markets were refined.
 In most cases, research teams present results in a~form which is convenient
 just for themselves, and it is difficult for experts to compare and interpret
 results, especially, in the case when the model is tested on
 fundamentally dissimilar  samples from different subject areas.
 The paper presents a~visualization tool for displaying parameter estimates
 independently of the used numerical methods. The tool is convenient for
 researchers and experts.}

\KWE{method of moving separation of mixtures; user interface; normal mixtures;
probabilistic models; data mining}

\DOI{10.14357/19922264140410}

\vspace*{-3pt}

\Ack
\noindent
The research was partially financially supported by the President 
Grant for Government Support of Young Russian Scientists 
MK-4103.2014.9.

%\vspace*{3pt}

  \begin{multicols}{2}

\renewcommand{\bibname}{\protect\rmfamily References}
%\renewcommand{\bibname}{\large\protect\rm References}



{\small\frenchspacing
 { %\baselineskip=10pt
 \addcontentsline{toc}{section}{References}
 \begin{thebibliography}{9}
\bibitem{1-gor}
\Aue{Korolev, V.\,Yu.} 2011.
\textit{Veroyatnostno-statisticheskie metody dekompozitsii volatil'nosti
khaoticheskikh protsessov} [Probabilistic and statistical methods of decomposition of volatility of
chaotic processes]. Moscow: Moscow University Publishing House. 512~p.

\bibitem{2-gor}
\Aue{Skvortsova, N.\,N., G.\,M.~Batanov, D.\,V.~Malakhov,
A.\,E.~Petrov, V.\,V.~Saenko, K.\,A.~Sarksyan, N.\,K.~Kharchev,
Yu.\,V.~Kholnov, V.\,Yu.~Korolev, Yu.\,V.~Zhukov, M.~Rey, A.\,S.~Merkulov,
S.\,V.~Shatalin, S.\,I.~Lashkul, E.\,O.~Vekshina, and A.\,Yu.~Popov}.
 2006. Estimation of
 dynamic and diffusive components in edge turbulent
particle fluxes in the L-2M stellarator and the FT-2 tokamak.
\textit{21st IAEA Fusion Energy
Conference}. Chengdu. IAEA-CN-149, PD/P6-3.

\bibitem{3-gor}
\Aue{Gorshenin, A.\,K.,  V.\,Yu.~Korolev, D.\,V.~Malakhov, and A.\,M.~Tursunbaev}.
2008.   Mediannye
modifikatsii EM- i~SEM-al\-go\-rit\-mov dlya razdeleniya smesey ve\-ro\-yat\-nost\-nykh
raspredeleniy i~ikh primenenie k~dekompozitsii volatil'nosti finansovykh
vremennykh ryadov [Median modification of EM- and SEM-algorithms for
separation of mixtures of probability distributions and their application to
the decomposition of volatility of financial time series].
\textit{Informatika i~ee Primeneniya}~--- \textit{Inform. Appl.} 2(4):12--47.

\bibitem{4-gor}
\Aue{Batanov, G.\,M., A.\,K.~Gorshenin,  V.\,Yu.~Korolev, D.\,V.~Malakhov,
 and N.\,N.~Skvortsova}. 2011.
Evolyutsiya veroyatnostnykh kharakteristik nizkochastotnoy turbulentnosti plazmy v mikrovolnovom pole
[The evolution of probability characteristics of low-frequency plasma turbulence].
\textit{Matematicheskoe Modelirovanie} [Mathematical Modeling]
23(5):35--55.
\end{thebibliography}

 }
 }

\end{multicols}

\vspace*{-6pt}

\hfill{\small\textit{Received November 13, 2014}}

\vspace*{-18pt}


\Contrl

\noindent
\textbf{Gorshenin Andrey K.}  (b.\ 1986)~---
Candidate of Science (PhD) in physics and mathematics, senior scientist,
Institute of Informatics Problems, Russian Academy of Sciences,
44-2~Vavilov Str., Moscow 119333, Russian Federation;
associate professor, Moscow Institute of Radio, Electronics,
and Automation (MIREA), 78 Prosp.\ Vernadskogo, Moscow 119454, Russian Federation;
agorshenin@ipiran.ru

\label{end\stat}

\renewcommand{\bibname}{\protect\rm Литература}