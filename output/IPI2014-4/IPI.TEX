\documentclass[10pt]{book}
\usepackage[utf8]{inputenc}

\usepackage{latexsym,amssymb,amsfonts,amsmath,indentfirst,shapepar,%fleqn,%
picinpar,shadow,floatflt,enumerate,multicol,colortbl,ipi}

\usepackage{rotating}
\usepackage{mathrsfs}
\usepackage[noend]{algorithmic}
\usepackage{ulem}

\input{epsf}

%\nofiles

%\includeonly{avtor} %+pdf
%\includeonly{obchak,avtor}
%\includeonly{pred}      %
%\includeonly{podgot-rus,podgot-eng}  %+pdf
%\includeonly{ocherk} %+
%\includeonly{nekrol} %+

%\includeonly{eroshenko}         %мор+pdf+авт+авт
%\includeonly{sorokin}           %англ+мор+авт+pdf
%\includeonly{grusho}            %+авт+(мор)+pdf
%\includeonly{zats}              %+авт+(мор)+pdf+авт
%\includeonly{bronstein}         %+авт(мор)+pdf+авт
%\includeonly{grusho-2}          %+авт(мор)+pdf
%\includeonly{korolev}           %+авт+мор+pdf
%\includeonly{mironov}           %+авт(мор)+pdf
%\includeonly{pechinkin}         %авт+(мор)pdf
%\includeonly{kozerenko}         %+авт+(мор)pdf
%\includeonly{vovchenko}         %+авт+(мор)+pdfавт
%\includeonly{chertok}           %авт+(мор)pdf
%\includeonly{gorshenin}         %+авт+(мор)pdf
%\includeonly{kalin}             %+авт+(мор)+pdf

%\includeonly{ubilei-budzko} %pdf


%\includeonly{toc-rus, toc-en}
%\includeonly{obchak} %,toc-en}

%\includeonly{rekl}
%\includeonly{rekl-1}
%\includeonly{reshal}  %
%\includeonly{eng-index}
%\includeonly{cover3}

\usepackage{acad}
%\usepackage{courier}
\usepackage{decor}
\usepackage{newton}
\usepackage{pragmatica}
\usepackage{zapfchan}
\usepackage{petrotex}
\usepackage{bm}                     % полужирные греческие буквы
\usepackage{upgreek}                % прямые греческие буквы
\usepackage{eufrak}
\usepackage{verbatim}

\renewcommand{\bottomfraction}{0.99}
\renewcommand{\topfraction}{0.99}
\renewcommand{\textfraction}{0.01}

\setcounter{secnumdepth}{1} %здесь - 3 + chapter = 4

\arraycolsep=1.5pt

%\usepackage[pdftex]{graphicx}

%\usepackage{oz}

%NEW COMMANDS


\renewcommand*{\hm}[1]{#1\nobreak\discretionary{}%
            {\hbox{$\mathsurround=0pt #1$}}{}} %% Дублирует знаки операций
                               %при переносе в формуле (перед знаком, который
                               %надо продублировать ставится команда \hm)

%\newcommand{\endproof}{\hfill$\Box$}
\renewcommand{\r}{\mathbb{R}}
\newcommand{\I}{{\rm I\hspace{-0.7mm}I}}
%\newcommand{\Ikl}{{\tt{1}}\hspace*{-1.44mm}\mathtt{1}}
\newcommand{\Ik}{\mbox{{\small \tt {1}}\hspace{-1.5mm}{\tt 1}}}
\newcommand{\argmin}{\mathop{\mathrm{arg}\,\mathrm{min}}}
\newcommand{\argmax}{\mathop{\mathrm{arg}\,\mathrm{max}}}
%\newcommand{\capr}{\mathop{\cap\,}}
%\newcommand{\cupr}{\mathop{\cup\,}}
%\def\argmin{\mathop{arg\,min}}

\def\vrp{\varphi}
\def\prt{\partial}
\def\mm{{\rm M}}
\def\modnop#1{\mathop{#1}\limits_{n}}
\def\eam{\mathbin{{\mathop{=}\limits^{\mathrm{def}}}}}
\def\dey#1#2{#1 (#2)}
\def\deyc#1#2{#1 \cdot  #2}
\def\ra#1{\;\mathop{\to}\limits^{#1}\;}
\def\raz#1{\;\mathop{\longrightarrow}\limits^{\!\!\!#1}\;}
\def\ral#1{\;\mathop{\longrightarrow}\limits^{#1}\;}

\newcommand{\Nor}{\mathcal{N}}
\newcommand{\T}{\mathbb{T}}
\newcommand{\Z}{\mathbb{Z}}



\newcommand{\il}[2]{\int\limits_{#1}^{#2}}%интеграл с пределами #1 и #2

%\def\ss2{\mathop {\sum\limits^p\sum\limits^p}}
\def\sss{\sum\limits}
\def\tr{,\,\ldots\,,\,}
\def\rk{\right]}
\def\lk{\left[}
\def\rf{\right\}}
\def\lf{\left\{}
\def\lv{\,\left\vert}
\def\rv{\right\vert\,}
\def\iii{\int\limits}
\def\iin{\int\limits_{-\infty}^\infty}
\def\rrv{\right\vert}


\def\ee{{\cal E}}
\def\ww{{\cal W}}
\def\yy{{\cal Y}}
\def\vv{{\cal V}}

\newcommand{\R}{\mathbb R}
\newcommand{\E}{\mathbb E}
\newcommand{\N}{\mathbb N}

\renewcommand{\P}{\mathbb{P}}

\newcommand{\h}{{\bf H}}
\newcommand{\p}{{\sf P}}  % вероятность

\newcommand{\e}{{\sf E}}  % мат. ожидание
\newcommand{\D}{{\sf D}}  % дисперсия
\newcommand{\eps}{\varepsilon}
\newcommand{\vp}{{\mathbf p}}
\newcommand{\vz}{{\mathbf z}}
\newcommand{\vx}{{\mathbf x}}
\newcommand{\vf}{{\mathbf f}}
\newcommand{\F}{{\mathcal F}}
\def\ap{{\mathrm{ЭР}}}
\newcommand{\ud}{\Delta_n} %uniform ditance
\newcommand{\nud}{\Delta_n(x)}
\renewcommand{\Re}{\mathrm{Re}\,}

\newcommand{\abs}[1]{\left\vert#1\right\vert}
\newcommand{\norm}[1]{\left\Vert#1\right\Vert}
\def\da{(\Delta_t,A)}

\newcommand{\corr}{\mathrm{corr}}

\newcommand{\cov}{\mathrm{cov}}
\newcommand{\Expect}{\mathbb{E}}

\def\w{\omega}
\def\W{\Omega}

\def\inh{\int\limits_{nh}^{(n+1)h}}

\def\sumin{\sum_{i=1}^N}


\def\bxt{(Y,t)}
\def\xt{(y,t)}

\def\ovth{{\fr{\tau-nh}{h}}}
\def\ov{\overline}
\def\tm{\tilde m}


\DeclareMathOperator{\sign}{sign}

%\newcommand{\gr}{{\geqslant}}


\newcommand{\g}{\mbox{\textit{g}}}

\renewcommand{\la}{\lambda}
\newcommand{\si}{\sigma}
\newcommand{\alp}{\alpha}

%\newcommand{\pto}{\stackrel{P}{\longrightarrow}} % сходимость по веpоятности

\newcommand{\eqd}{\stackrel{\mathrm{d}}{=}} % равенство по pаспpеделению
\newcommand{\eqdelta}{\stackrel{\Delta}{=}} % равенство по pаспpеделению

\def\be#1{\begin{equation}\label{#1}}
\def\ee{\end{equation}}
\def\re#1{(\ref{#1})}

\def\bn{\begin{enumerate}}
\def\en{\end{enumerate}}
\def\bi{\begin{itemize}}
\def\ei{\end{itemize}}
\def\i{\item}

%\newcommand{\kp}{\kappa}
%\def\Q{{\cal Q}} \def\H{{\cal H}}
%\newcommand{\bet}{\beta_{2+\delta}}


%\newtheorem{definition}{Определение}
%\renewcommand{\thedefinition}{\arabic{definition}.}
%END NEW COMMANDS

%\renewcommand{\baselinestretch}{1.2}

%\pagestyle{myheadings}

\setlength{\textwidth}{167mm}      % 122mm
\setlength{\textheight}{658pt}
%\setlength{\textheight}{635.6pt}
\setlength{\columnsep}{4.5mm}

\setcounter{secnumdepth}{4}

%\addtolength{\headheight}{2pt}
%\addtolength{\headsep}{-2mm}

%\addtolength{\topmargin}{-20mm}  % for printing


%\hoffset=-30mm  % From Yap
\hoffset=-23mm  % From Acrobat

%\voffset=0mm % From Yap
%\voffset=-15mm   % From Acrobat

\addtolength{\evensidemargin}{-9.5mm} % for printing
\addtolength{\oddsidemargin}{9.5mm}  % for printing

%\renewcommand{\thefootnote}{\fnsymbol{footnote}}
%\renewcommand{\thefootnote}{\arabic{footnote}}
\renewcommand{\figurename}{\protect\bf Рис.}
\renewcommand{\tablename}{\protect\bf Таблица}

\newcommand{\Caption}[1]{\caption{\protect\small %\baselineskip=2.5ex
#1}}

\renewcommand{\thefigure}{\arabic{figure}}
\renewcommand{\thetable}{\arabic{table}}
\renewcommand{\theequation}{\arabic{equation}}
\renewcommand{\thesection}{\arabic{section}}

\renewcommand{\contentsname}{СОДЕРЖАНИЕ}
\newcommand{\fr}[2]{\displaystyle\frac{\displaystyle #1\mathstrut}{\displaystyle #2\mathstrut}}

%\renewcommand{\thefootnote}{\fnsymbol{footnote}}
%\newcommand{\g}{\mbox{\textit{g}}}

%\newcommand{\Caption}[1]{\caption{\protect\small\baselineskip=2ex #1}}
\newcounter{razdel}
\setcounter{razdel}{0}


\newcommand{\titel}[4]{%
\

\vspace*{5pt}

\ifodd\therazdel {\raggedright\noindent\Large\textrm\textbf
 \lineskip .75em
  \baselineskip=3.2ex #1 \par}
\vskip 1em {\noindent\large\textrm\textbf #2 \par}
\addcontentsline{toc}{subsection}{{\textrm\textbf #3}\protect\newline #1}
\def\rightheadline{\underline{\noindent\hbox to \textwidth{\hfill\small\textrm{#4}
%\hfill \large\bf\thepage
}}}
\def\leftheadline{\underline{\noindent\parbox{\textwidth}{
%\raggedleft\large\bf\thepage \hfill
\small\textit{#3}\hfill}}}
\def\leftfootline{\small{\textbf{\thepage}
\hfill ИНФОРМАТИКА И ЕЁ ПРИМЕНЕНИЯ\ \ \ том~8\ \ \ выпуск 4\ \ \ 2014}
}%
 \def\rightfootline{\small{ИНФОРМАТИКА И ЕЁ ПРИМЕНЕНИЯ\ \ \ том~8\ \ \ выпуск~4\ \ \ 2014
\hfill \textbf{\thepage}}}
\vskip 2em \setcounter{figure}{0}
\setcounter{table}{0}
\setcounter{equation}{0}
\setcounter{section}{0}
\setcounter{subsection}{0}
\setcounter{subsubsection}{0}
\setcounter{footnote}{0}
\setcounter{razdel}{0}
%\end{flushleft}
\else {
 \raggedright\noindent\Large\textrm\textbf
 \lineskip .75em
\baselineskip=3.2ex #1 \par} \vskip 1em
%\begin{flushleft}
{\noindent\large\textrm\textbf #2 \par}
\addcontentsline{toc}{subsection}{{\textrm\textbf #3}\protect\newline #1}
\def\rightheadline{\underline{\noindent\hbox to \textwidth{\hfill\small\textrm{#4}
%\hfill \large\bf\thepage
}}}
\def\leftheadline{\underline{\noindent\parbox{\textwidth}{%\raggedleft\large\bf\thepage \hfill
\small\textit{#3}\hfill}}}
\def\leftfootline{\small{\textbf{\thepage}
\hfill ИНФОРМАТИКА И ЕЁ ПРИМЕНЕНИЯ\ \ \ том~8\ \ \ выпуск~4\ \ \ 2014}
}%
 \def\rightfootline{\small{ИНФОРМАТИКА И ЕЁ ПРИМЕНЕНИЯ\ \ \ том~8\ \ \ выпуск~4\ \ \ 2014
\hfill \textbf{\thepage}}} \vskip 2em \setcounter{figure}{0}
\setcounter{table}{0} \setcounter{equation}{0} \setcounter{section}{0}
\setcounter{subsection}{0} \setcounter{subsubsection}{0}
\setcounter{footnote}{0}
%\end{flushleft}
\fi}

\newcommand{\titelr}[2]{%
\

\vspace*{5pt}

\ifodd\therazdel {\raggedright\noindent%\Large\textrm\textbf
 \lineskip .75em
  \baselineskip=3.2ex #1 \par}
\vskip 1em {\noindent\normalsize\textrm\textbf #2 \par}
\else {
 \raggedright\noindent\Large\textrm\textbf
 \lineskip .75em
\baselineskip=3.2ex #1 \par} \vskip 1em
%\begin{flushleft}
{\noindent\large\textrm\textbf #2 \par
%\noindent\normalsize\textrm\textbf #2 \par
} \fi}

\newcommand{\titele}[5]{%
\

%\vspace*{5pt}

\ifodd\therazdel {\raggedright\noindent\large
\textrm\textbf
 \lineskip .75em
%  \baselineskip=3.2ex
#1 \par}
\vskip .5em {\noindent\large\textrm\textbf #2 \par}
\vskip .5em
 {\noindent\textrm #3 \par}
\addcontentsline{toc}{subsection}{{\textrm\textbf #1}\protect\newline #2}
\def\rightheadline{\underline{\noindent\hbox to \textwidth{\hfill\small\textrm{#4}
%\hfill \large\bf\thepage
}}}
\def\leftheadline{\underline{\noindent\parbox{\textwidth}{
%\raggedleft\large\bf\thepage \hfill
\small\textrm{#5}\hfill}}}
\def\leftfootline{\small{\textbf{\thepage}
\hfill ИНФОРМАТИКА И ЕЁ ПРИМЕНЕНИЯ\ \ \ том~8\ \ \ выпуск~4\ \ \ 2014}
}%
 \def\rightfootline{\small{ИНФОРМАТИКА И ЕЁ ПРИМЕНЕНИЯ\ \ \ том~8\ \ \ выпуск~4\ \ \ 2014
\hfill \textbf{\thepage}}} \vskip 1em \setcounter{figure}{0}
\setcounter{table}{0} \setcounter{equation}{0} \setcounter{section}{0}
\setcounter{subsection}{0} \setcounter{subsubsection}{0}
\setcounter{footnote}{0} \setcounter{razdel}{0}
%\end{flushleft}
\else {
 \raggedright\noindent\large
 \textrm\textbf
 \lineskip .75em
%\baselineskip=3.2ex
#1 \par} \vskip .5em
%\begin{flushleft}
{\noindent\large\textrm\textbf #2 \par} \vskip .5em
 {\noindent\textrm #3 \par}
\addcontentsline{toc}{subsection}{{\textrm\textbf #1}\protect\newline #2}
\def\rightheadline{\underline{\noindent\hbox to \textwidth{\hfill\small\textrm{#4}
%\hfill \large\bf\thepage
}}}
\def\leftheadline{\underline{\noindent\parbox{\textwidth}{%\raggedleft\large\bf\thepage \hfill
\small\textrm{#5}\hfill}}}
\def\leftfootline{\small{\textbf{\thepage}
\hfill ИНФОРМАТИКА И ЕЁ ПРИМЕНЕНИЯ\ \ \ том~8\ \ \ выпуск~4\ \ \ 2014}
}%
 \def\rightfootline{\small{ИНФОРМАТИКА И ЕЁ ПРИМЕНЕНИЯ\ \ \ том~8\ \ \ выпуск~4\ \ \ 2014
\hfill \textbf{\thepage}}} \vskip 1em \setcounter{figure}{0}
\setcounter{table}{0} \setcounter{equation}{0} \setcounter{section}{0}
\setcounter{subsection}{0} \setcounter{subsubsection}{0}
\setcounter{footnote}{0}
%\end{flushleft}
\fi}

\def\Abst#1{
\begin{center}\small\nwt
\parbox{150mm}{%\baselineskip=2.5ex
\textbf{Аннотация:}\ \
%\hspace*{\parindent}
#1}
\end{center}}
\def\Abste#1{
\begin{center}\small\nwt
\parbox{150mm}{%\baselineskip=2.5ex
\textbf{Abstract:}\ \
%\hspace*{\parindent}
#1}
\end{center}}

\def\DOI#1{
\begin{center}\small\nwt
\parbox{150mm}{%\baselineskip=2.5ex
\textbf{DOI:}\ \
%\hspace*{\parindent}
#1}
\end{center}}

\def\Abstend#1{
\begin{center}\small\nwt
\parbox{150mm}{%\baselineskip=2.5ex
%\hspace*{\parindent}
#1}
\end{center}}


\def\KW#1{
\begin{center}\small\nwt
\parbox{150mm}{%\baselineskip=2.5ex
\textbf{Ключевые слова:}\ \ #1}
\end{center}}

\def\KWE#1{
\begin{center}\small\nwt
\parbox{150mm}{%\baselineskip=2.5ex
\textbf{Keywords:}\ \ #1}
\end{center}}


\def\KWN#1{
%\begin{center}
%\small
%\parbox{150mm}\end{center}
}

\renewcommand{\thesubsection}{\thesection.\arabic{subsection}\hspace*{-5pt}}
\renewcommand{\thesubsubsection}{\thesubsection\hspace*{5pt}.\arabic{subsubsection}\hspace*{-3pt}}

\newcommand{\Ack}{\section*{\protect\rmfamily Acknowledgments}\noindent}
\newcommand{\Contr}{\section*{\protect\rmfamily Contributors}\noindent}
\newcommand{\Contrl}{\section*{\protect\rmfamily Contributor}\noindent}


\begin{document}
\Rus

\nwt
%\ptb

%\renewcommand{\contentsname}{\protect\Large\bf Содержание}

\setcounter{tocdepth}{2}

%\tableofcontents

\renewcommand{\bibname}{\protect\rmfamily Литература}
  \def\Au#1{{\it #1}}
    \def\Aue#1{{#1}}

%\newcommand{\No}{№}
  \newcommand{\tg}{\,\mathrm{tg}\,}
    \newcommand{\ctg}{\,\mathrm{ctg}\,}
  \newcommand{\arctg}{\,\mathrm{arctg}\,}

\def\forallb{\mathop{\forall}}
\def\cupb{\mathop{\cup}}
\def\existsb{\mathop{\exists}}


\newpage
\addtocounter{razdel}{1}
%\def\razd{РЕГУЛИРУЕМЫЙ ЭЛЕКТРОПРИВОД ДЛЯ ЭЛЕКТРОЭНЕРГЕТИКИ}


\setcounter{page}{3}


%   { %\Large  
   { %\baselineskip=16.6pt
   
   \vspace*{-48pt}
   \begin{center}\LARGE
   \textit{Предисловие}
   \end{center}
   
   %\vspace*{2.5mm}
   
   \vspace*{25mm}
   
   \thispagestyle{empty}
   
   { %\small 

    
Вниманию читателей журнала <<Информатика и её применения>> предлагается 
очередной тематический выпуск <<Вероятностно-статистические методы и 
задачи информатики и информационных технологий>>. Предыдущие тематические 
выпуски журнала по данному направлению вышли в 2008~г.\ (т.~2, вып.~2), 
в 2009~г.\ (т.~3, вып.~3) и в 2010~г.\ (т.~4, вып.~2). 

Статьи, собранные в данном журнале, посвящены разработке новых вероятностно-статистических 
методов, ориентированных на применение к решению конкретных задач информатики и информационных 
технологий, а также~--- в ряде случаев~--- и других прикладных задач. Проблематика, охватываемая 
публикуемыми работами, развивается в рамках научного сотрудничества между Институтом проблем 
информатики Российской академии наук (ИПИ РАН) и Факультетом вычислительной математики и 
кибернетики Московского государственного университета им.\ М.\,В.~Ломоносова в ходе работ 
над совместными научными проектами (в том числе в рамках функционирования 
Научно-образовательного центра <<Вероятностно-статистические методы анализа рисков>>). 
Многие из авторов статей, включенных в данный номер журнала, являются активными участниками 
традиционного международного семинара по проблемам устойчивости стохастических моделей, 
руководимого В.\,М.~Золотаревым и В.\,Ю.~Королевым; регулярные сессии этого семинара 
проводятся под эгидой МГУ и ИПИ РАН (в 2011~г.\ указанный семинар проводится в октябре 
в Калининградской области РФ). 

Наряду с представителями ИПИ РАН и МГУ в число авторов данного выпуска журнала входят 
ученые из Научно-исследовательского института системных исследований РАН, Института 
проблем технологии микроэлектроники и особочистых материалов РАН, Института 
прикладных математических исследований Карельского НЦ РАН, Московского 
авиационного института, Вологодского государственного педагогического университета, 
НИИММ им.\ Н.\,Г.~Чеботарева, Казанского государственного университета, Дебреценского 
университета (Венгрия).

Несколько статей выпуска посвящено разработке и применению стохастических методов и 
информационных технологий для решения различных прикладных задач. В~работе В.\,Г.~Ушакова 
и О.\,В.~Шестакова рассмотрена задача определения вероятностных характеристик случайных 
функций по распределениям интегральных преобразований, возникающих в задачах эмиссионной 
томографии. В~статье Д.\,О.~Яковенко и М.\,А.~Целищева рассмотрены некоторые вопросы 
математической теории риска и предложен новый подход к диверсификации инвестиционных 
портфелей. Работа И.\,А.~Кудрявцевой и А.\,В.~Пантелеева посвящена построению и 
исследованию математической модели, описывающей динамику сильноионизованной плазмы. 
В~статье П.\,П.~Кольцова изучается качество работы ряда алгоритмов сегментации изображений. 
Статья А.\,Н.~Чупрунова и И.~Фазекаша посвящена вероятностному анализу числа без\-оши\-бочных 
блоков при помехоустойчивом кодировании; получены усиленные законы больших чисел для указанных 
величин.

В данном выпуске традиционно присутствует тематика, весьма активно разрабатываемая в течение 
многих лет специалистами ИПИ РАН и МГУ,~--- методы моделирования и управления для 
информационно-телекоммуникационных и вычислительных систем, в частности методы 
теории массового обслуживания. В~статье А.\,И.~Зейфмана с соавторами рассматриваются 
модели обслуживания, описываемые марковскими цепями с непрерывным временем в случае 
наличия катастроф. В~работе М.\,М.~Лери и И.\,А.~Чеплюковой рассматриваются случайные 
графы Интернет-типа, т.\,е.\ графы, степени вершин которых имеют степенные распределения; 
такие задачи находят применение при исследовании глобальных сетей передачи данных. 
Работа Р.\,В.~Разумчика посвящена исследованию систем массового обслуживания специального 
вида~--- с отрицательными заявками и хранением вытесненных заявок.

Ряд статей посвящен развитию перспективных теоретических 
вероятностно-статистических методов, которые находят широкое применение в различных 
задачах информатики и информационных технологий. В~работе В.\,Е.~Бенинга, А.\,К.~Горшенина 
и В.\,Ю.~Королева рассмотрена задача статистической проверки гипотез о числе компонент 
смеси вероятностных распределений, приводится конструкция асимптотически наиболее мощного 
критерия. Результаты этой работы найдут применение в ряде прикладных задач, использующих 
математическую модель смеси вероятностных распределений (в информатике, моделировании 
финансовых рынков, физике турбулентной плазмы и~т.\,д.). В~статье В.\,Ю.~Королева, 
И.\,Г.~Шевцовой и С.\,Я.~Шоргина строится новая, улучшенная оценка точности нормальной 
аппроксимации для пуассоновских случайных сумм; как известно, указанные случайные суммы 
широко используются в качестве моделей многих реальных объектов, в том числе в информатике, 
физике и других прикладных областях. Работа В.\,Г.~Ушакова и Н.\,Г.~Ушакова посвящена 
исследованию ядерной оценки плотности распределения; эти результаты могут применяться, 
в част\-ности, при анализе трафика в телекоммуникационных системах. Серьезные приложения 
в статистике могут получить результаты работы О.\,В.~Шестакова, в которой доказаны оценки 
скорости сходимости распределения выборочного абсолютного медианного отклонения к нормальному 
закону. 

\smallskip

Редакционная коллегия журнала выражает надежду, что данный тематический  выпуск 
будет интересен специалистам в области теории вероятностей и математической статистики 
и их применения к решению задач информатики и информационных технологий.
     
     %\vfill 
     \vspace*{20mm}
     \noindent
     Заместитель главного редактора журнала <<Информатика и её 
применения>>,\\
     директор ИПИ РАН, академик  \hfill
     \textit{И.\,А.~Соколов}\\
     
     \noindent
     Редактор-составитель тематического выпуска,\\
     профессор кафедры математической статистики факультета\\
      вычислительной математики и кибернетики МГУ им.\ М.\,В.~Ломоносова,\\
     ведущий научный сотрудник ИПИ РАН,\\ 
доктор физико-математических наук \hfill
      \textit{В.\,Ю.~Королев}
     
     } }
     }

\def\stat{pechinkin}


\def\tit{СОВМЕСТНОЕ СТАЦИОНАРНОЕ РАСПРЕДЕЛЕНИЕ
ЧИСЛА ЗАЯВОК В~НАКОПИТЕЛЕ И~В~БУНКЕРЕ
ПЕРЕУПОРЯДОЧЕНИЯ В~МНОГОКАНАЛЬНОЙ СИСТЕМЕ
ОБСЛУЖИВАНИЯ С~ПЕРЕУПОРЯДОЧЕНИЕМ
ЗАЯВОК$^*$}


\def\titkol{Совместное стационарное распределение
числа заявок в~накопителе и~в~бункере
переупорядочения} % в~многоканальной системе обслуживания с~переупорядочением заявок}

\def\aut{\fbox{А.\,В.\~Печинкин}$^1$, Р.\,В.~Разумчик$^2$}

\def\autkol{А.\,В.\~Печинкин, Р.\,В.~Разумчик}

\titel{\tit}{\aut}{\autkol}{\titkol}

{\renewcommand{\thefootnote}{\fnsymbol{footnote}} \footnotetext[1]
{Работа выполнена при частичной поддержке РФФИ (проект 13-07-00223).}}


\renewcommand{\thefootnote}{\arabic{footnote}}
\footnotetext[1]{Институт проблем информатики Российской академии наук}
\footnotetext[2]{Институт проблем информатики Российской академии наук; Российский
университет дружбы народов, rrazumchik@ieee.org}

%\vspace*{3pt}

\Abst{Рассматривается функционирующая в~непрерывном времени
многоканальная система обслуживания с~накопителем
бесконечной емкости и переупорядочением заявок.
В~систему поступает пуассоновский поток заявок, время
обслуживания каждым прибором распределено по
экспоненциальному закону с~одним и~тем же параметром.
При поступлении в~систему всем заявкам  присваивается
порядковый номер. На выходе из системы сохраняется
порядок между заявками, установленный при входе в~нее.
Заявки, завершившие обслуживание и~нарушившие установленный порядок,
накапливаются на выходе системы
в~бункере переупорядочения (БП), который также имеет неограниченную емкость.
Найдено совместное стационарное распределение
числа заявок в~накопителе и~суммарного числа
заявок в~БП в~терминах
вычислительных алгоритмов и~производящих функций (ПФ).
Приведены примеры расчетов по полученным
соотношениям.}

\KW{многолинейная система массового обслуживания;
переупорядочение; стационарное распределение
числа заявок}

\DOI{10.14357/19922264140401}


%\vspace*{3pt}

\vskip 12pt plus 9pt minus 6pt

\thispagestyle{headings}

\begin{multicols}{2}

\label{st\stat}


\section{Введение}

Для функционирования ряда
ин\-фор\-ма\-ци\-он\-но-те\-ле\-ком\-му\-ни\-ка\-ци\-он\-ных сис\-тем
и для предоставления на их основе услуг
необходимо соблюдение\linebreak требования сохранения порядка в~потоке передаваемых сообщений.
Различные действия, необходимые для этого, можно объединить
в~одно понятие~--- переупорядочение.
Для изучения влияния\linebreak \mbox{переупорядочения} на качество
функционирования ин\-фор\-ма\-ци\-он\-но-те\-ле\-ком\-му\-ни\-ка\-ци\-он\-ных
сис\-тем к~настоящему времени предложено множество
моделей, которые в~своей основе используют методы
и~модели теории массового обслуживания.
Исследуемая сис\-те\-ма обычно представляется в~виде
системы или сети массового обслуживания с одним\linebreak или
несколькими входящими потоками сообщений.
Эффект переупорядочения часто моделируется с~помощью
дополнительной очереди (БП),
в~которую попадают сообщения, обработанные\linebreak в~системе,
и~ожидают там до тех пор, пока порядок следования сообщений
нельзя будет восстановить.
Некоторый обзор работ в~этом направлении можно найти
в~\cite{a1, a2},
а~некоторые последние результаты~--- в~[3--8].

Настоящая работа является развитием \cite{a8}, в~которой
рассматривается система массового обслуживания (СМО)
с~переупорядочением в~виде марковской многоканальной
системы обслуживания неограниченной емкости и~бункером
переупорядочения, также имеющим неограниченную
емкость.
В~\cite{a8} была получена система уравнений равновесия для
совместного стационарного распределения чис\-ла заявок в~системе
и~бункере переупорядочения и~приведены некоторые результаты
численных расчетов.
Однако несомненный интерес представляют
две задачи, не освещенные в~\cite{a8}, которые и~являются
предметом данной статьи, а~именно:
разработка рекуррентного алгоритма расчета вышеупомянутого
совместного стационарного распределения и~нахождение
этого распределения в~терминах ПФ.

Статья организована таким образом.
В~разд.~2 приводится подробное описание
системы.
В~разд.~3 дается рекуррентный алгоритм расчета
совместного стационарного распределения, а~в~разд.~4
показано, как совместное стационарное распределение
можно найти в~терминах ПФ.
Примеры расчетов, проведенных по формулам разд.~4,
представлены в~разд.~5.
В~заключении сформулированы основные результаты работы.

\section{Описание системы}

Рассмотрим функционирующую в~непрерывном времени
$N$-ли\-ней\-ную ($N\hm\ge 2$) СМО с накопителем
неограниченной емкости, входящим пуассоновским
потоком заявок интенсивности~$\lambda$ \mbox{и~экспоненциальным}
распределением времени
обслуживания заявки каждым прибором с~па\-ра\-мет\-ром~$\mu$.


При поступлении в~систему всем заявкам  присваивается
порядковый номер.
На выходе из СМО сохраняется порядок между заявками,
установленный при входе в~нее.
Заявки, завершившие обслуживание и~нарушившие
установленный порядок, накапливаются на выходе
системы в~БП и~покидают СМО только
после того, как закончится обслуживание всех заявок с~меньшими номерами.
Такая СМО носит название системы с переупорядочением
заявок.

Предполагается также выполнение необходимого и~достаточного условия
существования стационарного режима функционирования СМО
$$\tilde {\rho}\hm=\fr{\rho}{N}<1\,,
$$
 где $\rho\hm=\lambda/\mu$.

\vspace*{-9pt}

\section{Алгоритм нахождения совместного стационарного распределения}

Предположим, что на приборах находится $n$, $n\hm=\overline{1,N}$, заявок.
Тогда заявкой первого уровня будем называть ту из них,
которая в~систему поступила последней, второго уровня~--- предпоследней,
$\ldots,$ $n$-го уровня~--- первой. При этом если $n\hm=N$ (все приборы
заняты), то находящиеся в~БП заявки, поступившие между заявками
второго и~первого уровней, будем называть заявками первой очереди,
заявки, поступившие между заявками третьего и~второго уровней,~---
заявками второй очереди, $\ldots,$ заявки, поступившие между
заявками $N$-го и~$(N-1)$-го уровней,~--- заявками $(N-1)$-й
очереди. Если же $n<N$, то  заявками первой очереди будем называть
заявки из БП, поступившие после заявки первого уровня, заявками
второй очереди~--- заявки, поступившие между заявками второго и~первого уровней,
и~т.\,д.

При $n\ge N$ обозначим через
$p^{(m)}_{n;i}$, ${m\hm=\overline{1,N-1}}$, ${i\hm\ge 0}$,
стационарную вероятность того, что в~системе на
приборах и~в накопителе находится~$n$~заявок,
а~в~БП имеется в~сумме~$i$~заявок первой,
второй, $\ldots,$ $m$-й очереди.
Через
$p^{(m)}_{n;i}$, ${m\hm=\overline{1,n}}$, ${i\hm\ge 0}$,
обозначим аналогичную стационарную вероятность
при $n\hm=\overline{1,N-1}$.
Через~$p_n$, $n\hm\ge 0$, обозначим
стационарную вероятность того, что в~системе на
приборах и~в накопителе (без учета числа заявок в~БП) находится~$n$~заявок.
Очевидно, что стационарные вероятности~$p_n$
определяются теми же самыми формулами, что и~в~обычной
марковской СМО $M/M/N/\infty$
(см., например,~\cite{boch}):
\begin{align}
p_{0} &= \left( \sum\limits_{i=0}^{N-1} \fr{\rho^i}{i!} +
\fr{\rho^N}{(N-1)! (N-\rho)}
\right)^{-1} \,;\label{3-1}
\\
p_{i} &= \begin{cases}
\fr{\rho^i }{i!} p_{0}\,, &\ i=\overline{1,N}\,,
\\
%\label{3-3}
\fr{\rho^i}{N!\, N^{i-N}} p_{0}
= \tilde \rho^{i-N} p_{N}\,, &\ i\ge N+1\,.
\end{cases}
\label{3-2}
\end{align}

Наконец, через $p_{n;i}$, ${n\hm\ge 1}$, ${i\hm\ge 0}$, обозначим
стационарную вероятность того, что в~системе на
приборах и~в накопителе находится~$n$~заявок,
а~в~БП~--- $i$~заявок.

Используя принцип глобального баланса, можно выписать систему уравнений для
вероятностей~$p^{(m)}_{n;i}$.
Для вероятностей $p^{(1)}_{n;i}$, $n\hm\ge N$,
$i \hm\ge 0$, справедливы уравнения:
\begin{align}
\hspace*{-2.8mm}p^{(1)}_{n;0} (\lambda+N\mu) &= p^{(1)}_{n-1;0} \lambda +
p_{n+1} (N-1) \mu \,,\ n\ge N;
\!\!\label{eq-1-1}
\\
\hspace*{-2.8mm}p^{(1)}_{n;i} (\lambda+N\mu) &= p^{(1)}_{n-1;i} \lambda +
p^{(1)}_{n+1;i-1} \mu \,,\notag\\
&\hspace*{25mm} n\ge N\,,\enskip i \ge 1\,.
\label{eq-1-2}\!\!
\end{align}
%%%%%%%%%%%%%%%%%%%%%%%
%%%%%%%%%%%%%%%%%%%%%%%
Для вероятностей $p^{(1)}_{N-1;i}$,\ \ $i \ge 0$,
справедливы уравнения:
%%%%%%%%%%%%%%%%%%%
\begin{align}
\label{eq-1-3}
p^{(1)}_{N-1;0} [\lambda+(N-1)\mu] &=
p_{N-2} \lambda + p_{N} (N-1)\mu\,;
\\
\label{eq-1-4}
p^{(1)}_{N-1;i} [\lambda+(N-1)\mu] &= p^{(1)}_{N;i-1} \mu\,,\enskip i \ge 1\,.
\end{align}
Для вероятностей
$p^{(1)}_{n;i}$, $n\hm=\overline{1,N-2}$, $i \hm\ge 0$,
справедливы уравнения
\begin{align}
\label{eq-1-5}
\hspace*{-2mm}p^{(1)}_{n;0} (\lambda+n\mu) &= p_{n-1} \lambda +
p^{(1)}_{n+1;0} n\mu ,\  n=\overline{1,N-2};
\\
\label{eq-1-6}
\hspace*{-2mm}p^{(1)}_{n;i} (\lambda+n\mu) &= p^{(1)}_{n+1;i} n\mu
+ p^{(2)}_{n+1;i-1} \mu \,,\notag\\
&\hspace*{15mm}n=\overline{1,N-2},\ \ i \ge 1.
\end{align}


Для остальных вероятностей
$p^{(m)}_{n;i}$, $m\hm=\overline{2,N-1}$, справедливы формулы:
\begin{align}
p^{(m)}_{n;0} (\lambda+N\mu) &= p^{(m)}_{n-1;0} \lambda +
p^{(m-1)}_{n+1;0} (N-m) \mu\,,\notag\\
& \hspace*{30mm}n\ge N\,; \label{bat-1}
\\
p^{(m)}_{n;i} (\lambda+N\mu) &= p^{(m)}_{n-1;i} \lambda +
p^{(m-1)}_{n+1;i} (N-m) \mu +{}\notag\\
&\hspace*{-10mm}{}+p^{(m)}_{n+1;i-1} m \mu \,,\enskip
n\ge N\,,\ \ i\ge 1\,;
\label{bat-2}
\end{align}

\noindent
\begin{align}
p^{(m)}_{N-1;0} [\lambda+(N-1)\mu] &={}\notag\\
{}=p^{(m-1)}_{N-2;0} \lambda
&{}=+ p^{(m-1)}_{N;0} (N-m) \mu \,;
\label{bat-3}
\end{align}

\noindent
\begin{multline}
p^{(m)}_{N-1;i} [\lambda+(N-1)\mu] =p^{(m-1)}_{N-2;i} \lambda+{}\\
{}+
p^{(m-1)}_{N;i} (N-m) \mu +p^{(m)}_{N;i-1} m \mu\,,\enskip i\ge 1\,;
\label{bat-4}
\end{multline}

\vspace*{-12pt}



\noindent
\begin{multline}
\label{bat-5}
p^{(m)}_{n;0} (\lambda+n\mu) = p^{(m-1)}_{n-1;0} \lambda+
p^{(m)}_{n+1;0} (n-m+1) \mu \,,\\
 n=\overline{m,N-2}\,;
\end{multline}

\noindent
\begin{multline}
\label{bat-6}
p^{(m)}_{n;i} (\lambda+n\mu) = p^{(m-1)}_{n-1;i} \lambda
+
p^{(m)}_{n+1;i} (n-m+1) \mu +{}\\
{}+ p^{(m+1)}_{n+1;i-1} m \mu\,,\enskip
 n=\overline{m,N-2}\,,\ \ i\ge 1\,.
\end{multline}

Решение данной системы уравнений позволяет
найти совместное стационарное распределение
$p_{n;i}$ числа заявок на приборах и~в
накопителе и~суммарного числа заявок в~БП в~виде следующих ра\-венств:
\begin{alignat*}{2}
%\label{bat-7}
p_{n;i} &= p^{(N-1)}_{n;i}\,, &\quad  n&\ge N\,,\ \ i\ge 0\,,
\\
%\label{bat-8}
p_{n;i} &= p^{(n)}_{n;i} \,, &\quad n&=\overline{1,N-1}\,,\ \ i\ge 0\,.
\end{alignat*}

Анализ системы~\eqref{eq-1-1}--\eqref{bat-6}
показал, что вычисление стационарных
вероятностей $p^{(m)}_{n;i}$ можно проводить
рекуррентным образом по следующему алгоритму.

\bigskip

\noindent
А\,л\,г\,о\,р\,и\,т\,м~1\ (\textbf{Алгоритм решения системы уравнений равновесия}).

\noindent
\textit{Задать} $\lambda$, $\mu$ и $n$.

\noindent
\textit{Для $n\ge 0$ рассчитать $p_{n}$ по
формулам}~\eqref{3-1} и~\eqref{3-2}.

\noindent
\textit{Рассчитать $p^{(1)}_{N-1;0}$ по формуле}~\eqref{eq-1-3}.

\noindent
\textit{Для $n\ge N$ рассчитать $p^{(1)}_{n;0}$ по
формуле}~\eqref{eq-1-1}.

\noindent
\textit{Для $i\ge1$}


\textit{рассчитать $p^{(1)}_{N-1;i}$ по формуле}~\eqref{eq-1-4}.

\textit{для $n\ge N$ рассчитать $p^{(1)}_{n;i}$ по формуле}~\eqref{eq-1-2}.

\noindent
\textit{Для $n=\overline{N-2,1}$ рассчитать $p^{(1)}_{n;0}$
по формуле}~\eqref{eq-1-5}.

\noindent
\textit{Для $m=\overline{2,N-1}$}

\textit{рассчитать $p^{(m)}_{N-1;0}$ по формуле}~\eqref{bat-3}.


\textit{для $n\ge N$ рассчитать $p^{(m)}_{n;0}$
   по формуле}~\eqref{bat-1};

\textit{для} $i\hm\ge1$

    \hspace*{9pt}\textit{рассчитать $p^{(1)}_{N-m;i}$ по
    формуле}~\eqref{eq-1-6};


    \hspace*{9pt}\textit{если $m \ne 2$, для}  $j\hm=\overline{2,m-1}$ \textit{рассчитать}\linebreak\vspace*{-12pt}

 \hspace*{9pt}\textit{$p^{(j)}_{N-m+j-1;i}$ по формуле}~\eqref{bat-6};

\hspace*{9pt}\textit{рассчитать $p^{(m)}_{N-1;i}$ по формуле}~\eqref{bat-4};

\hspace*{9pt}\textit{для $n\ge N$ рассчитать $p^{(m)}_{n;i}$
    по формуле}~\eqref{bat-2};

\textit{если {$m \ne N-1$}, для $m\hm=\overline{N-2,m}$
   рассчитать}\linebreak

   \textit{$p^{(m)}_{n;0}$ по формуле}~\eqref{bat-5}.

\bigskip

В~связи с~тем, что вычисление моментов после расчета
вероятностей по представленному алгоритму
может давать погрешности, в~следующем разделе
находятся формулы для совместного стационарного
распределения в~терминах ПФ.


\section{Использование производящих функций}

Система уравнений~\eqref{eq-1-1}--\eqref{bat-6}
допускает также решение с~помощью ПФ.
Для нахождения этого решения положим
\begin{equation*}
\label{f-m}
f_m(u,z) = \lambda u^2 - (\lambda + N\mu) u + m \mu z\,,\
 m=\overline{1,N-1}\,.
\end{equation*}

Обозначим через $u_m\hm=u_m(z)$, $m\hm=\overline{1,N-1}$,
минимальное решение уравнения
$$
f_m(u,z) = 0\,,
$$
т.\,е.
\begin{equation*}
%\label{sqrt}
u_m = \fr{\lambda + N\mu - \sqrt{(\lambda + N\mu)^2 - 4 m \lambda \mu z}}
{2 \lambda }\,.
\end{equation*}


Введем ПФ
\begin{multline*}
P^{(m)}_{n}(z) = \sum\limits_{i=0}^{\infty}
z^{i} p^{(m)}_{n;i}\,, \\
0<z<1\,, \ \ n\ge1\,,\ \
m=\overline{1,\min(n,N-1)} \,;
\end{multline*}

\vspace*{-12pt}


\noindent
\begin{multline*}
P^{(m)}(u,z) = \sum\limits_{n=N}^{\infty} u^{n-N} P^{(m)}_{n}(z)\,, \\
0<u,z<1\,, \ \ m=\overline{1,N-1}\,,
\end{multline*}
и, кроме того, положим
$$
P(u) = \sum\limits_{n=N}^{\infty} u^{n-N} p_{n}
= \fr{1}{1 - \tilde{\rho} u}\, p_N \,.
$$

Тогда, умножая~\eqref{eq-1-1} и~\eqref{eq-1-2}
на~$z^i$ и~суммируя по всем~$i$ от нуля до
бесконечности, получаем:
\begin{multline*}
%\label{eq-z-1}
(\lambda+N\mu) P^{(1)}_{n}(z) =
\lambda P^{(1)}_{n-1}(z) +
(N-1) \mu p_{n+1}
+ {}\\
{}+\mu z P^{(1)}_{n+1}(z)\,,\enskip n\ge N\,.
\end{multline*}
Умножая последнее выражение на $u^{n-N}$ и~суммируя по всем значениям $n\hm\ge N$,
после приведения подобных слагаемых имеем:
\begin{multline}
\label{eq-z-2}
f_1(u,z) P^{(1)}(u,z) =
\mu z P^{(1)}_{N}(z) -{}\\
{}- \lambda u P^{(1)}_{N-1}(z) -
(N-1) \mu [P(u) - p_{N}] \,.
\end{multline}


Теперь умножим \eqref{bat-1} и~\eqref{bat-2}
на~$z^i$ и~просуммируем по всем значениям $i\hm\ge0$.
В~результате приходим к~выражению:
\begin{multline*}
%\label{bat-2*}
(\lambda+N\mu) P^{(m)}_{n}(z) = \lambda P^{(m)}_{n-1}(z)
+{}\\
{}+(N-m) \mu P^{(m-1)}_{n+1}(z) +
m \mu z P^{(m)}_{n+1}(z) \,,\enskip n\ge N\,.
\end{multline*}
Умножая последнее выражение на $u^{n-N}$, после
суммирования по всем $n\hm\ge N$ получаем:

\pagebreak

\noindent
\begin{multline}
\label{bat-2*}
f_m(u,z) P^{(m)}(u,z) = m \mu z P^{(m)}_{N}(z)
- \lambda u P^{(m)}_{N-1}(z) -{}\\
{}-
(N-m) \mu [P^{(m-1)}(u,z) - P^{(m-1)}_{N}(z)]\,,\\ m=\overline{2,N-1}\,.
\end{multline}

Из уравнений~\eqref{eq-1-3} и~\eqref{eq-1-4}
после умножения на~$z^i$ и~суммирования по
всем значениям $i \hm\ge 0$ находим:
\begin{multline}
\label{eq-z-3}
P^{(1)}_{N-1}(z)=\fr{\lambda p_{N-2} + (N-1)\mu p_{N}}
{\lambda+(N-1)\mu }+{}\\
{}+ \fr{\mu z}{\lambda+(N-1)\mu} \,P^{(1)}_N(z)\,.
\end{multline}

Действуя аналогичным образом
с~уравнениями~\eqref{bat-3} и~\eqref{bat-4}, как и~с~уравнениями~\eqref{eq-1-3}
и~\eqref{eq-1-4}, приходим к выражению:
\begin{multline}
\label{bat-4*}
P^{(m)}_{N-1}(z) = \fr{ \lambda P^{(m-1)}_{N-2}(z) + (N-m) \mu P^{(m-1)}_{N}(z)
}{\lambda+(N-1)\mu }+{}\\
{}+\fr{m \mu z}{\lambda+(N-1)\mu}\,P^{(m)}_{N}(z) \,,\enskip m=\overline{2,N-1}\,.
\end{multline}


Домножая уравнения~\eqref{eq-1-5} и~\eqref{eq-1-6}
на~$z^i$, после суммирования по всем
значениям $i \hm\ge 0$ имеем:
\begin{multline}
\label{eq-z-4}
P^{(1)}_{n}(z)= \fr{ \lambda p_{n-1} + n \mu P^{(1)}_{n+1}(z) }{
\lambda+n\mu }+ \fr{\mu z}{\lambda+n\mu}\,P^{(2)}_{n+1}(z) \,,\\
n=\overline{1,N-2}\,.
\end{multline}

Наконец, производя аналогичные преобразования
с~уравнениями~\eqref{bat-5} и~\eqref{bat-6}, получаем:
\begin{multline}
\label{bat-6*}
P^{(m)}_n(z)= \fr {\lambda P^{(m-1)}_{n-1}(z) +
(n-m+1) \mu P^{(m)}_{n+1}(z)} {\lambda+n\mu}
+{}
\\
{}+
\fr{m \mu z}{\lambda+n\mu} P^{(m+1)}_{n+1}(z)\,,\enskip
m=\overline{2,N-2}\,,\\
n=\overline{m,N-2}\,.
\end{multline}

Уравнения~\eqref{eq-z-2}--\eqref{bat-6*} позволяют
находить выражения для всех
ПФ $P^{(m)}_{n}(z)$, $m\hm=\overline{1,N-1}$,
$n\hm=\overline{1,N-1}$, а~так\-же совместное
стационарное распределение рекуррентным образом.
Подставляя выражение для $P^{(1)}_{N-1}(z)$ из
формулы~\eqref{eq-z-3} в~формулу~\eqref{eq-z-2}, получаем:
\begin{multline}
P^{(1)}(u,z) = \left(
\left[
\mu z - \fr{\lambda \mu z u}{\lambda+(N-1)\mu}
\right] P^{(1)}_N(z) -{}\right.\\
{}-
\left[
\lambda u \fr{\lambda p_{N-2} + (N-1)\mu p_{N}}{\lambda+(N-1)\mu}+{}\right.\\
\left.\left.{}+
 (N-1) \mu [P(u) - p_{N}]
\vphantom{\fr{\lambda p_{N-2} + (N-1)\mu p_{N}}{\lambda+(N-1)\mu}}\right]
\right)
\Bigg /
f_1(u,z)\,,
\label{m25}
\end{multline}
откуда из равенства нулю в~точке $u_1(z)$ числителя и~знаменателя
правой части формулы~\eqref{m25} следует:
\columnbreak


%%%%%%%%%%%%%%%%%%%%%%%%%%%
\noindent
\begin{multline*}
%\label{r1}
P^{(1)}_N(z)= \left(
\lambda u_1(z) [\lambda p_{N-2} + (N-1)\mu p_{N}]
+{}\right.\\
{}+
\left.(\lambda+(N-1)\mu)(N-1) \mu \left[P(u_1(z)) - p_{N}\right]\right)\!\!\Big/\!\!
\left(\mu z \left[\lambda+{}\right.\right.\\
\left.\left.{}+(N-1)\mu  - \lambda u_1(z)\right]\right)\,.
\end{multline*}
%%%%%%%%%%%%%%%%%%%%%%%%%%%%%%%%%%%%%%%%%%
%%%%%%%%%%%%%%%%%%%%%%%%%%%%%%%%%%%%%%%%
Теперь, возвращаясь к~формуле~\eqref{eq-z-3},
получаем выражение для $P^{(1)}_{N-1}(z)$:
\begin{multline*}
%\label{r2}
P^{(1)}_{N-1}(z)=
\left([\lambda p_{N-2} + (N-1)\mu p_{N}]+{}\right.\\
\left.{}
+ (N-1) \mu [P(u_1(z)) - p_{N}]\right)\Big /
\left(\lambda+(N-1)\mu  - {}\right.\\
\left.{}-\lambda u_1(z)\right)\,.
\end{multline*}

Далее из равенства~\eqref{eq-z-4} выражаем $P^{(1)}_{N-2}(z)$ через
$P^{(2)}_{N-1}(z)$. Из равенства~\eqref{bat-4*} выражаем
$P^{(2)}_{N-1}(z)$ через $P^{(2)}_{N}(z)$. Подставляя полученное
выражение для $P^{(2)}_{N-1}(z)$ в~формулу~\eqref{bat-2*}, из
равенства нулю в~точке~$u_2$ левой и~правой части получившегося
равенства находим $P^{(2)}(u,z)$. Затем из равенства~\eqref{eq-z-4}
выражаем $P^{(1)}_{N-3}(z)$ через $P^{(2)}_{N-2}(z)$ и~т.\,д.

Продолжая эту процедуру, можно найти
соотношения для вычисления всех
ПФ $P^{(m)}_{n}(z)$, $m\hm=\overline{1,N-1}$, $n\hm=\overline{1,N-1}$.

С каждым шагом выражение для очередной ПФ становится все сложнее,
и~в итоге при большом числе приборов выписать явный вид всех ПФ не
удается. Тем не менее нахождение значений ПФ в~каждой точке $z \hm\ne
0$ можно свести к последовательному решению систем линейных
уравнений. Для этого обозначим через $A_n(z)$, $n\hm =\overline{2,N-1}$,
мат\-ри\-цы размера $(n+1)\times (n+1)$, име\-ющие
следующую структуру:
\begin{gather*}
\setcounter{MaxMatrixCols}{3}
A_2(z)=
\begin{pmatrix}
 2 \mu z   & 0  & -2 \mu z         \\
 - \lambda u_2(z) & - \mu z   &  \lambda +(N-1) \mu       \\
0  & \lambda +(N-2)\mu &    - \lambda
\end{pmatrix}\,;
\end{gather*}

\vspace*{-12pt}

\noindent
{ %\scriptsize
\begin{multline*}
\setcounter{MaxMatrixCols}{7}
A_n(z)=\left(
\begin{matrix}
 n \mu z   & 0  & - n \mu z &      \!\cdots\!          \\
 - \lambda u_n(z) \! & 0  & \! \lambda +(N-1) \mu \! & \!\cdots\!  \\
  \vdots   & \vdots & \vdots &  \!\cdots\! \\
 0   & 0 & 0&  \!\cdots\!  \\
 0      & 0 & 0    &     \!\cdots\! \\
 0   & - \mu z  &0   &  \cdots\! \\
0 & \!\lambda +(N-n)\mu \!&0  &  \!\cdots\!
\end{matrix}\right.\\
\left.\begin{matrix}
    \cdots\!     & 0    & 0       \\
    \cdots\!  & 0 & 0 \\
    \cdots\! & \vdots     & \vdots  \\
    \cdots\!  & - 3 \mu z     & 0   \\
    \cdots\! & \! \lambda +(N-n+2) &-2\mu z\\
    \cdots\! & - \lambda  & \! \lambda+(N-n+1)\mu\\
    \cdots\!  & 0  & - \lambda
\end{matrix}\right)\,,
\\ n =\overline{3,N-1}\,.
\end{multline*}
}

\noindent
Определим вектор-стр$\acute{\mbox{о}}$\-ки $\vec{a}_n(z)$ и~$\vec{b}_n(z)$
длины $(n+1)$ следующим образом:
\begin{multline*}
\vec{a}_n(z) = \left (
P^{(n)}_{N}(z), P^{(n)}_{N-1}(z), \dots\right.\\
\left.\dots,  P^{(2)}_{N-n+1}(z), P^{(1)}_{N-n}(z)
\right )\,,\enskip
n =\overline{2,N-1}\,;
\end{multline*}

\vspace*{-12pt}
\noindent
\begin{multline*}
\vec{b}_2(z) = \left (
(N-2) \mu [P^{(1)}(u_2,z) - P^{(1)}_{N}(z)] ,
\lambda p_{N-3}+{}\right.\\
\left.{}+ (N-2) \mu P^{(1)}_{N-1}(z),
(N-2) \mu P^{(1)}_{N}(z) \right)\,;
\end{multline*}

\vspace*{-12pt}

\noindent
\begin{multline*}
\vec{b}_n(z) = \left (
(N-n) \mu
[P^{(n-1)}(u_n,z) - P^{(n-1)}_{N}(z)],\right.
\\
\lambda p_{N-1-n}+ (N-n) \mu P^{(1)}_{N-1-(n-2)}(z),\\
(N-n) \mu P^{(n-1)}_{N}(z), (N-n)\mu  P^{(n-1)}_{N-1}(z),
\dots ,
\\
\left.
(N-n)\mu  P^{(3)}_{N-n+3}(z), (N-n)\mu  P^{(2)}_{N-n+2}(z)
\right )\,,\\
n =\overline{3,N-1}\,.
\end{multline*}
Тогда алгоритм нахождения ПФ состоит в~последовательном начиная с~$n\hm=2$ решении
системы линейных уравнений
$$
\vec{a}_n(z) A_n(z) = \vec{b}_n(z) \,.
$$
Из структуры матрицы $A_n(z)$, $n \hm=\overline{3,N-1}$, видно, что
она неприводима и~обладает свойством диагонального преобладания
т.\,е.\ перестановкой строк и~столбцов можно добиться того,
что в~каждой строке модуль диагонального элемента будет либо строго
больше, либо не меньше суммы модулей всех остальных элементов в~строке.
Покажем это. Если определить матрицы перестановки~$P^L_n$ и~$P^R_n$
размера $(n+1)\times (n+1)$ при $n \hm=\overline{3,N-1}$
следующим образом:
\begin{gather*}
\setcounter{MaxMatrixCols}{5}
P^L_n=
\begin{pmatrix}
 0   & 0  & \cdots & 0& 1 \\
 1   & 0  & \cdots & 0& 0 \\
  \vdots   &  \vdots  & \cdots &  \vdots &  \vdots \\
  0   & 0  & \cdots & 0& 0 \\
   0   & 0  & \cdots & 1& 0
\end{pmatrix}\,;
\enskip
\setcounter{MaxMatrixCols}{5}
P^R_n=
\begin{pmatrix}
 0   & 1  & \cdots & 0& 0         \\
 1   & 0  & \cdots & 0& 0 \\
   \vdots   &  \vdots  & \cdots &  \vdots &  \vdots \\
  0   & 0  & \cdots & 1& 0 \\
   0   & 0  & \cdots & 0& 1
\end{pmatrix}\,,
\end{gather*}
то матрица $P^L_n A_n(z)P^R_n$, $n \hm=\overline{3,N-1}$,
примет вид:
\begin{multline*}
\setcounter{MaxMatrixCols}{7}
P^L_n A_n(z)P^R_n={}\\
{}=\left(
\begin{matrix}
 \lambda +(N-n)\mu &0 & 0  & \cdots\\
 0  &  n \mu z   & - n \mu z &  \cdots       \\
  0  & - \lambda u_n(z) & \lambda +(N-1) \mu  & \cdots  \\
  \vdots   & \vdots & \vdots & \cdots   \\
 0   & 0 & 0& \cdots \\
 0      & 0 & 0    & \cdots  \\
 - \mu z  & 0   &0   & \cdots
\end{matrix}\right.
\end{multline*}

\noindent
\begin{equation*}
\hspace*{15mm}\left.\begin{matrix}
\cdots  & 0  & - \lambda\\
\cdots        & 0    & 0       \\
\cdots    & 0    & 0      \\
\cdots   & \vdots     & \vdots       \\
\cdots  & - 3 \mu z     & 0       \\
\cdots   & \lambda +(N-n+2) \mu & - 2 \mu z       \\
\cdots      & - \lambda  & \lambda +(N-n+1)\mu
\end{matrix}\right).
\end{equation*}
Легко видеть, что в~каждой строке модуль диагонального
элемента либо строго больше, либо не меньше суммы
модулей всех остальных элементов в~строке.
Тогда, как вытекает из следствия~6.2.27 в~\cite{horn},
у~матрицы $A_n(z)$ существует обратная
и,~значит, система $\vec{a}_n(z) A_n(z) \hm= \vec{b}_n(z)$
при $z\hm\neq 0$ имеет единственное решение.

\vspace*{-4pt}

\section{Примеры расчетов}

На основе полученных в~разд.~4 результатов {были} проведены расчеты
среднего и~дисперсии чис\-ла заявок в~БП,
а~также коэффициента корреляции числа заявок в~накопителе и~числа
заявок в~БП для различного чис\-ла
приборов~$N$~и~значений загрузки системы $\rho/N$. \mbox{Напомним}, что аналогичные
показатели были рассчитаны в~\cite{a8} по определению, на основе
стационарных вероятностей, рассчитанных по приведенному выше
алгоритму. Далее можно видеть, что результаты, полученные с~по\-мощью
ПФ, как и~ожидалось, полностью совпадают с~результатами,
представленными в~\cite{a8}.

На рис.~1 отражено поведение значения среднего числа заявок
в~БП в~зависимости от загрузки системы $\rho/N$.
Отметим, что полученные в~предыдущих  разделах результаты позволяют
рассчитывать такие
 характеристики, как среднее число заявок только
в~первой очереди в~БП, в~сумме в~первой и~во второй очередях в~БП
(когда обе очереди существуют), в~сумме в~первой, второй,\ldots ,
$(N-1)$-й очере-\linebreak\vspace*{-12pt}

\vspace*{6pt}

\begin{center}  %fig1
\vspace*{2pt}
\mbox{%
 \epsfxsize=75.145mm
 \epsfbox{pec-1.eps}
 }
\end{center}

\noindent
{{\figurename~1}\ \ \small{Поведение
 среднего числа заявок в~БП в~зависимости от загрузки
системы  $\rho/N$: \textit{1}~--- $N\hm=4$; \textit{2}~--- 7;
\textit{3}~--- $N=9$}}

%\vspace*{9pt}


\addtocounter{figure}{1}


\begin{center}  %fig2
\vspace*{2pt}
 \mbox{%
 \epsfxsize=75.027mm
 \epsfbox{pec-2.eps}
 }
 \end{center}

\noindent
{{\figurename~2}\ \ \small{Поведение среднего числа заявок в~первой
очереди в~БП~(\textit{1}), в~сумме в~первой и~во второй очередях в~БП~(\textit{2}),
в~сумме в~первой, второй и~третьей очередях в~БП~(\textit{3})
в~зависимости от загрузки системы $\rho/N$. Число
приборов $N\hm=4$}}

\vspace*{18pt}


\begin{center}  %fig3
\vspace*{2pt}
 \mbox{%
 \epsfxsize=74.929mm
 \epsfbox{pec-3.eps}
 }
 \end{center}

\noindent
{{\figurename~3}\ \ \small{Поведение
 дисперсии числа заявок в~БП в~зависимости от загрузки
системы  $\rho/N$: \textit{1}~--- $N\hm=4$; \textit{2}~--- 7; \textit{3}~--- $N=9$}}

\vspace*{18pt}

\begin{center}  %fig4
\vspace*{2pt}
 \mbox{%
 \epsfxsize=75.192mm
 \epsfbox{pec-4.eps}
 }
 \end{center}

\noindent
{{\figurename~4}\ \ \small{Поведение
 коэффициента корреляции числа заявок в~накопителе и~числа
заявок в~БП в~зависимости от загрузки системы  $\rho/N$:
\textit{1}~--- $N\hm=4$; \textit{2}~--- 7; \textit{3}~--- $N=9$}}


%\vspace*{9pt}


\noindent
дях в~БП (когда каждая из очередей существует).
Поведение данных характеристик в~зависимости от загрузки системы
$\rho/N$ для случая $N\hm=4$ пред\-став\-ле\-но на рис.~2.

На рис.~3 и~4 изображено поведение дисперсии числа
заявок в~БП и~поведение
коэффициента корреляции числа заявок в~накопителе и~числа
заявок в~БП соответственно.

Во всех расчетах интенсивность обслуживания заявок~$\mu$ принималась
равной~1.

%\addtocounter{figure}{1}
%%%%%%%%%%%%%%%%%%%%%%%%%%%%%%%%%%%%%%%%%%%%%%%%%%%%%

Анализируя графики на рис.~1--4, стоит отметить два момента. Среднее
число заявок в~БП не уходит в~бесконечность с ростом загрузки
(и~даже при загрузке больше единицы), что следует из формулы Литтла.
Число заявок в~накопителе и~число заявок в~БП весьма слабо
коррелированы, и~с~рос\-том числа приборов коэффициент корреляции
уменьшается.

\section{Заключение}

В настоящей работе рассмотрена функционирующая в~непрерывном времени
многоканальная система обслуживания с~накопителем бесконечной емкости
и~переупорядочением заявок.
В~систему поступает пуассоновский поток заявок, время
обслуживания каждым прибором распределено по
экспоненциальному закону с~одним и~тем же параметром.
Для нахождения совместного стационарного распределения
числа заявок в~накопителе и~суммарного числа
заявок в~БП получен рекуррентный алгоритм.
Также показано, как можно находить совместное распределение
в~терминах ПФ, которые облегчают расчет его моментов.

{\small\frenchspacing
 {%\baselineskip=10.8pt
 \addcontentsline{toc}{section}{References}
 \begin{thebibliography}{99}
 \bibitem{a1} %1
\Au{Boxma O., Koole G., Liu~Z.}
Queueing-theoretic solution methods for
models of parallel and distributed systems~//
Performance Evaluation of Parallel and Distributed Systems Solution
Methods, 1994. CWI Tract~105 and~106. P.~1--24.

\bibitem{a2} %2
\Au{Dimitrov B.}
Queues with resequencing. A~survey and recent results~//
{2nd World Congress on Nonlinear Analysis,
Theory, Methods, Applications Proceedings}, 1997. Vol.~30. No.\,8. P.~5447--5456.

\bibitem{a3} %3
\Au{Huisman T., Boucherie R.\,J.}
The sojourn time distribution in an infinite server
resequencing queue with dependent interarrival and
service times~// J.~Appl. Probab., 2002.
Vol.~39. No.\,3. P.~590--603.

\bibitem{a5} %4
\Au{Xia Y., Tse D.\,N.\,C.}
On the large deviations of resequencing
queue size: 2-$M$/$M$/1 сase~// IEEE Trans. Inform. Theory, 2008.
Vol.~54. No.\,9. P.~4107--4118.

\bibitem{a4} %5
\Au{Leung K., Li V.\,O.\,K.}
A~resequencing model for high-speed packet-switching networks~//
J.~Comput. Commun., 2010.
Vol.~33. No.\,4. P.~443--453.

\bibitem{a7} %6
\Au{Матюшенко С.\,И.} Стационарные характеристики двухканальной
системы обслуживания с~переупорядочением заявок и~распределениями
фазового типа~// Информатика и~её применения, 2010. Т.~4. Вып.~4.
С.~67--71.

\bibitem{a6} %7
\Au{De Nicola C., Pechinkin A.\,V., Razumchik~R.\,V.}
Stationary characteristics of homogenous Geo/Geo/2
queue with resequencing in discrete time~//
27th European Conference on Modelling and
Simulation Proceedings.~---- Aalesund, 2013. P.~594--600.

\bibitem{a7+} %8
\Au{Pechinkin A.\,V., Caraccio~I., Razumchik~R.\,V.}
Joint stationary distribution of queues in
homogenous $M\vert M\vert$3 queue with resequencing~//
28th European Conference on
Modelling and Simulation Proceedings.~--- Brescia, 2014. P.~558--564.

\bibitem{a8}
\Au{Pechinkin A.\,V., Caraccio~I., Razumchik~R.\,V.}
On joint stationary distribution in exponential
multiserver reordering queue~// 12th  Conference (International) on
Numerical Analysis and Applied Mathematics Proceedings, 2014 (in press).

\bibitem{boch}
\Au{Bocharov P.\,P., D'Apice C., Pechinkin~A.\,V., Salerno~S.}
Queueing theory.~--- Urecht, Boston: VSP, 2004. 446~p.

\bibitem{horn}
\Au{Horn R.\,A., Johnson C.\,R.}
Matrix analysis.~--- 2nd ed.~--- Cambridge: Cambridge University Press, 2013.
662~p.
 \end{thebibliography}

 }
 }

\end{multicols}

\vspace*{-9pt}

\hfill{\small\textit{Поступила в редакцию 28.10.14}}

%\newpage

\vspace*{12pt}

\hrule

\vspace*{2pt}

\hrule

%\vspace*{12pt}

\def\tit{JOINT STATIONARY DISTRIBUTION OF~THE~NUMBER OF~CUSTOMERS IN~THE~SYSTEM
AND REORDERING BUFFER IN~THE~MULTISERVER REORDERING QUEUE}

\def\titkol{Joint stationary distribution of~the~number of~customers in~the~system
and reordering buffer in~the~multiserver reordering queue}



\def\aut{\fbox{A.\,V.~Pechinkin}$^1$ and R.\,V.~Razumchik$^{1,2}$}

\def\autkol{A.\,V.~Pechinkin and R.\,V.~Razumchik}

\titel{\tit}{\aut}{\autkol}{\titkol}

\vspace*{-9pt}

\noindent
$^1$Institute of Informatics Problems, Russian Academy of Sciences,
44-2 Vavilov Str., Moscow 119333, Russian\\
$\hphantom{^1}$Federation


\noindent
$^2$Peoples' Friendship University of Russia,
6~Miklukho-Maklaya Str., Moscow 117198, Russian Federation



\def\leftfootline{\small{\textbf{\thepage}
\hfill INFORMATIKA I EE PRIMENENIYA~--- INFORMATICS AND
APPLICATIONS\ \ \ 2014\ \ \ volume~8\ \ \ issue\ 4}
}%
 \def\rightfootline{\small{INFORMATIKA I EE PRIMENENIYA~---
INFORMATICS AND APPLICATIONS\ \ \ 2014\ \ \ volume~8\ \ \ issue\ 4
\hfill \textbf{\thepage}}}

\vspace*{3pt}



\Abste{The paper considers a continuous-time multiserver queueing
system with buffer on infinite capacity and reordering. The Poisson
flow of customers arrives at the system. Service times of customers at
each server are exponentially distributed with the same parameter.
Each customer obtains a~sequential number upon arrival. The order of
customers upon arrival should be preserved upon departure from the system.
Customers whose service finished but which violated the order are kept in
the reordering buffer of infinite capacity. A~joint stationary distribution
of the number of customers in the buffer, servers, and
reordering buffer is obtained in terms of a~computational algorithm and
a~generating function. A~numerical example is provided.}


\KWE{queueing system; reordering; infinite capacity; joint distribution}

\DOI{10.14357/19922264140401}

%\vspace*{3pt}

\Ack
\noindent
The research was partially financially supported by the Russian Foundation for
Basic Research (project 13-07-00223).


  \begin{multicols}{2}

\renewcommand{\bibname}{\protect\rmfamily References}
%\renewcommand{\bibname}{\large\protect\rm References}



{\small\frenchspacing
 {%\baselineskip=10.8pt
 \addcontentsline{toc}{section}{References}
 \begin{thebibliography}{99}


 \bibitem{a1-1}
\Aue{Boxma O., G. Koole, and Z.~Liu}. 1994.
Queueing-theoretic solution methods for
models of parallel and distributed systems.
\textit{Performance Evaluation of Parallel and
Distributed Systems Solution Methods}.  CWI Tract 105
and 106:1--24.

\bibitem{a2-1}
\Aue{Dimitrov, B.} 1997.
Queues with resequencing. A~survey and recent results.
\textit{2nd World Congress on Nonlinear
Analysis, Theory, Methods, Applications Proceedings}. 30(8):5447--5456.

\bibitem{a3-1}
\Aue{Huisman, T., and R.\,J.~Boucherie}. 2002.
The sojourn time distribution in an infinite server
resequencing queue with dependent interarrival and service times.
\textit{J.~Appl. Probab}. 39(3):590--603.

\bibitem{a5-1}
\Aue{Xia, Y., and D.\,N.\,C.~Tse}. 2008.
On the large deviations of resequencing
queue size: 2-$M$/$M$/1 case.
\textit{IEEE Trans. Inform. Theory} 54(9):4107--4118.

\bibitem{a4-1} %5
\Aue{Leung, K., and V.\,O.\,K.~Li}. 2010.
A~resequencing model for high-speed
packet-switching networks.
\textit{J.~ Comput. Commun.} 33(4):443--453.

\bibitem{a7-1} %6
\Aue{Matyushenko, S.\,I.} 2010.
 Statsionarnye kharakteristiki
dvukh\-ka\-nal'\-noy sistemy obsluzhivaniya s~pe\-re\-upo\-rya\-do\-chi\-va\-ni\-em zayavok
i~raspredeleniyami
fazovogo tipa [Stationary characteristics of the two-channel
queueing system with reordering customers and distributions of phase type].
\textit{Informatika i ee Primemeniya}~--- \textit{Inform. Appl.}
4(4):67--71.

\bibitem{a6-1} %7
\Aue{De Nicola, C., A.\,V.~Pechinkin, and R.\,V.~Razumchik}. 2013.
Stationary characteristics of homogenous Geo/Geo/2
queue with resequencing in discrete time.
\textit{27th European Conference
on Modelling and Simulation Proceedings}. Aalesund. 594--600.

\bibitem{a7+-1}
\Aue{Pechinkin, A.\,V., I.~Caraccio, and R.\,V.~Razumchik}. 2014.
joint stationary distribution of queues
in homogenous $M \vert M \vert3$ queue with resequencing.
\textit{28th European Conference
on Modelling and Simulation Proceedings}. Brescia. 558--564.

\bibitem{a8-1}
\Aue{Pechinkin, A.\,V., I.~Caraccio, and R.\,V.~Razumchik}. 2014 (in press).
On joint stationary distribution in exponential
multiserver reordering queue.
\textit{12th  Conference (International) on
Numerical Analysis and Applied Mathematics Proceedings}.

\bibitem{boch-1}
\Aue{Bocharov,  P.\,P., C.~D'Apice, A.\,V.~Pechinkin, and S.~Salerno}. 2004.
\textit{Queueing theory}. Urecht, Boston: VSP. 446~p.

\bibitem{horn-1}
\Aue{Horn, R.\,A., and C.\,R.~Johnson}. 2013.
\textit{Matrix analysis}. Cambridge: Cambridge University Press. 662~p.
\end{thebibliography}

 }
 }

\end{multicols}

\vspace*{-6pt}

\hfill{\small\textit{Received October 28, 2014}}

\vspace*{-18pt}

\Contr

\noindent
\textbf{Pechinkin Alexander V.} (1946--2014)~--- Doctor
of Science in physics and mathematics; principal
scientist, Institute of Informatics Problems of
the Russian Academy of Sciences, 44-2 Vavilov Str.,
Moscow 119333, Russian Federation


\vspace*{3pt}

\noindent
\textbf{Razumchik Rostislav V.} (b.\ 1984)~--- Candidate
of Science (PhD) in physics and mathematics,
senior scientist, Institute of Informatics
Problems of the Russian Academy of Sciences, 44-2 Vavilov Str.,
Moscow 119333, Russian Federation;
associate professor,
Peoples' Friendship University of Russia,
6~Miklukho-Maklaya Str., Moscow 117198, Russian Federation;
rrazumchik@ieee.org


\label{end\stat}

\renewcommand{\bibname}{\protect\rm Литература} %1
\def\stat{kor-kor}



\def\tit{МОДИФИЦИРОВАННЫЙ СЕТОЧНЫЙ МЕТОД РАЗДЕЛЕНИЯ ДИСПЕРСИОННО-СДВИГОВЫХ
СМЕСЕЙ НОРМАЛЬНЫХ ЗАКОНОВ$^*$}



\def\titkol{Модифицированный сеточный метод разделения дисперсионно-сдвиговых
смесей нормальных законов}

\def\aut{В.\,Ю.~Королев$^1$,  А.\,Ю.~Корчагин$^2$}

\def\autkol{В.\,Ю.~Королев,  А.\,Ю.~Корчагин}

\titel{\tit}{\aut}{\autkol}{\titkol}

{\renewcommand{\thefootnote}{\fnsymbol{footnote}} \footnotetext[1]
{Работа поддержана Российским научным фондом (проект 14-11-00364).}}


\renewcommand{\thefootnote}{\arabic{footnote}}
\footnotetext[1]{Факультет
вычислительной математики и кибернетики Московского государственного
университета им.\ М.\,В.~Ломоносова; Институт проблем информатики
Российской академии наук; victoryukorolev@yandex.ru}
\footnotetext[2]{Факультет вычислительной математики и кибернетики
Московского государственного университета им.\ М.\,В.~Ломоносова;
sasha.korchagin@gmail.com}

%\vspace*{2pt}



\Abst{Описывается модифицированный двухэтапный
сеточный метод разделения дис\-пер\-си\-он\-но-сдви\-го\-вых смесей нормальных
законов, представляющий собой альтернативу чистому ЕМ (expectation-maximization)
ал\-го\-рит\-му. На
первом этапе этого алгоритма строится дискретная аппроксимация для
смешивающего распределения, на втором этапе подбирается абсолютно
непрерывное распределение из заранее заданного семейства, например,
обобщенных обратных гауссовских законов, ближайшее к~дискретному
распределению, полученному на первом этапе. Обсуждаются вопросы
сходимости этого двухэтапного алгоритма. Доказана монотонность
сеточного итерационного метода, используемого на первом этапе.
Подробно обсуждается вопрос оптимального выбора параметров метода,
прежде всего сетки, накидываемой на носитель смешивающего
распределения. С~этой целью предложены статистические оценки
квантилей смешивающего распределения. Эффективность метода
иллюстрируется примерами конкретных вычислений оценок параметров
обобщенных гиперболических распределений.}

\KW{смесь распределений вероятностей;
дис\-пер\-си\-он\-но-сдви\-го\-вая смесь нормальных законов; обобщенное
гиперболическое распределение; ЕМ-ал\-го\-ритм; сеточный метод
разделения смесей}

\vspace*{1pt}

%\vspace*{2pt}

\DOI{10.14357/19922264140402}


\vskip 12pt plus 9pt minus 6pt

\thispagestyle{headings}

\begin{multicols}{2}

\label{st\stat}

\section{Введение}

При {\it практическом} решении задачи моделирования и исследования
волатильности (изменчивости) хаотических стохастических процессов
ключевым этапом является статистическое разделение смесей
вероятностных распределений. Задача разделения смесей~---
статистического оценивания параметров смесей вероятностных
распределений~--- в~деталях разобрана, например, в~книге~\cite{k2011}.

Для решения задачи разделения смесей вероятностных распределений
традиционно используются итерационные процедуры типа ЕМ-ал\-го\-рит\-ма.
К~сожалению, классический ЕМ-ал\-го\-ритм обладает рядом серьезных
недостатков при его применении к~смесям нормальных законов, а~именно:
он демонстрирует крайнюю неустойчивость по отношению к~исходным
данным и~начальным приближениям.

Для преодоления этих недостатков
предложено много модификаций ЕМ-ал\-го\-рит\-ма (см., например,~\cite{k2011}).
Вместе с тем в~указанной книге предложен и~исследован
принципиально новый~--- сеточный~--- метод приближенного решения
задачи разделения смесей. В~работе~\cite{n2013} подробно исследованы
вопросы сходимости сеточных методов разделения смесей.

В соответствии с подходом к~статистическому анализу хаотических
стохастических процессов, в~частности к~решению задачи декомпозиции
волатильности таких процессов, развитом в~книге~\cite{k2011},
в~общем случае на практике приходится решать задачу разделения
конечных смесей нормальных законов с~произвольно большим числом
неизвестных параметров (параметров компонент и~их весов).
И~хотя в~большинстве приложений возникают смеси не более чем с~пятью--семью
компонентами, даже при использовании таких смесей, скажем, в~задачах
анализа и~прогнозирования финансовых рисков приходится моделировать
траекторию движения точки в~пространствах, размерность которых
соответственно лежит в~пределах от~14 (для пятикомпонентных смесей)
до~20 (для семикомпонентных смесей), что существенно увеличивает
вычислительные и~временн$\acute{\mbox{ы}}$е ресурсы, необходимые для практического
решения указанных задач.

Поскольку во многих ситуациях (например,
при прогнозировании на основе высокочастотных данных) эти задачи
необходимо решать в~режиме, близком к~реальному времени, для
создания эффективных методов статистического анализа на основе
смешанных моделей на первый план выходит проб\-ле\-ма снижения
размерности решаемой задачи, т.\,е.\ параметрического пространства.

Одним из возможных подходов к~снижению размерности является
априорное сужение классов допусти\-мых смесей. К~примеру, при решении
многих задач, связанных с~анализом процессов атмосферной или
плазменной турбулентности, а~так\-же процессов, описывающих эволюцию
различных финансовых индексов, высочайшую адекватность
продемонстрировали модели, основанные на дис\-пер\-си\-он\-но-сдви\-го\-вых
смесях нормальных законов. Класс таких смесей очень обширен
и,~в~част\-ности, включает в~себя обобщенные гиперболические распределения,
которые были введены О.-Е.~Барн\-дорфф-Ниль\-се\-ном в~1977--1978~гг.\ как
класс специальных сдвиг-мас\-штаб\-ных смесей нормальных законов~\cite{BN1977, BN1978}.
Пусть $\alpha\hm\in\r$, $\beta\hm\in\r$. Если
функцию распределения обобщенного гиперболического закона
с~параметрами~$\alpha$, $\beta$, $\nu$, $\mu$, $\lambda$ обозначить
$P_{GH}(x;\alpha,\beta,\nu,\mu,\lambda)$, то по определению
\begin{multline}
P_{GH}(x;\alpha,\beta,\nu,\mu,\lambda)={}\\
{}=
\int\limits_{0}^{\infty}\Phi\left(\fr{x-\beta-\alpha
z}{\sqrt{z}}\right)\,p_{GIG}(z;\nu,\mu,\lambda)\,dz\,,\\
x\in\r\,,
\label{e1-kor}
\end{multline}
где $\Phi(x)$~--- стандартная нормальная функция распределения:
$$
\Phi(x)=\int\limits_{-\infty}^{x}\varphi(z)\,dz\,,\enskip
\varphi(x)=\fr{1}{\sqrt{2\pi}}e^{-x^2/2}\,,\enskip  x\in\mathbb{R}\,;
$$
$p_{GIG}(x;\nu,\mu,\lambda)$~--- плот\-ность обобщенного обратного
гауссовского распределения:
\begin{multline*}
p_{GIG}(x;\nu,\mu,\lambda)={}\\
{}=\fr{\lambda^{\nu/2}}{2\mu^{\nu/2}
K_{\nu}\left(\sqrt{\mu\lambda}\right)}\,
x^{\nu-1}\exp\left\{-\fr{1}{2}\left(\fr{\mu}{x}+\lambda
x\right)\right\}\,,\\ x>0\,.
\end{multline*}
Здесь $\nu\in\r$;
$$
\begin{array}{lll}
\mu>0\,, & \lambda\geqslant0\,, & \mbox{если }\nu<0\,;\\[6pt]
\mu>0\,, & \lambda>0\,, & \mbox{если }\nu=0\,;\\[6pt]
\mu\geqslant0\,, & \lambda>0\,, & \mbox{если }\nu>0\,;
\end{array}
$$
$K_{\nu}(z)$~--- модифицированная бесселева функция третьего рода
порядка~$\nu$:

\noindent
\begin{multline*}
K_{\nu}(z)=\fr{1}{2}\int\limits_{0}^{\infty}y^{\nu-1}\exp
\left\{-\fr{z}{2}\left(y+\fr{1}{y}\right)\right\}\,dy\,,\\
z\in\mathbb{C}\,,\enskip \mathrm{Re}\,z>0\,.
\end{multline*}
Обратим внимание, что в~(1) смешивание происходит одновременно и~по
параметру сдвига, и~по параметру масштаба, но так как эти параметры
в~(1)  связаны жесткой зависимостью, так что параметр сдвига
смешиваемого распределения пропорционален его дисперсии, то
фактически смесь~(1) является {\it однопараметрической} и~поэтому
называется {\it дис\-пер\-си\-он\-но-сдви\-го\-вой} (см., например,~\cite{BN1982}).

Другим примером дис\-пер\-си\-он\-но-сдви\-го\-вых смесей нормальных законов
являются обобщенные дисперсионные гам\-ма-рас\-пре\-де\-ле\-ния, в~которых
смешивающими являются обобщенные гам\-ма-рас\-пре\-де\-ле\-ния~\cite{ks2012, zk2013}.

В указанных семействах смесей число неизвестных параметров равно
пяти или шести (если\linebreak учитывать неслучайный сдвиг). Вместе
с~тем у~подоб\-ных моделей имеются довольно серьезные тео\-ре\-ти\-че\-ские
обоснования: в~работах~\cite{zk2013, k2013} показано, что указанные
модели являются асимптотическими аппроксимациями в~простой
предельной схеме случайного суммирования и~потому могут успешно
применяться для анализа процессов типа остановленных случайных
блужданий. Эти выводы подтверждены статистическим анализом
вы\-со\-ко\-час\-тот\-ных финансовых данных, в~результате которого выявлен
синхронизированный характер изменения интенсивностей потоков заявок
в~сис\-те\-мах электронных торгов, что естественно приводит к~синхронизированному
поведению па\-ра\-мет\-ров сдвига и~диффузии в~соответствующих моделях вида смесей
нормальных законов~\cite{kckg2013}.

\section{Описание моди\-фи\-ци\-ро\-ван\-но\-го
сеточного ме\-то\-да разделения дисперсионно-сдвиговых смесей
нормальных законов и~его свойства}

Оказывается, что сеточные методы разделения смесей довольно
эффективны не только при разделении конечных смесей нормальных
законов, но и~при разделении произвольных дис\-пер\-си\-он\-но-сдви\-го\-вых
смесей нормальных законов. Поясним сказанное на примере задачи
оценивания па\-ра\-мет\-ров обобщенных гиперболических распределений.

Для решения задачи оценивания параметров обобщенных гиперболических
распределений традиционно используется метод, предложенный в~статье~\cite{p2004}
и~по сути являющийся классическим ЕМ-ал\-го\-рит\-мом,
приспособленным к~конкретной задаче, и,~соответственно, наследующий
присущие ЕМ-ал\-го\-рит\-мам недостатки.

Рассмотрим следующий альтернативный двухэтапный метод. На первом
этапе на поло\-жи\-тельной полупрямой выделим основную часть носителя
смешивающего распределения, т.\,е.\ \mbox{ограниченный} интервал,
вероятность которого, вычисленная в~соответствии со смешивающим
распределением, практически равна единице. На этот интервал накинем
конечную сетку, содержащую, возможно, очень много {\it известных}
узлов $u_1,\ldots,u_K$. Считая параметр сдвига~$\beta$ равным нулю,
приблизим искомое обобщенное гиперболическое распределение конечной
смесью нормальных законов:

\noindent
\begin{multline}
P_{GH}(x;\,\alpha,0,\nu,\mu,\lambda)\approx{}\\
{}\approx \sum\limits_{i=1}^K
p_i\Phi\left(\fr{x-\alpha u_i}{\sqrt{u_i}}\right)\,,\enskip
x\in\mathbb{R}\,.\label{e2-kor}
\end{multline}
В смеси, стоящей в~правой части соотношения~(2), неизвестными
являются только параметры $p_1,\ldots,p_{K-1}$ и~$\alpha$. Пусть
$x_1,\ldots,x_n$~--- анализируемая выборка значений случайной
величины с~оцениваемым обобщенным гиперболическим распределением.
Итерационный процесс, определяющий сеточный ЕМ-ал\-го\-ритм для данной
задачи, задается следующим образом. Пусть
$p_1^{(m)},\ldots,p_{K-1}^{(m)}$ и~$\alpha^{(m)}$~--- оценки параметров
$p_1,\ldots,p_{K-1}$ и~$\alpha$ на $m$-й итерации,
$p_K^{(m)}\hm=1\hm-p_1^{(m)}-\cdots-p_{K-1}^{(m)}$. Обозначим

\noindent
\begin{align*}
\varphi_{ij}^{(m)}&=\fr{1}{\sqrt{u_i}}\varphi\left(\fr{x_j-\alpha^{(m)}u_i}{\sqrt{u_i}}\right)\,;
\\
g_{ij}^{(m)}&=\fr{p_i^{(m)}\varphi_{ij}^{(m)}}{\sum\limits_{r=1}^K
p_r^{(m)}\varphi_{rj}^{(m)}}\,,\\
&\hspace*{14mm}i=1,\ldots,K\,;\enskip j=1,\ldots,n\,.
\end{align*}
Тогда, используя стандартные рассуждения, определяющие
вычислительные формулы EM-ал\-го\-рит\-ма для параметров конечной смеси
нормальных законов (см, например,~[1, разд.~5.3.7--5.3.8]),
следует положить

\noindent
\begin{equation}
p_i^{(m+1)}=\fr{1}{n}\sum\limits_{j=1}^n g_{ij}^{(m)}\,, \enskip
i=1,\ldots,K\,.\label{e3-kor}
\end{equation}
Обозначим $\overline{x}=(1/n)\sum\limits_{j=1}^nx_j$. Используя
соотношение~(5.3.24) в~\cite{k2011}, с~учетом очевидного равенства
$\sum\limits_{i=1}^K g_{ij}^{(m)}\hm=1$ можно заметить, что уточненная
оценка параметра~$\alpha$ имеет вид:

\columnbreak

\noindent
\begin{equation}
\alpha^{(m+1)}=\fr{\overline{x}}{\sum\limits_{i=1}^K u_ip_i^{(m+1)}}\,,
\label{e4-kor}
\end{equation}
т.\,е.\ равна отношению генерального выборочного среднего и~текущего
эмпирического среднего смешивающего распределения, что вполне
согласуется с~тем, что в~соответствии с~приводимым ниже соотношением~(\ref{e5-kor})
в~данном случае ${\sf E}X\hm=\alpha{\sf E}U$.

В силу монотонности классического ЕМ-ал\-го\-рит\-ма справедливо следующее
утверждение.

\smallskip

\noindent
\textbf{Теорема~1.} {\it Пусть узлы $u_1,\ldots,u_K$ сетки различны,
неотрицательны и~известны. Тогда итерационный процесс $(3)$--$(4)$
является монотонным, т.\,е.\ каждая его итерация не уменьшает
целевую сеточную функцию правдоподобия}
\begin{multline*}
L(p_1,\ldots,p_K,\alpha;x_1,\ldots,x_n)={}\\
{}=
\prod\nolimits_{j=1}^n\left[\sum\nolimits_{i=1}^K
\fr{p_i}{\sqrt{u_i}}\,\varphi\left(\fr{x_j-\alpha^{(m)}u_i}{\sqrt{u_i}}\right)\right].
\end{multline*}

\smallskip

\noindent
\textbf{Замечание~1.} В~разд.~5.7.4 книги~\cite{k2011} показано, что
при каждом фиксированном значении параметра~$\alpha$ сеточная
функция правдоподобия\linebreak
$L(p_1,\ldots,p_{K-1},\alpha;\,x_1,\ldots,x_n)$ вогнута по
аргументам $p_1,\ldots,p_{K-1}$. Поэтому на каждом шаге
итерационного процесса вместо соотношения~(3) можно\linebreak использо\-вать
любой более быстрый алгоритм максимизации функции
$L(p_1,\ldots,p_{K-1},\alpha^{(m)};\,x_1,\ldots$\linebreak $\ldots,x_n)$ по переменным
$p_1,\ldots,p_{K-1}$. Например, оценки весов $p_1,\ldots,p_K$ можно
искать методом условного градиента~\cite{k2011, kn2010}.

\smallskip

Таким образом, на первом этапе получаются оценки параметра~$\alpha$
и~весов всех узлов~$u_i$ конечной сетки, накинутой на носитель
смешивающего обобщенного обратного гауссовского распределения
$P_{\mathrm{GIG}}(z;\,\nu,\mu,\lambda)$.

На втором этапе остается применить ка\-кой-ли\-бо стандартный метод
подгонки обобщенного обратного гауссовского распределения
$P_{\mathrm{GIG}}(z;\,\nu,\mu,\lambda)$ к~эмпирическим данным типа
гистограммы $(u_1, p_1),\ldots, (u_K, p_K)$. Например, параметры~$\nu$,
$\mu$ и~$\lambda$ можно оценить, минимизируя соответствующую
статистику хи-квад\-рат. Или же, например, можно решить задачу
наименьших квад\-ратов:
\begin{multline*}
(\nu^*,\mu^*,\lambda^*)={}\\
{}=\arg\min\limits_{\nu,\mu,\lambda}\sum\limits_{i=1}^K
\left[p_i- \!\!\!\!\!
\int\limits_{(1/2)\left(u_{i-1}+u_i\right)}^{(1/2)(u_i+u_{i+1})}\!\!\!\!\!\!\!\!\!\!\!\!\!\!\!
p_{GIG}(u;\,\nu,\mu,\lambda)\,du\right]^2,
\end{multline*}
где $u_0=0$; $u_{K+1}\hm=\infty$.

На практике хорошие результаты показал подход с решением задачи
наименьших квадратов. Для поиска параметров использовался алгоритм
ns2sol, описанный в~книге~\cite{DSch1983}. Указанный алгоритм
доступен во многих статистических пакетах, отличается высоким
быстродействием и~возможностью при желании задавать разумные
интервалы для поиска параметров.

%\vspace*{-9pt}

\section{О практическом выборе сетки
на~первом этапе моди\-фи\-ци\-ро\-ван\-но\-го
сеточного метода разделения дисперсионно-сдвиговых смесей нормальных
законов}

Естественно, что при использовании указанного двухэтапного метода
в~динамическом режиме крайне важным становится вопрос о~выборе
наиболее эффективных и~быстродействующих численных процедур и~их
параметров. В~частности, исключительную важность приобретает
правильный выбор сетки на первом этапе. Рассмотрим этот вопрос
подробнее.

Формально рассматриваемая задача выглядит так: по наблюдаемым
значениям $x_1,\ldots,x_n$ требуется построить статистическую оценку
верхней границы квантилей заданного порядка сме\-ши\-ва\-юще\-го закона так,
чтобы как можно точнее оценить носитель смешивающего распределения.

В дальнейшем будем считать, что $x_1,\ldots,x_n$~--- независимые
реализации случайной величины $X\hm=Y\sqrt{U}+\alpha U$, где $Y$~---
случайная величина со стандартным нормальным распределением, а~$U$~---
независимая от нее случайная величина с~обобщенным обратным
гауссовским распределением. Тогда, очевидно, распределение случайной
величины~$X$ имеет вид~(1). Предположим, что у~случайной величины~$U$
существуют моменты первых двух порядков. Тогда, как несложно видеть,
\begin{equation}
{\sf E}X={\sf E}Y\cdot{\sf E}\sqrt{U}+\alpha{\sf E}U=\alpha{\sf
E}U\,.\label{e5-kor}
\end{equation}
При этом по усиленному закону больших чисел с~вероятностью единица
$\overline x\hm\longrightarrow {\sf E}X$ $(n\hm\to\infty)$, так что при
больших~$n$ справедливо приближенное равенство ${\sf E}X\hm\approx\overline x$
и~с учетом~(\ref{e5-kor})
\begin{equation}
{\sf E}U\approx\fr{\overline x}{\alpha}\,.\label{e6-kor}
\end{equation}
Далее, очевидно,

\columnbreak

\noindent
\begin{multline}
{\sf E}X^2={\sf E}Y^2\cdot{\sf E}U+2\alpha{\sf E}X\cdot{\sf E}U^{3/2}+{}\\
{}+
\alpha^2{\sf E}U^2={\sf E}U+\alpha^2{\sf E}U^2\,.
\label{e7-kor}
\end{multline}

\noindent
Поэтому, обозначив
$$
m^2=\fr{1}{n}\sum\limits_{i=1}^nx_i^2\,,
$$
получаем приближенное равенство ${\sf E}X^2\hm\approx m^2$, так что
с~учетом~(\ref{e6-kor}) и~(\ref{e7-kor}) имеем:
\begin{equation}
{\sf E}U^2\approx\fr{1}{\alpha^2}\left(m^2-\fr{\overline
x}{\alpha}\right)\,.\label{e8-kor}
\end{equation}
Если параметр~$\alpha$ известен, то для определения верхней границы~$u^*$
сетки, накидываемой на носитель распределения случайной
величины~$U$, можно задать малое положительное число~$\varepsilon$
и~воспользоваться требованием
\begin{equation}
{\sf P}(U\geqslant u^*)\leqslant\varepsilon\,.\label{e9-kor}
\end{equation}
А~для гарантированного выполнения требования~(\ref{e9-kor}) можно использовать
неравенство Маркова:
$$
{\sf P}(U\geqslant u^*)\leqslant\fr{{\sf E}U^2}{(u^*)^2}\leqslant \varepsilon\,,
$$
откуда с учетом~(\ref{e8-kor})
$$
(u^*)^2\geqslant\fr{{\sf E}U^2}{\varepsilon}\approx
\fr{1}{\alpha^2\varepsilon}\left( m^2-\fr{\overline x}{\alpha}\right)
$$
или
\begin{equation}
u^*\approx\fr{1}{\alpha\sqrt{\varepsilon}}\sqrt{m^2-
\fr{\overline x}{\alpha}}\,.\label{e10-kor}
\end{equation}

\begin{figure*}[b] %fig1
\vspace*{1pt}
 \begin{center}
 \mbox{%
 \epsfxsize=161.718mm
 \epsfbox{kor-1.eps}
 }
 \end{center}
 \vspace*{-9pt}
\Caption{Примеры применения модифицированного двухэтапного сеточного
ЕМ-ал\-го\-рит\-ма для подгонки обобщенного гиперболического распределения
к искусственным данным, $\beta\hm=0$: (\textit{a})~$n\hm=1000$, $\alpha\hm=0{,}3$,
$\nu\hm=1{,}3$, $\mu\hm=1{,}6$, $\lambda\hm=0{,}2$;
(\textit{б})~$n\hm=1000$, $\alpha\hm=0{,}5$, $\nu\hm=1$, $\mu\hm=1$,
$\lambda\hm=3$;
(\textit{в})~$n\hm=1000$, $\alpha\hm=3$,
 $\nu\hm=1{,}3$, $\mu\hm=1{,}6$, $\lambda\hm=2$;
(\textit{г})~$n\hm=10\,000$,
$\alpha\hm=0{,}3$, $\nu\hm=1{,}3$, $\mu\hm=1{,}6$, $\lambda\hm=0{,}2$}
\end{figure*}


Если же параметр~$\alpha$, определяющий асим\-мет\-рию распределения
случайной величины~$X$, неизвестен, то можно воспользоваться
следующими рассуждениями. Обозначим
$$
q_n=\fr{1}{n}\sum\limits_{i=1}^n{\bf 1}(x_i<0)\,,
$$
где ${\bf 1}(A)$~--- индикаторная функция множества (события)~$A$.
При этом по усиленному закону больших чисел с~вероятностью единица
$q_n\hm\longrightarrow {\sf P}(X\hm<0)$ $(n\hm\to\infty)$, так что при
больших~$n$ справедливо приближенное равенство
\begin{equation}
q_n\approx{\sf P}(X<0)\,.\label{e11-kor}
\end{equation}
Но
\begin{multline}
{\sf P}(X<0)=\int\limits_{0}^{\infty}\Phi
\left(-\alpha\sqrt{u}\right) p_{\mathrm{GIG}}(u;\nu,\mu,\lambda)\,du={}\\
{}=
{\sf E}\Phi\left(-\alpha\sqrt{U}\right)\,.\label{e12-kor}
\end{multline}

\pagebreak

\noindent
Предположим сначала, что $q_n\hm<1/2$. Если~$n$ достаточно велико,
то можно с~большой степенью
 уверенности утверж\-дать, что тогда
$\overline x\hm>0$ и~$-\alpha\hm<0$, т.\,е.
 $\alpha\hm>0$ и,~стало быть, на
положительной полуоси значений аргумента~$u$ функция $\Phi(\alpha u)$
вогнута, т.\,е.\ выпукла вверх. Тогда из~(\ref{e11-kor}) и~(\ref{e12-kor}), дважды
применяя неравенство Иенсена, в~силу монотонности функции~$\Phi$
получаем:
\begin{multline}
1-q_n\approx 1-{\sf E}\Phi\left(-\alpha\sqrt{U}\right)=
          {\sf E}\Phi\left(\alpha\sqrt{U}\right)\leqslant{}\\
          {}\leqslant\Phi
          \left(\alpha{\sf E}\sqrt{U}\right)\leqslant
          \Phi\left(\alpha\sqrt{{\sf E}U}\right)\,.\label{e13-kor}
\end{multline}
Если теперь для $t\hm\in(0,1)$ символом~$v_t$ обозначить $t$-кван\-тиль
стандартного нормального закона, то из~(\ref{e13-kor}) и~(\ref{e6-kor}) вытекает
<<приближенное неравенство>>
$$
v_{1-q_n}\hm\leqslant \alpha\sqrt{{\sf E}U}\,,
$$
т.\,е.
$$
\alpha\geqslant\fr{v_{1-q_n}}{\sqrt{{\sf E}U}}\approx
\fr{v_{1-q_n}\sqrt{\alpha}}{\sqrt{\overline x}}\,,
$$
откуда получаем, что при достаточно больших~$n$
\begin{equation}
\alpha\geqslant\fr{v_{1-q_n}^2}{\overline x}\,.\label{e14-kor}
\end{equation}
Если теперь задать малое положительное число~$\varepsilon$, то
для определения верхней границы~$u^*$ сетки, накидываемой на
носитель распределения случайной величины~$U$, можно воспользоваться
требованием~(\ref{e9-kor}), для гарантированного выполнения которого
с~учетом~(\ref{e6-kor}) и~(\ref{e14-kor}) можно использовать неравенство Маркова:
$$
{\sf P}(U\geqslant u^*)\leqslant \fr{{\sf E}U}{u^*}\approx\fr{\overline
x}{\alpha u^*}\leqslant \fr{(\overline x)^2}{v_{1-q_n}^2 u^*}\leqslant
\varepsilon\,,
$$
откуда окончательно вытекает оценка
\begin{equation}
u^*\approx\fr{(\overline x)^2}{v_{1-q_n}^2 \varepsilon}\,.\label{e15-kor}
\end{equation}

\begin{figure*}[b] %fig2
\vspace*{18pt}
 \begin{center}
 \mbox{%
 \epsfxsize=162.433mm
 \epsfbox{kor-3.eps}
 }
 \end{center}
 \vspace*{-9pt}
\Caption{Примеры применения модифицированного двухэтапного
сеточного ЕМ-ал\-го\-рит\-ма для подгонки обобщенного гиперболического
распределения к~искусственным данным, $n=10\,000$, $\beta\hm=0$:
(\textit{а})~$\alpha\hm=0{,}3$,
$\nu\hm=2$, $\mu\hm=2$, $\lambda\hm=2{,}5$;
(\textit{б})~$\alpha\hm=0{,}5$,  $\nu\hm=1$, $\mu\hm=1$, $\lambda\hm=3$;
(\textit{в})~$\alpha\hm=0{,}8$,
$\nu\hm=1{,}3$, $\mu\hm=1{,}6$, $\lambda\hm=2$;
(\textit{г})~$\alpha\hm=1{,}3$, $\nu\hm=2$, $\mu\hm=2$, $\lambda\hm=2{,}5$}
\end{figure*}



В случае $q_n\hm\geqslant1/2$, если $n$ достаточно велико, то можно
с~большой степенью уверенности утверж\-дать, что $\overline x\hm\leqslant 0$
и~$-\alpha\hm\geqslant 0$, т.\,е.\ на положительной\linebreak\vspace*{-12pt}

\pagebreak

%\end{multicols}


%\begin{multicols}{2}

\noindent
 полуоси значений аргумента~$u$
функция $\Phi(-\alpha u)$ вогнута, т.\,е.\ выпукла вверх. Тогда
из~(\ref{e11-kor}) и~(\ref{e12-kor}), дважды применяя неравенство Иенсена, в~силу
монотонности функции~$\Phi$ получаем
$$
q_n\approx {\sf E}\Phi\left(-\alpha\sqrt{U}\right)\leqslant
\Phi\left(-\alpha\sqrt{{\sf E}U}\right)\,,
$$
откуда вытекает <<приближенное неравенство>> $v_{q_n}\hm \leqslant
-\alpha\sqrt{{\sf E}U}$,
т.\,е.
$$
-\alpha\geqslant\fr{v_{q_n}}{\sqrt{{\sf E}U}}\approx
\fr{v_{q_n}\sqrt{|\alpha|}}{\sqrt{|\overline x|}}
$$
и при достаточно больших~$n$
\begin{equation}
|\alpha|\geqslant\fr{v_{q_n}^2}{|\overline x|}\,.\label{e16-kor}
\end{equation}
Для определения верхней границы~$u^*$ сетки, накидываемой на
носитель распределения случайной величины~$U$, снова зададим малое
положительное число~$\varepsilon$ и~потребуем, чтобы было
справедливо условие~(\ref{e9-kor}), для гарантированного выполнения которого
с~учетом~(\ref{e6-kor}) и~(\ref{e16-kor}) используем неравенство Маркова и~тот факт, что
$\mathrm{sign}\, \overline x\hm=\mathrm{sign}\,\alpha$ при достаточно
больших~$n$:
\begin{multline}
{\sf P}(U\geqslant u^*)\leqslant \fr{{\sf E}U}{u^*}\approx
\fr{\overline x}{\alpha u^*}=
\fr{|\overline x|}{|\alpha| u^*} \leqslant{}\\
{}\leqslant
\fr{(\overline x)^2}{v_{q_n}^2 u^*}\leqslant
\varepsilon\,.\label{e17-kor}
\end{multline}
В силу симметричности нормального распределения $v_{t}\hm=-v_{1-t}$ для
любого $t\hm\in(0,1)$, поэтому $v_{q_n}^2\hm=v_{1-q_n}^2$ и~в~случае
$q_n\hm\geqslant1/2$ соотношение~(\ref{e17-kor}) снова приводит к~оценке~(\ref{e15-kor}).

Справедливости ради необходимо отметить, что оценки~(\ref{e10-kor}) и~(\ref{e15-kor})
являются завышенными, но они гарантируют, что
$(1-\varepsilon)$-почти-весь носитель распределения случайной
величины~$U$ будет лежать внутри интервала $[0, u^*]$.

\section{Результаты численных экспериментов}

Приводимые в~данном разделе графики иллюстрируют качество работы
модифицированного сеточного метода разделения дис\-пер\-си\-он\-но-сдви\-го\-вых
смесей нормальных законов на примере его\linebreak применения к~оцениванию
параметров обоб\-щенных гиперболических распределений с~ис\-поль\-зованием
указанного алгоритма выбора сетки\linebreak с~умеренным чис\-лом узлов $K\hm=40$.
Для вы\-чис\-ле\-ний использовались искусственно сгенерированные выборки
объемов $n\hm=1000$ и~$n\hm=10\,000$ с~разными наборами параметров, значения
которых указаны на рисунках. На рис.~1 и~2 изображены гистограммы
(серые столбики) и~графики
истинной плот\-ности (штриховые линии), промежуточной
оценки, полученной сеточным ЕМ-ал\-го\-рит\-мом (пунктирные линии)
и~итоговой оценки (непрерывные линии). На рис.~1 и~2 так\-же указаны
значения полученных оценок параметров. Как видно из приводимых
рисунков, параметры~$\alpha$ оцениваются очень точно. Точность
оценок остальных параметров удовлетворительная и~может быть повышена
за счет использования более частых сеток и~более чувствительных
критериев остановки ЕМ-ал\-го\-рит\-ма на первом этапе. Следует отметить,
что даже в~тех случаях, в~которых наблюдаются заметные расхождения
оценок параметров и~их точных значений, оценки самих плотностей
довольно \mbox{точны}.




{\small\frenchspacing
 {%\baselineskip=10.8pt
 \addcontentsline{toc}{section}{References}
 \begin{thebibliography}{99}
\bibitem{k2011}
\Au{Королев В.\,Ю.} Ве\-ро\-ят\-но\-ст\-но-ста\-ти\-сти\-че\-ские методы
декомпозиции волатильности хаотических процессов.~--- М.: Изд-во
Московского ун-та, 2011.

\bibitem{n2013}
\Au{Назаров А.\,Л.} Приближенные методы разделения смесей
вероятностных распределений: Дисс.\ \ldots\  канд. физ.-мат. наук.~--- М.:
МГУ им.\ М.\,В.~Ломоносова, 2013.

\bibitem{BN1977}
\Au{Barndorff-Nielsen~O.-E.} Exponentially decreasing distributions
for the logarithm of particle size~// Proc. Roy. Soc. Lond.~A,
1977. Vol.~353. P.~401--419.

\bibitem{BN1978}
\Au{Barndorff-Nielsen~O.-E.} Hyperbolic distributions and
distributions of hyperbolae~// Scand. J. Statist., 1978. Vol.~5.
P.~151--157.

\bibitem{BN1982}
\Au{Barndorff-Nielsen~O.-E., Kent~J., S\!{\!\ptb{\!\o}}\,rensen~M.} Normal
variance-mean mixtures and $z$-distributions~// Int. Statist. Rev.,
1982. Vol.~50. No.\,2. P.~145--159.

\bibitem{ks2012}
\Aue{Королев В.\,Ю., Соколов И.\,А.} Скошенные распределения
Стьюдента, дисперсионные гам\-ма-рас\-пре\-де\-ле\-ния и~их обобщения как
асимптотические аппроксимации~// Информатика и~её применения, 2012.
Т.~6. Вып.~1. С.~2--10.

\bibitem{zk2013}
\Au{Закс Л.\,М., Королев В.\,Ю.} Обобщенные дисперсионные
гам\-ма-рас\-пре\-де\-ле\-ния как предельные для случайных сумм~// Информатика
и её применения, 2013. Т.~7. Вып.~1. С.~105--115.

\bibitem{k2013}
\Au{Королев В.\,Ю.} Обобщенные гиперболические
распределения как предельные для случайных сумм~// Тео\-рия
вероятностей и~ее применения, 2013. Т.~58. Вып.~1. С.~117--132.

\bibitem{kckg2013}
\Au{Королев В.\,Ю., Черток А.\,В., Корчагин~А.\,Ю.,
Горшенин~А.\,К.} Ве\-ро\-ят\-но\-ст\-но-ста\-ти\-сти\-че\-ское моделирование
информационных потоков в~сложных финансовых системах на основе
высокочастотных данных~// Информатика и~её применения, 2013. Т.~7.
Вып.~1. С.~12--21.

\bibitem{p2004}
\Au{Protassov R.\,S.} EM-based maximum likelihood parameter
estimation for a~multivariate generalized hyperbolic distribution
with fixed~$\lambda$~// Statistics Computing, 2004. Vol.~14.
P.~67--77.

\bibitem{kn2010}
\Au{Королев В.\,Ю., Назаров А.\,Л.} Разделение смесей
вероятностных распределений при помощи сеточных методов моментов и~максимального правдоподобия~//
Автоматика и~телемеханика, 2010. Вып.~3. С.~98--116.

\bibitem{DSch1983}
\Au{Dennis J.\,E., Schnabel R.\,B.} Numerical methods for
unconstrained optimization and nonlinear equations.~--- Englewood
Cliffs: Prentice-Hall, 1983. 378~p.
 \end{thebibliography}

 }
 }

\end{multicols}

\vspace*{-6pt}

\hfill{\small\textit{Поступила в редакцию 01.10.14}}

\newpage

%\vspace*{12pt}

%\hrule

%\vspace*{2pt}

%\hrule

%\vspace*{12pt}

\def\tit{A MODIFIED GRID METHOD FOR~STATISTICAL SEPARATION
OF~NORMAL VARIANCE-MEAN MIXTURES}

\def\titkol{A modified grid method for statistical separation
of~normal variance-mean mixtures}

\def\aut{V.\,Yu.~Korolev$^{1,2}$ and~A.\,Yu.~Korchagin$^1$}

\def\autkol{V.\,Yu.~Korolev and~A.\,Yu.~Korchagin}

\titel{\tit}{\aut}{\autkol}{\titkol}

\vspace*{-9pt}


\noindent
$^1$Faculty of Computational Mathematics and Cybernetics,
M.\,V.~Lomonosov Moscow State University,\linebreak
$\hphantom{^1}$1-52 Leninskiye Gory, GSP-1, Moscow 119991, Russian Federation


\noindent
$^2$Institute of Informatics Problems, Russian Academy of Sciences,
44-2~Vavilov Str., Moscow 119333, Russian\linebreak
$\hphantom{^1}$Federation

\def\leftfootline{\small{\textbf{\thepage}
\hfill INFORMATIKA I EE PRIMENENIYA~--- INFORMATICS AND
APPLICATIONS\ \ \ 2014\ \ \ volume~8\ \ \ issue\ 4}
}%
 \def\rightfootline{\small{INFORMATIKA I EE PRIMENENIYA~---
INFORMATICS AND APPLICATIONS\ \ \ 2014\ \ \ volume~8\ \ \ issue\ 4
\hfill \textbf{\thepage}}}

\vspace*{3pt}

\Abste{A~modified two-stage grid method for
statistical separation of normal variance-mean mixtures is described
as an alternative to a pure EM (expectation-maximization) algorithm.
At the first stage of this
algorithm, a~discrete approximation is constructed to the mixing
distribution. At the second stage, the obtained discrete
distribution is approximated by an absolutely continuous
distribution from a~predetermined family, say, by a generalized
inverse Gaussian distribution. The convergence of this two-stage
procedure is discussed. The monotonicity of the grid procedure used
at the first stage is proved. The problem of the optimal choice of
the parameters of the method is discussed in detail. First of all,
the problem of the optimal choice of the grid thrown on the support
of the mixing distribution is considered. Statistical estimators are
proposed for the quantiles of the mixing law. The efficiency of the
method is illustrated by examples of its application to the
estimation of the parameters of generalized hyperbolic
distributions.}

\smallskip

\KWE{mixture of probability distributions; normal
variance-mean mixture; generalized hyperbolic distribution;
EM-algorithm; grid method of separation of mixtures}

\DOI{10.14357/19922264140402}

\Ack
\noindent
The research was supported by the Russian Science Foundation (project 14-11-00364).

%\vspace*{3pt}

  \begin{multicols}{2}

\renewcommand{\bibname}{\protect\rmfamily References}
%\renewcommand{\bibname}{\large\protect\rm References}



{\small\frenchspacing
 {%\baselineskip=10.8pt
 \addcontentsline{toc}{section}{References}
 \begin{thebibliography}{99}
 \bibitem{k2011eng}
 \Aue{Korolev, V.\,Yu.} 2011.
\textit{Veroyatnostno-statisticheskie metody dekompozitsii
volatil'nosti khaoticheskikh protsessov}
[Probabilistic and statistical methods for the decomposition of volatility
of chaotic processes].
Moscow: Moscow University Press. 510~p.

\bibitem{n2013eng}
\Aue{Nazarov, A.\,L.} 2013.
{Priblizhennye metody razdeleniya smesey veroyatnostnykh raspredeleniy}
[Approximate methods for the decomposition of volatility of chaotic processes].
Ph.D. Thesis. Moscow: Moscow State University.

\bibitem{BN1977eng}
\Aue{Barndorff-Nielsen, O.\,E.} 1977.
Exponentially decreasing distributions for the logarithm of particle size.
\textit{Proc. Roy. Soc. Lond. A} 353:401--419.

\bibitem{BN1978eng}
\Aue{Barndorff-Nielsen, O.\,E.} 1978.
Hyperbolic distributions and distributions of hyperbolae.
\textit{Scand. J. Statist.} 5:151--157.

\bibitem{BN1982eng}
\Aue{Barndorff-Nielsen, O.\,E., J.~Kent, and M.~S\!{\ptb{\o}}rensen}. 1982.
Normal variance-mean mixtures and $z$-distributions.
\textit{Int. Statist. Rev.} 50(2):145--159.

\bibitem{ks2012eng}
\Aue{Korolev, V.\,Yu., and I.\,A. Sokolov}. 2012.
{Skoshennye raspredeleniya St'yudenta, dispersionnye
gam\-ma-ras\-pre\-de\-le\-niya i~ikh obobshcheniya kak asimptoticheskie
approksimatsii}
[Skewed Student's distributions, variance gamma distributions, and their
generalizations as asymptotic approximations].
\textit{Informatika i ee Primeneniya}~--- \textit{Inform. Appl.} 6(1):2--10.

\bibitem{zk2013eng}
\Aue{Korolev, V.\,Yu., and L.\,M.~Zaks}. 2013.
{Obobshchennye dispersionnye gam\-ma-ras\-pre\-de\-le\-niya kak
predel'nye dlya sluchaynykh summ}
[Generalized variance gamma distributions as limiting for random sums].
\textit{Informatika i ee Primeneniya}~--- \textit{Inform. Appl.} 7(1):105--115.

\bibitem{k2013eng} \Aue{Korolev, V.\,Yu.} 2013.
{Obobshchennye giperbolicheskie raspredeleniya kak predel'nye dlya sluchaynykh summ}
[Generalized hyperbolic distributions as limiting for random sums]
\textit{Theory Probab. Appl.} 58(1):117--132.

\bibitem{kckg2013eng}
\Aue{Korolev, V.\,Yu., A.\,V. Chertok, A.\,Yu.~Korchagin, and A.\,K.~Gorshenin}.
2013. {Ve\-ro\-yat\-no\-st\-no-sta\-ti\-sti\-che\-skoe
mo\-de\-li\-ro\-va\-nie informatsionnykh potokov v~slozhnykh finansovykh sistemakh
na osnove vysokochastotnykh dannykh}
[Probability and statistical modeling of information flows in complex
financial systems from high-frequency data].
\textit{Informatika i~ee Primeneniya}~--- \textit{Inform.  Appl.} 7(1):12--21.

\bibitem{p2004eng-1}
\Aue{Protassov, R.\,S.} 2004.
EM-based maximum likelihood parameter estimation for a multivariate
generalized hyperbolic distribution with fixed~$\lambda$.
\textit{Statistics Computing} 14:67--77.

\bibitem{kn2010eng-1}
\Aue{Korolev, V.\,Yu., and A.\,L.~Nazarov}. 2010.
{Razdelenie smesey veroyatnostnykh raspredeleniy pri pomoshchi
setochnykh metodov momentov i~maksimal'nogo pravdopodobiya}
[Separation of mixtures using grid moment-based methods and maximum likelihood].
\textit{Avtomatika i~Telemekhanika} [Automatics and Telemechanics] 3:98--116.

\bibitem{DSch1983eng}
\Aue{Dennis, J.\,E., and R.\,B.~Schnabel}. 1983.
\textit{Numerical methods for unconstrained optimization and nonlinear equations}.
Englewood Cliffs: Prentice-Hall. 378~p.


\end{thebibliography}

 }
 }

\end{multicols}

\vspace*{-6pt}

\hfill{\small\textit{Received October 01, 2014}}

\vspace*{-18pt}

\Contr

\noindent
\textbf{Korolev Victor Yu.} (b.\ 1954)~---
Doctor of Science in physics and mathematics, professor,
Department of Mathematical Statistics, Faculty of Computational Mathematics
and Cybernetics, M.\,V.~Lomonosov Moscow State University,
1-52 Leninskiye Gory, GSP-1, Moscow 119991, Russian Federation;
leading scientist, Institute of Informatics Problems,
Russian Academy of Sciences, 44-2~Vavilov Str., Moscow 119333, Russian
Federation; victoryukorolev@yandex.ru

\vspace*{3pt}

\noindent
\textbf{Korchagin Alexander Yu.} (b.\ 1989)~---
PhD student, Faculty of Computational Mathematics and Cybernetics,
M.\,V.~Lomonosov Moscow State University,
1-52 Leninskiye Gory, GSP-1, Moscow 119991, Russian Federation;
sasha.korchagin@gmail.com


\label{end\stat}

\renewcommand{\bibname}{\protect\rm Литература} %2
\def\stat{chertok}

\def\tit{МЕТОД КУМУЛЯТИВНЫХ СУММ ДЛЯ~ПОИСКА СМЕНЫ РЕЖИМА В~ПРОЦЕССЕ 
ОРНШТЕЙНА--УЛЕНБЕКА\\ НА~ОСНОВЕ ПРОЦЕССА ЛЕВИ$^*$}

\def\titkol{Метод кумулятивных сумм для поиска смены режима в~процессе 
Орнштейна--Уленбека на основе процесса Леви}

\def\aut{А.\,В.~Черток$^1$, А.\,И.~Каданер$^2$, Г.\,Т.~Хазеева$^3$, И.\,А.~Соколов$^4$}

\def\autkol{А.\,В.~Черток, А.\,И.~Каданер, Г.\,Т.~Хазеева, И.\,А.~Соколов}

\titel{\tit}{\aut}{\autkol}{\titkol}

\index{Черток А.\,В.}
\index{Каданер А.\,И.}
\index{Хазеева Г.\,Т.}
\index{Соколов И.\,А.}
\index{Chertok A.\,V.}
\index{Kadaner A.\,I.}
\index{Khazeeva G.\,T.} 
\index{Sokolov I.\,A.}


{\renewcommand{\thefootnote}{\fnsymbol{footnote}} \footnotetext[1]
{Работа выполнена при частичной 
финансовой поддержке РФФИ (проект 14-07-00041).}}


\renewcommand{\thefootnote}{\arabic{footnote}}
\footnotetext[1]{Факультет вычислительной математики и~кибернетики 
Московского государственного университета им.\ М.\,В.~Ломоносова; Сбербанк России, 
\mbox{avchertok.sbt@sberbank.ru}}
\footnotetext[2]{Механико-математический 
факультет Московского государственного университета им.\ М.\,В.~Ломоносова; 
Сбербанк России, \mbox{aikadaner.sbt@sberbank.ru}}
\footnotetext[3]{Факультет вычислительной математики и~кибернетики 
Московского государственного университета им.~М.\,В.~Ломоносова, 
\mbox{gelana.khazeyeva@gmail.com}}
\footnotetext[4]{Институт проб\-лем информатики Федерального 
исследовательского центра <<Информатика и~управ\-ле\-ние>> Российской академии наук, 
\mbox{isokolov@ipiran.ru}}

\vspace*{-3pt}

\Abst{Рассматривается процесс Орн\-штей\-на--Улен\-бе\-ка (ОУ) с~трендом 
на основе процесса Леви для моделирования финансовых временных рядов. 
Продемонстрировано, что использование процесса Леви в~основе процесса 
ОУ дает больше гибкости для описания финансовых 
временных рядов по 
сравнению с~классической гауссовой моделью. В~частности, процесс Леви позволяет 
моделировать остатки с~тяжелыми хвостами, что является  распространенным 
свойством реальных временных рядов. Приводятся эффективные решения для 
оценивания параметров модели с~использованием таких методов, как OLS (ordinary least squares)
и~RLS (regularized least squares). 
Решается задача поиска моментов смены режима в~модели при условии поступления 
данных в~режиме реального времени. Приведен алгоритм, основанный на  
CUSUM (CUmulative SUM) ме\-то\-дах,  способный последовательно обрабатывать смены режима и~поддерживать 
параметры модели актуальными для каждого момента времени.  Решение задачи поиска 
разладки модели и~соответствующих смен режима имеет важное прикладное значение, 
поскольку в~большинстве случаев параметры моделей, описывающих динамику реальных 
систем, меняются во времени под действием внешних факторов.}

\KW{случайные процессы; процессы со свойством возвратности к~среднему; 
процесс Орн\-штей\-на--Улен\-бе\-ка, управляемый процессом Леви; процесс 
Орн\-штей\-на--Улен\-бе\-ка с~трендом; смена режима; CUSUM-ал\-го\-ритмы}

\DOI{10.14357/19922264160405}

\vspace*{-3pt} 


\vskip 10pt plus 9pt minus 6pt

\thispagestyle{headings}

\begin{multicols}{2}

\label{st\stat}

\section{Введение}

Процессы со свойством возвратности к~среднему играют важную роль в~моделировании 
динамики явлений из различных областей человеческой деятельности.  В~частности, 
эти процессы привлекательны для моделирования различных явлений в~эконометрике, 
таких как процентные ставки, курсы обмена валют и~цены на сырьевые товары, где 
свойство возвратности к~среднему имеет фундаментальную природу. 

В~работе~\cite{brigo2007} рассмотрено несколько видов случайных процессов со свойством 
возвратности к~среднему и~описаны их основные характеристики.
В~настоящей статье в~качестве такого процесса рассматривается процесс 
ОУ, управляемый процессом Леви. 

Классическая версия 
процесса была 
впервые представлена в~совместной работе голландских физиков Л.\,С.~Орнштейна 
и~Дж.\,Е.~Уленбека~\cite{ou1930} в~качестве модели, которая способна описать данные 
с~гауссовской и~диффузионной структурой. В~экономике же классический процесс 
ОУ известен как модель Васичека благодаря фундаментальной 
работе~\cite{vasicek1977}, где автор предлагает использовать ее для 
моделирования временн$\acute{\mbox{о}}$го ряда процентной ставки. Ее основной недостаток 
заключается в~том, что существует ненулевая вероятность появления отрицательных 
значений, нереалистичных для экономических процессов. Для решения данной 
проблемы позднее была разработана экспоненциальная модель Васичека, а~также 
модель процесса Кок\-са--Ин\-гер\-сол\-ла--Рос\-са, также называемая <<мо\-делью 
с~квад\-рат\-ным корнем>>, в~которой процентная ставка принимает только 
неотрицательные значения и~имеет гам\-ма-рас\-пре\-де\-ле\-ние~\cite{cox1985}.

        \begin{figure*} %fig1
        \vspace*{1pt}
\begin{center}
\mbox{%
\epsfxsize=109.749mm
\epsfbox{che-1.eps}
}
\end{center}
\vspace*{-9pt}
 \Caption{График соотношения цен для фьючерсов компаний 
<<Лукойл>> и~<<Роснефть>>}
                \label{rtsmixpic}
        \end{figure*}


В настоящей статье подтверждается тот факт, что предположение нормальности 
в~классической версии процесса ОУ не описывает реальную структуру 
данных, и~поэтому рассматривается обоб\-щение классического процесса~--- процесс 
ОУ,\linebreak управ\-ля\-емый процессом Леви. Некоторые его модификации 
изучены в~работе~\cite{GarOlk2000}. Предложено рас\-смат\-ри\-вать нормальный обратный 
гауссовский и~дис\-пер\-сионный гам\-ма-про\-цесс для описания динамики его остатков. 
Распределения прираще\-ний этих процессов имеют хвосты тяжелее, чем у~нормального 
распределения, что часто встречается в~реальных данных. 

Дополнительная мотивация 
в~использовании именно этих распределений исходит из приложений в~финансах. 
Например, дисперсионное гам\-ма-рас\-пре\-де\-ле\-ние используется для моделирования цен 
акций, как это делается в~работе~\cite{Fin2009}, а~нормальное обратное 
гауссовское распределение хорошо описывает логарифмические приращения цен 
активов, например в~работе А.\,В.~Кузьминой~\cite{Kuzmina2011} это подтверждается 
на примере данных о~цене фьючерса RTS.

Для более общего механизма построения моделей к~классической модели ОУ 
добавляется линейная составляющая, или тренд. Такой подход позволяет 
моделировать большее число явлений, не выходя за рамки одной модели.

Как известно, финансовые рынки являются динамическими и~нестационарными 
системами. Поэтому отношения, связывающие различные факторы рынка, склонны 
меняться во времени. Пример данного явления продемонстрирован на рис.~\ref{rtsmixpic}. 
По оси~$x$ отложены цены фьючерса на акции компании 
<<Роснефть>> (ROSN), а~по оси~$y$~--- цены фьючерса на акции компании <<Лукойл>> 
(LKOH) с~08.01.2013 по~28.10.2016. Видно, что параметры этой зависимости 
являются также изменяющимися во времени на протяжении дня, так как можно 
отчетливо выделить области, где точки группируются в~окрестностях прямых 
с~разными параметрами.

Все это ставит задачу определения моментов, в~которые предложенная для описания 
данных модель с~определенными параметрами перестает работать, после чего процесс 
начинает следовать той же самой модели, но уже с~другими параметрами. В~данной 
статье эта проблема решена для модели\linebreak
 ОУ. Более того, предлагается процедура 
оценивания параметров  и~обнаружения смен режима в~реаль\-ном времени 
с~использованием RLS или рекурсивного метода наименьших квадратов для оценивания 
параметров, а~также алгоритм, основанный на CUSUM-про\-це\-ду\-рах для обнаружения 
смен режима. В~конце статьи предложенная процедура применяется на различных 
данных.


\section{Моделирование временного ряда}

        \subsection {Одномерный процесс Орнштейна--Уленбека}

       Процесс ОУ с~трендом, управляемый процессом Леви, 
определяется как решение стохастического дифференциального уравнения  (СДУ):       
\begin{align*}
d\left(X_t -\mu -  \nu t\right)& = -\alpha\left(X_t - \mu - \nu 
t\right) dt +  dL_{\lambda t}\,,\\ 
&\hspace{45mm}\forall\  t>0\,;  \\
X(0) &= X_0\,,  
       \end{align*}
         где $\alpha, \mu \in \mathbb{R}$; $L_t$~--- процесс Леви; $X_0$~--- 
некоторая случайная величина, независимая от~$\{L_t\}$; $\nu$ определяет 
постоянный на всем промежутке времени линейный тренд. Параметр~$\mu$ здесь 
означает долгосрочное среднее, а~$\alpha$ определяет скорость стремления 
процесса возвращаться к~своему среднему~--- тренду.

Как показано в~\cite{Protter}, данное СДУ имеет сле\-ду\-ющее решение:
\begin{multline*}
X(t) =\nu t + \mu + \exp\left(-\alpha t\right) \times{}\\
{}\times
\left(
\left(X_0 - \mu \right)+ \int\limits_0^t\exp(\alpha  s)\,dL_{\lambda s}\right)\,, 
\quad X_0 = X(0)\,,
\end{multline*}
или
\begin{multline}
X(t + \tau) =\mu +\nu  (t + \tau) +   \exp
\left(-\alpha \tau\right)\times{}\\
\!\!{}\times\! \left(\!\!
(X(t) - \mu - \nu  t) + \exp(- \alpha t)\!\int\limits_t^{t + \tau}\!\!
\exp(\alpha s)\,dL_{\lambda s}\!\right).\!\!\!\!
\label{explicit_ou}
\end{multline}
Отсюда, в~частности, следует, что $X_t$~--- марковский процесс. Еще стоит 
заметить, что данное решение единственно с~точностью до неотличимости~\cite{sato}. 
Для более подробного рассмотрения процессов Леви см.~[8, 10].

Для удобства обозначим через $Y_t\hm = X_t \hm-\mu\hm- \nu t$ соответствующий приведенный 
процесс ОУ без тренда и~имеющий нулевое среднее.

Свойство возвратности  процесса~$Y_t$ к~нулевому уровню  при $\alpha \hm> 0$ может 
быть получено из~(\ref{explicit_ou}):  если~$Y_t$ стал больше~0 в~момент 
времени~$t$, то коэффициент при~$dt$ отрицательный и~$Y_t$ будет стремиться 
немедленно вернуться к~0; аналогично происходит, если случайный процесс 
становится меньше~0.

\subsubsection{Авторегрессия и~оценка параметров}

Пусть $ X ^ * = {\left(X^*_{t_i}\right)}_ {i = 1, \ldots, N} $~--- 
наблюдения с~интервалом 
$\Delta \hm=  1$ процесса, описываемого определенной выше моделью ОУ с~трендом. 
В~дискретном случае уравнение процесса~(\ref{explicit_ou}) выглядит следующим 
образом:
        \begin{multline}
         \label{OUtrend_d}
        X_{i+1} = \mu + \mu_0\left(1 - e^{-\lambda}\right) + 
        \mu \left(1 - e^{-\lambda}\right)i+ {}\\
        {}+e^{-\lambda } X_{i} + l_i\,,
        \end{multline}
        где $l_i $~--- некоторая случайная величина с~нулевым средним.

        Соотношение~(\ref{OUtrend_d}) описывается регрессионной моделью. Запишем 
его в~виде:
        \begin{equation*} 
%        \label{OUtrend_regr}
        X_{i+1} = c + b t_i + a X_{i} + l_i\,.
        \end{equation*}

        Чтобы оценить параметры $a$, $b$ и~$c$ регрессии, можно воспользоваться 
методом наименьших квад\-ра\-тов и~получить оценки~$\hat{a}$, $\hat{b}$ и~$\hat{c}.$ 
Тогда параметры исходного процесса ОУ с~трендом можно получить 
из соотношений:
        \begin{equation*}
                \hat{\lambda} = -\fr{\ln\hat{a}}{\tau}\,;\quad
                \hat{\mu} = \fr{\hat{b}}{1-\hat{a}}\,; \quad
                \hat{\mu}_0 = \fr{\hat{c} - \hat{\mu} \tau}{1 - \hat{a}}\,.
        \end{equation*}

        Из независимости приращений также можно явно посчитать логарифмическую 
функцию правдоподобия: 
\begin{multline*}
L\left(X^*, \theta\right) =  \sum\limits_{k = 2}^n \ln 
f_{Y_i |Y_{i -1}}(X_i , \theta) = {}\\
{}=  \sum\limits_{k = 2}^n \ln 
f\left(Y_i - a Y_{i - 1} - c,\theta\right)\,,
\end{multline*}
где $\theta$~--- параметры модели.

\subsubsection{Симуляция}

     Используя соотношения авторегрессионного вида процесса ОУ, можно 
смоделировать процесс ОУ итеративно, задав некоторую начальную 
точку~$X_0$. На рис.~2 проиллюстрирован построенный 
итеративно  процесс ОУ с~положительным трендом.

{ \begin{center}  %fig2
 \vspace*{6pt}
 \mbox{%
\epsfxsize=77.781mm
\epsfbox{che-2.eps}
}
\end{center}

%\vspace*{-3pt}


\noindent
{{\figurename~2}\ \ \small{Пример процесса ОУ с~трендом ($\alpha\hm=0{,}5$, 
$\nu\hm=0{,}1$, $\mu_0\hm=0$ и~$\sigma\hm=1$)}}
}

\addtocounter{figure}{1}

\begin{table*}[b]\small
\begin{center}
\Caption{Характеристики дисперсионного гам\-ма-про\-цес\-са $V \hm= (V_t)_{t \geqslant 0}$}
\label{table1}
\vspace*{2ex}

                \begin{tabular}{|c|c|c|c|}
                        \hline
                       \tabcolsep=0pt\begin{tabular}{c}
                        Математическое\\ ожидание\end{tabular} &                         Дисперсия &
                                               Асимметрия&                         Эксцесс\\
                                               \hline
                                               &&&\\[-9pt]
                        $\theta t$  & $\left(\sigma^2 + \nu \theta^2\right) t$ 
 & $\displaystyle
                        \fr{\theta \nu \left(3 \sigma^ 2 + 2 \nu 
\theta^2\right)}{t^{{1}/{2}}} \left(\sigma^2 + \nu \theta ^2\right)^{{3/}{2}}$ 
 & $\displaystyle 3 \left( 1 + \fr{2 \nu}{t} - \nu \theta^4 
t \left(\sigma^2 + \nu \theta^2\right)^{-2} \right) $ \\
                        \hline
                \end{tabular}
        \end{center}
%\end{table*}
%\begin{table*}\small
\begin{center}
\Caption{Характеристики нормального обратного гауссовского распределения}
\label{table2}
\vspace*{2ex}

                \begin{tabular}{|c|c|c|c|}
                        \hline
                        Математическое ожидание & 
                                                Дисперсия &
                                                                        Асимметрия &
                                    Эксцесс \\
 \hline
 &&&\\[-9pt]
 $\mu + \delta \tau$ &
 $\displaystyle\fr{\delta^2(1 + \tau^2)}{\xi}$ 
& $\displaystyle\fr{3}{\tau \sqrt{\xi (1 + \tau^2)}}$ 
& $\displaystyle\fr{3}{\xi} \left(  1 + 4 \fr{\tau^2}
                        {1 + \tau^2}  \right)$  \\[8pt]
                        \hline
                \end{tabular}
        \end{center}
\end{table*}
        

\subsection{Частные случаи моделирования остатков процесса Орнштейна--Уленбека}

В работе~\cite{taufer} авторы приводят быстрые и~эффективные методики для 
симуляции различных ОУ-про\-цес\-сов, управляемых процессом Леви, а~также  
описание множества различных частных его случаев. Будем рассматривать три типа 
процессов Леви: винеровский процесс, дисперсионный гам\-ма-про\-цесс (VG), а~также 
нормальный обратный гауссовский процесс (NIG).  Оба последних процесса 
моделируют тяжелые хвосты и~принадлежат классу обобщенных гиперболических 
распределений. Они часто применяются в~финансах и~эконометрике (для 
VG см.~[12,  13]), для NIG см.~[10, 14--16]).

\subsubsection {Дисперсионный гамма-процесс}

\noindent
\textbf{Определение 2.1.}\
Случайная величина~$\xi$ имеет дисперсионное гам\-ма-рас\-пре\-де\-ле\-ние, 
если ее плотность распределения имеет вид:
\begin{multline} \label{vgpdf}
 f_\xi(x) = \int\limits_{0}^{\infty} \fr{1}{\sigma  \sqrt{2 \pi g}} 
 \exp \left( - \fr{(x - \theta g)^2}{2 \sigma^2 g}  \right) \times{}\\
 {}\times
\fr{g ^{{1}/{\nu} - 1} \exp \left( - {g}/{\nu} \right)}
{\nu  ^{{1}/{\nu}} \Gamma \left({1}/{\nu}\right) }\, dg\,,\enskip x \in \mathbb{R},
                \end{multline}
                где $\Gamma (x)$, $x\hm>0$,~--- гам\-ма-функ\-ция, 
                а~$\sigma \hm> 0$, $\nu \hm> 0$,  
$\theta \hm\in \mathbb{R}$.


        Обозначение: $\xi \sim V(\sigma, \nu, \theta)$.

\smallskip

\noindent
\textbf{Определение~2.2.}\
        Случайный  процесс  $V \hm= (V_t)_{t \hm\geqslant 0} $  с~ параметрами        
$\sigma\hm >0$, $\nu \hm> 0$, $\theta \hm\in \mathbb{R} $, заданный на вероятностном 
пространстве $ (\Omega, F, \mathbb{P}) $ со
        значениями в~$ \mathbb{R}$, называется дисперсионным гам\-ма-про\-цес\-сом, 
если $V_0\overset{\mathrm{p.n.}}{=} 0$, $V$ имеет независимые приращения и~для любых 
$s \hm\geqslant 0$, $t \hm\geqslant 0$ 
$V$ имеет стационарные приращения с~дисперсионным 
гам\-ма-рас\-пре\-де\-ле\-ни\-ем~(\ref{vgpdf}) с~параметрами $\sigma \sqrt{t}\hm > 0$, 
$\nu/t \hm> 0$ и~$t \theta\hm > 0$,~т.\,е.\
$$
V_{t+s} - V_s \overset{\mathrm{d}}{=} V_t - V_0 \sim 
V\left(\sigma \sqrt{t}, \fr{\nu}{t}, t \theta\right) \,.
$$

\smallskip

        Характеристики дисперсионного гам\-ма-про\-цес\-са $V \hm= (V_t)_{t \geqslant 0}$ 
с параметрами~$\sigma$, $\nu$ и~$\theta$ представлены в~табл.~1.


       
        В работе~\cite{madancarr} показано, что плотность дисперсионного 
        гамма-процесса  $V \hm= (V_t)_{t \geqslant 0}$ выражается аналитически с~использованием 
модифицированной функции Бесселя второго рода с~индексом~$\nu$.

        \subsubsection {Нормальный обратный гауссовский процесс}

\noindent
\textbf{Определение 2.3.}\  Случайная величина~$\eta$ имеет 
нормальное обратное гауссовское распределение с~параметрами~$\alpha$, $\beta$, 
$\delta$ и~$\mu$ ($\eta \hm\sim \mathrm{NIG}\,(\alpha, \beta, \delta, \mu)$), если ее плотность 
распределения имеет вид:
                \begin{multline*} 
%                \label{nigpdf}
                   \hspace*{-3mm}f_\eta(x, \alpha, \beta, \delta, \mu) = \fr{\alpha 
\delta}{\pi} \exp \left(\delta \sqrt{\alpha^2 - \beta^2} + \beta (x - \mu)\right) 
\times{}\\
{}\times \fr{K_1\left(\alpha \sqrt{\delta^2 + (x - \mu)^2}\right)}{\sqrt{\delta ^2 + (x - 
\mu)^2}}\,,
                \end{multline*}
                где $K_1(z) = (1/2) \int\nolimits_{0}^{\infty} \exp 
                (-({1}/{2}) z (u \hm+ u^{-1}))\,du$, $z\hm>0$,~--- 
                модифицированная функция Бесселя 
второго рода с~индексом~1, $\alpha \hm> 0$, $-\alpha \hm< \beta \hm< \alpha$, 
$\delta \hm> 0$,  $\mu \hm\in  \mathbb{R}$, $x\hm>0$.

                Параметры $\alpha$, $\beta$, $\delta$ и~$\mu$ являются параметрами 
формы, асимметрии, масштаба и~расположения соответственно. 

\smallskip

\noindent
\textbf{Определение 2.4.}\
                Случайный процесс $N \hm= (N_t)_{t \geqslant 0}$ с~параметрами 
$\alpha$, $\beta$, $\delta$ и~$\mu$, заданный на вероятностном пространстве $(\Omega, 
F, P)$ со значениями  в~$\mathbb{R}$,\linebreak
 называется нормальным обратным гауссовским 
процессом, если $N_0 \overset{\mathrm{p.n.}}{=} 0$, $N$ имеет независимые приращения 
и~для любых $s \hm\geqslant 0$, $t \hm\geqslant 0$ $N$ имеет стационарные приращения 
с~нормальным обратным гауссовским распределением:
$$
    N_{t+s} - N_s  \overset{\mathrm{d}}{=} N_t - N_0 \sim 
\mathrm{NIG}\,( \alpha, \beta, \delta t, \mu t) 
  $$
        с~параметрами $\alpha \hm> 0$, 
        $-\alpha \hm< \beta \hm< \alpha$, $\delta t\hm > 0$ 
и~$\mu t \hm\in  \mathbb{R}$.

\smallskip

    Плотность нормального обратного гауссовского распределения может быть 
представлена в~аналитической форме.

        Характеристики нормального обратного гауссовского распределения 
представлены в~табл.~\ref{table2}.



\section {Оценивание параметров и~поиск смен режима в~реальном времени}

В~данной разделе рассмотрено моделирование и~описание данных при условии их 
поступления в~режиме реального времени, когда значения выборки данных поступают 
одно за другим. Специфика данной задачи заключается в~высокой скорости 
поступления данных в~ее приложениях и~их большом объеме, поэтому любые 
приводимые алгоритмы должны быть достаточно быстрыми и~эффективно использовать 
компьютерную память.

\subsection{Оценивание параметров}

Без каких-ли\-бо ограничений на компьютерные мощности самым очевидным решением для 
оценивания параметров было бы на каждом шаге использовать метод наименьших 
квадратов (OLS). Чтобы удовлетворить необходимость обрабатывать потоковые 
данные, воспользуемся рекурсивным методом наименьших квадратов (RLS). Данный 
алгоритм на каждом шаге обновляет рекурсивно оценку параметра~$\theta$ 
и~ковариационную матрицу~$X^{\mathrm{T}} X$  вмес\-то того, чтобы насчитываться с~нуля каждый 
раз. Данный алгоритм и~его реализация хорошо известны и~могут быть найдены 
в~\cite{haykin}.

\subsection{Постановка простейшей смены режима}

        В реальной жизни некоторые явления могут быть связаны отношениями, 
например линейными, параметры которых изменяются во времени. Самым простым 
подобным примером является процесс, описываемый следующим образом:      
 \begin{equation*}
        M_t= 
                        \begin{cases}
                M_t^1\,, &\ t \leqslant t^*\,; \\
                M_t^2\,, &\ t > t^*\,,
                \end{cases}
        \end{equation*}
        где $t \in [1,\ldots,T]$ обозначает время; $ t^*$~--- критическое 
значение внешней переменной~$t$, или момент смены режима (change point, regime 
switch);  $M^{1,2}$~--- это две различные модели, соответствующие разным 
временным промежуткам: до и~после.
        В общем случае нельзя точно определить значение~$t^*$. Задача состоит 
        в~том, чтобы наилучшим образом оценить ее значение, имея на входе выборку 
наблюдений, при условии что на данном временн$\acute{\mbox{о}}$м промежутке произошла ровно одна 
смена режима (рис.~3), а~также оценить параметры старой и~новой модели.   
В~данном случае 
рассматривается модель ОУ и~исследуемый процесс выглядит 
следующим образом:

\columnbreak

\noindent
 \begin{center}  %fig1
 \vspace*{-2pt}
 \mbox{%
\epsfxsize=77.781mm
\epsfbox{che-3.eps}
}

\vspace*{3pt}

\noindent
{{\figurename~3}\ \ \small{Пример смены режима}}
\end{center}



 \vspace*{6pt}



\addtocounter{figure}{1}


\noindent
\begin {equation*}
        X_t =                \begin{cases}
                \mathrm{OU}_t^1\,, &\ t \leqslant t^*\,; \\
                \mathrm{OU}_t^2\,, &\ t > t^*\,,
                \end{cases}
        \end {equation*}
        где $\mathrm{OU}^i$~--- процессы ОУ, описанные выше.

        

\subsection{Постановка задачи для~потоковых данных}

            В случае потока данных значения выборки $X_1,X_2,\ldots,X_n, \ldots$ 
поступают последовательно. В~этом случае нет предпосылок для того, чтобы 
в~ка\-кой-то момент произошла смена режима, а~сами смены могут происходить 
последовательно много раз. Задача состоит в~том, чтобы последовательно их 
обнаруживать и~предоставлять оценку для параметров новой модели. Эффективность 
методов определяется тем, как часто метод ошибочно определяет режимы и~как 
быстро он способен обнаруживать смену режима.

\subsection{Решение задачи}

В современной литературе можно найти множество методов  для определения смен 
режима. Основным применяемым подходом является по\-стро\-ение по наблюдаемой системе  
некоторого детектора (change-detector), который сигнализирует, когда параметры 
модели перестают соответствовать выборке наблюдений и~предположительно сменился 
режим. Одним из таких методов  является CUSUM-тест, или метод кумулятивных сумм, 
который рассматривается в~данной работе для обнаружения смен режима. Стоит 
отметить другие известные методы, такие как метод GLT (generalized likehood 
test)~\cite{appel} и~MLT (marginalized likelihood text)~\cite{gustafsson}.

\subsubsection{Краткое описание CUSUM-методов}

В данной статье будут рассмотрены два базовых CUSUM-ме\-то\-да: CUSUM-ме\-тод, 
основанный на максимизации правдоподобия выборки наблюдений, и~CUSUM-ме\-тод, 
определяющий смену среднего значения выборки наблюдений.

\subsubsection*{CUSUM-метод максимизации правдоподобия}

Пусть есть некоторый поток данных $X^* \hm= X^*_1,\ldots X^*_n,\ldots $, 
элементы которого 
являются выборкой независимых одинаково распределенных случайных величин. 
Обозначим плотность соответствующей случайной величины через $p_\theta(x).$ 
Предполагается, что в~ка\-кой-то момент времени~$r^*$ может произойти смена режима 
модели. Это означает, что до~$r^*$ в~модели действует параметр~$\theta_0$, 
а~после~---~$\theta_1$. Введем соответствующие гипотезы: гипотезу <<неизменности>> 
модели~$H_0$ (разладки не произошло) и~гипотезу~$H_1$ об <<одноразовой 
разладке>>. Одним из самым известных и~простых методов для нахождения разладки 
модели является тест отношения правдоподобий~\cite{kay}.

\noindent
\textbf{Алгоритм 3.1.}\ %\begin{algorithm}
 Определим логарифмическое отношение правдоподобий для моделей~$H_0$ и~$H_1$: 
\begin{equation*}
\mathrm{LLR} \left(X, r^*\right) = 
\ln\fr{p_{H_1}(X)}{p_{H_0}(X)}\,.
\end{equation*} 
Тогда  принимается гипотеза~$H_1$, если $\mathrm{LLR} \hm>h$, где параметр~$h$  
отвечает за чувствительность 
алгоритма: чем меньше~$h$, тем быстрее будет происходить обнаружение разладки, 
но при этом тем больше будет срабатывать ложных сигналов.

\smallskip

К~сожалению,  в~данном случае неизвестно значение~$r^*$ и~поэтому явно посчитать 
$p_{H_1}(X)$ не представляется возможным. Данную проблему можно решить, перебрав 
все значения~$r^*$ и~взяв то, которому соответствует максимальное значение~$\mathrm{LLR}$. 
Данный метод называется обобщенным методом максимального 
правдоподобия (GLT).


\smallskip

\noindent
\textbf{Алгоритм 3.2.} %\begin{algorithm}\label{algo2}
Определим обобщенное логарифмическое отношение 
правдоподобий для выборки размера~$N$: 
\begin{multline*}
\mathrm{GLLR}(X) =\max\limits_{1   \leqslant r^* \leqslant N } 
\mathrm{LLR}(X, r^*)  ={}\\
{}=\max\limits_{1  \leqslant r^* \leqslant N } 
\ln\fr{p_{H_1}(X)}{p_{H_0}(X)}  =\max\limits_{1  \leqslant r^* \leqslant N } 
\sum\limits_{i = r^*}^N \ln \fr{p_{\theta_1}(X_i)}{p_{\theta_0}(X_i)}\,. 
%\label{gllr} 
\end{multline*}
Тогда на каждом шаге~$n$ принимается гипотеза~$H_1$  против гипотезы~$H_0$, если 
$\mathrm{GLLR}(X)\hm>h.$

\smallskip

Введем  кумулятивную сумму точечных отношений правдоподобий:
\begin{equation*}
S(n) = \sum\limits_{i = 1}^n \ln  
\fr{p_{\theta_1}(X_i)}{p_{\theta_0}(X_i)}\,. 
\end{equation*}
Тогда 
\begin{align*}
\mathrm{LLR} (X, r^*) &= S(N) - S(r^*)\,; \\
\mathrm{GLLR}(X, N) &= S(N) - \min\limits_{1  \leqslant r^* \leqslant N  }S(r^*)\,,
\end{align*}
где
$$
\hat{r}^* = \mathop{\mathrm{argmin}}\limits_{1  \leqslant r^* \leqslant N}S(r^*)\,.
$$
Заметим, что для того чтобы использовать алгоритм~3.2, достаточно 
считать кумулятивную сумму~$S$. Более того, так как уровень~$h$ берется 
положительным, вместо того, чтобы явно насчитывать значение~$\mathrm{GLLR}$ на 
каждом шаге, достаточно рекурсивно считать функцию 
\begin{equation*}
 G(N)  =\max\left(G(N- 1) +\ln \fr{p_{\theta_1}(X_N)}{p_{\theta_0}(X_N)}\,, 
\;0\right),
\end{equation*}
которая совпадает с~$\mathrm{GLLR}$ там, где последняя положительна. В~этом 
и~заключается метод кумулятивных сумм. Из вышесказанного вытекает следующий 
алгоритм, эквивалентный алгоритму~3.2:

\smallskip

\noindent
\textbf{Алгоритм 3.3.} %\begin{algorithm}
На каждом шаге~$N$ принимается гипотеза~$H_1$ против гипотезы~$H_0$, если 
$G(N)\hm>h.$


\smallskip

На практике значение~$\theta_0$ можно оценить, а~значение~$\theta_1$ неизвестно. 
Поэтому  берут $\theta_1 \hm= \theta_0 \hm+ \delta$, где~$\delta$~--- минимальная 
величина, изменение которой хотят детектировать.

\subsubsection*{CUSUM-метод определения смены среднего выборки}

%\smallskip

\noindent
\textbf{Алгоритм 3.4.} %\begin{algorithm}
Определим рекурсивно 
\begin{equation*} 
S_n^+=\max\left (S_{n- 1}^+ + 
\fr{X_N - \mu_0}{\sigma} - k, \;0\right)\,, 
\end{equation*}
где $\mu_0$~--- среднее текущей модели; $\sigma$~--- среднее текущей выборки; $k$~--- 
уровень чувствительности метода к~разбросам. Тогда считается, что принимается 
гипотеза~$H_1$, если $S^+_n \hm> h.$

\smallskip

Подробнее с~алгоритмом можно 
ознакомиться, например, в~\cite{page61}.

\begin{figure*}[b] %fig4
 \vspace*{1pt}
 \begin{minipage}[t]{79mm}
\begin{center}
\mbox{%
\epsfxsize=77.835mm
\epsfbox{che-5.eps}
}
\end{center}
\vspace*{-9pt}
  \Caption{Частичная автокорреляционная функция}\label{pacfpic}
  \end{minipage}
%\end{figure*}
\hfill
%\begin{figure*} %fig5
        \vspace*{1pt}
         \begin{minipage}[t]{79mm}
\begin{center}
\mbox{%
\epsfxsize=78.035mm
\epsfbox{che-4.eps}
}
\end{center}
\vspace*{-9pt}
  \Caption{Автокорреляционная функция}\label{acfpic}
    \end{minipage}
\end{figure*}

\subsection{Общий алгоритм для процесса Орнштейна--Уленбека}

Как было замечено ранее, приведенный процесс ОУ с~трендом обладает 
авторегрессионным свойством AR(1). Поэтому можно применять CUSUM-ме\-тод 
максимального правдоподобия для приращений~$l_i$ данного процесса. Данный метод 
будем\linebreak
 применять для детектирования смены во\-ла\-тиль\-ности модели, в~то время как 
для обнаружения смены среднего, или тренда, будем применять CUSUM-ме\-тод поиска 
смены среднего. Применить первый алгоритм для детектирования среднего 
оказывается сложно из-за большого числа параметров, которые нельзя адекватно 
оценить, в~частности па\-ра\-мет\-ров~$\mu_0$ и~$\nu$.

Таким образом, общий алгоритм следующий:

\smallskip

\noindent
\textbf{Алгоритм 3.5.} %%rithm}

\noindent
\begin{enumerate}[1.]
\item На каждом шаге оцениваем наиболее вероятные параметры выборки~$\theta_0$ 
для выбранной модели с~помощью метода RLS.
\item На каждом шаге считаем значения детекторов CUSUM смены волатильности 
и~смены тренда.
\item В случае, когда детекторы сигнализируют\linebreak о~смене режима, проходимся общим 
методом обобщенного отношения правдоподобий (GLT) по выборке и~находим наиболее 
вероятную точку смены режима~$r^*$ с~оценкой параметров~$\theta_1$. Далее 
исключаем из выборки все ее элементы до~$r^*$ и~продолжаем процедуру алгоритма 
с~$\theta_0:=\theta_1$.
\end{enumerate}


\section{Анализ данных}

       В этом разделе описывается моделирование процессом ОУ 
реальных финансовых данных. В~качестве данных были выбраны секундные данные по 
ценам фьючерсов на индекс RTS ($x_t$) и~на акции компании <<Газпром>> ($y_t$) 
с~MOEX за~07.10.2014. Если наблюдения сделаны через равные промежутки времени, то 
можно рассматривать их как временной ряд.
Предполагается, что разность $z_t \hm= x_t \hm - 6  y_t$ обладает свойством 
стационарности и~может быть описана с~помощью процесса ОУ. Для 
того чтобы ряд имел свойство авторегрессии, вычитаем из ряда его скользящее 
среднее с~периодом 5~мин.

Тест Дики--Фуллера (с уровнем зна\-чи\-мости 0,05) подтверждает предположение 
о~стационар\-ности: значение статистики Ди\-ки--Фул\-ле\-ра: $-25{,}374$; $p$-зна\-че\-ние: 
0,001. Тест  отвергает нулевую гипотезу о существовании единичного корня 
с~уровнем значимости~0,05, что подтверждает стационарность данного ряда.  Для 
проверки наличия свойства AR(1) проанализируем вид автокорреляционной (ACF) 
и~частичной автокорреляционной (PACF) функций.

         Для модели AR(1) характерен следующий вид автокорреляционных 
         и~частичных автокорреляционных функций: график ACF экспоненциально убывает, 
         а~график PACF имеет пик при значении сдвига, равном~1, и~практически равен~0 при 
значениях сдвига более высокого порядка.

       
       На рис.~\ref{pacfpic} изображен график PACF для рас\-смат\-ри\-ва\-емых данных. 
Заметно, что он имеет пик при сдвиге~1 и~практически равен нулю для сдвигов 
более высокого порядка.
        На рис.~\ref{acfpic} ACF для исходных данных убывает экспоненциально.
        Такое поведение графиков ACF и~PACF соответствует модели AR(1).

        Теперь можно говорить о том, что данные представляют собой процесс 
ОУ, и~перейти к~оценке его параметров.

        Для начала оценим параметры~$\theta$ и~$\alpha$ с~по\-мощью метода 
наименьших квадратов, получаем оценки параметров $\hat \theta \hm= 0$ и~$\hat \alpha 
\hm= 0{,}7506$.


        Для того чтобы оценить качество полученных оценок для данного процесса, 
построим QQ-плот для оцененных параметров нормального распределения для остатков 
(рис.~\ref{qqplot}). Это график, где по оси~$x$~--- квантили теоретического 
распределения, а~по оси~$y$~--- эмпирические квантили данных. Если теоретическое 
распределение хорошо описывает\linebreak\vspace*{-12pt}

\pagebreak

\end{multicols}

\begin{figure*} %fig6
        \vspace*{1pt}
\begin{center}
\mbox{%
\epsfxsize=161.601mm
\epsfbox{che-6.eps}
}
\end{center}
\vspace*{-11pt}
\Caption{Графики QQ-plot для оценивания параметров}\label{qqplot}
\vspace*{-3pt}
\end{figure*}

\begin{multicols}{2}

\noindent
 реальные данные, то график <<кван\-тиль--кван\-тиль>> 
близок к~прямой $y \hm= x$.


        Видно, что нормальное распределение не очень\linebreak хорошо описывает 
распределение остатков. По\-пробуем вместо нормального распределение\linebreak использовать 
распределение с~более тяжелыми хвостами, например дисперсионное 
гам\-ма-рас\-пре\-де\-ле\-ние и~нормальное обратное гауссовское распределение. Для оценки 
параметров этих распределений используем метод максимального правдоподобия, 
описанный в~п.~3.2.3.

       

        Анализируя графики QQ-plot (см.\ рис.~\ref{qqplot}) для оцененных параметров 
дисперсионного гамма- и~нормального обратного гауссовского распределений для 
остатков, можно прийти к~выводу, что дисперсионное гамма- и~нормальное обратное 
гауссовское распределение лучше описывают структуру независимых приращений 
в~процессе~ОУ.

        Для того чтобы оценить качество полученных результатов, применим критерий 
согласия Колмогорова для новой выборки данных. Результат подсчета статистики 
представлен в~табл.~3.



        По результатам, представленным в~табл.~3, \mbox{можно} заключить, 
что гипотеза о нормальном рас\-пре\-делении остатков отвергнута при уровне 
зна\-чи\-мости~0,01, гипотеза о~дисперсионном гам\-ма-рас\-пре\-де\-ле\-нии остатков 
и~нормальном обратном гауссовском распределении остатков принята при уровне 
значимости~0,01.

\vspace*{12pt}

\noindent
{{\tablename~3}\ \ \small{Оценки параметров с~результатом критерия Колмогорова}}

\vspace*{1pt}

{\small
 \begin{center}  %
\tabcolsep=3pt
                        \begin{tabular}{|c|c|c|c|c|}
                                \hline
                                \multicolumn{3}{|c|} {Оценка параметра} &
\tabcolsep=0pt\begin{tabular}{c} Значение\\ статистики \end{tabular}&  
$p$-значение  \\
                                \hline
\multicolumn{1}{|c|}{\raisebox{-6pt}[0pt][0pt]{$N(\mu, \sigma^2)$}}
                                & $\hat \mu$ & 0 &  &   \\ 
                              %  \cline{2-3}
                                & $\hat \sigma$ & 
3,6885 &
\multicolumn{1}{c|}{\raisebox{6pt}[0pt][0pt]{ 0,087731}} & 
\multicolumn{1}{c|}{\raisebox{6pt}[0pt][0pt]{$1{,}6679\cdot 10^{-12}$}}\\
                                \hline
                                & $\hat \sigma$ & 3,6712 &  &   \\ 
                                %\cline{2-3}
$\mathrm{VG}\,(\sigma, \nu, \theta)$& $\hat \nu$ & 1,5226 & 0,026806 &   0,14799  \\ 
%\cline{2-3}
                                & $\hat \theta$ & 0,0379 &  & 
\\
                                \hline
\multicolumn{1}{|c|}{\raisebox{-18pt}[0pt][0pt]{$\mathrm{NIG}\,(\theta, \xi, \delta, \mu)$}}
                                & $\hat \theta$ & 0,0592 &  &   \\ 
                                %\cline{2-3}
                                & $\hat \xi$ & 0,3714 &  &   \\ 
                                %\cline{2-3}
                                & $\hat \delta$ & 2,2690 &  &   \\ 
                                %\cline{2-3}
                                & $\hat \mu$ & $-$0,0963 & 
 \multicolumn{1}{c|}{\raisebox{18pt}[0pt][0pt]{0,034654}} & 
 \multicolumn{1}{c|}{\raisebox{18pt}[0pt][0pt]{0,025953}}  
\\
                                \hline
                        \end{tabular}
                        \vspace*{3pt}
\end{center}
}

\pagebreak

\end{multicols}

 \begin{figure*} %fig7
         \vspace*{1pt}
         \begin{minipage}[t]{80mm}
\begin{center}
\mbox{%
\epsfxsize=78.067mm
\epsfbox{che-7.eps}
}
\end{center}
\vspace*{-9pt}
        \Caption{Пример применения CUSUM-алгоритма}
        \label{cusum_vola}
%        \end{figure*}
\end{minipage}
\hfill
%        \begin{figure*} %fig8[H]
                 \vspace*{1pt}
                          \begin{minipage}[t]{80mm}
\begin{center}
\mbox{%
\epsfxsize=78.067mm
\epsfbox{che-8.eps}
}
\end{center}
\vspace*{-9pt}
               \Caption{Пример применения CUSUM-алгоритма}
        \label{cusum_mean}
        \end{minipage}
        \end{figure*}

\begin{multicols}{2}

        Полученный результат говорит о том, что структура реальных данных 
сложнее, чем может описать классический процесс ОУ. 
Целесообразнее использовать обобщенный процесс ОУ, где процесс 
броуновского движения заменен на процесс Леви.

        \subsection{Применение алгоритма для~детектирования изменения 
волатильности}

       Будем рассматривать гауссовский процесс ОУ. Для этого 
были сгенерированы две выборки процесса размера~100 с~$\sigma_1\hm=1$ и~$\sigma_2\hm=3.$ 
Для CUSUM-тес\-та будем брать $\theta_1\hm=2$, т.\,е.\ $\delta\hm=1$. Уровень $h\hm=45.$ 
Результат применения алгоритма проиллюстрирован на рис.~7. На 
рис.~7,\,\textit{а} изображена выборка сгенерированного процесса ОУ со сменой 
режима, в~то время как на рис.~7,\,\textit{б} построено значение CUSUM-де\-тек\-то\-ра. Сплошная 
вертикальная линяя обозначает фактическую смену режима, а~пунктирная~--- время ее 
обнаружения. Заметим, что смена режима могла бы быть обнаружена быстрее при 
другом выборе уровня~$h$.

        \subsection{Применение алгоритма для~обнаружения тренда}

        Для обнаружения тренда также были сгенерированы две выборки гауссовского 
процесса ОУ, которые потом были склеены. Параметры процессов 
следующие: $\alpha_0\hm=0{,}5$, $\nu_0\hm=0$, $\mu_0^0\hm=0$, 
$\sigma_0\hm=0{,}6$, $\alpha_1\hm=0{,}5$, 
$\nu_1\hm=0{,}05$, $\mu_0^1\hm=0$ и~$\sigma_1\hm=0{,}6$. 
Аналогично построена выборка и~значения 
CUSUM-де\-тек\-то\-ра. Уровень $h\hm=13.$ Алгоритм успешно определил смену режима 
(рис.~\ref{cusum_mean}).
       



\section {Заключение}

В статье рассмотрен процесс ОУ с~трендом, управ\-ля\-емый процессом 
Леви, для описания финансовых временн$\acute{\mbox{ы}}$х рядов.
На реальных данных было показано, что дисперсионный гамма- и~нормальный обратный 
гауссовский процессы в~качестве процесса Леви способны гораздо точнее описывать 
реальные явления. Также были рас\-смот\-ре\-ны проб\-ле\-мы разладки модели и~поиска смены 
режима в~реальном времени. Была представлена процедура обнаружения разладки 
модели, а~также определения параметров новой модели. Данный алгоритм способен 
детектировать многократные смены режима последовательно, сохраняя текущую модель 
актуальной для текущего потока данных.

{\small\frenchspacing
 {%\baselineskip=10.8pt
 \addcontentsline{toc}{section}{References}
 \begin{thebibliography}{99}
\bibitem{brigo2007}
    \Au{Brigo D., Dalessandro~A., Neugebauer~M., Triki~F.} A~stochastic 
processes toolkit for risk management.~--- London: King's College 
London, November 2007.  Working paper. 48~p.


\bibitem{ou1930}
\Au{Ornstein L.\,S., Uhlenbeck~G.\,E.} On the theory of the Brownian motion~// 
Phys. Rev., 1930. Vol.~36. No.\,5. P.~823.
    
    \bibitem{vasicek1977}
    \Au{Vasicek O.} An equilibrium characterization of the term structure~// 
J.~Financ. Econ., 1977. Vol.~5. P.~177.

\bibitem{cox1985}
        \Au{Cox J.\,C., Ingersoll E., Jr., Ross~S.\,A.} A~theory of the term 
structure of interest rates~//  Econometrica, 1985. Vol.~53. No.\,2.  P.~385--407.



\bibitem{GarOlk2000}
    \Au{Garbaczewski P., Olkiewicz~R.} Ornstein--Uhlenbeck--Cauchy process~// 
J.~Math. Phys., 2000. Vol.~41. P.~6843.

\bibitem{Fin2009}
    \Au{Finlay R.} The variance gamma (VG) model with long range dependence: 
A~model for financial data incorporating long range dependence in squared 
returns.~--- Sydney, Australia: University of Sydney, School of 
Mathematics and Statistics, 2009. PhD Thesis. 144~p.

\bibitem{Kuzmina2011}
    \Au{Кузьмина А.\,В.} Моделирование нормального обратного гауссовского 
процесса и~оценивание его параметров~// Информатика, 2011. №\,2. С.~133--136.

    \bibitem{Protter}
    \Au{Protter P.} Stochastic integration and differential equations.~--- 
Heidelberg: Springer-Verlag, 1990. 415~p.
    
           \bibitem{sato}
    \Au{Sato K.\,I.} L$\acute{\mbox{e}}$vy processes and 
    infinitely divisible distributions.~--- Cambridge: Cambridge University Press, 1999.
    500~p.
    
        \bibitem{nielsen}
\Au{Barndorff-Nielsen O.\,E., Shephard~N.} Non-Gaussian Ornstein--Uhlenbeck-based 
models and some of their uses in financial economics~// 
J.~Roy. Stat. Soc. B, 2001. Vol.~63. P.~167--241.
    
    \bibitem{taufer}
\Au{Taufer E., Leonenko~N.} Simulation of L$\acute{\mbox{e}}$vy-driven 
Ornstein--Uhlenbeck processes with given marginal distribution~// 
Comput. Stat. Data An., 2008. Vol.~53.  P.~2427--2437.

\bibitem{madanseneta}
    \Au{Madan D.\,B., Seneta~E.} The VG model for share market returns~// 
    J.~Bus., 1990. Vol.~63. P.~511--524.

\bibitem{madancarr}
\Au{Madan D.\,B., Carr P.\,P., Chang~E.\,C.} The variance gamma 
process and option pricing~// Eur. Finance Rev., 1998. Vol.~2. P.~79--105.

\bibitem{nielsen2}
 \Au{Barndorff-Nielsen O.\,E.} Normal inverse Gaussian distributions and 
stochastic volatility modelling~// Scand. J.~Stat., 1997. Vol.~24. No.\,1. P.~1--13.
    
   
    
    \bibitem{rydberg}
\Au{Rydberg H.} The Normal inverse Gaussian L$\acute{\mbox{e}}$vy process: Simulation 
and approximation~// Commun. Stat. Stochastic Models, 1997. 
Vol.~13. No.\,4. P.~887--910.

 \bibitem{nielsen3}
   \Au{Barndorff-Nielsen O.\,E.} Processes of normal inverse Gaussian type~// 
Financ. Stoch., 1998. Vol.~2. P.~41--68.

\bibitem{haykin}
    \Au{Haykin S.} Adaptive filter theory.~--- 3rd ed.~--- Upper Saddle River, NJ, USA: 
Prentice Hall, 1996. 989~p.

\bibitem{appel}
\Au{Appel U., Brandt~A.\,V.} Adaptive sequential segmentation of 
piecewise stationary time series~// Inform. Sci., 1983. Vol.~29. P.~27--56.

\bibitem{gustafsson}
\Au{Gustafsson F.} The marginalized likelihood ratio test for detecting 
abrupt changes~// IEEE Trans. Automat. Contr., 1996. Vol.~41. P.~66--78.
    
    \bibitem{kay}
    \Au{Kay S.} Fundamentals of statistical signal processing. Vol.~I. 
Estimation theory.~--- Upper Saddle River, NJ, USA: Prentice Hall, 1993. 625~p.

\bibitem{page61}
   \Au{Page E.\,S.} Cumulative sum control chart~// Technometrics, 1961. 
Vol.~3. P.~1--9.
 \end{thebibliography}

 }
 }

\end{multicols}

\vspace*{-6pt}

\hfill{\small\textit{Поступила в~редакцию 20.10.16}}

\vspace*{8pt}

%\newpage

%\vspace*{-24pt}

\hrule

\vspace*{2pt}

\hrule

\vspace*{8pt}


\def\tit{REGIME SWITCHING DETECTION FOR~THE~LEVY DRIVEN ORNSTEIN--UHLENBECK PROCESS 
USING CUSUM METHODS}

\def\titkol{Regime switching detection for the Levy driven Ornstein--Uhlenbeck process 
using CUSUM methods}

\def\aut{A.\,V.~Chertok$^{1,2}$, A.\,I.~Kadaner$^{2,3}$, G.\,T.~Khazeeva$^1$, 
and~I.\,A.~Sokolov$^4$}

\def\autkol{A.\,V.~Chertok, A.\,I.~Kadaner, G.\,T.~Khazeyeva, 
and~I.\,A.~Sokolov}

\titel{\tit}{\aut}{\autkol}{\titkol}

\vspace*{-9pt}

\noindent
$^1$Faculty of Computational Mathematics and Cybernetics, 
M.\,V.~Lomonosov Moscow State University, 1-52~Lenin-\linebreak
$\hphantom{^1}$skiye Gory, GSP-1, 
Moscow 119991, Russian Federation

\noindent
$^2$Sberbank of Russia, 19~Vavilov Str., Moscow 117999, Russian Federation

\noindent
$^3$Faculty of Mechanics and Mathematics, 
M.\,V.~Lomonosov Moscow State University, Main Building, Leninskiye\linebreak
$\hphantom{^1}$Gory, 
GSP-1, Moscow 119991, Russian Federation

\noindent
$^4$Federal Research Center ``Computer Science and Control'' 
of the Russian Academy of Sciences, 44-2~Vavilov\linebreak 
$\hphantom{^1}$Str., Moscow 119333, 
Russian Federation


\def\leftfootline{\small{\textbf{\thepage}
\hfill INFORMATIKA I EE PRIMENENIYA~--- INFORMATICS AND
APPLICATIONS\ \ \ 2016\ \ \ volume~10\ \ \ issue\ 4}
}%
 \def\rightfootline{\small{INFORMATIKA I EE PRIMENENIYA~---
INFORMATICS AND APPLICATIONS\ \ \ 2016\ \ \ volume~10\ \ \ issue\ 4
\hfill \textbf{\thepage}}}

\vspace*{3pt}



\Abste{The article considers using a trending Ornstein--Uhlenbeck process, driven 
by a~Levy process, for modeling financial time series. The authors demonstrate 
that the Levy driven model gives more flexibility to describe financial time series 
than the simple classical model. In particular, the Levy driven model allows 
modeling distributions with heavy tails, which is a~common property of time series 
in real applications. The authors describe efficient methods for estimating model 
parameters using such methods as OLS (ordinary least squares)
and RLS (regularized least squares). The article also solves the regime 
switching problem in a~real time data stream. The authors built an algorithm based 
on CUSUM (CUmulative SUM) methods that is capable of determining regime switches consecutively as 
they happen online and keep model parameters up to date. Solution of the regime 
switching problem is important in real applications, since the dynamics of real 
systems tend to change over time under the influence of external factors.} 

\KWE{random process; mean-reverting process; Ornstein--Uhlenbeck process driven 
by Levy process; trending Ornstein--Uhlenbeck process; regime switch; 
change point detection; CUSUM algorithm}



\DOI{10.14357/19922264160405} 

\vspace*{-16pt}

\Ack
\noindent
The research was partially supported by the Russian Foundation for Basic Research 
(project 14-07-00041).



%\vspace*{3pt}

  \begin{multicols}{2}

\renewcommand{\bibname}{\protect\rmfamily References}
%\renewcommand{\bibname}{\large\protect\rm References}

{\small\frenchspacing
 {%\baselineskip=10.8pt
 \addcontentsline{toc}{section}{References}
 \begin{thebibliography}{99}

\bibitem{1-ch-1}
\Aue{Brigo, D., A.~Dalessandro, M.~Neugebauer, and F.~Triki}. 2007. 
{A~stochastic processes toolkit for risk management}. 
London: King's College London.  Working paper. 48~p.
\bibitem{2-ch-1}
\Aue{Ornstein, L.\,S., and G.\,E.~Uhlenbeck}. 1930. On the theory of the Brownian motion. 
\textit{Phys. Rev.} 36(5):823.
\bibitem{3-ch-1}
\Aue{Vasicek, O.} 1977. An equilibrium characterization of the term structure. 
\textit{J.~Financ. Econ.} 5(2):177--188.
\bibitem{4-ch-1}
\Aue{Cox, J.\,C., E.~Ingersoll, Jr., and S.\,A.~Ross}. 1985. 
A~theory of the term structure of interest rates. \textit{Econometrica} 53(2):385--407.
\bibitem{5-ch-1}
\Aue{Garbaczewski, P., and R.~Olkiewicz}. 2000. Ornstein--Uhlenbeck--Cauchy process. 
\textit{J.~Math. Phys.} 41:6843--6860.
\bibitem{6-ch-1}
\Aue{Finlay, R.} 2009. The variance gamma (VG) model with long range dependence: 
A~model for financial data incorporating long range dependence in squared returns.
Sydney, Australia: University of Sydney, School of Mathematics and Statistics. 
 PhD Thesis. 144~p.
\bibitem{7-ch-1}
\Aue{Kuzmina, A.\,V.} 2011. Modelirovanie normal'nogo obratnogo gaussovskogo 
protsessa i~otsenivanie ego papametrov [Normal inverse Gaussian distribution 
modeling and its parameters estimation]. 
\textit{Vestnik Belorusskogo gosudarstvennogo universiteta. Ser.~1: Fizika. Matematika. 
Informatika} [Herald of the Belarusian State University. Ser.~1: 
Physics. Mathematics. Informatics] 2:133--136. 
\bibitem{8-ch-1}
\Aue{Protter, P.} 1990. 
\textit{Stochastic integration and differential equations.}  
Heidelberg: Springer-Verlag. 415~p.
\bibitem{9-ch-1}
\Aue{Sato, K.\,I.} 1999. \textit{L$\acute{\mbox{e}}$vy processes and infinitely divisible 
distributions.}  Cambridge: Cambridge University Press. 500~p.
\bibitem{10-ch-1}
\Aue{Barndorff-Nielsen, O.\,E., and N.~Shephard}. 2001. Non-Gaussian 
Ornstein--Uhlenbeck-based models and some of their uses in financial economics. 
\textit{J.~Roy. Stat. Soc.~B} 63:167--241.
\bibitem{11-ch-1}
\Aue{Taufer, E., and N.~Leonenko.} 2008. Simulation of L$\acute{\mbox{e}}$vy-driven 
Ornstein--Uhlenbeck processes with given marginal distribution. 
\textit{Comput. Stat. Data An.} 53:2427--2437.
\bibitem{12-ch-1}
\Aue{Madan, D.\,B., and E.~Seneta}. 1990. 
The VG model for share market returns. \textit{J.~Bus.} 63:511--524.
\bibitem{13-ch-1}
\Aue{Madan, D.\,B, P.\,P.~Carr, and E.\,C.~Chang}. 1998. 
The variance gamma process and option pricing. \textit{Eur. Finance Rev.} 2:79--105.
\bibitem{14-ch-1}
\Aue{Barndorff-Nielsen, O.\,E.} 1997. 
Normal inverse Gaussian distributions and stochastic volatility modeling. 
\textit{Scand. J.~Stat.} 24(1):1--13.

\bibitem{16-ch-1}
\Aue{Rydberg, H.} 1997. The normal inverse Gaussian L$\acute{\mbox{e}}$vy process: 
Simulation and approximation. \textit{Commun. Stat. Stochastic Models} 
13(4):887--910.
\bibitem{15-ch-1}
\Aue{Barndorff-Nielsen, O.\,E.} 1998. Processes of normal inverse Gaussian type. 
\textit{Financ.  Stoch.} 2:41--68.
\bibitem{17-ch-1}
\Aue{Haykin, S.} 1996. \textit{Adaptive filter theory}. 3rd ed. 
Upper Saddle River, NJ: Prentice Hall. 989~p.
\bibitem{18-ch-1}
\Aue{Appel, U., and A.\,V.~Brandt}. 1983. 
Adaptive sequential segmentation of piecewise stationary time series.  
\textit{Inform. Sci.} 29:27--56.
\bibitem{19-ch-1}
\Aue{Gustafsson, F.} 1996. The marginalized likelihood ratio test for 
detecting abrupt changes. \textit{IEEE Trans. Automat. Contr.} 41:66--78.
\bibitem{20-ch-1}
\Aue{Kay, S.} 1993. \textit{Fundamentals of statistical signal processing. Vol.~I. 
Estimation theory}.  Upper Saddle River, NJ: Prentice Hall. 625~p.
\bibitem{21-ch-1}
\Aue{Page, E.\,S.} 1961. Cumulative sum control chart. 
\textit{Technometrics} 3:1--9.
\end{thebibliography}

 }
 }

\end{multicols}

\vspace*{-6pt}

\hfill{\small\textit{Received October 20, 2016}}

\vspace*{-18pt}

\Contr

\vspace*{-2pt}

\noindent
\textbf{Chertok Andrey V.} (b.\ 1987)~--- 
junior scientist, Faculty of Computational Mathematics and Cybernetics, 
M.\,V.~Lo\-monosov Moscow State University, 1-52~Leninskiye Gory, GSP-1, Moscow 119991, 
Russian Federation; Head of R\&D, Data Science, Sberbank of Russia, 19~Vavilov Str.,
Moscow 117999, Russian Federation; \mbox{avchertok.sbt@sberbank.ru}

 \vspace*{1pt}
 
\noindent
\textbf{Kadaner Arsenii I.} (b.\ 1995)~--- 
student, Faculty of Mechanics and Mathematics, 
M.\,V.~Lomonosov Moscow State University, 
Main Building, Leninskiye Gory, GSP-1, Moscow 119991, Russian Federation; 
data scientist, Sberbank of Russia, 19~Vavilov Str., Moscow 117999, 
Russian Federation; \mbox{aikadaner.sbt@sberbank.ru}

  \vspace*{1pt}
 
\noindent
\textbf{Khazeeva Gelana T.} (b.\ 1993)~---
 student,  Faculty of Computational Mathematics and Cybernetics, M.\,V.~Lo\-monosov 
 Moscow State University, 1-52~Leninskiye Gory, GSP-1, Moscow 119991, 
 Russian Federation; \mbox{gelana.khazeyeva@gmail.com} 

 
 \vspace*{1pt}
 
\noindent
\textbf{Sokolov Igor A.} (b.\ 1954)~---
Academician of the Russian Academy of Sciences, Doctor of Science in technology, 
Director, Federal Research Center ``Computer Science and Control'' of 
the Russian Academy of Sciences, 44-2~Vavilov Str., Moscow 119333, Russian Federation; 
\mbox{isokolov@ipiran.ru}
\label{end\stat}


\renewcommand{\bibname}{\protect\rm Литература}   %3

\newcommand{\Radon}{\textit{R}}
\newcommand{\Variance}{{\sf D}}

\def\stat{eroshenko}



\def\tit{АСИМПТОТИЧЕСКИЕ СВОЙСТВА ОЦЕНКИ РИСКА В ЗАДАЧЕ
ВОССТАНОВЛЕНИЯ ИЗОБРАЖЕНИЯ С КОРРЕЛИРОВАННЫМ
 ШУМОМ ПРИ ОБРАЩЕНИИ ПРЕОБРАЗОВАНИЯ РАДОНА$^*$}



\def\titkol{Асимптотические свойства оценки риска в задаче
восстановления изображения с коррелированным  шумом}
% при обращении преобразования Радона}

\def\aut{А.\,А.~Ерошенко$^1$,  О.\,В.~Шестаков$^2$}

\def\autkol{А.\,А.~Ерошенко,  О.\,В.~Шестаков}

\titel{\tit}{\aut}{\autkol}{\titkol}

{\renewcommand{\thefootnote}{\fnsymbol{footnote}} \footnotetext[1]
{Работа выполнена при финансовой поддержке РНФ (проект 14-11-00364).}}


\renewcommand{\thefootnote}{\arabic{footnote}}
\footnotetext[1]{Московский государственный университет им.\ М.\,В.~Ломоносова,
факультет вычислительной математики и кибернетики,
кафедра математической статистики,
aeroshik@gmail.com}
\footnotetext[2]{Московский государственный университет им.\ М.\,В.~Ломоносова, факультет вычислительной математики и кибернетики,
кафедра математической статистики; Институт проблем
информатики Российской академии наук, oshestakov@cs.msu.su}

\vspace*{-6pt}

\Abste{В последние годы вейвлет-методы, основанные на разложении проекций
по специальному базису и~последующей процедуре пороговой обработки,
широко используются при решении задач реконструкции томографических
 изображений. Их привлекательность заключается, во-пер\-вых, в~быстроте алгоритмов,
 а~во-вто\-рых, в~возможности реконструкции локальных участков изображения
 по неполным проекционным данным, что имеет ключевое значение, например,
 для медицинских приложений, где пациента нежелательно подвергать
 лишней дозе облучения. Анализ погрешностей этих методов представляет собой
 важную практическую задачу, поскольку позволяет оценить качество как
 самих методов, так и~используемого оборудования. В~работе рассматривается
 задача оценки функции при обращении оператора Радона в~модели
 с~коррелированным шумом. Исследуются асимптотические свойства оценки риска
 при пороговой обработке коэффициентов вейв\-лет-вейг\-лет-раз\-ло\-же\-ния
 функции изображения. Приводятся условия, при которых имеет место
 асимптотическая нормальность несмещенной оценки риска.}

\KW{вейвлеты; линейный однородный оператор; преобразование Радона;
пороговая обработка; несмещенная оценка риска; коррелированный шум;
асимптотическая нормальность}

\vspace*{-6pt}

\DOI{10.14357/19922264140404}


\vskip 10pt plus 9pt minus 6pt

\thispagestyle{headings}

\begin{multicols}{2}

\label{st\stat}

\section{Введение}

Методы статистического анализа часто применяются для решения задач,
в~которых данные наблюдаются не напрямую, в~частности для
задач компьютерной томографии, связанных с~обращением преобразования Радона.
Томографические методы революционизировали медицинскую диагностику,
поскольку позволили
<<увидеть>> внутренние органы человека, не подвергая пациента опасности.
Эти методы применяются также в~геологии, астрономии, сейсмологии, электронной
микроскопии, диагностике плазмы, химии и~во многих других областях.
Рассматривается следующая модель:
\begin{equation*}
X = R f + z\,,
\end{equation*}
где $R$~--- оператор Радона; $f$~--- искомая функция изображения; $z$~---
коррелированный шум с~нулевым математическим ожиданием.

Для того чтобы <<очистить>> целевую функцию (изображение) от шума,
используется пороговая обработка коэффициентов вейв\-лет-вейг\-лет-раз\-ло\-же\-ния
наблюдаемых данных. Наличие шума приводит к~погрешностям. Оценки этих погрешностей
(риска) и~их свойства в~моделях компьютерной томографии с~независимым шумом
исследовались в~работе~[1]. Показано, что при определенных условиях оценка
риска обладает свойствами состоятельности и~асимптотической нормальности.
В~данной работе исследуется асимптотическое поведение оценки риска в~модели
со стационарным коррелированным шумом.

\section{Преобразование Радона: вейвлет-вейглет-разложение функции}

Определим оператор Радона $\Radon$ как набор интегралов от функции~$f$
по всевозможным прямым плос\-кости:
\begin{equation*}
(\Radon f)(s,\theta) = \int\limits_{L_{s,\theta}} f(x,y)\, d l\,,
\end{equation*}
где
\begin{equation*}
{L_{s,\theta}} = \left\{(x,y)\colon x\cos\theta + y\sin\theta - s = 0 \right\}\,.
\end{equation*}

Задача томографии~--- восстановить функцию по наборам ее линейных интегралов,
 т.\,е.\ восстановить~$f$ по $\Radon f$. Для решения этой
 задачи воспользуемся методом вейв\-лет-вейг\-лет-раз\-ло\-же\-ния~[2].

Пусть заданы $\phi(x)$ и~$\psi(x)$~--- отцовский и~материнский вейвлеты.
Тогда можно определить:

\vspace*{2pt}

\noindent
\begin{equation}
\left.
\begin{array}{rl}
\psi_{j,k_1,k_2}^{[1]} (x,y) &= 2^j \phi(2^j x - k_1) \psi(2^j y - k_2)\,;\\[9pt]
\psi_{j,k_1,k_2}^{[2]} (x,y) &= 2^j \psi(2^j x - k_1) \phi(2^j y - k_2)\,;\\[9pt]
\psi_{j,k_1,k_2}^{[3]} (x,y) &= 2^j \psi(2^j x - k_1) \psi(2^j y - k_2),
\end{array}
\right\}
\label{vaguelette_2d}
\end{equation}

\vspace*{-2pt}

\noindent
семейство $\left\{\psi^{[\lambda]}_{j,k_1,k_2}\right\}_{\lambda,j,k_1,k_2}$
образует ортонормированный базис в~$L^2(\mathbb{R}^2)$. Индекс~$j$
в~(\ref{vaguelette_2d}) называется масштабом, а~индексы $k_1$ и~$k_2$~---
сдвигами. Функция~$\psi$ должна удовлетворять определенным
требованиям, однако ее можно выбрать таким образом, чтобы она
обладала некоторыми полезными свойствами, например была
дифференцируемой нужное число раз и~имела заданное число~$M_0$
нулевых моментов~[3], т.\,е.

\vspace*{4pt}

\noindent
$$
\int\limits_{-\infty}^{\infty}t^k\psi(t)\,dt=0\,,\quad k=0,\ldots,M_0-1\,.
$$

\vspace*{-2pt}

\noindent
В данной работе предполагается, что используются вейвлеты Мейера~[4]
с~достаточным количеством непрерывных производных.

Вейвлет-разложение функции~$f$ имеет вид:

\vspace*{4pt}

\noindent
\begin{equation}
\label{waveletdecomp}
f = \sum\limits_{\lambda,j,k_1,k_2} \left\langle f\,,\
\psi^{[\lambda]}_{j,k_1,k_2}\right\rangle \psi^{[\lambda]}_{j,k_1,k_2}\,.
\end{equation}


\vspace*{-2pt}

На практике в~задаче томографии функция~$f$ обычно задана в~дискретных отсчетах
на круге. Без ограничения общности будем считать, что это круг
единичного радиуса с~центром в~начале координат.
Будем рассматривать <<растянутую>> версию функции
$ \overline{f} (Nx,Ny) \hm= f(x,y)$.
Для дискретных вейв\-лет-ко\-эф\-фи\-ци\-ен\-тов справедливо
$f_{j,k_1,k_2}^{W^{[\lambda]}} \approx 2^{J} \left\langle f,
\psi^{[\lambda]}_{j,k_1,k_2} \right\rangle$.
Также потребуем, чтобы функция~$f$ была равномерно регулярной по
Липшицу с~некоторым параметром $0\hm<\gamma\hm<1$:

\noindent
\begin{equation*}
\left\vert f(x_1,y_1) - f(x_2,y_2)\right\vert
\leq C\!\left(\!\left\vert x_1 - x_2\right\vert^2\! + \!
\left\vert y_1 - y_2\right\vert^2\right)^{\!\gamma/2}\!\!,
\end{equation*}
где $C$~--- некоторая константа. Тогда существует константа~$A$ такая, что~[4]:

\noindent
\begin{align}
\label{Coeff_Decay}
\abs{f_{j,k_1,k_2}^{W^{[\lambda]}}} \leq \fr{A \cdot 2^{J}}{2^{j(\gamma+1)}}\,.
\end{align}

Для обращения оператора Радона определим вейглеты~[2]:
\begin{equation}
\xi_{j,k_1,k_2}^{[\lambda]} = I^{-1}\Radon\psi_{j,k_1,k_2}^{[\lambda]}\,,
\label{Radon_vaguelette}
\end{equation}
где $I$~--- потенциал Рисса:
$\hat{I}^{p} g(w) \hm= |w|^{-p} \hat{g}(w)$. Для вейглетов справедливо соотношение:
\begin{equation*}
\left[R f, \xi_{j,k_1,k_2}^{[\lambda]}\right] = \left\langle
f,\psi_{j,k_1,k_2}^{[\lambda]}\right\rangle\,,
\end{equation*}
которое позволяет использовать в~разложении~(\ref{waveletdecomp})
только проекционные данные и~тем самым предлагает метод обращения преобразования
Радона.


\section{Модель данных}

Предположим, что проекционные данные измеряются при $(s,\theta)\hm\in [-1,1]\times[0,\pi]$.
Пусть $\{e_{i,j}$, $i,j \hm \in \mathbb{Z}\}$~--- стационарный гауссовский
процесс с~ковариационной последовательностью $r_{k,p} \hm= \cov (e_{i,j},
e_{i+k,j+p})$.
Для выборки с~размерами $n \hm= m$ модель проекционных данных
с~шумом выглядит следующим образом:
\begin{multline*}
Y_{i,j} = R f\left(-1+\fr{2i}{n},\fr{j\pi}{n}\right) + e_{i,j}\,, \\
i = 1, \dots, n\,, \enskip j = 1, \dots, n\,, \enskip n=2^J\,.
\end{multline*}
%

Структура ковариации шума для преобразования Радона должна
отражать типичную ситуацию: на практике проекции для разных углов
регистрируются независимо друг от друга.
В~рассматриваемой модели ошибок получаются независимые наблюдения
в~случае разных углов и~стационарный гауссовский шум
с~нулевым математическим ожиданием, конечной дисперсией
и~ковариационной последовательностью $r_\delta \hm\sim A \delta^{-\alpha}$
($0\hm<\alpha\hm<1$) для одинаковых углов.

Как и~в одномерном случае~[5], создаем для равномерных отсчетов наблюдаемое
поле при $(s,\theta)\hm\in [-1,1]\times[0,\pi]$:
\begin{multline*}
Y_n(s,\theta) = \fr{1}{n^2} \sum\limits_{i=1}^{[n(s+1)/2]}\,
\sum\limits_{j=1}^{[n\theta/\pi]} Y_{i,j}=
RF_n(s,\theta) + {}\\
{}+\fr{1}{n^2} \sum\limits_{i=1}^{[n(s+1)/2]}\,
\sum\limits_{j=1}^{[n\theta/\pi]} e_{i,j}\,,
\end{multline*}
где
\begin{equation*}
RF_n(s,\theta)= \fr{1}{n^2}
\sum\limits_{i=1}^{[n(s+1)/2]}\sum\limits_{j=1}^{[n\theta/\pi]}
R f\left(-1+\fr{2i}{n},\fr{j\pi}{n}\right)
\end{equation*}
представляет собой суммарный <<сигнал>> поля.

Положим $H = 1- \alpha/2$, $H \hm\in (1/2,1)$
и~определим дробное броуновское движение $\mathbf{B}_H(s)$~---
гауссовский процесс на $\mathbb{R}$ с~нулевым средним  и~ковариационной функцией
\begin{equation*}
r(s,t) = \fr{V_H}{2}(|s|^{2H} + |s|^{2H} - |s-t|^{2H})\,, \enskip s,t \in \mathbb{R}\,,
\end{equation*}
где
$$
V_H = \Variance (\mathbf{B}_H(1)) =
\fr{-\Gamma(2-2H)\cos(\pi H)}{\pi H(2H-1)}\,.
$$

Далее по аналогии с~моделью, описанной в~[5],
рассматриваем непрерывную аппроксимацию:
\begin{equation}
Y(s,\theta) = R F(s,\theta) + n^{-(1+\alpha)/2} \tau \mathbf{B'}_H(s,\theta)\,,
\label{model_sum2}
\end{equation}
где $\tau = \sqrt{{2A}/({(1-\alpha)(2-\alpha)})}$~--- нормировочный множитель,
который без ограничения общности далее будем считать равным единице,
$R F(s,\theta)\hm=\int\limits_{-1}^{s}\int\limits_{0}^{\theta}Rf(t,q)\,dtdq$,
а~$\mathbf{B'}_H(s,\theta)$~--- случайная функция, которая для каждого
фиксированного угла~$\theta$ представляет собой дробное броуновское
движение $\mathbf{B}_H(s)$ и~имеет некоррелированные приращения по~$\theta$.

Применяя к~\eqref{model_sum2} вейглет-раз\-ло\-же\-ние, получаем:
\begin{align*}
\left[Y, \xi_{j,k_1,k_2}^{[\lambda]}\right] &\!\!= \!
\left[R f, \xi_{j,k_1,k_2}^{[\lambda]}\right]\! \!+\! n^{-(1+\alpha)/2}
\left[ \mathbf{B'}_H, \xi_{j,k_1,k_2}^{[\lambda]}\right]\!;\\
\left[Y, \xi_{j,k_1,k_2}^{[\lambda]}\right] &\!\!= \!
\left\langle f, \psi_{j,k_1,k_2}^{[\lambda]}\right\rangle \!+\! n^{-(1+\alpha)/2}
\left[ \mathbf{B'}_H, \xi_{j,k_1,k_2}^{[\lambda]}\right]\!.
\end{align*}
%

Переходя к~дискретному вейглет-пре\-обра\-зо\-ва\-нию и~вспоминая, что $n\hm=2^J$,
по аналогии с~дискретным вейв\-лет-пре\-обра\-зо\-ва\-ни\-ем
получаем модель дискретных вейг\-лет-ко\-эф\-фи\-ци\-ентов:
\begin{equation*}
X_{j,k_1,k_2}^{[\lambda]} = \mu_{j,k_1,k_2}^{[\lambda]} +
2^{(1-\alpha)J/2} e_{j,k_1,k_2}^{[\lambda]}\,,
\end{equation*}
где
\begin{align*}
\mu_{j,k_1,k_2}^{[\lambda]} &=  2^J\left[Rf, \xi_{j,k_1,k_2}^{[\lambda]}\right]\,;
\\
e_{j,k_1,k_2}^{[\lambda]}&=\left[ \mathbf{B'}_H, \xi_{j,k_1,k_2}^{[\lambda]}\right]
=\int\xi_{j,k_1,k_2}^{[\lambda]}\,d\mathbf{B'}_H\,.
\end{align*}

Из~\eqref{vaguelette_2d} и~\eqref{Radon_vaguelette} получаем
выражения для преобразований Фурье вейг\-лет-функ\-ций по первому аргументу:
\begin{align*}
\widehat{\xi}_{j,k_1,k_2}^{[1]}(w,\theta) &= |w|\cdot 2^{-j}e^{i(k_1\cos\theta+k_2
\sin\theta) 2^{-j} w}\times{}\\
&\hspace*{9mm}{}\times \widehat{\phi}(2^{-j}w\cos\theta) \widehat{\psi}(2^{-j}w\sin\theta)\,;\\
\widehat{\xi}_{j,k_1,k_2}^{[2]}(w,\theta) &=
|w|\cdot 2^{-j}e^{i(k_1\cos\theta+k_2\sin\theta) 2^{-j} w}{}\times{}\\
&\hspace*{9mm}{}\times \widehat{\psi}(2^{-j}w\cos\theta) \widehat{\phi}(2^{-j}w\sin\theta)\,;\\
\widehat{\xi}_{j,k_1,k_2}^{[3]}(w,\theta) &=
|w|\cdot 2^{-j}e^{i(k_1\cos\theta+k_2\sin\theta) 2^{-j} w} \times{}\\
&\hspace*{9mm}{}\times\widehat{\psi}(2^{-j}w\cos\theta) \widehat{\psi}(2^{-j}w\sin\theta).
\end{align*}
Рассмотрим ковариацию коэффициентов модели, например для $(\lambda_1 ,\lambda_2)
\hm=(3,3)$. Проведем интегрирование по углу и~воспользуемся свойствами
$\mathbf{B}_H(s)$~[5]. Без ограничения общности считаем, что  $j \hm\geq i$.
Обозначим $\Delta \hm= j \hm- i$. Имеем

\columnbreak

\noindent
\begin{multline*}
\left\vert \,\cov\left(X_{j,k_1,k_2}^{[3]},X_{i,l_1,l_2}^{[3]}\right)\right\vert
={}\\[3pt]
{}= \left\vert \fr{1}{2\pi}\,2^{(1-\alpha)J}\int \!\!\int
\widehat{\xi}_{j,k_1,k_2}^{[3]}(w,\theta) \overline{\widehat{\xi}_{i,l_1,l_2}^{[3]}
(w,\theta)} \times{}\right.\\[3pt]
\left.{}\times |w|^{-(1-\alpha)}\, dw d\theta \vphantom{\fr{1}{2\pi}}
\right\vert=
\left\vert \fr{1}{2\pi}\,2^{(1-\alpha)J}\int \!\!\int
 2^{-j-i} w^2\times{}\right.\\[3pt]
\left. {}\times e^{i((k_1\cos\theta+k_2\sin\theta) 2^{-j} -
 (l_1\cos\theta+l_2\sin\theta) 2^{-i}) w} \times{}\right.
\\[3pt]
{}\times \widehat{\psi}(2^{-j}w\cos\theta)
\widehat{\psi}(2^{-j}w\sin\theta) \times{}\\
\left.{}\times \overline{ \widehat{\psi}(2^{-i}w\cos\theta)
\widehat{\psi}(2^{-i}w\sin\theta)} |w|^{-(1-\alpha)} \,d w d\theta
\vphantom{\fr{1}{2\pi}}\right\vert={}
\end{multline*}

\vspace*{-2pt}

\noindent
(сделаем замену $w' = 2^{-i} w$ и~перейдем от полярных координат к~декартовым:
$w_1 \hm= w \cos\theta$, $w_2 \hm= w\sin\theta$)
%
\begin{multline*}
=\left\vert \fr{1}{2\pi}\, 2^{(1-\alpha)J}\cdot 2^{i\alpha-\Delta}\times{}\right.\\
{}\times \int\!\! \int
e^{i((k_1 2^{-\Delta} - l_1) w_1+(k_2 2^{-\Delta} - l_2) w_2 )} \times{}
\\
{}\times \left\vert w_1^2 +w_2^2\right\vert^{(1+\alpha)/2} \widehat{\psi}(2^{-\Delta}w_1)
 \widehat{\psi}(2^{-\Delta}w_2) \times{}\\
\left. {}\times\overline{ \widehat{\psi}(w_1) \widehat{\psi}(w_2)}\,  d w_1 d w_2
\vphantom{\fr{1}{2\pi}}\right\vert \leq{}
\end{multline*}


\vspace*{-2pt}

\noindent
(воспользуемся следующим свойством вейвлетов Мейера~[5]: при любом натуральном~$M_0$
существует константа $C_{M_0}\hm>0$ такая, что  $|\widehat{\psi}(w)|\hm\leq
C_{M_0}|w|^{M_0}\textbf{1}_{w\in \mathrm{supp}\left(\widehat{\psi}\right)}$)

\noindent
\begin{multline*}
\leq\left\vert \fr{C^2_{M_0}}{2\pi}\,
2^{(1-\alpha)J}\cdot 2^{i\alpha-\Delta}\times{}\right.\\
{}\times \int\limits_{\mathrm{supp}\left(\widehat{\psi}\right)}
\int\limits_{\mathrm{supp}\left(\widehat{\psi}\right)}
e^{i((k_1 2^{-\Delta} - l_1) w_1+(k_2 2^{-\Delta} - l_2) w_2 )} \times{}\\
{}\times |w_1^2 +w_2^2|^{(1+\alpha)/{2}}\cdot
2^{-2\Delta M_0}(w_1 w_2)^{M_0} \times{}\\
\left.{}\times\overline{ \widehat{\psi}(w_1)
\widehat{\psi}(w_2)} \, d w_1 d w_2
\vphantom{\fr{C^2_{M_0}}{2\pi}}\right\vert\leq{}\\
\end{multline*}

\vspace*{-14pt}

\noindent
(предположим, что выбранный вейвлет Мейера имеет достаточное число
непрерывных производных, чтобы функция $g(w_1,w_2) \hm=
|w_1^2\hm +w_2^2|^{({1+\alpha})/2} \times$\linebreak
$\times 2^{-2\Delta M_0}(w_1 w_2)^{M_0}
\overline{ \widehat{\psi}(w_1) \widehat{\psi}(w_2)}$ имела $M_1$
непрерывных производных по~$w_1$ и~$w_2$, и~воспользуемся свойствами
обратного преобразования Фурье)

\noindent
\begin{multline*}
\leq 2^{(1-\alpha)J} \cdot 2^{i\alpha}\cdot 2^{-(1+2M_0)\Delta}\times{}\\
{}\times
\fr{C''}{\left\vert k_1\cdot 2^{-\Delta} - l_1\right\vert^{M_1}
\left\vert k_2\cdot 2^{-\Delta} - l_2\right\vert^{M_1}}
\end{multline*}
с некоторой константой $C''\hm>0$.


Отдельно выделим случай $k_1 \cdot2^{-\Delta}\hm=l_1$ и~$k_2 \times$\linebreak
$\times 2^{-\Delta}\hm=l_2$.
Можно показать, что в~этом случае
\begin{equation*}
\left\vert\cov\left(X_{j,k_1,k_2}^{[\lambda]},X_{i,l_1,l_2}^{[\lambda]}\right)\right\vert
\leq C_e\cdot 2^{J(1-\alpha) + i\alpha -(2M_0+1)\Delta}
\end{equation*}



\noindent
с некоторой константой $C_e\hm>0$. Аналогично рассматриваются случаи, когда
$k_1 \cdot2^{-\Delta}\hm=l_1$ и~$k_2 \cdot2^{-\Delta}\hm=l_2$ 
выполнены не одновременно.
Варианты других комбинаций $(\lambda_1 ,\lambda_2)$ рассматриваются аналогично.


Обозначим $M' = -(2M_1-2M_0-1)$ и~выберем~$M_0$ так, что $M'\hm>0$.

Тогда

\noindent
\begin{multline}
\left\vert \cov\left(X_{j,k_1,k_2}^{[\lambda]},X_{i,l_1,l_2}^{[\lambda]}\right)\right\vert
\leq{}\\
\hspace*{-1mm}{}\leq
\begin{cases}
        C'\cdot 2^{J(1-\alpha) + i\alpha -(2M_0+1)\Delta} &
        \hspace*{-13mm}\mbox{ при }k_1\cdot 2^{-\Delta}=l_1, \\
        &\hspace*{-5.3mm} k_2\cdot 2^{-\Delta}=l_2\,;\\
        C' \fr{2^{(1-\alpha)J+i\alpha-M'\Delta}}
        {\left\vert k_2 - 2^{\Delta} l_2\right\vert^{M_1}} & \hspace*{-13mm}\mbox{ при } k_1 \cdot2^{-\Delta}=l_1\,;\\[11pt]
        C' \fr{2^{(1-\alpha)J+i\alpha-M'\Delta}}
        {\left\vert k_1 - 2^{\Delta} l_1\right\vert^{M_1}} & \hspace*{-13mm}\mbox{ при } k_2 \cdot2^{-\Delta}=l_2\,;\\[11pt]
        C' \fr{2^{(1-\alpha)J+i\alpha-M'\Delta}}
        {\left\vert k_1 - 2^{\Delta} l_1\right\vert^{M_1}
        \left\vert k_2 - 2^{\Delta} l_2\right\vert^{M_1}} & \mbox{ иначе}
    \end{cases}\!\!\!\!
    \label{Cov_Decay11}
\end{multline}

\vspace*{-3pt}

\noindent
с некоторой константой $C'\hm>0$.

Дисперсия для коэффициентов модели имеет вид:
\begin{equation}
\label{Var_Model}
\sigma_{\lambda,j}^2=C_{\lambda,\alpha} \cdot 2^{(1-\alpha)J}\cdot 2^{j\alpha}\,,
\end{equation}
где константа $C_{\lambda,\alpha}$ зависит от параметра~$\alpha$
и~выбранного вейв\-лет-ба\-зиса.

\vspace*{-6pt}

\section{Пороговая обработка и~оценка риска}

\vspace*{-1pt}

Смысл пороговой обработки коэффициентов вейв\-лет-вейг\-лет-раз\-ло\-же\-ния
заключается в~удалении достаточно маленьких коэффициентов, которые считаются шумом.
Будем рассматривать так называемую мягкую пороговую обработку с~порогом
$T_{\lambda,j}$, зависящим от уровня~$j$. К~каждому
коэффи\-ци\-енту применяется функция
$\rho_{T_{\lambda,j}}(x)\hm=\mathrm{sgn}\left(x\right)\left(\abs{x}\hm -
T_{\lambda,j}\right)_{+}$, т.\,е.\ при такой пороговой обработ\-ке
коэффициенты, которые по модулю меньше порога $T_{\lambda,j}$,
обнуляются, а~абсолютные величины остальных коэффициентов
уменьшаются на величину порога. Погрешность (или риск) мягкой
пороговой обработки определяется следующим образом:

\vspace*{-2pt}

\noindent
\begin{multline}
\label{Risk_def}
R_J(f) = \sum\limits_{j = 0}^{J - 1}\sum_{k_1=0}^{2^j-1}\sum\limits_{k_2=0}^{2^j-1}
\sum\limits_{\lambda=1}^3 \Expect\left(\mu_{j,k_1,k_2}^{[\lambda]} -{}\right.\\
\left.{}-
\rho_{T_{\lambda,j}}\left(X_{j,k_1,k_2}^{[\lambda]}\right)\right)^2.
\end{multline}

В~[6] предложено использовать порог $T_{\lambda,j} \hm= \sqrt{2\ln 2^{2j}}
\sigma_{\lambda,j}$, названный универсальным.
В~дальнейшем будет использоваться именно такой вид\linebreak\vspace*{-12pt}

\columnbreak

\noindent
 порога.
В~выражении~(\ref{Risk_def}) присутствуют неизвестные величины
$\mu_{j,k_1,k_2}^{[\lambda]}$, поэтому вычислить значение $R_J(f)$ нельзя.
Однако его можно оценить. В~качестве оценки риска используется следующая величина~[4]:
\begin{multline}
\label{risk_Est}
\widehat{R}_J (f) = {}\\
{}=\sum\limits_{j=0}^{J-1} \sum\limits_{k_1=0}^{2^j-1}
\sum\limits_{k_2=0}^{2^j-1} \sum\limits_{\lambda=1}^3 F\left[
\left(X_{j,k_1,k_2}^{[\lambda]}
\right)^2,T_{\lambda,j},\sigma_{\lambda,j}\right]\,,
\end{multline}
где
$F(x,T,\sigma) = (x - \sigma^2)\mathbf{1}_{|x|\leq T} \hm+
\left(\sigma^2\hm + T^2\right)\mathbf{1}_{|x|>T}.
$
Величина $\widehat{R}_J(f)$ является несмещенной оценкой для $R_J(f)$~[4].
В~работе~[1] исследовались асимптотические свойства оценки~(\ref{risk_Est})
в~модели с~независимым шумом.
Было показано, что при определенных условиях гладкости эта
оценка является состоятельной и~асимптотически нормальной.
Далее будет исследовано асимптотическое поведение оценки~(\ref{risk_Est})
в~модели данных с~долгосрочной зависимостью, рассматриваемой в~данной работе.

\vspace*{-5pt}

\section{Вспомогательные результаты}

\vspace*{-2pt}

Введем следующее обозначение для последовательностей: $a_J\hm\simeq b_J$,
если $\lim\limits_{J\rightarrow \infty} ({a_J}/{b_J})\hm = 1$.

\vspace*{3pt}

\noindent
\textbf{Лемма~1.}
\textit{Для любых $0<\alpha<1$ и~$\gamma>(1+\alpha)^{-1}$ выполняется
$D_J^2 \hm= \Variance \widehat{R}_J(f)\hm \simeq  \tilde{C}\cdot2^{4J}$,
где константа~$\tilde{C}$ зависит от~$\alpha$ и~выбранного вейвлет-базиса,
но не зависит от функции сигнала}~$f$.

\smallskip

\noindent
Д\,о\,к\,а\,з\,а\,т\,е\,л\,ь\,с\,т\,в\,о\,.\ \ Поскольку
$\gamma > {1}/(1+\alpha)$, то ${1}/(1+\gamma)\hm<(1\hm+\alpha)/(2\hm+\alpha)$.
Выберем~$p''$ так, что ${1}/(1+\gamma)\hm < p'' \hm<
(1\hm+\alpha)/(2\hm+\alpha)$ и~$p''J$~--- целое число.
Тогда в~силу~\eqref{Coeff_Decay} $\mu_{j,k_1,k_2}^{[\lambda]}\hm\to 0$
для всех~$j$: $p''J\hm\leq j\hm< J$ при $J\hm\to \infty$.
Разобьем выражение~\eqref{risk_Est} на две суммы:

\noindent
\begin{multline*}
\widehat{R}_J (f) ={}\\
{}= \sum\limits_{j=0}^{p''J} \sum\limits_{k_1=0}^{2^j-1}
\sum\limits_{k_2=0}^{2^j-1}
\sum\limits_{\lambda=1}^3 F\left[\left(X_{j,k_1,k_2}^{[\lambda]}\right)^2,T_{\lambda,j},
\sigma_{\lambda,j}\right] +{}\\
{}+
 \sum\limits_{j=p''J+1}^{J-1} \sum\limits_{k_1=0}^{2^j-1}
 \sum\limits_{k_2=0}^{2^j-1} \sum\limits_{\lambda=1}^3
 F\left[\left(X_{j,k_1,k_2}^{[\lambda]}\right)^2,T_{\lambda,j},\sigma_{\lambda,j}\right].
\end{multline*}
%
Поскольку существует константа $C_F \hm> 0$ такая, что
$F\left[\left(X_{j,k_1,k_2}^{[\lambda]}\right)^2,T_{\lambda,j},\sigma_{\lambda,j}\right] \hm
\leq C_F  j \cdot2^{(1-\alpha)J}\cdot2^{j\alpha}$, то для первой суммы имеем:

\noindent
\begin{multline*}
\abs{\sum_{j=0}^{p''J} \sum_{k_1=0}^{2^j-1}\sum_{k_2=0}^{2^j-1}
\sum_{\lambda=1}^3 F\left[\left(X_{j,k_1,k_2}^{[\lambda]}\right)^2,T_{\lambda,j},
\sigma_{\lambda,j}\right]}
\leq {}\\
{}\leq C_F\sum_{j=0}^{p''J}
\sum_{k_1=0}^{2^j-1}
\sum_{k_2=0}^{2^j-1}
\sum_{\lambda=1}^3 j \cdot2^{(1-\alpha)J}\cdot2^{j\alpha} \leq{}
\end{multline*}

\end{multicols}



\noindent
\begin{equation}
\label{Large_Coeff_Order}
\hspace*{10mm}{}\leq C_F\cdot 2^{(1-\alpha)J}
\sum_{j=0}^{p''J}
\sum_{\lambda=1}^3 j\cdot 2^{j(2+\alpha)}
\leq C'_F  J\cdot 2^{J(1 - \alpha + (\alpha + 2) p'')}
\end{equation}
с некоторой константой $C'_F\hm > 0$. Далее,
\begin{multline}                                                                        \label{Var_Struct}
\Variance \widehat{R}_J(f)
= \sum\limits_{j = 0}^{J - 1} \sum\limits_{k_1 = 0}^{2^j - 1}
\sum\limits_{k_2 = 0}^{2^j - 1} \sum\limits_{\lambda=1}^3
\Variance F\left[\left(X_{j,k_1,k_2}^{[\lambda]}\right)^2,T_{\lambda,j},\sigma_{\lambda,j}\right]+{}\\
{}+ \sum\limits_{i = 0}^{J - 1} \sum\limits_{l_1 = 0}^{2^i - 1}
\sum\limits_{l_2 = 0}^{2^i - 1} \sum\limits_{\lambda_1=1}^3
\sum\limits_{j = 0}^{J - 1}\sum\limits_{k_1 = 0}^{2^j - 1}
\sum\limits_{k_2 = 0}^{2^j - 1}\sum\limits_{\lambda_2=1}^3\!
\cov\left(F\left[\left(X_{i,l_1,l_2}^{[\lambda_1]}\right)^2,
T_{\lambda_1,i},
\sigma_{\lambda_1,i}\right], F\left[\left(X_{j,k_1,k_2}^{[\lambda_2]}\right)^2,
T_{\lambda_2,j},\sigma_{\lambda_2,j}\right]\right)\,,
\end{multline}
где во второй сумме предполагается, что $(\lambda_1,i,l_1,l_2)\hm\neq(\lambda_2,j,k_1,k_2)$.

Сумму дисперсий разобьем на две суммы:
\begin{equation*}
\sum\limits_{j = 0}^{p''J} \sum\limits_{k_1 = 0}^{2^j - 1}
\sum\limits_{k_2 = 0}^{2^j - 1}
\sum\limits_{\lambda=1}^3 \Variance F\left[\left(X_{j,k_1,k_2}^{[\lambda]}\right)^2,T_{\lambda,j},\sigma_{\lambda,j}\right]+
\sum\limits_{j = p''J + 1}^{J-1} \sum\limits_{k_1 = 0}^{2^j - 1}
\sum\limits_{k_2 = 0}^{2^j - 1}
\sum\limits_{\lambda=1}^3 \Variance F
\left[\left(X_{j,k_1,k_2}^{[\lambda]}\right)^2,T_{\lambda,j},\sigma_{\lambda,j}\right].
\end{equation*}
Поскольку $p''<(1+\alpha)/(2+\alpha)$, из~\eqref{Large_Coeff_Order}
следует, что первая сумма имеет меньший порядок, чем вторая. Для второй
в~силу выбора порога и~(\ref{Var_Model}) справедливо
\noindent
\begin{multline*}
\sum\limits_{j = p''J + 1}^{J - 1} \sum\limits_{k_1 = 0}^{2^j - 1}
\sum\limits_{k_2 = 0}^{2^j - 1} \sum_{\lambda=1}^3 \Variance
F\left[\left(X_{j,k_1,k_2}^{[\lambda]}\right)^2,T_j,\sigma_j\right]
\simeq \sum\limits_{j = p''J + 1}^{J - 1} \sum\limits_{k_1 = 0}^{2^j - 1}
\sum\limits_{k_2 = 0}^{2^j - 1}
\sum\limits_{\lambda=1}^3 \Variance \left(X_{j,k_1,k_2}^{[\lambda]}\right)^2={}
\\
{}=
\sum_{j = p''J + 1}^{J - 1} \sum_{k_1 = 0}^{2^j - 1}
\sum_{k_2 = 0}^{2^j - 1} C_1 \cdot2\sigma_j^2\left(\sigma_j^2 +\mu^2_{j,k_1,k_2}\right)
\simeq\sum\limits_{j = p''J + 1}^{J - 1} \sum\limits_{k_1 = 0}^{2^j - 1}
\sum\limits_{k_2 = 0}^{2^j - 1} 2C_1 \sigma_j^4
\simeq C'_\alpha \cdot2^{4J}\,,
\end{multline*}
где $C_1$ и~$C'_\alpha$~--- некоторые положительные константы.
%
Далее перейдем к~сумме ковариаций в~(\ref{Var_Struct}).
Аналогично сумме дисперсий имеем:
\begin{multline*}
\hspace*{-2.5mm}\sum\limits_{i = 0}^{J - 1} \sum\limits_{l_1 = 0}^{2^i - 1}
\sum\limits_{l_2 = 0}^{2^i - 1} \sum\limits_{\lambda_1=1}^3
\sum\limits_{j = 0}^{J - 1}\sum\limits_{k_1 = 0}^{2^j - 1}
\sum\limits_{k_2 = 0}^{2^j - 1}\sum\limits_{\lambda_2=1}^3\!\!
\cov\!\left(\!F\!\left[\left(X_{i,l_1,l_2}^{[\lambda_1]}\right)^2,
T_{\lambda_1,i},\sigma_{\lambda_1,i}
\vphantom{(X_{i,l_1,l_2}^{[\lambda_1]})^2}\right],
F\left[\left(X_{j,k_1,k_2}^{[\lambda_2]}\right)^2,T_{\lambda_2,j},\sigma_{\lambda_2,j}\right]\right)
\simeq {}\\
{}\simeq \sum\limits_{i = p''J + 1}^{J - 1} \sum\limits_{l_1 = 0}^{2^i - 1} \sum_{l_2 = 0}^{2^i - 1} \sum_{\lambda_1=1}^3\sum_{j = p''J + 1}^{J - 1}\sum_{k_1 = 0}^{2^j - 1} \sum_{k_2 = 0}^{2^j - 1}\sum_{\lambda_2=1}^3
\cov\left(\left(X_{i,l_1,l_2}^{[\lambda_1]}\right)^2,
\left(X_{j,k_1,k_2}^{[\lambda_2]}\right)^2\right)\,.
\end{multline*}
Известно, что $\cov\left(X^2,Y^2\right) \hm= 4\Expect X\Expect Y \cov\left(X, Y\right) + 2
\cov^2 \left(X, Y\right)$, если вектор $(X,Y)$ имеет двумерное нормальное распределение.
Так же, как в~работе~[7], можно показать, что асимптотика первого
слагаемого меньше второго, а для второго в~силу~(\ref{Cov_Decay11}) справедливо
\begin{multline*}
\fr{1}{2} \sum\limits_{i = p''J + 1}^{J - 1}
\sum\limits_{l_1 = 0}^{2^i - 1} \sum\limits_{l_2 = 0}^{2^i - 1}
\sum\limits_{\lambda_1=1}^3 \sum\limits_{j = p''J + 1}^{J - 1}
\sum\limits_{k_1 = 0}^{2^j - 1} \sum\limits_{k_2 = 0}^{2^j - 1}
\sum\limits_{\lambda_2=1}^3
\cov^2\left(X_{i,l_1,l_2}^{[\lambda_1]}, X_{j,k_1,k_2}^{[\lambda_2]}\right)=
{}\\
{}=\sum\limits_{\lambda_1=1}^3\sum\limits_{\lambda_2=1}^3
\sum\limits_{i = p''J + 1}^{J - 1} \sum\limits_{l_1 = 0}^{2^i - 1}
\sum\limits_{l_2 = 0}^{2^i - 1} \sum\limits_{\Delta = 0}^{J - 1 - i}
{\sum\limits^{2^{i+\Delta} - 1}_{k_1 = %\protect\tiny
\left\{
\begin{array}{ll} %\begin{cases}
{\scriptstyle l_1 + 1}, & {\scriptstyle \Delta = 0;} \\
{\scriptstyle 0,} & {\scriptstyle \Delta > 0.}
\end{array}\right.
%\end{cases}
}}
{\sum\limits^{2^{i+\Delta} - 1}_{k_2 =  %{ %\protect\tiny
\left\{  %\begin{cases}
\begin{array}{ll}
{\scriptstyle l_2 + 1,} & {\scriptstyle \Delta = 0;} \\
{\scriptstyle 0,} & {\scriptstyle \Delta > 0.}
\end{array}
\right.
%\end{cases}
} }
\cov^2\left(X_{i,l_1,l_2}^{[\lambda_1]},
X_{j,k_1,k_2}^{[\lambda_2]}\right) \leq{}\\
{}\leq \sum\limits_{i = p''J + 1}^{J - 1} \sum\limits_{l_1 = 0}^{2^i - 1}
\sum\limits_{l_2 = 0}^{2^i - 1}
\left(\sum\limits_{\delta_1 = 1}^{2^i - l_1} \sum\limits_{\delta_2 = 1}^{2^i - l_2}
\fr{ C'^2 \cdot2^{2J(1-\alpha)}2^{2i\alpha}}{\delta_1^{2M_1}\, \delta_2^{2M_1}} +{}
\right.\\
\left.{}+ \sum\limits_{\Delta = 1}^{J-i-1}\sum\limits_{k_1 = 0}^{2^{i+\Delta} - 1}
\sum\limits_{k_2 = 0}^{2^{i+\Delta} - 1}
2^{2J(1-\alpha)}\cdot 2^{2i\alpha}\cdot 2^{-2M'\Delta}
\fr{C'^2\mathbf{1}(k_1 \neq 2^{\Delta} l_1,k_2 \neq 2^{\Delta} l_2)}
{\left\vert k_1 - 2^{\Delta} l_1\right\vert^{2M_1}\left\vert k_2 - 2^{\Delta} l_2\right\vert^{2M_1}} \right) ={}
\end{multline*}

\noindent
\begin{multline*}
{}= C'^2 \cdot2^{2J(1-\alpha)} \sum\limits_{i = p''J + 1}^{J - 1}
2^{2i\alpha}\sum\limits_{l_1 = 0}^{2^i - 1} \sum\limits_{l_2 = 0}^{2^i - 1}
\left( \sum\limits_{\delta_1 = 1}^{2^i - l_1}
\fr{1}{\delta_1^{2M_1}}\sum\limits_{\delta_2 = 1}^{2^i - l_2} \fr{1}
{\delta_2^{2M_1}} +{}\right.\\
\left.{}+ \sum\limits_{\Delta = 1}^{J-i-1}\sum\limits_{k_1 = 0}^{2^{i+\Delta} - 1}
\sum\limits_{k_2 = 0}^{2^{i+\Delta} - 1} 2^{-2M'\Delta}
\fr{\mathbf{1}(k_1 \neq 2^{\Delta} l_1,k_2 \neq 2^{\Delta} l_2)}
{\left\vert k_1 - 2^{\Delta} l_1\right\vert^{2M_1} \left\vert k_2 - 2^{\Delta} l_2
\right\vert^{2M_1}} \right) \simeq{}
\\
{}\simeq C'^2 \cdot2^{2J(1-\alpha)} \!\sum\limits_{i = p''J + 1}^{J - 1} 2^{2i\alpha}
\sum\limits_{l_1 = 0}^{2^i - 1} \sum\limits_{l_2 = 0}^{2^i - 1} \left( H^2_0
+ \sum\limits_{\Delta = 1}^{J-i-1}2^{-2M'\Delta}
\sum\limits_{k_1 = 0}^{2^{i+\Delta} - 1}
\fr{\mathbf{1}\left(k_1 \neq 2^{\Delta} l_1\right)}
{\left\vert k_1 -2^{\Delta}l_1\right\vert^{2M_1}}
\sum\limits_{k_2 = 0}^{2^{i+\Delta} - 1} \fr{\mathbf{1}\left(k_2 \neq 2^{\Delta} l_2\right)}
{\left\vert k_2 -2^{\Delta}l_2\right\vert^{2M_1}} \right)\simeq{}\\
{}\simeq  C'^2 \cdot 2^{2J(1-\alpha)} \sum\limits_{i = p''J + 1}^{J - 1}
2^{2i(\alpha+1)} \left( H_0
+ \sum\limits_{\Delta = 1}^{J-i-1}2^{-2M'\Delta} H_1 \right)
\simeq C'^2\cdot 2^{2J(1-\alpha)}
\sum\limits_{i = p''J + 1}^{J - 1} 2^{2i(\alpha+1)} H_2
\simeq C''_\alpha \cdot 2^{4J},
\end{multline*}

\hrule
\begin{multicols}{2}

\noindent
где $H_0$, $H_1$, $H_2$ и~$C''_\alpha$~--- положительные константы
(в~случае $k_1 2^{-\Delta}\hm=l_1$ или $k_2 2^{-\Delta}\hm=l_2$
в~приведенных выкладках вместо соответствующих слагаемых используется первая,
вторая или третья оценки из~\eqref{Cov_Decay11}).

Объединяя результаты, получаем, что $\Variance \widehat{R}_J(f)\hm\simeq
\tilde{C} 2^{4J}$.
Лемма доказана.

\smallskip

Докажем еще одно свойство эмпирических вейг\-лет-ко\-эф\-фи\-ци\-ен\-тов.
Говорят, что последовательность случайных величин $\{Y_i\}_{i = 1}^\infty$
обладает свойством $\rho$-пе\-ре\-ме\-ши\-ва\-ния, если для функции
%
\begin{equation*}
\rho(m) = \sup\limits_{i,j:|i - j| > m} \abs{\corr\left(Y_i,Y_j\right)}
\end{equation*}
%
справедливо $\rho(m) \hm \to 0$ при $m \hm\to \infty$.

\smallskip

\noindent
\textbf{Лемма~2.}\
\textit{Последовательность
$\left\{F\left[\left(X_{j,k_1,k_2}^{[\lambda]}\right)^2,\right.\right.$\linebreak
$\left.\left.T_{\lambda,j},
\sigma_{\lambda,j}
\vphantom{\left(X_{j,k_1,k_2}^{[\lambda]}\right)^2}\right]\right\}$, $\lambda\hm=1,2,3, j = 0,\dots, J \hm- 1, k_1 \hm= 1,\dots, 2^j,
k_2 \hm= 1,\dots, 2^j$,
обладает свойством $\rho$-пе\-ре\-ме\-шивания,
причем для некоторой положительной константы~$C_\rho$}
$$
\rho(m) \leq
\begin{cases}
       \fr{C_\rho}{(m+1)^{2M_1}}  \,  & \mbox{ \textit{для элементов}}\\
       &\mbox{\textit{на одном уровне }} (i = j);  \\
      \fr{C_\rho}{2^{(m+1)(2M' +\alpha)}}   \,  & \mbox{ \textit{для элементов}}\\
      &\mbox{\textit{на разных уровнях}}.  \\
\end{cases}
$$

\noindent
Д\,о\,к\,а\,з\,а\,т\,е\,л\,ь\,с\,т\,в\,о\,.\ \  Сначала рассмотрим функцию
$\rho(m)$ для разных коэффициентов $X_{i,l_1,l_2}^{[\lambda]}$ на одном уровне~$i$.

Для некоторой константы $C_\rho \hm> 0$ при $k_1\hm\neq l_1$ и~$k_2\hm\neq l_2$
имеем:
%

\end{multicols}

\hrule

\begin{align}
\rho(m)
&=\sup\limits_{{\begin{smallmatrix}0\leq i\leq J-1,\\
\lambda_1, \lambda_2= 1,2,3, \\
k_1,k_2,l_1,l_2:\left[ \begin{gathered}{\scriptstyle |k_1-l_1|>m,}\\
  {\scriptstyle |k_2-l_2|>m.}\end{gathered} \right.\end{smallmatrix}}}
   \fr{\abs{\cov\left(F\left[\left(X_{i,l_1,l_2}^{[\lambda_1]}\right)^2,
   T_{\lambda_1,i},\sigma_{\lambda_1,i}\right],F\left[\left(X_{i,k_1,k_2}^{[\lambda_2]}\right)^2,
   T_{\lambda_2,i},\sigma_{\lambda_2,i}\right]\right)}}
   {\sqrt{\Variance F\left[\left(X_{i,l_1,l_2}^{[\lambda_1]}\right)^2,
   T_{\lambda_1,i},\sigma_{\lambda_1,i}\right]
   \Variance F\left[\left(X_{i,k_1,k_2}^{[\lambda_2]}\right)^2,T_{\lambda_2,i},
   \sigma_{\lambda_2,i}\right]}}\leq{}\notag\\
&\leq \sup\limits_{{\begin{smallmatrix}0\leq i\leq
J-1,\\k_1,k_2,l_1,l_2:\left[ \begin{gathered}
{\scriptstyle |k_1-l_1|>m,}\\
{\scriptstyle |k_2-l_2|>m.}
\end{gathered} \right.\end{smallmatrix}}}
C_\rho\fr{2^{2J(1-\alpha)}\cdot2^{2i\alpha}\left\vert k_1-l_1\right\vert^{-2M_1}
\left\vert k_2-l_2\right\vert^{-2M_1}}
{\sqrt{\sigma^4_i\sigma^4_i}}\leq{}\notag\\
&{}\leq \sup\limits_{{\begin{smallmatrix}0\leq i\leq J-1,\\
k_1,k_2,l_1,l_2:\left[ \begin{gathered}
{\scriptstyle |k_1-l_1|>m,}\\ {\scriptstyle  |k_2-l_2|>m.}\end{gathered}
\right.\end{smallmatrix}}}
C_\rho \fr{2^{2J(1-\alpha)}\cdot 2^{2i\alpha}\left\vert k_1-l_1\right\vert^{-2M_1}
\left\vert k_2-l_2\right\vert^{-2M_1}}
{2^{2J(1-\alpha)}\cdot 2^{2i\alpha}}\leq{}\notag\\
&{}\leq \sup\limits_{{\begin{smallmatrix}0\leq
i\leq J-1,\\k_1,k_2,l_1,l_2:\left[
\begin{gathered}
{\scriptstyle |k_1-l_1|>m,}\\ {\scriptstyle  |k_2-l_2|>m.}\end{gathered}
\right.\end{smallmatrix}}}
C_\rho\fr{1}{\left\vert k_1-l_1\right\vert^{2M_1}\left\vert k_2-l_2\right\vert^{2M_1}}
\leq \fr{C_\rho}{(m+1)^{2M_1}} \,.
\label{rho_1}
\end{align}
%
Случаи $k_1 = l_1$ или $k_2 = l_2$ рассматриваются аналогично.

\pagebreak

Далее рассмотрим функцию перемешивания для элементов
$$
F\left[\left(X_{i,l_1,l_2}^{[\lambda_1]}\right)^2,T_{\lambda_1,i},
\sigma_{\lambda_1,i}\right],\enskip
F\left[\left(X_{j,k_1,k_2}^{[\lambda_2]}\right)^2,T_{\lambda_2,j},\sigma_{\lambda_2,j}\right]
$$
на разных уровнях $i,j: j\hm>i, j\hm-i\hm=\Delta\hm>0$.
Рассмотрим случай $\left\vert k_{1}2^{-\Delta} \hm\neq l_{1}\right\vert$,
$\left\vert k_{2}2^{-\Delta} \hm\neq l_{2}\right\vert$:

\noindent
\begin{align}
\rho(m)
&=\sup\limits_{{\begin{smallmatrix}j-i=\Delta>m,\\
\lambda_1, \lambda_2= 1,2,3,\\ 0\leq l_1,l_2\leq2^i-1,
0\leq k_1,k_2\leq2^j-1,\\|k_{1}\cdot2^{-\Delta}\neq
l_{1}|,|k_{2}\cdot2^{-\Delta}\neq l_{2}|.\end{smallmatrix}}}
\fr{\abs{\cov\left(F\left[\left(X_{i,l_1,l_2}^{[\lambda_1]}\right)^2,T_{\lambda_1,i},
\sigma_{\lambda_1,i}\right],
F\left[\left(X_{j,k_1,k_2}^{[\lambda_2]}\right)^2,T_{\lambda_2,j},
\sigma_{\lambda_2,j}\right]\right)}}
{\sqrt{\Variance F\left[\left(X_{i,l_1,l_2}^{[\lambda_1]}\right)^2,T_{\lambda_1,i},
\sigma_{\lambda_1,i}\right] \Variance F
\left[\left(X_{j,k_1,k_2}^{[\lambda_2]}\right)^2,
T_{\lambda_2,j},\sigma_{\lambda_2,j}\right]}}\leq{}\notag\\
&{}\leq \sup\limits_{{\begin{smallmatrix}j-i=\Delta>m,\\
0\leq l_1,l_2\leq2^i-1, 0\leq k_1,k_2\leq2^j-1,\\
|k_{1}\cdot2^{-\Delta}\neq l_{1}|,|k_{2}\cdot2^{-\Delta}\neq l_{2}|.
\end{smallmatrix}}} C_\rho\fr{2^{2J(1-\alpha)}\cdot2^{2i\alpha}\cdot2^{-2 M' \Delta}
\left\vert \left(k_1 - 2^{\Delta}l_1\right)^{-2M_1} \left(k_2-
2^{\Delta}l_2\right)^{-2M_2}\right\vert}
{2^{J(1-\alpha)}\cdot 2^{i\alpha}\cdot 2^{J(1-\alpha)}\cdot2^{(i+\Delta)\alpha}}\leq{}\notag\\
&\hspace*{-10mm}{}\leq \sup\limits_{{\begin{smallmatrix}j-i=\Delta>m,\\
0\leq l_1,l_2\leq2^i-1, 0\leq k_1,k_2\leq2^j-1,\\
|k_{1}\cdot2^{-\Delta}\neq l_{1}|,|k_{2}\cdot2^{-\Delta}\neq l_{2}|.
\end{smallmatrix}}} \fr{C_\rho\cdot 2^{-2M'\Delta}}{2^{\Delta\alpha}}
\leq \sup\limits_{{\begin{smallmatrix}
j-i=\Delta>m,\\ 0\leq l_1,l_2\leq2^i-1, 0\leq k_1,k_2\leq2^j-1,\\
|k_{1}\cdot2^{-\Delta}\neq l_{1}|,|k_{2}\cdot2^{-\Delta}\neq l_{2}|.
\end{smallmatrix}}}\fr{C_\rho}{2^{\Delta(2M' +\alpha)}}
=\fr{C_\rho}{2^{(m+1)(2M' +\alpha)}}\,.
\label{rho_2}
\end{align}

\hrule

\vspace*{2pt}

\begin{multicols}{2}

\noindent
Аналогично рассматриваются случаи, когда выполнено хотя бы одно из равенств
 $\left\vert k_{1}\cdot2^{-\Delta} \hm= l_{1}\right\vert$,
 $\left\vert k_{2}\cdot 2^{-\Delta} \hm= l_{2}\right\vert$.

Из~\eqref{rho_1} и~\eqref{rho_2} следует утверждение леммы.

\vspace*{-6pt}


\section{Основная теорема}

Докажем асимптотическую нормальность оценки риска.

\smallskip

\noindent
\textbf{Теорема.}
\textit{Пусть $0\hm<\alpha<1$ и~функция~$f$ равномерно регулярна по Липшицу
с~параметром $\gamma\hm>(1\hm+\alpha)^{-1}$. Тогда при пороговой обработке
с~универсальным порогом $T_{\lambda,j}$ имеет место сходимость по распределению}:
\begin{equation}
\label{asnorm}
\fr{\widehat{R}_J(f) - R_J(f)}{ D_J } \Rightarrow \textbf{N}(0,1)\,, \enskip
J \rightarrow \infty\,,
\end{equation}
\textit{где $D_J^2 = \tilde{C} 2^{4J}$, а~константа $\tilde{C}$
зависит от~$\alpha$ и~выбранного вейв\-лет-ба\-зиса}.

\smallskip

\noindent
Д\,о\,к\,а\,з\,а\,т\,е\,л\,ь\,с\,т\,в\,о\,.\ \
Из леммы~1 следует, что $\Variance \widehat{R}_J(f)\hm\simeq D_J^2
\hm = \tilde{C}\cdot 2^{4J}$.
Разобьем выражение в~(\ref{asnorm}) на две суммы, как и~в лемме~1:


\noindent
\begin{multline*}
\fr{\widehat{R}_J(f) - R_J(f)}{ D_J } ={}\\
{}=\fr{1}{D_j}\left(
\sum\limits_{j = 0}^{p''J}\sum\limits_{k_1 = 0}^{2^j - 1}
\sum\limits_{k_2 = 0}^{2^j - 1}\sum\limits_{\lambda = 1}^{3}
\left(F\left[\left(X_{j,k_1,k_2}^{[\lambda]}\right)^2,\right.\right.\right.
\\
\left.\left.\left.T_{\lambda,j},
\sigma_{\lambda,j}
\vphantom{\left(X_{j,k_1,k_2}^{[\lambda]}\right)^2}
\right] -
\Expect F\left[\left(X_{j,k_1,k_2}^{[\lambda]}\right)^2,T_{\lambda,j},\sigma_{\lambda,j}
\right]
\right)\!
\vphantom{\sum\limits_{j = 0}^{p''J}\sum\limits_{k_1 = 0}^{2^j - 1}}
\right) +{}
\end{multline*}

\noindent
\begin{multline*}
{}+\fr{1}{D_j}\left( \sum\limits_{j = p''J+1}^{J-1}\sum\limits_{k_1 = 0}^{2^j - 1}
\sum\limits_{k_2 = 0}^{2^j - 1}\sum\limits_{\lambda = 1}^{3}
\left(F\left[\left(X_{j,k_1,k_2}^{[\lambda]}\right)^2,\right.\right.\right.\\[1pt]
\left.\left.\left.T_{\lambda,j},\sigma_{\lambda,j}
\vphantom{\left(X_{j,k_1,k_2}^{[\lambda]}\right)^2}
\right] -
 \Expect
F\left[\left(X_{j,k_1,k_2}^{[\lambda]}\right)^2,T_{\lambda,j},\sigma_{\lambda,j}
\right]\right)\!
\vphantom{\sum\limits_{j = 0}^{p''J}\sum\limits_{k_1 = 0}^{2^j - 1}}
\right)\,,
\end{multline*}
где ${1}/(1+\gamma) < p'' <(1+\alpha)/(2+\alpha)$. Тогда
вследствие~(\ref{Large_Coeff_Order}) первая сумма стремится к~нулю
п.~в.

Из леммы~2 следует, что последовательность
$\left\{F\left[\left(X_{j,k_1,k_2}^{[\lambda]}\right)^2,T_{\lambda,j},\sigma_{\lambda,j}\right]\right\}$,
$\lambda\hm=1,2,3$, $j \hm= 0,\dots, J \hm- 1$, $k_1 \hm= 1,\dots, 2^j$, $k_1 \hm=
1,\dots, 2^j$, обладает свойством $\rho$-пе\-ре\-ме\-ши\-ва\-ния
и,~следовательно, обладает свойством
$\alpha$-пе\-ре\-ме\-ши\-ва\-ния~[8].

Далее, действуя, как в~лемме~1, можно показать, что

\noindent
\begin{multline*}
\sup\limits_{J > 0} \fr{1}{D^2_J}
\sum\limits_{j = p''J + 1}^{J - 1}
\sum\limits_{k_1 = 0}^{2^j - 1} \sum\limits_{k_2 = 0}^{2^j - 1}
\sum\limits_{\lambda = 1}^{3} \Expect
\left( F\left[\left(X_{j,k_1,k_2}^{[\lambda]}\right)^2,\right.\right.\\[1pt]
\left.\left.T_{\lambda,j},\sigma_{\lambda,j}
\vphantom{\left(X_{j,k_1,k_2}^{[\lambda]}\right)^2}\right]
-
\Expect  F\left[\left(X_{j,k_1,k_2}^{[\lambda]}\right)^2,T_{\lambda,j},
\sigma_{\lambda,j}\right]\right)^2
= {}
\\[1pt]
{}=\sup\limits_{J > 0} \fr{1}{D^2_J}\sum\limits_{j = p''J + 1}^{J - 1}
\sum\limits_{k_1 = 0}^{2^j - 1} \sum\limits_{k_2 = 0}^{2^j - 1}
\sum\limits_{\lambda = 1}^{3} \Variance
F\left[\left(X_{j,k_1,k_2}^{[\lambda]}\right)^2,\right.\\[1pt]
\left.T_{\lambda,j},\sigma_{\lambda,j}
\left(X_{j,k_1,k_2}^{[\lambda]}\right)^2\right]
\leq \sup\limits_{J > 0} \fr{\tilde{C}'_\alpha \cdot2^{4J}}{\hat{C}\cdot 2^{4J}}
\leq \infty\,.
\end{multline*}
%

Также можно показать, что выполнено условие Линдеберга:
для любого $\epsilon \hm> 0$

\noindent
\begin{multline}
\label{Norm_Cond2}
\fr{1}{D^2_J} \sum\limits_{j = p''J + 1}^{J - 1}
\sum\limits_{k_1=0}^{2^j - 1}\sum\limits_{k_2=0}^{2^j - 1}
\sum\limits_{\lambda = 1}^{3} \Expect
\left( \!F\!\left[\left(X_{j,k_1,k_2}^{[\lambda]}\right)^2\!\!,\right.\right.\\
\left.\left.T_{\lambda,j},\sigma_{\lambda,j}
\vphantom{\left(X_{j,k_1,k_2}^{[\lambda]}\right)^2}
\right] -\Expect F\left[\left(X_{j,k_1,k_2}^{[\lambda]}\right)^2,
T_{\lambda,j},\sigma_{\lambda,j}\right]\right)^2\;\times\\
{}\times\mathbf{1}
\left( \left\vert F\left[\left(X_{j,k_1,k_2}^{[\lambda]}\right)^2,
T_{\lambda,j},\sigma_{\lambda,j}\right] -{}\right.\right.\\
\left.\left.{}-
\Expect  F\left[\left(X_{j,k_1,k_2}^{[\lambda]}\right)^2,T_{\lambda,j},\sigma_{\lambda,j}\right]\right\vert
> \epsilon D_J\right) \rightarrow 0\,,\\ J\rightarrow\infty.
\end{multline}


Действительно, поскольку
\begin{multline*}
\hspace*{-6pt}\left\vert F\left[\left(X_{j,k_1,k_2}^{[\lambda]}\right)^2,
T_{\lambda,j},\sigma_{\lambda,j}\right]\right\vert
\leq T_{\lambda,j}^2 =2\ln2^{2j}\sigma^2_{\lambda,j}
= {}\\
{}=\tilde{C}_{\lambda,\alpha} j\cdot 2^{J(1-\alpha)}\cdot 2^{j\alpha}
\end{multline*}
с~некоторой константой $\tilde{C}_{\lambda,\alpha}\hm>0$
и~$D_J^2 \hm\simeq \hat{C} \cdot2^{4J}$, то начиная с~некоторого~$J$ все
индикаторы в~(\ref{Norm_Cond2}) обращаются в~ноль.

Таким образом, выполнены все условия теоремы~2.1 из работы~[9]
и~справедлива сходимость по распределению~(\ref{asnorm}). Теорема доказана.



{\small\frenchspacing
 {%\baselineskip=10.8pt
 \addcontentsline{toc}{section}{References}
 \begin{thebibliography}{9}


\bibitem{1-ero}
\Au{Маркин А.\,В., Шестаков О.\,В.} Асимптотики оценки риска
при пороговой обработке вейв\-лет-вейг\-лет коэффициентов в~задаче
томографии~// Информатика и~её применения, 2010. Т.~4. Вып.~2.
С.~36--45.

\bibitem{2-ero}
\Au{Donoho D.} Nonlinear solution of linear inverse problems
by wavelet-vaguelette decomposition~// Appl. Comput.
Harmon. Anal., 1995. Vol.~2. P.~101--126.

\bibitem{3-ero}
\Au{Добеши И.} Десять лекций по вейвлетам~/
Пер. с англ.~--- Ижевск: НИЦ
Регулярная и~хаотическая динамика, 2001. 357~с.
(\Au{Daubechies~I.} Ten lectures on wavelets.
CBMF-NSF regional
conference ser. in applied mathematics. SIAM, 1992. 369~p.)

\bibitem{4-ero}
\Au{Mallat S.} A~wavelet tour of signal processing.~---
Academic Press, 1999. 662~p.

\bibitem{5-ero}
\Au{Johnstone I.\,M., Silverman~B.\,W.} Wavelet threshold
estimates for data with correlated noise~// J.~Roy. Stat. Soc.
 B, 1997. Vol.~59. P.~319--351.

\bibitem{6-ero}
\Au{Kolaczyk E.\,D.} Wavelet methods for the inversion of
certain homogeneous linear operators in the presence of noisy data.~---
Stanford: Stanford University, 1994. Ph.D. Dissertation. 163~p.

\bibitem{7-ero}
\Au{Ерошенко А.\,А., Шестаков~О.\,В.} Асимптотические свойства
оценки риска при пороговой обработке вейв\-лет-ко\-эф\-фи\-ци\-ен\-тов в~модели
с~коррелированным шумом~// Информатика и~её применения, 2014. Т.~8.
Вып.~1. С.~36--44.

\bibitem{8-ero}
\Au{Bradley R.\,C.} Basic properties of strong mixing
conditions. A~survey and some open questions~// Probab. Surveys,
2005. Vol.~2. P.~107--144.

\bibitem{9-ero}
\Au{Peligrad M.} On the asymptotic normality of sequences of
weak dependent random variables~// J.~Theor. Probab., 1996. Vol.~9.
No.~3. P.~703--715.
 \end{thebibliography}

 }
 }

\end{multicols}

\vspace*{-9pt}

\hfill{\small\textit{Поступила в редакцию 29.09.14}}

%\newpage

\vspace*{12pt}

\hrule

\vspace*{2pt}

\hrule

%\vspace*{12pt}

\def\tit{ASYMPTOTIC PROPERTIES OF~RISK ESTIMATE IN~THE~PROBLEM
OF~RECONSTRUCTING IMAGES WITH CORRELATED NOISE BY~INVERTING THE~RADON TRANSFORM}

\def\titkol{Asymptotic properties of risk estimate in~the~problem
of~reconstructing images} % with correlated noise by~inverting the Radon transform}

\def\aut{A.\,A.~Eroshenko$^1$ and~O.\,V.~Shestakov$^{1,2}$}

\def\autkol{A.\,A.~Eroshenko and~O.\,V.~Shestakov}

\titel{\tit}{\aut}{\autkol}{\titkol}

\vspace*{-9pt}

\noindent
$^1$Faculty of Computational Mathematics and Cybernetics,
M.\,V.~Lomonosov Moscow State University,
1-52~Lenin-\linebreak
$\hphantom{^1}$skiye Gory, GSP-1, Moscow 119991, Russian Federation

\noindent
$^2$Institute of Informatics Problems, Russian Academy of Sciences,
44-2~Vavilov Str., Moscow 119333, Russian\linebreak
$\hphantom{^1}$Federation


\def\leftfootline{\small{\textbf{\thepage}
\hfill INFORMATIKA I EE PRIMENENIYA~--- INFORMATICS AND
APPLICATIONS\ \ \ 2014\ \ \ volume~8\ \ \ issue\ 4}
}%
 \def\rightfootline{\small{INFORMATIKA I EE PRIMENENIYA~---
INFORMATICS AND APPLICATIONS\ \ \ 2014\ \ \ volume~8\ \ \ issue\ 4
\hfill \textbf{\thepage}}}

\vspace*{3pt}


\Abste{In recent years, wavelet methods based on the decomposition
of projections in a special basis and the following thresholding
procedure became widely used for solving the problems of tomographic
image reconstruction. These methods are easily implemented through
fast algorithms; so, they are very appealing in practical situations.
Besides, they allow the reconstruction of local parts of the images using
incomplete projection data, which is essential, for example, for medical
applications, where it is not desirable to expose the patient to the redundant
radiation dose. Wavelet thresholding risk analysis is an important practical
task, because it allows determining the quality of techniques themselves and
the equipment which is used. The present paper considers\linebreak\vspace*{-12pt}}

\Abstend{the problem of
estimating the function by inverting the Radon transform in the model of
data with correlated noise. The asymptotic properties of mean-square risk
estimate of wavelet-vaguelette thresholding technique are studied. The conditions
under which the unbiased risk estimate is asymptotically normal are given.}

\KWE{wavelets; linear homogeneous operator; Radon transform;
thresholding; unbiased risk estimate;
correlated noise; asymptotic normality}


\DOI{10.14357/19922264140404}

\vspace*{-3pt}

\Ack
\noindent
The research was financially supported by the Russian Science Foundation (project
14-11-00364).

%\vspace*{3pt}

  \begin{multicols}{2}

\renewcommand{\bibname}{\protect\rmfamily References}
%\renewcommand{\bibname}{\large\protect\rm References}



{\small\frenchspacing
 { %\baselineskip=10pt
 \addcontentsline{toc}{section}{References}
 \begin{thebibliography}{9}



\bibitem{1-ero-1}
\Aue{Markin, A.\,V., and O.\,V.~Shestakov}. 2010.
Asimptotiki otsen\-ki riska pri porogovoy obrabotke veyvlet-veyglet
koeffitsientov v~zadache tomografii
[Asymptotic properties of risk estimate for wavelet-vaguelette coefficients
thresholding in  tomographic reconstruction problem].
\textit{Informatika i ee Primeneniya}~--- \textit{Inform. Appl.} 4(2):36--45.

%\vspace*{-3pt}

\bibitem{2-ero-1}
\Aue{Donoho, D.} 1995. Nonlinear solution of linear inverse problems
by wavelet-vaguelette decomposition.
\textit{Appl. Comput. Harmon. Anal.}  2:101--126.

%\vspace*{-3pt}

\bibitem{3-ero-1}
\Aue{Daubechies, I.} 1992.\textit{Ten lectures on wavelets.}
CBMF-NSF regional
conference ser. in applied mathematics.
SIAM. 369~p.

%\vspace*{-3pt}

\bibitem{4-ero-1}
\Aue{Mallat, S.}  1999.
\textit{A~wavelet tour of signal processing.} Academic Press. 662~p.

\bibitem{5-ero-1}
\Aue{Johnstone, I.\,M., and B.\,W.~Silverman}. 1997.
Wavelet threshold estimates for data with correlated noise.
\textit{J.~Roy. Stat. Soc. B}  59:319--351.

\bibitem{6-ero-1}
\Aue{Kolaczyk, E.\,D.} 1994. Wavelet
methods for the inversion of certain homogeneous linear operators in
the presence of noisy data.  Stanford: Stanford University. Ph.D. Thesis. 163~p.

\smallskip

\bibitem{7-ero-1}
\Aue{Eroshenko, A.\,A., and O.\,V.~Shestakov}. 2014.
Asimptoticheskie svoystva otsenki riska pri porogovoy obra-\linebreak botke
veyvlet-koeffitsientov v~modeli s~korrelirovannym\linebreak shumom
[Asymptotic properties of wavelet thresholding risk estimate in
the model of data with correlated noise].
\textit{Informatika i ee Primeneniya}~--- \textit{Inform. Appl.} 8(1):\linebreak 36--44.

\smallskip

\bibitem{8-ero-1}
\Aue{Bradley, R.\,C.} 2005.
Basic properties of strong mixing conditions.
A~survey and some open questions. \textit{Probab. Surveys}  2:107--144.

\smallskip

\bibitem{9-ero-1}
\Aue{Peligrad, M.} 1996.
On the asymptotic normality of sequences of weak dependent random variables.
\textit{J.~Theor. Probab.} 9(3):703--715.
\end{thebibliography}

 }
 }

\end{multicols}

\vspace*{-6pt}

\hfill{\small\textit{Received September 29, 2014}}

\vspace*{-18pt}

\Contr

\noindent
\textbf{Eroshenko Alexander A.} (b.\ 1989)~---
PhD student, Department of Mathematical Statistics,
Faculty of Computational Mathematics and Cybernetics,
M.\,V.~Lomonosov Moscow State University,
1-52  Leninskiye Gory, GSP-1, Moscow 119991, Russian Federation;
aeroshik@gmail.com

\vspace*{3pt}

\noindent
\textbf{Shestakov Oleg V.} (b.\ 1976)~---
Doctor of Science in physics and mathematics, associate professor,
%Department of Mathematical Statistics,
Faculty of Computational Mathematics and Cybernetics,
M.\,V.~Lomonosov Moscow State University,
1-52  Leninskiye Gory, GSP-1, Moscow 119991, Russian Federation;
senior scientist, Institute of Informatics Problems, Russian Academy of Sciences,
44-2 Vavilov Str., Moscow 119333, Russian Federation;
oshestakov@cs.msu.su

\label{end\stat}

\renewcommand{\bibname}{\protect\rm Литература} %4
\def\stat{grusho}

\def\tit{АРХИТЕКТУРНЫЕ РЕШЕНИЯ В~ЗАДАЧЕ ВЫЯВЛЕНИЯ МОШЕННИЧЕСТВА ПРИ~АНАЛИЗЕ 
ИНФОРМАЦИОННЫХ ПОТОКОВ В~ЦИФРОВОЙ ЭКОНОМИКЕ$^*$}

\def\titkol{Архитектурные решения в~задаче выявления мошенничества при~анализе 
информационных потоков в
%~цифровой 
экономике}

\def\aut{А.\,А.~Грушо$^1$, М.\,И.~Забежайло$^2$, Н.\,А.~Грушо$^3$, 
Е.\,Е.~Тимонина$^4$}

\def\autkol{А.\,А.~Грушо, М.\,И.~Забежайло, Н.\,А.~Грушо, 
Е.\,Е.~Тимонина}

\titel{\tit}{\aut}{\autkol}{\titkol}

\index{Грушо А.\,А.}
\index{Забежайло М.\,И.}
\index{Грушо Н.\,А.}
\index{Тимонина Е.\,Е.}
\index{Grusho A.\,A.}
\index{Zabezhailo M.\,I.}
\index{Grusho N.\,A.}
\index{Timonina E.\,E.}


{\renewcommand{\thefootnote}{\fnsymbol{footnote}} \footnotetext[1]
{Работа частично поддержана РФФИ (проекты 18-29-03081 и~18-07-00274).}}


\renewcommand{\thefootnote}{\arabic{footnote}}
\footnotetext[1]{Институт проблем информатики Федерального исследовательского центра <<Информатика и~управление>> 
Российской академии наук, grusho@yandex.ru}
\footnotetext[2]{Институт проблем информатики Федерального исследовательского центра <<Информатика и~управление>> 
Российской академии наук, m.zabezhailo@yandex.ru}
\footnotetext[3]{Институт проблем информатики Федерального исследовательского центра <<Информатика и~управление>> 
Российской академии наук, info@itake.ru}
\footnotetext[4]{Институт проблем информатики Федерального исследовательского центра <<Информатика и~управление>> 
Российской академии наук, eltimon@yandex.ru}

\vspace*{-12pt}
   

 
  
  \Abst{Cформулирован подход к~исследованию некоторых видов мошенничества в~цифровой 
экономике с~использованием причинно-следственных связей. Во всех видах рассматриваемых 
мошенничеств должно наблюдаться несоответствие между целями финансовых транзакций 
и~реальной стоимостью достижения этих целей. Данные о транзакциях можно собирать, 
наблюдая информационные потоки, в~которых отражаются эти транзакции. Архитектура сбора 
данных и~их анализа может быть организована с~помощью распределенных реестров 
с~централизованным консенсусом, что позволяет создать аналог электронной бухгалтерской 
книги, фиксирующей финансово-экономическую деятельность субъектов цифровой экономики в~регионе. 
  Рассматриваемые методы выявления мошенничества основаны на противоречиях 
между действиями, описанными в~транзакциях, и~информацией, содержащейся в~планах, 
стандартах, прецедентах и~др. Рассмотрен метод, основанный на некоторой упрощенной схеме 
реализации абстрактного проекта. Для выявления противоречий необходимо проводить анализ 
от следствия к~причине, т.\,е.\ искать аномалии в~информации, описывающей порождение 
наблюдаемых следствий. 
  Показано, как в~реализации проекта можно выделять простые <<необходимые условия>> 
нарушения при\-чин\-но-след\-ст\-вен\-ных связей, т.\,е.\ множество <<необходимых условий>>, 
нарушение которых свидетельствует о наличии мошенничества. Это множество <<необходимых 
условий>> можно назвать метаданными для контроля проекта на выявление мошенничества.} 
 
 
  \KW{цифровая экономика; информационные потоки; при\-чин\-но-след\-ст\-вен\-ные связи; 
выявление мошеннических схем} 

\DOI{10.14357/19922264190204}
  
\vspace*{-4pt}


\vskip 10pt plus 9pt minus 6pt

\thispagestyle{headings}

\begin{multicols}{2}

\label{st\stat}

\section{Введение}

\vspace*{3pt}

  В работе сформулирован подход к~исследованию некоторых видов 
мошенничества в~цифровой экономике с~использованием  
при\-чин\-но-след\-ст\-вен\-ных связей. Рассматриваются три вида мошенничества, 
а именно:
  \begin{enumerate}[(1)]
\item отмыв денег; 
\item обман при выполнении договорных обязательств при реализации 
технических проектов (строительные проекты и~др.); 
\item незаконный вывод денег. 
\end{enumerate}

  Названные виды мошенничества могут быть сведены к~решению одного типа 
задач. Для отмывания денег источник должен заключать фиктивные контракты, 
в~соответствии с~которыми будут переводиться средства за заведомо ненужную 
работу и~материалы. 
  
  Мошенничество, связанное с~невыполнением договорных обязательств, связано 
со снижением качества услуг, качества и~количества закупаемых 
материалов, выполнением работ с~ненадлежащим качеством. 
  
  Вывод денег связан с~переводом средств фир\-мам-од\-но\-днев\-кам, которые 
заведомо не могут выполнить обязательства по контрактам, за которые им 
переводятся средства. 
  
  Таким образом, во всех трех видах рассматриваемых мошенничеств должно 
наблюдаться несоответствие между целями финансовых транзакций и~реальной 
стоимостью достижения этих целей. Данные о транзакциях можно собирать, 
наблюдая информационные потоки, в~которых отражаются эти транзакции. 
  
  Однако для наблюдения таких информационных потоков необходимо создавать 
архитектуру\linebreak телекоммуникационной системы, позволяющей перехватывать 
и~собирать данные о всех транзакциях. Например, такая архитектура может быть 
организована с~помощью распределенных реестров с~централизованным 
консенсусом, т.\,е.\ все информационные потоки, сформированные в~цифровой 
экономике и~несущие информацию о транзакциях, проходят через некоторый 
центральный узел, запоминающий их в~форме распределенного реестра. Такие 
реестры могут дублироваться в~аналогичных центрах различных регионов, что 
позволяет создать аналог электронной бухгалтерской книги, фиксирующей 
фи\-нан\-со\-во-эко\-но\-ми\-че\-скую деятельность субъектов цифровой экономики. Такой 
подход предложено реализовать на базе системы ситуационных центров, что 
отражено в~работах~[1, 2].
  
  Собранная из информационных потоков информация о~транзакциях, т.\,е.\ 
о~контрактах, договорах, платежах, отчетах, закупленных материалах, 
характеристиках исполнителей работ и~др., собирается в~базе данных в~указанном 
центре. Согласно теории интеллектуальных сис\-тем~[3], эту базу данных можно 
называть базой фактов (БФ). Базу фактов можно представить как бинарную мат\-ри\-цу, 
строки которой описывают характеристики, входящие в~транзакции, а столбцы 
нумеруются характеристиками. Строки матрицы будем называть 
\textit{объектами}~[4, 5]. 
  
  Рассматриваемые в~работе методы выявления мошенничества будут основаны 
на противоречиях между действиями, описанными в~транзакциях, и~информацией, 
содержащейся в~планах, стандартах, прецедентах и~др. Для нахождения 
противоречий в~архитектуре центра предусмотрена другая база данных~--- база 
знаний (БЗ)~\cite{3-gr, 6-gr}, которая устроена так же, как БФ. 
  
  Информация в~БЗ собирается на основе положительного опыта или расчетов. 
Используя БЗ, можно выводить факты нарушения при\-чин\-но-след\-ст\-вен\-ных 
связей. Нарушения при\-чин\-но-след\-ст\-вен\-ных связей будем называть 
\textit{аномалиями}. 
  
  Для упрощения дальнейшее изложение будет вестись в~рамках поиска 
противоречий при выполнении некоторого абстрактного проекта. Выявление 
аномалий будет происходить на основе фактов из БФ с~помощью знаний из БЗ 
методами искусственного интеллекта и~интеллектуального анализа 
данных~\cite{6-gr}. 

\vspace*{-10pt}
  
  \section{Модели}
  
  \vspace*{-3pt}
  
  Наиболее сложная из рассмотренных выше задач~--- выявление противоречий, 
т.\,е.\ использование БЗ для получения новых знаний и~выявление аномалий из 
полученных фактов. 
  
  Все способы выявления противоречий основаны на определении 
  причинно-следственных связей. При этом противоречия в~параметрах транзакций по 
отношению к~требуемым в~БЗ составляют сущность аномалий. 
  
   Далее будет рассмотрен метод, основанный на некоторой упрощенной схеме 
реализации абстрактного проекта. 
  
  Каждый проект имеет цель: например, цель представляет собой построение 
некоторой системы. Воспользуемся структурным подходом, который позволяет 
строить проект на основе разбиения системы на подсистемы и~определения 
взаимодействий подсистем~\cite{7-gr}. При этом каждая подсистема также 
представима структурной моделью. 
  
  Как сама система, так и~каждая ее подсистема имеют свой функционал 
и~спецификацию, па\-ра\-мет\-ры настройки и~домены параметров настройки. Кроме 
этих характеристик существует множество характеристик, связанных 
с~<<жизненным циклом>> создания системы. Сюда входят работы, ресурсы, 
сроки выполнения работ по созданию подсистем и~самой системы, стоимости 
компонентов и~материалов, стоимости работ, схемы поставок, договорные 
обязательства и~др. Все характеристики связаны между собой, поэтому можно 
говорить о стоимости и~времени изготовления структурных компонентов системы. 
  
  Одной из важнейших характеристик является смета (система смет для 
подсистем). Смета сопоставляет каждому компоненту системы стоимость его 
изготовления и~настройки. 
  
  Схема построения системы может быть пред\-став\-ле\-на диаграммой, 
изображенной на рис.~1. 

{ \begin{center}  %fig1
 \vspace*{9pt}
   \mbox{%
 \epsfxsize=79mm 
 \epsfbox{gru-1.eps}
 }


\vspace*{9pt}


\noindent
{{\figurename~1}\ \ \small{Диаграмма достижения цели}}
\end{center}
}

\vspace*{9pt}

\addtocounter{figure}{1}
  
  


  Представленная на рис.~1 диаграмма позволяет описать основные классы 
возможных противоречий при достижении цели. Противоречия возникают, когда 
данные БФ не соответствуют требуемым характеристикам. 
  
  
  \section{Потенциальные классы аномалий при~достижении цели}
  
  Выделим четыре потенциальных класса противоречий, которые показывают, 
каким образом нужно искать эти противоречия.
  
 
  Противоречие цели и~проекта (рис.~2) возникает при отсутствии обоснования 
или в~случае логического противоречия между возможностями проектируемого 
функционала и~целью системы. Отметим, что в~проект входят сроки, перечень 
работ, материалы, настройки, которые описываются соответствующими 
параметрами и~допустимыми значениями этих параметров. Проект формируется 
на основе БЗ и~расчетов, исходя из информации, полученной по аналогии 
с~другими проектами и~решениями, которые считаются апробированными. 
  
  Отметим, что цель порождает проект и~в этом смысле является причиной 
проекта. Однако для анализа противоречий необходимо двигаться по штриховой 
стрелке диаграммы (см.\ рис.~2) от проекта к~цели. В~самом деле, любой компонент 
проекта направлен на теоретическое достижение цели. Цель~--- сложный объект, 
поэтому в~проекте могут возникнуть характеристики, противоречащие хотя бы 
некоторым характеристикам цели. Это делает проект противоречивым, но вывод 
об этом может быть сделан только на уровне описания цели. 
  

  Противоречия между проектом и~его реализацией, исключая настройки 
(рис.~3), могут возникать, например, при закупке исполнителем материалов более 
низкого качества по более низким ценам, при попытках достижения требуемых 
сроков работы за счет снижения качества выполнения работ, за счет нахождения 
<<объективных>> причин для увеличения сроков работы и,~следовательно, 
увеличения цены реализации проекта. 


  Для выявления указанных противоречий необходимо двигаться по диаграмме 
(см.\ рис.~3) в~обратную сторону в~соответствии со~штриховыми стрелками. 
Действительно, выявить противоречия между характеристиками закупленных 
материалов и~требуемыми по проекту можно только при обращении к~проекту 
и~его спецификациям. Манипуляции со сроками работы также можно выявить 
только при обращении к~соответствующим расчетам в~проекте. Задержки в~сроках 
работы, связанные с~поставками материалов, можно определить только на 
предыдущем этапе диаграммы (см.\ рис.~3) в~описании проекта. 


  


  Противоречия между реализацией проекта и~его настройкой (рис.~4) возникает, 
когда не удается добиться требуемых значений параметров функционала, не 
удается обеспечить необходимый уровень\linebreak\vspace*{-12pt}

{ \begin{center}  %fig2
 \vspace*{-6pt}
   \mbox{%
 \epsfxsize=16mm 
 \epsfbox{gru-2.eps}
 }


\vspace*{6pt}


\noindent
{{\figurename~2}\ \ \small{Противоречия цели и~проекта}}
\end{center}
}

%\vspace*{9pt}

\addtocounter{figure}{1}

{ \begin{center}  %fig3
 \vspace*{6pt}
    \mbox{%
 \epsfxsize=79mm 
 \epsfbox{gru-3.eps}
 }


\end{center}

\vspace*{-2pt}


\noindent
{{\figurename~3}\ \ \small{Противоречия проекта и~его реализации (без настройки)}}
}

\vspace*{6pt}

\addtocounter{figure}{1}

{ \begin{center}  %fig4
 \vspace*{1pt}
   \mbox{%
 \epsfxsize=54.5mm 
 \epsfbox{gru-4.eps}
 }


\end{center}


\noindent
{{\figurename~4}\ \ \small{Противоречия реализации проекта и~его на\-стройки}}
}

%\vspace*{9pt}

\addtocounter{figure}{1}

{ \begin{center}  %fig5
 \vspace*{5pt}
    \mbox{%
 \epsfxsize=79mm 
 \epsfbox{gru-5.eps}
 }


\end{center}



\noindent
{{\figurename~5}\ \ \small{Противоречия цели и~достигнутой реализации проекта}}
}

\vspace*{6pt}

\addtocounter{figure}{1}

\noindent
 качества реализации проекта. Для 
определения противоречия в~настройках надо опять же двигаться по диаграмме 
(см.\ рис.~4) в~обратную сторону по штриховым стрелкам, так как для выявления 
характеристик результатов работы, которые не дают возможности реализации 
определенного функционала, необходимо иметь информацию о результатах этой 
работы. 


  



  Противоречие между целью и~достигнутой реализацией проекта (рис.~5) 
возникает, когда реализованная система не позволяет достичь цели. В~этом случае 
опять противоречие нужно искать, двигаясь от цели к~реальному достигнутому 
функционалу по штриховой стрелке (см.\ рис.~5).
  
  Суммируя положения, изложенные в~данном разделе, приходим к~выводу, что 
для выявления противоречий необходимо проводить анализ от следствия 
к~причине, т.\,е.\ искать аномалии в~информации, описывающей порождение 
наблюдаемых следствий. 
  
  
  \section{Связь противоречий и~причин}
  
  Прежде чем построить связь между причинами и~противоречиями, кратко 
опишем простейшую модель связи этих понятий. Причины и~противоречия будут 
сформулированы для представления компонентов системы как объектов, 
обладающих наборами известных характеристик~\cite{4-gr, 5-gr}. 
  
  Пусть $U\hm=\{\alpha, \beta, \ldots\}$~--- совокупность характеристик 
(пространство характеристик). Согласно~\cite{4-gr} \textit{объектом}~$O$ 
называется любое подмножество характеристик $O\hm\subseteq U$. Рассмотрим 
последовательность объектов, возможно в~различных пространствах 
характеристик. 
  
  \smallskip
  
  \noindent
  \textbf{Определение~1.}\ Объект~$P$ с~числом характеристик, большим или 
равным~2, является \textit{причиной} объекта (\textit{свойства})~$B$ в~цепочке 
наблюдаемых объектов тогда и~только тогда, когда выполнены следующие 
условия:
  \begin{enumerate}[(1)]
\item для каждого объекта~$C$, если $P\hm\subseteq C$, то $C\mapsto B$, где 
$C\mapsto B$ означает, что объект~$B$ присутствует в~объекте, следующем за 
объектом~$C$;
\item объект~$P$ является минимальным объектом, удовлетворяющим 
условию~1, а~именно: $\forall \alpha\hm\in P$ объект~$P\backslash \{\alpha\}$ 
не является причиной, т.\,е.\ $\exists C:\ \alpha\not\in C$, $P\backslash 
\{\alpha\}\hm\subseteq C$ и~$C\not\mapsto B$, где $C\not\mapsto B$ означает, 
что~$B$ не может содержаться в~объекте, следующем за объектом~$C$. 
\end{enumerate}

  Приведенное определение причины является упрощением причин, 
возникающих в~реальном мире. Например, реальные причины могут возникать\linebreak 
как совокупность характеристик из разных пространств. Одно следствие может 
порождаться разными причинами или возникать из внешних\linebreak и~ненаблюдаемых 
характеристик. Однако пред\-став\-лен\-ная далее формализация позволяет доступно 
изложить при\-чин\-но-след\-ст\-вен\-ные истоки противоречий, которые 
инициируют в~дальнейшем глубокое исследование рассматриваемых процессов.
  
  Будем считать, что для любого интересующего нас свойства~$B$ существует 
причина. Тогда справедлива следующая теорема.
  
  \smallskip
  
  \noindent
  \textbf{Теорема~1.}\ \textit{Для любого свойства~$B$ существует 
единственная причина}. 
  
  \smallskip
  
  \noindent
  Д\,о\,к\,а\,з\,а\,т\,е\,л\,ь\,с\,т\,в\,о\,.\ \ Доказательство будем вести от противного, 
т.\,е.\ предположим, что существуют две причины свойства~$B$: $P$ 
и~$P^\prime$, $P\hm\not= P^\prime$. Тогда существует $\alpha\hm\in U$, которое 
удовлетворяет одному из двух условий:
  \begin{itemize}
\item[(а)] $\alpha\in P$, $\alpha\notin P^\prime$;
\item[(б)] $\alpha\notin P$, $\alpha \in P^\prime$.
\end{itemize}

  Пусть выполняется условие~(б). Тогда $P^\prime\backslash \{\alpha\}$ не 
является причиной по условию~2 определения~1, т.\,е.\ $\exists C$ такое, что 
$\alpha\notin C$, $P^\prime\backslash \{\alpha\}\hm\subseteq C$ и~$C\not\mapsto B$. 
Но если~$B$ произошло и~$P$ его причина, то $C\mapsto B$, что противоречит 
предположению. Теорема~1 доказана.
  
  \smallskip
  
  \noindent
  \textbf{Лемма.} \textit{Если $P$~--- причина появления свойства~$B$, то 
объект~$B$ определяет существование свойства~$P$ в~объекте, 
предшествующем~$B$. }
  
  \smallskip
  
  \noindent
  Д\,о\,к\,а\,з\,а\,т\,е\,л\,ь\,с\,т\,в\,о\,.\ \ Из предположения, что у~каж\-до\-го 
свойства~$B$ есть причина, и~условия, что~$P$ является причиной~$B$, следует, 
что при появлении в~данных свойства~$B$ объект~$C$, предшествующий 
появлению~$B$, содержит как часть объект~$P$. Это следует из теоремы~1 
и~определения причины. 
  
  Докажем принцип <<необходимого условия>>, который, несмотря на простоту 
доказательства, будет играть в~дальнейшем существенную роль.
  
  \smallskip
  
  \noindent
  \textbf{Теорема~2.} \textit{Если~$P$~--- причина появления свойства~$B$ 
и~$A\hm\subseteq P$, то объект~$B$ определяет наличие свойства~$A$ 
в~объекте, предшествующем~$B$}. 
  
  \smallskip
  
  \noindent
  Д\,о\,к\,а\,з\,а\,т\,е\,л\,ь\,с\,т\,в\,о\,.\ \ Пусть в~данных имеется объект~$B$ 
и~$P\mapsto B$, тогда в~силу существования и~единственности причины~$B$ 
в~данных должен существовать объект~$C$, предшествующий~$B$ 
и~содержащий причину~$P$. Поскольку $A\hm\subseteq P$ и~$B$ содержит 
причину~$P$, то $B\mapsto A$. С~учетом леммы теорема~2 доказана.
  
  \smallskip
  
  Пусть даны пространства $U_1, U_2,\ldots$ и~имеется последовательность 
данных (процесс выполнения этапов проекта в~соответствии с~рис.~1) $A, B, 
\ldots$, где каждый объект является подмножеством некоторого 
пространства~$U_i$, $i\hm=1,\ldots$ Тогда в~объекте~$A$ присутствует 
причина~$P$ появления интересующего нас свойства~$C$ в~объекте~$B$. Пусть 
$P\hm\subseteq A$, тогда по теореме~2 $\forall \alpha\hm\in P$:  
$C\mapsto \{\alpha\}$, т.\,е.\ из появления~$C$ следует появление 
характеристики~$\alpha$ в~предшествующем объекте. Это необходимое условие 
того, что~$C$ удовлетворяет причинно-следственным связям развития процесса 
выполнения проекта. Если для~$C$ нет характеристики~$\alpha$, которую можно 
отнести к~причине~$C$, то можно считать, что нарушена  
при\-чин\-но-след\-ст\-вен\-ная связь и~$C$~--- аномальный объект. 
  
  \smallskip
  
  \noindent
  \textbf{Пример.} Если объект~$C$ состоит в~получении суммы~$a$ 
фирмой~$K$, то согласно теореме~2 в~пред\-шест\-ву\-ющем объекте должна 
существовать причина перевода суммы~$a$ на фирму~$K$. Если эта причина 
в~проекте отсутствует, то это можно считать признаком мошеннической схемы. 
Все проекты по предположению собираются из <<кубиков>>, содержащихся в~БЗ. 
Тогда можно сравнить цену объекта~$C$, породившего получение суммы~$a$, 
и~сумму, присутствующую в~смете проекта. Если разница велика, то это либо 
ошибка проекта, либо признак мошеннической схемы.
  
  \section{Поиск противоречий на~основе~принципа <<необходимых~условий>>}
   
  Как было показано в~разд.~3, нахождение противоречий соответствуют 
движению от следствия к~причине. Для каждого объекта в~наблюдаемых данных 
выявление причин его появления является трудоемкой задачей. Кроме того, при 
реализации контроля соблюдения при\-чин\-но-след\-ст\-вен\-ных связей на 
большом множестве участников экономической деятельности задача анализа 
причин становится трудоемкой. Поэтому процедуру контроля необходимо разбить 
на два этапа, где первый этап состоит в~анализе простых <<необходимых 
условий>> проявления мошенничества, когда используется хотя бы одна 
известная характеристика причины. Второй этап (в~режиме офлайн) состоит 
в~выявлении причин, позволяющих провести анализ источников мошеннических 
схем. 
  
  Один из подходов к~выбору <<необходимых условий>> состоит в~построении 
множества подцелей исходной цели проекта (структурный метод построения 
проекта~\cite{7-gr}). Каждая подцель описывается диаграммой на рис.~1, 
и~реализации подцелей должны образовывать полный функционал цели. Это 
является необходимым, но не достаточным условием достижения цели, так как 
при таком подходе отсутствует компонент согласования всех подцелей в~единую 
систему. Однако такой подход значительно упрощает анализ выполнения проекта 
на предмет поиска мошенничества. Если признаки мошенничества будут 
обнаружены в~реализации хотя бы одной из подцелей, то это значит, что 
мошенничество присутствует в~реализации всего проекта. 
  
  Аналогично в~реализации каждого этапа в~любой из подцелей можно выделять 
простые <<необходимые условия>> нарушения при\-чин\-но-след\-ст\-венн\-ых 
связей. 
  
  Таким образом, получается множество <<необходимых условий>>, нарушение 
которых свидетельствует о наличии мошенничества. Это множество 
<<необходимых условий>> можно назвать метаданными~[8, 9] для контроля 
проекта на выявление мошенничества. 
  
  
  \section{Заключение }
  
  В поиске противоречий необходимо от транзакций, соответствующих 
следствиям при\-чин\-но-след\-ст\-вен\-ных связей, переходить к~анализу причин 
наблюдаемых следствий. Это сложная задача, которая связана с~описанием причин 
определенных свойств. 
  
  В работе представлена модель, позволяющая строить множество необходимых 
условий соответствия наблюдаемого следствия вызвавшей его причине. Этот 
подход делает поиск противоречий вполне вычислимой задачей, но не гарантирует 
успех. 
  
  {\small\frenchspacing
 {%\baselineskip=10.8pt
 \addcontentsline{toc}{section}{References}
 \begin{thebibliography}{9}
\bibitem{1-gr}
\Au{Грушо А.\,А., Зацаринный~А.\,А., Тимонина~Е.\,Е.} Блокчейны цифровой экономики на базе 
системы ситуационных центров и~централизованного консенсуса~// Радиолокация, навигация, 
связь: Мат-лы XXV Междунар. научн.-технич. конф.~---
Воронеж: Издательский дом ВГУ, 2019. Т.~6. С.~183--191. 
\bibitem{2-gr}
\Au{Grusho A., Zatsarinny~A., Timonina~E.} A~system approach to information security in 
distributed ledgers on the situational centers platform.~---
Lecture notes in computer science ser.~--- Springer, 2019 
(in press).
\bibitem{3-gr}
\Au{Финн В.\,К.} Искусственный интеллект: Методология, применения, философия.~--- М.: 
Красанд, 2011. 448~с.

\bibitem{5-gr} %4
\Au{Аншаков~О.\,М., Фабрикантова~Е.\,Ф.} ДСМ-ме\-тод автоматического порождения 
гипотез: Логические и~эпистемологические основания.~--- М.: Либроком, 2009. 432~с.

\bibitem{4-gr} %5
\Au{Poelmans J., Elzinga~P., Viaene~S., Dedene~G.} Formal concept analysis in knowledge 
discovery: A~survey~// Conceptual structures: From information to intelligence~/ Eds.\ M.~Croitoru, 
S.~Ferr$\acute{\mbox{e}}$, and D.~Lukose.~--- Lecture notes in computer science 
ser.~--- Berlin--Heidelberg: Springer, 2010. Vol.~6208.  P.~139--153.

\bibitem{6-gr}
\Au{Панкратова~Е.\,С., Финн~В.\,К.} Автоматическое по\-рож\-де\-ние гипотез в~интеллектуальных 
системах.~--- М.: Либроком, 2009. 528~с. 
\bibitem{7-gr}
\Au{Денисов А.\,А., Колесников~Д.\,Н.} Теория больших систем управления.~--- Л.: Энергоиздат, 1982. 488~с.

\bibitem{9-gr}
\Au{Грушо А.\,А., Грушо Н.\,А., Забежайло~М.\,И., Смирнов~Д.\,В., Тимонина~Е.\,Е.} 
Параметризация в~прикладных задачах поиска эмпирических причин~// Информатика и~её 
применения, 2018. Т.~12. Вып.~3. С.~62--66.

\bibitem{8-gr}
\Au{Грушо А.\,А., Грушо Н.\,А., Левыкин~М.\,В., Тимонина~Е.\,Е.} Методы идентификации 
захвата хоста в~распределенной ин\-фор\-ма\-ци\-он\-но-вы\-чис\-ли\-тель\-ной сис\-те\-ме, 
защищенной с~помощью метаданных~// Информатика и~её применения, 2018. Т.~12. Вып.~4. 
С.~41--45.

 \end{thebibliography}

 }
 }

\end{multicols}

\vspace*{-3pt}

\hfill{\small\textit{Поступила в~редакцию 03.04.19}}

%\vspace*{8pt}

%\pagebreak

\newpage

\vspace*{-28pt}

%\hrule

%\vspace*{2pt}

%\hrule

%\vspace*{-2pt}

\def\tit{ARCHITECTURAL DECISIONS IN~THE~PROBLEM 
OF~IDENTIFICATION OF~FRAUD IN~THE~ANALYSIS 
OF~INFORMATION FLOWS IN~DIGITAL ECONOMY\\[-5pt]}


\def\titkol{Architectural decisions in~the~problem 
of~identification of~fraud in~the~analysis 
of~information flows in~digital economy}

\def\aut{A.\,A.~Grusho, M.\,I.~Zabezhailo, N.\,A.~Grusho, and~E.\,E.~Timonina}

\def\autkol{A.\,A.~Grusho, M.\,I.~Zabezhailo, N.\,A.~Grusho, and~E.\,E.~Timonina}

\titel{\tit}{\aut}{\autkol}{\titkol}

\vspace*{-13pt}


 \noindent
   Institute of Informatics Problems, Federal Research Center ``Computer Sciences and 
Control'' of the Russian Academy of Sciences; 44-2~Vavilov Str., Moscow 119133, 
Russian Federation

\def\leftfootline{\small{\textbf{\thepage}
\hfill INFORMATIKA I EE PRIMENENIYA~--- INFORMATICS AND
APPLICATIONS\ \ \ 2019\ \ \ volume~13\ \ \ issue\ 2}
}%
 \def\rightfootline{\small{INFORMATIKA I EE PRIMENENIYA~---
INFORMATICS AND APPLICATIONS\ \ \ 2019\ \ \ volume~13\ \ \ issue\ 2
\hfill \textbf{\thepage}}}

\vspace*{3pt}


   
     
   \Abste{An approach to a~research of some types of fraud in digital economy with the usage of relationships of 
cause and effect is formulated. In all types of the considered frauds, the discrepancy between the 
purposes of financial transactions and actual cost of achievement of these purposes
has to be observed. Data on 
transactions can be collected by observing information flows in which these transactions are reflected. 
The architecture of data collection and their analysis can be organized by means of the distributed 
ledgers with the centralized consensus that allows creating an analog of the electronic account book 
fixing financial and economic activity of subjects of digital economy in the region. 
   The methods of fraud identification considered are based on the contradictions 
between actions described in transactions and information, which is contained in plans, standards, 
precedents, etc. 
   The method based on a~simplified scheme of implementation of the abstract project is considered. 
For identification of contradictions, it is necessary to carry out the analysis from the effect to the cause, 
i.\,e., to look for anomalies in information describing the generation of the observed effects. 
   It is shown how in implementation of the project it is possible to allocate simple ``necessary 
conditions'' of violation of cause and effect relationships, i.\,e., a~set of ``necessary conditions'' 
violation of which demonstrates fraud existence. It is possible to call this set of "necessary conditions" 
by metadata for control of the project for fraud identification.} 
   
   \KWE{digital economy; information flows; relationships of reason and effect; detection of 
fraudulent schemes}
   
  

 \DOI{10.14357/19922264190204}

\vspace*{-20pt}

 \Ack
   \noindent
   The work was partially supported by the Russian Foundation for Basic Research (projects  
18-29-03081 and 18-07-00274).



%\vspace*{6pt}

  \begin{multicols}{2}

\renewcommand{\bibname}{\protect\rmfamily References}
%\renewcommand{\bibname}{\large\protect\rm References}

{\small\frenchspacing
 {\baselineskip=10.5pt
 \addcontentsline{toc}{section}{References}
 \begin{thebibliography}{9}
\bibitem{1-gr-1}
\Aue{Grusho, A.\,A., A.\,A.~Zatsarinny, and E.\,E.~Timonina.} 2019. Blokcheyny tsifrovoy ekonomiki 
na baze sistemy situatsionnykh tsentrov i~tsentralizovannogo konsensusa [Blockchains of digital 
economy on the basis of the system of the situational centres and the centralized consensus]. 
\textit{25th Scientific and Technical Conference (International) ``Radar-Location, Navigation, 
Communication'' Proceedings}. Voronezh: VSU Publs. 6:183--191.
\bibitem{2-gr-1}
\Aue{Grusho, A., A.~Zatsarinny, and E.~Timonina.} 2019 (in press). 
A~system approach to information security 
in distributed ledgers on the situational centers platform. 
Lecture notes in computer science ser. Springer.
\bibitem{3-gr-1}
\Aue{Finn, V.\,K.} 2011. \textit{Iskusstvennyy intellekt: Metodologiya, primeneniya, filosofiya} 
[Artificial intelligence: Methodology, applications, philosophy]. Moscow: KRASAND. 448~p.

\bibitem{5-gr-1}
\Aue{Anshakov, O.\,M., and E.\,F.~Fabrikantova}. 2009. \textit{DSM-metod avtomaticheskogo porozhdeniya gipotez: Logicheskie 
i~epistemologicheskie osnovaniya} [JSM-method of automatic hypothesis generation: Logical and 
epistemological]. Moscow: KD LIBROKOM. 432~p.
\bibitem{4-gr-1} %5
\Aue{Poelmans, J., P.~Elzinga, S.~Viaene, and G.~Dedene.} 2010. Formal concept analysis in 
knowledge discovery: A~survey. \textit{Conceptual structures: From information to intelligence}. 
Eds.\ M.~Croitoru, S.~Ferr$\acute{\mbox{e}}$, and D.~Lukose. Lecture notes in 
computer science ser. Berlin--Heidelberg: Springer. 6208:139--153.

\bibitem{6-gr-1}
\Aue{Pankratov, E.\,S., and V.\,K.~Finn}. 
2009. \textit{Avtomaticheskoe porozhdenie gipotez v~intellektual'nykh 
sistemakh} [Automatic hypotheses generation in intelligent systems]. Moscow: KD 
\mbox{LIBROKOM}.  528~p. 
\bibitem{7-gr-1}
\Aue{Denisov, A.\,A., and D.\,N.~Kolesnikov.} 1982. \textit{Teoriya bol'shikh 
sistem upravleniya} [Theory of big control systems]. Leningrad: Energoizdat. 488~p.

\bibitem{9-gr-1}
\Aue{Grusho, A.\,A., N.\,A.~Grusho, M.\,I.~Zabezhailo, D.\,V.~Smirnov, and 
E.\,E.~Timonina.} 2018. 
Parametrizatsiya v~prikladnykh zadachakh poiska empiricheskikh prichin 
[Parametrization in applied 
problems of search of the empirical reasons]. 
\textit{Informatika i~ee Primeneniya~--- 
Inform. Appl.} 12(3):62--66.

\bibitem{8-gr-1}
\Aue{Grusho, A.\,A., N.\,A.~Grusho, M.\,V.~Levykin, and E.\,E.~Timonina.} 2018. Metody 
identifikatsii zakhvata khosta v~raspredelennoy informatsionno-vychislitel'noy sisteme, 
zashchishchennoy s~pomoshch'yu metadannykh [Methods of identification of host capture 
in the  distributed information system which is protected on the base of meta data].
\textit{Informatika i~ee 
Primeneniya~--- Inform. Appl.} 12(4):41--45.
{ %\looseness=1

}

\end{thebibliography}

 }
 }

\end{multicols}

\vspace*{-12pt}

\hfill{\small\textit{Received April 3, 2019}}

%\pagebreak

%\vspace*{-18pt}

\Contr

\noindent
\textbf{Grusho Alexander A.} (b.\ 1946)~--- Doctor of Science in physics and 
mathematics, professor, principal scientist, Institute of Informatics Problems, 
Federal Research Center ``Computer Sciences and Control'' of the Russian 
Academy of Sciences; 44-2~Vavilov Str., Moscow 119133, Russian Federation; 
\mbox{grusho@yandex.ru} 

\vspace*{3pt}

\noindent
\textbf{Zabezhailo Michael I.} (b.\ 1956)~--- Doctor of Science in physics and 
mathematics, principal scientist, Institute of Informatics Problems, Federal Research 
Center ``Computer Sciences and Control'' of the Russian Academy of Sciences;  
44-2~Vavilov Str., Moscow 119133, Russian Federation; 
\mbox{m.zabezhailo@yandex.ru} 

\vspace*{3pt}


\noindent
\textbf{Grusho Nikolai A.} (b.\ 1982)~--- Candidate of Science (PhD) in physics 
and mathematics, senior scientist, Institute of Informatics Problems, Federal 
Research Center ``Computer Sciences and Control'' of the Russian Academy of 
Sciences; 44-2~Vavilov Str., Moscow 119133, Russian Federation; 
\mbox{info@itake.ru} 

\vspace*{3pt}


\noindent
\textbf{Timonina Elena E.} (b.\ 1952)~--- Doctor of Science in technology, 
professor, leading scientist, Institute of Informatics Problems, Federal Research 
Center ``Computer Sciences and Control'' of the Russian Academy of Sciences;  
44-2~Vavilov Str., Moscow 119133, Russian Federation; 
\mbox{eltimon@yandex.ru} 

\label{end\stat}

\renewcommand{\bibname}{\protect\rm Литература}   %5
\def\stat{grusho-2}



\def\tit{ВКЛЮЧЕНИЕ НОВЫХ ЗАПРЕТОВ В~СЛУЧАЙНЫЕ
ПОСЛЕДОВАТЕЛЬНОСТИ$^*$}



\def\titkol{Включение новых запретов в~случайные
последовательности}

\def\aut{А.\,А. Грушо$^1$, Н.\,А. Грушо$^2$, Е.\,Е. Тимонина$^3$}

\def\autkol{А.\,А. Грушо, Н.\,А. Грушо, Е.\,Е. Тимонина}

\titel{\tit}{\aut}{\autkol}{\titkol}

{\renewcommand{\thefootnote}{\fnsymbol{footnote}} \footnotetext[1]
{Работа частично поддержана РФФИ (проект 13-01-00215).}}


\renewcommand{\thefootnote}{\arabic{footnote}}
\footnotetext[1]{Институт проблем информатики Российской академии наук;
факультет вычислительной математики
и~кибернетики Московского государственного университета им.\ М.\,В. Ломоносова,
grusho@yandex.ru}
\footnotetext[2]{Институт проблем информатики Российской академии наук, info@itake.ru}
\footnotetext[3]{Институт проблем информатики Российской академии наук, eltimon@yandex.ru}





\Abst{Рассматривается задача порождения одних вероятностных мер на пространстве
бесконечных последовательностей над конечными алфавитами с~$\sigma$-ал\-геб\-рой,
порожденной цилиндрическими множествами, из других вероятностных мер на этом
пространстве. При этом новая вероятностная мера устроена так, чтобы определенным
образом сокращать множество допустимых траекторий случайных последовательностей.
Недопустимость траекторий определяется в~терминах спецификаций наименьших запретов.}

\KW{случайные последовательности; запреты вероятностных мер; порождение
вероятностных мер; статистические задачи на случайных последовательностях}

\DOI{10.14357/19922264140406}
%\vspace*{5pt}


\vskip 12pt plus 9pt minus 6pt

\thispagestyle{headings}

\begin{multicols}{2}

\label{st\stat}

\section{Введение}

    В системах контроля информационной безопасности и~при поиске
скрытых каналов часто возникает задача поиска аномалий в~наблюдаемой
случайной последовательности. Одним из основных инструментов в~решении
таких задач является математическая статистика. Однако при ненулевой ошибке
принятия решения об аномалии наблюдение процесса порождает большое
число ложных тревог~[1], которые затрудняют или делают невозможным анализ
причин аномалий. Для случайных процессов с~дискретным временем
и~конечным множеством состояний найден подход для проверки
последовательности гипотез о~распределении проекций вероятностных мер на
пространстве бесконечных последовательностей, при котором вероятность
ложной тревоги всегда равна нулю. При этом с~ростом размерностей
вероятность правильного решения о наличии аномалий стремится к~единице.
Этот подход основан на запретах вероятностных мер~[2, 3].

    В предыдущих исследованиях~[2, 3] было введено определение запрета
для вероятностной меры на конечном пространстве. Запрет означает
последовательность, имеющую нулевую вероятность на конечном пространстве.
Было показано, что введение понятия запрета полезно для решения указанных
выше задач. Запреты позволяют определять критические множества
статистических критериев прос\-тей\-шим для вычисления способом~\cite{2-grs}.
Были доказаны необходимые и~достаточные условия существования
состоятельной последовательности критериев, в~которых все критические
множества статистических критериев определяются с~по\-мощью
запретов~\cite{3-grs}.

    Статистические критерии, основанные на запретах, обладают важными
особенностями. Вероятности ложных решений равны нулю. Так как
критические множества статистических критериев определяются только
запретами, то эти критерии одинаковы для вероятностных мер, име\-ющих
одинаковые множества запретов (т.\,е.\ это робастные критерии). В~случае
проверки гипотез состоятельность определяется тем условием, что вероятность
появления хотя бы одного запрета при альтернативе стремится к единице.

    Рассмотрим задачу внесения запретов в~случайную последовательность.

Приведем следующий пример~[4]. Для поиска скрытых каналов в~случайную
последовательность вносится запрет. При этом организаторы скрытого канала
не знают о~внесенном запрете, поэтому, когда в~наблюдаемой
последовательности наблюдается запрет, контролер определяет, что
функционирует скрытый канал.

    В работе~[4] было показано, что другие статистические методы, отличные
от методов, основанных на запретах, очень чувствительны к~изменениям
вероятностных моделей, поэтому статистические\linebreak процедуры по\-ис\-ка скрытых
каналов необходимо строить на робастных процедурах, учитывающих
гетерогенный характер последовательностей передаваемых сообщений.
Условиям робастности удовле\-тво\-ря\-ют статистические методы, основанные на
запретах.

    Пусть исходная случайная последовательность определена с~помощью
вероятностной меры~$P$ на пространстве бесконечных последовательностей
с~$\sigma$-ал\-геб\-рой, порожденной цилиндрическими множествами.
Проекции этой меры удовлетворяют условию согласованности, т.\,е.\
распределение очередного ($n\hm+1$)-го знака в~последовательности
выражается через условное распределение появления этого знака при условии
появления предыду\-щих~$n$~знаков последовательности и~безусловного
распределения этих $n$ знаков. Внесение запрета не должно нарушать условие
согласованности, иначе мера на пространстве бесконечных
последовательностей может определяться некорректно. В~рассматриваемом
методе условных распределений для внесения запрета необходимо оперировать
вероятностями всех начальных участков случайной последовательности.

    В работе рассматривается другой способ внесения запрета в~случайную
последовательность. Этот метод основан на корректном определении некоторых
множеств функций и~учитывает спе\-ци\-фи\-кации наименьших запретов, которые
должны присутствовать в~новой построенной мере. В~предлагаемом методе
сохраняются условия со\-гла\-со\-ван\-ности, что позволяет использовать теорему
Каратеодори~[5] об однозначном продолжении меры.

  Статья имеет следующую структуру. В~разд.~2 приводятся определения
и~предыдущие результаты. Раздел~3 определяет условия для случая, когда
вероятностная мера генерируется согласно заданной спецификации
наименьших запретов. В~разд.~4 приводится пример использования
доказанных условий. В~разд.~5 кратко анализируются условия существования
состоятельной последовательности критериев для мер, построенных в~примере.

\vspace*{-10pt}

\section{Основные определения и~предыдущие результаты}

\vspace*{-2pt}

    Пусть $X_i$, $i=1,2,\ldots,$~--- конечные множества, $\prod\limits_{i=1}^n
X_i$~--- декартово произведение~$X_i$, $i\hm=1,2,\ldots, n$, $X^\infty$~---
множество всех последовательностей, где $i$-й элемент принадлежит~$X_i$.
Пусть $\mathcal{A}$~--- это $\sigma$-ал\-геб\-ра на~$X^\infty$, порожденная
цилиндрическими множествами; $\mathcal{A}$~также является борелевской
$\sigma$-ал\-геб\-рой в~тихоновском произведении~$X^\infty$, где $X_i$ имеют
дискретную топологию~[6, 7].

    На ($X^\infty, \mathcal{A}$) определена вероятностная мера~$P$.
Предположим, что $P_n$ является проекцией меры~$P$ на пространство
конечномерных векторов, по\-рож\-ден\-ных первыми $n$ координатами
последовательностей из~$X^\infty$. Обозначим $X_n^\infty \hm=
\prod\limits_{i=n+1}^\infty X_i$. Ясно, что для каждого $B_n\hm\subseteq
\prod\limits_{i=1}^n X_i$
    $$
    P_n(B_n) =P\left( B_n\times X_n^\infty\right)\,.
    $$

    Пусть $D_n$~--- носитель меры~$P_n$:
    $$
    D_n=\left\{ \vec{x}_n\in \prod\limits_{i=1}^n X_i\,\ \ P_n\left(\vec{x}_n\right)
>0\right\}\,.
    $$

    Обозначим $\Delta_n=D_n\times X^\infty$. Последовательность~$\Delta_n$,
$n\hm= 1,2,\ldots$, невозрастающая и
    $$
    \Delta_P=\lim\limits_{n\to\infty} \Delta_n = \mathop{\bigcap}\limits_{n=1}^\infty \Delta_n\,.
    $$

    Множество $\Delta_P$ замкнуто в~топологии тихоновского произведения
и~является носителем меры~$P$. Если $\overline{\omega}^{(k)} \hm\in
\prod\limits_{i=1}^k X_i$, то $\tilde{\omega}^{(k-1)}$ получается
из~$\overline{\omega}^{(k)}$ отбрасыванием последней координаты.

    \smallskip

    \noindent
    \textbf{Определение 1.}\ Запретом в~мере~$P_n$ называется вектор
$\overline{\omega}^{(k)} \hm\in \prod\limits_{i=1}^k X_i$, $k\hm\leq n$ , такой
что
    $$
    P_n \left( \overline{\omega}^{(k)} \times \prod\limits_{i=k+1}^n X_i\right)
=0\,.
    $$

    Если $P_{k-1}\left( \tilde{\omega}^{(k-1)}\right)\hm>0$, то
$\overline{\omega}^{(k)}$ называется наименьшим запретом.

    Если $\overline{\omega}^{(k)}$ является запретом в~мере~$P_n$, тогда для
любых $k\hm\leq s\hm\leq n$ и~любых
последовательностей~$\overline{\omega}^{(s)}$, начинающихся
с~последовательности~$\overline{\omega}^{(k)}$, имеем:
    $$
    P_s\left( \overline{\omega}^{(s)}\right)=0\,.
    $$

    Действительно, если $P_k\left(\overline{\omega}^{(k)}\right)\hm=0$, то
\begin{align*}
    P\left( \overline{\omega}^{(k)} \times X_k^\infty \right)&=0\,;
\\
P\left( \overline{\omega}^{(k)} \times \prod\limits_{i=k+1}^s X_i\times X_s^\infty
     \right)&=0\,.
    \end{align*}
Из этого следует, что
\begin{multline*}
P_s\left(\overline{\omega}^{(s)}\right) = P\left( \overline{\omega}^{(s)}\times
X_s^\infty\right) \leq{}\\
{}\leq P\left( \overline{\omega}^{(k)} \times
\prod\limits_{i=k+1}^s X_i\times X_s^\infty\right)=0\,.
\end{multline*}

    Если существует  $\overline{\omega}^{(n)} \in \prod\limits_{i=1}^n X_i$
    такое, что $P_n\left(\overline{\omega}^{(n)}\right)\hm=0$, то существует
наименьший запрет.

    Пусть для всех $n$ носители мер~$P_n$ совпадают с~$\prod\limits_{i=1}^n
X_i$. Тогда носитель меры~$P$ совпадает с~$X^\infty$. Предположим, что
задана спецификация исходных наименьших запретов
$\nu^\prime\hm= \{\nu^\prime_n, n\hm= 1,2,\ldots\}$, где
$\nu^\prime_n$~--- число наименьших запретов длины~$n$ в~мере~$P$. Пусть задана новая
спецификация, определяемая дополнительными ограничениями
$\nu\hm= \{\nu_n, \nu_n\hm\geq \nu_n^\prime, n\hm= 1,2,\ldots\}$.
Задача состоит в~том, чтобы, используя меру~$P$
и~спецификации~$\nu$   и~$\nu^\prime$, построить вероятностную меру~$Q$ на пространстве
$(X^\infty, \mathcal{A})$, у~которой множество наименьших запретов обладает
спецификацией~$\nu$. Для построения~$Q$ сначала построим согласованную
систему вероятностных мер~$Q_n$, $n\hm= 1,2,\ldots$, на пространствах,
определяемых проекциями~$X^\infty$ на первые $n$ координат. Эти меры
определяют аддитивную меру на алгебре цилиндрических множеств, которая по
теореме Каратеодори будет однозначно определять меру~$Q$ на $(X^\infty,
\mathcal{A})$. Далее будем обозначать через~$D^\prime_n$, $n\hm=1, 2,\ldots$,
$D_n^\prime\hm\subseteq \prod\limits_{i=1}^n X_i$, носители мер~$P_n$, через
$d_n^\prime$~--- мощности этих носителей, а через~$D_n$, $n\hm=1, 2,\ldots$,
$D_n\hm\subseteq \prod\limits_{i=1}^n X_i$, носители мер~$Q_n$ и~через
$d_n$~--- мощности этих носителей.

    В работе~\cite{3-grs} доказано, что числа~$d_n$, $n\hm= 1, 2,\ldots$,
однозначно связаны со спецификацией~$\nu$ сле\-ду\-ющи\-ми соотношениями:
    \begin{equation}
    \nu_1 \prod\limits_{i=2}^n m_i +\cdots + \nu_{n-1} m_n +\nu_n +d_n
=\prod\limits_{i=1}^n m_i\,.
    \label{e1-grs}
    \end{equation}
для всех $n = 1, 2,\ldots$

  Таким образом, необходимо построить согласованное семейство
вероятностных мер $\{Q_n\}$, мощности носителей которых однозначно
определены соотношениями~(1).

\vspace*{-4pt}

\section{Порождение вероятностных мер с~заданной спецификацией
наименьших запретов}

\vspace*{-2pt}

    Пусть $\{D_n$, $D_n\hm\subseteq D_n^\prime$, $D_n\hm\subseteq
\prod\limits_{i=1}^n X_i$, $n\hm=1,2,\ldots\}$~--- некоторое семейство множеств,
удовле\-тво\-ря\-ющих~(1), $\overline{x}_n$~--- произвольный элемент
$\prod\limits_{i=1}^n X_i$. Для любого~$n$, $n\hm=1, 2,\ldots$, определим
функцию
    $$
    g_{n+1}:\ D_{n+1} \to \prod_{i=1}^n X_i
    $$
следующим образом. Для любых $\overline{x}_{n+1}\hm\in D_{n+1}$,
$\overline{x}_{n+1}\hm= \overline{x}_nx$, где
$\overline{x}_n \hm\in \prod\limits_{i=1}^n X_i$, $x\hm\in X_{n+1}$, определяем
$$
g_{n+1}\left( \overline{x}_{n+1}\right) =\overline{x}_n\,.
$$

    Кроме~(1) на множества~$\{D_n\}$ наложим сле\-ду\-ющие два ограничения,
связанных с~функциями~$g_n$. Для любого~$n$, $n\hm= 1, 2,\ldots$, и~любых
$\overline{x}_{n+1}\hm\in D_{n+1}$
    \begin{gather}
    g_{n+1}\left(\overline{x}_{n+1}\right) \in D_n\,;\label{e2-grs}\\
    g_{n+1}:\ D_{n+1}\stackrel{\mbox{\tiny на}}{\to} D_n\,.\label{e3-grs}
    \end{gather}

    По аналогии с~функциями~$g_n$ определяются функции~$h_n$ для
последовательности множеств~$D_n^\prime$ так, что для любого~$n$,
$n\hm= 1, 2,\ldots$, и~любых~$\overline{x}_n x\hm\in D_{n+1}^\prime$
    $$
    h_{n+1}\left( \overline{x}_n x\right) =\overline{x}_n\,.
    $$

    \smallskip

    \noindent
    \textbf{Лемма.}\ \textit{Пусть~$P$~--- вероятностная мера на $(X^\infty,
\mathcal{A})$, $\nu^\prime$~--- спецификация наименьших запретов,
$\{D_n^\prime\}$~--- носители мер~$P_n$, $d_n^\prime \hm= \vert
D_n^\prime\vert$,  $n\hm=1,2,\ldots$ Тогда для~$\nu^\prime$, $\{D_n^\prime\}$,
$\{d_n^\prime\}$, $\{h_n\}$ выполняются соотношения}~(1)--(3).

    \smallskip

    \noindent
    Д\,о\,к\,а\,з\,а\,т\,е\,л\,ь\,с\,т\,в\,о\,.\ \  Выполнение~(1) доказано в~[3]. По
определению $\forall\ \overline{x}_{n+1}\hm= \overline{x}_nx\hm\in
D_{n+1}^\prime$
    $$
    h_{n+1}\left(\overline{x}_{n+1}\right) =\overline{x}_n\,.
    $$
По определению $D_{n+1}^\prime$ имеем:
 $$
 P_n\left( \overline{x}_{n+1}\right) >0\,;
 $$

 \vspace*{-14pt}

 \noindent
\begin{multline*}
0<P_{n+1}\left(\overline{x}_{n+1}\right) = P\left( \overline{x}_{n+1}\times
X^\infty_{n+1}\right) = {}\\[2pt]
{}=P\left(\overline{x}_n \times \{x\} \times
X^\infty_{n+1}\right) \leq {}\\[2pt]
{}\leq P\left(\overline{x}_n \times X_n \times X^\infty_{n+1}\right) =
P\left(\overline{x}_n \times X_n^\infty \right) =P_n\left(\overline{x}_n\right).\hspace*{-5.73422pt}
\end{multline*}
Отсюда следует, что $\overline{x}_n\hm\in D_n^\prime$. Это доказывает
соотношение~(2).

    Если $\overline{x}_n \in D_n^\prime$ и~$\forall x\hm\in X_{n+1}$
выполняется
    $$
    P_{n+1}\left(\overline{x}_nx\right)=0\,,
    $$
то $P_{n+1}\left(\overline{x}_n, X_{n+1}\right)\hm=0$, что противоречит
согласованности мер~$P_n$ и~$P_{n+1}$:
$$
P_{n+1}\left(\overline{x}_n, X_{n+1}\right) =P_n\left(\overline{x}_n\right)
$$
и~предположению, что $P_n\left(\overline{x}_n\right)\hm>0$. Значит,
существует такое~$x$, что $P_{n+1}\left(\overline{x}_n x\right) \hm>0$, т.\,е.\
$\overline{x}_n x\hm\in D^\prime_{n+1}$. Это доказывает соотношение~(3).

    Возьмем произвольную последовательность\linebreak сюрьективных функций
    $f_n:\ D_n^\prime \hm\to D_n$, $n\hm=1, 2,\ldots$

    Каждая такая функция порождает на~$D_n$ и,~следовательно, на
$\prod\limits_{i=1}^n X_i$ вероятностную меру~$Q_n$ с~носителем~$D_n$.

    Тогда для всех~$n$ функции~$g_{n+1}$ и~меры~$Q_{n+1}$ порождают
на $\prod\limits_{i=1}^n X_i$ вероятностные меры~$Q_n^\prime$
с~носителями~$D_n$ (это следует из сюрьективности~$f_{n+1}$ и~$g_{n+1}$).

    \smallskip

    \noindent
    \textbf{Теорема~1.}\ \textit{Пусть задано произвольное семейство
вероятных мер $\{Q_n\}$ с~носителями $\{D_n\}$ и~семейство функций
$\{g_n\}$, удовлетворяющих условиям~$(2)$ и~$(3)$. Семейство вероятностных
мер $\{Q_n\}$ является согласованным семейством тогда и~только тогда,
когда для всех~$n$ выполняются равенства} $Q_n\hm= Q_n^\prime$.

    \smallskip

    \noindent
    Д\,о\,к\,а\,з\,а\,т\,е\,л\,ь\,с\,т\,в\,о\,.\ \ Докажем достаточность. Для
согласованности мер достаточно, чтобы для любых $\overline{x}\hm\in
\prod\limits_{i=1}^n X_i$

\noindent
    $$
    Q_n\left(\overline{x}_n\right) = Q_{n+1}\left(\overline{x}_n, X_{n+1}\right)\,.
    $$

    Из конечности вероятностных схем
    $$
    Q_{n+1}\left(\overline{x}_n, X_{n+1}\right) = \sum\limits_{x\in X_{n+1}}
    \!\!\!\!\!Q_{n+1} \left (\overline{x}_n x\right)\,.
    $$

    По определению
    \begin{multline*}
    Q_n^\prime\left(\overline{x}_n\right) =Q_{n+1}\left( g_{n+1}^{-1}
    \left(\overline{x}_n\right)\right) = %{}\\
%{}=
    \!\!\!\!\!\!\sum\limits_{(\overline{x}_nx)\in D_{n+1}}\!\!\!\!\!\!\!\!\!\!Q_{n+1} \left(\overline{x}_n
x\right) ={}\\
    {}=    \!\!\!\!\!\!\sum\limits_{x\in X_{n+1}}\!\!\!\!\! Q_{n+1} \left(\overline{x}_n x\right) =
Q_{n+1} \left(\overline{x}_n, X_{n+1}\right)\,.
    \end{multline*}

    По условию теоремы
    $$
    Q_n^\prime \left (\overline{x}_n\right) =Q_n \left(\overline{x}_n\right)
    $$
    для любых
    $\overline{x}_n \hm\in \prod\limits_{i=1}^n X_i$. Отсюда следует, что для любых
$\overline{x}_n\hm\in \prod\limits_{i=1}^n X_i$
    $$
Q_n\left(\overline{x}_n\right) = Q_{n+1}\left(\overline{x}_n, X_{n+1}\right)\,.
$$

    Достаточность доказана. Докажем необходимость. Если $\{Q_n\}$~---
согласованное семейство вероятностных мер, то для любых
$\overline{x}_n\hm\in \prod\limits_{i=1}^n X_i$
    $$
    Q_{n+1}\left(\overline{x}_n, X_{n+1}\right) =Q_n \left(\overline{x}_n\right)\,.
    $$
Кроме того, для любых $\overline{x}_n \hm\in \prod\limits_{i=1}^n X_i$
\begin{multline*}
Q_n^\prime \left(\overline{x}_n\right) =  Q_{n+1} \left( g^{-
1}_{n+1}\left(\overline{x}_n\right)\right) ={}\\
{}=Q_{n+1}\left(\overline{x}_n,
X_{n+1}\right) =Q_n \left(\overline{x}_n\right)\,.
\end{multline*}

    Теорема доказана.
    \smallskip

    Семейство функций $\{f_n\}$ и~вероятностная мера~$P$ порождают
семейство вероятностных мер $\{Q_n\}$ с~носителями $\{D_n\}$. Пусть
функции $\{g_n\}$ удовле-\linebreak\vspace*{-12pt}

\columnbreak

\noindent
творяют условия~(2) и~(3). Тогда справедливо
следующее утверждение.

    \smallskip

    \noindent
    \textbf{Следствие 1.}\ Семейство функций $\{f_n\}$ и~вероятностная
мера~$P$ порождают единственную вероятностную меру~$Q$ тогда и~только
тогда, когда для всех $n \hm=1, 2,\ldots$ выполняется равенство $Q_n\hm=
Q_n^\prime$.

    \smallskip

    \noindent
    \textbf{Теорема 2.}\ \textit{Для согласованности множества
вероятностных мер $\{Q_n\}$, порожденных функциями $\{f_n\}$
и~проекциями меры~$P$, достаточно, чтобы функции $\{g_n\}$
удовлетворяли соотношениям~$(2)$ и~$(3)$ и~для всех~$n$ были
коммутативны следующие диаграммы}:
    \begin{equation}
    \begin{array}{ccccc}
     & D_n^\prime & \stackrel{h_{n+1}}{\longleftarrow} & D_{n+1}^\prime & \\
    f_n & \!\!\downarrow & & \downarrow &\!\! f_{n+1}\\
    & D_n & \stackrel{g_{n+1}}{\longleftarrow} & D_{n+1} &
    \end{array}
    \label{e4-grs}
    \end{equation}

    \noindent
    Д\,о\,к\,а\,з\,а\,т\,е\,л\,ь\,с\,т\,в\,о\,.\ \ Каждая функция~$f_n$
и~мера~$P_n$ на~$D_n^\prime$ порождают на~$D_n$ вероятностное
распределение~$Q_n$. В~силу согласованности проекций меры~$P$ каждая
функция~$h_{n+1}$ порождает меру~$P_n$ из меры~$P_{n+1}$. Поэтому
можно считать, что мера~$Q_n$ порождена из меры~$P_{n+1}$ с~помощью
композиции отображений $(f_n * h_{n+1})$.

  В свою очередь, функция~$f_{n+1}$ и~мера~$P_{n+1}$ по\-рож\-да\-ют
распределение вероятностей $Q_{n+1}$ на~$D_{n+1}$. Эта мера
и~функция~$g_{n+1}$ порождают меру~$Q_n^\prime$ на~$D_n$, т.\,е.\
мера~$Q_n^\prime$ на~$D_n$ порождена из меры~$P_{n+1}$ с~помощью
композиции функций $(g_{n+1}*f_{n+1})$. По условию~(4) функции
$(f_n*h_{n+1})$) и~$(g_{n+1}*f_{n+1})$ совпадают. Следовательно, эти функции
и~мера~$P_{n+1}$ порождают на~$D_n$ одну и~ту же меру, т.\,е.\ $Q_n\hm=
Q_n^\prime$. Отсюда и~из теоремы~1 следует согласованность семейства
вероятностных мер $\{Q_n\}$. Теорема доказана.

\section{Пример порождения вероятностных мер с~заданной~спецификацией
наименьших~запретов}

    Пусть $\nu\hm= \{ \nu_i\hm=1, i\hm=1,2,\ldots\}$, $X_m\hm=
\{0,1,\ldots , m-1\}$, $m\hm>2$, и~$P$~--- равномерная мера на~$X^\infty$.
Пусть элементы $\prod\limits_{i=1}^n X_i$ лексикографически упорядочены.

    Приведем пример построения меры~$Q$ со спецификацией наименьших
запретов~$\nu$ с~помощью подхода, описанного в~разд.~3.

    В каждом множестве $B_n\hm\subseteq \prod\limits_{i=1}^n X_i$ есть
наименьший вектор\ $\overline{x}_n$ с~точки зрения лексикографическо-\linebreak го
порядка. Требуемую меру будем строить индуктивно. В~$D_1$ наименьшим
запретом будем считать~0. Функция~$f_1$ отображает~$X_1$ в~$X_1\backslash
\{0\}$. Предположим, что определены~$D_n$ и~$f_n$. Определим~$D_{n+1}$.
Пусть $\overline{x}_n^0$~--- наименьший элемент в~$D_n$. В~множестве
$D_n\times X_{n+1}$ определим наименьший запрет~--- $\left (\overline{x}_n^0
0\right)$. Положим
    $$
    D_{n+1}= \left( D_n\times X_{n+1}\right) \backslash \{ (\overline{x}_n^0
0)\}\,.
    $$

    Построим сюрьективную функцию
    $$
    f_{n+1}:\ \prod\limits_{i=1}^{n+1} X_i \stackrel{\mbox{\tiny на}}{\rightarrow}
D_{n+1}\,.
    $$

    Для любых $\left(\overline{x}_n x\right) \hm\in \prod\limits_{i=1}^{n+1}
X_i$, кроме тех, у~которых $f_n\left(\overline{x}_n\right)\hm= \overline{x}_n^0$
и~$x\hm=0$, положим
    $$
    f_{n+1}\left(\overline{x}_n x\right) =\left( f_n\left(\overline{x}_n\right),x\right)
\in D_n\times X_{n+1}\,.
    $$

    Обозначим $\overline{y}_n^{(i)}$,  $i \hm= 1,\ldots ,k$, все элементы
множества $f_n^{-1}\left(\overline{x}_n^0\right)$. Определим
    $$
    f_{n+1}\left( \overline{y}_n^{(i)},0\right) = \left(\overline{x}_n^0,1\right) \in
D_n\times X_{n+1}\,.
    $$

    Отметим, что $\left(\overline{x}_n^0,1\right) $~--- наименьший элемент
в~$D_{n+1}$. По определению~$f_n$~--- это отображение $\prod\limits_{i=1}^n
X_i$ на~$D_n$. Поэтому $f_{n+1}$ отображает~$X^{n+1}$ на $D_{n+1}\hm=
(D_n\times X_{n+1}) \backslash \{ \overline{x}_n^0,0\}$. По
построению~$D_{n+1}$ из~$D_n$ и~из~(1) следует, что $\nu_{n+1}\hm=1$
и~этот элемент равен $\left( \overline{x}_n^0,0\right)$.

    Докажем коммутативность диаграмм~(4). По построению~$f_n$
и~$h_{n+1}$ для любых $x\hm\in X_{n+1}$
    $$
    h_{n+1}\left( \overline{y}_n^{(i)},x\right) =\overline{y}_n^{(i)}\,,
    $$
поэтому
$$
\left( f_n * h_{n+1}\right) \left( \overline{y}_n^{(i)},x\right) =\overline{x}_n^0\,.
$$
При $\overline{x}_n \not= \overline{y}_n^{(i)}$, $i \hm= 1,\ldots , k$,
$$
\left( f_n * h_{n+1}\right) \left( \overline{x}_n,x\right)
=f_n\left( \overline{x}_n\right)\,.
$$
Далее
\begin{gather*}
f_{n+1}\left( \overline{y}_n^{(i)},0\right) =\left( \overline{x}_n^0,1\right)\in
D_{n+1}\,,\enskip i=1,\ldots ,k\,;\\
\left( f_{n+1}* g_{n+1}\right) \left( \overline{y}_n^{(i)},0\right) =f_n
\left( \overline{y}_n^{(i)}\right) =\overline{x}_n^0\,.
\end{gather*}
Для элементов $\left( \overline{y}_n^{(i)}, x\right)$, $x\not=0$, $i \hm= 1,\ldots ,
k$,
$$
f_{n+1} \left( \overline{y}_n^{(i)},x\right) =\left( f_n
\left( \overline{y}_n^{(i)}\right),x\right) = \left(\overline{x}_n^0,x\right) \in
D_{n+1}\,.
$$
Поэтому при $x\not=0$
$$
\left( f_{n+1}*g_{n+1}\right) \left( \overline{y}_n^{(i)},x\right)
=\overline{x}_n^0\,.
$$
При  $\overline{x}_n\not= \overline{y}_n^{(i)}$, $i \hm= 1,\ldots , k$, по
определению~$f_{n+1}$ имеем, что для любых $x\hm\in X_{n+1}$
$$
\left( f_{n+1}\right) \left(\overline{x}_n,x\right)
=\left( f_n\left(\overline{x}_n\right)x\right) \in D_{n+1}\,.
$$
Тогда по построению
$$
\left( g_{n+1}*f_{n+1}\right) =\left( f_n* h_{n+1}\right)\,.
$$

    Коммутативность диаграмм~(4) доказана.

    \smallskip

  Отсюда следует существование меры~$Q$ на $(X^\infty, \mathcal{A})$ со
спецификацией наименьших запретов $\nu\hm= \{ \nu_i \hm=1,\ i\hm=
1,2,\ldots\}$.

\section{Применение к анализу состоятельности}

    Из соотношений~(1) получаем соотношения:
    \begin{equation}
    d_{n+1}-m_{n+1}d_n +\nu_{n+1} =0\,,\ n=1,2,\ldots
    \label{e5-grs}
    \end{equation}

    Пусть $P$~--- равномерная мера на $(X^\infty, \mathcal{A})$. Тогда
отношение $d_n\big/\prod\limits_{i=1}^n m_i$ есть вероятность
множества~$D_n$ в~мере~$P_n$.

    Для спецификации $\nu\hm=\{\nu_i\hm=1,\ i\hm= 1,2,\ldots\}$
из~(\ref{e5-grs}) для некоторого положительного~$\varepsilon$ получаем
следующие соотношения при $m_n\hm\geq3$, $n\hm=1, 2,\ldots$:
    $$
    \fr{d_n}{\prod\limits_{i=1}^n m_i}>\varepsilon>0\,.
    $$

    При $n\to\infty$ предел этой вероятности равен вероятности~$P$
носителя~$\Delta_Q$ меры~$Q$
    $$
    P\left( \Delta_Q\right) \geq \varepsilon >0\,.
    $$

  Из необходимых и~достаточных условий~[2] существования состоятельных
последовательностей критериев, определяемых запретами, для проверки
гипотез $H_{0,n}:Q_n$ против $H_{1,n}:P_n$ следует, что таких
последовательностей критериев нет.

\section{Заключение}

Получены условия корректного построения дополни\-тельных
ограничений на случайную последовательность с~помощью спецификации
наименьших запретов. Корректное построение стохастических моделей
позволяет использовать в~анализе\linebreak статистических данных хорошо
разработанный аппарат теории случайных последовательностей и~процессов.

{\small\frenchspacing
 {%\baselineskip=10.8pt
 \addcontentsline{toc}{section}{References}
 \begin{thebibliography}{9}
\bibitem{1-grs}
    \Au{Axelson S.} The base-rate fallacy and its implications for the difficulty of
intrusion detection~// 6th ACM Conference on Computer and Communications
Security Proceedings.~--- New York: ASM, 1999. P.~1--7.
\bibitem{2-grs}
    \Au{Грушо А., Тимонина Е.} Запреты в~дискретных
    ве\-ро\-ят\-но\-ст\-но-ста\-ти\-сти\-че\-ских задачах~// Дискретная
математика, 2011. Т.~23. Вып.~2. С.~53--58.
\bibitem{3-grs}
    \Au{Grusho A., Grusho N., Timonina~E.} Consistent sequences of tests defined
by bans~// Springer proceedings in mathematics \& statistics, optimization theory,
decision making, and operation research applications.~---
    New York\,--\,Heidelberg\,--\,Dordrecht\,--\,London: Springer, 2013.
    \mbox{P.~281--291.}
\bibitem{4-grs}
    \Au{Grusho A., Grusho N., Timonina~E.} Problems of modeling in the analysis
of covert channels~// Computer network security, 2010. Lecture notes in computer
science ser. Vol.~6258. P.~118--124. doi: 10.1007/978-3-642-14706-7\_9.
\bibitem{5-grs}
    \Au{Неве Ж.} Математические основы теории вероятностей~/
    Пер. с англ.~--- М.: Мир,
1969. 309~с.
(\Au{Neveu~J.} {Bases mathematiques du calcul des probabilites}.
Paris: Masson, 1964. 203~p.)
\bibitem{6-grs}
    \Au{Бурбаки Н.} Общая топология. Основные структуры~/ Пер.
     с~франц.~--- М.: Наука, 1968. 272~с. (\Au{Bourbaki~N.} Topologie
G$\acute{\mbox{e}}$n$\acute{\mbox{e}}$rale. Chapitre~1: Structures
topologiques. Chapitre~2: Structures uniformes.~--- Paris: Hermann, 1940. 129~p.)
\bibitem{7-grs}
    \Au{Прохоров Ю.\,В., Розанов Ю.\,А.} Теория вероятностей.~--- М.: Наука,
1993. 496~с.



 \end{thebibliography}

 }
 }

\end{multicols}

\vspace*{-10pt}

\hfill{\small\textit{Поступила в редакцию 31.10.14}}

%\newpage

\vspace*{8pt}

\hrule

\vspace*{2pt}

\hrule

\vspace*{-2pt}

\def\tit{SWITCHING ON OF NEW BANS IN RANDOM SEQUENCES}

\def\titkol{Switching on of new bans in random sequences}

\def\aut{A.\,A.~Grusho$^{1,2}$, N.\,A.~Grusho$^1$, and E.\,E.~Timonina$^1$}

\def\autkol{A.\,A.~Grusho, N.\,A.~Grusho, and E.\,E.~Timonina}

\titel{\tit}{\aut}{\autkol}{\titkol}

\vspace*{-9pt}


\noindent
$^1$Institute of Informatics Problems, Russian Academy of Sciences,
44-2~Vavilov Str., Moscow 119333, Russian\linebreak
$\hphantom{^1}$Federation

\noindent
$^2$Faculty of Computational Mathematics and Cybernetics,
M.\,V.~Lomonosov Moscow State University,\linebreak
$\hphantom{^1}$1-52~Leninskiye Gory, GSP-1, Moscow 119991, Russian Federation


\def\leftfootline{\small{\textbf{\thepage}
\hfill INFORMATIKA I EE PRIMENENIYA~--- INFORMATICS AND
APPLICATIONS\ \ \ 2014\ \ \ volume~8\ \ \ issue\ 4}
}%
 \def\rightfootline{\small{INFORMATIKA I EE PRIMENENIYA~---
INFORMATICS AND APPLICATIONS\ \ \ 2014\ \ \ volume~8\ \ \ issue\ 4
\hfill \textbf{\thepage}}}

\vspace*{3pt}

\Abste{The problem of generating one probability measure on space
of the infinite sequences on finite alphabets with  $\sigma$-algebra generated
by cylindrical sets out of another probability measure on this space is
considered. A~new probability measure is arranged
to reduce the set of admissible trajectories of random sequences definitely. Inadmissibility
of trajectories is defined in terms of specifications of the smallest bans.
If a~specification of the smallest bans is given, then the powers of support
of projections of the new measure can be determined. It gives conditions to
construct several sets of functions. These functions and projections of the
initial measure define a set of measures on finite spaces which define
 the only probability measure on the space of infinite sequences.}


\KWE{random sequences; bans of probability measures;
generation of probability measures; statistical problems on random sequences}


\DOI{10.14357/19922264140406}

\vspace*{-20pt}

\Ack

\vspace*{-2pt}

\noindent
The paper was partially supported by the
Russian Foundation for Basic Research  (project 13-01-00215).

%\vspace*{3pt}

  \begin{multicols}{2}

\renewcommand{\bibname}{\protect\rmfamily References}
%\renewcommand{\bibname}{\large\protect\rm References}



{\small\frenchspacing
 {%\baselineskip=10.8pt
 \addcontentsline{toc}{section}{References}
 \begin{thebibliography}{9}
\bibitem{1-grs-1}
\Aue{Axelsson, S.} 1999. The base-rate fallacy and its implications for the
difficulty of intrusion detection.
\textit{6th Conference on Computer and Communications Security Proceedings}.
New York: ASM. 1--7.
\bibitem{2-grs-1}
\Aue{Grusho, A., and E. Timonina.} 2011.
Prohibitions in discrete probabilistic statistical problems.
\textit{Discrete Mathematics Applications} 21(3):275--281.
\bibitem{3-grs-1}
\Aue{Grusho, A., N. Grusho, and E.~Timonina}. 2013. Consistent
sequences of tests defined by bans.
\textit{Springer proceedings in mathematics \& statistics, optimization theory,
decision making, and operation research applications}.
New York\,--\,Heidelberg\,--\,Dordrecht\,--\,London: Springer. 281--291.
\bibitem{4-grs-1}
\Aue{Grusho, A., N.~Grusho, and E.~Timonina}.
2010. Problems of modeling in the analysis of covert channels.
\textit{Computer network security}.
Lecture notes in computer science ser. 6258:118--124.
doi: 10.1007/978-3-642-14706-7\_9.
\bibitem{5-grs-1}
\Aue{Neveu, J.} 1964. \textit{Bases mathematiques du calcul des probabilites}.
Paris: Masson. 203~p.
\bibitem{6-grs-1}
\Aue{Bourbaki, N.} 1940. \textit{Topologie G$\acute{\mbox{e}}$n$\acute{\mbox{e}}$rale}.
Chapitre~1: Structures topologiques. Chapitre~2: Structures uniformes.
Paris: Hermann, 1940. 129~p.
\bibitem{7-grs-1}
\Aue{Prokhorov, Yu.\,V., and Yu.\,A.~Rozanov}. 1993. \textit{Teoriya veroyatnostey}
[Theory of probabilities].  Moscow: Nauka. 496~p.

\end{thebibliography}

 }
 }

\end{multicols}

\vspace*{-10pt}

\hfill{\small\textit{Received October 31, 2014}}

\pagebreak

%\vspace*{-18pt}


    \Contr

\noindent
\textbf{Grusho Alexander A.} (b.\ 1946)~---
Doctor of Science in physics and mathematics,
Corresponding member of the Russian Academy
of Cryptography; leading scientist, Institute of Informatics Problems,
Russian Academy of
Sciences, 44-2 Vavilov Str., Moscow 119333, Russian Federation;
professor, Faculty of
Computational Mathematics and Cybernetics, M.\,V.~Lomonosov Moscow State University,
1-52~Leninskiye Gory, GSP-1, Moscow 119991, Russian Federation;
grusho@yandex.ru

\vspace*{3pt}

\noindent
\textbf{Grusho Nikolai A.}\ (b.\ 1982)~---
Candidate of Science (PhD) in physics and mathematics, senior scientist, Institute of Informatics
Problems, Russian Academy of Sciences, 44-2 Vavilov Str., Moscow 119333, Russian
Federation; info@itake.ru


\vspace*{3pt}

\noindent
\textbf{Timonina Elena E.}\ (b.\ 1952)~---
Doctor of Science in technology, professor, leading scientist, Institute of Informatics Problems,
Russian Academy of Sciences, 44-2 Vavilov Str., Moscow 119333, Russian Federation;
eltimon@yandex.ru

\label{end\stat}

\renewcommand{\bibname}{\protect\rm Литература} %6
\def\stat{bronshtein}



\def\tit{ОБ ОПТИМАЛЬНОЙ ДОСТАВКЕ ГРУЗОВ ТРАНСПОРТНЫМ
СРЕДСТВОМ С~УЧЕТОМ ЗАВИСИМОСТИ СТОИМОСТИ
ПЕРЕВОЗОК ОТ~ЗАГРУЗКИ ТРАНСПОРТНЫХ СРЕДСТВ
ПО~НЕСКОЛЬКИМ ЦИКЛИЧЕСКИМ МАРШРУТАМ$^*$}

\def\titkol{Об оптимальной доставке грузов транспортным
средством с~учетом зависимости стоимости
перевозок от~загрузки} % транспортных средств по~нескольким циклическим маршрутам}

\def\aut{Е.\,М. Бронштейн$^1$, П.\,А.~Зелёв$^2$}

\def\autkol{Е.\,М. Бронштейн, П.\,А.~Зелёв}

\titel{\tit}{\aut}{\autkol}{\titkol}

{\renewcommand{\thefootnote}{\fnsymbol{footnote}} \footnotetext[1]
{Работа выполнена при поддержке РФФИ (проект 13-01-00005).}}


\renewcommand{\thefootnote}{\arabic{footnote}}
\footnotetext[1]{Уфимский государственный авиационный технический университет, bro-efim@yandex.ru}
\footnotetext[2]{Уфимский государственный авиационный технический университет, pz1988@yandex.ru}

  \Abst{Рассматривается задача построения маршрута доставки грузов
потребителям от одного производителя (базы, склада) транспортным средством
(ТС) с~минимальными затратами на перевозки. При этом учитывается
зависимость стоимости транспортировки от загрузки ТС и~качества дороги.
Предполагается, что ТС может возвращаться на базу для дозагрузки. Построена
соответствующая математическая модель. Для случая линейной зависимости
стоимости проезда от загрузки получена линейная целочисленная модель. Для
решения поставленной задачи наряду с~точным алгоритмом предложена
модификация известного эвристического алгоритма Клар\-ка--Рай\-та.
Проведен вычислительный эксперимент.}

  \KW{эвристический алгоритм; построение маршрута; транспортировка;
задача маршрутизации}

\DOI{10.14357/19922264140407}


\vskip 12pt plus 9pt minus 6pt

\thispagestyle{headings}

\begin{multicols}{2}

\label{st\stat}

\section{Введение}

  Систематическое изучение оптимизационных задач транспортной логистики
(Vehicle Routing Problem, VRP) началось с~работы~[1]. За последние полвека
поставлено множество задач этого типа, развиты как точные, так
и~эвристические методы решения. Частным случаем подобных задач является
известная задача коммивояжера.
  Классификация оптимизационных задач транспортной логистики приведена,
например, в~[2].

  Задача, рассматриваемая в~данной работе, перекликается с~CVRP
(Capacitated VRP), в~этих задачах учитываются ограничения на вместимость
ТС (см., например,~[3]). В~большинстве случаев при постановке задач
маршрутизации стоимость транспортировки является функцией, зависящей
только от расстояния между городами.

  В работе~[4] предлагается при постановке задач маршрутизации учитывать
ряд дополнительных факторов, часть из которых рассматривается в~настоящей
работе. В~частности, предполагается, что стоимость транспортировки зависит
от загрузки ТС и~состояния дороги.

  Рассматриваемая задача является NP-труд\-ной, как обобщение задачи
коммивояжера, в~связи с~чем точные методы применимы лишь при малых
размерностях задач (менее 10), а~значит, актуальной задачей является
разработка эвристических методов ее решения.

\section{Постановка задачи}

  Груз следует доставить потребителям из пункта производства
ТС грузоподъемностью~$Q$. Известны значения $q_i$~--- потребности
в~грузе $i$-го пункта потребления ($i\hm=1, \ldots, n$), $n$~--- число пунктов
потребления. Для единообразия будем считать пункт производства (базу, склад)
нулевым пунктом. Предполагается, что ТС, доставив груз в~некоторые пункты,
возвращается на базу и~загружается для доставки другим потребителям.

  Дороги между пунктами характеризуются двумя показателями:
расстоянием~$l_{ij}$ ($i, j\hm=0, \ldots , n$) и~коэффициен\-том слож\-ности
дороги~$k_{ij}$ между пунктами~$i$ и~$j$. Расстояние~$l_{ij}$ может зависеть от
направления движения (например, есть дороги с~односторонним движением),
коэффициент слож\-ности дороги влияет на расход топлива, его величина также
может зависеть от направления маршрута (пример: подъем в~гору или спуск
с~горы)~[5].

  Каждый пункт посещается в~точности один раз (не допускается split
delivery~[6]). Очевидно, что задача имеет решение тогда и~только тогда, когда
вместимость ТС не меньше потребности в~грузе в~каждом из пунктов
потребления. Число циклов, за которые происходит доставка, не
превосходит~$n$. Будем считать, что оно равно~$n$, но среди них могут быть
пустые.

  Предполагается известной зависимость $f(q)$ расхода топлива на единицу
пути дороги стандартного качества.

  Функция расхода топлива в~зависимости от веса перевозимого груза
является возрастающей и~(как правило) вогнутой на всем интервале области
определения функции  $[0;\,Q]$.

  Введем булевы переменные~$X_{ij}^t$ ($i,j,\,t\hm=0, \ldots , n$), равные~1
тогда и~только тогда, когда следующим после $i$-го пункта на пути
следования ТС в~$t$-м цикле является $j$-й пункт. Должны выполняться
следующие условия:
  \begin{align}
  \sum\limits_{i=0}^n X^t_{i0} &= \sum\limits_{j=0}^n X^t_{0j} =1\,,\enskip
t=1,\ldots ,n\,;\label{e1-br}\\
  \sum\limits_{t=1}^n \sum\limits_{i=0}^n X_{ij}^t &=1\,,\enskip j=1,\ldots,
n\,;\label{e2-br}\\
  \sum\limits_{t=1}^n \sum\limits_{i=0}^n X_{ij}^t &=1\,,\enskip i=1,\ldots, n\,.
  \label{e3-br}
  \end{align}

  Ограничение~(1) означает наличие в~каждом цик\-ле одного пункта
потребления (или базы для нулевого маршрута), из которого ТС попадает
непосредственно на базу, и~ровно одного пункта потребления (или базы для
нулевого маршрута), в~который ТС выезжает с~базы; ограничения~(2) и~(3)
описывают условие однократного посещения каж\-до\-го пункта потребления.

  Введем целочисленные переменные  $v_i^t$ ($i,t\hm=1,\ldots ,n)$), имеющие
смысл номеров потребителей в~порядке прохождения в~$t$-м цикле,
содержащем все пункты, за исключением начального:
  \begin{gather}
  1\leq v_i^t \leq \sum\limits_{i=0}^n \sum\limits_{j=0}^n X_{ij}^t\,, \enskip
i,t=1,\ldots, n\,;\label{e4-br}\\
  \left( v_i^t -v_j^t\right) +nX^t_{ij} \leq n-1\,,\enskip i,j,t=1,\ldots, n\,.
  \label{e5-br}
  \end{gather}

  Из условий~(4) и~(5) следует отсутствие под\-цик\-лов.

  Введем, наконец, булевы переменные $Z^t_{is}$, равные~1 тогда и~только
тогда, когда $v_s^t\hm>v_i^t$ ($i,s,t\hm=1,\ldots ,n$). Ограничения на введенные
переменные имеют следующий вид:
  \begin{equation}
  nZ_{is}^t \geq v_s^t -v_i^t\,,\enskip i,s,t=1,\ldots, n\,.
  \label{e6-br}
  \end{equation}


  Это условие обеспечивает выполнение равенства $Z_{is}^t\hm= 1$ при
$v_s^t\hm>v_i^t$.

  Для обеспечения равенства $Z_{is}^t\hm=0$ при $v_s^t\hm\leq v_i^t$
заметим, что число величин~$v_i^t$, меньших~$v_s^t$, равно $v_s^t\hm-1$.
Таким образом, выполнение нужного свойства обеспечивается равенствами:
  \begin{equation}
  \sum\limits_{i=1}^n Z_{is}^t =v_s^t -1\,, s\,,t=1,\ldots, n\,.
  \label{e7-br}
  \end{equation}

     Ограничение на вместимость:
     \begin{equation}
     \sum\limits_{i=0}^n \sum\limits_{j=0}^n q_i X_{ij}^t \leq Q\,,\enskip
t=1,\ldots, n\,.
     \label{e8-br}
     \end{equation}

  Целью является минимизация расходов на транспортировку:
  \begin{multline}
  R(X) = {}\\
  {}=\sum\limits_{t=1}^n \sum\limits_{j=0}^n \sum\limits_{i=0}^n X_{ij}^t f
\left ( \sum\limits_{s=0}^n Z^t_{is} q_i\right) l_{ij} k_{ij} \to \min\,.
  \label{e9-br}
  \end{multline}

\section{Линейная целочисленная модель}

  Предположим, что стоимость транспортировки является линейной функцией
от массы перевозимого груза:
$$
f(q)=vq+w\,.
$$
%
 Тогда при сохранении
условий~(1)--(8) целевая функция~(\ref{e9-br}) примет вид:
  \begin{multline}
  R(X) =v \sum\limits_{t=1}^n \sum\limits_{s=0}^n \sum\limits_{j=0}^n
\sum\limits_{i=0}^n X_{ij}^t Z_{is}^t q_s l_{ij} k_{ij} +{}\\
 {}+w \sum\limits_{t=1}^n
\sum\limits_{j=0}^n \sum\limits_{i=0}^n X_{ij}^t l_{ij} k_{ij} \to \min\,.
  \label{e10-br}
  \end{multline}

  Для приведения целевой функции к линейной форме введем булевы
переменные $P^t_{ijs} \hm= X_{ij}^t Z_{is}^t$ ($i,j,s,t\hm= 0,\ldots ,n$). Это
условие равносильно выполнению следующей системы неравенств:
  \begin{alignat}{2}
  P^t_{ijs} &\geq X_{ij}^t +Z_{is}^t -1\,, &\enskip i,j,s,t&=0,\ldots, n\,;\label{e11-br}\\
  P_{ijs}^t &\leq X_{ij}^t\,, &\enskip i,j,s,t&=0,\ldots ,n\,;\label{e12-br}\\
  P_{ijs}^t&\leq Z_{is}^t\,, &\enskip i,j,s,t&=0,\ldots, n\,.\label{e13-br}
  \end{alignat}
%
  Тогда целевая функция~(10) примет вид:
  \begin{multline}
  R(X) =v\sum\limits_{t=1}^n \sum\limits_{s=0}^n \sum\limits_{j=0}^n
\sum\limits_{i=0}^n P^t_{ijs} q_s l_{ij} k_{ij} +{}\\
{}+w \sum\limits_{t=1}^n
\sum\limits_{j=0}^n \sum\limits_{i=0}^n X_{ij}^t l_{ij} k_{ij}\to \min\,.
  \label{e14-br}
  \end{multline}

  Как число переменных задачи~(1)--(8), (11)--(14), так и~число ограничений
имеют порядок $O(n^4)$.

\section{Модифицированный алгоритм Кларка--Райта}

  Для решения задачи был модифицирован эвристический алгоритм
  Клар\-ка--Рай\-та~[7, 8]. Достоинствами метода являются его простота,
надежность и~гибкость.

  Опишем основные идеи алгоритма.

  Метод Кларка--Райта является итерационным, причем на каждой итерации
осуществляется попытка сращивания двух циклических маршрутов по
определенным правилам. Модификация метода Клар\-ка--Рай\-та заключается
в~возможности варьирования глубины сращивания маршрутов~$p$, а~также
в~учете зависимости стоимости перевозки от направления прохождения цикла.
В~классическом алгоритме Клар\-ка--Рай\-та глубина $p\hm=0$.

  Изначально генерируются~$n$~цик\-лов вида 0--$i$--0
($i\hm=\overline{1,n}$). На каждом из последующих шагов рассматриваются
всевозможные допустимые пары циклов (т.\,е.\ такие, для которых суммарная
загрузка не превосходит~$Q$), для каждой из которых строится не более
8~цик\-лов при $p\hm =0$ и~не более $16p$~цик\-лов при $p\hm\geq 1$
следующим образом: каждый из новых циклов состоит из начального или
конечного отрезка (длиной не более~$p$) одного из циклов, затем проходится
второй цикл (без нулевой вершины), затем продолжается движение по первому
циклу. При этом учитывается возможность изменения направления движения
по каждому из циклов на противоположное, поскольку задача не
предполагается симметричной и~загрузка ТС зависит от порядка прохождения
пунктов.

  Например, сращивание циклов $a_1^1,a_2^1,\ldots, a^1_{k_1}$ и~$a_1^2,
a_2^2,\ldots, a^2_{k_2}$ (с исключенной нулевой вершиной) при единичных
начальном и~конечном отрезках дает следующие циклы:
  \begin{gather*}
  0, a_1^1, a_1^2, a_2^2,\ldots, a^2_{k_2}, a_2^1,\ldots, a^1_{k_1}, 0;\\
   0, a_1^1, a^2_{k_2}, a^2_{k_2-1},\ldots , a_1^2, a_2^1,\ldots, a^1_{k_1},0;\\
  0,a_{k_1}^1, a^1_{k_1-1}, \ldots, a_2^1, a_1^2, a_2^2,\ldots, a^2_{k_2},
a_1^1,0;\\
  0,a^1_{k_1}, a^1_{k_1-1},\ldots , a_2^1, a^2_{k_2}, a^2_{k_2-1},\ldots, a_1^2,
a_1^1,0\,.
  \end{gather*}

  Еще четыре цикла возникнут при добавлении части второго цикла между
пунктами $a^1_{k_1}$ и~$a^1_{k_1-1}$. Затем первый и~второй циклы можно
поменять мес\-та\-ми, тогда получим еще 8~циклов.

  В каждом случае вычисляется разность между суммой расходов на доставку
грузов по отдельным циклам и~расходов на доставку грузов по их
объединению и~выбирается объединенный цикл, для которого эта разность
максимальная. При этом в~модификации алгоритма Клар\-ка--Рай\-та для
каждой пары циклов перебираются все возможные варианты их сращивания
при всех значениях глубины сращивания от~0 до заданного пользователем
значения~$p$. Процесс прекращается при невозможности дальнейшего
сращивания циклов или отсутствии пары циклов, сращивание которых
выгодно~[7,~8].

\vspace*{-9pt}

\section{Вычислительный эксперимент}

\vspace*{-2pt}

  Задача (1)--(8), (11)--(14) решалась двумя способами: точным методом
и~модифицированным алгоритмом Клар\-ка--Рай\-та, которые были
реализованы в~среде Scilab-4.1.2. Поскольку на задачах с~размерностями
$n\hm\geq 8$ время решения линейной целочисленной задачи оказалось весьма
продолжительным (в~некоторых случаях более 45~ч), сравнение
алгоритмов производилось для случайно сгенерированных задач при
размерностях $n\hm=\overline{4,8}$. Для каж\-дой размерности было решено по
20~задач каж\-дым из алгоритмов.

  Генерация тестовых примеров осуществлялась для значений
грузоподъемности ТС $Q\hm\in [1{,}5,\,5]$ (т); элементы матрицы~$L$
протяженности дорог между пунктами генерировались из диапазона
$L_{ij}\hm\in [1,\,100]$, $i,j\hm=\overline{1,n}$ (км); элементы матрицы
коэффициентов сложности дорог между пунктами генерировались целыми из
диапазона $[1,\,5]$; вектор потребностей пунктов в~грузе генерировался из
диапазона $q_i\hm\in [300,\,1500]$ (кг). Функция расхода топлива для
используемого типа ТС задана полиномом первой степени от величины
загрузки ТС в~виде $f(q)\hm=15\hm+3q$.

  Все сгенерированные задачи были решены при помощи точного алгоритма,
а~также с~помощью модифицированного алгоритма Клар\-ка--Рай\-та при
значении параметра $p\hm=3$. Для каждого значения $n\hm= \overline{4,8}$
в~таблице приведены результаты численного эксперимента~--- средние
значения отклонения\linebreak от оптимального значения расхода топлива, среднее время
работы точного алгоритма и~модифицированного алгоритма Клар\-ка--Рай\-та.
Также приведены сведения о доле задач, решение которых алгорит\-мом
  Клар\-ка--Рай\-та совпало с~оптимальным.

\begin{table*}\small
\begin{center}
\begin{tabular}{|c|c|c|c|c|}
\multicolumn{5}{c}{Результаты численного эксперимента}\\[6pt]
\hline
&Точный алгоритм&\multicolumn{3}{c|}{Модифицированный алгоритм
Кларка--Райта}\\
\cline{2-5}
\tabcolsep=0pt\begin{tabular}{c}$n$\\ \ \end{tabular}&
\tabcolsep=0pt\begin{tabular}{c}Среднее время\\ работы алгоритма, с\end{tabular}&
\tabcolsep=0pt\begin{tabular}{c}Среднее время\\ работы алгоритма, с\end{tabular}&
\tabcolsep=0pt\begin{tabular}{c}Среднее отклонение\\ расхода топлива, \%\end{tabular}
&\tabcolsep=0pt\begin{tabular}{c}Совпадения результатов\\ с~оптимумом, \%\end{tabular}\\
\hline
4&2,70&17,57&3,08&85\\
5&13,74&20,82&4,47&75\\
6&213,85&24,36&3,04&65\\
7&6050,91&46,40&9,15&50\\
8&100\,084&54,60&14,26\hphantom{9}&25\\
\hline
\end{tabular}
\end{center}
\end{table*}



  Как видно из таблицы, модифицированный алгоритм Клар\-ка--Рай\-та на
небольших размерностях показывает хорошие результаты. Вычислительное
время, затрачиваемое точным алгоритмом, оказывается неприемлемо велико
(более 27~ч уже при $n\hm=8$), в~то время как вычислительное время,
затрачиваемое модифицированным алгоритмом Клар\-ка--Рай\-та, остается
приемлемым при существенном увеличении размерности

  Рисунок иллюстрирует зависимость времени работы эвристического
алгоритма от размерности за-\linebreak\vspace*{-12pt}
 \begin{center}
 \mbox{%
 \epsfxsize=77.428mm
 \epsfbox{bro-1.eps}
 }
 \end{center}
% \vspace*{-9pt}

\noindent
{\small Зависимость вычислительного времени работы модификации алгоритма
Клар\-ка--Рай\-та от размерности задач}

\vspace*{9pt}


\noindent
дачи. По горизонтали указаны значения~$n$~---
размерности задачи, по вертикали~--- среднее время вычисления (в секундах).
Как видно, при увеличении размерности время работы модифицированного
алгоритма Клар\-ка--Рай\-та возрастает, однако точный алгоритм на
идентичных размерностях работает несравнимо дольше.



\section{Заключение}

  Рассматриваемая проблема является актуальной прикладной задачей.
Полученные результаты позволяют сделать вывод о~целесообразности
использования модификации алгоритма Клар\-ка--Рай\-та для решения задачи
транспортировки грузов с~учетом загрузки ТС для
снижения отклонения получаемых результатов от точно найденного оптимума.

{\small\frenchspacing
 {%\baselineskip=10.8pt
 \addcontentsline{toc}{section}{References}
 \begin{thebibliography}{9}
\bibitem{1-br}
\Au{Dantzig G.\,B., Ramser J.\,H.} The truck dispatching problem~// Management
Sci., 1959. No.\,1. P.~80--91.
\bibitem{2-br}
\Au{Бронштейн Е.\,М., Заико Т.\,А}. Детерминированные оптимизационные
задачи транспортной логистики~// Автоматика и~телемеханика, 2010. №\,10.
С.~133--147.
\bibitem{3-br}
\Au{Ralphs T.\,K., Kopman L., Pulleyblank~W.\ R., Trotter~L.\,E., Jr.} On the
capacitated vehicle routing problem~// Math. Program. Ser. B, 2003. Vol.~94.
P.~343--359.
\bibitem{4-br}
\Au{Kara I., Kara B.\,Y., Kadri Yetis~M.} Energy minimizing vehicle routing
problem~// Combinatorial optimization and applications~/
Eds. Dress~A.\,W.\,M., Xu~Y., Zhu~B.
Lecture notes in computer science ser.~--- Springer, 2007. Vol.~4616.
P.~62--71.
\bibitem{5-br}
\Au{Зелёв П.\,А., Бронштейн Е.\,М.} Задача транспортной логистики с~учетом
зависимости расходов на транспортировку от загрузки транспортного
средства~// Логистика и~управление цепями поставок, 2010. №\,4. С.~39--45.
\bibitem{6-br}
\Au{Dror M., Laporte G., Trudeau~P.} Vehicle routing with split deliveries~//
Discrete Appl. Math., 1994. Vol.~50. P.~239--254.

\bibitem{8-br}
\Au{Clarke G., Right J.\,W.} Scheduling of vehicles from a central depot to a~number
of delivery points~// Oper. Res., 1963. No.\,11. P.~568--581.

\bibitem{7-br} %8
\Au{Бронштейн Е.\,М., Зелёв П.\,А.} Задача маршрутизации транспортного
средства с~учетом зависимости стоимости перевозок от загрузки~//
Информационные технологии, 2014. №\,4. С.~33--37.

 \end{thebibliography}

 }
 }

\end{multicols}

\vspace*{-3pt}

\hfill{\small\textit{Поступила в редакцию 06.02.14}}

\newpage

%\vspace*{12pt}

%\hrule

%\vspace*{2pt}

%\hrule

%\vspace*{12pt}

\def\tit{ABOUT OPTIMUM DELIVERY OF FREIGHTS BY~THE~VEHICLE TAKING INTO ACCOUNT DEPENDENCE
OF~COST OF~TRANSPORTATIONS ON~LOADING OF~VEHICLES ON~SEVERAL CYCLIC ROUTES}

\def\titkol{About optimum delivery of freights by~the vehicle
taking into account dependence of~cost
of~transportations on~loading of~vehicles} % on several cyclic routes}

\def\aut{E.\,M.~Bronshtein and P.\,A.~Zelyov}

\def\autkol{E.\,M.~Bronshtein and P.\,A.~Zelyov}

\titel{\tit}{\aut}{\autkol}{\titkol}

\vspace*{-9pt}

\noindent
Ufa State Aviation Technical University,
12 K.~Marx Str., Ufa 450000, Russian Federation


\def\leftfootline{\small{\textbf{\thepage}
\hfill INFORMATIKA I EE PRIMENENIYA~--- INFORMATICS AND
APPLICATIONS\ \ \ 2014\ \ \ volume~8\ \ \ issue\ 4}
}%
 \def\rightfootline{\small{INFORMATIKA I EE PRIMENENIYA~---
INFORMATICS AND APPLICATIONS\ \ \ 2014\ \ \ volume~8\ \ \ issue\ 4
\hfill \textbf{\thepage}}}

\vspace*{3pt}




  \Abste{The problem of creation of a route of freights delivery from
  one producer (base, a warehouse) to consumers by the vehicle with
  the minimum costs of transportations is considered. Dependence of cost
  of transportation on loading of the vehicle and quality of the road is
  thus considered. It is supposed that the vehicle can come back to the base
  for additional charge. The corresponding mathematical model is constructed;
  for a case of linear dependence of fare from loading, the linear integer model
  is received. For the solution of an objective along with the exact algorithm,
  modification
  of the known heuristic algorithm of Clark and Right is suggested. Computing experiment
  has been made.}

  \KWE{heuristic algorithm; creation of a route; transportation; problem of routing}


  \DOI{10.14357/19922264140407}


\Ack
\noindent
The work was financially supported by the Russian  Foundation for
Basic Research (project 13-01-00005).



%\vspace*{3pt}

  \begin{multicols}{2}

\renewcommand{\bibname}{\protect\rmfamily References}
%\renewcommand{\bibname}{\large\protect\rm References}



{\small\frenchspacing
 {%\baselineskip=10.8pt
 \addcontentsline{toc}{section}{References}
 \begin{thebibliography}{9}
\bibitem{1-br-1}
\Aue{Dantzig, G.\,B., and J.\,H.~Ramser}. 1959. The truck dispatching problem.
\textit{Management Sci.} 1:80--91.
\bibitem{2-br-1}
\Aue{Bronshtein, E.\,M., and T.\,A.~Zaiko}. 2010. Deterministic optimizational
problems of transportation logistics. \textit{Automation Remote Control}
10:2132--2144.
\bibitem{3-br-1}
\Aue{Ralphs, T.\,K., L. Kopman, W.\,R.~Pulleyblank, and L.\,E.~Trotter, Jr.} 2003.
On the capacitated vehicle routing problem. \textit{Math. Program. Ser. B}.
94:343--359.
\bibitem{4-br-1}
\Au{Kara, I., B.\,Y. Kara, and M.~Kadri Yetis}. 2007. Energy minimizing vehicle
routing problem. \textit{Combinatorial optimization and applications}.
Eds.\ A.\,W.\,M.~Dress, Y.~Xu, and B.~Zhu.
Lecture notes in computer science ser. 4616:62--71.
\bibitem{5-br-1}
\Aue{Zelev, P.\,A., and E.\,M. Bronshteyn}. 2010. Zadacha transportnoy logistiki s~uchetom zavisimosti raskhodov na transportirovku ot zagruzki transportnogo sredstva
[The problem of transportation logistics, taking into account transportation costs,
depending on the load of the vehicle]. \textit{Logistika i~Upravlenie Tsepyami
Postavok} [Logistics and Supply Chain Management] 4:39--45.
\bibitem{6-br-1}
\Aue{Dror, M., G. Laporte, and P.~Trudeau}. 1994. Vehicle routing with split
deliveries. \textit{Discrete Appl. Math.} 50:239--254.
\bibitem{7-br-1} %8
\Aue{Bronshteyn, E.\,M., and P.\,A.~Zelev}. 2014. Zadacha marsh\-ru\-ti\-za\-tsii
transportnogo sredstva s~uchetom zavisimosti stoimosti perevozok ot zagruzki
[Vehicle routing problem, taking into account the cost of transportаtion, depending
on the load]. \textit{Informatsionnye Tekhnologii} [Information Technologies]
4:33--37.
\bibitem{8-br-1} %7
\Aue{Clarke, G., and J.\,W.~Right}. 1963. Scheduling of vehicles from a~central
depot to a~number of delivery points. \textit{Oper. Res.} 11:568--581.
\end{thebibliography}

 }
 }

\end{multicols}

\vspace*{-6pt}

\hfill{\small\textit{Received February 6, 2014}}

\vspace*{-18pt}

\Contr

\noindent
\textbf{Bronshtein Efim M.}\ (b.\ 1946)~--- Doctor of Science in physics and mathematics; professor,
Ufa State Aviation Technical University, 12 K.~Marx Street, Ufa 450000, Russian Federation; bro-efim@yandex.ru

\vspace*{3pt}

\noindent
\textbf{Zelyov Pavel A.}\ (b.\ 1988)~---
PhD student,  Ufa State Aviation Technical University, 12 K.~Marx Street, Ufa 450000, Russian Federation;
pz1988@yandex.ru
\label{end\stat}

\renewcommand{\bibname}{\protect\rm Литература}
		
 %7
\def\vo{\;\mathop{\to}\limits_{r}\;}
\def\eam{\mathbin{{\mathop{=}\limits^{\mathrm{def}}}}}


\def\stat{mironov}

\def\tit{МЕТОД ПОВЫШЕНИЯ ЭФФЕКТИВНОСТИ РЕШЕНИЯ
ЗАДАЧ  ВЕРОЯТНОСТНОЙ ВЕРИФИКАЦИИ ВЫЧИСЛИТЕЛЬНЫХ И~ТЕЛЕКОММУНИКАЦИОННЫХ СИСТЕМ$^*$}



\def\titkol{Метод повышения эффективности решения
задач  вероятностной верификации вычислительных %и телекоммуникационных
систем}

\def\aut{А.\,М.~Миронов$^1$, С.\,Л.\,Френкель$^2$}

\def\autkol{А.\,М.~Миронов, С.\,Л.\,Френкель}

\titel{\tit}{\aut}{\autkol}{\titkol}

{\renewcommand{\thefootnote}{\fnsymbol{footnote}} \footnotetext[1]
{Работа выполнена при частичной поддержке РФФИ (проект 12-07-00109).}}


\renewcommand{\thefootnote}{\arabic{footnote}}
\footnotetext[1]{Институт проблем информатики
Российской академии наук,  amironov66@gmail.com}
\footnotetext[2]{Институт проблем информатики Российской
академии наук; Московский государственный технический университет
радиотехники, электроники и автоматики (МГТУ МИРЭА), fsergei@mail.ru}

\vspace*{6pt}


\Abst{Рассматривается проблема снижения трудоемкости
вероятностной верификации при проектировании вычислительных систем.
Поставленная
цель достигается редукцией вероятностных систем переходов
(ВСП), моделирующих проектируемые системы.
Верификация ВСП заключается в~вычислении истинностных значений
формул вероятностной темпоральной логики (PCTL, Probabilistic
Com\-pu\-ta\-ti\-o\-nal Tree Logic) в~начальных состояниях ВСП.
Редукция ВСП  выполняется  по алгоритму удаления эквивалентных
состояний, в~результате работы которого получается такая ВСП,
у~которой все свойства, выражаемые формулами логики PCTL, совпадают со
свойствами исходной ВСП.}

\KW{верификация; вероятностные системы переходов; %цепи Маркова;
вероятностная темпоральная логика;
редукция вероятностных моделей}

\DOI{10.14357/19922264140408}

\vspace*{6pt}


\vskip 12pt plus 9pt minus 6pt

\thispagestyle{headings}

\begin{multicols}{2}

\label{st\stat}


\section{Введение}

\subsection{Постановка задачи}


Необходимость в~вероятностной верификации проектов цифровых систем
возникает либо при проектировании систем со стохастическим
поведением, например многоканальных телекоммуникационных систем,
либо в~случаях,  когда у~разработ\-чика есть основание полагать, что
проектируемая\linebreak система в~рабочем режиме может быть подвержена
различным не специфицированным при проектировании ошибкам
и~случайным сбоям как внутрен\-ней природы, так и~инициированных
внешними воздействиями. Поскольку точно локализация и~функциональные
последствия наличия таких\linebreak ошибок априори не известны, их можно
попытаться характеризовать вероятностью проявления в~результатах
работы и,~соответственно, говорить\linebreak о~вероятностной верификации.
Наиболее распространенным подходом к формальной вероятностной
верификации является Probabilistic Model Checking~[1,
гл.~11], дополняющий проверку соот\-ветствия формальной спецификации
про\-ек\-ти\-ру\-емой системы ее свойствам (properties) вычис\-ле\-нием
вероятностей выполнения этих свойств.\linebreak В~данном случае проектируемые
системы описываются моделью ВСП, которая
используется в~алгоритмах формальной верификации, основанных на
проверке моделей (Мodel Сhecking). Одна из главных проблем
использования Probabilistic Model Checking, как и~прочих формальных
методов спецификации, состоит в~их вычислительной сложности,
и~поэтому снижение размера соответствующих моделей, в~частности матриц
тех или иных переходов, является важнейшим фактором реализуемости
соответствующих методов.

В настоящей работе рассматривается задача редукции ВСП,
целью которой является понижение сложности
верификации свойств ВСП, выражаемых формулами вероятностной
темпоральной логики PCTL. Вероятностные системы переходов
представляют собой один из наиболее
широко используемых классов моделей дискретных динамических систем.
Понятие ВСП является обобщением понятия цепи Маркова~\cite{markov},
которое нашло широкое применение в~естественных  и~гуманитарных
науках. Понятие ВСП можно рассматривать также как\linebreak частный случай
понятия вероятностного автомата~\cite{buh}. Главной отличительной
особенностью понятия ВСП от понятий цепи Маркова и~вероятностного
автомата является наличие выразительного\linebreak логическо\-го формализма,
позволяющего эффективно описывать различные свойства поведения ВСП.
В качестве такого формализма выступает вероятностная темпоральная
логика PCTL~[4, 5], которая представляет собой вероятностный аналог
темпоральной логики ветвящегося времени CTL~\cite{peled},
использующейся для спецификации свойств параллельных и~распределенных программ,
и~является эффективным инструментом для
описания различных свойств дискретных вероятностных динамических
систем.

Формулы логики PCTL могут отражать различные вероятностные аспекты
поведения анализируемых систем, к~числу которых относятся, например,
частота выполнения тех или иных действий или переходов в~анализируемых системах,
вероятность отказа компонентов анализируемых
систем, вероятностный характер взаимодействия анализируемой системы
с ее окружением, например: час\-то\-та поступления входных запросов или
сообщений, частота получения искаженных сообщений (для протоколов
передачи сообщений в~компьютерных сетях) и~т.\,п.


В данной работе  уточняются основные формулировки  и~демонстрируется
на новом примере  решение задачи, сформулированной в~\cite{mf},
а~именно:  преобразование ВСП проектируемой системы в~эквивалентную
ВСП с меньшим числом со\-сто\-яний. Под эквивалентностью понимается, что
результаты верификации исходной и~редуцированной модели будут
одинаковы.

Некоторые подходы к~редукции ВСП изучались в~различных работах по
вероятностной верификации, однако в~этих исследованиях были
рас\-смот\-ре\-ны лишь частные методы редукции ВСП, такие как редукция
частичных порядков~[8, 9] и~редукция, основанная на понятии
симметрии множества состояний ВСП~[10, 11]. Данные методы можно
эффективно использовать лишь для ВСП достаточно специального вида:
как правило, это вероятностные модели параллельных и~распределенных
программ.


\subsection{Современное состояние проблемы вероятностной верификации}


Первые алгоритмы вероятностной верификации были предложены в~1980-е~гг.\
в~работах~[12--14]. Данные алгоритмы были предназначены для
верификации качественных вероятностных свойств (т.\,е.\ таких,
которые выполняются с~вероятностью~1 или~0). Затем эти алгоритмы
были обобщены на случай верификации количественных вероятностных
свойств (в~спецификации таких свойств могло присутствовать любое
значение вероятности). Эти алгоритмы были изложены в~работах~[4, 15, 16].
Программные реализации этих алгоритмов были представлены в~работах [17, 18].

Первые промышленные системы вероятностной верификации были
разработаны в~2000-х~гг.~[19, 20]. Эти системы  успешно
применяются во многих областях, таких как анализ распределенных
алгоритмов, телекоммуникационные протоколы, компьютерная
безопасность, криптографические протоколы, моделирование
биологических процессов. С~использованием этих систем верификации
были обнаружены уязвимости и~аномальные поведения анализируемых
систем (подробнее см.\ в~\cite{50}). При помощи систем вероятностной
верификации могут быть вычислены такие характеристики про\-грам\-мных
систем, как, например, вероятность вторжения злоумышленника
в~компьютерную сеть, математическое ожидание времени отклика
веб-сер\-ви\-са и~другие количественные и~качественные характеристики.

Наиболее популярной практической системой вероятностной верификации
в настоящее время является система PRISM~\cite{55},  разработанная
на факультете компьютерных наук Оксфордского университета
(Великобритания) в~группе Quan\-ti\-ta\-ti\-ve Ana\-ly\-sis and
Ve\-ri\-fi\-ca\-ti\-on под руководством Марты Квятковской.
Информация о~деятельности этой группы представлена на веб-сайте
{\sf http://qav.comlab.ox.ac.uk/}.

\section{Вероятностные системы переходов}
\label{fdgfdsgsdfgsdfgrr4444}

\subsection{Понятие вероятностной системы переходов}

Предположим, что задано конечное множество~AP, элементы которого
называются {\bf атомарными утверждениями}.
Ниже запись $2^{\mathrm{AP}}$ обозначает множество всех подмножеств~AP.

{\bf Вероятностная система переходов}
(называемая также в англоязычной литературе {\bf Discrete Time Markov Chain})~---
это четверка $D$ вида
\begin{equation*} %{dfsgdsfgf33dsgdsgfds}
D=(S, s^0, P, L)\,,
\end{equation*}
компоненты которой имеют следующий смысл:
\begin{enumerate}[(1)]
\item  $S$~---  множество,    элементы которого называются
    {\bf состояниями} ВСП~$D$;
\item  $s^0 \in S$~--- выделенное состояние, называемое
 {\bf начальным состоянием} ВСП~$D$;
 \item  $P$~--- функция вида $P:S\times S\hm\to [0,1]$,
   на\-зы\-ва\-емая {\bf функцией перехода} ВСП~$D$ и~ удовлетворяющая условию:
  $\forall\, s\hm\in S\quad
  \sum\limits_{s'\in S}P(s,s')\hm=1.$
Для каждой пары $(s_1,s_2)\hm \in S\times S$
число $P(s_1,s_2)$ понимается как вероятность
того, что если в~текущий момент времени~$D$ находится в~состоянии~$s_1$,
то через один такт времени~$D$ будет находиться
в~состоянии~$s_2$.
Если $P(s_1,s_2)\hm>0$, то будем называть тройку
$(s_1, s_2, P(s_1,s_2))$ {\bf переходом} из~$s_1$ в~$s_2$
с~вероятностью $P(s_1,s_2)$. Ниже запись $s_1\ra{a}s_2$ является другим обозначением
перехода $(s_1, s_2, a)$;\\[-14pt]

\item $L$~--- функция вида
$L:S\to 2^{\mathrm{AP}}$, называемая {\bf оценкой}, которая имеет следующий смысл:
   для каждого состояния    $s\hm\in S$ и~каждого атомарного утверждения
   $p\hm\in \mathrm{AP}$ утверждение~$p$  считается
{\bf истинным} в~ $s$, если $p\hm\in L(s)$, и~{\bf ложным} в~$s$,
если $p\hm\not\in L(s)$.
\end{enumerate}


Вероятностную систему переходов удобно рассматривать как помеченный граф, вершинами которого
являются состояния, помеченные элементами множества $2^{\mathrm{AP}}$: каждая
вершина $s\hm\in S$ имеет метку $L(s)$ и~для каждой пары $(s_1,s_2)\hm\in
S\times S$ такой, что  $P(s_1, s_2) \hm>0$, граф содержит
ребро из~$s_1$ в~$s_2$ с~меткой $P(s_1, s_2)$.

\vspace*{-7pt}

\subsection{Случайные функции}

Пусть $X$ и~$Y$~--- два конечных множества.

{\bf Случайной функцией} (СФ) из~$X$ в~$Y$
называется произвольная функция~$f$ вида
\be{dfgfdsgfds}
f: X\times Y \to [0,1]
\ee
такая, что
$\forall\,x\hm\in X\quad \sum\limits_{y\in Y}f(x,y)\hm=1$.

Для любых $x\in X$ и~$y\hm\in Y$ значение $f(x,y)$ можно интерпретировать
как вероятность того, что СФ~$f$ отображает~$x$ в~$y$.

Случайная функция~\re{dfgfdsgfds} называется {\bf детерминированной},
если для каждого $x\hm\in X$ существует
единственный $y\hm\in Y$, такой что
$f(x,y)\hm=1$. Если~$f$~--- детерминированная СФ
вида~\re{dfgfdsgfds} и~$x,y$~--- такие элемен\-ты~$X$
и~$Y$ соответственно, что $f(x,y)\hm=1$,
то  будем говорить, что {\bf $f$ отображает~$x$ в~$y$}.

Если $f$~--- СФ из~$X$ в~$Y$, то  будем обозначать этот факт
записью $f: X\vo Y$. Будем называть~$X$ {\bf об\-ластью определения} СФ~$f$,
а~$Y$~--- {\bf областью значений} СФ~$f$.

Для каждого конечного множества~$X$ запись id$_X$ обозначает детерминированную
СФ $X\hm\to X$, которая отображает каждый $x\hm\in X$ в~$x$.

\vspace*{-7pt}

\subsection{Матрицы, соответствующие случайным функциям}

Если СФ $f$ имеет вид $f: X\vo Y$ и~на множествах~$X$ и~$Y$
заданы  упорядочения их элементов, которые имеют вид
$(x_1,\ldots, x_m)$ и $(y_1,\ldots, y_n)$
со-\linebreak\vspace*{-12pt}

\columnbreak

\noindent
 ответственно, то СФ~$f$ можно представить в~виде матрицы
(обозначаемой тем же символом~$f$)
\be{dfgdsghdsfgdsg}
f=\begin{pmatrix}
f(x_1,y_1)&\cdots&f(x_1,y_n)\\
\vdots&\cdots&\vdots\\
f(x_m,y_1)&\cdots&f(x_m,y_n)
\end{pmatrix}\,.
\ee

Ниже  будем отождествлять СФ~$f$ с~матрицей~\re{dfgdsghdsfgdsg}.

Будем предполагать, что для каждого конечного множества~$X$, являющегося
областью определения или областью значений ка\-кой-либо из
рассматриваемых СФ, на~$X$ задано фиксированное упорядочение
его элементов. Таким образом, для каждой рассматриваемой СФ
соответствующая  ей матрица определена однозначно.

Для каждой СФ $f:X\vo Y$ и произвольных $x\hm\in X$, $y\hm\in Y$
 будем называть
 \bi
\item строку $(f(x,y_1),\ldots, f(x,y_n))$
   матрицы~$f$~--- {\bf строкой $x$};
\item столбец $\begin{pmatrix}
f(x_1,y)\\
\vdots\\
f(x_m,y)\end{pmatrix}$    матрицы~$f$~--- {\bf столбцом~$y$}.
\ei


Если $f$ и~$g$~--- СФ вида $f: X\vo Y$, $g: Y\vo Z$, то
их {\bf композицией} называется СФ $f\cdot g: X\vo Z$, определяемая следующим
образом:
\be{dsfgdsfgdsf}
\forall\,x\in X\quad
(f\cdot g)(x)\eam \sum\limits_{y\in Y}f(x,y)\cdot
g(y,z)\,.
\ee

По определению произведения матриц из~\re{dsfgdsfgdsf} следует, что
матрица $f\cdot g$ является произведением матриц~$f$ и~$g$.

\vspace*{-6pt}

\subsection{Случайные функции, соответствующие вероятностным системам переходов}
\label{dfgdgdsf44555333}

Пусть задана ВСП $D\hm=(S,s^0, P, L)$ и список элементов множества~$S$
имеет вид $(s_1,\ldots, s_n)$.

Будем использовать  следующие обозначения.
\bn
\item Символ {\bf 1} означает  множество,
состоящее из одного элемента, который будем обозначать символом~$e$.
\item Для каждого состояния $s\hm\in S$ запись~$I_s$
обозначает детерминированную СФ вида
$I_s: {\bf 1} \vo S$, отображающую элемент $e\hm\in {\bf 1}$ в~состояние~$s$
ВСП~$D$.

\item Для каждого $n\hm\geq 0$
обозначим записью~$P^n$ СФ вида $P^n: S\vo S$, определяемую индуктивно:
$P^0\eam \mathrm{id}_S$ и~$\forall\,n\geq 0\enskip P^{n+1}\eam P^n\cdot P$.
Нетрудно видеть, что матрицы, соответствующие
СФ~ $P^i$, имеют следующий вид:
$P^0$~--- единичная матрица и~$\forall\,n\hm>0$
матрица~$P^n$  является $n$-й степенью матрицы~$P$.

Для любых $n\geq 0$, $s_1, s_2\hm\in S$ число $P^n(s_1,s_2)$
можно понимать как вероятность того, что если в~текущий
момент времени ВСП~$D$ находится в~состоянии~$s_1$,
то через~$n$~тактов времени $D$ будет находиться
в~состоянии~$s_2$.
\en

\vspace*{-9pt}

\section{Логика PCTL}

{\bf Логика PCTL}~--- это темпоральная логика,
предназначенная для формального описания свойств ВСП.
Логика PCTL была введена Х.~Ханссоном  и~Б.~Джонссоном  в~работе~\cite{35}.


\subsection{Формулы логики PCTL}

В~определении понятия формулы логики PCTL
 будем использовать множество~AP атомарных утверждений, введенное в~разд.~2.

Формулы логики PCTL делятся на два класса: StateFm~--- {\bf формулы
состояний}~--- и~PathFm~--- {\bf формулы путей}. Формулы из классов
StateFm и~PathFm  будем обозначать символами~$\varphi$
и~$\alpha$ соответственно (возможно, с индексами), а~формулу
произвольного вида~--- символом~$f$ (возможно, с~индексом).

Классы StateFm и~PathFm определяются следующим образом.

\smallskip

StateFm:

\smallskip

\bn
\item Каждое атомарное утверждение~$p$ из~AP является формулой из
StateFm.
\item Символы $\top$ и~$\bot$ является формулами
из  StateFm. Данные символы обозначают тож\-дественно истинное
и тождественно ложное утверждение соответственно.
\item Если $\varphi_1$ и~$\varphi_2$~--- формулы
из StateFm, то  следующие знакосочетания являются формулами из StateFm:
$
\neg \varphi_1$; $\varphi_1\wedge \varphi_2$;
$\varphi_1\vee \varphi_2$;
$\varphi_1\to \varphi_2$;
$\varphi_1\leftrightarrow \varphi_2$.
\item Если
\begin{itemize}
\item $\triangle$~--- функциональный символ, которому соответствует функция
(обозначаемая тем же символом) вида
$$\triangle: [0,1]\times [0,1]\to \{0,1\}\,;
$$
  \item $a$~--- число из $[0,1]$;

    \item $\alpha$~--- формула из PathFm,
\end{itemize}
то знакосочетание $\mathcal{P}_{\triangle a} \alpha$ является формулой из StateFm.
\en

PathFm:

\smallskip

\bn
\item Если $f$~--- формула логики PCTL, то  знакосочетание ${\bf X}f$
является формулой из PathFm.
\item Если $\varphi_1$ и~$\varphi_2$~--- формулы
из StateFm, то  следующие знакосочетания являются формулами из PathFm:
\begin{itemize}
\item[(а)] $\varphi_1{\bf U}^{\leq n}\varphi_2$,
где $n$~--- натуральное число;
\item[(б)] $\varphi_1{\bf U}\varphi_2$.
\end{itemize}
\item Если $\alpha$~--- формула из PathFm,
то знакосочетание $\neg \alpha$ является формулой
из PathFm.
\en

В записи формул из PathFm могут использоваться символы {\bf F} и~${\bf G}$,
которые являются сокращением знакосочетаний $\top{\bf U}$
и $\neg{\bf F} \neg$ соответственно (т.\,е., например, знакосочетания
 ${\bf F} \alpha$ и~${\bf G}^{\leq n}\alpha$
обозначают формулы
$\top{\bf U} \alpha$ и~$\neg {\bf F}^{\leq n}\neg \alpha$
соответственно).

\subsection{Значения формул логики PCTL
в~состояниях вероятностных систем переходов}
\label{dsfgdsfgdfgdfgdsf445}

Пусть $D=(S, s^0, P, L)$~--- некоторая ВСП.

Для каждого состояния $s\hm\in S$ и~каждой фо\-р\-му\-лы~$f$ логики PCTL
определено {\bf значение} формулы~$f$ в~состоянии~$s$, которое обозначается
записью $s(f)$,~и
\begin{enumerate}[(1)]
\item если $f\hm\in \mathrm{StateFm}$, то $s(f) \hm\in \{0,1\}$
   и
   \begin{itemize}
   \item в~случае $s(f)=1$  формула~$f$ считается
   истинной в~$s$;
   \item в~случае $s(f)=0$
   формула~$f$ считается  ложной в~$s$;
   \end{itemize}
\item если $f\in \mathrm{PathFm}$, то значение
$s(f)$ является числом из  $[0,1]$ и~интерпретируется
как вероятность того, что  формула~$f$ истинна в~состоянии~$s$.
\end{enumerate}

Для каждой формулы~$f$ логики PCTL
будем обозначать записью $S(f)$ век\-тор-столбец
$
\begin{pmatrix}
 s_1(f)\\
 \vdots\\
 s_n(f)
\end{pmatrix}.$

Значения формул логики PCTL в~состояниях ВСП определяются индукцией
по структуре формул в~соответствии с~излагаемыми ниже правилами.
В~одних из этих правил определяется значение $s(f)$, в~других~--- определяется
век\-тор-стол\-бец $S(f)$ целиком. В~этих определениях будем использовать следующие обозначения:
\begin{itemize}
\item для любых векторов $U\hm=\begin{pmatrix}
    u_1\\ \vdots\\u_n
\end{pmatrix}$, $V\hm=\begin{pmatrix}
    v_1\\ \vdots\\v_n
\end{pmatrix}$
из $[0,1]^n$ записи $\max(U,V)$ и~$U\circ V$    обозначают векторы
      $ \begin{pmatrix}
      \max(u_1,v_1)\\ \vdots\\ \max(u_n,v_n)
\end{pmatrix}$ и
$\begin{pmatrix} u_1\cdot v_1\\ \vdots\\ u_n \cdot v_n
\end{pmatrix}
$
соответственно.
   \item
   если $A$ и~$B$~--- матрицы порядков $n\times n$
   и~$n\hm\times 1$ соответственно с~компонентами из $[0,1]$,
   то запись $[A^*\cdot B]$ обозначает матрицу, получаемую
%\begin{itemize}
%\i
заменой
   всех ненулевых компонентов $A$ и~$B$ на~1 и
%   \i
вычислением $(\sum\limits_{i\geq 0}A^i)\cdot B$, где сложение
понимается как дизъюнкция (т.~е.\ $\sum\limits_{i\geq 0}A^i$
является конечной). %\end{itemize}
\end{itemize}

Правила определения значений формул логики PCTL
в~состояниях ВСП имеют следующий вид:
\bi
\item
Для каждого $p\hm\in \mathrm{AP}$

\vspace*{-2pt}

\noindent
$$
s(p) \eam \begin{cases}
1, &\ \mbox{если\ } p\in L(s)\,;\\
0 &\ \mbox{иначе}\,;\end{cases}
$$

\vspace*{-6pt}

\item $s(\top)\eam 1,\;\;s(\bot)\eam 0$;\\[-13pt]
\item $s(\neg f)\eam 1-s(f)$, $s(\varphi_1\wedge \varphi_2)\eam
s(\varphi_1) \cdot s(\varphi_2)$
и~т.\,д. (т.\,е.\ значения формул коммутируют с~булевыми операциями);\\[-13pt]
\item
$s(\mathcal{P}_{\triangle a} \alpha) \eam \triangle(s(\alpha),a)$;\\[-13pt]

\item $S({\bf X}f) \eam P\cdot S(f)$;\\[-13pt]

\item пусть $\alpha_n \hm= \varphi_1{\bf U}^{\leq n}\varphi_2$
(где $n\hm\geq 0$). Тогда

\vspace*{-8pt}

\noindent
\begin{gather*}
S(\alpha_0)\eam S(\varphi_2)\,;\\
\hspace*{-3.5mm}\forall\, n>0\ S(\alpha_n)\eam
\max\left( S(\varphi_2),\;
S(\varphi_1)\circ S({\bf X} \alpha_{n-1})
\right)\,.
\end{gather*}

\vspace*{-5pt}

\item пусть $\alpha \hm= \varphi_1{\bf U}\varphi_2$.
Тогда $S(\alpha)$ определяется системой линейных уравнений

\vspace*{-8pt}

\noindent
\begin{multline*}
S(\alpha) ={}\\
\hspace*{-12pt}{}=\max\left ( S(\varphi_2),
\left[P^*\cdot
 S(\varphi_2)\right] \circ S(\varphi_1) \circ (P\cdot S(\alpha))
\right).
\end{multline*}
\ei

\vspace*{-9pt}


\section{Метод редукции вероятностных систем переходов}

\vspace*{-6pt}

\subsection{Задача редукции вероятностных систем переходов}

Если анализируемая ВСП имеет большой размер, то анализ ее свойств,
выражаемых формулами логики PCTL (т.\,е.\ вычисление значений формул
логики PCTL в~состояниях этой ВСП), может быть
 связан
с~трудновыполнимыми требованиями к вычислительным ресурсам,
с~использованием которых производится этот анализ.
В~связи с~этим представляет большую актуальность проблема редукции
ВСП, т.\,е.\ удаления части состояний и~переходов анализируемой ВСП
с~таким расчетом, чтобы получившая ВСП была эквивалентна исходной
в~следующем смысле: для каждой формулы состояний~$f$ логики PCTL
формула~$f$ истинна в~начальном состоянии исходной ВСП тогда и~только тогда,
когда она истинная в~начальном состоянии редуцированной ВСП.

Основная идея предлагаемого в~настоящей работе
метода редукции ВСП основана на понятии эквивалентности состояний ВСП:
будем называть состояния эквивалентными, если значения
всех формул логики PCTL в~этих состояниях совпадают.
Алгоритм редукции ВСП представляет собой вы\-чис\-ле\-ние классов эквивалентности
состояний анализируемой ВСП и~удаление эквивалентных состояний.

\vspace*{-6pt}

\subsection{Эквивалентность  вероятностных систем переходов}

Пусть заданы две ВСП:
\be{sdfgdsgdsfgdsf}
D_i=(S_i,s_i^0,P_i,L_i)\enskip
(i=1,2)\,.
\ee

Будем называть состояния $s_1\hm\in S_1$ и $s_2\hm\in S_2$ {\bf эквивалентными},
если для каждой формулы~$f$ логики PCTL верно равенство $s_1(f) \hm=  s_2(f).$

Если состояния $s_1$ и~$s_2$ эквивалентны, то будем обозначать это записью $s_1\hm\sim s_2$.

Будем называть ВСП $D_1$ и~$D_2$ вида~\re{sdfgdsgdsfgdsf} {\bf эквивалентными},
если $s^0_1\hm\sim s^0_2$. Если ВСП~$D_1$ и~$D_2$ эквивалентны,
то будем обозначать этот факт записью $D_1\hm\sim D_2$.

Если ВСП $D_1$ и~$D_2$ совпадают и~$S$~--- множество их состояний, то
бинарное отношение на~$S$, состоящее из всех пар $(s_1, s_2)$
таких, что $s_1\hm\sim s_2$, является отношением эквивалентности.
Будем обозначать это отношение символом~$\sim$.

Отношение $\sim$ может быть найдено при помощи алгоритма, излагаемого
в~параграфе~5.3.

\vspace*{-4pt}

\subsection{Редукция вероятностных систем переходов}
%\label{dfgdfgdsfgdeeeee}

\vspace*{-2pt}

\subsubsection{Задача редукции вероятностных систем переходов}

Пусть задана ВСП $D\hm=(S, s^0, P,L)$.

Задача редукции ВСП~$D$ заключается в~построении ВСП~$D'$,
которая эквивалентна~$D$ и~чис\-ло состояний которой меньше, чем
чис\-ло состояний ВСП~$D$.

Излагаемый в~настоящем пункте алгоритм редукции ВСП является
вероятностным обобщением алгоритма редукции детерминированных
автоматов. Идея данного алгоритма основана на отож\-де\-ст\-вле\-нии
неразличимых состояний ВСП:
\begin{enumerate}[(1)]
\item алгоритм вычисляет классы $S_1, \ldots, S_k$ разбиения множества~$S$,
соответствующего эквивалентности~$\sim$;
\item  ВСП $D$ преобразуется путем удаления
состояний в~классах $S_1, \ldots, S_k$ (и~соответствующего
переопределения функции перехода) до тех пор, пока не останется по
одному состоянию в~каждом из этих классов.
\end{enumerate}
В результате этих удалений получается искомая ВСП~$D'$.

\subsubsection{Построение разбиения множества состояний редуцируемой вероятностной системы переходов}
\label{sdfagdsagdsfgdsfgds44455}

Разбиение множества~$S$ состояний ВСП $D\hm=(S,s^0,P,L)$,  соответствующее
отношению эквивалентности~$\sim$, вычисляется следующим образом:
\begin{enumerate}[(1)]
\item вычисляется разбиение~$\Sigma^0$, соответствующее отношению
 эквивалентности $\{ (s_1,s_2)\hm\in S\times S\mid
L(s_1)\hm=L(s_2) \}$;
\item затем работает цикл, состоящий из следующих шагов.

Пусть для некоторого $i\hm\geq 0$ определены
\bi
\item  отношение эквивалентности $\rho^i$;
\item соответствующее ему разбиение~$\Sigma^i$, которое состоит из классов
$S^i_1, \ldots, S^i_k$.
\ei
Обозначим записями $\Sigma^i_1, \ldots, \Sigma^i_k$ строки матрицы
$\pi^i$, соответствующей детерминированной СФ $\pi^i: S\hm\to
\Sigma^i$, и~$\varphi^{\Sigma^i}_1,\ldots, \varphi^{\Sigma^i}_k$~---
список формул таких, что $\forall\,j=1,\ldots, k \quad
S(\varphi^{\Sigma^i}_j) \hm= \Sigma^i_j$.

Определим отношение эквивалентности
$\rho^{i+1}$ на~$S$:
\begin{multline*}
%\label{fdsgdsgds44545}
\rho^{i+1} \eam \rho^{i} \cap \left\{
\vphantom{\varphi^{\Sigma^i}_j}
(s_1,s_2)\in S^2\mid \forall\,j=1,\ldots, k\right.\\
\left. s_1({\bf X} \varphi^{\Sigma^i}_j) = s_2({\bf X} \varphi^{\Sigma^i}_j)
\right \}\,.
\end{multline*}


Разбиение $\Sigma^{i+1}$, соответствующее отношению~$\rho^{i+1}$, можно построить
следующим образом:
\bi
\item вычисляются век\-тор-столб\-цы
\be{fgfsdgdfgfdsgrrr444}S({\bf X}
\varphi^{\Sigma^i}_j) = P\cdot \Sigma^i_j
\ee
(каждый из которых, как нетрудно видеть, является
суммой некоторых столбцов мат\-ри\-цы~$P$: для каждого
$j\hm=1,\ldots, k$ век\-тор-стол\-бец~\re{fgfsdgdfgfdsgrrr444}
 является суммой таких столбцов~$s$ матрицы~$P$, для которых
 $s\hm\in S_j$);
 \item классы разбиения~$\Sigma^{i+1}$ получаются путем измельчения классов
 разбиения~$\Sigma^{i}$: в~один и~тот же класс разбиения~$\Sigma^{i+1}$
попадают такие состояния, для которых со\-от\-вет\-ст\-ву\-ющие им компоненты
векторов~\re{fgfsdgdfgfdsgrrr444} совпадают для каждого $j\hm=1,\ldots,
k$.
\ei

Возможны два случая:
\begin{itemize}
\item[(а)] $\Sigma^{i+1}=\Sigma^{i}$.
В~этом случае искомое раз\-би\-ение~$\sim$ найдено:
оно совпадает с~$\Sigma^i$;

\item[(б)] $\Sigma^i \neq \Sigma^{i+1}$.
   В~этом случае увеличиваем~$i$ на~1 и~возвращаемся в~начало цикла
   (т.\,е.\ выполняем шаг~2 с увеличенным значением~$i$).

Нетрудно видеть, что таких возвращений может быть не больше
числа элементов множества~$S$ (так как разбиение~$\Sigma^{i+1}$
является измельчением разбиения~$\Sigma^{i}$).
\end{itemize}
\end{enumerate}



\subsubsection{Удаление эквивалентных состояний из~вероятностных систем переходов}

Пусть ВСП $D\hm=(S,s^0,P,L)$ содержит пару эквивалентных состояний~$s_1$,
$s_2$, где $s_1\hm\neq s^0$. Определим ВСП
\be{fgdsgsdfrestettyt}
D_1 \eam (S_1,s^0,P_1,L_1)\,,
\ee
где $S_1\eam S\setminus \{s_1\}$,
$\forall\,s,s'\hm\in S_1$;
$$
P_1(s, s')\eam \begin{cases}
P(s, s')+P(s, s_1),&\ \mbox{если }
s'=s_2\,;\\
P(s, s'),&\ \mbox{если }
s'\neq s_2,
\end{cases}
$$
$\forall\,s\in S_1$ $L_1(s)\eam L(s)$.

Таким образом,  матрица~$P_1$ получается из  мат\-ри\-цы~$P$
 прибавлением к~столбцу~$s_2$ столбца~$s_1$ и~удалением строки~$s_1$
 и~столбца~$s_1$, а~матрица~$L_1$ получается из  матрицы~$L$
удалением строки~$s_1$.

Будем говорить что ВСП~\re{fgdsgsdfrestettyt} получается из ВСП~$D$
путем {\bf удаления состояния~$s_1$, эквивалентного состоянию~$s_2$}.
По определению ВСП~\re{fgdsgsdfrestettyt} каждое ее
состояние является также и~состоянием ВСП~$D$.

Для каждого  $s\in S_1$ и~каждой формулы~$f$ логики PCTL
будем обозначать записями $s_{D}(f)$ и~$s_{D_1}(f)$
значения формулы~$f$ в~состоянии~$s$
в~ВСП~$D$ и~$D_1$ соответственно и~записями
$S_{D}(f)$ и~$S_{D_1}(f)$~--- век\-тор-столб\-цы значений формулы~$f$
в~состояниях ВСП~$D$ и~$D_1$ соответственно.

\smallskip

\noindent
\textbf{Теорема~1.}\
\textit{Пусть ВСП}~\re{fgdsgsdfrestettyt}
\textit{получается из ВСП~$D$ путем удаления состояния~$s_1$,
эквивалентного состоянию~$s_2$. Тогда}
$\forall\,s\hm\in S_1\enskip s_{D_1}(f) \hm= s_{D}(f).$

\subsubsection{Описание алгоритма редукции вероятностной системы переходов}

Теорема~1 является обоснованием
излагаемого ниже алгоритма редукции ВСП
$D\hm=(S,s^0,P,L)$. Этот алгоритм имеет следующий вид.
\begin{enumerate}[1.]
\item Вычисляется разбиение множества состояний ВСП~$D$,
соответствующее отношению эквивалентности $R\eam \sim$ (для этого
выполняются действия, изложенные в~п.~5.3.2).
\item Искомая ВСП~$D'$ строится путем
удаления состояний из ВСП~$D$ и~переопределения функции перехода и~отношения~$R$
следующим образом:
\begin{itemize}
\item[(а)] если
отношение~$R$ содержит пару $(s_1, s_2)$, такую что
$s_1\hm\neq s_2$ и~$s_2\hm\neq s^0$, то выберем произвольную такую
пару $(s_1, s_2)$ и~преобразуем компоненты ВСП~$D$ описываемым ниже образом. Будем
излагать данное преобразование в~терминах графа, соответствующего
ВСП~$D$ (данный граф будем обозначать тем же символом~$D$):
\begin{itemize}
\item[(i)]
если граф $D$ содержит ребро с началом в~некоторой вершине~$s$
и~с концом~$s_2$, то данное ребро удаляется, а~к~метке ребра
с~началом в~$s$ и~с~концом в~$s_1$ прибавляется число, равное метке
удаленного ребра. Данная операция выполняется до тех пор, пока
имеются ребра с~концом в~$s_2$;
\item[(ii)] вершина $s_2$ удаляется и,~кроме того, удаляются все ребра,
выходящие из этой вершины;
\item[(iii)] из~$R$ удаляются все пары, содержащие~$s_2$, и~осуществляется
переход к шагу~2a;
\end{itemize}

\item[(б)] если каждая пара, входящая в~$R$, имеет
вид $(s,s)$, то   работа завершается.
\end{itemize}
\end{enumerate}

\section{Пример редукции вероятностной системы~переходов}

В этом разделе рассматривается пример редукции вероятностной модели
протокола передачи сообщений через ненадежный канал связи, в~котором
пересылаемые сообщения могут пропадать или искажаться. Протокол
представляет собой систему, состоящую из двух агентов~--- отправителя
и~получателя, а~также канала,  в~который помещаются сообщения,
 пересылаемые от одного агента другому.
 Предполагается, что факт искажения
 получаемых сообщений может быть установлен, и~если исходное сообщение не может быть восстановлено
 из искаженного, то отправитель получает  сигнал о~необходимости повторной посылки
 этого сообщения. Как только сообщение успешно доходит до получателя,
отправителю посылается сигнал подтверждения успешного
получения и~он переходит к отправке следующего сообщения.
  Предполагается, что сигналы и~подтверждения
  отправителю не пропадают и~не искажаются в~канале.

Графовая модель этого протокола имеет  вид, представленный на рис.~1.

\vspace*{6pt}

\begin{center}  %fig1
\vspace*{2pt}
\mbox{%
 \epsfxsize=72.154mm
 \epsfbox{mir-1.eps}
 }
 \end{center}
%  \vspace*{2pt}

\noindent
{{\figurename~1}\ \ \small{Графовая модель протокола передачи сообщений через
ненадежный канал связи}}


\vspace*{12pt}


\addtocounter{figure}{1}


Переходы в~этом графе имеют следующий смысл.

\begin{enumerate}[1.]
\item Переход $s_0\ra{1}s_1$ заключается в~получении
 отправителем от внешнего источника  сообщения, которое должно быть
 передано через канал получателю.
\item Переход $s_1\ra{0,8}s_2$ заключается в~помещении сообщения
в~канал отправителем, причем сообщение в~канале не искажается.
\item Переход $s_1\ra{0,2}s_4$ заключается в~помещении сообщения в~канал
отправителем, причем сообщение в~канале искажается.
\item Переход $s_2\ra{0,3} s_1$ заключается в~потере неискаженного сообщения
в~канале и~посылке отправителю сигнала о~необходимости повторной передачи.
\item Переход $s_4\ra{0,3} s_1$ заключается в~посылке
отправителю сигнала о~том, что исходное сообщение не может быть
восстановлено из искаженного сообщения и~должно быть передано
повторно.
\item Переход $s_2\ra{0,7}s_3$ заключается в~передаче
неискаженного сообщения из канала получателю.
\item Переход $s_4\ra{0,7}s_5$ заключается в~восстановлении исходного сообщения из
искаженного и~передаче восстановленного сообщения получа\-телю.
\item Переходы $s_3\ra{1}s_0$ и~$s_5\ra{1}s_0$ заключаются в~получении
  сообщения получателем и~посылке им отправителю
  уведомления о~том, что получение сообщения   было выполнено успешно.
\end{enumerate}

Одно из свойств   протокола, представленного моделью рис.~1,
   заключается в~том, что каждое сообщение, полученное отправителем
от внешнего источника, будет с вероятностью    $\hm\geq 0{,}9$  доставлено получателю
      не более чем через 5~единиц времени.
   Для формального представления этого свойства будем полагать, что
   множество~AP атомарных утверждений состоит из одной переменной~$p$ и~эта
   переменная принимает в~состоянии~$s_0$ (см.\ рис.~1)
   значение~1, а~в~остальных состояниях~--- значение~0.
Таким образом, множество $2^{\mathrm{AP}}$ состоит из двух элементов:
$\emptyset$ и~$\{p\}$. Будем обозначать эти элементы символами~0 и~1 соответственно.

         Формула логики PCTL, соответствующая  указанному выше свойству,
   имеет следующий вид:
\be{sdfsadfsa44}
{\bf G}((\neg p) \to {\cal P}_{\geq 0.9}({\bf F}^{\leq 5}p))\,,
\ee
где символ~$\geq$ обозначает функцию вида
$$
\geq: [0,1]\times [0,1]\to\{0,1\}\,,
$$
которая сопоставляет паре $(a,b)\hm\in [0,1]\times [0,1]$
элемент~1, если $a\hm\geq b$, и~0  иначе.

Анализируемая ВСП получается из графа, представленного на
рис.~1 приписыванием
к~каждой его вершине~$s$ метки $L(s)$, которая равна~1, если $s\hm=s_0$, и~0 иначе.
Для вычисления значения  формулы~\re{sdfsadfsa44}
в~состояниях этой ВСП можно использовать описанный выше метод редукции.

Матрицу~$P$, соответствующую данной ВСП, представим в~виде следующей таблицы:
$$
    \begin{tabular}{|c||c|c|c|c|c|c|}
  \hline
  & $s_0$ & $s_1$ & $s_2$ & $s_3$ & $s_4$ &
    $s_5$   \\
  \hline
  \hline $s_0$ & 0& $1$& 0& $0$& $0$& $0$\\
  \hline $s_1$ & 0& 0& $0,8$& 0& $0,2$& 0\\
  \hline $s_2$ & 0& 0,3& 0& 0,7& 0& 0 \\
  \hline $s_3$ & 1& 0& $0$& 0& 0& $0$\\
  \hline $s_4$ & $0$& $0,3$& 0& 0,7& 0& 0\\
  \hline $s_5$ & $1$& 0& 0& $0$& 0& 0 \\
  \hline
  \end{tabular}
$$



Вычисление эквивалентности~$\sim$ для анализируемой ВСП происходит следующим образом.
\begin{enumerate}[1.]
\item Вычисляется отношение эквивалентности~$\rho^0$,
которое состоит из  всех пар $(s_1, s_2)\hm\in S\times S$,
удовлетворяющих равенству $L(s_1) \hm=L(s_2)$.

По предположению значение~$p$ в~$s_0$  равно~1 и~в~каждом $s\hm\in S$
(где $S$~--- множество состояний анализируемой ВСП), таком что $s\hm\neq s_0$,
значение~$p$ равно~0, т.\,е.\ $L(s_0)\hm=1$ и~$\forall\,s\hm\in
S\setminus \{s_0\}$ $L(s)=0$. Cледовательно, $\Sigma^0$ состоит
из двух классов:

\noindent
\be{dfgfdsgdfgfd5r5}
\{s_0\}, \quad \{s_1, s_2, s_3, s_4, s_5\}\,.
\ee

\item Матрица $\pi^0$, соответствующая детерминированной
СФ $\pi^0: S\hm\to \Sigma^0$, имеет вид:
$$
  \begin{tabular}{|c||c|c|}
%  \hline   & $\,$ & $\,$  \\
 % \hline
 \hline
   $s_0$ & 1&0\\
  \hline $s_1$ &  0&1\\
   \hline $s_2$ & 0&1 \\
   \hline $s_3$ & 0&1\\
   \hline $s_4$ & 0&1 \\
   \hline $s_5$ & 0&1 \\
  \hline
  \end{tabular}
$$

Затем вычисляется матрица $P\cdot \pi^0$. Данная мат\-ри\-ца будет иметь сле\-ду\-ющий вид:
\be{sdfsdafsd3333}
  \begin{tabular}{|c||c|c|}
%  \hline   & $\,$& $\,$   \\
 % \hline
 \hline
   $s_0$ & 0&1\\
  \hline $s_1$ &  0&1\\
   \hline $s_2$ & 0&1 \\
   \hline $s_3$ & 1&0\\
   \hline $s_4$ & 0&1 \\
   \hline $s_5$ & 1& 0\\
  \hline
  \end{tabular}
\ee


По матрице~\re{sdfsdafsd3333} нетрудно вычислить отношение~$\rho^1$
и соответствующее ему разбиение~$\Sigma^1$.
Из определения отношения~$\rho^1$ непосредственно следует, что
состояния~$s$ и~$s'$ находятся в~одном и~том
же классе разбиения~$\Sigma^1$ тогда и~только тогда, когда они оба
находятся в~одном и~том же классе из списка~\re{dfgfdsgdfgfd5r5}
и,~кроме того, строки матрицы~\re{sdfsdafsd3333},
соответствующие состояниям~$s$ и~$s'$, совпадают.

Разбиение~$\Sigma^1$ будет состоять  из трех классов
(измельчится второй класс в~\re{dfgfdsgdfgfd5r5}, а~первый класс
останется тем же), эти классы имеют следующий вид:
\be{dfgfdsgdfgfd5r51331}
\{s_0\}, \quad \{s_1, s_2, s_4\}, \quad
\{s_3, s_5\}.
\ee

\item Затем вычисляется матрица $\pi^1$, соответст\-ву\-ющая
детерминированной СФ $\pi^1: S \vo \Sigma^1$. Она выглядит следующим
образом:
$$
  \begin{tabular}{|c||c|c|c|}
  %\hline   & $\,$  & $\,$ &$\,$\\
%  \hline
   \hline
   $s_0$ & 1& 0 & 0\\
   \hline $s_1$ & 0 & 1 & 0 \\
   \hline $s_2$ & 0 &1 & 0\\
   \hline $s_3$ & 0 &0 & 1\\
   \hline $s_4$ & 0 &1 & 0\\
   \hline $s_5$ & 0 &0 & 1\\
  \hline
  \end{tabular}
$$

Произведение $P\cdot \pi_1$ выглядит так:
$$
  \begin{tabular}{|c||c|c|c|}
 % \hline   & $\,$  & $\,$ &$\,$\\
%  \hline
   \hline $s_0$ & 0& 1 & 0\\
   \hline $s_1$ & 0 & 1 & 0 \\
   \hline $s_2$ & 0 &$0,3$ & $0,7$\\
   \hline $s_3$ & 1 &0 & 0\\
   \hline $s_4$ & 0 &$0,3$ & $0,7$\\
   \hline $s_5$ & 1 &0 & 0\\
  \hline
  \end{tabular}
$$

После этого, действуя так же, как и~в~предыдущем пункте, вычисляем
классы раз\-би\-ения~$\Sigma^2$, соответствующего эквивалентности~$\rho^2$.
Таких классов будет четыре (измельчится второй  класс
в~\re{dfgfdsgdfgfd5r51331}, а~первый и~третий классы останутся теми же),
эти классы имеют сле\-ду\-ющий вид:
\be{dfgfds4433dfgfd5r51331}
\{s_0\}, \quad \{s_1\}, \quad \{s_2, s_4\},\quad
\{s_3, s_5\}.
\ee


\item
Затем вычисляется матрица~$\pi^2$,
соответ\-ст\-ву\-ющая детерминированной СФ $\pi^2: S \vo \Sigma^2$:
$$
\begin{tabular}{|c||c|c|c|c|}
%  \hline   & $\,$  & $\,$ &$\,$&$\,$\\
%  \hline
   \hline $s_0$ & 1& 0 & 0& 0\\
   \hline $s_1$ & 0 & 1 & 0 & 0\\
   \hline $s_2$ & 0 &0 & 1& 0\\
   \hline $s_3$ & 0 &0 & 0& 1\\
   \hline $s_4$ & 0 &0 & 1& 0\\
   \hline $s_5$ & 0 &0 & 0& 1\\
  \hline
  \end{tabular}
$$

Произведение $P\cdot \pi_2$ имеет следующий вид:

\begin{equation*} %{sdfsdafsd333333323344}
  \begin{tabular}{|c||c|c|c|c|}
%  \hline   & $\,$  & $\,$ &$\,$&$\,$\\
%  \hline
   \hline $s_0$ & 0 & 1 & 0& 0 \\
   \hline $s_1$ & 0 &  0& 1 & 0\\
   \hline $s_2$ & 0 &0,3 & 0& 0,7\\
   \hline $s_3$ & 1 &0 & 0&  0\\
   \hline $s_4$ & 0 &0,3 & 0& 0,7\\
   \hline $s_5$ & 1 &0 & 0& 0\\
  \hline
  \end{tabular}
\end{equation*}

Далее, действуя так же, как и~в~предыдущем пункте, вычисляем классы
разбиения~$\Sigma^3$, соответствующего эквивалентности~$\rho^3$.
Нетрудно проверить, что классы разбиения~$\Sigma^3$ будут иметь
точно такой же вид, что и~классы  эквивалентности разбиения~$\Sigma^2$.
Это означает, что искомое разбиение множества~$S$  на
классы эквивалентных состояний построено, оно имеет вид~\re{dfgfds4433dfgfd5r51331}.
\end{enumerate}

Теперь можно приступить к~удалению избыточных состояний (так, чтобы
среди оставшихся состояний было ровно по одному состоянию
из каждого класса эквивалентности~\re{dfgfds4433dfgfd5r51331}).
Нетрудно видеть, что можно удалить состояния~$s_4$ и~$s_5$.
После удаления данных состояний граф, представленный
на рис.~1 примет вид, представленный на рис.~2.

\begin{center}  %fig2
\vspace*{2pt}
\mbox{%
 \epsfxsize=40.06mm
 \epsfbox{mir-2.eps}
 }
  \vspace*{2pt}

{{\figurename~2}\ \ \small{Модифицированный граф рис.~1}}
  \end{center}

%\vspace*{3pt}


\addtocounter{figure}{1}




Таким образом, задача вычисления
значений формулы~\re{sdfsadfsa44} в~состояниях ВСП (см.\ рис.~1)
сводится к задаче вычисления значений формулы~\re{sdfsadfsa44} в~состояниях
ВСП~(см.\ рис.~2), что требует выполнения
меньшего числа операций, чем задача вычисления
значений формулы~\re{sdfsadfsa44} в~состояниях исходной ВСП.

\vspace*{-9pt}


\section{Заключение}

В настоящей работе изложен алгоритм редукции
ВСП, идея которого
заключается в~удалении избыточных состояний.
Результатом применения этого алгоритма является ВСП,
чис\-ло со\-сто\-яний которой не превосходит
чис\-ла со\-сто\-яний исходной ВСП и~все свойства которой,
выражаемые формулами логики PCTL, совпадают со свойствами исходной ВСП.
Идея алгоритма заключается\linebreak в~по\-стро\-ении последовательности вложенных
разбиений мно\-жества состояний исходной ВСП. Алгоритм по\-стро\-ения
последовательности разбиений мно\-жества состояний заканчивает свою работу,\linebreak
когда эта последовательность стабилизируется. Редукция ВСП выполняется
методом удаления эквивалентных состояний и~переопределения вероятностей перехода.
На примере показана возможность применения предложенного алгоритма
к~задаче вероятностной верификации протокола передачи сообщений
через ненадежный канал связи, в~котором пересылаемые сообщения могут пропадать
или искажаться, с~возможной коррекцией искажения.
Отметим, что в~результате такой редукции может получиться ВСП,
которая, хотя и~не содержит различных эквивалентных состояний, тем не
менее, может не являться минимальной по числу состояний среди всех ВСП,
эквивалентных исходной ВСП.
В связи с~этим встает вопрос об
алгоритме нахождения минимальной по числу состояний ВСП,
эквивалентной заданной ВСП, и~исследовании
единственности такой минимальной ВСП (с~точ\-ностью до подходящим образом
сформулированного понятия изоморфизма).
Так\-же пред\-став\-ля\-ет интерес исследование проб\-лем минимизации других классов
моделей, связанных с~вероятностной верификацией, в~част\-ности минимизации
марковских решающих процессов.

\vspace*{-9pt}

{\small\frenchspacing
 {%\baselineskip=10.8pt
 \addcontentsline{toc}{section}{References}
 \begin{thebibliography}{99}


\bibitem{karpov} %1
\Au{Карпов Ю.\,Г.} Model checking.
Верификация параллельных и~распределенных программных систем.~---
СПб.: БХВ-Петербург, 2010. 560~с.

\bibitem{markov} %2
\Au{Кемени Дж., Снелл Дж.} Конечные цепи Маркова.~--- М.: Наука, 1970. 225~с.

\bibitem{buh} %3
\Au{Бухараев Р.\,Г.} Основы теории вероятностных автоматов.~--- М.: Наука, 1985. 288~с.


\bibitem{35} %4
\Au{Hansson H., Jonsson B.} A~logic for reasoning about time and
reliability~// Formal Aspects Computing, 1994. Vol.~6. No.\,5. P.~512--535.

\bibitem{tut} %5
\Au{Kwiatkowska M., Parker D.} Advances in probabilistic model
checking~// NATO Science for Peace and Security Series. Information
and Communication Security, 2012. Vol.~33. P.~126--151.

\bibitem{peled} %6
\Au{Кларк Э.\,М., Грамберг О., Пелед~Д.} Верификация моделей
программ. Model Checking.~--- М.: МЦНМО, 2002. 416~с.

\bibitem{mf} %7
\Au{Миронов А.\,М., Френкель С.\,Л.} Минимизация вероятностных
моделей программ~// Фундаментальная и~прикладная математика, 2014.
Т.~19. Вып.~1. С.~121--163.

\bibitem{6} %8
\Au{Baier C., Groesser M., Ciesinski~F.} Partial order reduction
for probabilistic systems~// 1st Conference (International)
on Quantitative Evaluation of Systems (QEST'04) Proceedings.~--- IEEE
Computer Society Press, 2004. P.~230--239.

\bibitem{23} %9
\Au{D'Argenio P., Niebert P.} Partial order reduction on
concurrent probabilistic programs~//  1st
Conference (International) on Quantitative Evaluation of Systems
(QEST'04) Proceedings.~--- IEEE Computer Society Press, 2004. P.~240--249.

\bibitem{52} %10
\Au{Kwiatkowska M., Norman G., Parker~D.} Symmetry reduction for
probabilistic model checking~// Computer aided verification~/
Eds. T.~Ball, R.\,B.~Jones.
Lecture notes in computer science ser.~--- Springer, 2006.  Vol.~4144. P.~234--248.

\bibitem{26} %11
\Au{Donaldson A., Miller A.} Symmetry reduction for probabilistic
model checking using generic representatives~//
Automated technology for verification
and analysis~/ Eds.\ S.~Graf, W.~Zhang. Lecture notes in computer science ser.~--- Springer, 2006.
Vol.~4218. P.~9--23.



\bibitem{36} %12
\Au{Hart S., Sharir M., Pnueli~A.} Termination of probabilistic
concurrent programs // ACM Trans. Programming Languages
Syst., 1983. Vol.~5. No.\,3. P.~356--380.

\bibitem{63}
\Au{Vardi M.} Automatic verification of probabilistic concurrent
finite state programs~// 26th Annual Symposium on
Foundations of Computer Science (FOCS'85) Proceedings.~--- IEEE Computer Society
Press, 1985. P.~327--338.

\bibitem{21}
\Au{Courcoubetis C., Yannakakis~M.} Verifying temporal properties
of finite state probabilistic programs~// 29th
Annual Symposium on Foundations of Computer Science (FOCS'88) Proceedings.~---
IEEE Computer Society Press, 1988. P.~338--345.


\bibitem{14} %15
\Au{Bianco A., de Alfaro L.} Model checking of probabilistic and
nondeterministic systems~// Foundations of software technology and
theoretical computer science~/
Ed. P.\,S.~Triagarejan.  Lecture notes in computer science ser.~---
Springer, 1995.  Vol.~1026. P.~499--513.

\bibitem{8}
\Au{Baier C., Haverkort B., Hermanns~H., Katoen~J.-P.}
Model-checking algorithms for continuous-time Markov chains~// IEEE
Trans. Software Eng., 2003. Vol.~29. No.\,6.
P.~524--541.

\bibitem{34} %17
\Au{Hansson H.} Time and probability in formal design of
distributed systems.~--- Elsevier, 1994. 304~p.

\bibitem{5} %18
\Au{Baier C., Clarke E., Hartonas-Garmhausen~V., Kwiatkowska~M.,
Ryan~M.} Symbolic model checking for probabilistic processes~//
Automata, languages and programming~/ Eds. P.~Degano, R.~Gorrieri, A.~Marinetti-Spaccamela.
Lecture notes in computer science ser.~--- Springer,
1997.  Vol.~1256. P.~430--440.

\bibitem{40} %19
\Au{Hermanns H., Katoen~J.-P., Meyer-Kayser~J., Siegle~M.}
A~Markov chain model checker~// Tools and algorithms for the construction and
analysis of systems~/
Eds. S.~Graf, M.\,I.~Schwartzbach.
Lecture notes in computer science ser.~--- Springer, 2000.
 Vol.~1785. P.~347--362.

\bibitem{24} %20
\Au{De Alfaro L., Kwiatkowska~M., Norman~G., Parker~D., Segala~R.}
Symbolic model checking of probabilistic processes using MTBDDs and
the Kronecker representation~//
Tools and algorithms for the construction and analysis
of systems~/
Eds. S.~Graf, M.\,I.~Schwartzbach. Lecture notes in computer science ser.~--- Springer, 2000.
 Vol.~1785. P.~395--410.



\bibitem{50} %21
\Au{Kwiatkowska M., Norman~G., Parker~D.} Probabilistic model
checking in practice: Case studies with PRISM~// ACM SIGMETRICS
Performance Evaluation Review, 2005. Vol.~32. No.\,4. P.~16--21.

\bibitem{55}
\Au{Kwiatkowska M., Norman~G., Parker~D.} PRISM 4.0: Verification
of probabilistic real-time systems~//
 Computer aided verification~/
 Eds. G.~Gopalakrishnan, S.~Qadeer.
Lecture notes in computer science ser.~--- Springer, 2011. Vol.~6806. P.~585--591.

 \end{thebibliography}

 }
 }

\end{multicols}

\vspace*{-12pt}

\hfill{\small\textit{Поступила в редакцию 05.11.14}}

%\newpage

\vspace*{8pt}

\hrule

\vspace*{2pt}

\hrule

%\vspace*{12pt}

\def\tit{A METHOD OF ENHANCING PROBABILISTIC VERIFICATION EFFICIENCY FOR COMPUTER AND TELECOMMUNICATION SYSTEMS\\[-7pt]}

\def\titkol{A method of enhancing probabilistic verification
efficiency for computer and telecommunication systems}

\def\aut{A.\,M.~Mironov$^1$ and~S.\,L.~Frenkel$^{1,2}$}

\def\autkol{A.\,M.~Mironov and~S.\,L.~Frenkel}

\titel{\tit}{\aut}{\autkol}{\titkol}

\vspace*{-10pt}

\noindent
$^1$Institute of Informatics Problems, Russian Academy of Sciences,
44-2~Vavilov Str., Moscow 119333, Russian\linebreak
$\hphantom{^1}$Federation

\noindent
$^2$Moscow Institute of Radio, Electronics, and Automation
(MIREA), 78\ Prosp. Vernadskogo, Moscow 119454,\linebreak
$\hphantom{^1}$Russian Federation

\def\leftfootline{\small{\textbf{\thepage}
\hfill INFORMATIKA I EE PRIMENENIYA~--- INFORMATICS AND
APPLICATIONS\ \ \ 2014\ \ \ volume~8\ \ \ issue\ 4}
}%
 \def\rightfootline{\small{INFORMATIKA I EE PRIMENENIYA~---
INFORMATICS AND APPLICATIONS\ \ \ 2014\ \ \ volume~8\ \ \ issue\ 4
\hfill \textbf{\thepage}}}

\vspace*{2pt}

\Abste{The paper considers the problem of reduction of probabilistic transition
systems (PTS) in order to reduce the complexity of model checking of such systems.
The problem of model checking of a~PTS is to calculate truth
values
of formulas of temporal probabilistic computational tree logic (PCTL)
in the initial state of the PTS. The\linebreak\vspace*{-12pt}}

\Abstend{paper introduces the concept of equivalence
of states of a~PTS and represents an algorithm for removing equivalent states.
The result of this algorithm is a~PTS such that all its properties expressed
by formulas of PCTL coincide with those of the original PTS.}

\KWE{verification; model checking;
probabilistic transition systems; probabilistic
temporal logic; reduction of probabilistic models}




  \DOI{10.14357/19922264140408}

  \vspace*{-14pt}


\Ack

\vspace*{-3pt}

\noindent
The research was partially supported by the Russian Foundation for Basic Research
 (project 12-07-00109).



\vspace*{-6pt}

  \begin{multicols}{2}

\renewcommand{\bibname}{\protect\rmfamily References}
%\renewcommand{\bibname}{\large\protect\rm References}



{\small\frenchspacing
 {%\baselineskip=10.8pt
 \addcontentsline{toc}{section}{References}
 \begin{thebibliography}{99}

 \vspace*{-2pt}

 \bibitem{karpov-1}
\Aue{Karpov, Yu.\,G.} 2010.
\textit{Model checking. Verification of parallel and distributed systems.}
St.\ Petersburg.: BHV-Peterburg. 560~p.


\bibitem{markov-1}
\Aue{Kemeny, J.\,G., and J.\,L.~Snell.} 1976.
\textit{Finite Markov chains.}
New York\,--\,Berlin\,--\,Heidelberg\,--\,Tokyo: Springer-Verlag.
225~p.

\bibitem{buh-1}
\Aue{Bukharaev, R.\,G.} 1985.
\textit{Foundations of probabilistic automata theory.}
Moscow: Nauka. 288~p.



\bibitem{35-1} %4
\Aue{Hansson, H., and B.~Jonsson}.
1994. A~logic for reasoning about time and reliability.
\textit{Formal Aspects Computing} 6(5):512--535.

\bibitem{tut-1} %5
\Aue{Kwiatkowska, M., and D.~Parker}. 2012.
Advances in probabilistic model checking.
\textit{NATO Science for Peace and Security Series,
Information and Communication Security} 33:126--151.

\bibitem{peled-1}
\Aue{Clarke, E.\,M., O. Grumberg, and D.~Peled.}
1999. \textit{Model checking.} MIT Press. 314~p.

\bibitem{mf-1}
\Aue{Mironov, A.\,M., and S.\,L.~Frenkel}.
2014. Minimization of probabilistic models of programs.
\textit{Fundamental Applied Mathematics} 19(1):121--163.

\bibitem{6-1} %8
\Aue{Baier, C., M. Groesser, and F.~Ciesinski.}
2004.
Partial order reduction for probabilistic systems.
\textit{1st Conference (International) on Quantitative
Evaluation of Systems (QEST'04) Proceedings}.
IEEE Computer Society Press. 230--239.

\bibitem{23-1}
\Aue{D'Argenio, P., and P. Niebert.}
2004.
Partial order reduction on
concurrent probabilistic programs.
\textit{1st Conference (International) on Quantitative
Evaluation of Systems (QEST'04) Proceedings}.
IEEE Computer Society Press. 240--249.

\bibitem{52-1} %10
\Aue{Kwiatkowska, M., G. Norman, and D.~Parker.}
2006.
Symmetry reduction for probabilistic model checking.
\textit{Computer aided verification}. Eds. T.~Ball and R.\,B.~Jones.
Lecture notes in computer science ser.
4144:234--248.

\bibitem{26-1} %11
\Aue{Donaldson, A., and A. Miller.}
2006. Symmetry reduction for
probabilistic model checking using
generic representatives.
\textit{Automated technology for verification
and analysis}.
Eds.\ S.~Graf and W.~Zhang. Lecture notes in computer science ser.
4218:9--23.



\bibitem{36-1} %12
\Aue{Hart, S., M. Sharir, and A.~Pnueli.}
1983. Termination of probabilistic concurrent programs.
\textit{ACM Trans. Programming Languages Syst.}
5(3):356--380.

\bibitem{63-1} %13
\Aue{Vardi, M.} 1985.
Automatic verification of probabilistic concurrent
finite state programs.
\textit{26th Annual Symposium on Foundations of Computer Science (FOCS'85) Proceedings.}
IEEE Computer Society Press. 327--338.

\bibitem{21-1} %14
\Aue{Courcoubetis, C., and M. Yannakakis.}
1988. Verifying temporal properties
of finite state probabilistic programs.
\textit{29th Annual Symposium on Foundations of Computer Science (FOCS'88)
Proceedings.}
IEEE Computer Society Press. 338--345.

\bibitem{14-1} %15
\Aue{Bianco, A., and L. de Alfaro.}
1995. Model checking of probabilistic and
nondeterministic systems.
\textit{Foundations of software technology and theoretical computer science}.
Ed. P.\,S.~Triagarejan. Lecture notes in computer science ser.
1026:499--513.

\bibitem{8-1} %16
\Aue{Baier, C., B. Haverkort, H.~Hermanns, and J.-P.~Katoen.}
2003.
Model-checking algorithms for continuous-time Markov chains.
\textit{IEEE Trans. Software Engineering} 29(6):524--541.

\bibitem{34-1} %17
\Aue{Hansson, H.} 1994.
\textit{Time and probability in formal design of distributed
systems.} Elsevier. 304~p.

\bibitem{5-1} %18
\Aue{Baier, C., E. Clarke, V.~Hartonas-Garmhausen, M.~Kwiatkowska, and M.~Ryan.}
1997. Symbolic model checking for probabilistic processes.
\textit{Automata,
languages and programming.}
Eds. P.~Degano,  R.~Gorrieri, and A.~Marinetti-Spaccamela.
Lecture notes in computer science ser.
1256:430--440.

\bibitem{40-1} %19
\Aue{Hermanns, H., J.-P.~Katoen, J.~Meyer-Kayser, and M.~Siegle.}
2000. A~Markov chain model checker.
\textit{Tools and algorithms for the construction and analysis
of systems.} Eds. S.~Graf and M.\,I.~Schwartzbach.
Lecture notes in computer science ser.
1785:347--362.

\bibitem{24-1} %20
\Aue{De Alfaro, L., M. Kwiatkowska, G.~Norman, D.~Parker, and R.~Segala}.
2000.
Symbolic model
checking of probabilistic processes using MTBDDs and the Kronecker representation.
\textit{Tools and algorithms for the construction and analysis
of systems.} Eds. S.~Graf and M.\,I.~Schwartzbach.
Lecture notes in computer science ser.
1785:395--410.


\bibitem{50-1} %21
\Aue{Kwiatkowska, M., G.~Norman, and D.~Parker.}
2005.
Probabilistic model checking in practice:
Case studies with PRISM.
\textit{ACM SIGMETRICS Performance Evaluation Review.}
32(4):16--21.

\bibitem{55-1} %23
\Aue{Kwiatkowska, M., G. Norman, and D.~Parker.}
2011.
PRISM 4.0: Verification of probabilistic real-time systems.
\textit{Computer aided verification.}  Eds. G.~Gopalakrishnan and S.~Qadeer.
Lecture notes in computer science ser.
6806:585--591.


\end{thebibliography}

 }
 }

\end{multicols}

\vspace*{-12pt}

\hfill{\small\textit{Received November 5, 2014}}

\pagebreak

%\vspace*{-18pt}


\Contr


\noindent
\textbf{Mironov Andrew M.} (b.\ 1966)~---
Candidate of Science (PhD) in physics and mathematics,
senior scientist, Institute of Informatics Problems,
Russian Academy of Sciences, 44-2 Vavilov Str., Moscow 119333,
Russian Federation; amironov66@gmail.com

\vspace*{6pt}

\noindent
\textbf{Frenkel Sergey L.} (b.\ 1951)~---
Candidate of Science (PhD) in technology, senior scientist,
Institute of Informatics Problems, Russian Academy of Sciences,
44-2 Vavilov Str., Moscow 119333, Russian Federation;
associate professor, Moscow Institute of Radio, Electronics, and Automation
(MIREA), 78 Prosp.\ Vernadskogo, Moscow 119454, Russian Federation; fsergei@mail.ru

\label{end\stat}

\renewcommand{\bibname}{\protect\rm Литература}
 %8
\def\stat{kozerenko}

\def\tit{КОГНИТИВНО-ЛИНГВИСТИЧЕСКИЕ ПРЕДСТАВЛЕНИЯ 
В~СИСТЕМАХ ОБРАБОТКИ ТЕКСТОВ}

\def\titkol{Когнитивно-лингвистические представления 
в~системах обработки текстов}

\def\autkol{Е.\,Б.~Козеренко, И.\,П.~Кузнецов}
\def\aut{Е.\,Б.~Козеренко$^1$, И.\,П.~Кузнецов$^2$}

\titel{\tit}{\aut}{\autkol}{\titkol}

%{\renewcommand{\thefootnote}{\fnsymbol{footnote}}\footnotetext[1]
%{Работа выполнена при поддержке Российского фонда фундаментальных
%исследований, проект~10-01-00480. Статья написана на основе материалов доклада, 
%представленного на IV Международном семинаре <<Прикладные задачи теории вероятностей 
%и математической статистики, связанные с моделированием информационных систем>> 
%(зимняя сессия, Аоста, Италия, январь--февраль 2010 г.).}}

\renewcommand{\thefootnote}{\arabic{footnote}}
\footnotetext[1]{Институт проблем информатики Российской академии наук, kozerenko@mail.ru}
\footnotetext[2]{Институт проблем информатики Российской академии наук, igor-kuz@mtu-net.ru}


\Abst{Рассмотрены вопросы проектирования и развития 
семантико-синтаксических и лексико-семантических представлений в 
лингвистических процессорах ряда систем, основанных на аппарате расширенных 
семантических сетей (РСС). Системы этого класса создаются для извлечения знаний из 
текстов на естественных языках, отображения извлеченных сущностей и связей в 
структуры базы знаний (БЗ) и использования знаний для поддержки экспертных 
аналитических решений в различных сферах приложения. В~фокусе внимания 
находятся ин\-же\-нер\-но-линг\-ви\-сти\-че\-ские представления, позволяющие 
построить целостную работающую лингвистическую модель, которая 
модифицируется в зависимости от конкретной задачи: от <<тяжелой>> формы на 
основе детальных глубинных представлений до фокусных редуцированных 
оболочек, настроенных на узкую предметную область (ПО) и ограниченный язык 
общения. Особое внимание уделяется способам описания 
дис\-три\-бу\-тив\-но-транс\-фор\-ма\-ци\-он\-ных признаков языковых объектов.}

\KW{интеллектуальные системы; семантические представления; лингвистические 
процессоры; обработка естественного языка; извлечение знаний}

       \vskip 14pt plus 9pt minus 6pt

      \thispagestyle{headings}

      \begin{multicols}{2}

      \label{st\stat}

\section{Введение}

     Данная работа посвящена проблемам создания\linebreak 
     когни\-тив\-но-линг\-ви\-сти\-че\-ских моделей естественного языка для 
различных классов информационных систем и описанию опыта создания 
линг\-ви\-сти\-че\-ских представлений для интеллектуальных\linebreak технологий 
обработки текстов. Вопросы извлечения знаний из текстов и создания модели 
естественного языка рассматриваются в единстве. В центре внимания будут 
находиться лингвистические процессоры интеллектуальных систем, 
разработанных на основе аппарата \textit{расширенных семантических 
сетей}~[1--5]. %\cite{1koz}--\cite{3koz}, \cite{18koz}--\cite{19koz}. 
Будем 
называть их \textit{РСС-сис\-те\-мы}. Эти системы создавались коллективом 
разработчиков, включая авторов данной статьи в Институте проб\-лем 
информатики РАН на протяжении целого ряда лет в рамках 
исследовательских проектов и прикладных систем, ориентированных на 
конкретные ПО заказчиков. Можно выделить четыре 
поколения РСС-систем. Ко\-гни\-тив\-но-линг\-ви\-сти\-че\-ские 
представления, заложенные в основу систем этого класса, прошли 
определенный эволюционный путь. 
     
     Интеллектуальные РСС-сис\-те\-мы содержат развитые \textit{базы 
знаний}, при этом знания представлены в виде записей на языке 
РСС, называемых 
     \textit{РСС-струк\-ту\-ра\-ми}. Лингвистические знания, таким 
образом, являются частным случаем <<знаний>> и также представлены в 
виде записей на языке РСС. Основным 
конструктивным элементом РСС\linebreak является именованный $N$-мест\-ный 
предикат, на\-зы\-ва\-емый <<\textit{фрагментом}>>. Все множество языковых 
объектов задается в виде системы пре\-ди\-кат\-но-ак\-тант\-ных структур, при этом 
поддерживаются механизмы представления вложенных структур, что дает 
очень мощные изобразительные возможности для описания объектов 
различных языковых уровней. Очень важными факторами являются 
однородность и единообразие лингвистических представлений. 
     
     В процессе анализа и синтеза предложений естественного языка 
используется фор\-маль\-но-грам\-ма\-ти\-че\-ский аппарат, сходный с 
грамматиками зависимостей. При этом подходе опорными элементами 
служат слова и конструкции, выполняющие роль предикатов в предложении, 
и результатом анализа предложения должен стать один предикат, 
соответствующий сказуемому рассматриваемого предложения (т.\,е.\ 
основному глаголу в личной форме или другому основному предикатному 
выражению). Таким образом, в процессе анализа происходит выявление 
\textit{когнитивных опор} предложения: <<слов-дейст\-вий>> и 
     <<слов-от\-но\-ше\-ний>>, т.\,е.\ глаголов и других слов, имеющих 
синтактико-семантические валентности. Примером <<слов-от\-но\-ше\-ний>> 
могут служить, например, слова <<отец>>, <<друг>> и~т.\,п., т.\,е.\ в данном 
случае <<отношения>> (или \textit{функции}~--- в терминах языка логики 
предикатов 1-го порядка)~--- это слова, которые задают сильные, четко 
выраженные син\-так\-ти\-ко-се\-ман\-ти\-че\-ские ожидания. 
     
     Семантический анализ в ин\-же\-нер\-но-линг\-ви\-сти\-че\-ском 
понимании~--- это процесс перевода ес\-тест\-вен\-но-язы\-ко\-вых 
выражений во <<внутренние>> структуры БЗ, в 
рассматриваемой ситуации этими <<внутренними>> структурами являются 
записи на языке РСС. Таким образом, структуры БЗ~--- это код смысла в 
интеллектуальных информационных системах подобного рода. 
     
     В работе рассматриваются ин\-же\-нер\-но-линг\-ви\-сти\-че\-ские 
решения в системах с <<пол\-ным>> линг\-ви\-сти\-че\-ским анализом~--- это 
     сис\-те\-мы 1-го и 2-го поколения: ДИЕС1, ДИЕС2, 
     Логос-Д~\cite{2koz, 3koz}~--- и сис\-те\-мах с <<фактографическим>> 
подходом: интеллектуальных системах поддержки аналитических решений 
(ИСПАР)~\cite{18koz, 19koz}, где целью анализа является выделение 
сущностей и связей из текстов,~--- это системы 3-го и 4-го поколения. 

\section{Процесс концептуально-лингвистического моделирования 
в системах, основанных на аппарате расширенных семантических сетей}
     
\subsection{Центральные вопросы семантического моделирования} %2.1
     
     Концептуально-лингвистическое моделирование (КЛМ)~--- это 
процесс построения ес\-тест\-вен\-но-язы\-ко\-вой модели ПО (рис.~1), синтезирующий в себе подходы 
концептуального и лингвистического моделирования~[1--3]. 
По\-стро\-ение концептуально-лингвистической модели некоторой 
ПО подразделяется на следующие этапы:
     \begin{itemize}
     \item построение собственно концептуальной модели, т.\,е.\ вычленение 
базовых понятий, организация их в ро\-до-ви\-до\-вые деревья и определение 
связей между ними;
     \item разработка идеографического словаря ПО, т.\,е.\ 
лексическое наполнение концептуальной модели;
     \item ввод базовых правил, описывающих на естественном языке 
<<модель мира>>, релевантную данной ПО.
     \end{itemize}
     
     
     Методика КЛМ на 
основе аппарата РСС базируется на следующих принципах:
     \begin{itemize}
\item модель должна быть <<открытой>>, т.\,е.\ поддерживать эффективный 
механизм расширения и обновления информации;
\begin{center} %fig1
%\vspace*{3pt}
\hspace*{-10.7158pt}\mbox{%
\epsfxsize=77.871mm
\epsfbox{koz-1.eps}
}\hspace{10.7158pt}
%\end{center}
\vspace*{4pt}
%\begin{center}
{{\figurename~1}\ \ \small{Процесс КЛМ}}
\end{center}
\vspace*{3pt}

%\bigskip
\addtocounter{figure}{1}
\item модель представления <<смысла>> должна учитывать факты 
экстралингвистической реаль\-ности, которые в виде правил и отношений 
составляют некоторую базовую <<модель мира>>, достраиваемую 
конкретными моделями ПО;
\item модель должна быть практичной, т.\,е.\ не перегруженной детальными 
описаниями связей и отношений между понятиями, чтобы обеспечить 
возможность ее реализации, но в то же время отражать всю релевантную 
конкретной задаче информацию.
\end{itemize}

     \begin{figure*} %fig2
%     \begin{center}
\hspace*{23mm}\{(ВЫРАБАТЫВА895\_\_)(DICSEM)\\
\hspace*{23mm}COORD(PROGNOZ1,RUS,ВЫРАБАТЫВА895\_\_,S50\_31\_51\_20,\%)\\
\hspace*{23mm}SUB(UNIV,0+)~SUB(UNIV,1+)~SUB(UNIV,2+)\\
\hspace*{23mm}ВЫРАБАТЫВ(0-,1-,2-/3+)~INFI(3-)~ПРИДЕТСЯ(3-)~ПРИДЕТСЯ(3$-$/4+) \\
\hspace*{23mm}FUT1(4$-$)~SUB(СРЕД,5+)
%\end{center}
%\vspace*{2pt}
\Caption{Пример записи представления глагола <<вырабатывать>> в семантическом 
словаре
\label{f2koz}}
%\vspace*{6pt}
\end{figure*}

     Реалистичный подход к постановке задачи диктует необходимость 
ограничения моделируемого подмножества естественного языка. Суть 
ограничений сводится к следующему:
     \begin{enumerate}[(1)]
     \item анализируемые текстовые материалы содержат 
экспертные знания из конкретных ПО (в разработанных 
авторами системах это были такие ПО, как диагностика 
брака при изготовлении микросхем, социальное прогнозирование, 
криминалистика и другие);
     \item в целях максимально возможного устранения 
неоднозначности словарь строится по модульному принципу: есть некоторая 
наиболее общая часть (1--2~уровня), которая достраивается специальными 
словарями для каж\-дой отдельной~ПО.
     \end{enumerate}
     
     Предлагаемая модель лексической семантики основана на принципе 
<<ядерного>> значения, реализуемого в контексте данной 
ПО, с последующим индуктивным наращиванием других значений (если 
они актуализируются в рас\-смат\-ри\-ва\-емых контекстах). Также используется 
таксономия, которая реализуется в виде иерархических деревьев классов 
слов. 
     
     Общая <<модель мира>> системы является основой для моделей ПО. 
Элементами этой модели служат классы слов, которые подразделяются на 
понятия/имена, отношения, действия, свойства, характеристики действий, 
временные и пространственные характеристики.
     
     Самым общим понятием является \textit{концепт}, или 
\textit{универсальный класс}, который подразделяется на объект, ситуацию, 
процесс и~др. 
     
     Слова, относящиеся к классам действий и отношений, представлены 
как се\-ман\-ти\-ко-син\-так\-си\-че\-ские фреймы, задающие 
     пре\-ди\-кат\-но-ак\-тант\-ные структуры (модель управления). Однако 
в описываемом подходе (назовем его РСС-под\-хо\-дом) существенно 
расширена область значений актантов. Суть расширения состоит, во-первых, 
в том, что в роли актантов могут выступать не только простые объекты, 
соответствующие отдельным словам, но и структурные объекты, 
представляющие словосочетания и фразы, а во-вторых, в том, что понятие 
падежа включает в себя не только семантические, но и синтаксические 
признаки.
     
     Подход, основанный на РСС, позволяет отражать произвольный 
уровень вложенности структур за счет пропозициональных вершин 
семантической сети. Это обеспечивает представление\linebreak сложных 
синтаксических конструкций фраз\linebreak естественного языка, а также позволяет 
отразить\linebreak структурный характер лексической семантики,\linebreak которая в 
предлагаемой модели имеет иерар\-хи\-че\-ски-се\-те\-вую структуру. 
Линг\-ви\-сти\-че\-ские зна-\linebreak ния пред\-став\-ле\-ны в системном словаре и 
декла\-ра\-тивных модулях линг\-ви\-сти\-че\-ско\-го процессора.\linebreak В РСС-сис\-те\-мах 
так\-же реализована функция динамически форми\-ру\-емо\-го семантического 
словаря, который на основе исходной лингвистической информации 
достраивается системой автоматически в процессе об\-ра\-бот\-ки конкретных 
текстов. На рис.~\ref{f2koz} пред\-став\-ле\-но \mbox{такое} <<внутреннее>> описание 
глагола в семантическом словаре. Этот словарь автоматически генерируется 
РСС-системами ДИЕС2, ЛОГОС-Д, ИКС в процессе обработки 
     естест\-вен\-но-язы\-ко\-вых \mbox{текстов}. 
     {\looseness=1
     
     }
     
     
\subsection{Особенности применения аппарата расширенных семантических сетей 
в~когнитивно-лингвистическом моделировании} %2.2
     
     Дадим краткое описание аппарата РСС и  
обос\-ну\-ем выбор именно этого метода представления для моделирования 
естественного языка. Классическое понятие семантической сети сводится к 
следующему: задаются некоторые вершины, соответствующие объектам,  
вершины связываются дугами, которые помечаются именами отношений. 
Однако с помощью подобных сетей оказывается трудно представлять 
сложные виды информации, например, когда объекты, связанные 
отношениями, образуют агрегаты и когда отношения связываются между 
собой отношениями и~др. Поэтому в сети вводятся вершины, 
соответствующие именам отношений, а также специальный композиционный 
элемент, называемый вершиной связи. Вершина связи как бы <<разрывает>> 
дугу и подсоединяется одним ребром к вершине-отношению, а другими 
ребрами~--- к вершинам-объектам. Расширенная семантическая сеть является развитием такого сорта 
сетей в направлении повышения изобразительных возможностей при 
сохранении свойства однородности.
     
     Основой РСС является множество вершин ($V$), из которых 
составляются элементарные фрагменты (ЭФ) вида
     $
     V_0(V_1,V_2,\ldots ,V_k/V_{k+1})
     $, 
     где
$V_0, V_1, V_2,\ldots , V_k, V_{k+1}>0$.
     
     
     Такой фрагмент представляет $k$-местное отношение. Позиции 
вершин в ЭФ определяют их роли. 
Вершина~$V_0$ ставится в соответствие имени отношения, 
вершины~$V_1$, $V_2$, \ldots , $V_k$~--- объектам, участ\-ву\-ющим в 
отношении, а вершина~$V_{k+1}$, отделенная косой линией,~--- всей 
совокупности упомянутых объектов с учетом их отношения. В~дальнейшем 
будем $V_{k+1}$ называть $C$-вершиной ЭФ.\linebreak 
Множество ЭФ образует РСС. 
С~помощью РСС представляются наборы отношений, различные ситуации, 
сце\-нарии. Сильной стороной РСС-под\-хо\-да является возможность 
однородного пред\-став\-ле\-ния как предметной (концептуальной), так и 
лингвистической информации, что обеспечивает эффективную обработку 
знаний и поддержание непротиворечи\-вости~БЗ.
          \begin{figure*} %fig3
     \vspace*{1pt}
\begin{center}
\mbox{%
\epsfxsize=125.039mm
\epsfbox{koz-3.eps}
}
\end{center}
\vspace*{-9pt}
     \Caption{Семантико-синтаксический анализ без выявления глагольных 
словоформ
      \label{f3koz}}
\vspace*{12pt}
 %     \end{figure*}
%            \begin{figure*} %fig4
           \vspace*{1pt}
\begin{center}
\mbox{%
\epsfxsize=103.129mm
\epsfbox{koz-4.eps}
}
\end{center}
\vspace*{-9pt}
      \Caption{Целостная семантическая структура предложения
      \label{f4koz}}
      \end{figure*}

     
     Посредством РСС в БЗ представлены лингвистические  и 
предметные знания. Обработка этих знаний осуществляется 
продукциями языка ДЕКЛ, на котором реализованы сле\-ду\-ющие шесть 
блоков: морфологического анализа, семанти\-ческого анализа слов, 
син\-так\-ти\-ко-се\-ман\-ти\-че\-ско\-го анализа форм, 
прагматических функций, организации системной активности и 
обратный лингвистический процессор. С~помощью продукций 
осущест\-вля\-ет\-ся последовательное преобразование сети~--- РСС. При этом 
проходятся фазы, соответствующие уровню понимания входного текста. 
Рас\-смот\-рим~их.
     \begin{enumerate}[1.]
     \item На первом шаге анализа строится 
пространственная структура предложения с морфологической информацией 
для каждого слова.\linebreak Каж\-дый член предложения представляется вершиной 
семантической сети. Вместо слова генерируется код (если слово 
многозначно, т.\,е.\ принадлежит к нескольким классам,~--- то более одного 
кода). Основой кода служит корень слова. На этом этапе предложение 
представляется в виде набора фрагментов типа LRR (специальных меток 
результатов 1-го этапа анализа), объединяемых в целостную структуру 
посредством вершины связи. Результат 1-го этапа постоянно обращается к 
словарю: <<Что значит данное слово?>>
     \item На втором этапе каждой вершине сопоставляется семантический 
класс и присваивается новый код. За словами (т.\,е.\ конкретными вершинами 
РСС) система видит объекты, действия, свойства, т.\,е.\ строит 
классификации. Производится се\-ман\-ти\-ко-син\-так\-си\-че\-ский анализ 
без выявления глагольных словоформ, при этом предложение представляется 
в виде совокупности фрагментов типа SEM и SEMD~--- специальных меток 
результатов 2-го этапа анализа (рис.~\ref{f3koz}).
     \item На третьем этапе происходит частичное <<сворачивание>> 
синтаксических структур в более компактные (например, свойство объекта и 
сам объект) с присваиванием нового кода и строится фрагмент для объекта, 
обладающего этим свойством.
     \begin{figure*}[b] %fig5
          \vspace*{12pt}
\begin{center}
\mbox{%
\epsfxsize=147.485mm
\epsfbox{koz-5.eps}
}
\end{center}
\vspace*{-9pt}
     \Caption{Глубинная структура предложений
      \label{f5koz}}
      \end{figure*}      
     \item На четвертом этапе выявляются отношения и действия и 
производится анализ непосредственного контекста на соответствие заданным 
семантическим падежам. Система проверяет, подходят ли объекты 
(концепты, понятия) на аргументные места данного действия или отношения. 
При этом отглагольные существительные (<<делатель>>, т.\,е.\ агент 
действия, или <<делание>>~--- процесс~--- анализируются как слова с 
двойной природой: вначале как действия, а затем как объекты). Результатом 
этого этапа является целостная семантическая структура предложения, 
которая представляется фрагментом типа SEMSTR~--- метки результата 4-го 
этапа анализа (рис.~\ref{f4koz}).
     \item На пятом этапе происходит анализ прагматики: установление 
кореференциальных отношений, частичное восстановление эллиптических 
конструкций, система производит дальнейшие действия с построенными 
фрагментами.
     \end{enumerate}

     
Система ДИЕС допускает ввод полисемичных форм глаголов. Для этого следует 
воспользоваться формальной записью лингвистических знаний. 
     В~сис\-те\-мах, основанных на РСС, все функции реализованы на 
единой основе~--- в рамках языков РСС и ДЕКЛ, которые были разработаны 
с ориентацией на задачи обработки естественного языка.

%\vspace*{-6pt}

\section{Представление семантики глаголов, глубинные 
и~поверхностные структуры}
     
     В процессе анализа выявляются семантические вершины предложения: 
происходит выявление <<слов-дей\-ст\-вий>>, т.\,е.\ глаголов, и 
     <<слов-от\-но\-ше\-ний>>. Что же является конструктивной основой\linebreak 
задания семантических представлений предикатных слов и выражений? Как 
убедительно показано в работе~\cite{4koz}, семантика глагола 
определяется его дис\-три\-бу\-тив\-но-транс\-фор\-ма\-ци\-он\-ны\-ми\linebreak 
свойствами. Поэтому смысл предикатных выражений должен кодироваться с 
учетом их дистрибутивных и трансформационных признаков. 
     
     Выдвинутая рядом лингвистов (Хомский, Филлмор) гипотеза о том, что 
все предложения имеют глубинные и поверхностные 
     структуры~[7--10], явилась очень продуктивным 
источником проектных решений при создании первых РСС-сис\-тем и 
развивалась в дальнейшем. 

В~тео\-ре\-ти\-ко-линг\-ви\-сти\-че\-ском 
понимании глубинная структура~--- это абстракция, содержащая все 
элементы, необходимые для образования поверхностных структур 
предложений со сходной семантикой. 

     В~ин\-же\-нер\-но-линг\-ви\-сти\-че\-ском понимании\linebreak глубинная 
структура~--- это запись на языке БЗ, например на языке РСС, 
которая может быть представлена в <<поверхностном>> виде на одном из 
естественных языков в результате конечного числа определенных 
преобразований. Например, предложения

\noindent
\begin{align*}    
(1)\ &\mbox{\textit{The programmer writes the code}}\\
(2)\ &\mbox{\textit{The code is written by the programmer}}
\end{align*}
имеют истоком одну глубинную структуру:

\medskip

\noindent
     \begin{verbatim}
  Programmer <---- write ----> Code
      agent                   object,
\end{verbatim}

\medskip

\noindent
хотя и отличаются своими поверхностными структурами. В~каждом из них 
имеется агент (the programmer), объект (the code) и действие (write).\linebreak Согласно 
концепции \textit{падежной грамматики} Филлмора~\cite{5koz} глубинная 
структура для обоих предложений инвариантна. Эту структуру можно 
представить в виде скобочной записи $V(\mathrm{AGENT}, \mathrm{OBJECT})$. В~графическом 
виде глубинная структура предложения также может быть представлена 
диаграммой в виде дерева, где отражены инвариантные отношения 
зависимости между предикатной вершиной и актантами (рис.~\ref{f5koz}), 
причем в таком представлении явным образом разграничиваются 
\textit{модальность} (MOD) и \textit{пропозиция} (PROP).
     

     В исходном варианте~\cite{5koz} теория признавала шесть падежей: 
агентив, инструменталис, датив, объектив, локатив и фактитив. По мере 
развития теории~\cite{8koz} происходило увеличение числа падежей, однако 
<<умножение>> количества падежей утяжеляет первоначальную 
конфигурацию, поэтому при построении инженерных семантических 
представлений требуется некоторый <<компромиссный>> вариант, 
сочетающий в себе необходимую полноту, с одной стороны, и простоту и 
гибкость, с другой.

\begin{figure*}[b] %fig6
\vspace*{24pt}
\begin{center}
\mbox{%
\epsfxsize=156.873mm
\epsfbox{koz-6.eps}
}
\end{center}
%\vspace*{-9pt}
\Caption{Обобщенное функциональное представление систем ИСПАР
\label{f6koz}}
\end{figure*}
     
%\vspace*{-6pt}

\section{Некоторые базовые аспекты построения многоязычных 
систем}
     
     Одним из приоритетных направлений развития РСС-сис\-тем является 
обеспечение обработки текстов на нескольких языках, прежде всего для 
рус\-ско-анг\-лий\-ской языковой пары. В системах 2-го поколения~--- ДИЕС2, 
ИКС, ЛОГОС-Д были реализованы лингвистические процессоры и словари 
для русского и английского языка, позволявшие обрабатывать тексты для 
ряда ПО. При этом поддерживался как режим ввода 
лингвистических знаний линг\-вис\-том-ана\-ли\-ти\-ком, так и 
автоматический режим самообучения системы по вводимым \mbox{текстам}. 
{\looseness=1

}

Проводились также эксперименты с итальянским и французским языком. 
При создании многоязычных систем авторы обращались к европейским 
языкам. Очевидно, что европейские языки обладают большим числом общих 
правил, чем любой из них с языками других групп. Но при этом все 
естественные языки обладают общей структурой на самом глубинном 
уровне. На этом уровне располагаются главные элементы естественного 
языка: \textit{предложение}, \textit{модальность}, \textit{пропозиция}.
     
     Моделирование смысловых представлений~--- это процесс, 
развивающийся в направлении от поверхностных семантических структур к 
глубинным. Поиск такого внутреннего представления смысла в условиях 
многоязычной ситуации является на\-прав\-ле\-ни\-ем развития методов 
     КЛМ на базе  РСС. 
     
%     \vspace*{-48pt}
     
\section{Интеллектуальные системы поддержки аналитических 
решений}
     
Системы РСС 3-го и 4-го поколения на\-прав\-ле\-ны на извлечение знаний 
в виде \textit{объектов}, или \textit{сущностей}, и связей между ними из 
пред\-мет\-но-ориен\-ти\-ро\-ван\-ных текстов на русском и английском 
языке~\cite{18koz, 19koz}.

    
В настоящее время во всем мире активно ведутся работы по созданию 
систем извлечения фактов из текстов на естественных языках~[11--14], создаются развитые тезаурусы и 
онтологии~\cite{17koz}. Сис\-те\-мы РСС функционально шире, поскольку 
имеют возможность не только извлекать факты, но и поддерживать 
механизмы логического анализа и экспертного вывода на основе 
извлеченных знаний. Сис\-те\-ма\-ми такого рода являются ИСПАР. В~целом это 
направление исследований требует дальнейшей проработки 
     лек\-си\-ко-се\-ман\-ти\-че\-ских представлений, создания 
     пред\-мет\-но-ориен\-ти\-ро\-ван\-ных семантических словарей. 

Обобщенное функциональное представление систем ИСПАР дано на 
рис.~\ref{f6koz}. 
     
     В рамках ИСПАР на основе РСС 
(\mbox{ИСПАР}--РСС) были реализованы полномасштабные и\linebreak пилотные 
проекты для ряда ПО: криминалистики, управления 
кадрами, мониторинга финансово-экономического кризиса и 
др.~\cite{18koz, 19koz}.

\section{Применение аппарата расширенных семантических сетей в~лингвистических 
исследованиях}
     
     В настоящее время в рамках проектов, на\-прав\-лен\-ных на создание 
открытых лингвистических ресурсов~\cite{20koz} для 
     на\-уч\-но-прак\-ти\-че\-ских целей, ведутся работы по выравниванию 
параллельных текстов научных статей, патентов и 
     фи\-нан\-со\-во-эко\-но\-ми\-че\-ских текстов. В~качестве одного из 
методов выравнивания используется РСС-под\-ход, поскольку он позволяет 
отразить глу\-бин\-но-се\-ман\-ти\-че\-ский уровень языковых структур. 

На  рис.~7 представлен фрагмент первого этапа лингвистического 
анализа в многоязычных системах. Для <<идеальной>> ситуации, когда 
структуры исходного текста и текста перевода практически совпадают, такая 
ситуация имеет место в меньшинстве случаев. Основные трудности 
возникают при наличии переводческих трансформаций в параллельных 
текстах. Особое внимание следует уделять гла\-голь\-но-имен\-ным 
трансформациям, например явлению \textit{номинализации}, поскольку она 
очень продуктивна для всех исследовавшихся языков.

     
     Ключевой задачей при разработке методов сопоставления 
параллельных текстов является выявление и детальное описание тех 
языковых трансформаций, которые имеют место при переводе 
     естест\-вен\-но-язы\-ко\-вых конструкций с одного языка на 
другой~\cite{9koz}, потому что далеко не всегда некое содержание 
передается струк\-тур\-но-по\-доб\-ны\-ми средствами в текстах на разных 
языках. Сравнительное исследование употребления различных частей речи в 
параллельных текстах на разных языках создает основу для выявления и 
описания языковых транс-\linebreak

\begin{center} %fig7
\vspace*{3pt}
\mbox{%
\epsfxsize=79.726mm
\epsfbox{koz-7.eps}
}
\end{center}
\vspace*{4pt}
%\begin{center}
{{\figurename~7}\ \ \small{Первый этап анализа параллельных текстов ($W_n$
обозначает словоформу с номером~$n$, $1\leq n\geq 5$)}}
%\end{center}
%\vspace*{9pt}

%\bigskip
\addtocounter{figure}{1}
      

\noindent 
формаций, при этом центральной трансформацией
является \textit{номинализация}. Явление номинализации
было исследовано в 
ряде работ отечественных и зарубежных лингвистов~[17--20]. 
Ближе всего к правильному, по мнению авторов данной статьи, 
пониманию этого явления следующие определения номинализации: 
<<конструкции\ldots называются номинализованными~--- в том смысле, что 
их естественно рассматривать как результат номинализации конструкций с 
предикативным употреблением глаголов и прилагательных>>; 
<<номинализация~--- это синтаксический процесс, который соотносит 
предложения с именными группами>>~\cite{9koz, 10koz}. Выявление 
номинализованных конструкций в параллельных научных и патентных 
текстах на русском, английском, французском и немецком языках в научных 
и патентных текстах и сопоставительное описание гла\-голь\-но-имен\-ных 
межъязыковых трансформаций~--- одна из центральных задач 
     ин\-же\-нер\-но-линг\-ви\-сти\-че\-ских исследований. 
     
     Следующей базовой трансформацией в исследуемых текстах на 
нескольких европейских языках является адъек\-тив\-но-ад\-вер\-би\-аль\-ное 
преобразование. Это означает, что при переводе с одного языка на другой 
происходит синтаксическое преобразование имен прилагательных в наречия 
и обратное преобразование~--- наречий в прилагательные. Установление 
семантических соответствий между этими языковыми объектами также 
возможно осуществить посредством аппарата~РСС. 
     
     При семантическом выравнивании непараллельных текстов, имеющих 
одну и ту же денотативную составляющую, аппарат РСС позволяет выявить в 
текстах когнитивные опоры (слова с сильной валентностью~--- 
     <<сло\-ва-дейст\-вия>> и <<сло\-ва-от\-но\-ше\-ния>>) и установить 
между ними семантические соответствия.

\section{Заключение}

     В данной работе представлен опыт создания и развития 
     когни\-тив\-но-линг\-ви\-сти\-че\-ских пред\-став\-ле\-ний в 
интеллектуальных информационных сис\-те\-мах, разработанных на основе 
аппарата РСС. Аппарат РСС 
обеспечивает мощные изобразительные возможности для описания всех 
уровней естественного языка, включая уровень 
     глу\-бин\-но-се\-ман\-ти\-че\-ских представлений и межъязыковых 
соответствий. Конкретные лингвистические процессоры, которые были 
созданы на основе этого подхода, прошли определенный путь развития и 
позволили выработать проектные решения для основных задач текущего 
этапа~--- извлечения и обработки содержательных знаний из текстов на 
естественных языках и сопоставления языковых структур в текстах на 
различных языках с учетом базовых трансформаций.
     
     Проблема извлечения и обработки знаний открывает перспективы 
развития интеллектуальных направлений компьютерной лингвистики, 
поскольку ее основной акцент смещен в сторону\linebreak глубинных представлений 
языка, в которых используются как грамматические (морфологические и 
синтаксические), так и семантические атрибуты для описания языковых 
объектов. Проводи-\linebreak мые авторами исследования параллельных текстов 
направлены также на рассмотрение этой проблемы~\cite{20koz}. 
Центральное место в проводящихся линг\-ви\-сти\-че\-ских исследованиях 
занимает изучение и формализация процессов трансформации языковых 
структур, особенно все варианты глагольно-но\-ми\-на\-тив\-ных трансформаций, 
создание развитых дис\-три\-бу\-тив\-но-транс\-фор\-ма\-ци\-он\-ных 
описаний предикатых структур для рассматриваемых языков. 
     
     Для задач извлечения знаний и создания \mbox{ИСПАР} 
     дис\-три\-бу\-тив\-но-транс\-фор\-ма\-ци\-он\-ные описания имеют 
особое значение, поскольку таким образом задаются все возможные способы 
перевода языковых структур в пре\-ди\-кат\-но-ар\-гу\-мент\-ные 
представления, которые затем используются в процедурах обработки знаний.

{\small\frenchspacing
{%\baselineskip=10.8pt
%\addcontentsline{toc}{section}{Литература}
\begin{thebibliography}{99}

     \bibitem{1koz}
     \Au{Кузнецов~И.\,П.}
     Семантические представления.~--- М.: Наука, 1986. 290~с.
     
     \bibitem{2koz}
     \Au{Козеренко~Е.\,Б.}
     Кон\-цеп\-ту\-аль\-но-линг\-вис\-ти\-че\-ское моделирование в среде 
интеллектуального редактора знаний ИКС~// Проблемы проектирования и 
использования баз знаний.~--- Киев: Ин-т кибернетики им.\ В.\,М.~Глушкова, 
1992. C.~73--79.
     
     \bibitem{3koz}
     \Au{Kozerenko~E.\,B.}
     Multilingual processors: A unified approach to semantic and syntactic 
knowledge presentation~// Conference (International ) on Artificial Intelligence 
IC-AI'2001 Proceedings. Las Vegas, Nevada, USA. June 25--28, 2001.~--- Las 
Vegas: CSREA Press, 2001. P.~1277--1282.

     \bibitem{18koz} %4
     \Au{Kuznetsov~I.\,P., Efimov~D.\,A., Kozerenko~E.\,B.}
     Tools for tuning the semantic processor to application areas~// ICAI'09 
Proceedings, WORLDCOMP'09. July 13--16, 2009. Las Vegas, Nevada, USA. 
Vol.~I.~--- Las Vegas: CRSEA Press, 2009. P.~467--472.
     
     \bibitem{19koz} %5
     \Au{Kuznetsov~I.\,P., Kozerenko~E.\,B., Kuznetsov~K.\,I., 
Timonina~N.\,O.}
     Intelligent system for entities extraction (ISEE) from natural language 
texts~// Workshop (International) on Conceptual Structures for Extracting Natural 
Language Semantics (Sense'09) at the 17th Conference 
(International ) on Conceptual Structures (ICCS'09) Proceedings. University Higher School of 
Economics. Moscow, Russia, 2009. P.~17--25.
     
     \bibitem{4koz} %6
     \Au{Апресян~Ю.\,Д.}
     Экспериментальное исследование семантики русского глагола.~--- М.: 
Наука, 1967.  252~с.
     
     \bibitem{5koz} %7
     \Au{Филлмор~Ч.}
     Дело о падеже~// Новое в зарубежной линг\-вистике, 1968. Вып.~X. С.~369--495.
     
     \bibitem{6koz} %8
     \Au{Хомский~Н.}
     Аспекты теории синтаксиса.~--- М.: МГУ, 1972.
     
     \bibitem{7koz} %9
     \Au{Хомский Н.}
     Язык и мышление.~--- М.: МГУ, 1972.
     
     
     \bibitem{8koz} %10
     \Au{Fillmore~C.}
     The case for case reopened~// Syntax and Semantics. Vol.~8.~--- N.Y.: 
Academic Press, 1977. 
     

          \bibitem{15koz} %11
     FASTUS: A cascaded finite-state trasducer for extracting information from 
natural-language text~// AIC, SRI International, Menlo Park, California, 1996. 
     
     \bibitem{16koz} %12
     \Au{Han~J., Pei~Y., Mao~R.}
     Mining frequent patterns without candidate generation: A frequent-pattern 
tree approach~// Data Mining and Knowledge Discovery, 2004. Vol.~8. No.\,1. 
P.~53--87.
     
     
     \bibitem{13koz} %13
     \Au{Cunningham~H.}
     Automatic information extraction~// Encyclopedia of Language and 
Linguistics. 2nd ed.~--- Elsevier, 2005.
     
     \bibitem{14koz} %14
     \Au{Han~J., Kamber~M.}
     Data mining: Concepts and techniques.~--- Morgan Kaufmann, 2006.
     
     
     \bibitem{17koz} %15
     \Au{Добров~Б.\,В., Лукашевич~Н.\,В.}
     Онтологии для автоматической обработки текстов: Описание понятий 
и лексических значений~// Компьютерная лингвистика и интеллектуальные 
технологии: Тр. межд. конф. <<Диалог'06>>. Бекасово, 31~мая\,--\,4~июня 
2006. С.~138--142.

     \bibitem{20koz} %16
     \Au{Kozerenko~E.\,B.}
     INTERTEXT: A multilingual knowledge base for machine translation~// 
Conference (International) on Machine Learning, Models, Technologies and 
Applications Proceedings. June 25--28, 2007. Las Vegas, USA.~--- Las Vegas: 
CSREA Press, 2007. P.~238--243.

     \bibitem{9koz} %17
     \Au{Жолковский~А.\,К., Мельчук~И.\,А.}
     О семантическом синтезе~// Проблемы кибернетики, 1967. Вып.~19.
     
         
     \bibitem{11koz} %18
     \Au{Jacobs~R.\,A., Rosenbaum P.\,S.}
     English transformational grammar.~--- Blaisdell, 1968.
     

\label{end\stat}
     
          \bibitem{12koz} %19
     \Au{Балли~Ш.}
     Общая лингвистика и вопросы французского языка. 2-е изд.~--- М.: 
УРСС, 2001.

\bibitem{10koz} %20
     \Au{Падучева~Е.\,В.}
     О~семантике синтаксиса: Мат-лы к трансформационной 
грамматике русского языка. 2-е изд.~--- М: КомКнига, 2007.  296~с. 
     
 \end{thebibliography}
}
}


\end{multicols} %9
\def\stat{gorshenin}

\def\tit{ЗАШУМЛЕНИЕ ДАННЫХ КОНЕЧНЫМИ СМЕСЯМИ НОРМАЛЬНЫХ 
И~ГАММА-РАСПРЕДЕЛЕНИЙ\\ С~ПРИМЕНЕНИЕМ К~ЗАДАЧЕ ОКРУГЛЕНИЯ НАБЛЮДЕНИЙ$^*$}

\def\titkol{Зашумление данных конечными смесями нормальных 
и~гамма-распределений с~применением к~задаче округления} % наблюдений}

\def\aut{А.\,К.~Горшенин$^1$}

\def\autkol{А.\,К.~Горшенин}

\titel{\tit}{\aut}{\autkol}{\titkol}

\index{Горшенин А.\,К.}
\index{Gorshenin A.\,K.}


{\renewcommand{\thefootnote}{\fnsymbol{footnote}} \footnotetext[1]
{Работа выполнена при поддержке РНФ (проект 18-71-00156).}}


\renewcommand{\thefootnote}{\arabic{footnote}}
\footnotetext[1]{Институт проблем информатики Федерального исследовательского центра 
<<Информатика и~управление>> Российской академии наук, \mbox{agorshenin@frccsc.ru}}

\vspace*{-12pt}




\Abst{Во многих реальных задачах проводится статистический анализ данных, 
содержащих дополнительные ошибки измерения, в~том числе в~виде округления, 
что в~ряде ситуаций может приводить к~достаточно существенным искажениям. 
В~настоящей статье для одной из возможных моделей округления получены оценки 
для неизвестного математического ожидания наблюдений в~предположении, что 
исходные данные дополнительно зашумлены с~по\-мощью случайных величин, 
име\-ющих распределения типа конечных смесей нормальных и~гам\-ма-за\-ко\-нов. 
Построены доверительные интервалы для неизвестного математического ожидания 
с~использованием уточненной оценки для дисперсии целой части случайной величины. 
Обсуждается алгоритм определения значения параметра для искусственного шума, 
добавление которого к~исходным данным способствует повышению качества работы 
метода скользящего разделения смесей.}

\KW{зашумленные данные; округленные наблюдения; конечные смеси нормальных 
распределений; конечные смеси гам\-ма-рас\-пре\-де\-ле\-ний; доверительные интервалы;  
метод скользящего разделения смесей}

\DOI{10.14357/19922264180304}
  
\vspace*{-4pt}


\vskip 10pt plus 9pt minus 6pt

\thispagestyle{headings}

\begin{multicols}{2}

\label{st\stat}


\section{Введение}

Во многих реальных задачах данные, являющиеся непрерывными по своей сути, 
регистрируются с~помощью инструментов, вносящих дополнительные ошибки 
измерения, в~том чис\-ле в~виде округления. Таким образом, статистический 
анализ проводится не для исходных, а для преобразованных некоторым 
случайным образом наблюдений, что в~ряде ситуаций может приводить к~достаточно
 существенным искажениям.

Для преодоления данной проблемы развивались различные подходы, в~том числе 
на основе смешанных моделей (см., например, статью~\cite{Wright2003}, в~которой 
различные компоненты  используются для пред\-став\-ле\-ния уровней округления). 
В~работе~\cite{Bai2009} приводятся результаты для моделей авторегрессии и~скользящего 
среднего для округленных данных, а~в~статье~\cite{Zhang2010} эти результаты 
развиваются и~исследуются их асимптотические свойства. 
В~статье~\cite{Zhao2012} исследован метод оценивания па\-ра\-мет\-ров конечных смесей 
вероятностных распределений (в~том чис\-ле, и~многомерных) 
на основе использования EM (expectation-maximization) 
алгоритма~\cite{Korolev2011-i} с~\mbox{целью} получения состоятельных 
и~асимптотически нормальных оценок.

В настоящей статье развиваются результаты для моделей округления, 
описанных в~работах~\cite{Ushakov2015,Ushakov2017a,Ushakov2017b}. 
В~их рамках будут получены оценки для неизве\-ст\-ного математического ожидания 
наблюдений в~предположении, что исходные данные зашумлены с~по\-мощью случайных 
величин, имеющих распределения типа конечных смесей нормальных и~гам\-ма-за\-ко\-нов. 
Это позволяет учесть большее количество случайных факторов, влия\-ющих на величину 
<<дополнительной>> ошибки. Также будут построены доверительные интервалы для 
неизвестного математического ожидания. Выражения для гам\-ма-рас\-пре\-де\-ле\-ний 
получены впервые. Также обсуждается алгоритм определения значения па\-ра\-мет\-ра для 
искусственного шума, добавление которого к~исходным данным способствует 
повышению качества работы метода скользящего разделения смесей~\cite{Gorshenin2016}.

\vspace*{-12pt}

\section{Предположения и~базовые отношения}

Для сокращения формулировок теорем в~сле\-ду\-ющих разделах сделаем ряд 
предположений, на которые будем ссылаться в~дальнейшем. Итак, пусть:
\begin{itemize}
\item[(A)] $X_1,X_2,\ldots$~--- независимые одинаково распределенные 
случайные величины с~неизвестным математическим ожиданием ${\sf E}_X\hm<+\infty$;
\item[(B)] $\varepsilon_1,\varepsilon_2,\ldots$~--- независимые одинаково 
распределенные случайные величины с~математическим ожиданием 
${\sf E}_\varepsilon\hm<+\infty$; %\label{B}
\item[(C)] $X_1,X_2,\ldots$ и~$\varepsilon_1,\varepsilon_2,\ldots$ 
являются независимыми;
\item[(D)] $Y_j=\left[X_j+\varepsilon_j+1/2\right]$ для всех $j\hm=1,2,\ldots$ 
представляют собой округление значения суммы случайных величин $X_j\hm+\varepsilon_j$ 
до ближайшего целого сверху (при этом запись~$[\cdot]$ соответствует целой 
части выражения).
\end{itemize}

В рамках данных предположений в~статье будут рассмотрены вопросы качества 
приближения неизвестного математического ожидания~${\sf E}_X$ для исходных данных 
в~ситуации, когда наблюдения для анализа получены с~аддитивной ошибкой c известными 
распределениями (см.\ предположение~(B)) и~дополнительно округляются до 
ближайшего целого (см.\ предположение~(D)).

Заметим, что в~силу усиленного закона больших чисел справедливы следующие выражения:
\begin{multline}
\fr{1}{n}\sum\limits_{j=1}^n Y_j\xrightarrow[n\to\infty]{\text{п.н.}}
{\sf E}_Y\equiv\mathbb{E}\left[X_1+\varepsilon_1+\fr{1}{2}\right]={}\\
{}=\mathbb{E}\left(X_j+\varepsilon_j+\fr{1}{2}\right)-\mathbb{E}
\left\{X_j+\varepsilon_j+\fr{1}{2}\right\}={}\\
{}={\sf E}_X+{\sf E}_\varepsilon+\fr{1}{2}-\mathbb{E}\left\{X_j+\varepsilon_j+\fr{1}{2}\right\}. 
\label{Law}
\end{multline}

Запись $\{\cdot\}$ в~формуле~\eqref{Law} соответствует дробной 
части выражения, а~п.н.\ обозначает сходимость в~смысле почти наверное.

Для доказательства результатов в~дальнейшем потребуется следующее 
представления для дробной части  абсолютно непрерывной случайной величины~$Z$ 
с~абсолютно  интегрируемой характеристической функцией~$\varphi_Z(t)$
 (см., например, Лемму~4 в~работе~\cite{Ushakov2017b}):
\begin{equation}
\label{Fract}
\mathbb{E}\{Z\}=\fr{1}{2}-\sum\limits_{n=1}^\infty 
\fr{\mathrm{Im}\left (\varphi_Z(2\pi n)\right)}{\pi n}\,.
\end{equation}

Через $\mathrm{Im}\,(\cdot)$ в~формуле~\eqref{Fract} обозначена мнимая часть 
соответствующей функции.

При построении доверительных интервалов в~дальнейшем будет 
использована следующая оценка, справедливая для любой случайной величины~$Z$:
\begin{equation}
\mathbb{D}[Z]\leqslant \left(\sqrt{\mathbb{D} Z}+\fr{1}{2}\right)^2.
\label{Var}
\end{equation}
Она может быть проверена непосредственно с~учетом представления 
$\mathbb{D} [Z]\hm=\mathbb{D}\left(Z\hm-\{Z\}\right)$, неравенства 
Ко\-ши--Бу\-ня\-ков\-ско\-го для ковариации и~соотношения 
 $\mathbb{D}\{Z\}\hm\leqslant 1/4$, справедливого для любой случайной величины~$Z$ 
 (см., например, статью~\cite{Ushakov2017b}). Отметим, что данная оценка 
 является более точной по сравнению с~использованным для аналогичных 
 целей в~работе~\cite{Ushakov2017b} соотношением 
 $\mathbb{D} [Z]\hm\leqslant 2\mathbb{D} Z\hm+1/2$. Действительно,
\begin{equation*}
2\mathbb{D} Z+\fr{1}{2}-\left(\sqrt{\mathbb{D} Z}+\fr{1}{2}\right)^2=
\left(\sqrt{\mathbb{D} Z}-\fr{1}{2}\right)^2\geqslant0\,,
\end{equation*}
причем для всех $\sqrt{\mathbb{D} Z}\hm\neq {1}/{2}$ 
данное неравенство является строгим.

\section{Конечные смеси нормальных законов}

Для случайной величины~$X$, имеющей распределение типа 
конечной смеси нормальных законов~\cite{Korolev2011-i} с~параметрами 
${\bf a}\hm=(a_1,\ldots, a_k)$, $a_j\hm\in \mathbb{R}$, 
$\boldsymbol{\sigma}\hm=(\sigma_1,\ldots, \sigma_k)$, 
$\sigma_j\hm>0$, ${\bf p}\hm=(p_1,\ldots, p_k)$, $p_j\hm\geqslant 0$, 
$\sum\nolimits_{j=1}^{k}p_j\hm=1$, плот\-ность которого задается выражением
\begin{equation}
f_X(x)=\sum\limits_{j=1}^{k}\fr{p_j}{\sigma_j\sqrt{2\pi}}\,e^{-(x-a_j)^2/(2\sigma_j^2)}\,,
\label{FinNormMixt}
\end{equation}
характеристическая функция имеет вид:
\begin{equation}
\varphi_X(t)=\int\limits_{-\infty}^{+\infty}\!\!e^{itx} f_X(x)\, dx = 
\sum\limits_{j=1}^{k}p_j e^{ita_j-\sigma_j^2 t^2/2}.
\label{ChiFinNormMixt}
\end{equation}

Абсолютная интегрируемость  $\varphi_X(t)$ вытекает из свойств 
характеристической функции нормального распределения. 
Заметим, что в~точке $t\hm=2\pi n$ выражение~\eqref{ChiFinNormMixt} принимает 
сле\-ду\-ющий вид:
\begin{equation}
\label{ChiFinNormMixt2npi}
\varphi_X(2\pi n)= \sum\limits_{j=1}^{k}p_j e^{-2\pi^2 \sigma_j^2 n^2}\,.
\end{equation}

Рассмотрим вопрос точности оценивания неизвестного математического ожидания~${\sf E}_X$ 
при до\-бав\-ле\-нии зашумления.

\smallskip

\noindent
\textbf{Теорема~1.}\ 
\textit{Пусть выполнены предположения}~(A)--(D), 
\textit{причем случайные величины~$\varepsilon_j$, $j\hm=1,2,\ldots$, 
имеют распределение типа конечной $k$-ком\-по\-нент\-ной смеси нормальных законов 
вида}~\eqref{FinNormMixt} \textit{с~па\-ра\-мет\-ра\-ми~${\bf a}$, $\boldsymbol{\sigma}$ 
и~${\bf p}$. Тогда}
\begin{equation}
\label{Th1Eq}
\left\lvert {\sf E}_Y-{\sf E}_X\right\rvert \leqslant 
A+\fr{1}{\pi}\left(1+\fr{1}{4\pi^2\sigma^2}\right)e^{-2\pi^2\sigma^2}\,, 
\end{equation}
\textit{где} $A=\max(|a_1|,\ldots,|a_k|)$, $\sigma\hm=\min(\sigma_1,\ldots,\sigma_k)$.

\smallskip


\noindent
Д\,о\,к\,а\,з\,а\,т\,е\,л\,ь\,с\,т\,в\,о\,.\ \
С~учетом пред\-став\-ле\-ний~\eqref{Law},~\eqref{Fract} и~\eqref{ChiFinNormMixt2npi}, 
ограниченности модуля характеристической функции, а~также не\-за\-ви\-си\-мости 
случайных величин~$X_j$ и~$\varepsilon_j$ имеем:
\begin{multline*}
\left\lvert {\sf E}_Y-{\sf E}_X\right\rvert =
\left\lvert {\sf E}_\varepsilon+\fr{1}{2}-\mathbb{E}\left\{X_j+
\varepsilon_j+\fr{1}{2}\right\}\right\rvert={}\\
{}=\left\lvert {\sf E}_\varepsilon+\sum\limits_{n=1}^\infty
\fr{\mathrm{Im} \left(\varphi_{X_j}(2\pi n)\varphi_{\varepsilon_j}(2\pi n)
\varphi_{1/2}(2\pi n)\right)}{\pi n}\right\rvert={}\\
=\left\lvert 
\vphantom{\fr{(-1)^n\sum\nolimits_{j=1}^{k}p_j e^{-2\pi^2 \sigma_j^2 n^2} 
\mathrm{Im} \left(\varphi_{X_j}(2\pi n)\right)}{\pi n}}
{\sf E}_\varepsilon+{}\right.\\
\left.{}+\sum\limits_{n=1}^\infty
\fr{\mathrm{Im} \left(\varphi_{X_j}(2\pi n) 
\sum\nolimits_{j=1}^{k}p_j e^{-2\pi^2 \sigma_j^2 n^2} 
e^{\pi n}\right)}{\pi n}\right\rvert={}\\
{}=\left\lvert 
\vphantom{\fr{(-1)^n\sum\nolimits_{j=1}^{k}p_j e^{-2\pi^2 \sigma_j^2 n^2} 
\mathrm{Im} \left(\varphi_{X_j}(2\pi n)\right)}{\pi n}}
{\sf E}_\varepsilon+{}\right.\\
\left.{}+\sum\limits_{n=1}^\infty
\fr{(-1)^n\sum\nolimits_{j=1}^{k}p_j e^{-2\pi^2 \sigma_j^2 n^2} 
\mathrm{Im} \left(\varphi_{X_j}(2\pi n)\right)}{\pi n}\right\rvert\leqslant{}\\
{}\leqslant \left\lvert {\sf E}_\varepsilon\right\rvert+\left\lvert\
\sum\limits_{j=1}^{k}p_j\sum\limits_{n=1}^\infty 
\fr{1}{\pi n} e^{-2\pi^2 \sigma_j^2 n^2}\right\rvert\leqslant {}\\
\\
{}\leqslant
\max\left(|a_1|,\ldots,|a_k|\right)+{}\\
{}+\sum\limits_{j=1}^{k} 
\fr{p_j}{\pi} \left(\!1+\fr{1}{4\pi^2\sigma_j^2}\!\right)\!e^{-2\pi^2\sigma_j^2}\leqslant{}\\
{}\leqslant
A+\fr{1}\pi\left(1+\fr{1}{4\pi^2\sigma^2}\right)e^{-2\pi^2\sigma^2}\,.
\end{multline*}

Справедливость использованной оценки 
\begin{equation*}
\sum\limits_{n=1}^\infty
\fr{e^{-2\pi^2 \sigma_j^2 n^2}}{n}\leqslant 
\left(1+\fr{1}{4\pi^2\sigma_j^2}\right)e^{-2\pi^2\sigma_j^2}
\end{equation*}
может быть проверена непосредственно (например, см.\ доказательство Теоремы~6 
в~статье~\cite{Ushakov2017b}).~\hfill$\square$

\smallskip

\noindent
\textbf{Замечание~1.}
В~случае, если зашумление производится нормально распределенными случайными 
величинами c нулевыми средними (т.\,е.\ в~формуле~\eqref{Th1Eq} необходимо считать 
$A\hm=0$, $k\hm=1$), то Тео\-ре\-ма~1 совпадает с~результатом, 
полученным в~работе~\cite{Ushakov2017b}.


\smallskip

Рассмотрим вопросы построения доверительного интервала для неизвестного 
математического ожидания~${\sf E}_X$ в~предположении, что случайные величины~$X_j$ не 
содержат ошибок измерения, а~все погрешности учтены исключительно в~за\-шум\-ля\-ющих 
элементах~$\varepsilon_j$.

\smallskip

\noindent
\textbf{Теорема~2.}\ 
\textit{Пусть выполнены предположения}~(A)--(D), 
\textit{причем случайные величины~$\varepsilon_j$, $j\hm=1,2,\ldots$, имеют 
распределение типа конечной $k$-ком\-по\-нент\-ной смеси нормальных законов 
вида}~\eqref{FinNormMixt} \textit{с~параметрами~${\bf a}$, $\boldsymbol{\sigma}$ 
и~${\bf p}$, а~случайные величины} $X_j\stackrel{\text{п.н.}}{=}{\sf E}_X$, $j\hm=1,2,\ldots$ 
\textit{Тогда доверительный интервал для~${\sf E}_X$ при условии $0\hm<\alpha\hm<1$ имеет вид}:
\begin{equation} 
\label{Th2Eq}
\hat{{\sf E}}_X - f({\bf a},\boldsymbol{\sigma},\alpha,n) 
\leqslant {\sf E}_X \leqslant  \hat{{\sf E}}_X + f({\bf a},\boldsymbol{\sigma},\alpha,n),
\end{equation}
\textit{где}

\vspace*{-2pt}

\noindent
\begin{align}
\hat{{\sf E}}_X&=\fr{1}{n} \sum\limits_{j=1}^{n} Y_j\,; \label{Th2hatE}\\
f({\bf a},\boldsymbol{\sigma},\alpha,n)&=
\fr{z_{1-{\alpha}/2}}{\sqrt{n}} \left(\sqrt{A^2+\Sigma^2}+\fr{1}{2}\right) +{}\notag\\
&{}+A+\fr{1}\pi\left(1+\fr{1}{4\pi^2\sigma^2}\right)e^{-2\pi^2\sigma^2}\,;
  \label{Th2f}
\end{align}
\textit{$z_{1-{\alpha}/2}$~--- $\left(1-{\alpha}/2\right)$-кван\-тиль 
стандартного нормального распределения; $A\hm=\max(|a_1|,\ldots,|a_k|)$; 
$\Sigma\hm=\max(\sigma_1,\ldots,\sigma_k)$; $\sigma\hm=\min(\sigma_1,\ldots,\sigma_k)$}. 


\smallskip

\noindent
\noindent
Д\,о\,к\,а\,з\,а\,т\,е\,л\,ь\,с\,т\,в\,о\,.\ \
Из центральной предельной тео\-ре\-мы с~учетом условия~(A) следует, 
что величина~$\hat{{\sf E}}_X$~\eqref{Th2hatE} асимптотически нормальна с~математическим 
ожиданием 
\begin{equation}
{\sf E}_Y\equiv \mathbb{E}\left[{\sf E}_X+\varepsilon_1+\fr{1}{2}\right] \label{EY}
\end{equation}
и дисперсией
\begin{equation}
\fr{1}{n} {\sf D}_Y\equiv \fr{1}{n}\mathbb{D}\left[{\sf E}_X+\varepsilon_1+
\fr{1}{2}\right]. \label{DY}
\end{equation}

Воспользовавшись оценкой~\eqref{Var}, получим:

\vspace*{-2pt}

\noindent
\begin{multline*}
{\sf D}_Y \leqslant  \left(\sqrt{\mathbb{D} \left({\sf E}_X+\varepsilon_1+\fr{1}{2}\right)}+
\fr{1}{2}\right)^2={}\\
{}=
\left(\sqrt{\mathbb{D}\varepsilon_1}+\fr{1}{2}\right)^2= {}\\
{}= \left(\sqrt{\sum\limits_{j=1}^{k}p_j\left(\left(a_j-\sum\limits_{t=1}^{k}
p_t a_t\right)^2+\sigma_j^2\right)}+\fr{1}{2}\right)^2\leqslant {}\\ 
{}\leqslant \left(\sqrt{A^2+\Sigma^2}+\fr{1}{2}\right)^2\,.
\end{multline*}
Тогда доверительный интервал уровня $1\hm-\alpha$ для математического ожидания~${\sf E}_Y$ 
имеет вид:
\begin{equation*}
\mathbb{P}\left(\left\lvert \hat{{\sf E}}_X-{\sf E}_Y\right\rvert \leqslant 
\fr{z_{1-{\alpha}/2}}{\sqrt{n}} 
\left(\sqrt{A^2+\Sigma^2}+\fr{1}{2}\right)\right)\geqslant 1-\alpha\,.
\end{equation*}

\begin{table*}[b]\small
\begin{center}

\begin{tabular}{|c|c|c|c|c|c|c|c|}
\multicolumn{7}{p{100mm}}{Численные решения уравнений~\eqref{f1} и~\eqref{f2} относительно 
параметра~$\sigma$ для некоторых значений~$n$ и~$\alpha$}\\
\multicolumn{7}{c}{\ }\\[-6pt]
\hline
\multicolumn{1}{|c|}{Размер}  & \multicolumn{2}{c|}{Уровень $\alpha=0{,}1$}& 
\multicolumn{2}{c|}{Уровень $\alpha=0{,}05$}& 
\multicolumn{2}{c|}{Уровень $\alpha=0{,}01$}\\
\cline{2-7}
\multicolumn{1}{|c|}{выборки $n$}&$\sigma_1$&$\sigma_2$&$\sigma_1$&$\sigma_2$&$\sigma_1$&$\sigma_2$\\
\hline
$\hphantom{000}100$&$0{,}4302$&$0{,}435$&$0{,}419$&$0{,}425$&$0{,}4002$&$0{,}408$\\
%\hline
$\hphantom{000}200$&$0{,}452$&$0{,}455$ &$0{,}441$&$0{,}445$&$0{,}424$&$0{,}429$\\
%\hline
$\hphantom{00}1000$&$0{,}499$&$0{,}499$ &$0{,}489$&$0{,}489$&$0{,}473$&$0{,}475$\\
%\hline
$\hphantom{0}10000$&$0{,}558$&$0{,}556$ &$0{,}549$&$0{,}547$&$0{,}536$&$0{,}534$\\
%\hline
$100000$&$0{,}611$&$0{,}607$ &$0{,}603$&$0{,}599$&$0{,}591$&$0{,}588$\\
\hline
\end{tabular}
\end{center}
\end{table*}


\noindent
Откуда следует справедливость соотношения~\eqref{Th2Eq} c~уче\-том 
очевидного неравенства

\pagebreak

\noindent
\begin{equation*}
\left\lvert \hat{{\sf E}}_X-{\sf E}_X\right\rvert \leqslant 
\left\lvert \hat{{\sf E}}_X-{\sf E}_Y\right\rvert +\left\lvert {\sf E}_Y-{\sf E}_X\right\rvert 
\end{equation*}
и оценки~\eqref{Th1Eq} из Теоремы~1.~\hfill$\square$

\smallskip

\noindent
\textbf{Замечание~2.}
В~работе~\cite{Gorshenin2016} было продемонстрировано повышение точ\-ности 
работы метода скользящего разделения конечных нормальных смесей за счет 
введения дополнительной компоненты, имеющей нормальное 
распределение $\mathcal{N}(0,\sigma^2)$ с~математическим ожиданием, равным~$0$, 
и~стандартным отклонением~$\sigma$. При этом была отмечена сложность выбора 
параметра~$\sigma$ для сохранения структуры выборки, близкой к~исходной. 
Результат Теоремы~2 может быть использован с~данной целью, если положить $k\hm=1$, 
$a_j\hm=0$ для всех $j\hm=1,2,\ldots$ и~выбирать величину~$\sigma$ как 
минимизирующую длину доверительного интервала~\eqref{Th2Eq}. Для 
этого необходимо найти производную функции $f(0,\sigma,\alpha,n)$~\eqref{Th2f} 
и~численно решить уравнение
\begin{multline}
f_\sigma'(0,\sigma,\alpha,n)\equiv \fr{z_{1-{\alpha}/2}}{\sqrt{n}} - {}\\
{}-
e^{-2\pi^2\sigma^2}\left(4\pi\sigma+\fr{1}{2\pi^3\sigma^3}+
\fr{1}{\pi\sigma}\right)=0
\label{f1}
\end{multline}
относительно неизвестного параметра~$\sigma$ при выбранных значениях величин~$n$ 
и~$\alpha$. В~качестве альтернативы можно использовать вид доверительного интервала 
из статьи~\cite{Ushakov2017b}, полученный с~помощью неравенства $\mathbb{D} [Z]
\hm\leqslant 2\mathbb{D} Z\hm+{1}/{2}$, и~искать решение уравнения вида:
\begin{multline}
\hspace*{-2.90578pt}\fr{2\sigma z_{1-{\alpha}/2}}{\sqrt{n (2\sigma^2+{1}/{2})}} -
 e^{-2\pi^2\sigma^2}\left(4\pi\sigma+\fr{1}{2\pi^3\sigma^3}+
 \fr{1}{\pi\sigma}\right)={}\\
 {}=0\,.\label{f2}
\end{multline}

Примеры найденных значений~$\sigma$ для типичных размеров выборок в~методе 
скользящего разделения смесей (учитываются как возможная ширина окна, 
так и~общее количество наблюдений в~анализируемом ряде) приведены в~таблице 
(использован метод оптимизации \verb"Trust-Region Dogleg" пакета \verb"MATLAB" 
c~настройками по умолчанию), в~которой через~$\sigma_1$ обозначено приближенное  
решение уравнения~\eqref{f1}, a~через $\sigma_2$~--- уравнения~\eqref{f2}.


Проверка практической эффективности данного подхода в~качестве 
критерия выбора параметров зашумляющего распределения для повышения 
точности работы метода скользящего разделения смесей может быть отмечена 
как задача для дальнейших исследований.


\section{Конечные смеси гамма-распределений}

Для случайной величины~$X$, имеющей распределение типа конечной смеси 
гам\-ма-рас\-пре\-де\-ле\-ний с~параметрами ${\bf r}\hm=(r_1,\ldots, r_k)$,
 $r_j\hm>0$, $\boldsymbol{\lambda}\hm=(\lambda_1,\ldots, \lambda_k)$, $\lambda_j\hm>0$, 
 ${\bf p}\hm=(p_1,\ldots, p_k)$, $p_j\hm\geqslant 0$, $\sum\nolimits_{j=1}^{k}p_j\hm=1$, 
 плот\-ность которого задается выражением
\begin{equation}
f_X(x)=\sum\limits_{j=1}^{k}p_j\fr{\lambda_j^{r_j} e^{-\lambda_j x}}
{\Gamma(r_j)}\,x^{r_j-1}\,,
\label{FinGammaMixt}
\end{equation}
характеристическая функция имеет следующий вид:
%характеристическая функция задается следующим выражением:
\begin{equation}
\varphi_X(t)=\!\int\limits_{-\infty}^{+\infty}\!\!\!e^{itx} f_X(x)\, dx = \!
\sum\limits_{j=1}^{k}p_j \left(\!1-\fr{it}{\lambda_j}\right)^{-r_j}\!.\!
\label{ChiFinGammaMixt}
\end{equation}

Отметим, что подобные модели зашумления разумно использовать в~случае, 
если известно, что данные сосредоточены на положительной полуоси, например 
при анализе различных информационных потоков (см., в~част\-ности, 
 работу~\cite{Gorshenin2013}). 

Проверим абсолютную интегрируемость функции $\varphi_X(t)$~\eqref{ChiFinGammaMixt}. 
Имеем:
\begin{multline*}
\int\limits_{-\infty}^{+\infty}\left\lvert\varphi_X(t)\right\rvert\, dt\leqslant 
\sum\limits_{j=1}^{k}p_j \int\limits_{-\infty}^{+\infty}\left\lvert \left(
1-\fr{it}{\lambda_j}\right)^{-r_j}\right\rvert \, dt={}\\
{}=\sum\limits_{j=1}^{k}p_j \int\limits_{-\infty}^{+\infty} \left\lvert\left(
\fr{\lambda_j(\lambda_j+it)}{\lambda_j^2+t^2}\right)^{r_j}\right\rvert\, dt \leqslant{}\\
{}\leqslant\sum\limits_{j=1}^{k}p_j \lambda_j \int\limits_{-\infty}^{+\infty}\left(
1+y^2\right)^{-{r_j}/{2}}\, dy\,.
\end{multline*}

Подынтегральное выражение при $r_j\hm\geqslant 2$ может быть оценено сверху 
функцией $1/({1+y^2})$, при этом соответствующий интеграл равен~$\pi$, что влечет 
абсолютную интегрируемость характеристической функции для конечной смеси 
гам\-ма-рас\-пре\-де\-ле\-ний. Поэтому в~дальнейшем будем предполагать,
 что $r_j\hm\geqslant 2$ для всех возможных значений $j\hm=1,2,\ldots$

Рассмотрим вопрос точ\-ности оценивания неизвестного математического ожидания ${\sf E}_X\hm>0$ 
при добавлении зашумления.

\smallskip

\noindent
\textbf{Теорема~3.}
\textit{Пусть выполнены предположения}~(A)--(D), 
\textit{причем случайные величины~$\varepsilon_j$, $j\hm=1,2,\ldots$, имеют 
распределение типа конечной $k$-ком\-по\-нент\-ной смеси 
гам\-ма-рас\-пре\-де\-ле\-ний вида}~\eqref{FinGammaMixt} 
\textit{с~па\-ра\-мет\-ра\-ми~${\bf r}$, $\boldsymbol{\lambda}$ и~${\bf p}$. Тогда}
\begin{equation}
\label{Th3Eq}
\left\lvert {\sf E}_Y-{\sf E}_X\right\rvert \leqslant \fr{R}{\lambda}+
\fr{\Lambda^{R}}{2^{r}\pi^{r+1}}\left(1+\frac1{r}\right)\,,
\end{equation}
\textit{где} $r=\min(r_1, \ldots,r_k)$; $R\hm=\max(r_1, \ldots,r_k)$; 
$\lambda\hm=\max(\lambda_1, \ldots,\lambda_k)$; 
$\Lambda\hm=\max(\lambda_1, \ldots,\lambda_k)$.

\smallskip

\noindent
Д\,о\,к\,а\,з\,а\,т\,е\,л\,ь\,с\,т\,в\,о\,.\ \
С~учетом пред\-став\-ле\-ний~\eqref{Law} и~\eqref{Fract}, ограниченности 
модуля характеристической функции, перехода от тригонометрической к~показательной 
записи комплексных чисел, а~также независимости случайных величин~$X_j$ 
и~$\varepsilon_j$ \mbox{имеем}:
\begin{multline*}
\left\lvert {\sf E}_Y-{\sf E}_X\right\rvert
\leqslant \left\lvert {\sf E}_\varepsilon\right\rvert+ {}\\
{}+\left\lvert\sum\limits_{n=1}^\infty
\left(
(-1)^n\mathrm{Im} \left(\sum\limits_{j=1}^{k}p_j \varphi_{X_j}(2\pi n)\left(
\vphantom{\fr{2\pi n}{\lambda_j}}
1-{}\right.\right.\right.\right.\\
\left.\left.\left.\left.{}-i\left(\fr{2\pi n}{\lambda_j}\right)\right)^{-r_j}\right)
\Bigg/ ({\pi n})
\vphantom{\sum\limits_{j=1}^{k}}
\right)\right\rvert={}\\
{}=\left\lvert {\sf E}_\varepsilon\right\rvert+ 
\left\lvert\sum\limits_{n=1}^\infty
\left(\!(-1)^n\mathrm{Im} \!\left(\sum\limits_{j=1}^{k}p_j \left(\!
1+\fr{4\pi^2 n^2}{\lambda_j^2}\right)^{- {r_j}/2}\!\times{}\right.\right.\right.\hspace*{-2.8663pt}\\
\left.\left.\left.{}\times \varphi_{X_j}(2\pi n)\,
e^{-ir_j\mathrm{arctan}\,({{t}/{\lambda_j}})}\right)
\Bigg/
({\pi n})
\vphantom{\left(
1+\fr{4\pi^2 n^2}{\lambda_j^2}\right)^{- {r_j}/2}}
\right)\right\rvert\leqslant{}\\
{}\leqslant \left\lvert {\sf E}_\varepsilon\right\rvert+\sum\limits_{j=1}^{k}
p_j\sum\limits_{n=1}^\infty\fr{1}{\pi n}\left(
1+\fr{4\pi^2 n^2}{\lambda_j^2}\right)^{-{r_j}/2}\leqslant{}\\
{}\leqslant  \fr{R}\lambda + \sum\limits_{j=1}^{k}p_j
\sum\limits_{n=1}^\infty\left(\fr{1}{\pi n}\,
\fr{\lambda_j^{r_j}}{(2\pi)^{r_j} n^{r_j}}\right)\leqslant {}
\\
{}\leqslant  \fr{R}{\lambda} + \sum\limits_{j=1}^{k}p_j 
\fr{\lambda_j^{r_j}}{2^{r_j}\pi^{r_j+1}}\left(1+\int\limits_{1}^{\infty}
\fr{1}{ x^{r_j+1}}\,dx\right)
\leqslant{}\\
{}\leqslant \fr{R}{\lambda}+\fr{\Lambda^{R}}{2^{r}\pi^{r+1}}\left(1+\fr{1}{r}\right).
\end{multline*}

При переходе от суммы к~интегралу используется факт убывания функции как переменной~$n$ 
(или~$x$).~\hfill$\square$


\smallskip

\noindent
\textbf{Замечание~3.}\
Теорема~3 описывает соответ\-ст\-ву\-ющий результат для гам\-ма-рас\-пре\-де\-лен\-ных 
за\-шум\-ля\-ющих случайных величин, если положить $k\hm=1$ в~выражении~\eqref{Th3Eq}. 
При этом, очевидно, $r\hm\equiv R$ и~$\lambda\hm\equiv \Lambda$.


\smallskip

Рассмотрим вопросы построения доверительного интервала для неизвестного 
математического ожидания ${\sf E}_X\hm>0$ в~предположении, что случайные величины~$X_j$ 
не содержат ошибок измерения, а все погрешности учтены исключительно в~за\-шум\-ля\-ющих 
элементах~$\varepsilon_j$.

\smallskip

\noindent
\textbf{Теорема~4.}
\textit{Пусть выполнены предположения}~(A)--(D), 
\textit{причем случайные величины~$\varepsilon_j$, $j\hm=1,2,\ldots$, имеют 
распределение типа конечной $k$-ком\-по\-нент\-ной смеси 
гам\-ма-рас\-пре\-де\-ле\-ний вида}~\eqref{FinGammaMixt} 
\textit{с~па\-ра\-мет\-ра\-ми~${\bf r}$, $\boldsymbol{\lambda}$ и~${\bf p}$, 
а~случайные величины} $X_j\stackrel{\text{п.н.}}{=}{\sf E}_X$, $j=1,2,\ldots$ 
\textit{Тогда доверительный интервал для~${\sf E}_X$ при условии $0\hm<\alpha\hm<1$ имеет вид}:
\begin{equation} 
\label{Th4Eq}
\left\lvert {\sf E}_X - \hat{{\sf E}}_X\right\rvert \leqslant  
f({\bf r},\boldsymbol{\lambda},\alpha,n),
\end{equation}
\textit{где}

\vspace*{-9pt}

\noindent
\begin{align}
\hat{{\sf E}}_X&=\fr{1}{n} \sum\limits_{j=1}^{n} Y_j\,; \label{Th4hatE}\\[-4pt]
f({\bf r}, \boldsymbol{\lambda},\alpha,n)&=\fr{z_{1-{\alpha}/2}}{\sqrt{n}} \left(
\sqrt{\fr{R(R+1)}{\lambda^2}-\fr{r^2}{\Lambda^2}}+\fr{1}{2}\right) +{}\notag\\[-1pt]
&\hspace*{20mm}{}+
\fr{R}{\lambda}+\fr{\Lambda^{R}}{2^{r}\pi^{r+1}}\left(1+\fr{1}{r}\right); \notag
\end{align}
\textit{$z_{1-{\alpha}/2}$~--- $\left(1-{\alpha}/2\right)$-кван\-тиль 
стандартного нормального распределения; $r\hm=\min(r_1, \ldots,r_k)$; 
$R\hm=\max(r_1, \ldots,r_k)$; $\lambda\hm=\max(\lambda_1, \ldots,\lambda_k)$; 
$\Lambda\hm=\max(\lambda_1, \ldots,\lambda_k)$}. 

\smallskip

\noindent
Д\,о\,к\,а\,з\,а\,т\,е\,л\,ь\,с\,т\,в\,о\,.\ \
Из центральной предельной теоремы с~учетом условия~(A) 
следует, что величина~$\hat{{\sf E}}_X$~\eqref{Th4hatE} асимптотически нормальна 
с~математическим ожиданием~${\sf E}_Y$~\eqref{EY} и~дисперсией $(1/n){\sf D}_Y$~\eqref{DY}. 
Пользуясь определением и~свойствами гам\-ма-функ\-ции, а~также оценкой~\eqref{Var} 
получим:

\noindent
\begin{multline*}
{\sf D}_Y \leqslant \left(\sqrt{\sum\limits_{j=1}^k p_j
\fr{\lambda_j^{r_j}}{\Gamma(r_j)} \int\limits_{0}^{+\infty} 
e^{\lambda_j x}x^{r_j+1}\, dx}+\fr{1}{2}\right)^2= {}\\[-0.5pt]
= \left(\sqrt{\sum\limits_{j=1}^{k}p_j
\fr{r_j(r_j+1)}{\lambda_j^2}-\left(\sum\limits_{j=1}^{k}p_j
\fr{r_j}{\lambda_j}\right) ^2}+\fr{1}{2}\right)^2\leqslant {}\\[-1.5pt]
{}\leqslant \left(\sqrt{\fr{R(R+1)}{\lambda^2}-\fr{r^2}{\Lambda^2}}+\fr{1}{2}\right)^2\,.
\end{multline*}

Аналогично доказательству Тео\-ре\-мы~2 с~учетом оценки~\eqref{Th3Eq} 
отсюда следует справедливость соотношения~\eqref{Th4Eq}.~\hfill$\square$

\vspace*{-12pt}

\section{Заключение}

Итак, в~работе получены оценки для математического ожидания наблюдений в~предположении 
зашумления конечными смесями нормальных\linebreak (Тео\-ре\-ма~1) 
и~гам\-ма-рас\-пре\-де\-ле\-ний (Тео\-ре\-ма~3). 
%
Построены доверительные интервалы 
для неизвестного математического ожидания в~этих случаях с~использованием 
уточненной оценки~\eqref{Var} 
(Тео\-ре\-мы~2 и~4 соответственно). Отметим, что соответствующие соотношения 
зависят только от <<экстремальных>> значений параметров смесей, но не от числа 
компонент и~весов в~распределении зашумляющих наблюдений. 
%
Замечание~2 
предлагает подход, который  может быть использован для определения неизвестного 
параметра искусственно добавляемого к~исходным данным шума для улучшения качества 
работы метода скользящего разделения смесей.

\smallskip
Автор выражает признательность доктору фи\-зи\-ко-ма\-те\-ма\-ти\-че\-ских наук, 
профессору Виктору Юрьевичу Королеву за идею использования оценки 
дисперсии вида~\eqref{Var} и~другие полезные обсуждения в~рамках 
работы над данной статьей.

\vspace*{-12pt}

{\small\frenchspacing
 {%\baselineskip=10.8pt
 \addcontentsline{toc}{section}{References}
 \begin{thebibliography}{99}
\bibitem{Wright2003} \Au{Wright~D.\,E., Bray~I.} 
A~mixture model for rounded data~// J.~Roy. Stat. Soc.~D 
Sta., 2003. Vol.~52. P.~3--13.

\columnbreak

\bibitem{Bai2009} \Au{Bai~Z., Zheng~S., Zhang~B., Hu~G.} 
Statistical analysis for rounded data~// J.~Stat. Plan.  Infer., 2009. 
Vol.~139. Iss.~8. P.~2526--2542.

\bibitem{Zhang2010} \Au{Zhang~B., Liu~T., Bai~Z.\,D.} 
Analysis of rounded data from dependent sequences~// 
Ann. I.~Stat. Math., 2010. Vol.~62. Iss.~6. P.~1143--1173.

\bibitem{Zhao2012} \Au{Zhao~N., Bai~Z.} 
Analysis of rounded data in mixture normal model~// Stat. Pap., 2012. 
Vol.~53. P.~895--914.

\bibitem{Korolev2011-i} \Au{Королев~В.\,Ю.} 
Ве\-ро\-ят\-но\-ст\-но-ста\-ти\-сти\-че\-ские методы декомпозиции волатильности 
хаотических процессов.~--- М.: Изд-во Моск. ун-та, 2011. 512~с.

\bibitem{Ushakov2015} \Au{Ушаков В.\,Г., Ушаков Н.\,Г.} 
Об усреднении округленных данных~// Информатика и~её применения, 2015. Т.~9. 
Вып.~4. С.~106--109.

\bibitem{Ushakov2017a} \Au{Ушаков~В.\,Г., Ушаков~Н.\,Г.} 
Границы точ\-ности восстановления информации, 
теряемой при округлении результатов наблюдений~// 
Вестник Московского университета. Серия~15: Вычислительная математика и~кибернетика, 
2017. №\,2. С.~26--30.

\bibitem{Ushakov2017b} \Au{Ushakov~N.\,G., Ushakov~V.\,G.} 
Statistical analysis of rounded data: Recovering of information lost due to rounding~// 
J.~Korean Stat. Soc., 2017.  Vol.~46. No.\,3. P.~426--437.

\bibitem{Gorshenin2016} \Au{Gorshenin~A.\,K., Korolev~V.\,Yu.} 
A~noising method for the identification of the stochastic structure of 
information flows~// Comm. Com. Inf. Sc., 2017. 
Vol.~678. P.~279--289.

\bibitem{Gorshenin2013} 
\Au{Gorshenin~A., Korolev~V.} Modelling of statistical
fluctuations of information flows by mixtures of gamma distributions~// 
27th European Conference on Modelling and Simulation Proceedings.~--- 
Dudweiler, Germany: Digitaldruck Pirrot GmbHP, 2013. P.~569--572.
 \end{thebibliography}

 }
 }

\end{multicols}

\vspace*{-6pt}

\hfill{\small\textit{Поступила в~редакцию 03.08.18}}

\vspace*{6pt}

%\newpage

%\vspace*{-24pt}

\hrule

\vspace*{2pt}

\hrule

\vspace*{-2pt}


\def\tit{DATA NOISING BY FINITE NORMAL AND~GAMMA MIXTURES WITH~APPLICATION 
TO~THE~PROBLEM OF~ROUNDED OBSERVATIONS}


\def\titkol{Data noising by finite normal and~gamma mixtures with~application 
to~the~problem of~rounded observations}



\def\aut{A.\,K.~Gorshenin}

\def\autkol{A.\,K.~Gorshenin}

\titel{\tit}{\aut}{\autkol}{\titkol}

\vspace*{-11pt}


\noindent
Institute of Informatics Problems, Federal Research Center ``Computer Science and
Control'' of the Russian Academy of Sciences, 44-2~Vavilov Str., Moscow 119333,
Russian Federation


\def\leftfootline{\small{\textbf{\thepage}
\hfill INFORMATIKA I EE PRIMENENIYA~--- INFORMATICS AND
APPLICATIONS\ \ \ 2018\ \ \ volume~12\ \ \ issue\ 3}
}%
 \def\rightfootline{\small{INFORMATIKA I EE PRIMENENIYA~---
INFORMATICS AND APPLICATIONS\ \ \ 2018\ \ \ volume~12\ \ \ issue\ 3
\hfill \textbf{\thepage}}}

\vspace*{3pt}



\Abste{In many real problems, statistical analysis of data containing additional 
measurement errors, including rounding, is performed, which in some situations can 
lead to sufficiently significant distortions. In this paper, estimates for an 
unknown expectation of observations are obtained for one of the possible 
rounding models under the assumption that the original data are additionally 
noised with random variables having distributions of the type of finite 
mixtures of normal and gamma laws. Confidence intervals for an 
unknown expectation are constructed using the refined estimate for 
the variance of the integer part of the random variable. An algorithm 
for determining the value of the parameter of artificial noise, which 
can be added to the initial data to improve the quality of the 
method of moving separation of mixtures, is discussed.}


\KWE{noisy data; rounded data; finite normal mixtures; finite gamma mixtures; 
confidence intervals; moving separation of mixtures}



\DOI{10.14357/19922264180304}

%\vspace*{-14pt}

\Ack
\noindent
The research was supported by the Russian Science Foundation (project 18-71-00156).



%\vspace*{6pt}

  \begin{multicols}{2}

\renewcommand{\bibname}{\protect\rmfamily References}
%\renewcommand{\bibname}{\large\protect\rm References}

{\small\frenchspacing
 {%\baselineskip=10.8pt
 \addcontentsline{toc}{section}{References}
 \begin{thebibliography}{99}
\bibitem{1-gor-1}
\Aue{Wright,~D.\,E., and I.~Bray.} 2003.
A~mixture model for rounded data.  \textit{J.~Roy. Stat. Soc.~D Sta.} 52:3--13.

\bibitem{2-gor-1}
\Aue{Bai,~Z., S.~Zheng, B.~Zhang, and G.~Hu.} 2009. 
Statistical analysis for rounded data. \textit{J.~Stat. Plan. 
Infer.} 139(8):2526--2542.

\bibitem{3-gor-1}
\Aue{Zhang,~B., T.~Liu, and Z.\,D.~Bai.} 2010. 
Analysis of rounded data from dependent sequences. 
\textit{Ann. I.~Stat. Math.} 62(6):1143--1173.

\bibitem{4-gor-1}
\Aue{Zhao,~N., and Z.~Bai.} 2012. Analysis of rounded data in mixture normal model. 
\textit{Stat. Pap.} 53:895--914.

\bibitem{5-gor-1}
\Aue{Korolev, V.\,Yu.} 2011. 
\textit{Veroyatnostno-statisticheskie metody dekompozitsii volatil'nosti 
khaoticheskikh protsessov} [Probabilistic and statistical methods of 
decomposition of volatility of chaotic processes]. 
Moscow: Moscow University Publishing House. 512~p.

\bibitem{6-gor-1}
\Aue{Ushakov, V.\,G., and N.\,G.~Ushakov.} 
2015. Ob usrednenii okruglennykh dannykh [On averaging of rounded data].
\textit{Informatika i~ee Primeneniya~--- Inform. Appl.} 9(4):106--109.

\bibitem{7-gor-1}
\Aue{Ushakov,~V.\,G., and N.\,G.~Ushakov.} 2017. 
Boundaries of the precision of restoring information lost after rounding
 the results from observations. 
 \textit{Moscow University Computational Math. Cybernetics} 41(2):76--80.

\bibitem{8-gor-1}
\Aue{Ushakov,~N.\,G., and  V.\,G.~Ushakov.} 2017. 
Statistical analysis of rounded data: Recovering of information lost due to rounding. 
\textit{J.~Korean Stat. Soc.} 46(3):426--437.

\bibitem{9-gor-1}
\Aue{Gorshenin,~A.\,K., and V.\,Yu.~Korolev.} 2016. 
A~noising method for the identification of the stochastic structure of information 
flows. \textit{Comm. Com. Inf. Sc.} 678:279--289.

\bibitem{10-gor-1}
\Aue{Gorshenin,~A., and V.~Korolev.} 2013.  Modelling of statistical fluctuations of
information flows by mixtures of gamma distributions. 
\textit{27th European Conference on Modelling and Simulation Proceedings}. 
Dudweiler, Germany: Digitaldruck Pirrot GmbHP. 569--572.

\end{thebibliography}

 }
 }

\end{multicols}

\vspace*{-6pt}

\hfill{\small\textit{Received August 3, 2018}}

%\pagebreak

%\vspace*{-18pt}

\Contrl

\noindent
\textbf{Gorshenin Andrey K.} (b.\ 1986)~--- Candidate of Science (PhD) in physics and
mathematics, associate professor, leading scientist, Institute of Informatics Problems,
Federal Research Center ``Computer Science and Control'' of the Russian Academy of
Sciences, 44-2 Vavilov Str., Moscow 119333, Russian Federation; 
\mbox{agorshenin@frccsc.ru}
\label{end\stat}

\renewcommand{\bibname}{\protect\rm Литература}         %10
\def\stat{zats-chu}



\def\tit{ОБ ЭРГОНОМИЧЕСКИХ ЗАВИСИМОСТЯХ
МЕЖДУ~ПАРАМЕТРАМИ СИТУАЦИОННОГО ЗАЛА С~ИСПОЛЬЗОВАНИЕМ~ИЗОГНУТОГО КОЛЛЕКТИВНОГО ЭКРАНА}



\def\titkol{Об эргономических зависимостях между
параметрами ситуационного зала с использованием
изогнутого %коллективного
экрана}

\def\aut{А.\,А.~Зацаринный$^1$, К.\,Г.~Чупраков$^2$}

\def\autkol{А.\,А.~Зацаринный, К.\,Г.~Чупраков}

\titel{\tit}{\aut}{\autkol}{\titkol}

%{\renewcommand{\thefootnote}{\fnsymbol{footnote}} \footnotetext[1]
%{Работа выполнена при финансовой поддержке РНФ (проект 14-11-00364).}}


\renewcommand{\thefootnote}{\arabic{footnote}}
\footnotetext[1]{Институт проблем информатики Российской академии наук, azatsarinny@ipiran.ru}
\footnotetext[2]{Институт проблем информатики Российской академии наук, chkos@rambler.ru}

\vspace*{-6pt}

\Abst{Рассмотрен подход к определению зависимостей между параметрами
ситуационного зала: размерами помещения, числом наблюдателей, информационной
емкостью контента (количеством знаков) и~шириной экрана. Эти зависимости позволяют
рассчитать неизвестный параметр ситуационного зала при известных других с~выполнением
требований государственных и~международных стандартов по эргономике рабочих мест.
Предложенные формулы применимы и~для изогнутых экранов, определяемых в~рамках
статьи углом кривизны~$\beta$ (для плоского экрана $\beta\hm=0$). Данный параметр может
быть интерпретирован как угол наклона между дисплеями в~полиэкране. Наличие этого
параметра позволяет оценить эффективность использования изогнутых экранов в~составе
систем отображения информации коллективного использования. Предложен общий подход
к~определению количества рабочих мест для коллективного экрана, который может быть
применен для их различных взаимных расположений.}

\KW{изогнутый экран коллективного пользования; ситуационный зал; диспетчерский пункт;
эргономические зависимости; область комфортного наблюдения; угол кривизны экрана;
видеостена; полиэкран; эффективность; оправданность цены}

\DOI{10.14357/19922264140411}




\vskip 10pt plus 9pt minus 6pt

\thispagestyle{headings}

\begin{multicols}{2}

\label{st\stat}

\section{Введение}
	
     Опыт создания ситуационных центров (СЦ) обозначил целый ряд проблем,
которые препятствуют разработке и~внедрению новых технологий в~процесс
управления~[1--4]. Эти трудности могут возникать в~результате
недостаточного применения системного подхода при проектировании, когда
имеющийся функционал прикладных средств не подкреплен достаточной
технологической или технической базой и~наоборот. Значительное число
проблем создания и~внедрения СЦ вызвано человеческим фактором: плохой
мотивацией персонала на обучение новым процессам и~слабой
заинтересованностью первого лица~[5, 6].

     Важно понимать, что любая проблема, будь то техническая или
организационная, может стать узким местом в~обеспечении функционирования
СЦ в~требуемых режимах~\cite{3-zat, 7-zat}. Для
выявления таких узких мест необходимы системные оценки эффективности
СЦ, которые должны учитывать оценки всех его отдельных
компонентов~\cite{8-zat}. При этом в~рамках технического
и~эксплуатационного компонентов важную роль играют эргономические
показатели, так как они определяют эффективность пользовательского
интерфейса в~СЦ. Эргономические требования к системам
отображения информации являются важными с~точки зрения не только
обеспечения комфортных условий работы, но и~более эффективного
использования пространственного и~материального ресурса~[9, 10] .

     Одним из наиболее заметных направлений в~развитии средств
отображения информации является применение так называемых изогнутых
экранов (curved screen), реализуемых на основе технологии OLED
(organic light-emitted diode), широко
применяемой для мобильных устройств. Отметим, что идея таких
<<изогнутых>> или вообще неплоских экранов не нова: помимо кинотеатров
неплоские экраны можно встретить и~в диспетчерских пунктах. Производители
профессиональных полиэкранов и~видеостен предусматривают возможность
создания изогнутых экранов за счет взаимного поворота между отдельными
дисплеями. Вместе с~тем ощутимых преимуществ, которые дают изогнутые
экраны, производители и~их маркетологи сформулировать не смогли~[11].

     В настоящей статье сделана попытка восполнить этот пробел.
Рассмотрены методические подходы к~определению зависимостей между
размерами помещения, разме\-рами экрана, числом наблюда\-телей
и~характеристиками отображаемого контента, сформулированные
в~работах~[9, 10], применительно к~изогнутому экрану.
Полученные зависимости позволяют получать оценки эффективности
применения изогнутых экранов в~сис\-те\-мах коллективного отображения
информации СЦ. Более того, в~отличие от известного
оценочного подхода по уже реализованным комплексам в~[12] предлагаемый
в~статье подход предусматривает \mbox{построение} сис\-те\-мы
     <<по\-ме\-ще\-ние--экран--наблю\-да\-те\-ли>> в~строгом соответствии
с~нор\-ма\-тив\-но-тех\-ни\-че\-ской базой, определяемой существующими
государственными и~международными стандартами в~части эргономических
норм.

\section{Общий подход. Термины и~определения}

	Создание и~оборудование ситуационного зала должно среди прочих
требований опираться на существующие стандарты по эргономике,
действующие на территории РФ. Большинство указаний, содержащихся
в~стандартах, основано на особенностях человеческого восприятия и~потому
может служить практическим руководством при оснащении помещений. Это
относится и~к~средствам отображения информации. Подход,
сформулированный в~[9, 10] и~используемый в~данной статье, опирается на
стандарты~[13--19].

    О существовании связей между основными параметрами системы
    <<по\-ме\-ще\-ние--дис\-плей--на\-блю\-да\-те\-ли>> можно судить на
основании сле\-ду\-юще\-го примера. Увеличение информационной емкости
контента за счет уменьшения символов незамедлительно приведет
к~уменьшению проектного расстояния наблюдения. Это, в~свою очередь,
уменьшит рабочую площадь наблюдения, а~следовательно, и~количество
персонала, который может работать в~нем одновременно.

	На основании рекомендаций, сформулированных в~[13--19]
и~выделенных в~[9, 10], можно при\-ступить к формированию взаимосвязей
между основны\-ми параметрами системы
<<по\-ме\-ще\-ние--экран--на\-блю\-да\-те\-ли>>. Далее в~работе будут
использованы следующие термины и~обозначения:
\begin{description}
\item[\,]
$D$~--- проектное расстояние наблюдения, измеряется в~метрах. Это
расстояние или диапазон расстояний между экраном и~глазами наблюдателей,
при котором изображение соответствует требованиям разборчивости
и~удобочита\-емости;
\item[\,]
$N$~--- количество людей, которые должны одновременно работать
с~коллективным экраном, получая с~него визуальную информацию
в~комфортных условиях;
\item[\,]
$Q$~--- диаметр помещения, ограниченного сте\-нами. В~большинстве случаев
помещение является прямоугольным, а~в~рамках статьи, обобщенно,~---
выпуклым. Параметр~$Q$~--- максимальное из расстояний между двумя произвольными
точками помещения, измеряется в~метрах.\linebreak В~некоторых задачах $Q$ может
быть ограничено искусственным образом ввиду особенностей геометрии
помещения, группировки наблюдателей. Отметим разницу между проектным
расстояни\-ем наблюдения~$D$ и~диаметром помещения~$Q$. Первый
параметр характеризуется свойствами системы <<экран--на\-блю\-да\-те\-ли>>,
а~второй~--- исключительно свойствами помещения. Ясно, что~$D$ не
может превосходить~$Q$;
\item[\,]
$W$~--- максимальное расстояние между двумя точками экрана
в~горизонтальной плоскости, <<плоская>> ширина экрана. Для плоского
экрана~$W$~соответствует его ширине, а~для кривого~--- расстоянию между
его краями. Измеряется в~мет\-рах;
\item[\,]
$I$~--- необходимая статичная информационная емкость отображаемого
контента~--- максимальное количество знаков или символов, которые должен
отобразить дисплей в~одном кадре или неподвижном изображении.
Определяется на основании задач ситуационного зала и~иных приложений
и~объемов отображаемого контента. Измеряется в~количестве отображаемых
знаков;
\item[\,]
$\alpha$~--- максимальный стягиваемый угол (угловой размер экрана) по
горизонтали, ограничиваемый уровнем концентрации наблюдения;
\item[\,]
\textit{изогнутый экран}~--- экран, сечение которого в~горизонтальной
плоскости является дугой некоторой окружности. Также под изогнутым
экраном будем понимать полиэкран, \mbox{составленный} из плоских дисплеев
одинакового размера и~расположенных друг к~другу под одним углом
(дискретное приближение к~окруж\-ности). В~рамках данной работы
в~вертикальной плоскости экран считается плоским;
\begin{figure*}[b] %fig1
\vspace*{1pt}
 \begin{center}
 \mbox{%
 \epsfxsize=91.03mm
 \epsfbox{zac-1.eps}
 }
 \end{center}
 \vspace*{-9pt}
\Caption{Область комфортного наблюдения экрана}
\end{figure*}
\item[\,]
$\beta$~--- угол, определяющий кривизну (изогнутость) экрана. Определяется
как половина дуги, которую стягивает экран на соответствующей ему
окружности. Такое определение эквивалентно углу наклона между отдельными
дисплеями в~случае полиэкрана из плоских дисплеев. Положительные
значения~$\beta$ соответствуют случаю, когда наблюдатели расположены
в~той же части пространства, что и~центр окружности, на которую ложится
горизонтальное сечение экрана, или экран, <<выпуклый от наблюдателя>>.
Отрицательные значения~$\beta$, соответственно,~--- когда экран <<выпуклый
к~наблюдателю>>. Если $\beta\hm=0$, то экран плоский;
\item[\,]
{ОКН}~--- область комфортного наблюдения~--- область пространства, где
выполнены рекомендации международных и~государственных стандартов по
оборудованию рабочих мест наблюдения с~коллективного экрана.
\end{description}

\section{Построение области комфортного наблюдения и~исследование ее свойств}


    По аналогии с~результатами~[9, 10] c~учетом ограничений~[13--19]
ОКН в~случае изогнутого экрана будет также
являться фигурой пересечения нескольких областей (рис.~1).



     Рассмотрим такую систему координат, ось абсцисс которой проходит
через крайние точки экрана, а~ось ординат является осью симметрии для
экрана и~направлена в~сторону наблюдателей. Единица измерения по обеим
осям равна 1~м. С~по\-мощью средств аналитической геометрии вы\-чис\-лим
координаты основных точек, которые будут участвовать в~дальнейших
оценках:
\begin{description}
\item[\,]     $O_1$~--- центр левого круга:
     $
     O_1\left( ({D}/{2})\sin\beta - W/2;\right.$ $\left.\left(D/2\right)\cos\beta\right);
     $
\item[\,]
      $O_2$~--- центр правого круга:
     $O_2\left( {W}/{2}-({D}/{2})\sin\beta;\right.$ $\left.({D}/{2})\cos\beta\right);$

\item[\,] $l_1$~--- прямая, выходящая из левого края экрана под углом~$\alpha$
к~$OY$;
\item[\,]
$l_2$~--- прямая, выходящая из правого края экрана под углом~$\alpha$
к~$OY$;
\item[\,]
$K$~--- левый край экрана: $K\left( -{W}/{2};\,0\right);$
\item[\,]
$L$~--- правый край экрана: $L\left( {W}/{2};\,0\right);$
\item[\,]
$M$~--- точка пересечения прямых~$l_1$ и~$l_2$:

\noindent
$$
M\left( 0;({W}/{2})\ctg(\alpha\hm+\beta)\right)\,;
$$

\vspace*{-4pt}
\item[\,]
$A$~--- верхняя точка пересечения кругов:


\noindent
$$
A\left( 0;({D}/{2})\left(\cos\beta\hm+\sqrt{1-(C-\sin\beta)^2}\,\right)\right)\,;
$$

\vspace*{-4pt}

\item[\,]
$B$~--- нижняя точка пересечения кругов:

\vspace*{-2pt}

\noindent
$$
B\left( 0;({D}/{2})\left( \cos\beta \hm-\sqrt{1-(C-\sin\beta)^2}\,\right)\right)\,;
$$

\vspace*{-2pt}

\item[\,]
$F$~--- точка пересечения прямой~$l_2$ и~правого круга, отличная от~$L$:
$F\left( -({D}/{2})\times\right.$\linebreak $\left.\times\left(2\sin(\alpha\hm+\beta\right)\cos\alpha\hm-
C);\,D\cos(\alpha\hm+\beta)\cos\alpha\right);$
\item[\,]
$E$~--- точка пересечения прямой $l_1$ и~левого круга, отличная от~$K$:
$E\left( ({D}/{2})\times\right.$ $\left.\times\left(2\sin(\alpha\hm+\beta)\cos\alpha\hm -
C\right);\,D\cos(\alpha\hm+\beta)\cos\alpha\right),$
где $C$~---
отношение <<плоской>> ширины экрана к~проектному расстоянию
наблюдения:

\vspace*{-2pt}

\noindent
\begin{equation}
C=\fr{W}{D}\,.
\label{e1-zat}
\end{equation}

\vspace*{-2pt}

\noindent
Как будет показано далее, оно зависит от информационной
емкости контента~$I$, отношения~$k$ высоты экрана к ширине,
отношения~$p$ ширины знака к его высоте и~угла~$\psi$, стя\-ги\-ва\-емо\-го одним
символом.
\end{description}

     Криволинейный четырехугольник $AFME$ и~есть ОКН, параметры
которой позволят оценить количество рабочих мест, которые можно
расположить в~ней, а~так\-же ширину экрана, которая при этом потребуется.
Согласно методике, сформулированной в~[9, 10], для оценки количества
рабочих мест и~ширины экрана необходимо рассчитать площадь ОКН и~ее
периметр, но сначала необходимо определить условия, при которых эта область
будет непустой и~будет приобретать различные формы. Четырехугольник
будет вырожденным, если круги с~центрами~$O_1$ и~$O_2$ не будут
пересекаться, т.\,е.\ точки~$A$ и~$B$ не будут существовать. Это условие
эквивалентно неравенству
     \begin{equation}
     1-\left (C-\sin \beta\right)^2<0\,.
     \label{e2-zat}
     \end{equation}


Далее полагаем, что это неравенство выполнено.

     Рассмотрим взаимное расположение точек~$A$, $B$ и~$M$. Возможны
три принципиально разных случая:
     \begin{enumerate}[1.]
\item {\bfseries\textit{Точка~{\boldmath{$M$}} находится выше точки~{\boldmath{$A$}}.}} Это условие
эквивалентно системе неравенств:

\vspace*{1pt}

\noindent
\begin{equation}\left.
\begin{array}{rl}
C&> \cos\beta \tg(\alpha+\beta) =R_1\,;\\[6pt]
C&> 2\sin(\alpha+\beta)\cos\alpha =R_2\,.
\end{array}
\right\}
\label{3-zat}
\end{equation}

\vspace*{-2pt}

\noindent
В этом случае решений нет~--- ОКН вырождена.

\item
{\bfseries\textit{Точка~{\boldmath{$M$}} принадлежит отрезку~{\boldmath{$[A, B]$}}}}.
Это условие
эквивалентно неравенству:
\begin{equation}
C\leq R_2\,.
\label{e4-zat}
\end{equation}
В этом случае ОКН ограничена дугами $\overset{\frown}{AE}$, $\overset{\frown}{AF}$
и~отрезками MF, ME (основной случай).

\item
{\bfseries\textit{Точка {\boldmath{$M$}} находится ниже точки~{\boldmath{$B$}}.}} Это условие
эквивалентно системе:
\begin{equation}
\left.
\begin{array}{rl}
C&> R_2\,;\\[6pt]
C&< R_1\,.
\end{array}
\right\}
\label{e5-zat}
\end{equation}
В этом случае ОКН ограничена дугами $\overset{\frown}{AB}$, принадлежащими левому
и~правому кругу. Решение этой системы существует, когда $R_1\hm>R_2$ или
$2\alpha \hm+\beta \hm> \pi/2$.
\end{enumerate}

     В случае~2 площадь и~периметр криволинейного четырехугольника
могут быть рассчитаны по формулам:

\noindent
     \begin{align}
     S_{\mathrm{ОКН}} &= \fr{D^2}{4}\left[ \left(
     \vphantom{\sqrt{1-(C-\sin\beta)^2}}
     \cos\beta -C\ctg(\alpha+\beta)
+{}\right.\right. \notag
\\
&\hspace*{-7mm}\left.{}+\sqrt{1-(C-\sin\beta)^2} \,\right)
(2\sin (\alpha+\beta)\cos\alpha -C)+ {}\notag
\\
     &{}+\{2\alpha +\beta -\arcsin (C-\sin\beta) -{}\notag\\
&     \left.{}-\sin (2\alpha+\beta -\arcsin (C-
\sin\beta))\}
     \vphantom{\sqrt{1-(C-\sin\beta)^2}}
     \right]\,;\label{e6-zat}\\
     P_{\mathrm{ОКН}} &= D\left[ \cos\alpha -\fr{C}{\sin(\alpha+\beta)}+ 2\alpha
+\beta -{}\right.\notag\\
&\hspace*{10mm}\left.{}-\arcsin (C-\sin\beta)
    \vphantom{\fr{C}{\sin(\alpha+\beta)}}
     \right]\,.
     \label{e7-zat}
     \end{align}
	
В случае~3 площадь и~периметр ОКН могут быть рассчитаны по формулам:
\begin{align}
S_{\mathrm{ОКН}} &= \fr{D^2}{4}\left[ 2\arcsin \sqrt{1-(C-\sin\beta)^2} - {}\right.\notag\\
&\left.{}-2(C-
\sin\beta)\sqrt{1-(C-\sin\beta)^2}\,\right]\,;\label{e8-zat}\\
P_{\mathrm{ОКН}} &= 2D \arcsin \sqrt{1-(C-\sin\beta)^2}\,.\label{e9-zat}
\end{align}

\smallskip

\noindent
\textbf{Замечание~1.} Данные формулы выполнимы при любом взаимном
расположении окружностей, $OY$ и~прямых~$l_1$ и~$l_2$ в~рамках
ограничений, заданных случаями.
\smallskip

\noindent
\textbf{Замечание~2.} Формулы~(2)--(9) действительны и~для
отрицательных~$\beta$, т.\,е.\ случаев, когда экран выпуклый к наблюдателям.

\smallskip

\noindent
\textbf{Замечание~3.} В случае~2 при $\beta\hm=0$ (экран плоский)
формулы~(\ref{e6-zat}) и~(\ref{e7-zat}) приобретают вид результатов,
полученных в~[9, 10], но в~отличие от них являются точными, так как не
пренебрегают малыми сла\-га\-емы\-ми, которые для неплоского случая могут
стать существенными.

\subsection{Количество рабочих мест в~области комфортного наблюдения}

    Расчет количества рабочих мест, которые могут быть размещены
в~области комфортного наблюдения, должен опираться на способ их
размещения. Например, для построения ситуационного зала, где в~центре
будет находиться овальный стол, расчет будет заключаться в~определении
максимального размера такого стола при заданных ограничениях по
характеристикам экрана и~информационной емкости контента, отображаемого
на нем.
%
В~[9, 10] рассмотрен способ рав\-но\-мер\-но-плот\-но\-го
распределения рабочих мест~--- <<сеточный>>. Согласно методике,
предложенной в~[9, 10], количество рабочих мест в~области комфортного
наблюдения может быть оценено сверху следующей величиной:
    \begin{equation}
    N=B+\Gamma =\fr{\sqrt{2}}{1{,}8^2}\,S+\fr{P}{3{,}6} +1\,.
    \label{e10-zat}
    \end{equation}

    \begin{figure*}[b] %fig2
\vspace*{1pt}
 \begin{center}
 \mbox{%
 \epsfxsize=116.938mm
 \epsfbox{zac-2.eps}
 }
 \end{center}
 \vspace*{-9pt}
\Caption{Зависимость количества рабочих мест от углов~$\alpha$ и~$\beta$ при $I\hm=4000$
и~$Q\hm=10$}
\end{figure*}

     Физический смысл единицы, входящей одним из слагаемых в~эту
формулу, в~том, что даже если ОКН вырождена в~точку и~ее площадь
и~периметр равны нулю, то все равно в~эту точку можно посадить хотя бы
одного наблюдателя. Также обратим внимание на то, что используемая
в~получении этой оценки формула Пика по своей сути смешивает
размерности~--- выводит безразмерную величину из м$^2$ и~м.
{\looseness=-1

}

     Формула~(\ref{e10-zat}) дает оценку сверху для случая расположения
рабочих мест равномерно плотно~--- в~вершинах треугольной сетки. Для
других случаев расположения рабочих мест необходимо определить другую
подходящую функцию
$$
N= N(S_{\mathrm{ОКН}}, P_{\mathrm{ОКН}})\,.
$$
Наличие такой функции для других случаев расстановки позволит использовать
оценки~(\ref{e6-zat})--(\ref{e9-zat}), предложенные в~данной статье.

\vspace*{-6pt}

\subsection{Оценка максимальной площади области комфортного
наблюдения и~максимального количества рабочих~мест в~этой области}

    Вследствие ограниченности помещения проектное расстояние~$D$ не
может превышать диаметра помещения~$Q$, поэтому ввиду оценки
    \begin{equation}
    C\leq \fr{\sqrt{I}}{193}
    \label{e11-zat}
    \end{equation}
(см.\ формулу~(21) в~[9]) для случая~2 формулы~(\ref{e6-zat}) и~(\ref{e7-zat})
могут быть преобразованы следующим образом:\\[-17pt]
\begin{multline}
S_{\mathrm{ОКН}} = \fr{Q^2}{4}\left[ \left(
\vphantom{\sqrt{1-\left( \fr{\sqrt{I}}{193}-\sin\beta\right)^2}}
\cos\beta -\fr{\sqrt{I}}{193}\ctg
(\alpha +\beta) +{}\right.\right.
\\[-1pt]
\hspace*{-5mm}\left.{}+\sqrt{1-\left( \fr{\sqrt{I}}{193}-\sin\beta\right)^2}\,\right)
\times{}
\\[-3pt]
\left.\hspace*{-12mm}{}\times
\left(2\sin(\alpha+\beta)\cos\alpha - \fr{\sqrt{I}}{193}\right) +
\{ \theta -\sin(\theta)\}
\vphantom{\sqrt{1-\left( \fr{\sqrt{I}}{193}-\sin\beta\right)^2}}
\right]\,;\label{e12-zat}
\end{multline}
\begin{equation*}
P_{\mathrm{ОКН}} = Q\left[ \cos\alpha -\fr{\sqrt{I}}{193}\sin
(\alpha+\beta)+\theta
\right]\,.
%\label{e13-zat}
\end{equation*}
Здесь\\[-15pt]
$$
\theta=2\alpha +\beta -\arcsin \left( \fr{\sqrt{I}}{193}-\sin\beta\right)\,.
$$

\vspace*{-3pt}



     Для случая~3 формулы для площади~(\ref{e8-zat})
     и~периметра~(\ref{e9-zat}) приобретут соответственно вид:
\begin{align*}
S_{\mathrm{ОКН}} &= \fr{Q^2}{4}\left[ 2\arcsin \sqrt{1-\left( \fr{\sqrt{I}}{193}-
\sin\beta\right)^2} - {}\right.\notag\\[-3pt]
&\hspace*{-10mm}\left.{}-2\left( \fr{\sqrt{I}}{193} -\sin\beta\right) \sqrt{1-\left(
\fr{\sqrt{I}}{193}-\sin\beta\right)^2}\,\right]\,; \notag%\label{e14-zat}
\\[-3pt]
P_{\mathrm{ОКН}} &= 2Q\arcsin \sqrt{1-\left( \fr{\sqrt{I}}{193}-
\sin\beta\right)^2}\,.
%\label{e15-zat}
\end{align*}

     Далее количество рабочих мест в~обоих случаях может быть посчитано
по формуле~(\ref{e10-zat}).

\begin{figure*}[b] %fig3
\vspace*{-6pt}
 \begin{center}
 \mbox{%
 \epsfxsize=110.869mm
 \epsfbox{zac-3.eps}
 }
 \end{center}
 \vspace*{-9pt}
\Caption{График <<плоской>> ширины экрана при $I\hm=4000$, $N = 10$ от углов~$\alpha$
и~$\beta$}
\end{figure*}

     Всплески, отображенные на рис.~2, происходят в~области, где
выполнены условия для случая~1, т.\,е.\ ОКН является вырожденной, поэтому
эти всплески не представляют интереса для исследований. Сами
невырожденные случаи~2 и~3 соответствуют гладким областям графика. Кроме
того, график показывает, что применение кривых экранов действительно может
быть эффективным для увеличения площади и~периметра ОКН, а~значит,
и~количества рабочих мест в~ней. Например, для случая $\alpha\hm = 45^\circ$
прирост количества рабочих мест при увеличении угла~$\beta$ с~0$^\circ$
до~15$^\circ$ при информационной емкости контента 9000~знаков может
достигать 40\%.


	
\subsection{Оценка минимальной ширины активной поверхности дисплея}

     Очевидно, что чем меньше размеры экрана, тем при прочих равных
условиях меньше его стоимость, но при этом снижаются и~функциональные
возможности экрана. Поэтому необходимо найти такие минимальные размеры
экрана, при которых обеспечивается достаточность решения требуемого
перечня функциональных задач.

     Пусть известна информационная емкость контента~$I$. Рассмотрим два
принципиально разных случая:
     \begin{enumerate}[(1)]
\item число наблюдателей неизвестно, необходимо оценить размеры экрана
(его ширину), позволяющие эффективно использовать пространство
помещения;\\[-13pt]
\item число наблюдателей известно, требуется оценить минимальные
размеры коллективного экрана, достаточные для одновременной работы
всех наблюдателей с~выполнением эргономических требований.
\end{enumerate}

\noindent
\textbf{Случай 1}. Из соотношений~(1) и~(\ref{e11-zat}) следует, что
\begin{equation*}
W_{\min} =QC\geq QC_{\min}=\fr{Q\sqrt{I}}{193}\,.
%\label{e16-zat}
\end{equation*}

\smallskip

\noindent
\textbf{Случай 2}. Согласно методике, предложенной в~[9, 10],
и~формулам~(\ref{e11-zat}) и~(\ref{e12-zat}) для системы
неравенств~(\ref{e4-zat}) получаем:

\vspace*{-2pt}

\noindent
\begin{multline*}
W_{\min} =
{}=2C_{\min} \sqrt{S_{\min}}
\left(
\left(
\vphantom{\sqrt{1-\left(\fr{\sqrt{I}}{193} -\sin\beta\right)^2}}
\cos\beta -{}\right.\right.\\
{}-\left(\fr{\sqrt{I}}{193}\right)\ctg
(\alpha+\beta) +\!
\left.\left.\sqrt{1-\left(\fr{\sqrt{I}}{193} -\sin\beta\right)^{\!2}}
\right) \times{}\right.\\[-2pt]
\left.{}\times
\left(2\sin(\alpha+\beta)\cos\alpha -
\fr{\sqrt{I}}{193}\right)+\{\theta -\sin(\theta)\}
\vphantom{\sqrt{1-\left(\fr{\sqrt{I}}{193} -\sin\beta\right)^2}}
\!\right)^{\!-1/2}\!\!\!\! \!\!={}\hspace*{-7pt}
\end{multline*}

\noindent
\begin{multline*}
{}=
\fr{2}{193}\sqrt{1{,}94I(N-2)} \left(
\left(
\vphantom{\sqrt{1-\left(\fr{\sqrt{I}}{193} -\sin\beta\right)^2}}
\cos\beta - \left(\fr{\sqrt{I}}{193}\right)\right.\right.\times{}\\
\left.\left.{}\times\ctg (\alpha+\beta)+
\sqrt{1-\left(\fr{\sqrt{I}}{193}-\sin\beta\right)^2}\,\right) \times{}\right.\\
\left.{}\times
\left(2\sin (\alpha+\beta)\cos\alpha -
\fr{\sqrt{I}}{193}\right)+\{\theta -\sin(\theta)\}
\vphantom{\sqrt{1-\left(\fr{\sqrt{I}}{193} -\sin\beta\right)^2}}
\right)^{-1/2}\!\!.\hspace*{-7.20935pt}
%\label{e17-zat}
\end{multline*}

Для системы неравенств~(\ref{e5-zat}) минимальная ширина экрана может быть
посчитана по формуле:
\begin{multline*}
W_{\min} ={}\\
{}= \fr{2}{193}\!\left(\!
1{,}94I(N-2)\!\Big/\!\!
\left( 2\arcsin \sqrt{1-(C-\sin\beta)^2}-{}\right.\right.\\
\left.\left.{}-2(C-\sin\beta ) \sqrt{1-(C-\sin\beta)^2}\right)
\vphantom{\Big/}
\right)^{1/2}\,.
%\label{e18-zat}
\end{multline*}


     Найденная по этим формулам ширина~--- это расстояние между
крайними точками экрана. Таким образом, действительная ширина экрана
с~учетом его кривизны может быть рассчитана умножением на коэффициент
$\beta/\sin\beta$:
     \begin{equation*}
     W_{\mathrm{экр}} = W_{\min} \fr{\beta}{\sin\beta}\,.
%     \label{e19-zat}
     \end{equation*}
Здесь при $\beta\hm=0$ используем предел
$$
\lim\limits_{\beta\to 0} \left(\fr{\beta}{\sin\beta}\right)=1\,,
$$
т.\,е.\ функция ширины экрана непрерывна и~в~окрестности
точки $\beta\hm=0$.



     На рис.~3 показан график зависимости ширины экрана от углов~$\alpha$
и~$\beta$. Основные потребности в~большой ширине экрана начинаются в~тот
момент, когда экран становится выпуклым в~сторону наблюдателей,
а~в~остальном для 10~наблюдателей понадобится экран шириной около
     1,5--2~м в~зависимости от угла наблюдения~$\alpha$.


\vspace*{-6pt}

\section{Заключение}

     Полученные в~статье результаты являются развитием методических
подходов, предложенных в~[9, 10], применительно к~изогнутому
экрану за счет введения нового параметра~--- угла кривизны~$\beta$, который
для случая полиэкрана представляет собой угол между двумя его дисплеями.
Важно, что случай $\beta\hm=0$ полностью отвечает результатам, полученным
в~литературе для плоских экранов.

     Предложен общий подход к~расчету максимального количества рабочих
мест на основании формулы , формализация которой для разных расположений
может позволить использовать полученные в~данной статье готовые формулы.

     Показано, что использование изогнутых экранов в~качестве
коллективных может увеличить количество рабочих мест с~соблюдением
эргономических требований за счет увеличения площади и~пери\-мет\-ра ОКН.

     Использование изогнутых экранов может быть особенно эффективным
в~условиях жестких требований к~ситуационному залу (например, маленький
допустимый угол наблюдения, большая информационная емкость контента или
большое количество наблюдателей). Именно в~таких случаях эффект от
применения изогнутых экранов (увеличение числа рабочих мест) может
оказаться наибольшим за счет расположения дисплеев полиэкрана под
определенным углом друг к~другу.

     Полученные зависимости позволяют оценить максимальную
дополнительную стоимость изогнутого экрана относительно плоского такой же
ширины при реализации проектов с~коллективным экраном.

\vspace*{-6pt}

{\small\frenchspacing
 {%\baselineskip=10.8pt
 \addcontentsline{toc}{section}{References}
 \begin{thebibliography}{99}
\bibitem{2-zat} %1
\Au{Зацаринный А.\,А., Сучков А.\,В., Босов~А.\,В.} Ситуационные центры в~современных
ин\-фор\-ма\-ци\-он\-но-те\-леком\-му\-ни\-ка\-ци\-он\-ных сис\-те\-мах специального
назначения~// Ведомственные корпоративные сети и~систе\-мы (ВКСС Connect!), 2007.
№\,5(44). С.~64--76.

\bibitem{4-zat} %2
\Au{Зацаринный А.\,А. Шабанов А.\,П.} Исследование и~разработка методического
обеспечения и~технологических решений по управлению производительностью
конт\-роль\-но-тех\-но\-ло\-ги\-че\-ских трактов~// Информационные технологии в~науке,
социологии, экономике и~бизнесе (IT\;+\;S\&E'10): Мат-лы XXXVII Междунар. конф.~//
Открытое образование, 2010. №\,6. Приложение. С.~44--45.

\bibitem{3-zat}
\Au{Зацаринный А.\,А., Шабанов А.\,П.} Ситуационные центры:
ин\-фор\-ма\-ция--про\-цес\-сы--ор\-га\-ни\-за\-ция~// Электросвязь, 2011. №\,6. С.~42--46.

\bibitem{1-zat} %4
\Au{Зацаринный А.\,А., Козлов С.\,В., Сучков~А.\,П.} Особенности проектирования
и~функционирования ситуационных цент\-ров~// Системы высокой доступности, 2012.
Т.~8. №\,1. С.~12--21.



\bibitem{5-zat} %5
\Au{Зацаринный А.\,А.} Организационные принципы сис\-тем\-но\-го подхода к разработке,
проектированию и~внедрению современных
ин\-фор\-ма\-ци\-он\-но-те\-ле\-ком\-му\-ни\-ка\-ци\-он\-ных сетей~// Ведомственные
корпоративные сети и~системы (ВКСС Connect!), 2007. №\,1(40). С.~60--67.
\bibitem{6-zat}
\Au{Зацаринный А.\,А., Шабанов~А.\,П.} Эффективность ситуационных центров
и~человеческий фактор~// Вестник Московского ун-та имени С.\,Ю.~Витте. Сер.~1:
Экономика и~управление, 2013. №\,3. С.~43--53.
\bibitem{7-zat}
\Au{Зацаринный А.\,А., Ионенков Ю.\,С., Шабанов~А.\,П.} Методиче\-ский подход к оценке
эффективности ситуационных центров~// Фундаментальные и~прикладные
исследования, разработка и~применение высоких технологий в~промышленности
и~экономике: Сб. статей 15-й Междунар. науч.-практич. конф.~--- СПб.: СПбГТУ,
2013. Т.~2. С.~37--39.
\bibitem{8-zat}
\Au{Зацаринный А.\,А., Шабанов А.\,П.} Системные аспекты эффективности
ситуационных центров~// Вестник Московского ун-та имени С.\,Ю.~Витте. Сер.~1:
Экономика и~управление, 2013. №\,2. С.~110--123.
\bibitem{9-zat}
\Au{Чупраков К.\,Г.} Исследование и~разработка методов построения систем
отображения информации для ситуационного центра. Дисс.\ \ldots\ канд. техн. наук.~---
М.: ИПИ РАН, 2010. 214~с.
\bibitem{10-zat}
\Au{Чупраков К.\,Г.} К~вопросу о~размещении коллективных средств отображения
в~ситуационном зале с~заданными параметрами~// Информатика и~её применения, 2010.
Т.~4. Вып.~4. С.~89--96.
\bibitem{11-zat}
\Au{Золотов Е.} Кривое ТВ: кому выгодно гнуть телевизор. {\sf
http://www.computerra.ru/100617/curved-tv}.
\bibitem{12-zat}
\Au{Новикова Е.\,В., Переверзев Б.\,Л., Лавренюк~С.\,Ю.} Метод расчета зоны
оптимальной видимости при работе с~экранами коллективного пользования~//
Информационные технологии в~проектировании и~производстве, 2011. №\,3.
С.~104--109.

\bibitem{19-zat} %13
ГОСТ 21958-76. Зал и~кабины операторов. Взаимное расположение рабочих мест.~--- М.:
Изд-во стандартов, 1976. 7~с.
\bibitem{16-zat} %14
ГОСТ 12.2.032-78. Система стандартов безопасности труда. Рабочее место при
выполнении работ сидя. Общие эргономические требования.~--- М.: Изд-во стандартов,
2001. 9~с.
\bibitem{17-zat} %15
ГОСТ 12.2.033-78. Система стандартов безопасности труда. Рабочее место при
выполнении работ стоя. Общие эргономические требования.~---М.: Изд-во стандартов,
2001. 9~с.

\bibitem{14-zat} %16
ГОСТ 26387-84. Система <<Че\-ло\-век--ма\-ши\-на>>. Термины и~определения.~--- М.:
Стандартинформ, 2006. 6~с.
\bibitem{15-zat} %17
ГОСТ 27833-88. Средства отображения информации. Термины и~определения.~--- М.:
Изд-во стандартов, 1988. 11~с.

\bibitem{18-zat} %18
ГОСТ Р ИСО 9241-3-2003. Эргономические требования при выполнении офисных работ
с~использованием видеодисплейных терминалов.~--- М.: Изд-во стандартов, 2003. 39~с.

\bibitem{13-zat} %19
ГОСТ Р 52324-2005. (ИСО 13406-2:2001). Эргономические требования к работе
с~визуальными дисплеями, основанными на плоских панелях.~--- М.: Стандартинформ,
2005. 13~с.

 \end{thebibliography}

 }
 }

\end{multicols}

\vspace*{-6pt}

\hfill{\small\textit{Поступила в редакцию 12.10.14}}

%\newpage

\vspace*{12pt}

\hrule

\vspace*{2pt}

\hrule

%\vspace*{12pt}

\def\tit{REGARDING ERGONOMIC DEPENDENCES BETWEEN~SITUATIONAL~HALL PARAMETERS
USING~COLLECTIVE~CURVED~SCREEN}

\def\titkol{Regarding ergonomic dependences between
situational hall parameters using collective curved
screen}

\def\aut{A.\,A.~Zatsarinnyy and~K.\,G.~Сhuprakov}

\def\autkol{A.\,A.~Zatsarinnyy and~K.\,G.~Сhuprakov}

\titel{\tit}{\aut}{\autkol}{\titkol}

\vspace*{-9pt}

\noindent
Institute of Informatics Problems, Russian Academy of Sciences,
44-2~Vavilov Str., Moscow 119333, Russian Federation


\def\leftfootline{\small{\textbf{\thepage}
\hfill INFORMATIKA I EE PRIMENENIYA~--- INFORMATICS AND
APPLICATIONS\ \ \ 2014\ \ \ volume~8\ \ \ issue\ 4}
}%
 \def\rightfootline{\small{INFORMATIKA I EE PRIMENENIYA~---
INFORMATICS AND APPLICATIONS\ \ \ 2014\ \ \ volume~8\ \ \ issue\ 4
\hfill \textbf{\thepage}}}

\vspace*{3pt}

\Abste{The paper presents an approach to determining dependences between
such parameters of a~situational hall as measurements of the hall,
quantity of people working with the screen,
information capacity of the content (the quantity of symbols),
 and screen width. These dependences make it possible to calculate an unknown
 parameter of a~situational hall using known parameters satisfying requirements
 of the Russian and International ergonomic standards. The presented formulas
 are applicable to the case of curved screens by using the angle of curvature~$\beta$
 (for a flat screen, $\beta=0$). This parameter may be interpreted as an angle
 between displays in a~polyscreen or a~videowall. This parameter makes it possible
 to evaluate the efficiency of curved screens as a collective screen compared
 to the flat screens. The paper also suggests an approach to estimating the
 quantity of workplaces that may be used for their different interpositions.}

\KWE{collective curved screen; situational hall; dispatch room; ergonomic
dependences; comfort observation area; curve angle; videowall, polyscreen; efficiency; price
justification}


\DOI{10.14357/19922264140411}

%\vspace*{3pt}

  \begin{multicols}{2}

\renewcommand{\bibname}{\protect\rmfamily References}
%\renewcommand{\bibname}{\large\protect\rm References}



{\small\frenchspacing
 {%\baselineskip=10.8pt
 \addcontentsline{toc}{section}{References}
 \begin{thebibliography}{99}

\bibitem{2-zat-1} %1
\Aue{Zatsarinnyy, A.\,A., A.\,V.~Suchkov, and A.\,V.~Bosov}. 2007. Situatsionnye tsentry
v~sovremennykh informatsionno-telekommunikatsionnykh sistemakh spetsial'nogo
naznacheniya [Situational centers in modern information-telecommunicational network of
special purposes]. \textit{VKSS Connect! (Vedomstvennye korporativnye seti i~sistemy)}
[VKSS Connect! (Departmental Corporate Networks and Systems] 5(44):64--76.

\bibitem{4-zat-1} %2
\Aue{Zatsarinnyy, A.\,A., and A.\,P.~Shabanov} 2010. Issledovanie i razrabotka metodicheskogo
obespecheniya i tekhnologicheskikh resheniy po upravleniyu proizvoditel'nost'yu
kontrol'no-tekhnologicheskikh traktov [Investigation and development of methodical base and
technologic solutions concerning the
 management of control and technologic tract performance].
\textit{Prilozhenie k zhurnalu ``Otkrytoe obrazovanie.'' Mat-ly XXXVII Mezhdunar. konf.
i~diskussionnogo nauchnogo kluba ``Informatsionnye Tekhnologii v~Nauke, Sotsiologii,
Ekonomike i~Biznese'' IT\;+\;SE'10} [Appendix to magazine ``Open Education.'' 37th
Conference (International) and Discussion Club ``Informational Technologies in Science, Education,
Telecommunications, and Business'' Proceedings].  Yalta. 44--45.
\bibitem{3-zat-1} %3
\Aue{Zatsarinnyy, A.\,A., and A.\,P.~Shabanov}. 2011. Situatsionnye centry:
Informatsiya--protsessy--organizatsiya. [Situational centers:
Information--procecces--organization].
Electrosvyaz' [Telecommunications] 6:42--46.
\bibitem{1-zat-1} %4
\Aue{Zatsarinnyy, A.\,A., S.\,V.~Kozlov, and A.\,P.~Suchkov}. 2012. Osobennosti
proektirovaniya i~funktsionirovaniya situ\-atsi\-on\-nykh tsentrov [Peculiarity of situational centers
design and operation]. \textit{Sistemy Vysokoy Dostupnosti} [Systems of High Accessability].
Moscow: Radiotechnika Publ. 8(1):12--21.

\bibitem{5-zat-1} %5
\Aue{Zatsarinnyy, A.\,A.} 2007. Organizatsionnye printsipy sistemnogo podkhoda k razrabotke,
proektirovaniyu i~vnedreniyu sovremennykh in\-for\-ma\-tsi\-on\-no-te\-le\-kom\-mu\-ni\-ka\-tsi\-on\-nykh setey
[Organizational principles of systemic approach to development, design, and implementation of
modern information-telecommunicational networks]. \textit{VKSS Connect! (Vedomstvennye
korporativnye seti i~sistemy)} [VKSS Connect! (Departmental Corporate Networks and
Systems] 1(40):60--67.

\bibitem{6-zat-1}
\Aue{Zatsarinnyy, A.\,A., and A.\,P.~Shabanov} 2013. Effektivnost' situatsionnykh tsentrov
i~chelovecheskiy faktor [Situational centers efficiency and the human factor]. \textit{Vestnik
Moskovskogo Un-ta imeni S.\,Yu.~Vitte. Ser.~1: Ekonomika i~Upravlenie} [Herald of
S.\,Y.~Vitte Moscow University. Ch.~1: Economics and Management]  3:43--53.

\bibitem{7-zat-1}
\Aue{Zatsarinnyy, A.\,A., Yu.\,S.~Ionenkov, and A.\,P.~Shabanov}. 2013. Metodicheskiy
podkhod k otsenke effektivnosti\linebreak si\-tu\-a\-tsi\-onnykh tsentrov [Methodical approach to situational
centers efficiency evaluation]. \textit{Sb. statey 15-y Mezhdunar. nauch.-praktich. konf.
``Fundamental'nye i~Prikladnye Issledovaniya, Razrabotka i~Primenenie Vysokikh Tekhnologiy
v~Promyshlennosti i~Ekonomike''} [15th Scientific and Practical Conference
(International) ``Fundamental and Applied Investigations, Development
and Applications of Fine Technologies in Industry and Economics Proceedings].
Ed.\ A.\,P.~Kudinova. St.\ Petersburg, Russia:
Polytechnical University Publ. 2(1):37--39.
\bibitem{8-zat-1}
\Aue{Zatsarinnyy, A.\,A., and A.\,P.~Shabanov}. 2013. Sistemnye aspekty effektivnosti
situatsionnykh tsentrov [Systemic peculiarities of situational centers efficiency]. \textit{Vestnik
Moskovskogo Un-ta imeni S.\,Yu.~Vitte} [Herald of~S.\,Yu.~Vitte Moscow University.
Chapter~1: Economics and Management]. 2:110--123.
\bibitem{9-zat-1}
\Aue{Chuprakov, K.\,G.} 2010. Issledovanie i razrabotka metodov postroeniya sistem
otobrazheniya informatsii dlya si\-tu\-a\-tsi\-on\-no\-go tsentra [Investigations and development of
methods of visualization systems creation for situational center]. PhD. Thesis. Moscow. 214~p.
\bibitem{10-zat-1}
\Aue{Chuprakov, K.\,G.} 2010. K~voprosu o razmeshchenii kollektivnykh sredstv
otobrazheniya v~situatsionnom zale s~zadannymi parametrami [On collective
display facilities placed in a~situational hall with prescribed parameters].
\textit{Informatika i ee Primeneniya}~--- \textit{Inform. Appl.} 4(4):89--96.
\bibitem{11-zat-1}
\Aue{Zolotov, E.} 2014. {Krivoe TV: Komu vygodno gnut' televizor} [Curved TV: Who
benefit from bending TV]. Available at: {\sf
http://www.computerra.ru/100617/curved-tv/} (accessed August~27, 2014).
\bibitem{12-zat-1}
\Aue{Novikova, E.\,V., B.\,L.~Pereverzev, and S.\,Yu.~Lavrenyuk}. 2011. Metod rascheta zony
optimal'noy vidimosti pri rabote s~ekranami kollektivnogo pol'zovaniya [A~calculation method
for area of optimal visibility during work with collective screen].
\textit{Informatsionnye Tekhnologii v~Proektirovanii i~proizvodstve} [Informational
Technologies in Design and Industry] 3:104--109.

\bibitem{19-zat-1} %13
GOST 21958-76. 1976. \textit{Zal i~kabiny operatorov. Vzaimnoe raspolozhenie rabochikh
mest} [Hall and operator's cabins. Mutual disposition of workplaces]. Moscow:
Standard Pubs. 7~p.
\bibitem{16-zat-1} %14
GOST 12.2.032-78. 2001. \textit{Sistema standartov bezopasnosti truda. Rabochee mesto pri
vypolnenii rabot sidya. Obshchie ergonomicheskie trebovaniya} [A~system of work safety
standards. Workplace for sitting operations. Main ergonomic requirements]. Moscow:
Standard Publs. 9~p.
\bibitem{17-zat-1} %15
GOST 12.2.033-78. 2001. \textit{Sistema standartov bezopasnosti truda. Rabochee mesto pri
vypolnenii rabot stoya. Obshchie ergonomicheskie trebovaniya} [A~system of work safety
standards. Workplace for standing operations. Main ergonomic requirements]. Moscow:
Standard Publs. 9~p.

\bibitem{14-zat-1} %16
GOST 26387-84. 2006. \textit{Sistema ``Chelovek--mashina.'' Terminy i~opredeleniya}
[A~system ``human-machine.'' Terms and definitions]. Moscow: StandardInform Publ. 6~p.

\bibitem{15-zat-1} %17
GOST 27833-88. 1990. \textit{Sredstva otobrazheniya informatsii.
Terminy i~opredeleniya}
[Information visualization means. Terms and definitions]. Moscow: Standards Publs. 11~p.
\bibitem{18-zat-1} %18
GOST R ISO 9241-3-2003. 2003. \textit{Ergonomicheskie trebovaniya pri vypolnenii ofisnykh rabot
s~ispol'zovaniem videodispleynykh terminalov} [Ergonomic requirement during office works
operations using videodisplay terminals]. Moscow: Standard Publs. 39~p.


\bibitem{13-zat-1} %19
GOST R 52324-2005 (ISO 13406-2:2001). 2005. \textit{Ergonomicheskie trebovaniya k~rabote
s~vizual'nymi displeyami, osnovannymi na ploskikh panelyakh} [Ergonomic requirements for
work with visual displays based on flat panels]. Moscow: Standardinform Publ. 13~p.

\end{thebibliography}

 }
 }

\end{multicols}

\vspace*{-6pt}

\hfill{\small\textit{Received October 12, 2014}}

\vspace*{-18pt}

\Contr

\noindent
\textbf{Zatsarinnyy Alexander A.}\ (b.\ 1951)~---
Doctor of Science in technology, professor, Deputy Director,
Institute of Informatics Problems, Russian Academy of Sciences,
44-2 Vavilov Str., Moscow 119333, Russian Federation; azatsarinny@ipiran.ru

\vspace*{3pt}

\noindent
\textbf{Сhuprakov Konstantin G.}\ (b.\ 1985)~---
Candidate of Sciences (PhD) in technology, leading mathematician,
Institute of Informatics Problems, Russian Academy of Sciences,
44-2 Vavilov Str., Moscow 119333, Russian Federation;  chkos@rambler.ru
\label{end\stat}

\renewcommand{\bibname}{\protect\rm Литература} %11
\def\stat{vovchenko}



\def\tit{МЕТОДЫ РАЗРЕШЕНИЯ СУЩНОСТЕЙ И~СЛИЯНИЯ ДАННЫХ
В~ETL-ПРОЦЕССЕ И~ИХ РЕАЛИЗАЦИЯ В~СРЕДЕ HADOOP$^*$}



\def\titkol{Методы разрешения сущностей и~слияния данных
в~ETL-процессе и~их реализация в~среде Hadoop}

\def\aut{А.\,Е.~Вовченко$^1$, Л.\,А.~Калиниченко$^2$, Д.\,Ю.~Ковалев$^3$}

\def\autkol{А.\,Е.~Вовченко, Л.\,А.~Калиниченко, Д.\,Ю.~Ковалев}

\titel{\tit}{\aut}{\autkol}{\titkol}

{\renewcommand{\thefootnote}{\fnsymbol{footnote}} \footnotetext[1]
{Работа выполнена при поддержке РФФИ (проекты 13-07-00579, 14-07-00548),
ИПИ РАН (Тема~38.25
<<Спецификация и~решение задач анализа данных в~концептуальных терминах
предметных областей
с~интенсивным использованием данных>> государственного задания ФГБУН ИПИ РАН)
и~Президиума РАН
(Программа фундаментальных исследований Президиума РАН №\,16 <<Фундаментальные
проблемы системного программирования>>).}}


\renewcommand{\thefootnote}{\arabic{footnote}}
\footnotetext[1]{Институт проблем информатики Российской академии наук, alexey.vovchenko@gmail.com}
\footnotetext[2]{Институт проблем информатики Российской академии наук;
Московский государственный университет им.\ М.\,В.~Ломоносова,
факультет вычислительной математики и кибернетики,
leonidk@synth.ipi.ac.ru}
\footnotetext[3]{Институт проблем информатики Российской академии наук, dm.kovalev@gmail.com}

\Abst{При интеграции данных из совокупности исходных коллекций важной задачей
является извлечение сущностей, их трансформация и~загрузка в~интегрированное
хранилище. Такие действия являются частью ETL-про\-цес\-са (extract--transform--loading).
Под сущностью здесь понимается некоторое цифровое представление объекта реального
мира (например, информация о~персонах). При извлечении сущностей возникает проб\-ле\-ма
их разрешения: из различных ресурсов можно извлечь различную информацию об одном
и~том же объекте реального мира. Проб\-ле\-ма разрешения сущностей ориентирована на
решение таких задач, как идентификация сущностей, выявление дубликатов, удаление
дубликатов, установление связей между сущностями, сопоставление сущностей
с~некоторым шаблонным образцом и~др. После разрешения сущностей следует этап их
слияния~--- формирование интегрированных сущностей (содержащих информацию из всех
связанных сущностей). Слияние сущностей является заключительным этапом интеграции
данных. В~работе дан обзор методов разрешения и~слияния сущностей. Рас\-смат\-ри\-ва\-ют\-ся
вопросы адаптации таких методов для применения в~ETL-про\-цес\-се при интеграции
больших данных в~Hadoop. Также рас\-смат\-ри\-ва\-ют\-ся способы программирования методов
разрешения и~слияния сущностей как частей ETL-про\-цес\-са. В~качестве языка
программирования используется HIL (High-Level Integration Language)~--- декларативный
язык, ориентированный на разрешение и~интеграцию сущностей в~Hadoop-инфра\-струк\-туре.}

  \KW{интеграция данных; ETL; разрешение сущностей; слияние сущностей; большие
данные; Hadoop; Jaql; HIL}

\DOI{10.14357/19922264140412}

\vskip 10pt plus 9pt minus 6pt

\thispagestyle{headings}

\begin{multicols}{2}

\label{st\stat}

\section{Введение}


  В течение нескольких последних лет информатика стала играть все возрастающую роль
в~широком наборе научных дисциплин, особенно заметную из-за существенных проблем,
вызванных взрывоподобным ростом данных в~таких науках.\linebreak
 X-информатика образовалась
как совокупность академических дисциплин, направленных на применение средств
информатики для федерализации, организации и~анализа данных в~конкретных областях
науки с~интенсивным использованием данных (НИИД): X\,=\,астро-, био-, гео-, нейро- и~пр.

  Сложность использования данных в~НИИД усугубляется еще и~вследствие естественной
разнородности моделей обрабатываемых данных, в~которых представляют тексты, графы,
структурированную и~слабоструктурированную информацию и~пр. Разнообразие
обрабатываемой информации вызывается, в~частности, не только большим числом
источников поступления обрабатываемой информации, но и~разнообразием объектов
исследования, непрерывным и~быстрым совершенствованием инструментов, вызывающим
адекватные изменения структуры и~содержания накапливаемой информации. Это приводит
к необходимости использования неоднородной, распределенной информации, накопленной
в~течение значительного периода наблюдений технологически различными инструментами.

  Для анализа больших объемов накапливаемых данных используются современные
распределенные инфраструктуры обработки массивных данных (например, Hadoop~[1, 2]).
Основной особенностью подобных инфраструктур является почти линейная горизонтальная
масштабируемость (производительность системы растет линейно относительно числа узлов
кластера), а~также высокая отказоустойчивость (отказ любого узла кластера не должен
влиять на работоспособность системы в~целом).

  Главным достоинством подобных инфраструктур является возможность анализировать
и~обрабатывать разноструктурированные данные, например реляционные, XML, JSON,
NoSQL, текстовые и~др. При этом возникает проблема интеграции информации,
извлекаемой из таких разноструктурированных данных.

  Процесс интеграции данных (рассматриваемый здесь как ETL-про\-цесс) можно
представить состоящим из следующих этапов:
  \begin{itemize}
\item сопоставление схем;
\item интеграция схем;
\item трансформация данных;
\item разрешение сущностей (Entity Resolution~[3--5]);
\item слияние сущностей (Data Fusion~[6]).
\end{itemize}

  В данной работе рассматриваются последние два этапа. При интеграции сырых
разноструктурированных данных задачей ETL-про\-цес\-са является извлечение сущностей
из исходных коллекций, их разрешение, трансформация и~загрузка в~интегрированное
хранилище. Под сущностью здесь понимается некоторое цифровое представление объекта\linebreak
реального мира (например, информация о~персонах). При извлечении сущностей возникает
проб\-ле\-ма их разрешения: из разных ресурсов можно извлечь разную информацию об одном
и~том же\linebreak объекте реального мира. В~общем случае под термином разрешения сущностей
(entity resolution (ER)~[3--5, 7--10]) понимают извлечение информации об одной и~той же
сущности реального мира из разнообразных структурированных, слабоструктурированных
и~неструктурированных коллекций данных и~приведение извлеченных данных
к~унифицированному представлению. При этом применяются методы извлечения,
сопоставления, группирования, связывания устранения дублирования различных
представлений информации. Подходы к~разрешению сущностей рассматриваются
в~разд.~2. Слияние сущностей является заключительным этапом интеграции
данных. Под слиянием сущностей~[6, 11--13] понимается образование интегрированного
представления информации об одной и~той же же сущности реального мира, полученной из разных
источников. Операции и~процедуры, ис\-поль\-зу\-емые при слиянии сущностей, приведены
в~разд.~3.

  В разд.~4 приводится описание средств программирования в~среде Hadoop,
а~также обоснование выбора в~работе средств (языки Jaql и~HIL). Наконец, в~разд.~5
показаны примеры программирования методов разрешения сущностей и~слияния данных
как части ETL-про\-цес\-са в~Hadoop.

\section{Методы разрешения сущностей}

  Этап разрешения сущностей важен для сохранения исходной информации
в~интегрированной коллекции. Кроме того, важно обнаруживать дуб\-ли\-ка\-ты сущностей,
поскольку, например, увеличение числа узлов и~ребер в~сетевых задачах может
существенно удлинять время работы простейших алгоритмов (например, поиска
кратчайшего пути). Алгоритмы разрешения сущностей (включая поиск дубликатов~---
duplicate detection) часто используются и~во всевозможных поисковых системах, таких как
Google, Amazon и~др. Аналогичная проблема встает остро в~различных агрегаторах
информации (например, новостных агрегаторах).

В общем случае процесс разрешения сущностей включает следующие этапы~\cite{8-vov}:
\begin{itemize}
\item подготовка данных;
\item выбор методов сопоставления значений;
\item определение методов разрешения пар сущностей;
\item определение зависимостей (constraints).
\end{itemize}

  \subsection{Методы сопоставления значений}

  Важным действием для успешного разрешения сущностей является подготовка данных,
которая включает нормализацию схем и~нормализацию данных. Нормализация схем~---
непростая задача, подробное ее рассмотрение выходит за рамки данной работы
и~содержится в~работах по интеграции данных в~традиционных архитектурах или
в~инфраструктурах больших данных. Ниже представлен пример списка действий, которые
могут быть отнесены к нормализации схем:
  \begin{itemize}
\item сопоставление атрибутов схем (например, <<контактный телефон>>
и~<<мобильный телефон>>);
\item слияние атрибутов (например, <<полный адрес>> образуется из атрибутов
<<город>>, <<индекс>>, <<улица>>, \ldots);
\item слияние множественных значений и~списков (например, <<контактные телефоны>>
и~<<основной номер телефона>>, <<дополнительный номер телефона>>)
 и~др.
\end{itemize}

  Нормализация данных может включать приведение к строчному или заглавному регистру;
удаление разделителей; поиск и~исправление опечаток; поиск сокращений и~аббревиатур
и~замена их на полные стандартные формы; использование словарей для нормализации
строк и~многое другое. Нормализация данных, так же как и~нормализация схем, не
рассматривается в~данной статье.

  Для сопоставления сущностей важно определиться с~выбором метода оценки сходства
(similarity) значений. Рассматриваются как булевы, так и~вещественные меры сходства.
Следующий список включает примеры часто используемых методов оценки сходства
значений:
  \begin{itemize}
\item эквивалентность булевых предикатов;\\[-13pt]
\item вычисление функции сходства простых значений (расстояние Левенштейна~[14],
алгоритм Сми\-та--Ва\-тер\-ма\-на~[14]);\\[-13pt]
\item вычисление функции сходства множеств (коэффициент Жаккара~[14], коэффициент
сходства Дайса~[14], коэффициент Адара~[15]);\\[-13pt]
\item вычисление функции сходства векторов (коэффициент косинусов~[14],
статистическая мера TFIDF (term frequency\,--\,inverse document frequency)~[14]);\\[-13pt]
\item оценка сходства на основе выравнивания (сходство Джа\-ро--Винк\-ле\-ра~[14],
статистическая мера Soft-TFIDF~[16], расстояние Монг--Эл\-ка\-на~[17]);\\[-13pt]
\item оценка сходства фонетических данных (алгоритм сравнения двух строк по их
звучанию Soundex~[14]);\\[-13pt]
\item оценка сходства, основанная на переводе (может использоваться для нормализации
аббревиатур);\\[-13pt]
\item оценка сходства, основанная на знаниях о~предметной области.
\end{itemize}

  Также существуют специальные методы для определения сходства отношений. Меры,
используемые для отношений, обычно основаны на сходстве множеств и~предполагают
использование функций вычисления сходства множеств.

\vspace*{-3pt}

  \subsection{Методы сопоставления пар сущностей}

  \vspace*{-3pt}

  \subsubsection{Традиционные методы сопоставления пар сущностей}

  Пусть даны две коллекции объектов с~атрибутами: author, venue, paper. Значение
некоторой меры сходства (одной из приведенных в~подразд.~2.1) для конкретного атрибута
будем обозначать

\vspace*{-3pt}

\noindent
\begin{multline*}
\mathrm{XX}\mbox{-}\mathrm{match}\mbox{-}\mathrm{score}
(\mathrm{author}\mbox{-}\mathrm{match}\mbox{-}\mathrm{score},\\
  \mathrm{venue}\mbox{-}\mathrm{match}\mbox{-}\mathrm{score},
  \mathrm{paper}\mbox{-}\mathrm{match}\mbox{-}\mathrm{score})\,.
\end{multline*}

\vspace*{-3pt}

  Традиционным методом сравнения объектов является подсчет сходства некоторым
алгоритмом (см.\ подразд.~2.1) для каждого из атрибутов неза-\linebreak\vspace*{-12pt}

\columnbreak

\noindent
висимо. Затем реализуется подсчет
взвешенной суммы.

  Например:

  \vspace*{-4pt}

  \noindent
  \begin{align*}
 & 0{,}5  \mathrm{autor}\mbox{-}\mathrm{match}\mbox{-}\mathrm{score}+
  0{,}2\mathrm{venue}\mbox{-}\mathrm{match}\mbox{-}\mathrm{score}\\
&  +\;  0{,}3\mathrm{paper}\mbox{-}\mathrm{match}\mbox{-}\mathrm{score}\,.
  \end{align*}

  \vspace*{-4pt}

  Недостатком такого подхода является сложность выбора весов для каждого из атрибутов
и~сложность выбора порога сходства сущностей.

  Другой метод предполагает задание булевого обобщенного правила, где условия
накладываются на каждый атрибут независимо.

  Например:

  \vspace*{-4pt}

  \noindent
  \begin{align*}
&(\mathrm{author\mbox{-}match\mbox{-}score > 0{,}7}\\
&\hspace*{10mm}\mathrm{AND\ venue\mbox{-}match\mbox{-}score >0{,}8)}\\
  &\mathrm{OR\ (paper\mbox{-}match\mbox{-}score > 0{,}9}\\
&\hspace*{10mm}\mathrm{AND\
  venue\mbox{-}match\mbox{-}score>0{,}9)}\,.
  \end{align*}

  \vspace*{-4pt}

  Недостатком этого подхода является сложность формулирования подобных правил
вручную.



  \subsubsection{Методы машинного обучения для~сопоставления пар
сущностей}

  Для сопоставления пар сущностей применяют также специальные методы машинного
обучения, которые позволяют автоматизировать процесс формулирования критериев для
сопоставления сущностей. Использование таких методов основано на применении теории
Фел\-ле\-ги и~Сантера~[14] для связывания сущностей. Рассмотрим этот подход подробнее.

  Пусть даны коллекции $A$ и~$B$.

  Пусть $r$~--- это пара $r(x,y)$, где $x\hm\in {A}$, $y\hm\in {B}$.

  Пусть $\gamma \hm=\gamma(r)$~--- это вектор сравнения, например:
    \begin{multline*}
  \gamma(r) = \{x.\mathrm{author} = y.\mathrm{author},x.\mathrm{venue}=y.\mathrm{venue},\\
  x.\mathrm{paper}=y.\mathrm{paper}\}\,,
\end{multline*}
  $\gamma(r) = \{\mathrm{true}, \mathrm{false}, \mathrm{true}\}$~--- пример, в~случае если
  $x.\mathrm{author}\hm=y.\mathrm{author}$, $x.\mathrm{venue}\hm\not= y.\mathrm{venue}$,
  $x.\mathrm{paper}\hm=y.\mathrm{paper}$.

  Пусть $M$~--- множество всех пар, являющихся дубликатами.

  Пусть $U$~--- множество всех пар, не являющихся дубликатами.

  Тогда правило для определения сходства сущностей можно описать следующей
формулой:
  $$
  R(r)= \fr{m(\gamma)}{u(\gamma)} = \fr{P(\gamma\vert r\in M)}{P(\gamma \vert r\in U)}\,.
  $$


  Определим два порога $T_I$ и~$t_U$, такие что
  \begin{itemize}
  \item $R(r)\leq t_I$~--- объекты (пара) не являются дубликатами;\\[-13pt]
\item $R(r)>t_I\ \mathrm{AND}\ R(r)<t_u$~--- невозможно определить, являются ли объекты (пара)
дубликатами или нет;
\item $R(r)\geq t_u$~--- объекты (пара) являются дубликатами.
\end{itemize}

  Правилом связывания, обозначаемым $L(t_I,t_U)$, называется пара порогов~$t_I$
и~$t_U$.

  При подобном подходе учитываются стандартные для задачи проверки статистических
гипотез ошибки первого и~второго рода. Ошибки первого рода~--- два объекта не являются
дубликатами ($r(x,y)\hm\in U$), однако правило~$L$ относит их к дуб\-ли\-ка\-там. Ошибки
первого рода обозначаются бук\-вой~$\mu$ и~их можно описать формулой:
  $$
  \mu= P(L_{\mathrm{match}}\vert U) =\sum\limits_\gamma u(\gamma) P\left(L_{\mathrm{match}}\vert
\gamma\right)\,.
  $$

  Ошибки второго рода~--- два объекта являются дубликатами ($r(x,y)\hm\in M$), однако
правило~$L$ определяет, что это не дубликаты. Ошибки второго рода означаются
буквой~$\lambda$ и~их можно описать формулой:
  $$
  \lambda= P(L_{\mathrm{nonmatch}}\vert M) =
  \sum\limits_\gamma m(\gamma) P\left(
L_{\mathrm{nonmatch}} \vert \gamma\right).
  $$

  Оптимальным правилом связывания $L^*(t_I^*,t_U^*)$ называется такое правило,
которое соответствует ограничениям на ошибки первого и~второго рода для правила,
  а~так\-же ограничения на неопределенности. Эти ограничения выражаются следующими
формулами:
\begin{itemize}
\item
  ограничения на ошибки:
  $$
  P\left( L^*_{\mathrm{match}}\vert U\right) \leq \mu\,;\qquad
  P\left( L^*_{\mathrm{nonmatch}}\vert M\right) \leq
\lambda\,;
  $$
\item
  ограничения на неопределенности:
  \begin{align*}
  P\left( L^*_{\mathrm{uncertain}}\vert U\right) &\leq P\left( L_{\mathrm{uncertain}} \vert U\right)\,;\\
  P\left( L^*_{\mathrm{uncertain}} \vert M\right)&\leq P\left( L_{\mathrm{uncertain}}\vert M\right)\,.
\end{align*}
\end{itemize}

  Нахождение оптимального правила является основной задачей при использовании теории
Феллеги и~Сантера. Классические (переборные) методы при таком подходе работают
неэффективно, поэтому нахождение оптимального правила достигается с~помощью средств
машинного обучения, например можно использовать наивный байесовский классификатор.
Одна из основных проблем при этом заключается в~том, что для вычисления $P(\gamma \vert
r \hm\in M)$ и~$P(\gamma\vert r \hm\in U)$ необходимы знания о~том, какие объекты
являются дубликатами, а~какие нет (знания о~множествах~$M$ и~$U$).

  Для разрешения сущностей применяются различные реализации подходов, основанных на
алгоритмах машинного обучения и~использовании теории Феллеги и~Сантера, например
использование:
  \begin{itemize}
\item деревьев решений~[18];
\item метода опорных векторов~[19, 20];
\item ансамблей классификаторов~[21];
\item метода условных случайных полей~[22].
\end{itemize}

  К недостаткам этих подходов можно отнести несбалансированность результирующих
классифицированных множеств (так, в~результате образуется значительно больше
непохожих объектов, чем похожих), а~также высокую вероятность того, что объект не будет
причислен ни к какому классу (из-за неопределенности). Но оба этих недостатка могут быть
устранены путем тонкой настройки алгоритмов. Ключевой проблемой при использовании
методов машинного обучения при сравнении пар сущностей является выбор обучающего
множества.

  Выделяют следующие методы классификации сущностей, не требующие построения
обучающей выборки:
  \begin{itemize}
\item обучение без учителя или с~частичным привлечением учителя~\cite{14-vov, 23-vov};
\item методы с~активным обучением;
\item ансамбли классификаторов~\cite{25-vov, 24-vov};
\item доказуемая оптимизация точ\-ности/пол\-но\-ты~\cite{26-vov, 27-vov};
\item краудсорсинг~\cite{28-vov, 29-vov}.
\end{itemize}

  Подводя итог методам разрешения пар сущностей, выделим методы, основанные на мерах
сходства, и~методы, основанные на использовании машинного обучения. Общим
недостатком первой группы методов является сложность формулирования критериев
сходства (подбор весов или явных формул). Методы машинного обучения лишены этого
недостатка в~силу своей структуры, но при этом ключевой проблемой является выбор
обучающего множества, да и~сами методы значительно сложнее. Перспективными (но все
еще мало из\-учен\-ны\-ми) представляются методы машинного обуче\-ния, не требующие
изначального определения обучающего множества, такие как методы, основанные на
активном обучении и~краудсор\-синге.
{\looseness=-1

}

\begin{figure*}[b] %fig1
\vspace*{1pt}
 \begin{center}
 \mbox{%
 \epsfxsize=157.76mm
 \epsfbox{vov-1.eps}
 }
 \end{center}
 \vspace*{-9pt}
\Caption{Различные аспекты проблемы слияния данных}
\end{figure*}


  \subsection{Использование ограничений} %2.3

  После того как определен метод разрешения конкретных пар сущностей, можно
определить зависимости. Далее представлены примеры зависимостей, используемых для
установления сходства сущностей:
  \begin{itemize}
\item транзитивность: если М1 и~М2 похожи и~М2 и~М3 похожи, то и~М1 и~М3 похожи;
\item эксклюзивность: если М1 и~М2 похожи, то М3 не может быть похож на М2;
\item функциональные зависимости: если М1 и~М2 похожи, то М3 и~М4 должны быть
похожи.
\end{itemize}

  Транзитивность часто используется в~методах удаления дубликатов, а~эксклюзивность
используется в~методах установления связей между сущностями.

  В заключение можно отметить, что разрешение сущностей является быстро развиваемой
областью. Исследуются новые меры сходства~\cite{14-vov}, ведутся работы по применению
перспективных методов машинного обучения~[24--29]. Развивается применение
функциональных зависимостей при очистке данных (data cleaning)~[30--32]. Ведутся работы
по построению сущностей с~наиболее представительными данными (включающими данные
из разнообразных дубликатов~--- методы канонизации сущностей~[33]). Также исследуются
методы, в~которых решения по сходству двух сущностей принимаются на основе анализа
совокупности сущностей, применения вероятностных логик сходства, латентной модели
Дирихле~[34--36].

\section{Методы слияния данных}

  Под слиянием данных~\cite{6-vov, 12-vov, 11-vov} понимается образование
интегрированного представления информации об одной и той же сущности реального мира,
полученной из разных источников данных. Процесс слияния данных включает следующие
задачи: слияние записей о~сущностях, разрешение возможных конфликтов, обнаружение
и~удаление ошибочных данных. Методы слияния данных, кратко рассмотренные в~данном
разделе, исследованы в~Потсдамском университете~\cite{11-vov}. Различные аспекты
проблемы слияния данных представлены на рис.~1.


  \subsection{Типы конфликтов при слиянии данных}

  Различают два типа конфликтов: конфликты, вызванные неопределенными значениями,
и~конфликты, вызванные противоречивыми значе\-ниями.

  Неопределенность означает, что в~одном источнике данных содержатся неизвестные
значения (null), а~в~другом~--- известные. Проблема заключается в~том, что семантика
неопределенных значений (null) может сильно отличаться. Различают три варианта:
неизвестные значения, несуществующие значения (например, атрибут <<имя супруга>>
всегда будет null для неженатых), скрытые значения (такие данные, которые по ка\-ким-то
причинам не позволено видеть).

  Противоречивость значений означает появление двух различных ненулевых (not null)
значений. Возможны различные стратегии обработки подобных конфликтов, о~чем
рассказывается в~сле\-ду\-ющем подразделе.



  \subsection{Стратегии разрешения конфликтов}

  Различают следующие подходы к разрешению конфликтов:
  \begin{itemize}
\item игнорирование конфликтов;
\item избегание конфликтов;
\item разрешение конфликтов.
\end{itemize}

  Стратегия игнорирования конфликтов предполагает извлечение всей доступной
информации.\linebreak\vspace*{-12pt}

\pagebreak

\end{multicols}

  \begin{table}\small
  \begin{center}

  \tabcolsep=5pt
  \begin{tabular}{|l|p{34mm}|c|p{40mm}|}
%  \hline
  \multicolumn{4}{c}{Примеры функций для разрешения конфликтов}\\[6pt]
  \hline
  \multicolumn{1}{|c|}{Функция}&\multicolumn{1}{c|}{Описание}&Стратегия&\multicolumn{1}{c|}{Пример
конфликта}\\
  \hline
Min, Max, Sum, Count, Avg&Обычная агрегация&Разрешение конфликта&Подсчет зарплаты
(средней или максимальной), подсчет возраста или количества детей\\
\hline
Random&Случайное значение&Разрешение конфликта&Размер участка\\
\hline
Longest, Shortest&Самое короткое или длинное значение&Разрешение
конфликта&Например, для имен\\
\hline
Choose (source)&Значение из конкретного ресурса&Избежание конфликта&Например, для
финансовых данных, если принято решение доверять информации из Yahoo больше, нежели
другим ресурсам\\
\hline
Choose Depending (val, col)&Выбирается значение в~зависимости от значения в~другом
атрибуте&Избежание конфликта&Например, если выбран атрибут <<город>> из одного
ресурса, то <<почтовый индекс>> разумно взять из того же самого ресурса\\
\hline
Vote&Голосование, решение по большинству&Разрешение конфликта&Например, для
подсчета рейтинга\\
\hline
Coalesce&Выбор первого ненулевого значения&Избежание конфликта&Например, для
имен\\
\hline
Group, Concat&Группировка или конкатенация всех значений&Избежание
конфликта&Например, для отзывов о~продуктах\\
\hline
MostRecent&Выбор наиболее свежего значения (недавно обновленного)&Разрешение
конфликта&Например, если интересует последний адрес местожительства\\
\hline
Escalate&Сохранение всех конфликтующих значений, с~тем чтобы пользователь сам решил,
какое выбрать&Игнорирование конфликта&Например, для атрибута <<пол>> сложно
придумать объективные причины выбора того или иного значения\\
\hline
\multicolumn{1}{|c|}{$\ldots$}&\multicolumn{1}{c|}{$\ldots$}&
\multicolumn{1}{|c|}{$\ldots$}&\multicolumn{1}{c|}{$\ldots$}\\
\hline
\end{tabular}
\end{center}
\end{table}

\begin{multicols}{2}

\noindent
 Например, для строк это может быть обычная конкатенация строк,
а~пользователь уже сам решает, какие данные верны.

  Стратегия избегания конфликтов предполагает выбор данных на основе самих данных (по
некоторому алгоритму) или на основе метаданных. Примером функции на основе данных
может служить функция coalesce (выбор первого ненулевого значения) или функция выбора
самого длинного значения.

Примером функций на основе метаданных может выступать
выбор в~зависимости от самого источника (например, известно, что один из источников
наиболее достоверный). Другим примером является функция, выбирающая значение из того
источника, в~котором большее число значений было выбрано для других атрибутов.



  Стратегии разрешения конфликтов учитывают все значения и~выбирают из них
<<достоверное>>. Примером подобной функции могут выступать всевозможные функции
голосования, функции выбора случайного значения, функции среднего значения, функции
наиболее часто встречающегося значения и~др.

  В~таблице представлены примеры функций для разрешения конфликтов.

  \begin{figure*}[b] %fig2
  \vspace*{1pt}
 \begin{center}
 \mbox{%
 \epsfxsize=120.997mm
 \epsfbox{vov-2.eps}
 }
 \end{center}
 \vspace*{-9pt}
   \Caption{Пример операции \textbf{Minimum Union}
  }
%  \end{figure*}
%    \begin{figure*} %fig3
  \vspace*{1pt}
 \begin{center}
 \mbox{%
 \epsfxsize=120.997mm
 \epsfbox{vov-3.eps}
 }
 \end{center}
 \vspace*{-9pt}
  \Caption{Пример операции \textbf{Complementation Union}}
  \end{figure*}

\vspace*{-6pt}

  \subsection{Основные функции разрешения конфликтов}

  Вводится операция {\sf outer union}~\cite{11-vov}, результатом которой
является объединение двух отношений. Если схемы не совпадают, то результирующая схема
является объединением двух исходных схем. Например, пусть даны два отношения:
A~с~набором атрибутов $\{\mathrm{a, b, c, d}\}$ и~отношение~B с~набором атрибутов
$\{\mathrm{c, d, e, f}\}$. Результирующая схема будет содержать набор
атрибутов\;=\;$\{\mathrm{a, b, c, d, e, f}\}$. В~результирующие кортежи для недостающих
атрибутов помещаются нулевые значения. Эта операция не является стандартной
и~отсутствует в~большинстве реляционных систем управления базами данных (СУБД).
В~реляционной алгебре подобная
операция может быть представлена как

\smallskip

  \noindent
  ({\sf  SELECT a, b, c, d, NULL as e, NULL as f FROM A})

\noindent
  {\sf UNION}

\noindent
  ({\sf SELECT NULL as a, NULL as b, c, d, e, f FROM B}).

  Вводится функция {\sf tuple subsumption}~\cite{11-vov}. Говорят, что
кортеж~t1 поглощает другой кортеж~t2 (поглощаемый кортеж), если у них
  \begin{itemize}
\item совпадают схемы;
\item в t2 больше неизвестных (null) значений, чем в~t1;
\item в t2 все известные значения совпадают со значениями в~t1.
\end{itemize}

  Например, пусть даны кортежи $\mathrm{t1} \hm= (5, \mbox{`text'}, \mbox{null}, 7)$
и~$\mathrm{t2} = (5, \mathrm{null}, \mathrm{null}, 7)$. Видно, что каждый атрибут в~t2 либо
совпадает с~аналогичным атрибутом в~t1, либо он null. Для этого примера кортеж~t1
поглощает кортеж~t2.

  Вводится функция {\sf tuple complementation}~\cite{11-vov}. Говорят, что
кортежи~t1 и~t2 дополняют друг друга, если
  \begin{itemize}
\item у них совпадают схемы;
\item они не совпадают;
\item значения соответствующих атрибутов в~t1 и~t2 совпадают, либо одно из них не
определено, либо оба не определены;
\item t1 и~t2 имеют как минимум один атрибут, значения которого совпадают.
\end{itemize}

  Например, пусть даны кортежи $\mathrm{t1} \hm= (5, \mbox{`text'}, \mathrm{null},
\mathrm{null})$ и~$\mathrm{t2} \hm= (5, \mathrm{null}, \mathrm{null}, 7)$. Видно, что кортежи
дополняют друг друга. Результатом операции дополнения для этих двух кортежей будет
новый кортеж $\mathrm{t} \hm= (5, \mbox{`text'}, \mathrm{null}, 7)$.


  \begin{figure*} %fig4
  \vspace*{1pt}
 \begin{center}
 \mbox{%
 \epsfxsize=120.997mm
 \epsfbox{vov-4.eps}
 }
 \end{center}
 \vspace*{-7pt}
   \Caption{Пример операции \textbf{Full Disjunction}}
   \vspace*{3pt}
  \end{figure*}

\vspace*{-6pt}

  \subsection{Операторы слияния данных}

  Различают два основных подхода к слиянию данных. Эти подходы основаны на операции
объединения (union-based) или на операции соединения (join-based). Различают следующие
основные операции.

  \textbf{Minimum Union}~\cite{11-vov} (union-based). Операция представляет собой
выполнение операции {\sf outer union}, а затем удаление из результата всех
поглощаемых (subsumed~\cite{11-vov}) кортежей. Пример операции представлен на рис.~2.


  \textbf{Complemetation Union}~\cite{11-vov} (union-based). Операция
представляет собой выполнение операции {\sf outer union}, а~затем дополнение
(complementation) всевозможных кортежей. Пример операции представлен на рис.~3.



  \textbf{Grouping and Aggregation}~\cite{11-vov} (union-based). Операция
предполагает выполнение {\sf outer union}, а~затем группировку по общему атрибуту
и~применение функции агрегации к остальным атрибутам. Пример операции на языке SQL
представлен ниже.
\smallskip

\noindent
{\sf   WITH OU AS ( }

\noindent
\hspace*{3pt}{\sf   (SELECT A, B, C, NULL AS D FROM U1)}

\noindent
\hspace*{12pt}{\sf UNION (ALL)}

\noindent
\hspace*{3pt}{\sf (SELECT A, B, NULL AS C, D FROM U2)),}


\noindent
{\sf SELECT A, MAX(B),  MIN(C), SUM(D)}

\noindent
{\sf FROM 	OU}

\noindent
{\sf GROUP BY A}

\smallskip

\textbf{Full Disjunction}~\cite{37-vov} (join-based). Операция представляет собой
{\sf full outer join} (стандартную реляционную операцию), после чего
применяется {\sf subsumption} к результату. Пример представлен на рис.~4.



  \textbf{Match Join}~\cite{11-vov} (union-\;+\;join-based). В операции выбираются
всевозможные комбинации значений атрибутов, после чего выполняется
{\sf full
outer join}. Фактически реализуется {\sf outer union} двух коллекций. Затем
определяется $N\hm-1$ вспомогательных отношений, где $N$~--- число атрибутов,
  а~каж\-дое из отношений содержит по два атрибута: один общий и~какой-то другой.
После чего происходит {\sf full outer join} ($N\hm-1$)-го отношения. Пример
реализации операции на языке SQL представлен ниже.

\smallskip

\noindent
{\sf   WITH}

\noindent
{\sf   \hspace*{3pt}OU(A,B,C,D) AS (}

\noindent
{\sf   \hspace*{3pt}(SELECT A, B, C, NULL AS D FROM U1)}

\noindent
{\sf  \hspace*{6pt}UNION}

\noindent
{\sf   \hspace*{3pt}(SELECT  A, B, NULL AS C, D FROM U2))},

\noindent
\ \ \ \ \ //\ $\leftarrow$\ \textbf{Outer Union}

\noindent
{\sf   \hspace*{3pt}B\_V(A,B) AS (SELECT DISTINCT A, B FROM OU)},


\noindent
\ \ \ \ \  //\ $\leftarrow$\ \textbf{1-е отношение} ($N=4$)

\noindent
  {\sf \hspace*{3pt}C\_V(A,C) AS (SELECT DISTINCT A, C FROM OU)},

  \noindent
\ \ \ \ \  // \ $\leftarrow$\ \textbf{ 2-е отношение} ($N=4$)

\noindent
  {\sf \hspace*{3pt}D\_V(A,D) AS (SELECT DISTINCT A, D FROM OU),}

\noindent
\ \ \ \ \ //\ $\leftarrow$\ \textbf{3-е отношение} ($N=4$)

\noindent
  {\sf SELECT A, B, C, D}

\noindent
  {\sf FROM B\_V FULL OUTER JOIN C\_V FULL OUTER

  \noindent
  \ \ \ \ \  JOIN D\_V}
  //\ $\leftarrow$\ \textbf{Full Outer Join}

 \textbf{Merge} (union-\;+\;join-based). Операция объединяет операции соединения
и~объединения. Для каждого общего атрибута формируются две версии значений, нулевые
значения удаляются функцией COALESCE (выбор первого ненулевого значения). Пусть
даны два отношения: A~с~набором атрибутов $\{\mathrm{a, b, c}\}$ и~B с~набором
атрибутов $\{\mathrm{a, b, d}\}$. Пусть a~--- конфликтующий атрибут, b~--- атрибут
с~нулевыми значениями. Пример реализации операции на языке SQL представлен ниже,
а~результат показан на рис.~5.

\noindent
{\sf   (SELECT A.a, COALESCE(A.b, B.b), A.c, B.d}

\noindent
  {\sf FROM A LEFT OUTER JOIN B ON A.a\;=\;B.a)}

\noindent
{\sf   UNION}

\noindent
{\sf   (SELECT B.a, COALESCE(B.b, A.b), A.c, B.d}

\noindent
{\sf   FROM A RIGHT OUTER JOIN B ON A.a\;=\;B.a)}



  \textbf{Grouping and Aggregation} (union-based). Операция представляет собой
группировку по некоторому атрибуту, а~затем использование разнообразных агрегирующих
функций. В~качестве достоинства данного подхода можно выделить его реализацию
в~большинстве СУБД и~эффективное выполнение. Пример реализации на SQL представлен
ниже.

  \begin{figure*} %fig5
  \vspace*{1pt}
 \begin{center}
 \mbox{%
 \epsfxsize=128.851mm
 \epsfbox{vov-5.eps}
 }
 \end{center}
 \vspace*{-9pt}
  \Caption{Пример операции \textbf{Merge}}
  \end{figure*}


\noindent
{\sf  WITH OU AS (}

\noindent
{\sf   \hspace*{3pt}(SELECT A, B, C, NULL AS D FROM U1)}

\noindent
{\sf         \hspace*{6pt}UNION (ALL)}

\noindent
{\sf    \hspace*{3pt}(SELECT A, B, NULL AS C, D FROM U2)),}

\noindent
{\sf   SELECT A, MAX(B), MIN(C), SUM(D)}

\noindent
{\sf   FROM OU}

\noindent
{\sf   GROUP BY A}

\textbf{Data Fusion оператор}~\cite{13-vov}~--- \textbf{Fuse By} (union-based).
В~некоторых системах пошли дальше использования стандартных операций
группировки и~агрегации. Ключевое слово FUSE BY используется вместо GROUP BY,
и~семантика у него аналогична. Вмес\-то использования стандартных функций агрегации
используется встроенная функция RESOLVE, которой параметром передается само значение
и~имя функции разрешения конфликтов. Пример реализации на SQL представлен ниже.

\noindent
{\sf SELECT ID,}

\noindent
{\sf  \hspace*{3pt}RESOLVE(Title, Choose(IMDB)),}

\noindent
{\sf   \hspace*{3pt}RESOLVE(Year, Max),}

\noindent
{\sf  \hspace*{3pt}RESOLVE(Director, COALESCE),}

\noindent
{\sf   \hspace*{3pt}RESOLVE(Rating, COALESCE),}

\noindent
{\sf   \hspace*{3pt}RESOLVE(Genre, Concat)}

\noindent
{\sf   FUSE FROM IMDB, Filmdienst}

\noindent
{\sf   FUSE BY (ID)}

\noindent
{\sf   ON ORDER Year DESC}

  \section{Разрешение сущностей в~больших данных}

  Для манипулирования большими разноструктурированными данными служат
  Hadoop-инф\-ра\-струк\-ту\-ры~[1, 2], предоставляющие масштабируемое хранилище
и~обеспечивающие высокую \mbox{скорость} анализа больших данных за счет распределенной их
обработки. Для применения методов разрешения сущностей в~такой среде нужна
\textit{адаптация} алгоритмов для их распределенного выполнения на различных узлах
Hadoop-клас\-тера.

  В среде Hadoop реализована парадигма распределенного программирования для анализа
данных Map-Reduce~\cite{38-vov, 39-vov}, называемая по именам основных функций.
Вначале на всех узлах кластера обрабатываются блоки данных независимо друг от друга
(Map). После чего данные группируются по заранее выбранным для алгоритма ключам
и~поступают на выполнение на один или более узлов в~зависимости от алгоритма (Reduce).

  Таким образом, для реализации любого алгоритма в~Hadoop-инфра\-струк\-ту\-ре
требуется его адап\-тация к виду Map-Reduce. Другим вариантом яв\-ля\-ется реализация
алгоритма на одном из языков \mbox{высокого} уровня, таких как Pig~[40], Hive~[41], Jaql~[42]. Все
эти языки автоматически переписывают программы, реализованные на них,
  в~Map-Reduce-при\-ло\-же\-ния для выполнения на Hadoop-клас\-тере.

  В случае больших данных и~распределенных инфраструктур традиционные подходы
требуют доработок. Различают два основных метода разрешения сущностей над большими
данными: разбиение данных на блоки (blocking~[43, 44]) и~распределенный метод
разрешения сущностей.

  Суть разбиения на блоки заключается в~сле\-ду\-ющем. Пусть имеется 1000~компаний
в~1000~городах. Требуется сравнить компании. Алгоритм полного попарного сравнения
потребует 10$^{12}$ сравнений. При этом если предположить, что компании из разных
городов не могут совпадать, то потребуется 10$^9$ сравнений. Ключевой проблемой данного
подхода является выбор критерия, по которому разбиваются данные.

Различают два
основных метода: основанный на хэш-функ\-ции~\cite{8-vov} и~основанный на сходстве
соседей~\cite{8-vov}. Метод, основанный на хэш-функ\-ции, предполагает разбиение на
блоки по хэш-клю\-чу. Основной проблемой алгоритма является выбор хэш-функ\-ции.
Метод, основанный на сходстве соседей, предполагает, что совпадать могут только объекты,
похожие по некоторой мере. Все объекты сортируются по ка\-ко\-му-то признаку (ключу~---
простому или составному, уникальность ключа не требуется). После этого выбирается
размер окна, внутри которого объекты сравниваются. Проблемой данного метода является
выбор ключа сортировки.

  Распределенный метод разрешения сущностей предполагает реализацию традиционных
алгоритмов этого семейства в~виде Map-Reduce-при\-ло\-же\-ния, что требует зачастую
полного пересмотра исходного алгоритма. Другой вариант~--- реализация алгоритма
разрешения сущностей на специализированных языках, чему будет посвящен сле\-ду\-ющий
раздел. Третий вариант~--- использование специализированных инструментов,
направленных на распределенное выполнение методов разрешения сущностей над
Hadoop~[45].

\vspace*{-6pt}

\section{Реализация операций разрешения сущностей и~слияния
данных в~среде Hadoop}

\vspace*{-2pt}

  Язык HIL~[46]~--- декларативный язык, ориентированный
на разрешение и~интеграцию сущностей в~Hadoop инфраструктуре. HIL компилируется
в~язык Jaql~[43, 44], который, в~свою очередь, автоматически переписывается
  в~Map-Reduce, если этого требует алгоритм.

  \vspace*{-6pt}

  \subsection{Реализация методов разрешения сущностей}

  \vspace*{-2pt}

  Пусть даны структуры данных, включающие три атрибута: id, value, name. Тогда
простейшее правило разрешения сущностей на языке HIL будет выглядеть следующим
образом:

{\sf \noindent
  declare \textbf{Duplicated}: ?;

\noindent
  declare \textbf{Generated}: ?;

\noindent
  declare \textbf{Deduplicated}: ?;

  \vspace*{6pt}

\noindent
  create link \textbf{Deduplicated} as

\noindent
  select

\noindent
  [gen: [id: g.id, name: g.name, value: g.value],

\noindent
  dup: [id: d.id, name: d.name, value: d.value]]

\noindent
  from Generated g, Duplicated d

\noindent
  match using

\noindent
    \hspace*{3pt}rule\_id: g.id\;=\;d.id,

\noindent
    \hspace*{3pt}rule\_name: g.name\;=\;d.name,

\noindent
    \hspace*{3pt}rule\_value: g.value\;=\;d.value;
    }

  В этом примере используется простое сопо\-став\-ле\-ние сущностей по совпадению значений.
Если требуется ввести какую-то функцию меры для значений, это можно реализовать
внешней функцией Jaql:
{\sf

\noindent
  @jaql\{

\noindent
  {\textbf{compareValue}} =

  javaudf("org.ipiran.similarity.ValueSimilarity");

\noindent
  \}
  }

  После этого такую функцию можно вызывать из языка HIL:

{\sf
\noindent
  declare compareValue: function ? to ?;

\noindent
  declare Duplicated: ?;

\noindent
  declare Generated: ?;

\noindent
  declare Deduplicated: ?;

  \vspace*{6pt}

\noindent
  create link Deduplicated as

\noindent
  select

\noindent
  [gen: [id: g.id, name: g.name, value: g.value],

\noindent
  dup: [id: d.id, name: d.name, value: d.value]]

\noindent
  from Generated g, Duplicated d

\noindent
  match using

\noindent
  rule\_id:

\noindent
    \hspace*{3pt}{\textbf{compareValue}}(g.id, d.id)\;$>$\;0.7,

\noindent
  rule\_name:

\noindent
    \hspace*{3pt}{\textbf{compareValue}}(g.name, d.name)\;$>$\;0.7,

\noindent
  rule\_value:

\noindent
    \hspace*{3pt}{\textbf{compareValue}}(g.value, d.value)\;$>$\;0.7;
    }

  Можно также ввести меру для сравнения объектов целиком.

  Пусть описана функция
{\sf compareObject}, которая принимает на вход объекты. Тогда правило на языке HIL
изменится, так как в~этом случае используется другой вид правил:

{\sf

\noindent
  insert into Deduplicated

\noindent
  select

\noindent
  [gen: [id: g.id, name: g.name, value: g.value],

\noindent
  dup: [id: d.id, name: d.name, value: d.value],

\noindent
  value: {\textbf{compareObject}}(g,d)]

\noindent
  from Generated g, Duplicated d

\noindent
  where {\textbf{compareObject}}(g, d)\;$>$\;0.7;
  }

  Во всех этих случаях происходит сравнение всех объектов со всеми, сложность подобного
сравнения $O(n^2)$. Несмотря на то что сравнения будут выполняться независимо
и~распределены на всех узлах кластера (так как HIL переписывается в~Jaql, а~тот, в~свою
очередь, в~Map-Reduce), время их выполнения может быть довольно большим. Для
уменьшения числа сравнений, как было описано в~разд.~4, можно разбивать
данные на блоки.

  Пусть имеется функция {\sf calcHash}, которая вычисляет хэш для объектов. В~результате
функция может выдавать столько уникальных значений, на сколько блоков требуется
разбить данные. Тогда, объединив правила, рассмотренные выше, выбрав вначале те
объекты, что совпадают по хэш-функ\-ции, а~далее, вычислив общую меру, можно получить
результат за более короткое время:

{\sf

\noindent
  declare {\textbf{calcHash}}: function ? to ?;

\noindent
  insert into GeneratedHash

\noindent
  select [\$.*, hash: {\bfseries\textit{calcHash}}(\$.*)]

\noindent
  from Generated;

\noindent
  insert into DuplicatedHash

\noindent
  select [\$.*,hash: {\textbf{calcHash}}(\$.*)]

\noindent
  from Duplicated;

 \vspace*{6pt}

\noindent
  create link Deduplicated as

\noindent
  select [

\noindent
  \hspace*{3pt}gen: [id: g.id, name: g.name, value:   g.value],

\noindent
  \hspace*{3pt}dup: [id: d.id, name: d.name, value: d.value]]

\noindent
  from GeneratedHash g, DuplicatedHash d

\noindent
  match using

\noindent
  \hspace*{3pt}rule\_id: g.hash\;=\;d.hash;

\noindent
  insert into Measured

\noindent
  select [gen: dd.gen, dup: dd.dup, value:

  \noindent
  \hspace*{6pt}{\textbf{compareObject}}(dd.gen, dd.dup)]

\noindent
  from Deduplicated dd

\noindent
  where {\textbf{compareObject}}(dd.gen, dd.dup)\;$>$\;0.8
  }

  \subsection{Реализация методов слияния данных}

     Будем считать, что этап разрешения сущностей уже пройден и~дана некоторая
коллекция Deduplicated, где уже установлены соответствия одним из вышеперечисленных
способов. Например, пусть имеются две коллекции: A (id, a, b, c) и~B (id, a, b, d). Атрибуты
a, b, c, d могут содержать null-значения, атрибуты id совпадают. Ниже дан пример подобных
данных для коллекции~А в~формате JSON:

{\sf
\noindent
  [\{"a":null,"b":null,"c":"wmqhxfgmac",

  \noindent
  "id":919132322\},

\noindent
  \{"a":null,"b":null,"c":"wmqhxfgmac",

  \noindent
  "id":919132322\}]
}
  Тогда коллекция разрешенных сущностей может быть получена следующим образом:

{\sf
\noindent
  create link Deduplicated as

\noindent
  select


\noindent
[gen: [id: a.id, a:a.a, b:a.b, c:a.c],

\noindent
  dup:  [id: b.id, a:b.a, b:b.b, d:b.d]]

\noindent
  from A a, B b

\noindent
  match using

\noindent
  \hspace*{3pt}rule1: a.id\;=\;b.id;
  }

  Рассмотрим теперь реализацию Minimum Union и~оператор Fusion~\cite{13-vov} на языке
HIL.

  Как было определено в~разд.~3, \textbf{Minimum Union}~--- это
последовательное применение операций {\sf outer union}
и~{\sf subsumption}~\cite{11-vov}. {\sf Outer
Union} фактически реализуется с~помощью индекса {\sf FusionIndex}. Использование индекса
оправдано, так как существует несколько записей, описывающих одну сущность. Ключом
является атрибут id. Ниже представлена реализация операции {\sf Outer Union}:

{\sf
\noindent
  insert into {\textbf{FusionIndex}}![id: f.gen.id] select [a: f.gen.a, b:
f.gen.b, c: f.gen.c] from Deduplicated f;

  \vspace*{6pt}

\noindent
  insert into {\textbf{FusionIndex}}![id: f.dup.id] select [a: f.dup.a, b: f.dup.b, d:
f.dup.d] from Deduplicated f;
}

  Далее для реализации subsumption требуется удалить все ненужные кортежи. Это делается
на языке Jaql. Для этого нужна функция, которая бы определяла, поглощается ли один
кортеж другим. К~сожалению, в~языке Jaql нет возможностей написания общих (generic)
методов, универсальных для всех коллекций, поэтому функцию сравнения можно
реализовать на Java и~подключить к~языку Jaql подобно тому, как демонстрировалось
в~подразд.~5.1 на примере функций вычисления меры. Либо же можно реализовать функцию
для сравнения конкретных коллекций на языке Jaql, как показано ниже:

{\sf
 \noindent
{\textbf{is\_subsumed}} = fn(i,j) ((


 \noindent isnull(j.a) or (i.a\;==\;j.a) ) and (isnull(j.b) or

 \noindent
 \ (i.b\;==\;j.b))  and (isnull(j.c) or (i.c\;==\;j.c)) and

 \noindent
 \ (snull(j.d) or (i.d\;==\;j.d)) and (i!\;=\;j));
  }

  Функция \textbf{is\_subsumed(i,j)} проверяет, поглощает ли один кортеж другой
при помощи попарного сравнения атрибутов или проверки на null.

{\sf \noindent
{\textbf{removeSubsumed}}\;=\;fn (a) (b = a,

\noindent
  subs = for (i0 in b) [a\;$\to$\;filter is\_subsumed(i0,\$)],

  s = subs\;$\to$\;expand,

\noindent
  a\;$\to$\;filter not \$ in s);
}

  Функция \textbf{removeSubsumed} удаляет все поглощенные записи из кортежа. Здесь
реализован наивный алгоритм, который попарно для каждого кортежа находит все
поглощенные им и~удаляет их.

{\sf
\noindent
{\textbf{minUnion}}\;=\;fn(id,a) ( \{id:id, minunion:


\noindent
\ {\textbf{removeSubsumed}}(a)\});
}

  Функция \textbf{minUnion} нужна для построения результирующих кортежей при
реализации Minimum Union. С~ее помощью операцию Minumum Union можно описать
следующим образом на языке HIL:

{\sf \noindent
  insert into {\textbf{MinimumUnion}}

\noindent
  select {\textbf{minUnion}}(i.dup.id,

\noindent
  \hspace*{10mm}{\textbf{FusionIndex}}![id : i.dup.id])

\noindent
  from Deduplicated i;
  }

  Для каждого id достаются все соответствующие записи и~удаляются те, которые ими
поглощаются.

  Оператор \textbf{Data Fusion}~\cite{13-vov} представляет собой особый вид функции,
использующий группировку для преодоления конфликтов. Основная идея заключается
в~группировке различных представлений одной и~той же сущности по общему атрибуту,
а~затем в~применении функций разрешения конфликтов для всех остальных атрибутов,
сливая данные в~одну сущность. Различают два вида стратегии для функций разрешения
конфликтов:
  \begin{enumerate}[(1)]
\item deciding-стратегия заключается в~выборе ка\-ко\-го-то одного значения каким-то
способом (минимум, максимум, случайное значение);
\item mediating-стратегия заключается в~агрегации всех значений (среднее значение,
сумма).
\end{enumerate}

  Пусть имеются две коллекции: A (id, name, age) и~B (id, name, info), пример которых дан
ниже:

{\sf
\noindent
  A

\noindent
  [\{"id":760046903,"name":null,"age":null\},

\noindent
  \{"id":15009544, "name":
"zvqcsxkzxk",

\noindent
"age":938781652\}]

  \vspace*{6pt}

\noindent
  B

\noindent
  [\{"id":15009544, "name":null,
"info":null\},

\noindent
  \{"id":760046903,
"name":"pjltaghyug","info":null\}]

}

  Пусть для них пройден этап разрешения сущностей и~построена коллекция Deduplicated,
как описано выше в~этом разделе. Пусть также для этих данных построен индекс
FusionIndex, как показано выше для операции Minimum Union. Тогда оператор Data Fusion на
языке HIL может быть описан следующим образом:

{\sf
\noindent
  @jaql\{

\noindent
  {\textbf{average}} = fn(\$a) avg(\$a[*].age);

\noindent
  {\textbf{any}} = fn(\$a) any(\$a[*].name);

\noindent
  {\textbf{concat}} = fn (\$a) strJoin(\$a[*].info,"\_");

\noindent
  \}


  \vspace*{3pt}

\noindent
insert into {\textbf{Fused}}

\noindent
select [

\noindent
id: i.dup.id,

\noindent
age:

\noindent
    \hspace*{3pt}{\textbf{average}}(FusionIndex![id: i.dup.id]),

\noindent
name:

\hspace*{6pt}

\noindent
{\textbf{any}}(FusionIndex![id: i.dup.id]),

\noindent
info:

\noindent
    \hspace*{3pt}{\textbf{concat}}(FusionIndex![id: i.dup.id])]

\noindent
from Deduplicated i;

}

  Функции вычисления среднего, выбора случайного ненулевого значения, а также
конкатенации реализованы на Jaql. Данное правило образует коллекцию \textbf{Fused},
причем для атрибута {\sf age} будет подсчитано среднее значение, для имени {\sf name}
выбрано любое ненулевое значение, а~для атрибута {\sf info} будет получена
конкатенация всех доступных значений. Таким образом, в~данном примере показана
реализация обеих стратегий для функций разрешения конфликтов в~операторе Data Fusion.

\section{Заключение}

  Рассмотренные методы и~операции извлечения и~интеграции информации о~сущностях
реального мира, представленной сырыми разноструктурированными коллекциями данных,
позволяют программировать интеграционные потоки вида ETL для образования
интегрированных структурированных данных, которые могут быть использованы
в~приложениях для дальнейшего анализа и~обработки. В~статье рассмотрены методы
разрешения сущностей и~слияния данных. В~статье показаны способы программирования
методов и~операций извлечения и~интеграции информации о~сущностях реального мира,
включая методы слияния данных, на декларативном языке HIL.


{\small\frenchspacing
 {%\baselineskip=10.8pt
 \addcontentsline{toc}{section}{References}
 \begin{thebibliography}{99}
\bibitem{1-vov}
\Au{White T.} Hadoop: The definitive guide.~--- 3rd ed.~--- O'Reilly Media, 2012.
688~p.
\bibitem{2-vov}
Apache Hadoop 2.5.1. {\sf http://hadoop.apache.org}.

\bibitem{5-vov} %3
\Au{Naumann F., Herschel~M.} An introduction to duplicate detection.
Synthesis lectures on
data management.~--- Morgan \& Claypool, 2010. Lecture No.\,3. 87~p.

\bibitem{3-vov} %4
\Au{Christen P.} Data matching~---  concepts and techniques for record linkage, entity
resolution, and duplicate detection. Data-centric systems and applications
ser.~--- Springer, 2012. 272~p.

\bibitem{4-vov} %5
\Au{Fan W., Geerts F.} Foundations of data quality management. Synthesis lectures on data
management.~--- Morgan \& Claypool, 2012. Lecture No.\,29. 217~p.


\bibitem{6-vov} %6
\Au{Bleiholder J., Naumann F.} Data fusion~// ACM Computing Surveys (CSUR), 2009.
Vol.~41. Iss.~1. Article No.\,1. doi: 10.1145/1456650.1456651.


\bibitem{9-vov} %7
\Au{K$\ddot{\mbox{o}}$pcke H., Thor A., Rahm~E.} Evaluation of entity resolution approaches
on real-world match problems~// Proc. VLDB Endowment, 2010. Vol.~3. Iss.~1-2.
P.~484--493.
\bibitem{10-vov} %8
\Au{K$\ddot{\mbox{o}}$pcke H., Rahm E.} Frameworks for entity matching: A~comparison~//
Data Knowledge Engineering, 2010. Vol.~69. Iss.~2. P.~197--210.
doi: 10.1016/j.datak. 2009.10.003.
\bibitem{7-vov} %9
\Au{Ganti V., Das Sarma A.} Data cleaning, a~practical perspective. Synthesis lectures on
data management.~--- Morgan \& Claypool, 2013. Lecture No.\,36. 85~p.
\bibitem{8-vov} %10
\Au{Getoor L., Machanavajjhala~A.} Entity resolution for big data~// KDD'13: 19th ACM
SIGKDD Conference on Knowledge Discovery and Data Mining Proceedings, 2013.
P.~1527--1527.

\bibitem{13-vov} %11
\Au{Bleiholder J., Naumann~F.} Declarative data fusion~--- syntax, semantics, and
implementation~// East European Conference on Advances in Databases and Information
Systems (ADBIS) Proceedings, 2005. P.~58--73.

\bibitem{12-vov} %12
\Au{Luna Dong X., Naumann F.} Data fusion~--- resolving data conflicts in integration~//
Proc. VLDB Endowment, 2009. Vol.~2. Iss.~2. P.~1654--1655.

\bibitem{11-vov} %13
\Au{Bleiholder J.} Data fusion and conflict resolution in integrated
information systems.~--- Potsdam: Hasso-Plattner-Institut, 2010. D.Sc. Diss. 184~p.


\bibitem{14-vov} %14
\Au{Winkler W.\,E.} Overview of record linkage and current research directions.
Research
report ser. (Statistics \#\,2006-2).~---
Washington, DC: Statistical Research Division, U.S. Census Bureau, 2006.
{\sf http://www.census.gov/srd/papers/pdf/rrs2006-02.pdf}.
\bibitem{15-vov}
\Au{Adamic L.\,A., Adar E.} Friends and neighbors on the Web~// Social networks, 2003.
Vol.~25. No.\,3. P.~211--230.
\bibitem{16-vov}
\Au{Bilenko M., Mooney R., Cohen~W., Ravikumar~P., Fienberg~S.} Adaptive name matching
in information integration~// IEEE Intell. Syst., 2003. Vol.~18. No.\,5. P.~16--23.
\bibitem{17-vov}
Monge--Elkan distance function.
{\sf http://www. gabormelli.com/RKB/Monge-Elkan\_Distance\_Function}.
\bibitem{18-vov}
\Au{Cochinwala M., Kurienb~V., Lalka~G., Shasha~D.} Efficient data reconciliation~//
Inform. Sci. Int.~J., 2001. Vol.~137. Iss.~1-4. P.~1--15.
\bibitem{19-vov}
\Au{Bilenko M., Mooney R.} Adaptve duplicate detecton using learnable string similarity
measures~// KDD'03: 9th ACM SIGKDD  Conference (International) on Knowledge Discovery
and Data Mining Proceedings, 2003. P.~39--48.
\bibitem{20-vov}
\Au{Christen P.} Automatic record linkage using seeded nearest neighbour and support vector
machine classification~// KDD'08:  14th ACM SIGKDD Conference (International) on
Knowledge Discovery and Data Mining  Proceedings, 2008. P.~151--159.
\bibitem{21-vov}
\Au{Chen Z., Kalashnikov D.\,V., Mehrotra~S.} Exploiting context analysis for combining
multiple entity resolution systems~// SIGMOD'09: 2009 ACM SIGMOD Conference
(International) on Management of Data Proceedings, 2009. P.~207--218.
\bibitem{22-vov}
\Au{Gupta R., Sarawagi S.} Answering table augmentaton queries from unstructured lists on the
Web~// Proc. VLDB Endowment, 2009. Vol.~2. Iss.~1. P.~289--300.
\bibitem{23-vov}
\Au{Ravikumar P., Cohen W.} A~hierarchical graphical model for record linkage~// UAI'04:
20th Conference on Uncertainty in Artificial Intelligence Proceedings, 2004. P.~454--461.

\bibitem{25-vov} %24
\Au{Tejada S., Knoblock C.\,A., Minton~S.} Learning object identification rules for information
integration~// Inform. Syst. Data Extraction Cleaning Reconciliation, 2001.
Vol.~26. Iss.~8. P.~607--633.

\bibitem{24-vov} %25
\Au{Sarawagi S., Bhamidipaty A.} Interactive deduplication using active learning~// KDD'02:
8th ACM SIGKDD Conference (International) on Knowledge Discovery and Data Mining
Proceedings, 2002. P.~269--278.

\bibitem{26-vov} %26
\Au{Arasu A., G$\ddot{\mbox{o}}$tz M., Kaushik~R}. On active learning of record matching
packages // SIGMOD'10: 2010 ACM SIGMOD Conference (International) on Management of
Data Proceedings, 2010. P.~783--794.
\bibitem{27-vov}
\Au{Bellare K., Iyengar~S, Parameswaran~A.\,G., Rastogi~V.} Active sampling for entity
matching // KDD'12: 18th ACM SIGKDD  Conference (International) on Knowledge Discovery
and Data Mining Proceedings, 2012. P.~1131--1139.
\bibitem{28-vov}
\Au{Adam K., Wu E., Karger~D., Madden~S., Miller~R.} Human-powered sorts and joins~//
Proc. VLDB Endowment, 2011. Vol.~5. Iss.~1. P.~13--24.
\bibitem{29-vov}
\Au{Wang J., Kraska T., Franklin~M.\,J., Feng~J.} CrowdER: Crowdsourcing Entity
Resolution~// Proc. VLDB Endowment, 2012. Vol.~5. Iss.~11. P.~1483--1494.
\bibitem{30-vov}
\Au{Ananthakrishna R., Chaudhuri~S., Ganti~V.} Eliminating fuzzy duplicates in data
warehouses~// VLDB'02:  28th Conference (International) on Very Large Data Bases
Proceedings, 2002. P.~586--597.
\bibitem{31-vov}
\Au{Fan W., Geerts F., Jia~X., Kementsietsidis~A.} Conditional functional dependencies for Data
cleaning~// ICDE'07: 23rd IEEE Conference (International) on Data Engineering Proceedings,
2007. P.~746--755.
\bibitem{32-vov}
\Au{Fan W.} Dependencies revisited for improving data quality~// PODS'08:  27th ACM
SIGMOD-SIGACT-SIGART Symposium on Principles of Database Systems Proceedings, 2008.
P.~159--170.
\bibitem{33-vov}
\Au{Benjelloun O., Garcia-Molina~H., Menestrina~D., Su~Q., Whang~S.\,E., Widom~J.}
Swoosh: A~generic approach to Entity Resolution~// VLDB Int.~J., 2009.
Vol.~18. Iss.~1. P.~255--276.
\bibitem{34-vov}
\Au{Bhattacharya I., Getoor L.} Collective Entity Resolution in relational data~// ACM
Transactions on Knowledge Discovery from Data (TKDD), 2007. Vol.~1. Iss.~1. Article No.\,5.
doi: 10.1145/1217299.1217304.
\bibitem{35-vov}
\Au{Bhattacharya I., Getoor L.} A~latent Dirichlet model for unsupervised Entity Resolution~//
6th SIAM Conference (International) on Data Mining Proceedings, 2007. P.~47--58.
\bibitem{36-vov}
\Au{Broecheler M., Getoor~L.} Probabilistic similarity logic~// UAI'10:  26th Conference on
Uncertainty in Artificial Intelligence Proceedings, 2010. P.~73--82.
\bibitem{37-vov}
\Au{Rajaraman A., Ullman~J.\,D.} Integrating information by outerjoins and full disjunctions~//
PODS'96: 15th ACM SIGACT-SIGMOD-SIGART Symposium on Principles of Database
Systems Proceedings, 1996. P.~238--248.
\bibitem{38-vov}
\Au{Dean J., Ghemawat S.} MapReduce: Simplified data processing on large clusters~//
Comm. ACM, 2008. Vol.~51. Iss.~1. P.~107--113.
\bibitem{39-vov}
MapReduce Tutorial. {\sf http://hadoop.apache.org/docs/\linebreak r1.2.1/mapred\_tutorial.html}.
\bibitem{40-vov}
Apache Pig Project. {\sf http://pig.apache.org}.
\bibitem{41-vov}
The Apache Hive data warehouse. {\sf http://hive. apache.org}.
\bibitem{42-vov}
IBM InfoSphere BigInsights Version~3.0, Jaql reference.~--- 2014.
{\sf
http://www-01.ibm.com/support/\linebreak knowledgecenter/SSPT3X\_3.0.0/com.ibm.swg.im.\linebreak infosphere.biginsights.jaql.doc/doc/c
0057749.html}.
\bibitem{43-vov}
\Au{Das Sarma A., Jain A., Machanavajjhala~A., Bohannon~P.} An automatic blocking
mechanism for large-scale de-duplication tasks~// CIKM'12:  21st ACM Conference
(International) on Information and Knowledge Management Proceedings, 2012. P.~1055--1064.
\bibitem{44-vov}
\Au{Papadakis G., Ioannou E., Nieder$\acute{\mbox{e}}$e~C., Palpanas~T., \mbox{Nejdl}~W.} Beyond
100 million entities: Large-scale blocking-based resolution for heterogeneous data~//
\mbox{WSDM'12}:
5th ACM Conference (International) on Web Search and Data Mining Proceedings, 2012.
P.~53--62.
\bibitem{45-vov}
\Au{Kolb L., Thor A., Rahm E.} Dedoop: Efficient deduplication with Hadoop~// Proceedings of
the VLDB Endowment, 2012. Vol.~5. Iss.~12. P.~1878--1881.
\bibitem{46-vov}
\Au{Hern$\acute{\mbox{a}}$ndez M., Koutrika G., Krishnamurthy~R., Popa~L.,
Wisnesky~R.}
HIL: A~high-level scripting language for entity integration~// EDBT'13: 16th Conference
(International) on Extending Database Technology Proceedings, 2013. P.~549--560.
 \end{thebibliography}

 }
 }

\end{multicols}

\vspace*{-6pt}

\hfill{\small\textit{Поступила в редакцию 09.11.14}}

%\newpage

\vspace*{10pt}

\hrule

\vspace*{2pt}

\hrule

%\vspace*{12pt}

\def\tit{METHODS OF ENTITY RESOLUTION AND DATA FUSION\\
IN~THE~ETL-PROCESS AND THEIR IMPLEMENTATION IN~THE~HADOOP~ENVIRONMENT}

\def\titkol{Methods of entity resolution and~data fusion
in~the~ETL-process and their implementation in the Hadoop environment}

\def\aut{A.\,E.~Vovchenko$^1$, L.\,A.~Kalinichenko$^{1,2}$,
and~D.\,Yu.~Kovalev$^1$}

\def\autkol{A.\,E.~Vovchenko, L.\,A.~Kalinichenko,
and~D.\,Yu.~Kovalev}

\titel{\tit}{\aut}{\autkol}{\titkol}

\vspace*{-9pt}

\noindent
$^1$Institute of Informatics Problems, Russian Academy of Sciences,
44-2~Vavilov Str., Moscow 119333, Russian\linebreak
$\hphantom{^1}$Federation

\noindent
$^2$Faculty of Computational Mathematics and Cybernetics,
M.\,V.~Lomonosov Moscow State University,
1-52~Lenin-\linebreak
$\hphantom{^1}$skiye Gory, GSP-1, Moscow 119991, Russian Federation

\def\leftfootline{\small{\textbf{\thepage}
\hfill INFORMATIKA I EE PRIMENENIYA~--- INFORMATICS AND
APPLICATIONS\ \ \ 2014\ \ \ volume~8\ \ \ issue\ 4}
}%
 \def\rightfootline{\small{INFORMATIKA I EE PRIMENENIYA~---
INFORMATICS AND APPLICATIONS\ \ \ 2014\ \ \ volume~8\ \ \ issue\ 4
\hfill \textbf{\thepage}}}

\vspace*{6pt}


  \Abste{Entities extraction, their transformation and loading
  in the integrated repository are the main problem of data integration.
  These actions are part of the ETL-process (extract--transform--loading).
An entity is a~digital representation of a real world object (for example,
  information about a~person). Entity resolution takes care of duplicate detection,
  deduplication, record linkage, object identification, reference matching,
  and other ETL-related tasks. After the entity resolution step, entities should
  be merged into the one reference entity (containing information from all
  related entities). Data fusion is the final step in the data integration
  process. The paper gives an overview of the entity resolution and data fusion
  methods. Also, the paper presents the techniques for programming the entity
  resolution and data fusion methods for implementing the ETL-process in the
  Hadoop environment.  High-Level Integration Language (HIL),
  a~declarative language that focuses on resolution and fusion of
 entities in the  Hadoop-infrastructure, is used in this part of the paper.}

 \vspace*{1pt}

  \KWE{data integration; ETL; entity resolution; data fusion; big data; Hadoop; Jaql; HIL}

  \DOI{10.14357/19922264140412}

\vspace*{-12pt}


\Ack
\noindent
This work was supported by the Russian Foundation for Basic Research
(projects 13-07-00579 and 14-07-00548), Institute of informatics Problems
of the Russian Academy of Sciences (IPI RAN) (theme 38.25
``Specification and problem solving of data analysis in conceptual terms
of subject areas with intensive use of data''
of the state task for IPI RAN), and the Presidium of the Russian Academy
of Sciences (Basic Research Program No.\,16 ``Fundamental problems of
system programming'').


\vspace*{3pt}

  \begin{multicols}{2}

\renewcommand{\bibname}{\protect\rmfamily References}
%\renewcommand{\bibname}{\large\protect\rm References}



{\small\frenchspacing
 {%\baselineskip=10.8pt
 \addcontentsline{toc}{section}{References}
 \begin{thebibliography}{99}


\bibitem{1-vov-1}
\Aue{White, T.} 2012. Hadoop: The definitive guide. 3rd ed.
O'Reilly Media. 688~p.
\bibitem{2-vov-1}
Apache Hadoop 2.5.1. Available at: {\sf http://hadoop. apache.org/}
(accessed November~01, 2014).

\bibitem{5-vov-1} %3
\Aue{Naumann, F., and M. Herschel}. 2010.
\textit{An introduction to duplicate detection}.
Synthesis lectures on data management.  Morgan \& Claypool. Lecture No.\,3. 87~p.

\bibitem{3-vov-1} %4
\Aue{Christen, P.} 2012.
\textit{Data matching~--- concepts and techniques for record linkage, entity resolution,
and duplicate detection}. {Data-centric systems and applications} ser.
Springer. 272~p.
\bibitem{4-vov-1} %5
\Aue{Fan, W., and F. Geerts.} 2012. \textit{Foundations of data quality management}.
Synthesis lectures on data management.
Morgan \& Claypool. Lecture No.\,29. 217~p.

\bibitem{6-vov-1}
\Aue{Bleiholder, J., and F.~Naumann}. 2009.
Data fusion. \textit{ACM Computing Surveys (CSUR)} 41(1). Article No.\,1.
doi: 10.1145/1456650.1456651.

\bibitem{9-vov-1} %7
\Aue{K$\ddot{\mbox{o}}$pcke, H., A.~Thor, and E.~Rahm}.
2010. Evaluation of entity resolution approaches on real-world match problems.
\textit{Proc. VLDB Endowment} 3(1-2):484--493.
\bibitem{10-vov-1} %8
\Aue{K$\ddot{\mbox{o}}$pcke, H., and E.~Rahm}. 2010. Frameworks for
entity matching: A~comparison. \textit{Data Knowledge Engineering} 69(2):197--210.
doi: 10.1016/j.datak.2009.10.003.

\bibitem{7-vov-1} %9
\Aue{Ganti, V., and A. Das Sarma}. 2013. Data cleaning: A~practical perspective.
{Synthesis lectures on data management}.  Morgan \& Claypool.
Lecture No.\,36. 85~p.
\bibitem{8-vov-1} %10
\Aue{Getoor, L., and A.~Machanavajjhala}.
2013. Entity resolution for big data. \textit{19th ACM SIGKDD
Conference (International) on Knowledge Discovery and Data Mining (KDD'13)
Proceedings}. Chicago. 1527--1527.

\bibitem{13-vov-1} %11
\Aue{Bleiholder, J., and F. Naumann}. 2005. Declarative data fusion~---
syntax, semantics, and implementation. \textit{East European Conference on
Advances in Databases and Information Systems (ADBIS) Proceedings}. Tallinn. 58--73.


\bibitem{12-vov-1} %12
\Aue{Dong, L.\,X., and F.~Naumann}. 2009. Data fusion~---
resolving data conflicts in Integration.
\textit{Proc. VLDB Endowment} 2(2):1654--1655.

\bibitem{11-vov-1} %13
\Aue{Bleiholder, J.} 2010. Data fusion and conflict resolution in integrated
information systems.  Potsdam. D.Sc. Diss. 184~p.


\bibitem{14-vov-1} %14
\Aue{Winkler, W.\,E.} 2006.
Overview of record linkage and current research directions.
 Research report ser. No.\,2006-2.
Washington, DC: Statistical Research Division, U.S. Census Bureau. 44~p. Available at:
{\sf http:// www.census.gov/srd/papers/pdf/rrs2006-02.pdf} (accessed November~01, 2014).
\bibitem{15-vov-1}
\Aue{Adamic, L.\,A., and E.~Adar}. 2003. Friends and neighbors on the Web.
\textit{Social Networks} 25:211--230.
\bibitem{16-vov-1}
\Aue{Bilenko, M., R. Mooney, W.~Cohen, P.~Ravikumar, and S.~Fienberg}.
2003. Adaptive name matching in information integration.
\textit{IEEE Intell. Syst.} 18(5):16--23.
\bibitem{17-vov-1}
Monge--Elkan distance function. Available at:
{\sf http://\linebreak www.gabormelli.com/RKB/Monge-Elkan\_Distance\_ Function}
(accessed November~01, 2014).
\bibitem{18-vov-1}
\Aue{Cochinwala, M., V. Kurienb, G.~Lalka, and D.~Shasha}.
2001. Efficient data reconciliation.
\textit{Inform. Sci. Int.~J.} 137(1-4):1--15.
\bibitem{19-vov-1}
\Aue{Bilenko, M., and R.~Mooney}. 2003.
Adaptve duplicate detecton using learnable string similarity measures.
\textit{9th ACM SIGKDD  Conference (International) on Knowledge Discovery and
Data Mining (SIGKDD 2003) Proceedings}. Washington. 39--48.
\bibitem{20-vov-1}
\Aue{Christen, P.} 2008. Automatic record linkage using seeded nearest
neighbour and support vector machine classification.
\textit{14th ACM SIGKDD  Conference (International) on Knowledge Discovery and
Data Mining (KDD'2008) Proceedings}. Las Vegas. 151--159.
\bibitem{21-vov-1}
\Aue{Chen, Z., D.\,V.~Kalashnikov, and S.~Mehrotra}. 2009.
Exploiting context analysis for combining multiple entity resolution systems.
\textit{2009 ACM SIGMOD  Conference (International) on Management of Вata (SIGMOD 2009)
Proceedings}. Providence. 207--218.
\bibitem{22-vov-1}
\Aue{Gupta, R., and S.~Sarawagi}. 2009.
Answering table augmentaton queries from unstructured lists on the Web.
\textit{Proc. VLDB Endowment} 2(1):289--300.
\bibitem{23-vov-1}
\Aue{Ravikumar, P., and W.~Cohen}. 2004. A~hierarchical
graphical model for record linkage. \textit{20th Conference on Uncertainty in
Artificial Intelligence (UAI 2004) Proceedings}.  Virginia. 454--461.

\bibitem{25-vov-1} %24
\Aue{Tejada, S., C.\,A. Knoblock, and S.~Minton}. 2001.
Learning object identification rules for information integration.
\textit{Inform. Syst. Data Extraction Cleaning Reconciliation} 26(8):607--633.

\bibitem{24-vov-1} %25
\Aue{Sarawagi, S., and A.~Bhamidipaty}. 2002.
Interactive deduplication using active learning.
\textit{8th ACM SIGKDD  Conference (International) on Knowledge Discovery and
Data Mining (KDD 2002) Proceedings}. Edmonton. 269--278.

\bibitem{26-vov-1}
\Aue{Arasu, A., M.~G$\ddot{\acute{o}}$tz, and R.~Kaushik}. 2010.
On active learning of record matching packages.
\textit{2010 ACM \mbox{SIGMOD}  Conference (International)
on Management of Data Proceedings}.  Indianapolis. 783--794.
\bibitem{27-vov-1}
\Aue{Bellare, K., S.~Iyengar, A.\,G.~Parameswaran, and V.~Rastogi}. 2012.
Active sampling for entity matching. \textit{18th ACM SIGKDD
Conference (International) on Knowledge Discovery and Data Mining (KDD 2012) Proceedings}. Beijing. 1131--1139.
\bibitem{28-vov-1}
\Aue{Adam, K., E. Wu, D.~Karger, S.~Madden, and R.~Miller}.
2011. Human-powered sorts and joins. \textit{Proc. VLDB Endowment} 5(1):13--24.
\bibitem{29-vov-1}
\Aue{Wang, J., T. Kraska, M.\,J.~Franklin, and J.~Feng}. 2012.
CrowdER: Crowdsourcing Entity Resolution.
\textit{Proc. VLDB Endowment} 5(11):1483--1494.
\bibitem{30-vov-1}
\Aue{Ananthakrishna, R., S.~Chaudhuri, and V.~Ganti}. 2002.
Eliminating fuzzy duplicates in data warehouses. \textit{28th Conference (International) on
Very Large Data Bases (VLDB 2002)
Proceedings}. Hong Kong. 586--597.
\bibitem{31-vov-1}
\Aue{Fan, W., F. Geerts, X.~Jia, and A.~Kementsietsidis}. 2007.
Conditional functional dependencies for data cleaning. \textit{2007 IEEE
23rd  Conference (International) on Data Engineering Proceeding}. Istanbul. 746--755.
\bibitem{32-vov-1}
\Aue{Fan, W.} 2008. Dependencies revisited for improving data quality.
\textit{27th ACM SIGMOD-SIGACT-SIGART Symposium on Principles of Database Systems
(PODS 2008) Proceedings}.  Vancouver.  159--170.
\bibitem{33-vov-1}
\Aue{Benjelloun, O., H.~Garcia-Molina, D.~Menestrina, Q.~Su, S.\,E.~Whang, and
J.~Widom}. 2009. Swoosh: A~generic approach to Entity Resolution.
\textit{VLDB Int.~J.} 18(1):255--276.
\bibitem{34-vov-1}
\Aue{Bhattacharya, I., and L.~Getoor}. 2007.
Collective Entity Resolution in relational data.
\textit{ACM Trans. Knowledge Discovery Data (TKDD)} 1(1). Article No.\,5.
doi: 10.1145/1217299.1217304.
\bibitem{35-vov-1}
\Aue{Bhattacharya, I., and L.~Getoor}. 2007. A~latent
Dirichlet model for unsupervised Entity Resolution.
\textit{6th SIAM  Conference (International) on Data Mining Proceedings}. Maryland. 47--58.
\bibitem{36-vov-1}
\Aue{Broecheler, M., and L.~Getoor}. 2010.
Probabilistic similarity logic. \textit{26th Conference on Uncertainty in Artificial Intelligence
Proceedings}. Corvallis. 73--82.
\bibitem{37-vov-1}
\Aue{Rajaraman, A., and J.\,D.~Ullman}. 1996. Integrating information by
outerjoins and full disjunctions. \textit{15th ACM SIGACT-SIGMOD-SIGART Symposium
on Principles of Database Systems (PODS1996) Proceedings}. Montreal. 238--248.
\bibitem{38-vov-1}
\Aue{Dean, J., and S.~Ghemawat}. 2008. MapReduce: Simplified
data processing on large clusters.
\textit{Comm. ACM} 51(1):107--113.
\bibitem{39-vov-1}
MapReduce tutorial. Available at:
{\sf http://hadoop. apache.org/docs/r1.2.1/mapred\_tutorial.html} (accessed November~01, 2014).
\bibitem{40-vov-1}
Apache Pig Project. Available at: {\sf http://pig.apache.org/} (accessed November~01, 2014).
\bibitem{41-vov-1}
The Apache Hive data warehouse. Available at:
{\sf http:// hive.apache.org/} (accessed November~01, 2014).
\bibitem{42-vov-1}
IBM InfoSphere BigInsights Version~3.0, Jaql reference.
Available at: {\sf http://www-01.ibm.com/\linebreak
support/knowledgecenter/SSPT3X\_3.0.0/com.ibm.swg. im.infosphere.biginsights.jaql.doc/doc/c0057749.html} (accessed November~01, 2014).
\bibitem{43-vov-1}
\Aue{Sarma, D.\,A., A.~Jain, A.~Machanavajjhala, and P.~Bohannon}.
2012. An automatic blocking mechanism for large-scale de-duplication tasks.
\textit{21st ACM  Conference (International) on Information and Knowledge Management
Proceedings}. Maui. 1055--1064.
\bibitem{44-vov-1}
\Aue{Papadakis, G., E.~Ioannou, C.~Nieder$\acute{\mbox{e}}$e, T.~Palpanas, and
W.~Nejdl}. 2012. Beyond 100~million entities: Large-scale blocking-based
resolution for heterogenous data. \textit{5th ACM  Conference (International) on
Web Search and Data Mining Proceedings}. Seattle. 53--62.
\bibitem{45-vov-1}
\Aue{Kolb, L., A. Thor, and E.~Rahm}. 2012. Dedoop: Efficient deduplication with
Hadoop. \textit{Proceedings of the VLDB Endowment} 5(12):1878--1881.
\bibitem{46-vov-1}
\Aue{Hern$\acute{\mbox{a}}$ndez, M., G.~Koutrika, R.~Krishnamurthy, L.~Popa, and
R.~Wisnesky}. 2013. HIL: A~high-level scripting language for entity integration.
\textit{16th  Conference (International) on Extending Database Technology (EDBT'13)
Proceedings}.  Genoa. 549--560.
\end{thebibliography}

 }
 }

\end{multicols}

\vspace*{-6pt}

\hfill{\small\textit{Received November 9, 2014}}

\vspace*{-18pt}

\Contr

\noindent
\textbf{Vovchenko Alexey E.} (b.\ 1984)~---
Candidate of Science (PhD) in technology, senior researcher,
Institute of Informatics Problems, Russian Academy of Sciences;
44-2 Vavilov Str., Moscow 119333, Russian Federation;
alexey.vovchenko@gmail.com

\vspace*{3pt}

\noindent
\textbf{Kalinichenko Leonid A.} (b.\ 1937)~---
Doctor of Science in physics and mathematics, professor; Head of Laboratory,
Institute of Informatics Problems,
Russian Academy of Sciences;
44-2 Vavilov Str., Moscow 119333, Russian Federation;
professor, Faculty of Computational Mathematics and Cybernetics,
M.\,V.~Lomonosov Moscow State University, 1-52  Leninskiye Gory, GSP-1, Moscow 119991, Russian Federation;
 leonidk@synth.ipi.ac.ru

\vspace*{3pt}

\noindent
\textbf{Kovalev Dmitry  Yu.} (b.\ 1988)~---
junior researcher, Institute of Informatics Problems,
Russian Academy of Sciences, 44-2 Vavilov Str., Moscow 119333, Russian Federation;
 dm.kovalev@gmail.com




\label{end\stat}

\renewcommand{\bibname}{\protect\rm Литература}
 %12
\renewcommand{\figurename}{\protect\bf Figure}
\renewcommand{\tablename}{\protect\bf Table}

\def\stat{kalinich}


\def\tit{CONCEPTUAL MODELING OF~MULTIDIALECT WORKFLOWS}

\def\titkol{Conceptual modeling of~multidialect workflows}

\def\autkol{L.~Kalinichenko, S.~Stupnikov, A.~Vovchenko,
and~D.~Kovalev}

\def\aut{L.~Kalinichenko$^{1,2}$, S.~Stupnikov$^1$, A.~Vovchenko$^1$,
and~D.~Kovalev$^1$}

\titel{\tit}{\aut}{\autkol}{\titkol}

%{\renewcommand{\thefootnote}{\fnsymbol{footnote}}
%\footnotetext[1] {}}

\renewcommand{\thefootnote}{\arabic{footnote}}
\footnotetext[1]{Institute of Informatics Problems, Russian Academy of Sciences,
44-2 Vavilov Str., Moscow 119333, Russian Federation}
\footnotetext[2]{Faculty of
Computational Mathematics and Cybernetics, M.\,V.~Lomonosov Moscow State University,
1-52~Leninskiye Gory, GSP-1, Moscow 119991, Russian Federation}


%\vspace*{6pt}

\def\leftfootline{\small{\textbf{\thepage}
\hfill INFORMATIKA I EE PRIMENENIYA~--- INFORMATICS AND APPLICATIONS\ \ \ 2014\ \ \ volume~8\ \ \ issue\ 4}
}%
 \def\rightfootline{\small{INFORMATIKA I EE PRIMENENIYA~--- INFORMATICS AND APPLICATIONS\ \ \ 2014\ \ \ volume~8\ \ \ issue\ 4
\hfill \textbf{\thepage}}}

%\vspace*{6pt}


\Abste{This paper contributes to the techniques for conceptual representation of
data analysis algorithms and data integration facilities as well as processes to
specify data and behavior semantics in one paradigm. An investigation of a~novel
approach for applying a~combination of semantically different
platform-independent rule-based languages (dialects) for interoperable conceptual
specifications over various rule-based systems (RSs) relying on the rule-based
program transformation technique recommended by the W3C Rule Interchange
Format (RIF) is extended here. Such approach is combined with the facilities
aimed at the semantic rule-based mediation intended for the heterogeneous data
base integration. This paper extends a~previous research of the authors in the
direction of workflow modeling for definition of compositions of algorithmic
modules in a~process structure. A~capability of the multidialect workflow
support specifying the tasks in semantically different languages mostly suited to
the task orientation is presented. A~practical workflow use case, the
interoperating tasks of which are specified in several rule-based languages
(RIF-CASPD, RIF-BLD, RIF-PRD), is introduced. In addition, OWL~2 is used
for the conceptual schema definition, RIF-PRD is used also for the workflow
orchestration. The use case implementation infrastructure includes a~production
rule-based system (IBM ILOG), a~logic rule-based system (DLV), and
a~mediation system.}

\KWE{conceptual specification; workflow; RIF; production rule languages;
database integration; mediators; PRD; multidialect infrastructure}

\DOI{10.14357/19922264140413}

%\vspace*{6pt}


\vskip 12pt plus 9pt minus 6pt

      \thispagestyle{myheadings}

      \begin{multicols}{2}

                  \label{st\stat}

\section{Introduction}

  \noindent
  This work keeps on the intention of developing the facilities for conceptual
declarative problem specification and solving in data intensive domains (DID). In [1]
it was claimed that conceptual data semantics alone (e.\,g., formalized in ontology
languages based on description logic) are insufficient, so that conceptual
representation of data analysis algorithms as well as processes for problem solving are
required to specify data and behavior semantics in one paradigm.

The results presented in this paper\footnote[3]{This paper is an extended for the journal version of
the results presented in the ``Multidialect Workflows'' report at the ADBIS'2014
Conference.} extend the research~[1] aimed at the definition and implementation of the
facilities for conceptually-driven problems specification and solving in DID aiming at
ensuring eventually the following capabilities for expressing the specifications:
\begin{enumerate}[(1)]
\item an ability to provide complete and precise specification of the abstract
structure and behavior of the domain entities, their consistency, relationship, and
interaction;
\item well-grounded diversity of semantics of the modeling facilities providing for
the best attainable expressiveness, compactness, and precision of the definition of
the problem solving algorithm specifications;
\item arrangements for the extensions of the modeling facilities satisfying the
changing technological and practical needs;
\item specification independence from implementation platforms (languages,
systems);
\item specification independence from concrete information resources (IRs)
(databases,
services, ontologies, etc.)\ combined with facilities for their semantic
integration and interoperability; and
\item built-in methodologies for creation of unifying specification languages
providing for construction of semantics-preserving mappings of conceptual
specifications into their implementations in specific platforms.
\end{enumerate}

  The research reported in~[1] investigated the conceptual modeling facilities for
DID applying rule-based declarative logic languages possessing different,
complementary semantics and capabilities combined with the methods and languages
for heterogeneous data mediation and integration. Two fundamental techniques were
combined: ($i$)~constructing of the unifying extensible language providing for
semantics-preserving mapping into it of various IR
specification languages (e.\,g., such as data definition (DDL) and
data manipulation (DML) languages for databases); and ($ii$)~creation of
the unified extensible family of rule-based languages (dialects) and a~model of
interoperability of the programs expressed in such dialects with different semantics.

  The first technique is based on the experience obtained in course of the
SYNTHESIS language development~[2]. The kernel of the SYNTHESIS language is
based on the object-frame data model used together with the declarative rule-based
facilities in the logic language similar to a~stratified Datalog with functions and
negation. The extensions of the kernel are constructed in such a~way that each
extension together with the kernel is a~result of semantic preserving mapping of some
IR language into the SYNTHESIS~[2]. The canonical information model is
constructed as a~union of the kernel with such extensions defined for various resource
languages. Canonical model is used for development of \textit{mediators} positioned
between the users, conceptually formulating problems in terms of the mediators, and
distributed resources. A~schema of a~subject mediator for a~class of problems
includes the specification of the domain concepts defined by the respective
ontologies.
{ %\looseness=1

}

  Another, multidialect technique for rule-based programs interoperability applied is
based on the RIF standard [3] of W3C. The RIF standard introduces a~unified family of rule-based
languages together with a~methodology for constructing of semantic preserving
mappings of specific languages used in various RSs into RIF
dialects. Examples of RSs include {SILK}, {OntoBroker}, {DLV},
{IBM Websphere ILOG JRules}, {RIF4J}\;+\;{IRIS}, and others
(more examples can be found at {\sf
http:// www.w3.org/2005/rules/wiki/Implementations}). From the RIF point of view, an
IR is a~program developed in a~specific language of some RS.

  In [1], the first results obtained were presented including the description of an
approach and an infrastructure supporting:
  \begin{itemize}
\item application domain conceptual specification and problem solving algorithms
definitions based on the combination of the heterogeneous database mediation
technique and the rule-based multidialect facilities;
\item interoperability of distributed multidialect rule-based programs and mediators
integrating heterogeneous databases; and
\item rule delegation approach for the peer interactions in the multidialect
environment.
\end{itemize}

  The proof-of-concept prototype of the infrastructure based on the SYNTHESIS
environment and RIF standards has been implemented. The approach for multidialect
conceptualization of a~problem domain, rule delegation, rule-based programs, and
mediators interoperability were explained in detail and illustrated on an use-case in
the finance domain~[1]. For the conceptual definition of the use-case problem, the
OWL was used for the domain concepts definition and two RIF logic dialects
RIF-BLD~[4] and RIF-CASPD~[5] were used and mapped for implementation into the
SYNTHESIS formula language and the ASP (answer set programming)
based DLV~[6] language, respectively.

  The results obtained so far are quite encouraging for future work: they show that
the mentioned in the beginning capabilities~(1)--(6) sought for conceptual modeling
become feasible. This paper reports the results of extending the research in the
direction of modeling of the processes for the problem solving following the approach
briefly outlined above. These results include extensions of the infrastructure and
specification languages considered in~[1] to the workflow level keeping the same
approach and paradigm as well as aiming at the capabilities of the
conceptualization~(1)--(6) that were stated in~[1] and mentioned in the beginning of
the introduction.

  For investigation of such extension with respect to the choice of rule-based languages, it was
decided not to go outside the limits of the existing set of the published RIF dialects.
Such decision would allow to retain well-defined semantics of the conceptual
rule-based languages with a~possibility to check preservation of their semantics by various
languages of the implementing systems.

  The production rule dialect RIF PRD~[7] has been chosen as the language for the
workflow modeling in such a~way that the tasks of the workflow can have
multidialect rule-based representation (as defined in~[1]). This paper reporting the results
of such investigation is structured as follows. To make the paper self-contained, the
next section provides a~brief overview of the infrastructure supporting multidialect
programming defined in details in~[1]. Here, it is stressed
 that this infrastructure is
suitable for the workflow tasks specification. Workflow-oriented extension of the
multidialect infrastructureis considered in section~3. Use case implementation in
the proof-of-concept prototype is given in section~4. Related works are reviewed
in section~5. Concluding Remarks summarize contributions of the research.

%\vspace*{-24pt}

\begin{figure*} %fig1
\vspace*{1pt}
 \begin{center}
 \mbox{%
 \epsfxsize=164.734mm
 \epsfbox{kal-1.eps}
 }
 \end{center}
 \vspace*{-9pt}
\Caption{Conceptual schema and peer specifications }
\vspace*{-2pt}
  \end{figure*}

\section{Basic Principles of the Workflow Tasks Representation
in~the~Multidialect Infrastructure}

  \noindent
  Each workflow task (besides those that for pragmatic reasons are defined as
externally specified functions) is assumed to be represented in the novel infrastructure
defined in details in~[1]. Conceptual programming of tasks is performed using the
RIF dialects (now not only logic but also PRDs can be
used).

Conceptual tasks are implemented by their transformation into the rule-based
programs of the respective RSs and mediation systems (MSs). \textit{Conceptual
specification of a~task} is defined in the context of a~subject domain and consists of
a~set of RIF-documents (document is a~specification unit of RIF). The
\textit{conceptual schema} of the domain is defined using OWL 2~\cite{8-kal}
ontologies. Such usage of ontology is analogous to~\cite{22-kal}; however, it is
specifically important in the multidialect environment due to the formally defined
compatibility between RIF and OWL. The ontologies contain entities of the domain
and their relationships (Fig.~1, right-hand part). Conceptual specification of a~task is
defined over conceptual schema. Ontologies are imported into the RIF-documents
specifying an import profile, for instance, {OWL Direct}. Documents
\textit{import} other documents having the same semantics (the \textit{Import}
directive), \textit{link} documents defined using other dialects and having different
semantics (remote module directive \textit{Module}) or \textit{refer} to entities
contained in other documents using \textit{external terms}.
{\looseness=1

}

  Semantics of a~conceptual task definition in such setting becomes a~multidialect
one. The specification modules of a~task are treated as peers. Mediation modules are
assumed to be defined in RIF-BLD for representation of the mediator rules (to be
interpreted in SYNTHESIS) supporting schema mapping and semantic integration of
the IRs. Multidialect task is implemented by means of
transformation of conceptual specifications into modular, component-based
peer-to-peer (P2P)
program represented in the languages of the MSs and RSs
with the respective semantics. Interoperability of logic rule components of such
distributed program is carried out by means of the delegation technique [1,
section~3.3]. Production rule components are considered as external functions,
interoperability is achieved through the mechanism of external terms.

  A schema $S_R$ of a~peer~$R$ is a~set of entities (classes or relations and their
attributes) corresponding to extensional and intensional predicates of the resource
implementing the peer~$R$.

  The RS or the MS of each peer~$R$ should be
a~conformant~$D_R$ consumer where~$D_R$ is the~respective RIF dialect (Fig.~1,
left-hand part). Conformance is formally defined using formula entailment and
language mappings~[3].

  The peer $R$ is relevant to a~RIF-document~$d$ of a~conceptual specification of a~problem
  (Fig.~1, right-hand part) if ($i$)~$D_R$ is a~subdialect of the document~$d$
dialect (subdialect is a~language obtained from some dialect by removing certain
syntactic constructsand imposing respective restrictions on its semantics~[4]; each
program that conforms with the subdialect also conforms with the dialect) and
($ii$)~entities of the peer schema~$S_R$ (if they exist) are \textit{ontologically
relevant} to entities of the conceptual schema the names of which are used
in~$d$~for extensional predicates.

  The schema of a~relevant peer is mapped into the conceptual schema. The mapping
establishes the correspondence of the conceptual entities referred in the
document~$d$ to their expressions in terms of entities of the schema~$S_R$ using
rules of the $D_R$ dialect. These schema mapping rules constitute separate
  RIF-document (Fig.~1, middle part).

  Peers communicate using a~technique for distributed execution of the rule-based
programs. The basic notion of the technique is delegation-transferring facts and
rules from one peer to another. A~peer is installed on a~node of the multidialect
infrastructure. A~node is a~combination of a~wrapper, an RS or an MS, and a~peer
(for the details, refer~[1, Fig.~3]). A~wrapper
transforms programs and facts from the specific RIF dialect into the language of the
RS or MS and \textit{vice versa}. A~wrapper also implements the delegation mechanism.
Transferring facts and rules among peers is performed in the RIF dialects.

  A~special component (\textit{Supervisor}) of the architecture defined in~[1] stores
shared information of the environment, i.\,e., conceptual specifications related to the
domain and to the problem, a~list of the relevant resources, RIF-documents combining
rules for the conceptual specification and a~resource schema mapping.

  Implementation of the conceptual specification includes the following steps:
  \begin{enumerate}[(1)]
\item rewriting of the conceptual documents into the RIF-programs of the peers
performed by the \textit{Supervisor}. The rewriting includes also ($i$)~replacing the
document identifiers (used to mark predicates) by peer identifiers and ($ii$)~adding
schema mapping rules to programs (Fig.~1, middle part); %\\[-14pt]
\item a transfer of the rewritten programs to nodes containing peers relevant to the
respective conceptual documents. The transfer is performed by the \textit{Supervisor}
by calling the method \textit{loadRules} of the respective node wrappers; %\\[-14pt]
\item a transformation of the RIF-programs into the concrete RS or MS languages.
The transformation is performed by the \textit{NodeWrapper} or by the RS or MS
itself (if the RS or MS supports the respective RIF dialect); and %\\[-14pt]
\item an execution of the produced programs in P2P environment.
\end{enumerate}

  During the process of rewriting of the conceptual schema into the resource
programs, the relationships between RIF-documents of the conceptual schema defined
by remote or imported terms are replaced by relationships between peers also defined
by remote or imported terms. To implement remote and imported terms, a~\textit{rule
delegation} mechanism is used to transfer facts and rules from one peer to another.
The details of rule delegation approach including description of the related algorithms
are provided in~[1].

\vspace*{-7pt}

\section{Workflow-Oriented Extension of~the~Multidialect
Infrastructure}

\vspace*{-2pt}

  \noindent
  The aim of the infrastructure proposed is a~conceptual programming of problems in
the RIF-dialects and an implementation of conceptual specifications using rule-based
languages of the RSs and MSs. One of the objectives of this particular paper is to
introduce an extension of the existing multidialect infrastructure~[1] aiming at the
conceptual specification of rule-based workflows.

  Conceptual specification of a~problem (class of problems) is defined in the context
of a~subject domain and consists of a~set of
  RIF-documents. Besides the documents expressed in the logic dialects of RIF, the
documents expressed in the production rule dialect (RIF-PRD) also can be a~part of
conceptual specification of a~problem. In particular, these documents are aimed to
express a~process of solving the problem as the production rule-based workflow.

\vspace*{-4pt}

\subsection{Specification of~workflow orchestration}

  A workflow consists of a~set of tasks orchestrated by specific constructs
(\textit{workflow patterns}~\cite{9-kal}, for instance, \textit{sequence}, \textit{split},
\textit{join}) defining the order of tasks execution. The specification of such
orchestration is called here a~\textit{workflow skeleton}. A~skeleton is defined using
RIF-PRD production rules. Workflows and workflow patterns can be represented
using production rules in various ways, e.\,g., as in~\cite{9-kal, 17-kal}. The
approach applied in this paper to represent workflows requires the extension of
  RIF-PRD dialect by several built-in predicates (they are considered to be a~part of
\textit{wkfl} namespace referenced by
  {\sf http://www.w3.org/2014/rif-workflow-predicate\#} URI similarly to
\textit{func} and \textit{pred} namespaces defined in~\cite{21-kal} for built-in
functions and predicates of RIF):
  \begin{itemize}
\item predicate {\sf wkfl:end-of-task(?arg)} where \textit{\sf ?arg} is an identifier of a~task. The value space of
{\sf ?arg} is the XML-Schema built-in data type
{\sf xsd:Name} representing XML names. The predicate turns into true if a~task
\textit{?arg} has been completed;
\item predicate {\sf wkfl:variable-definition(?arg1\, ?arg2)} where {\sf ?arg1}
is the identifier of a~variable and {\sf ?arg2} is the identifier of a~type of the
variable.
The value space for both arguments is {\sf xsd:Name}. Turning the predicate into
true means that a~variable {\sf ?arg1} of type {\sf ?arg2} is defined in the
context of a~workflow;
\item predicate {\sf wkfl:variable-value(?arg1\,?arg2)} where {\sf ?arg1} is the
identifier of a~variable and {\sf ?arg2} is the value of the variable. The value space for
the first argument is {\sf xsd:Name}, the value space for the second argument is the
union of value spaces of all RIF built-in datatypes. Turning the predicate into true
means that a~variable {\sf ?arg1} has the value {\sf ?arg2};
\item predicate {\sf wkfl:parameter-definition(?arg1\,?arg2\,
?arg3)} where
{\sf ?arg1} is the identifier of a~workflow parameter; {\sf ?arg2} is the identifier
of a~type of the parameter; and {\sf ?arg3} is the direction of the parameter. The value
space for the first and for the second arguments is {\sf xsd:Name}. The value space
for the third argument is {\sf \{IN, OUT, IN\_OUT\}}
(\textit{input}, \textit{output}, or
\textit{input--output} parameter).
Turning the predicate into true means that a~parameter {\sf ?arg1} of
type {\sf ?arg2}, and direction {\sf ?arg3} is defined
for a~workflow; and
\item predicate {\sf wkfl:parameter-value(?arg1\,?arg2)} defines values of
workflow parameters in the same way as {\sf wkfl:variable-value} defines values
of workflow variables.
  \end{itemize}

  Predicates {\sf wkfl:variable-definition} and
  {\sf wkfl:}\linebreak {\sf variable-value} allow
to specify workflow variables and their values and thus to organize the data flow
within a~workflow. Predicates {\sf wkfl:parameter-definition} and
{\sf wkfl:parameter-value} allow to specify workflow parameters and their values
and thus to define the interface of a~workflow in terms of input and output parameters.
Using of workflow parameters and variables is illustrated in the Appendix.

  The predicate {\sf wkfl:end-of-task(?arg)} allows to orchestrate the order of
execution of workflows tasks using conditions and actions of production rules. In this
section, the template rules intended for representation of several basic workflow
patterns (Fig.~2) are provided.

\begin{center}  %fig2
\vspace*{6pt}
\mbox{%
 \epsfxsize=76.913mm
 \epsfbox{kal-2.eps}
 }
  \vspace*{2pt}

{{\figurename~2}\ \ \small{Basic workflow patterns}}
  \end{center}

\vspace*{6pt}


\addtocounter{figure}{1}

  Three well-known workflow patterns are considered below: {\sf Sequence},
{\sf AND-Split}, and {\sf AND-Join}.

  The \textit{AND-Split}\footnote{In this paper, the simplified \textit{presentation
syntax}~\cite{7-kal} is used.} workflow pattern is represented in RIF-PRD by the
following production ruletemplate using {\sf wkfl:end-of-task} predicate:
  \begin{verbatim}
If Not(External(wkfl:end-of-task(A)))
Then Do (Act(A)
 Assert(External(wkfl:end-of-task(A))))
If And(Not(External(wkfl:end-of-task(B)))
 External(wkfl:end-of-task(A)))
Then Do (Act(B)
 Assert(External(wkfl:end-of-task(B))))
If And(Not(External(wkfl:end-of-task(C)))
 External(wkfl:end-of-task(A)))
Then Do (Act(C)
 Assert(External(wkfl:end-of-task(C))))
\end{verbatim}

  The template includes three rules for tasks~$A$, $B$, and~$C$, respectively.
${\sf Act}(A)$, ${\sf Act}(B)$, and ${\sf Act}(C)$ denote \textit{actions} associated with tasks~$A$,
$B$, and $C$. Orchestration (tasks~$B$ and~$C$ are executed concurrently right after
task~$A$ is completed) is specified using {\sf wkfl:end-of-task} predicate in
conditions and {\sf Assert} actions of rules.

  Similarly, the {\sf AND-Split} pattern is represented in RIF-PRD by the
following production rule template:

\vspace*{-1.5pt}

\noindent
  \begin{verbatim}
If Not(External(wkfl:end-of-task(A)))
Then Do (Act(A)
 Assert(External(wkfl:end-of-task(A))))
If And(Not(External(wkfl:end-of-task(B)))
 External(wkfl:end-of-task(A)))
Then Do (Act(B)
 Assert(External(wkfl:end-of-task(B))))
If And(Not(External(wkfl:end-of-task(C)))
 External(wkfl:end-of-task(A)))
Then Do (Act(C)
 Assert(External(wkfl:end-of-task(C))))
\end{verbatim}

\vspace*{-1.5pt}

  The {\sf Sequence} pattern is represented in RIF-PRD by the following
production rule template:

\vspace*{-1.5pt}

\noindent
  \begin{verbatim}
If Not(External(wkfl:end-of-task(A)))
Then Do (Act(A)
 Assert(External(wkfl:end-of-task(A))))
If And(Not(External(wkfl:end-of-task(B)))
 External(wkfl:end-of-task(A)))
Then Do (Act(B)
 Assert(External(wkfl:end-of-task(B))))
\end{verbatim}

\vspace*{-1.5pt}

  More complicated patterns like OR-, XOR- splits and joins, structured loops,
subflows, and others are represented in RIF-PRD similarly.

\vspace*{-6pt}

\subsection{Workflow tasks specification}

  \noindent
  Workflow taskscan be specified as:
  \begin{itemize}
\item separate RIF-documents in various logic RIF-dialects (this is the way how
multidialect infrastructure~[1] is extended with workflow capabilities);\\[-15pt]
\item separate RIF-documents in the RIF-PRD dialect;\\[-15pt]
\item set of production rules embedded into the workflow skeleton; and\\[-15pt]
\item external functions treated as ``black boxes.''
\end{itemize}

  Semantics of tasks specified as multidialect logic programs are defined in
accordance with the RIF-FLD~[3] standard and standards for the respective
  RIF-dialects (BLD, CASPD, etc.). Semantics of tasks specified as production
rule programs are defined in accordance with the RIF-PRD standard. Semantics of
external functions ``are assumed to be specified externally in some document''~[3].

  All kinds of tasks (except those that are embedded into a~workflow skeleton) are
referenced in the workflow skeleton as \textit{external terms}~[3] like
${\sf External}\left({\sf t}\right)$
where term~{\sf t} is defined by an external resource identified by internationalized
resource identifier (IRI)~[3].

\begin{figure*} %fig3
\vspace*{1pt}
 \begin{center}
 \mbox{%
 \epsfxsize=163.675mm
 \epsfbox{kal-3.eps}
 }
 \end{center}
 \vspace*{-9pt}
\Caption{Extended multidialect infrastructure}
\end{figure*}


\vspace*{-6pt}

\subsection{Workflow implementation infrastructure}

  \noindent
  Workflows defined in the conceptual specification are implemented in the
environment shown in Fig.~3. Peer-\linebreak\vspace*{-12pt}

\pagebreak

\noindent
to-peer environment~[1] intended to implement logic
programs is extended with a~production rule-based system (PRS) (for
instance, a~production system compliant with the OMG Production Rule
Representation~\cite{16-kal}) and with external functions, implemented as
  web-services. Implementation of the conceptual specification includes the
following steps:
  \begin{enumerate}[(1)]
\item transfer of the conceptual RIF-documents constituting a~workflow skeleton to
the production rule-based system node (performed by the \textit{Supervisor} component);\\[-14pt]
\item transformation of the conceptual RIF-documents constituting a~workflow
skeleton into the language of the production rule-based system (performed by the PRS
Wrapper component);\\[-14pt]
\item transferring RIF logic programs related to tasks to the relevant nodes of the
environment and transformation of the RIF-programs into the concrete RS or MS
languages~[1]; and\\[-14pt]
\item execution of the workflow.
\end{enumerate}


  The interface of the \textit{Supervisor} includes methods for submitting and
executing a~workflow represented as a~set of RIF-documents, and for getting the
result of the workflow execution.

  To provide a~proof of the multidialect infrastructure concept, a~use case in the
financial domain has been implemented. The problem to be solved in the use case is
called the \textit{investment portfolio diversification problem}. The detailed
description of the use case is included in the Appendix.

\vspace*{-9pt}

\section{Related Work}

  \noindent
  Two types of workflow models, namely, abstract and concrete, were
identified~\cite{15-kal}. In the abstract model, a~workflow is described in an abstract
form, without re-\linebreak\vspace*{-12pt}
\columnbreak

\noindent
ferring to specific resources. In this paper, workflow
representation in abstract and platform-independent  form is suggested.

  A classification model for scientific workflow characteristics~\cite{9-kal}
contributes to better understanding of scientific workflow requirements. The list of
structural patterns discovered during this analysis (including sequential, parallel,
parallel-split, parallel-merge, and mesh) influenced the choice of the required workflow
patterns.

  The OMG standard~\cite{16-kal} reflects an attitude to production rules from the
industrial side providing an OMG MDA (model-driven architecture)
platform-independent model (PIM)  with a~high probability of
support at the PSM (platform-specific model) level from the rule engine vendors.
Similar capabilities though formally defined are used as the basis for the
RIF-PRD~\cite{7-kal}.

  Some vendors of such production rule engines have extended their languages with
the workflow specification capabilities. IBM has extended ILOG to provide the
ruleflow capability. Microsoft supports Windows Workflow Foundation as a~platform
providing the workflow and rules capabilities. The examples of specific formalisms for
PIM rule-based process specifications are also provided in~\cite{17-kal}.

  Comparing to the known variants of the PIM production rule representations,
  selection of the RIF-PRD is considered to be well grounded:
  \begin{enumerate}[(1)]
\item the RIF-PRD is formally defined;
\item RIF ensures support of interoperability of modules written in different
rule-based dialects with different semantics;
\item RIF provides foundations for PIM to PSM semantic preserving
transformation; and
\item RIF also provides ability for specification of the concepts in application
domain terms combining rule-based specifications with the OWL ontologies.
\end{enumerate}

  Importance of providing the interdialect interoperation is advocated
  in~\cite{18-kal} for combining the functionalities of production systems and logic
programs for abductive logic programming (ALP). The ALP framework gives a~model-theoretic semantics to both kinds of rules and provides them with powerful
proof procedures, combining backward and forward reasoning.

  Papers related to RIF-PRD experimentations are focused mainly on the issue of the
PRD programs transformation to an implementation system. In~\cite{19-kal}, a~case
study of bridging the ILOG Rule Language (IRL) to RIF-PRD and vice versa is
considered. In~\cite{20-kal}, implementation of RIF-PRD in three different
paradigms: Answer Set Programming, Production Rules, and Logic Programming
(XSB) is investigated.

  The contribution of this paper with regard to previous works of the authors~[1] consists in
extensions of the infrastructure and specification languages considered in~[1] to the
workflow level.

\section{Concluding Remarks}

  \noindent
  Progress in the investigation of the infrastructure~[1] for the conceptual
multidialect interoperable programming in the abstract, rule-based,
platform-independent notations is reported. An extension of the coherent
combination of the multidialect rule-based programming technique recommended by
the W3C RIF with the approach for unifying modeling of heterogeneous data bases
for their semantic mediation is presented. The extension of the infrastructure and specification
languages considered in~[1] in the direction of the workflow modeling is presented.

  Sticking to the limits of the existing set of the published RIF dialects,
   a~capability of the multidialect workflow support
   is presented with the tasks specified in
semantically different languages mostly suited to the task orientation.
Also, a~realistic problem solving use case containing the interoperating tasks
specified in
several platform-independent rule-based languages: RIF-CASPD, RIF-BLD,
  RIF-PRD, is presented. In addition, OWL~2 is used for the conceptual schema definition,
  RIF-PRD is applied for the workflow orchestration. The platforms selected for
implementation of the tasks include: DLV, SYNTHESIS, IBM ILOG. Such approach
retains well-defined semantics of the platform-independent rule-based languages with
a possibility to check preservation of their semantics by various languages of the
implementing systems. The principle of independence of tasks from the specific IRs is
carried out by the heterogeneous database mediation facilitates contributing to the
  reuse of tasks and workflows. Alongside with the further extension of the
approach, in the future work, the authors plan to apply the conceptual multidialect
programming philosophy for support of the experiments in data intensive sciences. In
particular, they plan to investigate modeling hypotheses in astronomy representing them
as a~set of rules applying the multiplicity of the dialects required.


%\setcounter{equation}{0}

\vspace*{12pt}

{{\hfill \textbf{APPENDIX A}}}

\vspace*{-12pt}

\subsection*{MULTIDIALECT WORKFLOW USE CASE}


{\small


 %\section*{\raggedleft Appendix~A.\\ Multidialect Workflow Use-Case}


%\renewcommand{\thesection}{A\arabic{equation}}

\subsection*{A.1\ Investment portfolio diversification\\
\hspace{20pt}problem extended}

  \noindent
  Motivation of the use case that illustrates the proposed approach comes from the
finance area. The use case extends the \textit{investment portfolio diversification
problem} defined in~[1, Appendix] by adding workflow orchestration applying the
RIF-PRD. The idea of the portfolio diversification problem is
as follows. The portfolio is a~collection of securities of companies, and its size is the
number of securities in the portfolio. The problem is to build a~diversified portfolio of
maximum size. Diversification means that the prices of the securities in portfolio
should be almost independent of each other. If the price of one security falls, it will
not significantly affect the prices of others. Thus, the risk of a~portfolio sharp decrease
is reduced.

  The input data for the problem is a~set of securities and respective time series of
indicators of the security price for each security. Time series for each security is a~set
of pairs $(d, v)$ where $d$ is a~date and~$v$ is an indicator of the security price (for
instance, closing price). The financial services \textit{Google Finance} ({\sf
https://www.google.com/finance}) and \textit{Yahoo! Finance} ({\sf
http://finance.yahoo.com/}) are considered. They include various indicators of the
security price for all trading days of the last decades. For the diversified portfolio, the
securities having noncorrelated time series should be used. Noncorrelation of the
time series means that their correlation is less than some predetermined price
correlation value. The output data for the problem is a~set of subsets of securities of
the maximum size, for which the pair wise correlation will be less than the
predetermined one.

  The maximum satisfying subset of securities is calculated in the following way.
Let~$G$ be a~graph where the vertices are the securities. An edge between two
securities exists if absolute value of their correlation is less than a~specified number.
So, any two securities connected by an edge are considered as noncorrelated. In such
case, the problem of finding the portfolio of the maximum size is exactly the problem
of finding a~maximum clique in an undirected graph. A~maximal clique is a~maximal
portfolio. Note that several different maximal portfolios can be found.

  The conceptual specification of the use case~[1] used two RIF-dialects: RIF-BLD
and RIF-CASPD. The use case was implemented in the environment containing a~mediation system used as a~platform for RIF-BLD~[4] and ASP-based DLV
system~\cite{6-kal}~--- a~platform for RIF-CASPD. The RIF-BLD was used to
specify the problem of data integration, and RIF-CASPD~--- the problem of finding
a~maximum clique in an undirected graph.

In this work, the portfolio use case is extended in the following way. The goal is
not only to build a~set of diversified portfolios, but
also to choose the ``best'' of them
according to some criteria. There are several approaches to choose the most
appropriate portfolio.

  The most recognized one is based on the Markovitz portfolio theory~\cite{10-kal}.
The idea is to choose the portfolio, which has the maximum risk/return ratio. The
most well-known metric to operate with risk/return is Sharpe-ratio~\cite{11-kal}:
  $(r_p -r_f)/\sigma^2$. Here, $r_p$ denotes the expected return of the portfolio;
$r_f$ denotes the~risk free rate; and $\sigma^2$ denotes the~portfolio standard deviation (risk).
The more the Sharpe-ratio is, the better the investment is.

  Another approach is based on an idea that with the advent of social networks, it
became possible to monitor ideas, sentiments, actions of people and lots of available
information has to do with the markets and investments. In~\cite{12-kal}, Bollen
\textit{et al.}\ draw the connection between the mood of investor tweets and the move
of Dow Jones Index, stating that correlation between them is more than 80\%. The
idea of using tweets to assess market movements has been implemented in several
hedge funds.

  Combining these two strategies could provide benefits of both of them, which
leads to the following problem statement: having S\&P500 (a stock market index
maintained by the Standard\,\&\,Poor's, comprising 500~large-cap American
companies) list of companies, compute the diversified portfolio of maximum size
with the best risk/return and sentiment ratios.

%\vspace*{-6pt}

\subsection*{A.2\	Conceptual specification\\
\hspace*{20pt}of~the~application domain\\
\hspace*{20pt}and~the~problem}

%\vspace*{-2pt}

  \noindent
  Conceptual schema (ontology) of the application domain of historical prices of
securities is written in the simplified OWL functional syntax~\cite{8-kal}
({\sf Declaration} keyword is omitted; {\sf property}, {\sf domain}, and {\sf range}
declarations are combined).
  \begin{verbatim}
Ontology(<http://synthesis.ipi.ac.ru/portfolio/
    ontology>
 Class(Portfolio)
  ObjectProperty(securities domain(Portfolio)
   range(Portfolio))
  DataProperty(expected_return domain(Portfolio)
   range(xsd:double))
  DataExactCardinality(1 expected_return
   Portfolio)
  DataProperty(std_dev domain(Portfolio)
   range(xsd:double))
  DataExactCardinality(1 std_dev Portfolio)
  DataProperty(sharpe_ratio domain(Portfolio)
   range(xsd:double))
  DataExactCardinality(1 sharpe_ratio Portfolio)
  DataProperty(twitter_positive_ratio
   domain(Portfolio) range(xsd:double))
  DataExactCardinality(1 twitter_positive_ratio
   Portfolio)
  DataProperty(risk_free_rate domain(Portfolio)
   range(xsd:double))
  DataExactCardinality(1 risk_free_rate
   Portfolio)
  DataProperty(recommended domain(Portfolio)
   range(xsd:boolean))
  DataExactCardinality(1 recommended Portfolio)

 Class(Security)
  DataProperty(ticker  domain(Security)
   range(xsd:string))
  DataExactCardinality(1 ticker Security)
  DataProperty(rates  domain(Security)
   range(StockRate))
  DataProperty(positive_tweets domain(Security)
   range(xsd:double))
  DataExactCardinality(1 positive_tweets
   Security)
  DataProperty(sec_expected_return
   domain(Security) range(xsd:double))
  DataExactCardinality(1 sec_expected_return
   Security)
  DataProperty(sec_std_dev domain(Security)
   range(xsd:double))
  DataExactCardinality(1 sec_std_dev Security)

 Class(StockRate)
  DataProperty(date domain(StockRate)
   range(xsd:date))
  DataExactCardinality(1 date StockRate)
  DataProperty(price domain(StockRate)
   range(xsd:double))
  DataExactCardinality(1 price StockRate)
)
  \end{verbatim}

  \vspace*{-6pt}

  A~portfolio (the {\sf Portfolio} class) is characterized by a~set of securities
({\sf securities} attribute) contained in the portfolio, by several metrics: expected
return ({\sf expected\_return} attribute), standard deviation ({\sf std\_dev}
attribute), Sharpe ratio ({\sf sharpe\_ratio attribute}), risk free rate
({\sf risk\_free\_rate} attribute), and
ratio of positive tweets mentioning securities of
the portfolio ({\sf twitter\_positive\_ratio} attribute).

  A security (the {\sf Security} class) is characterized by identifier ({\sf ticker}
attribute), time series of historical prices (attribute {\sf
rates}), ratio of positive
tweets mentioning the security ({\sf positive\_tweets} attribute), expected return
({\sf sec\_expected\_return} attribute), and standard deviation ({\sf sec\_std\_dev}
attribute).

\begin{figure*} %fig4
\vspace*{1pt}
 \begin{center}
 \mbox{%
 \epsfxsize=126.24mm
 \epsfbox{kal-4.eps}
 }
 \end{center}
 \vspace*{-11pt}
\Caption{Portfolio workflow}
\vspace*{-6pt}
  \end{figure*}

The workflow of the extended portfolio problem is demonstrated in Fig. 4. The workflow
contains six tasks\footnote{To save space, specifications are provided only for
{\sf getPortfolios}, {\sf getPositiveTweetRatio}, and
{\sf computePortfolioTwitterMetrics} tasks.}:
\begin{enumerate}[(1)]
\item {\sf getPortfolios}. A~set of diversified portfolio candidates is computed. The
multidialect task specification consists of two RIF-documents in BLD and CASPD
dialects~[1, Appendix]. Portfolios received as a~result contain only security tickers,
they have to be augmented by financial and sentiments ratios;
\item {\sf getPositiveTweetRatio}. This task is responsible for computing a~sentiment ratio of tweets for every security. Every tweet is assessed
to be positive,
negative, or neutral. The task is specified as a~call of external function;
\item {\sf computePortfolioTwitterMetrics}. The portfolio sentiment ratio is
computed as the average of its securities sentiment ratio. The task is specified using
RIF-PRD;
{\looseness=1

}
\item {\sf getSecurityFinancialMetrics}. For every security in a~portfolio the
financial rates (the {\sf expected return} and the {\sf standard deviation}) are
calculated on the basis of historical rates of securities specified as an OWL~2 class of
the ontology of the application domain. The task is specified using RIF-BLD dialect;
\item {\sf computePortfolioFinancialMetrics}. The computation of the portfolio
expected return, risk, and Sharpe-ratio is done within this task. The task is specified
using RIF-PRD dialect; and
\item {\sf choosePortfolio}. The best portfolio is chosen according to maximizing
the (\textit{Sharpe ratio * sentiment ratio}) coefficient. The task is specified using
RIF-PRD dialect.
  \end{enumerate}

  Workflow skeleton is specified as a~RIF-PRD document importing the ontology of
the application domain:
  \begin{verbatim}
Document( Dialect(RIF-PRD)
 Base(<http://synthesis.ipi.ac.ru/portfolio/
  workflow#>)
 Import(<http://synthesis.ipi.ac.ru/portfolio/
  ontology#>
 <http://www.w3.org/ns/entailment/OWL-Direct>)
Prefix(ont<http://synthesis.ipi.ac.ru/portfolio/
 ontology#>)
Prefix(ofws<http://synthesis.ipi.ac.ru/
 synthesis/projects/RuleInt/OpinionFinderWS#>)
Prefix(mws<http://synthesis.ipi.ac.ru/
 synthesis/projects/RuleInt/MediatorWS#>)

Group 2 (
 Do(
  Assert(External(wkfl:parameter-definition(
   startDatexsd:string IN)))
  Assert(External(wkfl:parameter-definition(
   endDatexsd:string IN)))
  Assert(External(wkfl:parameter-definition(
   bestPortfolioont:Portfolio OUT)))
  Assert(External(wkfl:variable-definition(
   ps  List<ont:Portfolio> IN)))
  Assert(External(wkfl:
   variable-value(ps List())))
 )
)
\end{verbatim}

\noindent
\begin{verbatim}

Group 1 (
 Forall ?sd ?ed such that (
  External(wkfl:parameter-value(startDate ?sd))
  External(wkfl:parameter-value(endDate ?ed))  )
( If Not(External(wkfl:
   end-of-task(getPortfolios)))
  Then
   Do( Modify(External(wkfl:variable-value(ps
    External(mws:getPortfolios(?sd ?ed) )))
   Assert(External(wkfl:
    end-of-task(getPortfolios))) )
 )

 Forall ?ps ?p ?scs ?s ?t such that (
  External(wkfl:variable-value(ps ?ps))
  ?p#?ps  ?p[securities->?scs]
   ?s#?scs ?s[ticker->?t] )
( If And( Not(External(wkfl:
     end-of-task(getTweets)))
   External(wkfl:end-of-task(getPortfolios)))
  Then
  Do( Modify(?s[positive_tweets->
   External(ofws:computeSecPosTweets(?t))] )
   Assert(External(wkfl:
    end-of-task(getTweets))) )
)

Forall ?ps ?p such that (
 External(wkfl:variable-value(ps  ?ps))
  ?p#?ps)
( If And(Not(External(wkfl:
   end-of-task(countTwitterMetrics)))
   External(wkfl:end-of-task(getTweets)) )
  Then Do(
   Modify(?p[twitter_positive_ratio->
    External(func:numeric-divide(
    Sum{?pt | Exists
     ?scs ?s(?p[securities->?scs]
     ?s#?scs  ?s[positive_tweets->?pt])}
    External(func:count(?ps))))])
   Assert(External(wkfl:
    end-of-task(countTwitterMetrics)))
)	)) )
\end{verbatim}

\begin{figure*}[b] %fig5
\vspace*{-4pt}
 \begin{center}
 \mbox{%
 \epsfxsize=162.319mm
 \epsfbox{kal-5.eps}
 }
 \end{center}
 \vspace*{-9pt}
\Caption{Portfolio problem implementation infrastructure}
  \end{figure*}

  Production rules of the document are divided into two groups. The first group with
priority~2 contains rules defining workflow parameters and variable. Input parameters
are \textit{start date} and \textit{end date} of historical rates used for calculation of
\textit{portfolio metrics}. Workflow variable {\sf ps} denotes a~set containing
\textit{portfolio candidates}.

  The second group with priority~1 contains the orchestration rules~--- workflow
skeleton. The only orchestration rule provided in the example above corresponds to
the task {\sf getPortfolios}. The external function {\sf getPortfolios}
encapsulates a~multidialect logic program calculating portfolio candidates~[1,
Appendix]. A~{\sf Modify} action is used to call the function and to put the
returned result into the {\sf ps} variable.

\vspace*{-6pt}

\subsection*{A.3\	Revised portfolio problem infrastructure}

  \noindent
  The implementation structure of the use case is shown in Fig.~5.




  The RIF-PRD workflow skeleton was transformed into a~program (rule set) in the
ILOG~\cite{13-kal} language combining production rules and workflow facilities
(like {\sf fork} and {\sf sequence}). The ILOG program was executed in the
{IBM Operational Decision Manager} tool~\cite{24-kal}. In order to execute
ILOG programs, the underlying execution model (XOM)~\cite{25-kal}
was defined as a~set of
Java classes: {\sf Portfolio}, {\sf Security}, and {\sf StockRate}. The
{\sf Portfolio class} corresponds to a~financial portfolio and contains as attributes a~set of
securities in it, its expected return, standard deviation, Sharpe ratio, and twitter
positive ratio. Code of this class is provided below:
  \begin{verbatim}
public class Portfolio {
 private Collection<Security> securities;
 private double expected_return;
 private double std_dev;
 private double sharpe_ratio;
 private double twitter_positive_ratio;
 // as of 05.04.14 US 5-year treasuries
 private static double risk_free_rate = 0.0169;
 private boolean recommended;
}
\end{verbatim}

  Class {\sf Security} corresponds to real world financial securities. The class
contains as attributes a~ticker, ratio of positive tweet number to the sum of positive
and negative tweets, a~set of stock rates, security's standard deviation, and expected
return. These attributes are set as responses to corresponding web services queries:

\vspace*{-2pt}

\noindent
  \begin{verbatim}
public class Security {
 public String ticker;
 public double positive_tweets;
 public Collection<StockRate> rates;
 public double std_dev;
 public double expected_return;
 public static int number_of_periods = 5;
}	
\end{verbatim}

\vspace*{-2pt}

{\sf StockRate} is a~simple class and contains just two attributes~--- price and date:

\vspace*{-2pt}

\noindent
  \begin{verbatim}
public class StockRate {
 public float price;
 public String date;
}
\end{verbatim}

\vspace*{-2pt}

  It is easy to see that the one-to-one mapping exists between conceptual schema
entities and execution model entities.

  Parameters of RIF-PRD workflow skeleton ({\sf startDate}, {\sf endDate}, and
{\sf bestPortfolio}) are mapped into the respective parameters of ILOG rule set
(Fig.~6).

\begin{figure*} %fig6
\vspace*{1pt}
 \begin{center}
 \mbox{%
 \epsfxsize=115mm
 \epsfbox{kal-6.eps}
 }
 \end{center}
 \vspace*{-9pt}
\Caption{Rule set parameters}
  \end{figure*}

  The variable of RIF-PRD workflow skeleton ({\sf ps}) is mapped into a~local variable
of the rule set. Specification of the variable looks as follows:

\vspace*{-2pt}

\noindent
  \begin{verbatim}
<?xml version="1.0" encoding="UTF-8"?>
<ilog.rules.studio.model.base:VariableSetxmi:
  version="2.0"
xmlns:xmi="http://www.omg.org/XMI"
  xmlns:ilog.rules.studio.model.base =
"http://ilog.rules.studio/model/base.ecore">
 <name>local_vars</name>
 <variables name="ps" type="java.util.ArrayList"
   initialValue=""verbalization="ps"/>
</ilog.rules.studio.model.base:VariableSet>
\end{verbatim}

  Rules of the RIF-PRD workflow skeleton are mapped into ILOG
\textit{ruleflow}~\cite{25-kal}:

\vspace*{-6pt}

\noindent
  \begin{verbatim}
flowtask portfolio$_$flow {
 property mainflowtask = true;
 property ilog.rules.business_name =
  "portfolio_flow";
 body {
  portfolio$_$flow#getPortfolios;
  fork {
   portfolio$_$flow#getRates;
   portfolio$_$flow
   #computePortfolioFinancialMetrics;} &&
  { portfolio$_$flow#getTweets;
   portfolio$_$flow#
    computePortfolioTwitterMetrics;}
  portfolio$_$flow#choosePortfolio;
 }
};

ruletask portfolio$_$flow#getPortfolios {
 property ilog.rules.business_name =
   "portfolio_flow>getPortfolios";
 body { getPortfolios.*}
};

ruletask portfolio$_$flow#
   computePortfolioTwitterMetrics {
 propertyilog.rules.business_name =
  "portfolio_flow>
   computePortfolioTwitterMetrics";
 body { computePortfolioTwitterMetrics.* }
};

ruletask portfolio$_$flow#getTweets {
 property ilog.rules.business_name =
  "portfolio_flow>getTweets";
 property ilog.rules.package_name = "";
 body {getTweets.*}
};
\end{verbatim}

  The {\sf computePortfolioTwitterMetrics},
{\sf computePortfolioFinancialMetrics}, and {\sf choosePortfolio} tasks are
implemented as production rules in ILOG:

\vspace*{-6pt}

\noindent
  \begin{verbatim}
package computePortfolioTwitterMetrics {
 use ps;
 import portfolio.*;

 rule computePortfolioTwitterMetrics {
  property status = "new";
  when {	IlrContext() from ?context;	}
  then {
   foreach (Portfolio p in ps) {
    double ?twitter_metrics = 0;
    int ?length = 0;
     foreach (Security security
       in p.securities) {
      ?twitter_metrics= ?twitter_metrics +
       security.positive_tweets;
      ?length = ?length + 1; }
     p.twitter_positive_ratio=
      ?twitter_metrics / ?length;
}}}}
\end{verbatim}

  The {\sf getPortfolios} and {\sf computeSecurityFinancialMetrics} tasks are
implemented by the following production rules in ILOG:


\noindent
  \begin{verbatim}
package getPortfolios {
 use ps;
 import portfolio.*;

 rule getPortfolios {
  when { IlrContext() from ?context; }
  then {
   ps = Supervisor.getPortfolios(startDate,
    endDate);
} } }
\end{verbatim}

\begin{table*}\small
\begin{center}
\Caption{Metrics for the securities}
  \vspace*{2ex}

  \begin{tabular}{cccc}
  \hline
Security identifier&Expected return&Standard deviation&Positive tweet ratio\\
\hline
COG&0.163&0.201&0.507\\
DO&0.015&0.019&0.651\\
EQR&0.150&0.022&0.846\\
FOSL&0.513&0.030&0.579\\
SCG&0.050&0.010&0.622\\
\hline
\end{tabular}
\end{center}
%\end{table*}
%\begin{table*}\small
\begin{center}
\Caption{Metrics for the portfolio candidates}
  \vspace*{2ex}

  \begin{tabular}{lcccccc}
  \hline
\multicolumn{1}{c}{\tabcolsep=0pt\begin{tabular}{c}Portfolio\\ identifier\end{tabular}}&
\tabcolsep=0pt\begin{tabular}{c}Expected\\ return\end{tabular}&
\tabcolsep=0pt\begin{tabular}{c}Standard\\ deviation\end{tabular}&
\tabcolsep=0pt\begin{tabular}{c}Risk free\\ rate\end{tabular}&
\tabcolsep=0pt\begin{tabular}{c}Sharpe\\ ratio\end{tabular}&
\tabcolsep=0pt\begin{tabular}{c}Positive\\ tweet ratio\end{tabular}&
\tabcolsep=0pt\begin{tabular}{c}Sharpe ratio\\$\times$\;Positive tweet ratio\end{tabular}\\
\hline
1&0.111&0.008&0.0169&11.755&0.660&7.758\\
2&2.400&0.507&0.0169&\hphantom{9}4.701&0.508&2.388\\
3&2.381&0.508&0.0169&\hphantom{9}4.662&0.557&2.597\\
4&2.347&0.505&0.0169&\hphantom{9}4.606&0.708&3.261\\
5 (best)&0.178&0.011&0.0169&14.227&0.641&9.120\\
6&0.147&0.008&0.0169&15.577&0.521&8.166\\
\hline
\end{tabular}
\end{center}
\vspace*{-3pt}
\end{table*}



\noindent
  Here, the {\sf Supervisor} is the~Java class wrapping execution of logic programs
in multidialect infrastructure including two nodes~[1]. The nodes correspond to the
mediation system (which integrates \textit{Google Finance} and the \textit{Yahoo!
Finance} services) and to a~rule-based programming system DLV.

  The {\sf getSecurityFinancialMetrics} task uses the same instance of the
mediation system as the {\sf getPortfolios} task. The reason is that financial
metrics are calculated using the historical rates of the securities. This is exactly the
information that is extracted by the mediation system from {Google Finance}
and {Yahoo! Finance}. The difference between two tasks is that the
{\sf getPortfolios} is implemented as a~submission of a~query to the DLV node, but
the {\sf getSecurityFinancialMetrics} is implemented as a~submission of a~different
query to the Mediation Node.

  The {\sf getPositiveTweetRatio} task is implemented by the following
production rule in ILOG:

%\pagebreak

\noindent
  \begin{verbatim}
package getTweets {
 use ps;
 import portfolio.*;

 rule getTweets {
  when { IlrContext() from ?context; }
  then {
   foreach (Portfolio p in ps) {
    foreach (Security s in p.securities) {
     s.positive_tweets =
      WebServices.computeSecPosTweets(s.ticker);
} } } } }
\end{verbatim}




\noindent
Here, {\sf WebServices} is the~Java-class wrapping invocation of a~web-service.
The WSDL specification of the web-service can be found at {\sf
http://synthesis.ipi.ac.ru/synthesis/ projects/RuleInt/OpinionFinderWS}. The
  web-service, in its turn, encapsulates a~Java-program. The program first collects
tweets using the {\sf Twitter Streaming API}. After that, a~sentiment analysis is
done by the {\sf Polarity Classifier} of the {\sf OpinionFinder}
  tool~\cite{14-kal} which assesses if tweet is positive, negative, or neutral. Finally,
the sentiment ratio for every security in a~portfolio is calculated and returned as the
result.

\vspace*{-6pt}

\subsection*{A.4\	Result of~the~use case workflow execution}

  \noindent
  The results obtained by one of the use case runs are as follows. The task
{\sf getPortfolios} computes portfolio candidates on the basis of historical rates of
daily closing prices of securities from S\&P500 list for the 2011--2013. Six portfolios
of size~5 were calculated. Each portfolio is a~set of identifiers (tickers) of
companies:
  \begin{verbatim}
Candidate 1: { ALXN, BF.B, EW, POM, VNO }
Candidate 2: { BMC, JBL, LUK, MNST, POM }
Candidate 3: { AVP, BMC, JPL, MNST, POM }
Candidate 4: { ALTR, BF.B, BMC, DGX, PEG }
Candidate 5: { COG, DO, EQR, FOSL, SCG }
Candidate 6: { ADSK, GILD, INTC, POM, TJX }
\end{verbatim}

  The task {\sf getSecurityFinancialMetrics} computes the expected return and
the standard deviation for every security mentioned in portfolio candidates.
 The task
{\sf getPositiveTweetRatio} computes positive sentiment ratios for every security
mentioned in portfolio candidates (500~latest tweets for every security were used for
the computation). Financial and twitter metrics for several securities are provided in
Table~1.



  The task {\sf computePortfolioFinancialMetrics} computes financial metrics for
every portfolio candidate on the basis of respective metrics for
securities in a~portfolio. The task {\sf computePortfolioTwitterMetrics} computes sentiment
metrics for every portfolio candidate on the basis of sentiment metrics for securities in
a~portfolio. Financial and twitter metrics for portfolio candidates are provided in
Table~2. The task {\sf choosePortfolio} identifies the best portfolio by maximum
value of the products of Sharpe ratio and positive tweet ratio obtained for every
portfolio (see Table~2).


}

\vspace*{-9pt}

\Ack
\noindent
This research has been done under the support of the \mbox{RFBR} (projects13-07-00579,
14-07-00548) and the Program for Basic Research of the Presidium of RAS.

\renewcommand{\bibname}{\protect\rmfamily References}

\vspace*{-9pt}

{\small\frenchspacing
{%\baselineskip=10.8pt
\begin{thebibliography}{99}

\bibitem{1-kal}
\Aue{Kalinichenko, L.\,A., S.\,A.~Stupnikov, A.\,E.~Vovchenko, and D.\,Y.~Kovalev}.
2013. Conceptual declarative problem specification and solving in data intensive domains.
\textit{Informatics and Applications}~--- \textit{Inform \mbox{Appl.}} 7(4):112--139.
Available at: {\sf http://synthesis.ipi.ac.\linebreak ru/synthesis/publications/13ia-multidialect}
 (accessed December~9, 2014).
\bibitem{2-kal}
\Aue{Kalinichenko, L.\,A., S.\,A.~Stupnikov, and D.\,O.~Martynov}. 2007.
\textit{SYNTHESIS: A~language for canonical information modeling and mediator
definition for problem solving in heterogeneous information resource environments}.
Moscow: IPIRAN. 171~p.
\bibitem{3-kal}
Boley, H., and M.~Kifer, eds. 2013. {RIF framework for logic dialects. W3C
recommendation}. 2nd ed. Available at:
{\sf http://www.w3.org/TR/2013/REC-rif-fld-20130205/}
(accessed December~9, 2014).

\bibitem{4-kal}
Boley, H., and M.~Kifer, eds. 2013. {RIF basic logic dialect. W3C Recommendation}.
2nd ed. Available at:
{\sf http://www.w3.org/TR/2013/REC-rif-bld-20130205/}
(accessed December~9, 2014).


\bibitem{5-kal}
Heymans, S., and M.~Kifer, eds. 2009. {RIF core answer set programming dialect}.
Available at: {\sf http:// ruleml.org/rif/RIF-CASPD.html} (accessed November~5, 2014).
\bibitem{6-kal}
\Aue{Leone, N., G.~Pfeifer, W.~Faber, T.~Eiter,  G.~Gottlob, S.~Perri, and F.~Scarcello}.
2006. The DLV system for knowledge representation and reasoning. \textit{ACM Trans.
Comput. Logic} 7(3):499--562.
\bibitem{7-kal}
DeSante, M.\,C., G.~Hallmark, and A.~Paschke, eds. 2013. {RIF production rule
dialect. W3C Recommendation}. 2nd ed.
Available at: {\sf http://www.w3.org/TR/2013/REC-rif-prd-20130205/}
(accessed December~9, 2014).

\bibitem{8-kal}
Motik, B., P.\,F.~Patel-Schneider, and B.~Parsia, eds.
2012. {OWL~2 Web Ontology Language structural
specification and functional-style syntax. W3C Recommendation}. 2nd ed.
Available at:
{\sf http://www.w3.org/TR/owl2-syntax/} (accessed November~5, 2014).
\bibitem{22-kal} %9
\Aue{Calvanese, D., G.~De Giacomo, D.~Lembo, M.~Lenzerini, A.~Poggi, and
R.~Rosati}.
 2007. Ontology-based database access. \textit{15th
Italian Symposium on Advanced Database Systems Proceedings}. 324--331.

\bibitem{9-kal} %10
\Aue{Ramakrishnan, L., and B.~Plale}. 2010. A~multi-dimensional classification model for
scientific workflow\linebreak characteristics. \textit{1st Workshop (International) on Workflow
Approaches to New Data-Centric Science Proceedings}. New York: ACM.
Article No.\,4. 12~p.
Available at: {\sf http://dl.acm.org/citation.cfm?id=1833402}\linebreak
(accessed December~9, 2014).

\bibitem{17-kal} %11
\Aue{Boukhebouze, M, Y.~Amghar, A.-N.~Benharkat,  and Z.~Maamar}. 2011.
A~rule-based approach to model and verify flexible business processes. \textit{Int.
J.~Business Process Integration Management} 5(4):287--307.

\bibitem{21-kal} %12
Polleres, A., H.~Boley, and M.~Kifer, eds. 2013. {RIF datatypes and Built-Ins~1.0
W3C Recommendation.} 2nd ed.
Available at: {\sf http://www.w3.org/TR/2013/REC-rif-dtb-20130205/}
(accessed December~9, 2014).




\bibitem{16-kal} %13
Production Rule Representation (PRR), Version 1.0. OMG Document Number:
formal/2009-12-01. Available at: {\sf http://www.omg.org/spec/PRR/1.0} (accessed
November~5, 2014).

\bibitem{15-kal} %14
\Aue{Yu,~J., and R.~Buyya}. 2005. A~taxonomy of scientific workflow systems for grid
computing. \textit{ACM SIGMOD Records} 34(3):44--49.

\bibitem{18-kal} %15
\Aue{Kowalski, R., and F.~Sadri}. 2009. Integrating logic programming and production
systems in abductive logic programming agents.
\textit{Web reasoning and rule systems}. Eds. A.~Polleres and T.~Swift.
Lecture notes in computer science ser. Springer.
5837:1--23.

\bibitem{19-kal} %16
\Aue{Cosentino, V., M.\,D.~Del Fabro, and A.~El Ghali}. 2012. A~model driven approach
for bridging ILOG rule language and RIF. \textit{6th Symposium (International) on Rules
RuleML Proceedings}. CEUR-WS.org. 874:96--102.
\bibitem{20-kal} %17
\Aue{Veiga, F.\,D.\,J.} 2011. Implementation of the RIF-PRD.  Universidade
Nova de Lisboa. Master Thesis. Available at: {\sf
http://run.unl.pt/bitstream/10362/6310/1/Veiga\_\linebreak 2011.pdf} (accessed November~5, 2014).

\bibitem{10-kal} %18
\Aue{Markowitz, H.\,M.} 1991. \textit{Portfolio selection: Efficient diversification of
investments}. Wiley. 402~p.
\bibitem{11-kal} %19
\Aue{Sharpe, W.\,F.} 1966. Mutual fund performance. \textit{J.~Business}
39(S1):119--138.
\bibitem{12-kal} %20
\Aue{Bollen,~J., H.~Maoa, and X.~Zeng}. 2011. Twitter mood predicts the stockmarket.
\textit{J.~Comput. Sci.} 2(1):1--8.
\bibitem{13-kal} %21
IBM WebSphere ILOG JRules Version~7.0. Online documentation. Available at: {\sf
http://pic.dhe.ibm.com/\linebreak infocenter/brjrules/v7r0/index.jsp} (accessed November~5, 2014).

\bibitem{24-kal} %22
IBM Operational Decision Manager. Available at:
{\sf http:// www-03.ibm.com/software/products/en/odm} (accessed November~5, 2014).
\bibitem{25-kal} %23
IBM Operational Decision Manager Version~8.5 Information Center. Available at: {\sf
http://pic.dhe.ibm.com/\linebreak infocenter/dmanager/v8r5/index.jsp} (accessed November~5, 2014).


\bibitem{14-kal} %24
\Aue{Wilson, T., J.~Wiebe, and P.~Hoffmann}. 2005.
Recognizing contextual polarity in phrase-level sentiment Analysis. \textit{Conference on
Human Language Technology and Empirical Methods in Natural Language Processing
Proceedings}. Stroudsburg: Association for Computational Linguistics. 347--354.


%\bibitem{23-kal}
%Bock, C., \textit{et. al.}, eds. 2012. \textit{OWL~2 Web Ontology Language Structural
%Specification and Functional-Style Syntax. W3C Recommendation}. 2nd ed.


\end{thebibliography} } }

\end{multicols}

\vspace*{-9pt}

\hfill{\small\textit{Received November 3, 2014}}

\vspace*{-18pt}

\Contr

\noindent
\textbf{Kalinichenko Leonid A.} (b.\ 1937)~---
 Doctor of Science in physics and mathematics, professor;
 Head of Laboratory, Institute of Informatics Problems, 44-2 Vavilov Str.,
 Moscow 119333, Russian Federation; professor,
 Faculty of Computational Mathematics and Cybernetics, M.\,V.~Lomonosov Moscow
 State University, 1-52 Leninskiye Gory, GSP-1, Moscow 119991,
 Russian Federation; leonidandk@gmail.com

 \vspace*{3pt}

 \noindent
 \textbf{Stupnikov Sergey A.} (b.\ 1978)~---
 Candidate of Science (PhD) in technology, senior scientist,
 Institute of Informatics Problems, Russian Academy of Sciences,
 44-2 Vavilov Str.,
 Moscow 119333, Russian Federation; ssa@ipi.ac.ru

 \vspace*{3pt}

 \noindent
 \textbf{Vovchenko Alexey E.} (b.\ 1984)~---
 Candidate of Science (PhD) in technology, senior scientist,
 Institute of Informatics Problems, Russian Academy of Sciences,
 44-2 Vavilov Str.,
 Moscow 119333, Russian Federation; itsnein@gmail.com

 \vspace*{3pt}

 \noindent
 \textbf{Kovalev Dmitry Yu.} (b.\ 1988)~---
 junior scientist, Institute of Informatics Problems, Russian Academy of Sciences,
 44-2 Vavilov Str.,
 Moscow 119333, Russian Federation; dm.kovalev@gmail.com


%\vspace*{24pt}

%\hrule

%\vspace*{2pt}

%\hrule

%\vspace*{-6pt}

\newpage


\def\tit{КОНЦЕПТУАЛЬНОЕ МОДЕЛИРОВАНИЕ МУЛЬТИДИАЛЕКТНЫХ ПОТОКОВ РАБОТ$^*$}

\def\aut{Л.\,А.~Калиниченко$^{1,2}$, С.~Ступников$^1$, А.~Вовченко$^1$, Д.~Ковалев$^1$}


\def\titkol{Концептуальное моделирование мультидиалектных потоков работ}

\def\autkol{Л.\,А.~Калиниченко, С. Ступников, А. Вовченко, Д. Ковалев}

{\renewcommand{\thefootnote}{\fnsymbol{footnote}}
\footnotetext[1]{Работа выполнена при поддержке РФФИ (проекты
13-07-00579, 14-07-00548) и~Программы фундаментальных исследований Президиума РАН.}}


\titel{\tit}{\aut}{\autkol}{\titkol}

\vspace*{-12pt}

\noindent
$^1$Институт проблем информатики Российской академии наук

\noindent
$^2$Московский государственный университет им.\ М.\,В.~Ломоносова, факультет вычислительной
матема-\linebreak
$\hphantom{^1}$тики и~кибернетики

\vspace*{6pt}

\def\leftfootline{\small{\textbf{\thepage}
\hfill ИНФОРМАТИКА И ЕЁ ПРИМЕНЕНИЯ\ \ \ том\ 8\ \ \ выпуск\ 4\ \ \ 2014}
}%
 \def\rightfootline{\small{ИНФОРМАТИКА И ЕЁ ПРИМЕНЕНИЯ\ \ \ том\ 8\ \ \ выпуск\ 4\ \ \ 2014
\hfill \textbf{\thepage}}}


\Abst{Рассматриваются методы концептуального представления
алгоритмов анализа данных, средств интеграции данных, а~также процессов,
направленных на спецификацию семантики данных и~поведения в~единой парадигме.
Расширяется новый подход к~применению комбинации семантически различных
плат\-фор\-мо\-не\-за\-ви\-си\-мых языков на правилах (диалектов) для создания
интероперабельных концептуальных спецификаций над различными системами на правилах.
Подход опирается на методику трансформации программ на правилах, рекомендованную
стандартом W3C Rule Interchange Format (RIF). Подход, предлагаемый в~стандарте RIF,
сочетается со технологией семантической интеграции неоднородных баз данных
в~предметных посредниках. Статья расширяет предыдущие исследования авторов
в~направлении моделирования потоков работ для определения композиций
алгоритмических модулей в~процессной структуре. Рассмотрены возможности
спецификации задач в~мультидиалектных потоках работ с~применением семантически
различных языков, наиболее подходящих для конкретных задач. Приведен практический
пример потока работ, задачи которого специфицированы с~использованием нескольких
 языков на правилах (RIF-CASPD, RIF-BLD, RIF-PRD). Для определения концептуальной
 схемы использован язык OWL~2, для оркестровки потока работ использован язык
 RIF-PRD. Инфраструктура реализации примера включает систему на продукционных
 правилах (IBM ILOG), систему на логических правилах (DLV) и~предметный посредник.}

\KW{концептуальные спецификации; потоки работ; RIF; языки продукционных правил;
интеграция баз данных; посредники; PRD; мультидиалектная инфраструктура}

\DOI{10.14357/19922264140413}

\vspace*{6pt}


 \begin{multicols}{2}

\renewcommand{\bibname}{\protect\rmfamily Литература}
%\renewcommand{\bibname}{\large\protect\rm References}

{\small\frenchspacing
{%\baselineskip=10.8pt
\begin{thebibliography}{99}

\bibitem{1-kal-1}
\Au{Kalinichenko L.\,A., Stupnikov S.\,A.. Vovchenko~A.\,E.,
Kovalev~D.\,Y.}
Conceptual declarative problem specification and solving in data intensive domains~//
Информатика и~её применения, 2013. Т.~7. Вып.~4. С.~112--139.
{\sf http://synthesis.ipi.ac.ru/synthesis/publications/13ia-multidialect}.
\bibitem{2-kal-1}
\Au{Kalinichenko L.\,A., Stupnikov~S.\,A., Martynov~D.\,O.}
 SYNTHESIS: A~language for canonical information modeling and mediator definition
 for problem solving in heterogeneous information resource environments.~---
 Moscow: IPI RAN, 2007. 171~p.
\bibitem{3-kal-1}
RIF framework for logic dialects. W3C Recommendation~/
Eds. H.~Boley, M.~Kifer. 2nd ed.
{\sf http:// www.w3.org/TR/2013/REC-rif-fld-20130205/}.
\bibitem{4-kal-1}
RIF basic logic dialect. W3C Recommendation~/
Eds. H.~Boley, M.~Kifer. 2nd ed.
{\sf http://www.w3.org/ TR/2013/REC-rif-bld-20130205/}.
\bibitem{5-kal-1}
RIF core answer set programming dialect~/
Eds.\ S.~Heymans, M.~Kifer, 2009. {\sf  http://ruleml.org/rif/RIF-CASPD.html}.
\bibitem{6-kal-1}
\Au{Leone N., Pfeifer G., Faber~W., Eiter~T., Gottlob~G., Perri~S., Scarcello~F.}
The DLV system for knowledge representation and reasoning~//
 ACM Trans. Comput. Logic, 2006. Vol.~7. No.\,3. P.~499--562.
\bibitem{7-kal-1}
RIF production rule dialect. W3C Recommendation~/
Eds.\ De Sante Marie~C., Hallmark~G., A.~Paschke.~ 2nd ed.
{\sf http://www.w3.org/TR/2013/REC-rif-prd-20130205/}.
\bibitem{8-kal-1}
OWL~2 Web Ontology Language Structural Specification and Functional-Style Syntax.
W3C Recommendation~/ Eds. B.~Motik, P.\,F.~Patel-Schneider, B.~Parsia. 2nd ed.
{\sf http://www.w3.org/TR/owl2-syntax/}.

\bibitem{22-kal-1} %9
\Au{Calvanese, D., De Giacomo~G., Lembo~D., Lenzerini~M., Poggi~A.,
Rosati~R.}
Ontology-based database access~// 15th Italian Symposium on Advanced
Database Systems Proceedings, 2007. P.~324--331.

\bibitem{9-kal-1} %10
\Au{Ramakrishnan L., Plale~B.}
A~multi-dimensional classification model for scientific workflow characteristics~//
1st  Workshop (International) on Workflow Approaches to New Data-Centric Science
Proceedings.  New York: ACM, 2010. Aricle No.\,4. 12~p.
{\sf http://dl. acm.org/citation.cfm?id=1833402}.
\pagebreak

\bibitem{17-kal-1} %11
\Au{Boukhebouze M., Amghar~Y., Benharkat~A.-N., Maamar~Z.}
A~rule-based approach to model and verify flexible business processes~//
Int. J.~Business Process Integration Management, 2011. Vol.~5. No.\,4. P.~287--307.

\bibitem{21-kal-1} %12
RIF Datatypes and Built-Ins 1.0. W3C Recommendation~/
Eds.\ A.~Polleres, H.~Boley, M.~Kifer. 2nd ed.
{\sf http://www.w3.org/TR/2013/REC-rif-dtb-20130205/}.



%\pagebreak

\bibitem{16-kal-1} %13
Production Rule Representation (PRR), Version~1.0.
OMG Document Number: formal/2009-12-01. {\sf http:// www.omg.org/spec/PRR/1.0}.

\bibitem{15-kal-1} %14
\Au{Yu J., Buyya~R.}
 A~taxonomy of scientific workflow systems for grid computing~//
 ACM SIGMOD Records, 2005. Vol.~34. No.\,3. P.~44--49.

 \bibitem{18-kal-1} %15
\Au{Kowalski R., Sadri~F.}
Integrating logic programming and production systems in abductive logic programming
agents~//
Web reasoning and rule systems~/ Eds. A.~Polleres, T.~Swift.
Lecture notes in computer science ser.~--- Springer, 2009. Vol.~5837. P.~1--23.
\bibitem{19-kal-1} %16
\Au{Cosentino V., Del Fabro~M.\,D., El Ghali~A.}
 A~model driven approach for bridging ILOG rule language and RIF~//
 6th  Symposium (International) on Rules, RuleML 2012 Proceedings.  2012.
 CEUR-WS.org. Vol.~874. P.~96--102.
\bibitem{20-kal-1} %17
\Au{Veiga F.\,D.\,J.}
Implementation of the RIF-PRD. Universidade Nova de Lisboa, 2011. Master Thesis.


\bibitem{10-kal-1} %18
\Au{Markowitz H.\,M.}
Portfolio selection: Efficient diversification of investments. Wiley, 1991.
402~p.
\bibitem{11-kal-1} %19
\Au{Sharpe, W.\,F.}
Mutual fund performance~// J.~Business, 1966. Vol.~39(S1). P.~119--138.

\bibitem{12-kal-1} %20
\Au{Bollen J., Maoa H., Zeng~X.}
Twitter mood predicts the stock market~// J.~Comput. Sci., 2011. Vol.~2. No.\,1.
P.~1--8.

\bibitem{13-kal-1} %21
IBM WebSphere ILOG JRules Version 7.0. Online documentation.
{\sf http://pic.dhe.ibm.com/infocenter/\linebreak brjrules/v7r0/index.jsp}.

 \bibitem{24-kal-1} %22
 IBM Operational Decision Manager.
 {\sf http://www-03.\linebreak ibm.com/software/products/en/odm}.
\bibitem{25-kal-1} %23
IBM Operational Decision Manager Version~8.5 Information Center.
{\sf http://pic.dhe.ibm.com/infocenter/\linebreak dmanager/v8r5/index.jsp}.

\bibitem{14-kal-1} %24
\Au{Wilson T., Wiebe~J., Hoffmann~P.}
Recognizing contextual polarity in phrase-level sentiment analysis.
\textit{Conference on Human Language Technology and Empirical Methods in Natural
Language Processing Proceedinhgs}.
Stroudsburg: Association for Computational Linguistics, 2005. P.~347--354.

\end{thebibliography}
} }

\end{multicols}

 \label{end\stat}

 \vspace*{-3pt}

\hfill{\small\textit{Поступила в редакцию 03.11.2014}}
%\renewcommand{\bibname}{\protect\rm Литература}
\renewcommand{\figurename}{\protect\bf Рис.}  %13
\renewcommand{\figurename}{\protect\bf Figure}

\def\stat{sorokin}


\def\tit{AUTOMATION BEYOND WEB 2.0}

\def\titkol{Automation beyond WEB 2.0}

\def\autkol{A.~Sorokin}

\def\aut{A.~Sorokin$^1$}

\titel{\tit}{\aut}{\autkol}{\titkol}

%{\renewcommand{\thefootnote}{\fnsymbol{footnote}}
%\footnotetext[1] {The work of first and second  authors is partially supported by the
%Program of Strategy development of Petrozavodsk State University in
%the framework of the research activity. The third author is a
%postdoctoral fellow with the Research Foundation-Flanders
%(FWO-Vlaanderen).}}

\renewcommand{\thefootnote}{\arabic{footnote}}
\footnotetext[1]{IBM EE/A, % University Relations Manager for Russia \& CIS,
10 Presnenskaya Nab., Moscow 123317, Russian Federation}


%\vspace*{-12pt}

\def\leftfootline{\small{\textbf{\thepage}
\hfill INFORMATIKA I EE PRIMENENIYA~--- INFORMATICS AND APPLICATIONS\ \ \ 2014\ \ \ volume~8\ \ \ issue\ 4}
}%
 \def\rightfootline{\small{INFORMATIKA I EE PRIMENENIYA~--- INFORMATICS AND APPLICATIONS\ \ \ 2014\ \ \ volume~8\ \ \ issue\ 4
\hfill \textbf{\thepage}}}

\vspace*{6pt}


\Abste{This paper introduces a~new approach to the analysis of information systems
(IS) evolution based
on a~range of technological activities. The issue centres on the prospect that
Web-driven IS will be expanded from business processes to other domains of activities. The classical
approach by which automation eliminates bottlenecks in business processes does not work under
these conditions. Current trends in information technologies
(IT) increase the capability for Web integration that leads to new
types of virtual systems that will create a~new Web architecture, conditionally named a~Web
``spiral.''
The spiral type of integration on Web
supported by integrated cross-industry solutions is more promising and effective in comparison with
the ``radial'' ones. The paper describes this new class of IT systems.}

\KWE{automation; business process reengineering; collaborative software;
economies of scale; Internet topology; sociotechnical systems; systems of systems; virtual enterprises; Web 2.0}

\DOI{10.14357/19922264140414}

\vspace*{6pt}


\vskip 12pt plus 9pt minus 6pt

      \thispagestyle{myheadings}

      \begin{multicols}{2}

                  \label{st\stat}

\section{Introduction}

  \noindent
  Not much time has passed~--- on a~historical scale~--- since computers came on the technology scene.
However, the time and origin of invention and even authorship are still under discussion. Clearly, the
improvement in computing is a~process that combines a~variety of ideas, technologies,
 and drivers. In
Japan, for example, according to some sources, computing after the war had tight links with the emergence
of telephone switchboards
({\sf http://museum.ipsj.or.jp/en/computer/dawn/0005.\linebreak html}).

 In the U.S., one of the motivations for searching new ways of information processing was difficulties in
working with clerical card indexes in the first 30~years of the XX~century. In the USSR, the first
prototypes of computing technology were closely associated with defense programmes and space
research and development (R\&D).
After its birth, computing technology began to show not only rapid development, but also the ability to
penetrate virtually all spheres of human activities, going far beyond the range of tasks originally referred to.
With the emergence of local area network (LAN) and Internet, computing evolved along with telecommunications originating
the term ``information and communications technology'' (ICT).

 Current ICT development reflects the issues that are relevant to the professional roles of specialists
involved. Information managers view as dominant the phenomenal data avalanche that has to be overcome
with the help of high performance parallel computing. The
IT architects are concerned about the complexity of
systems integration. Business analysts are trying to cope with the growing
number of approaches and
notations defining business process modeling.

 Narrow professional observations on the issues lead to multiple conclusions in which direction the Web of
the Future will go. The center of gravity of such observations, if focused on the technical qualities of new
IT, are considered here in comparison with each other, in isolation from
the economy and lifecycle (LC) environment
in which they operate.

 Dominant approaches are looking like a~kind of technological Darwinism, most characteristically
expressed by Gartner's hypercycle of emerging technologies~\cite{1-sor}. Meanwhile,
 automation has
firmly implemented in the overall human environment, the conditions of existence
which are regulated by
known factors, such as economic crises, resource depletion, population growth,
and many others. The IT
fashion is changing rapidly. In one to two years, new bright idea is highlighted.
One can endlessly study
trends whilst more practical question stays in shadow~--- what artifacts will appear at their crossroads?

   \begin{figure*} %fig1
      \vspace*{1pt}
 \begin{center}
 \mbox{%
 \epsfxsize=103.339mm
 \epsfbox{sor-1.eps}
 }
 \end{center}
 \vspace*{-9pt}
\Caption{Evolution of automation scale and impact criteria}
\vspace*{6pt}
\end{figure*}

 A more general conception for describing the evolution of IT systems has been proposed by IBM
Almaden Lab. They noted the marked complexity of models and objects, which designers of modern
IS face today. The concept is presented in the form of a~new Science of Services,
Management, and Engineering (SSME) aimed at developing design techniques for highly complex objects.
Stakeholders may include large organizations, cities, and even whole states. ({\sf
http://campustechnology. com/articles/2009/04/13/ibm-and-higher-ed-push-for-a-smarter-planet-with-ssme-curriculum.aspx}).

 Viewing IT evolution from the SSME perspective, one may come to the conclusion that a~classical
approach to IT system design, based on business processes analysis, does not work here at all, or must be
extensively revised. The reason lies in the multiplicity of such processes that must be simultaneously
analyzed and automated.
Additionally, advanced complex IT systems automate not one or a~few business
processes but whole domains of activities in general. In such cases,
one is faced with a~large number of
processes and technologies embedded in living environments; so, each of them needs to be repeatedly
changed, removed, or substituted. This leads to the necessity of an initial analysis of the domain
itself and,
second, the related business processes, implying transition to a~meta-design approach, and the
modification of basic ICT system design, scaling to \textit{domain} of activity. Under the term `activity,'
the bunch of connected business processes, covering professional areas (domains), supported by
groups of technologies with the purpose of improving economic efficiency
is considered.
Hence,
{\bfseries\textit{automation and virtualization}} in this work are regarded as group technologies for the
reproduction of the artificial environment with the purpose of significantly improving business
performance.

 In this paper, the system and design aspects of future Web evolution
 are investigated that will
undoubtedly affect the architecture and LC of IS. Below,
{\bfseries\textit{the stack of activities}} is introduced which analysis can help
to make some predictions on ICT
development in the near future. Figure~1 shows automation changes in terms of scale and added-value
criteria.


 In this paper, the present author will try to find some answers on described above issues considering Web as
 self-organizing, self-sustaining, and evolving system with multiple direct and feedback connections
between domains.

\vspace*{-3pt}

\section{Lifecycle Ecosystem}

 \noindent
 An area of activity (domain) as functionally homogeneous set of business processes and
technologies appear in division of labor is considered.
Quantity, composition, and relationship between such activities
form products LC and environment that represent a~model of technological expansion. It uses
the principle of procreation, when the original activity generates the following one as a~result of internal
conflict that limited its capacity (Fig.~2).


\begin{figure*} %fig2
\vspace*{1pt}
 \begin{center}
 \mbox{%
 \epsfxsize=134.29mm
 \epsfbox{sor-2.eps}
 }
 \end{center}
 \vspace*{-9pt}
  \Caption{LC Ecosystem}
  \end{figure*}

 First of all~--- LC realization is impossible without resources. Therefore,
  the root domain will be
1.~``Access to resources'' activity. This domain is denoted by
number one, and main types of resources that are
consumed by the technology community create a~set of subdomains. Let list them in a~logical
 sequence~--- from basic to the more complex ones appearing later:
 \begin{enumerate}[{1}.1]
 \item Data (information resources) collection.
 \item  Natural resources extraction.
 \item Finances.
 \end{enumerate}

 1.1\ ``Data collection.'' This is the root activity in number of basic types playing fundamental role in every
human's life and society in general. Probably for this reason, the history of automated systems started with
the automation of information processes. Since that and going on,
automated systems are called as
``information.'' Without this subdomain, the very existence of society is impossible. Initial stage here is
observation. After that, people are beginning to measure the observed phenomena. Next stage is
introduction measures of weight, distance, and time. Then development moves towards the inventions of better
tools to collect information. Thus, new tools and correspondently domain states appeared: microscope and
telescope, photography, microphone, audio recording, telephone, x-ray, filming, radar, video, electronic
microscope, sensors and telemetry, space probe, databases, search engines, scanner, geopositioning, global
information systems. Each tool opened new and more powerful opportunities for high quality information
activities.

 1.2\ ``Natural resources extraction.'' To implement new technologies and inventions, natural
resources are needed. They determine the place of this activity. Stages evolve in the direction from the exploitation of
the biosphere and then~--- extraction and use of minerals and metals (mining), extraction of hydrocarbons,
uranium enrichment. The sequence of stages is moving from easy to more difficult availability and after
that~--- to creation of artificial resources, recycling, and meta-materials~\cite{3-sor}.

 1.3\ ``Finances.'' Recourses evolution also came from natural to artificial and led to emergence of money
as more general and valuable recourse.  Milestones indicating the progress of this type of activity are:
introduction into circulation of money substitutes, coinage, emergences of usury and banks, invention of
paper money, introduction into circulation of securities to financial markets, globalization and automated
support of financial markets.

 Intra conflict: resources cannot be used instantly at the place of extraction. This conflict is an origin of
new type of activity~--- 2. ``Transfer.''  Subdomains in this area are also ordered by complexity:
 \begin{enumerate}[{2}.1]
 \item  Information channels.
 \item  Material channels (transport, pipelines, energy lines).
 \end{enumerate}

 2.1\ ``Information channels.''  Stages that mark the domain expansion
 form a~sequence starting of
personal contacts between people, and further on~--- postal service, electrical transmission of discrete
information (telegraph, teletype), electromagnetic analog transmission (telephone, radio), digital
communications, computer  networks, satellite repeaters.

 2.2\ ``Material channels.''  Transfer of resources absorbs achievements of related domains and is
developing toward modern supply chain. They provide movement of a~large number of people,
transportation of goods, raw materials and energy. Stages marked expansion in this domain are: road
construction, use of natural ways (river and sea), the emergence of railways, air transport, container
transport, and, finally, modern automated supply chain.

 Intra conflict: inability of immediate use resources by end user. To improve this, they must be processed,
which generates the next logical domain~--- 3. ``Production.''  Its subdomains:
 \begin{enumerate}[{3}.1]
 \item  Power production.
 \item Production of goods and services.
 \end{enumerate}

 3.1\ ``Power production.'' Previous activity of extractive industries makes basis of energy production. On
qualitative scale of stages, one may mark use of muscular energy as
a~starting point and further on, opening of
fire, use of wind  and water energy, steam energy (in the beginning of the industrial revolution),
electricity, nuclear energy, alternative energy sources in a~modern, high-tech version, and, finally,
thermonuclear (forecast). Evolution of this domain makes it possible mass productive activity in the next
one.

 3.2\ ``Production of goods and services.'' Initial milestone of this activity
 is served to satisfy primary needs
with the help of manual production of food, clothing, and footwear.
Next step is providing of services. Then
goes the chain of stages: deployment of production of consumer goods on a~commercial scale, industrial and
residential construction, heavy machinery, entertainment business, high-tech manufacturing of
communications and computing equipment.

 Intra conflict: manufactured products and services must be delivered to consumer. This function is outside
this domain, which generates the following
area~4.~``Distribution.'' Communities cooperate with each other
by information, services, and products exchange they produced as objects of trade.
So, according to this logic,
trade became the following domain of activity in
the constructed technologic stack. It has evolved from
a~simple barter towards the emergences of money trade, after that~--- major trading houses, laying large
trade routes and distribution channels, unification of production,
transportation, and sales businesses, global
international trade, sales through communication channels and networks (television, mail, catalogs, and
over the Internet).

 Stages of this area include: barter, monetary trade, wholesale, trading networks, integration of trade and
production, global trade, and trade through communication channels.

 Intra conflict: Customers are buying goods which quality is entirely determined by the manufacturer. This
may not comply what customers are really need. In domain evolution, certain number of compensational
tools was born and developed: advertising, marketing, etc. However, these tools are not always fully able to
suppress this conflict manifestations. Flexible production is a~way to solve it getting feedbacks and
experience from products use history.

The fifth domain is ``Use.'' Its subdomains, as well as subdomains of all other activities arranged by the
principle ``from simple to complex'' are taken here from  classical
Maslow pyramid~\cite{2-sor}:
 \begin{enumerate}[{5.}1]
 \item  Satisfaction of physiological needs.
 \item Safety needs.
 \item Social needs realization.
 \item Esteem support.
 \item Realizing of personal potential.
 \end{enumerate}

  Intra conflict: produced items with time become morally and physically outdated and must be recycled.

 6. ``Recycling''~--- next and the last activity in the row of domains that form LC chain of activities. Its
implementation became more difficult with increasing scales of production, using modern synthetic
materials, pollution, and many other factors.

 Main stages: indiscriminate dumping of waste; disintegration technologies;
 and recycling and reuse.

 Intra conflict: environmental pollution. It can be solved by sending of extracted reusable resources  into
initial first domain with  cross-domain feedback.

 Technology development makes LC more complex, and an inherent intra conflicts increase their
vulnerability. Compensatory mechanisms bring to life two other large domains
 7.~``Lifecycles support'' and 8.~``Lifecycles update.''

 7. ``Lifecycles support'' includes business processes and technology that support existing industries and
all domains included in LC ecosystem. It has four subdomains:
 \begin{enumerate}[{7}.1]
 \item ``Learning.''
 \item  ``Management and organization.''
 \item ``LC services.''
 \item  ``Meta-security.''
 \end{enumerate}

 7.1\ ``Learning.'' Information shared through communication channels has to be analyzed. Analysis results
in the new knowledge. Large number of professions is associated with the knowledge accumulation and
sharing. Development is moving from primary artifacts in the form of oral tradition, education, and then,
after invention of writing, towards book printing, calculation, research, establishment of educational and
research institutions, lecturing by radio and television, knowledge bases/expert systems,
computer-aided
learning systems, social networks, virtual universities, and, finally,
to virtual labs and universities. However,
knowledge consumption must be effective and purposely managed. Later stages include knowledge
socialization where new role functions appeared:  leaders, experts, facilitators, and others that support
the process of knowledge exchange in networked social groups. They also need to be managed.

 7.2 ``Management and organization.'' Every domain has its specialties in division
 of labor processes. It increases performance of society in general. On top
 of this division, management plays special role to
improve effectiveness of every domain by better organization and resources utilization.

 Initial stages here are: management of row information, communications management,
 and knowledge
consuming management. Then, path of evolution goes to emergence of organizational skills, processes
management, organizations structures management, material objects control
(tools, machines), asset
management, and
territories and global structures governing (states, transnational corporations).

 Improving social performance by organizational means quickly achieves
 the limits where performance stops growing. To complete the
 mission of forth domain, additional instruments are needed. Technology, innovations,
and inventions are these tools. Needs in their development are calling for
life next area of activity.

 7.3\ ``LC services.'' This subdomain of meta-activity provides B2B
 (Business-to-Business) is services that differ than B2C
 (Business-to-Customer) serviced
which are producing by subdomain~3.2. Technology expansion is growing from repair on demand to
subscription for services, CALS (Continuous Acquisition and Lifecycle Support),
web-services for LC updating, and embedded self-services.

 7.4\ ``Meta-security.'' Very important subdomain that defends all other
 activities from various threats.
Collapse and damage to any of its elements are
fraught with losses and even economic or social disasters.
That is why, all other domains contribute to its maintenance, and, in its turn, domain provides a~feedback to
all other stack's elements. Milestones and specific implementations are: the emergence of the concepts of
society, property, including intellectual property, and from here onwards~--- personal rights and duties,
emergence of law and legislation, tools for protection of family and possessions, health, judiciary, police
and security guards, armed forces, international law, and cyber defense.

 Intra conflict: multiple LC, heterogeneous technologies and manufactured products cannot be
effectively organized by supporting tools. Quality of training, maintenance,
management, and security
measures often follows by incidents that show backlogs off requirements to LC sustainability. This
demands permanent improvements and innovations. For such purposes, the ecosystem of LC evolution
contends 8th domain~--- ``Lifecycles update'' that is dedicated to modernize, renovate, optimize, and
integrate LC.

 Unlike ``Lifecycle support,'' this domain's technologies replace old LC
 by the new ones or, at least, introduce
new elements and provide LC adjustment and optimization. It also contents three main subdomains:
 \begin{enumerate}[{8}.1]
 \item ``Meta-design.''
 \item ``Meta-automation.''
 \item ``Meta-economy.''
 \end{enumerate}

 8.1\ ``Meta-design.'' In context of this paper, the term is associated with innovative and inventive activity,
the result of which is modernization and replacement domains~(1--6) with new and modern components. It
also means design of LC, their integration and optimization.

 Accumulated knowledge as well as managerial and organizational skills allow invent and produce tools
of increasingly perfect design on industrial scale. Its creative stages and expansion phenomena in some
senses repeat the picture of development in ``Management'' for current domain's
purpose and is also directed to
improvement of technological culture. Very important inventions were made in basic domains starting
progress with inventions in data collection (computer images that represent designing objects, etc.),
communications (mapping, navigation instruments, etc.),
and knowledge accumulation (writing, printing, etc.).

 Further evolution goes through inventions of organizing technologies, towards specialization of industries
and crafts, drawing, technology development and manufacturing processes of these objects, production of
technology equipment, automation, and virtualization of manufacturing.

 8.2\ ``Meta-automation.'' Unlike single IS implementations, ``Meta-automation'' belongs
to a~class of ``system of systems'' for it affects not only separate business processes but
the parts of the whole
domains. It can be regarded as an instrument for domains efficiency management; however, unlike
economy tools, ``Meta-automation'' makes it with different means.

Initially, automation pasts the simplest form allowing reduction of LC cost by replacing manual labor. Then, it
goes in a~way of creation an artificial environment that increases business processes performance in all
others domains. That is the reason why evolving stages of automation repeat sequence of described human
activities expansion. Evolution started with the first milestone that represents the concepts and initial
prototypes of IS and databases, which at one time were considered as main computer
applications for indefinitely long term. According to some sources, in 1962, American company System
Development Corporation first coined the term database ({\sf http://en.wikipedia.org/wiki/Database}).

 Next milestone was circuit switching. ARPANET project where packet switching was
 first implemented
in communications between computers became the third one.


 This key technology opened the era of networking. Next is a~milestone that marks the period when
efforts were focused on management control system (MCS) later evolved in ERP
(Enterprise Resource Planning). Boom of CAD/CAM
(computer aided design\,/\,computer aided manufacturing) evolved in CIM (computer integrated
manufacturing) and factory automation made the following states of development. It produced complete
systems including not only computer graphics, but robotic workstations, information retrieval systems,
plotters, and various versions of LAN. Seventh milestone: noticeable progress in simulation modeling of
economic processes and the development of computer models of the economy.

 Communication protocol TCP/IP (Transmission Control Protocol\,/\,Internet
 Protocol) opened new era. Speed and scale of Internet as unprecedented in human
history monster artificial environment mean a~new phase of automation~--- spread or even absorption of all
kinds of human activities by sociotechnosphere. With the implementation of Internet technologies,
 sociotechnical systems, concepts of which appeared in the 1960s~\cite{4-sor},
nowadays reached a~new level. It leveraged by deep penetration of IS and all kinds of
technology in social and organizational structures, as well as the quality of innovations.
The influence of this phenomenon will be discussed in the following sections.

 8.3\ ``Meta-economy'' is a~toolkit of economic regulation not inside but under LC. Activity that
determines behavior of industrial units.

\pagebreak

 Intra conflict: similar to the case of ``LC support'' domains, the contradictions lie in multiplicity of
LC that make it difficult to effectively coordinate all instruments of updating and innovations. Predicted
solution is the development of Meta-design technologies to leverage LC modernization.


 It should be noted that both domains (7 and~8) have also an external additional conflict between them
since stability and modernization are inherently contradictory concepts. For example, implementation of
advanced and saving car engines was periodically hampered by oil-producing companies interested in
increase consuming of petroleum for internal combustion engines. Nevertheless, economic conditions
tightening shifts the equilibrium between the domains in favor of modernization and forces manufacturers
to accelerate the transition to new LC design. Measures taken to energy saving in the current crisis
stimulated the development of ``green'' technologies and production of economic components for
electronics and lighting.

\section{Technologic Stack of~Activities}

\noindent
Stack of activities is a~model for presenting a~set of key areas of lines of business (LOB), or business
domains, in logical connection between the individual domains. In definition of ``activity'' which was
introduced above, two entities are presented~--- the set of business processes and a~variety of technologies.
For this reason, one can build up two types of stacks~--- first, the processes stack and, second, the
technological stack of activities. This couple gives an overall model for domain description. However, in
this paper, we, in the first turn, are interested in IT analysis and,
 therefore, should focus on building
a~technological stack. Then, it will be applied to exploration of Web evolution.

\begin{figure*} %fig3
\vspace*{1pt}
 \begin{center}
 \mbox{%
 \epsfxsize=120.259mm
 \epsfbox{sor-3.eps}
 }
 \end{center}
\end{figure*}

Theoretical model of technology stack is \mbox{$N$-dimensional}. In fact, the stages also have their history,
description, and content parameters, which imply the possibility of including additional elements called
states. So, three-dimensional (3D) stack will include a~description of each stage decomposed by states. For example, for stage
``construction industry'' in domain ``Production of goods and services'' (8\;$\to$\;8.4), additional chain of
stage decomposition will appear: 8.4.1~``Use of natural shelters''\;$\to$\;8.4.2~``Construction of stone and
wood''\;$\to$\;8.4.3~``Construction of the man made materials''\;$\to$\;8.4.4~``Use of 3D printers.''

 Each state may be subjected to further decomposition, etc.\ until
 the desired degree of granularity is
obtained. For studies in local business areas, it is possible to make slices of this model. For the immediate
objectives of the present study, the two dimensions of model are enough and will not sufficiently distort the
results.
{\looseness=-1

}

 Stack element (domain or milestone) affects others with \textit{direct links} and \textit{feedbacks}. Direct
links mean that progress could be measured by emergences of new milestones in higher positioned
domains. Feedbacks are measured by the progress observed in previous domains.

 When direct links and feedbacks form a~\textit{loop}, it gives an enhanced effect of activities interaction
often described as a~``\textit{revolution}.'' Thus, feedback from banks and users capital (activity
``Finance'') to the activity ``Production of goods and services'' and backward made a~loop and gave direct
effect in the creation of capitalist industry (the first industrial revolution) and a~modern market economy.
Direct links from activity~7.1. (``Learning'') to 8.2. ``Meta-design'' led to invention of computer as a~mean
of information processing. Implementation of computing in lower domains in all subsequent forms
provided an effect, often defined as the \textit{second industrial revolution}.

 Stages of technology evolution form the space that may be
 defined as ``Space of Technology'' (Fig.~3).
Technologic stack enables one also to fix the level of social and technology development. If one draws a~line
across the selected milestones, it will be a~line of development level.

 The lower level of development crosses the initial stages of domains and forms the foundation of ``Space
of technologies.'' Higher stages of domains mark the front of development in technological space.

 North-West corner of built technologic space covers basic technologies taken from nature and natural
analogs. Large scale automated systems are concentrated in the South-Eastern corner.  They leverage
technologically closed communities with artificial internal rules of existence.

 Direction from the North-West to the South-Eastern corner figuratively plays a~role of vector of
development, on which the most crucial inventions, technologies, and discoveries are located.

\section{Special Role of Automation in~``Space of~Technologies''}

\noindent
 	Mentioning this special role of meta-automation in shaping of technology,
 let make the closer look on
evolution of this domain. Web is the space for interaction. Men, machine, and system are the general parts
of such interaction between them in IT environment. Let
also fix three modern types of Web environment
in which interaction is taking place: user centered, machine dominated,
and systems integrated.

 Computing platforms and algorithms also play very important roles. But in this
 sense,
  these roles supporting and computing evolution will go in the direction
  to satisfy\linebreak\vspace*{-12pt}

  \pagebreak

%  \begin{figure*} %fig4


 \begin{center}
 \vspace*{1pt}
 \mbox{%
 \epsfxsize=77.459mm
 \epsfbox{sor-4.eps}
 }

% \vspace*{-9pt}
 {{\figurename~4}\ \ \small{Basic IT classes supporting interaction in Web}}
  \end{center}

  \vspace*{6pt}
%\end{figure*}

\addtocounter{figure}{1}

  \noindent
   the growing demands coming from
further development of interaction needs. Hence, let build classification that presents generic classes of
automation that support Web interaction using these 6~entities (Fig.~4).

\addtocounter{figure}{1}



 Constructing this classification, the rules introduced early have
 been followed: classes evolve from simple to
sophisticated~--- from left higher corner to the
right low one. And each higher class may absorb and use all
technology solutions of lower classes. And the present model is 3D that gives possibilities for classes to provide
new artifacts as vertical components of the model. Interaction tools between human and
user centered
environment set up the initial class. It is \textit{data flow} supporting technology~--- e-mail, e-messengers,
Skype, and so forth.

 Second class contents the means for intercommunion among machine and user centered environment.
They are \textit{Interface} design technology and examples include standard API
(application programming interface), speech-gesture
interfaces, Google glasses, and so on.

 The next class is \textit{Enterprise}. This type of technology intermediates between work stations (user
centered environment) and systems. Commercial ERP systems also satisfy this class definition.

 Then, class with \textit{Smart devices} is going. It helps one to operate in machine dominated area: smart
houses, machine control with embedded processors, etc.

 Evolution of machine--machine interaction is based on intelligent protocol class of technology. The
examples are SDN (software defined networking), intelligent navigation software.

 Systems managing machine class demands use of robotic conception. Artificial intelligence mechanisms
must be embedded in drones (UAV~--- unmanned aerial vehicle)), industrial and military robots for their navigation, interaction and
mission execution.

 Next step in Web interaction evolution is emerging of the new technology class that permits human to
solve complex tasks in system integrated media by use of natural language
(NLP~--- Natural Language
Processing). This class does not include relatively simple voice recognition technology related to interface
class.  It has a~deal with Q\&A (questions and answers)
system, artificial intelligence for decision-making, advanced training
systems, etc. IBM Watson is a~good example of commercial system
that occupied this position.

 Further, let come to \textit{Smart systems} class of interaction technologies permitting any machine
to be integrated and interact in systems environment. This class is a~child of fast developing cloud
technology, social networking~--- from computing side, and demands to smart consuming of resources~---
from side of economy. Examples embrace IBM series of Smarter Planet system that
will be considered in more details later on.

 The last, ninth, and the highest class is \textit{System of systems} technology. The most spectacular example of
this type is Web itself. Applying the famous analogy of Internet and real web made by spider~---
a~wonderful example of natural stress-resistant construction~---
it is possible to answer the following
question: What quality of our IT Web that is the greatest artificial system in human history, provides its
integrity instead possible fragmentation with increasing complexity?

 In the model shown in Fig.~3, one may recognize ``radial'' ``filaments'' of WWW
 (world wide web) representing
industrial/LOB directions of automation technologies development. Cross-industries connections of
automation milestones shown there represent ``spiral'' way of integration. Cloud computing,
social networks, and M2M (machine-to-machine) technologies open practically unlimited prospects for such
integration. Projects of global free wi-fi like Outernet ({\sf https://www.outernet.is/}) have to accelerate
movement toward ``spiral'' at the humanitarian (healthcare, education, etc.)\ parts of it.
Some assessments
show that 15~billion devices will be connected by 2015 (Intel) in Internet of Everything (IoE) that will
cover of \$14.4 trillion (Cisco)~\cite{10-sor}. One
may foresee that further development of AHCP (Ad-Hoc
Configuration Protocol) inside \textit{intelligent protocol} class will open the possibility of \textit{smarter
integration} when IT platforms of ``spiral'' systems could ``negotiate'' with each other and install proper ad
hoc configuration. Large potential of \textit{intellectual integration} is opened by IBM Watson. With
ability to deep intelligent search of absent and necessary system's components based on artificial
intelligence, Watson computer will be very useful assistant in design of ``spiral'' systems.

 Mentioned technologies  produce powerful synergetic effect for automation development providing
WWW as platform for further deployment of extralarge complex systems, predicted above as class~9
(see Fig.~4). Growing in scale and accumulating innovations taken from different
domains, such types of
systems will reinforce total WWW performance and lead to new web topology which is similar to weaving
of web spirals. One may call it ``spiral'' architecture, employing mentioned parallel.
And, correspondingly, such future systems will be called as Web ``spiral'' systems.

In meta-automation expansion scenario outlined here, the described 9~classes of interaction
technologies are used
as reference solutions that determine general direction of developments.	

 Each domain produces technology that tends to expand into others domains ``territory.'' As a~result of
expansion, communication technologies are observed
everywhere, e-learning is everywhere. Power
and transport are also penetrating in every part of technological space.
It will be shown that automation like
each other activity enjoys this ability but beside this, it also possesses
some special features that make it meta-activity.

 As was noted above, each domain may be decomposed into subdomains. Technological breakthroughs
corresponding to each of them can be represented as innovative milestones that determine ``vertical'' or
``radial'' directions of technology diffusion. New milestone also provides impacts to neighborhood
environment in horizontal directions through direct and feedback connections.  As a~result, all of the
domains in varying degrees contribute to the technological development of each activity. In general, it
looks like the acceleration of scientific and technological progress. Automation follows the innovations in
all directions but sometimes precedes and accelerates them.

 Taking as an example, breakthrough in nanotechnology with the graphene invention (8.2. ``Meta-design'')
leads to implementations into domains 3.1 (rechargeable battery, solar panels), 3.2 (processors), 1.1 and
7.4 (sensors), etc. Collaborative works in these areas performed onto IT integrated platform will be more
productive than process of ``natural'' technology diffusion.

 Thus, automation as a~special kind of activity, from the one hand, by means of innovations is tightly
connected with Meta-design domain and from the second~--- is a~tool for managing of economic performance
set by ``Meta-economy.'' Automation fulfills this role in two ways. First~--- in labor allocation scheme it
transfers to machine only those operations that man performs worse than computer. The second way is
integration that adds value for multitude of enterprises helping them to manage common data flows and
effectively use their huge computing resources. That, in fact, makes it possible to combine stack elements
to obtain economic benefits in new models of entrepreneurship.

 The innovative quality of automation provides results and artifacts
 that cannot be achieved without
automation. The most famous examples are robots, 3D printers, and calculations of satellites flights
trajectories.  Cognitive computing, cloud technology, and M2M are taking out automation beyond the role
of just service tool that helps to solve different tasks inside LC processes.
 With latest advent, automation
obtains the ability not just redesign and improves LC themselves but to invent new LC for new
products. In accordance with such new quality, this activity can be called
 the ``Meta-automation'' as
a~part of ``Update of lifecycles'' domain, ensuring LC design, renewal, and integration.

\section{Steps to ``Spiral'' Systems}

\noindent
 Let consider main properties of the ``spiral'' systems. First of all:
 \begin{itemize}
\item ``Spiral'' is a~``system of systems'' component of Web architecture and more advanced form in
evolution of Web hosting and data centers;\\[-14pt]
\item ``Spiral'' system belongs to different owners and supports multitenant mode;\\[-14pt]
\item ``Spiral'' integrates self-organized communities: emerge
collaborative projects and
maintain and implement their outcomes with the help of rented instruments and resources;\\[-14pt]
\item ``Spiral'' system projects management is conducted in a~virtualized environment (domain
``Management and Organization'') and also may attract social lending alongside with traditional financial
sources;\\[-14pt]
\item ``Spiral'' integrates, first, larger parts of LC: information
and research, economics and finance, and
design and production as secured distributed clods; and\\[-14pt]
\item ``Spiral'' platforms predominantly provide utility computing in the mode of EaaS (Everything as
a~Service).
 \end{itemize}

 Deployment corresponding architecture of Web  brings forth the task of new design methods or even new
approaches to Internet topology.

 The constructed model allows to look at automation with more general civilization positions, defining
its place and role in overall progress.

 For qualitative assessment of possible value for synergetic effect of automation, it
  is necessary to involve some economic considerations.

  \begin{figure*} %fig5
\vspace*{1pt}
 \begin{center}
 \mbox{%
 \epsfxsize=162.416mm
 \epsfbox{sor-5.eps}
 }
 \end{center}
 \vspace*{-11pt}
\Caption{World economy model according IBM Institute for Business Value:
\textit{1}~--- same industry; \textit{2}~--- business support;
\textit{3}~--- IT systems; \textit{4}~--- energy resources; \textit{5}~---
machinery; \textit{6}~--- materials; and \textit{7}~--- trade}
\vspace*{-3pt}
\end{figure*}

 IBM Institute for Business Value published a~study model of the world economy representing 11~core
systems~\cite{5-sor} (Fig.~5). These core systems include: Infrastructure, Electricity, Finance, Education,
Food, \mbox{Communications}, Water Resources, Transportation, Entertainment, Fashion, Leisure, Health,
Governance, and Security. In Fig.~5, each core is represented by a~bubble whose value is proportional to
the size of the economy. For scaling, a~bubble of 1 trillion dollars is shown in the lower right corner. The
model has the fractal properties. Production, business, IT systems, engineering, energy resources, materials,
and trade are incorporated in each core system as components.
In many cases, the component's name and the name of
core system may be the same. The components inside cores are also indicated by scalable
bubbles. Arrows, whose width characterizes the degree of influence of one core system to another, show
contribution of all core systems in the operation of each separate one.

IBM analysts based on regression economic models argue that the total loss of world economy is estimated
as 15~trillion dollars resulted due to inefficiency. They also found that the loss of at least 4~trillion dollars
could be prevented through more rational organization and use of modern information technologies.
Certain part of such losses occurs at the level of ``system of systems'' as poor integration between the cores.
Executive Report made by Korsten and Seider was published by IBM in 2010 in the period
when global economic crisis has begun~\cite{5-sor}. At that time,
IBM developed and presented the ``Smarter Planet''
initiative~\cite{6-sor} as a~technological response to the crisis. The purpose of this initiative was to
increase value delivered to the end users with the help of~``smart'' IS aimed at the
automation of complex activities that approximately correspond to system cores in the economic model
described above. Such systems were labeled as ``Smarter Grid,'' ``Smarter Healthcare,'' ``Smarter Finance,''
``Smarter Work,'' and so on.
%Their design should provide cost savings and effective business processes in
%the target domain (power production, customer services, water resources, etc.)\ and in the maintenance of
%information systems themselves (green technology, dynamic infrastructure). These open the era of cloud
%computing and social networks.




 Their design should provide cost savings and effective business processes
 in the target domain (power
production, customer services, water resources, etc.)\ for interdomains integration and building of
heterogeneous systems at a~very broad scale. They can be considered
as the first steps to Web-``spiral'' systems of systems (class~8, see Fig.~4).

 Platforms of this series are different in structure, but have many common characteristics:
 \begin{itemize}
  \item  high degree of security provided by software products of IBM Tivoli family;
  \item dynamic infrastructure supported by cloud technologies;
\item meta-models and tools for business processes management  that make possible  quick adaption to
changes in business environment;
\item ability to accumulate knowledge and provide access to cognitive resources with help of social tools;
\item high scalability and mobile access to services; and
\item good  performance records (cost savings and use of ``green technologies'' reducing total cost of
ownership).
  \end{itemize}

 Platforms presented in LC ecosystem (see Fig.~2) are: Smarter SCADA for Oil and Gas (1.~Access to
resources), Smarter Transportation (2.~Transfer); IBM Smart Grid and Rational software platform for
automotive systems (3.~Production); IBM Smarter Commerce (4.~Distribution);
and IBM i2, IBM
Defense Operations Platform, and IBM i2 Defense Solution (7.~LC support).

 Early examples of complex automation corresponding to ``spiral'' concept were DiFac project launched in
second framework program of the European Commission~\cite{7-sor}
and BioVLAB~\cite{8-sor}.

DiFac is a~complex sociotechnical system designed to boost both economic efficiency and performance of
human labor in production area. Collaborative participants of the project developed methods for industrial
control, interaction in the network team, whose members are located in different countries and were
connected through 3D virtual reality.

 BioVLAB is a~cloud environment for microRNA and mRNA (ribonucleic acid) integrated analysis (MMIA)
on Amazon EC2. It makes vast amount of microRNA expression profile data publicly available.  BioVLAB
is positioned by its developers as an easy-to-use computing environment for researchers who plan to
perform genome-wide integrated analysis tasks with advanced features:
 \begin{itemize}
  \item readily expanded computational tools;\\[-14pt]
  \item easily modifiable by reconfiguring in the workflow;\\[-14pt]
  \item on-demand cloud computing resources; and\\[-14pt]
  \item  distributed orchestration supports complex and long running workflows asynchronously.
  \end{itemize}

Special place in the row is occupied by IBM Intelligent Operations Center (IOC) introduced by IBM as
a~part of Smarter Planet initiative. IBM IOC had been applied in many target areas of Smarter Planet in
a~purpose to integrate and use data from multiple sources and present results of their processing in single
interface. Covered sources may belong to absolutely different domains of activity and this complex
integration permits to monitor and manage their states and support operative decisions. Data processing and
decision-making use advanced analytics, asset management, and collaboration tools. Smarter City is one of
the most complex and promising platforms introductions in modern Web. Perhaps, IBM IOC is the largest
commercial solutions currently distributed at the IT market. It provides
the following functions~\cite{9-sor}:
\begin{itemize}
  \item visual workspace;\\[-14pt]
  \item events and incident management;\\[-14pt]
  \item resource, response, and activity management;\\[-14pt]
  \item status monitoring;\\[-14pt]
  \item  collaboration, instant notification, and messaging;\\[-14pt]
  \item  reports; and\\[-14pt]
  \item semantic model.
  \end{itemize}

System's architecture includes multilevel SOA (service-oriented architecture)
structure, power infrastructure based on IBM Tivoli
software, including clouds and system security. Key performance indicator
managed dashboard uses event management and
workflows engine to react on real-world situation
and to keep specified policy and performance level.

 All above mentioned systems are designed for collaborative works performed by legally independent
or-\linebreak\vspace*{-12pt}

\columnbreak

\noindent
ganizations acting as a~single Web alliance. As such, they meet the definition of a~virtual enterprise
and the 9th class ``system of systems'' as well. Consequently, ``spiral'' systems may also be regarded as
virtual organization of next generation.

\vspace*{-6pt}

\section{Concluding Remarks}

\vspace*{-2pt}

\noindent
\begin{enumerate}
 \item  Further economy development, as it follows from the IBM Institute for Business Value model will
be not for intensification of natural resources consumption but for losses reduction. This requires the new
type of Web systems and Web architecture with the ability to automate not just business processes but
domain of activities.\\[-14pt]
 \item Design of this type of information systems based on classical approach that automation eliminates
bottlenecks in business process does not work and to BPM (business process management) must be added
AMS (activity management system).\\[-14pt]
 \item Analysis of technologic stack and requirements of the modern economy permits
 to expect with
a~high probability that new type of IS
for domain automation conditionally defined as
``spiral'' will evolve in the direction responding the introduced requirements.
 \end{enumerate}

\renewcommand{\bibname}{\protect\rmfamily References}

\vspace*{-6pt}

{\small\frenchspacing
{%\baselineskip=10.8pt
\begin{thebibliography}{99}

\vspace*{-2pt}

\bibitem{1-sor}
Gartner hype cycle. Available at: {\sf http://www.gartner.\linebreak
com/technology/research/methodologies/hype-cycle. jsp}
(accessed November~21, 2014).

\bibitem{3-sor} %2
News tagged with metamaterials. Available at:
{\sf http:// phys.org/tags/metamaterials/} (accessed November~21, 2014).

\bibitem{2-sor} %3
\Aue{Maslow, A.} 1954. \textit{Motivation and personality}. New York, N.Y.: Harper\&Row Publs.
Inc. 15--31.

\bibitem{4-sor}
\Aue{Emery, F.\,E., and E.\,L.~Trist}. 1960. Socio-technical systems. \textit{Management science, models
and techniques}. Eds.\  C.\,W.~Churchman and  M.~Verhurst.
London: Pergamon Press. 2:83--97.

\bibitem{10-sor} %5
Internet of things market forecast. Available at:
{\sf http://\linebreak postscapes.com/internet-of-things-market-size}
(accessed November~21, 2014).

\bibitem{5-sor} %6
\Aue{Korsten,~P., and Ch.~Seider.} 2010.
The world's 4 trillion dollar challenge: Using a~system-of-systems approach to build a~smarter
planet.  IBM
Institute for Business Value.  IBM Global Business Services Executive Report.
Available at: {\sf
http://www-05.ibm.com/tr/events/\linebreak ibmcozumlerzirvesi2011/pdf/GBE03278USEN.PDF} (accessed June~17,
2014).

\bibitem{6-sor} %7
IBM Smarter Planet publications. Available at:\linebreak {\sf
http://www.ibm.com/smarterplanet/us/en/overview/\linebreak
 ideas/index.html?re=sph};
  {\sf http://www.ibm.com/\newline smarterplanet/ru/ru/};
% \vspace*{-12pt}
{\sf http://en.wikipedia.org/wiki/}

\pagebreak

\noindent

{\sf  Smarter\_Planet};
{\sf http://www.ibm.com/smarterplanet/}
{\sf us/en/?ca=v\_smarterplanet} (accessed June~17, 2014).
\bibitem{7-sor} %8
DiFac success story. {\sf
http://www.ims.org/wp-content/\linebreak
uploads/2012/03/DiFac-SUCCESS-STORY\_100917.\linebreak pdf} (accessed
June~17, 2014).
\bibitem{8-sor} %9
\Aue{Lee, H., Y.~Yang, H.~Chae, S.~Nam, D.~Choi, P.~Tangchaisin, C.~Herath, S.~Marru,
K.\,P.~Nephew, and S.~Kim.} 2012.
BioVLAB-MMIA: A~cloud environment for microRNA and mRNA integrated analysis (MMIA) on
Amazon EC2. \textit{IEEE Trans. Nanobiosci.} 11(3):266--272. doi: 10.1109/TNB.2012.2212030.
\bibitem{9-sor}
IBM Corp., International Technical Support Organization.
November~15, 2012.
IBM Intelligent Operations Center for Smarter Cities.
IBM Redbooks Solution Guide.



\end{thebibliography} } }

\end{multicols}

\vspace*{-9pt}

\hfill{\small\textit{Received June 10, 2014}}

\vspace*{-24pt}

\Contrl

\noindent
\textbf{Sorokin Alexander V.} (b.\ 1946)~---
Candidate of Science (PhD) in technology,
University Relations Manager for Russia \& CIS, IBM EE/A;
asorokin27@gmail.com

\vspace*{8pt}

\hrule

\vspace*{2pt}

\hrule

\vspace*{-6pt}

%\newpage


\def\tit{АВТОМАТИЗАЦИЯ ЗА ПРЕДЕЛАМИ WEB 2.0}

\def\aut{А.~Сорокин}


\def\titkol{Автоматизация за пределами Web 2.0}

\def\autkol{А.~Сорокин}

%{\renewcommand{\thefootnote}{\fnsymbol{footnote}}
%\footnotetext[1]{Работа проводится при финансовой поддержке Программы
%стратегического развития Петрозаводского государственного университета в рамках
%на\-уч\-но-ис\-сле\-до\-ва\-тель\-ской деятельности.}}


\titel{\tit}{\aut}{\autkol}{\titkol}

\vspace*{-12pt}

\noindent
IBM EE/A, Пресненская наб. 10, Москва 123317, Россия

\vspace*{6pt}

\def\leftfootline{\small{\textbf{\thepage}
\hfill ИНФОРМАТИКА И ЕЁ ПРИМЕНЕНИЯ\ \ \ том\ 8\ \ \ выпуск\ 4\ \ \ 2014}
}%
 \def\rightfootline{\small{ИНФОРМАТИКА И ЕЁ ПРИМЕНЕНИЯ\ \ \ том\ 8\ \ \ выпуск\ 4\ \ \ 2014
\hfill \textbf{\thepage}}}



\Abst{Рассматривается новый подход к анализу эволюции информационных
систем, основанный на разработанном автором стеке активностей. С~помощью введенного
подхода исследуются перспективные тенденции построения на платформе Вэб  информационных
систем, которые начинаются с~автоматизации отдельных биз\-нес-про\-цес\-сов и~затем,
в~результате дальнейшей экспансии информационных технологий (ИТ), охватывают области профессиональной деятельности.
В~результате классический подход к~проектированию информационных систем, базирующийся на
устранении посредством автоматизации узких мест биз\-нес-про\-цес\-сов, перестает работать.
Текущие тенденции в~развитии ИТ, связанные с~новыми возможностями <<ортогональной>>
интеграции систем, делают вероятным появление нового типа больших информационных систем и
нового типа их Вэб-ар\-хи\-тек\-ту\-ры, условно названного в~данной работе <<спиралью
паутины>>. По сравнению с~<<радиальной>> интеграцией Вэб в~рамках одной
профессиональной области такой тип архитектуры является более эффективным.}

\KW{автоматизация; реинжиниринг бизнес-процессов;
совместная разработка программных продуктов; экономика масштабирования;
ин\-тер\-нет-то\-по\-ло\-гия; социотехнические системы; системы систем;
виртуальные предприятия; Вэб 2.0}

\DOI{10.14357/19922264140414}

%\vspace*{6pt}


 \begin{multicols}{2}

\renewcommand{\bibname}{\protect\rmfamily Литература}
%\renewcommand{\bibname}{\large\protect\rm References}

{\small\frenchspacing
{%\baselineskip=10.8pt
\begin{thebibliography}{99}


\bibitem{1-sor-1}
Gartner's hype cycle. {\sf http://www.gartner.com/ technology/research/methodologies/hype-cycle.jsp}.

\bibitem{3-sor-1} %2
News tagged in metamaterials.
 {\sf http://phys.org/tags/ metamaterials/}.

 \bibitem{2-sor-1} %3
\Au{Maslow A.}
{Motivation and personality}. New York, N.Y.: Harper\&Row Publs.
Inc., 1954. P.~15--31.

\bibitem{4-sor-1}
\Au{Emery F.\,E., Trist E.\,L.} Socio-technical systems~// \textit{Management science, models
and techniques}~/ Eds.\  C.\,W.~Churchman and  M.~Verhurst.~---  London: Pergamon Press,
 1960. Vol.~2. P.~83--97.

 \bibitem{10-sor-1} %5
Internet of things market forecast.
{\sf http://postscapes. com/internet-of-things-market-size}.

\bibitem{5-sor-1} %6
\Au{Korsten~P., Seider~Ch.}
 The world's 4~trillion dollar challenge. Using a~system-of-systems approach to build a~smarter
planet.
 IBM Institute for Business Value, 2010.
 IBM Global Business Services Executive Report. {\sf
http://www-05.ibm.com/tr/events/ ibmcozumlerzirvesi2011/pdf/GBE03278USEN.PDF}.
\bibitem{6-sor-1} %7
IBM Smarter Planet publications. {\sf
http://www.ibm. com/smarterplanet/us/en/overview/ideas/index.html? re=sph};
{\sf http://www.ibm.com/smarterplanet/ru/ru/};
{\sf http://en.wikipedia.org/wiki/Smarter\_Planet};
{\sf http://www.ibm.com/smarterplanet/us/en/?ca=v\_\linebreak smarterplanet} (accessed June~17, 2014).
\bibitem{7-sor-1}
DiFac Success story. {\sf
http://www.ims.org/wp-content/uploads/2012/03/DiFac-SUCCESS-STORY\_\linebreak 100917.pdf}.

\bibitem{8-sor-1}
\Au{Lee H., Yang Y., Chae H., Nam S., Choi D., Tang\-chai\-sin~P., Herath~C., Marru~S.,
Nephew~K.\,P., Kim~S.}
BioVLAB-MMIA: A~cloud environment for microRNA and mRNA integrated analysis (MMIA) on
Amazon EC2~// IEEE Trans. Nanobiosci., 2012. Vol.~11. No.\,3. P.~266--272.
doi: 10.1109/TNB.2012.2212030.
\bibitem{9-sor-1} %10
IBM Corporation, International Technical Support Organization.
IBM Intelligent Operations Center for Smarter Cities.
IBM Redbooks Solution Guide. November~15, 2012.


\end{thebibliography}
} }

\end{multicols}

 \label{end\stat}

 \vspace*{-6pt}

\hfill{\small\textit{Поступила в редакцию 10.06.2014}}
%\renewcommand{\bibname}{\protect\rm Литература}
\renewcommand{\figurename}{\protect\bf Рис.} %14

%%%%%%%%%%%%%%%%%%%%%%%%%%%%%%%%%%%%%%%%%%%%%%%

%\def\stat{rez}
{%\hrule\par
%\vskip 7pt % 7pt
\raggedleft\Large \bf%\baselineskip=3.2ex
Р\,Е\,Ц\,Е\,Н\,З\,И\,И \vskip 17pt
    \hrule
    \par
\vskip 6pt plus 6pt minus 3pt }

%\thispagestyle{headings} %с верхним колонтитулом
%\thispagestyle{myheadings} %с нижним колонтитулом, но в верхнем РЕЦЕНЗИИ

\def\tit{НОВАЯ КНИГА И.\,Н.~СИНИЦЫНА, А.\,С.~ШАЛАМОВА <<ЛЕКЦИИ ПО ТЕОРИИ 
ИНТЕГРИРОВАННОЙ ЛОГИСТИЧЕСКОЙ ПОДДЕРЖКИ>> (М.: ТОРУС ПРЕСС, 2012. 624~с.)}

%1
\def\aut{Д.ф.-м.н., профессор С.\,Я.~Шоргин}

\def\auf{\ }

\def\leftkol{\ % РЕЦЕНЗИИ
}

\def\rightkol{ \ } 

%\def\leftkol{\ } % ENGLISH ABSTRACTS}

%\def\rightkol{\ } %ENGLISH ABSTRACTS}

%\def\leftkol{РЕЦЕНЗИИ}

%\def\rightkol{РЕЦЕНЗИИ}

\titele{\tit}{\aut}{\auf}{\leftkol}{\rightkol}
\vspace*{-18pt}


     \label{st\stat}

     \begin{multicols}{2}
     {\small
     {\baselineskip=10.1pt
     

      В книге представлено системное изложение теоретических основ одного из новейших 
направлений в \mbox{об\-ласти} экономики послепродажного обслуживания изделий наукоемкой 
продукции (ИНП) длительного пользования~--- интегрированной логистической поддержки
(ИЛП). 
{\looseness=1

}

Приведены также результаты новых работ, выполненных в Институте проблем информатики 
Российской академии наук в рамках научного направления <<Информационные технологии и 
анализ сложных сис\-тем>>.
 {%\looseness=1

}
     
      Излагаемые в книге научные подходы позво\-ляют карди\-наль\-но реформировать 
существующие системы производства и эксплуатации ИНП путем создания и внед\-ре\-ния 
методов рационального и оптимального управ\-ле\-ния процессами расходования 
вре\-мен\-н$\acute{\mbox{ы}}$х, 
мате\-ри\-аль\-ных, трудовых и других ресурсов на всех стадиях жизненного цикла изделий (ЖЦИ) по 
критериям экономической целесообразности и эф\-фек\-тив\-ности.
  {\looseness=1

}
    
      В книге приведен краткий обзор причин возник\-новения и
      развития CALS-методологии как основы 
современных международных стандартов по созданию и функционированию глобальных 
ин\-фор\-ма\-ци\-он\-но-ком\-му\-ни\-ка\-ци\-он\-ных систем, ее ключевых возможностей и эффективности 
результатов ее использования. 
Авторы %\linebreak 
предлагают ряд научных обоснований для разработки 
единой теории проектирования и управления систем ИЛП для полноценного использования 
преимуществ %\linebreak
 суще\-ст\-ву\-ющей методологии, определяют \mbox{общую} структурную схему 
комплексной системы <<ИНП-СППО>> и необходимость разработки для ее описания 
гибридных стохастических моделей.
{%\looseness=1

}

%\columnbreak
      
      Книга состоит из пяти частей, где последовательно излагается материал по каждой из 
следующих тем: <<Интегрированная логистическая поддержка>>, <<Теория гибридных 
стохастических систем и компьютерная поддержка исследований и разработок>>, <<Основы 
математического моделирования, анализа и синтеза систем послепродажного обслуживания>>, 
<<Определение и анализ показателей экспортного потенциала ИНП при проектировании>>, 
<<Задачи управления поддержкой послепродажного обслуживания>>, а также 
<<Моделирование инвестиционных процессов ИЛП в условиях неравновесных финансовых 
рынков>>. 
   
      В конце каждой главы приведены выводы и даны вопросы и задания для 
самоконтроля. В~приложениях содержатся основные определения по программам работ по 
анализу ИЛП, логистическим базам данных и компьютерным решениям, эквивалентной статистической 
линеаризации нелинейных преобразований ИЛП, справочный материал, а также развернутые 
уравнения для вероятностных характеристик.


      \def\leftkol{РЕЦЕНЗИИ}

\def\rightkol{РЕЦЕНЗИИ} 

      
      Книга заинтересует широкий круг специалистов и может быть использована научными 
проектными организациями в сфере промышленного производства ИНП. Большое количество 
иллюстраций, примеров и вопросов, обращенных к читателю, позволяет использовать книгу 
также в качестве учебного пособия для студентов и аспирантов машиностроительных, 
транспортных и~других специальностей, а также для самостоятельного изучения. 
{%\looseness=-1

}

Книга 
представляет несомненный интерес для специалистов и студентов в области прикладной 
математики и информатики.
    

}

}
\end{multicols}

%\newpage

\def\stat{authorsrus}
{%\hrule\par
%\vskip 7pt % 7pt
\raggedleft\Large \bf%\baselineskip=3.2ex
О\,Б\ \ А\,В\,Т\,О\,Р\,А\,Х \vskip 17pt
    \hrule
    \par
\vskip 21pt plus 8pt minus 4pt }


\def\tit{\ }

\def\aut{\ }

\def\auf{\ }

\def\leftkol{\ } % ENGLISH ABSTRACTS}

\def\rightkol{ОБ АВТОРАХ} %ENGLISH ABSTRACTS}

\titele{\tit}{\aut}{\auf}{\leftkol}{\rightkol}
      
            \label{st\stat}



\vspace*{24pt}

\begin{multicols}{2}




\noindent
\textbf{Архипов Олег Петрович} (р.\ 1948)~---
кандидат технических наук, директор Орловского филиала Института проб\-лем информатики
Российской академии наук
%302025, г.Орел, Московское шоссе, д.137

\vspace*{3pt}

\noindent
\textbf{Бирюкова Татьяна Константиновна} (р.\ 1968)~---
кандидат фи\-зи\-ко-ма\-те\-ма\-ти\-че\-ских наук, старший научный сотрудник Института проб\-лем информатики
Российской академии наук

\vspace*{3pt}

\noindent 
\textbf{Бобков  Сергей Геннадьевич} (р.\ 1955)~---
доктор технических наук,  заведующий отделением На\-уч\-но-ис\-сле\-до\-ва\-тель\-ско\-го 
института системных исследований Российской академии наук
%117218, Москва, Нахимовский просп., 36, к.1 

\vspace*{3pt}

\noindent \textbf{Васильев Николай Семенович} (р.\ 1952)~--- доктор 
фи\-зи\-ко-ма\-те\-ма\-ти\-че\-ских наук, профессор, 
МГТУ им.\ Н.\,Э.~Баумана 
%, Москва 105005, 2-я Бауманская ул., д.~5,

\vspace*{3pt}

\noindent
\textbf{Гершкович Максим Михайлович} (р.\ 1968)~---
старший научный сотрудник Института проб\-лем информатики
Российской академии наук

\vspace*{3pt}

\noindent 
\textbf{Дьяченко Юрий Георгиевич} (р.\ 1958)~--- кандидат технических наук, 
старший научный сотрудник Института проб\-лем информатики
Российской академии наук

\vspace*{3pt}

\noindent 
\textbf{Ерошенко Александр Андреевич} (р.\ 1989)~--- аспирант кафедры 
математической статистики факультета вычисли\-тельной математики и кибернетики 
Московского государственного университета им.\ М.\,В.~Ломоносова
%119991, Москва ГСП-1, Ленинские горы, д.\ 1, стр. 52

\vspace*{3pt}
 
\noindent 
\textbf{Захаров Виктор Николаевич} (р.\ 1948)~--- 
доктор технических наук, доцент, ученый секретарь Института проб\-лем информатики
Российской академии наук

\vspace*{3pt}

\noindent
\textbf{Зейфман Александр Израилевич} (р.\ 1954)~---
доктор фи\-зи\-ко-ма\-те\-ма\-ти\-че\-ских наук, профессор, 
заведующий кафедрой Вологодского государственного университета; 
старший научный сотрудник Института проб\-лем информатики
Российской академии наук; главный научный сотрудник ИСЭРТ Российской академии наук

\vspace*{3pt}

\noindent
\textbf{Зыкин Сергей Владимирович} (р.\ 1959)~--- 
доктор технических наук, профессор, заведующий лабораторией Института математики 
им.\ С.\,Л.~Соболева Сибирского отделения Российской академии наук, Новосибирск 
%630090, пр.\ ак.\ Коптюга, 4 

\vspace*{4pt}

\noindent
\textbf{Киреев Владимир Иванович} (р.\ 1938)~---
доктор фи\-зи\-ко-ма\-те\-ма\-ти\-че\-ских наук, профессор Московского 
государственного горного университета
%Адрес: Россия, 119991, г. Москва, Ленинский проспект, д. 6

%\columnbreak

\vspace*{4pt}

\noindent
\textbf{Козеренко Елена Борисовна} (р.\ 1959)~---
кандидат филологических наук, заведующая лабораторией Института проб\-лем информатики
Российской академии наук

\vspace*{4pt}

\noindent
\textbf{Королев Виктор Юрьевич} (р.\ 1954)~--- доктор
фи\-зи\-ко-ма\-те\-ма\-ти\-че\-ских наук, профессор кафедры математической 
статистики факультета вычисли\-тельной математики и кибернетики 
Московского государственного университета; 
ведущий научный сотрудник Института проб\-лем информатики
Российской академии наук

\vspace*{4pt}

\noindent
\textbf{Коротышева Анна Владимировна} (р.\ 1988)~---
старший преподаватель Вологодского государственного университета

\vspace*{4pt}

\noindent 
\textbf{Кун Де Турк} (р.\ 1981)~--- научный сотрудник 
исследовательской группы SMACS факультета телекоммуникаций и обработки информации
Университета Гента, Бельгия
%В-9000 Гент, Бельгия

\vspace*{4pt}

\noindent
\textbf{Лупенцов Олег Сергеевич} (р.\ 1986)~---
аспирант Омского государственного института сервиса
%Омск 644043, ул.\ Певцова 13

\vspace*{4pt}

\noindent
\textbf{Лучко Олег Николаевич} (р.\ 1961)~---
кандидат педагогических наук, профессор, заведующий кафедрой 
Омского государственного института сервиса
%Омск 644043, ул.\ Певцова 13

\vspace*{4pt}

\noindent
\textbf{Малашенко Юрий Евгеньевич} (р.\ 1946)~---
доктор фи\-зи\-ко-ма\-те\-ма\-ти\-че\-ских наук, заведующий сектором 
Вычислительного центра им.\ А.\,А.~Дородницына Российской академии наук
%Адрес: 119333, Москва, ул. Вавилова, 40,

\vspace*{4pt}

\noindent
\textbf{Маньяков Юрий Анатольевич} (р.\ 1984)~---
кандидат технических наук, научный сотрудник Орловского филиала Института проб\-лем информатики
Российской академии наук
%302025, г.Орел, Московское шоссе, д.137

\vspace*{4pt}

\noindent
\textbf{Маренко Валентина Афанасьевна} (р.\ 1951)~---
кандидат технических наук, доцент, старший научный сотрудник 
Института математики им.\ С.\,Л.~Соболева Сибирского отделения Российской академии наук
%Новосибирск 630090, пр. ак. Коптюга, 4 

\vspace*{3pt}

\noindent 
\textbf{Морозов Евсей Викторович} (р.\ 1947)~--- доктор 
фи\-зи\-ко-ма\-те\-ма\-ти\-че\-ских, профессор, ведущий научный сотрудник 
Института прикладных математических исследований Карельского научного центра Российской
академии наук; 
%%185910 Россия, Республика Карелия, г.\ Петрозаводск, ул.\ Пушкинская, 11
профессор Петрозаводского государственного университета, Петрозаводск
%185910 Россия, Республика Карелия, г.\ Петрозаводск, пр.\ Ленина, 33

%\pagebreak

\vspace*{3pt}

\noindent
\textbf{Назарова Ирина Александровна} (р.\ 1966)~---
кандидат фи\-зи\-ко-ма\-те\-ма\-ти\-че\-ских наук, 
научный сотрудник Вычислительного центра им.\ А.\,А.~Дородницына Российской академии наук 
%Адрес: 119333, Москва, ул. Вавилова, 40

\vspace*{3pt}

\noindent
\textbf{Павлов Игорь Валерианович} (р.\ 1945)~--- 
доктор фи\-зи\-ко-ма\-те\-ма\-ти\-че\-ских наук, профессор МГТУ им.\ Н.\,Э.~Баумана 
%Москва 105005, 2-я Бауманская ул., д.~5 

%\pagebreak

\vspace*{3pt}

\noindent 
\textbf{Потахина Любовь Викторовна} (р.\ 1989)~--- аспирантка
Института прикладных математических исследований Карельского научного центра
Российской академии наук; 
%%185910 Россия, Республика Карелия, г.\ Петрозаводск, ул.\ Пушкинская, 11
инженер Петрозаводского государственного университета, Петрозаводск
%185910 Россия, Республика Карелия, г.\ Петрозаводск, пр.\ Ленина, 33

\vspace*{3pt}

\noindent 
\textbf{Рождественский Юрий Владимирович} (р.\ 1952)~--- 
кандидат технических наук, заведующий сектором Института проб\-лем информатики
Российской академии наук

\vspace*{3pt}

\noindent 
\textbf{Синицын Игорь Николаевич} (р.\ 1940)~--- доктор технических наук,
профессор, заслуженный деятель\linebreak\vspace*{-12pt}

\columnbreak

\noindent
 науки РФ, заведующий отделом Института проб\-лем информатики
Российской академии наук

\vspace*{7pt}


\noindent
\textbf{Сиротинин Денис Олегович} (р.\ 1984)~---
кандидат технических наук, научный сотрудник Орловского филиала Института проб\-лем информатики
Российской академии наук
%302025, г.Орел, Московское шоссе, д.137

\vspace*{7pt}

%\columnbreak

\noindent 
\textbf{Соколов  Игорь Анатольевич} (р.\ 1954)~--- академик (действительный член) Российской 
академии наук, доктор технических наук, директор Института проб\-лем информатики
Российской академии наук

\vspace*{7pt}

\noindent
\textbf{Степченков Юрий Афанасьевич} (р.\ 1951)~---
кандидат технических наук, заведующий отделом Института проб\-лем информатики
Российской академии наук

\vspace*{7pt}

\noindent
\textbf{Сурков Алексей Викторович} (р.\ 1978)~--- 
старший научный сотрудник На\-уч\-но-ис\-сле\-до\-ва\-тель\-ско\-го 
института системных исследований Российской академии наук
%117218, Москва, Нахимовский просп., 36, к.1 

\vspace*{7pt}

\noindent 
\textbf{Шестаков Олег Владимирович} (р.\ 1976)~--- доктор 
фи\-зи\-ко-ма\-те\-ма\-ти\-че\-ских, доцент кафедры математической статистики 
факультета вычисли\-тельной математики и кибернетики Московского 
государственного университета им.\ М.\,В.~Ломоносова; 
%119991, Москва ГСП-1, Ленинские горы, д.\ 1, стр. 52
старший научный сотрудник Института проб\-лем информатики
Российской академии наук
%, Москва 119333, ул. Вавилова, д.~44, корп.~2

\vspace*{7pt}

\noindent 
\textbf{Шоргин Сергей Яковлевич} (р.\ 1952.)~--- доктор
фи\-зи\-ко-ма\-те\-ма\-ти\-че\-ских наук, профессор, заместитель директора Института 
проб\-лем информатики Российской академии наук





%%%%%%%%%%%%%%%%%%%%%%%%%%%%%%%%%%%%%%%%%%%%%%%%%%%%%%%%%%%%%%%%%%%%%%%%%%%%%%%




%\def\rightkol{ОБ АВТОРАХ}
%\def\leftkol{ОБ АВТОРАХ}

 \label{end\stat}





%\def\leftfootline{\small{\textbf{\thepage}
%\hfill ИНФОРМАТИКА И ЕЁ ПРИМЕНЕНИЯ\ \ \ том~7\ \ \ выпуск~1\ \ \ 2013}
%}%
% \def\rightfootline{\small{ИНФОРМАТИКА И ЕЁ ПРИМЕНЕНИЯ\ \ \ том~7\ \ \ выпуск~1\ \ \ 2013
%\hfill \textbf{\thepage}}}


%\thispagestyle{myheadings}



\end{multicols}

\newpage

%\end{document}

%
\def\stat{rekl}
%\label{preobr}

%\def\tit{АКАДЕМИК ПУГАЧЁВ  ВЛАДИМИР СЕМЁНОВИЧ\\
%25.03.1911--25.03.1998}


%   \vspace*{-48pt}
%   \begin{center}\LARGE
%Академик Пугачёв  Владимир Семёнович\\ (25.03.1911--25.03.1998)
%   \end{center}

   %\vspace*{2.5mm}

   \begin{center}

{\prgsh\LARGE
ЮБИЛЕИ}

\end{center}
%\hrule

\vspace*{6pt}


   \vspace*{8mm}

   \thispagestyle{empty}


%\def\stat{emel}


\section*{К 70-летию заместителя директора ИПИ РАН,\\ члена редколлегии журнала
<<Информатика и её применения>>\\ доктора технических наук В.\,И.~Будзко}

\vspace*{18pt}




          \begin{multicols}{2}

%            \label{st\stat}

\begin{center}
\vspace*{1pt}
\mbox{%
\epsfxsize=78mm
\epsfbox{bud-1.eps}
}
\end{center}

\vspace*{12pt}

      14 августа 2014~г.\ исполнилось 70~лет за\-мес\-ти\-те\-лю директора ИПИ РАН по
научной работе доктору технических наук Владимиру Игоревичу Будзко.

      Владимир Игоревич Будзко родился в г.~Москве. Высшее образование получил на факультете
элект\-рон\-но-вы\-чис\-ли\-тель\-ных устройств в Московском
ин\-же\-нер\-но-фи\-зи\-че\-ском институте
(МИФИ), который он окончил в 1968~г., после чего был на\-прав\-лен для прохождения
службы в одну из войс\-ко\-вых частей, где прошел путь от инженера до первого заместителя
командира войсковой части.

      С приходом В.\,И.~Будзко в ИПИ РАН (2001~г.)\ в институте
сформировалось новое научное на\-прав\-ле\-ние теоретических исследований~--- <<Постро\-ение
ин\-фор\-ма\-ци\-он\-но-те\-ле\-ком\-му\-ни\-ка\-ци\-он\-ных\linebreak сис\-тем
высокой до\-ступ\-ности>>. В~рамках этого
направления выполнен широкий круг фундаментальных исследований по поиску подходов и
определению принципов построения средств обеспечения доступности, конфиденциальности
и целостности современных крупномасштабных
ин\-фор\-ма\-ци\-он\-но-те\-ле\-ком\-му\-ни\-ка\-ци\-он\-ных
сис\-тем (ИТС). Разработаны основные сис\-тем\-но-тех\-ни\-че\-ские принципы и базовые
архитектурные решения построения перспективных для условий России ИТС с
централизованной обработкой и хранением информации, сочетающих в себе свойства
высокой доступности, отказо- и катастрофоустойчивости, информационной защищенности.
Определены принципы, методы и математические основы рационального построения и
оптимизации средств восстановления функционирования центров обработки данных (ЦОД)
после возникновения отказов и катастроф, передачи и хранения данных, обеспечения
информационной безопасности при достижении минимальной совокупной стоимости
владения такими системами. Результаты нашли практическое воплощение при реализации
проектов в интересах ряда отечественных государственных и негосударственных
организаций, таких как Банк России (БР), Внешторгбанк, ОАО <<ГМК <<Норильский Никель>>,
<<Газпром>>, Минэкономразвития России, Правительство Москвы, а также ряд силовых
ведомств.

      Под руководством В.\,И.~Будзко начиная с 2001~г.\ выполнен комплекс
      на\-уч\-но-ис\-сле\-до\-ва\-тель\-ских и
      опыт\-но-кон\-ст\-рук\-тор\-ских работ (свыше 100~проектов),
направленных на развитие электронной информационной технологии БР.
Разработаны концепции развития ИТС БР сначала до 2008~г., а затем до 2013~г., которые
были приняты в качестве основы проведения технической политики. За реализацию проекта
<<Катастрофоустойчивая тер\-ри\-то\-ри\-аль\-но-рас\-пре\-де\-лен\-ная
      ин\-фор\-ма\-ци\-он\-но-те\-ле\-ком\-му\-ни\-ка\-ци\-он\-ная сис\-те\-ма централизованной
обработки банковской информации>> В.\,И.~Будзко удостоен Премии Правительства РФ в
области науки и техники за 2010~г.

      В.\,И.~Будзко возглавлял и возглавляет работы по ряду других прикладных проектов,
связанных с созданием, совершенствованием и развитием крупномасштабных ИТС.

      В.\,И.~Будзко~--- генерал-майор, доктор технических наук, член-кор\-рес\-пон\-дент
Академии криптографии РФ, известный ученый в области информатики и применения
информационных технологий при построении территориально распределенных ИТС
различного назначения. Является автором свыше 250~научных работ, опубликованных в
на\-уч\-но-тех\-ни\-че\-ских и специальных изданиях.

    \thispagestyle{empty}

      В.\,И.~Будзко уделяет большое внимание подготовке научных кадров. Под его
руководством защищено 6~диссертаций на соискание ученой степени кандидата
технических наук. Свыше 30~лет он читает лекции в ИКСИ Академии ФСБ, профессор
кафедры НИЯУ МИФИ. Является членом двух диссертационных советов, главным
редактором журнала <<Системы высокой доступности>> и членом редколлегии журнала
<<Информатика и её применения>>.

      \bigskip

      Редакционный совет и Редакционная коллегия журнала <<Информатика и её
применения>> сердечно поздравляют Владимира Игоревича Будзко с 70-ле\-ти\-ем и желают
крепкого здоровья и новых научных достижений.

\end{multicols}

\def\stat{cont}
{%\hrule\par
%\vskip 7pt % 7pt
\raggedleft\Large \bf%\baselineskip=3.2ex
А\,В\,Т\,О\,Р\,С\,К\,И\,Й\ \ У\,К\,А\,З\,А\,Т\,Е\,Л\,Ь\ \ З\,А\ \ 2\,0\,1\,0 г. \vskip 17pt
    \hrule
    \par
\vskip 21pt plus 6pt minus 3pt }

\label{st\stat}

\def\tit{\ }

\def\aut{\ }
\def\auf{\ }

\def\leftkol{\ } % ENGLISH ABSTRACTS}

\def\rightkol{\ } %АВТОРСКИЙ УКАЗАТЕЛЬ ЗА 2010 г.} %ENGLISH ABSTRACTS}

\titele{\tit}{\aut}{\auf}{\leftkol}{\rightkol}

\vspace*{-12pt}

{\tabcolsep=3pt
\begin{tabular}{p{388pt}rr}
&\textbf{Выпуск} & \textbf{Стр.}\\[6pt]
\hangindent=23pt\noindent\textbf{Арутюнян~А.\,Р.} Моделирование влияния деформаций отпечатков пальцев на 
точность\linebreak
\vspace*{-12pt}\\
\hspace*{23pt}дактилоскопической идентификации$\dotfill$&1&51\\
\hangindent=23pt\noindent\textbf{Архипов~О.\,П., Зыкова~З.\,П.} Интеграция гетерогенной информации о цветных 
пикселях\linebreak
\vspace*{-12pt}\\
\hspace*{23pt}и их цветовосприятии$\dotfill$&4&15\\
\hangindent=23pt\noindent\textbf{Баранов~С.\,И., Френкель~С.\,Л., Захаров~В.\,Н.} Полуформальная верификация 
цифрового устройства с конвейером, основанная на использовании алгоритмических машин\linebreak
\vspace*{-12pt}\\
\hspace*{23pt}состояния$\dotfill$&4&49\\
\textbf{Бекетова~И.\,В.} см.~Каратеев~С.\,Л.&&\\
\textbf{Белоусов~В.\,В.} см.~Синицын~И.\,Н.&&\\
\hangindent=23pt\noindent\textbf{Бенинг~В.\,Е., Королев~Р.\,А.} О предельном поведении мощностей критериев в 
случае\linebreak
\vspace*{-12pt}\\
\hspace*{23pt}распределения Лапласа$\dotfill$&2&63\\
\hangindent=23pt\noindent\textbf{Бенинг~В.\,Е., Сипина~А.\,В.} Асимптотическое разложение для мощности 
критерия,\linebreak
\vspace*{-12pt}\\
\hspace*{23pt}основанного на выборочной медиане, в случае распределения Лапласа$\dotfill$&1&18\\
\textbf{Бондаренко~А.\,В.} см.~Каратеев~С.\,Л.&&\\
\hangindent=23pt\noindent\textbf{Бородина~А.\,В., Морозов~Е.\,В.} Об оценивании асимптотики вероятности 
большого\linebreak
\vspace*{-12pt}\\
\hspace*{23pt}уклонения стационарной регенеративной очереди с одним прибором$\dotfill$&3&29\\
\hangindent=23pt\noindent\textbf{Бунтман~Н.\,В., Минель~Ж.-Л., Ле~Пезан~Д., Зацман~И.\,М.} Типология и 
компьютерное\linebreak
\vspace*{-12pt}\\
\hspace*{23pt}моделирование трудностей перевода$\dotfill$&3&77\\
\textbf{Визильтер~Ю.\,В.} см.~Каратеев~С.\,Л.&&\\
\hangindent=23pt\noindent\textbf{Гавриленко~С.\,В.} Оценки скорости сходимости распределений случайных сумм с 
безгранично делимыми индексами к нормальному закону$\dotfill$&4&81\\
\hangindent=23pt\noindent\textbf{Григорьева~М.\,Е., Шевцова~И.\,Г.} Уточнение неравенства 
Каца--Берри--Эссеена$\dotfill$&2&75\\
\hangindent=23pt\noindent\textbf{Грушо~А.\,А., Грушо~Н.\,А., Тимонина~Е.\,Е.} Поиск конфликтов в политиках 
безопасности: модель случайных графов$\dotfill$&3&38\\
\textbf{Грушо~Н.\,А.} см.~Грушо~А.\,А.&&\\
\hangindent=23pt\noindent\textbf{Гудков~В.\,Ю.} Математические модели изображения отпечатка пальца на основе 
описания линий$\dotfill$&1&58\\
\textbf{Гуртов~А.\,В.} см.~Лукьяненко~А.\,С.&&\\
\textbf{Желтов~С.\,Ю.} см.~Каратеев~С.\,Л.&&\\
\hangindent=23pt\noindent\textbf{Захаров~А.\,А., Серебряков~В.\,А.} Система управления электронной библиотекой 
LibMeta$\dotfill$&4&2\\
\textbf{Захаров~В.\,Н.} см.~Баранов~С.\,И.&&\\
\textbf{Захарова~Т.\,В.} см.~Матвеева~С.\,С.&&\\
\hangindent=23pt\noindent\textbf{Зацаринный~А.\,А., Чупраков~К.\,Г.} Некоторые аспекты выбора технологии для 
постро-\linebreak
\vspace*{-12pt}\\
\hspace*{23pt}ения систем отображения информации ситуационного центра$\dotfill$&3&59\\
\textbf{Зацман~И.\,М.} см.~Бунтман~Н.\,В.&&\\
\hangindent=23pt\noindent\textbf{Зейфман~А.\,И., Коротышева~А.\,В., Сатин~Я.\,А., Шоргин~С.\,Я.} Об 
устойчивости нестаци-\linebreak
\vspace*{-12pt}\\
\hspace*{23pt}онарных систем обслуживания с катастрофами$\dotfill$&3&9\\
\textbf{Зыкова~З.\,П.} см.~Архипов~О.\,П.&&\\
\hangindent=23pt\noindent\textbf{Илюшин~Г.\,Я., Соколов~И.\,А.} Организация управляемого доступа пользователей 
к\linebreak
\vspace*{-12pt}\\
\hspace*{23pt}разнородным ведомственным информационным ресурсам$\dotfill$&1&24\\
\hangindent=23pt\noindent\textbf{Кавагучи~Ю., Ульянов~В.\,В., Фуджикоши~Я.} Приближения для статистик, 
описывающих\linebreak
\vspace*{-12pt}\\
\hspace*{23pt}геометрические свойства данных большой размерности, с оценками 
ошибок$\dotfill$&1&12\\
\hangindent=23pt\noindent\textbf{Каратеев~С.\,Л., Бекетова~И.\,В., Ососков~М.\,В., Князь~В.\,А., 
Визильтер~Ю.\,В., Бондаренко~А.\,В., Желтов~С.\,Ю.} Автоматизированный контроль 
качества цифровых\linebreak
\vspace*{-12pt}\\
\hspace*{23pt}изображений для персональных документов$\dotfill$&1&65\\
\end{tabular}
}

\pagebreak

\def\leftkol{АВТОРСКИЙ УКАЗАТЕЛЬ ЗА 2010 г.} % ENGLISH ABSTRACTS}

\def\rightkol{АВТОРСКИЙ УКАЗАТЕЛЬ ЗА 2010 г.} %ENGLISH ABSTRACTS}

{\tabcolsep=3pt
\begin{tabular}{p{388pt}rr}
&\textbf{Выпуск} & \textbf{Стр.}\\[3pt]
\hangindent=23pt\noindent\textbf{Козеренко~Е.\,Б.} Лингвистические фильтры в статистических моделях машинного\linebreak
\vspace*{-12pt}\\
\hspace*{23pt}перевода$\dotfill$&2&83\\
\hangindent=23pt\noindent\textbf{Козеренко~Е.\,Б., Кузнецов~И.\,П.} Когнитивно-лингвистические представления в 
систе-\linebreak
\vspace*{-12pt}\\
\hspace*{23pt}мах обработки текстов$\dotfill$&3&69\\
\textbf{Князь~В.\,А.} см.~Каратеев~С.\,Л.&&\\
\hangindent=23pt\noindent\textbf{Колесников~А.\,В., Солдатов~С.\,А.} Алгоритм координации для гибридной 
интеллектуальной системы решения сложной задачи оперативно-производственного\linebreak
\vspace*{-12pt}\\
\hspace*{23pt}планирования$\dotfill$&4&61\\
\hangindent=23pt\noindent\textbf{Коновалов~М.\,Г.} О планировании потоков в системах вычислительных 
ресурсов$\dotfill$&2&3\\
\textbf{Конушин~А.\,С.} см.~Конушин~В.\,С.&&\\
\hangindent=23pt\noindent\textbf{Конушин~В.\,С., Кривовязь~Г.\,Р., Конушин~А.\,С.} Алгоритм распознавания людей 
в видео-\linebreak
\vspace*{-12pt}\\
\hspace*{23pt}последовательности по одежде$\dotfill$&1&74\\
\textbf{Корепанов~Э.\, Р.} см.~Синицын~И.\,Н.&&\\
\textbf{Королев~В.\,Ю.} см.~Соколов~И.\,А.&&\\
\textbf{Королев~Р.\,А.} см.~Бенинг~В.\,Е.&&\\
\textbf{Коротышева~А.\,В.} см.~Зейфман~А.\,И.&&\\
\hangindent=23pt\noindent\textbf{Кривенко~М.\,П.} Непараметрическое оценивание элементов байесовского 
клас\-си-\linebreak
\vspace*{-12pt}\\
\hspace*{23pt}фикатора$\dotfill$&2&13\\
\textbf{Кривовязь~Г.\,Р.} см.~Конушин~В.\,С.&&\\
\textbf{Крылов~А.\,С.} см.~Павельева~Е.\,А.&&\\
\hangindent=23pt\noindent\textbf{Крылов~В.\,А.} Моделирование и классификация многоканальных дистанционных\linebreak
\vspace*{-12pt}\\
\hspace*{23pt}изображений с использованием копул$\dotfill$&4&34\\
\hangindent=23pt\noindent\textbf{Крючин~О.\,В.} Разработка параллельных эвристических алгоритмов подбора 
весовых\linebreak
\vspace*{-12pt}\\
\hspace*{23pt}коэффициентов искусственной нейтронной сети$\dotfill$&2&53\\
\hangindent=23pt\noindent\textbf{Кудрявцев~А.\,А., Шоргин~С.\,Я.} Байесовские модели массового обслуживания и 
надеж-\linebreak
\vspace*{-12pt}\\
\hspace*{23pt}ности: характеристики среднего числа заявок в системе $M\vert M \vert 1\vert 
\infty$$\dotfill$&3&16\\
\hangindent=23pt\noindent\textbf{Кузнецов~А.\,А.} Связь между временными и структурно-топологическими 
характери-\linebreak
\vspace*{-12pt}\\
\hspace*{23pt}стиками диаграмм ритма сердца здоровых людей$\dotfill$&4&39\\
\textbf{Кузнецов~И.\,П.} см.~Козеренко~Е.\,Б.&&\\
\textbf{Ле~Пезан~Д.} см.~Бунтман~Н.\,В.&&\\
\hangindent=23pt\noindent\textbf{Лукьяненко~А.\,С., Морозов~Е.\,В., Гуртов~А.\,В.} Анализ сетевого протокола с общей 
функ-\linebreak
\vspace*{-12pt}\\
\hspace*{23pt}цией расширения окна передачи сообщения при конфликтах$\dotfill$&2&46\\
\hangindent=23pt\noindent\textbf{Лямин~О.\,О.} О предельном поведении мощностей критериев в случае обобщенного\linebreak
\vspace*{-12pt}\\
\hspace*{23pt}распределения Лапласа$\dotfill$&3&47\\
\hangindent=23pt\noindent\textbf{Маркин~А.\,В., Шестаков~О.\,В.} Асимптотики оценки риска при пороговой 
обработке\linebreak
\vspace*{-12pt}\\
\hspace*{23pt}вейвлет-вейглет коэффициентов в задаче томографии$\dotfill$&2&36\\
\hangindent=23pt\noindent\textbf{Матвеева~С.\,С., Захарова~Т.\,В.} Сети массового обслуживания с наименьшей 
длиной\linebreak
\vspace*{-12pt}\\
\hspace*{23pt}очереди$\dotfill$&3&22\\
\hangindent=23pt\noindent\textbf{Матюшенко~С.\,И.} Стационарные характеристики двухканальной системы 
обслужива-\linebreak
\vspace*{-12pt}\\
\hspace*{23pt}ния с переупорядочиванием заявок и распределениями фазового типа$\dotfill$&4&68\\
\textbf{Минель~Ж.-Л.} см.~Бунтман~Н.\,В.&&\\
\textbf{Морозов~Е.\,В.} см.~Бородина~А.\,В.&&\\
\textbf{Морозов~Е.\,В.} см.~Лукьяненко~А.\,С.&&\\
\textbf{Ососков~М.\,В.} см.~Каратеев~С.\,Л.&&\\
\hangindent=23pt\noindent\textbf{Павельева~Е.\,А., Крылов~А.\,С.} Поиск и анализ ключевых точек радужной 
оболочки\linebreak
\vspace*{-12pt}\\
\hspace*{23pt}глаза методом преобразования Эрмита$\dotfill$&1&79\\
\textbf{Печинкин~А.\,В.} см.~Френкель~С.\,Л.,&&\\
\hangindent=23pt\noindent\textbf{Протасов~В.\,И.} Составление субъективного портрета с использованием 
эволюционно-\linebreak
\vspace*{-12pt}\\
\hspace*{23pt}го морфинга и квалиметрия метода$\dotfill$&1&83\\
\hangindent=23pt\noindent\textbf{Рудаков~К.\,В., Торшин~И.\,Ю.} Вопросы разрешимости задачи распознавания 
вторичной\linebreak
\vspace*{-12pt}\\
\hspace*{23pt}структуры белка$\dotfill$&2&25\\
\textbf{Сатин~Я.\,А.} см.~Зейфман~А.\,И.&&\\
\hangindent=23pt\noindent\textbf{Сейфуль-Мулюков~Р.\,Б.} Нефть как носитель информации о своем 
происхождении,\linebreak
\vspace*{-12pt}\\
\hspace*{23pt}структуре и эволюции$\dotfill$&1&41\\
\end{tabular}
}

{\tabcolsep=3pt
\begin{tabular}{p{388pt}rr}
&\textbf{Выпуск} & \textbf{Стр.}\\[6pt]
\textbf{Семендяев~Н.\,Н.} см.~Синицын~И.\,Н.&&\\
\textbf{Серебряков~В.\,А.} см.~Захаров~А.\,А.&&\\
\textbf{Синицын~В.\,И.} см.~Синицын~И.\,Н.&&\\
\hangindent=23pt\noindent\textbf{Синицын~И.\,Н., Синицын~В.\,И., Корепанов~Э.\, Р., Белоусов~В.\,В., 
Семендяев~Н.\,Н.} Оперативное построение информационных моделей движения полюса 
Земли\linebreak
\vspace*{-12pt}\\
\hspace*{23pt}методами линейных и линеаризованных фильтров$\dotfill$&1&2\\
\textbf{Сипина~А.\,В.} см.~Бенинг~В.\,Е.&&\\
\hangindent=23pt\noindent\textbf{Соколов~И.\,А.} О работах заслуженного деятеля науки Российской Федерации 
И.\,Н.~Синицына в области информационных технологий и автоматизации (к 70-летию\linebreak
\vspace*{-12pt}\\
\hspace*{23pt}со дня рождения)$\dotfill$&3&84\\
\textbf{Соколов~И.\,А.} см.~Илюшин~Г.\,Я.&&\\
\hangindent=23pt\noindent\textbf{Соколов~И.\,А., Королев~В.\,Ю.} Предисловие$\dotfill$&2&2\\
\textbf{Солдатов~С.\,А.} см.~Колесников~А.\,В.&&\\
\hangindent=23pt\noindent\textbf{Степанов~С.\,Ю.} Использование координатного метода фрагментации 
коммутаторной\linebreak
\vspace*{-12pt}\\
\hspace*{23pt}нейронной сети для сокращения трафика$\dotfill$&2&57\\
\textbf{Тимонина~Е.\,Е.} см.~Грушо~А.\,А.&&\\
\textbf{Торшин~И.\,Ю.} см.~Рудаков~К.\,В.&&\\
\textbf{Ульянов~В.\,В.} см.~Кавагучи~Ю.&&\\
\textbf{Фазекаш~И.} см.~Чупрунов~А.\,Н.&&\\
\textbf{Френкель~С.\,Л.} см.~Баранов~С.\,И.&&\\
\hangindent=23pt\noindent\textbf{Френкель~С.\,Л., Печинкин~А.\,В.} Оценка времени самовосстановления в 
цифровых\linebreak
\vspace*{-12pt}\\
\hspace*{23pt}системах после сбоев, вызываемых переходными помехами$\dotfill$&3&2\\
\textbf{Фуджикоши~Я.} см.~Кавагучи~Ю.&&\\
\hangindent=23pt\noindent\textbf{Цискаридзе~А.\,К.} Математическая модель и метод восстановления позы человека 
по\linebreak
\vspace*{-12pt}\\
\hspace*{23pt}стереопаре силуэтных изображений$\dotfill$&4&27\\
\hangindent=23pt\noindent\textbf{Чупраков~К.\,Г.} К вопросу о размещении коллективных средств отображения в 
ситуа-\linebreak
\vspace*{-12pt}\\
\hspace*{23pt}ционном зале с заданными параметрами$\dotfill$&4&89\\
\textbf{Чупраков~К.\,Г.} см.~Зацаринный~А.\,А.&&\\
\hangindent=23pt\noindent\textbf{Чупрунов~А.\,Н., Фазекаш~И.} Законы повторного логарифма для числа 
безошибочных\linebreak
\vspace*{-12pt}\\
\hspace*{23pt}блоков при помехоустойчивом кодировании$\dotfill$&3&42\\
\textbf{Шевцова~И.\,Г.} см.~Григорьева~М.\,Е.&&\\
\hangindent=23pt\noindent\textbf{Шестаков~О.\,В.} Аппроксимация распределения оценки риска пороговой 
обработки вейвлет-коэффициентов нормальным распределением при использовании 
выбо-\linebreak
\vspace*{-12pt}\\
\hspace*{23pt}рочной дисперсии$\dotfill$&4&73\\
\textbf{Шестаков~О.\,В.} см.~Маркин~А.\,В.&&\\
\textbf{Шоргин~С.\,Я.} см.~Зейфман~А.\,И.&&\\
\textbf{Шоргин~С.\,Я.} см.~Кудрявцев~А.\,А.&&\\
\end{tabular}
}

%\thispagestyle{myheadings}
\def\leftfootline{\small{\textbf{\thepage}
\hfill ИНФОРМАТИКА И ЕЁ ПРИМЕНЕНИЯ\ \ \ том~4\ \ \ выпуск~4\ \ \ 2010}
}%
 \def\rightfootline{\small{ИНФОРМАТИКА И ЕЁ ПРИМЕНЕНИЯ\ \ \ том~4\ \ \ выпуск~4\ \ \ 2010
 \hfill \textbf{\thepage}}}
 \label{end\stat}


%Том 10 Выпуск 1-4 Год 2016

\def\stat{cont-e}
{%\hrule\par
%\vskip 7pt % 7pt
\raggedleft\Large \bf%\baselineskip=3.2ex
2\,0\,1\,6\ \ A\,U\,T\,H\,O\,R\ \ I\,N\,D\,E\,X \vskip 17pt
 \hrule
 \par
\vskip 21pt plus 6pt minus 3pt }

\label{st\stat}

\def\tit{\ }

\def\aut{\ }
\def\auf{\ }

\def\leftkol{\ } %2016 AUTHOR INDEX} % ENGLISH ABSTRACTS}

\def\rightkol{\ } %2016 AUTHOR INDEX} %ENGLISH ABSTRACTS}

\titele{\tit}{\aut}{\auf}{\leftkol}{\rightkol}

\def\leftfootline{\small{\textbf{\thepage}
\hfill INFORMATIKA I EE PRIMENENIYA~--- INFORMATICS AND APPLICATIONS\ \ \ 2016\
\ \ volume~10\ \ \ issue\ 4}
}%
 \def\rightfootline{\small{INFORMATIKA I EE PRIMENENIYA~--- INFORMATICS AND APPLICATIONS\ \ \ 2016\ \ \ volume~10\ \ \ issue\ 4
\hfill \textbf{\thepage}}}

\vspace*{-12pt}
\vspace*{-18pt}

{\tabcolsep=2.8pt
\begin{tabular}{p{382pt}cc}
&\textbf{Issue} & \textbf{Page}\\[6pt]
\Avtors{Agalarov~M.\,Ya.} see~Agalarov~Ya.\,M.&&\\
\Avtors{Agalarov~Ya.\,M., Agalarov~M.\,Ya., and
Shorgin~V.\,S.} About the optimal threshold of queue\linebreak
\\[-12pt]
\hspace*{23pt}length in a~particular problem of profit maximization
in the $M/G/1$ queuing system&2&70--79\\
\Avtors{Alexeyevsky~D.\,A.} BioNLP ontology extraction from 
a~restricted language corpus with\linebreak
\\[-12pt]
\hspace*{23pt}context-free grammars&1&119--128\\
\Avtors{Andreev~S.\,D.} see~Gaidamaka~Yu.\,V.&&\\
\Avtors{Andreev~S.\,D.} see~Ometov~A.\,Ya.&&\\
\Avtors{Arkhipov~O.\,P., Arkhipov~P.\,O., and Sidorkin~I.\,I.} The
option to create a~local coordinate\linebreak
\\[-12pt]
\hspace*{23pt}system for synchronization of selected images&3&91--97\\
\Avtors{Arkhipov~P.\,O.} see~Arkhipov~O.\,P.&&\\
\Avtors{Belousov~V.\,V.} see~Shnurkov~P.\,V.&&\\
\Avtors{Belousov~V.\,V.} see~Shnurkov~P.\,V.&&\\
\Avtors{Bening~V.\,E.} Calculation of~the~asymptotic deficiency
of~some statistical procedures based\linebreak
\\[-12pt]
\hspace*{23pt}on~samples with~random sizes&4&34--45\\
\Avtors{Borisov~A.\,V., Bosov~A.\,V., and Miller~G.\,B.} Modeling and
monitoring of VoIP connection&2&\hphantom{1}2--13\\
\Avtors{Bosov~A.\,V.} see~Borisov~A.\,V.&&\\
\Avtors{Briukhov~D.\,O.} see~Stupnikov~S.\,A.&&\\
\Avtors{Callaos~N.\,K.\ and Seyful-Mulyukov~R.\,B.} Complexity and
its information content&1&129--139\\
\Avtors{Chertok~A.\,V., Kadaner~A.\,I., Khazeeva~G.\,T., and
Sokolov~I.\,A.} Regime switching detection\linebreak
\\[-12pt]
\hspace*{23pt}for~the~Levy driven
Ornstein--Uhlenbeck process using CUSUM methods&4&46--56\\
\Avtors{Chichagov~V.\,V.} Asymptotic expansions of mean absolute
error of uniformly minimum variance unbiased and maximum likelihood
estimators on the one-parameter exponential\linebreak
\\[-12pt]
\hspace*{23pt}family model of lattice distributions&3&66--76\\
\Avtors{Danishevsky~V.\,I.} see~Kolesnikov A.\,V.&&\\
\Avtors{Fazliev~A.\,Z.} see~Kalinichenko~L.\,A.&&\\
\Avtors{Fedoseev~A.\,A.} What is behind the concept of ``knowledge in
small packages''&3&105--110\\
\Avtors{Gaidamaka~Yu.\,V., Andreev~S.\,D., Sopin~E.\,S.,
Samouylov~K.\,E., and Shorgin~S.\,Ya.} Interference analysis
of~the~device-to-device communications model with~regard to~a~signal\linebreak
\\[-12pt]
\hspace*{23pt}propagation environment&4&\hphantom{1}2--10\\
\Avtors{Gasilov~A.\,V.} see~Yakovlev~O.\,A.&&\\
\Avtors{Goncharov~A.\,V.\ and Strijov~V.\,V.} Metric time series
classification using weighted dynamic\linebreak
\\[-12pt]
\hspace*{23pt}warping relative to centroids of classes&2&36--47\\
\Avtors{Gordov~E.\,P.} see~Kalinichenko~L.\,A.&&\\
\Avtors{Gorshenin~A.\,K.} Concept of online service for stochastic
modeling of real processes&1&72--81\\
\Avtors{Gorshenin~A.\,K.} see~Shnurkov~P.\,V.&&\\
\Avtors{Gorshenin~A.\,K.} see~Shnurkov~P.\,V.&&\\
\Avtors{Grusho~A.\,A., Grusho~N.\,A., Zabezhailo~M.\,I., and
Timonina~E.\,E.} Integration of statistical and\linebreak
\\[-12pt]
\hspace*{23pt}deterministic methods for
analysis of information security&3&2--8\\
\Avtors{Grusho~A.\,A., Zabezhailo~M.\,I., and Zatsarinny~A.\,A.} On
the advanced procedure to reduce\linebreak
\\[-12pt]
\hspace*{23pt}calculation of Galois closures&4&\hphantom{1}96--104\\
\Avtors{Grusho~N.\,A.} see~Grusho~A.\,A.&&\\
\Avtors{Havanskov~V.\,A.} see~Minin~V.\,A.&&\\
\Avtors{Inkova~O.\,Yu.} see~Zatsman~I.\,M.&&\\
\Avtors{Isachenko~R.\,V.\ and Strijov~V.\,V.} Metric learning in
multiclass time series classification\linebreak
\\[-12pt]
\hspace*{23pt}problem&2&48--57\\
\end{tabular}
}
\pagebreak

\def\leftfootline{\small{\textbf{\thepage}
\hfill INFORMATIKA I EE PRIMENENIYA~--- INFORMATICS AND APPLICATIONS\ \ \ 2016\
\ \ volume~10\ \ \ issue\ 4}
}%
 \def\rightfootline{\small{INFORMATIKA I EE PRIMENENIYA~---
INFORMATICS AND APPLICATIONS\ \ \ 2016\ \ \ volume~10\ \ \ issue\ 4
\hfill \textbf{\thepage}}}

\def\leftkol{2016 AUTHOR INDEX} % ENGLISH ABSTRACTS}

\def\rightkol{2016 AUTHOR INDEX} %ENGLISH ABSTRACTS}


{\tabcolsep=2.83pt
\begin{tabular}{p{382pt}cc}
&\textbf{Issue} & \textbf{Page}\\[6pt]
\Avtors{Kadaner~A.\,I.} see~Chertok~A.\,V.&&\\[.255pt]
\Avtors{Kalinichenko~L.\,A., Volnova~A.\,A., Gordov~E.\,P.,
Kiselyova~N.\,N., Kovaleva~D.\,A., Malkov~O.\,Yu., Okladnikov~I.\,G.,
Podkolodnyy~N.\,L., Pozanenko~A.\,S., Ponomareva~N.\,V.,
Stupnikov~S.\,A.,} \textbf{and Fazliev~A.\,Z.} Data access challenges for data
intensive\linebreak
\\[-12pt]
\hspace*{23pt}research in Russia&1& 2--22\\[.255pt]
\Avtors{Karasikov~M.\,E.\ and Strijov~V.\,V.} Feature-based
time-series classification&4&121--131\\[.255pt]
\Avtors{Khazeeva~G.\,T.} see~Chertok~A.\,V.&&\\[.255pt]
\Avtors{Khokhlov~Yu.\,S.} Multivariate fractional Levy motion and its
applications&2&\hphantom{1}98--106\\[.255pt]
\Avtors{Kirikov~I.\,A., Kolesnikov~A.\,V., Listopad~S.\,V., and
Rumovskaya~S.\,B.} Fine-grained hybrid\linebreak
\\[-12pt]
\hspace*{23pt}intelligent systems. Part 2:
Bidirectional hybridization&1&\hphantom{1}96--105\\[.255pt]
\Avtors{Kirikov~I.\,A., Kolesnikov~A.\,V., Listopad~S.\,V., and
Rumovskaya~S.\,B.} ``Virtual council''~---\linebreak
\\[-12pt]
\hspace*{23pt}source environment
supporting complex diagnostic decision making&3&81--90\\[.255pt]
\Avtors{Kiselyova~N.\,N.} see~Kalinichenko~L.\,A.&&\\[.255pt]
\Avtors{Kolesnikov A.\,V., Listopad~S.\,V., Rumovskaya~S.\,B., and
Danishevsky~V.\,I.} Informal axiomatic\linebreak
\\[-12pt]
\hspace*{23pt}theory of~the~role visual models&4&114--120\\[.255pt]
\Avtors{Kolesnikov~A.\,V.} see~Kirikov~I.\,A.&&\\[.255pt]
\Avtors{Kolesnikov~A.\,V.} see~Kirikov~I.\,A.&&\\[.255pt]
\Avtors{Kolin~K.\,K.} Humanitarian aspects of information
security&3&111--121\\[.255pt]
\Avtors{Konovalov~M.\,G.\ and Razumchik~R.\,V.} Dispatching
to~two parallel nonobservable queues using\linebreak
\\[-12pt]
\hspace*{23pt}only static
information&4&57--67\\[.255pt]
\Avtors{Korchagin~A.\,Yu.} see~Korolev~V.\,Yu.&&\\[.255pt]
\Avtors{Korchagin~A.\,Yu.} see~Korolev~V.\,Yu.&&\\[.255pt]
\Avtors{Korepanov~E.\,R.} see~Sinitsyn~I.\,N.&&\\[.255pt]
\Avtors{Korepanov~E.\,R.} see~Sinitsyn~I.\,N.&&\\[.255pt]
\Avtors{Korolev~V.\,Yu., Korchagin~A.\,Yu., and Zeifman~A.\,I.} The
Poisson theorem for Bernoulli trials\linebreak
\\[-12pt]
\hspace*{23pt}with~a~random probability
of~success and~a~discrete analog of~the~Weibull distribution&4&11--20\\[.255pt]
\Avtors{Korolev~V.\,Yu., Zeifman~A.\,I., and Korchagin~A.\,Yu.}
Asymmetric Linnik distributions as~limit\linebreak
\\[-12pt]
\hspace*{23pt}laws for~random sums
of~independent random variables with~finite variances&4&21--33\\[.255pt]
\Avtors{Koucheryavy~E.\,A.} see~Ometov~A.\,Ya.&&\\[.255pt]
\Avtors{Kovaleva~D.\,A.} see~Kalinichenko~L.\,A.&&\\[.255pt]
\Avtors{Kovalyov~S.\,P.} Metaprogramming to increase
manufacturability of large-scale software-\linebreak
\\[-12pt]
\hspace*{23pt}intensive systems&1&56--66\\[.255pt]
\Avtors{Krivenko~M.\,P.} Significance tests of feature selection for
classification&3&32--40\\[.255pt]
\Avtors{Kruzhkov~M.\,G.} see~Zalizniak~Anna~A.&&\\[.255pt]
\Avtors{Kruzhkov~M.\,G.} see~Zatsman~I.\,M.&&\\[.255pt]
\Avtors{Kudryavtsev~A.\,A.} Bayesian queueing and reliability models:
\textit{A~priori} distributions with\linebreak
\\[-12pt]
\hspace*{23pt}compact support&1&67--71\\[.255pt]
\Avtors{Kudryavtsev~A.\,A.} Characteristics dependent on the balance
coefficient in Bayesian models\linebreak
\\[-12pt]
\hspace*{23pt}with compact support of \textit{a priori}
distributions&3&77--80\\[.255pt]
\Avtors{Kudryavtsev~A.\,A.\ and Palionnaia~S.\,I.} Bayesian recurrent
model of reliability growth:\linebreak
\\[-12pt]
\hspace*{23pt}Parabolic distribution of parameters&2&80--83\\[.255pt]
\Avtors{Kudryavtsev~A.\,A.\ and Titova~A.\,I.} Bayesian queuing
and~reliability models: Degenerate-\linebreak
\\[-12pt]
\hspace*{23pt}Weibull case&4&68--71\\[.255pt]
\Avtors{Leontyev~N.\,D.\ and Ushakov~V.\,G.} Analysis of a queueing
system with autoregressive arrivals\linebreak
\\[-12pt]
\hspace*{23pt}and nonpreemptive priority&3&15--22\\[.255pt]
\Avtors{Listopad~S.\,V.} see~Kirikov~I.\,A.&&\\[.255pt]
\Avtors{Listopad~S.\,V.} see~Kirikov~I.\,A.&&\\[.255pt]
\Avtors{Listopad~S.\,V.} see~Kolesnikov A.\,V.&&\\[.255pt]
\Avtors{Malkov~O.\,Yu.} see~Kalinichenko~L.\,A.&&\\[.255pt]
\Avtors{Markov~A.\,S., Monakhov~M.\,M., and
Ulyanov~V.\,V.} Generalized Cornish--Fisher expansions\linebreak
\\[-12pt]
\hspace*{23pt}for distributions of statistics based on samples
of random size&2&84--91\\[.255pt]
\Avtors{Melnikov~A.\,K.\ and Ronzhin~A.\,F.} Generalized statistical
method of~text analysis based\linebreak
\\[-12pt]
\hspace*{23pt}on~calculation of~probability distributions
of~statistical values&4&89--95\\
\end{tabular}
}
\pagebreak

\def\leftfootline{\small{\textbf{\thepage}
\hfill INFORMATIKA I EE PRIMENENIYA~--- INFORMATICS AND APPLICATIONS\ \ \ 2016\
\ \ volume~10\ \ \ issue\ 4}
}%
 \def\rightfootline{\small{INFORMATIKA I EE PRIMENENIYA~---
INFORMATICS AND APPLICATIONS\ \ \ 2016\ \ \ volume~10\ \ \ issue\ 4
\hfill \textbf{\thepage}}}

\def\leftkol{2016 AUTHOR INDEX} % ENGLISH ABSTRACTS}

\def\rightkol{2016 AUTHOR INDEX} %ENGLISH ABSTRACTS}


{\tabcolsep=3pt
\begin{tabular}{p{381pt}cc}
&\textbf{Issue} & \textbf{Page}\\[6pt]
\Avtors{Meykhanadzhyan~L.\,A.} Stationary characteristics of the finite
capacity queueing system with\linebreak
\\[-12pt]
\hspace*{23pt}inverse service order and generalized
probabilistic priority&2&123--131\\[.23pt]
\Avtors{Miller~G.\,B.} see~Borisov~A.\,V.&&\\[.23pt]
\Avtors{Minin~V.\,A., Zatsman~I.\,M., Havanskov~V.\,A., and
Shubnikov~S.\,K.} Intensity of citation of scientific publications in
inventions on information and computer technologies patented\linebreak
\\[-12pt]
\hspace*{23pt}in Russia by domestic and foreign applicants&2&107--122\\[.23pt]
\Avtors{Monakhov~M.\,M.} see~Markov~A.\,S.&&\\[.23pt]
\Avtors{Naumov~V.\,A.\ and Samouylov~K.\,E.} On relationship
between queuing systems with resources\linebreak
\\[-12pt]
\hspace*{23pt}and Erlang networks&3&\hphantom{1}9--14\\[.23pt]
\Avtors{Okladnikov~I.\,G.} see~Kalinichenko~L.\,A.&&\\[.23pt]
\Avtors{Ometov~A.\,Ya., Andreev~S.\,D., Turlikov~A.\,M., and
Koucheryavy~E.\,A.} Performance analysis of\linebreak
\\[-12pt]
\hspace*{23pt}a wireless data
aggregation system with contention for contemporary sensor
networks&3&23--31\\[.23pt]
\Avtors{Palionnaia~S.\,I.} see~Kudryavtsev~A.\,A.&&\\[.23pt]
\Avtors{Podkolodnyy~N.\,L.} see~Kalinichenko~L.\,A.&&\\[.23pt]
\Avtors{Ponomareva~N.\,V.} see~Kalinichenko~L.\,A.&&\\[.23pt]
\Avtors{Popkova~N.\,A.} see~Zatsman~I.\,M.&&\\[.23pt]
\Avtors{Pozanenko~A.\,S.} see~Kalinichenko~L.\,A.&&\\[.23pt]
\Avtors{Razumchik~R.\,V.} see~Konovalov~M.\,G.&&\\[.23pt]
\Avtors{Ronzhin~A.\,F.} see~Melnikov~A.\,K.&&\\[.23pt]
\Avtors{Rumovskaya~S.\,B.} see~Kirikov~I.\,A.&&\\[.23pt]
\Avtors{Rumovskaya~S.\,B.} see~Kirikov~I.\,A.&&\\[.23pt]
\Avtors{Rumovskaya~S.\,B.} see~Kolesnikov A.\,V.&&\\[.23pt]
\Avtors{Samouylov~K.\,E.} see~Gaidamaka~Yu.\,V.&&\\[.23pt]
\Avtors{Samouylov~K.\,E.} see~Naumov~V.\,A.&&\\[.23pt]
\Avtors{Serebryanskii~S.\,M.} see~Tyrsin~A.\,N.&&\\[.23pt]
\Avtors{Seyful-Mulyukov~R.\,B.} see~Callaos~N.\,K.&&\\[.23pt]
\Avtors{Shestakov~O.\,V.} Statistical properties of the denoising method
based on the stabilized hard\linebreak
\\[-12pt]
\hspace*{23pt}thresholding&2&65--69\\[.23pt]
\Avtors{Shestakov~O.\,V.} The strong law of large numbers for the risk
estimate in the problem of\linebreak
\\[-12pt]
\hspace*{23pt}tomographic image reconstruction from
projections with a correlated noise&3&41--45\\[.23pt]
\Avtors{Shestakov~O.\,V.} see~Zakharova~T.\,V.&&\\[.23pt]
\Avtors{Shnurkov~P.\,V., Gorshenin~A.\,K., and Belousov~V.\,V.}
Analytical solution of~the~optimal control\linebreak
\\[-12pt]
\hspace*{23pt}task of~a~semi-Markov
process with~finite set of~states&4&72--88\\[.23pt]
\Avtors{Shnurkov~P.\,V., Zasypko~V.\,V., Belousov~V.\,V., and
Gorshenin~A.\,K.} Development of the algorithm of numerical solution
of the optimal investment control problem\linebreak
\\[-12pt]
\hspace*{23pt}in the closed dynamical model of three-sector economy&1&82--95\\[.23pt]
\Avtors{Shorgin~S.\,Ya.} see~Gaidamaka~Yu.\,V.&&\\[.23pt]
\Avtors{Shorgin~V.\,S.} see~Agalarov~Ya.\,M.&&\\[.23pt]
\Avtors{Shubnikov~S.\,K.} see~Minin~V.\,A.&&\\[.23pt]
\Avtors{Sidorkin~I.\,I.} see~Arkhipov~O.\,P.&&\\[.23pt]
\Avtors{Sinitsyn~I.\,N.} Analytical modeling of processes in stochastic
systems with complex fractional\linebreak
\\[-12pt]
\hspace*{23pt}order Bessel nonlinearities&3&55--65\\[.23pt]
\Avtors{Sinitsyn~I.\,N.} Orthogonal supoptimal filters for nonlinear
stochastic systems on manifolds&1&34--44\\[.23pt]
\Avtors{Sinitsyn~I.\,N.\ and Korepanov~E.\,R.} Normal Pugachev
conditionally-optimal filters and extra-\linebreak
\\[-12pt]
\hspace*{23pt}polators for state linear stochastic systems&2&14--23\\[.23pt]
\Avtors{Sinitsyn~I.\,N.\ and Sinitsyn~V.\,I.} Analytical modeling of
distributions in stochastic systems on\linebreak
\\[-12pt]
\hspace*{23pt}manifolds based on ellipsoidal approximation&1&45--55\\[.23pt]
\Avtors{Sinitsyn~I.\,N., Sinitsyn~V.\,I., and
Korepanov~E.\,R.} Ellipsoidal suboptimal filters for nonlinear\linebreak
\\[-12pt]
\hspace*{23pt}stochastic systems on manifolds&2&24--35\\[.23pt]
\Avtors{Sinitsyn~V.\,I.} see~Sinitsyn~I.\,N.&&\\[.23pt]
\Avtors{Sinitsyn~V.\,I.} see~Sinitsyn~I.\,N.&&\\[.23pt]
\Avtors{Skvortsov~N.\,A.} see~Stupnikov~S.\,A.&&\\[.23pt]
\Avtors{Sokolov~I.\,A.} see~Chertok~A.\,V.&&\\
\end{tabular}
}
\pagebreak

\def\leftfootline{\small{\textbf{\thepage}
\hfill INFORMATIKA I EE PRIMENENIYA~--- INFORMATICS AND APPLICATIONS\ \ \ 2016\
\ \ volume~10\ \ \ issue\ 4}
}%
 \def\rightfootline{\small{INFORMATIKA I EE PRIMENENIYA~---
INFORMATICS AND APPLICATIONS\ \ \ 2016\ \ \ volume~10\ \ \ issue\ 4
\hfill \textbf{\thepage}}}

\def\leftkol{2016 AUTHOR INDEX} % ENGLISH ABSTRACTS}

\def\rightkol{2016 AUTHOR INDEX} %ENGLISH ABSTRACTS}


{\tabcolsep=3pt
\begin{tabular}{p{382pt}cc}
&\textbf{Issue} & \textbf{Page}\\[6pt]
\Avtors{Sopin~E.\,S.} see~Gaidamaka~Yu.\,V.&&\\
\Avtors{Strijov~V.\,V.} see~Goncharov~A.\,V.&&\\
\Avtors{Strijov~V.\,V.} see~Isachenko~R.\,V.&&\\
\Avtors{Strijov~V.\,V.} see~Karasikov~M.\,E.&&\\
\Avtors{Stupnikov~S.\,A., Briukhov~D.\,O., and Skvortsov~N.\,A.}
Co-lending systemic risk analysis over\linebreak
\\[-12pt]
\hspace*{23pt}heterogeneous data collections&1&23--33\\
\Avtors{Stupnikov~S.\,A.} see~Kalinichenko~L.\,A.&&\\
\Avtors{Suchkov~A.\,P.} see~Zatsarinny~A.\,A.&&\\
\Avtors{Timonina~E.\,E.} see~Grusho~A.\,A.&&\\
\Avtors{Titova~A.\,I.} see~Kudryavtsev~A.\,A.&&\\
\Avtors{Turlikov~A.\,M.} see~Ometov~A.\,Ya.&&\\
\Avtors{Tyrsin~A.\,N.\ and Serebryanskii~S.\,M.} Recognition of
dependences on the basis of inverse\linebreak
\\[-12pt]
\hspace*{23pt}mapping&2&58--64\\
\Avtors{Ulyanov~V.\,V.} see~Markov~A.\,S.&&\\
\Avtors{Ushakov~V.\,G.} Queueing system with working vacations and
hyperexponential input stream&2&92--97\\
\Avtors{Ushakov~V.\,G.} see~Leontyev~N.\,D.&&\\
\Avtors{Volnova~A.\,A.} see~Kalinichenko~L.\,A.&&\\
\Avtors{Yakovlev~O.\,A.\ and Gasilov~A.\,V.} Speeded-up stereo
matching using geodesic support weights&3&\hphantom{1}98--104\\
\Avtors{Zabezhailo~M.\,I.} see~Grusho~A.\,A.&&\\
\Avtors{Zabezhailo~M.\,I.} see~Grusho~A.\,A.&&\\
\Avtors{Zakharova~T.\,V.\ and Shestakov~O.\,V.} Precision analysis of
wavelet processing of aerodynamic\linebreak
\\[-12pt]
\hspace*{23pt}flow patterns&3&46--54\\
\Avtors{Zalizniak~Anna~A.\ and Kruzhkov~M.\,G.} Database
of~Russian impersonal verbal constructions&4&132--141\\
\Avtors{Zasypko~V.\,V.} see~Shnurkov~P.\,V.&&\\
\Avtors{Zatsarinny~A.\,A.\ and Suchkov~A.\,P.} Systems engineering
approaches to~the~establishment of\linebreak
\\[-12pt]
\hspace*{23pt}a~system for~decision support based
on~situational analysis&4&105--113\\
\Avtors{Zatsarinny~A.\,A.} see~Grusho~A.\,A.&&\\
\Avtors{Zatsman~I.\,M., Inkova~O.\,Yu., Kruzhkov~M.\,G., and
Popkova~N.\,A.} Representation of cross-\linebreak
\\[-12pt]
\hspace*{23pt}lingual knowledge about
connectors in supracorpora databases&1&106--118\\
\Avtors{Zatsman~I.\,M.} see~Minin~V.\,A.&&\\
\Avtors{Zeifman~A.\,I.} see~Korolev~V.\,Yu.&&\\
\Avtors{Zeifman~A.\,I.} see~Korolev~V.\,Yu.&&\\
\end{tabular}
}

%\thispagestyle{myheadings}
\def\leftfootline{\small{\textbf{\thepage}
\hfill INFORMATIKA I EE PRIMENENIYA~--- INFORMATICS AND APPLICATIONS\ \ \ 2016\
\ \ volume~10\ \ \ issue\ 4}
}%
 \def\rightfootline{\small{INFORMATIKA I EE PRIMENENIYA~---
INFORMATICS AND APPLICATIONS\ \ \ 2016\ \ \ volume~10\ \ \ issue\ 4
\hfill \textbf{\thepage}}}

 \label{end\stat}

\newpage

%\def\stat{rekl}
%\label{preobr}

%\def\tit{АКАДЕМИК ПУГАЧЁВ  ВЛАДИМИР СЕМЁНОВИЧ\\
%25.03.1911--25.03.1998}


%   \vspace*{-48pt}
%   \begin{center}\LARGE
%Академик Пугачёв  Владимир Семёнович\\ (25.03.1911--25.03.1998)
%   \end{center}
   
   %\vspace*{2.5mm}
   
   \begin{center}

{\prgsh\LARGE
ОБЪЯВЛЕНИЯ О КОНФЕРЕНЦИЯХ}

\end{center}
%\hrule

\vspace*{6pt}

   
   \vspace*{10mm}
   
   \thispagestyle{empty}

\noindent
\begin{tabular}{cc}
%\begin{center}
\multicolumn{1}{c}{\raisebox{-40pt}[0pt][0pt]{\mbox{%
\epsfxsize=33mm
\epsfbox{vspu.eps}
}}}
%\end{center}
&
\tabcolsep=0pt\begin{tabular}{c}
{\prg{\Large\textbf{XII Всероссийское совещание}}}\\[6pt]
{\prg{\Large\textbf{по проблемам управления}}}\\[12pt]
{\prg{\large 16--19 июня 2014~г.}}\\[6pt] 
{\prg{\large Институт проблем управления имени В.\,А.~Трапезникова РАН}}\\[6pt]
{\prg{\large Москва, Россия}}
\end{tabular}
\end{tabular}

\vspace*{60pt}

     
 { %\large    
 XII Всероссийское совещание по проблемам управления (ВСПУ XII), посвященное 75-летию 
Института проблем управления (ИПУ) имени В.\,А.~Трапезникова РАН, проводится 16--19~июня 
2014~г.\ 
в ИПУ РАН (г.~Москва, Россия). ВСПУ XII организуется ИПУ РАН при поддержке РФФИ, Отделения 
энергетики, машиностроения, механики и процессов управления Российской академии наук, 
Российского 
национального комитета по автоматическому управлению, Академии навигации и управ\-ле\-ния 
движением, 
Научного совета РАН по комплексным проблемам управления и автоматизации, Совета по 
мехатронике и робототехнике РАН. Официальный язык Совещания~--- русский.

\vspace*{24pt}
     
     \textbf{Направления работы}
     \begin{enumerate}[1.]
\item Теория систем управления
\item Управление подвижными объектами и навигация
\item Интеллектуальные системы управления
\item Управление в промышленности, транспортом и логистикой
\item Управление системами междисциплинарной природы
\item Средства измерения, вычислений и контроля в управлении
\item Системный анализ и принятие решений в задачах управления
\item Информационные технологии в управлении
\item Проблемы образования в области управления: современное содержание и технологии обучения
\end{enumerate}

\vspace*{24pt}

     Подробная информация о Совещании находится на сайте {\sf http://vspu2014.ipu.ru}. Срок 
окончательной подачи докладов через систему подачи докладов на сайте~--- \textbf{30~ноября} 
2013~г.
}

%\include{rekl-1}

%\end{document}

   \vspace*{-48pt}

\begin{center}
\vspace*{6pt}
\mbox{%
\epsfxsize=53.502mm
\epsfbox{foto-1.eps}
}
\end{center}

\vspace*{6pt} %Академик


   \begin{center}
\fbox{\Large\textbf{Профессор Игорь Алексеевич Ушаков}}\\[12pt]
\textbf{\large 22.01.1935--27.02.2015}
   \end{center}


   %\vspace*{2.5mm}

   \vspace*{5mm}

   \thispagestyle{empty}

%\

%\vspace*{-12pt}


Редакционный совет и редакционная коллегия журнала <<Информатика и~её применения>> с~глубоким прискорбием извещают, что 27~февраля 2015~г.\ после тяжелой
и~продолжительной болезни скончался Игорь Алексеевич Ушаков~--- доктор технических наук, профессор, член редколлегии журнала <<Информатика и ее применения>>.

Игорь Алексеевич Ушаков окончил Московский авиационный институт, в~1963~г.\ защитил кандидатскую, а~в~1968~г.~--- докторскую диссертацию. С~1958 по 1989~гг.\ работал в~ряде научно-исследовательских организаций СССР, в~том числе руководил отделами в~НИИ АА и~ВЦ АН СССР; с 1969 по 1989 гг. преподавал в~МФТИ (был профессором, а~затем заведующим кафедрой) и~в~МЭИ. С~1989~г.~---- в~США: являлся профессором университета Дж.\ Вашингтона, университета Дж.\ Мэйсона и~Калифорнийского университета, сотрудником компаний MCI, Qualcomm и Hughes.

И.\,А.~Ушаков с момента основания журнала <<Надежность и~контроль качества>> был заместителем ответственного редактора, а~затем на протяжении многих лет членом редколлегии. В~2006~г.\ основал электронный международный журнал ``Reliability: Theory \& Application'', главным редактором которого оставался до конца жизни.

Учебниками и справочниками по теории надежности, написанными И.\,А.~Ушаковым, пользовались и~пользуются несколько поколений ученых и~специалистов в~разных странах мира.

Игорь Алексеевич всегда уделял огромное внимание работе с~молодежью; более~50 его учеников защитили докторские и~кандидатские диссертации.

И.\,А.~Ушаков вел активную научно-про\-све\-ти\-тель\-скую деятельность. В~частности, он был одним из организаторов и~руководителей Московского кабинета качества и~надежности при Политехническом музее (целью этого Кабинета было оказание консультаций работникам промышленных предприятий и~чтение курсов лекций для инженеров, занимающихся проблемой надежности). Находясь в~США, И.\,А.~Ушаков создал международный ин\-тер\-нет-фо\-рум им.\ Б.\,В.~Гнеденко, объединивший около~400~видных специалистов по приложениям теории вероятностей и~математической статистики, преимущественно в~об\-ласти теории надежности и~анализа риска, из десятков стран мира; коллективным членов этого Форума является и~наш журнал. Цели Форума~--- содействие контактам между специалистами из разных стран, организация обмена профессиональными 
новостями и~информацией (новые публикации, предстоящие события и~др.). Также необходимо отметить большое число на\-уч\-но-по\-пу\-ляр\-ных работ, опубликованных И.\,А.~Ушаковым.

И.\,А.~Ушаков обладал большим личным обаянием, имел широкий круг интересов. Все знавшие И.\,А.~Ушакова всегда будут помнить его как замечательного ученого и~прекрасного человека.

\bigskip

Редакционный совет и редакционная коллегия журнала <<Информатика и~её применения>> 
выражают глубокие соболезнования родным и близким покойного, всем, кто его знал и~работал с~ним.



%\end{document}

%\include{IPPM-25}

\def\stat{cont-rus}
{%\hrule\par
%\vskip 7pt % 7pt
\vspace*{-24pt}
\raggedleft\Large \bf%\baselineskip=3.2ex
Правила подготовки рукописей  для публикации в журнале
<<Информатика~и~её~применения>> \vskip 8pt
    \hrule
    \par
\vskip 14pt plus 6pt minus 3pt }

\label{st\stat}

\def\tit{\ }

\def\aut{\ }
\def\auf{\ }

\def\leftkol{\ }
% Правила подготовки рукописей  для публикации в журнале
%<<Информатика и её применения>>

\def\rightkol{\ }
%Правила подготовки рукописей  для публикации в журнале
%<<Информатика и её применения>>}


\titele{\tit}{\aut}{\auf}{\leftkol}{\rightkol}


\vspace*{-60pt}
{ %\small

Журнал <<Информатика и её применения>>
публикует теоретические, обзорные и дискуссионные статьи,
посвященные научным исследованиям и разработкам в области
информатики и ее приложений.

Журнал издается на русском языке. По специальному решению
редколлегии отдельные статьи могут печататься на английском языке.

Тематика журнала охватывает следующие направления:
\begin{itemize}
\item теоретические основы информатики;\\[-15pt]
      \item
математические методы исследования сложных систем и процессов;\\[-15pt]
           \item
информационные системы и сети;\\[-15pt]
                \item
информационные технологии;\\[-15pt]
                     \item
архитектура и программное обеспечение вычислительных комплексов и сетей.\\[-15pt]
\end{itemize}


\noindent
\begin{enumerate}[1.]
\item В журнале печатаются статьи, содержащие результаты, ранее не опубликованные и
не предназначенные к одновременной публикации в других изданиях.

%Публикация не должна нарушать закон об авторских правах.
Публикация предоставленной автором(ами) рукописи не должна нарушать 
положений глав~69, 70 раздела~VII части~IV Гражданского кодекса, 
которые определяют права на результаты интеллектуальной деятельности 
и~средства индивидуализации, в~том числе авторские права, в~РФ.

Ответственность за нарушение авторских прав, в~случае предъявления претензий к~редакции журнала,  
несут авторы статей.



Направляя рукопись в редакцию, авторы сохраняют свои права на данную
рукопись и при этом передают учредителям и редколлегии журнала неисключительные права на
издание статьи на русском языке 
(или на языке статьи, если он отличен от рус\-ско\-го) и~на перевод ее на английский
язык, а~также на
ее распространение в России и за рубежом. 
Каждый автор должен представить в~редакцию подписанный 
с~его стороны <<Лицензионный договор о~передаче неисключительных прав 
на использование произведения>>, текст которого размещен по адресу 
{\sf http://www.ipiran.ru/publications/licence.doc}. 
Этот договор может быть пред\-став\-лен в~бумажном (в~2-х экз.)\ 
или в~электронном виде (отсканированная копия заполненного и~подписанного документа).




Редколлегия вправе запросить у авторов экспертное заключение о возможности
пуб\-ли\-ка\-ции пред\-став\-лен\-ной статьи в открытой печати.\\[-13.5pt]

\item К статье прилагаются данные автора (авторов) (см.\ п.~8). При наличии нескольких
авторов указывается фамилия автора, ответственного за переписку с редакцией.\\[-13.5pt]

\item Редакция журнала осуществляет экспертизу присланных статей в соответствии с
принятой в журнале процедурой рецензирования.

Возвращение рукописи на доработку не означает ее принятия к печати.

Доработанный вариант с ответом на замечания рецензента необходимо прислать в
редакцию.\\[-13.5pt]

\item Решение редколлегии о публикации статьи или ее отклонении сообщается авторам.

Редколлегия может также направить авторам текст рецензии на их статью. Дискуссия по
поводу отклоненных статей не ведется.\\[-13.5pt]

%\pagebreak

\item Редактура статей высылается авторам для просмотра. Замечания к редактуре должны
быть присланы авторами в кратчайшие сроки.\\[-13.5pt]

\item Рукопись предоставляется в электронном виде в форматах MS WORD (.doc или
.docx) или \LaTeX\  (.tex), дополнительно~--- в формате .pdf, на дискете, лазерном диске
или электронной почтой. Предоставление бумажной рукописи необязательно.\\[-13.5pt]

\item При подготовке рукописи в MS Word рекомендуется использовать следующие
настройки.

Параметры страницы:
формат~--- А4; ориентация~--- книжная; поля (см): внутри~--- 2,5, снаружи~--- 1,5,
сверху~--- 2, снизу~--- 2, от края до нижнего колонтитула~--- 1,3.

Основной текст: стиль~--- <<Обычный>>, шрифт~--- Times New Roman, размер~---
14~пунк\-тов, абзацный отступ~--- 0,5~см, 1,5~интервала, выравнивание~--- по ширине.

\pagebreak

\def\leftkol{Правила подготовки рукописей  для публикации в журнале
<<Информатика и её применения>>}

\def\rightkol{Правила подготовки рукописей  для публикации в журнале
<<Информатика и её применения>>}



Рекомендуемый объем рукописи~--- не свыше 10~страниц указанного формата.
При превышении указанного объема редколлегия вправе потребовать от 
автора сокращения объема рукописи.


Сокращения слов, помимо стандартных, не допускаются. Допускается минимальное
количество аббревиатур.


Все страницы рукописи нумеруются.

Шаблоны оформления представлены в интернете:

\noindent
 {\sf
http://www.ipiran.ru/journal/template\_iiep\_ssi\_2024.zip}\\[-14pt]

\item Статья должна содержать следующую информацию на {\bfseries\textit{русском и
английском языках}}:\\[-16pt]

\begin{itemize}
\item название статьи;\\[-15pt]
\item Ф.И.О.\ авторов, на английском можно только имя и фамилию;\\[-15pt]
\item место работы, с указанием почтового адреса организации и электронного адреса каждого
автора;\\[-15pt]
\item сведения об авторах, в соответствии с форматом, образцы которого
представлены на страницах:



\def\leftfootline{\small{\textbf{\thepage}
\hfill ИНФОРМАТИКА И ЕЁ ПРИМЕНЕНИЯ\ \ \ том\ 18\ \ \ выпуск\ 3\ \ \ 2024}
}%
 \def\rightfootline{\small{ИНФОРМАТИКА И ЕЁ ПРИМЕНЕНИЯ\ \ \ том\ 18\ \ \ выпуск\ 3\ \ \ 2024
\hfill \textbf{\thepage}}}



{\sf http://www.ipiran.ru/journal/issues/2013\_07\_01/authors.asp} и

{\sf http://www.ipiran.ru/journal/issues/2013\_07\_01\_eng/authors.asp};
\item аннотация (не менее 100~слов на каждом из языков). Аннотация~--- это краткое
резюме работы, которое может публиковаться отдельно. Она является основным
источником информации в~ин\-фор\-ма\-ци\-он\-ных системах и базах данных. Английская
аннотация должна быть оригинальной, может не быть дословным переводом русского
текста и должна быть написана хорошим английским языком. В~аннотации не должно
быть ссылок на литературу и, по возможности, формул;\\[-15pt]
\item ключевые слова~--- желательно из принятых в мировой
на\-уч\-но-тех\-ни\-че\-ской литературе тематических тезаурусов. Предложения не
могут быть ключевыми словами;\\[-15pt]
\item источники финансирования работы (ссылки на гранты, проекты,
поддерживающие организации и~т.\,п.).
\end{itemize}



%\pagebreak

\item  Требования к спискам литературы.\\[-14pt]

Ссылки на литературу в тексте статьи нумеруются (в квадратных скобках) и
располагаются в каждом из списков литературы в порядке  первых упоминаний. Если источник имеет DOI и/или EDN,
то их необходимо указывать.

Списки литературы представляются в двух вариантах:\\[-14pt]


\noindent
\begin{enumerate}[(1)]
\item \textbf{Список литературы к русскоязычной части}. Русские и английские
работы~---  на языке и в алфавите оригинала;\\[-14.5pt]
\item  \textbf{References}. Русские работы и работы на других языках~--- в латинской
транслитерации с переводом на английский язык; английские работы и работы на других
языках~--- на языке оригинала.
\end{enumerate}

Необходимо для составления списка ``References'' пользоваться размещенной на сайте
{\sf http://www. translit.net/ru/bgn/} бесплатной программой транслитерации русского
 текста в~латиницу. %, при этом в~за\-клад\-ке <<варианты\ldots>> следует выбратьопцию BGN.

Список литературы ``References'' приводится полностью отдельным блоком, повторяя все
позиции из списка литературы к русскоязычной части, независимо от того, имеются или
нет в нем иностранные источники. Если в списке литературы к русскоязычной части есть
ссылки на иностранные публикации, набранные латиницей, они полностью повторяются в
списке ``References''.

Ниже приведены примеры ссылок на различные виды публикаций в списке ``References''.

\def\leftfootline{\small{\textbf{\thepage}
\hfill ИНФОРМАТИКА И ЕЁ ПРИМЕНЕНИЯ\ \ \ том\ 18\ \ \ выпуск\ 3\ \ \ 2024}
}%
 \def\rightfootline{\small{ИНФОРМАТИКА И ЕЁ ПРИМЕНЕНИЯ\ \ \ том\ 18\ \ \ выпуск\ 3\ \ \ 2024
\hfill \textbf{\thepage}}}

{\small

\noindent
\textbf{Описание статьи из журнала:}

\Aue{Zagurenko, A.\,G., V.\,A.~Korotovskikh, A.\,A.~Kolesnikov, A.\,V.~Timonov, and D.\,V.~Kardymon}. 2008.
Tekhniko-ekonomicheskaya optimizatsiya dizayna gidrorazryva plasta [Technical and
economic optimization of the design
of hydraulic fracturing]. \textit{Neftyanoe hozyaystvo} [\textit{Oil Industry}] 11:54--57.

\Aue{Zhang, Z., and D.~Zhu}. 2008. Experimental research on the localized
electrochemical micromachining. \textit{Russ. J.~Electrochem.}  44(8):926--930.
{\sf doi:10.1134/S1023193508080077}.

\noindent
\textbf{Описание статьи из электронного журнала:}

\Aue{Swaminathan, V., E.~Lepkoswka-White, and B.\,P.~Rao}. 1999. Browsers or buyers in cyberspace? An
investigation of electronic factors influencing electronic exchange. \textit{JCMC}
5(2). Available at: {\sf http://www.ascusc.org/jcmc/vol5/issue2/} (accessed April~28, 2011).

\def\leftkol{Правила подготовки рукописей  для публикации в журнале
<<Информатика и её применения>>}

\def\rightkol{Правила подготовки рукописей  для публикации в журнале
<<Информатика и её применения>>}


\noindent
\textbf{Описание статьи из продолжающегося издания (сборника трудов):}

\Aue{Astakhov, M.\,V., and T.\,V.~Tagantsev}. 2006. Eksperimental'noe
issledovanie prochnosti soedineniy ``stal'--kompozit'' [Experimental study of
the strength of joints ``steel--composite'']. \textit{Trudy MGTU
``Matematicheskoe modelirovanie slozhnykh tekh\-ni\-che\-skikh sistem''}
[\textit{Bauman MSTU ``Mathematical Modeling of Complex Technical
Systems'' Proceedings}]. 593:125--130.


\pagebreak



\noindent
\textbf{Описание материалов конференций:}

\Aue{Usmanov, T.\,S., A.\,A.~Gusmanov, I.\,Z.~Mullagalin, R.\,Ju.~Muhametshina, A.\,N.~Chervyakova, and
A.\,V.~Sveshnikov}. 2007. Osobennosti proektirovaniya razrabotki mestorozhdeniy
s primeneniem gidrorazryva
plasta [Features of the design of field development with the use of hydraulic fracturing].
\textit{Trudy 6-go
Mezhdu\-na\-rod\-no\-go Simpoziuma ``Novye resursosberegayushchie tekhnologii nedropol'zovaniya i povysheniya
neftegazootdachi''} [\textit{6th  Symposium (International) ``New Energy Saving Subsoil Technologies and
the Increasing of the Oil and Gas Impact'' Proceedings}]. Moscow. 267--272.



\def\leftfootline{\small{\textbf{\thepage}
\hfill ИНФОРМАТИКА И ЕЁ ПРИМЕНЕНИЯ\ \ \ том\ 18\ \ \ выпуск\ 3\ \ \ 2024}
}%
 \def\rightfootline{\small{ИНФОРМАТИКА И ЕЁ ПРИМЕНЕНИЯ\ \ \ том\ 18\ \ \ выпуск\ 3\ \ \ 2024
\hfill \textbf{\thepage}}}



\noindent
\textbf{Описание книги (монографии, сборники):}



Lindorf, L.\,S., and L.\,G.~Mamikoniants, eds. 1972.
\textit{Ekspluatatsiya turbogeneratorov s neposredstvennym
okhlazhdeniem} [\textit{Operation of turbine generators with direct cooling}].
Moscow: Energy Publs. 352~p.


\Aue{Latyshev, V.\,N.} 2009. \textit{Tribologiya rezaniya. Kn.~1: Friktsionnye protsessy
pri rezanii metallov}
[\textit{Tribology of cutting. Vol.~1: Frictional processes in metal cutting}]. Ivanovo: Ivanovskii
State Univ. 108~p.

\def\leftkol{Правила подготовки рукописей  для публикации в журнале
<<Информатика и её применения>>}

\def\rightkol{Правила подготовки рукописей  для публикации в журнале
<<Информатика и её применения>>}

\noindent
\textbf{Описание переводной книги}
(в списке литературы к русскоязычной части необходимо указать:~/ Пер.\ с англ.~---
после названия книги, а в конце ссылки указать оригинал книги в круглых скобках):
\begin{enumerate}[1.]
\item  В русскоязычной части:

\def\leftfootline{\small{\textbf{\thepage}
\hfill ИНФОРМАТИКА И ЕЁ ПРИМЕНЕНИЯ\ \ \ том\ 18\ \ \ выпуск\ 3\ \ \ 2024}
}%
 \def\rightfootline{\small{ИНФОРМАТИКА И ЕЁ ПРИМЕНЕНИЯ\ \ \ том\ 18\ \ \ выпуск\ 3\ \ \ 2024
\hfill \textbf{\thepage}}}

\Au{Тимошенко С.\,П., Янг Д.\,Х., Уивер~У.}
Колебания в инженерном деле~/ Пер.\ с англ.~--- М.: Машиностроение, 1985. 472~с.
(\Au{Timoshenko~S.\,P., Young~D.\,H., Weaver~W.}
Vibration problems in engineering.~--- 4th ed.~--- New York, NY, USA: Wiley, 1974. 521~p.)\\[-13.5pt]
\item  В англоязычной части:

\Aue{Timoshenko, S.\,P., D.\,H.~Young, and W.~Weaver}.
1974. \textit{Vibration problems in engineering}. 4th ed. New York: 
Wiley. 521~p.
\end{enumerate}

\vspace*{-3pt}


\noindent
\textbf{Описание неопубликованного документа:}


\Aue{Latypov, A.\,R., M.\,M.~Khasanov, and V.\,A.~Baikov}.
2004 (unpubl.). Geologiya i~dobycha (NGT GiD) [Geology and production (NGT GiD)]. Certificate on official registration of the computer program
No.\,2004611198. 

\noindent
\textbf{Описание интернет-ресурса:}


Pravila tsitirovaniya istochnikov [Rules for the citing of sources]. Available at: {\sf
http://www.scribd.com/doc/1034528/} (accessed February~7, 2011).

%\pagebreak

\noindent
\textbf{Описание диссертации или автореферата диссертации:}

\Aue{Semenov, V.\,I.}
2003. Matematicheskoe modelirovanie plazmy v sisteme kompaktnyy tor [Mathematical
modeling of the plasma in the compact torus].  Moscow.  D.Sc.\ Diss. 272~p.

\Aue{Kozhunova, O.\,S.} 2009. Tekhnologiya razrabotki semanticheskogo
slovarya informatsionnogo monitoringa [Technology of development of
semantic dictionary of information monitoring system].  Moscow: IPI RAN. PhD Thesis. 23~p.


\noindent
\textbf{Описание ГОСТа:}

GOST 8.586.5-2005. 2007. Metodika vypolneniya izmereniy. Izmerenie raskhoda i~kolichestva zhidkostey i~gazov
s~pomoshch'yu standartnykh suzhayushchikh ustroystv [Method of measurement.
Measurement of flow rate and volume of liquids and gases by means of orifice devices]. Moscow:
Standardinform  Publs. 10~p.

\noindent
\textbf{Описание патента:}

\Aue{Bolshakov, M.\,V., A.\,V.~Kulakov, A.\,N.~Lavrenov, and M.\,V.~Palkin}.
2006. Sposob orientirovaniya po krenu letatel'nogo
apparata s opti\-che\-skoy golovkoy
samonavedeniya [The way to orient on the roll of aircraft with optical homing head].
Patent RF No.\,2280590.
}

\item Присланные в редакцию материалы авторам не возвращаются.\\[-13.5pt]

\item При отправке файлов по электронной почте просим придерживаться следующих
правил:
\begin{itemize}
\item указывать в поле subject (тема) название журнала и фамилию автора;\\[-13.5pt]
\item указывать в тексте письма название статьи, авторов и~журнал, в~который направляется статья;\\[-13.5pt]
\item использовать attach (присоединение);\\[-13.5pt]
\item в состав электронной версии статьи должны входить: файл, содержащий текст
статьи, и файл(ы), содержащий(е) иллюстрации.\\[-13.5pt]
\end{itemize}

\item Журнал <<Информатика и её применения>> является некоммерческим изданием.
Плата за публикацию не взимается, гонорар авторам не выплачивается.
\end{enumerate}



\def\leftfootline{\small{\textbf{\thepage}
\hfill ИНФОРМАТИКА И ЕЁ ПРИМЕНЕНИЯ\ \ \ том\ 18\ \ \ выпуск\ 3\ \ \ 2024}
}%
 \def\rightfootline{\small{ИНФОРМАТИКА И ЕЁ ПРИМЕНЕНИЯ\ \ \ том\ 18\ \ \ выпуск\ 3\ \ \ 2024
\hfill \textbf{\thepage}}}


\vspace*{-1mm}

\begin{center}

\textbf{Адрес редакции журнала <<Информатика и её применения>>:} \\




Москва 119333, ул.~Вавилова, д.~44, корп.~2, ФИЦ ИУ РАН\\[-10pt]

\

Тел.: +7\,(499)\,135-86-92\ \ Факс:  +7\,(495)\,930-45-05\\[-10pt]

 \

e-mail:   {\sf iiep@frccsc.ru} (Стригина Светлана Николаевна)\\[-10pt]

\

{\sf http://www.ipiran.ru/journal/issues/}
\end{center}
}


\def\leftkol{Правила подготовки рукописей  для публикации в журнале
<<Информатика и её применения>>}

\def\rightkol{Правила подготовки рукописей  для публикации в журнале
<<Информатика и её применения>>}


\def\leftfootline{\small{\textbf{\thepage}
\hfill ИНФОРМАТИКА И ЕЁ ПРИМЕНЕНИЯ\ \ \ том\ 18\ \ \ выпуск\ 3\ \ \ 2024}
}%
 \def\rightfootline{\small{ИНФОРМАТИКА И ЕЁ ПРИМЕНЕНИЯ\ \ \ том\ 18\ \ \ выпуск\ 3\ \ \ 2024
\hfill \textbf{\thepage}}} 
\def\stat{podg-e}
{%\hrule\par
%\vskip 7pt % 7pt
\vspace*{-24pt}
\raggedleft\Large \bf%\baselineskip=3.2ex
Requirements for manuscripts submitted to Journal
``Informatics~and~Applications'' \vskip 8pt
    \hrule
    \par
\vskip 21pt plus 6pt minus 3pt }

\label{st\stat}

\def\tit{\ }

\def\aut{\ }
\def\auf{\ }

\def\leftkol{\ }

\def\rightkol{\ }
%Requirements for manuscripts submitted to Journal
%``Informatics~and~Applications''}

\titele{\tit}{\aut}{\auf}{\leftkol}{\rightkol}

\def\leftfootline{\small{\textbf{\thepage}
\hfill INFORMATIKA I EE PRIMENENIYA~--- INFORMATICS AND APPLICATIONS\ \ \ 2019\
\ \ volume~13\ \ \ issue\ 4}
}%
 \def\rightfootline{\small{INFORMATIKA I EE PRIMENENIYA~--- INFORMATICS AND APPLICATIONS\ \ \ 2019\ \ \ volume~13\ \ \ issue\ 4
\hfill \textbf{\thepage}}}

\vspace*{-60pt}

{\small

\noindent
Journal ``Informatics and Applications'' (Inform.\ Appl.)
publishes theoretical, review, and discussion
articles on the research and development in the
field of informatics and its applications.

The journal is published in Russian.
By a special decision of the editorial
board, some articles can be published in English.


The topics covered include the following areas:
\begin{itemize}
               \item
     theoretical fundamentals of informatics; \\[-14pt]
\item
mathematical methods for studying complex systems and processes; \\[-14pt]
\item
information systems and networks;\\[-14pt]
\item
information technologies; and \\[-14pt]
\item
architecture and software of computational complexes and networks. \\[-14pt]
\end{itemize}

\noindent
\begin{enumerate}[1.]
\item The Journal publishes original articles which have not been published before and are not
intended for simultaneous publication in other editions. An article submitted to the Journal must not violate the
Copyright law. Sending the manuscript to the Editorial Board, the authors retain all rights of the
owners of the manuscript and transfer the nonexclusive rights to publish the article in Russian
(or the language of the article, if not Russian) and its distribution in Russia and abroad to the
Founders and the Editorial Board. Authors should submit a letter to the Editorial Board in the
following form:

{\bfseries\textit{Agreement on the transfer of rights to publish:}}

``\textit{We, the undersigned authors of the manuscript ``\ldots'', pass to the
Founder and the Editorial Board of the Journal ``Informatics and Applications''
the nonexclusive right to publish the manuscript of the article in Russian (or
in English) in both print and electronic versions of the Journal. We affirm
that this publication does not violate the Copyright of other persons or
organizations.}

\textit{Author(s) signature(s): (name(s), address(es), date).}

This agreement should be submitted in paper form or in the form of a scanned copy (signed by
the authors).


%The Editorial Board has the right to request from the authors an official expert conclusion that
%the submitted article has no secret data prohibited for publication. \\[-13.5pt]
\item
A submitted article should be attached with \textbf{the data on the author(s)} (see item~8). If
there are several authors, the contact person should be indicated who is responsible for
correspondence with the Editorial Board and other authors about revisions and final approval
of the proofs.\\[-13.5pt]

\item The Editorial Board of the Journal examines the article according to the established
reviewing procedure. If the authors receive their article for correction after reviewing, it does not
mean that the article is approved for publication. The corrected article should be sent to the
Editorial Board for the subsequent review and approval.\\[-13.5pt]

\item The decision on the article publication or its rejection is communicated to the authors. The
Editorial Board may also send the reviews on the submitted articles to the authors. Any
discussion upon the rejected articles is not possible.\\[-13.5pt]

\item The edited articles will be sent to the authors for proofread. The comments of the authors
to the edited text of the article should be sent to the Editorial Board as soon as possible.\\[-13.5pt]

\item The manuscript of the article should be presented electronically in the MS WORD (.doc or
.docx) or \LaTeX\ (.tex) formats, and additionally in the .pdf format. All documents
 may be sent
by e-mail or provided on a CD or diskette. A~hard copy submission is not necessary.\\[-13.5pt]

\item The recommended typesetting instructions for manuscript.

Pages parameters: format A4, portrait orientation, document margins (cm): left~--- 2.5, right~---
1.5, above~--- 2.0, below~--- 2.0, footer 1.3.

Text: font~---Times New Roman, font size~--- 14, paragraph indent~--- 0.5, line spacing~--- 1.5,
justified alignment.

The recommended manuscript size: not more than 15~pages of the specified format.
If the specified size exceeded, the editorial board is entitled to require the author
to reduce the manuscript.

Use only standard abbreviations. Avoid  abbreviations in the title and
abstract. The full term for which an abbreviation stands should precede
its first use in the text unless it is a standard unit of measurement.

All pages of the manuscript should be numbered.

The templates for the manuscript typesetting are presented on site: {\sf
http://www.ipiran.ru/journal/template.doc}.\\[-13.5pt]


%\def\leftkol{Requirements for manuscripts submitted to Journal
%``Informatics~and~Applications''}

\item The articles should enclose data both in \textbf{Russian and English}:
\begin{itemize}
\item title;\\[-13.5pt]
\item author's name and surname;\\[-13.5pt]
\item affiliation~--- organization, its address with ZIP code, city, country, and
official e-mail address;\\[-13.5pt]
\item data on authors according to the format: (see site)

{\sf http://www.ipiran.ru/journal/issues/2013\_07\_01/authors.asp}  and

{\sf  http://www.ipiran.ru/journal/issues/2013\_07\_01\_eng/authors.asp};\\[-13.5pt]

\pagebreak

\def\leftfootline{\small{\textbf{\thepage}
\hfill INFORMATIKA I EE PRIMENENIYA~--- INFORMATICS AND APPLICATIONS\ \ \ 2019\
\ \ volume~13\ \ \ issue\ 4}
}%
 \def\rightfootline{\small{INFORMATIKA I EE PRIMENENIYA~--- INFORMATICS AND APPLICATIONS\ \ \ 2019\ \ \ volume~13\ \ \ issue\ 4
\hfill \textbf{\thepage}}}


%\def\leftkol{Requirements for manuscripts submitted to Journal
%``Informatics~and~Applications''}

%\def\rightkol{Requirements for manuscripts submitted to Journal
%``Informatics~and~Applications''}



\item abstract (not less than 100 words) both in Russian and in English. Abstract is a short
summary of the article that can be published separately. The abstract is the
main source of information on the article and it could be included in leading information
systems and data bases. The abstract in English has to be an original text and should
not be an exact translation of the Russian one. Good English is required.
In abstracts, avoid references and formulae;\\[-13.5pt]
\item indexing is performed on the basis of keywords. The use of keywords from the
internationally accepted thematic Thesauri is recommended.

%\def\leftkol{Requirements for manuscripts submitted to Journal
%``Informatics~and~Applications''}

%\def\rightkol{Requirements for manuscripts submitted to Journal
%``Informatics~and~Applications''}

Important! Keywords must not be sentences;
\item Acknowledgments.
\end{itemize}

\item References. Russian references have to be presented both in English translation and Latin
transliteration (refer {\sf http://www.translit.net/ru/bgn/}).

Please take into account the following examples of Russian references appearance:

\noindent
\textbf{Article in journal:}

\Aue{Zhang, Z., and D.~Zhu}. 2008. Experimental research on the localized electrochemical
micromachining.
\textit{Rus. J.~Electrochem.}  44(8):926--930. {\sf doi:10.1134/S1023193508080077}.


\noindent
\textbf{Journal article in electronic format:}

\Aue{Swaminathan, V., E.~Lepkoswka-White, and B.\,P.~Rao}. 1999. Browsers or buyers in
cyberspace? An
investigation of electronic factors influencing electronic exchange. \textit{JCMC}
5(2). Available at: {\sf http://www.ascusc.org/jcmc/vol5/issue2/} (accessed April~28, 2011).




\noindent
\textbf{Article from the continuing publication (collection of works, proceedings):}

\Aue{Astakhov, M.\,V., and T.\,V.~Tagantsev}. 2006. Eksperimental'noe
issledovanie prochnosti soedineniy ``stal'--kompozit'' [Experimental study of
the strength of joints ``steel--composite'']. \textit{Trudy MGTU
``Matematicheskoe modelirovanie slozhnykh tekh\-ni\-che\-skikh sistem''}
[\textit{Bauman MSTU ``Mathematical Modeling of Complex Technical
Systems'' Proceedings}]. 593:125--130.

\def\leftfootline{\small{\textbf{\thepage}
\hfill INFORMATIKA I EE PRIMENENIYA~--- INFORMATICS AND APPLICATIONS\ \ \ 2019\
\ \ volume~13\ \ \ issue\ 4}
}%
 \def\rightfootline{\small{INFORMATIKA I EE PRIMENENIYA~--- INFORMATICS AND APPLICATIONS\ \ \ 2019\ \ \ volume~13\ \ \ issue\ 4
\hfill \textbf{\thepage}}}

\def\leftkol{Requirements for manuscripts submitted to Journal
``Informatics~and~Applications''}

\def\rightkol{Requirements for manuscripts submitted to Journal
``Informatics~and~Applications''}

\noindent
\textbf{Conference proceedings:}

\Aue{Usmanov, T.\,S., A.\,A.~Gusmanov, I.\,Z.~Mullagalin, R.\,Ju.~Muhametshina,
A.\,N.~Chervyakova, and
A.\,V.~Sveshnikov}. 2007. Osobennosti proektirovaniya razrabotki mestorozhdeniy
s primeneniem gidrorazryva
plasta [Features of the design of field development with the use of hydraulic fracturing].
\textit{Trudy 6-go
Mezhdu\-na\-rod\-no\-go Simpoziuma ``Novye resursosberegayushchie tekhnologii
nedropol'zovaniya i povysheniya
neftegazootdachi''} [\textit{6th  Symposium (International) ``New Energy Saving Subsoil
Technologies and
the Increasing of the Oil and Gas Impact'' Proceedings}]. Moscow. 267--272.


\noindent
\textbf{Books and other monographs:}




Lindorf, L.\,S., and L.\,G.~Mamikoniants, eds. 1972.
\textit{Ekspluatatsiya turbogeneratorov s neposredstvennym
okhlazhdeniem} [\textit{Operation of turbine generators with direct cooling}].
Moscow: Energy Publs. 352~p.


%\Aue{Latyshev, V.\,N.} 2009. \textit{Tribologiya rezaniya. Kn.~1: Frikcionnye prosessy
%pri rezanii metallov}
%[\textit{Tribology of cutting. Vol.~1: Frictional processes in metal cutting}]. Ivanovo: Ivanovskii
%State Univ. 108~p.


%\noindent
%\textbf{Unpublished material:}

%\Aue{Latypov, A.\,R., M.\,M.~Khasanov, and V.\,A.~Baikov}.
%2004. Geology and production (NGT GiD). Certificate on official registration of the computer
%program
%No.\,2004611198. (In Russian, unpubl.)

%\noindent
%\textbf{Internet-source:}

%APA Style. 2011. Available at: {\sf http://www.apastyle.org/apa-style-help.aspx} (accessed
%February~5, 2011).

%Pravila citirovaniya istochnikov [Rules for the citing of sources]. Available at: {\sf
%http://www.scribd.com/doc/1034528/} (accessed February~7, 2011).


\noindent
\textbf{Dissertation and Thesis:}

%\Aue{Semenov, V.\,I.}
%2003. Matematicheskoe modelirovanie plazmy v sisteme kompaktnyy tor. [Mathematical
%modeling of the plasma in the compact torus]. D.Sc.\ Diss. Moscow. 272~p.

\Aue{Kozhunova, O.\,S.} 2009. Tekhnologiya razrabotki semanticheskogo
slovarya informatsionnogo monitoringa [Technology of development of
semantic dictionary of information monitoring system]. PhD Thesis. Moscow: IPI RAN. 23~p.


\noindent
\textbf{State standards and patents:}

GOST 8.586.5-2005. 2007. Metodika vypolneniya izmereniy. Izmerenie raskhoda i~kolichestva
zhidkostey i gazov 
s~pomoshch'yu standartnykh suzhayushchikh ustroystv [Method of measurement.
Measurement of flow rate and volume of liquids and gases by means of orifice devices]. M.:
Standardinform
Publs. 10~p.

%\noindent
%\textbf{Patent:}

\Aue{Bolshakov, M.\,V., A.\,V.~Kulakov, A.\,N.~Lavrenov, and M.\,V.~Palkin}.
2006. Sposob orientirovaniya po krenu letatel'nogo
apparata s opti\-che\-skoy golovkoy
samonavedeniya [The way to orient on the roll of aircraft with optical homing head].
Patent RF No.\,2280590.

References in Latin transcription are presented in the original language.

References in the text are numbered according to the order of their
first appearance; the number is
placed in square brackets. All items from the reference list should be
cited.\\[-13.5pt]

\item Manuscripts and additional materials are not returned to Authors by the Editorial Board.\\[-13.5pt]

\item Submissions of files by e-mail must include:\\[-13.5pt]
\begin{itemize}
\item   the journal title and author's name in the ``Subject'' field; \\[-13.5pt]
\item   an article and additional materials have to be attached using the ``attach'' function;\\[-13.5pt]
\item   an electronic version of the article should contain the file with the text and a separate file
with figures.\\[-13.5pt]
\end{itemize}

\item ``Informatics and Applications'' journal is not a profit publication. There are no
charges for the authors as well as there are no royalties.\\[-13.5pt]
\end{enumerate}

\def\leftfootline{\small{\textbf{\thepage}
\hfill INFORMATIKA I EE PRIMENENIYA~--- INFORMATICS AND APPLICATIONS\ \ \ 2019\
\ \ volume~13\ \ \ issue\ 4}
}%
 \def\rightfootline{\small{INFORMATIKA I EE PRIMENENIYA~--- INFORMATICS AND APPLICATIONS\ \ \ 2019\ \ \ volume~13\ \ \ issue\ 4
\hfill \textbf{\thepage}}}

\def\leftkol{Requirements for manuscripts submitted to Journal
``Informatics~and~Applications''}

\def\rightkol{Requirements for manuscripts submitted to Journal
``Informatics~and~Applications''}


%\vspace*{5mm}


\begin{center}
\textbf{Editorial Board address:} \\

%ABOUT AUTHORS



FRC CSC RAS, 44, block~2, Vavilov Str., Moscow 119333, Russia\\[-10pt]

\

Ph.: +7\,(499)\,135\,86\,92,\ \ Fax: +7\,(495)\,930\,45\,05\\[-10pt]

\

 e-mail: {\sf rust@ipiran.ru} (to Prof.\ Rustem Seyful-Mulyukov)\\[-10pt]

\

 {\sf http://www.ipiran.ru/english/journal.asp}
\end{center}
 }
%\thispagestyle{myheadings}

\def\leftkol{Requirements for manuscripts submitted to Journal
``Informatics~and~Applications''}

\def\rightkol{Requirements for manuscripts submitted to Journal
``Informatics~and~Applications''}

\def\leftfootline{\small{\textbf{\thepage}
\hfill INFORMATIKA I EE PRIMENENIYA~--- INFORMATICS AND APPLICATIONS\ \ \ 2019\
\ \ volume~13\ \ \ issue\ 4}
}%
 \def\rightfootline{\small{INFORMATIKA I EE PRIMENENIYA~--- INFORMATICS AND APPLICATIONS\ \ \ 2019\ \ \ volume~13\ \ \ issue\ 4
\hfill \textbf{\thepage}}}

 \label{end\stat}

\newpage


%\vspace*{-60pt} {\small
{\baselineskip=9.1pt
\section*{Правила подготовки рукописей статей для публикации в журнале
<<Информатика и её применения>>}

\thispagestyle{empty}

 Журнал <<Информатика и её применения>> публикует
теоретические, обзорные и дискуссионные статьи, посвященные научным
исследованиям и разработкам в области информатики и ее приложений. Журнал
издается на русском языке. По специальному решению редколлегии отдельные статьи,
в виде исключения, могут печататься на английском языке.
Тематика журнала охватывает следующие направления:
\begin{itemize}
\item теоретические основы информатики; %\\[-13.5pt]
\item математические методы исследования сложных систем и процессов; %\\[-13.5pt]
\item информационные системы и сети; %\\[-13.5pt]
\item информационные технологии; %\\[-13.5pt]
\item архитектура и программное
обеспечение вычислительных комплексов и сетей.
\end{itemize}
\begin{enumerate}
\item В журнале печатаются результаты, ранее не
опубликованные и не предназначенные к одновременной публикации в других
изданиях. Публикация не должна нарушать закон об авторских правах. Направляя
свою рукопись в редакцию, авторы автоматически передают учредителям и
редколлегии неисключительные права на издание данной статьи на русском языке и
на ее распространение в России и за рубежом. При этом за авторами сохраняются
все права как собственников данной рукописи. В связи с этим авторами должно
быть представлено в редакцию письмо в следующей форме:
Соглашение о передаче права на публикацию:

\textit{<<Мы, нижеподписавшиеся, авторы рукописи <<$\qquad\qquad$>>, передаем
учредителям и редколлегии журнала <<Информатика и её применения>>
неисключительное право опубликовать данную рукопись статьи на русском языке как
в печатной, так и в электронной версиях журнала. Мы подтверждаем, что данная
публикация не нарушает авторского права других лиц или организаций. Подписи
авторов: (ф.\,и.\,о., дата, адрес)>>.}

Указанное соглашение может быть представлено 
как в бумажном виде, так и в виде отсканированной копии (с подписями авторов).


Редколлегия вправе запросить у авторов экспертное заключение о возможности
опубликования представленной статьи в открытой печати. %\\[-13.5pt]
\item Статья
подписывается всеми авторами. На отдельном листе представляются данные автора
(или всех авторов): фамилия, полные имя и отчество, телефон, факс, e-mail,
почтовый адрес. Если работа выполнена несколькими авторами, указывается фамилия
одного из них, ответственного за переписку с редакцией. %\\[-13.5pt]
\item Редакция журнала
осуществляет самостоятельную экспертизу присланных статей. Возвращение рукописи
на доработку не означает, что статья уже принята к печати. Доработанный вариант
с ответом на замечания рецензента необходимо прислать в редакцию. %\\[-13.5pt]
\item Решение
редакционной коллегии о принятии статьи к печати или ее отклонении сообщается
авторам. Редколлегия не обязуется направлять рецензию авторам отклоненной
статьи. %\\[-13.5pt]
\item Корректура статей высылается авторам для просмотра. Редакция
просит авторов присылать свои замечания в кратчайшие сроки. %\\[-13.5pt]
\item При
подготовке рукописи в MS Word рекомендуется использовать следующие настройки.
Параметры страницы: формат~--- А4; ориентация~--- книжная; поля (см): внутри~---
2,5, снаружи~--- 1,5, сверху~--- 2, снизу~--- 2, от края до нижнего
колонтитула~--- 1,3. Основной текст: стиль~--- <<Обычный>>: шрифт Times New
Roman, размер 14~пунктов, абзацный отступ~--- 0,5~см, 1,5 интервала,
выравнивание~--- по ширине. Рекомендуемый объем рукописи~--- не свыше
25~страниц указанного формата. Ознакомиться с шаблонами, содержащими примеры
оформления, можно по адресу в Интернете:
\textsf{http://www.ipiran.ru/journal/template.doc}.
\item К рукописи, предоставляемой в 2-х
экземплярах, обязательно прилагается электронная версия статьи (как правило, в
форматах MS WORD (.doc) или \LaTeX\ (.tex), а также~--- дополнительно~--- в
формате .pdf) на дискете, лазерном диске или по электронной почте. Сокращения
слов, кроме стандартных, не применяются. Все страницы рукописи должны быть
пронумерованы. %\\[-13.5pt]
\item Статья должна содержать следующую информацию на русском и
английском языках: название, Ф.И.О. авторов, места работы авторов и их
электронные адреса, подробные сведения об авторах, оформленные в соответствии с форматом, 
определяемым файлами {\sf http://www.ipiran.ru/journal/issues/2011\_05\_01/authors.asp} и 
{\sf http://www.ipiran.ru/journal/issues/2011\_01\_eng/authors.asp},
аннотация (не более 100~слов), ключевые слова. Ссылки на
литературу в тексте статьи нумеруются (в квадратных скобках) и располагаются в
порядке их первого упоминания. В~списке литературы не должно быть позиций, на которые нет ссылки в тексте статьи.
Все фамилии авторов, заглавия статей, названия
книг, конференций и~т.\,п.\ даются на языке оригинала, если этот язык
использует кириллический или латинский алфавит. %\\[-13.5pt]
\item Присланные в редакцию материалы авторам не возвращаются.
\item При отправке файлов по электронной
почте просим придерживаться следующих правил:
\begin{itemize}
\item указывать в поле subject (тема) название журнала и фамилию автора; %\\[-13.5pt]
\item использовать attach (присоединение); %\\[-13.5pt]
\item в случае больших объемов информации возможно
использование общеизвестных архиваторов (ZIP, RAR); %\\[-13.5pt]
\item в состав электронной версии статьи должны входить: файл, содержащий текст статьи, и файл(ы),
содержащий(е) иллюстрации. %\\[-13.5pt]
\end{itemize}
\item Журнал <<Информатика и её применения>> является некоммерческим изданием. 
Плата за публикацию с авторов не взимается, гонорар авторам не выплачивается.
\end{enumerate}
\thispagestyle{empty}
\textbf{Адрес редакции:} Москва 119333,
ул.~Вавилова, д.~44, корп.~2, ИПИ РАН\\
\hphantom{\textbf{Адрес редакции:} }Тел.: +7 (499) 135-86-92\ \
Факс:  +7 (495) 930-45-05\ \  E-mail:   rust@ipiran.ru }
}

\end{document}


%\tableofcontents

%\end{document}





%\def\stat{cont}
{%\hrule\par
%\vskip 7pt % 7pt
\raggedleft\Large \bf%\baselineskip=3.2ex
А\,В\,Т\,О\,Р\,С\,К\,И\,Й\ \ У\,К\,А\,З\,А\,Т\,Е\,Л\,Ь\ \ З\,А\ \ 2\,0\,0\,7 г. \vskip 17pt
    \hrule
    \par
\vskip 21pt plus 6pt minus 3pt }

\label{st\stat}

\def\tit{\ }

\def\aut{\ }
\def\auf{\ }

\def\leftkol{\ } % ENGLISH ABSTRACTS}

\def\rightkol{\ } %ENGLISH ABSTRACTS}

\titele{\tit}{\aut}{\auf}{\leftkol}{\rightkol}


\contentsline {chapter}{\ }{Выпуск \quad Стр.} 
\contentsline {section}{\textbf{Батракова Д.\,А., Королев В.\,Ю., Шоргин С.\,Я.}\ \ Новый метод вероятностно-ста\-ти\-сти\-че\-ско\-го анализа информационных потоков в\nobreakspace {}телекоммуникационных сетях}{\qquad 1 \qquad 40} 
\contentsline {section}{\textbf{Борисов А.\,В.}\ \ Байесовское оценивание в системах наблюдения с\nobreakspace {}марковскими скачкообразными процессами: игровой подход}{\qquad 2 \qquad 65}
\contentsline {section}{\textbf{Босов А.\,В., Иванов А.\,В.}\ \ Программная инфраструктура информационного Web-пор\-тала}{\qquad 2 \qquad 50}
\contentsline {section}{\textbf{Захаров В.\,Н., Калиниченко Л.\,А., Соколов И.\,А., Ступников С.\,А.}\ \ Конструирование канонических информационных моделей для интегрированных информационных систем}{\qquad 2 \qquad 15}
\contentsline {section}{\textbf{Захаров В.\,Н., Козмидиади В.\,А.}\ \ Средства обеспечения отказоустойчивости при\-ло\-жений}{\qquad 1 \qquad 14} 
\contentsline {section}{\textbf{Иванов А.\,В.}\ \ см. Босов А.\,В.\hfill\hfill\hfill\hfill\hfill\hfill\hfill\hfill\hfill\hfill\hfill\hfill\hfill\hfill\hfill\hfill\hfill\hfill\hfill\hfill\hfill\hfill\hfill\hfill\hfill\hfill\hfill\hfill\hfill\hfill\hfill\hfill\hfill\hfill\hfill}{\ }
\contentsline {section}{\textbf{Ильин В.\,Д., Соколов И.\,А.}\ \ Символьная модель системы знаний информатики в\nobreakspace {}че\-ло\-ве\-ко-автоматной среде}{\qquad 1 \qquad 66} 
\contentsline {section}{\textbf{Калиниченко Л.\,А.}\ \ см. Захаров В.\,Н.\hfill\hfill\hfill\hfill\hfill\hfill\hfill\hfill\hfill\hfill\hfill\hfill\hfill\hfill\hfill\hfill\hfill\hfill\hfill\hfill\hfill\hfill\hfill\hfill\hfill\hfill\hfill\hfill\hfill\hfill\hfill\hfill\hfill\hfill\hfill}{\ }
\contentsline {section}{\textbf{Козеренко Е.\,Б.}\ \ Лингвистическое моделирование для систем машинного перевода и обработки знаний}{\qquad 1 \qquad 54} 
\contentsline {section}{\textbf{Козмидиади В.\,А.}\ \ см. Захаров В.\,Н.\hfill\hfill\hfill\hfill\hfill\hfill\hfill\hfill\hfill\hfill\hfill\hfill\hfill\hfill\hfill\hfill\hfill\hfill\hfill\hfill\hfill\hfill\hfill\hfill\hfill\hfill\hfill\hfill\hfill\hfill\hfill\hfill\hfill\hfill\hfill }{\ } 
\contentsline {section}{\textbf{Королев В.\,Ю.}\ \ см. Батракова Д.\,А.\hfill\hfill\hfill\hfill\hfill\hfill\hfill\hfill\hfill\hfill\hfill\hfill\hfill\hfill\hfill\hfill\hfill\hfill\hfill\hfill\hfill\hfill\hfill\hfill\hfill\hfill\hfill\hfill\hfill\hfill\hfill\hfill\hfill\hfill\hfill}{\ } 
\contentsline {section}{\textbf{Кудрявцев А.\,А., Шоргин С.\,Я.}\ \ Байесовский подход к\nobreakspace {}анализу систем массового обслуживания и\nobreakspace {}показателей надежности}{\qquad 2 \qquad 76}
\contentsline {section}{\textbf{Печинкин А.\,В., Соколов И.\,А., Чаплыгин В.\,В.}\ \ Многолинейная система массового обслуживания с конечным накопителем и ненадежными приборами}{\qquad 1 \qquad 27} 
\contentsline {section}{\textbf{Печинкин А.\,В., Соколов И.\,А., Чаплыгин В.\,В.}\ \ Стационарные характеристики многолинейной\nobreakspace {}системы массового обслуживания с\nobreakspace {}одновременными отказами приборов}{\qquad 2 \qquad 39}
\contentsline {section}{\textbf{Синицын И.\,Н.}\ \ Корреляционные методы построения аналитических информационных моделей флуктуаций полюса Земли по априорным данным}{\qquad 2 \qquad \hphantom{9}2}
\contentsline {section}{\textbf{Синицын И.\,Н.}\ \ Развитие теории фильтров Пугачева для оперативной обработки информации в стохастических системах}{{\qquad 1 \qquad \hphantom{9}3}} 
\contentsline {section}{\textbf{Соколов И.\,А.}\ \ см. Захаров В.\,Н.\hfill\hfill\hfill\hfill\hfill\hfill\hfill\hfill\hfill\hfill\hfill\hfill\hfill\hfill\hfill\hfill\hfill\hfill\hfill\hfill\hfill\hfill\hfill\hfill\hfill\hfill\hfill\hfill\hfill\hfill\hfill\hfill\hfill\hfill\hfill}{\ }
\contentsline {section}{\textbf{Соколов И.\,А.}\ \ см. Ильин В.\,Д.\hfill\hfill\hfill\hfill\hfill\hfill\hfill\hfill\hfill\hfill\hfill\hfill\hfill\hfill\hfill\hfill\hfill\hfill\hfill\hfill\hfill\hfill\hfill\hfill\hfill\hfill\hfill\hfill\hfill\hfill\hfill\hfill\hfill\hfill\hfill}{\ } 
\contentsline {section}{\textbf{Соколов И.\,А.}\ \ см. Печинкин А.\,В.\hfill\hfill\hfill\hfill\hfill\hfill\hfill\hfill\hfill\hfill\hfill\hfill\hfill\hfill\hfill\hfill\hfill\hfill\hfill\hfill\hfill\hfill\hfill\hfill\hfill\hfill\hfill\hfill\hfill\hfill\hfill\hfill\hfill\hfill\hfill}{\ } 
\contentsline {section}{\textbf{Соколов И.\,А.}\ \ см. Печинкин А.\,В.\hfill\hfill\hfill\hfill\hfill\hfill\hfill\hfill\hfill\hfill\hfill\hfill\hfill\hfill\hfill\hfill\hfill\hfill\hfill\hfill\hfill\hfill\hfill\hfill\hfill\hfill\hfill\hfill\hfill\hfill\hfill\hfill\hfill\hfill\hfill}{\ }
\contentsline {section}{\textbf{Ступников С.\,А.}\ \ см. Захаров В.\,Н.\hfill\hfill\hfill\hfill\hfill\hfill\hfill\hfill\hfill\hfill\hfill\hfill\hfill\hfill\hfill\hfill\hfill\hfill\hfill\hfill\hfill\hfill\hfill\hfill\hfill\hfill\hfill\hfill\hfill\hfill\hfill\hfill\hfill\hfill\hfill}{\ }
\contentsline {section}{\textbf{Чаплыгин В.\,В.}\ \ см. Печинкин А.\,В.\hfill\hfill\hfill\hfill\hfill\hfill\hfill\hfill\hfill\hfill\hfill\hfill\hfill\hfill\hfill\hfill\hfill\hfill\hfill\hfill\hfill\hfill\hfill\hfill\hfill\hfill\hfill\hfill\hfill\hfill\hfill\hfill\hfill\hfill\hfill}{\ } 
\contentsline {section}{\textbf{Чаплыгин В.\,В.}\ \ см. Печинкин А.\,В.\hfill\hfill\hfill\hfill\hfill\hfill\hfill\hfill\hfill\hfill\hfill\hfill\hfill\hfill\hfill\hfill\hfill\hfill\hfill\hfill\hfill\hfill\hfill\hfill\hfill\hfill\hfill\hfill\hfill\hfill\hfill\hfill\hfill\hfill\hfill}{\ }
\contentsline {section}{\textbf{Шоргин С.\,Я.}\ \ см. Батракова Д.\,А.\hfill\hfill\hfill\hfill\hfill\hfill\hfill\hfill\hfill\hfill\hfill\hfill\hfill\hfill\hfill\hfill\hfill\hfill\hfill\hfill\hfill\hfill\hfill\hfill\hfill\hfill\hfill\hfill\hfill\hfill\hfill\hfill\hfill\hfill\hfill}{\ } 
\contentsline {section}{\textbf{Шоргин С.\,Я.}\ \ см. Кудрявцев А.\,А.\hfill\hfill\hfill\hfill\hfill\hfill\hfill\hfill\hfill\hfill\hfill\hfill\hfill\hfill\hfill\hfill\hfill\hfill\hfill\hfill\hfill\hfill\hfill\hfill\hfill\hfill\hfill\hfill\hfill\hfill\hfill\hfill\hfill\hfill\hfill}{\ }
%\thispagestyle{myheadings}
\def\leftfootline{\small{\textbf{\thepage}
\hfill ИНФОРМАТИКА И ЕЁ ПРИМЕНЕНИЯ\ \ \ том~1\ \ \ выпуск~2\ \ \ 2007}
}%
 \def\rightfootline{\small{ИНФОРМАТИКА И ЕЁ ПРИМЕНЕНИЯ\ \ \ том~1\ \ \ выпуск~2\ \ \ 2007
 \hfill \textbf{\thepage}}}
 \label{end\stat}

%\def\stat{cont-e}
{%\hrule\par
%\vskip 7pt % 7pt
\raggedleft\Large \bf%\baselineskip=3.2ex
2\,0\,0\,7\ \ A\,U\,T\,H\,O\,R\ \ I\,N\,D\,E\,X \vskip 17pt
    \hrule
    \par
\vskip 21pt plus 6pt minus 3pt }

\label{st\stat}

\def\tit{\ }

\def\aut{\ }
\def\auf{\ }

\def\leftkol{\ } % ENGLISH ABSTRACTS}

\def\rightkol{\ } %ENGLISH ABSTRACTS}

\titele{\tit}{\aut}{\auf}{\leftkol}{\rightkol}


\contentsline {chapter}{\ }{Issue \quad Page} 
\contentsline {subsection}{\textbf{Batrakova D.\,A., Korolev V.\,Yu., Shorgin S.\,Ya.}\ \ A New Method for the Probabilistic and Statistical Analysis of Information Flows in Telecommunication Networks}{\qquad 1 \qquad 40} 
\contentsline {subsection}{\textbf{Borisov A.\,V.}\ \ Bayesian Estimation in\nobreakspace {}Observation Systems with\nobreakspace {}Markov Jump Processes: Game-Theoretic Approach}{\qquad 2 \qquad 65} 
\contentsline {subsection}{\textbf{Bosov A.\,V., Ivanov A.\,V.}\ \ Linguistic Simulation for Machine Translation and Knowledge Management Systems}{\qquad 2 \qquad 50} 
\contentsline {subsection}{\textbf{Chaplygin V.\,V.} see Pechinkin A.\,V.\hfill\hfill\hfill\hfill\hfill\hfill\hfill\hfill\hfill\hfill\hfill\hfill\hfill\hfill\hfill\hfill\hfill\hfill\hfill\hfill\hfill\hfill\hfill\hfill\hfill\hfill\hfill\hfill\hfill\hfill\hfill\hfill\hfill\hfill\hfill}{\ }
\contentsline {subsection}{\textbf{Chaplygin V.\,V.} see Pechinkin A.\,V.\hfill\hfill\hfill\hfill\hfill\hfill\hfill\hfill\hfill\hfill\hfill\hfill\hfill\hfill\hfill\hfill\hfill\hfill\hfill\hfill\hfill\hfill\hfill\hfill\hfill\hfill\hfill\hfill\hfill\hfill\hfill\hfill\hfill\hfill\hfill}{\ }
\contentsline {subsection}{\textbf{Ilyin V.\,D., Sokolov I.\,A.}\ \ The Symbol Model of Informatics Knowledge System in Human-Automaton Environment}{\qquad 1 \qquad 66} 
\contentsline {subsection}{\textbf{Ivanov A.\,V.} see Bosov A.\,V.\hfill\hfill\hfill\hfill\hfill\hfill\hfill\hfill\hfill\hfill\hfill\hfill\hfill\hfill\hfill\hfill\hfill\hfill\hfill\hfill\hfill\hfill\hfill\hfill\hfill\hfill\hfill\hfill\hfill\hfill\hfill\hfill\hfill\hfill\hfill}{\ }
\contentsline {subsection}{\textbf{Kalinichenko L.\,A.} see Zakharov V.\,N.\hfill\hfill\hfill\hfill\hfill\hfill\hfill\hfill\hfill\hfill\hfill\hfill\hfill\hfill\hfill\hfill\hfill\hfill\hfill\hfill\hfill\hfill\hfill\hfill\hfill\hfill\hfill\hfill\hfill\hfill\hfill\hfill\hfill\hfill\hfill}{\ }
\contentsline {subsection}{\textbf{Korolev V.\,Yu.} see Batrakova D.\,A.\hfill\hfill\hfill\hfill\hfill\hfill\hfill\hfill\hfill\hfill\hfill\hfill\hfill\hfill\hfill\hfill\hfill\hfill\hfill\hfill\hfill\hfill\hfill\hfill\hfill\hfill\hfill\hfill\hfill\hfill\hfill\hfill\hfill\hfill\hfill}{\ }
\contentsline {subsection}{\textbf{Kozerenko E.\,B.}\ \ Linguistic Simulation for Machine Translation and Knowledge Management Systems}{\qquad 1 \qquad 54} 
\contentsline {subsection}{\textbf{Kozmidiady V.\,A.} see Zakharov V.\,N.\hfill\hfill\hfill\hfill\hfill\hfill\hfill\hfill\hfill\hfill\hfill\hfill\hfill\hfill\hfill\hfill\hfill\hfill\hfill\hfill\hfill\hfill\hfill\hfill\hfill\hfill\hfill\hfill\hfill\hfill\hfill\hfill\hfill\hfill\hfill}{\ }
\contentsline {subsection}{\textbf{Kudryavtsev A.\,A., Shorgin S.\,Ya.}\ \ Bayesian Approach to Queueing Systems and Reliability Characteristics}{\qquad 2 \qquad 76} 
\contentsline {subsection}{\textbf{Pechinkin A.\,V., Sokolov I.\,A., Chaplygin V.\,V.}\ \ Multichannel Queuing System with Finite Buffer and Unreliable Servers}{\qquad 1 \qquad 27} 
\contentsline {subsection}{\textbf{Pechinkin A.\,V., Sokolov I.\,A., Chaplygin V.\,V.}\ \ Stationary Characteristics of a Multichannel Queueing System with\nobreakspace {}Simultaneous Refusals of Servers}{\qquad 2 \qquad 39} 
\contentsline {subsection}{\textbf{Shorgin S.\,Ya.} see Batrakova D.\,A.\hfill\hfill\hfill\hfill\hfill\hfill\hfill\hfill\hfill\hfill\hfill\hfill\hfill\hfill\hfill\hfill\hfill\hfill\hfill\hfill\hfill\hfill\hfill\hfill\hfill\hfill\hfill\hfill\hfill\hfill\hfill\hfill\hfill\hfill\hfill}{\ }
\contentsline {subsection}{\textbf{Shorgin S.\,Ya.} see Kudryavtsev A.\,A.\hfill\hfill\hfill\hfill\hfill\hfill\hfill\hfill\hfill\hfill\hfill\hfill\hfill\hfill\hfill\hfill\hfill\hfill\hfill\hfill\hfill\hfill\hfill\hfill\hfill\hfill\hfill\hfill\hfill\hfill\hfill\hfill\hfill\hfill\hfill}{\ }
\contentsline {subsection}{\textbf{Sinitsyn I.\,N.}\ \ Correlational Methods for Analytical Informational Models of the Earth Pole Fluctuations Design Based on a priori Data}{\qquad 2 \qquad \hphantom{9}2}
\contentsline {subsection}{\textbf{Sinitsyn I.\,N.}\ \ Development of Pugachev Filtering for Stochastic Systems}{\qquad 1 \qquad \hphantom{9}3}
\contentsline {subsection}{\textbf{Sokolov I.\,A.} see Ilyin V.\,D.\hfill\hfill\hfill\hfill\hfill\hfill\hfill\hfill\hfill\hfill\hfill\hfill\hfill\hfill\hfill\hfill\hfill\hfill\hfill\hfill\hfill\hfill\hfill\hfill\hfill\hfill\hfill\hfill\hfill\hfill\hfill\hfill\hfill\hfill\hfill}{\ }
\contentsline {subsection}{\textbf{Sokolov I.\,A.} see Pechinkin A.\,V.\hfill\hfill\hfill\hfill\hfill\hfill\hfill\hfill\hfill\hfill\hfill\hfill\hfill\hfill\hfill\hfill\hfill\hfill\hfill\hfill\hfill\hfill\hfill\hfill\hfill\hfill\hfill\hfill\hfill\hfill\hfill\hfill\hfill\hfill\hfill}{\ }
\contentsline {subsection}{\textbf{Sokolov I.\,A.} see Pechinkin A.\,V.\hfill\hfill\hfill\hfill\hfill\hfill\hfill\hfill\hfill\hfill\hfill\hfill\hfill\hfill\hfill\hfill\hfill\hfill\hfill\hfill\hfill\hfill\hfill\hfill\hfill\hfill\hfill\hfill\hfill\hfill\hfill\hfill\hfill\hfill\hfill}{\ }
\contentsline {subsection}{\textbf{Sokolov I.\,A.} see Zakharov V.\,N.\hfill\hfill\hfill\hfill\hfill\hfill\hfill\hfill\hfill\hfill\hfill\hfill\hfill\hfill\hfill\hfill\hfill\hfill\hfill\hfill\hfill\hfill\hfill\hfill\hfill\hfill\hfill\hfill\hfill\hfill\hfill\hfill\hfill\hfill\hfill}{\ }
\contentsline {subsection}{\textbf{Stupnikov S.\,A.} see Zakharov V.\,N.\hfill\hfill\hfill\hfill\hfill\hfill\hfill\hfill\hfill\hfill\hfill\hfill\hfill\hfill\hfill\hfill\hfill\hfill\hfill\hfill\hfill\hfill\hfill\hfill\hfill\hfill\hfill\hfill\hfill\hfill\hfill\hfill\hfill\hfill\hfill}{\ }
\contentsline {subsection}{\textbf{Zakharov V.\,N., Kalinichenko L.\,A., Sokolov I.\,A., Stupnikov S.\,A.}\ \ Development of Canonical Information Models for Integrated Information Systems}{\qquad 2 \qquad 15} 
\contentsline {subsection}{\textbf{Zakharov V.\,N., Kozmidiady V.\,A.}\ \ Means Providing Applications Fault Tolerance}{\qquad 1 \qquad 14} 
\def\leftfootline{\small{\textbf{\thepage}
\hfill ИНФОРМАТИКА И ЕЁ ПРИМЕНЕНИЯ\ \ \ том~1\ \ \ выпуск~2\ \ \ 2007}
}%
 \def\rightfootline{\small{ИНФОРМАТИКА И ЕЁ ПРИМЕНЕНИЯ\ \ \ том~1\ \ \ выпуск~2\ \ \ 2007
 \hfill \textbf{\thepage}}}
 \label{end\stat}


%\tableofcontents


\end{document}

\newcommand{\Ack}{\subsection*{\protect\large\bf Acknowledgments}}