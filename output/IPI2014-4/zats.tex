\def\stat{zats-chu}



\def\tit{ОБ ЭРГОНОМИЧЕСКИХ ЗАВИСИМОСТЯХ
МЕЖДУ~ПАРАМЕТРАМИ СИТУАЦИОННОГО ЗАЛА С~ИСПОЛЬЗОВАНИЕМ~ИЗОГНУТОГО КОЛЛЕКТИВНОГО ЭКРАНА}



\def\titkol{Об эргономических зависимостях между
параметрами ситуационного зала с использованием
изогнутого %коллективного
экрана}

\def\aut{А.\,А.~Зацаринный$^1$, К.\,Г.~Чупраков$^2$}

\def\autkol{А.\,А.~Зацаринный, К.\,Г.~Чупраков}

\titel{\tit}{\aut}{\autkol}{\titkol}

%{\renewcommand{\thefootnote}{\fnsymbol{footnote}} \footnotetext[1]
%{Работа выполнена при финансовой поддержке РНФ (проект 14-11-00364).}}


\renewcommand{\thefootnote}{\arabic{footnote}}
\footnotetext[1]{Институт проблем информатики Российской академии наук, azatsarinny@ipiran.ru}
\footnotetext[2]{Институт проблем информатики Российской академии наук, chkos@rambler.ru}

\vspace*{-6pt}

\Abst{Рассмотрен подход к определению зависимостей между параметрами
ситуационного зала: размерами помещения, числом наблюдателей, информационной
емкостью контента (количеством знаков) и~шириной экрана. Эти зависимости позволяют
рассчитать неизвестный параметр ситуационного зала при известных других с~выполнением
требований государственных и~международных стандартов по эргономике рабочих мест.
Предложенные формулы применимы и~для изогнутых экранов, определяемых в~рамках
статьи углом кривизны~$\beta$ (для плоского экрана $\beta\hm=0$). Данный параметр может
быть интерпретирован как угол наклона между дисплеями в~полиэкране. Наличие этого
параметра позволяет оценить эффективность использования изогнутых экранов в~составе
систем отображения информации коллективного использования. Предложен общий подход
к~определению количества рабочих мест для коллективного экрана, который может быть
применен для их различных взаимных расположений.}

\KW{изогнутый экран коллективного пользования; ситуационный зал; диспетчерский пункт;
эргономические зависимости; область комфортного наблюдения; угол кривизны экрана;
видеостена; полиэкран; эффективность; оправданность цены}

\DOI{10.14357/19922264140411}




\vskip 10pt plus 9pt minus 6pt

\thispagestyle{headings}

\begin{multicols}{2}

\label{st\stat}

\section{Введение}
	
     Опыт создания ситуационных центров (СЦ) обозначил целый ряд проблем,
которые препятствуют разработке и~внедрению новых технологий в~процесс
управления~[1--4]. Эти трудности могут возникать в~результате
недостаточного применения системного подхода при проектировании, когда
имеющийся функционал прикладных средств не подкреплен достаточной
технологической или технической базой и~наоборот. Значительное число
проблем создания и~внедрения СЦ вызвано человеческим фактором: плохой
мотивацией персонала на обучение новым процессам и~слабой
заинтересованностью первого лица~[5, 6].

     Важно понимать, что любая проблема, будь то техническая или
организационная, может стать узким местом в~обеспечении функционирования
СЦ в~требуемых режимах~\cite{3-zat, 7-zat}. Для
выявления таких узких мест необходимы системные оценки эффективности
СЦ, которые должны учитывать оценки всех его отдельных
компонентов~\cite{8-zat}. При этом в~рамках технического
и~эксплуатационного компонентов важную роль играют эргономические
показатели, так как они определяют эффективность пользовательского
интерфейса в~СЦ. Эргономические требования к системам
отображения информации являются важными с~точки зрения не только
обеспечения комфортных условий работы, но и~более эффективного
использования пространственного и~материального ресурса~[9, 10] .

     Одним из наиболее заметных направлений в~развитии средств
отображения информации является применение так называемых изогнутых
экранов (curved screen), реализуемых на основе технологии OLED
(organic light-emitted diode), широко
применяемой для мобильных устройств. Отметим, что идея таких
<<изогнутых>> или вообще неплоских экранов не нова: помимо кинотеатров
неплоские экраны можно встретить и~в диспетчерских пунктах. Производители
профессиональных полиэкранов и~видеостен предусматривают возможность
создания изогнутых экранов за счет взаимного поворота между отдельными
дисплеями. Вместе с~тем ощутимых преимуществ, которые дают изогнутые
экраны, производители и~их маркетологи сформулировать не смогли~[11].

     В настоящей статье сделана попытка восполнить этот пробел.
Рассмотрены методические подходы к~определению зависимостей между
размерами помещения, разме\-рами экрана, числом наблюда\-телей
и~характеристиками отображаемого контента, сформулированные
в~работах~[9, 10], применительно к~изогнутому экрану.
Полученные зависимости позволяют получать оценки эффективности
применения изогнутых экранов в~сис\-те\-мах коллективного отображения
информации СЦ. Более того, в~отличие от известного
оценочного подхода по уже реализованным комплексам в~[12] предлагаемый
в~статье подход предусматривает \mbox{построение} сис\-те\-мы
     <<по\-ме\-ще\-ние--экран--наблю\-да\-те\-ли>> в~строгом соответствии
с~нор\-ма\-тив\-но-тех\-ни\-че\-ской базой, определяемой существующими
государственными и~международными стандартами в~части эргономических
норм.

\section{Общий подход. Термины и~определения}

	Создание и~оборудование ситуационного зала должно среди прочих
требований опираться на существующие стандарты по эргономике,
действующие на территории РФ. Большинство указаний, содержащихся
в~стандартах, основано на особенностях человеческого восприятия и~потому
может служить практическим руководством при оснащении помещений. Это
относится и~к~средствам отображения информации. Подход,
сформулированный в~[9, 10] и~используемый в~данной статье, опирается на
стандарты~[13--19].

    О существовании связей между основными параметрами системы
    <<по\-ме\-ще\-ние--дис\-плей--на\-блю\-да\-те\-ли>> можно судить на
основании сле\-ду\-юще\-го примера. Увеличение информационной емкости
контента за счет уменьшения символов незамедлительно приведет
к~уменьшению проектного расстояния наблюдения. Это, в~свою очередь,
уменьшит рабочую площадь наблюдения, а~следовательно, и~количество
персонала, который может работать в~нем одновременно.

	На основании рекомендаций, сформулированных в~[13--19]
и~выделенных в~[9, 10], можно при\-ступить к формированию взаимосвязей
между основны\-ми параметрами системы
<<по\-ме\-ще\-ние--экран--на\-блю\-да\-те\-ли>>. Далее в~работе будут
использованы следующие термины и~обозначения:
\begin{description}
\item[\,]
$D$~--- проектное расстояние наблюдения, измеряется в~метрах. Это
расстояние или диапазон расстояний между экраном и~глазами наблюдателей,
при котором изображение соответствует требованиям разборчивости
и~удобочита\-емости;
\item[\,]
$N$~--- количество людей, которые должны одновременно работать
с~коллективным экраном, получая с~него визуальную информацию
в~комфортных условиях;
\item[\,]
$Q$~--- диаметр помещения, ограниченного сте\-нами. В~большинстве случаев
помещение является прямоугольным, а~в~рамках статьи, обобщенно,~---
выпуклым. Параметр~$Q$~--- максимальное из расстояний между двумя произвольными
точками помещения, измеряется в~метрах.\linebreak В~некоторых задачах $Q$ может
быть ограничено искусственным образом ввиду особенностей геометрии
помещения, группировки наблюдателей. Отметим разницу между проектным
расстояни\-ем наблюдения~$D$ и~диаметром помещения~$Q$. Первый
параметр характеризуется свойствами системы <<экран--на\-блю\-да\-те\-ли>>,
а~второй~--- исключительно свойствами помещения. Ясно, что~$D$ не
может превосходить~$Q$;
\item[\,]
$W$~--- максимальное расстояние между двумя точками экрана
в~горизонтальной плоскости, <<плоская>> ширина экрана. Для плоского
экрана~$W$~соответствует его ширине, а~для кривого~--- расстоянию между
его краями. Измеряется в~мет\-рах;
\item[\,]
$I$~--- необходимая статичная информационная емкость отображаемого
контента~--- максимальное количество знаков или символов, которые должен
отобразить дисплей в~одном кадре или неподвижном изображении.
Определяется на основании задач ситуационного зала и~иных приложений
и~объемов отображаемого контента. Измеряется в~количестве отображаемых
знаков;
\item[\,]
$\alpha$~--- максимальный стягиваемый угол (угловой размер экрана) по
горизонтали, ограничиваемый уровнем концентрации наблюдения;
\item[\,]
\textit{изогнутый экран}~--- экран, сечение которого в~горизонтальной
плоскости является дугой некоторой окружности. Также под изогнутым
экраном будем понимать полиэкран, \mbox{составленный} из плоских дисплеев
одинакового размера и~расположенных друг к~другу под одним углом
(дискретное приближение к~окруж\-ности). В~рамках данной работы
в~вертикальной плоскости экран считается плоским;
\begin{figure*}[b] %fig1
\vspace*{1pt}
 \begin{center}
 \mbox{%
 \epsfxsize=91.03mm
 \epsfbox{zac-1.eps}
 }
 \end{center}
 \vspace*{-9pt}
\Caption{Область комфортного наблюдения экрана}
\end{figure*}
\item[\,]
$\beta$~--- угол, определяющий кривизну (изогнутость) экрана. Определяется
как половина дуги, которую стягивает экран на соответствующей ему
окружности. Такое определение эквивалентно углу наклона между отдельными
дисплеями в~случае полиэкрана из плоских дисплеев. Положительные
значения~$\beta$ соответствуют случаю, когда наблюдатели расположены
в~той же части пространства, что и~центр окружности, на которую ложится
горизонтальное сечение экрана, или экран, <<выпуклый от наблюдателя>>.
Отрицательные значения~$\beta$, соответственно,~--- когда экран <<выпуклый
к~наблюдателю>>. Если $\beta\hm=0$, то экран плоский;
\item[\,]
{ОКН}~--- область комфортного наблюдения~--- область пространства, где
выполнены рекомендации международных и~государственных стандартов по
оборудованию рабочих мест наблюдения с~коллективного экрана.
\end{description}

\section{Построение области комфортного наблюдения и~исследование ее свойств}


    По аналогии с~результатами~[9, 10] c~учетом ограничений~[13--19]
ОКН в~случае изогнутого экрана будет также
являться фигурой пересечения нескольких областей (рис.~1).



     Рассмотрим такую систему координат, ось абсцисс которой проходит
через крайние точки экрана, а~ось ординат является осью симметрии для
экрана и~направлена в~сторону наблюдателей. Единица измерения по обеим
осям равна 1~м. С~по\-мощью средств аналитической геометрии вы\-чис\-лим
координаты основных точек, которые будут участвовать в~дальнейших
оценках:
\begin{description}
\item[\,]     $O_1$~--- центр левого круга:
     $
     O_1\left( ({D}/{2})\sin\beta - W/2;\right.$ $\left.\left(D/2\right)\cos\beta\right);
     $
\item[\,]
      $O_2$~--- центр правого круга:
     $O_2\left( {W}/{2}-({D}/{2})\sin\beta;\right.$ $\left.({D}/{2})\cos\beta\right);$

\item[\,] $l_1$~--- прямая, выходящая из левого края экрана под углом~$\alpha$
к~$OY$;
\item[\,]
$l_2$~--- прямая, выходящая из правого края экрана под углом~$\alpha$
к~$OY$;
\item[\,]
$K$~--- левый край экрана: $K\left( -{W}/{2};\,0\right);$
\item[\,]
$L$~--- правый край экрана: $L\left( {W}/{2};\,0\right);$
\item[\,]
$M$~--- точка пересечения прямых~$l_1$ и~$l_2$:

\noindent
$$
M\left( 0;({W}/{2})\ctg(\alpha\hm+\beta)\right)\,;
$$

\vspace*{-4pt}
\item[\,]
$A$~--- верхняя точка пересечения кругов:


\noindent
$$
A\left( 0;({D}/{2})\left(\cos\beta\hm+\sqrt{1-(C-\sin\beta)^2}\,\right)\right)\,;
$$

\vspace*{-4pt}

\item[\,]
$B$~--- нижняя точка пересечения кругов:

\vspace*{-2pt}

\noindent
$$
B\left( 0;({D}/{2})\left( \cos\beta \hm-\sqrt{1-(C-\sin\beta)^2}\,\right)\right)\,;
$$

\vspace*{-2pt}

\item[\,]
$F$~--- точка пересечения прямой~$l_2$ и~правого круга, отличная от~$L$:
$F\left( -({D}/{2})\times\right.$\linebreak $\left.\times\left(2\sin(\alpha\hm+\beta\right)\cos\alpha\hm-
C);\,D\cos(\alpha\hm+\beta)\cos\alpha\right);$
\item[\,]
$E$~--- точка пересечения прямой $l_1$ и~левого круга, отличная от~$K$:
$E\left( ({D}/{2})\times\right.$ $\left.\times\left(2\sin(\alpha\hm+\beta)\cos\alpha\hm -
C\right);\,D\cos(\alpha\hm+\beta)\cos\alpha\right),$
где $C$~---
отношение <<плоской>> ширины экрана к~проектному расстоянию
наблюдения:

\vspace*{-2pt}

\noindent
\begin{equation}
C=\fr{W}{D}\,.
\label{e1-zat}
\end{equation}

\vspace*{-2pt}

\noindent
Как будет показано далее, оно зависит от информационной
емкости контента~$I$, отношения~$k$ высоты экрана к ширине,
отношения~$p$ ширины знака к его высоте и~угла~$\psi$, стя\-ги\-ва\-емо\-го одним
символом.
\end{description}

     Криволинейный четырехугольник $AFME$ и~есть ОКН, параметры
которой позволят оценить количество рабочих мест, которые можно
расположить в~ней, а~так\-же ширину экрана, которая при этом потребуется.
Согласно методике, сформулированной в~[9, 10], для оценки количества
рабочих мест и~ширины экрана необходимо рассчитать площадь ОКН и~ее
периметр, но сначала необходимо определить условия, при которых эта область
будет непустой и~будет приобретать различные формы. Четырехугольник
будет вырожденным, если круги с~центрами~$O_1$ и~$O_2$ не будут
пересекаться, т.\,е.\ точки~$A$ и~$B$ не будут существовать. Это условие
эквивалентно неравенству
     \begin{equation}
     1-\left (C-\sin \beta\right)^2<0\,.
     \label{e2-zat}
     \end{equation}


Далее полагаем, что это неравенство выполнено.

     Рассмотрим взаимное расположение точек~$A$, $B$ и~$M$. Возможны
три принципиально разных случая:
     \begin{enumerate}[1.]
\item {\bfseries\textit{Точка~{\boldmath{$M$}} находится выше точки~{\boldmath{$A$}}.}} Это условие
эквивалентно системе неравенств:

\vspace*{1pt}

\noindent
\begin{equation}\left.
\begin{array}{rl}
C&> \cos\beta \tg(\alpha+\beta) =R_1\,;\\[6pt]
C&> 2\sin(\alpha+\beta)\cos\alpha =R_2\,.
\end{array}
\right\}
\label{3-zat}
\end{equation}

\vspace*{-2pt}

\noindent
В этом случае решений нет~--- ОКН вырождена.

\item
{\bfseries\textit{Точка~{\boldmath{$M$}} принадлежит отрезку~{\boldmath{$[A, B]$}}}}.
Это условие
эквивалентно неравенству:
\begin{equation}
C\leq R_2\,.
\label{e4-zat}
\end{equation}
В этом случае ОКН ограничена дугами $\overset{\frown}{AE}$, $\overset{\frown}{AF}$
и~отрезками MF, ME (основной случай).

\item
{\bfseries\textit{Точка {\boldmath{$M$}} находится ниже точки~{\boldmath{$B$}}.}} Это условие
эквивалентно системе:
\begin{equation}
\left.
\begin{array}{rl}
C&> R_2\,;\\[6pt]
C&< R_1\,.
\end{array}
\right\}
\label{e5-zat}
\end{equation}
В этом случае ОКН ограничена дугами $\overset{\frown}{AB}$, принадлежащими левому
и~правому кругу. Решение этой системы существует, когда $R_1\hm>R_2$ или
$2\alpha \hm+\beta \hm> \pi/2$.
\end{enumerate}

     В случае~2 площадь и~периметр криволинейного четырехугольника
могут быть рассчитаны по формулам:

\noindent
     \begin{align}
     S_{\mathrm{ОКН}} &= \fr{D^2}{4}\left[ \left(
     \vphantom{\sqrt{1-(C-\sin\beta)^2}}
     \cos\beta -C\ctg(\alpha+\beta)
+{}\right.\right. \notag
\\
&\hspace*{-7mm}\left.{}+\sqrt{1-(C-\sin\beta)^2} \,\right)
(2\sin (\alpha+\beta)\cos\alpha -C)+ {}\notag
\\
     &{}+\{2\alpha +\beta -\arcsin (C-\sin\beta) -{}\notag\\
&     \left.{}-\sin (2\alpha+\beta -\arcsin (C-
\sin\beta))\}
     \vphantom{\sqrt{1-(C-\sin\beta)^2}}
     \right]\,;\label{e6-zat}\\
     P_{\mathrm{ОКН}} &= D\left[ \cos\alpha -\fr{C}{\sin(\alpha+\beta)}+ 2\alpha
+\beta -{}\right.\notag\\
&\hspace*{10mm}\left.{}-\arcsin (C-\sin\beta)
    \vphantom{\fr{C}{\sin(\alpha+\beta)}}
     \right]\,.
     \label{e7-zat}
     \end{align}
	
В случае~3 площадь и~периметр ОКН могут быть рассчитаны по формулам:
\begin{align}
S_{\mathrm{ОКН}} &= \fr{D^2}{4}\left[ 2\arcsin \sqrt{1-(C-\sin\beta)^2} - {}\right.\notag\\
&\left.{}-2(C-
\sin\beta)\sqrt{1-(C-\sin\beta)^2}\,\right]\,;\label{e8-zat}\\
P_{\mathrm{ОКН}} &= 2D \arcsin \sqrt{1-(C-\sin\beta)^2}\,.\label{e9-zat}
\end{align}

\smallskip

\noindent
\textbf{Замечание~1.} Данные формулы выполнимы при любом взаимном
расположении окружностей, $OY$ и~прямых~$l_1$ и~$l_2$ в~рамках
ограничений, заданных случаями.
\smallskip

\noindent
\textbf{Замечание~2.} Формулы~(2)--(9) действительны и~для
отрицательных~$\beta$, т.\,е.\ случаев, когда экран выпуклый к наблюдателям.

\smallskip

\noindent
\textbf{Замечание~3.} В случае~2 при $\beta\hm=0$ (экран плоский)
формулы~(\ref{e6-zat}) и~(\ref{e7-zat}) приобретают вид результатов,
полученных в~[9, 10], но в~отличие от них являются точными, так как не
пренебрегают малыми сла\-га\-емы\-ми, которые для неплоского случая могут
стать существенными.

\subsection{Количество рабочих мест в~области комфортного наблюдения}

    Расчет количества рабочих мест, которые могут быть размещены
в~области комфортного наблюдения, должен опираться на способ их
размещения. Например, для построения ситуационного зала, где в~центре
будет находиться овальный стол, расчет будет заключаться в~определении
максимального размера такого стола при заданных ограничениях по
характеристикам экрана и~информационной емкости контента, отображаемого
на нем.
%
В~[9, 10] рассмотрен способ рав\-но\-мер\-но-плот\-но\-го
распределения рабочих мест~--- <<сеточный>>. Согласно методике,
предложенной в~[9, 10], количество рабочих мест в~области комфортного
наблюдения может быть оценено сверху следующей величиной:
    \begin{equation}
    N=B+\Gamma =\fr{\sqrt{2}}{1{,}8^2}\,S+\fr{P}{3{,}6} +1\,.
    \label{e10-zat}
    \end{equation}

    \begin{figure*}[b] %fig2
\vspace*{1pt}
 \begin{center}
 \mbox{%
 \epsfxsize=116.938mm
 \epsfbox{zac-2.eps}
 }
 \end{center}
 \vspace*{-9pt}
\Caption{Зависимость количества рабочих мест от углов~$\alpha$ и~$\beta$ при $I\hm=4000$
и~$Q\hm=10$}
\end{figure*}

     Физический смысл единицы, входящей одним из слагаемых в~эту
формулу, в~том, что даже если ОКН вырождена в~точку и~ее площадь
и~периметр равны нулю, то все равно в~эту точку можно посадить хотя бы
одного наблюдателя. Также обратим внимание на то, что используемая
в~получении этой оценки формула Пика по своей сути смешивает
размерности~--- выводит безразмерную величину из м$^2$ и~м.
{\looseness=-1

}

     Формула~(\ref{e10-zat}) дает оценку сверху для случая расположения
рабочих мест равномерно плотно~--- в~вершинах треугольной сетки. Для
других случаев расположения рабочих мест необходимо определить другую
подходящую функцию
$$
N= N(S_{\mathrm{ОКН}}, P_{\mathrm{ОКН}})\,.
$$
Наличие такой функции для других случаев расстановки позволит использовать
оценки~(\ref{e6-zat})--(\ref{e9-zat}), предложенные в~данной статье.

\vspace*{-6pt}

\subsection{Оценка максимальной площади области комфортного
наблюдения и~максимального количества рабочих~мест в~этой области}

    Вследствие ограниченности помещения проектное расстояние~$D$ не
может превышать диаметра помещения~$Q$, поэтому ввиду оценки
    \begin{equation}
    C\leq \fr{\sqrt{I}}{193}
    \label{e11-zat}
    \end{equation}
(см.\ формулу~(21) в~[9]) для случая~2 формулы~(\ref{e6-zat}) и~(\ref{e7-zat})
могут быть преобразованы следующим образом:\\[-17pt]
\begin{multline}
S_{\mathrm{ОКН}} = \fr{Q^2}{4}\left[ \left(
\vphantom{\sqrt{1-\left( \fr{\sqrt{I}}{193}-\sin\beta\right)^2}}
\cos\beta -\fr{\sqrt{I}}{193}\ctg
(\alpha +\beta) +{}\right.\right.
\\[-1pt]
\hspace*{-5mm}\left.{}+\sqrt{1-\left( \fr{\sqrt{I}}{193}-\sin\beta\right)^2}\,\right)
\times{}
\\[-3pt]
\left.\hspace*{-12mm}{}\times
\left(2\sin(\alpha+\beta)\cos\alpha - \fr{\sqrt{I}}{193}\right) +
\{ \theta -\sin(\theta)\}
\vphantom{\sqrt{1-\left( \fr{\sqrt{I}}{193}-\sin\beta\right)^2}}
\right]\,;\label{e12-zat}
\end{multline}
\begin{equation*}
P_{\mathrm{ОКН}} = Q\left[ \cos\alpha -\fr{\sqrt{I}}{193}\sin
(\alpha+\beta)+\theta
\right]\,.
%\label{e13-zat}
\end{equation*}
Здесь\\[-15pt]
$$
\theta=2\alpha +\beta -\arcsin \left( \fr{\sqrt{I}}{193}-\sin\beta\right)\,.
$$

\vspace*{-3pt}



     Для случая~3 формулы для площади~(\ref{e8-zat})
     и~периметра~(\ref{e9-zat}) приобретут соответственно вид:
\begin{align*}
S_{\mathrm{ОКН}} &= \fr{Q^2}{4}\left[ 2\arcsin \sqrt{1-\left( \fr{\sqrt{I}}{193}-
\sin\beta\right)^2} - {}\right.\notag\\[-3pt]
&\hspace*{-10mm}\left.{}-2\left( \fr{\sqrt{I}}{193} -\sin\beta\right) \sqrt{1-\left(
\fr{\sqrt{I}}{193}-\sin\beta\right)^2}\,\right]\,; \notag%\label{e14-zat}
\\[-3pt]
P_{\mathrm{ОКН}} &= 2Q\arcsin \sqrt{1-\left( \fr{\sqrt{I}}{193}-
\sin\beta\right)^2}\,.
%\label{e15-zat}
\end{align*}

     Далее количество рабочих мест в~обоих случаях может быть посчитано
по формуле~(\ref{e10-zat}).

\begin{figure*}[b] %fig3
\vspace*{-6pt}
 \begin{center}
 \mbox{%
 \epsfxsize=110.869mm
 \epsfbox{zac-3.eps}
 }
 \end{center}
 \vspace*{-9pt}
\Caption{График <<плоской>> ширины экрана при $I\hm=4000$, $N = 10$ от углов~$\alpha$
и~$\beta$}
\end{figure*}

     Всплески, отображенные на рис.~2, происходят в~области, где
выполнены условия для случая~1, т.\,е.\ ОКН является вырожденной, поэтому
эти всплески не представляют интереса для исследований. Сами
невырожденные случаи~2 и~3 соответствуют гладким областям графика. Кроме
того, график показывает, что применение кривых экранов действительно может
быть эффективным для увеличения площади и~периметра ОКН, а~значит,
и~количества рабочих мест в~ней. Например, для случая $\alpha\hm = 45^\circ$
прирост количества рабочих мест при увеличении угла~$\beta$ с~0$^\circ$
до~15$^\circ$ при информационной емкости контента 9000~знаков может
достигать 40\%.


	
\subsection{Оценка минимальной ширины активной поверхности дисплея}

     Очевидно, что чем меньше размеры экрана, тем при прочих равных
условиях меньше его стоимость, но при этом снижаются и~функциональные
возможности экрана. Поэтому необходимо найти такие минимальные размеры
экрана, при которых обеспечивается достаточность решения требуемого
перечня функциональных задач.

     Пусть известна информационная емкость контента~$I$. Рассмотрим два
принципиально разных случая:
     \begin{enumerate}[(1)]
\item число наблюдателей неизвестно, необходимо оценить размеры экрана
(его ширину), позволяющие эффективно использовать пространство
помещения;\\[-13pt]
\item число наблюдателей известно, требуется оценить минимальные
размеры коллективного экрана, достаточные для одновременной работы
всех наблюдателей с~выполнением эргономических требований.
\end{enumerate}

\noindent
\textbf{Случай 1}. Из соотношений~(1) и~(\ref{e11-zat}) следует, что
\begin{equation*}
W_{\min} =QC\geq QC_{\min}=\fr{Q\sqrt{I}}{193}\,.
%\label{e16-zat}
\end{equation*}

\smallskip

\noindent
\textbf{Случай 2}. Согласно методике, предложенной в~[9, 10],
и~формулам~(\ref{e11-zat}) и~(\ref{e12-zat}) для системы
неравенств~(\ref{e4-zat}) получаем:

\vspace*{-2pt}

\noindent
\begin{multline*}
W_{\min} =
{}=2C_{\min} \sqrt{S_{\min}}
\left(
\left(
\vphantom{\sqrt{1-\left(\fr{\sqrt{I}}{193} -\sin\beta\right)^2}}
\cos\beta -{}\right.\right.\\
{}-\left(\fr{\sqrt{I}}{193}\right)\ctg
(\alpha+\beta) +\!
\left.\left.\sqrt{1-\left(\fr{\sqrt{I}}{193} -\sin\beta\right)^{\!2}}
\right) \times{}\right.\\[-2pt]
\left.{}\times
\left(2\sin(\alpha+\beta)\cos\alpha -
\fr{\sqrt{I}}{193}\right)+\{\theta -\sin(\theta)\}
\vphantom{\sqrt{1-\left(\fr{\sqrt{I}}{193} -\sin\beta\right)^2}}
\!\right)^{\!-1/2}\!\!\!\! \!\!={}\hspace*{-7pt}
\end{multline*}

\noindent
\begin{multline*}
{}=
\fr{2}{193}\sqrt{1{,}94I(N-2)} \left(
\left(
\vphantom{\sqrt{1-\left(\fr{\sqrt{I}}{193} -\sin\beta\right)^2}}
\cos\beta - \left(\fr{\sqrt{I}}{193}\right)\right.\right.\times{}\\
\left.\left.{}\times\ctg (\alpha+\beta)+
\sqrt{1-\left(\fr{\sqrt{I}}{193}-\sin\beta\right)^2}\,\right) \times{}\right.\\
\left.{}\times
\left(2\sin (\alpha+\beta)\cos\alpha -
\fr{\sqrt{I}}{193}\right)+\{\theta -\sin(\theta)\}
\vphantom{\sqrt{1-\left(\fr{\sqrt{I}}{193} -\sin\beta\right)^2}}
\right)^{-1/2}\!\!.\hspace*{-7.20935pt}
%\label{e17-zat}
\end{multline*}

Для системы неравенств~(\ref{e5-zat}) минимальная ширина экрана может быть
посчитана по формуле:
\begin{multline*}
W_{\min} ={}\\
{}= \fr{2}{193}\!\left(\!
1{,}94I(N-2)\!\Big/\!\!
\left( 2\arcsin \sqrt{1-(C-\sin\beta)^2}-{}\right.\right.\\
\left.\left.{}-2(C-\sin\beta ) \sqrt{1-(C-\sin\beta)^2}\right)
\vphantom{\Big/}
\right)^{1/2}\,.
%\label{e18-zat}
\end{multline*}


     Найденная по этим формулам ширина~--- это расстояние между
крайними точками экрана. Таким образом, действительная ширина экрана
с~учетом его кривизны может быть рассчитана умножением на коэффициент
$\beta/\sin\beta$:
     \begin{equation*}
     W_{\mathrm{экр}} = W_{\min} \fr{\beta}{\sin\beta}\,.
%     \label{e19-zat}
     \end{equation*}
Здесь при $\beta\hm=0$ используем предел
$$
\lim\limits_{\beta\to 0} \left(\fr{\beta}{\sin\beta}\right)=1\,,
$$
т.\,е.\ функция ширины экрана непрерывна и~в~окрестности
точки $\beta\hm=0$.



     На рис.~3 показан график зависимости ширины экрана от углов~$\alpha$
и~$\beta$. Основные потребности в~большой ширине экрана начинаются в~тот
момент, когда экран становится выпуклым в~сторону наблюдателей,
а~в~остальном для 10~наблюдателей понадобится экран шириной около
     1,5--2~м в~зависимости от угла наблюдения~$\alpha$.


\vspace*{-6pt}

\section{Заключение}

     Полученные в~статье результаты являются развитием методических
подходов, предложенных в~[9, 10], применительно к~изогнутому
экрану за счет введения нового параметра~--- угла кривизны~$\beta$, который
для случая полиэкрана представляет собой угол между двумя его дисплеями.
Важно, что случай $\beta\hm=0$ полностью отвечает результатам, полученным
в~литературе для плоских экранов.

     Предложен общий подход к~расчету максимального количества рабочих
мест на основании формулы , формализация которой для разных расположений
может позволить использовать полученные в~данной статье готовые формулы.

     Показано, что использование изогнутых экранов в~качестве
коллективных может увеличить количество рабочих мест с~соблюдением
эргономических требований за счет увеличения площади и~пери\-мет\-ра ОКН.

     Использование изогнутых экранов может быть особенно эффективным
в~условиях жестких требований к~ситуационному залу (например, маленький
допустимый угол наблюдения, большая информационная емкость контента или
большое количество наблюдателей). Именно в~таких случаях эффект от
применения изогнутых экранов (увеличение числа рабочих мест) может
оказаться наибольшим за счет расположения дисплеев полиэкрана под
определенным углом друг к~другу.

     Полученные зависимости позволяют оценить максимальную
дополнительную стоимость изогнутого экрана относительно плоского такой же
ширины при реализации проектов с~коллективным экраном.

\vspace*{-6pt}

{\small\frenchspacing
 {%\baselineskip=10.8pt
 \addcontentsline{toc}{section}{References}
 \begin{thebibliography}{99}
\bibitem{2-zat} %1
\Au{Зацаринный А.\,А., Сучков А.\,В., Босов~А.\,В.} Ситуационные центры в~современных
ин\-фор\-ма\-ци\-он\-но-те\-леком\-му\-ни\-ка\-ци\-он\-ных сис\-те\-мах специального
назначения~// Ведомственные корпоративные сети и~систе\-мы (ВКСС Connect!), 2007.
№\,5(44). С.~64--76.

\bibitem{4-zat} %2
\Au{Зацаринный А.\,А. Шабанов А.\,П.} Исследование и~разработка методического
обеспечения и~технологических решений по управлению производительностью
конт\-роль\-но-тех\-но\-ло\-ги\-че\-ских трактов~// Информационные технологии в~науке,
социологии, экономике и~бизнесе (IT\;+\;S\&E'10): Мат-лы XXXVII Междунар. конф.~//
Открытое образование, 2010. №\,6. Приложение. С.~44--45.

\bibitem{3-zat}
\Au{Зацаринный А.\,А., Шабанов А.\,П.} Ситуационные центры:
ин\-фор\-ма\-ция--про\-цес\-сы--ор\-га\-ни\-за\-ция~// Электросвязь, 2011. №\,6. С.~42--46.

\bibitem{1-zat} %4
\Au{Зацаринный А.\,А., Козлов С.\,В., Сучков~А.\,П.} Особенности проектирования
и~функционирования ситуационных цент\-ров~// Системы высокой доступности, 2012.
Т.~8. №\,1. С.~12--21.



\bibitem{5-zat} %5
\Au{Зацаринный А.\,А.} Организационные принципы сис\-тем\-но\-го подхода к разработке,
проектированию и~внедрению современных
ин\-фор\-ма\-ци\-он\-но-те\-ле\-ком\-му\-ни\-ка\-ци\-он\-ных сетей~// Ведомственные
корпоративные сети и~системы (ВКСС Connect!), 2007. №\,1(40). С.~60--67.
\bibitem{6-zat}
\Au{Зацаринный А.\,А., Шабанов~А.\,П.} Эффективность ситуационных центров
и~человеческий фактор~// Вестник Московского ун-та имени С.\,Ю.~Витте. Сер.~1:
Экономика и~управление, 2013. №\,3. С.~43--53.
\bibitem{7-zat}
\Au{Зацаринный А.\,А., Ионенков Ю.\,С., Шабанов~А.\,П.} Методиче\-ский подход к оценке
эффективности ситуационных центров~// Фундаментальные и~прикладные
исследования, разработка и~применение высоких технологий в~промышленности
и~экономике: Сб. статей 15-й Междунар. науч.-практич. конф.~--- СПб.: СПбГТУ,
2013. Т.~2. С.~37--39.
\bibitem{8-zat}
\Au{Зацаринный А.\,А., Шабанов А.\,П.} Системные аспекты эффективности
ситуационных центров~// Вестник Московского ун-та имени С.\,Ю.~Витте. Сер.~1:
Экономика и~управление, 2013. №\,2. С.~110--123.
\bibitem{9-zat}
\Au{Чупраков К.\,Г.} Исследование и~разработка методов построения систем
отображения информации для ситуационного центра. Дисс.\ \ldots\ канд. техн. наук.~---
М.: ИПИ РАН, 2010. 214~с.
\bibitem{10-zat}
\Au{Чупраков К.\,Г.} К~вопросу о~размещении коллективных средств отображения
в~ситуационном зале с~заданными параметрами~// Информатика и~её применения, 2010.
Т.~4. Вып.~4. С.~89--96.
\bibitem{11-zat}
\Au{Золотов Е.} Кривое ТВ: кому выгодно гнуть телевизор. {\sf
http://www.computerra.ru/100617/curved-tv}.
\bibitem{12-zat}
\Au{Новикова Е.\,В., Переверзев Б.\,Л., Лавренюк~С.\,Ю.} Метод расчета зоны
оптимальной видимости при работе с~экранами коллективного пользования~//
Информационные технологии в~проектировании и~производстве, 2011. №\,3.
С.~104--109.

\bibitem{19-zat} %13
ГОСТ 21958-76. Зал и~кабины операторов. Взаимное расположение рабочих мест.~--- М.:
Изд-во стандартов, 1976. 7~с.
\bibitem{16-zat} %14
ГОСТ 12.2.032-78. Система стандартов безопасности труда. Рабочее место при
выполнении работ сидя. Общие эргономические требования.~--- М.: Изд-во стандартов,
2001. 9~с.
\bibitem{17-zat} %15
ГОСТ 12.2.033-78. Система стандартов безопасности труда. Рабочее место при
выполнении работ стоя. Общие эргономические требования.~---М.: Изд-во стандартов,
2001. 9~с.

\bibitem{14-zat} %16
ГОСТ 26387-84. Система <<Че\-ло\-век--ма\-ши\-на>>. Термины и~определения.~--- М.:
Стандартинформ, 2006. 6~с.
\bibitem{15-zat} %17
ГОСТ 27833-88. Средства отображения информации. Термины и~определения.~--- М.:
Изд-во стандартов, 1988. 11~с.

\bibitem{18-zat} %18
ГОСТ Р ИСО 9241-3-2003. Эргономические требования при выполнении офисных работ
с~использованием видеодисплейных терминалов.~--- М.: Изд-во стандартов, 2003. 39~с.

\bibitem{13-zat} %19
ГОСТ Р 52324-2005. (ИСО 13406-2:2001). Эргономические требования к работе
с~визуальными дисплеями, основанными на плоских панелях.~--- М.: Стандартинформ,
2005. 13~с.

 \end{thebibliography}

 }
 }

\end{multicols}

\vspace*{-6pt}

\hfill{\small\textit{Поступила в редакцию 12.10.14}}

%\newpage

\vspace*{12pt}

\hrule

\vspace*{2pt}

\hrule

%\vspace*{12pt}

\def\tit{REGARDING ERGONOMIC DEPENDENCES BETWEEN~SITUATIONAL~HALL PARAMETERS
USING~COLLECTIVE~CURVED~SCREEN}

\def\titkol{Regarding ergonomic dependences between
situational hall parameters using collective curved
screen}

\def\aut{A.\,A.~Zatsarinnyy and~K.\,G.~Сhuprakov}

\def\autkol{A.\,A.~Zatsarinnyy and~K.\,G.~Сhuprakov}

\titel{\tit}{\aut}{\autkol}{\titkol}

\vspace*{-9pt}

\noindent
Institute of Informatics Problems, Russian Academy of Sciences,
44-2~Vavilov Str., Moscow 119333, Russian Federation


\def\leftfootline{\small{\textbf{\thepage}
\hfill INFORMATIKA I EE PRIMENENIYA~--- INFORMATICS AND
APPLICATIONS\ \ \ 2014\ \ \ volume~8\ \ \ issue\ 4}
}%
 \def\rightfootline{\small{INFORMATIKA I EE PRIMENENIYA~---
INFORMATICS AND APPLICATIONS\ \ \ 2014\ \ \ volume~8\ \ \ issue\ 4
\hfill \textbf{\thepage}}}

\vspace*{3pt}

\Abste{The paper presents an approach to determining dependences between
such parameters of a~situational hall as measurements of the hall,
quantity of people working with the screen,
information capacity of the content (the quantity of symbols),
 and screen width. These dependences make it possible to calculate an unknown
 parameter of a~situational hall using known parameters satisfying requirements
 of the Russian and International ergonomic standards. The presented formulas
 are applicable to the case of curved screens by using the angle of curvature~$\beta$
 (for a flat screen, $\beta=0$). This parameter may be interpreted as an angle
 between displays in a~polyscreen or a~videowall. This parameter makes it possible
 to evaluate the efficiency of curved screens as a collective screen compared
 to the flat screens. The paper also suggests an approach to estimating the
 quantity of workplaces that may be used for their different interpositions.}

\KWE{collective curved screen; situational hall; dispatch room; ergonomic
dependences; comfort observation area; curve angle; videowall, polyscreen; efficiency; price
justification}


\DOI{10.14357/19922264140411}

%\vspace*{3pt}

  \begin{multicols}{2}

\renewcommand{\bibname}{\protect\rmfamily References}
%\renewcommand{\bibname}{\large\protect\rm References}



{\small\frenchspacing
 {%\baselineskip=10.8pt
 \addcontentsline{toc}{section}{References}
 \begin{thebibliography}{99}

\bibitem{2-zat-1} %1
\Aue{Zatsarinnyy, A.\,A., A.\,V.~Suchkov, and A.\,V.~Bosov}. 2007. Situatsionnye tsentry
v~sovremennykh informatsionno-telekommunikatsionnykh sistemakh spetsial'nogo
naznacheniya [Situational centers in modern information-telecommunicational network of
special purposes]. \textit{VKSS Connect! (Vedomstvennye korporativnye seti i~sistemy)}
[VKSS Connect! (Departmental Corporate Networks and Systems] 5(44):64--76.

\bibitem{4-zat-1} %2
\Aue{Zatsarinnyy, A.\,A., and A.\,P.~Shabanov} 2010. Issledovanie i razrabotka metodicheskogo
obespecheniya i tekhnologicheskikh resheniy po upravleniyu proizvoditel'nost'yu
kontrol'no-tekhnologicheskikh traktov [Investigation and development of methodical base and
technologic solutions concerning the
 management of control and technologic tract performance].
\textit{Prilozhenie k zhurnalu ``Otkrytoe obrazovanie.'' Mat-ly XXXVII Mezhdunar. konf.
i~diskussionnogo nauchnogo kluba ``Informatsionnye Tekhnologii v~Nauke, Sotsiologii,
Ekonomike i~Biznese'' IT\;+\;SE'10} [Appendix to magazine ``Open Education.'' 37th
Conference (International) and Discussion Club ``Informational Technologies in Science, Education,
Telecommunications, and Business'' Proceedings].  Yalta. 44--45.
\bibitem{3-zat-1} %3
\Aue{Zatsarinnyy, A.\,A., and A.\,P.~Shabanov}. 2011. Situatsionnye centry:
Informatsiya--protsessy--organizatsiya. [Situational centers:
Information--procecces--organization].
Electrosvyaz' [Telecommunications] 6:42--46.
\bibitem{1-zat-1} %4
\Aue{Zatsarinnyy, A.\,A., S.\,V.~Kozlov, and A.\,P.~Suchkov}. 2012. Osobennosti
proektirovaniya i~funktsionirovaniya situ\-atsi\-on\-nykh tsentrov [Peculiarity of situational centers
design and operation]. \textit{Sistemy Vysokoy Dostupnosti} [Systems of High Accessability].
Moscow: Radiotechnika Publ. 8(1):12--21.

\bibitem{5-zat-1} %5
\Aue{Zatsarinnyy, A.\,A.} 2007. Organizatsionnye printsipy sistemnogo podkhoda k razrabotke,
proektirovaniyu i~vnedreniyu sovremennykh in\-for\-ma\-tsi\-on\-no-te\-le\-kom\-mu\-ni\-ka\-tsi\-on\-nykh setey
[Organizational principles of systemic approach to development, design, and implementation of
modern information-telecommunicational networks]. \textit{VKSS Connect! (Vedomstvennye
korporativnye seti i~sistemy)} [VKSS Connect! (Departmental Corporate Networks and
Systems] 1(40):60--67.

\bibitem{6-zat-1}
\Aue{Zatsarinnyy, A.\,A., and A.\,P.~Shabanov} 2013. Effektivnost' situatsionnykh tsentrov
i~chelovecheskiy faktor [Situational centers efficiency and the human factor]. \textit{Vestnik
Moskovskogo Un-ta imeni S.\,Yu.~Vitte. Ser.~1: Ekonomika i~Upravlenie} [Herald of
S.\,Y.~Vitte Moscow University. Ch.~1: Economics and Management]  3:43--53.

\bibitem{7-zat-1}
\Aue{Zatsarinnyy, A.\,A., Yu.\,S.~Ionenkov, and A.\,P.~Shabanov}. 2013. Metodicheskiy
podkhod k otsenke effektivnosti\linebreak si\-tu\-a\-tsi\-onnykh tsentrov [Methodical approach to situational
centers efficiency evaluation]. \textit{Sb. statey 15-y Mezhdunar. nauch.-praktich. konf.
``Fundamental'nye i~Prikladnye Issledovaniya, Razrabotka i~Primenenie Vysokikh Tekhnologiy
v~Promyshlennosti i~Ekonomike''} [15th Scientific and Practical Conference
(International) ``Fundamental and Applied Investigations, Development
and Applications of Fine Technologies in Industry and Economics Proceedings].
Ed.\ A.\,P.~Kudinova. St.\ Petersburg, Russia:
Polytechnical University Publ. 2(1):37--39.
\bibitem{8-zat-1}
\Aue{Zatsarinnyy, A.\,A., and A.\,P.~Shabanov}. 2013. Sistemnye aspekty effektivnosti
situatsionnykh tsentrov [Systemic peculiarities of situational centers efficiency]. \textit{Vestnik
Moskovskogo Un-ta imeni S.\,Yu.~Vitte} [Herald of~S.\,Yu.~Vitte Moscow University.
Chapter~1: Economics and Management]. 2:110--123.
\bibitem{9-zat-1}
\Aue{Chuprakov, K.\,G.} 2010. Issledovanie i razrabotka metodov postroeniya sistem
otobrazheniya informatsii dlya si\-tu\-a\-tsi\-on\-no\-go tsentra [Investigations and development of
methods of visualization systems creation for situational center]. PhD. Thesis. Moscow. 214~p.
\bibitem{10-zat-1}
\Aue{Chuprakov, K.\,G.} 2010. K~voprosu o razmeshchenii kollektivnykh sredstv
otobrazheniya v~situatsionnom zale s~zadannymi parametrami [On collective
display facilities placed in a~situational hall with prescribed parameters].
\textit{Informatika i ee Primeneniya}~--- \textit{Inform. Appl.} 4(4):89--96.
\bibitem{11-zat-1}
\Aue{Zolotov, E.} 2014. {Krivoe TV: Komu vygodno gnut' televizor} [Curved TV: Who
benefit from bending TV]. Available at: {\sf
http://www.computerra.ru/100617/curved-tv/} (accessed August~27, 2014).
\bibitem{12-zat-1}
\Aue{Novikova, E.\,V., B.\,L.~Pereverzev, and S.\,Yu.~Lavrenyuk}. 2011. Metod rascheta zony
optimal'noy vidimosti pri rabote s~ekranami kollektivnogo pol'zovaniya [A~calculation method
for area of optimal visibility during work with collective screen].
\textit{Informatsionnye Tekhnologii v~Proektirovanii i~proizvodstve} [Informational
Technologies in Design and Industry] 3:104--109.

\bibitem{19-zat-1} %13
GOST 21958-76. 1976. \textit{Zal i~kabiny operatorov. Vzaimnoe raspolozhenie rabochikh
mest} [Hall and operator's cabins. Mutual disposition of workplaces]. Moscow:
Standard Pubs. 7~p.
\bibitem{16-zat-1} %14
GOST 12.2.032-78. 2001. \textit{Sistema standartov bezopasnosti truda. Rabochee mesto pri
vypolnenii rabot sidya. Obshchie ergonomicheskie trebovaniya} [A~system of work safety
standards. Workplace for sitting operations. Main ergonomic requirements]. Moscow:
Standard Publs. 9~p.
\bibitem{17-zat-1} %15
GOST 12.2.033-78. 2001. \textit{Sistema standartov bezopasnosti truda. Rabochee mesto pri
vypolnenii rabot stoya. Obshchie ergonomicheskie trebovaniya} [A~system of work safety
standards. Workplace for standing operations. Main ergonomic requirements]. Moscow:
Standard Publs. 9~p.

\bibitem{14-zat-1} %16
GOST 26387-84. 2006. \textit{Sistema ``Chelovek--mashina.'' Terminy i~opredeleniya}
[A~system ``human-machine.'' Terms and definitions]. Moscow: StandardInform Publ. 6~p.

\bibitem{15-zat-1} %17
GOST 27833-88. 1990. \textit{Sredstva otobrazheniya informatsii.
Terminy i~opredeleniya}
[Information visualization means. Terms and definitions]. Moscow: Standards Publs. 11~p.
\bibitem{18-zat-1} %18
GOST R ISO 9241-3-2003. 2003. \textit{Ergonomicheskie trebovaniya pri vypolnenii ofisnykh rabot
s~ispol'zovaniem videodispleynykh terminalov} [Ergonomic requirement during office works
operations using videodisplay terminals]. Moscow: Standard Publs. 39~p.


\bibitem{13-zat-1} %19
GOST R 52324-2005 (ISO 13406-2:2001). 2005. \textit{Ergonomicheskie trebovaniya k~rabote
s~vizual'nymi displeyami, osnovannymi na ploskikh panelyakh} [Ergonomic requirements for
work with visual displays based on flat panels]. Moscow: Standardinform Publ. 13~p.

\end{thebibliography}

 }
 }

\end{multicols}

\vspace*{-6pt}

\hfill{\small\textit{Received October 12, 2014}}

\vspace*{-18pt}

\Contr

\noindent
\textbf{Zatsarinnyy Alexander A.}\ (b.\ 1951)~---
Doctor of Science in technology, professor, Deputy Director,
Institute of Informatics Problems, Russian Academy of Sciences,
44-2 Vavilov Str., Moscow 119333, Russian Federation; azatsarinny@ipiran.ru

\vspace*{3pt}

\noindent
\textbf{Сhuprakov Konstantin G.}\ (b.\ 1985)~---
Candidate of Sciences (PhD) in technology, leading mathematician,
Institute of Informatics Problems, Russian Academy of Sciences,
44-2 Vavilov Str., Moscow 119333, Russian Federation;  chkos@rambler.ru
\label{end\stat}

\renewcommand{\bibname}{\protect\rm Литература}