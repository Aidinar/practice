\def\vo{\;\mathop{\to}\limits_{r}\;}
\def\eam{\mathbin{{\mathop{=}\limits^{\mathrm{def}}}}}


\def\stat{mironov}

\def\tit{МЕТОД ПОВЫШЕНИЯ ЭФФЕКТИВНОСТИ РЕШЕНИЯ
ЗАДАЧ  ВЕРОЯТНОСТНОЙ ВЕРИФИКАЦИИ ВЫЧИСЛИТЕЛЬНЫХ И~ТЕЛЕКОММУНИКАЦИОННЫХ СИСТЕМ$^*$}



\def\titkol{Метод повышения эффективности решения
задач  вероятностной верификации вычислительных %и телекоммуникационных
систем}

\def\aut{А.\,М.~Миронов$^1$, С.\,Л.\,Френкель$^2$}

\def\autkol{А.\,М.~Миронов, С.\,Л.\,Френкель}

\titel{\tit}{\aut}{\autkol}{\titkol}

{\renewcommand{\thefootnote}{\fnsymbol{footnote}} \footnotetext[1]
{Работа выполнена при частичной поддержке РФФИ (проект 12-07-00109).}}


\renewcommand{\thefootnote}{\arabic{footnote}}
\footnotetext[1]{Институт проблем информатики
Российской академии наук,  amironov66@gmail.com}
\footnotetext[2]{Институт проблем информатики Российской
академии наук; Московский государственный технический университет
радиотехники, электроники и автоматики (МГТУ МИРЭА), fsergei@mail.ru}

\vspace*{6pt}


\Abst{Рассматривается проблема снижения трудоемкости
вероятностной верификации при проектировании вычислительных систем.
Поставленная
цель достигается редукцией вероятностных систем переходов
(ВСП), моделирующих проектируемые системы.
Верификация ВСП заключается в~вычислении истинностных значений
формул вероятностной темпоральной логики (PCTL, Probabilistic
Com\-pu\-ta\-ti\-o\-nal Tree Logic) в~начальных состояниях ВСП.
Редукция ВСП  выполняется  по алгоритму удаления эквивалентных
состояний, в~результате работы которого получается такая ВСП,
у~которой все свойства, выражаемые формулами логики PCTL, совпадают со
свойствами исходной ВСП.}

\KW{верификация; вероятностные системы переходов; %цепи Маркова;
вероятностная темпоральная логика;
редукция вероятностных моделей}

\DOI{10.14357/19922264140408}

\vspace*{6pt}


\vskip 12pt plus 9pt minus 6pt

\thispagestyle{headings}

\begin{multicols}{2}

\label{st\stat}


\section{Введение}

\subsection{Постановка задачи}


Необходимость в~вероятностной верификации проектов цифровых систем
возникает либо при проектировании систем со стохастическим
поведением, например многоканальных телекоммуникационных систем,
либо в~случаях,  когда у~разработ\-чика есть основание полагать, что
проектируемая\linebreak система в~рабочем режиме может быть подвержена
различным не специфицированным при проектировании ошибкам
и~случайным сбоям как внутрен\-ней природы, так и~инициированных
внешними воздействиями. Поскольку точно локализация и~функциональные
последствия наличия таких\linebreak ошибок априори не известны, их можно
попытаться характеризовать вероятностью проявления в~результатах
работы и,~соответственно, говорить\linebreak о~вероятностной верификации.
Наиболее распространенным подходом к формальной вероятностной
верификации является Probabilistic Model Checking~[1,
гл.~11], дополняющий проверку соот\-ветствия формальной спецификации
про\-ек\-ти\-ру\-емой системы ее свойствам (properties) вычис\-ле\-нием
вероятностей выполнения этих свойств.\linebreak В~данном случае проектируемые
системы описываются моделью ВСП, которая
используется в~алгоритмах формальной верификации, основанных на
проверке моделей (Мodel Сhecking). Одна из главных проблем
использования Probabilistic Model Checking, как и~прочих формальных
методов спецификации, состоит в~их вычислительной сложности,
и~поэтому снижение размера соответствующих моделей, в~частности матриц
тех или иных переходов, является важнейшим фактором реализуемости
соответствующих методов.

В настоящей работе рассматривается задача редукции ВСП,
целью которой является понижение сложности
верификации свойств ВСП, выражаемых формулами вероятностной
темпоральной логики PCTL. Вероятностные системы переходов
представляют собой один из наиболее
широко используемых классов моделей дискретных динамических систем.
Понятие ВСП является обобщением понятия цепи Маркова~\cite{markov},
которое нашло широкое применение в~естественных  и~гуманитарных
науках. Понятие ВСП можно рассматривать также как\linebreak частный случай
понятия вероятностного автомата~\cite{buh}. Главной отличительной
особенностью понятия ВСП от понятий цепи Маркова и~вероятностного
автомата является наличие выразительного\linebreak логическо\-го формализма,
позволяющего эффективно описывать различные свойства поведения ВСП.
В качестве такого формализма выступает вероятностная темпоральная
логика PCTL~[4, 5], которая представляет собой вероятностный аналог
темпоральной логики ветвящегося времени CTL~\cite{peled},
использующейся для спецификации свойств параллельных и~распределенных программ,
и~является эффективным инструментом для
описания различных свойств дискретных вероятностных динамических
систем.

Формулы логики PCTL могут отражать различные вероятностные аспекты
поведения анализируемых систем, к~числу которых относятся, например,
частота выполнения тех или иных действий или переходов в~анализируемых системах,
вероятность отказа компонентов анализируемых
систем, вероятностный характер взаимодействия анализируемой системы
с ее окружением, например: час\-то\-та поступления входных запросов или
сообщений, частота получения искаженных сообщений (для протоколов
передачи сообщений в~компьютерных сетях) и~т.\,п.


В данной работе  уточняются основные формулировки  и~демонстрируется
на новом примере  решение задачи, сформулированной в~\cite{mf},
а~именно:  преобразование ВСП проектируемой системы в~эквивалентную
ВСП с меньшим числом со\-сто\-яний. Под эквивалентностью понимается, что
результаты верификации исходной и~редуцированной модели будут
одинаковы.

Некоторые подходы к~редукции ВСП изучались в~различных работах по
вероятностной верификации, однако в~этих исследованиях были
рас\-смот\-ре\-ны лишь частные методы редукции ВСП, такие как редукция
частичных порядков~[8, 9] и~редукция, основанная на понятии
симметрии множества состояний ВСП~[10, 11]. Данные методы можно
эффективно использовать лишь для ВСП достаточно специального вида:
как правило, это вероятностные модели параллельных и~распределенных
программ.


\subsection{Современное состояние проблемы вероятностной верификации}


Первые алгоритмы вероятностной верификации были предложены в~1980-е~гг.\
в~работах~[12--14]. Данные алгоритмы были предназначены для
верификации качественных вероятностных свойств (т.\,е.\ таких,
которые выполняются с~вероятностью~1 или~0). Затем эти алгоритмы
были обобщены на случай верификации количественных вероятностных
свойств (в~спецификации таких свойств могло присутствовать любое
значение вероятности). Эти алгоритмы были изложены в~работах~[4, 15, 16].
Программные реализации этих алгоритмов были представлены в~работах [17, 18].

Первые промышленные системы вероятностной верификации были
разработаны в~2000-х~гг.~[19, 20]. Эти системы  успешно
применяются во многих областях, таких как анализ распределенных
алгоритмов, телекоммуникационные протоколы, компьютерная
безопасность, криптографические протоколы, моделирование
биологических процессов. С~использованием этих систем верификации
были обнаружены уязвимости и~аномальные поведения анализируемых
систем (подробнее см.\ в~\cite{50}). При помощи систем вероятностной
верификации могут быть вычислены такие характеристики про\-грам\-мных
систем, как, например, вероятность вторжения злоумышленника
в~компьютерную сеть, математическое ожидание времени отклика
веб-сер\-ви\-са и~другие количественные и~качественные характеристики.

Наиболее популярной практической системой вероятностной верификации
в настоящее время является система PRISM~\cite{55},  разработанная
на факультете компьютерных наук Оксфордского университета
(Великобритания) в~группе Quan\-ti\-ta\-ti\-ve Ana\-ly\-sis and
Ve\-ri\-fi\-ca\-ti\-on под руководством Марты Квятковской.
Информация о~деятельности этой группы представлена на веб-сайте
{\sf http://qav.comlab.ox.ac.uk/}.

\section{Вероятностные системы переходов}
\label{fdgfdsgsdfgsdfgrr4444}

\subsection{Понятие вероятностной системы переходов}

Предположим, что задано конечное множество~AP, элементы которого
называются {\bf атомарными утверждениями}.
Ниже запись $2^{\mathrm{AP}}$ обозначает множество всех подмножеств~AP.

{\bf Вероятностная система переходов}
(называемая также в англоязычной литературе {\bf Discrete Time Markov Chain})~---
это четверка $D$ вида
\begin{equation*} %{dfsgdsfgf33dsgdsgfds}
D=(S, s^0, P, L)\,,
\end{equation*}
компоненты которой имеют следующий смысл:
\begin{enumerate}[(1)]
\item  $S$~---  множество,    элементы которого называются
    {\bf состояниями} ВСП~$D$;
\item  $s^0 \in S$~--- выделенное состояние, называемое
 {\bf начальным состоянием} ВСП~$D$;
 \item  $P$~--- функция вида $P:S\times S\hm\to [0,1]$,
   на\-зы\-ва\-емая {\bf функцией перехода} ВСП~$D$ и~ удовлетворяющая условию:
  $\forall\, s\hm\in S\quad
  \sum\limits_{s'\in S}P(s,s')\hm=1.$
Для каждой пары $(s_1,s_2)\hm \in S\times S$
число $P(s_1,s_2)$ понимается как вероятность
того, что если в~текущий момент времени~$D$ находится в~состоянии~$s_1$,
то через один такт времени~$D$ будет находиться
в~состоянии~$s_2$.
Если $P(s_1,s_2)\hm>0$, то будем называть тройку
$(s_1, s_2, P(s_1,s_2))$ {\bf переходом} из~$s_1$ в~$s_2$
с~вероятностью $P(s_1,s_2)$. Ниже запись $s_1\ra{a}s_2$ является другим обозначением
перехода $(s_1, s_2, a)$;\\[-14pt]

\item $L$~--- функция вида
$L:S\to 2^{\mathrm{AP}}$, называемая {\bf оценкой}, которая имеет следующий смысл:
   для каждого состояния    $s\hm\in S$ и~каждого атомарного утверждения
   $p\hm\in \mathrm{AP}$ утверждение~$p$  считается
{\bf истинным} в~ $s$, если $p\hm\in L(s)$, и~{\bf ложным} в~$s$,
если $p\hm\not\in L(s)$.
\end{enumerate}


Вероятностную систему переходов удобно рассматривать как помеченный граф, вершинами которого
являются состояния, помеченные элементами множества $2^{\mathrm{AP}}$: каждая
вершина $s\hm\in S$ имеет метку $L(s)$ и~для каждой пары $(s_1,s_2)\hm\in
S\times S$ такой, что  $P(s_1, s_2) \hm>0$, граф содержит
ребро из~$s_1$ в~$s_2$ с~меткой $P(s_1, s_2)$.

\vspace*{-7pt}

\subsection{Случайные функции}

Пусть $X$ и~$Y$~--- два конечных множества.

{\bf Случайной функцией} (СФ) из~$X$ в~$Y$
называется произвольная функция~$f$ вида
\be{dfgfdsgfds}
f: X\times Y \to [0,1]
\ee
такая, что
$\forall\,x\hm\in X\quad \sum\limits_{y\in Y}f(x,y)\hm=1$.

Для любых $x\in X$ и~$y\hm\in Y$ значение $f(x,y)$ можно интерпретировать
как вероятность того, что СФ~$f$ отображает~$x$ в~$y$.

Случайная функция~\re{dfgfdsgfds} называется {\bf детерминированной},
если для каждого $x\hm\in X$ существует
единственный $y\hm\in Y$, такой что
$f(x,y)\hm=1$. Если~$f$~--- детерминированная СФ
вида~\re{dfgfdsgfds} и~$x,y$~--- такие элемен\-ты~$X$
и~$Y$ соответственно, что $f(x,y)\hm=1$,
то  будем говорить, что {\bf $f$ отображает~$x$ в~$y$}.

Если $f$~--- СФ из~$X$ в~$Y$, то  будем обозначать этот факт
записью $f: X\vo Y$. Будем называть~$X$ {\bf об\-ластью определения} СФ~$f$,
а~$Y$~--- {\bf областью значений} СФ~$f$.

Для каждого конечного множества~$X$ запись id$_X$ обозначает детерминированную
СФ $X\hm\to X$, которая отображает каждый $x\hm\in X$ в~$x$.

\vspace*{-7pt}

\subsection{Матрицы, соответствующие случайным функциям}

Если СФ $f$ имеет вид $f: X\vo Y$ и~на множествах~$X$ и~$Y$
заданы  упорядочения их элементов, которые имеют вид
$(x_1,\ldots, x_m)$ и $(y_1,\ldots, y_n)$
со-\linebreak\vspace*{-12pt}

\columnbreak

\noindent
 ответственно, то СФ~$f$ можно представить в~виде матрицы
(обозначаемой тем же символом~$f$)
\be{dfgdsghdsfgdsg}
f=\begin{pmatrix}
f(x_1,y_1)&\cdots&f(x_1,y_n)\\
\vdots&\cdots&\vdots\\
f(x_m,y_1)&\cdots&f(x_m,y_n)
\end{pmatrix}\,.
\ee

Ниже  будем отождествлять СФ~$f$ с~матрицей~\re{dfgdsghdsfgdsg}.

Будем предполагать, что для каждого конечного множества~$X$, являющегося
областью определения или областью значений ка\-кой-либо из
рассматриваемых СФ, на~$X$ задано фиксированное упорядочение
его элементов. Таким образом, для каждой рассматриваемой СФ
соответствующая  ей матрица определена однозначно.

Для каждой СФ $f:X\vo Y$ и произвольных $x\hm\in X$, $y\hm\in Y$
 будем называть
 \bi
\item строку $(f(x,y_1),\ldots, f(x,y_n))$
   матрицы~$f$~--- {\bf строкой $x$};
\item столбец $\begin{pmatrix}
f(x_1,y)\\
\vdots\\
f(x_m,y)\end{pmatrix}$    матрицы~$f$~--- {\bf столбцом~$y$}.
\ei


Если $f$ и~$g$~--- СФ вида $f: X\vo Y$, $g: Y\vo Z$, то
их {\bf композицией} называется СФ $f\cdot g: X\vo Z$, определяемая следующим
образом:
\be{dsfgdsfgdsf}
\forall\,x\in X\quad
(f\cdot g)(x)\eam \sum\limits_{y\in Y}f(x,y)\cdot
g(y,z)\,.
\ee

По определению произведения матриц из~\re{dsfgdsfgdsf} следует, что
матрица $f\cdot g$ является произведением матриц~$f$ и~$g$.

\vspace*{-6pt}

\subsection{Случайные функции, соответствующие вероятностным системам переходов}
\label{dfgdgdsf44555333}

Пусть задана ВСП $D\hm=(S,s^0, P, L)$ и список элементов множества~$S$
имеет вид $(s_1,\ldots, s_n)$.

Будем использовать  следующие обозначения.
\bn
\item Символ {\bf 1} означает  множество,
состоящее из одного элемента, который будем обозначать символом~$e$.
\item Для каждого состояния $s\hm\in S$ запись~$I_s$
обозначает детерминированную СФ вида
$I_s: {\bf 1} \vo S$, отображающую элемент $e\hm\in {\bf 1}$ в~состояние~$s$
ВСП~$D$.

\item Для каждого $n\hm\geq 0$
обозначим записью~$P^n$ СФ вида $P^n: S\vo S$, определяемую индуктивно:
$P^0\eam \mathrm{id}_S$ и~$\forall\,n\geq 0\enskip P^{n+1}\eam P^n\cdot P$.
Нетрудно видеть, что матрицы, соответствующие
СФ~ $P^i$, имеют следующий вид:
$P^0$~--- единичная матрица и~$\forall\,n\hm>0$
матрица~$P^n$  является $n$-й степенью матрицы~$P$.

Для любых $n\geq 0$, $s_1, s_2\hm\in S$ число $P^n(s_1,s_2)$
можно понимать как вероятность того, что если в~текущий
момент времени ВСП~$D$ находится в~состоянии~$s_1$,
то через~$n$~тактов времени $D$ будет находиться
в~состоянии~$s_2$.
\en

\vspace*{-9pt}

\section{Логика PCTL}

{\bf Логика PCTL}~--- это темпоральная логика,
предназначенная для формального описания свойств ВСП.
Логика PCTL была введена Х.~Ханссоном  и~Б.~Джонссоном  в~работе~\cite{35}.


\subsection{Формулы логики PCTL}

В~определении понятия формулы логики PCTL
 будем использовать множество~AP атомарных утверждений, введенное в~разд.~2.

Формулы логики PCTL делятся на два класса: StateFm~--- {\bf формулы
состояний}~--- и~PathFm~--- {\bf формулы путей}. Формулы из классов
StateFm и~PathFm  будем обозначать символами~$\varphi$
и~$\alpha$ соответственно (возможно, с индексами), а~формулу
произвольного вида~--- символом~$f$ (возможно, с~индексом).

Классы StateFm и~PathFm определяются следующим образом.

\smallskip

StateFm:

\smallskip

\bn
\item Каждое атомарное утверждение~$p$ из~AP является формулой из
StateFm.
\item Символы $\top$ и~$\bot$ является формулами
из  StateFm. Данные символы обозначают тож\-дественно истинное
и тождественно ложное утверждение соответственно.
\item Если $\varphi_1$ и~$\varphi_2$~--- формулы
из StateFm, то  следующие знакосочетания являются формулами из StateFm:
$
\neg \varphi_1$; $\varphi_1\wedge \varphi_2$;
$\varphi_1\vee \varphi_2$;
$\varphi_1\to \varphi_2$;
$\varphi_1\leftrightarrow \varphi_2$.
\item Если
\begin{itemize}
\item $\triangle$~--- функциональный символ, которому соответствует функция
(обозначаемая тем же символом) вида
$$\triangle: [0,1]\times [0,1]\to \{0,1\}\,;
$$
  \item $a$~--- число из $[0,1]$;

    \item $\alpha$~--- формула из PathFm,
\end{itemize}
то знакосочетание $\mathcal{P}_{\triangle a} \alpha$ является формулой из StateFm.
\en

PathFm:

\smallskip

\bn
\item Если $f$~--- формула логики PCTL, то  знакосочетание ${\bf X}f$
является формулой из PathFm.
\item Если $\varphi_1$ и~$\varphi_2$~--- формулы
из StateFm, то  следующие знакосочетания являются формулами из PathFm:
\begin{itemize}
\item[(а)] $\varphi_1{\bf U}^{\leq n}\varphi_2$,
где $n$~--- натуральное число;
\item[(б)] $\varphi_1{\bf U}\varphi_2$.
\end{itemize}
\item Если $\alpha$~--- формула из PathFm,
то знакосочетание $\neg \alpha$ является формулой
из PathFm.
\en

В записи формул из PathFm могут использоваться символы {\bf F} и~${\bf G}$,
которые являются сокращением знакосочетаний $\top{\bf U}$
и $\neg{\bf F} \neg$ соответственно (т.\,е., например, знакосочетания
 ${\bf F} \alpha$ и~${\bf G}^{\leq n}\alpha$
обозначают формулы
$\top{\bf U} \alpha$ и~$\neg {\bf F}^{\leq n}\neg \alpha$
соответственно).

\subsection{Значения формул логики PCTL
в~состояниях вероятностных систем переходов}
\label{dsfgdsfgdfgdfgdsf445}

Пусть $D=(S, s^0, P, L)$~--- некоторая ВСП.

Для каждого состояния $s\hm\in S$ и~каждой фо\-р\-му\-лы~$f$ логики PCTL
определено {\bf значение} формулы~$f$ в~состоянии~$s$, которое обозначается
записью $s(f)$,~и
\begin{enumerate}[(1)]
\item если $f\hm\in \mathrm{StateFm}$, то $s(f) \hm\in \{0,1\}$
   и
   \begin{itemize}
   \item в~случае $s(f)=1$  формула~$f$ считается
   истинной в~$s$;
   \item в~случае $s(f)=0$
   формула~$f$ считается  ложной в~$s$;
   \end{itemize}
\item если $f\in \mathrm{PathFm}$, то значение
$s(f)$ является числом из  $[0,1]$ и~интерпретируется
как вероятность того, что  формула~$f$ истинна в~состоянии~$s$.
\end{enumerate}

Для каждой формулы~$f$ логики PCTL
будем обозначать записью $S(f)$ век\-тор-столбец
$
\begin{pmatrix}
 s_1(f)\\
 \vdots\\
 s_n(f)
\end{pmatrix}.$

Значения формул логики PCTL в~состояниях ВСП определяются индукцией
по структуре формул в~соответствии с~излагаемыми ниже правилами.
В~одних из этих правил определяется значение $s(f)$, в~других~--- определяется
век\-тор-стол\-бец $S(f)$ целиком. В~этих определениях будем использовать следующие обозначения:
\begin{itemize}
\item для любых векторов $U\hm=\begin{pmatrix}
    u_1\\ \vdots\\u_n
\end{pmatrix}$, $V\hm=\begin{pmatrix}
    v_1\\ \vdots\\v_n
\end{pmatrix}$
из $[0,1]^n$ записи $\max(U,V)$ и~$U\circ V$    обозначают векторы
      $ \begin{pmatrix}
      \max(u_1,v_1)\\ \vdots\\ \max(u_n,v_n)
\end{pmatrix}$ и
$\begin{pmatrix} u_1\cdot v_1\\ \vdots\\ u_n \cdot v_n
\end{pmatrix}
$
соответственно.
   \item
   если $A$ и~$B$~--- матрицы порядков $n\times n$
   и~$n\hm\times 1$ соответственно с~компонентами из $[0,1]$,
   то запись $[A^*\cdot B]$ обозначает матрицу, получаемую
%\begin{itemize}
%\i
заменой
   всех ненулевых компонентов $A$ и~$B$ на~1 и
%   \i
вычислением $(\sum\limits_{i\geq 0}A^i)\cdot B$, где сложение
понимается как дизъюнкция (т.~е.\ $\sum\limits_{i\geq 0}A^i$
является конечной). %\end{itemize}
\end{itemize}

Правила определения значений формул логики PCTL
в~состояниях ВСП имеют следующий вид:
\bi
\item
Для каждого $p\hm\in \mathrm{AP}$

\vspace*{-2pt}

\noindent
$$
s(p) \eam \begin{cases}
1, &\ \mbox{если\ } p\in L(s)\,;\\
0 &\ \mbox{иначе}\,;\end{cases}
$$

\vspace*{-6pt}

\item $s(\top)\eam 1,\;\;s(\bot)\eam 0$;\\[-13pt]
\item $s(\neg f)\eam 1-s(f)$, $s(\varphi_1\wedge \varphi_2)\eam
s(\varphi_1) \cdot s(\varphi_2)$
и~т.\,д. (т.\,е.\ значения формул коммутируют с~булевыми операциями);\\[-13pt]
\item
$s(\mathcal{P}_{\triangle a} \alpha) \eam \triangle(s(\alpha),a)$;\\[-13pt]

\item $S({\bf X}f) \eam P\cdot S(f)$;\\[-13pt]

\item пусть $\alpha_n \hm= \varphi_1{\bf U}^{\leq n}\varphi_2$
(где $n\hm\geq 0$). Тогда

\vspace*{-8pt}

\noindent
\begin{gather*}
S(\alpha_0)\eam S(\varphi_2)\,;\\
\hspace*{-3.5mm}\forall\, n>0\ S(\alpha_n)\eam
\max\left( S(\varphi_2),\;
S(\varphi_1)\circ S({\bf X} \alpha_{n-1})
\right)\,.
\end{gather*}

\vspace*{-5pt}

\item пусть $\alpha \hm= \varphi_1{\bf U}\varphi_2$.
Тогда $S(\alpha)$ определяется системой линейных уравнений

\vspace*{-8pt}

\noindent
\begin{multline*}
S(\alpha) ={}\\
\hspace*{-12pt}{}=\max\left ( S(\varphi_2),
\left[P^*\cdot
 S(\varphi_2)\right] \circ S(\varphi_1) \circ (P\cdot S(\alpha))
\right).
\end{multline*}
\ei

\vspace*{-9pt}


\section{Метод редукции вероятностных систем переходов}

\vspace*{-6pt}

\subsection{Задача редукции вероятностных систем переходов}

Если анализируемая ВСП имеет большой размер, то анализ ее свойств,
выражаемых формулами логики PCTL (т.\,е.\ вычисление значений формул
логики PCTL в~состояниях этой ВСП), может быть
 связан
с~трудновыполнимыми требованиями к вычислительным ресурсам,
с~использованием которых производится этот анализ.
В~связи с~этим представляет большую актуальность проблема редукции
ВСП, т.\,е.\ удаления части состояний и~переходов анализируемой ВСП
с~таким расчетом, чтобы получившая ВСП была эквивалентна исходной
в~следующем смысле: для каждой формулы состояний~$f$ логики PCTL
формула~$f$ истинна в~начальном состоянии исходной ВСП тогда и~только тогда,
когда она истинная в~начальном состоянии редуцированной ВСП.

Основная идея предлагаемого в~настоящей работе
метода редукции ВСП основана на понятии эквивалентности состояний ВСП:
будем называть состояния эквивалентными, если значения
всех формул логики PCTL в~этих состояниях совпадают.
Алгоритм редукции ВСП представляет собой вы\-чис\-ле\-ние классов эквивалентности
состояний анализируемой ВСП и~удаление эквивалентных состояний.

\vspace*{-6pt}

\subsection{Эквивалентность  вероятностных систем переходов}

Пусть заданы две ВСП:
\be{sdfgdsgdsfgdsf}
D_i=(S_i,s_i^0,P_i,L_i)\enskip
(i=1,2)\,.
\ee

Будем называть состояния $s_1\hm\in S_1$ и $s_2\hm\in S_2$ {\bf эквивалентными},
если для каждой формулы~$f$ логики PCTL верно равенство $s_1(f) \hm=  s_2(f).$

Если состояния $s_1$ и~$s_2$ эквивалентны, то будем обозначать это записью $s_1\hm\sim s_2$.

Будем называть ВСП $D_1$ и~$D_2$ вида~\re{sdfgdsgdsfgdsf} {\bf эквивалентными},
если $s^0_1\hm\sim s^0_2$. Если ВСП~$D_1$ и~$D_2$ эквивалентны,
то будем обозначать этот факт записью $D_1\hm\sim D_2$.

Если ВСП $D_1$ и~$D_2$ совпадают и~$S$~--- множество их состояний, то
бинарное отношение на~$S$, состоящее из всех пар $(s_1, s_2)$
таких, что $s_1\hm\sim s_2$, является отношением эквивалентности.
Будем обозначать это отношение символом~$\sim$.

Отношение $\sim$ может быть найдено при помощи алгоритма, излагаемого
в~параграфе~5.3.

\vspace*{-4pt}

\subsection{Редукция вероятностных систем переходов}
%\label{dfgdfgdsfgdeeeee}

\vspace*{-2pt}

\subsubsection{Задача редукции вероятностных систем переходов}

Пусть задана ВСП $D\hm=(S, s^0, P,L)$.

Задача редукции ВСП~$D$ заключается в~построении ВСП~$D'$,
которая эквивалентна~$D$ и~чис\-ло состояний которой меньше, чем
чис\-ло состояний ВСП~$D$.

Излагаемый в~настоящем пункте алгоритм редукции ВСП является
вероятностным обобщением алгоритма редукции детерминированных
автоматов. Идея данного алгоритма основана на отож\-де\-ст\-вле\-нии
неразличимых состояний ВСП:
\begin{enumerate}[(1)]
\item алгоритм вычисляет классы $S_1, \ldots, S_k$ разбиения множества~$S$,
соответствующего эквивалентности~$\sim$;
\item  ВСП $D$ преобразуется путем удаления
состояний в~классах $S_1, \ldots, S_k$ (и~соответствующего
переопределения функции перехода) до тех пор, пока не останется по
одному состоянию в~каждом из этих классов.
\end{enumerate}
В результате этих удалений получается искомая ВСП~$D'$.

\subsubsection{Построение разбиения множества состояний редуцируемой вероятностной системы переходов}
\label{sdfagdsagdsfgdsfgds44455}

Разбиение множества~$S$ состояний ВСП $D\hm=(S,s^0,P,L)$,  соответствующее
отношению эквивалентности~$\sim$, вычисляется следующим образом:
\begin{enumerate}[(1)]
\item вычисляется разбиение~$\Sigma^0$, соответствующее отношению
 эквивалентности $\{ (s_1,s_2)\hm\in S\times S\mid
L(s_1)\hm=L(s_2) \}$;
\item затем работает цикл, состоящий из следующих шагов.

Пусть для некоторого $i\hm\geq 0$ определены
\bi
\item  отношение эквивалентности $\rho^i$;
\item соответствующее ему разбиение~$\Sigma^i$, которое состоит из классов
$S^i_1, \ldots, S^i_k$.
\ei
Обозначим записями $\Sigma^i_1, \ldots, \Sigma^i_k$ строки матрицы
$\pi^i$, соответствующей детерминированной СФ $\pi^i: S\hm\to
\Sigma^i$, и~$\varphi^{\Sigma^i}_1,\ldots, \varphi^{\Sigma^i}_k$~---
список формул таких, что $\forall\,j=1,\ldots, k \quad
S(\varphi^{\Sigma^i}_j) \hm= \Sigma^i_j$.

Определим отношение эквивалентности
$\rho^{i+1}$ на~$S$:
\begin{multline*}
%\label{fdsgdsgds44545}
\rho^{i+1} \eam \rho^{i} \cap \left\{
\vphantom{\varphi^{\Sigma^i}_j}
(s_1,s_2)\in S^2\mid \forall\,j=1,\ldots, k\right.\\
\left. s_1({\bf X} \varphi^{\Sigma^i}_j) = s_2({\bf X} \varphi^{\Sigma^i}_j)
\right \}\,.
\end{multline*}


Разбиение $\Sigma^{i+1}$, соответствующее отношению~$\rho^{i+1}$, можно построить
следующим образом:
\bi
\item вычисляются век\-тор-столб\-цы
\be{fgfsdgdfgfdsgrrr444}S({\bf X}
\varphi^{\Sigma^i}_j) = P\cdot \Sigma^i_j
\ee
(каждый из которых, как нетрудно видеть, является
суммой некоторых столбцов мат\-ри\-цы~$P$: для каждого
$j\hm=1,\ldots, k$ век\-тор-стол\-бец~\re{fgfsdgdfgfdsgrrr444}
 является суммой таких столбцов~$s$ матрицы~$P$, для которых
 $s\hm\in S_j$);
 \item классы разбиения~$\Sigma^{i+1}$ получаются путем измельчения классов
 разбиения~$\Sigma^{i}$: в~один и~тот же класс разбиения~$\Sigma^{i+1}$
попадают такие состояния, для которых со\-от\-вет\-ст\-ву\-ющие им компоненты
векторов~\re{fgfsdgdfgfdsgrrr444} совпадают для каждого $j\hm=1,\ldots,
k$.
\ei

Возможны два случая:
\begin{itemize}
\item[(а)] $\Sigma^{i+1}=\Sigma^{i}$.
В~этом случае искомое раз\-би\-ение~$\sim$ найдено:
оно совпадает с~$\Sigma^i$;

\item[(б)] $\Sigma^i \neq \Sigma^{i+1}$.
   В~этом случае увеличиваем~$i$ на~1 и~возвращаемся в~начало цикла
   (т.\,е.\ выполняем шаг~2 с увеличенным значением~$i$).

Нетрудно видеть, что таких возвращений может быть не больше
числа элементов множества~$S$ (так как разбиение~$\Sigma^{i+1}$
является измельчением разбиения~$\Sigma^{i}$).
\end{itemize}
\end{enumerate}



\subsubsection{Удаление эквивалентных состояний из~вероятностных систем переходов}

Пусть ВСП $D\hm=(S,s^0,P,L)$ содержит пару эквивалентных состояний~$s_1$,
$s_2$, где $s_1\hm\neq s^0$. Определим ВСП
\be{fgdsgsdfrestettyt}
D_1 \eam (S_1,s^0,P_1,L_1)\,,
\ee
где $S_1\eam S\setminus \{s_1\}$,
$\forall\,s,s'\hm\in S_1$;
$$
P_1(s, s')\eam \begin{cases}
P(s, s')+P(s, s_1),&\ \mbox{если }
s'=s_2\,;\\
P(s, s'),&\ \mbox{если }
s'\neq s_2,
\end{cases}
$$
$\forall\,s\in S_1$ $L_1(s)\eam L(s)$.

Таким образом,  матрица~$P_1$ получается из  мат\-ри\-цы~$P$
 прибавлением к~столбцу~$s_2$ столбца~$s_1$ и~удалением строки~$s_1$
 и~столбца~$s_1$, а~матрица~$L_1$ получается из  матрицы~$L$
удалением строки~$s_1$.

Будем говорить что ВСП~\re{fgdsgsdfrestettyt} получается из ВСП~$D$
путем {\bf удаления состояния~$s_1$, эквивалентного состоянию~$s_2$}.
По определению ВСП~\re{fgdsgsdfrestettyt} каждое ее
состояние является также и~состоянием ВСП~$D$.

Для каждого  $s\in S_1$ и~каждой формулы~$f$ логики PCTL
будем обозначать записями $s_{D}(f)$ и~$s_{D_1}(f)$
значения формулы~$f$ в~состоянии~$s$
в~ВСП~$D$ и~$D_1$ соответственно и~записями
$S_{D}(f)$ и~$S_{D_1}(f)$~--- век\-тор-столб\-цы значений формулы~$f$
в~состояниях ВСП~$D$ и~$D_1$ соответственно.

\smallskip

\noindent
\textbf{Теорема~1.}\
\textit{Пусть ВСП}~\re{fgdsgsdfrestettyt}
\textit{получается из ВСП~$D$ путем удаления состояния~$s_1$,
эквивалентного состоянию~$s_2$. Тогда}
$\forall\,s\hm\in S_1\enskip s_{D_1}(f) \hm= s_{D}(f).$

\subsubsection{Описание алгоритма редукции вероятностной системы переходов}

Теорема~1 является обоснованием
излагаемого ниже алгоритма редукции ВСП
$D\hm=(S,s^0,P,L)$. Этот алгоритм имеет следующий вид.
\begin{enumerate}[1.]
\item Вычисляется разбиение множества состояний ВСП~$D$,
соответствующее отношению эквивалентности $R\eam \sim$ (для этого
выполняются действия, изложенные в~п.~5.3.2).
\item Искомая ВСП~$D'$ строится путем
удаления состояний из ВСП~$D$ и~переопределения функции перехода и~отношения~$R$
следующим образом:
\begin{itemize}
\item[(а)] если
отношение~$R$ содержит пару $(s_1, s_2)$, такую что
$s_1\hm\neq s_2$ и~$s_2\hm\neq s^0$, то выберем произвольную такую
пару $(s_1, s_2)$ и~преобразуем компоненты ВСП~$D$ описываемым ниже образом. Будем
излагать данное преобразование в~терминах графа, соответствующего
ВСП~$D$ (данный граф будем обозначать тем же символом~$D$):
\begin{itemize}
\item[(i)]
если граф $D$ содержит ребро с началом в~некоторой вершине~$s$
и~с концом~$s_2$, то данное ребро удаляется, а~к~метке ребра
с~началом в~$s$ и~с~концом в~$s_1$ прибавляется число, равное метке
удаленного ребра. Данная операция выполняется до тех пор, пока
имеются ребра с~концом в~$s_2$;
\item[(ii)] вершина $s_2$ удаляется и,~кроме того, удаляются все ребра,
выходящие из этой вершины;
\item[(iii)] из~$R$ удаляются все пары, содержащие~$s_2$, и~осуществляется
переход к шагу~2a;
\end{itemize}

\item[(б)] если каждая пара, входящая в~$R$, имеет
вид $(s,s)$, то   работа завершается.
\end{itemize}
\end{enumerate}

\section{Пример редукции вероятностной системы~переходов}

В этом разделе рассматривается пример редукции вероятностной модели
протокола передачи сообщений через ненадежный канал связи, в~котором
пересылаемые сообщения могут пропадать или искажаться. Протокол
представляет собой систему, состоящую из двух агентов~--- отправителя
и~получателя, а~также канала,  в~который помещаются сообщения,
 пересылаемые от одного агента другому.
 Предполагается, что факт искажения
 получаемых сообщений может быть установлен, и~если исходное сообщение не может быть восстановлено
 из искаженного, то отправитель получает  сигнал о~необходимости повторной посылки
 этого сообщения. Как только сообщение успешно доходит до получателя,
отправителю посылается сигнал подтверждения успешного
получения и~он переходит к отправке следующего сообщения.
  Предполагается, что сигналы и~подтверждения
  отправителю не пропадают и~не искажаются в~канале.

Графовая модель этого протокола имеет  вид, представленный на рис.~1.

\vspace*{6pt}

\begin{center}  %fig1
\vspace*{2pt}
\mbox{%
 \epsfxsize=72.154mm
 \epsfbox{mir-1.eps}
 }
 \end{center}
%  \vspace*{2pt}

\noindent
{{\figurename~1}\ \ \small{Графовая модель протокола передачи сообщений через
ненадежный канал связи}}


\vspace*{12pt}


\addtocounter{figure}{1}


Переходы в~этом графе имеют следующий смысл.

\begin{enumerate}[1.]
\item Переход $s_0\ra{1}s_1$ заключается в~получении
 отправителем от внешнего источника  сообщения, которое должно быть
 передано через канал получателю.
\item Переход $s_1\ra{0,8}s_2$ заключается в~помещении сообщения
в~канал отправителем, причем сообщение в~канале не искажается.
\item Переход $s_1\ra{0,2}s_4$ заключается в~помещении сообщения в~канал
отправителем, причем сообщение в~канале искажается.
\item Переход $s_2\ra{0,3} s_1$ заключается в~потере неискаженного сообщения
в~канале и~посылке отправителю сигнала о~необходимости повторной передачи.
\item Переход $s_4\ra{0,3} s_1$ заключается в~посылке
отправителю сигнала о~том, что исходное сообщение не может быть
восстановлено из искаженного сообщения и~должно быть передано
повторно.
\item Переход $s_2\ra{0,7}s_3$ заключается в~передаче
неискаженного сообщения из канала получателю.
\item Переход $s_4\ra{0,7}s_5$ заключается в~восстановлении исходного сообщения из
искаженного и~передаче восстановленного сообщения получа\-телю.
\item Переходы $s_3\ra{1}s_0$ и~$s_5\ra{1}s_0$ заключаются в~получении
  сообщения получателем и~посылке им отправителю
  уведомления о~том, что получение сообщения   было выполнено успешно.
\end{enumerate}

Одно из свойств   протокола, представленного моделью рис.~1,
   заключается в~том, что каждое сообщение, полученное отправителем
от внешнего источника, будет с вероятностью    $\hm\geq 0{,}9$  доставлено получателю
      не более чем через 5~единиц времени.
   Для формального представления этого свойства будем полагать, что
   множество~AP атомарных утверждений состоит из одной переменной~$p$ и~эта
   переменная принимает в~состоянии~$s_0$ (см.\ рис.~1)
   значение~1, а~в~остальных состояниях~--- значение~0.
Таким образом, множество $2^{\mathrm{AP}}$ состоит из двух элементов:
$\emptyset$ и~$\{p\}$. Будем обозначать эти элементы символами~0 и~1 соответственно.

         Формула логики PCTL, соответствующая  указанному выше свойству,
   имеет следующий вид:
\be{sdfsadfsa44}
{\bf G}((\neg p) \to {\cal P}_{\geq 0.9}({\bf F}^{\leq 5}p))\,,
\ee
где символ~$\geq$ обозначает функцию вида
$$
\geq: [0,1]\times [0,1]\to\{0,1\}\,,
$$
которая сопоставляет паре $(a,b)\hm\in [0,1]\times [0,1]$
элемент~1, если $a\hm\geq b$, и~0  иначе.

Анализируемая ВСП получается из графа, представленного на
рис.~1 приписыванием
к~каждой его вершине~$s$ метки $L(s)$, которая равна~1, если $s\hm=s_0$, и~0 иначе.
Для вычисления значения  формулы~\re{sdfsadfsa44}
в~состояниях этой ВСП можно использовать описанный выше метод редукции.

Матрицу~$P$, соответствующую данной ВСП, представим в~виде следующей таблицы:
$$
    \begin{tabular}{|c||c|c|c|c|c|c|}
  \hline
  & $s_0$ & $s_1$ & $s_2$ & $s_3$ & $s_4$ &
    $s_5$   \\
  \hline
  \hline $s_0$ & 0& $1$& 0& $0$& $0$& $0$\\
  \hline $s_1$ & 0& 0& $0,8$& 0& $0,2$& 0\\
  \hline $s_2$ & 0& 0,3& 0& 0,7& 0& 0 \\
  \hline $s_3$ & 1& 0& $0$& 0& 0& $0$\\
  \hline $s_4$ & $0$& $0,3$& 0& 0,7& 0& 0\\
  \hline $s_5$ & $1$& 0& 0& $0$& 0& 0 \\
  \hline
  \end{tabular}
$$



Вычисление эквивалентности~$\sim$ для анализируемой ВСП происходит следующим образом.
\begin{enumerate}[1.]
\item Вычисляется отношение эквивалентности~$\rho^0$,
которое состоит из  всех пар $(s_1, s_2)\hm\in S\times S$,
удовлетворяющих равенству $L(s_1) \hm=L(s_2)$.

По предположению значение~$p$ в~$s_0$  равно~1 и~в~каждом $s\hm\in S$
(где $S$~--- множество состояний анализируемой ВСП), таком что $s\hm\neq s_0$,
значение~$p$ равно~0, т.\,е.\ $L(s_0)\hm=1$ и~$\forall\,s\hm\in
S\setminus \{s_0\}$ $L(s)=0$. Cледовательно, $\Sigma^0$ состоит
из двух классов:

\noindent
\be{dfgfdsgdfgfd5r5}
\{s_0\}, \quad \{s_1, s_2, s_3, s_4, s_5\}\,.
\ee

\item Матрица $\pi^0$, соответствующая детерминированной
СФ $\pi^0: S\hm\to \Sigma^0$, имеет вид:
$$
  \begin{tabular}{|c||c|c|}
%  \hline   & $\,$ & $\,$  \\
 % \hline
 \hline
   $s_0$ & 1&0\\
  \hline $s_1$ &  0&1\\
   \hline $s_2$ & 0&1 \\
   \hline $s_3$ & 0&1\\
   \hline $s_4$ & 0&1 \\
   \hline $s_5$ & 0&1 \\
  \hline
  \end{tabular}
$$

Затем вычисляется матрица $P\cdot \pi^0$. Данная мат\-ри\-ца будет иметь сле\-ду\-ющий вид:
\be{sdfsdafsd3333}
  \begin{tabular}{|c||c|c|}
%  \hline   & $\,$& $\,$   \\
 % \hline
 \hline
   $s_0$ & 0&1\\
  \hline $s_1$ &  0&1\\
   \hline $s_2$ & 0&1 \\
   \hline $s_3$ & 1&0\\
   \hline $s_4$ & 0&1 \\
   \hline $s_5$ & 1& 0\\
  \hline
  \end{tabular}
\ee


По матрице~\re{sdfsdafsd3333} нетрудно вычислить отношение~$\rho^1$
и соответствующее ему разбиение~$\Sigma^1$.
Из определения отношения~$\rho^1$ непосредственно следует, что
состояния~$s$ и~$s'$ находятся в~одном и~том
же классе разбиения~$\Sigma^1$ тогда и~только тогда, когда они оба
находятся в~одном и~том же классе из списка~\re{dfgfdsgdfgfd5r5}
и,~кроме того, строки матрицы~\re{sdfsdafsd3333},
соответствующие состояниям~$s$ и~$s'$, совпадают.

Разбиение~$\Sigma^1$ будет состоять  из трех классов
(измельчится второй класс в~\re{dfgfdsgdfgfd5r5}, а~первый класс
останется тем же), эти классы имеют следующий вид:
\be{dfgfdsgdfgfd5r51331}
\{s_0\}, \quad \{s_1, s_2, s_4\}, \quad
\{s_3, s_5\}.
\ee

\item Затем вычисляется матрица $\pi^1$, соответст\-ву\-ющая
детерминированной СФ $\pi^1: S \vo \Sigma^1$. Она выглядит следующим
образом:
$$
  \begin{tabular}{|c||c|c|c|}
  %\hline   & $\,$  & $\,$ &$\,$\\
%  \hline
   \hline
   $s_0$ & 1& 0 & 0\\
   \hline $s_1$ & 0 & 1 & 0 \\
   \hline $s_2$ & 0 &1 & 0\\
   \hline $s_3$ & 0 &0 & 1\\
   \hline $s_4$ & 0 &1 & 0\\
   \hline $s_5$ & 0 &0 & 1\\
  \hline
  \end{tabular}
$$

Произведение $P\cdot \pi_1$ выглядит так:
$$
  \begin{tabular}{|c||c|c|c|}
 % \hline   & $\,$  & $\,$ &$\,$\\
%  \hline
   \hline $s_0$ & 0& 1 & 0\\
   \hline $s_1$ & 0 & 1 & 0 \\
   \hline $s_2$ & 0 &$0,3$ & $0,7$\\
   \hline $s_3$ & 1 &0 & 0\\
   \hline $s_4$ & 0 &$0,3$ & $0,7$\\
   \hline $s_5$ & 1 &0 & 0\\
  \hline
  \end{tabular}
$$

После этого, действуя так же, как и~в~предыдущем пункте, вычисляем
классы раз\-би\-ения~$\Sigma^2$, соответствующего эквивалентности~$\rho^2$.
Таких классов будет четыре (измельчится второй  класс
в~\re{dfgfdsgdfgfd5r51331}, а~первый и~третий классы останутся теми же),
эти классы имеют сле\-ду\-ющий вид:
\be{dfgfds4433dfgfd5r51331}
\{s_0\}, \quad \{s_1\}, \quad \{s_2, s_4\},\quad
\{s_3, s_5\}.
\ee


\item
Затем вычисляется матрица~$\pi^2$,
соответ\-ст\-ву\-ющая детерминированной СФ $\pi^2: S \vo \Sigma^2$:
$$
\begin{tabular}{|c||c|c|c|c|}
%  \hline   & $\,$  & $\,$ &$\,$&$\,$\\
%  \hline
   \hline $s_0$ & 1& 0 & 0& 0\\
   \hline $s_1$ & 0 & 1 & 0 & 0\\
   \hline $s_2$ & 0 &0 & 1& 0\\
   \hline $s_3$ & 0 &0 & 0& 1\\
   \hline $s_4$ & 0 &0 & 1& 0\\
   \hline $s_5$ & 0 &0 & 0& 1\\
  \hline
  \end{tabular}
$$

Произведение $P\cdot \pi_2$ имеет следующий вид:

\begin{equation*} %{sdfsdafsd333333323344}
  \begin{tabular}{|c||c|c|c|c|}
%  \hline   & $\,$  & $\,$ &$\,$&$\,$\\
%  \hline
   \hline $s_0$ & 0 & 1 & 0& 0 \\
   \hline $s_1$ & 0 &  0& 1 & 0\\
   \hline $s_2$ & 0 &0,3 & 0& 0,7\\
   \hline $s_3$ & 1 &0 & 0&  0\\
   \hline $s_4$ & 0 &0,3 & 0& 0,7\\
   \hline $s_5$ & 1 &0 & 0& 0\\
  \hline
  \end{tabular}
\end{equation*}

Далее, действуя так же, как и~в~предыдущем пункте, вычисляем классы
разбиения~$\Sigma^3$, соответствующего эквивалентности~$\rho^3$.
Нетрудно проверить, что классы разбиения~$\Sigma^3$ будут иметь
точно такой же вид, что и~классы  эквивалентности разбиения~$\Sigma^2$.
Это означает, что искомое разбиение множества~$S$  на
классы эквивалентных состояний построено, оно имеет вид~\re{dfgfds4433dfgfd5r51331}.
\end{enumerate}

Теперь можно приступить к~удалению избыточных состояний (так, чтобы
среди оставшихся состояний было ровно по одному состоянию
из каждого класса эквивалентности~\re{dfgfds4433dfgfd5r51331}).
Нетрудно видеть, что можно удалить состояния~$s_4$ и~$s_5$.
После удаления данных состояний граф, представленный
на рис.~1 примет вид, представленный на рис.~2.

\begin{center}  %fig2
\vspace*{2pt}
\mbox{%
 \epsfxsize=40.06mm
 \epsfbox{mir-2.eps}
 }
  \vspace*{2pt}

{{\figurename~2}\ \ \small{Модифицированный граф рис.~1}}
  \end{center}

%\vspace*{3pt}


\addtocounter{figure}{1}




Таким образом, задача вычисления
значений формулы~\re{sdfsadfsa44} в~состояниях ВСП (см.\ рис.~1)
сводится к задаче вычисления значений формулы~\re{sdfsadfsa44} в~состояниях
ВСП~(см.\ рис.~2), что требует выполнения
меньшего числа операций, чем задача вычисления
значений формулы~\re{sdfsadfsa44} в~состояниях исходной ВСП.

\vspace*{-9pt}


\section{Заключение}

В настоящей работе изложен алгоритм редукции
ВСП, идея которого
заключается в~удалении избыточных состояний.
Результатом применения этого алгоритма является ВСП,
чис\-ло со\-сто\-яний которой не превосходит
чис\-ла со\-сто\-яний исходной ВСП и~все свойства которой,
выражаемые формулами логики PCTL, совпадают со свойствами исходной ВСП.
Идея алгоритма заключается\linebreak в~по\-стро\-ении последовательности вложенных
разбиений мно\-жества состояний исходной ВСП. Алгоритм по\-стро\-ения
последовательности разбиений мно\-жества состояний заканчивает свою работу,\linebreak
когда эта последовательность стабилизируется. Редукция ВСП выполняется
методом удаления эквивалентных состояний и~переопределения вероятностей перехода.
На примере показана возможность применения предложенного алгоритма
к~задаче вероятностной верификации протокола передачи сообщений
через ненадежный канал связи, в~котором пересылаемые сообщения могут пропадать
или искажаться, с~возможной коррекцией искажения.
Отметим, что в~результате такой редукции может получиться ВСП,
которая, хотя и~не содержит различных эквивалентных состояний, тем не
менее, может не являться минимальной по числу состояний среди всех ВСП,
эквивалентных исходной ВСП.
В связи с~этим встает вопрос об
алгоритме нахождения минимальной по числу состояний ВСП,
эквивалентной заданной ВСП, и~исследовании
единственности такой минимальной ВСП (с~точ\-ностью до подходящим образом
сформулированного понятия изоморфизма).
Так\-же пред\-став\-ля\-ет интерес исследование проб\-лем минимизации других классов
моделей, связанных с~вероятностной верификацией, в~част\-ности минимизации
марковских решающих процессов.

\vspace*{-9pt}

{\small\frenchspacing
 {%\baselineskip=10.8pt
 \addcontentsline{toc}{section}{References}
 \begin{thebibliography}{99}


\bibitem{karpov} %1
\Au{Карпов Ю.\,Г.} Model checking.
Верификация параллельных и~распределенных программных систем.~---
СПб.: БХВ-Петербург, 2010. 560~с.

\bibitem{markov} %2
\Au{Кемени Дж., Снелл Дж.} Конечные цепи Маркова.~--- М.: Наука, 1970. 225~с.

\bibitem{buh} %3
\Au{Бухараев Р.\,Г.} Основы теории вероятностных автоматов.~--- М.: Наука, 1985. 288~с.


\bibitem{35} %4
\Au{Hansson H., Jonsson B.} A~logic for reasoning about time and
reliability~// Formal Aspects Computing, 1994. Vol.~6. No.\,5. P.~512--535.

\bibitem{tut} %5
\Au{Kwiatkowska M., Parker D.} Advances in probabilistic model
checking~// NATO Science for Peace and Security Series. Information
and Communication Security, 2012. Vol.~33. P.~126--151.

\bibitem{peled} %6
\Au{Кларк Э.\,М., Грамберг О., Пелед~Д.} Верификация моделей
программ. Model Checking.~--- М.: МЦНМО, 2002. 416~с.

\bibitem{mf} %7
\Au{Миронов А.\,М., Френкель С.\,Л.} Минимизация вероятностных
моделей программ~// Фундаментальная и~прикладная математика, 2014.
Т.~19. Вып.~1. С.~121--163.

\bibitem{6} %8
\Au{Baier C., Groesser M., Ciesinski~F.} Partial order reduction
for probabilistic systems~// 1st Conference (International)
on Quantitative Evaluation of Systems (QEST'04) Proceedings.~--- IEEE
Computer Society Press, 2004. P.~230--239.

\bibitem{23} %9
\Au{D'Argenio P., Niebert P.} Partial order reduction on
concurrent probabilistic programs~//  1st
Conference (International) on Quantitative Evaluation of Systems
(QEST'04) Proceedings.~--- IEEE Computer Society Press, 2004. P.~240--249.

\bibitem{52} %10
\Au{Kwiatkowska M., Norman G., Parker~D.} Symmetry reduction for
probabilistic model checking~// Computer aided verification~/
Eds. T.~Ball, R.\,B.~Jones.
Lecture notes in computer science ser.~--- Springer, 2006.  Vol.~4144. P.~234--248.

\bibitem{26} %11
\Au{Donaldson A., Miller A.} Symmetry reduction for probabilistic
model checking using generic representatives~//
Automated technology for verification
and analysis~/ Eds.\ S.~Graf, W.~Zhang. Lecture notes in computer science ser.~--- Springer, 2006.
Vol.~4218. P.~9--23.



\bibitem{36} %12
\Au{Hart S., Sharir M., Pnueli~A.} Termination of probabilistic
concurrent programs // ACM Trans. Programming Languages
Syst., 1983. Vol.~5. No.\,3. P.~356--380.

\bibitem{63}
\Au{Vardi M.} Automatic verification of probabilistic concurrent
finite state programs~// 26th Annual Symposium on
Foundations of Computer Science (FOCS'85) Proceedings.~--- IEEE Computer Society
Press, 1985. P.~327--338.

\bibitem{21}
\Au{Courcoubetis C., Yannakakis~M.} Verifying temporal properties
of finite state probabilistic programs~// 29th
Annual Symposium on Foundations of Computer Science (FOCS'88) Proceedings.~---
IEEE Computer Society Press, 1988. P.~338--345.


\bibitem{14} %15
\Au{Bianco A., de Alfaro L.} Model checking of probabilistic and
nondeterministic systems~// Foundations of software technology and
theoretical computer science~/
Ed. P.\,S.~Triagarejan.  Lecture notes in computer science ser.~---
Springer, 1995.  Vol.~1026. P.~499--513.

\bibitem{8}
\Au{Baier C., Haverkort B., Hermanns~H., Katoen~J.-P.}
Model-checking algorithms for continuous-time Markov chains~// IEEE
Trans. Software Eng., 2003. Vol.~29. No.\,6.
P.~524--541.

\bibitem{34} %17
\Au{Hansson H.} Time and probability in formal design of
distributed systems.~--- Elsevier, 1994. 304~p.

\bibitem{5} %18
\Au{Baier C., Clarke E., Hartonas-Garmhausen~V., Kwiatkowska~M.,
Ryan~M.} Symbolic model checking for probabilistic processes~//
Automata, languages and programming~/ Eds. P.~Degano, R.~Gorrieri, A.~Marinetti-Spaccamela.
Lecture notes in computer science ser.~--- Springer,
1997.  Vol.~1256. P.~430--440.

\bibitem{40} %19
\Au{Hermanns H., Katoen~J.-P., Meyer-Kayser~J., Siegle~M.}
A~Markov chain model checker~// Tools and algorithms for the construction and
analysis of systems~/
Eds. S.~Graf, M.\,I.~Schwartzbach.
Lecture notes in computer science ser.~--- Springer, 2000.
 Vol.~1785. P.~347--362.

\bibitem{24} %20
\Au{De Alfaro L., Kwiatkowska~M., Norman~G., Parker~D., Segala~R.}
Symbolic model checking of probabilistic processes using MTBDDs and
the Kronecker representation~//
Tools and algorithms for the construction and analysis
of systems~/
Eds. S.~Graf, M.\,I.~Schwartzbach. Lecture notes in computer science ser.~--- Springer, 2000.
 Vol.~1785. P.~395--410.



\bibitem{50} %21
\Au{Kwiatkowska M., Norman~G., Parker~D.} Probabilistic model
checking in practice: Case studies with PRISM~// ACM SIGMETRICS
Performance Evaluation Review, 2005. Vol.~32. No.\,4. P.~16--21.

\bibitem{55}
\Au{Kwiatkowska M., Norman~G., Parker~D.} PRISM 4.0: Verification
of probabilistic real-time systems~//
 Computer aided verification~/
 Eds. G.~Gopalakrishnan, S.~Qadeer.
Lecture notes in computer science ser.~--- Springer, 2011. Vol.~6806. P.~585--591.

 \end{thebibliography}

 }
 }

\end{multicols}

\vspace*{-12pt}

\hfill{\small\textit{Поступила в редакцию 05.11.14}}

%\newpage

\vspace*{8pt}

\hrule

\vspace*{2pt}

\hrule

%\vspace*{12pt}

\def\tit{A METHOD OF ENHANCING PROBABILISTIC VERIFICATION EFFICIENCY FOR COMPUTER AND TELECOMMUNICATION SYSTEMS\\[-7pt]}

\def\titkol{A method of enhancing probabilistic verification
efficiency for computer and telecommunication systems}

\def\aut{A.\,M.~Mironov$^1$ and~S.\,L.~Frenkel$^{1,2}$}

\def\autkol{A.\,M.~Mironov and~S.\,L.~Frenkel}

\titel{\tit}{\aut}{\autkol}{\titkol}

\vspace*{-10pt}

\noindent
$^1$Institute of Informatics Problems, Russian Academy of Sciences,
44-2~Vavilov Str., Moscow 119333, Russian\linebreak
$\hphantom{^1}$Federation

\noindent
$^2$Moscow Institute of Radio, Electronics, and Automation
(MIREA), 78\ Prosp. Vernadskogo, Moscow 119454,\linebreak
$\hphantom{^1}$Russian Federation

\def\leftfootline{\small{\textbf{\thepage}
\hfill INFORMATIKA I EE PRIMENENIYA~--- INFORMATICS AND
APPLICATIONS\ \ \ 2014\ \ \ volume~8\ \ \ issue\ 4}
}%
 \def\rightfootline{\small{INFORMATIKA I EE PRIMENENIYA~---
INFORMATICS AND APPLICATIONS\ \ \ 2014\ \ \ volume~8\ \ \ issue\ 4
\hfill \textbf{\thepage}}}

\vspace*{2pt}

\Abste{The paper considers the problem of reduction of probabilistic transition
systems (PTS) in order to reduce the complexity of model checking of such systems.
The problem of model checking of a~PTS is to calculate truth
values
of formulas of temporal probabilistic computational tree logic (PCTL)
in the initial state of the PTS. The\linebreak\vspace*{-12pt}}

\Abstend{paper introduces the concept of equivalence
of states of a~PTS and represents an algorithm for removing equivalent states.
The result of this algorithm is a~PTS such that all its properties expressed
by formulas of PCTL coincide with those of the original PTS.}

\KWE{verification; model checking;
probabilistic transition systems; probabilistic
temporal logic; reduction of probabilistic models}




  \DOI{10.14357/19922264140408}

  \vspace*{-14pt}


\Ack

\vspace*{-3pt}

\noindent
The research was partially supported by the Russian Foundation for Basic Research
 (project 12-07-00109).



\vspace*{-6pt}

  \begin{multicols}{2}

\renewcommand{\bibname}{\protect\rmfamily References}
%\renewcommand{\bibname}{\large\protect\rm References}



{\small\frenchspacing
 {%\baselineskip=10.8pt
 \addcontentsline{toc}{section}{References}
 \begin{thebibliography}{99}

 \vspace*{-2pt}

 \bibitem{karpov-1}
\Aue{Karpov, Yu.\,G.} 2010.
\textit{Model checking. Verification of parallel and distributed systems.}
St.\ Petersburg.: BHV-Peterburg. 560~p.


\bibitem{markov-1}
\Aue{Kemeny, J.\,G., and J.\,L.~Snell.} 1976.
\textit{Finite Markov chains.}
New York\,--\,Berlin\,--\,Heidelberg\,--\,Tokyo: Springer-Verlag.
225~p.

\bibitem{buh-1}
\Aue{Bukharaev, R.\,G.} 1985.
\textit{Foundations of probabilistic automata theory.}
Moscow: Nauka. 288~p.



\bibitem{35-1} %4
\Aue{Hansson, H., and B.~Jonsson}.
1994. A~logic for reasoning about time and reliability.
\textit{Formal Aspects Computing} 6(5):512--535.

\bibitem{tut-1} %5
\Aue{Kwiatkowska, M., and D.~Parker}. 2012.
Advances in probabilistic model checking.
\textit{NATO Science for Peace and Security Series,
Information and Communication Security} 33:126--151.

\bibitem{peled-1}
\Aue{Clarke, E.\,M., O. Grumberg, and D.~Peled.}
1999. \textit{Model checking.} MIT Press. 314~p.

\bibitem{mf-1}
\Aue{Mironov, A.\,M., and S.\,L.~Frenkel}.
2014. Minimization of probabilistic models of programs.
\textit{Fundamental Applied Mathematics} 19(1):121--163.

\bibitem{6-1} %8
\Aue{Baier, C., M. Groesser, and F.~Ciesinski.}
2004.
Partial order reduction for probabilistic systems.
\textit{1st Conference (International) on Quantitative
Evaluation of Systems (QEST'04) Proceedings}.
IEEE Computer Society Press. 230--239.

\bibitem{23-1}
\Aue{D'Argenio, P., and P. Niebert.}
2004.
Partial order reduction on
concurrent probabilistic programs.
\textit{1st Conference (International) on Quantitative
Evaluation of Systems (QEST'04) Proceedings}.
IEEE Computer Society Press. 240--249.

\bibitem{52-1} %10
\Aue{Kwiatkowska, M., G. Norman, and D.~Parker.}
2006.
Symmetry reduction for probabilistic model checking.
\textit{Computer aided verification}. Eds. T.~Ball and R.\,B.~Jones.
Lecture notes in computer science ser.
4144:234--248.

\bibitem{26-1} %11
\Aue{Donaldson, A., and A. Miller.}
2006. Symmetry reduction for
probabilistic model checking using
generic representatives.
\textit{Automated technology for verification
and analysis}.
Eds.\ S.~Graf and W.~Zhang. Lecture notes in computer science ser.
4218:9--23.



\bibitem{36-1} %12
\Aue{Hart, S., M. Sharir, and A.~Pnueli.}
1983. Termination of probabilistic concurrent programs.
\textit{ACM Trans. Programming Languages Syst.}
5(3):356--380.

\bibitem{63-1} %13
\Aue{Vardi, M.} 1985.
Automatic verification of probabilistic concurrent
finite state programs.
\textit{26th Annual Symposium on Foundations of Computer Science (FOCS'85) Proceedings.}
IEEE Computer Society Press. 327--338.

\bibitem{21-1} %14
\Aue{Courcoubetis, C., and M. Yannakakis.}
1988. Verifying temporal properties
of finite state probabilistic programs.
\textit{29th Annual Symposium on Foundations of Computer Science (FOCS'88)
Proceedings.}
IEEE Computer Society Press. 338--345.

\bibitem{14-1} %15
\Aue{Bianco, A., and L. de Alfaro.}
1995. Model checking of probabilistic and
nondeterministic systems.
\textit{Foundations of software technology and theoretical computer science}.
Ed. P.\,S.~Triagarejan. Lecture notes in computer science ser.
1026:499--513.

\bibitem{8-1} %16
\Aue{Baier, C., B. Haverkort, H.~Hermanns, and J.-P.~Katoen.}
2003.
Model-checking algorithms for continuous-time Markov chains.
\textit{IEEE Trans. Software Engineering} 29(6):524--541.

\bibitem{34-1} %17
\Aue{Hansson, H.} 1994.
\textit{Time and probability in formal design of distributed
systems.} Elsevier. 304~p.

\bibitem{5-1} %18
\Aue{Baier, C., E. Clarke, V.~Hartonas-Garmhausen, M.~Kwiatkowska, and M.~Ryan.}
1997. Symbolic model checking for probabilistic processes.
\textit{Automata,
languages and programming.}
Eds. P.~Degano,  R.~Gorrieri, and A.~Marinetti-Spaccamela.
Lecture notes in computer science ser.
1256:430--440.

\bibitem{40-1} %19
\Aue{Hermanns, H., J.-P.~Katoen, J.~Meyer-Kayser, and M.~Siegle.}
2000. A~Markov chain model checker.
\textit{Tools and algorithms for the construction and analysis
of systems.} Eds. S.~Graf and M.\,I.~Schwartzbach.
Lecture notes in computer science ser.
1785:347--362.

\bibitem{24-1} %20
\Aue{De Alfaro, L., M. Kwiatkowska, G.~Norman, D.~Parker, and R.~Segala}.
2000.
Symbolic model
checking of probabilistic processes using MTBDDs and the Kronecker representation.
\textit{Tools and algorithms for the construction and analysis
of systems.} Eds. S.~Graf and M.\,I.~Schwartzbach.
Lecture notes in computer science ser.
1785:395--410.


\bibitem{50-1} %21
\Aue{Kwiatkowska, M., G.~Norman, and D.~Parker.}
2005.
Probabilistic model checking in practice:
Case studies with PRISM.
\textit{ACM SIGMETRICS Performance Evaluation Review.}
32(4):16--21.

\bibitem{55-1} %23
\Aue{Kwiatkowska, M., G. Norman, and D.~Parker.}
2011.
PRISM 4.0: Verification of probabilistic real-time systems.
\textit{Computer aided verification.}  Eds. G.~Gopalakrishnan and S.~Qadeer.
Lecture notes in computer science ser.
6806:585--591.


\end{thebibliography}

 }
 }

\end{multicols}

\vspace*{-12pt}

\hfill{\small\textit{Received November 5, 2014}}

\pagebreak

%\vspace*{-18pt}


\Contr


\noindent
\textbf{Mironov Andrew M.} (b.\ 1966)~---
Candidate of Science (PhD) in physics and mathematics,
senior scientist, Institute of Informatics Problems,
Russian Academy of Sciences, 44-2 Vavilov Str., Moscow 119333,
Russian Federation; amironov66@gmail.com

\vspace*{6pt}

\noindent
\textbf{Frenkel Sergey L.} (b.\ 1951)~---
Candidate of Science (PhD) in technology, senior scientist,
Institute of Informatics Problems, Russian Academy of Sciences,
44-2 Vavilov Str., Moscow 119333, Russian Federation;
associate professor, Moscow Institute of Radio, Electronics, and Automation
(MIREA), 78 Prosp.\ Vernadskogo, Moscow 119454, Russian Federation; fsergei@mail.ru

\label{end\stat}

\renewcommand{\bibname}{\protect\rm Литература}
