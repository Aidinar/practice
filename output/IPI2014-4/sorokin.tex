\renewcommand{\figurename}{\protect\bf Figure}

\def\stat{sorokin}


\def\tit{AUTOMATION BEYOND WEB 2.0}

\def\titkol{Automation beyond WEB 2.0}

\def\autkol{A.~Sorokin}

\def\aut{A.~Sorokin$^1$}

\titel{\tit}{\aut}{\autkol}{\titkol}

%{\renewcommand{\thefootnote}{\fnsymbol{footnote}}
%\footnotetext[1] {The work of first and second  authors is partially supported by the
%Program of Strategy development of Petrozavodsk State University in
%the framework of the research activity. The third author is a
%postdoctoral fellow with the Research Foundation-Flanders
%(FWO-Vlaanderen).}}

\renewcommand{\thefootnote}{\arabic{footnote}}
\footnotetext[1]{IBM EE/A, % University Relations Manager for Russia \& CIS,
10 Presnenskaya Nab., Moscow 123317, Russian Federation}


%\vspace*{-12pt}

\def\leftfootline{\small{\textbf{\thepage}
\hfill INFORMATIKA I EE PRIMENENIYA~--- INFORMATICS AND APPLICATIONS\ \ \ 2014\ \ \ volume~8\ \ \ issue\ 4}
}%
 \def\rightfootline{\small{INFORMATIKA I EE PRIMENENIYA~--- INFORMATICS AND APPLICATIONS\ \ \ 2014\ \ \ volume~8\ \ \ issue\ 4
\hfill \textbf{\thepage}}}

\vspace*{6pt}


\Abste{This paper introduces a~new approach to the analysis of information systems
(IS) evolution based
on a~range of technological activities. The issue centres on the prospect that
Web-driven IS will be expanded from business processes to other domains of activities. The classical
approach by which automation eliminates bottlenecks in business processes does not work under
these conditions. Current trends in information technologies
(IT) increase the capability for Web integration that leads to new
types of virtual systems that will create a~new Web architecture, conditionally named a~Web
``spiral.''
The spiral type of integration on Web
supported by integrated cross-industry solutions is more promising and effective in comparison with
the ``radial'' ones. The paper describes this new class of IT systems.}

\KWE{automation; business process reengineering; collaborative software;
economies of scale; Internet topology; sociotechnical systems; systems of systems; virtual enterprises; Web 2.0}

\DOI{10.14357/19922264140414}

\vspace*{6pt}


\vskip 12pt plus 9pt minus 6pt

      \thispagestyle{myheadings}

      \begin{multicols}{2}

                  \label{st\stat}

\section{Introduction}

  \noindent
  Not much time has passed~--- on a~historical scale~--- since computers came on the technology scene.
However, the time and origin of invention and even authorship are still under discussion. Clearly, the
improvement in computing is a~process that combines a~variety of ideas, technologies,
 and drivers. In
Japan, for example, according to some sources, computing after the war had tight links with the emergence
of telephone switchboards
({\sf http://museum.ipsj.or.jp/en/computer/dawn/0005.\linebreak html}).

 In the U.S., one of the motivations for searching new ways of information processing was difficulties in
working with clerical card indexes in the first 30~years of the XX~century. In the USSR, the first
prototypes of computing technology were closely associated with defense programmes and space
research and development (R\&D).
After its birth, computing technology began to show not only rapid development, but also the ability to
penetrate virtually all spheres of human activities, going far beyond the range of tasks originally referred to.
With the emergence of local area network (LAN) and Internet, computing evolved along with telecommunications originating
the term ``information and communications technology'' (ICT).

 Current ICT development reflects the issues that are relevant to the professional roles of specialists
involved. Information managers view as dominant the phenomenal data avalanche that has to be overcome
with the help of high performance parallel computing. The
IT architects are concerned about the complexity of
systems integration. Business analysts are trying to cope with the growing
number of approaches and
notations defining business process modeling.

 Narrow professional observations on the issues lead to multiple conclusions in which direction the Web of
the Future will go. The center of gravity of such observations, if focused on the technical qualities of new
IT, are considered here in comparison with each other, in isolation from
the economy and lifecycle (LC) environment
in which they operate.

 Dominant approaches are looking like a~kind of technological Darwinism, most characteristically
expressed by Gartner's hypercycle of emerging technologies~\cite{1-sor}. Meanwhile,
 automation has
firmly implemented in the overall human environment, the conditions of existence
which are regulated by
known factors, such as economic crises, resource depletion, population growth,
and many others. The IT
fashion is changing rapidly. In one to two years, new bright idea is highlighted.
One can endlessly study
trends whilst more practical question stays in shadow~--- what artifacts will appear at their crossroads?

   \begin{figure*} %fig1
      \vspace*{1pt}
 \begin{center}
 \mbox{%
 \epsfxsize=103.339mm
 \epsfbox{sor-1.eps}
 }
 \end{center}
 \vspace*{-9pt}
\Caption{Evolution of automation scale and impact criteria}
\vspace*{6pt}
\end{figure*}

 A more general conception for describing the evolution of IT systems has been proposed by IBM
Almaden Lab. They noted the marked complexity of models and objects, which designers of modern
IS face today. The concept is presented in the form of a~new Science of Services,
Management, and Engineering (SSME) aimed at developing design techniques for highly complex objects.
Stakeholders may include large organizations, cities, and even whole states. ({\sf
http://campustechnology. com/articles/2009/04/13/ibm-and-higher-ed-push-for-a-smarter-planet-with-ssme-curriculum.aspx}).

 Viewing IT evolution from the SSME perspective, one may come to the conclusion that a~classical
approach to IT system design, based on business processes analysis, does not work here at all, or must be
extensively revised. The reason lies in the multiplicity of such processes that must be simultaneously
analyzed and automated.
Additionally, advanced complex IT systems automate not one or a~few business
processes but whole domains of activities in general. In such cases,
one is faced with a~large number of
processes and technologies embedded in living environments; so, each of them needs to be repeatedly
changed, removed, or substituted. This leads to the necessity of an initial analysis of the domain
itself and,
second, the related business processes, implying transition to a~meta-design approach, and the
modification of basic ICT system design, scaling to \textit{domain} of activity. Under the term `activity,'
the bunch of connected business processes, covering professional areas (domains), supported by
groups of technologies with the purpose of improving economic efficiency
is considered.
Hence,
{\bfseries\textit{automation and virtualization}} in this work are regarded as group technologies for the
reproduction of the artificial environment with the purpose of significantly improving business
performance.

 In this paper, the system and design aspects of future Web evolution
 are investigated that will
undoubtedly affect the architecture and LC of IS. Below,
{\bfseries\textit{the stack of activities}} is introduced which analysis can help
to make some predictions on ICT
development in the near future. Figure~1 shows automation changes in terms of scale and added-value
criteria.


 In this paper, the present author will try to find some answers on described above issues considering Web as
 self-organizing, self-sustaining, and evolving system with multiple direct and feedback connections
between domains.

\vspace*{-3pt}

\section{Lifecycle Ecosystem}

 \noindent
 An area of activity (domain) as functionally homogeneous set of business processes and
technologies appear in division of labor is considered.
Quantity, composition, and relationship between such activities
form products LC and environment that represent a~model of technological expansion. It uses
the principle of procreation, when the original activity generates the following one as a~result of internal
conflict that limited its capacity (Fig.~2).


\begin{figure*} %fig2
\vspace*{1pt}
 \begin{center}
 \mbox{%
 \epsfxsize=134.29mm
 \epsfbox{sor-2.eps}
 }
 \end{center}
 \vspace*{-9pt}
  \Caption{LC Ecosystem}
  \end{figure*}

 First of all~--- LC realization is impossible without resources. Therefore,
  the root domain will be
1.~``Access to resources'' activity. This domain is denoted by
number one, and main types of resources that are
consumed by the technology community create a~set of subdomains. Let list them in a~logical
 sequence~--- from basic to the more complex ones appearing later:
 \begin{enumerate}[{1}.1]
 \item Data (information resources) collection.
 \item  Natural resources extraction.
 \item Finances.
 \end{enumerate}

 1.1\ ``Data collection.'' This is the root activity in number of basic types playing fundamental role in every
human's life and society in general. Probably for this reason, the history of automated systems started with
the automation of information processes. Since that and going on,
automated systems are called as
``information.'' Without this subdomain, the very existence of society is impossible. Initial stage here is
observation. After that, people are beginning to measure the observed phenomena. Next stage is
introduction measures of weight, distance, and time. Then development moves towards the inventions of better
tools to collect information. Thus, new tools and correspondently domain states appeared: microscope and
telescope, photography, microphone, audio recording, telephone, x-ray, filming, radar, video, electronic
microscope, sensors and telemetry, space probe, databases, search engines, scanner, geopositioning, global
information systems. Each tool opened new and more powerful opportunities for high quality information
activities.

 1.2\ ``Natural resources extraction.'' To implement new technologies and inventions, natural
resources are needed. They determine the place of this activity. Stages evolve in the direction from the exploitation of
the biosphere and then~--- extraction and use of minerals and metals (mining), extraction of hydrocarbons,
uranium enrichment. The sequence of stages is moving from easy to more difficult availability and after
that~--- to creation of artificial resources, recycling, and meta-materials~\cite{3-sor}.

 1.3\ ``Finances.'' Recourses evolution also came from natural to artificial and led to emergence of money
as more general and valuable recourse.  Milestones indicating the progress of this type of activity are:
introduction into circulation of money substitutes, coinage, emergences of usury and banks, invention of
paper money, introduction into circulation of securities to financial markets, globalization and automated
support of financial markets.

 Intra conflict: resources cannot be used instantly at the place of extraction. This conflict is an origin of
new type of activity~--- 2. ``Transfer.''  Subdomains in this area are also ordered by complexity:
 \begin{enumerate}[{2}.1]
 \item  Information channels.
 \item  Material channels (transport, pipelines, energy lines).
 \end{enumerate}

 2.1\ ``Information channels.''  Stages that mark the domain expansion
 form a~sequence starting of
personal contacts between people, and further on~--- postal service, electrical transmission of discrete
information (telegraph, teletype), electromagnetic analog transmission (telephone, radio), digital
communications, computer  networks, satellite repeaters.

 2.2\ ``Material channels.''  Transfer of resources absorbs achievements of related domains and is
developing toward modern supply chain. They provide movement of a~large number of people,
transportation of goods, raw materials and energy. Stages marked expansion in this domain are: road
construction, use of natural ways (river and sea), the emergence of railways, air transport, container
transport, and, finally, modern automated supply chain.

 Intra conflict: inability of immediate use resources by end user. To improve this, they must be processed,
which generates the next logical domain~--- 3. ``Production.''  Its subdomains:
 \begin{enumerate}[{3}.1]
 \item  Power production.
 \item Production of goods and services.
 \end{enumerate}

 3.1\ ``Power production.'' Previous activity of extractive industries makes basis of energy production. On
qualitative scale of stages, one may mark use of muscular energy as
a~starting point and further on, opening of
fire, use of wind  and water energy, steam energy (in the beginning of the industrial revolution),
electricity, nuclear energy, alternative energy sources in a~modern, high-tech version, and, finally,
thermonuclear (forecast). Evolution of this domain makes it possible mass productive activity in the next
one.

 3.2\ ``Production of goods and services.'' Initial milestone of this activity
 is served to satisfy primary needs
with the help of manual production of food, clothing, and footwear.
Next step is providing of services. Then
goes the chain of stages: deployment of production of consumer goods on a~commercial scale, industrial and
residential construction, heavy machinery, entertainment business, high-tech manufacturing of
communications and computing equipment.

 Intra conflict: manufactured products and services must be delivered to consumer. This function is outside
this domain, which generates the following
area~4.~``Distribution.'' Communities cooperate with each other
by information, services, and products exchange they produced as objects of trade.
So, according to this logic,
trade became the following domain of activity in
the constructed technologic stack. It has evolved from
a~simple barter towards the emergences of money trade, after that~--- major trading houses, laying large
trade routes and distribution channels, unification of production,
transportation, and sales businesses, global
international trade, sales through communication channels and networks (television, mail, catalogs, and
over the Internet).

 Stages of this area include: barter, monetary trade, wholesale, trading networks, integration of trade and
production, global trade, and trade through communication channels.

 Intra conflict: Customers are buying goods which quality is entirely determined by the manufacturer. This
may not comply what customers are really need. In domain evolution, certain number of compensational
tools was born and developed: advertising, marketing, etc. However, these tools are not always fully able to
suppress this conflict manifestations. Flexible production is a~way to solve it getting feedbacks and
experience from products use history.

The fifth domain is ``Use.'' Its subdomains, as well as subdomains of all other activities arranged by the
principle ``from simple to complex'' are taken here from  classical
Maslow pyramid~\cite{2-sor}:
 \begin{enumerate}[{5.}1]
 \item  Satisfaction of physiological needs.
 \item Safety needs.
 \item Social needs realization.
 \item Esteem support.
 \item Realizing of personal potential.
 \end{enumerate}

  Intra conflict: produced items with time become morally and physically outdated and must be recycled.

 6. ``Recycling''~--- next and the last activity in the row of domains that form LC chain of activities. Its
implementation became more difficult with increasing scales of production, using modern synthetic
materials, pollution, and many other factors.

 Main stages: indiscriminate dumping of waste; disintegration technologies;
 and recycling and reuse.

 Intra conflict: environmental pollution. It can be solved by sending of extracted reusable resources  into
initial first domain with  cross-domain feedback.

 Technology development makes LC more complex, and an inherent intra conflicts increase their
vulnerability. Compensatory mechanisms bring to life two other large domains
 7.~``Lifecycles support'' and 8.~``Lifecycles update.''

 7. ``Lifecycles support'' includes business processes and technology that support existing industries and
all domains included in LC ecosystem. It has four subdomains:
 \begin{enumerate}[{7}.1]
 \item ``Learning.''
 \item  ``Management and organization.''
 \item ``LC services.''
 \item  ``Meta-security.''
 \end{enumerate}

 7.1\ ``Learning.'' Information shared through communication channels has to be analyzed. Analysis results
in the new knowledge. Large number of professions is associated with the knowledge accumulation and
sharing. Development is moving from primary artifacts in the form of oral tradition, education, and then,
after invention of writing, towards book printing, calculation, research, establishment of educational and
research institutions, lecturing by radio and television, knowledge bases/expert systems,
computer-aided
learning systems, social networks, virtual universities, and, finally,
to virtual labs and universities. However,
knowledge consumption must be effective and purposely managed. Later stages include knowledge
socialization where new role functions appeared:  leaders, experts, facilitators, and others that support
the process of knowledge exchange in networked social groups. They also need to be managed.

 7.2 ``Management and organization.'' Every domain has its specialties in division
 of labor processes. It increases performance of society in general. On top
 of this division, management plays special role to
improve effectiveness of every domain by better organization and resources utilization.

 Initial stages here are: management of row information, communications management,
 and knowledge
consuming management. Then, path of evolution goes to emergence of organizational skills, processes
management, organizations structures management, material objects control
(tools, machines), asset
management, and
territories and global structures governing (states, transnational corporations).

 Improving social performance by organizational means quickly achieves
 the limits where performance stops growing. To complete the
 mission of forth domain, additional instruments are needed. Technology, innovations,
and inventions are these tools. Needs in their development are calling for
life next area of activity.

 7.3\ ``LC services.'' This subdomain of meta-activity provides B2B
 (Business-to-Business) is services that differ than B2C
 (Business-to-Customer) serviced
which are producing by subdomain~3.2. Technology expansion is growing from repair on demand to
subscription for services, CALS (Continuous Acquisition and Lifecycle Support),
web-services for LC updating, and embedded self-services.

 7.4\ ``Meta-security.'' Very important subdomain that defends all other
 activities from various threats.
Collapse and damage to any of its elements are
fraught with losses and even economic or social disasters.
That is why, all other domains contribute to its maintenance, and, in its turn, domain provides a~feedback to
all other stack's elements. Milestones and specific implementations are: the emergence of the concepts of
society, property, including intellectual property, and from here onwards~--- personal rights and duties,
emergence of law and legislation, tools for protection of family and possessions, health, judiciary, police
and security guards, armed forces, international law, and cyber defense.

 Intra conflict: multiple LC, heterogeneous technologies and manufactured products cannot be
effectively organized by supporting tools. Quality of training, maintenance,
management, and security
measures often follows by incidents that show backlogs off requirements to LC sustainability. This
demands permanent improvements and innovations. For such purposes, the ecosystem of LC evolution
contends 8th domain~--- ``Lifecycles update'' that is dedicated to modernize, renovate, optimize, and
integrate LC.

 Unlike ``Lifecycle support,'' this domain's technologies replace old LC
 by the new ones or, at least, introduce
new elements and provide LC adjustment and optimization. It also contents three main subdomains:
 \begin{enumerate}[{8}.1]
 \item ``Meta-design.''
 \item ``Meta-automation.''
 \item ``Meta-economy.''
 \end{enumerate}

 8.1\ ``Meta-design.'' In context of this paper, the term is associated with innovative and inventive activity,
the result of which is modernization and replacement domains~(1--6) with new and modern components. It
also means design of LC, their integration and optimization.

 Accumulated knowledge as well as managerial and organizational skills allow invent and produce tools
of increasingly perfect design on industrial scale. Its creative stages and expansion phenomena in some
senses repeat the picture of development in ``Management'' for current domain's
purpose and is also directed to
improvement of technological culture. Very important inventions were made in basic domains starting
progress with inventions in data collection (computer images that represent designing objects, etc.),
communications (mapping, navigation instruments, etc.),
and knowledge accumulation (writing, printing, etc.).

 Further evolution goes through inventions of organizing technologies, towards specialization of industries
and crafts, drawing, technology development and manufacturing processes of these objects, production of
technology equipment, automation, and virtualization of manufacturing.

 8.2\ ``Meta-automation.'' Unlike single IS implementations, ``Meta-automation'' belongs
to a~class of ``system of systems'' for it affects not only separate business processes but
the parts of the whole
domains. It can be regarded as an instrument for domains efficiency management; however, unlike
economy tools, ``Meta-automation'' makes it with different means.

Initially, automation pasts the simplest form allowing reduction of LC cost by replacing manual labor. Then, it
goes in a~way of creation an artificial environment that increases business processes performance in all
others domains. That is the reason why evolving stages of automation repeat sequence of described human
activities expansion. Evolution started with the first milestone that represents the concepts and initial
prototypes of IS and databases, which at one time were considered as main computer
applications for indefinitely long term. According to some sources, in 1962, American company System
Development Corporation first coined the term database ({\sf http://en.wikipedia.org/wiki/Database}).

 Next milestone was circuit switching. ARPANET project where packet switching was
 first implemented
in communications between computers became the third one.


 This key technology opened the era of networking. Next is a~milestone that marks the period when
efforts were focused on management control system (MCS) later evolved in ERP
(Enterprise Resource Planning). Boom of CAD/CAM
(computer aided design\,/\,computer aided manufacturing) evolved in CIM (computer integrated
manufacturing) and factory automation made the following states of development. It produced complete
systems including not only computer graphics, but robotic workstations, information retrieval systems,
plotters, and various versions of LAN. Seventh milestone: noticeable progress in simulation modeling of
economic processes and the development of computer models of the economy.

 Communication protocol TCP/IP (Transmission Control Protocol\,/\,Internet
 Protocol) opened new era. Speed and scale of Internet as unprecedented in human
history monster artificial environment mean a~new phase of automation~--- spread or even absorption of all
kinds of human activities by sociotechnosphere. With the implementation of Internet technologies,
 sociotechnical systems, concepts of which appeared in the 1960s~\cite{4-sor},
nowadays reached a~new level. It leveraged by deep penetration of IS and all kinds of
technology in social and organizational structures, as well as the quality of innovations.
The influence of this phenomenon will be discussed in the following sections.

 8.3\ ``Meta-economy'' is a~toolkit of economic regulation not inside but under LC. Activity that
determines behavior of industrial units.

\pagebreak

 Intra conflict: similar to the case of ``LC support'' domains, the contradictions lie in multiplicity of
LC that make it difficult to effectively coordinate all instruments of updating and innovations. Predicted
solution is the development of Meta-design technologies to leverage LC modernization.


 It should be noted that both domains (7 and~8) have also an external additional conflict between them
since stability and modernization are inherently contradictory concepts. For example, implementation of
advanced and saving car engines was periodically hampered by oil-producing companies interested in
increase consuming of petroleum for internal combustion engines. Nevertheless, economic conditions
tightening shifts the equilibrium between the domains in favor of modernization and forces manufacturers
to accelerate the transition to new LC design. Measures taken to energy saving in the current crisis
stimulated the development of ``green'' technologies and production of economic components for
electronics and lighting.

\section{Technologic Stack of~Activities}

\noindent
Stack of activities is a~model for presenting a~set of key areas of lines of business (LOB), or business
domains, in logical connection between the individual domains. In definition of ``activity'' which was
introduced above, two entities are presented~--- the set of business processes and a~variety of technologies.
For this reason, one can build up two types of stacks~--- first, the processes stack and, second, the
technological stack of activities. This couple gives an overall model for domain description. However, in
this paper, we, in the first turn, are interested in IT analysis and,
 therefore, should focus on building
a~technological stack. Then, it will be applied to exploration of Web evolution.

\begin{figure*} %fig3
\vspace*{1pt}
 \begin{center}
 \mbox{%
 \epsfxsize=120.259mm
 \epsfbox{sor-3.eps}
 }
 \end{center}
\end{figure*}

Theoretical model of technology stack is \mbox{$N$-dimensional}. In fact, the stages also have their history,
description, and content parameters, which imply the possibility of including additional elements called
states. So, three-dimensional (3D) stack will include a~description of each stage decomposed by states. For example, for stage
``construction industry'' in domain ``Production of goods and services'' (8\;$\to$\;8.4), additional chain of
stage decomposition will appear: 8.4.1~``Use of natural shelters''\;$\to$\;8.4.2~``Construction of stone and
wood''\;$\to$\;8.4.3~``Construction of the man made materials''\;$\to$\;8.4.4~``Use of 3D printers.''

 Each state may be subjected to further decomposition, etc.\ until
 the desired degree of granularity is
obtained. For studies in local business areas, it is possible to make slices of this model. For the immediate
objectives of the present study, the two dimensions of model are enough and will not sufficiently distort the
results.
{\looseness=-1

}

 Stack element (domain or milestone) affects others with \textit{direct links} and \textit{feedbacks}. Direct
links mean that progress could be measured by emergences of new milestones in higher positioned
domains. Feedbacks are measured by the progress observed in previous domains.

 When direct links and feedbacks form a~\textit{loop}, it gives an enhanced effect of activities interaction
often described as a~``\textit{revolution}.'' Thus, feedback from banks and users capital (activity
``Finance'') to the activity ``Production of goods and services'' and backward made a~loop and gave direct
effect in the creation of capitalist industry (the first industrial revolution) and a~modern market economy.
Direct links from activity~7.1. (``Learning'') to 8.2. ``Meta-design'' led to invention of computer as a~mean
of information processing. Implementation of computing in lower domains in all subsequent forms
provided an effect, often defined as the \textit{second industrial revolution}.

 Stages of technology evolution form the space that may be
 defined as ``Space of Technology'' (Fig.~3).
Technologic stack enables one also to fix the level of social and technology development. If one draws a~line
across the selected milestones, it will be a~line of development level.

 The lower level of development crosses the initial stages of domains and forms the foundation of ``Space
of technologies.'' Higher stages of domains mark the front of development in technological space.

 North-West corner of built technologic space covers basic technologies taken from nature and natural
analogs. Large scale automated systems are concentrated in the South-Eastern corner.  They leverage
technologically closed communities with artificial internal rules of existence.

 Direction from the North-West to the South-Eastern corner figuratively plays a~role of vector of
development, on which the most crucial inventions, technologies, and discoveries are located.

\section{Special Role of Automation in~``Space of~Technologies''}

\noindent
 	Mentioning this special role of meta-automation in shaping of technology,
 let make the closer look on
evolution of this domain. Web is the space for interaction. Men, machine, and system are the general parts
of such interaction between them in IT environment. Let
also fix three modern types of Web environment
in which interaction is taking place: user centered, machine dominated,
and systems integrated.

 Computing platforms and algorithms also play very important roles. But in this
 sense,
  these roles supporting and computing evolution will go in the direction
  to satisfy\linebreak\vspace*{-12pt}

  \pagebreak

%  \begin{figure*} %fig4


 \begin{center}
 \vspace*{1pt}
 \mbox{%
 \epsfxsize=77.459mm
 \epsfbox{sor-4.eps}
 }

% \vspace*{-9pt}
 {{\figurename~4}\ \ \small{Basic IT classes supporting interaction in Web}}
  \end{center}

  \vspace*{6pt}
%\end{figure*}

\addtocounter{figure}{1}

  \noindent
   the growing demands coming from
further development of interaction needs. Hence, let build classification that presents generic classes of
automation that support Web interaction using these 6~entities (Fig.~4).

\addtocounter{figure}{1}



 Constructing this classification, the rules introduced early have
 been followed: classes evolve from simple to
sophisticated~--- from left higher corner to the
right low one. And each higher class may absorb and use all
technology solutions of lower classes. And the present model is 3D that gives possibilities for classes to provide
new artifacts as vertical components of the model. Interaction tools between human and
user centered
environment set up the initial class. It is \textit{data flow} supporting technology~--- e-mail, e-messengers,
Skype, and so forth.

 Second class contents the means for intercommunion among machine and user centered environment.
They are \textit{Interface} design technology and examples include standard API
(application programming interface), speech-gesture
interfaces, Google glasses, and so on.

 The next class is \textit{Enterprise}. This type of technology intermediates between work stations (user
centered environment) and systems. Commercial ERP systems also satisfy this class definition.

 Then, class with \textit{Smart devices} is going. It helps one to operate in machine dominated area: smart
houses, machine control with embedded processors, etc.

 Evolution of machine--machine interaction is based on intelligent protocol class of technology. The
examples are SDN (software defined networking), intelligent navigation software.

 Systems managing machine class demands use of robotic conception. Artificial intelligence mechanisms
must be embedded in drones (UAV~--- unmanned aerial vehicle)), industrial and military robots for their navigation, interaction and
mission execution.

 Next step in Web interaction evolution is emerging of the new technology class that permits human to
solve complex tasks in system integrated media by use of natural language
(NLP~--- Natural Language
Processing). This class does not include relatively simple voice recognition technology related to interface
class.  It has a~deal with Q\&A (questions and answers)
system, artificial intelligence for decision-making, advanced training
systems, etc. IBM Watson is a~good example of commercial system
that occupied this position.

 Further, let come to \textit{Smart systems} class of interaction technologies permitting any machine
to be integrated and interact in systems environment. This class is a~child of fast developing cloud
technology, social networking~--- from computing side, and demands to smart consuming of resources~---
from side of economy. Examples embrace IBM series of Smarter Planet system that
will be considered in more details later on.

 The last, ninth, and the highest class is \textit{System of systems} technology. The most spectacular example of
this type is Web itself. Applying the famous analogy of Internet and real web made by spider~---
a~wonderful example of natural stress-resistant construction~---
it is possible to answer the following
question: What quality of our IT Web that is the greatest artificial system in human history, provides its
integrity instead possible fragmentation with increasing complexity?

 In the model shown in Fig.~3, one may recognize ``radial'' ``filaments'' of WWW
 (world wide web) representing
industrial/LOB directions of automation technologies development. Cross-industries connections of
automation milestones shown there represent ``spiral'' way of integration. Cloud computing,
social networks, and M2M (machine-to-machine) technologies open practically unlimited prospects for such
integration. Projects of global free wi-fi like Outernet ({\sf https://www.outernet.is/}) have to accelerate
movement toward ``spiral'' at the humanitarian (healthcare, education, etc.)\ parts of it.
Some assessments
show that 15~billion devices will be connected by 2015 (Intel) in Internet of Everything (IoE) that will
cover of \$14.4 trillion (Cisco)~\cite{10-sor}. One
may foresee that further development of AHCP (Ad-Hoc
Configuration Protocol) inside \textit{intelligent protocol} class will open the possibility of \textit{smarter
integration} when IT platforms of ``spiral'' systems could ``negotiate'' with each other and install proper ad
hoc configuration. Large potential of \textit{intellectual integration} is opened by IBM Watson. With
ability to deep intelligent search of absent and necessary system's components based on artificial
intelligence, Watson computer will be very useful assistant in design of ``spiral'' systems.

 Mentioned technologies  produce powerful synergetic effect for automation development providing
WWW as platform for further deployment of extralarge complex systems, predicted above as class~9
(see Fig.~4). Growing in scale and accumulating innovations taken from different
domains, such types of
systems will reinforce total WWW performance and lead to new web topology which is similar to weaving
of web spirals. One may call it ``spiral'' architecture, employing mentioned parallel.
And, correspondingly, such future systems will be called as Web ``spiral'' systems.

In meta-automation expansion scenario outlined here, the described 9~classes of interaction
technologies are used
as reference solutions that determine general direction of developments.	

 Each domain produces technology that tends to expand into others domains ``territory.'' As a~result of
expansion, communication technologies are observed
everywhere, e-learning is everywhere. Power
and transport are also penetrating in every part of technological space.
It will be shown that automation like
each other activity enjoys this ability but beside this, it also possesses
some special features that make it meta-activity.

 As was noted above, each domain may be decomposed into subdomains. Technological breakthroughs
corresponding to each of them can be represented as innovative milestones that determine ``vertical'' or
``radial'' directions of technology diffusion. New milestone also provides impacts to neighborhood
environment in horizontal directions through direct and feedback connections.  As a~result, all of the
domains in varying degrees contribute to the technological development of each activity. In general, it
looks like the acceleration of scientific and technological progress. Automation follows the innovations in
all directions but sometimes precedes and accelerates them.

 Taking as an example, breakthrough in nanotechnology with the graphene invention (8.2. ``Meta-design'')
leads to implementations into domains 3.1 (rechargeable battery, solar panels), 3.2 (processors), 1.1 and
7.4 (sensors), etc. Collaborative works in these areas performed onto IT integrated platform will be more
productive than process of ``natural'' technology diffusion.

 Thus, automation as a~special kind of activity, from the one hand, by means of innovations is tightly
connected with Meta-design domain and from the second~--- is a~tool for managing of economic performance
set by ``Meta-economy.'' Automation fulfills this role in two ways. First~--- in labor allocation scheme it
transfers to machine only those operations that man performs worse than computer. The second way is
integration that adds value for multitude of enterprises helping them to manage common data flows and
effectively use their huge computing resources. That, in fact, makes it possible to combine stack elements
to obtain economic benefits in new models of entrepreneurship.

 The innovative quality of automation provides results and artifacts
 that cannot be achieved without
automation. The most famous examples are robots, 3D printers, and calculations of satellites flights
trajectories.  Cognitive computing, cloud technology, and M2M are taking out automation beyond the role
of just service tool that helps to solve different tasks inside LC processes.
 With latest advent, automation
obtains the ability not just redesign and improves LC themselves but to invent new LC for new
products. In accordance with such new quality, this activity can be called
 the ``Meta-automation'' as
a~part of ``Update of lifecycles'' domain, ensuring LC design, renewal, and integration.

\section{Steps to ``Spiral'' Systems}

\noindent
 Let consider main properties of the ``spiral'' systems. First of all:
 \begin{itemize}
\item ``Spiral'' is a~``system of systems'' component of Web architecture and more advanced form in
evolution of Web hosting and data centers;\\[-14pt]
\item ``Spiral'' system belongs to different owners and supports multitenant mode;\\[-14pt]
\item ``Spiral'' integrates self-organized communities: emerge
collaborative projects and
maintain and implement their outcomes with the help of rented instruments and resources;\\[-14pt]
\item ``Spiral'' system projects management is conducted in a~virtualized environment (domain
``Management and Organization'') and also may attract social lending alongside with traditional financial
sources;\\[-14pt]
\item ``Spiral'' integrates, first, larger parts of LC: information
and research, economics and finance, and
design and production as secured distributed clods; and\\[-14pt]
\item ``Spiral'' platforms predominantly provide utility computing in the mode of EaaS (Everything as
a~Service).
 \end{itemize}

 Deployment corresponding architecture of Web  brings forth the task of new design methods or even new
approaches to Internet topology.

 The constructed model allows to look at automation with more general civilization positions, defining
its place and role in overall progress.

 For qualitative assessment of possible value for synergetic effect of automation, it
  is necessary to involve some economic considerations.

  \begin{figure*} %fig5
\vspace*{1pt}
 \begin{center}
 \mbox{%
 \epsfxsize=162.416mm
 \epsfbox{sor-5.eps}
 }
 \end{center}
 \vspace*{-11pt}
\Caption{World economy model according IBM Institute for Business Value:
\textit{1}~--- same industry; \textit{2}~--- business support;
\textit{3}~--- IT systems; \textit{4}~--- energy resources; \textit{5}~---
machinery; \textit{6}~--- materials; and \textit{7}~--- trade}
\vspace*{-3pt}
\end{figure*}

 IBM Institute for Business Value published a~study model of the world economy representing 11~core
systems~\cite{5-sor} (Fig.~5). These core systems include: Infrastructure, Electricity, Finance, Education,
Food, \mbox{Communications}, Water Resources, Transportation, Entertainment, Fashion, Leisure, Health,
Governance, and Security. In Fig.~5, each core is represented by a~bubble whose value is proportional to
the size of the economy. For scaling, a~bubble of 1 trillion dollars is shown in the lower right corner. The
model has the fractal properties. Production, business, IT systems, engineering, energy resources, materials,
and trade are incorporated in each core system as components.
In many cases, the component's name and the name of
core system may be the same. The components inside cores are also indicated by scalable
bubbles. Arrows, whose width characterizes the degree of influence of one core system to another, show
contribution of all core systems in the operation of each separate one.

IBM analysts based on regression economic models argue that the total loss of world economy is estimated
as 15~trillion dollars resulted due to inefficiency. They also found that the loss of at least 4~trillion dollars
could be prevented through more rational organization and use of modern information technologies.
Certain part of such losses occurs at the level of ``system of systems'' as poor integration between the cores.
Executive Report made by Korsten and Seider was published by IBM in 2010 in the period
when global economic crisis has begun~\cite{5-sor}. At that time,
IBM developed and presented the ``Smarter Planet''
initiative~\cite{6-sor} as a~technological response to the crisis. The purpose of this initiative was to
increase value delivered to the end users with the help of~``smart'' IS aimed at the
automation of complex activities that approximately correspond to system cores in the economic model
described above. Such systems were labeled as ``Smarter Grid,'' ``Smarter Healthcare,'' ``Smarter Finance,''
``Smarter Work,'' and so on.
%Their design should provide cost savings and effective business processes in
%the target domain (power production, customer services, water resources, etc.)\ and in the maintenance of
%information systems themselves (green technology, dynamic infrastructure). These open the era of cloud
%computing and social networks.




 Their design should provide cost savings and effective business processes
 in the target domain (power
production, customer services, water resources, etc.)\ for interdomains integration and building of
heterogeneous systems at a~very broad scale. They can be considered
as the first steps to Web-``spiral'' systems of systems (class~8, see Fig.~4).

 Platforms of this series are different in structure, but have many common characteristics:
 \begin{itemize}
  \item  high degree of security provided by software products of IBM Tivoli family;
  \item dynamic infrastructure supported by cloud technologies;
\item meta-models and tools for business processes management  that make possible  quick adaption to
changes in business environment;
\item ability to accumulate knowledge and provide access to cognitive resources with help of social tools;
\item high scalability and mobile access to services; and
\item good  performance records (cost savings and use of ``green technologies'' reducing total cost of
ownership).
  \end{itemize}

 Platforms presented in LC ecosystem (see Fig.~2) are: Smarter SCADA for Oil and Gas (1.~Access to
resources), Smarter Transportation (2.~Transfer); IBM Smart Grid and Rational software platform for
automotive systems (3.~Production); IBM Smarter Commerce (4.~Distribution);
and IBM i2, IBM
Defense Operations Platform, and IBM i2 Defense Solution (7.~LC support).

 Early examples of complex automation corresponding to ``spiral'' concept were DiFac project launched in
second framework program of the European Commission~\cite{7-sor}
and BioVLAB~\cite{8-sor}.

DiFac is a~complex sociotechnical system designed to boost both economic efficiency and performance of
human labor in production area. Collaborative participants of the project developed methods for industrial
control, interaction in the network team, whose members are located in different countries and were
connected through 3D virtual reality.

 BioVLAB is a~cloud environment for microRNA and mRNA (ribonucleic acid) integrated analysis (MMIA)
on Amazon EC2. It makes vast amount of microRNA expression profile data publicly available.  BioVLAB
is positioned by its developers as an easy-to-use computing environment for researchers who plan to
perform genome-wide integrated analysis tasks with advanced features:
 \begin{itemize}
  \item readily expanded computational tools;\\[-14pt]
  \item easily modifiable by reconfiguring in the workflow;\\[-14pt]
  \item on-demand cloud computing resources; and\\[-14pt]
  \item  distributed orchestration supports complex and long running workflows asynchronously.
  \end{itemize}

Special place in the row is occupied by IBM Intelligent Operations Center (IOC) introduced by IBM as
a~part of Smarter Planet initiative. IBM IOC had been applied in many target areas of Smarter Planet in
a~purpose to integrate and use data from multiple sources and present results of their processing in single
interface. Covered sources may belong to absolutely different domains of activity and this complex
integration permits to monitor and manage their states and support operative decisions. Data processing and
decision-making use advanced analytics, asset management, and collaboration tools. Smarter City is one of
the most complex and promising platforms introductions in modern Web. Perhaps, IBM IOC is the largest
commercial solutions currently distributed at the IT market. It provides
the following functions~\cite{9-sor}:
\begin{itemize}
  \item visual workspace;\\[-14pt]
  \item events and incident management;\\[-14pt]
  \item resource, response, and activity management;\\[-14pt]
  \item status monitoring;\\[-14pt]
  \item  collaboration, instant notification, and messaging;\\[-14pt]
  \item  reports; and\\[-14pt]
  \item semantic model.
  \end{itemize}

System's architecture includes multilevel SOA (service-oriented architecture)
structure, power infrastructure based on IBM Tivoli
software, including clouds and system security. Key performance indicator
managed dashboard uses event management and
workflows engine to react on real-world situation
and to keep specified policy and performance level.

 All above mentioned systems are designed for collaborative works performed by legally independent
or-\linebreak\vspace*{-12pt}

\columnbreak

\noindent
ganizations acting as a~single Web alliance. As such, they meet the definition of a~virtual enterprise
and the 9th class ``system of systems'' as well. Consequently, ``spiral'' systems may also be regarded as
virtual organization of next generation.

\vspace*{-6pt}

\section{Concluding Remarks}

\vspace*{-2pt}

\noindent
\begin{enumerate}
 \item  Further economy development, as it follows from the IBM Institute for Business Value model will
be not for intensification of natural resources consumption but for losses reduction. This requires the new
type of Web systems and Web architecture with the ability to automate not just business processes but
domain of activities.\\[-14pt]
 \item Design of this type of information systems based on classical approach that automation eliminates
bottlenecks in business process does not work and to BPM (business process management) must be added
AMS (activity management system).\\[-14pt]
 \item Analysis of technologic stack and requirements of the modern economy permits
 to expect with
a~high probability that new type of IS
for domain automation conditionally defined as
``spiral'' will evolve in the direction responding the introduced requirements.
 \end{enumerate}

\renewcommand{\bibname}{\protect\rmfamily References}

\vspace*{-6pt}

{\small\frenchspacing
{%\baselineskip=10.8pt
\begin{thebibliography}{99}

\vspace*{-2pt}

\bibitem{1-sor}
Gartner hype cycle. Available at: {\sf http://www.gartner.\linebreak
com/technology/research/methodologies/hype-cycle. jsp}
(accessed November~21, 2014).

\bibitem{3-sor} %2
News tagged with metamaterials. Available at:
{\sf http:// phys.org/tags/metamaterials/} (accessed November~21, 2014).

\bibitem{2-sor} %3
\Aue{Maslow, A.} 1954. \textit{Motivation and personality}. New York, N.Y.: Harper\&Row Publs.
Inc. 15--31.

\bibitem{4-sor}
\Aue{Emery, F.\,E., and E.\,L.~Trist}. 1960. Socio-technical systems. \textit{Management science, models
and techniques}. Eds.\  C.\,W.~Churchman and  M.~Verhurst.
London: Pergamon Press. 2:83--97.

\bibitem{10-sor} %5
Internet of things market forecast. Available at:
{\sf http://\linebreak postscapes.com/internet-of-things-market-size}
(accessed November~21, 2014).

\bibitem{5-sor} %6
\Aue{Korsten,~P., and Ch.~Seider.} 2010.
The world's 4 trillion dollar challenge: Using a~system-of-systems approach to build a~smarter
planet.  IBM
Institute for Business Value.  IBM Global Business Services Executive Report.
Available at: {\sf
http://www-05.ibm.com/tr/events/\linebreak ibmcozumlerzirvesi2011/pdf/GBE03278USEN.PDF} (accessed June~17,
2014).

\bibitem{6-sor} %7
IBM Smarter Planet publications. Available at:\linebreak {\sf
http://www.ibm.com/smarterplanet/us/en/overview/\linebreak
 ideas/index.html?re=sph};
  {\sf http://www.ibm.com/\newline smarterplanet/ru/ru/};
% \vspace*{-12pt}
{\sf http://en.wikipedia.org/wiki/}

\pagebreak

\noindent

{\sf  Smarter\_Planet};
{\sf http://www.ibm.com/smarterplanet/}
{\sf us/en/?ca=v\_smarterplanet} (accessed June~17, 2014).
\bibitem{7-sor} %8
DiFac success story. {\sf
http://www.ims.org/wp-content/\linebreak
uploads/2012/03/DiFac-SUCCESS-STORY\_100917.\linebreak pdf} (accessed
June~17, 2014).
\bibitem{8-sor} %9
\Aue{Lee, H., Y.~Yang, H.~Chae, S.~Nam, D.~Choi, P.~Tangchaisin, C.~Herath, S.~Marru,
K.\,P.~Nephew, and S.~Kim.} 2012.
BioVLAB-MMIA: A~cloud environment for microRNA and mRNA integrated analysis (MMIA) on
Amazon EC2. \textit{IEEE Trans. Nanobiosci.} 11(3):266--272. doi: 10.1109/TNB.2012.2212030.
\bibitem{9-sor}
IBM Corp., International Technical Support Organization.
November~15, 2012.
IBM Intelligent Operations Center for Smarter Cities.
IBM Redbooks Solution Guide.



\end{thebibliography} } }

\end{multicols}

\vspace*{-9pt}

\hfill{\small\textit{Received June 10, 2014}}

\vspace*{-24pt}

\Contrl

\noindent
\textbf{Sorokin Alexander V.} (b.\ 1946)~---
Candidate of Science (PhD) in technology,
University Relations Manager for Russia \& CIS, IBM EE/A;
asorokin27@gmail.com

\vspace*{8pt}

\hrule

\vspace*{2pt}

\hrule

\vspace*{-6pt}

%\newpage


\def\tit{АВТОМАТИЗАЦИЯ ЗА ПРЕДЕЛАМИ WEB 2.0}

\def\aut{А.~Сорокин}


\def\titkol{Автоматизация за пределами Web 2.0}

\def\autkol{А.~Сорокин}

%{\renewcommand{\thefootnote}{\fnsymbol{footnote}}
%\footnotetext[1]{Работа проводится при финансовой поддержке Программы
%стратегического развития Петрозаводского государственного университета в рамках
%на\-уч\-но-ис\-сле\-до\-ва\-тель\-ской деятельности.}}


\titel{\tit}{\aut}{\autkol}{\titkol}

\vspace*{-12pt}

\noindent
IBM EE/A, Пресненская наб. 10, Москва 123317, Россия

\vspace*{6pt}

\def\leftfootline{\small{\textbf{\thepage}
\hfill ИНФОРМАТИКА И ЕЁ ПРИМЕНЕНИЯ\ \ \ том\ 8\ \ \ выпуск\ 4\ \ \ 2014}
}%
 \def\rightfootline{\small{ИНФОРМАТИКА И ЕЁ ПРИМЕНЕНИЯ\ \ \ том\ 8\ \ \ выпуск\ 4\ \ \ 2014
\hfill \textbf{\thepage}}}



\Abst{Рассматривается новый подход к анализу эволюции информационных
систем, основанный на разработанном автором стеке активностей. С~помощью введенного
подхода исследуются перспективные тенденции построения на платформе Вэб  информационных
систем, которые начинаются с~автоматизации отдельных биз\-нес-про\-цес\-сов и~затем,
в~результате дальнейшей экспансии информационных технологий (ИТ), охватывают области профессиональной деятельности.
В~результате классический подход к~проектированию информационных систем, базирующийся на
устранении посредством автоматизации узких мест биз\-нес-про\-цес\-сов, перестает работать.
Текущие тенденции в~развитии ИТ, связанные с~новыми возможностями <<ортогональной>>
интеграции систем, делают вероятным появление нового типа больших информационных систем и
нового типа их Вэб-ар\-хи\-тек\-ту\-ры, условно названного в~данной работе <<спиралью
паутины>>. По сравнению с~<<радиальной>> интеграцией Вэб в~рамках одной
профессиональной области такой тип архитектуры является более эффективным.}

\KW{автоматизация; реинжиниринг бизнес-процессов;
совместная разработка программных продуктов; экономика масштабирования;
ин\-тер\-нет-то\-по\-ло\-гия; социотехнические системы; системы систем;
виртуальные предприятия; Вэб 2.0}

\DOI{10.14357/19922264140414}

%\vspace*{6pt}


 \begin{multicols}{2}

\renewcommand{\bibname}{\protect\rmfamily Литература}
%\renewcommand{\bibname}{\large\protect\rm References}

{\small\frenchspacing
{%\baselineskip=10.8pt
\begin{thebibliography}{99}


\bibitem{1-sor-1}
Gartner's hype cycle. {\sf http://www.gartner.com/ technology/research/methodologies/hype-cycle.jsp}.

\bibitem{3-sor-1} %2
News tagged in metamaterials.
 {\sf http://phys.org/tags/ metamaterials/}.

 \bibitem{2-sor-1} %3
\Au{Maslow A.}
{Motivation and personality}. New York, N.Y.: Harper\&Row Publs.
Inc., 1954. P.~15--31.

\bibitem{4-sor-1}
\Au{Emery F.\,E., Trist E.\,L.} Socio-technical systems~// \textit{Management science, models
and techniques}~/ Eds.\  C.\,W.~Churchman and  M.~Verhurst.~---  London: Pergamon Press,
 1960. Vol.~2. P.~83--97.

 \bibitem{10-sor-1} %5
Internet of things market forecast.
{\sf http://postscapes. com/internet-of-things-market-size}.

\bibitem{5-sor-1} %6
\Au{Korsten~P., Seider~Ch.}
 The world's 4~trillion dollar challenge. Using a~system-of-systems approach to build a~smarter
planet.
 IBM Institute for Business Value, 2010.
 IBM Global Business Services Executive Report. {\sf
http://www-05.ibm.com/tr/events/ ibmcozumlerzirvesi2011/pdf/GBE03278USEN.PDF}.
\bibitem{6-sor-1} %7
IBM Smarter Planet publications. {\sf
http://www.ibm. com/smarterplanet/us/en/overview/ideas/index.html? re=sph};
{\sf http://www.ibm.com/smarterplanet/ru/ru/};
{\sf http://en.wikipedia.org/wiki/Smarter\_Planet};
{\sf http://www.ibm.com/smarterplanet/us/en/?ca=v\_\linebreak smarterplanet} (accessed June~17, 2014).
\bibitem{7-sor-1}
DiFac Success story. {\sf
http://www.ims.org/wp-content/uploads/2012/03/DiFac-SUCCESS-STORY\_\linebreak 100917.pdf}.

\bibitem{8-sor-1}
\Au{Lee H., Yang Y., Chae H., Nam S., Choi D., Tang\-chai\-sin~P., Herath~C., Marru~S.,
Nephew~K.\,P., Kim~S.}
BioVLAB-MMIA: A~cloud environment for microRNA and mRNA integrated analysis (MMIA) on
Amazon EC2~// IEEE Trans. Nanobiosci., 2012. Vol.~11. No.\,3. P.~266--272.
doi: 10.1109/TNB.2012.2212030.
\bibitem{9-sor-1} %10
IBM Corporation, International Technical Support Organization.
IBM Intelligent Operations Center for Smarter Cities.
IBM Redbooks Solution Guide. November~15, 2012.


\end{thebibliography}
} }

\end{multicols}

 \label{end\stat}

 \vspace*{-6pt}

\hfill{\small\textit{Поступила в редакцию 10.06.2014}}
%\renewcommand{\bibname}{\protect\rm Литература}
\renewcommand{\figurename}{\protect\bf Рис.}