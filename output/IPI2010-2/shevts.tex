
%\renewcommand{\r}{\mathbb{R}}

\newcommand{\abs}[1]{\left|#1\right|}
\newcommand{\ex}{C_0}
%\newcommand{\lowex}{C_1}
\newcommand{\exlow}{C_1}
\newcommand{\exlowk}{C_k}
\renewcommand{\le}{\leqslant}
\renewcommand{\ge}{\geqslant}
\renewcommand{\d}{\delta}
\newcommand{\bet}{\beta_{2+\delta}}

\newcommand{\gd}{\gamma(\delta)}
\newcommand{\kd}{\varkappa(\delta)}
\newcommand{\sign}{\mathrm{sign}\,}
\newcommand{\R}{\mathbb R}
\newcommand{\C}{\mathbb C}
\newcommand{\To}{\longrightarrow}

\newcommand{\ud}{\rho(F_n,\Phi)} %uniform distance


\def\stat{sevts}

\def\tit{УТОЧНЕНИЕ НЕРАВЕНСТВА КАЦА--БЕРРИ--ЭССЕЕНА$^*$}

\def\titkol{Уточнение неравенства Каца--Берри--Эссеена}

\def\autkol{М.\,Е.\,Григорьева, И.\,Г.~Шевцова}
\def\aut{М.\,Е.\,Григорьева$^1$, И.\,Г.~Шевцова$^2$}

\titel{\tit}{\aut}{\autkol}{\titkol}

{\renewcommand{\thefootnote}{\fnsymbol{footnote}}\footnotetext[1]
{Работа выполнена при поддержке Российского фонда фундаментальных
исследований (проекты 08-01-00563, 08-01-00567, 08-07-00152 и
09-07-12032-офи-м), а также Министерства образования и науки
(государственные контракты П1181 и П958, грант МК-581.2010.1).}}

\renewcommand{\thefootnote}{\arabic{footnote}}
\footnotetext[1]{Московский
государственный университет имени М.\,В.~Ломоносова, факультет
вычислительной математики и кибернетики, maria-grigoryeva@yandex.ru}
\footnotetext[2]{Московский государственный университет
имени М.\,В.~Ломоносова, факультет вычислительной математики и
кибернетики, ishevtsova@cs.msu.su}

\Abst{Уточнены верхние оценки абсолютной константы в
неравенстве Каца--Берри--Эссеена для сумм независимых одинаково
распределенных случайных величин с конечными абсолютными моментами
порядка $2+\d$, $0<\d<1$. Предложена альтернатива неравенству
Каца--Берри--Эссеена, имеющая более тонкую структуру, и построены
верхние оценки входящих в уточненное неравенство констант.}

\KW{центральная предельная теорема; неравенство
Каца--Берри--Эссеена; дробь Ляпунова}

       \vskip 18pt plus 9pt minus 6pt

      \thispagestyle{headings}

      \begin{multicols}{2}

      \label{st\stat}
  

\section{Введение}

При решении многих прикладных задач приходится учитывать эффекты,
возникающие в результате суммарного воздействия большого числа
случайных факторов, отдельный вклад каждого из которых в сумму
пренебрежимо мал. Чаще всего в таких ситуациях статистические
закономерности поведения суммы в силу центральной предельной теоремы
аппроксимируются нормальным распределением вероятностей. При этом
точность нормальной аппроксимации зависит от наличия у случайных
слагаемых моментов достаточно высокого порядка или, другими
словами, тяжестью (или легкостью) их <<хвостов>>. Известно, что при
некоторых достаточно общих условиях нормальная аппроксимация
адекватна, если случайные слагаемые имеют моменты хотя бы второго
порядка, причем чем выше порядок момента, тем, как правило, выше
точность нормальной аппроксимации. При этом большой интерес
представляет ситуация, когда случайные слагаемые имеют моменты,
порядок которых заключен между двумя и тремя: с одной стороны, для
распределений, имеющих моменты порядка, большего трех, скорость
сходимости в центральной предельной теореме остается в общем случае
такой же, как для распределений с третьими моментами; с другой
стороны, во многих прикладных задачах важно оценивать точность
нормальной аппроксимации, когда центральная предельная теорема все
еще выполняется, но слагаемые имеют распределения со столь тяжелыми
<<хвостами>>, что третьего момента уже не существует. Такие задачи
возникают, например, в страховании, когда речь заходит о
маловероятных, но экстремально больших выплатах по тому или иному
страховому случаю. Другие примеры связаны с практическим применением
моделей типа распределения Парето с <<хвостами>>, убывающими степенным
образом, при анализе трафика в телекоммуникационных сис\-те\-мах. Часто
статистический анализ таких моделей позволяет сделать вывод, что
показатель степени заключен между тремя и четырьмя, т.\,е.\
дисперсия существует, а третий момент отсутствует. Улучшению оценок
точности нормальной аппроксимации именно для таких ситуаций и
посвящена данная работа.

Для $0\le\d\le 1$ обозначим через $\F_{2+\d}$ множество функций
распределения с нулевым средним, единичной дисперсией и конечным
абсолютным моментом $\bet$ порядка ${2+\d}$. При $\d=0$ полагаем
$\beta_2=1$ и $\F_2$~--- класс всех распределений с нулевым средним и
единичной дисперсией. Пусть $X_1,\ldots,X_n$~--- независимые
одинаково распределенные случайные величины с общей функцией
распределения $F\in\F_{2+\d}$, заданные на некотором вероятностном
пространстве $(\Omega,\mathcal{A},\p)$. Обозначим
\begin{align*}
F_n(x)&=F^{*n}(x\sqrt{n})=
\p\left(\fr{X_1+\ldots+X_n}{\sqrt{n}}<x\right)\,;
\\
\Phi(x)&=\fr{1}{\sqrt{2\pi}}\int\limits_{-\infty}^xe^{-t^2/2}\,dt\,,\quad
x\in\R\,.
\end{align*}

Классическое неравенство Каца--Берри--Эс\-се\-ена устанавливает
существование конечной положительной постоянной $\ex=\ex(\d)$,
зависящей только от~$\d$, которая гарантирует справедливость
неравенства

\noindent
\begin{equation}
\!\left.
\begin{array}{rl}
\ud &\equiv\sup_x|F_n(x)-\Phi(x)|\le \ex(\delta)
L_n^{2+\delta}\,;\\[6pt]
 L_n^{2+\delta}&=\fr{\bet}{n^{\delta/2}}
 \end{array}\!
 \right \}\!\!\!\!\!
\label{Bikelis}
\end{equation}
для всех $n\ge1$ и $F\in\F_3$. 

Для  $\d=1$
неравенство~(\ref{Bikelis}) было доказано независимо и одновременно
Э.~Берри~\cite{Berry1941} и К.-Г.~Эссе-\linebreak еном~\cite{Esseen1942}. В~1960-е~гг.\
разными авторами были предприняты успешные попытки обобщить\linebreak
результат Берри--Эссеена. Так, в 1963~г.\ М.~Кац~\cite{Katz1963}
доказал аналог~(\ref{Bikelis}) для независимых одинаково
распределенных случайных величин с $\e X_1^2g(X_1)<\infty$ для
функций $g$ из некоторого класса, включающего $g(x)=|x|^\d$. 
В~1965~г.\ В.\,В.~Пет\-ров~\cite{Petrov1965} обобщил неравенство Каца
на разнораспределенные слагаемые. В~1966~г.\
А.~Бикялис~\cite{Bikelis1966} доказал неравномерную оценку для
разнораспределенных случайных величин, имеющих конечные абсолютные
моменты порядка $2+\d$, $0<\d\le1$, из которой также вытекает
неравенство~(\ref{Bikelis}). Точные формулировки  упомянутых
результатов вместе с их доказательствами можно найти, например, в
монографии В.\,В.~Петрова~\cite{Petrov1972}.

Относительно константы $\ex(1)$ известно, что
$$
\fr{\sqrt{10}+3}{6\sqrt{2\pi}}\le\ex(1)\le 0{,}4784
$$
(нижняя оценка найдена К.-Г.~Эссееном~\cite{Esseen1956}, верхняя~--- В.\,Ю.~Королевым и 
И.\,Г.~Шевцовой~\cite{KorolevShevtsova2010}).

Верхние оценки величины $\ex(\d)$ при некоторых $0<\d<1$ впервые
были получены в 1983~г.\ В.~Тысиаком~\cite{Tysiak1983} 
(см.\ также~\cite{Paditz1996}) и недавно были уточнены в 
работе~\cite{GaponovaKorchaginShevtsova2009}. В 1986~г.\
Г.~Падитц~\cite{Paditz1986} показал, что для всех $F\in\F_2$ и
$n\ge1$ имеет место неравенство
$$
\rho(F_n,\Phi)\le 3{,}51{\e}\left(X_1^2\min\left\{1,
\fr{|X_1|}{\sqrt{n}}\right\}\right)\,,
$$
откуда вытекает равномерная по $\delta\in[0,1]$ оценка
$\ex(\delta)\le 3{,}51$, так как при любом $\delta\in[0,1]$ выражение
в правой части последнего неравенства не превосходит $3{,}51\cdot
L_n^{2+\delta}$.

Несмотря на то, что со времени первой публикации верхних оценок
прошло более 25~лет, нижние оценки для величины $\ex(\d)$ получены
совсем недавно (см.~\cite{Shevtsova2010}). Перечисленные
оценки указаны в табл.~1: во втором
столбце~--- верхние оценки Тысиака~\cite{Tysiak1983}, в третьем~---
верхние оценки из работы~\cite{GaponovaKorchaginShevtsova2009}, в
пятом~--- нижние оценки из~\cite{Shevtsova2010}; в четвертом же
столбце указаны новые оценки, доказанные в данной работе. Для
удобства в первой строке таблицы приведен год соответствующей
публикации.


Из табл.~1 видно, что представленные в данной работе верхние оценки
константы $\ex(\d)$ не очень далеки от неулучшаемых: зазор между
найденной\linebreak\vspace*{-12pt}
\columnbreak

%\bigskip

%\begin{center} %tabl1
\noindent
{{\tablename~1}\ \ \small{История верхних оценок, а также нижние оценки
константы $\ex(\d)$}}
%\end{center}
\vspace*{2pt}

{\small \begin{center}
\tabcolsep=11.8pt
\begin{tabular}{|c|c|c|c|c|}
\hline
Год & 1983  & 2009 & 2010 &   2010  \\
\hline
$\d$ & $\ex \le$ & $\ex\le$ & $\ex\le$ & $\ex\ge$ \\
\hline
0,9 & 0,802  & 0,7671 & 0,5383 &   0,2133  \\
0,8 & 0,812  & 0,7720 & 0,5723 &   0,2245  \\
0,7 & 0,833  & 0,7876 & 0,6026 &   0,2376  \\
0,6 & 0,863  & 0,8126 & 0,6276 &   0,2530  \\
0,5 & 0,902  & 0,8454 & 0,6413 &   0,2715  \\
0,4 & 0,950  & 0,8876 & 0,6342 &   0,2939  \\
0,3 & 1,008  & 0,9407 & 0,6195 &   0,3220  \\
0,2 & 1,076  & 1,0001 & 0,6094 &   0,3585  \\
0,1 & 1,102  & 1,0739 & 0,6028 &   0,4092  \\
\hline
\end{tabular}
\end{center}
}
%\vspace*{6pt}


\bigskip

\begin{center} %tabl2
\noindent
\parbox{56mm}{{\tablename~2}\ \ \small{Двусторонние оценки
константы $\exlow(\d)$}}
%\end{center}
\vspace*{4pt}

{\small 
\tabcolsep=16pt
\begin{tabular}{|c|c|c|c|c|}
\hline \vphantom{$\frac{\displaystyle R}2$}
  $\d$ & $\exlow\le$ & $\exlow\ge$ \\
\hline
0,9 &0,3089 &0,0323 \\
0,8 &0,3187 &0,0356 \\
0,7 &0,3334 &0,0396 \\
0,6 &0,3528 &0,0444 \\
0,5 &0,3775 &0,0503 \\
0,4 &0,4080 &0,0575 \\
0,3 &0,4450 &0,0665 \\
0,2 &0,4901 &0,0780 \\
0,1 &0,5451 &0,0939 \\
  \hline
\end{tabular}
}
\end{center}

\addtocounter{table}{2}

\bigskip

\noindent
 мажорантой и соответствующей минорантой со\-став\-ля\-ет всего
0,2--0,3, а их отношение колеблется в пределах 1,5--2,5 в
зависимости от~$\d$.


Наряду с уточнением константы $\ex(\d)$ в~(\ref{Bikelis}) данная
работа ставит своей целью уточнение и самой структуры
неравенства~(\ref{Bikelis}). А~именно, в качестве альтернативы
предлагается рассмотреть неравенство
\begin{equation}
\label{K-B-E-sharpened}
\ud\le \exlow(\d)\fr{\bet+1}{n^{\d/2}}\,,\enskip n\ge1\,,\
F\in\F_{2+\d}\,,
\end{equation}
справедливость которого с некоторым положительным и конечным
$\exlow(\d)$ вытекает тривиальным образом из~(\ref{Bikelis})
(например, с $\exlow(\d)=2\ex(\d)$ в силу условия $\bet\ge1$).
Однако константа $\exlow(\d)$ в~(\ref{K-B-E-sharpened})\linebreak
 оказывается
гораздо меньше, чем $\ex(\d)$ в~(\ref{Bikelis}), поэтому
неравенство~(\ref{K-B-E-sharpened}) при достаточно больших~$\bet$
дает оценку, заведомо лучшую, чем~(\ref{Bikelis}), несмотря на то
что для его справедливости необходима та же априорная информация о
распределении~$F$ (а именно, только значение абсолютного момента~$\bet$). 
Кроме того, более оптимистичными оказываются и нижние
оценки~$\exlow(\d)$, построенные в работе~\cite{Shevtsova2010}.
Например, для $\d=1$ двусторонняя оценка имеет вид
$0{,}2659\le\exlow(1)\le 0{,}3041$~[13--16]. Для $0<\d<1$ верхние
оценки константы~$\exlowk(\d)$, устанавливаемые в данной статье, и
нижние, полученные в
работе~\cite{Shevtsova2010}, приведены в
табл.~2 во втором и третьем
столбцах соответст\-венно.


Рассуждения, приводящие к форме неравенства~(\ref{K-B-E-sharpened}),
основаны на используемых оценках для характеристических функций и
подробно описаны в~\cite{KorolevShevtsova2010, KorolevShevtsova2009}.

\section{Главный результат и~основные идеи его доказательства}

\noindent
\textbf{Теорема 1.}
\textit{Для константы $\exlowk(\d)$ в неравенстве}
$$
\ud\le
\exlowk(\d)\fr{\bet+k}{n^{\d/2}}\,,\quad n\ge1\,,\ F\in\F_{2+\d}\,,
$$
\textit{при $k=0$ и $k=1$ имеют место оценки, приведенные в
табл.}~3.

\medskip

\begin{center} %tabl1
\noindent
\parbox{56mm}{{\tablename~3}\ \ \small{Верхние оценки
констант $\ex(\d)$ и $\exlow(\d)$}}
%\end{center}
\vspace*{4pt}

{\small 
\tabcolsep=16pt
\begin{tabular}{|c|c|c|}
  \hline $\d$ & $\ex\le$ & $\exlow\le$ \\
\hline
0,9 & 0,5383 &  0,3089  \\
0,8 & 0,5723 &  0,3187  \\
0,7 & 0,6026 &  0,3334  \\
0,6 & 0,6276 &  0,3528  \\
0,5 & 0,6413 &  0,3775  \\
0,4 & 0,6342 &  0,4080  \\
0,3 & 0,6195 &  0,4450  \\
0,2 & 0,6094 &  0,4901  \\
0,1 & 0,6028 &  0,5451  \\
  \hline
\end{tabular}
}
\end{center}

\addtocounter{table}{1}

\bigskip


%\medskip


При доказательстве теоремы~1 будем придерживаться
подхода, предложенного и развитого В.\,М.~Золотарёвым в его
работах~[17--19]. Этот
подход основан на применении неравенств сглаживания, которые
позволяют оценить расстояние между функциями распределения через
расстояние между соответствующими характеристическими функциями. 
В~рамках этого подхода ключевыми моментами являются: ($i$)~выбор
надлежащего неравенства сглаживания; ($ii$) выбор в нем сглаживающего
ядра; ($iii$)~конструирование оценок для характеристических функций и
($i\nu$) выбор вычислительной оптимизационной процедуры. Опишем эти
моменты в той последовательности, в которой они появляются при
доказательстве неравенств~(\ref{Bikelis}) и~(\ref{K-B-E-sharpened}).
Соответствующие утверждения сформулируем в виде лемм.
{\looseness=1

}

\medskip

Обозначим $f(t)$ и $f_n(t)$ характеристические функции случайных
величин~$X_1$ и стандартизованной суммы $(X_1+\ldots+X_n)/\sqrt{n}$
соответственно:

\noindent
\begin{align*}
f(t)&=\int\limits_{-\infty}^\infty e^{itx}\,dF(x)\,;\\
f_n(t)&=\int\limits_{-\infty}^{\infty}e^{itx}\,dF_n(x)=
\left(f\left(\fr{t}{\sqrt n}\right)\right)^n\,,\quad t\in\R\,.
\end{align*}
Пусть
$$
r_n(t)=\left|f_n(t)-e^{-t^2/2}\right|\,.
$$

\medskip

\noindent
\textbf{Лемма 1} (см.~\cite{Prawitz1972}). %\label{LemPrawitzSmoothIneq}
\textit{Для произвольной функции распределения~$F$ при всех $n\ge1$,
$0<t_0\le1$ и $U>0$ имеет место неравенство}
\begin{multline*}
\ud\le 2\int\limits_0^{t_0}\left|K(t)\right|r_n(Ut)\,dt+{}\\
{}+
2\int\limits_{t_0}^{1}|K(t)\left|\cdot|f_n(Ut)\right|\,dt+{}\\
{}+
2\int\limits_0^{t_0}\left|K(t)-\fr i{2\pi t}\right|e^{-U^2t^2/2}\,dt +
\fr{1}{\pi}\int\limits_{t_0}^\infty e^{-U^2t^2/2}\,\fr{dt}t\,,\hspace*{-0.98pt}
\end{multline*}
\textit{где}
\begin{multline*}
K(t)=\fr{1}{2}\left(1-|t|\right)+\fr {i}{2}\left[(1-|t|)\cot\pi
t+\frac{\sign t}\pi\right]\,,\\
-1\le t\le1\,. %\eqno(8)
\end{multline*}

\smallskip

Перейдем теперь к оцениванию характеристических функций,
фигурирующих в лемме~1. Пусть
$\theta_0(\d)$~--- единственный корень уравнения
$$
\fr{\d\theta^2}2+ \theta\sin \theta + (2+\d)(\cos \theta - 1)=0\,,
\quad \pi\le\theta\le2\pi\,;
$$

\vspace*{-12pt}

\noindent
\begin{multline*}
\kd \equiv \sup_{x>0}\fr{\left|\cos
x-1+x^2/2\right|}{x^{2+\d}}={}\\
{}=\fr{\cos
x-1+x^2/2}{x^{2+\d}}\Bigg|_{x=\theta_0(\d)}\,;
\end{multline*}

\vspace*{-12pt}

\noindent
\begin{multline*}
\!\!\gd= \sup_{x>0}\sqrt{\left(\fr{\cos x-1+x^2/2}{x^{2+\d}}\right)^2\! +\!
\left(\fr{\sin x-x}{x^{2+\d}}\right)^2 }\,,\\ 0<\d\le 1\,.
\end{multline*}
Для $\eps>0$ положим
$$
\psi_\d(t,\eps)= 
\begin{cases}
  t^2/2-\kd\eps|t|^{2+\d}\,, & |t|<\theta_0(\d)\eps^{-1/\d}\,;\\[6pt]
  \fr{1-\cos\big(\eps^{1/\d} t\big)}{\eps^{2/\d}}\,,
      &\!\!\!\!\! \theta_0(\d)\le\eps^{1/\d}|t|\le 2\pi\,;\\[6pt]
  0\,, & |t|>2\pi\eps^{-1/\d}\,.
\end{cases}
$$
Несложно убедиться, что функция~$\psi_\delta(t,\eps)$ монотонно убывает 
по~$\eps$ при каждом фиксированном $t\in\R$.

Ляпуновская дробь будет обозначаться $\ell=$\linebreak $=\bet n^{-\d/2}$.
Дополнительно обозначим
$$
\ell_n=\ell+n^{-\d/2}\,.
$$

\smallskip

\noindent
\textbf{Лемма 2}.
\textit{При всех $F\in\F_{2+\d}$, $0<\d\le1$, $n\ge1$ и $t\in\R$ (если не
оговорено иное) справедливы оценки}
$$
|f_n(t)|\le  \left[1-\fr{2}{n}\psi_\d(t,\ell_n)\right]^{n/2} \equiv
f_1(t,\ell_n,n)\,;
$$
$$
|f_n(t)|\le \exp\{-\psi_\d(t,\ell_n)\}\equiv  f_2(t,\ell_n)\,;
$$
$$
|f_n(t)|\le \exp\left\{-\fr{t^2}{2}+\kd\ell_n|t|^{2+\d} \right\}\equiv
f_3(t,\ell_n)\,;
$$

\vspace*{-6pt}

\noindent
\begin{multline*}
\!r_n(t)\le e^{-t^2/2}\left[\exp\left\{\! \gd\ell|t|^{2+\d}-
n\ln\left(\!1-\fr{t^2}{2n}\!\right)-{}\right.\right.\hspace*{-0.67pt}\\
{}-\left.\left. \fr{t^2}2\right\}-1\right]\equiv
r_1(t,\ell,n), \enskip |t|<\sqrt{2n};
\end{multline*}

\vspace*{-9pt}

\noindent
\begin{multline*}
r_n(t)\le  \left(\gd{\ell|t|^{2+\d}}+\fr{|t|^4}{8n}\right)
\times{}\\
{}\times 
\left(\max\left\{f_1(t,\ell_n,n),\,e^{-t^2/2}\right\}\right)^{(n-1)/n}
\equiv r_2(t,\ell,n)\,;
\end{multline*}

\vspace*{-9pt}

\noindent
\begin{multline*}
r_n(t)\le  \fr{1}{2}\left(\gd{\ell|t|^{2+\d}}+\fr{|t|^4}{8n}\right)
\left(
e^{-t^2/2}+{}\right.\\
{}+\left.\max\left\{f_1(t,\ell_n,n),\,e^{-t^2/2}\right\}\right)
e^{t^2/(2n)}\equiv r_3(t,\ell,n)\,;
\end{multline*}

\vspace*{-3pt}

\noindent
$$
r_n(t)\le f_1(t,\ell_n,n)+e^{-t^2/2}\equiv r_4(t,\ell,n)\,.
$$

\medskip

\noindent
\textbf{Замечание 1.}
Очевидно, $f_1(t,\eps,n)\le f_2(t,\eps)$ при всех $n\ge1$, $\eps>0$
и $t\in\R$. Более того, из результата работы~\cite{Shevtsova2009}
вытекает, что $f_2(t,\eps)\le f_3(t,\eps)$ для всех $\eps>0$ и
$t\in\R$, так что самую точную оценку для~$|f_n(t)|$ дает 
$f_1(t,\ell_n,n)$, тогда как функции $f_j(t,\ell_n)$, $j=2,3$,
обладают полезным свойством монотонности по~$\ell_n$, играющим
важную роль в оптимизационной процедуре.

\medskip

\noindent
Д\,о\,к\,а\,з\,а\,т\,е\,л\,ь\,с\,т\,в\,о\,.\
Первые три оценки ($f_j$, $j=$\linebreak $=1,2,3$) являются тривиальными
следствиями оценок
$$
|f(t)|^2\le 1-2\psi_\d(t,\bet+1)\,,\quad t\in\R\,;
$$
$$
|f(t)|^2\le 1-t^2 + 2\kd(\bet+1)|t|^{2+\d}\,,\quad t\in\R\,,
$$
полученных Шевцовой в~\cite{Shevtsova2009}. Четвертая оценка ($r_1$)
впервые объявлена В.\,М.~Зо\-ло\-та\-рё\-вым~\cite{Zolotarev1966}
для $\d=1$ (без доказательства), ниже будет приведено полное
доказательство для всех $0<\d\le1$. Пятая ($r_2$) и шестая ($r_3$)
оценки являются несложной комбинацией методов и результатов работ
Правитца~\cite{Prawitz1975}, Шевцовой~\cite{Shevtsova2009},
Гапоновой и Шевцовой~\cite{GaponovaShevtsova2009}. Последняя 
оценка~($r_4$) тривиальна.

Докажем неравенство $r_n(t)\le r_1(t,\ell,n)$, $|t|<$\linebreak $<\sqrt{2n}$. Из
неравенств $|e^{ix}-1-ix|\le|x|^2/2$ и $|e^{ix}-1-ix-(ix)^2/2|\le
\gd|x|^{2+\d}$, $x\in\R$, с учетом моментных условий для
распределения $F\in\F_{2+\d}$ вытекают соотношения
\begin{gather*}
|f(t)-1|\le\fr{t^2}{2}\,,\quad|t|\le\sqrt2\,,\\
f(t)=1-\fr{t^2}2 + \theta_1 \gd \bet|t|^{2+\d}\,,\quad t\in\R\,,
%\label{ch_f_expansion}
\end{gather*}
с некоторым $\theta_1\in\C$, $|\theta_1|\le1$. Следовательно, для
всех $|t|<\sqrt{2n}$ определен логарифм~$\ln f(t)$ (условимся всегда
выбирать главную ветвь логарифма) и
\begin{multline*}
\abs{\ln f(t)+\fr{t^2}2}=\abs{\ln[1-(1-f(t))]+\fr{t^2}2}={}\\
{}=
\left|-\sum_{k=1}^\infty\fr{(1-f(t))^k}k+\fr{t^2}2\right|\le{}\\
{}\le
\sum_{k=2}^\infty\fr{1}{k}\left(\fr{t^2}2\right)^k+
\abs{f(t)-1+\fr{t^2}2}\le{}\\
{}\le  -\left[\ln\left(1-\fr{t^2}{2}\right)
+\fr{t^2}2\right]+\gd\bet|t|^{2+\d}\,,\\ |t|<\sqrt{2n}\,,
\end{multline*}
откуда с учетом неравенства $\abs{e^z-1}\le e^{|z|}-1$, $z\in\C$,
получаем
\begin{multline*}
r_n(t)=\abs{f_n(t)-e^{-t^2/2}}={}\\
{}=e^{-t^2/2} \left|\exp\left\{n\ln
f\left(\fr{t}{\sqrt n}\right)+\fr{t^2}2\right\}- 1\right|\le{}
\\
{}\le e^{-t^2/2} \left(\exp\left\{n\left|\ln f\left(\fr{t}{\sqrt
n}\right)+\fr{t^2}{2n}\right|\right\}-1\right)\le{}
\\
{}\le e^{-t^2/2} \left(\exp\left\{\gd \fr{\bet|t|^{2+\d}}{n^{\d/2}}
-n\ln \left(1-\fr{t^2}{2n}\right)-{}\right.\right.\\
{}-\left.\left.\fr{t^2}{2}\right\}-1\right)\equiv
r_1(t,\ell,n)\,,
\end{multline*}
что и требовалось доказать.

\medskip

Следующая лемма позволяет ограничить сверху множество
рассматриваемых значений~$n$ при оценивании констант~$\exlowk(\d)$ в
неравенстве~(\ref{K-B-E-sharpened}) с $0\le k\le 1$.

\medskip

\noindent
\textbf{Лемма 3.} %\begin{lemma}\label{LemMonotone}
\textit{Для любых положительных $k\le1$, $T$, $\eps$,
\begin{multline*}
N_1 \ge N_1(T) \equiv{}\\
{}\equiv T^2 \left( \fr{1}{8k\d\gd} + \sqrt{1 + \left(
\fr{1}{8k\d\gd}\right)^2}\ \right)^2\,;
\end{multline*}
$$
N_3 \ge N_3(T,\eps) \equiv \left( \fr{ T^{2-\delta}}{4\d\gd} +
\fr{\eps T^2}{\delta}\right)^{2/(2-\delta)}
$$
и таких, что $N_j\ge((1+k)/\eps)^{2/\d}$, при всех $|t| \le T$
справедливы оценки}
\begin{multline*}
\sup\limits_{n\ge N_1}r_1 \left(t, \eps - kn^{-\d/2}, n \right) \le{}\\
{}\le e^{-t^2/2}
\left( \exp \left\{ \gd \eps |t|^{2+\d} \right\} - 1
\right)\equiv\widetilde r_1(t, \eps)\,;
\end{multline*}

\vspace*{-9pt}

\noindent
\begin{multline*}
\sup\limits_{n\ge N_3}r_3\left(t,\eps-n^{-\delta/2},n\right)\le{}\\
{}\le \fr{\gd \eps
|t|^{2+\delta}}{2} \left( e^{\kd\eps |t|^{2+\delta}} + 1 \right)
e^{-t^2/2}\equiv \widetilde r_3(t,\eps)\,.
\end{multline*}

\medskip

\noindent
Д\,о\,к\,а\,з\,а\,т\,е\,л\,ь\,с\,т\,в\,о\,.\
Для доказательства первой оценки запишем~$r_1$ в виде
\begin{multline*}
r_1 \left(t, \eps - kn^{-\d/2}, n \right) = {}\\
{}=e^{-t^2/2} \left( \exp
\left\{ \gd \eps |t|^{2+\d} + g(n, |t|) \right\} - 1 \right)\,,
\end{multline*}
где
$$
g(n, |t|) = -\fr{k\gd}{n^{\d/2}} |t|^{2+\d} - n \ln \left(1 -
\fr{t^2}{2n} \right) - \fr{t^2}{2}\,.
$$
Тогда достаточно показать, что $g(x, t) \le 0$ для всех $0 \le t \le
T$ и $x \ge N_1(T,\eps)$.

Используя разложение логарифма в степенной ряд, для всех $0 \le t
\le \sqrt{2x}$ и $x > 0$ получаем
\begin{multline*}
g(x, t) = -\fr{k\gd}{x^{\d/2}}|t|^{2+\d} + x
\sum\limits_{j=2}^{\infty} \fr{1}{j} \left(\fr{t^2}{2x} \right)^j
\le{}\\
{}\le
 -\fr{k\gd}{x^{\d/2}}|t|^{2+\d} + \fr{x}{2}
\sum\limits_{j=2}^{\infty} \left(\fr{t^2}{2x} \right)^j ={}
\\
{}
= -\fr{k\gd}{x^{\d/2}}\,|t|^{2+\d} + \fr{t^4}{4(2x - t^2)} \equiv
\widetilde{g}(x, t)\,.
\end{multline*}
Заметим, что для $0 \le t \le \sqrt{2x}$ и $x > 0$
\begin{multline*}
\fr{\partial \widetilde{g}(x, t)}{\partial x} = \fr{k\d\gd
t^{2+\d}}{2 x^{1+\d/2}} - \fr{t^4}{2(2x - t^2)^2} > 0
\Longleftrightarrow{} \\
{}\Longleftrightarrow h(x, t) \equiv k\d\gd(2x - t^2)^2 -
t^{2-\d} x^{1+\d/2} > 0\,.
\end{multline*}

Пусть $T$~--- произвольное число из интервала $(0, \sqrt{2x})$.
Несложно видеть, что функция~$h(x, t)$ монотонно убывает по~$t$,
поэтому для всех $0 \le t \le T$
\begin{multline*}
h(x, t) \ge h(x, T) = x \left( 4k\d\gd x - 4k\d\gd T^2 -{}\right.\\
{}-\left.
x^{\d/2}T^{2-\d}\right) + k\d\gd T^4 > \\
> x \left( 4k\d\gd x - 4k\d\gd T^2 - x^{\d/2}T^{2-\d}\right)\,.
\end{multline*}
Для неотрицательности последнего выражения достаточно, чтобы
$$
H(x)\equiv 4k\d\gd (x - T^2) - x^{\d/2}T^{2-\d}\ge0\,.
$$
Очевидно, для всех достаточно больших~$x$ функция~$H(x)$ монотонно
возрастает и не ограничена при $x\to\infty$. Следовательно, найдется
такая точка $x_0>0$, что $H(x)\ge0$ для всех $x\ge x_0$. Будем
искать эту точку в виде $x_0=zT^2$. Не ограничивая общности, можно
считать, что $z>1$, поскольку $H(T^2)=-T^2<0$. Имеем
\begin{multline*}
H(zT^2) = 4k\d\gd T^2(z-1) - T^2z^{\d/2} > 0 
\Longleftrightarrow{}\\
{}\Longleftrightarrow{} 4k\d\gd(z-1) - z^{\d/2} > 0\,.
\end{multline*}
Поскольку $z^{\d/2} \le \sqrt{z}$ при $z>1$, для справедли\-вости
последнего условия достаточно, чтобы
$$
4k\d\gd z - \sqrt{z} - 4k\d\gd > 0\,,
$$
откуда, решив квадратное уравнение, получаем
$$
\sqrt{z} >\fr{1 + \sqrt{1 + 64(\d\gd k)^2}}{8\d\gd k}\equiv z_0\,.
$$
Следовательно, $x_0=z_0^2T^2$ и $H(x) > 0$ при
$$
x\ge  T^2 \left( \fr{1}{8\d\gd k} + \sqrt{1 + \left( \fr{1}{8\d\gd
k}\right)^2} \right)^2\,.
$$

Таким образом, для всех $0 \le t \le T$ и $x \ge N_1(T,\eps)$ имеем
$h(x, t)>xH(x)>0$, а значит~$\widetilde{g}(x, t)$ монотонно
возрастает по $x \ge N_1(T,\eps)$ при каждом фиксированном $0<t\le
T$ и для всех $N_1 \ge N_1(T,\eps)$
\begin{multline*}
\sup\limits_{0 \le t \le T} \sup_{x \ge N_1} g(x, t) \le \sup\limits_{0 \le t \le
T} \sup\limits_{x \ge N_1} \widetilde{g}(x, t) ={}\\
{}= \sup\limits_{0 \le t \le T}
\lim_{x\rightarrow\infty} \widetilde{g}(x, t) = 0\,,
\end{multline*}
что и требовалось доказать.

Далее заметим, что в силу неравенства $f_1(t,\ell_n,n)\le
f_3(t,\ell_n)$, $t\in\R$, величину~$r_3$ можно оценить следующим
образом:
\begin{multline*}
r_3\left(t,\eps-n^{-\delta/2},n\right)\le{}\\
{}\le \fr{1}{2} \left( \gd
\eps|t|^{2+\delta} - \fr{\gd |t|^{2+\delta}}{n^{\delta/2}} +
\fr{t^4}{8n}\right) \left(
\vphantom{\fr{t^2}{2}} e^{-{t^2}/{2}} +{}\right.\\
{}\left. \exp \left\{-\fr{t^2}{2} +
\varkappa(\delta) \eps |t|^{2+\delta} - \fr{\varkappa(\delta)
|t|^{2+\delta}}{n^{\delta/2}}\right\} \right) e^{t^2/(2n)} \le{}
\\
{}\le \fr{1}{2} \left( \gd \eps|t|^{2+\delta} - \fr{\gd
|t|^{2+\delta}}{n^{\delta/2}} + \fr{t^4}{8n}\right)\times{}\\
{}\times \left( 1 +
e^{\varkappa(\delta) \eps |t|^{2+\delta}}\right)e^{t^2/(2n)-t^2/2} ={}
\\
{}= \fr{|t|^{2+\delta}}{2} \left( \gd \eps - \fr{\gd}{n^{\delta/2}}
+ \fr{|t|^{2 - \delta}}{8n}\right)\times{}\\
{}\times \left( 1 + e^{\varkappa(\delta)
\eps |t|^{2+\delta}}\right) e^{t^2/(2n)-t^2/2} \equiv{}
\\
{}\equiv \fr{|t|^{2+\delta}}{2} \left( 1 + e^{\varkappa(\delta) \eps
|t|^{2+\delta}}\right) \exp \left( -\fr{t^2}{2} + g(n, |t|)
\right)\,,
\end{multline*}
где
\begin{multline*}
g(x, t) = \fr{t^2}{2x} + \ln \left( \gd\eps  -
\fr{\gd}{x^{{\delta}/{2}}} + \fr{t^{2 - \delta}}{8x}
\right)\,,\\
 x>\left(\fr{2}{\eps}\right)^{2/\delta}\,,\quad  \eps,\ t > 0\,.
\end{multline*}
Заметим, что выражение под знаком логарифма положительно.

Покажем, что при всех фиксированных положительных~$\eps$ и $t\le T$
функция~$g(x, t)$ монотонно возрастает по~$x$ при $x \ge
N_3(T,\eps)$. Вычислим производную
\begin{multline*}
\!\!\!\!g_x' (x, t) = -\fr{t^2}{2x^2} + \frac{(\delta/2) \gd
x^{-1-\d/2} - t^{2-\delta}/(8x^2)}{\gd\eps - {\gd}{x^{-\d/2}}
+ {t^{2-\delta}}/(8x)}=
\\
{}= \left (-8\gd\eps x t^2 + 8\gd x^{1-\d/2}t^2 - t^{4-\delta} +{}\right.\\
\left.{}+ 8\d\gd
x^{2-\d/2} - 2x t^{2-\delta}\right) \Big/
\left (
\vphantom{8\gd x^{1-\d/2} +
t^{2-\delta}}
2x^2 \left(
\vphantom{8\gd x^{1-\d/2} +
t^{2-\delta}}
8\gd\eps x -{}\right.\right.\\
\left.\left.{}- 8\gd x^{1-\d/2} +
t^{2-\delta}\right)\right)\,.
\end{multline*}
Знаменатель в последнем выражении совпадает с точностью до множителя
$2x^3$ с выражением под знаком логарифма в определении $g(x,t)$, а
следовательно, он положителен. В таком случае условие $g_x'(x,t)>0$
равносильно неравенству
\begin{multline*}
x\left(8\d\gd x^{1-\d/2} - \left(2t^{2-\delta} + 8\gd\eps t^2\right)\right) +{}\\
{}+ 8\gd
x^{1-\d/2} t^2 - t^{4-\delta} \ge 0\,,
\end{multline*}
для чего достаточно, чтобы

\noindent
\begin{multline*}
x^{1-\d/2} \ge \max\left\{\fr{t^{2-\delta}}{4\d\gd} + \fr{\eps
t^2}{\delta},\, \frac{t^{2-\delta}}{8\gd}\right\}={}\\
{}=
\fr{t^{2-\delta}}{4\d\gd} + \fr{\eps t^2}{\delta}\,.
\end{multline*}

Таким образом, при всех $0\le t\le T$ и
\begin{multline*}
x \ge  N_3(T,\eps)={}\\
{}= \max \left\{
\left(\fr{2}{\eps}\right)^{2/\delta}, \left( \fr{
T^{2-\delta}}{4\d\gd} + \fr{\eps
T^2}{\delta}\right)^{2/(2-\delta)} \right\}
\end{multline*}
функция $g(x,t)$ монотонно возрастает по~$x$, а следовательно, при
всех $N_3\ge N_3(T,\eps)$
$$
\sup\limits_{0\le t\le T}\sup\limits_{n \ge N_3} g(n,t) = \sup\limits_{0\le t\le
T}\lim_{x\rightarrow\infty} g(x,t) = \ln(\gd\eps)\,,
$$
что и требовалось доказать. Лемма доказана.

\medskip

Наконец, правильно организовать процесс вычислительной оптимизации
позволяют следующие утверждения.

\noindent
\textbf{Лемма 4} (см.~[24]). %\begin{lemma} [см. \cite{BhatRangaRao1982}] \label{LemBhRRao}
\textit{Для любого распределения~$F$ с нулевым средним и единичной
дисперсией справедливо неравенство}
\begin{multline*}
\sup\limits_{x}|F(x)-\Phi(x)|\le
\sup\limits_{x>0}\left(\Phi(x)-\fr{x^2}{1+x^2}\right)={}\\
{}= 0{,}54093654\ldots
\end{multline*}

\medskip

\noindent
\textbf{Лемма 5.} (см.~[23]). %\begin{lemma}[см. \cite{GaponovaShevtsova2009}]
%\label{LemEps_le_0.3}
\textit{Для любой функции распределения $F\in\F_{2+\d}$ и всех $n\ge2$
таких, что $(\bet+1)/n^{\d/2}\le0{,}6$, справедливо неравенство
$$
\rho(F_n,\Phi) \le
C'(\d)\fr{\bet}{n^{\d/2}}+\fr{C''(\d)}{n^{\d/2}}
$$
с $C'(\d)$ и $C''(\d)$, указанными в
табл.}~4.

\smallskip

\begin{center} %tabl4
\noindent
\parbox{56mm}{{\tablename~4}\ \ \small{Значения $C'(\d)$ и
$C''(\d)$ из леммы~5 при некоторых $\d$}}
%\end{center}
\vspace*{2pt}

{\small 
\tabcolsep=16.1pt
\begin{tabular}{|c|c|c|}
\hline
$\d$ & $C'(\d)$ & $C''(\d)$ \\
\hline
0,9 & 0,3085 & 0,2399  \\
0,8 & 0,2987 & 0,2166  \\
0,7 & 0,2912 & 0,1921  \\
0,6 & 0,2852 & 0,1655  \\
0,5 & 0,2800 & 0,1382  \\
0,4 & 0,2765 & 0,1044  \\
0,3 & 0,2776 & 0,0714  \\
0,2 & 0,2915 & 0,0327  \\
0,1 & 0,1500 & 0,0021  \\
  \hline
\end{tabular}
}
\end{center}

\addtocounter{table}{1}

%\bigskip

Из леммы~5 вытекает, что при всех~$n$ и~$\bet$
таких, что ${(\bet+k)/n^{\d/2}<0.3(1+k)}$,
неравенство~(\ref{K-B-E-sharpened}) имеет место при любых
$k\in[0,\,1]$ и $\exlowk(\d)>0$, удовлетворяющих условию
$(k+1)\exlowk(\d) \ge C'(\d) + C''(\d)$.

Подставляя оценки для характеристических функций из леммы~2 в правую часть 
неравенства сглаживания Правитца из леммы~1, получаем некоторую функцию 
$D(\ell,n,t_0,U)$, мажорирующую равномерное расстояние 
$\rho(F_n,\Phi)$ при всех $U>0$, $t_0\in(0,1]$, $n\geqslant1$ и~$F$ 
с фиксированной ляпуновской дробью $\beta_{2+\delta}n^{-\delta/2}=\ell$.
Приведенные леммы дают основание ограничить область рассматриваемых
значений величины $\eps=(\bet+k)/n^{\d/2}$ некоторым конечным
отрезком, отделенным от нуля (подробнее об этом будет сказано ниже),
и искать константу~$C_k$ при каждом $k\in[0,\,1]$ в виде
\begin{equation}
\left.
\begin{array}{rl}
\exlowk(\d)&=\max\limits_{\eps}C_\d(\eps)\,;\\[6pt]
C_\d(\eps)&=\fr{D_\d(\eps)}{\eps}\,;\\[6pt]
D_\d(\eps)&=\sup\left\{D_\d(\eps,n)\colon n\ge n_*\right\}\,,
\end{array}
\right \}
\label{FormMax}
\end{equation}
где
\begin{gather*}
D_\d(\eps,n) =\inf\limits_{t_0,\,U>0} D\left(\eps-\fr{k}{n^{\d/2}},n,t_0,U\right)\,,\\
n_*=\max\left\{1,\,\left\lceil\left(\fr{1+k}{\ell}\right)^{2/\d}\right\rceil\right\}\,.
\end{gather*}
Здесь $\lceil x\rceil$~--- минимальное целое, не меньшее~$x$. Условие
$n\ge n_*$ является следствием неравенства $\bet\ge1$. При этом для
оценивания супремума по~$n$ вместо входящих в $D_\d(\eps,n)$ величин
$r_j$, $j=1,2,3,4$, для достаточно больших~$n$ используются их
монотонные мажоранты. Вычисление максимума по~$\eps$ существенно
опирается на свойство монотонного возрастания по~$\eps$ всех
используемых оценок для функций~$|f_n(t)|$ и~$r_n(t)$, а
следовательно, и величины $D_\d(\eps)=\eps C_\d(\eps)$. Это свойство
позволяет оценить $\max\limits_{\eps}C_\d(\eps)$ по значениям~$C_\d(\eps)$
лишь в конечном числе точек. А~именно имеет место

\medskip

\noindent
\textbf{Лемма 6.} %\begin{lemma}\label{LemMonDeps}
\textit{Для всех $\eps_2>\eps_1>0$ имеет место неравенство}
$$
\max\limits_{\eps_1\le\eps\le\eps_2}C_\d(\eps)\le
C_\d(\eps_2)\fr{\eps_2}{\eps_1}\,.
$$

\medskip

Минимизация функции $D(\eps-k{n}^{-\d/2},\,n,\,t_0,\,U)$ по~$t_0$ и
$U$ проводится численно c использованием стандартных процедур в
системе Matlab~7.3 (R2006b).

\medskip

Перейдем теперь к описанию алгоритма вычисления константы~$\ex(\d)$
в неравенстве~(\ref{Bikelis}). Положим
$$
\eps = \fr{\bet}{n^{\d/2}}\,.
$$
%\vspace*{-12pt}


\noindent
\begin{center} %tabl5
\vspace*{-8pt}

\noindent
\parbox{79mm}{{\tablename~5}\ \ \small{Экстремальные значения
$n^*$, $\eps^*=(n^*)^{-\d/2}$ и оптимальные $t_0,U$ при вычислении
константы $\ex(\d)$}}
%\end{center}

\vspace*{2ex}

{\small 
\tabcolsep=11pt
\begin{tabular}{|c|c|c|c|c|c|}
  \hline
  $\d$ & $\eps_{\max}$ & $n^*$ & $\eps^*$ & $t_0$ & $U$\\
  \hline 
  0,9& 1,006& 3& 0,610& 0,35& 4,94\\
  0,8& 0,946& 3& 0,644& 0,39& 4,40\\
  0,7& 0,898& 3& 0,681& 0,46& 3,81\\
  0,6& 0,862& 4& 0,660& 0,53& 3,71\\
  0,5& 0,844& 5& 0,669& 0,66& 3,35\\
  0,4& 0,853& 6& 0,699& 0,82& 3,09\\
  0,3& 0,874& 5& 0,786& 1,00& 2,69\\
  0,2& 0,888& 4& 0,871& 0,78& 2,42\\
  0,1& 0,898& 9& 0,896& 0,87& 2,53\\
  \hline
\end{tabular}
}
\end{center}

\addtocounter{table}{1}

\vspace*{12pt}


\noindent
Тогда при $\eps\le0.3$ неравенство~(\ref{Bikelis}) с~$\ex(\d)$,
указанной в теореме~1, вытекает из
леммы~5. C~другой стороны, лемма~4
позволяет ограничить сверху область рассматриваемых значений~$\eps$
величиной $0{,}5409\ldots/\ex(\d)\equiv\eps_{\max}(\d)$, фиксированной
при каждом~$\d$. Таким образом, при вычислении~$\ex(\d)$
максимизация по~$\eps$ в формулах~(\ref{FormMax}) проводится на
конечном отрезке $0{,}3\le\eps\le\eps_{\max}(\d)$. Значения правой
границы~$\eps_{\max}(\d)$ приведены в
табл.~5. Для оценки
характеристических функций при $n<100$ используется~$f_1$, а при
$n\ge100$~--- функция~$f_2$, монотонно убывающая по~$n$, что
позволяет при каждом~$\eps$ оценивать супремум $D_\d(\eps,n)$ по
значениям~$n$ лишь в конечном числе точек:
$n_*,\ldots,\max\{n_*,100\}$, где $n_*=\max\{1,\eps^{-2/\d}\}$.
Максимум $C_\d(\eps)=D_\d(\eps)/\eps$ по
$0{,}3\le\eps\le\eps_{\max}(\d)$ оценивается  с помощью
леммы~6 и не превосходит тех значений~$\ex(\d)$,
которые указаны в формулировке теоремы~1.
Экстремальные значения $n=n^*$ и $\eps=(\ell^*)^{-\d/2}$ указаны в
табл.~5 в третьем и четвертом
столбцах, а соответствующие оптимальные значения параметров~$t_0$ и~$U$~--- 
в пятом и шестом столбцах. Отметим, что точке экстремума соответствует $\bet=1$.


Пусть теперь $k=1$. Обозначим
$$
\eps=\fr{\bet+1}{n^{\d/2}}\,.
$$
Тогда при $\eps\le0{,}3$ неравенство~(\ref{K-B-E-sharpened}) является
следствием леммы~5, а при $\eps\ge
0{,}5409\ldots/\exlow(\d)\equiv$\linebreak $\equiv\eps_{\max}(\d)$~--- следствием
леммы~4. Таким образом, при вычислении~$\exlow(\d)$
максимизацию по~$\eps$ в формулах~(\ref{FormMax}) достаточно
проводить на отрезке $0{,}3\le\eps\le\eps_{\max}(\d)$. Значения
$\eps_{\max}(\d)$ приведены в
табл.~6. Из этой таблицы видно,
что максимальное рассматриваемое значение~$\eps$ не превосходит~1,76. 
Для вычисления супремума по $n\ge n_*$ используется
лемма~3 с $T=2{,}2$, соответствующие значения
$N_1=$\linebreak $=N_1(2{,}2)$ и $N_3=N_3(2{,}2,\,1{,}76)$ приведены в
табл.~6 (для $N_3(T,\eps)$ взято
<<с запасом>> значение $\eps=1{,}76$). Как видно, уже при $n\ge184$
для всех рассматриваемых значений~$\d$ можно использовать оценки
$\widetilde r_1(t,\eps)$, \linebreak\vspace*{-12pt}
\pagebreak

%\vspace*{1pt}

\noindent
\begin{center} %tabl6
\vspace*{-8pt}

\noindent
\parbox{79mm}{{\tablename~6}\ \ \small{Экстремальные значения
$\eps^*$ и оптимальные $t_0,U$ при вычислении константы
$\exlow(\d)$}}
%\end{center}
\vspace*{2ex}

{\small 
\tabcolsep=8pt
\begin{tabular}{|c|c|c|c|c|c|c|}
  \hline
$\d$ & $\eps_{\max}$ & $N_1$ & $N_3$ & $\eps^*$ & $t_0$ & $U$\\
\hline 0,9& 1,752& 38& 142& 1,061& 0,38& 2,14\\
0,8& 1,698& 37& 110& 1,108& 0,40& 2,12\\
0,7& 1,623& 36& 91& 1,135& 0,43& 2,09\\
0,6& 1,534& 36& 79& 1,143& 0,44& 2,06\\
0,5& 1,434& 37& 73& 1,136& 0,46& 2,02\\
0,4& 1,326& 40& 71& 1,114& 0,47& 1,99\\
0,3& 1,213& 49& 75& 1,079& 0,48& 1,96\\
0,2& 1,099& 72& 90& 1,032& 0,49& 1,93\\
0,1& 0,985& 184& 144& 0,976& 0,49& 1,90\\
\hline
\end{tabular}
}
\end{center}

\addtocounter{table}{1}

\vspace*{12pt}


\noindent
$\widetilde r_3(t,\eps)$ из
леммы~3. Таким  образом, супремум по~$n$ достаточно оценивать
по значениям~$n$ лишь в конечном числе точек:
$n_*,\ldots,\max\{n_*,184\}$, где
 $n_*=\max\{1,(2/\eps)^{2/\d}\}$.
При этом экстремум целевой функции не превосходит значений,
указанных  в теореме~1 и достигается при $n\to\infty$
и $\eps=\eps^*(\d)$~--- указано в пятом столбце
табл.~6. Соответствующие
оптимальные значения~$t_0$ и~$U$ приведены в шестом и седьмом
столбцах табл.~6.



\bigskip

В заключение авторы выражают свою признательность В.\,Ю.~Королеву
за поддержку и постоянное внимание к работе.

{\small\frenchspacing
{%\baselineskip=10.8pt
\addcontentsline{toc}{section}{Литература}
\begin{thebibliography}{99}

\bibitem{Berry1941} %1
\Au{Berry A.\,C.} The accuracy of the Gaussian approximation to the
sum of independent variates~// Trans. Amer. Math. Soc., 1941.
Vol.~49. P.~122--139.

\bibitem{Esseen1942} %2
\Au{Esseen C.-G.} On the Liapunoff limit of error in the theory of
probability~// Ark. Mat. Astron. Fys., 1942. Vol.~A28. No.~9.
P.~1--19.

\bibitem{Katz1963} %3
\Au{Katz M.} A note on the Berry--Esseen theorem~// Ann. Math.
Statist., 1963. Vol.~34. P.~1107--1108.

\bibitem{Petrov1965} %4
\Au{Петров В.\,В.} Одна оценка отклонения распределения суммы
независимых случайных величин от нормального закона~// ДАН СССР,
1965. Т.~160. Вып.~5. С.~1013--1015.

\bibitem{Bikelis1966} %5
\Au{Бикялис А.} Оценки остаточного члена в центральной предельной
теореме~// Литовский математический сб., 1966. Т.~6. Вып.~3.
С.~323--346.

\bibitem{Petrov1972} %6
\Au{Петров В.\,В.} Суммы независимых случайных величин.~--- М.: Наука, 1972.

\bibitem{Esseen1956} %7
\Au{Esseen~C.-G.} A moment inequality with an application to the
central limit theorem~// Skand. Aktuarietidskr., 1956. Vol.~39.
P.~160--170.

\bibitem{KorolevShevtsova2010} %8
\Au{Королев В.\,Ю., Шевцова И.\,Г.} Уточнение неравенства
Берри--Эссеена с приложениями к пуассоновским и смешанным
пуассоновским случайным суммам~// Обозрение прикладной и
промышленной математики, 2010. Т.~17. Вып.~1. С.~25--56.

\bibitem{Tysiak1983} %9
\Au{Tysiak W.} Gleichm$\ddot{\mbox{a}}${\!\!\ptb\ss}ige und
nicht-gleichm$\ddot{\mbox{a}}${\!\!\ptb\ss}ige Berry--\linebreak Esseen--Absch{\"a}tzungen.
Dissertation. --- Wuppertal, 1983.

\bibitem{Paditz1996} %10
\Au{Paditz H.} On the error-bound in the nonuniform version of
Esseen's inequality in the $L_p$-metric~// Statistics, 1996.
Vol.~27. P.~379--394.

\bibitem{GaponovaKorchaginShevtsova2009} %11
\Au{Гапонова М.\,О., Корчагин А.\,Ю., Шевцова~И.\,Г.} Об абсолютных
константах в равномерной оценке точности нормальной аппроксимации
для распределений, не имеющих третьего момента~// Сб.\ статей
молодых ученых факультета ВМК МГУ. Вып.~6.~--- М.: Макс Пресс, 2009.
С.~81--89.

\bibitem{Paditz1986} %12
\Au{Paditz H.} $\ddot{\mbox{U}}$ber eine Fehlerabsch$\ddot{\mbox{a}}$tzung im zentralen
Grenzwertsatz~// Wiss. Z. Hochschule f$\ddot{\mbox{u}}$r Verkehswesen
``Friedrich List.''~--- Dresden, 1986. Bd.~33. H.~2. S.~399--404.

\bibitem{Shevtsova2010} %13
\Au{Шевцова И.\,Г.} Об асимптотически варл правильных постоянных в
центральной предельной теореме~// Тео\-рия вероятностей и ее
применения, 2010 (в пе\-ча\-ти). Т.~55. Вып.~2.

\bibitem{KorolevShevtsova2009} %14
\Au{Королев В.\,Ю., Шевцова И.\,Г.} О верхней оценке абсолютной
постоянной в неравенстве Берри--Эссеена~// Теория вероятностей и ее
применения, 2009. Т.~54. Вып.~4. С.~671--695.

\bibitem{Shevtsova2010a} %15
\Au{Шевцова И.\,Г.} Нижняя асимптотически правильная постоянная в
центральной предельной теореме~// Докл. РАН, 2010.
Т.~430. Вып.~4. С.~466--469.

\bibitem{KorolevShevtsova2010a} %16
\Au{Королев В.\,Ю., Шевцова И.\,Г.} Уточнение неравенства
Берри--Эссеена~// Докл. РАН, 2010. Т.~430. Вып.~6.
С.~738--742.

\bibitem{Zolotarev1966} %17
\Au{Золотарёв В.\,М.} Абсолютная оценка остаточного члена в
центральной предельной теореме~// Теория вероятностей и ее
применения, 1966. Т.~11. Вып.~1. С.~108--119.

\bibitem{Zolotarev1967a} %18
\Au{Золотарёв В.\,М.} Некоторые неравенства теории вероятностей и их
применение к уточнению теоремы А.\,М.~Ляпунова~// ДАН СССР, 1967.
Т.~177. №\,3. С.~501--504.

\bibitem{Zolotarev1967b} %19
\Au{Zolotarev V.\,M.} A sharpening of the inequality of
Berry--Esseen~// Z. Wahrsch. verw. Geb., 1967. Bd.~8. P.~332--342.

\bibitem{Prawitz1972} %20
\Au{Prawitz H.} Limits for a distribution, if the characteristic
function is given in a finite domain~// Scand. Aktuar Tidskr., 1972.
P.~138--154.

\bibitem{Shevtsova2009} %21
\Au{Шевцова И.\,Г.} Некоторые оценки для характеристических функций
с применением к уточнению неравенства Мизеса~// Информатика и её
применения, 2009. Т.~3. Вып.~3. С.~69--78.


\bibitem{Prawitz1975} %22
\Au{Prawitz H.} On the remainder in the central limit theorem.~I.
Onedimensional independent variables with finite absolute moments of
third order~// Scand. Actuarial J., 1975. No.~3. P.~145--156.

\bibitem{GaponovaShevtsova2009} %23
\Au{Гапонова М.\,О., Шевцова  И.\,Г.} Асимптотические оценки
абсолютной постоянной в неравенстве Берри--Эссеена для
распределений, не имеющих третьего момента~// Информатика и её
применения, 2009. Т.~3. Вып.~4. С.~41--56.

\label{end\stat}

\bibitem{BhatRangaRao1982} %24
\Au{Бхаттачария Р.\,Н., Ранга~Р.\,Р.} 
Аппроксимация нормальным распределением.~--- М.: Наука, 1982.
 \end{thebibliography}
}
}

\end{multicols}