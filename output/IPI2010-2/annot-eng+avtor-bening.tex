
ON ASYMPTOTIC BEHAVIOR OF THE POWERS OF THE TESTS FOR THE CASE OF LAPLACE DISTRIBUTION

V.\,E.\,Bening$^1$ and R.\,A.\,Korolev$62$
Faculty of Computational Mathematics and Cybernetics, 
M.V. Lomonosov Moscow State University, bening@yandex.ru

Faculty of Computational Mathematics and Cybernetics, 
M.V. Lomonosov Moscow State University, stochastique@gmail.com}

In the paper we prove a formula for the limit of thх normalized difference between 
the power of the asymptotically most powerful test and the power of the asymptotically optimal test 
for the case of Laplace distribution (see theorem~2.8). Due to the nonregularity of the Laplace distribution 
the logarithm of the likelihood ratio admits nonregular stochastic expansion, 
and an analog of Cram\'er's~(C) condition is not valid for the sign statistic which bases the asymptotically optimal test. 
Then direct use of the theorem~3.2.1 from~[1] or theorem~2.1 from~[2] is difficult, 
and in the present paper we revisit their proofs for the case of Laplace distribution.

power function; conditional probability measure; conditional moment; Laplace distribution.


\textbf{Бхэшэу ВырфшьшЁ Етухэ№хтшў} (Ё.\,1954) --- фюъЄюЁ Їшчшъю-ьрЄхьрЄшўхёъшї эрєъ, яЁюЇхёёюЁ ърЇхфЁ√ ьрЄхьрЄшўхёъющ ёЄрЄшёЄшъш
Їръєы№ЄхЄр т√ўшёышЄхы№эющ ьрЄхьрЄшъш ш ъшсхЁэхЄшъш МГУ шь.\,М.\,В.\,Люьюэюёютр, ёЄрЁ°шщ эрєўэ√щ ёюЄЁєфэшъ ИПИ РАН

\textbf{Bening Vladimir E.} (b. 1954) --- Doctor of Science in physics and mathematics; professor, Department of Mathematical
Statistics, Faculty of Computational Mathematics and Cybernetics, M.\,V.\,Lomonosov Moscow State University; senior researcher,
Institute of Informatics Problems, Russian Academy of Sciences

\textbf{КюЁюыхт Рюьрэ АэрЄюы№хтшў} (Ё.\,1977) --- рёяшЁрэЄ, МГУ шь.\,М.\,В.\,Люьюэюёютр;
рёёшёЄхэЄ, Рюёёшщёъшщ єэштхЁёшЄхЄ фЁєцс√ эрЁюфют (РУДН)

\textbf{Korolev Roman A.} (b.\,1977) --- PhD student, M.\,V.\,Lomonosov Moscow State University; assistant, Peoples' Friendship
University of Russia

\end{document}