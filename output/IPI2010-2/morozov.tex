\def\stat{morozov}

\def\tit{АНАЛИЗ  СЕТЕВОГО  ПРОТОКОЛА С~ОБЩЕЙ
ФУНКЦИЕЙ   РАСШИРЕНИЯ ОКНА   ПЕРЕДАЧИ СООБЩЕНИЯ ПРИ~КОНФЛИКТАХ$^*$}

\def\titkol{Анализ  сетевого  протокола с общей
функцией   расширения окна   передачи сообщения при конфликтах}

\def\autkol{А.~Лукьяненко, Е.~Морозов,  А.~Гуртов}
\def\aut{А.~Лукьяненко$^1$, Е.~Морозов$^2$,  А.~Гуртов$^3$}

\titel{\tit}{\aut}{\autkol}{\titkol}

{\renewcommand{\thefootnote}{\fnsymbol{footnote}}\footnotetext[1]
{Работа поддерживается грантом РФФИ, 10-07-00017.}}

\renewcommand{\thefootnote}{\arabic{footnote}}
\footnotetext[1]{Helsinki Institute for Information Technology HIIT, Aalto, Finland, firstname.secondname@hiit.fi}
\footnotetext[2]{Институт прикладных математических исследований КарНЦ РАН, emorozov@krc.karelia.ru}
\footnotetext[3]{Helsinki Institute for Information Technology HIIT, Aalto, Finland, gurtov@hiit.fi}

%\newcommand{\todo}[1]{{\bf\color{blue} TODO: #1}}


\Abst{Исследован класс сетевых протоколов
контроля несущей среды, где окно передачи  сообщения является
призвольной возрастающей функцией числа  конфликтов сообщения,
посланного с данной   станции. В общепринятых предположениях,
накладываемых  на свойства сети,   исследована функция протокола,
определяемая правилом расширения окна в зависимости от числа
конфликтов. Найдено выражение для функции протокола, обеспечивающей
минимальное среднее время передачи сообщения. Проведен
асимптотический анализ протокола при неограниченно растущем числе
станций.  Рассмотрены протоколы как с неограниченным, так и 
с ограниченным числом  попыток  передачи сообщения. Предложена модель
распределения доступа к каналу в непрерывном времени, допускающая
слоты различной длины.}

\KW{передача данных; оценка производительности;
моделирование протокола; доступ к каналу}


     \vskip 18pt plus 9pt minus 6pt

      \thispagestyle{headings}

      \begin{multicols}{2}

      \label{st\stat}
      
            
\section{Введение}

В данной работе  рассмотрен  сетевой прото\-кол, обес\-пе\-чи\-ва\-ющий
отсрочку передачи данных, %\linebreak 
вызванную конфликтом   в сети с
кол\-лек\-тивным доступом с контролем несущей и обнаружением/устранением
конфликтов~\cite{METCALFE}. Далее для этого протокола будет
использовано  обозначение BP (backoff protocol). Протокол BP служит
механизмом %\todo{для?}
успешной передачи информации и приводит к по\-стро\-ению легко
развертываемых и недорогих локальных сетей (ЛС). По\-стро\-ение ЛС,
использующее прямые связи всех станций друг с другом, весьма дорого,
и добавление каждой  новой станции ведет к стремительному увеличению
затрат и сложности ЛС. Альтернативное решение, состоящее в том, что
сообщение передается не напрямую, требует  уверенности в том, что
промежуточная станция, используемая для передачи,  не уйдет из сети.
В~ЛС центральным элементом является  {\it передающая среда}, которую
назовем системой передачи или просто {\it системой}. Когда система
развернута, новые узлы (рабочие станции, терминалы, принтеры,
серверы) просто подключаются к ней и могут мгновенно начинать
функционировать.
   Однако такая простота и быстрота  развертывания и организации ЛС  создает технические
трудности. Когда некоторая станция начинает передавать сигнал,
возможна ситуация, при которой  другая станция, увидев систему
пус\-той, также  начинает  передачу (поскольку она еще не обнаружила
ранее посланный в систему сигнал). Тогда данные обеих станций
перекрываются и, как следствие, разрушаются, т.\,е.\ происходит {\it
конфликт}. Существуют методы избавления от подобных конфликтов.
Например, при использовании радиосигнала или оптоволоконной среды
сигналы  разных станций можно передавать на различающихся час\-то\-тах,
однако это приведет к удорожанию технологии. К~тому же число
различных доступных частот обычно ограничено.

После того как станция узнает о том, что ее данные разрушены, она
(как правило) инициирует повторную попытку передачи. Но если такие
попытки будут предприниматься  через детерминированные промежутки
времени, то  те же сообщения столкнутся вновь. Именно для решения
этой проб\-ле\-мы и предназначен BP, в котором окно передачи растет
вместе с числом неудачных попыток послать сообщение. Так называемый
{\it константный} BP был впервые использован как часть протокола
Aloha~\cite {ABRAMSON85}. Позднее его модификацию (обрезанный
бинарный экспоненциальный BP) успешно применили в сети Ethernet~\cite {METCALFE, SHOCH}. 
Несмотря на то, что сейчас Ethernet ушел от
использования BP, алгоритм расширения окна все еще активно
используется в различных сетевых протоколах, в частности в
беспроводных сетях (например, IEEE 802.11~\cite {IEEE80211}) и
транспортных протоколах таких, как SCTP и TFRC.

Несмотря на свою относительно  долгую историю, важность  и простоту,
BP долгое время не поддавался удовлетворительному теоретическому
анализу. В этой связи укажем на  работу~\cite {HASTAD}, содержащую
подробную предысторию вопроса, анализ
устойчивости, а также детальное исследование важных
частных случаев BP. Существует множество работ, связанных с анализом
BP, однако, как правило, используемые в них предпосылки являются
слишком ограничительными (например, рассматривается  бесконечное
число станций $N=\infty$ или функция расширения окна~$f$ имеет
специальную форму)~[6--8].  В~данной работе
исследование проводится при  конечном числе станций $N$, а затем
рассматривается асимптотика при  $N\to \infty$. В~действительности,
 рассматривается  целый класс протоколов описанного
выше типа, где каждый протокол специфицируется выбором конкретной
(возрастающей) функции~$f$.

\section{Описание протокола}

Рассмотрим~ $N$ станций, подключенных к передающей среде, причем
предполагается, что в каж\-дой станции постоянно имеется очередь
сообщений, готовых к отправке. Считается, что все\linebreak станции идентичны
(в статистическом смысле) и работают независимо друг от друга.
Рассмотрим работу одной такой (произвольной) станции более подробно.
По алгоритму BP выбирается первое сообщение из очереди и в
специальном счетчике, который  называется  BC, устанавливается
начальное значение $i=0$. В~любой момент времени значение BC равно
числу   последовательных  безуспешных попыток отправки, накопленных
станцией к данному моменту времени.
Если в некоторый момент значение BC равно~$i$, то  для
осуществления следующей передачи BP строит   {\it окно отсрочки
передачи сообщения} $W(i)= [1,f(i)]$,
 где $f$~--- некоторая  заданная монотонно возрастающая целочисленная
 функция, $i\ge 0$, причем  $f(0)\ge 1$.
  (Случай произвольной функции  $f$, требующий  незначительного изменения
  модели,  представлен в~\cite{LUKYA}.) Таким образом,  окно
  расширяется  с увеличением   значения BC.

 Естественной  единицей времени  при сетевом анализе
является  {\it слот}, в  течение которого может произойти одно {\it
элементарное событие}, скажем  столкновение сообщений, передача
пакета и~т.\,д.  Вообще говоря, величина реального (физического)
времени, соответствующего различным слотам, не является постоянной.
 Основные    результаты данной работы получены в терминах слотов,
а  в  разд.~5   показано, как переформулировать модель (и
полученные результаты) в терминах реального времени. Таким образом,
величина~$f(i)$ равна числу слотов, в течение одного из  которых
произойдет сле\-ду\-ющая попытка  отправки сообщения (при условии, что
уже произошло ровно $i$~конфликтов).
 Внутри  окна $W(i)$ слот отправки сообщения~$D_i$
выбирается {\it равномерно}. Таким образом, $D_i$ является временем
задержки (в слотах) до осуществления следующей попытки отправки
сообщения, если это сообщение уже имело $i\ge 0$ неудачных попыток.
Если сообщение отправлено успешно, то значение BC полагается равным
нулю. Если же (при значении BC равном~$i$) происходит конфликт, то
значение BC увеличивается до $i+1$. Для следующей  отправки
используется окно  $W(i+1)=[1,\, f(i+1)]$ и~т.\,д. В~дальнейшем
рассматриваются  {\it ограниченный} и {\it неограниченный}
протоколы. В~ограниченном протоколе задана верхняя граница  $M$
значений BC: если сообщение не удалось отправить в течение~$M$
попыток, то оно выбрасывается  из очереди. В неограниченном
протоколе $M=\infty$. Далее подробно исследуется неограниченный
протокол. Для ограниченного протокола достаточно использовать
очевидную модификацию анализа, применяемого для неограниченного
протокола, и поэтому в данном случае   приведены лишь окончательные
результаты.

Размер  любого  слота ограничен снизу величиной RTT (round trip
time). Это необходимо,  чтобы станции получали информацию  об отсутствии столкновений
 в течение одного слота, поскольку  RTT~--- это величина
времени, за которое  сигнал об отправке сообщения оповестит  сеть и
вернется  на станцию  отправки. Поэтому исключена  ситуация, когда
одна станция   ведет отправку сообщения в течение нескольких слотов,
в то время  как  другая  станция уменьшила число  {\it пустых
слотов} до очередной попытки отправки сообщения.  (Пустым слотом для
данной станции является любой слот внутри окна передачи, когда
станция не пытается передавать сообщение.)
 Кроме того, считается, что время передачи сообщения занимает один слот.
 Это ограничение основано в первую очередь на поведении сети Wi-Fi (протокол
IEEE802.11). Например, для  протокола IEEE802.11 время
распространения сигнала в  сети равно 1~мкс (т.\,е.\  $\mathrm{RTT}= 2$~мкс), а
пустой слот равен 50 мкс~\cite{IEEE80211, VISHN}.

%{\bf может быть где-то  здесь сослаться на работы Ляхова ? -- Сослался на книгу}


Для сети, где станции продолжают отсчитывать время до отправки, даже
видя сеть занятой,  рассматриваемая в данной статье модель верна,
только если одно сообщение  передается в  течении одного  слота, т.\,е.\ лишь при
 малых размерах  отправляемых пакетов. Иначе после
успешной передачи резко возрастает вероятность столкновения.
Действительно, если передача сообщения требует  {\it намного больше
одного  слота}, то многие другие станции, исчерпав время до отправки, попытаются
передать свои сообщения одновременно, сразу же  после завершения
данной передачи, что приведет к конфликтам.
 Заметим, что  в сети Ethernet станции продолжают отсчитывать время (пустые
слоты) до отправки, даже если  видят сеть занятой.  Поэтому, как
было отмечено, предлагаемая модель применима в данном случае    лишь
при малых размерах  отправляемых пакетов.

\section{Математическая модель в~дискретном времени  и~ее~анализ}

Суммируем предположения, положенные в основу модели. Они  приняты  в литературе, посвященной данному вопросу,
и являются вполне естественными и согласованными с имеющимися данными
о работе BP (см., например,~[3, 11, 12]).

\medskip

\noindent
\textbf{Предположение~1.} {\it Станции идентичны}.

\medskip

\noindent
\textbf{Предположение~2.}  {\it Стационарность}: вероятность конфликта
$p_c\in (0,1)$ постоянна  и является одной и той же для каждой
станции.

\medskip

\noindent
\textbf{Предположение~3.} {\it Условие насыщения}: каждая станция
всегда имеет непустую очередь.


\medskip

\noindent
\textbf{Предположение~4.} Если конфликт не происходит в первый слот
передачи сообщения, то  данное сообщение передается успешно.


\medskip

\noindent
\textbf{Предположение~5.} Когда какая-либо станция начинает передавать
сообщение, то  либо 1)~время до отправки на  других станциях не
уменьшается, либо 2)~время отправки одного сообщения занимает один
слот. При анализе протокола в терминах слотов оба эти предположения
эквивалентны.

Далее, состоянием данной станции считается  значение ее  BC. Если
$\mathrm{BC}=i$, то
$$
\p (D_i=k)=\fr{1}{f(i)}\,,\quad k=1,\dots,f(i)\,.
$$
Поэтому  среднее  время пребывания станции в данном состоянии определяется по формуле
$\e D_i\;=$\linebreak $=\;(f(i)+1)/2$. Далее, вероятность того, что значение $\mathrm{BC}=i$ и
в этом состоянии произойдет успешная отправка, есть
$P_i=(1-p_c)p_c^i$, $i\ge 0$. Назовем {\it циклом передачи
сообщения} время с момента первой попытки отправки до
успешной  его отправки и обозначим его длительность через~$S$.

Заметим, что стационарная вероятность $\gamma_i$ того, что станция
находится в состоянии~$i$ (в произвольный момент времени) равна доле
времени, проводимом станцией в этом состоянии на одном  цикле
передачи. Это приводит к сле\-ду\-юще\-му соотношению:
\begin{equation}
\gamma_i =
   \fr {P_i  \e D_i }{\sum_{i=0}^{\infty} P_i  \e D_i}
    =
   \fr {(f(i)+1)(1-p_c)p_c^i}
         {(1-p_c)F(p_c)+1}\,, \enskip i\ge 0\,,\!\!\!
         \label{eq:gamma}
\end{equation}
где использовано обозначение
\begin{equation*}
    F(p_c)
    =
    \sum_{i=0}^{\infty}f(i)p_c^i\,.
%    \label{eq:fz}
\end{equation*}

 Функцию~$F$, которая   играет в дальнейшем анализе ключевую роль,
 назовем {\it функцией протокола}. Заметим, что отправка сообщения
происходит (внутри окна передачи) в слоте с номером~ $N_R$,
распределение которого имеет вид $ \p(N_R=i)=P_i$, $i\ge 0.$ Поэтому величина
$$
\sum_{i=0}^{\infty} \e D_i P_i=\e D_{N_R}\,,
$$
стоящая в знаменателе~\eqref{eq:gamma}, является  средней  величиной
окна, в котором происходит успешная отправка  сообщения. В~\cite{KWAK} доказано следующее равенство:
$$
\fr{\gamma_i}{\e D_i}=
   \fr { P_i}{\e D_{N_R}}\,,\quad i\ge 0\,.
$$

Из формулы полной вероятности следует   такое выражение для
стационарной вероятности успешной отправки сообщения (в
произвольном слоте):
\begin{equation}
p_{\mathrm{st}} = \sum_{i=0}^{\infty}
   \fr{P_i}{\e
D_{N_R}}=\fr{1}{\e D_{N_R}}
   = \fr {2}{(1-p_c)F(p_c)+1}\,.\!\!\!
   \label{eq:pt}
\end{equation}

Вероятность~$p_{\mathrm{st}}$ можно получить  иначе. Заметим, что средняя
длина цикла передачи может быть записана таким образом:
\begin{multline}
\e\left(\sum_{i=0}^{N_R}D_i\right)=\sum_{i=0}^\infty \e D_i
\p\left(N_R\ge i\right)={}\\
{}=\fr{1}{2}\sum_{i\ge 0}\left(f(i)+1\right)p_c^i=\fr{1}{2} \left(F(p_c)+\fr{1}{(1-p_c)}\right)={}\\
{}=
\fr{(1-p_c)F(p_c)+1}{2(1-p_c)}\,.
\label{eq:4}
\end{multline}

Поскольку $P(N_R\ge i)= p_c^i$, то среднее число попыток до успешной
отправки 
\begin{eqnarray}
\e N_R =\fr{p_c}{(1-p_c)}\,. 
\label{eq:5}
\end{eqnarray}

Понятно, что в стационарном режиме вероятность отправки (в
произвольный момент времени) равна отношению среднего числа попыток
к средней длине цикла передачи. Поэтому~(\ref{eq:4}) и~(\ref {eq:5})
дают выражение
\begin{equation*}
p_{\mathrm{st}}=\fr{\e N_R}{\e\Bigl (\sum_{i=0}^{N_R}D_i\Bigr)},
\end{equation*}
которое совпадает с~(\ref{eq:pt}). Учтем, что данная станция (как
и остальные  $N-1$ станций) находится в стационарном режиме.
Поскольку конфликт возникает,  если не менее двух станций пытаются
отправить сообщения одновременно, то нетрудно получить следующее
соотношение (см.\ также~[9, 12]):
\begin{equation}
  p_c=1-(1-p_{\mathrm{st}})^{N-1}\,.
  \label{eq:pc}
\end{equation}

Объединяя выражения~(\ref{eq:pt}) и~(\ref{eq:pc}), приходим к
следующему выражению для функции протокола:
\begin{equation}
   F(p_c) = \fr{1+(1-p_c)^{1/(N-1)}}{(1-p_c)\left(1-(1-p_c)^{1/(N-1)}\right)}\,.
   \label{eq:Fpc}
\end{equation}

В~\cite{LUKYA} показано, что для существования единственного решения
$p_c\in (0,1)$ уравнения~(\ref{eq:Fpc}) достаточно   монотонного
возрастания функции $f$. Заметим, что различные протоколы,
определяемые различными функциями~$F$, вообще говоря, приводят к различным решениям~$p_c$.

Теперь  найдем  при каком значении~$p_c$ станция будет работать
оптимально, т.\,е.\ обеспечит минимальную  среднюю  длину  цикла
передачи~$\e S$.   Используя независимость случайных величин~$N_R$ и~$D_i$, 
можно записать
\begin{multline}
   \e S   = \e \left (\sum_{i=0}^{N_R}D_i\right)
        ={}\\
        {}= \fr{1}{2}\,\e \left(\sum_{i=0}^{\infty}(f(i)+1)\Ik (N_R\ge
        i)\right)={}\\
{}=\fr{1}{2} \left( F(p_c) +\fr{1}{1-p_c}\right)\,,
       \label{eq:es2}
\end{multline}
где $\Ik$ обозначает индикатор. 

Объединяя~(\ref{eq:Fpc}) и~(\ref{eq:es2}), получаем
\begin{equation}
   \e S  = \fr{1}{(1-p_c)\left(1-(1-p_c)^{1/(N-1)}\right)}\,. 
   \label{eq:eqm}
\end{equation}

Нетрудно проверить стандартным способом, что   минимальное значение
среднего цикла передачи $\e S$, являющегося функцией  вероятности
конфликта~$p_c$, достигается в точке $p_c=p_c^*$, где
\begin{equation}
   p_c^* = 1 - \left(1-\frac1N\right)^{N-1}\,.
    \label{eq:pcopt}
\end{equation}

Подстановка~(\ref{eq:pcopt}) в~(\ref{eq:eqm}) дает
\begin{equation}
\e S =\fr{N}{\left(1-1/N \right)^{N-1}}\,. 
\label{eq:8}
\end{equation}

Если рассматривать  вероятность конфликта как функцию числа станций~$N$ 
в сети, т.\,е.\ $p_c^*=p_c^*(N)$, то из~(\ref{eq:pcopt}) следует
такой асимптотический результат:
\begin{equation*}
p_c^* \to 1-e^{-1}\,,\quad N\to \infty\,. 
%\label{eq:asympt}
\end{equation*}

Подстановка~(\ref{eq:pcopt}) в~(\ref{eq:Fpc}) показывает, что
протокол является оптимальным, если его функция~$F$ удовлетворяет
условию
\begin{equation}
   F(p_c^*) =
   \fr{2N-1}{\left(1-1/N\right)^{N-1}}:=A(N)\,.
   \label{optimum}
  % \nonumber
\end{equation}
Отметим, что этому соотношению  может удовлетворять, вообще говоря,
не единственная функция расширения окна $f$ (т.\,е.\ не  единственный
протокол), а целый класс функций~$f$.

Рассмотрим класс наиболее важных с точки зрения практики  {\it
экспоненциальных протоколов}, где $f(i)=a^i$ для некоторого числа
$a>0$. В этом случае  в предположении $a p_c^*<1 $ соотношение~(\ref{optimum}) принимает вид
\begin{equation*}
\fr{1}{1-ap_c^*} =A(N)\,. 
%\label{6}
\end{equation*}

Таким образом,
$$
a=\fr{A(N)-1}{A(N)
p_c^*}=\fr{2N-1-\left(1-1/N\right)^{N-1}}{\left(2N-1\right)\left(1-\left(1-1/N\right)^{N-1}\right)}\,.
$$
Нетрудно показать, что  при $N\to \infty$
$$
a:=a(N)\to \fr{1}{1-e^{-1}}\approx 1{,}58
$$
 и   что $a(N)$ монотонно убывает с ростом
 числа станций~$N$.  Кроме того,  $a(2)=5/3\approx 1{,}66.$
Таким образом,  {\it оптимальное значение  параметра~$a$ в формуле}~(\ref{optimum})
\textit{определяется однозначно для каждого~$N$, находится в
диапазоне $[1{,}58,\, 1{,}66]$ и  с ростом  числа станций приближается к~1,58  сверху}.

Коснемся условия стационарности системы, считая
упрощенно, что суммарный входной поток в систему является процессом
восстановления с интенсивностью $\lambda \in (0,\,\infty)$.
Естественно считать, что тогда каждая станция получает входной поток
интенсивности~$\lambda/N$. Такое  предположение вполне оправдано при
большом числе идентичных станций. С~другой стороны, нетрудно
показать (используя, скажем,  аргументы из теории регенерирующих процессов), что  условие стационарности (для каждой
станции) имеет хорошо известный вид:
\begin{equation}
\fr{\lambda}{N}\e S< 1\,. 
\label{eq:9}
\end{equation}

Такого рода условия {\it отрицательного сноса} хорошо известны в
анализе стационарности марковских цепей~\cite {MEYN}. Как правило,
они исключают {\it уход процесса в бесконечность} и
обеспечивают существование стационарного режима для широкого класса
процессов, описывающих реальные коммуникационные системы. Отметим,
что хотя $\e S\to \infty$ при $N\to \infty$ (см.~(\ref{eq:8})),
однако интенсивность входного потока в отдельный узел
пропорционально убывает, сохраняя неравенство~(\ref{eq:9}). При
оптимальном значении $p_c=p_c^*$ неравенство~(\ref{eq:9}) принимает
форму (см.~(\ref{eq:eqm}), (\ref{eq:pcopt})):
\begin{equation*}
\lambda<\left(1-\fr{1}{N}\right)^{N-1}\,.
\end{equation*}
 Это неравенство в пределе  при  $N\to \infty$ переходит в
 неравенство $ \lambda<e^{-1}, $ возникающее, например,  при анализе
 стационарности   синхронной  системы ALOHA~\cite{ROBERTS, Kleinrock}.
 %\todo{пробел в цитировании?}.
 Таким образом,  при
оптимальном выборе протокола и большом числе станций интенсивность
входного потока на каждую станцию, обеспечивающая устойчивость,
должна быть меньше~$e^{-1}$. Следует ожидать, что для произвольного
протокола эта область устойчивости еще меньше.  Отметим, что
проблема неустойчивости BP неоднократно отмечалась  ранее (см.~[6--8]).

\section{Ограниченный протокол отсрочки}

Проведенный выше анализ можно перенести   на случай, когда число
конфликтов ограничено величиной  $M<\infty$. В~этом случае формула~(\ref{eq:Fpc}), 
определяющая функцию протокола, преобразуется к виду:
\begin{multline}
  F(p_c):= F_M(p_c)={}\\
  {}=\fr{\left(1-p_c^{M+1}\right)\left(1+\left(1-p_c\right)^{1/(N-1)}\right)}
   {\left(1-p_c\right)\left(1-\left(1-p_c\right)^{1/(N-1)}\right)}\,, 
   \label{eq2:FMpc}
\end{multline}
а средняя длина цикла передачи принимает вид:
\begin{equation*}
   \e S = \fr{\left(1-p_c^{M+1}\right)}
   {\left(1-p_c\right)\left(1-\left(1-p_c\right)^{1/(N-1)}\right)}\,. 
%   \label{eq2:ES}
\end{equation*}

 Рассматриваемое  ограничение применяется, например, в  алгоритме
протокола Ethernet, где используется ограниченный бинарный
экспоненциальный протокол. Более точно, $f(i)=2^i,$ $i=1,\dots,10$ и
$f(i)=1024$, $i=11,\ldots,16$. После 16~конфликтов сообщение
выбрасывается из очереди. Это правило дает следующую функцию протокола:
\begin{multline*}
   F_{16}(p_c)=\sum_{i=0}^{10} 2^ip_c^i + \sum_{i=11}^{16} 2^{10} p_c^i ={}\\
   {}=\fr{1-(2p_c)^{11}}{1-2p_c}+2^{10}p_c^{11}\fr{1-p_c^6}{1-p_c}\,.
\end{multline*}
Поэтому уравнение~(\ref{eq2:FMpc}) (относительно~$p_c$) принимает вид
\begin{multline}
\fr{1-(2p_c)^{11}}{1-2p_c}+2^{10}p_c^{11}\fr{1-p_c^6}{1-p_c} ={}\\
{}=
\fr{\left(1-p_c^{17}\right)\left(1+\left(1-p_c\right)^{1/(N-1)}\right)}
   {\left(1-p_c\right)\left(1-\left(1-p_c\right)^{1/(N-1)}\right)}\,.
   \label{17}
\end{multline}
Решив это уравнение,  можно подсчитать  среднюю длину цикла передачи
сообщения~$\e S $   и вероятность  потери  сообщения~$p_c^{17}$.
Явное  решение уравнения~(\ref{17}) в общем случае найти трудно,
однако его  можно исследовать численно. Некоторые численные результаты
представлены в  табл.~\ref{t1mor}.
Заметим, что вероятность столкновения и вероятность потери
возрастают с ростом числа станций. Так, при $N=11$  вероятность
столкновения $p_c=0{,}621$ (и достаточно близка к оптимальной
вероятности столкновения $p_c=p_c^*\approx 1-e^{-1}=0{,}6817$,
полученной  для неограниченного протокола). Вероятность потери в
этом случае незначительна. При $N=501$ и $N=1001$ вероятность
столкновения  больше~0,9, а вероятность потери  равна
соответственно~0,35 и~0,81. При этом среднее время передачи
сообщения  в сети $\e S/N$ (в отличие от среднего цикла передачи~$\e
S$ для отдельной станции) практически не изменяется  с ростом числа
станций~$N$.  Причина такой устойчивости, по-видимому, состоит в
том, что увеличение вероятности потери, вызванное ростом числа
станций, сдерживает рост среднего времени передачи.

\begin{table*}\small
\begin{center}
 \Caption{Численный анализ  BEB
 \label{t1mor}}
 \vspace*{2ex}
 
    \begin{tabular}{|c|c|c|c|}
        \hline  Число станций   &  
        \tabcolsep=0pt\begin{tabular}{c}Вероятность\\ столкновения $p_c$\end{tabular} &
\tabcolsep=0pt\begin{tabular}{c} Среднее время\\ передачи сообщения\\ в сети $\e S/N$\end{tabular}& 
\tabcolsep=0pt\begin{tabular}{c}Вероятность\\  потери $p_c^{17}$\end{tabular}\\
        \hline 
        $11$   & 0,621 &  2,593  &  \hphantom{9}0,0003\\
        \hline 
        51   & 0,743 &  2,828  &  \hphantom{9}0,0064 \\
        \hline 
        101\hphantom{9}  & 0,799 &  3,026  &  0,022 \\
        \hline 
        501\hphantom{9}  & 0,94\hphantom{9}  &  3,858  &  0,349 \\
        \hline 
        1001\hphantom{99} & 0,988 &  3,515  &  0,809 \\
        \hline
    \end{tabular}
\end{center}
\end{table*}

\section{Модель  протокола в~непрерывном времени}

Выше  рассмотрена дискретная модель в терминах слотов, которые,
вообще говоря, не равны по величине. Полезно рассмотреть модель в
непрерывном времени, что может привести к существенному изменению   вида
оптимизационной задачи. Для этого введем величину пустого слота~$T_i$,  
величину слота~$T_{c}$, где произошло столкновение,  и
величину~$T_{s}$ слота, где произошла успешная отправка. Если
размеры слотов одинаковы (т.\,е.\ $T_{c}=T_{s}=T_{i}$), то среднее
(физическое) время цикла передачи $\e_T S= \e S \cdot T_{i}$ и
модель сводится к модели, рассмотренной выше.

В случае неограниченного протокола среднее число слотов в цикле
передачи, равное  $\e \left (\sum_{i=0}^{N_R}D_i\right)$, содержит в
среднем $\e N_R+1$ {\it непустых  слотов}, причем в течение   $\e
N_R$ слотов происходят столкновения  и  один слот содержит успешную
отправку.  Пусть $T^*_{i}$~--- размер  слота, в течение которого {\it
данная произвольная станция}  не отправляет сообщения 
(в то время как   другие станции могут отправлять  или  не отправлять).
 Таким образом, средний цикл передачи 
\begin{multline}
   \e_T S   = \e N_R T_{c} + {}\\
   {}+\left (\e \left[\sum_{i=0}^{N_R}D_i\right]-\e N_R-1\right)T^*_{i}+
   T_{s}\,.
   \label{time}
\end{multline}
C учетом~(\ref{eq:4}) и~(\ref{eq:5}) соотношение~(\ref{time})
принимает вид:
\begin{multline}
   \e_T S   = \fr{p_c}{1-p_c} T_{c} + {}\\
   {}+\fr1{2(1-p_c)}\left((1-p_c)F(p_c)-1\right)T^*_{i}+ T_{s}\,. 
   \label{eq:ets}
\end{multline}

Вообще говоря, $T^*_{i}\not = T_{i}$, поскольку
 величина  пустого слота для
данной станции {\it растягивается} на время возможных передач
другими станциями. Для нахождения~$T^*_{i}$ используем  формулу
полной вероятности. Во-первых, $T^*_{i}=T_{i}$ с вероятностью
$1-p_c$, с которой  остальные  $N-1$~станций  не пытаются передавать
(в произвольном слоте). Далее (предполагаем, что $N>2$), лишь одна из
$N-1$ оставшихся станций передает сообщение (а остальные $N-2$
станций молчат) в течение данного  слота с вероятностью
$(N-1)p_{\mathrm{st}}(1-p_{\mathrm{st}})^{N-2}$, и с этой вероятностью
$T^*_{i}=T_{s}$. Оставшаяся вероятность  соответствует событию, при
котором, по крайней мере, две станции пытаются передавать сообщение в
данном слоте, и тогда $T^*_{i}=T_{c}$. Таким образом, используя
соотношения~(\ref{eq:pt}) и~(\ref{eq:pc}), имеем
\begin{multline*}
    T^*_{i} = (1-p_c)T_{i}+
   (1-p_c)\fr{2(N-1)}{(1-p_c)F(p_c)-1}T_{s}+{}\\
{}+\left(1-(1-p_c)-
   (1-p_c)\fr{2(N-1)}{(1-p_c)F(p_c)-1}\right)T_{c}={}\\
{}= (1-p_c)T_{i} +{}\\
{}+ p_c T_{c} + \fr{2(1-p_c)}{(1-p_c)F(p_c)-1}(N-1)(T_{s}-T_{c})\,.
\end{multline*}
Подставляя это выражение  в~(\ref{eq:ets}), получаем
    \begin{multline*}
   \e_T S   = N(T_{s}-T_{c}) +\fr{1}{1-p_c}T_{c} +{}\\
   {}+
   \fr{2(N-1)}{(1-p_c)F(p_c)-1}\left(\left(1-p_c\right)T_{i} + p_c
   T_{c}\right)\,.
   \end{multline*}

Используя явный вид~(\ref{eq:Fpc}) функции~$F(p_c)$, окончательно
получаем  величину среднего цикла передачи в  виде
\begin{multline*}
\e_T S   = N(T_{s}-T_{c}) +\fr{1}{1-p_c}T_{c} +   (1-p_c)(N-1)\times{}\\
{}\times
\left((1-p_c)^{1-1/(N-1)}-1\right)\left((1-p_c)T_{i} + p_c
   T_{c}\right)\,.\!
%   \label{itog}
\end{multline*}

Таким образом, задача оптимизации сводится к  отысканию минимума
$\e_T S$ как функции~$p_c$ для заданных величин~$T_{s}$, $T_{c}$ и~$T_{i}$. 
Заметим, что~$T_{s}$ не влияет на задачу оптимизации.

 В работе~\cite{BIANCHI} исследован  частный случай данной функции (для
бинарного экспоненциального протокола) при фиксированных величинах~$T_{s}$, $T_{c}$ и~$T_{i}$, 
описанных в спецификации~\cite{IEEE80211}. В~частности, $T_{i}=50$ мкс,
 средняя длина   фрейма (пакета MAC-уровня) $T_{s}$ определяется
настройщиком сети,  величина~$T_{c}$ определяется
 видом конкретно используемого протокола.
Из работы~\cite{BIANCHI} следует, что предлагаемая в данной статье
модель адекватно описывает работу реальных сетевых протоколов.

\section{Заключение}

В статье предложен метод исследования сетевого протокола контроля
несущей,  в котором окно отсрочки передачи сообщения при конфликте
сообщений, посланных разными станциями, меняется  в соответствии с
произвольной (не обязательно экспоненциальной) функцией~$f$,
монотонно возрастающей с ростом числа последовательных конфликтов
при попытке передачи данного сообщения данной станцией. Метод
базируется на нескольких естественных и неоднократно использованных
ранее предположениях. В~частности, предполагается, что станции сети
идентичны и работают независимо друг от друга, находятся в состоянии
насыщения (т.\,е.\ сообщения  для  передачи есть всегда), искомая
вероятность конфликта~$p_c$ является стационарной (и одинаковой для
всех станций). %\linebreak 
Кроме того, предполагается, что время отправки
сообщения распределено равномерно внутри \mbox{окна} передачи.
 В~отличие от предшествующих работ, исследование охватывает весь  класс протоколов с рас\-ши\-ря\-ющим\-ся
окном. Ключевым элементом    является введение функции протокола~$F$, 
которая строится на базе функции~$f$. 
%
С~помощью установленных
соотношений между веро\-ят\-ностью конфликта~$p_c$  и ве\-ро\-ят\-ностью
отправки сообщения удалось \mbox{найти} явное  выражение для~$F$ как функции
ве\-ро\-ят\-ности~$p_c$. 
%
В~статье объясняются некоторые эффекты,
обсуждавшиеся  при исследовании данного протокола в ряде
предшествующих работ (например,~\cite{SHOCH}). В~частности,
показано, что в нагруженном состоянии и при растущем числе станций в
оптимальном режиме работы (т.\,е.\ при минимальном среднем цикле
передачи~$\e S$) вероятность конфликта $p_c\to 1-e^{-1}$. При тех же
условиях найдено, что область стационарности сети при суммарном
входном потоке интенсивности~$\lambda$ имеет вид $\lambda<e^{-1}$.
{\looseness=-1

}

Исследован протокол как с неограниченным, так и с ограниченным
числом попыток передачи сообщения.

\vspace*{-12pt}

{\small\frenchspacing
{\baselineskip=12pt
\addcontentsline{toc}{section}{Литература}
\begin{thebibliography}{99}

  \bibitem{METCALFE}
  \Au{Metcalfe R., Boggs~D.}
  Ethernet: Distributed packet switching for local computer
networks~// Communications of the ACM, 1976. Vol.~19. No.\,7. P.~395--404.

  \bibitem{ABRAMSON85}
  \Au{Abramson N.} 
  Development of the ALOHANET~//
  IEEE Trans. on Inform. Theory, 1985. Vol.~31. No.\,2. P.~119--123.

  \bibitem{SHOCH}
  \Au{Shoch J.\,F., Hupp J.\,A.} 
  Measured performance of an Ethernet local network~//
  Commun. ACM, 1980. Vol.~23. No.\,12. P.~711--721.

  \bibitem{IEEE80211}
  \Au{IEEE 802.11 Standard.}
  IEEE Standard for Information technology~--- Telecommunications and information exchange between systems~---
  Local and metropolitan area networks~--- Specific requirements.
  Part~11: Wireless LAN Medium Access Control (MAC)
  and Physical Layer (PHY) Specifications.~--- N.Y.: IEEE, 2007.

  \bibitem{HASTAD}
  \Au{H{\tiny$\stackrel{\circ}{\mbox{\normalsize a}}$}stad J., Leighton~T., Rogoff~B.}
  Analysis of backoff protocols for multiple access channel~// SIAM J. Comput., 1996. Vol.~25. No.\,4. P.~740--774.

  \bibitem{KELLY85}
  \Au{Kelly F.\,P.} 
  Stochastic models of computer   communication systems~//J.~Roy. Statist. Soc. B, 1985. Vol.~47. P.~379--395.

  \bibitem{KELLY87}
  \Au{Kelly F.\,P.,  MacPhee~I.\,M.} 
  The number of packets   transmitted by collision detect random access scheme~//
  Annals of Prob., 1987. Vol.~15. P.~1557--1568.

  \bibitem{ALDOUS}
  \Au{Aldous D.\,J.} 
  Ultimate instability of exponential back-off
  protocol for acknowledgement-based transmission control of random
  access communication channel~// IEEE Trans. on Information Theory, 1987. Vol.~33. No.\,2. P.~219--223.


  \bibitem{LUKYA}
  \Au{Lukyanenko A., Gurtov~A.} 
  Performance analysis of general backoff  protocols~// J.~Communications Software and Systems, 2008. Vol.~4. No.\,1. P.~13--22.


  \bibitem{VISHN} 
  \Au{Вишневский В., Ляхов А., Портной~С., Шахнович~И.}
  Широкополосные беспроводные сети передачи информации.~--- М.: Техносфера, 2005.

  \bibitem{BIANCHI}
  {\it Bianchi G.} 
  Performance analysis of the IEEE 802.11 Distributed Coordination Function~//
  IEEE J.~Selected Areas in Communications, 2000. Vol.~18. No.\,3. P.~535--547.
  
    \bibitem{KWAK}
  \Au{Kwak B., Song N., Miller~L.\,E.} 
  Performance analysis   of exponential backoff~// IEEE/ACM Transactions on Networking (TON), 2005. 
  Vol.~13. No.\,2. P.~343--355.

  \bibitem{MEYN}
  \Au{Meyn S., Tweedie R.\,L.} 
  Markov chains and stochastic stability.~--- New York: Cambridge University Press, 2009.

  \bibitem{ROBERTS}
  \Au{Roberts L.\,G.} 
  ALOHA packet system with and without slots and capture~// SIGCOMM Comput. Commun., 1975. Vol.~5. No.\,2. P.~28--42.
  
  \label{end\stat}

    \bibitem{Kleinrock}
    \Au{Клейнрок Л.} 
    Вычислительные системы с очередями.~--- М.: Мир, 1979.

%  \bibitem{ABRAMSON70}
%  \Au{Abramson N.} 
%  The ALOHA system~--- another alternative  for computer communications~// AFIPS, 1970. Vol.~37. P.~281--285.

%  \bibitem{GOODMAN}
%  \Au{Goodman J., Greenberg A.\,G., Madras~N., March~P.} 
%  Stability of binary exponential backoff~//  J.~ACM, 1988. Vol.~35. No.\,3. P.~579--602.

 % \bibitem{FAYOLLE}
%  \Au{Fayolle G., Flajolet P.,  Hofri~M.}
%  On a functional equation arising in the analysis of a protocol for a
%  multi-access broadcast channel~// Adv. Appl. Prob., 1986. Vol.~18. P.~441--472.

 % \bibitem{ROSENKRANTZ}
%  \Au{Rosenkrantz W.\,A.} 
%  Some theorems on the instability   of the exponential back-off protocol~//  Proc. of Performance'84,
%  1985. P.~199--205.

 % \bibitem{GOLDBERG98}
%  \Au{Goldberg L.\,A., MacKenzie~P.} 
%  Analysis of practical
%  backoff protocols for contention resolution with multiple servers~//
%  J. Comp. System Sci., 1999. Vol.~58. P.~232--258.

 % \bibitem{GOLDBERG00}
%  \Au{Goldberg L.\,A., MacKenzie P., Paterson~M., Srinivasan~A.} 
%  Contention resolution with constant
%  expected delay~// J. ACM (JACM), 2000. Vol.~47. No.\,6. P.~1048--1096.

 % \bibitem{JACOBSON}
%  \Au{Jacobson V., Karels M.\,J.} 
%  Congestion   avoidance and control~// Proc. of ACM SIGCOMM, 1998.
  
%  \label{end\stat}

 % \bibitem{AKELLA} 
%  \Au{Akella A., Seshan S., Karp~R., Shenker~S.}
%  Selfish behavior and stability of the Internet: A
%  game-theoretic analysis of TCP~// Proc. of ACM SIGCOMM, 2002.
 \end{thebibliography}
}
}

\end{multicols}