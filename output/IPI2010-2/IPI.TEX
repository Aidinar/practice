\documentclass[10pt]{book}
\usepackage[utf8]{inputenc}

\usepackage{latexsym,amssymb,amsfonts,amsmath,indentfirst,shapepar,%fleqn,%
picinpar,shadow,floatflt,enumerate,multicol,colortbl,ipi}

\usepackage{rotating}
\input{epsf}

%\nofiles

%\includeonly{avtor,avtor-eng}
%\includeonly{avtor-eng}
%\includeonly{pred}  %+

%\includeonly{konovalov}  %+pdf
%\includeonly{markin}     %+pdf
%\includeonly{kruchin}    %+
%\includeonly{krivenko}   %+pdf
%\includeonly{morozov}    %+pdf
%\includeonly{stepanov}   %+pdf
%\includeonly{torhin-rud} %+pdf
%\includeonly{shevts}     %+pdf
%\includeonly{bening}     %+pdf
%\includeonly{kozerenko}  %+pdf


%\includeonly{toc-rus,toc-en}
%\includeonly{toc-en}


%\includeonly{obchak}
%\includeonly{reshal}
%\includeonly{eng-index}
%\includeonly{cover3}

\usepackage{acad}
\usepackage{courier}
\usepackage{decor}
\usepackage{newton}
\usepackage{pragmatica}
\usepackage{zapfchan}
\usepackage{petrotex}
\usepackage{bm}                     % полужирные греческие буквы
\usepackage{upgreek}                % прямые греческие буквы
%\usepackage{verbatim}

\renewcommand{\bottomfraction}{0.99}
\renewcommand{\topfraction}{0.99}
\renewcommand{\textfraction}{0.01}

\setcounter{secnumdepth}{1} %здесь - 3 + chapter = 4

\arraycolsep=1.5pt

%\usepackage[pdftex]{graphicx}

%\usepackage{oz}

%NEW COMMANDS



\renewcommand{\r}{{\rm I\hspace{-0.7mm}\rm R}}
\newcommand{\I}{{\rm I\hspace{-0.7mm}I}}
\newcommand{\Ik}{\mbox{{\small \tt {1}}\hspace{-1.5mm}{\tt 1}}}
%\newcommand{\Ikl}{{\small \tt{1}}\hspace*{-0.4mm}\mathtt{1}}

%\mathrm{I}\hspace*{-0.7mm}\mathrm{R}

\newcommand{\il}[2]{\int\limits_{#1}^{#2}}%интеграл с пределами #1 и #2

\newcommand{\h}{{\bf H}}
\newcommand{\p}{{\sf P}}  % вероятность
\newcommand{\e}{{\sf E}}  % мат. ожидание
\newcommand{\D}{{\sf D}}  % дисперсия
\newcommand{\eps}{\varepsilon}
\newcommand{\vp}{\mathrm{v.p.}}
\newcommand{\F}{{\mathcal F}}
%\def\iint{\int\limits_{-\infty}^{\infty}}

%\newcommand{\gr}{{\geqslant}}

\newcommand{\g}{\mbox{\textit{g}}}

%\renewcommand{\la}{\lambda}
\newcommand{\si}{\sigma}
%\renewcommand{\a}{\alpha}

%\newcommand{\pto}{\stackrel{P}{\longrightarrow}} % сходимость по веpоятности

%\newcommand{\eqd}{\stackrel{d}{=}} % равенство по pаспpеделению

%\newcommand{\kp}{\kappa}
%\def\Q{{\cal Q}} \def\H{{\cal H}}
%\newcommand{\bet}{\beta_{2+\delta}}


%\newtheorem{definition}{Определение}
%\renewcommand{\thedefinition}{\arabic{definition}.}
%END NEW COMMANDS

%\renewcommand{\baselinestretch}{1.2}

%\pagestyle{myheadings}

\setlength{\textwidth}{167mm}      % 122mm
\setlength{\textheight}{658pt}
%\setlength{\textheight}{635.6pt}
\setlength{\columnsep}{4.5mm}

\setcounter{secnumdepth}{4}

%\addtolength{\headheight}{2pt}
%\addtolength{\headsep}{-2mm}

%\addtolength{\topmargin}{-20mm}  % for printing


\hoffset=-30mm  % From Yap
%\hoffset=-20mm  % From Acrobat

%\voffset=0mm % From Yap
%\voffset=-15mm   % From Acrobat

\addtolength{\evensidemargin}{-9.5mm} % for printing
\addtolength{\oddsidemargin}{9.5mm}  % for printing

%\renewcommand{\thefootnote}{\fnsymbol{footnote}}
%\renewcommand{\thefootnote}{\arabic{footnote}}
\renewcommand{\figurename}{\protect\bf Рис.}
\renewcommand{\tablename}{\protect\bf Таблица}

\newcommand{\Caption}[1]{\caption{\protect\small %\baselineskip=2.5ex
#1}}

\renewcommand{\thefigure}{\arabic{figure}}
\renewcommand{\thetable}{\arabic{table}}
\renewcommand{\theequation}{\arabic{equation}}
\renewcommand{\thesection}{\arabic{section}}

\renewcommand{\contentsname}{СОДЕРЖАНИЕ}
\newcommand{\fr}[2]{\displaystyle\frac{\displaystyle #1\mathstrut}{\displaystyle #2\mathstrut}}

%\renewcommand{\thefootnote}{\fnsymbol{footnote}}
%\newcommand{\g}{\mbox{\textit{g}}}

%\newcommand{\Caption}[1]{\caption{\protect\small\baselineskip=2ex #1}}
\newcounter{razdel}
\setcounter{razdel}{0}


\newcommand{\titel}[4]{%
\

\vspace*{5pt}

\ifodd\therazdel {\raggedright\noindent\Large\textrm\textbf
 \lineskip .75em
  \baselineskip=3.2ex #1 \par}
\vskip 1em {\noindent\large\textrm\textbf #2 \par}
\addcontentsline{toc}{subsection}{{\textrm\textbf #3}\protect\newline #1}
\def\rightheadline{\underline{\noindent\hbox to \textwidth{\hfill\small\textrm{#4}
%\hfill \large\bf\thepage
}}}
\def\leftheadline{\underline{\noindent\parbox{\textwidth}{
%\raggedleft\large\bf\thepage \hfill
\small\textit{#3}\hfill}}}
\def\leftfootline{\small{\textbf{\thepage}
\hfill ИНФОРМАТИКА И ЕЁ ПРИМЕНЕНИЯ\ \ \ том~4\ \ \ выпуск 2\ \ \ 2010}
}%
 \def\rightfootline{\small{ИНФОРМАТИКА И ЕЁ ПРИМЕНЕНИЯ\ \ \ том~4\ \ \ выпуск~2\ \ \ 2010
\hfill \textbf{\thepage}}} \vskip 2em \setcounter{figure}{0}
\setcounter{table}{0} \setcounter{equation}{0} \setcounter{section}{0}
\setcounter{subsection}{0} \setcounter{subsubsection}{0}
\setcounter{footnote}{0} \setcounter{razdel}{0}
%\end{flushleft}
\else {
 \raggedright\noindent\Large\textrm\textbf
 \lineskip .75em
\baselineskip=3.2ex #1 \par} \vskip 1em
%\begin{flushleft}
{\noindent\large\textrm\textbf #2 \par}
\addcontentsline{toc}{subsection}{{\textrm\textbf #3}\protect\newline #1}
\def\rightheadline{\underline{\noindent\hbox to \textwidth{\hfill\small\textrm{#4}
%\hfill \large\bf\thepage
}}}
\def\leftheadline{\underline{\noindent\parbox{\textwidth}{%\raggedleft\large\bf\thepage \hfill
\small\textit{#3}\hfill}}}
\def\leftfootline{\small{\textbf{\thepage}
\hfill ИНФОРМАТИКА И ЕЁ ПРИМЕНЕНИЯ\ \ \ том~4\ \ \ выпуск~2\ \ \ 2010}
}%
 \def\rightfootline{\small{ИНФОРМАТИКА И ЕЁ ПРИМЕНЕНИЯ\ \ \ том~4\ \ \ выпуск~2\ \ \ 2010
\hfill \textbf{\thepage}}} \vskip 2em \setcounter{figure}{0}
\setcounter{table}{0} \setcounter{equation}{0} \setcounter{section}{0}
\setcounter{subsection}{0} \setcounter{subsubsection}{0}
\setcounter{footnote}{0}
%\end{flushleft}
\fi}

\newcommand{\titelr}[2]{%
\

\vspace*{5pt}

\ifodd\therazdel {\raggedright\noindent\large\textrm\textbf
 \lineskip .75em
  \baselineskip=3.2ex #1 \par}
\vskip 1em {\noindent\normalsize\textrm\textbf #2 \par}
\else {
 \raggedright\noindent\large\textrm\textbf
 \lineskip .75em
\baselineskip=3.2ex #1 \par} \vskip 1em
%\begin{flushleft}
{\noindent\normalsize\textrm\textbf #2 \par}
\fi}

\newcommand{\titele}[5]{%
\

%\vspace*{5pt}

\ifodd\therazdel {\raggedright\noindent%\large
\textrm\textbf
 \lineskip .75em
%  \baselineskip=3.2ex
#1 \par}
\vskip .5em {\noindent\large\textrm\textbf #2 \par}
\vskip .5em
 {\noindent\textrm #3 \par}
\addcontentsline{toc}{subsection}{{\textrm\textbf #1}\protect\newline #2}
\def\rightheadline{\underline{\noindent\hbox to \textwidth{\hfill\small\textrm{#4}
%\hfill \large\bf\thepage
}}}
\def\leftheadline{\underline{\noindent\parbox{\textwidth}{
%\raggedleft\large\bf\thepage \hfill
\small\textrm{#5}\hfill}}}
\def\leftfootline{\small{\textbf{\thepage}
\hfill ИНФОРМАТИКА И ЕЁ ПРИМЕНЕНИЯ\ \ \ том~4\ \ \ выпуск~2\ \ \ 2010}
}%
 \def\rightfootline{\small{ИНФОРМАТИКА И ЕЁ ПРИМЕНЕНИЯ\ \ \ том~4\ \ \ выпуск~2\ \ \ 2010
\hfill \textbf{\thepage}}} \vskip 1em \setcounter{figure}{0}
\setcounter{table}{0} \setcounter{equation}{0} \setcounter{section}{0}
\setcounter{subsection}{0} \setcounter{subsubsection}{0}
\setcounter{footnote}{0} \setcounter{razdel}{0}
%\end{flushleft}
\else {
 \raggedright\noindent%\large
 \textrm\textbf
 \lineskip .75em
%\baselineskip=3.2ex
#1 \par} \vskip .5em
%\begin{flushleft}
{\noindent\large\textrm\textbf #2 \par} \vskip .5em
 {\noindent\textrm #3 \par}
\addcontentsline{toc}{subsection}{{\textrm\textbf #1}\protect\newline #2}
\def\rightheadline{\underline{\noindent\hbox to \textwidth{\hfill\small\textrm{#4}
%\hfill \large\bf\thepage
}}}
\def\leftheadline{\underline{\noindent\parbox{\textwidth}{%\raggedleft\large\bf\thepage \hfill
\small\textrm{#5}\hfill}}}
\def\leftfootline{\small{\textbf{\thepage}
\hfill ИНФОРМАТИКА И ЕЁ ПРИМЕНЕНИЯ\ \ \ том~4\ \ \ выпуск~2\ \ \ 2010}
}%
 \def\rightfootline{\small{ИНФОРМАТИКА И ЕЁ ПРИМЕНЕНИЯ\ \ \ том~4\ \ \ выпуск~2\ \ \ 2010
\hfill \textbf{\thepage}}} \vskip 1em \setcounter{figure}{0}
\setcounter{table}{0} \setcounter{equation}{0} \setcounter{section}{0}
\setcounter{subsection}{0} \setcounter{subsubsection}{0}
\setcounter{footnote}{0}
%\end{flushleft}
\fi}

\def\Abst#1{
\begin{center}\small\nwt
\parbox{150mm}{%\baselineskip=2.5ex
\textbf{Аннотация:}\ \
%\hspace*{\parindent}
#1}
\end{center}}
\def\Abste#1{
\begin{center}\small\nwt
\parbox{150mm}{%\baselineskip=2.5ex
\textbf{Abstract:}\ \
%\hspace*{\parindent}
#1}
\end{center}}

\def\KW#1{
\begin{center}\small\nwt
\parbox{150mm}{%\baselineskip=2.5ex
\textbf{Ключевые слова:}\ \ #1}
\end{center}}

\def\KWE#1{
\begin{center}\small\nwt
\parbox{150mm}{%\baselineskip=2.5ex
\textbf{Keywords:}\ \ #1}
\end{center}}


\def\KWN#1{
%\begin{center}
%\small
%\parbox{150mm}\end{center}
}

\renewcommand{\thesubsection}{\thesection.\arabic{subsection}\hspace*{-5pt}}
\renewcommand{\thesubsubsection}{\thesubsection\hspace*{5pt}.\arabic{subsubsection}\hspace*{-3pt}}

\begin{document}
\Rus

\nwt
%\ptb

%\renewcommand{\contentsname}{\protect\Large\bf Содержание}

\setcounter{tocdepth}{2}

%\tableofcontents

\renewcommand{\bibname}{\protect\rmfamily Литература}
  \def\Au#1{{\it #1}}

%\newcommand{\No}{№}
  \newcommand{\tg}{\,\mathrm{tg}\,}
    \newcommand{\ctg}{\,\mathrm{ctg}\,}
  \newcommand{\arctg}{\,\mathrm{arctg}\,}
  
\def\forallb{\mathop{\forall}}
\def\existsb{\mathop{\exists}}

\setcounter{page}{1}

\newpage
\addtocounter{razdel}{1}
%\def\razd{РЕГУЛИРУЕМЫЙ ЭЛЕКТРОПРИВОД ДЛЯ ЭЛЕКТРОЭНЕРГЕТИКИ}
%\newpage
%\def\stat{zakh}
\def\tit{СРЕДСТВА ОБЕСПЕЧЕНИЯ ОТКАЗОУСТОЙЧИВОСТИ ПРИЛОЖЕНИЙ}
\def\titkol{Средства обеспечения отказоустойчивости приложений}

\def\aut{В.\,Н.~Захаров$^1$, В.\,А.~Козмидиади$^2$}
\titel{\razd}{\tit}{\aut}{\titkol}


\Abst{Рассмотрены проблемы построения отказоустойчивых серверов, возникающие в связи с недетерминированностью поведения приложений. Предложена формальная модель, описывающая поведение приложения, основными объектами которой являются ресурсы и события. Предложены алгоритмы протоколирования работы приложения на резервном узле кластера, а также восстановления и продолжения его работы при отказе основного узла. При этом для клиентов сбой остается незаметным, за исключением некоторого увеличения времени обслуживания.}

\KW{сервер приложений $\bullet$ прозрачная отказоустойчивость $\diamond$
 процесс $\diamond$ ресурс $\diamond$ событие $\diamond$ контрольная точка
$\bullet$ детерминированность}

\vskip 12pt plus 6pt minus 3pt

\begin{multicols}{2}

\section*{ВВЕДЕНИЕ}

Средства вычислительной техники стали использоваться в областях,
требующих безотказной работы систем в течение многих лет (24 часа
в сутки, 365 дней в году).

\label{st\stat}

\footnotetext{$^1$ФГУП Центральный институт авиационного моторостроения
им. П.И. Баранова, Москва, Россия}
\footnotetext{$^2$ФГУП Центральный институт авиационного моторостроения
им. П.И. Баранова, Москва, Россия}

К таким областям относятся, например, центры хранения и обработки данных  в сетях (системы резервирования билетов, биллинговые,  банковские и т.д.), массированные распределенные вычисления (GRID-вычисления) и другие.

\thispagestyle{headings}

Обычно в подобных системах применяются частные решения, ориентированные в основном на обеспечение надежного хранения данных (например, файловые серверы, использующие для хранения RAID-контроллеры) и корректного их состояния при отказах (серверы баз данных с транзакционным выполнением запросов). Однако большинство приложений не гарантируют, что не произойдет потери части данных при отказе системы. Обычно предполагается, что клиентские средства должны повторять запросы после восстановления серверов, для того, чтобы данные не были потеряны, или что можно сделать возврат по времени на некоторое время назад и повторить работу с этого места. Однако далеко не все клиентские средства и условия применения приложений допускают это.

Отказоустойчивые системы для критически важных приложений, корректно решающие проблемы восстановления после сбоев,   предлагаемые ведущими производителями, как правило, дороги. Кроме того, они включают специфические серверные и клиентские приложения, не совместимые со стандартными приложениями, не обеспечивающими отказоустойчивость. Примером такого подхода к решению проблемы отказоустойчивости  хранения данных являются системы NetApp FAS компании Network Appliance, работающие на базе специализированной операционной системы Data ONTAP [1].

Построение отказоустойчивых систем, использующих серверы со стандартными приложениями, в свете вышесказанного, является актуальной проблемой, вызывающей значительный интерес. Рассмотрение методов достижения прозрачной отказоустойчивости таких систем и является предметом статьи.
\begin{figure*} %fig1
\vspace*{1pt}
\begin{center}
\mbox{%
\epsfxsize=1.6in
\epsfxsize=100mm
\epsfbox{BbR-1.eps}
}
\end{center}
\vspace*{-9pt}
\Caption{Базовый вариант трубы с разными выходными устройствами
(цилиндрическое, расширяющееся и сужающееся сопло)
\label{f1bab}}
\vspace*{-3pt}
\end{figure*}


\section{ОСНОВНЫЕ ПОНЯТИЯ И ПОДХОДЫ}

Под сервером в данной работе понимается вычислительный центр
(отдельный компьютер или кластер) в сети, предоставляющий клиентам
(пользователям, клиентским компьютерам) определенные услуги, разделяя
между ними свои ресурсы. Подобные серверы названы серверами приложений.
Широко распространенным примером сервера такого типа является файловый сервер, обеспечивающий удаленный коллективный доступ к файловой системе. Часто используются вычислительные серверы, предоставляющие клиентам возможность выполнять на них свои программы (например, в центрах коллективного пользования).


Обычно приложение представляет собой программу или группу программ, работающих в операционной среде, создаваемой операционной системой (в другой терминологии - один или несколько взаимодействующих процессов или потоков (threads)), которые реализуют функциональность сервера. Для построения отказоустойчивых серверов приложений широко используется кластерная технология. Следуя [2], кластером, названа разновидность параллельной или распределенной системы, которая:
\begin{itemize}
\item состоит из нескольких компьютеров (узлов кластера), связанных как минимум необходимыми коммуникационными каналами;
\item используется как единый, унифицированный компьютерный ресурс.
\end{itemize}

Прозрачная отказоустойчивость (Transparent Fault Tolerance, TFT) сервера приложений - это такое его поведение при возникновении аппаратных или программных отказов либо отказов в сети, при котором:
\begin{itemize}
\item отказ не вызывает потери или искажения данных, находящихся в базе данных сервера;
\item сервер продолжает нормально функционировать, несмотря на имевшие место отказы.
\end{itemize}

Клиенты сервера "не замечают" произошедших отказов. Единственным\footnote{допустимым
отклонением сервера от нормального поведения с точки зрения клиента является
некоторое увеличение времени обслуживания} (на несколько секунд или десятков секунд).

Обычно приложения, работающие на серверах приложений, не ориентированы на прозрачную отказоустойчивость. Они "заботятся" лишь о собственной целостности (например, состояния файловой системы или базы данных). Восстановление работоспособности сервера приводит к разрыву соединений с клиентами и потере их запросов. Это замечают клиенты - запросы следует повторять, на что клиентские приложения далеко не всегда рассчитаны. В данной работе предполагается, что приложения (прикладные программные средства), выполняемые на сервере, являются стандартными, то есть не имеют специальных средств, обеспечивающих отказоустойчивость.
\begin{figure*}[b] %fig1
\vspace*{1pt}
\begin{center}
\mbox{%
\epsfxsize=1.6in
\epsfxsize=100mm
\epsfbox{BbR-1.eps}
}
\end{center}
\vspace*{-9pt}
\Caption{Базовый вариант трубы с разными выходными устройствами
(цилиндрическое, расширяющееся и сужающееся сопло)
\label{f1bab}}
\vspace*{-3pt}
\end{figure*}

Серьезные исследования в области обеспечения отказоустойчивости серверов были развернуты после создания вычислительных серверов, предназначенных для решения задач, требующих больших вычислительных ресурсов. Решение этих задач выполняется на суперкомпьютерах, обеспечивающих массово-параллельные вычисления и представляющих собой кластеры из сотен и тысяч узлов (процессоров). Однако даже на этих "монстрах" решение может требовать десятков или сотен часов, и одиночный сбой, если не предприняты специальные меры, может привести к необходимости начинать работу сначала. Обычно решение вычислительной задачи в таких случаях осуществляется в модели относительно редко взаимодействующих между собой процессов, выполняемых на разных узлах кластера. Эти взаимодействия нужны для координации работы процессов, в частности, для обмена данными и промежуточными результатами. Взаимодействия опираются на специальный протокол, называемый MPI (Message-Passing Interface) и представляющий собой стандарт "de facto" [3].

Для преодоления последствий сбоя достаточно давно была разработана и широко применяется технология, опирающаяся на механизм контрольных точек (checkpoints) [4-6]. По этой технологии система должна иметь стабильную память, которая не меняется при отказах. Соответствующие программные средства периодически сохраняют информацию о состоянии процессов приложения в стабильной памяти. Все процессы также имеют доступ к устройству стабильной памяти.  В случае отказа или сбоя, записанная в стабильную память информация используется для повторения вычисления с момента, когда была записана эта информация, то есть выполняется откат назад по времени. Данные, сохранение которых позволяет выполнить откат, называются контрольной точкой. В качестве устройства стабильной памяти может использоваться дисковый том, энергонезависимая оперативная память, память другого узла или узлов кластера. В последнем случае узел, которому требуется сохранить информацию, пересылает ее через быстрый канал связи на другой узел. Стабильная память после отказа одного из узлов должна быть доступной узлу, на котором делается повтор.

Однако решение, опирающееся только на контрольные точки, не является прозрачным, поскольку не скрывает от клиентов факт отказа системы и требует от них выполнения определенных действий. Так как при работе процессы обмениваются сообщениями, возможны два варианта решения проблемы. Первый - все процессы выполняют записи контрольных точек одновременно, что затруднительно. Второй вариант, при несоблюдении синхронности, - возврат в каждом процессе к такому скоординированному набору контрольных точек, при котором невозможна противоречивая ситуация. Такая ситуация возникает, когда один процесс вернулся к контрольной точке, после которой он должен получить сообщение от другого процесса, а этот другой процесс вернулся к точке, которая следует за выдачей этого сообщения. Однако при повторе ожидаемое первым процессом сообщение не поступит. В этом случае  возможен эффект домино, в результате процессы оказываются отброшены как угодно далеко назад.

В этом состоит первая проблема, которую необходимо преодолеть.

Если нужно, чтобы последствия отказа узла не были видны клиенту,  это означает:
\begin{itemize}
\item клиент не должен терять и потом восстанавливать соединения с сервером;
\item клиент не должен повторять свои запросы;
\item клиент не должен повторно получать сообщения, которые он уже получил.
\end{itemize}

Вторая проблема, которую надо решать, связана с недетерминированностью поведения сервера приложений. Приведем пример.  Пусть имеется система продажи билетов на самолеты. Два клиента одновременно обратились к системе с запросом билета на один и тот же рейс. Клиентам безразлично, какие места им зарезервирует система. Система выполняет запросы клиентов параллельно, поэтому в какой-то момент между процессами, обрабатывающими эти запросы, может возникнуть конкуренция за ресурс - в данном случае, скажем, рейс. Один из процессов захватывает ресурс первым, резервирует место и освобождает ресурс. Потом второй процесс проделывает то же самое.

Порядок, в котором в этом примере процессы захватили ресурс, зависит от многих факторов и, в конечном счете, случаен. Однако  это не мешает правильному функционированию системы, поскольку клиентам важно одно - получить билеты, причем на разные места. Однако отсутствие детерминизма в поведении приложения приводит к тому, что при повторном выполнении могут быть получены другие результаты: например, клиенту уже сообщено, что ему зарезервировано место №5, а при повторе может получиться, что зарезервировано место №6. Система должна устранить это несоответствие и сделать его невидимым для клиента.
\begin{figure*} %fig1
\vspace*{1pt}
\begin{center}
\mbox{%
\epsfxsize=1.6in
\epsfxsize=100mm
\epsfbox{BbR-1.eps}
}
\end{center}
\vspace*{-9pt}
\Caption{Базовый вариант трубы с разными выходными устройствами
(цилиндрическое, расширяющееся и сужающееся сопло)
\label{f1bab}}
\vspace*{-3pt}
\end{figure*}

Недетерминированность поведения системы это следствие, по крайней мере, двух обстоятельств. Во-первых, это присущая системам с разделением времени неопределенность в порядке выполнения процессов. Во-вторых, это конкуренция процессов за общие ресурсы. Перечислим некоторые причины недетерминированного поведения приложений:
\begin{itemize}
\item синхронизация процессов с помощью семафоров или атомарных операций над операндами в общей памяти процессов;
\item зависимость от порядка получения клиентских запросов;
\item время, затраченное процессом на обработку полученного запроса;
\item генераторы случайных чисел;
\item системное управление процессами и потоками;
\item локальные таймеры;
\item доступ к реальному времени.
\end{itemize}

По различным  причинам время, которое тратится на выполнение вычислительной задачи с одними и теми же исходными данными, не является константой, то есть повторное выполнение может дать другое время. Процессы используют общие ресурсы, обращение к которым требует организации очередности выполнения (сериализации) - первым пришел, первым захватил. И, наконец,  результат работы процесса может зависеть от состояния ресурса, а это состояние может изменить другой процесс, ранее захвативший ресурс. Все это создает значительные трудности при попытках воспроизведения поведения процессов с сохраненной контрольной точки.

Прозрачная отказоустойчивость серверов приложений обычно осуществляется переносом приложения на другой узел кластера, идентичный первому по конфигурации аппаратных средств и операционной среды. Это делается методом, называемым snapshot/restore. На основном узле (оригинале)  периодически фиксируется состояние приложения на этом узле кластера (так называемый снимок или snapshot). После отказа оригинала на резервном узле (копии) делается восстановление (restore), то есть восстанавливается последнее зафиксированное состояние приложения. Операционная среда при этом приводится в состояние, которое соответствует моменту изготовления снимка. После этого узел-копия продолжает работу с зафиксированного места. Сравнение метода  snapshot/restore с другими подходами приведено в [7].

Ниже рассматриваются информационные  технологии, позволяющие решить ряд принципиальных вопросов, связанных с реализацией прозрачной отказоустойчивости серверов приложений. Ими являются:
\begin{itemize}
\item виртуализация операционной среды, в которой работает серверное приложение;
\item отказоустойчивая реализация протокола TCP;
\item создание контрольных точек состояния приложения и файловой системы, которые делаются внешним по отношению к приложению образом;
\item восстановление серверного приложения на основании контрольной точки.
\end{itemize}
\begin{figure*} %fig1
\vspace*{1pt}
\begin{center}
\mbox{%
\epsfxsize=1.6in
\epsfxsize=100mm
\epsfbox{BbR-1.eps}
}
\end{center}
\vspace*{-9pt}
\Caption{Базовый вариант трубы с разными выходными устройствами
(цилиндрическое, расширяющееся и сужающееся сопло)
\label{f1bab}}
\vspace*{-3pt}
\end{figure*}

\section{МОДЕЛЬ ОПИСАНИЯ ПОВЕДЕНИЯ ПРИЛОЖЕНИЯ}

Предлагаемый подход опирается на построение модели вычислений, связанной с использованием понятия времени в многопроцессных приложениях. Впервые подобные проблемы были изучены в классической работе Л. Лампорта [8].

Многопроцессными приложения называются потому, что в них параллельно работают несколько процессов. Процесс ведет себя детерминированно, пока в предписанном кодом порядке выполняет процессорные инструкции. Конечно, его работа может быть прервана практически в любой момент и процессор передан другому процессу или ядру. Поэтому абсолютное время, которое затрачивает процесс на выполнение определенной работы, не  константа, а случайная  величина. То же  относится к относительному времени, то есть времени, которое процесс занимал процессор,  поскольку одни и те же обращения к операционной среде могут вызвать работы разной длительности, а значит потребовать разное время на свое выполнение.

Кэшированность инструкций и данных, а также длина хэш-списков влияют на действительное время пребывания в операционной среде. Утрачивает смысл понятие одновременность действий, поскольку  нельзя установить, выполнили ли два разных процесса какие-либо действия одновременно или одно из них предшествовало другому. Таким образом, с процессом можно связать только его локальное время, которое линейно упорядочивает события,  происходившие в этом процессе.  Глобальное время, линейно упорядочивающее действия во всех процессах, отсутствует. Расстояние (в этом качестве используется время) между действиями оказывается случайной величиной.

Эти соображения важны, поскольку процессы в интересующих нас приложениях взаимодействуют и используют общие ресурсы. Для взаимодействия они используют средства синхронизации, предоставляемые операционной средой - например, наборы семафоров SVR4 (System V Release 4), POSIX-семафоры, бинарные семафоры и другие примитивы взаимного исключения (POSIX- mutual exclusion locks) и т.д. Подобные средства операционной среды, которые позволяют процессам синхронизировать свою деятельность друг с другом или сериализовать обращения к совместно используемым объектам,  будут ниже  называться ресурсами.

С каждым ресурсом связано свое локальное время, линейно упорядочивающее события в жизни ресурса. Например, в случае двоичных семафоров это создание семафора, а также его захват и освобождение процессом. Заметим, что событие - это не намерение процесса (например, захватить бинарный семафор), а сам факт захвата семафора процессом (т.е. успешное выполнение намерения). От изъявления намерения до его осуществления может многое произойти. Например, семафор, который хочет захватить рассматриваемый процесс, принадлежал другому процессу, потом тот процесс его освободил, но семафор был сначала передан операционной средой третьему процессу, который также на него претендовал, и т.д. Поведение рассматриваемого процесса в это время нас не интересует - он ресурсом еще не овладел, а только его захват определяет его дальнейшее поведение. По причинам,  изложенным выше, расстояние между двумя событиями - случайная величина. Однако, события замечательны тем, что они одновременно присутствуют и в локальном времени процесса, и в локальном времени ресурса. Поэтому все, что произошло в истории процесса или/и ресурса до этого события, предшествует ему. Далее  будет считаться, что истории и ресурсов и процессов состоят только из событий, причем между двумя последовательными событиями в жизни процесса последний ведет себя детерминированно.

Это означает, что на  поведении процесса сказывается только его предыдущая история, то есть состояние ресурсов, с которыми он взаимодействовал. Это свойство процессов ниже будет называться локальной детерминированностью. Этим свойством не обладают ресурсы, поскольку - следующее событие в истории ресурса не определяется однозначно по его предыдущей истории. Утверждение, касающееся детерминированного поведения процессов, неявно опирается на предположение,  что учтены все ресурсы, которые могут привести к  недетерминированности процессов.

Таким образом, описанное нами очень неформально время в многопроцессном комплексе представляет собой отношение частичного порядка, введенное на множестве событий. Зная полное состояние комплекса в некоторый момент времени,  нельзя однозначно определить, какое событие в истории ресурса наступит следующим. Можно говорить только о вероятности наступления того или иного события. Недетерминированность поведения есть следствие двух обстоятельств. Во-первых, это неопределенность времени, которое тратит процесс на переход от одного события к другому. Во-вторых, конкуренция процессов за общие ресурсы.

Выполнение приложения, на множестве событий которого введена частичная упорядоченность, можно описать направленным ациклическим графом выполнения. Вершинами этого графа являются события, с каждым  из которых связаны две входящие в него дуги. Одна дуга начинается в событии, которое непосредственно предшествует данному событию в истории процесса, другая - в истории ресурса.

Построение средств обеспечения прозрачной отказоустойчивости приложений опирается на следующее утверждение: для восстановления работы приложения после отказа достаточно располагать:
\begin{itemize}
\item контрольной точкой, которая отражает на некоторый момент времени состояния процессов и других ресурсов, образующих приложение;
\item графом выполнения приложения, который описывает работу приложения, начинающуюся с контрольной точки и заканчивающуюся отказом. Данные, которые нужны для построения графа выполнения, далее называются протоколом.
\end{itemize}
\begin{figure*} %fig1
\vspace*{1pt}
\begin{center}
\mbox{%
\epsfxsize=1.6in
\epsfxsize=100mm
\epsfbox{BbR-1.eps}
}
\end{center}
\vspace*{-9pt}
\Caption{Базовый вариант трубы с разными выходными устройствами
(цилиндрическое, расширяющееся и сужающееся сопло)
\label{f1bab}}
\vspace*{-3pt}
\end{figure*}

Вся эта информация должна находиться в стабильной памяти, не разрушающейся при отказе.

Ниже неформально описан алгоритм восстановления работы приложения после отказа, который опирается на наличие контрольной точки и графа выполнения. Будем считать, что в распоряжении имеются средства, позволяющие остановить процесс в тот момент, когда он намерен совершить некоторую операцию над ресурсом. Заметим, что событие в графе выполнения соответствует не изъявлению намерения, а его удовлетворению, то есть завершению выполнения операции.

Предварительно сделаем следующее:
\begin{itemize}
\item используя контрольную точку, приведем приложение в состояние, соответствующее этой контрольной точке;
\item в графе выполнения пометим все вершины (события) как "не наступившие". У некоторых вершин графа отсутствуют им непосредственно предшествующие; соответствующие события наступили сразу же после создания контрольной точки. Для каждой такой вершины включим в граф дополнительную вершину, ей предшествующую в истории процесса, и отметим эту дополнительную вершину как "наступившую";
\item разрешим процессам приложения выполняться.
\end{itemize}

Пусть некоторый процесс проявляет намерение выполнить операцию над каким-либо ресурсом. Отыщем для этого процесса в его истории последнее наступившее событие. Следующее в его истории событие - это то, которое соответствует требуемой операции. Посмотрим, наступило ли событие в истории ресурса, которое ему предшествует. Если нет, переведем процесс в состояния ожидания, отметив в предшествующем событии, что данный процесс ожидает его наступления. Если да, разрешим процессу выполняться, то есть выполнить операцию над ресурсом.

Пусть некоторый процесс объявляет, что он выполнил операцию над каким-либо ресурсом (это соответствует моменту протоколирования при оригинальном выполнении). Отыщем для этого процесса в его истории последнее наступившее событие и перейдем к следующему событию в его истории. Это опять то событие, которое мы рассматриваем. Отметим его как "наступившее". Если наступления этого события ожидал какой-нибудь процесс, выведем этот процесс из состояния ожидания. Наконец, разрешим процессу, выполнившему операцию, продолжаться дальше.

Когда выясняется, что наступили все события графа выполнения, повторное выполнение считается законченным.

Естественным следствием из сказанного является следующее утверждение: для того, чтобы размер протокола не рос неограниченно, нужно периодически создавать контрольные точки, очищая при этом протокол.

\section{ФОРМАЛЬНОЕ ОПИСАНИЕ МОДЕЛИ ПОВЕДЕНИЯ МНОГОПРОЦЕССНОГО ПРИЛОЖЕНИЯ}
\begin{figure*} %fig1
\vspace*{1pt}
\begin{center}
\mbox{%
\epsfxsize=1.6in
\epsfxsize=100mm
\epsfbox{BbR-1.eps}
}
\end{center}
\vspace*{-9pt}
\Caption{Базовый вариант трубы с разными выходными устройствами
(цилиндрическое, расширяющееся и сужающееся сопло)
\label{f1bab}}
\vspace*{-3pt}
\end{figure*}

Опишем формально поведение приложения, неформальное описание которого было приведено выше. Рассматриваются два типа объектов:
\begin{itemize}
\item ресурсы (r), например, наборы семафоров (POSIX- или SVR4-семафоры), бинарные семафоры (POSIX-mutex's), таймер реального времени, сокеты (sockets), то есть двусторонние виртуальные соединения с внешним миром;
\item процессы (p), например, процессы или потоки (threads) пользователя.
\end{itemize}

\end{multicols}

\label{end\stat}

%\def\stat{batr}

\def\tit{НОВЫЙ МЕТОД ВЕРОЯТНОСТНО-СТАТИСТИЧЕСКОГО\newline
АНАЛИЗА ИНФОРМАЦИОННЫХ ПОТОКОВ
В~ТЕЛЕКОММУНИКАЦИОННЫХ СЕТЯХ$^*$}
\def\titkol{Новый метод вероятностно-статистического
анализа информационных потоков
в~телекоммуникационных сетях}
\def\autkol{Д.\,А.~Батракова, В.\,Ю.~Королев, С.\,Я.~Шоргин}
\def\aut{Д.\,А.~Батракова$^1$, В.\,Ю.~Королев$^2$, С.\,Я.~Шоргин$^3$}

\titel{\tit}{\aut}{\autkol}{\titkol}

{\renewcommand{\thefootnote}{\fnsymbol{footnote}}\footnotetext[1]{Работа 
выполнена при поддержке РФФИ, проекты №№\,04-01-00671, 05-07-90103.} 
\renewcommand{\thefootnote}{\arabic{footnote}}}
 \footnotetext[1]{ИПИ РАН, 
daria.batrakova@gmail.com} \footnotetext[2]{Факультет вычислительной математики 
и кибернетики МГУ им.~М.\,В.~Ломоносова, ИПИ РАН, vkorolev@comtv.ru} 
\footnotetext[3]{ИПИ РАН, sshorgin@ipiran.ru}



\Abst{В данной работе предлагается метод исследования стохастической структуры
хаотических информационных потоков в сложных телекоммуникационных
сетях. Предлагаемый метод основан на стохастической модели
телекоммуникационной сети, в рамках которой она представляется в виде
суперпозиции некоторых простых последовательно-параллельных структур.
Эта модель естественно порождает смеси гамма-распределений для времени
выполнения (обработки) запроса сетью. Параметры получаемой смеси
гамма-распределений характеризуют стохастическую структуру
информационных потоков в сети. Для решения задачи статистического
оценивания параметров смесей экспоненциальных и гамма-распределений
(задачи разделения смесей) используется ЕМ-алгоритм. Чтобы проследить
изменение стохастической структуры информационных потоков во времени,
ЕМ-алгоритм применяется в режиме скользящего окна. Описывается
программный инструментарий для применения полученного решения к
реальным статистическим данным. Приводится интерпретация результатов.}

\KW{телекоммуникационные сети; информационные потоки;
разделение смесей  распределений;
метод скользящего окна;  программа для разделения смесей}

\vskip 24pt plus 9pt minus 6pt

\thispagestyle{headings}

\begin{multicols}{2}


\label{st\stat}

\section{Введение}

Развитие телекоммуникационных сетей, их усложнение поставило перед
инженерами важную прикладную задачу исследования характеристик
информационных потоков, возникающих в этих сетях. Здесь под
информационным потоком мы будем понимать упорядоченное движение
любого вида информации по сети.

Если на заре эры телекоммуникаций, в эпоху первых телефонных линий и
телеграфа эта проблема не была столь насущной, то со временем, при
постепенном охвате мирового пространства сетями возникла необходимость в
построении и исследовании адекватных моделей сетей и процессов,
происходящих в них.

\thispagestyle{headings}


Современные сети~--- это \textit{конвергентные} сети, т.е.\ совокупность крайне
разнородных как по топологии, так и по физической архитектуре сетей, которые
предлагают конечному пользователю самые разнообразные сервисы. Это~--- огромное
виртуальное и физическое пространство, состоящее из миллионов процессоров,
операционных платформ, линий передачи данных и стыковочных узлов.
%
Существует множество классификаций телекоммуникационных сетей по различным
признакам:
\begin{itemize}
\item масштабу (локальные сети~--- LAN, масштаба города~---
MAN, широкого масштаба~--- WAN);
\item топологии, или логической организации (<<звезда>>,
<<кольцо>>, <<шина>>);
\item физической организации (оптические сети, радио);
\item предлагаемым услугам (сотовые сети, для доступа в
Интернет);
\item назначению (военные, гражданские) и~др.
\end{itemize}


Конвергентная сеть входит во все эти классы, причем, как правило,
обладает всеми этими признаками. Поэтому построение модели для ее анализа
является и очень важной, и очень сложной задачей.

Существуют достаточно многочисленные математические методы, ориентированные на
моделирование и анализ телекоммуникационных сетей. В~большинстве своем они
основываются на теории массового обслуживания, то есть разделе теории
вероятностей, посвященном описанию функционирования сложных систем обслуживания
(в том чис\-ле телекоммуникационных сетей и систем) с помощью стохастических
процессов особого вида и анализу таких процессов. Указанная теория является
весьма развитой и широко применяется на практике. Тем не менее, ее применимость
ограничена~--- во-первых, все возрастающей сложностью структур и дисциплин
обслуживания в рас\-смат\-ри\-ва\-емых реальных сетях. Эта сложность во многих
случаях принципиально не может найти адекватного отображения в моделях
массового обслуживания, даже несмотря на постоянно растущую сложность самих
этих моделей. В результате даже модели, допускающие точный математический
анализ, дают возможность расчета всего лишь приближенных значений характеристик
реальных сетей, ибо предположения, принимаемые при построении моделей, во
многих случаях не соответствуют практике. Во-вторых, для описания
телекоммуникационной сети в виде сети массового обслуживания исследователь
должен располагать детальным описанием структуры сети, что далеко не всегда
имеет мес\-то на практике. В-третьих, разработано крайне мало моделей массового
обслуживания, в которых используется в качестве входной информация о
наблюдаемых (статистических) показателях функционирования сети; в то же время,
такая информация очень часто доступна исследователю, и ее использование при
анализе сети весьма целесообразно.

В данной работе предлагается в определенной степени альтернативный подход к
моделированию сложных телекоммуникационных сетей. Строится и исследуется
вероятностная модель сложной телекоммуникационной сети как суперпозиции
достаточно простых структур. При этом практически никакая априорная информация
о структуре исследуемой сети не используется~--- наоборот, в результате
исследования модели исследователь получает приближенное представление об этой
структуре. Характеристики типовых простых структур, составляющих в совокупности
модель сети, и сети в целом при этом подходе описываются
гам\-ма-рас\-пре\-де\-ле\-ни\-я\-ми. Ставится задача оценки параметров модели
на основе статистических данных о функционировании сети, а также предлагается
математическое решение этой задачи. В статье описан также созданный на основе
разработанных математических методов программный инструментарий и приведены
результаты расчетов для реального трафика. {\looseness=-1

}

\section{Математическая модель и~постановка задачи}

\subsection{Логическая модель сети}
 %1.1

Рассмотрим абстрактную \textit{конвергентную телекоммуникационную
сеть}. Это может быть как крупномасштабная транспортная сеть (WAN), сеть
отдельного оператора масштаба города (MAN) с различными сервисами, так и
локальная сеть (LAN).

Любой из этих случаев можно описать как ($p,\,q$)-\textit{сеть}.

\medskip
\textbf{Определение 1.} В теории графов и сетей под ($p,\,q)$-сетью понимается
набор вида $S =$\linebreak $=(G,\,V^\prime ,\,V^{\prime\prime})$, где $G$~---
граф, а $V^\prime$ и $V^{\prime\prime}$~--- выборки из множества $V(G)$ (вершин
графа) длины~$p$ и $q$ соответственно. При этом выборка $V^\prime$
($V^{\prime\prime}$) считается \textit{входной} (\textit{выходной}) выборкой, а
ее $i$-я вершина называется $i$-\textit{м} \textit{входным} (\textit{выходным})
\textit{полюсом} или, иначе, $i$-\textit{м} \textit{входом} (\textit{выходом})
сети~$S$. Вершины, не участвующие во входной и выходной выборках сети,
считаются ее внутренними вершинами~\cite{1bat}.

Сеть $S$ (рис.~\ref{f1bat}) имеет $p$ точек входа~--- точек соединения
с внешней средой (это могут быть точки стыковки разнородных сетей, сетей
различных операторов, физические подключения к интерфейсам
маршрутизаторов и~т.п.). Под \textit{внешней средой} будем понимать другие
сети, которые передают данные в сеть~$S$. Отдельные <<единицы>> данных
(кадры, сообщения, датаграммы, пакеты) поступают на входы сети~$S$,
обрабатываются и подаются на каждый из $q$ выходов, которые могут быть
соединены как с конечными пользователями, так и с другими сетями.
\begin{figure*} %fig1
\vspace*{1pt}
\begin{center}
\mbox{%
\epsfxsize=139.7mm \epsfbox{bat-1.eps}
%\epsfxsize=139.698mm
%\epsfbox{bek-3.eps}
}
\end{center}
\vspace*{-9pt} \Caption{Абстрактная телекоммуникационная сеть \label{f1bat}}
\end{figure*}

Следует отметить, что структура сложных телекоммуникационных сетей обладает
свойством некоторого самоподобия, т.е.\ на каком бы уровне сетевой архитектуры
мы ни рассматривали поведение информационных потоков, мы можем выделить
некоторые базовые структуры, субпотоки, суперпозицией которых мы можем получить
модель конкретной сети, какой бы уровень <<детализации>> сегментов сети мы ни
взяли. Так, например, физические подключения к интерфейсам оптического
кросс-коннекта в этом смысле подобны <<виртуальным>> подключениям к портам TCP
на сервере приложений.

Итак, независимо от уровня сетевой архитектуры мы можем
рассматривать некоторую величину, характеризующую количество каких-либо
ресурсов сети~$S$, занимаемых в процессе передачи и обработки данных.  Это
могут быть ресурсы, относящиеся как к <<объему>> (памяти сетевого
устройства, количеству занятых линий, размеру пакета), так и ко <<времени>>
(времени обслуживания заявки, времени простоя). Эта величина случайна, т.к.\
мы не можем абсолютно точно сказать для сложной телекоммуникационной
сети, какое сообщение на какой из входов поступит и какого размера оно будет.
Таким образом, случайный характер данной величины определяется
случайностью поведения внешней среды.

Пусть $R$~--- это описанная выше случайная величина, $R>0$. Далее, не
ограничивая общности, будем подразумевать под ней время, необходимое для
какой-либо операции сети (обработки <<единицы>> данных), предполагая, что
время обработки прямо зависит от объема сообщения.

\subsection{Вероятностная модель сети} %1.2.

Даже не зная реальной топологии сети, мы можем описать
функционирование некоторых ее участков как процесс выполнения операций
(задач сети) в последовательном  порядке (например, если доступен только
один канал данных) или как процесс одновременного выполнения субопераций
(когда доступно более одного пути выполнения). Это значит, что мы можем
представить функционирование сложной телекоммуникационной сети как
\textit{суперпозицию} таких <<последовательных>> и <<параллельных>>
блоков.

Для построения вероятностной модели распределения~$R$ используется
комбинация асимптотического подхода, основанного на предельных теоремах
теории вероятностей, и принципа максимальной неопределенности (энтропии).

Рассмотрим следующую модель. Предположим, что мы можем разделить
сеть~$S$ на несколько сегментов $S_i$. Пусть $T$~--- случайная величина,
время выполнения операции в отдельно взятом блоке $S_i$ (сегменте сети).

Если операции выполняются \textit{параллельно}, то время, необходимое
для их выполнения~--- это максимальное время, затрачиваемое на какую-либо
субоперацию:
$$
T = \underset{i}{\max}\, T_i\,,
$$
где $T_i$~--- случайные величины для со\-от\-вет\-ст\-ву\-ющих субопераций.
Модель такого выполнения пред\-став\-ле\-на на рис.~\ref{f2bat}.

\begin{figure*} %fig2
\vspace*{1pt}
\begin{center}
\mbox{%
\epsfxsize=117.271mm
\epsfbox{bat-2.eps}
}
\end{center}
\vspace*{-9pt}
\Caption{Параллельное выполнение
\label{f2bat}}
\end{figure*}

Известно, что предельное распределение экстремальных значений для
выборок ~--- это экспоненциальное распределение с плотностью~\cite{2bat}
$$
f(x) =
\begin{cases}
\lambda e^{-\lambda x}\,, & x>0\,,\\
0\,, & x\leq 0\,,
\end{cases}
$$
где $\lambda >0$~--- параметр масштаба.

 Учитывая также энтропийный подход, естественно будет считать
распределение $T$ экспоненциальным, т.к.\ экспоненциальное распределение
обладает наибольшей энтропией среди всех распределений с $x>0$.

Если же операции сети выполняются \textit{последовательно}, то величина
$T$~--- это сумма времен $T_i$, необходимых для выполнения каждой
субоперации:
$$
T = \sum\limits_i T_i\,,
$$
где $T_i$~--- случайные величины для со\-от\-вет\-ст\-ву\-ющих субопераций.
%
Такая модель представлена на рис.~\ref{f3bat}.

\begin{figure*} %fig3
\vspace*{1pt}
\begin{center}
\mbox{%
\epsfxsize=139.592mm
\epsfbox{bat-3.eps}
}
\end{center}
\vspace*{-9pt}
\Caption{Последовательное  выполнение
\label{f3bat}}
\end{figure*}

Это значит, что распределение общей длительности $T$ выполнения
блока~--- это свертка распределений <<элементарных>> времен $T_i$
(экспоненциально распределенных).

Известно, что результатом свертки экспоненциальных распределений
является гамма-распределение, определяемое плотностью
$$
\g_{\lambda , \alpha} (x) =
\begin{cases}
\fr{\lambda_0^{\alpha_0}}{\Gamma (\alpha_0 )}\,x^{\alpha_0-1}
e^{\lambda_0 x}\,, & x>0\,,\\
0,\, & x\leq 0\,,
\end{cases}
$$
где $\alpha >0$~--- параметр формы,  $\lambda >0$  параметр масштаба, а
$\Gamma (z)$~--- гамма-функция Эйлера:
$$
\Gamma (z) = \int\limits_0^\infty x^{z-1} e^{-x}\,dx\,.
$$

\begin{figure*} %fig4
\vspace*{1pt}
\begin{center}
\mbox{%
\epsfxsize=120.831mm
\epsfbox{bat-4.eps}
}
\end{center}
\vspace*{-9pt}
\Caption{Модель пути  обработки сообщения сетью~$S$
\label{f4bat}}
\end{figure*}

Известно~\cite{2bat}, что класс гамма-распределений замкнут над операцией
свертки, поэтому ре\-зуль\-ти\-ру\-ющее распределение случайной величины~$R$
будет также гамма-распределением
$$
\g_{\lambda , \alpha} (x) =
\begin{cases}
\fr{\lambda^{\alpha}}{\Gamma (\alpha )}\,x^{\alpha -1} e^{-\lambda x}\,, &
x>0\,,\\
0,\, & x\leq 0\,.
\end{cases}
$$

В силу случайного характера ввода данных в сеть~$S$ из внешней среды маршрут
передачи данных становится случайным, что представлено на рис.~\ref{f4bat}. Это
означает, что параметры ре\-зуль\-ти\-ру\-юще\-го распределения~$R$ тоже
случайны. Отсюда имеем следующую модель \textit{смеси
гам\-ма-рас\-пре\-де\-ле\-ний}, определяемой плотностью

\begin{equation} %1
p(x) = \iint \g_{\lambda , \alpha}(x)\,dH (\lambda ,\,\alpha )\,,
\end{equation}
где $H(\lambda , \alpha )$~--- смешивающая функция, функция распределения
входных параметров.

Поясним понятие \textit{смеси распределений}.

\medskip
\textbf{Определение~2.} Пусть имеется двух\-па\-ра\-мет\-ри\-че\-ское
семейство $n$-мерных плотностей  распределения
\begin{equation}
F = \{ f_\omega (x;\, \theta (\omega ))\}\,,
\end{equation}
где одномерный (целочисленный или непрерывный) параметр $\omega$ в
качестве нижнего индекса функции $f$ определяет специфику общего вида
каж\-до\-го компонента~--- распределения смеси, а в качестве аргумента при
многомерном, вообще говоря, параметре $\theta$ определяет зависимость
значений хотя бы части компонентов этого параметра от того, в каком именно
составляющем распределении $f_\omega$ он присутствует. Кроме того, пусть
$P = \{P(\omega )\}$~--- \textit{семейство смешивающих функций}
распределения.

Функция плотности распределения
\begin{equation}
f(x) = \int f_\omega (x;\,\theta(\omega ))\,dP (\omega )
\end{equation}
называется $P$-\textit{смесью} (или просто \textit{смесью})
\textit{распределений} семейства~$F$,  интеграл в~(3) понимается в смысле
Лебега--Стильтьеса~\cite{3bat}.

\medskip
\textbf{Определение 3.} Семейство смесей~(3) называется
\textit{идентифицируемым} (\textit{различимым}), если из равенства
$$
\int f_\omega (x;\,\theta(\omega ))\,dP (\omega ) =\int f_\omega
(x,\,\theta(\omega )) dP^*(\omega )
$$
следует, что $P(\omega ) \equiv P^*(\omega )$ для всех $P \in P(\omega
)$~\cite{3bat}.

\subsection{Постановка задачи} %1.3.

Перед нами встает задача \textit{разделения} такой смеси. Вообще говоря,
задача разделения смесей распределений со смешивающими функциями
общего вида является \textit{некорректно поставленной}, т.к.\ она допускает
существование нескольких решений. Поэтому будем искать решение в классе
\textit{конечных идентифицируемых смесей распределений}, где смешивающая
функция дискретна.

Для этого сузим данное выше определение и будем рассматривать в дальнейшем лишь 
случай конечного числа $k$ возможных значений па\-ра\-мет\-ра~$\omega$, что 
соответствует конечному числу скачков смешивающих функций $P(\omega )$.  
Величины этих скачков как раз и будут играть роль \textit{удельных весов} 
(\textit{априорных вероятностей}) $p_j$ компонентов смеси. Более того, в нашем 
случае мы постулируем также однотипность компонентов распределений $f_j$, т.е.\ 
принадлежность всех $f_j$ к одному общему па\-ра\-мет\-ри\-че\-ско\-му 
семейству $\{ f(X;\,\theta )\}$, где $\theta$~--- многомерный, вообще говоря, 
параметр. Так что~(3) в этом случае может быть записано в виде
\begin{equation} %4
p(x) = \sum\limits^k_{j=1} p_j f_j (x;\,\theta_j )\,.
\end{equation}

Переформулируем понятие идентифицируемости (различимости) смесей
специально применительно к такому виду смесей.

\medskip
\textbf{Определение 4.} \textit{Конечная смесь}~(3) называется
\textit{идентифицируемой} (\textit{различимой}), если из равенства
$$
\sum\limits_{j=1}^k p_j f_j (x;\,\theta_j ) = \sum\limits_{l=1}^{k^*} p_l^* f_l
(x;\,\theta_l^* )
$$
следует, что $k=k^*$ и для любого $j$ ($1\leq j \leq k$) найдется такое $l$ 
($1\leq l \leq k^*$), что $p_j = p_l^*$ и $f_j (x;\,\theta_j ) = f_l 
(x;\,\theta_l^* )$~\cite{3bat}.

Решить эту задачу в выборочном варианте~--- значит по выборке
классифицируемых наблюдений
$X_1,\ldots , X_n, $ извлеченной из генеральной совокупности, яв\-ля\-ющей\-ся смесью~(3)
генеральных совокупностей типа~(2) (при заданном общем виде составляющих
смесь функций $f_j (x;\,\theta_j )$), построить статистические оценки для числа
компонентов смеси~$k$, их удельных весов $p_j$ и, главное, для каждого из
компонентов %f_j (x;\,\theta_j )$ анализируемой смеси. Далее будет считать, что
функции $f_j$ однозначно определяются своими параметрами $\theta_j$: $f_j
=f(x;\,\theta_j)$.

Однако не следует ставить знак тождества между задачей разделения смеси
и задачей статистического оценивания параметров в модели~(4) по выборке $
X_1,\ldots , X_n$, поскольку задача разделения сохраняет смысл и
применительно к генеральным совокупностям, т.е.\ в теоретическом
варианте~\cite{3bat}.

Итак, для статистического анализа на основе реальных данных мы
аппроксимируем нашу общую модель~(1) следующей:
$$
p(x) \approx \hat{p}(x) = \sum\limits_{j=1}^k p_j \g_{\lambda_j , \alpha_j}
(x)\,,
$$
где $p_j$~--- дискретные смешивающие параметры, $\g_{\lambda_j , \alpha_j}
(x)$~--- плотности гамма-распределений.

Такая аппроксимация не только позволяет решить поставленную статистическую
задачу, но и полу\-чить наглядную визуализацию результатов. Существуют
достаточно эффективные методики разделения смесей распределений, среди них~---
семейство так называемых \textit{ЕМ-алгоритмов}
(\textit{Expectation-Maximization Algorithms}).

Полученные результаты могут быть достаточно просто интерпретированы. Число
компонентов смеси символизирует число типичных параллельных или
последовательных структур. Значения параметров составляющих смесь
гам\-ма-рас\-пре\-де\-ле\-ний показывают <<степень параллелизма>>
соответствующей структуры: чем ближе параметр формы к~1, тем выше эта
<<степень>>. И наоборот, чем дальше значение параметра формы от~1, тем больше
последовательных операций выполняется в соответствующем блоке.

Веса компонентов характеризуют примерную долю использования
ресурсов для сообщений, соответствующих каждому распределению входных
данных.

Итак, предложенный подход позволяет получить представление о
стохастической структуре телекоммуникационной сети.

\section{ЕМ-алгоритм разделения смесей распределений}

\subsection{Описание алгоритма} %2.1.

Определяемый ниже итерационный алгоритм решения поставленной в
предыдущем разделе задачи относится к процедурам, базирующимся на
\textit{методе максимального правдоподобия}.

Этот алгоритм позволяет находить максимум логарифмической функции
правдоподобия по параметрам $p_1,\,p_2,\ldots ,\,p_k$, $\theta_1 ,\,\theta_2,\ldots ,\,
\theta_k$ при фиксированном $k$ по выборке $X_1, \ldots , X_n$, т.е.\ решение
оптимизационной задачи вида

\begin{equation} \sum\limits_{i=1}^n \ln \left ( \sum\limits_{j=1}^k p_j f_j
(X_i;\,\theta_j )\right ) \rightarrow \underset{p_j,\,\theta_j}{\max}\,.
\end{equation}

Конкретные алгоритмы, построенные по этой схеме, часто называют
\textit{алгоритмами типа ЕМ}, поскольку в каждом из них можно выделить два
этапа, находящихся по отношению друг к другу в последовательности
итерационного взаимодействия: \textit{оценивание} (\textit{Estimation}) и
\textit{максимизация} (\textit{Maximization})~\cite{4bat}.

Введем в рассмотрение так называемые апостериорные вероятности
$\g_{ij}$ принадлежности наблюдения $X_i$ к $j$-му классу:
\begin{equation} %6
\g_{ij} = \fr{p_j f(X_i;\,\theta_j )}{\sum\limits_{l=1}^k p_l f(X_i;\,\theta_l 
)} \ (i=1,\ldots , n;\ j=1,\ldots ,k)\,.\!\!\end{equation} 
Очевидно, что для 
всех $i=1,\ldots ,n$ и $j=1,\ldots ,k$
$$
\g_{ij} \geq 0,\quad \sum_{j=1}^k \g_{ij} =1\,.
$$


Далее обозначим $\Theta = (p_1,\ldots p_k,\,\theta_1,\ldots ,\theta_k )$ и
представим анализируемую логарифмическую функцию правдоподобия
$$
\ln L(\Theta ) = \sum\limits_{i=1}^n \ln \left (\sum\limits_{j=1}^k p_j f_j
(X_i;\,\theta_j )\right )
$$
в виде
\begin{multline}
\ln L (\Theta ) = \sum\limits_{j=1}^k\sum\limits_{i=1}^n \g_{ij} \ln p_j+{}\\
{}+\sum\limits_{j=1}^k\sum\limits_{i=1}^n \g_{ij} f(X_i;\,\theta_j)-
\sum\limits_{j=1}^k\sum\limits_{i=1}^n \g_{ij} \ln \g_{ij}\,.
\end{multline}

Справедливость этого тождества легко проверяется с учетом
$$
\sum\limits_{j=1}^k \g_{ij} =1\,.
$$

Далее идея построения итерационного алгоритма вычисления оценок
$\hat{\Theta} = (\hat{p}_1,\ldots , \hat{p}_k,\
\hat{\theta}_1,\ldots , \hat{\theta}_k)$
для параметров $\Theta = (p_1,\ldots , p_k,\ \theta_1,\ldots , \theta_k)$ состоит в
следующем:
\begin{enumerate}[1.]
\item Выбирается некоторое \textit{начальное приближение}~$\hat{\Theta}^0$.
\item \textbf{E-step:} вычисляются по формулам~(6) начальные приближения
$\g_{ij}^0$ для апостериорных вероятностей $\g_{ij}$~--- \textit{этап
оценивания}.
\item \textbf{M-step:} затем, возвращаясь к~(7), при вычисленных значениях
$\g^0_{ij}$ следует определить значения $\hat{\Theta}^1$ из условия
максимизации отдельно каждого из первых двух слагаемых правой
части~(7), поскольку первое слагаемое
$$
\sum_{j=1}^k \sum_{i=1}^n \g_{ij} \ln p_j
$$
зависит только от параметров $p_j$, а второе слагаемое
$$
\sum_{j=1}^k \sum_{i=1}^n \g_{ij} f(X_i;\,\theta_j )
$$
зависит только от параметров $\theta_j$~--- \textit{этап максимизации}.
\item Проверяется \textit{условие останова}:
$$
\parallel \Theta^{(t)} - \Theta^{t-1}\parallel <\varepsilon\,,
$$
где $t$~--- номер итерации, а
$\parallel\bullet\parallel$~--- евклидова норма, для некоторого $\varepsilon
>0$.
\end{enumerate}

Очевидно, решение оптимизационной задачи
$$
\sum\limits_{j=1}^k\sum\limits_{i=1}^n \g_{ij}^{(t)}\ln p_j \rightarrow
\underset{p_j}{\max}
$$
дается выражением (с учетом $\sum_{j=1}^k p_j =1$):
$$
p_{ij}^{(t+1)} =\fr{1}{n}\,\sum\limits_{i=1}^n \g_{ij}^{(t)}\,,
$$
где $t$~--- номер итерации, $t = 0$, 1, 2,\,\ldots

Решение оптимизационной задачи
$$
\sum\limits_{j=1}^k \sum\limits_{i=1}^n \g_{ij}^{(t)} f(X_i;\,\theta_j )
\rightarrow \underset{\theta_j}{\max}
$$
получить намного проще решения задачи~(5): выражение для $\theta_j$
записывается с учетом знания конкретного вида функций
$f(X,\,\theta)$~\cite{3bat}.

\subsection{О сходимости алгоритма} %2.2.

В работе М.\,И.~Шлезингера~\cite{5bat}, где эта схема (позднее названная
ЕМ-схемой) впервые предложена, установлены и основные свойства
реа\-ли\-зу\-ющих ее алгоритмов. В частности, было доказано, что при достаточно
широких предположениях \textit{предельные точки} всякой последовательности,
порожденной итерациями ЕМ-алгоритма, являются стационарными точками
оптимизируемой логарифмической функции правдоподобия $\ln L(\Theta )$ и что
найдется неподвижная точка алгоритма, к которой будет сходиться каждая из таких
последовательностей. Если дополнительно потребовать положительной
определенности информационной мат\-ри\-цы Фишера для $\ln L(\Theta )$ при
истинных зна\-че\-ни\-ях па\-ра\-мет\-ра $\Theta$, то можно показать, что
асимптотически по $n\rightarrow\infty$ (т.е.\ при больших выборках) существует
единственное сходящееся (по веро\-ят\-но\-сти) решение $\hat{\Theta}(n)$
уравнений метода максимального правдоподобия и, кроме того, существует в
пространстве параметров $\Theta$ норма, в которой последовательность
$\Theta^{(t)}(n)$, порожденная ЕМ-ал\-го\-рит\-мом, сходится к $\hat{\Theta}
(n)$, если только начальная аппроксимация $\hat{\Theta}^0$ не была слишком
далека от~$\hat{\Theta} (n)$. {%\looseness=1

}

Таким образом, результаты исследования свойств ЕМ-алгоритмов метода
максимального правдоподобия разделения смеси и их практическое
использование показали, что они являются достаточно работоспособными (при
известном чис\-ле компонентов смеси) даже при большом чис\-ле $k$ компонентов и
при высоких размерностях анализируемого признака~$X$~\cite{3bat}.

\subsection{Уравнения для смеси экспоненциальных распределений}
%2.3.

Применим описанный выше алгоритм к разделению смеси
экспоненциальных распределений:
$$
p(x) = \sum\limits_{j=1}^k p_j \lambda_j e^{-\lambda_j x}\,.
$$
Получаем следующие итерационные уравнения:
\begin{align*}
\g_{ij}^{(t+1)} & = \fr{p_j^{(t)}\lambda_j^{(t)}e^{-
\lambda_j^{(t)}X_i}}{\sum\limits_{l=1}^k p_l^{(t)}\lambda_l^{(t)}
e^{-\lambda_l^{(t)}X_i}}\,,\\
p_j^{(t+1)} & = \fr{1}{n}\,\sum\limits_{i=1}^n \g_{ij}^{(t)}\,.
\end{align*}

Чтобы найти  оценки $\lambda_j$, подсчитаем первую производную функции
$$\sum_{j=1}^k\sum_{i=1}^n \g_{ij}^{(t)} \ln (\lambda_j e^{-\lambda_j X_i}):$$
\vspace*{-8pt}
\begin{multline*}
\left ( \sum\limits_{j=1}^k \sum\limits_{i=1}^n
\g_{ij}^{(t)}\ln \left ( \lambda_j
e^{-\lambda_j X_i} \right ) \right )^\prime \lambda_j =\\[-3pt]
{}= \left (
\sum\limits_{j=1}^k\sum\limits_{i=1}^n \g_{ij}^{(t)}\ln (\lambda_j -\lambda_j X_i )
\right )^\prime \lambda_j =\\[-3pt]
{}= \sum\limits_{i=1}^n \g_{ij}^{(t)}\left (
\fr{1}{\lambda_j} - X_i \right )\,.
\end{multline*}

Разрешая уравнение
$$
\sum\limits_{i=1}^n \g_{ij}^{(t)}\left ( \fr{1}{\lambda_j} -X_i\right ) =0
$$
относительно $\lambda_j$, получаем следующее итерационное уравнение:
$$
\lambda_j^{(t+1)} = \fr{\sum\limits_{i=1}^n
\g_{ij}^{(t)}}{\sum\limits_{i=1}^n \g_{ij}^{(t)} X_i}\,.
$$

\subsection{Уравнения для смеси гамма-распределений } %2.4.

Применим теперь ЕМ-алгоритм к смеси гам\-ма-рас\-пре\-де\-ле\-ний вида
$$
p(x) = \sum\limits_{j=1}^k p_j \fr{\alpha_j^{\alpha_j} x^{\alpha_j -
1}}{\lambda_j^{\alpha_j} \Gamma (\alpha_j )}\,e^{-(\alpha_j / \lambda_j)x}\,.
$$

Такая параметризация удобна для нахождения
оценок~$\alpha_j$~\cite{6bat}.

Аналогичным способом выписываются итерационные уравнения:
\begin{align*}
\g_{ij}^{(t+1)} & =   \fr{p_j^{(t)}\fr{(\alpha_j^{\alpha_j} )^{(t)}
x^{\alpha_j - 1}}{(\lambda_j^{\alpha_j} )^{(t)}\Gamma (\alpha_j)}\,
e^{-(\alpha_j /\gamma_j)^{(t)}x}}{\sum\limits_{l=1}^k
p_l^{(t)}\fr{(\alpha_l^{\alpha_l})^{(t)} x^{\alpha_l -
1}}{(\lambda_l^{\alpha_l})^{(t)}\Gamma (\alpha_l )}\,
e^{-(\alpha_l /\lambda_l)^{(t)} x}}\,,\\
p_j^{(t+1)} & = \fr{1}{n}\,\sum\limits_{i=1}^n \g_{ij}^{(t)}\,.
\end{align*}

Далее найдем оценки $\lambda_j$ для данного случая, приравнивая
производную
\begin{equation} %8
\sum\limits_{j=1}^k \sum\limits_{i=1}^n \g_{ij}^{(t)} \ln \left (
\fr{\alpha_j^{\alpha_j} x^{\alpha_j -1}}{\lambda_j^{\alpha_j}\Gamma
(\alpha_j)}\,e^{-(\alpha_j /\lambda_j) x}\right )
\end{equation}
по $\lambda_j$ к нулю и разрешая относительно~$\lambda_j$ уравнение:
$$
\sum\limits_{i=1}^n \g_{ij}^{(t+1)}\left ( \fr{\alpha_j^{(t)}}{\lambda_j^{(t)}}
- \fr{\alpha_j^{(t)}X_i}{\left ( \lambda_j^{(t)}\right )^2}\right ) =0 \,.
$$
Получаем
$$
\lambda_j^{(t+1)} = \fr{\sum\limits_{i=1}^n \g_{ij}^{(t)}
X_i}{\sum\limits_{i=1}^n \g_{ij}^{(t)}}\,.
$$

Для того чтобы получить итерационные уравнения для $\alpha_j$, найдем
первую производную~(8):
\begin{multline*}
\left ( \sum\limits_{j=1}^k\sum\limits_{i=1}^n \g_{ij}^{(t)}\ln \left (
\fr{\alpha_j^{\alpha_j} x^{\alpha_j -1}}{\lambda_j^{\alpha_j}\Gamma (\alpha_j
)}\,e^{-(\alpha_j /\lambda_j ) x} \right ) \right )^\prime \alpha_j ={}\\[-3pt]
{}=\left ( \sum\limits_{j=1}^k\sum\limits_{i=1}^n \g_{ij}^{(t)}\ln \left (
\fr{\alpha_j^{\alpha_j}}{\lambda_j^{\alpha_j}}\right ) - \ln \Gamma (\alpha_j )+{} \right.\\[-3pt]
{}+\left.
(\alpha_j -1 )\ln X_i - \fr{\alpha_j}{\lambda_j}\,X_i \right )^\prime \alpha_j =\\[-3pt]
{}=\sum\limits_{i=1}^n \g_{ij}^{(t)} \left ( \ln \alpha_j +1-\ln \lambda_j -
\fr{\Gamma^\prime (\alpha_j )}{\Gamma (\alpha_j)}\right.+\\[-3pt]
{}+\left. \ln X_i - \fr{X_i}{\lambda_j}\right )\,;
\end{multline*}
\begin{multline*}
\sum\limits_{i=1}^n \g_{ij}^{(t)} \left(  \ln \alpha_j +1 -\ln \lambda_j -{}\right. \\[-3pt]
\left. {}-\fr{\Gamma^\prime (\alpha_j )}{\Gamma (\alpha_j )}+\ln X_i 
-\fr{X_i}{\lambda_j} \right) =0\,;
\end{multline*}
\begin{multline}
\fr{\Gamma^\prime (\alpha_j )}{\Gamma (\alpha_j )} ={}\\[-3pt]
{}= \fr{\sum\limits_{i=1}^n \g_{ij}^{(t)} \left ( \ln \alpha_j +1-\ln\lambda_j 
+\ln X_i -\fr{X_i}{\lambda_j} \right )}{\sum\limits_{i=1}^n \g_{ij}^{(t)}}\,.
\end{multline}
%
Здесь $\Gamma^\prime (\alpha_j ) / \Gamma (\alpha_j )$~--- это
\textit{логарифмическая производная гамма-функции}. Для нее существует так
называемое \textit{разложение Абрамовитца}--\textit{Стигана}~\cite{4bat}:
$$
\fr{\Gamma^\prime (\alpha ) }{ \Gamma (\alpha )} = \mathrm{log}\,\alpha -
\fr{1}{2\alpha }-\fr{1}{12\alpha^2 }+\ldots
$$

Подставим первые три члена разложения в~(9) и разрешим это уравнение
относительно~$\alpha_j$:
$$
\alpha_{ij}^{(t+1)} = \fr{\sum\limits_{i=1}^n
\g_{ij}^{(t+1)}}{2\sum\limits_{i=1}^n \g_{ij}^{(t +1)}\left ( \fr{X_i}{\lambda_j^{(t)}} -
\ln \fr{X_i}{\lambda_j^{(t)}} -1\right )}\,.
$$
В итоге получаем итерационные уравнения для ~$\alpha_j$.

\section{Описание программного обеспечения (программа~ЕМ)}

\subsection{Назначение программы} %3.1.

Разработанная авторами статьи программа ЕМ предназначена для решения задачи
разделения смесей экспоненциальных и гамма-распределений, поставленной в
разд.~2, с использованием ЕМ-ал\-го\-рит\-ма и формул, описанных в разд.~3.

\subsection{Инструменты разработки} %3.2.

Для создания программы была использована среда разработки Microsoft
Visual Studio .NET 2005 и объектно-ориентированный язык C\#. Для
визуализации результатов была использована свободно распространяемая
графическая библиотека ZedGraph~\cite{7bat}.


\subsection{Возможности  программы} %3.3.

\noindent
\begin{itemize}
\item Загрузка выборочных данных из текстового файла
\item Оценивание по выборке параметров смеси экспоненциальных
распределений
\item Оценивание по выборке параметров смеси гамма-распределений
\item Отслеживание изменений параметров смесей распределений во
времени в режиме <<скользящего окна>>
\item Построение гистограммы по выборке
\end{itemize}

\subsection{Входные и выходные данные. Функционирование
программы} %3.4.

В качестве \textit{входных данных} программа ЕМ получает:
\begin{itemize}
\item выборочные данные из текстового файла;
\item число компонентов смеси;
\item размер <<скользящего окна>>;
\item размер класса гистограммы.
\end{itemize}

На \textit{выходе} мы получаем:
\begin{itemize}
\item точечные оценки параметров смеси экспоненциальных
распределений;
\item точечные оценки параметров смеси гамма-распределений;
\item графическое изображение результирующей смеси распределения;
\item графическое изображение компонентов каж\-дой смеси;
\item графическое изображение того, как меняются параметры смесей
распределений с течением времени в режиме <<скользящего окна>>;
\item гистограмма, построенная по выборке;
\item значение статистического теста.
\end{itemize}

Выборочные данные загружаются из текстового файла в память программы и подаются
на вход двум независимо работающим реализациям ЕМ-алгоритма~--- для
идентификации смеси экспоненциальных распределений и для идентификации смеси
гамма-распределений. Результатом их работы являются наборы значений оцениваемых
параметров модели, предложенной в разд.~2. Кроме того, результирующие
распределения визуализируются в виде графиков. В программе можно запустить
режим <<скользящего окна>>, который для всех подвыборок заданного
размера с помощью ЕМ-алгоритма оценивает параметры смесей распределений этих
подвыборок. Все действия программы документируются в окне информации.

\section{Описание тестовых расчетов}

С использованием разработанной программы были проведены тестовые
расчеты на выборочных данных реального сетевого трафика.

На вход программы EM были поданы выборки трафика:
\begin{enumerate}[I]
\item Между лабораторией Lawrence Berkeley (Berkeley, California) и
внешним миром размера примерно 7000~\cite{8bat}~--- \textit{выборка~1}.
\item
Сети радиодоступа ЗАО <<Синтерра>> размера примерно 1000~\cite{9bat}~---
 \textit{выборка~2}.
\end{enumerate}

\subsection{Выборка 1 ``Berkeley''} %5.1.

При числе компонентов смеси~5 и случайном начальном приближении
были получены результаты, представленные в табл.~\ref{t1bat}.


Данные результаты иллюстрирует рис.~\ref{f5bat}.

Гистограмма  на рис.~\ref{f6bat} показывает статистическую значимость
полученных результатов.

Данная выборка обладает той особенностью, что она собиралась в течение
достаточно длительного времени и в ней агрегирован самый разнородный
трафик. Поэтому в ней присутствует не только большое количество
<<коротких>> сообщений (что обычно для выборок из телетрафика), но и
некоторый массив сообщений средней длины, а также определенный
<<выброс>> больших сообщений. Это свидетельствует о \textit{пиковости}
телетрафика на довольно больших промежутках времени.

Как мы видим, ЕМ-алгоритм удачно справился с задачей идентификации
смеси.

\subsection{Выборка~2 ``Synterra''} %5.2.

Результаты применения ЕМ-алгоритма к выборке ``Synterra''
представлены в табл.~\ref{t2bat}.
\begin{table*}\small
\begin{minipage}[t]{76mm}
\begin{center}
\Caption{Результаты применения ЕМ-алго\-рит\-ма к выборке~1 ``Berkeley'' 
\label{t1bat}} \vspace*{2ex}

\tabcolsep=8.7pt
\begin{tabular}{|c|c|c|}
\hline
№&Начальное приближение&Результат\\
\hline
\multicolumn{3}{|c|}{$P$}\\
\hline
0&0,2&0,1896\\
1&0,2&0,1858\\
2&0,2&0,1830\\
3&0,2&0,2259\\
4&0,2&0,2154\\
\hline
\multicolumn{3}{|c|}{$\alpha$}\\
\hline
0&2,7028&10,9783\hphantom{9}\\
1&3,6273&5,8621 \\
2&5,7598&2,7092\\
3&0,2315&1,0235\\
4&0,9110&0,4772\\
\hline
\multicolumn{3}{|c|}{$\lambda$}\\
\hline
0&85,2066&137,1714  \\
1&23,9592&136,7349\\
2&63,8425&132,6482\\
3&\hphantom{9}1,8026&116,7317\\
4&98,3882&102,5278\\
\hline
\end{tabular}
\end{center}
\end{minipage}\hfill
\begin{minipage}[t]{76mm}
%\end{table*}
%\begin{table*}\small
\begin{center}
\Caption{Результаты применения ЕМ-алго\-рит\-ма к выборке~2 ``Synterra'' 
\label{t2bat}} \vspace*{2ex}

\tabcolsep=8.7pt
\begin{tabular}{|c|c|c|}
\hline
№&Начальное приближение&Результат\\
\hline
\multicolumn{3}{|c|}{$P$}\\
\hline
0&0,2&$0{,}3815\hphantom{{}\cdot 10^{-9}}$\\
1&0,2&$0{,}3594\hphantom{{}\cdot 10^{-9}}$\\
2&0,2&$0{,}2589\hphantom{{}\cdot 10^{-9}}$\\
3&0,2&$0{,}4401\cdot 10^{-9}$\\
4&0,2&$0{,}0\hphantom{{}\cdot 10^{-9}999}$\\
\hline
\multicolumn{3}{|c|}{$\alpha$}\\
\hline
0&6,0804&1,5833\\
1&3,1838&0,8554\\
2&1,4886&0,4557\\
3&4,6407&0,2278\\
4&3,7843&0,1139\\
\hline
\multicolumn{3}{|c|}{$\lambda$}\\
\hline
0&17,3387&15,8682\\
1&47,8294&16,9150\\
2&54,1984&19,2866\\
3&\hphantom{1}8,6254&19,2866\\
4&\hphantom{1}5,7252&19,2866\\
\hline
\end{tabular}
\end{center}
\end{minipage}
\end{table*}


Данные результаты иллюстрируют рис.~\ref{f7bat}.


Эти результаты также отражают действительную картину, как показано на
рис.~\ref{f8bat}.


Этот трафик был снят с базовой станции <<Лукойл-Юго-Запад>> сети
широкополосного радиодоступа ЗАО <<Синтерра>>. Сеть радиодоступа
является реализацией так называемой <<последней мили>>, переносящей два
разных вида трафика: данные (Ethernet пакеты) и голос (IP-телефония, VoIP).
Поэтому здесь присутствуют в качестве основной массы короткие, но
интенсивные сообщения (пакеты SIP и голосовые фреймы), а также длинные
сообщения, содержащие данные.

Как мы видим, программная реализация ЕМ-ал\-го\-рит\-ма успешно справилась с
задачей разделения смесей распределений для этих двух выборок, что делает
данную программу удобным инструментом построения стохастической картины
конкретной сети. По полученным данным, используя метод интерпретации,
предложенный в разд.~2, можно получить представление о количестве
последовательных и параллельных структур вероятностной модели сети.

\subsection{Режим <<скользящего окна>>} %5.3.

Результаты для выборки
``Berkeley'' в режиме <<скользящего окна>>  представлены
на рис.~\ref{f9bat}.


Данные графики показывают изменение параметров распределений подвыборок выборки 
``Berkeley''. Видно, что параметры распределений подвыборок не остаются 
неизменными во времени, наоборот, они имеют внешне случайный характер. На 
рис.~\ref{f9bat},\,\textit{в} видна даже своеобразная пульсация первой 
компоненты.
%
На основании расчетов можно сделать вывод о том, что пиковость трафика
обусловливается как формой, так и интенсивностью сообщений.

\section{Заключение}

В данной работе исследована вероятностная модель  информационных потоков,
возникающих в сложных телекоммуникационных конвергентных сетях, построенная с
помощью асимптотического и энтропийного подходов. Эта модель предполагает, что
функционирование сложной телекоммуникационной сети можно представить в виде
суперпозиции довольно простых стохастических структур~--- последовательных и
параллельных, которые по\-рож\-да\-ют смеси гамма-распределений для случайной
величины времени обработки и передачи сообщений в сети. Предложена простая
интерпретация параметров данной модели.
\begin{figure*} %fig5
\vspace*{1pt}
\begin{center}
\mbox{%
\epsfxsize=130mm %145.109mm 
\epsfbox{bat-5.eps} }
\end{center}
\vspace*{-13pt} \Caption{Компоненты смеси начального приближения~(\textit{а}) и 
результата~(\textit{б}) для выборки~1 ``Berkeley'' \label{f5bat}}
%\end{figure*}
%\begin{figure*} %fig6
\vspace*{12pt}
\begin{center}
\mbox{%
\epsfxsize=130mm %148.256mm 
\epsfbox{bat-7.eps} }
\end{center}
\vspace*{-13pt} \Caption{График смеси распределений~(\textit{1}) и гистограмма 
для выборки~1 ``Berkeley''~(\textit{2}) \label{f6bat}}
\end{figure*}



\begin{figure*} %fig7
\vspace*{1pt}
\begin{center}
\mbox{%
\epsfxsize=130mm %144.283mm 
\epsfbox{bat-8.eps} }
\end{center}
\vspace*{-16pt} \Caption{Компоненты смеси начального приближения~(\textit{а}) и 
результата~(\textit{б}) для выборки~2 ``Synterra'' \label{f7bat}}
%\end{figure*}
%\begin{figure*} %fig8
\vspace*{12pt}
\begin{center}
\mbox{%
\epsfxsize=130mm %148.256mm 
\epsfbox{bat-10.eps} }
\end{center}
\vspace*{-11pt} \Caption{График смеси распределений~(\textit{1}) и гистограмма
для выборки~2 ``Synterra''~(\textit{2}) \label{f8bat}}
\end{figure*}

\begin{figure*} %fig9
\vspace*{1pt}
\begin{center}
\mbox{%
\epsfxsize=119.041mm
\epsfbox{bat-11.eps} }
\end{center}
\vspace*{-9pt} \Caption{Изменение  смешивающих параметров~(\textit{а}), 
параметров формы~(\textit{б}) и параметров масштаба~(\textit{в}) во времени для 
выборки~1 ``Berkeley'' \label{f9bat}}
\end{figure*}

Для решения вытекающей из модели задачи предложен итерационный алгоритм,
базирующийся на методе максимального правдоподобия~--- ЕМ-ал\-го\-ритм, для
которого получены формулы для конкретного вида смесей~--- экспоненциальных и
гамма-распределений.
%
Кроме того, разработан программный инструментарий для оценки параметров 
предложенной модели на выборках из реальных трафиковых данных. Проведены 
исследования, которые подтвердили предположения вероятностной модели. 


Получение информации о стохастической структуре
телекоммуникационных сетей и наличие программных инструментов для
выявления более или менее стабильных структур позволит понять причины
возникновения неожиданных больших нагрузок, предотвратить такие нагрузки,
а также поможет в будущем в проектировании надежных, оптимальных по
стоимости и уровню сервиса телекоммуникационных сетей нового поколения.

%\vspace*{-15pt} 
{\small\frenchspacing
{%\baselineskip=10.8pt
\addcontentsline{toc}{section}{Литература}
\begin{thebibliography}{9}
\bibitem{1bat}
Teletraffic Engeneering Handbook. International Telecommunication Union, 
Geneva, 2005 {\sf http://www.itu.int}. \vspace*{5pt} 
\bibitem{2bat}
\Au{Севастьянов~Б.\,А.} Курс теории вероятностей и математической статистики. 
М., 2004. \vspace*{5pt} 
\bibitem{3bat}
\Au{Айвазян~C.\,А., Бухштабер~В.\,М., Енюков~И.\,С, Мешалкин~Л.\,Д.} Прикладная 
статистика. Классификация и снижение размерности~// Финансы и статистика. М., 
1989. \vspace*{5pt} 
\bibitem{4bat}
\Au{Bilmes~J.\,A.} A gentle tutorial of the EM algorithm and its application to 
parameter estimation for Gaussian mixture and hidden Markov models. Berkeley, 
CA, USA: International Computer Science Institute,  1998. \vspace*{5pt} 
\bibitem{5bat}
\Au{Шлезингер~М.\,И.} О самопроизвольном различении образов~// Шлезингер~М.\,И. 
Читающие. автоматы. Киев: Наукова думка, 1965. С.~38--45. \vspace*{5pt} 
\bibitem{6bat}
\Au{Hsiao~I.-T., Rangarajan~A., Gindi~G.}. Joint-MAP 
reconstruction/segmentation for transmission tomography using mixture-models as 
priors. Yale University, 1998. \vspace*{5pt} 
\bibitem{7bat}
{\sf http://zedgraph.org}. \vspace*{4pt} 
\bibitem{8bat}
{\sf http://ita.ee.lbl.gov/html/contrib/LBL-PKT.html}. \vspace*{5pt} 
\bibitem{9bat}
{\sf http://www.synterra.ru}.
\end{thebibliography}

} } \label{end\stat}
\end{multicols}


%\addtocounter{razdel}{1}
%\def\razd{НЕРЕГУЛИРУЕМЫЙ ЭЛЕКТРОПРИВОД ДЛЯ ЭЛЕКТРОЭНЕРГЕТИКИ}

\setcounter{page}{2}

   { %\Large  
   { %\baselineskip=16.6pt
   
   \vspace*{-48pt}
   \begin{center}\LARGE
   \textit{Предисловие}
   \end{center}
   
   %\vspace*{2.5mm}
   
   \vspace*{25mm}
   
   \thispagestyle{empty}
   
   { %\small 

    
Вниманию читателей журнала <<Информатика и её применения>> предлагается 
очередной тематический выпуск <<Вероятностно-статистические методы и 
задачи информатики и информационных технологий>>. Предыдущие тематические 
выпуски журнала по данному направлению вышли в 2008~г.\ (т.~2, вып.~2), 
в 2009~г.\ (т.~3, вып.~3) и в 2010~г.\ (т.~4, вып.~2). 

Статьи, собранные в данном журнале, посвящены разработке новых вероятностно-статистических 
методов, ориентированных на применение к решению конкретных задач информатики и информационных 
технологий, а также~--- в ряде случаев~--- и других прикладных задач. Проблематика, охватываемая 
публикуемыми работами, развивается в рамках научного сотрудничества между Институтом проблем 
информатики Российской академии наук (ИПИ РАН) и Факультетом вычислительной математики и 
кибернетики Московского государственного университета им.\ М.\,В.~Ломоносова в ходе работ 
над совместными научными проектами (в том числе в рамках функционирования 
Научно-образовательного центра <<Вероятностно-статистические методы анализа рисков>>). 
Многие из авторов статей, включенных в данный номер журнала, являются активными участниками 
традиционного международного семинара по проблемам устойчивости стохастических моделей, 
руководимого В.\,М.~Золотаревым и В.\,Ю.~Королевым; регулярные сессии этого семинара 
проводятся под эгидой МГУ и ИПИ РАН (в 2011~г.\ указанный семинар проводится в октябре 
в Калининградской области РФ). 

Наряду с представителями ИПИ РАН и МГУ в число авторов данного выпуска журнала входят 
ученые из Научно-исследовательского института системных исследований РАН, Института 
проблем технологии микроэлектроники и особочистых материалов РАН, Института 
прикладных математических исследований Карельского НЦ РАН, Московского 
авиационного института, Вологодского государственного педагогического университета, 
НИИММ им.\ Н.\,Г.~Чеботарева, Казанского государственного университета, Дебреценского 
университета (Венгрия).

Несколько статей выпуска посвящено разработке и применению стохастических методов и 
информационных технологий для решения различных прикладных задач. В~работе В.\,Г.~Ушакова 
и О.\,В.~Шестакова рассмотрена задача определения вероятностных характеристик случайных 
функций по распределениям интегральных преобразований, возникающих в задачах эмиссионной 
томографии. В~статье Д.\,О.~Яковенко и М.\,А.~Целищева рассмотрены некоторые вопросы 
математической теории риска и предложен новый подход к диверсификации инвестиционных 
портфелей. Работа И.\,А.~Кудрявцевой и А.\,В.~Пантелеева посвящена построению и 
исследованию математической модели, описывающей динамику сильноионизованной плазмы. 
В~статье П.\,П.~Кольцова изучается качество работы ряда алгоритмов сегментации изображений. 
Статья А.\,Н.~Чупрунова и И.~Фазекаша посвящена вероятностному анализу числа без\-оши\-бочных 
блоков при помехоустойчивом кодировании; получены усиленные законы больших чисел для указанных 
величин.

В данном выпуске традиционно присутствует тематика, весьма активно разрабатываемая в течение 
многих лет специалистами ИПИ РАН и МГУ,~--- методы моделирования и управления для 
информационно-телекоммуникационных и вычислительных систем, в частности методы 
теории массового обслуживания. В~статье А.\,И.~Зейфмана с соавторами рассматриваются 
модели обслуживания, описываемые марковскими цепями с непрерывным временем в случае 
наличия катастроф. В~работе М.\,М.~Лери и И.\,А.~Чеплюковой рассматриваются случайные 
графы Интернет-типа, т.\,е.\ графы, степени вершин которых имеют степенные распределения; 
такие задачи находят применение при исследовании глобальных сетей передачи данных. 
Работа Р.\,В.~Разумчика посвящена исследованию систем массового обслуживания специального 
вида~--- с отрицательными заявками и хранением вытесненных заявок.

Ряд статей посвящен развитию перспективных теоретических 
вероятностно-статистических методов, которые находят широкое применение в различных 
задачах информатики и информационных технологий. В~работе В.\,Е.~Бенинга, А.\,К.~Горшенина 
и В.\,Ю.~Королева рассмотрена задача статистической проверки гипотез о числе компонент 
смеси вероятностных распределений, приводится конструкция асимптотически наиболее мощного 
критерия. Результаты этой работы найдут применение в ряде прикладных задач, использующих 
математическую модель смеси вероятностных распределений (в информатике, моделировании 
финансовых рынков, физике турбулентной плазмы и~т.\,д.). В~статье В.\,Ю.~Королева, 
И.\,Г.~Шевцовой и С.\,Я.~Шоргина строится новая, улучшенная оценка точности нормальной 
аппроксимации для пуассоновских случайных сумм; как известно, указанные случайные суммы 
широко используются в качестве моделей многих реальных объектов, в том числе в информатике, 
физике и других прикладных областях. Работа В.\,Г.~Ушакова и Н.\,Г.~Ушакова посвящена 
исследованию ядерной оценки плотности распределения; эти результаты могут применяться, 
в част\-ности, при анализе трафика в телекоммуникационных системах. Серьезные приложения 
в статистике могут получить результаты работы О.\,В.~Шестакова, в которой доказаны оценки 
скорости сходимости распределения выборочного абсолютного медианного отклонения к нормальному 
закону. 

\smallskip

Редакционная коллегия журнала выражает надежду, что данный тематический  выпуск 
будет интересен специалистам в области теории вероятностей и математической статистики 
и их применения к решению задач информатики и информационных технологий.
     
     %\vfill 
     \vspace*{20mm}
     \noindent
     Заместитель главного редактора журнала <<Информатика и её 
применения>>,\\
     директор ИПИ РАН, академик  \hfill
     \textit{И.\,А.~Соколов}\\
     
     \noindent
     Редактор-составитель тематического выпуска,\\
     профессор кафедры математической статистики факультета\\
      вычислительной математики и кибернетики МГУ им.\ М.\,В.~Ломоносова,\\
     ведущий научный сотрудник ИПИ РАН,\\ 
доктор физико-математических наук \hfill
      \textit{В.\,Ю.~Королев}
     
     } }
     }



%   { %\Large  
   { %\baselineskip=16.6pt
   
   \vspace*{-48pt}
   \begin{center}\LARGE
   \textit{Предисловие}
   \end{center}
   
   %\vspace*{2.5mm}
   
   \vspace*{25mm}
   
   \thispagestyle{empty}
   
   { %\small 

    
Вниманию читателей журнала <<Информатика и её применения>> предлагается 
очередной тематический выпуск <<Вероятностно-статистические методы и 
задачи информатики и информационных технологий>>. Предыдущие тематические 
выпуски журнала по данному направлению вышли в 2008~г.\ (т.~2, вып.~2), 
в 2009~г.\ (т.~3, вып.~3) и в 2010~г.\ (т.~4, вып.~2). 

Статьи, собранные в данном журнале, посвящены разработке новых вероятностно-статистических 
методов, ориентированных на применение к решению конкретных задач информатики и информационных 
технологий, а также~--- в ряде случаев~--- и других прикладных задач. Проблематика, охватываемая 
публикуемыми работами, развивается в рамках научного сотрудничества между Институтом проблем 
информатики Российской академии наук (ИПИ РАН) и Факультетом вычислительной математики и 
кибернетики Московского государственного университета им.\ М.\,В.~Ломоносова в ходе работ 
над совместными научными проектами (в том числе в рамках функционирования 
Научно-образовательного центра <<Вероятностно-статистические методы анализа рисков>>). 
Многие из авторов статей, включенных в данный номер журнала, являются активными участниками 
традиционного международного семинара по проблемам устойчивости стохастических моделей, 
руководимого В.\,М.~Золотаревым и В.\,Ю.~Королевым; регулярные сессии этого семинара 
проводятся под эгидой МГУ и ИПИ РАН (в 2011~г.\ указанный семинар проводится в октябре 
в Калининградской области РФ). 

Наряду с представителями ИПИ РАН и МГУ в число авторов данного выпуска журнала входят 
ученые из Научно-исследовательского института системных исследований РАН, Института 
проблем технологии микроэлектроники и особочистых материалов РАН, Института 
прикладных математических исследований Карельского НЦ РАН, Московского 
авиационного института, Вологодского государственного педагогического университета, 
НИИММ им.\ Н.\,Г.~Чеботарева, Казанского государственного университета, Дебреценского 
университета (Венгрия).

Несколько статей выпуска посвящено разработке и применению стохастических методов и 
информационных технологий для решения различных прикладных задач. В~работе В.\,Г.~Ушакова 
и О.\,В.~Шестакова рассмотрена задача определения вероятностных характеристик случайных 
функций по распределениям интегральных преобразований, возникающих в задачах эмиссионной 
томографии. В~статье Д.\,О.~Яковенко и М.\,А.~Целищева рассмотрены некоторые вопросы 
математической теории риска и предложен новый подход к диверсификации инвестиционных 
портфелей. Работа И.\,А.~Кудрявцевой и А.\,В.~Пантелеева посвящена построению и 
исследованию математической модели, описывающей динамику сильноионизованной плазмы. 
В~статье П.\,П.~Кольцова изучается качество работы ряда алгоритмов сегментации изображений. 
Статья А.\,Н.~Чупрунова и И.~Фазекаша посвящена вероятностному анализу числа без\-оши\-бочных 
блоков при помехоустойчивом кодировании; получены усиленные законы больших чисел для указанных 
величин.

В данном выпуске традиционно присутствует тематика, весьма активно разрабатываемая в течение 
многих лет специалистами ИПИ РАН и МГУ,~--- методы моделирования и управления для 
информационно-телекоммуникационных и вычислительных систем, в частности методы 
теории массового обслуживания. В~статье А.\,И.~Зейфмана с соавторами рассматриваются 
модели обслуживания, описываемые марковскими цепями с непрерывным временем в случае 
наличия катастроф. В~работе М.\,М.~Лери и И.\,А.~Чеплюковой рассматриваются случайные 
графы Интернет-типа, т.\,е.\ графы, степени вершин которых имеют степенные распределения; 
такие задачи находят применение при исследовании глобальных сетей передачи данных. 
Работа Р.\,В.~Разумчика посвящена исследованию систем массового обслуживания специального 
вида~--- с отрицательными заявками и хранением вытесненных заявок.

Ряд статей посвящен развитию перспективных теоретических 
вероятностно-статистических методов, которые находят широкое применение в различных 
задачах информатики и информационных технологий. В~работе В.\,Е.~Бенинга, А.\,К.~Горшенина 
и В.\,Ю.~Королева рассмотрена задача статистической проверки гипотез о числе компонент 
смеси вероятностных распределений, приводится конструкция асимптотически наиболее мощного 
критерия. Результаты этой работы найдут применение в ряде прикладных задач, использующих 
математическую модель смеси вероятностных распределений (в информатике, моделировании 
финансовых рынков, физике турбулентной плазмы и~т.\,д.). В~статье В.\,Ю.~Королева, 
И.\,Г.~Шевцовой и С.\,Я.~Шоргина строится новая, улучшенная оценка точности нормальной 
аппроксимации для пуассоновских случайных сумм; как известно, указанные случайные суммы 
широко используются в качестве моделей многих реальных объектов, в том числе в информатике, 
физике и других прикладных областях. Работа В.\,Г.~Ушакова и Н.\,Г.~Ушакова посвящена 
исследованию ядерной оценки плотности распределения; эти результаты могут применяться, 
в част\-ности, при анализе трафика в телекоммуникационных системах. Серьезные приложения 
в статистике могут получить результаты работы О.\,В.~Шестакова, в которой доказаны оценки 
скорости сходимости распределения выборочного абсолютного медианного отклонения к нормальному 
закону. 

\smallskip

Редакционная коллегия журнала выражает надежду, что данный тематический  выпуск 
будет интересен специалистам в области теории вероятностей и математической статистики 
и их применения к решению задач информатики и информационных технологий.
     
     %\vfill 
     \vspace*{20mm}
     \noindent
     Заместитель главного редактора журнала <<Информатика и её 
применения>>,\\
     директор ИПИ РАН, академик  \hfill
     \textit{И.\,А.~Соколов}\\
     
     \noindent
     Редактор-составитель тематического выпуска,\\
     профессор кафедры математической статистики факультета\\
      вычислительной математики и кибернетики МГУ им.\ М.\,В.~Ломоносова,\\
     ведущий научный сотрудник ИПИ РАН,\\ 
доктор физико-математических наук \hfill
      \textit{В.\,Ю.~Королев}
     
     } }
     }


\def\stat{konovalov}

\def\tit{ОБ АДАПТИВНЫХ СТРАТЕГИЯХ И~УСЛОВИЯХ~ИХ~СУЩЕСТВОВАНИЯ$^*$}

\def\titkol{Об адаптивных стратегиях и~условиях их 
существования}

\def\autkol{М.\,Г.~Коновалов}

\def\aut{М.\,Г.~Коновалов$^1$}

\titel{\tit}{\aut}{\autkol}{\titkol}

{\renewcommand{\thefootnote}{\fnsymbol{footnote}}\footnotetext[1]
{Работа выполнена при поддержке РФФИ, грант № 11-07-00112.}}

\renewcommand{\thefootnote}{\arabic{footnote}}
\footnotetext[1]{Институт проблем информатики Российской академии наук, mkonovalov@ipiran.ru}



\Abst{Рассматривается задача оптимального управления в отсутствие априорной 
информации об управляемом объекте. Решением задачи является построение адаптивных 
стратегий на основе наблюдений, доступных в процессе управления. Изучаются 
некоторые условия адаптивной управляемости объекта. В~качестве математической 
модели используются управляемые случайные последовательности.}

\KW{управляемые случайные последовательности; адаптивные стратегии; условия 
существования}

\vskip 14pt plus 9pt minus 6pt

      \thispagestyle{headings}

      \begin{multicols}{2}

            \label{st\stat}


\section{Введение}

  Тема статьи относится к области адаптивных методов обработки информации с целью 
принятия оптимальных решений. Потребность в адаптивном\linebreak
подходе возникает в задачах 
с большой информационной неопределенностью, что наиболее характерно для 
телекоммуникационных систем, автоматизированных производственных процессов, 
робототех\-ни\-ки и других сфер, неразрывно связанных с компьютерной обработкой 
информации. Понятие неопределенности многозначно и связано с отсутствием априорных 
сведений, недетерминированностью, а также с неполнотой наблюдений. 
К~перечисленным факторам в нарастающей степени добавляется <<избыточность>> 
информации, которая порождается чрезмерно прогрессирующими объемами 
передаваемой и хранимой информации и обусловлена экспоненциальным ростом 
пропускной способности телекоммуникационных сетей, а также емкостей носителей 
информации.
  
  Идея адаптации (приспособления, самоорганизации), заимствованная из 
биологического мира, начала активно эксплуатироваться в науке примерно с середины 
прошлого века. Кратко, она заключается в том, чтобы, целенаправленно взаимодействуя с 
окружающей средой, отбирать и использовать поступающую информацию, необходимую 
для принятия оптимальных решений с точки зрения поставленной цели.
  
  Данная статья посвящена теоретическим аспектам адаптации. В~качестве исходного 
пред\-став\-ле\-ния использована схема, которая опирается на пред\-став\-ление о паре 
  <<объект--субъект>>, взаимодействующей в дискретном времени путем 
попеременного обмена сигналами. При этом субъект воздействует на объект с помощью 
управлений, получая в ответ сигналы, называемые наблюдениями. Действия субъекта 
преследуют цель, выраженную в наличии определенных свойств у траектории 
наблюдений.
  
  Основная отличительная особенность заключается в предположении, что действия 
субъекта происходят при недостаточной информации об объекте. В~качестве 
математической модели объекта взята конструкция управляемой случайной 
последовательности. В~терминах этого аппарата легко очерчиваются четыре аспекта 
информационной неопределенности:
  \begin{enumerate}[(1)]
\item недетерминированность понимается как стохастичность;
\item недостаток информации об объекте трактуется как неполное знание вероятностного 
распределения, задающего процесс;
  \item неполнота наблюдений означает, что состояния процесса наблюдаются лишь 
частично;
  \item недостаток знаний выражается в неумении \mbox{найти} или рассчитать ту или иную 
характеристику, связанную со случайной последовательностью, даже при наличии 
априорной информации о распределении процесса и полной его наблюдаемости.
  \end{enumerate}
  
  Субъект ассоциируется с алгоритмом, согласно которому выбираются управления, 
регулирующие траекторию случайной последовательности. Такой алгоритм принято 
называть стратегией управ\-ле\-ния. Задача заключается в том, чтобы выбрать стратегию, 
достигающую цели в ситуации, когда информация субъекта об объекте ограничена. 
По-дру\-го\-му можно сказать, что речь идет о построении стратегии, достигающей цели (в 
данном случае~--- максимизации предельного среднего дохода) для любого процесса из 
некоторого заданного класса объектов. Такие стратегии называют адаптивными по 
отношению к заданному классу объектов~[1].
  
  В разд.~2 даются формальные определения объекта, цели и адаптивной стратегии 
управления.
  
  В разд.~3 анализируются условия существования адаптивной стратегии. В~качестве 
необходимых условий обсуждаются два требования, которые, как представляется, должны 
выполняться из интуитивных соображений.
  
  Первое из необходимых условий связано с принципиальной особенностью адаптивных 
стратегий, которые, прежде чем выйти на <<оптимальный режим>>, должны затратить 
некоторое время на <<обуче\-ние>>. (На самом деле в рассматриваемой постановке процесс 
обучения для адаптивных стратегий длится даже неограниченно долго.) Естественно 
предположить, что подобные стратегии могут реализоваться, только если в процессе 
обучения не будут совершены <<непоправимые ошибки>>. Это соображение 
раскрывается на примерах и получает формальное описание.
  
  Второе необходимое условие является менее очевидным. Оно связано с гипотезой о 
том, что адаптивная стратегия управления классом случайных последовательностей 
существует лишь тогда, когда для данного класса возможно построение так называемой 
адаптивной стратегии перебора. Это выражается в том, что существует и заранее известно 
некоторое счетное множество вариантов поведения, среди которого для данного класса 
обязательно найдется оптимальный или близкий к нему вариант. Данное соображение 
также иллюстрировано примерами и приведена теорема о критерии существования 
адаптивной стратегии для определенного класса объектов.
  
  Подход, использованный в статье, а также полученные результаты являются 
продолжением направления, представленного в работе~[2].
  
\section{Постановка задачи адаптивного управления}
  
  Пусть  время $t$ пробегает значения 0, 1, \ldots\ и пусть заданы измеримые 
пространства $(X,\mathbf{X})$, $(Y,\mathbf{Y})$, $(Z,\mathbf{Z})$ (соответственно 
пространства \textit{состояний}, \textit{управлений} и \textit{наблюдений}).
  
  Общая траектория процесса упорядочена в виде последовательности $x_0, y_1, 
z_1,x_1,\ldots$\linebreak $\ldots , x_{t-1},y_t,z_t,x_t,\ldots$ Предыстория процесса до момента~$t$ 
включительно обозначается как

\noindent
  \begin{gather*}
 \! x^t=x_0^t=(x_0,\ldots , x_{t-1});\ \ \ y^t=y_1^t=(y_1, \ldots , y_{t-1});\\
  z^t=z_1^t=(z_1,  \ldots , z_{t-1})\,.
  \end{gather*}
  
  Траектории процесса определяются последовательностями условных вероятностных 
распределений~$\mu$, $\nu$ и~$\sigma$.
  
  Последовательность $\mu\hm=(\mu_0,\mu_1,\ldots ,\mu_t, \ldots)$ задает механизм 
смены состояний. В~этой последовательности $\mu_0$~--- вероятностное распределение 
на $(X,\mathbf{X})$; $\mu_t=\mu_t(A\vert x^{t-1},y^t)$, $t\hm>0$~---  условная 
(переходная) вероятность, которая при любых наборах $(x^{t-1},y^t)$ является 
вероятностной мерой на $(X,\mathbf{X})$ и при любом $A\hm\in X$ является измеримой 
функцией относительно $x^{t-1},y^t$.
  
  Последовательность $\nu\hm=(\nu_1, \ldots , \nu_t, \ldots)$ задает механизм появления 
наблюдений. В~этой последовательности каждый элемент $\nu_t\hm=\nu_t(C\vert x^{t-1}, 
y^t)$, $t\hm>0$, представляет собой условное распределение, которое при любом условии 
является вероятностной мерой на $(Z,\mathbf{Z})$ и для любого $C\hm\in Z$ является 
измеримой функцией относительно переменных, стоящих в условии. Пара $o\hm= 
(\mu,\nu)$ называется объектом.
  
  Последовательность $\sigma\hm= (\sigma_1, \ldots , \sigma_t. \ldots)$ называется 
(допустимой) \textit{стратегией} и определяет выбор управлений. В~этой 
последовательности:
%\smallskip
   $\sigma_1\hm=\sigma_1(\cdot)$~--- вероятностная мера на $(Y,\mathbf{Y})$; 
      $\sigma_{t+1}\hm=\sigma_{t+1}(B\vert y^t,z^t)$, $t\hm>0$,~--- условная вероятность, 
которая при любых $y^t,z^t$ является вероятностной мерой на $(Y,\mathbf{Y})$ и при 
любом $B\hm\in Y$ является измеримой функцией относительно $y^t,z^t$. Элементы 
последовательности~$\sigma$ называются (допустимыми) \textit{правилами}.

%\smallskip
  
  Введем обозначение для прямых произведений множеств:
  $$
  \Omega_0=X\,;\enskip \Omega_t=X^{t+1}\times Y^t\times Z^t\,,\enskip t>0\,,
  $$
а также для наименьших $\sigma$-ал\-гебр, порожденных соответствующими 
$\sigma$-ал\-геб\-рами:
$$
\mathbf{F}_0=\mathbf{X}\,;\enskip \mathbf{F}_t=\mathbf{X}\otimes \mathbf{Y}\otimes 
\mathbf{Z}\otimes \mathbf{X}\otimes \cdots \otimes \mathbf{Y}\otimes \mathbf{Z}\otimes 
\mathbf{X}
$$
($\mathbf{X}$ повторяется $t+1$ раз, $\mathbf{Y}$ и $\mathbf{Z}$~--- $t$ раз, $t\hm>0$).
  
  Положим
  
  \vspace*{3pt}
  
  \noindent
  $$
  \Omega =\prod\limits_{t\geq 0}\Omega_t\,;\enskip 
\mathbf{F}=\mathop{\otimes}\limits_{t\geq0}\mathbf{F}_t\,.
  $$ 
  
  Согласно общей теории~\cite{3-kon} последовательности $o\hm=(\mu,\nu)$ и~$\sigma$ 
порождают на пространстве $(\Omega, \mathbf{F})$ вероятностную меру $\mathbf{P}\hm= 
\mathbf{P}_{o,\sigma}\hm=\mathbf{P}_{\mu,\nu,\sigma}$, которая согласована с 
элементами этих последовательностей следующим образом. Случайные 
последова\-тель\-ности

\vspace*{-3pt}

\noindent
  \begin{gather*}
  x_t=x_t(\omega)\,;\enskip  
  y_{t+1}=y_{t+1}(\omega)\,;\\
  z_{t+1}= z_{t+1}(\omega)\,,\enskip  \omega\in \Omega\,,\  t\geq 0\,,
  \end{gather*}
удовлетворяют соотношениям:

\pagebreak

\noindent
$$
\mathbf{P}(x_0(\omega)\in A_0)=\int\limits_{A_0} \mu_0(dx_0)\,;
$$

\vspace*{-12pt}

\noindent
\begin{multline*}
\mathbf{P}\left(x_0(\omega)\in A_0\,,\  y_1(\omega)\in B_1\,,\ 
z_1(\omega)\in C_1, \ldots \right.\\[1pt]
\left.{}\ldots\,,
y_t(\omega)\in B_t\,,\  z_t(\omega)\in C_t\,,\  x_t(\omega)\in A_t\right)={}\\[1pt]
{}=\int\limits_{A_0}\mu_0(dx_0)\int\limits_{B_1}\sigma_1(dy_1)\int\limits_{C_1}\nu_1(dz_1
\vert x_0, y_1)\cdots{}\\[1pt]
{}\cdots
\int\limits_{B_t}\sigma_t\left(dy_t\vert y^{t-1},z^{t-1}\right) 
\int\limits_{C_t} \nu_t\left( dz_t\vert x^{t-
1},y^t\right) \times{}\\[1pt]
{}\times
\int\limits_{A_t} \mu_t\left( dx_t\vert x^{t-1},y^t\right)
\end{multline*}
для любых $A_t\in X$, $B_{t+1}\hm\in Y$, $C_{t+1}\hm\in Z$, $t\hm\geq 0$.
  
  По определению стратегии, ее правила зависят от предыдущих управлений и 
наблюдений, но не от предыдущих состояний. Это соответствует предположению о том, 
что состояния объекта не наблюдаемы в ходе процесса управления. В~частных случаях 
объект $o\hm=(\mu,\nu)$ может, конечно, описывать полностью наблюдаемый процесс. 
Например, если все множества $X_t$ содержат один и тот же единственный элемент. 
Другой простой пример~--- когда наблюдения тождественны состояниям. Однако на 
самом деле, как показывает лемма~1, с формальной точки зрения рассмотрение объекта с 
<<ненаблюдаемой>> компонентой всегда можно заменить изучением полностью 
наблюдаемого процесса.
  
  \medskip
  
  \noindent
  \textbf{Лемма 1.} \textit{Для любого объекта $o\hm=(\mu,\nu)$ условная вероятность 
$\mathbf{P}\left(dz_t\vert y^t,z^{t-1}\right)$ не зависит от стратегии~$\sigma$ при любых 
$t\hm>0$.}
  
  \medskip
  
  \noindent
  Д\,о\,к\,а\,з\,а\,т\,е\,л\,ь\,с\,т\,в\,о\,.\ Согласно отмеченной выше согласованности 
условных распределений $\mu,\nu,o$ и порождаемой ими меры~\textbf{P} имеем 
соотношения:
  \begin{multline*}
  I_1=\mathbf{P}\left(
  y_1(\omega)\in B_1,\ z_1(\omega)\in C_1\right) ={}\\[1pt]
  {}=
  \mathbf{P}\left( x_0(\omega)\in X_0\,,\ y_1(\omega)\in B_1\,,\ z_1(\omega)\in 
C_1\right)={}\\[1pt]
  {}=\int\limits_{X_0} \int\limits_{B_1} \int\limits_{C_1} \mu_0\left(dx_0\right) 
\sigma_1\left(dy_1\right) \nu_1\left(dz_1\vert x_0,y_1\right)={}\\[1pt]
  {}= \int\limits_{B_1}\int\limits_{C_1}\sigma_1\left(dy_1\right) \int\limits_{X_0}\mu_0\left( 
dx_0\right) \nu_1\left( dz_1\vert x_0,y_1\right)\,,
  \end{multline*}
справедливые при любых $B_1\hm\in Y$ и $C_1\in Z$. Кроме того, по определению 
условной вероятности
$$
I_1=\int\limits_{B_1}\int\limits_{C_1}\sigma_1\left(dy_1\right) \mathbf{P}\left(dz_1\vert 
y_1\right)\,.
$$
  
  Сравнивая оба выражения для~$I_1$, получаем, что
  $$
  \mathbf{P}\left( dz_1\vert 
y_1\right)=\int\limits_{X_0}\mu_0\left(dx_0\right)\nu_1\left(dz_1\vert x_0, y_1\right)\,,
  $$
т.\,е.\ утверждение леммы справедливо для $t\hm=1$. Пусть оно верно для $n\hm=1, 2, 
\ldots , t\hm-1$. Для любых $B_1\hm\in Y$, $C_1\hm\in Z$, \ldots , $B_{t-1}\hm\in Y$, 
$C_t\hm\in Z$ имеем:

\noindent
\begin{multline*}
I_t=\mathbf{P}\left( y_1(\omega)\in B_1\,,\ z_1(\omega)\in C_1, \ldots{}\right.\\[1pt]
\left.{}\ldots , y_t(\omega)\in B_t\,,\ 
z_t(\omega) \in C_t\right)={}\\[1pt]
{}=
\mathbf{P}\left( x_0(\omega)\in X\,,\ y_1(\omega)\in B_1\,,\ z_1(\omega)\in C_1\,, \ldots\right.\\[1pt]
\left.{}\ldots , x_{t-
1}(\omega)\in X\,,\ y_t(\omega)\in B_t\,,\ z_t(\omega)\in C_t\right)={}\\[1pt]
{}=
\int\limits_X \int\limits_{B_1} \int\limits_{C_1}\ldots \\[1pt]
\ldots\int\limits_X \int\limits_{B_t} 
\int\limits_{C_t} \mu_0\left( dx_0\right) \sigma_1\left( dy_1\right) \nu_1\left( dz_1\vert 
x_0,y_1\right)\cdots{}\\[1pt]
\cdots \mu_{t-1}\left( dx_{t-1}\vert x^{t-2} y^{t-1}\right) \sigma_t \left( dy_t\vert y^{t-
1},z^{t-1}\right)\times{}\\[1pt]
{}\times \nu_t\left( dz_t\vert x^{t-1},y^t\right)={}\\[1pt]
{}=\int\limits_{B_1} \sigma_1\left( dy_1\right) \int\limits_{C_1} \int\limits_{B_2} 
\sigma_2\left( dy_2\vert z_1\right)\cdots\\[1pt]
\cdots \int\limits_{C_{t-1}}\int\limits_{B_t} \sigma_t \left( 
dy_t\vert y^{t-1},z^{t-1}\right)\times{}\\[1pt]
{}\times \int\limits_X \mu_0\left(dx_0\right) \nu_1\left( dz_1\vert 
x_o,y_1\right)\cdots{}\\[1pt]
{}\cdots \int\limits_{X_{t-1}}\mu_{t-1}\left( dx_{t-1}\vert x^{t-2} y^{t-1}\right) \nu_t \left( 
dz_t\vert x^{t-1}, y^t\right)={}\\[1pt]
{}=\int\limits_{B_1} \sigma_1\left( dy_1\right) \int\limits_{C_1} 
\int\limits_{B_2}\sigma_2\left( dy_2\vert z_1\right)\cdots\\[1pt]
\cdots \int\limits_{C_{t-1}} 
\int\limits_{B_t} \sigma_t\left( dy_t\vert y^{t-1},z^{t-1}\right) \int\limits_{C_t} 
\mathbf{P}\left( dz_1\vert y_1\right)\ldots{}\\[1pt]
{}\cdots \mathbf{P}\left( dz_{t-1}\vert y^{t-1},z^{t-2}\right) \mathbf{P}\left( dz_t\vert y^t, 
z^{t-1}\right)\,.
\end{multline*}
Отсюда получаем, что

\noindent
  \begin{multline*}
\hspace*{-6.95218pt}\mathbf{P}\left( dz_1\vert y_1\right)\cdots \mathbf{P}\left( dz_{t-1}\vert y^{t-1},z^{t-
2}\right) \mathbf{P}\left( dz_t\vert y^t,z^{t-1}\right)={}\\[1pt]
  {}=\int\limits_X \mu_0\left( dx_0\right) \nu_1\left( dz_1\vert x_o,y_1\right)\cdots \\[1pt]
  \cdots
\int\limits_X \mu_{t-1}\left( dx_{t-1}\vert x^{t-2}y^{t-1}\right) \nu_t\left( dz_t\vert x^{t-
1},y^t\right)\,.
  \end{multline*}
  
  Следовательно, по предположению индукции $\mathbf{P}\left( dz_t\vert y^t,z^{t-
1}\right)$ не зависит от~$\sigma$.
  
  Таким образом, не уменьшая общности, можно ограничиться (что и будет сделано в 
оставшейся части текста) рассмотрением полностью наблюда-\linebreak\vspace*{-12pt}

\pagebreak

\noindent
емых объектов $o\hm=\mu$, 
управляемых (допустимыми) стратегиями~$\sigma$ c правилами вида
  $$
  \sigma_1=\sigma_1\left(\cdot\right)\,;\enskip \sigma_{t+1}=\sigma_{t+1}\left( \cdot \vert 
y^t,x^t\right)\,,\enskip t>0\,.
  $$
(Множество всех таких стратегий при заданных пространствах состояний и управлений 
далее обозначается через~$\Sigma$.) В~этом случае вероятностная мера 
$\mathbf{P}\hm=\mathbf{P}_{\mu,\sigma}$ определена на пространстве $(\Omega, 
\mathbf{F})$, в котором $\Omega\hm=\prod\limits_{t\geq0} X^{t+1}\times Y^t$, 
$\mathbf{F}\mathop{\otimes}\limits_{t\geq0} \mathbf{F}_t$, где $\mathbf{F}_0\hm=\mathbf{X}$; 
$\mathbf{F}_t=\mathbf{X}\otimes \mathbf{Y}\otimes \mathbf{X}\otimes \cdots \otimes 
\mathbf{Y}\otimes \mathbf{X}$ и согласована с последовательностями~$\mu$ и~$\sigma$. 
Через $\mathbf{F}_t$ обозначена $\sigma$-ал\-геб\-ра, порожденная предысторией 
$(x^t,y^t)$ до момента~$t$ включительно.
  
  В то же время необходимо заметить, что предположение о наличии 
<<двухступенчатой>> структуры у объектов (со\-сто\-яние--наблю\-де\-ние) может 
принести пользу при их изучении. Так происходит, например, в теории частично 
наблюдаемых управляемых марковских процессов.
  
  Предположим далее, что на наблюдаемой части траектории процесса задан 
одношаговый доход (в момент~$t$), и будем считать, что этот доход имеет вид 
$g_t\hm=g(x_t)$, где $g:\ X\rightarrow (0,\,1)\subset \mathbb{R}$~--- измеримая числовая 
функция со значениями из интервала (0,\,1).
  
  Обозначим через $v_{t,s}\hm=s^{-1}\sum\limits_{n=1}^s g_{t+n}$ среднее 
арифметическое доходов на промежутке от $t+1$ до $t\hm+s$ ($t\hm\geq0$, $s\hm\geq 1$).
  
  Если объект~$\mu$ управляется согласно стратегии~$\sigma$, то число
  $$
  w_t(\mu,\sigma) =\sup \left\{ c:\ \mathbf{P}_{\mu,\sigma} \left( 
\lim\limits_{\overline{s\rightarrow\infty}} v_{t,s}>c\right) =1\right\}
  $$
характеризует получаемый при этом гарантированный предельный средний доход 
начиная с момента $t=1$. Поскольку $\lim\limits_{\overline{s\rightarrow\infty}} v_{t,s}$ не 
зависит от~$t$, то $w_0(\mu,\sigma)\hm=w_1(\mu,\sigma)\hm=w_2(\mu,\sigma)\hm=\cdots$. 
Величина $w(\mu,\sigma)\hm=w_0(\mu,\sigma)$ играет в дальнейшем роль целевой 
функции и называется просто \textit{доходом} (при управлении объектом~$\mu$ с 
помощью стратегии~$\sigma$).
  
  Из определения дохода следует, что для любого $t>0$ выполняется условие
  $$
  \mathbf{P}_{\mu,\sigma}\left( \lim\limits_{\overline{s\rightarrow\infty}} v_{t,s}\geq 
w(\mu,\sigma)\vert \mathbf{F}_{t-1}\right)=1
  $$
почти наверное.
  Столь общее определение дохода, без предположений об эргодичности, оказывается 
полезным в теоретических рассмотрениях, однако на практике все же среднее 
арифметическое ведет себя более или менее регулярным образом. Поэтому введем 
следующее определение.
{ %\looseness=1

}
  
  Стратегия~$\sigma$ называется \textit{эргодической} по отношению к классу~$M$, 
если для любого объекта $\mu\hm\in M$ и любого $\varepsilon\hm>0$ выполняется 
условие $\sum\limits_{s=1}^\infty a_s\hm<\infty$, где $a_s\hm= 
a_s(\mu,\sigma,\varepsilon)\hm=\sup\limits_{t\geq0} \mathbf{P}_{\mu,\sigma}\left( \left\vert 
v_{t,s}-w(\mu,\sigma)\right\vert >\varepsilon\vert \mathbf{F}_t\right)$. Обозначим еще
  $$
  W=W(\mu) =\sup\limits_\sigma w(\mu,\sigma)\,,
  $$
где точная верхняя грань берется по всем допустимым стратегиям. Стратегия~$\sigma$ 
называется $\varepsilon$-\textit{оп\-ти\-маль\-ной}, если выполняется неравенство
$$
w(\mu,\sigma)\geq W-\varepsilon\,,\enskip \varepsilon\geq 0\,.
$$
  
  Далее объекты будут объединяться в множества объектов (классы объектов). При этом 
без дополнительных оговорок всюду предполагается, что
  \begin{itemize}
  \item все объекты из класса имеют одинаковые пространства состояний, управлений (и 
наблюдений);
  \item в качестве множества допустимых стратегий берется определенное выше 
множество~$\Sigma$;
  \item функция одношаговых доходов~$g$ одна и та же для всех объектов.
  \end{itemize}
  
  Пусть $M$~--- класс объектов. Стратегия~$\sigma$ является равномерно 
  $\varepsilon$-оп\-ти\-маль\-ной относительно этого класса, если последнее неравенство 
выполняется для всех $\mu\hm\in M$. Такую стратегию будем называть также 
  $\varepsilon$-\textit{адап\-тив\-ной} по отношению к классу~$M$. Класс объектов, для 
которого существует $\varepsilon$-адап\-тив\-ная стратегия, называется 
  $\varepsilon$-\textit{адап\-тив\-но управ\-ля\-емым}. (Если $\varepsilon\hm=0$, то 
приставка <<$\varepsilon$->> в этих определениях опускается.)
  
  Основная задача адаптивного управления заключается в построении адаптивных 
стратегий для различных классов объектов. 

К~настоящему вре\-ме\-ни получено много 
решений для многочисленных вариантов этой задачи. Подобные результаты являются 
фактически достаточными условиями адаптивной управ\-ля\-емости. Ниже, однако, будет 
уделено внимание также необходимым условиям существования адаптивных стратегий. 
Подчеркнем, что рассматриваемая постановка задачи предполагает, по сути, наличие 
лишь минимальной априорной информации об объекте управления~--- необходимо знать 
множество управлений~$Y$.

\section{Некоторые условия адаптивной управляемости}

  Пусть $\mu\in M$~--- фиксированный объект, а $\sigma\hm\in \Sigma$~--- 
фиксированная стратегия из некоторой среды. Набор, состоящий из первых $t$ правил 
стратегии~$\sigma$, будем обозначать через $\sigma^t\hm=(\sigma_1, \ldots , \sigma_t)$. 
Таким образом, $\sigma\hm=(\sigma^t, \sigma_{t+1},\sigma_{t+2}, \ldots)$. Положим
  $$
  w_t^*(\mu,\sigma) =w_t^*(\mu,\sigma^t)=\sup\limits_{\sigma_{t+1},\sigma_{t+2}, \ldots} 
w_t(\mu,\sigma)\,,
  $$
где верхняя грань берется по всем допустимым правилам начиная с момента $t\hm+1$. В 
этих обозначениях $w_0^*(\mu,\sigma) \hm=W(\mu)$. Ясно, что $W(\mu)\hm\geq 
w_1^*(\mu,\sigma)\hm\geq w_2^*(\mu,\sigma)\geq \cdots$
  
  Стратегию~$\sigma$ назовем $\varepsilon$-\textit{по\-вреж\-да\-ющей} для 
объекта~$\mu$, если
  $$
  \inf\left\{ t:\ w_t^*(\mu,\sigma)<W(\mu)-\varepsilon\right\} <\infty\,,\enskip \varepsilon>0\,.
  $$
  
  Пример~1 показывает, что существуют объекты, для которых каждая стратегия~--- 
$\varepsilon$-по\-вреж\-да\-ющая (с разными значениями~$\varepsilon$).
  
  \medskip
  
  \noindent
  \textbf{Пример~1.} Множество~$X$ состояний объекта~$\mu$ образовано точками с 
неотрицательными целочисленными координатами на плоскости, $X\hm=\{ (i,j), 
i\hm\geq0,\ j\hm\geq0\}$. Множество управлений $Y\hm=\{1;2\}$. Начальное состояние 
$x_0=(0,\,0)$. Детерминированные переходы между состояниями заданы следующим 
образом ($t\hm>0$, $i\hm\geq0$):
  \begin{align*}
  \mu_t\left( x_t=(i+1{,}0)\vert x_{t-1}=(i,0),y_t=1\right)&=1\,;\\
  \mu_t\left( x_t=(i,j+1)\vert x_{t-1}=(i,j),y_t=1\right)&=1\,,\ j>0\,;\\
  \mu_t\left( x_t=(i,j+1)\vert x_{t-1}=(i,j),y_t=2\right)&=1\,, j\geq 0\,.
  \end{align*}
  
  Одношаговые доходы определены как $g(i,0)\hm=0$, $g(i,j)\hm=1-2^{-i}$ для $i\geq 0$, 
$j\hm>0$.
  
  Стратегия, состоящая из бесконечного повторения управления~1, приносит доход~0. 
Стратегия, в которой управление~2 первый раз применяется (детерминировано) в 
момент~$t$, приносит доход $1\hm-2^{t-1}$, что меньше максимально возможного 
на~$2^{t-1}$. Рандомизация правил и их зависимость от предыстории не вносит 
принципиальных изменений~--- каждая стратегия остается 
  $\varepsilon$-по\-вреж\-да\-ющей относительно предельно наибольшего, но 
недостижимого значения~1.
  
  В примере~2 оптимальная стратегия для любого объекта из класса является 
повреждающей для остальных объектов.
  
  \medskip
  
  \noindent
  \textbf{Пример~2.} Пусть $X\hm= \{0, 1, 2, \ldots\}\cup \{a,b\}$; $Y\hm=\{0;\,1\}$; 
$g(a)\hm=1$; $g(b)\hm=g(i)\hm=0$, $i\hm\geq0$. Зададим счетное множество объектов 
$M\hm=\{\mu^{(k)},\ k\hm=0, 1, \ldots\}$. Пусть для всех~$k$:
  \begin{align*}
  \mu^{(k)}(x_0=0)&=1\,;\\
  \mu^{(k)}(x_{t+1}=i+1\vert x_t=i, y_t=0)&=1\,,\enskip i\geq0\,;\\
     \mu^{(k)}(x_{t+1}=a\vert x_t=k,y_t=1)&=1\,;\\
     \mu^{(k)}(x_{t+1}=b\vert x_t=i,y_t=1) &=1\,,\enskip i\not=k\,;\\
     \mu^{(k)}(x_{t+1}=a\vert x_t=a,y_t=j)&={}\\
&\hspace*{-45mm}{}=\mu^{(k)}(x_{t+1}=b\vert 
x_t=b,y_t=j)=1\,,\enskip j=0\vee 1\,.
     \end{align*}
  
  Таким образом, состояния $a$ и $b$~--- погло\-ща\-ющие, причем в состояние~$a$, 
приносящее максимальный доход, объект~$\mu^{(k)}$ может попасть, только если 
применить управление~1, находясь в со\-сто\-янии~$k$. Первые (существенные) правила 
оптимальной стратегии для объекта~$\mu^{(k)}$ требуют применения управления~0 до 
достижения состояния~$k$, а затем применения в этом состоянии управления~1. Однако 
такая стратегия является повреждающей для всех остальных объектов. Следовательно, для 
класса~$M$ не существует равномерно оптимальной стра\-тегии.
{\looseness=1

}
  
  Пусть $M$~--- класс объектов. Обозначим через $\Sigma_\varepsilon(\mu)$ множество 
$\varepsilon$-по\-вреж\-да\-ющих стратегий для объекта~$\mu$, $\mu\hm\in M$. Положим 
$\Sigma_\varepsilon(M)\bigcap\limits_{\mu\in M}\left( \Sigma\backslash 
\Sigma_\varepsilon(\mu)\right)$.
  
  \medskip
  
  \noindent
  \textbf{Лемма~2.} \textit{Для того чтобы существовала $\varepsilon$-адап\-тив\-ная 
стратегия, необходимо, чтобы $\Sigma_\varepsilon(M)\not=\emptyset$.}
  \medskip
  
  \noindent
  Д\,о\,к\,а\,з\,а\,т\,е\,л\,ь\,с\,т\,в\,о\,.\ Если $\Sigma_\varepsilon\not= \emptyset$, то любая 
допустимая стратегия хотя бы для одного из объектов является 
  $\varepsilon$-по\-вреж\-да\-ющей и, следовательно, не является 
  $\varepsilon$-оп\-ти\-маль\-ной, а потому не может быть равномерно 
  $\varepsilon$-оп\-ти\-маль\-ной по отношению к классу~$M$.
  
  В примере~3, несмотря на наличие по\-вреж\-да\-ющих стратегий, адаптивная стратегия 
существует.
  
  \medskip
  
  \noindent
  \textbf{Пример~3.} Пусть $X\hm=Y\hm=\{1, \ldots , K\}$ и пусть задана 
детерминированная функция~$f:\ X\hm\rightarrow X$, которая представляет собой 
циклическую подстановку на множестве~$X$,  т.\,е.\ $f(i)\not= f(j)$, если $i\not= j$; 
$i,j\hm=1, \ldots , K$. Рассмотрим следующий неоднородный во времени 
детерминированный объект. Положим
  \begin{align*}
  \mu_0(x_0=1)&=1\,;\\
  \mu_t(x_t=f(k)\vert x^{t-1},y^t) &= I_{\{y_t=k\}}\,,\ 0<k\,,\ t\leq K\,;\\
  \mu_t(x_t=f(k)\vert x^{t-1},y^t) &=I_{\{y_{K+1}=k}\,,\\
  & \hspace*{10mm}0<k\leq K\,,\enskip t>K
  \end{align*}
($I_A$~--- индикатор события~$A$).
  
  Одношаговые доходы определим как $g(i)\hm=i$, $i\hm\in X$.
  
  Так определенный объект обозначим через~$\mu^f$. Ясно, что для этого объекта 
траектория управ\-ля\-емо\-го процесса, начиная с момента $K+1$, и, следовательно, доход 
зависят исключительно от управ\-ле\-ния, примененного в момент $K+1$. Доход будет 
максимален (и равен~$K$) тогда и только тогда, когда $y_{K+1}\hm= k^\prime \hm= 
k^*(f)\hm=\argmax\limits_{1\leq k\leq K} f(k)$.
  
  Пусть $M=\{\mu^f\}$~--- совокупность всех объектов данного вида (которая содержит 
$K!$ элементов). Очевидно, для класса~$M$ существует равномерно оптимальная 
стратегия, доставляющая доход, равный~$K$. Например, достаточно вначале в моменты 
$t\hm=1, \ldots , K$ по одному разу применить каждое из управлений, а затем в момент 
$K+1$ применить управление~$k^*$, которое будет выявлено путем наблюдения за 
полученными одношаговыми доходами. Таким образом, на первых тактах необходимо совершить 
<<обучение>>~--- выявить управление, приносящее наибольший одношаговый доход. 
В~то же время существуют и повреждающие стратегии. Например, стратегия, в которой 
первые $K$ правил заключаются в применении управления~1. Правило~$\sigma_{K+1}$ 
такой стратегии может быть построено только в виде зависимости от управления~1 и от 
значения $f(1)$, поэтому при любом его определении найдется объект~$\mu^f$, для 
которого в момент $K+1$ будет с положительной вероятностью предписано применение 
неоптимального управления, и, следовательно, доход будет меньше~$K$.
  
  В примере 3 <<обучение>> оказалось возможным только благодаря знанию структуры 
процессов. Если бы заранее не было известно, что необходимо на первых тактах по разу 
<<испробовать>> все управ\-ле\-ния, то легко можно было пропустить период, когда 
возможно обучение, и совершить тем самым <<непоправимую ошибку>>. Следовательно, 
для того чтобы конструктивно построить равномерно оптимальную стратегию, 
необходима дополнительная информация. Это противоречит избранному принципу 
постановки задачи~--- минимальности априорной информации об объекте. 

Введем более 
жесткое определение адаптивной стратегии, которое, в част\-ности, устраняет указанное 
несоответствие.
  
  Пусть $M$~--- некоторый класс объектов. Эргодическая стратегия~$\sigma$ (ее 
определение дано в конце разд.~2) называется \textit{устойчивой} по отношению к 
классу~$M$, если для любого объекта $\mu\hm\in M$ стратегия~$\tilde{\sigma}$, 
полученная из стратегии~$\sigma$ путем произвольной (допустимой) замены конечного 
числа правил, (1)~имеет одинаковый со стратегией доход 
$w(\mu,\sigma)\hm=w(\mu,\tilde{\sigma})$ и (2)~является эргодической по отношению к 
классу~$M$.

%\columnbreak
  
  Адаптивная стратегия для класса~$M$ называется \textit{строго адаптивной}, если она 
устойчивая по отношению к этому классу.
  
  \medskip
  
  \noindent
  \textbf{Пример~4.} Легко показать, что строго адаптивными являются 
многочисленные адаптивные стратегии для класса управляемых конечных связных 
марковских цепей~[1, 2].
  
  Рассмотрим еще один мотив, выдвигаемый в качестве необходимого условия 
адаптивной управ\-ля\-емости.
  
  \medskip
  
  \noindent
  \textbf{Пример~5.} Пусть класс объектов состоит из функций вещественного 
аргумента~$u$ вида $\mu^y\hm=\mu^y(u)\hm=I_{\{u=y\}}$, $y\hm\in [0,\,1]$. (В~терминах 
управляемых случайных последовательностей: $X\hm= \{0;1]\}$, $Y\hm=[0,1]$; 
$\mu_t(x_t\vert x^{t-1},y^t)\hm=x_t I_{\{y_t=y\}}+ (1-x_t)I_{\{y_t=y\}}$; $g(x)\hm=x$, 
$x\hm\in X$.) Интуитивно представляется очевидным, что невозможно найти максимум 
такой функции за счетное число шагов, если не знать значение, в котором она обращается 
в единицу. В~то же время формально для каждого объекта~$\mu^y$ существует 
оптимальная стратегия. Например, можно постоянно повторять управление~$y$. Однако 
не существует стратегии, равномерно оптимальной по отношению к классу 
$M\hm=\{\mu^y\}$. В~такой стратегии для каждого $y\hm\in [0,\,1]$ необходимо должно 
было бы выполняться следующее условие: $\sigma_t(y_t=y\vert \cdot)>0$ хотя бы для 
одного значения~$t$. Но это невозможно, поскольку для фиксированного значения~$t$ 
данное неравенство может быть выполнено лишь для счетного множества значений~$y$, а 
$t$ также пробегает счетное множество значений. Счетное объединение счетных 
множеств само счетно, поэтому необходимое неравенство не может быть выполнено для 
всех точек на отрезке [0,\,1].
  
  Аналогичные рассуждения показывают, что в данном примере не существует счетного 
множества стратегий, обладающего тем свойством, что для любого объекта найдется 
$\varepsilon$-оп\-ти\-маль\-ная стратегия из этого множества.
  
  Конечное или счетное множество стратегий $\Sigma\hm=\{\sigma(1),\sigma(2), \ldots \}$ 
назовем \textit{базовым} по отношению к классу объектов $M\hm\in \mathcal{M}$, если:
  \begin{enumerate}[(1)]
  \item для любого объекта из $M$ и любого $\varepsilon\hm>0$ существует оптимальная 
стратегия из множества~$\Sigma$;
  \item любая стратегия $\sigma(i)$ является устойчивой по отношению к классу~$M$.
  \end{enumerate}
  
  \smallskip
  
  \noindent
  \textbf{Теорема.} \textit{Строго адаптивная стратегия для класса объектов~$M$ 
существует тогда и только тогда, когда для этого класса существует базовое 
множество стратегий~$\Sigma$.}


%\hfill {\large Приложение~1}

\bigskip

%\pagebreak

\noindent
Д\,о\,к\,а\,з\,а\,т\,е\,л\,ь\,с\,т\,в\,о\ \ теоремы.

Необходимость условий в данном случае является тривиальной, поскольку строго 
адаптивная стратегия, если она существует, образует базовое множество 
стратегий~$\Sigma$, состоящее из одного элемента.
  
  Докажем достаточность. Определим с по\-мощью стратегий из~$\Sigma$ новую 
стратегию $a$ следующим образом. Обозначим
  $$
  \theta_{t,n}=\mathrm{Int}\left(\left( 1-v_{t,n}\right)^{-n}\right)\,,
  $$
где $\mathrm{Int}\left(a\right)$ означает целую часть числа~$a$, и зададим 
последовательность марковских моментов $\tau\hm=\{\tau_n\}$ с помощью рекуррентных 
соотношений

\pagebreak

\noindent
$$
\tau_0=0\,,\enskip \tau_n=\tau_{n-1}+n+\theta_n\,,
$$
где $\theta_n\hm=\theta_{\tau_{n-1},n}$. Соответствующие $\sigma$-ал\-геб\-ры обозначим 
$\mathbf{F}_{(n)}\hm=\mathbf{F}_{\tau_{n-1}}$.
  
  Будем считать, что на пространстве $(\Omega,\mathbf{F})$ задана последовательность 
случайных величин $\beta\hm=\{\beta_n\}$, независимых 
относительно~$\mathbf{F}_{(n)}$. Каждая случайная величина имеет одно и то же 
невырожденное распределение $\{b_i\}$ на множестве номеров стратегий из~$\Sigma$.
  
  Определим правила стратегии $a\hm=a(\Sigma,\beta)$ формулой
  $$
  a_t=\sum\limits_{n=1}^\infty \sigma_t(\beta_n) I_{\{\tau_{n-1}<t\leq \tau_n\}}\,,
  $$
где $\sigma_t(\beta_n)$~--- правило стратегии $\sigma(i)\hm\in\Sigma$ в момент~$t$, если 
$\beta_n\hm=i$.
  
  Наглядно работа стратегии~$a$ выглядит следующим образом. Процесс управления 
разбивается на этапы. Этап с номером $n$ начинается в момент $\tau_{n-1}+1$ и 
оканчивается в момент~$\tau_n;\tau_0\hm=0$. В~момент, предшествующий началу 
очередного этапа, определяется номер стратегии в множестве~$\Sigma$, из которой будут 
взяты правила для применения на данном этапе. Этот номер равен значению случайной 
величины~$\beta_n$. Продолжительность $n$-го этапа равна $n\hm+\theta_n$ и зависит, 
следовательно, от номера этапа и от оценки качества применяемой стратегии, полученной 
в течение первых $n$ тактов этапа. Стратегия~$a$ называется стратегией перебора~[2]. 
Таким образом, последовательность~$\beta$ определяет на каждом этапе выбор стратегии 
из множества~$\Sigma$, правила из которой применяются на этом этапе.
  
  Пусть задан объект $\mu\hm\in M$ и пусть $W\hm=W(\mu)$~--- точная верхняя грань 
доходов для этого объекта, взятая по всем допустимым стратегиям, и пусть %также
  \begin{alignat*}{2}
  W_i&=w(\mu,\sigma(i))\,; &\enskip v_n^{(1)}&=v_{\tau_{n-1},n}\,;\\
  v_n^{(2)}&=v_{\tau_{n-1},n+\theta_n}\,; &\enskip \Delta_n&=\tau_n-\tau_{n-1}=n+\theta_n\,.
  \end{alignat*}
  
  Для произвольного $\varepsilon>0$ определим множества
  $$
  A_n^{(k)}(\varepsilon)=\left\{ v_n^{(k)}\geq W-\varepsilon\right\}\,,
  $$
обозначая их дополнения $\overline{A_n^{(k)}(\varepsilon)}$, $k=1, 2$.
  
  Обозначим
  \begin{align*}
  s_n^{(1)} &= \sum\limits_{l=1}^n I_{A_l^{(1)}(\varepsilon)\cap 
{A_l^{(2)}(2\varepsilon)}} \Delta_l\,;\\
  s_n^{(2)} &= \sum\limits_{l=1}^n I_{A_l^{(1)}\cap 
\overline{A_l^{(2)}(2\varepsilon)}}\Delta_l\,;\\
  s_n^{(3)} &= \sum\limits_{l=1}^n I_{\overline{A_l^{(1)}(\varepsilon)}}\Delta_l\,,
  \end{align*}
так что $\tau_n\hm=\sum\limits_{l=1}^n \Delta_l\hm= s_n^{(1)}\hm+ s_n^{(2)}\hm+ 
s_n^{(3)}$.

\columnbreak

  
  С~помощью введенных обозначений запишем оценку для усредненного дохода к 
моменту~$\tau_n$:
  \begin{multline}
  w_n=\fr{1}{\tau_n}\sum\limits_{t=1}^{\tau_n} g_t=\fr{\sum\limits_{l=1}^n 
v_l^{(2)}\Delta_l} {\sum\limits_{l=1}^n \Delta_l}\geq{}\\
{}\geq (W-2\varepsilon) \fr{s_n^{(1)}} 
{s_n^{(1)}+s_n^{(2)}+s_n^{(3)}}\,.
  \label{e1-kon}
  \end{multline}
  
  Для оценки суммы $s_n^{(1)}$ запишем неравенство
  $$
  s_n^{(1)}\geq \Delta_{v_n}\,,
  $$
в котором обозначено
$$
v_n=\max\left\{ l:\ l\leq n,\ A_l^{(1)}(\varepsilon)\cap A_l^{(2)}(2\varepsilon)\right\}\,.
$$
  
  Оценим вероятность события $B_n\hm=\{v_n\hm\leq n-\ln n\}$, для которого выполняется 
включение
  $$
  B_n\subset \bigcap\limits_{n-\ln n<l\leq n} 
  \overline{A_l^{(1)}(\varepsilon)}\cap \overline{A_l^{(2)}(2\varepsilon)}\,.
  $$
  
  Согласно определениям эргодической стратегии, базового множества стратегий и 
семейства случайных величин~$\beta$ имеем:
  \begin{multline*}
  \mathbf{P}_{a} \left( \overline{A_l^{(1)}(\varepsilon)}\cup\overline{A_l^{(2)} 
(2\varepsilon)}\,\Big\vert \mathbf{F}_{(l)}\right)\leq{}\\
  {}\leq
  \sum\limits_{\substack{{i\in \mathcal{I};}\\ {W_i\leq W-\varepsilon/2}}}\!\!\!\!
   \mathbf{P}_{a}\left(\beta_l=i\vert 
\mathbf{F}_{(l)}\right)+{}\\
{}+  %\substack{{i=\overline{1,n}}\\ {j=\overline{1,l}}}
\sum\limits_{\substack{{i\in \mathcal{I};}\\ {W_i\leq W-\varepsilon/2}}}\!\!\!\!
\mathbf{P}_{a}\left( \overline{A_l^{(1)}(\varepsilon)}, \ \beta_l=i
\vert \mathbf{F}_{(l)}\right)\leq{}\\
  {}\leq \sum\limits_{\substack{{i\in \mathcal{I};}\\ {W_i\leq W-\varepsilon/2}}}\!\!\!\!
  \mathrm{P}_{a}(\beta_l=i)+{}\\
{}+\sum\limits_{\substack{{i\in \mathcal{I};}\\ {W_i> W-
\varepsilon/2}}}
\!\!\!\!\mathbf{P}_{a}\left( v_l^{(1)}\leq W_i-\fr{\varepsilon}{2}, \beta_l=i\vert 
\mathbf{F}_{(l)}\right) \leq{}\\
  {}\leq \sum\limits_{\substack{{i\in \mathcal{I};}\\ {W_i\leq W-\varepsilon/2}}}\!\!\!\!
   b_i+a_l\left( 
\fr{\varepsilon}{2}\right) \leq q<1
  \end{multline*}
при всех достаточно больших~$l$. Отсюда следует, что для всех достаточно больших 
значений~$n$ выполняется неравенство
$$
\mathbf{P}_a(B_n)\leq q^{n-\ln n}\,.
$$
  
  Следовательно, согласно лемме Бо\-ре\-ля--Кан\-тел\-ли
  \begin{equation}
  \mathbf{P}_{a}\left( \overline{\lim\limits_{n\rightarrow\infty}} B_n\right)=0\,.
  \label{e2-kon}
  \end{equation}
  
  Это означает, что
  $$
  s_n^{(1)}\geq \Delta_{v_n}\geq (1-W-\varepsilon)^{-n+\ln n}\,.
  $$
  
  Оценим сумму $s_n^{(2)}$. Обозначив 
$C_n\hm=A_n^{(1)}(\varepsilon)\cap$\linebreak 
$\cap\overline{A_n^{(2)}(2\varepsilon)}$ и $W_{(n)}\hm=\sum\limits_{i\in 
I} W_i I_{\{\beta_n=i\}}$, получим:
  \begin{multline*}
  \mathrm{P}_{a}\left(C_n\vert \mathrm{ F}_{(n)}\right)=
  \mathrm{P}_{a|} \left( C_n, W_{(n)}<W-\fr{3\varepsilon}{2}\vert \mathrm{
  F}_{(n)}\right) +{}\\
  {}+ \mathrm{P}_{a}\left( 
  C_n, W_{(n)}\geq W-\fr{3\varepsilon}{2}\vert \mathrm{
  F}_{(n)}\right)\leq{}\\
  {}\leq \mathrm{P}_{a}\left( v_n^{(1)}>W-\varepsilon,\, W_{(n)}<W-\fr{3\varepsilon}{2}\vert \mathrm{
  F}_{(n)}\right)+{}\\
  {}+
  \mathrm{P}_{a} \left( v_n^{(2)}\leq W-2\varepsilon,\, W_{(n)}\geq W-
\fr{3\varepsilon}{2}\vert \mathrm{
  F}_{(n)}\right)\leq{}\\
  {}\leq \sum\limits_{i\in \mathcal{I}; W_i\leq W- \varepsilon/2} \mathrm{P}_{a}\left(
  v_{\tau_n,n}>W_i+\fr{\varepsilon}{2},\, \beta_l=i\vert\mathrm{F}_{(n)}\right)+{}\\
  {}+\sum\limits_{\substack{{i\in \mathcal{I};}\\ {W_i> W- 3\varepsilon/2}}}\!\!\!\!
   \mathbf{P}_{a} \left( 
v_{\tau_n,n+\theta_n}\leq W_i-\fr{\varepsilon}{2},\,\beta_l=i\vert\mathbf{F}_{(n)}\right)\leq {}\\
{}\leq
a_n\left( \fr{\varepsilon}{2}\right)\,.
  \end{multline*}
  
  Из определения базового множества стратегий следует, что
  $$
  \sum\limits_{n=1}^\infty \mathbf{P}_{a} (C_n)<\infty\,,
  $$
поэтому согласно лемме Бо\-ре\-ля--Кан\-тел\-ли полу\-чаем:
\begin{equation}
\mathbf{P}_{a}\left( \overline{\lim\limits_{n\rightarrow\infty}} C_n\right) =0\,.
\label{e3-kon}
\end{equation}
  
  Отсюда следует, что
  $$
  \sup\limits_n s_n^{(2)}\leq c<\infty\,.
  $$
  
  Для суммы $s_n^{(3)}$ имеем следующую оценку:
  $$
  s_n^{(3)}\geq \sum\limits_{l=1}^n \left(n+(1-W+\varepsilon)^{-l}\right)< n^2+n(1-
W+\varepsilon)^{-n}.
  $$
  
  Подставляя оценки, полученные для сумм $s_n^{(k)}$, в неравенство~(\ref{e1-kon}), 
получаем:
  \begin{multline*}
  w_n\geq (W-\varepsilon) \left( 1+\fr{s_n^{(2)}+s_n^{(3)}}{s_n^{(1)}}\right)^{-1}\geq 
{}\\
  {}\geq (W-\varepsilon)\left( 1+\fr{c+n^2+n(1-W+\varepsilon)^{-n}}{(1-W-\varepsilon/2)^{-
n+\ln n}}\right)^{-1}\geq{}\\
{}\geq W-3\varepsilon
  \end{multline*}
для всех достаточно больших значений~$n$. Отсюда
\begin{equation}
\lim\limits_{\overline{n\rightarrow\infty}} w_n\geq W\,.
\label{e4-kon}
\end{equation}
  
  Рассмотрим далее множество
  $$
  \Omega^\prime =\left\{ \lim\limits_{n\rightarrow\infty} w_n =W\right\}\cap 
\overline{B}\cap\overline{C}\,,
  $$
где $\overline{B}$ и $\overline{C}$ означают соответственно дополнения к множествам 
$B\hm= \overline{\lim\limits_{n\rightarrow\infty}} B_n$ и $C\hm= 
\overline{\lim\limits_{n\rightarrow\infty}} C_n$.
  
  Согласно формулам~(\ref{e2-kon})--(\ref{e4-kon})
  $$
  \mathbf{P}_{a}\left(\Omega^\prime\right) =1\,.
  $$
  
  Определим следующие события:
  
  \noindent
  \begin{align*}
  D_{n,t}^{(1)} &= \left\{ \tau_{n-1}<t\leq \tau_{n-1}+n\right\} \cap \Omega^\prime\,;\\
  D_{n,t}^{(2)} &= \left\{\tau_{n-1}+n<t\leq \tau_n\right\}\cap \Omega^\prime\,;\\
  D_{n,t}^{(3)} &= \left\{ \tau_{n-1}<t\leq \tau_n\right\} \cap \Omega^\prime\,.
  \end{align*}
  
  На множестве $D_{n,t}^{(1)}$ усредненный доход $v_t\hm=v_{0,t}\hm=
  t^{-1}\sum\limits_{s=1}^t g_s$ оценивается с помощью формулы~(\ref{e1-kon}) как
  
    \noindent
  $$
  v_t\geq \fr{\tau_{n-1} w_n}{\tau_{n-1}+n+\theta_n}\geq W-\varepsilon_n^{(1)}\,,
  $$
где $\varepsilon_n^{(1)}\hm\rightarrow0$ при $n\hm\rightarrow\infty$.
  
  Пусть событие $D_{n,t}^{(2)}$ имеет место. Тогда $\theta_n\geq (1\hm- 
W\hm+\varepsilon)^{-n}$. Кроме того, из определения событий $B_n$, $B$, 
$D_{n,t}^{(2)}$ следует, что для всех достаточно больших значений~$n$ выполняется 
неравенство $v_n\hm> n-\ln n$. Следовательно, на множестве~$D_n^{(2)}$ справедлива 
оценка

  \noindent
  $$
  v_t\geq \fr{\tau_{n-1} w_n}{\tau_{n-1}+n+\theta_n}\geq W-\varepsilon_n^{(2)}\,,
  $$
где $\varepsilon_n^{(2)}\hm\rightarrow0$ при $n\hm\rightarrow\infty$.
  
  Из определения событий $C_n$, $C$, $D_{n,t}^{(3)}$ вытекает, что
  
    \noindent
  $$
  D_{n,t}^{(3)} \subset \left\{ \min\limits_{n<m\leq n+\theta_n} v_{n,m}\geq W-
2\varepsilon\right\}\,,
  $$
поэтому на множестве $D_n^{(3)}$ справедливы неравенства:

  \noindent
\begin{multline*}
\!\!v_t\geq \fr{\tau_{n-1} w_n}{t}+\left(1- \fr{\tau_{n-1}}{t}\right) \left( 1-\tau_n\right)^{-1} 
\!\!\sum\limits_{s=\tau_{n-1}+1}^t \!\!\!\!g_s\geq{}\\
{}\geq \fr{\tau_{n-1} w_n}{t}+\left( 1-\fr{\tau_{n-1}}{t}\right)\left( W-2\varepsilon\right) \geq 
W-2\varepsilon -\varepsilon_n^{(3)},
\end{multline*}
где $\varepsilon_n^{(3)}\rightarrow0$ при $n\hm\rightarrow\infty$.

\pagebreak
  
  Таким образом, на множестве
  $$
  D_{n,t}=\bigcup\limits_{k=1}^3 D_{n,t}^{(k)} = \left\{ \tau_{n-1}<t\leq \tau_n\right\} \cap 
\Omega^\prime
  $$
имеет место оценка $v_n\hm\geq W-\varepsilon-\varepsilon_n$, где 
$\varepsilon_n\hm\rightarrow 0$ при $n\hm\rightarrow\infty$. Достаточность утверждения 
теоремы следует из соотношений $\Omega\hm= \bigcup\limits_{n=1}^\infty \left\{ \tau_{n-
1}\hm<t\hm\leq \tau_n\right\}$ и $\lim\limits_{t\rightarrow\infty} I_{D_{n,t}}\hm=0$.

\section{Заключение}

  Адаптивные стратегии, позволяющие достигать цели в условиях информационной 
неопреде\-лен\-ности, основываясь на <<обучении>> в процессе взаимодействия с объектом, 
находят все более широкое практическое применение. 

В~этой работе было уделено 
внимание теоретическим аспектам адаптивного подхода. Сформулированы определения 
адаптивных стратегий и приведена формальная постановка задачи адаптивного 
управления. Сформулированы и доказаны некоторые утверждения о необходимых 
условиях и достаточных условиях адап\-тив\-ной управляемости. 

Продолжение исследований 
в данном на\-прав\-ле\-нии позволит найти ответы на принципиальные вопросы, в каких 
ситуациях можно рассчитывать на <<приспособление к неизвестной среде>> и сколь 
универсальными могут быть <<обучающиеся>> алгоритмы.



{\small\frenchspacing
{%\baselineskip=10.8pt
\addcontentsline{toc}{section}{Литература}
\begin{thebibliography}{9}


  \bibitem{1-kon}
  \Au{Sragovich~V.\,G.}
  Mathematical theory of adaptive control.~--- Singapore: World Scientific, 2006.
  \bibitem{2-kon}
  \Au{Коновалов~М.\,Г.}
  Методы адаптивной обработки информации и их приложения.~--- М.: ИПИ РАН, 2007.
  
  \label{end\stat}
  
  \bibitem{3-kon}
  \Au{Неве~Ж.}
  Математические основы теории вероятностей.~--- М.: Мир, 1969.
\end{thebibliography}
}
}


\end{multicols}   %1
\def\stat{krivenko}

\def\tit{МНОГОМЕРНЫЙ РЕФЕРЕНСНЫЙ РЕГИОН\\ ВЫСОКОЙ ПЛОТНОСТИ}

\def\titkol{Многомерный референсный регион высокой плотности}

\def\aut{М.\,П.~Кривенко$^1$}

\def\autkol{М.\,П.~Кривенко}

\titel{\tit}{\aut}{\autkol}{\titkol}

\index{Кривенко М.\,П.}
\index{Krivenko M.\,P.}


%{\renewcommand{\thefootnote}{\fnsymbol{footnote}} \footnotetext[1]
%{Работа выполнена при финансовой поддержке РФФИ (проекты 16-07-00677 
%и~15-37-20611-мол\_а\_вед).}}


\renewcommand{\thefootnote}{\arabic{footnote}}
\footnotetext[1]{Институт проблем информатики Федерального исследовательского центра <<Информатика и~управление>> Российской академии наук,
\mbox{mkrivenko@ipiran.ru}}

\vspace*{4pt}



\Abst{Рассматриваются принципы построения многомерных референсных регионов
(MRR~--- multivariate reference region). 
Предложен оригинальный метод построения региона на основе областей с~высокой 
плотностью точек и~аппроксимации распределения данных с~помощью смеси нормальных 
распределений. Для оценки порога для плотности распределения используется  
бут\-стреп-ме\-тод. В~качестве эксперимента рассмотрена задача построения 
и~использования эталонной области для прогнозирования типа мочевого камня. Обработка 
реальных данных продемонстрировала преимущества предлагаемых решений.}

\KW{многомерный референсный регион; область высокой плотности; бут\-стреп-ме\-тод; 
смесь многомерных нормальных распределений}

\vspace*{6pt}

\DOI{10.14357/19922264170207} 


\vskip 10pt plus 9pt minus 6pt

\thispagestyle{headings}

\begin{multicols}{2}

\label{st\stat}

\section{Введение}

     Многомерный референсный регион 
был предложен в~литературе по клинической химии в~начале 1970-х~гг.\ как 
альтернатива одномерным референсным интервалам~[1]. Там излагались 
преимущества предлагаемых множественных тестов, хоть и~имеющих 
упрощенный вид, но снижающих (по отношению к~одномерным вариантам) 
число ложных положительных результатов. Появление MRR оказалось 
особенно привлекательным для интерпретации результатов наборов 
медицинских тестов. Тем не менее возникали трудности в~построении 
и~использовании процедур многомерного анализа (см., например,~[2]), 
связанные, в~частности, с~быстрым увеличением числа параметров, которые 
должны быть оценены. Немногие лаборатории использовали MRR в~своей 
практике, причем в~экспериментальном режиме, и,~как следствие, на 
сегодняшний день имеется относительно малое количество соответствующих 
публикаций. 

\vspace*{-6pt}

\section{Многомерный референсный регион на основе расстояния Махалонобиса}

\vspace*{-2pt}

     Одномерный референсный интервал, полученный статистическим путем, 
использует центральную часть значений анализируемого показателя, обычно 
соответствующую~95\% некоторой популяции~--- совокупности особей 
определенного вида (например, здоровой части населения определенного пола 
из некоторого диапазона возрастов). Одномерные референсные интервалы 
применялись в~течение многих лет в~качестве стандартного приема 
интерпретации лабораторных данных. Они легко формируются, хранятся, 
извлекаются и~передаются в~лабораторных информационных системах, просты 
в~понимании, хорошо воспринимаются медицинским сообществом в~ходе 
длительного использования. Тем не менее одномерные референсные интервалы 
при классификации данных могут дать большое число ложно аномальных 
результатов. Этот далеко не единственный недостаток однофакторного 
референсного интервала может быть полностью или частично устранен 
с~помощью MRR.
     
     Простейшим и~весьма распространенным способом построения MRR 
является использование прямого произведения отдельных референсных 
интервалов в~предположении, что они статистически независимы. Пусть 
$(1\hm-\alpha)$~--- вероятность попадания в~MRR, а~$p_0$~--- вероятность 
попадания в~референсный интервал для любого из~$d$~признаков, тогда 
$p_0\hm= \sqrt[d]{1-\alpha}$. С~ростом размерности~$d$ значения~$p_0$ 
быстро приближаются к~1, что фактически лишает смысла применение MRR.
     
     Как и~в одномерном случае, отправной точкой для построения MRR 
может стать нормальное распределение. Идеи центрального расположения 
референсного региона и~заданной вероятности попадания в~него приводят для 
$d$-мер\-но\-го нормального распределения, имеющего плотность 
распределения
     \begin{multline*}
     \varphi(y,\mu,\Sigma) ={}\\
     {}=(2\pi)^{-d/2}\vert\Sigma\vert^{-1/2}\exp \left( -\fr{\left(y-
\mu\right)^{\mathrm{T}} \Sigma^{-1}(y-\mu)}{2}\right),
   \end{multline*}
где величина $(y-\mu)^{\mathrm{T}} \Sigma^{-1} (y-\mu)$ есть квадрат так 
называемого расстояния Махаланобиса между~$y$ и~$\mu$, к~использованию 
многомерного эллипсоида
\begin{multline*}
(2\pi)^{-d/2}\vert\Sigma\vert^{-1/2}\exp \left( -\fr{\left(y-\mu\right)^{\mathrm{T}}
\Sigma^{-1} 
(y-\mu)}{2}\right) ={}\\
{}=const
\end{multline*}
или, что то же самое, 
$$ 
(y-\mu)^{\mathrm{T}} \Sigma^{-1}(y-\mu)=const\,.
$$
Его называют эллипсоидом равной плотности распределения (или просто 
эллипсоидом равной вероятности). 
     
     Если задаться вероятностью $(1\hm-\alpha)$ попадания в~эллипсоид 
равной вероятности вида $(y\hm-\mu)^{\mathrm{T}}\Sigma^{-1} (y\hm-\mu)\hm= 
\rho$, то параметр~$\rho$ удовлетворяет уравнению $\mathrm{Pr}\left\{ 
\chi_d^2\leq \rho\right\} \hm=1\hm-\alpha$.
     
     Использование эллипсоида в~качестве MRR будет оправдано только 
тогда, когда исходное распределение данных есть многомерное нормаль-\linebreak ное. 
Поэтому становятся актуальными критерии\linebreak подгонки, а~также использование 
процедур норма\-ли\-зации распределения данных в~многомерном\linebreak случае.
 Если 
с~помощью тестов выявляется, что распределение не является нормальным, то 
Международная федерация клинической химии и~лабораторной медицины 
рекомендует, согласно~[3], использовать двухступенчатую процедуру 
нормализации. Следует обратить внимание, что многошаговость здесь 
относится не к~многомерности, а касается лишь покоординатного 
преобразования распределения данных к~нормальному.
     
     Первые же попытки применения MRR на основе расстояния 
Махалонобиса (фактически это означает принятие модели нормального 
распределения референсных значений) выявили ряд недостатков (более 
подробно смотри в~\cite[разд.~6.2]{4-kri}):
     \begin{itemize}
\item проявление <<проклятий>> размерности при механическом 
увеличении~$d$, в~особенности если игнорируется этап анализа состава 
признаков~[1, 5, 6];
\item из-за небольших объемов обучающей выборки невысокая устойчивость 
при применении, в~частности чувствительность к~увеличению неточностей 
измерений после того, как регион был установлен~\cite{5-kri, 7-kri}. 
\item предположение о нормальном распределении и~попытки <<подправить>> 
действительность с~помощью преобразований реальных данных для их 
нормализации при увеличении размерности данных становятся все более 
шаткими~\cite{5-kri};
\item представление и~интерпретация выводов на основе MRR трудно 
понимаемы не только для специалистов в~предметной области~[8].
\end{itemize}

\vspace*{-9pt}

\section{Многомерный референсный регион высокой плотности}

\vspace*{-2pt}

     Заметим, что в~случае нормального распределения референсных значений 
для точек внут\-ри построенного эллипсоида значения плотности\linebreak распределения 
больше, чем на границе, а~вне~--- меньше. Это замечание позволяет 
предложить другой подход к~построению MRR.
     
     \smallskip
     
     \noindent
     \textbf{Определение.}\ Eсли плотность распределения референсных 
значений есть $f(y)$, то MRR есть область $A_t\hm= \left\{ y\in 
\mathcal{R}^d\vert f(y)\hm\geq t\right\}$ для некоторого порогового 
значения~$t$. 
     
     \smallskip
     
     Для нормального распределения это уже упомянутый эллипсоид равной 
вероятности. Если задается вероятность $(1\hm-\alpha)$ попадания в~$A_t$, то 
пороговое значение~$t$ есть решение уравнения $\int\nolimits_{A_t} 
f(u)\,du\hm=1\hm-\alpha$, получить которое аналитически в~случае 
произвольной плотности распределения вряд ли возможно. Здесь присутствуют 
две проблемы: вычисление многомерного интеграла и~зависимость области 
интегрирования от неизвестного значения. Для решения их предлагается 
привлечь метод моделирования.
     
     Сгенерируем выборку из $f(y)$, которую обозначим как $Y^f\hm= \left\{ 
y_1^f, \ldots, y_m^f\right\}$. Для оценки $\int\nolimits_{A_t} f(u)\,du$ 
используем отношение:

\noindent
\begin{multline*}
     \fr{\left\vert \left\{ y_i^f\vert y_i^f\in A_t\right\}\right\vert }{m} =
      \fr{\left\vert\left\{ y_i^f\vert 
f\left(y_i^f\right) \geq t\right\}\right\vert }{m} ={}\\
{}= 1-\fr{\left\vert \left\{ y_i^f\vert f(y_i^f)<t\right\}\right\vert }{m}=1-
F_m(t)\,,
     \end{multline*}
где $F_m(t)$~--- эмпирическая функция распределения случайной 
величины~$f(y)$, т.\,е.\ случайной величины, являющейся результатом 
преобразования с~помощью функции~$f(\cdot)$ случайной величины, име\-ющей 
плотность распределения~$f(u)$.

     Таким образом, искомая оценка~$t^*$ должна удовле\-тво\-рять уравнению 
$F_m(t^*)\hm=\alpha$ и~может быть получена как непараметрическая оценка 
квантиля\linebreak\vspace*{-12pt}

\pagebreak

\noindent
 порядка~$\alpha$ из распределения $F_m(\cdot)$. Если обозначить 
$f_i\hm= f(y_i^f)$, то~$t^*$ есть~$f_{(r)}$, где
     $$
     r= \begin{cases}
     m\alpha, &\ m\alpha~\mbox{---~целое}\,;\\
     \lfloor m\alpha+1\rfloor\,, & m\alpha~\mbox{--- не целое}\,.
     \end{cases}
     $$
     Заметим, что для такой оценки можно указать доверительный интервал.
     
     Для построения MRR необходимо знать распределение данных. При 
реализации принципа точек высокой плотности в~первую очередь следует 
обратиться к~параметрическим моделям, в~част\-ности к~смеси нормальных 
распределений, име\-ющей плотность распределения
     $$
     f(u) =\sum\limits_{j=1}^k p_j \varphi\left (u,\mu_j, \Sigma_j\right)\,.
     $$
Если $\hat{f}(u)$~--- оценка смеси, то~$t^*$ строится сле\-ду\-ющим образом:
\begin{itemize}
\item генерируется выборка $\left\{ y_1^f,\ldots , y_m^f\right\}$ из $\hat{f}(u)$ и~
для каждого ее $i$-го элемента подсчитывается значение $\hat{f}\left( 
y_i^f\right)$;
\item в~качестве~$t^*$ берется непараметрическая оценка квантиля 
порядка~$\alpha$ (в случае необходимости дополнительно находится 
непараметрическая оценка доверительного интервала для~$t^*$, что 
может характеризовать правильность выбранного объема для 
генерируемой выборки).
\end{itemize}

     Пусть для $f(u)$ имеется~$A_t$, а также получена $\hat{f}(u)$ 
и~соответствующий MRR вида~$\hat{A}_t$. Качество аппроксимации~$A_t$ 
с~по\-мощью~$\hat{A}_t$ можно оценить с~по\-мощью вероятности совпадения 
этих областей, т.\,е. 
     $$
     P_c= \int\limits_{\{ u\in A_t\}\cup \{u\in \hat{A}_t\}} \hspace*{-6mm}
f(u)\,du+\int\limits_{\{u\not\in A_t\} \cup\{ u\not\in \hat{A}_t\}}\hspace*{-6mm} f(u)\,du\,.
     $$
     
     Для оценки  $P_c$ можно использовать величину
     \begin{multline*}
     \hat{P}_c= \fr{\left\vert \left\{ 
     y_i^f\vert y_i^f \in \left\{\left\{ y_i^f\in A_t\right\}\cup \left\{y_i^f\in 
\hat{A}_t\right\}\right\}\right\}\right\vert}{m}+{}\\
{}+\fr{\left\vert \left\{ y_i^f\vert y_i^f \in \left\{\left\{ y_i^f\not\in A_t\right\}\cup 
\left\{ y_i^f\not\in \hat{A}_t\right\}\right\}\right\}\right\vert}{m}\,.
     \end{multline*}
     
     Использование MRR высокой плотности для диагностирования сводится 
к~реализации так называемого слабого критерия значимости для наблюденного 
значения~$x$: нулевая гипотеза заключается в~том, что $x\hm\in A_t$, 
статистика критерия есть $\hat{f}(x)$ и~решение о~принадлежности 
критической об\-ласти~$A_t$ принимается при больших значениях~$\hat{f}(x)$.
     
     Для медицинской практики важна возможность использования 
референсного региона при интерпретации результатов обследования 
некоторого пациента с~вектором признаков~$x$. В~подобных случаях 
сложившейся практикой для слабых критериев значимости является 
использование критического уровня~$\alpha_{\mathrm{cr}}$ (более распространенным 
в~медицине является употребление термина $p$-зна\-че\-ние)  $\alpha_{\mathrm{cr}}\hm= 
\mathrm{Pr}\left\{ \hat{f}(y)\hm\leq \hat{f}(x)\right\}$, где $y$~--- случайная 
величина, имеющая плотность распределения~$\hat{f}(u)$, а $\hat{f}(x)$~--- 
значение плотности распределения~$\hat{f}(u)$ в~точке~$x$. Эта 
характеристика дает представление о~том, насколько сильно данное 
наблюденное значение~$x$ противоречит гипотезе (или подкрепляет ее) 
о~принадлежности данных MRR. При выбранном же заранее уровне 
значимости с~помощью~$\alpha_{\mathrm{cr}}$ сразу же можно принять конкретное 
решение. 

\vspace*{-9pt}

\section{Эксперименты}

\vspace*{-2pt}

     Для демонстрации возможностей MRR использовались данные по 
прогнозу химического состава мочевых камней по метаболическим 
показателям мочи и~сыворотки крови, а также антропологическим 
характеристикам пациентов~[9]. В качестве исходной классификации камней 
рассматривалась следующая: чисто оксалатные (далее обозначены как O), чисто 
уратные (U), чисто фосфатные (P), смесь только оксалатных и~уратных (OU), 
смесь только оксалатных и~фосфатных (OP), смесь только уратных 
и~фосфатных (UP), все остальные. Данная классификация была построена 
в~[10] на основе доминирующих частот встречаемости основных компонентов. 
В~качестве референсных значений рассматривались наборы метаболических 
и~антропологических показателей (их всего было~14), соответствующих 
определенному классу камней.

\begin{table*}\small
\begin{center}


\begin{tabular}{|c|c|c|c|c|c|c|}
\multicolumn{7}{c}{Качество классификации с~помощью MRR}\\
\multicolumn{7}{c}{\ }\\[-6pt]
\hline
\multicolumn{1}{|c|}{\raisebox{-6pt}[0pt][0pt]{\tabcolsep=0pt\begin{tabular}{c}Тип\\ камня\end{tabular}}}&
\multicolumn{1}{c|}{\raisebox{-6pt}[0pt][0pt]{$N$}}&$(1-\alpha)$, 
&\multicolumn{2}{c|}{MRR(5)}&\multicolumn{2}{c|}{MRR(1)}\\
\cline{4-7}
&&&&&&\\[-9pt]
&&\%&$(1-\hat{\alpha})$, \%&$\hat{\beta}$, \%&$(1-\hat{\alpha})$, \%&$\hat{\beta}$, \%\\
\hline
\multicolumn{1}{|c|}{\raisebox{-18pt}[0pt][0pt]{O}}&
\multicolumn{1}{c|}{\raisebox{-18pt}[0pt][0pt]{82}}
&95&100\hphantom{9}&71&90&24\\
&&85&96&78&89&36\\
&&75&91&85&77&44\\
&&65&76&88&74&50\\
\hline
\multicolumn{1}{|c|}{\raisebox{-18pt}[0pt][0pt]{U}}&
\multicolumn{1}{c|}{\raisebox{-18pt}[0pt][0pt]{76}}&95&100\hphantom{9}&75&91&24\\
&&85&99&85&80&35\\
&&75&82&89&74&48\\
&&65&71&91&68&56\\
\hline
\multicolumn{1}{|c|}{\raisebox{-18pt}[0pt][0pt]{P}}&
\multicolumn{1}{c|}{\raisebox{-18pt}[0pt][0pt]{83}}&95&100\hphantom{9}&66&87&25\\
&&85&94&78&86&33\\
&&75&86&82&82&41\\
&&65&77&87&75&47\\
\hline
\end{tabular}
\end{center}
\end{table*}
     
     
     Для каждого из основных классов O, U, P, OU, OP и~UP перед построением 
MRR проводилась селекция признаков и~принималось то значение размерности 
признакового пространства~$d$ и~соответствующий набор показателей, 
которые позволяли прогнозировать состав камней без потери качества 
(методика описана в~\cite{9-kri} и~привела к~значению $d\hm=9$). В~качестве 
модели данных в~первую очередь рассматривалась смесь многомерных 
нормальных распределений из пяти элементов (подбор числа элементов смеси 
проводился с~по\-мощью AIC~--- Akaike information criterion), для соответствующего региона было принято 
обозначение MRR(5). Для сравнения также использовалась модель 
нормального распределения, которой соответствовал MRR(1). Полученные 
результаты приводятся час\-тич\-но в~таблице, где $N$~--- объем 
классифицируемых данных; $\hat{\alpha}$~--- оценка для~$\alpha$; 
$\hat{\beta}$~--- оценка мощности критерия при определении типа камня на 
основании MRR.


     Одной из базовых характеристик является вероятность попадания в~MRR 
$(1\hm-\alpha)$ и~ее оценка $(1\hm-\hat{\alpha})$. Сравнение соответствующих 
столбцов с~учетом значений~$N$ и~ориентировочных значений разброса 
(стандартные отклонения на основе биномиального распределения) не 
позволило выявить явных отклонений. Необходимо, правда, отметить, что во 
всех проанализированных случаях для MRR(5) оказалось, что $1\hm-
\hat{\alpha}\hm\geq 1\hm-\alpha$.
     
     Назначение MRR, заключающееся в~сжатом представлении референсных 
значений, в~многомерном случае практически не проявляется. Для задания 
MRR(5) необходимо указать следующие величины: $1\hm-\alpha$, $t$, 
$p_1,\ldots, p_{k-1}$, $\mu_1, \Sigma_1,\ldots , \mu_k,\Sigma_k$, общее 
количество которых равно  $[2\hm+ (k\hm-1)\hm+ k(d\hm+ d(d\hm+1)/2)]$ 
и,~в~частности, в~рассматриваемых экспериментах~--- 276. Для MRR(1) это 
значение меньше и~равно~56. При этом для обрабатываемой обучающей 
выборки в~зависимости от класса камней речь идет о~порядка~10$^2$ векторах 
данных (см.\ столбец со значениями~$N$), что приблизительно 
дает~10$^3$~скалярных величин.
     
     Другое назначение MRR состоит в~его использовании для 
диагностирования (классификации). В~этой связи в~первую очередь 
проводился сравнительный анализ MRR(1) (фактически это означает, что 
построение региона осуществляется на основе расстояния Махаланобиса) 
и~MRR(5) (модель смеси нормальных распределений и~предложенный 
в~данной работе метод оценивания па\-ра\-мет\-ров региона). Показателем 
информативности метода построения многомерного региона выступала 
мощность соответствующего слабого критерия значимости, а~именно: 
вероятность не попасть в~MRR при условии, что данные берутся из дополнения 
к~классу, для которого построена MRR. Сравнение соответствующих столбцов 
говорит о~явном преимуществе двух предложенных моментов: усложнение 
модели данных путем перехода от нормального распределения к~смеси 
нормальных распределений и~построение региона высокой плотности.
     
     Использование критического уровня можно продемонстрировать  
с~по\-мощью зависимости результатов сравнения двух классов от того, какой 
класс взять за основу. Введем для возможных значений $p$-ве\-ли\-чи\-ны три 
интервала: $(-\infty, 1\%)$, $[1\%, 5\%)$, $[5\%, 100\%)$ с~соответствующей 
интерпретацией положения наблюденного набора показателей для пациента 
относительно построенного MRR: уверенное непопадание, неуверенное 
попадание, уверенное попадание. Если MRR построить для оксалатных камней, 
то результаты для анализа пациентов с~фосфатными камнями дадут следующий 
вектор относительных частот попадания $p$-ве\-ли\-чин в~указанные 
интервалы: $(60\%, 18\%, 22\%)$. Если же MRR строить для фосфатных 
камней, то получим $(71\%, 5\%, 24\%)$. Таким образом, для классификации 
указанных камней при приблизительно одинаковых частотах попадания в~MRR 
(22\% или~24\%) уверенный отказ от референсного региона происходит чаще, 
если принять за базовый MRR регион для фосфатных камней. Построение 
шкалы, подобной рассмотренной, является прерогативой специалистов 
в~предметной области, в~данной работе она использовалась только для 
иллюстрации. 

\vspace*{-6pt}

\section{Заключение}

\vspace*{-2pt}

     На настоящий момент имеется относительно мало примеров применения 
MRR в~клинической практике. Тому есть несколько причин. Математическое 
обеспечение, необходимое для получения и~применения MRR, не отвечает 
возможностям большинства клинических лабораторий. Лаборатории слабо 
оснащены программными средствами\linebreak для реализации достаточно сложного 
математического аппарата многомерного анализа, а~еще важнее, что 
отсутствуют методики, инструкции по\linebreak использованию соответствующих 
средств. Лишь немногие клинические применения демонстрируют 
преимущества MRR, хотя свидетельств неудачных попыток больше.
     
     Несмотря на сложности внедрения мно\-го\-мерно\-го анализа референсных 
значений, можно сформулировать некоторые рекомендации по иссле\-до\-ва\-нию 
и~разработке MRR. Во-пер\-вых, эффективная размерность в~MRR должна 
быть как можно меньше, чтобы избежать затенения диагностически полезной 
информации тестами, со\-зда\-ющи\-ми шум. Низкая размерность также должна 
уменьшить неблагоприятные последствия увеличения неточности результатов 
в~связи с~ростом числа анализируемых показателей. Во-вто\-рых, показатели 
(тес\-ты), включенные в~MRR, должны быть физиологически релевантными 
исследуемому кругу расстройств, чтобы максимизировать информацию, 
полученную от MRR. В-треть\-их, чтобы учесть эффекты долгосрочной 
лабораторной из\-мен\-чи\-вости, данные, используемые для получения MRR, 
долж\-ны быть собраны и~проанализированы в~течение достаточно большого 
периода времени (от нескольких недель до нескольких месяцев).  
В-чет\-вер\-тых, представление результатов лабораторных исследований 
следует осуществлять в~графическом виде, чтобы помочь врачам лучше понять 
MRR. Различные подходы к~уменьшению размерности помогут выполнить это 
требование.
     
     Необходима дальнейшая разработка пояснительных инструментов, 
способных воспринять результаты анализа MRR. При этом дополнительно 
необходима информация о~том, какие именно тес\-ты являются важнейшими 
факторами нарушения нормы. Надо признать, что соответствующий 
математический аппарат еще предстоит разработать. Решение перечисленных 
вопросов играет важную роль для обеспечения постоянного клинического 
применения MRR. 

\vspace*{-6pt}
     
{\small\frenchspacing
 {%\baselineskip=10.8pt
 \addcontentsline{toc}{section}{References}
 \begin{thebibliography}{99}
 
 \vspace*{-2pt}
 
\bibitem{1-kri}
\Au{Boyd J.\,C.} Reference regions of two or more dimensions~// Clin. Chem. Lab. 
Med., 2004. Vol.~42. No.\,7. P.~739--746.
\bibitem{2-kri}
\Au{Winkel P.} Patterns and clusters~--- multivariate approach for interpreting 
clinical chemistry results~// Clin. Chem., 1973. Vol.~19. No.\,12. P.~1329--1333.
\bibitem{3-kri}
IFCC. Expert panel on theory of reference values. Approved recommendation on the 
theory of reference values. Part~5. Statistical treatment of collected reference values. 
Determination of reference limits~// J.~Clin. Chem. Clin. Biochem., 1987. Vol.~25. 
No.\,9. P.~645--656.
\bibitem{4-kri}
\Au{Кривенко М.\,П.} Статистические методы представления и~предварительной 
обработки референсных значений.~--- М.: ФИЦ ИУ РАН, 2016. 160~с.
\bibitem{5-kri}
\Au{Boyd J.\,C., Lacher~D.\,A.} The multivariate reference range: An alternative 
interpretation of multi-test profiles~// Clin. Chem., 1982. Vol.~28. No.\,2.  
P.~259--265.
\bibitem{6-kri}
\Au{Albert A., Harris~E.\,K.} Multivariate interpretation of clinical laboratory  
data.~--- New York, NY, USA: CRC Press, 1987. 328~p.
\bibitem{7-kri}
\Au{Linnet K.} Influence of sampling variation and analytical errors on the 
performance of the multivariate reference region~// Meth. Inf. Med., 1988. Vol.~27. 
No.\,1. P.~37--42.
\bibitem{8-kri}
\Au{Durbridge T.\,C.} Clinical acceptance of a multi-test reference region for 
biochemical-panel results~// Clin. Chem., 1983. Vol.~29. No.\,10. P.~1724--1726.
\bibitem{9-kri}
\Au{Кривенко М.\,П.} Критерии значимости отбора признаков классификации~// 
Информатика и~её применения, 2016. Т.~10. Вып.~3. С.~32--40.
\bibitem{10-kri}
\Au{Кривенко М.\,П., Голованов~С.\,А., Сивков~А.\,В.} Анализ однородности 
данных о химическом составе камней при уролитиазе~// Информатика и~её 
применения, 2013. Т.~7. Вып.~4. С.~94--104.
 \end{thebibliography}

 }
 }

\end{multicols}

\vspace*{-10pt}

\hfill{\small\textit{Поступила в~редакцию 5.12.16}}

\vspace*{4pt}

%\newpage

%\vspace*{-24pt}

\hrule

\vspace*{2pt}

\hrule

\vspace*{-3pt}


\def\tit{HIGH-DENSITY MULTIVARIATE REFERENCE REGION\\[-5pt]}

\def\titkol{High-density multivariate reference region}

\def\aut{M.\,P.~Krivenko\\[-7pt]}

\def\autkol{M.\,P.~Krivenko}

\titel{\tit}{\aut}{\autkol}{\titkol}

\vspace*{-16pt}


\noindent
Institute of Informatics Problems, Federal Research Center 
``Computer Science and Control'' of the Russian
Academy of Sciences,  44-2~Vavilov Str., Moscow 119333, Russian Federation



\def\leftfootline{\small{\textbf{\thepage}
\hfill INFORMATIKA I EE PRIMENENIYA~--- INFORMATICS AND
APPLICATIONS\ \ \ 2017\ \ \ volume~11\ \ \ issue\ 2}
}%
 \def\rightfootline{\small{INFORMATIKA I EE PRIMENENIYA~---
INFORMATICS AND APPLICATIONS\ \ \ 2017\ \ \ volume~11\ \ \ issue\ 2
\hfill \textbf{\thepage}}}

\vspace*{2pt}




\Abste{The paper considers the principles of construction of multivariate 
reference regions. An original method of construction of 
a~region on the basis of areas of high density of points and approximation 
of data distribution with a~mixture of normal distributions is suggested. 
To estimate the threshold for the probability density, the bootstrap method is used. 
As an experiment, the paper considers the problem of description and use of 
the reference region for predicting the type of urinary stones. 
Real data treatment demonstrated the benefits of the proposed solutions.}

\KWE{multivariate reference region; high-density region; bootstrap method; 
multivariate normal mixture}

\DOI{10.14357/19922264170207} 

%\vspace*{-18pt}

%\Ack
%\noindent



%\vspace*{3pt}

  \begin{multicols}{2}

\renewcommand{\bibname}{\protect\rmfamily References}
%\renewcommand{\bibname}{\large\protect\rm References}

{\small\frenchspacing
 {%\baselineskip=10.8pt
 \addcontentsline{toc}{section}{References}
 \begin{thebibliography}{99}
\bibitem{1-kri-1}
\Aue{Boyd, J.\,C.} 2004. Reference regions of two or more dimensions. \textit{Clin. 
Chem. Lab. Med.} 42(7):739--746.

\bibitem{2-kri-1}
\Aue{Winkel, P.} 1973. Patterns and clusters~--- multivariate approach for interpreting 
clinical chemistry results. \textit{Clin. Chem.} 19(12):1329--1333.
\bibitem{3-kri-1}
IFCC. 1987. Expert panel on theory of reference values. Approved recommendation on the 
theory of reference values. Part~5. Statistical treatment of collected reference values. 
Determination of reference limits. \textit{J.~Clin. Chem. Clin. Biochem.} 
25(9):645--656.
\bibitem{4-kri-1}
\Aue{Krivenko, M.\,P.} 2016. \textit{Statisticheskie metody predstavleniya 
i~predvaritel'noy obrabotki referensnykh znacheniy}
[Statistical methods for representation and preliminary processing of
reference values]. Moscow: FRC CSC RAS. 160~p.

\bibitem{5-kri-1}
\Aue{Boyd, J.\,C., and D.\,A.~Lacher.} 1982. The multivariate reference range: An 
alternative interpretation of multi-test profiles. \textit{Clin. Chem.}  
28(2):259--265.
\bibitem{6-kri-1}
\Aue{Albert, A., and E.\,K.~Harris.} 1987. \textit{Multivariate interpretation of 
clinical laboratory data}. New York, NY: CRC Press. 328~p.
\bibitem{7-kri-1}
\Aue{Linnet, K.} 1988. Influence of sampling variation and analytical errors on the 
performance of the multivariate reference region. \textit{Meth. Inf. Med.}  
27(1):37--42.
\bibitem{8-kri-1}
\Aue{Durbridge, T.\,C.} 1983. Clinical acceptance of a multi-test reference region 
for biochemical-panel results. \textit{Clin. Chem.} 29(10):1724--1726.
\bibitem{9-kri-1}
\Aue{Krivenko, M.\,P.} 2016. Kriterii znachimosti otbora priznakov klassifikatsii
[Significance tests of feature selection for~classification]. \textit{Informatika i~ee 
Primeneniya~--- Inform. Appl.} 10(3):32--40.
\bibitem{10-kri-1}
\Aue{Krivenko, M.\,P., S.\,A.~Golovanov, and A.\,V.~Sivkov}. 2013. Analiz 
odnorodnosti dannykh o~khimicheskom sostave kamney pri urolitiaze
[Analysis of data homogeneity of~the~chemical compositions 
of~stones in~case of~urolithiasis]. \textit{Informatika i~ee Primeneniya~---
Inform Appl.} 7(4):94--104.
\end{thebibliography}

 }
 }

\end{multicols}

\vspace*{-3pt}

\hfill{\small\textit{Received December 5, 2016}}


\Contrl

\noindent
\textbf{Krivenko Michail P.} (b.\ 1946)~--- Doctor of Science in technology, 
professor, leading scientist, Institute of Informatics Problems, Federal Research 
Center ``Computer Science and Control'' of the Russian Academy of Sciences, 
\mbox{44-2}~Vavilov Str., Moscow 119333, Russian Federation; \mbox{mkrivenko@ipiran.ru}

\label{end\stat}


\renewcommand{\bibname}{\protect\rm Литература}       %2
\include{torhin-rud}   %3
\newcommand{\cov}{\textrm{cov}}
%\newcommand{\indic}{\mathbb{1}}
\newcommand{\Obig}{\textsf{O}}
\newcommand{\osml}{\textsf{o}}
\newcommand{\hsig}{\hat\sigma^2}
\newcommand{\Yljk}{Y_{\lambda;j,\mathbf{k}}}

\newcommand{\Yljks}{Y_{\lambda';j',\mathbf{k'}}}
\newcommand{\muljk}{\mu_{\lambda;j,\mathbf{k}}}
\newcommand{\solj}{\sigma_{\lambda;j}}
\newcommand{\soljs}{\sigma_{\lambda';j'}}
\newcommand{\solz}{\sigma_{\lambda;0}}
\newcommand{\silz}{\sigma_{1;0}}
\newcommand{\siilz}{\sigma_{2;0}}
\newcommand{\siiilz}{\sigma_{3;0}}
\newcommand{\slj}{\solj^2}
\newcommand{\sljs}{\soljs^2}
\newcommand{\hslj}{\hat\sigma^2_{\lambda;j}}
\newcommand{\hsljs}{\hat\sigma^2_{\lambda';j'}}
\newcommand{\hslz}{\hat\sigma_{\lambda;0}^2}
\newcommand{\Tlj}{T_{\lambda;j}}
\newcommand{\hTlj}{\hat T_{\lambda;j}}

\newcommand{\indYjklTj}{\Ik_{\left|\Yljk\right|\leqslant \Tlj}}
\newcommand{\indYjkgTj}{\Ik_{\left|\Yljk\right|>\Tlj}}
\newcommand{\prbYjkgTj}{\p\left( \left|\Yljk\right|>\Tlj \right)}
\newcommand{\indYjklhTj}{\Ik_{\left|\Yljk\right|\leqslant \hTlj}}
\newcommand{\indYjkghTj}{\Ik_{\left|\Yljk\right|>\hTlj}}
\newcommand{\sumljk}{\sum\limits_{\lambda,j,\mathbf{k}}}

\def\stat{markin}

\def\tit{АСИМПТОТИКИ ОЦЕНКИ РИСКА ПРИ ПОРОГОВОЙ ОБРАБОТКЕ ВЕЙВЛЕТ-ВЕЙГЛЕТ КОЭФФИЦИЕНТОВ В ЗАДАЧЕ ТОМОГРАФИИ}

\def\titkol{Асимптотики оценки риска при пороговой обработке вейвлет-вейглет коэффициентов в задаче томографии}

\def\autkol{А.\,В.~Маркин, О.\,В.~Шестаков}
\def\aut{А.\,В.~Маркин$^1$, О.\,В.~Шестаков$^2$}

\titel{\tit}{\aut}{\autkol}{\titkol}

%{\renewcommand{\thefootnote}{\fnsymbol{footnote}}\footnotetext[1]
%{Исследования выполнены при частичной поддержке РФФИ, гранты 08-01-00567, 08-01-91205, 09-01-12180.}}

\renewcommand{\thefootnote}{\arabic{footnote}}
\footnotetext[1]{Московский государственный университет им.\ М.\,В.~Ломоносова, 
факультет вычислительной математики и кибернетики, кафедра математической статистики, artem.v.markin@mail.ru}
\footnotetext[2]{Московский государственный университет им.\ М.\,В.~Ломоносова, 
факультет вычислительной математики и кибернетики, кафедра математической статистики,
oshestakov@cs.msu.su}


\Abst{Рассмотрена задача реконструкции изображения по радоновскому образу с помощью вейв\-лет-вейг\-лет разложения. 
Исследованы свойства оценки риска пороговой обработки вейг\-лет-коэф\-фи\-ци\-ен\-тов, такие как состоятельность и 
асимптотическая нормальность.}

\KW{вейвлеты; томография; пороговая обработка; оценка риска; предельное распределение}

     \vskip 18pt plus 9pt minus 6pt

      \thispagestyle{headings}

      \begin{multicols}{2}

      \label{st\stat}


\section{Введение}

Вейвлет-преобразование является весьма популярным и удобным методом обработки нестационарных сигналов и 
изображений. Одна из основных задач, для которых используются вейв\-ле\-ты,~---
удаление шума и сжатие. Эти операции производятся путем пороговой обработки 
вейв\-лет-коэффициентов. Кроме того, вейвлеты могут быть использованы для обращения 
линейных операторов, таких, например, как преобразование Радона. В этом случае 
пороговая обработка выполняет задачу регуляризации соответствующей формулы обращения.

Пусть на плоскости $(x,\,y)$ задана функция~$f$. Определим образ 
(или проекции) Радона~$\mathcal{R}f$ как набор интегралов от~$f$ по всевозможным прямым плоскости
\begin{equation}
\label{eq_radonTransform}
\mathcal{R}f(s,\theta)=\int\limits_{L_{s,\theta}}f\left(x,\,y\right)\,dl\,,
\end{equation}
где
\begin{equation*}
L_{s,\theta}=\left\{ (x,\,y): x\cos\theta+y\sin\theta-s=0 \right \}\,.
\end{equation*}
Формула обращения преобразования~(\ref{eq_radonTransform}) впервые была получена 
Радоном, ее можно записать в следующем виде~\cite{Natterer}:
\begin{equation}
\label{eq_radonInverse}
f = \fr{1}{2}\mathcal{R}^{\#}\mathcal{I}^{-1}\mathcal{R}f\,,
\end{equation}
где $\mathcal{R^{\#}}$~--- оператор обратного проецирования:
\begin{equation*}
\left(\mathcal{R^{\#}}g\right)(x,y)=\int\limits_0^{2\pi}g(x\cos\theta+y\sin\theta,\theta)\,d\theta\,;
\end{equation*}
$\mathcal{I}$~--- потенциал Рисса:
\begin{equation}
\label{eq_RieszPoten}
\left(\mathcal{F}_1\mathcal{I}^\alpha g\right) (\omega) = |\omega|^{-\alpha}\left(\mathcal{F}_1 g\right)(\omega)\,,
\end{equation}
а $\mathcal{F}_k$~--- $k$-мерное преобразование Фурье.

Для точного восстановления~$f$ требуется точное знание всевозможных проекций~$\mathcal{R}f(s,\,\theta)$. 
На практике же имеют дело с конечным числом проекций, причем в проекциях присутствует шум.
При этом задача томографии является некорректной, т.\,е.\ малые изменения в проекциях могут 
при\-вес\-ти к восстановлению изображения, существенно отличающегося от исходного. Математически
это выражается в наличии множителя~$|\omega|$ в формуле~(\ref{eq_RieszPoten}) (и, следовательно, 
в~(\ref{eq_radonInverse})), который <<подчеркивает>> высокие частоты.

Выход видится в регуляризации~(\ref{eq_radonInverse}) путем умножения~$|\omega|$ на некоторый множитель, 
называемый частотным фильтром (или стабилизирующим множителем)~\cite{TikhonovArsenin}. 
Общая идея регуляризации такова:\linebreak
немного <<испортить>> проекционные данные, подавив влияние 
высоких частот, но при этом обеспечить реконструкцию, близкую к оригиналу. Подроб\-нее о 
регуляризации формулы обращения можно прочитать в монографии~\cite{Herman}.

\section{Вейвлет-вейглет разложение}

Задачу томографии можно решить и с помощью вейвлетов. Пусть~$\phi(t)$ и~$\psi(t)$~--- 
одномерные отцовский и материнский вейвлеты. Определим
\begin{align*}
\phi_{j,k_1,k_2}(x,y) &= 2^{j} \phi\left(2^jx-k_1\right) \phi\left(2^jy-k_2\right)\,;\\
\psi^{[1]}_{j,k_1,k_2}(x,y) &= 2^{j} \phi\left(2^jx-k_1\right) \psi\left(2^jy-k_2\right)\,;
\end{align*}

\noindent
\begin{align*}
\psi^{[2]}_{j,k_1,k_2}(x,y) &= 2^{j} \psi\left(2^jx-k_1\right) \phi\left(2^jy-k_2\right)\,;\\
\psi^{[3]}_{j,k_1,k_2}(x,y) &= 2^{j} \psi\left(2^jx-k_1\right) \psi\left(2^jy-k_2\right)\,.
\end{align*}
Заметим, что параметр масштаба~$j$ контролирует сразу обе функции в произведении. 
Это так называемое тензорное произведение двух одномерных кратномасштабных анализов~\cite{Daub}. 
Тогда набор функций $\left\{ \phi_{j_0,k_1,k_2}, \, \psi^{[\lambda]}_{j,k_1,k_2}, \right\}$, 
где $j$, $k_1$, $k_2\in\mathbb{Z}$, $j\geq j_0$, $\lambda=\overline{1,3}$, 
будет ортонормированным базисом~$\mathbf{L}^2(\mathbb{R}^2)$.

Донохо~\cite{DonohoWVD} решил задачу обращения ряда линейных операторов 
с помощью вейвлетов и родственных им функций специального вида, названных вейглетами (\textit{vaguelettes}). 
Вейглеты для обращения оператора Радона выглядят так:
\begin{multline*}
\xi^{[\lambda]}_{j,k_1,k_2}(s,\,\theta)=
\int\limits_{-\infty}^\infty|\omega|\left(\mathcal{F}_2\psi^{[\lambda]}_{j,k_1,k_2}\right)\times{}\\
{}\times \left( \omega\cos\theta,\,\omega\sin\theta \right)\exp(i2\pi s\omega)\,d\omega\,.
%\label{eq_vagueletteDef}
\end{multline*}
Идея метода реконструкции заключается в том, что вейглет-коэффициенты проекций~$\mathcal{R}f(s,\theta)$ 
равны вейвлет-коэффициентам исходной функции~$f(x,y)$:
\begin{equation*}
\left[\mathcal{R}f,\,\xi^{[\lambda]}_{j,k_1,k_2}\right] = \left\langle f,\,\psi^{[\lambda]}_{j,k_1,k_2}\right\rangle\,,
\end{equation*}
и поэтому
\begin{multline}
f = \sum\limits_{k_1,k_2}\left[\mathcal{R}f,\,\tau_{j_0,k_1,k_2}\right] \phi_{j_0,k_1,k_2} +{}\\
{}+ \sum\limits_{j\geqslant j_0,k_1,k_2,\lambda} \left[\mathcal{R}f,\,\xi^{[\lambda]}_{j,k_1,k_2}\right] \psi^{[\lambda]}_{j,k_1,k_2}\,,
\label{eq_radonInverseWVD}
\end{multline}
где
\begin{multline*}
\tau_{j_0,k_1,k_2}(s,\,\theta)=\int\limits_{-\infty}^\infty|\omega|\left(\mathcal{F}_2\phi_{j_0,k_1,k_2}\right)\times{}\\
{}\times \left( \omega\cos\theta,\,\omega\sin\theta \right)
\exp\left(i2\pi s\omega\right)\,d\omega\,.
\end{multline*}
Регуляризация вейвлет-вейглет формулы~(\ref{eq_radonInverseWVD}) производится с 
помощью мягкой пороговой обработки вейглет-коэффициентов (см.\ разд.~\ref{sect_ThreshholdingTomo}).

\section{Дискретизация и модель шума}

Пусть функция $f(x,y)$ задана на квадрате $[0,\,1]\;\times$\linebreak $\times\;[0,\,1]$. Разбив стороны квадрата на~$N=2^J$ 
равных частей и вычислив значения~$f$ в точках отсчета, получим дискретизованную версию~$f$. Одна\-ко на практике 
нередко бывает удобно нормировать длину отрезка разбиения и рас\-сматривать вместо~$f$ ее <<растянутую>>
версию~--- функцию~$\bar f(Nx,Ny)\;=$\linebreak $={f}(x,y)$. Тогда для вейвлет-коэффициентов функции~$f$ справедливо равенство:
\begin{multline}
\left\langle f,\, \psi^{[\lambda]}_{j,u_1,u_2}\right\rangle ={}\\
{}= \iint f(x,y)\,2^j\overline{\psi^{[\lambda]}\left(2^jx-k_1,\,2^jy-k_2\right)}\,dx\,dy ={}\\
{}=\left(\mathcal{W}^{[\lambda]}f\right)\left(2^{-j},k_1,k_2\right)={}\\
{}=\fr{1}{N}\left(\mathcal{W}^{[\lambda]}\bar f\right)\left(N\,2^{-j},k_1,k_2\right)\,.
\label{eq_contToDiscrCoeff}
\end{multline}
Заметим, что при работе с растянутой функцией растягиваются и вейвлет-функции.
Коэффициенты аппроксимации, получаемые через скалярное произведение~$f$ и~$\phi$, не рассматриваются, 
так как пороговая обработка (см.\ разд.~4) применяется к коэффициентам деталей, которые дают функции~$\psi^{[\lambda]}$. 
Далее везде, кроме разд.~\ref{sect_RegularityTomo}, предполагается, что используются именно коэффициенты 
растянутой версии функции~$f$.

Задача томографии ставится следующим образом. Имеются наблюдения~$X$, состоящие из 
проекций~$\mathcal{R}f$ функции~$f$ и шума~$\epsilon$:
\begin{equation*}%\label{eq_tomoTask}
X=\mathcal{R}f+\epsilon\,, 
\end{equation*}
$\epsilon$~--- независимые нормальные случайные величины с нулевым средним и дисперсией~$\sigma^2$. 
Необходимо восстановить~$f$ по~$X$. При этом при достаточно большом~$N$~\cite{KolaczykArticle}
\begin{equation}
\left.
\begin{array}{rl}
\e \left[X,\,\xi^{[\lambda]}_{j,k_1,k_2}\right] &= \left[\mathcal{R}f,\,\xi^{[\lambda]}_{j,k_1,k_2}\right]\,;\\[9pt]
\D \left[X,\,\xi^{[\lambda]}_{j,k_1,k_2}\right] &= \sigma^2 \left\|\xi^{[\lambda]}_{j,k_1,k_2}\right\|_2^2=\sigma^2_{\lambda;j}\,;\\[9pt]
\left\|\xi^{[\lambda]}_{j,k_1,k_2}\right\|_2^2 &= 2^j \left\|\xi^{[\lambda]}_{0,0,0}\right\|_2^2\,.
\end{array}
\right \}
\label{eq_expctVageuletteCoef}
\end{equation}
Как видим, дисперсия коэффициентов растет вмес\-те с уровнем разложения. Это является следствием 
некорректности задачи томографии. При этом вейглеты не ортогональны, а почти ортогональны. И, 
стало быть, вейг\-лет-коэф\-фи\-ци\-ен\-ты не независимы, а почти независимы. Однако нередко 
этим фактом пренебрегают, так как исследование этой зависимости сопряжено с рядом трудностей. 
И потому порог выбирается исходя из предположения независимости коэффициентов. Как будет видно 
далее, уже только тот факт, что дисперсия растет на каждом уровне, заметно влияет на оценку 
риска пороговой обработки.

\section{Пороговая обработка}\label{sect_ThreshholdingTomo}

Мягкая пороговая функция определяется следующим образом:
\begin{equation*}
\rho(x, T)=
\begin{cases}
x-T & \text{при } x>T\,;\\
x+T & \text{при } x<-T\,;\\
0 & \text{при } |x|\leq T\,.
\end{cases} 
\end{equation*}
Эта функция применяется к вейглет-ко\-эф\-фи\-ци\-ен\-там проекций.

Допустим, что размер изображения равен $N^2\;=$\linebreak $=2^{2J}=L$, разложение идет до уровня~$J-1$. 
В~качестве порога взят порог Колашика~\cite{KolaczykArticle, KolaczykThesis}:
\begin{equation*}
\Tlj = \sqrt{2\ln 2^{2j}} \, 2^{j/2}\sigma  \left\|\xi^{[\lambda]}_{0,0,0}\right\|_2\,.
\end{equation*}
В случае использования оценки дисперсии шума~$\hsig$ порог принимает вид
\begin{equation*}
\hTlj = \sqrt{2\ln 2^{2j}}\,2^{j/2}\hat\sigma \left\|\xi^{[\lambda]}_{0,0,0}\right\|_2\,.
\end{equation*}
Идея выбора такого порога схожа с идеей выбора порога~\textit{VisuShrink} 
$T=\sigma\sqrt{2\ln N}$ (одномерный случай, $N$~--- размер сигнала): при таком пороге 
убирается почти весь шум~\cite{DJideal, DJunkn}. Это следует из того факта, что если $Z_1,\ldots,Z_N$~--- 
независимые стандартные нормальные случайные величины, то
\begin{equation*}
\p\left( \underset{1\leqslant i\leqslant N}{\max}|Z_i| > \sqrt{2\ln N} \right) \rightarrow 0\
\mbox{при}\ N\rightarrow\infty\,.
\end{equation*}

Пороговая обработка идет с уровня~$j_M$, т.\,е.\ в формуле~(\ref{eq_radonInverseWVD}) 
$j_0=j_M$ ($j_M$ определим ниже). Риск~$r(f)$ такой пороговой обработки определяется следующим образом:
\begin{multline}
r(f)=\sum\limits_{j=j_M}^{J-1}\sum_{\lambda=1}^3\sum_{k_1=0}^{2^j-1}
\sum_{k_2=0}^{2^j-1}\e
\left\{ \left\langle f,\,\psi^{[\lambda]}_{j,k_1,k_2}\right\rangle - {}\right.\\
\left.{}-\rho\left(\left[X,\,\xi^{[\lambda]}_{j,k_1,k_2}\right],\,\Tlj\right) \right\}^2\,.
\label{eq_riskEstimDefTomo}
\end{multline}
Так как на практике коэффициенты $\left\langle f,\,\psi^{[\lambda]}_{j,k_1,k_2}\right\rangle$ 
неизвестны, то строят оценку риска. Например,
на основе функции~$\Phi(x,T)$~\cite{Mallat}:
\begin{equation*}
\Phi(x,\Tlj)=
\begin{cases}
x-\slj & \text{при } x\leqslant \Tlj^2\,;\\
\slj+\Tlj^2 & \text{при } x> \Tlj^2\,.
\end{cases} 
\end{equation*}
Оценка риска принимает вид:
\begin{equation*}
\tilde r(f)=\sum\limits_{j=j_M}^{J-1}\sum\limits_{\lambda,k_1,k_2}\Phi\left( 
\left| \left[X,\,\xi^{[\lambda]}_{j,k_1,k_2}\right] \right|^2 ,\,\Tlj\right)\,.
\end{equation*}
Если вместо~$\sigma^2$ используется оценка~$\hsig$, то
\begin{equation*}
\hat r(f)=\sum\limits_{j=j_M}^{J-1}
\sum\limits_{\lambda,k_1,k_2}\hat\Phi\left( 
\left| \left[X,\,\xi^{[\lambda]}_{j,k_1,k_2}\right] \right|^2 ,\,\hTlj\right)\,,
\end{equation*}
где
\begin{equation*}
\hat\Phi(x,\hTlj)=
\begin{cases}
x-\hslj & \text{при } x\leqslant \hTlj^2\,;\\
\hslj+\hTlj^2 & \text{при } x> \hTlj^2\,.
\end{cases} 
\end{equation*}


В работах~\cite{MarkinShestakovConsist, MarkinLimitDistr} рассмотрены асимптотические свойства оценки 
риска пороговой обработки вейв\-лет-коэффициентов в одномерном случае при прямом наблюдении~$f$. 
Показано, что $(\hat r -r)/N^a$ сходится по вероятности к нулю и по распределению к нормальному 
закону при соответствующих~$a$.
Величина~$a$ существенно зависит от свойств оценки~$\hsig$. Однако даже при весьма общих ограничениях 
на моменты~$\hsig$ порядок $a=1$ обеспечивал
сходимость по вероятности к нулю. Ниже будет показано, что в задаче томографии для сходимости 
по вероятности к нулю недостаточно делить на число коэффициентов ($N^2=L$),
т.\,е.\ некоторый аналог закона больших чисел уже не выполнен. Важнейшим фактором 
является то, что~$f$ наблюдается через оператор Радона~$\mathcal{R}$, обратный к 
которому не является непрерывным (т.\,е.\ ограниченным).

\section{Регулярность функции и~вейвлет-коэффициенты}\label{sect_RegularityTomo}

Известно (см., например,~\cite{Mallat}), что если функция~$f(x,y)$ является регулярной по Липшицу 
с параметром $0\leq\alpha\leq 1$, т.\,е.\
\begin{multline*}
\left|f(x_1,y_1)-f(x_2,y_2)\right|\leq{}\\
{}\leq C \left( |x_1-x_2|^2 + |y_1-y_2|^2 \right)^{\alpha/2}
\end{multline*}
для некоторой константы~$C$, не зависящей от $(x_1,y_1)$ и $(x_2,y_2)$, то существует не зависящая от~$J$, 
$j$, $k_1$ и $k_2$ константа~$A$ такая, что
\begin{equation*}
\left(\mathcal{W}^{[\lambda]}f\right)\left(2^{-j},k_1,k_2\right)\leq \fr{A}{2^{j(\alpha+1)}}\,.
\end{equation*}
В отечественной литературе вместо регулярности по Липшицу обычно используется термин <<непрерывность 
по Гёльдеру>>.
С учетом~(\ref{eq_contToDiscrCoeff}) получаем
\begin{equation*}
\left(\mathcal{W}^{[\lambda]}\bar f\right)\left(N\cdot 2^{-j},k_1,k_2\right) \leqslant \frac{A\cdot 2^J}{2^{j(\alpha+1)}}\,.
\end{equation*}

\textit{Предположение о регулярности~$f$: будем полагать, что функция~$f$ является регулярной по Липшицу с 
показателем~$\alpha>0$}. Будем считать, что пороговая обработка ведется с уровня 
$j_M\geq J/(\alpha+1)$. Заметим, что $J-j_M\rightarrow\infty$ при $J\rightarrow\infty$. 
Тогда при определенном выборе вейвлет-базиса~\cite{Mallat} найдется константа~$C_1$ такая, что для 
всех $j\geq j_M$ выполнено
\begin{equation}
\label{eq_WaveletCoeffUpperBoundTomo}
\left(\mathcal{W}^{[\lambda]}\bar f\right)\left(N\cdot 2^{-j},k_1,k_2\right) \leqslant C_1\,,
\end{equation}
причем $C_1$ не зависит от~$N$. Значит, математические ожидания в~(\ref{eq_expctVageuletteCoef}) ограничены.

В работе используется буква~$C$ (с индексом или без индекса) для обозначения констант, причем в 
разных местах~--- вообще говоря, разных.

\section{Асимптотика оценки риска при~известной дисперсии шума}\label{sect_ConsitKnownSTomo}

В работе~\cite{MarkinLimitDistr} показано, что в одномерном случае при известной дисперсии шума 
разность риска и оценки риска при делении на $\sqrt{N}$ сходится по распределению к нормальному 
закону. В задаче томографии уже надо делить не на~$\sqrt{L}$, а на~$L$.

Для краткости введем обозначения:
\begin{align*}
\Yljk &= \left[X,\,\xi^{[\lambda]}_{j,k_1,k_2}\right]\,;\\
\muljk &= \left\langle f,\,\psi^{[\lambda]}_{j,k_1,k_2}\right\rangle\,,
\end{align*}
где $\mathbf{k}=\left(k_1,\,k_2\right)$. Еще раз напомним, что~$\muljk$ рассматриваются 
как коэффициенты растянутой версии дискретизованной функции~$f$. С учетом предположения об 
ортогональности вейглетов получаем
\begin{equation}
\label{eq_YljkNormalDistributed}
\Yljk \sim \mathcal{N}\left(\muljk,\,\slj\right)\,,
\end{equation}
причем $\Yljk$~--- независимые случайные величины.

\medskip

\noindent
\textbf{Теорема 1.}
\textit{Пусть справедливы предположения о регулярности~$f$ из разд.~\ref{sect_RegularityTomo}. 
При известной дисперсии шума в задаче томографии}
\begin{equation*}
\fr{\tilde r(f)-r(f)}{L \sqrt{ b_2 \left( \silz^4 + \siilz^4 + \siiilz^4 \right) }} \Rightarrow \mathcal{N}(0,\,1)
\end{equation*}
\textit{при} $L\rightarrow\infty$, \textit{где} $b_2=2/(2^4-1)=2/15$.

\medskip

\noindent
Д\,о\,к\,а\,з\,а\,т\,е\,л\,ь\,с\,т\,в\,о.\ 
Представим разность оценки риска и самого риска в виде
\begin{multline*}
\tilde r-r=\sumljk\left(\Yljk^2-\slj\right)\indYjklTj +{}\\
\!\!\!\!{}+ \sumljk\left(\slj+\Tlj^2\right)\indYjkgTj - {}
\end{multline*}

\noindent
\begin{multline}
\ \ {}- \sumljk\e\left(\Yljk^2-\slj\right)\indYjklTj -{}\\
{}- \sumljk\e\left(\slj+\Tlj^2\right)\indYjkgTj ={}\\
{}= \sumljk\left(\Yljk^2-\e\Yljk^2\right) -{}\\
{}- \sumljk\left(\Yljk^2-\slj\right)\indYjkgTj + {}\\
{}+ \sumljk\e\left(\Yljk^2-\slj\right)\indYjkgTj +{}\\
{}+ \sumljk\left(\slj+\Tlj^2\right)\indYjkgTj -{}\\
{}- \sumljk\left(\slj+\Tlj^2\right)\prbYjkgTj\,.
\label{eq_diffRiskEstimKnownS}
\end{multline}
Покажем, что при делении на~$L$ первая сумма в~(\ref{eq_diffRiskEstimKnownS}) сходится по распределению
к нормальному закону, а остальные суммы~--- к нулю по вероятности.

Итак, рассмотрим первую сумму в~(\ref{eq_diffRiskEstimKnownS}). Имеем
\begin{multline}
D_L^2 = \D\sumljk\Yljk^2 = \sumljk \D \Yljk^2={}\\
{}=\sum\limits_\lambda \sum_{j=j_M}^{J-1} \sum_{\mathbf{k}}
\left( 2\solj^4 + 4\muljk^2\slj \right)={}\\
{}= \sum\limits_\lambda \sum_{j=j_M}^{J-1}\left\{ 2\cdot 2^{2j}
\solz^4 \cdot 2^{2j} + \sum_{\mathbf{k}}4\muljk^2 2^j \solz^2 \right\} \simeq{}\\
{}\simeq \sum\limits_\lambda \sum_{j=j_M}^{J-1} 2\cdot 2^{4j}
\solz^4 = \sum\limits_\lambda 2\solz^4 \frac{2^{4J} - 2^{4j_M}}{2^4 - 1} \simeq{}\\
{}\simeq \fr{2}{15} 2^{4J} \left( \silz^4 + \siilz^4 + \siiilz^4 \right)\,.
\label{eq_DLknownS}
\end{multline}
Знак~$\simeq$ означает, что при $J\rightarrow\infty$ предел отношения левой и правой частей~(\ref{eq_DLknownS}) 
равен единице. Если выполнено условие Линдеберга, т.\,е.\ для любого~$\delta>0$
\begin{multline}
\fr{1}{D_L^2}\sumljk\e\left\{ \left( \Yljk^2 - \muljk^2 - \slj \right)^2\times{}\right.\\
\left.{}\times \Ik_{\left|\Yljk^2 - \muljk^2 - \slj\right|>\delta D_L} \right\} \rightarrow 0\,,
\label{eq_LindCondTomo}
\end{multline}
то будет иметь место сходимость к нормальному распределению. Так как~$D_L$ имеет порядок~$L$ и чис\-ло слагаемых 
в~(\ref{eq_LindCondTomo}) имеет порядок~$L$, то достаточно показать, что при $L\rightarrow\infty$
\begin{multline*}
\e\left\{ \fr{\left( \Yljk^2 - \muljk^2 - \slj \right)^2}{D_L} \times{}\right.\\
\left.{}\times\Ik_{\left(\Yljk^2 - \muljk^2 - \slj\right)^2/D_L>\delta^2 D_L} \right\} \rightarrow 0\,.
\end{multline*}
А последнее выполнено потому, что у случайных величин вида $\left( \Yljk^2 - \muljk^2 - \slj \right)^2\!/D_L$ 
конечные математические ожидания и $D_L\rightarrow\infty$.

Теперь рассмотрим вторую сумму в~(\ref{eq_diffRiskEstimKnownS}). В силу~(\ref{eq_YljkNormalDistributed}) имеем
\begin{multline*}
\p\left( |\Yljk| > \Tlj \right) < {}\\
{}< \frac{\exp\left( -(\Tlj-\muljk)^2/(2\slj) \right)}{\Tlj} +{}\\
{}+ \frac{\exp\left( -(\Tlj+\muljk)^2/(2\slj) \right)}{\Tlj} \leqslant \fr{C}{2^{5j/2} \sqrt{j} }
\end{multline*}
при $J \rightarrow \infty$ (и, следовательно, $j \rightarrow \infty$). Это можно получить из 
следующей цепочки равенств:
\begin{multline*}
\exp\left( -\fr{(\Tlj-\muljk)^2}{2\slj} \right) = {}\\
{}=\exp\left( -\fr{\Tlj^2}{2\slj} + \fr{2\Tlj\muljk}{2\slj} - \fr{\muljk^2}{2\slj} \right) ={}\\
{}= \exp\left( -\ln 2^{2j} + \fr{\sqrt{2\ln(2^{2j})}\muljk}{2^{j/2}\solz} - 
\fr{\muljk^2}{2\slj} \right) \simeq{}\\
{}\simeq 2^{-2j}\mbox{ при }j \rightarrow \infty\,,
\end{multline*}
так как
\begin{equation*}
\fr{\sqrt{2\ln(2^{2j})}\muljk}{2^{j/2}\solz} \rightarrow 0\quad\text{и}\quad\fr{\muljk^2}{2\slj} \rightarrow 0\,.
\end{equation*}
С помощью неравенств Чебышёва и Коши--Бу\-ня\-ков\-ского получаем для любого $\delta>0$ при $J\rightarrow\infty$
\begin{multline*}
\p\left( \fr{ \left|\sumljk\left(\Yljk^2-\slj\right)\indYjkgTj \right|}{D_L} > \delta \right) \leq{}\\
{}\leq \fr{ \e\left| \sumljk\left(\Yljk^2-\slj\right)\indYjkgTj \right| }{\delta D_L} \leq {}\\
{}\leq \fr{ \sumljk \e\left| \Yljk^2-\slj\right|\indYjkgTj }{\delta D_L} \leq{}\\
{}\leq \fr{ \sumljk \sqrt{ \e\left( \Yljk^2-\slj\right)^2 \p\left( |\Yljk| > \Tlj \right) } }{\delta D_L} 
\leq{}
\end{multline*}

\noindent
\begin{multline*}
{}\leq \fr{1}{\delta D_L}\sumljk 
\left  ( \vphantom{4\cdot 2^j\solz^2\muljk^2  C\cdot 2^{-5j/2} j^{-1/2}}
\left(
\muljk^4 + 2\cdot 2^{2j}\solz^4 + {}\right.\right.\\
\left.\left.{}+4\cdot 2^j\solz^2\muljk^2 
\right) C\cdot 2^{-5j/2} j^{-1/2} 
\right )^{1/2} \rightarrow 0\,.
\end{multline*}
Аналогично проводятся рассуждения для оставшихся сумм в~(\ref{eq_diffRiskEstimKnownS}).~$\square$

\section{Свойства оценки риска при~использовании оценки дисперсии шума}

В работе~\cite{MarkinShestakovConsist} показано, что при достаточно слабых ограничениях 
на моменты оценки дисперсии шума для сходимости разности риска и его оценки к нулю по 
вероятности ее надо нормировать числом вейвлет-коэффициентов, т.\,е.\ порядок знаменателя 
вырастает почти на~1/2. Покажем, что в задаче томографии порядок тоже повышается почти на~1/2, 
но знаменатель уже будет много больше числа коэффициентов.

Введем обозначение
\begin{equation*}
\hslj = 2^j \hsig \left\|\xi^{[\lambda]}_{0,0,0}\right\|_2^2\,.
\end{equation*}

\medskip
\noindent
\textbf{Теорема 2.} \textit{Пусть справедливы предположения о регулярности~$f$. 
Пусть $\hsig$~--- оценка дисперсии, $\e\hsig-\sigma^2=\nu_L$ и 
$\D\hsig=\theta_L=\Obig(L^{-\beta})$, $\nu_L=\osml(1)$, $\beta>0$. 
Тогда при $L\rightarrow\infty$ выполнено}
\begin{equation}
\label{eq_ConsistTomo32}
\fr{\hat r(f)-r(f)}{L^{3/2}} \xrightarrow{\textsf{P}} 0\,.
\end{equation}

\medskip

\noindent
Д\,о\,к\,а\,з\,а\,т\,е\,л\,ь\,с\,т\,в\,о.\
Подобно доказательству теоремы~3 в~\cite{MarkinShestakovConsist} запишем
\begin{equation*}
\hat r-r = S_1 + S_2\,,
\end{equation*}
где
\begin{multline}
S_1 = \sumljk\left(\Yljk^2-\hslj\right) -{}\\
{}- \sumljk\e\left(\Yljk^2-\slj\right)\,; \label{eq_riskSplitSoTomo}
\end{multline}

\vspace*{-6pt}

\noindent
\begin{multline*}
S_2 = - \sumljk\left(\Yljk^2-\hslj\right)\indYjkghTj +{}\\
\!\!{}+ \sumljk\left(\hslj+\hTlj^2\right)\indYjkghTj +{} 
\end{multline*}

\noindent
\begin{multline}
{}+ \sumljk\e\left(\Yljk^2-\slj\right)\indYjkgTj -{}\\
{}- \sumljk\e\left(\slj+\Tlj^2\right)\indYjkgTj\,.
\label{eq_riskSplitStTomo}
\end{multline}
Далее будет показано, что при делении на $L^{3/2}$ и~$S_1$, и~$S_2$ сходятся к нулю по вероятности.

Сначала рассмотрим~$S_1$: по неравенству Чебышёва при любом $\delta>0$
\begin{multline}
\p\left( \fr{|S_1|}{L^{3/2}} > \delta \right) \leq{}\\
{}\leq
\fr{ \e\left( \sumljk \left( \Yljk^2 - \hslj - \e\Yljk^2 + \slj \right) \right)^2 }{\delta^2 L^3} ={}\\
{}= \fr{ \sumljk \e\left( \Yljk^2 - \hslj - \e\Yljk^2 + \slj \right)^2 }{ \delta^2 L^3 } + {}\\
{}+ \fr{1}{\delta^2 L^3}
 \sum \e\left( \Yljk^2 - \hslj - \e\Yljk^2 + \slj \right)\times{}\\
 {}\times \left( \Yljks^2 - \hsljs - \e\Yljks^2 + \sljs \right)\,.
\label{eq_riskSplitUnknSTomo}
\end{multline}
Во второй сумме~(\ref{eq_riskSplitUnknSTomo}) суммирование идет по индексам 
$(\lambda,j,\mathbf{k})\ne(\lambda',j',\mathbf{k}')$. Понятно, что первое слагаемое в~(\ref{eq_riskSplitUnknSTomo}) 
стремится к нулю~--- в сумме всего порядка~$L$ слагаемых, они имеют порядок не выше~$L$ и 
сумма делится на~$L^3$ (напомним, что $L=2^{2J}$).

Рассмотрим одно из слагаемых второй суммы~(\ref{eq_riskSplitUnknSTomo}):
\begin{multline*}
\e\left( \Yljk^2 - \hslj - \e\Yljk^2 + \slj \right) \times{}\\
{}\times\left( \Yljks^2 - \hsljs - \e\Yljks^2 + \sljs \right) = {}\\
{}= \e\Yljk^2\Yljks^2 - \e\Yljk^2\hsljs - \e\Yljk^2 \e\Yljks^2 +{}\\
{}+ \sljs \e\Yljk^2 - 
 \e\hslj\Yljks^2 + \e\hslj\hsljs +{}\\
 {}+ \e\hslj\e\Yljks^2 - \sljs\e\hslj 
- \e\Yljk^2 \e\Yljks^2 +{}\\
{}+ \e\hsljs\e\Yljk^2 + \e\Yljk^2 \e\Yljks^2 - \sljs\e\Yljk^2 
+ {}\\
{}+\slj\e\Yljks^2 - \slj\e\hsljs - \slj\e\Yljks^2 +{}\\
{}+ \slj\sljs = 
 - \cov\left( \hsljs,\,\Yljk^2 \right) -{}\\
 {}- \cov\left( \hslj,\,\Yljks^2 \right) +
 \fr{\slj\sljs}{\sigma^4}\left( \nu_L^2 + \theta_L \right)\,.
\end{multline*}
С учетом того, что $\D\Yljk^2$ имеет порядок~$2^{2j}$, а ковариацию можно оценить по неравенству Коши--Бу\-ня\-ков\-ско\-го, 
получаем, что каждое слагаемое второй суммы~(\ref{eq_riskSplitUnknSTomo}) можно оценить как 
$2^{j+j'}\cdot\osml(1)$. Всего таких слагаемых порядка~$L^2$, а максимальное значение $2^{j+j'}$ 
равно $2^{J-1+J-1}=L/4$. Следовательно, после суммирования получаем, что второе сла\-га\-емое 
в~(\ref{eq_riskSplitUnknSTomo}) оценивается как~$\osml(1)$. Значит, $S_1/L^{3/2}$ сходится к нулю по вероятности.

Для оценки~$S_2$ используем другую модификацию неравенства Чебышёва:
\begin{equation*}
\p\left( \fr{|S_2|}{L^{3/2}} > \delta \right) \leq \fr{\e|S_2|}{\delta L^{3/2}} = 
\fr{\e\left[|S_2|/L^{1/2}\right]}{\delta L}\,.
\end{equation*}
Величину $\e|S_2|$ можно оценить сверху суммой математических ожиданий модулей сумм, входящих в~$S_2$, 
а эти суммы, в свою очередь,~--- суммой математических ожиданий входящих в них слагаемых.

По формуле полной вероятности для некоторого $0<\gamma<1$ получаем
\begin{multline*}
\p\left( |\Yljk| > \hTlj \right)={}\\
{}=\p\left( |\Yljk| > \hTlj \,|\, \hTlj \leqslant (1-\gamma)\solj\sqrt{2\ln 2^{2j}} \right)\times{}\\
{}\times \p\left( \hTlj \leq (1-\gamma)\solj\sqrt{2\ln 2^{2j}} \right) + {}\\
{}+ \p\left( |\Yljk| > \hTlj \,,\, \hTlj > (1-\gamma)\solj\sqrt{2\ln 2^{2j}} \right)\,.
\end{multline*}
В силу свойств~$\hsig$
\begin{multline*}
\p\left( \hTlj \leq (1-\gamma)\solj\sqrt{2\ln 2^{2j}} \right)={}\\
{}=\p\left(\hsig\leqslant (1-\gamma)^2\sigma^2\right)\leq{}\\
{}\leq\p\left(|\hsig-\sigma^2-\nu_L|\geqslant(2\gamma-\gamma^2)\sigma^2+\nu_L\right)\leq{}\\
{}\leq \fr{\D\hsig}{\left((2\gamma-\gamma^2)\sigma^2+\nu_L\right)^2}=\Obig\left(L^{-\beta}\right)
\end{multline*}
для достаточно большого~$L$. Далее
\begin{multline}
\p\left( |\Yljk| > \hTlj \,,\, \hTlj > (1-\gamma)\solj\sqrt{2\ln 2^{2j}} \right) \leq{}\\
{}\leq \p\left( |\Yljk| > (1-\gamma)\solj\sqrt{2\ln 2^{2j}} \right) = {}\\
{}=
\fr{ C }{ 2^{2j(1-\gamma)^2}\cdot 2^{j/2}\sqrt{j} }\,.
\label{eq_prbSplitGammaTomo}
\end{multline}
Теперь оцениваем математические ожидания компонентов сумм из~$S_2$ при делении на~$L^{1/2}$:
\begin{multline*}
\e\left[\fr{\left| \Yljk^2 - \hslj \right|}{L^{1/2}}\indYjkghTj\right] \leq {}\\
{}\leq\sqrt{ \e\left[\fr{ \left(\Yljk^2 - \hslj\right)^2 }{2^{2J}}\right]  \p\left( |\Yljk| > \hTlj \right) } 
\rightarrow 0\,;
\end{multline*}

\vspace*{-6pt}
\noindent
\begin{multline*}
\e\left[\fr{\left| \hslj + \hTlj^2 \right|}{L^{1/2}}\indYjkghTj\right] \leq{}\\
{}\leq 
\left( 2\ln 2^{2j} + 1 \right) 2^{j-J} \times{}\\
{}\times\sqrt{ \e\left(\hslz\right)^2 \p\left( |\Yljk| > \hTlj \right) } \rightarrow 0
\end{multline*}
при $j\geq j_M$ и $J\rightarrow\infty$.
Остальные слагаемые оцениваются аналогично. Итак, $S_2/L^{3/2}$ тоже сходится к нулю по вероятности.~$\square$

\smallskip

Как и в одномерном случае (см.~\cite{MarkinLimitDistr}), порядок знаменателя в~(\ref{eq_ConsistTomo32}) 
можно понизить, введя дополнительные ограничения на~$\nu_L$.

\medskip

\noindent
\textbf{Теорема 3.}
\textit{Пусть справедливы предположения о регулярности~$f$. Пусть $\hsig$~--- 
оценка дисперсии, $\e\hsig-\sigma^2=$\linebreak $=\nu_L=\Obig(L^{-\upsilon})$ и 
$\D\hsig=\theta_L=\Obig(L^{-\beta})$, $\upsilon$, $\beta>0$. Тогда
при любом $a>1/2-c$, $c=\min\left\{1/2, \upsilon, \beta/2\right\}$ и $L\rightarrow\infty$ выполнено
\begin{equation*}
\fr{\hat r(f)-r(f)}{L^{a+1}} \xrightarrow{\textsf{P}} 0\,.
\end{equation*}}
\medskip

\noindent
Д\,о\,к\,а\,з\,а\,т\,е\,л\,ь\,с\,т\,в\,о.
Заметим, что $0<c\leq 1/2$ и, стало быть, $a>0$. Так же, как и в доказательстве теоремы~2, 
разобьем $\hat r - r$ на те же суммы~$S_1$ и~$S_2$ (см.\ формулы~(\ref{eq_riskSplitSoTomo})
и~(\ref{eq_riskSplitStTomo})), только~$S_1$ запишем в виде
\begin{multline*}
S_1 = \sumljk\left(\Yljk^2-\e\Yljk^2\right) - \sumljk \left(\hslj-\slj\right) = {}\\
{}= \sumljk\left(\Yljk^2-\e\Yljk^2\right) - {}\\
{}-\sum\limits_\lambda \sum_j 2^{2j}\,2^j \left(\hslz-\solz^2\right)\,.
%\label{eq_riskSplitSoLimTomo}
\end{multline*}
Первая сумма при делении на~$L$ сходится по распределению к нормальному закону 
(см.\ разд.~\ref{sect_ConsitKnownSTomo}) и, следовательно, сходится по вероятности к нулю при делении 
на~$L^{a+1}$, где $a>0$. Вторая сумма пред\-став\-ля\-ет собой произведение 
$\left(\hsig-\sigma^2\right)$ и множителя, имеющего порядок $2^{3J}=L^{3/2}$. Легко видеть, что
\begin{equation*}
\fr{L^{3/2}\left(\hsig-\sigma^2\right)}{L^{a+1}} \xrightarrow{\textsf{P}} 0
\end{equation*}
при указанных в формулировке теоремы ограничениях на~$a$.

Покажем теперь, что $S_2/L^{a+1}$ сходится к нулю по вероятности. 
Обозначим $\varkappa = a-1/2+c>$\linebreak $>\;0$. В теореме~2 есть оценки для вероятности 
$\p\left( |\Yljk| > \hTlj \right)$:
\begin{multline}
\p\left( |\Yljk| > \hTlj \right) = {}\\
{}=\max\left\{ \fr{C_1}{2^{2J\beta}},\,\fr{C_2}{2^{2j(1-\gamma)^2}\cdot 2^{j/2}\sqrt{j}} \right\} 
\label{eq_ProbYghTOrdersTomo}
\end{multline}
для некоторого $0<\gamma<1$. При $J\rightarrow\infty$ имеем
\begin{multline}
\label{eq_restEstimConsistTomo1}
\fr{\e \left(\Yljk^2\right)^2 C_1/2^{2J\beta}}{L^{2a}} \leq \fr{C_3\cdot 2^{2j}
\cdot 2^{-2j\beta}}{2^{2J(1-2c+2\varkappa)}} ={}\\
{}= \fr{C_3\cdot 2^{2j}\cdot2^{2J\min\left\{1, 2\upsilon, \beta\right\}}}{2^{2J}\cdot 2^{2J\beta}\cdot 2^{4J\varkappa}}  \rightarrow 0,
\end{multline}

\columnbreak 
%\vspace*{-6pt}

\noindent
\begin{multline}
\fr{\e \left(\Yljk^2\right)^2 C_2\cdot 2^{-2j(1-\gamma)^2}\cdot 2^{-j/2}/\sqrt{j}}{L^{2a}} \leq {}\\
{}\leq
\fr{C_4\cdot 2^{2j}\cdot 2^{2J\min\left\{1, 2\upsilon, \beta\right\}}}{2^{2j(1-\gamma)^2+j/2}\cdot 2^{2J}\cdot 2^{4J\varkappa}\sqrt{j}} \rightarrow 0
\label{eq_restEstimConsistTomo2}
\end{multline}
для достаточно малого~$\gamma$. Отсюда имеем для произвольного $\delta>0$
\begin{multline*}
\p\left(\fr{\sumljk \Yljk^2 \indYjkghTj }{L^{a+1}}>\delta\right) \leq{}\\
{}\leq
\fr{\sumljk \e \left[\Yljk^2/L^a\right] \indYjkghTj  }{\delta L} \rightarrow 0
\end{multline*}
при $J\rightarrow\infty$ в силу неравенств Чебышёва и Коши--Бу\-ня\-ков\-ско\-го. Оценки для суммы 
с членами вида $\hslj \indYjkghTj$ получаются аналогично. А для сумм, в которые входят $\indYjkgTj$, 
оценки получены в теореме~1.~$\square$

Можно сформулировать и доказать теорему сходимости по распределению к нетривиальному пределу.

\medskip

\noindent
\textbf{Теорема 4.} 
\textit{Пусть справедливы предположения о регулярности~$f$. 
Пусть $\hsig$~--- оценка дисперсии, 
$\e\hsig-\sigma^2=\nu_L=\Obig(L^{-\upsilon})$ и 
$\D\hsig=\theta_L=\Obig(L^{-\beta})$, $\upsilon>0$, $\beta>1/2$. 
Пусть $\hsig$ не зависит от $\Yljk$ и $\sqrt{L}\left( \hsig - \sigma^2 \right) 
\Rightarrow \mathcal{N}\left(0,\,\Sigma^2\right)$ при $L\rightarrow\infty$, тогда
\begin{multline*}
\fr{\hat r(f)-r(f)}{ L \sqrt{ b_2 \left( \silz^4 + \siilz^4 + \siiilz^4 \right) } } \Rightarrow{}\\
{}\Rightarrow \mathcal{N}\left( 0, 1+\frac{ \left( \silz^2 + \siilz^2 + \siiilz^2 \right)^2 \Sigma^2 }{ d_2 \,\sigma^4 \left( \silz^4 + \siilz^4 + \siiilz^4 \right) } \right)\,,
\end{multline*}
где $b_2=2/(2^4-1)=2/15$, $d_2 = (2(2^3-1)^2)/(2^4-1)=$\linebreak $=98/15$.}

\medskip

\noindent
Д\,о\,к\,а\,з\,а\,т\,е\,л\,ь\,с\,т\,в\,о.
В теореме~3 было существенным наличие~$\varkappa>0$, которое давало сходимость к нулю 
в~(\ref{eq_restEstimConsistTomo1}) (в~(\ref{eq_restEstimConsistTomo2}) это несущественно). 
Сейчас же $\varkappa=0$, поэтому доказательство необходимо изменить.

Оценим $S_2$ более тонко. Имеем
\vspace*{-9pt}

\noindent
\begin{multline*}
\Yljk^2 \indYjkghTj - \e \Yljk^2 \indYjkgTj = {}\\[3pt]
{}= \Yljk^2 \indYjkghTj - \Yljk^2 \indYjkgTj +{}\\[3pt]
{}+ \Yljk^2 \indYjkgTj - \e \Yljk^2 \indYjkgTj\,.
\vspace*{-3pt}
\end{multline*}
\vspace*{-18pt}

\pagebreak

Вопрос о двух последних слагаемых решен в теореме~1. Рассмотрим два первых:
\begin{multline*}
\e \left| \Yljk^2 \indYjkghTj - \Yljk^2 \indYjkgTj \right| = {}\\
{}=\e \Yljk^2 \Ik_{\Tlj<|\Yljk|\leqslant\hTlj} +{}\\
{}+ \e \Yljk^2 \Ik_{\hTlj<|\Yljk|\leqslant\Tlj}\,.
\end{multline*}
При этом
\begin{multline*}
\e \Yljk^2 \Ik_{\Tlj<|\Yljk|\leqslant\hTlj} \leq \e \hTlj^2 \indYjkgTj \leq{}\\
{}\leq \sqrt{\fr{C\cdot j^2\cdot 2^{2j}}{2^{2j+j/2}\sqrt{j}}}\rightarrow 0,\quad J\rightarrow\infty\,,
\end{multline*}

%\vspace*{-3pt}

\noindent
и

%\vspace*{-3pt}
\noindent
\begin{multline}
\e \Yljk^2 \Ik_{\hTlj<|\Yljk|\leq\Tlj} \leq {}\\
{}\leq \Tlj^2 \e \Ik_{\hTlj<|\Yljk|\leq 
 \Tlj} \leq{}\\
{}\leq C j\cdot 2^{j}
\e \indYjkghTj\,.
\label{eq_FineRestEstimTomo}
\end{multline}
С учетом~(\ref{eq_ProbYghTOrdersTomo}) получаем, что
\begin{equation*}
\e \Yljk^2 \Ik_{\hTlj<|\Yljk|\leqslant\Tlj} \rightarrow 0
\end{equation*}
при $J\rightarrow\infty$ и $\beta>1/2$. Отметим, что, в отличие от работы~\cite{MarkinLimitDistr}, 
требование на~$\beta$ повысилось (там требовалось только $\beta>0$). Это является следствием роста дисперсии с 
ростом~$j$, которое выражается в наличии множителя~$2^j$ в~(\ref{eq_FineRestEstimTomo}). 
Аналогично получаем соотношения для~$\hTlj$:
\begin{multline*}
\e \hTlj^2 \Ik_{\Tlj<|\Yljk|\leq\hTlj} \leq \e \hTlj^2 \indYjkgTj \leq{}\\
{}\leq \sqrt{ \fr{ C j^2 \cdot 2^{2j} }{ 2^{2j+j/2}\sqrt{j} } } \rightarrow 0\,;
\end{multline*}

\vspace*{-12pt}

\noindent
\begin{multline*}
\e \hTlj^2 \Ik_{\hTlj<|\Yljk|\leq\Tlj} \leq{}\\
{}\leq \Tlj^2 \e\Ik_{\hTlj<|\Yljk|\leq\Tlj} \rightarrow 0\,.
\end{multline*}
Для $\hslj$ заметим, что $\hslj\leqslant\hTlj^2$. После применения неравенства Чебышёва получим, что 
$S_2/L$ сходится к нулю по вероятности.

В~$S_1$ оба слагаемых сходятся по распределению к нормальному закону и при этом независимы. 
Поэтому их сумма тоже сходится по распределению к нормальному закону. Осталось убедиться в 
правильности параметров. Имеем
\begin{multline*}
\sumljk \left(\hslj-\slj\right) = \sum\limits_\lambda \sum_j 2^{2j}\cdot 2^j \left(\hslz-\solz^2\right) = {}\\
{}= \left( \left\|\xi^{[1]}_{0,0,0}\right\|_2^2 + \left\|\xi^{[2]}_{0,0,0}\right\|_2^2 + 
\left\|\xi^{[3]}_{0,0,0}\right\|_2^2 \right)\times{}\\
{}\times \fr{2^{3J}-2^{3j_M}}{2^3-1} \left(\hsig-\sigma^2\right) = {}
\end{multline*}

\noindent
$$%\begin{multline*}
{}= \fr{  \silz^2 + \siilz^2 + \siiilz^2 }{ \sigma^2 }\, \fr{2^{3J}-2^{3j_M}}{7} \left(\hsig-\sigma^2\right)\,.\hfil\square
$$%\end{multline*}

%\columnbreak
\medskip

\noindent
\textbf{Замечание}. 
Если функция $f$ регулярная с параметром $\alpha\geq 1/4$, а $j_M\geq 4J/5$, то можно ослабить требования 
на~$\hsig$. Достаточно потребовать только состоятельность, асимптотическую нормальность и независимость от~$\Yljk$.

\smallskip

В теоремах~2 и~3 при оценке $\p\left( |\Yljk| > \hTlj \right)$ использовалось число $0<\gamma<1$. 
Можно заменить~$\gamma$ бесконечно малой последовательностью~$\gamma_L$, которая не испортит порядок знаменателя 
в~(\ref{eq_prbSplitGammaTomo}).

По формуле полной вероятности для любого $\delta>0$
\begin{multline}
\label{eq_sumOfIndicTomo}
\p\left( \sumljk\indYjkghTj > \delta \right) ={}\\
{}= \p\left( \hTlj \leqslant \left( 1-\gamma_L \right)\solj\sqrt{2\ln 2^{2j}} \right) \times{} \\
{}\times \p\left( \sumljk \indYjkghTj>\delta\,|\,\hTlj \leqslant{}\right.\\
\left.{}\vphantom{\sumljk\indYjkghTj}\leq \left(1-\gamma_L\right) \solj\sqrt{2\ln 2^{2j}} \right) +{} \\
{}+ \p\left(\sumljk\indYjkghTj>\delta\,,\right. \\
\left.\vphantom{\sumljk\indYjkghTj}\hTlj > \left( 1-\gamma_L \right) \solj\sqrt{2\ln 2^{2j}}\right)\,,
\end{multline}
где
$\gamma_L = 1/J$.
При таком $\gamma_L$ получаем
\begin{multline*}
\p\left( |\Yljk| > (1-\gamma_L)\solj\sqrt{2\ln 2^{2j}} \right) = {}\\
{}=\fr{C}{ 2^{2j(1-\gamma_L)^2}\cdot 2^{j/2}\sqrt{j} } \leq \fr{C_1}{ 2^{2J} \sqrt{j} }
\end{multline*}
в силу выбора $j_M$ и того, что
\begin{equation*}
2^{2j\left(1-\gamma_L\right)^2} = 2^{2j\left( 1-2/J + 1/J^2 \right)} > 2^{2j-4}\,.
\end{equation*}
По неравенству Чебышёва
\begin{multline*}
\p\left(\sumljk\indYjkghTj>\delta\,, \right.\\
\left. \vphantom{\sumljk\indYjkghTj}\hTlj > \left( 1-\gamma_L \right) \solj\sqrt{2\ln 2^{2j}}\right) \leq{}\\
{}\leq \p\left( \sumljk \Ik_{ |\Yljk| > (1-\gamma_L)\solj\sqrt{2\ln 2^{2j}} } > \delta \right) \leq{}
\end{multline*}

\noindent
\begin{multline*}
{}\leq \fr{ \sumljk \p\left( |\Yljk| > (1-\gamma_L)\solj\sqrt{2\ln 2^{2j}} \right) }{\delta} = {}\\
{}=\Obig\left( \fr{1}{\sqrt{j}} \right)\,.
\end{multline*}

Используя свойство асимптотической нормальности~$\hsig$, можно для любого $\delta'>0$ оценить
\begin{equation}
\label{eq_hatTdevProbAsympTomo}
\p\left( \hTlj \leq (1-\gamma_L)\solj\sqrt{2\ln 2^{2j}} \right)< \delta'\,,
\end{equation}
причем отметим, что~$\delta$ здесь фиксировано, а~$\delta'$ можно делать произвольно малым. Имеем
\begin{multline*}
%\label{eq_hatTdevProbAsymp}
\p\left(\hTlj\leqslant(1-\gamma_L)\solj\sqrt{2\ln 2^{2j}}\right) ={}\\
{}= \p\left(\hsig\leqslant(1-\gamma_L)^2\sigma^2\right)={}\\
{}= \p\left( \left(\hsig-\sigma^2\right) \leqslant \sigma^2\left(-2\gamma_L+\gamma_L^2\right) \right) ={}\\
{}= \p\left( \sqrt{L}\left(\hsig-\sigma^2\right) \leqslant -\fr{\sqrt{L}\sigma^2(2J-1)}{J^2} \right)\,.
\end{multline*}
Для произвольного~$\delta'>0$ найдется $J_0$ ($L_0=2^{2J_0}$) такое, что
\begin{equation*}
F_\Sigma\left( -\fr{\sqrt{L_0}\sigma^2(2J_0-1 )}{J_0^2} \right) < \fr{\delta'}{2}\,,
\end{equation*}
где $F_\Sigma$~--- функция распределения нормального закона с нулевым средним и дисперсией~$\Sigma^2$. 
При этом для любого $J\geq J_0$
\begin{multline*}
\p\left( \sqrt{L}\left(\hsig-\sigma^2\right) \leq -\fr{\sqrt{L}\sigma^2(J -1 )}{J^2} \right) \leq{} \\
{}\leq \p\left( \sqrt{L}\left(\hsig-\sigma^2\right) \leq -\fr{\sqrt{L_0}\sigma^2(2J_0 -1 )}{J_0^2}  \right)\,.
\end{multline*}
В силу асимптотической нормальности~$\hsig$ и непрерывности~$F_\Sigma$ для этого же~$\delta'$ 
найдется~$J_1$ $\left(L_1=2^{2J_1}\right)$ такое, что для любого $J\geq J_1$
\begin{equation*}
\left| \p\left( \sqrt{L_1}\left(\hsig-\sigma^2\right) \leq x \right) - F_\Sigma(x)\right| < \fr{\delta'}{2}\,,
\end{equation*}
причем $J_1$ не зависит от~$x$. Возьмем $x_0 =$\linebreak $= -\sqrt{L_0}\sigma^2(2J_0-1)/J_0^2$ и 
$J_2 = \max\{J_0,J_1\}$. Для любого $J\geq J_2$ имеем
\begin{equation*}
\p\left( \sqrt{L}\left(\hsig-\sigma^2\right) \leq x_0 \right) < \delta'\,,
\end{equation*}
а значит, справедливо~(\ref{eq_hatTdevProbAsympTomo}).

Получаем, что сумма индикаторов в~(\ref{eq_sumOfIndicTomo}) сходится к нулю по вероятности:
\begin{equation*}
%\label{eq_sumIndConsisthTTomo}
\p\left( \sumljk\indYjkghTj > \delta \right) \rightarrow 0 \mbox{ при }J\rightarrow\infty\,.
\end{equation*}
Для суммы индикаторов с неслучайным порогом аналогично получаем
\begin{equation*}%\label{eq_sumIndConsistTTomo}
\p\left( \sumljk\indYjkgTj > \delta \right) \rightarrow 0\,.
\end{equation*}
Далее воспользуемся дискретной версией неравенства Коши--Буняковского:
\begin{multline*}
\fr{ \sumljk \Yljk^2\indYjkghTj }{ L } \leq{}\\
{}\leq \sqrt{ \fr{\sumljk \Yljk^4/L}{L} \, \sumljk \indYjkghTj } \,\xrightarrow{\textsf{P}} 0\,,
\end{multline*}
так как $\e\left[ \Yljk^4/L \right]$ ограничено,
\begin{equation*}
\fr{ \sumljk \hTlj^2 \indYjkghTj }{L} \leq \fr{ \sumljk \Yljk^2 \indYjkghTj }{L} \xrightarrow{\mathsf{P}} 0
\end{equation*}
и
\begin{equation*}
\hslj\indYjkghTj \leqslant \hTlj^2\indYjkghTj\,.
\end{equation*}
Оценки для слагаемых с $\indYjkgTj$ получены в теореме~1.

\medskip

\noindent
\textbf{Замечание}. 
Всюду выше в этом разделе предполагалось, что пороговая обработка и суммирование в выражении для 
риска~(\ref{eq_riskEstimDefTomo}) ведутся с уровня~$j_M$, причем $j_M\rightarrow\infty$ при 
$J\rightarrow\infty$. Однако если ввести дополнительные ограничения на регулярность~$f$, 
то можно вести пороговую обработку и суммирование с уровня $j_0\nrightarrow\infty$. Если 
$j_M=J/(\alpha+1)$, то для коэффициентов, соответствующих $j<j_M$, неравенство~(\ref{eq_WaveletCoeffUpperBoundTomo}), 
вообще говоря, не выполнено. Оценим вклад больших коэффициентов в оценку риска:
\begin{multline*}
L^{-1} \sum\limits_{j=j_0}^{j_M-1}\sum\limits_{\lambda,\mathbf{k}} \left\{\left|\Yljk^2-\hslj\right|\indYjklhTj +{}\right.\\
\left.{}+ \left(\hslj+\hTlj^2\right)\indYjkghTj \right\} \leq{}\\
{}\leq L^{-1} \sum\limits_{j=j_0}^{j_M-1}\sum\limits_{\lambda,\mathbf{k}} \left\{ \left(\hslj+\hTlj^2\right) +
 \left(\hslj+\hTlj^2\right) \right\} \xrightarrow{\mathsf{P}}{}\\
\xrightarrow{\mathsf{P}} {} 0
\end{multline*}
в силу состоятельности~$\hsig$ и того, что

\noindent
\begin{multline*}
L^{-1}\left\{\sum\limits_{j=j_0}^{j_M-1}j2^j\cdot2^{2j}\right\} \leq 2^{-2J}
\left\{j_M\sum\limits_{j=j_0}^{j_M-1}2^{3j}\right\} \simeq{}\\
{}\simeq 2^{2J}\cdot j_M\cdot2^{3j_M}\rightarrow 0
\end{multline*}
при $J\rightarrow\infty$, если $3j_M<2J$, т.\,е.\ 
$\alpha>1/2$. Слагаемые риска оцениваются аналогично. Итак, 
при $\alpha>1/2$ суммирование в~(\ref{eq_riskEstimDefTomo}) можно начинать с произвольного~$j_0$.


{\small\frenchspacing
{%\baselineskip=10.8pt
\addcontentsline{toc}{section}{Литература}
\begin{thebibliography}{99}

\bibitem{Natterer} %1
\Au{Наттерер Ф.} 
Математические аспекты компьютерной томографии.~--- М.: Мир, 1990.

\bibitem{TikhonovArsenin}  %2
\Au{Тихонов А.\,Н., Арсенин В.\,Я.} 
Методы решения некорректных задач.~--- М.: Наука, 1979.

\bibitem{Herman}  %3
\Au{Хермен Г.} 
Восстановление изображений по проекциям: основы реконструктивной томографии.~--- М.: Наука, 1983.

\bibitem{Daub}  %4
\Au{Добеши И.} 
Десять лекций по вейвлетам.~--- Ижевск: НИЦ <<Регулярная и хаотическая динамика>>, 2001.

\bibitem{DonohoWVD} 
\Au{Donoho D.\,L.} 
Nonlinear solution of linear inverse problems by wavelet-vaguelette decomposition~// 
Appl. Comput. Harmonic Anal., 1995. Vol.~2. P.~101--126.

\bibitem{KolaczykArticle}  %7
\Au{Kolaczyk E.\,D.} 
A wavelet shrinkage approach to tomographic image reconstruction~// J. Amer. Statistical Association, 1996. 
Vol.~91. No.\,435. P.~1079--1090.

\bibitem{KolaczykThesis} %6
\textit{Kolaczyk E.\,D.} 
Wavelet methods for the inversion of certain homogeneous linear operators in the presence of noisy data.  Ph.D.\ 
Thesis, 1994.

\bibitem{DJideal} 
\textit{Donoho D.\,L., Johnstone I.\,M.} 
Ideal spatial adaptation via wavelet shrinkage~// Biometrika, 1994. Vol.~81. No.\,3. P.~425--455.

\bibitem{DJunkn}  %9
\textit{Donoho D.\,L., Johnstone I.\,M.} 
Adapting to unknown smoothness via wavelet shrinkage~// J. Amer.\ Statistical Association, 1995. Vol.~90. P.~1200--1224.

\bibitem{Mallat} %10
\Au{Mallat S.} 
A wavelet tour of signal processing.~--- Academic Press, 1999.


\bibitem{MarkinLimitDistr}  %11
\Au{Маркин А.\,В.} 
Предельное распределение оценки риска при пороговой обработке вейвлет-ко\-эф\-фи\-ци\-ен\-тов~// 
Информатика и её применения, 2009. Т.~3. Вып.~4. С.~57--63.

\label{end\stat}

\bibitem{MarkinShestakovConsist}  %12
\Au{Маркин А.\,В., Шестаков О.\,В.} 
О состоятельности оценки риска при пороговой обработке вейвлет-ко\-эф\-фи\-ци\-ен\-тов~// Вестник Московского университета. 
Сер.~15. Вычислительная математика и кибернетика, 2010. №\,1. С.~26--33.


 \end{thebibliography}
}
}

\end{multicols}    %pdf %4
\def\stat{morozov}

\def\tit{АНАЛИЗ  СЕТЕВОГО  ПРОТОКОЛА С~ОБЩЕЙ
ФУНКЦИЕЙ   РАСШИРЕНИЯ ОКНА   ПЕРЕДАЧИ СООБЩЕНИЯ ПРИ~КОНФЛИКТАХ$^*$}

\def\titkol{Анализ  сетевого  протокола с общей
функцией   расширения окна   передачи сообщения при конфликтах}

\def\autkol{А.~Лукьяненко, Е.~Морозов,  А.~Гуртов}
\def\aut{А.~Лукьяненко$^1$, Е.~Морозов$^2$,  А.~Гуртов$^3$}

\titel{\tit}{\aut}{\autkol}{\titkol}

{\renewcommand{\thefootnote}{\fnsymbol{footnote}}\footnotetext[1]
{Работа поддерживается грантом РФФИ, 10-07-00017.}}

\renewcommand{\thefootnote}{\arabic{footnote}}
\footnotetext[1]{Helsinki Institute for Information Technology HIIT, Aalto, Finland, firstname.secondname@hiit.fi}
\footnotetext[2]{Институт прикладных математических исследований КарНЦ РАН, emorozov@krc.karelia.ru}
\footnotetext[3]{Helsinki Institute for Information Technology HIIT, Aalto, Finland, gurtov@hiit.fi}

%\newcommand{\todo}[1]{{\bf\color{blue} TODO: #1}}


\Abst{Исследован класс сетевых протоколов
контроля несущей среды, где окно передачи  сообщения является
призвольной возрастающей функцией числа  конфликтов сообщения,
посланного с данной   станции. В общепринятых предположениях,
накладываемых  на свойства сети,   исследована функция протокола,
определяемая правилом расширения окна в зависимости от числа
конфликтов. Найдено выражение для функции протокола, обеспечивающей
минимальное среднее время передачи сообщения. Проведен
асимптотический анализ протокола при неограниченно растущем числе
станций.  Рассмотрены протоколы как с неограниченным, так и 
с ограниченным числом  попыток  передачи сообщения. Предложена модель
распределения доступа к каналу в непрерывном времени, допускающая
слоты различной длины.}

\KW{передача данных; оценка производительности;
моделирование протокола; доступ к каналу}


     \vskip 18pt plus 9pt minus 6pt

      \thispagestyle{headings}

      \begin{multicols}{2}

      \label{st\stat}
      
            
\section{Введение}

В данной работе  рассмотрен  сетевой прото\-кол, обес\-пе\-чи\-ва\-ющий
отсрочку передачи данных, %\linebreak 
вызванную конфликтом   в сети с
кол\-лек\-тивным доступом с контролем несущей и обнаружением/устранением
конфликтов~\cite{METCALFE}. Далее для этого протокола будет
использовано  обозначение BP (backoff protocol). Протокол BP служит
механизмом %\todo{для?}
успешной передачи информации и приводит к по\-стро\-ению легко
развертываемых и недорогих локальных сетей (ЛС). По\-стро\-ение ЛС,
использующее прямые связи всех станций друг с другом, весьма дорого,
и добавление каждой  новой станции ведет к стремительному увеличению
затрат и сложности ЛС. Альтернативное решение, состоящее в том, что
сообщение передается не напрямую, требует  уверенности в том, что
промежуточная станция, используемая для передачи,  не уйдет из сети.
В~ЛС центральным элементом является  {\it передающая среда}, которую
назовем системой передачи или просто {\it системой}. Когда система
развернута, новые узлы (рабочие станции, терминалы, принтеры,
серверы) просто подключаются к ней и могут мгновенно начинать
функционировать.
   Однако такая простота и быстрота  развертывания и организации ЛС  создает технические
трудности. Когда некоторая станция начинает передавать сигнал,
возможна ситуация, при которой  другая станция, увидев систему
пус\-той, также  начинает  передачу (поскольку она еще не обнаружила
ранее посланный в систему сигнал). Тогда данные обеих станций
перекрываются и, как следствие, разрушаются, т.\,е.\ происходит {\it
конфликт}. Существуют методы избавления от подобных конфликтов.
Например, при использовании радиосигнала или оптоволоконной среды
сигналы  разных станций можно передавать на различающихся час\-то\-тах,
однако это приведет к удорожанию технологии. К~тому же число
различных доступных частот обычно ограничено.

После того как станция узнает о том, что ее данные разрушены, она
(как правило) инициирует повторную попытку передачи. Но если такие
попытки будут предприниматься  через детерминированные промежутки
времени, то  те же сообщения столкнутся вновь. Именно для решения
этой проб\-ле\-мы и предназначен BP, в котором окно передачи растет
вместе с числом неудачных попыток послать сообщение. Так называемый
{\it константный} BP был впервые использован как часть протокола
Aloha~\cite {ABRAMSON85}. Позднее его модификацию (обрезанный
бинарный экспоненциальный BP) успешно применили в сети Ethernet~\cite {METCALFE, SHOCH}. 
Несмотря на то, что сейчас Ethernet ушел от
использования BP, алгоритм расширения окна все еще активно
используется в различных сетевых протоколах, в частности в
беспроводных сетях (например, IEEE 802.11~\cite {IEEE80211}) и
транспортных протоколах таких, как SCTP и TFRC.

Несмотря на свою относительно  долгую историю, важность  и простоту,
BP долгое время не поддавался удовлетворительному теоретическому
анализу. В этой связи укажем на  работу~\cite {HASTAD}, содержащую
подробную предысторию вопроса, анализ
устойчивости, а также детальное исследование важных
частных случаев BP. Существует множество работ, связанных с анализом
BP, однако, как правило, используемые в них предпосылки являются
слишком ограничительными (например, рассматривается  бесконечное
число станций $N=\infty$ или функция расширения окна~$f$ имеет
специальную форму)~[6--8].  В~данной работе
исследование проводится при  конечном числе станций $N$, а затем
рассматривается асимптотика при  $N\to \infty$. В~действительности,
 рассматривается  целый класс протоколов описанного
выше типа, где каждый протокол специфицируется выбором конкретной
(возрастающей) функции~$f$.

\section{Описание протокола}

Рассмотрим~ $N$ станций, подключенных к передающей среде, причем
предполагается, что в каж\-дой станции постоянно имеется очередь
сообщений, готовых к отправке. Считается, что все\linebreak станции идентичны
(в статистическом смысле) и работают независимо друг от друга.
Рассмотрим работу одной такой (произвольной) станции более подробно.
По алгоритму BP выбирается первое сообщение из очереди и в
специальном счетчике, который  называется  BC, устанавливается
начальное значение $i=0$. В~любой момент времени значение BC равно
числу   последовательных  безуспешных попыток отправки, накопленных
станцией к данному моменту времени.
Если в некоторый момент значение BC равно~$i$, то  для
осуществления следующей передачи BP строит   {\it окно отсрочки
передачи сообщения} $W(i)= [1,f(i)]$,
 где $f$~--- некоторая  заданная монотонно возрастающая целочисленная
 функция, $i\ge 0$, причем  $f(0)\ge 1$.
  (Случай произвольной функции  $f$, требующий  незначительного изменения
  модели,  представлен в~\cite{LUKYA}.) Таким образом,  окно
  расширяется  с увеличением   значения BC.

 Естественной  единицей времени  при сетевом анализе
является  {\it слот}, в  течение которого может произойти одно {\it
элементарное событие}, скажем  столкновение сообщений, передача
пакета и~т.\,д.  Вообще говоря, величина реального (физического)
времени, соответствующего различным слотам, не является постоянной.
 Основные    результаты данной работы получены в терминах слотов,
а  в  разд.~5   показано, как переформулировать модель (и
полученные результаты) в терминах реального времени. Таким образом,
величина~$f(i)$ равна числу слотов, в течение одного из  которых
произойдет сле\-ду\-ющая попытка  отправки сообщения (при условии, что
уже произошло ровно $i$~конфликтов).
 Внутри  окна $W(i)$ слот отправки сообщения~$D_i$
выбирается {\it равномерно}. Таким образом, $D_i$ является временем
задержки (в слотах) до осуществления следующей попытки отправки
сообщения, если это сообщение уже имело $i\ge 0$ неудачных попыток.
Если сообщение отправлено успешно, то значение BC полагается равным
нулю. Если же (при значении BC равном~$i$) происходит конфликт, то
значение BC увеличивается до $i+1$. Для следующей  отправки
используется окно  $W(i+1)=[1,\, f(i+1)]$ и~т.\,д. В~дальнейшем
рассматриваются  {\it ограниченный} и {\it неограниченный}
протоколы. В~ограниченном протоколе задана верхняя граница  $M$
значений BC: если сообщение не удалось отправить в течение~$M$
попыток, то оно выбрасывается  из очереди. В неограниченном
протоколе $M=\infty$. Далее подробно исследуется неограниченный
протокол. Для ограниченного протокола достаточно использовать
очевидную модификацию анализа, применяемого для неограниченного
протокола, и поэтому в данном случае   приведены лишь окончательные
результаты.

Размер  любого  слота ограничен снизу величиной RTT (round trip
time). Это необходимо,  чтобы станции получали информацию  об отсутствии столкновений
 в течение одного слота, поскольку  RTT~--- это величина
времени, за которое  сигнал об отправке сообщения оповестит  сеть и
вернется  на станцию  отправки. Поэтому исключена  ситуация, когда
одна станция   ведет отправку сообщения в течение нескольких слотов,
в то время  как  другая  станция уменьшила число  {\it пустых
слотов} до очередной попытки отправки сообщения.  (Пустым слотом для
данной станции является любой слот внутри окна передачи, когда
станция не пытается передавать сообщение.)
 Кроме того, считается, что время передачи сообщения занимает один слот.
 Это ограничение основано в первую очередь на поведении сети Wi-Fi (протокол
IEEE802.11). Например, для  протокола IEEE802.11 время
распространения сигнала в  сети равно 1~мкс (т.\,е.\  $\mathrm{RTT}= 2$~мкс), а
пустой слот равен 50 мкс~\cite{IEEE80211, VISHN}.

%{\bf может быть где-то  здесь сослаться на работы Ляхова ? -- Сослался на книгу}


Для сети, где станции продолжают отсчитывать время до отправки, даже
видя сеть занятой,  рассматриваемая в данной статье модель верна,
только если одно сообщение  передается в  течении одного  слота, т.\,е.\ лишь при
 малых размерах  отправляемых пакетов. Иначе после
успешной передачи резко возрастает вероятность столкновения.
Действительно, если передача сообщения требует  {\it намного больше
одного  слота}, то многие другие станции, исчерпав время до отправки, попытаются
передать свои сообщения одновременно, сразу же  после завершения
данной передачи, что приведет к конфликтам.
 Заметим, что  в сети Ethernet станции продолжают отсчитывать время (пустые
слоты) до отправки, даже если  видят сеть занятой.  Поэтому, как
было отмечено, предлагаемая модель применима в данном случае    лишь
при малых размерах  отправляемых пакетов.

\section{Математическая модель в~дискретном времени  и~ее~анализ}

Суммируем предположения, положенные в основу модели. Они  приняты  в литературе, посвященной данному вопросу,
и являются вполне естественными и согласованными с имеющимися данными
о работе BP (см., например,~[3, 11, 12]).

\medskip

\noindent
\textbf{Предположение~1.} {\it Станции идентичны}.

\medskip

\noindent
\textbf{Предположение~2.}  {\it Стационарность}: вероятность конфликта
$p_c\in (0,1)$ постоянна  и является одной и той же для каждой
станции.

\medskip

\noindent
\textbf{Предположение~3.} {\it Условие насыщения}: каждая станция
всегда имеет непустую очередь.


\medskip

\noindent
\textbf{Предположение~4.} Если конфликт не происходит в первый слот
передачи сообщения, то  данное сообщение передается успешно.


\medskip

\noindent
\textbf{Предположение~5.} Когда какая-либо станция начинает передавать
сообщение, то  либо 1)~время до отправки на  других станциях не
уменьшается, либо 2)~время отправки одного сообщения занимает один
слот. При анализе протокола в терминах слотов оба эти предположения
эквивалентны.

Далее, состоянием данной станции считается  значение ее  BC. Если
$\mathrm{BC}=i$, то
$$
\p (D_i=k)=\fr{1}{f(i)}\,,\quad k=1,\dots,f(i)\,.
$$
Поэтому  среднее  время пребывания станции в данном состоянии определяется по формуле
$\e D_i\;=$\linebreak $=\;(f(i)+1)/2$. Далее, вероятность того, что значение $\mathrm{BC}=i$ и
в этом состоянии произойдет успешная отправка, есть
$P_i=(1-p_c)p_c^i$, $i\ge 0$. Назовем {\it циклом передачи
сообщения} время с момента первой попытки отправки до
успешной  его отправки и обозначим его длительность через~$S$.

Заметим, что стационарная вероятность $\gamma_i$ того, что станция
находится в состоянии~$i$ (в произвольный момент времени) равна доле
времени, проводимом станцией в этом состоянии на одном  цикле
передачи. Это приводит к сле\-ду\-юще\-му соотношению:
\begin{equation}
\gamma_i =
   \fr {P_i  \e D_i }{\sum_{i=0}^{\infty} P_i  \e D_i}
    =
   \fr {(f(i)+1)(1-p_c)p_c^i}
         {(1-p_c)F(p_c)+1}\,, \enskip i\ge 0\,,\!\!\!
         \label{eq:gamma}
\end{equation}
где использовано обозначение
\begin{equation*}
    F(p_c)
    =
    \sum_{i=0}^{\infty}f(i)p_c^i\,.
%    \label{eq:fz}
\end{equation*}

 Функцию~$F$, которая   играет в дальнейшем анализе ключевую роль,
 назовем {\it функцией протокола}. Заметим, что отправка сообщения
происходит (внутри окна передачи) в слоте с номером~ $N_R$,
распределение которого имеет вид $ \p(N_R=i)=P_i$, $i\ge 0.$ Поэтому величина
$$
\sum_{i=0}^{\infty} \e D_i P_i=\e D_{N_R}\,,
$$
стоящая в знаменателе~\eqref{eq:gamma}, является  средней  величиной
окна, в котором происходит успешная отправка  сообщения. В~\cite{KWAK} доказано следующее равенство:
$$
\fr{\gamma_i}{\e D_i}=
   \fr { P_i}{\e D_{N_R}}\,,\quad i\ge 0\,.
$$

Из формулы полной вероятности следует   такое выражение для
стационарной вероятности успешной отправки сообщения (в
произвольном слоте):
\begin{equation}
p_{\mathrm{st}} = \sum_{i=0}^{\infty}
   \fr{P_i}{\e
D_{N_R}}=\fr{1}{\e D_{N_R}}
   = \fr {2}{(1-p_c)F(p_c)+1}\,.\!\!\!
   \label{eq:pt}
\end{equation}

Вероятность~$p_{\mathrm{st}}$ можно получить  иначе. Заметим, что средняя
длина цикла передачи может быть записана таким образом:
\begin{multline}
\e\left(\sum_{i=0}^{N_R}D_i\right)=\sum_{i=0}^\infty \e D_i
\p\left(N_R\ge i\right)={}\\
{}=\fr{1}{2}\sum_{i\ge 0}\left(f(i)+1\right)p_c^i=\fr{1}{2} \left(F(p_c)+\fr{1}{(1-p_c)}\right)={}\\
{}=
\fr{(1-p_c)F(p_c)+1}{2(1-p_c)}\,.
\label{eq:4}
\end{multline}

Поскольку $P(N_R\ge i)= p_c^i$, то среднее число попыток до успешной
отправки 
\begin{eqnarray}
\e N_R =\fr{p_c}{(1-p_c)}\,. 
\label{eq:5}
\end{eqnarray}

Понятно, что в стационарном режиме вероятность отправки (в
произвольный момент времени) равна отношению среднего числа попыток
к средней длине цикла передачи. Поэтому~(\ref{eq:4}) и~(\ref {eq:5})
дают выражение
\begin{equation*}
p_{\mathrm{st}}=\fr{\e N_R}{\e\Bigl (\sum_{i=0}^{N_R}D_i\Bigr)},
\end{equation*}
которое совпадает с~(\ref{eq:pt}). Учтем, что данная станция (как
и остальные  $N-1$ станций) находится в стационарном режиме.
Поскольку конфликт возникает,  если не менее двух станций пытаются
отправить сообщения одновременно, то нетрудно получить следующее
соотношение (см.\ также~[9, 12]):
\begin{equation}
  p_c=1-(1-p_{\mathrm{st}})^{N-1}\,.
  \label{eq:pc}
\end{equation}

Объединяя выражения~(\ref{eq:pt}) и~(\ref{eq:pc}), приходим к
следующему выражению для функции протокола:
\begin{equation}
   F(p_c) = \fr{1+(1-p_c)^{1/(N-1)}}{(1-p_c)\left(1-(1-p_c)^{1/(N-1)}\right)}\,.
   \label{eq:Fpc}
\end{equation}

В~\cite{LUKYA} показано, что для существования единственного решения
$p_c\in (0,1)$ уравнения~(\ref{eq:Fpc}) достаточно   монотонного
возрастания функции $f$. Заметим, что различные протоколы,
определяемые различными функциями~$F$, вообще говоря, приводят к различным решениям~$p_c$.

Теперь  найдем  при каком значении~$p_c$ станция будет работать
оптимально, т.\,е.\ обеспечит минимальную  среднюю  длину  цикла
передачи~$\e S$.   Используя независимость случайных величин~$N_R$ и~$D_i$, 
можно записать
\begin{multline}
   \e S   = \e \left (\sum_{i=0}^{N_R}D_i\right)
        ={}\\
        {}= \fr{1}{2}\,\e \left(\sum_{i=0}^{\infty}(f(i)+1)\Ik (N_R\ge
        i)\right)={}\\
{}=\fr{1}{2} \left( F(p_c) +\fr{1}{1-p_c}\right)\,,
       \label{eq:es2}
\end{multline}
где $\Ik$ обозначает индикатор. 

Объединяя~(\ref{eq:Fpc}) и~(\ref{eq:es2}), получаем
\begin{equation}
   \e S  = \fr{1}{(1-p_c)\left(1-(1-p_c)^{1/(N-1)}\right)}\,. 
   \label{eq:eqm}
\end{equation}

Нетрудно проверить стандартным способом, что   минимальное значение
среднего цикла передачи $\e S$, являющегося функцией  вероятности
конфликта~$p_c$, достигается в точке $p_c=p_c^*$, где
\begin{equation}
   p_c^* = 1 - \left(1-\frac1N\right)^{N-1}\,.
    \label{eq:pcopt}
\end{equation}

Подстановка~(\ref{eq:pcopt}) в~(\ref{eq:eqm}) дает
\begin{equation}
\e S =\fr{N}{\left(1-1/N \right)^{N-1}}\,. 
\label{eq:8}
\end{equation}

Если рассматривать  вероятность конфликта как функцию числа станций~$N$ 
в сети, т.\,е.\ $p_c^*=p_c^*(N)$, то из~(\ref{eq:pcopt}) следует
такой асимптотический результат:
\begin{equation*}
p_c^* \to 1-e^{-1}\,,\quad N\to \infty\,. 
%\label{eq:asympt}
\end{equation*}

Подстановка~(\ref{eq:pcopt}) в~(\ref{eq:Fpc}) показывает, что
протокол является оптимальным, если его функция~$F$ удовлетворяет
условию
\begin{equation}
   F(p_c^*) =
   \fr{2N-1}{\left(1-1/N\right)^{N-1}}:=A(N)\,.
   \label{optimum}
  % \nonumber
\end{equation}
Отметим, что этому соотношению  может удовлетворять, вообще говоря,
не единственная функция расширения окна $f$ (т.\,е.\ не  единственный
протокол), а целый класс функций~$f$.

Рассмотрим класс наиболее важных с точки зрения практики  {\it
экспоненциальных протоколов}, где $f(i)=a^i$ для некоторого числа
$a>0$. В этом случае  в предположении $a p_c^*<1 $ соотношение~(\ref{optimum}) принимает вид
\begin{equation*}
\fr{1}{1-ap_c^*} =A(N)\,. 
%\label{6}
\end{equation*}

Таким образом,
$$
a=\fr{A(N)-1}{A(N)
p_c^*}=\fr{2N-1-\left(1-1/N\right)^{N-1}}{\left(2N-1\right)\left(1-\left(1-1/N\right)^{N-1}\right)}\,.
$$
Нетрудно показать, что  при $N\to \infty$
$$
a:=a(N)\to \fr{1}{1-e^{-1}}\approx 1{,}58
$$
 и   что $a(N)$ монотонно убывает с ростом
 числа станций~$N$.  Кроме того,  $a(2)=5/3\approx 1{,}66.$
Таким образом,  {\it оптимальное значение  параметра~$a$ в формуле}~(\ref{optimum})
\textit{определяется однозначно для каждого~$N$, находится в
диапазоне $[1{,}58,\, 1{,}66]$ и  с ростом  числа станций приближается к~1,58  сверху}.

Коснемся условия стационарности системы, считая
упрощенно, что суммарный входной поток в систему является процессом
восстановления с интенсивностью $\lambda \in (0,\,\infty)$.
Естественно считать, что тогда каждая станция получает входной поток
интенсивности~$\lambda/N$. Такое  предположение вполне оправдано при
большом числе идентичных станций. С~другой стороны, нетрудно
показать (используя, скажем,  аргументы из теории регенерирующих процессов), что  условие стационарности (для каждой
станции) имеет хорошо известный вид:
\begin{equation}
\fr{\lambda}{N}\e S< 1\,. 
\label{eq:9}
\end{equation}

Такого рода условия {\it отрицательного сноса} хорошо известны в
анализе стационарности марковских цепей~\cite {MEYN}. Как правило,
они исключают {\it уход процесса в бесконечность} и
обеспечивают существование стационарного режима для широкого класса
процессов, описывающих реальные коммуникационные системы. Отметим,
что хотя $\e S\to \infty$ при $N\to \infty$ (см.~(\ref{eq:8})),
однако интенсивность входного потока в отдельный узел
пропорционально убывает, сохраняя неравенство~(\ref{eq:9}). При
оптимальном значении $p_c=p_c^*$ неравенство~(\ref{eq:9}) принимает
форму (см.~(\ref{eq:eqm}), (\ref{eq:pcopt})):
\begin{equation*}
\lambda<\left(1-\fr{1}{N}\right)^{N-1}\,.
\end{equation*}
 Это неравенство в пределе  при  $N\to \infty$ переходит в
 неравенство $ \lambda<e^{-1}, $ возникающее, например,  при анализе
 стационарности   синхронной  системы ALOHA~\cite{ROBERTS, Kleinrock}.
 %\todo{пробел в цитировании?}.
 Таким образом,  при
оптимальном выборе протокола и большом числе станций интенсивность
входного потока на каждую станцию, обеспечивающая устойчивость,
должна быть меньше~$e^{-1}$. Следует ожидать, что для произвольного
протокола эта область устойчивости еще меньше.  Отметим, что
проблема неустойчивости BP неоднократно отмечалась  ранее (см.~[6--8]).

\section{Ограниченный протокол отсрочки}

Проведенный выше анализ можно перенести   на случай, когда число
конфликтов ограничено величиной  $M<\infty$. В~этом случае формула~(\ref{eq:Fpc}), 
определяющая функцию протокола, преобразуется к виду:
\begin{multline}
  F(p_c):= F_M(p_c)={}\\
  {}=\fr{\left(1-p_c^{M+1}\right)\left(1+\left(1-p_c\right)^{1/(N-1)}\right)}
   {\left(1-p_c\right)\left(1-\left(1-p_c\right)^{1/(N-1)}\right)}\,, 
   \label{eq2:FMpc}
\end{multline}
а средняя длина цикла передачи принимает вид:
\begin{equation*}
   \e S = \fr{\left(1-p_c^{M+1}\right)}
   {\left(1-p_c\right)\left(1-\left(1-p_c\right)^{1/(N-1)}\right)}\,. 
%   \label{eq2:ES}
\end{equation*}

 Рассматриваемое  ограничение применяется, например, в  алгоритме
протокола Ethernet, где используется ограниченный бинарный
экспоненциальный протокол. Более точно, $f(i)=2^i,$ $i=1,\dots,10$ и
$f(i)=1024$, $i=11,\ldots,16$. После 16~конфликтов сообщение
выбрасывается из очереди. Это правило дает следующую функцию протокола:
\begin{multline*}
   F_{16}(p_c)=\sum_{i=0}^{10} 2^ip_c^i + \sum_{i=11}^{16} 2^{10} p_c^i ={}\\
   {}=\fr{1-(2p_c)^{11}}{1-2p_c}+2^{10}p_c^{11}\fr{1-p_c^6}{1-p_c}\,.
\end{multline*}
Поэтому уравнение~(\ref{eq2:FMpc}) (относительно~$p_c$) принимает вид
\begin{multline}
\fr{1-(2p_c)^{11}}{1-2p_c}+2^{10}p_c^{11}\fr{1-p_c^6}{1-p_c} ={}\\
{}=
\fr{\left(1-p_c^{17}\right)\left(1+\left(1-p_c\right)^{1/(N-1)}\right)}
   {\left(1-p_c\right)\left(1-\left(1-p_c\right)^{1/(N-1)}\right)}\,.
   \label{17}
\end{multline}
Решив это уравнение,  можно подсчитать  среднюю длину цикла передачи
сообщения~$\e S $   и вероятность  потери  сообщения~$p_c^{17}$.
Явное  решение уравнения~(\ref{17}) в общем случае найти трудно,
однако его  можно исследовать численно. Некоторые численные результаты
представлены в  табл.~\ref{t1mor}.
Заметим, что вероятность столкновения и вероятность потери
возрастают с ростом числа станций. Так, при $N=11$  вероятность
столкновения $p_c=0{,}621$ (и достаточно близка к оптимальной
вероятности столкновения $p_c=p_c^*\approx 1-e^{-1}=0{,}6817$,
полученной  для неограниченного протокола). Вероятность потери в
этом случае незначительна. При $N=501$ и $N=1001$ вероятность
столкновения  больше~0,9, а вероятность потери  равна
соответственно~0,35 и~0,81. При этом среднее время передачи
сообщения  в сети $\e S/N$ (в отличие от среднего цикла передачи~$\e
S$ для отдельной станции) практически не изменяется  с ростом числа
станций~$N$.  Причина такой устойчивости, по-видимому, состоит в
том, что увеличение вероятности потери, вызванное ростом числа
станций, сдерживает рост среднего времени передачи.

\begin{table*}\small
\begin{center}
 \Caption{Численный анализ  BEB
 \label{t1mor}}
 \vspace*{2ex}
 
    \begin{tabular}{|c|c|c|c|}
        \hline  Число станций   &  
        \tabcolsep=0pt\begin{tabular}{c}Вероятность\\ столкновения $p_c$\end{tabular} &
\tabcolsep=0pt\begin{tabular}{c} Среднее время\\ передачи сообщения\\ в сети $\e S/N$\end{tabular}& 
\tabcolsep=0pt\begin{tabular}{c}Вероятность\\  потери $p_c^{17}$\end{tabular}\\
        \hline 
        $11$   & 0,621 &  2,593  &  \hphantom{9}0,0003\\
        \hline 
        51   & 0,743 &  2,828  &  \hphantom{9}0,0064 \\
        \hline 
        101\hphantom{9}  & 0,799 &  3,026  &  0,022 \\
        \hline 
        501\hphantom{9}  & 0,94\hphantom{9}  &  3,858  &  0,349 \\
        \hline 
        1001\hphantom{99} & 0,988 &  3,515  &  0,809 \\
        \hline
    \end{tabular}
\end{center}
\end{table*}

\section{Модель  протокола в~непрерывном времени}

Выше  рассмотрена дискретная модель в терминах слотов, которые,
вообще говоря, не равны по величине. Полезно рассмотреть модель в
непрерывном времени, что может привести к существенному изменению   вида
оптимизационной задачи. Для этого введем величину пустого слота~$T_i$,  
величину слота~$T_{c}$, где произошло столкновение,  и
величину~$T_{s}$ слота, где произошла успешная отправка. Если
размеры слотов одинаковы (т.\,е.\ $T_{c}=T_{s}=T_{i}$), то среднее
(физическое) время цикла передачи $\e_T S= \e S \cdot T_{i}$ и
модель сводится к модели, рассмотренной выше.

В случае неограниченного протокола среднее число слотов в цикле
передачи, равное  $\e \left (\sum_{i=0}^{N_R}D_i\right)$, содержит в
среднем $\e N_R+1$ {\it непустых  слотов}, причем в течение   $\e
N_R$ слотов происходят столкновения  и  один слот содержит успешную
отправку.  Пусть $T^*_{i}$~--- размер  слота, в течение которого {\it
данная произвольная станция}  не отправляет сообщения 
(в то время как   другие станции могут отправлять  или  не отправлять).
 Таким образом, средний цикл передачи 
\begin{multline}
   \e_T S   = \e N_R T_{c} + {}\\
   {}+\left (\e \left[\sum_{i=0}^{N_R}D_i\right]-\e N_R-1\right)T^*_{i}+
   T_{s}\,.
   \label{time}
\end{multline}
C учетом~(\ref{eq:4}) и~(\ref{eq:5}) соотношение~(\ref{time})
принимает вид:
\begin{multline}
   \e_T S   = \fr{p_c}{1-p_c} T_{c} + {}\\
   {}+\fr1{2(1-p_c)}\left((1-p_c)F(p_c)-1\right)T^*_{i}+ T_{s}\,. 
   \label{eq:ets}
\end{multline}

Вообще говоря, $T^*_{i}\not = T_{i}$, поскольку
 величина  пустого слота для
данной станции {\it растягивается} на время возможных передач
другими станциями. Для нахождения~$T^*_{i}$ используем  формулу
полной вероятности. Во-первых, $T^*_{i}=T_{i}$ с вероятностью
$1-p_c$, с которой  остальные  $N-1$~станций  не пытаются передавать
(в произвольном слоте). Далее (предполагаем, что $N>2$), лишь одна из
$N-1$ оставшихся станций передает сообщение (а остальные $N-2$
станций молчат) в течение данного  слота с вероятностью
$(N-1)p_{\mathrm{st}}(1-p_{\mathrm{st}})^{N-2}$, и с этой вероятностью
$T^*_{i}=T_{s}$. Оставшаяся вероятность  соответствует событию, при
котором, по крайней мере, две станции пытаются передавать сообщение в
данном слоте, и тогда $T^*_{i}=T_{c}$. Таким образом, используя
соотношения~(\ref{eq:pt}) и~(\ref{eq:pc}), имеем
\begin{multline*}
    T^*_{i} = (1-p_c)T_{i}+
   (1-p_c)\fr{2(N-1)}{(1-p_c)F(p_c)-1}T_{s}+{}\\
{}+\left(1-(1-p_c)-
   (1-p_c)\fr{2(N-1)}{(1-p_c)F(p_c)-1}\right)T_{c}={}\\
{}= (1-p_c)T_{i} +{}\\
{}+ p_c T_{c} + \fr{2(1-p_c)}{(1-p_c)F(p_c)-1}(N-1)(T_{s}-T_{c})\,.
\end{multline*}
Подставляя это выражение  в~(\ref{eq:ets}), получаем
    \begin{multline*}
   \e_T S   = N(T_{s}-T_{c}) +\fr{1}{1-p_c}T_{c} +{}\\
   {}+
   \fr{2(N-1)}{(1-p_c)F(p_c)-1}\left(\left(1-p_c\right)T_{i} + p_c
   T_{c}\right)\,.
   \end{multline*}

Используя явный вид~(\ref{eq:Fpc}) функции~$F(p_c)$, окончательно
получаем  величину среднего цикла передачи в  виде
\begin{multline*}
\e_T S   = N(T_{s}-T_{c}) +\fr{1}{1-p_c}T_{c} +   (1-p_c)(N-1)\times{}\\
{}\times
\left((1-p_c)^{1-1/(N-1)}-1\right)\left((1-p_c)T_{i} + p_c
   T_{c}\right)\,.\!
%   \label{itog}
\end{multline*}

Таким образом, задача оптимизации сводится к  отысканию минимума
$\e_T S$ как функции~$p_c$ для заданных величин~$T_{s}$, $T_{c}$ и~$T_{i}$. 
Заметим, что~$T_{s}$ не влияет на задачу оптимизации.

 В работе~\cite{BIANCHI} исследован  частный случай данной функции (для
бинарного экспоненциального протокола) при фиксированных величинах~$T_{s}$, $T_{c}$ и~$T_{i}$, 
описанных в спецификации~\cite{IEEE80211}. В~частности, $T_{i}=50$ мкс,
 средняя длина   фрейма (пакета MAC-уровня) $T_{s}$ определяется
настройщиком сети,  величина~$T_{c}$ определяется
 видом конкретно используемого протокола.
Из работы~\cite{BIANCHI} следует, что предлагаемая в данной статье
модель адекватно описывает работу реальных сетевых протоколов.

\section{Заключение}

В статье предложен метод исследования сетевого протокола контроля
несущей,  в котором окно отсрочки передачи сообщения при конфликте
сообщений, посланных разными станциями, меняется  в соответствии с
произвольной (не обязательно экспоненциальной) функцией~$f$,
монотонно возрастающей с ростом числа последовательных конфликтов
при попытке передачи данного сообщения данной станцией. Метод
базируется на нескольких естественных и неоднократно использованных
ранее предположениях. В~частности, предполагается, что станции сети
идентичны и работают независимо друг от друга, находятся в состоянии
насыщения (т.\,е.\ сообщения  для  передачи есть всегда), искомая
вероятность конфликта~$p_c$ является стационарной (и одинаковой для
всех станций). %\linebreak 
Кроме того, предполагается, что время отправки
сообщения распределено равномерно внутри \mbox{окна} передачи.
 В~отличие от предшествующих работ, исследование охватывает весь  класс протоколов с рас\-ши\-ря\-ющим\-ся
окном. Ключевым элементом    является введение функции протокола~$F$, 
которая строится на базе функции~$f$. 
%
С~помощью установленных
соотношений между веро\-ят\-ностью конфликта~$p_c$  и ве\-ро\-ят\-ностью
отправки сообщения удалось \mbox{найти} явное  выражение для~$F$ как функции
ве\-ро\-ят\-ности~$p_c$. 
%
В~статье объясняются некоторые эффекты,
обсуждавшиеся  при исследовании данного протокола в ряде
предшествующих работ (например,~\cite{SHOCH}). В~частности,
показано, что в нагруженном состоянии и при растущем числе станций в
оптимальном режиме работы (т.\,е.\ при минимальном среднем цикле
передачи~$\e S$) вероятность конфликта $p_c\to 1-e^{-1}$. При тех же
условиях найдено, что область стационарности сети при суммарном
входном потоке интенсивности~$\lambda$ имеет вид $\lambda<e^{-1}$.
{\looseness=-1

}

Исследован протокол как с неограниченным, так и с ограниченным
числом попыток передачи сообщения.

\vspace*{-12pt}

{\small\frenchspacing
{\baselineskip=12pt
\addcontentsline{toc}{section}{Литература}
\begin{thebibliography}{99}

  \bibitem{METCALFE}
  \Au{Metcalfe R., Boggs~D.}
  Ethernet: Distributed packet switching for local computer
networks~// Communications of the ACM, 1976. Vol.~19. No.\,7. P.~395--404.

  \bibitem{ABRAMSON85}
  \Au{Abramson N.} 
  Development of the ALOHANET~//
  IEEE Trans. on Inform. Theory, 1985. Vol.~31. No.\,2. P.~119--123.

  \bibitem{SHOCH}
  \Au{Shoch J.\,F., Hupp J.\,A.} 
  Measured performance of an Ethernet local network~//
  Commun. ACM, 1980. Vol.~23. No.\,12. P.~711--721.

  \bibitem{IEEE80211}
  \Au{IEEE 802.11 Standard.}
  IEEE Standard for Information technology~--- Telecommunications and information exchange between systems~---
  Local and metropolitan area networks~--- Specific requirements.
  Part~11: Wireless LAN Medium Access Control (MAC)
  and Physical Layer (PHY) Specifications.~--- N.Y.: IEEE, 2007.

  \bibitem{HASTAD}
  \Au{H{\tiny$\stackrel{\circ}{\mbox{\normalsize a}}$}stad J., Leighton~T., Rogoff~B.}
  Analysis of backoff protocols for multiple access channel~// SIAM J. Comput., 1996. Vol.~25. No.\,4. P.~740--774.

  \bibitem{KELLY85}
  \Au{Kelly F.\,P.} 
  Stochastic models of computer   communication systems~//J.~Roy. Statist. Soc. B, 1985. Vol.~47. P.~379--395.

  \bibitem{KELLY87}
  \Au{Kelly F.\,P.,  MacPhee~I.\,M.} 
  The number of packets   transmitted by collision detect random access scheme~//
  Annals of Prob., 1987. Vol.~15. P.~1557--1568.

  \bibitem{ALDOUS}
  \Au{Aldous D.\,J.} 
  Ultimate instability of exponential back-off
  protocol for acknowledgement-based transmission control of random
  access communication channel~// IEEE Trans. on Information Theory, 1987. Vol.~33. No.\,2. P.~219--223.


  \bibitem{LUKYA}
  \Au{Lukyanenko A., Gurtov~A.} 
  Performance analysis of general backoff  protocols~// J.~Communications Software and Systems, 2008. Vol.~4. No.\,1. P.~13--22.


  \bibitem{VISHN} 
  \Au{Вишневский В., Ляхов А., Портной~С., Шахнович~И.}
  Широкополосные беспроводные сети передачи информации.~--- М.: Техносфера, 2005.

  \bibitem{BIANCHI}
  {\it Bianchi G.} 
  Performance analysis of the IEEE 802.11 Distributed Coordination Function~//
  IEEE J.~Selected Areas in Communications, 2000. Vol.~18. No.\,3. P.~535--547.
  
    \bibitem{KWAK}
  \Au{Kwak B., Song N., Miller~L.\,E.} 
  Performance analysis   of exponential backoff~// IEEE/ACM Transactions on Networking (TON), 2005. 
  Vol.~13. No.\,2. P.~343--355.

  \bibitem{MEYN}
  \Au{Meyn S., Tweedie R.\,L.} 
  Markov chains and stochastic stability.~--- New York: Cambridge University Press, 2009.

  \bibitem{ROBERTS}
  \Au{Roberts L.\,G.} 
  ALOHA packet system with and without slots and capture~// SIGCOMM Comput. Commun., 1975. Vol.~5. No.\,2. P.~28--42.
  
  \label{end\stat}

    \bibitem{Kleinrock}
    \Au{Клейнрок Л.} 
    Вычислительные системы с очередями.~--- М.: Мир, 1979.

%  \bibitem{ABRAMSON70}
%  \Au{Abramson N.} 
%  The ALOHA system~--- another alternative  for computer communications~// AFIPS, 1970. Vol.~37. P.~281--285.

%  \bibitem{GOODMAN}
%  \Au{Goodman J., Greenberg A.\,G., Madras~N., March~P.} 
%  Stability of binary exponential backoff~//  J.~ACM, 1988. Vol.~35. No.\,3. P.~579--602.

 % \bibitem{FAYOLLE}
%  \Au{Fayolle G., Flajolet P.,  Hofri~M.}
%  On a functional equation arising in the analysis of a protocol for a
%  multi-access broadcast channel~// Adv. Appl. Prob., 1986. Vol.~18. P.~441--472.

 % \bibitem{ROSENKRANTZ}
%  \Au{Rosenkrantz W.\,A.} 
%  Some theorems on the instability   of the exponential back-off protocol~//  Proc. of Performance'84,
%  1985. P.~199--205.

 % \bibitem{GOLDBERG98}
%  \Au{Goldberg L.\,A., MacKenzie~P.} 
%  Analysis of practical
%  backoff protocols for contention resolution with multiple servers~//
%  J. Comp. System Sci., 1999. Vol.~58. P.~232--258.

 % \bibitem{GOLDBERG00}
%  \Au{Goldberg L.\,A., MacKenzie P., Paterson~M., Srinivasan~A.} 
%  Contention resolution with constant
%  expected delay~// J. ACM (JACM), 2000. Vol.~47. No.\,6. P.~1048--1096.

 % \bibitem{JACOBSON}
%  \Au{Jacobson V., Karels M.\,J.} 
%  Congestion   avoidance and control~// Proc. of ACM SIGCOMM, 1998.
  
%  \label{end\stat}

 % \bibitem{AKELLA} 
%  \Au{Akella A., Seshan S., Karp~R., Shenker~S.}
%  Selfish behavior and stability of the Internet: A
%  game-theoretic analysis of TCP~// Proc. of ACM SIGCOMM, 2002.
 \end{thebibliography}
}
}

\end{multicols}      %5
\include{kruchin}      %6
\include{stepanov}       %7
\def\stat{bening}


\def\tit{АСИМПТОТИЧЕСКОЕ
РАЗЛОЖЕНИЕ ДЛЯ МОЩНОСТИ КРИТЕРИЯ, ОСНОВАННОГО НА ВЫБОРОЧНОЙ
МЕДИАНЕ, В~СЛУЧАЕ РАСПРЕДЕЛЕНИЯ ЛАПЛАСА$^*$}
\def\titkol{Асимптотическое
разложение для мощности критерия, основанного на выборочной
медиане} %, в случае распределения Лапласа}

\def\autkol{В.\,Е.~Бенинг, А.\,В.~Сипина}
\def\aut{В.\,Е.~Бенинг$^1$, А.\,В.~Сипина$^2$}

\titel{\tit}{\aut}{\autkol}{\titkol}

{\renewcommand{\thefootnote}{\fnsymbol{footnote}}\footnotetext[1]
{Работа выполнена
при финансовой поддержке РФФИ, проекты 08-01-00567 и
08-07-00152.}}

\renewcommand{\thefootnote}{\arabic{footnote}}
\footnotetext[1]{Московский государственный университет им.\
М.\,В.~Ломоносова, факультет вычислительной математики и
кибернетики; Институт проблем информатики Российской академии наук, bening@yandex.ru}
\footnotetext[2]{Московский государственный университет им.\
М.\,В.~Ломоносова, факультет вычислительной математики и
кибернетики, anna@sipin.ru}



\Abst{В работе прямыми методами, использующими асимптотические разложения,
получена формула для предельного отклонения мощности критерия, 
основанного на выборочной медиане, от мощности наилучшего критерия в случае распределения Лапласа.}

\KW{выборочная медиана; асимптотичсекое разложение; функция мощности; распределение Лапласа}

      \vskip 18pt plus 9pt minus 6pt

      \thispagestyle{headings}

      \begin{multicols}{2}

      \label{st\stat}


\section{Введение}

Следуя работе~\cite{3ben}, рассмотрим задачу проверки гипотезы
\begin{equation*}
{\sf H}_0: \theta = 0     
%\label{e1.1b}
\end{equation*}
против последовательности сложных близких альтернатив вида
\begin{equation*}
{\sf H}_{n,1}: \theta = \fr{t}{\sqrt{n}}\,,\quad 0<t<C\,,\quad
 C > 0
% \label{e1.2b}
\end{equation*}
на основе выборки $(X_1, \ldots , X_n)$~--- независимых одинаково распределенных наблюдений, имеющих распределение Лапласа 
с плотностью
\begin{equation}
p(x, \theta) = \fr{1}{2}e^{-|x-\theta|}\,, \quad x,\:
\theta \in{\sf R}^1\,. 
\label{e1.3b}
\end{equation}
Распределение Лапласа широко применяется в прикладной статистике, например
в задачах вы\-де\-ле\-ния полезного сигнала на фоне помех~[2--4].
Естественность возникновения этого распределения обоснована в
работе~\cite{6ben}.

Для каждого фиксированного $t\in (0,C]$
обозначим через~$\beta_n^*(t)$ мощность наилучшего критерия размера
$\alpha\in (0,1)$. По лемме Неймана--Пирсона %\linebreak 
[6, с.~94]
такой критерий всегда существует и  основан на логарифме отношения правдоподобия
\begin{equation}
\Lambda_n(t) = 
\sum_{i=1}^{n}\left( \left|X_i\right|-\left|X_i-tn^{-1/2}\right|\right)\,.
 \label{e1.4b}
\end{equation}
В работах~\cite{3ben, 2ben} рассмотрен критерий, основанный на знаковой статистике,
и получена формула для предельного отклонения мощности данного
критерия от мощности наилучшего критерия, основанного на~$\Lambda_n(t)$.
Поскольку у плотности~$p(x,\theta)$ не существует производной по~$\theta$ в 
точке $\theta = 0$, то это семейство не является регулярным.
Это выражается в нарушении естественного порядка~$n^{-1}$ разности мощностей
этих критериев и приводит к порядку~$n^{-1/2}$.

В  работе рассматривается статистика
\begin{equation*}
T_n = \sqrt{2k}\, \zeta_n\,,\quad k=\left[\fr{n}{2}\right]\,, 
%\label{e1.5b}
\end{equation*}
где $\zeta_n$~--- выборочная медиана:
\begin{equation*}
\zeta_n= 
\begin{cases}
X_{(k+1)}\,, & n=2k+1\,; \\
\fr{X_{(k)}+X_{(k+1)}}{2}\,, &  n=2k\,.
\end{cases}
%\label{e1.6b}
\end{equation*}
Заметим, что в случае распределения Лапласа выборочная медиана
совпадает с оценкой максимального правдоподобия (см.~\cite{1ben}).

Обозначим через~$\beta_n(t)$ мощность критерия размера $\alpha\in (0,1)$,
основанного на статистике~$T_n$. В работе получено асимптотическое
разложение для~$\beta_n(t)$ и вычислен предел разности мощностей~$\beta_n^*(t)$ и~$\beta_n(t)$
$$
r(t)\equiv\lim_{n\to\infty}\sqrt n\left(\beta_n^*(t)-\beta_n(t)\right)
$$
критериев (см.~(\ref{e2.14b})),
основанных соответственно на статистиках~$\Lambda_n$ и~$T_n$.

В работе также приведено полное доказательство  (см.~\cite{5ben})
представления выборочной медианы в виде случайной суммы
независимых экспоненциально распределенных  случайных величин.


\section{Асимптотическое разложение для мощности критерия,
основанного на выборочной медиане}

В этом разделе будет построено  асимптотическое разложение  для мощности~$\beta_n(t)$.
Основой для его получения служит  работа~\cite{1ben} (см.\ теорему~2.1),
в которой получено разложение для функции распределения выборочной медианы.
Члены порядка~$n^{-1/2}$ в разложении для функции распределения выборочной медианы
без доказательства приведены  также в работе~\cite{9ben}.

\medskip
\noindent
\textbf{Теорема 1.} {\it Для мощности~$\beta_n(t)$ равномерно по
$t\in(0,C]$, $C>0$,
справедливо следующее асимптотическое разложение:
\begin{equation*}
\beta_n(t)=
\begin{cases}
\Phi(t-u_\alpha)-\fr{t(2u_\alpha-t)}{2\sqrt{n}}\,\varphi(u_\alpha-t)+{} \\
\hspace*{8mm}{}+o\left(n^{-1/2}\right)\,,  \quad t \le u_\alpha\,,\enskip  \alpha <\fr{1}{2}\,;\\
\Phi(t-u_\alpha)-\fr{2u_\alpha^2+t^2-2u_\alpha t}{2\sqrt{n}}\,\varphi(u_\alpha -t)+{}\\
\hspace*{8mm}{}+o\left(n^{-1/2}\right)\,, \quad t>u_\alpha\,, \enskip \alpha <\fr{1}{2}\,;\\
\Phi(t-u_\alpha)+\fr{t(2u_\alpha-t)}{2\sqrt{n}}\,\varphi(u_\alpha -t)+{}\\
\hspace*{22mm}{}+{} o\left(n^{-1/2}\right)\,, \quad 
\alpha \ge \fr{1}{2}\,,
\end{cases}\hspace*{-6pt}
%\label{e2.1b}
\end{equation*}
где  $\Phi(x)$  и~ $\varphi(x)$~---  функция распределения и
плотность стандартного нормального закона и $\Phi(u_\alpha)=1-\alpha$.}

\medskip

\noindent
Д\,о\,к\,а\,з\,а\,т\,е\,л\,ь\,с\,т\,в\,о\,.\
Для доказательства теоремы воспользуемся асимптотическим разложением
для функции распределения выборочной медианы в случае
распределения Лапласа из работы~\cite{1ben} (см.\ формулу~(1.3)):
\begin{multline}
\p_{n,\theta} \left( \sqrt{2k}(\zeta_n - \theta) < x \right) = 
\Phi(x)-\fr{x|x|}{2\sqrt{2k}}\,\varphi(x)+{}\\
{}+
\fr{x(18+10x^2-3x^4)}{48k}\,\varphi(x)+ o(n^{-1})\,.
\label{e2.2b}
\end{multline}
Подберем критическое значение~$d_n$, исходя из условия
\begin{equation*}
\p_{n,0}(T_n>d_n)=\alpha+ o(n^{-1})\,.
%\label{e2.3b}
\end{equation*}
Будем искать $d_n$ в виде
\begin{equation*}
d_n = u_\alpha +\fr{a}{\sqrt{2k}}+\fr{b}{2k}\,.
%\label{e2.4b}
\end{equation*}
Из формулы~(\ref{e2.2b}) следует, что

\noindent
\begin{multline}
\p_{n, 0} \left( T_n> d_n \right) = 1 -
\Phi(d_n)+\fr{d_n|d_n|}{2\sqrt{2k}}\varphi(d_n)-{}\\
{}-
\fr{d_n(18+10d_n^2-3d_n^4)}{48k}\,\varphi(d_n)+ o(n^{-1})\,.
\label{e2.5b}
\end{multline}
Чтобы раскрыть модуль в выражении~(\ref{e2.5b}),  рас\-смот\-рим два случая:
$\alpha<1/2$ и $\alpha \ge 1/2$.

Рассмотрим случай $\alpha < 1/2$. Это означает, что при достаточно
больших $n$ справедливо неравенство $d_n > 0$.
Подставляя выражение для~$d_n$ в формулу~(\ref{e2.5b}) и применяя следующие разложения:
\begin{multline*}
\Phi(d_n)=\Phi\left(u_\alpha+\fr{a}{\sqrt{2k}}+\fr{b}{2k}\right)=
\Phi(u_\alpha)+{}\\
{}+
\left(\fr{a}{\sqrt{2k}}+\fr{b}{2k}\right)\varphi(u_\alpha)-
\fr{u_\alpha a^2}{4k}\varphi(u_\alpha)+ o(n^{-1})\,;
\end{multline*}
\vspace*{-12pt}

\noindent
\begin{multline*}
\varphi(d_n)=\varphi\left(u_\alpha+\fr{a}{\sqrt{2k}}
+\fr{b}{2k}\right)= {}\\
{}=
\varphi(u_\alpha)-\left(\fr{a}{\sqrt{2k}}+\fr{b}{2k}\right)u_\alpha
\varphi(u_\alpha)+ o(n^{-1})\,,
\end{multline*}
получаем
\begin{multline*}
1-\Phi(u_\alpha)-\left(\fr{a}{\sqrt{2k}}+
\fr{b}{2k}\right)\varphi(u_\alpha)+\fr{u_\alpha a^2}{4k}\,\phi(u_\alpha)
+{}\\
{}+\fr{(u_\alpha+(a/\sqrt{2k})+b/(2k))^2}
{2\sqrt{2k}}\times{}\\
{}\times \left(\varphi(u_\alpha) - \fr{a}{\sqrt{2k}}\,u_\alpha
\varphi(u_\alpha)\right)-{}\\
{}-
\fr{u_\alpha(18+10u_\alpha^2-3u_\alpha^4)}{48k}\,\varphi(u_\alpha)=
\alpha + o(n^{-1})\,.
\end{multline*}
Приравнивая коэффициенты при~$1/\sqrt{2k}$ и~$1/(2k)$ к нулю,
находим выражения для~$a$ и~$b$:
\begin{gather*}
a=\fr{u_\alpha^2}{2}\,;
\\
b=-\fr{3}{4}\,u_\alpha+\fr{1}{12}\,u_\alpha^3\,;
\\
d_n = u_\alpha+\fr{u_\alpha^2}{2\sqrt{2k}}-\fr{3}{8k}\,
u_\alpha+\fr{1}{24k}\,u_\alpha^3\,.
\end{gather*}
Теперь для получения асимптотического разложения мощности критерия используем
разложение
\begin{multline*}
\p_{n,tn^{-1/2}}(T_n<x)= \Phi\left(x-t\sqrt{2k}n^{-1/2}\right) -{}\\
{}-
\fr{\left(x-t\sqrt{2k}n^{-1/2}\right)\left| x\:-\:t\sqrt{2k}\,n^{-1/2}\right|}{2\sqrt{2k}}\,
{}\times{}\\
{}\times\varphi(x-t\sqrt{2k}\,n^{-1/2})+ {}
\end{multline*}
\begin{multline*}
{}+
\fr{ x-t\sqrt{2k}\,n^{-1/2}}{48k}
\left(18+10(x-
t\sqrt{2k}\,n^{-1/2})^2-{}\right.\\
\left.{}-3(x-t\sqrt{2k}\,n^{-1/2})^4\right)\times{}
\\
{}\times\varphi\left(x-t\sqrt{2k}\,n^{-1/2}\right)+ o\left(n^{-1}\right)\,,
%\label{e2.6b}
\end{multline*}
которое  получается при подстановке $\theta=tn^{-1/2}$ в
формулу~(\ref{e2.2b}).

Имеем
\begin{multline*}
\beta_n(t)=\p_{n,tn^{-1/2}}\left(T_n>d_n\right) ={}\\
{}=
1-\Phi\left(d_n-t\right) +
\fr{\left(d_n-t\right)\left|d_n-t\right|}{2\sqrt{2k}}\,\varphi\left(d_n-t\right)-{}
\\\!
{}-\fr{d_n-t}{48k}\left(18+10\left(d_n-t\right)^2
-3(d_n-t)^4\right)\, \varphi\left(d_n-t\right)+{}\\
{}+ o\left(n^{-1}\right)\,.
%\label{e2.7b}
\end{multline*}
Аналогично предыдущему, рассмотрим  два случая: $t\le u_\alpha$ и
$t>u_\alpha$.

Пусть сначала $t \le u_\alpha$.
Используя разложения
\begin{multline*}
\Phi\left(d_n-t\right)={}\\
{}=\Phi\left(u_\alpha-t+
\fr{u_\alpha^2}{2\sqrt{2k}}-\fr{3}{8k}\,u_\alpha+
\fr{1}{24k}\,u_\alpha^3\right)={}\\
{}=\Phi\left(u_\alpha-t\right)+
\left(\fr{u_\alpha^2}{2\sqrt{2k}}-\fr{3}{8k}\,u_\alpha+
\fr{1}{24k}\,u_\alpha^3\right)\times{}\\
{}\times\varphi\left(u_\alpha-t\sqrt{2k}\,n^{-1/2}\right)-{}
\\
{}-
\fr{\left(u_\alpha-t\sqrt{2k}\,n^{-1/2}\right)\varphi\left(u_\alpha-
t\sqrt{2k}\,n^{-1/2}\right)u_\alpha^4}{16k}+{}\\
{}+ o\left(n^{-1}\right)\,; 
%\label{e2.8b}
\end{multline*}

\vspace*{-12pt}

\noindent
\begin{multline*}
\varphi\left(d_n-t\right)={}\\
{}= \varphi\left(u_\alpha-t+
\fr{u_\alpha^2}{2\sqrt{2k}}-\fr{3}{8k}\,u_\alpha+
\fr{1}{24k}\,u_\alpha^3\right)={}\\
{}=
\varphi\left(u_\alpha-t\right)-\left(u_\alpha-t\right)
\varphi\left(u_\alpha-t\right)\fr{u_\alpha^2}{2\sqrt{2k}}+{}\\
{}+
o\left(n^{-1/2}\right)\,,
%\label{e2.9}
\end{multline*}
получаем, что
\begin{multline*}
\beta_n(t)=1-\Phi\left(u_\alpha-t\right)-
\fr{u_\alpha^2}{2\sqrt{2k}}\,\varphi\left(u_\alpha-t\right)+{}\\
{}+\fr{u_\alpha^2}{2\sqrt{2k}}\,\varphi(u_\alpha-t)-
\fr{2u_\alpha t - t^2}{2\sqrt{2k}}\,\varphi(u_\alpha-t)+{}\\
{}+
o\left(n^{-1/2}\right)=
\Phi\left(t-u_\alpha\right)-\fr{t\left(2u_\alpha - t\right)}{2\sqrt{2k}}\,
\varphi\left(u_\alpha - t\right)+{}\\
{}+ o\left(n^{-1/2}\right)\,.
%\label{e2.10b}
\end{multline*}
Во втором случае при $t > u_\alpha$  выражение
для мощности приобретает вид:

\noindent
\begin{multline*}
\beta_n(t)=\Phi\left(t-u_\alpha\right)-{}\\
{}-
\fr{t\left(2u_\alpha^2+t^2 -2u_\alpha t\right)}{2\sqrt{n}}\,
\varphi\left(u_\alpha-t\right)+ o\left(n^{-1/2}\right)\,.
%\label{e2.11b}
\end{multline*}
При $\alpha \ge 1/2$  аналогичным образом имеем
\begin{multline*}
\beta_n(t)={}\\
{}=
 \Phi\left(t-u_\alpha\right)+
\fr{t\left(2u_\alpha - t\right)}{2\sqrt{n}}\,\varphi\left(u_\alpha - t\right)+
o\left(n^{-1/2}\right)\,.
%\label{e2.12b}
\end{multline*}
Из этих формул следует утверждение теоремы.~$\Box$

\medskip

В работе~\cite{2ben} было показано, что для мощ\-ности~$\beta_n^*(t)$ 
критерия размера $\alpha\in (0,1)$, осно\-ван\-но\-го на
логарифме отношения прав\-до\-подобия~$\Lambda_n(t)$~(\ref{e1.4b}),
справедливо  асимптотическое\linebreak разложение
\begin{equation*}
\beta_n^*(t)=\Phi(t-u_\alpha) - \fr{t^2}{6\sqrt{n}}\,
\varphi(t-u_\alpha)+ o(n^{-1/2})\,.
%\label{e2.13b}
\end{equation*}
Используя это разложение и теорему~1, получаем формулу
для предельного отклонения нормированной разности мощностей
рассматриваемых критериев:
\begin{multline}
r(t)= \lim_{n \to \infty}\sqrt{n}(\beta_n^*(t)-\beta_n(t))
={}\\
{}=
\begin{cases}
\left(t u_\alpha-\fr{2t^2}{3}\right)
\varphi(u_\alpha-t)\,,\\
\hspace*{30mm} t \le u_\alpha\,,\enskip \alpha < \fr{1}{2}\,; \\
\left(u_\alpha^2+\fr{t^2}{3}-u_\alpha t \right)
\varphi(u_\alpha - t)\,,\\
\hspace*{30mm}  t>u_\alpha\,,\enskip \alpha<\fr{1}{2}\,; \\
\left(\fr{t^2}{3}-t u_\alpha\right)\varphi(u_\alpha-t)\,, \quad\quad\ \  \alpha \ge \fr{1}{2}\,. 
\end{cases}
\label{e2.14b}
\end{multline}

\section{Представление выборочной медианы в~виде случайной суммы}

В этом разделе докажем лемму о представлении выборочной медианы
в случае распределения Лапласа в виде суммы случайного числа
независимых экспоненциально распределенных случайных величин.
Формулы для представления порядковых статистик в случае распределения
Лапласа в виде подобной суммы приведены в работе~[4, с.~63],
но без строгого доказательства.

\bigskip

\noindent
\textbf{Лемма 1.}
{\it В случае распределения Лапласа выборочную медиану
можно представить в следующем виде (здесь равенства по распределению):
\begin{align}
\zeta_{2k+1} &\stackrel{d}{=}\delta_{2k+1}
\sum\limits_{j=k+1}^{K_{2k+1}}{\fr{W_j}{j}}\,;
\label{e3.1b}\\
\zeta_{2k}&\stackrel{d}{=}\fr{W_1-W_2}{2k}\,\mathbf{1}(B_{2k+1}=k)+{}\notag\\[1pt]
&\!\!\!\!\!\!\!\!\!\!\!\!\!\!{}+
\left(\delta_{2k}\sum\limits_{j=k+1}^{K_{2k+1}}\fr{W_j}{j}+
\delta_{2k}\fr{W_k}{2k}\right)\mathbf{1}\left(B_{2k+1} \ne k\right)\,,
\label{e3.2b}
\end{align}
где
$$
\delta_n=\mathrm{sign}\left(B_n-k-\fr{1}{2}\right)\,,
$$
$W_j$~--- независимые экспоненциально (с параметром~1) распределенные
случайные величины; $B_n$~--- бернуллиевские случайные величины с параметрами
$p=1/2$ и~$n$, независимые от~$W_j$;
\begin{equation*}
K_n = \max\left(B_n, \bar{B_n}\right)\,,\quad
\bar{B_n}= n - B_n\,.
\end{equation*}
}

\smallskip

\noindent
Д\,о\,к\,а\,з\,а\,т\,е\,л\,ь\,с\,т\,в\,о\,.

Вначале докажем две вспомогательные формулы, справедливые для любого
действительного чис\-ла~$s$
и любых натуральных чисел~$a$ и~ $b$:
\begin{gather}
\prod\limits_{j=a}^{a+b}{\fr{1}{j+is}}=
\sum\limits_{j=0}^b \fr{(-1)^j}{(a+j+is)(b-j)!j!}\,;
\label{e3.3b}
\\[3pt]
\!\!\!\!\!\!\!\!\sum\limits_{l=0}^k\fr{k!}{l!} \prod\limits_{j=a}^{a+k-l}\fr{1}{j+is}=
\sum\limits_{l=0}^k \begin{pmatrix}
k\\ l\end{pmatrix}
\fr{(-1)^l 2^{k-l}}{a+l+is}\,.
\label{e3.4b}
\end{gather}
Формулу~(\ref{e3.3b}) докажем методом математической индукции.

При $b=1$ формула верна. Предполагая ее верной при $b\ge1$,
докажем что она  верна и  при~$b+1$:
\begin{multline*}
\prod\limits_{j=a}^{a+b+1}\fr{1}{j+is}=\fr{1}{a+b+1+is}\prod\limits_{j=a}^{a+b}
\fr{1}{j+is}={}\\[2pt]
{}=
\fr{1}{a+b+1+is}\left(\sum\limits_{l=0}^k 
\begin{pmatrix}
k\\ l
\end{pmatrix}
\fr{(-1)^l 2^{k-l}}
{a+l+is}\right)={}\\[2pt]
{}=
\sum\limits_{j=0}^{b}\fr{(-1)^j}{(b-j)!j!} \left(\fr{1}{(b+1-j)(a+j+is)}
- {}\right.\\[2pt]
\left.{}-\fr{1}{(b+1-j)(a+b+1+is)} \right)={}
\end{multline*}
\begin{multline*}
{}=
\sum\limits_{j=0}^{b}\fr{(-1)^j}{(a+j+is)(b+1-j)!j!}-{}\\
{}-
\fr{1}{a+b+1+is}\sum\limits_{j=0}^{b}\fr{(-1)^j}{(b-j+1)!j!}\,.
\end{multline*}
Заметим, что
\begin{multline*}
\!\!\sum\limits_{j=0}^b\fr{(-1)^j}{(b-j+1)!j!}=
\sum\limits_{j=0}^{b+1}\fr{(-1)^j}{(b-j+1)!j!}
-\fr{(-1)^{b+1}}{(b+1)!}={}\\
{}=
\fr{1}{(b+1)!}(1-1)^{b+1}-\fr{(-1)^{b+1}}{(b+1)!}=
-\fr{(-1)^{b+1}}{(b+1)!}\,.
\end{multline*}
И следовательно, формула~(\ref{e3.3b}) доказана.
Формула~(\ref{e3.4b}) следует  из доказанной формулы~(\ref{e3.3b}), по\-скольку
\begin{multline*}
\sum_{l=0}^k{\fr{k!}{l!}}\prod\limits_{j=a}^{a+k-l}\fr{1}{j+is}={}\\
{}=
\sum\limits_{l=0}^{k}\fr{k!}{l!}\sum\limits_{j=0}^{k-l}
\fr{(-1)^j}{(a+j+is)(k-l-j)! j!}={}\\
{}
=\sum\limits_{j=0}^{k}\fr{(-1)^j}{a+j+is}\sum\limits_{l=0}^{k-j}
\begin{pmatrix}
k\\ j
\end{pmatrix}
\begin{pmatrix}
k-j\\  l
\end{pmatrix}={}\\
{}=
\sum\limits_{j=0}^k\fr{(-1)^j}{a+j+is}
\begin{pmatrix}
k\\ j
\end{pmatrix}
2^{k-j}\,.
\end{multline*}
Теперь приступим к доказательству основного утверждения леммы.
Рассмотрим сначала случай $n=2k+1$.
Плотность $(k+1)$-й порядковой статистики, как известно,
выражается формулой (см.~\cite{4ben})
\begin{equation*}
p_{2k+1}(x) = (2k+1)
\begin{pmatrix}
2k\\  k\end{pmatrix}
f(x)(F(x)(1-F(x))^k\,,
%\label{3.5b}
\end{equation*}
где $f(x)$ и  $F(x)$~--- соответственно плотность и
функция распределения исходных случайных величин.

Найдем характеристическую функцию~$\phi_{2k+1}(s)$ выборочной
медианы~$\zeta_{2k+1}$:
\begin{multline*}
\phi_{2k+1}(s)=\e e^{is\zeta_{2k+1}}=
\int\limits_{-\infty}^{\infty}e^{isx}f(x)\,dx={}\\
{}=
(2k+1)
\begin{pmatrix}
2k\\  k\end{pmatrix}
2^{-(k+1)}\times{}\\
{}\times
\sum\limits_{j=0}^k (-1)^j 2^{-j}
\begin{pmatrix}
k\\ e j\end{pmatrix}
\fr{2(k+1+j)}{(k+1+j)^2+s^2}\,.
%\label{e3.6b}
\end{multline*}
Теперь найдем характеристическую функцию~$f_{2k+1}(s)$ случайной величины, определенной\linebreak\vspace*{-12pt}\pagebreak

\noindent
в правой части  формулы~(\ref{e3.1b}).
С учетом того, что
 характеристическая функция стандартной экспоненциальной
случайной величины равна $1/(1-is)$, имеем
\begin{multline*}
f_{2k+1}(s)={}\\
{}=
\sum\limits_{l=0}^{2k+1}\e \exp \left(is\delta_{2k+1}
\sum\limits_{j=k+1}^{K_{2k+1}}\fr{W_j}{j}\right)\mathbf{1}(B_{2k+1}=l)={}
\\
=2^{-(2k+1)}\left(\sum\limits_{l=0}^k \begin{pmatrix}
2k+1\\  l\end{pmatrix}
\prod\limits_{j=k+1}^{2k+1-l}\fr{j}{j+is}+{}\right.\\
\left.{}+
\sum\limits_{l=k+1}^{2k+1}
\begin{pmatrix}
2k+1\\ l\end{pmatrix}
\prod\limits_{j=k+1}^{l}\fr{j}{j-is}
\right)={}\\
{}
=2^{-(2k+1)}(2k+1)
\begin{pmatrix}
2k\\ k\end{pmatrix}
\sum\limits_{l=0}^k\fr{k!}{l!}
\left(\prod\limits_{j=k+1}^{2k+1-l}\fr{1}{j+is} +{}\right.\\
\left.{}+
\prod\limits_{j=k+1}^{2k+1-l}\fr{1}{j-is}\right)\,.
%\label{e3.7b}
\end{multline*}
Применяя формулу~(\ref{e3.4b}), получаем
\begin{multline*}
f_{2k+1}(s)=(2k+1)
\begin{pmatrix}
2k\\  k
\end{pmatrix}
2^{-(k+1)}\times{}\\
{}\times
\sum\limits_{j=0}^k(-1)^j 2^{-j}
\begin{pmatrix}
k\\  j\end{pmatrix}
\fr{2(k+1+j)}{(k+1+j)^2+s^2}\,.
%\label{e3.8b}
\end{multline*}
Значит, $f_{2k+1}(s)\equiv\phi_{2k+1}(s)$ и представление~(\ref{e3.1b}) доказано.
\medskip

Перейдем теперь к рассмотрению случая четного $n=2k$.
Совместная плотность двух порядковых статистик~$X_{(k)}$ и~$X_{(k+1)}$
определяется формулой (см.~\cite{4ben})
\begin{equation*}
p(x,y)=\fr{(2k)!}{((k-1)!)^2}\,(F(x)(1-F(y)))^{k-1}f(x)f(y)\,.
%\label{e3.9b}
\end{equation*}
Из этой формулы нетрудно получить, что плотность случайной величины
$$
\zeta_{2k}=\fr{X_{(k)}+X_{(k+1)}}{2}
$$
равна
\begin{multline*}
p_{2k}(x) = \fr{(2k)!}{2^k ((k-1)!)^2}\times{}\\
{}\times
\left(\sum_{j=0}^{k-2}\fr{(-1)^j
\begin{pmatrix}
k-1\\ j
\end{pmatrix}
2^{-j}}{k-1-j}
e^{-(k+1+j)|x|}\times{}\right.
\end{multline*}
\begin{multline}
\left.{}\times \left(1-e^{-(k-1-j)|x|}\right)- \right.{}\\
{}\left.
- \fr{(-1)^k}{2^{k-1}}|x|e^{-2k|x|} + \fr{1}{k2^k}e^{-2k|x|}
\vphantom{\fr{(-1)^j
\begin{pmatrix}
k-1\\ j
\end{pmatrix}
2^{-j}}{k-1-j}}\right)\,.
\label{e3.10b}
\end{multline}
Подробный вывод этой формулы приведен в работе~\cite{8ben}.
Исходя их формулы~(\ref{e3.10b}), найдем характеристическую функцию~$\phi_{2k}(s)$
выборочной медианы~$\zeta_{2k}$:
\begin{multline*}
\!\phi_{2k}(s)=
\fr{(2k)!}{2^k ((k-1)!)^2}
\left( \sum\limits_{j=0}^{k-2}
\fr{(-1)^j
\begin{pmatrix}
k-1\\ j
\end{pmatrix}
2^{-j}}{k-1-j}\times{}\right.
\\
\left.{}\times
\left(
\fr{2(k+1+j)}{(k+1+j)^2+s^2}  -
 \fr{4k}{4k^2+s^2} \right)-{}\right.\\
\left. {}- 
 \fr{(-1)^k}{2^{k-2}(4k^2+s^2)} + \fr{1}{2^{k-2}(4k^2+s^2)}
 \vphantom{\sum\limits_{j=0}^{k-2}
\fr{(-1)^j
\begin{pmatrix}
k-1\\ j
\end{pmatrix}
2^{-j}}{k-1-j}}
\right)\,.
%\label{e3.11b}
\end{multline*}
Найдем теперь характеристическую функ-\linebreak цию~$f_{2k}(s)$ случайной величины,
определенной
 в правой части формулы~(\ref{e3.2b}). Учитывая формулу~(\ref{e3.4b}), получим
\begin{multline*}
f_{2k}(s)=\sum\limits_{l=0}^{k-1}{\p(B_{2k}=l)
\fr{2k}{2k+is}\prod\limits_{j=k+1}^{2k-l}{\fr{j}{j+is}}}+{}\\
{}+
\sum\limits_{j=k+1}^{2k}{\p(B_{2k}=l)\fr{2k}{2k-is}\prod\limits_{j=k+1}^{2k-l}\fr{j}{j-is}}+{}\\
{}+
\p(B_{2k}=k)\fr{4k^2}{4k^2+s^2}={}\\
{}=
\fr{(2k)!}{2^k ((k-1)!)^2} \left( \fr{1}{2^{k-2}(4k^2+s^2)}
+{}\right.\\
\left.{}+2^{1-k}\sum\limits_{l=0}^{k-1}(-1)^l 2^{k-l-1}
\begin{pmatrix}
k-1\\ l\end{pmatrix}\times\right.{}\\
{}\times
\left( \fr{1}{(2k+is)(k+1+l-is)}+{}\right.\\
\left.\left.{}+ 
\fr{1}{(2k-is)(k+1+l-is)}\right) \right)\,.
\end{multline*}
Применяя при $l \ne k-1$ следующее соотношение:
\begin{multline*}
\fr{1}{(2k+is)(k+1+l+is)}={}\\
{}=
\fr{1}{k-1-l}\left( \fr{1}{k+1+l+is} - \fr{1}{2k+is}\right)\,,
\end{multline*}
получаем равенство

\noindent
\begin{multline*}
f_{2k}(s)=
\fr{(2k)!}{2^k ((k-1)!)^2}
\left( \sum\limits_{j=0}^{k-2}
\fr{(-1)^j 
\begin{pmatrix}
k-1\\ j
\end{pmatrix}
2^{-j}}{k-1-j}\times{}\right.\\
\left.{}\times
\left(
\fr{2\left(k+1+j\right)}{(k+1+j)^2+s^2} 
-  \fr{4k}{4k^2+s^2} \right)
-{}\right. \\
\left.{}- \fr{\left(-1\right)^k}{2^{k-2}\left(4k^2+s^2\right)} + \fr{1}{2^{k-2}(4k^2+s^2)}
\vphantom{\sum_{j=0}^{k-2}
\fr{(-1)^j 
\begin{pmatrix}
k-1\\ j
\end{pmatrix}
2^{-j}}{k-1-j}}
\right)\,.
%\label{e3.12b}
\end{multline*}
Таким образом,  $\phi_{2k}(s)\equiv f_{2k}(s)$ и утверждение~(\ref{e3.2b})
доказано.~$\Box$

{\small\frenchspacing
{%\baselineskip=10.8pt
\addcontentsline{toc}{section}{Литература}
\begin{thebibliography}{9}

\bibitem{3ben} %1
\Au{Королев Р.\,А., Тестова  А.\,В., Бенинг~В.\,Е.} 
О мощ\-ности асимптотически оптимального критерия в случае 
распределения Лапласа~// Вестник Тверского Государственного Университета, 
серия Прикладная математика, 2008. Вып.~8. №\,4(64). С.~5--23.

\bibitem{9ben} %2
\Au{Takeuchi K.} 
Asymptotic theory of statistical estimation.~---  Tokyo, 1974. (In Japanese.)

\bibitem{1ben} %3
\Au{Бурнашев М.\,В.} 
Асимптотические разложения для 
медианной оценки параметра~// Теор. вероятн. и ее
прим., 1996. Т.~41. Вып.~4. С.~738--753.

\bibitem{5ben}  %4
\Au{Kotz S., Kozubowski~T.\,J., Podgorski~K.}
The Laplace distribution and generalizations: 
A revisit with applications to communications, economics, engineering, 
and finance.~--- Birkhauser, 2001.  P.~349.

\bibitem{6ben}  %5
\Au{Бенинг В.\,Е., Королев В.\,Ю.}
Некоторые статистические  задачи, связанные с распределением Лапласа~// 
Информатика и её применения, 2008. Т.~2.  Вып.~2. С.~19--34.

\bibitem{7ben}  %6
\Au{Леман Э.} 
Проверка статистических гипотез.~--- М.: Наука, 1964. 498~с.

\bibitem{2ben} %7
\Au{Королев Р.\,А., Бенинг В.\,Е.}
Асимптотические 
разложения для мощностей критериев в случае распределения Лапласа~//
Вестник Тверского Государственного Университета, серия 
Прикладная математика, 2008. Вып.~3(10). №\,26(86). С.~97--107.

\bibitem{4ben} %8
\Au{David H.\,A., Nagaraja H.\,N.}
Order Statistics.  3rd ed.~--- New Jersey: Wiley, 2003.  P.~458.

\label{end\stat}

\bibitem{8ben} %9
\Au{Asrabadi B.\,R.} 
The exact confidence interval for 
the scale parameter and the MVUE of the Laplace distribution~// 
Communications in statistics. Theory and methods, 1985. Vol.~14. No.\,3. 
P.~713--733.

 \end{thebibliography}
}
}
\end{multicols}  %8
\def\stat{shevtsova}

\def\tit{ОБ АБСОЛЮТНЫХ КОНСТАНТАХ В НЕРАВЕНСТВЕ БЕРРИ--ЭССЕЕНА И ЕГО
СТРУКТУРНЫХ И~НЕРАВНОМЕРНЫХ~УТОЧНЕНИЯХ$^*$}

\def\titkol{Об абсолютных константах в неравенстве Берри--Эссеена и его
структурных и неравномерных уточнениях}

\def\autkol{И.\,Г.~Шевцова}

\def\aut{И.\,Г.~Шевцова$^1$}

\titel{\tit}{\aut}{\autkol}{\titkol}

{\renewcommand{\thefootnote}{\fnsymbol{footnote}}\footnotetext[1]
{Работа поддержана
грантом МК--2256.2012.1, а также Российским фондом фундаментальных
исследований (проекты 12-01-31125-а, 11-01-00515а, 11-07-00112а,
12-07-00115а).}}

\renewcommand{\thefootnote}{\arabic{footnote}}
\footnotetext[1]{Факультет вычислительной математики и
кибернетики Московского государственного университета им.\
М.\,В.\,Ломоносова; Институт проблем информатики Российской академии
наук, ishevtsova@cs.msu.su}


\Abst{Для равномерного расстояния $\Delta_n$ между
функцией распределения (ф.р.)\ стандартного нормального закона и
ф.р.\ нормированной суммы~$n$ независимых случайных величин (с.в.)\
$X_1,\ldots,X_n$ с $\e X_j=0$, $\e X_j^2=\sigma_j^2$,
${j=1,\ldots,n}$, при всех $n\ge1$ приведены оценки
$$
\ud\le \min\{0{,}5583 \ell_n,\, 0{,}3723(\ell_n+0{,}5\tau_n),
\,0{,}3057(\ell_n+\tau_n)\},
$$
$$
\ud\le \min\{0{,}4690\ell_n,\, 0{,}3322(\ell_n+0{,}429\tau_n),
\,0{,}3031(\ell_n+0{,}646\tau_n)\}, \text{ если } X_1\eqd\cdots\eqd X_n,
$$
где $\ell_n\hm=\sum\limits_{j=1}^n\e|X_j|^3$, $\tau_n\hm=\sum\limits_{j=1}^n\sigma_j^3$,
$\sum\limits_{j=1}^n\sigma_j^2\hm=1$. Получены уточненные результаты для
случая симметричного распределения слагаемых. Также показано, что в
неравенстве На\-га\-ева--Би\-кя\-ли\-са (неравномерном аналоге неравенства
Бер\-ри--Эс\-се\-ена) абсолютная константа не превосходит 21,82 в общем
случае и 17,36 в случае одинаково распределенных слагаемых.}

\vspace*{3pt}

\KW{центральная предельная теорема; оценка
скорости сходимости; нормальная аппроксимация; неравенство
Бер\-ри--Эс\-се\-ена; неравенство На\-га\-ева--Би\-кя\-ли\-са; абсолютная константа}

\vspace*{4pt}

\vskip 14pt plus 9pt minus 6pt

      \thispagestyle{headings}

      \begin{multicols}{2}

            \label{st\stat}


Обозначим $\F_3$ множество всех ф.р.~$F$
с.в.~$X$ с ${\e X\hm=0}$ и ${\e|X|^3\hm< \infty}$.
Через $\F_{3,\,s}$ обозначим множество всех симметричных ф.р.\ из
$\F_3$. Пусть $X_1,X_2,\ldots,X_n$~--- независимые с.в.\ с ф.р.\
$F_1,\ldots,F_n\in\F_3$ соответственно. Положим $\sigma_j^2\hm= \e
X_j^2,$ $\beta_{3,\,j}\hm=\e|X_j|^3,$ $j\hm=1,\ldots,n,$
$\sum\limits_{j=1}^n\sigma_j^2\hm=1$, $\ell_n\hm=\sum\limits_{j=1}^n\beta_{3,\,j}$,
$\tau_n\hm=\sum\limits_{j=1}^n\sigma_j^3$,
$\overline F_n(x)\hm=\p(X_1+\cdots+X_n\hm<xB_n)$. Пусть $\phi(x)$ и
$\Phi(x)$~--- соответственно плотность и ф.р.\ стандартного
нормального закона,
\begin{align*}
\nud&=|\overline F_n(x)-\Phi(x)|\,,\quad x\in\R\,;\\
\ud &=\sup\limits_x|\overline F_n(x)-\Phi(x)|\,,\quad n=1,2,\ldots\,
\end{align*}

В работе~\cite{Shevtsova2012ISSPSM3} с использованием техники
преобразования смещения формы и новой оценки точности аппроксимации
характеристической функции первыми членами ее разложения в ряд
Тейлора для всех $n\hm\ge1$ были получены оценки:
\begin{multline*}
\ud\le \min\{0{,}5584\ell_n, 0{,}36266(\ell_n+0{,}54\tau_n),\\
0{,}3129(\ell_n+0{,}922\tau_n)\}\,,\quad F_1,\ldots,F_n\in\F_3\,;
\end{multline*}


\noindent
\begin{multline*}
\ud\le \min\{ 0{,}4693\ell_n, 0{,}3322(\ell_n+0{,}429\tau_n),\\
0{,}3031(\ell_n+0{,}646\tau_n)\}\,,\quad  F_1=\cdots=F_n\in\F_3\,;
\end{multline*}

\vspace*{-9pt}

\noindent
\begin{multline*}
\ud\le \min\{ 0{,}3425(\ell_n+0{,}63\tau_n),\\
0{,}2895(\ell_n+\tau_n) \}\,,\quad  F_1,\ldots,F_n\in\F_{3,\,s}\,;
\end{multline*}

\vspace*{-9pt}

\noindent
\begin{multline*}
\ud\le \min\{ 0{,}29489(\ell_n+0{,}587\tau_n)\,,\\
0{,}2730(\ell_n+0{,}732\tau_n)\}\,, \quad F_1=\cdots=F_n\in\F_{3,\,s}\,.
\end{multline*}

С помощью алгоритма, использованного в~\cite{Shevtsova2012ISSPSM3},
и оценки
$$
|f'(t)|\le \sigma\sin\left(\sigma|t| \wedge\fr{\pi}{2}\right)\,,\quad t\in\R\,,
$$
для производной характеристической функции $f(t)\hm=\e e^{itX}$ с $\e
X\hm=0$, $\sigma^2\hm\equiv\e X^2\hm<\infty$, вытекающей из результатов
работ~\cite{Rossberg1991, MatysiakSzablowski2001}, указанные
результаты можно уточнить и получить следующие оценки, справедливые
при всех $n\hm\ge1$:
\begin{multline*}
\ud\le \min\{0{,}5583\ell_n,\, 0{,}3723(\ell_n+0{,}5\tau_n),\\
\,0{,}3057(\ell_n+\tau_n)\}\,,\quad F_1,\ldots,F_n\in\F_3\,;
\end{multline*}

\vspace*{-12pt}

\noindent
\begin{multline*}
\ud\le \min\{0{,}4690\ell_n,\, 0{,}3322(\ell_n+0{,}429\tau_n),\\
\,0{,}3031(\ell_n+0{,}646\tau_n)\}\,,\quad F_1=\cdots=F_n\in\F_3\,;
\end{multline*}

\vspace*{-12pt}

\noindent
\begin{multline*}
\ud\le \min\{ 0{,}34245(\ell_n+0{,}63\tau_n),\\
\,0{,}2873(\ell_n+\tau_n)\}\,,\quad F_1,\ldots,F_n\in\F_{3,\,s}\,;
\end{multline*}

\vspace*{-12pt}

\noindent
\begin{multline*}
\ud\le \min\{ 0{,}29353(\ell_n+0{,}593\tau_n),\\
\,0{,}2730(\ell_n+0{,}729\tau_n)\}\,,\quad F_1=\cdots=F_n\in\F_{3,\,s}\,.
\end{multline*}

Кроме того, исправляя неточность, допущенную
в~\cite{NefedovaShevtsova2012}, и используя приведенные выше
неравенства, можно показать, что справедливы следующие неравномерные
оценки:
\begin{multline*}
\sup\limits_{x\in\R}(1+|x|^3)\nud\le \min\{21{,}82\ell_n,\,
18{,}19(\ell_n+\tau_n)\}\,,\\ F_1,\ldots,F_n\in\F_3\,;
\end{multline*}

\vspace*{-12pt}

\noindent
\begin{multline*}
\sup\limits_{x\in\R}(1+|x|^3)\nud\le \min\{ 17{,}36\ell_n,\\
15{,}70(\ell_n+0{,}646\tau_n)\}\,,\quad F_1=\cdots=F_n\in\F_3\,.
\end{multline*}
Более того, исправленный метод работы~\cite{NefedovaShevtsova2012}
позволяет построить монотонно убывающую функцию $C(t)$ с
$\lim\limits_{t\to\infty}C(t)\hm=1\hm+e\hm=3{,}71\ldots$ и такую, что
\begin{multline*}
\sup\limits_{|x|\ge t}|x|^3\nud\le C(t)\ell_n,\quad t\ge0,\quad n\ge1,\\
F_1,\ldots,F_n\in\F_3\,.
\end{multline*}
В частности, для $C(t)$ справедливы верхние оценки
\\
$C(0)\le21{,}26$, $C(4)\le17{,}19$, $C(5)\le12{,}35$, $C(10)\hm\le7{,}36$ для
любых $F_1,\ldots,F_n\in\F_3$;
\\
$C(0)\le16{,}90$, $C(4)\le14{,}58$, $C(5)\le11{,}56$, $C(10)\hm\le5{,}85$, если
$F_1\hm=\cdots=F_n\hm\in\F_3$.

С помощью методов, использованных в данной работе, можно получить
аналогичные равномерные и неравномерные оценки для случая, когда
слагаемые обладают моментами порядка лишь $2+\delta$ с некоторым
$0<\delta\le1$.

{\small\frenchspacing
{%\baselineskip=10.8pt
\addcontentsline{toc}{section}{Литература}
\begin{thebibliography}{9}

\bibitem{Shevtsova2012ISSPSM3}
\Au{Shevtsova I.} On the absolute constants in the
Berry--Esseen-type inequalities~// 30th  Seminar (International) on
Stability Problems for Stochastic Models (Svetlogorsk, 2012): Book
of Abstracts.~--- М.: ИПИ РАН, 2012. С.~71--72.



\bibitem{Rossberg1991}
\Au{Ro\!\!\!{\ptb{\ss}}\,berg~H.-J.} Positiv definite Verteilungsdichten~//
Appendix to:  \Au{Gnedenko B.\,W.} Einf$\ddot{\mbox{u}}$hrung in die
Wahrscheinlichkeitstheorie.~--- 9th ed.~--- Berlin:
Akademie--Verlag, 1991.

\bibitem{MatysiakSzablowski2001}
\Au{Matysiak~W., Szab\!\!{\ptb{\l}}owski~P.~J.} Some inequalities for
characteristic functions~// J.~Math. Sci., 2001. Vol.~105. No.\,6.
P.~2594--2598.

\label{end\stat}

\bibitem{NefedovaShevtsova2012}
\Au{Нефедова Ю.\,С., Шевцова И.\,Г.} О~неравномерных оценках
скорости сходимости в центральной предельной теореме~// Теория
вероятн. и ее примен., 2012, Т.~57. Вып.~1. С.~62--97.
\end{thebibliography}
}
}

\end{multicols}  %9
\def\stat{kozerenko}

\def\tit{КОГНИТИВНО-ЛИНГВИСТИЧЕСКИЕ ПРЕДСТАВЛЕНИЯ 
В~СИСТЕМАХ ОБРАБОТКИ ТЕКСТОВ}

\def\titkol{Когнитивно-лингвистические представления 
в~системах обработки текстов}

\def\autkol{Е.\,Б.~Козеренко, И.\,П.~Кузнецов}
\def\aut{Е.\,Б.~Козеренко$^1$, И.\,П.~Кузнецов$^2$}

\titel{\tit}{\aut}{\autkol}{\titkol}

%{\renewcommand{\thefootnote}{\fnsymbol{footnote}}\footnotetext[1]
%{Работа выполнена при поддержке Российского фонда фундаментальных
%исследований, проект~10-01-00480. Статья написана на основе материалов доклада, 
%представленного на IV Международном семинаре <<Прикладные задачи теории вероятностей 
%и математической статистики, связанные с моделированием информационных систем>> 
%(зимняя сессия, Аоста, Италия, январь--февраль 2010 г.).}}

\renewcommand{\thefootnote}{\arabic{footnote}}
\footnotetext[1]{Институт проблем информатики Российской академии наук, kozerenko@mail.ru}
\footnotetext[2]{Институт проблем информатики Российской академии наук, igor-kuz@mtu-net.ru}


\Abst{Рассмотрены вопросы проектирования и развития 
семантико-синтаксических и лексико-семантических представлений в 
лингвистических процессорах ряда систем, основанных на аппарате расширенных 
семантических сетей (РСС). Системы этого класса создаются для извлечения знаний из 
текстов на естественных языках, отображения извлеченных сущностей и связей в 
структуры базы знаний (БЗ) и использования знаний для поддержки экспертных 
аналитических решений в различных сферах приложения. В~фокусе внимания 
находятся ин\-же\-нер\-но-линг\-ви\-сти\-че\-ские представления, позволяющие 
построить целостную работающую лингвистическую модель, которая 
модифицируется в зависимости от конкретной задачи: от <<тяжелой>> формы на 
основе детальных глубинных представлений до фокусных редуцированных 
оболочек, настроенных на узкую предметную область (ПО) и ограниченный язык 
общения. Особое внимание уделяется способам описания 
дис\-три\-бу\-тив\-но-транс\-фор\-ма\-ци\-он\-ных признаков языковых объектов.}

\KW{интеллектуальные системы; семантические представления; лингвистические 
процессоры; обработка естественного языка; извлечение знаний}

       \vskip 14pt plus 9pt minus 6pt

      \thispagestyle{headings}

      \begin{multicols}{2}

      \label{st\stat}

\section{Введение}

     Данная работа посвящена проблемам создания\linebreak 
     когни\-тив\-но-линг\-ви\-сти\-че\-ских моделей естественного языка для 
различных классов информационных систем и описанию опыта создания 
линг\-ви\-сти\-че\-ских представлений для интеллектуальных\linebreak технологий 
обработки текстов. Вопросы извлечения знаний из текстов и создания модели 
естественного языка рассматриваются в единстве. В центре внимания будут 
находиться лингвистические процессоры интеллектуальных систем, 
разработанных на основе аппарата \textit{расширенных семантических 
сетей}~[1--5]. %\cite{1koz}--\cite{3koz}, \cite{18koz}--\cite{19koz}. 
Будем 
называть их \textit{РСС-сис\-те\-мы}. Эти системы создавались коллективом 
разработчиков, включая авторов данной статьи в Институте проб\-лем 
информатики РАН на протяжении целого ряда лет в рамках 
исследовательских проектов и прикладных систем, ориентированных на 
конкретные ПО заказчиков. Можно выделить четыре 
поколения РСС-систем. Ко\-гни\-тив\-но-линг\-ви\-сти\-че\-ские 
представления, заложенные в основу систем этого класса, прошли 
определенный эволюционный путь. 
     
     Интеллектуальные РСС-сис\-те\-мы содержат развитые \textit{базы 
знаний}, при этом знания представлены в виде записей на языке 
РСС, называемых 
     \textit{РСС-струк\-ту\-ра\-ми}. Лингвистические знания, таким 
образом, являются частным случаем <<знаний>> и также представлены в 
виде записей на языке РСС. Основным 
конструктивным элементом РСС\linebreak является именованный $N$-мест\-ный 
предикат, на\-зы\-ва\-емый <<\textit{фрагментом}>>. Все множество языковых 
объектов задается в виде системы пре\-ди\-кат\-но-ак\-тант\-ных структур, при этом 
поддерживаются механизмы представления вложенных структур, что дает 
очень мощные изобразительные возможности для описания объектов 
различных языковых уровней. Очень важными факторами являются 
однородность и единообразие лингвистических представлений. 
     
     В процессе анализа и синтеза предложений естественного языка 
используется фор\-маль\-но-грам\-ма\-ти\-че\-ский аппарат, сходный с 
грамматиками зависимостей. При этом подходе опорными элементами 
служат слова и конструкции, выполняющие роль предикатов в предложении, 
и результатом анализа предложения должен стать один предикат, 
соответствующий сказуемому рассматриваемого предложения (т.\,е.\ 
основному глаголу в личной форме или другому основному предикатному 
выражению). Таким образом, в процессе анализа происходит выявление 
\textit{когнитивных опор} предложения: <<слов-дейст\-вий>> и 
     <<слов-от\-но\-ше\-ний>>, т.\,е.\ глаголов и других слов, имеющих 
синтактико-семантические валентности. Примером <<слов-от\-но\-ше\-ний>> 
могут служить, например, слова <<отец>>, <<друг>> и~т.\,п., т.\,е.\ в данном 
случае <<отношения>> (или \textit{функции}~--- в терминах языка логики 
предикатов 1-го порядка)~--- это слова, которые задают сильные, четко 
выраженные син\-так\-ти\-ко-се\-ман\-ти\-че\-ские ожидания. 
     
     Семантический анализ в ин\-же\-нер\-но-линг\-ви\-сти\-че\-ском 
понимании~--- это процесс перевода ес\-тест\-вен\-но-язы\-ко\-вых 
выражений во <<внутренние>> структуры БЗ, в 
рассматриваемой ситуации этими <<внутренними>> структурами являются 
записи на языке РСС. Таким образом, структуры БЗ~--- это код смысла в 
интеллектуальных информационных системах подобного рода. 
     
     В работе рассматриваются ин\-же\-нер\-но-линг\-ви\-сти\-че\-ские 
решения в системах с <<пол\-ным>> линг\-ви\-сти\-че\-ским анализом~--- это 
     сис\-те\-мы 1-го и 2-го поколения: ДИЕС1, ДИЕС2, 
     Логос-Д~\cite{2koz, 3koz}~--- и сис\-те\-мах с <<фактографическим>> 
подходом: интеллектуальных системах поддержки аналитических решений 
(ИСПАР)~\cite{18koz, 19koz}, где целью анализа является выделение 
сущностей и связей из текстов,~--- это системы 3-го и 4-го поколения. 

\section{Процесс концептуально-лингвистического моделирования 
в системах, основанных на аппарате расширенных семантических сетей}
     
\subsection{Центральные вопросы семантического моделирования} %2.1
     
     Концептуально-лингвистическое моделирование (КЛМ)~--- это 
процесс построения ес\-тест\-вен\-но-язы\-ко\-вой модели ПО (рис.~1), синтезирующий в себе подходы 
концептуального и лингвистического моделирования~[1--3]. 
По\-стро\-ение концептуально-лингвистической модели некоторой 
ПО подразделяется на следующие этапы:
     \begin{itemize}
     \item построение собственно концептуальной модели, т.\,е.\ вычленение 
базовых понятий, организация их в ро\-до-ви\-до\-вые деревья и определение 
связей между ними;
     \item разработка идеографического словаря ПО, т.\,е.\ 
лексическое наполнение концептуальной модели;
     \item ввод базовых правил, описывающих на естественном языке 
<<модель мира>>, релевантную данной ПО.
     \end{itemize}
     
     
     Методика КЛМ на 
основе аппарата РСС базируется на следующих принципах:
     \begin{itemize}
\item модель должна быть <<открытой>>, т.\,е.\ поддерживать эффективный 
механизм расширения и обновления информации;
\begin{center} %fig1
%\vspace*{3pt}
\hspace*{-10.7158pt}\mbox{%
\epsfxsize=77.871mm
\epsfbox{koz-1.eps}
}\hspace{10.7158pt}
%\end{center}
\vspace*{4pt}
%\begin{center}
{{\figurename~1}\ \ \small{Процесс КЛМ}}
\end{center}
\vspace*{3pt}

%\bigskip
\addtocounter{figure}{1}
\item модель представления <<смысла>> должна учитывать факты 
экстралингвистической реаль\-ности, которые в виде правил и отношений 
составляют некоторую базовую <<модель мира>>, достраиваемую 
конкретными моделями ПО;
\item модель должна быть практичной, т.\,е.\ не перегруженной детальными 
описаниями связей и отношений между понятиями, чтобы обеспечить 
возможность ее реализации, но в то же время отражать всю релевантную 
конкретной задаче информацию.
\end{itemize}

     \begin{figure*} %fig2
%     \begin{center}
\hspace*{23mm}\{(ВЫРАБАТЫВА895\_\_)(DICSEM)\\
\hspace*{23mm}COORD(PROGNOZ1,RUS,ВЫРАБАТЫВА895\_\_,S50\_31\_51\_20,\%)\\
\hspace*{23mm}SUB(UNIV,0+)~SUB(UNIV,1+)~SUB(UNIV,2+)\\
\hspace*{23mm}ВЫРАБАТЫВ(0-,1-,2-/3+)~INFI(3-)~ПРИДЕТСЯ(3-)~ПРИДЕТСЯ(3$-$/4+) \\
\hspace*{23mm}FUT1(4$-$)~SUB(СРЕД,5+)
%\end{center}
%\vspace*{2pt}
\Caption{Пример записи представления глагола <<вырабатывать>> в семантическом 
словаре
\label{f2koz}}
%\vspace*{6pt}
\end{figure*}

     Реалистичный подход к постановке задачи диктует необходимость 
ограничения моделируемого подмножества естественного языка. Суть 
ограничений сводится к следующему:
     \begin{enumerate}[(1)]
     \item анализируемые текстовые материалы содержат 
экспертные знания из конкретных ПО (в разработанных 
авторами системах это были такие ПО, как диагностика 
брака при изготовлении микросхем, социальное прогнозирование, 
криминалистика и другие);
     \item в целях максимально возможного устранения 
неоднозначности словарь строится по модульному принципу: есть некоторая 
наиболее общая часть (1--2~уровня), которая достраивается специальными 
словарями для каж\-дой отдельной~ПО.
     \end{enumerate}
     
     Предлагаемая модель лексической семантики основана на принципе 
<<ядерного>> значения, реализуемого в контексте данной 
ПО, с последующим индуктивным наращиванием других значений (если 
они актуализируются в рас\-смат\-ри\-ва\-емых контекстах). Также используется 
таксономия, которая реализуется в виде иерархических деревьев классов 
слов. 
     
     Общая <<модель мира>> системы является основой для моделей ПО. 
Элементами этой модели служат классы слов, которые подразделяются на 
понятия/имена, отношения, действия, свойства, характеристики действий, 
временные и пространственные характеристики.
     
     Самым общим понятием является \textit{концепт}, или 
\textit{универсальный класс}, который подразделяется на объект, ситуацию, 
процесс и~др. 
     
     Слова, относящиеся к классам действий и отношений, представлены 
как се\-ман\-ти\-ко-син\-так\-си\-че\-ские фреймы, задающие 
     пре\-ди\-кат\-но-ак\-тант\-ные структуры (модель управления). Однако 
в описываемом подходе (назовем его РСС-под\-хо\-дом) существенно 
расширена область значений актантов. Суть расширения состоит, во-первых, 
в том, что в роли актантов могут выступать не только простые объекты, 
соответствующие отдельным словам, но и структурные объекты, 
представляющие словосочетания и фразы, а во-вторых, в том, что понятие 
падежа включает в себя не только семантические, но и синтаксические 
признаки.
     
     Подход, основанный на РСС, позволяет отражать произвольный 
уровень вложенности структур за счет пропозициональных вершин 
семантической сети. Это обеспечивает представление\linebreak сложных 
синтаксических конструкций фраз\linebreak естественного языка, а также позволяет 
отразить\linebreak структурный характер лексической семантики,\linebreak которая в 
предлагаемой модели имеет иерар\-хи\-че\-ски-се\-те\-вую структуру. 
Линг\-ви\-сти\-че\-ские зна-\linebreak ния пред\-став\-ле\-ны в системном словаре и 
декла\-ра\-тивных модулях линг\-ви\-сти\-че\-ско\-го процессора.\linebreak В РСС-сис\-те\-мах 
так\-же реализована функция динамически форми\-ру\-емо\-го семантического 
словаря, который на основе исходной лингвистической информации 
достраивается системой автоматически в процессе об\-ра\-бот\-ки конкретных 
текстов. На рис.~\ref{f2koz} пред\-став\-ле\-но \mbox{такое} <<внутреннее>> описание 
глагола в семантическом словаре. Этот словарь автоматически генерируется 
РСС-системами ДИЕС2, ЛОГОС-Д, ИКС в процессе обработки 
     естест\-вен\-но-язы\-ко\-вых \mbox{текстов}. 
     {\looseness=1
     
     }
     
     
\subsection{Особенности применения аппарата расширенных семантических сетей 
в~когнитивно-лингвистическом моделировании} %2.2
     
     Дадим краткое описание аппарата РСС и  
обос\-ну\-ем выбор именно этого метода представления для моделирования 
естественного языка. Классическое понятие семантической сети сводится к 
следующему: задаются некоторые вершины, соответствующие объектам,  
вершины связываются дугами, которые помечаются именами отношений. 
Однако с помощью подобных сетей оказывается трудно представлять 
сложные виды информации, например, когда объекты, связанные 
отношениями, образуют агрегаты и когда отношения связываются между 
собой отношениями и~др. Поэтому в сети вводятся вершины, 
соответствующие именам отношений, а также специальный композиционный 
элемент, называемый вершиной связи. Вершина связи как бы <<разрывает>> 
дугу и подсоединяется одним ребром к вершине-отношению, а другими 
ребрами~--- к вершинам-объектам. Расширенная семантическая сеть является развитием такого сорта 
сетей в направлении повышения изобразительных возможностей при 
сохранении свойства однородности.
     
     Основой РСС является множество вершин ($V$), из которых 
составляются элементарные фрагменты (ЭФ) вида
     $
     V_0(V_1,V_2,\ldots ,V_k/V_{k+1})
     $, 
     где
$V_0, V_1, V_2,\ldots , V_k, V_{k+1}>0$.
     
     
     Такой фрагмент представляет $k$-местное отношение. Позиции 
вершин в ЭФ определяют их роли. 
Вершина~$V_0$ ставится в соответствие имени отношения, 
вершины~$V_1$, $V_2$, \ldots , $V_k$~--- объектам, участ\-ву\-ющим в 
отношении, а вершина~$V_{k+1}$, отделенная косой линией,~--- всей 
совокупности упомянутых объектов с учетом их отношения. В~дальнейшем 
будем $V_{k+1}$ называть $C$-вершиной ЭФ.\linebreak 
Множество ЭФ образует РСС. 
С~помощью РСС представляются наборы отношений, различные ситуации, 
сце\-нарии. Сильной стороной РСС-под\-хо\-да является возможность 
однородного пред\-став\-ле\-ния как предметной (концептуальной), так и 
лингвистической информации, что обеспечивает эффективную обработку 
знаний и поддержание непротиворечи\-вости~БЗ.
          \begin{figure*} %fig3
     \vspace*{1pt}
\begin{center}
\mbox{%
\epsfxsize=125.039mm
\epsfbox{koz-3.eps}
}
\end{center}
\vspace*{-9pt}
     \Caption{Семантико-синтаксический анализ без выявления глагольных 
словоформ
      \label{f3koz}}
\vspace*{12pt}
 %     \end{figure*}
%            \begin{figure*} %fig4
           \vspace*{1pt}
\begin{center}
\mbox{%
\epsfxsize=103.129mm
\epsfbox{koz-4.eps}
}
\end{center}
\vspace*{-9pt}
      \Caption{Целостная семантическая структура предложения
      \label{f4koz}}
      \end{figure*}

     
     Посредством РСС в БЗ представлены лингвистические  и 
предметные знания. Обработка этих знаний осуществляется 
продукциями языка ДЕКЛ, на котором реализованы сле\-ду\-ющие шесть 
блоков: морфологического анализа, семанти\-ческого анализа слов, 
син\-так\-ти\-ко-се\-ман\-ти\-че\-ско\-го анализа форм, 
прагматических функций, организации системной активности и 
обратный лингвистический процессор. С~помощью продукций 
осущест\-вля\-ет\-ся последовательное преобразование сети~--- РСС. При этом 
проходятся фазы, соответствующие уровню понимания входного текста. 
Рас\-смот\-рим~их.
     \begin{enumerate}[1.]
     \item На первом шаге анализа строится 
пространственная структура предложения с морфологической информацией 
для каждого слова.\linebreak Каж\-дый член предложения представляется вершиной 
семантической сети. Вместо слова генерируется код (если слово 
многозначно, т.\,е.\ принадлежит к нескольким классам,~--- то более одного 
кода). Основой кода служит корень слова. На этом этапе предложение 
представляется в виде набора фрагментов типа LRR (специальных меток 
результатов 1-го этапа анализа), объединяемых в целостную структуру 
посредством вершины связи. Результат 1-го этапа постоянно обращается к 
словарю: <<Что значит данное слово?>>
     \item На втором этапе каждой вершине сопоставляется семантический 
класс и присваивается новый код. За словами (т.\,е.\ конкретными вершинами 
РСС) система видит объекты, действия, свойства, т.\,е.\ строит 
классификации. Производится се\-ман\-ти\-ко-син\-так\-си\-че\-ский анализ 
без выявления глагольных словоформ, при этом предложение представляется 
в виде совокупности фрагментов типа SEM и SEMD~--- специальных меток 
результатов 2-го этапа анализа (рис.~\ref{f3koz}).
     \item На третьем этапе происходит частичное <<сворачивание>> 
синтаксических структур в более компактные (например, свойство объекта и 
сам объект) с присваиванием нового кода и строится фрагмент для объекта, 
обладающего этим свойством.
     \begin{figure*}[b] %fig5
          \vspace*{12pt}
\begin{center}
\mbox{%
\epsfxsize=147.485mm
\epsfbox{koz-5.eps}
}
\end{center}
\vspace*{-9pt}
     \Caption{Глубинная структура предложений
      \label{f5koz}}
      \end{figure*}      
     \item На четвертом этапе выявляются отношения и действия и 
производится анализ непосредственного контекста на соответствие заданным 
семантическим падежам. Система проверяет, подходят ли объекты 
(концепты, понятия) на аргументные места данного действия или отношения. 
При этом отглагольные существительные (<<делатель>>, т.\,е.\ агент 
действия, или <<делание>>~--- процесс~--- анализируются как слова с 
двойной природой: вначале как действия, а затем как объекты). Результатом 
этого этапа является целостная семантическая структура предложения, 
которая представляется фрагментом типа SEMSTR~--- метки результата 4-го 
этапа анализа (рис.~\ref{f4koz}).
     \item На пятом этапе происходит анализ прагматики: установление 
кореференциальных отношений, частичное восстановление эллиптических 
конструкций, система производит дальнейшие действия с построенными 
фрагментами.
     \end{enumerate}

     
Система ДИЕС допускает ввод полисемичных форм глаголов. Для этого следует 
воспользоваться формальной записью лингвистических знаний. 
     В~сис\-те\-мах, основанных на РСС, все функции реализованы на 
единой основе~--- в рамках языков РСС и ДЕКЛ, которые были разработаны 
с ориентацией на задачи обработки естественного языка.

%\vspace*{-6pt}

\section{Представление семантики глаголов, глубинные 
и~поверхностные структуры}
     
     В процессе анализа выявляются семантические вершины предложения: 
происходит выявление <<слов-дей\-ст\-вий>>, т.\,е.\ глаголов, и 
     <<слов-от\-но\-ше\-ний>>. Что же является конструктивной основой\linebreak 
задания семантических представлений предикатных слов и выражений? Как 
убедительно показано в работе~\cite{4koz}, семантика глагола 
определяется его дис\-три\-бу\-тив\-но-транс\-фор\-ма\-ци\-он\-ны\-ми\linebreak 
свойствами. Поэтому смысл предикатных выражений должен кодироваться с 
учетом их дистрибутивных и трансформационных признаков. 
     
     Выдвинутая рядом лингвистов (Хомский, Филлмор) гипотеза о том, что 
все предложения имеют глубинные и поверхностные 
     структуры~[7--10], явилась очень продуктивным 
источником проектных решений при создании первых РСС-сис\-тем и 
развивалась в дальнейшем. 

В~тео\-ре\-ти\-ко-линг\-ви\-сти\-че\-ском 
понимании глубинная структура~--- это абстракция, содержащая все 
элементы, необходимые для образования поверхностных структур 
предложений со сходной семантикой. 

     В~ин\-же\-нер\-но-линг\-ви\-сти\-че\-ском понимании\linebreak глубинная 
структура~--- это запись на языке БЗ, например на языке РСС, 
которая может быть представлена в <<поверхностном>> виде на одном из 
естественных языков в результате конечного числа определенных 
преобразований. Например, предложения

\noindent
\begin{align*}    
(1)\ &\mbox{\textit{The programmer writes the code}}\\
(2)\ &\mbox{\textit{The code is written by the programmer}}
\end{align*}
имеют истоком одну глубинную структуру:

\medskip

\noindent
     \begin{verbatim}
  Programmer <---- write ----> Code
      agent                   object,
\end{verbatim}

\medskip

\noindent
хотя и отличаются своими поверхностными структурами. В~каждом из них 
имеется агент (the programmer), объект (the code) и действие (write).\linebreak Согласно 
концепции \textit{падежной грамматики} Филлмора~\cite{5koz} глубинная 
структура для обоих предложений инвариантна. Эту структуру можно 
представить в виде скобочной записи $V(\mathrm{AGENT}, \mathrm{OBJECT})$. В~графическом 
виде глубинная структура предложения также может быть представлена 
диаграммой в виде дерева, где отражены инвариантные отношения 
зависимости между предикатной вершиной и актантами (рис.~\ref{f5koz}), 
причем в таком представлении явным образом разграничиваются 
\textit{модальность} (MOD) и \textit{пропозиция} (PROP).
     

     В исходном варианте~\cite{5koz} теория признавала шесть падежей: 
агентив, инструменталис, датив, объектив, локатив и фактитив. По мере 
развития теории~\cite{8koz} происходило увеличение числа падежей, однако 
<<умножение>> количества падежей утяжеляет первоначальную 
конфигурацию, поэтому при построении инженерных семантических 
представлений требуется некоторый <<компромиссный>> вариант, 
сочетающий в себе необходимую полноту, с одной стороны, и простоту и 
гибкость, с другой.

\begin{figure*}[b] %fig6
\vspace*{24pt}
\begin{center}
\mbox{%
\epsfxsize=156.873mm
\epsfbox{koz-6.eps}
}
\end{center}
%\vspace*{-9pt}
\Caption{Обобщенное функциональное представление систем ИСПАР
\label{f6koz}}
\end{figure*}
     
%\vspace*{-6pt}

\section{Некоторые базовые аспекты построения многоязычных 
систем}
     
     Одним из приоритетных направлений развития РСС-сис\-тем является 
обеспечение обработки текстов на нескольких языках, прежде всего для 
рус\-ско-анг\-лий\-ской языковой пары. В системах 2-го поколения~--- ДИЕС2, 
ИКС, ЛОГОС-Д были реализованы лингвистические процессоры и словари 
для русского и английского языка, позволявшие обрабатывать тексты для 
ряда ПО. При этом поддерживался как режим ввода 
лингвистических знаний линг\-вис\-том-ана\-ли\-ти\-ком, так и 
автоматический режим самообучения системы по вводимым \mbox{текстам}. 
{\looseness=1

}

Проводились также эксперименты с итальянским и французским языком. 
При создании многоязычных систем авторы обращались к европейским 
языкам. Очевидно, что европейские языки обладают большим числом общих 
правил, чем любой из них с языками других групп. Но при этом все 
естественные языки обладают общей структурой на самом глубинном 
уровне. На этом уровне располагаются главные элементы естественного 
языка: \textit{предложение}, \textit{модальность}, \textit{пропозиция}.
     
     Моделирование смысловых представлений~--- это процесс, 
развивающийся в направлении от поверхностных семантических структур к 
глубинным. Поиск такого внутреннего представления смысла в условиях 
многоязычной ситуации является на\-прав\-ле\-ни\-ем развития методов 
     КЛМ на базе  РСС. 
     
%     \vspace*{-48pt}
     
\section{Интеллектуальные системы поддержки аналитических 
решений}
     
Системы РСС 3-го и 4-го поколения на\-прав\-ле\-ны на извлечение знаний 
в виде \textit{объектов}, или \textit{сущностей}, и связей между ними из 
пред\-мет\-но-ориен\-ти\-ро\-ван\-ных текстов на русском и английском 
языке~\cite{18koz, 19koz}.

    
В настоящее время во всем мире активно ведутся работы по созданию 
систем извлечения фактов из текстов на естественных языках~[11--14], создаются развитые тезаурусы и 
онтологии~\cite{17koz}. Сис\-те\-мы РСС функционально шире, поскольку 
имеют возможность не только извлекать факты, но и поддерживать 
механизмы логического анализа и экспертного вывода на основе 
извлеченных знаний. Сис\-те\-ма\-ми такого рода являются ИСПАР. В~целом это 
направление исследований требует дальнейшей проработки 
     лек\-си\-ко-се\-ман\-ти\-че\-ских представлений, создания 
     пред\-мет\-но-ориен\-ти\-ро\-ван\-ных семантических словарей. 

Обобщенное функциональное представление систем ИСПАР дано на 
рис.~\ref{f6koz}. 
     
     В рамках ИСПАР на основе РСС 
(\mbox{ИСПАР}--РСС) были реализованы полномасштабные и\linebreak пилотные 
проекты для ряда ПО: криминалистики, управления 
кадрами, мониторинга финансово-экономического кризиса и 
др.~\cite{18koz, 19koz}.

\section{Применение аппарата расширенных семантических сетей в~лингвистических 
исследованиях}
     
     В настоящее время в рамках проектов, на\-прав\-лен\-ных на создание 
открытых лингвистических ресурсов~\cite{20koz} для 
     на\-уч\-но-прак\-ти\-че\-ских целей, ведутся работы по выравниванию 
параллельных текстов научных статей, патентов и 
     фи\-нан\-со\-во-эко\-но\-ми\-че\-ских текстов. В~качестве одного из 
методов выравнивания используется РСС-под\-ход, поскольку он позволяет 
отразить глу\-бин\-но-се\-ман\-ти\-че\-ский уровень языковых структур. 

На  рис.~7 представлен фрагмент первого этапа лингвистического 
анализа в многоязычных системах. Для <<идеальной>> ситуации, когда 
структуры исходного текста и текста перевода практически совпадают, такая 
ситуация имеет место в меньшинстве случаев. Основные трудности 
возникают при наличии переводческих трансформаций в параллельных 
текстах. Особое внимание следует уделять гла\-голь\-но-имен\-ным 
трансформациям, например явлению \textit{номинализации}, поскольку она 
очень продуктивна для всех исследовавшихся языков.

     
     Ключевой задачей при разработке методов сопоставления 
параллельных текстов является выявление и детальное описание тех 
языковых трансформаций, которые имеют место при переводе 
     естест\-вен\-но-язы\-ко\-вых конструкций с одного языка на 
другой~\cite{9koz}, потому что далеко не всегда некое содержание 
передается струк\-тур\-но-по\-доб\-ны\-ми средствами в текстах на разных 
языках. Сравнительное исследование употребления различных частей речи в 
параллельных текстах на разных языках создает основу для выявления и 
описания языковых транс-\linebreak

\begin{center} %fig7
\vspace*{3pt}
\mbox{%
\epsfxsize=79.726mm
\epsfbox{koz-7.eps}
}
\end{center}
\vspace*{4pt}
%\begin{center}
{{\figurename~7}\ \ \small{Первый этап анализа параллельных текстов ($W_n$
обозначает словоформу с номером~$n$, $1\leq n\geq 5$)}}
%\end{center}
%\vspace*{9pt}

%\bigskip
\addtocounter{figure}{1}
      

\noindent 
формаций, при этом центральной трансформацией
является \textit{номинализация}. Явление номинализации
было исследовано в 
ряде работ отечественных и зарубежных лингвистов~[17--20]. 
Ближе всего к правильному, по мнению авторов данной статьи, 
пониманию этого явления следующие определения номинализации: 
<<конструкции\ldots называются номинализованными~--- в том смысле, что 
их естественно рассматривать как результат номинализации конструкций с 
предикативным употреблением глаголов и прилагательных>>; 
<<номинализация~--- это синтаксический процесс, который соотносит 
предложения с именными группами>>~\cite{9koz, 10koz}. Выявление 
номинализованных конструкций в параллельных научных и патентных 
текстах на русском, английском, французском и немецком языках в научных 
и патентных текстах и сопоставительное описание гла\-голь\-но-имен\-ных 
межъязыковых трансформаций~--- одна из центральных задач 
     ин\-же\-нер\-но-линг\-ви\-сти\-че\-ских исследований. 
     
     Следующей базовой трансформацией в исследуемых текстах на 
нескольких европейских языках является адъек\-тив\-но-ад\-вер\-би\-аль\-ное 
преобразование. Это означает, что при переводе с одного языка на другой 
происходит синтаксическое преобразование имен прилагательных в наречия 
и обратное преобразование~--- наречий в прилагательные. Установление 
семантических соответствий между этими языковыми объектами также 
возможно осуществить посредством аппарата~РСС. 
     
     При семантическом выравнивании непараллельных текстов, имеющих 
одну и ту же денотативную составляющую, аппарат РСС позволяет выявить в 
текстах когнитивные опоры (слова с сильной валентностью~--- 
     <<сло\-ва-дейст\-вия>> и <<сло\-ва-от\-но\-ше\-ния>>) и установить 
между ними семантические соответствия.

\section{Заключение}

     В данной работе представлен опыт создания и развития 
     когни\-тив\-но-линг\-ви\-сти\-че\-ских пред\-став\-ле\-ний в 
интеллектуальных информационных сис\-те\-мах, разработанных на основе 
аппарата РСС. Аппарат РСС 
обеспечивает мощные изобразительные возможности для описания всех 
уровней естественного языка, включая уровень 
     глу\-бин\-но-се\-ман\-ти\-че\-ских представлений и межъязыковых 
соответствий. Конкретные лингвистические процессоры, которые были 
созданы на основе этого подхода, прошли определенный путь развития и 
позволили выработать проектные решения для основных задач текущего 
этапа~--- извлечения и обработки содержательных знаний из текстов на 
естественных языках и сопоставления языковых структур в текстах на 
различных языках с учетом базовых трансформаций.
     
     Проблема извлечения и обработки знаний открывает перспективы 
развития интеллектуальных направлений компьютерной лингвистики, 
поскольку ее основной акцент смещен в сторону\linebreak глубинных представлений 
языка, в которых используются как грамматические (морфологические и 
синтаксические), так и семантические атрибуты для описания языковых 
объектов. Проводи-\linebreak мые авторами исследования параллельных текстов 
направлены также на рассмотрение этой проблемы~\cite{20koz}. 
Центральное место в проводящихся линг\-ви\-сти\-че\-ских исследованиях 
занимает изучение и формализация процессов трансформации языковых 
структур, особенно все варианты глагольно-но\-ми\-на\-тив\-ных трансформаций, 
создание развитых дис\-три\-бу\-тив\-но-транс\-фор\-ма\-ци\-он\-ных 
описаний предикатых структур для рассматриваемых языков. 
     
     Для задач извлечения знаний и создания \mbox{ИСПАР} 
     дис\-три\-бу\-тив\-но-транс\-фор\-ма\-ци\-он\-ные описания имеют 
особое значение, поскольку таким образом задаются все возможные способы 
перевода языковых структур в пре\-ди\-кат\-но-ар\-гу\-мент\-ные 
представления, которые затем используются в процедурах обработки знаний.

{\small\frenchspacing
{%\baselineskip=10.8pt
%\addcontentsline{toc}{section}{Литература}
\begin{thebibliography}{99}

     \bibitem{1koz}
     \Au{Кузнецов~И.\,П.}
     Семантические представления.~--- М.: Наука, 1986. 290~с.
     
     \bibitem{2koz}
     \Au{Козеренко~Е.\,Б.}
     Кон\-цеп\-ту\-аль\-но-линг\-вис\-ти\-че\-ское моделирование в среде 
интеллектуального редактора знаний ИКС~// Проблемы проектирования и 
использования баз знаний.~--- Киев: Ин-т кибернетики им.\ В.\,М.~Глушкова, 
1992. C.~73--79.
     
     \bibitem{3koz}
     \Au{Kozerenko~E.\,B.}
     Multilingual processors: A unified approach to semantic and syntactic 
knowledge presentation~// Conference (International ) on Artificial Intelligence 
IC-AI'2001 Proceedings. Las Vegas, Nevada, USA. June 25--28, 2001.~--- Las 
Vegas: CSREA Press, 2001. P.~1277--1282.

     \bibitem{18koz} %4
     \Au{Kuznetsov~I.\,P., Efimov~D.\,A., Kozerenko~E.\,B.}
     Tools for tuning the semantic processor to application areas~// ICAI'09 
Proceedings, WORLDCOMP'09. July 13--16, 2009. Las Vegas, Nevada, USA. 
Vol.~I.~--- Las Vegas: CRSEA Press, 2009. P.~467--472.
     
     \bibitem{19koz} %5
     \Au{Kuznetsov~I.\,P., Kozerenko~E.\,B., Kuznetsov~K.\,I., 
Timonina~N.\,O.}
     Intelligent system for entities extraction (ISEE) from natural language 
texts~// Workshop (International) on Conceptual Structures for Extracting Natural 
Language Semantics (Sense'09) at the 17th Conference 
(International ) on Conceptual Structures (ICCS'09) Proceedings. University Higher School of 
Economics. Moscow, Russia, 2009. P.~17--25.
     
     \bibitem{4koz} %6
     \Au{Апресян~Ю.\,Д.}
     Экспериментальное исследование семантики русского глагола.~--- М.: 
Наука, 1967.  252~с.
     
     \bibitem{5koz} %7
     \Au{Филлмор~Ч.}
     Дело о падеже~// Новое в зарубежной линг\-вистике, 1968. Вып.~X. С.~369--495.
     
     \bibitem{6koz} %8
     \Au{Хомский~Н.}
     Аспекты теории синтаксиса.~--- М.: МГУ, 1972.
     
     \bibitem{7koz} %9
     \Au{Хомский Н.}
     Язык и мышление.~--- М.: МГУ, 1972.
     
     
     \bibitem{8koz} %10
     \Au{Fillmore~C.}
     The case for case reopened~// Syntax and Semantics. Vol.~8.~--- N.Y.: 
Academic Press, 1977. 
     

          \bibitem{15koz} %11
     FASTUS: A cascaded finite-state trasducer for extracting information from 
natural-language text~// AIC, SRI International, Menlo Park, California, 1996. 
     
     \bibitem{16koz} %12
     \Au{Han~J., Pei~Y., Mao~R.}
     Mining frequent patterns without candidate generation: A frequent-pattern 
tree approach~// Data Mining and Knowledge Discovery, 2004. Vol.~8. No.\,1. 
P.~53--87.
     
     
     \bibitem{13koz} %13
     \Au{Cunningham~H.}
     Automatic information extraction~// Encyclopedia of Language and 
Linguistics. 2nd ed.~--- Elsevier, 2005.
     
     \bibitem{14koz} %14
     \Au{Han~J., Kamber~M.}
     Data mining: Concepts and techniques.~--- Morgan Kaufmann, 2006.
     
     
     \bibitem{17koz} %15
     \Au{Добров~Б.\,В., Лукашевич~Н.\,В.}
     Онтологии для автоматической обработки текстов: Описание понятий 
и лексических значений~// Компьютерная лингвистика и интеллектуальные 
технологии: Тр. межд. конф. <<Диалог'06>>. Бекасово, 31~мая\,--\,4~июня 
2006. С.~138--142.

     \bibitem{20koz} %16
     \Au{Kozerenko~E.\,B.}
     INTERTEXT: A multilingual knowledge base for machine translation~// 
Conference (International) on Machine Learning, Models, Technologies and 
Applications Proceedings. June 25--28, 2007. Las Vegas, USA.~--- Las Vegas: 
CSREA Press, 2007. P.~238--243.

     \bibitem{9koz} %17
     \Au{Жолковский~А.\,К., Мельчук~И.\,А.}
     О семантическом синтезе~// Проблемы кибернетики, 1967. Вып.~19.
     
         
     \bibitem{11koz} %18
     \Au{Jacobs~R.\,A., Rosenbaum P.\,S.}
     English transformational grammar.~--- Blaisdell, 1968.
     

\label{end\stat}
     
          \bibitem{12koz} %19
     \Au{Балли~Ш.}
     Общая лингвистика и вопросы французского языка. 2-е изд.~--- М.: 
УРСС, 2001.

\bibitem{10koz} %20
     \Au{Падучева~Е.\,В.}
     О~семантике синтаксиса: Мат-лы к трансформационной 
грамматике русского языка. 2-е изд.~--- М: КомКнига, 2007.  296~с. 
     
 \end{thebibliography}
}
}


\end{multicols} %10



%   { %\Large  
   { %\baselineskip=16.6pt
   
   \vspace*{-48pt}
   \begin{center}\LARGE
   \textit{Предисловие}
   \end{center}
   
   %\vspace*{2.5mm}
   
   \vspace*{25mm}
   
   \thispagestyle{empty}
   
   { %\small 

    
Вниманию читателей журнала <<Информатика и её применения>> предлагается 
очередной тематический выпуск <<Вероятностно-статистические методы и 
задачи информатики и информационных технологий>>. Предыдущие тематические 
выпуски журнала по данному направлению вышли в 2008~г.\ (т.~2, вып.~2), 
в 2009~г.\ (т.~3, вып.~3) и в 2010~г.\ (т.~4, вып.~2). 

Статьи, собранные в данном журнале, посвящены разработке новых вероятностно-статистических 
методов, ориентированных на применение к решению конкретных задач информатики и информационных 
технологий, а также~--- в ряде случаев~--- и других прикладных задач. Проблематика, охватываемая 
публикуемыми работами, развивается в рамках научного сотрудничества между Институтом проблем 
информатики Российской академии наук (ИПИ РАН) и Факультетом вычислительной математики и 
кибернетики Московского государственного университета им.\ М.\,В.~Ломоносова в ходе работ 
над совместными научными проектами (в том числе в рамках функционирования 
Научно-образовательного центра <<Вероятностно-статистические методы анализа рисков>>). 
Многие из авторов статей, включенных в данный номер журнала, являются активными участниками 
традиционного международного семинара по проблемам устойчивости стохастических моделей, 
руководимого В.\,М.~Золотаревым и В.\,Ю.~Королевым; регулярные сессии этого семинара 
проводятся под эгидой МГУ и ИПИ РАН (в 2011~г.\ указанный семинар проводится в октябре 
в Калининградской области РФ). 

Наряду с представителями ИПИ РАН и МГУ в число авторов данного выпуска журнала входят 
ученые из Научно-исследовательского института системных исследований РАН, Института 
проблем технологии микроэлектроники и особочистых материалов РАН, Института 
прикладных математических исследований Карельского НЦ РАН, Московского 
авиационного института, Вологодского государственного педагогического университета, 
НИИММ им.\ Н.\,Г.~Чеботарева, Казанского государственного университета, Дебреценского 
университета (Венгрия).

Несколько статей выпуска посвящено разработке и применению стохастических методов и 
информационных технологий для решения различных прикладных задач. В~работе В.\,Г.~Ушакова 
и О.\,В.~Шестакова рассмотрена задача определения вероятностных характеристик случайных 
функций по распределениям интегральных преобразований, возникающих в задачах эмиссионной 
томографии. В~статье Д.\,О.~Яковенко и М.\,А.~Целищева рассмотрены некоторые вопросы 
математической теории риска и предложен новый подход к диверсификации инвестиционных 
портфелей. Работа И.\,А.~Кудрявцевой и А.\,В.~Пантелеева посвящена построению и 
исследованию математической модели, описывающей динамику сильноионизованной плазмы. 
В~статье П.\,П.~Кольцова изучается качество работы ряда алгоритмов сегментации изображений. 
Статья А.\,Н.~Чупрунова и И.~Фазекаша посвящена вероятностному анализу числа без\-оши\-бочных 
блоков при помехоустойчивом кодировании; получены усиленные законы больших чисел для указанных 
величин.

В данном выпуске традиционно присутствует тематика, весьма активно разрабатываемая в течение 
многих лет специалистами ИПИ РАН и МГУ,~--- методы моделирования и управления для 
информационно-телекоммуникационных и вычислительных систем, в частности методы 
теории массового обслуживания. В~статье А.\,И.~Зейфмана с соавторами рассматриваются 
модели обслуживания, описываемые марковскими цепями с непрерывным временем в случае 
наличия катастроф. В~работе М.\,М.~Лери и И.\,А.~Чеплюковой рассматриваются случайные 
графы Интернет-типа, т.\,е.\ графы, степени вершин которых имеют степенные распределения; 
такие задачи находят применение при исследовании глобальных сетей передачи данных. 
Работа Р.\,В.~Разумчика посвящена исследованию систем массового обслуживания специального 
вида~--- с отрицательными заявками и хранением вытесненных заявок.

Ряд статей посвящен развитию перспективных теоретических 
вероятностно-статистических методов, которые находят широкое применение в различных 
задачах информатики и информационных технологий. В~работе В.\,Е.~Бенинга, А.\,К.~Горшенина 
и В.\,Ю.~Королева рассмотрена задача статистической проверки гипотез о числе компонент 
смеси вероятностных распределений, приводится конструкция асимптотически наиболее мощного 
критерия. Результаты этой работы найдут применение в ряде прикладных задач, использующих 
математическую модель смеси вероятностных распределений (в информатике, моделировании 
финансовых рынков, физике турбулентной плазмы и~т.\,д.). В~статье В.\,Ю.~Королева, 
И.\,Г.~Шевцовой и С.\,Я.~Шоргина строится новая, улучшенная оценка точности нормальной 
аппроксимации для пуассоновских случайных сумм; как известно, указанные случайные суммы 
широко используются в качестве моделей многих реальных объектов, в том числе в информатике, 
физике и других прикладных областях. Работа В.\,Г.~Ушакова и Н.\,Г.~Ушакова посвящена 
исследованию ядерной оценки плотности распределения; эти результаты могут применяться, 
в част\-ности, при анализе трафика в телекоммуникационных системах. Серьезные приложения 
в статистике могут получить результаты работы О.\,В.~Шестакова, в которой доказаны оценки 
скорости сходимости распределения выборочного абсолютного медианного отклонения к нормальному 
закону. 

\smallskip

Редакционная коллегия журнала выражает надежду, что данный тематический  выпуск 
будет интересен специалистам в области теории вероятностей и математической статистики 
и их применения к решению задач информатики и информационных технологий.
     
     %\vfill 
     \vspace*{20mm}
     \noindent
     Заместитель главного редактора журнала <<Информатика и её 
применения>>,\\
     директор ИПИ РАН, академик  \hfill
     \textit{И.\,А.~Соколов}\\
     
     \noindent
     Редактор-составитель тематического выпуска,\\
     профессор кафедры математической статистики факультета\\
      вычислительной математики и кибернетики МГУ им.\ М.\,В.~Ломоносова,\\
     ведущий научный сотрудник ИПИ РАН,\\ 
доктор физико-математических наук \hfill
      \textit{В.\,Ю.~Королев}
     
     } }
     }

%%%%%%%%%%%%%%%%%%%%%%%%%%%%%%%%%%%%%%%%%%%%%%%


                       
%\end{document}

%\def\stat{rez}
{%\hrule\par
%\vskip 7pt % 7pt
\raggedleft\Large \bf%\baselineskip=3.2ex
Р\,Е\,Ц\,Е\,Н\,З\,И\,И \vskip 17pt
    \hrule
    \par
\vskip 6pt plus 6pt minus 3pt }

%\thispagestyle{headings} %с верхним колонтитулом
%\thispagestyle{myheadings} %с нижним колонтитулом, но в верхнем РЕЦЕНЗИИ

\def\tit{НОВАЯ КНИГА И.\,Н.~СИНИЦЫНА, А.\,С.~ШАЛАМОВА <<ЛЕКЦИИ ПО ТЕОРИИ 
ИНТЕГРИРОВАННОЙ ЛОГИСТИЧЕСКОЙ ПОДДЕРЖКИ>> (М.: ТОРУС ПРЕСС, 2012. 624~с.)}

%1
\def\aut{Д.ф.-м.н., профессор С.\,Я.~Шоргин}

\def\auf{\ }

\def\leftkol{\ % РЕЦЕНЗИИ
}

\def\rightkol{ \ } 

%\def\leftkol{\ } % ENGLISH ABSTRACTS}

%\def\rightkol{\ } %ENGLISH ABSTRACTS}

%\def\leftkol{РЕЦЕНЗИИ}

%\def\rightkol{РЕЦЕНЗИИ}

\titele{\tit}{\aut}{\auf}{\leftkol}{\rightkol}
\vspace*{-18pt}


     \label{st\stat}

     \begin{multicols}{2}
     {\small
     {\baselineskip=10.1pt
     

      В книге представлено системное изложение теоретических основ одного из новейших 
направлений в \mbox{об\-ласти} экономики послепродажного обслуживания изделий наукоемкой 
продукции (ИНП) длительного пользования~--- интегрированной логистической поддержки
(ИЛП). 
{\looseness=1

}

Приведены также результаты новых работ, выполненных в Институте проблем информатики 
Российской академии наук в рамках научного направления <<Информационные технологии и 
анализ сложных сис\-тем>>.
 {%\looseness=1

}
     
      Излагаемые в книге научные подходы позво\-ляют карди\-наль\-но реформировать 
существующие системы производства и эксплуатации ИНП путем создания и внед\-ре\-ния 
методов рационального и оптимального управ\-ле\-ния процессами расходования 
вре\-мен\-н$\acute{\mbox{ы}}$х, 
мате\-ри\-аль\-ных, трудовых и других ресурсов на всех стадиях жизненного цикла изделий (ЖЦИ) по 
критериям экономической целесообразности и эф\-фек\-тив\-ности.
  {\looseness=1

}
    
      В книге приведен краткий обзор причин возник\-новения и
      развития CALS-методологии как основы 
современных международных стандартов по созданию и функционированию глобальных 
ин\-фор\-ма\-ци\-он\-но-ком\-му\-ни\-ка\-ци\-он\-ных систем, ее ключевых возможностей и эффективности 
результатов ее использования. 
Авторы %\linebreak 
предлагают ряд научных обоснований для разработки 
единой теории проектирования и управления систем ИЛП для полноценного использования 
преимуществ %\linebreak
 суще\-ст\-ву\-ющей методологии, определяют \mbox{общую} структурную схему 
комплексной системы <<ИНП-СППО>> и необходимость разработки для ее описания 
гибридных стохастических моделей.
{%\looseness=1

}

%\columnbreak
      
      Книга состоит из пяти частей, где последовательно излагается материал по каждой из 
следующих тем: <<Интегрированная логистическая поддержка>>, <<Теория гибридных 
стохастических систем и компьютерная поддержка исследований и разработок>>, <<Основы 
математического моделирования, анализа и синтеза систем послепродажного обслуживания>>, 
<<Определение и анализ показателей экспортного потенциала ИНП при проектировании>>, 
<<Задачи управления поддержкой послепродажного обслуживания>>, а также 
<<Моделирование инвестиционных процессов ИЛП в условиях неравновесных финансовых 
рынков>>. 
   
      В конце каждой главы приведены выводы и даны вопросы и задания для 
самоконтроля. В~приложениях содержатся основные определения по программам работ по 
анализу ИЛП, логистическим базам данных и компьютерным решениям, эквивалентной статистической 
линеаризации нелинейных преобразований ИЛП, справочный материал, а также развернутые 
уравнения для вероятностных характеристик.


      \def\leftkol{РЕЦЕНЗИИ}

\def\rightkol{РЕЦЕНЗИИ} 

      
      Книга заинтересует широкий круг специалистов и может быть использована научными 
проектными организациями в сфере промышленного производства ИНП. Большое количество 
иллюстраций, примеров и вопросов, обращенных к читателю, позволяет использовать книгу 
также в качестве учебного пособия для студентов и аспирантов машиностроительных, 
транспортных и~других специальностей, а также для самостоятельного изучения. 
{%\looseness=-1

}

Книга 
представляет несомненный интерес для специалистов и студентов в области прикладной 
математики и информатики.
    

}

}
\end{multicols}

%\newpage

\include{obchak}


\def\stat{authorsrus}
{%\hrule\par
%\vskip 7pt % 7pt
\raggedleft\Large \bf%\baselineskip=3.2ex
О\,Б\ \ А\,В\,Т\,О\,Р\,А\,Х \vskip 17pt
    \hrule
    \par
\vskip 21pt plus 8pt minus 4pt }


\def\tit{\ }

\def\aut{\ }

\def\auf{\ }

\def\leftkol{\ } % ENGLISH ABSTRACTS}

\def\rightkol{ОБ АВТОРАХ} %ENGLISH ABSTRACTS}

\titele{\tit}{\aut}{\auf}{\leftkol}{\rightkol}
      
            \label{st\stat}



\vspace*{24pt}

\begin{multicols}{2}




\noindent
\textbf{Архипов Олег Петрович} (р.\ 1948)~---
кандидат технических наук, директор Орловского филиала Института проб\-лем информатики
Российской академии наук
%302025, г.Орел, Московское шоссе, д.137

\vspace*{3pt}

\noindent
\textbf{Бирюкова Татьяна Константиновна} (р.\ 1968)~---
кандидат фи\-зи\-ко-ма\-те\-ма\-ти\-че\-ских наук, старший научный сотрудник Института проб\-лем информатики
Российской академии наук

\vspace*{3pt}

\noindent 
\textbf{Бобков  Сергей Геннадьевич} (р.\ 1955)~---
доктор технических наук,  заведующий отделением На\-уч\-но-ис\-сле\-до\-ва\-тель\-ско\-го 
института системных исследований Российской академии наук
%117218, Москва, Нахимовский просп., 36, к.1 

\vspace*{3pt}

\noindent \textbf{Васильев Николай Семенович} (р.\ 1952)~--- доктор 
фи\-зи\-ко-ма\-те\-ма\-ти\-че\-ских наук, профессор, 
МГТУ им.\ Н.\,Э.~Баумана 
%, Москва 105005, 2-я Бауманская ул., д.~5,

\vspace*{3pt}

\noindent
\textbf{Гершкович Максим Михайлович} (р.\ 1968)~---
старший научный сотрудник Института проб\-лем информатики
Российской академии наук

\vspace*{3pt}

\noindent 
\textbf{Дьяченко Юрий Георгиевич} (р.\ 1958)~--- кандидат технических наук, 
старший научный сотрудник Института проб\-лем информатики
Российской академии наук

\vspace*{3pt}

\noindent 
\textbf{Ерошенко Александр Андреевич} (р.\ 1989)~--- аспирант кафедры 
математической статистики факультета вычисли\-тельной математики и кибернетики 
Московского государственного университета им.\ М.\,В.~Ломоносова
%119991, Москва ГСП-1, Ленинские горы, д.\ 1, стр. 52

\vspace*{3pt}
 
\noindent 
\textbf{Захаров Виктор Николаевич} (р.\ 1948)~--- 
доктор технических наук, доцент, ученый секретарь Института проб\-лем информатики
Российской академии наук

\vspace*{3pt}

\noindent
\textbf{Зейфман Александр Израилевич} (р.\ 1954)~---
доктор фи\-зи\-ко-ма\-те\-ма\-ти\-че\-ских наук, профессор, 
заведующий кафедрой Вологодского государственного университета; 
старший научный сотрудник Института проб\-лем информатики
Российской академии наук; главный научный сотрудник ИСЭРТ Российской академии наук

\vspace*{3pt}

\noindent
\textbf{Зыкин Сергей Владимирович} (р.\ 1959)~--- 
доктор технических наук, профессор, заведующий лабораторией Института математики 
им.\ С.\,Л.~Соболева Сибирского отделения Российской академии наук, Новосибирск 
%630090, пр.\ ак.\ Коптюга, 4 

\vspace*{4pt}

\noindent
\textbf{Киреев Владимир Иванович} (р.\ 1938)~---
доктор фи\-зи\-ко-ма\-те\-ма\-ти\-че\-ских наук, профессор Московского 
государственного горного университета
%Адрес: Россия, 119991, г. Москва, Ленинский проспект, д. 6

%\columnbreak

\vspace*{4pt}

\noindent
\textbf{Козеренко Елена Борисовна} (р.\ 1959)~---
кандидат филологических наук, заведующая лабораторией Института проб\-лем информатики
Российской академии наук

\vspace*{4pt}

\noindent
\textbf{Королев Виктор Юрьевич} (р.\ 1954)~--- доктор
фи\-зи\-ко-ма\-те\-ма\-ти\-че\-ских наук, профессор кафедры математической 
статистики факультета вычисли\-тельной математики и кибернетики 
Московского государственного университета; 
ведущий научный сотрудник Института проб\-лем информатики
Российской академии наук

\vspace*{4pt}

\noindent
\textbf{Коротышева Анна Владимировна} (р.\ 1988)~---
старший преподаватель Вологодского государственного университета

\vspace*{4pt}

\noindent 
\textbf{Кун Де Турк} (р.\ 1981)~--- научный сотрудник 
исследовательской группы SMACS факультета телекоммуникаций и обработки информации
Университета Гента, Бельгия
%В-9000 Гент, Бельгия

\vspace*{4pt}

\noindent
\textbf{Лупенцов Олег Сергеевич} (р.\ 1986)~---
аспирант Омского государственного института сервиса
%Омск 644043, ул.\ Певцова 13

\vspace*{4pt}

\noindent
\textbf{Лучко Олег Николаевич} (р.\ 1961)~---
кандидат педагогических наук, профессор, заведующий кафедрой 
Омского государственного института сервиса
%Омск 644043, ул.\ Певцова 13

\vspace*{4pt}

\noindent
\textbf{Малашенко Юрий Евгеньевич} (р.\ 1946)~---
доктор фи\-зи\-ко-ма\-те\-ма\-ти\-че\-ских наук, заведующий сектором 
Вычислительного центра им.\ А.\,А.~Дородницына Российской академии наук
%Адрес: 119333, Москва, ул. Вавилова, 40,

\vspace*{4pt}

\noindent
\textbf{Маньяков Юрий Анатольевич} (р.\ 1984)~---
кандидат технических наук, научный сотрудник Орловского филиала Института проб\-лем информатики
Российской академии наук
%302025, г.Орел, Московское шоссе, д.137

\vspace*{4pt}

\noindent
\textbf{Маренко Валентина Афанасьевна} (р.\ 1951)~---
кандидат технических наук, доцент, старший научный сотрудник 
Института математики им.\ С.\,Л.~Соболева Сибирского отделения Российской академии наук
%Новосибирск 630090, пр. ак. Коптюга, 4 

\vspace*{3pt}

\noindent 
\textbf{Морозов Евсей Викторович} (р.\ 1947)~--- доктор 
фи\-зи\-ко-ма\-те\-ма\-ти\-че\-ских, профессор, ведущий научный сотрудник 
Института прикладных математических исследований Карельского научного центра Российской
академии наук; 
%%185910 Россия, Республика Карелия, г.\ Петрозаводск, ул.\ Пушкинская, 11
профессор Петрозаводского государственного университета, Петрозаводск
%185910 Россия, Республика Карелия, г.\ Петрозаводск, пр.\ Ленина, 33

%\pagebreak

\vspace*{3pt}

\noindent
\textbf{Назарова Ирина Александровна} (р.\ 1966)~---
кандидат фи\-зи\-ко-ма\-те\-ма\-ти\-че\-ских наук, 
научный сотрудник Вычислительного центра им.\ А.\,А.~Дородницына Российской академии наук 
%Адрес: 119333, Москва, ул. Вавилова, 40

\vspace*{3pt}

\noindent
\textbf{Павлов Игорь Валерианович} (р.\ 1945)~--- 
доктор фи\-зи\-ко-ма\-те\-ма\-ти\-че\-ских наук, профессор МГТУ им.\ Н.\,Э.~Баумана 
%Москва 105005, 2-я Бауманская ул., д.~5 

%\pagebreak

\vspace*{3pt}

\noindent 
\textbf{Потахина Любовь Викторовна} (р.\ 1989)~--- аспирантка
Института прикладных математических исследований Карельского научного центра
Российской академии наук; 
%%185910 Россия, Республика Карелия, г.\ Петрозаводск, ул.\ Пушкинская, 11
инженер Петрозаводского государственного университета, Петрозаводск
%185910 Россия, Республика Карелия, г.\ Петрозаводск, пр.\ Ленина, 33

\vspace*{3pt}

\noindent 
\textbf{Рождественский Юрий Владимирович} (р.\ 1952)~--- 
кандидат технических наук, заведующий сектором Института проб\-лем информатики
Российской академии наук

\vspace*{3pt}

\noindent 
\textbf{Синицын Игорь Николаевич} (р.\ 1940)~--- доктор технических наук,
профессор, заслуженный деятель\linebreak\vspace*{-12pt}

\columnbreak

\noindent
 науки РФ, заведующий отделом Института проб\-лем информатики
Российской академии наук

\vspace*{7pt}


\noindent
\textbf{Сиротинин Денис Олегович} (р.\ 1984)~---
кандидат технических наук, научный сотрудник Орловского филиала Института проб\-лем информатики
Российской академии наук
%302025, г.Орел, Московское шоссе, д.137

\vspace*{7pt}

%\columnbreak

\noindent 
\textbf{Соколов  Игорь Анатольевич} (р.\ 1954)~--- академик (действительный член) Российской 
академии наук, доктор технических наук, директор Института проб\-лем информатики
Российской академии наук

\vspace*{7pt}

\noindent
\textbf{Степченков Юрий Афанасьевич} (р.\ 1951)~---
кандидат технических наук, заведующий отделом Института проб\-лем информатики
Российской академии наук

\vspace*{7pt}

\noindent
\textbf{Сурков Алексей Викторович} (р.\ 1978)~--- 
старший научный сотрудник На\-уч\-но-ис\-сле\-до\-ва\-тель\-ско\-го 
института системных исследований Российской академии наук
%117218, Москва, Нахимовский просп., 36, к.1 

\vspace*{7pt}

\noindent 
\textbf{Шестаков Олег Владимирович} (р.\ 1976)~--- доктор 
фи\-зи\-ко-ма\-те\-ма\-ти\-че\-ских, доцент кафедры математической статистики 
факультета вычисли\-тельной математики и кибернетики Московского 
государственного университета им.\ М.\,В.~Ломоносова; 
%119991, Москва ГСП-1, Ленинские горы, д.\ 1, стр. 52
старший научный сотрудник Института проб\-лем информатики
Российской академии наук
%, Москва 119333, ул. Вавилова, д.~44, корп.~2

\vspace*{7pt}

\noindent 
\textbf{Шоргин Сергей Яковлевич} (р.\ 1952.)~--- доктор
фи\-зи\-ко-ма\-те\-ма\-ти\-че\-ских наук, профессор, заместитель директора Института 
проб\-лем информатики Российской академии наук





%%%%%%%%%%%%%%%%%%%%%%%%%%%%%%%%%%%%%%%%%%%%%%%%%%%%%%%%%%%%%%%%%%%%%%%%%%%%%%%




%\def\rightkol{ОБ АВТОРАХ}
%\def\leftkol{ОБ АВТОРАХ}

 \label{end\stat}





%\def\leftfootline{\small{\textbf{\thepage}
%\hfill ИНФОРМАТИКА И ЕЁ ПРИМЕНЕНИЯ\ \ \ том~7\ \ \ выпуск~1\ \ \ 2013}
%}%
% \def\rightfootline{\small{ИНФОРМАТИКА И ЕЁ ПРИМЕНЕНИЯ\ \ \ том~7\ \ \ выпуск~1\ \ \ 2013
%\hfill \textbf{\thepage}}}


%\thispagestyle{myheadings}



\end{multicols}

\newpage


%\vspace*{-48pt}
\begin{center}\LARGE
\textit{About Authors}
\end{center}

\thispagestyle{empty}
\def\tit{\ }

\def\aut{\ }

\def\auf{\ }


\def\leftkol{ABOUT AUTHORS}

\def\rightkol{ABOUT AUTHORS}

\vspace*{-18pt}

\titele{\tit}{\aut}{\auf}{\leftkol}{\rightkol}

%\vspace*{36pt}

\def\rightmark{{\noindent\hbox to \textwidth{\hfill\small ABOUT AUTHORS
%\hfill \large\bf\thepage
}}}
\def\leftmark{{\noindent\parbox{\textwidth}{
%\raggedleft\large\bf\thepage \hfill
\small\textrm{ABOUT AUTHORS}\hfill}}}


\def\leftfootline{\small{\textbf{\thepage}
\hfill ИНФОРМАТИКА И ЕЁ ПРИМЕНЕНИЯ\ \ \ том~6\ \ \ выпуск~2\ \ \ 2012}
}%
 \def\rightfootline{\small{ИНФОРМАТИКА И ЕЁ ПРИМЕНЕНИЯ\ \ \ том~6\ \ \ выпуск~2\ \ \ 2012
\hfill \textbf{\thepage}}}


\begin{multicols}{2}

\noindent
\textbf{Agalarov Yaver M.} (b.\ 1952)~--- Candidate of Science (PhD)
in technology, 
leading scientist, Institute of Informatics Problems, Russian Academy of Sciences

\vspace*{5pt}


  \noindent
\textbf{Bosov Alexey V.} (b.\ 1969)~--- Doctor of Science in technology, Head of
Laboratory, Institute of Informatics Problems, Russian Academy of Sciences

\vspace*{5pt}


\noindent
\textbf{Dulin Sergey K.} (b.\ 1950)~--- Doctor of Science in technology, 
professor, senior scientist, Institute of Informatics Problems, Russian Academy of Sciences

\vspace*{5pt}

\noindent
\textbf{Gorshenin Andrey K.}~--- (b.\ 1986)~--- Candidate of Science (PhD)
in physics and mathematics,
senior scientist, Institute of Informatics Problems, Russian Academy of Sciences

\vspace*{5pt}

\noindent
\textbf{Kalenov Nikolay E.}  (b.\ 1945)~--- Doctor of Science in technology,
professor, Director, Library for Natural Sciences,  Russian Academy of Sciences 

\vspace*{5pt}

\noindent
\textbf{Kalinichenko Leonid A.} (b.\ 1937)~--- Doctor of Science in physics and mathematics, 
professor, Honored scientist of RF, 
Head of Laboratory, Institute of Informatics Problems, Russian Academy of Sciences 

\vspace*{5pt}

\noindent
\textbf{Karpov Alexey A.} (b.\ 1978)~--- Candidate of Science (PhD) in technology, 
senior scientist, St.\ Petersburg Institute for
Informatics and Automation,  Russian Academy of Sciences

\vspace*{5pt}

\noindent
\textbf{Kuznetsov Igor P.} (b.\ 1938)~--- Doctor of Science in technology, 
professor, principal scientist, Institute of Informatics Problems, Russian Academy of Sciences

\vspace*{5pt}


\noindent
\textbf{Markova Natalia A.} (b.\ 1950)~--- Candidate of Science (PhD) in
physics and mathematics, leading scientist,  
Institute of Informatics Problems, Russian Academy of Sciences

\vspace*{5pt}

\noindent
\textbf{Nikolaev Andrey V.} (b.\ 1985)~--- Candidate of Science (PhD) in technology, 
senior lecturer, Tchaikovsky Technological Institute, Branch of the Izhevsk State Technical 
University

\vspace*{6pt}

\noindent
\textbf{Pavlov Igor V.} (b.\ 1945)~---  Doctor of Science in physics and mathematics,
professor, Bauman Moscow State Technical University

\vspace*{6pt}

%\columnbreak

\noindent
\textbf{Rozenberg Igor N.} (b.\ 1965)~--- Doctor of Science in technology, 
First Deputy Director General, Research \& Design Institute for Information 
Technology, Signalling and Telecommunications on Railway Transport (JSC NIIAS)

\vspace*{6pt}


\noindent
\textbf{Semenov Konstantin K.} (b.\ 1986)~--- MPhil, 
associate professor, St.\ Petersburg State Polytechnical University

\vspace*{6pt}

\noindent
\textbf{Sharnin Mikhail M.} (b.\ 1959)~--- Candidate of Science (PhD) 
in technology, senior scientist, Institute of Informatics Problems, Russian Academy of Sciences

\vspace*{6pt}

\noindent 
\textbf{Shestakov Oleg V.} (b.\ 1976)~--- Candidate of Science (PhD) in physics and mathematics,
associate professor, Department of Mathematical Statistics, Faculty of Computational Mathematics and Cybernetics,
M.\,V.~Lomonosov Moscow State University; senior scientist, Institute of Informatics Problems, 
Russian Academy of Sciences

\vspace*{6pt}

\noindent
\textbf{Stupnikov Sergey A.} (b.\ 1978)~--- Candidate of Science (PhD) in technology, 
senior scientist, Institute of Informatics Problems, Russian Academy of Sciences 

\vspace*{6pt}

\noindent
\textbf{Umansky Vladimir I.} (b.\ 1954)~--- Candidate of Science (PhD) in technology, 
Director General, ``IntechGeoTrans'' Closed Joint Stock Company

\vspace*{6pt}

\noindent
\textbf{Zhevnerchuk Dmitry V.} (b.\ 1978)~--- Candidate of Science (PhD) in technology, 
associate professor, Tchaikovsky Technological Institute, Branch of the Izhevsk State 
Technical University

%\vspace*{6pt}

\def\leftfootline{\small{\textbf{\thepage}
\hfill ИНФОРМАТИКА И ЕЁ ПРИМЕНЕНИЯ\ \ \ том~6\ \ \ выпуск~2\ \ \ 2012}
}%
 \def\rightfootline{\small{ИНФОРМАТИКА И ЕЁ ПРИМЕНЕНИЯ\ \ \ том~6\ \ \ выпуск~2\ \ \ 2012
\hfill \textbf{\thepage}}}



%\thispagestyle{myheadings}

\end{multicols}
\newpage

\vspace*{-60pt} {\small
{\baselineskip=9.1pt
\section*{Правила подготовки рукописей статей для публикации в журнале
<<Информатика и её применения>>}

\thispagestyle{empty}

 Журнал <<Информатика и её применения>> публикует
теоретические, обзорные и дискуссионные статьи, посвященные научным
исследованиям и разработкам в области информатики и ее приложений. Журнал
издается на русском языке. По специальному решению редколлегии отдельные статьи,
в виде исключения, могут печататься на английском языке.
Тематика журнала охватывает следующие направления:
\begin{itemize}
\item теоретические основы информатики; %\\[-13.5pt]
\item математические методы исследования сложных систем и процессов; %\\[-13.5pt]
\item информационные системы и сети; %\\[-13.5pt]
\item информационные технологии; %\\[-13.5pt]
\item архитектура и программное
обеспечение вычислительных комплексов и сетей.
\end{itemize}
\begin{enumerate}
\item В журнале печатаются результаты, ранее не
опубликованные и не предназначенные к одновременной публикации в других
изданиях. Публикация не должна нарушать закон об авторских правах. Направляя
свою рукопись в редакцию, авторы автоматически передают учредителям и
редколлегии неисключительные права на издание данной статьи на русском языке и
на ее распространение в России и за рубежом. При этом за авторами сохраняются
все права как собственников данной рукописи. В связи с этим авторами должно
быть представлено в редакцию письмо в следующей форме:
Соглашение о передаче права на публикацию:

\textit{<<Мы, нижеподписавшиеся, авторы рукописи <<$\qquad\qquad$>>, передаем
учредителям и редколлегии журнала <<Информатика и её применения>>
неисключительное право опубликовать данную рукопись статьи на русском языке как
в печатной, так и в электронной версиях журнала. Мы подтверждаем, что данная
публикация не нарушает авторского права других лиц или организаций. Подписи
авторов: (ф.\,и.\,о., дата, адрес)>>.}

Указанное соглашение может быть представлено 
как в бумажном виде, так и в виде отсканированной копии (с подписями авторов).


Редколлегия вправе запросить у авторов экспертное заключение о возможности
опубликования представленной статьи в открытой печати. %\\[-13.5pt]
\item Статья
подписывается всеми авторами. На отдельном листе представляются данные автора
(или всех авторов): фамилия, полные имя и отчество, телефон, факс, e-mail,
почтовый адрес. Если работа выполнена несколькими авторами, указывается фамилия
одного из них, ответственного за переписку с редакцией. %\\[-13.5pt]
\item Редакция журнала
осуществляет самостоятельную экспертизу присланных статей. Возвращение рукописи
на доработку не означает, что статья уже принята к печати. Доработанный вариант
с ответом на замечания рецензента необходимо прислать в редакцию. %\\[-13.5pt]
\item Решение
редакционной коллегии о принятии статьи к печати или ее отклонении сообщается
авторам. Редколлегия не обязуется направлять рецензию авторам отклоненной
статьи. %\\[-13.5pt]
\item Корректура статей высылается авторам для просмотра. Редакция
просит авторов присылать свои замечания в кратчайшие сроки. %\\[-13.5pt]
\item При
подготовке рукописи в MS Word рекомендуется использовать следующие настройки.
Параметры страницы: формат~--- А4; ориентация~--- книжная; поля (см): внутри~---
2,5, снаружи~--- 1,5, сверху~--- 2, снизу~--- 2, от края до нижнего
колонтитула~--- 1,3. Основной текст: стиль~--- <<Обычный>>: шрифт Times New
Roman, размер 14~пунктов, абзацный отступ~--- 0,5~см, 1,5 интервала,
выравнивание~--- по ширине. Рекомендуемый объем рукописи~--- не свыше
25~страниц указанного формата. Ознакомиться с шаблонами, содержащими примеры
оформления, можно по адресу в Интернете:
\textsf{http://www.ipiran.ru/journal/template.doc}.
\item К рукописи, предоставляемой в 2-х
экземплярах, обязательно прилагается электронная версия статьи (как правило, в
форматах MS WORD (.doc) или \LaTeX\ (.tex), а также~--- дополнительно~--- в
формате .pdf) на дискете, лазерном диске или по электронной почте. Сокращения
слов, кроме стандартных, не применяются. Все страницы рукописи должны быть
пронумерованы. %\\[-13.5pt]
\item Статья должна содержать следующую информацию на русском и
английском языках: название, Ф.И.О. авторов, места работы авторов и их
электронные адреса, подробные сведения об авторах, оформленные в соответствии с форматом, 
определяемым файлами {\sf http://www.ipiran.ru/journal/issues/2011\_05\_01/authors.asp} и 
{\sf http://www.ipiran.ru/journal/issues/2011\_01\_eng/authors.asp},
аннотация (не более 100~слов), ключевые слова. Ссылки на
литературу в тексте статьи нумеруются (в квадратных скобках) и располагаются в
порядке их первого упоминания. В~списке литературы не должно быть позиций, на которые нет ссылки в тексте статьи.
Все фамилии авторов, заглавия статей, названия
книг, конференций и~т.\,п.\ даются на языке оригинала, если этот язык
использует кириллический или латинский алфавит. %\\[-13.5pt]
\item Присланные в редакцию материалы авторам не возвращаются.
\item При отправке файлов по электронной
почте просим придерживаться следующих правил:
\begin{itemize}
\item указывать в поле subject (тема) название журнала и фамилию автора; %\\[-13.5pt]
\item использовать attach (присоединение); %\\[-13.5pt]
\item в случае больших объемов информации возможно
использование общеизвестных архиваторов (ZIP, RAR); %\\[-13.5pt]
\item в состав электронной версии статьи должны входить: файл, содержащий текст статьи, и файл(ы),
содержащий(е) иллюстрации. %\\[-13.5pt]
\end{itemize}
\item Журнал <<Информатика и её применения>> является некоммерческим изданием. 
Плата за публикацию с авторов не взимается, гонорар авторам не выплачивается.
\end{enumerate}
\thispagestyle{empty}
\textbf{Адрес редакции:} Москва 119333,
ул.~Вавилова, д.~44, корп.~2, ИПИ РАН\\
\hphantom{\textbf{Адрес редакции:} }Тел.: +7 (499) 135-86-92\ \
Факс:  +7 (495) 930-45-05\ \  E-mail:   rust@ipiran.ru }
}

\end{document}


%\tableofcontents

%\end{document}

\def\stat{cont}
{%\hrule\par
%\vskip 7pt % 7pt
\raggedleft\Large \bf%\baselineskip=3.2ex
А\,В\,Т\,О\,Р\,С\,К\,И\,Й\ \ У\,К\,А\,З\,А\,Т\,Е\,Л\,Ь\ \ З\,А\ \ 2\,0\,1\,0 г. \vskip 17pt
    \hrule
    \par
\vskip 21pt plus 6pt minus 3pt }

\label{st\stat}

\def\tit{\ }

\def\aut{\ }
\def\auf{\ }

\def\leftkol{\ } % ENGLISH ABSTRACTS}

\def\rightkol{\ } %АВТОРСКИЙ УКАЗАТЕЛЬ ЗА 2010 г.} %ENGLISH ABSTRACTS}

\titele{\tit}{\aut}{\auf}{\leftkol}{\rightkol}

\vspace*{-12pt}

{\tabcolsep=3pt
\begin{tabular}{p{388pt}rr}
&\textbf{Выпуск} & \textbf{Стр.}\\[6pt]
\hangindent=23pt\noindent\textbf{Арутюнян~А.\,Р.} Моделирование влияния деформаций отпечатков пальцев на 
точность\linebreak
\vspace*{-12pt}\\
\hspace*{23pt}дактилоскопической идентификации$\dotfill$&1&51\\
\hangindent=23pt\noindent\textbf{Архипов~О.\,П., Зыкова~З.\,П.} Интеграция гетерогенной информации о цветных 
пикселях\linebreak
\vspace*{-12pt}\\
\hspace*{23pt}и их цветовосприятии$\dotfill$&4&15\\
\hangindent=23pt\noindent\textbf{Баранов~С.\,И., Френкель~С.\,Л., Захаров~В.\,Н.} Полуформальная верификация 
цифрового устройства с конвейером, основанная на использовании алгоритмических машин\linebreak
\vspace*{-12pt}\\
\hspace*{23pt}состояния$\dotfill$&4&49\\
\textbf{Бекетова~И.\,В.} см.~Каратеев~С.\,Л.&&\\
\textbf{Белоусов~В.\,В.} см.~Синицын~И.\,Н.&&\\
\hangindent=23pt\noindent\textbf{Бенинг~В.\,Е., Королев~Р.\,А.} О предельном поведении мощностей критериев в 
случае\linebreak
\vspace*{-12pt}\\
\hspace*{23pt}распределения Лапласа$\dotfill$&2&63\\
\hangindent=23pt\noindent\textbf{Бенинг~В.\,Е., Сипина~А.\,В.} Асимптотическое разложение для мощности 
критерия,\linebreak
\vspace*{-12pt}\\
\hspace*{23pt}основанного на выборочной медиане, в случае распределения Лапласа$\dotfill$&1&18\\
\textbf{Бондаренко~А.\,В.} см.~Каратеев~С.\,Л.&&\\
\hangindent=23pt\noindent\textbf{Бородина~А.\,В., Морозов~Е.\,В.} Об оценивании асимптотики вероятности 
большого\linebreak
\vspace*{-12pt}\\
\hspace*{23pt}уклонения стационарной регенеративной очереди с одним прибором$\dotfill$&3&29\\
\hangindent=23pt\noindent\textbf{Бунтман~Н.\,В., Минель~Ж.-Л., Ле~Пезан~Д., Зацман~И.\,М.} Типология и 
компьютерное\linebreak
\vspace*{-12pt}\\
\hspace*{23pt}моделирование трудностей перевода$\dotfill$&3&77\\
\textbf{Визильтер~Ю.\,В.} см.~Каратеев~С.\,Л.&&\\
\hangindent=23pt\noindent\textbf{Гавриленко~С.\,В.} Оценки скорости сходимости распределений случайных сумм с 
безгранично делимыми индексами к нормальному закону$\dotfill$&4&81\\
\hangindent=23pt\noindent\textbf{Григорьева~М.\,Е., Шевцова~И.\,Г.} Уточнение неравенства 
Каца--Берри--Эссеена$\dotfill$&2&75\\
\hangindent=23pt\noindent\textbf{Грушо~А.\,А., Грушо~Н.\,А., Тимонина~Е.\,Е.} Поиск конфликтов в политиках 
безопасности: модель случайных графов$\dotfill$&3&38\\
\textbf{Грушо~Н.\,А.} см.~Грушо~А.\,А.&&\\
\hangindent=23pt\noindent\textbf{Гудков~В.\,Ю.} Математические модели изображения отпечатка пальца на основе 
описания линий$\dotfill$&1&58\\
\textbf{Гуртов~А.\,В.} см.~Лукьяненко~А.\,С.&&\\
\textbf{Желтов~С.\,Ю.} см.~Каратеев~С.\,Л.&&\\
\hangindent=23pt\noindent\textbf{Захаров~А.\,А., Серебряков~В.\,А.} Система управления электронной библиотекой 
LibMeta$\dotfill$&4&2\\
\textbf{Захаров~В.\,Н.} см.~Баранов~С.\,И.&&\\
\textbf{Захарова~Т.\,В.} см.~Матвеева~С.\,С.&&\\
\hangindent=23pt\noindent\textbf{Зацаринный~А.\,А., Чупраков~К.\,Г.} Некоторые аспекты выбора технологии для 
постро-\linebreak
\vspace*{-12pt}\\
\hspace*{23pt}ения систем отображения информации ситуационного центра$\dotfill$&3&59\\
\textbf{Зацман~И.\,М.} см.~Бунтман~Н.\,В.&&\\
\hangindent=23pt\noindent\textbf{Зейфман~А.\,И., Коротышева~А.\,В., Сатин~Я.\,А., Шоргин~С.\,Я.} Об 
устойчивости нестаци-\linebreak
\vspace*{-12pt}\\
\hspace*{23pt}онарных систем обслуживания с катастрофами$\dotfill$&3&9\\
\textbf{Зыкова~З.\,П.} см.~Архипов~О.\,П.&&\\
\hangindent=23pt\noindent\textbf{Илюшин~Г.\,Я., Соколов~И.\,А.} Организация управляемого доступа пользователей 
к\linebreak
\vspace*{-12pt}\\
\hspace*{23pt}разнородным ведомственным информационным ресурсам$\dotfill$&1&24\\
\hangindent=23pt\noindent\textbf{Кавагучи~Ю., Ульянов~В.\,В., Фуджикоши~Я.} Приближения для статистик, 
описывающих\linebreak
\vspace*{-12pt}\\
\hspace*{23pt}геометрические свойства данных большой размерности, с оценками 
ошибок$\dotfill$&1&12\\
\hangindent=23pt\noindent\textbf{Каратеев~С.\,Л., Бекетова~И.\,В., Ососков~М.\,В., Князь~В.\,А., 
Визильтер~Ю.\,В., Бондаренко~А.\,В., Желтов~С.\,Ю.} Автоматизированный контроль 
качества цифровых\linebreak
\vspace*{-12pt}\\
\hspace*{23pt}изображений для персональных документов$\dotfill$&1&65\\
\end{tabular}
}

\pagebreak

\def\leftkol{АВТОРСКИЙ УКАЗАТЕЛЬ ЗА 2010 г.} % ENGLISH ABSTRACTS}

\def\rightkol{АВТОРСКИЙ УКАЗАТЕЛЬ ЗА 2010 г.} %ENGLISH ABSTRACTS}

{\tabcolsep=3pt
\begin{tabular}{p{388pt}rr}
&\textbf{Выпуск} & \textbf{Стр.}\\[3pt]
\hangindent=23pt\noindent\textbf{Козеренко~Е.\,Б.} Лингвистические фильтры в статистических моделях машинного\linebreak
\vspace*{-12pt}\\
\hspace*{23pt}перевода$\dotfill$&2&83\\
\hangindent=23pt\noindent\textbf{Козеренко~Е.\,Б., Кузнецов~И.\,П.} Когнитивно-лингвистические представления в 
систе-\linebreak
\vspace*{-12pt}\\
\hspace*{23pt}мах обработки текстов$\dotfill$&3&69\\
\textbf{Князь~В.\,А.} см.~Каратеев~С.\,Л.&&\\
\hangindent=23pt\noindent\textbf{Колесников~А.\,В., Солдатов~С.\,А.} Алгоритм координации для гибридной 
интеллектуальной системы решения сложной задачи оперативно-производственного\linebreak
\vspace*{-12pt}\\
\hspace*{23pt}планирования$\dotfill$&4&61\\
\hangindent=23pt\noindent\textbf{Коновалов~М.\,Г.} О планировании потоков в системах вычислительных 
ресурсов$\dotfill$&2&3\\
\textbf{Конушин~А.\,С.} см.~Конушин~В.\,С.&&\\
\hangindent=23pt\noindent\textbf{Конушин~В.\,С., Кривовязь~Г.\,Р., Конушин~А.\,С.} Алгоритм распознавания людей 
в видео-\linebreak
\vspace*{-12pt}\\
\hspace*{23pt}последовательности по одежде$\dotfill$&1&74\\
\textbf{Корепанов~Э.\, Р.} см.~Синицын~И.\,Н.&&\\
\textbf{Королев~В.\,Ю.} см.~Соколов~И.\,А.&&\\
\textbf{Королев~Р.\,А.} см.~Бенинг~В.\,Е.&&\\
\textbf{Коротышева~А.\,В.} см.~Зейфман~А.\,И.&&\\
\hangindent=23pt\noindent\textbf{Кривенко~М.\,П.} Непараметрическое оценивание элементов байесовского 
клас\-си-\linebreak
\vspace*{-12pt}\\
\hspace*{23pt}фикатора$\dotfill$&2&13\\
\textbf{Кривовязь~Г.\,Р.} см.~Конушин~В.\,С.&&\\
\textbf{Крылов~А.\,С.} см.~Павельева~Е.\,А.&&\\
\hangindent=23pt\noindent\textbf{Крылов~В.\,А.} Моделирование и классификация многоканальных дистанционных\linebreak
\vspace*{-12pt}\\
\hspace*{23pt}изображений с использованием копул$\dotfill$&4&34\\
\hangindent=23pt\noindent\textbf{Крючин~О.\,В.} Разработка параллельных эвристических алгоритмов подбора 
весовых\linebreak
\vspace*{-12pt}\\
\hspace*{23pt}коэффициентов искусственной нейтронной сети$\dotfill$&2&53\\
\hangindent=23pt\noindent\textbf{Кудрявцев~А.\,А., Шоргин~С.\,Я.} Байесовские модели массового обслуживания и 
надеж-\linebreak
\vspace*{-12pt}\\
\hspace*{23pt}ности: характеристики среднего числа заявок в системе $M\vert M \vert 1\vert 
\infty$$\dotfill$&3&16\\
\hangindent=23pt\noindent\textbf{Кузнецов~А.\,А.} Связь между временными и структурно-топологическими 
характери-\linebreak
\vspace*{-12pt}\\
\hspace*{23pt}стиками диаграмм ритма сердца здоровых людей$\dotfill$&4&39\\
\textbf{Кузнецов~И.\,П.} см.~Козеренко~Е.\,Б.&&\\
\textbf{Ле~Пезан~Д.} см.~Бунтман~Н.\,В.&&\\
\hangindent=23pt\noindent\textbf{Лукьяненко~А.\,С., Морозов~Е.\,В., Гуртов~А.\,В.} Анализ сетевого протокола с общей 
функ-\linebreak
\vspace*{-12pt}\\
\hspace*{23pt}цией расширения окна передачи сообщения при конфликтах$\dotfill$&2&46\\
\hangindent=23pt\noindent\textbf{Лямин~О.\,О.} О предельном поведении мощностей критериев в случае обобщенного\linebreak
\vspace*{-12pt}\\
\hspace*{23pt}распределения Лапласа$\dotfill$&3&47\\
\hangindent=23pt\noindent\textbf{Маркин~А.\,В., Шестаков~О.\,В.} Асимптотики оценки риска при пороговой 
обработке\linebreak
\vspace*{-12pt}\\
\hspace*{23pt}вейвлет-вейглет коэффициентов в задаче томографии$\dotfill$&2&36\\
\hangindent=23pt\noindent\textbf{Матвеева~С.\,С., Захарова~Т.\,В.} Сети массового обслуживания с наименьшей 
длиной\linebreak
\vspace*{-12pt}\\
\hspace*{23pt}очереди$\dotfill$&3&22\\
\hangindent=23pt\noindent\textbf{Матюшенко~С.\,И.} Стационарные характеристики двухканальной системы 
обслужива-\linebreak
\vspace*{-12pt}\\
\hspace*{23pt}ния с переупорядочиванием заявок и распределениями фазового типа$\dotfill$&4&68\\
\textbf{Минель~Ж.-Л.} см.~Бунтман~Н.\,В.&&\\
\textbf{Морозов~Е.\,В.} см.~Бородина~А.\,В.&&\\
\textbf{Морозов~Е.\,В.} см.~Лукьяненко~А.\,С.&&\\
\textbf{Ососков~М.\,В.} см.~Каратеев~С.\,Л.&&\\
\hangindent=23pt\noindent\textbf{Павельева~Е.\,А., Крылов~А.\,С.} Поиск и анализ ключевых точек радужной 
оболочки\linebreak
\vspace*{-12pt}\\
\hspace*{23pt}глаза методом преобразования Эрмита$\dotfill$&1&79\\
\textbf{Печинкин~А.\,В.} см.~Френкель~С.\,Л.,&&\\
\hangindent=23pt\noindent\textbf{Протасов~В.\,И.} Составление субъективного портрета с использованием 
эволюционно-\linebreak
\vspace*{-12pt}\\
\hspace*{23pt}го морфинга и квалиметрия метода$\dotfill$&1&83\\
\hangindent=23pt\noindent\textbf{Рудаков~К.\,В., Торшин~И.\,Ю.} Вопросы разрешимости задачи распознавания 
вторичной\linebreak
\vspace*{-12pt}\\
\hspace*{23pt}структуры белка$\dotfill$&2&25\\
\textbf{Сатин~Я.\,А.} см.~Зейфман~А.\,И.&&\\
\hangindent=23pt\noindent\textbf{Сейфуль-Мулюков~Р.\,Б.} Нефть как носитель информации о своем 
происхождении,\linebreak
\vspace*{-12pt}\\
\hspace*{23pt}структуре и эволюции$\dotfill$&1&41\\
\end{tabular}
}

{\tabcolsep=3pt
\begin{tabular}{p{388pt}rr}
&\textbf{Выпуск} & \textbf{Стр.}\\[6pt]
\textbf{Семендяев~Н.\,Н.} см.~Синицын~И.\,Н.&&\\
\textbf{Серебряков~В.\,А.} см.~Захаров~А.\,А.&&\\
\textbf{Синицын~В.\,И.} см.~Синицын~И.\,Н.&&\\
\hangindent=23pt\noindent\textbf{Синицын~И.\,Н., Синицын~В.\,И., Корепанов~Э.\, Р., Белоусов~В.\,В., 
Семендяев~Н.\,Н.} Оперативное построение информационных моделей движения полюса 
Земли\linebreak
\vspace*{-12pt}\\
\hspace*{23pt}методами линейных и линеаризованных фильтров$\dotfill$&1&2\\
\textbf{Сипина~А.\,В.} см.~Бенинг~В.\,Е.&&\\
\hangindent=23pt\noindent\textbf{Соколов~И.\,А.} О работах заслуженного деятеля науки Российской Федерации 
И.\,Н.~Синицына в области информационных технологий и автоматизации (к 70-летию\linebreak
\vspace*{-12pt}\\
\hspace*{23pt}со дня рождения)$\dotfill$&3&84\\
\textbf{Соколов~И.\,А.} см.~Илюшин~Г.\,Я.&&\\
\hangindent=23pt\noindent\textbf{Соколов~И.\,А., Королев~В.\,Ю.} Предисловие$\dotfill$&2&2\\
\textbf{Солдатов~С.\,А.} см.~Колесников~А.\,В.&&\\
\hangindent=23pt\noindent\textbf{Степанов~С.\,Ю.} Использование координатного метода фрагментации 
коммутаторной\linebreak
\vspace*{-12pt}\\
\hspace*{23pt}нейронной сети для сокращения трафика$\dotfill$&2&57\\
\textbf{Тимонина~Е.\,Е.} см.~Грушо~А.\,А.&&\\
\textbf{Торшин~И.\,Ю.} см.~Рудаков~К.\,В.&&\\
\textbf{Ульянов~В.\,В.} см.~Кавагучи~Ю.&&\\
\textbf{Фазекаш~И.} см.~Чупрунов~А.\,Н.&&\\
\textbf{Френкель~С.\,Л.} см.~Баранов~С.\,И.&&\\
\hangindent=23pt\noindent\textbf{Френкель~С.\,Л., Печинкин~А.\,В.} Оценка времени самовосстановления в 
цифровых\linebreak
\vspace*{-12pt}\\
\hspace*{23pt}системах после сбоев, вызываемых переходными помехами$\dotfill$&3&2\\
\textbf{Фуджикоши~Я.} см.~Кавагучи~Ю.&&\\
\hangindent=23pt\noindent\textbf{Цискаридзе~А.\,К.} Математическая модель и метод восстановления позы человека 
по\linebreak
\vspace*{-12pt}\\
\hspace*{23pt}стереопаре силуэтных изображений$\dotfill$&4&27\\
\hangindent=23pt\noindent\textbf{Чупраков~К.\,Г.} К вопросу о размещении коллективных средств отображения в 
ситуа-\linebreak
\vspace*{-12pt}\\
\hspace*{23pt}ционном зале с заданными параметрами$\dotfill$&4&89\\
\textbf{Чупраков~К.\,Г.} см.~Зацаринный~А.\,А.&&\\
\hangindent=23pt\noindent\textbf{Чупрунов~А.\,Н., Фазекаш~И.} Законы повторного логарифма для числа 
безошибочных\linebreak
\vspace*{-12pt}\\
\hspace*{23pt}блоков при помехоустойчивом кодировании$\dotfill$&3&42\\
\textbf{Шевцова~И.\,Г.} см.~Григорьева~М.\,Е.&&\\
\hangindent=23pt\noindent\textbf{Шестаков~О.\,В.} Аппроксимация распределения оценки риска пороговой 
обработки вейвлет-коэффициентов нормальным распределением при использовании 
выбо-\linebreak
\vspace*{-12pt}\\
\hspace*{23pt}рочной дисперсии$\dotfill$&4&73\\
\textbf{Шестаков~О.\,В.} см.~Маркин~А.\,В.&&\\
\textbf{Шоргин~С.\,Я.} см.~Зейфман~А.\,И.&&\\
\textbf{Шоргин~С.\,Я.} см.~Кудрявцев~А.\,А.&&\\
\end{tabular}
}

%\thispagestyle{myheadings}
\def\leftfootline{\small{\textbf{\thepage}
\hfill ИНФОРМАТИКА И ЕЁ ПРИМЕНЕНИЯ\ \ \ том~4\ \ \ выпуск~4\ \ \ 2010}
}%
 \def\rightfootline{\small{ИНФОРМАТИКА И ЕЁ ПРИМЕНЕНИЯ\ \ \ том~4\ \ \ выпуск~4\ \ \ 2010
 \hfill \textbf{\thepage}}}
 \label{end\stat}


%Том 10 Выпуск 1-4 Год 2016

\def\stat{cont-e}
{%\hrule\par
%\vskip 7pt % 7pt
\raggedleft\Large \bf%\baselineskip=3.2ex
2\,0\,1\,6\ \ A\,U\,T\,H\,O\,R\ \ I\,N\,D\,E\,X \vskip 17pt
 \hrule
 \par
\vskip 21pt plus 6pt minus 3pt }

\label{st\stat}

\def\tit{\ }

\def\aut{\ }
\def\auf{\ }

\def\leftkol{\ } %2016 AUTHOR INDEX} % ENGLISH ABSTRACTS}

\def\rightkol{\ } %2016 AUTHOR INDEX} %ENGLISH ABSTRACTS}

\titele{\tit}{\aut}{\auf}{\leftkol}{\rightkol}

\def\leftfootline{\small{\textbf{\thepage}
\hfill INFORMATIKA I EE PRIMENENIYA~--- INFORMATICS AND APPLICATIONS\ \ \ 2016\
\ \ volume~10\ \ \ issue\ 4}
}%
 \def\rightfootline{\small{INFORMATIKA I EE PRIMENENIYA~--- INFORMATICS AND APPLICATIONS\ \ \ 2016\ \ \ volume~10\ \ \ issue\ 4
\hfill \textbf{\thepage}}}

\vspace*{-12pt}
\vspace*{-18pt}

{\tabcolsep=2.8pt
\begin{tabular}{p{382pt}cc}
&\textbf{Issue} & \textbf{Page}\\[6pt]
\Avtors{Agalarov~M.\,Ya.} see~Agalarov~Ya.\,M.&&\\
\Avtors{Agalarov~Ya.\,M., Agalarov~M.\,Ya., and
Shorgin~V.\,S.} About the optimal threshold of queue\linebreak
\\[-12pt]
\hspace*{23pt}length in a~particular problem of profit maximization
in the $M/G/1$ queuing system&2&70--79\\
\Avtors{Alexeyevsky~D.\,A.} BioNLP ontology extraction from 
a~restricted language corpus with\linebreak
\\[-12pt]
\hspace*{23pt}context-free grammars&1&119--128\\
\Avtors{Andreev~S.\,D.} see~Gaidamaka~Yu.\,V.&&\\
\Avtors{Andreev~S.\,D.} see~Ometov~A.\,Ya.&&\\
\Avtors{Arkhipov~O.\,P., Arkhipov~P.\,O., and Sidorkin~I.\,I.} The
option to create a~local coordinate\linebreak
\\[-12pt]
\hspace*{23pt}system for synchronization of selected images&3&91--97\\
\Avtors{Arkhipov~P.\,O.} see~Arkhipov~O.\,P.&&\\
\Avtors{Belousov~V.\,V.} see~Shnurkov~P.\,V.&&\\
\Avtors{Belousov~V.\,V.} see~Shnurkov~P.\,V.&&\\
\Avtors{Bening~V.\,E.} Calculation of~the~asymptotic deficiency
of~some statistical procedures based\linebreak
\\[-12pt]
\hspace*{23pt}on~samples with~random sizes&4&34--45\\
\Avtors{Borisov~A.\,V., Bosov~A.\,V., and Miller~G.\,B.} Modeling and
monitoring of VoIP connection&2&\hphantom{1}2--13\\
\Avtors{Bosov~A.\,V.} see~Borisov~A.\,V.&&\\
\Avtors{Briukhov~D.\,O.} see~Stupnikov~S.\,A.&&\\
\Avtors{Callaos~N.\,K.\ and Seyful-Mulyukov~R.\,B.} Complexity and
its information content&1&129--139\\
\Avtors{Chertok~A.\,V., Kadaner~A.\,I., Khazeeva~G.\,T., and
Sokolov~I.\,A.} Regime switching detection\linebreak
\\[-12pt]
\hspace*{23pt}for~the~Levy driven
Ornstein--Uhlenbeck process using CUSUM methods&4&46--56\\
\Avtors{Chichagov~V.\,V.} Asymptotic expansions of mean absolute
error of uniformly minimum variance unbiased and maximum likelihood
estimators on the one-parameter exponential\linebreak
\\[-12pt]
\hspace*{23pt}family model of lattice distributions&3&66--76\\
\Avtors{Danishevsky~V.\,I.} see~Kolesnikov A.\,V.&&\\
\Avtors{Fazliev~A.\,Z.} see~Kalinichenko~L.\,A.&&\\
\Avtors{Fedoseev~A.\,A.} What is behind the concept of ``knowledge in
small packages''&3&105--110\\
\Avtors{Gaidamaka~Yu.\,V., Andreev~S.\,D., Sopin~E.\,S.,
Samouylov~K.\,E., and Shorgin~S.\,Ya.} Interference analysis
of~the~device-to-device communications model with~regard to~a~signal\linebreak
\\[-12pt]
\hspace*{23pt}propagation environment&4&\hphantom{1}2--10\\
\Avtors{Gasilov~A.\,V.} see~Yakovlev~O.\,A.&&\\
\Avtors{Goncharov~A.\,V.\ and Strijov~V.\,V.} Metric time series
classification using weighted dynamic\linebreak
\\[-12pt]
\hspace*{23pt}warping relative to centroids of classes&2&36--47\\
\Avtors{Gordov~E.\,P.} see~Kalinichenko~L.\,A.&&\\
\Avtors{Gorshenin~A.\,K.} Concept of online service for stochastic
modeling of real processes&1&72--81\\
\Avtors{Gorshenin~A.\,K.} see~Shnurkov~P.\,V.&&\\
\Avtors{Gorshenin~A.\,K.} see~Shnurkov~P.\,V.&&\\
\Avtors{Grusho~A.\,A., Grusho~N.\,A., Zabezhailo~M.\,I., and
Timonina~E.\,E.} Integration of statistical and\linebreak
\\[-12pt]
\hspace*{23pt}deterministic methods for
analysis of information security&3&2--8\\
\Avtors{Grusho~A.\,A., Zabezhailo~M.\,I., and Zatsarinny~A.\,A.} On
the advanced procedure to reduce\linebreak
\\[-12pt]
\hspace*{23pt}calculation of Galois closures&4&\hphantom{1}96--104\\
\Avtors{Grusho~N.\,A.} see~Grusho~A.\,A.&&\\
\Avtors{Havanskov~V.\,A.} see~Minin~V.\,A.&&\\
\Avtors{Inkova~O.\,Yu.} see~Zatsman~I.\,M.&&\\
\Avtors{Isachenko~R.\,V.\ and Strijov~V.\,V.} Metric learning in
multiclass time series classification\linebreak
\\[-12pt]
\hspace*{23pt}problem&2&48--57\\
\end{tabular}
}
\pagebreak

\def\leftfootline{\small{\textbf{\thepage}
\hfill INFORMATIKA I EE PRIMENENIYA~--- INFORMATICS AND APPLICATIONS\ \ \ 2016\
\ \ volume~10\ \ \ issue\ 4}
}%
 \def\rightfootline{\small{INFORMATIKA I EE PRIMENENIYA~---
INFORMATICS AND APPLICATIONS\ \ \ 2016\ \ \ volume~10\ \ \ issue\ 4
\hfill \textbf{\thepage}}}

\def\leftkol{2016 AUTHOR INDEX} % ENGLISH ABSTRACTS}

\def\rightkol{2016 AUTHOR INDEX} %ENGLISH ABSTRACTS}


{\tabcolsep=2.83pt
\begin{tabular}{p{382pt}cc}
&\textbf{Issue} & \textbf{Page}\\[6pt]
\Avtors{Kadaner~A.\,I.} see~Chertok~A.\,V.&&\\[.255pt]
\Avtors{Kalinichenko~L.\,A., Volnova~A.\,A., Gordov~E.\,P.,
Kiselyova~N.\,N., Kovaleva~D.\,A., Malkov~O.\,Yu., Okladnikov~I.\,G.,
Podkolodnyy~N.\,L., Pozanenko~A.\,S., Ponomareva~N.\,V.,
Stupnikov~S.\,A.,} \textbf{and Fazliev~A.\,Z.} Data access challenges for data
intensive\linebreak
\\[-12pt]
\hspace*{23pt}research in Russia&1& 2--22\\[.255pt]
\Avtors{Karasikov~M.\,E.\ and Strijov~V.\,V.} Feature-based
time-series classification&4&121--131\\[.255pt]
\Avtors{Khazeeva~G.\,T.} see~Chertok~A.\,V.&&\\[.255pt]
\Avtors{Khokhlov~Yu.\,S.} Multivariate fractional Levy motion and its
applications&2&\hphantom{1}98--106\\[.255pt]
\Avtors{Kirikov~I.\,A., Kolesnikov~A.\,V., Listopad~S.\,V., and
Rumovskaya~S.\,B.} Fine-grained hybrid\linebreak
\\[-12pt]
\hspace*{23pt}intelligent systems. Part 2:
Bidirectional hybridization&1&\hphantom{1}96--105\\[.255pt]
\Avtors{Kirikov~I.\,A., Kolesnikov~A.\,V., Listopad~S.\,V., and
Rumovskaya~S.\,B.} ``Virtual council''~---\linebreak
\\[-12pt]
\hspace*{23pt}source environment
supporting complex diagnostic decision making&3&81--90\\[.255pt]
\Avtors{Kiselyova~N.\,N.} see~Kalinichenko~L.\,A.&&\\[.255pt]
\Avtors{Kolesnikov A.\,V., Listopad~S.\,V., Rumovskaya~S.\,B., and
Danishevsky~V.\,I.} Informal axiomatic\linebreak
\\[-12pt]
\hspace*{23pt}theory of~the~role visual models&4&114--120\\[.255pt]
\Avtors{Kolesnikov~A.\,V.} see~Kirikov~I.\,A.&&\\[.255pt]
\Avtors{Kolesnikov~A.\,V.} see~Kirikov~I.\,A.&&\\[.255pt]
\Avtors{Kolin~K.\,K.} Humanitarian aspects of information
security&3&111--121\\[.255pt]
\Avtors{Konovalov~M.\,G.\ and Razumchik~R.\,V.} Dispatching
to~two parallel nonobservable queues using\linebreak
\\[-12pt]
\hspace*{23pt}only static
information&4&57--67\\[.255pt]
\Avtors{Korchagin~A.\,Yu.} see~Korolev~V.\,Yu.&&\\[.255pt]
\Avtors{Korchagin~A.\,Yu.} see~Korolev~V.\,Yu.&&\\[.255pt]
\Avtors{Korepanov~E.\,R.} see~Sinitsyn~I.\,N.&&\\[.255pt]
\Avtors{Korepanov~E.\,R.} see~Sinitsyn~I.\,N.&&\\[.255pt]
\Avtors{Korolev~V.\,Yu., Korchagin~A.\,Yu., and Zeifman~A.\,I.} The
Poisson theorem for Bernoulli trials\linebreak
\\[-12pt]
\hspace*{23pt}with~a~random probability
of~success and~a~discrete analog of~the~Weibull distribution&4&11--20\\[.255pt]
\Avtors{Korolev~V.\,Yu., Zeifman~A.\,I., and Korchagin~A.\,Yu.}
Asymmetric Linnik distributions as~limit\linebreak
\\[-12pt]
\hspace*{23pt}laws for~random sums
of~independent random variables with~finite variances&4&21--33\\[.255pt]
\Avtors{Koucheryavy~E.\,A.} see~Ometov~A.\,Ya.&&\\[.255pt]
\Avtors{Kovaleva~D.\,A.} see~Kalinichenko~L.\,A.&&\\[.255pt]
\Avtors{Kovalyov~S.\,P.} Metaprogramming to increase
manufacturability of large-scale software-\linebreak
\\[-12pt]
\hspace*{23pt}intensive systems&1&56--66\\[.255pt]
\Avtors{Krivenko~M.\,P.} Significance tests of feature selection for
classification&3&32--40\\[.255pt]
\Avtors{Kruzhkov~M.\,G.} see~Zalizniak~Anna~A.&&\\[.255pt]
\Avtors{Kruzhkov~M.\,G.} see~Zatsman~I.\,M.&&\\[.255pt]
\Avtors{Kudryavtsev~A.\,A.} Bayesian queueing and reliability models:
\textit{A~priori} distributions with\linebreak
\\[-12pt]
\hspace*{23pt}compact support&1&67--71\\[.255pt]
\Avtors{Kudryavtsev~A.\,A.} Characteristics dependent on the balance
coefficient in Bayesian models\linebreak
\\[-12pt]
\hspace*{23pt}with compact support of \textit{a priori}
distributions&3&77--80\\[.255pt]
\Avtors{Kudryavtsev~A.\,A.\ and Palionnaia~S.\,I.} Bayesian recurrent
model of reliability growth:\linebreak
\\[-12pt]
\hspace*{23pt}Parabolic distribution of parameters&2&80--83\\[.255pt]
\Avtors{Kudryavtsev~A.\,A.\ and Titova~A.\,I.} Bayesian queuing
and~reliability models: Degenerate-\linebreak
\\[-12pt]
\hspace*{23pt}Weibull case&4&68--71\\[.255pt]
\Avtors{Leontyev~N.\,D.\ and Ushakov~V.\,G.} Analysis of a queueing
system with autoregressive arrivals\linebreak
\\[-12pt]
\hspace*{23pt}and nonpreemptive priority&3&15--22\\[.255pt]
\Avtors{Listopad~S.\,V.} see~Kirikov~I.\,A.&&\\[.255pt]
\Avtors{Listopad~S.\,V.} see~Kirikov~I.\,A.&&\\[.255pt]
\Avtors{Listopad~S.\,V.} see~Kolesnikov A.\,V.&&\\[.255pt]
\Avtors{Malkov~O.\,Yu.} see~Kalinichenko~L.\,A.&&\\[.255pt]
\Avtors{Markov~A.\,S., Monakhov~M.\,M., and
Ulyanov~V.\,V.} Generalized Cornish--Fisher expansions\linebreak
\\[-12pt]
\hspace*{23pt}for distributions of statistics based on samples
of random size&2&84--91\\[.255pt]
\Avtors{Melnikov~A.\,K.\ and Ronzhin~A.\,F.} Generalized statistical
method of~text analysis based\linebreak
\\[-12pt]
\hspace*{23pt}on~calculation of~probability distributions
of~statistical values&4&89--95\\
\end{tabular}
}
\pagebreak

\def\leftfootline{\small{\textbf{\thepage}
\hfill INFORMATIKA I EE PRIMENENIYA~--- INFORMATICS AND APPLICATIONS\ \ \ 2016\
\ \ volume~10\ \ \ issue\ 4}
}%
 \def\rightfootline{\small{INFORMATIKA I EE PRIMENENIYA~---
INFORMATICS AND APPLICATIONS\ \ \ 2016\ \ \ volume~10\ \ \ issue\ 4
\hfill \textbf{\thepage}}}

\def\leftkol{2016 AUTHOR INDEX} % ENGLISH ABSTRACTS}

\def\rightkol{2016 AUTHOR INDEX} %ENGLISH ABSTRACTS}


{\tabcolsep=3pt
\begin{tabular}{p{381pt}cc}
&\textbf{Issue} & \textbf{Page}\\[6pt]
\Avtors{Meykhanadzhyan~L.\,A.} Stationary characteristics of the finite
capacity queueing system with\linebreak
\\[-12pt]
\hspace*{23pt}inverse service order and generalized
probabilistic priority&2&123--131\\[.23pt]
\Avtors{Miller~G.\,B.} see~Borisov~A.\,V.&&\\[.23pt]
\Avtors{Minin~V.\,A., Zatsman~I.\,M., Havanskov~V.\,A., and
Shubnikov~S.\,K.} Intensity of citation of scientific publications in
inventions on information and computer technologies patented\linebreak
\\[-12pt]
\hspace*{23pt}in Russia by domestic and foreign applicants&2&107--122\\[.23pt]
\Avtors{Monakhov~M.\,M.} see~Markov~A.\,S.&&\\[.23pt]
\Avtors{Naumov~V.\,A.\ and Samouylov~K.\,E.} On relationship
between queuing systems with resources\linebreak
\\[-12pt]
\hspace*{23pt}and Erlang networks&3&\hphantom{1}9--14\\[.23pt]
\Avtors{Okladnikov~I.\,G.} see~Kalinichenko~L.\,A.&&\\[.23pt]
\Avtors{Ometov~A.\,Ya., Andreev~S.\,D., Turlikov~A.\,M., and
Koucheryavy~E.\,A.} Performance analysis of\linebreak
\\[-12pt]
\hspace*{23pt}a wireless data
aggregation system with contention for contemporary sensor
networks&3&23--31\\[.23pt]
\Avtors{Palionnaia~S.\,I.} see~Kudryavtsev~A.\,A.&&\\[.23pt]
\Avtors{Podkolodnyy~N.\,L.} see~Kalinichenko~L.\,A.&&\\[.23pt]
\Avtors{Ponomareva~N.\,V.} see~Kalinichenko~L.\,A.&&\\[.23pt]
\Avtors{Popkova~N.\,A.} see~Zatsman~I.\,M.&&\\[.23pt]
\Avtors{Pozanenko~A.\,S.} see~Kalinichenko~L.\,A.&&\\[.23pt]
\Avtors{Razumchik~R.\,V.} see~Konovalov~M.\,G.&&\\[.23pt]
\Avtors{Ronzhin~A.\,F.} see~Melnikov~A.\,K.&&\\[.23pt]
\Avtors{Rumovskaya~S.\,B.} see~Kirikov~I.\,A.&&\\[.23pt]
\Avtors{Rumovskaya~S.\,B.} see~Kirikov~I.\,A.&&\\[.23pt]
\Avtors{Rumovskaya~S.\,B.} see~Kolesnikov A.\,V.&&\\[.23pt]
\Avtors{Samouylov~K.\,E.} see~Gaidamaka~Yu.\,V.&&\\[.23pt]
\Avtors{Samouylov~K.\,E.} see~Naumov~V.\,A.&&\\[.23pt]
\Avtors{Serebryanskii~S.\,M.} see~Tyrsin~A.\,N.&&\\[.23pt]
\Avtors{Seyful-Mulyukov~R.\,B.} see~Callaos~N.\,K.&&\\[.23pt]
\Avtors{Shestakov~O.\,V.} Statistical properties of the denoising method
based on the stabilized hard\linebreak
\\[-12pt]
\hspace*{23pt}thresholding&2&65--69\\[.23pt]
\Avtors{Shestakov~O.\,V.} The strong law of large numbers for the risk
estimate in the problem of\linebreak
\\[-12pt]
\hspace*{23pt}tomographic image reconstruction from
projections with a correlated noise&3&41--45\\[.23pt]
\Avtors{Shestakov~O.\,V.} see~Zakharova~T.\,V.&&\\[.23pt]
\Avtors{Shnurkov~P.\,V., Gorshenin~A.\,K., and Belousov~V.\,V.}
Analytical solution of~the~optimal control\linebreak
\\[-12pt]
\hspace*{23pt}task of~a~semi-Markov
process with~finite set of~states&4&72--88\\[.23pt]
\Avtors{Shnurkov~P.\,V., Zasypko~V.\,V., Belousov~V.\,V., and
Gorshenin~A.\,K.} Development of the algorithm of numerical solution
of the optimal investment control problem\linebreak
\\[-12pt]
\hspace*{23pt}in the closed dynamical model of three-sector economy&1&82--95\\[.23pt]
\Avtors{Shorgin~S.\,Ya.} see~Gaidamaka~Yu.\,V.&&\\[.23pt]
\Avtors{Shorgin~V.\,S.} see~Agalarov~Ya.\,M.&&\\[.23pt]
\Avtors{Shubnikov~S.\,K.} see~Minin~V.\,A.&&\\[.23pt]
\Avtors{Sidorkin~I.\,I.} see~Arkhipov~O.\,P.&&\\[.23pt]
\Avtors{Sinitsyn~I.\,N.} Analytical modeling of processes in stochastic
systems with complex fractional\linebreak
\\[-12pt]
\hspace*{23pt}order Bessel nonlinearities&3&55--65\\[.23pt]
\Avtors{Sinitsyn~I.\,N.} Orthogonal supoptimal filters for nonlinear
stochastic systems on manifolds&1&34--44\\[.23pt]
\Avtors{Sinitsyn~I.\,N.\ and Korepanov~E.\,R.} Normal Pugachev
conditionally-optimal filters and extra-\linebreak
\\[-12pt]
\hspace*{23pt}polators for state linear stochastic systems&2&14--23\\[.23pt]
\Avtors{Sinitsyn~I.\,N.\ and Sinitsyn~V.\,I.} Analytical modeling of
distributions in stochastic systems on\linebreak
\\[-12pt]
\hspace*{23pt}manifolds based on ellipsoidal approximation&1&45--55\\[.23pt]
\Avtors{Sinitsyn~I.\,N., Sinitsyn~V.\,I., and
Korepanov~E.\,R.} Ellipsoidal suboptimal filters for nonlinear\linebreak
\\[-12pt]
\hspace*{23pt}stochastic systems on manifolds&2&24--35\\[.23pt]
\Avtors{Sinitsyn~V.\,I.} see~Sinitsyn~I.\,N.&&\\[.23pt]
\Avtors{Sinitsyn~V.\,I.} see~Sinitsyn~I.\,N.&&\\[.23pt]
\Avtors{Skvortsov~N.\,A.} see~Stupnikov~S.\,A.&&\\[.23pt]
\Avtors{Sokolov~I.\,A.} see~Chertok~A.\,V.&&\\
\end{tabular}
}
\pagebreak

\def\leftfootline{\small{\textbf{\thepage}
\hfill INFORMATIKA I EE PRIMENENIYA~--- INFORMATICS AND APPLICATIONS\ \ \ 2016\
\ \ volume~10\ \ \ issue\ 4}
}%
 \def\rightfootline{\small{INFORMATIKA I EE PRIMENENIYA~---
INFORMATICS AND APPLICATIONS\ \ \ 2016\ \ \ volume~10\ \ \ issue\ 4
\hfill \textbf{\thepage}}}

\def\leftkol{2016 AUTHOR INDEX} % ENGLISH ABSTRACTS}

\def\rightkol{2016 AUTHOR INDEX} %ENGLISH ABSTRACTS}


{\tabcolsep=3pt
\begin{tabular}{p{382pt}cc}
&\textbf{Issue} & \textbf{Page}\\[6pt]
\Avtors{Sopin~E.\,S.} see~Gaidamaka~Yu.\,V.&&\\
\Avtors{Strijov~V.\,V.} see~Goncharov~A.\,V.&&\\
\Avtors{Strijov~V.\,V.} see~Isachenko~R.\,V.&&\\
\Avtors{Strijov~V.\,V.} see~Karasikov~M.\,E.&&\\
\Avtors{Stupnikov~S.\,A., Briukhov~D.\,O., and Skvortsov~N.\,A.}
Co-lending systemic risk analysis over\linebreak
\\[-12pt]
\hspace*{23pt}heterogeneous data collections&1&23--33\\
\Avtors{Stupnikov~S.\,A.} see~Kalinichenko~L.\,A.&&\\
\Avtors{Suchkov~A.\,P.} see~Zatsarinny~A.\,A.&&\\
\Avtors{Timonina~E.\,E.} see~Grusho~A.\,A.&&\\
\Avtors{Titova~A.\,I.} see~Kudryavtsev~A.\,A.&&\\
\Avtors{Turlikov~A.\,M.} see~Ometov~A.\,Ya.&&\\
\Avtors{Tyrsin~A.\,N.\ and Serebryanskii~S.\,M.} Recognition of
dependences on the basis of inverse\linebreak
\\[-12pt]
\hspace*{23pt}mapping&2&58--64\\
\Avtors{Ulyanov~V.\,V.} see~Markov~A.\,S.&&\\
\Avtors{Ushakov~V.\,G.} Queueing system with working vacations and
hyperexponential input stream&2&92--97\\
\Avtors{Ushakov~V.\,G.} see~Leontyev~N.\,D.&&\\
\Avtors{Volnova~A.\,A.} see~Kalinichenko~L.\,A.&&\\
\Avtors{Yakovlev~O.\,A.\ and Gasilov~A.\,V.} Speeded-up stereo
matching using geodesic support weights&3&\hphantom{1}98--104\\
\Avtors{Zabezhailo~M.\,I.} see~Grusho~A.\,A.&&\\
\Avtors{Zabezhailo~M.\,I.} see~Grusho~A.\,A.&&\\
\Avtors{Zakharova~T.\,V.\ and Shestakov~O.\,V.} Precision analysis of
wavelet processing of aerodynamic\linebreak
\\[-12pt]
\hspace*{23pt}flow patterns&3&46--54\\
\Avtors{Zalizniak~Anna~A.\ and Kruzhkov~M.\,G.} Database
of~Russian impersonal verbal constructions&4&132--141\\
\Avtors{Zasypko~V.\,V.} see~Shnurkov~P.\,V.&&\\
\Avtors{Zatsarinny~A.\,A.\ and Suchkov~A.\,P.} Systems engineering
approaches to~the~establishment of\linebreak
\\[-12pt]
\hspace*{23pt}a~system for~decision support based
on~situational analysis&4&105--113\\
\Avtors{Zatsarinny~A.\,A.} see~Grusho~A.\,A.&&\\
\Avtors{Zatsman~I.\,M., Inkova~O.\,Yu., Kruzhkov~M.\,G., and
Popkova~N.\,A.} Representation of cross-\linebreak
\\[-12pt]
\hspace*{23pt}lingual knowledge about
connectors in supracorpora databases&1&106--118\\
\Avtors{Zatsman~I.\,M.} see~Minin~V.\,A.&&\\
\Avtors{Zeifman~A.\,I.} see~Korolev~V.\,Yu.&&\\
\Avtors{Zeifman~A.\,I.} see~Korolev~V.\,Yu.&&\\
\end{tabular}
}

%\thispagestyle{myheadings}
\def\leftfootline{\small{\textbf{\thepage}
\hfill INFORMATIKA I EE PRIMENENIYA~--- INFORMATICS AND APPLICATIONS\ \ \ 2016\
\ \ volume~10\ \ \ issue\ 4}
}%
 \def\rightfootline{\small{INFORMATIKA I EE PRIMENENIYA~---
INFORMATICS AND APPLICATIONS\ \ \ 2016\ \ \ volume~10\ \ \ issue\ 4
\hfill \textbf{\thepage}}}

 \label{end\stat}

\newpage



%\def\stat{cont}
{%\hrule\par
%\vskip 7pt % 7pt
\raggedleft\Large \bf%\baselineskip=3.2ex
А\,В\,Т\,О\,Р\,С\,К\,И\,Й\ \ У\,К\,А\,З\,А\,Т\,Е\,Л\,Ь\ \ З\,А\ \ 2\,0\,0\,7 г. \vskip 17pt
    \hrule
    \par
\vskip 21pt plus 6pt minus 3pt }

\label{st\stat}

\def\tit{\ }

\def\aut{\ }
\def\auf{\ }

\def\leftkol{\ } % ENGLISH ABSTRACTS}

\def\rightkol{\ } %ENGLISH ABSTRACTS}

\titele{\tit}{\aut}{\auf}{\leftkol}{\rightkol}


\contentsline {chapter}{\ }{Выпуск \quad Стр.} 
\contentsline {section}{\textbf{Батракова Д.\,А., Королев В.\,Ю., Шоргин С.\,Я.}\ \ Новый метод вероятностно-ста\-ти\-сти\-че\-ско\-го анализа информационных потоков в\nobreakspace {}телекоммуникационных сетях}{\qquad 1 \qquad 40} 
\contentsline {section}{\textbf{Борисов А.\,В.}\ \ Байесовское оценивание в системах наблюдения с\nobreakspace {}марковскими скачкообразными процессами: игровой подход}{\qquad 2 \qquad 65}
\contentsline {section}{\textbf{Босов А.\,В., Иванов А.\,В.}\ \ Программная инфраструктура информационного Web-пор\-тала}{\qquad 2 \qquad 50}
\contentsline {section}{\textbf{Захаров В.\,Н., Калиниченко Л.\,А., Соколов И.\,А., Ступников С.\,А.}\ \ Конструирование канонических информационных моделей для интегрированных информационных систем}{\qquad 2 \qquad 15}
\contentsline {section}{\textbf{Захаров В.\,Н., Козмидиади В.\,А.}\ \ Средства обеспечения отказоустойчивости при\-ло\-жений}{\qquad 1 \qquad 14} 
\contentsline {section}{\textbf{Иванов А.\,В.}\ \ см. Босов А.\,В.\hfill\hfill\hfill\hfill\hfill\hfill\hfill\hfill\hfill\hfill\hfill\hfill\hfill\hfill\hfill\hfill\hfill\hfill\hfill\hfill\hfill\hfill\hfill\hfill\hfill\hfill\hfill\hfill\hfill\hfill\hfill\hfill\hfill\hfill\hfill}{\ }
\contentsline {section}{\textbf{Ильин В.\,Д., Соколов И.\,А.}\ \ Символьная модель системы знаний информатики в\nobreakspace {}че\-ло\-ве\-ко-автоматной среде}{\qquad 1 \qquad 66} 
\contentsline {section}{\textbf{Калиниченко Л.\,А.}\ \ см. Захаров В.\,Н.\hfill\hfill\hfill\hfill\hfill\hfill\hfill\hfill\hfill\hfill\hfill\hfill\hfill\hfill\hfill\hfill\hfill\hfill\hfill\hfill\hfill\hfill\hfill\hfill\hfill\hfill\hfill\hfill\hfill\hfill\hfill\hfill\hfill\hfill\hfill}{\ }
\contentsline {section}{\textbf{Козеренко Е.\,Б.}\ \ Лингвистическое моделирование для систем машинного перевода и обработки знаний}{\qquad 1 \qquad 54} 
\contentsline {section}{\textbf{Козмидиади В.\,А.}\ \ см. Захаров В.\,Н.\hfill\hfill\hfill\hfill\hfill\hfill\hfill\hfill\hfill\hfill\hfill\hfill\hfill\hfill\hfill\hfill\hfill\hfill\hfill\hfill\hfill\hfill\hfill\hfill\hfill\hfill\hfill\hfill\hfill\hfill\hfill\hfill\hfill\hfill\hfill }{\ } 
\contentsline {section}{\textbf{Королев В.\,Ю.}\ \ см. Батракова Д.\,А.\hfill\hfill\hfill\hfill\hfill\hfill\hfill\hfill\hfill\hfill\hfill\hfill\hfill\hfill\hfill\hfill\hfill\hfill\hfill\hfill\hfill\hfill\hfill\hfill\hfill\hfill\hfill\hfill\hfill\hfill\hfill\hfill\hfill\hfill\hfill}{\ } 
\contentsline {section}{\textbf{Кудрявцев А.\,А., Шоргин С.\,Я.}\ \ Байесовский подход к\nobreakspace {}анализу систем массового обслуживания и\nobreakspace {}показателей надежности}{\qquad 2 \qquad 76}
\contentsline {section}{\textbf{Печинкин А.\,В., Соколов И.\,А., Чаплыгин В.\,В.}\ \ Многолинейная система массового обслуживания с конечным накопителем и ненадежными приборами}{\qquad 1 \qquad 27} 
\contentsline {section}{\textbf{Печинкин А.\,В., Соколов И.\,А., Чаплыгин В.\,В.}\ \ Стационарные характеристики многолинейной\nobreakspace {}системы массового обслуживания с\nobreakspace {}одновременными отказами приборов}{\qquad 2 \qquad 39}
\contentsline {section}{\textbf{Синицын И.\,Н.}\ \ Корреляционные методы построения аналитических информационных моделей флуктуаций полюса Земли по априорным данным}{\qquad 2 \qquad \hphantom{9}2}
\contentsline {section}{\textbf{Синицын И.\,Н.}\ \ Развитие теории фильтров Пугачева для оперативной обработки информации в стохастических системах}{{\qquad 1 \qquad \hphantom{9}3}} 
\contentsline {section}{\textbf{Соколов И.\,А.}\ \ см. Захаров В.\,Н.\hfill\hfill\hfill\hfill\hfill\hfill\hfill\hfill\hfill\hfill\hfill\hfill\hfill\hfill\hfill\hfill\hfill\hfill\hfill\hfill\hfill\hfill\hfill\hfill\hfill\hfill\hfill\hfill\hfill\hfill\hfill\hfill\hfill\hfill\hfill}{\ }
\contentsline {section}{\textbf{Соколов И.\,А.}\ \ см. Ильин В.\,Д.\hfill\hfill\hfill\hfill\hfill\hfill\hfill\hfill\hfill\hfill\hfill\hfill\hfill\hfill\hfill\hfill\hfill\hfill\hfill\hfill\hfill\hfill\hfill\hfill\hfill\hfill\hfill\hfill\hfill\hfill\hfill\hfill\hfill\hfill\hfill}{\ } 
\contentsline {section}{\textbf{Соколов И.\,А.}\ \ см. Печинкин А.\,В.\hfill\hfill\hfill\hfill\hfill\hfill\hfill\hfill\hfill\hfill\hfill\hfill\hfill\hfill\hfill\hfill\hfill\hfill\hfill\hfill\hfill\hfill\hfill\hfill\hfill\hfill\hfill\hfill\hfill\hfill\hfill\hfill\hfill\hfill\hfill}{\ } 
\contentsline {section}{\textbf{Соколов И.\,А.}\ \ см. Печинкин А.\,В.\hfill\hfill\hfill\hfill\hfill\hfill\hfill\hfill\hfill\hfill\hfill\hfill\hfill\hfill\hfill\hfill\hfill\hfill\hfill\hfill\hfill\hfill\hfill\hfill\hfill\hfill\hfill\hfill\hfill\hfill\hfill\hfill\hfill\hfill\hfill}{\ }
\contentsline {section}{\textbf{Ступников С.\,А.}\ \ см. Захаров В.\,Н.\hfill\hfill\hfill\hfill\hfill\hfill\hfill\hfill\hfill\hfill\hfill\hfill\hfill\hfill\hfill\hfill\hfill\hfill\hfill\hfill\hfill\hfill\hfill\hfill\hfill\hfill\hfill\hfill\hfill\hfill\hfill\hfill\hfill\hfill\hfill}{\ }
\contentsline {section}{\textbf{Чаплыгин В.\,В.}\ \ см. Печинкин А.\,В.\hfill\hfill\hfill\hfill\hfill\hfill\hfill\hfill\hfill\hfill\hfill\hfill\hfill\hfill\hfill\hfill\hfill\hfill\hfill\hfill\hfill\hfill\hfill\hfill\hfill\hfill\hfill\hfill\hfill\hfill\hfill\hfill\hfill\hfill\hfill}{\ } 
\contentsline {section}{\textbf{Чаплыгин В.\,В.}\ \ см. Печинкин А.\,В.\hfill\hfill\hfill\hfill\hfill\hfill\hfill\hfill\hfill\hfill\hfill\hfill\hfill\hfill\hfill\hfill\hfill\hfill\hfill\hfill\hfill\hfill\hfill\hfill\hfill\hfill\hfill\hfill\hfill\hfill\hfill\hfill\hfill\hfill\hfill}{\ }
\contentsline {section}{\textbf{Шоргин С.\,Я.}\ \ см. Батракова Д.\,А.\hfill\hfill\hfill\hfill\hfill\hfill\hfill\hfill\hfill\hfill\hfill\hfill\hfill\hfill\hfill\hfill\hfill\hfill\hfill\hfill\hfill\hfill\hfill\hfill\hfill\hfill\hfill\hfill\hfill\hfill\hfill\hfill\hfill\hfill\hfill}{\ } 
\contentsline {section}{\textbf{Шоргин С.\,Я.}\ \ см. Кудрявцев А.\,А.\hfill\hfill\hfill\hfill\hfill\hfill\hfill\hfill\hfill\hfill\hfill\hfill\hfill\hfill\hfill\hfill\hfill\hfill\hfill\hfill\hfill\hfill\hfill\hfill\hfill\hfill\hfill\hfill\hfill\hfill\hfill\hfill\hfill\hfill\hfill}{\ }
%\thispagestyle{myheadings}
\def\leftfootline{\small{\textbf{\thepage}
\hfill ИНФОРМАТИКА И ЕЁ ПРИМЕНЕНИЯ\ \ \ том~1\ \ \ выпуск~2\ \ \ 2007}
}%
 \def\rightfootline{\small{ИНФОРМАТИКА И ЕЁ ПРИМЕНЕНИЯ\ \ \ том~1\ \ \ выпуск~2\ \ \ 2007
 \hfill \textbf{\thepage}}}
 \label{end\stat}

%\def\stat{cont-e}
{%\hrule\par
%\vskip 7pt % 7pt
\raggedleft\Large \bf%\baselineskip=3.2ex
2\,0\,0\,7\ \ A\,U\,T\,H\,O\,R\ \ I\,N\,D\,E\,X \vskip 17pt
    \hrule
    \par
\vskip 21pt plus 6pt minus 3pt }

\label{st\stat}

\def\tit{\ }

\def\aut{\ }
\def\auf{\ }

\def\leftkol{\ } % ENGLISH ABSTRACTS}

\def\rightkol{\ } %ENGLISH ABSTRACTS}

\titele{\tit}{\aut}{\auf}{\leftkol}{\rightkol}


\contentsline {chapter}{\ }{Issue \quad Page} 
\contentsline {subsection}{\textbf{Batrakova D.\,A., Korolev V.\,Yu., Shorgin S.\,Ya.}\ \ A New Method for the Probabilistic and Statistical Analysis of Information Flows in Telecommunication Networks}{\qquad 1 \qquad 40} 
\contentsline {subsection}{\textbf{Borisov A.\,V.}\ \ Bayesian Estimation in\nobreakspace {}Observation Systems with\nobreakspace {}Markov Jump Processes: Game-Theoretic Approach}{\qquad 2 \qquad 65} 
\contentsline {subsection}{\textbf{Bosov A.\,V., Ivanov A.\,V.}\ \ Linguistic Simulation for Machine Translation and Knowledge Management Systems}{\qquad 2 \qquad 50} 
\contentsline {subsection}{\textbf{Chaplygin V.\,V.} see Pechinkin A.\,V.\hfill\hfill\hfill\hfill\hfill\hfill\hfill\hfill\hfill\hfill\hfill\hfill\hfill\hfill\hfill\hfill\hfill\hfill\hfill\hfill\hfill\hfill\hfill\hfill\hfill\hfill\hfill\hfill\hfill\hfill\hfill\hfill\hfill\hfill\hfill}{\ }
\contentsline {subsection}{\textbf{Chaplygin V.\,V.} see Pechinkin A.\,V.\hfill\hfill\hfill\hfill\hfill\hfill\hfill\hfill\hfill\hfill\hfill\hfill\hfill\hfill\hfill\hfill\hfill\hfill\hfill\hfill\hfill\hfill\hfill\hfill\hfill\hfill\hfill\hfill\hfill\hfill\hfill\hfill\hfill\hfill\hfill}{\ }
\contentsline {subsection}{\textbf{Ilyin V.\,D., Sokolov I.\,A.}\ \ The Symbol Model of Informatics Knowledge System in Human-Automaton Environment}{\qquad 1 \qquad 66} 
\contentsline {subsection}{\textbf{Ivanov A.\,V.} see Bosov A.\,V.\hfill\hfill\hfill\hfill\hfill\hfill\hfill\hfill\hfill\hfill\hfill\hfill\hfill\hfill\hfill\hfill\hfill\hfill\hfill\hfill\hfill\hfill\hfill\hfill\hfill\hfill\hfill\hfill\hfill\hfill\hfill\hfill\hfill\hfill\hfill}{\ }
\contentsline {subsection}{\textbf{Kalinichenko L.\,A.} see Zakharov V.\,N.\hfill\hfill\hfill\hfill\hfill\hfill\hfill\hfill\hfill\hfill\hfill\hfill\hfill\hfill\hfill\hfill\hfill\hfill\hfill\hfill\hfill\hfill\hfill\hfill\hfill\hfill\hfill\hfill\hfill\hfill\hfill\hfill\hfill\hfill\hfill}{\ }
\contentsline {subsection}{\textbf{Korolev V.\,Yu.} see Batrakova D.\,A.\hfill\hfill\hfill\hfill\hfill\hfill\hfill\hfill\hfill\hfill\hfill\hfill\hfill\hfill\hfill\hfill\hfill\hfill\hfill\hfill\hfill\hfill\hfill\hfill\hfill\hfill\hfill\hfill\hfill\hfill\hfill\hfill\hfill\hfill\hfill}{\ }
\contentsline {subsection}{\textbf{Kozerenko E.\,B.}\ \ Linguistic Simulation for Machine Translation and Knowledge Management Systems}{\qquad 1 \qquad 54} 
\contentsline {subsection}{\textbf{Kozmidiady V.\,A.} see Zakharov V.\,N.\hfill\hfill\hfill\hfill\hfill\hfill\hfill\hfill\hfill\hfill\hfill\hfill\hfill\hfill\hfill\hfill\hfill\hfill\hfill\hfill\hfill\hfill\hfill\hfill\hfill\hfill\hfill\hfill\hfill\hfill\hfill\hfill\hfill\hfill\hfill}{\ }
\contentsline {subsection}{\textbf{Kudryavtsev A.\,A., Shorgin S.\,Ya.}\ \ Bayesian Approach to Queueing Systems and Reliability Characteristics}{\qquad 2 \qquad 76} 
\contentsline {subsection}{\textbf{Pechinkin A.\,V., Sokolov I.\,A., Chaplygin V.\,V.}\ \ Multichannel Queuing System with Finite Buffer and Unreliable Servers}{\qquad 1 \qquad 27} 
\contentsline {subsection}{\textbf{Pechinkin A.\,V., Sokolov I.\,A., Chaplygin V.\,V.}\ \ Stationary Characteristics of a Multichannel Queueing System with\nobreakspace {}Simultaneous Refusals of Servers}{\qquad 2 \qquad 39} 
\contentsline {subsection}{\textbf{Shorgin S.\,Ya.} see Batrakova D.\,A.\hfill\hfill\hfill\hfill\hfill\hfill\hfill\hfill\hfill\hfill\hfill\hfill\hfill\hfill\hfill\hfill\hfill\hfill\hfill\hfill\hfill\hfill\hfill\hfill\hfill\hfill\hfill\hfill\hfill\hfill\hfill\hfill\hfill\hfill\hfill}{\ }
\contentsline {subsection}{\textbf{Shorgin S.\,Ya.} see Kudryavtsev A.\,A.\hfill\hfill\hfill\hfill\hfill\hfill\hfill\hfill\hfill\hfill\hfill\hfill\hfill\hfill\hfill\hfill\hfill\hfill\hfill\hfill\hfill\hfill\hfill\hfill\hfill\hfill\hfill\hfill\hfill\hfill\hfill\hfill\hfill\hfill\hfill}{\ }
\contentsline {subsection}{\textbf{Sinitsyn I.\,N.}\ \ Correlational Methods for Analytical Informational Models of the Earth Pole Fluctuations Design Based on a priori Data}{\qquad 2 \qquad \hphantom{9}2}
\contentsline {subsection}{\textbf{Sinitsyn I.\,N.}\ \ Development of Pugachev Filtering for Stochastic Systems}{\qquad 1 \qquad \hphantom{9}3}
\contentsline {subsection}{\textbf{Sokolov I.\,A.} see Ilyin V.\,D.\hfill\hfill\hfill\hfill\hfill\hfill\hfill\hfill\hfill\hfill\hfill\hfill\hfill\hfill\hfill\hfill\hfill\hfill\hfill\hfill\hfill\hfill\hfill\hfill\hfill\hfill\hfill\hfill\hfill\hfill\hfill\hfill\hfill\hfill\hfill}{\ }
\contentsline {subsection}{\textbf{Sokolov I.\,A.} see Pechinkin A.\,V.\hfill\hfill\hfill\hfill\hfill\hfill\hfill\hfill\hfill\hfill\hfill\hfill\hfill\hfill\hfill\hfill\hfill\hfill\hfill\hfill\hfill\hfill\hfill\hfill\hfill\hfill\hfill\hfill\hfill\hfill\hfill\hfill\hfill\hfill\hfill}{\ }
\contentsline {subsection}{\textbf{Sokolov I.\,A.} see Pechinkin A.\,V.\hfill\hfill\hfill\hfill\hfill\hfill\hfill\hfill\hfill\hfill\hfill\hfill\hfill\hfill\hfill\hfill\hfill\hfill\hfill\hfill\hfill\hfill\hfill\hfill\hfill\hfill\hfill\hfill\hfill\hfill\hfill\hfill\hfill\hfill\hfill}{\ }
\contentsline {subsection}{\textbf{Sokolov I.\,A.} see Zakharov V.\,N.\hfill\hfill\hfill\hfill\hfill\hfill\hfill\hfill\hfill\hfill\hfill\hfill\hfill\hfill\hfill\hfill\hfill\hfill\hfill\hfill\hfill\hfill\hfill\hfill\hfill\hfill\hfill\hfill\hfill\hfill\hfill\hfill\hfill\hfill\hfill}{\ }
\contentsline {subsection}{\textbf{Stupnikov S.\,A.} see Zakharov V.\,N.\hfill\hfill\hfill\hfill\hfill\hfill\hfill\hfill\hfill\hfill\hfill\hfill\hfill\hfill\hfill\hfill\hfill\hfill\hfill\hfill\hfill\hfill\hfill\hfill\hfill\hfill\hfill\hfill\hfill\hfill\hfill\hfill\hfill\hfill\hfill}{\ }
\contentsline {subsection}{\textbf{Zakharov V.\,N., Kalinichenko L.\,A., Sokolov I.\,A., Stupnikov S.\,A.}\ \ Development of Canonical Information Models for Integrated Information Systems}{\qquad 2 \qquad 15} 
\contentsline {subsection}{\textbf{Zakharov V.\,N., Kozmidiady V.\,A.}\ \ Means Providing Applications Fault Tolerance}{\qquad 1 \qquad 14} 
\def\leftfootline{\small{\textbf{\thepage}
\hfill ИНФОРМАТИКА И ЕЁ ПРИМЕНЕНИЯ\ \ \ том~1\ \ \ выпуск~2\ \ \ 2007}
}%
 \def\rightfootline{\small{ИНФОРМАТИКА И ЕЁ ПРИМЕНЕНИЯ\ \ \ том~1\ \ \ выпуск~2\ \ \ 2007
 \hfill \textbf{\thepage}}}
 \label{end\stat}


%\tableofcontents


\end{document}