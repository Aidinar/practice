\documentclass[10pt]{book}
\usepackage[utf8]{inputenc}

\usepackage{latexsym,amssymb,amsfonts,amsmath,indentfirst,shapepar,%fleqn,%
picinpar,shadow,floatflt,enumerate,multicol,colortbl,ipi}

\usepackage{rotating}
\input{epsf}

%\nofiles

%\includeonly{avtor,avtor-eng}
%\includeonly{avtor-eng}
%\includeonly{pred}  %+

%\includeonly{konovalov}  %+pdf
%\includeonly{markin}     %+pdf
%\includeonly{kruchin}    %+
%\includeonly{krivenko}   %+pdf
%\includeonly{morozov}    %+pdf
%\includeonly{stepanov}   %+pdf
%\includeonly{torhin-rud} %+pdf
%\includeonly{shevts}     %+pdf
%\includeonly{bening}     %+pdf
%\includeonly{kozerenko}  %+pdf


%\includeonly{toc-rus,toc-en}
%\includeonly{toc-en}


%\includeonly{obchak}
%\includeonly{reshal}
%\includeonly{eng-index}
%\includeonly{cover3}

\usepackage{acad}
\usepackage{courier}
\usepackage{decor}
\usepackage{newton}
\usepackage{pragmatica}
\usepackage{zapfchan}
\usepackage{petrotex}
\usepackage{bm}                     % полужирные греческие буквы
\usepackage{upgreek}                % прямые греческие буквы
%\usepackage{verbatim}

\renewcommand{\bottomfraction}{0.99}
\renewcommand{\topfraction}{0.99}
\renewcommand{\textfraction}{0.01}

\setcounter{secnumdepth}{1} %здесь - 3 + chapter = 4

\arraycolsep=1.5pt

%\usepackage[pdftex]{graphicx}

%\usepackage{oz}

%NEW COMMANDS



\renewcommand{\r}{{\rm I\hspace{-0.7mm}\rm R}}
\newcommand{\I}{{\rm I\hspace{-0.7mm}I}}
\newcommand{\Ik}{\mbox{{\small \tt {1}}\hspace{-1.5mm}{\tt 1}}}
%\newcommand{\Ikl}{{\small \tt{1}}\hspace*{-0.4mm}\mathtt{1}}

%\mathrm{I}\hspace*{-0.7mm}\mathrm{R}

\newcommand{\il}[2]{\int\limits_{#1}^{#2}}%интеграл с пределами #1 и #2

\newcommand{\h}{{\bf H}}
\newcommand{\p}{{\sf P}}  % вероятность
\newcommand{\e}{{\sf E}}  % мат. ожидание
\newcommand{\D}{{\sf D}}  % дисперсия
\newcommand{\eps}{\varepsilon}
\newcommand{\vp}{\mathrm{v.p.}}
\newcommand{\F}{{\mathcal F}}
%\def\iint{\int\limits_{-\infty}^{\infty}}

%\newcommand{\gr}{{\geqslant}}

\newcommand{\g}{\mbox{\textit{g}}}

%\renewcommand{\la}{\lambda}
\newcommand{\si}{\sigma}
%\renewcommand{\a}{\alpha}

%\newcommand{\pto}{\stackrel{P}{\longrightarrow}} % сходимость по веpоятности

%\newcommand{\eqd}{\stackrel{d}{=}} % равенство по pаспpеделению

%\newcommand{\kp}{\kappa}
%\def\Q{{\cal Q}} \def\H{{\cal H}}
%\newcommand{\bet}{\beta_{2+\delta}}


%\newtheorem{definition}{Определение}
%\renewcommand{\thedefinition}{\arabic{definition}.}
%END NEW COMMANDS

%\renewcommand{\baselinestretch}{1.2}

%\pagestyle{myheadings}

\setlength{\textwidth}{167mm}      % 122mm
\setlength{\textheight}{658pt}
%\setlength{\textheight}{635.6pt}
\setlength{\columnsep}{4.5mm}

\setcounter{secnumdepth}{4}

%\addtolength{\headheight}{2pt}
%\addtolength{\headsep}{-2mm}

%\addtolength{\topmargin}{-20mm}  % for printing


\hoffset=-30mm  % From Yap
%\hoffset=-20mm  % From Acrobat

%\voffset=0mm % From Yap
%\voffset=-15mm   % From Acrobat

\addtolength{\evensidemargin}{-9.5mm} % for printing
\addtolength{\oddsidemargin}{9.5mm}  % for printing

%\renewcommand{\thefootnote}{\fnsymbol{footnote}}
%\renewcommand{\thefootnote}{\arabic{footnote}}
\renewcommand{\figurename}{\protect\bf Рис.}
\renewcommand{\tablename}{\protect\bf Таблица}

\newcommand{\Caption}[1]{\caption{\protect\small %\baselineskip=2.5ex
#1}}

\renewcommand{\thefigure}{\arabic{figure}}
\renewcommand{\thetable}{\arabic{table}}
\renewcommand{\theequation}{\arabic{equation}}
\renewcommand{\thesection}{\arabic{section}}

\renewcommand{\contentsname}{СОДЕРЖАНИЕ}
\newcommand{\fr}[2]{\displaystyle\frac{\displaystyle #1\mathstrut}{\displaystyle #2\mathstrut}}

%\renewcommand{\thefootnote}{\fnsymbol{footnote}}
%\newcommand{\g}{\mbox{\textit{g}}}

%\newcommand{\Caption}[1]{\caption{\protect\small\baselineskip=2ex #1}}
\newcounter{razdel}
\setcounter{razdel}{0}


\newcommand{\titel}[4]{%
\

\vspace*{5pt}

\ifodd\therazdel {\raggedright\noindent\Large\textrm\textbf
 \lineskip .75em
  \baselineskip=3.2ex #1 \par}
\vskip 1em {\noindent\large\textrm\textbf #2 \par}
\addcontentsline{toc}{subsection}{{\textrm\textbf #3}\protect\newline #1}
\def\rightheadline{\underline{\noindent\hbox to \textwidth{\hfill\small\textrm{#4}
%\hfill \large\bf\thepage
}}}
\def\leftheadline{\underline{\noindent\parbox{\textwidth}{
%\raggedleft\large\bf\thepage \hfill
\small\textit{#3}\hfill}}}
\def\leftfootline{\small{\textbf{\thepage}
\hfill ИНФОРМАТИКА И ЕЁ ПРИМЕНЕНИЯ\ \ \ том~4\ \ \ выпуск 2\ \ \ 2010}
}%
 \def\rightfootline{\small{ИНФОРМАТИКА И ЕЁ ПРИМЕНЕНИЯ\ \ \ том~4\ \ \ выпуск~2\ \ \ 2010
\hfill \textbf{\thepage}}} \vskip 2em \setcounter{figure}{0}
\setcounter{table}{0} \setcounter{equation}{0} \setcounter{section}{0}
\setcounter{subsection}{0} \setcounter{subsubsection}{0}
\setcounter{footnote}{0} \setcounter{razdel}{0}
%\end{flushleft}
\else {
 \raggedright\noindent\Large\textrm\textbf
 \lineskip .75em
\baselineskip=3.2ex #1 \par} \vskip 1em
%\begin{flushleft}
{\noindent\large\textrm\textbf #2 \par}
\addcontentsline{toc}{subsection}{{\textrm\textbf #3}\protect\newline #1}
\def\rightheadline{\underline{\noindent\hbox to \textwidth{\hfill\small\textrm{#4}
%\hfill \large\bf\thepage
}}}
\def\leftheadline{\underline{\noindent\parbox{\textwidth}{%\raggedleft\large\bf\thepage \hfill
\small\textit{#3}\hfill}}}
\def\leftfootline{\small{\textbf{\thepage}
\hfill ИНФОРМАТИКА И ЕЁ ПРИМЕНЕНИЯ\ \ \ том~4\ \ \ выпуск~2\ \ \ 2010}
}%
 \def\rightfootline{\small{ИНФОРМАТИКА И ЕЁ ПРИМЕНЕНИЯ\ \ \ том~4\ \ \ выпуск~2\ \ \ 2010
\hfill \textbf{\thepage}}} \vskip 2em \setcounter{figure}{0}
\setcounter{table}{0} \setcounter{equation}{0} \setcounter{section}{0}
\setcounter{subsection}{0} \setcounter{subsubsection}{0}
\setcounter{footnote}{0}
%\end{flushleft}
\fi}

\newcommand{\titelr}[2]{%
\

\vspace*{5pt}

\ifodd\therazdel {\raggedright\noindent\large\textrm\textbf
 \lineskip .75em
  \baselineskip=3.2ex #1 \par}
\vskip 1em {\noindent\normalsize\textrm\textbf #2 \par}
\else {
 \raggedright\noindent\large\textrm\textbf
 \lineskip .75em
\baselineskip=3.2ex #1 \par} \vskip 1em
%\begin{flushleft}
{\noindent\normalsize\textrm\textbf #2 \par}
\fi}

\newcommand{\titele}[5]{%
\

%\vspace*{5pt}

\ifodd\therazdel {\raggedright\noindent%\large
\textrm\textbf
 \lineskip .75em
%  \baselineskip=3.2ex
#1 \par}
\vskip .5em {\noindent\large\textrm\textbf #2 \par}
\vskip .5em
 {\noindent\textrm #3 \par}
\addcontentsline{toc}{subsection}{{\textrm\textbf #1}\protect\newline #2}
\def\rightheadline{\underline{\noindent\hbox to \textwidth{\hfill\small\textrm{#4}
%\hfill \large\bf\thepage
}}}
\def\leftheadline{\underline{\noindent\parbox{\textwidth}{
%\raggedleft\large\bf\thepage \hfill
\small\textrm{#5}\hfill}}}
\def\leftfootline{\small{\textbf{\thepage}
\hfill ИНФОРМАТИКА И ЕЁ ПРИМЕНЕНИЯ\ \ \ том~4\ \ \ выпуск~2\ \ \ 2010}
}%
 \def\rightfootline{\small{ИНФОРМАТИКА И ЕЁ ПРИМЕНЕНИЯ\ \ \ том~4\ \ \ выпуск~2\ \ \ 2010
\hfill \textbf{\thepage}}} \vskip 1em \setcounter{figure}{0}
\setcounter{table}{0} \setcounter{equation}{0} \setcounter{section}{0}
\setcounter{subsection}{0} \setcounter{subsubsection}{0}
\setcounter{footnote}{0} \setcounter{razdel}{0}
%\end{flushleft}
\else {
 \raggedright\noindent%\large
 \textrm\textbf
 \lineskip .75em
%\baselineskip=3.2ex
#1 \par} \vskip .5em
%\begin{flushleft}
{\noindent\large\textrm\textbf #2 \par} \vskip .5em
 {\noindent\textrm #3 \par}
\addcontentsline{toc}{subsection}{{\textrm\textbf #1}\protect\newline #2}
\def\rightheadline{\underline{\noindent\hbox to \textwidth{\hfill\small\textrm{#4}
%\hfill \large\bf\thepage
}}}
\def\leftheadline{\underline{\noindent\parbox{\textwidth}{%\raggedleft\large\bf\thepage \hfill
\small\textrm{#5}\hfill}}}
\def\leftfootline{\small{\textbf{\thepage}
\hfill ИНФОРМАТИКА И ЕЁ ПРИМЕНЕНИЯ\ \ \ том~4\ \ \ выпуск~2\ \ \ 2010}
}%
 \def\rightfootline{\small{ИНФОРМАТИКА И ЕЁ ПРИМЕНЕНИЯ\ \ \ том~4\ \ \ выпуск~2\ \ \ 2010
\hfill \textbf{\thepage}}} \vskip 1em \setcounter{figure}{0}
\setcounter{table}{0} \setcounter{equation}{0} \setcounter{section}{0}
\setcounter{subsection}{0} \setcounter{subsubsection}{0}
\setcounter{footnote}{0}
%\end{flushleft}
\fi}

\def\Abst#1{
\begin{center}\small\nwt
\parbox{150mm}{%\baselineskip=2.5ex
\textbf{Аннотация:}\ \
%\hspace*{\parindent}
#1}
\end{center}}
\def\Abste#1{
\begin{center}\small\nwt
\parbox{150mm}{%\baselineskip=2.5ex
\textbf{Abstract:}\ \
%\hspace*{\parindent}
#1}
\end{center}}

\def\KW#1{
\begin{center}\small\nwt
\parbox{150mm}{%\baselineskip=2.5ex
\textbf{Ключевые слова:}\ \ #1}
\end{center}}

\def\KWE#1{
\begin{center}\small\nwt
\parbox{150mm}{%\baselineskip=2.5ex
\textbf{Keywords:}\ \ #1}
\end{center}}


\def\KWN#1{
%\begin{center}
%\small
%\parbox{150mm}\end{center}
}

\renewcommand{\thesubsection}{\thesection.\arabic{subsection}\hspace*{-5pt}}
\renewcommand{\thesubsubsection}{\thesubsection\hspace*{5pt}.\arabic{subsubsection}\hspace*{-3pt}}

\begin{document}
\Rus

\nwt
%\ptb

%\renewcommand{\contentsname}{\protect\Large\bf Содержание}

\setcounter{tocdepth}{2}

%\tableofcontents

\renewcommand{\bibname}{\protect\rmfamily Литература}
  \def\Au#1{{\it #1}}

%\newcommand{\No}{№}
  \newcommand{\tg}{\,\mathrm{tg}\,}
    \newcommand{\ctg}{\,\mathrm{ctg}\,}
  \newcommand{\arctg}{\,\mathrm{arctg}\,}
  
\def\forallb{\mathop{\forall}}
\def\existsb{\mathop{\exists}}

\setcounter{page}{1}

\newpage
\addtocounter{razdel}{1}
%\def\razd{РЕГУЛИРУЕМЫЙ ЭЛЕКТРОПРИВОД ДЛЯ ЭЛЕКТРОЭНЕРГЕТИКИ}
%\newpage
%\include{zakh-old}

%\def\stat{batr}

\def\tit{НОВЫЙ МЕТОД ВЕРОЯТНОСТНО-СТАТИСТИЧЕСКОГО\newline
АНАЛИЗА ИНФОРМАЦИОННЫХ ПОТОКОВ
В~ТЕЛЕКОММУНИКАЦИОННЫХ СЕТЯХ$^*$}
\def\titkol{Новый метод вероятностно-статистического
анализа информационных потоков
в~телекоммуникационных сетях}
\def\autkol{Д.\,А.~Батракова, В.\,Ю.~Королев, С.\,Я.~Шоргин}
\def\aut{Д.\,А.~Батракова$^1$, В.\,Ю.~Королев$^2$, С.\,Я.~Шоргин$^3$}

\titel{\tit}{\aut}{\autkol}{\titkol}

{\renewcommand{\thefootnote}{\fnsymbol{footnote}}\footnotetext[1]{Работа 
выполнена при поддержке РФФИ, проекты №№\,04-01-00671, 05-07-90103.} 
\renewcommand{\thefootnote}{\arabic{footnote}}}
 \footnotetext[1]{ИПИ РАН, 
daria.batrakova@gmail.com} \footnotetext[2]{Факультет вычислительной математики 
и кибернетики МГУ им.~М.\,В.~Ломоносова, ИПИ РАН, vkorolev@comtv.ru} 
\footnotetext[3]{ИПИ РАН, sshorgin@ipiran.ru}



\Abst{В данной работе предлагается метод исследования стохастической структуры
хаотических информационных потоков в сложных телекоммуникационных
сетях. Предлагаемый метод основан на стохастической модели
телекоммуникационной сети, в рамках которой она представляется в виде
суперпозиции некоторых простых последовательно-параллельных структур.
Эта модель естественно порождает смеси гамма-распределений для времени
выполнения (обработки) запроса сетью. Параметры получаемой смеси
гамма-распределений характеризуют стохастическую структуру
информационных потоков в сети. Для решения задачи статистического
оценивания параметров смесей экспоненциальных и гамма-распределений
(задачи разделения смесей) используется ЕМ-алгоритм. Чтобы проследить
изменение стохастической структуры информационных потоков во времени,
ЕМ-алгоритм применяется в режиме скользящего окна. Описывается
программный инструментарий для применения полученного решения к
реальным статистическим данным. Приводится интерпретация результатов.}

\KW{телекоммуникационные сети; информационные потоки;
разделение смесей  распределений;
метод скользящего окна;  программа для разделения смесей}

\vskip 24pt plus 9pt minus 6pt

\thispagestyle{headings}

\begin{multicols}{2}


\label{st\stat}

\section{Введение}

Развитие телекоммуникационных сетей, их усложнение поставило перед
инженерами важную прикладную задачу исследования характеристик
информационных потоков, возникающих в этих сетях. Здесь под
информационным потоком мы будем понимать упорядоченное движение
любого вида информации по сети.

Если на заре эры телекоммуникаций, в эпоху первых телефонных линий и
телеграфа эта проблема не была столь насущной, то со временем, при
постепенном охвате мирового пространства сетями возникла необходимость в
построении и исследовании адекватных моделей сетей и процессов,
происходящих в них.

\thispagestyle{headings}


Современные сети~--- это \textit{конвергентные} сети, т.е.\ совокупность крайне
разнородных как по топологии, так и по физической архитектуре сетей, которые
предлагают конечному пользователю самые разнообразные сервисы. Это~--- огромное
виртуальное и физическое пространство, состоящее из миллионов процессоров,
операционных платформ, линий передачи данных и стыковочных узлов.
%
Существует множество классификаций телекоммуникационных сетей по различным
признакам:
\begin{itemize}
\item масштабу (локальные сети~--- LAN, масштаба города~---
MAN, широкого масштаба~--- WAN);
\item топологии, или логической организации (<<звезда>>,
<<кольцо>>, <<шина>>);
\item физической организации (оптические сети, радио);
\item предлагаемым услугам (сотовые сети, для доступа в
Интернет);
\item назначению (военные, гражданские) и~др.
\end{itemize}


Конвергентная сеть входит во все эти классы, причем, как правило,
обладает всеми этими признаками. Поэтому построение модели для ее анализа
является и очень важной, и очень сложной задачей.

Существуют достаточно многочисленные математические методы, ориентированные на
моделирование и анализ телекоммуникационных сетей. В~большинстве своем они
основываются на теории массового обслуживания, то есть разделе теории
вероятностей, посвященном описанию функционирования сложных систем обслуживания
(в том чис\-ле телекоммуникационных сетей и систем) с помощью стохастических
процессов особого вида и анализу таких процессов. Указанная теория является
весьма развитой и широко применяется на практике. Тем не менее, ее применимость
ограничена~--- во-первых, все возрастающей сложностью структур и дисциплин
обслуживания в рас\-смат\-ри\-ва\-емых реальных сетях. Эта сложность во многих
случаях принципиально не может найти адекватного отображения в моделях
массового обслуживания, даже несмотря на постоянно растущую сложность самих
этих моделей. В результате даже модели, допускающие точный математический
анализ, дают возможность расчета всего лишь приближенных значений характеристик
реальных сетей, ибо предположения, принимаемые при построении моделей, во
многих случаях не соответствуют практике. Во-вторых, для описания
телекоммуникационной сети в виде сети массового обслуживания исследователь
должен располагать детальным описанием структуры сети, что далеко не всегда
имеет мес\-то на практике. В-третьих, разработано крайне мало моделей массового
обслуживания, в которых используется в качестве входной информация о
наблюдаемых (статистических) показателях функционирования сети; в то же время,
такая информация очень часто доступна исследователю, и ее использование при
анализе сети весьма целесообразно.

В данной работе предлагается в определенной степени альтернативный подход к
моделированию сложных телекоммуникационных сетей. Строится и исследуется
вероятностная модель сложной телекоммуникационной сети как суперпозиции
достаточно простых структур. При этом практически никакая априорная информация
о структуре исследуемой сети не используется~--- наоборот, в результате
исследования модели исследователь получает приближенное представление об этой
структуре. Характеристики типовых простых структур, составляющих в совокупности
модель сети, и сети в целом при этом подходе описываются
гам\-ма-рас\-пре\-де\-ле\-ни\-я\-ми. Ставится задача оценки параметров модели
на основе статистических данных о функционировании сети, а также предлагается
математическое решение этой задачи. В статье описан также созданный на основе
разработанных математических методов программный инструментарий и приведены
результаты расчетов для реального трафика. {\looseness=-1

}

\section{Математическая модель и~постановка задачи}

\subsection{Логическая модель сети}
 %1.1

Рассмотрим абстрактную \textit{конвергентную телекоммуникационную
сеть}. Это может быть как крупномасштабная транспортная сеть (WAN), сеть
отдельного оператора масштаба города (MAN) с различными сервисами, так и
локальная сеть (LAN).

Любой из этих случаев можно описать как ($p,\,q$)-\textit{сеть}.

\medskip
\textbf{Определение 1.} В теории графов и сетей под ($p,\,q)$-сетью понимается
набор вида $S =$\linebreak $=(G,\,V^\prime ,\,V^{\prime\prime})$, где $G$~---
граф, а $V^\prime$ и $V^{\prime\prime}$~--- выборки из множества $V(G)$ (вершин
графа) длины~$p$ и $q$ соответственно. При этом выборка $V^\prime$
($V^{\prime\prime}$) считается \textit{входной} (\textit{выходной}) выборкой, а
ее $i$-я вершина называется $i$-\textit{м} \textit{входным} (\textit{выходным})
\textit{полюсом} или, иначе, $i$-\textit{м} \textit{входом} (\textit{выходом})
сети~$S$. Вершины, не участвующие во входной и выходной выборках сети,
считаются ее внутренними вершинами~\cite{1bat}.

Сеть $S$ (рис.~\ref{f1bat}) имеет $p$ точек входа~--- точек соединения
с внешней средой (это могут быть точки стыковки разнородных сетей, сетей
различных операторов, физические подключения к интерфейсам
маршрутизаторов и~т.п.). Под \textit{внешней средой} будем понимать другие
сети, которые передают данные в сеть~$S$. Отдельные <<единицы>> данных
(кадры, сообщения, датаграммы, пакеты) поступают на входы сети~$S$,
обрабатываются и подаются на каждый из $q$ выходов, которые могут быть
соединены как с конечными пользователями, так и с другими сетями.
\begin{figure*} %fig1
\vspace*{1pt}
\begin{center}
\mbox{%
\epsfxsize=139.7mm \epsfbox{bat-1.eps}
%\epsfxsize=139.698mm
%\epsfbox{bek-3.eps}
}
\end{center}
\vspace*{-9pt} \Caption{Абстрактная телекоммуникационная сеть \label{f1bat}}
\end{figure*}

Следует отметить, что структура сложных телекоммуникационных сетей обладает
свойством некоторого самоподобия, т.е.\ на каком бы уровне сетевой архитектуры
мы ни рассматривали поведение информационных потоков, мы можем выделить
некоторые базовые структуры, субпотоки, суперпозицией которых мы можем получить
модель конкретной сети, какой бы уровень <<детализации>> сегментов сети мы ни
взяли. Так, например, физические подключения к интерфейсам оптического
кросс-коннекта в этом смысле подобны <<виртуальным>> подключениям к портам TCP
на сервере приложений.

Итак, независимо от уровня сетевой архитектуры мы можем
рассматривать некоторую величину, характеризующую количество каких-либо
ресурсов сети~$S$, занимаемых в процессе передачи и обработки данных.  Это
могут быть ресурсы, относящиеся как к <<объему>> (памяти сетевого
устройства, количеству занятых линий, размеру пакета), так и ко <<времени>>
(времени обслуживания заявки, времени простоя). Эта величина случайна, т.к.\
мы не можем абсолютно точно сказать для сложной телекоммуникационной
сети, какое сообщение на какой из входов поступит и какого размера оно будет.
Таким образом, случайный характер данной величины определяется
случайностью поведения внешней среды.

Пусть $R$~--- это описанная выше случайная величина, $R>0$. Далее, не
ограничивая общности, будем подразумевать под ней время, необходимое для
какой-либо операции сети (обработки <<единицы>> данных), предполагая, что
время обработки прямо зависит от объема сообщения.

\subsection{Вероятностная модель сети} %1.2.

Даже не зная реальной топологии сети, мы можем описать
функционирование некоторых ее участков как процесс выполнения операций
(задач сети) в последовательном  порядке (например, если доступен только
один канал данных) или как процесс одновременного выполнения субопераций
(когда доступно более одного пути выполнения). Это значит, что мы можем
представить функционирование сложной телекоммуникационной сети как
\textit{суперпозицию} таких <<последовательных>> и <<параллельных>>
блоков.

Для построения вероятностной модели распределения~$R$ используется
комбинация асимптотического подхода, основанного на предельных теоремах
теории вероятностей, и принципа максимальной неопределенности (энтропии).

Рассмотрим следующую модель. Предположим, что мы можем разделить
сеть~$S$ на несколько сегментов $S_i$. Пусть $T$~--- случайная величина,
время выполнения операции в отдельно взятом блоке $S_i$ (сегменте сети).

Если операции выполняются \textit{параллельно}, то время, необходимое
для их выполнения~--- это максимальное время, затрачиваемое на какую-либо
субоперацию:
$$
T = \underset{i}{\max}\, T_i\,,
$$
где $T_i$~--- случайные величины для со\-от\-вет\-ст\-ву\-ющих субопераций.
Модель такого выполнения пред\-став\-ле\-на на рис.~\ref{f2bat}.

\begin{figure*} %fig2
\vspace*{1pt}
\begin{center}
\mbox{%
\epsfxsize=117.271mm
\epsfbox{bat-2.eps}
}
\end{center}
\vspace*{-9pt}
\Caption{Параллельное выполнение
\label{f2bat}}
\end{figure*}

Известно, что предельное распределение экстремальных значений для
выборок ~--- это экспоненциальное распределение с плотностью~\cite{2bat}
$$
f(x) =
\begin{cases}
\lambda e^{-\lambda x}\,, & x>0\,,\\
0\,, & x\leq 0\,,
\end{cases}
$$
где $\lambda >0$~--- параметр масштаба.

 Учитывая также энтропийный подход, естественно будет считать
распределение $T$ экспоненциальным, т.к.\ экспоненциальное распределение
обладает наибольшей энтропией среди всех распределений с $x>0$.

Если же операции сети выполняются \textit{последовательно}, то величина
$T$~--- это сумма времен $T_i$, необходимых для выполнения каждой
субоперации:
$$
T = \sum\limits_i T_i\,,
$$
где $T_i$~--- случайные величины для со\-от\-вет\-ст\-ву\-ющих субопераций.
%
Такая модель представлена на рис.~\ref{f3bat}.

\begin{figure*} %fig3
\vspace*{1pt}
\begin{center}
\mbox{%
\epsfxsize=139.592mm
\epsfbox{bat-3.eps}
}
\end{center}
\vspace*{-9pt}
\Caption{Последовательное  выполнение
\label{f3bat}}
\end{figure*}

Это значит, что распределение общей длительности $T$ выполнения
блока~--- это свертка распределений <<элементарных>> времен $T_i$
(экспоненциально распределенных).

Известно, что результатом свертки экспоненциальных распределений
является гамма-распределение, определяемое плотностью
$$
\g_{\lambda , \alpha} (x) =
\begin{cases}
\fr{\lambda_0^{\alpha_0}}{\Gamma (\alpha_0 )}\,x^{\alpha_0-1}
e^{\lambda_0 x}\,, & x>0\,,\\
0,\, & x\leq 0\,,
\end{cases}
$$
где $\alpha >0$~--- параметр формы,  $\lambda >0$  параметр масштаба, а
$\Gamma (z)$~--- гамма-функция Эйлера:
$$
\Gamma (z) = \int\limits_0^\infty x^{z-1} e^{-x}\,dx\,.
$$

\begin{figure*} %fig4
\vspace*{1pt}
\begin{center}
\mbox{%
\epsfxsize=120.831mm
\epsfbox{bat-4.eps}
}
\end{center}
\vspace*{-9pt}
\Caption{Модель пути  обработки сообщения сетью~$S$
\label{f4bat}}
\end{figure*}

Известно~\cite{2bat}, что класс гамма-распределений замкнут над операцией
свертки, поэтому ре\-зуль\-ти\-ру\-ющее распределение случайной величины~$R$
будет также гамма-распределением
$$
\g_{\lambda , \alpha} (x) =
\begin{cases}
\fr{\lambda^{\alpha}}{\Gamma (\alpha )}\,x^{\alpha -1} e^{-\lambda x}\,, &
x>0\,,\\
0,\, & x\leq 0\,.
\end{cases}
$$

В силу случайного характера ввода данных в сеть~$S$ из внешней среды маршрут
передачи данных становится случайным, что представлено на рис.~\ref{f4bat}. Это
означает, что параметры ре\-зуль\-ти\-ру\-юще\-го распределения~$R$ тоже
случайны. Отсюда имеем следующую модель \textit{смеси
гам\-ма-рас\-пре\-де\-ле\-ний}, определяемой плотностью

\begin{equation} %1
p(x) = \iint \g_{\lambda , \alpha}(x)\,dH (\lambda ,\,\alpha )\,,
\end{equation}
где $H(\lambda , \alpha )$~--- смешивающая функция, функция распределения
входных параметров.

Поясним понятие \textit{смеси распределений}.

\medskip
\textbf{Определение~2.} Пусть имеется двух\-па\-ра\-мет\-ри\-че\-ское
семейство $n$-мерных плотностей  распределения
\begin{equation}
F = \{ f_\omega (x;\, \theta (\omega ))\}\,,
\end{equation}
где одномерный (целочисленный или непрерывный) параметр $\omega$ в
качестве нижнего индекса функции $f$ определяет специфику общего вида
каж\-до\-го компонента~--- распределения смеси, а в качестве аргумента при
многомерном, вообще говоря, параметре $\theta$ определяет зависимость
значений хотя бы части компонентов этого параметра от того, в каком именно
составляющем распределении $f_\omega$ он присутствует. Кроме того, пусть
$P = \{P(\omega )\}$~--- \textit{семейство смешивающих функций}
распределения.

Функция плотности распределения
\begin{equation}
f(x) = \int f_\omega (x;\,\theta(\omega ))\,dP (\omega )
\end{equation}
называется $P$-\textit{смесью} (или просто \textit{смесью})
\textit{распределений} семейства~$F$,  интеграл в~(3) понимается в смысле
Лебега--Стильтьеса~\cite{3bat}.

\medskip
\textbf{Определение 3.} Семейство смесей~(3) называется
\textit{идентифицируемым} (\textit{различимым}), если из равенства
$$
\int f_\omega (x;\,\theta(\omega ))\,dP (\omega ) =\int f_\omega
(x,\,\theta(\omega )) dP^*(\omega )
$$
следует, что $P(\omega ) \equiv P^*(\omega )$ для всех $P \in P(\omega
)$~\cite{3bat}.

\subsection{Постановка задачи} %1.3.

Перед нами встает задача \textit{разделения} такой смеси. Вообще говоря,
задача разделения смесей распределений со смешивающими функциями
общего вида является \textit{некорректно поставленной}, т.к.\ она допускает
существование нескольких решений. Поэтому будем искать решение в классе
\textit{конечных идентифицируемых смесей распределений}, где смешивающая
функция дискретна.

Для этого сузим данное выше определение и будем рассматривать в дальнейшем лишь 
случай конечного числа $k$ возможных значений па\-ра\-мет\-ра~$\omega$, что 
соответствует конечному числу скачков смешивающих функций $P(\omega )$.  
Величины этих скачков как раз и будут играть роль \textit{удельных весов} 
(\textit{априорных вероятностей}) $p_j$ компонентов смеси. Более того, в нашем 
случае мы постулируем также однотипность компонентов распределений $f_j$, т.е.\ 
принадлежность всех $f_j$ к одному общему па\-ра\-мет\-ри\-че\-ско\-му 
семейству $\{ f(X;\,\theta )\}$, где $\theta$~--- многомерный, вообще говоря, 
параметр. Так что~(3) в этом случае может быть записано в виде
\begin{equation} %4
p(x) = \sum\limits^k_{j=1} p_j f_j (x;\,\theta_j )\,.
\end{equation}

Переформулируем понятие идентифицируемости (различимости) смесей
специально применительно к такому виду смесей.

\medskip
\textbf{Определение 4.} \textit{Конечная смесь}~(3) называется
\textit{идентифицируемой} (\textit{различимой}), если из равенства
$$
\sum\limits_{j=1}^k p_j f_j (x;\,\theta_j ) = \sum\limits_{l=1}^{k^*} p_l^* f_l
(x;\,\theta_l^* )
$$
следует, что $k=k^*$ и для любого $j$ ($1\leq j \leq k$) найдется такое $l$ 
($1\leq l \leq k^*$), что $p_j = p_l^*$ и $f_j (x;\,\theta_j ) = f_l 
(x;\,\theta_l^* )$~\cite{3bat}.

Решить эту задачу в выборочном варианте~--- значит по выборке
классифицируемых наблюдений
$X_1,\ldots , X_n, $ извлеченной из генеральной совокупности, яв\-ля\-ющей\-ся смесью~(3)
генеральных совокупностей типа~(2) (при заданном общем виде составляющих
смесь функций $f_j (x;\,\theta_j )$), построить статистические оценки для числа
компонентов смеси~$k$, их удельных весов $p_j$ и, главное, для каждого из
компонентов %f_j (x;\,\theta_j )$ анализируемой смеси. Далее будет считать, что
функции $f_j$ однозначно определяются своими параметрами $\theta_j$: $f_j
=f(x;\,\theta_j)$.

Однако не следует ставить знак тождества между задачей разделения смеси
и задачей статистического оценивания параметров в модели~(4) по выборке $
X_1,\ldots , X_n$, поскольку задача разделения сохраняет смысл и
применительно к генеральным совокупностям, т.е.\ в теоретическом
варианте~\cite{3bat}.

Итак, для статистического анализа на основе реальных данных мы
аппроксимируем нашу общую модель~(1) следующей:
$$
p(x) \approx \hat{p}(x) = \sum\limits_{j=1}^k p_j \g_{\lambda_j , \alpha_j}
(x)\,,
$$
где $p_j$~--- дискретные смешивающие параметры, $\g_{\lambda_j , \alpha_j}
(x)$~--- плотности гамма-распределений.

Такая аппроксимация не только позволяет решить поставленную статистическую
задачу, но и полу\-чить наглядную визуализацию результатов. Существуют
достаточно эффективные методики разделения смесей распределений, среди них~---
семейство так называемых \textit{ЕМ-алгоритмов}
(\textit{Expectation-Maximization Algorithms}).

Полученные результаты могут быть достаточно просто интерпретированы. Число
компонентов смеси символизирует число типичных параллельных или
последовательных структур. Значения параметров составляющих смесь
гам\-ма-рас\-пре\-де\-ле\-ний показывают <<степень параллелизма>>
соответствующей структуры: чем ближе параметр формы к~1, тем выше эта
<<степень>>. И наоборот, чем дальше значение параметра формы от~1, тем больше
последовательных операций выполняется в соответствующем блоке.

Веса компонентов характеризуют примерную долю использования
ресурсов для сообщений, соответствующих каждому распределению входных
данных.

Итак, предложенный подход позволяет получить представление о
стохастической структуре телекоммуникационной сети.

\section{ЕМ-алгоритм разделения смесей распределений}

\subsection{Описание алгоритма} %2.1.

Определяемый ниже итерационный алгоритм решения поставленной в
предыдущем разделе задачи относится к процедурам, базирующимся на
\textit{методе максимального правдоподобия}.

Этот алгоритм позволяет находить максимум логарифмической функции
правдоподобия по параметрам $p_1,\,p_2,\ldots ,\,p_k$, $\theta_1 ,\,\theta_2,\ldots ,\,
\theta_k$ при фиксированном $k$ по выборке $X_1, \ldots , X_n$, т.е.\ решение
оптимизационной задачи вида

\begin{equation} \sum\limits_{i=1}^n \ln \left ( \sum\limits_{j=1}^k p_j f_j
(X_i;\,\theta_j )\right ) \rightarrow \underset{p_j,\,\theta_j}{\max}\,.
\end{equation}

Конкретные алгоритмы, построенные по этой схеме, часто называют
\textit{алгоритмами типа ЕМ}, поскольку в каждом из них можно выделить два
этапа, находящихся по отношению друг к другу в последовательности
итерационного взаимодействия: \textit{оценивание} (\textit{Estimation}) и
\textit{максимизация} (\textit{Maximization})~\cite{4bat}.

Введем в рассмотрение так называемые апостериорные вероятности
$\g_{ij}$ принадлежности наблюдения $X_i$ к $j$-му классу:
\begin{equation} %6
\g_{ij} = \fr{p_j f(X_i;\,\theta_j )}{\sum\limits_{l=1}^k p_l f(X_i;\,\theta_l 
)} \ (i=1,\ldots , n;\ j=1,\ldots ,k)\,.\!\!\end{equation} 
Очевидно, что для 
всех $i=1,\ldots ,n$ и $j=1,\ldots ,k$
$$
\g_{ij} \geq 0,\quad \sum_{j=1}^k \g_{ij} =1\,.
$$


Далее обозначим $\Theta = (p_1,\ldots p_k,\,\theta_1,\ldots ,\theta_k )$ и
представим анализируемую логарифмическую функцию правдоподобия
$$
\ln L(\Theta ) = \sum\limits_{i=1}^n \ln \left (\sum\limits_{j=1}^k p_j f_j
(X_i;\,\theta_j )\right )
$$
в виде
\begin{multline}
\ln L (\Theta ) = \sum\limits_{j=1}^k\sum\limits_{i=1}^n \g_{ij} \ln p_j+{}\\
{}+\sum\limits_{j=1}^k\sum\limits_{i=1}^n \g_{ij} f(X_i;\,\theta_j)-
\sum\limits_{j=1}^k\sum\limits_{i=1}^n \g_{ij} \ln \g_{ij}\,.
\end{multline}

Справедливость этого тождества легко проверяется с учетом
$$
\sum\limits_{j=1}^k \g_{ij} =1\,.
$$

Далее идея построения итерационного алгоритма вычисления оценок
$\hat{\Theta} = (\hat{p}_1,\ldots , \hat{p}_k,\
\hat{\theta}_1,\ldots , \hat{\theta}_k)$
для параметров $\Theta = (p_1,\ldots , p_k,\ \theta_1,\ldots , \theta_k)$ состоит в
следующем:
\begin{enumerate}[1.]
\item Выбирается некоторое \textit{начальное приближение}~$\hat{\Theta}^0$.
\item \textbf{E-step:} вычисляются по формулам~(6) начальные приближения
$\g_{ij}^0$ для апостериорных вероятностей $\g_{ij}$~--- \textit{этап
оценивания}.
\item \textbf{M-step:} затем, возвращаясь к~(7), при вычисленных значениях
$\g^0_{ij}$ следует определить значения $\hat{\Theta}^1$ из условия
максимизации отдельно каждого из первых двух слагаемых правой
части~(7), поскольку первое слагаемое
$$
\sum_{j=1}^k \sum_{i=1}^n \g_{ij} \ln p_j
$$
зависит только от параметров $p_j$, а второе слагаемое
$$
\sum_{j=1}^k \sum_{i=1}^n \g_{ij} f(X_i;\,\theta_j )
$$
зависит только от параметров $\theta_j$~--- \textit{этап максимизации}.
\item Проверяется \textit{условие останова}:
$$
\parallel \Theta^{(t)} - \Theta^{t-1}\parallel <\varepsilon\,,
$$
где $t$~--- номер итерации, а
$\parallel\bullet\parallel$~--- евклидова норма, для некоторого $\varepsilon
>0$.
\end{enumerate}

Очевидно, решение оптимизационной задачи
$$
\sum\limits_{j=1}^k\sum\limits_{i=1}^n \g_{ij}^{(t)}\ln p_j \rightarrow
\underset{p_j}{\max}
$$
дается выражением (с учетом $\sum_{j=1}^k p_j =1$):
$$
p_{ij}^{(t+1)} =\fr{1}{n}\,\sum\limits_{i=1}^n \g_{ij}^{(t)}\,,
$$
где $t$~--- номер итерации, $t = 0$, 1, 2,\,\ldots

Решение оптимизационной задачи
$$
\sum\limits_{j=1}^k \sum\limits_{i=1}^n \g_{ij}^{(t)} f(X_i;\,\theta_j )
\rightarrow \underset{\theta_j}{\max}
$$
получить намного проще решения задачи~(5): выражение для $\theta_j$
записывается с учетом знания конкретного вида функций
$f(X,\,\theta)$~\cite{3bat}.

\subsection{О сходимости алгоритма} %2.2.

В работе М.\,И.~Шлезингера~\cite{5bat}, где эта схема (позднее названная
ЕМ-схемой) впервые предложена, установлены и основные свойства
реа\-ли\-зу\-ющих ее алгоритмов. В частности, было доказано, что при достаточно
широких предположениях \textit{предельные точки} всякой последовательности,
порожденной итерациями ЕМ-алгоритма, являются стационарными точками
оптимизируемой логарифмической функции правдоподобия $\ln L(\Theta )$ и что
найдется неподвижная точка алгоритма, к которой будет сходиться каждая из таких
последовательностей. Если дополнительно потребовать положительной
определенности информационной мат\-ри\-цы Фишера для $\ln L(\Theta )$ при
истинных зна\-че\-ни\-ях па\-ра\-мет\-ра $\Theta$, то можно показать, что
асимптотически по $n\rightarrow\infty$ (т.е.\ при больших выборках) существует
единственное сходящееся (по веро\-ят\-но\-сти) решение $\hat{\Theta}(n)$
уравнений метода максимального правдоподобия и, кроме того, существует в
пространстве параметров $\Theta$ норма, в которой последовательность
$\Theta^{(t)}(n)$, порожденная ЕМ-ал\-го\-рит\-мом, сходится к $\hat{\Theta}
(n)$, если только начальная аппроксимация $\hat{\Theta}^0$ не была слишком
далека от~$\hat{\Theta} (n)$. {%\looseness=1

}

Таким образом, результаты исследования свойств ЕМ-алгоритмов метода
максимального правдоподобия разделения смеси и их практическое
использование показали, что они являются достаточно работоспособными (при
известном чис\-ле компонентов смеси) даже при большом чис\-ле $k$ компонентов и
при высоких размерностях анализируемого признака~$X$~\cite{3bat}.

\subsection{Уравнения для смеси экспоненциальных распределений}
%2.3.

Применим описанный выше алгоритм к разделению смеси
экспоненциальных распределений:
$$
p(x) = \sum\limits_{j=1}^k p_j \lambda_j e^{-\lambda_j x}\,.
$$
Получаем следующие итерационные уравнения:
\begin{align*}
\g_{ij}^{(t+1)} & = \fr{p_j^{(t)}\lambda_j^{(t)}e^{-
\lambda_j^{(t)}X_i}}{\sum\limits_{l=1}^k p_l^{(t)}\lambda_l^{(t)}
e^{-\lambda_l^{(t)}X_i}}\,,\\
p_j^{(t+1)} & = \fr{1}{n}\,\sum\limits_{i=1}^n \g_{ij}^{(t)}\,.
\end{align*}

Чтобы найти  оценки $\lambda_j$, подсчитаем первую производную функции
$$\sum_{j=1}^k\sum_{i=1}^n \g_{ij}^{(t)} \ln (\lambda_j e^{-\lambda_j X_i}):$$
\vspace*{-8pt}
\begin{multline*}
\left ( \sum\limits_{j=1}^k \sum\limits_{i=1}^n
\g_{ij}^{(t)}\ln \left ( \lambda_j
e^{-\lambda_j X_i} \right ) \right )^\prime \lambda_j =\\[-3pt]
{}= \left (
\sum\limits_{j=1}^k\sum\limits_{i=1}^n \g_{ij}^{(t)}\ln (\lambda_j -\lambda_j X_i )
\right )^\prime \lambda_j =\\[-3pt]
{}= \sum\limits_{i=1}^n \g_{ij}^{(t)}\left (
\fr{1}{\lambda_j} - X_i \right )\,.
\end{multline*}

Разрешая уравнение
$$
\sum\limits_{i=1}^n \g_{ij}^{(t)}\left ( \fr{1}{\lambda_j} -X_i\right ) =0
$$
относительно $\lambda_j$, получаем следующее итерационное уравнение:
$$
\lambda_j^{(t+1)} = \fr{\sum\limits_{i=1}^n
\g_{ij}^{(t)}}{\sum\limits_{i=1}^n \g_{ij}^{(t)} X_i}\,.
$$

\subsection{Уравнения для смеси гамма-распределений } %2.4.

Применим теперь ЕМ-алгоритм к смеси гам\-ма-рас\-пре\-де\-ле\-ний вида
$$
p(x) = \sum\limits_{j=1}^k p_j \fr{\alpha_j^{\alpha_j} x^{\alpha_j -
1}}{\lambda_j^{\alpha_j} \Gamma (\alpha_j )}\,e^{-(\alpha_j / \lambda_j)x}\,.
$$

Такая параметризация удобна для нахождения
оценок~$\alpha_j$~\cite{6bat}.

Аналогичным способом выписываются итерационные уравнения:
\begin{align*}
\g_{ij}^{(t+1)} & =   \fr{p_j^{(t)}\fr{(\alpha_j^{\alpha_j} )^{(t)}
x^{\alpha_j - 1}}{(\lambda_j^{\alpha_j} )^{(t)}\Gamma (\alpha_j)}\,
e^{-(\alpha_j /\gamma_j)^{(t)}x}}{\sum\limits_{l=1}^k
p_l^{(t)}\fr{(\alpha_l^{\alpha_l})^{(t)} x^{\alpha_l -
1}}{(\lambda_l^{\alpha_l})^{(t)}\Gamma (\alpha_l )}\,
e^{-(\alpha_l /\lambda_l)^{(t)} x}}\,,\\
p_j^{(t+1)} & = \fr{1}{n}\,\sum\limits_{i=1}^n \g_{ij}^{(t)}\,.
\end{align*}

Далее найдем оценки $\lambda_j$ для данного случая, приравнивая
производную
\begin{equation} %8
\sum\limits_{j=1}^k \sum\limits_{i=1}^n \g_{ij}^{(t)} \ln \left (
\fr{\alpha_j^{\alpha_j} x^{\alpha_j -1}}{\lambda_j^{\alpha_j}\Gamma
(\alpha_j)}\,e^{-(\alpha_j /\lambda_j) x}\right )
\end{equation}
по $\lambda_j$ к нулю и разрешая относительно~$\lambda_j$ уравнение:
$$
\sum\limits_{i=1}^n \g_{ij}^{(t+1)}\left ( \fr{\alpha_j^{(t)}}{\lambda_j^{(t)}}
- \fr{\alpha_j^{(t)}X_i}{\left ( \lambda_j^{(t)}\right )^2}\right ) =0 \,.
$$
Получаем
$$
\lambda_j^{(t+1)} = \fr{\sum\limits_{i=1}^n \g_{ij}^{(t)}
X_i}{\sum\limits_{i=1}^n \g_{ij}^{(t)}}\,.
$$

Для того чтобы получить итерационные уравнения для $\alpha_j$, найдем
первую производную~(8):
\begin{multline*}
\left ( \sum\limits_{j=1}^k\sum\limits_{i=1}^n \g_{ij}^{(t)}\ln \left (
\fr{\alpha_j^{\alpha_j} x^{\alpha_j -1}}{\lambda_j^{\alpha_j}\Gamma (\alpha_j
)}\,e^{-(\alpha_j /\lambda_j ) x} \right ) \right )^\prime \alpha_j ={}\\[-3pt]
{}=\left ( \sum\limits_{j=1}^k\sum\limits_{i=1}^n \g_{ij}^{(t)}\ln \left (
\fr{\alpha_j^{\alpha_j}}{\lambda_j^{\alpha_j}}\right ) - \ln \Gamma (\alpha_j )+{} \right.\\[-3pt]
{}+\left.
(\alpha_j -1 )\ln X_i - \fr{\alpha_j}{\lambda_j}\,X_i \right )^\prime \alpha_j =\\[-3pt]
{}=\sum\limits_{i=1}^n \g_{ij}^{(t)} \left ( \ln \alpha_j +1-\ln \lambda_j -
\fr{\Gamma^\prime (\alpha_j )}{\Gamma (\alpha_j)}\right.+\\[-3pt]
{}+\left. \ln X_i - \fr{X_i}{\lambda_j}\right )\,;
\end{multline*}
\begin{multline*}
\sum\limits_{i=1}^n \g_{ij}^{(t)} \left(  \ln \alpha_j +1 -\ln \lambda_j -{}\right. \\[-3pt]
\left. {}-\fr{\Gamma^\prime (\alpha_j )}{\Gamma (\alpha_j )}+\ln X_i 
-\fr{X_i}{\lambda_j} \right) =0\,;
\end{multline*}
\begin{multline}
\fr{\Gamma^\prime (\alpha_j )}{\Gamma (\alpha_j )} ={}\\[-3pt]
{}= \fr{\sum\limits_{i=1}^n \g_{ij}^{(t)} \left ( \ln \alpha_j +1-\ln\lambda_j 
+\ln X_i -\fr{X_i}{\lambda_j} \right )}{\sum\limits_{i=1}^n \g_{ij}^{(t)}}\,.
\end{multline}
%
Здесь $\Gamma^\prime (\alpha_j ) / \Gamma (\alpha_j )$~--- это
\textit{логарифмическая производная гамма-функции}. Для нее существует так
называемое \textit{разложение Абрамовитца}--\textit{Стигана}~\cite{4bat}:
$$
\fr{\Gamma^\prime (\alpha ) }{ \Gamma (\alpha )} = \mathrm{log}\,\alpha -
\fr{1}{2\alpha }-\fr{1}{12\alpha^2 }+\ldots
$$

Подставим первые три члена разложения в~(9) и разрешим это уравнение
относительно~$\alpha_j$:
$$
\alpha_{ij}^{(t+1)} = \fr{\sum\limits_{i=1}^n
\g_{ij}^{(t+1)}}{2\sum\limits_{i=1}^n \g_{ij}^{(t +1)}\left ( \fr{X_i}{\lambda_j^{(t)}} -
\ln \fr{X_i}{\lambda_j^{(t)}} -1\right )}\,.
$$
В итоге получаем итерационные уравнения для ~$\alpha_j$.

\section{Описание программного обеспечения (программа~ЕМ)}

\subsection{Назначение программы} %3.1.

Разработанная авторами статьи программа ЕМ предназначена для решения задачи
разделения смесей экспоненциальных и гамма-распределений, поставленной в
разд.~2, с использованием ЕМ-ал\-го\-рит\-ма и формул, описанных в разд.~3.

\subsection{Инструменты разработки} %3.2.

Для создания программы была использована среда разработки Microsoft
Visual Studio .NET 2005 и объектно-ориентированный язык C\#. Для
визуализации результатов была использована свободно распространяемая
графическая библиотека ZedGraph~\cite{7bat}.


\subsection{Возможности  программы} %3.3.

\noindent
\begin{itemize}
\item Загрузка выборочных данных из текстового файла
\item Оценивание по выборке параметров смеси экспоненциальных
распределений
\item Оценивание по выборке параметров смеси гамма-распределений
\item Отслеживание изменений параметров смесей распределений во
времени в режиме <<скользящего окна>>
\item Построение гистограммы по выборке
\end{itemize}

\subsection{Входные и выходные данные. Функционирование
программы} %3.4.

В качестве \textit{входных данных} программа ЕМ получает:
\begin{itemize}
\item выборочные данные из текстового файла;
\item число компонентов смеси;
\item размер <<скользящего окна>>;
\item размер класса гистограммы.
\end{itemize}

На \textit{выходе} мы получаем:
\begin{itemize}
\item точечные оценки параметров смеси экспоненциальных
распределений;
\item точечные оценки параметров смеси гамма-распределений;
\item графическое изображение результирующей смеси распределения;
\item графическое изображение компонентов каж\-дой смеси;
\item графическое изображение того, как меняются параметры смесей
распределений с течением времени в режиме <<скользящего окна>>;
\item гистограмма, построенная по выборке;
\item значение статистического теста.
\end{itemize}

Выборочные данные загружаются из текстового файла в память программы и подаются
на вход двум независимо работающим реализациям ЕМ-алгоритма~--- для
идентификации смеси экспоненциальных распределений и для идентификации смеси
гамма-распределений. Результатом их работы являются наборы значений оцениваемых
параметров модели, предложенной в разд.~2. Кроме того, результирующие
распределения визуализируются в виде графиков. В программе можно запустить
режим <<скользящего окна>>, который для всех подвыборок заданного
размера с помощью ЕМ-алгоритма оценивает параметры смесей распределений этих
подвыборок. Все действия программы документируются в окне информации.

\section{Описание тестовых расчетов}

С использованием разработанной программы были проведены тестовые
расчеты на выборочных данных реального сетевого трафика.

На вход программы EM были поданы выборки трафика:
\begin{enumerate}[I]
\item Между лабораторией Lawrence Berkeley (Berkeley, California) и
внешним миром размера примерно 7000~\cite{8bat}~--- \textit{выборка~1}.
\item
Сети радиодоступа ЗАО <<Синтерра>> размера примерно 1000~\cite{9bat}~---
 \textit{выборка~2}.
\end{enumerate}

\subsection{Выборка 1 ``Berkeley''} %5.1.

При числе компонентов смеси~5 и случайном начальном приближении
были получены результаты, представленные в табл.~\ref{t1bat}.


Данные результаты иллюстрирует рис.~\ref{f5bat}.

Гистограмма  на рис.~\ref{f6bat} показывает статистическую значимость
полученных результатов.

Данная выборка обладает той особенностью, что она собиралась в течение
достаточно длительного времени и в ней агрегирован самый разнородный
трафик. Поэтому в ней присутствует не только большое количество
<<коротких>> сообщений (что обычно для выборок из телетрафика), но и
некоторый массив сообщений средней длины, а также определенный
<<выброс>> больших сообщений. Это свидетельствует о \textit{пиковости}
телетрафика на довольно больших промежутках времени.

Как мы видим, ЕМ-алгоритм удачно справился с задачей идентификации
смеси.

\subsection{Выборка~2 ``Synterra''} %5.2.

Результаты применения ЕМ-алгоритма к выборке ``Synterra''
представлены в табл.~\ref{t2bat}.
\begin{table*}\small
\begin{minipage}[t]{76mm}
\begin{center}
\Caption{Результаты применения ЕМ-алго\-рит\-ма к выборке~1 ``Berkeley'' 
\label{t1bat}} \vspace*{2ex}

\tabcolsep=8.7pt
\begin{tabular}{|c|c|c|}
\hline
№&Начальное приближение&Результат\\
\hline
\multicolumn{3}{|c|}{$P$}\\
\hline
0&0,2&0,1896\\
1&0,2&0,1858\\
2&0,2&0,1830\\
3&0,2&0,2259\\
4&0,2&0,2154\\
\hline
\multicolumn{3}{|c|}{$\alpha$}\\
\hline
0&2,7028&10,9783\hphantom{9}\\
1&3,6273&5,8621 \\
2&5,7598&2,7092\\
3&0,2315&1,0235\\
4&0,9110&0,4772\\
\hline
\multicolumn{3}{|c|}{$\lambda$}\\
\hline
0&85,2066&137,1714  \\
1&23,9592&136,7349\\
2&63,8425&132,6482\\
3&\hphantom{9}1,8026&116,7317\\
4&98,3882&102,5278\\
\hline
\end{tabular}
\end{center}
\end{minipage}\hfill
\begin{minipage}[t]{76mm}
%\end{table*}
%\begin{table*}\small
\begin{center}
\Caption{Результаты применения ЕМ-алго\-рит\-ма к выборке~2 ``Synterra'' 
\label{t2bat}} \vspace*{2ex}

\tabcolsep=8.7pt
\begin{tabular}{|c|c|c|}
\hline
№&Начальное приближение&Результат\\
\hline
\multicolumn{3}{|c|}{$P$}\\
\hline
0&0,2&$0{,}3815\hphantom{{}\cdot 10^{-9}}$\\
1&0,2&$0{,}3594\hphantom{{}\cdot 10^{-9}}$\\
2&0,2&$0{,}2589\hphantom{{}\cdot 10^{-9}}$\\
3&0,2&$0{,}4401\cdot 10^{-9}$\\
4&0,2&$0{,}0\hphantom{{}\cdot 10^{-9}999}$\\
\hline
\multicolumn{3}{|c|}{$\alpha$}\\
\hline
0&6,0804&1,5833\\
1&3,1838&0,8554\\
2&1,4886&0,4557\\
3&4,6407&0,2278\\
4&3,7843&0,1139\\
\hline
\multicolumn{3}{|c|}{$\lambda$}\\
\hline
0&17,3387&15,8682\\
1&47,8294&16,9150\\
2&54,1984&19,2866\\
3&\hphantom{1}8,6254&19,2866\\
4&\hphantom{1}5,7252&19,2866\\
\hline
\end{tabular}
\end{center}
\end{minipage}
\end{table*}


Данные результаты иллюстрируют рис.~\ref{f7bat}.


Эти результаты также отражают действительную картину, как показано на
рис.~\ref{f8bat}.


Этот трафик был снят с базовой станции <<Лукойл-Юго-Запад>> сети
широкополосного радиодоступа ЗАО <<Синтерра>>. Сеть радиодоступа
является реализацией так называемой <<последней мили>>, переносящей два
разных вида трафика: данные (Ethernet пакеты) и голос (IP-телефония, VoIP).
Поэтому здесь присутствуют в качестве основной массы короткие, но
интенсивные сообщения (пакеты SIP и голосовые фреймы), а также длинные
сообщения, содержащие данные.

Как мы видим, программная реализация ЕМ-ал\-го\-рит\-ма успешно справилась с
задачей разделения смесей распределений для этих двух выборок, что делает
данную программу удобным инструментом построения стохастической картины
конкретной сети. По полученным данным, используя метод интерпретации,
предложенный в разд.~2, можно получить представление о количестве
последовательных и параллельных структур вероятностной модели сети.

\subsection{Режим <<скользящего окна>>} %5.3.

Результаты для выборки
``Berkeley'' в режиме <<скользящего окна>>  представлены
на рис.~\ref{f9bat}.


Данные графики показывают изменение параметров распределений подвыборок выборки 
``Berkeley''. Видно, что параметры распределений подвыборок не остаются 
неизменными во времени, наоборот, они имеют внешне случайный характер. На 
рис.~\ref{f9bat},\,\textit{в} видна даже своеобразная пульсация первой 
компоненты.
%
На основании расчетов можно сделать вывод о том, что пиковость трафика
обусловливается как формой, так и интенсивностью сообщений.

\section{Заключение}

В данной работе исследована вероятностная модель  информационных потоков,
возникающих в сложных телекоммуникационных конвергентных сетях, построенная с
помощью асимптотического и энтропийного подходов. Эта модель предполагает, что
функционирование сложной телекоммуникационной сети можно представить в виде
суперпозиции довольно простых стохастических структур~--- последовательных и
параллельных, которые по\-рож\-да\-ют смеси гамма-распределений для случайной
величины времени обработки и передачи сообщений в сети. Предложена простая
интерпретация параметров данной модели.
\begin{figure*} %fig5
\vspace*{1pt}
\begin{center}
\mbox{%
\epsfxsize=130mm %145.109mm 
\epsfbox{bat-5.eps} }
\end{center}
\vspace*{-13pt} \Caption{Компоненты смеси начального приближения~(\textit{а}) и 
результата~(\textit{б}) для выборки~1 ``Berkeley'' \label{f5bat}}
%\end{figure*}
%\begin{figure*} %fig6
\vspace*{12pt}
\begin{center}
\mbox{%
\epsfxsize=130mm %148.256mm 
\epsfbox{bat-7.eps} }
\end{center}
\vspace*{-13pt} \Caption{График смеси распределений~(\textit{1}) и гистограмма 
для выборки~1 ``Berkeley''~(\textit{2}) \label{f6bat}}
\end{figure*}



\begin{figure*} %fig7
\vspace*{1pt}
\begin{center}
\mbox{%
\epsfxsize=130mm %144.283mm 
\epsfbox{bat-8.eps} }
\end{center}
\vspace*{-16pt} \Caption{Компоненты смеси начального приближения~(\textit{а}) и 
результата~(\textit{б}) для выборки~2 ``Synterra'' \label{f7bat}}
%\end{figure*}
%\begin{figure*} %fig8
\vspace*{12pt}
\begin{center}
\mbox{%
\epsfxsize=130mm %148.256mm 
\epsfbox{bat-10.eps} }
\end{center}
\vspace*{-11pt} \Caption{График смеси распределений~(\textit{1}) и гистограмма
для выборки~2 ``Synterra''~(\textit{2}) \label{f8bat}}
\end{figure*}

\begin{figure*} %fig9
\vspace*{1pt}
\begin{center}
\mbox{%
\epsfxsize=119.041mm
\epsfbox{bat-11.eps} }
\end{center}
\vspace*{-9pt} \Caption{Изменение  смешивающих параметров~(\textit{а}), 
параметров формы~(\textit{б}) и параметров масштаба~(\textit{в}) во времени для 
выборки~1 ``Berkeley'' \label{f9bat}}
\end{figure*}

Для решения вытекающей из модели задачи предложен итерационный алгоритм,
базирующийся на методе максимального правдоподобия~--- ЕМ-ал\-го\-ритм, для
которого получены формулы для конкретного вида смесей~--- экспоненциальных и
гамма-распределений.
%
Кроме того, разработан программный инструментарий для оценки параметров 
предложенной модели на выборках из реальных трафиковых данных. Проведены 
исследования, которые подтвердили предположения вероятностной модели. 


Получение информации о стохастической структуре
телекоммуникационных сетей и наличие программных инструментов для
выявления более или менее стабильных структур позволит понять причины
возникновения неожиданных больших нагрузок, предотвратить такие нагрузки,
а также поможет в будущем в проектировании надежных, оптимальных по
стоимости и уровню сервиса телекоммуникационных сетей нового поколения.

%\vspace*{-15pt} 
{\small\frenchspacing
{%\baselineskip=10.8pt
\addcontentsline{toc}{section}{Литература}
\begin{thebibliography}{9}
\bibitem{1bat}
Teletraffic Engeneering Handbook. International Telecommunication Union, 
Geneva, 2005 {\sf http://www.itu.int}. \vspace*{5pt} 
\bibitem{2bat}
\Au{Севастьянов~Б.\,А.} Курс теории вероятностей и математической статистики. 
М., 2004. \vspace*{5pt} 
\bibitem{3bat}
\Au{Айвазян~C.\,А., Бухштабер~В.\,М., Енюков~И.\,С, Мешалкин~Л.\,Д.} Прикладная 
статистика. Классификация и снижение размерности~// Финансы и статистика. М., 
1989. \vspace*{5pt} 
\bibitem{4bat}
\Au{Bilmes~J.\,A.} A gentle tutorial of the EM algorithm and its application to 
parameter estimation for Gaussian mixture and hidden Markov models. Berkeley, 
CA, USA: International Computer Science Institute,  1998. \vspace*{5pt} 
\bibitem{5bat}
\Au{Шлезингер~М.\,И.} О самопроизвольном различении образов~// Шлезингер~М.\,И. 
Читающие. автоматы. Киев: Наукова думка, 1965. С.~38--45. \vspace*{5pt} 
\bibitem{6bat}
\Au{Hsiao~I.-T., Rangarajan~A., Gindi~G.}. Joint-MAP 
reconstruction/segmentation for transmission tomography using mixture-models as 
priors. Yale University, 1998. \vspace*{5pt} 
\bibitem{7bat}
{\sf http://zedgraph.org}. \vspace*{4pt} 
\bibitem{8bat}
{\sf http://ita.ee.lbl.gov/html/contrib/LBL-PKT.html}. \vspace*{5pt} 
\bibitem{9bat}
{\sf http://www.synterra.ru}.
\end{thebibliography}

} } \label{end\stat}
\end{multicols}


%\addtocounter{razdel}{1}
%\def\razd{НЕРЕГУЛИРУЕМЫЙ ЭЛЕКТРОПРИВОД ДЛЯ ЭЛЕКТРОЭНЕРГЕТИКИ}

\setcounter{page}{2}

{ %\Large  
{ %\baselineskip=16.6pt

\vspace*{-48pt}
\begin{center}\LARGE
\textit{Уважаемый читатель!}
\end{center}

%\vspace*{2.5mm}

\vspace*{4mm}

\thispagestyle{empty}

{\small

 
В~2017~г.\ исполняется 10~лет со времени выхода в~свет первого 
номера журнала <<Информатика и~её применения>>~--- 
научного журнала Российской академии наук, издающегося под 
на\-уч\-но-ме\-то\-ди\-че\-ским руководством Отделения нанотехнологий 
и~информационных технологий Российской академии наук. Учредителем журнала 
является Федеральный исследовательский центр <<Информатика и~управ\-ле\-ние>> 
Российской академии наук (ФИЦ ИУ РАН) (до~2015~г.~--- 
Институт проб\-лем информатики РАН).

Необходимость издания такого журнала была вызвана активным развитием 
информатики и~информационных технологий, большой важностью этого научного 
направления для развития страны, проникновением информационных технологий 
во все сферы жизни современного общества.

Тематику журнала определяет тот факт, что информатика~--- это комплексная 
фундаментальная научная дисциплина, опирающаяся на достижения 
ряда других наук, в~том числе математики, физики, лингвистики и~др. 
Одновременно журнал уделяет большое внимание современным информационным технологиям, 
являющимся приложениями результатов информатики как фундаментальной науки.

За прошедшие 10~лет (2007--2016~гг.)\ издано~38~выпусков журнала. В~них 
размещено~452~публикации, в~том числе~430~научных статей и~22~информационных 
публикации (обзоры, рецензии и~др.). Среди авторов журнала представители ведущих 
научных организаций и~университетов страны, в~том числе Московского государственного 
университета им.\ М.\,В.~Ломоносова, ФИЦ ИУ РАН (в~том числе ИПИ РАН, ВЦ 
им.\ А.\,А.~Дородницына РАН, ИСА РАН), Института точной механики и~вычислительной 
техники им.\ С.\,А.~Лебедева РАН, Института космических исследований РАН, 
Института астрономии РАН, ряда институтов Сибирского отделения РАН, МФТИ, МИФИ, 
Высшей школы экономики, Санкт-Пе\-тер\-бург\-ско\-го государственного университета, 
Санкт-Пе\-тер\-бург\-ско\-го государственного политехнического университета 
Петра Великого, Санкт-Пе\-тер\-бург\-ско\-го государственного университета 
телекоммуникаций им.\ проф.\ М.\,А.~Бонч-Бруе\-ви\-ча, 
Российского университета дружбы народов, Балтийского федерального университета 
имени Иммануила Канта, Вологодского государственного университета и~др. 
Публиковались статьи зарубежных авторов, в~том числе ученых из Израиля, 
США, Финляндии, Франции, Швейцарии, Швеции и~других стран. 

В конце настоящего выпуска журнала помещен указатель статей, 
опуб\-ли\-ко\-ван\-ных в~томах~1--10 (2007--2016~гг.).

Журнал включен в~Российский индекс научного цитирования и~в~базу 
данных RSCI Web of Science, перечень ВАК, базу данных CrossRef 
и~информационную систему <<Общероссийский математический портал MathNet>>. 
С~2015~г.\ журнал индексируется в~библиографической и~реферативной базе 
данных SCOPUS.

Мы всегда будем помнить ушедших из жизни членов редакционного совета 
и~редакционной коллегии журнала: академика С.\,К.~Коровина, профессоров 
А.\,В.~Печинкина и~И.\,А.~Ушакова, которые внесли неоценимый вклад в~становление 
и~развитие журнала.

После объединения в~2015~г.\ трех учреждений Российской академии наук~--- 
Института проблем информатики, Вычислительного центра им.\ А.\,А.~Дородницына 
и~Института системного анализа~--- в~Федеральное государственное учреждение 
<<Федеральный исследовательский центр <<Информатика и~управ\-ле\-ние>> 
Российской академии наук>> (ФИЦ ИУ РАН) именно этот Центр стал базовой организацией 
для издания журнала, что существенно расширило как тематику журнала, 
так и~его возможности по привлечению новых авторов, в~том числе и~зарубежных.

В настоящее время тематику журнала в~первую очередь составляют:
\begin{itemize}
\item    теоретические основы информатики;\\[-14.5pt] 
\item    математические методы исследования сложных систем и~процессов;\\[-14.5pt]
\item    информационные системы и~сети;\\[-14.5pt]
\item    информационные технологии;\\[-14.5pt]
\item    архитектура и~программное обеспечение вычислительных комплексов и~сетей. 
\end{itemize}

Эти направления особенно важны в~связи с необходимостью решения задач 
формирования технологической базы инновационного развития, обеспечения 
на\-уч\-но-тех\-но\-ло\-ги\-че\-ско\-го прорыва в~области создания и~развития 
отечественных информационных и~коммуникационных технологий в~интересах 
достижения высокого качества и~стабильности систем управления и~предоставления 
услуг в~экономической и~социальной сферах. 

Мы, как и~ранее, приглашаем авторов представлять для публикации в~журнале 
статьи как с достижениями в~области теоретических проблем информатики, так 
и~с~изложением результатов ее практического приложения, а~также 
рецензии на наиболее интересные книжные новинки в~области информатики 
и~информационных технологий, объявления о~крупнейших международных 
и~всероссийских конференциях, различных научных мероприятиях 
по этой тематике и~другие информационные материалы.

Надеемся, что и~в~дальнейшем содержание статей, помещаемых в~журнале, 
будет вызывать интерес научной общественности. Редакционный совет, редколлегия 
и~редакция журнала, со своей стороны, сделают все для того, 
чтобы журнал и~впредь своевременно и~подробно информировал читателей 
о~новейших достижениях информатики и~ее актуальных практических приложениях.

                

      
\vfill
\noindent
Главный редактор журнала <<Информатика и~её применения>>,\\
академик  РАН\hfill
\textit{И.\,А.~Соколов}\\[-6pt]

%\noindent
%Редактор-составитель тематического выпуска, профессор кафедры математической статистики\\
%факультета вычислительной математики и~кибернетики МГУ им.~М.\,В.~Ломоносова,\\
%ведущий научный сотрудник ИПИ РАН, доктор физико-математических наук\hfill
% \textit{В.\,Ю.~Королев}


} }
}
      



%{ %\Large  
{ %\baselineskip=16.6pt

\vspace*{-48pt}
\begin{center}\LARGE
\textit{Уважаемый читатель!}
\end{center}

%\vspace*{2.5mm}

\vspace*{4mm}

\thispagestyle{empty}

{\small

 
В~2017~г.\ исполняется 10~лет со времени выхода в~свет первого 
номера журнала <<Информатика и~её применения>>~--- 
научного журнала Российской академии наук, издающегося под 
на\-уч\-но-ме\-то\-ди\-че\-ским руководством Отделения нанотехнологий 
и~информационных технологий Российской академии наук. Учредителем журнала 
является Федеральный исследовательский центр <<Информатика и~управ\-ле\-ние>> 
Российской академии наук (ФИЦ ИУ РАН) (до~2015~г.~--- 
Институт проб\-лем информатики РАН).

Необходимость издания такого журнала была вызвана активным развитием 
информатики и~информационных технологий, большой важностью этого научного 
направления для развития страны, проникновением информационных технологий 
во все сферы жизни современного общества.

Тематику журнала определяет тот факт, что информатика~--- это комплексная 
фундаментальная научная дисциплина, опирающаяся на достижения 
ряда других наук, в~том числе математики, физики, лингвистики и~др. 
Одновременно журнал уделяет большое внимание современным информационным технологиям, 
являющимся приложениями результатов информатики как фундаментальной науки.

За прошедшие 10~лет (2007--2016~гг.)\ издано~38~выпусков журнала. В~них 
размещено~452~публикации, в~том числе~430~научных статей и~22~информационных 
публикации (обзоры, рецензии и~др.). Среди авторов журнала представители ведущих 
научных организаций и~университетов страны, в~том числе Московского государственного 
университета им.\ М.\,В.~Ломоносова, ФИЦ ИУ РАН (в~том числе ИПИ РАН, ВЦ 
им.\ А.\,А.~Дородницына РАН, ИСА РАН), Института точной механики и~вычислительной 
техники им.\ С.\,А.~Лебедева РАН, Института космических исследований РАН, 
Института астрономии РАН, ряда институтов Сибирского отделения РАН, МФТИ, МИФИ, 
Высшей школы экономики, Санкт-Пе\-тер\-бург\-ско\-го государственного университета, 
Санкт-Пе\-тер\-бург\-ско\-го государственного политехнического университета 
Петра Великого, Санкт-Пе\-тер\-бург\-ско\-го государственного университета 
телекоммуникаций им.\ проф.\ М.\,А.~Бонч-Бруе\-ви\-ча, 
Российского университета дружбы народов, Балтийского федерального университета 
имени Иммануила Канта, Вологодского государственного университета и~др. 
Публиковались статьи зарубежных авторов, в~том числе ученых из Израиля, 
США, Финляндии, Франции, Швейцарии, Швеции и~других стран. 

В конце настоящего выпуска журнала помещен указатель статей, 
опуб\-ли\-ко\-ван\-ных в~томах~1--10 (2007--2016~гг.).

Журнал включен в~Российский индекс научного цитирования и~в~базу 
данных RSCI Web of Science, перечень ВАК, базу данных CrossRef 
и~информационную систему <<Общероссийский математический портал MathNet>>. 
С~2015~г.\ журнал индексируется в~библиографической и~реферативной базе 
данных SCOPUS.

Мы всегда будем помнить ушедших из жизни членов редакционного совета 
и~редакционной коллегии журнала: академика С.\,К.~Коровина, профессоров 
А.\,В.~Печинкина и~И.\,А.~Ушакова, которые внесли неоценимый вклад в~становление 
и~развитие журнала.

После объединения в~2015~г.\ трех учреждений Российской академии наук~--- 
Института проблем информатики, Вычислительного центра им.\ А.\,А.~Дородницына 
и~Института системного анализа~--- в~Федеральное государственное учреждение 
<<Федеральный исследовательский центр <<Информатика и~управ\-ле\-ние>> 
Российской академии наук>> (ФИЦ ИУ РАН) именно этот Центр стал базовой организацией 
для издания журнала, что существенно расширило как тематику журнала, 
так и~его возможности по привлечению новых авторов, в~том числе и~зарубежных.

В настоящее время тематику журнала в~первую очередь составляют:
\begin{itemize}
\item    теоретические основы информатики;\\[-14.5pt] 
\item    математические методы исследования сложных систем и~процессов;\\[-14.5pt]
\item    информационные системы и~сети;\\[-14.5pt]
\item    информационные технологии;\\[-14.5pt]
\item    архитектура и~программное обеспечение вычислительных комплексов и~сетей. 
\end{itemize}

Эти направления особенно важны в~связи с необходимостью решения задач 
формирования технологической базы инновационного развития, обеспечения 
на\-уч\-но-тех\-но\-ло\-ги\-че\-ско\-го прорыва в~области создания и~развития 
отечественных информационных и~коммуникационных технологий в~интересах 
достижения высокого качества и~стабильности систем управления и~предоставления 
услуг в~экономической и~социальной сферах. 

Мы, как и~ранее, приглашаем авторов представлять для публикации в~журнале 
статьи как с достижениями в~области теоретических проблем информатики, так 
и~с~изложением результатов ее практического приложения, а~также 
рецензии на наиболее интересные книжные новинки в~области информатики 
и~информационных технологий, объявления о~крупнейших международных 
и~всероссийских конференциях, различных научных мероприятиях 
по этой тематике и~другие информационные материалы.

Надеемся, что и~в~дальнейшем содержание статей, помещаемых в~журнале, 
будет вызывать интерес научной общественности. Редакционный совет, редколлегия 
и~редакция журнала, со своей стороны, сделают все для того, 
чтобы журнал и~впредь своевременно и~подробно информировал читателей 
о~новейших достижениях информатики и~ее актуальных практических приложениях.

                

      
\vfill
\noindent
Главный редактор журнала <<Информатика и~её применения>>,\\
академик  РАН\hfill
\textit{И.\,А.~Соколов}\\[-6pt]

%\noindent
%Редактор-составитель тематического выпуска, профессор кафедры математической статистики\\
%факультета вычислительной математики и~кибернетики МГУ им.~М.\,В.~Ломоносова,\\
%ведущий научный сотрудник ИПИ РАН, доктор физико-математических наук\hfill
% \textit{В.\,Ю.~Королев}


} }
}
      

\def\stat{konovalov}

\def\tit{О ПЛАНИРОВАНИИ ПОТОКОВ В~СИСТЕМАХ ВЫЧИСЛИТЕЛЬНЫХ 
РЕСУРСОВ$^*$}

\def\titkol{О планировании потоков в~системах вычислительных 
ресурсов}

\def\autkol{М.\,Г.~Коновалов}
\def\aut{М.\,Г.~Коновалов$^1$}

\titel{\tit}{\aut}{\autkol}{\titkol}

{\renewcommand{\thefootnote}{\fnsymbol{footnote}}\footnotetext[1]
{Работа выполнена при поддержке РФФИ, гранты 09-07-12032-офи\_м, 08-07-00152-a.}}

\renewcommand{\thefootnote}{\arabic{footnote}}
\footnotetext[1]{Институт проблем информатики Российской академии наук, mkonovalov@ipiran.ru}


\Abst{Рассмотрена проблема анализа и оптимизации распределения потоков 
заданий и ценообразования в системах коллективного использования распределенных 
вычислительных ресурсов. Проведен обзор литературных источников. Предложен 
подход к построению математических моделей систем вычислительных ресурсов, 
основанный на укрупненном описании потоков заданий в виде балансовых 
соотношений и использовании функций качества. Участники системы представляются 
как субъекты, обладающие собственными стратегиями поведения и преследующие 
индивидуальные цели, сформулированные в терминах качества обслуживания и 
стоимости. В качестве варианта стратегии участников рассматривается распределенный 
децентрализованный алгоритм градиентного типа. Приведен численный пример и 
обсуждены перспективы развития и использования модели.}
      
      \KW{системы вычислительных ресурсов; распределение потоков; качество 
обслуживания; коллективное поведение}

     \vskip 18pt plus 9pt minus 6pt

      \thispagestyle{headings}

      \begin{multicols}{2}

      \label{st\stat}

\section{Введение}
  В современном мире к ряду наиболее существенных для человечества ресурсов, таких как 
вода, нефть и~т.\,п., добавился еще один, хотя и искусственный, но жизненно важный 
ресурс, который можно назвать вычислительным (или, более широко, ин\-фор\-ма\-ци\-он\-но-вы\-чис\-ли\-тель\-ным). 
Его запасы, в отличие от естественных ресурсов, увеличиваются. 
<<Можно констатировать, что экспоненциальный характер роста вычислительных 
мощностей\ldots и систем хранения информации сохранится еще на многие годы\ldots>>~[1].
  
Несмотря на то, что количество компьютеров увеличивается, а их мощность продолжает рас\-ти, 
эффективность использования вычислительной техники, по общераспространенному 
мнению, отстает от этого процесса, оставаясь невысокой. Имеет место ситуация, при которой 
сопряженное с огромными материальными и интеллектуальными затратами наращивание 
вычислительных ресурсов (ВР) не дает должной отдачи, что, учитывая 
подверженность данного специфического ресурса моральному старению, делает проблему 
особенно острой.
  
  Вполне естественная идея увеличить действенность имеющегося и обновляющегося парка 
компьютеров за счет обеспечения более широкого, стандартизованного и облегченного 
доступа нашла свою реализацию в виде систем коллективного использования 
распределенных ВР, в первую очередь гридов~[2]. Данная статья касается одного из общих 
аспектов, связанных с разработкой и эксплуатацией систем коллективного использования 
ВР, и посвящена проблематике выбора принципов и алгоритмов распределения заданий 
между имеющимися ресурсами.
  
  Всякая система коллективного использования ВР предполагает наличие двух типов 
со\-став\-ля\-ющих, участвующих в ее работе: потребителей ресурсов и собственно ресурсов, т.\,е.\ 
вычислительной техники. С узкотехнической точки зрения потребители ассоциируются 
с источниками заданий. Последние могут и должны быть выполнены на том или ином ресурсе, 
они имеют определенную трудоемкость, сопряжены с передачей определенного объема 
информации, а также обладают рядом других показателей, включая требования к качеству 
обслуживания. Точно так же ресурс в узком смысле является той или иной разновидностью 
компьютера (персональный компьютер, кластер, суперкомпьютер и~т.\,д.) или хранилища 
данных и обладает определенными техническими па\-ра\-мет\-ра\-ми: производительностью, 
емкостью памяти, программным обеспечением и~пр. Таким образом, проблема 
планирования потоков заданий в узком смысле может пониматься как составление 
расписания, предписывающего место и очередность выполнения того или иного задания и 
увязывающего рабочие характеристики исполняемых заданий и используемых ресурсов.
  
  В то же время в современных больших вычислительных системах, за счет их масштаба, 
разнородности оборудования, различий в административном подчинении и других 
особенностей, при планировании появляются качественно новые моменты, не учитываемые 
в классической теории расписаний и связанные с взаимозависимостью отдельных 
участников системы. 

К~примеру, приходится принимать во внимание, что потребители не 
только порождают потоки заданий, но и преследуют при их выполнении довольно сложные 
цели, порожденные соображениями финансового, или приоритетного, или секретного и~т.\,д. 
характера. К~тому же потребители часто объединяются в группы или сообщества, оставаясь 
при этом в определенной мере самостоятельными, и к тому же разделенными географически. 

Аналогично за ресурсом как техническим устройством обычно стоит еще и владелец 
ресурса со своими собственными интересами. Приходится констатировать, что отдельные 
составные части распределенной системы ВР обладают собственным поведением, 
целенаправленность которого, вообще говоря, не совпадает с приоритетами сис\-те\-мы в 
целом. Проблема планирования в такой системе в широком смысле не может быть сведена к 
составлению расписания и созданию со\-от\-вет\-ст\-ву\-юще\-го связующего программного 
обеспечения. Необходимо концептуальное понимание принципов и наличие методов, 
позволяющих организовать поведение участников системы и учитывающих как собственные 
интересы участников, так и цели, стоящие перед всей системой.
  
  Термин \textit{система вычислительных ресурсов} не относится к числу точных или даже 
вызывающих однозначные ассоциации. По-видимому, безусловно общими для всех систем 
ВР являются следующие признаки: 
  \begin{enumerate}[(1)]
  \item наличие собственно ВР как технических устройств (процессоров и структур, 
составленных из процессоров, носителей информации);
  \item наличие источников заданий, выполняемых с помощью этих устройств;
  \item объединение поименованных в первых двух пунктах элементов в систему, 
позволяющее различным источникам заданий обращаться к разным ресурсам, 
обмениваться информацией между элементами, осуществлять более или менее 
согласованную политику функционирования составных частей.
  \end{enumerate}
  
  Перечисленным признакам удовлетворяют самые различные и по масштабу, и по целям 
системы ВР. Основным <<примером>> для данной работы можно считать системы грид с их, 
как правило, глобальным масштабом, гетерогенным составом и автономностью поведения 
подсистем. Однако было бы неверно ограничивать статью только об\-ластью проблематики 
грида. Например, в крупных и не очень крупных компаниях повсеместно приходят к мысли о 
необходимости организации более эффективного менеджмента в области использования ВР. 
<<Территориально-распределенное предприятие~--- более 1700~км магистральных 
трубопроводов, которые протянулись от Полярного круга до юга Тюменской области, 
17~компрессорных станций, один из крупнейших в России завод стабилизации газового 
конденсата\ldots Сбой в любом звене может привести к достаточно серьезным последствиям. 
Поэтому года три назад\ldots была осознана необходимость создания системы управления 
ин\-фор\-ма\-ци\-он\-но-вы\-чис\-ли\-тель\-ны\-ми ресурсами>>. (Это цитата 2010~г.\ из 
публикации о компании ООО <<Сургутгазпром>>~[3].)
  
  Данная работа направлена на изучение таких систем ВР, в которых составные части 
обладают определенной самостоятельностью в выборе критериев деятельности и стратегий 
управления потоками заданий и распределения ресурсов. Другая особенность постановки 
задачи заключается в том, что отдельные субъекты, составляющие систему, 
многофункциональны. Они могут одновременно являться как источниками заданий, так и 
пред\-став\-лять запас вычислительных ресурсов или выполнять посреднические функции.

\section{Краткий обзор литературы}
  
  За последние годы опубликовано много работ по планированию в системах ВР, прежде 
всего, в сис\-те\-мах грид. Это отражено в обзорах, посвященных данной тематике~[4, 5]. 
Публикации самого последнего времени, которые упоминаются ниже, представляют 
большое разнообразие постановок задач, методов и приложений.
  
  Проблема управления ресурсами в системах ВР может трактоваться как имеющая две 
основные со\-став\-ляющие. Первая~--- это нахождение алгоритмов, которые бы эффективно 
обслуживали конкретные задания, направляя их на конкретные\linebreak ресурсы. Эта задача в целом 
находится в об\-ласти классических оптимизационных задач (за рубежом для ее обозначения 
повсеместно используется термин scheduling), хотя и осложнена большой размерностью и 
обилием специфических особен\-ностей. Вторая~--- связана с уже упоминавшейся 
авто\-номностью субъектов, обладающих собственными  %\linebreak 
целями и стратегиями поведения. 
Реализация стратегии, оптимизирующей глобальную для всей системы целевую функцию 
(даже если такая стратегия будет найдена), натолкнется на со\-про\-тив\-ле\-ние отдельных 
элементов системы, если не будут учтены их локальные интересы.
  
  Работы первого направления продолжают широко использовать классические статические 
оптимизационные постановки задач с <<интегральным>> описанием потоков заданий~[6] и 
применением традиционных потоковых алгоритмов~[7]. По-преж\-нему популярна 
\textit{задача о рюкзаке}, которая применительно к проблеме планирования ресурсов 
дополня\-ется использованием функций полезности и метрик качества обслуживания~[8], 
стоимостными соображениями~[9], а также эвристическими методами~[10]. 

Во многих работах, 
однако, рассматривается <<штучная>> обработка заданий. В~этих случаях описание 
моделей осуществляется в терминах случайных процессов, а для оптимизации часто 
употребляется марковский процесс принятия решений~[11, 12]. В~[5] предложена модель, в 
которой для оперативного управления коллективным доступом к распределенным 
вычислительным ресурсам используются алгоритмы, основанные на адаптивном варианте 
теории управления марковскими цепями.
  
  Поскольку точные методы решения классических оптимизационных задач часто 
неэффективны, то широкую популярность приобрели эвристические алгоритмы~[13--15]. Не 
являясь сугубой принадлежностью планирования потоков заданий, они фигурируют за 
рубежом под экзотическими названиями (artificial fish swarm algorithm, genetic algorithm, 
simulated annealing и~пр.)
  
  Некоторые работы используют менее распространенные подходы. 
  
  Так, в~[16] 
предлагается механизм планирования заданий в гриде на основе предварительного 
резервирования в режиме онлайн. В~[17] описан алгоритм диспетчеризации заданий также в 
режиме реального времени, но на основе балансировки нагрузки путем прогнозирования 
производительности серверов. Алгоритмы балансировки нагрузки разработаны~[18, 19], а 
статистическое прогнозирование явилось исходной посылкой для создания стратегии 
планирования в~[20]. В~[21] для распределения заданий в вычислительном гриде 
используются нечеткие множества и нейронные сети. В~[22] та же задача решается с 
применением техники так называемых \textit{сложных сетей} (complex network). В~[23] 
обсуждаются специфические проблемы планирования, возникающие при обслуживании 
заданий разного типа: ориентированных на вычисления и связанных преимущественно с 
передачей данных. В~[24] описана необычная постановка задачи распределения мобильных 
ресурсов.
\columnbreak
  
  Отмеченное выше второе направление исследований, связанное с автономностью 
поведения участников системы ВР, представлено значительно беднее. Работы, которые 
можно было бы отнести к этому направлению, часто отражают в большей степени 
гетерогенность составных блоков системы, нежели их самостоятельность.
  
  В~[25] предлагается модель размещения заданий, параллельно выполняемых в 
автономных доменах. Другой подход к организации параллельных вы\-чис\-ле\-ний в 
гетерогенной среде основан на привлечении так называемых систем с агентами~[26].

 В~[27] 
рассмотрена технология организации планиро\-вания в гриде, составные части которого 
основаны на разных стандартах, что затрудняет коллективное использование ресурсов. 
Рассмотренное в~[27] понятие \textit{федерации гридов} (grid-federation) 
широко разрабатывается авторами статьи~[28], в которой дана схема кооперированного 
использования ресурсов в гриде, основанная на концепции соглашений об уровне 
обслуживания (service level agreement). 

Стратегии \textit{согласований} (negotiations) 
рассмотрены в~[29].
  
  Большие надежды в организации менеджмента в системах ВР возлагаются на 
экономический подход, побуждающий, как принято считать, разнородные и конкурирующие 
элементы действовать, соблюдая интересы системы в целом. Эта часть публикаций 
заслуживает отдельного обзора (в некоторой степени он проведен в~[4]). Здесь упомянем в 
качестве примера сравнительный анализ менеджмента в гриде, основанный на различных 
механизмах рыночной экономики~[30], а также две модели планирования ресурсов, 
основанные на оценке их стоимости~[31, 32].
  
  Проведенный беглый обзор показывает, прежде всего, что проблема управления потоками 
заданий в системах вычислительных ресурсов находится в стадии интенсивного изучения. 
В~настоящее время нельзя говорить о том, что для ее решения сформирован в какой-то 
степени окончательный круг теоретических положений и практических методов. 
Существующие разработки отличаются большим многообразием используемых подходов и 
средств.
{\looseness=1

}
  
  В данной работе сделана попытка дать простое математическое описание распределенной 
системы ВР, которое, не отражая частных деталей, годилось бы для изучения общих 
вопросов управления потоками заданий в широком классе таких систем. Избранный способ 
моделирования продолжает подход, начатый в~[33].

\section{Моделирование процесса распределения потоков в~системе 
вычислительных ресурсов}

\subsection{Балансовое соотношение для потоков заданий}

  Рассмотрим модель распределенной системы ВР, составные элементы которой будем 
называть субъектами (или иногда, для разнообразия, участниками). Всего система содержит 
$N$~субъектов, пронумерованных от~1 до~$N$. Каждый субъект способен 
осуществлять, вообще говоря, троякую функцию:
  \begin{itemize}
  \item являться источником потоков заданий;
  \item выполнять задания на имеющихся у него ресурсах;
\item являться транзитным и коммутационным пунктом для перемещения заданий между 
субъек\-тами. 
\end{itemize}
Таким образом, в разрабатываемой модели каждый субъект может одновременно выступать 
в роли потребителя ресурсов, владельца ресурса и посредника.
  
  Субъекты взаимодействуют через коммуникационную сеть, которая не выделяется как 
самостоятельный элемент системы, но ее наличие будет подтверждено косвенно при 
дальнейшем описании.
  
  Условимся измерять объемы заданий (нагрузку) в некоторых условных единицах, 
физический смысл которых в данном случае не играет особой роли. (Можно, например, 
представлять себе, что единица объема задания измеряется временем его выполнения на 
стандартном процессоре.) Поток заданий, порождаемый субъектом~$i$, представляет собой 
случайный процесс~$Z_i(t)$, заданный, как и все функционирование системы, в дискретном 
времени, $t=0,1,\ldots$ Этот процесс определяет объем нагрузки, поступающей в систему 
извне на каждом такте времени через посредство субъекта~$i$.
  
  Помимо экзогенного потока заданий~$Z_i(t)$, субъект получает входные потоки от 
других участников. Объем заданий, который накапливается у субъекта~$i$ в момент~$t$, 
обозначается через~$X_i(t)$. Этим объемом заданий (будем называть его для краткости 
очередью в момент~$t$) субъект обязан распорядиться в момент~$t$, имея 
следующие возможности:
  \begin{itemize}
  \item выполнение заданий на собственном ресурсе;
  \item отправка заданий другим субъектам;
  \columnbreak
  \item оставление заданий в собственной очереди;
  \item уничтожение заданий (потери).
  \end{itemize}
  
  Чтобы описать реализацию указанных возможностей, определим вектор
  $$
  \alpha_i = \left( \alpha_{i0},\alpha_{i1},\ldots ,\alpha_{iN}\right)\,,
  $$
компоненты которого удовлетворяют условиям
$$
\sum\limits_{i=0}^N \alpha_{ij} =1\,,\quad 0\leq \alpha_{ij}\leq1\,,\enskip 0\leq i,j\leq N\,,
$$
и означают следующее:
  $\alpha_{i0}$~--- доля объема заданий, направляемая для выполнения на собственный 
ресурс; $\alpha_{ij}$, $i\not=j$~--- доля объема заданий, направляемая субъекту~$j$; 
$\alpha_{ii}$~--- доля объема заданий, остав\-ля\-емая в собственной очереди (часть этих 
заданий может быть уничтожена).
  
  Вектор $\alpha_i$ определяет способ, которым субъект~$i$ распоряжается тем объемом 
заданий, который у него уже имеется. В~частности, параметры $\alpha_{ij}$, $i\not= j$,
определяют объем \textit{заявки} на передачу части заданий на адрес субъекта~$j$. Однако 
субъект~$j$, в соответствии с обсуждавшимся выше предположением о самостоятельности 
поведения участников системы, имеет возможность отказаться от предложения. Для 
регулирования объема вновь поступающих заданий необходим дополнительный механизм. 
Он задается вектором
  $$
  \beta_i =\left ( \beta_{i1},\ldots , \beta_{iN}\right)\,,
  $$
компоненты которого подчиняются условиям $0\leq$\linebreak $\leq \beta_{ij}\leq 1$ и при 
$i\not=  j$ означают долю от пред\-ла\-га\-емо\-го субъектом~$j$ объема заданий, которую субъект~$i$ согласен 
принять. Параметру~$\beta_{ij}$ придадим смысл регулирования объема заданий, 
сознательно удаляемых из системы субъектом~$i$.
  
  Согласно сказанному, в каждый момент~$t$ при передаче заданий от субъекта~$i$ к 
субъекту~$j$ происходит следующего рода согласование. Субъект~$i$ делает заявку на 
передачу заданий в объеме $\alpha_{ij}X_i(t)$. Субъект~$j$ подтверждает согласие на 
передачу части этого объема, и фактически передаваемый объем заданий составляет 
$\alpha_{ij}\beta_{ji}X_i(t)$.
  
  Таким образом, стратегии участников системы описываются с помощью матриц~$\alpha$ 
и~$\beta$ с компонентами~$\alpha_{ij}$ и~$\beta_{ij}$, имеющих размерности 
соответственно $N\times (N+1)$ и $N\times N$. О~матрице~$\alpha$ будем говорить как о 
стратегии \textit{маршрутизации}, а матрицу~$\beta$ будем называть стратегией 
\textit{согласования}.
  
  Заметим, что стратегии поведения субъекта, описываемые с помощью матриц~$\alpha$ 
и~$\beta$, могут быть сколь угодно сложными, поскольку компоненты этих матриц могут 
зависеть в общем случае от момента времени~$t$ и даже от всей предыстории системы до 
этого момента.
  
  Прежде чем описать динамику потоков в системе, необходимо еще оговорить, что 
происходит с теми заданиями, которые были заявлены на передачу другим субъектам, но не 
были согласованы. В принципе, это вопрос описания, а не существа дела, и он может быть 
разрешен по-разному. Договоримся в данном случае, что все несогласованные задания 
автоматически отправляются для выполнения на собственный ресурс.
  
  Введем в рассмотрение матрицу~$\lambda$ размерностью $N\times N$ с компонентами
  $$
  \lambda_{ij} =\alpha_{ij}\beta_{ji}\,,\quad 1\leq i,j\leq N\,.
  $$
Объем заданий у субъекта~$i$ в момент $t+1$ складывается из экзогенного потока, а также из 
потоков, поступающих от всех участников системы, поэтому
  $$
  X_i(t+1)=Z_i(t+1)+\sum\limits_{j=1}^N \lambda_{ij}X_j(t)
  $$
или в матричной форме
\begin{equation}
X(t+1)=Z(t)+\lambda X(t)\,,
\label{e1konov}
\end{equation}
где $X(t)$ и $Z(t)$~--- векторы-строки с компонентами~$X_i(t)$ и~$Z_i(t)$.
  
  Объем заданий, направленный для выполнения на собственный ресурс, с учетом 
сделанной договоренности составляет для субъекта~$i$ величину
  \begin{equation*}
  Y_i(t) =\lambda_{i0}X_i(t)\,,
%  \label{e2konov}
  \end{equation*}
где 
$$
\lambda_{i0} =\sum\limits_{j=0}^N\alpha_{ij}-\sum\limits_{j\not= i}\lambda_{ij}=1-\alpha_{ii}-
\sum\limits_{j=1}^N \lambda_{ij}\,.
$$
  
  Потери заданий выражаются как
  \begin{equation}
  \alpha_{ii}(1-\beta_{ii})X_i(t)=\gamma_iX_i(t)\,.
  \label{e3konov}
  \end{equation}

  Величину $\gamma_i =\alpha_{ii}(1-\beta_{ii})$~--- долю удаляемых заданий~--- назовем 
\textit{коэффициентом потерь}.
  
  В последующих рассуждениях будем предполагать, что экзогенные потоки имеют не 
зависящие от времени средние $z_i = \mathrm{M}Z_i(t)$.
  
  Пусть элементы матрицы~$\lambda$ являются постоянными величинами и при этом для 
всех $i$ выполняются соотношения $\sum\limits_{j=1}^N \lambda_{ij}<1$. Тогда 
из~(\ref{e1konov}) следует, что существуют пределы 
$x_i=\lim\limits_{t\rightarrow\infty} \mathrm{M} X(t)$, которые определяются соотношением
  \begin{equation}
  x=z+x\lambda
  \label{e4konov}
  \end{equation}
или
$$
x=z(E-\lambda)^{-1}\,,
$$
где $x=(x_1,x_2,\ldots ,x_N)$ и $z=(z_1,z_2,\ldots ,z_N)$, а $E$~--- единичная матрица. 
Дальнейшие соотношения также будут относиться к предельным средним значениям 
переменных, участвующих в модели.

\subsection{Качество обслуживания}

  Качество обслуживания в системе вычислительных ресурсов определяется целым рядом 
факторов, которые зачастую плохо поддаются количественному описанию. Тем не менее 
одно из самых существенных предположений, которое делается в данной статье, 
заключается именно в том, что количественное описание качества обслуживания имеется.
  
  Говоря неформально о стремлении к качественному обслуживанию, можно было бы 
выразиться словами <<быстрее, дешевле, лучше>>. Что такое <<быстрее>> и 
<<дешевле>>~--- вполне понятно. Например, временной фактор связан со скоростью 
процессоров, наличием или отсутствием очередей невыполненных заданий в буферах, 
задержкой при транспортировке заданий по сети и~т.\,д. Главное в том, что временной 
фактор естественным образом выражается скалярно. Так же, как и стоимостный фактор. 
Конечно, подсчет и времени, и стоимости выполнения заданий может оказаться непростой 
задачей, но результатом ее решения будет количественное выражение. Сложнее может 
оказаться выразить числом, что значит <<лучше>>, поскольку речь может идти о плохо 
формализуемых критериях (например, об использовании более или менее подходящего 
программного обеспечения и~т.\,п.).
   
   Не расшифровывая, что понимается конкретно под словом <<лучше>>, предположим, 
что существует неотрицательный показатель качества субъекта в момент времени~$t$, 
который будем обозначать через~$q_i(t)$. При этом условимся, что значение показателя 
качества, равное нулю, характеризует некоторый идеальный уровень качества обслуживания, а 
увеличение числовой оценки качества обслуживания соответствует потере качества. Таким 
образом, чем больше значение показателя качества, тем хуже реальное качество 
обслуживания.
   
    
   Следующее предположение заключается в существовании неотрицательных 
\textit{функций качества}~$Q_i$,\linebreak которые будут использоваться при вычислении\linebreak 
показателей качества субъектов~$q_i(t)$ и которые характеризуют качество обслуживания 
на ресурсе субъекта~$i$. Функция~$Q_i$ имеет аргументом объем заданий, поступающий на 
ресурс~$i$. Характер зависимости от аргумента для каждой из функций каче-\linebreak\vspace*{-12pt}
\pagebreak

%\medskip

\begin{center} %fig1
\vspace*{-3pt}
\mbox{%
\epsfxsize=65.328mm
\epsfbox{kon-1.eps}
}
\vspace*{4pt}
\end{center}
\begin{center}
{{\figurename~1}\ \ \small{Функция оценки качества~$Q$ от объема заданий~$x$}}
\end{center}
\vspace*{12pt}


%\bigskip
\addtocounter{figure}{1}

   
\noindent
ства 
предполагается примерно таким, как у функции~$Q(x)$ на рис.~1. Эта функция 
зависит от трех параметров: $p$, $r$ и~$s$. Параметр $p>0$ характеризует тот уровень 
нагрузки, достижение и превышение которого влечет существенное ухудшение качества 
обслуживания. Параметр~$p$ будем условно называть \textit{емкостью}. Параметр $r\geq 0$ 
означает минимальную потерю качества, которая возможна при обслуживании и которая 
достигается при минимальной загрузке данного участника. Этот параметр будем условно 
называть \textit{интенсивностью (обслуживания)}. Параметр $s>0$ означает числовую 
оценку ситуации, в которой качественное обслуживание заданий практически отсутствует. 
Можно условно употребить выражение \textit{(максимальный) штраф} за плохое 
обслуживание. Пример аналитического выражения для функции~$Q(x)$:
   \begin{equation}
   Q(x) =\fr{2s \arctg p+\pi r+2(s-r)\arctg(x-p)}{\pi+2\arctg p}\,.
   \label{e5konov}
   \end{equation}
В этом примере $p$~--- точка перегиба функции~$Q(x)$, $Q(0) = r$,  
$\lim\limits_{x\rightarrow\infty} Q(x)=s$.
   
   При составлении соотношений для показателей качества будем исходить из следующих 
наводящих соображений. Располагая определенным объемом заданий, субъект обеспечивает 
их обслуживание, оставляя часть заданий на собственном ресурсе и отправляя остальные 
другим субъектам, либо уничтожая их. Эти действия приводят к разному качеству 
обслуживания для тех долей заданий, которые подвергаются тому или иному управлению. 
Например, передача заданий другому субъекту означает, что тот принимает полную 
ответственность за их дальнейшее обслуживание, которое будет происходить с тем уровнем 
качества, который обеспечивает принимающая сторона. Чтобы не усложнять дальнейшее 
описание, предположим, что в системе нет потерь ($\gamma_i=0$ для всех~$i$). Тогда 
распределение потоков заданий субъекта~$i$ задается набором 
$\lambda_{i0},\lambda_{i1},\ldots ,\lambda_{iN}$, и для показателя качества субъекта~$i$ 
запишем соотношение
   $$
   q_i=\lambda_{i0}Q_i+\sum\limits_{n=1}^N \lambda_{ij} q_j\,,
   $$
где $Q_i=Q_i(y_i)$, $y_i=\lambda_{i0}x_i$~--- функции качества,  имеющие, например, 
вид~(\ref{e5konov}), с индивидуальными для каждого субъекта параметрами $(p_i, r_i, s_i)$. 
В~мат\-рич\-ной форме имеем систему
$$
q=\kappa+q\lambda^{\mathrm{T}}
$$
или
$$
q=\kappa\left(E-\lambda^{\mathrm{T}}\right)^{-1}\,,
$$
где $q=(q_1,\ldots ,q_n)$; $\kappa=(\lambda_{10}Q_1,\ldots ,\lambda_{N0}Q_N)$; T~--- знак 
транспонирования.

  \medskip
  
  \noindent
  \textbf{Замечание.} В начале этого подраздела были выделены три составляющие 
качества обслуживания, которые можно определить как временн$\acute{\mbox{у}}$ю, денежную и третью 
составляющую, учитывающую специфические трудно формализуемые критерии. 
Определенные выше показатели качества, в том числе функции качества, можно трактовать 
как \textit{штраф за потерю качества}. Это предполагает, что упомянутая третья 
составляющая качества имеет не только количественное, но и, фактически, денежное 
выражение. Поскольку и временн$\acute{\mbox{а}}$я компонента, очевидно, легко может 
быть переведена на тот же язык, например с помощью \textit{штрафа за задержку}, то 
напрашивается естественная возможность все обсуждение качества в модели свести к 
стоимостным соотношениям. В~данном случае, однако, такая возможность не используется, 
а рассматривается некий компромиссный вариант. Стоимостная составляющая выступает 
самостоятельно в виде арендной платы (см.\ следующий подраздел). Что касается фактора, 
связанного с задержками, то он, хотя и не входит явным образом в модель, частично 
отражен, поскольку большие значения функции качества при высокой нагрузке могут 
интерпретироваться как \textit{штраф за задержку на ресурсе}.

\subsection{Стоимостные факторы}

  Предполагается, что каждый субъект~$i$ имеет набор ценовых параметров $(a_i,b_{ij})$, 
где $a_i$~--- стоимость обслуживания единицы объема заданий, полученных от других 
субъектов; $b_{ij}$~--- тариф за передачу единицы объема заданий от субъекта~$i$ к 
субъекту~$j$.
  
  Кроме того, предположим, что задана цена~$c$~--- штраф за потерю (уничтожение) 
единицы объема заданий~--- одинаковая для всей системы.
  
  Доходы  $f_i$ субъекта~$i$ представляют собой арендную плату, равную объему 
поставляемых ему заданий, умноженную на коэффициент~$a_i$. Расходы этого же субъекта 
складываются из платы~$g_i$ за дальнейшее обслуживание той части заданий, которая 
пересылается другим субъектам, и штрафа за потери~$h_i$. Первая из указанных величин, в 
свою очередь, составляется из арендной платы другим субъектам и оплаты транспортировки. 
Согласно~(\ref{e3konov}) и~(\ref{e4konov}) перечисленные компоненты равняются:
  \begin{align*}
  f_i & = a_i\sum\limits_{j=1}^N \lambda_{ij}x_j+a_iz_i=a_ix_i\,;\\
  g_i & = \sum\limits_{j=1}^N 
a_j\lambda_{ij}x_i+\sum\limits_{j=1}^Nb_{ij}\lambda_{ij}x_i\,;\\
  h_i& = c\gamma_i x_i\,.
  \end{align*}

  Общий денежный баланс субъекта~$i$ (превышение доходов над расходами) 
определяется как 
  $$
  d_i=f_i-g_i-h_i=\delta_ix_i\,,
  $$
где $\delta_i=a_i-\sum\limits_{j=1}^N (a_j+b_{ij})\lambda_{ij}-c\gamma_i$.

\subsection{Целевые функции и~алгоритм распределения потоков}
  
  Целевые функции участников системы должны учитывать как стоимостные аспекты, так и 
стремление оптимизировать качество выполнения заданий.
  
  Чтобы соединить в одном критерии эти два обычно противоречивых фактора, введем два 
дополнительных пороговых параметра: $\overline{d}_i$ и $\overline{q}_i$~--- бюджет и 
требуемый уровень качества субъекта~$i$. Субъект стремится действовать так, чтобы 
расходы не выходили за рамки бюджета, а качество соответствовало заданному уровню. 
Нарушение ограничений по одному из критериев заставляет субъект заботиться именно об 
этом показателе. Определим целевую функцию в этих случаях как
  $$
  w_i = 
  \begin{cases}
  -d_i\,, & \mbox{если}\ -d_i>\overline{d}_i,\enskip q_i\leq \overline{q}_i\,;\\
  q_i\,, & \mbox{если}\ -d_i\leq \overline{d}_i\,,\enskip q_i>\overline{q}_i\,.
  \end{cases}
  $$
Такое определение соответствует стремлению потребителя: 
\begin{enumerate}[(1)]
\item минимизировать расходы, не 
заботясь о качестве, если он выходит за рамки бюджета, а качество при этом находится в 
допустимых пределах; 
\item оптимизировать только качество обслуживания в ситуации, когда 
он укладывается в бюджет, а требования по качеству нарушены.
\end{enumerate}
  
  В ситуации, когда оба пороговых соотношения нарушены или, наоборот, оба выполнены, 
можно воспользоваться какой-либо сверкой критериев. Например, можно использовать 
механизм штрафов, мотивируя стремление участников улучшать качество обслуживания. 
Определим за ухудшение качества обслуживания штраф, размер которого прогрессивно 
зависит от устанавливаемых тарифов за услуги. Положим
$$
  w_i =-d_i-(a_i)^{1+a}q_i\,,
  $$
  если $-d_i\leq\overline{d}_i$, $q_i\leq\overline{q}_i$ или
$-d_i >\overline{d}_i$, $q_i>\overline{q}_i$, где $a\geq 0$~--- фиксированный параметр.

  Управлениями в системе будем считать определенные в п.~3.1 матрицы 
маршрутизации~$\alpha$ и согласования~$\beta$, которые задают распределение потоков, а 
также определенные в п.~3.3 стоимостные характеристики~$a_i$. (Матрица сетевых 
тарифов~$b_{ij}$ считается фиксированной.) Исходное представление о 
децентрализованном и независимом поведении участников означает, что субъект~$i$ 
выбирает собственный вектор маршрутизации~$\alpha_i$, согласования~$\beta_i$, а также 
тариф~$a_i$.
  
  Рассмотрим такой вариант поведения участников, при котором они синхронно и 
независимо\linebreak изменя\-ют <<свои>> управления согласно алгоритму проекции градиента, 
стремясь минимизировать определенные выше целевые функции~$w_i$. Формальную запись 
алгоритма, однотипного для всех субъектов и для всех управлений, приведем для 
маршрутизации субъекта~$i$. Пусть 
$\alpha_i^{(n)}=$\linebreak $=(\alpha_{i0}^{(n)},\alpha_{i1}^{(n)},\ldots ,\alpha_{iN}^{(n)})$~--- значение 
вектора~$\alpha_i$ для\linebreak
 $n$-й итерации алгоритма, $n=0, 1,\ldots ,$ и пусть $\nabla 
w_i^{(n)}$~--- градиент функции~$w_i$ в точке~$\alpha_i^{(n)}$. Рекуррентное 
соотношение для последовательных значений вектора~$\alpha_i^{(n)}$ имеет следующий 
вид:
  \begin{equation}
  \alpha_i^{(n+1)} =\Pi \left( \alpha_k^{(n)}-a^{(n)}\nabla w_i^{(n)}\right)\,,
  \label{e6konov}
  \end{equation}
где $\Pi$ означает оператор проектирования на единичный симплекс размерности $N+1$, а 
положительная последовательность~$a^{(n)}$ удовлетворяет обычным для такого рода 
алгоритмов условиям $a^{(n)}\rightarrow 0$, $\sum\limits_n a^{(n)}=\infty$.

\subsection{Пример}
  
  Рассмотрим систему, состоящую из трех субъектов, параметры которых указаны в 
табл.~1.
  
\begin{table*}\small
\begin{center}
\Caption{Параметры системы
\label{t1konov}}
\vspace*{2ex}

\begin{tabular}{|c|c|c|c|c|c|}
\hline
\multicolumn{1}{|c|}{\raisebox{-6pt}[0pt][0pt]{Субъект, $i$}} &
\multicolumn{3}{c|}{Параметры функции качества $Q_i$}&\multicolumn{1}{|c|}{\raisebox{-6pt}[0pt][0pt]{Поток, 
$z_i$}}&\multicolumn{1}{|c|}{\raisebox{-6pt}[0pt][0pt]{Сетевой тариф, $b_{ij}$}}\\
\cline{2-4}
&$p_i$&$r_i$&$s_i$&&\\
\hline
1&20&10&100&100&0; 100; 1\\
2&120&1&20&25&100; 0; 1\\
3&0,01&1&1000&0&1; 1; 0\\
\hline
\end{tabular}
\end{center}
\end{table*}

\begin{table*}\small
\begin{center}
\Caption{Показатели системы до и после оптимизации
\label{t2konov}}
\vspace*{2ex}

\begin{tabular}{|c|c|c|c|c|c|c|}
\hline
Субъект, $i$&\multicolumn{2}{c|}{Маршрутизация, $\alpha_i$ }&\multicolumn{2}{c|}{Очередь, $x_i$}  &\multicolumn{2}{c|}{Качество,  $q_i$}\\
\cline{2-7}
&в начале&в конце&в начале&в конце&в начале&в конце\\
\hline
1&0,25&
\tabcolsep=0pt\begin{tabular}{c}0\\ 0\\ 0,8031\\ 0,1969\end{tabular}& 
225&101\hphantom{9}&295&19,69\\
\hline
2&0,25&0, 0, 0, 1&150&97&271&\hphantom{9}1,22\\
\hline
3&0,25&0, 1, 0, 0&125&80&985&13,4\hphantom{9}\\
\hline
\end{tabular}
\end{center}
\end{table*}


  Функции качества для всех субъектов имеют вид~(\ref{e5konov}). Требования по уровню 
качества для участников 1, 2, 3 равняются соответственно 20, 1,2 и~20.
  
  Приведенные значения параметров говорят о следующем. Субъекты~1 и~2 выполняют в 
системе одновременно роль источников потоков заданий и ресурсов. При этом субъект~1 
создает основную нагрузку, но располагает значительно менее емким, производительным и 
качественным ресурсом, чем субъект~2. Субъект~3 не порождает потока заданий, но он 
также и не располагает сколько-нибудь значимым ресурсом. Потенциальная роль этого 
субъекта определяется матрицей сетевых тарифов, из которой следует, что данный участник 
обладает значительно более экономными возможностями общения с субъектами~1 и~2, чем 
те сами между собой.
  
  Начальная маршрутизация всех субъектов\linebreak (векторы~$\alpha_i$) равномерная, а все 
векторы согласования~$\beta_i$ имеют компоненты, равные~1. Ценовые факто\-ры в этом 
примере не рассматриваются. Основные показатели, которые дает это очень неэффективное 
распределение потоков, содержатся в табл.~2 (округленно).
  

  После оптимизации с помощью алгоритма~(\ref{e6konov}) распределение потоков 
изменилось. Субъект~1 стал использовать участника~3 в качестве транзитного пункта для 
передачи большей части заданий на более мощный ресурс~2. На собственный ресурс 
направляется около 20\% потока~--- больший объем нарушил бы требуемый уровень 
качества~20. Субъект~2 перешел полностью на самообслуживание и стал посылать весь 
поток на собственный ресурс. Характерно, что из всех компонент матрицы 
согласования~$\beta$ (в табл.~2 она не отражена) изменился только 
коэффициент~$\beta_{23}$, который стал равным приблизительно~0,9. Это вызвано 
необходимостью для субъекта~2 ограничить поток, получаемый от посредника~3, и тем 
самым обеспечить заданный уровень качества~1,2.

\section{Заключение}
  
  Рассмотрена проблема анализа и оптимизации распределения потоков заданий и 
ценообразования в системах коллективного использования распределенных вычислительных 
ресурсов. Проведенный литературный обзор показал, что, хотя существуют разнообразные 
подходы и методы решения проблемы, она пока далека от окончательного решения.
  
  Предложена математическая модель системы вычислительных ресурсов, которая 
представляет собой интегральное описание потоков заданий в виде динамических 
балансовых соотношений. Необычность модели в том, что участники системы, вообще 
говоря, одновременно играют роль источников заданий, владельцев ресурсов и посредников 
в передаче потоков. Благодаря этому субъекты сис\-те\-мы имеют одинаковое, причем 
математически простое, описание в плане стратегии распределения потоков заданий и 
выделения ресурсов. В~то же время в сравнительно простую модель удалось включить 
целый ряд факторов, имеющих принципиальное значение для любой системы 
вычислительных ресурсов: планирование выбора ресурсов для заданий, степень готовности 
ресурса обслуживать задания, качество обслуживания, учет потерь, затраты и 
ценообразование, потери в системе и~т.\,д. При этом построение модели является, в 
сущ\-ности, многовариантным. Перечисленные факторы в зависимости от потребностей той 
или иной сис\-те\-мы, той или иной задачи могут быть полностью или частично отражены в 
модели или, наоборот, устранены из нее. В~этом смысле можно говорить о том, что 
предложен способ моделирования системы вычислительных ресурсов.
  
  Изложенный подход к моделированию не привязан к конкретной системе 
вычислительных ресурсов, поэтому его использование видится прежде всего в изучении 
вопросов, связанных с по\-стро\-ени\-ем таких систем вообще. При этом надо еще раз 
подчеркнуть, что принципиальным соображением в работе было представление об 
участниках системы как о независимо действующих субъектах, преследующих 
индивидуальные цели. С~учетом последнего замечания к числу важных вопросов, 
разрешение которых можно надеяться получить с помощью предложенной методологии, 
относятся, например, следующие.
  
  Определение показателей, которых может достичь система ресурсов, участники которой 
действуют, исходя из эгоистических интересов. Определение оптимальных стратегий 
поведения\linebreak участников.
  
  Установление механизма ценообразования в сис\-те\-мах ресурсов.
  
  Определение роли и разумного числа посредников в распределении ресурсов.
  
  Эти и другие вопросы являются направлениями дальнейших исследований.


{\small\frenchspacing
{%\baselineskip=10.8pt
\addcontentsline{toc}{section}{Литература}
\begin{thebibliography}{99}
  \bibitem{1konov}
  Информатика: состояние, проблемы, перспективы~/ Под ред. И.\,А.~Соколова.~--- М.: 
ИПИ РАН, 2009. 46~с. ISBN-978-5-902030-69-0.
  
  \bibitem{2konov}
  \Au{Демичев А.\,П., Ильин В.\,А., Крюков А.\,П.}
  Введение в грид-технологии: Препринт.~--- М.: НИИЯФ МГУ, 2007.  87~с.
  
  \bibitem{3konov}
  {\sf http://www.systematic.ru/publikatsii/sx/art/310033/ po/309844/cp/1/br/309438/discart/310033.html}.
  
  \bibitem{5konov}
  \Au{Коновалов М.\,Г., Малашенко Ю.\,Е., Назарова~И.\,А.}
  Модели и методы управления заданиями в системах распределенных вычислительных 
ресурсов: Препринт.~--- М.: ВЦ РАН, 2009.  110~с.

  \bibitem{4konov} %5
  \Au{Xhafa F., Abraham~A.}
  Computational models and heuristic methods for Grid scheduling problems~// Future Generation 
Comput. Syst., 2010. Vol.~26. P.~608--621.
  
  
  \bibitem{6konov}
  \Au{Cho S., Lee M., In~J., Kim~B., Choi~E.}
  Policy based scheduling for resource allocation on grid~/ Eds. K.\,C.~Chang \textit{et al}.~// 
APWeb/WAIM 2007 Ws, LNCS, 2007. Vol.~4537. P.~229--234.
  
  \bibitem{7konov}
  \Au{Топорков В.\,В.}
Потоковые и жадные алгоритмы согласованного выделения ресурсов в распределенных системах~// Известия РАН. 
Теория и системы управ\-ле\-ния, 2007. No.\,2. P.~109--119.

  \bibitem{8konov}
  \Au{Vanderster D.\,C., Dimopoulos N.\,J., Sobie~R.\,J.}
  Metascheduling multiple resource types using the MMKP grid~// 7th IEEE/ACM Conference 
(International) on Grid Computing Proceedings, 2006. P.~231--237.
  
  \bibitem{9konov}
  \Au{Душин Ю.\,А.}
  Модель оценки стоимости гетерогенных ресурсов в Грид~// Системы и средства 
информатики. Спец. вып. Математические модели в информационных технологиях.~--- М.: 
ИПИ РАН, 2006. С.~163--172.
  
  \bibitem{10konov}
  \Au{Gamst M.}
  Greedy and metaheuristics for the offline scheduling problem in grid computing~// DTU 
Management Engeneering.~--- Technical University of Danemark, 2010. {\sf 
http://www.man.dtu.dk/\linebreak upload/institutter/ipl/publ/publikationer\%202010/\linebreak rapport2.2010.pdf}.
  
  \bibitem{11konov}
  \Au{Агаларов~Я.\,М.}
  Динамическая стратегия распределения вычислительных ресурсов локального узла 
GRID~// Системы и средства информатики. Вып.~17.~--- М.: ИПИ РАН, 2007. С.~17--29.
  
  \bibitem{12konov}
  \Au{Slegers J., Mitrani~I., Thomas~N.}
  Optimal dynamic server allocation in systems with on/off sources~/ Ed.\ K.~Wolter~// EPEW 
2007, LNCS, 2007. Vol.~4748. P.~186--199.
  
  \bibitem{13konov}
  \Au{Farzi S.}
  Efficient job scheduling in grid computing with modified artificial fish swarm algorithm~// 
Int. J. Comput. Theory Eng., 2009. Vol.~1. No.~1. 
{\sf http://www.ijcte.org/papers/003.pdf}.
  
  \bibitem{14konov}
  \Au{Mathiyalagan P., Dhepthie~U.\,R., Sivanandam~S.\,N.}
  Grid scheduling using enhanced PSQ algorithm~// Int.\ J. Comput. Sci. Eng., 
2010. Vol.~2. No.~2. P.~140--145.
  
  \bibitem{15konov}
  \Au{Kamalam G.\,K., Muralibhaskaran~V.}
  A~new heuristic approach: min-mean algorithm for scheduling meta-tasks on heterogenous 
computing systems~// Int. J. Comput. Sci. Network Security, 2010. Vol.~10. No.~1.
  
  \bibitem{16konov}
  \Au{Li B., Zhao~D.}
  Online algorithms for single machine schedulers to support advance reservations from grid 
jobs~/ Eds.\ R.~Perrott, B.~Chapman, J.~Subhlok, \textit{et al}.~// HPCC 2007, LNCS, 2007. Vol.~4782. P.~239--248.
  
  \bibitem{17konov}
  \Au{Nou R., Kounev~S., Torres~J.}
  Building online performance models of grid middleware with fine-grained load-balancing: A 
Globus Toolkit case study~/ Ed.\ K.~Wolter~// EPEW 2007, LNCS, 2007. Vol.~4748. P.~125--140.
  
  \bibitem{18konov}
  \Au{Yagoubi B., Slimani~Y.}
  Dynamic load balancing strategy for grid computing~// Trans. Eng. Comput. 
Technol., 2006. Vol.~13. P.~260--265.
  
  \bibitem{19konov}
  \Au{Saravanakumar E., Gomathy~P.}
   A~novel load balancing algorithm for computational grid~// Int.\ J. Comput.
Intelligence, 2010. Vol.~1. Issue~1. P.~20--26.
  
  \bibitem{20konov}
  \Au{Berten V., Gaujal B.}
  Brokering strategies in computational grids using stochastic prediction models~// Parallel 
Comput., 2007. Vol.~33. P.~238--249. {\sf www.sciencedirect.com}.
  
  \bibitem{21konov}
  \Au{Yu K.-M., Luo Z.-J., Chou~C.-H., Chen~C.-K., Zhou~J.}
  A~fuzzy neural network based scheduling algorithm for job assignment on computational 
grids~/ Eds. T.~Enokido, L.~Barolli, M.~Takizawa~// NBiS 2007, LNCS, 2007. Vol.~4658. 
P.~533--542.
  
  \bibitem{22konov}
  \Au{Ishii R.\,P., De Mello~R.\,F., Yang~L.\,T.}
  A~complex network-based approach for job scheduling in grid environments~/ Eds.\ R.~Perrott,
  B.~Chapman, J.~Subhlok, 
\textit{et al}.~// HPCC 2007, LNCS, 2007. Vol.~4782. P.~204--215.
  
  \bibitem{23konov}
  \Au{Al-Khateeb A., Abdullah~R., Rashid~N.\,A.}
  Job type approach for deciding job scheduling in grid computing systems~// J. Comput. 
Sci., 2009. Vol.~5. No.~10. P.~745--750.
  {\sf http://www.scipub.org/fulltext/jcs/jcs510745-750.pdf}.
  
  \bibitem{24konov}
  \Au{Shah S.\,C., Chahdary~S.\,H., Bashir~A.\,K., Park~M.\,S.}
  A~centralized location-based job scheduling algorithm for interdependent jobs in mobile ad hoc 
computational grids~// J.~Appl. Sci., 2010. Vol.~10. No.\,3. P.~174--181.
  
  \bibitem{25konov}
  \Au{Wei X., Ding~Z., Xing~S., Yuan~Y.}
  VJM: A~novel grid resource co-allocation model for parallel jobs~// Int. J. Grid  
Distributed Comput., 2009. Vol.~1. No.\,2.
  
  \bibitem{26konov}
  \Au{Ali G., Shaikh N.\,A., Shaikh~Z.\,A.}
  Integration of grid and agent systems to perform parallel computations in a heterogeneous and 
distributed environment~// Aust. J. Basic Appl. Sci., 2009. Vol.~3. No.\,4. 
P.~3857--3863.
  
  \bibitem{27konov}
  \Au{Vazquez C., Huedo~E.\,S. Montero~R.\,S., Llorente~I.\,M.}
  Federation of TeraGrid, EGEE and OSG infrastructures through a metascheduler. Preprint 
submitted to Future Generation Computer Systems, 2010. 
  {\sf http://dsa-research.org/doku.php?id=publications:grid:utility}.
  
  \bibitem{28konov}
  \Au{Ranjan R., Harwood A., Buyya~R.}
  SLA-based cooperative superscheduling algorithms for computational grids~// 8th IEEE  
Conference (International) on Cluster Computing (Cluster 2006) Proceedings~// IEEE Computer 
Society Press, 2006. abs/cs/0605057.
  
  \bibitem{29konov}
  \Au{Li J., Sim K.\,M., Yahyapour~R.}
  Negotiation strategies considering opportunity functions for grid scheduling~/ Eds.\ 
  A.-M.~Kermarrec, L.~Boug$\acute{\mbox{e}}$, T.~Priol~// Euro-Par, 2007, LNCS, 2007. 
Vol.~4641. P.~447--456.
  
  \bibitem{30konov}
  \Au{Vanmechelen K., Broeckhove~J.}
  A~comparative analysis of single-unit vickrey auctions and commodity markets for realizing 
grid economies with dynamic pricing~/ Eds.\ D.\,J.~Veit, J.~Altmann~// GECON 2007, LNCS, 
2007. Vol.~4685. P.~98--111.
  
  \bibitem{31konov}
  \Au{Krasnotcshekov V., Vakhitov~A.}
  Adaptive scheduling and resource assessment in grid~/ Ed.\ V.~Malyshkin~// PaCT 2007, LNCS, 
2007. Vol.~4671. P.~240--244.
  
  \bibitem{32konov}
  \Au{Агаларов Я.\,М.}
  Функция стоимости ресурсов в экономической модели грид~// Информатика и её 
применения, 2008. Т.~2. Вып.~3. С.~27--34.

\label{end\stat}
  
  \bibitem{33konov}
  \Au{Коновалов М.\,Г., Душин~Ю.\,А., Малашенко~Ю.\,Е., Шоргин~С.\,Я.}
  Модель взаимодействия потребителей с удаленными вычислительными ресурсами через 
посредников~// Системы и средства информатики. Вып.~19.~--- М.: Наука,1989. С.~5--33.
 \end{thebibliography}
}
}

\end{multicols}
 
 
 
    %1
\def\stat{krivenko}

\def\tit{ОБУЧАЕМАЯ КЛАССИФИКАЦИЯ ДАННЫХ\\ С~УЧЕТОМ АНАЛИЗА ГЛАВНЫХ 
КОМПОНЕНТ}

\def\titkol{Обучаемая классификация данных с~учетом анализа главных 
компонент}

\def\aut{М.\,П.~Кривенко$^1$}

\def\autkol{М.\,П.~Кривенко}

\titel{\tit}{\aut}{\autkol}{\titkol}

\index{Кривенко М.\,П.}
\index{Krivenko M.\,P.}




%{\renewcommand{\thefootnote}{\fnsymbol{footnote}} \footnotetext[1]
%{Работа выполнена при финансовой поддержке Российского научного фонда (проект 18-11-00155).}}


\renewcommand{\thefootnote}{\arabic{footnote}}
\footnotetext[1]{Институт проблем информатики Федерального исследовательского центра 
<<Информатика и~управление>>
Российской академии наук, \mbox{mkrivenko@ipiran.ru}}

\vspace*{8pt}





\Abst{Рассматриваются вопросы обучаемой классификации с~учетом результатов анализа 
главных компонент (PCA~--- Principal Component 
Analysis). Построение байесовского классификатора становится возможным 
после представления ковариаций через параметры вероятностной модели PCA. Выделен 
случай сингулярных распределений данных, для него оценивание параметров модели 
предлагается проводить при ограничениях на собственные значения ковариационных 
матриц. Исследуется качество классификации с~учетом реальной размерности данных. 
Продемонстрировано, что при ее правильном задании классификатор обладает наименьшими 
вероятностями ошибки. Превышение наилучшего значения размерности обычно ухудшает 
качество классификации в~меньшей степени, чем его занижение. Смесь вероятностных 
анализаторов главных компонент позволяет моделировать объемные данные с~помощью 
относительно небольшого числа свободных параметров. Число свободных параметров 
можно контролировать с~помощью выбора латентной размерности данных.}

\KW{анализ главных компонент; смеси нормальных распределений; EM-ал\-го\-ритм; 
обучаемая классификация}

\DOI{10.14357/19922264180308}
  
%\vspace*{4pt}


\vskip 10pt plus 9pt minus 6pt

\thispagestyle{headings}

\begin{multicols}{2}

\label{st\stat}

\section{Введение}

     Один из способов снижения размерности данных заключается 
в~применении анализа главных компонент (PCA). 
Популярность PCA определяется рядом свойств, важнейшим из 
которых является его оптимальность при сжатии множества векторов высокой 
размерности в~множество векторов более низкой размерности, а~затем их 
восстановления.
     
     Использовать PCA в~задаче обучаемой классификации данных можно 
двояко. Во-пер\-вых, без-\linebreak от\-но\-си\-тель\-но к~сложной структуре 
результатов\linebreak 
наблюдений, подразумевающей наличие классов данных. В~этом случае 
данные без уточнения их статистической модели сжимаются, а~затем 
подвергаются анализу. Более сложным оказывается второй подход, когда PCA 
проводится индивидуально для каждого класса в~отдельности. В~связи с~его 
применением возникают два вопроса:
     \begin{enumerate}[1.]
\item Как проводить классификацию данных, объединяя результаты PCA для 
каждого класса в~отдельности?
\item Может ли подобное сжатие данных стать источником повышения качества 
классификации данных?
\end{enumerate}

     Задача классификации данных становится традиционной после перехода 
к~PPCA$(k)$~--- вероятностной модели PCA (PPCA~--- probabilistic PCA) для 
сниженной размерности~$k$:
     $$
     \mathbf{y}=\mathbf{Wx}+\mathbf{a}+\boldsymbol{\varepsilon}\,,
     $$
где $\mathbf{y}$~--- $d$-мер\-ная наблюдаемая переменная, $\mathbf{y}\sim$\linebreak
$\sim 
N(\mathbf{a},\mathbf{C}(k))$; $\mathbf{W}$~---  $(d\times k)$-мат\-ри\-ца 
преобразования; $\mathbf{x}$~--- $k$-мер\-ная латентная переменная, 
$\mathbf{x}\sim$\linebreak $\sim N(\mathbf{0},\mathbf{I})$; $\boldsymbol{\varepsilon$}~--- 
$d$-мер\-ная переменная, 
$\boldsymbol{\varepsilon}\sim N(\mathbf{0},\sigma^2\mathbf{I})$; 
$\mathbf{C}(k)\hm=\mathbf{W}\mathbf{W}^{\mathrm{T}}\hm 
+\sigma^2\mathbf{I}$. Здесь $d$~--- исходная размерность данных; $k$~--- 
сниженная размерность сжатых данных; $\sigma^2$ и~$\mathbf{W}$ суть 
параметры модели (данные принимаются центрированными).

     Пусть задана $(d\times N)$-мат\-ри\-ца  
<<при\-знак--объ\-ект>>~$\mathbf{Y}$ и~найдена выборочная ковариационная 
мат\-ри\-ца~$\mathbf{S}$. Справедливо спектральное разложение вида 
$\mathbf{S}\hm= \mathbf{UVU}^{\mathrm{T}}$, где $\mathbf{V}$~--- 
диагональная матрица, ее элементы $v_1,\ldots , v_d$ суть собственные 
значения матрицы~$\mathbf{S}$, а~$\mathbf{U}$ является ортогональной 
матрицей, столбцы которой~--- ортонормированные собственные векторы  
мат\-ри\-цы~$\mathbf{S}$. Тогда согласно~[1] могут быть найдены оценки 
параметров модели:
     $$
     \hat{\sigma}^2=\fr{1}{d-k}\sum\limits^d_{i=k+1} v_i\,;\enskip 
\hat{\mathbf{W}}= \mathbf{U}_k\left( \mathbf{V}_k-
\hat{\sigma}^2\mathbf{I}\right)^{1/2}\,,
     $$
где столбцы $(d\times k)$-мат\-ри\-цы~$\mathbf{U}_k$ суть оси 
первых~$k$~главных компонент; $\mathbf{V}_k$~--- диагональная  
$(k\times k)$-мат\-ри\-ца соответствующих дисперсий. После этого можно 
рассматривать случайную нормально распределенную величину 
$\mathbf{y}\sim N(\mathbf{a},\hat{\mathbf{C}}(k))$,\linebreak где
$$
\hat{\mathbf{C}}(k) 
=\hat{\mathbf{W}}\hat{\mathbf{W}}^{\mathrm{T}}+\hat{\sigma}^2\mathbf{I}\,,
$$
и становится возможным построение байесовского классификатора. 
Заметим, что
для различных значений~$k$ ковариационная матрица 
$\hat{\mathbf{C}}(k)\not=\mathbf{S}$, кроме случаев $k\hm= d\hm- 1,d$.

\section{Смесь вероятностных моделей анализа главных компонент}

     Анализ главных компонент 
     определяет только линейную проекцию данных, по этой причине 
область его применения несколько ограничена. Это, естественно, мотивировало 
различные разработки нелинейного PCA. 

Связь вероятностной модели со 
стандартным PCA открывает заманчивую перспективу моделировать сложные 
структуры данных с~по\-мощью комбинации локальных подмоделей PCA 
и~реализации механизма смеси вероятностных анализаторов главных 
компонент. 

Этот подход позволяет определять все параметры модели путем 
максимизации правдоподобия, в~ходе которого автоматически происходит 
раз\-би\-ение данных и~определение соответствующих главных осей. Логарифм 
правдоподобия для такой модели смеси есть
     $$
     L=\sum\limits^N_{n=1} \ln \left\{ p\left( \mathbf{y}_n\right)\right\} 
=\sum\limits^N_{n=1}\ln \left\{ \sum\limits^M_{j=1} \pi_i p(\mathbf{y}_n\vert 
j)\right\}\,,
     $$
где $p(\mathbf{y}_n\vert j)$ отвечает элементарной PPCA-мо\-де\-ли;  
$\pi_j$~--- соответствующий вес элемента смеси с~$\pi_j\hm\geq 0$ 
и~$\sum\nolimits^M_{j=1} \pi_j\hm=1$. Заметим, что с~каждым $j$-м элементом 
смеси связаны свои параметры~$\mathbf{a}_j$, $\mathbf{W}_j$ и~$\sigma_j^2$.

     При этом генерирующая модель для смеси требует случайного выбора 
элемента в~соответствии с~пропорциями~$\pi_j$, после чего формирование 
наблюдений для~$\mathbf{x}$ и~$\boldsymbol{\varepsilon}$ происходит согласно модели 
PPCA$(k)$  с~соответствующими параметрами. Кроме того, для некоторой 
точки~$\mathbf{y}$ теперь имеется апостериорное распределение, связанное 
с~каждым латентным пространством.
     
     Можно разработать итеративный EM (expectation-maximization)
     алгоритм для оценивания всех 
параметров модели $\pi_j$, $\mathbf{a}_j$, $\mathbf{W}_j$ и~$\sigma_j^2$. 
Если\linebreak
\vspace*{-12pt}

\columnbreak

\noindent
$q_{nj}\hm= p(j\vert \mathbf{y}_n)$~--- вероятность 
принадлежности~$\mathbf{y}_n$ к~$j$-му элементу смеси и
     $$
     q_{nj}= \fr{\pi_j p(\mathbf{y}_n\vert j)}{p(\mathbf{y}_n)}\,,
     $$
     то согласно приложению~C~[1] обновления для параметров принимают 
обычный вид для смеси нормальных распределений. Более того, 
в~приложении~C~[1] также показано: комбинация E-\linebreak и~M-ша\-гов приводит 
к~интуитивно ясному результату, что оси~$\mathbf{W}_j$ и~дисперсии 
шума~$\sigma_j^2$ определяются из взвешенной ковариационной матрицы
     $$
     \mathbf{S}_j= \fr{1}{\tilde{\pi}_j N} \sum\limits^N_{n=1} q_{nj} \left( 
\mathbf{y}_n -\hat{\mathbf{a}}_j\right) \left( \mathbf{y}_n-
\hat{\mathbf{a}}_j\right)^{\mathrm{T}}
     $$
с помощью обычной факторизации так же, как и~для элементарного PPCA. 
Однако, как отмечено в~разд.~3.4 и~приложении~A.5~[1], для больших 
значений размерности данных~$d$ могут быть получены вычислительные 
преимущества, если оценки~$\mathbf{W}_j$ и~$\sigma^2_j$ обновляются 
итеративно в~соответствии со схемой EM-ал\-го\-ритма.

     До сих пор предполагалось, что $\vert \mathbf{S}_j\vert\not= 0$. Иная 
ситуация с~наличием сингулярных распределений может возникать при малых 
выборках, когда их объем не превосходит размерности выборочного 
пространства, или при применении EM-ал\-го\-рит\-ма оценивания параметров, 
характеризующих смесь нормальных распределений. Если это так, то на 
помощь может прийти подход и~результаты из разд.~2.2.2~[2], заключающиеся 
в~использовании уточненной (невырожденной) модели многомерного 
нормального распределения, для которой введены ограничения на множество 
возможных значений ковариационной матрицы.
     
     Пусть плотность распределения смеси нормальных распределений есть
     $$
     f(\mathbf{u}) =\sum\limits^M_{j=1} \pi_j \varphi\left( \mathbf{u}, 
\mathbf{a}_j, \mathbf{C}_j\right)\,,
     $$ 
где $\varphi(\mathbf{u},\mathbf{a}_j,\mathbf{C}_j)$~--- плотность нормального 
рас\-пределения со средним~$\mathbf{a}_j$ и~ковариационной\linebreak 
мат\-ри\-цей~$\mathbf{C}_j$. При этом все собственные 
значения~$v_i(\mathbf{C}_j)$ ковариационных матриц~$\mathbf{C}_j$ 
ограничены снизу некоторой положительной константой~$v_0$, т.\,е.\ 
$v_i(\mathbf{C}_j)\hm\geq v_0\hm>0$, $i\hm=1,\ldots ,d$. Доказано, что при этих 
условиях на $t$-м шаге итерации EM-ал\-го\-рит\-ма максимум функции 
$\sum\nolimits^M_{j=1} \sum\nolimits^N_{n=1} q_{ij}^{(t)} \ln \varphi 
(\mathbf{y}_j, \mathbf{a}_j, \mathbf{C}_j)$ достигается при значениях 
параметров, которые для каждого значения допустимого~$j$ последовательно 
находятся следующим образом (реализация М-шага):
\begin{enumerate}[(1)]
\item вычислить 
$$
\mathbf{a}_j^{(t+1)}= \fr{\sum\nolimits^N_{n=1} q_{nj}^{(t)} \mathbf{y}_n} 
{\sum\nolimits^N_{n=1} q_{nj}^{(t)}}\,;
     $$
\item вычислить 
$$
\hspace*{-4.69887pt}\tilde{\mathbf{C}}_j=
\fr{\sum\nolimits^N_{n=1}\! q_{nj}^{(t)} \left( 
\mathbf{y}_n -\mathbf{a}_j^{(t+1)}\right) \left( \mathbf{y}_n-
\mathbf{a}_j^{(t+1)}\right)^{\mathrm{T}}} {\sum\nolimits^N_{n=1} 
q_{nj}^{(t)}}\,;
     $$
\item найти матрицы $\tilde{\mathbf{U}}_j$ и~$\tilde{\mathbf{V}}_j$, которые 
задают спектральное разложение матрицы~$\tilde{\mathbf{C}}_j$;
\item определить элементы~$v_{jl}^{(t+1)}$ диагональной 
мат\-ри\-цы~$\mathbf{V}_j^{(t+1)}$ через элементы~$\tilde{v}_{jl}$ 
мат\-ри\-цы~$\tilde{\mathbf{V}}_j$ по формулам $v_{jl}^{(t+1)}\hm= \max 
\left\{ \tilde{v}_{jl}, v_0\right\}$, $l\hm=1, \ldots , d$;
\item вычислить $\mathbf{C}_j^{(t+1)} \hm= \tilde{\mathbf{U}}_j 
\mathbf{V}_j^{(t+1)} \left( \tilde{\mathbf{U}}_j\right)^{\mathrm{T}}$.
     \end{enumerate}
     
     Заметим, что в~рассматриваемом случае пространство параметров 
ограничено и~максимум функции правдоподобия может лежать на границе. 
В~силу этого полученные оценки не подпадают под обычные условия 
о~сходимости EM-ал\-го\-рит\-ма (см., например,~[3]). Требуемые результаты 
были получены в~[4].
     
     Задание ограничения снизу на собственные значение ковариационных 
матриц необходимо для предотвращения появления недопустимо больших 
(малых) значений функции правдоподобия. При этом возникает необходимость 
выбора этого ограничения~$v_0$. С~одной стороны, значение~$v_0$ должно 
быть достаточно большим, чтобы обеспечить корректное выполнение операций 
с~плавающей точкой. С~другой стороны, неразумное увеличение этого 
значения может дать снижение качества классификации данных на основе 
модели смеси (например, слишком большие значения~$v_0$ могут привести 
к~потере индивидуальности отдельных элементов смеси).
     
\section{Последствия неправильного выбора размерности}

     Исследуем влияние ошибочного представления о реальной модели 
данных PPCA, принятой при классификации. Подобная постановка задачи 
актуальна в~связи с~вопросом, может ли снижение размерности данных на 
основе анализа главных компонент привести к~повышению качества 
классификации данных.
     
     Аналогичная ситуация рассматривалась в~[1]. На примере задачи 
распознавания рукописных цифр исследовалась эффективность представления 
плотности распределения данных с~помощью модели смеси PPCA. Было 
продемонстрировано снижение ошибочной классификации за счет 
рассмотрения не просто нормального распределения, а~смеси нормальных 
распределений (увеличение числа элементов смеси с~$M\hm=1$ 
до~10) и~снижения размерности данных (с~$d\hm=64$ 
до~10). Но данное улучшение, скорее всего, является просто 
результатом использования смеси, а~выбор малых значений~$k$ лишь 
обеспечивает снижение вычислительной слож\-ности, но не ясно, как он влияет 
на качество классификации. Таким образом, фактически было лишь показано, 
что качество классификации повышается в~связи с~усложнением модели 
данных.
     
     Чтобы составить представление о поведении качества классификации 
с~учетом PPCA, рассмотрим различение двух классов~$\omega_1$ 
и~$\omega_2$ нормально распределенных данных $N(\mathbf{a}_1, 
\mathbf{C}_1(k))$ и~$N(\mathbf{a}_2, \mathbf{C}_2(k))$, формируемых 
в~соответствии с~моделью PPCA$(k)$. При этом байесовский классификатор 
будет строиться с~помощью модели PPCA$(q)$. В~этом случае он будет 
обучаться по выборке~$\mathbf{Y}_n$ объема~$n$ в~соответствии с~моделью 
PPCA$(q)$, что даст для каждого класса оценки~$\hat{\mathbf{a}}_i$ 
и~$\hat{\mathbf{C}}_i(q)$, $i\hm=1,2$. Поэтому решающая функция для 
некоторого вектора~$\mathbf{x}$ примет вид:
    \begin{multline*}
     d_i(\mathbf{x})=\ln \pi_i -{}\\
     {}-\fr{1}{2}\ln \left\vert 
\hat{\mathbf{C}}_i(q)\right\vert -\fr{1}{2}\left( \mathbf{x}-
\hat{\mathbf{a}}_i\right)^{\mathrm{T}}\hat{\mathbf{C}}^{-1}(q) \left( \mathbf{x}-
\hat{\mathbf{a}}_i\right)\,,
\end{multline*}
где $\pi_i$~--- вероятности появления классов.

     Для построенного классификатора теперь можно найти условную 
вероятность ошибки классификатора~$P_e$ или ее оценку~$\hat{P}_e$, т.\,е.\ 
реализовать следующие шаги:
     \begin{itemize}
\item генерирование выборки~$\mathbf{Y}_n$ для 
$\pi_1 N(\mathbf{a}_1,\mathbf{C}_1(k))\hm+ \pi_2 N(\mathbf{a}_2, 
\mathbf{C}_2(k))$;
\item обучение классификатора в~соответствии с~моделью PPCA$(q)$, т.\,е.\ 
нахождение на основе сгенерированной выборки~$\mathbf{Y}_n$ оценок 
параметров модели PPCA$(q)$ и~с их помощью~$\hat{\mathbf{C}}_i(q)$;
\item нахождение~$P_e$ или~$\hat{P}_e$.
\end{itemize}
     
     Данные шаги могут быть многократно повторены для различных 
выборок~$\mathbf{y}_n$, что позволит \mbox{найти} оценки безусловных 
характеристик вероятности ошибки классификации.
     
     Нахождение $\hat{P}_e$ можно реализовать с~помощью метода 
моделирования, т.\,е.\ реализовать следующие шаги:
     \begin{itemize}
\item генерирование обучающей выборки~$\mathbf{X}_m$ для 
$f(\mathbf{u})\hm= \hat{\pi}_1 N\left( \hat{\mathbf{a}}_1, 
\hat{\mathbf{C}}_1(q)\right) +\hat{\pi}_2 N\left( \hat{\mathbf{a}}_2, 
\hat{\mathbf{C}}_2(q)\right)$;
\item классификация элементов~$\mathbf{X}_m$ на основе $f(\mathbf{u})$, 
затем получение значения~$\hat{P}_e$ путем сравнения смоделированной 
и~оцененной классификаций.
\end{itemize}
     
     С помощью некоторого упрощения постановки задачи классификации 
удается добиться того, что~$P_e$ можно найти аналитически, в~част\-ности, 
обобщая прием разд.~4.4~[5].
     
     Действительно, рассмотрим различение двух классов   и~нормально 
распределенных данных с~различными векторами средних~$\mathbf{a}_1$ 
и~$\mathbf{a}_2$, но одинаковыми ковариационными матрицами. Пусть 
наблюдения формируются в~соответствии с~моделью PPCA$(k)$, а~для их 
классификации используется модель PPCA$(q)$. Обучение байесовского 
классификатора по выборке~$\mathbf{Y}_n$ в~соответствии с~моделью 
PPCA$(q)$ дает оценку~$\hat{\mathbf{C}}(q)$. После этого классификация 
некоторого вектора~$\mathbf{x}$ осуществляется с~помощью функции
     \begin{multline*}
     u_{12}(\mathbf{x}) =2\mathbf{x}^{\mathrm{T}} \hat{\mathbf{C}}^{-1}(q) 
\left( \mathbf{a}_1 - \mathbf{a}_2\right)^{\mathrm{T}}-{}\\
{}-\left( \mathbf{a}_1 
+\mathbf{a}_2\right)^{\mathrm{T}}\hat{\mathbf{C}}^{-1} (q) \left(\mathbf{a}_1-
\mathbf{a}_2\right)\,.
   \end{multline*}
     
     При единичной функции потерь и~вероятностях появления 
классов~$\pi_i$ условие, определяющее принадлежность~$\mathbf{x}$ 
к~$\omega_1$, имеет вид $u_{12}(\mathbf{x})\hm>t$, где $t\hm= \ln 
(\pi_2/\pi_1)$. Случайная величина $u_{12}(\mathbf{x})$, где\linebreak $\mathbf{x}\sim 
N(\mathbf{a}_i, \mathbf{C}(k))$ для $i\hm=1,2$, как линейная ком\-бинация 
нормально распределенных случайных ве\-личин имеет также нормальное 
распределение.\linebreak Поэтому достаточно найти первые моменты 
распределений~$u_{12}(\mathbf{x})$ для каждого из двух классов, а~именно:
     $$
     2E_i\left\{ u_{12}(\mathbf{x})\right\} =\begin{cases}
     \rho\,, & i=1\,;\\
     -\rho\,, & i=2\,,
     \end{cases}
     $$
     где 
     $$
     \rho=(\mathbf{a}_1-\mathbf{a}_2)^{\mathrm{T}}\hat{\mathbf{C}}^{-
1} (\mathbf{a}_1-\mathbf{a}_2)\,;
$$
\vspace*{-12pt}

\noindent
     \begin{multline*}
\hspace*{-9pt}E_i\!\left\{ \!\left( u_{12}(\mathbf{x})-E_i\left\{ 
u_{12}(\mathbf{x})\right\}\right)^2\right\}^{\!2}=\left( \mathbf{a}_1-
\mathbf{a}_2\right)^{\mathrm{T}}\hat{\mathbf{C}}^{-1} (q) \times{}\\
{}\times E_i \left\{ \left( 
\mathbf{x}-\mathbf{a}_i\right) \left( \mathbf{x}-\mathbf{a}_i\right)^{\mathrm{T}} 
\hat{\mathbf{C}}^{-1} (q) \left( \mathbf{a}_1-\mathbf{a}_2\right)\right\}= {}\\
     {}=
     \left( \mathbf{a}_1-\mathbf{a}_2\right)^{\mathrm{T}}\hat{\mathbf{C}}^{-1}
     (q)\cdot \mathbf{C}(k)\hat{\mathbf{C}}^{-1}(q)\cdot \left( \mathbf{a}_1-
\mathbf{a}_2\right) = v^2\\ 
\mbox{(по\ определению)}.
     \end{multline*}
          В результате имеем:
     \begin{multline*}
     \hspace*{-6.08414pt}P_e=\pi_1\mathrm{Pr}\left\{ u_{12}(\mathbf{x}) < t\vert \omega_1\right\}+
     \pi_2\mathrm{Pr}\left\{ u_{12}(\mathbf{x})>t\vert \omega_2\right\}={}\\
     {}=\pi_1 \Phi\left( \fr{t-\rho/2}{\sqrt{v^2}}\right)+\pi_2 \left( 1-\Phi\left( 
\fr{t+\rho/2}{\sqrt{v^2}}\right)\right)\,,
     \end{multline*}
где $\Phi(u)$~--- функция стандартного нормального распределения.

     При $\pi_1=\pi_2=1/2$
     
     \noindent
     $$
     P_e=\Phi\left( -\fr{\rho}{2\sqrt{v^2}}\right)\,.
     $$
     Если $q=k$, то $v^2\hm=\rho$ и~получаем ранее известную формулу:
     
     \noindent
$$
P_e=\Phi\left(-
\fr{\sqrt{\rho}}{2}\right).
$$
     
     Для того чтобы составить представление о реальной зависимости 
качества классификации от знания фактической размерности данных, 
рассматривался случай: $d\hm=50$; $k\hm=5$;  $\mathbf{a}_1\hm-
\mathbf{a}_2\hm=(0{,}1; \ldots ; 0{,}1)^{\mathrm{T}}$; $\mathbf{C}$~--- 
некоторая случайно выбранная ковариационная матрица. Параметры выборок 
были таковы: $n\hm=300$; $N_{\mathrm{exp}}\hm=100$. Результаты моделирования 
позволили получить оценки~95\%-ных доверительных интервалов для оценки 
вероятности ошибки байесовского классификатора при различных значениях~$q$. 

В~первую очередь ковариационные матрицы для классов принимались 
одинаковыми: 
$$
\mathbf{C}_1=\mathbf{C}_2= (\mathbf{u}_1,\ldots 
,\mathbf{u}_k)\mathbf{V}_k(\mathbf{u}_1,\ldots , \mathbf{u}_k)^{\mathrm{T}}\,.
$$ 
Соответствующие результаты отражены на рисунке в~виде практически 
слившихся линий с~пометкой~\textit{1} (границы доверительных интервалов 
приблизительно равны).

Затем рассматривался случай ковариационных матриц: 

\noindent
\begin{align*}
\mathbf{C}_1&= (\mathbf{u}_1, \ldots ,\mathbf{u}_k)\mathbf{V}_k 
(\mathbf{u}_1, \ldots ,\mathbf{u}_k)^{\mathrm{T}};\\
\mathbf{C}_2&= 
(\mathbf{u}_2, \ldots ,\mathbf{u}_{k+1})\mathbf{V}_k (\mathbf{u}_2, \ldots 
,\mathbf{u}_{k+1})^{\mathrm{T}},
\end{align*}

{ \begin{center}  %fig1
 \vspace*{6pt}
  \mbox{%
 \epsfxsize=77.961mm 
 \epsfbox{kri-1.eps}
 }


\end{center}

\vspace*{-3pt}

\noindent
{\small{Зависимость качества классификации~$P_e$ от размерности 
данных~$q$, принятой при анализе данных}}
}

%\vspace*{11pt}
\pagebreak

\noindent
 использующих несколько разные 
подпространства главных компонент и~совпадающие дисперсии для них. 

Соответствующие результаты представлены на рисунке уже в~виде двух линий, 
помеченных как~\textit{2}. При оценивании вероятности ошибки классификации 
бралась контрольная выборка объема $N_{\mathrm{contr}}=300$. Заметим, что только 
изменение взаимной ориентации главных компонент для классов приводит 
к~существенному повышению качества классификации.
     
     Продемонстрированный пример, а~также множество дополнительно 
проведенных экспериментов позволяют сформулировать следующие 
результаты:
     \begin{itemize}
     \item оценка $P_e$ как функция от размерности~$q$ для модели данных 
имеет минимум для $q\hm=k$, т.\,е.\ при правильном задании реальной 
раз\-мер\-ности данных классификатор обладает наилучшим качеством (на 
рисунке наилучшее значение   выделено вертикальной штриховой прямой);
\item превышение наилучшего значения~$q$ обычно ухудшает качество 
классификации в~меньшей степени, чем его занижение (на рисунке это более 
ярко проявляется для графика~\textit{2}).
\end{itemize}


\section{Заключение}

     Внимание к~модели смеси в~рамках PPCA определяется в~первую очередь 
необходимостью повышения эффективности сжатия и~восстановления\linebreak данных. 
Но эта модель позволяет также детализировать описание реальных данных 
и~тем самым\linebreak создать предпосылки для повышения качества классификации. 
Как показывают проведенные эксперименты, вероятность ошибок 
классификатора может снижаться достаточно существенно.
     
Модель смеси нормальных распределений является основой для 
популярного подхода к~комбинированной оценке плотности. 

Однако такая 
модель %\linebreak 
обладает следующим недостатком: если каж\-дая гауссовская компонента 
описывается полной ковариационной мат\-ри\-цей, то для каждой компоненты 
смеси должны оцениваться $d(d+1)/2$ отдельных ковариационных\linebreak па\-ра\-метров.
Очевидно, что по мере роста размер\-ности пространства данных, да еще при 
естест\-вен\-ном желании увеличить число элементов смеси,\linebreak количество точек 
данных, необходимых для надежного определения этих па\-ра\-мет\-ров, становится 
непомерно высоким.
%
 Альтернативный подход заключается в~уменьшении  
чис\-ла параметров путем введения ограничения на форму ковариационной 
мат\-ри\-цы (другой прием со\-сто\-ял во введении предположений  
о~па\-ра\-мет\-рах полной ковариационной матрицы~\cite{6-kri}). При этом 
обычно используются два общих ограничения: задать мат\-ри\-цу ковариаций 
изотропной или диагональной. 

Изотропная модель сильно ограничена, 
поскольку она присваивает только один параметр для описания всей структуры 
ковариации для полноразмерных данных. Диагональная модель более гибкая, 
с~$d$ па\-ра\-мет\-ра\-ми, но главные оси эллипсоидов для элементов смеси 
должны быть выровнены\linebreak
 с~осями данных, и,~таким образом, каж\-дый 
от\-дельный элемент смеси не способен описывать корреляции меж\-ду 
переменными. Поэтому смесь моделей PPCA, где ковариации каждого элемента\linebreak 
пара\-мет\-ри\-зуются с~по\-мощью соотношения $\mathbf{C}\hm= 
\sigma^2\mathbf{I}\hm+\mathbf{WW}^{\mathrm{T}}$, может содержать 
существенно\linebreak меньшее чис\-ло па\-ра\-метров.
     
     Одним из преимуществ методологии PPCA является то, что определение 
модели плотности позволяет вы\-чис\-лять апостериорные вероятности 
принадлежности некоторого наблюденного значения элементу смеси 
(некоторому классу) и~использовать их для последующей классификации, не 
находя ошибку восстановления.
     
     Возможным недостатком вероятностного подхо\-да к~объединению 
локальных моделей PCA является то, что, оптимизируя функцию 
правдоподобия, модель смеси PPCA напрямую не минимизирует квадратичную 
ошибку реконструкции. Для приложений, где это ключевой критерий, следует 
ожидать, что алгоритмы, которые явно минимизируют ошибку восстановления, 
будут эффективнее. Эксперименты действительно показали, что это, как 
правило, имеет место, но важны две оговорки, прежде чем можно будет сделать 
ка\-кие-ли\-бо твердые выводы относительно пригодности данной модели.  

Во-пер\-вых, есть задачи, где окончательная модель смеси PPCA оказывалась 
фактически лучше в~смысле ошибки реконструкции, даже на обуча\-ющем 
наборе.

 Второе соображение заключается в~том, что имеются также 
свидетельства того, что сглаживание, подразумеваемое мягкой кластеризацией, 
присущей модели смеси PPCA, помогает уменьшить переобучение, особенно 
в~случае эксперимента сжатия данных, где статистика набора \mbox{тестовых} данных 
отличается от данных обучения гораздо больше, чем для других примеров (см., 
например,~[1]).

     
     В терминах модели гауссовой смеси смесь вероятностных анализаторов 
главных компонент позволяет моделировать данные больших размеров 
с~относительно небольшим числом свободных параметров, не налагая в~целом 
неуместного ограничения на структуру ковариации. 
%
Число свободных 
параметров можно контролировать с~по\-мощью выбора скрытой 
пространственной размерности~$k$, что позволяет проводить интерполяцию по 
сложности модели от изотропной до полной ковариационной структуры.
     
{\small\frenchspacing
 {%\baselineskip=10.8pt
 \addcontentsline{toc}{section}{References}
 \begin{thebibliography}{9}
 
 \bibitem{1-kri}
\Au{Tipping M.\,E., Bishop~C.\,M.} Mixtures of probabilistic principal component 
analyzers~// Neural Comput., 1999. Vol.~11. Iss.~2. P.~443--482.
\bibitem{2-kri}
\Au{Кривенко М.\,П.} Прикладные методы оценивания распределения 
многомерных данных малой выборки.~--- М.: ИПИ РАН, 2011. 146~с.
\bibitem{3-kri}
\Au{Wu C.\,F.\,J.} On convergence properties of the EM algorithm~// Ann. Stat., 
1983. Vol.~11. P.~95--103.

%\columnbreak

\bibitem{4-kri}
\Au{Nettleton D.} Convergence properties of the EM algorithm in constrained 
parameter spaces~// Can. J.~Stat., 1999. Vol.~27. Iss.~3. P.~639--648.



\bibitem{5-kri}
\Au{Ту Дж., Гонсалес~Р.} Принципы распознавания образов~/
Пер. с~англ.~--- М.: Мир, 1978. 
414~с. (\Au{Tou~J., Gonzalez~R.\,C.}  {Pattern 
recognition principles}.~--- Reading, MA, USA: Addison-Wesley Publ. 
Co., 1974.\linebreak 377~p.)
\bibitem{6-kri}
\Au{Ormoneit D., Tresp~V.} Improved gaussian mixture density estimates using 
Bayesian penalty terms and network averaging~// Advances in neural information 
rocessing systems~/
Eds. D.\,S.~Touretzky, M.\,C.~Mozer, M.\,E.~Hasselmo.~--- 
Cambridge, MA, USA: MIT Press, 1996. Vol.~8. P.~542--548.
 \end{thebibliography}

 }
 }

\end{multicols}

\vspace*{-7pt}

\hfill{\small\textit{Поступила в~редакцию 30.05.18}}

\vspace*{4pt}

%\newpage

%\vspace*{-24pt}

\hrule

\vspace*{2pt}

\hrule

\vspace*{-2pt}


\def\tit{SUPERVISED LEARNING CLASSIFICATION OF~DATA\\ TAKING INTO~ACCOUNT 
PRINCIPAL COMPONENT ANALYSIS}

\def\titkol{Supervised learning classification of~data taking into~account 
principal component analysis}

\def\aut{M.\,P.~Krivenko}

\def\autkol{M.\,P.~Krivenko}

\titel{\tit}{\aut}{\autkol}{\titkol}

\vspace*{-13pt}


\noindent
Institute of Informatics Problems, Federal Research Center ``Computer 
Science and Control'' of the Russian Academy of Sciences, 44-2~Vavilov Str., 
Moscow 119333, Russian Federation


\def\leftfootline{\small{\textbf{\thepage}
\hfill INFORMATIKA I EE PRIMENENIYA~--- INFORMATICS AND
APPLICATIONS\ \ \ 2018\ \ \ volume~12\ \ \ issue\ 3}
}%
 \def\rightfootline{\small{INFORMATIKA I EE PRIMENENIYA~---
INFORMATICS AND APPLICATIONS\ \ \ 2018\ \ \ volume~12\ \ \ issue\ 3
\hfill \textbf{\thepage}}}

\vspace*{1pt}



\Abste{The article examines questions of supervised learning 
classification of data taking into account principal component 
analysis (PCA) results. Сonstruction of a Bayesian classifier 
becomes possible after representation of covariances through the 
parameters of the probabilistic PCA model. The case of singular 
data distributions is singled out; for this case, it is suggested to 
estimate the parameters of the model under constraints on the 
eigenvalues of covariance matrices. The quality of classification is 
studied in respect to the actual data dimension. It is demonstrated 
that, when correctly assigned, the classifier has the least error 
probabilities. Exceeding the best value of the dimension usually 
worsens the quality of the classification to a lesser extent than its 
underestimation. The mixture of probabilistic principal component 
analyzer allows modeling big data by means of a relatively small 
number of free parameters. The number of free parameters can be 
controlled by choosing the latent dimension of the data.}

\KWE{principal component analysis; mixtures of normal 
distributions; EM algorithm; supervised learning classification}

\DOI{10.14357/19922264180308}

%\vspace*{-14pt}

 %\Ack
%\noindent



\vspace*{-4pt}

  \begin{multicols}{2}

\renewcommand{\bibname}{\protect\rmfamily References}
%\renewcommand{\bibname}{\large\protect\rm References}

{\small\frenchspacing
 {%\baselineskip=10.8pt
 \addcontentsline{toc}{section}{References}
 \begin{thebibliography}{9}
\bibitem{1-kri-1}
\Aue{Tipping, M.\,E., and C.\,M.~Bishop.} 1999. Mixtures of 
probabilistic principal component analyzers. \textit{Neural 
Comput.} 11(2):443--482.
\bibitem{2-kri-1}
\Aue{Krivenko, M.\,P.} 2011. \textit{Prikladnye metody 
otsenivaniya raspredeleniya mnogomernykh dannykh maloy 
vyborki} [Applied methods for estimating the distribution of small 
sample multidimensional data].~--- Moscow: IPI RAN. 146~p.
\bibitem{3-kri-1}
\Aue{Wu, C.\,F.\,J.} 1983. On convergence properties of the EM 
algorithm. \textit{Ann. Stat.} 11:95--103.
\bibitem{4-kri-1}
\Aue{Nettleton, D.} 1999. Convergence properties of the EM 
algorithm in constrained parameter spaces. \textit{Can. 
J.~Stat.} 27(3):639--648.
\bibitem{5-kri-1}
\Aue{Tou, J., and R.\,C.~Gonzalez.} 1974.  \textit{Pattern 
recognition principles}. Reading, MA: Addison-Wesley Publ. 
Co. 377~p.
\bibitem{6-kri-1}
\Aue{Ormoneit, D., and V.~Tresp.} 1996. Improved gaussian 
mixture density estimates using Bayesian penalty terms and 
network averaging. Eds. D.\,S.~Touretzky, M.\,C.~Mozer, and 
M.\,E.~Hasselmo. \textit{Advances in neural information 
processing systems}.  Cambridge, MA: MIT Press.  
8:542--548. 
 %1120 p.
\end{thebibliography}

 }
 }

\end{multicols}

\vspace*{-9pt}

\hfill{\small\textit{Received May 30, 2018}}

%\pagebreak

\vspace*{-18pt}

\Contrl

\noindent
\textbf{Krivenko Michail P.} (b.\ 1946)~--- Doctor of Science in 
technology, professor, leading scientist, Institute of Informatics 
Problems, Federal Research Center ``Computer Science and 
Control'' of the Russian Academy of Sciences, 44-2~Vavilov Str., 
Moscow 119333, Russian Federation; 
\mbox{mkrivenko@ipiran.ru}


\label{end\stat}


\renewcommand{\bibname}{\protect\rm Литература}       %2
\include{torhin-rud}   %3


%\usepackage{bm}                     % полужирные греческие буквы
%\usepackage{upgreek}                % прямые греческие буквы

\newcommand{\sgn}{\textrm{sgn}}
\newcommand{\cov}{\textrm{cov}}



\newcommand{\indYilT}{\Ik_{|Y_i|\leq T}}
\newcommand{\indYigT}{\Ik_{|Y_i|>T}}

\newcommand{\indYighT}{\Ik_{|Y_i|>\hat T}}

\newcommand{\expct}{\textrm{E}}
\newcommand{\prbblty}{\textrm{P}}
\newcommand{\varnce}{\textrm{D}}
\newcommand{\Obig}{\textrm{O}}
\newcommand{\osml}{\textrm{o}}
\newcommand{\hsig}{\hat\sigma^2}
\newcommand{\ovrV}{\overline{V}}


\def\stat{markin}


\def\tit{ПРЕДЕЛЬНОЕ РАСПРЕДЕЛЕНИЕ ОЦЕНКИ РИСКА ПРИ~ПОРОГОВОЙ ОБРАБОТКЕ ВЕЙВЛЕТ-КОЭФФИЦИЕНТОВ}
\def\titkol{Предельное распределение оценки риска при пороговой обработке вейвлет-коэффициентов} 

\def\autkol{А.\,В. Маркин}
\def\aut{А.\,В. Маркин$^1$}

\titel{\tit}{\aut}{\autkol}{\titkol}

%{\renewcommand{\thefootnote}{\fnsymbol{footnote}}\footnotetext[1]
%{Работа выполнена при частичной поддержке РФФИ, проекты 08-07-00152-а и
%      09-07-12032-офи-м.}}

\renewcommand{\thefootnote}{\arabic{footnote}}
\footnotetext[1]{Московский государственный университет им.~М.\,В.~Ломоносова, 
 кафедра математической статистики факультета ВМиК, artem.v.markin@mail.ru}

\Abst{Исследованы асимптотические свойства оценки риска при пороговой обработке 
коэффициентов разложения функции сигнала по вейвлет-базису. Приведены условия, при 
которых будет иметь место сходимость по распределению к нетривиальному пределу разности 
теоретического риска и его оценки. Работа является продолжением статьи~[1].}

\KW{вейвлеты; пороговая обработка; оценка риска; предельное распределение}

      \vskip 18pt plus 9pt minus 6pt

      \thispagestyle{headings}

      \begin{multicols}{2}

      \label{st\stat}


\section{Введение}

Рассмотрена следующая задача: имеются на\-блю\-де\-ния~$X$, состоящие из полезного сигнала~$f$ 
и аддитивного белого гауссовского шума~$\epsilon$ с нулевым средним и  дисперсией~$\sigma^2$:
\begin{equation*}
%\label{eq_theSignalModel}
X=f+\epsilon\,.
\end{equation*}
Необходимо оценить~$f$ по~$X$. Параметр~$\sigma$ практически всегда неизвестен, 
для него существует лишь некоторый разумный диапазон. Сигнал дискретный, размер 
равен $N=2^J$. К наблюдениям применяется вейвлет-преобразование вида
\begin{equation}
\label{eq_DiscreteWaveletTrans}
y=\langle y,\phi_{0,0}\rangle\phi_{0,0}+\sum_{j=0}^{J-1}\sum_{n=0}^{2^j-1}\langle y,\psi_{j,n}\rangle\psi_{j,n}\,,
\end{equation}
где $\phi(t)$~--- отцовский вейвлет; $\psi(t)$~--- материнский вейвлет; $\phi_{j,n}(t)=2^{j/2}\phi\left(2^jt-n\right)$; 
$\psi_{j,n}(t)=$\linebreak 
$=2^{j/2}\psi\left(2^jt-n\right)$. Преобразование~(\ref{eq_DiscreteWaveletTrans}) 
справедливо для любой функции $y(t)\in\mathbf{L}^2(\mathbb{R})$~\cite{3mar, 4mar}.

После применения вейвлет-преобразования получим
\begin{equation*}
%\label{eq_theSignalModelWavelet}
X_W=f_W+\epsilon_W,
\end{equation*}
где~$\epsilon_W$ также будет белым гауссовским шумом с нулевым средним и дисперсией~$\sigma^2$. 
Предположим, что сигнал гладкий по Липшицу с параметром $\alpha>1/2$ (см.\ ниже). 
Будем считать, что существует константа~$C$ такая, что $|f_W|<C$ для любого~$N$. Можно показать, 
что число коэффициентов, для которых последнее неравенство не выполнено, мало, и эти коэффициенты 
не повлияют на приведенные ниже результаты.

Для удаления шума используется пороговая обработка коэффициентов. Коэффициенты, не 
превышающие некоторого порога, считаются коэффициентами шума. Величина порога определяется 
параметрами шума и свойствами полезного сигнала. Простейшая пороговая обработка~--- \textit{жесткая пороговая обработка}:
%\noindent
\begin{equation*}
%\label{eq_hardTreshold}
\rho_H(x, T)=\begin{cases}
x & \text{при } |x|>T\,;\\
0 & \text{при } |x|\leq T\,.
\end{cases}
\end{equation*}
Однако функция~$\rho_H$ разрывна и дает смещенные оценки~[1]. 
Поэтому на практике обычно используют \textit{мягкую пороговую обработку}:
\begin{equation*}
%\label{eq_softTreshold}
\rho_S(x, T)=
\begin{cases}
x-T & \text{при } x>T\,;\\
x+T & \text{при } x<-T\,;\\
0 & \text{при } |x|\leq T\,.
\end{cases}
\end{equation*}

Риск пороговой обработки~$r(f,T)$ определяется следующим образом:
\begin{equation*}
%\label{eq_idealRiskDef}
r(f,T)=\sum_{i=1}^{N}\expct\left\{f_W[i]-\rho(X_W[i])\right\}^2\,.
\end{equation*}
Однако вычислить явно~$r(f,T)$ нельзя, так как в выражении присутствуют неизвестные величины~$f_W[i]$, 
поэтому вместо теоретического риска используют его оценку. Например, можно использовать такую оценку:
\begin{equation}
\label{eq_estimRiskDef}
\tilde{r}(f,T)=\sum_{i=1}^{N}\Phi\left((X_W[i])^2\right)\,,
\end{equation}
где $\Phi(x)$ для мягкой пороговой обработки равна
\begin{equation*}
%\label{eq_riskPhiFuncDef}
\Phi_S(x)=
\begin{cases}
x-\sigma^2 & \text{при } x\leq T^2\,;\\
\sigma^2+T^2 & \text{при } x> T^2\,,
\end{cases}
\end{equation*}
а для жесткого порога
\begin{equation*}
%\label{eq_riskPhiFuncDefHard}
\Phi_H(x)=
\begin{cases}
x-\sigma^2 & \text{при } x\leq T^2\,;\\
\sigma^2 & \text{при } x> T^2\,.
\end{cases}
\end{equation*}
\pagebreak

Можно представить пороговую функцию в виде $\rho(x)=x-g(x)$. Имеем для жесткого и мягкого порога соответственно
\begin{align*}
%\label{eq_gHard}
g_H(x)&=x\Ik_{|x|\leq T}\,;\\
%\label{eq_gSoft}
g_S(x)&=T\sgn\,x+(x-T\sgn\, x)\Ik_{|x|\leq T}\,.
\end{align*}
При этом функция~$g_S(x)$ почти дифференци\-ру\-ема (в смысле Стейна~\cite{6mar}), 
а потому оценка~$\tilde{r}_S(f,T)$ будет несмещенной оценкой риска~$r_S(f,T)$ (SURE-оценка~--- Stein
Unbiased Risk Estimator~--- несмещенная оценка Стейна для риска)~\cite{3mar, 5mar}:
\begin{equation*}
\expct\{\tilde{r}_S(f,T)\}=r_S(f,T)\,.
\end{equation*}
Оценка риска жесткой пороговой обработки будет смещенной~\cite{3mar}:
\begin{multline*}
\!\!\expct\{\tilde{r}_H(f,T)\}=r_H(f,T)-{}\\
{}-2T\sigma^2\sum_{i=1}^N
\left(\varphi_\sigma(T-f_W[i])+\varphi_\sigma(T+f_W[i])\right)\,,
\end{multline*}
где $\varphi_\sigma(u)=\varphi\left(u/\sigma\right)$.
В качестве порога будем использовать универсальный порог $T =\sqrt{2 \ln N}$, 
обла\-да\-ющий хорошими асимптотическими свойствами~\cite{2mar}.

В работе~\cite{1mar} показано, что для $\delta>0$ и любого $\alpha>0$ 
при использовании универсального порога для мягкой и жесткой пороговой обработки выполнено
\begin{equation*}
\prbblty\left(\fr{\left|\tilde{r}(f,T)-r(f,T)\right|}{N^{\alpha+1/2}}>\delta\right)\rightarrow 0\,.
\end{equation*}
Естественно поставить вопрос о том, что будет в случае $\alpha=0$, т.\,е.\ когда знаменатель имеет порядок~$\sqrt{N}$.

На практике дисперсия шума заранее не известна, и ее необходимо оценить. В этом случае в выражении~(\ref{eq_estimRiskDef}) 
для оценки риска будут использоваться функции
\begin{equation*}
%\label{eq_riskPhiFuncEstDef}
\hat\Phi_S(x)=\begin{cases}
x-\hat\sigma^2 & \text{при } x\leq \hat T^2\,;\\
\hat\sigma^2+\hat T^2 & \text{при } x> \hat T^2
\end{cases}
\end{equation*}
и
\begin{equation*}
%\label{eq_riskPhiFuncEstDefHard}
\hat\Phi_H(x)=\begin{cases}
x-\hat\sigma^2 & \text{при } x\leq \hat T^2\,;\\
\hat\sigma^2 & \text{при } x> \hat T^2\,.
\end{cases}
\end{equation*}
Соответствующие оценки риска обозначим через~$\hat{r}_S(f,\hat T)$ и~$\hat{r}_H(f,\hat T)$. 
Известно~\cite{1mar}, что ес\-ли~$\hsig$~--- оценка дисперсии, 
$\expct\hsig\sigma^2+\nu_N$, $\nu_N=\osml(1)$, 
$\varnce\hsig=\theta_N=\Obig(N^{-\beta})$, $\beta>0$, 
то для любого $\delta>0$ при использовании порога $\hat T=\hat\sigma\sqrt{2\ln N}$ вы\-пол\-нено
\begin{equation}
\label{eq_ConsistencyUnknownSigma}
\prbblty\left(\fr{\left|\hat{r}(f,\hat T)-r(f,T)\right|}{N}>\delta\right)\rightarrow 0\,.
\end{equation}
Вопрос о сходимости при более низком порядке знаменателя в~(\ref{eq_ConsistencyUnknownSigma}) 
также рассмотрен в настоящей статье.

\section{Предельное распределение оценки риска при известной дисперсии шума}

\noindent
\textbf{Теорема 1.} \textit{ %\label{theo_knownSweak}
При мягкой или жесткой пороговой обработке с порогом $T=\sigma\sqrt{2\ln N}$ имеет место сходимость по распределению
\begin{equation*}
\fr{\tilde{r}(f,T)-r(f,T)}{D_N}\Rightarrow \mathcal{N}(0,1)\,,
\end{equation*}
где $D_N^2=2\sigma^2\left(N\sigma^2+2\sum\limits_{i=1}^N \left(f_W[i]\right)^2\right)$.
}

\smallskip

\noindent
Д\,о\,к\,а\,з\,а\,т\,е\,л\,ь\,с\,т\,в\,о\,.\ 
Введем обозначения: $X_W[i]\equiv$\linebreak $\equiv Y_i$, $f_W[i]\equiv \mu_i$. 
Тогда $\expct Y_i^2=\mu_i^2+\sigma^2$ и $\varnce Y_i^2 =$\linebreak 
$= 2\sigma^2\left(\sigma^2+2\mu_i^2\right)$. 
Для мягкой пороговой обработки имеем
\begin{multline}
\label{eq_riskMainComponentSknown}
\tilde{r}_S(f,T)-r_S(f,T)=\sum_{i=1}^N\left(Y_i^2-\sigma^2\right)\indYilT+{}\\
{}+\sum_{i=1}^N\left(\sigma^2+T^2\right)\indYigT- %{}\\
%{}-
\sum_{i=1}^N\expct\left(Y_i^2-\sigma^2\right)\indYilT-{}\\
{}-\left(\sigma^2+T^2\right)\sum_{i=1}^N\prbblty(\left|Y_i\right|>T)={}\\
{}=\sum_{i=1}^N\left(Y_i^2-\expct Y_i^2\right)-\sum_{i=1}^N\left(Y_i^2-\sigma^2\right)\indYigT+{}\\
{}+\sum_{i=1}^N\expct\left(Y_i^2-\sigma^2\right)\indYigT+ %{}\\
%{}+
\sum_{i=1}^N\left(\sigma^2+T^2\right)\indYigT-{}\\
{}-\left(\sigma^2+T^2\right)\sum_{i=1}^N\prbblty(\left|Y_i\right|>T).
\end{multline}
Достаточно показать, что при делении на~$D_N$ первая сумма в~(\ref{eq_riskMainComponentSknown}) 
сходится по распределению к стандартному нормальному закону, а все оставшиеся суммы сходятся к нулю по вероятности.

Понятно, что случайные величины $Y_i^2$ независимы и
\begin{equation*}
D_N^2=\sum_{i=1}^N\varnce Y_i^2=\varnce\sum_{i=1}^N Y_i^2\,.
\end{equation*}
Для обеспечения указанной сходимости по распределению достаточно выполнения условия Линдеберга~\cite{7mar}. 
Проверим, что для любого $\delta>0$

\noindent
\begin{multline}
\label{eq_LindSknown}
\fr{1}{D_N^2}\sum_{i=1}^N\expct\left\{
\left(Y_i^2-\mu_i^2-\sigma^2\right)^2\Ik_{\left|Y_i^2-\mu_i^2-\sigma^2\right|\geq \delta D_N}\right\}\rightarrow\\
{}\rightarrow 0\,.
\end{multline}
Имеем
\begin{equation*}
D_N^2=2\sigma^2\left(N\sigma^2+2\sum_{i=1}^N \mu_i^2\right)\geq 2\sigma^4N=\Obig(N)\,,
\end{equation*}
поэтому $D_N\rightarrow\infty$, и в силу ограниченности дис\-пер\-сии~$Y_i^2$ получаем для любого~$i$
\begin{equation*}
\expct\left\{\left(Y_i^2-\mu_i^2-\sigma^2\right)^2\Ik_{\left|Y_i^2-\mu_i^2-\sigma^2\right|\geq \delta D_N}\right\}=\osml(1)\,.
\end{equation*}
Значит, условие~(\ref{eq_LindSknown}) выполнено и
\begin{equation*}
\fr{\sum\limits_{i=1}^N\left(Y_i^2-\expct Y_i^2\right)}{D_N}\Rightarrow \mathcal{N}(0,1)\,.
\end{equation*}

В работе~[1] показано, что
\begin{multline*}
\expct\indYigT=\expct\left(\indYigT\right)^2=
\prbblty\left(\left|Y_i\right|>T\right)={}\\
{}=\Obig\left(\fr{1}{N\sqrt{\ln N}}\right)\,.
\end{multline*}
Тогда по неравенствам Чебышёва и Коши--Бу\-ня\-ков\-ско\-го
\begin{multline*}
\prbblty\left(\fr{\left|\sum\limits_{i=1}^N\left(Y_i^2-\sigma^2\right)\indYigT\right|}{D_N}>\delta\right) \leq {}\\
{}\leq
\fr{\expct\left|\sum\limits_{i=1}^N\left(Y_i^2-\sigma^2\right)\indYigT\right|}{\delta D_N}\leq{}\\
{}\leq\fr{\sum\limits_{i=1}^N\expct\left|\left(Y_i^2-\sigma^2\right)\indYigT\right|}{\delta\sigma^2\sqrt{2N}}\leq{}\\
{}\leq\fr{\sum\limits_{i=1}^N\sqrt{\expct\left(Y_i^2-\sigma^2\right)^2\prbblty\left(\left|Y_i\right|>T\right)}}{\delta\sigma^2\sqrt{2N}}\rightarrow 0\,.
\end{multline*}
Для трех оставшихся сумм в~(\ref{eq_riskMainComponentSknown}) рассуждения проводятся аналогично. 
Таким образом, для мягкой пороговой обработки теорема доказана.

Учитывая суть отличий~$\Phi_H(x)$ от~$\Phi_S(x)$, доказательство 
для жесткого порога практически повторяет рассуждения для случая 
мягкого порога. Остается только добавить, что порядок смещения 
оценки риска при жестком пороге равен $\Obig \left(\sqrt{\ln N}\right)$~[1], 
стало быть, при делении на~$D_N$ смещение будет стремиться к нулю.

\section{Риск в случае использования оценки дисперсии}
\vspace*{-2pt}

В работе~[1] (теоремы~3 и~6) показано, что выполнено~(\ref{eq_ConsistencyUnknownSigma}), 
при этом ограничения на~$\hsig$ носят довольно общий характер. Например, от~$\nu_N$ 
требуется лишь бесконечная малость, т.\,е.\ стремление к нулю может быть как угодно 
медленным. Возникает вопрос: можно ли понизить степень~$N$ в знаменателе, наложив 
некоторые более сильные ограничения на свойства моментов~$\hsig$?

В доказательстве теоремы~3 из статьи~[1] есть такая оценка:
\vspace*{-6pt}

\noindent
\begin{multline*}
E_{ij}={}\\
\!\!{}=\expct\left(Y_i^2-\hsig-\expct Y_i^2+\sigma^2\right)\left(Y_j^2-\hsig-\expct Y_j^2+\sigma^2\right)={}\\
{}=-\cov\left(\hsig,Y_i^2+Y_j^2\right)+\nu_N^2+\theta_N=\osml(1)\,.
\end{multline*}
Число элементов~$E_{ij}$ равно $N^2-N$, поэтому сумма всех этих элементов имеет порядок~$\osml(N^2)$. 
Понятно, что эту оценку можно улучшить, если потребовать, скажем, $\nu_N=\Obig\left(N^{-\upsilon}\right)$.

Помимо этого, из доказательства той же теоремы можно получить следующую оценку~$\expct\indYighT$:
\vspace*{-3pt}

\noindent
\begin{equation*}
\expct\indYighT=\max\left\{ \Obig\left(\fr{1}{N^{(1-\gamma)^2}\sqrt{\ln N}}\right), 
\Obig\left(N^{-\beta}\right) \right\}\,,
\end{equation*}
где $\gamma$~--- фиксированное сколь угодно малое положительное число. 
Принимая, что $\nu_N=\Obig\left(N^{-\upsilon}\right)$, $\upsilon>0$, эту оценку также можно улучшить.

Представим разность оценки риска и самого риска при мягком пороге в таком виде:
\vspace*{-6pt}

\noindent
\begin{multline}
\label{eq_hrSmrSsplit}
\hat r_S-r_S=\sum_{i=1}^{N}\left(Y_i^2-\sigma^2\right)-N\left(\hsig-\sigma^2\right)-{}\\
{}-\sum_{i=1}^{N}\left(Y_i^2-\hsig\right)\indYighT+\sum_{i=1}^{N}\left(\hsig+\hat T^2\right)\indYighT-{}\\
{}-\sum_{i=1}^{N}\expct\left(Y_i^2-\sigma^2\right)+\sum_{i=1}^{N}\expct\left(Y_i^2-\sigma^2\right)\indYigT-{}\\
{}-\sum_{i=1}^{N}\expct\left(\sigma^2+T^2\right)\indYigT\,.
\end{multline}
Вопрос о невырожденном предельном распределении 
\vspace*{-3pt}

\noindent
$$
\sum\limits_{i=1}^{N}\left(Y_i^2-\sigma^2\right)-\sum\limits_{i=1}^{N}\expct\left(Y_i^2-\sigma^2\right)
$$ 
решен в предыдущем разделе, а в статье~[1] показано, что при делении на~$N^{\alpha+1/2}$ 
будет иметь мес\-то сходимость к нулю по вероятности при любом $\alpha>0$.
\pagebreak

Асимптотика $N\left(\hsig-\sigma^2\right)$ зависит в том чис\-ле и от 
скорости убывания~$\nu_N$ и~$\theta_N$. Если $\nu_N=\Obig\left(N^{-1/2}\right)$ и 
$\theta_N=\Obig\left(N^{-1}\right)$, то при $\alpha>0$
\begin{equation}
\label{eq_hsigConsist}
\fr{N\left(\hsig-\sigma^2\right)}{N^{\alpha+1/2}}\stackrel{\prbblty}{\longrightarrow} 0\,.
\end{equation}
Это можно показать с помощью неравенства Чебышёва. Более того, 
если $\nu_N=\Obig\left(N^{-\upsilon}\right)$, $\theta_N=$\linebreak $=\Obig\left(N^{-\beta}\right)$, 
то~(\ref{eq_hsigConsist}) справедливо для
\begin{equation*}
\alpha>\fr{1}{2}-\min\left\{\upsilon, \fr{\beta}{2}\right\}\,.
\end{equation*}

Покажем теперь, что если справедливы последние предположения о~$\nu_N$ и~$\theta_N$, 
то все суммы с индикаторами в~(\ref{eq_hrSmrSsplit}) при делении на~$\sqrt{N}$ 
сходятся по вероятности к нулю. По формуле полной вероятности имеем
\begin{multline}
\label{eq_fullProbForGammaN}
\prbblty\left(\sum_{i=1}^{N}\indYighT>\delta\right)={}\\
{}=\prbblty\left(\sum_{i=1}^{N}\indYighT>\delta\,, \enskip\hat T>(1-\gamma_N)\sigma\sqrt{2\ln N}\right)+{}\\
{}+\prbblty\left(\sum_{i=1}^{N}\indYighT>\delta|\hat T\leq(1-\gamma_N)\sigma\sqrt{2\ln N}\right)\times{}\\
{}\times \prbblty\left(\hat T\leq(1-\gamma_N)\sigma\sqrt{2\ln N}\right)\,,
\end{multline}
где
\begin{equation*}
\gamma_N=\fr{1}{\ln N}\gg |\nu_N|\,;\enskip \gamma_N^2\gg\theta_N\,.
\end{equation*}
При этом
\begin{multline*}
\prbblty\left(|Y_i|>(1-\gamma_N)\sigma\sqrt{2\ln N}\right)={}\\
=\Obig\left(\fr{1}{N^{(1-\gamma_N)^2}\sqrt{\ln N}}\right)=
\Obig\left(\fr{1}{N\sqrt{\ln N}}\right)\,,
\end{multline*}
так как
\begin{multline*}
N^{2\gamma_N-\gamma_N^2}=
\exp\left(\ln N\left(2\gamma_N-\gamma_N^2\right)\right)={}\\
{}=\exp\left(2-\fr{1}{\ln N}\right)=\Obig(1)\,.
\end{multline*}

Далее, пользуясь неравенством Чебышёва, получаем
$$
\prbblty\left(\sum_{i=1}^{N}\indYighT>\delta, \quad\hat T>(1-\gamma_N)\sigma\sqrt{2\ln N}\right)\leq{}
$$$$%\\
{}\leq \prbblty\left(\sum_{i=1}^{N}\Ik_{|Y_i|>(1-\gamma_N)\sigma\sqrt{2\ln N}}>\delta\right)\leq{}$$
%\columnbreak

\noindent
$${}\leq \fr{\sum\limits_{i=1}^{N}\prbblty\left(|Y_i|>(1-\gamma_N)\sigma\sqrt{2\ln N}\right)}{\delta}={}$$
\vspace*{-6pt}

\noindent
$$%\\
\hspace*{120pt}{}=\Obig\left(\fr{1}{\sqrt{\ln N}}\right)\rightarrow 0\,.
$$%\end{multline*}
Теперь воспользуемся свойствами моментов~$\hsig$ для оценки второго слагаемого в~(\ref{eq_fullProbForGammaN}):
\vspace*{-6pt}

\noindent
\begin{multline*}
\prbblty\left(\hat T\leq(1-\gamma_N)\sigma\sqrt{2\ln N}\right)={}\\
{}=
\prbblty\left(\hsig\leq(1-\gamma_N)^2\sigma^2\right)={}\\
{}=\prbblty\left(\hsig+\nu_N\leq\nu_N+\sigma^2-2\sigma^2\gamma_N+\gamma_N^2\sigma^2\right)\leq{}\\
{}\leq\prbblty\left(\left|\hsig-\nu_N-\sigma^2\right|\geq\nu_N+2\sigma^2\gamma_N-\gamma_N^2\sigma^2\right)\,.
\end{multline*}
Последнее неравенство справедливо, так как для достаточно большого~$N$ 
$$
\nu_N+2\sigma^2\gamma_N-\gamma_N^2\sigma^2>0
$$ 
в силу выбора~$\gamma_N$. Далее по неравенству Чебышёва
\vspace*{-3pt}

\noindent
\begin{multline*}
\prbblty\left(\left|\hsig-\nu_N-\sigma^2\right|\geq\nu_N+2\sigma^2\gamma_N-\gamma_N^2\sigma^2\right)\leq{}\\
{}\leq
\fr{\varnce\hsig}{\left(\nu_N+2\sigma^2\gamma_N-\gamma_N^2\sigma^2\right)^2}\sim{}\\
{}\sim\fr{\varnce\hsig}{\gamma_N^2}=\Obig\left(\fr{\ln^2 N}{N^\beta}\right)\rightarrow 0\,.
\end{multline*}
Таким образом,
\begin{equation*}
\prbblty\left(\sum_{i=1}^{N}\indYighT>\delta\right)\rightarrow 0\,.
\end{equation*}

Опять же пользуясь неравенством Чебышёва, имеем
\vspace*{-6pt}

\noindent
\begin{multline*}
\prbblty\left(\sum_{i=1}^{N}\indYigT>\delta\right)\leq
\fr{\sum\limits_{i=1}^{N}\prbblty\left(|Y_i|>T\right)}{\delta}={}\\
{}=
\Obig\left(\fr{1}{\sqrt{\ln N}}\right)\rightarrow 0\,.
\end{multline*}
Теперь перейдем непосредственно к суммам в~(\ref{eq_hrSmrSsplit}):
\begin{equation*}
\fr{\sum\limits_{i=1}^{N}Y_i^2\indYighT}{\sqrt{N}} \leq \sqrt{ \fr{\sum\limits_{i=1}^{N}Y_i^4}{N}
\sum\limits_{i=1}^{N}\indYighT }\xrightarrow{P}0\,,
\end{equation*}
так как
\begin{equation*}
\fr{\sum\limits_{i=1}^{N}\left(Y_i^4-\expct Y_i^4\right)}{N}\xrightarrow{P}0\,;\enskip
0\leq\fr{\sum\limits_{i=1}^{N}\expct Y_i^4}{N}\leq C
\end{equation*}
для некоторой константы~$C$.\ 
Далее
\pagebreak

\noindent
\begin{equation*}
\left(2\ln N+1\right)\fr{\sum\limits_{i=1}^{N}\hsig\indYighT}{\sqrt{N}}\xrightarrow{P}0\,,
\end{equation*}
так как

\noindent
\begin{equation*}
\fr{\left(2\ln N+1\right)}{\sqrt{N}}\hsig\xrightarrow{P}0\,.
\end{equation*}
Суммы с~$\indYigT$ оцениваются аналогично. При жестком пороге приведенные оценки тоже верны, 
смещение оценки риска при делении на~$\sqrt{N}$ стремится к нулю. Обобщая полученные результаты, 
можно сформулировать следующую теорему.

\smallskip

\noindent
\textbf{Теорема 2.} \textit{Пусть~$\hsig$~--- оценка дисперсии, 
$\expct\hsig=$\linebreak $=\sigma^2+\Obig\left(N^{-\upsilon}\right)$ и 
$\varnce\hsig=\Obig(N^{-\beta})$, $\upsilon>0$, $\beta>0$, 
а константа $c=\min\left\{1/2, \upsilon, \beta/2 \right\}$. 
Пусть выбран порог $\hat T=\hat\sigma\sqrt{2\ln N}$. 
Тогда для любого $\delta>0$ и любого $\alpha > 1/2-c$ при мягком и жестком пороге выполнено
\begin{equation*}
\prbblty\left(\fr{\left|\hat r(f,\hat T)-r(f,T)\right|}{N^{\alpha+1/2}}>\delta\right)\rightarrow 0\,.
\end{equation*}
}

%\smallskip
\vspace*{-4pt}

Как видно из приведенных выше рассуждений, предельное распределение 
$\left(\hat r-r\right)/\sqrt N$ определяется предельным распределением

\noindent
\begin{equation*}
\fr{\sum\limits_{i=1}^{N}\left(Y_i^2-\expct Y_i^2\right)}{\sqrt{N}}-\sqrt{N}\left(\hsig-\sigma^2\right)\,.
\end{equation*}

%\smallskip
\vspace*{-2pt}

\noindent %\label{theo_unknSweak}
\textbf{Теорема 3.} \textit{Пусть $\hsig$~--- оценка дисперсии, 
$\expct\hsig=$\linebreak $=\sigma^2+\Obig\left(N^{-\upsilon}\right)$ и 
$\varnce\hsig=\Obig(N^{-\beta})$, $\upsilon>0$, $\beta>$\linebreak $>0$. 
Пусть~$\hsig$ не зависит от~$X_W[i]$ и выполнено
$\sqrt{\mathcal{N}}\left ( \hat{\sigma}^2-\sigma^2\right )\rightarrow \mathcal{N}\left (0,\Sigma^2\right)$.
%асимптотически нормальна с параметрами~$\sigma^2$ и~$\Sigma^2/N$: 
%$\hsig\sim\mathcal{N}\left(\sigma^2,\Sigma^2/N\right)$ и . 
Пусть выбран порог $\hat T=\hat\sigma\sqrt{2\ln N}$. Тогда при мягком и жестком пороге выполнено}
\vspace*{3pt}

\noindent
\begin{equation*}
\fr{\hat{r}(f,\hat{T})-r(f,T)}{D_N}\Rightarrow \mathcal{N}\left(0,1+\fr{\Sigma^2}{2\sigma^2\left(\sigma^2+2R\right)}\right)\,,
\end{equation*}
\textit{где}

\noindent
\begin{align*}
D_N^2 &= 2\sigma^2\left(N\sigma^2+2\sum_{i=1}^N \left(f_W[i]\right)^2\right)\,;\\
R &= \lim\limits_{N\rightarrow\infty}\sum\limits_{i=1}^N\fr{\left(f_W[i]\right)^2}{N}\,.
\end{align*}


\noindent
Теорема элементарно доказывается c помощью метода характеристических функций и свойства 
воспроизводимости нормального закона. Понятно, что если распределение оценки дисперсии слабо 
сходится к некоторому пределу, но при нормировочном множителе порядка $0<\alpha<1/2$, то 
распределение оценки риска при делении на $N^{1-\alpha}$ будет слабо сходиться к 
тому же пределу. Причем в этом случае даже не требуется независимости~$X_W[i]$ и~$\hsig$.

\section{Оценки дисперсии шума}

В работе~[1] рассмотрены два примера оценок: 
$\hsig_1=S^2$ как оценка дисперсии и интерквартильный размах~$\hat\sigma_2$ как оценка стандартного отклонения. 
Пусть $Z_1, Z_2, \ldots, Z_M$~--- наблюдения, по которым вычисляется оценка, $\overline{Z}$~--- выборочное среднее, 
$Z_{M,p}$~--- выборочная квантиль порядка~$p$, $0\leq p\leq 1$:
\begin{equation*}
Z_{M,p}=
\begin{cases}
Z_{\left([Mp]+1\right)} & \text{при } Mp~\text{ дробном}\,;\\
Z_{\left(Mp\right)} & \text{при } Mp~\text{ целом}\,.
\end{cases}
\end{equation*}
Тогда
\begin{equation*}
\hsig_1=\fr{1}{M}\sum_{i=1}^M(Z_i-\overline{Z})^2\,;\enskip
\hat\sigma_2=\fr{Z_{M,3/4}-Z_{M,1/4}}{2\zeta_{3/4}}
\end{equation*}
будут соответственно состоятельными оценками дисперсии и стандартного отклонения, 
если~$Z$~--- выборка из нормального распределения с нулевым средним и дисперсией~$\sigma^2$, а 
$\zeta_{3/4}=F^{-1}\left(3/4\right)$~--- квантиль порядка~3/4 стандартного 
нормального распределения~[1]. Известно, что в этом случае~$\hsig_1$ асимптотически нормальна~\cite{8mar}:
\begin{equation*}
\sqrt{M}\left(\hsig_1-\sigma^2\right)\Rightarrow\mathcal{N}\left(0,2\sigma^4\right)\,.
\end{equation*}
Оценка $\hat\sigma_2$ тоже является асимптотически нормальной~\cite{9mar}:
\begin{equation}
\label{eq_weaklimIQR}
\sqrt{M}\left(\hat\sigma_2-\sigma\right)\Rightarrow\mathcal{N}\left(0,\fr{1}{\left\{4\zeta_{3/4}\phi_\sigma
\left(\zeta_{\sigma,3/4}\right)\right\}^2}\right)\,,
\end{equation}
где $\zeta_{\sigma,3/4}=F_\sigma^{-1}\left(3/4\right)$, а~$F_\sigma$ и~$\phi_\sigma$~--- 
функция распределения и плотность распределения~$\mathcal{N}\left(0,\sigma^2\right)$ соответственно. 
Обозначив предельную дисперсию в~(\ref{eq_weaklimIQR}) через~$d^2$, получим, что
\begin{multline}
\label{eq_weaklimIQR2}
\sqrt{M}\left(\hat\sigma_2^2-\sigma^2\right)={}\\
{}=\sqrt{M}\left(\hat\sigma_2-\sigma\right)\left(\hat\sigma_2+\sigma\right)\Rightarrow\mathcal{N}\left(0,4\sigma^2d^2\right)
\end{multline}
в силу сходимости $\hat\sigma_2$ к~$\sigma$ по вероятности. Значит, если $M$ имеет порядок роста~$N$, 
то для оценок~$\hsig_1$ и~$\hsig_2$ справедлива теорема~3. 
Заметим, что~(\ref{eq_weaklimIQR2}) можно показать, используя одну из теорем непрерывности~\cite{8mar}: 
$y\left(\hat\sigma_2\right)=\hsig_2$, $y'(\sigma)=2\sigma$, 
$\sqrt{M}\left(y(\hat\sigma_2)-y(\sigma)\right)\Rightarrow\mathcal{N}\left(0,\left(y'(\sigma)\right)^2d^2\right)$. 
В англоязычной литературе эти теоремы носят название Delta method (см., например,~\cite{10mar}).

Рассмотрим теперь случай, когда дисперсия оценивается по выборке~$X$, а точнее по ее части. 
Можно считать, что вейвлет-коэффициенты на самом мелком масштабе являются коэффициентами шума~\cite{2mar}. 
Ниже это предположение будет обобщено. Пусть $Y_{N/2+1},\ldots,Y_N$ имеют распределение 
$\mathcal{N}\left(0,\sigma^2\right)$. Оценка~$\hsig_1$ принимает вид
\begin{equation*}
\hsig_1=S^2=\fr{1}{N/2}\sum\limits_{i=N/2+1}^N Y_i^2-\left(\fr{2}{N}\sum\limits_{i=N/2+1}^N Y_i\right)^2\,.
\end{equation*}
Тогда
\begin{multline*}
\sum\limits_{i=1}^N \left(Y_i^2-\hsig_1\right)=
\sum\limits_{i=1}^{N/2} Y_i^2 -{}\\
{}- \sum\limits_{i=N/2+1}^N Y_i^2 + N\left(\fr{2}{N}\sum\limits_{i=N/2+1}^N Y_i\right)^2\,.
\end{multline*}
При делении на~$\sqrt{N}$ получаем
\begin{equation*}
%\label{eq_weaklimAver2}
\sqrt{N}\left(\fr{2}{N}\sum\limits_{i=N/2+1}^N Y_i\right)^2 \rightarrow 0
\end{equation*}
по вероятности, так как последовательность $\lambda_N\equiv$\linebreak $\equiv \sqrt{N}\left(2/N\sum_{i=N/2+1}^N Y_i\right)$ 
сходится по распределению к нормальному закону~\cite{8mar}, а следовательно, в силу теоремы 
непрерывности последовательность~$\lambda_N^2/\sqrt{N}$ сходится по вероятности к нулю.

Обобщая вышесказанное, получаем, что
\begin{multline}
\label{eq_weaklimriskS2}
\fr{\hat{r}(f,\hat{T})-r(f,T)}{D_N}\Rightarrow{}\\
{}\Rightarrow \mathcal{N}\left(0,1+\fr{2\sigma^4}{2\sigma^2\left(\sigma^2+2R\right)}\right)\,,
\end{multline}
где
\begin{equation}
\label{eq_DNforS2}
D_N^2=2\sigma^2\left(\fr{N}{2}\sigma^2+2\sum_{i=1}^{N/2} \left(f_W[i]\right)^2\right)\,.
\end{equation}

Используя результаты работы~\cite{11mar}, для совместного распределения второго выборочного момента~$\overline{Z^2}$ 
и интерквартильного размаха $\mathrm{IQR}=Z_{M,3/4}\;-$\linebreak $-\;Z_{M,1/4}$ имеем

\noindent
\begin{multline}
\label{eq_jointDistrIQR_S2}
\sqrt{M}\left(\left(\begin{array}{c}\overline{Z^2}\\\mathrm{IQR}\end{array}\right)-
\left(
\begin{array}{c}
\sigma^2\\2\sigma\zeta_{3/4}\end{array}
\right)\right)
\Rightarrow{}\\
{}\Rightarrow  \mathcal{N}
\left(0,\left(
\begin{array}{cc} 2\sigma^4 & 2\sigma^3\zeta_{3/4} \\[3pt]
2\sigma^3\zeta_{3/4} & 
\displaystyle\fr{\sigma^2}{4\phi^2\left(\zeta_{3/4}\right)} \end{array}
\right)\right)\,.
\end{multline}
Здесь учтено, что $\expct Z_i=0$ и $\phi_\sigma\left(\zeta_{\sigma,3/4}\right)=
\phi\left(\zeta_{3/4}\right)/\sigma$, $\phi(\cdot)$~--- плотность распределения~$\mathcal{N}(0,1)$. 
Применим к $\left(\overline{Z^2},\mathrm{IQR}\right)$ преобразование~$z$ с матрицей частных производных~$W$, вычисленных в
точ\-ке~$\left(\sigma^2, 2\sigma\zeta_{3/4}\right)^T$:
\begin{equation*}
z(x,y)=\left(x,\fr{y^2}{4\zeta^2_{3/4}}\right)^T\,, 
\quad W=\left(
\begin{array}{cc}
1 & 0\\0 & \displaystyle\fr{\sigma}{\zeta_{3/4}}
\end{array}\right)\,.
\end{equation*}
Если обозначить матрицу ковариаций в~(\ref{eq_jointDistrIQR_S2}) через~$\Omega^2$, то 
по векторному варианту теоремы непрерывности
\begin{equation*}
\sqrt{M}\left(z\left(\overline{Z^2},\mathrm{IQR}\right)-\left(\begin{array}{c}\sigma^2\\\sigma^2\end{array}\right)\right)
\Rightarrow \mathcal{N}\left(0,W\Omega^2W^T\right)\,,
\end{equation*}
где
\begin{equation*}
W\Omega^2W^T=\left(\begin{array}{cc} 2\sigma^4 & 2\sigma^4 \\ 2\sigma^4 & \fr{\sigma^4}{4\zeta_{3/4}^2\phi^2\left(\zeta_{3/4}\right)} 
\end{array}\right)\,.
\end{equation*}
Дисперсия разности~$\overline{Z^2}$ и нормированного интерквартильного размаха равна
\begin{equation*}
\varnce\left(\overline{Z^2}-\fr{\mathrm{IQR}^2}{4\zeta_{3/4}^2}\right)=
\fr{\sigma^4}{4\zeta_{3/4}^2\phi^2\left(\zeta_{3/4}\right)}-2\sigma^4\equiv\Sigma^2\,.
\end{equation*}
Теперь, полагая $M=N/2$, $Z_i=Y_{N/2+i}$, получаем для~$\hat\sigma_2$:
\begin{multline}
\label{eq_weaklimriskIQR}
\fr{\hat{r}(f,\hat{T})-r(f,T)}{D_N}\Rightarrow{}\\
{}\Rightarrow \mathcal{N}\left(0,1+\fr{\Sigma^2}{2\sigma^2\left(\sigma^2+2R\right)}\right)\,,
\end{multline}
где~$D_N$ определена в~(\ref{eq_DNforS2}).

Предполагалось, что $Y_{N/2+1},\ldots,Y_N$ имеют распределение 
$\mathcal{N}\left(0,\sigma^2\right)$, т.\,е.\ коэффициенты разложения на самом 
мелком масштабе содержат только\linebreak информацию о шуме. Однако такой подход несколько 
нестрогий, поэтому рассмотрим такое обобщение. Пусть~$Y_{N/2+i}$ имеет распределение 
$\mathcal{N}\left(\mu_i,\sigma^2\right)$, $i=\overline{N/2+1,N}$, при этом $\mu_i=\osml(1)$. 
Какой конкретно порядок малости достаточно потребовать, чтобы результаты для $\mu_i=0$ 
остались справедливыми, будет выяснено ниже. Такое обобщение вполне обосновано, так как, если функция~$f$ 
достаточно гладкая, вейвлет-коэффициенты убывают к нулю достаточно быстро~\cite{3mar}.

Функция~$f$ называется гладкой по Липшицу с параметром~$\alpha$ на от\-рез\-ке~$[a,b]$, 
если существуют константа~$K$ и полином~$p_x$ степени $m=\lfloor\alpha\rfloor$ такие, 
что для любого $x\in[a,b]$ и любого $t\in\mathbb{R}$
\begin{equation*}
\left| f(t)-p_x(t) \right|\leq K|t-x|^\alpha\,,
\end{equation*}
где
\begin{equation*}
p_x(t)=\sum\limits_{k=0}^{m}c_k(t-x)^k\,.
\end{equation*}
Понятно, что если~$f$ является $m$~раз дифференцируемой, то 
$c_k=f^{(k)}(x)/k!$, $k=\overline{0,m-1}$. 
При $0\leq\alpha<1$ имеем $p_x(t)=f(x)$. Известно~\cite{3mar}, что если~$f$ гладкая по 
Липшицу с параметром~$\alpha$ на отрезке~$[a,b]$, то существует константа~$\Upsilon$ такая, что
\begin{equation}
\label{eq_aiBound}
|\mu_i|\leq\fr{\Upsilon}{N^{\alpha}}\,,\quad i=\overline{\fr{N}{2}+1,N}\,,
\end{equation}
причем~$\Upsilon$ не зависит от~$N$.

Обозначим $V_i=Y_{N/2+i}-\mu_{N/2+i}$, $i=\overline{1,N/2}$, 
$\ovrV=2/N\sum\limits_{i=1}^{N/2}V_i$, $\mu=2/N\sum\limits_{i=N/2+1}^{N}\mu_i$. Тогда
\begin{multline}
\label{eq_Yi2Vi}
\fr{1}{\sqrt{N/2}}\sum\limits_{i=N/2+1}^N Y_i^2 = 
\fr{1}{\sqrt{N/2}}\sum\limits_{i=1}^{N/2}V_i^2 + {}\\
{}+
\fr{1}{\sqrt{N/2}}\sum\limits_{i=1}^{N/2}V_i\mu_{N/2+i} +
\fr{1}{\sqrt{N/2}}\sum\limits_{i=N/2+1}^N \mu_i^2\,.
\end{multline}
Допустим, что выполнено условие~(\ref{eq_aiBound}), тогда $\mu=\Obig(N^{-\alpha})$. 
В этом случае последняя сумма в~(\ref{eq_Yi2Vi}) стремится к нулю при $\alpha>1/4$. Далее,
\begin{multline*}
\!\!\left|\fr{1}{\sqrt{N/2}}\sum\limits_{i=1}^{N/2}V_i\mu_{N/2+i}\right| \leq 
\fr{1}{\sqrt{N/2}}\sum\limits_{i=1}^{N/2}|V_i\mu_{N/2+i}| \leq {}\\
{}\leq
\sqrt{\fr{\sum\limits_{i=1}^{N/2}V_i^2}{N/2}\sum\limits_{i=N/2+1}^N\mu_i^2}
\end{multline*}
сходится к нулю по вероятности при $\sum\limits_{i=N/2+1}^N\mu_i^2\rightarrow 0$, т.\,е.\ 
при $\alpha>1/2$. Кроме того,

\noindent
\begin{multline*}
\sqrt{N/2}\left(
\fr{1}{N/2}\sum\limits_{i=N/2+1}^N Y_i\right)^2 = \sqrt{\fr{N}{2}}\left(\ovrV\right)^2 +{}\\
{}+ 2\mu\sqrt{\fr{N}{2}}\ovrV + \sqrt{\fr{N}{2}}\mu^2
\end{multline*}
сходится к нулю по вероятности при $\mu\rightarrow 0$ $\left(\alpha>0\right)$. 
Итак, получаем достаточное требование на~$\alpha$: при $\alpha>1/2$ остается справедливым~(\ref{eq_weaklimriskS2}). 
При переходе от~$Y_i$ к~$V_i$ и при введенных ограничениях предельные распределения совпадают.

Теперь рассмотрим случай использования интерквартильного размаха. 
Пусть $Z_{N/2,p}$~--- выборочная квантиль порядка~$p$, построенная по $Y_{N/2+1},\ldots,Y_N$, 
а $Z_{N/2,p}'$~--- выборочная квантиль порядка~$p$, построенная по $V_1,\ldots,V_{N/2}$. Заметим, что почти всюду
\begin{equation}
\label{eq_quanDiff}
\left| Z_{N/2,p} - Z_{N/2,p}' \right| \leq 5 \max_{N/2+1\leq i\leq N} |\mu_i| \leq \fr{5\Upsilon}{N^{\alpha}}\,.
\end{equation}
Нужно, чтобы правая часть в~(\ref{eq_quanDiff}) стремилась к нулю при умножении на~$\sqrt{N/2}$. 
Тогда~$Z_{N/2,p}'$ можно использовать вместо~$Z_{N/2,p}$, и с учетом~(\ref{eq_Yi2Vi}) все сводится к 
случаю $\mu_i=0$, для которого уже все доказано (см.\ формулу~(\ref{eq_weaklimriskIQR})). 
Таким образом, достаточно потребовать $\alpha>1/2$.

Итак, если функция гладкая по Липшицу с положительным параметром  $\alpha>1/2$, 
то при использовании оценок дисперсии на основе~$S^2$ и интерквартильного размаха 
предельные распределения оказываются такими же, как и в случае $\mu_i=0$, $i=\overline{N/2+1,N}$.


{\small\frenchspacing
{%\baselineskip=10.8pt
\addcontentsline{toc}{section}{Литература}
\begin{thebibliography}{99}    

\bibitem{1mar}
\Au{Маркин А.\,В., Шестаков О.\,В.} 
О состоятельности оценки риска при пороговой обработке вейвлет-коэффициентов~// 
Вестник Моск.\ ун-та. Серия~15: Вычислительная математика и кибернетика (в пе\-чати).


\bibitem{3mar} %2
\Au{Mallat S.} 
A wavelet tour of signal processing.~--- Academic Press, 1999.

\bibitem{4mar} %3
\Au{Добеши И.} 
Десять лекций по вейвлетам.~--- Ижевск: НИЦ <<Регулярная и хаотическая динамика>>, 2001.

\bibitem{6mar} %4
\Au{Stein C.\,M.} 
Estimation of the mean of a multivariate normal distribution~// Annals Statistics, 1981. Vol.~9. No.\,6. P.~1135--1151.

\bibitem{5mar} %5
\Au{Donoho D.\,L., Johnstone I.\,M.}
 Adapting to unknown smoothness via wavelet shrinkage~// J. Amer. Statistical Association, 1995. 
 Vol.~90. P.~1200--1224.

 \bibitem{2mar} %6
\Au{Donoho D.\,L., Johnstone I.\,M.} 
Ideal spatial adaptation via wavelet shrinkage~// Biometrika, 1994. Vol.~81. No.\,3. P.~425--455.


\bibitem{7mar}
\Au{Ширяев А.\,Н.} Вероятность-1.~--- М.: МЦНМО, 2004.

\bibitem{8mar}
\Au{Боровков А.\,А.}
 Математическая статистика.~--- М.: Наука, 1984.

\bibitem{9mar}
\Au{Serfling R.} 
Approximation theorems of mathematical statistics.~--- John Wiley \& Sons, 1980.

\bibitem{10mar}
\Au{Van der Vaart A.\,W.} 
Asymptotic statistics.~--- Cambridge University Press, 2000.

\label{end\stat}

\bibitem{11mar}
\Au{DasGupta A.} 
Asymptotic values and expansions for the correlation between different measures of spread~// 
J.\ Statistical Planning Inference, 2006. Vol.~136. No.\,7. P.\,2197--2212.
 \end{thebibliography}
}
}
\end{multicols}    %pdf %4
%\renewcommand{\figurename}{\protect\bf Figure}

\def\stat{lukmorozov}


\def\tit{ON THE  OVERFLOW PROBABILITY ASYMPTOTICS IN~A~GAUSSIAN~QUEUE}

\def\titkol{On the  overflow probability asymptotics in a Gaussian
queue}

\def\autkol{O.\,V.~Lukashenko, E.\,V.~Morozov, and~M.~Pagano}

\def\aut{O.\,V.~Lukashenko$^{1,2}$, E.\,V.~Morozov$^{1,2}$, and~M.~Pagano$^3$}

\titel{\tit}{\aut}{\autkol}{\titkol}

%{\renewcommand{\thefootnote}{\fnsymbol{footnote}}
%\footnotetext[1] {The work of first and second  authors is partially supported by the
%Program of Strategy development of Petrozavodsk State University in
%the framework of the research activity. The third author is a
%postdoctoral fellow with the Research Foundation-Flanders
%(FWO-Vlaanderen).}}

\renewcommand{\thefootnote}{\arabic{footnote}}
\footnotetext[1]{Institute of Applied Mathematical
Research, Karelian Research Center, Russian Academy of Sciences,
11 Pushkinskaya Str., Petrozavodsk 185910,
Russian Federation}
\footnotetext[2]{Petrozavodsk State University, 33 Lenin Str., Petrozavodsk 185910,
Russian Federation}
\footnotetext[3]{University of Pisa, 43 Lungarno Pacinotti, Pisa 56126,
Italy}


\vspace*{-12pt}

\def\leftfootline{\small{\textbf{\thepage}
\hfill INFORMATIKA I EE PRIMENENIYA~--- INFORMATICS AND APPLICATIONS\ \ \ 2014\ \ \ volume~8\ \ \ issue\ 2}
}%
 \def\rightfootline{\small{INFORMATIKA I EE PRIMENENIYA~--- INFORMATICS AND APPLICATIONS\ \ \ 2014\ \ \ volume~8\ \ \ issue\ 2
\hfill \textbf{\thepage}}}




\Abste{Gaussian processes are a powerful tool in network modeling since they permit
to capture the long memory property of actual traffic flows. In more detail,
under realistic assumptions, fractional Brownian motion (FBM) arise as the
limit process when a huge number of on-off sources (with heavy-tailed sojourn
times) are multiplexed in backbone networks. This paper  studies  fluid queuing
systems  with a constant service rate fed by a  sum of independent FBMs,
corresponding to the aggregation of heterogeneous traffic flows. For such
queuing systems, logarithmic asymptotics of the overflow probability, an upper
bound for the loss probability in the corresponding finite-buffer queues, are
derived, highlighting that the FBM with the largest Hurst parameter dominates
in the estimation. Finally, asymptotic results for the workload maximum  in
the more general case of a Gaussian input with slowly varying at infinity
variance are given.}

\KWE{Gaussian fluid system; overflow probability;
logarithmic asymptotics}

\DOI{10.14357/19922264140203}


\vskip 10pt plus 9pt minus 6pt

      \thispagestyle{myheadings}

      \begin{multicols}{2}

                  \label{st\stat}



\section{Introduction}

\noindent
Gaussian processes are well-recognized models to describe the
traffic dynamics of a wide class of modern telecommunication
networks. The main motivation to apply these  models is
their ability of capturing, in a simple and parsimonious way,
the properties of self-similarity and long-range dependence, which are
inherent in multimedia
network traffic~\cite{Leland, Willinger}. Self-similarity means that
the distribution of the process remains unchanged under suitable
scaling of time and space, while long-range dependence implies a
slow decay of the autocorrelation function. These properties make
difficult the probabilistic analysis and, as a consequence, to
obtain  key performance characteristics, crucial  to evaluate   the Quality
of Service (QoS) provided  by the considered networks,
 in an explicit form.

The FBM
is one of the most studied self-similar long-range dependent
Gaussian processes. Its use as a traffic model is supported  by the
following theoretical analysis~\cite{Taqqu}:
the sum of an increasing  number of the so-called on-off inputs,
with either on-times or off-times having a heavy-tailed distribution
with infinite variance, converges weakly to an FBM, after an
appropriate time  scaling. If an FBM is  the  input to a queueing
system, then  let call it  fractional Brownian (FB) input.


One of the main characteristics of the queueing systems is the  {\it
overflow probability}, i.\,e.,
the probability that the workload process exceeds a finite threshold.

In  Gaussian queueing systems with infinite buffer, the
analysis of the overflow probability (closely related  to the
workload maximum) is reduced to the analysis of the extremes of
Gaussian processes~\cite{Reich}.

There are  no explicit expressions for the overflow probability in
queueing  systems with general Gaussian input (including FB input),
while   a few   asymptotic results    are available. In this regard,
let  mention the following key works~[5--7].
It is important to stress that in a general setting, the asymptotic
analysis of the overflow probability is
 based on a number of the assumptions which sometimes are  difficult
to verify. The loss probability in the Gaussian queues with the FB
input  and a finite buffer  is  studied in~[8--12]. Also,  let mention  closely related
 works~[13--17], where the maximum of the
workload process is studied.  Since explicit analysis is
unavailable in general case, the numerical analysis of the
overflow probability  presented in~[18--22]  plays an important
role in the studying of the Gaussian queueing systems. Note that
analysis of the systems with the Brownian input is much easier
because this process has independent increments. This property
allows to obtain the tail probability  for the maximum of the
Brownian motion~\cite{Takacs} in an explicit form. In turn, this
result is directly connected with the overflow probability in the
queueing system fed by the Brownian input.

In this paper, the authors first present the asymptotic analysis of the
overflow probability in the  queueing system where the input is a
sum of the independent FBMs. Thus, they extend the result which has
been proved in a seminal work~\cite{Duffield} for   only FB input.
The present authors follow  mainly the approach developed in~\cite{Duffield} and
discuss in brief  inevitable differences in the proofs. For this
reason, the  proof is straightforward and more transparent than that
can be extracted from  the related works~\cite{Debicki2,Duffy},
where generalizations of the basic model from~\cite{Duffield} are
studied. In particular,  the proof in~\cite{Duffy}
is based on a number of rather complicated  assumptions some of
which, as was mentioned, are not easy to verify for the specific
models.

In summary, the present work  reviews the main  results on the
asymptotics of the overflow probability  in Gaussian  queueing
systems
and also discusses some new results on the workload maximum.

The  paper is organized as follows.  Section~2 contains the description
of Gaussian queueing systems, while section~3 presents the proof of the
overflow probability asymptotics for the
superposition of independent identically distributed  (i.i.d.)\
FB inputs.  Finally, in section~4, the  results concerning the workload
maximum,  when  the input process belongs to  a wide class of the
Gaussian processes, are  analyzed.

\vspace*{-6pt}


\section{Theoretical Background}

\noindent
First of all, the authors motivate their interest to Gaussian queueing  systems.
To this aim, they consider~$N$ i.i.d.\ {\it on-off sources}, modelling
the traffic flows generated by independent connections.  Each  source~$k$ is
described by the process $\{I_k(t)$, $t \geq 0\}$, $k=1,\ldots ,N$,
where
\begin{equation*}
I_k(t)=\begin{cases}
 1\,, &\ t\in \mbox{ on-period}\,; \\
 0\,, &\ t\in \mbox{ off-period.} \\
\end{cases}
%\label{Luk-l1}
\end{equation*}
During an {\it on-period}, a source  is active, while  it keeps
silence (inactive) during the following {\it off-period}.  The
on-off periods are i.i.d.\ and  form an {\it alternating renewal
process}. Furthermore, the processes formed by different sources are
assumed to be independent. As a result, the aggregated traffic
(cumulative workload) generated by all $N$ sources during
time interval $[0,t]$ is given by
\begin{equation*}
A_N (t):=\int\limits_0^{t} \left( \sum\limits_{k=1}^N {I_{k}(u)}
\right)du\,.
\end{equation*}
It is assumed  that there are $M$ types of sources , and  $N_i$ is
the number of the $i$th type sources, $i=1,\ldots ,M$, so  $\sum\limits_{i=1}^M
N_i=N$. The statistical behavior of the cumulative workload
crucially depends on the distribution of on-off periods. Let
$F_{\mathrm{on}}^i,\, F_{\mathrm{off}}^i$ be the distribution of on- and off-period,
respectively. Let assume that the following conditions hold:
\begin{equation}
\left.
\begin{array}{rl}
1-F_{\mathrm{on}}^i(x)& \sim  \ell_{\mathrm{on}}^i x^{-\alpha_{\mathrm{on}}^i}L_{\mathrm{on}}^i(x)\,;
\\[9pt]
1-F^i_{\mathrm{off}}(x)& \sim  \ell_{\mathrm{off}}^i
x^{-\alpha_{\mathrm{off}}^i}L_{\mathrm{off}}^i(x),\,\,\,x\to \infty\,,
\end{array}
\right\}
\label{3}
\end{equation}
where $\ell_{\mathrm{on}}^i$ and $\ell_{\mathrm{off}}^i$ are the
positive constants; exponents
$\alpha_{\mathrm{on}}^i,\alpha_{\mathrm{off}}^i\in (1,\,2)$;
and functions $L_{\mathrm{on}}^i$ and
$L_{\mathrm{off}}^i$ are slowly varying at infinity, i.\,e., for any $t >0$,
$$
\lim\limits_{x \to \infty} \fr{L^i(tx)}{L^i(x)}=1\,,\enskip i=1,\ldots,M\,.
$$
(Relation  $a\sim b$ means that $a/b\to 1$.) Indeed, conditions~(\ref{3})
mean that the distributions $F_{\mathrm{on}}^i$ and  $F_{\mathrm{off}}^i$
are {\it heavy-tailed}. For each~$i$, denote by $\mu_{\mathrm{on}}^i$,
$\mu_{\mathrm{off}}^i$ the mean length of  on-  and off-period, respectively
(note that $\mu_{\mathrm{on}}^i$ and $\mu_{\mathrm{off}}^i<\infty$ because
$\alpha_{\mathrm{on}}^i$ and $\alpha_{\mathrm{off}}^i>1$).
It has been  shown in~\cite{Taqqu}  that the scaled cumulative workload arrived during
period $[0,\,Tt]$ converges weakly to a sum of independent
FBMs provided that:
\begin{enumerate}[($i$)]
\item $N_i\to \infty$ such that
 $\lim\limits_{N\to \infty}N_i/N>0$ for each~$i$;
 and
 \item the scaling factor $T\to \infty$.
 \end{enumerate}
 This functional limit theorem leads to the following approximation:
\begin{multline*}
A(tT)\approx T\left( \sum\limits_{i=1}^M N_i
\fr{\mu_{\mathrm{on}}^i}{\mu_{\mathrm{on}}^i+\mu_{\mathrm{off}}^i} \right)t\\
{} + \sum\limits_{i=1}^M
T^{H_i} \sqrt{L_i(T)N_i}c_i B_{H_i}(t)
\end{multline*}
where $c_i$ are the positive constants; $L_i$ are the slowly varying at
infinity functions (expressed in the terms of given  parameters); and
$B_{H_i}$ are the independent FBMs with the Hurst parameters~$H_i$
 given by
$$
H_i=\fr{3-\min(\alpha_{\mathrm{on}}^i,\,\alpha_{\mathrm{off}}^i)}{2}\in
\left(\fr{1}{2},\,1 \right)\,,\enskip i=1,\ldots, M\,.
$$
Thus, the aggregated traffic generated by a large number of
i.i.d.\ heavy-tailed on-off sources is approximated by a
superposition of  the independent FBMs with a linear drift. This
result gives a motivation to consider  a  queueing system  fed by a
sum of independent FBMs as  a suitable model for a wide class of
modern telecommunication systems.

Now, let describe  a {\it fluid queue} with a constant service rate~$C$
driven by the input process $\{A(t),\,t\ge 0\}$ which is defined
as follows:
 $$
 A(t)=mt+X(t)
 $$
 where $m>0$ is the  mean input rate and the process $X:=\{X(t)\}$
 is the  sum of~$M$ independent
 FB inputs such that  the $i$th summand  has
the   Hurst parameter $H_i \in (1/2,1)$.
 Obviously, $A(t)$ describes the amount of data (workload)
arrived into a communication node within time interval $[0, t]$.
Thus, the variance of the input process in  interval $[0, t]$ is
 $$
 v(t)=\sum\limits_{i=1}^M t^{H_i}\,.
 $$
Introduce the parameter
%{\it traffic intensity}
$r:=C-m$. Denote
$W(t)=X(t)-rt$ and  let  $Q(t)$ be the current workload  at instant~$t$.
If $Q(0)=0$, then the workload $Q(t)$ satisfies the  following
equation~\cite{Reich}:
\begin{multline*}
Q(t)=_d \sup\limits_{0 \leq s \leq t}(A(t)-A(s)-C(t-s))\\
{}= \sup\limits_{0 \leq s \leq t}(X(t)-X(s)-r(t-s))\\
{}=\sup\limits_{0 \leq s \leq t}(W(t)-W(s))
%\label{6a}
\end{multline*}
where symbol $=_d$ stands for the equality in distribution.
 If, moreover, $r>0$, then the
system is stable and a stationary  workload process~$Q$ exists such
that~\cite{Mandjes}:
\begin{multline}
Q=_d \sup\limits_{t \in \T} \left( A(t)-Ct \right)\\
{}= \sup\limits_{t \in \T} W(t)\;\;\; (\T=\Z_+\, \mbox{or}\,\,
\T=\R_+)\,.
\label{asymp-l1}
\end{multline}
Hence, for an  arbitrary threshold $b\in [0,\,\infty)$, the {\it overflow
probability} is defined as
\begin{equation*}
\P(Q>b)=\P\left( \sup\limits_{t \in \T} W(t) >b \right).
%\label{largebuff-l1}
\end{equation*}
It is worth mentioning that some nonasymptotic upper bounds for the
overflow probability have been proposed.
For instance, in case of ordinary FB input ($H$ is the Hurst parameter)
and $\T=\Z_+$, it was shown that~\cite{Fidler1, Fidler2}
$$
\P(Q>b)\le \fr{\Gamma\left( 1/(2\beta) \right)}{2\beta(-\log \eta)^{{1}/(2\beta)}}
$$
where $\Gamma$ denotes the Gamma-function, $\beta \in (0,1-H)$ is the
free parameter and
\begin{multline*}
\eta=\exp\left( -\fr{1}{2} \left( \fr{C-m}{H+\beta} \right)^{2(H+\beta)}\right.\\
\left.{}\times
\left( \fr{b}{1-(H+\beta)} \right)^{2-2(H+\beta)} \right).
\end{multline*}
This result can be extended to general Gaussian inputs, but the value of~$\eta$
can be estimated only by numerical methods (see~\cite{Luk0}).

\vspace*{-6pt}

\section{Asymptotics of~the~Overflow Probability for~a~Superposition of~Fractional
Brownian Inputs}

\noindent
The following result shows that the  FB input  with the largest
Hurst parameter dominates in the asymptotic analysis of the overflow
probability. Recall that the  presented proof is mainly based on
the technique developed in~\cite{Duffield} where
a  system with  single FB input process has been  analyzed.

\smallskip

\noindent
\textbf{Theorem 3.1.}\
\textit{For the  stationary workload~$(\ref{asymp-l1})$, the following
asymptotic holds}:

\noindent
\begin{multline*}
\lim_{b \to \infty}b^{2H-2}\log \P (Q>b)\\
{}=-\fr{r^{2H}}
{ 2H^{2H}(1-H)^{2(1-H)} }:=-\Theta
%\label{ar-l5}
\end{multline*}
\textit{where} $H=\max(H_1,\ldots ,H_M).$

\smallskip

\noindent
P\,r\,o\,o\,f.\ \ Consider the following  relations:

\noindent
\begin{multline}
\P(W(t)/t>x)=\P\left(\Nor(0,1)>
\fr{(x+r)t}{\sqrt{v(t)}} \right)\\
{}=\Psi\left( \fr{(x+r)t}{\sqrt{v(t)}} \right)
\label{ar-l2}
\end{multline}
where

\noindent
$$
\Psi(x):=\fr{1}{\sqrt{2\pi}}\int\limits_x^\infty e^{-y^2/2}dy
$$
is the tail distribution of the standard normal variable
$\Nor(0,1)$.   Function~$\Psi$ satisfies   the following
inequalities~\cite{Mandjes} for   $x>0$:

\noindent
\begin{equation*}
\fr{1-x^{-2}}{x\sqrt{2\pi}}\, e^{-{x^2}/2} \le \Psi(x) \le
\fr{1}{x\sqrt{2\pi}}\, e^{-x^2/2},
%\label{norm-l2}
\end{equation*}
which, in turn, imply  the approximation

\noindent
\begin{equation}
\log \Psi(x) \sim -\fr{x^2}{2}\,,\enskip x \to \infty\,.
\label{ar-l1}
\end{equation}
Denote

\noindent
$$
\nu(t)=\fr{t^2}{v(t)}
$$
and note that $\nu(t) \sim t^{2-2H} \to \infty$ as $t \to \infty$.
It now follows from~\eqref{ar-l1} and~\eqref{ar-l2} that the
following limit exists:

\vspace*{-6pt}

\noindent
\begin{multline}
\lim\limits_{t \to \infty} \fr{1}{\nu(t)}\,\log \P\left( W(t)/t >x
\right){}\\
{}=-\fr{1}{2}\left(x+r\right)^2:=-\lambda(x)\,.
\label{ar-l4}
\end{multline}
It is easy to check that
$$
\inf\limits_{c>0}c^{2H-2}\lambda(c)=\Theta\,.
$$
Let emphasize  that, in contrast to this  straightforward
analysis, the proof of~\eqref{ar-l4} in~\cite{Duffield, Duffy}
(obtained for general non-Gaussian case)  is  based on a large
deviation principle and  some  technical conditions placed on the
logarithmic moment generating function.

To prove the statement of theorem~3.1, it suffices to establish the
following lower and upper bounds:
\begin{align}
\liminf_{b \to \infty}\fr{\log \P (Q>b)}{\nu(b)}&\geq
-\inf\limits_{c>0}c^{2H-2}\lambda(c)\,;\label{asymp-l2}\\
\limsup\limits_{b \to \infty}\fr{\log \P (Q>b)}{\nu(b)}&\leq -\inf_{c>0}
c^{2H-2}\lambda(c) \label{asymp-l3}\,.
\end{align}
First, let note that for each $c>0$,
\begin{multline*}
\liminf\limits_{b \to \infty}\fr{\log \P (Q>b)}{\nu(b)} \geq
\liminf\limits_{b \to \infty} \fr{\log \P \left( W(b/c)>b \right)}{\nu(b)}\\
{}=\lim_{t \to \infty}\fr{ \log \P (W(t)/t >c)}{\nu(t\,c)}
= -c^{2H-2}\lambda(c)
\end{multline*}
and, thus,  the lower bound follows.

The proof of the upper bound is  more challenging. First, let consider the
 discrete time case $\T=\Z_+$ and then  verify some technical conditions
to extend this  result to continuous time case $\T=\R_+$.

Consider an arbitrary $d>0$; then, one obtains

\noindent
\begin{multline}
\P (Q>b) \leq \P \left( \sup_{n<b/d}W(n)>b \right)\\
{}+\P \left(
\sup\limits_{n \geq b/d}W(n)>b \right)\\
{}\leq \fr{b}{d}\sup_{c>d}\P (W(b/c)>b) +\sum\limits_{n \geq b/d}\P
(W(n)>b).
\label{asymp-l7}
\end{multline}
Denote

\noindent
\begin{equation}
\left.
\begin{array}{c}
f_1(b)=\displaystyle\fr{b}{d}\sup\limits_{c>d}\P (W(b/c)>b)\,;\\[9pt]
f_2(b)=\displaystyle\sum\limits_{n \geq b/d}\P (W(n)>b)\,.
\end{array}
\right\}
\label{16}
\end{equation}
It is easy to check that if  $(b+rk)/\sqrt{v(k)}>1$, then
\begin{multline}
\P(W(k) >b) = \Psi \left( \fr{b+rk}{\sqrt{v(k)}} \right) \le
\exp \left( -\fr{(b+rk)^2}{2v(k)} \right)\\
{}\le \exp\left(-\fr{r^2}{2}\nu(k)\right)\,.
\label{ar-l3}
\end{multline}
Also, note   that  for $t\ge1$,
\begin{equation}
\nu(t)=\fr{t^2}{\sum\limits_{i=1}^M t^{H_i}}\geq
\fr{1}{M}t^{2-2H}\,.
\label{logbuff-l13}
\end{equation}
Denote

\noindent
$$
 a=2-2H>0\,,\enskip \gamma=\fr{r^2}{2M}\,.
 $$
It  then follows from~\eqref{ar-l3} and~\eqref{logbuff-l13} that

\columnbreak

\noindent
\begin{multline}
\limsup\limits_{b \to \infty} \fr{\log f_2(b)}{\nu(b)}\\
{} \le M\limsup\limits_{b \to
\infty}\fr{1}{ b^{a}}\,\log \left[\sum\limits_{k=\lfloor b/d
\rfloor}^\infty e^{-\gamma k^a}\right]\,.
\label{asymp-l4}
\end{multline}
 Note that if $k \geq \lfloor b/d \rfloor$ and $k-1\leq x \leq k$,
then the inequality $ e^{-\gamma k^a}\leq e^{-\gamma x^a}$ holds. Hence,

\vspace*{-2pt}

\noindent
\begin{multline}
\sum\limits_{k=\lfloor b/d \rfloor}^\infty e^{-\gamma k^a} \leq
\int\limits_{\lfloor b/d \rfloor-1}^\infty e^{-\gamma x^a}dx\\[-1pt]
{} \leq
\int\limits_{b/d-2}^\infty e^{-\gamma x^a}dx\,.
\label{asymp-l5}
\end{multline}
It follows from~(\ref{asymp-l4}) and~(\ref{asymp-l5}) that

\vspace*{-2pt}

\noindent
\begin{multline}
\limsup\limits_{b \to \infty}\fr{ \log f_2(b)}{\nu(b)}\\
{}\leq M
\limsup\limits_{b \to \infty}\fr{1}{ b^{a}}\,\log
\left[ \int\limits_{b/d-2}^\infty
e^{-\gamma x^a}dx\right]
\label{ar-l6}
\end{multline}
and by applying  the L'H$\hat{\mbox{o}}$pital's rule twice, one obtains the
following limit:
\vspace*{-2pt}

\noindent
\begin{multline*}
\lim\limits_{b \to \infty} \fr{1}{b^a}\log \int\limits_{b/d-2}^\infty e^{-\gamma x^a}dx\\
{} =-\fr{1}{da}\,\lim\limits_{b \to \infty} \left
[e^{-\gamma(b/d-2)^a}b^{1-a} \fr{1}{\int\limits_{b/d-2}^\infty
e^{-\gamma x^a}dx}\right]\\
{}=-\fr{\gamma}{d}\,\lim\limits_{b \to \infty}\left(
\fr{b/d-2}{b} \right)^{a-1} =-\gamma d^{-a}\,.
\end{multline*}
 Now, let  choose  $d \in \left(0,\,((1-H)r)/H \right)$ such  that
$$
-\gamma d^{-a}\le -\inf\limits_{c>0} c^{-a}\lambda(c)\,.
$$
It then follows from~\eqref{ar-l6} that

\noindent
\begin{equation}
\hspace*{-3mm}\limsup\limits_{b \to \infty} \fr{\log\left[ \sum\limits_{n \geq b/d} \P
(W(n)>b)\right]}{\nu(b)}  \leq -\inf\limits_{c>0}\fr{\lambda(c)}{c^{a}}.\!\!
\label{asymp-l8}
\end{equation}
Consider the term~$f_1$ from~(\ref{16}) and note that
\begin{multline*}
\limsup\limits_{b \to \infty}\fr{1}{\nu(b)}\,\log f_1(b)\\
{}=
\limsup\limits_{b \to \infty}\fr{1}{\nu(b)}\,\log \left[
\fr{b}{d}\sup\limits_{c>d} \P
\left(W(b/c)>b\right) \right]
\end{multline*}


\noindent
\begin{multline*}
{}=\limsup\limits_{b \to \infty}\fr{1}{\nu(b)}\,\log \left[\sup\limits_{c>d} \P
\left(W(b/c)>b\right)\right]\\
{}=\limsup\limits_{b \to \infty}\,\sup\limits_{c>d}\fr{1}{\nu(b)}\,\log  \P
\left(W(b/c)>b\right)\\
{}=\limsup\limits_{n \to \infty}\,\sup\limits_{c>d}\fr{1}{\nu(nc)}\,\log \P
(W(n)/n>c)\,.
 \end{multline*}
 By~\eqref{ar-l4}, for   any given $\delta>0$, let
 choose  sufficiently large~$n$ such that
 $$
 \log \P (W(n)/n>x) \leq \nu(n) (\delta -\lambda(x))\,.
 $$
 Using the last inequality,
 \begin{multline}
\limsup\limits_{b \to \infty}\fr{\log f_1(b)}{\nu(b)} \leq
\limsup\limits_{n \to \infty}\sup\limits_{c>d}
\fr{\nu(n)}{\nu(cn)}\left[\delta-\lambda(c)\right]\\
{}=\limsup\limits_{n \to \infty} \sup\limits_{c>d}\left[ \fr{\nu(n)
\delta}{h(cn)}-\fr{\lambda(c)\nu(n)}{\nu(cn)} \right]\\
{}\leq\limsup\limits_{n \to \infty} \left[ \sup\limits_{c>d}\fr{\nu(n)
\delta}{h(cn)}-\inf\limits_{c>d}\fr{\lambda(c)\nu(n)}{\nu(cn)}
\right]\\
{}=-\limsup\limits_{n \to \infty}\inf_{c>d}\fr{\lambda(c)\nu(n)}{\nu(cn)}+
\limsup\limits_{n \to \infty}\fr{\nu(n) \delta}{\nu(dn)}\\[3pt]
{}=-\limsup\limits_{n \to \infty}\inf\limits_{c>d}\fr{\lambda(c)\nu(n)}{\nu(cn)}+
\fr{\delta}{ d^{a}}\,.\label{22}
\end{multline}
Consider the  following function:
$$
f(x):=\fr{1}{2}\left(x+r\right)^2 x^{2H-2}\,,\enskip x \geq 0\,.
$$
It is easy to check that $\min f(x)$ is attained at the point
$x=(1-H)r/H>0$. Thus,  for any $d \in \left(0,\,((1-H)r)/H\right)$,
$$
\inf\limits_{c>0} f(c)=\inf\limits_{c>d} f(c)\,.
$$
Moreover,
\begin{multline*}
\inf\limits_{c>d} \fr{\lambda(c)\nu(t)}{\nu(ct)}=\inf\limits_{c>d}
\fr{(c+r)^2 \sum_{i=1}^M c^{2H_i-2}t^{2H_i}}{2\sum_{i=1}^M t^{2H_i}}
\\
{}\geq \fr{\sum_{i=1}^M \inf_{c>d} t^{2H_i} (c+r)^2  c^{2H_i-2}}{2\sum_{i=1}^M t^{2H_i}}\\
{}= \fr{\sum_{i=1}^M t^{2H_i} f\left( ((1-H_i)r)/H_i\right)}
{\sum_{i=1}^M t^{2H_i}}\,.
\end{multline*}
It now easily follows  as $t\to \infty$ that
\begin{multline}
 \fr{\sum_{i=1}^M t^{2H_i} f\left( ((1-H_i)r)/H_i\right)}
 {\sum_{i=1}^M t^{2H_i}}\to
\fr{1}{2}\,f\left( \fr{(1-H)r}{H} \right)\\
{} =
\inf\limits_{c>d}c^{2H-2}\lambda(c)\,.
\end{multline}
Thus,  let obtain  from~(\ref{22})  that
\begin{equation*}
\limsup\limits_{b \to \infty}\fr{\log f_1(b)}{\nu(b)} \le - \inf\limits_{c>0}
c^{2H-2}\lambda(c)+\fr{\delta}{ d^{a}}\,,
\end{equation*}
and because $\delta$ is arbitrary,
\begin{equation}
\limsup\limits_{b \to \infty}\fr{\log f_1(b)}{\nu(b)} \leq -\inf\limits_{c>0}
c^{2H-2}\lambda(c)\,.
\label{asymp-l11}
\end{equation}
Now, let take  into account the following  inequality
\begin{multline}
\limsup\limits_{b \to \infty}\fr{\log \left(f_1(b)+f_2(b)\right)}{h(b)}\\
{}\leq \limsup\limits_{b \to \infty}\fr{ \log\left(\max(f_1(b),f_2(b))\right)+
\log 2}{h(b)}\,.
\label{asymp-l12}
\end{multline}
Then, let combine~(\ref{asymp-l12})  with~(\ref{asymp-l7}),
(\ref{asymp-l8}),  and~(\ref{asymp-l11}) to obtain~(\ref{asymp-l3}).
In turn, inequalities~(\ref{asymp-l2}) and~(\ref{asymp-l3})  imply
$$
\lim\limits_{b \to \infty}  \fr{\log \P
(Q>b)}{\nu(b)}=-\inf\limits_{c>0}c^{2H-2}\lambda(c)\,,
$$
and the proof for $\T=\Z_+$ is completed.


\smallskip

To consider the case  $\T=\R_+$, let  define the process
$\{W^*(n),\,n \in \Z_+\}$ as
$$
W^*(n)=\sup\limits_{0 \le s \le 1} W(n+s)\,.
$$
Note that
$$
\P \left( \sup\limits_{t \in \R_+}W(t) >b\right)=
\P \left( \sup\limits_{n \in \Z_+} W^*(n) >b \right)\,.
$$
Thus, the asymptotics for the process $\{W^*(n),\, n \in \Z_+\}$
implies  the asymptotics for the continuous-time process $\{W(t),\,
t \in \R_+\}$. The lower bound  follows from the  obvious inequality:
$$
\P\left(\sup\limits_{n \in \Z_+} W^*(n)>b\right) \ge
\P\left(\sup\limits_{n \in \Z_+} W(n)>b\right)\,.
$$
Denote
$$
Y(s)=W(n+s)-W(n)\,, \enskip s \geq 0\,,\ \ n \in \Z_+\,,
$$
 and note that $\E Y(s)=-rs$.  Applying  Borell--Sudakov--Tsirelson
 inequality~\cite{Adler, Debicki}, one obtains
\begin{multline}
\P \left( \sup\limits_{0 \leq s \leq 1} Y(s) \geq u \right) \leq
\P \left( \sup\limits_{0 \leq s \leq 1} \left(Y(s)
+rs \right)\geq u \right)\\
{}\leq 2 \P\left( \Nor(a,\sigma^2)>u \right)
\label{asymp-l13}
\end{multline}
where
\begin{align*}
a&:=\mathrm{med}\, \left( \sup\limits_{0 \leq s \leq 1}\left(Y(s) +rs
\right)\right)\,;\\
\sigma^2&:=\sup\limits_{0 \leq s \leq 1} \mathbb{D} Y(s)\,.
\end{align*}
Denote  $\varphi(n)=\theta\nu(n)/n$. It now follows from~(\ref{asymp-l13})
 that
\begin{multline*}
\E \exp\left[\varphi(n)\left(W^*(n)-W(n)\right)\right]\\
{}=
\E \exp\left[\varphi(n)\sup\limits_{0 \leq s \leq 1}Y(s)\right]
\leq 2 \E \exp\left[\varphi(n) \Nor(a,\sigma^2)\right]\\
{}=\exp\left[\fr{\sigma^2\varphi^2(n)+2\varphi(n) a}{2}\right].
\end{multline*}
Thus, one obtains
\begin{multline}
\limsup\limits_{n \to \infty}\fr{1}{\nu(n)}\,\log \E
e^{\theta\nu(n)\left(W^*(n)-W(n)\right)/n} \\
{}\leq \limsup\limits_{n \to \infty}
\fr{\sigma^2\varphi^2(n)+2\varphi(n) a}{2\nu(n)}\\
{}= \limsup\limits_{n \to \infty} \fr{\sigma^2\theta^2\nu(n)+2\theta an}
{2n^2}=0\,.
\label{Duff2}
\end{multline}
Note that the statement
\begin{equation}
\limsup\limits_{n \to \infty}\fr{1}{\nu(n)}\,\log \E
e^{\theta\nu(n)\left(W^*(n)-W(n)\right)/n}=0
\label{ar-l7}
\end{equation}
 is used  as  an assumption  in  the  paper~\cite{Duffield},
while  above,  a detailed proof of~(\ref{ar-l7}) is given for the system with
Gaussian input.  By the H$\ddot{\mbox{o}}$lder's inequality,  one obtains for
$1<p,g<\infty$, ${1}/{p}+{1}/{q}=1$ that
\begin{multline}
\lambda_1(\theta) := \limsup\limits_{n \to \infty } \fr{1}{\nu(n)}\,
\log \E e^{\theta \nu(n) W^*(n)/n}\\
{}= \limsup\limits_{n \to \infty}\fr{1}{\nu(n)}\\
{}\times \log \E e^{\theta
 \nu(n)(W^*(n)-W(n))/n}e^{\theta \nu(n) W(n)/n}
\\
{}\leq \limsup\limits_{n \to \infty} \fr{1}{\nu(n)}\,\log \left\{ \left[ \E
e^{\theta q \nu(n) (W^*(n)-W(n))/n} \right]^{1/q}\right.\\
\left.{}\times \left[ \E
e^{\theta p \nu(n) W(n)/n } \right]^{1/p} \right\}\\
{}= \fr{1}{q}\,\limsup\limits_{n \to \infty} \fr{1}{\nu(n)}\,\log \E
e^{(\theta q\nu(n) \left(W^*(n)-W(n)\right))/n}\\
{}+
\fr{1}{p}\,\limsup\limits_{n \to \infty} \fr{1}{\nu(n)}\, \log
\E e^{(\theta p \nu(n) W(n))/n}
\label{logbuff-l9}
\end{multline}

\vspace*{-12pt}

\noindent
\begin{equation*}
{}\leq \limsup\limits_{n \to \infty}\fr{1}{\nu(n)} \,\log \E
e^{\theta p \nu(n) W(n)/n}=\fr{1}{2}\left(\theta p\right)^2 - \theta p r
\label{logbuff-l10}
\end{equation*}
where  the first term in~(\ref{logbuff-l9}) equals zero
by~(\ref{Duff2}). Taking $p \to 1$, one gets
$$
\lambda_1(\theta)\leq \fr{1}{2}\,\theta^2 - \theta r\,.
$$
On the other  hand,

\noindent
\begin{multline*}
\liminf\limits_{n \to \infty} \fr{1}{\nu(n)}\, \log \E e^{\theta \nu(n)
W^*(n)/n} \\
{}\geq \liminf\limits_{n \to \infty}\fr{1}{\nu(n)}\, \log \E
e^{\theta \nu(n) W(n)/n}=\fr{1}{2}\theta^2 - \theta r\,.
\end{multline*}
Thus, the following limit exists:

\noindent
$$
\lim\limits_{n \to \infty} [\nu(n)]^{-1} \log \E e^{\theta \nu(n)
W^*(n)/n}=\fr{1}{2}\,\theta^2 - \theta r=\lambda_1(\theta)\,.
$$
It now follows  by the G$\ddot{\mbox{a}}$rtner--Ellis theorem~\cite {Dembo}
that the sequence  $\{W^*(n)/n,$ $\, \nu(n)\}$  satisfies a large
deviation principle with the following rate function $\lambda$ which
is   the Fenchel--Legendre transform of function~$\lambda_1$:

\noindent
\begin{multline*}
\sup\limits_{\theta \in \R}\{\theta x-\lambda_1(\theta)\}=
\sup\limits_{\theta \in \R}\left\{ -\fr{\theta^2}{2}+(x+r)\theta
\right\}\\
{}=\fr{1}{2}\left(x+r\right)^2:=\lambda(x)\,.
%\label{logbuff-l11}
\end{multline*}
(For more details, see~\cite{Dembo}.) As a consequence, one obtains
$$
\lim\limits_{n \to \infty} \fr{\P (W^*(n)/n > x)}{\nu(n)} =-\lambda (x)\,.
$$
Thus,   equation~\eqref{ar-l4} for the process $\{ W^*(n) \}$ is
proved.

\smallskip

To establish   the upper bound,  let  repeat
steps~\eqref{asymp-l7}--\eqref{asymp-l12} with $W(n)$ replaced by
$W^*(n)$. The proof
remains unchanged with exception of the point, where another
arguments are used to come to the  upper bound~\eqref{ar-l3}. More exactly,
at this point, Chernoff's  inequality is applied which gives, for any
$\varepsilon>0$,

\noindent
\begin{multline*}
\P \left(W^*(k)>b \right) \leq e^{-\sup\limits_{\theta}\left( \theta b -\log \E
e^{\theta W^*(k)} \right)}\\
{}\leq  e^{-\theta \nu(k) b/k}e^{(\lambda_1(\theta)+\varepsilon)\nu(k)}
\leq e^{(\lambda_1(\theta)+\varepsilon)\nu(k)}
\end{multline*}
where $\theta$ is chosen in such a way  that
$\lambda_1(\theta)+\varepsilon<0$.~\hfill$\square$

\vspace*{-6pt}


\section{Closely Related Results}

\noindent
In this section,  a number of results for
Gaussian queueing systems are discussed in brief which are closely connected  with the
asymptotics of the overflow probability  analyzed above.
  More exactly,  the   asymptotics   for the workload
 maximum are obtained
in the more general case when the variance $v$ of the input process~$X$
is regu-\linebreak\vspace*{-12pt}

\pagebreak

\noindent
larly varying at infinity function with index $0<V<2$, i.\,e.,
for any $y>0$,
$$
\lim_{t \to \infty} \fr{v(yt) }{v(t)}=y^V.
$$
The asymptotic has the following form~\cite{Debicki2, Duffy}
\begin{equation*}
\lim_{b \to \infty} \fr{v(b)}{b^2} \,\log \P(Q>b)=-\Theta
%\label{asymp1-l13}
\end{equation*}
where
\begin{eqnarray}
\Theta=\fr{2}{(2-V)^{2-V}}\left( \fr{r}{V} \right)^V\,.
\label{reg}
\end{eqnarray}
It is well-known that every regularly varying at infinity function can be
represented~as
\begin{equation}
v(t)=t^V L(t)
\label{MEV:1}
\end{equation}
 where function $L(t)$  is slowly varying (as $t\to\infty$) and
index $V\in (0,\,2)$~\cite{Seneta}. Denote $\beta=(2-V)^{-1}$ and
take (arbitrary) $\varepsilon \in (0,2-V)$. Moreover, it is assumed
that the following conditions hold as $t \to \infty$:
\begin{equation}
L(tL^\beta(t)) \sim L(t)\,;
\label{MEV:2}
\end{equation}
 function $L(t)$ is twice differentiable on~$\mathbb{R}_+$ and
\begin{equation}
L''(t)=o\left( \fr{1}{t^{V+\varepsilon}} \right)\,.
\label{MEV:3}
\end{equation}
It follows from~\eqref{MEV:3}  that
\begin{equation}
v''(t)\log t \to 0\,,\enskip t \to \infty\,.
\label{2.2.l20}
\end{equation}
Also, recall that a stationary version $Q^*(t)$ of the workload
process $Q(t)$ exists~\cite{MEV:Konstantopoulos}. Let
\begin{gather*}
\gamma(t)=L[\left(\log t \right)^\beta]  \log t \,;
\\
M(t):=\max\limits_{0 \leq s \leq t}Q(s)\,;\enskip M^*(t):=\max\limits_{0 \leq s \leq t}Q^*(s)\,.
\end{gather*}
The following asymptotic result for the workload maximum  has been
established in~\cite{Luk1} (see also~\cite{Luk3, Luk2}).

\smallskip

\noindent
\textbf{Theorem~4.1.}\
\textit{If the variance $v(t)$ of the Gaussian component~$X$ satisfies
conditions~$(\ref{MEV:1})$--$(\ref{MEV:3})$ and $r>0$, then
\begin{equation}
\frac{M^*(t)}{\gamma^\beta(t)} \Rightarrow
\left(\fr{1}{\Theta}\right)^\beta;\,\,
\frac{M(t)}{\gamma^\beta(t)} \Rightarrow
\left(\fr{1}{\Theta}\right)^\beta,\,\,\,t \to \infty\,,
\label{MEV:4}
\end{equation}
where the constant $\Theta$ is given by expression \eqref{reg},  and
$\Rightarrow$ stands for convergence in probability}.


\smallskip

\noindent
P\,r\,o\,o\,f.\ \  Let give a sketch of the proof which  mainly follows the
technique developed in~\cite{Zeevi}. It is sufficient to prove that
for any $\delta>0$,
\begin{align}
&\P \left( \fr{M^*(t)}{\gamma^\beta(t)} \geq \left(
\fr{1-\delta}{\theta} \right)^\beta \right ) \to 1\,,\enskip t \to
\infty \,; \label{pr-l1}\\
&\P \left( \fr{M^*(t)}{\gamma^\beta(t)} \geq \left(
\fr{1+\delta}{\theta} \right)^\beta \right) \to 0\,,\enskip t \to
\infty\,. \label{pr-l2}
\end{align}
To prove~\eqref{pr-l1}, let fix $\Delta \in (0,t)$ and note that

\noindent
$$
Q^*(t) \geq W(t)-W(t-\Delta)\,.
$$
Denote

\noindent
$$
Y_k^{(\Delta)}=W(k\Delta)-W((k-1)\Delta)\,,
$$
then for each~$t$, one has

\noindent
$$
M^*(t)\geq \max\limits_{1\leq k \leq \lfloor t/\Delta
\rfloor}Y_k^{(\Delta)}\,.
$$
Thus, the original  problem reduces to the analysis of the extremes
of a stationary normal sequence $\{Z_k\}$ (with $Z_k =_d \Nor(0,1)$) since

\noindent
\begin{multline*}
\P \left( \fr{M^*(t)}{\gamma^\beta(t)}\geq \left(
\fr{1-\delta}{\theta}\right)^\beta \right) \\
{}\geq \P \left(
\max\limits_{i=1,\ldots ,\lfloor t/\Delta \rfloor}Z_i \geq u(t)\right) := \P_1(t)\,,
\end{multline*}
where

\noindent
$$
u(t):= \fr{ \alpha(t)+r\Delta(t)}{\sqrt{v(\Delta(t))}}\,,\enskip
\alpha(t):=\left( \fr{1-\delta}{\theta}\,\,\gamma (t) \right)^\beta.
$$
If $\Delta:=\Delta(t)=A\gamma^\beta(t)$, where  $A>0$ is the
constant, is chosen, then it is possible to show that

\noindent
$$
\P_1(t) \to 1\,,\enskip  t \to \infty\,,
$$
as required. To establish~\eqref{pr-l2}, let  consider another stationary sequence:

\noindent
$$
Y_i:=\sup\limits_{s \in [i-1,i)} Q^*(s)\,,\enskip
i=1,\ldots ,\lceil t \rceil\,,
$$
and, thus,

\noindent
$$
M^*(t) \leq \max\{Y_i:\ \, i=1,\ldots ,\lceil t \rceil\}\,.
$$
This  immediately implies

\noindent
\begin{multline*}
\P \left( M^*(t) \geq \left(\fr{1+\delta}{\theta}\right)^\beta
\gamma^\beta(t) \right)\\
{} \leq \P \left( \max\limits_{i=1,\ldots ,\lceil t
\rceil} Y_i \geq \left(\fr{1+\delta}{\theta}\right)^\beta
\gamma^\beta(t)  \right)\\
{}\leq \lceil t \rceil \P \left(Y \geq
\left(\fr{1+\delta}{\theta}\right)^\beta \gamma^\beta(t)
\right) :=\P_2(t)\,.
\end{multline*}
A careful analysis allows to conclude   that

\noindent
$$
\P_2(t) \to 0\,,\enskip  t \to \infty\,,
$$
that completes  the proof.  (More detailed analysis can be found in~\cite{Luk1}.)~\hfill$\square$

\smallskip

Under  slightly less general  assumptions, the  previous result can
be generalized  to the convergence in the  space~$L_p$ for any $p\ge 1$.


\smallskip

\noindent
\textbf{Theorem 4.2.}\
\textit{Let conditions of  theorem~4.1 hold.  If,
moreover,
\begin{equation}
\liminf_{t\to \infty} L(t)>0\,;\enskip \limsup\limits_{t\to \infty} L(t)<\infty\,,
\label{10a}
\end{equation}
then convergence in~$\eqref{MEV:4}$ holds in the space~$L_p$  for any}
$p \in [1,\infty)$.


\smallskip

To prove this statement, it is sufficient to establish the uniform
integrability of the following sequence:\linebreak
$\left\{\left({ M^*(t)}/{\gamma^\beta(t)}\right)^p,\,t\ge T\right\}$
where $T$ is the finite positive constant. Indeed, it is shown
in~\cite{Luk2} that
\begin{equation*}
%\label{2.2.l1}
\sup\limits_{t \geq T} \E \left[
\fr{M^*(t)}{\gamma^\beta(t)} \right]^{p+1} < \infty\,.
\end{equation*}


\noindent
\textbf{Remark.}\
If the  limit

\noindent
\begin{equation*}
\lim\limits_{t\to \infty} L(t)= A\in (0,\,\infty)
%\label{38}
\end{equation*}
 exists, then conditions~\eqref{10a} are automatically fulfilled.


\smallskip

Theorems~4.1 and~4.2 can be applied to some specific
processes.  First, let  consider the following input:
\begin{equation*}
X(t)=\sum\limits_{i=1}^n B_{H_i}(t)\,,\enskip t\ge 0\,,
\end{equation*}
where $B_{H_i}$ are the independent FBMs with the Hurst parameters
$H_i\in(0,\,1)$ and  $H_1>\max\limits_{i>1}H_i$. Then,  the variance
$v(t)$ of the process $X(t)$ satisfies condition~(\ref{MEV:1}), and
 the statements of theorems~4.1 and~4.2 hold  with $V=2H_1$.
 This is an extension of the results derived in~\cite{Zeevi}.

The second example is the integrated Gaussian process~$X$, that is,

\noindent
\begin{equation*}
X(t)=\int\limits_0^t Z(s)\,ds
%\label{41}
\end{equation*}
where $Z$ is the centered stationary Gaussian process with the
covariance function  $R(u):=\mathbb{C}\mathrm{ov}(Z(0),Z(u))$. Such models
 have been considered in~\cite{Kulkarni, Debicki1}. It is easy to
check that the variance $v(t)$ of  $X(t)$ can be written as
\begin{equation}
v(t)=2 \int\limits_0^t \int\limits_0^s R(u)\,duds\,.
\label{42a}
\end{equation}
Then $v''(t)=2 R(t)$ and condition~\eqref{2.2.l20} is equivalent to
\begin{equation*}
R(t)\log t \to 0\,,\enskip t \to \infty\,.
%\label{2.2.l21}
\end{equation*}
If, in addition, $A \in (0,\ \infty)$ and  exists $V \in (0,2)$
such that
\begin{equation}
\fr{\int\limits_0^t \int\limits_0^s R(u)\,duds}{t^V} \to A\,,\enskip
t \to \infty\,,
\label{2.2.l22}
\end{equation}
then conditions of  theorem~4.2 are
satisfied as well. For example, let~$Z$ be the Ornstein--Uhlenbeck
process with $R(t)=\lambda e^{-\alpha t}$ and parameters
$\lambda,\alpha>0$. It then follows from~(\ref{42a}) that
$$
v(t)=\fr{2 \lambda}{\alpha}\,t+\fr{2\lambda}{\alpha^2}\left(e^{-\alpha t}-1\right)
$$
and, hence, condition~\eqref{2.2.l22} is satisfied with  $V=1$ and
$A={\lambda}/{\alpha}$.

Note that the integrated Ornstein--Uhlenbeck process is the Gaussian
counterpart of the well-known Anick--Mitra--Sondi fluid model~\cite{Anick}
(see also~\cite{Addie}), and its relevance for the
modeling of the  traffic in communications systems is  motivated in~\cite{Kulkarni}.

Theorem~4.1 yields the asymptotics for another important
characteristic, called hitting time,  the time required to reach  a
threshold~$b$,
$$
T(b)=\inf\left\{t \geq 0:\,Q^*(t)\geq b\right\}\,.
$$
The analysis of $T(b)$ is based on the  following relation between
$M^*(t)$ and~$T(b)$:
\begin{equation*}
\left\{ T(b) \leq t\right\}=\left\{M^*(t) \geq b \right\}\,.
%\label{time-l0}
\end{equation*}
Finally, the following result proved in~\cite{Luk2} has been got as well.


\smallskip

\noindent
\textbf{Theorem~4.3.}
\textit{If conditions of  theorem~$4.1$ hold and  function
$\gamma(t)$ monotonically increases in  $[t_0,\infty)$, for some
$t_0<\infty$, then}
\begin{equation*}
\fr{\gamma(T(b))}{b^{1/\beta}} \Rightarrow \Theta\,,\enskip b \to \infty\,.
%\label{time-l1}
\end{equation*}

\vspace*{-6pt}

\Ack
This work is done under financial   support
of  the Program of Strategy Development of  Petrozavodsk State
University  in the framework
of the research activity.



\renewcommand{\bibname}{\protect\rmfamily References}

{\small\frenchspacing
{%\baselineskip=10.8pt
\begin{thebibliography}{99}

\bibitem{Leland} %1
\Aue{Leland,~W.\,E., M.\,S. Taqqu, W.~Willinger, and D.\,V.~Wilson.}  1994.
On the self-similar nature of ethernet traffic (extended version).
\textit{IEEE/ACM Trans. Networking} 2(1):1--15.

\bibitem{Willinger} %2
\Aue{Willinger,~W., M.\,S.~Taqqu, W.\,E.~Leland, and D.~Wilson.}
1995. Self-similarity in high-speed packet traffic: Analysis and
modeling of Ethernet traffic measurements. \textit{Stat. Sci}.
10(1):67--85.

\bibitem{Taqqu} %3
\Aue{Taqqu,~M.\,S., W.~Willinger, and R.~Sherman.} 1997. Proof of a
fundamental result in self-similar traffic modeling. \textit{Comp.
Comm. Rev.} 27:5--23.

\bibitem{Reich} %4
\Aue{Reich,~E.} 1958.
On the integrodifferential equation of Takacs~I.
\textit{Ann. Math. Stat.} 29:563--570.

\bibitem{Duffield} %5
\Aue{Duffield,~N., and N.~O'Connell.}  1995.
Large deviations and overflow
probabilities for the general single server queue, with
applications. \textit{Proc. Cambridge Phil. Soc.} 118:363--374.

\bibitem{Debicki2}  %6
\Aue{Debicki,~K.} 1999.
A~note on LDP for supremum of Gaussian processes over infinite
horizon. \textit{Stat. Probab. Lett.} 44:211--220.

\bibitem{Duffy} %7
\Aue{Duffy,~K.,  J.\,T.~Lewis, and W.\,G.~Sullivan.} 2003.
Logarithmic asymptotics for the supremum of a stochastic process.
\textit{Ann. Appl. Probab.} 13(2):430--445.

\bibitem{Kim1} %8
\Aue{Kim,~H.\,S., and N.\,B.~Shroff.}  2001.
Loss probability calculations and asymptotic analysis for finite buffer
multiplexers. \textit{IEEE/ACM Trans. Networking} 9:755--768.

\bibitem{Kim2} %9
\Aue{Kim,~H.\,S., and N.\,B.~Shroff.}  2001.
On the asymptotic relationship between the overflow
probability and the loss ratio. \textit{Adv. Appl. Probab.}
33:836--863.

\bibitem{Luk6} %10
\Aue{Goricheva,~R.\,S., O.\,V.~Lukashenko, E.\,V.~Morozov, and
M.~Pagano.} 2010. Regenerative
analysis of a finite buffer fluid queue. \textit{2010
Congress (International) on Ultra Modern Telecommunications and Control Systems and
Workshops (ICUMT) Proceedings}. 1132--1136.

\bibitem{Luk4} %11
\Aue{Lukashenko,~O.\,V., E.\,V.~Morozov, and M.~Pagano.} 2011.
Estimation of loss probability
in Gaussian queues. \textit{Conference (International) ``Modern
Probabilistic Methods for Analysis and Optimization of Information and
Telecommunication Networks'' Proceedings}. 142--147.

\bibitem{Luk7} %12
\Aue{Lukashenko,~O.\,V., E.\,V.~Morozov, R.\,S.~Nekrasova, and M.~Pagano.} 2013.
Performance evaluation of finite buffer queues through regenerative simulation.
\textit{Comm. Comp. Inform. Sci. BWWQT 2013.} 356:131--139.

\bibitem{Zeevi} %13
\Aue{Zeevi,~A., and P.~Glynn.} 2000. On the maximum workload in a queue fed
by fractional Brownian motion. \textit{Ann. Appl. Probab}. 10:1084--1099.

\bibitem{Husler1} %14
\Aue{H$\ddot{\mbox{u}}$sler,~J., and V.\,I.~Piterbarg.}
2004. Limit theorem for maximum
of the storage process with fractional Brownian as input.
\textit{Stochastic Proc. Their Appl.} 114:231--250.

\bibitem{Luk1} %15
\Aue{Lukashenko,~O.\,V., and E.\,V.~Morozov.} 2012. Asymptotics of the maximum
workload for a class of Gaussian queues. \textit{Informatics and
Its Applications}~--- \textit{Inform. Appl.}
6(3):81--89.

\bibitem{Luk3} %16
\Aue{Lukashenko~O.~V., and E.\,V.~Morozov.} 2012. On the maximum workload for
a class of Gaussian queues. \textit{Conference (International) ``Probability
Theory and Its Applications'' in Commemoration of the Centennial of
B.\,V.~Gnedenko}. 231--232.

\bibitem{Luk2} %17
\Aue{Lukashenko,~O.\,V., and E.\,V.~Morozov.} 2013.
On convergence in the $L_p$
space of the workload maximum for a class of Gaussian queueing
systems. \textit{Informatics and Its Applications}~---
\textit{Inform. Appl.} 7(1):36--43.


\bibitem{Dieker1} %18
\Aue{Dieker,~A.\,B., and M.~Mandjes.} 2005.
On asymptotically efficient simulation of large deviation probabilities.
 \textit{Adv. Appl. Probab.} 37:539--552.

 \bibitem{Michele1} %19
\Aue{Giordano,~S., M.~Gubinelli, and M.~Pagano.}
2005. Bridge Monte-Carlo: A~novel approach to rare events of Gaussian processes.
\textit{5th St.\ Petersburg Workshop on Simulation Proceedings}.
St.\ Petersburg, Russia. 281--286.

\bibitem{Dieker2} %20
\Aue{Dieker,~A.\,B., and M.~Mandjes.} 2006. Fast simulation of overflow
probabilities in a queue with Gaussian input. \textit{ACM Trans. Model.
Comput. Simul.} 16(2):119--151.

\bibitem{Michele2} %21
\Aue{Giordano,~S., M.~Gubinelli, and M.~Pagano.}
2007. Rare events of Gaussian processes:
A~performance comparison between Bridge Monte-Carlo and Importance Sampling.
\textit{Next generation teletraffic and wired/wireless advanced networking}.
St.\ Petersburg, Russia. 269--280.

\bibitem{Luk5} %22
\Aue{Lukashenko,~O.\,V., E.\,V.~Morozov, and M.~Pagano.} 2012.
Performance analysis of bridge Monte-Carlo estimator.
\textit{Trans. KarRC RAS} 3:54--60.

\bibitem{Takacs} %23
\Aue{Takacs,~L.} 1967. \textit{Combinatorial methods in the theory of stochastic processes}.
John Wiley\&Sons. 262~p.

\bibitem{Mandjes} %24
\Aue{Mandjes,~M.} 2007. \textit{Large deviations of Gaussian queues}.
Chichester: Wiley. 339~p.



\bibitem{Fidler1} %25
\Aue{Rizk,~A., and M.~Fidler.} 2010.
Sample path bounds for long memory fbm traffic.
\textit{29th Conference on Information Communications, INFOCOM'10 Proceedings}.
Piscataway, NJ, USA: IEEE Press. 61--65.

\bibitem{Fidler2} %26
\Aue{Rizk,~A., and M.~Fidler.} 2012.
Non-asymptotic end-to-end performance bounds
for networks with long range dependent FBM cross traffic. \textit{Comp. Networks}.
56(1):127--141.


\bibitem{Luk0} %27
\Aue{Lukashenko,~O., E.~Morozov, and M.~Pagano.}  2014.
On the effective envelopes for fluid queues with Gaussian input.
\textit{Comm. Comp. Inform. Sci. DCCN 2013.} 279:178--189.

\bibitem{Adler} %28
\Aue{Adler,~R.\,J.} 1990. \textit{An introduction to continuity, extrema, and
related topics for general Gaussian processes}. Hayward, CA: Institute of
Mathematical Statistics. 160~p.

\bibitem{Debicki} %29
\Aue{Debicki,~K.} 2004. Gaussian processes. \textit{Encyclopedia of actuarial
sciences} 2:752--757.

\bibitem{Dembo} %30
\Aue{Dembo,~A., and O.~Zeitouni.} 1998.
\textit{Large deviations techniques and applications}. Springer. 396~p.

\bibitem{Seneta} %31
\Aue{Seneta,~E.} 1985. \textit{Regularly varying functions}. Springer. 116~p.

\bibitem{MEV:Konstantopoulos} %32
\Aue{Konstantopoulos,~T., M.~Zazanis, and G.~De Veciana}. 1996.
Conservation
laws and reflection mappings with application to multiclass mean
value analysis for stochastic fluid queues. \textit{Stochastic
Proc. Their Appl.} 65:139--146.

\bibitem{Kulkarni} %33
\Aue{Kulkarni,~V., and T.~Rolski.} 1994.
Fluid model driven by an Ornstein--Uhlenbeck process.
\textit{Prob. Eng. Inform. Sci.} 8:403--417.

\bibitem{Debicki1} %34
\Aue{Debicki,~K., and T.~Rolski.}  1995.
A~Gaussian fluid model. \textit{Queueing Syst.} 20:433--452.

\bibitem{Anick} %35
\Aue{Anick,~D., D.~Mitra, and M.\,M.~Sondhi.}  1982.
Stochastic theory of a data handling system with multiple resources.
\textit{Bell Syst. Techn.~J.} 61:1871--1894.

\bibitem{Addie} %36
\Aue{Addie,~R., P.~Mannersalo, and I.~Norros.} 2002. Most probable paths and performance
formulae for buffers with Gaussian input traffic.
\textit{Eur. Trans. Telecomm.} 13:183--196.



\end{thebibliography} } }

\end{multicols}

\vspace*{-6pt}

\hfill{\small\textit{Received March 8, 2014}}

\vspace*{-6pt}

\Contr

\noindent
\textbf{Lukashenko Oleg V.} (b.\ 1986)~---
Candidate of Science (PhD) in physics and mathematics,
junior scientist, Institute of Applied Mathematical Research of Karelian
Research Center, Russian Academy of Sciences; lecturer,
Petrozavodsk State University; lukashenko-oleg@mail.ru

\vspace*{3pt}

\noindent
\textbf{Morozov Evsei V.} (b.\ 1947)~--- Doctor of Science in physics and
mathematics, professor,
leading scientist, Institute of Applied Mathematical Research of Karelian Research Center,
Russian Academy of Sciences, 11 Pushkinskaya Str., Petrozavodsk 185910,
Republic of Karelia, Russian Federation; professor, Petrozavodsk State University,
33 Lenin Str., Petrozavodsk 185910, Republic of Karelia,
Russian Federation; emorozov@karelia.ru

\vspace*{3pt}

\noindent
\textbf{Pagano Michele} (b.\ 1968)~---
 PhD in electronics engineering, associate professor, University of Pisa,
 43 Lungarno Pacinotti, Pisa 56126, Italy;
 m.pagano@iet.unipi.it

\vspace*{12pt}

\hrule

\vspace*{2pt}

\hrule

\vspace*{6pt}

%\newpage


\def\tit{ОБ АСИМПТОТИКЕ ВЕРОЯТНОСТИ ПЕРЕПОЛНЕНИЯ ГАУССОВСКОЙ ОЧЕРЕДИ$^*$}

\def\aut{О.\,В.~Лукашенко$^1$, Е.\,В.~Морозов$^2$,  М.~Пагано$^3$}


\def\titkol{Об асимптотике вероятности переполнения гауссовской очереди}

\def\autkol{О.\,В.~Лукашенко, Е.\,В.~Морозов,  М.~Пагано}

{\renewcommand{\thefootnote}{\fnsymbol{footnote}}
\footnotetext[1]{Работа проводится при финансовой поддержке Программы
стратегического развития Петрозаводского государственного университета в рамках
на\-уч\-но-ис\-сле\-до\-ва\-тель\-ской деятельности.}}


\titel{\tit}{\aut}{\autkol}{\titkol}

\vspace*{-12pt}

\noindent $^1$Институт прикладных математических исследований КарНЦ РАН,
 Россия, Республика Карелия,\\
 $\hphantom{^1}$г.~Петрозаводск 185910, ул.\ Пушкинская 11;
Петрозаводский государственный университет,\\
$\hphantom{^1}$Россия, Республика Карелия, г.~Петрозаводск 185910,
пр.\ Ленина 33; lukashenko-oleg@mail.ru\\
\noindent $^2$Институт прикладных математических исследований КарНЦ РАН,
 Россия, Республика Карелия,\\
  $\hphantom{^1}$г.~Петрозаводск 185910, ул.\ Пушкинская 11;
Петрозаводский государственный университет,\\
 $\hphantom{^1}$Россия, Республика Карелия, г.~Петрозаводск 185910,
 пр.\ Ленина 33; emorozov@karelia.ru\\
\noindent
$^3$Университет г.\ Пиза, Италия; m.pagano@iet.unipi.it


\vspace*{6pt}

\def\leftfootline{\small{\textbf{\thepage}
\hfill ИНФОРМАТИКА И ЕЁ ПРИМЕНЕНИЯ\ \ \ том\ 8\ \ \ выпуск\ 2\ \ \ 2014}
}%
 \def\rightfootline{\small{ИНФОРМАТИКА И ЕЁ ПРИМЕНЕНИЯ\ \ \ том\ 8\ \ \ выпуск\ 2\ \ \ 2014
\hfill \textbf{\thepage}}}



\Abst{Гауссовские процессы являются мощным инструментом в моделировании сетей,
так как они позволяют описать эффект долгой памяти реальных сетевых потоков.
Более точно, при  реалистичных предположениях, дробное броуновское движение
(ДБД) возникает как предельный процесс, когда огромное число on-off
источников (с тяжелыми хвостами) мультиплексируются в магистральных сетях.
В~данной работе изучается жидкостная  система массового обслуживания с постоянной
скоростью обслуживания, с суммой независимых ДБД на входе, что соответствует
агрегации гетерогенных сетевых потоков. Для таких систем массового обслуживания
получена  логарифмическая асимптотика вероят\-ности переполнения, которая является
верхней границей  вероятности потери в соответствующих очередях с конечным буфером
и которая показывает, что в оценке доминирует ДБД с наибольшим параметром Херста.
Наконец, приведены асимптотические результаты для максимума  нагрузки в более
общем случае гауссовского входного процесса с дисперсией, которая правильно
меняется на бесконечности}

\KW{гауссовские жидкостные системы; вероятность переполнения;
логарифмические асимптотики}

\DOI{10.14357/19922264140203}

\vspace*{6pt}


 \begin{multicols}{2}

\renewcommand{\bibname}{\protect\rmfamily Литература}
%\renewcommand{\bibname}{\large\protect\rm References}

{\small\frenchspacing
{%\baselineskip=10.8pt
\addcontentsline{toc}{section}{References}
\begin{thebibliography}{99}

\bibitem{Leland-1} %1
\Au{Leland~W.\,E., Taqqu M.\,S., Willinger~W., Wilson~D.\,V.}
On the self-similar nature of ethernet traffic (extended version).
{IEEE/ACM Transactions of Networking}, 1994. Vol.~2. No.\,1. P.~1--15.

\bibitem{Willinger-1} %2
\Au{Willinger~W., Taqqu M.\,S., Leland~W.\,E., Wilson~D.}
Self-similarity in high-speed packet traffic: Analysis and
modeling of Ethernet traffic measurements~// {Stat. Sci.}, 1995.
Vol.~10. No.\,1. P.~67--85.

\bibitem{Taqqu-1} %3
\Au{Taqqu~M.\,S., Willinger W., Sherman~R.}  Proof of a
fundamental result in self-similar traffic modeling~// {Comp.
Comm. Rev.}, 1997. Vol.~27. P.~5--23.

\bibitem{Reich-1} %4
\Au{Reich~E.} On the integrodifferential equation of Takacs~I~//
Ann. Math. Stat., 1958. Vol.~29. P.~563--570.

\bibitem{Duffield-1} %5
\Au{Duffield~N., O'Connell N.}
Large deviations and overflow
probabilities for the general single server queue, with
applications~// {Proc. Cambridge Phil.
Soc.}, 1995. Vol.~118. P.~363--374.

\pagebreak

\bibitem{Debicki2-1} %6
\Au{Debicki~K.} A~note on LDP for supremum of Gaussian processes over infinite
horizon~// {Stat. Probab. Lett.}, 1999. Vol.~44. P.~211--220.

\bibitem{Duffy-1} %7
\Au{Duffy~K.,  Lewis J.\,T.,  Sullivan~W.\,G.}
Logarithmic asymptotics for the supremum of a stochastic process~//
{Ann. Appl. Probab.}, 2003. Vol.~13. No.\,2. P.~430--445.

\bibitem{Kim1-1} %8
\Au{Kim~H.\,S., Shroff N.\,B.}
Loss probability calculations and asymptotic analysis for finite buffer
multiplexers~// {IEEE/ACM Trans. Networking}, 2001. Vol.~9. P.~755--768.

\bibitem{Kim2-1} %9
\Au{Kim~H.\,S., Shroff N.\,B.}
On the asymptotic relationship between the overflow
probability and the loss ratio~// {Adv. Appl. Probab.},  2001.
Vol.~33. P.~836--863.

\bibitem{Luk6-1} %10
\Au{Goricheva~R.\,S., Lukashenko O.\,V., Morozov~E.\,V.,
Pagano~M.}  Regenerative
analysis of a finite buffer fluid queue~// \textit{2010
Congress (International) on Ultra Modern Telecommunications and Control Systems and
Workshops (ICUMT) Proceedings}, 2010. P.~1132--1136.

\bibitem{Luk4-1} %11
\Au{Lukashenko~O.\,V., Morozov E.\,V.,  Pagano~M.}
Estimation of loss probability
in Gaussian queues~// {Conference (International) ``Modern
Probabilistic Methods for Analysis and Optimization of Information and
Telecommunication Networks'' Proceedings}, 2011. P.~142--147.

\bibitem{Luk7-1} %12
\Au{Lukashenko~O.\,V., Morozov E.\,V., Nekrasova~R.\,S., Pagano~M.}
Performance evaluation of finite buffer queues through regenerative simulation~//
Comm. Comp. Inform. Sci. BWWQT 2013, 2013.
Vol.~356. P.~131--139.


\bibitem{Zeevi-1} %13
\Au{Zeevi~A., Glynn~P.} On the maximum workload in a queue fed
by fractional Brownian motion~// {Ann. Appl. Probab}., 2000. Vol.~10. P.~1084--1099.

\bibitem{Husler1-1} %14
\Au{H$\ddot{\mbox{u}}$sler~J., Piterbarg~V.\,I.}
 Limit theorem for maximum
of the storage process with fractional Brownian as input~//
{Stochastic Proc. Their Appl.}, 2004. Vol.~114. P.~231--250.


\bibitem{Luk1-1} %15
\Au{Lukashenko~O.\,V., Morozov E.\,V.}  Asymptotics of the maximum
workload for a class of Gaussian queues~// Информатика и её применения, 2012.
Т.~6. Вып.~3. С.~81--89.

\bibitem{Luk3-1} %16
\Au{Lukashenko~O.~V., Morozov~E.~V.} On the maximum workload for
a class of Gaussian queues~// {Conference (International) ``Probability
Theory and Its Applications'' in Commemoration of the Centennial of
B.\,V.~Gnedenko}, 2012. P.~231--232.

\bibitem{Luk2-1} %17
\Au{Lukashenko~O.\,V., Morozov  E.\,V.}
On convergence in the $L_p$
space of the workload maximum for a class of Gaussian queueing
systems~// Информатика и её применения, 2013.
Т.~7. Вып.~1. С.~36--43.

\bibitem{Dieker1-1} %18
\Au{Dieker~A.\,B., Mandjes~M.}
On asymptotically efficient simulation of large deviation probabilities~//
{Adv. Appl. Probab.}, 2005. Vol.~37. P.~539--552.





\bibitem{Michele1-1} %19
\Au{Giordano~S., Gubinelli M., Pagano~M.}
Bridge Monte-Carlo: A~novel approach to rare events of Gaussian processes~//
{5th St.\ Petersburg Workshop on Simulation Proceedings}.
St.\ Petersburg, Russia, 2005. P.~281--286.

\bibitem{Dieker2-1} %20
\Au{Dieker,~A.\,B., Mandjes M.}  Fast simulation of overflow
probabilities in a queue with Gaussian input~// {ACM Trans. Model.
Comput. Simul.}, 2006. Vol.~16. No.\,2. P.~119--151.

\bibitem{Michele2-1} %21
\Au{Giordano~S., Gubinelli M., Pagano~M.}
Rare events of Gaussian processes:
A~performance comparison between Bridge Monte-Carlo and Importance Sampling~//
{Next Generation Teletraffic and Wired/Wireless Advanced Networking}.
St.\ Petersburg, Russia, 2007. P.~269--280.


\bibitem{Luk5-1} %22
\Au{Lukashenko~O.\,V., Morozov E.\,V., Pagano~M.}
Performance analysis of bridge Monte-Carlo estimator~//
Trans. KarRC RAS, 2012. Vol.~3. P.~54--60.


\bibitem{Takacs-1} %23
\Au{Takacs~L.}
{Combinatorial methods in the theory of stochastic processes}.~---
John Wiley\&Sons, 1967. 262~p.

\bibitem{Mandjes-1} %24
\Au{Mandjes~M.}  {Large deviations of Gaussian queues}.
Chichester: Wiley, 2007. 339~p.


\bibitem{Fidler1-1} %25
\Au{Rizk~A., Fidler M.}
Sample path bounds for long memory fbm traffic~//
{29th Conference on Information Communications, INFOCOM'10 Proceedings}.~---
Piscataway, NJ, USA: IEEE Press, 2010. P.~61--65.

\bibitem{Fidler2-1} %26
\Au{Rizk~A., Fidler M.}
Non-asymptotic end-to-end performance bounds
for networks with long range dependent fbm cross traffic~//
{Computer Networks}, 2012. Vol.~56. No.\,1. P.~127--141.

\bibitem{Luk0-1} %27
\Au{Lukashenko~O., Morozov E.,  Pagano~M.}
On the effective envelopes for fluid queues with Gaussian input~//
Comm. Comp. Inform. Sci. DCCN 2013,  2014.
Vol.~279. P.~178--189.

\bibitem{Adler-1} %28
\Au{Adler~R.\,J.} {An introduction to continuity, extrema, and
related topics for general Gaussian processes}.~--- Hayward, CA: Institute of
Mathematical Statistics, 1990. 160~p.


\bibitem{Debicki-1} %29
\Au{Debicki~K.} Gaussian processes~// {Encyclopedia of actuarial
sciences}, 2004. Vol.~2. P.~752--757.

\bibitem{Dembo-1} %30
\Au{Dembo~A., Zeitouni O.}
{Large deviations techniques and applications}. Springer, 1998. 396~p.



\bibitem{Seneta-1} %31
\Au{Seneta~E.}  {Regularly varying functions}.~--- Springer, 1985. 116~p.

\bibitem{MEV:Konstantopoulos-1} %32
\Au{Konstantopoulos~T., Zazanis M., De~Veciana~G.}
Conservation
laws and reflection mappings with application to multiclass mean
value analysis for stochastic fluid queues~// {Stochastic
Proc. Their Appl.}, 1996. Vol.~65. P.~39--146.

\bibitem{Kulkarni-1} %33
\Au{Kulkarni~V., Rolski T.}
Fluid model driven by an Ornstein--Uhlenbeck process~//
{Prob. Eng. Inform. Sci.}, 1994.
Vol.~8. P.~403--417.

\bibitem{Debicki1-1} %34
\Au{Debicki~K., Rolski T.}
A~Gaussian fluid model~// Queueing Syst.,  1995. Vol.~20. P.~433--452.

\bibitem{Anick-1} %35
\Au{Anick~D., Mitra D., Sondhi~M.\,M.}
Stochastic theory of a data handling system with multiple resources~//
Bell Syst. Techn.~J.,  1982. Vol.~61. P.~1871--1894.



\bibitem{Addie-1} %36
\Au{Addie~R., Mannersalo P., Norros I.} Most probable paths and performance
formulae for buffers with Gaussian input traffic~//
Eur. Trans. Telecommunications, 2002. Vol.~13. P.~183--196.


 \label{end\stat}

\end{thebibliography}
} }

\end{multicols}



\hfill{\small\textit{Поступила в редакцию 08.03.2014}}
%\renewcommand{\bibname}{\protect\rm Литература}
\renewcommand{\figurename}{\protect\bf Рис.}      %5
\include{kruchin}      %6
\include{stepanov}       %7
\def\stat{bening}

\def\tit{АСИМПТОТИЧЕСКИЕ РАЗЛОЖЕНИЯ ДЛЯ ФУНКЦИЙ РАСПРЕДЕЛЕНИЯ
СТАТИСТИК, ПОСТРОЕННЫХ\\ ПО ВЫБОРКАМ СЛУЧАЙНОГО ОБЪЕМА$^*$}

\def\titkol{Асимптотические разложения для функций распределения
статистик, построенных по выборкам случайного объема}

\def\autkol{В.\,Е.~Бенинг, Н.\,К.~Галиева, В.\,Ю.~Королев}

\def\aut{В.\,Е.~Бенинг$^1$, Н.\,К.~Галиева$^2$, В.\,Ю.~Королев$^3$}

\titel{\tit}{\aut}{\autkol}{\titkol}

{\renewcommand{\thefootnote}{\fnsymbol{footnote}}\footnotetext[1]
{Работа
поддержана Российским фондом фундаментальных исследований (проекты
11-01-00515а, 11-07-00112а, 11-01-12026-офи-м), Министерством
образования и науки РФ (госконтракт 16.740.11.0133).}}

\renewcommand{\thefootnote}{\arabic{footnote}}
\footnotetext[1]{Факультет вычислительной
математики и кибернетики Московского государственного университета
им.\ М.\,В.~Ломоносова; Институт проблем информатики Российской
академии наук, bening@yandex.ru}
\footnotetext[2]{Казахстанский филиал Московского государственного
университета им.\ М.\,В.~Ломоносова, nurgul\_u@mail.ru}
\footnotetext[3]{Факультет вычислительной
математики и кибернетики Московского государственного университета
им.\ М.\,В.~Ломоносова; Институт проблем информатики Российской
академии наук, vkorolev@cs.msu.su}

\vspace*{4pt}

\Abst{Доказана общая теорема переноса,
позволяющая получать асимптотические разложения (а.р.)\ для функций
распределения (ф.р.)\ статистик, основанных на выборках случайного объема,
из а.р.\  для ф.р.\ случайного
объема выборки и а.р.\ для  ф.р.\ статистик, построенных по выборкам неслучайного
объема.}

\vspace*{2pt}

\KW{выборка случайного объема;
асимптотическое разложение; теорема переноса; смесь вероятностных
законов; распределение Лапласа; распределение Стьюдента}

\vspace*{4pt}

\vskip 14pt plus 9pt minus 6pt

      \thispagestyle{headings}

      \begin{multicols}{2}

            \label{st\stat}


\section{Введение}

В классических задачах математической статистики объем выборки,
доступной исследователю, традиционно считается детерминированным и в
асимптотических постановках играет роль (как правило, неограниченно
возрастающего) {\it известного} параметра. В~то же время на практике
час\-то возникают ситуации, когда размер выборки не является заранее
определенным и может рас\-смат\-ри\-вать\-ся как случайный. Такие ситуации,
как правило, связаны с тем, что статистические данные накапливаются
в течение фиксированного времени. Это \mbox{имеет} место, в частности, в
страховании, когда в течение разных отчетных периодов одинаковой
длины (скажем, месяцев) происходит разное число страховых событий
(страховых выплат и/или заключений страховых контрактов); в
медицине, когда число пациентов с тем или иным заболеванием
варьируется от года к году; в технике, когда при испытании на
надежность (скажем, при определении наработки на отказ) разных
партий приборов (изделий), чис\-ло отказавших приборов в разных
партиях оказывается разным. В~таких ситуациях заранее не известное
число наблюдений, которые будут доступны исследователю, разумно
считать случайной величиной (с.в.). Другими словами, в таких ситуациях
объем выборки является не (известным) параметром, а сам становится
{\it наблюдением}, т.\,е.\ статистикой. В~силу указанных
обстоятельств вполне естественным становится изуче\-ние
асимптотического поведения распределений статистик достаточно общего
вида, основанных на выборках случайного объема.

На естественность такого подхода, в частности, обратил внимание
Б.\,В.~Гнеденко в работе~\cite{2-ben}, в которой рассматривались
асимптотические свойства\linebreak распределений выборочных квантилей,
построенных по выборкам случайного объема, и было
продемонстрировано, что при замене неслучайного \mbox{объема} выборки
случайной величиной асимптотические свойства статистик могут
радикально измениться. К~примеру, если объем выборки является
геометрически распределенной с.в., то вместо
ожидаемого в соответствии с классической теорией нормального закона
в качестве асимптотического распределения выборочной медианы
возникает распределение Стьюдента с двумя степенями свободы, хвосты
которого столь тяжелы, что у него отсутствуют моменты порядков,
б$\acute{\mbox{о}}$льших второго. <<Тяжесть>> хвостов асимптотических распределений
имеет же  критически важное значение, в частности, в задачах проверки
гипотез.

Простейшей статистикой является сумма наблюдений. Для выборок
случайного объема число слагаемых в таких суммах само становится
случайным. Асимптотическим свойствам распределений сумм случайного
числа с.в.\ посвящено много работ (см., например,~[1--7]). 
Такого рода суммы находят широкое применение в страховании,
экономике, биологии и~т.\,п.~[2, 5, 7, 8]. В~классической статистике
суммирование наблюдений, как правило, возникает при определении
выборочных средних. При статистическом анализе, основанном на
моделях, в которых объем выборки считается неслучайным,
асимптотическое поведение статистик типа сумм и статистик типа
средних арифметических одинаково~--- эти статистики после нормировки,
обязательной для получения нетривиальных предельных распределений,
становятся неразличимыми. Однако, как уже говорилось, в реальной
практике очень часто объем выборки сам является статистикой и, как
недавно показано, например, в работе~\cite{24-ben}, асимптотическое
поведение статистик типа сумм и статистик типа средних
арифметических при их неслучайной нормировке оказывается различным.
Заметим, что, конечно же, формально допустима и случайная
нормировка, но для построения разумных асимптотических аппроксимаций
для распределений статистик (а именно это и является целью
асимптотической статистики) она не применима. Именно использованием
неслучайной нормировки и объясняется возникновение не <<чис\-то\-го>>
нормального закона, а (разных!) смешанных нормальных предельных
распределений у статистик типа сумм и типа средних арифметических.
При этом различие этих предельных законов может дать дополнительную
информацию о структуре исходных данных.

Более того, в математической статистике и ее приложениях часто
встречаются статистики, которые не являются суммами наблюдений.
Примерами могут служить ранговые статистики, $U$-ста\-ти\-сти\-ки,
линейные комбинации порядковых статистик\linebreak ($L$-ста\-ти\-сти\-ки) и~т.\,п.

В данной работе получены а.р.\ для
ф.р.\ статистик, построенных по выборкам
случайного объема. Эти а.р.\ непосредственно зависят от а.р.\ ф.р.\
случайного объема выборки и а.р.\ ф.р.\ статистики, основанной на
неслучайной выборке. Подобного рода утверждения принято называть
тео\-ре\-ма\-ми переноса. Таким образом, в данной работе доказаны теоремы
переноса для а.р.\ статистик, построенных по выборкам случайного
объема.

В работе приняты следующие обозначения: $\R$~--- множество
вещественных чисел; $\N$~--- множество натуральных чисел; $\Phi(x)$ и
$\varphi(x)$~--- соответственно ф.р.\ и плот\-ность стандартного
нормального закона.

В разд.~2 приведен эвристический вывод основного результата, в
разд.~3--5 содержатся строгая формулировка основной теоремы, ее
доказательство и примеры.

Рассмотрим с.в.\ $N_1, N_2, \ldots$ и  $X_1, X_2,\ldots$, 
заданные на одном и том же вероятностном пространстве
$(\Omega, {\cal A}, {\p})$. В~статистике с.в.\ $X_1, X_2, \ldots X_n$
имеют смысл наблюдений, $n$~--- неслучайный объем выборки, а с.в.\
$N_n$~--- случайный объем выборки, зависящий от натурального
параметра $n\hm\in \N$. Например, если с.в.~$N_n$ имеет геометрическое
распределение вида
$$
{\p}(N_n = k) =  \fr{1}{n} \left(1 - \fr{1}{n}\right)^{k-1}\,,\enskip
 k\in\N\,,
$$
то
$$
\e N_n = n\,,
$$
т.\,е.\ среднее значение случайного объема выборки равно~$n$.

Предположим, что при каждом $n\hm\geq1$ с.в.~$N_n$ принимают только
натуральные значения (т.\,е.\ $N_n\hm\in \N$) и независимы от
последовательности с.в.\ $X_1, X_2, \ldots$ Всюду далее считаем с.в.\
$X_1, X_2, \ldots$ независимыми и одинаково распределенными.

Обозначим через  $T_n=T_n(X_1,\ldots ,X_n)$ некоторую статистику, т.\,е.\
действительную измеримую функцию от наблюдений $X_1,\ldots ,X_n$.
Для каждого  $n\hm\geq1$ определим с.в.\ $T_{N_n}$, полагая
$$
T_{N_n}(\omega)\equiv
T_{N_n(\omega)}(X_1(\omega),\ldots ,X_{N_n(\omega)}(\omega))\,, \enskip
\omega\in\Omega\,.
$$
Можно сказать, что $T_{N_n}$~--- это статистика, построенная на
основе статистики $T_n$ по выборке случайного объема $N_n$.

Сформулируем условие, определяющее а.р.\ для ф.р.\ статистики $T_n$
при неслучайном объеме вы\-борки.

\smallskip

\noindent
\textbf{Условие 1.1.} \textit{Существуют константы  $l\hm\in\N$, $\mu\hm\in\R$,
$\sigma\hm>0$, $\alpha\hm>l/2$, $\gamma\hm\ge0$, $C_1\hm>0$, дифференцируемая
ф.р.\ $F(x)$ и дифференцируемые ограниченные функции $f_j(x)$,
$j=1,\ldots ,l$, такие что}
\begin{multline*}
\sup\limits_x \left|\vphantom{\sum\limits_{j=1}^l}
{\p}\left(\sigma n^\gamma(T_n - \mu) < x\right) -\ F(x)
-{}\right.\\
\left.{}- \sum\limits_{j=1}^l n^{-j/2} f_j(x) \right|  \leq \fr{C_1}{n^{\alpha}}\,,
  \ \ \ n\in\N\,.
\end{multline*}

\smallskip

Следующее условие определяет а.р.\ ф.р.\ нормированного случайного
индекса $N_n$.

\smallskip

\noindent
\textbf{Условие 1.2.} \textit{Существуют константы  $m\hm\in\N$, $\beta\hm>m/2$,
$C_2\hm>0$, функция $0\hm<g(n)\uparrow \infty$, $n\hm\to\infty$, ф.р.~$H(x),
H(0+)\hm=0$, и функции ограниченной вариации $h_i(x)$, $i\hm=1,\ldots ,m$, такие
что}
\begin{multline*}
\sup\limits_{x\ge0} \left|{\p}\left(\fr{N_n}{g(n)} < x\right) - H(x)\ -
\sum\limits_{i=1}^m n^{-i/2} h_i(x) \right|  \leq{}\\
{}\leq \fr{C_2}{n^{\beta}}\,, \ \
\  n\in\N\,.
\end{multline*}

\smallskip

В данной работе с помощью а.р.\ для ф.р.\ нормированной статистики
$T_{N_n}$, основанной на выборке случайного объема, получена
аппроксимация вида
\begin{equation*}
{\p}\left(\sigma g^\gamma(n)(T_{N_n}  -  \mu)  <  x\right)\ \approx
G_{n}(x)\,, \ \  n \to \infty\,,
%\label{e1.1-ben}
\end{equation*}
где функция $G_n(x)$ имеет вид (см.\ условия~1.1, 1.2):
\begin{multline}
G_n(x) = {}\\
{}=\!\!\int\limits_{1/g(n)}^\infty\!\!\!\! F(x y^\gamma)\, dH(y) +
\sum\limits_{i=1}^m n^{-i/2}\! \int\limits_{1/g(n)}^\infty \!\!F(xy^\gamma)\,
dh_i(y)\ +{}
\\
{}+ \sum\limits_{j=1}^l g^{-j/2}(n)\! \int\limits_{1/g(n)}^\infty \!
y^{-j/2}f_j(xy^\gamma)\, dH(y) +{}\\
\hspace*{-3mm}{}+
\sum\limits_{j=1}^l \sum\limits_{i=1}^m n^{-i/2}g^{-j/2}(n)
\!\!\!\int\limits_{1/g(n)}^\infty y^{-j/2}\!\!f_j(xy^\gamma)
\,dh_i(y).\!
\label{e1.2-ben}
\end{multline}
Для пояснения этой формулы, идеи доказательства и удобства
дальнейших ссылок приведем ее эвристический вывод.

\section{Эвристический вывод основного результата}

В идейном плане доказательство основного результата данной работы~---
теоремы~3.1~--- близко к доказательству теорем 6.6.1 и 6.7.3 для
случайных сумм из работы~\cite{6-ben} и оценкам скорости сходимости
распределений случайно индексированных последовательностей из работы~\cite{14-ben} 
($\S$~1.3, с.~63).

По формуле полной вероятности имеем:
\begin{multline}
{\p}\left(\sigma g^\gamma(n)(T_{N_n} - \mu) < x\right)  ={}\\
{}=
{\p}\left(\sigma N_n^\gamma(T_{N_n} - \mu) <
\left(\fr{N_n}{g(n)}\right)^\gamma x\right)  ={}\\
{}
= {\e} {\p}\left(\sigma N_n^\gamma(T_{N_n} - \mu) <
\left(\fr{N_n}{g(n)}\right)^\gamma x \Big| N_n\right)  ={}\\
\!{}=
\sum\limits_{k=1}^{\infty} {\p}\left(\!\sigma k^\gamma(T_{k}  -  \mu) <
\left(\fr{k}{g(n)}\right)^\gamma\!\! x\right){\p}(N_n=k).\!\!
\label{e2.1-ben}
\end{multline}
Используя условие~1.1, вероятность под знаком ряда аппроксимируем
следующим образом:
\begin{multline*}
\!\!\!\!{\p}\left(\sigma g^\gamma(n)(T_{N_n} - \mu) < x\right) \approx
\sum\limits_{k=1}^{\infty} \!\left(\!
\vphantom{\sum\limits_{j=1}^l\
k^{-j/2}f_j\left(x\left(\fr{k}{g(n)}\right)^\gamma\right)}
F\left(x
\left(\fr{k}{g(n)}\right)^\gamma\right) + {}\right.\hspace*{-1.8927pt}\\
\left.{}+\sum\limits_{j=1}^l\
k^{-j/2}f_j\left(x\left(\fr{k}{g(n)}\right)^\gamma\right)\right)
{\p}(N_n=k) ={}
\end{multline*}

\noindent
\begin{multline}
{}
= {\e} \left( \vphantom{\sum\limits_{j=1}^l\
k^{-j/2}f_j\left(x\left(\fr{k}{g(n)}\right)^\gamma\right)}
F\left(x \left(\fr{N_n}{g(n)}\right)^\gamma\right) +{}\right.\\
\left.{}+
\sum\limits_{j=1}^l 
N_n^{-j/2}f_j\left(x\left(\fr{N_n}{g(n)}\right)^\gamma\right)\right)
= \!\int\limits_{1/g(n)}^\infty \!\left(
\vphantom{\sum\limits_{j=1}^l}
F(x y^\gamma) +{}\right.\\
\left.{}+ \sum\limits_{j=1}^l
\left(yg(n)\right)^{-j/2}f_j(xy^\gamma)\right)\, d
{\p}\left(\fr{N_n}{g(n)} < y\right)\,.\label{e2.2}
\end{multline}
Теперь, аппроксимируя вероятность под знаком последнего интеграла с
помощью условия~1.2, получим формулу~(\ref{e1.2-ben}):
\begin{multline}
\!\!\!{\p}\left(\sigma g^\gamma(n)(T_{N_n} - \mu) < x\right) \approx G_n(x)
=\!\!\! \int\limits_{1/g(n)}^\infty\! \!\!\left(\!
\vphantom{\sum\limits_{j=1}^l}
F(x y^\gamma) + {}\right.\hspace*{-3.19522pt}\\
\left.{}+\sum\limits_{j=1}^l
\left(yg(n)\right)^{-j/2}f_j(xy^\gamma)\right)\, d \left(
\vphantom{\sum\limits_{i=1}^m}
H(y) +{}\right.\\
\left.{}+
\sum\limits_{i=1}^m n^{-i/2} h_i(y)\right) ={}\\
{}
 = \!\!\int\limits_{1/g(n)}^\infty\!\! F(x y^\gamma)\, dH(y) + 
 \sum\limits_{i=1}^m n^{-i/2} \!\!\int\limits_{1/g(n)}^\infty \!\!F(xy^\gamma) \,dh_i(y) +{}
\\
{}+ \sum\limits_{j=1}^l g^{-j/2}(n) \!\int\limits_{1/g(n)}^\infty\!
y^{-j/2}f_j(xy^\gamma) \,dH(y) +{}\\
\!\!\!{}+
\sum\limits_{j=1}^l \sum\limits_{i=1}^m n^{-i/2}g^{-j/2}(n)\!\!
\int\limits_{1/g(n)}^\infty \!\!y^{-j/2}f_j(xy^\gamma)
\,dh_i(y).\!\!
\label{e2.3-ben}
\end{multline}
Если в условии~1.1 статистика $T_n$ не нормирована, т.\,е.\
$\gamma\hm=0$, то полученное а.р.\ приобретает вид:
\begin{multline*}
G_n(x) =   F(x)\left(1 - H\left(\fr{1}{g(n)}\right)\right) +
 \sum\limits_{j=1}^l \left(g(n)\right)^{-j/2}\times{}\\
{}\times f_j(x)
\int\limits_{1/g(n)}^\infty y^{-j/2} d \left(H(y) + \sum\limits_{i=1}^m
n^{-i/2} h_i(y)\right)\,. %\label{e2.4-ben}
\end{multline*}
Таким образом, у ненормированной статистики $T_n$ исходное а.~р.
$$
{\p}\left(\sigma (T_n  -  \mu)  <  x\right) \approx  F(x) +
\sum\limits_{j=1}^l n^{-j/2} f_j(x)
$$
при переходе к выборке случайного объема $N_n$  заменяется на а.р.\
вида:

\noindent
\begin{multline*}
{\p}\left(\sigma (T_{N_n}  -  \mu)  <  x\right) \approx {}\\
{}\approx  F(x)\left(1 -
H\left(\fr{1}{g(n)}\right)\right) + \sum\limits_{j=1}^l c_{jn} f_j(x)\,,
\end{multline*}
где
\begin{multline*}
c_{jn} = {}\\
\hspace*{-4pt}{}=\left(g(n)\right)^{-j/2} \!\!\!\int\limits_{1/g(n)}^\infty\!\!\!
y^{-j/2} \,d \left(H(y) + \sum\limits_{i=1}^m n^{-i/2} h_i(y)\right).
\end{multline*}

\section {Формулировка основного результата}

\noindent
\textbf{Теорема~3.1.} \textit{Пусть  статистика
$T_n\hm=T_n(X_1,\ldots$\linebreak $\ldots ,X_n)$ удовлетворяет условию~$1.1,$ а случайный
объем выборки $N_n$~--- условию~$1.2$. Тогда  существует константа
$C_3\hm>0$ такая, что справедливо неравенство:
\begin{multline*}
\sup\limits_x \left| \p\left(\sigma g^\gamma(n)(T_{N_n}  -  \mu) < x\right)
-  G_n(x)\right| \leq {}\\
{}\leq C_1 \e N_n^{-\alpha} + \fr{C_3 + C_2
M_n}{n^\beta}\,,
\end{multline*}
где
\begin{multline*}
M_n = \sup_x \int\limits_{1/g(n)}^\infty
\Bigl|\fr{\partial}{\partial y} \bigl(F(xy^\gamma) +{}\\
{}+ \sum_{j=1}^l
(yg(n))^{-j/2}\ f_j(xy^\gamma)\bigr)\Bigr|\, dy
\end{multline*}
и а.р.\ $G_n(x)$ определено по формуле}~(\ref{e1.2-ben}).

\smallskip

\noindent
\textbf{Следствие 3.1.} \textit{Если моменты $\e (N_n/g(n))^{-\alpha}$
равномерно по $n$ ограничены, т.\,е.\
$$
\e \left(\fr{N_n}{g(n)}\right)^{-\alpha} \leq C_4\,,  \enskip C_4 >
0\,,\ \ n \in \N\,,
$$
то правая часть неравенства в формулировке теоремы~$3.1$ приобретает
вид:}
$
{C_1 C_4}/{g^\alpha(n)} \hm+  (C_3 \hm+ C_2 M_n)/{n^\beta}.
$


\smallskip

\noindent
\textbf{Следствие~3.2.} \textit{В силу неравенства Гёльдера при
$0\hm<\alpha\hm\leq1$ справедлива оценка
$$
\e N_n^{-\alpha} \leq \left(\e N_n^{-1}\right)^\alpha\,,
$$
которая может быть полезной при практическом применении теоремы. 
В~этом случае правая часть неравенства из формулировки теоремы может
быть записана в виде:}
$
C_1 \left(\e N_n^{-1}\right)^\alpha  +  ({C_3 + C_2
M_n})/{n^\beta}.
$


\smallskip

Для вычисления  $\e N_n^{-1}$ можно использовать следующую формулу
(см., например,~\cite{25-ben},  с.~93, задача~40,\,б). Если
неотрицательная целочисленная с.в.~$N$  имеет производящую функцию
$$
\Psi(s) =  \e s^{N}\,,   \enskip  |s| \le 1\,,
$$
то из теоремы Фубини непосредственно следует, что
\begin{equation}
\e N^{-1} = \int\limits_0^1 \fr{\Psi(s)}{s}\, ds\,. \label{e3.1-ben}
\end{equation}
Используя это соотношение, оценку из формулировки теоремы можно
представить  в виде:
\begin{equation}
C_1 \left(\int\limits_0^1 \fr{\Psi_n(s)}{s}\, ds\right)^\alpha  +
\fr{C_3 + C_2 M_n}{n^\beta}\,,\label{e3.2-ben}
\end{equation}
где $\Psi_n(s)$~--- производящая функция с.в.~$N_n$.

Приведем пример использования формулы~(\ref{e3.1-ben}). Пусть с.в.~$N_n$ имеет
геометрическое распределение:
$$
{\p}(N_n = k) =  \fr{1}{n} \left(1 - \fr{1}{n}\right)^{k-1}\,,\enskip
 k\in\N\,.
$$
В этом случае производящая функция имеет вид:
\begin{multline*}
\Psi_n(s) =  \e s^{N_n} =  \fr{s}{n}\left[1 -  s\left(1 -
\fr{1}{n}\right)\right]^{-1}={}\\
{}=\fr{s}{n-s(n-1)}\,, \enskip   |s| \le 1\,,
\end{multline*}
поэтому
\begin{equation}
\e N_n^{-1} = \int\limits_0^1 \fr{\Psi_n(s)}{s}\, ds = \fr{1}{n -
1} \,\log n\,,  \enskip  n > 1\,. \label{e3.3-ben}
\end{equation}
С учетом формулы~(\ref{e3.3-ben}) оценка~(\ref{e3.2-ben}) принимает вид:
$(C_1 \log^{\alpha} n)/(n - 1)^\alpha  +  (C_3 + C_2
M_n)/n^\beta$, $n \hm> 1$.

\smallskip

\textbf{Замечание~3.1.} Заметим, что из условия~1.2, в частности,
вытекает, что с.в.\ $N_n/g(n)$ слабо сходится к с.в.~$V$,  имеющей
ф.р.~$H(x)$. Из определения слабой сходимости с функцией
$x^{-\alpha}$, $ x \hm\ge 1$, в случае, если $N_n \hm\ge g(n)$, $n\hm\in\N$,
следует, что
$$
\e \left(\fr{N_n}{g(n)}\right)^{-\alpha} \longrightarrow   \e
\fr{1}{V^\alpha}\,,  \enskip  n\to\infty\,,
$$
т.\,е.\  моменты  $\e (N_n/g(n))^{-\alpha}$ равномерно ограничены по~$n$  
и справедливо утверждение из следствия~3.1.

\smallskip

Случай, когда $N_n\hm\ge g(n)$, возникает, например, если с.в.\ $N_n$
принимает значения $g(n), 2g(n), \ldots , kg(n)$  с равными
вероятностями    $1/k$ при любом фиксированном $k\hm\in\N$. В~этом
случае с.в.\ $N_n/g(n)$  вообще не зависит от~$n$ и, значит, слабо
сходится к с.в.~$V$,  которая принимает значения $1, 2, \ldots , k$ с
равными вероятностями~$1/k$.

\section{Доказательство теоремы~3.1}

Используя формулы~(\ref{e2.1-ben})--(\ref{e2.3-ben}),  получаем оценку:
\begin{multline}
\sup\limits_{x}\left| \p\left(\sigma g^\gamma(n)(T_{N_n}  -  \mu)  <
x\right) -  G_n(x)\right| \leq {}\\
{}\leq I_{1n} + I_{2n}\,, \label{e4.1-ben}
\end{multline}
где
\begin{multline}
I_{1n} =
 \sup\limits_{x}\left| \, \int\limits_{1/g(n)}^\infty \left(
 \vphantom{\sum\limits_{j=1}^l}
 F(x y^\gamma) +{}\right.\right.\\
\left. {}+
\sum\limits_{j=1}^l \left(yg(n)\right)^{-j/2}f_j(xy^\gamma)\right)  d
\left({\p}\left(\fr{N_n}{g(n)} < y\right) -{}\right.\\
\left.\left.{}- H(y) - \sum\limits_{i=1}^m
n^{-i/2} h_i(y)\right)
\vphantom{\int\limits_{1/g(n)}^\infty}\right|\,; \label{e4.2-ben}
\end{multline}

\vspace*{-12pt}

\noindent
\begin{multline}
I_{2n}  =  \sum\limits_{k=1}^{\infty} \sup\limits_{x}  \left|
\vphantom{\sum\limits_{j=1}^l}
 {\p}\left(\sigma
k^\gamma(T_{k}  -  \mu)  < x\left(\fr{k}{g(n)}\right)^\gamma\right)\right.
-{}
\\
- F\left(x \left(\fr{k}{g(n)}\right)^\gamma\right) - {}\\
\left.{}-\sum\limits_{j=1}^l
k^{-j/2}f_j\left(x\left(\fr{k}{g(n)}\right)^\gamma\right)\right|
{\p}(N_n=k)\,.    \label{e4.3-ben}
\end{multline}
Для оценки величины  $I_{1n}$ используем равенство~(\ref{e4.2-ben}),  
условие~1.2, формулу интегрирования по частям (см., например,~\cite{5-ben},
теорема~2.6.11, с.~222 или~\cite{15-ben}, теорема~18.4, с.~236) и
ограниченность функций $f_j(z)$, $j=1,\ldots ,l$, получим, что существует
константа $C_3\hm>0$ такая, что
\begin{multline*}
I_{1n} \le \fr{C_3}{n^\beta} + \sup\limits_{x}\left|\,
\int\limits_{1/g(n)}^\infty \left({\p}\left(\fr{N_n}{g(n)} <
y\right) -{}\right.\right.\\
\left.{}- H(y) - \sum\limits_{i=1}^m n^{-i/2} h_i(y)\right) 
d \left(\vphantom{\sum\limits_{j=1}^l}
F(x y^\gamma) + {} \right.\\
\left.\left.{}+\sum\limits_{j=1}^l
\left(yg(n)\right)^{-j/2}f_j(xy^\gamma)\right)
\vphantom{\int\limits_{1/g(n)}^\infty} \right| \le{}
\\
{}
\le \fr{C_3}{n^\beta} +   \sup\limits_{x}  \int\limits_{1/g(n)}^\infty
\left| \vphantom{\sum\limits_{j=1}^l}
{\p}\left(\fr{N_n}{g(n)} < y\right) -{}\right.\\
\left.{}- H(y) - \sum\limits_{i=1}^m
n^{-i/2} h_i(y)\right| \times{}
\end{multline*}

\noindent
\begin{multline}
{}\times  \left|\fr{\partial}{\partial y} \left(F(xy^\gamma) +
\sum\limits_{j=1}^l (yg(n))^{-j/2} f_j(xy^\gamma)\right)\right|\, dy \le{}\\
{}\le
\fr{C_3}{n^\beta} + \fr{C_1M_n}{n^\beta}\,.  \label{e4.4-ben}
\end{multline}
Ряд  в определении $I_{2n}$   (см.~(\ref{e4.3-ben})) оценим с помощью условия~1.1 и получим:
\begin{equation}
I_{2n}  \leq C_1  \sum\limits_{k=1}^{\infty} \fr{1}{k^\alpha}\, {\p}(N_n=k)
= C_1 \e N_n^{-\alpha}\,.    \label{e4.5-ben}
\end{equation}
Теперь утверждение теоремы следует из неравенств~(\ref{e4.1-ben}), (\ref{e4.4-ben}) и~(\ref{e4.5-ben}).
Теорема доказана.


\section{Примеры}

Приведем два примера применения теоремы~3.1 с вполне конкретными
предельными функциями распределения статистик, построенных по
выборкам случайного объема. Рассмотрим а.р.\
для ф.р.\ выборочных средних, построенных по выборкам случайного
объема. Аналогичные результаты могут быть получены для статистик,
допускающих а.р.\ типа Эдж\-вор\-та для ф.р.\ при
неслучайном объеме выборки. Например, используя результаты работ~[13--18], 
можно получить а.р.\ для  ф.р.\
ранговых статистик, $L$-ста\-ти\-стик и $U$-ста\-ти\-стик.

Пусть $X_1,X_2,\ldots$~--- независимые одинаково распределенные
случайные величины с ${\sf E}X_1 \hm= \mu$, $0\hm<{\sf D}X_1
\hm=\sigma^{-2}$, $\e |X_1|^{3+2\delta} \hm< \infty$,
$\delta\hm\in(0,1/2)$ и ${\sf E}(X_1 - \mu)^3 \hm= \mu_3$. Для
натурального~$n$ обозначим
\begin{equation}
T_n = \fr{X_1 + \cdots + X_n}{n}\,.\label{e5.1-ben}
\end{equation}
Предположим также, что случайная величина $X_1$ удовлетворяет
условию Крам$\acute{\mbox{е}}$ра ($C$)
$$
\limsup\limits_{|t|\to\infty} |\e \exp\{itX_1\}| < 1\,,
$$
тогда с учетом  теоремы~6.3.2 из~\cite{22-ben} получаем, что
\begin{multline}
\sup\limits_x \left|\vphantom{\fr{\mu_3 \sigma^3}{6 \sqrt n}}
{\p}\left(\sigma \sqrt n(T_n - \mu) < x\right) - \Phi(x)
- {}\right.\\
\left.{}-\fr{\mu_3 \sigma^3}{6 \sqrt n} \left(1 - x^2\right) \varphi(x) \right| \leq
\fr{C_1}{n^{1/2+\delta}}\,,  \\  C_1 > 0\,, \  \delta \in
\left(0,\fr{1}{2}\right)\,,
\   n \in \N\,. \label{e5.2-ben}
\end{multline}
Таким образом, статистика~(\ref{e5.1-ben}) удовлетворяет условию~1.1~с

\noindent
\begin{gather}
\gamma = \fr{1}{2}\,; \ \   \alpha = \fr{1}{2} +\delta\,; \ \ \   l = 1\,;
\label{e5.3-ben}
\\
F(x) = \Phi(x)\,;\ \    f_1(x) = \fr{\mu_3 \sigma^3}{6} (1 -\ x^2)
\varphi(x)\,. \label{e5.4-ben}
\end{gather}
Справедлива следующая лемма.

\smallskip

\noindent
\textbf{Лемма 5.1.} \textit{Пусть $l \hm= 1$,  $0 \hm< g(n) \uparrow \infty$,
$F(x) \hm= \Phi(x)$,
$$
f_1(x) = \fr{\mu_3 \sigma^3}{6} (1 - x^2) \varphi(x)\,.
$$
Тогда для величины $M_n$ $($см.\ теорему~$3.1)$ справедливо
неравенство
$$
M_n \le  2 + \widetilde C|\mu_3|\sigma^3\,,
$$
где}
\begin{multline*}
\widetilde C = \fr{1}{3}\,\sup\limits_{u\ge0}
\left\{\varphi(u)(u^4+2u^2+1)\right\}=\fr{16}{3\sqrt{2\pi
e^3}}\approx {}\\
{}\approx 0{,}474752293191785\ldots
\end{multline*}


\smallskip

\noindent
Д\,о\,к\,а\,з\,а\,т\,е\,л\,ь\,с\,т\,в\,о\,.\ \  С~учетом формул~(\ref{e5.1-ben})--(\ref{e5.4-ben}) 
имеем (см.\ теорему~3.1):
\begin{multline*}
M_n = \sup\limits_x \int\limits_{(g(n))^{-1}}^\infty
\left|
\fr{\partial}{\partial y} \left( \vphantom{\sum\limits_{j=1}^l}
F(xy^\gamma) + {}\right.\right.\\
\left.\left.{}+\sum\limits_{j=1}^l
(yg(n))^{-j/2} f_j(xy^\gamma)\right)\right|\, dy  ={}
\\
{}= \sup\limits_x \int\limits_{(g(n))^{-1}}^\infty
\left|\fr{\partial}{\partial y} \left(
\vphantom{\fr{\mu_3\sigma^3(1-x^2 y)\varphi(x\sqrt y)}{6
\sqrt{yg(n)}}}
\Phi(x\sqrt y) +{}\right.\right.\\
\left.\left.{}+
\fr{\mu_3\sigma^3(1-x^2 y)\varphi(x\sqrt y)}{6
\sqrt{yg(n)}}\right)\right|\, dy \le{}
\\
{}\le \sup\limits_{x\ge0} \int\limits_{x(g(n))^{-1/2}}^\infty
\left|\fr{\partial}{\partial u} \left(
\vphantom{\fr{x\mu_3\sigma^3(1-u^2)\varphi(u)}{6u \sqrt{g(n)}}}
\Phi(u) +{}\right.\right.\\
\left.\left.{}+\fr{x\mu_3\sigma^3(1-u^2)\varphi(u)}{6u \sqrt{g(n)}}\right)\right|
\,du +{}
\\
{}+ \sup\limits_{x<0} \int\limits_{-\infty}^{x(g(n))^{-1/2}}
\left|\fr{\partial}{\partial u} \left(
\vphantom{\fr{x\mu_3\sigma^3(1-u^2)\varphi(u)}{6u \sqrt{g(n)}}}
\Phi(u) +{}\right.\right.\\
\left.\left.{}+
\fr{x\mu_3\sigma^3(1-u^2)\varphi(u)}{6u \sqrt{g(n)}}\right)\right|
\,du ={}
\\
{}= \sup\limits_{x\ge0} \int\limits_{x(g(n))^{-1/2}}^\infty \!\!\!\!\!\!\!\!\!\!\varphi(u)
\left|1 + \fr{x\mu_3\sigma^3}{6
\sqrt{g(n)}}\fr{(u^4-2u^2-1)}{u^2}\right|\, du +{}\hspace*{-5.40286pt}
\end{multline*}

\noindent
\begin{multline}
{}+ \sup\limits_{x<0} \!\!\!\int\limits_{-\infty}^{x(g(n))^{-1/2}} \!\!\!\!\!\!\!\!\!\!\varphi(u)
\left|1 + \fr{x\mu_3\sigma^3}{6
\sqrt{g(n)}}\fr{(u^4-2u^2-1)}{u^2}\right|\, du \le{}\hspace*{-0.40298pt}
\\
\hspace*{-1.8mm}{}\le 2 + \fr{|\mu_3|\sigma^3}{3\sqrt{g(n)}} \sup\limits_{x\ge0} x
\!\!\!\int\limits_{x(g(n))^{-1/2}}^\infty\!\!\!\!\!\!\!\!\!\!\varphi(u)
\fr{u^4+2u^2+1}{u^2} \,du.\!\!\label{e5.5-ben}
\end{multline}
Далее заметим, что справедливо соотношение
$$
\sup\limits_{u\ge0}
\left\{\varphi(u)(u^4+2u^2+1)\right\}=\fr{1}{\sqrt{2\pi}}\,\sup\limits_{u\ge0}e^{-u}(2u+1)^2\,.
$$
Легко видеть, что $\left(e^{-u}(2u+1)^2\right)^\prime\hm=e^{-u}(2u\hm+1)(3\hm-2u)\hm=0$
при $u={3}/{2}$. Таким образом,
\begin{multline*}
\hspace*{-1.36589pt}\fr{1}{\sqrt{2\pi}}\,\sup\limits_{u\ge0}e^{-u}(2u+1)^2
=\fr{1}{\sqrt{2\pi}}\,e^{-u}(2u+1)^2\Big|_{u=3/2}={}\\
{}=\fr{16}{\sqrt{2\pi
e^3}}\approx 1{,}42425687951535\ldots ,
\end{multline*}
так что
\begin{multline}
\widetilde C=\fr{1}{3}\sup\limits_{u\ge0}
\left\{\varphi(u)(u^4+2u^2+1)\right\}=\fr{16}{3\sqrt{2\pi
e^3}}\approx{}\\
{}\approx 0{,}474752293191785\ldots 
\label{e5.6-ben}
\end{multline}
При этом из неравенства~(\ref{e5.5-ben}) следует, что
\begin{multline*}
M_n \le 2 + \widetilde C \fr{|\mu_3|\sigma^3}{\sqrt{g(n)}}
\sup\limits_{x\ge0} x \!\int\limits_{x(g(n))^{-1/2}}^\infty \!\!u^{-2}\, du
={}\\
{}= 2 + \widetilde C|\mu_3|\sigma^3\,.
\end{multline*}
Таким образом, справедливо неравенство:
\begin{equation}
M_n \le  2 + \widetilde C|\mu_3|\sigma^3\,,\label{e5.7-ben}
\end{equation}
где постоянная $\widetilde C$ определена в соотношении~(\ref{e5.6-ben}). Из
неравенства~(\ref{e5.7-ben}) следует утверждение леммы. Лемма доказана.

\subsection{Распределение Стьюдента}

В работе~\cite{3-ben} показано, что если случайный объем выборки $N_n$
имеет отрицательно биномиальное распределение с параметрами $p \hm=
1/n$ и $r \hm> 0$,  т.\,е.\
$$
{\p}(N_n = k) = \fr{(k+r-2)\cdots r}{(k-1)!}\, \fr{1}{n^r} \left(1
- \fr{1}{n}\right)^{k-1}\!\!, \   k\in\N
$$
(при $r=1$ имеем геометрическое распределение), то для
асимптотически нормальной статистики $T_n$ справедливо предельное
соотношение~(\cite{3-ben}, следствие~2.1)
\begin{equation}
{\p}(\sigma\sqrt{n} (T_{N_n} - \mu) < x)\longrightarrow G_{2r}(x
\sqrt r), \ \  n\to\infty,\!\! \label{e5.8-ben}
\end{equation}
где $G_{f}(x)$~--- функция распределения Стьюдента с параметром $f \hm= 2r$, соответствующая 
плотности вида

\noindent
$$
p_{f}(x) = \fr{\Gamma(f+1/2)}{\sqrt{\pi f}\, \Gamma(f/2)}\left(
1+\fr{x^2}{f}\right)^{-(\gamma+1)/2}\,,\enskip   x\in\R\,,
$$
где $\Gamma(\cdot)$~--- эйлерова гам\-ма-функ\-ция, а $f\hm>0$~--- параметр
формы (если параметр $f$ натурален, то он называется числом степеней
свободы). В~рас\-смат\-ри\-ва\-емой ситуации он может быть произвольно мал,
т.\,е.\ может иметь место типичное распределение с тяжелыми
хвостами. Если $f\hm=2$, т.\,е.\ $r\hm=1$, то ф.р.\ $G_2(x)$ выражается в
явном виде:

\noindent
$$
G_2(x) = \fr{1}{2}\left( 1+\fr{x}{\sqrt{2+x^2}} \right)\,, \enskip  x\in\R\,.
$$
При $r=1/2$ имеем распределение Коши.

В книге~\cite{22-ben} (формула~(6.112)) приведена следующая оценка
скорости сходимости:

\noindent
\begin{multline}
\sup_{x\ge0} \left|{\p}\left(\fr{N_n}{\e N_n} <  x\right) -
H_r(x)\right| \leq 
\begin{cases}
\displaystyle\fr{C_r}{n}\,, &  r \ge 1;\\
\displaystyle\fr{C_r}{n^r}\,, &  r \in (0,1),
\end{cases}\\  C_r > 0\,,\ \
  n\in\N\,,\label{e5.9-ben}
\end{multline}
где $H_r(x)$~--- функция гам\-ма-рас\-пре\-де\-ле\-ния с параметром $r \hm> 0$:

\noindent
\begin{equation}
H_r(x) = \fr{r^r}{\Gamma(r)} \int\limits_0^x e^{-ry} y^{r-1} \,dy\,, 
\enskip
 x\ge 0\,,\label{e5.10-ben}
\end{equation}
 При этом
\begin{equation}
\e N_n = r(n - 1) + 1\,. \label{e5.11-ben}
\end{equation}
Таким образом, из соотношений~(\ref{e5.9-ben})--(\ref{e5.11-ben}) следует, что случайный
индекс $N_n$ удовлетворяет условию~1.2 с
\begin{gather*}
g(n) = r(n - 1) + 1\,; \enskip  H(x) = H_r(x)\,; \enskip  m = 1\,; % \label{e5.12-ben}
\\
h_1(x) \equiv 0\,;  \enskip   C_2 = C_r > 0\,; %  \label{e5.13-ben}
\\
\beta = \begin{cases} 1\,, &  r \ge 1\,;\\ 
r\,, &  r \in (0,1)\,.
\end{cases}
%\label{e5.14-ben}
\end{gather*}
Далее, используя равенство

\noindent
\begin{multline*}
(1 + x)^\gamma = \sum\limits_{k=0}^\infty \fr{\gamma(\gamma -
1)\cdots(\gamma - k + 1 )}{k!}\,  x^k\,, \\
    |x| < 1\,,\enskip  \gamma
 \in \R\,,
\end{multline*}
нетрудно получить, что
\columnbreak


\noindent
\begin{multline}
\e N_n^{-1} = \fr{1}{(n - 1) \left(1 - r\right)}\left(\fr{1}{n^{r-1}} -
1\right) ={}\\
{}= {O}(n^{-r})\,, \ \ \   r > 0\,, \ \   r \ne 1\,, \ \  n \in \N\,.
\label{e5.15-ben}
\end{multline}
Для случая $r=1$, используя формулу~(\ref{e3.3-ben}), имеем:
\begin{equation}
\e N_n^{-1} = \fr{1}{n - 1} \log n\,, \enskip  n > 1\,. \label{e5.16-ben}
\end{equation}
Таким образом, учитывая теорему~3.1, следствие~3.2, формулы~(\ref{e5.2-ben})--(\ref{e5.4-ben}),
а также лемму~5.1, соотношения~(\ref{e5.15-ben}), (\ref{e5.16-ben}) и
равенства (справедливые равномерно по~$x$)
\begin{gather}
\int\limits_{(r(n-1)+1)^{-1}}^\infty \!\!\!\Phi(x\sqrt y)\, dH_r(y) =
 \int\limits_{0}^\infty \Phi(x\sqrt y)\, dH_r(y) +{}\notag\\
 \hspace*{15mm}{}+
{O}\left(\fr{1}{n}\right) =
 G_{2r}(x) + O\left(\fr{1}{n}\right)\,;\label{e5.17-ben}
\\
\int\limits_{(r(n-1)+1)^{-1}}^\infty \!\!\!\varphi(x\sqrt y)
\fr{1-x^2y}{\sqrt y}\, dH_r(y) ={}\notag\\
{}=
 \int\limits_{0}^\infty \varphi(x\sqrt y) \fr{1-x^2y}{\sqrt y}\,
dH_r(y) +  o(1) \equiv\notag\\
{}\hspace*{30mm}\equiv g_{r}(x) + {\it o}(1)\,,\label{e5.18-ben}
\end{gather}
получаем следующее утверждение.

\smallskip

\noindent
\textbf{Теорема 5.1.} \textit{Пусть статистика $T_n$ имеет вид $(\ref{e5.1-ben})$,
где  $X_1,X_2,\ldots$~--- независимые одинаково распределенные с.в.\ с
${\sf E}X_1 \hm= \mu$, $0\hm<{\sf D}X_1 \hm=\sigma^{-2}$, $\e
|X_1|^{3+2\delta} \hm< \infty$, $\delta\hm\in(0,1/2)$ и ${\sf E}(X_1 -
\mu)^3 \hm= \mu_3$, причем с.в.\ $X_1$ удовлетворяет условию
Крам$\acute{\mbox{е}}$ра $(C)$. Предположим, что при некотором $r\hm>0$
случайная величина $N_n$ имеет распределение вида:
\begin{multline*}
{\p}(N_n = k) = {}\\
{}=\fr{(k+r-2)\cdots r}{(k-1)!} \,\fr{1}{n^r} \left(1
- \fr{1}{n}\right)^{k-1}\,, \quad   k\in\N.
\end{multline*}
Тогда при $r > 1/(1+2\delta)$ для ф.р.\ нормированной статистики
$T_{N_n}$ при $n\to\infty$ справедливо а.р.\ вида
\begin{multline*}
\hspace*{-7.7pt}\sup\limits_x \left|{\p}\left(\sigma\sqrt {r(n-1)+1} (T_{N_n} - \mu) <
x\right) - G_{2r}(x) -{}\right.\\
\left.{}- \fr{\mu_3\sigma^3g_r(x)}{6\sqrt{r(n-1)+1}}
\right| ={}
\\
{}= \begin{cases} 
{O}\left(\left(\fr{\log
n}{n}\right)^{1/2+\delta}\right)\,, & \quad \hspace*{2pt} r = 1\,;\\[4pt]
{O}\left(\fr{1}{n^{\min(1, r(1/2+\delta))}}\right)\,,
&\quad \hspace*{2pt}  r > 1\,;\\[4pt]
{O}\left(\fr{1}{n^{r(1/2+\delta)}}\right)\,, &
\hspace*{-11mm}\fr{1}{1+2\delta} < r < 1\,,
\end{cases}
\end{multline*}
где функции $G_{2r}(x)$ и $g_r(x)$ определены в соотношениях
$(\ref{e5.17-ben})$ и $(\ref{e5.18-ben})$.}

\subsection{Распределение Лапласа}

Рассмотрим распределение Лапласа с ф.р.\ $\Lambda_\theta(x)$ и
плотностью
$$
\lambda_\theta(x)=\fr{1}{\theta\sqrt 2}\exp\left\{
-\fr{\sqrt{2}|x|}{\theta} \right\}\,, \ \ \   \theta > 0\,,\  x\in\R\,.
$$
В работе~\cite{9-ben} была построена последовательность с.в. $N_n(s)$,
зависящая от параметра $s \in \N$, сле\-ду\-юще\-го вида. Пусть $Y_1, Y_2,
\ldots$~--- независимые одинаково распределенные с.в., имеющие
непрерывную ф.р. Определим с.в.
$$
N(s) = \min\left\{ i\geq1: \max\limits_{1\leq j\leq s} Y_j < \max\limits_{s+1\leq k
\leq s+i} Y_k \right\}\,.
$$
Хорошо известно, что так определенные с.в.\ имеют распределение вида
\begin{equation}
{\p}(N(s) \geq k) = \fr{s}{s+k-1}\,, \enskip   k \geq \ 1\label{e5.19-ben}
\end{equation}
(см., например,~\cite{26-ben, 27-ben}). Пусть теперь  $N^{(1)}(s),
N^{(2)}(s),\ldots$~--- независимые одинаково распределенные с.в.,
имеющие распределение~(\ref{e5.19-ben}). Определим с.в.\
$$
N_n(s) = \max\limits_{1\leq j\leq n} N^{(j)}(s)\,,
$$
тогда, как показано в работе~\cite{9-ben},
\begin{equation}
\lim\limits_{n\to\infty} {\p}\left( \fr{N_n(s)}{n} < x \right) = e^{-s/x}\,,
\enskip  x>0\,,\label{e5.20-ben}
\end{equation}
и для асимптотически нормальной статистики $T_n$ справедливо соотношение:
\begin{multline*}
{\p}\left(\sigma\sqrt{n}(T_{N_n(s)} - \mu) < x\right) \longrightarrow{}\\
{}\longrightarrow
\Lambda_{1/s}(x)\,,  \quad n\to\infty\,,\enskip  x\in\R\,,
\end{multline*}
где $\Lambda_{1/s}(x)$~--- функция распределения Лапласа с параметром
$\theta\hm=1/s$.

В работе~\cite{11-ben} была получена следующая оценка скорости
сходимости в соотношении~(\ref{e5.20-ben}):
\begin{multline}
\sup\limits_{x\ge0} \left|\:{\p}\left(\fr{N_n(s)}{n} <  x\right) -
e^{-s/x} \right| \leq \fr{C_s}{n}\,, \\    C_s > 0\,, \quad
n\in\N\,.\label{e5.21-ben}
\end{multline}
Таким образом, из соотношения~(\ref{e5.21-ben}) следует, что случайный индекс
$N_n(s)$ удовлетворяет усло\-вию~1.2~с
\begin{equation}
g(n) = n\,; \ \  H(x) = e^{-s/x}\,; \ \  m = 1\,; \label{e5.22-ben}
\end{equation}
\begin{equation}
h_1(x) \equiv 0\,; \ \    C_2 = C_s > 0\,;  \ \  \beta = 1\,.\label{e5.23-ben}
\end{equation}
Рассмотрим более подробно величину $ \e N_n^{-1}(s)$. Из определения
с.в.\ $N_n(s)$ и равенства~(\ref{e5.19-ben}) имеем
\begin{multline*}
{\p}(N_n(s) = k) = \left( \fr{k}{s+k}\right)^n - \left( \fr{k -
1}{s+k-1}\right)^n ={}\\
{}=
 sn \int\limits_{k-1}^ k \fr{x^{n-1}}{(s + x)^{n+1}}\, dx\,,
\end{multline*}
поэтому
\begin{multline*}
\e N_n^{-1}(s) = \sum\limits_{k=1}^\infty \fr{1}{k} \,{\p}(N_n(s) = k) ={}\\
{}=
 sn \sum\limits_{k=1}^\infty \fr{1}{k} \int\limits_{k-1}^ k
\fr{x^{n-1}}{(s + x)^{n+1}}\, dx \leq{}
\\
{}\leq sn \sum\limits_{k=1}^\infty \int\limits_{k-1}^k \fr{x^{n-2}}{(s +
x)^{n+1}}\, dx =
 sn \int\limits_{0}^\infty \fr{x^{n-2}}{(s + x)^{n+1}} \,dx\,.\hspace*{-0.76227pt}
\end{multline*}
Для вычисления последнего интеграла используем формулу (см.~\cite{13-ben} формула~856.12, с.~184):
$$
\int\limits_{0}^\infty \fr{x^{s-1}}{(a + bx)^{s+n}} \,dx =
\fr{\Gamma(s)\Gamma(n)}{a^n b^s\Gamma(s+n)}\,,  \ \ \  a, b, s, n>0\,.
$$
Получим
\begin{equation*}
\e N_n^{-1}(s) \leq sn \fr{\Gamma(n-1)\Gamma(2)}{s^2\Gamma(n+1)} =
\fr{1}{s(n - 1)} = {O}(n^{-1})\,. \label{e5.24-ben}
\end{equation*}
Таким образом, учитывая теорему~3.1, следствие~3.2, формулы~(\ref{e5.2-ben})--(\ref{e5.4-ben}), 
а также лемму~5.1, соотношения~(\ref{e5.22-ben}), (\ref{e5.23-ben}) и
равенства (справедливые равномерно по~$x$)
\begin{gather}
\int\limits_{n^{-1}}^\infty \Phi(x\sqrt y)\, de^{-s/y} =
\int\limits_{0}^\infty \Phi(x\sqrt y) \,de^{-s/y} +
{O}\left(\fr{1}{n}\right) ={}\notag\\
\hspace*{30mm}{}= \Lambda_{1/s}(x) +
{O}\left(\fr{1}{n}\right)\,;\label{e5.25-ben}
\\
\hspace*{-20mm}\int\limits_{n^{-1}}^\infty \varphi(x\sqrt y) \fr{1-x^2y}{\sqrt
y}\, de^{-s/y} = {}\notag\\
{}=\int\limits_{0}^\infty \varphi(x\sqrt y)
\fr{1-x^2y}{\sqrt y}\, de^{-s/y} + {\it o}(1) \equiv{}\notag\\
\hspace*{40mm}{}\equiv l_{s}(x) +
{\it o}(1)\,,\label{e5.26-ben}
\end{gather}
непосредственно получаем следующую теорему.

\smallskip

\noindent
\textbf{Теорема 5.2.} \textit{Пусть статистика $T_n$ имеет вид $(\ref{e5.1-ben})$,
где $X_1,X_2,\ldots$~--- независимые одинаково распределенные с.в.\ с
${\sf E}X_1 \hm= \mu$, $0<{\sf D}X_1 \hm=\sigma^{-2}$, $\e
|X_1|^{3+2\delta} \hm< \infty$, $\delta\hm\in(0,1/2)$ и ${\sf E}(X_1 -
\mu)^3 \hm= \mu_3$, причем с.в.\ $X_1$ удовлетворяет условию
Крам$\acute{\mbox{е}}$ра $(C)$. Предположим, что при некотором $s\hm\in\N$ 
с.в.\ $N_n(s)$ имеет распределение вида:
$$
{\p}(N_n(s) = k) = \left( \fr{k}{s+k}\right)^n - \left( \fr{k -
1}{s+k-1}\right)^n\,, \ \ \   k\in\N\,.
$$
Тогда для ф.р.\ нормированной статистики $T_{N_n(s)}$ справедливо а.р.\ вида:
\begin{multline*}
\sup\limits_x \left| \vphantom{\fr{\mu_3\sigma^3l_s(x)}{6\sqrt{n}}}
{\p}\left(\sigma\sqrt {n} (T_{N_n(s)} - \mu) < x\right)
- \Lambda_{1/s}(x) - {}\right.\\
\left.{}-\fr{\mu_3\sigma^3l_s(x)}{6\sqrt{n}} \right| =
{O}\left(\fr{1}{n^{1/2+\delta}}\right)\,, \ \ n \to \infty\,,
\end{multline*}
где функции $\Lambda_{1/s}(x)$ и $l_s(x)$ определены соответственно
в соотношениях $(\ref{e5.25-ben})$ и~$(\ref{e5.26-ben})$.}

{\small\frenchspacing
{%\baselineskip=10.8pt
\addcontentsline{toc}{section}{Литература}
\begin{thebibliography}{99}

\bibitem{2-ben} %1
\Au{Гнеденко Б.\,В.} Об оценке неизвестных параметров
распределения при случайном числе независимых наблюдений~// Тр.
Тбилисского математического института, 1989. Т.~92. С.~146--150.

\bibitem{1-ben} %2
\Au{Гнеденко Б.\,В., Фахим Х.} Об одной теореме переноса~// Докл. АН
СССР, 1969. Т.~187. С.~15--17.

\bibitem{12-ben} %3
\Au{Von Ghossy R., Rappl G.} Some approximation methods for the
distri\-bution of random sums~// Insurance: Mathematics and
Economics, 1983. Vol.~2. P.~251--270.

\bibitem{6-ben}  %4
\Au{Круглов В.\,М., Королев~В.\,Ю.} Предельные теоремы для случайных
сумм.~--- М.: Изд-во Московского ун-та, 1990.

\bibitem{14-ben} %5
\Au{Королев В.\,Ю.} Предельные распределения для случайно
индексированных   последовательностей и их применения: Дисс. \ldots\
докт. физ.-мат. наук.~--- М.: МГУ, 1993.

\bibitem{7-ben} %6
\Au{Gnedenko B.\,V., Korolev V.\,Yu.} Random summation. Limit
theorems and applications.~--- Boca Raton: CRC Press, 1996.

\bibitem{8-ben}  %7
\Au{Bening V.\,E., Korolev V.\,Yu.} Generalized Poisson models and
their applications in insurance and finance.~--- Utrecht: VSP, 2002.

\bibitem{4-ben} %8
\Au{Гнеденко Б.\,В.} Курс теории вероятностей.~--- М.: Наука, 1988.

\bibitem{24-ben} %9
\Au{Королев В.\,Ю.} О~взаимосвязи обобщенного распределения
Стьюдента и дисперсионного гам\-ма-рас\-пре\-де\-ле\-ния при статистическом
анализе выборок случайного объема~// Докл. РАН, 2012. Т.~445.
Вып.~6. С.~622--627.

\bibitem{25-ben} %10
\Au{Климов Г.\,П.} Теория вероятностей и
математическая статистика.~--- М.: Изд-во Московского ун-та,
1983.

\bibitem{5-ben} 
\Au{Ширяев А.\,Н.} Вероятность.~--- М.: Наука, 1989.

\bibitem{15-ben} 
\Au{Billingsley P.} Probability and measure.~--- John Wiley \&
Sons, 1995.

\bibitem{16-ben} 
\Au{Bickel P.\,G.} Edgeworth expansions in nonparametric
statistics~// Ann. Stat., 1974. Vol.~2. P.~1--21.

\bibitem{17-ben} %14
\Au{Albers W.} Asymptotic  expansions and the deficiency
concept in statistics~// Mathematical Centre Tracts 58.~---
Amsterdam: Mathematisch Centrum, 1974.

\bibitem{19-ben} %15
\Au{Albers W., Bickel P.\,G., Van Zwet~W.\,R.} Asymptotic
expansions for the power of distribution free tests in the
one-sample problem~// Ann. Stat., 1976. Vol.~4. P.~108--156.

\bibitem{20-ben} %16
\Au{Bickel P.\,G., Van Zwet~W.\,R.} Asymptotic expansions for the
power of distribution free tests in the two-sample problem~// Ann.
Stat., 1978. Vol.~6. P.~947--1004.

\bibitem{18-ben} %17
\Au{Helmers R.} Edgeworth   expansions for linear combinations
of order statistics~// Mathematical Centre Tracts 105.~--- Amsterdam:
Mathematisch Centrum, 1984.

\bibitem{21-ben} %18
\Au{Bentkus V., Gotze F., Van Zwet~W.\,R.} An Edgeworth
expansions for symmetric statistics~// Ann. Stat., 1997. Vol.~25.
P.~851--896.

\bibitem{22-ben} 
\Au{Бенинг В.\,Е., Королев~В.\,Ю., Соколов~И.\,А., Шоргин~С.\,Я.}
Рандомизированные модели и методы теории надежности информационных и
технических систем.~--- М.: ТОРУС ПРЕСС, 2007.

\bibitem{3-ben} 
\Au{Бенинг В.\,Е., Королев В.\,Ю.} Об использовании распределения
Стьюдента в задачах теории вероятностей и математической статистики~// 
Теория вероятностей и ее применения, 2004. Т.~49. Вып.~3. С.~417--435.

\bibitem{9-ben} 
\Au{Бенинг В.\,Е., Королев В.\,Ю.} Некоторые статистические задачи,
связанные с распределением Лапласа~// Информатика и её применения,
2008. Т.~2. Вып.~2. С.~19--34.

\bibitem{26-ben} 
\Au{Wilks S.\,S.} Recurrence of extreme observations~// J.~Amer. Math. Soc., 
1959. Vol.~1. No.~1. P.~106--112.

\bibitem{27-ben} 
\Au{Невзоров В.\,Б.} Рекорды. Математическая теория.~--- М.: Фазис, 2000.

\bibitem{11-ben} 
\Au{Лямин О.\,О.} О~скорости сходимости распределений некоторых
статистик к распределению Лапласа и Стьюдента~// Вестник Московского
ун-та. Сер.~15: Вычислительная  математика и кибернетика,
2011. Вып.~1. С.~39--47.

\label{end\stat}


\bibitem{13-ben} 
\Au{Двайт Г.\,Б.} 
Таблицы интегралов и другие математические формулы.~--- М.: Наука, 1977.
\end{thebibliography}
}
}

\end{multicols}  %8

%\renewcommand{\r}{\mathbb{R}}

\newcommand{\abs}[1]{\left|#1\right|}
\newcommand{\ex}{C_0}
%\newcommand{\lowex}{C_1}
\newcommand{\exlow}{C_1}
\newcommand{\exlowk}{C_k}
\renewcommand{\le}{\leqslant}
\renewcommand{\ge}{\geqslant}
\renewcommand{\d}{\delta}
\newcommand{\bet}{\beta_{2+\delta}}

\newcommand{\gd}{\gamma(\delta)}
\newcommand{\kd}{\varkappa(\delta)}
\newcommand{\sign}{\mathrm{sign}\,}
\newcommand{\R}{\mathbb R}
\newcommand{\C}{\mathbb C}
\newcommand{\To}{\longrightarrow}

\newcommand{\ud}{\rho(F_n,\Phi)} %uniform distance


\def\stat{sevts}

\def\tit{УТОЧНЕНИЕ НЕРАВЕНСТВА КАЦА--БЕРРИ--ЭССЕЕНА$^*$}

\def\titkol{Уточнение неравенства Каца--Берри--Эссеена}

\def\autkol{М.\,Е.\,Григорьева, И.\,Г.~Шевцова}
\def\aut{М.\,Е.\,Григорьева$^1$, И.\,Г.~Шевцова$^2$}

\titel{\tit}{\aut}{\autkol}{\titkol}

{\renewcommand{\thefootnote}{\fnsymbol{footnote}}\footnotetext[1]
{Работа выполнена при поддержке Российского фонда фундаментальных
исследований (проекты 08-01-00563, 08-01-00567, 08-07-00152 и
09-07-12032-офи-м), а также Министерства образования и науки
(государственные контракты П1181 и П958, грант МК-581.2010.1).}}

\renewcommand{\thefootnote}{\arabic{footnote}}
\footnotetext[1]{Московский
государственный университет имени М.\,В.~Ломоносова, факультет
вычислительной математики и кибернетики, maria-grigoryeva@yandex.ru}
\footnotetext[2]{Московский государственный университет
имени М.\,В.~Ломоносова, факультет вычислительной математики и
кибернетики, ishevtsova@cs.msu.su}

\Abst{Уточнены верхние оценки абсолютной константы в
неравенстве Каца--Берри--Эссеена для сумм независимых одинаково
распределенных случайных величин с конечными абсолютными моментами
порядка $2+\d$, $0<\d<1$. Предложена альтернатива неравенству
Каца--Берри--Эссеена, имеющая более тонкую структуру, и построены
верхние оценки входящих в уточненное неравенство констант.}

\KW{центральная предельная теорема; неравенство
Каца--Берри--Эссеена; дробь Ляпунова}

       \vskip 18pt plus 9pt minus 6pt

      \thispagestyle{headings}

      \begin{multicols}{2}

      \label{st\stat}
  

\section{Введение}

При решении многих прикладных задач приходится учитывать эффекты,
возникающие в результате суммарного воздействия большого числа
случайных факторов, отдельный вклад каждого из которых в сумму
пренебрежимо мал. Чаще всего в таких ситуациях статистические
закономерности поведения суммы в силу центральной предельной теоремы
аппроксимируются нормальным распределением вероятностей. При этом
точность нормальной аппроксимации зависит от наличия у случайных
слагаемых моментов достаточно высокого порядка или, другими
словами, тяжестью (или легкостью) их <<хвостов>>. Известно, что при
некоторых достаточно общих условиях нормальная аппроксимация
адекватна, если случайные слагаемые имеют моменты хотя бы второго
порядка, причем чем выше порядок момента, тем, как правило, выше
точность нормальной аппроксимации. При этом большой интерес
представляет ситуация, когда случайные слагаемые имеют моменты,
порядок которых заключен между двумя и тремя: с одной стороны, для
распределений, имеющих моменты порядка, большего трех, скорость
сходимости в центральной предельной теореме остается в общем случае
такой же, как для распределений с третьими моментами; с другой
стороны, во многих прикладных задачах важно оценивать точность
нормальной аппроксимации, когда центральная предельная теорема все
еще выполняется, но слагаемые имеют распределения со столь тяжелыми
<<хвостами>>, что третьего момента уже не существует. Такие задачи
возникают, например, в страховании, когда речь заходит о
маловероятных, но экстремально больших выплатах по тому или иному
страховому случаю. Другие примеры связаны с практическим применением
моделей типа распределения Парето с <<хвостами>>, убывающими степенным
образом, при анализе трафика в телекоммуникационных сис\-те\-мах. Часто
статистический анализ таких моделей позволяет сделать вывод, что
показатель степени заключен между тремя и четырьмя, т.\,е.\
дисперсия существует, а третий момент отсутствует. Улучшению оценок
точности нормальной аппроксимации именно для таких ситуаций и
посвящена данная работа.

Для $0\le\d\le 1$ обозначим через $\F_{2+\d}$ множество функций
распределения с нулевым средним, единичной дисперсией и конечным
абсолютным моментом $\bet$ порядка ${2+\d}$. При $\d=0$ полагаем
$\beta_2=1$ и $\F_2$~--- класс всех распределений с нулевым средним и
единичной дисперсией. Пусть $X_1,\ldots,X_n$~--- независимые
одинаково распределенные случайные величины с общей функцией
распределения $F\in\F_{2+\d}$, заданные на некотором вероятностном
пространстве $(\Omega,\mathcal{A},\p)$. Обозначим
\begin{align*}
F_n(x)&=F^{*n}(x\sqrt{n})=
\p\left(\fr{X_1+\ldots+X_n}{\sqrt{n}}<x\right)\,;
\\
\Phi(x)&=\fr{1}{\sqrt{2\pi}}\int\limits_{-\infty}^xe^{-t^2/2}\,dt\,,\quad
x\in\R\,.
\end{align*}

Классическое неравенство Каца--Берри--Эс\-се\-ена устанавливает
существование конечной положительной постоянной $\ex=\ex(\d)$,
зависящей только от~$\d$, которая гарантирует справедливость
неравенства

\noindent
\begin{equation}
\!\left.
\begin{array}{rl}
\ud &\equiv\sup_x|F_n(x)-\Phi(x)|\le \ex(\delta)
L_n^{2+\delta}\,;\\[6pt]
 L_n^{2+\delta}&=\fr{\bet}{n^{\delta/2}}
 \end{array}\!
 \right \}\!\!\!\!\!
\label{Bikelis}
\end{equation}
для всех $n\ge1$ и $F\in\F_3$. 

Для  $\d=1$
неравенство~(\ref{Bikelis}) было доказано независимо и одновременно
Э.~Берри~\cite{Berry1941} и К.-Г.~Эссе-\linebreak еном~\cite{Esseen1942}. В~1960-е~гг.\
разными авторами были предприняты успешные попытки обобщить\linebreak
результат Берри--Эссеена. Так, в 1963~г.\ М.~Кац~\cite{Katz1963}
доказал аналог~(\ref{Bikelis}) для независимых одинаково
распределенных случайных величин с $\e X_1^2g(X_1)<\infty$ для
функций $g$ из некоторого класса, включающего $g(x)=|x|^\d$. 
В~1965~г.\ В.\,В.~Пет\-ров~\cite{Petrov1965} обобщил неравенство Каца
на разнораспределенные слагаемые. В~1966~г.\
А.~Бикялис~\cite{Bikelis1966} доказал неравномерную оценку для
разнораспределенных случайных величин, имеющих конечные абсолютные
моменты порядка $2+\d$, $0<\d\le1$, из которой также вытекает
неравенство~(\ref{Bikelis}). Точные формулировки  упомянутых
результатов вместе с их доказательствами можно найти, например, в
монографии В.\,В.~Петрова~\cite{Petrov1972}.

Относительно константы $\ex(1)$ известно, что
$$
\fr{\sqrt{10}+3}{6\sqrt{2\pi}}\le\ex(1)\le 0{,}4784
$$
(нижняя оценка найдена К.-Г.~Эссееном~\cite{Esseen1956}, верхняя~--- В.\,Ю.~Королевым и 
И.\,Г.~Шевцовой~\cite{KorolevShevtsova2010}).

Верхние оценки величины $\ex(\d)$ при некоторых $0<\d<1$ впервые
были получены в 1983~г.\ В.~Тысиаком~\cite{Tysiak1983} 
(см.\ также~\cite{Paditz1996}) и недавно были уточнены в 
работе~\cite{GaponovaKorchaginShevtsova2009}. В 1986~г.\
Г.~Падитц~\cite{Paditz1986} показал, что для всех $F\in\F_2$ и
$n\ge1$ имеет место неравенство
$$
\rho(F_n,\Phi)\le 3{,}51{\e}\left(X_1^2\min\left\{1,
\fr{|X_1|}{\sqrt{n}}\right\}\right)\,,
$$
откуда вытекает равномерная по $\delta\in[0,1]$ оценка
$\ex(\delta)\le 3{,}51$, так как при любом $\delta\in[0,1]$ выражение
в правой части последнего неравенства не превосходит $3{,}51\cdot
L_n^{2+\delta}$.

Несмотря на то, что со времени первой публикации верхних оценок
прошло более 25~лет, нижние оценки для величины $\ex(\d)$ получены
совсем недавно (см.~\cite{Shevtsova2010}). Перечисленные
оценки указаны в табл.~1: во втором
столбце~--- верхние оценки Тысиака~\cite{Tysiak1983}, в третьем~---
верхние оценки из работы~\cite{GaponovaKorchaginShevtsova2009}, в
пятом~--- нижние оценки из~\cite{Shevtsova2010}; в четвертом же
столбце указаны новые оценки, доказанные в данной работе. Для
удобства в первой строке таблицы приведен год соответствующей
публикации.


Из табл.~1 видно, что представленные в данной работе верхние оценки
константы $\ex(\d)$ не очень далеки от неулучшаемых: зазор между
найденной\linebreak\vspace*{-12pt}
\columnbreak

%\bigskip

%\begin{center} %tabl1
\noindent
{{\tablename~1}\ \ \small{История верхних оценок, а также нижние оценки
константы $\ex(\d)$}}
%\end{center}
\vspace*{2pt}

{\small \begin{center}
\tabcolsep=11.8pt
\begin{tabular}{|c|c|c|c|c|}
\hline
Год & 1983  & 2009 & 2010 &   2010  \\
\hline
$\d$ & $\ex \le$ & $\ex\le$ & $\ex\le$ & $\ex\ge$ \\
\hline
0,9 & 0,802  & 0,7671 & 0,5383 &   0,2133  \\
0,8 & 0,812  & 0,7720 & 0,5723 &   0,2245  \\
0,7 & 0,833  & 0,7876 & 0,6026 &   0,2376  \\
0,6 & 0,863  & 0,8126 & 0,6276 &   0,2530  \\
0,5 & 0,902  & 0,8454 & 0,6413 &   0,2715  \\
0,4 & 0,950  & 0,8876 & 0,6342 &   0,2939  \\
0,3 & 1,008  & 0,9407 & 0,6195 &   0,3220  \\
0,2 & 1,076  & 1,0001 & 0,6094 &   0,3585  \\
0,1 & 1,102  & 1,0739 & 0,6028 &   0,4092  \\
\hline
\end{tabular}
\end{center}
}
%\vspace*{6pt}


\bigskip

\begin{center} %tabl2
\noindent
\parbox{56mm}{{\tablename~2}\ \ \small{Двусторонние оценки
константы $\exlow(\d)$}}
%\end{center}
\vspace*{4pt}

{\small 
\tabcolsep=16pt
\begin{tabular}{|c|c|c|c|c|}
\hline \vphantom{$\frac{\displaystyle R}2$}
  $\d$ & $\exlow\le$ & $\exlow\ge$ \\
\hline
0,9 &0,3089 &0,0323 \\
0,8 &0,3187 &0,0356 \\
0,7 &0,3334 &0,0396 \\
0,6 &0,3528 &0,0444 \\
0,5 &0,3775 &0,0503 \\
0,4 &0,4080 &0,0575 \\
0,3 &0,4450 &0,0665 \\
0,2 &0,4901 &0,0780 \\
0,1 &0,5451 &0,0939 \\
  \hline
\end{tabular}
}
\end{center}

\addtocounter{table}{2}

\bigskip

\noindent
 мажорантой и соответствующей минорантой со\-став\-ля\-ет всего
0,2--0,3, а их отношение колеблется в пределах 1,5--2,5 в
зависимости от~$\d$.


Наряду с уточнением константы $\ex(\d)$ в~(\ref{Bikelis}) данная
работа ставит своей целью уточнение и самой структуры
неравенства~(\ref{Bikelis}). А~именно, в качестве альтернативы
предлагается рассмотреть неравенство
\begin{equation}
\label{K-B-E-sharpened}
\ud\le \exlow(\d)\fr{\bet+1}{n^{\d/2}}\,,\enskip n\ge1\,,\
F\in\F_{2+\d}\,,
\end{equation}
справедливость которого с некоторым положительным и конечным
$\exlow(\d)$ вытекает тривиальным образом из~(\ref{Bikelis})
(например, с $\exlow(\d)=2\ex(\d)$ в силу условия $\bet\ge1$).
Однако константа $\exlow(\d)$ в~(\ref{K-B-E-sharpened})\linebreak
 оказывается
гораздо меньше, чем $\ex(\d)$ в~(\ref{Bikelis}), поэтому
неравенство~(\ref{K-B-E-sharpened}) при достаточно больших~$\bet$
дает оценку, заведомо лучшую, чем~(\ref{Bikelis}), несмотря на то
что для его справедливости необходима та же априорная информация о
распределении~$F$ (а именно, только значение абсолютного момента~$\bet$). 
Кроме того, более оптимистичными оказываются и нижние
оценки~$\exlow(\d)$, построенные в работе~\cite{Shevtsova2010}.
Например, для $\d=1$ двусторонняя оценка имеет вид
$0{,}2659\le\exlow(1)\le 0{,}3041$~[13--16]. Для $0<\d<1$ верхние
оценки константы~$\exlowk(\d)$, устанавливаемые в данной статье, и
нижние, полученные в
работе~\cite{Shevtsova2010}, приведены в
табл.~2 во втором и третьем
столбцах соответст\-венно.


Рассуждения, приводящие к форме неравенства~(\ref{K-B-E-sharpened}),
основаны на используемых оценках для характеристических функций и
подробно описаны в~\cite{KorolevShevtsova2010, KorolevShevtsova2009}.

\section{Главный результат и~основные идеи его доказательства}

\noindent
\textbf{Теорема 1.}
\textit{Для константы $\exlowk(\d)$ в неравенстве}
$$
\ud\le
\exlowk(\d)\fr{\bet+k}{n^{\d/2}}\,,\quad n\ge1\,,\ F\in\F_{2+\d}\,,
$$
\textit{при $k=0$ и $k=1$ имеют место оценки, приведенные в
табл.}~3.

\medskip

\begin{center} %tabl1
\noindent
\parbox{56mm}{{\tablename~3}\ \ \small{Верхние оценки
констант $\ex(\d)$ и $\exlow(\d)$}}
%\end{center}
\vspace*{4pt}

{\small 
\tabcolsep=16pt
\begin{tabular}{|c|c|c|}
  \hline $\d$ & $\ex\le$ & $\exlow\le$ \\
\hline
0,9 & 0,5383 &  0,3089  \\
0,8 & 0,5723 &  0,3187  \\
0,7 & 0,6026 &  0,3334  \\
0,6 & 0,6276 &  0,3528  \\
0,5 & 0,6413 &  0,3775  \\
0,4 & 0,6342 &  0,4080  \\
0,3 & 0,6195 &  0,4450  \\
0,2 & 0,6094 &  0,4901  \\
0,1 & 0,6028 &  0,5451  \\
  \hline
\end{tabular}
}
\end{center}

\addtocounter{table}{1}

\bigskip


%\medskip


При доказательстве теоремы~1 будем придерживаться
подхода, предложенного и развитого В.\,М.~Золотарёвым в его
работах~[17--19]. Этот
подход основан на применении неравенств сглаживания, которые
позволяют оценить расстояние между функциями распределения через
расстояние между соответствующими характеристическими функциями. 
В~рамках этого подхода ключевыми моментами являются: ($i$)~выбор
надлежащего неравенства сглаживания; ($ii$) выбор в нем сглаживающего
ядра; ($iii$)~конструирование оценок для характеристических функций и
($i\nu$) выбор вычислительной оптимизационной процедуры. Опишем эти
моменты в той последовательности, в которой они появляются при
доказательстве неравенств~(\ref{Bikelis}) и~(\ref{K-B-E-sharpened}).
Соответствующие утверждения сформулируем в виде лемм.
{\looseness=1

}

\medskip

Обозначим $f(t)$ и $f_n(t)$ характеристические функции случайных
величин~$X_1$ и стандартизованной суммы $(X_1+\ldots+X_n)/\sqrt{n}$
соответственно:

\noindent
\begin{align*}
f(t)&=\int\limits_{-\infty}^\infty e^{itx}\,dF(x)\,;\\
f_n(t)&=\int\limits_{-\infty}^{\infty}e^{itx}\,dF_n(x)=
\left(f\left(\fr{t}{\sqrt n}\right)\right)^n\,,\quad t\in\R\,.
\end{align*}
Пусть
$$
r_n(t)=\left|f_n(t)-e^{-t^2/2}\right|\,.
$$

\medskip

\noindent
\textbf{Лемма 1} (см.~\cite{Prawitz1972}). %\label{LemPrawitzSmoothIneq}
\textit{Для произвольной функции распределения~$F$ при всех $n\ge1$,
$0<t_0\le1$ и $U>0$ имеет место неравенство}
\begin{multline*}
\ud\le 2\int\limits_0^{t_0}\left|K(t)\right|r_n(Ut)\,dt+{}\\
{}+
2\int\limits_{t_0}^{1}|K(t)\left|\cdot|f_n(Ut)\right|\,dt+{}\\
{}+
2\int\limits_0^{t_0}\left|K(t)-\fr i{2\pi t}\right|e^{-U^2t^2/2}\,dt +
\fr{1}{\pi}\int\limits_{t_0}^\infty e^{-U^2t^2/2}\,\fr{dt}t\,,\hspace*{-0.98pt}
\end{multline*}
\textit{где}
\begin{multline*}
K(t)=\fr{1}{2}\left(1-|t|\right)+\fr {i}{2}\left[(1-|t|)\cot\pi
t+\frac{\sign t}\pi\right]\,,\\
-1\le t\le1\,. %\eqno(8)
\end{multline*}

\smallskip

Перейдем теперь к оцениванию характеристических функций,
фигурирующих в лемме~1. Пусть
$\theta_0(\d)$~--- единственный корень уравнения
$$
\fr{\d\theta^2}2+ \theta\sin \theta + (2+\d)(\cos \theta - 1)=0\,,
\quad \pi\le\theta\le2\pi\,;
$$

\vspace*{-12pt}

\noindent
\begin{multline*}
\kd \equiv \sup_{x>0}\fr{\left|\cos
x-1+x^2/2\right|}{x^{2+\d}}={}\\
{}=\fr{\cos
x-1+x^2/2}{x^{2+\d}}\Bigg|_{x=\theta_0(\d)}\,;
\end{multline*}

\vspace*{-12pt}

\noindent
\begin{multline*}
\!\!\gd= \sup_{x>0}\sqrt{\left(\fr{\cos x-1+x^2/2}{x^{2+\d}}\right)^2\! +\!
\left(\fr{\sin x-x}{x^{2+\d}}\right)^2 }\,,\\ 0<\d\le 1\,.
\end{multline*}
Для $\eps>0$ положим
$$
\psi_\d(t,\eps)= 
\begin{cases}
  t^2/2-\kd\eps|t|^{2+\d}\,, & |t|<\theta_0(\d)\eps^{-1/\d}\,;\\[6pt]
  \fr{1-\cos\big(\eps^{1/\d} t\big)}{\eps^{2/\d}}\,,
      &\!\!\!\!\! \theta_0(\d)\le\eps^{1/\d}|t|\le 2\pi\,;\\[6pt]
  0\,, & |t|>2\pi\eps^{-1/\d}\,.
\end{cases}
$$
Несложно убедиться, что функция~$\psi_\delta(t,\eps)$ монотонно убывает 
по~$\eps$ при каждом фиксированном $t\in\R$.

Ляпуновская дробь будет обозначаться $\ell=$\linebreak $=\bet n^{-\d/2}$.
Дополнительно обозначим
$$
\ell_n=\ell+n^{-\d/2}\,.
$$

\smallskip

\noindent
\textbf{Лемма 2}.
\textit{При всех $F\in\F_{2+\d}$, $0<\d\le1$, $n\ge1$ и $t\in\R$ (если не
оговорено иное) справедливы оценки}
$$
|f_n(t)|\le  \left[1-\fr{2}{n}\psi_\d(t,\ell_n)\right]^{n/2} \equiv
f_1(t,\ell_n,n)\,;
$$
$$
|f_n(t)|\le \exp\{-\psi_\d(t,\ell_n)\}\equiv  f_2(t,\ell_n)\,;
$$
$$
|f_n(t)|\le \exp\left\{-\fr{t^2}{2}+\kd\ell_n|t|^{2+\d} \right\}\equiv
f_3(t,\ell_n)\,;
$$

\vspace*{-6pt}

\noindent
\begin{multline*}
\!r_n(t)\le e^{-t^2/2}\left[\exp\left\{\! \gd\ell|t|^{2+\d}-
n\ln\left(\!1-\fr{t^2}{2n}\!\right)-{}\right.\right.\hspace*{-0.67pt}\\
{}-\left.\left. \fr{t^2}2\right\}-1\right]\equiv
r_1(t,\ell,n), \enskip |t|<\sqrt{2n};
\end{multline*}

\vspace*{-9pt}

\noindent
\begin{multline*}
r_n(t)\le  \left(\gd{\ell|t|^{2+\d}}+\fr{|t|^4}{8n}\right)
\times{}\\
{}\times 
\left(\max\left\{f_1(t,\ell_n,n),\,e^{-t^2/2}\right\}\right)^{(n-1)/n}
\equiv r_2(t,\ell,n)\,;
\end{multline*}

\vspace*{-9pt}

\noindent
\begin{multline*}
r_n(t)\le  \fr{1}{2}\left(\gd{\ell|t|^{2+\d}}+\fr{|t|^4}{8n}\right)
\left(
e^{-t^2/2}+{}\right.\\
{}+\left.\max\left\{f_1(t,\ell_n,n),\,e^{-t^2/2}\right\}\right)
e^{t^2/(2n)}\equiv r_3(t,\ell,n)\,;
\end{multline*}

\vspace*{-3pt}

\noindent
$$
r_n(t)\le f_1(t,\ell_n,n)+e^{-t^2/2}\equiv r_4(t,\ell,n)\,.
$$

\medskip

\noindent
\textbf{Замечание 1.}
Очевидно, $f_1(t,\eps,n)\le f_2(t,\eps)$ при всех $n\ge1$, $\eps>0$
и $t\in\R$. Более того, из результата работы~\cite{Shevtsova2009}
вытекает, что $f_2(t,\eps)\le f_3(t,\eps)$ для всех $\eps>0$ и
$t\in\R$, так что самую точную оценку для~$|f_n(t)|$ дает 
$f_1(t,\ell_n,n)$, тогда как функции $f_j(t,\ell_n)$, $j=2,3$,
обладают полезным свойством монотонности по~$\ell_n$, играющим
важную роль в оптимизационной процедуре.

\medskip

\noindent
Д\,о\,к\,а\,з\,а\,т\,е\,л\,ь\,с\,т\,в\,о\,.\
Первые три оценки ($f_j$, $j=$\linebreak $=1,2,3$) являются тривиальными
следствиями оценок
$$
|f(t)|^2\le 1-2\psi_\d(t,\bet+1)\,,\quad t\in\R\,;
$$
$$
|f(t)|^2\le 1-t^2 + 2\kd(\bet+1)|t|^{2+\d}\,,\quad t\in\R\,,
$$
полученных Шевцовой в~\cite{Shevtsova2009}. Четвертая оценка ($r_1$)
впервые объявлена В.\,М.~Зо\-ло\-та\-рё\-вым~\cite{Zolotarev1966}
для $\d=1$ (без доказательства), ниже будет приведено полное
доказательство для всех $0<\d\le1$. Пятая ($r_2$) и шестая ($r_3$)
оценки являются несложной комбинацией методов и результатов работ
Правитца~\cite{Prawitz1975}, Шевцовой~\cite{Shevtsova2009},
Гапоновой и Шевцовой~\cite{GaponovaShevtsova2009}. Последняя 
оценка~($r_4$) тривиальна.

Докажем неравенство $r_n(t)\le r_1(t,\ell,n)$, $|t|<$\linebreak $<\sqrt{2n}$. Из
неравенств $|e^{ix}-1-ix|\le|x|^2/2$ и $|e^{ix}-1-ix-(ix)^2/2|\le
\gd|x|^{2+\d}$, $x\in\R$, с учетом моментных условий для
распределения $F\in\F_{2+\d}$ вытекают соотношения
\begin{gather*}
|f(t)-1|\le\fr{t^2}{2}\,,\quad|t|\le\sqrt2\,,\\
f(t)=1-\fr{t^2}2 + \theta_1 \gd \bet|t|^{2+\d}\,,\quad t\in\R\,,
%\label{ch_f_expansion}
\end{gather*}
с некоторым $\theta_1\in\C$, $|\theta_1|\le1$. Следовательно, для
всех $|t|<\sqrt{2n}$ определен логарифм~$\ln f(t)$ (условимся всегда
выбирать главную ветвь логарифма) и
\begin{multline*}
\abs{\ln f(t)+\fr{t^2}2}=\abs{\ln[1-(1-f(t))]+\fr{t^2}2}={}\\
{}=
\left|-\sum_{k=1}^\infty\fr{(1-f(t))^k}k+\fr{t^2}2\right|\le{}\\
{}\le
\sum_{k=2}^\infty\fr{1}{k}\left(\fr{t^2}2\right)^k+
\abs{f(t)-1+\fr{t^2}2}\le{}\\
{}\le  -\left[\ln\left(1-\fr{t^2}{2}\right)
+\fr{t^2}2\right]+\gd\bet|t|^{2+\d}\,,\\ |t|<\sqrt{2n}\,,
\end{multline*}
откуда с учетом неравенства $\abs{e^z-1}\le e^{|z|}-1$, $z\in\C$,
получаем
\begin{multline*}
r_n(t)=\abs{f_n(t)-e^{-t^2/2}}={}\\
{}=e^{-t^2/2} \left|\exp\left\{n\ln
f\left(\fr{t}{\sqrt n}\right)+\fr{t^2}2\right\}- 1\right|\le{}
\\
{}\le e^{-t^2/2} \left(\exp\left\{n\left|\ln f\left(\fr{t}{\sqrt
n}\right)+\fr{t^2}{2n}\right|\right\}-1\right)\le{}
\\
{}\le e^{-t^2/2} \left(\exp\left\{\gd \fr{\bet|t|^{2+\d}}{n^{\d/2}}
-n\ln \left(1-\fr{t^2}{2n}\right)-{}\right.\right.\\
{}-\left.\left.\fr{t^2}{2}\right\}-1\right)\equiv
r_1(t,\ell,n)\,,
\end{multline*}
что и требовалось доказать.

\medskip

Следующая лемма позволяет ограничить сверху множество
рассматриваемых значений~$n$ при оценивании констант~$\exlowk(\d)$ в
неравенстве~(\ref{K-B-E-sharpened}) с $0\le k\le 1$.

\medskip

\noindent
\textbf{Лемма 3.} %\begin{lemma}\label{LemMonotone}
\textit{Для любых положительных $k\le1$, $T$, $\eps$,
\begin{multline*}
N_1 \ge N_1(T) \equiv{}\\
{}\equiv T^2 \left( \fr{1}{8k\d\gd} + \sqrt{1 + \left(
\fr{1}{8k\d\gd}\right)^2}\ \right)^2\,;
\end{multline*}
$$
N_3 \ge N_3(T,\eps) \equiv \left( \fr{ T^{2-\delta}}{4\d\gd} +
\fr{\eps T^2}{\delta}\right)^{2/(2-\delta)}
$$
и таких, что $N_j\ge((1+k)/\eps)^{2/\d}$, при всех $|t| \le T$
справедливы оценки}
\begin{multline*}
\sup\limits_{n\ge N_1}r_1 \left(t, \eps - kn^{-\d/2}, n \right) \le{}\\
{}\le e^{-t^2/2}
\left( \exp \left\{ \gd \eps |t|^{2+\d} \right\} - 1
\right)\equiv\widetilde r_1(t, \eps)\,;
\end{multline*}

\vspace*{-9pt}

\noindent
\begin{multline*}
\sup\limits_{n\ge N_3}r_3\left(t,\eps-n^{-\delta/2},n\right)\le{}\\
{}\le \fr{\gd \eps
|t|^{2+\delta}}{2} \left( e^{\kd\eps |t|^{2+\delta}} + 1 \right)
e^{-t^2/2}\equiv \widetilde r_3(t,\eps)\,.
\end{multline*}

\medskip

\noindent
Д\,о\,к\,а\,з\,а\,т\,е\,л\,ь\,с\,т\,в\,о\,.\
Для доказательства первой оценки запишем~$r_1$ в виде
\begin{multline*}
r_1 \left(t, \eps - kn^{-\d/2}, n \right) = {}\\
{}=e^{-t^2/2} \left( \exp
\left\{ \gd \eps |t|^{2+\d} + g(n, |t|) \right\} - 1 \right)\,,
\end{multline*}
где
$$
g(n, |t|) = -\fr{k\gd}{n^{\d/2}} |t|^{2+\d} - n \ln \left(1 -
\fr{t^2}{2n} \right) - \fr{t^2}{2}\,.
$$
Тогда достаточно показать, что $g(x, t) \le 0$ для всех $0 \le t \le
T$ и $x \ge N_1(T,\eps)$.

Используя разложение логарифма в степенной ряд, для всех $0 \le t
\le \sqrt{2x}$ и $x > 0$ получаем
\begin{multline*}
g(x, t) = -\fr{k\gd}{x^{\d/2}}|t|^{2+\d} + x
\sum\limits_{j=2}^{\infty} \fr{1}{j} \left(\fr{t^2}{2x} \right)^j
\le{}\\
{}\le
 -\fr{k\gd}{x^{\d/2}}|t|^{2+\d} + \fr{x}{2}
\sum\limits_{j=2}^{\infty} \left(\fr{t^2}{2x} \right)^j ={}
\\
{}
= -\fr{k\gd}{x^{\d/2}}\,|t|^{2+\d} + \fr{t^4}{4(2x - t^2)} \equiv
\widetilde{g}(x, t)\,.
\end{multline*}
Заметим, что для $0 \le t \le \sqrt{2x}$ и $x > 0$
\begin{multline*}
\fr{\partial \widetilde{g}(x, t)}{\partial x} = \fr{k\d\gd
t^{2+\d}}{2 x^{1+\d/2}} - \fr{t^4}{2(2x - t^2)^2} > 0
\Longleftrightarrow{} \\
{}\Longleftrightarrow h(x, t) \equiv k\d\gd(2x - t^2)^2 -
t^{2-\d} x^{1+\d/2} > 0\,.
\end{multline*}

Пусть $T$~--- произвольное число из интервала $(0, \sqrt{2x})$.
Несложно видеть, что функция~$h(x, t)$ монотонно убывает по~$t$,
поэтому для всех $0 \le t \le T$
\begin{multline*}
h(x, t) \ge h(x, T) = x \left( 4k\d\gd x - 4k\d\gd T^2 -{}\right.\\
{}-\left.
x^{\d/2}T^{2-\d}\right) + k\d\gd T^4 > \\
> x \left( 4k\d\gd x - 4k\d\gd T^2 - x^{\d/2}T^{2-\d}\right)\,.
\end{multline*}
Для неотрицательности последнего выражения достаточно, чтобы
$$
H(x)\equiv 4k\d\gd (x - T^2) - x^{\d/2}T^{2-\d}\ge0\,.
$$
Очевидно, для всех достаточно больших~$x$ функция~$H(x)$ монотонно
возрастает и не ограничена при $x\to\infty$. Следовательно, найдется
такая точка $x_0>0$, что $H(x)\ge0$ для всех $x\ge x_0$. Будем
искать эту точку в виде $x_0=zT^2$. Не ограничивая общности, можно
считать, что $z>1$, поскольку $H(T^2)=-T^2<0$. Имеем
\begin{multline*}
H(zT^2) = 4k\d\gd T^2(z-1) - T^2z^{\d/2} > 0 
\Longleftrightarrow{}\\
{}\Longleftrightarrow{} 4k\d\gd(z-1) - z^{\d/2} > 0\,.
\end{multline*}
Поскольку $z^{\d/2} \le \sqrt{z}$ при $z>1$, для справедли\-вости
последнего условия достаточно, чтобы
$$
4k\d\gd z - \sqrt{z} - 4k\d\gd > 0\,,
$$
откуда, решив квадратное уравнение, получаем
$$
\sqrt{z} >\fr{1 + \sqrt{1 + 64(\d\gd k)^2}}{8\d\gd k}\equiv z_0\,.
$$
Следовательно, $x_0=z_0^2T^2$ и $H(x) > 0$ при
$$
x\ge  T^2 \left( \fr{1}{8\d\gd k} + \sqrt{1 + \left( \fr{1}{8\d\gd
k}\right)^2} \right)^2\,.
$$

Таким образом, для всех $0 \le t \le T$ и $x \ge N_1(T,\eps)$ имеем
$h(x, t)>xH(x)>0$, а значит~$\widetilde{g}(x, t)$ монотонно
возрастает по $x \ge N_1(T,\eps)$ при каждом фиксированном $0<t\le
T$ и для всех $N_1 \ge N_1(T,\eps)$
\begin{multline*}
\sup\limits_{0 \le t \le T} \sup_{x \ge N_1} g(x, t) \le \sup\limits_{0 \le t \le
T} \sup\limits_{x \ge N_1} \widetilde{g}(x, t) ={}\\
{}= \sup\limits_{0 \le t \le T}
\lim_{x\rightarrow\infty} \widetilde{g}(x, t) = 0\,,
\end{multline*}
что и требовалось доказать.

Далее заметим, что в силу неравенства $f_1(t,\ell_n,n)\le
f_3(t,\ell_n)$, $t\in\R$, величину~$r_3$ можно оценить следующим
образом:
\begin{multline*}
r_3\left(t,\eps-n^{-\delta/2},n\right)\le{}\\
{}\le \fr{1}{2} \left( \gd
\eps|t|^{2+\delta} - \fr{\gd |t|^{2+\delta}}{n^{\delta/2}} +
\fr{t^4}{8n}\right) \left(
\vphantom{\fr{t^2}{2}} e^{-{t^2}/{2}} +{}\right.\\
{}\left. \exp \left\{-\fr{t^2}{2} +
\varkappa(\delta) \eps |t|^{2+\delta} - \fr{\varkappa(\delta)
|t|^{2+\delta}}{n^{\delta/2}}\right\} \right) e^{t^2/(2n)} \le{}
\\
{}\le \fr{1}{2} \left( \gd \eps|t|^{2+\delta} - \fr{\gd
|t|^{2+\delta}}{n^{\delta/2}} + \fr{t^4}{8n}\right)\times{}\\
{}\times \left( 1 +
e^{\varkappa(\delta) \eps |t|^{2+\delta}}\right)e^{t^2/(2n)-t^2/2} ={}
\\
{}= \fr{|t|^{2+\delta}}{2} \left( \gd \eps - \fr{\gd}{n^{\delta/2}}
+ \fr{|t|^{2 - \delta}}{8n}\right)\times{}\\
{}\times \left( 1 + e^{\varkappa(\delta)
\eps |t|^{2+\delta}}\right) e^{t^2/(2n)-t^2/2} \equiv{}
\\
{}\equiv \fr{|t|^{2+\delta}}{2} \left( 1 + e^{\varkappa(\delta) \eps
|t|^{2+\delta}}\right) \exp \left( -\fr{t^2}{2} + g(n, |t|)
\right)\,,
\end{multline*}
где
\begin{multline*}
g(x, t) = \fr{t^2}{2x} + \ln \left( \gd\eps  -
\fr{\gd}{x^{{\delta}/{2}}} + \fr{t^{2 - \delta}}{8x}
\right)\,,\\
 x>\left(\fr{2}{\eps}\right)^{2/\delta}\,,\quad  \eps,\ t > 0\,.
\end{multline*}
Заметим, что выражение под знаком логарифма положительно.

Покажем, что при всех фиксированных положительных~$\eps$ и $t\le T$
функция~$g(x, t)$ монотонно возрастает по~$x$ при $x \ge
N_3(T,\eps)$. Вычислим производную
\begin{multline*}
\!\!\!\!g_x' (x, t) = -\fr{t^2}{2x^2} + \frac{(\delta/2) \gd
x^{-1-\d/2} - t^{2-\delta}/(8x^2)}{\gd\eps - {\gd}{x^{-\d/2}}
+ {t^{2-\delta}}/(8x)}=
\\
{}= \left (-8\gd\eps x t^2 + 8\gd x^{1-\d/2}t^2 - t^{4-\delta} +{}\right.\\
\left.{}+ 8\d\gd
x^{2-\d/2} - 2x t^{2-\delta}\right) \Big/
\left (
\vphantom{8\gd x^{1-\d/2} +
t^{2-\delta}}
2x^2 \left(
\vphantom{8\gd x^{1-\d/2} +
t^{2-\delta}}
8\gd\eps x -{}\right.\right.\\
\left.\left.{}- 8\gd x^{1-\d/2} +
t^{2-\delta}\right)\right)\,.
\end{multline*}
Знаменатель в последнем выражении совпадает с точностью до множителя
$2x^3$ с выражением под знаком логарифма в определении $g(x,t)$, а
следовательно, он положителен. В таком случае условие $g_x'(x,t)>0$
равносильно неравенству
\begin{multline*}
x\left(8\d\gd x^{1-\d/2} - \left(2t^{2-\delta} + 8\gd\eps t^2\right)\right) +{}\\
{}+ 8\gd
x^{1-\d/2} t^2 - t^{4-\delta} \ge 0\,,
\end{multline*}
для чего достаточно, чтобы

\noindent
\begin{multline*}
x^{1-\d/2} \ge \max\left\{\fr{t^{2-\delta}}{4\d\gd} + \fr{\eps
t^2}{\delta},\, \frac{t^{2-\delta}}{8\gd}\right\}={}\\
{}=
\fr{t^{2-\delta}}{4\d\gd} + \fr{\eps t^2}{\delta}\,.
\end{multline*}

Таким образом, при всех $0\le t\le T$ и
\begin{multline*}
x \ge  N_3(T,\eps)={}\\
{}= \max \left\{
\left(\fr{2}{\eps}\right)^{2/\delta}, \left( \fr{
T^{2-\delta}}{4\d\gd} + \fr{\eps
T^2}{\delta}\right)^{2/(2-\delta)} \right\}
\end{multline*}
функция $g(x,t)$ монотонно возрастает по~$x$, а следовательно, при
всех $N_3\ge N_3(T,\eps)$
$$
\sup\limits_{0\le t\le T}\sup\limits_{n \ge N_3} g(n,t) = \sup\limits_{0\le t\le
T}\lim_{x\rightarrow\infty} g(x,t) = \ln(\gd\eps)\,,
$$
что и требовалось доказать. Лемма доказана.

\medskip

Наконец, правильно организовать процесс вычислительной оптимизации
позволяют следующие утверждения.

\noindent
\textbf{Лемма 4} (см.~[24]). %\begin{lemma} [см. \cite{BhatRangaRao1982}] \label{LemBhRRao}
\textit{Для любого распределения~$F$ с нулевым средним и единичной
дисперсией справедливо неравенство}
\begin{multline*}
\sup\limits_{x}|F(x)-\Phi(x)|\le
\sup\limits_{x>0}\left(\Phi(x)-\fr{x^2}{1+x^2}\right)={}\\
{}= 0{,}54093654\ldots
\end{multline*}

\medskip

\noindent
\textbf{Лемма 5.} (см.~[23]). %\begin{lemma}[см. \cite{GaponovaShevtsova2009}]
%\label{LemEps_le_0.3}
\textit{Для любой функции распределения $F\in\F_{2+\d}$ и всех $n\ge2$
таких, что $(\bet+1)/n^{\d/2}\le0{,}6$, справедливо неравенство
$$
\rho(F_n,\Phi) \le
C'(\d)\fr{\bet}{n^{\d/2}}+\fr{C''(\d)}{n^{\d/2}}
$$
с $C'(\d)$ и $C''(\d)$, указанными в
табл.}~4.

\smallskip

\begin{center} %tabl4
\noindent
\parbox{56mm}{{\tablename~4}\ \ \small{Значения $C'(\d)$ и
$C''(\d)$ из леммы~5 при некоторых $\d$}}
%\end{center}
\vspace*{2pt}

{\small 
\tabcolsep=16.1pt
\begin{tabular}{|c|c|c|}
\hline
$\d$ & $C'(\d)$ & $C''(\d)$ \\
\hline
0,9 & 0,3085 & 0,2399  \\
0,8 & 0,2987 & 0,2166  \\
0,7 & 0,2912 & 0,1921  \\
0,6 & 0,2852 & 0,1655  \\
0,5 & 0,2800 & 0,1382  \\
0,4 & 0,2765 & 0,1044  \\
0,3 & 0,2776 & 0,0714  \\
0,2 & 0,2915 & 0,0327  \\
0,1 & 0,1500 & 0,0021  \\
  \hline
\end{tabular}
}
\end{center}

\addtocounter{table}{1}

%\bigskip

Из леммы~5 вытекает, что при всех~$n$ и~$\bet$
таких, что ${(\bet+k)/n^{\d/2}<0.3(1+k)}$,
неравенство~(\ref{K-B-E-sharpened}) имеет место при любых
$k\in[0,\,1]$ и $\exlowk(\d)>0$, удовлетворяющих условию
$(k+1)\exlowk(\d) \ge C'(\d) + C''(\d)$.

Подставляя оценки для характеристических функций из леммы~2 в правую часть 
неравенства сглаживания Правитца из леммы~1, получаем некоторую функцию 
$D(\ell,n,t_0,U)$, мажорирующую равномерное расстояние 
$\rho(F_n,\Phi)$ при всех $U>0$, $t_0\in(0,1]$, $n\geqslant1$ и~$F$ 
с фиксированной ляпуновской дробью $\beta_{2+\delta}n^{-\delta/2}=\ell$.
Приведенные леммы дают основание ограничить область рассматриваемых
значений величины $\eps=(\bet+k)/n^{\d/2}$ некоторым конечным
отрезком, отделенным от нуля (подробнее об этом будет сказано ниже),
и искать константу~$C_k$ при каждом $k\in[0,\,1]$ в виде
\begin{equation}
\left.
\begin{array}{rl}
\exlowk(\d)&=\max\limits_{\eps}C_\d(\eps)\,;\\[6pt]
C_\d(\eps)&=\fr{D_\d(\eps)}{\eps}\,;\\[6pt]
D_\d(\eps)&=\sup\left\{D_\d(\eps,n)\colon n\ge n_*\right\}\,,
\end{array}
\right \}
\label{FormMax}
\end{equation}
где
\begin{gather*}
D_\d(\eps,n) =\inf\limits_{t_0,\,U>0} D\left(\eps-\fr{k}{n^{\d/2}},n,t_0,U\right)\,,\\
n_*=\max\left\{1,\,\left\lceil\left(\fr{1+k}{\ell}\right)^{2/\d}\right\rceil\right\}\,.
\end{gather*}
Здесь $\lceil x\rceil$~--- минимальное целое, не меньшее~$x$. Условие
$n\ge n_*$ является следствием неравенства $\bet\ge1$. При этом для
оценивания супремума по~$n$ вместо входящих в $D_\d(\eps,n)$ величин
$r_j$, $j=1,2,3,4$, для достаточно больших~$n$ используются их
монотонные мажоранты. Вычисление максимума по~$\eps$ существенно
опирается на свойство монотонного возрастания по~$\eps$ всех
используемых оценок для функций~$|f_n(t)|$ и~$r_n(t)$, а
следовательно, и величины $D_\d(\eps)=\eps C_\d(\eps)$. Это свойство
позволяет оценить $\max\limits_{\eps}C_\d(\eps)$ по значениям~$C_\d(\eps)$
лишь в конечном числе точек. А~именно имеет место

\medskip

\noindent
\textbf{Лемма 6.} %\begin{lemma}\label{LemMonDeps}
\textit{Для всех $\eps_2>\eps_1>0$ имеет место неравенство}
$$
\max\limits_{\eps_1\le\eps\le\eps_2}C_\d(\eps)\le
C_\d(\eps_2)\fr{\eps_2}{\eps_1}\,.
$$

\medskip

Минимизация функции $D(\eps-k{n}^{-\d/2},\,n,\,t_0,\,U)$ по~$t_0$ и
$U$ проводится численно c использованием стандартных процедур в
системе Matlab~7.3 (R2006b).

\medskip

Перейдем теперь к описанию алгоритма вычисления константы~$\ex(\d)$
в неравенстве~(\ref{Bikelis}). Положим
$$
\eps = \fr{\bet}{n^{\d/2}}\,.
$$
%\vspace*{-12pt}


\noindent
\begin{center} %tabl5
\vspace*{-8pt}

\noindent
\parbox{79mm}{{\tablename~5}\ \ \small{Экстремальные значения
$n^*$, $\eps^*=(n^*)^{-\d/2}$ и оптимальные $t_0,U$ при вычислении
константы $\ex(\d)$}}
%\end{center}

\vspace*{2ex}

{\small 
\tabcolsep=11pt
\begin{tabular}{|c|c|c|c|c|c|}
  \hline
  $\d$ & $\eps_{\max}$ & $n^*$ & $\eps^*$ & $t_0$ & $U$\\
  \hline 
  0,9& 1,006& 3& 0,610& 0,35& 4,94\\
  0,8& 0,946& 3& 0,644& 0,39& 4,40\\
  0,7& 0,898& 3& 0,681& 0,46& 3,81\\
  0,6& 0,862& 4& 0,660& 0,53& 3,71\\
  0,5& 0,844& 5& 0,669& 0,66& 3,35\\
  0,4& 0,853& 6& 0,699& 0,82& 3,09\\
  0,3& 0,874& 5& 0,786& 1,00& 2,69\\
  0,2& 0,888& 4& 0,871& 0,78& 2,42\\
  0,1& 0,898& 9& 0,896& 0,87& 2,53\\
  \hline
\end{tabular}
}
\end{center}

\addtocounter{table}{1}

\vspace*{12pt}


\noindent
Тогда при $\eps\le0.3$ неравенство~(\ref{Bikelis}) с~$\ex(\d)$,
указанной в теореме~1, вытекает из
леммы~5. C~другой стороны, лемма~4
позволяет ограничить сверху область рассматриваемых значений~$\eps$
величиной $0{,}5409\ldots/\ex(\d)\equiv\eps_{\max}(\d)$, фиксированной
при каждом~$\d$. Таким образом, при вычислении~$\ex(\d)$
максимизация по~$\eps$ в формулах~(\ref{FormMax}) проводится на
конечном отрезке $0{,}3\le\eps\le\eps_{\max}(\d)$. Значения правой
границы~$\eps_{\max}(\d)$ приведены в
табл.~5. Для оценки
характеристических функций при $n<100$ используется~$f_1$, а при
$n\ge100$~--- функция~$f_2$, монотонно убывающая по~$n$, что
позволяет при каждом~$\eps$ оценивать супремум $D_\d(\eps,n)$ по
значениям~$n$ лишь в конечном числе точек:
$n_*,\ldots,\max\{n_*,100\}$, где $n_*=\max\{1,\eps^{-2/\d}\}$.
Максимум $C_\d(\eps)=D_\d(\eps)/\eps$ по
$0{,}3\le\eps\le\eps_{\max}(\d)$ оценивается  с помощью
леммы~6 и не превосходит тех значений~$\ex(\d)$,
которые указаны в формулировке теоремы~1.
Экстремальные значения $n=n^*$ и $\eps=(\ell^*)^{-\d/2}$ указаны в
табл.~5 в третьем и четвертом
столбцах, а соответствующие оптимальные значения параметров~$t_0$ и~$U$~--- 
в пятом и шестом столбцах. Отметим, что точке экстремума соответствует $\bet=1$.


Пусть теперь $k=1$. Обозначим
$$
\eps=\fr{\bet+1}{n^{\d/2}}\,.
$$
Тогда при $\eps\le0{,}3$ неравенство~(\ref{K-B-E-sharpened}) является
следствием леммы~5, а при $\eps\ge
0{,}5409\ldots/\exlow(\d)\equiv$\linebreak $\equiv\eps_{\max}(\d)$~--- следствием
леммы~4. Таким образом, при вычислении~$\exlow(\d)$
максимизацию по~$\eps$ в формулах~(\ref{FormMax}) достаточно
проводить на отрезке $0{,}3\le\eps\le\eps_{\max}(\d)$. Значения
$\eps_{\max}(\d)$ приведены в
табл.~6. Из этой таблицы видно,
что максимальное рассматриваемое значение~$\eps$ не превосходит~1,76. 
Для вычисления супремума по $n\ge n_*$ используется
лемма~3 с $T=2{,}2$, соответствующие значения
$N_1=$\linebreak $=N_1(2{,}2)$ и $N_3=N_3(2{,}2,\,1{,}76)$ приведены в
табл.~6 (для $N_3(T,\eps)$ взято
<<с запасом>> значение $\eps=1{,}76$). Как видно, уже при $n\ge184$
для всех рассматриваемых значений~$\d$ можно использовать оценки
$\widetilde r_1(t,\eps)$, \linebreak\vspace*{-12pt}
\pagebreak

%\vspace*{1pt}

\noindent
\begin{center} %tabl6
\vspace*{-8pt}

\noindent
\parbox{79mm}{{\tablename~6}\ \ \small{Экстремальные значения
$\eps^*$ и оптимальные $t_0,U$ при вычислении константы
$\exlow(\d)$}}
%\end{center}
\vspace*{2ex}

{\small 
\tabcolsep=8pt
\begin{tabular}{|c|c|c|c|c|c|c|}
  \hline
$\d$ & $\eps_{\max}$ & $N_1$ & $N_3$ & $\eps^*$ & $t_0$ & $U$\\
\hline 0,9& 1,752& 38& 142& 1,061& 0,38& 2,14\\
0,8& 1,698& 37& 110& 1,108& 0,40& 2,12\\
0,7& 1,623& 36& 91& 1,135& 0,43& 2,09\\
0,6& 1,534& 36& 79& 1,143& 0,44& 2,06\\
0,5& 1,434& 37& 73& 1,136& 0,46& 2,02\\
0,4& 1,326& 40& 71& 1,114& 0,47& 1,99\\
0,3& 1,213& 49& 75& 1,079& 0,48& 1,96\\
0,2& 1,099& 72& 90& 1,032& 0,49& 1,93\\
0,1& 0,985& 184& 144& 0,976& 0,49& 1,90\\
\hline
\end{tabular}
}
\end{center}

\addtocounter{table}{1}

\vspace*{12pt}


\noindent
$\widetilde r_3(t,\eps)$ из
леммы~3. Таким  образом, супремум по~$n$ достаточно оценивать
по значениям~$n$ лишь в конечном числе точек:
$n_*,\ldots,\max\{n_*,184\}$, где
 $n_*=\max\{1,(2/\eps)^{2/\d}\}$.
При этом экстремум целевой функции не превосходит значений,
указанных  в теореме~1 и достигается при $n\to\infty$
и $\eps=\eps^*(\d)$~--- указано в пятом столбце
табл.~6. Соответствующие
оптимальные значения~$t_0$ и~$U$ приведены в шестом и седьмом
столбцах табл.~6.



\bigskip

В заключение авторы выражают свою признательность В.\,Ю.~Королеву
за поддержку и постоянное внимание к работе.

{\small\frenchspacing
{%\baselineskip=10.8pt
\addcontentsline{toc}{section}{Литература}
\begin{thebibliography}{99}

\bibitem{Berry1941} %1
\Au{Berry A.\,C.} The accuracy of the Gaussian approximation to the
sum of independent variates~// Trans. Amer. Math. Soc., 1941.
Vol.~49. P.~122--139.

\bibitem{Esseen1942} %2
\Au{Esseen C.-G.} On the Liapunoff limit of error in the theory of
probability~// Ark. Mat. Astron. Fys., 1942. Vol.~A28. No.~9.
P.~1--19.

\bibitem{Katz1963} %3
\Au{Katz M.} A note on the Berry--Esseen theorem~// Ann. Math.
Statist., 1963. Vol.~34. P.~1107--1108.

\bibitem{Petrov1965} %4
\Au{Петров В.\,В.} Одна оценка отклонения распределения суммы
независимых случайных величин от нормального закона~// ДАН СССР,
1965. Т.~160. Вып.~5. С.~1013--1015.

\bibitem{Bikelis1966} %5
\Au{Бикялис А.} Оценки остаточного члена в центральной предельной
теореме~// Литовский математический сб., 1966. Т.~6. Вып.~3.
С.~323--346.

\bibitem{Petrov1972} %6
\Au{Петров В.\,В.} Суммы независимых случайных величин.~--- М.: Наука, 1972.

\bibitem{Esseen1956} %7
\Au{Esseen~C.-G.} A moment inequality with an application to the
central limit theorem~// Skand. Aktuarietidskr., 1956. Vol.~39.
P.~160--170.

\bibitem{KorolevShevtsova2010} %8
\Au{Королев В.\,Ю., Шевцова И.\,Г.} Уточнение неравенства
Берри--Эссеена с приложениями к пуассоновским и смешанным
пуассоновским случайным суммам~// Обозрение прикладной и
промышленной математики, 2010. Т.~17. Вып.~1. С.~25--56.

\bibitem{Tysiak1983} %9
\Au{Tysiak W.} Gleichm$\ddot{\mbox{a}}${\!\!\ptb\ss}ige und
nicht-gleichm$\ddot{\mbox{a}}${\!\!\ptb\ss}ige Berry--\linebreak Esseen--Absch{\"a}tzungen.
Dissertation. --- Wuppertal, 1983.

\bibitem{Paditz1996} %10
\Au{Paditz H.} On the error-bound in the nonuniform version of
Esseen's inequality in the $L_p$-metric~// Statistics, 1996.
Vol.~27. P.~379--394.

\bibitem{GaponovaKorchaginShevtsova2009} %11
\Au{Гапонова М.\,О., Корчагин А.\,Ю., Шевцова~И.\,Г.} Об абсолютных
константах в равномерной оценке точности нормальной аппроксимации
для распределений, не имеющих третьего момента~// Сб.\ статей
молодых ученых факультета ВМК МГУ. Вып.~6.~--- М.: Макс Пресс, 2009.
С.~81--89.

\bibitem{Paditz1986} %12
\Au{Paditz H.} $\ddot{\mbox{U}}$ber eine Fehlerabsch$\ddot{\mbox{a}}$tzung im zentralen
Grenzwertsatz~// Wiss. Z. Hochschule f$\ddot{\mbox{u}}$r Verkehswesen
``Friedrich List.''~--- Dresden, 1986. Bd.~33. H.~2. S.~399--404.

\bibitem{Shevtsova2010} %13
\Au{Шевцова И.\,Г.} Об асимптотически варл правильных постоянных в
центральной предельной теореме~// Тео\-рия вероятностей и ее
применения, 2010 (в пе\-ча\-ти). Т.~55. Вып.~2.

\bibitem{KorolevShevtsova2009} %14
\Au{Королев В.\,Ю., Шевцова И.\,Г.} О верхней оценке абсолютной
постоянной в неравенстве Берри--Эссеена~// Теория вероятностей и ее
применения, 2009. Т.~54. Вып.~4. С.~671--695.

\bibitem{Shevtsova2010a} %15
\Au{Шевцова И.\,Г.} Нижняя асимптотически правильная постоянная в
центральной предельной теореме~// Докл. РАН, 2010.
Т.~430. Вып.~4. С.~466--469.

\bibitem{KorolevShevtsova2010a} %16
\Au{Королев В.\,Ю., Шевцова И.\,Г.} Уточнение неравенства
Берри--Эссеена~// Докл. РАН, 2010. Т.~430. Вып.~6.
С.~738--742.

\bibitem{Zolotarev1966} %17
\Au{Золотарёв В.\,М.} Абсолютная оценка остаточного члена в
центральной предельной теореме~// Теория вероятностей и ее
применения, 1966. Т.~11. Вып.~1. С.~108--119.

\bibitem{Zolotarev1967a} %18
\Au{Золотарёв В.\,М.} Некоторые неравенства теории вероятностей и их
применение к уточнению теоремы А.\,М.~Ляпунова~// ДАН СССР, 1967.
Т.~177. №\,3. С.~501--504.

\bibitem{Zolotarev1967b} %19
\Au{Zolotarev V.\,M.} A sharpening of the inequality of
Berry--Esseen~// Z. Wahrsch. verw. Geb., 1967. Bd.~8. P.~332--342.

\bibitem{Prawitz1972} %20
\Au{Prawitz H.} Limits for a distribution, if the characteristic
function is given in a finite domain~// Scand. Aktuar Tidskr., 1972.
P.~138--154.

\bibitem{Shevtsova2009} %21
\Au{Шевцова И.\,Г.} Некоторые оценки для характеристических функций
с применением к уточнению неравенства Мизеса~// Информатика и её
применения, 2009. Т.~3. Вып.~3. С.~69--78.


\bibitem{Prawitz1975} %22
\Au{Prawitz H.} On the remainder in the central limit theorem.~I.
Onedimensional independent variables with finite absolute moments of
third order~// Scand. Actuarial J., 1975. No.~3. P.~145--156.

\bibitem{GaponovaShevtsova2009} %23
\Au{Гапонова М.\,О., Шевцова  И.\,Г.} Асимптотические оценки
абсолютной постоянной в неравенстве Берри--Эссеена для
распределений, не имеющих третьего момента~// Информатика и её
применения, 2009. Т.~3. Вып.~4. С.~41--56.

\label{end\stat}

\bibitem{BhatRangaRao1982} %24
\Au{Бхаттачария Р.\,Н., Ранга~Р.\,Р.} 
Аппроксимация нормальным распределением.~--- М.: Наука, 1982.
 \end{thebibliography}
}
}

\end{multicols}  %9

\def\stat{kozerenko}

\def\tit{СЕМАНТИЧЕСКАЯ ОБРАБОТКА НЕСТРУКТУРИРОВАННЫХ ТЕКСТОВЫХ 
ДАННЫХ НА~ОСНОВЕ ЛИНГВИСТИЧЕСКОГО ПРОЦЕССОРА PullEnti}

\def\titkol{Семантическая обработка неструктурированных текстовых 
данных на~основе лингвистического процессора PullEnti}

\def\aut{Е.\,Б.~Козеренко$^1$, К.\,И.~Кузнецов$^2$, Д.\,А.~Романов$^3$}

\def\autkol{Е.\,Б.~Козеренко, К.\,И.~Кузнецов, Д.\,А.~Романов}

\titel{\tit}{\aut}{\autkol}{\titkol}

\index{Козеренко Е.\,Б.}
\index{Кузнецов К.\,И.}
\index{Романов Д.\,А.}
\index{Kozerenko E.\,B.}
\index{Kuznetsov K.\,I.}
\index{Romanov D.\,A.}




%{\renewcommand{\thefootnote}{\fnsymbol{footnote}} \footnotetext[1]
%{Работа выполнена при частичной поддержке РФФИ (проект 16-07-00677).}}


\renewcommand{\thefootnote}{\arabic{footnote}}
\footnotetext[1]{Институт проблем информатики Федерального исследовательского центра <<Информатика и~управ\-ле\-
ние>> Российской академии наук, \mbox{kozerenko@mail.ru}}
\footnotetext[2]{Институт проблем информатики Федерального исследовательского центра <<Информатика 
и~управление>> Российской академии наук, \mbox{k.smith@mail.ru}}
\footnotetext[3]{Национальный исследовательский университет <<Высшая школа экономики>>, DRomanov@it.ru}

%\vspace*{8pt}
  

  
   
     \Abst{Представлена методика создания систем извлечения 
знаний, основанная на подходе, главным инструментом которого является 
программный пакет PullEnti, включающий алгоритмы морфологического 
и~се\-ман\-ти\-ко-син\-так\-си\-че\-ско\-го анализа для выделения сущностей 
определенных типов из текстов естественного языка 
(персоны, организации, локации и~другие целевые семантические объекты). 
В~сис\-те\-ме PullEnti используются динамически подключаемые компоненты 
(плагины), что позволяет без перекомпилирования активировать различные 
функциональные воз\-мож\-ности. Именно таким образом запускается блок 
семантического анализа. В~процессе анализа выделяются семантические 
единицы (токены), которые представляют собой типизированные фразы: 
текстовые, чис\-ло\-вые и~др. Приводятся примеры реализованных проектов 
для различных предметных областей.}
     
     \KW{семантическое моделирование; извлечение именованных 
сущностей; области с~интенсивным использованием данных; 
автоматизированные сис\-те\-мы извлечения знаний; семантический поиск; 
интеллектуальные ин\-тер\-нет-тех\-но\-логии}

\DOI{10.14357/19922264180313}
  
\vspace*{-1pt}


\vskip 10pt plus 9pt minus 6pt

\thispagestyle{headings}

\begin{multicols}{2}

\label{st\stat}
     
     \section{Введение}
     
     \vspace*{-4pt}
     
     Задача автоматического анализа текстовой информации, 
представленной в~Интернете, является актуальной во всем мире. В~данной 
статье пред\-ставле\-ны результаты исследований и~разработок, на\-прав\-лен\-ных 
на решение научной проб\-ле\-мы со\-зда\-ния оптимальной методики  
ло\-ги\-ко-ста\-ти\-сти\-че\-ско\-го моделирования механизмов целевого 
семантического анализа в~информационных сис\-те\-мах с~интенсивным 
использованием знаний, выполняющих функции извлечения знаний, 
поддержки аналитических решений, в~том чис\-ле в~среде нескольких 
естественных языков. 

Для решения полного спектра задач обработки 
естественного языка создан се\-ман\-ти\-че\-ски-ори\-ен\-ти\-ро\-ван\-ный 
лингвистический процессор (СОЛП). Центральным компонентом СОЛП 
является инструментальный пакет (SDK-мо\-дуль) PullEnti. Этот процессор 
в~рамках проводимых соревнований конференции <<Диа\-лог-2016>> занял 
несколько первых мест при анализе текс\-тов в~рамках решения задач 
извлечения именованных сущностей. Разработчик PullEnti~--- Кузнецов 
Константин Игоревич. В~сис\-те\-ме PullEnti используются динамически 
подключаемые компоненты (плагины), что позволяет без 
перекомпилирования запускать различные функциональные возможности. 
Именно таким образом активируется блок семантического анализа. 
     
     В процессе анализа выделяются семантические единицы (токены), 
которые представляют собой типизированные фразы, такие как текс\-то\-вые, 
чис\-ло\-вые и~др. Например, в~результате анализа фразы <<В~2017~году>> 
будут выделены три токена: <<В>>~--- текс\-то\-вый; <<году>>~--- текстовый; 
<<2017>>~--- чис\-ло\-вой. Такие токены можно назвать прос\-ты\-ми. Кроме того, 
выделяются \textit{метатокены}~--- слож\-ные токены, которые объединяют 
несколько прос\-тых токенов, например существительные с~определителями, 
скобки, кавычки и~т.\,п.
     
     В системе существует пополняемый статический словарь терминов. 
В~него можно добавлять термины и~затем проверять их наличие в~тексте. 
Кроме того, в~сис\-те\-ме можно формировать динамически подобные словари 
на основе анализа текста.
     
     При анализе текста создается аналитический контейнер, в~который 
помещаются вы\-де\-ля\-емые сущности, токены в~определенной 
последовательности, статистические данные и~др.
     
    \section{Лингвистическое моделирование в~системах 
обработки знаний в~многоязычной среде}
     
     Способы представления информации, знаний многообразны. Огромный 
объем данных пред\-став\-лен в~виде текс\-тов естественного языка, что делает 
задачу извлечения и~структурирования информации из текстов весьма 
важной. Это относится к~различным предметным областям. Для 
оперирования данными на компьютере необходимо выделить из текста 
объекты, их атрибуты, связи между объектами, процессы, в~которых эти 
объекты задействованы, другую важ\-ную информацию, которая бы позволяла 
не только описать ситуацию, но и~строить выводы, характерные для 
конкретной предметной об\-ласти, прогнозировать развитие ситуации.
     
     Для решения поставленных задач проведены эксперименты 
с~различными грамматическими формализмами, в~том чис\-ле с~грамматикой 
категориального типа~[1]. Проведено сравнительное\linebreak
 исследование методов 
классификации применительно к~лингвистическим задачам; выработан\linebreak 
эффективный метод отображения вектора ес\-те\-ст\-вен\-но-язы\-ко\-вых 
структур в~расширенное пространство признаков для классификации новых 
языковых объектов и~структур; сформирована фокусная выборка 
параллельных текстов деловых и~научных документов на русском 
и~английском языках по различным отраслям науки и~техники;\linebreak 
сформирована расширенная система новых категорий для повышения 
изобразительных возможностей двуязычной грамматики; выработаны пути\linebreak 
расширения базовых представлений на основе аппарата расширенных 
семантических сетей~[2]\linebreak и~результатов применения метода векторных 
пространств, направленного на разрешение не\-од\-но\-знач\-ности языковых 
структур для синтаксического разбора при распознавании текста в~процессе 
извлечения знаний из текстов на разных естественных языках. Разработаны 
алгоритмы автоматического выравнивания параллельных текстов для 
развития грамматических компонент сис\-тем обработки знаний 
в~многоязычном режиме. 

%o
Основной результат исследований~--- модель 
лингвистической со\-став\-ля\-ющей интеллектуальных информационных сис\-тем, 
работающих в~многоязычном пространстве для поиска информации, 
обеспечения оптимальных аналитических и~управ\-лен\-че\-ских решений 
в~сферах деятельности с~интенсивным использованием данных. Результаты 
исследований применяются в~ло\-ги\-ко-се\-ман\-ти\-че\-ских 
и~статистических процедурах обработки слабоструктурированной текс\-то\-вой 
информации, при разработке технологии и~инструментальных средств 
построения лингвистических компонент интеллектуальных сис\-тем и~сис\-тем 
машинного перевода.
     
    \section{Представление лингвистических знаний на~основе 
векторных пространств }
    
     Процедуры анализа и~синтеза ес\-те\-ст\-вен\-но-язы\-ко\-вых высказываний 
отражают динамический характер языка как деятельности; соответственно, 
в~модели, которая кладется в~основу проекта сис\-те\-мы обработки  
ес\-те\-ст\-вен\-но-язы\-ко\-вых высказываний, дол\-жен быть заложен 
механизм, позволяющий строить пред\-став\-ле\-ния движения. 
     
     Методы машинного обучения на основе векторных моделей 
развиваются и~используются в~различных областях знаний, применительно 
к~лингвистическим задачам эти методы вполне эффективны для разрешения 
лексической мно\-го\-знач\-ности~[3--8]. 
     
     Более сложной задачей и~новым направлением исследований 
возможности применения векторных моделей для пред\-став\-ле\-ния и~обработки 
лингвистических данных является моделирование грамматических 
преобразований на основе векторных пространств и~тензоров. Тензор (от 
лат.\ \textit{tensus}, напряженный)~--- объект линейной алгебры, пре\-об\-ра\-зу\-ющий 
элементы одного линейного пространства в~элементы другого. Часто тензор 
представляют как многомерную таб\-ли\-цу, заполненную чис\-ла\-ми~--- 
компонентами тензора $d \cdot d \cdots d$, где $d$~--- раз\-мер\-ность, 
над которой задан тензор, а~чис\-ло сомножителей совпадает с~так называемой 
валентностью, или рангом тензора. Важно, что такое представление (кроме 
скаляров, т.\,е.\ тензоров валентности ноль) возможно только после выбора 
базиса (или системы координат): при смене базиса компоненты тензора 
меняются определенным образом. Сам тензор как <<геометрическая 
сущность>> от выбора базиса не зависит, компоненты вектора меняются при 
смене координатных осей, но сам вектор, образом которого может быть 
прос\-то нарисованная стрелка, от этого не изменяется. Тензор обычно 
обозначают некоторой буквой с~совокупностью верх\-них (контрвариантных) и~ниж\-них (ковариантных) индексов: $X_{j_1 j_2\ldots j_s}^{i_1i_2\ldots i_r}$. 
При смене базиса ковариантные компоненты меняются так же, как и~базис 
(с~по\-мощью того же преобразования), а~контрвариантные~--- обратно 
изменению базиса (обратным преобразованием). Тензор является сущностью 
любой системы реального мира и~сохраняется, несмотря на происходящие 
изменения в~этой системе~\cite{9-koz}. Эта особенность тензора чрезвычайно 
актуальна для моделирования языковых преобразований в~лингвистических 
процессорах, когда необходимо выявлять сходные значения, выраженные 
многочисленными способами, сис\-те\-мой разнородных языковых средств. 
     
     В работе, представленной в~данной статье, используются два основных 
подхода к~пред\-став\-ле\-нию смысла в~вы\-чис\-ли\-тель\-ной лингвистике: 
символьный подход~\cite{10-koz, 11-koz} и~подход на основе 
дистрибутивной семантики~\cite{5-koz, 7-koz, 8-koz, 9-koz}; решение 
заключается\linebreak в~сочетании методов компьютерной лингвистики и~когнитивной 
науки, в~котором символьное и~<<\textit{коннекционистское}>> (от англ.\ 
\textit{connectionist}, т.\,е.\ основанное на нейронных сетях как модели 
машинного обучения) представления объединяются с~по\-мощью тензорных 
произведений. Исследованы возможные применения данного метода для 
обработки синтаксических структур и~контекстов в~русском и~английском 
языках и~проведены межъязыковые сопоставления.
\vspace*{-3pt}
     
\section{Семантико-ориентированный лингвистический процессор}
     
     Методы, описанные выше, используются в~процедурах семантической 
обработки текстовых знаний в СОЛП, который решает задачу извлечения 
структурированной информации из текс\-тов на русском и~английском языках. 
Ядром СОЛП является программный пакет PullEnti, вклю\-ча\-ющий алгоритмы 
морфологического и~синтаксического анализа, который позволяет выделять 
сущности определенных типов из текс\-тов естественного языка (персоны, 
организации, локации и~другие семантические объекты). <<Именованная 
сущность>>~--- это объект, содержащий набор значений атрибутов, 
отличающий его от других объектов этого же типа. В~тексте находятся 
именованные сущности и~устанавливаются семантические связи между 
ними, при этом учитывается воз\-мож\-ность обозначения одной сущ\-ности 
несколькими способами (синонимия). Все множество сущностей, 
выделенных из текста или нескольких текс\-тов, представляет собой 
ориентированный граф.
     
     Предварительный этап обработки текстов включает в~себя 
морфологический и~синтаксический анализ. При морфологическом анализе 
текст разбивается на словоформы, так\-же называемые токенами (от англ.\ 
\textit{token}~--- пример использования лингвистической единицы в~тексте). 
Основными наследными классами базового класса Token являются TextToken и~MetaToken. 
TextToken~--- это исходный фрагмент текста, содержащий результат 
морфологического анализа. TextToken ссылается\linebreak на MorphToken, 
содержащий все морфологические варианты разбора. MetaToken~--- это 
группа токенов, соответствующих одной синтаксической или семантической 
группе. К~классу метатокенов относятся NumberToken, пред\-став\-ля\-ющий 
чис\-ло, и~\mbox{ReferentToken}, представляющий сущ\-ность.
     
     Приведем пример морфологического анализа предложения.
     
     Исходный текст: 
     
     <<По словам директора департамента экономического развития 
автономного округа Павла Сидорова, на эти цели планируется при\-влечь 
200~млн рублей из федерального бюджета и~еще 450~млн 
рублей из внебюджетных источников>>.
     
     В результате морфологического разбора текст был разбит на 
словоформы, для каждой словоформы указана начальная форма, часть речи 
и~морфологические характеристики. В~случае мно\-го\-знач\-ности или 
омонимии указываются все варианты морфологического разбора. 
     
     Также были выделены сле\-ду\-ющие метатокены:
     
      Павел Сидоров~--- текстовый фрагмент <<директора департамента 
экономического развития автономного округа Павла Сидорова>>
      
      200.000.000 RUB~--- текстовый фрагмент <<200~миллионов рублей>>
      
      450.000.000 RUB~--- текстовый фрагмент <<450~миллионов рублей>>
     
     Вслед за морфологическим анализом проводится выделение 
именованных сущностей различных типов. Се\-ман\-ти\-ко-ори\-ен\-ти\-ро\-ван\-ный 
лингвистический процессор извлекает из текстов 
объекты следующих типов: дата, временной период, территориальное 
образование, денежная сумма, телефон, URL, адрес, организация, транспорт, 
свойство персоны, персона, декрет, часть декрета. Каждому типу 
соответствуют свои свойства и~связи с~объектами других типов. 

\begin{figure*}[b] %fig1
  \vspace*{6pt}
 \begin{center}
 \mbox{%
 \epsfxsize=163mm 
 \epsfbox{koz-1.eps}
 }
 \end{center}
\vspace*{-9pt}
\Caption{Результаты работы программы <<Доктор Ватсон>>. Выделение сущностей}
\end{figure*}
     
     Базовым классом для сущностей является класс Referent. Тип 
сущностей задается классом Referent Class, на\-след\-ным от Referent, который 
содержит набор атрибутов. Значение может быть как прос\-тым (строка, 
число), так и~ссылкой на другую сущ\-ность. Помимо значений атрибутов 
сущность содержит список ссылок на участки исходного текс\-та, в~которых 
эта сущность располагается. Для задач, в~которых требуется обрабатывать 
множество текстов и~хранить по\-лу\-ча\-емые сущности, в~СОЛП используется 
базовый класс Repository Base, об\-лег\-ча\-ющий реализацию хранилища 
сущностей. Мес\-то хранения сериализованных данных от сущностей 
определяется в~наследном классе (например, это может быть реляционная 
система управления базами данных или файловая сис\-те\-ма). 
Repository Base берет на себя функции 
отож\-де\-ст\-вле\-ния новых данных со старыми данными и~поддержки 
не\-про\-ти\-во\-ре\-чи\-вости семантической сети.
     
Извлекаемая из текс\-та информация должна быть адресной, поэтому из 
одного и~того же текста можно извлекать совершенно различные виды 
информации, характерные для конкретной предметной об\-ласти. 
В~результате анализа текстовой информации выделяются типизированные 
объекты предметной об\-ласти. Программа PullEnti стала основой для 
построения множества сис\-тем, таких как программа <<Доктор Ватсон>>, 
система поиска экспертов, процессор BRef и~др. Программа <<Доктор 
Ватсон>> предназначена для исследования массивов текс\-то\-вой информации с~целью выявления сущностей и~связей между ними. При этом пользователь 
может добавить недостающие сущности и~связи (которые не были выделены 
программой), настроить выдаваемую информацию, сформировать отчет 
о~результатах работы программы. Данная программа может использоваться в~таких сферах деятельности, как криминалистика, конкурентная разведка, 
маркетинг, реклама, безопас\-ность. Результат работы программы~--- отчет об 
исследуемом объекте, диаграммы сущностей и~связей~--- пред\-став\-лен на 
рис.~1. Из текущего текста выделены организации, персоны, их связи.
   
   На рис.~2 представлены выделенные объекты, их связи. Для каждой связи 
выделяется тип связи и~название (например, тип связи~--- <<родственные>>, 
заголовок связи~--- <<отец>>; тип связи~--- <<владение>>, заголовок  
связи~--- <<особняк в~центре Вашингтона>> и~т.\,д.), определяются попарно 
объекты-участники связи. Для более полного определения ситуации 
выделяется не только время, характеризующее текущую ситуацию, но 
и~интервалы времени. Дополнительные параметры поз\-во\-ля\-ют выяснить, 
является ли связь симметричной для данной пары выделенных объектов 
(субъектов). Также для каж\-до\-го выделенного объекта (субъекта) выделяются 
атрибуты. Например, для типа объекта <<персона>> выделяются имя, 
фамилия, отчество, дата рождения и~др. 


   Результаты работы программы могут быть пред\-став\-ле\-ны в~виде графа 
(вкладка <<Диаграммы>>) (см.\ рис.~3). В~отчете выводятся обнаруженные 
объекты (персоны, организации, локации, атрибуты), их связи в~удоб\-ном для 
анализа виде.


   
   Программа <<Логика ECM. Правовая экспертиза>> предназначена для 
автоматизации процесса\linebreak проведения экспертизы проектов  
нор\-ма\-тив\-но-пра\-во\-вых актов,  
ор\-га\-ни\-за\-ци\-он\-но-рас\-по\-ря\-ди\-тель\-ных документов, договоров 
и~других документов. Сис\-те\-ма значительно упрощает процесс проведения 
правовой экспертизы и~сокращает его сроки, выполняя рутинные операции 
и~кардинально снижая за\-тра\-ты рабочего времени квалифицированных 
юристов. Сис\-те\-ма <<Логика ECM. Правовая экспер-\linebreak\vspace*{-12pt}

\pagebreak

\end{multicols}

\begin{figure*} %fig2
\vspace*{1pt}
 \begin{center}
 \mbox{%
 \epsfxsize=163mm 
 \epsfbox{koz-2.eps}
 }
 \end{center}
\vspace*{-11pt}
\Caption{Выделение связей, периодов}
%\end{figure*}
%\begin{figure*} %fig3
\vspace*{6pt}
 \begin{center}
 \mbox{%
 \epsfxsize=163mm 
 \epsfbox{koz-3.eps}
 }
 \end{center}
\vspace*{-11pt}
\Caption{Графическое представление результатов работы программы <<Доктор 
Ватсон>>}
\vspace*{-2pt}
\end{figure*}
   

\begin{multicols}{2}

\noindent
тиза>> автоматически за 
несколько секунд поможет, например, установить:
   \begin{itemize}
   \item не содержатся ли в~проверяемом документе ссылки на нормативные 
правовые акты, которые утратили силу;
   \item нет ли в~проверяемом документе фрагментов других документов, не 
возникает ли избыточное дублирование нормативной документации;
   \item соответствует ли оформление и~структура документа уста\-нов\-лен\-ным 
   в~организации правилам;\\[-13pt]
   \item нет ли ошибок в~оформлении цифровой информации в~договоре, 
соответствуют ли друг другу суммы, указанные цифрами и~про\-писью, 
правильно ли рассчитан налог на добавленную стоимость и~т.\,п. На основе лингвистического процессора 
PullEnti был реализован процессор обработки ссылок и~списка литературы 
BREF,\linebreak
\vspace*{-12pt}

\pagebreak

\noindent
который позволяет по выделенной информации по\-строить \textit{Граф 
Цитирования} и~\textit{Граф Соавторов}, отражающие формальные связи 
в~коллекции документов.
   \end{itemize}
    
   Лингвистический процессор PullEnti под псевдонимом Pink на 
соревновании FactRuEval конференции <<Диа\-лог-2016>> занял первые места 
на большинстве дорожек~\cite{12-koz}.
   
   Соревнование проводилось на следующих дорожках:
   \begin{itemize}
   \item определение в~тексте границ именованных сущностей, таких как 
персона, организация, локация;
   \item выделение именованных сущностей с~определением атрибутов 
в~нормализованном виде. Для персон это фамилия, имя и~отчество. Для 
организаций и~локаций~--- нормализованное название;
   \item извлечение фактов (например: <<встреча>>, <<покупка>>, $\ldots$) 
и~наборов строковых полей (например: <<участник встречи~1>>, 
<<участник встречи~2>>, <<место встречи>>, <<да\-та/вре\-мя начала 
встречи>>, $\ldots$).
   \end{itemize}
   
   \vspace*{-13pt}
   
  \section{Заключение}
  \vspace*{-3pt}
  
   Лингвистические процессоры на основе программы PullEnti могут быть 
использованы в~различных областях, в~которых информация пред\-став\-ле\-на 
в~текс\-то\-вом виде. Особенно это важно в~тех случаях, когда необходимо 
выделять важ\-ную информацию из большого потока документов на 
естественном языке. Очень хорошо данная технология работает в~задачах 
кластеризации текстов по определенным признакам. При этом существует 
воз\-мож\-ность автоматической настройки программы на требования 
пользователя.
   
   Описанные выше системы, созданные на основе технологии PullEnti, 
доказывают ее эффективность в~самых различных областях. Сле\-ду\-ющи\-ми 
шагами исследований станут: методы уточнения границ мо\-де\-ли\-ру\-емых 
предметных областей за счет построения семантического ядра каж\-дой 
области (в~том чис\-ле с~использованием методов вероятностного 
тематического моделирования); выделение массива неявных ссылок 
(упоминаний персон и~идей, выраженных ключевыми фра\-за\-ми/зна\-чи\-мы\-ми 
словосочетаниями); расчет корреляции между явными и~неявными ссылками 
в~рамках созданной коллекции, формирование  
функ\-цио\-на\-ль\-но-грам\-ма\-ти\-че\-ских модулей естественных языков, 
включаемых в~лингвистический процессор.

    
{\small\frenchspacing
 {%\baselineskip=10.8pt
 \addcontentsline{toc}{section}{References}
 \begin{thebibliography}{99}

 \vspace*{-4pt}

\bibitem{1-koz}
\Au{Shaumyan S.} Categorial grammar and semiotic universal grammar~// 
Conference (International) on Artificial Intelligence Proceedings.~--- 
Las Vegas, NV, USA: CSREA Press, 2003. 
P.~623--629.

\bibitem{2-koz}
\Au{Kuznetsov I.\,P., Kozerenko~E.\,B., Matskevich~A.\,G.} 
Intelligent extraction of knowledge structures from natural language texts~// 
IEEE/WIC/ACM Joint Conferences (International) on Web Intelligence and 
Intelligent Agent Technology Proceedings.~---
Washington, DC, USA: IEEE Computer Society, 2011. Vol.~3. P.~269--272.

\bibitem{3-koz}
\Au{Dempster A.\,P., Laird N.\,M., Rubin~D.\,B.} 
Maximum likelihood from incomplete data via the EM algorithm~// 
J.~Roy. Stat. Soc.~B, 1977. Vol.~39. Iss.~1. P.~1--22.

\bibitem{7-koz} %4
\Au{Lund K., Burgess~C.} Producing high-dimensional semantic spaces from 
lexical co-occurrence~// 
Behav. Res. Meth. Ins. C., 1996. Vol.~28. Iss.~2. 
P.~203--208.

\bibitem{5-koz} %5
\Au{Curran J.\,R.} From distributional to semantic similarity.~--- 
Edinburgh: University of 
Edinburgh, 2004. PhD Thesis. 177~p.
{\sf https://www.inf.ed.ac.uk/publications/thesis/ online/IP030023.pdf}

\bibitem{8-koz} %6
\Au{McCarthy D., Koeling R., Weeds~J., Carroll~J.} Finding predominant 
senses in untagged text~// 42nd Annual Meeting of the Association for 
Computational Linguistics Proceedings.~---
Stroudsburg, PA, USA: Association for 
Computational Linguistics, 2004. P.~280--287. doi: 10.3115/1218955.1218991.

\bibitem{4-koz} %7
\Au{Clark S., Pulman S.} Combining symbolic and distributional models of 
meaning~// AAAI Spring Symposium on Quantum Interaction Proceedings.~--- 
Palo Alto, CA, USA: AAAI Press, 2007. 4~p. {\sf 
http://www.cl.cam.ac.uk/ $\sim$sc609/pubs/aaai07.pdf.}

\bibitem{6-koz} %8
\Au{Kozerenko E.\,B.} Parallel texts alignment strategies~// 
\textit{Conference (International) on Artificial Intelligence Proceedings}.~--- 
Las Vegas, NV, USA: CSREA Press, 
2012. Vol.~2. P.~945--951.

\bibitem{9-koz}
\Au{Danielson D.\,A.} Vectors and tensors in engineering and physics.~--- 
2nd ed.~--- Boulder, CO, USA: Westview (Perseus), 2003. 287~p.

\bibitem{11-koz} %10
\Au{Montague R.} Universal grammar~// Theoria, 1970. Vol.~36. P.~373--398. 
(Reprinted in: Formal philosophy: Selected 
papers of Richard Montague~/ 
Ed. R.\,H.~Thomason.~--- 
New Haven, CT, USA: Yale University Press, 1974. P.~7--27.)

\bibitem{10-koz}
\Au{Pang B., Knight K., Marcu~D.} Syntax-based alignment of multiple translations: 
Extracting paraphrases and generating new sentences~// 
Conference of the North 
American Chapter of the Association for Computational Linguistics on Human Language 
Technology Proceedings.~--- 
Stroudsburg, PA, USA: Association for Computational Linguistics.
2003. Vol.~1. P.~102--109. doi: 10.3115/1073445.1073469.

\bibitem{12-koz}
FACRUEVAL. Evaluation of named entity recognition and fact extraction systems for 
Russian, 2016. {\sf http:// ww.dialog-21.ru/media/3430/starostinaetal.pdf.}
 \end{thebibliography}

 }
 }

\end{multicols}

%\vspace*{-12pt}

\hfill{\small\textit{Поступила в~редакцию 13.07.18}}

%\vspace*{-36pt}

\pagebreak

\vspace*{-36pt}

%\hrule

%\vspace*{2pt}

%\hrule

\vspace*{-2pt}


\def\tit{SEMANTIC PROCESSING OF~UNSTRUCTURED TEXTUAL DATA BASED 
ON~THE~LINGUISTIC PROCESSOR PullEnti}


\def\titkol{Semantic processing of~unstructured textual data based 
on~the~linguistic processor PullEnti}


\def\aut{E.\,B.~Kozerenko$^1$, K.\,I.~Kuznetsov$^1$, and~D.\,A.~Romanov$^2$}

\def\autkol{E.\,B.~Kozerenko, K.\,I.~Kuznetsov, and~D.\,A.~Romanov}

\titel{\tit}{\aut}{\autkol}{\titkol}

\vspace*{-11pt}


\noindent
$^1$Institute of Informatics Problems, Federal Research Center ``Computer Science 
and Control'' of the Russian 
$\hphantom{^1}$Academy of Sciences,  44-2~Vavilov Str., Moscow 119333, 
Russian Federation

\noindent
$^2$National Research University ``Higher School of Economics,'' 
20~Myasnitskaya Str., Moscow 101000, Russian 
$\hphantom{^1}$Federation


\def\leftfootline{\small{\textbf{\thepage}
\hfill INFORMATIKA I EE PRIMENENIYA~--- INFORMATICS AND
APPLICATIONS\ \ \ 2018\ \ \ volume~12\ \ \ issue\ 3}
}%
 \def\rightfootline{\small{INFORMATIKA I EE PRIMENENIYA~---
INFORMATICS AND APPLICATIONS\ \ \ 2018\ \ \ volume~12\ \ \ issue\ 3
\hfill \textbf{\thepage}}}

\vspace*{3pt}



\Abste{The paper presents the method for creation of knowledge extraction 
systems based on the approach employing the software tool system 
PullEnti comprising the algorithms for morphological and semantic-syntactical 
analysis which makes it possible to extract entities of certain types 
from natural language texts (persons, organizations, locations, and other 
target semantic objects). The PullEnti system uses dynamically connected 
components (plugins) which makes it possible to activate various functions 
without recompiling. This is how the semantic analysis unit is incorporated. 
During the analysis, the semantic units (tokens) are established, which are 
typed phrases: text, numerical data, etc. 
Examples of implemented projects for different subject areas are given.}

\KWE{semantic modeling; named entities recognition, data intensive domains; 
automated systems of knowledge extraction; semantic search; intelligent Internet 
technologies}




\DOI{10.14357/19922264180313} %

%\vspace*{-14pt}

%\Ack
%\noindent



%\vspace*{6pt}

  \begin{multicols}{2}

\renewcommand{\bibname}{\protect\rmfamily References}
%\renewcommand{\bibname}{\large\protect\rm References}

{\small\frenchspacing
 {%\baselineskip=10.8pt
 \addcontentsline{toc}{section}{References}
 \begin{thebibliography}{99}
     
\bibitem{1-koz-1}
 \Aue{Shaumyan, S.} 2003. Categorial grammar and semiotic universal grammar.  
\textit{Conference (International) on Artificial Intelligence Proceedings}. 
Las Vegas, NV: CSREA Press. 623--629.

\bibitem{2-koz-1}
\Aue{Kuznetsov, I.\,P., E.\,B.~Kozerenko, and A.\,G.~Matskevich.} 
2011. Intelligent extraction of knowledge structures from natural language texts. 
\textit{IEEE/WIC/ACM Conferences (International) on Web Intelligence and 
Intelligent Agent Technology Proceeding}. 
Washington, DC: IEEE Computer Society. 3:269--272. doi: 10.1109/WI-IAT.2011.235.

\bibitem{3-koz-1}
\Aue{Dempster, A.\,P., N.\,M.~Laird, and D.\,B.~Rubin.} 1977. Maximum likelihood 
from incomplete data via the EM algorithm. 
\textit{J.~Roy. Stat. Soc.~B} 39(1):1--22.

 \bibitem{7-koz-1} %4
\Aue{Lund, K., and C.~Burgess.} 1996. Producing high-dimensional semantic spaces 
from lexical co-occurrence. \textit{Behav. Res. Meth. Ins. C.}  
28(2):203--208.

\bibitem{5-koz-1} %5
\Aue{Curran, J.\,R.} 2004. From distributional to semantic similarity. Edinburgh: 
University of Edinburgh. PhD Thesis. 177~p.  Available at: 
{\sf https://www.inf.ed.ac.uk/\linebreak publications/thesis/online/IP030023.pdf} 
(accessed July~19, 2018).

 \bibitem{8-koz-1} %6
\Aue{McCarthy, D., R.~Koeling, J.~Weeds, and J.~Carroll.} 2004. 
Finding predominant senses in untagged text. 
\textit{42nd Annual Meeting of Association for Computational Linguistics Proceedings}. 
Stroudsburg, PA: Association for 
Computational Linguistics. 280--287. doi: 10.3115/1218955.1218991.

\bibitem{4-koz-1}%7
\Aue{Clark, S., and S.~Pulman.} 2007. Combining symbolic and distributional 
models of meaning. \textit{AAAI Spring Symposium on Quantum Interaction Proceedings}. 
Palo Alto, CA: AAAI Press. 4~p. Available at: 
{\sf http://www.cl.cam.ac.uk/ $\sim$sc609/pubs/aaai07.pdf} (accessed July~19, 2018).

 \bibitem{6-koz-1} %8
 \Aue{Kozerenko, E.\,B.} 2012. Parallel texts alignment strategies. 
\textit{Conference (International) on Artificial Intelligence Proceedings}. 
Las Vegas, NV: CSREA Press. 2:945--951.

\bibitem{9-koz-1}
\Aue{Danielson, D.\,A.} 2003. \textit{Vectors and tensors in engineering and 
physics.} 2nd ed. Boulder, CO: Westview Press. 287~p.

\bibitem{11-koz-} %10
\Aue{Montague, R.} 1970. Universal grammar. \textit{Theoria} 36:373--398. 
(Reprinted in: 1974.
\textit{Formal philosophy: Selected papers of Richard Montague}. 
Ed. R.\,H.~Thomason. New Haven, CT: Yale University Press. 7--27.)

\bibitem{10-koz-1} %11
\Aue{Pang, B., K.~Knight, and D.~Marcu.} 2003. Syntax-based alignment of 
multiple translations: Extracting paraphrases and generating new sentences. 
\textit{Conference of the North American Chapter of the Association for 
Computational Linguistics on Human Language Technology Proceedings}. 
Stroudsburg, PA: Association 
for Computational Linguistics. 1:102--109. doi: 10.3115/1073445.1073469.

\bibitem{12-koz-1}
FACRUEVAL. 2016. Evaluation of named entity recognition and fact extraction 
systems for Russian. Available at: {\sf http://www.dialog-21.ru/media/3430/\linebreak starostinaetal.pdf} 
(accessed July~19, 2018).

\end{thebibliography}

 }
 }

\end{multicols}

\vspace*{-6pt}

\hfill{\small\textit{Received July 13, 2018}}

\pagebreak

%\vspace*{-18pt}
     
     \Contr
     
     \noindent
     \textbf{Kozerenko Elena B.} (b.\ 1959)~--- Candidate of Science (PhD) in linguistics, leading scientist, 
Institute of Informatics Problems, Federal Research Center ``Computer Science and Control'' of the Russian 
Academy of Sciences,  44-2 Vavilov Str., Moscow 119333, Russian Federation; \mbox{kozerenko@mail.ru} 
      
       \vspace*{6pt}
      
     \noindent
       \textbf{Kuznetsov Konstantin I.} (b.\ 1968)~--- leading engineer, Institute 
of Informatics Problems, Federal Research Center ``Computer Science and 
Control'' of the Russian Academy of Sciences,  44-2~Vavilov Str., Moscow 
119333, Russian Federation; \mbox{k.smith@mail.ru} 
       
       \vspace*{6pt}
       
     \noindent
       \textbf{Romanov Dmitri A.} (b.\ 1967)~--- Candidate of Science (PhD) in 
technology, associate professor, National Research University ``Higher School of 
Economics,'' 20~Myasnitskaya Str., Moscow 101000, Russian Federation; 
\mbox{DRomanov@it.ru} 

\label{end\stat}

\renewcommand{\bibname}{\protect\rm Литература}       
        %10



%{ %\Large  
{ %\baselineskip=16.6pt

\vspace*{-48pt}
\begin{center}\LARGE
\textit{Уважаемый читатель!}
\end{center}

%\vspace*{2.5mm}

\vspace*{4mm}

\thispagestyle{empty}

{\small

 
В~2017~г.\ исполняется 10~лет со времени выхода в~свет первого 
номера журнала <<Информатика и~её применения>>~--- 
научного журнала Российской академии наук, издающегося под 
на\-уч\-но-ме\-то\-ди\-че\-ским руководством Отделения нанотехнологий 
и~информационных технологий Российской академии наук. Учредителем журнала 
является Федеральный исследовательский центр <<Информатика и~управ\-ле\-ние>> 
Российской академии наук (ФИЦ ИУ РАН) (до~2015~г.~--- 
Институт проб\-лем информатики РАН).

Необходимость издания такого журнала была вызвана активным развитием 
информатики и~информационных технологий, большой важностью этого научного 
направления для развития страны, проникновением информационных технологий 
во все сферы жизни современного общества.

Тематику журнала определяет тот факт, что информатика~--- это комплексная 
фундаментальная научная дисциплина, опирающаяся на достижения 
ряда других наук, в~том числе математики, физики, лингвистики и~др. 
Одновременно журнал уделяет большое внимание современным информационным технологиям, 
являющимся приложениями результатов информатики как фундаментальной науки.

За прошедшие 10~лет (2007--2016~гг.)\ издано~38~выпусков журнала. В~них 
размещено~452~публикации, в~том числе~430~научных статей и~22~информационных 
публикации (обзоры, рецензии и~др.). Среди авторов журнала представители ведущих 
научных организаций и~университетов страны, в~том числе Московского государственного 
университета им.\ М.\,В.~Ломоносова, ФИЦ ИУ РАН (в~том числе ИПИ РАН, ВЦ 
им.\ А.\,А.~Дородницына РАН, ИСА РАН), Института точной механики и~вычислительной 
техники им.\ С.\,А.~Лебедева РАН, Института космических исследований РАН, 
Института астрономии РАН, ряда институтов Сибирского отделения РАН, МФТИ, МИФИ, 
Высшей школы экономики, Санкт-Пе\-тер\-бург\-ско\-го государственного университета, 
Санкт-Пе\-тер\-бург\-ско\-го государственного политехнического университета 
Петра Великого, Санкт-Пе\-тер\-бург\-ско\-го государственного университета 
телекоммуникаций им.\ проф.\ М.\,А.~Бонч-Бруе\-ви\-ча, 
Российского университета дружбы народов, Балтийского федерального университета 
имени Иммануила Канта, Вологодского государственного университета и~др. 
Публиковались статьи зарубежных авторов, в~том числе ученых из Израиля, 
США, Финляндии, Франции, Швейцарии, Швеции и~других стран. 

В конце настоящего выпуска журнала помещен указатель статей, 
опуб\-ли\-ко\-ван\-ных в~томах~1--10 (2007--2016~гг.).

Журнал включен в~Российский индекс научного цитирования и~в~базу 
данных RSCI Web of Science, перечень ВАК, базу данных CrossRef 
и~информационную систему <<Общероссийский математический портал MathNet>>. 
С~2015~г.\ журнал индексируется в~библиографической и~реферативной базе 
данных SCOPUS.

Мы всегда будем помнить ушедших из жизни членов редакционного совета 
и~редакционной коллегии журнала: академика С.\,К.~Коровина, профессоров 
А.\,В.~Печинкина и~И.\,А.~Ушакова, которые внесли неоценимый вклад в~становление 
и~развитие журнала.

После объединения в~2015~г.\ трех учреждений Российской академии наук~--- 
Института проблем информатики, Вычислительного центра им.\ А.\,А.~Дородницына 
и~Института системного анализа~--- в~Федеральное государственное учреждение 
<<Федеральный исследовательский центр <<Информатика и~управ\-ле\-ние>> 
Российской академии наук>> (ФИЦ ИУ РАН) именно этот Центр стал базовой организацией 
для издания журнала, что существенно расширило как тематику журнала, 
так и~его возможности по привлечению новых авторов, в~том числе и~зарубежных.

В настоящее время тематику журнала в~первую очередь составляют:
\begin{itemize}
\item    теоретические основы информатики;\\[-14.5pt] 
\item    математические методы исследования сложных систем и~процессов;\\[-14.5pt]
\item    информационные системы и~сети;\\[-14.5pt]
\item    информационные технологии;\\[-14.5pt]
\item    архитектура и~программное обеспечение вычислительных комплексов и~сетей. 
\end{itemize}

Эти направления особенно важны в~связи с необходимостью решения задач 
формирования технологической базы инновационного развития, обеспечения 
на\-уч\-но-тех\-но\-ло\-ги\-че\-ско\-го прорыва в~области создания и~развития 
отечественных информационных и~коммуникационных технологий в~интересах 
достижения высокого качества и~стабильности систем управления и~предоставления 
услуг в~экономической и~социальной сферах. 

Мы, как и~ранее, приглашаем авторов представлять для публикации в~журнале 
статьи как с достижениями в~области теоретических проблем информатики, так 
и~с~изложением результатов ее практического приложения, а~также 
рецензии на наиболее интересные книжные новинки в~области информатики 
и~информационных технологий, объявления о~крупнейших международных 
и~всероссийских конференциях, различных научных мероприятиях 
по этой тематике и~другие информационные материалы.

Надеемся, что и~в~дальнейшем содержание статей, помещаемых в~журнале, 
будет вызывать интерес научной общественности. Редакционный совет, редколлегия 
и~редакция журнала, со своей стороны, сделают все для того, 
чтобы журнал и~впредь своевременно и~подробно информировал читателей 
о~новейших достижениях информатики и~ее актуальных практических приложениях.

                

      
\vfill
\noindent
Главный редактор журнала <<Информатика и~её применения>>,\\
академик  РАН\hfill
\textit{И.\,А.~Соколов}\\[-6pt]

%\noindent
%Редактор-составитель тематического выпуска, профессор кафедры математической статистики\\
%факультета вычислительной математики и~кибернетики МГУ им.~М.\,В.~Ломоносова,\\
%ведущий научный сотрудник ИПИ РАН, доктор физико-математических наук\hfill
% \textit{В.\,Ю.~Королев}


} }
}
      

%%%%%%%%%%%%%%%%%%%%%%%%%%%%%%%%%%%%%%%%%%%%%%%


                       
%\end{document}

%\def\stat{rez}
{%\hrule\par
%\vskip 7pt % 7pt
\raggedleft\Large \bf%\baselineskip=3.2ex
Р\,Е\,Ц\,Е\,Н\,З\,И\,И \vskip 17pt
    \hrule
    \par
\vskip 6pt plus 6pt minus 3pt }

%\thispagestyle{headings} %с верхним колонтитулом
%\thispagestyle{myheadings} %с нижним колонтитулом, но в верхнем РЕЦЕНЗИИ

\def\tit{НОВАЯ КНИГА И.\,Н.~СИНИЦЫНА, А.\,С.~ШАЛАМОВА <<ЛЕКЦИИ ПО ТЕОРИИ 
ИНТЕГРИРОВАННОЙ ЛОГИСТИЧЕСКОЙ ПОДДЕРЖКИ>> (М.: ТОРУС ПРЕСС, 2012. 624~с.)}

%1
\def\aut{Д.ф.-м.н., профессор С.\,Я.~Шоргин}

\def\auf{\ }

\def\leftkol{\ % РЕЦЕНЗИИ
}

\def\rightkol{ \ } 

%\def\leftkol{\ } % ENGLISH ABSTRACTS}

%\def\rightkol{\ } %ENGLISH ABSTRACTS}

%\def\leftkol{РЕЦЕНЗИИ}

%\def\rightkol{РЕЦЕНЗИИ}

\titele{\tit}{\aut}{\auf}{\leftkol}{\rightkol}
\vspace*{-18pt}


     \label{st\stat}

     \begin{multicols}{2}
     {\small
     {\baselineskip=10.1pt
     

      В книге представлено системное изложение теоретических основ одного из новейших 
направлений в \mbox{об\-ласти} экономики послепродажного обслуживания изделий наукоемкой 
продукции (ИНП) длительного пользования~--- интегрированной логистической поддержки
(ИЛП). 
{\looseness=1

}

Приведены также результаты новых работ, выполненных в Институте проблем информатики 
Российской академии наук в рамках научного направления <<Информационные технологии и 
анализ сложных сис\-тем>>.
 {%\looseness=1

}
     
      Излагаемые в книге научные подходы позво\-ляют карди\-наль\-но реформировать 
существующие системы производства и эксплуатации ИНП путем создания и внед\-ре\-ния 
методов рационального и оптимального управ\-ле\-ния процессами расходования 
вре\-мен\-н$\acute{\mbox{ы}}$х, 
мате\-ри\-аль\-ных, трудовых и других ресурсов на всех стадиях жизненного цикла изделий (ЖЦИ) по 
критериям экономической целесообразности и эф\-фек\-тив\-ности.
  {\looseness=1

}
    
      В книге приведен краткий обзор причин возник\-новения и
      развития CALS-методологии как основы 
современных международных стандартов по созданию и функционированию глобальных 
ин\-фор\-ма\-ци\-он\-но-ком\-му\-ни\-ка\-ци\-он\-ных систем, ее ключевых возможностей и эффективности 
результатов ее использования. 
Авторы %\linebreak 
предлагают ряд научных обоснований для разработки 
единой теории проектирования и управления систем ИЛП для полноценного использования 
преимуществ %\linebreak
 суще\-ст\-ву\-ющей методологии, определяют \mbox{общую} структурную схему 
комплексной системы <<ИНП-СППО>> и необходимость разработки для ее описания 
гибридных стохастических моделей.
{%\looseness=1

}

%\columnbreak
      
      Книга состоит из пяти частей, где последовательно излагается материал по каждой из 
следующих тем: <<Интегрированная логистическая поддержка>>, <<Теория гибридных 
стохастических систем и компьютерная поддержка исследований и разработок>>, <<Основы 
математического моделирования, анализа и синтеза систем послепродажного обслуживания>>, 
<<Определение и анализ показателей экспортного потенциала ИНП при проектировании>>, 
<<Задачи управления поддержкой послепродажного обслуживания>>, а также 
<<Моделирование инвестиционных процессов ИЛП в условиях неравновесных финансовых 
рынков>>. 
   
      В конце каждой главы приведены выводы и даны вопросы и задания для 
самоконтроля. В~приложениях содержатся основные определения по программам работ по 
анализу ИЛП, логистическим базам данных и компьютерным решениям, эквивалентной статистической 
линеаризации нелинейных преобразований ИЛП, справочный материал, а также развернутые 
уравнения для вероятностных характеристик.


      \def\leftkol{РЕЦЕНЗИИ}

\def\rightkol{РЕЦЕНЗИИ} 

      
      Книга заинтересует широкий круг специалистов и может быть использована научными 
проектными организациями в сфере промышленного производства ИНП. Большое количество 
иллюстраций, примеров и вопросов, обращенных к читателю, позволяет использовать книгу 
также в качестве учебного пособия для студентов и аспирантов машиностроительных, 
транспортных и~других специальностей, а также для самостоятельного изучения. 
{%\looseness=-1

}

Книга 
представляет несомненный интерес для специалистов и студентов в области прикладной 
математики и информатики.
    

}

}
\end{multicols}

%\newpage

\include{obchak}


\def\stat{authorsrus}
{%\hrule\par
%\vskip 7pt % 7pt
\raggedleft\Large \bf%\baselineskip=3.2ex
О\,Б\ \ А\,В\,Т\,О\,Р\,А\,Х \vskip 17pt
    \hrule
    \par
\vskip 21pt plus 8pt minus 6pt }


\def\tit{\ }

\def\aut{\ }

\def\auf{\ }

\def\leftkol{ОБ АВТОРАХ}

\def\rightkol{\ }

\titele{\tit}{\aut}{\auf}{\leftkol}{\rightkol}
\addcontentsline{toc}{subsection}{\textrm\textbf ОБ АВТОРАХ}
\label{st\stat}



\vspace*{-38pt}

\begin{multicols}{2}

\noindent
\textbf{Агаларов Явер Мирзабекович} (р.\ 1952)~--- 
кандидат технических наук, доцент, ведущий научный сотрудник 
Института проб\-лем информатики Федерального исследовательского центра 
<<Информатика и~управ\-ле\-ние>> Российской академии наук

\vspace*{3pt}

\noindent
\textbf{Битюков Юрий Иванович} (р.\ 1972)~---
доктор технических наук, доцент Московского авиационного института 
(национального исследовательского университета) 

\vspace*{3pt}

\noindent
\textbf{Буянов Михаил Владимирович} (р.\ 1994)~--- 
аспирант Московского авиационного института (национального исследовательского 
университета)

\vspace*{3pt}

\noindent
\textbf{Вихрова Ольга Геннадиевна} (р.\ 1990)~---
 аспирант Российского университета дружбы народов
 
 \vspace*{3pt}
 

\noindent
\textbf{Гайдамака Юлия Васильевна} (р.\ 1971)~--- 
кандидат фи\-зи\-ко-ма\-те\-ма\-ти\-че\-ских наук, доцент Российского университета 
дружбы народов; старший научный сотрудник Института проб\-лем информатики 
Федерального исследовательского центра <<Информатика и~управ\-ле\-ние>> 
Российской академии наук 

\vspace*{3pt}

\noindent
\textbf{Горшенин Андрей Константинович} (р.\ 1986)~--- 
кандидат фи\-зи\-ко-ма\-те\-ма\-ти\-че\-ских наук, доцент, 
ведущий научный сотрудник Института проб\-лем информатики Федерального 
исследовательского\linebreak
 центра <<Информатика и~управ\-ле\-ние>> Российской академии наук;
 старший научный сотрудник
 Института океанологии им.\ П.\,П.~Ширшова Российской академии наук

\vspace*{3pt}


\noindent
\textbf{Гребешков Александр Юрьевич} (р.\ 1967)~--- 
кандидат технических наук, старший научный сотрудник Поволжского 
государственного университета телекоммуникаций и информатики

\vspace*{3pt}

\noindent
\textbf{Грушо Александр Александрович} (р.\ 1946)~--- доктор 
фи\-зи\-ко-ма\-те\-ма\-ти\-че\-ских наук, профессор, заведующий лабораторией 
Института проб\-лем информатики Федерального исследовательского центра 
<<Информатика и~управ\-ле\-ние>> Российской академии наук 

\vspace*{3pt}

\noindent
\textbf{Забежайло Михаил Иванович} (р.\ 1956)~--- 
кандидат фи\-зи\-ко-ма\-те\-ма\-ти\-че\-ских наук, доцент, заведующий лабораторией 
Института проб\-лем информатики Федерального исследовательского центра 
<<Информатика и~управ\-ле\-ние>> Российской академии наук 

%\vspace*{3pt}
\columnbreak

\noindent
\textbf{Зарипова Эльвира Ринатовна} (р.\ 1979)~--- 
кандидат фи\-зи\-ко-ма\-те\-ма\-ти\-че\-ских наук, доцент Российского университета 
дружбы народов

\vspace*{3pt}

\noindent
\textbf{Иванов Сергей Валерьевич} (р.\ 1989)~--- 
кандидат фи\-зи\-ко-ма\-те\-ма\-ти\-че\-ских наук, доцент Московского 
авиационного института (национального исследовательского университета)

\vspace*{3pt}

\noindent
\textbf{Кибзун Андрей Иванович}  (р.\ 1951)~--- 
доктор фи\-зи\-ко-ма\-те\-ма\-ти\-че\-ских наук, профессор, 
заведующий кафедрой Московского авиационного института 
(национального исследовательского университета)

\vspace*{3pt}

\noindent
\textbf{Королев Виктор Юрьевич} (р.\ 1954)~--- доктор 
фи\-зи\-ко-ма\-те\-ма\-ти\-че\-ских наук, профессор, 
заведующий кафедрой математической статистики факультета вычислительной 
математики и~кибернетики МГУ им.\ М.\,В.~Ломоносова; 
ведущий научный сотрудник Института проб\-лем информатики 
Федерального исследовательского центра <<Информатика и~управ\-ле\-ние>> 
Российской академии наук; профессор Университета Дианьзи города Ханчжоу (Китай)

\vspace*{3pt}


\noindent
\textbf{Кружков Михаил Григорьевич} (р.\ 1975)~--- 
старший научный сотрудник Института проб\-лем 
информатики Федерального исследовательского центра 
<<Информатика и~управ\-ле\-ние>> Российской академии наук

\vspace*{3pt}

\noindent
\textbf{Кудрявцев Алексей Андреевич} (p.\ 1978)~--- кандидат 
фи\-зи\-ко-ма\-те\-ма\-ти\-че\-ских наук, 
доцент кафедры математической статистики факультета вычислительной математики 
и~кибернетики Московского государственного университета им.\ М.\,В.~Ломоносова

\vspace*{3pt}

\noindent
\textbf{Лисовская Екатерина Юрьевна} (р.\ 1992)~--- 
аспирант Национального исследовательского 
Томского государственного университета 

\vspace*{3pt}

\noindent
\textbf{Малашенко Юрий Евгеньевич} (р.\ 1946)~---
доктор фи\-зи\-ко-ма\-те\-ма\-ти\-че\-ских наук, заведующий сектором 
Вычислительного центра им.\ А.\,А.~Дородницына Федерального исследовательского центра 
<<Информатика и~управ\-ле\-ние>> Российской академии \mbox{наук}

\vspace*{3pt}


\noindent
\textbf{Моисеева Светлана Петровна} (р.\ 1971)~--- 
доктор фи\-зи\-ко-ма\-те\-ма\-ти\-че\-ских наук, доцент; 
профессор Национального исследовательского Томского государственного 
университета  

%\vspace*{3pt}
\pagebreak

\noindent
\textbf{Мокров Евгений Владимирович} (р.\ 1988)~--- 
аспирант Российского университета дружбы народов 

\vspace*{3pt}

\noindent
\textbf{Назарова Ирина Александровна} (р.\ 1966)~---
 кандидат фи\-зи\-ко-ма\-те\-ма\-ти\-че\-ских наук, научный сотрудник 
 Вычислительного центра им.\ А.\,А.~Дородницына Федерального исследовательского центра 
 <<Информатика и~управ\-ле\-ние>> Российской академии наук

\vspace*{3pt}

\noindent
\textbf{Наумов Андрей Викторович} (р.\ 1966)~--- 
доктор фи\-зи\-ко-ма\-те\-ма\-ти\-че\-ских наук, доцент, 
профессор\linebreak Московского авиационного института (национального исследовательского 
университета)

\vspace*{3pt}

\noindent
\textbf{Наумов Валерий Арсентьевич} (р.\ 1950)~--- 
кандидат фи\-зи\-ко-ма\-те\-ма\-ти\-че\-ских наук, 
научный руководитель Исследовательского института инноваций, 
г.~Хельсинки, Финляндия

\vspace*{3pt}

\noindent
\textbf{Новикова Наталья Михайловна} (р.\ 1953)~--- 
доктор фи\-зи\-ко-ма\-те\-ма\-ти\-че\-ских наук, профессор, ведущий научный сотрудник 
Вычислительного центра им.\ А.\,А.~Дородницына Федерального исследовательского центра 
<<Информатика и~управ\-ле\-ние>> Российской академии наук

\vspace*{3pt}

\noindent
\textbf{Пагано Микеле} (р.\ 1968)~---
PhD по информационным технологиям, профессор Университета 
г.\ Пиза (Италия) 

\vspace*{3pt}

\noindent
\textbf{Платонов Евгений Николаевич} (р.\ 1976)~---  
кандидат фи\-зи\-ко-ма\-те\-ма\-ти\-че\-ских наук, 
доцент Московского авиационного института (национального исследовательского 
университета)

\vspace*{3pt}

\noindent
\textbf{Потатуева Виктория Владимировна} (р.\ 1993)~---  
студентка магистратуры Национального исследовательского 
Томского государственного университета

\vspace*{3pt}


\noindent
\textbf{Разумчик Ростислав Валерьевич} (р.\ 1984)~--- 
кандидат фи\-зи\-ко-ма\-те\-ма\-ти\-че\-ских наук, 
ведущий научный сотрудник Института проб\-лем 
информатики Федерального исследовательского центра <<Информатика и~управ\-ле\-ние>>
Российской академии наук;  доцент Российского университета дружбы народов

\vspace*{3pt}

\noindent
\textbf{Самуйлов Константин Евгеньевич} (р.\ 1955)~---
доктор технических наук, профессор, заведующий ка\-фед\-рой Российского 
университета дружбы наро-\linebreak дов, директор Института прикладной математики\linebreak 
и~телекоммуникаций Российского университета дружбы народов; 
старший научный сотрудник Института проб\-лем информатики Федерального 
исследовательского центра <<Информатика и~управ\-ле\-ние>> 
Российской академии наук

\vspace*{3pt}

\noindent
\textbf{Смирнов Дмитрий Владимирович} (р.\ 1984)~--- 
биз\-нес-парт\-нер по информационным технологиям Департамента безопасности ПАО 
<<Сбербанк России>>

\vspace*{3pt}

\noindent
\textbf{Тимонина Елена Евгеньевна} (р.\ 1952)~--- 
доктор технических наук, профессор, ведущий научный\linebreak сотрудник 
Института проб\-лем информатики Федерального исследовательского центра 
<<Информатика и~управ\-ле\-ние>> Российской академии наук 

\vspace*{3pt}

\noindent
\textbf{Титова Анастасия Игоревна} (p.\ 1995)~--- 
студентка кафедры математической статистики факультета вычисли\-тельной математики 
и~кибернетики Московского государственного университета им.\ М.\,В.~Ломоносова

\vspace*{3pt}

\noindent
\textbf{Шоргин Всеволод Сергеевич} (р.\ 1978)~---
кандидат технических наук, старший научный сотрудник Института проб\-лем 
информатики Федерального исследовательского центра <<Информатика и~управ\-ле\-ние>> 
Российской академии наук

\vspace*{3pt}

\noindent
\textbf{Шоргин Сергей Яковлевич} (р.\ 1952)~--- 
доктор фи\-зи\-ко-ма\-те\-ма\-ти\-че\-ских наук, профессор, заместитель директора 
Федерального исследовательского цент\-ра <<Информатика и~управ\-ле\-ние>> 
Российской академии наук (ФИЦ ИУ РАН); главный научный сотрудник Института проб\-лем 
информатики ФИЦ ИУ РАН
 



 \label{end\stat}

%\def\leftfootline{\small{\textbf{\thepage}
%\hfill ИНФОРМАТИКА И ЕЁ ПРИМЕНЕНИЯ\ \ \ том~11\ \ \ выпуск~4\ \ \ 2017}
%}%
% \def\rightfootline{\small{ИНФОРМАТИКА И ЕЁ ПРИМЕНЕНИЯ\ \ \ том~11\ \ \ выпуск~4\ \ \ 2017
%\hfill \textbf{\thepage}}}


%\thispagestyle{myheadings}



\end{multicols}

\newpage


\def\stat{authors}
{%\hrule\par
%\vskip 7pt % 7pt
\raggedleft\Large \bf%\baselineskip=3.2ex
A\,B\,O\,U\,T\  \  A\,U\,T\,H\,O\,R\,S \vskip 17pt
    \hrule
    \par
\vskip 21pt plus 8pt minus 3pt }

\label{st\stat}


\def\leftkol{ABOUT AUTHORS} % 
\def\rightkol{\ } %ABOUT AUTHORS} 


\vspace*{36pt}

\begin{multicols}{2}

%\vspace*{4pt}

\noindent \textbf{Andreev Arkady M.} (b.\ 1943)~--- Candidate of Science (PhD) in 
technology, assistant professor, Bauman Moscow State Technical University

\vspace*{4pt}

\vspace*{4pt}

\noindent
\textbf{Belyaev Mikhail G.} (b.\ 1987)~--- PhD student, Institute for Information 
Transmission Problems, Russian Academy of Sciences; junior scientist, Moscow 
Institute of Physics and Technology; scientist, Datadvance LLC

\vspace*{4pt}

\noindent
\textbf{Berezkin Dmitry V.} (b.\ 1966)~--- Candidate of Science (PhD) in technology, senior 
scientist, Bauman Moscow State Technical University


\vspace*{4pt}

\noindent
\textbf{Burnaev Evgeny V.} (b.\ 1983)~--- Candidate of Science (PhD) in physics and 
mathematics, associate professor; Head of Laboratory, Institute for Information 
Transmission Problems, Russian Academy of Sciences; senior scientist, Moscow 
Institute of Physics and Technology; Head of Laboratory, Datadvance LLC

\vspace*{4pt}

\noindent
\textbf{Glushanovskiy Alexey V.} (b.\ 1944)~--- senior scientist, Library for Natural 
Sciences, Russian Academy of Sciences

\vspace*{4pt}

\noindent
\textbf{Kaganov Vladislav Yu.} (b.\ 1993)~--- student, Faculty of Computational Mathematics 
and Cybernetics, M.\,V.~Lomonosov Moscow State University


\vspace*{4pt}

\noindent
\textbf{Kalenov Nikolay E.} (b.\ 1945)~--- Doctor of Science in technology, professor, 
Director, Library for Natural Sciences, Russian Academy of Sciences 

\vspace*{4pt}

\noindent
\textbf{Kapnin Alexey V.} (b.\ 1986)~--- PhD student, assistant professor of Lipetsk State 
Technical University


\vspace*{4pt}

\noindent \textbf{Kirikov Igor  A.} (b.\ 1955)~--- Candidate of Science (PhD) 
in technology, Director, Kaliningrad Branch of Institute of Informatics 
Problems, Russian Academy of Sciences 

\vspace*{4pt}

\noindent
\textbf{Klemenkov Pavel A.} (b.\ 1986)~--- PhD student, Department of  System Programming, 
Faculty of Computational Mathematics and Cybernetics, M.\,V.~Lomonosov Moscow State 
University


\vspace*{4pt}

\noindent
\textbf{Kolesnikov Alexander V.} (b.\ 1948)~--- Doctor of Science in technology; professor, 
Immanuel Kant Baltic Federal University; senior scientist, Kaliningrad Branch 
of Institute of Informatics Problems, Russian Academy of Sciences 

\columnbreak

\noindent
\textbf{Korenkov Vladimir V.} (b.\ 1953)~--- Candidate of Science (PhD) in physics and 
mathematics; Director, Laboratory of Information Technologies, Joint 
Institute for Nuclear Research (JINR); Head of Department, International 
University of Nature, Society and Man ``Dubna'' 

\vspace*{5.5pt}


\noindent
\textbf{Korolev Victor Yu.} (b.\ 1954)~--- Doctor of Science in physics and mathematics, 
professor, Department of Mathematical Statistics, Faculty of Computational 
Mathematics and Cybernetics, M.\,V.~Lomonosov Moscow State University; leading 
scientist, Institute of Informatics Problems, Russian Academy of Sciences

\vspace*{5.5pt}

\noindent
\textbf{Korolyov Andrey K.} (b.\ 1992)~--- student, Faculty of Computational Mathematics 
and Cybernetics, M.\,V.~Lomonosov Moscow State University

\vspace*{5.5pt}

\noindent
\textbf{Kovalyov Sergey P.} (b.\ 1972)~--- Candidate of Science (PhD) in physics and 
mathematics, senior scientist, Institute of Control Problems, Russian Academy 
of Sciences

\vspace*{5.5pt}

\noindent
\textbf{Kozhunova Olga S.} (b.\ 1982)~--- Candidate of Science (PhD) in technology, Head of 
Laboratory, Institute of Informatics Problems, Russian Academy of Sciences

\vspace*{5.5pt}

\noindent
\textbf{Kozlov Ilya A.} (b.\ 1989)~--- MD student, Department of 
Informatics and Control Systems, Bauman Moscow State 
Technical University

\vspace*{5.5pt}

\noindent
\textbf{Krylov Michael N.} (b.\ 1992)~--- student, Faculty of Computational Mathematics and 
Cybernetics, M.\,V.~Lomonosov Moscow State University

\vspace*{5.5pt}

\noindent
\textbf{Kuznetsov Leonid A.} (b.\ 1942)~--- Doctor of Science in technology, professor, 
Head of Department, Russian Presidential Academy of National Economy and Public 
Administration  (Lipetsk Branch)

\vspace*{5.5pt}

\noindent
\textbf{Kuznetsova Vera F.} (b.\ 1948)~--- Candidate of Science (PhD) in technology, 
associate professor of the Russian Presidential Academy of National Economy and 
Public Administration  (Lipetsk Branch)


\vspace*{5.5pt}

\noindent
\textbf{Listopad Sergey V.} (b.\ 1984)~--- Candidate of Science (PhD) in technology, 
scientist, Kaliningrad Branch of Institute of Informatics Problems, Russian 
Academy of Sciences

\vspace*{5.5pt}

\noindent
\textbf{Mashechkin Igor V.} (b.\ 1956)~--- Doctor of Science in physics and mathematics, 
professor, Faculty of Computational Mathematics and Cybernetics, M.\,V.~Lomonosov Moscow State University

%\pagebreak

\vspace*{2pt}



\noindent
\textbf{Nechaevskiy Andrey V.} (b.\ 1982)~--- programmer, Laboratory of Information 
Technologies, Joint Institute for 
nuclear research (JINR) 

\vspace*{2pt}

\noindent
\textbf{Petrovskiy Michael I.} (b.\ 1975)~--- Candidate of Science (PhD) in physics and 
mathematics,  associate professor, Faculty of Computational Mathematics and 
Cybernetics, M.\,V.~Lomonosov Moscow State University

\def\leftkol{ABOUT AUTHORS} %
\def\rightkol{ABOUT AUTHORS} 

\vspace*{2pt}

\noindent
\textbf{Shkotin Alexander V.} (b.\ 1952)~--- software engineer, GIS Department,  State 
Geological Museum of Russian Academy of Sciences

\columnbreak

%\vspace*{4pt}

\noindent
\textbf{Simakov Konstantin V.} (b.\ 1980)~--- Candidate of Science 
(PhD) in technology, senior scientist, Bauman Moscow State Technical 
University

%\columnbreak

\def\leftkol{ABOUT AUTHORS} %
\def\rightkol{ABOUT AUTHORS} 

\vspace*{7pt}

\noindent
\textbf{Stupnikov Sergey A.} (b.\ 1978)~--- Candidate of Science (PhD) in technology, 
senior scientist, Institute of Informatics Problems, Russian Academy of 
Sciences

\vspace*{7pt}

\noindent
\textbf{Trofimov Vladimir V.} (b.\ 1955)~--- leading programmer, 
Laboratory of Information Technologies, Joint Institute for nuclear research 
(JINR)


\vspace*{7pt}

\noindent
\textbf{Zaks Lily M.} (b.\ 1989)~--- principal officer, Department of Modeling and 
Mathematical Statistics, Alpha-Bank

\def\leftkol{ABOUT AUTHORS} %
\def\rightkol{ABOUT AUTHORS} 


\end{multicols}
\newpage

\vspace*{-60pt} {\small
{\baselineskip=9.1pt
\section*{Правила подготовки рукописей статей для публикации в журнале
<<Информатика и её применения>>}

\thispagestyle{empty}

 Журнал <<Информатика и её применения>> публикует
теоретические, обзорные и дискуссионные статьи, посвященные научным
исследованиям и разработкам в области информатики и ее приложений. Журнал
издается на русском языке. По специальному решению редколлегии отдельные статьи,
в виде исключения, могут печататься на английском языке.
Тематика журнала охватывает следующие направления:
\begin{itemize}
\item теоретические основы информатики; %\\[-13.5pt]
\item математические методы исследования сложных систем и процессов; %\\[-13.5pt]
\item информационные системы и сети; %\\[-13.5pt]
\item информационные технологии; %\\[-13.5pt]
\item архитектура и программное
обеспечение вычислительных комплексов и сетей.
\end{itemize}
\begin{enumerate}
\item В журнале печатаются результаты, ранее не
опубликованные и не предназначенные к одновременной публикации в других
изданиях. Публикация не должна нарушать закон об авторских правах. Направляя
свою рукопись в редакцию, авторы автоматически передают учредителям и
редколлегии неисключительные права на издание данной статьи на русском языке и
на ее распространение в России и за рубежом. При этом за авторами сохраняются
все права как собственников данной рукописи. В связи с этим авторами должно
быть представлено в редакцию письмо в следующей форме:
Соглашение о передаче права на публикацию:

\textit{<<Мы, нижеподписавшиеся, авторы рукописи <<$\qquad\qquad$>>, передаем
учредителям и редколлегии журнала <<Информатика и её применения>>
неисключительное право опубликовать данную рукопись статьи на русском языке как
в печатной, так и в электронной версиях журнала. Мы подтверждаем, что данная
публикация не нарушает авторского права других лиц или организаций. Подписи
авторов: (ф.\,и.\,о., дата, адрес)>>.}

Указанное соглашение может быть представлено 
как в бумажном виде, так и в виде отсканированной копии (с подписями авторов).


Редколлегия вправе запросить у авторов экспертное заключение о возможности
опубликования представленной статьи в открытой печати. %\\[-13.5pt]
\item Статья
подписывается всеми авторами. На отдельном листе представляются данные автора
(или всех авторов): фамилия, полные имя и отчество, телефон, факс, e-mail,
почтовый адрес. Если работа выполнена несколькими авторами, указывается фамилия
одного из них, ответственного за переписку с редакцией. %\\[-13.5pt]
\item Редакция журнала
осуществляет самостоятельную экспертизу присланных статей. Возвращение рукописи
на доработку не означает, что статья уже принята к печати. Доработанный вариант
с ответом на замечания рецензента необходимо прислать в редакцию. %\\[-13.5pt]
\item Решение
редакционной коллегии о принятии статьи к печати или ее отклонении сообщается
авторам. Редколлегия не обязуется направлять рецензию авторам отклоненной
статьи; дискуссия с авторами по поводу отклоненных статей не ведется. %\\[-13.5pt]
\item Корректура статей высылается авторам для просмотра. Редакция
просит авторов присылать свои замечания в кратчайшие сроки. %\\[-13.5pt]
\item При
подготовке рукописи в MS Word рекомендуется использовать следующие настройки.
Параметры страницы: формат~--- А4; ориентация~--- книжная; поля (см): внутри~---
2,5, снаружи~--- 1,5, сверху~--- 2, снизу~--- 2, от края до нижнего
колонтитула~--- 1,3. Основной текст: стиль~--- <<Обычный>>: шрифт Times New
Roman, размер 14~пунктов, абзацный отступ~--- 0,5~см, 1,5 интервала,
выравнивание~--- по ширине. Рекомендуемый объем рукописи~--- не свыше
25~страниц указанного формата. Ознакомиться с шаблонами, содержащими примеры
оформления, можно по адресу в Интернете:
\textsf{http://www.ipiran.ru/journal/template.doc}.
\item К рукописи, предоставляемой в 2-х
экземплярах, обязательно прилагается электронная версия статьи (как правило, в
форматах MS WORD (.doc) или \LaTeX\ (.tex), а также~--- дополнительно~--- в
формате .pdf) на дискете, лазерном диске или по электронной почте. Сокращения
слов, кроме стандартных, не применяются. Все страницы рукописи должны быть
пронумерованы. %\\[-13.5pt]
\item Статья должна содержать следующую информацию на русском и
английском языках: название, Ф.И.О. авторов, места работы авторов и их
электронные адреса, подробные сведения об авторах, оформленные в соответствии с форматом, 
определяемым файлами {\sf http://www.ipiran.ru/journal/issues/2011\_05\_01/authors.asp} и 
{\sf http://www.ipiran.ru/journal/issues/2011\_01\_eng/authors.asp},
аннотация (не более 100~слов), ключевые слова. Ссылки на
литературу в тексте статьи нумеруются (в квадратных скобках) и располагаются в
порядке их первого упоминания. В~списке литературы не должно быть позиций, на которые нет ссылки в тексте статьи.
Все фамилии авторов, заглавия статей, названия
книг, конференций и~т.\,п.\ даются на языке оригинала, если этот язык
использует кириллический или латинский алфавит. %\\[-13.5pt]
\item Присланные в редакцию материалы авторам не возвращаются.
\item При отправке файлов по электронной
почте просим придерживаться следующих правил:
\begin{itemize}
\item указывать в поле subject (тема) название журнала и фамилию автора; %\\[-13.5pt]
\item использовать attach (присоединение); %\\[-13.5pt]
\item в случае больших объемов информации возможно
использование общеизвестных архиваторов (ZIP, RAR); %\\[-13.5pt]
\item в состав электронной версии статьи должны входить: файл, содержащий текст статьи, и файл(ы),
содержащий(е) иллюстрации. %\\[-13.5pt]
\end{itemize}
\item Журнал <<Информатика и её применения>> является некоммерческим изданием. 
Плата за публикацию с авторов не взимается, гонорар авторам не выплачивается.
\end{enumerate}
\thispagestyle{empty}
\textbf{Адрес редакции:} Москва 119333,
ул.~Вавилова, д.~44, корп.~2, ИПИ РАН\\
\hphantom{\textbf{Адрес редакции:} }Тел.: +7 (499) 135-86-92\ \
Факс:  +7 (495) 930-45-05\ \  E-mail:   rust@ipiran.ru }
}

\end{document}


%\tableofcontents

%\end{document}

\def\stat{cont}
{%\hrule\par
%\vskip 7pt % 7pt
\raggedleft\Large \bf%\baselineskip=3.2ex
А\,В\,Т\,О\,Р\,С\,К\,И\,Й\ \ У\,К\,А\,З\,А\,Т\,Е\,Л\,Ь\ \ З\,А\ \ 2\,0\,1\,0 г. \vskip 17pt
    \hrule
    \par
\vskip 21pt plus 6pt minus 3pt }

\label{st\stat}

\def\tit{\ }

\def\aut{\ }
\def\auf{\ }

\def\leftkol{\ } % ENGLISH ABSTRACTS}

\def\rightkol{\ } %АВТОРСКИЙ УКАЗАТЕЛЬ ЗА 2010 г.} %ENGLISH ABSTRACTS}

\titele{\tit}{\aut}{\auf}{\leftkol}{\rightkol}

\vspace*{-12pt}

{\tabcolsep=3pt
\begin{tabular}{p{388pt}rr}
&\textbf{Выпуск} & \textbf{Стр.}\\[6pt]
\hangindent=23pt\noindent\textbf{Арутюнян~А.\,Р.} Моделирование влияния деформаций отпечатков пальцев на 
точность\linebreak
\vspace*{-12pt}\\
\hspace*{23pt}дактилоскопической идентификации$\dotfill$&1&51\\
\hangindent=23pt\noindent\textbf{Архипов~О.\,П., Зыкова~З.\,П.} Интеграция гетерогенной информации о цветных 
пикселях\linebreak
\vspace*{-12pt}\\
\hspace*{23pt}и их цветовосприятии$\dotfill$&4&15\\
\hangindent=23pt\noindent\textbf{Баранов~С.\,И., Френкель~С.\,Л., Захаров~В.\,Н.} Полуформальная верификация 
цифрового устройства с конвейером, основанная на использовании алгоритмических машин\linebreak
\vspace*{-12pt}\\
\hspace*{23pt}состояния$\dotfill$&4&49\\
\textbf{Бекетова~И.\,В.} см.~Каратеев~С.\,Л.&&\\
\textbf{Белоусов~В.\,В.} см.~Синицын~И.\,Н.&&\\
\hangindent=23pt\noindent\textbf{Бенинг~В.\,Е., Королев~Р.\,А.} О предельном поведении мощностей критериев в 
случае\linebreak
\vspace*{-12pt}\\
\hspace*{23pt}распределения Лапласа$\dotfill$&2&63\\
\hangindent=23pt\noindent\textbf{Бенинг~В.\,Е., Сипина~А.\,В.} Асимптотическое разложение для мощности 
критерия,\linebreak
\vspace*{-12pt}\\
\hspace*{23pt}основанного на выборочной медиане, в случае распределения Лапласа$\dotfill$&1&18\\
\textbf{Бондаренко~А.\,В.} см.~Каратеев~С.\,Л.&&\\
\hangindent=23pt\noindent\textbf{Бородина~А.\,В., Морозов~Е.\,В.} Об оценивании асимптотики вероятности 
большого\linebreak
\vspace*{-12pt}\\
\hspace*{23pt}уклонения стационарной регенеративной очереди с одним прибором$\dotfill$&3&29\\
\hangindent=23pt\noindent\textbf{Бунтман~Н.\,В., Минель~Ж.-Л., Ле~Пезан~Д., Зацман~И.\,М.} Типология и 
компьютерное\linebreak
\vspace*{-12pt}\\
\hspace*{23pt}моделирование трудностей перевода$\dotfill$&3&77\\
\textbf{Визильтер~Ю.\,В.} см.~Каратеев~С.\,Л.&&\\
\hangindent=23pt\noindent\textbf{Гавриленко~С.\,В.} Оценки скорости сходимости распределений случайных сумм с 
безгранично делимыми индексами к нормальному закону$\dotfill$&4&81\\
\hangindent=23pt\noindent\textbf{Григорьева~М.\,Е., Шевцова~И.\,Г.} Уточнение неравенства 
Каца--Берри--Эссеена$\dotfill$&2&75\\
\hangindent=23pt\noindent\textbf{Грушо~А.\,А., Грушо~Н.\,А., Тимонина~Е.\,Е.} Поиск конфликтов в политиках 
безопасности: модель случайных графов$\dotfill$&3&38\\
\textbf{Грушо~Н.\,А.} см.~Грушо~А.\,А.&&\\
\hangindent=23pt\noindent\textbf{Гудков~В.\,Ю.} Математические модели изображения отпечатка пальца на основе 
описания линий$\dotfill$&1&58\\
\textbf{Гуртов~А.\,В.} см.~Лукьяненко~А.\,С.&&\\
\textbf{Желтов~С.\,Ю.} см.~Каратеев~С.\,Л.&&\\
\hangindent=23pt\noindent\textbf{Захаров~А.\,А., Серебряков~В.\,А.} Система управления электронной библиотекой 
LibMeta$\dotfill$&4&2\\
\textbf{Захаров~В.\,Н.} см.~Баранов~С.\,И.&&\\
\textbf{Захарова~Т.\,В.} см.~Матвеева~С.\,С.&&\\
\hangindent=23pt\noindent\textbf{Зацаринный~А.\,А., Чупраков~К.\,Г.} Некоторые аспекты выбора технологии для 
постро-\linebreak
\vspace*{-12pt}\\
\hspace*{23pt}ения систем отображения информации ситуационного центра$\dotfill$&3&59\\
\textbf{Зацман~И.\,М.} см.~Бунтман~Н.\,В.&&\\
\hangindent=23pt\noindent\textbf{Зейфман~А.\,И., Коротышева~А.\,В., Сатин~Я.\,А., Шоргин~С.\,Я.} Об 
устойчивости нестаци-\linebreak
\vspace*{-12pt}\\
\hspace*{23pt}онарных систем обслуживания с катастрофами$\dotfill$&3&9\\
\textbf{Зыкова~З.\,П.} см.~Архипов~О.\,П.&&\\
\hangindent=23pt\noindent\textbf{Илюшин~Г.\,Я., Соколов~И.\,А.} Организация управляемого доступа пользователей 
к\linebreak
\vspace*{-12pt}\\
\hspace*{23pt}разнородным ведомственным информационным ресурсам$\dotfill$&1&24\\
\hangindent=23pt\noindent\textbf{Кавагучи~Ю., Ульянов~В.\,В., Фуджикоши~Я.} Приближения для статистик, 
описывающих\linebreak
\vspace*{-12pt}\\
\hspace*{23pt}геометрические свойства данных большой размерности, с оценками 
ошибок$\dotfill$&1&12\\
\hangindent=23pt\noindent\textbf{Каратеев~С.\,Л., Бекетова~И.\,В., Ососков~М.\,В., Князь~В.\,А., 
Визильтер~Ю.\,В., Бондаренко~А.\,В., Желтов~С.\,Ю.} Автоматизированный контроль 
качества цифровых\linebreak
\vspace*{-12pt}\\
\hspace*{23pt}изображений для персональных документов$\dotfill$&1&65\\
\end{tabular}
}

\pagebreak

\def\leftkol{АВТОРСКИЙ УКАЗАТЕЛЬ ЗА 2010 г.} % ENGLISH ABSTRACTS}

\def\rightkol{АВТОРСКИЙ УКАЗАТЕЛЬ ЗА 2010 г.} %ENGLISH ABSTRACTS}

{\tabcolsep=3pt
\begin{tabular}{p{388pt}rr}
&\textbf{Выпуск} & \textbf{Стр.}\\[3pt]
\hangindent=23pt\noindent\textbf{Козеренко~Е.\,Б.} Лингвистические фильтры в статистических моделях машинного\linebreak
\vspace*{-12pt}\\
\hspace*{23pt}перевода$\dotfill$&2&83\\
\hangindent=23pt\noindent\textbf{Козеренко~Е.\,Б., Кузнецов~И.\,П.} Когнитивно-лингвистические представления в 
систе-\linebreak
\vspace*{-12pt}\\
\hspace*{23pt}мах обработки текстов$\dotfill$&3&69\\
\textbf{Князь~В.\,А.} см.~Каратеев~С.\,Л.&&\\
\hangindent=23pt\noindent\textbf{Колесников~А.\,В., Солдатов~С.\,А.} Алгоритм координации для гибридной 
интеллектуальной системы решения сложной задачи оперативно-производственного\linebreak
\vspace*{-12pt}\\
\hspace*{23pt}планирования$\dotfill$&4&61\\
\hangindent=23pt\noindent\textbf{Коновалов~М.\,Г.} О планировании потоков в системах вычислительных 
ресурсов$\dotfill$&2&3\\
\textbf{Конушин~А.\,С.} см.~Конушин~В.\,С.&&\\
\hangindent=23pt\noindent\textbf{Конушин~В.\,С., Кривовязь~Г.\,Р., Конушин~А.\,С.} Алгоритм распознавания людей 
в видео-\linebreak
\vspace*{-12pt}\\
\hspace*{23pt}последовательности по одежде$\dotfill$&1&74\\
\textbf{Корепанов~Э.\, Р.} см.~Синицын~И.\,Н.&&\\
\textbf{Королев~В.\,Ю.} см.~Соколов~И.\,А.&&\\
\textbf{Королев~Р.\,А.} см.~Бенинг~В.\,Е.&&\\
\textbf{Коротышева~А.\,В.} см.~Зейфман~А.\,И.&&\\
\hangindent=23pt\noindent\textbf{Кривенко~М.\,П.} Непараметрическое оценивание элементов байесовского 
клас\-си-\linebreak
\vspace*{-12pt}\\
\hspace*{23pt}фикатора$\dotfill$&2&13\\
\textbf{Кривовязь~Г.\,Р.} см.~Конушин~В.\,С.&&\\
\textbf{Крылов~А.\,С.} см.~Павельева~Е.\,А.&&\\
\hangindent=23pt\noindent\textbf{Крылов~В.\,А.} Моделирование и классификация многоканальных дистанционных\linebreak
\vspace*{-12pt}\\
\hspace*{23pt}изображений с использованием копул$\dotfill$&4&34\\
\hangindent=23pt\noindent\textbf{Крючин~О.\,В.} Разработка параллельных эвристических алгоритмов подбора 
весовых\linebreak
\vspace*{-12pt}\\
\hspace*{23pt}коэффициентов искусственной нейтронной сети$\dotfill$&2&53\\
\hangindent=23pt\noindent\textbf{Кудрявцев~А.\,А., Шоргин~С.\,Я.} Байесовские модели массового обслуживания и 
надеж-\linebreak
\vspace*{-12pt}\\
\hspace*{23pt}ности: характеристики среднего числа заявок в системе $M\vert M \vert 1\vert 
\infty$$\dotfill$&3&16\\
\hangindent=23pt\noindent\textbf{Кузнецов~А.\,А.} Связь между временными и структурно-топологическими 
характери-\linebreak
\vspace*{-12pt}\\
\hspace*{23pt}стиками диаграмм ритма сердца здоровых людей$\dotfill$&4&39\\
\textbf{Кузнецов~И.\,П.} см.~Козеренко~Е.\,Б.&&\\
\textbf{Ле~Пезан~Д.} см.~Бунтман~Н.\,В.&&\\
\hangindent=23pt\noindent\textbf{Лукьяненко~А.\,С., Морозов~Е.\,В., Гуртов~А.\,В.} Анализ сетевого протокола с общей 
функ-\linebreak
\vspace*{-12pt}\\
\hspace*{23pt}цией расширения окна передачи сообщения при конфликтах$\dotfill$&2&46\\
\hangindent=23pt\noindent\textbf{Лямин~О.\,О.} О предельном поведении мощностей критериев в случае обобщенного\linebreak
\vspace*{-12pt}\\
\hspace*{23pt}распределения Лапласа$\dotfill$&3&47\\
\hangindent=23pt\noindent\textbf{Маркин~А.\,В., Шестаков~О.\,В.} Асимптотики оценки риска при пороговой 
обработке\linebreak
\vspace*{-12pt}\\
\hspace*{23pt}вейвлет-вейглет коэффициентов в задаче томографии$\dotfill$&2&36\\
\hangindent=23pt\noindent\textbf{Матвеева~С.\,С., Захарова~Т.\,В.} Сети массового обслуживания с наименьшей 
длиной\linebreak
\vspace*{-12pt}\\
\hspace*{23pt}очереди$\dotfill$&3&22\\
\hangindent=23pt\noindent\textbf{Матюшенко~С.\,И.} Стационарные характеристики двухканальной системы 
обслужива-\linebreak
\vspace*{-12pt}\\
\hspace*{23pt}ния с переупорядочиванием заявок и распределениями фазового типа$\dotfill$&4&68\\
\textbf{Минель~Ж.-Л.} см.~Бунтман~Н.\,В.&&\\
\textbf{Морозов~Е.\,В.} см.~Бородина~А.\,В.&&\\
\textbf{Морозов~Е.\,В.} см.~Лукьяненко~А.\,С.&&\\
\textbf{Ососков~М.\,В.} см.~Каратеев~С.\,Л.&&\\
\hangindent=23pt\noindent\textbf{Павельева~Е.\,А., Крылов~А.\,С.} Поиск и анализ ключевых точек радужной 
оболочки\linebreak
\vspace*{-12pt}\\
\hspace*{23pt}глаза методом преобразования Эрмита$\dotfill$&1&79\\
\textbf{Печинкин~А.\,В.} см.~Френкель~С.\,Л.,&&\\
\hangindent=23pt\noindent\textbf{Протасов~В.\,И.} Составление субъективного портрета с использованием 
эволюционно-\linebreak
\vspace*{-12pt}\\
\hspace*{23pt}го морфинга и квалиметрия метода$\dotfill$&1&83\\
\hangindent=23pt\noindent\textbf{Рудаков~К.\,В., Торшин~И.\,Ю.} Вопросы разрешимости задачи распознавания 
вторичной\linebreak
\vspace*{-12pt}\\
\hspace*{23pt}структуры белка$\dotfill$&2&25\\
\textbf{Сатин~Я.\,А.} см.~Зейфман~А.\,И.&&\\
\hangindent=23pt\noindent\textbf{Сейфуль-Мулюков~Р.\,Б.} Нефть как носитель информации о своем 
происхождении,\linebreak
\vspace*{-12pt}\\
\hspace*{23pt}структуре и эволюции$\dotfill$&1&41\\
\end{tabular}
}

{\tabcolsep=3pt
\begin{tabular}{p{388pt}rr}
&\textbf{Выпуск} & \textbf{Стр.}\\[6pt]
\textbf{Семендяев~Н.\,Н.} см.~Синицын~И.\,Н.&&\\
\textbf{Серебряков~В.\,А.} см.~Захаров~А.\,А.&&\\
\textbf{Синицын~В.\,И.} см.~Синицын~И.\,Н.&&\\
\hangindent=23pt\noindent\textbf{Синицын~И.\,Н., Синицын~В.\,И., Корепанов~Э.\, Р., Белоусов~В.\,В., 
Семендяев~Н.\,Н.} Оперативное построение информационных моделей движения полюса 
Земли\linebreak
\vspace*{-12pt}\\
\hspace*{23pt}методами линейных и линеаризованных фильтров$\dotfill$&1&2\\
\textbf{Сипина~А.\,В.} см.~Бенинг~В.\,Е.&&\\
\hangindent=23pt\noindent\textbf{Соколов~И.\,А.} О работах заслуженного деятеля науки Российской Федерации 
И.\,Н.~Синицына в области информационных технологий и автоматизации (к 70-летию\linebreak
\vspace*{-12pt}\\
\hspace*{23pt}со дня рождения)$\dotfill$&3&84\\
\textbf{Соколов~И.\,А.} см.~Илюшин~Г.\,Я.&&\\
\hangindent=23pt\noindent\textbf{Соколов~И.\,А., Королев~В.\,Ю.} Предисловие$\dotfill$&2&2\\
\textbf{Солдатов~С.\,А.} см.~Колесников~А.\,В.&&\\
\hangindent=23pt\noindent\textbf{Степанов~С.\,Ю.} Использование координатного метода фрагментации 
коммутаторной\linebreak
\vspace*{-12pt}\\
\hspace*{23pt}нейронной сети для сокращения трафика$\dotfill$&2&57\\
\textbf{Тимонина~Е.\,Е.} см.~Грушо~А.\,А.&&\\
\textbf{Торшин~И.\,Ю.} см.~Рудаков~К.\,В.&&\\
\textbf{Ульянов~В.\,В.} см.~Кавагучи~Ю.&&\\
\textbf{Фазекаш~И.} см.~Чупрунов~А.\,Н.&&\\
\textbf{Френкель~С.\,Л.} см.~Баранов~С.\,И.&&\\
\hangindent=23pt\noindent\textbf{Френкель~С.\,Л., Печинкин~А.\,В.} Оценка времени самовосстановления в 
цифровых\linebreak
\vspace*{-12pt}\\
\hspace*{23pt}системах после сбоев, вызываемых переходными помехами$\dotfill$&3&2\\
\textbf{Фуджикоши~Я.} см.~Кавагучи~Ю.&&\\
\hangindent=23pt\noindent\textbf{Цискаридзе~А.\,К.} Математическая модель и метод восстановления позы человека 
по\linebreak
\vspace*{-12pt}\\
\hspace*{23pt}стереопаре силуэтных изображений$\dotfill$&4&27\\
\hangindent=23pt\noindent\textbf{Чупраков~К.\,Г.} К вопросу о размещении коллективных средств отображения в 
ситуа-\linebreak
\vspace*{-12pt}\\
\hspace*{23pt}ционном зале с заданными параметрами$\dotfill$&4&89\\
\textbf{Чупраков~К.\,Г.} см.~Зацаринный~А.\,А.&&\\
\hangindent=23pt\noindent\textbf{Чупрунов~А.\,Н., Фазекаш~И.} Законы повторного логарифма для числа 
безошибочных\linebreak
\vspace*{-12pt}\\
\hspace*{23pt}блоков при помехоустойчивом кодировании$\dotfill$&3&42\\
\textbf{Шевцова~И.\,Г.} см.~Григорьева~М.\,Е.&&\\
\hangindent=23pt\noindent\textbf{Шестаков~О.\,В.} Аппроксимация распределения оценки риска пороговой 
обработки вейвлет-коэффициентов нормальным распределением при использовании 
выбо-\linebreak
\vspace*{-12pt}\\
\hspace*{23pt}рочной дисперсии$\dotfill$&4&73\\
\textbf{Шестаков~О.\,В.} см.~Маркин~А.\,В.&&\\
\textbf{Шоргин~С.\,Я.} см.~Зейфман~А.\,И.&&\\
\textbf{Шоргин~С.\,Я.} см.~Кудрявцев~А.\,А.&&\\
\end{tabular}
}

%\thispagestyle{myheadings}
\def\leftfootline{\small{\textbf{\thepage}
\hfill ИНФОРМАТИКА И ЕЁ ПРИМЕНЕНИЯ\ \ \ том~4\ \ \ выпуск~4\ \ \ 2010}
}%
 \def\rightfootline{\small{ИНФОРМАТИКА И ЕЁ ПРИМЕНЕНИЯ\ \ \ том~4\ \ \ выпуск~4\ \ \ 2010
 \hfill \textbf{\thepage}}}
 \label{end\stat}


%Том 10 Выпуск 1-4 Год 2016

\def\stat{cont-e}
{%\hrule\par
%\vskip 7pt % 7pt
\raggedleft\Large \bf%\baselineskip=3.2ex
2\,0\,1\,6\ \ A\,U\,T\,H\,O\,R\ \ I\,N\,D\,E\,X \vskip 17pt
 \hrule
 \par
\vskip 21pt plus 6pt minus 3pt }

\label{st\stat}

\def\tit{\ }

\def\aut{\ }
\def\auf{\ }

\def\leftkol{\ } %2016 AUTHOR INDEX} % ENGLISH ABSTRACTS}

\def\rightkol{\ } %2016 AUTHOR INDEX} %ENGLISH ABSTRACTS}

\titele{\tit}{\aut}{\auf}{\leftkol}{\rightkol}

\def\leftfootline{\small{\textbf{\thepage}
\hfill INFORMATIKA I EE PRIMENENIYA~--- INFORMATICS AND APPLICATIONS\ \ \ 2016\
\ \ volume~10\ \ \ issue\ 4}
}%
 \def\rightfootline{\small{INFORMATIKA I EE PRIMENENIYA~--- INFORMATICS AND APPLICATIONS\ \ \ 2016\ \ \ volume~10\ \ \ issue\ 4
\hfill \textbf{\thepage}}}

\vspace*{-12pt}
\vspace*{-18pt}

{\tabcolsep=2.8pt
\begin{tabular}{p{382pt}cc}
&\textbf{Issue} & \textbf{Page}\\[6pt]
\Avtors{Agalarov~M.\,Ya.} see~Agalarov~Ya.\,M.&&\\
\Avtors{Agalarov~Ya.\,M., Agalarov~M.\,Ya., and
Shorgin~V.\,S.} About the optimal threshold of queue\linebreak
\\[-12pt]
\hspace*{23pt}length in a~particular problem of profit maximization
in the $M/G/1$ queuing system&2&70--79\\
\Avtors{Alexeyevsky~D.\,A.} BioNLP ontology extraction from 
a~restricted language corpus with\linebreak
\\[-12pt]
\hspace*{23pt}context-free grammars&1&119--128\\
\Avtors{Andreev~S.\,D.} see~Gaidamaka~Yu.\,V.&&\\
\Avtors{Andreev~S.\,D.} see~Ometov~A.\,Ya.&&\\
\Avtors{Arkhipov~O.\,P., Arkhipov~P.\,O., and Sidorkin~I.\,I.} The
option to create a~local coordinate\linebreak
\\[-12pt]
\hspace*{23pt}system for synchronization of selected images&3&91--97\\
\Avtors{Arkhipov~P.\,O.} see~Arkhipov~O.\,P.&&\\
\Avtors{Belousov~V.\,V.} see~Shnurkov~P.\,V.&&\\
\Avtors{Belousov~V.\,V.} see~Shnurkov~P.\,V.&&\\
\Avtors{Bening~V.\,E.} Calculation of~the~asymptotic deficiency
of~some statistical procedures based\linebreak
\\[-12pt]
\hspace*{23pt}on~samples with~random sizes&4&34--45\\
\Avtors{Borisov~A.\,V., Bosov~A.\,V., and Miller~G.\,B.} Modeling and
monitoring of VoIP connection&2&\hphantom{1}2--13\\
\Avtors{Bosov~A.\,V.} see~Borisov~A.\,V.&&\\
\Avtors{Briukhov~D.\,O.} see~Stupnikov~S.\,A.&&\\
\Avtors{Callaos~N.\,K.\ and Seyful-Mulyukov~R.\,B.} Complexity and
its information content&1&129--139\\
\Avtors{Chertok~A.\,V., Kadaner~A.\,I., Khazeeva~G.\,T., and
Sokolov~I.\,A.} Regime switching detection\linebreak
\\[-12pt]
\hspace*{23pt}for~the~Levy driven
Ornstein--Uhlenbeck process using CUSUM methods&4&46--56\\
\Avtors{Chichagov~V.\,V.} Asymptotic expansions of mean absolute
error of uniformly minimum variance unbiased and maximum likelihood
estimators on the one-parameter exponential\linebreak
\\[-12pt]
\hspace*{23pt}family model of lattice distributions&3&66--76\\
\Avtors{Danishevsky~V.\,I.} see~Kolesnikov A.\,V.&&\\
\Avtors{Fazliev~A.\,Z.} see~Kalinichenko~L.\,A.&&\\
\Avtors{Fedoseev~A.\,A.} What is behind the concept of ``knowledge in
small packages''&3&105--110\\
\Avtors{Gaidamaka~Yu.\,V., Andreev~S.\,D., Sopin~E.\,S.,
Samouylov~K.\,E., and Shorgin~S.\,Ya.} Interference analysis
of~the~device-to-device communications model with~regard to~a~signal\linebreak
\\[-12pt]
\hspace*{23pt}propagation environment&4&\hphantom{1}2--10\\
\Avtors{Gasilov~A.\,V.} see~Yakovlev~O.\,A.&&\\
\Avtors{Goncharov~A.\,V.\ and Strijov~V.\,V.} Metric time series
classification using weighted dynamic\linebreak
\\[-12pt]
\hspace*{23pt}warping relative to centroids of classes&2&36--47\\
\Avtors{Gordov~E.\,P.} see~Kalinichenko~L.\,A.&&\\
\Avtors{Gorshenin~A.\,K.} Concept of online service for stochastic
modeling of real processes&1&72--81\\
\Avtors{Gorshenin~A.\,K.} see~Shnurkov~P.\,V.&&\\
\Avtors{Gorshenin~A.\,K.} see~Shnurkov~P.\,V.&&\\
\Avtors{Grusho~A.\,A., Grusho~N.\,A., Zabezhailo~M.\,I., and
Timonina~E.\,E.} Integration of statistical and\linebreak
\\[-12pt]
\hspace*{23pt}deterministic methods for
analysis of information security&3&2--8\\
\Avtors{Grusho~A.\,A., Zabezhailo~M.\,I., and Zatsarinny~A.\,A.} On
the advanced procedure to reduce\linebreak
\\[-12pt]
\hspace*{23pt}calculation of Galois closures&4&\hphantom{1}96--104\\
\Avtors{Grusho~N.\,A.} see~Grusho~A.\,A.&&\\
\Avtors{Havanskov~V.\,A.} see~Minin~V.\,A.&&\\
\Avtors{Inkova~O.\,Yu.} see~Zatsman~I.\,M.&&\\
\Avtors{Isachenko~R.\,V.\ and Strijov~V.\,V.} Metric learning in
multiclass time series classification\linebreak
\\[-12pt]
\hspace*{23pt}problem&2&48--57\\
\end{tabular}
}
\pagebreak

\def\leftfootline{\small{\textbf{\thepage}
\hfill INFORMATIKA I EE PRIMENENIYA~--- INFORMATICS AND APPLICATIONS\ \ \ 2016\
\ \ volume~10\ \ \ issue\ 4}
}%
 \def\rightfootline{\small{INFORMATIKA I EE PRIMENENIYA~---
INFORMATICS AND APPLICATIONS\ \ \ 2016\ \ \ volume~10\ \ \ issue\ 4
\hfill \textbf{\thepage}}}

\def\leftkol{2016 AUTHOR INDEX} % ENGLISH ABSTRACTS}

\def\rightkol{2016 AUTHOR INDEX} %ENGLISH ABSTRACTS}


{\tabcolsep=2.83pt
\begin{tabular}{p{382pt}cc}
&\textbf{Issue} & \textbf{Page}\\[6pt]
\Avtors{Kadaner~A.\,I.} see~Chertok~A.\,V.&&\\[.255pt]
\Avtors{Kalinichenko~L.\,A., Volnova~A.\,A., Gordov~E.\,P.,
Kiselyova~N.\,N., Kovaleva~D.\,A., Malkov~O.\,Yu., Okladnikov~I.\,G.,
Podkolodnyy~N.\,L., Pozanenko~A.\,S., Ponomareva~N.\,V.,
Stupnikov~S.\,A.,} \textbf{and Fazliev~A.\,Z.} Data access challenges for data
intensive\linebreak
\\[-12pt]
\hspace*{23pt}research in Russia&1& 2--22\\[.255pt]
\Avtors{Karasikov~M.\,E.\ and Strijov~V.\,V.} Feature-based
time-series classification&4&121--131\\[.255pt]
\Avtors{Khazeeva~G.\,T.} see~Chertok~A.\,V.&&\\[.255pt]
\Avtors{Khokhlov~Yu.\,S.} Multivariate fractional Levy motion and its
applications&2&\hphantom{1}98--106\\[.255pt]
\Avtors{Kirikov~I.\,A., Kolesnikov~A.\,V., Listopad~S.\,V., and
Rumovskaya~S.\,B.} Fine-grained hybrid\linebreak
\\[-12pt]
\hspace*{23pt}intelligent systems. Part 2:
Bidirectional hybridization&1&\hphantom{1}96--105\\[.255pt]
\Avtors{Kirikov~I.\,A., Kolesnikov~A.\,V., Listopad~S.\,V., and
Rumovskaya~S.\,B.} ``Virtual council''~---\linebreak
\\[-12pt]
\hspace*{23pt}source environment
supporting complex diagnostic decision making&3&81--90\\[.255pt]
\Avtors{Kiselyova~N.\,N.} see~Kalinichenko~L.\,A.&&\\[.255pt]
\Avtors{Kolesnikov A.\,V., Listopad~S.\,V., Rumovskaya~S.\,B., and
Danishevsky~V.\,I.} Informal axiomatic\linebreak
\\[-12pt]
\hspace*{23pt}theory of~the~role visual models&4&114--120\\[.255pt]
\Avtors{Kolesnikov~A.\,V.} see~Kirikov~I.\,A.&&\\[.255pt]
\Avtors{Kolesnikov~A.\,V.} see~Kirikov~I.\,A.&&\\[.255pt]
\Avtors{Kolin~K.\,K.} Humanitarian aspects of information
security&3&111--121\\[.255pt]
\Avtors{Konovalov~M.\,G.\ and Razumchik~R.\,V.} Dispatching
to~two parallel nonobservable queues using\linebreak
\\[-12pt]
\hspace*{23pt}only static
information&4&57--67\\[.255pt]
\Avtors{Korchagin~A.\,Yu.} see~Korolev~V.\,Yu.&&\\[.255pt]
\Avtors{Korchagin~A.\,Yu.} see~Korolev~V.\,Yu.&&\\[.255pt]
\Avtors{Korepanov~E.\,R.} see~Sinitsyn~I.\,N.&&\\[.255pt]
\Avtors{Korepanov~E.\,R.} see~Sinitsyn~I.\,N.&&\\[.255pt]
\Avtors{Korolev~V.\,Yu., Korchagin~A.\,Yu., and Zeifman~A.\,I.} The
Poisson theorem for Bernoulli trials\linebreak
\\[-12pt]
\hspace*{23pt}with~a~random probability
of~success and~a~discrete analog of~the~Weibull distribution&4&11--20\\[.255pt]
\Avtors{Korolev~V.\,Yu., Zeifman~A.\,I., and Korchagin~A.\,Yu.}
Asymmetric Linnik distributions as~limit\linebreak
\\[-12pt]
\hspace*{23pt}laws for~random sums
of~independent random variables with~finite variances&4&21--33\\[.255pt]
\Avtors{Koucheryavy~E.\,A.} see~Ometov~A.\,Ya.&&\\[.255pt]
\Avtors{Kovaleva~D.\,A.} see~Kalinichenko~L.\,A.&&\\[.255pt]
\Avtors{Kovalyov~S.\,P.} Metaprogramming to increase
manufacturability of large-scale software-\linebreak
\\[-12pt]
\hspace*{23pt}intensive systems&1&56--66\\[.255pt]
\Avtors{Krivenko~M.\,P.} Significance tests of feature selection for
classification&3&32--40\\[.255pt]
\Avtors{Kruzhkov~M.\,G.} see~Zalizniak~Anna~A.&&\\[.255pt]
\Avtors{Kruzhkov~M.\,G.} see~Zatsman~I.\,M.&&\\[.255pt]
\Avtors{Kudryavtsev~A.\,A.} Bayesian queueing and reliability models:
\textit{A~priori} distributions with\linebreak
\\[-12pt]
\hspace*{23pt}compact support&1&67--71\\[.255pt]
\Avtors{Kudryavtsev~A.\,A.} Characteristics dependent on the balance
coefficient in Bayesian models\linebreak
\\[-12pt]
\hspace*{23pt}with compact support of \textit{a priori}
distributions&3&77--80\\[.255pt]
\Avtors{Kudryavtsev~A.\,A.\ and Palionnaia~S.\,I.} Bayesian recurrent
model of reliability growth:\linebreak
\\[-12pt]
\hspace*{23pt}Parabolic distribution of parameters&2&80--83\\[.255pt]
\Avtors{Kudryavtsev~A.\,A.\ and Titova~A.\,I.} Bayesian queuing
and~reliability models: Degenerate-\linebreak
\\[-12pt]
\hspace*{23pt}Weibull case&4&68--71\\[.255pt]
\Avtors{Leontyev~N.\,D.\ and Ushakov~V.\,G.} Analysis of a queueing
system with autoregressive arrivals\linebreak
\\[-12pt]
\hspace*{23pt}and nonpreemptive priority&3&15--22\\[.255pt]
\Avtors{Listopad~S.\,V.} see~Kirikov~I.\,A.&&\\[.255pt]
\Avtors{Listopad~S.\,V.} see~Kirikov~I.\,A.&&\\[.255pt]
\Avtors{Listopad~S.\,V.} see~Kolesnikov A.\,V.&&\\[.255pt]
\Avtors{Malkov~O.\,Yu.} see~Kalinichenko~L.\,A.&&\\[.255pt]
\Avtors{Markov~A.\,S., Monakhov~M.\,M., and
Ulyanov~V.\,V.} Generalized Cornish--Fisher expansions\linebreak
\\[-12pt]
\hspace*{23pt}for distributions of statistics based on samples
of random size&2&84--91\\[.255pt]
\Avtors{Melnikov~A.\,K.\ and Ronzhin~A.\,F.} Generalized statistical
method of~text analysis based\linebreak
\\[-12pt]
\hspace*{23pt}on~calculation of~probability distributions
of~statistical values&4&89--95\\
\end{tabular}
}
\pagebreak

\def\leftfootline{\small{\textbf{\thepage}
\hfill INFORMATIKA I EE PRIMENENIYA~--- INFORMATICS AND APPLICATIONS\ \ \ 2016\
\ \ volume~10\ \ \ issue\ 4}
}%
 \def\rightfootline{\small{INFORMATIKA I EE PRIMENENIYA~---
INFORMATICS AND APPLICATIONS\ \ \ 2016\ \ \ volume~10\ \ \ issue\ 4
\hfill \textbf{\thepage}}}

\def\leftkol{2016 AUTHOR INDEX} % ENGLISH ABSTRACTS}

\def\rightkol{2016 AUTHOR INDEX} %ENGLISH ABSTRACTS}


{\tabcolsep=3pt
\begin{tabular}{p{381pt}cc}
&\textbf{Issue} & \textbf{Page}\\[6pt]
\Avtors{Meykhanadzhyan~L.\,A.} Stationary characteristics of the finite
capacity queueing system with\linebreak
\\[-12pt]
\hspace*{23pt}inverse service order and generalized
probabilistic priority&2&123--131\\[.23pt]
\Avtors{Miller~G.\,B.} see~Borisov~A.\,V.&&\\[.23pt]
\Avtors{Minin~V.\,A., Zatsman~I.\,M., Havanskov~V.\,A., and
Shubnikov~S.\,K.} Intensity of citation of scientific publications in
inventions on information and computer technologies patented\linebreak
\\[-12pt]
\hspace*{23pt}in Russia by domestic and foreign applicants&2&107--122\\[.23pt]
\Avtors{Monakhov~M.\,M.} see~Markov~A.\,S.&&\\[.23pt]
\Avtors{Naumov~V.\,A.\ and Samouylov~K.\,E.} On relationship
between queuing systems with resources\linebreak
\\[-12pt]
\hspace*{23pt}and Erlang networks&3&\hphantom{1}9--14\\[.23pt]
\Avtors{Okladnikov~I.\,G.} see~Kalinichenko~L.\,A.&&\\[.23pt]
\Avtors{Ometov~A.\,Ya., Andreev~S.\,D., Turlikov~A.\,M., and
Koucheryavy~E.\,A.} Performance analysis of\linebreak
\\[-12pt]
\hspace*{23pt}a wireless data
aggregation system with contention for contemporary sensor
networks&3&23--31\\[.23pt]
\Avtors{Palionnaia~S.\,I.} see~Kudryavtsev~A.\,A.&&\\[.23pt]
\Avtors{Podkolodnyy~N.\,L.} see~Kalinichenko~L.\,A.&&\\[.23pt]
\Avtors{Ponomareva~N.\,V.} see~Kalinichenko~L.\,A.&&\\[.23pt]
\Avtors{Popkova~N.\,A.} see~Zatsman~I.\,M.&&\\[.23pt]
\Avtors{Pozanenko~A.\,S.} see~Kalinichenko~L.\,A.&&\\[.23pt]
\Avtors{Razumchik~R.\,V.} see~Konovalov~M.\,G.&&\\[.23pt]
\Avtors{Ronzhin~A.\,F.} see~Melnikov~A.\,K.&&\\[.23pt]
\Avtors{Rumovskaya~S.\,B.} see~Kirikov~I.\,A.&&\\[.23pt]
\Avtors{Rumovskaya~S.\,B.} see~Kirikov~I.\,A.&&\\[.23pt]
\Avtors{Rumovskaya~S.\,B.} see~Kolesnikov A.\,V.&&\\[.23pt]
\Avtors{Samouylov~K.\,E.} see~Gaidamaka~Yu.\,V.&&\\[.23pt]
\Avtors{Samouylov~K.\,E.} see~Naumov~V.\,A.&&\\[.23pt]
\Avtors{Serebryanskii~S.\,M.} see~Tyrsin~A.\,N.&&\\[.23pt]
\Avtors{Seyful-Mulyukov~R.\,B.} see~Callaos~N.\,K.&&\\[.23pt]
\Avtors{Shestakov~O.\,V.} Statistical properties of the denoising method
based on the stabilized hard\linebreak
\\[-12pt]
\hspace*{23pt}thresholding&2&65--69\\[.23pt]
\Avtors{Shestakov~O.\,V.} The strong law of large numbers for the risk
estimate in the problem of\linebreak
\\[-12pt]
\hspace*{23pt}tomographic image reconstruction from
projections with a correlated noise&3&41--45\\[.23pt]
\Avtors{Shestakov~O.\,V.} see~Zakharova~T.\,V.&&\\[.23pt]
\Avtors{Shnurkov~P.\,V., Gorshenin~A.\,K., and Belousov~V.\,V.}
Analytical solution of~the~optimal control\linebreak
\\[-12pt]
\hspace*{23pt}task of~a~semi-Markov
process with~finite set of~states&4&72--88\\[.23pt]
\Avtors{Shnurkov~P.\,V., Zasypko~V.\,V., Belousov~V.\,V., and
Gorshenin~A.\,K.} Development of the algorithm of numerical solution
of the optimal investment control problem\linebreak
\\[-12pt]
\hspace*{23pt}in the closed dynamical model of three-sector economy&1&82--95\\[.23pt]
\Avtors{Shorgin~S.\,Ya.} see~Gaidamaka~Yu.\,V.&&\\[.23pt]
\Avtors{Shorgin~V.\,S.} see~Agalarov~Ya.\,M.&&\\[.23pt]
\Avtors{Shubnikov~S.\,K.} see~Minin~V.\,A.&&\\[.23pt]
\Avtors{Sidorkin~I.\,I.} see~Arkhipov~O.\,P.&&\\[.23pt]
\Avtors{Sinitsyn~I.\,N.} Analytical modeling of processes in stochastic
systems with complex fractional\linebreak
\\[-12pt]
\hspace*{23pt}order Bessel nonlinearities&3&55--65\\[.23pt]
\Avtors{Sinitsyn~I.\,N.} Orthogonal supoptimal filters for nonlinear
stochastic systems on manifolds&1&34--44\\[.23pt]
\Avtors{Sinitsyn~I.\,N.\ and Korepanov~E.\,R.} Normal Pugachev
conditionally-optimal filters and extra-\linebreak
\\[-12pt]
\hspace*{23pt}polators for state linear stochastic systems&2&14--23\\[.23pt]
\Avtors{Sinitsyn~I.\,N.\ and Sinitsyn~V.\,I.} Analytical modeling of
distributions in stochastic systems on\linebreak
\\[-12pt]
\hspace*{23pt}manifolds based on ellipsoidal approximation&1&45--55\\[.23pt]
\Avtors{Sinitsyn~I.\,N., Sinitsyn~V.\,I., and
Korepanov~E.\,R.} Ellipsoidal suboptimal filters for nonlinear\linebreak
\\[-12pt]
\hspace*{23pt}stochastic systems on manifolds&2&24--35\\[.23pt]
\Avtors{Sinitsyn~V.\,I.} see~Sinitsyn~I.\,N.&&\\[.23pt]
\Avtors{Sinitsyn~V.\,I.} see~Sinitsyn~I.\,N.&&\\[.23pt]
\Avtors{Skvortsov~N.\,A.} see~Stupnikov~S.\,A.&&\\[.23pt]
\Avtors{Sokolov~I.\,A.} see~Chertok~A.\,V.&&\\
\end{tabular}
}
\pagebreak

\def\leftfootline{\small{\textbf{\thepage}
\hfill INFORMATIKA I EE PRIMENENIYA~--- INFORMATICS AND APPLICATIONS\ \ \ 2016\
\ \ volume~10\ \ \ issue\ 4}
}%
 \def\rightfootline{\small{INFORMATIKA I EE PRIMENENIYA~---
INFORMATICS AND APPLICATIONS\ \ \ 2016\ \ \ volume~10\ \ \ issue\ 4
\hfill \textbf{\thepage}}}

\def\leftkol{2016 AUTHOR INDEX} % ENGLISH ABSTRACTS}

\def\rightkol{2016 AUTHOR INDEX} %ENGLISH ABSTRACTS}


{\tabcolsep=3pt
\begin{tabular}{p{382pt}cc}
&\textbf{Issue} & \textbf{Page}\\[6pt]
\Avtors{Sopin~E.\,S.} see~Gaidamaka~Yu.\,V.&&\\
\Avtors{Strijov~V.\,V.} see~Goncharov~A.\,V.&&\\
\Avtors{Strijov~V.\,V.} see~Isachenko~R.\,V.&&\\
\Avtors{Strijov~V.\,V.} see~Karasikov~M.\,E.&&\\
\Avtors{Stupnikov~S.\,A., Briukhov~D.\,O., and Skvortsov~N.\,A.}
Co-lending systemic risk analysis over\linebreak
\\[-12pt]
\hspace*{23pt}heterogeneous data collections&1&23--33\\
\Avtors{Stupnikov~S.\,A.} see~Kalinichenko~L.\,A.&&\\
\Avtors{Suchkov~A.\,P.} see~Zatsarinny~A.\,A.&&\\
\Avtors{Timonina~E.\,E.} see~Grusho~A.\,A.&&\\
\Avtors{Titova~A.\,I.} see~Kudryavtsev~A.\,A.&&\\
\Avtors{Turlikov~A.\,M.} see~Ometov~A.\,Ya.&&\\
\Avtors{Tyrsin~A.\,N.\ and Serebryanskii~S.\,M.} Recognition of
dependences on the basis of inverse\linebreak
\\[-12pt]
\hspace*{23pt}mapping&2&58--64\\
\Avtors{Ulyanov~V.\,V.} see~Markov~A.\,S.&&\\
\Avtors{Ushakov~V.\,G.} Queueing system with working vacations and
hyperexponential input stream&2&92--97\\
\Avtors{Ushakov~V.\,G.} see~Leontyev~N.\,D.&&\\
\Avtors{Volnova~A.\,A.} see~Kalinichenko~L.\,A.&&\\
\Avtors{Yakovlev~O.\,A.\ and Gasilov~A.\,V.} Speeded-up stereo
matching using geodesic support weights&3&\hphantom{1}98--104\\
\Avtors{Zabezhailo~M.\,I.} see~Grusho~A.\,A.&&\\
\Avtors{Zabezhailo~M.\,I.} see~Grusho~A.\,A.&&\\
\Avtors{Zakharova~T.\,V.\ and Shestakov~O.\,V.} Precision analysis of
wavelet processing of aerodynamic\linebreak
\\[-12pt]
\hspace*{23pt}flow patterns&3&46--54\\
\Avtors{Zalizniak~Anna~A.\ and Kruzhkov~M.\,G.} Database
of~Russian impersonal verbal constructions&4&132--141\\
\Avtors{Zasypko~V.\,V.} see~Shnurkov~P.\,V.&&\\
\Avtors{Zatsarinny~A.\,A.\ and Suchkov~A.\,P.} Systems engineering
approaches to~the~establishment of\linebreak
\\[-12pt]
\hspace*{23pt}a~system for~decision support based
on~situational analysis&4&105--113\\
\Avtors{Zatsarinny~A.\,A.} see~Grusho~A.\,A.&&\\
\Avtors{Zatsman~I.\,M., Inkova~O.\,Yu., Kruzhkov~M.\,G., and
Popkova~N.\,A.} Representation of cross-\linebreak
\\[-12pt]
\hspace*{23pt}lingual knowledge about
connectors in supracorpora databases&1&106--118\\
\Avtors{Zatsman~I.\,M.} see~Minin~V.\,A.&&\\
\Avtors{Zeifman~A.\,I.} see~Korolev~V.\,Yu.&&\\
\Avtors{Zeifman~A.\,I.} see~Korolev~V.\,Yu.&&\\
\end{tabular}
}

%\thispagestyle{myheadings}
\def\leftfootline{\small{\textbf{\thepage}
\hfill INFORMATIKA I EE PRIMENENIYA~--- INFORMATICS AND APPLICATIONS\ \ \ 2016\
\ \ volume~10\ \ \ issue\ 4}
}%
 \def\rightfootline{\small{INFORMATIKA I EE PRIMENENIYA~---
INFORMATICS AND APPLICATIONS\ \ \ 2016\ \ \ volume~10\ \ \ issue\ 4
\hfill \textbf{\thepage}}}

 \label{end\stat}

\newpage



%\def\stat{cont}
{%\hrule\par
%\vskip 7pt % 7pt
\raggedleft\Large \bf%\baselineskip=3.2ex
А\,В\,Т\,О\,Р\,С\,К\,И\,Й\ \ У\,К\,А\,З\,А\,Т\,Е\,Л\,Ь\ \ З\,А\ \ 2\,0\,0\,7 г. \vskip 17pt
    \hrule
    \par
\vskip 21pt plus 6pt minus 3pt }

\label{st\stat}

\def\tit{\ }

\def\aut{\ }
\def\auf{\ }

\def\leftkol{\ } % ENGLISH ABSTRACTS}

\def\rightkol{\ } %ENGLISH ABSTRACTS}

\titele{\tit}{\aut}{\auf}{\leftkol}{\rightkol}


\contentsline {chapter}{\ }{Выпуск \quad Стр.} 
\contentsline {section}{\textbf{Батракова Д.\,А., Королев В.\,Ю., Шоргин С.\,Я.}\ \ Новый метод вероятностно-ста\-ти\-сти\-че\-ско\-го анализа информационных потоков в\nobreakspace {}телекоммуникационных сетях}{\qquad 1 \qquad 40} 
\contentsline {section}{\textbf{Борисов А.\,В.}\ \ Байесовское оценивание в системах наблюдения с\nobreakspace {}марковскими скачкообразными процессами: игровой подход}{\qquad 2 \qquad 65}
\contentsline {section}{\textbf{Босов А.\,В., Иванов А.\,В.}\ \ Программная инфраструктура информационного Web-пор\-тала}{\qquad 2 \qquad 50}
\contentsline {section}{\textbf{Захаров В.\,Н., Калиниченко Л.\,А., Соколов И.\,А., Ступников С.\,А.}\ \ Конструирование канонических информационных моделей для интегрированных информационных систем}{\qquad 2 \qquad 15}
\contentsline {section}{\textbf{Захаров В.\,Н., Козмидиади В.\,А.}\ \ Средства обеспечения отказоустойчивости при\-ло\-жений}{\qquad 1 \qquad 14} 
\contentsline {section}{\textbf{Иванов А.\,В.}\ \ см. Босов А.\,В.\hfill\hfill\hfill\hfill\hfill\hfill\hfill\hfill\hfill\hfill\hfill\hfill\hfill\hfill\hfill\hfill\hfill\hfill\hfill\hfill\hfill\hfill\hfill\hfill\hfill\hfill\hfill\hfill\hfill\hfill\hfill\hfill\hfill\hfill\hfill}{\ }
\contentsline {section}{\textbf{Ильин В.\,Д., Соколов И.\,А.}\ \ Символьная модель системы знаний информатики в\nobreakspace {}че\-ло\-ве\-ко-автоматной среде}{\qquad 1 \qquad 66} 
\contentsline {section}{\textbf{Калиниченко Л.\,А.}\ \ см. Захаров В.\,Н.\hfill\hfill\hfill\hfill\hfill\hfill\hfill\hfill\hfill\hfill\hfill\hfill\hfill\hfill\hfill\hfill\hfill\hfill\hfill\hfill\hfill\hfill\hfill\hfill\hfill\hfill\hfill\hfill\hfill\hfill\hfill\hfill\hfill\hfill\hfill}{\ }
\contentsline {section}{\textbf{Козеренко Е.\,Б.}\ \ Лингвистическое моделирование для систем машинного перевода и обработки знаний}{\qquad 1 \qquad 54} 
\contentsline {section}{\textbf{Козмидиади В.\,А.}\ \ см. Захаров В.\,Н.\hfill\hfill\hfill\hfill\hfill\hfill\hfill\hfill\hfill\hfill\hfill\hfill\hfill\hfill\hfill\hfill\hfill\hfill\hfill\hfill\hfill\hfill\hfill\hfill\hfill\hfill\hfill\hfill\hfill\hfill\hfill\hfill\hfill\hfill\hfill }{\ } 
\contentsline {section}{\textbf{Королев В.\,Ю.}\ \ см. Батракова Д.\,А.\hfill\hfill\hfill\hfill\hfill\hfill\hfill\hfill\hfill\hfill\hfill\hfill\hfill\hfill\hfill\hfill\hfill\hfill\hfill\hfill\hfill\hfill\hfill\hfill\hfill\hfill\hfill\hfill\hfill\hfill\hfill\hfill\hfill\hfill\hfill}{\ } 
\contentsline {section}{\textbf{Кудрявцев А.\,А., Шоргин С.\,Я.}\ \ Байесовский подход к\nobreakspace {}анализу систем массового обслуживания и\nobreakspace {}показателей надежности}{\qquad 2 \qquad 76}
\contentsline {section}{\textbf{Печинкин А.\,В., Соколов И.\,А., Чаплыгин В.\,В.}\ \ Многолинейная система массового обслуживания с конечным накопителем и ненадежными приборами}{\qquad 1 \qquad 27} 
\contentsline {section}{\textbf{Печинкин А.\,В., Соколов И.\,А., Чаплыгин В.\,В.}\ \ Стационарные характеристики многолинейной\nobreakspace {}системы массового обслуживания с\nobreakspace {}одновременными отказами приборов}{\qquad 2 \qquad 39}
\contentsline {section}{\textbf{Синицын И.\,Н.}\ \ Корреляционные методы построения аналитических информационных моделей флуктуаций полюса Земли по априорным данным}{\qquad 2 \qquad \hphantom{9}2}
\contentsline {section}{\textbf{Синицын И.\,Н.}\ \ Развитие теории фильтров Пугачева для оперативной обработки информации в стохастических системах}{{\qquad 1 \qquad \hphantom{9}3}} 
\contentsline {section}{\textbf{Соколов И.\,А.}\ \ см. Захаров В.\,Н.\hfill\hfill\hfill\hfill\hfill\hfill\hfill\hfill\hfill\hfill\hfill\hfill\hfill\hfill\hfill\hfill\hfill\hfill\hfill\hfill\hfill\hfill\hfill\hfill\hfill\hfill\hfill\hfill\hfill\hfill\hfill\hfill\hfill\hfill\hfill}{\ }
\contentsline {section}{\textbf{Соколов И.\,А.}\ \ см. Ильин В.\,Д.\hfill\hfill\hfill\hfill\hfill\hfill\hfill\hfill\hfill\hfill\hfill\hfill\hfill\hfill\hfill\hfill\hfill\hfill\hfill\hfill\hfill\hfill\hfill\hfill\hfill\hfill\hfill\hfill\hfill\hfill\hfill\hfill\hfill\hfill\hfill}{\ } 
\contentsline {section}{\textbf{Соколов И.\,А.}\ \ см. Печинкин А.\,В.\hfill\hfill\hfill\hfill\hfill\hfill\hfill\hfill\hfill\hfill\hfill\hfill\hfill\hfill\hfill\hfill\hfill\hfill\hfill\hfill\hfill\hfill\hfill\hfill\hfill\hfill\hfill\hfill\hfill\hfill\hfill\hfill\hfill\hfill\hfill}{\ } 
\contentsline {section}{\textbf{Соколов И.\,А.}\ \ см. Печинкин А.\,В.\hfill\hfill\hfill\hfill\hfill\hfill\hfill\hfill\hfill\hfill\hfill\hfill\hfill\hfill\hfill\hfill\hfill\hfill\hfill\hfill\hfill\hfill\hfill\hfill\hfill\hfill\hfill\hfill\hfill\hfill\hfill\hfill\hfill\hfill\hfill}{\ }
\contentsline {section}{\textbf{Ступников С.\,А.}\ \ см. Захаров В.\,Н.\hfill\hfill\hfill\hfill\hfill\hfill\hfill\hfill\hfill\hfill\hfill\hfill\hfill\hfill\hfill\hfill\hfill\hfill\hfill\hfill\hfill\hfill\hfill\hfill\hfill\hfill\hfill\hfill\hfill\hfill\hfill\hfill\hfill\hfill\hfill}{\ }
\contentsline {section}{\textbf{Чаплыгин В.\,В.}\ \ см. Печинкин А.\,В.\hfill\hfill\hfill\hfill\hfill\hfill\hfill\hfill\hfill\hfill\hfill\hfill\hfill\hfill\hfill\hfill\hfill\hfill\hfill\hfill\hfill\hfill\hfill\hfill\hfill\hfill\hfill\hfill\hfill\hfill\hfill\hfill\hfill\hfill\hfill}{\ } 
\contentsline {section}{\textbf{Чаплыгин В.\,В.}\ \ см. Печинкин А.\,В.\hfill\hfill\hfill\hfill\hfill\hfill\hfill\hfill\hfill\hfill\hfill\hfill\hfill\hfill\hfill\hfill\hfill\hfill\hfill\hfill\hfill\hfill\hfill\hfill\hfill\hfill\hfill\hfill\hfill\hfill\hfill\hfill\hfill\hfill\hfill}{\ }
\contentsline {section}{\textbf{Шоргин С.\,Я.}\ \ см. Батракова Д.\,А.\hfill\hfill\hfill\hfill\hfill\hfill\hfill\hfill\hfill\hfill\hfill\hfill\hfill\hfill\hfill\hfill\hfill\hfill\hfill\hfill\hfill\hfill\hfill\hfill\hfill\hfill\hfill\hfill\hfill\hfill\hfill\hfill\hfill\hfill\hfill}{\ } 
\contentsline {section}{\textbf{Шоргин С.\,Я.}\ \ см. Кудрявцев А.\,А.\hfill\hfill\hfill\hfill\hfill\hfill\hfill\hfill\hfill\hfill\hfill\hfill\hfill\hfill\hfill\hfill\hfill\hfill\hfill\hfill\hfill\hfill\hfill\hfill\hfill\hfill\hfill\hfill\hfill\hfill\hfill\hfill\hfill\hfill\hfill}{\ }
%\thispagestyle{myheadings}
\def\leftfootline{\small{\textbf{\thepage}
\hfill ИНФОРМАТИКА И ЕЁ ПРИМЕНЕНИЯ\ \ \ том~1\ \ \ выпуск~2\ \ \ 2007}
}%
 \def\rightfootline{\small{ИНФОРМАТИКА И ЕЁ ПРИМЕНЕНИЯ\ \ \ том~1\ \ \ выпуск~2\ \ \ 2007
 \hfill \textbf{\thepage}}}
 \label{end\stat}

%\def\stat{cont-e}
{%\hrule\par
%\vskip 7pt % 7pt
\raggedleft\Large \bf%\baselineskip=3.2ex
2\,0\,0\,7\ \ A\,U\,T\,H\,O\,R\ \ I\,N\,D\,E\,X \vskip 17pt
    \hrule
    \par
\vskip 21pt plus 6pt minus 3pt }

\label{st\stat}

\def\tit{\ }

\def\aut{\ }
\def\auf{\ }

\def\leftkol{\ } % ENGLISH ABSTRACTS}

\def\rightkol{\ } %ENGLISH ABSTRACTS}

\titele{\tit}{\aut}{\auf}{\leftkol}{\rightkol}


\contentsline {chapter}{\ }{Issue \quad Page} 
\contentsline {subsection}{\textbf{Batrakova D.\,A., Korolev V.\,Yu., Shorgin S.\,Ya.}\ \ A New Method for the Probabilistic and Statistical Analysis of Information Flows in Telecommunication Networks}{\qquad 1 \qquad 40} 
\contentsline {subsection}{\textbf{Borisov A.\,V.}\ \ Bayesian Estimation in\nobreakspace {}Observation Systems with\nobreakspace {}Markov Jump Processes: Game-Theoretic Approach}{\qquad 2 \qquad 65} 
\contentsline {subsection}{\textbf{Bosov A.\,V., Ivanov A.\,V.}\ \ Linguistic Simulation for Machine Translation and Knowledge Management Systems}{\qquad 2 \qquad 50} 
\contentsline {subsection}{\textbf{Chaplygin V.\,V.} see Pechinkin A.\,V.\hfill\hfill\hfill\hfill\hfill\hfill\hfill\hfill\hfill\hfill\hfill\hfill\hfill\hfill\hfill\hfill\hfill\hfill\hfill\hfill\hfill\hfill\hfill\hfill\hfill\hfill\hfill\hfill\hfill\hfill\hfill\hfill\hfill\hfill\hfill}{\ }
\contentsline {subsection}{\textbf{Chaplygin V.\,V.} see Pechinkin A.\,V.\hfill\hfill\hfill\hfill\hfill\hfill\hfill\hfill\hfill\hfill\hfill\hfill\hfill\hfill\hfill\hfill\hfill\hfill\hfill\hfill\hfill\hfill\hfill\hfill\hfill\hfill\hfill\hfill\hfill\hfill\hfill\hfill\hfill\hfill\hfill}{\ }
\contentsline {subsection}{\textbf{Ilyin V.\,D., Sokolov I.\,A.}\ \ The Symbol Model of Informatics Knowledge System in Human-Automaton Environment}{\qquad 1 \qquad 66} 
\contentsline {subsection}{\textbf{Ivanov A.\,V.} see Bosov A.\,V.\hfill\hfill\hfill\hfill\hfill\hfill\hfill\hfill\hfill\hfill\hfill\hfill\hfill\hfill\hfill\hfill\hfill\hfill\hfill\hfill\hfill\hfill\hfill\hfill\hfill\hfill\hfill\hfill\hfill\hfill\hfill\hfill\hfill\hfill\hfill}{\ }
\contentsline {subsection}{\textbf{Kalinichenko L.\,A.} see Zakharov V.\,N.\hfill\hfill\hfill\hfill\hfill\hfill\hfill\hfill\hfill\hfill\hfill\hfill\hfill\hfill\hfill\hfill\hfill\hfill\hfill\hfill\hfill\hfill\hfill\hfill\hfill\hfill\hfill\hfill\hfill\hfill\hfill\hfill\hfill\hfill\hfill}{\ }
\contentsline {subsection}{\textbf{Korolev V.\,Yu.} see Batrakova D.\,A.\hfill\hfill\hfill\hfill\hfill\hfill\hfill\hfill\hfill\hfill\hfill\hfill\hfill\hfill\hfill\hfill\hfill\hfill\hfill\hfill\hfill\hfill\hfill\hfill\hfill\hfill\hfill\hfill\hfill\hfill\hfill\hfill\hfill\hfill\hfill}{\ }
\contentsline {subsection}{\textbf{Kozerenko E.\,B.}\ \ Linguistic Simulation for Machine Translation and Knowledge Management Systems}{\qquad 1 \qquad 54} 
\contentsline {subsection}{\textbf{Kozmidiady V.\,A.} see Zakharov V.\,N.\hfill\hfill\hfill\hfill\hfill\hfill\hfill\hfill\hfill\hfill\hfill\hfill\hfill\hfill\hfill\hfill\hfill\hfill\hfill\hfill\hfill\hfill\hfill\hfill\hfill\hfill\hfill\hfill\hfill\hfill\hfill\hfill\hfill\hfill\hfill}{\ }
\contentsline {subsection}{\textbf{Kudryavtsev A.\,A., Shorgin S.\,Ya.}\ \ Bayesian Approach to Queueing Systems and Reliability Characteristics}{\qquad 2 \qquad 76} 
\contentsline {subsection}{\textbf{Pechinkin A.\,V., Sokolov I.\,A., Chaplygin V.\,V.}\ \ Multichannel Queuing System with Finite Buffer and Unreliable Servers}{\qquad 1 \qquad 27} 
\contentsline {subsection}{\textbf{Pechinkin A.\,V., Sokolov I.\,A., Chaplygin V.\,V.}\ \ Stationary Characteristics of a Multichannel Queueing System with\nobreakspace {}Simultaneous Refusals of Servers}{\qquad 2 \qquad 39} 
\contentsline {subsection}{\textbf{Shorgin S.\,Ya.} see Batrakova D.\,A.\hfill\hfill\hfill\hfill\hfill\hfill\hfill\hfill\hfill\hfill\hfill\hfill\hfill\hfill\hfill\hfill\hfill\hfill\hfill\hfill\hfill\hfill\hfill\hfill\hfill\hfill\hfill\hfill\hfill\hfill\hfill\hfill\hfill\hfill\hfill}{\ }
\contentsline {subsection}{\textbf{Shorgin S.\,Ya.} see Kudryavtsev A.\,A.\hfill\hfill\hfill\hfill\hfill\hfill\hfill\hfill\hfill\hfill\hfill\hfill\hfill\hfill\hfill\hfill\hfill\hfill\hfill\hfill\hfill\hfill\hfill\hfill\hfill\hfill\hfill\hfill\hfill\hfill\hfill\hfill\hfill\hfill\hfill}{\ }
\contentsline {subsection}{\textbf{Sinitsyn I.\,N.}\ \ Correlational Methods for Analytical Informational Models of the Earth Pole Fluctuations Design Based on a priori Data}{\qquad 2 \qquad \hphantom{9}2}
\contentsline {subsection}{\textbf{Sinitsyn I.\,N.}\ \ Development of Pugachev Filtering for Stochastic Systems}{\qquad 1 \qquad \hphantom{9}3}
\contentsline {subsection}{\textbf{Sokolov I.\,A.} see Ilyin V.\,D.\hfill\hfill\hfill\hfill\hfill\hfill\hfill\hfill\hfill\hfill\hfill\hfill\hfill\hfill\hfill\hfill\hfill\hfill\hfill\hfill\hfill\hfill\hfill\hfill\hfill\hfill\hfill\hfill\hfill\hfill\hfill\hfill\hfill\hfill\hfill}{\ }
\contentsline {subsection}{\textbf{Sokolov I.\,A.} see Pechinkin A.\,V.\hfill\hfill\hfill\hfill\hfill\hfill\hfill\hfill\hfill\hfill\hfill\hfill\hfill\hfill\hfill\hfill\hfill\hfill\hfill\hfill\hfill\hfill\hfill\hfill\hfill\hfill\hfill\hfill\hfill\hfill\hfill\hfill\hfill\hfill\hfill}{\ }
\contentsline {subsection}{\textbf{Sokolov I.\,A.} see Pechinkin A.\,V.\hfill\hfill\hfill\hfill\hfill\hfill\hfill\hfill\hfill\hfill\hfill\hfill\hfill\hfill\hfill\hfill\hfill\hfill\hfill\hfill\hfill\hfill\hfill\hfill\hfill\hfill\hfill\hfill\hfill\hfill\hfill\hfill\hfill\hfill\hfill}{\ }
\contentsline {subsection}{\textbf{Sokolov I.\,A.} see Zakharov V.\,N.\hfill\hfill\hfill\hfill\hfill\hfill\hfill\hfill\hfill\hfill\hfill\hfill\hfill\hfill\hfill\hfill\hfill\hfill\hfill\hfill\hfill\hfill\hfill\hfill\hfill\hfill\hfill\hfill\hfill\hfill\hfill\hfill\hfill\hfill\hfill}{\ }
\contentsline {subsection}{\textbf{Stupnikov S.\,A.} see Zakharov V.\,N.\hfill\hfill\hfill\hfill\hfill\hfill\hfill\hfill\hfill\hfill\hfill\hfill\hfill\hfill\hfill\hfill\hfill\hfill\hfill\hfill\hfill\hfill\hfill\hfill\hfill\hfill\hfill\hfill\hfill\hfill\hfill\hfill\hfill\hfill\hfill}{\ }
\contentsline {subsection}{\textbf{Zakharov V.\,N., Kalinichenko L.\,A., Sokolov I.\,A., Stupnikov S.\,A.}\ \ Development of Canonical Information Models for Integrated Information Systems}{\qquad 2 \qquad 15} 
\contentsline {subsection}{\textbf{Zakharov V.\,N., Kozmidiady V.\,A.}\ \ Means Providing Applications Fault Tolerance}{\qquad 1 \qquad 14} 
\def\leftfootline{\small{\textbf{\thepage}
\hfill ИНФОРМАТИКА И ЕЁ ПРИМЕНЕНИЯ\ \ \ том~1\ \ \ выпуск~2\ \ \ 2007}
}%
 \def\rightfootline{\small{ИНФОРМАТИКА И ЕЁ ПРИМЕНЕНИЯ\ \ \ том~1\ \ \ выпуск~2\ \ \ 2007
 \hfill \textbf{\thepage}}}
 \label{end\stat}


%\tableofcontents


\end{document}