\def\stat{abstr}
{%\hrule\par
%\vskip 7pt % 7pt
\raggedleft\Large \bf%\baselineskip=3.2ex
A\,B\,S\,T\,R\,A\,C\,T\,S \vskip 17pt
    \hrule
    \par
\vskip 21pt plus 6pt minus 3pt }


\def\tit{ON TASK FLOW PLANNING IN~COMPUTATIONAL RESOURCE SYSTEMS}

%1
\def\aut{M.\,G.~Konovalov}

\def\auf{IPI RAN, mkonovalov@ipiran.ru}

\def\leftkol{\ } % ENGLISH ABSTRACTS}

\def\rightkol{\ } %ENGLISH ABSTRACTS}

\titele{\tit}{\aut}{\auf}{\leftkol}{\rightkol}

%\vspace*{-2pt}

\noindent 
The problem of flow distribution analysis, optimization, and pricing in shared computational 
resource systems is examined. The appropriate literature review is given. An approach to mathematical 
models construction is suggested where task flows being described as dynamic balance equations and quality 
of service relations are used. The subjects in the systems possess own strategies of behavior and purpose 
individual aims in terms of cost and quality of service. Distributed gradient algorithm is one of possible 
system member strategies. A numerical example is given and model development and employment for future trends 
are discussed.

\label{st\stat}

%\vspace*{-5pt}

\KWN{computational resource systems; flow distribution; quality of service; cooperative behavior
}


\vskip 14pt plus 6pt minus 3pt

%\vfil



%2
\def\tit{NONPARAMETRIC ESTIMATION OF BAYESIAN CLASSIFIER ELEMENTS}

\def\aut{M.\,P.~Krivenko}

\def\auf{IPI RAN, mkrivenko@ipiran.ru}


\def\leftkol{\ } % ENGLISH ABSTRACTS}

\def\rightkol{\ } %ENGLISH ABSTRACTS}

\titele{\tit}{\aut}{\auf}{\leftkol}{\rightkol}

%\vspace*{-2pt}

\noindent
The problem of constructing an empirical 
Bayesian classifier, providing recognition of the text, where 
some symbols have different picture sizes, is considered. A combined method 
of constructing an evaluation of Bayesian classifier is proposed. 
The method includes nonparametric kernel estimation and parametric 
estimation with the help of the density of normal distribution. This combined 
assessment allows to deal effectively with the task of handling small amounts 
of training set. Productivity of the proposed ideas is illustrated by an example of recognizing the real text.

%\vspace*{-5pt}

\KWN{Bayesian classifier; combined multivariate density estimation; adaptive kernel estimation; text recognition}
%\pagebreak

\vskip 14pt plus 6pt minus 3pt
%\vskip 14pt plus 9pt minus 6pt

%3
\def\tit{SOLVABILITY PROBLEMS IN THE PROTEIN SECONDARY STRUCTURE RECOGNITION}

\def\aut{K.\,V.~Rudakov$^{1}$ and I.\,Yu.~Torshin$^2$}
\def\auf{$^1$Computing Center of RAS;
Moscow Institute of Physics and Technology, rudakov@ccas.ru\\[1pt]
$^2$Russian Center of the Trace Element Institute for UNESCO, tiy135@yahoo.com}

\def\leftkol{\ } % ENGLISH ABSTRACTS}

\def\rightkol{\ } %ENGLISH ABSTRACTS}

%\def\leftkol{ENGLISH ABSTRACTS}

%\def\rightkol{ENGLISH ABSTRACTS}

\titele{\tit}{\aut}{\auf}{\leftkol}{\rightkol}

%\vspace*{-2pt}
\noindent
The purpose of the work is to develop a formalism for the application of the algebraic 
approach to recognition of the protein secondary structure. Paper presents rigorous formal 
description of the problem and considers its solvability, regularity, and locality. Key terms 
for the analysis of locality, such as a neighborhood,  mask, mask system, monotony, and irreducibility 
of the mask systems were proposed. An algorithm for constructing nonredundant mask systems was formulated. 
The formalism has allowed to formulate a correct description of the hypothesis of the local character of the 
dependence of the secondary structure on the primary and to obtain constructive criteria of the solvability 
of the problem. 

%\vspace*{-5pt}

\KWN{algebraic approach; the secondary structure of protein; bioinformatics}
\pagebreak


%\vfil
%4
%\vskip 14pt plus 6pt minus 3pt


\def\tit{ASYMPTOTIC PROPERTIES OF RISK ESTIMATE OF WAVELET-VAGUELETTE COEFFICIENTS THRESHOLDING 
IN~TOMOGRAPHIC RECONSTRUCTION PROBLEM}


\def\aut{A.\,V.~Markin$^1$ and O.\,V.~Shestakov$^2$}
\def\auf{$^1$Department of Mathematical Statistics, Faculty of
Computational Mathematics and Cybernetics,\\  
\hphantom{$^1$}M.\,V.~Lomonosov Moscow State University, artem.v.markin@mail.ru\\[1pt]
$^2$Department of Mathematical Statistics, Faculty of
Computational Mathematics and Cybernetics,\\  
\hphantom{$^1$}M.\,V.~Lomonosov Moscow State University,
oshestakov@cs.msu.su}

\def\leftkol{\ } % ENGLISH ABSTRACTS}

\def\rightkol{\ } %ENGLISH ABSTRACTS}

\titele{\tit}{\aut}{\auf}{\leftkol}{\rightkol}

\noindent
Tomographic image reconstruction problem using wavelet-vaguelette 
decomposition is considered. Consistency and asymptotic normality 
of risk estimate of vaguelette coefficients thresholding are studied.

%\label{st\stat}

\KWN{wavelets; tomography; thresholding; risk estimate; limit distribution}

%\pagebreak

% \thispagestyle{headings}



%\vful

 %\vskip 14pt plus 6pt minus 3pt

% \vskip 24pt plus 9pt minus 6pt
\vskip 8pt plus 3pt minus 3pt


%5
\def\tit{ANALYSIS OF A LINK PROTOCOL WITH A~GENERAL CONTENTION WINDOW BACKOFF FUNCTION}


\def\aut{A.\,S.~Lukyanenko$^1$, E.\,V.~Morozov$^2$, and A.~Gurtov$^3$}

\def\auf{$^1$Helsinki Institute for Information Technology HIIT, Aalto, Finland, firstname.secondname@hiit.fi\\[1pt]
$^2$Institute of Applied Mathematical  Research, Karelian Research Centre RAS; Petrozavodsk State University,\\
$\hphantom{^1}$emorozov@krc.karelia.ru\\[1pt]
$^3$Helsinki Institute for Information Technology HIIT, Aalto, Finland, gurtov@hiit.fi}


\def\leftkol{ENGLISH ABSTRACTS}

\def\rightkol{ENGLISH ABSTRACTS}

\titele{\tit}{\aut}{\auf}{\leftkol}{\rightkol}

%\vspace*{-2pt}

\noindent
A set of medium access (backoff) protocols, where the
collision resolution window for a station depends on the number of
successive collisions, is analyzed. Under mild common assumptions for the network
properties, a general backoff protocol is studied. An optimal criterion
yields a backoff protocol possessing the minimal service time. An
asymptotic analysis is offered for the unboundly growing number of
stations. Bounded and unbounded protocols are considered in the
analysis. Finally, a continuous time model is introduced as an extension
for the slotted model, which allows slots of different sizes.

%\vspace*{-5pt}

\KWN{data communications; performance analysis; backoff protocol; stability; medium access control}

%\vskip 18pt plus 6pt minus 3pt

 \vskip 8pt plus 6pt minus 3pt

% \pagebreak

%6
\def\tit{DEVELOPMENT OF PARALLEL HEURISTIC ALGORITHMS OF~WEIGHTS COEFFICIENTS SELECTION FOR~ARTIFICIAL 
NEURAL NETWORK}

\def\aut{O.\,V.~Kryuchin}
\def\auf{G.\,R.~Derzhavin Tambov State University, kryuchov@gmail.com}

\titele{\tit}{\aut}{\auf}{\leftkol}{\rightkol}

%\vspace*{-2pt}

\noindent
A gradients alogirithm of artificial neural network and heuristic algoritms 
QuickProp and RProp which are based on it are described. Possible applications of cluster systems have been considered.


%\vspace*{-5pt}

\KWN{artificial neural network; heuristic algorithms of teaching; cluster systesm}
%\pagebreak

\vskip 8pt plus 6pt minus 3pt

%7
\def\tit{APPLICATION OF THE COORDINATE METHOD OF COMMUTATED NEURAL NETWORK FRAGMENTATION FOR TRAFFIC~REDUCTION}

\def\aut{S.\,Y.~Stepanov}
\def\auf{Moscow State Technological Institute STANKIN, 
cympak\_shade@rambler.ru}


\titele{\tit}{\aut}{\auf}{\leftkol}{\rightkol}

%\vspace*{-2pt}

\noindent
The problem of increasing traffic in scaling commutated neural network, 
a method and algorithm of its solution are outlined. An example of the developed algorithm
is also presented.

%\vspace*{-5pt}

\KWN{commutated neural network; scaling; traffic}
%\pagebreak

 \vskip 14pt plus 6pt minus 3pt

%8
\def\tit{ON ASYMPTOTIC BEHAVIOR OF THE POWERS OF~THE~TESTS FOR~THE~CASE OF~LAPLACE DISTRIBUTION
}

\def\aut{V.\,E.\,Bening$^1$ and R.\,A.\,Korolev$^2$}
\def\auf{$^1$Faculty of Computational Mathematics and Cybernetics, 
M.\,V.~Lomonosov Moscow State University,\\
$\hphantom{^1}$ bening@yandex.ru\\[1pt]
$^2$Faculty of Computational Mathematics and Cybernetics, 
M.\,V.~Lomonosov Moscow State University,\\
$\hphantom{^1}$stochastique@gmail.com
}

\titele{\tit}{\aut}{\auf}{\leftkol}{\rightkol}

%\vspace*{-2pt}

\noindent
A formula for the limit of the normalized difference between 
the power of the asymptotically most powerful test and the power of the asymptotically optimal test 
for the case of Laplace distribution was proved. Due to the nonregularity of the Laplace distribution, 
the logarithm of the likelihood ratio admits nonregular stochastic expansion, 
and an analog of Cram$\acute{\mbox{e}}$r condition is not valid for the sign 
statistic which is the basis of the asymptotically optimal test. 
Then direct use of theorem~3.2.1 from~[1] or theorem~2.1 from~[2] is difficult, 
and in the present paper, their proofs for the case of Laplace distribution are revisited.

%\vspace*{-5pt}


\KWN{power function; conditional probability measure; conditional moment; Laplace distribution}
%\pagebreak

 \vskip 14pt plus 6pt minus 3pt

%9

\def\tit{AN IMPROVEMENT OF THE KATZ--BERRY--ESSEEN INEQUALITY}

\def\aut{M.\,E.~Grigorieva$^1$ and I.\,G.~Shevtsova$^2$}
\def\auf{$^1$Department of Mathematical Statistics, Faculty of Computational Mathematics and Cybernetics,\\
$\hphantom{^1}$M.\,V.~Lomonosov Moscow State University,
maria-grigorieva@yandex.su\\[1pt]
$^2$Department of Mathematical Statistics, Faculty of Computational Mathematics and Cybernetics,\\
$\hphantom{^1}$M.\,V.~Lomonosov Moscow State University, ishevtsova@cs.msu.su
}

\titele{\tit}{\aut}{\auf}{\leftkol}{\rightkol}

\noindent
The upper estimates of the absolute constant in the Katz--Berry--Esseen inequality 
for sums of independent identically distributed random variables with finite absolute moments of order between~2 
and~3 are sharpened and an alternative inequality with sharpened structure and evaluated constants is proposed.

\KWN{central limit theorem; Katz--Berry--Esseen inequality; Lyapounov fraction}
%\pagebreak



\vskip 14pt plus 6pt minus 3pt

%10
\def\tit{LINGUISTIC FILTERS IN STATISTICAL MACHINE TRANSLATION MODELS
}
\def\aut{E.\,B.~Kozerenko}


\def\auf{IPI RAN, kozerenko@mail.ru}


%\def\leftkol{ENGLISH ABSTRACTS}

%\def\rightkol{ENGLISH ABSTRACTS}

\titele{\tit}{\aut}{\auf}{\leftkol}{\rightkol}

 \label{end\stat}

\noindent
The paper focuses on the problems of linguistic filters development for statistical machine 
translation and advancement of parallel texts alignment methods for enhancing precision and 
adequacy of translations. Statistical and heuristic models of alignment and translation are 
considered. The solutions proposed are based on the hybrid grammar formalism comprising the 
linguistic rules and probability characteristics of language structures.


\KWN{statistical models; machine translation; parallel texts; alignment; linguistic filters
}

%\pagebreak

 