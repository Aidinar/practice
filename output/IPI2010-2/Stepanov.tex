\def\stat{stepanov}

\def\tit{ИСПОЛЬЗОВАНИЕ КООРДИНАТНОГО МЕТОДА 
ФРАГМЕНТАЦИИ КОММУТАТОРНОЙ НЕЙРОННОЙ СЕТИ 
ДЛЯ~СОКРАЩЕНИЯ ТРАФИКА}

\def\titkol{Использование координатного метода 
фрагментации коммутаторной нейронной сети 
для~сокращения трафика}

\def\autkol{С.\,Ю.~Степанов}
\def\aut{С.\,Ю.~Степанов$^1$}

\titel{\tit}{\aut}{\autkol}{\titkol}

%{\renewcommand{\thefootnote}{\fnsymbol{footnote}}\footnotetext[1]
%{Исследования выполнены при частичной поддержке РФФИ, гранты 08-01-00567, 08-01-91205, 09-01-12180.}}

\renewcommand{\thefootnote}{\arabic{footnote}}
\footnotetext[1]{ГОУ ВПО МГТУ <<Станкин>>, cympak\_shade@rambler.ru}

  
\Abst{Описана проблема возрастания трафика при масштабировании 
коммутаторной нейронной сети, предложен метод и алгоритм ее решения. Приведен 
пример работы разработанного алгоритма.}
  
  \KW{коммутаторная нейронная сеть; масштабирование; трафик}

  \vspace*{24pt}
  
       \vskip 18pt plus 9pt minus 6pt

      \thispagestyle{headings}

      \begin{multicols}{2}

      \label{st\stat}
  
\section{Введение}
     
     Коммутаторная нейронная сеть~--- новая технология построения 
нейронных сетей, позволяющая создавать большие искусственные 
нейронные сети для задач управления сложными техническими объектами и 
обработки информации~[1].
     
     Традиционный способ построения нейронных сетей предусматривает, 
что такая сеть состоит только из одного типа элементов~--- нейронов. 
Нейрон имеет  несколько входов и один выход. Каждому входу приписано 
некоторое число, называемое весом. Полученный на входные каналы сигнал 
называется входным сигналом. Он преобразуется в сигнал внут\-рен\-не\-го 
возбуждения нейрона и рассчитывается как сумма произведений значений 
каждого сигнала на вес соответствующего канала. Выходной сигнал нейрона 
получается путем воздействия активационной функции на сигнал 
внут\-рен\-не\-го возбуждения. Связываясь между собой, нейроны образуют 
нейронную сеть~--- совокупность нейронов, удовлетворяющую следующим 
требованиям: 
     \begin{itemize}
     \item  из совокупности всех входных каналов нейронов выделяется 
множество каналов, на\-зы\-ва\-емое множеством входных каналов сети; 
     \item если вход нейрона не является входным каналом сети, то на него 
подается сигнал выходного канала одного и только одного нейрона этой 
сети; 
     \item из совокупности всех нейронов сети выделяется некоторая 
подсовокупность, элементы которой называются выходными нейронами 
сети, выходные каналы которых образуют множество выходных каналов 
сети; 
     \item если выходной канал нейрона сети не является выходным 
каналом сети, то его сигнал подается на входной канал хотя бы одного 
нейрона сети.
     \end{itemize}
     
      Таким образом, все нейроны выполняют две общие для них функции: 
обработку получаемой информации, масштабирование и передачу 
информации к соответствующим входам других нейронов сети. 
     
     При таком способе организации возникает ощутимая проблема 
масштабирования нейронной сети при значительном увеличении числа 
нейронов. Существенно усложняется работа нейрона по решению 
транспортной задачи из-за увеличения объема передаваемой информации и 
размера внутренних таблиц связи. При аппаратной реализации нейронной 
сети квадратично увеличивается число потенциально поддерживаемых линий 
передачи информации~[2].
     
     Коммутаторная нейронная сеть (рис.~1) отличается от традиционной нейронной 
сети тем, что задачи транспорта и обработки информации разделены. Такая 
сеть содержит два типа элементов (каждый из которых выполняет только 
одну функцию):
     \begin{itemize}
     \item
нейрон, обрабатывающий информацию;
\item коммутатор, обеспечивающий транспорт и масштабирование 
информации.
\end{itemize}

   

     Нейрон коммутаторной нейронной сети~--- устройство, имеющее один 
вход и один выход и выполняющее функцию принятия решений. Нейрон 
изменяет свое состояние дискретно и может
 быть различного типа в 
соответствии с требуемой активационной функцией.
\pagebreak
\end{multicols}

\begin{figure} %fig1
\vspace*{1pt}
\begin{center}
\mbox{%
\epsfxsize=164.361mm
\epsfbox{ste-1.eps}
}
\end{center}
\vspace*{-3pt}
\Caption{Коммутаторная нейронная сеть
\label{f1step}}
\vspace*{14pt}
\end{figure}

\begin{multicols}{2}


     
     Коммутатор коммутаторной нейронной сети~--- устройство, имеющее 
много входов и много выходов и перераспределяющее информацию между
другими коммутаторами и нейронами коммутаторной нейронной сети. Он 
состоит из таблицы связей нейронов, составляющих фрагмент сети, и 
устройства, масштабирующего и передающего информацию между 
нейронами на основе этой таблицы. Число входов и выходов коммутатора 
ограничено и определяется его реализацией. При обучении нейронной сети 
изменяется таблица связей нейронов в коммутаторе, а сами нейроны при 
этом не  затрагиваются. В~результате применения коммутаторов  нейронная 
сеть приобретает иерархическую древовидную структуру~[3].
     
     Базовая нейронная сеть~--- нейронная сеть, которую обслуживает один 
коммутатор~[4]. Мас\-шта\-бирование нейронной сети осуществляется 
увеличением числа базовых нейронных сетей. Для\linebreak интеграции отдельных 
базовых сетей в единую нейронную сеть используются коммутаторы 
верхнего уровня.

\section{Задача сокращения трафика в~коммутаторной 
нейронной сети}
\vspace*{-12pt}

     При связи нейронов между собой путь передаваемой информации 
(трафик) существенно зависит от взаимного расположения нейронов в 
древовидной структуре~[5]. Чем дальше нейроны расположены друг от 
друга, тем больше трафик информации. Поэтому правомерна задача 
оптимизации трафика в нейронной сети за счет рационального расположения 
нейронов на коммутаторах. Внутри одной базовой нейронной сети сократить 
трафик невозможно. Переставляя нейроны между различными базовыми 
нейронными сетями, можно обеспечить\linebreak существенное снижение трафика в 
нейронной\linebreak сети. Важным аспектом этой задачи является логическая 
группировка нейронов, по результатам которой возможно создание 
отдельных групп (подмножеств нейронов), в целом отвечающих за 
отдельные подзадачи общей задачи нейронной сети.
\pagebreak
     
     При решении оптимизационной задачи необходимо:
     \begin{itemize}
     \item определить, как по обученной нейронной сети выделить 
подмножества нейронов, отвечающих за отдельные подзадачи общей задачи 
нейронной сети;
     \item сгруппировать нейроны в базовые элементарные сегменты, 
сократив информационный трафик в сети;
     \item рационально расположить базовые элементарные сегменты на 
коммутаторах верхнего уровня, сократив информационный трафик в сети.
     \end{itemize}
     
     Группировку следует производить таким образом, чтобы связанные 
между собой нейроны  были расположены максимально близко друг от 
друга, при этом не потребуется физически перемещать нейроны от одного 
коммутатора к другому. Достаточно изменить идентификаторы нейронов в 
дереве, содержащем информацию о структуре сети.
     
     Элементарный сегмент обслуживается одним коммутатором нижнего 
уровня, емкость которого имеет определенные ограничения. Эти 
ограничения задают предельный размер базового элементарного сегмента.
     
     Рациональная группировка нейронов позволит по обученной 
нейронной сети определить, какие подмножества нейронов отвечают за 
отдельные подзадачи, поскольку существование более или менее 
изолированной группы нейронов позволяет говорить, что эта группа 
выполняет более или менее изолированную подзадачу общей задачи. При 
этом чем меньше связей эта группа имеет с другими группами нейронов, тем 
более изолированную подзадачу она выполняет.
     
     В полученных таким образом группах большинство связей между 
нейронами будут внутренними. Решение задачи оптимальной декомпозиции 
нейронной сети позволит перейти к доменной орга\-низации коммутаторной 
нейронной сети~[6]. Для этого необходимо ввести дополнительный\linebreak 
элемент~--- шлюз домена, отделяющий адресное пространство нейронов 
домена от адресного пространства всей нейронной сети. Доменная струк\-тура 
позволяет выделить автономный элемент,\linebreak име\-ющий интерфейс 
     ввода-вывода, облегчить проведение обучения больших нейронных 
сетей и анализ подзадач, которые они выполняют~[7].

\section{Математическая постановка задачи}

     Сопоставим нейронной сети ориентированный граф. Множеству 
вершин этого графа будет соответствовать множество нейронов сети, а 
множеству дуг~--- множество связей между нейронами. 
Направление дуг графа определяется направлением потока информации в 
нейронной сети. Граф необходимо разбить на фиксированное число 
подграфов таким образом, чтобы суммарное число дуг, связывающих 
подграфы, было минимальным. Все вершины графа должны войти в эти 
подграфы, причем каждая вершина должна войти только в один подграф.

%\medskip


\setcounter{figure}{2}
\begin{figure*} %fig3
\vspace*{1pt}
\begin{center}
\mbox{%
\epsfxsize=126.93mm
\epsfbox{ste-3.eps}
}
\end{center}
\vspace*{-9pt}
\Caption{Действующие на объекты импульсы
\label{f3step}}
\end{figure*}
     
     В связи с физическим ограничением на число нейронов, одновременно 
подключенных к коммутатору первого уровня, следует ввести 
дополнительное условие: размер каждого подграфа должен быть меньше или 
равен числу нейронов, подключенных к коммутатору первого уровня.
     
     Решение задачи разбиения графа на подграфы с учетом заданных 
условий позволит решить задачу оптимизации трафика в коммутаторной 
нейронной сети.

\vspace*{-6pt}

\section{Метод решения задачи}

\vspace*{-3pt}
     
     Рассмотрим плоскость, на которой расположена сетка, в узлах которой 
в произвольном порядке размещена группа объектов, соответствующих  
вершинам данного графа (рис.~2).  Введем импульс притяжения между объектами 
таким образом, что если между двумя вершинами в графе существует дуга, 
то эти две вершины притягиваются друг к другу. Импульс притяжения 
должен быть постоянным для каждой пары объектов (рис.~3). 
     
     Для предотвращения слияния объектов в общее ядро введем импульс 
отталкивания между всеми объектами. Этот импульс  обратно 
пропорционален расстоянию между объектами. Для того чтобы объекты 
могли объединяться в группы,  следует уменьшить импульс отталкивания,  
когда объекты сблизились на определенное расстояние.


%\vspace*{-9pt}
\noindent
\begin{center} %fig2
\vspace*{12pt}
\mbox{%
\epsfxsize=78.484mm
\epsfbox{ste-2.eps}
}
\end{center}
\vspace*{2pt}
%\begin{center}
{{\figurename~2}\ \ \small{Работа алгоритма: (\textit{а})~исходное состояние; (\textit{б})~результат 
группирования}}
%\end{center}
%\vspace*{-9pt}
%\medskip
\addtocounter{figure}{1}


\pagebreak
     
     Определив результирующий импульс как сумму векторов всех 
импульсов, действующих на объект, получим направление движения для 
каждого объекта и значение перемещения. Через некоторое время объекты, 
имеющие между собой импульсы притяжения, должны оказаться рядом на 
плос\-кости. Изменяя параметры импульсов притяжения и отталкивания, 
можно варьировать количество и размеры получаемых элементарных 
сегментов.
     
     Используя данный метод группирования многократно, можно решить 
задачу оптимизации трафика информации,  разделив ее на следующие 
стадии:
     \begin{itemize}
     \item разбиение и группировка  нейронов в группы;
     \item разбиение каждой группы на подгруппы определенного размера;
     \item группировка подгрупп между собой.
     \end{itemize}



     Разработанный алгоритм группирования можно описать следующим 
образом:
\begin{description}     
\item[\ ] $A$~--- коэффициент притяжения, постоянный и положительный;
\item[\ ]
$W(i,k)$~--- вес связи между нейронами $i$ и~$k$ в рассматриваемой 
нейронной сети;
\item[\ ]
$B$~--- коэффициент отталкивания, постоянный и положительный;
\item[\ ]
$S(i,k)$~--- расстояние между объектами~$i$ и~$k$ на плоскости;
\item[\ ]
$P_{\mathrm{прит}}=A\vert 
W(i,k)\vert$~--- импульс притяжения [кг$\cdot$м/с];
\item[\ ]
$P_{\mathrm{отт}}= B/S$~--- импульс отталкивания [кг$\cdot$м/с];
\item[\ ]
$n$~--- число нейронов в сети.
\end{description}
     
     \noindent
     \begin{enumerate}[1.]
     \item Для каждого нейрона сети провести следующие действия:
     \begin{enumerate}[{1.}1]
     \item для каждой пары нейронов:
     \begin{enumerate}[{1.1.}1]
     \item найти расстояние между соответствующими объектами на 
плос\-кости;
     \item  рассчитать воздействующий на объекты импульс как разность 
$P_{\mathrm{отт}}$ и~$P_{\mathrm{прит}}$;
     \item рассчитать $\Delta x(i,k)$ и~$\Delta x(k,i)$~--- перемещение 
каждого объекта из \mbox{пары};
     \item рассчитать $\Delta y(i,k)$ и $\Delta y(k,i)$~--- перемещение 
каждого объекта из \mbox{пары}.
     \end{enumerate}
     \end{enumerate}
     \item Переместить каждый объект на плоскости в соответствии с 
рассчитанными~$\Delta x(i)$ и~$\Delta y(i)$.
     \item Определить, пришла ли система из объектов на плоскости в 
состояние покоя, сравнив значения перемещений каждого объекта с заданной 
минимальной величиной.
     \item Повторить пп.\,1--3 $m$~раз, пока система не придет в состояние 
равновесия.
     \end{enumerate}
     
     Значение $m$ зависит от числа нейронов, величины связей между 
ними, топологии сети и выбранных коэффициентов~$A$ и~$B$. Для 
рассматриваемых в примерах сетей экспериментально определено, что 
значение коэффициента~$m$ сопоставимо с~$n^3$.
     
     Определим вычислительную трудоемкость разработанного алгоритма и 
сравним с вычислительной трудоемкостью алгоритма полного 
перебора~\cite{9step}.
     
     Трудоемкость алгоритма: $O(m)(O(n^2)+O(n)\;+$\linebreak $+\;O(n)) = O(m(n^2+n))$.
     
     Определим вычислительную трудоемкость алгоритма перебора 
вариантов размещения нейронов в коммутаторной нейронной сети. Для этого 
формализуем алгоритм:
     \begin{enumerate}[1.]
     \item Для $n!$ возможных комбинаций расположения нейронов:
     \begin{enumerate}[{1.}1]
     \item  подсчитать общее число линий передачи данных в 
коммутаторной нейронной сети для связи каждой пары нейронов друг с 
другом;
     \item определить оптимальный вариант из $n!$~комбинаций.
     \end{enumerate}
     \end{enumerate}
     
     Трудоемкость алгоритма перебора: $O(n!\cdot n^2)$.
     
     По полученной оценке трудоемкости алгоритмов можно сказать, что 
для фрагментации нейронных сетей, состоящих из большого числа нейронов 
(от сотен до десятков тысяч), применение координатного метода дает 
значительное преимущество по затратам на оптимизацию структуры сети.

\section{Пример использования алгоритма группирования}
     
     Наиболее сложной структурой нейронной сети является полносвязная 
нейронная сеть, в которой выходные сигналы всех 
нейронов подаются на вход каждого нейрона сети, выходные каналы всех 
нейронов образуют выходной сигнал сети, а также каждый нейрон сети 
имеет как минимум один канал входной информации, являющейся внешней 
по отношению к этой сети. В~реальных задачах используются сети с 
неполной топологией,  поэтому для исследования работы алгоритма были 
выбраны полносвязные нейронные сети с произвольными весовыми 
коэффициентами связей, подвергнутые редукции связей.
     
     Простейшим критерием редукции считается учет величины весов. 
Веса, которые значительно меньше средних значений (нулевые или близкие к 
нулю), оказывают незначительное влияние на общий уровень выходного 
сигнала связанного с ними\linebreak\vspace*{-12pt}
\columnbreak

\vspace*{-18pt}

\noindent
\begin{center} %fig4
\vspace*{12pt}
\mbox{%
\epsfxsize=78.405mm
\epsfbox{ste-4.eps}
}
\end{center}
\vspace*{4pt}
%\begin{center}
{{\figurename~4}\ \ \small{Пример группирования: (\textit{а})~исходная сеть; (\textit{б})~результат 
группирования}}
%\end{center}
%\vspace*{9pt}

\bigskip
\addtocounter{figure}{1}

\noindent 
 нейрона. Поэтому их можно отсечь без 
существенного вреда для его функционирования~\cite{8step}.
     
     Рассмотрим алгоритм подключения полученных групп на примере 
оптимизации работы тес\-товой полносвязной нейронной сети с 
реду\-цированным числом связей. Предположим, что коммутаторная 
нейронная сеть состоит из 18~нейронов,  двух коммутаторов первого уровня 
с десятью входами и одним коммутатором второго уровня (рис.~4).
     
     Исходя из структуры сети необходимо подобрать такие коэффициенты, 
чтобы число полученных групп равнялось~2, а число элементов в одной 
группе не превышало~10. В~результате подбора и группирования были 
получены 2~группы размерами~10 и~8, что удовлетворяет требованиям к 
структуре сети.
     
     Заметим, что по сравнению с первым результатом группирования 
уменьшилось число нейронов, связанных с другой группой, и стало 
равным~9.

     
     Всего в рассматриваемой сети 126~связей между нейронами. Для 
сравнения объема трафика в коммутаторной нейронной сети до 
использования алгоритма группирования и после, возьмем первоначальную 
группу нейронов и произвольно разобьем ее на 2~подгруппы по 9~нейронов 
в каждой. Подключив эти подгруппы к коммутаторам первого уровня, 
получим следующий результат: в первой подгруппе будет 45~связей между 
элементами, во второй~29, а через коммутатор второго уровня пройдет 
52~связи.
     
     Затем  подключим подгруппы, полученные в процессе работы 
алгоритма группирования. На первом коммутаторе первого уровня будет 
находиться 10~элементов и 65~связей между ними. На втором коммутаторе 
первого уровня будет находиться 8~элементов и 48~связей между ними. 
Через коммутатор второго уровня пройдет 13~связей.
     
     В результате, если рассчитать~$S$, суммарное число линий передачи 
информации коммутатор--коммутатор и коммутатор--нейрон, пройденных 
потоком информации, получим:
     
     для первого случая $S_1 = (45 + 29)\cdot 2 + 52\cdot 4 = 356$;
     
     для второго случая $S_2 = (65 + 48)\cdot 2 + 13\cdot 4 = 278$.
     
     Таким образом, для рассматриваемого примера при использовании 
предложенного алгоритма оптимизации происходит сокращение трафика 
информации на 21,91\% относительно первого варианта подключения.


     На каждом шаге работы алгоритма число объектов, связанных между 
собой и принадлежащих различ\-ным группам, не увеличивается. Целью 
группирования является получение локального минимума суммарного числа 
связей между коммутаторами разных уровней, необходимых для работы
сети. Таким образом, результаты работы предложенного алгоритма 
удовлетворяют решению по\-став\-лен\-ной задачи оптимизации трафика 
информации в коммутаторной нейронной сети.
     
     Методом случайной генерации связей были получены три нейронные 
сети, состоящие из 169~нейронов, коэффициенты и наличие связей между\linebreak 
двумя любыми объектами получены случайно с одинаковой для всех 
вероятностью~\cite{10step}. В~результате группирования каждой сети было 
получено около 30~групп с максимальным размером группы, равным~20. 
Суммарное число связей во всех случаях группирования каждой из 
рассматриваемых сетей находилось на отрезке~[1254,\,1302].


     В среднем сокращение трафика информации при использовании 
предложенных алгоритмов по сравнению с произвольным подключением к 
коммутаторам для данных групп составило~5,5\%.
     
     Для множества сетей из 400~нейронов и приблизительно таким же 
распределением связей сокращение трафика составило около~9\%.
     
     Таким образом, даже для сети, по топологии близкой к полносвязной, в 
которой отсутствуют фрагменты, явно отвечающие за отдельные подзадачи, 
а соответственно, и более изолированные от всей сети, можно получить 
достаточно заметное сокращение трафика.

\vspace*{-3pt}
\section{Заключение}
\vspace*{-1pt}


     Разработанный алгоритм может подвергаться 
масштабированию~\cite{11step}. Для этого необходимо провести 
фрагментацию нейронной сети~---  произвольным образом разбить большую 
нейронную сеть на группы требуемого размера и проводить группирование 
каждой группы по отдельности. Затем провести замену полученных в 
результате работы алгоритма групп на объекты, имеющие соответствующие 
связи с другими группами сети. И~в дальнейшем применять алгоритм к 
полученным объектам. Необходимым условием в таком случае является 
ограничение на размер получаемых групп, который должен быть при 
первичном группировании заведомо меньше размера базовой нейронной 
сети. В~ситуации получения итогового размера групп нейронов значительно 
большего, чем размер базовой нейронной сети, следует дополнительно 
фрагментировать группы с использованием разработанного алгоритма.
     
     Коммутаторная нейронная сеть дает возможность создавать нейронные 
сети любой архитектуры и практически любого объема, что позволяет, в 
свою очередь, создавать системы управления оборудованием и 
технологическими процессами, а также реализовывать системы 
искусственного интеллекта. Разработанный метод фрагментации 
коммутаторной нейронной сети позволяет оптимизировать ее структуру и 
сократить затраты на передачу информации между нейронами.

\vspace*{-6pt}
    
{\small\frenchspacing
{\baselineskip=10.36pt
\addcontentsline{toc}{section}{Литература}
\begin{thebibliography}{99}
%\vspace*{-1pt}

\bibitem{1step}
\Au{Кабак И.\,С., Суханова Н.\,В.}
Нейронная сеть. Патент на ПМ №\,75247 РФ.

\bibitem{2step}
Теория нейронных сетей: Учеб.\ пособие для вузов~/ Под общ. ред. 
А.\,И.~Галушкина. Кн.~1.~--- М.: ИПРЖР, 2000.  416~с.

\bibitem{3step}
\Au{Кабак И.\,С., Суханова~Н.\,В.}
Большие нейронные сети в системах управления~// Тр. XVI 
междунар. научно-технич. конф. <<Информационные 
средства и технологии>>. Т.~3.~--- М.: МЭИ, 2008. С.~204--210.

\bibitem{4step}
\Au{Кабак И.\,С.}
Коммутаторная архитектура больших нейронных сетей~// Тр. XV 
междунар. научно-технич. конф. <<Информационные 
средства и технологии>>. Т.~3.~--- М.: МЭИ, 2007. С.~124--127.

\bibitem{5step}
\Au{Кабак И.\,С., Степанов~С.\,Ю.}
Оптимизация трафика информации в коммутаторной нейронной сети~// 
Тр. XIV междунар. научно-технич. конф. 
<<Информационные средства и технологии>>. Т.~3.~--- М.: МЭИ, 2006. 
С.~163--167.

\bibitem{6step}
\Au{Кабак И.\,С., Суханова Н.\,В.}
Доменная нейронная сеть. Патент на ПМ №\,72084 РФ. 

\bibitem{7step}
\Au{Кабак И.\,С.}
Доменная организация коммутаторных нейронных сетей~// Тр. XV 
междунар. научно-технич. конф. <<Информационные 
средства и технологии>>. Т.~3.~--- М.: МЭИ, 2007. С.~128--131.

\bibitem{9step}
\Au{Кормен Т., Лейзерсон Ч., Ривест~Р., Штайн~К.}
Алгоритмы: построение и анализ~// Introduction to Algorithms~/ Под ред. 
И.\,В.~Красикова. 2-е изд.~--- М.: Вильямс, 2005. 1296~с.

\bibitem{8step}
\Au{Осовский С.}
Нейронные сети для обработки информации~/ Пер. с польск. 
И.\,Д.~Рудинского.~--- М.: Финансы и статистика, 2004.  344~с.

\bibitem{10step}
\Au{Степанов С.\,Ю.}
Группирование нейронов в двух\-уров\-не\-вой коммутаторной нейронной сети~// 
Тр. XV междунар. научно-технич. конф. 
<<Информационные средства и технологии>>. Т.~3.~--- М.: МЭИ, 2007. 
С.~178--181.

\label{end\stat}

\bibitem{11step}
\Au{Степанов С.\,Ю.}
Снижение трафика информации в коммутаторной нейронной сети на основе 
ее фрагментации~// Тр. XVI междунар. научно-технич. 
конф. <<Информационные средства и технологии>>. Т.~3.~--- М.: 
МЭИ, 2008. С.~236--241.
 \end{thebibliography}
}
}

\end{multicols}