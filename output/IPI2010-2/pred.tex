{ %\Large  
{ %\baselineskip=16.6pt

\vspace*{-48pt}
\begin{center}\LARGE
\textit{Предисловие}
\end{center}

%\vspace*{2.5mm}

\vspace*{9mm}

\thispagestyle{empty}

{\small

    
      Вниманию читателей журнала <<Информатика и её применения>> предлагается 
очередной тематический выпуск <<Вероятностно-статистические методы и задачи 
информатики и информационных технологий>>. Предыдущие тематические выпуски 
журнала по данному направлению вышли в 2008~г.\ (том~2, вып.~2) и в 2009~г.\ (том~3, 
вып.~3). 
      
      Статьи, собранные в данном журнале, посвящены разработке новых вероятностно-статистических 
      методов, ориентированных на решение конкретных задач 
информатики и информационных технологий, а также~--- в ряде случаев~--- и других 
прикладных задач. Проблематика, охватываемая публикуемыми работами, развивается в 
рамках научного сотрудничества между Институтом проблем информатики Российской 
академии наук (ИПИ РАН) и Факультетом вычислительной математики и кибернетики 
Мос\-ков\-ско\-го государственного университета им.~М.\,В.~Ломоносова в ходе работ над 
сов\-местными научными проектами (в том числе в рамках функционирования
научно-образовательного центра ИПИ РАН\,--\,ВМК МГУ <<Вероятностно-статистические 
методы анализа рисков>>). Многие авторы статей данного выпуска
журнала являются активными участниками традиционного международного семинара по 
проблемам устойчивости стохастических моделей, руководимого В.\,М.~Золотарёвым и 
В.\,Ю.~Королевым; регулярные сессии этого семинара проводятся под эгидой МГУ и 
ИПИ РАН. 
      
      Наряду с представителями ИПИ РАН и МГУ, в число авторов 
      входят ученые из ВЦ РАН им.~А.\,А.~Дородницына, Института прикладных 
математических исследований Карельского НЦ РАН, Московского физико-технического 
института, МГТУ <<Станкин>>, Тамбовского государственного университета им.\ 
Г.\,Р.~Державина, Российского отделения Института микроэлементов ЮНЕСКО, Helsinki 
Institute for Information Technology (Хельсинки, Финляндия).
      
      В данном выпуске традиционно присутствует тематика, весьма активно 
разрабатываемая в течение многих лет специалистами ИПИ РАН и МГУ,~--- методы 
моделирования и управления для ин\-фор\-ма\-ци\-он\-но-те\-ле\-ком\-му\-ни\-ка\-ци\-он\-ных и 
вы\-чис\-ли\-тель\-ных систем. В~статье М.\,Г.~Коновалова рассматривается проблема анализа 
и оптимизации распределения потоков заданий и ценообразования в системах 
коллективного использования распределенных вы\-чис\-ли\-тель\-ных ресурсов. Предложен 
подход к построению математических моделей систем вы\-чис\-ли\-тель\-ных ресурсов, 
основанный на укрупненном описании потоков заданий в виде динамических балансовых 
соотношений. В~статье А.\,С.~Лукьяненко и Е.\,В.~Морозова исследован класс сетевых 
протоколов контроля несущей среды, где окно передачи сообщения является 
произвольной возрастающей функцией числа конфликтов сообщения, посланного с 
данной станции. 
      
      Ряд статей выпуска посвящен разработке и применению стохастических методов и 
информационных технологий для решения различных прикладных задач. В~работе 
М.\,П.~Кривенко рассматривается задача построения эмпирического байесовского 
классификатора, обеспечивающего распознавание текста, когда отдельные знаки имеют 
различные размеры. Представлен комбинированный метод построения оценки элементов 
байесовского классификатора, включающий непараметрическую ядерную оценку и 
параметрическую оценку с помощью плотности нормального распределения. В~статье 
К.\,В.~Рудакова и И.\,Ю.~Торшина разработан формализм для применения 
алгебраического подхода к проблеме распознавания вторичной структуры белка. 
Этот формализм позволил сформулировать математическое описание принятой 
у биологов гипотезы о локальном характере зависимости вторичной структуры от 
первичной и получить конструктивные критерии разрешимости задачи. Работа 
А.\,В.~Маркина и О.\,В.~Шес\-та\-ко\-ва посвящена решению задачи реконструкции 
изображения по радоновскому образу с помощью вейв\-лет-вейг\-лет разложения. Статьи 
О.\,В.~Крючина и С.\,Ю.~Степанова представляют результаты в области нейронных 
сетей. В~работе Е.\,Б.~Козеренко рассматриваются задачи создания лингвистических 
фильт\-ров в статистических моделях машинного перевода и совершенствования 
механизмов выравнивания параллельных текстов для повышения точности и адекватности 
переводов. 
      
      Две статьи посвящены развитию перспективных теоретических вероятностно-статистических 
      методов, которые могут найти широкое применение в различных задачах 
информатики и информационных технологий. В~работе В.\,Е.~Бенинга и Р.\,А.~Королева 
рассмотрены вопросы анализа статистических критериев проверки гипотез о параметрах 
обобщенного распределения Лапласа, которое применяется при 
математическом моделировании многих процессов в телекоммуникационных системах, в 
экономике, финансовом деле, технике и других областях. Работа 
М.\,Е.~Григорьевой и И.\,Г.~Шевцовой 
посвящена уточнению оценок абсолютной константы в оценке точности нормальной 
аппроксимации (такая аппроксимация весьма важна при математическом моделировании 
и анализе многих реальных систем, в том числе информационных и 
телекоммуникационных). 
      
      Редакционная коллегия журнала выражает надежду, что данный тематический  
выпуск будет интересен специалистам в области теории вероятностей и математической 
статистики и их применения к решению задач информатики и информационных 
технологий.
      
\vspace*{6mm}
\noindent
Заместитель главного редактора журнала <<Информатика и её применения>>,\\
директор ИПИ РАН, академик  \hfill
\textit{И.\,А.~Соколов}\\[-6pt]

\noindent
Редактор-составитель тематического выпуска, профессор кафедры математической статистики\\
факультета вычислительной математики и кибернетики МГУ им.~М.\,В.~Ломоносова,\\
ведущий научный сотрудник ИПИ РАН, доктор физико-математических наук\hfill
 \textit{В.\,Ю.~Королев}


} }
}
      