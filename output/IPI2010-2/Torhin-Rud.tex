\def\stat{tor}

\def\tit{ВОПРОСЫ РАЗРЕШИМОСТИ ЗАДАЧИ РАСПОЗНАВАНИЯ ВТОРИЧНОЙ 
СТРУКТУРЫ БЕЛКА$^*$}

\def\titkol{Вопросы разрешимости задачи распознавания вторичной 
структуры белка}

\def\autkol{К.\,В.~Рудаков, И.\,Ю.~Торшин}
\def\aut{К.\,В.~Рудаков$^1$, И.\,Ю.~Торшин$^2$}

\titel{\tit}{\aut}{\autkol}{\titkol}

{\renewcommand{\thefootnote}{\fnsymbol{footnote}}\footnotetext[1]
{Работа выполнена при поддержке РФФИ, гранты 09-07-12098, 09-07-00212-а и 09-07-00211-а.}}

\renewcommand{\thefootnote}{\arabic{footnote}}
\footnotetext[1]{ВЦ РАН им. А.\,А.~Дородницына, Московский физико-технический институт, 
rudakov@ccas.ru}
\footnotetext[2]{Российское отделение Института микроэлементов ЮНЕСКО, tiy135@yahoo.com}

%\vspace*{6pt}

\Abst{Цель работы~--- разработка формализма для последующего применения 
алгебраического подхода к проблеме распознавания вторичной структуры белка. Проведено 
формальное описание задачи, рассмотрена ее разрешимость, регулярность и локальность. 
Введены ключевые понятия для анализа локальности, такие как окрестность, маска, система 
масок, монотонность и тупиковость систем масок; предложен метод построения 
безызбыточных систем масок. Разработанный формализм позволил сформулировать 
математическое описание принятой у биологов гипотезы о локальном характере зависимости 
вторичной структуры от первичной и получить конструктивные критерии разрешимости 
задачи.}

%\vspace*{6pt}

\KW{алгебраический подход; вторичная структура белка; биоинформатика; окрестность; 
локальность; разрешимость; регулярность}

     \vskip 24pt plus 9pt minus 6pt

      \thispagestyle{headings}

      \begin{multicols}{2}

      \label{st\stat}

\section{Введение. Мотивация и~постановка проблемы}

    Клетка~--- мельчайшая структурная единица организма, а белки~--- 
активные молекулярные образования, поддерживающие жизнь клетки. 
В~современной биологии любой белок рассматривается с нескольких точек 
зрения:
\begin{enumerate}[(1)]
\item как одномерная аминокислотная последовательность (так на\-зы\-ва\-емая 
<<первичная структура белка>>, 1D); 
\item как одномерная последовательность 
характерных локальных конфигураций (<<вторичная структура>>, 2D); 
\item как 
трехмерный объект (<<третичная структура>>, <<пространственная 
структура>>, 3D);
\item как особый механизм, выполняющий определенную 
роль в функционировании клет\-ки~[1]. 
\end{enumerate}

    В настоящее время основным постулатом биологии белка является 
утверждение о том, что первичная структура однозначно определяет вторичную 
и третичную структуры, а третичная структура определяет биологическую роль 
белка. Поэтому основной задачей теоретической биологии белка считается 
установление закономерностей, определяющих взаимосвязь первичной и 
третичной структур. Как показал проведенный ранее анализ~[1], для решения 
этой задачи целесообразно решить промежуточную~--- установить взаимосвязь 
между первичным и вторичным уровнями структуры белка или решить задачу 
<<распознавания вторичной структуры белка>>. 

    Следует отметить, что имеющиеся данные о первич\-ном, вторичном и 
третичном уровнях описания структуры белка получены на основании 
существенно  разных экспериментов. Первичная структура 
(последовательность символов в 20-бук\-вен\-ном алфавите) устанавливается 
посредством <<секвенирования>> (дословно <<установления 
последова\-тельности>>)~--- процедуры последовательной хи\-мической 
деградации  молекулы белка. Вторичная структура (последовательность 
локальных конфигураций) и третичная структура (набор координат атомов) 
устанавливаются дифракционными методами (как правило, 
рентгеноструктурный анализ) или посредством исследования 
внутримолекулярного спин-спи\-но\-во\-го расщепления с использованием 
ЯМР (ядерного магнитного резо\-нанса).  

\begin{figure*} %fig1
\vspace*{1pt}
\begin{center}
\mbox{%
\epsfxsize=164.044mm
\epsfbox{rud-1.eps}
}
\end{center}
\vspace*{-9pt}
\Caption{Задача распознавания вторичной структуры белка
\label{f1tor}}
\end{figure*}

    В то время как точность секвенирования определяется однозначно как 
сов\-па\-де\-ние--не\-сов\-па\-де\-ние символов и достигает 100\%, трудно даже 
дать определение <<точности>> структурного эксперимента. Установленные 
исследователями координаты атомов молекулы белка (третичная структура) 
зависят не только от внутримолекулярных взаимодействий, но и от 
многочисленных условий проведения структурного эксперимента: выбора 
метода (дифракция, ЯМР), температуры, кислотности среды (pH), качества 
кристалла (в случае дифракционного метода), концентрации раствора белка (в 
случае ЯМР), присутствия других молекул и~т.\,д. Как результат, в разных 
экспериментах по определению структуры одного и того же белка 
устанавливаются отличающиеся друг от друга наборы  координат атомов 
молекул этого белка. 

    Экспертный анализ третичных структур белков указал на существование 
ряда характерных пространственных конфигураций локальных участков 
молекулы белка: <<спиралей>>, <<стрэндов>> и <<петель>>. 
Последовательности этих пространственных конфигураций были названы 
<<вторичной структурой белка>>. Вторичная структура как 
последовательность символов некоего алфавита (различные способы 
определения этого алфавита  рассмотрены далее)~--- результат интерпретации 
набора координат атомов молекулы белка. Так как координаты атомов и 
межатомные расстояния подвержены вариациям вследствие упомянутых выше 
особен\-ностей структурного эксперимента, то и вторичная структура как 
последовательность символов также подвержена вариациям. В~то же время 
существующие алгоритмы расчета вторичной структуры на основе третичной, 
основанные на эвристиках из проблемной области, дают результаты, более чем 
в 90\% случаев совпадающие с мнением экс\-пер-\linebreak та~[1--3]. 
    
В целом, распознавание вторичной структуры белка на основе его первичной 
структуры (амино\-кислотной последовательности)~--- одна из важней\-ших задач  
современной теоретической биологии. Актуальность задачи обусловлена 
значительным объемом данных по первичной структуре белка (миллионы 
аминокислотных последовательностей) и в сотни раз меньшим количеством 
экспериментальных данных по третичной и, следовательно, вторичной 
структуре белка. Это позволяет рассматривать накопленный материал о 
третичном и вторичном уровнях структуры белка как обуча\-ющую выборку для 
задачи распознавания вторичной структуры белка по первичной и применять 
алгебраический подход к проблемам распознавания~[4--12].  В~настоящей 
работе данная задача рассматривается как перевод последовательности 
символов из одного алфавита в другой (рис.~\ref{f1tor}).

     Особого внимания заслуживает локальность рассматриваемой задачи. 
Применение современных физических методов для исследования структуры и 
свойств белковых молекул (в частности, экспериментальные исследования 
наносекундной динамики белка, реалистичное моделирование молекулярной 
динамики свертывания белка и молекулярные механизмы биосинтеза 
     белка~\cite{1tor, 13tor, 14tor}), равно как  и существующие эвристики для 
<<предсказания вторичной структуры>>~\cite{3tor}, позволяют предположить 
локальный характер зависимости вторичной структуры от пер\-вичной.
     
     Таким образом, противоречивость экспериментальных данных, 
обусловленная особенностями структурного эксперимента, неоднозначность 
определения алфавита для описания вторичной структуры и необходимость 
систематического исследования гипотезы о локальном характере взаимосвязи 
между первичной и вторичной структурами указывают на целесообразность 
разработки специализированного формализма для корректной постановки 
изучаемой проб\-лемы. 
     
     Основная цель настоящей работы~--- разработать формализм для 
постановки задачи распознавания вторичной структуры белка в терминах 
современной теории распознавания~[4--12]. Особое внимание 
уделяется развитию формализма для тестирования гипотезы о локальном 
характере зависимости вторичной структуры белка от его первичной 
струк\-туры.
     
     Приводимые далее статистические оценки были сделаны на основании 
анализа общедоступных экспериментальных данных по первичной, вторичной 
и третичной структурам (на сегодняшний день имеются данные  
рентгеноструктурного анализа или ЯМР для более 50\,000~белков, помещенные 
в базу данных Protein Data Bank (PDB)~\cite{2tor}). 

\section{Введение. Об алфавитах для~описания вторичной 
структуры белка}

    В рамках разрабатываемого формализма используются два алфавита: 
алфавит~$A$ для описа\-ния первичной структуры белка (в дальнейшем 
<<верхнее слово>>) и алфавит~$B$ для описа\-ния вторичной структуры 
(<<нижнее слово>>). В~случае задачи распознавания вторичной структуры 
белка алфавит~$A$ определяется однозначно и соответствует множеству 
20~типов аминокислот, образующих цепи белков, $n=\vert A\vert 20$, $A = \{A,C,$\linebreak  
$D, E, F, G, H, I, K, L, M, N, P, R, S, T, V, W, Y\}$. Ал\-фа\-вит~$B$ может быть 
определен существенно раз\-личающимися способами: через использование\linebreak 
базовых алфавитов или через использование производных от них алфавитов. 
Ниже рассматриваются три способа определения алфавита~$B$:
\begin{enumerate}[(1)]
\item   базовые алфавиты; 
\item    производные алфавиты на основе последовательностей литер базового 
алфавита; 
\item расширение базового алфавита с учетом сегментов вторичной структуры 
нижнего слова. 
\end{enumerate}

    Во-первых, \textit{базовый алфавит} типа~$B$ можно определить как 
трехбуквенный, $m = \vert B\vert = 3$, $B = \{H, S, L\}$, где~$H$ (от 
\textit{англ}. helix) обозначает конфигурацию типа <<спираль>>, 
$S$ (strand)~--- конфигурацию <<стрэнд>> (плоский вытянутый участок 
белковой цепи) и~$L$ (loop)~--- участок произвольной структуры, т.\,е.\ 
не~$H$ и не~$S$. Частота встреча\-емости каждого из символов при 
использовании 3-бук\-вен\-но\-го алфавита отражена в табл.~1, 
примеры конфигураций~$H$, $S$, $L$ показаны на рис.~2.

    Трехбуквенный алфавит отображает принципиально различающиеся 
трехмерные пространственные конфигурации, которые может принимать тот 
или иной участок белковой цепи. На этом алфавите построено предыдущее 
поколение программ анализа структуры белка, которое не вполне точно 
отображает более тонкие геометрические различия известных конфигураций 
вторичной структуры. Так, при использовании более современных программ 
(основанных, прежде всего, на модификациях алгоритма STRIDE~\cite{13tor}), 
3-бук\-вен\-ный алфавит трансформируется в 8-бук\-вен\-ный. Хотя основными 
конфигурациями по-прежнему остаются~$H$, $S$ и~$L$, алфавит заметно 
усложняется (табл.~2). Как будет показано далее, сложность 
$B$-алфавита может сказаться на разрешимости рассматриваемой задачи 
распозна\-вания.

\bigskip

%\begin{center} %tabl1
\noindent
{{\tablename~1}\ \ \small{Частота встречаемости типов вторичной структуры при использовании 
3-буквенного алфавита~$B$ (АК~--- аминокислота)}}
%\end{center}
\vspace*{2pt}

{\small \begin{center}
%\tabcolsep=2pt
\begin{tabular}{|c|c|c|}
\hline
Литера&Тип вторичной структуры&Встречаемость\\
\hline
\multicolumn{3}{|c|}{Спирали}\\
\hline
$H$&$A$-спираль, 4 АК на виток&0,36\\
\hline
\multicolumn{3}{|c|}{Стрэнды}\\
\hline
$E (S)$&\tabcolsep=0pt\begin{tabular}{c}Стандартный стрэнд,\\ минимальная длина 2 АК\end{tabular}&0,22\\
\hline
\multicolumn{3}{|c|}{Петли}\\
\hline
$L$&Петля&0,42\\
\hline
\end{tabular}
\end{center}
}
%\vspace*{6pt}


%\bigskip
\addtocounter{table}{1}

\medskip

\begin{center} %fig2
\vspace*{12pt}
\mbox{%
\epsfxsize=69.946mm
\epsfbox{rud-2.eps}
}
\end{center}
\vspace*{4pt}
\begin{center}
{{\figurename~2}\ \ \small{Три основные формы вторичной структуры}}
\end{center}
\vspace*{9pt}


%\bigskip
\addtocounter{figure}{1}

\noindent
{{\tablename~2}\ \ \small{Частота встречаемости различных вариантов вторичной структуры в 8-бук\-вен\-ном 
алфавите~$B$ (АК~--- аминокислота)}}
%\end{center}
\vspace*{2pt}

{\small
\begin{center}
\tabcolsep=5pt
\begin{tabular}{|c|c|c|}
\hline
Литера&Тип вторичной структуры&Встречаемость\\
\hline
\multicolumn{3}{|c|}{Спирали}\\
\hline
$H$&$A$-спираль, 4 АК на виток&0,31\\
\hline
$G$&3$_{10}$-спираль, 3 АК на виток&0,04\\
\hline
$I$&$\pi$-спираль, 5 АК на виток&\hphantom{99}0,0002\\
\hline
\multicolumn{3}{|c|}{Стрэнды}\\
\hline
$E (S)$&\tabcolsep=0pt\begin{tabular}{c}Стандартный стрэнд,\\ минимальная\ длина 2 АК\end{tabular}&0,20\\
\hline
\multicolumn{3}{|c|}{Петли}\\
\hline
$L$&Петля&0,24\\
\hline
$T$&Поворот петли (turn)&0,11\\
\hline
$S^\prime$&Изгиб петли (bend)&0,09\\
\hline
$B$&Мост к стрэнду (bridge)&0,01\\
\hline
\end{tabular}
\end{center}}
\vspace*{12pt}

\addtocounter{table}{1}

    Во-вторых, возможно \textit{определение произ\-вод\-ного\linebreak алфавита}~$B$ 
через пары букв базового $B$-ал\-фа\-ви\-та. При этом последовательность из 
$N$~символов в $A$-алфавите переводится в последовательность из $N-1$~пар 
базового $B$-ал\-фа\-ви\-та.\linebreak Напри\-мер, для трехбуквенного базового алфавита~$H$, 
$S$, $L$ ($m = 3$) число пар составит $m^2 = 9$ и $B$-ал\-фа\-вит будет 
записываться как $\{b_{ij}\}$, $B\;=$\linebreak $=\;\{HH, HS, HL, SS, SH, SL, LL, LH, LS\}$. Для 
примера на рис.~\ref{f1tor} литере~$V$ в 1-й позиции верхнего слова будет 
соответствовать пара литер~$LL$ нижнего слова, литере~$T$ в 4-й позиции~--- 
пара литер~$HH$ и~т.\,д. Очевидно, что $B$-по\-сле\-до\-ва\-тель\-ность из 
$N-1$\linebreak\vspace*{-12pt}
\pagebreak

%\bigskip

\begin{center}
\noindent
\parbox{44.5mm}{{{\tablename~3}\ \ \small{Частота встре\-ча\-емости пар (трехбуквенный алфавит)}}
}
\end{center}
%\vspace*{2pt}

{\small
\begin{center}
\tabcolsep=7pt
\begin{tabular}{|c|c|}
\hline
Пары букв & \tabcolsep=0pt\begin{tabular}{c}Частота\\ встречаемости\end{tabular}\\
\hline
$HH$&0,33\\
$HS$&\hphantom{9}0,001\\
$HL$&0,03\\
$SS$&0,19\\
$SH$&\hphantom{9}0,003\\
$SL$&0,05\\
$LL$&0,32\\
$LH$&0,03\\
$LS$&0,05\\
\hline
\end{tabular}
\end{center}}
\vspace*{12pt}

\addtocounter{table}{1}

\noindent
двухбуквенных пар соответствует $N$~символам нижнего слова. 
Частота встречаемости каждой из пар представляет определенный интерес 
(табл.~3).


    Несколько важных выводов о структуре нижнего слова следует из данных 
табл.~3:
\begin{enumerate}[1.]
\item Пары~$HH$, $SS$ и~$LL$ встречаются наиболее часто и 
покрывают более 80\%  всех нижних слов. 
\item Пары $HL$, $LH$, $SL$, $LS$ встречаются на порядок реже  и 
соответствуют границам сегментов. Тем не менее границы сегментов 
встречаются довольно часто (0,03--0,05) и могут распознаваться 
отдельным семейством алгоритмов. 
\item Из пп.\,1 и~2 следует преобладание во вторичной структуре 
\textit{достаточно длинных однобуквенных сегментов} типа $HHHHH$, 
$SSSSS$, $LLLLL$. Частотные распределения длин сегментов показаны на 
рис.~3.
\end{enumerate}

%\vspace*{3pt}
\begin{center} %fig3
\vspace*{12pt}
\mbox{%
\epsfxsize=79.353mm
\epsfbox{rud-3.eps}
}
\end{center}
\vspace*{4pt}
%\begin{center}
{{\figurename~3}\ \ \small{Частота встре\-ча\-емости длин сегментов вторичной структуры. Пик в распределении 
$HHH$ при длине сегмента~3 соответствует спиралям типа~$G$ (8-бук\-вен\-ный алфавит): 
\textit{1}~--- $HHH$ (спирали); \textit{2}~--- $SSS$ (стрэнды); \textit{3}~--- $LLL$ (петли)}}
%\end{center}
%\vspace*{9pt}
%\columnbreak
%\bigskip
\addtocounter{figure}{1}


%\bigskip
\begin{center}
\noindent
\parbox{42mm}{{{\tablename~4}\ \ \small{Частота встре\-ча\-емости литер расширенного трех\-бук\-вен\-ного алфавита~$B$}}}
\end{center}
%\vspace*{2pt}

{\small
\begin{center}
\tabcolsep=8pt
\begin{tabular}{|c|c|}
\hline
Литера&\tabcolsep=0pt\begin{tabular}{c}Частота\\ встречаемости\end{tabular}\\
\hline
$H$&0,33\hphantom{9}\\
$S$&0,19\hphantom{9}\\
$L$&0,32\hphantom{9}\\
$H_S^b$&\hphantom{9}0,0015\\
$H_L^b$&0,015\\
$H_S^e$&\hphantom{9}0,0005\\
$H_L^e$&0,015\\
$S_H^b$&\hphantom{9}0,0005\\
$S_L^b$&0,025\\
$S_H^e$&\hphantom{9}0,0015\\
$S_L^e$& 0,025\\
$L_H^b$&0,015\\
$L_S^b$&0,025\\
$L_H^e$&0,015\\
$L_S^e$&0,025\\
\hline
\end{tabular}
\end{center}}
\vspace*{12pt}

\addtocounter{table}{1}

\begin{enumerate}[4.]\item Частоты переходов~$SH$ и~$HS$ крайне низки, т.\,е.\ границы 
между сегментами~$SSSSS$  и~$HHHHH$~--- редкое явление. 
С~физической точки зрения переходы~$SH$ и~$HS$ соответствуют 
резкой смене конфигурации и энергетически невыгодны вследствие 
Ван-дер-Вааль\-со\-ва отталкивания.
    \end{enumerate}

В-третьих, возможно \textit{расширение базового алфавита с учетом 
сегментной структуры нижних слов}. При этом базовый $B$-ал\-фа\-вит из 
$m$~литер трансформируется в ($m + 2m(m-1)$)-бук\-вен\-ный. Элементами 
расширенного алфавита являются элементы базового алфавита плюс $2m(m - 
1)$ элементов, соответствующих границам сегментов. Например, 
трехбуквенный базовый алфавит ($m = 3$) становится 15-бук\-вен\-ным 
расширенным $B$-ал\-фа\-ви\-том $B\;=$\linebreak $=\;\{ H,S,L,H_S^b, H_L^b, H_S^e,H_L^e, 
S_H^b,S_L^bS_H^e,S_L^e,L_H^b,L_S^b,$\linebreak $L_H^e,L_S^e\}$, частота встречаемости 
литер которого представлена в табл.~4.

    \begin{figure*}[b] %fig4
    \vspace*{1pt}
\begin{center}
\mbox{%
\epsfxsize=116.717mm
\epsfbox{rud-4.eps}
}
\end{center}
\vspace*{-9pt}
\Caption{Пример противоречивого набора экспериментальных данных и соответствующие 
условия экспериментов: ЯМР~--- ядерный магнитный резонанс (в скобках указано число 
модельных структур), рентген~--- рентгеноструктурная кристаллография белка, pH~--- 
кислотность среды, $T$~--- температура, $+$~ДНК~--- структура была определена в 
комплексе с фрагментом дезоксирибонуклеиновой кислоты:  \textit{1}~--- ЯМР (35~стр.), 
$\mathrm{pH} = 4{,}5$, $T = 300$~K; 
\textit{2}~--- рентген, $\mathrm{pH} = 6{,}5$, $T = 123$~K, $+$~ДНК;
\textit{3}~--- ЯМР (50~стр.), $\mathrm{pH} = 5{,}0$, $T = 323$~K, $+$~ДНК;
\textit{4}~--- рентген, $\mathrm{pH} = 6{,}5$, $T = 287$~K, $+$~ДНК
\label{f4tor}}
\end{figure*}


    В целом граница между сегментами $\ldots LLL$ и $HHH\ldots$ 
описывается четырьмя различными типами $B$-ли\-тер: $\ldots 
LLL_H^eH_L^bHH\ldots$\ \ С~физической\linebreak точки зрения такой расширенный 
алфавит позволяет подчеркнуть различия в пространственных конфигурациях 
индивидуальных аминокислотных остатков вдоль белковой цепи. Например, 
литера~$L_H^e$ расширенного алфавита соответствует последней 
(\textit{англ}. \textbf{e}nd) литере сегмента петли $\ldots LLL$, который 
переходит в спиральный $HHH\ldots$\ сегмент. Соответственно, 
литера~$L_H^b$ расширенного алфавита соответствует первой (\textit{англ}. 
\textbf{b}eginning) литере сегмента петли~$\ldots LLL$, которому предшествует 
спиральный сег\-мент~$\ldots HHH$.

    Возможно использование еще более сложных $B$-ал\-фа\-ви\-тов: 
трехбуквенных, четырехбуквенных и более сложных комбинаций литер 
базового алфавита, использование расширенных алфавитов на основе таких 
комбинаций и~т.\,д. Так или иначе, \textit{предлагаемый в настоящей работе 
математический аппарат применим при использовании любого способа 
определения} $B$-\textit{ал\-фа\-вита}. 

    Выбор того или иного способа определения $B$-ал\-фа\-ви\-та важен для 
конкретных реализаций решения данной задачи распознавания. Ниже будем 
говорить, что алфавит~$B^\prime$ есть \textit{расширение} алфавита~$B$, если 
существует такая функция $f: B^\prime\rightarrow B$, что ею определяется 
однозначный перевод всех слов алфавита~$B^\prime$ в соответствующие слова 
ал\-фа\-ви\-та~$B$.

%\vspace*{-12pt}

\section{Исходные определения}

Пусть заданы два алфавита: $A$ и~$B$, $A =$\linebreak $= \{a_1, a_2, \ldots , a_n\}$, $n > 0$, 
$B = \{b_1, b_2, \ldots , b_m\}$, $m > 0$. Обозначим множества слов длины~$k$ в 
каждом из алфавитов~$A^k$ и~$B^k$ соответственно. Тогда множество всех 
исходных слов в алфавите~$A$ есть $A^* =\bigcup\protect\limits_{l=1}^\infty A^l$, а 
множество всех слов в алфавите~$B$ есть $B^*=\bigcup\limits_{l=1}^\infty B^l$. 
Решение исследуемой задачи распознавания сводится к поиску некоторой 
функции $F: A^* \rightarrow B^*$, причем $\vert F(\vec{a})\vert =\vert 
\vec{a}\vert$ ($\vert\vec{a}\vert$~--- длина слова~$\vec{a}$). Пусть $\mathbf{F} 
= \{F: A^*\rightarrow B^*;\ \vert F (\vec{a})\vert =\vert\vec{a}\vert\}$~--- 
множество функций этого типа.

    Пусть $\Delta$~--- неопределенность. Введем $\Delta$-рас\-ши\-рен\-ные 
алфавиты $\tilde{A}=A\bigcup\{\Delta\}$ и $\tilde{B}=B\bigcup\{\Delta\}$ и 
$\Delta$-рас\-ши\-рен\-ные множества слов $\tilde{A}^*=\bigcup\limits_{l=1}^\infty 
\tilde{A}^l$ и  $\tilde{B}^*=\bigcup\limits_{l=1}^\infty \tilde{B}^l$ соответственно. 

Пусть задано конечное множество прецедентов $\mathbf{Pr}\subseteq 
\tilde{A}^*\times \tilde{B}^*$, $\mathbf{Pr}\not= \varnothing$, где~<<$\times$>> 
обозначает декартово произведение. Прецедент, таким образом, представляет 
собой пару слов $\left(\vec{a},\vec{b}\right) \in \mathbf{Pr}$, $\vert\vec{a}\vert 
=\vert\vec{b}\vert$. Назовем $V=\vec{a}$ <<верхним словом>>, а 
$Q=\vec{b}$~--- <<нижним словом>> прецедента. Будем называть 
функцию~$F$ \textit{корректной}, если $\forallb\limits_{\mathbf{Pr}} 
\left(\vec{a},\vec{b}\right) :\ F\left(\vec{a}\right)=\vec{b}$. Тогда оче\-видна 

\medskip

\noindent
\textbf{Теорема 1.} Функция $F$ \textit{существует тогда и только тогда, когда} 
\begin{equation}
\forallb\limits_{\mathbf{Pr}}\left (\vec{a}_1,\vec{b}_1\right), 
\left(\vec{a}_2,\vec{b}_2\right):\ \left (\vec{a}_1=\vec{a}_2\right) \Rightarrow 
\left(\vec{b}_1=\vec{b}_2\right)\,.
\label{1tor}
\end{equation}

\medskip

    В соответствии с теоремой~1 существование корректной функции~$F$ 
(т.\,е.\ разрешимость ис\-сле\-ду\-емой задачи) зависит от выбора 
множества~$\mathbf{Pr}$.  Таким образом, в жесткой постановке задача о 
распознавании вторичной структуры разрешима при заданном множестве 
прецедентов~$\mathbf{Pr}$, если для нее существует корректное решение~$F$, 
т.\,е.\ верхнему слову каждого прецедента однозначно поставлено в 
соответствие нижнее слово. Все возможные $\mathbf{Pr}\in \tilde{A}^*\times 
\tilde{B}^*$, $\mathbf{Pr}\not=\varnothing$ подразделяются на те, на которых 
задача разрешима, и те, на которых задача не\-раз\-ре\-шима.

    В идеале некое множество~$\mathbf{Pr}$, на котором задача разрешима, 
должно включать все известные структуры белков (например, все данные из 
PDB~\cite{2tor}). Но в такой формулировке существование корректной~$F$, к 
сожалению, опровергается наличием в базах данных о структуре белков 
примеров прецедентов, в которых совпадают верхние слова и не совпадают 
нижние. Прецеденты с одинаковыми верхними, но отличающимися нижними 
словами возникают в результате параллельной работы разных 
исследовательских групп, использования существенно различающихся 
физических методов установления структуры белка, разных условий 
структурного эксперимента, разных способов выделения и очистки белка 
и~т.\,д. На рис.~\ref{f4tor} приведены примеры таких противоречивых 
пре\-це\-дентов. 


    По теореме~1 при наличии противоречивых прецедентов корректных 
функций~$F$ не существует. Анализ противоречивых прецедентов в реальных 
данных и ка\-кая-ли\-бо форма обработки этих противоречивых прецедентов 
(исключение, усреднение, нахождение медианы, составление профиля\linebreak 
вероятностей, введение весов позиций, использование данных об условиях 
структурного эксперимента и~т.\,п.) представляет собой отдельное направление 
исследований. В~дальнейшем будут\linebreak рассматриваться только \textit{множества 
непротиворечивых прецедентов}, для которых существует корректная 
функ\-ция~$F$. 

    Исследуемую задачу~$Z$, определяемую множеством 
прецедентов~\textbf{Pr}, будем  называть \textit{разрешимой}, если для нее 
существует корректная функция~$F$, т.\,е.\ выполнено условие~(1) 
разрешимости задачи. Наряду с разрешимостью в современной теории 
распознавания~ [4--12] обычно изучается \textit{регулярность} 
задач. Под регулярностью задачи понимается разрешимость самой задачи, 
со\-про\-вож\-да\-ющаяся разрешимостью задач из некоторой ее окрестности в 
изучаемом множестве \mbox{задач}. 

    Понятие регулярности определяется тем, как задаются окрестности 
задачи. Следуя идеологии научной школы академика Ю.\,И.~Журавлева, 
определим окрестность задачи~$Z$ с множеством прецедентов $\mathbf{Pr} =\{ 
\left (\vec{a}_1,\vec{b}_1\right), \left(\vec{a}_2,\vec{b}_2\right), \ldots , 
\left(\vec{a}_q,\vec{b}_q\right) \}$ как множество задач~$Z^\prime$ с множеством 
прецедентов $\mathbf{Pr}^\prime =\{ \left (\vec{a}_1,\vec{b}_1^\prime\right), 
\left(\vec{a}_2,\vec{b}_2^\prime\right), \ldots , \left(\vec{a}_q,\vec{b}_q^\prime\right) \}$ 
при произвольных  $\vec{b}_1^\prime, \vec{b}_2^\prime, \ldots , 
\vec{b}_q^\prime$. Отсюда следует, что задача~$Z$ будет регулярной на 
множестве прецедентов~\textbf{Pr} тогда и только тогда, когда выполняется 
следующее \textit{условие регулярности}: 
\begin{equation}
\forallb\limits_{\mathbf{Pr}}\left (\vec{a}_i, \vec{b}_i\right),  \left (\vec{a}_j, 
\vec{b}_j\right), \left(i\not=j\right)\Rightarrow \left(\vec{a}_i\not= \vec{a}_j\right)\,.
\label{e2tor}
\end{equation}

\section{Локальность}

Предлагаемый формализм разрабатывается с целью тестирования гипотезы о 
локальном характере взаимосвязи между первичным и вторичным уровнями 
структуры белка. В~рамках формализма \textit{локальность} означает то, что 
каждая литера нижнего слова определяется неким подсловом верхнего слова. 
Пусть дано слово $\vec{U}=\{u_1,u_2, \ldots ,u_n\}$ длины~$n$. Это может 
быть верхнее слово~($V$) или нижнее слово~($W$). Определим некую 
\textit{ведущую позицию}~$i$, $1 \leq i \leq n$. Дана также <<маска>> 
$\hat{m}=\{\mu_1,\mu_2,\ldots ,\mu_m\}$, где $\mu_i\in Z$, 
$\mu_1<\mu_2<\ldots <\mu_m$. Будем называть $\mu_i$ \textit{позициями 
маски}. Параметр~$m$ назовем \textit{раз\-мер\-ностью  маски}~$\hat{m}$ и 
будем обозначать как~$\vert \hat{m}\vert$, а параметр $\mu_m-\mu_1+1$ 
назовем \textit{протяженностью маски} и обозначим как~[$\hat{m}$]. 
Определим \textit{оператор выбора подслова}~$\eta (i, \hat{m}, \vec{U})$:
\begin{equation*}
\eta (i, \hat{m}, \vec{U}) =
\begin{cases}
u_{i+\mu_1}u_{i+\mu_2}\ldots u_{i+\mu_m}\,, &\\
& \hspace*{-30mm}\mbox{если } i+\mu_1\geq 1\,,\enskip i+\mu_m\leq n\,;\\
\varnothing &\hspace*{-30mm} \mbox{в противном случае.}
\end{cases}
%\label{e3tor}
\end{equation*}

    Иначе говоря, оператор~$\eta$ выбирает определенную 
подпоследовательность слова~$\vec{U}$ по маске~$\hat{m}$, помещенной на 
позицию~$i$. \textit{Подсловом} или $(\hat{m},i)$-\textit{под\-сло\-вом} будем 
называть конкретное значе\-ние оператора~$\eta$ на определенной позиции~$i$ 
не\-которого слова~$\vec{U}$, выбранное по маске~$\hat{m}$.\linebreak С~точки зрения 
алгебраического подхода пара <<маска>>\,--\,<<ведущая позиция>> может 
быть рассмотрена как аналог <<опорного множества>>, а 
$(\hat{m},i)$-под\-сло\-во~--- как аналог <<представительного на\-бора>>.

    Пусть имеется система масок 
$$
M=\{ \hat{m}_1, \hat{m}_2, \ldots , 
\hat{m}_N\}\,.
$$
Будем считать, что 
\begin{multline*}
\hat{m}_1=\left( \mu_1^1,\mu_2^1, \ldots , 
\mu^N_{\vert m_N\vert}\right), \ldots ,\\
 \hat{m}_N=\left ( \mu_1^N, \mu_2^N, 
\ldots , \mu^N_{\vert m_N\vert}\right )\,.
\end{multline*} 

Определим одноэлементную систему 
масок~$\hat{M}_\Sigma (M)$ как \textit{объединенную маску}~$\hat{m}$ такую, 
что $\hat{m} = \bigcup\limits_{k=1}^{\vert M\vert}\hat{m}_k$. Очевидно, что
\begin{equation*}
\forallb\limits_{k=1}^{\vert M\vert} \hat{m}_k:\ \left (\left | \hat{m}_k:\right | \leq 
\left | \hat{m}_\Sigma\right |\right)\&\left(\left [ \hat{m}_k\right]\leq \left [ 
\hat{M}_\Sigma\right]\right )\,.
%\label{e4tor}
\end{equation*}

Слова в множестве прецедентов~\textbf{Pr} имеют конечную длину, поэтому 
для точности изложения следует описать краевые эффекты с учетом области 
определения оператора~$\eta$. Пусть~$L$, $R\;\in$\linebreak $\in\;N\bigcup\{0\}$. Функцию~$F$ 
назовем ($L, R$)-\textit{корректной}, если для $\forall 
\left(\vec{a},\vec{b}\right)\in\mathbf{Pr}$ выполнено 
$F\left (\vec{a}\right)=\vec{b}^\prime$, где $b_1^\prime =b_2^\prime =\ldots = 
b_L^\prime =\Delta$, $b^\prime_{\vert \vec{a}\vert -R+1}=\ldots = 
b^\prime_{\vert\vec{a}\vert}=\Delta$ и $b_i^\prime = b_i$ при $L<i\leq \left 
|\vec{a}\right |-R$. Иначе говоря, для прецедента~$\left(\vec{a},\vec{b}\right)$ 
$F$ является ($L, R$)-корректной, если она вычисляет слово~$\vec{b}$ с 
точностью до $L, R$ от краев нижнего слова.

Считаем, что имеется система масок~$M$. Можно предложить два 
существенно различающихся способа определения границ для описания 
краевых эффектов. В~первом случае~$L(M)$ и~$R(M)$ определяются как 
минимальные отступы от краев верхнего слова (слева и справа соответственно), 
при которых применимы все маски из~$M$. При этом $L(M) + 1 = \min (i) :\ 
\forallb\limits_{k=1}^N\left( i+\mu_1^k\geq 1\right)$, а $R(M) = \left | \vec{a}\right | 
-\max (i):\ \forallb\limits_{k=1}^N \left(i+\mu^k_{\vert m_k\vert}\leq \left 
|\vec{a}\right |\right )$. Тогда $L(M) = \max\left(-\underset{k=1,N}{\min}\mu_1^k, 
0\right)$ и аналогично $R(M) = \max\left ( \underset{k=1,N}{\max} \mu^k_{\vert 
m_k\vert},0\right)$.

    Во втором случае границы определяются как такие минимальные 
значения~$i$ (значения ведущей позиции), при которых применима, по крайней 
мере, одна маска из~$M$. Обозначим левую границу (левый отступ) 
как~$l(M)$. В~этом случае $l(M) + 1 = \min (i):\ \existsb\limits_{k=1}^N 
\left(i+\mu_1^k=1\right)$, и тогда $l(M)  = \max \left ( -\underset{k=1,N}{\max} 
\mu_1^k,0\right)$. Аналогично правая граница~$r(M)$ определяется как~$r(M)=$\linebreak 
$=\max \left (\underset{k=1,N}{\min} \mu^k_{k=1,N},0\right)$.

\medskip

\noindent
\textbf{Теорема 2.} ($l(M), r(M)$)-\textit{корректная функция также}  
($L(M), R(M$))-\textit{корректна}. 

\medskip

\noindent
Д\,о\,к\,а\,з\,а\,т\,е\,л\,ь\,с\,т\,в\,о\,.\ Нетрудно показать, что $l(M) \leq L(M)$. 
Рассмотрим произвольное множество масок, первый элемент каждой 
маски~$\mu_1^k$, множество всех первых элементов~$\{\mu_1^k\}$, 
$\mu_1^k\in Z$. $L(M)$ определяется через поиск минимума~$\mu_1^k$, $-
\underset{k=1,N}{\min} \mu_1^k$, а~$l(M)$ определяется через 
максимум~$\mu_1^k$, $-\underset{k=1,N}{\max} \mu_1^k$. Очевидно, что 
минимум не может превышать максимум на одном и том же множестве целых 
чисел, так что $\underset{k=1,N}{\min}\mu_1^k\leq 
\underset{k=1,N}{\max}\mu_1^k$, $-\underset{k=1,N}{\min}\mu_1^k \geq -
\underset{k=1,N}{\max}\mu_1^k$ и, следовательно, $l(M) \leq L(M)$. 
Аналогично $r(M) \leq R(M)$. В~случае любой ($L, R$)-кор\-рект\-ной функции 
$F\left(\vec{a}\right)=\vec{b}^\prime$, где $b_1^\prime =b_2^\prime=\ldots 
=b_L^\prime-\Delta$, $b^\prime_{\vert\vec{a}\vert -R+1}=\ldots 
=b^\prime_{\vert\vec{a}\vert}=\Delta$ и $b_i^\prime=b_i$ для любых 
$\left(\vec{a},\vec{b}\right)\in\mathbf{Pr}$. Б$\acute{\mbox{о}}$льшие 
значения~$L$ или~$R$ со\-от\-ветствуют б$\acute{\mbox{о}}$льшему числу 
ведущих позиций, в которых $b_i^\prime=\Delta$ при $i<L$ и при 
$i>\vert\vec{a}\vert -R$. Поэтому, если функция~$F$ является 
($l(M),r(M))$-кор\-рект\-ной, ($L(M),R(M)$)-кор\-рект\-ность этой же 
функции прос\-то соответствует увеличению числа позиций, в которых 
$b_i^\prime =\Delta$, так как $l(M) \leq L(M)$ и $r(M) \leq R(M)$. 
Неопределенность ($\Delta$) не влияет на корректность функции, так как не 
нарушает условия $F\left(\vec{a}\right)=\vec{b}^\prime$. Теорема до\-ка\-зана.

\medskip

    ($L,R$)-корректность~$F$ важна для практических применений 
предлагаемого формализма. Так, ($l(M),r(M)$)-кор\-рект\-ность гарантирует 
корректность распознавания на концах верхнего слова, а 
($L(M),R(M)$)-кор\-рект\-ность необходима для выбора безызбыточных сис\-тем 
масок (см.\ далее).

\section{Условие существования локальных функций}

    Гипотеза о локальности исследуемой задачи распознавания 
формулируется ниже как гипотеза о существовании некоторой локальной 
функции. Пусть дана система масок~$M$. Определим класс корректных 
локальных функций $f(M)\subseteq \mathbf{F}$. Функция~$F$ принадлежит 
$f(M)$ тогда и только тогда, когда существует функция~$f:\ \left(\tilde{A}\right) 
^{\vert \hat{M}_\Sigma(M)\vert} \rightarrow \tilde{B}$ такая, что для любого 
$\vec{a}=(a_1, \ldots ,a_n)$ выполнено $F(\vec{a})=\vec{b}=(b_1, \ldots , b_n)$, где 
$b_1=b_2=\ldots =$\linebreak $= b_{l(M)}=\Delta$, $b_{n-r(M)+1}=\ldots = b_n=\Delta$, а при 
$l(M)<i\leq n-r(M)$ выполнено условие
\begin{equation}
b_i  =f\left(\eta\left( i \hat{M}_\Sigma(M),\vec{a}\right)\right)\,.
\label{e5tor}
\end{equation}

Отметим, что при построении корректных алгоритмов с использованием 
конструкций алгебраического подхода функция~$f$ будет искаться в виде:
\begin{multline*}
f\left(\eta \left(i, \hat{M}_\Sigma (M),\vec{a}\right)\right) ={}
\\
{}=
g\left( h_1\left(\eta \left(i, \hat{m}_1,\vec{a}\right)\right), \ldots , h_{\vert M\vert} 
\left( \eta\left (i, \hat{m}_{\vert M\vert}, \vec{a}\right)\right)\right)\,.
%\eqno{5'}
\end{multline*}

Таким образом, условие разрешимости задачи распознавания вторичной 
структуры белка может быть сформулировано следующим образом: дано такое 
множество масок~$M$ и множество прецедентов~\textbf{Pr}, что существует 
локальная функция $F\in f(M)$ такая, что $F$ корректна на данном множестве 
прецедентов~\textbf{Pr}. Очевидно, что приведенные выше соображения о 
выборе~\textbf{Pr} и о существовании противоречивых прецедентов в реальных 
выборках данных относятся и к локальной формулировке за\-дачи.

    При поиске решения $F\in f(M)$ естественно требовать именно 
($l(M),r(M)$)-кор\-рект\-ности. \textit{Критерием локальной разрешимости} 
будем называть следующее условие:
\begin{multline}
\!\!\!\!\!\!\forallb\limits_{\mathbf{Pr}} \left(\vec{V}_1,\vec{W}_1\right)\!, \!
\left(\vec{V}_2,\vec{W}_2\right)\! \forallb\limits_{i,j\in N} (i,j): \ \eta \left(i 
\hat{M}_\Sigma (M),\vec{V}_1\right)\!=\\
{}=\eta\left( j, 
\hat{M}_\Sigma(M),\vec{V}_2\right) \Rightarrow W_1^i=W_2^j\,,\\ 
l(M)<i\leq \left | \vec{V}_1\right | -r(M)\,,\\ l(M)<j\leq \left | \vec{V}_2\right | -
r(M)\,,\quad i\not=j\,.
\label{e6tor}
\end{multline}
        
\noindent
\textbf{Теорема 3.} \textit{Локальная функция~$F$, корректная на множестве 
прецедентов}~\textbf{Pr}\textit{, существует тогда и только тогда, когда 
выполняется критерий локальной разрешимости}. 

\medskip

\noindent
Д\,о\,к\,а\,з\,а\,т\,е\,л\,ь\,с\,т\,в\,о\,.\ Предположим, что условие~(\ref{e6tor}) не 
выполняется и в~\textbf{Pr} существует хотя бы одна пара прецедентов, для 
которой при некоторых значениях~$i$ и~$j$ окрестности 
по~$\hat{M}_\Sigma(M)$ в верхнем слове идентичны  $\eta (i, \hat{M}_\Sigma , 
\vec{V}_1) =\eta (j, \hat{M}_\Sigma , \vec{V}_2)=\eta^\prime$, а литеры нижнего 
слова различаются: $W_1^i\not= W_2^j$. Тогда в соответствии с~(\ref{e5tor}) 
получается, что $W_1^i=f(\eta^\prime)$ и $W_2^j=f(\eta^\prime)$, т.\,е.\ 
происходит нарушение тождественности: $f(\eta^\prime)\not=f(\eta^\prime)$. 
Следовательно, критерий~(\ref{e6tor}) является необходимым условием 
существования $F\in f(M)$. Теорема до\-ка\-зана.

\medskip

\noindent
\textbf{Следствие 1.} \textit{Из теорем~2 и~3 следует критерий локальной} 
($L(M),R(M)$)-\textit{разрешимости}
\setcounter{equation}{3}\renewcommand{\theequation}{\arabic{equation}$^\prime$}
\begin{multline*}
\!\!\!\!\!\!\forallb\limits_{\mathbf{Pr}}\! \left(\vec{V}_1,\vec{W}_1\right)\!, \!
\left(\vec{V}_2,\vec{W}_2\right)\!\! \forallb\limits_{i,j\in N} \left(i,j\right):\ \eta \left(i, 
\hat{M}_\Sigma(M), \vec{V}_1\right)\! =\hspace*{-0.80466pt}\\
{}=\eta\left( j, \hat{M}_\Sigma(M), 
\vec{V}_2\right) \Rightarrow W_1^i=W_2^j\,,
\end{multline*}
\vspace*{-12pt}

\noindent
\begin{multline}
L(M)<i\leq \left |\vec{V}_1\right | -R(M)\,,
\\
L(M)<j\leq \left | \vec{V}_1\right | -
R(M)\,,\quad i\not= j\,.
%\eqno{(4')}
\end{multline}
\setcounter{equation}{4}\renewcommand{\theequation}{\arabic{equation}}
    
\medskip

\noindent
\textbf{Следствие 2.} \textit{Расширение B-ал\-фа\-ви\-та может приводить к 
потере разрешимости задачи}~$Z(\mathrm{Pr}, M)$. 

\medskip

$B$-алфавит может быть расширен за счет использования более сложного 
базового алфавита (3-бук\-вен\-но\-го, 8-бук\-вен\-но\-го), использования пар 
литер базового алфавита, а также за счет расширения с учетом сегментной  
структуры нижних слов (см.\ разд.~1). Расширенный 
алфавит~$B^{\mathrm{р}}$ может быть однозначно переведен в базовый 
алфавит~$B^{\mathrm{б}}$ посредством функции $f_B:\ 
B^{\mathrm{р}}\rightarrow B^{\mathrm{б}}$. Например, в случае 3- и 
8-бук\-вен\-ных базовых алфавитов (см.\ табл.~2) функция~$f_B$ 
определяется как $f_B:\ \{H,G,I\rightarrow H;\ S\rightarrow S;\ 
T,S^{\prime\prime},B,L\rightarrow L\}$. Тогда, если есть разрешимость в 
алфавите~$B^{\mathrm{р}}$, она существует и в алфавите $B^{\mathrm{б}} 
=f_B(B^{\mathrm{р}})$, так как $\underset{\mathrm{б}}{W_1^i} 
f_B(\underset{\mathrm{р}}{W_1^i})$ и $\underset{\mathrm{б}}{W_2^j} 
f_B(\underset{\mathrm{р}}{W_2^j})$. Обратное неверно: некоторым литерам 
из~$B^{\mathrm{б}}$ соответствуют несколько литер~$B^{\mathrm{р}}$, так 
что не существует $f_B:\ B^{\mathrm{б}}\rightarrow B^{\mathrm{р}}$. 
Следовательно, хотя из разрешимости задачи в алфавите~$B^{\mathrm{р}}$ 
следует разрешимость в~$B^{\mathrm{б}}$, из разрешимости в более прос\-том 
алфавите~$B^{\mathrm{б}}$ не следует разрешимость в более 
сложном~$B^{\mathrm{р}}$, т.\,е.\ при расширении алфавита может 
происходить потеря раз\-ре\-ши\-мости за\-дачи.

    По теореме~3 критерий локальной раз\-ре\-ши\-мости является необходимым 
и достаточным условием существования~$F$. На основе условия~(\ref{e6tor}) 
становится возможным проведение экспериментов по оценке параметров 
индивидуальных масок, при которых существование~$F$ воз\-можно. 

    Введем \textit{критерий локальной разрешимости} с использованием 
отдельных масок:
\setcounter{equation}{3}\renewcommand{\theequation}{\arabic{equation}$^{\prime\prime}$}
\begin{multline*}
\forallb\limits_{\mathbf{Pr}} \left(\vec{V}_1,\vec{W}_1\right), \left( \vec{V}_2, 
\vec{W}_2\right)\\ \forall (i,j) \left( \forallb\limits_{k=1}^{\vert M\vert} \hat{m}_k:\ 
\eta \left(i,\hat{m}_k, \vec{V}_1\right) ={}\right.
\end{multline*}
\begin{multline}
\left.{}=
\eta\left(j,\hat{m}_k,\vec{V}_2\right)\right) 
\Rightarrow W_1^i=W_2^j\,,
\\
l(M)<i\leq \left|\vec{V}_1\right| -r(M)\,,\\ l(M)<j\leq \left|\vec{V}_2\right|-
r(M)\,,\enskip
 i\not=j\,.
%\eqno{(4'')}
\end{multline}

\setcounter{equation}{4}\renewcommand{\theequation}{\arabic{equation}}

\noindent
\textbf{Теорема 4.} \textit{Условия}~(4) \textit{и}~(4$^{\prime\prime}$) \textit{эквивалентны}.

\medskip

\noindent
Д\,о\,к\,а\,з\,а\,т\,е\,л\,ь\,с\,т\,в\,о\,.\ Любые два подслова $\vec{v}^1\;=$\linebreak $=\;\{v_1^1, 
v_2^1,\ldots , v_m^1\}$ и $\vec{v}^2=\{v_1^2, v_2^2, \ldots , v_m^2\}$ равны, 
$\vec{v}^1=\vec{v}^2$, когда совпадают литеры в каждой позиции, т.\,е.\ 
$\forallb\limits_{i=1}^m v_i^1=v_i^2$. Поэтому выражение $\eta (i, 
\hat{M}_\Sigma (M), \vec{V}_1)=\eta (j,\hat{M}_\Sigma(M),\vec{V}_2)$ 
в~(\ref{e6tor}) соответствует системе равенств~$S$ такой, что $\forall \mu \in 
M_\Sigma(M):\ v^1_{i+\mu}=v^2_{j+\mu}$, т.\,е.
$$
S = \begin{cases}
v^1_{i+\mu_1} = v^2_{j+\mu_1}\,;\\
v^1_{i+\mu_2} = v^2_{j+\mu_2}\,;\\
\ldots\,;\\
v^1_{i+\mu_m} = v^2_{j+\mu_m} \,,
\end{cases}
$$
где  $M_\Sigma(M)=\{\mu_1,\mu_2,\ldots , \mu_m\}$, $m=\left| M_\Sigma(M)\right |$

    Очевидно, что каждому из $\mu_1, \mu_2, \ldots , \mu_m$ соответствует 
определенное равенство в~$S$. Любой $\hat{m}_k \subset M_\Sigma(M)$ 
соответствует подсистема равенств~$s_k$, которая может быть записана как 
$\eta (i, m_k \vec{V}_1) =\eta(j,m_k,\vec{V}_2)$. Так как 
$M_\Sigma(M)=\bigcup\limits_{k=1}^{\vert M\vert} \hat{m}_k$, то 
$S=\bigcup\limits_{k=1}^{\vert M\vert} s_k$, так что выражения $\eta (i, 
\hat{M}_\Sigma(M), \vec{V}_1)=\eta(j,\hat{M}_\Sigma(M),\vec{V}_2)$ и $\left ( 
\forallb\limits_{k=1}^{\vert M\vert } \hat{m}_k:\ \eta(i,\hat{m}_k, 
\vec{V}_1)=\eta(j,\hat{m}_k,\vec{V}_2)\right)$~--- эквивалентны. Последнее 
соответствует эквивалентности утверждений~(\ref{e6tor}) 
и~(4$^{\prime\prime}$). Теорема доказана.

\medskip

    Задачу $Z(Pr, M)$ будем  называть локально разрешимой на~\textbf{Pr}, 
если для нее существует ($l(M),r(M)$)-кор\-рект\-ная функция $F\in f(M)$. 
Таким образом, условия~(\ref{e6tor})--(4$^{\prime\prime}$) 
определяют наличие у задачи~$Z$ свойства \textit{локальной разрешимости}.

Критерием, определяющим наличие у задачи свойства \textit{локальной 
регулярности}, будем называть следующее условие:  
\begin{multline}
\!\!\forallb\limits_{\mathbf{Pr}} \!\left(\vec{V}_1,\vec{W}_1\right), 
\left(\vec{V}_1,\vec{W}_2\right) \forallb\limits_{i,j\in N}(i,j), i\not=j,\ 
\vec{V}_1\not=\vec{V}_2\Rightarrow {}\\
{}\Rightarrow \eta \left 
(i,M_\Sigma(M),\vec{V}_1\right)\not= \eta\left(j,M_\Sigma(M),\vec{V}_2\right)\,,\\
l(M)<i\leq \left | \vec{V}_1\right | -r(M)\,,\\
l(M)<j\leq \left |\vec{V}_2\right |-
r(M)\,.
\label{e7tor}
\end{multline}
        
Отметим, что из условия~(\ref{e7tor}) и теоремы~2 следует 
\begin{multline*}
\!\!\forallb\limits_{\mathbf{Pr}}\! \left(\vec{V}_1,\vec{W}_2\right),\left 
(\vec{V}_2,\vec{W}_2\right)\forallb\limits_{i,j\in N} (i,j), i\not=j, 
\vec{V_1}\not=\vec{V}_2\Rightarrow{}\\
{}\Rightarrow \eta\left(i, M_\Sigma(M), 
\vec{V}_1\right)\not=\eta \left( j, M_\Sigma(M), \vec{V}_2\right)\,,
\\
L(M)<i\leq \left |\vec{V}_1\right |-R(M)\,,\\ L(M) <j\leq \left |\vec{V}_2\right |-
R(M)\,,\quad i\not=j\,.
%\eqno{7''}
\end{multline*}

    Локальная регулярность $Z(Pr, M)$~--- предельный случай локальной 
разрешимости  этой задачи. Как и условие регулярности~(\ref{e2tor}), 
\textit{критерий локальной регулярности} позволяет сформулировать условие 
разрешимости задачи~$Z$ с учетом гипотезы о локальности. В~случае 
распознавания вторичной структуры белка регулярность задачи может быть 
достигнута при использовании систем масок избыточно высокой размерности и 
протяженности. 

\vspace*{6pt}

\section{Разрешимость задачи и~монотонность условия 
разрешимости}

При выполнении условия существования локальных 
функций~(\ref{e6tor})--(4$^{\prime\prime}$), задача распознавания вторичной 
структуры \textit{разрешима}, в противном случае~--- \textit{неразрешима}. 
Наличие разрешимости задачи~$Z(Pr,M)$ можно определить при заданных~Pr 
и~$M$. Ниже будем считать, что имеется непротиворечивое множество 
прецедентов~\textbf{Pr},  и рассмотрим возможности варьирования множества 
масок~$M$. 

    Варьирование~$M$ заключается в добавлении или удалении отдельных 
масок. В~общем случае условие разрешимости~(\ref{e6tor}) \textit{немонотонно} 
по~$M$, т.\,е.\ из существования разрешимости при~$M$ не 
следует разрешимость при произвольном~$M^\prime$ таком, что $M\subseteq 
M^\prime$. Иначе говоря, не исключена возможность нахождения 
маски~$\hat{m} \notin M$ такой, что при включении ее в~$M$ изменятся 
значения~$l(M)$ и~$r(M)$, так что условие разрешимости~(\ref{e6tor}) также 
нарушится вследствие невыполнения требования 
($l(M),r(M)$)-кор\-рект\-ности. Поэтому целесообразно исследовать 
монотонность условия раз\-ре\-ши\-мости~(\ref{e6tor}) при условии неизменности 
значений~$l(M)$ и~$r(M)$: если для $Z(Pr, M)$ есть разрешимость, $M 
\subseteq M^\prime$, $l(M)= l(M^\prime)$ и $r(M)= r(M^\prime)$, то 
разрешимость есть и для~$Z(Pr, M^\prime)$. Условие 
раз\-ре\-ши\-мости~(\ref{e6tor}) также в общем случае не монотонно по~$M$ и при 
$M^\prime \subseteq M$:  из существования разрешимости задачи~$Z(Pr, M)$ не 
следует разрешимость~$Z(Pr, M^\prime)$, где $M^\prime$ получено при 
удалении масок из~$M$. Рассмотрение монотонности свойства разрешимости 
задачи при $M^\prime\subseteq M$ имеет особое значение для нахождения 
безызбыточных и тупиковых систем масок.
{ %\looseness=1

}

\section{0-тупиковость и~тупиковость систем масок}

    В общем случае система масок~$M$, при которой задача разрешима, 
может быть избыточной в том смысле, что разрешимость сохранится при 
удалении некоторых масок из~$M$. Вообще говоря, определение 
безызбыточной системы масок~$M$~--- задача, разрешимая полным перебором 
подмножеств множества~$M$. Однако в практически интересных случаях 
полный перебор неосуществим. Полный  перебор может быть сокращен по 
аналогии с поиском минимальных д.н.ф.\ (дизъюнктивных нормальных 
форм)~\cite{15tor}. 

    Пусть имеется система масок~$M$, удовлетворяющая 
условиям~(\ref{e6tor}) и~(4$''$). Условия~(\ref{e6tor}) и~(\ref{e7tor})~--- необходимые 
условия, которые ограничивают рамки рассмотрения при анализе систем масок: 
поиск безызбыточных систем масок должен проводиться среди множества 
подмножеств объединенной маски $\hat{m} =M_\Sigma(M)$ при $L, R = const$. 

    Условие~(\ref{e6tor}) монотонно по~$M_\Sigma(M)$ в следующем 
смысле: если~(\ref{e6tor}) выполнено для $M^\prime \subset M$ такой, что 
$M_\Sigma (M^\prime )\subset M_\Sigma(M)$, то (\ref{e6tor}) выполнено и 
для~$M$. Если же условие~(\ref{e6tor}) выполнено для~$M$, но не выполнено 
для любой $M^\prime \subset M$ такой, что $M_\Sigma(M^\prime)\subset 
M_\Sigma(M)$, то систему масок~$M$ назовем \textit{0-ту\-пи\-ко\-вой}. Далее 
рассматриваются только 0-ту\-пи\-ко\-вые~$M$. \textit{Тупиковой} назовем 
такую систему масок, в которой условие~(\ref{e6tor}) нарушается для любой 
$M^\prime\subset M$. Отметим, что не все 0-ту\-пи\-ко\-вые~$M$ являются 
тупиковыми. 
{\looseness=1

}

    Пусть дана 0-тупиковая система масок $M =$\linebreak $= \{\hat{m}_1, \hat{m}_2, \ldots 
, \hat{m}_N\}$. По аналогии с задачей упрощения д.н.ф.~\cite{15tor} 
маску~$\hat{m}_{i_0}$, $i_0\in \{1,N\}$ будем называть \textit{ядерной}, если 
$\hat{m}_{i_0} \notin \bigcup\limits_{j=1,N}^{j\not= i_0} \hat{m}_j$. Ядерными 
системами или подсистемами масок будем называть~$M$, обладающие 
\textit{свойством ядерности}:
\begin{equation}
\forallb\limits_{i=1}^N i \existsb\limits_{\hat{m}_i} \mu:\ j \left(\mu\notin 
\hat{m}_j\right)\,.
\label{e8tor}
\end{equation}

\noindent
\textbf{Теорема 5.} $M$~--- \textit{тупиковая система масок тогда и только 
тогда, когда $M$ обладает свойством ядер\-ности.} 

%\medskip
\pagebreak

\noindent
Д\,о\,к\,а\,з\,а\,т\,е\,л\,ь\,с\,т\,в\,о\,.\  Необходимость доказывается от 
противного. Условие~(\ref{e8tor}) соответствует тому, что для каждой маски 
из~$M$ выполнено условие
\setcounter{equation}{5}\renewcommand{\theequation}{\arabic{equation}$^\prime$}
\begin{equation}
\forall i\in  \{ 1\ldots N\} \hat{m}_i\notin \bigcup\limits_{j=1,N}^{j\not= i} \hat{m}_j\,.
\end{equation}
\renewcommand{\theequation}{\arabic{equation}}


\noindent
Допустим, что~(6$^\prime$) нарушено и $\hat{m}_i\subseteq 
\bigcup\limits_{i=1,N}^{j\not= i} \hat{m}_j$. Тогда $\bigcup\limits_{i=1,N} \hat{m}_i 
=\bigcup\limits_{j=1,N}^{j\not= i} \hat{m}_j$, т.\,е.\ $M_\Sigma (M^\prime) 
=M_\Sigma(M)$. Поэтому условие~(\ref{e6tor}) будет выполняться и для 
$M^\prime = \{ \hat{m}_1, \hat{m}_2, \ldots , \hat{m}_{i-1}, \hat{m}_{i+1}, \ldots 
, \hat{m}_N\}$, так что~$M$ не является тупиковой. Последнее противоречит\linebreak 
исходной предпосылке, а следовательно, тупиковым~$M$ присуще свойство 
ядер\-ности. Достаточность следует из определения свойства 
ядер\-ности~(\ref{e8tor}): так как любая~$\hat{m}_i$ ядерной системы\linebreak 
масок~$M$ содержит определенную уникальную позицию~$\mu$ 
объединенной маски~$M_\Sigma(M)$, удаление любой~$\hat{m}_i$ из~$M$ 
неизбежно приведет к образованию такой~$M^\prime$, что $M_\Sigma 
(M^\prime )\subset M_\Sigma(M)$, так что условие разрешимости~(\ref{e6tor}) 
будет нарушено для~$M^\prime$. Последнее соответствует нарушению 
определения тупиковости, так что любая  ядерная~$M$ является и тупиковой. 
Теорема доказана. 

\medskip

\noindent
\textbf{Следствие 1.} \textit{Из тупиковости следует 0-ту\-пи\-ковость.} 
В~самом деле, тупиковая~$M$ обладает свойством ядерности, т.\,е.\ 
каждой~$\hat{m}_i$ соответствует некая уникальная позиция~$\mu$ в 
объединенной маске~$M_\Sigma(M)$. Поэтому при удалении 
любой~$\hat{m}_i$ удалится соответствующая позиция~$\mu$ с образованием 
измененной объединенной маски~$M_\Sigma(M^\prime)$ такой, что 
$M_\Sigma(M^\prime)\in M_\Sigma(M)$, причем $M^\prime$~--- не 0-тупиковая.

\medskip

\noindent
\textbf{Следствие 2.} \textit{Если в некоторой 0-тупиковой сис\-те\-ме масок $M$ 
имеется ядерная маска~$\hat{m}_{i_0}$, то $\hat{m}_{i_0}$ входит во все 
тупиковые подсистемы}~$M$. Если ядерная~$\hat{m}_{i_o}$ представлена в 
0-тупиковой~$M$, то при удалении~$\hat{m}_{i_0}$ из $M$ произойдет потеря 
0-ту\-пи\-ко\-вости~$M$ (так как $M_\Sigma(M^\prime)\in M_\Sigma(M)$). 
Поскольку\linebreak 0-тупиковость~--- необходимый признак тупиковости (см.\ следствие~1), 
то о тупиковости всех подсистем не может быть и речи. 

\medskip

\noindent
\textbf{Следствие 3.} \textit{Пусть в 0-тупиковой $M$ есть несколько ядерных 
масок $\hat{m}_{i_1}, \hat{m}_{i_1}, \ldots , \hat{m}_{i_L}$ (ядерная 
подсистема). Если некоторая $\hat{m}\subseteq \bigcup\limits_{j=1.L} 
\hat{m}_{i_j}$, то $\hat{m}$ не входит ни в одну тупиковую~$M$.} По 
теореме~5 любая маска в тупиковой~$M$ является ядерной и не может быть 
удалена без потери разрешимости в соответствии с условием~(\ref{e6tor}). 
Маска $\hat{m}\subseteq \bigcup\limits_{j=1,L} \hat{m}_{i_j}$ может быть удалена 
без потери разрешимости, так как ее удаление не приводит к изменению 
объединенной маски~$M_\Sigma (M)$.

\medskip

    Теорема~5 и ее следствия полезны для разработки алгоритма построения 
безызбыточных~$M$. На первом этапе находится 0-тупиковая система масок. 
Затем производится поиск еще менее избыточных систем масок на основе 
свойства ядерности: ядерная подсистема входит во все тупиковые системы 
масок, а маски, покрытые ядерной подсистемой, исключаются. Поиск 
ограничен снизу тупиковыми системами. 

\section{Заключение}

Существующие методы распознавания вторичной структуры белка имеют в 
своей основе ряд предположений. В~настоящей работе предложен 
математический формализм, основанный на предварительно проведенном 
анализе экспериментальных данных. Этот формализм в дальнейшем будет 
использован для анализа задачи о распознавании вторичной структуры белка с 
точки зрения ал\-геб\-раи\-че\-ско\-го подхода к проблемам распознавания. 

{\small\frenchspacing
{\baselineskip=11.8pt
\addcontentsline{toc}{section}{Литература}
\begin{thebibliography}{99}

\bibitem{1tor}
\Au{Torshin I.\,Y.}
Bioinformatics in the post-genomic era: The role of biophysics.~--- N.Y.: Nova 
Biomedical Books, 2006. %ISBN: 1-60021-048.

\bibitem{2tor}
\Au{Berman H.\,M., Henrick~K., Nakamura~H.}
Announcing the worldwide Protein Data Bank~// Nature Structural Biology, 2003. 
Vol.~10. No.\,12. P.~980--982.

\bibitem{3tor}
\Au{Simossis V.\,F., Herringa~J.}
Integrating protein secondary structure prediction and multiple sequence alignment~// 
Curr. Protein Pept. Sci., 2004. Vol.~5. No.\,2. P.~249--266.

\bibitem{4tor}
\Au{Журавлев Ю.\,И.}
Корректные алгебры над множествами некорректных (эвристических) 
алгоритмов. I~// Кибернетика, 1977. №\,4. С.~5--17.

\bibitem{5tor}
\Au{Журавлев Ю.\,И.}
Корректные алгебры над множествами некорректных (эвристических) 
алгоритмов. II~// Кибернетика, 1977. №\,6. С.~21--27.

\bibitem{6tor}
\Au{Журавлев Ю.\,И.}
Корректные алгебры над множествами некорректных (эвристических) 
алгоритмов. III~// Кибернетика, 1978. №\,2. С.~35--43.

\bibitem{7tor}
\Au{Журавлев Ю.\,И.}
Об алгебраическом подходе к решению задач распознавания или 
классификации~// Проблемы кибернетики, 1978. Вып.~33. 
С.~5--68.

\bibitem{8tor}
\Au{Журавлев Ю.\,И., Рудаков К.\,В.}
Об алгебраической коррекции процедур обработки (преобразования) 
информации~// Проблемы прикладной математики и информатики.~--- М.: Наука, 1987. С.~187--198.

\bibitem{9tor}
\Au{Рудаков К.\,В.}
Универсальные и локальные ограничения в проблеме коррекции эвристических 
алгоритмов~// Кибернетика, 1987. №\,2. С.~30--35.

\bibitem{10tor}
\Au{Рудаков К.\,В.}
Полнота и универсальные ограничения в проблеме коррекции эвристических 
алгоритмов классификации~// Кибернетика, 1987. №\,3. С.~106--109.

\bibitem{11tor}
\Au{Рудаков К.\,В.}
Симметрические и функциональные ограничения в проблеме коррекции 
эвристических алгоритмов классификации~// Кибернетика, 1987. №\,4. 
С.~73--77.

\bibitem{12tor}
\Au{Рудаков К.\,В.}
О применении универсальных ограничений при исследовании алгоритмов 
классификации~// Кибернетика, 1988. №\,1. С.~1--5.

\bibitem{13tor}
\Au{Frishman D., Argos~P.}
Knowledge-based protein secondary structure assignment~// Proteins, 1995. Vol.~23. 
No.\,4. P.~566--579.

\bibitem{14tor}
\Au{Torshin I.\,Yu.}
Bioinformatics in the post-genomic era: Sensing the change from molecular genetics 
to personalized medicine.~--- N.Y.: Nova Biomedical Books, 2009. 
%ISBN: 978-1-60692-217.

  \label{end\stat}

\bibitem{15tor}
\Au{Журавлев Ю.\,И.}
Теоретико-множественные методы в алгебре логики~//  Проблемы 
кибернетики, 1962. Т.~8. №\,1. С.~25--45.

 \end{thebibliography}
}
}

\end{multicols}