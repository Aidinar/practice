\def\stat{konovalov}

\def\tit{О ПЛАНИРОВАНИИ ПОТОКОВ В~СИСТЕМАХ ВЫЧИСЛИТЕЛЬНЫХ 
РЕСУРСОВ$^*$}

\def\titkol{О планировании потоков в~системах вычислительных 
ресурсов}

\def\autkol{М.\,Г.~Коновалов}
\def\aut{М.\,Г.~Коновалов$^1$}

\titel{\tit}{\aut}{\autkol}{\titkol}

{\renewcommand{\thefootnote}{\fnsymbol{footnote}}\footnotetext[1]
{Работа выполнена при поддержке РФФИ, гранты 09-07-12032-офи\_м, 08-07-00152-a.}}

\renewcommand{\thefootnote}{\arabic{footnote}}
\footnotetext[1]{Институт проблем информатики Российской академии наук, mkonovalov@ipiran.ru}


\Abst{Рассмотрена проблема анализа и оптимизации распределения потоков 
заданий и ценообразования в системах коллективного использования распределенных 
вычислительных ресурсов. Проведен обзор литературных источников. Предложен 
подход к построению математических моделей систем вычислительных ресурсов, 
основанный на укрупненном описании потоков заданий в виде балансовых 
соотношений и использовании функций качества. Участники системы представляются 
как субъекты, обладающие собственными стратегиями поведения и преследующие 
индивидуальные цели, сформулированные в терминах качества обслуживания и 
стоимости. В качестве варианта стратегии участников рассматривается распределенный 
децентрализованный алгоритм градиентного типа. Приведен численный пример и 
обсуждены перспективы развития и использования модели.}
      
      \KW{системы вычислительных ресурсов; распределение потоков; качество 
обслуживания; коллективное поведение}

     \vskip 18pt plus 9pt minus 6pt

      \thispagestyle{headings}

      \begin{multicols}{2}

      \label{st\stat}

\section{Введение}
  В современном мире к ряду наиболее существенных для человечества ресурсов, таких как 
вода, нефть и~т.\,п., добавился еще один, хотя и искусственный, но жизненно важный 
ресурс, который можно назвать вычислительным (или, более широко, ин\-фор\-ма\-ци\-он\-но-вы\-чис\-ли\-тель\-ным). 
Его запасы, в отличие от естественных ресурсов, увеличиваются. 
<<Можно констатировать, что экспоненциальный характер роста вычислительных 
мощностей\ldots и систем хранения информации сохранится еще на многие годы\ldots>>~[1].
  
Несмотря на то, что количество компьютеров увеличивается, а их мощность продолжает рас\-ти, 
эффективность использования вычислительной техники, по общераспространенному 
мнению, отстает от этого процесса, оставаясь невысокой. Имеет место ситуация, при которой 
сопряженное с огромными материальными и интеллектуальными затратами наращивание 
вычислительных ресурсов (ВР) не дает должной отдачи, что, учитывая 
подверженность данного специфического ресурса моральному старению, делает проблему 
особенно острой.
  
  Вполне естественная идея увеличить действенность имеющегося и обновляющегося парка 
компьютеров за счет обеспечения более широкого, стандартизованного и облегченного 
доступа нашла свою реализацию в виде систем коллективного использования 
распределенных ВР, в первую очередь гридов~[2]. Данная статья касается одного из общих 
аспектов, связанных с разработкой и эксплуатацией систем коллективного использования 
ВР, и посвящена проблематике выбора принципов и алгоритмов распределения заданий 
между имеющимися ресурсами.
  
  Всякая система коллективного использования ВР предполагает наличие двух типов 
со\-став\-ля\-ющих, участвующих в ее работе: потребителей ресурсов и собственно ресурсов, т.\,е.\ 
вычислительной техники. С узкотехнической точки зрения потребители ассоциируются 
с источниками заданий. Последние могут и должны быть выполнены на том или ином ресурсе, 
они имеют определенную трудоемкость, сопряжены с передачей определенного объема 
информации, а также обладают рядом других показателей, включая требования к качеству 
обслуживания. Точно так же ресурс в узком смысле является той или иной разновидностью 
компьютера (персональный компьютер, кластер, суперкомпьютер и~т.\,д.) или хранилища 
данных и обладает определенными техническими па\-ра\-мет\-ра\-ми: производительностью, 
емкостью памяти, программным обеспечением и~пр. Таким образом, проблема 
планирования потоков заданий в узком смысле может пониматься как составление 
расписания, предписывающего место и очередность выполнения того или иного задания и 
увязывающего рабочие характеристики исполняемых заданий и используемых ресурсов.
  
  В то же время в современных больших вычислительных системах, за счет их масштаба, 
разнородности оборудования, различий в административном подчинении и других 
особенностей, при планировании появляются качественно новые моменты, не учитываемые 
в классической теории расписаний и связанные с взаимозависимостью отдельных 
участников системы. 

К~примеру, приходится принимать во внимание, что потребители не 
только порождают потоки заданий, но и преследуют при их выполнении довольно сложные 
цели, порожденные соображениями финансового, или приоритетного, или секретного и~т.\,д. 
характера. К~тому же потребители часто объединяются в группы или сообщества, оставаясь 
при этом в определенной мере самостоятельными, и к тому же разделенными географически. 

Аналогично за ресурсом как техническим устройством обычно стоит еще и владелец 
ресурса со своими собственными интересами. Приходится констатировать, что отдельные 
составные части распределенной системы ВР обладают собственным поведением, 
целенаправленность которого, вообще говоря, не совпадает с приоритетами сис\-те\-мы в 
целом. Проблема планирования в такой системе в широком смысле не может быть сведена к 
составлению расписания и созданию со\-от\-вет\-ст\-ву\-юще\-го связующего программного 
обеспечения. Необходимо концептуальное понимание принципов и наличие методов, 
позволяющих организовать поведение участников системы и учитывающих как собственные 
интересы участников, так и цели, стоящие перед всей системой.
  
  Термин \textit{система вычислительных ресурсов} не относится к числу точных или даже 
вызывающих однозначные ассоциации. По-видимому, безусловно общими для всех систем 
ВР являются следующие признаки: 
  \begin{enumerate}[(1)]
  \item наличие собственно ВР как технических устройств (процессоров и структур, 
составленных из процессоров, носителей информации);
  \item наличие источников заданий, выполняемых с помощью этих устройств;
  \item объединение поименованных в первых двух пунктах элементов в систему, 
позволяющее различным источникам заданий обращаться к разным ресурсам, 
обмениваться информацией между элементами, осуществлять более или менее 
согласованную политику функционирования составных частей.
  \end{enumerate}
  
  Перечисленным признакам удовлетворяют самые различные и по масштабу, и по целям 
системы ВР. Основным <<примером>> для данной работы можно считать системы грид с их, 
как правило, глобальным масштабом, гетерогенным составом и автономностью поведения 
подсистем. Однако было бы неверно ограничивать статью только об\-ластью проблематики 
грида. Например, в крупных и не очень крупных компаниях повсеместно приходят к мысли о 
необходимости организации более эффективного менеджмента в области использования ВР. 
<<Территориально-распределенное предприятие~--- более 1700~км магистральных 
трубопроводов, которые протянулись от Полярного круга до юга Тюменской области, 
17~компрессорных станций, один из крупнейших в России завод стабилизации газового 
конденсата\ldots Сбой в любом звене может привести к достаточно серьезным последствиям. 
Поэтому года три назад\ldots была осознана необходимость создания системы управления 
ин\-фор\-ма\-ци\-он\-но-вы\-чис\-ли\-тель\-ны\-ми ресурсами>>. (Это цитата 2010~г.\ из 
публикации о компании ООО <<Сургутгазпром>>~[3].)
  
  Данная работа направлена на изучение таких систем ВР, в которых составные части 
обладают определенной самостоятельностью в выборе критериев деятельности и стратегий 
управления потоками заданий и распределения ресурсов. Другая особенность постановки 
задачи заключается в том, что отдельные субъекты, составляющие систему, 
многофункциональны. Они могут одновременно являться как источниками заданий, так и 
пред\-став\-лять запас вычислительных ресурсов или выполнять посреднические функции.

\section{Краткий обзор литературы}
  
  За последние годы опубликовано много работ по планированию в системах ВР, прежде 
всего, в сис\-те\-мах грид. Это отражено в обзорах, посвященных данной тематике~[4, 5]. 
Публикации самого последнего времени, которые упоминаются ниже, представляют 
большое разнообразие постановок задач, методов и приложений.
  
  Проблема управления ресурсами в системах ВР может трактоваться как имеющая две 
основные со\-став\-ляющие. Первая~--- это нахождение алгоритмов, которые бы эффективно 
обслуживали конкретные задания, направляя их на конкретные\linebreak ресурсы. Эта задача в целом 
находится в об\-ласти классических оптимизационных задач (за рубежом для ее обозначения 
повсеместно используется термин scheduling), хотя и осложнена большой размерностью и 
обилием специфических особен\-ностей. Вторая~--- связана с уже упоминавшейся 
авто\-номностью субъектов, обладающих собственными  %\linebreak 
целями и стратегиями поведения. 
Реализация стратегии, оптимизирующей глобальную для всей системы целевую функцию 
(даже если такая стратегия будет найдена), натолкнется на со\-про\-тив\-ле\-ние отдельных 
элементов системы, если не будут учтены их локальные интересы.
  
  Работы первого направления продолжают широко использовать классические статические 
оптимизационные постановки задач с <<интегральным>> описанием потоков заданий~[6] и 
применением традиционных потоковых алгоритмов~[7]. По-преж\-нему популярна 
\textit{задача о рюкзаке}, которая применительно к проблеме планирования ресурсов 
дополня\-ется использованием функций полезности и метрик качества обслуживания~[8], 
стоимостными соображениями~[9], а также эвристическими методами~[10]. 

Во многих работах, 
однако, рассматривается <<штучная>> обработка заданий. В~этих случаях описание 
моделей осуществляется в терминах случайных процессов, а для оптимизации часто 
употребляется марковский процесс принятия решений~[11, 12]. В~[5] предложена модель, в 
которой для оперативного управления коллективным доступом к распределенным 
вычислительным ресурсам используются алгоритмы, основанные на адаптивном варианте 
теории управления марковскими цепями.
  
  Поскольку точные методы решения классических оптимизационных задач часто 
неэффективны, то широкую популярность приобрели эвристические алгоритмы~[13--15]. Не 
являясь сугубой принадлежностью планирования потоков заданий, они фигурируют за 
рубежом под экзотическими названиями (artificial fish swarm algorithm, genetic algorithm, 
simulated annealing и~пр.)
  
  Некоторые работы используют менее распространенные подходы. 
  
  Так, в~[16] 
предлагается механизм планирования заданий в гриде на основе предварительного 
резервирования в режиме онлайн. В~[17] описан алгоритм диспетчеризации заданий также в 
режиме реального времени, но на основе балансировки нагрузки путем прогнозирования 
производительности серверов. Алгоритмы балансировки нагрузки разработаны~[18, 19], а 
статистическое прогнозирование явилось исходной посылкой для создания стратегии 
планирования в~[20]. В~[21] для распределения заданий в вычислительном гриде 
используются нечеткие множества и нейронные сети. В~[22] та же задача решается с 
применением техники так называемых \textit{сложных сетей} (complex network). В~[23] 
обсуждаются специфические проблемы планирования, возникающие при обслуживании 
заданий разного типа: ориентированных на вычисления и связанных преимущественно с 
передачей данных. В~[24] описана необычная постановка задачи распределения мобильных 
ресурсов.
\columnbreak
  
  Отмеченное выше второе направление исследований, связанное с автономностью 
поведения участников системы ВР, представлено значительно беднее. Работы, которые 
можно было бы отнести к этому направлению, часто отражают в большей степени 
гетерогенность составных блоков системы, нежели их самостоятельность.
  
  В~[25] предлагается модель размещения заданий, параллельно выполняемых в 
автономных доменах. Другой подход к организации параллельных вы\-чис\-ле\-ний в 
гетерогенной среде основан на привлечении так называемых систем с агентами~[26].

 В~[27] 
рассмотрена технология организации планиро\-вания в гриде, составные части которого 
основаны на разных стандартах, что затрудняет коллективное использование ресурсов. 
Рассмотренное в~[27] понятие \textit{федерации гридов} (grid-federation) 
широко разрабатывается авторами статьи~[28], в которой дана схема кооперированного 
использования ресурсов в гриде, основанная на концепции соглашений об уровне 
обслуживания (service level agreement). 

Стратегии \textit{согласований} (negotiations) 
рассмотрены в~[29].
  
  Большие надежды в организации менеджмента в системах ВР возлагаются на 
экономический подход, побуждающий, как принято считать, разнородные и конкурирующие 
элементы действовать, соблюдая интересы системы в целом. Эта часть публикаций 
заслуживает отдельного обзора (в некоторой степени он проведен в~[4]). Здесь упомянем в 
качестве примера сравнительный анализ менеджмента в гриде, основанный на различных 
механизмах рыночной экономики~[30], а также две модели планирования ресурсов, 
основанные на оценке их стоимости~[31, 32].
  
  Проведенный беглый обзор показывает, прежде всего, что проблема управления потоками 
заданий в системах вычислительных ресурсов находится в стадии интенсивного изучения. 
В~настоящее время нельзя говорить о том, что для ее решения сформирован в какой-то 
степени окончательный круг теоретических положений и практических методов. 
Существующие разработки отличаются большим многообразием используемых подходов и 
средств.
{\looseness=1

}
  
  В данной работе сделана попытка дать простое математическое описание распределенной 
системы ВР, которое, не отражая частных деталей, годилось бы для изучения общих 
вопросов управления потоками заданий в широком классе таких систем. Избранный способ 
моделирования продолжает подход, начатый в~[33].

\section{Моделирование процесса распределения потоков в~системе 
вычислительных ресурсов}

\subsection{Балансовое соотношение для потоков заданий}

  Рассмотрим модель распределенной системы ВР, составные элементы которой будем 
называть субъектами (или иногда, для разнообразия, участниками). Всего система содержит 
$N$~субъектов, пронумерованных от~1 до~$N$. Каждый субъект способен 
осуществлять, вообще говоря, троякую функцию:
  \begin{itemize}
  \item являться источником потоков заданий;
  \item выполнять задания на имеющихся у него ресурсах;
\item являться транзитным и коммутационным пунктом для перемещения заданий между 
субъек\-тами. 
\end{itemize}
Таким образом, в разрабатываемой модели каждый субъект может одновременно выступать 
в роли потребителя ресурсов, владельца ресурса и посредника.
  
  Субъекты взаимодействуют через коммуникационную сеть, которая не выделяется как 
самостоятельный элемент системы, но ее наличие будет подтверждено косвенно при 
дальнейшем описании.
  
  Условимся измерять объемы заданий (нагрузку) в некоторых условных единицах, 
физический смысл которых в данном случае не играет особой роли. (Можно, например, 
представлять себе, что единица объема задания измеряется временем его выполнения на 
стандартном процессоре.) Поток заданий, порождаемый субъектом~$i$, представляет собой 
случайный процесс~$Z_i(t)$, заданный, как и все функционирование системы, в дискретном 
времени, $t=0,1,\ldots$ Этот процесс определяет объем нагрузки, поступающей в систему 
извне на каждом такте времени через посредство субъекта~$i$.
  
  Помимо экзогенного потока заданий~$Z_i(t)$, субъект получает входные потоки от 
других участников. Объем заданий, который накапливается у субъекта~$i$ в момент~$t$, 
обозначается через~$X_i(t)$. Этим объемом заданий (будем называть его для краткости 
очередью в момент~$t$) субъект обязан распорядиться в момент~$t$, имея 
следующие возможности:
  \begin{itemize}
  \item выполнение заданий на собственном ресурсе;
  \item отправка заданий другим субъектам;
  \columnbreak
  \item оставление заданий в собственной очереди;
  \item уничтожение заданий (потери).
  \end{itemize}
  
  Чтобы описать реализацию указанных возможностей, определим вектор
  $$
  \alpha_i = \left( \alpha_{i0},\alpha_{i1},\ldots ,\alpha_{iN}\right)\,,
  $$
компоненты которого удовлетворяют условиям
$$
\sum\limits_{i=0}^N \alpha_{ij} =1\,,\quad 0\leq \alpha_{ij}\leq1\,,\enskip 0\leq i,j\leq N\,,
$$
и означают следующее:
  $\alpha_{i0}$~--- доля объема заданий, направляемая для выполнения на собственный 
ресурс; $\alpha_{ij}$, $i\not=j$~--- доля объема заданий, направляемая субъекту~$j$; 
$\alpha_{ii}$~--- доля объема заданий, остав\-ля\-емая в собственной очереди (часть этих 
заданий может быть уничтожена).
  
  Вектор $\alpha_i$ определяет способ, которым субъект~$i$ распоряжается тем объемом 
заданий, который у него уже имеется. В~частности, параметры $\alpha_{ij}$, $i\not= j$,
определяют объем \textit{заявки} на передачу части заданий на адрес субъекта~$j$. Однако 
субъект~$j$, в соответствии с обсуждавшимся выше предположением о самостоятельности 
поведения участников системы, имеет возможность отказаться от предложения. Для 
регулирования объема вновь поступающих заданий необходим дополнительный механизм. 
Он задается вектором
  $$
  \beta_i =\left ( \beta_{i1},\ldots , \beta_{iN}\right)\,,
  $$
компоненты которого подчиняются условиям $0\leq$\linebreak $\leq \beta_{ij}\leq 1$ и при 
$i\not=  j$ означают долю от пред\-ла\-га\-емо\-го субъектом~$j$ объема заданий, которую субъект~$i$ согласен 
принять. Параметру~$\beta_{ij}$ придадим смысл регулирования объема заданий, 
сознательно удаляемых из системы субъектом~$i$.
  
  Согласно сказанному, в каждый момент~$t$ при передаче заданий от субъекта~$i$ к 
субъекту~$j$ происходит следующего рода согласование. Субъект~$i$ делает заявку на 
передачу заданий в объеме $\alpha_{ij}X_i(t)$. Субъект~$j$ подтверждает согласие на 
передачу части этого объема, и фактически передаваемый объем заданий составляет 
$\alpha_{ij}\beta_{ji}X_i(t)$.
  
  Таким образом, стратегии участников системы описываются с помощью матриц~$\alpha$ 
и~$\beta$ с компонентами~$\alpha_{ij}$ и~$\beta_{ij}$, имеющих размерности 
соответственно $N\times (N+1)$ и $N\times N$. О~матрице~$\alpha$ будем говорить как о 
стратегии \textit{маршрутизации}, а матрицу~$\beta$ будем называть стратегией 
\textit{согласования}.
  
  Заметим, что стратегии поведения субъекта, описываемые с помощью матриц~$\alpha$ 
и~$\beta$, могут быть сколь угодно сложными, поскольку компоненты этих матриц могут 
зависеть в общем случае от момента времени~$t$ и даже от всей предыстории системы до 
этого момента.
  
  Прежде чем описать динамику потоков в системе, необходимо еще оговорить, что 
происходит с теми заданиями, которые были заявлены на передачу другим субъектам, но не 
были согласованы. В принципе, это вопрос описания, а не существа дела, и он может быть 
разрешен по-разному. Договоримся в данном случае, что все несогласованные задания 
автоматически отправляются для выполнения на собственный ресурс.
  
  Введем в рассмотрение матрицу~$\lambda$ размерностью $N\times N$ с компонентами
  $$
  \lambda_{ij} =\alpha_{ij}\beta_{ji}\,,\quad 1\leq i,j\leq N\,.
  $$
Объем заданий у субъекта~$i$ в момент $t+1$ складывается из экзогенного потока, а также из 
потоков, поступающих от всех участников системы, поэтому
  $$
  X_i(t+1)=Z_i(t+1)+\sum\limits_{j=1}^N \lambda_{ij}X_j(t)
  $$
или в матричной форме
\begin{equation}
X(t+1)=Z(t)+\lambda X(t)\,,
\label{e1konov}
\end{equation}
где $X(t)$ и $Z(t)$~--- векторы-строки с компонентами~$X_i(t)$ и~$Z_i(t)$.
  
  Объем заданий, направленный для выполнения на собственный ресурс, с учетом 
сделанной договоренности составляет для субъекта~$i$ величину
  \begin{equation*}
  Y_i(t) =\lambda_{i0}X_i(t)\,,
%  \label{e2konov}
  \end{equation*}
где 
$$
\lambda_{i0} =\sum\limits_{j=0}^N\alpha_{ij}-\sum\limits_{j\not= i}\lambda_{ij}=1-\alpha_{ii}-
\sum\limits_{j=1}^N \lambda_{ij}\,.
$$
  
  Потери заданий выражаются как
  \begin{equation}
  \alpha_{ii}(1-\beta_{ii})X_i(t)=\gamma_iX_i(t)\,.
  \label{e3konov}
  \end{equation}

  Величину $\gamma_i =\alpha_{ii}(1-\beta_{ii})$~--- долю удаляемых заданий~--- назовем 
\textit{коэффициентом потерь}.
  
  В последующих рассуждениях будем предполагать, что экзогенные потоки имеют не 
зависящие от времени средние $z_i = \mathrm{M}Z_i(t)$.
  
  Пусть элементы матрицы~$\lambda$ являются постоянными величинами и при этом для 
всех $i$ выполняются соотношения $\sum\limits_{j=1}^N \lambda_{ij}<1$. Тогда 
из~(\ref{e1konov}) следует, что существуют пределы 
$x_i=\lim\limits_{t\rightarrow\infty} \mathrm{M} X(t)$, которые определяются соотношением
  \begin{equation}
  x=z+x\lambda
  \label{e4konov}
  \end{equation}
или
$$
x=z(E-\lambda)^{-1}\,,
$$
где $x=(x_1,x_2,\ldots ,x_N)$ и $z=(z_1,z_2,\ldots ,z_N)$, а $E$~--- единичная матрица. 
Дальнейшие соотношения также будут относиться к предельным средним значениям 
переменных, участвующих в модели.

\subsection{Качество обслуживания}

  Качество обслуживания в системе вычислительных ресурсов определяется целым рядом 
факторов, которые зачастую плохо поддаются количественному описанию. Тем не менее 
одно из самых существенных предположений, которое делается в данной статье, 
заключается именно в том, что количественное описание качества обслуживания имеется.
  
  Говоря неформально о стремлении к качественному обслуживанию, можно было бы 
выразиться словами <<быстрее, дешевле, лучше>>. Что такое <<быстрее>> и 
<<дешевле>>~--- вполне понятно. Например, временной фактор связан со скоростью 
процессоров, наличием или отсутствием очередей невыполненных заданий в буферах, 
задержкой при транспортировке заданий по сети и~т.\,д. Главное в том, что временной 
фактор естественным образом выражается скалярно. Так же, как и стоимостный фактор. 
Конечно, подсчет и времени, и стоимости выполнения заданий может оказаться непростой 
задачей, но результатом ее решения будет количественное выражение. Сложнее может 
оказаться выразить числом, что значит <<лучше>>, поскольку речь может идти о плохо 
формализуемых критериях (например, об использовании более или менее подходящего 
программного обеспечения и~т.\,п.).
   
   Не расшифровывая, что понимается конкретно под словом <<лучше>>, предположим, 
что существует неотрицательный показатель качества субъекта в момент времени~$t$, 
который будем обозначать через~$q_i(t)$. При этом условимся, что значение показателя 
качества, равное нулю, характеризует некоторый идеальный уровень качества обслуживания, а 
увеличение числовой оценки качества обслуживания соответствует потере качества. Таким 
образом, чем больше значение показателя качества, тем хуже реальное качество 
обслуживания.
   
    
   Следующее предположение заключается в существовании неотрицательных 
\textit{функций качества}~$Q_i$,\linebreak которые будут использоваться при вычислении\linebreak 
показателей качества субъектов~$q_i(t)$ и которые характеризуют качество обслуживания 
на ресурсе субъекта~$i$. Функция~$Q_i$ имеет аргументом объем заданий, поступающий на 
ресурс~$i$. Характер зависимости от аргумента для каждой из функций каче-\linebreak\vspace*{-12pt}
\pagebreak

%\medskip

\begin{center} %fig1
\vspace*{-3pt}
\mbox{%
\epsfxsize=65.328mm
\epsfbox{kon-1.eps}
}
\vspace*{4pt}
\end{center}
\begin{center}
{{\figurename~1}\ \ \small{Функция оценки качества~$Q$ от объема заданий~$x$}}
\end{center}
\vspace*{12pt}


%\bigskip
\addtocounter{figure}{1}

   
\noindent
ства 
предполагается примерно таким, как у функции~$Q(x)$ на рис.~1. Эта функция 
зависит от трех параметров: $p$, $r$ и~$s$. Параметр $p>0$ характеризует тот уровень 
нагрузки, достижение и превышение которого влечет существенное ухудшение качества 
обслуживания. Параметр~$p$ будем условно называть \textit{емкостью}. Параметр $r\geq 0$ 
означает минимальную потерю качества, которая возможна при обслуживании и которая 
достигается при минимальной загрузке данного участника. Этот параметр будем условно 
называть \textit{интенсивностью (обслуживания)}. Параметр $s>0$ означает числовую 
оценку ситуации, в которой качественное обслуживание заданий практически отсутствует. 
Можно условно употребить выражение \textit{(максимальный) штраф} за плохое 
обслуживание. Пример аналитического выражения для функции~$Q(x)$:
   \begin{equation}
   Q(x) =\fr{2s \arctg p+\pi r+2(s-r)\arctg(x-p)}{\pi+2\arctg p}\,.
   \label{e5konov}
   \end{equation}
В этом примере $p$~--- точка перегиба функции~$Q(x)$, $Q(0) = r$,  
$\lim\limits_{x\rightarrow\infty} Q(x)=s$.
   
   При составлении соотношений для показателей качества будем исходить из следующих 
наводящих соображений. Располагая определенным объемом заданий, субъект обеспечивает 
их обслуживание, оставляя часть заданий на собственном ресурсе и отправляя остальные 
другим субъектам, либо уничтожая их. Эти действия приводят к разному качеству 
обслуживания для тех долей заданий, которые подвергаются тому или иному управлению. 
Например, передача заданий другому субъекту означает, что тот принимает полную 
ответственность за их дальнейшее обслуживание, которое будет происходить с тем уровнем 
качества, который обеспечивает принимающая сторона. Чтобы не усложнять дальнейшее 
описание, предположим, что в системе нет потерь ($\gamma_i=0$ для всех~$i$). Тогда 
распределение потоков заданий субъекта~$i$ задается набором 
$\lambda_{i0},\lambda_{i1},\ldots ,\lambda_{iN}$, и для показателя качества субъекта~$i$ 
запишем соотношение
   $$
   q_i=\lambda_{i0}Q_i+\sum\limits_{n=1}^N \lambda_{ij} q_j\,,
   $$
где $Q_i=Q_i(y_i)$, $y_i=\lambda_{i0}x_i$~--- функции качества,  имеющие, например, 
вид~(\ref{e5konov}), с индивидуальными для каждого субъекта параметрами $(p_i, r_i, s_i)$. 
В~мат\-рич\-ной форме имеем систему
$$
q=\kappa+q\lambda^{\mathrm{T}}
$$
или
$$
q=\kappa\left(E-\lambda^{\mathrm{T}}\right)^{-1}\,,
$$
где $q=(q_1,\ldots ,q_n)$; $\kappa=(\lambda_{10}Q_1,\ldots ,\lambda_{N0}Q_N)$; T~--- знак 
транспонирования.

  \medskip
  
  \noindent
  \textbf{Замечание.} В начале этого подраздела были выделены три составляющие 
качества обслуживания, которые можно определить как временн$\acute{\mbox{у}}$ю, денежную и третью 
составляющую, учитывающую специфические трудно формализуемые критерии. 
Определенные выше показатели качества, в том числе функции качества, можно трактовать 
как \textit{штраф за потерю качества}. Это предполагает, что упомянутая третья 
составляющая качества имеет не только количественное, но и, фактически, денежное 
выражение. Поскольку и временн$\acute{\mbox{а}}$я компонента, очевидно, легко может 
быть переведена на тот же язык, например с помощью \textit{штрафа за задержку}, то 
напрашивается естественная возможность все обсуждение качества в модели свести к 
стоимостным соотношениям. В~данном случае, однако, такая возможность не используется, 
а рассматривается некий компромиссный вариант. Стоимостная составляющая выступает 
самостоятельно в виде арендной платы (см.\ следующий подраздел). Что касается фактора, 
связанного с задержками, то он, хотя и не входит явным образом в модель, частично 
отражен, поскольку большие значения функции качества при высокой нагрузке могут 
интерпретироваться как \textit{штраф за задержку на ресурсе}.

\subsection{Стоимостные факторы}

  Предполагается, что каждый субъект~$i$ имеет набор ценовых параметров $(a_i,b_{ij})$, 
где $a_i$~--- стоимость обслуживания единицы объема заданий, полученных от других 
субъектов; $b_{ij}$~--- тариф за передачу единицы объема заданий от субъекта~$i$ к 
субъекту~$j$.
  
  Кроме того, предположим, что задана цена~$c$~--- штраф за потерю (уничтожение) 
единицы объема заданий~--- одинаковая для всей системы.
  
  Доходы  $f_i$ субъекта~$i$ представляют собой арендную плату, равную объему 
поставляемых ему заданий, умноженную на коэффициент~$a_i$. Расходы этого же субъекта 
складываются из платы~$g_i$ за дальнейшее обслуживание той части заданий, которая 
пересылается другим субъектам, и штрафа за потери~$h_i$. Первая из указанных величин, в 
свою очередь, составляется из арендной платы другим субъектам и оплаты транспортировки. 
Согласно~(\ref{e3konov}) и~(\ref{e4konov}) перечисленные компоненты равняются:
  \begin{align*}
  f_i & = a_i\sum\limits_{j=1}^N \lambda_{ij}x_j+a_iz_i=a_ix_i\,;\\
  g_i & = \sum\limits_{j=1}^N 
a_j\lambda_{ij}x_i+\sum\limits_{j=1}^Nb_{ij}\lambda_{ij}x_i\,;\\
  h_i& = c\gamma_i x_i\,.
  \end{align*}

  Общий денежный баланс субъекта~$i$ (превышение доходов над расходами) 
определяется как 
  $$
  d_i=f_i-g_i-h_i=\delta_ix_i\,,
  $$
где $\delta_i=a_i-\sum\limits_{j=1}^N (a_j+b_{ij})\lambda_{ij}-c\gamma_i$.

\subsection{Целевые функции и~алгоритм распределения потоков}
  
  Целевые функции участников системы должны учитывать как стоимостные аспекты, так и 
стремление оптимизировать качество выполнения заданий.
  
  Чтобы соединить в одном критерии эти два обычно противоречивых фактора, введем два 
дополнительных пороговых параметра: $\overline{d}_i$ и $\overline{q}_i$~--- бюджет и 
требуемый уровень качества субъекта~$i$. Субъект стремится действовать так, чтобы 
расходы не выходили за рамки бюджета, а качество соответствовало заданному уровню. 
Нарушение ограничений по одному из критериев заставляет субъект заботиться именно об 
этом показателе. Определим целевую функцию в этих случаях как
  $$
  w_i = 
  \begin{cases}
  -d_i\,, & \mbox{если}\ -d_i>\overline{d}_i,\enskip q_i\leq \overline{q}_i\,;\\
  q_i\,, & \mbox{если}\ -d_i\leq \overline{d}_i\,,\enskip q_i>\overline{q}_i\,.
  \end{cases}
  $$
Такое определение соответствует стремлению потребителя: 
\begin{enumerate}[(1)]
\item минимизировать расходы, не 
заботясь о качестве, если он выходит за рамки бюджета, а качество при этом находится в 
допустимых пределах; 
\item оптимизировать только качество обслуживания в ситуации, когда 
он укладывается в бюджет, а требования по качеству нарушены.
\end{enumerate}
  
  В ситуации, когда оба пороговых соотношения нарушены или, наоборот, оба выполнены, 
можно воспользоваться какой-либо сверкой критериев. Например, можно использовать 
механизм штрафов, мотивируя стремление участников улучшать качество обслуживания. 
Определим за ухудшение качества обслуживания штраф, размер которого прогрессивно 
зависит от устанавливаемых тарифов за услуги. Положим
$$
  w_i =-d_i-(a_i)^{1+a}q_i\,,
  $$
  если $-d_i\leq\overline{d}_i$, $q_i\leq\overline{q}_i$ или
$-d_i >\overline{d}_i$, $q_i>\overline{q}_i$, где $a\geq 0$~--- фиксированный параметр.

  Управлениями в системе будем считать определенные в п.~3.1 матрицы 
маршрутизации~$\alpha$ и согласования~$\beta$, которые задают распределение потоков, а 
также определенные в п.~3.3 стоимостные характеристики~$a_i$. (Матрица сетевых 
тарифов~$b_{ij}$ считается фиксированной.) Исходное представление о 
децентрализованном и независимом поведении участников означает, что субъект~$i$ 
выбирает собственный вектор маршрутизации~$\alpha_i$, согласования~$\beta_i$, а также 
тариф~$a_i$.
  
  Рассмотрим такой вариант поведения участников, при котором они синхронно и 
независимо\linebreak изменя\-ют <<свои>> управления согласно алгоритму проекции градиента, 
стремясь минимизировать определенные выше целевые функции~$w_i$. Формальную запись 
алгоритма, однотипного для всех субъектов и для всех управлений, приведем для 
маршрутизации субъекта~$i$. Пусть 
$\alpha_i^{(n)}=$\linebreak $=(\alpha_{i0}^{(n)},\alpha_{i1}^{(n)},\ldots ,\alpha_{iN}^{(n)})$~--- значение 
вектора~$\alpha_i$ для\linebreak
 $n$-й итерации алгоритма, $n=0, 1,\ldots ,$ и пусть $\nabla 
w_i^{(n)}$~--- градиент функции~$w_i$ в точке~$\alpha_i^{(n)}$. Рекуррентное 
соотношение для последовательных значений вектора~$\alpha_i^{(n)}$ имеет следующий 
вид:
  \begin{equation}
  \alpha_i^{(n+1)} =\Pi \left( \alpha_k^{(n)}-a^{(n)}\nabla w_i^{(n)}\right)\,,
  \label{e6konov}
  \end{equation}
где $\Pi$ означает оператор проектирования на единичный симплекс размерности $N+1$, а 
положительная последовательность~$a^{(n)}$ удовлетворяет обычным для такого рода 
алгоритмов условиям $a^{(n)}\rightarrow 0$, $\sum\limits_n a^{(n)}=\infty$.

\subsection{Пример}
  
  Рассмотрим систему, состоящую из трех субъектов, параметры которых указаны в 
табл.~1.
  
\begin{table*}\small
\begin{center}
\Caption{Параметры системы
\label{t1konov}}
\vspace*{2ex}

\begin{tabular}{|c|c|c|c|c|c|}
\hline
\multicolumn{1}{|c|}{\raisebox{-6pt}[0pt][0pt]{Субъект, $i$}} &
\multicolumn{3}{c|}{Параметры функции качества $Q_i$}&\multicolumn{1}{|c|}{\raisebox{-6pt}[0pt][0pt]{Поток, 
$z_i$}}&\multicolumn{1}{|c|}{\raisebox{-6pt}[0pt][0pt]{Сетевой тариф, $b_{ij}$}}\\
\cline{2-4}
&$p_i$&$r_i$&$s_i$&&\\
\hline
1&20&10&100&100&0; 100; 1\\
2&120&1&20&25&100; 0; 1\\
3&0,01&1&1000&0&1; 1; 0\\
\hline
\end{tabular}
\end{center}
\end{table*}

\begin{table*}\small
\begin{center}
\Caption{Показатели системы до и после оптимизации
\label{t2konov}}
\vspace*{2ex}

\begin{tabular}{|c|c|c|c|c|c|c|}
\hline
Субъект, $i$&\multicolumn{2}{c|}{Маршрутизация, $\alpha_i$ }&\multicolumn{2}{c|}{Очередь, $x_i$}  &\multicolumn{2}{c|}{Качество,  $q_i$}\\
\cline{2-7}
&в начале&в конце&в начале&в конце&в начале&в конце\\
\hline
1&0,25&
\tabcolsep=0pt\begin{tabular}{c}0\\ 0\\ 0,8031\\ 0,1969\end{tabular}& 
225&101\hphantom{9}&295&19,69\\
\hline
2&0,25&0, 0, 0, 1&150&97&271&\hphantom{9}1,22\\
\hline
3&0,25&0, 1, 0, 0&125&80&985&13,4\hphantom{9}\\
\hline
\end{tabular}
\end{center}
\end{table*}


  Функции качества для всех субъектов имеют вид~(\ref{e5konov}). Требования по уровню 
качества для участников 1, 2, 3 равняются соответственно 20, 1,2 и~20.
  
  Приведенные значения параметров говорят о следующем. Субъекты~1 и~2 выполняют в 
системе одновременно роль источников потоков заданий и ресурсов. При этом субъект~1 
создает основную нагрузку, но располагает значительно менее емким, производительным и 
качественным ресурсом, чем субъект~2. Субъект~3 не порождает потока заданий, но он 
также и не располагает сколько-нибудь значимым ресурсом. Потенциальная роль этого 
субъекта определяется матрицей сетевых тарифов, из которой следует, что данный участник 
обладает значительно более экономными возможностями общения с субъектами~1 и~2, чем 
те сами между собой.
  
  Начальная маршрутизация всех субъектов\linebreak (векторы~$\alpha_i$) равномерная, а все 
векторы согласования~$\beta_i$ имеют компоненты, равные~1. Ценовые факто\-ры в этом 
примере не рассматриваются. Основные показатели, которые дает это очень неэффективное 
распределение потоков, содержатся в табл.~2 (округленно).
  

  После оптимизации с помощью алгоритма~(\ref{e6konov}) распределение потоков 
изменилось. Субъект~1 стал использовать участника~3 в качестве транзитного пункта для 
передачи большей части заданий на более мощный ресурс~2. На собственный ресурс 
направляется около 20\% потока~--- больший объем нарушил бы требуемый уровень 
качества~20. Субъект~2 перешел полностью на самообслуживание и стал посылать весь 
поток на собственный ресурс. Характерно, что из всех компонент матрицы 
согласования~$\beta$ (в табл.~2 она не отражена) изменился только 
коэффициент~$\beta_{23}$, который стал равным приблизительно~0,9. Это вызвано 
необходимостью для субъекта~2 ограничить поток, получаемый от посредника~3, и тем 
самым обеспечить заданный уровень качества~1,2.

\section{Заключение}
  
  Рассмотрена проблема анализа и оптимизации распределения потоков заданий и 
ценообразования в системах коллективного использования распределенных вычислительных 
ресурсов. Проведенный литературный обзор показал, что, хотя существуют разнообразные 
подходы и методы решения проблемы, она пока далека от окончательного решения.
  
  Предложена математическая модель системы вычислительных ресурсов, которая 
представляет собой интегральное описание потоков заданий в виде динамических 
балансовых соотношений. Необычность модели в том, что участники системы, вообще 
говоря, одновременно играют роль источников заданий, владельцев ресурсов и посредников 
в передаче потоков. Благодаря этому субъекты сис\-те\-мы имеют одинаковое, причем 
математически простое, описание в плане стратегии распределения потоков заданий и 
выделения ресурсов. В~то же время в сравнительно простую модель удалось включить 
целый ряд факторов, имеющих принципиальное значение для любой системы 
вычислительных ресурсов: планирование выбора ресурсов для заданий, степень готовности 
ресурса обслуживать задания, качество обслуживания, учет потерь, затраты и 
ценообразование, потери в системе и~т.\,д. При этом построение модели является, в 
сущ\-ности, многовариантным. Перечисленные факторы в зависимости от потребностей той 
или иной сис\-те\-мы, той или иной задачи могут быть полностью или частично отражены в 
модели или, наоборот, устранены из нее. В~этом смысле можно говорить о том, что 
предложен способ моделирования системы вычислительных ресурсов.
  
  Изложенный подход к моделированию не привязан к конкретной системе 
вычислительных ресурсов, поэтому его использование видится прежде всего в изучении 
вопросов, связанных с по\-стро\-ени\-ем таких систем вообще. При этом надо еще раз 
подчеркнуть, что принципиальным соображением в работе было представление об 
участниках системы как о независимо действующих субъектах, преследующих 
индивидуальные цели. С~учетом последнего замечания к числу важных вопросов, 
разрешение которых можно надеяться получить с помощью предложенной методологии, 
относятся, например, следующие.
  
  Определение показателей, которых может достичь система ресурсов, участники которой 
действуют, исходя из эгоистических интересов. Определение оптимальных стратегий 
поведения\linebreak участников.
  
  Установление механизма ценообразования в сис\-те\-мах ресурсов.
  
  Определение роли и разумного числа посредников в распределении ресурсов.
  
  Эти и другие вопросы являются направлениями дальнейших исследований.


{\small\frenchspacing
{%\baselineskip=10.8pt
\addcontentsline{toc}{section}{Литература}
\begin{thebibliography}{99}
  \bibitem{1konov}
  Информатика: состояние, проблемы, перспективы~/ Под ред. И.\,А.~Соколова.~--- М.: 
ИПИ РАН, 2009. 46~с. ISBN-978-5-902030-69-0.
  
  \bibitem{2konov}
  \Au{Демичев А.\,П., Ильин В.\,А., Крюков А.\,П.}
  Введение в грид-технологии: Препринт.~--- М.: НИИЯФ МГУ, 2007.  87~с.
  
  \bibitem{3konov}
  {\sf http://www.systematic.ru/publikatsii/sx/art/310033/ po/309844/cp/1/br/309438/discart/310033.html}.
  
  \bibitem{5konov}
  \Au{Коновалов М.\,Г., Малашенко Ю.\,Е., Назарова~И.\,А.}
  Модели и методы управления заданиями в системах распределенных вычислительных 
ресурсов: Препринт.~--- М.: ВЦ РАН, 2009.  110~с.

  \bibitem{4konov} %5
  \Au{Xhafa F., Abraham~A.}
  Computational models and heuristic methods for Grid scheduling problems~// Future Generation 
Comput. Syst., 2010. Vol.~26. P.~608--621.
  
  
  \bibitem{6konov}
  \Au{Cho S., Lee M., In~J., Kim~B., Choi~E.}
  Policy based scheduling for resource allocation on grid~/ Eds. K.\,C.~Chang \textit{et al}.~// 
APWeb/WAIM 2007 Ws, LNCS, 2007. Vol.~4537. P.~229--234.
  
  \bibitem{7konov}
  \Au{Топорков В.\,В.}
Потоковые и жадные алгоритмы согласованного выделения ресурсов в распределенных системах~// Известия РАН. 
Теория и системы управ\-ле\-ния, 2007. No.\,2. P.~109--119.

  \bibitem{8konov}
  \Au{Vanderster D.\,C., Dimopoulos N.\,J., Sobie~R.\,J.}
  Metascheduling multiple resource types using the MMKP grid~// 7th IEEE/ACM Conference 
(International) on Grid Computing Proceedings, 2006. P.~231--237.
  
  \bibitem{9konov}
  \Au{Душин Ю.\,А.}
  Модель оценки стоимости гетерогенных ресурсов в Грид~// Системы и средства 
информатики. Спец. вып. Математические модели в информационных технологиях.~--- М.: 
ИПИ РАН, 2006. С.~163--172.
  
  \bibitem{10konov}
  \Au{Gamst M.}
  Greedy and metaheuristics for the offline scheduling problem in grid computing~// DTU 
Management Engeneering.~--- Technical University of Danemark, 2010. {\sf 
http://www.man.dtu.dk/\linebreak upload/institutter/ipl/publ/publikationer\%202010/\linebreak rapport2.2010.pdf}.
  
  \bibitem{11konov}
  \Au{Агаларов~Я.\,М.}
  Динамическая стратегия распределения вычислительных ресурсов локального узла 
GRID~// Системы и средства информатики. Вып.~17.~--- М.: ИПИ РАН, 2007. С.~17--29.
  
  \bibitem{12konov}
  \Au{Slegers J., Mitrani~I., Thomas~N.}
  Optimal dynamic server allocation in systems with on/off sources~/ Ed.\ K.~Wolter~// EPEW 
2007, LNCS, 2007. Vol.~4748. P.~186--199.
  
  \bibitem{13konov}
  \Au{Farzi S.}
  Efficient job scheduling in grid computing with modified artificial fish swarm algorithm~// 
Int. J. Comput. Theory Eng., 2009. Vol.~1. No.~1. 
{\sf http://www.ijcte.org/papers/003.pdf}.
  
  \bibitem{14konov}
  \Au{Mathiyalagan P., Dhepthie~U.\,R., Sivanandam~S.\,N.}
  Grid scheduling using enhanced PSQ algorithm~// Int.\ J. Comput. Sci. Eng., 
2010. Vol.~2. No.~2. P.~140--145.
  
  \bibitem{15konov}
  \Au{Kamalam G.\,K., Muralibhaskaran~V.}
  A~new heuristic approach: min-mean algorithm for scheduling meta-tasks on heterogenous 
computing systems~// Int. J. Comput. Sci. Network Security, 2010. Vol.~10. No.~1.
  
  \bibitem{16konov}
  \Au{Li B., Zhao~D.}
  Online algorithms for single machine schedulers to support advance reservations from grid 
jobs~/ Eds.\ R.~Perrott, B.~Chapman, J.~Subhlok, \textit{et al}.~// HPCC 2007, LNCS, 2007. Vol.~4782. P.~239--248.
  
  \bibitem{17konov}
  \Au{Nou R., Kounev~S., Torres~J.}
  Building online performance models of grid middleware with fine-grained load-balancing: A 
Globus Toolkit case study~/ Ed.\ K.~Wolter~// EPEW 2007, LNCS, 2007. Vol.~4748. P.~125--140.
  
  \bibitem{18konov}
  \Au{Yagoubi B., Slimani~Y.}
  Dynamic load balancing strategy for grid computing~// Trans. Eng. Comput. 
Technol., 2006. Vol.~13. P.~260--265.
  
  \bibitem{19konov}
  \Au{Saravanakumar E., Gomathy~P.}
   A~novel load balancing algorithm for computational grid~// Int.\ J. Comput.
Intelligence, 2010. Vol.~1. Issue~1. P.~20--26.
  
  \bibitem{20konov}
  \Au{Berten V., Gaujal B.}
  Brokering strategies in computational grids using stochastic prediction models~// Parallel 
Comput., 2007. Vol.~33. P.~238--249. {\sf www.sciencedirect.com}.
  
  \bibitem{21konov}
  \Au{Yu K.-M., Luo Z.-J., Chou~C.-H., Chen~C.-K., Zhou~J.}
  A~fuzzy neural network based scheduling algorithm for job assignment on computational 
grids~/ Eds. T.~Enokido, L.~Barolli, M.~Takizawa~// NBiS 2007, LNCS, 2007. Vol.~4658. 
P.~533--542.
  
  \bibitem{22konov}
  \Au{Ishii R.\,P., De Mello~R.\,F., Yang~L.\,T.}
  A~complex network-based approach for job scheduling in grid environments~/ Eds.\ R.~Perrott,
  B.~Chapman, J.~Subhlok, 
\textit{et al}.~// HPCC 2007, LNCS, 2007. Vol.~4782. P.~204--215.
  
  \bibitem{23konov}
  \Au{Al-Khateeb A., Abdullah~R., Rashid~N.\,A.}
  Job type approach for deciding job scheduling in grid computing systems~// J. Comput. 
Sci., 2009. Vol.~5. No.~10. P.~745--750.
  {\sf http://www.scipub.org/fulltext/jcs/jcs510745-750.pdf}.
  
  \bibitem{24konov}
  \Au{Shah S.\,C., Chahdary~S.\,H., Bashir~A.\,K., Park~M.\,S.}
  A~centralized location-based job scheduling algorithm for interdependent jobs in mobile ad hoc 
computational grids~// J.~Appl. Sci., 2010. Vol.~10. No.\,3. P.~174--181.
  
  \bibitem{25konov}
  \Au{Wei X., Ding~Z., Xing~S., Yuan~Y.}
  VJM: A~novel grid resource co-allocation model for parallel jobs~// Int. J. Grid  
Distributed Comput., 2009. Vol.~1. No.\,2.
  
  \bibitem{26konov}
  \Au{Ali G., Shaikh N.\,A., Shaikh~Z.\,A.}
  Integration of grid and agent systems to perform parallel computations in a heterogeneous and 
distributed environment~// Aust. J. Basic Appl. Sci., 2009. Vol.~3. No.\,4. 
P.~3857--3863.
  
  \bibitem{27konov}
  \Au{Vazquez C., Huedo~E.\,S. Montero~R.\,S., Llorente~I.\,M.}
  Federation of TeraGrid, EGEE and OSG infrastructures through a metascheduler. Preprint 
submitted to Future Generation Computer Systems, 2010. 
  {\sf http://dsa-research.org/doku.php?id=publications:grid:utility}.
  
  \bibitem{28konov}
  \Au{Ranjan R., Harwood A., Buyya~R.}
  SLA-based cooperative superscheduling algorithms for computational grids~// 8th IEEE  
Conference (International) on Cluster Computing (Cluster 2006) Proceedings~// IEEE Computer 
Society Press, 2006. abs/cs/0605057.
  
  \bibitem{29konov}
  \Au{Li J., Sim K.\,M., Yahyapour~R.}
  Negotiation strategies considering opportunity functions for grid scheduling~/ Eds.\ 
  A.-M.~Kermarrec, L.~Boug$\acute{\mbox{e}}$, T.~Priol~// Euro-Par, 2007, LNCS, 2007. 
Vol.~4641. P.~447--456.
  
  \bibitem{30konov}
  \Au{Vanmechelen K., Broeckhove~J.}
  A~comparative analysis of single-unit vickrey auctions and commodity markets for realizing 
grid economies with dynamic pricing~/ Eds.\ D.\,J.~Veit, J.~Altmann~// GECON 2007, LNCS, 
2007. Vol.~4685. P.~98--111.
  
  \bibitem{31konov}
  \Au{Krasnotcshekov V., Vakhitov~A.}
  Adaptive scheduling and resource assessment in grid~/ Ed.\ V.~Malyshkin~// PaCT 2007, LNCS, 
2007. Vol.~4671. P.~240--244.
  
  \bibitem{32konov}
  \Au{Агаларов Я.\,М.}
  Функция стоимости ресурсов в экономической модели грид~// Информатика и её 
применения, 2008. Т.~2. Вып.~3. С.~27--34.

\label{end\stat}
  
  \bibitem{33konov}
  \Au{Коновалов М.\,Г., Душин~Ю.\,А., Малашенко~Ю.\,Е., Шоргин~С.\,Я.}
  Модель взаимодействия потребителей с удаленными вычислительными ресурсами через 
посредников~// Системы и средства информатики. Вып.~19.~--- М.: Наука,1989. С.~5--33.
 \end{thebibliography}
}
}

\end{multicols}
 
 
 
 