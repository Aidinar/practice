\newcommand{\cov}{\textrm{cov}}
%\newcommand{\indic}{\mathbb{1}}
\newcommand{\Obig}{\textsf{O}}
\newcommand{\osml}{\textsf{o}}
\newcommand{\hsig}{\hat\sigma^2}
\newcommand{\Yljk}{Y_{\lambda;j,\mathbf{k}}}

\newcommand{\Yljks}{Y_{\lambda';j',\mathbf{k'}}}
\newcommand{\muljk}{\mu_{\lambda;j,\mathbf{k}}}
\newcommand{\solj}{\sigma_{\lambda;j}}
\newcommand{\soljs}{\sigma_{\lambda';j'}}
\newcommand{\solz}{\sigma_{\lambda;0}}
\newcommand{\silz}{\sigma_{1;0}}
\newcommand{\siilz}{\sigma_{2;0}}
\newcommand{\siiilz}{\sigma_{3;0}}
\newcommand{\slj}{\solj^2}
\newcommand{\sljs}{\soljs^2}
\newcommand{\hslj}{\hat\sigma^2_{\lambda;j}}
\newcommand{\hsljs}{\hat\sigma^2_{\lambda';j'}}
\newcommand{\hslz}{\hat\sigma_{\lambda;0}^2}
\newcommand{\Tlj}{T_{\lambda;j}}
\newcommand{\hTlj}{\hat T_{\lambda;j}}

\newcommand{\indYjklTj}{\Ik_{\left|\Yljk\right|\leqslant \Tlj}}
\newcommand{\indYjkgTj}{\Ik_{\left|\Yljk\right|>\Tlj}}
\newcommand{\prbYjkgTj}{\p\left( \left|\Yljk\right|>\Tlj \right)}
\newcommand{\indYjklhTj}{\Ik_{\left|\Yljk\right|\leqslant \hTlj}}
\newcommand{\indYjkghTj}{\Ik_{\left|\Yljk\right|>\hTlj}}
\newcommand{\sumljk}{\sum\limits_{\lambda,j,\mathbf{k}}}

\def\stat{markin}

\def\tit{АСИМПТОТИКИ ОЦЕНКИ РИСКА ПРИ ПОРОГОВОЙ ОБРАБОТКЕ ВЕЙВЛЕТ-ВЕЙГЛЕТ КОЭФФИЦИЕНТОВ В ЗАДАЧЕ ТОМОГРАФИИ}

\def\titkol{Асимптотики оценки риска при пороговой обработке вейвлет-вейглет коэффициентов в задаче томографии}

\def\autkol{А.\,В.~Маркин, О.\,В.~Шестаков}
\def\aut{А.\,В.~Маркин$^1$, О.\,В.~Шестаков$^2$}

\titel{\tit}{\aut}{\autkol}{\titkol}

%{\renewcommand{\thefootnote}{\fnsymbol{footnote}}\footnotetext[1]
%{Исследования выполнены при частичной поддержке РФФИ, гранты 08-01-00567, 08-01-91205, 09-01-12180.}}

\renewcommand{\thefootnote}{\arabic{footnote}}
\footnotetext[1]{Московский государственный университет им.\ М.\,В.~Ломоносова, 
факультет вычислительной математики и кибернетики, кафедра математической статистики, artem.v.markin@mail.ru}
\footnotetext[2]{Московский государственный университет им.\ М.\,В.~Ломоносова, 
факультет вычислительной математики и кибернетики, кафедра математической статистики,
oshestakov@cs.msu.su}


\Abst{Рассмотрена задача реконструкции изображения по радоновскому образу с помощью вейв\-лет-вейг\-лет разложения. 
Исследованы свойства оценки риска пороговой обработки вейг\-лет-коэф\-фи\-ци\-ен\-тов, такие как состоятельность и 
асимптотическая нормальность.}

\KW{вейвлеты; томография; пороговая обработка; оценка риска; предельное распределение}

     \vskip 18pt plus 9pt minus 6pt

      \thispagestyle{headings}

      \begin{multicols}{2}

      \label{st\stat}


\section{Введение}

Вейвлет-преобразование является весьма популярным и удобным методом обработки нестационарных сигналов и 
изображений. Одна из основных задач, для которых используются вейв\-ле\-ты,~---
удаление шума и сжатие. Эти операции производятся путем пороговой обработки 
вейв\-лет-коэффициентов. Кроме того, вейвлеты могут быть использованы для обращения 
линейных операторов, таких, например, как преобразование Радона. В этом случае 
пороговая обработка выполняет задачу регуляризации соответствующей формулы обращения.

Пусть на плоскости $(x,\,y)$ задана функция~$f$. Определим образ 
(или проекции) Радона~$\mathcal{R}f$ как набор интегралов от~$f$ по всевозможным прямым плоскости
\begin{equation}
\label{eq_radonTransform}
\mathcal{R}f(s,\theta)=\int\limits_{L_{s,\theta}}f\left(x,\,y\right)\,dl\,,
\end{equation}
где
\begin{equation*}
L_{s,\theta}=\left\{ (x,\,y): x\cos\theta+y\sin\theta-s=0 \right \}\,.
\end{equation*}
Формула обращения преобразования~(\ref{eq_radonTransform}) впервые была получена 
Радоном, ее можно записать в следующем виде~\cite{Natterer}:
\begin{equation}
\label{eq_radonInverse}
f = \fr{1}{2}\mathcal{R}^{\#}\mathcal{I}^{-1}\mathcal{R}f\,,
\end{equation}
где $\mathcal{R^{\#}}$~--- оператор обратного проецирования:
\begin{equation*}
\left(\mathcal{R^{\#}}g\right)(x,y)=\int\limits_0^{2\pi}g(x\cos\theta+y\sin\theta,\theta)\,d\theta\,;
\end{equation*}
$\mathcal{I}$~--- потенциал Рисса:
\begin{equation}
\label{eq_RieszPoten}
\left(\mathcal{F}_1\mathcal{I}^\alpha g\right) (\omega) = |\omega|^{-\alpha}\left(\mathcal{F}_1 g\right)(\omega)\,,
\end{equation}
а $\mathcal{F}_k$~--- $k$-мерное преобразование Фурье.

Для точного восстановления~$f$ требуется точное знание всевозможных проекций~$\mathcal{R}f(s,\,\theta)$. 
На практике же имеют дело с конечным числом проекций, причем в проекциях присутствует шум.
При этом задача томографии является некорректной, т.\,е.\ малые изменения в проекциях могут 
при\-вес\-ти к восстановлению изображения, существенно отличающегося от исходного. Математически
это выражается в наличии множителя~$|\omega|$ в формуле~(\ref{eq_RieszPoten}) (и, следовательно, 
в~(\ref{eq_radonInverse})), который <<подчеркивает>> высокие частоты.

Выход видится в регуляризации~(\ref{eq_radonInverse}) путем умножения~$|\omega|$ на некоторый множитель, 
называемый частотным фильтром (или стабилизирующим множителем)~\cite{TikhonovArsenin}. 
Общая идея регуляризации такова:\linebreak
немного <<испортить>> проекционные данные, подавив влияние 
высоких частот, но при этом обеспечить реконструкцию, близкую к оригиналу. Подроб\-нее о 
регуляризации формулы обращения можно прочитать в монографии~\cite{Herman}.

\section{Вейвлет-вейглет разложение}

Задачу томографии можно решить и с помощью вейвлетов. Пусть~$\phi(t)$ и~$\psi(t)$~--- 
одномерные отцовский и материнский вейвлеты. Определим
\begin{align*}
\phi_{j,k_1,k_2}(x,y) &= 2^{j} \phi\left(2^jx-k_1\right) \phi\left(2^jy-k_2\right)\,;\\
\psi^{[1]}_{j,k_1,k_2}(x,y) &= 2^{j} \phi\left(2^jx-k_1\right) \psi\left(2^jy-k_2\right)\,;
\end{align*}

\noindent
\begin{align*}
\psi^{[2]}_{j,k_1,k_2}(x,y) &= 2^{j} \psi\left(2^jx-k_1\right) \phi\left(2^jy-k_2\right)\,;\\
\psi^{[3]}_{j,k_1,k_2}(x,y) &= 2^{j} \psi\left(2^jx-k_1\right) \psi\left(2^jy-k_2\right)\,.
\end{align*}
Заметим, что параметр масштаба~$j$ контролирует сразу обе функции в произведении. 
Это так называемое тензорное произведение двух одномерных кратномасштабных анализов~\cite{Daub}. 
Тогда набор функций $\left\{ \phi_{j_0,k_1,k_2}, \, \psi^{[\lambda]}_{j,k_1,k_2}, \right\}$, 
где $j$, $k_1$, $k_2\in\mathbb{Z}$, $j\geq j_0$, $\lambda=\overline{1,3}$, 
будет ортонормированным базисом~$\mathbf{L}^2(\mathbb{R}^2)$.

Донохо~\cite{DonohoWVD} решил задачу обращения ряда линейных операторов 
с помощью вейвлетов и родственных им функций специального вида, названных вейглетами (\textit{vaguelettes}). 
Вейглеты для обращения оператора Радона выглядят так:
\begin{multline*}
\xi^{[\lambda]}_{j,k_1,k_2}(s,\,\theta)=
\int\limits_{-\infty}^\infty|\omega|\left(\mathcal{F}_2\psi^{[\lambda]}_{j,k_1,k_2}\right)\times{}\\
{}\times \left( \omega\cos\theta,\,\omega\sin\theta \right)\exp(i2\pi s\omega)\,d\omega\,.
%\label{eq_vagueletteDef}
\end{multline*}
Идея метода реконструкции заключается в том, что вейглет-коэффициенты проекций~$\mathcal{R}f(s,\theta)$ 
равны вейвлет-коэффициентам исходной функции~$f(x,y)$:
\begin{equation*}
\left[\mathcal{R}f,\,\xi^{[\lambda]}_{j,k_1,k_2}\right] = \left\langle f,\,\psi^{[\lambda]}_{j,k_1,k_2}\right\rangle\,,
\end{equation*}
и поэтому
\begin{multline}
f = \sum\limits_{k_1,k_2}\left[\mathcal{R}f,\,\tau_{j_0,k_1,k_2}\right] \phi_{j_0,k_1,k_2} +{}\\
{}+ \sum\limits_{j\geqslant j_0,k_1,k_2,\lambda} \left[\mathcal{R}f,\,\xi^{[\lambda]}_{j,k_1,k_2}\right] \psi^{[\lambda]}_{j,k_1,k_2}\,,
\label{eq_radonInverseWVD}
\end{multline}
где
\begin{multline*}
\tau_{j_0,k_1,k_2}(s,\,\theta)=\int\limits_{-\infty}^\infty|\omega|\left(\mathcal{F}_2\phi_{j_0,k_1,k_2}\right)\times{}\\
{}\times \left( \omega\cos\theta,\,\omega\sin\theta \right)
\exp\left(i2\pi s\omega\right)\,d\omega\,.
\end{multline*}
Регуляризация вейвлет-вейглет формулы~(\ref{eq_radonInverseWVD}) производится с 
помощью мягкой пороговой обработки вейглет-коэффициентов (см.\ разд.~\ref{sect_ThreshholdingTomo}).

\section{Дискретизация и модель шума}

Пусть функция $f(x,y)$ задана на квадрате $[0,\,1]\;\times$\linebreak $\times\;[0,\,1]$. Разбив стороны квадрата на~$N=2^J$ 
равных частей и вычислив значения~$f$ в точках отсчета, получим дискретизованную версию~$f$. Одна\-ко на практике 
нередко бывает удобно нормировать длину отрезка разбиения и рас\-сматривать вместо~$f$ ее <<растянутую>>
версию~--- функцию~$\bar f(Nx,Ny)\;=$\linebreak $={f}(x,y)$. Тогда для вейвлет-коэффициентов функции~$f$ справедливо равенство:
\begin{multline}
\left\langle f,\, \psi^{[\lambda]}_{j,u_1,u_2}\right\rangle ={}\\
{}= \iint f(x,y)\,2^j\overline{\psi^{[\lambda]}\left(2^jx-k_1,\,2^jy-k_2\right)}\,dx\,dy ={}\\
{}=\left(\mathcal{W}^{[\lambda]}f\right)\left(2^{-j},k_1,k_2\right)={}\\
{}=\fr{1}{N}\left(\mathcal{W}^{[\lambda]}\bar f\right)\left(N\,2^{-j},k_1,k_2\right)\,.
\label{eq_contToDiscrCoeff}
\end{multline}
Заметим, что при работе с растянутой функцией растягиваются и вейвлет-функции.
Коэффициенты аппроксимации, получаемые через скалярное произведение~$f$ и~$\phi$, не рассматриваются, 
так как пороговая обработка (см.\ разд.~4) применяется к коэффициентам деталей, которые дают функции~$\psi^{[\lambda]}$. 
Далее везде, кроме разд.~\ref{sect_RegularityTomo}, предполагается, что используются именно коэффициенты 
растянутой версии функции~$f$.

Задача томографии ставится следующим образом. Имеются наблюдения~$X$, состоящие из 
проекций~$\mathcal{R}f$ функции~$f$ и шума~$\epsilon$:
\begin{equation*}%\label{eq_tomoTask}
X=\mathcal{R}f+\epsilon\,, 
\end{equation*}
$\epsilon$~--- независимые нормальные случайные величины с нулевым средним и дисперсией~$\sigma^2$. 
Необходимо восстановить~$f$ по~$X$. При этом при достаточно большом~$N$~\cite{KolaczykArticle}
\begin{equation}
\left.
\begin{array}{rl}
\e \left[X,\,\xi^{[\lambda]}_{j,k_1,k_2}\right] &= \left[\mathcal{R}f,\,\xi^{[\lambda]}_{j,k_1,k_2}\right]\,;\\[9pt]
\D \left[X,\,\xi^{[\lambda]}_{j,k_1,k_2}\right] &= \sigma^2 \left\|\xi^{[\lambda]}_{j,k_1,k_2}\right\|_2^2=\sigma^2_{\lambda;j}\,;\\[9pt]
\left\|\xi^{[\lambda]}_{j,k_1,k_2}\right\|_2^2 &= 2^j \left\|\xi^{[\lambda]}_{0,0,0}\right\|_2^2\,.
\end{array}
\right \}
\label{eq_expctVageuletteCoef}
\end{equation}
Как видим, дисперсия коэффициентов растет вмес\-те с уровнем разложения. Это является следствием 
некорректности задачи томографии. При этом вейглеты не ортогональны, а почти ортогональны. И, 
стало быть, вейг\-лет-коэф\-фи\-ци\-ен\-ты не независимы, а почти независимы. Однако нередко 
этим фактом пренебрегают, так как исследование этой зависимости сопряжено с рядом трудностей. 
И потому порог выбирается исходя из предположения независимости коэффициентов. Как будет видно 
далее, уже только тот факт, что дисперсия растет на каждом уровне, заметно влияет на оценку 
риска пороговой обработки.

\section{Пороговая обработка}\label{sect_ThreshholdingTomo}

Мягкая пороговая функция определяется следующим образом:
\begin{equation*}
\rho(x, T)=
\begin{cases}
x-T & \text{при } x>T\,;\\
x+T & \text{при } x<-T\,;\\
0 & \text{при } |x|\leq T\,.
\end{cases} 
\end{equation*}
Эта функция применяется к вейглет-ко\-эф\-фи\-ци\-ен\-там проекций.

Допустим, что размер изображения равен $N^2\;=$\linebreak $=2^{2J}=L$, разложение идет до уровня~$J-1$. 
В~качестве порога взят порог Колашика~\cite{KolaczykArticle, KolaczykThesis}:
\begin{equation*}
\Tlj = \sqrt{2\ln 2^{2j}} \, 2^{j/2}\sigma  \left\|\xi^{[\lambda]}_{0,0,0}\right\|_2\,.
\end{equation*}
В случае использования оценки дисперсии шума~$\hsig$ порог принимает вид
\begin{equation*}
\hTlj = \sqrt{2\ln 2^{2j}}\,2^{j/2}\hat\sigma \left\|\xi^{[\lambda]}_{0,0,0}\right\|_2\,.
\end{equation*}
Идея выбора такого порога схожа с идеей выбора порога~\textit{VisuShrink} 
$T=\sigma\sqrt{2\ln N}$ (одномерный случай, $N$~--- размер сигнала): при таком пороге 
убирается почти весь шум~\cite{DJideal, DJunkn}. Это следует из того факта, что если $Z_1,\ldots,Z_N$~--- 
независимые стандартные нормальные случайные величины, то
\begin{equation*}
\p\left( \underset{1\leqslant i\leqslant N}{\max}|Z_i| > \sqrt{2\ln N} \right) \rightarrow 0\
\mbox{при}\ N\rightarrow\infty\,.
\end{equation*}

Пороговая обработка идет с уровня~$j_M$, т.\,е.\ в формуле~(\ref{eq_radonInverseWVD}) 
$j_0=j_M$ ($j_M$ определим ниже). Риск~$r(f)$ такой пороговой обработки определяется следующим образом:
\begin{multline}
r(f)=\sum\limits_{j=j_M}^{J-1}\sum_{\lambda=1}^3\sum_{k_1=0}^{2^j-1}
\sum_{k_2=0}^{2^j-1}\e
\left\{ \left\langle f,\,\psi^{[\lambda]}_{j,k_1,k_2}\right\rangle - {}\right.\\
\left.{}-\rho\left(\left[X,\,\xi^{[\lambda]}_{j,k_1,k_2}\right],\,\Tlj\right) \right\}^2\,.
\label{eq_riskEstimDefTomo}
\end{multline}
Так как на практике коэффициенты $\left\langle f,\,\psi^{[\lambda]}_{j,k_1,k_2}\right\rangle$ 
неизвестны, то строят оценку риска. Например,
на основе функции~$\Phi(x,T)$~\cite{Mallat}:
\begin{equation*}
\Phi(x,\Tlj)=
\begin{cases}
x-\slj & \text{при } x\leqslant \Tlj^2\,;\\
\slj+\Tlj^2 & \text{при } x> \Tlj^2\,.
\end{cases} 
\end{equation*}
Оценка риска принимает вид:
\begin{equation*}
\tilde r(f)=\sum\limits_{j=j_M}^{J-1}\sum\limits_{\lambda,k_1,k_2}\Phi\left( 
\left| \left[X,\,\xi^{[\lambda]}_{j,k_1,k_2}\right] \right|^2 ,\,\Tlj\right)\,.
\end{equation*}
Если вместо~$\sigma^2$ используется оценка~$\hsig$, то
\begin{equation*}
\hat r(f)=\sum\limits_{j=j_M}^{J-1}
\sum\limits_{\lambda,k_1,k_2}\hat\Phi\left( 
\left| \left[X,\,\xi^{[\lambda]}_{j,k_1,k_2}\right] \right|^2 ,\,\hTlj\right)\,,
\end{equation*}
где
\begin{equation*}
\hat\Phi(x,\hTlj)=
\begin{cases}
x-\hslj & \text{при } x\leqslant \hTlj^2\,;\\
\hslj+\hTlj^2 & \text{при } x> \hTlj^2\,.
\end{cases} 
\end{equation*}


В работах~\cite{MarkinShestakovConsist, MarkinLimitDistr} рассмотрены асимптотические свойства оценки 
риска пороговой обработки вейв\-лет-коэффициентов в одномерном случае при прямом наблюдении~$f$. 
Показано, что $(\hat r -r)/N^a$ сходится по вероятности к нулю и по распределению к нормальному 
закону при соответствующих~$a$.
Величина~$a$ существенно зависит от свойств оценки~$\hsig$. Однако даже при весьма общих ограничениях 
на моменты~$\hsig$ порядок $a=1$ обеспечивал
сходимость по вероятности к нулю. Ниже будет показано, что в задаче томографии для сходимости 
по вероятности к нулю недостаточно делить на число коэффициентов ($N^2=L$),
т.\,е.\ некоторый аналог закона больших чисел уже не выполнен. Важнейшим фактором 
является то, что~$f$ наблюдается через оператор Радона~$\mathcal{R}$, обратный к 
которому не является непрерывным (т.\,е.\ ограниченным).

\section{Регулярность функции и~вейвлет-коэффициенты}\label{sect_RegularityTomo}

Известно (см., например,~\cite{Mallat}), что если функция~$f(x,y)$ является регулярной по Липшицу 
с параметром $0\leq\alpha\leq 1$, т.\,е.\
\begin{multline*}
\left|f(x_1,y_1)-f(x_2,y_2)\right|\leq{}\\
{}\leq C \left( |x_1-x_2|^2 + |y_1-y_2|^2 \right)^{\alpha/2}
\end{multline*}
для некоторой константы~$C$, не зависящей от $(x_1,y_1)$ и $(x_2,y_2)$, то существует не зависящая от~$J$, 
$j$, $k_1$ и $k_2$ константа~$A$ такая, что
\begin{equation*}
\left(\mathcal{W}^{[\lambda]}f\right)\left(2^{-j},k_1,k_2\right)\leq \fr{A}{2^{j(\alpha+1)}}\,.
\end{equation*}
В отечественной литературе вместо регулярности по Липшицу обычно используется термин <<непрерывность 
по Гёльдеру>>.
С учетом~(\ref{eq_contToDiscrCoeff}) получаем
\begin{equation*}
\left(\mathcal{W}^{[\lambda]}\bar f\right)\left(N\cdot 2^{-j},k_1,k_2\right) \leqslant \frac{A\cdot 2^J}{2^{j(\alpha+1)}}\,.
\end{equation*}

\textit{Предположение о регулярности~$f$: будем полагать, что функция~$f$ является регулярной по Липшицу с 
показателем~$\alpha>0$}. Будем считать, что пороговая обработка ведется с уровня 
$j_M\geq J/(\alpha+1)$. Заметим, что $J-j_M\rightarrow\infty$ при $J\rightarrow\infty$. 
Тогда при определенном выборе вейвлет-базиса~\cite{Mallat} найдется константа~$C_1$ такая, что для 
всех $j\geq j_M$ выполнено
\begin{equation}
\label{eq_WaveletCoeffUpperBoundTomo}
\left(\mathcal{W}^{[\lambda]}\bar f\right)\left(N\cdot 2^{-j},k_1,k_2\right) \leqslant C_1\,,
\end{equation}
причем $C_1$ не зависит от~$N$. Значит, математические ожидания в~(\ref{eq_expctVageuletteCoef}) ограничены.

В работе используется буква~$C$ (с индексом или без индекса) для обозначения констант, причем в 
разных местах~--- вообще говоря, разных.

\section{Асимптотика оценки риска при~известной дисперсии шума}\label{sect_ConsitKnownSTomo}

В работе~\cite{MarkinLimitDistr} показано, что в одномерном случае при известной дисперсии шума 
разность риска и оценки риска при делении на $\sqrt{N}$ сходится по распределению к нормальному 
закону. В задаче томографии уже надо делить не на~$\sqrt{L}$, а на~$L$.

Для краткости введем обозначения:
\begin{align*}
\Yljk &= \left[X,\,\xi^{[\lambda]}_{j,k_1,k_2}\right]\,;\\
\muljk &= \left\langle f,\,\psi^{[\lambda]}_{j,k_1,k_2}\right\rangle\,,
\end{align*}
где $\mathbf{k}=\left(k_1,\,k_2\right)$. Еще раз напомним, что~$\muljk$ рассматриваются 
как коэффициенты растянутой версии дискретизованной функции~$f$. С учетом предположения об 
ортогональности вейглетов получаем
\begin{equation}
\label{eq_YljkNormalDistributed}
\Yljk \sim \mathcal{N}\left(\muljk,\,\slj\right)\,,
\end{equation}
причем $\Yljk$~--- независимые случайные величины.

\medskip

\noindent
\textbf{Теорема 1.}
\textit{Пусть справедливы предположения о регулярности~$f$ из разд.~\ref{sect_RegularityTomo}. 
При известной дисперсии шума в задаче томографии}
\begin{equation*}
\fr{\tilde r(f)-r(f)}{L \sqrt{ b_2 \left( \silz^4 + \siilz^4 + \siiilz^4 \right) }} \Rightarrow \mathcal{N}(0,\,1)
\end{equation*}
\textit{при} $L\rightarrow\infty$, \textit{где} $b_2=2/(2^4-1)=2/15$.

\medskip

\noindent
Д\,о\,к\,а\,з\,а\,т\,е\,л\,ь\,с\,т\,в\,о.\ 
Представим разность оценки риска и самого риска в виде
\begin{multline*}
\tilde r-r=\sumljk\left(\Yljk^2-\slj\right)\indYjklTj +{}\\
\!\!\!\!{}+ \sumljk\left(\slj+\Tlj^2\right)\indYjkgTj - {}
\end{multline*}

\noindent
\begin{multline}
\ \ {}- \sumljk\e\left(\Yljk^2-\slj\right)\indYjklTj -{}\\
{}- \sumljk\e\left(\slj+\Tlj^2\right)\indYjkgTj ={}\\
{}= \sumljk\left(\Yljk^2-\e\Yljk^2\right) -{}\\
{}- \sumljk\left(\Yljk^2-\slj\right)\indYjkgTj + {}\\
{}+ \sumljk\e\left(\Yljk^2-\slj\right)\indYjkgTj +{}\\
{}+ \sumljk\left(\slj+\Tlj^2\right)\indYjkgTj -{}\\
{}- \sumljk\left(\slj+\Tlj^2\right)\prbYjkgTj\,.
\label{eq_diffRiskEstimKnownS}
\end{multline}
Покажем, что при делении на~$L$ первая сумма в~(\ref{eq_diffRiskEstimKnownS}) сходится по распределению
к нормальному закону, а остальные суммы~--- к нулю по вероятности.

Итак, рассмотрим первую сумму в~(\ref{eq_diffRiskEstimKnownS}). Имеем
\begin{multline}
D_L^2 = \D\sumljk\Yljk^2 = \sumljk \D \Yljk^2={}\\
{}=\sum\limits_\lambda \sum_{j=j_M}^{J-1} \sum_{\mathbf{k}}
\left( 2\solj^4 + 4\muljk^2\slj \right)={}\\
{}= \sum\limits_\lambda \sum_{j=j_M}^{J-1}\left\{ 2\cdot 2^{2j}
\solz^4 \cdot 2^{2j} + \sum_{\mathbf{k}}4\muljk^2 2^j \solz^2 \right\} \simeq{}\\
{}\simeq \sum\limits_\lambda \sum_{j=j_M}^{J-1} 2\cdot 2^{4j}
\solz^4 = \sum\limits_\lambda 2\solz^4 \frac{2^{4J} - 2^{4j_M}}{2^4 - 1} \simeq{}\\
{}\simeq \fr{2}{15} 2^{4J} \left( \silz^4 + \siilz^4 + \siiilz^4 \right)\,.
\label{eq_DLknownS}
\end{multline}
Знак~$\simeq$ означает, что при $J\rightarrow\infty$ предел отношения левой и правой частей~(\ref{eq_DLknownS}) 
равен единице. Если выполнено условие Линдеберга, т.\,е.\ для любого~$\delta>0$
\begin{multline}
\fr{1}{D_L^2}\sumljk\e\left\{ \left( \Yljk^2 - \muljk^2 - \slj \right)^2\times{}\right.\\
\left.{}\times \Ik_{\left|\Yljk^2 - \muljk^2 - \slj\right|>\delta D_L} \right\} \rightarrow 0\,,
\label{eq_LindCondTomo}
\end{multline}
то будет иметь место сходимость к нормальному распределению. Так как~$D_L$ имеет порядок~$L$ и чис\-ло слагаемых 
в~(\ref{eq_LindCondTomo}) имеет порядок~$L$, то достаточно показать, что при $L\rightarrow\infty$
\begin{multline*}
\e\left\{ \fr{\left( \Yljk^2 - \muljk^2 - \slj \right)^2}{D_L} \times{}\right.\\
\left.{}\times\Ik_{\left(\Yljk^2 - \muljk^2 - \slj\right)^2/D_L>\delta^2 D_L} \right\} \rightarrow 0\,.
\end{multline*}
А последнее выполнено потому, что у случайных величин вида $\left( \Yljk^2 - \muljk^2 - \slj \right)^2\!/D_L$ 
конечные математические ожидания и $D_L\rightarrow\infty$.

Теперь рассмотрим вторую сумму в~(\ref{eq_diffRiskEstimKnownS}). В силу~(\ref{eq_YljkNormalDistributed}) имеем
\begin{multline*}
\p\left( |\Yljk| > \Tlj \right) < {}\\
{}< \frac{\exp\left( -(\Tlj-\muljk)^2/(2\slj) \right)}{\Tlj} +{}\\
{}+ \frac{\exp\left( -(\Tlj+\muljk)^2/(2\slj) \right)}{\Tlj} \leqslant \fr{C}{2^{5j/2} \sqrt{j} }
\end{multline*}
при $J \rightarrow \infty$ (и, следовательно, $j \rightarrow \infty$). Это можно получить из 
следующей цепочки равенств:
\begin{multline*}
\exp\left( -\fr{(\Tlj-\muljk)^2}{2\slj} \right) = {}\\
{}=\exp\left( -\fr{\Tlj^2}{2\slj} + \fr{2\Tlj\muljk}{2\slj} - \fr{\muljk^2}{2\slj} \right) ={}\\
{}= \exp\left( -\ln 2^{2j} + \fr{\sqrt{2\ln(2^{2j})}\muljk}{2^{j/2}\solz} - 
\fr{\muljk^2}{2\slj} \right) \simeq{}\\
{}\simeq 2^{-2j}\mbox{ при }j \rightarrow \infty\,,
\end{multline*}
так как
\begin{equation*}
\fr{\sqrt{2\ln(2^{2j})}\muljk}{2^{j/2}\solz} \rightarrow 0\quad\text{и}\quad\fr{\muljk^2}{2\slj} \rightarrow 0\,.
\end{equation*}
С помощью неравенств Чебышёва и Коши--Бу\-ня\-ков\-ского получаем для любого $\delta>0$ при $J\rightarrow\infty$
\begin{multline*}
\p\left( \fr{ \left|\sumljk\left(\Yljk^2-\slj\right)\indYjkgTj \right|}{D_L} > \delta \right) \leq{}\\
{}\leq \fr{ \e\left| \sumljk\left(\Yljk^2-\slj\right)\indYjkgTj \right| }{\delta D_L} \leq {}\\
{}\leq \fr{ \sumljk \e\left| \Yljk^2-\slj\right|\indYjkgTj }{\delta D_L} \leq{}\\
{}\leq \fr{ \sumljk \sqrt{ \e\left( \Yljk^2-\slj\right)^2 \p\left( |\Yljk| > \Tlj \right) } }{\delta D_L} 
\leq{}
\end{multline*}

\noindent
\begin{multline*}
{}\leq \fr{1}{\delta D_L}\sumljk 
\left  ( \vphantom{4\cdot 2^j\solz^2\muljk^2  C\cdot 2^{-5j/2} j^{-1/2}}
\left(
\muljk^4 + 2\cdot 2^{2j}\solz^4 + {}\right.\right.\\
\left.\left.{}+4\cdot 2^j\solz^2\muljk^2 
\right) C\cdot 2^{-5j/2} j^{-1/2} 
\right )^{1/2} \rightarrow 0\,.
\end{multline*}
Аналогично проводятся рассуждения для оставшихся сумм в~(\ref{eq_diffRiskEstimKnownS}).~$\square$

\section{Свойства оценки риска при~использовании оценки дисперсии шума}

В работе~\cite{MarkinShestakovConsist} показано, что при достаточно слабых ограничениях 
на моменты оценки дисперсии шума для сходимости разности риска и его оценки к нулю по 
вероятности ее надо нормировать числом вейвлет-коэффициентов, т.\,е.\ порядок знаменателя 
вырастает почти на~1/2. Покажем, что в задаче томографии порядок тоже повышается почти на~1/2, 
но знаменатель уже будет много больше числа коэффициентов.

Введем обозначение
\begin{equation*}
\hslj = 2^j \hsig \left\|\xi^{[\lambda]}_{0,0,0}\right\|_2^2\,.
\end{equation*}

\medskip
\noindent
\textbf{Теорема 2.} \textit{Пусть справедливы предположения о регулярности~$f$. 
Пусть $\hsig$~--- оценка дисперсии, $\e\hsig-\sigma^2=\nu_L$ и 
$\D\hsig=\theta_L=\Obig(L^{-\beta})$, $\nu_L=\osml(1)$, $\beta>0$. 
Тогда при $L\rightarrow\infty$ выполнено}
\begin{equation}
\label{eq_ConsistTomo32}
\fr{\hat r(f)-r(f)}{L^{3/2}} \xrightarrow{\textsf{P}} 0\,.
\end{equation}

\medskip

\noindent
Д\,о\,к\,а\,з\,а\,т\,е\,л\,ь\,с\,т\,в\,о.\
Подобно доказательству теоремы~3 в~\cite{MarkinShestakovConsist} запишем
\begin{equation*}
\hat r-r = S_1 + S_2\,,
\end{equation*}
где
\begin{multline}
S_1 = \sumljk\left(\Yljk^2-\hslj\right) -{}\\
{}- \sumljk\e\left(\Yljk^2-\slj\right)\,; \label{eq_riskSplitSoTomo}
\end{multline}

\vspace*{-6pt}

\noindent
\begin{multline*}
S_2 = - \sumljk\left(\Yljk^2-\hslj\right)\indYjkghTj +{}\\
\!\!{}+ \sumljk\left(\hslj+\hTlj^2\right)\indYjkghTj +{} 
\end{multline*}

\noindent
\begin{multline}
{}+ \sumljk\e\left(\Yljk^2-\slj\right)\indYjkgTj -{}\\
{}- \sumljk\e\left(\slj+\Tlj^2\right)\indYjkgTj\,.
\label{eq_riskSplitStTomo}
\end{multline}
Далее будет показано, что при делении на $L^{3/2}$ и~$S_1$, и~$S_2$ сходятся к нулю по вероятности.

Сначала рассмотрим~$S_1$: по неравенству Чебышёва при любом $\delta>0$
\begin{multline}
\p\left( \fr{|S_1|}{L^{3/2}} > \delta \right) \leq{}\\
{}\leq
\fr{ \e\left( \sumljk \left( \Yljk^2 - \hslj - \e\Yljk^2 + \slj \right) \right)^2 }{\delta^2 L^3} ={}\\
{}= \fr{ \sumljk \e\left( \Yljk^2 - \hslj - \e\Yljk^2 + \slj \right)^2 }{ \delta^2 L^3 } + {}\\
{}+ \fr{1}{\delta^2 L^3}
 \sum \e\left( \Yljk^2 - \hslj - \e\Yljk^2 + \slj \right)\times{}\\
 {}\times \left( \Yljks^2 - \hsljs - \e\Yljks^2 + \sljs \right)\,.
\label{eq_riskSplitUnknSTomo}
\end{multline}
Во второй сумме~(\ref{eq_riskSplitUnknSTomo}) суммирование идет по индексам 
$(\lambda,j,\mathbf{k})\ne(\lambda',j',\mathbf{k}')$. Понятно, что первое слагаемое в~(\ref{eq_riskSplitUnknSTomo}) 
стремится к нулю~--- в сумме всего порядка~$L$ слагаемых, они имеют порядок не выше~$L$ и 
сумма делится на~$L^3$ (напомним, что $L=2^{2J}$).

Рассмотрим одно из слагаемых второй суммы~(\ref{eq_riskSplitUnknSTomo}):
\begin{multline*}
\e\left( \Yljk^2 - \hslj - \e\Yljk^2 + \slj \right) \times{}\\
{}\times\left( \Yljks^2 - \hsljs - \e\Yljks^2 + \sljs \right) = {}\\
{}= \e\Yljk^2\Yljks^2 - \e\Yljk^2\hsljs - \e\Yljk^2 \e\Yljks^2 +{}\\
{}+ \sljs \e\Yljk^2 - 
 \e\hslj\Yljks^2 + \e\hslj\hsljs +{}\\
 {}+ \e\hslj\e\Yljks^2 - \sljs\e\hslj 
- \e\Yljk^2 \e\Yljks^2 +{}\\
{}+ \e\hsljs\e\Yljk^2 + \e\Yljk^2 \e\Yljks^2 - \sljs\e\Yljk^2 
+ {}\\
{}+\slj\e\Yljks^2 - \slj\e\hsljs - \slj\e\Yljks^2 +{}\\
{}+ \slj\sljs = 
 - \cov\left( \hsljs,\,\Yljk^2 \right) -{}\\
 {}- \cov\left( \hslj,\,\Yljks^2 \right) +
 \fr{\slj\sljs}{\sigma^4}\left( \nu_L^2 + \theta_L \right)\,.
\end{multline*}
С учетом того, что $\D\Yljk^2$ имеет порядок~$2^{2j}$, а ковариацию можно оценить по неравенству Коши--Бу\-ня\-ков\-ско\-го, 
получаем, что каждое слагаемое второй суммы~(\ref{eq_riskSplitUnknSTomo}) можно оценить как 
$2^{j+j'}\cdot\osml(1)$. Всего таких слагаемых порядка~$L^2$, а максимальное значение $2^{j+j'}$ 
равно $2^{J-1+J-1}=L/4$. Следовательно, после суммирования получаем, что второе сла\-га\-емое 
в~(\ref{eq_riskSplitUnknSTomo}) оценивается как~$\osml(1)$. Значит, $S_1/L^{3/2}$ сходится к нулю по вероятности.

Для оценки~$S_2$ используем другую модификацию неравенства Чебышёва:
\begin{equation*}
\p\left( \fr{|S_2|}{L^{3/2}} > \delta \right) \leq \fr{\e|S_2|}{\delta L^{3/2}} = 
\fr{\e\left[|S_2|/L^{1/2}\right]}{\delta L}\,.
\end{equation*}
Величину $\e|S_2|$ можно оценить сверху суммой математических ожиданий модулей сумм, входящих в~$S_2$, 
а эти суммы, в свою очередь,~--- суммой математических ожиданий входящих в них слагаемых.

По формуле полной вероятности для некоторого $0<\gamma<1$ получаем
\begin{multline*}
\p\left( |\Yljk| > \hTlj \right)={}\\
{}=\p\left( |\Yljk| > \hTlj \,|\, \hTlj \leqslant (1-\gamma)\solj\sqrt{2\ln 2^{2j}} \right)\times{}\\
{}\times \p\left( \hTlj \leq (1-\gamma)\solj\sqrt{2\ln 2^{2j}} \right) + {}\\
{}+ \p\left( |\Yljk| > \hTlj \,,\, \hTlj > (1-\gamma)\solj\sqrt{2\ln 2^{2j}} \right)\,.
\end{multline*}
В силу свойств~$\hsig$
\begin{multline*}
\p\left( \hTlj \leq (1-\gamma)\solj\sqrt{2\ln 2^{2j}} \right)={}\\
{}=\p\left(\hsig\leqslant (1-\gamma)^2\sigma^2\right)\leq{}\\
{}\leq\p\left(|\hsig-\sigma^2-\nu_L|\geqslant(2\gamma-\gamma^2)\sigma^2+\nu_L\right)\leq{}\\
{}\leq \fr{\D\hsig}{\left((2\gamma-\gamma^2)\sigma^2+\nu_L\right)^2}=\Obig\left(L^{-\beta}\right)
\end{multline*}
для достаточно большого~$L$. Далее
\begin{multline}
\p\left( |\Yljk| > \hTlj \,,\, \hTlj > (1-\gamma)\solj\sqrt{2\ln 2^{2j}} \right) \leq{}\\
{}\leq \p\left( |\Yljk| > (1-\gamma)\solj\sqrt{2\ln 2^{2j}} \right) = {}\\
{}=
\fr{ C }{ 2^{2j(1-\gamma)^2}\cdot 2^{j/2}\sqrt{j} }\,.
\label{eq_prbSplitGammaTomo}
\end{multline}
Теперь оцениваем математические ожидания компонентов сумм из~$S_2$ при делении на~$L^{1/2}$:
\begin{multline*}
\e\left[\fr{\left| \Yljk^2 - \hslj \right|}{L^{1/2}}\indYjkghTj\right] \leq {}\\
{}\leq\sqrt{ \e\left[\fr{ \left(\Yljk^2 - \hslj\right)^2 }{2^{2J}}\right]  \p\left( |\Yljk| > \hTlj \right) } 
\rightarrow 0\,;
\end{multline*}

\vspace*{-6pt}
\noindent
\begin{multline*}
\e\left[\fr{\left| \hslj + \hTlj^2 \right|}{L^{1/2}}\indYjkghTj\right] \leq{}\\
{}\leq 
\left( 2\ln 2^{2j} + 1 \right) 2^{j-J} \times{}\\
{}\times\sqrt{ \e\left(\hslz\right)^2 \p\left( |\Yljk| > \hTlj \right) } \rightarrow 0
\end{multline*}
при $j\geq j_M$ и $J\rightarrow\infty$.
Остальные слагаемые оцениваются аналогично. Итак, $S_2/L^{3/2}$ тоже сходится к нулю по вероятности.~$\square$

\smallskip

Как и в одномерном случае (см.~\cite{MarkinLimitDistr}), порядок знаменателя в~(\ref{eq_ConsistTomo32}) 
можно понизить, введя дополнительные ограничения на~$\nu_L$.

\medskip

\noindent
\textbf{Теорема 3.}
\textit{Пусть справедливы предположения о регулярности~$f$. Пусть $\hsig$~--- 
оценка дисперсии, $\e\hsig-\sigma^2=$\linebreak $=\nu_L=\Obig(L^{-\upsilon})$ и 
$\D\hsig=\theta_L=\Obig(L^{-\beta})$, $\upsilon$, $\beta>0$. Тогда
при любом $a>1/2-c$, $c=\min\left\{1/2, \upsilon, \beta/2\right\}$ и $L\rightarrow\infty$ выполнено
\begin{equation*}
\fr{\hat r(f)-r(f)}{L^{a+1}} \xrightarrow{\textsf{P}} 0\,.
\end{equation*}}
\medskip

\noindent
Д\,о\,к\,а\,з\,а\,т\,е\,л\,ь\,с\,т\,в\,о.
Заметим, что $0<c\leq 1/2$ и, стало быть, $a>0$. Так же, как и в доказательстве теоремы~2, 
разобьем $\hat r - r$ на те же суммы~$S_1$ и~$S_2$ (см.\ формулы~(\ref{eq_riskSplitSoTomo})
и~(\ref{eq_riskSplitStTomo})), только~$S_1$ запишем в виде
\begin{multline*}
S_1 = \sumljk\left(\Yljk^2-\e\Yljk^2\right) - \sumljk \left(\hslj-\slj\right) = {}\\
{}= \sumljk\left(\Yljk^2-\e\Yljk^2\right) - {}\\
{}-\sum\limits_\lambda \sum_j 2^{2j}\,2^j \left(\hslz-\solz^2\right)\,.
%\label{eq_riskSplitSoLimTomo}
\end{multline*}
Первая сумма при делении на~$L$ сходится по распределению к нормальному закону 
(см.\ разд.~\ref{sect_ConsitKnownSTomo}) и, следовательно, сходится по вероятности к нулю при делении 
на~$L^{a+1}$, где $a>0$. Вторая сумма пред\-став\-ля\-ет собой произведение 
$\left(\hsig-\sigma^2\right)$ и множителя, имеющего порядок $2^{3J}=L^{3/2}$. Легко видеть, что
\begin{equation*}
\fr{L^{3/2}\left(\hsig-\sigma^2\right)}{L^{a+1}} \xrightarrow{\textsf{P}} 0
\end{equation*}
при указанных в формулировке теоремы ограничениях на~$a$.

Покажем теперь, что $S_2/L^{a+1}$ сходится к нулю по вероятности. 
Обозначим $\varkappa = a-1/2+c>$\linebreak $>\;0$. В теореме~2 есть оценки для вероятности 
$\p\left( |\Yljk| > \hTlj \right)$:
\begin{multline}
\p\left( |\Yljk| > \hTlj \right) = {}\\
{}=\max\left\{ \fr{C_1}{2^{2J\beta}},\,\fr{C_2}{2^{2j(1-\gamma)^2}\cdot 2^{j/2}\sqrt{j}} \right\} 
\label{eq_ProbYghTOrdersTomo}
\end{multline}
для некоторого $0<\gamma<1$. При $J\rightarrow\infty$ имеем
\begin{multline}
\label{eq_restEstimConsistTomo1}
\fr{\e \left(\Yljk^2\right)^2 C_1/2^{2J\beta}}{L^{2a}} \leq \fr{C_3\cdot 2^{2j}
\cdot 2^{-2j\beta}}{2^{2J(1-2c+2\varkappa)}} ={}\\
{}= \fr{C_3\cdot 2^{2j}\cdot2^{2J\min\left\{1, 2\upsilon, \beta\right\}}}{2^{2J}\cdot 2^{2J\beta}\cdot 2^{4J\varkappa}}  \rightarrow 0,
\end{multline}

\columnbreak 
%\vspace*{-6pt}

\noindent
\begin{multline}
\fr{\e \left(\Yljk^2\right)^2 C_2\cdot 2^{-2j(1-\gamma)^2}\cdot 2^{-j/2}/\sqrt{j}}{L^{2a}} \leq {}\\
{}\leq
\fr{C_4\cdot 2^{2j}\cdot 2^{2J\min\left\{1, 2\upsilon, \beta\right\}}}{2^{2j(1-\gamma)^2+j/2}\cdot 2^{2J}\cdot 2^{4J\varkappa}\sqrt{j}} \rightarrow 0
\label{eq_restEstimConsistTomo2}
\end{multline}
для достаточно малого~$\gamma$. Отсюда имеем для произвольного $\delta>0$
\begin{multline*}
\p\left(\fr{\sumljk \Yljk^2 \indYjkghTj }{L^{a+1}}>\delta\right) \leq{}\\
{}\leq
\fr{\sumljk \e \left[\Yljk^2/L^a\right] \indYjkghTj  }{\delta L} \rightarrow 0
\end{multline*}
при $J\rightarrow\infty$ в силу неравенств Чебышёва и Коши--Бу\-ня\-ков\-ско\-го. Оценки для суммы 
с членами вида $\hslj \indYjkghTj$ получаются аналогично. А для сумм, в которые входят $\indYjkgTj$, 
оценки получены в теореме~1.~$\square$

Можно сформулировать и доказать теорему сходимости по распределению к нетривиальному пределу.

\medskip

\noindent
\textbf{Теорема 4.} 
\textit{Пусть справедливы предположения о регулярности~$f$. 
Пусть $\hsig$~--- оценка дисперсии, 
$\e\hsig-\sigma^2=\nu_L=\Obig(L^{-\upsilon})$ и 
$\D\hsig=\theta_L=\Obig(L^{-\beta})$, $\upsilon>0$, $\beta>1/2$. 
Пусть $\hsig$ не зависит от $\Yljk$ и $\sqrt{L}\left( \hsig - \sigma^2 \right) 
\Rightarrow \mathcal{N}\left(0,\,\Sigma^2\right)$ при $L\rightarrow\infty$, тогда
\begin{multline*}
\fr{\hat r(f)-r(f)}{ L \sqrt{ b_2 \left( \silz^4 + \siilz^4 + \siiilz^4 \right) } } \Rightarrow{}\\
{}\Rightarrow \mathcal{N}\left( 0, 1+\frac{ \left( \silz^2 + \siilz^2 + \siiilz^2 \right)^2 \Sigma^2 }{ d_2 \,\sigma^4 \left( \silz^4 + \siilz^4 + \siiilz^4 \right) } \right)\,,
\end{multline*}
где $b_2=2/(2^4-1)=2/15$, $d_2 = (2(2^3-1)^2)/(2^4-1)=$\linebreak $=98/15$.}

\medskip

\noindent
Д\,о\,к\,а\,з\,а\,т\,е\,л\,ь\,с\,т\,в\,о.
В теореме~3 было существенным наличие~$\varkappa>0$, которое давало сходимость к нулю 
в~(\ref{eq_restEstimConsistTomo1}) (в~(\ref{eq_restEstimConsistTomo2}) это несущественно). 
Сейчас же $\varkappa=0$, поэтому доказательство необходимо изменить.

Оценим $S_2$ более тонко. Имеем
\vspace*{-9pt}

\noindent
\begin{multline*}
\Yljk^2 \indYjkghTj - \e \Yljk^2 \indYjkgTj = {}\\[3pt]
{}= \Yljk^2 \indYjkghTj - \Yljk^2 \indYjkgTj +{}\\[3pt]
{}+ \Yljk^2 \indYjkgTj - \e \Yljk^2 \indYjkgTj\,.
\vspace*{-3pt}
\end{multline*}
\vspace*{-18pt}

\pagebreak

Вопрос о двух последних слагаемых решен в теореме~1. Рассмотрим два первых:
\begin{multline*}
\e \left| \Yljk^2 \indYjkghTj - \Yljk^2 \indYjkgTj \right| = {}\\
{}=\e \Yljk^2 \Ik_{\Tlj<|\Yljk|\leqslant\hTlj} +{}\\
{}+ \e \Yljk^2 \Ik_{\hTlj<|\Yljk|\leqslant\Tlj}\,.
\end{multline*}
При этом
\begin{multline*}
\e \Yljk^2 \Ik_{\Tlj<|\Yljk|\leqslant\hTlj} \leq \e \hTlj^2 \indYjkgTj \leq{}\\
{}\leq \sqrt{\fr{C\cdot j^2\cdot 2^{2j}}{2^{2j+j/2}\sqrt{j}}}\rightarrow 0,\quad J\rightarrow\infty\,,
\end{multline*}

%\vspace*{-3pt}

\noindent
и

%\vspace*{-3pt}
\noindent
\begin{multline}
\e \Yljk^2 \Ik_{\hTlj<|\Yljk|\leq\Tlj} \leq {}\\
{}\leq \Tlj^2 \e \Ik_{\hTlj<|\Yljk|\leq 
 \Tlj} \leq{}\\
{}\leq C j\cdot 2^{j}
\e \indYjkghTj\,.
\label{eq_FineRestEstimTomo}
\end{multline}
С учетом~(\ref{eq_ProbYghTOrdersTomo}) получаем, что
\begin{equation*}
\e \Yljk^2 \Ik_{\hTlj<|\Yljk|\leqslant\Tlj} \rightarrow 0
\end{equation*}
при $J\rightarrow\infty$ и $\beta>1/2$. Отметим, что, в отличие от работы~\cite{MarkinLimitDistr}, 
требование на~$\beta$ повысилось (там требовалось только $\beta>0$). Это является следствием роста дисперсии с 
ростом~$j$, которое выражается в наличии множителя~$2^j$ в~(\ref{eq_FineRestEstimTomo}). 
Аналогично получаем соотношения для~$\hTlj$:
\begin{multline*}
\e \hTlj^2 \Ik_{\Tlj<|\Yljk|\leq\hTlj} \leq \e \hTlj^2 \indYjkgTj \leq{}\\
{}\leq \sqrt{ \fr{ C j^2 \cdot 2^{2j} }{ 2^{2j+j/2}\sqrt{j} } } \rightarrow 0\,;
\end{multline*}

\vspace*{-12pt}

\noindent
\begin{multline*}
\e \hTlj^2 \Ik_{\hTlj<|\Yljk|\leq\Tlj} \leq{}\\
{}\leq \Tlj^2 \e\Ik_{\hTlj<|\Yljk|\leq\Tlj} \rightarrow 0\,.
\end{multline*}
Для $\hslj$ заметим, что $\hslj\leqslant\hTlj^2$. После применения неравенства Чебышёва получим, что 
$S_2/L$ сходится к нулю по вероятности.

В~$S_1$ оба слагаемых сходятся по распределению к нормальному закону и при этом независимы. 
Поэтому их сумма тоже сходится по распределению к нормальному закону. Осталось убедиться в 
правильности параметров. Имеем
\begin{multline*}
\sumljk \left(\hslj-\slj\right) = \sum\limits_\lambda \sum_j 2^{2j}\cdot 2^j \left(\hslz-\solz^2\right) = {}\\
{}= \left( \left\|\xi^{[1]}_{0,0,0}\right\|_2^2 + \left\|\xi^{[2]}_{0,0,0}\right\|_2^2 + 
\left\|\xi^{[3]}_{0,0,0}\right\|_2^2 \right)\times{}\\
{}\times \fr{2^{3J}-2^{3j_M}}{2^3-1} \left(\hsig-\sigma^2\right) = {}
\end{multline*}

\noindent
$$%\begin{multline*}
{}= \fr{  \silz^2 + \siilz^2 + \siiilz^2 }{ \sigma^2 }\, \fr{2^{3J}-2^{3j_M}}{7} \left(\hsig-\sigma^2\right)\,.\hfil\square
$$%\end{multline*}

%\columnbreak
\medskip

\noindent
\textbf{Замечание}. 
Если функция $f$ регулярная с параметром $\alpha\geq 1/4$, а $j_M\geq 4J/5$, то можно ослабить требования 
на~$\hsig$. Достаточно потребовать только состоятельность, асимптотическую нормальность и независимость от~$\Yljk$.

\smallskip

В теоремах~2 и~3 при оценке $\p\left( |\Yljk| > \hTlj \right)$ использовалось число $0<\gamma<1$. 
Можно заменить~$\gamma$ бесконечно малой последовательностью~$\gamma_L$, которая не испортит порядок знаменателя 
в~(\ref{eq_prbSplitGammaTomo}).

По формуле полной вероятности для любого $\delta>0$
\begin{multline}
\label{eq_sumOfIndicTomo}
\p\left( \sumljk\indYjkghTj > \delta \right) ={}\\
{}= \p\left( \hTlj \leqslant \left( 1-\gamma_L \right)\solj\sqrt{2\ln 2^{2j}} \right) \times{} \\
{}\times \p\left( \sumljk \indYjkghTj>\delta\,|\,\hTlj \leqslant{}\right.\\
\left.{}\vphantom{\sumljk\indYjkghTj}\leq \left(1-\gamma_L\right) \solj\sqrt{2\ln 2^{2j}} \right) +{} \\
{}+ \p\left(\sumljk\indYjkghTj>\delta\,,\right. \\
\left.\vphantom{\sumljk\indYjkghTj}\hTlj > \left( 1-\gamma_L \right) \solj\sqrt{2\ln 2^{2j}}\right)\,,
\end{multline}
где
$\gamma_L = 1/J$.
При таком $\gamma_L$ получаем
\begin{multline*}
\p\left( |\Yljk| > (1-\gamma_L)\solj\sqrt{2\ln 2^{2j}} \right) = {}\\
{}=\fr{C}{ 2^{2j(1-\gamma_L)^2}\cdot 2^{j/2}\sqrt{j} } \leq \fr{C_1}{ 2^{2J} \sqrt{j} }
\end{multline*}
в силу выбора $j_M$ и того, что
\begin{equation*}
2^{2j\left(1-\gamma_L\right)^2} = 2^{2j\left( 1-2/J + 1/J^2 \right)} > 2^{2j-4}\,.
\end{equation*}
По неравенству Чебышёва
\begin{multline*}
\p\left(\sumljk\indYjkghTj>\delta\,, \right.\\
\left. \vphantom{\sumljk\indYjkghTj}\hTlj > \left( 1-\gamma_L \right) \solj\sqrt{2\ln 2^{2j}}\right) \leq{}\\
{}\leq \p\left( \sumljk \Ik_{ |\Yljk| > (1-\gamma_L)\solj\sqrt{2\ln 2^{2j}} } > \delta \right) \leq{}
\end{multline*}

\noindent
\begin{multline*}
{}\leq \fr{ \sumljk \p\left( |\Yljk| > (1-\gamma_L)\solj\sqrt{2\ln 2^{2j}} \right) }{\delta} = {}\\
{}=\Obig\left( \fr{1}{\sqrt{j}} \right)\,.
\end{multline*}

Используя свойство асимптотической нормальности~$\hsig$, можно для любого $\delta'>0$ оценить
\begin{equation}
\label{eq_hatTdevProbAsympTomo}
\p\left( \hTlj \leq (1-\gamma_L)\solj\sqrt{2\ln 2^{2j}} \right)< \delta'\,,
\end{equation}
причем отметим, что~$\delta$ здесь фиксировано, а~$\delta'$ можно делать произвольно малым. Имеем
\begin{multline*}
%\label{eq_hatTdevProbAsymp}
\p\left(\hTlj\leqslant(1-\gamma_L)\solj\sqrt{2\ln 2^{2j}}\right) ={}\\
{}= \p\left(\hsig\leqslant(1-\gamma_L)^2\sigma^2\right)={}\\
{}= \p\left( \left(\hsig-\sigma^2\right) \leqslant \sigma^2\left(-2\gamma_L+\gamma_L^2\right) \right) ={}\\
{}= \p\left( \sqrt{L}\left(\hsig-\sigma^2\right) \leqslant -\fr{\sqrt{L}\sigma^2(2J-1)}{J^2} \right)\,.
\end{multline*}
Для произвольного~$\delta'>0$ найдется $J_0$ ($L_0=2^{2J_0}$) такое, что
\begin{equation*}
F_\Sigma\left( -\fr{\sqrt{L_0}\sigma^2(2J_0-1 )}{J_0^2} \right) < \fr{\delta'}{2}\,,
\end{equation*}
где $F_\Sigma$~--- функция распределения нормального закона с нулевым средним и дисперсией~$\Sigma^2$. 
При этом для любого $J\geq J_0$
\begin{multline*}
\p\left( \sqrt{L}\left(\hsig-\sigma^2\right) \leq -\fr{\sqrt{L}\sigma^2(J -1 )}{J^2} \right) \leq{} \\
{}\leq \p\left( \sqrt{L}\left(\hsig-\sigma^2\right) \leq -\fr{\sqrt{L_0}\sigma^2(2J_0 -1 )}{J_0^2}  \right)\,.
\end{multline*}
В силу асимптотической нормальности~$\hsig$ и непрерывности~$F_\Sigma$ для этого же~$\delta'$ 
найдется~$J_1$ $\left(L_1=2^{2J_1}\right)$ такое, что для любого $J\geq J_1$
\begin{equation*}
\left| \p\left( \sqrt{L_1}\left(\hsig-\sigma^2\right) \leq x \right) - F_\Sigma(x)\right| < \fr{\delta'}{2}\,,
\end{equation*}
причем $J_1$ не зависит от~$x$. Возьмем $x_0 =$\linebreak $= -\sqrt{L_0}\sigma^2(2J_0-1)/J_0^2$ и 
$J_2 = \max\{J_0,J_1\}$. Для любого $J\geq J_2$ имеем
\begin{equation*}
\p\left( \sqrt{L}\left(\hsig-\sigma^2\right) \leq x_0 \right) < \delta'\,,
\end{equation*}
а значит, справедливо~(\ref{eq_hatTdevProbAsympTomo}).

Получаем, что сумма индикаторов в~(\ref{eq_sumOfIndicTomo}) сходится к нулю по вероятности:
\begin{equation*}
%\label{eq_sumIndConsisthTTomo}
\p\left( \sumljk\indYjkghTj > \delta \right) \rightarrow 0 \mbox{ при }J\rightarrow\infty\,.
\end{equation*}
Для суммы индикаторов с неслучайным порогом аналогично получаем
\begin{equation*}%\label{eq_sumIndConsistTTomo}
\p\left( \sumljk\indYjkgTj > \delta \right) \rightarrow 0\,.
\end{equation*}
Далее воспользуемся дискретной версией неравенства Коши--Буняковского:
\begin{multline*}
\fr{ \sumljk \Yljk^2\indYjkghTj }{ L } \leq{}\\
{}\leq \sqrt{ \fr{\sumljk \Yljk^4/L}{L} \, \sumljk \indYjkghTj } \,\xrightarrow{\textsf{P}} 0\,,
\end{multline*}
так как $\e\left[ \Yljk^4/L \right]$ ограничено,
\begin{equation*}
\fr{ \sumljk \hTlj^2 \indYjkghTj }{L} \leq \fr{ \sumljk \Yljk^2 \indYjkghTj }{L} \xrightarrow{\mathsf{P}} 0
\end{equation*}
и
\begin{equation*}
\hslj\indYjkghTj \leqslant \hTlj^2\indYjkghTj\,.
\end{equation*}
Оценки для слагаемых с $\indYjkgTj$ получены в теореме~1.

\medskip

\noindent
\textbf{Замечание}. 
Всюду выше в этом разделе предполагалось, что пороговая обработка и суммирование в выражении для 
риска~(\ref{eq_riskEstimDefTomo}) ведутся с уровня~$j_M$, причем $j_M\rightarrow\infty$ при 
$J\rightarrow\infty$. Однако если ввести дополнительные ограничения на регулярность~$f$, 
то можно вести пороговую обработку и суммирование с уровня $j_0\nrightarrow\infty$. Если 
$j_M=J/(\alpha+1)$, то для коэффициентов, соответствующих $j<j_M$, неравенство~(\ref{eq_WaveletCoeffUpperBoundTomo}), 
вообще говоря, не выполнено. Оценим вклад больших коэффициентов в оценку риска:
\begin{multline*}
L^{-1} \sum\limits_{j=j_0}^{j_M-1}\sum\limits_{\lambda,\mathbf{k}} \left\{\left|\Yljk^2-\hslj\right|\indYjklhTj +{}\right.\\
\left.{}+ \left(\hslj+\hTlj^2\right)\indYjkghTj \right\} \leq{}\\
{}\leq L^{-1} \sum\limits_{j=j_0}^{j_M-1}\sum\limits_{\lambda,\mathbf{k}} \left\{ \left(\hslj+\hTlj^2\right) +
 \left(\hslj+\hTlj^2\right) \right\} \xrightarrow{\mathsf{P}}{}\\
\xrightarrow{\mathsf{P}} {} 0
\end{multline*}
в силу состоятельности~$\hsig$ и того, что

\noindent
\begin{multline*}
L^{-1}\left\{\sum\limits_{j=j_0}^{j_M-1}j2^j\cdot2^{2j}\right\} \leq 2^{-2J}
\left\{j_M\sum\limits_{j=j_0}^{j_M-1}2^{3j}\right\} \simeq{}\\
{}\simeq 2^{2J}\cdot j_M\cdot2^{3j_M}\rightarrow 0
\end{multline*}
при $J\rightarrow\infty$, если $3j_M<2J$, т.\,е.\ 
$\alpha>1/2$. Слагаемые риска оцениваются аналогично. Итак, 
при $\alpha>1/2$ суммирование в~(\ref{eq_riskEstimDefTomo}) можно начинать с произвольного~$j_0$.


{\small\frenchspacing
{%\baselineskip=10.8pt
\addcontentsline{toc}{section}{Литература}
\begin{thebibliography}{99}

\bibitem{Natterer} %1
\Au{Наттерер Ф.} 
Математические аспекты компьютерной томографии.~--- М.: Мир, 1990.

\bibitem{TikhonovArsenin}  %2
\Au{Тихонов А.\,Н., Арсенин В.\,Я.} 
Методы решения некорректных задач.~--- М.: Наука, 1979.

\bibitem{Herman}  %3
\Au{Хермен Г.} 
Восстановление изображений по проекциям: основы реконструктивной томографии.~--- М.: Наука, 1983.

\bibitem{Daub}  %4
\Au{Добеши И.} 
Десять лекций по вейвлетам.~--- Ижевск: НИЦ <<Регулярная и хаотическая динамика>>, 2001.

\bibitem{DonohoWVD} 
\Au{Donoho D.\,L.} 
Nonlinear solution of linear inverse problems by wavelet-vaguelette decomposition~// 
Appl. Comput. Harmonic Anal., 1995. Vol.~2. P.~101--126.

\bibitem{KolaczykArticle}  %7
\Au{Kolaczyk E.\,D.} 
A wavelet shrinkage approach to tomographic image reconstruction~// J. Amer. Statistical Association, 1996. 
Vol.~91. No.\,435. P.~1079--1090.

\bibitem{KolaczykThesis} %6
\textit{Kolaczyk E.\,D.} 
Wavelet methods for the inversion of certain homogeneous linear operators in the presence of noisy data.  Ph.D.\ 
Thesis, 1994.

\bibitem{DJideal} 
\textit{Donoho D.\,L., Johnstone I.\,M.} 
Ideal spatial adaptation via wavelet shrinkage~// Biometrika, 1994. Vol.~81. No.\,3. P.~425--455.

\bibitem{DJunkn}  %9
\textit{Donoho D.\,L., Johnstone I.\,M.} 
Adapting to unknown smoothness via wavelet shrinkage~// J. Amer.\ Statistical Association, 1995. Vol.~90. P.~1200--1224.

\bibitem{Mallat} %10
\Au{Mallat S.} 
A wavelet tour of signal processing.~--- Academic Press, 1999.


\bibitem{MarkinLimitDistr}  %11
\Au{Маркин А.\,В.} 
Предельное распределение оценки риска при пороговой обработке вейвлет-ко\-эф\-фи\-ци\-ен\-тов~// 
Информатика и её применения, 2009. Т.~3. Вып.~4. С.~57--63.

\label{end\stat}

\bibitem{MarkinShestakovConsist}  %12
\Au{Маркин А.\,В., Шестаков О.\,В.} 
О состоятельности оценки риска при пороговой обработке вейвлет-ко\-эф\-фи\-ци\-ен\-тов~// Вестник Московского университета. 
Сер.~15. Вычислительная математика и кибернетика, 2010. №\,1. С.~26--33.


 \end{thebibliography}
}
}

\end{multicols}