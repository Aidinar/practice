\def\stat{bening}

\def\tit{О ПРЕДЕЛЬНОМ ПОВЕДЕНИИ МОЩНОСТЕЙ КРИТЕРИЕВ В~СЛУЧАЕ РАСПРЕДЕЛЕНИЯ ЛАПЛАСА}

\def\titkol{О предельном поведении мощностей критериев в случае распределения Лапласа}

\def\autkol{В.\,Е.~Бенинг, Р.\,А.~Королев}
\def\aut{В.\,Е.~Бенинг$^1$, Р.\,А.~Королев$^2$}

\titel{\tit}{\aut}{\autkol}{\titkol}

%{\renewcommand{\thefootnote}{\fnsymbol{footnote}}\footnotetext[1]
%{Работа выполнена при поддержке Российского фонда фундаментальных
%исследований (проекты 08-01-00563, 08-01-00567, 08-07-00152 и
%09-07-12032-офи-м), а также Министерства образования и науки
%(государственные контракты П1181 и П958, грант МК-581.2010.1).}}

\renewcommand{\thefootnote}{\arabic{footnote}}
\footnotetext[1]{Московский
государственный университет им.\ М.\,В.~Ломоносова, факультет
вычислительной математики и кибернетики, bening@yandex.ru}
\footnotetext[2]{Московский государственный университет
им.\ М.\,В.~Ломоносова, факультет вычислительной математики и
кибернетики, stochastique@gmail.com}


\Abst{С использованием сходимости условных мер,
зависящих от параметра, доказана формула для предела
нормированной разности мощностей наилучшего и асимптотически
оптимального критериев в случае распределения Лапласа.
В~связи с нерегулярностью распределения Лапласа логарифм отношения
правдоподобия допускает нерегулярное стохастическое разложение; кроме
того, для знаковой статистики, лежащей в основе асимптотически оптимального
критерия, не выполняется аналог условия Крамера (C). Поэтому непосредственное
применение теоремы~3.2.1 работы~[1] или теоремы~2.1 работы~[2]  затруднительно
и в настоящей работе их доказательства пересмотрены для случая распределения
Лапласа.}

\KW{функция мощности; условная вероятностная мера; условный момент;
распределение Лапласа}

       \vskip 14pt plus 9pt minus 6pt

      \thispagestyle{headings}

      \begin{multicols}{2}

      \label{st\stat}

\section{Введение}

В 1774~г.\ Пьер-Симон Лаплас в своей статье <<Sur la
probabilit$\acute{\mbox{e}}$ des causes par les $\acute{\mbox{e}}$v$\acute{\mbox{e}}$nements>>
(см.~\cite{10ben} и литературу там же) предложил естественный вероятностный
закон для ошибки измерений в такой формулировке: логарифм частоты ошибки
есть линейная функция абсолютного значения ошибки. Естественность
этого вероятностного закона он объяснял так: <<Чем
дальше результат измерения от истинного значения, тем менее
вероятным он должен быть, при этом такое уменьшение вероятности не
может быть постоянным. Поскольку нет причин считать, что с ростом
ошибки сами вероятности и последовательные разности между
вероятностями убывают по-разному, то следует приравнять отношения
двух бесконечно близких разностей вероятностей и двух бесконечно
близких вероятностей. Интегрирование этого равенства показывает, что
вероятность ошибки выражается как экспоненциальная функция самой
ошибки независимо от ее знака>>. 

Назвав этот закон
первым законом для ошибки измерений, который исторически является
первым вероятностным распределением с неограниченным носителем,
Лаплас уже через 4~года в своей фундаментальной работе
``Th$\acute{\mbox{e}}$orie Analytique'' (см.~\cite{10ben} и
литературу там же) рассматривает второй вероятностный закон, который
гласит: логарифм частоты ошибки измерений есть квадратичная функция
ошибки. Именно этот второй закон из-за хороших аналитических свойств
будет детально исследоваться все последующее время, получит название
<<нормальное распределение>> и займет
главное мес\-то в теории вероятностей благодаря центральной предельной
теореме. Лишь спус\-тя почти 150~лет известный экономист и математик
Дж.~Кейнс (см.~\cite{10ben} и литературу там же) напомнит о существовании
первого закона для ошибки измерений и получит его вновь из
предположения, что наиболее вероятное значение измеряемой величины
есть ее медиана. Следом за ним известный математик Э.~Уилсон (см.~\cite{10ben}
и литературу там же) с помощью непараметрических методов покажет
на одном примере, что распределение отклонений от медианы измерений
является скорее первым законом Лапласа, нежели нормальным законом.
Спустя еще почти 50~лет в научной литературе (см.~\cite{10ben} и литературу
там же) все чаще стали появляться призывы к использованию первого
закона Лапласа в качестве основного распределения для экономических,
биометрических и демографических данных в противовес нормальному
распределению. 

В наши дни первый закон Лапласа называют
распределением Лапласа или двойным экспоненциальным распределением,
указывая на возможность получения его как разности двух независимых
одинаково распределенных экспоненциальных величин, которые часто
используются для описания продолжительности жизни наблюдаемых
объектов. Связь с $\chi^2$-распределением также способствует
распространению распределения Лапласа. Оно приковывает внимание
многих математиков и находит все более широкое применение в
прикладных областях экономики и науки. Нерегулярность распределения
Лапласа долгое время вызывала трудности при получении результатов в
задачах проверки статистических гипотез. Однако развитые в последние
десятилетия асимптотические методы теории статистических гипотез
(см.~[1, 2, 4, 5] и литературу там же) позволяют теперь решать многие
из таких задач. Одна из них~--- задача о потере мощности
асимптотически оптимального критерия~--- рас\-смат\-ри\-ва\-ет\-ся в данной
работе.

В работе исследуется формула для предела нормированной разности
мощностей критериев в случае распределения Лапласа (см.\
теорему~3.2.1\linebreak работы~\cite{1ben} и теорему~2.1 работы~\cite{2ben}). Следуя
рабо-\linebreak там~[6--8],
рассмотрим задачу проверки простой гипотезы
\begin{equation}
{\sf H}_0:\theta=0  
\label{e1.1}
\end{equation}
против последовательности сложных близких альтернатив вида
\begin{equation*}
{\sf H}_{n,1}: \theta = \fr{t}{\sqrt{n}}\,,\quad  0<t\le C\,,\quad
C>0\,,
%\label{e1.2}
\end{equation*}
на основе выборки ${\bf X}_n = (X_1,\ldots,X_n)$, являющейся элементом
выборочного пространства $({\cal{X}}_n,{\cal{A}}_n)$ и
состоящей из независимых и одинаково распределенных наблюдений с плотностью
$$
p(x,\theta)=\fr{1}{2}\,e^{-\left| x-\theta\right| }\,,\quad x,\,\theta\in {\sf R}^1\,.
$$
Обозначим через~${\p}_{n,0}$ и~${\p}_{n,1}$ распределения~${\cal
L}({\bf X}_n)$ при гипотезе~${\sf H}_0$ и альтернативе~${\sf
H}_{n,1}$, соответственно плотности распределений выборки будем
обозначать~$p_{n,0}(x)$ и~$p_{n,1}(x)$.

В работе~\cite{5ben} на основании сходимости условных моментов и с помощью
асимптотических разложений была доказана формула для предела
нормированной разности мощности асимптотически оптимального
критерия, основанного на статистике
\begin{equation}
S_n=\fr{t}{\sqrt{n}}\sum\limits_{i=1}^{n}\mathrm{sign}\,\left(X_i\right)-\fr{t^2}{2}\,,
\label{e1.3}
\end{equation}
и мощности наилучшего критерия, основанного на логарифме отношения правдоподобия
\begin{equation}
\Lambda_n=\sum\limits_{i=1}^{n}\left( |X_i| - |X_i - tn^{-1/2}| \right)\,.
\label{e1.4}
\end{equation}
При этом асимптотические разложения для функций распределения знаковой статистики~$S_n$ 
при гипотезе~${\sf H}_0$ и альтернативе~${\sf H}_{n,1}$ были получены и для точек разрыва, 
что соответствовало статистической постановке задачи.

Естественным выглядит отказ от применения асимптотических разложений с целью 
формирования достаточных условий, подобных условиям основных теорем работ~\cite{1ben, 2ben}, 
при доказательстве формулы для предела нормированной разности мощностей критериев (см.~(\ref{e2.6}) и теорему~2.8).

В настоящей работе в доказательстве формулы~(\ref{e2.6}) используются главным образом асимптотические 
свойства статистик~$\Lambda_n$ (см.~(\ref{e1.4})) и раз\-ности
$\Delta_n=S_n-\Lambda_n$
(см.\ леммы~2.1--2.3) и методы сходи\-мости условных моментов и условных мер, за\-висящих от параметра 
(см.~\cite{1ben, 2ben}). При этом асимптотические свойства логарифма отношения правдоподобия~$\Lambda_n$ 
описываются с помощью теорем об асимптотических разложениях функций распределения при гипотезе и альтернативе 
из работы~\cite{6ben}.

Заметим, что теорема~3.2.1 работы~\cite{1ben} не может\linebreak непосредственно быть применена к случаю 
распределения Лапласа, поскольку достаточное условие~3~(ii) (см.~\cite{1ben}, с.~79)~--- 
аналог условия\linebreak Крамера~(С)~--- не выполняется для решетчатой статистики~$S_n$. Также в связи с 
нерегулярностью распре\-деления Лапласа статистика~$\Lambda_n$ допускает нерегулярное стохастическое 
разложение, поэтому скорость сходимости мощностей критериев есть~$n^{-1/2}$ в отличие от случая~$n^{-1}$ 
в формулировке теоремы~2.1 работы~\cite{2ben}. Таким образом, в настоящей работе доказательства теоремы~3.2.1 
работы~\cite{1ben} и теоремы~2.1 работы~\cite{2ben} пересматриваются для случая распределения Лапласа.

\section{Формула для разности мощностей критериев}

В этом разделе докажем формулу для предела нормированной разности мощностей критериев в случае распределения Лапласа.

Рассмотрим последовательность критериев
\begin{equation}
\Psi_n^*(\Lambda_n)=
\begin{cases}
0\,, & \Lambda_n<c_n\,;\\
\gamma_n^*\,, & \Lambda_n=c_n\,;\\
1\,, & \Lambda_n>c_n\,,
\end{cases}
\label{e2.1}
\end{equation}
основанную на последовательности~$\Lambda_n$ (см.~(\ref{e1.4})), и последовательность критериев
\begin{equation*}
\Psi_n(S_n)=
\begin{cases}
0\,, & S_n<\bar d_n\,;\\
\gamma_n\,, & S_n=\bar d_n\,;\\
1\,, & S_n>\bar d_n\,,
\end{cases}
%\label{e2.2}
\end{equation*}
основанную на последовательности~$S_n$ (см.~(\ref{e1.3})), уровня~$\alpha,$ $\alpha\in(0,1)$, так что
\begin{equation}
{\e}_{n,0}\Psi_n^*(\Lambda_n)={\e}_{n,0}\Psi_n(S_n)=\alpha+ o\left(\tau_n^2\right)\,,
\label{e2.3}
\end{equation}
где $\tau_n\equiv n^{-1/4}$, а ${\e}_{n,0}$ и ${\e}_{n,1}$~--- математические ожидания по отношению 
к~${\p}_{n,0}$ и~${\p}_{n,1}$ соответственно.

Обозначим через
%\begin{align}
$\beta_n^*={\e}_{n,1}\Psi_n^*(\Lambda_n)$ и
$\beta_n=$\linebreak $=\;{\e}_{n,1}\Psi_n(S_n)$
мощности соответствующих критериев.

Докажем, что в случае распределения Лапласа справедлива формула для предела разности мощностей критериев~$r$ 
в терминах условной дисперсии
\begin{equation}
r\equiv\lim\limits_{n\to\infty}\tau_n^{-2}\left(\beta_n^*-\beta_n\right)=\fr{1}{2}\,e^b{\D}\left[
\Delta \vert \Lambda = b\right] p(b)\,,\!
\label{e2.6}
\end{equation}
где $\Delta$ и~$\Lambda$~--- случайные величины, такие что
\begin{equation*}
{\cal L}((\tau_n^{-1}\Delta_n,\Lambda_n)\vert {\sf H}_0) \to 
{\cal L}(\Delta, \Lambda)\,;
%\label{e2.7}
\end{equation*}
\begin{equation}
\Delta_n=
\begin{cases}
0\,, & S_n=\Lambda_n=+\infty\,,\\
S_n-\Lambda_n\,, &\mbox{в остальных случаях}\,;
\end{cases}
\label{e2.8}
\end{equation}
\begin{equation}
b=\Phi_1^{-1}(1-\alpha)\,.
\label{e2.9}
\end{equation}
Здесь~$\Phi_1(x)$ обозначает предельную функцию распределения случайных величин~$\Lambda_n$ 
при гипотезе~${\sf H}_0$ (см.~(\ref{e1.1})), $p(x)$~--- ее функция плотности.

Учитывая, что в случае распределения Лапласа плотности~$p_{n,0}(x),$ $p_{n,1}(x)$ не обращаются в нуль, т.\,е.\
\begin{equation}
p_{n,0}(x)>0\,,\enskip p_{n,1}(x)>0\,,\enskip \forall x\in{\cal{X}}_n\,,
\label{e2.10}
\end{equation}
получаем из~(\ref{e1.3}) и~(\ref{e1.4}) для любого~$n$
$$
\left|S_n\right|<\infty\,,\quad  \left|\Lambda_n\right| <\infty\,.
$$
Тогда~(\ref{e2.8}) запишется с точностью до множеств меры ноль (см.~(3.4) работы~\cite{3ben})
\begin{multline*}
\Delta_n=S_n-\Lambda_n=-\fr{t^2}{2}-{}\\
{}-2\sum_{i=1}^{n}(X_i-tn^{-1/2}){\Ik}_{[0,tn^{-1/2}]}(X_i)\,.
%\label{e2.11}
\end{multline*}
Леммы~2.1 и~3.1 работы~\cite{3ben} утверждают, что
\begin{align}
{\cal L}(\Lambda_n \vert {\sf H}_0) &\to
{\cal N}\left(-\fr{t^2}{2}, t^2 \right)\,;
\label{e2.12}\\
{\cal L}(\tau_n^{-1} \Delta_n\vert{\sf H}_0) & \to 
{\cal N}\left(0, \fr{2t^3}{3} \right)\,;\notag
\\
\!\!\!{\cal L}\left(\left(\tau_n^{-1} \Delta_n,\Lambda_n\right)\vert {\sf H}_0\right) & \to 
{\cal N}_2 \left(0, \fr{2t^3}{3},0, -\fr{t^2}{2}, t^2\right),\!\!
\label{e2.14}
\end{align}
где ${\cal N}_2$~--- двумерный нормальный закон с соответствующими
параметрами.

Таким образом, необходимо показать, что формулы~(\ref{e2.6}) и~(\ref{e2.9}) справедливы с
$$
b=tu_{\alpha}-\fr{t^2}{2}\,,\quad  r=\fr{t^2}{3}\varphi(u_{\alpha}-t)\,,
$$
где $\Phi(x)$~--- функция распределения стандартного нормального закона, $\varphi(x)$~--- его плотность, 
$\Phi(u_{\alpha})=1-\alpha$.

Приведем некоторые свойства статистик~$\Lambda_n$ и~$\Delta_n$ в случае распределения 
Лапласа (леммы~2.1--2.3). Из леммы~3.2 работы~\cite{4ben} следует

\smallskip

\noindent
\textbf{Лемма~2.1.} {\it Для $\tau_n=n^{-1/4}$ и непрерывно дифференцируемой функции~$\Phi_1(x)$, не зависящей от~$n$ 
и имеющей ограниченную производную $p(x)=\Phi_1'(x)>0$, выполняется
\begin{align*}
(i) \ \ \ &
\sup\limits_{x}\bigl|{\p}_{n,0}(\Lambda_n < x) - \Phi_1(x) \bigr|={\cal O}(\tau_n^2)\,;\\
(ii) \ \ \ &
\sup\limits_{x\leq x_0}{\p}_{n,1}\left(x\leq \Lambda_n \leq x + \tau_n^{2+\beta} \right)=o\left(\tau_n^2\right)
\end{align*}
для некоторого~$\beta>0$ и любого $x_0\in{\sf R}^1$.
}

%\medskip

В работе~[6] (см.\ лемму~3.4) была доказана.

\smallskip

\noindent
\textbf{Лемма~2.2.} {\it Для любого $x>0$ справедливо следующее неравенство:
$$
{\p}_{n,0}\left( \tau_n^{-1} \left|\Delta_n\right| \geq x \right)\leq Ce^{-x}\,,\enskip C>0\,.
$$
 }

Отсюда, а также из~(\ref{e2.12}) (см.\ лемму~2.2 работы~\cite{5ben}) следует

\smallskip

\noindent
\textbf{Лемма~2.3.} {\it Для $\tau_n=n^{-1/4}$ и $\eta_n=n^{-1/8}$ выполняются соотношения:
%\begin{enumerate}[($i$)]
%\item
\begin{align*}
(i)\ \ \ \ & \eta_n^{-1}{\e}_{n,0}\Delta_n^2 {\Ik}_{\left(\eta_n,\infty\right)}\left(|\Delta_n|\right)= o\left(\tau_n^2\right)\,;
\\
(ii)\ \ \ \  &{\e}_{n,0}e^{\Lambda_n}{\Ik}_{\left(\eta_n,\infty\right)}\left(|\Delta_n|\right)= o\left(\tau_n^2\right)\,,
\end{align*}
где ${\Ik}_{A}(\cdot)$~--- индикаторная функция множества~$A$. 
}

\smallskip

Рассмотрим теперь свойства мощностей критериев~$\Psi_n$ и~$\Psi_n^*$ в случае распределения Лапласа 
(леммы~2.4,~2.6). Из леммы~2.1 следует


\smallskip

\noindent
\textbf{Лемма~2.4.} {\it Для мощности~$\beta_n^*$ справедливо равенство
$$
\beta_n^*={\p}_{n,1}\left(\Lambda_n > c_n \right)+ o\left(\tau_n^2\right)\,.
$$
}

%\medskip


\noindent
Д\,о\,к\,а\,з\,а\,т\,е\,л\,ь\,с\,т\,в\,о\,.\
Доказательство следует из равенства
\begin{multline*}
\beta_n^*={\e}_{n,1}\Psi_n^*(\Lambda_n)={\p}_{n,1}\left(\Lambda_n > c_n \right)+{}\\
{}+
\gamma_n^*{\p}_{n,1}\left(\Lambda_n = c_n \right)\,,
\end{multline*}
неравенства $0\leq\gamma_n^*\leq1$ и соотношения (см.\ лемму~2.1~($ii$))
\begin{multline*}
{\p}_{n,1}\left(\Lambda_n = c_n \right)\leq
{\p}_{n,1}\left(c_n\leq \Lambda_n \leq c_n + \tau_n^{2+\beta} \right)={}\\
{}= o\left(\tau_n^2\right)\,.
\end{multline*}
Ограниченность~$c_n$ следует из леммы~2.1~($i$).\hfill$\Box$

\smallskip

Введем в рассмотрение сглаженные случайные величины
\begin{equation}
\tilde\Lambda_n=\Lambda_n+\xi_n\,, \quad \xi_n\sim{\cal N}(0,\sigma_n^2)\,,
\label{e2.15}
\end{equation}
где
\begin{equation}
\sigma_n^2={\cal O}\left(\tau_n^{4+4\beta}\right)\,,
\label{e2.16}
\end{equation}
$\beta>0$~--- постоянная, такая же как в лемме~2.1, причем случайная величина~$\xi_n$ не зависит 
от~${\bf X}_n$ при~${\sf H}_0$, и ${\cal N}(\mu,\sigma^2)$~--- нормальный закон с 
параметрами~$(\mu,\sigma^2)$.

Определим
\begin{align*}
\tilde\beta_n^*&={\e}_{n,0}e^{\tilde\Lambda_n}{\Ik}_{(c_n,\infty)}\left(\tilde\Lambda_n\right)\,;
%\label{e2.17}
\\
\tilde\beta_n &= {\e}_{n,0}e^{\tilde\Lambda_n}\Psi_n(S_n)\,. %\label{e2.18}
\end{align*}

Следующее утверждение хорошо известно, но ради полноты изложения приведем его доказательство.

\smallskip

\noindent
\textbf{Лемма~2.5.} {\it Пусть $X\sim{\cal N}(0,\sigma^2)$, тогда для любого $a\in{\sf R}^1$
$$
{\e}e^{aX}=e^{a^2\sigma^2/2}\,.
$$}


\smallskip

\noindent
Д\,о\,к\,а\,з\,а\,т\,е\,л\,ь\,с\,т\,в\,о\,.\
\begin{multline*}
{\e}e^{aX}=\fr{1}{\sqrt{2\pi}\sigma}\int\limits_{-\infty}^{\infty}\exp\left\{ax-\fr{x^2}{2\sigma^2}\right\}\,dx={}\\
{}=\fr{1}{\sqrt{2\pi}\sigma}\int\limits_{-\infty}^{\infty}
\exp \left\{-\fr{1}{2\sigma^2}\left(x^2-2xa\sigma^2+a^2\sigma^4-\right.\right.{}\\
\left.\left.{}-a^2\sigma^4\right)
\vphantom{-\fr{1}{2\sigma^2}} \right\}\,dx={}\\
{}
=e^{{a^2\sigma^2}/{2}}\fr{1}{\sqrt{2\pi}\sigma}\int\limits_{-\infty}^{\infty}
e^{-{(x-a\sigma^2)^2}/(2\sigma^2)}\,dx=
e^{{a^2\sigma^2}/2}\,.~\Box\hspace*{-5.86pt}
\end{multline*}


%\medskip

Далее справедлива следующая

\smallskip

\noindent
\textbf{Лемма~2.6.} {\it Для мощностей~$\beta_n$ и~$\beta_n^*$ справедливы равенства
\begin{align*}
\beta_n &=\tilde\beta_n+ o \left(\tau_n^2\right)\,;
%\label{e2.19}
\\
\beta_n^* &=\tilde\beta_n^*+ o\left(\tau_n^2\right)\,.  %\label{e2.20}
\end{align*}}

\smallskip

\noindent
Д\,о\,к\,а\,з\,а\,т\,е\,л\,ь\,с\,т\,в\,о\,.
Учитывая~(\ref{e2.10}), имеем
$$
{\e}_{n,1}\Psi_n(S_n)={\e}_{n,0}e^{\Lambda_n}\Psi_n(S_n)\,.
$$
Тогда
\begin{equation*}
\left|\beta_n-\tilde\beta_n\right|=\left|{\e}_{n,1}\Psi_n(S_n)-{\e}_{n,0}e^{\tilde\Lambda_n}\Psi_n(S_n)\right|={}
\end{equation*}
%\columnbreak

\noindent
\begin{multline*}
{}
=\left|{\e}_{n,0}e^{\Lambda_n}\Psi_n(S_n)-{\e}_{n,0}e^{\tilde\Lambda_n}\Psi_n(S_n)\right|\leq{}\\
{}\leq
{\e}_{n,0}e^{\Lambda_n}\left|e^{\xi_n}-1\right|\,.
\end{multline*}
В силу равенства
$$
{\e}_{n,0}e^{\Lambda_n}=1
$$
и независимости~${\bf X}_n$ и~$\xi_n$ имеем
\begin{multline*}
\left|\beta_n-\tilde\beta_n\right|\leq{\e}\left|e^{\xi_n}-1\right|\leq\left({\e}(e^{\xi_n}-1)^2\right)^{1/2}={}\\
{}=
\left({\e}e^{2\xi_n}-2{\e}e^{\xi_n}+1\right)^{1/2}\,.
\end{multline*}
Используя теперь лемму~2.5, получим неравенство
\begin{multline*}
\left|\beta_n-\tilde\beta_n\right|\leq\left(
e^{2\sigma_n^2}-2e^{\sigma_n^2/2}+1\right)^{1/2}\leq{}\\
{}\leq
\left(1+2\sigma_n^2-2-\sigma_n^2+1+ o(\sigma_n^2) \right)^{1/2}={}\\
{}={\cal O}(\sigma_n)=o\left(\tau_n^2\right)\,.
\end{multline*}
Аналогично, используя лемму~2.4, имеем
\begin{multline*}
\left|\beta_n^*-\tilde\beta_n^*\right|={}\\
{}=\left|{\e}_{n,0}e^{\Lambda_n}{\Ik}_{\left(c_n,\infty\right)}\left(\Lambda_n\right)-{\e}_{n,0}
e^{\tilde\Lambda_n}{\Ik}_{\left(c_n,\infty\right)}\left(\tilde\Lambda_n\right)\right|+{}\\
{}+ o\left(\tau_n^2\right)\leq{\e}\left|e^{\xi_n}-1\right|+{}\\
\!\!+\left|{\e}_{n,0}e^{\tilde\Lambda_n}\!\left(\!{\Ik}_{\left(c_n,\infty\right)}
\left(\tilde\Lambda_n\right)\!-\!{\Ik}_{\left(c_n,\infty\right)}
\left(\Lambda_n\right)\right)\right|\!+o\left(\tau_n^2\right)\leq\\
\!\!\leq\left|{\e}_{n,0}e^{\tilde\Lambda_n}
\!\left(\!{\Ik}_{\left(c_n,\infty\right)}\left(\tilde\Lambda_n\right)\!-\!{\Ik}_{\left(c_n,\infty\right)}
\left(\Lambda_n\right)\right)\right|+o\left(\tau_n^2\right).
\end{multline*}
Докажем, что первое слагаемое в правой части последнего неравенства есть величина 
порядка~$o\left(\tau_n^2\right)$. Заметим, что
\begin{multline*}
{\p}(\left|\xi_n\right|>\tau_n^{2+\beta})={\p}\left(|\xi|>\fr{\tau_n^{2+\beta}}{\sigma_n}\right)={}\\
{}={\p}(|\xi_n|>\tau_n^{-\beta})= 2\left(1-\Phi\left(\tau_n^{-\beta}\right)\right)\,,
\end{multline*}
где $\xi\sim{\cal N}(0,1)$.

Используя теперь лемму~7.1.2 из~\cite{7ben}, получаем для любого $\lambda>0$
\begin{equation}
{\p}\left(\left|\xi_n\right|>\tau_n^{2+\beta}\right)\leq\fr{2}{\sqrt{2\pi}}
\tau_n^{\beta}e^{-\tau_n^{-2\beta}/2}= o\left(\tau_n^{\lambda}\right)\,.
\label{e2.21}
\end{equation}
Запишем
$\left|{\e}_{n,0}e^{\tilde\Lambda_n}\!\left(\!
{\Ik}_{\left(c_n,\infty\right)}\left(\tilde\Lambda_n\right)-
{\Ik}_{\left(c_n,\infty\right)}\left(\Lambda_n\right)\right)\right|
\equiv$\linebreak $\equiv I_1+I_2$,
где с учетом леммы~2.5 для $\lambda=4$ имеем
\pagebreak

\end{multicols}

\noindent
\begin{multline*}
I_1=
\left|{\e}_{n,0}e^{\tilde\Lambda_n}\left({\Ik}_{\left(c_n,\infty\right)}
\left(\tilde\Lambda_n\right)-{\Ik}_{\left(c_n,\infty\right)}(\Lambda_n)\right)
{\Ik}_{(\tau_n^{2+\beta},\infty)}(|\xi_n|)\right|\leq
 {\e}_{n,0}e^{\Lambda_n} {\e}e^{\xi_n}{\Ik}_{\left(\tau_n^{2+\beta},\infty\right)}\left(\left|\xi_n\right|\right)\leq{}\\
{}\leq
\left({\e}e^{2\xi_n}\right)^{1/2}\left({\p}\left(\left|\xi_n\right|>\tau_n^{2+\beta}\right) \right)^{1/2}= 
o\left(\tau_n^2\right)\,.
\end{multline*}
Далее
\begin{multline*}
I_2=
\left| {\e}_{n,0}e^{\tilde\Lambda_n}\left({\Ik}_{\left(c_n,\infty\right)}\left(\tilde\Lambda_n\right)-
{\Ik}_{\left(c_n,\infty\right)}\left(\Lambda_n\right)\right)
{\Ik}_{\left[0,\tau_n^{2+\beta}\right]}\left(\left|\xi_n\right|\right) \right|={}\\
{}
=\left| \fr{1}{\sqrt{2\pi}\sigma_n}\int\limits_{-\tau_n^{2+\beta}}^{\tau_n^{2+\beta}}
e^{y-{y^2}/(2\sigma_n^2)}
\left( {\e}_{n,0}e^{\Lambda_n}\left({\Ik}_{\left(c_n,\infty\right)}\left(\Lambda_n+y\right) - {\Ik}_{\left(c_n,\infty\right)}\left(\Lambda_n\right)\right) \right)\,dy \right|={}\\
{}
=\fr{1}{\sqrt{2\pi}\sigma_n}\left| \int\limits_{-\tau_n^{2+\beta}}^{\tau_n^{2+\beta}}
e^{y-{y^2}/(2\sigma_n^2)}
\left( {\p}_{n,1}\left(\Lambda_n>c_n-y\right) - {\p}_{n,1}\left(\Lambda_n>c_n\right) \right)\,dy \right|\leq{}\\
{}
\leq\fr{1}{\sqrt{2\pi}\sigma_n}\!\int\limits_{0}^{\tau_n^{2+\beta}}\!
e^{y-{y^2}/(2\sigma_n^2)}{\p}_{n,1}\left(c_n-y<\Lambda_n\leq c_n\right)\,dy
+\fr{1}{\sqrt{2\pi}\sigma_n}\!\int\limits_{-\tau_n^{2+\beta}}^{0}\!
e^{y-{y^2}/(2\sigma_n^2)}{\p}_{n,1}\left(c_n<\Lambda_n\leq c_n-y\right)\,dy={}
\\
{}=e^{{\sigma_n^2}/2}\fr{1}{\sqrt{2\pi}\sigma_n}\int\limits_{0}^{\tau_n^{2+\beta}}
e^{-{(y-\sigma_n^2)^2}/(2\sigma_n^2)}{\p}_{n,1}\left(c_n-y\leq\Lambda_n\leq c_n\right)\,dy+{}\\
{}
+e^{\sigma_n^2/2}\fr{1}{\sqrt{2\pi}\sigma_n}\int\limits_{-\tau_n^{2+\beta}}^{0}
e^{-{(y-\sigma_n^2)^2}/(2\sigma_n^2)}{\p}_{n,1}\left(c_n\leq\Lambda_n\leq c_n-y\right)\,dy\,.
\end{multline*}
Поскольку величина~$c_n$ ограничена (см.\ лемму~2.1~($i$)), то с учетом леммы~2.1~($ii$) имеем
$$
{\p}_{n,1}\left(c_n-\tau_n^{2+\beta}\leq\Lambda_n\leq c_n\right)= o\left(\tau_n^2\right)\,.
$$
Отсюда следует утверждение леммы.\hfill$\Box$

%\medskip

Докажем еще одну вспомогательную лемму.


\medskip

\noindent
\textbf{Лемма~2.7.} {\it В случае распределения Лапласа справедливо равенство
$$
{\e}_{n,0}\Psi_n^*(\tilde\Lambda_n)={\e}_{n,0}{\Ik}_{\left(c_n,\infty\right)}
\left(\tilde\Lambda_n\right)=\alpha+ o\left(\tau_n^2\right)\,.
$$ }


%\medskip

\noindent
Д\,о\,к\,а\,з\,а\,т\,е\,л\,ь\,с\,т\,в\,о\,.
В силу равенства~(\ref{e2.3}) имеем
\begin{equation*}
{\e}_{n,0}\Psi_n^*\left(\tilde\Lambda_n\right)={\e}_{n,0}\Psi_n^*\left(\Lambda_n\right)+
{\e}_{n,0}\left(\Psi_n^*\left(\tilde\Lambda_n\right)-\Psi_n^*\left(\Lambda_n\right)\right)
=\alpha+{\e}_{n,0}\left(\Psi_n^*\left(\tilde\Lambda_n\right)-\Psi_n^*\left(\Lambda_n\right)\right)+ o\left(\tau_n^2\right)\,.
\end{equation*}
Покажем, что второе слагаемое в правой части этого равенства есть $o\left(\tau_n^2\right)$. Имеем
$$
\left|{\e}_{n,0}\left(\Psi_n^*\left(\tilde\Lambda_n\right)-\Psi_n^*\left(\Lambda_n\right)\right) \right|\equiv J_1+J_2\,,
$$
где c учетом~(\ref{e2.21}) для $\lambda=2$

\noindent
\begin{align*}
J_1&=\left|{\e}_{n,0}\left(\Psi_n^*\left(\tilde\Lambda_n\right)-\Psi_n^*\left(\Lambda_n\right)\right)
{\Ik}_{\left(\tau_n^{2+\beta},\infty\right)}\left(\left|\xi_n\right|\right) \right|
\leq{\p}\left(\left|\xi_n\right|>\tau_n^{2+\beta}\right)= o\left(\tau_n^2\right)\,;\\
J_2&\leq \fr{1}{\sqrt{2\pi}\sigma_n}\int\limits_{-\tau_n^{2+\beta}}^{\tau_n^{2+\beta}}
e^{-y^2/(2\sigma_n^2)}
\left| {\e}_{n,0}\left(\Psi_n^*\left(\Lambda_n+y\right) - \Psi_n^*\left(\Lambda_n\right)\right) \right|\,dy\,.
\end{align*}
Из определения~(\ref{e2.1}) критической функции~$\Psi_n^*$ следует, что
\pagebreak

\noindent
\begin{multline*}
J_2\leq{\p}_{n,0}\left(\Lambda_n=c_n\right)+\fr{1}{\sqrt{2\pi}\sigma_n}\int\limits_{-\tau_n^{2+\beta}}^{\tau_n^{2+\beta}}
e^{-y^2/(2\sigma_n^2)}{\p}_{n,0}(\Lambda_n=c_n-y)\,dy+{}\\
{}+
\fr{1}{\sqrt{2\pi}\sigma_n}\int\limits_{-\tau_n^{2+\beta}}^{\tau_n^{2+\beta}}
e^{-{y^2}/(2\sigma_n^2)}\left | {\p}_{n,0}\left(\Lambda_n>c_n-y\right) - {\p}_{n,0}\left(\Lambda_n>c_n\right) \right |\,dy\,.
\end{multline*}
Поскольку $0<c\leq|c_n|\leq C$, то с учетом леммы~2.1 имеем
\begin{equation*}
{\p}_{n,0}\left(\Lambda_n=c_n\right)={\e}_{n,1}e^{-\Lambda_n}{\Ik}_{\{c_n\}}
\left(\Lambda_n\right)=e^{-c_n}{\p}_{n,1}\left(\Lambda_n=c_n\right)
\leq e^C\sup_{x\leq C}{\p}_{n,1}\left(x\leq\Lambda_n\leq x+\tau_n^{2+\beta}\right)= o\left(\tau_n^2\right)\,.
\end{equation*}
Аналогично ${\p}_{n,0}\left(\Lambda_n=c_n-y\right)= o\left(\tau_n^2\right)$ при всех $|y|\leq\tau_n^{2+\beta}$.
Таким образом, имеем оценку
\begin{multline*}
J_2\leq\fr{1}{\sqrt{2\pi}\sigma_n}\int\limits_{0}^{\tau_n^{2+\beta}}
e^{-{y^2}/(2\sigma_n^2)}{\p}_{n,0}\left(c_n-y\leq\Lambda_n\leq c_n\right)\,dy+{}\\
{}
+\fr{1}{\sqrt{2\pi}\sigma_n}\int\limits_{-\tau_n^{2+\beta}}^{0}
e^{-{y^2}/(2\sigma_n^2)}{\p}_{n,0}\left(c_n\leq\Lambda_n\leq c_n-y\right)\,dy+ o\left(\tau_n^2\right)\,.
\end{multline*}

\vspace*{-6pt}

\noindent
Но в силу леммы~2.1
\begin{multline*}
{\p}_{n,0}\left(c_n-y\leq\Lambda_n\leq c_n\right)={\e}_{n,1}e^{-\Lambda_n}{\Ik}_{\left[c_n-y,c_n\right]}
\left(\Lambda_n\right)\leq
\\
{}
\leq e^{y-c_n}{\p}_{n,1}\left(c_n-y\leq\Lambda_n\leq c_n\right)\leq e^{\tau_n^{2+\beta}+C} o\left(\tau_n^2\right)=
 o\left(\tau_n^2\right)\,.
\end{multline*}
Аналогично получаем
${\p}_{n,0}\left(c_n\leq\Lambda_n\leq c_n-y\right)= o\left(\tau_n^2\right)$.
Отсюда следует утверждение леммы.\hfill$\Box$
%\pagebreak

%\smallskip

Из лемм~2.6 и~2.7 и ограниченности~$\bar d_n$ (см.\ лемму~3.5) разность $\beta_n^*-\beta_n$ 
может быть представлена в виде:
\begin{multline*}
\beta_n^*-\beta_n=\tilde\beta_n^*-\tilde\beta_n+ o\left(\tau_n^2\right)
=
{\e}_{n,0}e^{\tilde\Lambda_n}\left({\Ik}_{\left(c_n,\infty\right)}\left(\tilde\Lambda_n\right) - \Psi_n(S_n)\right)+ 
o\left(\tau_n^2\right)={}
\\
{}={\e}_{n,0}\left(e^{\tilde\Lambda_n}-e^{\bar d_n}\right)
\left({\Ik}_{\left(c_n,\infty\right)}\left(\tilde\Lambda_n\right) - \Psi_n(S_n)\right)+ o\left(\tau_n^2\right)
\equiv \tilde A_n+\tilde B_n+ o\left(\tau_n^2\right)\,,
\end{multline*}
где
\begin{align}
\tilde A_n &=
{\e}_{n,0}\left(e^{\tilde\Lambda_n}-e^{\bar d_n}\right)
\left({\Ik}_{\left(-\infty,\bar d_n\right)}\left(\tilde\Lambda_n\right) - {\Ik}_{(-\infty,c_n)}\left(\tilde\Lambda_n\right)\right)\,;\label{e2.22}\\
\tilde B_n &=
{\e}_{n,0}\left(e^{\tilde\Lambda_n}-e^{\bar d_n}\right)\left(1 - \Psi_n(S_n) - {\Ik}_{\left(-\infty,\bar d_n\right)}\left(\tilde\Lambda_n\right)\right)\,.
\label{e2.23}
\end{align}
Обозначим
\begin{equation}
D_n=c_n-\bar d_n\,.
\label{e2.24}
\end{equation}
В леммах~3.5--3.7 следующего раздела показано, что
\begin{align}
D_n &=-\tau_n{\e}\left[\Delta\vert \Lambda=b \right]+ o\left(\tau_n\right)\,;\label{e2.25}\\
\tilde A_n &=-\fr{1}{2}\,D_n^2e^bp(b)+ o\left(\tau_n^2\right)\,; \label{e2.26}\\
\tilde B_n &=\fr{1}{2}\,\tau_n^2e^b{\e}\left[\Delta^2\vert \Lambda=b \right] p(b)+ o\left(\tau_n^2\right)\,,\label{e2.27}
\end{align}
где $b=\Phi_1^{-1}(1-\alpha)$ (см.\ лемму~2.1).

Из (\ref{e2.25})--(\ref{e2.27}) следует основная

\medskip

\noindent
\textbf{Теорема~2.8.} {\it В случае распределения Лапласа справедлива формула
$$
\lim\limits_{n\to\infty}\tau_n^{-2}\left(\beta_n^*-\beta_n\right)=\fr{1}{2}\,e^b{\D}\left[\Delta\vert \Lambda = b\right] p(b)\,.
$$ 
}

\section{Схема доказательства основного результата}

В этом разделе приведены формулировки вспомогательных лемм для
случая распределения Лапласа, составляющие схему доказательства
теоремы~2.8. Для удобства изложения доказательство вспомогательных
лемм вынесено в следующий раздел.

Обозначим
\begin{align*}
\bar Q_{n,l}(x)&=
\int\limits_{|z|\leq\eta_n\tau_n^{-1}} z^{l}
\left[{\p}_{n,0}\left(\tau_n^{-1}\Delta_n < z\big| \tilde\Lambda_n=x-\tau_nz\right)-
{\Ik}_{(0,\infty)}(z)\right]p_n\left(x-\tau_nz\right)\,dz\,;\\
\tilde Q_{n,l}(x)&=\int\limits_{-\infty}^{\infty} z^{l}\left[{\p}_{n,0}
\left(\tau_n^{-1}\Delta_n < z\big| \tilde\Lambda_n=x-\tau_nz\right)-
{\Ik}_{(0,\infty)}(z)\right]p_n(x-\tau_nz)\,dz\,,
\end{align*}
где $l=0,1$, а $p_n(x)$~--- плотность случайной величины
$\tilde \Lambda_n=\Lambda_n+\xi_n$ (см.~(\ref{e2.15})).

\medskip

\noindent
\textbf{Лемма~3.1.} {\it Справедливо соотношение
$
\sup\limits_{x} \left|\bar Q_{n,l}(x)-\tilde Q_{n,l}(x) \right|\rightarrow 0$, $l=0,1$.}

\medskip

Введем обозначения для условных мер на~${\sf R}^1$, зависящих от параметров 
$u\in{\sf R}^1$ и $t\in(0,C]$, $C>0$. Положим для $B\in{\cal B}^1$
\begin{align*}
\tilde Q_{n,u,t}^{*}(B)&={\p}_{n,0}\left(\tau_n^{-1}\Delta_n(t) \in B\big| \tilde\Lambda_n(t)=u\right)p_n(u,t)\,;
\\
Q_{u,t}^{*}(B)&={\p}\left(\Delta(t) \in B\big| \Lambda(t)=u\right)p(u,t)\,;
\end{align*}
их характеристические функции будем обозначать:
\begin{align*}
\tilde q_{n,u,t}^{*}(s)&\equiv \int e^{isx}\tilde Q_{n,u,t}^{*}(dx)
={\e}_{n,0}\left[ e^{is\tau_n^{-1}\Delta_n(t)}\vert \tilde\Lambda_n(t)=u\right]\,p_n(u,t)\,;
%\label{e3.3}
\\
q_{u,t}^{*}(s)&\equiv \int e^{isx}\tilde Q_{u,t}^{*}(dx)=
{\e}\left[ e^{is\Delta(t)}\vert \Lambda(t)=u\right]\,p(u,t)\,.
%\label{e3.4}
\end{align*}
Здесь зависимость случайных величин и плотностей от параметра~$t$ 
указана в явном виде, в то время как в остальной части работы такая зависимость опускается.

\medskip

\noindent
\textbf{Лемма~3.2.} {\it Для любого $s\in{\sf R}^1$
$
\sup\limits_{0<t\leq C}\sup\limits_{u}\left|
\tilde q_{n,u,t}^{*}(s)-q_{u,t}^{*}(s)  \right|\longrightarrow0$ при  $n\rightarrow\infty$.
}

Введем обозначения для функций распределения, зависящих от параметров~$u$ и~$t$:
\begin{align*}
\tilde F_{n,u,t}^{*}(z)&={\p}_{n,0}\left(\tau_n^{-1}\Delta_n(t) < z\big| \tilde\Lambda_n(t)=u\right)p_n(u,t)\,;
\\
F_{u,t}^{*}(z)&={\p}\left(\Delta(t) < z\big| \Lambda(t)=u\right)p(u,t)\,.
\end{align*}

\medskip

\noindent
\textbf{Лемма~3.3.} {\it  Для любой последовательности $\varepsilon_n\rightarrow 0$
$$
\sup\limits_{0<t\leq C}\sup\limits_{u}L\left( \tilde F_{n,u+\varepsilon_n,t}^{*},\,  
F_{u,t}^{*}\right)\longrightarrow 0\ \mbox{при}\  n\rightarrow\infty\,,
$$
где $L(F_1,\,F_2)$~--- расстояние Леви между функциями распределения~$F_1$ и~$F_2$ на~${\sf R}^1$. 
}

Будем обозначать для $l=0,1$
\begin{equation*}
Q_l(x)\equiv p(x)\int z^l\left[{\p}\left(\Delta < z\big| \Lambda=x\right)-{\Ik}_{(0,\infty)}(z)\right]\,dz
=-\fr{1}{l+1}{\e}\left(\Delta^{l+1}\big|\Lambda=x \right)p(x)\,.
\end{equation*}

\medskip

\noindent
\textbf{Лемма~3.4.} {\it  Справедливо соотношение $\sup\limits_{x} \left|\tilde Q_{n,l}(x)- Q_{l}(x) \right|\rightarrow 0$ 
для  $l=0,1$. 
}

\medskip

\noindent
\textbf{Лемма~3.5} {\it  Для величины~$D_n$} (\textit{см.}~(\ref{e2.24})) \textit{справедливо представление
$D_n=-\tau_n{\e}\left[\Delta\vert\Lambda=b \right]+ o\left(\tau_n\right)$
и $\bar d_n\rightarrow b$.
}

\medskip

\noindent
\textbf{Лемма~3.6} {\it Для величины~$\tilde A_n$} (\textit{см.}~(\ref{e2.22})) \textit{справедливо следующее представление:
$$
\tilde A_n=-\fr{1}{2}\,D_n^2e^{\bar d_n}p(\bar d_n)+ o\left(\tau_n^2\right)\,.
$$ }

\medskip

\noindent
\textbf{Лемма~3.7} {\it Для величины} $\tilde B_n$ (\textit{см.}~(\ref{e2.23})) \textit{справедливо следующее представление:
$$
\tilde B_n=\fr{1}{2}\,\tau_n^2 e^{\bar d_n}{\e}\left[\Delta^2\vert \Lambda=\bar d_n\right] p(\bar d_n)+ o\left(\tau_n^2\right)\,.\ \ \ \ \ \ \ \ \Box
$$ 
}


\section{Доказательство вспомогательных лемм}

\noindent
Д\,о\,к\,а\,з\,а\,т\,е\,л\,ь\,с\,т\,в\,о~леммы~3.1. Справедливо равенство
\begin{equation*}
\left|\bar Q_{n,l}(x)-\tilde Q_{n,l}(x) \right|
=\int\limits_{|z|\geq\eta_n/\tau_n} z^{l}\left[
{\p}_{n,0}\left(\tau_n^{-1}\Delta_n < z\big| \tilde\Lambda_n=x-\tau_nz\right)-
{\Ik}_{(0,\infty)}(z)\right]
p_n(x-\tau_nz)\,dz\equiv
I_{n,l}^+ +I_{n,l}^-\,,
\end{equation*}
где
\begin{equation}
I_{n,l}^+=\int\limits_{\eta_n/\tau_n}^{\infty} z^{l} {\p}_{n,0}
\left(\tau_n^{-1}\Delta_n \geq z \big| \tilde\Lambda_n=x-\tau_nz\right) p_n(x-\tau_nz)\,dz\,;\label{e4.1}
\end{equation}
\begin{multline}
I_{n,l}^-=\int\limits_{-\infty}^{-\eta_n/\tau_n} |z|^{l} 
{\p}_{n,0}\left(\tau_n^{-1}\Delta_n < z\big| \tilde\Lambda_n=x-\tau_nz\right)p_n(x-\tau_nz)\,dz={}\\
{}
=\int\limits_{\eta_n/\tau_n}^{\infty} u^{l} {\p}_{n,0}\left(
-\tau_n^{-1}\Delta_n > u\big| \tilde\Lambda_n=x+\tau_nu\right) p_n\left(x+\tau_nu\right)\,du\,.
\label{e4.2}
\end{multline}
Последнее равенство получено из предшествующего путем замены $z=u$.

Оценим $I_{n,l}^+$ и~$I_{n,l}^-$ с помощью неравенства Чебышёва вида:
$$
{\p}(X>x|Y)\leq\fr{{\e}\left[ |X|^{l+1}{\Ik}_{(x,\infty)}(|X|)\vert Y \right]}{x^{l+1}}\,,\enskip x>0\,.
$$
Получим
\begin{multline*}
I_{n,l}^+\leq \int\limits_{\eta_n/\tau_n}^{\infty} z^{-1}
{\e}_{n,0}\left[\left| \tau_n^{-1}\Delta_n \right|^{l+1} {\Ik}_{[z,\infty)}\left(\left|\tau_n^{-1}\Delta_n\right|\right)
\big| \tilde\Lambda_n=x-\tau_nz\right]p_n \left(x-\tau_nz\right)\,dz\leq{}\\
{}
\leq \tau_n\eta_n^{-1}\int\limits_{\eta_n/\tau_n}^{\infty}
{\e}_{n,0}\left[\left|\tau_n^{-1}\Delta_n\right|^{l+1} {\Ik}_{(\eta_n,\infty)}\left(\left|\Delta_n\right|\right)
\big| \tilde\Lambda_n=x-\tau_nz\right]p_n(x-\tau_nz)\,dz\leq{}\\
{}\leq
\tau_n\eta_n^{-1}\int\limits_{-\infty}^{\infty}
{\e}_{n,0}\left[\left|\tau_n^{-1}\Delta_n\right|^{l+1} {\Ik}_{(\eta_n,\infty)}\left(\left|\Delta_n\right|\right)
\big| \tilde\Lambda_n=x-\tau_nz\right]p_n\left(x-\tau_nz\right)\,dz={}
\end{multline*}
(с заменой $v=x-\tau_nz$)
\begin{equation*}
{}=\eta_n^{-1}\int\limits_{-\infty}^{\infty}
{\e}_{n,0}\left[|\tau_n^{-1}\Delta_n|^{l+1} {\Ik}_{(\eta_n,\infty)}(|\Delta_n|)
\big| \tilde\Lambda_n=v\right]p_n(v)\,dv=
\eta_n^{-1}{\e}_{n,0}|\tau_n^{-1}\Delta_n|^{l+1} {\Ik}_{(\eta_n,\infty)}(|\Delta_n|)\,.
\end{equation*}
Аналогично
\begin{multline*}
I_{n,l}^-\leq\int\limits_{\eta_n/\tau_n}^{\infty} u^{-1}
{\e}_{n,0}\left[|\tau_n^{-1}\Delta_n|^{l+1} {\Ik}_{(u,\infty)}(|\tau_n^{-1}\Delta_n|)
\big| \tilde\Lambda_n=x+\tau_nu\right]p_n(x+\tau_nu)\,du\leq{}\\
{}\leq\tau_n\eta_n^{-1}\int\limits_{\eta_n/\tau_n}^{\infty}
{\e}_{n,0}\left[|\tau_n^{-1}\Delta_n|^{l+1} {\Ik}_{(\eta_n,\infty)}(|\Delta_n|)
\big| \tilde\Lambda_n=x+\tau_nu\right]p_n(x+\tau_nu)\,du\leq{}
\\
{}\leq\tau_n\eta_n^{-1}\int\limits_{-\infty}^{\infty}
{\e}_{n,0}\left[|\tau_n^{-1}\Delta_n|^{l+1} {\Ik}_{(\eta_n,\infty)}(|\Delta_n|)
\big| \tilde\Lambda_n=x+\tau_nu\right]p_n(x+\tau_nu)\,du\leq
\end{multline*}
(с заменой $\nu=x+\tau_nu$)
\begin{equation*}
{}=\eta_n^{-1}\int\limits_{-\infty}^{\infty}
{\e}_{n,0}\left[|\tau_n^{-1}\Delta_n|^{l+1} {\Ik}_{(\eta_n,\infty)}(|\Delta_n|)
\big| \tilde\Lambda_n=\nu\right]p_n(\nu)\,d\nu=
\eta_n^{-1}{\e}_{n,0}|\tau_n^{-1}\Delta_n|^{l+1} {\Ik}_{(\eta_n,\infty)}(|\Delta_n|)\,.
\end{equation*}
Теперь для доказательства леммы достаточно показать, что правая часть последнего неравенства стремится к нулю. 
Используя~($i$) из леммы~2.3, получаем при $l=0$
\begin{multline*}
\eta_n^{-1}{\e}_{n,0}|\tau_n^{-1}\Delta_n| {\Ik}_{(\eta_n,\infty)}(|\Delta_n|)=
\eta_n^{-1}\tau_n^{-1}{\e}_{n,0}\fr{\Delta_n^2}{|\Delta_n|}\, {\Ik}_{(\eta_n,\infty)}(|\Delta_n|)\leq{}
\\
{}
\leq\eta_n^{-2}\tau_n^{-1}{\e}_{n,0}\Delta_n^2 {\Ik}_{(\eta_n,\infty)}(|\Delta_n|)= o\left(n^{-1/8}\right)\to 0\,,
\end{multline*}
а при $l=1$ аналогично
\begin{equation*}
\quad\quad\quad\eta_n^{-1}{\e}_{n,0}(\tau_n^{-1}\Delta_n)^2 {\Ik}_{(\eta_n,\infty)}(|\Delta_n|)
=
\eta_n^{-1}\tau_n^{-2}{\e}_{n,0}\Delta_n^2 {\Ik}_{(\eta_n,\infty)}(|\Delta_n|)= o\left(1\right) \to 0\,.
\quad\quad\quad\quad\quad\quad\quad\quad\quad\Box
\end{equation*}



\medskip

\noindent
Д\,о\,к\,а\,з\,а\,т\,е\,л\,ь\,с\,т\,в\,о~леммы~3.2.
Обозначим для любого $t\in(0,C]$, $C>0$, характеристические функции случайных векторов 
$(\tau_n^{-1} \Delta_n(t),\Lambda_n(t))$ и $(\Delta(t),\Lambda(t))$
\begin{align*}
\tilde q_{n,t}(s,y)&\equiv\int e^{iyu}\tilde q_{n,u,t}^{*}(s)\,du
={\e}_{n,0} e^{is\tau_n^{-1}\Delta_n(t)+iy\tilde\Lambda_n(t)}
={\e}e^{iy\xi_n}{\e}_{n,0} e^{is\tau_n^{-1}\Delta_n(t)+iy\Lambda_n(t)}
\equiv \omega_n(y)q_{n,t}(s,y)\,;
\\
q_{t}(s,y)&\equiv\int e^{iyu} q_{u,t}^{*}(s)\,du={\e} e^{is\Delta(t)+iy\Lambda(t)}\,.
\end{align*}
Из~(\ref{e2.14}) для~$q_{u,t}^{*}(s)$ справедлива формула обращения
$$
\tilde q_{u,t}^{*}(s)=\fr{1}{2\pi}\int e^{-iuy}q_{t}(s,y)\,dy\,.
$$
Тогда для каждого $s\in{\sf R}^1$ рассмотрим оценку
\begin{equation*}
\left|\tilde q_{n,u,t}^{*}(s)-q_{u,t}^{*}(s)  \right|\leq\int \left| 
\tilde q_{n,t}(s,y) - q_{t}(s,y) \right|\,dy=
\int\limits_{|y|<\delta\sqrt n}+\int\limits_{\delta\sqrt{n}\leq|y|<n}+\int\limits_{|y|>n}\,,
\end{equation*}
где $\delta>0$ выбирается для каждого $s\in{\sf R}^1$ так, чтобы в рамках центральной предельной теоремы 
(см.~(\ref{e2.14})) для каждого $s\in{\sf R}^1$ и из свойств случайной величины~$\xi_n$ 
(см.~(\ref{e2.16})) по схеме доказательства~(6.15) из работы~\cite{2ben} 
с использованием теоремы о мажорируемой сходимости выполнялось
$$
\int\limits_{|y|<\delta\sqrt n} \left|\tilde q_{n,t}(s,y) - q_{t}(s,y) \right|\,dy\rightarrow 0\,.
$$
Применяя рассуждения для~(6.16) из работы~\cite{2ben} к случаю распределения Лапласа (см.\ 
также пример~1.3 из~\cite{6ben}), получаем для каждого $s\in{\sf R}^1$
$$
\int\limits_{\delta\sqrt{n}\leq|y|<n} \left| q_{n,t}(s,y)\right|\,dy \longrightarrow 0\,.
$$
Сходимость к нулю остальных интегралов следует из свойств случайной величины~$\xi_n$ (см.~(\ref{e2.16})) 
и случайного вектора~$(\Delta,\Lambda)$ (см.~(\ref{e2.14})), что завершает доказательство леммы.\hfill$\Box$

\smallskip

\noindent
Д\,о\,к\,а\,з\,а\,т\,е\,л\,ь\,с\,т\,в\,о~леммы~3.3.
Из леммы~6.1 работы~\cite{8ben}, леммы~3.2 и ограниченности~$\varepsilon_n$ следует, что
$$
L(\tilde Q_{n,u+\varepsilon_n,t}^{*}, Q_{u,t}^{*})\rightarrow 0
$$
равномерно по $u$ и $t$, поскольку семейство мер~$\{ Q_{u,t}^{*} \}$ плотно (см.\ лемму~6.2 из~\cite{2ben}) 
и ограничено по~$u$ и~$t$ (см.\ лемму~2.1). Тогда из свойств меры~$Q_{u,t}^{*}$ и эквивалентного 
определения для слабой сходимости мер (см., например, теорему~2.1 из работы~\cite{9ben}) вытекает утверждение леммы.
\hfill$\Box$

\smallskip

\noindent
Д\,о\,к\,а\,з\,а\,т\,е\,л\,ь\,с\,т\,в\,о~леммы~3.4. Разобьем пределы интегрирования на
три части:
$$
\left|\tilde Q_{n,l}(x)\:-\: Q_{l}(x) \right|=\left|\,\, \int\limits_{-\infty}^{-a_n}+\int\limits_{-a_n}^{a_n}
+\int\limits_{a_n}^{\infty} \right|\,,
$$
$a_n\rightarrow+\infty$ будет выбрано далее. Поскольку $Q_{l}(x)\rightarrow0$ 
на $(-\infty,-a_n)$, $(a_n,\infty)$, то рассмотрим на этих интервалах~$\tilde Q_{n,l}(x)$. 
Объединяя~(\ref{e4.1}) и~(\ref{e4.2}) и применяя неравенство Чебышёва, как в доказательстве леммы~3.1, получаем
\begin{multline*}
\left|\,\int\limits_{a_n}^{\infty} z^{l} {\p}_{n,0}\left(
\mp\tau_n^{-1}\Delta_n \geq z\big| \tilde\Lambda_n=x\pm\tau_nz\right)p_n(x\pm\tau_nz)\,dz\right|\leq{}
\\
{}
\leq\left|\, \int\limits_{a_n}^{\infty} z^{-1}
{\e}_{n,0}\left[|\tau_n^{-1}\Delta_n|^{l+1} {\Ik}_{[z,\infty)}(|\tau_n^{-1}\Delta_n|)
\big| \tilde\Lambda_n=x\pm\tau_nz\right]p_n(x\pm\tau_nz)\,dz\right|\leq{}\\
\end{multline*}
(с заменой $x\pm\tau_nz=v$)
\begin{multline*}
{}\leq\tau_n^{-1}a_n^{-1}\left|\,\, \int\limits_{-\infty}^{\infty}
{\e}_{n,0}\left[|\tau_n^{-1}\Delta_n|^{l+1} {\Ik}_{[a_n,\infty)}(|\tau_n^{-1}\Delta_n|)
\big| \tilde\Lambda_n=v\right]p_n(v)dv\right|={}\\
{}=\tau_n^{-(l+2)}a_n^{-1}{\e}_{n,0}|\Delta_n|^{l+1}{\Ik}_{[a_n,\infty)}
(|\tau_n^{-1}\Delta_n|)={\cal O}(e^{-a_n})\rightarrow 0\,,
\end{multline*}
где последовательность $a_n\rightarrow+\infty$ может быть выбрана любая. Последняя оценка получена из 
неравенства Гёльдера и леммы~2.2.
Тогда имеем
\begin{multline*}
\!\!\left|\tilde Q_{n,l}(x)- Q_{l}(x) \right|\leq
\left|\, \int\limits_{|z|\leq a_n} z^{l}\left( {\p}_{n,0}\left(\tau_n^{-1}\Delta_n < 
z\vert \tilde\Lambda_n=x-\tau_nz\right)p_n(x-\tau_nz)-
{\p}\left(\Delta < z\big| \Lambda=x\right)p(x)\right)\,dz \right|+{}\\
{}
+\left| \int\limits_{0}^{a_n} z^{l}\left( p_n(x-\tau_nz)\:-\:p(x)\right)\,dz \right|+{\cal O}\left(e^{-a_n}\right)\,.
\end{multline*}
Используя леммы~3.2 и 3.3 из~\cite{5ben} (см.\ также~\cite{6ben}) и технику доказательства 
локальной предельной теоремы, несложно показать, что
$
\kappa_n\equiv\sup\limits_{0<t\leq C}\sup\limits_{x}\left| p_n(x)-p(x) \right|\rightarrow 0\,.
$
Из леммы~3.3 следует, что для $|z|\leq a_n={\it o}(\tau_n^{-1})$
$$
\lambda_n\equiv\sup\limits_{0<t\leq C}\sup\limits_{x}L\left( 
\tilde F_{n,x-\tau_nz,t}^{*},\, F_{x,t}^{*}\right)\rightarrow 0\,.
$$
Теперь, если выбрать, например,
$
a_n=\min(\lambda_n^{-1/3},\kappa_n^{-1/3},n^{1/8})$, получим утверждение леммы.\hfill
$\Box$
%\pagebreak

%\smallskip

\noindent
Д\,о\,к\,а\,з\,а\,т\,е\,л\,ь\,с\,т\,в\,о~леммы~3.5. Доказательство леммы проведено в работе~\cite{5ben} (см.
 там лемму~3.5).

\smallskip

\noindent
Д\,о\,к\,а\,з\,а\,т\,е\,л\,ь\,с\,т\,в\,о~леммы~3.6. Доказательство леммы без изменений переносится из 
работы~\cite{1ben} (см.\ там лемму~3.4.3).

\smallskip

\noindent
Д\,о\,к\,а\,з\,а\,т\,е\,л\,ь\,с\,т\,в\,о~леммы~3.7.
Представим $\tilde B_n$ как
\begin{equation*}
\tilde B_n={\e}_{n,0}\left(e^{\tilde \Lambda_n}-e^{\bar d_n}\right)
\left({\Ik}_{(-\infty,\bar d_n)}(\tilde S_n)-{\Ik}_{(-\infty,\bar d_n)}\left(\tilde\Lambda_n\right)\right)
+
\rho_{n1}+\rho_{n2}\,,
\end{equation*}
где
\begin{align*}
\rho_{n1}&\equiv{\e}_{n,0}\left(e^{\tilde \Lambda_n}-e^{\bar d_n}\right)
\left({\Ik}_{[\bar d_n,\infty)}\bigl(S_n\bigr)-\Psi_n(S_n)\right)
=
(1-\gamma){\e}_{n,0}\left(e^{\tilde \Lambda_n}-e^{\bar d_n}\right){\Ik}_{\{\bar d_n\}}\left(S_n\right)\,;
\\
\rho_{n2}&\equiv{\e}_{n,0}\left(e^{\tilde \Lambda_n}-e^{\bar d_n}\right)
\left({\Ik}_{(-\infty,\bar d_n)}(S_n)-{\Ik}_{(-\infty,\bar d_n)}\left(\tilde S_n\right)\right)\,.
\end{align*}
Имеем
\begin{multline*}
|\rho_{n1}|\leq{\e}_{n,0}\left|e^{\tilde \Lambda_n}-e^{\bar d_n}\right|{\Ik}_{\{\bar d_n\}}\left(S_n\right)
\leq{\e}_{n,0}\left|e^{\tilde \Lambda_n}-e^{\bar d_n}\right|{\Ik}_{\{\bar d_n\}}
\left(S_n\right){\Ik}_{[0,n^{-1/8})}\left(|\Delta_n|\right)
+{}\\
{}+{\e}_{n,0}\left|e^{\tilde \Lambda_n}-e^{\bar d_n}\right|{\Ik}_{[n^{-1/8},\infty)}
\left(|\Delta_n|\right)\leq{}
\end{multline*}
(с учетом леммы~2.5)
\begin{multline*}
\leq e^{\bar d_n}{\e}\left|e^{\xi_n}-1\right| {\e}_{n,0}e^{-\Delta_n}{\Ik}_{\{\bar d_n\}}
\left(S_n\right){\Ik}_{[0,n^{-1/8})}\left(|\Delta_n|\right)+
e^{\bar d_n}{\e}_{n,0}\left|e^{-\Delta_n}-1\right|{\Ik}_{\{\bar d_n\}}\left(S_n\right)
{\Ik}_{[0,n^{-1/8})}\left(|\Delta_n|\right)+{}\\
{}
+e^{\sigma_n^2/2}{\e}_{n,0}e^{\Lambda_n}{\Ik}_{[n^{-1/8},\infty)}\left(|\Delta_n|\right)+
e^{\bar d_n}{\e}_{n,0}{\Ik}_{[n^{-1/8},\infty)}\left(|\Delta_n|\right)\leq{}
\end{multline*}
(с учетом неравенства Гёльдера и $|e^y-1|<|y|e^{|y|}$ для любого ограниченного~$y$)
\begin{multline*}
\leq e^{\bar d_n+n^{-1/8}}{\e}\left|e^{\xi_n}-1\right|+
e^{\bar d_n+n^{-1/8}}n^{-1/8}{\p}_{n,0}\left (S_n = \bar d_n\right)+{}\\
{}
+e^{\sigma_n^2/2}\left({\e}_{n,0}e^{2\Lambda_n}\right)^{1/2}\left({\p}_{n,0}\left(|\Delta_n|\geq n^{-1/8}\right)
\right)^{1/2}+
e^{\bar d_n}{\p}_{n,0}\left(|\Delta_n|\geq n^{-1/8}\right)\,.
\end{multline*}
С учетом того, что ${\p}_{n,0}(S_n=\bar d_n)={\cal O}(\tau_n^2)$,
в связи с леммой~2.2
${\p}_{n,0}\left(\left|\Delta_n\right|\geq n^{-1/8}\right)\leq
Ce^{-n^{-1/8}n^{1/4}}=$\linebreak $=Ce^{-n^{1/8}}$, из доказательства леммы~2.6
${\e}\left|e^{\xi_n}-1\right|= o\left(\tau_n^2\right)$
и~(2.10) работы~\cite{3ben} (см.\ там лемму~2.1) для $s=-ix$
$$
{\e}_{n,1}e^{x\Lambda_n}\longrightarrow e^{-{t^2(-x^2-x)}/2}
$$
для любого фиксированного~$x$, имеем $\rho_{n1}= o\left(\tau_n^2\right)$.
Аналогичные рассуждения применим для оценки $|\rho_{n2}|$:
\begin{multline*}
|\rho_{n2}|\leq\left| {\e}_{n,0}\left(e^{\tilde \Lambda_n}-e^{\bar d_n}\right)
\left({\Ik}_{(-\infty,\bar d_n)}(S_n)-{\Ik}_{(-\infty,\bar d_n)}\left(\tilde S_n\right)\right){\Ik}_{[0,n^{-1/8})}
\left(\left|\Delta_n\right|\right) \right|+{}\\
{}+
\left| {\e}_{n,0}\left(e^{\tilde \Lambda_n}-e^{\bar d_n}\right)
\left({\Ik}_{(-\infty,\bar d_n)}(S_n)-{\Ik}_{(-\infty,\bar d_n)}\left(\tilde S_n\right)\right){\Ik}_{[n^{-1/8},\infty)}
\left(|\Delta_n|\right)  \right|\leq{}\\
{}
\leq {\e}_{n,0}\left|e^{\tilde \Lambda_n}-e^{\bar d_n}\right|
\left|{\Ik}_{(-\infty,\bar d_n)}(S_n)-{\Ik}_{(-\infty,\bar d_n)}\left(\tilde S_n\right)\right|
{\Ik}_{[0,n^{-1/8})}\left(|\Delta_n|\right){\Ik}_{[0,\tau_n^{2+\beta})}\left(|\xi_n|\right)+{}\\
{}
+{\e}_{n,0}\left|e^{\tilde \Lambda_n}-e^{\bar d_n}\right|
\left|{\Ik}_{(-\infty,\bar d_n)}(S_n)-{\Ik}_{(-\infty,\bar d_n)}\left(\tilde S_n\right)\right|
{\Ik}_{[0,n^{-1/8})}\left(\left|\Delta_n\right|\right){\Ik}_{[\tau_n^{2+\beta},\infty)}\left(\left|\xi_n\right|\right)+
{\cal O}(e^{-n^{1/8}})\leq{}
\end{multline*}
(с учетом~(\ref{e2.21}))
\begin{multline*}
\leq {\e}\left|e^{\xi_n}-1\right|{\e}_{n,0}e^{\Lambda_n}
\left|{\Ik}_{(-\infty,\bar d_n)}(S_n)-{\Ik}_{(-\infty,\bar d_n)}\left(
\tilde S_n\right)\right|{\Ik}_{[0,n^{-1/8})}\left(
\left|\Delta_n\right|\right){\Ik}_{[0,\tau_n^{2+\beta})}\left(\left|\xi_n\right|\right)+{}\\
{}
+ e^{\bar d_n}{\e}_{n,0}\left| e^{\Lambda_n-\bar d_n}-1\right|
\left|{\Ik}_{(-\infty,\bar d_n)}(S_n)-{\Ik}_{(-\infty,\bar d_n)}\left(
\tilde S_n\right)\right|{\Ik}_{[0,n^{-1/8})}\left(\left|\Delta_n\right|\right)
{\Ik}_{[0,\tau_n^{2+\beta})}\left(\left|\xi_n\right|\right)+
{\it o}\left(\tau_n^2\right)\leq{}
\\
{}
\leq e^{\bar d_n}{\e}_{n,0}\left| e^{\Lambda_n-\bar d_n}-1\right|
{\Ik}_{[\bar d_n-\tau_n^{2+\beta},\bar d_n+\tau_n^{2+\beta}]}\left(S_n\right)
{\Ik}_{[0,n^{-1/8})}\left(\left|\Delta_n\right| \right)+{\it o}\left(\tau_n^2\right)\leq{}
\end{multline*}
%\pagebreak

\begin{multicols}{2}

\noindent
(поскольку $\left|\Lambda_n - \bar d_n\right|\leq\left|S_n-\bar d_n\right|+\left|\Delta_n\right|
\leq\tau_n^{2+\beta}+$\linebreak $+n^{-1/8}$)
\begin{multline*}
\leq C n^{-1/8}{\p}_{n,0}\left( 
\bar d_n-\tau_n^{2+\beta}\leq S_n\leq\bar d_n+\tau_n^{2+\beta} \right)+{}\\
{}+{\it o}\left(\tau_n^2\right)={\it o}\left(\tau_n^2\right)\,.
\end{multline*}
Теперь $\tilde B_n$ запишем как
\begin{multline*}
\tilde B_n={\e}_{n,0}\left(e^{\tilde \Lambda_n}-e^{\bar d_n}\right)
\left({\Ik}_{(-\infty,\bar d_n)}\left(\tilde S_n\right)-{}\right.\\
{}\left. -
{\Ik}_{(-\infty,\bar d_n)}\left(\tilde\Lambda_n\right)\right)
{\Ik}_{[0,\eta_n)}\left(\left|\Delta_n\right|\right)
+\rho_{n3}+{\it o}\left(\tau_n^2\right)\,,
\end{multline*}
где аналогично рассуждениям выше
\begin{multline*}
\rho_{n3}\equiv {\e}_{n,0}\left(e^{\tilde \Lambda_n}-e^{\bar d_n}\right)
\left({\Ik}_{(-\infty,\bar d_n)}(\tilde S_n)-{}\right.\\
\left.{}-{\Ik}_{(-\infty,\bar d_n)}
\left(\tilde\Lambda_n\right)\right){\Ik}_{[\eta_n,\infty)}\left(\left|\Delta_n\right|\right)
=
{\it o}\left(\tau_n^2\right)\,.
\end{multline*}
Теперь
\begin{multline*}
\tilde B_n=\tau_n e^{\bar d_n}\int\limits_{|z|\leq\eta_n/\tau_n} \left(e^{-\tau_n z}-1 \right)\times{}\\
{}\times\left[{\p}_{n,0}(\tau_n^{-1}\Delta_n < z\big| \tilde\Lambda_n=\bar d_n-\tau_n z )-{\Ik}_{(0,\infty)}(z) \right]\times{}\\
{}\times
p_n\left(\bar d_n - \tau_n z\right)\,dz+{\it o}\left(\tau_n^2\right)+\rho_{n4}\,,
\end{multline*}
где
\begin{multline*}
\left|\rho_{n4}\right|\leq 2\tau_n e^{\bar d_n}\times{}\\
{}\times\int\limits_{|z|\leq\eta_n/\tau_n}\left|e^{-\tau_nz}-1\right|
{\p}_{n,0}\left(\left|\Delta_n\right|>\eta_n\big| \tilde\Lambda_n={}\right.\\
\left.{}=
\bar d_n-\tau_nz
\vphantom{\tilde{\Lambda_n}}\right)
 p_n\left(\bar d_n-\tau_nz\right)\,dz\leq{}\\
{}\leq 2\eta_n e^{\bar d_n+\eta_n}{\p}_{n,0}\left(\left|\Delta_n\right|>\eta_n\right)={\it o}\left(\tau_n^2\right)\,.
\end{multline*}
Используя неравенство
$
\left|e^{-s}-1+s\right|\leq (1/2)\gamma^2e^{\gamma}$, $\left|s\right|\leq\gamma$,
где $s=\tau_nz$, можно представить $\tilde B_n$ как
\begin{multline*}
\tilde B_n\equiv -\tau_n^2e^{\bar d_n}\int\limits_{|z|\leq\eta_n/\tau_n}
z\left[ {\p}_{n,0}\left(\tau_n^{-1}\Delta_n<z \big| \tilde\Lambda_n={}\right.\right.\\
\left.\left.{}=\bar d_n-\tau_nz
\vphantom{\tilde{\Lambda_n}}\right) - 
{\Ik}_{(0,\infty)}(z) \right]
p_n\left(\bar d_n - \tau_nz\right)\,dz+{}\\
{}+{\it o}\left(\tau_n^2\right)+\rho_{n5}\,,
\end{multline*}
где
\begin{multline*}
\!\!\!\!\left|\rho_{n5}\right|\leq \fr{1}{2}\,\tau_n^2\eta_n 
e^{\bar d_n+\eta_n}\!\!\!\!\!\!\int\limits_{\!|z|\leq\eta_n/\tau_n}
\!\!\!\!\!\!\!\!\!\!|z|\left| {\p}_{n,0}\left(\tau_n^{-1}\Delta_n<z\big| \tilde\Lambda_n={}\right.\right.\\
\left.{}=\left.\bar d_n-\tau_nz\vphantom{\tilde{\Lambda}_n}\right) - {\Ik}_{(0,\infty)}(z) 
\vphantom{\tilde{\Lambda}_n}\right|
 p_n\left(\bar d_n - \tau_nz\right)\,dz\,.
\end{multline*}
Последний интеграл равен $-\bar Q_{n,1}(\bar d_n)$, и 
из леммы~3.4 для $l=1$ имеем
$\rho_{n5}={\it o}\left(\tau_n^2\right)$.
Тогда
$\tilde B_n=-\tau_n^2 e^{\bar d_n}\bar Q_{n,1}(\bar d_n)+{\it o}\left(\tau_n^2\right)$,
и, поскольку
$$
\bar Q_{n,1}(\bar d_n)\rightarrow Q_{1}(\bar d_n)\equiv -\fr{1}{2}\,{\e}\left[\Delta^2\big|\Lambda=\bar d_n\right]
p\left(\bar d_n\right)\,,
$$
отсюда и следует утверждение леммы.
\hfill$\Box$

%\begin{multicols}{2}

{\small\frenchspacing
{%\baselineskip=10.8pt
\addcontentsline{toc}{section}{Литература}
\begin{thebibliography}{99}

\bibitem{1ben}  %1
\Au{Bening~V.\,E.}
Asymptotic theory of testing statistical hypotheses.~---
Utrecht: VSP,  2000. 277~p.

\bibitem{2ben} %2
\Au{Чибисов~Д.\,М.} 
Вычисление дефекта асимптотически эффективных критериев~// Теория вероятностей и ее применения, 1985. 
Т.~30. Вып.\,2. С.~289--310.

\bibitem{10ben} %3
\Au{Kotz~S., Kozubowski~T.\,J., Podgorski~K.}
The Laplace distribution and generalizations:
A revisit with applications to communications, economics, engineering,
and finance.~--- Birkhauser, 2001. 349~p.

\bibitem{8ben} %4
\Au{Chibisov~D.\,M.}
An asymptotic expansion for distributions of C($\alpha$) test statistics~// Lecture Notes in Statistics, 1980. 
Vol.~2. P.~63--96.

\bibitem{6ben} %5
\Au{Chibisov~D.\,M., van~Zwet~W.\,R.}
On the edgeworth expansion for the logarithm of the likelihood
ratio. I~// Теория вероятностей и ее применения, 1984. Т.~29. Вып.\,3. С.~417--439.

\bibitem{4ben} %6
\Au{Королев~Р.\,А., Бенинг~В.\,Е.}
Асимптотические
разложения для мощностей критериев в случае распределения Лапласа~//
Вестник Тверского государственного университета. Сер.\
Прикладная математика, 2008. Вып.\,3(10). №\,26(86). С.~97--107.

\bibitem{3ben}  %7
\Au{Королев~Р.\,А., Тестова~А.\,В., Бенинг~В.\,Е.}
О мощности асимптотически оптимального критерия в случае
распределения Лапласа~//
Вестник Тверского государственного университета.
Сер.\ Прикладная математика, 2008. Вып.\,8. №\,4(64). С.~5--23.

\bibitem{5ben} %8
\Au{Королев~Р.\,А.}
Формула для предела нормированной разности мощностей критериев в случае распределения Лапласа~//
Вестник Тверского государственного университета. Сер.\
Прикладная математика, 2010. В печати.

\bibitem{7ben} %9
\Au{Феллер~В.}
Введение в теорию вероятностей и ее приложения. Т.~1.~---
М.: Мир, 1984. 528~с.


\label{end\stat}

\bibitem{9ben} %10
\Au{Billingsley~P.} Convergence of probability measures.~--- Wiley, Canada, 1999. 278~p.



 \end{thebibliography}
}
}

\end{multicols}