\def\stat{kudr}

\def\tit{ПРИБЛИЖЕННЫЕ МЕТОДЫ РЕШЕНИЯ ЗАДАЧИ ДИАГНОСТИКИ ПЛОСКИМ 
ЗОНДОМ СИЛЬНОИОНИЗОВАННОЙ ПЛАЗМЫ С~УЧЕТОМ КУЛОНОВСКИХ 
СТОЛКНОВЕНИЙ}

\def\titkol{Приближенные методы решения задачи диагностики плоским 
зондом сильноионизованной плазмы} %с~учетом Кулоновских  столкновений}

\def\autkol{И.\,А.~Кудрявцева, А.\,В.~Пантелеев}
\def\aut{И.\,А.~Кудрявцева$^1$, А.\,В.~Пантелеев$^2$}

\titel{\tit}{\aut}{\autkol}{\titkol}

%{\renewcommand{\thefootnote}{\fnsymbol{footnote}}\footnotetext[1]
%{Работа поддержана Российским фондом фундаментальных исследований
%(проекты 11-01-00515а и 11-07-00112а), а также Министерством
%образования и науки РФ в рамках ФЦП <<Научные и
%научно-педагогические кадры инновационной России на 2009--2013~годы>>.}}


\renewcommand{\thefootnote}{\arabic{footnote}}
\footnotetext[1]{Московский авиационный институт, irina.home.mail@mail.ru}
\footnotetext[2]{Московский авиационный институт, avpanteleev@inbox.ru}

\vspace*{-2pt}

\Abst{Сформирована математическая модель, описывающая динамику сильноионизованной 
плазмы с учетом столкновений заряженных частиц вблизи плоского зонда. Модель включает уравнение 
Фоккера--Планка и уравнение Пуассона. Предложено два подхода к решению задачи: на основе метода 
статистических испытаний Мон\-те-Кар\-ло и на основе композиции метода крупных частиц и метода 
расщепления.} 

\vspace*{-2pt}

\KW{телекоммуникационные системы; метод Монте-Карло; метод крупных частиц; метод 
расщепления; зонд; уравнение Фоккера--Планка; уравнение Пуассона} 

\vspace*{-4pt}

 \vskip 8pt plus 9pt minus 6pt

      \thispagestyle{headings}

      \begin{multicols}{2}
      
            \label{st\stat}

\section{Введение}

В настоящее время в области телекоммуникаций все более востребованными становятся 
информационные технологии, основанные на использовании математических моделей и численных 
методов физики плазмы. Поэтому особенно актуальным является решение разнообразных задач анализа 
поведения плазмы, включающих в себя формирование новых моделей и методов их исследования. 
Помимо этого, в разработке телекоммуникационного оборудования эффективно используются 
собственно физические свойства плазмы. В~частности, изготовлена антенна, работа которой основана 
на газовом разряде низкотемпературной плазмы~[1], интенсивно ведутся разработки по созданию и 
усовершенствованию источников бесперебойного питания на основе плазменных элементов~[2, 3]. 
      
      Одним из наиболее перспективных направлений для построения систем оптической 
беспроводной связи является использование лазеров~\cite{4-k, 5-k}. В~этой связи большое внимание 
уделяется использованию плазмы при разработке импульсных сильноточных коммутаторов~\cite{6-k}, 
так как практическое применение подобных разработок требует повышения уровня надежности и 
быстродействия лазерных систем.
      
      Исследования низкотемпературной плазмы также связаны с разработками в области дальней 
космической связи, так как моделирование процессов взаимодействия заряженного тела с верхними 
слоями атмосферы позволяет предлагать способы улучшения существующих систем радиосвязи с 
космическими летательными аппаратами~\cite{7-k}. 
      
      Наряду с этим актуальными также являются задачи диагностики плазмы, поскольку перспективы 
ее использования в области телекоммуникаций после более полного изучения физических свойств 
могут значительно расшириться. 

Для диагностики плазмы применяют зондовые методы исследования~[8--11]. Эти методы относятся к 
классу контактных методов; как следствие, возникает сложность в исследовании пристеночной области 
вблизи зонда, которая характеризуется достаточно сложным распределением потенциала и функциями 
распределения, отличными от максвелловских. 

Данная работа посвящена исследованию переходного режима обтекания заряженного тела плазмой. Для 
переходного режима выполняется следующее условие: длина свободного пробега иона до столкновения 
с нейтральным атомом или другим ионом невелика по сравнению с характерными размерами тела. 
В~этом случае возникает необходимость учета столкновений заряженных частиц с нейтральными 
атомами и кулоновских столкновений. В~работах~\cite{10-k, 11-k} подробно рассмотрена модель с 
учетом столкновений заряженных частиц с нейтральными атомами. В~настоящей статье представлена 
теоретическая модель, описывающая влияния ион-ионных и ион-элек\-т\-рон\-ных столкновений на 
измеряемые характеристики плазмы, что ранее детально не исследовалось.
      
      В~рамках данной работы предлагается модель, описывающая динамику сильноионизованной 
плазмы с учетом кулоновских столкновений. Эта модель учитывает такие процессы взаимодействия, 
как перенос частиц и столкновения между заряженными частицами типа <<ион--ион>> и 
      <<ион--электрон>> под влиянием макроскопического электрического поля. Перечисленные 
процессы описываются самосогласованной системой уравнений, включающей уравнение 
      Фок\-ке\-ра--План\-ка и уравнение Пуассона~[12].
      
      Вычислительная модель задачи строится на основе двух методов: метода статистических 
испытаний Мон\-те-Кар\-ло и композиции метода крупных частиц и метода расщепления. Приведены 
результаты численного моделирования, полученные с использованием вышеперечисленных методов.

\vspace*{-4pt}

\section{Постановка задачи}

\vspace*{-2pt}

Рассматривается следующая физическая постановка зондовой задачи~[11]. В~невозмущенную 
бесконечно протяженную плазму, состоящую из электронов и однозарядных ионов, внесена большая\linebreak 
заряженная до потенциала $\varphi_p$ плоскость. Плоскость, расположенная поперек потока плазмы, 
является идеально поглощающей для электронов. Ионы при ударе о плоскость нейтрализуются. 
Предполагается, что частицы в плазме движутся под действием внешнего электрического поля, 
магнитное поле отсутствует. Концентрации ионов $n_{i\infty}$ и электронов $n_{e\infty}$, а также 
температуры данных час\-тиц~$T_{i\infty}$ 
и~$T_{e\infty}$ в невозмущенной плазме заданы. За начальные 
функции распределения обоих типов час\-тиц принимаются функции распределения Максвелла. 
      
      Требуется с учетом столкновений между заряженными частицами найти напряженность 
самосогласованного электрического поля $\vec{E}(\vec{r},t)$, функции распределения однозарядных 
ионов $f_i(\vec{r}, \vec{v}, t)$ и электронов $f_e(\vec{r}, \vec{v}, t)$, 
а также их моменты (плотности 
токов ионов и электронов  $j_i(\vec{r},t)\hm
=q\int f_i(\vec{r}, \vec{v}, t)\vec{v}\,d\vec{v}$, $j_e(\vec{r},t) 
\hm={\sf e}\int f_e(\vec{r},\vec{v},t)\vec{v}\,d\vec{v}$, где $q=Z_i{\sf e}$, $Z_i=1$~--- заряд иона, ${\sf 
e}$~--- заряд электрона; концентрации ионов и электронов $n_i(\vec{r},t)\hm=\int 
f_i(\vec{r},\vec{v},t)\,d\vec{v}$, $n_e(\vec{r},t)\hm=\int f_e(\vec{r},\vec{v}, t)\,d\vec{v}$). 
Поведение частиц во 
времени~$t$ характеризуется ра\-ди\-ус-век\-то\-ром~$\vec{r}$ и вектором скорости~$\vec{v}$.
      
      Математическая модель, соответствующая данной физической постановке задачи, имеет 
вид~\cite{11-k, 13-k}:

\noindent
      \begin{equation}
      \left.
      \begin{array}{c}
      \fr{\partial f_\alpha (\vec{r},\vec{v},t)}{\partial t}+
      \vec{v}\fr{\partial f_\alpha (\vec{r},\vec{v},t)}{ 
\partial \vec{r}}+
\fr{\vec{F}_\alpha(\vec{r},t)}{m_\alpha}\times{}\\[4pt]
{}\times\fr{\partial f_\alpha(\vec{r},\vec{v},t)}{ \partial 
\vec{v}}=
\left(\fr{\partial f_\alpha(\vec{r},\vec{v},t)}{ \partial t}\right)_{\mathrm{с}}+S_\alpha 
(\vec{r},\vec{v},t)\,;\\[6pt]
      \Delta\varphi(\vec{r},t)=-\fr{{\sf e}}{\varepsilon_0}\left( n_i(\vec{r},t)-n_e(\vec{r},t)\right)\,;\\[6pt]
      \vec{E}(\vec{r},t)=-\nabla \varphi(\vec{r},t)\,.
      \end{array}\!\!
      \right\}\!\!
      \label{e1-k}
      \end{equation}
Здесь первое уравнение~--- уравнение Фок\-ке\-ра--План\-ка для частиц сорта~$\alpha$ ($\alpha=i,e$), 
второе~--- уравнение Пуассона для самосогласованного электрического поля; 
$f_\alpha(\vec{r},\vec{v},t)$~--- функция\linebreak
распределения час\-тиц сорта~$\alpha$; $(\partial 
f_\alpha(\vec{r},\vec{v},t)/\partial t)_{\mathrm{с}}$~--- 
оператор столкновений Фок\-ке\-ра--План\-ка; 
функция~$S_\alpha(\vec{r},\vec{v},t)$ описывает источники или стоки\linebreak
 час\-тиц; 
$\vec{F}_\alpha(\vec{r},t)=q_\alpha\vec{E}(\vec{r},t)$, где $\vec{E}(\vec{r},t)$~--- напряженность 
самосогласованного электрического поля, 
$$
q_\alpha =
\begin{cases}
-{\sf e}\,, & \alpha=e\,,\\
{\sf e}\,, & \alpha=i\,;
\end{cases}
$$
$\varphi(\vec{r},t)$~--- потенциал самосогласованного электрического поля; $n_\alpha(\vec{r},t)$ ($\alpha 
\hm=i,e$)~--- концентрация частиц сорта~$\alpha$; $m_\alpha$~--- масса частицы сорта~$\alpha$; 
$\varepsilon_0$~--- электрическая постоянная. 

Оператор столкновений Фок\-ке\-ра--План\-ка имеет вид~\cite{13-k, 14-k}
\begin{multline*}
\fr{1}{\Gamma_\alpha}\left( \fr{\partial f_\alpha}{\partial t}\right)_{\mathrm{с}} 
=\fr{1}{2}\,\nabla_v\nabla_v:\left(f_\alpha\nabla_v\nabla_vg_\alpha(\vec{r},\vec{v},t)\right)-{}\\
{}-
\nabla_v\cdot\left(f_\alpha\nabla_v h_\alpha\right)\,,
\end{multline*}
где $\nabla_v\nabla_v g_\alpha(\vec{r},\vec{v},t)$~--- ковариантная тензорная производная второго ранга, 
знак двоеточия ($:$) обозначает операцию двойного суммирования:
\begin{gather*}
\Gamma_\alpha=\fr{Z_\alpha^4 {\sf e}^4}{4\pi \varepsilon_0^2 m^2_\alpha}\,\ln D_\alpha\,;
\\
D_\alpha =\fr{12\pi\varepsilon_0 kT_{\alpha\infty}}{Z_\alpha^2 {\sf e}^2}\left( \fr{\varepsilon_0 k 
T_{e\infty}}{n_{e\infty} {\sf e}^2}\right)^{1/2}\,;\\
g_\alpha (\vec{r},\vec{v},t)=\sum\limits_{b=i,e}\left( \fr{Z_b}{Z_\alpha}\right) \int f_b 
(\vec{r},{\vec{v}}^{\,\prime},t)\left\vert \vec{v}-{\vec{v}}^{\,\prime}\right\vert\,d\vec{v}^{\,\prime}\,;\\
h_\alpha (\vec{r},\vec{v},t)=\sum\limits_{b=i,e} \fr{m_\alpha+m_b}{m_b} 
\left(\fr{Z_b}{Z_\alpha}\right)
\int
\fr{f_b(\vec{r},{\vec{v}}^{\,\prime}, t)}{\vert \vec{v}-{\vec{v}}^{\,\prime}\vert}
\,d{\vec{v}}^{\,\prime}\,;\\
Z_\alpha =1\,, \quad \alpha=i,e\,.
\end{gather*}
 
К системе уравнений~(\ref{e1-k}) необходимо добавить начальные и краевые условия:
\begin{equation}
\!\left.
\begin{array}{rrl}
t=0:\ & f_\alpha(\vec{r},\vec{v},0)&=f_\alpha^{\mathrm{maksv}}\,,\enskip \alpha=i,e;\\[9pt]
\vec{r}\in \Omega_p:\ & f_\alpha(\vec{r},\vec{v},t)\big\vert_{\vec{r}\in\Omega_p}&=0\,,\enskip \alpha=i,e\,;\\[9pt]
&\varphi(\vec{r},t)\big\vert_{\vec{r}\in\Omega_p}&=\varphi_p\,;\\[9pt]
\vec{r}\in\Omega_\infty:\ & 
f_\alpha(\vec{r},\vec{v},t)\big\vert_{\vec{r}\in\Omega_\infty}&= %{}\\[9pt]
f_\alpha^{\mathrm{maksv}}\,,\enskip \alpha=i,e\,;\\[9pt]
&\varphi(\vec{r},t)\big\vert_{\vec{r}\in\Omega_\infty}&=0\,,
\end{array}\!\!
\right\}\!\!\!\!
\label{e2-k}
\end{equation}
    где 
    
    \noindent
    \begin{multline*}
    f_\alpha^{\mathrm{maksv}}=n_{\alpha\infty}\left(\fr{m_\alpha}{2k\pi T_{\alpha\infty}}\right)^{3/2}\times{}\\
    {}\times
    \exp\left( -
\fr{m_\alpha}{2kT_{\alpha\infty}}\left\vert\vec{v}-\vec{v}_\infty\right\vert^2\right)\,,
\enskip \alpha=i, e\,;
\end{multline*} 
$\Omega_p$ и $\Omega_\infty$~--- множество радиус-векторов час\-тиц, концы которых принадлежат плоскости зонда и 
границе возмущенной зоны соответственно.

Для решения поставленной задачи введем декартову систему координат таким образом, чтобы 
заряженная плоскость совпала с плоскостью~$0xz$. Тогда положение частицы в пространстве будет 
определяться координатами $x,y,z$, а скорость~--- координатами $v_x, v_y, v_z$. В~силу того что 
плоскость является бесконечно большой в сравнении с характерным размером задачи, функции 
распределения частиц будут зависеть только от переменных $y, v_y, t$.

Поставленную задачу предлагается решать независимо двумя методами. Первый метод основывается на 
методе статистических испытаний Мон\-те-Кар\-ло, второй метод является композицией метода 
расщепления и метода крупных частиц.

\section{Применение метода Монте-Карло}

Запишем самосогласованную систему уравнений~(\ref{e1-k}) и~(\ref{e2-k}) в декартовой системе 
координат с учетом сделанных предположений:
\begin{equation}
\left.
\begin{array}{l}
\fr{\partial f_\alpha}{\partial t}+
v_y\fr{\partial f_\alpha}{\partial y}+\fr{F_y^\alpha}{m_\alpha}\,\fr{\partial 
f_\alpha}{\partial v_y}=\fr{1}{2}\,\fr{\partial^2 }{\partial [v_y]^2}\times{}\\
{}\times \left( 
f_\alpha\fr{\partial^2 g_\alpha  }{\partial [v_y]^2}\right) -
\fr{\partial}{\partial v_y}\left( f_\alpha\fr{\partial h_\alpha}{\partial v_y}\right)\,,
\enskip \alpha=i,e\,;\\[6pt]
    \fr{\partial^2\varphi}{\partial y^2} =-\fr{{\sf e}}{\varepsilon_0}\left(n_i-n_e\right)\,;
    \enskip E_y=-
\fr{\partial\varphi}{\partial y}\,;\\[6pt]
\hspace*{3.1mm}    t=0:\  \hspace*{2.6mm}f_\alpha(y,v_y,0)=f_\alpha^{\mathrm{maksv}}\,,\ \alpha=i,e\,;\\[9pt]
\hspace*{2.9mm} y=0:\ \hspace*{2.8mm}f_\alpha(0,v_y,t)=0\,,\ \alpha=i,e\,;\\[9pt]
\hspace*{24.3mm}\varphi(0,t)=\varphi_p\,;\\[9pt]
y=y_\infty:\ f_\alpha(y_\infty, v_y, t)=f_\alpha^{\mathrm{maksv}}\,,\ \alpha=i,e\,;\\[9pt]
\hspace*{21.5mm}\varphi(y_\infty, t)=0\,.
\end{array}
\right \}
\label{e3-k}
\end{equation}

В полученной системе уравнений~(\ref{e3-k}) перейдем к безразмерным величинам, применив 
соотношение $X=M_X \hat{X}$, где $M_X$~--- масштаб размерной величины~$X$, $\hat{X}$~--- 
безразмерная величина~$X$. В~качестве используемых масштабов были взяты следующие: радиус 
Дебая, скорость теплового движения частиц, концентрация частиц в невозмущенной плазме, потенциал, 
возникающий при разделении зарядов в дебаевской сфере, и производные от них величины.

Система безразмерных уравнений имеет следующий вид:
%\noindent
\begin{equation}
\left.
\begin{array}{l}
\fr{\partial 
\hat{f}_\alpha}{\partial\hat{t}}+A_\alpha\fr{\partial\hat{f}_\alpha}{\partial\hat{y}}+
B_\alpha\hat{E}_y\fr{\partial\hat{f}_\alpha}{\partial \hat{v}_y}={}\\
\!{}=
\fr{\partial^2}{\partial[\hat{v}_y]^2}\left(D_\alpha 
\hat{f}_\alpha\right)-\fr{\partial}{\partial\hat{v}_y}\left(K_\alpha \hat{f}_\alpha\right),\enskip 
\alpha=i,e;\\[9pt]
\fr{\partial^2\hat{\varphi}}{\partial\hat{y}^2}=-\left(\hat{n}_i-\hat{n}_e\right)\,;\enskip \hat{e}_y=-
\fr{\partial\hat\varphi}{\partial\hat{y}}\,;\\[9pt]
\hspace*{3.1mm}\hat{t}=0:\ \hspace*{2.6mm}\hat{f}_\alpha(\hat{y},\hat{v}_y,0)=\hat{f}_\alpha^{\mathrm{maksv}}\,,\enskip \alpha-i,e\,;\\[9pt]
\hspace*{2.9mm}\hat{y}=0:\ \hspace*{2.8mm}\hat{f}_\alpha(0,\hat{v}_y,\hat{t})=0\,,\enskip \alpha=i,e\,;\\[9pt]
\hspace*{24.3mm}\hat\varphi(0,\hat{t})=\hat{\varphi}_p\,;\\[9pt]
\hat{y}=\hat{y}_\infty:\ \hat{f}_\alpha(\hat{y}_\infty, \hat{v}_y, \hat{t})=\hat{f}^{\mathrm{maksv}}_\alpha\,,\enskip 
\alpha=i,e\,;\\[9pt]
\hspace*{21.5mm}\hat\varphi(\hat{y}_\infty,\hat{t})=0\,.
\end{array}
\right\}
\label{e4-k}
\end{equation}
Здесь 

\vspace*{-2pt}

\noindent
\begin{gather*}
A_\alpha=\sqrt{\delta_\alpha }\,\hat{v}_y\,;\enskip 
B_\alpha=\sqrt{\delta_\alpha}\,\fr{z_\alpha}{2\varepsilon_\alpha}\,;\\
\delta_\alpha=\fr{\varepsilon_\alpha}{\mu_\alpha}\,;\enskip 
\varepsilon_\alpha=\fr{T_{\alpha\infty}}{T_{i\infty}}\,;\\
\mu_\alpha=\fr{m_\alpha}{m_i}\,;\enskip 
D_\alpha=A_g^\alpha\fr{\partial^2\hat{g}_\alpha}{\partial  [\hat{v}_y]^2}\,;\\
K_\alpha=A_h^\alpha \fr{\partial \hat{h}_\alpha}{\partial \hat{v}_y}\,,\enskip \alpha=i,e\,,
\end{gather*}
где $A_g^\alpha$ и $A_h^\alpha$~--- коэффициенты, определяемые характерными параметрами 
задачи~\cite{15-k}.

Поиск решения самосогласованной системы уравнений~(\ref{e4-k}) осуществляется по следующей 
схе-\linebreak ме. Вначале находятся значения напряженности\linebreak
 электрического поля по значениям потенциала, 
полученным из граничной задачи для уравнения Пуассона. Далее, используя найденные значения 
напряженности, решается уравнение Фок\-ке\-ра--План\-ка путем перехода к стохастическому 
дифференциальному уравнению (СДУ) Ито:

\noindent
\begin{multline*}
d\Theta_\alpha(\hat{t}) = a_\alpha \left(\hat{t},\Theta_\alpha(\hat{t})\right)+{}\\
{}+\sigma\left(
\hat{t},\Theta_\alpha(\hat{t})\right)\,dW(\hat{t})\,,\quad \alpha=i,e\,,
%\label{e5-k}
\end{multline*}
где 

\noindent
\begin{align*}
\Theta_\alpha(\hat{t})&=\begin{bmatrix}
\hat{y}(\hat{t})\\ \hat{v}_y(\hat{t})
\end{bmatrix}\,;\\
a_\alpha\left(\hat{t},\Theta_\alpha(\hat{t})\right)&=\begin{bmatrix}
-A_\alpha\\ -K_\alpha -B_\alpha \hat{E}_y
\end{bmatrix}\,;\\
\sigma_\alpha\left(\hat{t},\Theta_\alpha(\hat{t})\right)\sigma_\alpha^{\mathrm{T}}\left( 
\hat{t},\Theta_\alpha(\hat{t})\right)&=D_\alpha\,,\enskip \alpha=i,e\,;
\end{align*} 
$W(\hat{t})$~--- стандартный винеровский случайный процесс.
\pagebreak

Для нахождения значений вектора состояния~$\Theta_\alpha(\hat{t})$ применим явную разностную 
схему стохастического метода Эйлера~\cite{16-k}:
\begin{multline*}
\Theta_\alpha^{n+1}=\Theta_\alpha^n +h_\tau a_\alpha \left( \hat{t}_n, \Theta_\alpha^n\right)+\sigma_\alpha 
\left( \hat{t}_n, \Theta_\alpha^n\right)\Delta W_n\,,\\ 
n=0,\ldots , N\,,\ \alpha=i,e\,,
%\label{e6-k}
\end{multline*}
где $\Theta_\alpha^n$, $n=0,\ldots , N$,~--- приближенное значение вектора 
состояния~$\Theta_\alpha(\hat{t})$, $\alpha=i,e$, в момент времени $\hat{t}\hm=\hat{t}_n$, 
$\hat{t}_n\hm=n h_\tau$, $n=0,\ldots , N$; $h_\tau$~--- достаточно малый шаг интегрирования; $\Delta 
W_n$, $n=0,\ldots ,N$,~--- величина приращения винеровского процесса~$W(\hat{t})$ на отрезке $\left[ 
\hat{t}_n,\,\hat{t}_{n+1}\right]$, по определению независимая от~$\Theta_\alpha^0$, 
$\Delta W_0,\ldots , 
\Delta W_{n-1}$: $\Delta W_n\hm=W(\hat{t}_{n-1})\hm-W(\hat{t}_n)$; $\Delta W_n\hm\sim N(0,\,h_\tau)$, 
т.\,е.\ $\Delta W_n$ представляют собой гауссовские случайные величины с нулевыми математическими 
ожиданиями и дисперсиями, равными шагу интегрирования; $\Theta_\alpha^0$~--- значение вектора 
состояния $\Theta_\alpha(\hat{t})$, $\alpha\hm=i,e$, в момент времени $\hat{t}=0$, 
$\Theta_\alpha^0\hm\sim \hat{f}_\alpha^{\mathrm{maksv}}$. 

Частные производные $\partial^2\hat{g}_\alpha/\partial[\hat{v}_y]^2$ и $\partial \hat{h}_\alpha/\partial 
\hat{v}_y$, являющиеся составляющими матрицы $\sigma_\alpha (\hat{t}_n, 
\Theta_\alpha^n)\sigma_\alpha^{\mathrm{T}}(\hat{t}_n,\Theta_\alpha^n)$ и вектора $a_\alpha(\hat{t}_n, 
\Theta_\alpha^n)$ соответственно, аппроксимируются со вторым порядком точности на трехточечном 
шаблоне на основе значений~$\hat{g}_\alpha$ и~$\hat{h}_\alpha$~\cite{17-k}.
      
      В выражения для функций~$\hat{g}_\alpha$ и~$\hat{h}_\alpha$ входят интегралы, которые 
вычисляются методом Мон\-те-Кар\-ло с использованием набора значений скоростной компоненты 
вектора состояния~$\hat{v}_y$, полученных из решения СДУ Ито:
      \begin{equation*}
      \int \hat{f}_\alpha \left\vert \hat{v}_y-
\hat{v}_y^\prime\right\vert\,dv_y^\prime=M\left(\zeta\left(\hat{V}_y\right)\right)\,,
\end{equation*}
где
$$
      \zeta\left(\hat{V}_y\right)=\left\vert \hat{v}_y-\hat{V}_y\right\vert\,,\enskip \hat{V}_y\sim 
\hat{f}_\alpha\,.
  $$
      
      Для вычисления напряженности самосогласованного электрического поля $\hat{E}_y=-
\partial\hat{\varphi}/\partial\hat{y}$, входящей в вектор $a_\alpha(\hat{t}_n, \Theta_\alpha^n)$, необходимо 
аналогично аппроксимировать со вторым порядком точности производную 
$\partial\hat{\varphi}/\partial\hat{y}$ на трехточечном шаблоне с использованием значений 
потенциала~$\hat{\varphi}$~\cite{17-k}. Значения потенциала~$\hat\varphi$ находятся из решения 
уравнения Пуассона. 
      
      Граничную задачу для уравнения Пуассона 
      \begin{align*}
      \fr{\partial^2 \hat\varphi}{\partial \hat{y}^2} & = -\left(\hat{n}_i-\hat{n}_e\right)\,;\\
      \hat{\varphi}\big|_{\hat{y}=0} &=\hat{\varphi}_p\,;\\
      \hat{\varphi}\big|_{\hat{y}_\infty=0} &=0
      \end{align*}
    предлагается решать путем перехода к конечно-разностной системе с последующим ее решением 
методом прогонки~\cite{17-k}:

\noindent
\begin{gather*}
\hat{\varphi}^n_{l-1}+2\hat{\varphi}_l^n+\hat{\varphi}^n_{l+1}=
h_y\hat{\delta}_l^n\,,\enskip l=1,\ldots , 
N_y\,;\\
\hat{\delta}_l^n=-\left( \hat{n}^n_{i,l}-\hat{n}^n_{e,l}\right)\,;\enskip 
\hat{\varphi}_0=\hat{\varphi}_p\,;\enskip \hat{\varphi}_{N_y}=0\,,
\end{gather*}
где $N_y$~--- число шагов по переменной~$\hat{y}$, $h_y$~--- величина шагов разбиения по~$\hat{y}$. 
      
      Концентрации $\hat{n}_\alpha$, $\alpha=i,e$, и плотности токов частиц на зонд~$\hat{f}_\alpha$, 
$\alpha=i,e$, вычисляются согласно описанному выше методу Мон\-те-Карло.

\section{Применение метода расщепления и~метода крупных~частиц}

Решение задачи в данном случае предлагается начать с записи правой части уравнения 
Фок\-ке\-ра--План\-ка в декартовой системе координат в виде:
$$
\mathbf{Q} f_\alpha = \fr{1}{2}\,\fr{\partial^2 f_\alpha}{\partial [v_y]^2}\,\fr{\partial^2 g_\alpha}{\partial 
[v_y]^2}+\fr{\partial f_\alpha}{\partial v_y}\,\fr{\partial C_\alpha}{\partial v_y}+H_\alpha\,,\enskip 
\alpha=i,e\,,
$$  
где 
\begin{align*}
C_\alpha(\vec{r},\vec{v},t)&=
\begin{cases}
\fr{1-\gamma}{Z_i^2}\int\fr{f_e(\vec{r},{\vec{v}}^{\,\prime},t)}{|\vec{v}-{\vec{v}}^{\,\prime} |}\,d{\vec{v}}^{\,\prime}\,, 
&\alpha=i\,;\\[9pt]
\fr{Z_i^2(\gamma-1)}{\gamma}\int \fr{f_i(\vec{r},{\vec{v}}^{\,\prime}, t)}
{|\vec{v}-{\vec{v}}^{\,\prime} 
|}\,d{\vec{v}}^{\,\prime}\,, &\alpha=e\,;
\end{cases} 
\\
H_\alpha&=
\begin{cases}
4\pi \left( \fr{\gamma f_e}{Z_i^2}+f_i\right)f_i\,, & \alpha=i\,;\\[9pt]
4\pi\left(\fr{Z_i^2 f_i}{\gamma}+f_e\right)f_e\,, &\alpha=e\,.
\end{cases}
\end{align*}
Тогда при переходе к безразмерным величинам (см.\ разд.~3) система~(\ref{e1-k}) запишется 
следующим образом:
      \begin{equation}
      \left.
\!\!\begin{array}{l}
      \fr{\partial 
\hat{f}_\alpha}{\partial\hat{t}}+A_\alpha\fr{\partial\hat{f}_\alpha}{\partial\hat{y}}+
B_\alpha  \hat{E}_y
\fr{\partial\hat{f}_\alpha}{\partial\hat{v}_\alpha}=\tilde{\mathbf{Q}}\hat{f}_\alpha\,,\enskip 
\alpha=i,e;\\[9pt]
      \fr{\partial^2\hat{\varphi}}{\partial\hat{y}^2}=-\left( \hat{n}_i-\hat{n}_e\right)\,,\enskip \hat{E}_y=-
\fr{\partial\hat\varphi}{\partial\hat{y}}\,,\\[9pt]
\hspace*{3.1mm}\hat{t}=0:\ \hspace*{2.6mm}\hat{f}_\alpha(\hat{y},\hat{v}_y, 0)=\hat{f}_\alpha^{\mathrm{maksv}}\,,\enskip \alpha=i,e\,,\\[9pt]
\hspace*{2.9mm} \hat{y}=0:\ \hspace*{2.8mm}\hat{f}_\alpha(0,\hat{v}_y,\hat{t})=0\,,\enskip \alpha=i,e\,;\\[9pt]
\hspace*{24.3mm}\hat\varphi(0,\hat{t})=\hat{\varphi}_p\,;\\[9pt]
      \hat{y}=\hat{y}_\infty:\ \hat{f}_\alpha(\hat{y}_\infty, 
\hat{v}_y,\hat{t})=\hat{f}_\alpha^{\mathrm{maksv}}\,,\enskip \alpha=i,e\,;\\[9pt]
\hspace*{21.5mm}\hat{\varphi}(\hat{y}_\infty,\hat{t})=0\,,\\[9pt]
    \end{array}
\right\}\!\!
\label{e7-k}
\end{equation}
где 
\begin{gather*}
\tilde{\mathbf{Q}} \hat{f}_\alpha=D_\alpha\fr{\partial^2\hat{f}_\alpha}{\partial 
[\hat{v}_y]^2}+K_\alpha\fr{\partial\hat{f}_\alpha}{\partial\hat{v}_y}+H_\alpha\,;\\
D_\alpha=A_g^\alpha\fr{\partial^2\hat{g}_\alpha}{\partial [\hat{v}_y]^2}\,;\enskip 
K_\alpha=A_h^\alpha \fr{\partial \hat{h}_\alpha}{\partial\hat{v}_y}\,,\ \alpha=i,e\,.
\end{gather*}

Для решения системы уравнений~(\ref{e7-k}) применяется модификация метода 
расщепления~\cite{17-k}, согласно которой исходная задача разбивается на две вспомогательные. Такое 
разбиение можно осуществить, переписав уравнение Фок\-ке\-ра--План\-ка в следующем виде:
$$
\fr{\partial\hat{f}_\alpha}{\partial\hat{t}} =
\tilde{\mathbf{Q}}_1\hat{f}_\alpha+\tilde{\mathbf{Q}}_2\hat{f}_\alpha\,,
$$
где 
\begin{align*}
\tilde{\mathbf{Q}}_1\hat{f}_\alpha &=-
\left(A_\alpha\fr{\partial\hat{f}_\alpha}{\partial\hat{y}}+
B_\alpha\fr{\partial\hat{f}_\alpha}{\partial\hat{y}}
\right)\,;\\
\tilde{\mathbf{Q}}_2\hat{f}_\alpha 
&=\left(D_\alpha\fr{\partial^2\hat{f}_\alpha}{\partial[\hat{v}_y]^2}+K_\alpha\fr{\partial 
\hat{f}_\alpha}{\partial\hat{v}_y}+H_\alpha\right)\,.
\end{align*}

      Правая часть уравнения Фок\-ке\-ра--План\-ка представляет собой сумму двух операторов, 
первый из которых отвечает за перенос частиц, второй~--- за столкновения заряженных частиц. 
В~результате образуются следующие задачи, которые решаются последовательно:
      \begin{itemize}
\item первая задача:
\begin{align*}
&\fr{\partial w_\alpha(\hat{y},\hat{v}_y,\hat{t})}{\partial\hat{t}} =\mathbf{Q}_1 
w_\alpha(\hat{y},\hat{v}_y,\hat{t})\,,\enskip \alpha=i,e\,;\\[9pt]
&\fr{\partial^2\hat\varphi}{\partial\hat{y}^2}=-\left(\hat{n}_i-\hat{n}_e\right)\,;\enskip
\hat{E}_y=-
\fr{\partial\hat\varphi}{\partial\hat{y}}\,;\\[9pt]
&w_\alpha(\hat{y},\hat{v}_y,\hat{t}^n)=\hat{f}_\alpha(\hat{y},\hat{v}_y,\hat{t}^n)\,,\enskip n=0,\ldots ,N-
1\,;\\[9pt]
&\hspace{2.9mm}\hat{y}=0:\ \hspace*{2.9mm}w_\alpha(0,\hat{v}_y,\hat{t})=0\,,\enskip \alpha=i,e\,;\\[9pt]
&\hspace*{25.1mm}\hat\varphi(0,\hat{t})=\hat{\varphi}_p\,;\\[9pt]
&\hat{y}=\hat{y}_\infty:\ w_\alpha(\hat{y}_\infty, \hat{v}_y, \hat{t})=
\hat{f}_\alpha^{\mathrm{maksv}}\,,\enskip 
\alpha=i,e\,;\\[9pt]
&\hspace*{22.5mm}\hat\varphi(\hat{y}_\infty,\hat{t})=0\,;
\end{align*}
\item вторая задача:
\begin{align*}
\!\!\!\!\!\!\!\fr{\partial s_\alpha(\hat{y},\hat{v}_y,\hat{t})}{\partial \hat{t}} &=\mathbf{Q}_2 
s_\alpha(\hat{y},\hat{v}_y,\hat{t})\,, & \alpha&=i,e\,;\\
\!\!\!\!\!\!\!s_\alpha (\hat{y},\hat{v}_y,\hat{t}^n) &=w_\alpha (\hat{y},\hat{v}_y, \hat{t}^{n+1}),& n&=0,\ldots ,N-
1.
\end{align*}
\end{itemize}

Первая задача представляет собой систему безразмерных уравнений Вла\-со\-ва--Пуас\-со\-на. Для ее 
решения применяется метод крупных частиц~\cite{18-k}. Согласно этому методу решение задачи 
осуществляется путем расщепления на два этапа: на первом этапе не учитываются конвективные члены 
и решение получается обычным интегрированием на неподвижной эйлеровой сетке, а на втором этапе 
рассматривается система, которая описывает перенос частиц в лагранжевой системе координат. Кроме 
того, на первом этапе необходимо решить уравнение Пуассона для получения значений потенциала 
самосогласованного электрического поля. Для этого применяется метод, описанный в разд.~3. 

Вторая задача решается путем перехода к ко\-неч\-но-раз\-ност\-ной сис\-те\-ме. При этом частные 
производные $\partial^2\hat{g}_\alpha/\partial[\hat{v}_y]^2$ и $\partial\hat{h}_\alpha/\partial\hat{v}_y$ 
аппроксимируются со вторым порядком точности с использованием трехточечного шаблона, а 
производная $\partial s_\alpha/\partial\hat{t}$ аппроксимируется на двухточечном шаблоне с первым 
порядком точности~\cite{16-k}. К~полученной системе разностных уравнений предлагается применить 
один из классических методов решения систем линейных уравнений, например метод 
Гаусса~\cite{19-k}.
      
      Решением первой задачи является функция $w_\alpha(\hat{y}, \hat{v}_y, \hat{t}^n)$, 
$n\hm=0,\ldots ,N$, , которая дает начальное условие для второй задачи. Решая вторую задачу, находим 
функцию $s_\alpha(\hat{y},\hat{v}_y,\hat{t}^n)\hm=\hat{f}_\alpha(\hat{y},\hat{v}_y,\hat{t}^n)$, 
$n=1,\ldots ,N$, $\alpha=i,e$, которая определяет решение $\hat{f}_\alpha(\hat{y},\hat{v}_y,\hat{t}^n)$, 
$\alpha=i,e$, исходной системы~(\ref{e7-k}) для рассматриваемых моментов времени $n=1,\ldots ,N$.

Моменты функций распределения $\hat{f}_\alpha$, $\alpha=i,e$, находятся с помощью методов 
численного интегрирования, например метода трапеций~\cite{19-k}.

\section{Результаты численного моделирования}

Для двух описанных выше методов реализованы две отдельные программы в среде {Matlab~7.0}. 
Эти программы позволяют по заданным значениям концентраций и температур частиц $n_{i\infty}$, 
$n_{e\infty}$, $T_{i\infty}$ и~$T_{e\infty}$ в невозмущенной плазме, а также потенциала~$\varphi_p$, 
подаваемого на зонд, изучить эволюцию во времени плотностей тока частиц~$j_i$ и~$j_e$, концентраций 
частиц~$n_i$  и~$n_e$ в произвольной точке пространства в возмущенной зоне, а также динамику 
изменения напряженности~$E_y$ самосогласованного электрического поля во времени и пространстве.

С использованием разработанных программ проведены серии расчетных экспериментов, в которых 
значение концентраций варьировалось в пределах $n_{i\infty} \hm = n_{e\infty}\hm =10^{18}\div 
10^{22}$~м$^{-3}$. Значение температур было выбрано неизменным и равным $T_{i\infty}\hm = 
T_{e\infty}\hm=3000$~K, а значения потенциала, подаваемого на зонд, изменялись в пределах 
$\varphi_p\hm=0\div 2{,}6$~В.

На рис.~1  и~2 приведены графики изменения напряженности самосогласованного электрического
 поля (см.\ рис.~1) и плотности токов ионов (см.\linebreak\vspace*{-12pt}

\pagebreak

\end{multicols}

\begin{figure} %fig1
\vspace*{1pt}
\begin{center}
\mbox{%
\epsfxsize=162.594mm
\epsfbox{kud-1.eps}
}
\end{center}
\vspace*{-9pt}
\Caption{Динамика изменения плотности тока ионов во времени в фиксированной точке возмущенной 
зоны для значений потенциала: \textit{1}~--- $\varphi_p=-6$; 
\textit{2}~--- $\varphi_p=-16$; \textit{3}~--- $\varphi_p=- 30$ 
в случае применения методов Монте-Карло~(\textit{а}) 
и крупных частиц~(\textit{б})}
\end{figure}

\begin{figure} %fig2
\vspace*{1pt}
\begin{center}
\mbox{%
\epsfxsize=162.713mm
\epsfbox{kud-2.eps}
}
\end{center}
\vspace*{-9pt}
\Caption{Динамика изменения напряженности электрического поля во времени в фиксированной точке 
возмущенной зоны для значений потенциала: 
\textit{1}~--- $\varphi_p=-6$; \textit{2}~--- $\varphi_p=-16$; 
\textit{3}~--- $\varphi_p=-30$ в случае применения методов Монте-Карло~(\textit{а}) и
крупных частиц~(\textit{б})
}
\end{figure}

\begin{multicols}{2}

\noindent
 рис.~2) во времени в фиксированной точке пространства 
возмущенной зоны в случае применения обоих разработанных алгоритмов.


На основании полученных результатов можно отметить похожее поведение зависимостей 
напряженности электрического поля и плотности тока от времени в двух рассматриваемых случаях. 
Графики кривых сначала убывают, затем начинают возрастать, выходя в некоторый момент 
времени~$t^\prime$ (момент установления) на стационарные значения. 

Одинаковое поведение 
напряженности и плот\-ности тока можно объяснить из следующих соображений: плотность тока ионов в 
данной области пространства равна произведению концентрации ионов на их направленную скорость и 
на заряд иона. Скорость ионов, в свою очередь, зависит от заряда, массы и напряженности 
электрического поля. 
%\columnbreak

При внесении в плазму отрицательно заряженного зонда возникает электрическое поле, которое 
нарушает квазинейтральность плазмы. Для того чтобы компенсировать действие внешнего 
электрического поля, ионы устремляются к зонду, а электроны~--- от зонда. Это приводит к дисбалансу 
концентраций вблизи зонда и, как следствие, к увеличению разности потенциалов; график 
напряженности электрического поля убывает. Вскоре разделение зарядов компенсирует внешнее 
электрическое поле; график выходит на стационарное значение. 

Также можно отметить, что значения 
напряженности электрического поля и плотности тока частиц на зонд в момент установления для двух 
методов совпадают. 

Момент установления~$t^\prime$ зависит от при\-ме\-ня\-емо\-го метода решения. В~случае метода 
Мон\-те-Кар\-ло $t^\prime=3{,}5\div 4$~ед., а для метода крупных частиц совместно с методом 
расщепления $t^\prime\hm=5\div 5{,}5$~ед. Используя ко\-неч\-но-раз\-ност\-ный метод, можно 
получить динамику изменения функций распределения частиц~$f_\alpha$, $\alpha=i,e$, во времени и 
пространстве. Функции распределения позволяют наглядно представить влияние на картину 
распределения частиц вблизи зонда самой поверхности зонда и электрического поля.

\section{Заключение}
      
      В работе найдено решение задачи диагностики плоским зондом сильноионизованной плазмы с 
учетом столкновений заряженных частиц. Разработана математическая модель исследуемого явления, 
описываемая уравнениями Фок\-ке\-ра--План\-ка и Пуассона. Решение получено двумя методами:\linebreak 
статистическим и ко\-неч\-но-раз\-ност\-ным на основе\linebreak сформированных алгоритмов. Приведены 
резуль-\linebreak таты численного моделирования при различных\linebreak характерных параметрах задачи.
 Из  проведенных 
вычислительных экспериментов вытекает, что искомые величины: напряженность 
электрического поля, плотности токов частиц на зонд, концентрации частиц вблизи зонда~--- как по 
характеру зависимости, так и по числовым значениям совпадают. При применении метода 
      Мон\-те-Кар\-ло момент установления наступает быстрее по сравнению с конечно-разностным 
методом, однако конечно-разностный метод позволяет получить более наглядные результаты.

{\small\frenchspacing
{%\baselineskip=10.8pt
\addcontentsline{toc}{section}{Литература}
\begin{thebibliography}{99}

\bibitem{1-k}
\Au{Alexeff I., Anderson T.}
Experimental and theoretical results with plasma antenna~// IEEE Trans. Plasma Sci., 2006. Vol.~34. 
No.\,2. P.~166--172.

\bibitem{2-k}
\Au{Сысун В.\,И.}
Сильноионизованная низкотемпературная плазма в приборах электронной техники: Методы 
исследования, свойства, применение. Дисс. \ldots д-ра физ.-мат. наук в форме науч. докл.: 
01.04.08.~--- Пет\-ро\-за\-водск, 1996.

\bibitem{3-k}
\Au{Тухас В.\,А.}
Методология создания средств измерений и испытаний на устойчивость к кондуктивным помехам~// 
Мат-лы VI Междунар. симп. по электромагнитной совместимости и 
электромагнитной экологии.~--- СПб., 2005. С.~231--234.

\bibitem{4-k}
\Au{Гудзенко Л.\,И., Яковленко С.\,И.}
Плазменные лазеры.~--- М.: Атомиздат, 1978.  256~с.

\bibitem{5-k}
\Au{Звелто О.}
Принципы лазеров.~--- М.: Мир, 1990.  560~с.

\bibitem{6-k}
\Au{Сысун В.\,И., Хромой Ю.\,Д.}
Расширение канала мощного импульсного разряда в парах ртути~// Электронная техника, 1974. 
Сер.~4. Вып.~10. С.~80--85. 

\bibitem{7-k}
\Au{Винклер Дж.\,Р.}
Искусственные пучки частиц в космической плазме.~--- М.: Мир, 1985.  451~с.

\bibitem{8-k}
\Au{Bernstein I.\,B., Rabinowitz I.\,N.}
Theory of electrostatic probes in low-density plasma~// Phys. Fluids, 1959. Vol.~2. No.\,2. P.~112--121. 

\bibitem{9-k}
\Au{Альперт Я.\,Л., Гуревич А.\,В., Питаевский~Л.\,П.}
Искусственные спутники в разреженной плазме.~--- М.: Наука, 1964.  282~с.

\bibitem{10-k}
\Au{Чан П., Тэлбот Л., Турян~К.}
Электрические зонды в неподвижной и движущейся плазме.~--- М.: Мир, 1978.  202~с.

\bibitem{11-k}
\Au{Алексеев Б.\,В., Котельников В.\,А.}
Зондовый метод диагностики плазмы.~--- М.: Энергоатомиздат, 1989.  240~с.

\bibitem{12-k}
\Au{Пантелеев А.\,В., Кудрявцева И.\,А.}
Формирование математической модели двухкомпонентной плазмы с учетом столкновений 
заряженных частиц в случае плоского зонда~// Теоретические вопросы вычислительной техники и 
программного обеспечения: Межвузовский сб. научн. тр.~--- М.: МИРЭА, 2006. С.~11--21.

\bibitem{13-k}
\Au{Олдер Б.}
Вычислительные методы в физике плазмы.~--- М.: Мир, 1974.  111~с.

\bibitem{14-k}
\Au{Montgomery D.\,C., Tidman D.\,A.}
Plasma kinetic theory.~--- New York, 1964. 

\bibitem{15-k}
\Au{Кудрявцева И.\,А., Пантелеев А.\,В.}
Применение метода Мон\-те-Кар\-ло для анализа поведения двухкомпонентной плазмы с учетом 
столкновений между заряженными частицами~// Теоретические вопросы\linebreak
вычислительной техники и 
программного обеспечения: Межвузовский сб. научн. тр.~--- М.: МИРЭА, 2008. С.~122--128. 

\bibitem{16-k}
\Au{Семенов В.\,В., Пантелеев А.\,В., Руденко~Е.\,А., Бор\-та\-ков\-ский~А.\,С.}
Методы описания, анализа и синтеза нелинейных систем управления.~--- М.: МАИ, 1993.  312~с.

\bibitem{17-k}
\Au{Киреев В.\,И., Пантелеев А.\,В.}
Численные методы в примерах и задачах.~--- М.: Высшая школа, 2006.  480~с.

\bibitem{18-k}
\Au{Белоцерковский О.\,М., Давыдов~Ю.\,М.}
Метод крупных частиц в газовой динамике. Вычислительный эксперимент.~--- М.: Наука, 
Физматгиз, 1982.

\label{end\stat}

\bibitem{19-k}
\Au{Вержбицкий В.\,М.}
Основы численных методов.~--- М.: Высшая школа, 2002.  840~с.
 \end{thebibliography}
}
}


\end{multicols}       