\def\stat{shestakov}

\def\tit{О СКОРОСТИ СХОДИМОСТИ РАСПРЕДЕЛЕНИЯ
ВЫБОРОЧНОГО АБСОЛЮТНОГО МЕДИАННОГО ОТКЛОНЕНИЯ
К~НОРМАЛЬНОМУ ЗАКОНУ$^*$}

\def\titkol{О скорости сходимости распределения
выборочного абсолютного медианного отклонения
к~нормальному закону}

\def\autkol{О.\,В.~Шестаков}
\def\aut{О.\,В.~Шестаков$^1$}

\titel{\tit}{\aut}{\autkol}{\titkol}

{\renewcommand{\thefootnote}{\fnsymbol{footnote}}\footnotetext[1]
{Работа выполнена при финансовой поддержке РФФИ (гранты 11-01-00515а и 11-01-12-26-офи-м).}}

\renewcommand{\thefootnote}{\arabic{footnote}}
\footnotetext[1]{Московский государственный
университет им.\ М.\,В.~Ломоносова, кафедра математической статистики 
факультета вычислительной математики и кибернетики; Институт проблем информатики
Российской академии наук,  oshestakov@cs.msu.su}

\vspace*{-9pt}


\Abst{Получены оценки скорости сходимости распределения выборочного абсолютного 
медианного отклонения к нормальному закону в общем и симметричном случае.}

\vspace*{-3pt}

\KW{порядковые статистики; выборочная медиана; абсолютное медианное отклонение; нормальное распределение; оценка скорости сходимости}

  \vskip 8pt plus 9pt minus 6pt

      \thispagestyle{headings}

      \begin{multicols}{2}
      
            \label{st\stat}

\section{Введение}

Абсолютное медианное отклонение от медианы представляет собой меру разброса, 
которую можно использовать, например, в тех случаях, когда дисперсия не существует. 
Для выборочного абсолютного медианного отклонения получены результаты, касающиеся 
сходимости почти всюду (п.\,в.) и асимптотической совместной нормальности выборочного 
абсолютного медианного отклонения и выборочной медианы, а также экспоненциальные 
оценки вероятностей больших отклонений и разложение Бахадура~[1--4]. 
В~данной работе будут получены некоторые оценки скорости сходимости выборочного 
абсолютного медианного отклонения к нормальному закону.

Введем необходимые обозначения. Пусть случайная величина~$X$ имеет функцию распределения~$F$. 
Медиана~$F$ определяется как $m^*\hm=F^{-1}(1/2)\hm=\inf\{x:F(x)\hm\geqslant1/2\}$. 
Если функция~$F$ непрерывна в точке $m^*$, то $F(m^*)=1/2$. Медиану функции распределения~$H$ 
случайной величины $\abs{X-m^*}$ (т.\,е. $H(y)\hm=F(m^*+y)\hm-F(m^*-y-)$) 
обозначим через~$M^*$. Величина~$M^*$ называется абсолютным медианным отклонением 
от медианы функции распределения~$F$.

Выборочные аналоги медианы и абсолютного медианного отклонения определяются следующим образом. 
Пусть $\{X_1,\ldots,X_n\}$~--- независимая выборка из распределения~$F$, 
а $\{X_{(1)},\ldots,X_{(n)}\}$~--- соответствующий ей ряд из порядковых статистик. 
Тогда выборочная медиана равна
$$
\mathrm{med}_n=\fr{1}{2}\left(X_{\left(\lfloor\frac{n+1}{2}\rfloor\right)}+
X_{\left(\lfloor\frac{n+2}{2}\rfloor\right)}\right)\,.
$$
Обозначим через $\{Z_{(1)},\ldots,Z_{(n)}\}$ ряд из упорядоченных значений 
$Z_i=\abs{X_i-\mathrm{med}_n}$, $1\leqslant i\leqslant n$. То-\linebreak\vspace*{-12pt}
\columnbreak

\noindent
гда выборочное абсолютное 
медианное отклонение равно

\noindent
$$
\mathrm{MAD}_n=\fr{1}{2}\,\left(Z_{\left(\lfloor\frac{n+1}{2}\rfloor\right)}+
Z_{\left(\lfloor\frac{n+2}{2}\rfloor\right)}\right)\,.
$$
Преимуществом med$_n$ и MAD$_n$ как мер центра и разброса значений
случайной величины является их робастность, т.\,е.\ устойчивость по
отношению к выбросам. При определенных условиях имеет место
сходимость~med$_n$ и~MAD$_n$ п.\,в.\ к~$m^*$ и~$M^*$
соответственно~\cite{1-sh, 3-sh}. Кроме того, довольно слабые ограничения на
регулярность~$F$ обеспечивают асимптотическую совместную
нормальность~med$_n$ и MAD$_n$~\cite{2-sh}. В~работах~\cite{3-sh, 5-sh} получены
экспоненциальные неравенства для вероятностей больших уклонений
med$_n$ и MAD$_n$ от~$m^*$ и~$M^*$, а недавно получен сильный (п.\,в.) 
вариант разложения Бахадура для MAD$_n$~\cite{4-sh}. В~данной работе
приведены оценки скорости сходимости выборочного абсолютного
медианного отклонения к нормальному закону. Причем в случае
симметричного распределения~$F$ оценка имеет лучший порядок, чем в
общем случае. Для простоты изложения будем рассматривать вместо
med$_n$ и MAD$_n$ выборочные квантили, т.\,е.\ величины~$\hat{m}_n$ и~$\hat{M}_n$, 
которые определяются как
$\hat{m}_n\hm=\hat{F}_n^{-1}(1/2)\hm=\inf\{x:\hat{F}_n(x)\hm\geqslant1/2\}$ и
$\hat{M}_n\hm=\hat{H}_n^{-1}(1/2)\hm=\inf\{y:\hat{H}_n(y)\hm\geqslant1/2\}$,
где $\hat{F}_n$~--- выборочная функция распределения, соответствующая
распределению~$F$, а
$\hat{H}_n(y)\hm=\hat{F}_n(\hat{m}_n+y)\hm-\hat{F}_n(\hat{m}_n-y)$. Это
позволяет рассматривать в определениях выборочной медианы и
выборочного абсолютного медианного отклонения одну порядковую
статистику вместо полусуммы порядковых статистик. Можно показать,
что все результаты данной работы останутся справедливыми и для
величин~med$_n$ и~MAD$_n$.

\pagebreak

\section{Оценка скорости сходимости абсолютного медианного отклонения к~нормальному закону}

Оценки скорости сходимости выборочных квантилей (включая выборочную медиану) 
к нормальному закону известны довольно давно (см., например,~\cite{5-sh, 7-sh}). 
Используя разложение Бахадура~\cite{6-sh}, полученное в работе~\cite{4-sh}, 
можно получить оценку скорости сходимости соот\-вет\-ст\-ву\-ющим образом центрированного 
и нормированного выборочного абсолютного медианного отклонения к нормальному закону.

\medskip

\noindent
\textbf{Теорема 1.} \textit{Пусть функция распределения~$F$ в окрестности точек~$m^*$ 
и~$m^*\pm M^*$ имеет непрерывную положительную плотность~$f$ и ограниченную вторую 
производную~$F''$. Тогда существует такая константа~$C^*$, что начиная с некоторого~$n$
\begin{multline}
\!\!\!\sup\limits_{t\in\mathbf{R}}\left\vert\mathrm{P}\left(
\fr{2h(M^*)f(m^*)}{(f^2(m^*)+g_F(M^*))^{1/2}}\,
n^{1/2}(\hat{M}_n-M^*)\leqslant{}\right.\right.\\
\left.\left.{}\leqslant t
\vphantom{\fr{M^*}{g_F(M^*)^{1/2}}}
\right)-\Phi(t)
\vphantom{\fr{2h(M^*)f(m^*)}{(f^2(m^*)+g_F(M^*))^{1/2}}}
\right\vert\leqslant C^*n^{-1/4}(\ln n)^{3/4}\,,\label{e1-sh}
\end{multline}
где $g_F(M^*)\hm=[f(m^*-M^*)\hm-f(m^*+M^*)]^2\hm-4[1-F(m^*-M^*)\hm-F(m^*+M^*)]f(m^*)$, 
а $\Phi(t)$~--- функция распределения стандартного нормального закона.}

\medskip

\noindent
Д\,о\,к\,а\,з\,а\,т\,е\,л\,ь\,с\,т\,в\,о\,.\ В~работе~\cite{4-sh} 
получено сле\-ду\-ющее представление (разложение Бахадура) для $\hat{M}_n$:
$$
\hat{M}_n-M^*\overset{\mbox{\tiny п.\,в.}}{=}Y_n+R_n\,,
$$
где
\begin{multline*}
Y_n=\fr{1/2-[\hat{F}_n(m^*+M^*)-\hat{F}_n(m^*-M^*)]}{h(M^*)}-{}\\
{}-\fr{f(m^*-M^*)-f(m^*+M^*)}{h(M^*)}\,\fr{1/2-\hat{F}_n(m^*)}{f(m^*)}\,,
\end{multline*}
а $R_n$~--- остаточный член, о котором будет сказано ниже. Величина~$Y_n$ 
представляет собой среднее арифметическое независимых одинаково распределенных 
случайных величин, причем математическое ожидание~$Y_n$ равно нулю, а дисперсия 
равна~$\sigma^2/n$, где
$$
\sigma^{2}=\fr{f^2(m^*)+g_F(M^*)}{4[h(M^*)f(m^*)]^{2}}\,.
$$
Воспользуемся неравенством
\begin{multline}
\sup\limits_{t\in\mathbf{R}}\abs{\mbox{P}\left(\fr{n^{1/2}}{\sigma}(\hat{M}_n-M^*)\leqslant t\right)-\Phi(t)}\leqslant{}\\
{}\leqslant\sup\limits_{t\in\mathbf{R}}\abs{\mbox{P}\left(\fr{n^{1/2}}{\sigma}Y_n\leqslant t\right)
-\Phi(t)}+\fr{\eps_n}{\sqrt{2\pi}}+{}\\
{}+\mbox{P}\left(\abs{n^{1/2}R_n}>\sigma\eps_n\right)\,,
\label{e2-sh}
\end{multline}
где $\eps_n=Bn^{-1/4}(\ln n)^{3/4}$ с некоторой константой~$B$, 
ограничения на которую будут наложены ниже. Оценим 
$\mbox{P}\left(\abs{n^{1/2}R_n}\hm>\sigma\eps_n\right)$. 
Величина~$R_n$ имеет следующую структуру~\cite{4-sh}:
$$
R_n=\fr{\Delta_n^{(1)}-\Delta_n^{(2)}+\Delta_n^{(3)}+\Delta_n^{(4)}-\Delta_n^{(5)}+\Delta_n^{(6)}}{h(M^*)}\,,
$$ 
где
\begin{align*}
\Delta_n^{(1)}&=\hat{F}_n(\hat{m}_n+\hat{M}_n)-\hat{F}_n(\hat{m}_n-\hat{M}_n)-\fr{1}{2}\,;\\
\Delta_n^{(2,3)}&=F(\hat{m}_n\pm\hat{M}_n)-F(m^*\pm M^*)-{}\\
&\hspace*{3mm}{}-f(m^*\pm M^*)[\hat{m}_n\pm\hat{M}_n-(m^*\pm M^*)]\,;\\
\Delta_n^{(4,5)}&=F(\hat{m}_n\pm\hat{M}_n)-F(m^*\pm M^*)-{}\\
&\hspace*{10mm}{}-[\hat{F}_n(\hat{m}_n\pm\hat{M}_n)-\hat{F}_n(m^*\pm M^*)]\,;\\
\Delta_n^{(6)}&=\left(\hat{m}_n-m^*-\fr{1/2-\hat{F}_n(m^*)}{f(m^*)}\right)\times{}\\
&\hspace*{16mm}{}\times[f(m^*-M^*)-f(m^*+M^*)]\,.
\end{align*}
Далее
\begin{multline}
\mbox{P}\left(\abs{n^{1/2}R_n}>\sigma\eps_n\right)\leqslant{}\\
{}\leqslant
\sum\limits_{i=1}^{6}\mbox{P}\left(\abs{n^{1/2}\Delta_n^{(i)}}>\fr{\sigma h(M^*)\eps_n}{6}\right)\,.
\label{e3-sh}
\end{multline}
Пользуясь ограниченностью~$F''$, можно показать~[3--5], что найдутся
такие константы $C_i$, $i=1,\ldots,6$, что начиная с некоторого~$n$
(в зависимости от величины~$F''$ в окрестности точек~$m^*$ и $m^*\pm
M^*$)
\begin{multline}
\mbox{P}\left(\abs{n^{1/2}\Delta_n^{(i)}}>\fr{\sigma h(M^*)\eps_n}{6}\right)\leqslant{}\\
{}\leqslant\fr{C_i(\ln n)^{1/2}}{n^{1/4}},\;\;i=1,\ldots,6\,,
\label{e4-sh}
\end{multline}
если $B>\max\{B_1,B_2\}$, где 
\begin{align*}
B_1&>6[\sigma h(M^*)]^{-1}(2D_1 D_2)^{1/2}\,;\\
B_2&>6[\sigma h(M^*)]^{-1}\times{}\\
&\hspace*{3mm}{}\times\abs{f(m^*-M^*)-f(m^*+M^*)}(2D'_1 D'_2)^{1/2}\,;
\end{align*}

\noindent
\begin{align*}
D_1&>\min\left\{\fr{2}{f(m^*)},\fr{2}{h(M^*)}\right\}\,;\\
D_2&>\max\{f(m^*+M^*),f(m^*-M^*)\}\,;\\
D'_1&>\fr{1}{2f(m^*)}\,;\\
D'_2&>f(m^*)\,.
\end{align*}
Кроме того, в силу неравенства Берри--Эссеена найдется такая константа~$C_7$, что для 
первого слагаемого в~(\ref{e2-sh}) справедливо
\begin{equation}
\sup\limits_{t\in\mathbf{R}}\abs{\mbox{P}\left(\fr{n^{1/2}}{\sigma}Y_n\leqslant t\right)-
\Phi(t)}\leqslant\fr{C_7}{n^{1/2}}\,.
\label{e5-sh}
\end{equation}

Объединяя (\ref{e2-sh})--(\ref{e5-sh}), получаем~(\ref{e1-sh}). Теорема доказана.

%\medskip

\section{Случай симметричного распределения}

В случае, когда функция распределения~$F$ симметрична относительно медианы, оценку, 
приведенную в теореме~1, можно улучшить.

\medskip

\noindent
\textbf{Теорема 2.} \textit{Пусть функция распределения~$F$ сим\-мет\-рич\-на относительно 
медианы~$m^*$ и в окрестности точек~$m^*$ и $m^*\pm M^*$ имеет непрерывную 
положительную плотность~$f$ и ограниченную вторую производную~$F''$. 
Тогда существует такая константа~$\tilde{C}^*$, что начиная с некоторого~$n$
\begin{multline}
\sup\limits_{t\in\mathbf{R}}\abs{\mbox{P}\left(2h(M^*)n^{1/2}(\hat{M}_n-M^*)\leqslant 
t\right)-\Phi(t)}\leqslant{}\\
{}\leqslant \tilde{C}^*n^{-1/2}\ln n\,,
\label{e6-sh}
\end{multline}
где $\Phi(t)$~--- функция распределения стандартного нормального закона.}

\medskip

\noindent
Д\,о\,к\,а\,з\,а\,т\,е\,л\,ь\,с\,т\,в\,о\,.\ 
Обозначим 
$$G_n(t)\hm=\mbox{P}\left(2h(M^*)n^{1/2}(\hat{M}_n-M^*)\leqslant t\right)\,.
$$ 
Положим $A_n=A(\ln n)^{1/2}$, где $A$~--- некоторая константа, ограничения на 
которую будут наложены позднее. Имеем
\begin{multline}
\sup\limits_{\abs{t}>A_n}\abs{G_n(t)-\Phi(t)}=
\max\left\{\sup\limits_{t>A_n}\abs{G_n(t)-\Phi(t)},\right.\\
\left.\sup\limits_{t<-A_n}\abs{G_n(t)-\Phi(t)}\right\}\leqslant{}\\
{}\leqslant\max\left\{1-G_n(A_n)+1-\Phi(A_n),\right.\\
\left.G_n(-A_n)+\Phi(-A_n)\right\}\leqslant{}\\
{}\leqslant G_n(-A_n)+1-G_n(A_n)+1-\Phi(A_n)={}\\
{}=\mbox{P}\left(\abs{\hat{M}_n-M^*}\geqslant\fr{A_n n^{-1/2}}{2h(M^*)}\right)+1-\Phi(A_n)\,.
\label{e7-sh}
\end{multline}
Известно~\cite{5-sh}, что
$$
1-\Phi(t)\leqslant\fr{e^{-t^2/2}}{\sqrt{2\pi}t}\;\;\mbox{ при }t>0\,,
$$
поэтому
$$
1-\Phi(A_n)\leqslant\fr{e^{-A_n^2/2}}{\sqrt{2\pi}A_n}\leqslant(2\pi n\ln n)^{-1/2}\,,
$$
если $A\geqslant1$. Чтобы оценить первое слагаемое в~(\ref{e7-sh}), 
воспользуемся экспоненциальным неравенством для медианного абсолютного 
отклонения, доказанного в работе~\cite{3-sh}:
\begin{equation}
\mbox{P}\left(\abs{\hat{M}_n-M^*}\geqslant\eps\right)\leqslant6e^{-2n\Delta^2_{\eps,n}}\,.
\label{e8-sh}
\end{equation}
Здесь $\Delta_{\eps,n}=\min\{a,b,c,d\}$, где
\begin{align*}
a&=F\left(m^*+\fr{\eps}{2}\right)-\fr{1}{2}\,; \quad b=\fr{1}{2}-F\left(m^*-\fr{\eps}{2}\right)\,,\\
c&=H\left(M^*+\fr{\eps}{2}\right)-\fr{1}{2}\,; \quad d=\fr{1}{2}-H\left(M^*-\fr{\eps}{2}\right)\,.
\end{align*}
Возьмем $\eps=\eps_n=A_n n^{-1/2}[2h(M^*)]^{-1}$. 
Воспользовавшись формулой Тейлора и тем, что $F(m^*)=1/2$ и $H(M^*)=1/2$, получаем
\begin{align*}
a&=\fr{1}{2}\,f(m^*)\eps_n+\fr{1}{8}\,F''(t_1^*)\eps^2_n\,;\\
b&=\fr{1}{2}\,f(m^*)\eps_n-\fr{1}{8}\,F''(t_2^*)\eps^2_n\,,
\end{align*}
где $t_1^*$ лежит между $m^*$ и $m^*+\eps_n$, а $t_2^*$ лежит между~$m^*$ и $m^*-\eps_n$. 
Аналогично
\begin{align*}
c&=\fr{1}{2}\,h(M^*)\eps_n+\fr{1}{8}\,H''(t_3^*)\eps^2_n\,;\\
d&=\fr{1}{2}\,h(M^*)\eps_n-\fr{1}{8}\,H''(t_4^*)\eps^2_n\,,
\end{align*}
где $t_3^*$ лежит между $M^*$ и $M^*+\eps_n$, а $t_4^*$ лежит между $M^*$ и $M^*-\eps_n$.
Следовательно,
\begin{multline*}
\Delta^2_{\eps_n,n}\geqslant\fr{1}{4}\,\eps_n^2\min\left\{f^2(m^*)-
\fr{M_1 f(m^*)\eps_n}{2},\right.\\
\left. h^2(M^*)-\fr{M_2 h(M^*)\eps_n}{2}\right\}\,,
\end{multline*}
где
\begin{align*}
&\sup\limits_{\abs{z}\leqslant\eps_n}\abs{F''(m^*+z)}\leqslant M_1<\infty\,;\\
&\sup\limits_{\abs{z}\leqslant\eps_n}\abs{H''(M^*+z)}\leqslant M_2<\infty\,.
\end{align*}
Тогда
\begin{multline*}
-2n\Delta^2_{\eps_n,n}\leqslant{}\\
{}\leqslant -\fr{1}{2}\,A^2[2h(M^*)]^{-2}\ln n\min\left\{
\vphantom{\fr{M_2M^*}{2}}
f^2(m^*)-{}\right.\\
{}-\fr{M_1 f(m^*)\eps_n}{2},\,
\left.h^2(M^*)-\fr{M_2 h(M^*)\eps_n}{2}\right\}\,.
\end{multline*}
Начиная с некоторого $n$ (в зависимости от $A$, $M_1$, $M_2$, $f(m^*)$ и~$h(M^*)$),
\begin{multline*}
A^2[2h(M^*)]^{-2}\min\left\{f^2(m^*)-\fr{M_1 f(m^*)\eps_n}{2},\right.\\
\left.h^2(M^*)-\fr{M_2 h(M^*)\eps_n}{2}\right\}\geqslant1
\end{multline*}
при $A>2h(M^*)\min\left\{[f(m^*)]^{-1},[h(M^*)]^{-1}\right\}$. Следовательно, для этих $n$
\begin{align}
\mbox{P}\left(\abs{\hat{M}_n-M^*}\geqslant\eps_n\right)&\leqslant6n^{-1/2}\,;\label{e9-sh}
\\
\sup\limits_{\abs{t}>A_n}\abs{G_n(t)-\Phi(t)}&\leqslant6n^{-1/2}+{}\notag\\
&\hspace*{5mm}{}+(2\pi n\ln n)^{-1/2}
\label{e10-sh}
\end{align}
при 
$$
A>\max\left\{1,2h(M^*)\min\left\{[f(m^*)]^{-1},[h(M^*)]^{-1}\right\}\right\}\,.
$$

Рассмотрим теперь $\sup\limits_{\abs{t}\leqslant A_n}\abs{G_n(t)\hm-\Phi(t)}$.
Имеем
\begin{multline*}
G_n(t)=\mbox{P}\left(2h(M^*)n^{1/2}(\hat{M}_n-M^*)\leqslant t\right)={}\\
{}=\mbox{P}\left(\hat{M}_n\leqslant M^*+\fr{tn^{-1/2}}{2h(M^*)}\right)={}\\
{}=\mbox{P}\left(\fr{1}{2}\leqslant\hat{H}_n\left(M^*+\fr{tn^{-1/2}}{2h(M^*)}\right)\right)={}\\
{}=\mbox{P}\left(\fr{n}{2}\leqslant\sum\limits_{i=1}^{n}
I\left(\hat{m}_n-M^*-\fr{tn^{-1/2}}{2h(M^*)}\leqslant X_i\leqslant{}\right.\right.\\
\left.\left.{}\leqslant \hat{m}_n+M^*+
\fr{tn^{-1/2}}{2h(M^*)}\right)\right)\,.
\end{multline*}
Следовательно,
\begin{multline*}
\sup\limits_{\abs{t}\leqslant A_n}\abs{G_n(t)-\Phi(t)}={}\\
{}=\sup\limits_{\abs{t}\leqslant A_n}\left\vert\mbox{P}\left(\fr{n}{2}\leqslant
\sum\limits_{i=1}^{n}
I\left(\hat{m}_n-M^*-\fr{tn^{-1/2}}{2h(M^*)}\leqslant{}\right.\right.\right.\\
\left.\left.\left.{}\leqslant X_i
\leqslant \hat{m}_n+M^*+\fr{tn^{-1/2}}{2h(M^*)}\right)\right)-\Phi(t)\right\vert={}\\
{}=\sup\limits_{\abs{t}\leqslant A_n}\left|\mbox{P}\left(\fr{n}{2}\leqslant\sum\limits_{i=1}^{n}
I\left(m^*+(\hat{m}_n-m^*)-{}\right.\right.\right.
\end{multline*}

\noindent
\begin{multline*}
\hspace*{8mm}\left.\left.\left.{}-M^*-\fr{tn^{-1/2}}{2h(M^*)}\leqslant X_i\leqslant{}\right.\right.\right.\\
\!\left.\left.\left.{}\leqslant
 m^*+(\hat{m}_n-m^*)+M^*+\fr{tn^{-1/2}}{2h(M^*)}\right)\right)-\Phi(t)\right|.
 \end{multline*}
Обозначим события под индикатором через $B_i$. Тогда
\begin{multline}
\sup\limits_{\abs{t}\leqslant A_n}\abs{G_n(t)-\Phi(t)}={}\\
{}=\sup\limits_{\abs{t}\leqslant A_n}\left|\mbox{P}\left(
\fr{n}{2}\leqslant\sum\limits_{i=1}^{n}I(B_i), 
\;\abs{\hat{m}_n-m^*}\leqslant\eps_n\right)+{}\right.\\
\left.{}+\mbox{P}\left(\fr{n}{2}\leqslant\sum\limits_{i=1}^{n}I(B_i)\;| \;
\abs{\hat{m}_n-m^*}>\eps_n\right)\times{}\right.\\
\left.{}\times \mbox{P}\left(\abs{\hat{m}_n-m^*}>\eps_n\right)-\Phi(t)
\vphantom{\sum\limits_{i=1}^n}\right|\leqslant{}\\
{}\leqslant\sup\limits_{\abs{t}\leqslant A_n}
\sup\limits_{\abs{\eps}\leqslant \eps_n}\left|\mbox{P}\left(\fr{n}{2}
\leqslant\sum\limits_{i=1}^{n}I\left(m^*+\eps-M^*-{}\right.\right.\right.\\
\left.\left.\left.{}-\fr{tn^{-1/2}}{2h(M^*)}\leqslant
 X_i\leqslant m^*+\eps+M^*+\fr{tn^{-1/2}}{2h(M^*)}\right)\right)-{}\right.\\
\left. {}-\Phi(t)
\vphantom{\sum\limits^n_{i=1}}\right|+\mbox{P}\left(\abs{\hat{m}_n-m^*}>\eps_n\right)\,.
\label{e11-sh}
\end{multline}
Для второго слагаемого при соответствующих ограничениях на $n$ и~$A$ 
справедливы оценки, аналогичные~(\ref{e8-sh}) и~(\ref{e9-sh})~\cite{5-sh}:
\begin{equation}
\mbox{P}\left(\abs{\hat{m}_n-m^*}\geqslant\eps_n\right)\leqslant2n^{-1/2}\,.
\label{e12-sh}
\end{equation}
Оценим первое слагаемое в~(\ref{e11-sh}). Введем обозначения:
\begin{align*}
D_{\eps,n,t}&=F\left(m^*+\eps+M^*+\fr{tn^{-1/2}}{2h(M^*)}\right)-{}\\
&\hspace*{5mm}{}-F\left(m^*+\eps-M^*-\fr{tn^{-1/2}}{2h(M^*)}\right)\,;
\\
S(D_{\eps,n,t})&=\sum\limits_{i=1}^{n}I\left(m^*+\eps-M^*-
\fr{tn^{-1/2}}{2h(M^*)}\leqslant {}\right.\\
&\hspace*{5mm}\left.{}{}\leqslant X_i\leqslant m^*+\eps+M^*+\fr{tn^{-1/2}}{2h(M^*)}\right)\,;
\\
S^*(D_{\eps,n,t})&=\fr{S(D_{\eps,n,t})-nD_{\eps,n,t}}{[nD_{\eps,n,t}(1-D_{\eps,n,t})]^{1/2}}\,;
\\
x_{\eps,n,t}&=\fr{n^{1/2}(D_{\eps,n,t}-1/2)}{[D_{\eps,n,t}(1-D_{\eps,n,t})]^{1/2}}\,.
\end{align*}
Сумма $S(D_{\eps,n,t})$ представляет собой биномиальную случайную величину с параметрами $(n,D_{\eps,n,t})$, поэтому в силу неравенства Берри--Эссеена
\begin{multline}
\sup\limits_{\abs{t}\leqslant A_n}\sup\limits_{\abs{\eps}\leqslant \eps_n}\left|
\mbox{P}\left(\fr{n}{2}\leqslant\sum\limits_{i=1}^{n}I
\left(
\vphantom{\fr{tn^{-1/2}}{2h(M^*)}}
m^*+\eps-M^*-{}\right.\right.\right.\\
\left.\left.\left.{}-\fr{tn^{-1/2}}{2h(M^*)}
\leqslant X_i\leqslant m^*+\eps+M^*+
\fr{tn^{-1/2}}{2h(M^*)}\right)\right)-{}\right.\\
\left.{}-\Phi(t)
\vphantom{\fr{tn^{-1/2}}{2h(M^*)}}\right|={}\\
{}=\!\!\!\sup\limits_{\abs{t}\leqslant A_n}\sup\limits_{\abs{\eps}\leqslant 
\eps_n}\abs{\mbox{P}\left(S^*(D_{\eps,n,t})\geqslant-x_{\eps,n,t}\right)-\Phi(t)}={}\\
{}=\sup\limits_{\abs{t}\leqslant A_n}
\sup\limits_{\abs{\eps}\leqslant \eps_n}\vert1-\Phi(t)-
\mbox{P}\left(S^*(D_{\eps,n,t})<{}\right.\\
\left.{}<-x_{\eps,n,t}\right)\vert
\leqslant\sup\limits_{\abs{t}\leqslant A_n}\sup\limits_{\abs{\eps}\leqslant 
\eps_n}\vert\Phi(-x_{\eps,n,t})-{}\\
{}-\mbox{P}\left(S^*(D_{\eps,n,t})<-x_{\eps,n,t}\right)\vert+{}\\
{}+\sup\limits_{\abs{t}\leqslant A_n}\sup\limits_{\abs{\eps}\leqslant \eps_n}
\abs{\Phi(x_{\eps,n,t})-\Phi(t)}\leqslant{}\\
{}\leqslant\sup\limits_{\abs{t}\leqslant A_n}
\sup\limits_{\abs{\eps}\leqslant \eps_n}\abs{n^{-1/2}C_0\rho(D_{\eps,n,t})}+{}\\
{}+\sup\limits_{\abs{t}\leqslant A_n}\sup\limits_{\abs{\eps}\leqslant \eps_n}\abs{\Phi(x_{\eps,n,t})-\Phi(t)}\,,
\label{e13-sh}
\end{multline}
где
$$
\rho(D_{\eps,n,t})=\fr{\mu^3(D_{\eps,n,t})}{\sigma^3(D_{\eps,n,t})}=
\fr{(1-D_{\eps,n,t})^2+D^2_{\eps,n,t}}{[D_{\eps,n,t}(1-D_{\eps,n,t})]^{1/2}}\,,
$$

\vspace*{-6pt}

\noindent
\begin{multline*}
\mu^3(D_{\eps,n,t})=D_{\eps,n,t}(1-D_{\eps,n,t})\left[(1-D_{\eps,n,t})^2+{}\right.\\
\left.{}+D^2_{\eps,n,t}\right]\,,
\end{multline*}

\noindent
$$
\sigma^2(D_{\eps,n,t})=D_{\eps,n,t}(1-D_{\eps,n,t})\,,
$$
а $C_0$~--- константа из неравенства Берри--Эссеена.
Обозначим: 
\begin{align*}
H_\eps(y)&=F(m^*+\eps+y)-F(m^*+\eps-y)\,;\\ 
h_\eps(y)&=f(m^*+\eps+y)+f(m^*+\eps-y)\,;\\ 
g_\eps(z)&=[H_\eps(M^*+z)-H^2_\eps(M^*+z)]^{-1/2}\,.
\end{align*} 
Тогда
\begin{multline*}
g'_\eps(z)=-\fr{1}{2}h_\eps(M^*+z)\left[1-2H_\eps(M^*+z)\right]\times{}\\
{}\times\left[H_\eps(M^*+z)-H^2_\eps(M^*+z)\right]^{-3/2}\,;
\end{multline*}

\noindent
$$
[D_{\eps,n,t}(1-D_{\eps,n,t})]^{-1/2}=g_\eps(0)+g'_\eps(z^*)\fr{tn^{-1/2}}{2h(M^*)}\,,
$$
где $z^*$ лежит между~0 и $tn^{-1/2}/[2h(M^*)]$. Величины

\noindent
\begin{align*}
g_n&=\sup\limits_{\abs{\eps}\leqslant \eps_n}\abs{g_\eps(0)}\\
g^*_n&=\sup\limits_{\abs{z}\leqslant A_n n^{-1/2}/[2h(M^*)],\;\abs{\eps}\leqslant \eps_n}\abs{g'_\eps(z)}
\end{align*}
конечны (по крайней мере, начиная с некоторого~$n$). Более того,
$$
g_n\rightarrow2\;\;\mbox{ и }\;\;g^*_n\rightarrow0\;\;\mbox{ при }n\rightarrow\infty\,.
$$
Следовательно, величина
$$
\rho_n=\sup\limits_{\abs{t}\leqslant A_n}\sup\limits_{\abs{\eps}\leqslant \eps_n}\abs{\rho(D_{\eps,n,t})}
$$
также конечна. Более того,
$$
\rho(D_{\eps,n,t})\rightarrow1 \;\;\mbox{ при }n\rightarrow\infty\,.
$$
Таким образом, первое слагаемое в~(\ref{e13-sh}) имеет порядок $C_n n^{-1/2}$, где величина~$C_n$ 
ограничена и $C_n\rightarrow C_0$ при $n\rightarrow\infty$.

Далее по формуле Тейлора получаем
\begin{multline*}
n^{1/2}\left(D_{\eps,n,t}-\fr{1}{2}\right)={}\\
{}=n^{1/2}\left(H_\eps\left(M^*+\fr{tn^{-1/2}}{2h(M^*)}\right)-\fr{1}{2}\right)={}\\
{}=n^{1/2}\left(H_\eps\left(M^*+\fr{tn^{-1/2}}{2h(M^*)}\right)-H_\eps\left(M^*\right)+{}\right.\\
\left.{}+
H_\eps\left(M^*\right)-\fr{1}{2}
\vphantom{\fr{tn^{-1/2}}{2h(M^*)}}\right)={}\\
{}=\fr{t}{2h(M^*)}\left(h_\eps(M^*)+\fr{H''_\eps(y^*)}{4h(M^*)}tn^{-1/2}\right)+{}\\
{}+n^{1/2}\left(F(m^*+\eps+M^*)-F(m^*+\eps-M^*)-{}\right.\\
\left.{}-(F(m^*+M^*)-F(m^*-M^*))\right)\,,
\end{multline*}
где $y^*$ лежит между $M^*$ и $M^*+tn^{-1/2}/(2h(M^*))$. 
Учитывая симметричность распределения~$F$, имеем:
{\looseness=-2

}

\noindent
\begin{multline*}
F(m^*+\eps+M^*)-F(m^*+\eps-M^*)-(F(m^*+M^*)-{}\\
{}-F(m^*-M^*))=\fr{F''(z^*)-F''(z^{**})}{2}\eps^2\,,
\end{multline*}
где $z^*$ лежит между $m^*+M^*$ и $m^*+M^*+\eps$, 
а $z^{**}$ лежит между $m^*-M^*$ и $m^*-M^*+\eps$.

Далее в силу симметричности~$F$
$$
g_\eps(0)=2+\beta(\eps)\eps\,,
$$ 
где $\beta(\eps)\rightarrow 0$  при $\eps\hm\rightarrow 0$;
$$
h_\eps(M^*)=h(M^*)+\gamma(\eps)\eps\,,
$$
где $\gamma(\eps)\rightarrow0$  при $\eps\hm\rightarrow0$.
Следовательно,

\pagebreak

\noindent
\begin{multline*}
x_{\eps,n,t}=\left(\fr{t}{2h(M^*)}\left(h_\eps(M^*)+
\fr{H''_\eps(y^*)}{4h(M^*)}tn^{-1/2}\right)+{}\right.\\
\left.{}+
\fr{F''(z^*)-F''(z^{**})}{2}n^{1/2}\eps^2\right)\times{}\\
{}\times\left(g_\eps(0)+g'_\eps(z^*)\fr{tn^{-1/2}}{2h(M^*)}\right)= {}\\
{}=
\left(2+\beta(\eps)\eps+g'_\eps(z^*)\fr{tn^{-1/2}}{2h(M^*)}\right)\times{}\\
{}\times\left(\fr{t}{2h(M^*)}\left(h(M^*)+\gamma(\eps)\eps+
\fr{H''_\eps(y^*)}{4h(M^*)}tn^{-1/2}\right)+{}\right.\\
\left.{}+
\fr{F''(z^*)-F''(z^{**})}{2}n^{1/2}\eps^2\right)
=t+B_{\eps,n,t}\,,
\end{multline*}
где для $B_{\eps,n,t}$ с некоторой константой~$A_0$ начиная с некоторого~$n$ выполнено
$$
B_n=\sup\limits_{\abs{t}\leqslant A_n}\sup\limits_{\abs{\eps}\leqslant \eps_n}\abs{B_{\eps,n,t}}\leqslant A_0n^{-1/2}\ln n\,.
$$
Таким образом,
\begin{multline}
\sup\limits_{\abs{t}\leqslant A_n}\sup\limits_{\abs{\eps}\leqslant
\eps_n}\abs{\Phi(x_{\eps,n,t})-\Phi(t)}\leqslant{}\\
{}\leqslant  A_0(2\pi n)^{-1/2}\ln n\,.
\label{e14-sh}
\end{multline}

Объединяя (\ref{e7-sh})--(\ref{e14-sh}), получаем~(\ref{e6-sh}). Теорема доказана.

{\small\frenchspacing
{%\baselineskip=10.8pt
\addcontentsline{toc}{section}{Литература}
\begin{thebibliography}{9}

\bibitem{1-sh}
\Au{Hall P., Welsh A.\,H.} 
Limits theorems for median deviation~// Annals of the Institute of
Statistical Math., 1985. Vol.~37. No.\,1. P.~27--36.
\par

\bibitem{2-sh}
\Au{Falk M.} Asymptotic independence of median and MAD~// 
Stat. Prob. Lett., 1997. Vol.~34. P.~341--345.


\bibitem{3-sh}
\Au{Serfling R., Mazumder S.} Exponential probability inequality and convergence results for
the median absolute deviation and its modifications~// Stat. Prob. Lett.,
2009. Vol.~79. No.\,16. P.~1767--1773.

\bibitem{4-sh}
\Au{Mazumder S., Serfling R.} Bahadur representations for the median absolute deviation
and its modifications~// Stat. Prob. Lett., 2009. Vol.~79. No.\,16. P.~1774--1783.

\bibitem{5-sh}
\Au{Serfling R.\,J.} Approximation theorems of mathematical statistics.~--- 
New York: John Wiley \&~Sons, 1980.


\label{end\stat}

\bibitem{7-sh}
\Au{Reiss R.\,D.} On the accuracy of the normal approximation for quantiles~// Ann.
Prob., 1974. Vol.~2. No.\,4. P.~741--744.

\bibitem{6-sh}
\Au{Bahadur R.\,R.} A note on quantiles in large samples~// 
The Annals of Math. Statistics, 1966. Vol.~37. No.\,3. P.~577--580.

 \end{thebibliography}
}
}


\end{multicols}       