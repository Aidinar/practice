\def\stat{zats}

\def\tit{К 60-ЛЕТИЮ ЧЛЕНА РЕДКОЛЛЕГИИ ЖУРНАЛА <<ИНФОРМАТИКА И ЕЁ 
ПРИМЕНЕНИЯ>>, ЗАМЕСТИТЕЛЯ ДИРЕКТОРА ИПИ РАН ПО~НАУЧНОЙ РАБОТЕ, 
ДОКТОРА ТЕХНИЧЕСКИХ НАУК А.\,А.~ЗАЦАРИННОГО}

\def\titkol{К 60-летию члена редколлегии журнала <<Информатика и её 
применения>>, заместителя директора ИПИ РАН по научной работе, 
доктора технических наук А.\,А.~Зацаринного}

\def\autkol{И.\,А.~Соколов, С.\,Я.~Шоргин}
\def\aut{И.\,А.~Соколов$^1$, С.\,Я.~Шоргин$^2$}

\titel{\tit}{\aut}{\autkol}{\titkol}

%{\renewcommand{\thefootnote}{\fnsymbol{footnote}}\footnotetext[1]
%{Работа поддержана Российским фондом
%фундаментальных исследований (проекты 11-01-00515а и 11-07-00112а),
%Федеральной целевой программой <<Научные и научно-педагогические
%кадры инновационной России на 2009--2013~годы>> и грантом Президента
%РФ МК--581.2010.1.}}

\renewcommand{\thefootnote}{\arabic{footnote}}
\footnotetext[1]{Институт проблем информатики Российской академии наук, isokolov@ipiran.ru}
\footnotetext[2]{Институт проблем информатики Российской академии наук,
sshorgin@ipiran.ru}

\vskip 14pt plus 9pt minus 6pt

      \thispagestyle{headings}

      \begin{multicols}{2}
      
            \label{st\stat}

\vspace*{-6pt}
\begin{center}
\mbox{%
\epsfxsize=79mm
\epsfbox{foto-zac.eps}
}
\end{center}
\vspace*{9pt}

25~апреля 2011~года исполнилось 60~лет Александру Алексеевичу 
Зацаринному~--- члену редколлегии журнала <<Информатика и её 
применения>>, лауреату премии Правительства РФ в области науки и 
техники, известному ученому в области создания 
информационно-телекоммуникационных систем и разработки методов оценки их 
эффективности.
     
     А.\,А.~Зацаринный родился в Киеве в семье военнослужащего. 
В~1968~г.\ поступил в Киевское высшее военное инженерное училище связи 
им.\ М.\,И.~Калинина. После окончания учебы в 1973~г.\ был направлен для 
прохождения службы в 16~Центральный 
     на\-уч\-но-ис\-сле\-до\-ва\-тель\-ский испытательном институт связи 
Министерства обороны. За 26~лет на\-уч\-но-ис\-сле\-до\-ва\-тель\-ской 
работы в этой ведущей научной организации в области систем и комплексов 
военной связи прошел путь от младшего научного сотрудника~--- 
лей\-те\-нан\-та-ин\-же\-не\-ра до заместителя начальника института по научной 
работе (1992--1999~гг.)~--- полковника. В~1984~г.\ защитил кандидатскую, а 
в 1996~г.~--- докторскую диссертацию.
     
     В сентябре 1999~г.\ Указом Президента РФ А.\,А.~Зацаринный был 
назначен начальником Управ\-ле\-ния развития систем связи и АСУ 
Вооруженных Сил (в составе Управления начальника связи ВС РФ), которым 
руководил в течение шести лет (1999--2005). В~2000~г.\ ему присвоено 
воинское звание ге\-не\-рал-майор, в 2002~г.~--- ге\-не\-рал-лей\-те\-нант. 
     
     После увольнения из рядов Вооруженных Сил с января 2006~г.\ по 
настоящее время занимает должность заместителя директора Института 
проблем информатики Российской академии наук по научной работе.

Вся трудовая и служебная деятельность А.\,А.~Зацаринного неразрывно 
связана с на\-уч\-но-ис\-сле\-до\-ва\-тель\-ской, 
на\-уч\-но-ор\-га\-ни\-за\-ци\-он\-ной и на\-уч\-но-прак\-ти\-че\-ской работой 
в области создания систем и комплексов военной связи, а также 
автоматизированных информационных систем.
     
     За период работы в ИПИ РАН А.\,А.~Зацаринным получен ряд 
важнейших результатов в области путей и методов создания 
     ин\-фор\-ма\-ци\-он\-но-те\-ле\-ком\-му\-ни\-ка\-ци\-он\-ных систем.
     
     А.\,А.~Зацаринный является действительным членом Академии 
военных наук, Международной\linebreak академии связи и Российской академии 
инженерных наук имени А.\,М.~Прохорова (Отделение <<Информационные 
сети, связь и радиотехника>>),\linebreak членом Экспертного совета ВАК России по 
электронике, измерительной технике, радиотехнике и связи, членом 
диссертационных советов при ИПИ РАН и при НИИ АА им.\ академика 
В.\,С.~Семенихина. Имеет ряд правительственных и ведомственных наград, 
в том числе орден <<За военные заслуги>> (2000~г.) и 15~медалей. 

\label{end\stat}

     \bigskip
     От имени редколлегии сердечно поздравляем А.\,А.~Зацаринного с 
юбилеем и желаем ему счастья, здоровья, семейного благополучия и 
дальнейших творческих успехов. 
\end{multicols}  

%\newpage