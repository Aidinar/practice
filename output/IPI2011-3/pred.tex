   { %\Large  
   { %\baselineskip=16.6pt
   
   \vspace*{-48pt}
   \begin{center}\LARGE
   \textit{Предисловие}
   \end{center}
   
   %\vspace*{2.5mm}
   
   \vspace*{25mm}
   
   \thispagestyle{empty}
   
   { %\small 

    
Вниманию читателей журнала <<Информатика и её применения>> предлагается 
очередной тематический выпуск <<Вероятностно-статистические методы и 
задачи информатики и информационных технологий>>. Предыдущие тематические 
выпуски журнала по данному направлению вышли в 2008~г.\ (т.~2, вып.~2), 
в 2009~г.\ (т.~3, вып.~3) и в 2010~г.\ (т.~4, вып.~2). 

Статьи, собранные в данном журнале, посвящены разработке новых вероятностно-статистических 
методов, ориентированных на применение к решению конкретных задач информатики и информационных 
технологий, а также~--- в ряде случаев~--- и других прикладных задач. Проблематика, охватываемая 
публикуемыми работами, развивается в рамках научного сотрудничества между Институтом проблем 
информатики Российской академии наук (ИПИ РАН) и Факультетом вычислительной математики и 
кибернетики Московского государственного университета им.\ М.\,В.~Ломоносова в ходе работ 
над совместными научными проектами (в том числе в рамках функционирования 
Научно-образовательного центра <<Вероятностно-статистические методы анализа рисков>>). 
Многие из авторов статей, включенных в данный номер журнала, являются активными участниками 
традиционного международного семинара по проблемам устойчивости стохастических моделей, 
руководимого В.\,М.~Золотаревым и В.\,Ю.~Королевым; регулярные сессии этого семинара 
проводятся под эгидой МГУ и ИПИ РАН (в 2011~г.\ указанный семинар проводится в октябре 
в Калининградской области РФ). 

Наряду с представителями ИПИ РАН и МГУ в число авторов данного выпуска журнала входят 
ученые из Научно-исследовательского института системных исследований РАН, Института 
проблем технологии микроэлектроники и особочистых материалов РАН, Института 
прикладных математических исследований Карельского НЦ РАН, Московского 
авиационного института, Вологодского государственного педагогического университета, 
НИИММ им.\ Н.\,Г.~Чеботарева, Казанского государственного университета, Дебреценского 
университета (Венгрия).

Несколько статей выпуска посвящено разработке и применению стохастических методов и 
информационных технологий для решения различных прикладных задач. В~работе В.\,Г.~Ушакова 
и О.\,В.~Шестакова рассмотрена задача определения вероятностных характеристик случайных 
функций по распределениям интегральных преобразований, возникающих в задачах эмиссионной 
томографии. В~статье Д.\,О.~Яковенко и М.\,А.~Целищева рассмотрены некоторые вопросы 
математической теории риска и предложен новый подход к диверсификации инвестиционных 
портфелей. Работа И.\,А.~Кудрявцевой и А.\,В.~Пантелеева посвящена построению и 
исследованию математической модели, описывающей динамику сильноионизованной плазмы. 
В~статье П.\,П.~Кольцова изучается качество работы ряда алгоритмов сегментации изображений. 
Статья А.\,Н.~Чупрунова и И.~Фазекаша посвящена вероятностному анализу числа без\-оши\-бочных 
блоков при помехоустойчивом кодировании; получены усиленные законы больших чисел для указанных 
величин.

В данном выпуске традиционно присутствует тематика, весьма активно разрабатываемая в течение 
многих лет специалистами ИПИ РАН и МГУ,~--- методы моделирования и управления для 
информационно-телекоммуникационных и вычислительных систем, в частности методы 
теории массового обслуживания. В~статье А.\,И.~Зейфмана с соавторами рассматриваются 
модели обслуживания, описываемые марковскими цепями с непрерывным временем в случае 
наличия катастроф. В~работе М.\,М.~Лери и И.\,А.~Чеплюковой рассматриваются случайные 
графы Интернет-типа, т.\,е.\ графы, степени вершин которых имеют степенные распределения; 
такие задачи находят применение при исследовании глобальных сетей передачи данных. 
Работа Р.\,В.~Разумчика посвящена исследованию систем массового обслуживания специального 
вида~--- с отрицательными заявками и хранением вытесненных заявок.

Ряд статей посвящен развитию перспективных теоретических 
вероятностно-статистических методов, которые находят широкое применение в различных 
задачах информатики и информационных технологий. В~работе В.\,Е.~Бенинга, А.\,К.~Горшенина 
и В.\,Ю.~Королева рассмотрена задача статистической проверки гипотез о числе компонент 
смеси вероятностных распределений, приводится конструкция асимптотически наиболее мощного 
критерия. Результаты этой работы найдут применение в ряде прикладных задач, использующих 
математическую модель смеси вероятностных распределений (в информатике, моделировании 
финансовых рынков, физике турбулентной плазмы и~т.\,д.). В~статье В.\,Ю.~Королева, 
И.\,Г.~Шевцовой и С.\,Я.~Шоргина строится новая, улучшенная оценка точности нормальной 
аппроксимации для пуассоновских случайных сумм; как известно, указанные случайные суммы 
широко используются в качестве моделей многих реальных объектов, в том числе в информатике, 
физике и других прикладных областях. Работа В.\,Г.~Ушакова и Н.\,Г.~Ушакова посвящена 
исследованию ядерной оценки плотности распределения; эти результаты могут применяться, 
в част\-ности, при анализе трафика в телекоммуникационных системах. Серьезные приложения 
в статистике могут получить результаты работы О.\,В.~Шестакова, в которой доказаны оценки 
скорости сходимости распределения выборочного абсолютного медианного отклонения к нормальному 
закону. 

\smallskip

Редакционная коллегия журнала выражает надежду, что данный тематический  выпуск 
будет интересен специалистам в области теории вероятностей и математической статистики 
и их применения к решению задач информатики и информационных технологий.
     
     %\vfill 
     \vspace*{20mm}
     \noindent
     Заместитель главного редактора журнала <<Информатика и её 
применения>>,\\
     директор ИПИ РАН, академик  \hfill
     \textit{И.\,А.~Соколов}\\
     
     \noindent
     Редактор-составитель тематического выпуска,\\
     профессор кафедры математической статистики факультета\\
      вычислительной математики и кибернетики МГУ им.\ М.\,В.~Ломоносова,\\
     ведущий научный сотрудник ИПИ РАН,\\ 
доктор физико-математических наук \hfill
      \textit{В.\,Ю.~Королев}
     
     } }
     }