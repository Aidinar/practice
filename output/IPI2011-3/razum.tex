
\def\stat{razum}

\def\tit{СИСТЕМА МАССОВОГО ОБСЛУЖИВАНИЯ С~ОТРИЦАТЕЛЬНЫМИ ЗАЯВКАМИ,
БУНКЕРОМ ДЛЯ ВЫТЕСНЕННЫХ ЗАЯВОК И~РАЗЛИЧНЫМИ ИНТЕНСИВНОСТЯМИ
ОБСЛУЖИВАНИЯ$^*$}

\def\titkol{СМО с~отрицательными заявками,
бункером для вытесненных заявок и~различными интенсивностями
обслуживания}

\def\autkol{Р.\,В.~Разумчик}
\def\aut{Р.\,В.~Разумчик$^1$}

\titel{\tit}{\aut}{\autkol}{\titkol}

{\renewcommand{\thefootnote}{\fnsymbol{footnote}}\footnotetext[1]
{Работа выполнена при финансовой поддержке Российского фонда
фундаментальных исследований (проект №\,11-07-00112).}}

\renewcommand{\thefootnote}{\arabic{footnote}}
\footnotetext[1]{Институт проблем информатики
Российской академии наук, rrazumchik@ieee.org}


\Abst{Рассмотрена система массового обслуживания (СМО), в которую поступают пуассоновские
потоки положительных и отрицательных заявок. Для положительных заявок имеется
накопитель неограниченной емкости. Отрицательная заявка, поступающая в систему,
выбивает положительную заявку из очереди в накопителе и перемещает ее в другой накопитель
неограниченной емкости (бункер). Если накопитель
пуст, отрицательная заявка покидает систему, не оказывая на нее никакого воздействия.
После окончания обслуживания очередной заявки на прибор поступает заявка из накопителя
или, если накопитель пуст, из бункера. Длительности обслуживания заявок
из накопителя и бункера имеют экспоненциальные распределения с различными
параметрами. Получены соотношения, позволяющие вычислять стационарные
распределения очередей в накопителе и бункере.}

\KW{система массового обслуживания; отрицательные заявки; бункер; различные интенсивности
обслуживания}

  \vskip 14pt plus 9pt minus 6pt

      \thispagestyle{headings}

      \begin{multicols}{2}
      
            \label{st\stat}



\section{Введение}

В~настоящее время изучению систем и сетей массового обслуживания
с отрицательными заявками по-прежнему уделяется значительное внимание.
Об этом свидетельствует большое число работ в данной области, которые
публикуются каждый год. Подробный обзор публикаций до 2003~г.\
приведен в~\cite{bib0}.
Среди недавних работ можно отметить~[2--9].
В~настоящей статье, которая является продолжением работы~\cite{mandzo},
рассматривается другой, отличный от классического, вид отрицательных заявок.
Поступающие отрицательные
заявки не разрушают заявки, ожидающие в очереди, а перемещают
их в дополнительную очередь, откуда те
обслуживаются с относительным приоритетом.

Рассмотрим однолинейную СМО, в
которую поступает пуассоновский поток заявок интен\-сив\-ности~$\lambda$. 
Заявки этого потока, как и в~\cite{mandzo}, будем называть
положительными.
Для положительных заявок имеется накопитель неограниченной ем\-кости.

Помимо положительных заявок в
систему поступает пуассоновский поток отрицательных заявок
интенсивности~$\lambda^-$. Отрицательная заявка, поступающая в
систему, вытесняет одну (положительную) заявку из конца очереди в накопителе и
перемещает ее в накопитель для вытесненных заявок или бункер,
который также имеет неограниченную емкость.

Если в момент поступления отрицательной заявки в накопителе нет
положительных заявок, а на приборе обслуживается заявка, то
отрицательная заявка, не прерывая обслуживания на приборе,
покидает систему, не оказывая на нее никакого воздействия.
То же самое происходит и в случае, когда в момент поступления
отрицательной заявки накопитель и обслуживающий прибор пусты.

Выбор заявок на обслуживание производится следующим образом.
После окончания обслуживания очередной заявки на прибор
становится заявка из накопителя.
Если же накопитель пуст, на прибор поступает заявка из бункера.
Обслуживание заявок не прерывается новыми как положительными, так
и отрицательными заявками.

Длительности обслуживания заявок из накопителя имеют экспоненциальное
распределение с параметром~$\mu_1$, а из бункера~--- экспоненциальное
распределение с параметром~$\mu_2$.

\section{Система уравнений равновесия}

Обозначим через $\nu(t)$ число заявок, находящихся в накопителе
в момент времени~$t$, через $\eta(t)$~---
число заявок в бункере, а через $\psi(t)$~--- тип заявки,
которую обслуживает прибор в момент~$t$.\linebreak
Положим $X(t)=(\nu(t),\eta(t),\psi(t))$. Случайный процесс
$\{X(t),\ \ t\ge 0\}$, описывающий стохастическое поведение
рассматриваемой СМО во времени, является марковским процессом
с непрерывным временем и дискретным множеством состояний.
Множество состояний процесса $\{X(t), \ \ t\ge 0\}$ имеет вид
$\mathcal{X}\hm= \mathcal{X}_0 \cup \mathcal{X}_1$,
где $\mathcal{X}_0\hm=\{0\}$, 
${\mathcal{X}_1}\hm=\{(i,j,k), \ i \ge 0, \ j \ge 0, \ k=\overline{0,1} \}$.
Состояние~$(0)$ соответствует полностью свободной системе;
состояние~$(i,j,k)$
означает, что в момент времени~$t$ в
накопителе находится $i$~заявок, в бункере ожидают $j$~заявок, вытесненных из накопителя, и на приборе обслуживается
заявка либо из накопителя (при $k=0$), либо из бункера (при $k=1$).

Обозначим через $p_{i,j,k}$ стационарную вероятность состояния $(i,j,k)$, а через $p_0$~--- 
стационарную вероятность состояния~$(0)$.
При некотором условии, о котором будет сказано позже, стационарное распределение
существует и удовлетворяет сле\-ду\-ющей системе уравнений равновесия:
\begin{gather}
\label{e1-r}
\lambda p_0 = \mu_1 p_{0,0,0}+\mu_2 p_{0,0,1}\,;\\[4pt]
\label{e2-r}
(\lambda +\mu_1)p_{0,0,0}=\lambda p_0 + \mu_1 p_{1,0,0} + 
\mu_2 p_{1,0,1}\,;
\\[4pt]
\label{e3-r}
(\lambda +\mu_1+\lambda^-)p_{i,0,0}=\lambda p_{i-1,0,0} + \mu_1 p_{i+1,0,0} +{}\notag\\ 
{}+\mu_2 p_{i+1,0,1}\,; \quad
i \ge 1\,, \\[4pt]
\label{e4-r}
(\lambda+\mu_1)p_{0,j,0}=\lambda^- p_{1,j-1,0} + \mu_1 p_{1,j,0} + \mu_2 p_{1,j,1}\,,\notag\\
\hspace*{50mm}j \ge 1\,;
\end{gather}

\vspace*{-9pt}

\noindent
\begin{multline}
\label{e5-r}
(\lambda+\mu_1+\lambda^-) p_{i,j,0}=\lambda p_{i-1,j,0} +  \lambda^- p_{i+1,j-1,0}  +{}\\
{}+ \mu_1 p_{i+1,j,0} + \mu_2 p_{i+1,j,1},\ i \ge 1\,, j \ge 1\,;
\end{multline}

\vspace*{-6pt}

\noindent
\begin{gather}
\label{e6-r}
(\lambda+\mu_2)p_{0,0,1}= \mu_1 p_{0,1,0} + \mu_2 p_{0,1,1}\,;
\\[4pt]
\label{e7-r}
(\lambda+\mu_2+\lambda^-)p_{i,0,1}=\lambda p_{i-1,0,1}\,,\ \ i \ge 1\,;
\\[4pt]
\label{e8-r}
(\lambda+\mu_2)p_{0,j,1}=\lambda^- p_{1,j-1,1}+ \mu_1 p_{0,j+1,0} +{}\notag\\
\hspace*{5mm}{}+ \mu_2 p_{0,j+1,1}\,,\ \ j \ge 1\,;
\\[4pt]
(\lambda+\mu_2+\lambda^-)p_{i,j,1}=\lambda^- p_{i+1,j-1,1} + \lambda 
p_{i-1,j,1}\,,\notag\\ 
\hspace*{30mm} i \ge 1, \ j \ge 1\,. \label{e9-r}
\end{gather}
К этим уравнениям необходимо добавить условие нормировки
\begin{equation}
\label{e10-r}
p_{0} + \sum\limits_{i=0}^{\infty} \sum\limits_{j=0}^{\infty}
\sum\limits_{k=0}^1 p_{i,j,k} = 1\,.
\end{equation}

\section{Совместное распределение числа заявок в~накопителе и~бункере}

Найдем совместное стационарное распределение
числа заявок в накопителе и бункере.
Для этого введем две производящие функции (ПФ):

\noindent
\begin{gather*}
%\label{e11-r}
P(u,v) = \sum\limits_{i=0}^{\infty}
\sum\limits_{j=0}^{\infty} p_{i,j,0} u^i v^j\,,\\
N(u,v) = \sum\limits_{i=0}^{\infty} \sum\limits_{j=0}^{\infty}
p_{i,j,1} u^i v^j\,,
\ \ 0 \le u \le 1\,, \ \ 0 \le v \le 1\,.
\end{gather*}

Сначала найдем выражение для $N(u,v)$. Умножая уравнения~(\ref{e6-r})---(\ref{e9-r})
на~$u^i$ и~$v^j$ и суммируя по всем значениям $i=0,1,\dots$ и $j=0,1,\dots$,
после элементарных выкладок получаем:
\begin{multline}
N(u,v)={}\\
{}=\left((\lambda^- v^2 - \lambda^- uv -  \mu_2 u ) S_1(v) - \mu_1 u S_0(v) + {}\right.\\
\!\!\!\!\!\left.{}+ \lambda u p_0
\vphantom{v^2}\right)\Big /
\left({\lambda u^2v - (\lambda + \mu_2 + \lambda^-)uv + \lambda^- v^2
}\right),\!\!
\label{e12-r}
\end{multline}
где
$$
S_0(v)=\sum\limits_{j=0}^{\infty}
p_{0,j,0} v^j\,; \quad
S_1(v)=\sum\limits_{j=0}^{\infty}
p_{0,j,1} v^j.
$$

Знаменатель в выражении~(\ref{e12-r}) представляет собой квадратный
трехчлен по~$u$, корни которого имеют вид:
\begin{multline*}
%\label{e13-r}
u_{1,2}=u_{1,2}(v)={}\\
{}=\fr{\lambda + \mu_2 + \lambda^-
\pm
\sqrt{(\lambda + \mu_2 + \lambda^-)^2-4\lambda \lambda^- v}
}{2 \lambda}\,,
\end{multline*}
причем $0<u_2<1<u_1$ при $0<v\le 1$. Поскольку ПФ $N(u,v)$ является непрерывной
функцией в области $\{0\le u \le 1,\ 0\le v \le 1\}$, то в точке $(u_2,v)$ вместе
со знаменателем должен обращаться в нуль и числитель. Отсюда приходим к равенству
\begin{multline}
\label{e14-r}
(\lambda^- v^2 - \lambda^- u_2 v -  \mu_2 u_2 )
S_1(v)-\mu_1 u_2 S_0(v)+ {}\\
{}+\lambda u_2 p_0 =0\,.
\end{multline}

Для нахождения выражения для $P(u,v)$ умножим уравнения~(\ref{e2-r})--(\ref{e5-r})
на~$u^i$ и~$v^j$ и просуммируем по всем значениям $i=0,1,\dots$ и $j=0,1,\dots$.
В~итоге получаем:
\begin{multline}
\label{e12-rp}
P(u,v)=\left((\lambda^- v + \mu_1 - \lambda^- u )
S_0(v) -\mu_2 N(u,v)
+{}\right.\\
\left.{}+\mu_2 S_1(v) -  \lambda u p_0\right)\Big /\left(
\lambda u^2 - (\lambda + \mu_1 + \lambda^-)u +{}\right.\\
\left.{}+ \mu_1 + \lambda^- v\right)\,.
\end{multline}

Корни квадратного трехчлена по~$u$ в знаменателе~(\ref{e12-rp})
имеют вид:
\begin{multline*}
%\label{e13-rp)}
u_{3,4}=u_{3,4}(v)={}\\
\!{}=\fr{\lambda + \mu_1 + \lambda^-\!\pm\!\sqrt{(\lambda + \mu_1 + \lambda^-)^2-4\lambda 
(\mu_1+\lambda^- v)}}{2 \lambda},\hspace*{-5pt}
\end{multline*}
причем $0<u_4<1<u_3$ при $0\le v < 1$. Поскольку ПФ $P(u,v)$ является непрерывной
функцией в области $\{0\le u \le 1\,,\ 0\le v \le 1\}$, то в точке $(u_4,v)$ вместе
со знаменателем должен обращаться в нуль и числитель, т.\,е.\
\begin{multline*}
%\label{e14-rp}
(\lambda^- v + \mu_1 - \lambda^- u_4 )S_0(v)
-\mu_2 N(u_4,v)+\mu_2 S_1(v) - {}\\
{}- \lambda u_4 p_0 =0\,.
\end{multline*}

Подставляя в последнее равенство вид $N(u_4,v)$ из~(\ref{e12-r}), 
после арифметических преобразований получаем:
\begin{multline}
\label{e15-rp}
\mu_2 (\lambda v u_4 - (\lambda +\mu_2)v + \mu_2)
S_1(v)-{}\\
{}-\lambda p_0 (\lambda v (u_4-u_2)(u_4-u_1)+\mu_2)-{}\\
{}- 
(\lambda v (u_4-u_2)(u_4-u_1)(\lambda^- - \lambda u_3)-{}\\
{}- \mu_1 \mu_2) S_0(v) = 0\,.
\end{multline}
Напомним, что здесь $u_i=u_i(v)$, $i =\overline{1,4}$.
Решая систему из двух уравнений~(\ref{e14-r}) и~(\ref{e15-rp}),
находим выражения для~$S_0(v)$ и~$S_1(v)$ с точностью до
вероятности~$p_0$:
\begin{multline}
\label{e15-rp1}
S_0(v) = \lambda p_0 \left ( \vphantom{\lambda^-_2}
\mu_2 u_2 (\mu_2 + (\lambda u_4 - \lambda - \mu_2)v)
- {}\right.\\
\left.{}-
(\mu_2+\lambda v (u_4-u_2)(u_4-u_1)) \left(\lambda^- v (u_2 -v) +{}\right.\right.\\
\left.\left.{}+ \mu_2 u_2\right)
\right )
\left (
\left(\lambda v (u_4-u_2)(u_4-u_1)(\lambda^- - \lambda u_3) - {}\right.\right.\\
\left.{}-\mu_1 \mu_2\right)
(\lambda^- v (u_2 -v) + \mu_2 u_2)
+{}
\\
\left.{}+
\mu_1 \mu_2 u_2 (\mu_2 + (\lambda u_4 - \lambda - \mu_2)v)
\vphantom{\lambda_2^-}
\right )^{-1}\,;
\end{multline}

\vspace*{-6pt}

\noindent
\begin{multline}
\label{e15-rp2}
S_1(v) = \fr{ \mu_1 u_2}{\lambda^- v^2 - \lambda^- u_2 v -  \mu_2 u_2}\,
S_0(v) -{}\\
{}- \fr{\lambda u_2}{\lambda^- v^2 - \lambda^- u_2 v -  \mu_2 u_2}\,
p_0 \,.
\end{multline}

Подставляя выражения для $S_0(v)$ и~$S_1(v)$ в~(\ref{e12-r})
и~(\ref{e12-rp}), получаем выражения для ПФ $N(u,v)$ и~$P(u,v)$
с точностью до вероятности~$p_0$, которая будет определена далее.

Для нахождения вероятности~$p_0$ сделаем ряд предварительных выкладок.
Введем следующие обозначения:
\begin{align*}
%\label{e17-r}
\tilde{p}_{n,0}&=\sum_{i+j=n-1} p_{i,j,0}\,, \quad n \ge 1\,; \ \\
\tilde{p}_{n,1}&= \sum_{i+j=n-1} p_{i,j,1}\,, \quad n \ge 1\,.
\end{align*}

Очевидно, что вероятность $\tilde{p}_{n,0}$ ($\tilde{p}_{n,1}$)
обозначает вероятность того, что общее число заявок в системе равно~$n$ 
и на приборе обслуживается заявка из накопителя (бункера).
Суммируя последовательно для каждого $n=1,2,\dots$ соответствующие
уравнения системы уравнений равновесия при $i+j=n-1$, приходим к следующей системе уравнений:
\begin{equation}
\left.
\begin{array}{rlr}
\!\!\!\!\!\!\lambda p_0 &= \mu_1 \tilde{p}_{1,0} + \mu_2 \tilde{p}_{1,1}\,, & n =0\,;\\
\!\!\!\!\!\!\lambda (\tilde{p}_{n,0} + \tilde{p}_{n,1}) &=&\\
&\!\!\!\!{}= \mu_1 \tilde{p}_{n+1,0} + \mu_2 \tilde{p}_{n+1,1}\,, & \ \ n \ge 1\,.\!   
\end{array}
\right\}\!\!
\label{e18-r}
\end{equation}

Если теперь просуммировать все уравнения~(\ref{e18-r}), получается
\begin{equation*}
\label{e19-r}
\lambda = \mu_1 \tilde{p}_{\cdot,0} + \mu_2 \tilde{p}_{\cdot,1},
\end{equation*}
которое с учетом того, что $\tilde{p}_{\cdot,0}=p_{\cdot, \cdot, 0}$ и
$\tilde{p}_{\cdot,1}=p_{\cdot, \cdot, 1}$, примет вид:
\begin{equation}
\label{e20-r}
\lambda = \mu_1 p_{\cdot, \cdot,0} + \mu_2 p_{\cdot, \cdot,1}\,,
\end{equation}
где символ <<$\cdot$>> обозначает суммирование по
всем значениям дискретного аргумента.

Это равенство свидетельствует о том, что, как и должно быть, в стационарном
режиме среднее число заявок, принятых в систему за единицу времени,
равно среднему числу заявок, обслуженных системой в единицу времени.

Заметим, что для системы $M|H_2|1|\infty$ равенство,
связывающее интенсивность принятого в систему потока и интенсивность
обслуженного потока полностью повторяет равенство~(\ref{e20-r})
(см., например,~[11, с.~176]).
Отличие состоит только в том, что для системы $M|H_2|1|\infty$
в равенстве, аналогичном~(\ref{e20-r}),
вместо вероятностей того, что прибор занят обслуживанием заявки
из накопителя и из бункера, стоят вероятности того, что прибор обслуживает
заявку на первой и на второй фазе прибора.
Отсюда следует, что соответствующие вероятности совпадают.

Более того, для рассматриваемой системы и для системы $M|H_2|1|\infty$
совпадают вероятности простоя, поскольку равенство~(\ref{e20-r})
с помощью условия нормировки~(\ref{e10-r}) приводится к виду:
\begin{equation}
\label{e21n-r}
\lambda= \mu_1 (1-p_0) + (\mu_2-\mu_1) p_{\cdot, \cdot, 1}\,.
\end{equation}

Таким образом, вероятность простоя~$p_0$ для рассматриваемой системы
находится по той же формуле, что и вероятность простоя для СМО $M|H_2|1|\infty$,
т.\,е.\
\begin{equation}
\label{e22-r}
p_0= 1-\beta_1 \fr{\lambda}{\mu_1} - \beta_2 \fr{\lambda}{\mu_2}\,,
\end{equation}
где $\beta_1$~--- вероятность того, что прибор занят обслуживанием
заявки из накопителя, а $\beta_2=1-\beta_1$~--- вероятность того, что прибор
занят обслуживанием заявки из бункера.

Покажем, как найти эти вероятности. Вероятность $\beta_2$
совпадает с вероятностью того, что поступившая в систему заявка
будет перемещена в бункер. Поступающая с интенсивностью~$\lambda^-$ отрицательная заявка
<<убивает>> заявку в накопителе, если он не пуст, что
происходит с вероятностью $(1-p_0-p_{0,\cdot,0} - p_{0,\cdot,1})$.
Учитывая, что в единицу времени в систему поступает $\lambda$ заявок, имеем
\begin{equation*}
%\label{e21-r}
\beta_2= \fr{\lambda^- (1-p_0-p_{0,\cdot,0} - p_{0,\cdot,1})}{\lambda}\,.
\end{equation*}

Подставляя в~(\ref{e22-r}) выражение для~$\beta_{1}$ и~$\beta_{2}$,
получаем
\begin{multline}
\label{e23-r}
p_0 \left ( 1+ \fr{\lambda^- }{\mu_1} - \fr{\lambda^-}{\mu_2}
\right ) =
1-\fr{\lambda}{\mu_1} +{}\\
{}+ \fr{\lambda^- (\mu_2 - \mu_1)}{\mu_1 \mu_2} (1- p_{0,\cdot,0} - p_{0,\cdot,1})\,.
\end{multline}

Система из четырех уравнений: (\ref{e12-r}) при $u=v\hm=1$, (\ref{e15-rp2}) при $v\hm=1$,
(\ref{e21n-r}) и (\ref{e23-r})~--- является не\-вы\-рож\-ден\-ной, и ее решение имеет вид
\begin{multline*}
%\label{e24-r}
p_0 =1-\fr{ \lambda}{\mu_1}+{}\\
{}+
\lambda^2 \lambda^- \left(\mu_2 - \mu_1\right)\!\Bigg /\!\left(\mu_1^2 \mu_2
\left(
\left(\mu_2 - \mu_1\right) \fr{u_2(1)}{(u_2(1) - 1)} +{}\right.\right.\\
\!\!\!\left.\left.{}+\lambda^-
\vphantom{\fr{u_2(1)}{(u_2(1) - 1)}}\right)
+\mu_1\left(\mu_1^2 \mu_2 + \lambda^- (\mu_2 - \mu_1) (\mu_1 + \lambda)\right)
\vphantom{\fr{u_2(1)}{(u_2(1) - 1)}}\right)\!\!
\,;
\end{multline*}

\vspace*{-6pt}

\noindent
\begin{multline*}
%\label{e25-r}
p_{0, \cdot, 0}
=
\fr{1}{\lambda^- (\mu_2 - \mu_1)
(\lambda^- - (\mu_2 - \mu_1 + \lambda^- ) u_2(1))
}
\times{}
\\
{}\times
\left (
p_0 \lambda^- (\mu_2 - \mu_1) ((\mu_2 + \lambda + \lambda^- ) u_2(1)- \lambda^- )
+{}\right.\\
+
(
\lambda^- u_2(1) + \mu_2 u_2(1) - \lambda^- )\left(
p_0 \mu_1 \mu_2 + \lambda \mu_2 - {}\right.\\
\left.\left.{}-\mu_1 \mu_2 -
 \lambda^- (\mu_2 - \mu_1)\right)
\right)
\,;
\end{multline*}

\vspace*{-6pt}

\noindent
\begin{multline*}
%\label{e15-r}
p_{0, \cdot,1} = \fr{ \mu_1 u_2(1)}
{\lambda^- - \lambda^- u_2(1) -  \mu_2 u_2(1)} p_{0, \cdot,0} -{}\\
{}- \fr{\lambda u_2(1)}
{\lambda^- - \lambda^- u_2(1) -  \mu_2 u_2(1) } p_0\,,
\end{multline*}

\noindent
\begin{equation*}
\label{e16-r}
p_{\cdot, \cdot,1} = p_{0, \cdot,1}
+\fr{\mu_1}{\mu_2} p_{0, \cdot,0} - \fr{\lambda}{\mu_2} p_0\,.
\end{equation*}

Нетрудно видеть, что если $\mu_1=\mu_2=\mu$, т.\,е.\
имеет место одинаковая интенсивность обслуживания заявок как из накопителя, так и из бункера,
то вероятность простоя $p_0=1- \lambda / \mu$, что совпадает с вероятностью простоя
 в системе из~\cite{mandzo}.

Зная вероятность~$p_0$, можно выписать алгоритм
расчета совместного стационарного распределения $p_{i,j,k}$:
\begin{itemize}
%\setcounter{cyritem}{0}
\item сначала находится вероятность $p_{0,0,0}$ по
формуле, которая следует из~(\ref{e15-rp1}) при $v = 0$:

\noindent
\begin{multline*}
\!\!\!\!\!p_{0,0,0} \!=\! \left(\!\lambda p_0 \left( u^{'}_2(0) \left[
\vphantom{\lambda_2^-}
 \lambda u_4(0) ( 1 - u_4(0) + u_1(0) )
 -{}\right.\right.\right.\hphantom{-4.68806pt}\\
\!\!\!\!\!\left.\left.\left.{}-\lambda - \mu_2 -  \lambda^- \right] + \lambda^-
\vphantom{u_2^{'}}\right)\right)\!\!\Big /\!\!\left( u^{'}_2(0)
\left[ \vphantom{\lambda^-}
\lambda u_4(0) \left(u_4(0) -{}\right.\right.\right.
\end{multline*}

\noindent
\begin{multline*}
\left.\left.\left.{}- u_1(0)\right) \left(\lambda^- - \lambda u_3(0)\right)
+
\mu_1 \left(\lambda u_4(0) - \lambda -{}\right.\right.\right.\\
\left.\left.\left.{}- \mu_2 -  \lambda^- \right)
\right] +
\mu_1 \lambda^- 
\vphantom{u_2^{'}}\right)\,;
\end{multline*}
\item затем, используя~(\ref{e1-r}), находится вероятность $p_{0,0,1}$:
$$ p_{0,0,1}
= \fr{\lambda p_0 - \mu_1 p_{0,0,0}
}{\mu_2 }\,;
$$
\item далее из~(\ref{e7-r}) находятся вероятности $p_{i,0,1}$, $i \ge 1$:
$$
p_{i,0,1}= \delta^i p_{0,0,1}\,; \enskip \delta=\fr{\lambda}{\lambda+\mu_2+\lambda^- }\,;
$$
\item используя~(\ref{e1-r}), (\ref{e2-r}) и~(\ref{e7-r}), находится
вероятность~$p_{1,0,0}$:
$$
p_{1,0,0}= \fr{\lambda p_{0,0,0} - \mu_2 (1+\delta) p_{0,0,1}}{\mu_1}\,;
$$
\item из (\ref{e3-r}) вычисляются вероятности $p_{i,0,0}$, $i \ge 2$:
\begin{multline*}
p_{i,0,0}= \left ( \fr{ \lambda + \mu_1 + \lambda^-}{\mu_1}
\right )
p_{i-1,0,0}
-{}\\
{}-\fr{\lambda}{\mu_1} p_{i-2,0,0} 
-
\fr{\mu_2 \delta^i p_{0,0,1}}{\mu_1}\,;
\end{multline*}
\item далее из~(\ref{e15-rp1}) и~(\ref{e15-rp2}) находятся
вероятности $p_{0,j,0}$,  $j \ge 1$, и $p_{0,j,1}$, $ j \ge 1$,
по формулам:
\begin{align*}
p_{0,j,0} &= \fr{1}{j!} 
\fr{d^{j}S_0(v) }{d v^{j}}
\Big|_{v=0}
\,; \\
p_{0,j,1}&= \fr{1}{j!}
\fr{ d^{j}S_1(v)}{d v^{j}}
\Big|_{v=0}\,;
\end{align*}
\item затем из~(\ref{e9-r}) последовательно
для каждого $j\hm=1, 2, \dots$ вычисляются вероятности $p_{i,j,1}$, $i \hm\ge 1$:
\begin{multline*}
p_{i,j,1}=\fr{\lambda^- }{\lambda+\mu_2+\lambda^-}\, p_{i+1,j-1,1} + {}\\
{}+
\fr{\lambda}{(\lambda+\mu_2+\lambda^-)} \,p_{i-1,j,1}\,;
\end{multline*}
\item из~(\ref{e4-r}) находятся вероятности $p_{1,j,0}$, $j \ge 1$:
$$
p_{1,j,0}= \fr{\lambda+\mu_1}{\mu_1}\, p_{0,j,0} - \fr{\lambda^-}
{\mu_1}\, p_{1,j-1,0} - \fr{\mu_2}{\mu_1} \,p_{1,j,1}\,;
$$
\item последними, из~(\ref{e5-r}), последовательно
для каж\-до\-го $j=1, 2, \dots$ находятся вероятности $p_{i,j,0}$, $i \ge 2$:
\begin{multline*}
p_{i,j,0}= \left((\lambda+\mu_1+\lambda^-) p_{i-1,j,0} - \lambda p_{i-2,j,0} - {}\right.\\
\left.{}-
\lambda^- p_{i,j-1,0} - \mu_2 p_{i,j,1}\right)\big /\mu_1\,.
\end{multline*}
\end{itemize}

Таким образом, совместное стационарное распределение
найдено. Поскольку заявки поступают в систему с интенсивностью~$\lambda$,
а обслуживаются с интенсивностью
$\mu^{*}= \left ( {\beta_1}/{\mu_1} + {\beta_2}/{\mu_2} \right )^{-1}$,
неравенство $\lambda \hm< \mu^{*}$ является необходимым и достаточным условием
для его существования.

\section{Заключение}

В статье представлен анализ СМО с отрицательными заявками
и бункером для вытесненных заявок, в которой заявки из накопителя
и бункера обслуживаются с различными интенсивностями.
Найдено совместное стационарное распределение числа
заявок в накопителе и бункере как в терминах производящих функций,
так и в терминах вычислительных алгоритмов.

С помощью программных средств GPSS 
была разработана имитационная модель системы. Результаты моделирования показали
хорошие совпадения с результатами численных расчетов, проведенных по полученным формулам.

\bigskip
Автор глубоко признателен профессору А.\,В.~Печинкину за постановку задачи, 
ряд ценных замечаний и помощь при оформлении статьи.

{\small\frenchspacing
{%\baselineskip=10.8pt
\addcontentsline{toc}{section}{Литература}
\begin{thebibliography}{99}

\bibitem{bib0}
\Au{Бочаров П.\,П., Вишневский В.\,М. }
G-сети: развитие теории мультипликативных сетей~//
Автоматика и телемеханика, 2003. №\,5. C.~70--74.


\bibitem{bib1}
\Au{Muthu Ganapathi Subramanian A., Ayyappan~G., Gopal Sekar}.
$M|M|1$ retrial queueing system with negative arrival under
non-pre-emptive priority service~// J.~Fundamental Sciences,
2009. Vol.~5. No.\,2. P.~129--145.

\bibitem{bib2}
\Au{D'Apice C., Manzo R., Pechinkin~A., Shorgin~S.}
Queueing network with negative customers and the route change~//
 Conference (International) on Ultra Modern Telecommunications Proceedings, 2009. P.~1--5.

\bibitem{bib3}
\Au{Pechinkin A., Razumchik~R.}
A queueing system with negative claims and a bunker for superseded claims in discrete time~//
Automation and Remote Control, 2009. Vol.~70. No.\,12. P.~109--120.

\bibitem{bib4} %5
\Au{Ayyappan G., Gopal Sekar, Muthu Ganapthi Subramanian~A.}
$M|M|1$ retrial queueing system with negative
arrival under erlang-k service by matrix
geometric method~//
Appl. Math. Sci., 2010. Vol.~4. No.\,48. P.~2355--2367.

\bibitem{bib6} %6
\Au{Pechinkin A.\,V., Razumchik~R.\,V.}
Waiting characteristics of queueing system $Geo|Geo|1$ with negative claims and a bunker for superseded claims in discrete 
time~//
Conference (International) on Ultra Modern Telecommunications Proceedings, 2010. P.~1051--1055.

\bibitem{bib5} %7
\Au{Krishna Kumar B., Pavai Madheswari~S., Anantha Lakshmi~S.\,R.}
An $M/G/1$ Bernoulli feedback retrial queueing system with negative customers~//
Operational Res., 2011. Vol.~1. P.~1--24.

\bibitem{bib7}
\Au{Songfang Jia, Yanheng Chen}.
A discrete time queueing system
with negative customers and single working vacation~// 3rd
 Conference (International) on Computer Research and Development
(ICCRD) Proceedings, 2011. Vol.~4. P.~15--19.

\bibitem{bib8}
\Au{Tien Van Do}. A new solution for a queueing model of a
manufacturing cell with negative customers under a rotation rule~//
J.~Performance Evaluation, 2011. Vol.~68. Issue~4. P.~330--337.

\bibitem{mandzo}
\Au{Мандзо~Р., Касконе~Н., Разумчик Р.\,В.}
Экспоненциальная система массового обслуживания с отрицательными заявками и
бункером для вытесненных заявок~// Автоматика и телемеханика, 2008. №\,9. C.~103--113.

\label{end\stat}

\bibitem{tmo}
\Au{Бочаров П.\,П., Печинкин А.\,В.}
Теория массового обслуживания.~--- М.: РУДН, 1995. 529~с.




 \end{thebibliography}
}
}


\end{multicols}       