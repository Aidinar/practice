
\def\stat{ush1+ush}

\def\tit{ОБ ОДНОЙ ЯДЕРНОЙ ОЦЕНКЕ ПЛОТНОСТИ$^*$}

\def\titkol{Об одной ядерной оценке плотности}

\def\autkol{В.\,Г.~Ушаков, Н.\,Г.~Ушаков}
\def\aut{В.\,Г.~Ушаков$^1$, Н.\,Г.~Ушаков$^2$}

\titel{\tit}{\aut}{\autkol}{\titkol}

{\renewcommand{\thefootnote}{\fnsymbol{footnote}}\footnotetext[1]
{Работа поддержана Российским фондом фундаментальных исследований
(проекты 11-01-00515а и 11-07-00112а), а также Министерством
образования и науки РФ в рамках ФЦП <<Научные и
научно-педагогические кадры инновационной России на 2009--2013~годы>>.}}


\renewcommand{\thefootnote}{\arabic{footnote}}
\footnotetext[1]{Московский государственный 
университет им.\ М.\,В.~Ломоносова, факультет вычислительной математики и кибернетики; Институт проблем информатики Российской
академии наук, vgushakov@mail.ru}
\footnotetext[2]{Институт проблем технологии
микроэлектроники и особочистых материалов Российской академии наук, ushakov@math.ntnu.no}

\Abst{Исследуется ядерная оценка плотности
распределения, основанная на так называемом синк-ядре.
Основное внимание уделено анализу среднеквадратической ошибки оценки
при конечных объемах выборки.
Рассмотрены проблемы оценивания моды и производных плотности.}

\KW{непараметрическое оценивание
плотности; ядерная оценка; ядро бесконечного порядка}

  \vskip 14pt plus 9pt minus 6pt

      \thispagestyle{headings}

      \begin{multicols}{2}
      
            \label{st\stat}

\section{Введение}

Непараметрическое оценивание плотности вероятностных распределений с помощью 
ядерных оценок является одной из важнейших задач интеллектуального
анализа данных, в частности при анализе трафика в телекоммуникационных системах.

Рассмотрим выборку $X_1,\ldots,X_n$, состоящую из $n$ независимых наблюдений,
имеющих одинаковые распределения. Всюду в настоящей работе будем
предполагать, что $X_k$ имеет абсолютно непрерывное распределение.
Обозначим функцию распределения, плотность распределения и характеристическую
функцию соответственно~$F(x)$, $f(x)$ и $\varphi(t)$. Ядерной оценкой
плотности распределения~$f(x)$, построенной по выборке $X_1,\ldots,X_n$,
называется случайная функция
\begin{equation*}
\hat f_n(x;h)=\fr{1}{nh}\sum\limits_{k=1}^nK\left(\fr{x-X_k}{h}\right)\,,
%\label{e1-u1}
\end{equation*}
где $K(x)$~--- некоторая функция, называемая \mbox{ядром}; $h$~--- параметр
сглаживания (неотрицательное\linebreak чис\-ло).

Обычно в качестве ядра выбирается плотность распределения вероятностей, т.\,е.\
неотрицательная интегрируемая функция такая, что
$\int\limits_{-\infty}^\infty K(x)\,dx=1$. В~данной работе, однако, будем иметь дело
с нестандартным ядром, а именно:
\begin{equation}
K(x)=\fr{\sin x}{\pi x}\,,
\label{e2-u1}
\end{equation}
чье преобразование Фурье (характеристическая функция) равно

\noindent
$$
\psi(t)=\begin{cases}
1\,,& \mbox{\ если\ } |t|\le 1\,;\\
0\,,& \mbox{\ если\ } |t|>1\,.
\end{cases}
$$
Это так называемое синк-ядро. Всюду в данной работе будем обозначать ядерную оценку с
ядром~(\ref{e2-u1}) как~$f_n(x;h)$. Оценка $f_n(x;h)$ изучалась в работах~[1, 2]. 
В~[1] она исследовалась для класса плотностей, чьи
характеристические функции убывают регулярно при $|t|\to\infty$.
Было показано, что при экспоненциальном убывании скорость сходимости
к нулю среднеквадратической ошибки отличается от $1/n$ лишь медленно
меняющейся функцией, т.\,е.\ ядро~(\ref{e2-u1}) имеет бесконечный порядок. Если
$|\varphi(t)|$ убывает <<алгебраически>>: $|\varphi(t)|\sim
1/{t^p}$, $t\to\infty$, то, как было показано в~[1], при $p>5$
оценка $f_n(x;h)$ имеет лучший порядок состоятельности, чем ядерные
оценки с ядрами, являющимися плотностями распределения. В~[2]
доказано, что оценка $f_n(x;h)$ имеет оптимальный порядок состоятельности.

Несмотря на очень хорошие асимптотические свойства, оценка с ядром~(\ref{e2-u1})
не нашла пока еще широкого применения. Одной из причин является то,
что, поскольку ядро принимает отрицательные значения и
неинтегрируемо, реализации соответствующей ядерной оценки не
являются плотностями распределения. Этот дефект, однако, легко может быть
исправлен без потери точности~[3]. Другая причина: не вполне понятно,
при каких объемах выборки проявляются асимптотические преимущества
оценки $f_n(x;h)$. Многие считают, что эти объемы должны быть очень велики.
В настоящей работе проводится более углубленное исследование оценки с ядром~(\ref{e2-u1}).
Получены неравенства для интегральной среднеквадратической ошибки,
позволяющие оценить величину ошибки при конечных объемах выборки
(в том числе в случае оценивания производных плотности).
Кроме того, исследуется равномерная сходимость оценки и
проблема оценивания моды плотности.

Пусть $\hat f(x)$~--- какая-либо оценка плотности~$f(x)$.
В качестве меры точности оценки будем использовать интегральную
среднеквадратическую ошибку, которую будем обозначать $J(\hat f)$ и
которая определяется следующим образом:
$$
J(\hat f)={\rm E}\int\limits_{-\infty}^\infty\left[\hat f(x)-f(x)\right]^2\,dx\,.
$$
Для ядерной оценки $f_n(x;h)$ с ядром~(\ref{e2-u1}) $J(f_n)$ может быть записана
как (см.~[2])

\noindent
\begin{multline}
J(f_n)=\fr{1}{2\pi}\int\limits_{|t|>1/h}|\varphi(t)|^2\,dt+{}\\
{}+
\fr{1}{n}\,\fr{1}{2\pi}
\int\limits_{-1/h}^{1/h}(1-|\varphi(t)|^2)\,dt\,.
\label{e3-u1}
\end{multline}
Отметим, что первое слагаемое в правой части является интегрированным
квадратом смещения оценки, а второе~--- интегрированной дисперсией.
Всюду в данной работе будем предполагать, что квадрат оцениваемой
плотности интегрируем. В~противном случае $J(f_n)=\infty$ и исследование
оценки на основе интегральной среднеквадратической ошибки становится бессмысленным.

Ядра, являющиеся плотностями распределения вероятностей, будем для
краткости называть стандартными ядрами.

\vspace*{-3pt}

\section{Неравенства}

\vspace*{-1pt}

Пусть $\tilde f(x)$~--- какая-либо оценка плотности~$f(x)$. 
Обозначим интегрированный квадрат смещения и интегрированную
дисперсию соответственно ${\rm B}(\tilde f)$ и ${\rm V}(\tilde f)$, т.\,е.\

\noindent
\begin{align*}
{\rm B}(\tilde f)&=\int\limits_{-\infty}^\infty\left[{\rm E}\tilde f(x)-f(x)\right]^2\,dx\,;\\
{\rm V}(\tilde f)&=
\int\limits_{-\infty}^\infty\left[\tilde f(x)-{\rm E}\tilde f(x)\right]^2\,dx\,.
\end{align*}
В этом параграфе будут получены верхние оценки интегральной среднеквадратической
ошибки, которые позволяют оценить реальный уровень погрешности
при конечных объемах выборки.
Определим нулевую производную функции как саму функцию. Ниже будем
использовать следующий вариант равенства Парсеваля.
Предположим, что плотность распределения~$f(x)$ $m$~раз дифференци-\linebreak\vspace*{-12pt}
\columnbreak

\noindent
руема,
$m\ge0$, а квадрат $m$-й производной интегрируем.
Пусть $\varphi(t)$~--- соответствующая характеристическая функция.
Тогда
\begin{equation}
\int\limits_{-\infty}^\infty(f^{(m)}(x))^2\,dx=
\fr{1}{2\pi}\int\limits_{-\infty}^\infty t^{2m}|\varphi(t)|^2\,dt\,.
\label{e4-u1}
\end{equation}

Пусть $g(x)$~--- некоторая функция, квадрат которой интегрируем. Обозначим
$R(g)=\int\limits_{-\infty}^\infty g^2(x)\,dx$.

\medskip

\noindent
\textbf{Теорема 1.} \textit{Пусть $m$-я производная плотности~$f(x)$
существует, а ее квадрат интегрируем. Тогда}
\begin{equation}
J(f_n)<\varepsilon(h)h^{2m}R(f^{(m)})+\fr{1}{\pi nh}\,,
\label{e5-u1}
\end{equation}
\textit{где $\varepsilon(h)\le1$ для всех~$h$ и $\varepsilon(h)\to0$ при $h\to0$.}

\medskip

\noindent
Д\,о\,к\,а\,з\,а\,т\,е\,л\,ь\,с\,т\,в\,о\,.\ 
Оценим сначала первое слагаемое в правой части~(\ref{e3-u1}).
Применяя~(\ref{e4-u1}), получаем:
\begin{multline*}
\fr{1}{2\pi}\int\limits_{|t|>1/h}|\varphi(t)|^2\,dt={}\\
{}=
h^{2m}\fr{1}{2\pi}\int\limits_{|t|>1/h}(1/h)^{2m}|\varphi(t)|^2\,dt\le{}\\
{}\le h^{2m}\fr{1}{2\pi}\int\limits_{|t|>1/h}t^{2m}|\varphi(t)|^2\,dt={}\\
{}=
h^{2m}\fr{1}{2\pi}\int\limits_{-\infty}^\infty t^{2m}|\varphi(t)|^2\,dt-{}\\
{}-
h^{2m}\fr{1}{2\pi}\int\limits_{-1/h}^{1/h} t^{2m}|\varphi(t)|^2\,dt
={}\\
{}=
h^{2m}\fr{1}{2\pi}\int\limits_{-\infty}^\infty t^{2m}|\varphi(t)|^2\,dt\times{}\\
{}\times
\left(1-\fr{\int\limits_{-1/h}^{1/h}
t^{2m}|\varphi(t)|^2\,dt}{\int\limits_{-\infty}^\infty t^{2m}|\varphi(t)|^2dt}
\right)={}\\
{}=\varepsilon(h)h^{2m}\int\limits_{-\infty}^\infty(f^{(m)}(x))^2\,dx=
\varepsilon(h)h^{2m}R(f^{(m)})\,,
\end{multline*}
где
$$
\varepsilon(h)= 1-\fr{\int\limits_{-1/h}^{1/h}
t^{2m}|\varphi(t)|^2\,dt}{\int\limits_{-\infty}^\infty t^{2m}|\varphi(t)|^2\,dt}
$$
удовлетворяет условиям теоремы:
$\varepsilon(h)\le1$ и $\varepsilon(h)\to0$ при $h\to0$.

Для второго слагаемого в правой части~(\ref{e3-u1}) имеем
$$
\fr{1}{n}\,\fr{1}{2\pi}\int\limits_{-1/h}^{1/h}(1-|\varphi(t)|^2)\,dt<
\fr{1}{n}\,\fr{1}{2\pi}\int\limits_{-1/h}^{1/h}\,dt=\fr{1}{\pi nh}\,.
$$
Таким образом, получаем~(\ref{e5-u1}).

\medskip

\noindent
\textbf{Следствие 1.} \textit{Пусть выполнены условия теоремы~$1$. Тогда}
\begin{equation}
J(f_n)<h^{2m}R(f^{(m)})+\fr{1}{\pi nh}\,.
\label{e6-u1}
\end{equation}

Полагая в~(\ref{e6-u1})
$$
h=\left[\fr{1}{2\pi nmR(f^{(m)})}\right]^{1/(2m+1)}
$$
(это значение $h$ минимизирует правую часть~(\ref{e6-u1})), получаем

\smallskip

\noindent
\textbf{Следствие 2.} \textit{Пусть выполнены условия теоремы~$1$. Тогда}
\begin{multline*}
\inf\limits_{h>0}J(f_n)<{}\\
{}< \fr{1+2m}{(2\pi m)^{2m/(2m+1)}}
R(f^{(m)})^{1/(2m+1)}n^{-2m/(2m+1)}\,.
\end{multline*}

\smallskip

\noindent
\textbf{Следствие 3.} \textit{Пусть выполнены условия теоремы~$1$. Тогда}
$$
\inf\limits_{h>0}J(f_n)=o\left(n^{-2m/(2m+1)}\right)\,,\quad n\to\infty\,.
$$

\smallskip

\noindent
\textbf{Следствие 4.} \textit{Если~$f(x)$ дважды дифференцируема и квадрат
второй производной интегрируем, то}
$$
\inf\limits_{h>0}J(f_n)=o(n^{-4/5})\,,\quad  n\to\infty\,,
$$
и
$$
\inf\limits_{h>0}J(f_n)<\fr{5}{4\pi}\left(4\pi R(f'')\right)^{1/5}n^{-4/5}\,.
$$

\smallskip

Чтобы получить более точные оценки, необходима дополнительная информация
об оцениваемой плотности. Пусть $g(x)$~--- некоторая функция. Обозначим~$Vr(g)$ 
ее полную вариацию.

\medskip

\noindent
\textbf{Теорема 2.} \textit{Если $f(x)$ $m$ раз дифференцируема ($m\ge0$),
и ее $m$-я производная имеет ограниченную полную вариацию, то}
\begin{equation}
J(f_n)\le h^{2m+1}\fr{Vr(f^{(m)})^2}{(2m+1)\pi}+\fr{1}{\pi nh}\,.
\label{e7-u1}
\end{equation}

\medskip

\noindent
Д\,о\,к\,а\,з\,а\,т\,е\,л\,ь\,с\,т\,в\,о\,.\ Для всех~$t$~[4]
$$
\left\vert\varphi(t)\right\vert\le \fr{Vr(f^{(m)})}{|t|^{m+1}}\,.
$$
Применяя это неравенство, оценим первое слагаемое в правой части~(\ref{e3-u1}):
\begin{multline*}
\fr{1}{2\pi}\int\limits_{|t|>1/h}|\varphi(t)|^2\,dt\le
\fr{Vr(f^{(m)})^2}{\pi}\int\limits_{1/h}^\infty \fr{dt}{t^{2m+2}}
={}\\
{}= \fr{Vr(f^{(m)})^2}{(2m+1)\pi}\,h^{2m+1}\,.
\end{multline*}
Для второго слагаемого в правой части~(\ref{e3-u1})
имеем (см.\ доказательство теоремы~1)
$$
\fr{1}{n}\,\fr{1}{2\pi}\int\limits_{-1/h}^{1/h}(1-|\varphi(t)|^2)\,dt<
\fr{1}{\pi nh}\,.
$$
Таким образом, получаем~(\ref{e7-u1}).

\smallskip

\noindent
\textbf{Следствие.} \textit{Если выполнены условия теоремы~$2$, то}
\begin{multline*}
\inf\limits_{h>0}J(f_n)\le{}\\
{}\le \fr{2(m+1)}{(2m+1)\pi}\,Vr(f^{(m)})^{1/(m+1)}n^{-(2m+1)/(2m+2)}\,.
\end{multline*}

\smallskip
Например, если $m=2$, получаем
$$
\inf\limits_{h>0}J(f_n)\le\fr{6}{5\pi}\,Vr(f'')^{1/3}n^{-5/6}\,,
$$
что по порядку лучше, чем в случае стандартных оценок ($n^{-4/5}$).

Следуя Ватсону и Лидбеттеру~[5] и Дэвис~[1] будем говорить, что
характеристическая функция~$\varphi(t)$ убывает экспоненциально с показателем~$\alpha$ 
и коэффициентом~$\rho$ ($\rho>0$, $0<\alpha\le2$), если
\begin{equation}
\left\vert\varphi(t)\right\vert \le Ae^{-\rho|t|^\alpha}\,,
\label{e8-u1}
\end{equation}
где $A$~--- постоянная. В~[1] доказано (теорема~4.1), что если для
характеристической функции оцениваемой плотности выполняется~(\ref{e8-u1}),
то
$$
\lim\limits_{n\to\infty}he^{\rho/h^\alpha}\left\vert{\rm B}(f_n)\right\vert=0\,.
$$
Теорема~3 уточняет это утверждение.

\smallskip

\noindent
\textbf{Теорема 3.} \textit{Если}
$$
\fr{1}{2\pi}\int\limits_{-\infty}^\infty e^{\rho|t|^\alpha}\left\vert\varphi(t)\right\vert^2\,dt
=C<\infty\,,
$$
\textit{то}
\begin{equation}
J(f_n)\le\varepsilon(h)Ce^{-\rho/h^\alpha}+\fr{1}{\pi nh}\,,
\label{e9-u1}
\end{equation}
\textit{где $0<\varepsilon(h)<1$ и $\varepsilon(h)\to0$ при $h\to0$}.
%\pagebreak


%\medskip

\noindent
Д\,о\,к\,а\,з\,а\,т\,е\,л\,ь\,с\,т\,в\,о\ аналогично доказательству\linebreak
теоремы~1.
Для первого слагаемого в правой час\-ти~(\ref{e3-u1}) имеем
\begin{multline*}
\fr{1}{2\pi}\int\limits_{|t|>1/h}|\varphi(t)|^2\,dt<{}\\
{}<
e^{-\rho/h^\alpha}\fr{1}{2\pi}
\int\limits_{|t|>1/h}e^{\rho|t|^\alpha}|\varphi(t)|^2\,dt
=\varepsilon(h)Ce^{-\rho/h^r}\,,
\end{multline*}
где
$$
\varepsilon(h)= 1-\fr{\int\limits_{-1/h}^{1/h}e^{\rho|t|^\alpha}\left\vert\varphi(t)\right\vert^2\,dt}
{\int e^{\rho|t|^\alpha}|\varphi(t)|^2\,dt}\,.
$$
Второе слагаемое оцениваем так же, как в доказательстве теоремы~1.

Трудно найти в явном виде значение~$h$, минимизирующее правую часть~(\ref{e9-u1}), 
поэтому возьмем такое~$h$, при котором правая часть~(\ref{e9-u1}) имеет
простой вид, а именно:
$$
h=\left(\fr{1}{\rho}\,\ln n\right)^{-1/\alpha}\,.
$$
Тогда
\begin{multline*}
J(f_n)< \left(C+\fr{(\ln n)^{1/\alpha}}{\pi\rho^{1/\alpha}}\right)\fr{1}{n}<{}\\
{}<
\left(C+\fr{1}{\pi\rho^{1/\alpha}}\right)\fr{(\ln n)^{1/\alpha}}{n}
\end{multline*}
при условии, что $n>2$. Если, например, $f(x)$~--- стандартная нормальная
плотность распределения, то
$$
J(f_n)< \left(\fr{1}{\sqrt{2\pi}}+ \fr{\sqrt 2}{\pi}\right)\fr{\sqrt{\ln
n}}{n}\,.
$$

\medskip

\noindent
\textbf{Следствие.} \textit{Пусть характеристическая функция~$\varphi(t)$ 
убывает экспоненциально с показателем~$\alpha$ и коэффициентом~$\rho$.
Тогда}
$$
\lim\limits_{n\to\infty}e^{c\rho/h^\alpha}\left\vert{\rm B}(f_n)\right\vert=0
$$
\textit{для любого $c<2$}.

\section {Оценивание производных}

Те преимущества, которые имеет оценка $f_n(x;h)$ по сравнению со стандартными
ядерными оценками, особенно проявляются в случае, когда оценивается не
сама плотность, а ее производная. Предположим, что~$f(x)$ является 
$r$~раз дифференцируемой и необходимо оценить
$r$-ю производную $f^{(r)}(x)$. Естественной оценкой является
$r$-я производная ядерной оценки плотности~$f(x)$
(при условии, что ядро $r$~раз дифференцируемо). Далее, пусть
$$
\hat f_n(x;h)=\fr{1}{nh}\sum\limits_{j=1}^nK\left(\fr{x-X_j}{h}\right)
$$
есть ядерная оценка плотности~$f(x)$ и существует производная
$K^{(r)}(x)$. Тогда в качестве оценки производной $f^{(r)}(x)$ берется
$$
\hat f_n^{(r)}(x;h)= \fr{1}{nh^{r+1}}\sum\limits_{j=1}^nK^{(r)}\left(
\fr{x-X_j}{h}\right)\,.
$$
Пусть $J(\hat f_n^{(r)})$~--- интегральная среднеквадратическая ошибка
$\hat f_n^{(r)}(x)$ как оценки производной $f^{(r)}(x)$, т.\,е.\
$$
J(\hat f_n^{(r)})={\rm E}\int\limits_{-\infty}^\infty
\left[\hat f_n^{(r)}(x)-f^{(r)}(x)\right]^2\,dx\,.
$$
Если $K(x)$~--- стандартное ядро, то интегральная среднеквадратическая
ошибка оценки $\hat f_n^{(r)}(x;h)$
может быть записана следующим образом (при условии, что $f(x)$ имеет
$r+2$ производные и дисперсия ядра конечна):
\begin{multline*}
J(\hat f_n^{(r)})=\fr{1}{4}\,h^4\mu_2^2R(f^{(r+2)})+
\fr{1}{nh^{2r+1}}\,R(K^{(r)})+{}\\
{}+
o\left(h^4+\fr{1}{nh^{2r+1}}\right)\,,\quad h\to0\,,
\end{multline*}
где $\mu_2=\int\limits_{-\infty}^\infty x^2K(x)\,dx$.
То есть при оптимальном выборе параметра сглаживания порядок ошибки
равен $n^{-4/(2r+5)}$ и он существенно ухудшается с ростом~$r$.
В~данном параграфе будет показано, что оценка $f_n^{(r)}(x;h)$, основанная
на ядре~(\ref{e2-u1}), при определенных условиях почти свободна от этого недостатка
или во всяком случае в гораздо меньшей степени подвержена ему.

\medskip

\noindent
\textbf{Лемма~1.} \textit{Для синк-оценки}
\begin{multline}
J(f_n^{(r)})=\fr{1}{2\pi}\int\limits_{|t|>1/h}t^{2r}\left\vert\varphi(t)\right\vert^2\,dt+{}\\
{}+
\fr{1}{n}\,\fr{1}{2\pi}\int\limits_{-1/h}^{1/h}t^{2r}\left(1-
\left\vert\varphi(t)\right\vert^2\right)\,dt\,.
\label{e10-u1}
\end{multline}

\medskip

\noindent
Д\,о\,к\,а\,з\,а\,т\,е\,л\,ь\,с\,т\,в\,о\,.\ 
Обозначим $\varphi_n(t)$ выборочную характеристическую
функцию, т.\,е.\
$$
\varphi_n(t)=\fr{1}{n}\sum\limits_{j=1}^ne^{itX_j}\,.
$$
Применяя равенство Парсеваля, получаем
\begin{multline}
J(f_n^{(r)})={\rm E}\int\limits_{-\infty}^\infty(f_n^{(r)}(x;h)-f^{(r)}(x))^2\,dx={}\\
{}=
\fr{1}{2\pi}\,
{\rm E}\int\limits_{-\infty}^\infty
t^{2r}\left\vert\varphi_n(t)I_{[-1/h,1/h]}(t)-\varphi(t)\right\vert^2\,dt={}\\
{}=
\fr{1}{2\pi}\int\limits_{-\infty}^\infty
{\rm E}\left[
\vphantom{\overline{\varphi(t)}}
\left(\varphi_n(t)I_{[-1/h,1/h]}(t)-{}\right.\right.\\
\left.\left.{}-\varphi(t)\right)
(\overline{\varphi_n(t)}\ I_{[-1/h,1/h]}(t)-
\overline{\varphi(t)})\right]\,dt={}\\
{}=\fr{1}{2\pi}\int\limits_{-1/h}^{1/h} t^{2r}{\rm E}|\varphi_n(t)|^2\,dt-{}\\
{}-
\fr{1}{2\pi}\int\limits_{-1/h}^{1/h} t^{2r}
\left(\varphi(t){\rm E}\overline{\varphi_n(t)}+
\overline{\varphi(t)}{\rm E}\varphi_n(t)\right)\,dt+{}\\
{}+
\fr{1}{2\pi}\int\limits_{-\infty}^\infty t^{2r}\left\vert\varphi(t)\right\vert^2\,dt\,.
\label{e11-u1}
\end{multline}
Легко видеть, что
\begin{align}
{\rm E}\varphi_n(t)&=\varphi(t)\,;\label{e12-u1}\\
{\rm E}\overline{\varphi_n(t)}&=\overline{\varphi(t)}\,;\label{e13-u1}\\
{\rm E}\left\vert\varphi_n(t)\right\vert^2&=
{\rm E}\left\vert\fr{1}{n}\sum\limits_{j=1}^ne^{itX_j}\right\vert^2
={}\notag\\
&{}={\rm E}\left[\fr{1}{n}\sum\limits_{j=1}^ne^{itX_j}\cdot\fr{1}{n}
\sum\limits_{k=1}^ne^{-itX_k}\right]={}\notag
\end{align}

\noindent
\begin{align}
&{}=\fr{1}{n^2}\left[n+\sum\limits_{j\not=k}e^{it(X_j-X_k)}\right]={}\notag\\
&\hspace*{15mm}{}=
\fr{1}{n}+\left(1-\fr{1}{n}\right)\left\vert\varphi(t)\right\vert^2\,.
\label{e14-u1}
\end{align}
Подставляя~(\ref{e12-u1})--(\ref{e14-u1}) в правую часть~(\ref{e11-u1}), получаем~(\ref{e10-u1}).

\smallskip

Применяя лемму~1, получаем аналоги теорем предыдущего параграфа для
случая оценивания производных.

\medskip

\noindent
\textbf{Теорема 4.} \textit{Если $f(x)$ $r+m$ раз дифференцируема и квадрат ее
$(r+m)$-й производной интегрируем, то}
$$
J(f_n^{(r)})< \varepsilon(h)h^{2m}R(f^{(r+m)})+\fr{1}{\pi(2r+1)nh^{2r+1}}\,,
$$
\textit{где $\varepsilon(h)\le1$ для всех $h$ и $\varepsilon(h)\to0$ при $h\to0$}.


\medskip

\noindent
\textbf{Следствие.} \textit{Пусть выполнены условия теоремы~$4$. Тогда}
\begin{multline*}
\inf\limits_{h>0}J(f_n^{(r)})
\le {}\\
{}\le C_{m,r}R(f^{(r+m)})^{(2r+1)/(2r+2m+1)}n^{-2m/(2r+2m+1)},\hspace*{-1.15225pt}
\end{multline*}
\textit{где}
\begin{multline*}
C_{m,r}=(2\pi m)^{-2m/(2r+2m+1)}+{}\\
{}+
\fr{(2\pi m)^{(2r+1)/(2r+2m+1)}}{\pi(2r+1)}\,.
\end{multline*}


\medskip

\noindent
\textbf{Теорема 5.} \textit{Пусть}
$$
\fr{1}{2\pi}\int\limits_{-\infty}^\infty
t^{2r}e^{\rho|t|^\alpha}\left\vert\varphi(t)\right\vert^2\,dt=C<\infty\,.
$$
\textit{Тогда}
$$
J(f_n^{(r)})\le\varepsilon(h)
Ce^{-\rho/h^\alpha}+\fr{1}{\pi nh^{2r+1}}\,,
$$
\textit{где $0<\varepsilon(h)<1$ и $\varepsilon(h)\to0$ при $h\to0$}.


\medskip

\noindent
\textbf{Следствие.} \textit{Пусть выполнены условия теоремы~$5$. Тогда}
$$
\inf\limits_{h>0}J(f_n^{(r)})< \left(
C+\fr{(\ln n)^{(2r+1)/\alpha}}{\pi\rho^{(2r+1)/\alpha}}\right)
\fr{1}{n}\,.
$$

\medskip

\noindent
\textbf{Теорема~6.} \textit{Пусть характеристическая функция $\varphi(t)$
плотности~$f(x)$ удовлетворяет условию: существует $T>0$ такое, что
$f(t)=0$ при $|t|>T$. Тогда если}
$$h\le \fr{1}{T}\,,$$
\textit{то}
$$
J(f_n^{(r)})\le\fr{1}{\pi nh^{2r+1}}\,.
$$
\textit{В частности, если $h=const=1/T$, то}
$$
J(f_n^{(r)})\le\fr{T^{2r+1}}{\pi n}\,.
$$

\smallskip

Доказательства теорем~4--6 сходны с доказательствами предыдущего
параграфа, и авторы оставляют их читателю.


\section{Равномерная состоятельность и~оценивание моды}

В данном параграфе будет доказано, что оценка $f_n(x;h)$ равномерно
состоятельна:
она сходится по вероятности к оцениваемой плотности распределения
равномерно на всей действительной прямой. Кроме того, докажем,
что мода оценки является состоятельной оценкой моды оцениваемой
плотности.

Пусть $K(x)$~--- симметричное и дифферен\-ци\-ру\-емое стандартное ядро
(т.\,е.\ ядро, являющееся плотностью распределения вероятностей)
с конечной дисперсией~$\sigma^2$. Предположим, что производная
ядра имеет конечную полную вариацию, которую обозначим~$v$.
Характеристическую функцию ядра и ядерную оценку, основанную
на этом ядре, обозначим соответственно~$\psi(t)$ и~$\hat f_n(x;h)$.

\medskip

\noindent
\textbf{Лемма~2.} \textit{Если характеристическая функция $\varphi(t)$
интегрируема:}
$$
\int\limits_{-\infty}^\infty\left\vert\varphi(t)\right\vert\,dt<\infty\,,
$$
\textit{то}
$$\sup\limits_x|f_n(x;h)-\hat f_n(x;h)|\xrightarrow{a.s.} 0$$
\textit{при $n\hm\to\infty$, $h\hm\to0$, $nh\hm\to\infty$}.

\medskip

\noindent
Д\,о\,к\,а\,з\,а\,т\,е\,л\,ь\,с\,т\,в\,о\,.\
\begin{multline*}
\sup\limits_x\left\vert f_n(x;h)-\hat f_n(x;h)\right\vert\le{}\\
{}\le\fr{1}{2\pi}
\left[ \vphantom{\int\limits_{-1/h}^{1/h}}
\int\limits_{-\infty}^\infty|\psi(ht)|\cdot|\varphi_n(t)-\varphi(t)|\,dt+{}\right.\\
\left.{}+
\int\limits_{-\infty}^\infty|\varphi(t)|\cdot|\psi(ht)-I_{[-1/h,1/h]}(t)|\,dt+{}\right.\\
\left.{}+\int\limits_{-1/h}^{1/h}\left\vert \varphi_n(t)-\varphi(t)\right\vert\,dt
\right]\,.
\end{multline*}
Докажем, что каждый из трех интегралов в правой части сходится к нулю
при $n\hm\to\infty$, $h\hm\to0$, $nh\hm\to\infty$. Обозначим эти интегралы соответственно
$I_1$, $I_2$ и~$I_3$. Тогда

\noindent
$$
I_1\le\int\limits_{|t|\le n^2}\left\vert \varphi_n(t)-\varphi(t)\right\vert\,dt+
4\int\limits_{n^2}^\infty|\psi(ht)|\,dt\,.
$$
Первый интеграл в правой части сходится почти наверное к нулю при
$n\hm\to\infty$ в силу теоремы~1 из~[6]. Чтобы оценить второй интеграл,
применим неравенство

\noindent
$$
|\psi(t)|\le\fr{v}{|t|^2}\,,
$$
справедливое для всех~$t$~[4]. Применяя это неравенство, получаем

\noindent
$$
\int\limits_{n^2}^\infty|\psi(ht)|\,dt\le\fr{v}{n^2h^2}\to 0
$$
при $nh\to\infty$.
Таким образом, $I_1\xrightarrow{a.s.} 0$ при
$n\hm\to\infty$, $nh\hm\to\infty$;

\noindent
$$
I_2\le2\int\limits_0^{1/\sqrt h}\left[1-\psi(ht)\right]\,dt+
4\int\limits_{1/\sqrt h}^\infty|\varphi(t)|\,dt\,.
$$
Второй интеграл в правой части стремится к нулю при $h\hm\to 0,$ поскольку
функция $|\varphi(t)|$ интег\-ри\-ру\-ема.
Чтобы оценить первый интеграл, используем неравенство

\noindent
$$
\psi(t)\ge 1-\fr{\sigma^2t^2}{2}\,,
$$
справедливое для всех~$t$ (см., например,~[7]).
Применяя это неравенство, получаем

\noindent
$$
\int\limits_0^{1/\sqrt h}\left[1-\psi(ht)\right]\,dt\le
\fr{\sigma^2}{6}\,\sqrt h\to 0
$$
при $h\to 0$.
Таким образом, $I_2\hm\to0$ при $h\hm\to0$.

Наконец, если $nh\hm\to\infty$, то $1/h\hm\le cn$, где $c$~--- некоторая постоянная,
и, следовательно,

\noindent
$$
I_3\le\int\limits_{|t|\le cn}\left\vert \varphi_n(t)-\varphi(t)\right\vert\,dt
\xrightarrow{a.s.} 0
$$ 
при $n\to\infty$ в силу
указанной выше теоремы~1 из~[6]. Таким образом, все три интеграла
стремятся к нулю, что доказывает лемму.

\smallskip

\noindent
\textbf{Замечание.} Из условия леммы об интегрируемости функции~$\varphi$
следует, что плотность~$f(x)$ равномерно
непрерывна, однако оно несколько более ограничительно.
Выполняется, например, в том случае,
когда $f(x)$ дифференцируема и ее
производная $f'(x)$ имеет ограниченную вариацию.
\pagebreak

%\medskip

\noindent
\textbf{Теорема 6.} \textit{Если $\varphi(t)$ интегрируема, то}
\begin{equation}
\sup\limits_x|f_n(x;h)-f(x)|\xrightarrow{P} 0
\label{e15-u1}
\end{equation}
\textit{при $n\hm\to\infty$, $h\hm\to0, nh^2\hm\to\infty$}.

\medskip

\noindent
Д\,о\,к\,а\,з\,а\,т\,е\,л\,ь\,с\,т\,в\,о\,.\ 
Пусть $K(x)$~--- произвольное стандартное ядро,
удовлетворяющее условиям леммы~2 и теоремы~3A  работы~[8].
Тогда в силу указанной теоремы~3A
\begin{equation}
\sup\limits_x \left\vert\hat f_n(x;h)-f(x)\right\vert
\xrightarrow{P} 0
\label{e16-u1}
\end{equation}
при $n\hm\to\infty$, $h\hm\to0$, $nh^2\hm\to\infty$
и в силу леммы~$2$
\begin{equation}
\sup\limits_x|f_n(x;h)-\hat f_n(x;h)|\xrightarrow{P} 0
\label{e17-u1}
\end{equation}
при $n\hm\to\infty$, $h\hm\to0$, $nh^2\hm\to\infty$.
Из~(\ref{e16-u1}) и~(\ref{e17-u1}) очевидным образом следует~(\ref{e15-u1}).

\smallskip

Обозначим $\theta$ моду плотности распределения $f(x)$.
Предположим, что она единственна. Пусть $\theta_n$~---
мода оценки~$f_n(x;h)$.

\medskip

\noindent
\textbf{Теорема 7.} \textit{Если $\varphi(t)$ интегрируема, то}
$$
\theta_n\xrightarrow{P}\theta
$$
\textit{при $n\hm\to\infty$, $h\hm\to0$, $nh^2\hm\to\infty$}.


\medskip

\noindent
Д\,о\,к\,а\,з\,а\,т\,е\,л\,ь\,с\,т\,в\,о\ 
теоремы аналогично доказательству второй части теоремы~3A из~[8].


{\small\frenchspacing
{%\baselineskip=10.8pt
\addcontentsline{toc}{section}{Литература}
\begin{thebibliography}{9}

\bibitem{1-u1}
\Au{Davis~K.\,B.} Mean square error properties of density
estimates~// Ann. Statist., 1975. Vol.~3. No.\,4. P.~1025--1030.

\bibitem{2-u1}
\Au{Davis~K.\,B.} Mean integrated square error properties of
density estimates~// Ann. Statist., 1977. Vol.~5. No.\,3. P.~530--535.

\bibitem{3-u1}
\Au{Glad~I.\,K., Hjort~N.\,L., Ushakov~N.\,G.} Correction of
density estimators that are not densities~// Scand. J.~Statist.,
2003. Vol.~30. No.\,2. P.~415--427.

\bibitem{4-u1}
\Au{Ушаков~В.\,Г., Ушаков~Н.\,Г.} Некоторые неравенства для
характеристических функций плотностей с ограниченной вариацией~//
Вестн. Моск. ун-та. Сер.~15. Вычисл. матем. и киберн., 2000. №\,3. С.~40--45.

\bibitem{5-u1}
\Au{Watson~G.\,S., Leadbetter~M.\,R.} On the estimation of the
probability density~I~//  Ann. Math. Statist., 1963. Vol.~34. P.~480--491.


\bibitem{6-u1}
\Au{Cs$\ddot{\mbox{o}}$rg$\mbox{\H{o}}$ S., Totik~V.} On how long interval is the
empirical characteristic function uniformly consistent?~// Acta Sci.
Math., 1983. Vol.~45. P.~141--149.


\bibitem{7-u1}
\Au{Ushakov~N.\,G.} Selected topics in characteristic functions.~--- Utrecht: VSP, 1999.

\label{end\stat}

\bibitem{8-u12}
\Au{Parzen~E.} On estimation of a probability density function
and its mode~// Ann. Math. Statist., 1962. Vol.~33. No.\,3. P.~1065--1076.
 \end{thebibliography}
}
}


\end{multicols}       