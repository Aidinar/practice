


\newcommand{\norm}[1]{\left\Vert#1\right\Vert}
%\newcommand{\abs}[1]{\left\vert#1\right\vert}
%\newcommand{\set}[1]{\left\{#1\right\}}
%\newcommand{\Real}{\mathbb R}
%\newcommand{\eps}{\varepsilon}
%\newcommand{\To}{\longrightarrow}
%\newcommand{\BX}{\mathbf{B}(X)}
%\newcommand{\A}{\mathcal{A}}

\def\stat{ushakov+shest}

\def\tit{РЕКОНСТРУКЦИЯ РАСПРЕДЕЛЕНИЙ СЛУЧАЙНЫХ ФУНКЦИЙ
В~ЗАДАЧАХ ОДНОФОТОННОЙ ЭМИССИОННОЙ ТОМОГРАФИИ
ПРИ~ПОМОЩИ АППРОКСИМАЦИИ ЭКСПОНЕНЦИАЛЬНОГО
МНОЖИТЕЛЯ ТРИГОНОМЕТРИЧЕСКИМИ МНОГОЧЛЕНАМИ$^*$}

\def\titkol{Реконструкция распределений случайных функций
в~задачах однофотонной эмиссионной томографии}
%при~помощи аппроксимации экспоненциального
%множителя тригонометрическими многочленами}

\def\autkol{В.\,Г.~Ушаков, О.\,В.~Шестаков}
\def\aut{В.\,Г.~Ушаков$^1$, О.\,В.~Шестаков$^2$}

\titel{\tit}{\aut}{\autkol}{\titkol}

{\renewcommand{\thefootnote}{\fnsymbol{footnote}}\footnotetext[1]
{Работа выполнена при финансовой поддержке РФФИ (гранты 11-01-00515а и 11-01-12026-офи-м).}}

\renewcommand{\thefootnote}{\arabic{footnote}}
\footnotetext[1]{Московский государственный университет им.\ М.\,В.~Ломоносова, 
кафедра математической статистики факультета вычислительной математики и кибернетики; 
Институт проблем информатики Российской академии наук, vgushakov@mail.ru}
\footnotetext[2]{Московский государственный университет им.\ М.\,В.~Ломоносова, 
кафедра математической статистики факультета вычислительной математики и кибернетики; 
Институт проблем информатики Российской академии наук, oshestakov@cs.msu.su}


\Abst{Рассмотрена задача определения вероятностных характеристик 
случайных функций по распределениям интегральных преобразований, возникающих 
в задачах эмиссионной томографии. В~классе дискретных случайных функций 
разработан метод восстановления распределений.}

\KW{эмиссионная томография; преобразование
Радона; проекции; случайные функции}

  \vskip 14pt plus 9pt minus 6pt

      \thispagestyle{headings}

      \begin{multicols}{2}
      
            \label{st\stat}


\section{Введение}

При исследовании биологических и других чув\-ст\-ви\-тель\-ных объектов с помощью методов 
однофотонной эмиссионной томографии возникает задача определения вероятностных 
характеристик\linebreak двумерных случайных функций
по характеристикам их интегральных преобразований вида
\begin{multline}
R_\mu f(s,\theta)=\int\limits_{x\theta=s}f(x)e^{D\mu(x,\theta^{\perp})}\,dl\,,\\
\theta\in S^1\,,\quad s\in\mathbf{R}\,,
\label{e1-us}
\end{multline} 
(см.\ [1--4]), где $f(x)$~--- непрерывная функция
с компактным носителем, имеющая смысл интенсивности излучения,
интеграл берется вдоль прямой $x\theta=s$, $S^1$~---
множество направлений, задаваемых единичными векторами в~$\mathbf{R}^2$ 
с центром в начале координат, а
$D\mu(x,\theta^{\perp})$~--- весовая функция, равная
$$
D\mu(x,\theta^{\perp})=\int\limits_{0}^{\infty}\mu(x+t\theta^{\perp})\,dt\,.
$$
Здесь $\mu(x)$~--- известная функция с компактным носителем,
$\theta=(\cos\alpha,\sin\alpha)$, а
$\theta^{\perp}=(-\sin\alpha,\cos\alpha)$. Функция~$-\mu(x)$ имеет
смысл коэффициента поглощения. Если~$\mu(x)$ равна константе на носителе~$f(x)$, то
$R_\mu(s,\theta)$ превращается в экспоненциальное преобразование
Радона~[1]. По аналогии с экспоненциальным преобразованием Радона 
будем называть интегральные преобразования вида~(\ref{e1-us}) проекциями, 
т.\,е.\ проекция представляет собой функцию от~$s$ при фиксированном~$\theta$. 
Вопрос возможности обращения преобразования~(\ref{e1-us}) оставался открытым более
20 лет. В последние годы проблема была успешно решена и были
получены различные формулы обращения (см., например,~[2--4]).

Часто в задачах однофотонной эмиссионной томографии вполне естественно 
считать функцию, описывающую интенсивность излучения, случайной.
При этом состояния (реализации) этой функции меняются во время
процесса получения проекций. Это приводит к тому, что
восстановление даже одной реализации случайной функции обычными
томографическими методами становится невозможным.

В работах~[5--9] рассмотрена задача определения вероятностных характеристик 
двумерных случайных функций
по характеристикам одномерных проекций без учета поглощения или в
предположении, что коэффициент поглощения равен константе.
Показано, что в общем случае эта задача характеризуется сильной
неоднозначностью и содержательные результаты удается
получить лишь в том случае, когда случайная функция дискретна (имеет не более
чем счетное число состояний). В~работах~\cite{7-us, 9-us} для класса таких
функций разработан метод восстановления распределений
двумерных случайных функций. В~работе~\cite{10-us} 
показано, что при коэффициенте поглощения, равном известной функции~$-\mu(x)$, 
имеющей компактный носитель и принадлежащей классу Гёльдера (с некоторым параметром), 
возможно восстановить распределения двумерных дискретных случайных функций. 
В~данной работе предложен метод такого вос\-ста\-нов\-ления.

\section{Постановка задачи и теорема единственности}

Пусть дискретная двумерная случайная функция~$\xi(x)$ имеет вид 
$$
\xi(x) = f_{\nu}(x)\,,
$$ 
где
$f_{1}(x),f_{2}(x),\;\ldots$~--- последовательность непрерывных
интегрируемых функций, имеющих компактный носитель 
(без потери общности будем считать, что этим носителем
является единичный круг $U=\{x\in R^2:x_1^2+x_2^2\leq 1\}$), а
$\nu$~--- случайная величина, принимающая целые положительные значения.

Вероятностная структура таких случайных функций полностью определяется распределением,
т.\,е.\ набором 
$
(f_{1}(x),f_{2}(x),\ldots;p_{1},p_{2},\ldots)
$, где 
$p_{i}=P(\xi(x) = f_{i}(x)),$ $i \hm= 1,2,\ldots$, 
$\sum\limits_{i=1}^{\infty}p_{i} \hm= 1$. Распределение~$\xi(x)$ будем
обозначать через~$P_\xi$, а распределение проекции через $P_{R_{\mu}\xi_{\theta}}$. 
Будем также полагать, что носителем функции~$\mu(x)$ является единичный круг~$U$.

Как показано в~\cite{10-us}, в рамках этой модели распределение двумерной 
случайной функции полностью определяется
распределениями проекций. Справедлива следующая теорема.

\medskip

\noindent
\textbf{Теорема.} \textit{Пусть случайные функции~$\xi(x)$ и~$\eta(x)$ имеют 
описанный выше вид и
$
P_{R_{\mu}\xi_{\theta}}=P_{R_{\mu}\eta_{\theta}}
$
для всех
$\theta\in\Lambda$, где $\Lambda$~--- любое подмножество~$S^1$, имеющее
положительную меру, тогда 
$$
P_\xi=P_\eta\,.
$$}

Другими словами, при сделанных предположениях распределение любой двумерной
случайной функции однозначно определяется распределениями
проекций, зарегистрированных в любом сколь угодно узком диапазоне
углов обзора.

\section{Метод группировки проекций}

Итак, в рамках описанной модели возможно восстановить
распределение двумерной случайной функции, зная распределения
ее проекций на множестве~$\Lambda$, имеющем положительную меру. В~этом пункте будет предложен
метод, позволяющий разделить множество зарегистрированных проекций
на группы, соответствующие различным состояниям случайной функции. 
Для удобства будем полагать, что $\Lambda$ совпадает с~$S^1$.

Для простоты изложения в данной работе будут рассматриваться случайные функции,
име\-ющие всего два состояния. Обобщение на любое конечное число
состояний очевидно, а для случая счетного числа состояний можно
произвести процедуру <<усечения>> распределений проекций, так же как
это делается в~\cite{7-us} для случая отсутствия поглощения. Кроме того,
поскольку в условиях рассматриваемой задачи функции~$f_{i}(x)$
описывают распределение плотности источников излучения в объекте,
будем предполагать, что они неотрицательны. Также будем
предполагать, что функции нормированы, т.\,е.\ интегралы от них по
всей об\-ласти определения равны~1 (при выполнении этих
предположений функции~$f_{i}(x)$ являются плотностями
вероятностных распределений).

Итак, пусть случайная функция~$\xi(x)$ принимает два состояния~$f_{1}(x)$ и~$f_{2}(x)$ с
вероятностями~$p_{1}$ и~$p_{2}$ соответственно.
Предполагается, что известны распределения проекций для всех $\theta\in S^1$, 
т.\,е.\ для каж\-до\-го $\theta\in S^1$ известны функции $R_\mu f_{i}(s,\theta)$, $i\hm=1,2,$ 
являющиеся проекциями функций $f_{i}(x)$, $i\hm=1,2,$ и
реализующиеся с вероятностями $p_{1}$ и~$p_{2}$ соответственно.
Причем, вообще говоря, заранее неизвестно, какое состояние
проекции соответствует какому состоянию функции, т.\,е.\ может оказаться
так, что $R_\mu f_{1}(s,\theta)$ является проекцией~$f_{2}(x)$, а
$R_\mu f_{2}(\theta,s)$~--- проекцией $f_{1}(x)$. Необходимо разделить
функции $R_\mu f_{i}(s,\theta)$, $i\hm=1,2,$ для всех $\theta\in S^1$ на
группы так, чтобы каждая группа состояний проекций относилась к
одному состоянию случайной функции.

Если $p_{1}\neq p_{2}$, тогда
такое разделение можно произвести по вероятностям состояний
проекций, т.\,е.\ для всех $\theta\in S^1$ то значение
$R_\mu f_{i}(s,\theta)$, которое реализуется с вероятностью~$p_{1}$,
относится к первой группе, а значение $R_\mu f_{i}(s,\theta)$, которое
реализуется с вероятностью~$p_{2}$,~--- ко второй.

В случае, когда $p_{1}=p_{2}=1/2$, метод группировки проекций
основан на использовании некоторых свойств так называемых моментов проекций.

Для $m=0,1,\ldots$ определим функции
\begin{equation}
J^{(m)}(\theta)=\int\limits_{\mathbf{R}}R_\mu f(s,\theta)s^{m}\,ds\,.
\label{e2-us}
\end{equation}
Интеграл $J^{(m)}(\theta)$ называют $m$-м моментом проекции
$R_\mu f(s,\theta)$ для данного направления~$\theta$. Подставим в~(\ref{e2-us})
выражение~(\ref{e1-us}) для $R_\mu f(s,\theta)$. Имеем

\noindent
\begin{multline*}
\int\limits_{\mathbf{R}}R_\mu f(s,\theta)s^{m}\,ds={}\\
{}=\int\limits_{\mathbf{R}}s^{m}\int\limits_{x\theta=s}f(x)e^{D\mu(x,\theta^{\perp})}\,dlds={}\\
{}=
\int\limits_{\mathbf{R}^2}(x\cdot \theta)^m e^{D\mu(x,\theta^{\perp})} f(x)\,dx={}\\
\int\limits_{U}(x\cdot \theta)^m e^{D\mu(x,\theta^{\perp})} f(x)\,dx\,.
\end{multline*}
Пусть $P_n(x,\theta)$~--- тригонометрический многочлен степени~$n$ 
наилучшего приближения для функции $e^{D\mu(x,\theta^{\perp})}$. 
Если функция~$\mu(x)$ непрерывно дифференцируема на~$U$ и
\begin{equation*}
\sup\limits_{x\in U}\abs{\mu(x)}=A_\mu\,;\quad
\sup\limits_{x\in U}\norm{\mbox{grad}\mu(x)}=C_\mu\,,
\end{equation*}
тогда, учитывая, что носителем функции $\mu(x)$ является круг~$U$, 
с помощью теоремы Джексона~\cite{11-us} можно показать, что для всех $x\in U$ справедливо
\begin{equation}
\sup\limits_{x\in U}\abs{e^{D\mu(x,\theta^{\perp})}-P_n(x,\theta)}\leq
\fr{24e^{2A_\mu}C_\mu}{n}\,.
\label{e3-us}
\end{equation}
Определим функцию
\begin{equation}
I^{(m,n)}(\theta)=\int\limits_{U}(x \cdot\theta)^m P_n(x,\theta) f(x)\,dx.
\label{e4-us}
\end{equation}
Легко видеть, что $I^{(m,n)}(\theta)$ представляет собой многочлен степени $m+n$. 
В~силу~(\ref{e3-us}) для всех $\theta\in S^1$
\begin{multline}
\abs{J^{(m)}(\theta)-I^{(m,n)}(\theta)}\leq{}
\\ {}\leq
\int\limits_{U}\abs{(x\cdot \theta)^m}\abs{e^{D\mu(x,\theta^{\perp})}-
P_n(x,\theta)}f(x)\,dx\leq{}\\
{}\leq\fr{24e^{2A_\mu}C_\mu}{n}\,.
\label{e5-us}
\end{multline}

Заметим, что в силу компактности носителя, если все
моменты двух непрерывных функций совпадают между собой, то эти функции
равны. Значит,
если функции $R_\mu f_{i}(s,\theta)$, $i\hm=1,2,$ различны, то найдется
номер~$m$, для которого моменты $J^{(m)}_{i}(\theta)$, $i\hm=1,2,$
различаются. Посчитаем по формуле~(\ref{e2-us}) моменты $J^{(m)}_i(\theta_k)$ в
равноотстоящих интерполяционных точках $\theta_k$~\cite{12-us}, 
$i\hm=1,2,$ $k\hm=0,\ldots,2m+2n$. Всего существует $2^{2m+2n+1}$ способов распределить
значения $J^{(m)}_i(\theta_k)$, $i\hm=1,2,$ по двум группам. Обозначим через~$H$ множество
всех возможных распределений. Решим линейные системы уравнений
\begin{equation}
G^{(m,n)}_h(\theta_{n})=J^{(m)}_{i^h_n}(\theta_{n})\,,\label{e6-us}
\end{equation}

\noindent
где
$i^h_n=1$ или~2 в зависимости от $h$, $n\hm=0,\ldots, 2m+2n$,
для всех возможных распределений~$h$ из~$H$. Здесь $G^{(m,n)}_h(\theta)$~--- 
многочлен степени $m+n$, коэффициенты которого определяются из системы~(\ref{e6-us}). 
В~результате получим
$2^{2m+2n+1}$ многочленов $G^{(m,n)}_h(\theta)$,
$h=1,\ldots,2^{2m+2n+1}$, претендующих на роль приближения для функции 
$I^{(m,n)}_1(\theta)$ или $I^{(m,n)}_2(\theta)$, определяемой выражением~(\ref{e4-us}). 
Чтобы оценить погрешность, с
которой вычисляются приближения функций
$I^{(m,n)}_i(\theta)$, $i=1,2$, воспользуемся известной
оценкой погрешности интерполяции тригонометрическими многочленами~\cite{11-us}.
Для многочлена $G^{(m,n)}_h(\theta)$, претендующего на роль
приближения для $I^{(m,n)}_1(\theta)$ или $I^{(m)}_2(\theta)$, в силу~(\ref{e5-us}) 
должно выполняться
\begin{multline*}
\abs{G^{(m,n)}_h(\theta)-I^{(m,n)}_1(\theta)}\leq{}\\
{}\leq
\fr{24e^{2A_\mu}C_\mu}{n}\left(8+\fr{4}{\pi}\ln(n+m)\right)
\end{multline*}
или
\begin{multline*}
\abs{G^{(m,n)}_h(\theta)-I^{(m,n)}_2(\theta)}\leq{}\\
{}\leq
\fr{24e^{2A_\mu}C_\mu}{n}\left(8+\fr{4}{\pi}\ln(n+m)\right)
\end{multline*}
для всех $\theta\in S^1$. Следовательно, при всех
$\theta\in S^1$ должно выполняться
\renewcommand{\theequation}{\arabic{equation}$^\prime$}
\setcounter{equation}{6}
\begin{multline}
\abs{G^{(m,n)}_h(\theta)-J^{(m)}_1(\theta)}
\leq{}\\
{}\leq
\fr{24e^{2A_\mu}C_\mu}{n}\left(9+\fr{4}{\pi}\ln(n+m)\right)
%\label{eqno{(7')}
\end{multline}
или
\renewcommand{\theequation}{\arabic{equation}$^{\prime\prime}$}
\setcounter{equation}{6}
\begin{multline}
\abs{G^{(m,n)}_h(\theta)-J^{(m)}_2(\theta)}\leq{}\\
{}\leq
\fr{24e^{2A_\mu}C_\mu}{n}\left(9+\fr{4}{\pi}\ln(n+m)\right).
%\eqno{(7'')}
\end{multline}
Начиная с некоторого~$n$, найдется всего два распределения $h_1$ и~$h_2$
из~$H$, для которых эти неравенства справедливы при всех
$\theta\in S^1$. Для этих распределений многочлены
$G^{(m,n)}_{h_i}(\theta)$, $i=1,2$, и будут приближениями
для функций $J^{(m)}_i(\theta)$.
При фиксированном~$m$ выражение в правой части
$(7')$ и~$(7'')$ стремится к нулю при $n$, стремящемся к бесконечности.
Поэтому можно сколь угодно близко равномерно аппроксимировать
функцию $J^{(m)}_i(\theta)$ многочленом
$G^{(m,n)}_{h_i}(\theta)$.
\renewcommand{\theequation}{\arabic{equation}}
\setcounter{equation}{7}

Далее при вычислении для каждого $\theta\in S^1$ значений интегралов
$J^{(m)}_i(\theta),\;i=1,2$, по формуле~(\ref{e2-us}) 
проекции относятся к той или иной группе в зависимости от того, к значению
какого из найденных многочленов $G^{(m,n)}_{h_i}(\theta)$,
$i\hm=1,2$, в точке~$\theta$ ближе значения этих интегралов.

После того как проекции распределены по группам, можно восстановить
каждое состояние случайной функции, а значит и ее распределение, 
с помощью формул Новикова или Наттерера (см., например,~\cite{4-us, 3-us}).

Описанный метод является точным в том смыс\-ле, что если проекции
функций известны точно, то по ним, в принципе, можно сколь угодно
точно приблизить функцию $J^{(m)}_i(\theta)$ и по ее
значениям осуществить группировку проекций. Однако на практике
реализовать подобное точное вос\-ста\-нов\-ле\-ние невозможно. Этому
препятствуют, по крайней мере, две причины. Первая кроется в самой
сущности метода, поскольку при аппроксимации функции $J^{(m)}_i(\theta)$ 
многочленом $G^{(m,n)}_{h_i}(\theta)$ возникает погрешность интерполяции. Вторая связана с
невозможностью точного измерения проекций. Если проекции заданы с
некоторой ошибкой, не превышающей уровень~$\eps$, то можно
получить следующую оценку для погрешности 
приближения функции $J^{(m)}_i(\theta)$:
\begin{multline*}
\abs{G^{(m,n)}_{h_i}(\theta)-J^{(m)}_i(\theta)}\leq{}\\
{}\leq
\left(\fr{2\eps}{m+1}+\fr{24e^{2A_\mu}C_\mu}{n}\right)\left(9+\fr{4}{\pi}\ln(n+m)\right)\,.
\end{multline*}

{\small\frenchspacing
{%\baselineskip=10.8pt
\addcontentsline{toc}{section}{Литература}
\begin{thebibliography}{99}
\bibitem{1-us}
\Au{Федоров Г.\,А., Терещенко С.\,А.} 
Вычислительная эмиссионная томография.~--- М.: Энергоатомиздат, 1990.
\par
\bibitem{2-us}
\Au{Arbuzov E.\,V., Bukhgeim A.\,L., Kazantsev~S.\,G.}
Two-dimensional tomogra\-phy problems and the theory of\linebreak A-analytic
functions~// Siberian Adv. Math., 1998. Vol.~8. P.~1--20.


\bibitem{4-us}
\Au{Natterer F.} Inversion of the attenuated Radon transform~//
Inverse Problems, 2001. Vol.~17. P.~113--119.

\bibitem{3-us}
\Au{Novikov R.\,G.} 
An inversion formula for the attenuated\linebreak X-ray
transformation~// Ark. Mat., 2002. Vol.~40. P.~145--167.

\bibitem{5-us}
\Au{Ушаков В.\,Г., Ушаков Н.\,Г.} Восстановление вероятностных
характеристик многомерных случайных функций по проекциям~// Вестн.
Моск. ун-та. Сер.~15. Вычисл. матем. и киберн., 2001. №\,4. C.~32--39.

\bibitem{7-us}
\Au{Shestakov O.\,V.} An algorithm to reconstruct probabilistic
distributions of multivariate random functions from the
distributions of their projections~// J.~Math.
Sci., 2002. Vol.~112. No.\,2. P.~4198--4204.

\bibitem{6-us}
\Au{Шестаков О.\,В.} О единственности восстановления вероятностных
характеристик многомерных случайных функций по вероятностным
характеристикам их проекций~// Вестн. Моск. ун-та. Сер.~15. Вычисл.
матем. и киберн., 2003. №\,3. С.~37--41.


\bibitem{8-us}
\Au{Ушаков В.\,Г., Шестаков О.\,В.} Экспоненциальное преобразование
Радона случайных функций~// Вестн. Моск. ун-та. Сер.~15. Вычисл.
матем. и киберн., 2005. №\,1. C.~49--55.

\bibitem{9-us}
\Au{Shestakov O.\,V.} Inversion of exponential Radon transform of
random func\-tions~// Transactions of XXV Seminar on Stability
Problems for Stochastic Models, 2005. P.~264--269.

\bibitem{10-us}
\Au{Ушаков В.\,Г., Шестаков О.\,В.} Восстановление вероятностных характеристик
случайных функций в задачах однофотонной эмиссионной томографии~//
Информатика и её применения, 2009. Т.~3. №\,1. С.~20--24.

\bibitem{11-us}
\Au{Натансон И.\,П.} Конструктивная теория функций.~--- М.--Л.: ГИТТЛ, 1949.

\label{end\stat}

\bibitem{12-us}
\Au{Гончаров В.\,Л.} Теория интерполирования и приближения функций.~---
М.: ГТТИ, 1933.
 \end{thebibliography}
}
}


\end{multicols}       