%\def\mytheorem#1#2{\begin{my_the} #1 \end{my_the} \begin{proof} #2 \end{proof} }
%\def\mylemma#1#2{\begin{my_lem} #1 \end{my_lem} \begin{proof} #2 \end{proof} }
%\def\mystatement#1#2{\begin{my_state} #1 \end{my_state} \begin{proof} #2 \end{proof} }
%\def\mydefinition#1{\begin{my_def} #1 \end{my_def}}
%\def\mysample#1{\begin{my_sam} #1 \end{my_sam}}


\def\Set#1#2{\{{#1\ \colon\ #2 \}}}
\def\SetA#1#2{\{{#1\ \mid\ #2 \}}}

\def\sigmaF{\mathscr{F}}
\def\sigmaB{\mathscr{B}}
\def\classT{\mathbb {T}}
\def\divers{\succcurlyeq}


\newcommand{\Expect}{\mathsf{E}}

%\renewcommand{\ge}{\geqslant}
%\renewcommand{\le}{\leqslant}

\def\stat{yakovenko}

\def\tit{ДИВЕРСИФИКАЦИЯ И ЕЕ СВЯЗЬ С МЕРАМИ РИСКА}

\def\titkol{Диверсификация и ее связь с мерами риска}

\def\autkol{Д.\,О.~Яковенко, М.\,А.~Целищев}
\def\aut{Д.\,О.~Яковенко$^1$, М.\,А.~Целищев$^2$}

\titel{\tit}{\aut}{\autkol}{\titkol}

%{\renewcommand{\thefootnote}{\fnsymbol{footnote}}\footnotetext[1]
%{Работа выполнена при финансовой поддержке РФФИ (гранты 11-01-00515а и 11-01-12026-офи-м).}}

\renewcommand{\thefootnote}{\arabic{footnote}}
\footnotetext[1]{Гроссмейстер ФИДЕ, ms@cs.msu.su}
\footnotetext[2]{Московский государственный университет им.\ М.\,В.~Ломоносова, 
кафедра математической статистики факультета вычислительной математики и кибернетики; 
ms@cs.msu.su}

\Abst{Предложен новый подход к понятию
диверсификации инвестиционных портфелей, которое определяется как
бинарное отношение во множестве портфелей с конечным первым
моментом. Показано, что это бинарное отношение является (в
определенном смысле) частичной упорядоченностью. Рассмотрены важные
свойства этого определения, а также необходимое и достаточное
условия срав\-ни\-мости портфелей, важнейшую роль в которых играет
когерентная мера риска \textit{Expected Shortfall} (ожидаемый
дефицит). В~качестве примера приводится интерпретация диверсификации
информационного риска.}

\KW{диверсификация; инвестиционные портфели;
сравнение портфелей; когерентные меры риска; \textit{Expected
Shortfall}; информационный риск}

  \vskip 14pt plus 9pt minus 6pt

      \thispagestyle{headings}

      \begin{multicols}{2}
      
            \label{st\stat}

\section{Введение}

Диверсификация является одним из основных способов управления
рисками практически во всех областях экономической деятельности. 
В~основе диверсификации лежит идея распределения риска по различным
источникам, недопущение ситуации, когда одно неблагоприятное событие~--- 
резкое изменение цены какого-либо товара, нарушение контрагентом
графика поставок и~т.\,п. --- может привести к катастрофическим
последствиям. В~статье предложено формальное определение
диверсификации и рассмотрены свойства этого определения. В~качестве
примера рассматривается диверсификация информационных сис\-тем как
способ управления информационными рисками, под которыми понимаются
риски возникновения убытков из-за неправильной организации или
умышленного нарушения информационных потоков и сис\-тем организации.


\section{Формализация понятия диверсификации}

Введем вероятностное пространство $(\Omega,\sigmaF,\mu)$, где
$\Omega=[0;1)$, $\sigmaF$~--- сиг\-ма-ал\-геб\-ра борелевских множеств на
$[0;1)$, $\mu$ --- мера Лебега. $\Omega$~интерпретируется как
пространство возможных вариантов развития событий (траекторий) на
рынке. Под инвестиционным портфелем в этой статье понимается не
набор ценных бумаг, а некая стратегия, которой придерживается
инвестор. Стратегия представляет собой случайную величину~$\xi$,
определенную на введенном выше вероятностном пространстве и такую,
что на траектории $\omega \hm\in \Omega$ инвестор \textbf{несет
потери}, равные $\xi(\omega)$ (если $\xi(\omega) \hm< 0 $, то инвестор
получает прибыль в размере $ - \xi(\omega)$). Класс всех
рассматриваемых стратегий полагаем таким: $V\hm=\Set{\xi}{\exists\
\Expect|\xi|< \infty}$.

Будем использовать следующие обозначения:
\begin{enumerate}
\item $\xi_1 \sim \xi_2 \Leftrightarrow F_{\xi_1}(x) \equiv F_{\xi_2}(x)$, т.\,е.\ 
$\xi_1$ и $\xi_2$ одинаково распределены.
\item $\xi^{(\alpha)} = \inf \Set{x}{F_{\xi}(x) \hm> \alpha}$, $\alpha \hm\in [0,1]$,~--- 
верхняя квантиль порядка~$\alpha$.
\end{enumerate}

В дальнейшем потребуются две операции над портфелями: сложение двух
портфелей и умножение портфеля на константу. Под суммой двух
портфелей~$\xi_1$ и~$\xi_2$ понимается портфель, который на
траектории $\omega \hm\in \Omega$ дает потери
$\xi_1(\omega)\hm+\xi_2(\omega)$, а под умножением на константу $\alpha
\hm\in [0,1]$ (другие константы не понадобятся) портфеля~$\xi$
понимается портфель, который на траектории $\omega \hm\in \Omega$ дает
потери $\alpha \xi(\omega)$. С~экономической точки зрения стратегия
$\xi_1\hm+\xi_2$ подразумевает параллельное исполнение стратегий~$\xi_1$ и~$\xi_2$, 
а стратегия $\alpha \xi$ заключается в выполнении
всех финансовых операций стратегии~$\xi$, но только <<в доле>> с
другими инвесторами (так, чтобы собственная доля во всех операциях
составляла~$\alpha$).

\medskip

\noindent
\textbf{Определение 1.} Будем говорить, что портфель~$\xi_2 \hm\in V$ является
\textbf{результатом диверсификации} портфеля~$\xi_1 \hm\in V$
(обозначается~$\xi_2 \divers \xi_1$), если $\forall\eps\hm>0$
существуют портфели $\eta_0, \ldots , \eta_n, \eta_0', \ldots,
\eta_{n-1}'\hm \in V$ и числа $\alpha_1, \ldots, \alpha_n, 0 \hm\le
\alpha_i \hm\le 1$, такие~что $\eta_0 \hm= \xi_1$, $\eta_i \hm\sim \eta_i'$,
$\eta_i \hm= \alpha_i \eta_{i-1} \hm+ (1 \hm- \alpha_i)\eta_{i-1}'$ и $\eta_n
\hm\ge \xi_2 \hm- \eps$ почти всюду (п.\,в).

\medskip

\noindent
\textbf{Пример 1.} Пусть $X_1$, $X_2$, $X_3$~--- независимые одинаково
распределенные случайные величины (портфели). Рассмотрим портфель
вида $\xi_1 \hm= p_1 X_1 \hm+ p_2 X_2 \hm+ p_3 X_3$, $p_i \hm\ge 0$,
$p_1\hm+p_2\hm+p_3\hm=1$, и покажем, что портфель $\xi_2 \hm= (1/3)X_1 \hm+
(1/3) X_2 \hm+ (1/3) X_3$ является результатом диверсификации
портфеля~$\xi_1$. Без ограничения общности считаем, что $0 \hm\le p_1
\hm\le p_2 \hm\le p_3 \hm\le 1$ и $p_1 \hm< 1/3$ (случай, когда все $p_i \hm=
1/3$, не интересен в силу его тривиальности). Определим $\eta_0 \hm=
\xi_1 \hm= p_1 X_1 \hm+ p_2 X_2 \hm+ p_3 X_3$ , $\eta_0' \hm= p_3 X_1 \hm+ p_2 X_2
\hm+ p_1 X_3$ ($\eta_0 \hm\sim \eta_0'$, так как $X_1$, $X_2$, $X_3$
независимы и одинаково распределены), $\alpha_1 \hm= (p_3 \hm-
1/3)/(p_3 \hm- p_1)$ ($0 \hm< \alpha_1 \hm< 1$ в силу предположений на
$p_1$, $p_2$, $p_3$). Тогда 

\noindent
$$
\eta_1 = \alpha_1 \eta_0 + (1 - \alpha_1) \eta_0' 
= \fr{1}{3}\, X_1 + p_2 X_2 + \left(\fr{2}{3}- p_2\right) X_3\,.
$$ 
Теперь определим 
$$
\eta_1' = \fr{1}{3}\, X_1 + \left(\fr{2}{3}- p_2\right) X_2 + p_2 X_3
$$ и
$\alpha_2 \hm= 1/2 $. 
Тогда 



\noindent
\begin{multline*}
\eta_2 = \alpha_2 \eta_1 + (1 - \alpha_2) \eta_1' = 
\fr{1}{3}\, X_1 + \fr{1}{3}\, X_2 + \fr{1}{3}\, X_3 ={}\\
{}=
    \xi_2 > \xi_2 - \eps\,,\enskip \forall \eps\hm>0\,.
    \end{multline*} 
    Тем самым, согласно введенному определению, $\xi_2 \divers \xi_1$.

\vspace*{-3pt}

\section{Свойства введенного определения}

\vspace*{-1pt}

Введем класс отображений $ \classT \hm= \Set {T}{\Omega \rightarrow
\Omega} $, так чтобы $T \hm\in \classT $ тогда и только тогда, когда
$\exists A,B \hm\in \sigmaF$, $\mu(A)\hm=\mu(B)=1$, $\exists T' \colon B
\hm\rightarrow A$ такое, что $T'$ измеримо, обратимо и $\forall
G\hm\in\sigmaF\cap B$ $\exists \mu(T'(G)) \hm= \mu(G)$, причем
$T(\omega)\hm=T'(\omega)$  для  $\omega \hm\in B$ (на множестве
нулевой меры $\Omega \backslash B$ отображение~$T$ принимает
произвольные значения из~$\Omega$).

\medskip

\noindent
\textbf{Лемма~1.} \textit{$\forall \xi \in V $, $\forall T \in \classT $
выполняется: $\xi \sim \xi(T) $.} 

\smallskip

\noindent
Д\,о\,к\,а\,з\,а\,т\,е,л\,ь\,с\,т\,в\,о.\  
Пусть $\ T \hm\in \classT$.
Тогда $\forall G \hm\in \sigmaB$ ($\sigmaB$~--- сиг\-ма-ал\-геб\-ра
борелевских множеств на прямой) в силу того, что
$\mu(A)\hm=\mu(B)\hm=1$, имеем:

\noindent
\begin{multline*}
\p(\xi(T) \in G) = \mu( \Set{\omega}{T(\omega) \in \xi^{-1}(G)} ) = {}\\
{}=\mu(\Set{\omega}{T(\omega) \in \xi^{-1}(G)} \cap B)={} \\
{}=\mu(\Set{\omega}{T(\omega) \in \xi^{-1}(G)\cap A} \cap B)={} \\
{}=\mu(\Set{\omega \in B}{T'(\omega) \in \xi^{-1}(G)\cap A})={} \\
{}=\mu(B\cap (T')^{-1}(\xi^{-1}(G)\cap A))={}\\
{}= \mu((T')^{-1}(\xi^{-1}(G)\cap A))={}\\
{}= \mu(\xi^{-1}(G)\cap A)= \mu(\xi^{-1}(G)) = \p(\xi \in G)\,.
\end{multline*}
Лемма доказана.\hfill$\square$

\columnbreak
%\medskip

\noindent

\textbf{Лемма~2.}
\textit{Пусть $\xi_1, \xi_2 \hm\in V $ и $\xi_1 \hm\sim \xi_2$. Тогда
$ \forall \eps \hm> 0$ $\exists  T \hm\in \classT$ такое, что $
|\xi_1(T)\hm-\xi_2|\hm<\eps $ почти наверное.} 

\medskip

\noindent
Д\,о\,к\,а\,з\,а\,т\,е\,л\,ь\,с\,т\,в\,о\,.\ Фиксируем произвольное
$\eps \hm> 0 $. Обозначим
$$
A_i = \xi_1^{-1}\left(  \left[  i \eps ; (i+1) \eps \right) \right )\,,  \\
B_i = \xi_2^{-1}\left( \left [  i \eps ; (i+1) \eps \right ) \right )\,.
$$
Поскольку $\xi_1 \hm\sim \xi_2$, то $\forall i$  $\mu(A_i) \hm=
\mu(B_i)$. Согласно~[1, с. 74], $\forall i$ $\exists
A_i'$, $B_i' \hm\in \sigmaF$, $A_i' \subseteq A_i$,
$B_i' \hm\subseteq B_i$, $\mu(A_i') \hm= \mu(A_i)$, $\mu(B_i')\hm =
\mu(B_i)$ и $\exists T_i \colon B_i' \hm\to A_i'$ такое, что $T_i$
измеримо, $\exists T_i^{-1}$ и $\forall G \hm\in \sigmaF \cap B_i': 
\mu(T_i(G)) \hm= \mu(G)$. Определим $T(\omega) \hm= T_i(\omega)$ для 
$\omega \hm\in B_i'$. На множестве нулевой меры $\Omega
\backslash (\bigsqcup\limits_{-\infty}^\infty B_i')$ доопределим~$T$ 
произвольно. Обозначим $B \hm= \bigsqcup\limits_{-\infty}^\infty
B_i'$, $A \hm= \bigsqcup\limits_{-\infty}^\infty A_i'$. В~этом случае
$\mu(A) = \mu(B) = 1$. Таким образом, $T \hm\in \classT$ и $\forall i
\ \forall \omega \hm\in B_i' $:
\begin{multline*}
| \xi_1(T(\omega)) - \xi_2(\omega) | = | \xi_1(T_i(\omega)) -
\xi_2(\omega) | <{}\\
{}< (i+1) \varepsilon - i \varepsilon =
\varepsilon\,.
\end{multline*} 
Лемма доказана.\hfill$\square$ 

\medskip

\noindent
\textbf{Определение 2.}
\textbf{Оператором диверсификации} назовем оператор
$D_{\alpha,T} : V \hm\rightarrow V$, $\alpha \hm\in [0;1]$, $T \hm\in \classT$,
такой что $D_{\alpha,T}(\xi) \hm= \alpha \xi \hm+ ( 1 \hm- \alpha ) \xi(T)$.

Важные свойства операторов диверсификации:
\begin{enumerate}
\item Если $\xi_1 \hm= \xi_2$ ($\xi_1 \hm\le \xi_2$) почти наверное, то
$D_{\alpha,T}(\xi_1) \hm= D_{\alpha,T}(\xi_2)$ 
($D_{\alpha,T}(\xi_1) \hm\le D_{\alpha,T}(\xi_2)$  соответственно)
 почти наверное.
\item Если $\xi,\eta\hm\in\,V$, $\eta\hm=const$ п.\,в., 
то $D_{\alpha,T}(\xi+\eta) \hm= D_{\alpha,T}(\xi) \hm+ \eta$.
\end{enumerate}

\medskip

\noindent
\textbf{Лемма 3.} \textit{$\xi_2 \divers \xi_1$ тогда и только тогда, когда $
\forall \eps\hm>0 $ существует последовательность операторов
диверсификации $ D_1=D_{\alpha_1,T_1},\, \ldots\, ,
D_n\hm=D_{\alpha_n,T_n} $ такая, что $ D_n(\ldots(D_1(\xi_1))\ldots)
\ge \xi_2 \hm- \eps $ почти наверное.}

\medskip

\noindent
Д\,о\,к\,а\,з\,а\,т\,е\,л\,ь\,с\,т\,в\,о\,.\
%$\phantom{hello,world}$\\

\noindent
\fbox{$\Leftarrow$} Непосредственно вытекает из определений~1, 2 и леммы~1.

\noindent
\fbox{$\Rightarrow$} Пусть $\xi_2 \divers \xi_1$. Фиксируем $\eps
\hm> 0$ и строим $\eta_0, \ldots , \eta_n, \eta_0', \ldots,
\eta_{n-1}', \alpha_1, \ldots, \alpha_n$ из определения~1, так чтобы
$ \eta_n \hm\ge \xi_2 \hm- \eps/2 $ почти наверное. Пользуясь
леммой~2, построим $ T_i \hm\in \classT $ так, что $ | \eta_{i-1}' \hm-
\eta_{i-1}(T_i) | \hm\le \eps/(2n) $ почти наверное. Определим
$D_i \hm= D_{\alpha_i,T_i}$, $i \hm=1,\ldots,n,$ и покажем, что эта
последовательность операторов диверсификации будет искомой.
Определим $\zeta_0 \hm= \xi_1$, $\zeta_k \hm= D_k(\zeta_{k-1})$, $k \hm=
1,\ldots,n$. Если доказать, что $ | \zeta_n \hm- \eta_n | \hm\le
\eps/2$ почти наверное, то в силу того, что $ \eta_n \hm\ge
\xi_2 \hm- \eps/2 $ почти наверное, получим $ \zeta_n \hm\ge \xi_2
\hm- \eps $ почти наверное, и лемма будет доказана. Обоснуем
необходимое неравенство по индукции: будем доказывать, что $\forall
k=0,\dots,n$ $| \zeta_k \hm- \eta_k | \hm\le k \eps/(2n)$ почти
наверное. При $k\hm=0$ утверждение справедливо: $|\zeta_0 \hm- \eta_0| \hm=
|\xi_1 \hm- \xi_1| \hm= 0 $. Пусть утверждение справедливо для некоторого
$k<n$, докажем его для $k+1$:
\begin{multline*}
\left\vert \zeta_{k+1} - \eta_{k+1} \right\vert = 
\left\vert\alpha_{k+1}\zeta_k + (1-\alpha_{k+1})\zeta_k(T_{k+1}) -{}\right.\\
\left.{}-  \alpha_{k+1}\eta_k -                  (1-\alpha_{k+1})\eta_k' \right\vert \le \\
\le \alpha_{k+1} \left\vert \zeta_k - \eta_k \right\vert 
+ \left(1-\alpha_{k+1}\right) \left\vert\zeta_k(T_{k+1}) - \eta_k'\right\vert \le {}\\
\!{}\le \alpha_{k+1} \left| \zeta_k - \eta_k \right| + \left(1-\alpha_{k+1}\right) 
\left\vert\zeta_k(T_{k+1}) - \eta_k(T_{k+1})\right\vert +{}\\
{}+    \left(1-\alpha_{k+1}\right) \left\vert\eta_k\left(T_{k+1}\right) - \eta_k'\right\vert \le{} \\
{}\le \fr{(k+1)\eps}{2n} 
\end{multline*}
почти наверное.

Тем самым лемма доказана.\hfill$\square$


\section{Необходимое условие диверсифицируемости}

В теории управления рисками хорошо известна когерентная мера риска
под названием \textit{Expected Shortfall} (ожидаемый дефицит):

\noindent
$$
{ES}_\gamma(\xi) = \fr{1}{\gamma} \int\limits_{1-\gamma}^1 \xi^{(t)} \,dt\,,\enskip
\gamma \in  (0,1]\,.
$$
Согласно~\cite{es}, \textit{Expected Shortfall} обладает следующими
свойствами:
\begin{enumerate}
\item $\xi\in V$, $\xi\hm\le0$  \textit{почти наверное} ${}\Rightarrow {ES}_\gamma(\xi)\hm\le0$, 
$\forall \gamma\hm \in (0;1]$;
\item ${ES}_\gamma(\xi_1+\xi_2) \le {ES}_\gamma(\xi_1) \hm+ {ES}_\gamma(\xi_2)$, 
$\forall \xi_1,\xi_2\hm\in V$;
\item ${ES}_\gamma(a\xi) = a {ES}_\gamma(\xi)$, $\forall \xi\hm\in V$, $a\hm\ge 0$;
\item ${ES}_\gamma(\xi+a) = {ES}_\gamma(\xi)\hm+a$, $\forall \xi\in V$, $a\hm\in\mathbb{R}$;
\item $\xi_1,\xi_2\in V$, $\xi_1\hm\sim\xi_2 \hm\Rightarrow{ES}_\gamma(\xi_1)\hm={ES}_\gamma(\xi_2)$.
\end{enumerate}

Простым следствием из свойств~1 и~2 является свойство монотонности:
\begin{enumerate}[6.]
\item $\xi_1,\xi_2\in V$, $\xi_1 \hm\le \xi_2$ \textit{почти наверное} $
{}\Rightarrow {ES}_\gamma(\xi_1) \hm\le {ES}_\gamma(\xi_2)$, $\forall \gamma \in (0;1]$.
\end{enumerate}

\medskip

\noindent
\textbf{Утверждение 1.} \textit{$\forall \xi \in V$ $\forall \alpha \in [0;1] 
\forall T \hm\in \classT$ справедливо неравенство 
${ES}_\gamma ( D_{\alpha , T}(\xi)) \hm\le {ES}_\gamma (\xi)$  для $\forall
\gamma \hm\in (0;1]$. }

\smallskip

\noindent
Д\,о\,к\,а\,з\,а\,т\,е\,л\,ь\,с\,т\,в\,о\,.\ В~силу леммы~1 \ $\xi \hm\sim \xi(T)$. По
указанным выше свойствам \textit{Expected Shortfall} имеем

\noindent
\begin{multline*}
{ES}_\gamma ( D_{\alpha , T}(\xi)) =
{ES}_\gamma ( \alpha \xi + (1 - \alpha) \xi(T)) \stackrel{\text{св.2}}{\le}{} \\
{}\le {ES}_\gamma ( \alpha \xi ) + {ES}_\gamma  ( (1 - \alpha) \xi(T)) \stackrel{\text{св.3}}{=}{}\\
{}=\alpha {ES}_\gamma ( \xi ) + (1 - \alpha) {ES}_\gamma ( \xi(T)) \stackrel{\text{св.5}}{=}  \\
= \alpha {ES}_\gamma ( \xi ) + (1 - \alpha) {ES}_\gamma ( \xi ) =
{ES}_\gamma ( \xi )\,.
\end{multline*}
Утверждение доказано.\hfill$\square$

\columnbreak 

%\medskip

\noindent
\textbf{Теорема 1.}
\textit{Если $\xi_2 \divers \xi_1 $, то ${ES}_\gamma(\xi_2) \hm\le
{ES}_\gamma(\xi_1)$ для всех $ \gamma \hm\in (0;1] $.} 

\medskip

\noindent
Д\,о\,к\,а\,з\,а\,т\,е\,л\,ь\,с\,т\,в\,о\,.\ По лемме~3
для любого $\eps\hm>0$ существует последовательность операторов
диверсификации $ D_1,\, \ldots\, , D_n $ такая, что
$D_n(\ldots(D_1(\xi_1))\ldots) \hm\ge \xi_2 \hm- \eps$ почти наверное.
Используя свойства \textit{Expected Shortfall} и утверждение~1, получим
цепочку неравенств
\begin{multline*}
{ES}_\gamma (\xi_2) - \eps \stackrel{\text{св.4}}{=} {ES}_\gamma (\xi_2 - \eps)
 \stackrel{\text{св.6}}{\le}{}\\
 {}\le {ES}_\gamma ( D_n(\ldots(D_1(\xi_1))\ldots) ) \stackrel{\text{утв.1}}{\le}{} \\
{}\le {ES}_\gamma( D_{n-1}(\ldots(D_1(\xi_1))\ldots) ) \stackrel{\text{утв.1}}{\le}\! \ldots
\! \stackrel{\text{утв.1}}{\le} {ES}_\gamma (\xi_1).\hspace*{-4.50763pt}
\end{multline*}
Так как это справедливо для всех $\eps \hm> 0$, то, переходя к пределу
при $\eps \hm\to +0$, получим ${ES}_\gamma(\xi_2) \hm\le
{ES}_\gamma(\xi_1)$. Теорема доказана.\hfill$\square$

\medskip

\noindent
\textbf{Теорема~2.}
\textit{Бинарное отношение <<$\divers$>> является час\-тич\-ной
упорядоченностью на множестве~$V$ (антисимметричность при этом
понимается в том смысле, что из $\xi_1 \divers \xi_2$, $\xi_2
\divers \xi_1$ следует, что $\xi_1 \hm\sim \xi_2$).} 

\medskip


\noindent
Д\,о\,к\,а\,з\,а\,т\,е\,л\,ь\,с\,т\,в\,о\,.\ 
\textit{Рефлексивность} очевидна.

\textit{Транзитивность.} Пусть $\xi_2 \divers \xi_1$, $\xi_3 \divers
\xi_2$. По лемме~3 существуют последовательности операторов
диверсификации $D_1, \ldots , D_n, D_1', \ldots, D_m'$ такие, что
\begin{align*}
D_n(\ldots(D_1(\xi_1))\ldots) &\ge \xi_2 - \fr{\eps}{2}\ \mbox{п.\,в.}\,;
\\
D_m'(\ldots(D_1'(\xi_2))\ldots) &\ge \xi_3 - \fr{\eps}{2}\ \mbox{п.\,в.}
\end{align*}
Тогда 
\begin{multline*}
D_m'(\ldots(D_1'(   D_n(\ldots(D_1(\xi_1))\ldots) ))\ldots)
\ge{}\\
{}\ge D_m'(\ldots\left(D_1'\left(  \xi_2 - \fr{\eps}{ 2}\right )\right)\ldots) ={}\\
{}=
D_m'(\ldots(D_1'(  \xi_2 ))\ldots) - \fr{\eps}{2}\ge  \xi_3 - \fr{\eps}{2}
- \fr{\eps}{2}= \xi_3 - \eps\,. 
\end{multline*}
По лемме~3 $\xi_3 \divers \xi_1$.

\textit{Антисимметричность.} Пусть $\xi_1 \divers \xi_2$, $\xi_2
\hm\divers \xi_1$. Тогда по теореме~1 $\forall \gamma \hm\in (0,1]$
справедливо ${ES}_\gamma(\xi_1)\hm = {ES}_\gamma(\xi_2)$, т.\,е.\
$$
\fr{1}{\gamma} \int\limits_{1-\gamma}^1 \xi_1^{(t)} \,dt = \fr{1}{\gamma}
\int\limits_{1-\gamma}^1 \xi_2^{(t)} \,dt\,.
$$ 
Отсюда вытекает, что
$\xi_1^{(t)} \hm= \xi_2^{(t)}$ почти наверное и, следовательно, $\xi_1
\hm\sim \xi_2$. Теорема доказана.\hfill$\square$


\section{Достаточное условие диверсифицируемости}

\noindent
\textbf{Лемма~4.} \textit{Пусть $[a;b),[c;d) \subset \Omega$,
$[a;b)\cap[c;d)=\varnothing$, $b-a\hm=d\hm-c$; $\xi\hm\in V$,
$\xi(\omega) \hm= y_1$  при $\omega \hm\in [a;b)$, $\xi(\omega)\hm =
y_2$ при $\omega \in [c;d)$; числа $y_1'$, $y_2'$ таковы, что
$0 \hm\le y_1 \hm\le y_1' \hm\le y_2' \hm\le y_2$ и $y_1\hm+y_2 \hm= y_1'\hm+
y_2'$. Тогда существует оператор диверсификации $D \hm= D_{\alpha,T}$
такой, что $D(\xi) \hm= y_1'$ при $\omega \hm\in [a;b)$, $D(\xi) \hm= y_2'$
при $\omega \hm\in [c;d)$,  $D(\xi) \hm= \xi$ при остальных~$\omega $.}

\medskip

\noindent
Д\,о\,к\,а\,з\,а\,т\,е\,л\,ь\,с\,т\,в\,о\,.\ 
Построим~$T(\omega)$:  $T(\omega) \hm= c \hm+ (\omega \hm- a)$ при $\omega
\hm\in [a;b)$, $T(\omega) \hm= a \hm+ (\omega \hm- c)$ при $\omega \hm\in [c;d)$,
$T(\omega) \hm= \omega$ при остальных~$\omega$. Ясно, что $T \hm\in
\classT$. Возьмем $\alpha \hm= (y_2-y_1')/(y_2-y_1)$. Оператор
$D_{\alpha,T}$ и будет искомым. Лемма доказана.\hfill$\square$

\medskip

\noindent
\textbf{Лемма~5.} \textit{Пусть $[a;b),[c;d) \subset \Omega$,
$[a;b)\cap[c;d)\hm=\varnothing$; $\xi\hm\in V$, $\xi(\omega) \hm= y_1$ при
$\omega\hm\in[a;b)$, $\xi(\omega)\hm = y_2$ при $\omega\hm\in[c;d)$. Тогда
если $ 0 \hm\le y_1 \hm\le y_1' \hm\le y_2' \hm\le y_2 $ и 
$(y_2 \hm- y_2')(d-c) \hm > (y_1' \hm- y_1)(b-a)$, 
то существует конечная последовательность
операторов диверсификации $D_n,\,\ldots\,,D_1$ такая, что
$\xi'(\omega) \hm= D_n(\ldots(D_1(\xi(\omega)))\ldots) \hm= y_1'$ при  
$\omega \hm\in [a;b)$, $\xi'(\omega) \hm\ge y_2'$  при 
$\omega \hm\in [c;d)$ и  $\xi'(\omega) \hm= \xi(\omega)$ при
остальных~$\omega$.} 

\medskip

\noindent
Д\,о\,к\,а\,з\,а\,т\,е\,л\,ь\,с\,т\,в\,о\,.\  Так как 
$(y_2 \hm- y_2')(d-c) \hm> (y_1' \hm- y_1)(b-a)$, то существует $d' \hm\in [c,d)$ такое, что 
$(y_2 \hm- y_2')(d'-c)\hm>(y_1' \hm- y_1)(b-a)$ и $(d'-c)/(b-a)$ рационально, т.\,е.\ 
$(d'-c)/(b-a) \hm=m/n$, $m,n \hm\in \mathbb{N}$.
Разделим полуинтервалы $[a;b)$ и $[c;d')$ на $n$ и~$m$ равных
полуинтервалов, т.\,е.\
$a\hm=a_0\hm<a_1<\ldots$\linebreak $\ldots<a_n\hm=b$; $c\hm=c_0\hm<c_1\hm<\ldots\hm<c_m\hm=b$ и 
$\forall i,j:
a_i\hm-a_{i-1}\hm=c_j\hm-c_{j-1}$.
Укажем теперь алгоритм построения искомой последовательности операторов диверсификации.

\smallskip

\noindent
\fbox{Шаг 1.} Пусть $y_2-y_2' \hm\ge y_1'-y_1$ (если это не так, то
переходим к следующему шагу). Тогда по предыдущей лемме существует~$D_1$ 
такой, что $\xi_1 \hm\equiv D_1(\xi) \hm= y_1'$ при $\omega \hm\in [a;a_1)$, 
$\xi_1 \hm= z \hm= y_2\hm-(y_1'\hm-y_1) \hm\ge y_2'$ при $\omega \hm\in [c;c_1)$, 
$\xi_1 \hm= \xi$ при остальных~$\omega$. Далее, если
$z-y_2' \hm\ge y_1'\hm-y_1$, то таким же образом строим (с заменой~$y_2$
на~$z$ и $[a;a_1)$ на $[a_1;a_2)$) оператор~$D_2$, получим случайную величину
$\xi_2 \hm\equiv D_2(\xi_1)$, которая равна~$y_1'$ уже на полуинтервале
$[a;a_2)$. Затем перейдем к полуинтервалу $[a_2;a_3)$ и~т.\,д.

\smallskip

\noindent
\fbox{Шаг 2.} Когда же выполнится неравенство $z-y_2' \hm< y_1'\hm-y_1$
(пусть это произойдет после применения оператора~$D_k$), построим
оператор~$D_{k+1}$ такой, что случайная величина $\xi_{k+1}(\omega)
\hm\equiv D_{k+1}(\xi_k(\omega)) \hm= y_2'$ при $\omega \hm\in [c;c_1)$,
$\xi_{k+1}(\omega) \hm= y_1\hm+(z\hm-y_2') \hm< y_1'$ при $\omega \hm\in [a_k;a_{k+1})$, 
$\xi_{k+1}(\omega) \hm= \xi_k(\omega)$ при остальных~$\omega$ 
($\xi_{k+1}\hm=y_1'$ на $[a;a_k)$). Теперь переходим от
$[c;c_1)$ к $[c_1;c_2)$ и продолжаем процесс c шага~1 (с заменой~$y_1$ 
на $y_1+(z-y_2')$, $[a;a_1)$ на $[a_k;a_{k+1})$, $[c;c_1)$ на
$[c_1;c_2)$) и~т.\,д.

Так как $(y_2 \hm- y_2')(d'\hm-c)>(y_1'\hm -
y_1)(b-a)$, или $(y_2 \hm- y_2')m>(y_1' \hm- y_1)n$, то через $l \hm\le m+n$
шагов получим случайную величину~$\xi_l$ такую, что
$\xi_l(\omega)\hm=y_1'$ при $\omega \hm\in [a;b)$, $\xi_l(\omega) \hm\ge
y_2'$ при $\omega \hm\in [c;d)$ и $\xi_l(\omega)\hm=\xi(\omega)$ при
остальных~$\omega$. Лемма доказана.\hfill$\square$

\smallskip

\noindent
\textbf{Лемма~6.}
\textit{Пусть $[a;b)\in \Omega$, $C = \bigsqcup\limits_{i=1}^n
[c_i;d_i) \hm\in \Omega$, $[a;b) \cap C=\varnothing$
$\xi\hm\in V$, $\xi(\omega)=x$ при $\omega \hm\in [a;b)$,
$\xi(\omega)\hm=y_i$  при  $\omega \hm\in
[c_i;d_i)$, $i\hm=\overline{1,n}$. Пусть чис\-ла~$x'$ и
$y'_i$, $i=\overline{1,n},$ таковы, что $\forall i$ $0 \hm\le x \hm\le x'
\hm\le y_i' \hm\le y_i$ и $\sum\limits_{i=1}^n(d_i-c_i)(y_i-y_i') \hm>
(b-a)(x'-x)$. Тогда существует последовательность операторов
диверсификации $D_n,\ldots,D_1$ такая, что $\xi'(\omega) \hm\equiv
D_n(\ldots(D_1(\xi(\omega)))\ldots) \hm= x'$ при $\omega \hm\in [a;b)$,
$\xi'(\omega) \hm\ge y_i'$ при $\omega \hm\in
[c_i;d_i)$, $i\hm=\overline{1,n}$, и $\xi'(\omega) \hm= \xi(\omega)$ при
остальных~$\omega$.} 

\smallskip

\noindent
Д\,о\,к\,а\,з\,а\,т\,е\,л\,ь\,с\,т\,в\,о\,.\ 
Обозначим
$\delta\hm=\sum\limits_{i=1}^n(d_i\hm-c_i)(y_i-y_i')\hm-(b\hm-a)(x'\hm-x) \hm>0$.
Построим последовательность $x_0\hm=x$, $x_i\hm=x_{i-1} \hm+((d_i-c_i)(y_i-y_i')\hm-\delta / n)/
(b-a)$, $i\hm=\overline{1,n}$.

Тогда $x_n\hm=x'$ и $(d_i\hm-c_i)(y_i\hm-y_i') \hm> (x_i\hm-x_{i-1})(b\hm-a)$. По
предыдущей лемме существуют последовательности операторов
диверсификации $D_{ik_i},\ldots,D_{i1}$, $i\hm=\overline{1,n},$ и
случайные величины $\xi_i\hm=D_{ik_i}(\ldots(D_{i1}(\xi_{i-1}))\ldots)$
такие, что $\xi_i(\omega) \hm= x_i$ при $\omega \hm\in [a;b)$,
$\xi_i(\omega) \hm\ge y_i'$ при $\omega \hm\in [c_i;d_i)$,
$\xi_i(\omega) \hm= \xi_{i-1}(\omega)$ при остальных~$\omega$,
$i\hm=\overline{1,n}$. Последовательность
$D_{nk_n},\ldots,D_{n1},\ldots,D_{1k_1},\ldots,D_{11}$ и будет
искомой. Лемма доказана.\hfill$\square$

\medskip

\noindent
\textbf{Теорема~3.}
\textit{Пусть $\xi_1,\xi_2 \in V$~--- почти наверное
ограниченные случайные величины. Тогда если $\forall
\gamma \hm \in (0;1]$ выполняется ${ES}_\gamma(\xi_2) \hm\le
{ES}_\gamma(\xi_1)$, то $\xi_2 \divers \xi_1$.}

\smallskip

\noindent
Д\,о\,к\,а\,з\,а\,т\,е\,л\,ь\,с\,т\,в\,о\,.\  \fbox{I}~Сначала
докажем теорему для простых, монотонно неубывающих и непрерывных
справа случайных величин~$\xi_1$ и~$\xi_2$. В~этом случае $\forall
\omega \hm\in \Omega\hm=[0;1)$ имеем
$\xi_i(\omega)\hm\equiv\xi_i^{(\omega)}$, $i\hm=1,2,$ и

\noindent
$$
{ES}_\gamma(\xi_i) = \fr{1}{\gamma} 
\int\limits_{1-\gamma}^1{\xi_i(\omega)\,d\omega}\,,\enskip \gamma\in (0;1]\,.
$$
Разобьем множество $\Omega\hm=[0;1)$ на конечное число непересекающихся
интервалов, на каждом из которых~$\xi_1$ и~$\xi_2$ постоянны:
$0{=}x_0\hm<x_1\hm<\ldots\hm<x_n{=}1$;
$\xi_1(\omega)\hm=a_i$, $\xi_2(\omega)\hm=b_i$ при
$\omega\hm\in [x_{i-1};x_i)$; $a_i\hm\le a_{i+1}$, $b_i \hm\le b_{i+1}$.
Пусть $1\le k_m \hm< \ldots \hm< k_1\le n$~--- номера, такие что $a_{k_i}
\hm< b_{k_i}$ (если таких номеров нет, то $\xi_1\hm\ge\xi_2$ и теорема
верна). Обозначим $p_i\hm=x_i\hm-x_{i-1}$,
$A\hm=\Set{\omega\hm\in\Omega}{\xi_1(\omega)\ge\xi_2(\omega)}\hm=\Omega
\backslash \bigsqcup\limits_{i=1}^m [x_{k_i-1};x_{k_i})$.
Множество~$A$ можно представить в виде объединения непересекающихся полуинтервалов.

По условию
\begin{multline*}
{ES}_{1-\gamma}(\xi_1) \ge {ES}_{1-\gamma}(\xi_2) \forall \gamma\in [0;1) \Rightarrow{}\\
{}\Rightarrow
\int\limits_\gamma^1 \xi_1(\omega)\,d\omega \ge \int\limits_\gamma^1 \xi_2(\omega)\,d\omega \Rightarrow {}\\
{}
\Rightarrow \int\limits_\gamma^1 (\xi_1(\omega)-\xi_2(\omega))\,d\omega \ge 0 \; \ \ \forall \gamma \in [0;1)\,.
\end{multline*}
Подставляя в последнее неравенство $x_{k_i-1}$ вмес\-то~$\gamma$ и используя равенство
$$
[x_{k_i-1};1) \backslash A = \bigsqcup\limits_{j=1}^{i} [x_{k_j-1};x_{k_j})\,,\enskip
i=\overline{1,m}\,,
$$
получим, что $\forall i \hm\in \overline{1,m}$ верно:
\begin{multline*}
\int\limits_{[x_{k_i-1};1)\cap A}(\xi_1(\omega)-\xi_2(\omega))\,d\omega \ge{}\\
{}\ge
-\!\!\! \int\limits_{[x_{k_i-1};1)\backslash A}\!\!\!(\xi_1(\omega)-\xi_2(\omega))\,d\omega =
\sum\limits_{j=1}^{i} p_{k_j}(b_{k_j} - a_{k_j})\,.
\end{multline*}
Отсюда по индукции легко получить, что существует
последовательность чисел
$0\hm \le c_m \hm\le \ldots \hm\le c_1 \hm\le c_0 \hm= 1$ такая, что $c_i\hm\ge
x_{k_i}$ и 
$$
\int\limits_{[c_i;c_{i-1})\cap A}(\xi_1(\omega)-\xi_2(\omega))\,d\omega
=p_{k_i}(b_{k_i} - a_{k_i})\,.
$$

Обозначим $A_i \hm= [c_i;c_{i-1}) \cap A$. Заметим, что
$A_i\cap A_j \hm= \varnothing$ при $i\hm\neq j$ и каждое из множеств~$A_i$ 
можно представить в виде объединения конечного чис\-ла
непересекающихся полуинтервалов, на каждом из которых~$\xi_1$ и~$\xi_2$ постоянны.

Фиксируем произвольное $\eps\hm>0$. Обозначим $\eta_0 \hm\equiv \xi_1$. Поскольку
$$
\int\limits_{[c_i;c_{i-1})\cap A}(\eta_0(\omega)-(\xi_2(\omega)-\eps))\,d\omega>p_{k_i}( (b_{k_i} - \eps) - a_{k_i})\,,
$$
то согласно лемме~6 существуют последовательности операторов
диверсификации $D_{il_i},\ldots,D_{i1}$ (обозначим через~$D_i'$ их
суперпозицию), $i\hm=\overline{1,m}$, и случайные величины
$\eta_i\hm=D_i'(\eta_{i-1})$, $i\hm=\overline{1,m}$, такие, что
$\eta_i(\omega)\hm=b_{k_i}\hm-\eps\hm=\xi_2(\omega)\hm-\eps$ при
$\omega \hm\in [x_{k_i-1};x_{k_i})$,
$\eta_i(\omega)\hm\ge\xi_2(\omega)\hm-\eps$ при $\omega \hm\in A_i$,
$\eta_i(\omega)\hm=\eta_{i-1}(\omega)$ при остальных~$\omega$. Легко
видеть, что построена последовательность операторов диверсификации
такая, что $\eta_m\hm\ge\xi_2\hm-\eps$. По лемме~3 $\xi_2 \divers \xi_1$.

\smallskip

\noindent
\fbox{II} Теперь докажем теорему для ограниченных, монотонно
неубывающих и непрерывных справа~$\xi_1$ и~$\xi_2$. Пусть $M\hm>0$
таково, что $|\xi_1| \hm< M$, $|\xi_2|\hm<M$. Фиксируем произвольные
$\eps\hm>0$ и $n\hm\in\mathbb{N}$ такие, что $n\hm\ge 8M/\eps$.
Разобьем отрезок $[-M;M]$ на $n$~равных частей точками
$-M\hm=a_0\hm<a_1\hm<\ldots$\linebreak $\ldots<a_n\hm=M$. Введем новые случайные величины
$\xi_1'(\omega)\hm=a_i$, если $a_{i-1}\hm\le\xi_1(\omega)\hm<a_i$,
$\xi_2'(\omega)\hm=a_{i-1}$, если $a_{i-1}\hm\le\xi_2(\omega)\hm<a_i$.
$\xi_1'$, $\xi_2'$~--- простые монотонно неубывающие случайные величины,
причем $\xi_1'\hm\ge\xi_1$, $\xi_2'\hm\le\xi_2$. Поэтому
${ES}_\gamma(\xi_2')\hm\le{ES}_\gamma(\xi_2)\hm\le{ES}_\gamma(\xi_1)\hm
\le{ES}_\gamma(\xi_1')$
$\forall \gamma\hm\in(0;1]$. По построению~$\xi_1'$ и~$\xi_2'$
непрерывны справа. Следовательно, по первой части доказательства,
существует последовательность операторов диверсификации
$D_n,\ldots,D_1$ такая, что $D_n(\ldots(D_1(\xi_1'))\ldots) \hm\ge
\xi_2' \hm- \eps / 2$. В~силу того что $n\hm\ge 8M/\eps$, имеем:
$|\xi_1'\hm-\xi_1|\hm\le M/n \hm\le \eps/4$ и аналогично $|\xi_2'\hm-\xi_2|\hm\le
\eps/4$. Таким образом, $D_n(\ldots(D_1(\xi_1))\ldots) \hm\ge \xi_2 \hm- \eps $. 
По лемме~3 $\xi_2 \divers \xi_1$.

\smallskip

\noindent
\fbox{III} Докажем теорему в общем виде. Пусть $\xi_1$, $\xi_2$~---
почти наверное ограниченные случайные величины и
${ES}_\gamma(\xi_2)\hm\le{ES}_\gamma(\xi_1)\ \forall
\gamma\hm\in(0;1]$. Фиксируем произвольное $\eps\hm>0$. Введем
$\xi_i'(\omega) \hm\equiv \xi_i^{(\omega)}$, $i\hm=1,2$. $\xi_i'$~---
ограниченные монотонно неубывающие непрерывные справа случайные
величины. В~то же время $\xi_i'\hm\sim\xi_i$, поэтому
${ES}_\gamma(\xi_2') \hm= {ES}_\gamma(\xi_2) \hm\le {ES}_\gamma(\xi_1) \hm=
{ES}_\gamma(\xi_1')$. Согласно второму пункту доказательства
существует последовательность операторов диверсификации
$D_n,\ldots,D_1$ такая, что
$D_n(\ldots(D_1(\xi_1'))\ldots)\hm\ge\xi_2'\hm-\eps/2$. Так как
$\xi_i'\hm\sim\xi_i$, то по лемме~2 существуют
$T_0,T_{n+1} \hm\in \classT$ такие, что $|\xi_1'\hm-\xi_1(T_0)| \hm\le
\eps/4$ почти наверное и $|\xi_2'\hm-\xi_2(T_{n+1})| \hm\le \eps/4$ почти
наверное. Обозначим $D_0\hm=D_{0,T_0}$, $D_{n+1}\hm=D_{0,T_{n+1}}$. Тогда
$D_{n+1}(D_n(\ldots(D_0(\xi_1))\ldots)) \hm\ge \xi_2 \hm- \eps$ и по лемме~3 
$\xi_2 \divers \xi_1$. Теорема дока-\linebreak зана.\hfill$\square$

\smallskip

Последняя теорема дает возможность про любые два портфеля с
ограниченными возможными потерями и прибылями сказать, является ли
один из них результатом диверсификации другого. Заметим, что на
практике большинство финансовых инструментов предполагает
неограниченную возможность потерь.


Определим последовательности

\noindent
$$
x_n=(n+1)!\,;\quad y_n=(n+2)n!\,;\quad
\mu_n=\fr{\nu_n}{\sum\limits_{i=1}^\infty \nu_i}\,,
$$ 
где 

\noindent
$$
\nu_n=\fr{\delta^{n^2}}{(n+2)!}\,,\quad \delta\in (0;1)\,.
$$
Разобьем множество $\Omega\hm=[0;1)$ на полуинтервалы точками
$0\hm=a_1\hm<a_2\hm=b_1\hm<b_2\hm=a_3\hm<a_4\hm=b_3\hm<\ldots$, так что
$a_{2i}\hm-a_{2i-1}\hm=b_{2i}\hm-b_{2i-1}\hm=\mu_i/2$, $i\hm\ge1$. Пронумеровав
произвольным способом все рациональные числа из интервала (0;\,0,5),
получим последовательность $\hat{\alpha}_1,\ldots$ Определим теперь
$\xi(\omega)\hm=x_i$ при $\omega\hm\in[a_{2i-1};\,a_{2i})$,
                                 $\xi(\omega)\hm=y_i$ при $\omega\hm\in[b_{2i-1};b_{2i})$,
                                 $\xi'(\omega)\hm=x_i+\hat{\alpha}_i(y_i-x_i)$ при $\omega\hm\in[a_{2i-1};a_{2i})$,
                                 $\xi'(\omega)\hm=y_i-\hat{\alpha}_i(y_i-x_i)$ при $\omega\hm\in[b_{2i-1};b_{2i})$.
Можно показать, что ${ES}_\gamma(\xi') \hm\le {ES}_\gamma(\xi)$,
$\forall \gamma \hm\in(0;1]$, но в то же время~$\xi'$ не является
результатом диверсификации портфеля~$\xi$. Таким образом,
полностью отказаться от требования ограниченности портфелей в
теореме~3 нельзя.


\section{Диверсификация и~информационные риски}

Информационным риск~--- риск возникновения убытков из-за неправильной
организации или умышленного нарушения информационных потоков и
систем организации. Можно применить предложенное понятие
диверсификации к информационным рискам. Под информационной системой
предприятия будем понимать систему взаимосвязанных информационных
объектов, которые реализуют информационный процесс в целях
эффективного функционирования предприятия. Будем рассматривать ее
как набор неких компонент, подсистем, каждая из которых
характеризуется случайной величиной $\xi(\omega)$ ($\Expect|\xi|\hm<
\infty$), определенной на вероятностном пространстве
$(\Omega=[0;1),\sigmaF,\mu)$ и отражающей убытки, которые несет
подсистема на траектории $\omega \hm\in \Omega$. Под сложением двух
подсистем $\xi_1(\omega)\hm+\xi_2(\omega)$ понимается система,
состоящая из совокупности подсистем~$\xi_1$ и~$\xi_2$, а под
умножением подсистемы~$\xi$ на константу $a\hm\in [0;1]$ понимается
информационная система, использующая $a\cdot 100\%$ ресурсов
подсистемы~$\xi$. Информационную систему~$\xi_2$ будем называть
результатом диверсификации системы~$\xi_1$, если систему~$\xi_1$
можно <<пересобрать>> (согласно правилам, указанным в определении~1)
так, чтобы $\forall \eps \hm> 0$ \  $\eta_n \hm\ge \xi_2 \hm- \eps$ почти
наверное.

Преобразуем пример~1 для случая информационного риска. Пусть имеется
три вычислительных машины, характеризующихся независимыми одинаково
распределенными убытками $X_1$, $X_2$, $X_3$, и имеется возможность
построить систему вида $p_1 X_1 \hm+ p_2 X_2 \hm+ p_3 X_3$, $p_i \hm\ge 0$,
$p_1\hm+p_2\hm+p_3\hm=1$ (это можно интерпретировать как факт огра\-ни\-чен\-ности
суммарной мощности системы). Рассуждения, приведенные в примере~1,
показывают, что система $(1/3) X_1 + (1/3) X_2 + (1/3) X_3$
является результатом диверсификации любой системы указанного выше
вида.


{\small\frenchspacing
{%\baselineskip=10.8pt
\addcontentsline{toc}{section}{Литература}
\begin{thebibliography}{9}

%\bibitem{halmosh}
%\Au{Halmos P.\,R.} Lectures on ergodic theory.~---- N.Y.: Chelsea Publishing Company, 1956.

\label{end\stat}

\bibitem{es}
\Au{Acerbi C., Tasche D.} On the coherence of expected Shortfall~//
J.~Banking Finance, 2002. Vol.~26. No.\,7. P.~1487--1503.
%{\sf http://www-m4.ma.tum.de/pers/ tasche/shortfall.pdf}.
 \end{thebibliography}
}
}


\end{multicols}       