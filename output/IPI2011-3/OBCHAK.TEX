\def\stat{abstr}
{%\hrule\par
%\vskip 7pt % 7pt
\raggedleft\Large \bf%\baselineskip=3.2ex
A\,B\,S\,T\,R\,A\,C\,T\,S \vskip 17pt
    \hrule
    \par
\vskip 21pt plus 6pt minus 3pt }

\label{st\stat}

%\def\rightmark{\ }

%1
\def\tit{AN ASYMPTOTICALLY OPTIMAL TEST FOR~THE~NUMBER OF~COMPONENTS OF~A~MIXTURE 
OF~PROBABILITY DISTRIBUTIONS}

\def\aut{V.\,E.~Bening$^1$, A.\,K.~Gorshenin$^2$, and~V.\,Yu.~Korolev$^3$}

\def\auf{$^1$Faculty of Computational Mathematics and Cybernetics, 
M.\,V.~Lomonosov Moscow State University; IPI RAN,\\
$\hphantom{^1}$bening@yandex.ru\\[1pt]
$^2$Faculty of Computational Mathematics and Cybernetics, 
M.\,V.~Lomonosov Moscow State University; IPI RAN,\\ 
$\hphantom{^1}$a.k.gorshenin@gmail.com\\
$^3$Faculty of Computational Mathematics and Cybernetics, 
M.\,V.~Lomonosov Moscow State University; IPI RAN,\\
$\hphantom{^1}$vkorolev@cs.msu.su}

\def\leftkol{\ } % ENGLISH ABSTRACTS}
\def\rightkol{\ } %ENGLISH ABSTRACTS}

\titele{\tit}{\aut}{\auf}{\leftkol}{\rightkol}

\vspace*{-2pt}

\noindent
The problem of statistical testing of hypotheses concerning 
the number of components in a mixture of probability distributions 
is considered. An asymptotically most powerful test is presented. 
Under rather weak conditions, the limit distributions, power loss, 
and the asymptotic deficiency are found. The application of this 
test to verification of hypotheses concerning the number of components in 
a mixture of uniform, normal, and gamma distributions is considered in detail.

\vspace*{-5pt}

\KWN{mixtures of probability distributions; asymptotically most powerful test; 
power loss; asymptotic deficiency}

%\thispagestyle{myheadings}


\vskip 10pt plus 6pt minus 3pt

%2
\def\tit{RECONSTRUCTION OF RANDOM FUNCTION DISTRIBUTIONS IN~SINGLE PHOTON 
EMISSION TOMOGRAPHY PROBLEMS USING TRIGONOMETRIC POLYNOMIAL APPROXIMATION 
OF~EXPONENTIAL MULTIPLIER}

\def\aut{V.\,G.~Ushakov$^1$ and O.\,V.~Shestakov$^2$}

\def\auf{$^1$Department of Mathematical Statistics, Faculty of Computational
Mathematics and Cybernetics,\\
$\hphantom{^1}$M.\,V.~Lomonosov Moscow State University; IPI RAN, vgushakov@mail.ru\\[1pt]
$^2$Department of Mathematical Statistics, Faculty of Computational
Mathematics and Cybernetics,\\
$\hphantom{^1}$M.\,V.~Lomonosov Moscow State University; IPI RAN, oshestakov@cs.msu.su}


\def\leftkol{\ } % ENGLISH ABSTRACTS}

\def\rightkol{\ } %ENGLISH ABSTRACTS}

\titele{\tit}{\aut}{\auf}{\leftkol}{\rightkol}

\vspace*{-2pt}

\noindent
This paper deals with the problem of reconstructing probabilistic distribution 
of random functions from distribution of integral transforms arising in the problems 
of emission tomography. The method of reconstruction is developed for the class of 
discrete random functions.

\vspace*{-5pt}

\KWN{emission tomography; Radon transform; projections; random functions}
%\pagebreak


\vskip 10pt plus 6pt minus 3pt


%3
\def\tit{DIVERSIFICATION AND ITS LINKS WITH RISK MEASURES}

\def\aut{D.\,O.~Jakovenko$^1$ and M.\,A.~Tselishchev$^2$}

\def\auf{$^1$FIDE Grandmaster, ms@cs.msu.su\\[1pt]
$^2$Department of Mathematical Statistics, Faculty of Computational Mathematics and
Cybernetics,\\
$\hphantom{^1}$M.\,V.~Lomonosov Moscow State University,  ms@cs.msu.su}


\def\leftkol{\ } % ENGLISH ABSTRACTS}

\def\rightkol{\ } %ENGLISH ABSTRACTS}

\titele{\tit}{\aut}{\auf}{\leftkol}{\rightkol}

\vspace*{-2pt}

\noindent
A new approach is proposed to the concept of diversification of 
investment portfolios which is defined as a binary relationship in 
the set of portfolios with finite first moments. It is shown that 
this relationship is, in some sense, a partial ordering. Important 
properties of such a definition are considered as well as  
necessary and sufficient condition of the comparability of 
portfolios, based on the coherent risk measure Expected Shortfall. 
As an example, an interpretation of the diversification of information 
risks is presented.

\vspace*{-5pt}

\KWN{diversification; investment portfolios; comparison of portfolios; 
coherent risk measure; Expected Shortfall; information risk}
\pagebreak

% \vskip 14pt plus 6pt minus 3pt

\def\leftkol{\ } % ENGLISH ABSTRACTS}
\def\rightkol{\ } %ENGLISH ABSTRACTS}

 %4
\def\tit{STABILITY BOUNDS FOR SOME QUEUEING SYSTEMS WITH CATASTROPHES}

\def\aut{A.\,I.~Zeifman$^1$, A.\,V.~Korotysheva$^2$, T.\,L.~Panfilova, and S.\,Ya.~Shorgin$^4$}

\def\auf{$^1$Vologda State Pedagogical University;
IPI RAN; VSCC CEMI RAS, a\_zeifman@mail.ru\\[1pt]
$^2$Vologda State Pedagogical University, a\_korotysheva@mail.ru\\[1pt]
$^3$Vologda State Pedagogical University, ptl-70@mail.ru\\[1pt]
$^4$IPI RAN, SShorgin@ipiran.ru}

\def\leftkol{ENGLISH ABSTRACTS}

\def\rightkol{ENGLISH ABSTRACTS}

\titele{\tit}{\aut}{\auf}{\leftkol}{\rightkol}


\noindent
Continuous-time Markovian queueing
models with catastrophes are considered. 
The bounds of stability for some characteristics of such systems are obtained.
Also, a queueing example is considered.

%\vspace*{-5pt}

\KWN{nonstationary queues; Markovian models with
catastrophes; stability bounds; approximations for limiting characteristics}

 \vskip 14pt plus 6pt minus 3pt

%5
\def\tit{ON A STATISTICAL PROBLEM FOR~RANDOM INTERNET-TYPE GRAPHS}

\def\aut{M.\,M.~Leri$^1$ and I.\,A.~Cheplyukova$^2$}

\def\auf{$^1$Institute of Applied Mathematical
Research, Karelian Research Center of the Russian Academy of
Sciences,\\ 
$\hphantom{^1}$leri@krc.karelia.ru\\[1pt]
$^2$Institute of Applied Mathematical
Research, Karelian Research Center of the Russian Academy of
Sciences,\\
$\hphantom{^1}$chia@krc.karelia.ru}

%\def\leftkol{ENGLISH ABSTRACTS}
%\def\rightkol{ENGLISH ABSTRACTS}

\titele{\tit}{\aut}{\auf}{\leftkol}{\rightkol}

%\vspace*{-2pt}

\noindent
There are considered random graphs of
Internet-type, i.\,e., graphs with vertex degrees drawn independently
from power-law distributions. By means of Monte-Carlo simulations,
a possibility of using the chi-square
goodness of fit test was investigated for verification of hypothesis that graph vertex
degrees are identically distributed. There were obtained the models
of the dependency of the strength of chi-square test on the graph
volume and vertex degrees distributions parameters and 
recommendations on choosing the number of intervals were given.


\KWN{random graphs; chi-square
goodness of fit test; simulation modeling}

 \vskip 14pt plus 6pt minus 3pt

%6
\def\tit{QUEUEING SYSTEM WITH NEGATIVE CUSTOMERS, BUNKER FOR~OUSTED CUSTOMERS, 
AND~DIFFERENT SERVICE RATES}

\def\aut{R.\,V.~Razumchik}
\def\auf{IPI RAN, rrazumchik@ieee.org}

\def\leftkol{ENGLISH ABSTRACTS}

\def\rightkol{ENGLISH ABSTRACTS}

\titele{\tit}{\aut}{\auf}{\leftkol}{\rightkol}

%\vspace*{-2pt}

\noindent
Consideration was given to the queuing system with 
Poisson flows of incoming positive and negative customers. 
For the positive customers, there is an infinite-capacity buffer.
The arriving negative customer knocks out a positive customer queued in the buffer 
and moves it to an infinite-capacity buffer of ousted customers (bunker). If the buffer 
is empty, then the
negative customer discharges the system without affecting it. After servicing the current 
customer, the server receives a customer from the buffer or, if the buffer is empty, the bunker.
The service times of customers arriving from buffer and bunker are distributed exponentially  
but with different parameters. Relations for calculation of the stationary distributions of 
the queues in the buffer and bunker are obtained.

\KWN{queueing system; negative customers; bunker; different service rates}
\pagebreak

%\vskip 14pt plus 6pt minus 3pt

%7
\def\tit{APPLICATION OF~THE~STATISTICAL METHOD 
AND~FINITE-DIFFERENCE METHOD FOR~STRONGLY IONIZED COLLISIONAL 
PLASMA DIAGNOSTICS PROBLEM SOLUTION BY~THE~FLAT PROBE}

\def\aut{I.\,A.~Kudryavtseva$^1$ and A.\,V.~Panteleyev$^2$}

\def\auf{$^1$Department of Mathematics and Cybernetics, Moscow Aviation Institute,
irina.home.mail@mail.ru\\[1pt]
$^2$Department of Mathematics and Cybernetics, Moscow Aviation Institute, avpanteleev@inbox.ru}


\def\leftkol{ENGLISH ABSTRACTS}

\def\rightkol{ENGLISH ABSTRACTS}

\titele{\tit}{\aut}{\auf}{\leftkol}{\rightkol}

%\vspace*{-2pt}

\noindent
A mathematical model, describing strongly ionized 
collisional plasma dynamics near the flat probe, is formulated. 
The mathematical model includes the Fokker--Planck  and 
Poisson equations. Two methods of getting solution are presented. 
One of these methods is the Monte-Carlo method, another is the combination 
of the splitting method and the Particle-In-Cell method. 


%\vspace*{-6pt}

\KWN{Monte-Carlo method; Particle-In-Cell method; 
splitting method; probe;  Fokker-Planck equation; Poisson equation}

%\vskip 18pt plus 6pt minus 3pt

 \vskip 14pt plus 6pt minus 3pt

%8
\def\tit{COMPARATIVE STUDY OF IMAGE SEGMENTATION
 ALGORITHMS PROCESSING QUALITY ON~METRIC BASE}

\def\aut{P.\,P.~Koltsov}

\def\auf{Scientific Research Institute for System Analysis of the Russian Academy 
of Sciences, koltsov@niisi.msk.ru}


\def\leftkol{ENGLISH ABSTRACTS}

\def\rightkol{ENGLISH ABSTRACTS}

\titele{\tit}{\aut}{\auf}{\leftkol}{\rightkol}

%\vspace*{-2pt}

\noindent 
The processing quality of four well-known 
digital image segmentation algorithms is under study. The set of artificial images 
under supervised distortions is used with \textit{a priori} given reference 
ground truth images. Algorithms processing results are compared with 
reference images by metrics with different features. The use of different 
metrics for image segmentation algorithms processing quality estimation 
and comparative study of the results helps to clear more exactly the 
features of the investigated algorithms.

\KWN{image processing; image processing quality estimation; image segmentation; 
edge detection; energy methods}

  \vskip 14pt plus 6pt minus 3pt
  
  %9
\def\tit{ON THE BERRY--ESSEEN TYPE INEQUALITIES FOR~POISSON RANDOM SUMS}


\def\aut{V.\,Yu.~Korolev$^1$, I.\,G.~Shevtsova$^2$, and~S.\,Ya.~Shorgin$^3$}
\def\auf{$^1$Department of Mathematical Statistics, Faculty of Computational
Mathematics and Cybernetics,\\
$\hphantom{^1}$M.\,V.~Lomonosov Moscow State University;
IPI RAN, vkorolev@cs.msu.su\\[1pt]
$^2$Department of Mathematical Statistics, Faculty of Computational
Mathematics and Cybernetics,\\
$\hphantom{^1}$M.\,V.~Lomonosov Moscow State University;
IPI RAN, ishevtsova@cs.msu.su\\[1pt]
$^3$IPI RAN, sshorgin@ipiran.ru}

\def\leftkol{ENGLISH ABSTRACTS}

\def\rightkol{ENGLISH ABSTRACTS}

\titele{\tit}{\aut}{\auf}{\leftkol}{\rightkol}

%\vspace*{-2pt}

\noindent
For the uniform distance between the distribution function $\Phi(x)$
of the standard normal random variable and the distribution function
$F_\lambda(x)$ of the Poisson random sum of independent identically
distributed random variables $X_1, X_2,\ldots$ with finite third
absolute moment, $\lambda\hm>0$ being the parameter of the Poisson
index, it is proved the inequality 
\begin{equation*}
\sup\limits_{x}\left\vert F_\lambda(x)-\Phi(x)\right\vert \le 0.4532\fr{{\sf E}|X_1-{\sf E}
X_1|^3}{(\D X_1)^{3/2}\sqrt{\lambda}}\,,\quad \la>0\,, 
\end{equation*}
which is similar to the Berry--Esseen estimate and uses the central
moments, unlike the known analogous inequalities based on the
noncentral moments.


%\vspace*{-5pt}

\KWN{Poisson random sum; central limit theorem; convergence rate estimate;
Berry--Esseen inequality; absolute constant}


\vskip 14pt plus 6pt minus 3pt


%10
\def\tit{ON ONE KERNEL DENSITY ESTIMATOR}

\def\aut{V.\,G.~Ushakov$^1$ and N.\,G.~Ushakov$^2$}

\def\auf{$^1$Department of Mathematical Statistics, 
Faculty of Computational Mathematics and Cybernetics,\\ 
$\hphantom{^1}$M.\,V.~Lomonosov Moscow State University; 
IPI RAN, vgushakov@mail.ru\\[1pt]
$^2$Institute of Microelectronics Technology and High Purity Materials,
Russian Academy of Sciences,\\
$\hphantom{^1}$ushakov@math.ntnu.no}

\titele{\tit}{\aut}{\auf}{\leftkol}{\rightkol}

%\vspace*{-6pt}

\noindent
The kernel density estimator based on the sinc kernel is investigated.
The main attention is paid to the analysis of the integrated mean squared for finite sample
sizes (nanosymptotic). The problems of estimation of the mode and 
of estimation of density derivatives are also considered.

%\vspace*{-2pt}

\KWN{nonparametric density estimator; kernel estimator; kernel of infinite order}
%\pagebreak

\vskip 14pt plus 6pt minus 3pt


%11
\def\tit{ON THE RATE OF CONVERGENCE OF~SAMPLE MEDIAN ABSOLUTE
DEVIATION DISTRIBUTION TO~THE~NORMAL LAW}

\def\aut{O.\,V.~Shestakov}

\def\auf{Department of Mathematical Statistics, Faculty of Computational
Mathematics and Cybernetics,\\
M.\,V.~Lomonosov Moscow State University;
IPI RAN, oshestakov@cs.msu.su}


\def\leftkol{ENGLISH ABSTRACTS}
\def\rightkol{ENGLISH ABSTRACTS}

\titele{\tit}{\aut}{\auf}{\leftkol}{\rightkol}

%\vspace*{-2pt}
\noindent
Some estimates for the rate of convergence of sample median absolute deviation
distribution to the normal law are obtained in the general and symmetric cases.

%\vspace*{-5pt}

\KWN{order statistics; sample median; median absolute deviation; normal distribution;
rate of convergence}

\vskip 14pt plus 6pt minus 3pt


%12
\def\tit{STRONG LAWS OF LARGE NUMBERS FOR A~NUMBER OF~ERROR-FREE BLOCKS
\mbox{UNDER~ERROR-CORRECTED~CODING}}

\def\aut{A.\,N.~Chuprunov$^2$ and I.~Fazekas$^2$}

\def\auf{$^1$Department of Mathematical Statistics and Probability, Chebotarev Institute of
Mathematics and Mechanics, $\hphantom{^1}$Kazan State
University, achuprunov@mail.ru\\[1pt]
$^2$Faculty of Informatics, University of Debrecen, Hungary, fazekas.istvan@inf.unideb.hu}


\def\leftkol{ENGLISH ABSTRACTS}

\def\rightkol{ENGLISH ABSTRACTS}

\titele{\tit}{\aut}{\auf}{\leftkol}{\rightkol}

%\vspace*{-2pt}
\noindent
The messages which contain blocks are considered. Each block 
was coded by error-corrected coding which can correct not more than 
$r$ errors. It is assumed that the number of errors in a block 
is Poissonian random variable with parameter~$\lambda$. 
Also, it is assumed that the number of errors in a message belongs 
to a subset of nonnegative integer numbers. Under 
there assumptions, the laws of large numbers for a number of error-free blocks
in the message were obtained.

 \label{end\stat}

%\vspace*{-5pt}

\KWN{allocation scheme; conditional probability; law of large numbers; error-corrected code}



%\pagebreak

 