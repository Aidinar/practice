\renewcommand*{\Pr}{\mathbb P}
\newcommand*{\E}{\mathbb E}
\newcommand*{\Dd}{\mathbb D}
%\newcommand*{\R}{\mathbb R}
\newcommand*{\Fd}{\mathfrak F}
\newcommand*{\N}{\mathbb N}
\newcommand*{\Id}{\mathcal I}

\def\stat{korgor}

\def\tit{АСИМПТОТИЧЕСКИ
ОПТИМАЛЬНЫЙ КРИТЕРИЙ ПРОВЕРКИ ГИПОТЕЗ О ЧИСЛЕ КОМПОНЕНТ СМЕСИ
ВЕРОЯТНОСТНЫХ РАСПРЕДЕЛЕНИЙ$^*$}

\def\titkol{Асимптотически
оптимальный критерий проверки гипотез о числе компонент смеси
вероятностных распределений}

\def\autkol{В.\,Е.~Бенинг, А.\,К.~Горшенин, В.\,Ю.~Королев}
\def\aut{В.\,Е.~Бенинг$^1$, А.\,К.~Горшенин$^2$, В.\,Ю.~Королев$^3$}

\titel{\tit}{\aut}{\autkol}{\titkol}

{\renewcommand{\thefootnote}{\fnsymbol{footnote}}\footnotetext[1]
{Работа поддержана Российским
фондом фундаментальных исследований (проекты 09-07-12032-офи-м,
11-07-00112а, 11-01-00515а и 11-01-12026-офи-м), а также Министерством образования и
науки РФ в рамках ФЦП <<Научные и научно-педагогические кадры
инновационной России на 2009--2013 годы>>.}}

\renewcommand{\thefootnote}{\arabic{footnote}}
\footnotetext[1]{Московский государственный
университет им.\ М.\,В.~Ломоносова, факультет вычислительной математики и кибернетики;
Институт проблем информатики Российской академии наук,
 bening@yandex.ru}
 \footnotetext[2]{Московский государственный
университет им.\ М.\,В.~Ломоносова, факультет вычислительной математики и кибернетики;
Институт проблем информатики Российской академии наук,
a.k.gorshenin@gmail.com}
  \footnotetext[3]{Московский государственный
университет им.\ М.\,В.~Ломоносова, факультет вычислительной математики и кибернетики;
Институт проблем информатики Российской академии наук,
vkorolev@cs.msu.su}
 
 \vspace*{4pt}

\Abst{Рассмотрена задача статистической
проверки гипотез о числе компонент смеси вероятностных
распределений. Приведен асимптотически наиболее мощный критерий.
При выполнении достаточно слабых условий найдены предельные
распределения, потеря мощности и асимптотический дефект. Подробно
рассмотрено применение данного критерия к проверке гипотез о числе
компонент смесей равномерных, нормальных и гам\-ма-рас\-пре\-де\-лений.}

\vspace*{2pt}

\KW{смеси вероятностных распределений;
асимптотически наиболее мощный критерий; потеря мощности;
асимптотический дефект}

\vspace*{4pt}

  \vskip 14pt plus 9pt minus 6pt

      \thispagestyle{headings}

      \begin{multicols}{2}
      
            \label{st\stat}




\section{Введение}

В некоторых конкретных прикладных задачах, использующих
математическую модель смеси ве\-роятностных распределений, крайне
важна кор\-ректная интерпретация полученных результатов\linebreak (например, в
физике турбулентной плазмы необходимо соотносить полученные
компоненты смеси с наблюдаемыми в плазме процессами). Поэтому при
выборе модели смеси важное значение имеет выбор не только типа
смеси (сдвиговая, масштабная или сдвиг-масштабная) и ядер
(смешиваемых распределений), но и числа компонент в подгоняемой к
реальным данным модели. С увеличением числа параметров возрастает
и согласие модели с данными. Однако известны примеры, когда
использование моделей с б$\acute{\mbox{o}}$льшим числом параметров не только
усложняет практические вычисления, но и приводит к неадекватной
интерпретации (более подробно об этом см.~\cite{Korolev2007, Korolev2010}).

Многие популярные алгоритмы (EM, SEM, MCEM-ал\-го\-рит\-мы и их
всевозможные модификации) для статистической декомпозиции смесей
используют заданное число компонент и в процессе итерационной
процедуры практически не могут менять это \textit{заранее} заданное
число. При этом в большинстве прикладных задач число компонент смеси
является неизвестным, и в лучшем случае возможно задать лишь
некоторую оценку сверху большим значением, что, как было
отмечено ранее, может приводить к неадекватным результатам. Известны
так называемые информационные критерии Акаике~\cite{Akaike1973} и
байесовский~\cite{Schwartz1978}, основанные на функции
правдоподобия. Однако в ряде практически значимых моделей нарушаются
условия регулярности, в частности в случае конечной смеси
нормальных законов, вообще говоря, функция правдоподобия не является
ограниченной, что приводит к необходимости использовать другие
критерии, а также накладывать дополнительные искусственные
технические условия для гарантии соблюдения формальных условий
регулярности.

В частности, в работах~\cite{Lo2001, Lo2005} для проверки гипотезы о
том, что $k_0$-ком\-по\-нент\-ная смесь нормальных распределений и
$k_1$-ком\-по\-нент\-ная смесь нормальных распределений одинаково близки к
истинному распределению, т.\,е.
$$
\E_h\left\{\log f(x;\,\theta^*)\right\}=\E_h\left\{\log g(x;\,\gamma^*)\right\}\,,
$$
против альтернативы, что одна из смесей приближает лучше, т.\,е.
$$
\E_h\left\{\log f(x;\,\theta^*)\right\}>\E_h\left\{\log g(x;\,\gamma^*)\right\}\,,
$$
предложено использовать статистику
\begin{equation}
\label{LR}
LR=LR(\Hat\theta,\Hat\gamma;\,x)=\sum\limits_{j=1}^n\log\fr{f(X_j;\,\Hat\theta)}
{g(X_j;\,\Hat\gamma)}\,.
\end{equation}

В качестве меры расстояния между данным и истинным распределением в
указанных работах используется информационный критерий
Куль\-ба\-ка--Лейб\-ле\-ра (KLIC)
\begin{multline*}
I(h:f;\,\theta)=\E_h\left\{\log\fr{h(X;\,\vartheta)}{f(X;\,\theta)}\right\}={}\\
\!{}=\int\log h(x;\,\vartheta)\,dH(x;\,\vartheta)-\int\log f(x;\,\theta)\,dH(x;\,\vartheta)\,.
\end{multline*}
Здесь $h(\cdot)$~--- истинная плотность, $f(\cdot)$~---
$k_1$-ком\-по\-нент\-ная смесь нормальных распределений, $g(\cdot)$~---
$k_0$-ком\-по\-нент\-ная смесь нормальных распределений, а символ $\E_h$
обозначает математическое\linebreak ожидание относительно~$h(\cdot)$;
$\theta^*$ и $\gamma^*$ минимизируют $I(h:f;\,\theta)$ и
$I(h:g;\,\gamma)$ соответственно, а величины~$\Hat\theta$ 
и~$\Hat\gamma$ являются оценками максимального правдоподобия для~$\theta^*$ 
и~$\gamma^*$ соответственно.

При этом для смесей нормальных законов в предположении
справедливости некоторых условий регулярности (а также
дополнительных предположений, позволяющих избегать неограниченности
функции правдоподобия) в работах~\cite{Lo2001, Lo2005} отмечено, что
статистика~\eqref{LR} в случае спра\-вед\-ли\-вости нулевой гипотезы имеет
асимптотическое распределение, являющееся взвешенной суммой
$\chi^2$-рас\-пре\-де\-ле\-ний. Но данный критерий не является оптимальным
хотя бы в каком-то смысле. Однако в работе~\cite{Vuong1989}
показано, что для произвольной смеси вероятностных распределений,
удовлетворяющих условиям регулярности, в случае справедливости
альтернативы значение статистики стремится к бесконечности, т.\,е.\
такой критерий является состоятельным.

В данной работе предлагается альтернативный асимптотически
оптимальный критерий проверки гипотез о числе компонент смеси
вероятностных распределений в смысле максимизации предельной
мощности критерия (см., например, книгу~\cite{Bening2000}).

\section{Постановка задачи}

Предположим, что каждое из независимых наблюдений
$\textbf{X}_n\hm=(X_1,\ldots,X_n)$ имеет плотность, представимую в виде
конечной $K$-ком\-по\-нент\-ной смеси плотностей некоторых законов
распределения $\psi_i(x),\,i\hm=1,\ldots,K,$ вида
$$
\sum_{i=1}^{K}p_i\psi_i(x),\,\sum_{i=1}^{K}p_i=1,\,p_i\geqslant0\,, \,i=1,\ldots,K\,.
$$

Отметим, что всюду предполагается, что смесь идентифицируема (если
для конкретных распределений для этого требуются дополнительные
условия, они оговариваются отдельно). Пусть $k$~--- некоторое
известное натуральное число. Требуется проверить гипотезу
$$
H_0:\ K=k
$$
против альтернативы
$$
H_1: K=k+1\,.
$$
Другими словами, требуется проверить значимость $(k+1)$-й компоненты
(например, если веса $p_i,\,i\hm=1,\ldots,k+1,$ упорядочены по
убыванию). Такая задача довольно типична и возникает, когда нужно
убедиться в значимости компоненты с малым весом или отбросить ее без
значимой потери информативности модели. Подобные проблемы особенно
существенны для так называемых сеточных методов разделения смесей
(см., например,~\cite{Korolev2007, Korolev2010}).

Для удобства асимптотического анализа предлагаемых критериев сведем
описанную выше задачу проверки гипотез о значении \textit{дискретного}
па\-ра\-мет\-ра~$K$ к задаче проверки гипотез о значении
\textit{непрерывного} параметра. С~этой целью для некоторого
$\theta\in[0,1]$ будем считать, что $X_1$ имеет плотность
\begin{equation}
\label{p} p(x,\theta)=(1-\theta) f(x)+\theta  g(x)\,,
\end{equation}
где
$$
f(x)=\sum\limits_{i=1}^{k}p_i\psi_i\,;\quad 
\sum\limits_{i=1}^{k}p_i=1\,;\ \  g(x)=\psi_{k+1}\,.
$$

Требуется проверить простую гипотезу
$$
H_0:\,\theta=0\\
$$
против последовательности сложных альтернатив вида (так
как равномерно наиболее мощного критерия для проверки простой
гипотезы против сложной, как правило, не существует)
$$
H_{n,1}:\,\theta=\fr{t}{\sqrt{n}}\,,\enskip 0<t\leqslant C\,,\  C>0\,,
$$
где параметр $t$ неизвестен. Фактически  осуществляется проверка
гипотезы о том, является ли рас\-смат\-ри\-ва\-емая смесь $k$-ком\-по\-нент\-ной
(при спра\-вед\-ли\-вости нулевой гипотезы~$H_0$) или $(k+1)$-ком\-по\-нент\-ной
(при справедливости альтернативы~$H_{n,1}$).
{\looseness=1

}

\section{Асимптотически
наиболее мощный критерий проверки гипотез о~числе компонент смеси}

Используем асимптотический подход, подробно описанный, например, в
книге~\cite{Bening2000}. Согласно лемме Ней\-ма\-на--Пир\-со\-на, для любого
фиксированного $t\in(0,C]$ наилучший критерий проверки гипотезы~$H_0$ 
против простой альтернативы
\begin{equation*}
H_{n,1}:\theta=\fr{t}{\sqrt{n}} 
\end{equation*}
основан на логарифме отношения правдоподобия
\begin{equation}
\label{Lambda}
\Lambda_n(t)=\sum\limits_{i=1}^{n}\left(l(X_i,tn^{-1/2})-l(X_i,0)\right)\,,
\end{equation}
где
$$
l(x,\theta)=\log p(x,\theta)\,.
$$

Мощность такого критерия уровня $\alpha\in(0,1)$ обозначим через
$\beta^*_n(t)$. Хотя статистика $\Lambda_n(t)$ не может быть
использована для построения критерия проверки гипотезы~$H_0$ против
альтернативы $H_{n,1}$ из-за того, что $t$ неизвестно, однако
$\beta^*_n(t)$ задает верхнюю границу для мощности любого критерия
при проверке гипотезы~$H_0$ против фиксированной альтернативы~$H_{n,1}$, $t>0$.

В дальнейшем понадобятся следующие функции:
$$
l^{(j)}(x)=\fr{\partial^{j}}{\partial\theta^{j}}\,l(x,\theta)\biggl|_{\theta=0},
\enskip j=1,2,\ldots\,,
$$
т.\,е.\
\begin{equation}
\left.
\begin{array}{rl}
\!\!\!l^{(1)}(x)&=\fr{g(x)-f(x)}{(1-\theta)f(x)+\theta g(x)}\biggl|_{\theta=0}=
\fr{g(x)}{f(x)}-1\,;\\[12pt]
\!\!\!l^{(2)}(x)&=-\fr{(g(x)-f(x))^2}{((1-\theta)f(x)+\theta
g(x))^2}\biggl|_{\theta=0}={}\\[12pt]
&\hspace*{30mm}=-\left(\fr{g(x)}{f(x)}-1\right)^2\,;\\[9pt]
l^{(j)}(x)&={}\\[9pt]
&\!\!\!\!\!\!\!\!\!\!\!\!\!\!\!\!\!\!\!\!\!\!\!\!{}=(-1)^{j+1}(j-1)!\fr{(g(x)-f(x))^j}{((1-\theta)f(x)+\theta
g(x))^j}\biggl|_{\theta=0}={}\\[9pt]
&\!\!\!\!\!\!\!\!\!\!\!\!\!\!\!\!\!\!\!\!\!\!{}=(-1)^{j+1}(j-1)!\left(\fr{g(x)}{f(x)}-1\right)^j,\,j\geqslant1\,.
\end{array}\!
\right\}\!\!
\label{lj(x)}
\end{equation}

Фишеровская информация имеет вид:
\begin{multline}
\label{I}
I=\E_0\left(l^{(1)}(X_1)\right)^2=
\int\limits_{-\infty}^\infty\left(\fr{g(x)}{f(x)}-1\right)^2
f(x)\,dx={}\\
{}=\int\limits_{-\infty}^\infty
f(x)\,dx-2\int\limits_{-\infty}^\infty
g(x)\,dx+\int\limits_{-\infty}^\infty
\fr{g^2(x)}{f(x)}\,dx={}\\
{}=\int\limits_{-\infty}^\infty
\fr{g^2(x)}{f(x)}\,dx-1\,.
\end{multline}
В дальнейшем будем пользоваться известными соотношениями:
\begin{equation}
\label{Fisher} \E_0 l^{(1)}(X_1)=0,\quad\E_0 l^{(2)}(X_1)=-I\,.
\end{equation}
Запишем разложение в ряд Тейлора логарифмической
производной из выражения~\eqref{Lambda}:
\begin{multline*}
 l(X_i,tn^{-1/2})-l(X_i,0)=\fr
{t}{\sqrt{n}}l^{(1)}(X_i)+{}\\
{}+\fr{t^2}{2n}l^{(2)}(X_i)+\cdots
\end{multline*}
Тогда
\begin{equation}
\label{Lambda_expansion} 
\Lambda_n(t)=tL_n^{(1)}-\fr{t^2I}{2}+\fr {t^2}{2\sqrt{n}} L_n^{(2)}+o(n^{-3/2})\,,%\cdots
\end{equation}
где
\begin{equation*}
\label{Ln1}
L_n^{(j)}=\fr{1}{\sqrt{n}}\sum\limits_{i=1}^{n}(l^{(j)}(X_i)-\E_0
l^{(j)}(X_i))\,,\enskip j=1,2,\ldots
\end{equation*}

Для получения формулы~\eqref{Lambda_expansion} были использованы
соотношения~\eqref{Fisher}. Критерий, основанный на статистике
$\Lambda_n(t)$, отвергает гипотезу~$H_0$ в пользу альтернативы~$H_{n,1}$, если
\begin{equation*}
\Lambda_n(t)>c_{n,t},
\end{equation*}
где критическое значение $c_{n,t}$ выбирается из условия
$$
\Pr_{n,0}(\Lambda_n(t)>c_{n,t})=\alpha\,,
$$
где символом $\Pr_{n,\theta}$ обозначено распределение~$\textbf{X}_n$ при $\theta\geqslant0$.

Так как $\Lambda_n(t)$ представляет собой сумму независимых
одинаково распределенных случайных величин, то из центральной
предельной теоремы (ЦПТ) при $n\rightarrow\infty$ следует, что
$\Lambda_n(t)$ имеет асимптотически нормальное распределение вида
\begin{align}
\label{DistrLn1} 
\mathfrak{L}(\Lambda_n(t)\mid H_0)&\rightarrow
N(-\fr{1}{2}\,t^2I,t^2I)\,;\\
\label{DistrLn2}
\mathfrak{L}(\Lambda_n(t)\mid H_{n,1})&\rightarrow
N(\fr{1}{2}\,t^2I,t^2I)\,.
\end{align}
Отсюда несложно получить предельное значение мощности (см.,
например, книгу~\cite{Bening2000}):

\noindent
$$
\beta^*_n(t)\rightarrow\beta^*(t)=\Phi(t\sqrt{I}-u_\alpha)\,,
$$
где $\Phi(u_\alpha)\hm=1\hm-\alpha$, символ $\Phi(\cdot)$ обозначает
функцию распределения стандартного нормального закона.

В качестве критерия с предельной мощностью $\beta^*(t)$ для
различения гипотез о числе компонент рассмотрим критерий, основанный
на статистике $L_n^{(1)}$, т.\,е.\
\begin{equation}
\label{L_n^1}L_n^{(1)}=\fr{1}{\sqrt{n}}\sum\limits_{i=1}^{n}\left(\fr{g(X_i)}{f(X_i)}-1\right)
\end{equation}

Рассмотрим достаточные условия, при которых логарифм отношения
правдоподобия асимптотически нормален, т.\,е.\ выполнены
условия~\eqref{DistrLn1} и~\eqref{DistrLn2} при $n\hm\to\infty$,
$0\hm<t\hm\leqslant C$. Это так называемое условие локальной асимптотической
нор\-маль\-ности, точнее возможность представить логарифм отношения
правдоподобия $\Lambda_n(t)$ в виде
\begin{equation}
\label{LAN} \Lambda_n(t)=t L_n^{(1)}-\fr{t^2 I}{2}+\xi_n(t)\,,
\end{equation}
где остаточный член $\xi_n(t)$ стремится к нулю по вероятности при
гипотезе~$H_0$ при $n\to\infty$.

Хорошо известен следующий результат (см., например,
статью~\cite{Hajek1962}). Предположим, что плотность $p(x,\theta)$
удовлетворяет следующим условиям:
\begin{description}
\item{{(A)}} При каждом $x\in\r$ плотность $p(x,\theta)$
абсолютно непрерывна по~$\theta$ в некоторой окрестности точки
$\theta=0$.
\item{{(B)}} При каждом $\theta$ из этой окрестности производная
$\partial p(x,\theta)/\partial\theta$ существует при почти
всех (по мере Лебега) $x\in\r$.
\item{{(C)}} Функция 
$$
I(\theta)\equiv\E_\theta\left(\fr{\partial
\log p(x,\theta)}{\partial\theta}\right)^2<\infty
$$ 
положительна и непрерывна в этой окрест\-ности.
\end{description}
Тогда выполнено условие~\eqref{LAN}.

Проверим выполнение этих условий для плотности $p(x,\theta)$ из
равенства~\eqref{p} и сформулируем результат в следующем виде.

\medskip

\noindent
\textbf{Лемма~1.} \textit{Пусть фишеровская информация~$I$ из
соотношения}~\eqref{I} \textit{конечна. Тогда для плотности $p(x,\theta)$ из
равенства}~\eqref{p} \textit{выполнено соотношение}~\eqref{LAN}.

\medskip

\noindent
Д\,о\,к\,а\,з\,а\,т\,е\,л\,ь\,с\,т\,в\,о\ будет заключаться в последовательной проверке условий
$(A)$--$(C)$. Обозначим через~$\delta$ необходимую правую окрестность
точки $\theta\hm=0$. Всюду в дальнейшем подразумевается выполнение
условия $0\leqslant\theta<\delta$.

\noindent {(\textit{A})} Очевидно, что линейная функция является абсолютно
непрерывной, поскольку если
\begin{equation*}
\sum\limits_i(b_i-a_i)<\delta_1\,,
\end{equation*}
где $(a_i,b_i)$~--- произвольная система попарно непересекающихся
интервалов, то для произвольной линейной функции вида $y(x)=ax+b$, $a$ и $b$~--- 
конечные фиксированные числа, получим
\begin{equation*}
\sum\limits_i|y(b_i)-y(a_i)|=|a|\sum\limits_i(b_i-a_i)<|a|\delta_1=\varepsilon
\end{equation*}
при соответствующем выборе~$\delta_1$. Плотность $p(x,\theta)$
представима в виде:
$
\theta (g(x)-f(x))+f(x)$,
т.\,е.\ при каждом фиксированном $x\in\r$
является линейной функцией по~$\theta$, а значит, является абсолютно
непрерывной по~$\theta$ из правой $\delta$-окрест\-ности нуля.

\noindent {(\textit{B})} Найдем производную
\begin{equation*}
\fr{\partial p(x,\theta)}{\partial\theta}=\fr{\partial
((1-\theta) f(x)+\theta g(x))}{\partial\theta}=g(x)-f(x)\,.
\end{equation*}
Очевидно, что данная производная существует при почти всех (по мере
Лебега) $x\in\r$ для любого~$\theta$ из правой $\delta$-окрест\-ности
нуля.

\noindent {(\textit{C})} Запишем выражение для функции $I(\theta)$ более
подробно:
\begin{multline*}
I(\theta)=\E_\theta\left(\fr{\partial \log
p(x,\theta)}{\partial\theta}\right)^2={}\\
{}=\E_\theta\left(\fr{g(x)-f(x)}{(1-\theta)
f(x)+\theta g(x)}\right)^2\,.
\end{multline*}
Функция $g(x)-f(x)\neq0$ почти наверное, а в силу известного
свойства интеграла Лебега это означает, что функция $I(\theta)\hm>0$ (в
силу неотри\-ца\-тель\-ности подынтегрального выражения условие
$I(\theta)\hm\neq 0$ эквивалентно $I(\theta)\hm>0$) почти наверное для
любого~$\theta$ из правой $\delta$-окрест\-ности нуля (включая и
значение $\theta=0$).

Далее, используя тот факт, что $0\hm\leqslant\theta\hm<\delta$, получаем оценку
для подынтегральной функции
\begin{multline}
\label{Ival} 
\!\!\!\fr{(g(x)-f(x))^2}{(1-\theta) f(x)+\theta
g(x)}\leqslant\fr{(g(x)-f(x))^2}{(1-\theta) f(x)+\theta
g(x)}\;\leqslant\\
{}\leqslant \fr{1}{(1-\delta)}\fr{(g(x)-f(x))^2}{f(x)}\,.
\end{multline}

Функция, стоящая в правой части~\eqref{Ival}, пред\-став\-ля\-ет собой
подынтегральное выражение для интеграла в фишеровской информации~$I$ 
из соотношения~\eqref{I}, который конечен по предположению леммы.

Воспользовавшись теоремой Лебега о мажорируемой сходимости (см.,
например, книгу~\cite{KF1976}), получим:
\begin{multline*}
\lim\limits_{\theta\to\theta_0}\int\fr{(g(x)-f(x))^2}{(1-\theta)
f(x)+\theta
g(x)}\,dx={}\\
{}=\int\fr{(g(x)-f(x))^2}{(1-\theta_0)
f(x)+\theta_0 g(x)}\,dx\,.
\end{multline*}
Данное соотношение означает непрерывность функции $I(\theta)$ в
правой $\delta$-окрест\-ности нуля (включая и значение $\theta=0$).\hfill$\square$

\medskip

Лемма~1 означает, что критерий $L_n^{(1)}$ из
равенства~\eqref{L_n^1} является асимптотически наиболее мощным.
Согласно ЦПТ, $L_n^{(1)}$ при $n\hm\rightarrow\infty$ имеет нормальное
распределение с параметрами~$0$ и~$I$ (при справедливости нулевой
гипотезы). Тогда критическое значение~$c^{(1)}_n$ может быть найдено
из соотношений:
$$
\Pr_{n,0}(L_n^{(1)}>c^{(1)}_n)=\alpha\,;\enskip 
c^{(1)}_n=\sqrt{I}u_\alpha+o(1)\,.
$$

Для отыскания точного распределения $L_n^{(1)}$ выпишем его
характеристическую функцию, пользуясь тем, что элементы выборки~---
независимые одинаково распределенные случайные величины
\begin{multline}
\label{Xf}\phi_{L_n^{(1)}}(z)=\left(e^{-iz/\sqrt{n}}\phi_\xi\left(\fr{z}{\sqrt
n}\right)\right)^n={}\\
{}=e^{-iz\sqrt{n}}\left(\phi_\xi\left(
\fr {z}{\sqrt n}\right)\right)^n,\,z\in\r\,;
\end{multline}

\vspace*{-6pt}

\noindent
\begin{multline*}
\xi=\fr{g(X_1)}{f(X_1)}\,;\ \ \ \ \phi_\xi(\fr {z}{\sqrt
n})=\E_\theta \exp{\left\{i\fr{z}{\sqrt
n}\,\xi\right\}}={}\\
{}=\int\limits_{-\infty}^\infty\exp{\left\{\fr {iz}{\sqrt
n}\,\fr{g(x)}{f(x)}\right\}}((1-\theta) f(x)+\theta
 g(x))\,dx\,.
\end{multline*}
Аналогично, используя соотношения
\begin{align*}
L_n^{(1)}&=\fr{1}{\sqrt{n}}\,\sum_{i=1}^n\fr{g(X_i)}{f(X_i)}-\sqrt{n}\,;
\\
\fr{g(X_i)}{f(X_i)}&\geqslant 0\,,\enskip  i=1,\ldots,n\,,
\end{align*}
можно применить аппарат преобразования Лапласа и записать
преобразование Лапласа случайной величины
$$
\eta_n=\fr{1}{\sqrt{n}}\sum_{i=1}^n\fr{g(X_i)}{f(X_i)}
$$
в виде
$$
\Phi_{\eta_n}(s)=\left(\Phi_\xi\left(\fr {s}{\sqrt
n}\right)\right)^n\,,
$$
где

\noindent
\begin{multline*}
\Phi_\xi\left(s\right)=\E_\theta
\exp{\left\{-s\xi\right\}}={}\\
{}=\int\limits_{-\infty}^\infty\exp{\left\{-s\fr{g(x)}{f(x)}\right\}}((1-\theta)
f(x)+\theta  g(x))\,dx\,.
\end{multline*}
Теперь, так как распределение неотрицательной случайной величины~$\eta_n$ 
однозначно определяется ее преобразованием Лапласа (см.,
например,~\cite{Feller2010}), то для практического определения
распределения случайной величины $L_n^{(1)}$ при конкретных~$f(x)$ и~$g(x)$ 
мож\-но использовать процедуры численного обращения
преобразования Лапласа.

\subsection{Асимптотическое поведение разности мощностей}

В работе~\cite{Bening2000} для нормированного предела раз\-ности
мощностей (также называемого \textit{потерей мощ\-ности}) для критерия,
основанного на статистике~\eqref{L_n^1}, получено выражение:

\noindent
\begin{multline}
\label{r(t)}
r(t)=\fr{t^3}{8\sqrt{I}}\varphi\left(u_\alpha-t\sqrt{I}\right)\left[\Dd_0
l^{(2)}(X_1)-{}\right.\\
\left.{}-I^{-1}\E_0^2l^{(1)}(X_1)l^{(2)}(X_1)\right]\,,
\end{multline}
где символ $\varphi(\cdot)$ обозначает плотность стандартного
нормального распределения. Введем обозначение для моментов порядка~$s$ 
случайной величины $\xi={{g(X_1)}/{f(X_1)}}$:

\noindent
\begin{multline}
\label{Psis}
\Psi_s=\E_0\xi=\E_0\left(\fr{g(X_1)}{f(X_1)}\right)^s=
\int\limits_{-\infty}^{+\infty}\fr{g^s(x)}{f^{s-1}(x)}\,dx,
\\ \ s=2,3,4\,.
\end{multline}
В этих обозначениях фишеровская информация~\eqref{I} равна

\noindent
\begin{equation}
\label{PsiI}
I=\Psi_2-1\,.
\end{equation}
Запишем величины, входящие в формулу~\eqref{r(t)}, с учетом
обозначений~\eqref{Psis} и формулы~\eqref{lj(x)} в виде

\noindent
\begin{align*}
\!\E_0 l^{(1)}(X_1)l^{(2)}(X_1)&=-\int\limits_{-\infty}^{+\infty}
\left(\fr{g(x)}{f(x)}-1\right)^3f(x)\,dx={}\hspace*{-1.25272pt}\\
&\hspace*{-27mm}{}=-\int\limits_{-\infty}^{+\infty}\left(\fr{g^3(x)}{f^3(x)}-3\fr{g^2(x)}{f^2(x)}
+3\fr{g(x)}{f(x)}-1\right)f(x)\,dx={}\\
&{}=-\Psi_3+3\Psi_2-2\,;\\
\E_0^2l^{(1)}(X_1)l^{(2)}(X_1)&=\left(-\Psi_3+3\Psi_2-2\right)^2={}\\
&\hspace*{-17mm}{}=\Psi_3^2-6\Psi_2\Psi_3+4\Psi_3+9\Psi_2^2-12\Psi_2+4\,.
\end{align*}


\noindent
С учетом формул~\eqref{Fisher} и~\eqref{PsiI} имеем:
\begin{multline*}
\Dd_0
l^{(2)}(X_1)=\E_0\left(l^{(2)}(X_1)\right)^2-\left(\E_0l^{(2)}(X_1)\right)^2={}\\
{}=\E_0\left(l^{(2)}(X_1)\right)^2-\left(\Psi_2-1\right)^2\,;
\end{multline*}

\vspace*{-6pt}

\noindent
\begin{multline*}
\E_0\left(l^{(2)}(X_1)\right)^2=\int\limits_{-\infty}^{+\infty}\left(
\fr{g(x)}{f(x)}-1\right)^4f(x)\,dx={}\\
{}=\int\limits_{-\infty}^{+\infty}\left(\fr{g^4(x)}{f^4(x)}-4\fr{g^3(x)}{f^3(x)}
+6\fr{g^2(x)}{f^2(x)}-4\fr{g(x)}{f(x)}+{}\right.\\
\left.{}+1
\vphantom{\fr{g^2(x)}{f^2(x)}}\right)f(x)\,dx
=\Psi_4-4\Psi_3+6\Psi_2-3\,;
\end{multline*}

\vspace*{-6pt}
\noindent
$$
\Dd_0 l^{(2)}(X_1)=\Psi_4-4\Psi_3+6\Psi_2-3-\left(\Psi_2-1\right)^2\,.
$$
Окончательно получаем для величины $r(t)$ из формулы~\eqref{r(t)}
следующее соотношение:
\begin{multline}
r(t)=\fr{t^3}{8\sqrt{\Psi_2-1}}\varphi\left(u_\alpha-t\sqrt{\Psi_2-1}\right)\times{}\\
{}\times\left(\Dd_0
l^{(2)}(X_1)-I^{-1}\E_0^2 l^{(1)}(X_1)l^{(2)}(X_1)\right)={}\\
{}=\fr{t^3}{8\sqrt{\Psi_2-1}}\varphi\left(u_\alpha-t\sqrt{\Psi_2-1}\right)\times
\\
{}\times\left(\Psi_4-4\Psi_3+6\Psi_2-3-\Psi^2_2+2\Psi_2-1+{}\right.\\
\left.{}+6\Psi_3-9\Psi_2+3
-\fr{\left(\Psi_3-1\right)^2}{\Psi_2-1}\right)={}\\
{}=
\fr{t^3}{8\sqrt{\Psi_2-1}}\,\varphi\left(u_\alpha-t\sqrt{\Psi_2-1}\right)\times{}\\
{}\times
\left(\Psi_4+2\Psi_3-\Psi^2_2-\Psi_2-\fr{\left(\Psi_3-1\right)^2}{\Psi_2-1}-1\right)\,.
\label{r(t)Ln}
\end{multline}
С помощью величины~$r(t)$ можно найти асимптотический дефект, так
как (см., например, книгу~\cite{Bening2000}) с учетом формул~\eqref{PsiI}
и~\eqref{r(t)Ln}
%\end{multicols}
%\hrule
\begin{multline}
d=\lim\limits_{n\to\infty}
d_n\equiv\lim\limits_{n\to\infty}(k_n-n)=
\fr{2r(t)}{t\sqrt{I}\varphi(t\sqrt{I}-u_\alpha)}={}\\
{}=2\fr{t^3}{8\sqrt{\Psi_2-1}}\,\varphi
\left(u_\alpha-t\sqrt{\Psi_2-1}\right)\times\\
{}\times
\left(\Psi_4+2\Psi_3-\Psi^2_2-\Psi_2-\fr{\left(\Psi_3-1\right)^2}{\Psi_2-1}-1\right)\Bigg /{}\\
{}
\left(t\sqrt{\Psi_2-1}\varphi(t\sqrt{\Psi_2-1}-u_\alpha)\right)={}\\
{}=\fr{t^2}{4\left(\Psi_2-1\right)}
\left(\vphantom{\fr{\left(\Psi_3-1\right)^2}{\Psi_2-1}}
\Psi_4+2\Psi_3-\Psi^2_2-\Psi_2-{}\right.\\
\left.{}-\fr{\left(\Psi_3-1\right)^2}{\Psi_2-1}-1\right)\,.
\label{d}
\end{multline}
Здесь через $d_n$ обозначен дефект, $k_n$~--- число наблюдений,
необходимых критерию, основанному на статистике $L_n^{(1)}$ из
формулы~\eqref{L_n^1}, для достижения той же мощности, что и
критерию, основанному на статистике $\Lambda_n(t)$ из
формулы~\eqref{Lambda}, при альтернативах вида $t/\sqrt{n}$. Первое
равенство в соотношении~\eqref{d} понимается в том смысле, что если
предел существует и конечен, то он, по определению, называется
асимптотическим дефектом.

\subsection{Условия сходимости моментных характеристик~$\Psi_s$}

Рассмотрим условия, гарантирующие конеч\-ность моментных
характеристик~\eqref{Psis} для не\-ко\-торых частных случаев смесей
распределений. Приведем подробный вывод для фишеровской\linebreak
информации~\eqref{I}, т.\,е.\ для моментной характеристики~$\Psi_2$.
Для остальных условия получаются аналогично с учетом того, что
\begin{align*}
\left(\sum\limits_i a_i\right)^2&=\sum\limits_i
a_i^2+2\sum\limits_{i\neq j} a_i a_j\,;\\
\left(\sum\limits_{i=1}^k
a_i\right)^3&=\sum\limits_i a_i^3+3\sum\limits_{i\neq j} a_i
a_j^2+k\prod\limits_i a_i\,,
\end{align*}
и для любых $a>0$, $b\geqslant0$ справедливо неравенство
$$
\fr{1}{a+b}\leqslant\fr{1}{a}\,.
$$

\textbf{Нормальное распределение.} В~этом случае
\begin{align*}
f(x)&=\sum\limits_{i=1}^{k}p_i\fr{1}{\sigma_i\sqrt{2\pi}}
\exp{\left\{-\fr{(x-a_i)^2}{2\sigma_i^2}\right\}};\
\sum\limits_{i=1}^{k}p_i=1\,,\\
g(x)&=\fr{1}{\sigma_{k+1}\sqrt{2\pi}}\exp{\left\{-\fr{(x-a_{k+1})^2}{2\sigma_{k+1}^2}\right\}}\,.
\end{align*}
Поэтому
\begin{multline*}
\fr{g^2(x)}{f(x)}=
{\exp\left\{-\fr{(x-a_{k+1})^2}{\sigma_{k+1}^2}\right\}}\Bigg/\\[3pt]
\left({\sqrt{2\pi}\sigma^2_{k+1}\sum_{j=1}^k\fr{p_j}{\sigma_j}
\exp\left\{-\fr{(x-a_j)^2}{2\sigma_j^2}\right\}}\right)={}\\[3pt]
{}={\exp\left\{-\fr{1}{\sigma_{k+1}^2}(x^2-2a_{k+1}x+a_{k+1}^2)\right\}}\Bigg /\\[3pt]
\!\!\!\!\left(\!{\sqrt{2\pi}\sigma^2_{k+1}
\sum_{j=1}^k\fr{p_j}{\sigma_j}
\exp\!\left\{\!-\fr{1}{2\sigma_j^2}(x^2-2a_jx+a_j^2)\!\right\}}\!\right)={}
\end{multline*}


\noindent
\begin{multline}
{}=\left(
\sqrt{2\pi}\sigma^2_{k+1}\sum_{j=1}^k\fr{p_j}{\sigma_j}
\exp\left\{x^2\left(\fr{1}{\sigma_{k+1}^2}-\fr{1}{2\sigma_j^2}\right)
+{}\right.\right.\\
\!\!\!\left.\left.{}+2x\left(\fr{a_j}{2\sigma_j^2}-\frac{a_{k+1}}{\sigma_{k+1}^2}\right)
+\left(\fr{a_{k+1}^2}{\sigma_{k+1}^2}-\fr{a_j^2}{2\sigma_j^2}\right)\right\}
\vphantom{\left(
\sqrt{2\pi}\sigma^2_{k+1}\sum_{j=1}^k\fr{p_j}{\sigma_j}
\exp\left\{x^2\left(\fr{1}{\sigma_{k+1}^2}-\fr{1}{2\sigma_j^2}\right)
+{}\right.\right.}
\right)^{-1}\!\!\!.\!\!\!
\label{inv}
\end{multline}
При $|x|\to\infty$ парабола относительно~$x$ вида

\noindent
\begin{multline*}
y_j(x)=x^2\left(\fr{1}{\sigma_{k+1}^2}-\fr{1}{2\sigma_j^2}\right)
+2x\left(\fr{a_j}{2\sigma_j^2}-\frac{a_{k+1}}
{\sigma_{k+1}^2}\right)+{}\\
{}+\left(\fr{a_{k+1}^2}{\sigma_{k+1}^2}-\fr{a_j^2}{2\sigma_j^2}\right)
\end{multline*}
может иметь пределом либо $+\infty$, либо $-\infty$. Первая
ситуация, очевидно, имеет место, если
$$
\fr{1}{\sigma_{k+1}^2}-\fr{1}{2\sigma_j^2}=
\fr{2\sigma_j^2-\sigma_{k+1}^2}{2\sigma_j^2\sigma_{k+1}^2}>0\,,
$$
что выполнено тогда и только тогда, когда
$$
\sigma_{k+1}^2<2\sigma_j^2\,.
$$
Поэтому, если выполнено условие
\begin{equation}
\label{CondSigmaNorm}
\sigma_{k+1}^2<2\max_{1\le j\le k}\sigma_j^2\,,
\end{equation}
то функция в знаменателе соотношения~\eqref{inv} неограниченно
возрастает при $|x|\hm\to\infty$, являясь при этом величиной порядка
$O\left(e^{\alpha x^2}\right)$ при некотором $\alpha\hm>0$. Обозначим
$\sigma_{j_0}^2=\max_{1\le j\le k}\sigma_j^2$. В~таком случае,
поскольку $\sigma_{k+1}^{-2}-({1}/{2})\sigma_{j_0}^{-2}>0$ и
\begin{multline*}
\fr{1}{\sigma_{k+1}^2}-\frac{1}{2\sigma_{j_0}^2}-
\left(\fr{1}{\sigma_{k+1}^2}-\fr{1}{2\sigma_i^2}\right)={}\\
{}=
\fr{1}{2}\left(\fr{1}{\sigma_{j_0}^2}-\fr{1}{\sigma_i^2}\right)<0\,,\enskip i=1,\ldots,k\,,
\end{multline*}

\vspace*{-12pt}

\noindent
\begin{multline*}
\fr{g^2(x)}{f(x)}=\bigg(C\exp\left\{x^2
\left(\fr{1}{\sigma_{k+1}^2}-\fr{1}{2\sigma_{j_0}^2}\right)
+{}\right.\\
\left.{}+2x\left(\fr{a_{j_0}}{2\sigma_{j_0}^2}-\fr{a_{k+1}}{\sigma_{k+1}^2}\right)
+\left(\fr{a_{k+1}^2}{\sigma_{k+1}^2}-\fr{a_{j_0}^2}{2\sigma_{j_0}^2}\right)\right\}
\times{}\\
{}\times\left(1+O\left(e^{-x^2}\right)\right)\bigg)^{-1}={}\\
{}=\left({O\left(e^{\alpha x^2}\right)\left(1+O\left(e^{-x^2}\right)\right)}\right)^{-1}=
O\left(e^{-\alpha x^2}\right)\,,\\
\alpha>0\,,\enskip C>0\,,\enskip  |x|\to\infty\,,
\end{multline*}
и интеграл в соотношении~\eqref{I} конечен.
%\columnbreak

Пусть вместо условия~\eqref{CondSigmaNorm} выполнено соотношение
$$
\sigma_{k+1}^2=2\sigma_{j_0}^2\,,
$$

\columnbreak


\noindent
Тогда для любого $\alpha$
\begin{multline*}
\fr{g^2(x)}{f(x)} =\fr{1}{e^{\alpha x+C}
\left(1+O\left(e^{-x^2}\right)\right)}={}\\
{}=\fr{1}{O\left(e^{\alpha
x}\right)\left(1+O\left(e^{-x^2}\right)\right)}=O\left(e^{-\alpha
x}\right)\,,\ |x|\to\infty.\hspace*{-2.57826pt}
\end{multline*}

Таким образом, можно заключить, что для конечности фишеровской
информации из формулы~\eqref{I} достаточно, чтобы было выполнено
условие~\eqref{CondSigmaNorm}.

Предположим теперь, что
\begin{equation}
\label{invCondSigmaNorm}
\sigma_{k+1}^2>2\max_{1\le j\le
k}\sigma_j^2\,,
\end{equation}
а интеграл из соотношения~\eqref{I} сходится. Из
неравенства~\eqref{invCondSigmaNorm} следует, что все коэффициенты
при $x^2$ в равенстве~\eqref{inv} отрицательны:
\begin{multline*}
\fr{1}{\sigma_{k+1}^2}-\fr{1}{2\sigma_{j_0}^2}-
\left(\fr{1}{\sigma_{k+1}^2}-\frac{1}{2\sigma_i^2}\right)={}\\
{}=
\fr{1}{2}\left(\fr{1}{\sigma_{j_0}^2}-\fr{1}{\sigma_i^2}\right)<0\,,\enskip i=1,\ldots,k\,.
\end{multline*}
Поэтому
\begin{multline*}
\fr{g^2(x)}{f(x)}=\Bigg( C\exp\left\{x^2
\left(\fr{1}{\sigma_{k+1}^2}-\fr{1}{2\sigma_{j_0}^2}\right)
+{}\right.\\
\left.{}+2x\left(\fr{a_{j_0}}{2\sigma_{j_0}^2}-\fr{a_{k+1}}{\sigma_{k+1}^2}\right)
+\left(\fr{a_{k+1}^2}{\sigma_{k+1}^2}-\fr{a_{j_0}^2}{2\sigma_{j_0}^2}\right)\right\}
\times{}\\
{}\times\sqrt{2\pi}\sigma^2_{k+1}\cdot
\left(1+O\left(e^{-x^2}\right)\right)\Bigg)^{-1}={}\\
{}=\left({O\left(e^{-\alpha
x^2}\right)\left(1+O\left(e^{-x^2}\right)\right)}\right)^{-1}=
O\left(e^{\alpha x^2}\right)\,,\\ \alpha>0\,,\enskip C>0\,,\enskip    |x|\to\infty\,,
\end{multline*}
поэтому интеграл из равенства~\eqref{I} расходится (аналогично
вычислению интеграла $\int\limits_{-\infty}^\infty e^{-x^2}\,dx$ переходим
к полярным координатам и получаем расходящийся интеграл
$\int\limits_0^\infty e^t\,dt$). Полученное противоречие доказывает, что
условие~\eqref{CondSigmaNorm} является необходимым и достаточным для
существования фишеровской информации в данном случае.

Вообще, достаточным условием сходимости моментных характеристик,
определяемых соотношением~\eqref{Psis}, в случае конечной смеси
нормальных законов является условие:

\noindent
$$
\sigma_{k+1}^2<\fr{s}{s-1}\max_{1\le j\le
k}\sigma_j^2\,,\enskip s=2,3,4\,,
$$
для каждой из моментных характеристик $\Psi_s$. Таким образом, для
конечности моментных характеристик~\eqref{Psis} сразу для всех
$s\hm=2,3,4$ достаточно выполнения условия:
\begin{equation}
\label{CondPsiNorm} 
\sigma_{k+1}^2<\fr{4}{3}\,\max_{1\le j\le
k}\sigma_j^2\,.
\end{equation}

\textbf{Гамма-распределение.} В~этом случае
\begin{gather*}
f(x)=\sum\limits_{i=1}^{k}p_i\fr{\alpha_i^{\beta_i}}
{\Gamma(\beta_i)}x^{\beta_i-1}e^{-\alpha_ix}\,;
\quad\sum\limits_{i=1}^{k}p_i=1\,;\\
g(x)=\fr{\alpha_{k+1}^{\beta_{k+1}}}
{\Gamma(\beta_{k+1})}x^{\beta_{k+1}-1}e^{-\alpha_{k+1}x},\,x\geqslant0\,.
\end{gather*}
Имеем
\begin{multline*}
\fr{g^2(x)}{f(x)}=\fr{\alpha_{k+1}^{2\beta_{k+1}}}
{\Gamma^2(\beta_{k+1})}\,x^{2\beta_{k+1}-2}e^{-2\alpha_{k+1}x}\Bigg/\\
{\sum\limits_{i=1}^{k}p_i \fr{\alpha_i^{\beta_i}}
{\Gamma(\beta_i)}x^{\beta_i-1}e^{-\alpha_ix}}={}\\
\!{}=\fr{\alpha_{k+1}^{2\beta_{k+1}}}{\Gamma^2(\beta_{k+1})}\,
\left({\sum\limits_{i=1}^{k}\fr{p_i\alpha_i^{\beta_i}}{\Gamma(\beta_i)}
{x^{\beta_i-2\beta_{k+1}+1}
e^{\left(2\alpha_{k+1}-\alpha_i\right)x}}}\right)^{-1}\!\!\!.\hspace*{-7.35753pt}
\end{multline*}
Сходимость интеграла из равенства~\eqref{I} зависит от показателей
функций в знаменателе. Отметим, что любое слагаемое, содержащее
множитель вида~$x^{1-\nu}$, $\nu\hm>1,$ имеет особенность в нуле, а
любое слагаемое, содержащее множитель вида~$e^{\mu x}$, $\mu\hm>0,$
имеет особенность на~$+\infty$. Таким образом, для сходимости
подынтегральной функции в~$0$ и на~$+\infty$ в знаменателе должны
быть слагаемые с подобными множителями. Итак, для ограниченности
подынтегральной функции хотя бы для одного номера~$i$ должно
существовать $\nu_i\hm>1$ и хотя бы для одного номера~$j$ должно
существовать $\mu_j>0$:
\begin{equation*}
\fr{g^2(x)}{f(x)}=O\left(x^{\nu_i-1}e^{-\mu_j x}\right) A(x)\,,
\end{equation*}
где $A(x)$~--- ограниченная функция, не имеющая особенностей в~0 
и на~$+\infty$, поэтому можно считать, что $A(x)\leqslant C$ и
\begin{equation*}
\fr{g^2(x)}{f(x)}=O\left(x^{\nu_i-1}e^{-\mu_j x}\right)\,,\enskip x\to0,+\infty\,.
\end{equation*}

Учитывая вид гамма-функ\-ции и сказанное ранее, получаем, что для
интегрируемости необходимо, чтобы неравенства $\nu_i\hm>1$, $\mu_j\hm>0$
выполнялись хотя бы для одного номера~$i$ и~$j$ соответственно.
Получим
\begin{align*}
\nu_i=2\beta_{k+1}-\beta_i>1&\Rightarrow\beta_{k+1}>\frac12\min\limits_{1\le
i\le k}
{(\beta_i+1)}\,;\\
\mu_j=2\alpha_{k+1}-\alpha_j>0&\Rightarrow\alpha_{k+1}>\frac12\min\limits_{1\le
j\le k}{\alpha_j}\,.
\end{align*}

Отдельно рассмотрим случай наличия  в знаменателе слагаемых вида~$e^{\mu_j x}$ 
(т.\,е.\ случай, когда $\nu_i\hm=1$ для некоторого номера~$i$). 
Из очевидного неравенства 
$$
\fr{1}{e^{\mu x}+b}\leqslant\fr{1}{e^{\mu x}}\,,\quad b\geqslant0\,,
$$ 
следует, что интеграл из соотношения~\eqref{I} сходится при $\nu_i\hm=1$, если $\mu_j\hm>0$. Если
же $\mu_j\hm\leqslant0$, то выводы о сходимости интеграла делаются на
основании поведения функций в сумме, для которой предполагается
отсутствие в ней слагаемых вида~$e^{\mu_j x}$, а этот случай был
рассмотрен ранее. Итак, интеграл сходится в случае, когда выполнены
условия
$$ %\label{CondSigmaGamma}
\beta_{k+1}\geqslant\fr{1}{2}\min\limits_{1\le i\le
k}{(\beta_i+1)}\,;\enskip \alpha_{k+1}>\fr{1}{2}\min\limits_{1\le j\le k}{\alpha_j}\,.
$$

Вообще, достаточным условием сходимости моментных характеристик в
соотношении~\eqref{Psis} в случае конечной смеси гам\-ма-рас\-пре\-де\-ле\-ний
являются условия
\begin{align*}
\beta_{k+1}&\geqslant s^{-1}\min\limits_{1\le i\le
k}{((s-1)\beta_i+1)}\,;\\
\alpha_{k+1}&>\fr{s-1}s\min\limits_{1\le j\le k}{\alpha_j}\,,\enskip s=2,3,4\,,
\end{align*}
для каждой из моментных характеристик~$\Psi_s$. Заметим, что при
$s\hm\geqslant2$, $\beta_0\hm=\min\limits_{1\le i\le k}{\beta_i}$ неравенство
$$
\fr{1}{2}\,\beta_0+\fr{1}{2}\leqslant\fr{s-1}{s}\,\beta_0+\fr{1}{s}
$$
справедливо только при условии $\beta_0\hm>1$. Таким образом, для
конечности моментных характеристик~\eqref{Psis} сразу для всех
$s\hm=2,3,4$ достаточные условия приобретают вид:
\begin{equation}
\left.
\begin{array}{rl}
\beta_{k+1}&\geqslant\\
&\!\!\!\!\!\!\!\!\!\!\!\!\!\!\!\!\!\geqslant\max\left\{\fr{1}{4}\min\limits_{1\le
i\le k}{(3\beta_i+1)},\fr{1}{2}\min\limits_{1\le i\le
k}{(\beta_i+1)}\right\}\,;\\[9pt]
\alpha_{k+1}&>\fr{3}{4}\min\limits_{1\le j\le
k}{\alpha_j}\,.
\end{array}\!
\right\}\!\!
\label{CondPsiGamma}
\end{equation}

\medskip

\noindent
\textbf{Замечание 1.} Предположим, что при проверке гипотез на реальных
данных добавляется компонента, для которой условия
вида~\eqref{CondPsiNorm} или~\eqref{CondPsiGamma} не выполнены.
Приведем алгоритм, позволяющий считать данные условия выполненными
без потери общности. Будем последовательно из плотности~$f(x)$
исключать компоненты, для которых условия
вида~\eqref{CondPsiNorm},~\eqref{CondPsiGamma}
выполнены. Вместо этой компоненты в~$f(x)$ добавим компоненту~$g(x)$
и перенормируем веса так, чтобы они в сумме равнялись единице. Тогда
мы получим некоторую смесь
\begin{equation*}
p_i(x,\theta)=(1-\theta) f_i(x)+\theta g_i(x)\,,
\end{equation*}
где
$$
f(x)=\sum\limits_{j\neq i}p_j\psi_j;\quad \sum\limits_{j\neq i}p_j=1;\  g_i(x)=\psi_i\,.
$$
В таких обозначениях условия вида~\eqref{CondPsiNorm}
или~\eqref{CondPsiGamma} уже будут выполнены. А~значит,
последовательно перебирая все номера~$i$, для которых указанные
условия будут выполнены, можно проверить, является ли смесь $k$- или
$(k+1)$-ком\-по\-нент\-ной. Если хотя бы в одном случае получится, что
верна нулевая гипотеза, значит, смесь является $k$-ком\-по\-нентной.

Итак, из вышесказанного получаем следующую теорему.

\medskip

\noindent
\textbf{Теорема 1.}
\textit{Пусть моментные характеристики $\Psi_s$, $s=2,3,4$, из
соотношения}~\eqref{Psis} \textit{конечны. Тогда для проверки гипотез о числе
компонент идентифицируемой смеси вероятностных распределений может
быть использован критерий, основанный на статистике
\begin{equation*}
L_n^{(1)}=\fr{1}{\sqrt{n}}\sum\limits_{i=1}^{n}\left(\fr{g(X_i)}{f(X_i)}-1\right)
\end{equation*}
и обладающий следующими свойствами:}
\begin{enumerate}
\item \textit{При справедливости нулевой гипотезы эта статистика имеет
нормальное распределение с па\-ра\-мет\-ра\-ми~$0$ и $\Psi_2-1$:}
\begin{equation*}
\mathfrak{L}(L_n^{(1)}\mid H_0)\rightarrow N(0,\Psi_2-1)\,.
\end{equation*}
\item \textit{При справедливости альтернативы эта статистика имеет
нормальное распределение с па\-ра\-мет\-ра\-ми $t\left(\Psi_2-1\right)$ и
$\Psi_2-1$:}
\begin{equation*}
\mathfrak{L}(L_n^{(1)}\mid H_{n,1})\rightarrow
N(t\left(\Psi_2-1\right),\Psi_2-1)\,.
\end{equation*}
\item \textit{Данный критерий является асимптотически наиболее мощным критерием с предельной мощностью
\emph{(}для заданного уровня $\alpha\in(0,1)$\emph{)} вида}
\begin{equation*}
\beta^*(t)=\Phi(t\sqrt{\Psi_2-1}-u_\alpha)\,.
\end{equation*}
\item \textit{Потеря мощности этого критерия равна}
\begin{multline*}
r(t)=\fr{t^3}{8\sqrt{\Psi_2-1}}\varphi\left(u_\alpha-t\sqrt{\Psi_2-1}\right)
\left(\vphantom{\fr{\left(\Psi_3-1\right)^2}{\Psi_2-1}-1}\Psi_4+{}\right.\\
\left.{}+2\Psi_3-\Psi^2_2-\Psi_2-\fr{\left(\Psi_3-1\right)^2}{\Psi_2-1}-1\right)\,.
\end{multline*}
\item \textit{Асимптотический дефект этого критерия равен}
\begin{multline*}
d=\fr{t^2}{4\left(\Psi_2-1\right)}
\left(\vphantom{\fr{\left(\Psi_3-1\right)^2}{\Psi_2-1}}
\Psi_4+2\Psi_3-\Psi^2_2-\Psi_2-{}\right.\\
\left.{}-\fr{\left(\Psi_3-1\right)^2}{\Psi_2-1}-1\right)\,.
\end{multline*}
\end{enumerate}


%\medskip

\noindent
\textbf{Замечание 2.}
В~случае рассмотрения конечной смеси нормальных
законов для конечности моментных характеристик~$\Psi_s$ достаточно
потребовать выполнение условий~\eqref{CondPsiNorm}, а в случае
рассмотрения конечной смеси гам\-ма-рас\-пре\-де\-ле\-ний~---
условий~\eqref{CondPsiGamma}.

\medskip

\noindent
\textbf{Замечание 3.}
Отметим, что в теореме~1 подразумевается
выполнение условия $\Psi_2>1$. Его справедливость в терминах
фишеровской информации была доказана при проверке условия~$(C)$
леммы~1.


\section{Примеры конкретных смесей вероятностных распределений}

В этом разделе рассмотрим частные случаи смесей, для которых в явном
виде можно выписать выражения для интегралов~\eqref{Psis}. Всюду
далее предполагается, что рассматриваются идентифицируемые смеси
(для нормального и гам\-ма-рас\-пре\-де\-ле\-ний это условие конечности чис\-ла
компонент смеси, что следует из результатов
работы~\cite{Teicher1963}). Для равномерного распределения
воспользуемся следующим утверждением.

\medskip

\noindent
\textbf{Утверждение 1.} \textit{Пусть $A(M)=\bigcup\limits_{i\in M}\left[a_i,b_i\right]$, 
где $M$~--- некоторое подмножество номеров. Обозначим семейство конечных смесей
равномерных распределений через
$$
H=\left\{F(x)=\sum\limits_{i=1}^k p_iF_i(x),\,
\sum\limits_{i=1}^k p_i=1,\,F_i\in\Fd\right\}\,,
$$ 
где
$\Fd=\{F(x,a_i,b_i),\,x\hm\in\r$, $-\infty<a_i<b_i<\infty$, $i\hm\in\N\}$~--- 
некоторое множество функций распределения равномерных законов
(возможно, конечное). Семейство~$H$ идентифицируемо тогда и только
тогда, когда
\begin{equation}
\label{Uident} A(M_1)\diagdown A(M_2)\neq\varnothing
\end{equation}
для всех возможных различных $M_1$ и $M_2$, $M_i\subseteq\N$.}

%\pagebreak
%\medskip

\noindent
Д\,о\,к\,а\,з\,а\,т\,е\,л\,ь\,с\,т\,в\,о\,.\ \ 
Для доказательства утверж\-де\-ния нужно показать, основываясь на
результатах работы~\cite{Yakowitz1968}, что условие~\eqref{Uident}
является необходимым и достаточным для линейной независимости
множества~$\Fd$ над полем действительных чисел.

Будем пользоваться тем фактом, что функции распределения линейно
зависимы тогда и только тогда, когда линейно зависимы их плотности. 
В~этом просто убедиться, дифференцируя
\begin{equation}
\label{sumFi}
\sum\limits_{i=1}^N \alpha_iF_i=0\,,\enskip\forall x\in\r\,,
\end{equation}
где $\alpha_i$~--- действительные числа, одновременно не
равные нулю. Для плотностей имеем соответственно тождество

\begin{equation}
\label{sumfi}
\sum\limits_{i=1}^N \alpha_if_i=0\,.
\end{equation}
Очевидно, что интегрирование тождества~\eqref{sumfi} приводит к
соотношению~\eqref{sumFi}.

\smallskip

\textit{Необходимость.} Покажем, что если $\Fd$ является линейно
независимым множеством над~$\r$, то условие~\eqref{Uident}
выполнено. Предположим, что это не так. Тогда
\begin{equation*}
\sum\limits_{i=1}^N \alpha_iF_i=0
\end{equation*}
только для тривиального набора чисел $\{\alpha_i\}$ и
условие~\eqref{Uident} не выполняется, т.\,е.\ для некоторых параметров
$a<b<c<d$ (без ограничения общности рассмотрим случай совпадения
числа элементов в $M_1$ и $M_2$; сказанное останется верным и для
разного числа элементов в сумме в равенстве~\eqref{sumFi})
$$
[a,b]\cup[b,d]=[a,c]\cup[c,d]\,.
$$
Но
\begin{multline*}
\alpha_1\fr{1}{b-a}\Id([a,b])+\alpha_2\fr{1}{d-b}\Id([b,d])+{}\\
{}+
\alpha_3\fr{1}{c-a}\Id([a,c])+\alpha_4\fr{1}{d-c}\Id([c,d])=0
\end{multline*}
не только для тривиального набора коэффициентов. Так, в качестве
такого набора можно рассмотреть $(b-a,d-b,-c+a,-d+c)$. Получаем
противоречие с линейной независимостью множества~$\Fd$.

\smallskip

\textit{Достаточность.} Теперь предположим, что условие~\eqref{Uident}
выполнено, а $\Fd$ является линейно зависимым множеством над~$\r$.
Справедливость условия~\eqref{Uident} означает, что существует
отрезок $[c_k,d_k]$ такой, что
$$
[c_k,d_k]\subseteq[a_k, b_k]:[c_k,d_k]\nsubseteq\left( A(M_1)\cap
A(M_2)\right)\,.
$$
Теперь, рассматривая линейно зависимые плот\-ности в~\eqref{sumfi} при
$x\in[c_k,d_k]$, получаем
%\noindent
$$
\alpha_k C\equiv0\,,\enskip C>0\,,
$$
поэтому $\alpha_k=0$, причем равенство нулю коэффициента справедливо
для всех $x\hm\in\r$, так как коэффициенты в сумме в
тождестве~\eqref{sumfi} от~$x$ не зависят и тождество~\eqref{sumfi}
должно выполняться сразу для всех~$x$. Таким образом, коэффициент
перед плотностью равномерного на сегменте $[a_k, b_k]$ распределения
в~\eqref{sumfi} равен нулю, при этом условие~\eqref{Uident}
выполняется, а $\Fd$~--- линейно зависимое множество. Далее, поступая
описанным выше способом, получаем, что все коэффициенты при
плотностях равны нулю. Получаем противоречие с предположением о
линейной зависимости. Значит, множество~$\Fd$ над~$\r$ не может быть
линейно зависимым при выполнении условия~\eqref{Uident}.\hfill$\square$

\smallskip

\textbf{Равномерное распределение.} В~данном случае

\noindent
\begin{gather*}
f(x)=\sum\limits_{i=1}^{k}p_i\fr{1}{b_i-a_i}\Id([a_i,b_i])\,;\quad 
\sum\limits_{i=1}^{k}p_i=1;\,\\
g(x)=\fr{1}{b_{k+1}-a_{k+1}}\Id([a_{k+1},b_{k+1}])\,,
\end{gather*}
где символом $\Id(\cdot)$ обозначен индикатор соответствующего
множества. Предположим, что
%\noindent
$$
[a_{k+1},b_{k+1}]\bigcap\left(\bigcup\limits_{i=1}^k[a_i,b_i]\right)\neq\emptyset\,.
$$
Тогда моменты~\eqref{Psis}, которые могут быть использованы для
нахождения величины $r(t)$ из равенства~\eqref{r(t)Ln}, равны
(отметим, что в данном случае в силу определения равномерного
распределения интегрирование ведется по ограниченному множеству, а
подынтегральная функция не имеет особенностей)

\noindent
\begin{multline*}
\Psi_s=\int\limits_{-\infty}^\infty\fr{g^s(x)}{f^s(x)}f(x)\,dx=
\fr{1}{(b_{k+1}-a_{k+1})^s}\times\\
{}\times
\sum\limits_{j=1}^k\fr{1}{b_j-a_j}
\int\limits_{a_j}^{b_j}\Id_{k+1}([a_j,b_j])
\bigg(\fr{1}{b_j-a_j}+{}\\
{}+\sum_{\substack{i=1,\ldots,k\\i\neq
j}}\fr{\Id_i([a_j,b_j])}{b_i-a_i}\bigg)^{-s}\,dx=
\fr{1}{(b_{k+1}-a_{k+1})^s}\times{}\\
\!\!\!{}\times
\sum\limits_{j=1}^k\fr{1}{b_j-a_j}
\int\limits_{A_j}\!\!\bigg(\fr{1}{b_j-a_j}+\!\!\sum_{\substack{i=1,\ldots,k\\i\neq
j}}\!\!\!\fr{\Id_i([a_j,b_j])}{b_i-a_i}\bigg)^{-s}\!\!dx={}\hspace*{-3.11491pt}\\
{}=\fr{1}{(b_{k+1}-a_{k+1})^s}\sum\limits_{j=1}^k\fr{1}{b_j-a_j}
\sum\limits_{t=1}^{m^j-1}(c_{t+1}^j-c_t^j)\times{}
\\
\!\!\!\!{}\times
\bigg(\fr{1}{b_j-a_j}+\!\!\sum_{\substack{i=1,\ldots,k\\i\neq
j}}\fr{\Id_i([c_t^j,c_{t+1}^j])}{b_i-a_i}\bigg)^{-s},\
s=2,3,4,
\end{multline*}

\end{multicols}

%\hrule

\noindent
где
\begin{equation*}
\Id_j([c_i,c_{i+1}])=
\begin{cases}
1,&\mbox{\ если }a_j\leqslant c_i\leqslant c_{i+1}\leqslant b_j\,;\\
0,&\ [c_i,c_{i+1}]\notin[a_j,b_j]\,;
\end{cases}
\end{equation*}

\noindent
\begin{equation*}
A_j=[a_j,b_j]\cap[a_{k+1},b_{k+1}]\,;
\end{equation*}
набор $c_1^j,\ldots,c_{m^j}^j$ составляет разбиение множества~$A_j$
на непересекающиеся (за исключением границ) сегменты
$[c_t^j,c_{t+1}^j])$. При этом считаем, что если $A_j=\emptyset$, то
$m^j\hm=1$ и все $c_i^j\equiv0$.

Теперь выпишем характеристическую функцию из соотношения~\eqref{Xf}
для данного случая:
\begin{multline*}
\phi_{L_n^{(1)}}(z)=\left(e^{-i{z}/{\sqrt{n}}}\phi_\xi\left(\fr
{z}{\sqrt n}\right)\right)^n=e^{-iz\sqrt{n}}\left(\phi_\xi\left(\fr {z}{\sqrt n}\right)\right)^n={}\\
{}= 
e^{-iz\sqrt{n}}\bigg(\int\limits_{-\infty}^\infty\exp{\bigg\{\fr
{iz}{\sqrt n}\,\fr{g(x)}{f(x)}\bigg\}}((1-\theta)
f(x)+\theta  g(x))\,dx\bigg)^{n}={}\\
{}=e^{-iz\sqrt{n}}\bigg((1-\theta)\sum\limits_{j=1}^k\fr{1}{b_j-a_j}
\int\limits_{a_j}^{b_j}\exp{\left\{\fr {iz}{\sqrt n}
\fr{{\Id_{k+1}([a_j,b_j])}/({b_{k+1}-a_{k+1}})}
{(b_j-a_j)^{-1}+\sum_{\substack{i=1,\ldots,k\\i\neq
j}}{\Id_i([a_j,b_j])}/(b_i-a_i)}\right\}}\,dx+{}\\
{}+\theta\fr{1}{b_{k+1}-a_{k+1}}
\int\limits_{a_{k+1}}^{b_{k+1}}\exp\left\{\fr {iz}{\sqrt n}
\left({(b_{k+1}-
a_{k+1})\sum\limits_{i=1}^k\fr{\Id_i([a_{k+1},b_{k+1}])}
{b_i-a_i}}\right)^{-1}\right\}\,dx\bigg)^{n}={}\\
{}=e^{-iz\sqrt{n}}\bigg((1-\theta)\sum\limits_{j=1}^k\fr{1}{b_j-a_j}
\sum\limits_{t=1}^{m^j-1}(c_{t+1}^j-c_t^j)\times{}\\
{}\times\exp{\bigg\{\fr {iz}{\sqrt n}\fr{1}{b_{k+1}-a_{k+1}}
\bigg(\fr{1}{b_j-a_j}+\sum_{\substack{i=1,\ldots,k\\i\neq
j}}\frac{\Id_i([a_j,b_j])}{b_i-a_i}\bigg)^{-1}\bigg\}}+{}\\
{}+\theta\fr{1}{b_{k+1}-a_{k+1}}
\sum\limits_{t=1}^{l-1}(c_{t+1}^{k+1}-c_t^{k+1})\exp{\bigg\{\fr
{iz}{\sqrt n}
\bigg({(b_{k+1}-a_{k+1})\sum\limits_{i=1}^k\fr{\Id_i([a_{k+1},b_{k+1}])}
{b_i-a_i}}\bigg)^{-1}\bigg\}}\bigg)^{n}\,.
\end{multline*}

Аналогично получается выражение для преобразования Лапласа:
\begin{multline*}
\Phi_{L_n^{(1)}}
(s)=e^{s\sqrt{n}}\bigg((1-\theta)\sum\limits_{j=1}^k\fr{1}{b_j-a_j}
\sum\limits_{t=1}^{m^j-1}(c_{t+1}^j-c_t^j)\times{}\\
{}\times\exp{\bigg\{-\fr{s}{\sqrt n}\,\fr{1}{b_{k+1}-a_{k+1}}
\bigg(\fr{1}{b_j-a_j}+\sum_{\substack{i=1,\ldots,k\\i\neq
j}}\fr{\Id_i([a_j,b_j])}{b_i-a_i}\bigg)^{-1}\bigg\}}+{}\\
{}+\theta\fr{1}{b_{k+1}-a_{k+1}}
\sum\limits_{t=1}^{l-1}(c_{t+1}^{k+1}-c_t^{k+1})\exp{\bigg\{-\fr
{s}{\sqrt n}
\,\bigg({(b_{k+1}-a_{k+1})\sum\limits_{i=1}^k\fr{\Id_i([a_{k+1},b_{k+1}])}
{b_i-a_i}}\bigg)^{-1}\bigg\}}\bigg)^{n}\,,\enskip
s>0\notag\,.
\end{multline*}

\textbf{Нормальное распределение.} Пусть выполнены
условия~\eqref{CondPsiNorm}. Проверяется гипотеза о том, что
плотность каждого наблюдения является нормальным законом, против
альтернативы, что плотность представляет собой смесь двух нормальных
законов. В~данном случае

\noindent
\begin{align*}
f(x)&=\fr{1}{\sigma_1\sqrt{2\pi}}\exp{\left\{-\fr{(x-a_1)^2}{2\sigma_1^2}\right\}}\,;\\
g(x)&=\fr{1}{\sigma_2\sqrt{2\pi}}\exp{\left\{-\fr{(x-a_2)^2}{2\sigma_2^2}\right\}}\,.
\end{align*}
Моменты, определенные в соотношении~\eqref{Psis}, которые входят в
величину $r(t)$ (см.~\eqref{r(t)Ln}), равны
\begin{multline*}
\Psi_s=\fr{\sigma^{s-1}_1}{\sigma_2^s\sqrt{2\pi}}\int\limits_{-\infty}^\infty
\exp{\left\{-s\fr{(x-a_2)^2}{2\sigma_2^2}+(s-1)\fr{(x-a_1)^2}{2\sigma_1^2}\right\}}\,dx={}\\[3pt]
{}=\fr{\sigma^{s-1}_1}{\sigma_2^s\sqrt{2\pi}}\int\limits_{-\infty}^\infty
\exp{\left\{-x^2\left(\fr{s}{2\sigma_2^2}-\fr{s-1}{2\sigma_1^2}\right)+x\left(
\fr{sa_2}{\sigma_2^2}-\fr{(s-1)a_1}{\sigma_1^2}
\right)+\fr{(s-1)a_1^2}{2\sigma_1^2}-\fr{s a_2^2}{2\sigma_2^2}\right\}}\,dx={}\\[3pt]
{}=\fr{\sigma^{s-1}_1}{\sigma_2^s\sqrt{2\pi}}
\sqrt{\fr\pi{{s}/({2\sigma_2^2})-(s-1)/(2\sigma_1^2)}}\times{}\\[3pt]
{}\times\exp{\left\{\left(\fr{s a_2}{\sigma_2^2}-
\fr{(s-1)a_1}{\sigma_1^2}
\right)^2\biggl{/}\left(4\left(\fr{s}{2\sigma_2^2}-\fr{s-1}{2\sigma_1^2}\right)\right)
+\fr{(s-1)a_1^2}{2\sigma_1^2}-\fr{s a_2^2}{2\sigma_2^2}\right\}}={}\\[3pt]
{}=\fr{\sigma_1^s}{\sigma_2^{s-1}\sqrt{s\sigma_1^2-(s-1)\sigma_2^2}}\times{}\\[3pt]
{}\times\exp{\left\{\left(\fr{s a_2}{\sigma_2^2}-\fr{(s-1)a_1}{\sigma_1^2}
\right)^2\biggl{/}\left(2\left(\fr{s}{\sigma_2^2}-\fr{s-1}{\sigma_1^2}\right)\right)
+\fr{(s-1)a_1^2}{2\sigma_1^2}-\fr{s a_2^2}{2\sigma_2^2}\right\}}={}\\[3pt]
{}=\fr{\sigma_1^s}{\sigma_2^{s-1}\sqrt{s\sigma_1^2-(s-1)\sigma_2^2}}\times{}\\[3pt]
{}\times\exp{\left\{\left(s a_2 \fr{\sigma_1}{\sigma_2}-(s-1)a_1
\fr{\sigma_2}{\sigma_1}\right)^2
\biggl{/}\left(2\left(s\sigma_1^2-(s-1)\sigma_2^2\right)\right)
+\fr{(s-1)a_1^2}{2\sigma_1^2}-\fr{s a_2^2}{2\sigma_2^2}\right\}}\,,\enskip
s=2,3,4\,.
\end{multline*}

Отметим, что при нахождении значения интеграла использовалось
следующее соотношение:
$$
\int\limits_{-\infty}^{+\infty}
\exp{\left\{-ax^2+bx+c\right\}}\,dx=\sqrt{\fr\pi
a}\exp{\left\{\fr {b^2}{4a}+c\right\}}\,.
$$

%\smallskip

\textbf{Гамма-распределение.} Пусть выполнены
условия~\eqref{CondPsiGamma}. Проверяем гипотезу о том, что
плот\-ность каждого наблюдения определяется гам\-ма-рас\-пре\-де\-ле\-ни\-ем,
против альтернативы, что плот\-ность представляет собой смесь двух
гамма-распределений. В~данном случае
\begin{align*}
f(x)&=\fr{\alpha_1^{\beta_1}}{\Gamma(\beta_1)}x^{\beta_1-1}e^{-\alpha_1x}\,;\\[6pt]
g(x)&=\fr{\alpha_2^{\beta_2}}{\Gamma(\beta_2)}x^{\beta_2-1}e^{-\alpha_2x}\,,\enskip x\geqslant0\,.
\end{align*}
Моменты, определенные в соотношении~\eqref{Psis}, которые входят в
величину~$r(t)$ (см.~\eqref{r(t)Ln}), равны:
\begin{multline*}
\Psi_s=\fr{\Gamma^{s-1}(\beta_1)\alpha_2^{s\beta_2}}{\Gamma^s(\beta_2)\alpha_1^{(s-1)\beta_1}}
\int\limits_0^\infty x^{s\beta_2-(s-1)\beta_1-1}e^{-x(s\alpha_2-(s-1)\alpha_1)}\,dx={}\\[3pt]
{}=\fr{\alpha_2^{s\beta_2}}{\alpha_1^{(s-1)\beta_1}}\,
\fr{1}{(s\alpha_2-(s-1)\alpha_1)^{s\beta_2-(s-1)\beta_1}}
\fr{\Gamma^{s-1}(\beta_1)\Gamma(s\beta_2-(s-1)\beta_1)}{\Gamma^s(\beta_2)}\,,\enskip
s=2,3,4\,.
\end{multline*}

\pagebreak


%\hrule

\begin{multicols}{2}
%\vspace*{-12pt}

{\small\frenchspacing
{%\baselineskip=10.8pt
\addcontentsline{toc}{section}{Литература}
\begin{thebibliography}{99}

\bibitem{Korolev2007}
\Au{Королев~В.\,Ю.} Ве\-ро\-ят\-ност\-но-ста\-ти\-сти\-че\-ский
анализ хаотических процессов с помощью смешанных гауссовских моделей.
Декомпозиция волатильности финансовых индексов и турбулентной
плазмы.~--- М.: ИПИ РАН, 2007. 363~c.

\bibitem{Korolev2010}\Au{Королев~В.\,Ю.} Ве\-ро\-ят\-ност\-но-ста\-ти\-сти\-че\-ские
методы декомпозиции волатильности хаотических процессов.~--- М.: МГУ,
2011. 510~c.

\bibitem{Akaike1973}\Au{Akaike~H.} Information theory and an extension
of the maximum likelihood principle~// 2nd Symposium (International) 
on Information Theory~/ Eds. B.\,N.~Petrov, F.~Csake.~--- Budapest, 1973. P.~267--281.

\bibitem{Schwartz1978}\Au{Schwartz~G.} Estimating the
dimension of a model~// The Annals of Statistics, 1978. Vol.~6.
P.~461--464.

\bibitem{Lo2001}\Au{Lo~Y., Mendell~N.\,R., Rubin~D.\,B.}
Testing the number of components in a normal mixture~// Biometrika,
2001. Vol.~88. No.\,3. P.~767--778.

\bibitem{Lo2005}\Au{Lo~Y.}  Likelihood ratio tests of the number of
components in a normal mixture with unequal variances~// Statistics
and Probability Lett., 2005. Vol.~71. P.~225--235.

\bibitem{Vuong1989}\Au{Vuong Q.\,H.} Likelihood ratio tests for model
selection and non-nested hypotheses~//~Econometrica, 1989. Vol.~57. Iss.~2. P.~307--333.

\bibitem{Bening2000}\Au{Bening~V.~E.} Asymptotic theory of testing
statistical hypothesis: Efficient statistics, optimality, power loss
and deficiency.~--- Untrecht: VSP, 2000. 277~p.

\bibitem{Hajek1962}\Au{H$\acute{\mbox{a}}$jek J.} Asymptotically most powerful
rank-order tests~// Ann. Math. Statist., 1962. Vol.~33. P.~1124--1147.

\bibitem{KF1976}\Au{Колмогоров~А.\,Н., Фомин~С.\,В.}
Элементы теории функций и функционального анализа.~--- 4-е изд.~---
М.:~Наука, 1976.

\bibitem{Feller2010}\Au{Феллер В.} Введение в теорию
вероятностей и ее приложения. Т.~2.~--- М.: Либроком, 2010. 766~с.

\bibitem{Teicher1963} \Au{Teicher~H.} Identifiability of finite mixtures~// The
Annals of Statistics, 1963. Vol.~34. No.\,4. P.~1265--1269.

\label{end\stat}

\bibitem{Yakowitz1968} \Au{Yakowitz~S.\,J., Spragins~J.\,D.} On the
identifiability of finite mixtures~// The Annals of Statistics,
1968. Vol.~39. No.\,1. P.~209--214.
 \end{thebibliography}
}
}


\end{multicols}       