

%\newcommand{\Z}{\bf Z}
%\newcommand{\R}{\bf R}
%\newcommand{\N}{\bf N}
%\newcommand{\cov}{\operatorname{cov}}
%\newcommand{\ctab}{\centering\arraybackslash}
%\makeatletter
%\def\D{\mathop{\kern\z@\mbox{\bfseries\sffamily\upshape D}}\nolimits}
%\def\M{\mathop{\kern\z@\mbox{\bfseries\sffamily\upshape E}}\nolimits}
%\def\P{\mathop{\kern\z@\mbox{\bfseries\sffamily\upshape P}}\nolimits}
%\makeatother
%\renewcommand{\le}{\leqslant}
%\renewcommand{\ge}{\geqslant}

\def\stat{leri}

\def\tit{ОБ ОДНОЙ СТАТИСТИЧЕСКОЙ ЗАДАЧЕ  ДЛЯ~СЛУЧАЙНЫХ ГРАФОВ ИНТЕРНЕТ-ТИПА}

\def\titkol{Об одной статистической задаче  для~случайных графов Интернет-типа}

\def\autkol{М.\,М.~Лери, И.\,А.~Чеплюкова}
\def\aut{М.\,М.~Лери$^1$, И.\,А.~Чеплюкова$^2$}

\titel{\tit}{\aut}{\autkol}{\titkol}

%{\renewcommand{\thefootnote}{\fnsymbol{footnote}}\footnotetext[1]
%{Работа выполнена при финансовой поддержке РФФИ (грант 11-01-00515).}}

\renewcommand{\thefootnote}{\arabic{footnote}}
\footnotetext[1]{Институт прикладных математических
исследований КарНЦ РАН, leri@krc.karelia.ru}
\footnotetext[2]{Институт прикладных математических
исследований КарНЦ РАН, chia@krc.karelia.ru}

\vspace*{4pt}

 
\Abst{Рассматриваются случайные графы
Ин\-тер\-нет-ти\-па, т.\,е.\ графы, степени вершин которых независимы и
имеют степенные распределения. С~помощью методов имитационного
моделирования проведено исследование возможности использования
критерия согласия~$\chi^2$ Пирсона для проверки гипотезы о том,
что степени вершин графа одинаково распределены. Построены модели
зависимости мощности критерия~$\chi^2$ от объема графа и
параметров распределений степеней вершин и даны рекомендации по
выбору числа интервалов группирования.}

\vspace*{2pt}

\KW{случайные графы; критерий согласия~$\chi^2$ Пирсона; имитационное моделирование}

\vspace*{2pt}


  \vskip 14pt plus 9pt minus 6pt

      \thispagestyle{headings}

      \begin{multicols}{2}
      
            \label{st\stat}



\section{Введение}

Быстрое развитие и широкое использование глобальных сетей
передачи данных, таких как сети телекоммуникаций, электрические и
телефонные сети, а также сети Интернет, за последнее десятилетие
привело к появлению множества работ (см., например,~[1--4]) 
как теоретического, так и прикладного
характера, направленных на описание структуры и функционирования
таких сетей. 

Авторами~\cite{Fa} было предложено описание такого
рода сетей с помощью случайных графов, степени вершин которых
представляют собой независимые случайные величины, общее
распределение которых является дискретным аналогом распределения
Парето. Такие графы иногда называют Ин\-тер\-нет-гра\-фа\-ми (см.,
например,~\cite{Pav}).

В работе рассматриваются случайные графы, состоящие из~$N$
основных вершин и одной вспомогательной, занумерованных числами
от~0 до~$N$. Степени основных вершин $1,2,\dots,N$ заданы
независимыми случайными величинами $\xi_1,\xi_2,\dots,\xi_N$,
распределения которых имеют следующий вид:
\begin{multline}
\label{eq1}
{\sf P}\{\xi_i\geqslant k\}= k^{-\tau_i}, \qquad i=1,\dots,N, \\
k=1,2,\dots, \quad \tau_i > 0\,.
\end{multline}

Для описания структуры графа будем использовать понятие полуребра~\cite{RN}: 
ребра, инцидентного некоторой вершине, для которой
смежная вершина не определена. Все полуребра графа различны. Для
образования ребер графа все полуребра соединяются между собой
равновероятно. Ясно, что построенный граф может иметь петли и
кратные ребра. Кроме того, необходимо, чтобы общее число
полуребер было четным. Поэтому степень вспомогательной вершины
(имеющей номер~0) задается равной~1, если суммарное число
полуребер основных вершин нечетно, и~0~--- в противном случае.

В~\cite{Fa} был проведен анализ реальных сетей, на основании
которого оказалось возможным считать $\tau_1,\dots,\tau_N$
одинаковыми и принадлежащими интервалу $(1,2)$. В~этом случае
распределение~(\ref{eq1}) имеет конечное математическое ожидание
и бесконечную дисперсию.

Наряду с теоретическими подходами одним из средств изучения
такого рода объектов является имитационное моделирование. В~\cite{RN, Tang} 
был предложен алгоритм генерации случайных графов,
степени вершин которых имеют распределение~(\ref{eq1}). На основе
этого алгоритма была создана~\cite{Ler} имитационная модель
случайного графа Ин\-тер\-нет-ти\-па и с помощью метода Мон\-те-Кар\-ло
изучалось поведение некоторых структурных характеристик таких
графов при $\tau_1=\tau_2=\dots=\tau_N=\tau$ для $\tau\in(1,2)$.

Согласно~\cite{Claus}, на практике при исследовании реальных сетей
редко можно точно утверждать, что имеющиеся данные взяты из
распределения Парето. Самое большее, что можно сказать, это то, что
выборка согласуется с распределением~(\ref{eq1}) с параметром~$\tau$, 
принадлежащим некоторому интервалу. В~частности, в статье~\cite{Claus} 
рассматривались модели случайных графов, распределение
степеней вершин которых имеет вид
\begin{equation}
\label{eq2}
\p\{\xi = k\} = \fr{k^{-\tau}}{\zeta(\tau,k_{\min})}\,, \qquad \tau
\in (2,3)\,,
\end{equation}
%\pagebreak

\noindent
где $k\geqslant k_{\min}$~--- натуральные числа, равные степеням
вершин графа, $k_{\min}\hm\geqslant 1$~--- некоторая заданная нижняя граница
значения степени вершины, а
$\zeta(\tau,k_{\min})\hm=\sum\limits_{j=0}^\infty{(j+k_{\min})^{-\tau}}$~---
обобщенная дзета-функция. Авторами~\cite{Claus} был использован
подход к исследованию согласования выборки с распределением~(\ref{eq2}) 
с помощью критерия согласия Кол\-мо\-го\-ро\-ва--Смир\-но\-ва, где
параметр~$\tau$ оценивался методом максимального правдоподобия.


Данная работа посвящена изучению воз\-мож\-ности использования
критерия~$\chi^2$ Пирсона для проверки гипотезы о согласии
распределения степеней вершин случайного Интернет-графа с
распределением~(\ref{eq1}).

\section{Выбор числа интервалов}

В качестве нулевой гипотезы~$H_0$ будем рас\-смат\-ри\-вать гипотезу о
том, что распределение степеней вершин случайного графа
Ин\-тер\-нет-ти\-па совпадает с распределением~(\ref{eq1}), где
$\tau_1\hm=\tau_2\hm=\dots\hm=\tau_N\hm=\tau$. Наряду с нулевой
рассматривается сложная альтернативная гипотеза~$H_1$ о том, что
выборка либо соответствует закону распределения, отличному от
распределения~(\ref{eq1}), либо не все $\tau_i$ равны~$\tau$.

Первая серия вычислительных экспериментов проводилась для
случайных графов, распределение степеней вершин которых имело вид~(\ref{eq1}), 
где $\tau_1\hm=\tau_2\hm=\dots=\tau_N\hm=\tau^*$. Параметр~$\tau^*$ принимал 
9~значений из интервала $(1,2)$ с шагом~0,1,
объем графа $N$ был равен следующим значениям: $10^3$,
$5\cdot10^3$, $10^4$, $2{,}5\cdot10^4$, $5\cdot10^4$,
$7{,}5\cdot10^4$, $10^5$. Для каждого из значений $N$ и~$\tau^*$
было сгенерировано по 100 случайных графов. Для каждого графа
по критерию~$\chi^2$ Пирсона проверялась нулевая гипотеза~$H_0$
при стандартном уровне значимости, равном~0,05. 

Способ раз\-би\-ения
на интервалы состоял в следующем. Первый интервал содержал
вершины степени~1, второй~--- вершины степени~2 и~т.\,д. Если
число вершин, попавших в интервал, оказывалось меньше~7, то
согласно рекомендациям,\linebreak
%\begin{table*}\small %tabl1
{\small \begin{center}
{{\tablename~1}\ \ \small{Число отвергнутых гипотез при $\tau = \tau^*$ }}
%\Caption{Число отвергнутых гипотез при $\tau = \tau^*$ }
\vspace*{2ex}

\tabcolsep=4.5pt
\begin{tabular}{|r|c|c|c|c|c|c|c|c|c|}
\hline
\multicolumn{1}{|c|}{\raisebox{-6pt}[0pt][0pt]{$N$}} &\multicolumn{9}{c|}{$\tau^*$}\\
\cline{2-10}
& 1,1 & 1,2 & 1,3 & 1,4 & 1,5 & 1,6 & 1,7 & 1,8 & 1,9 \\ 
\hline
1\,000 & \hphantom{9}7 & \hphantom{9}7 & \hphantom{9}6 & \hphantom{9}5& \hphantom{9}8 & 10 & 11 & \hphantom{9}4 & \hphantom{9}1 \\
5\,000 & 12 & 13 & \hphantom{9}9 & \hphantom{9}9 & \hphantom{9}3 & 11 & \hphantom{9}7 & \hphantom{9}5 & \hphantom{9}6 \\
10\,000 & \hphantom{9}6 & \hphantom{9}9 & 11 & \hphantom{9}7 & 10 &\hphantom{9}8 & \hphantom{9}5 & \hphantom{9}9 & \hphantom{9}7 \\
25\,000 & \hphantom{9}9 & \hphantom{9}9 & \hphantom{9}6 & \hphantom{9}6 & 13 & 13 & \hphantom{9}5 & \hphantom{9}4 & \hphantom{9}9 \\
50\,000 & 16 & 10 & \hphantom{9}7 & 10 & \hphantom{9}9 & \hphantom{9}4 & \hphantom{9}7 & \hphantom{9}3 & \hphantom{9}6 \\
75\,000 & 11 & 10 & 11 & 10 & \hphantom{9}8 & \hphantom{9}9 & 14 & \hphantom{9}4 & \hphantom{9}7 \\
100\,000 & 15 & 10 & \hphantom{9}9 & \hphantom{9}8 & 16 & \hphantom{9}9 & \hphantom{9}5 & \hphantom{9}6 & \hphantom{9}3 \\ 
\hline
\end{tabular}
\end{center}}
%\vspace*{6pt}
%\end{table*}
\columnbreak

%\begin{table*}\small %tabl2
\noindent
{{\tablename~2}\ \ \small{Число отвергнутых гипотез при $\tau$, оцени\-ва\-емом
по выборке}}
%\Caption{Число отвергнутых гипотез при $\tau$, оцениваемом
%по выборке}
\vspace*{1.5ex}

{\small 
\tabcolsep=4.5pt
\begin{center}
\begin{tabular}{|r|c|c|c|c|c|c|c|c|c|}
\hline
\multicolumn{1}{|c|}{\raisebox{-6pt}[0pt][0pt]{$N$}} &\multicolumn{9}{c|}{$\tau^*$}\\
\cline{2-10}
& 1,1 & 1,2 & 1,3 & 1,4 & 1,5 & 1,6 & 1,7 & 1,8 & 1,9 \\ 
\hline
1\,000 & \hphantom{9}5 & \hphantom{9}9 & \hphantom{9}7 & \hphantom{9}4 & \hphantom{9}7 & \hphantom{9}9 & 11 & \hphantom{9}4 & \hphantom{9}4 \\
5\,000 & 13 & 13 & \hphantom{9}8 & \hphantom{9}9 & \hphantom{9}4 & 11 & \hphantom{9}6 & \hphantom{9}3 & \hphantom{9}5 \\
10\,000 &\hphantom{9}4 & \hphantom{9}8 & \hphantom{9}9 & \hphantom{9}7 & 10 & \hphantom{9}7 & \hphantom{9}7 & \hphantom{9}9 & \hphantom{9}5 \\
25\,000 &\hphantom{9}9 & 11 & \hphantom{9}6 & \hphantom{9}7 & 12 & 15 & \hphantom{9}6 & \hphantom{9}4 & \hphantom{9}7 \\
50\,000 &16 &10 & \hphantom{9}9 & 11 & \hphantom{9}9 & \hphantom{9}3 & \hphantom{9}7 & \hphantom{9}3 & \hphantom{9}5 \\
75\,000 &\hphantom{9}8 & 11 & 11 & \hphantom{9}9 & 10 & 10 & 12 & \hphantom{9}5 & \hphantom{9}7 \\
100\,000 &15 & 10 & 10 & 10 & 17 & \hphantom{9}7 & \hphantom{9}6 & \hphantom{9}8 & \hphantom{9}4 \\ 
\hline
\end{tabular}
\end{center}}
%\end{table*}

\vspace*{18pt}

\addtocounter{table}{2}

\noindent
сформулированным в~\cite{Aivaz}, он
объединялся с последующим интервалом. 

При проверке гипотезы $H_0$
было рассмотрено два варианта: первый~--- значение параметра~$\tau$ 
задавалось равным~$\tau^*$ и второй~--- $\tau$ оценивалось
по выборке методом максимального правдоподобия. В~табл.~1 и~2
приведены значения числа отвергнутых нулевых гипотез для этих
двух рассмотренных вариантов соответственно.


Из приведенных таблиц видно, что число отвергнутых правильных
гипотез  в среднем равно~8, что не соответствует заданному уровню
зна\-чи\-мости критерия. Известно (см., например,~\cite{Aivaz,Lem}), что
статистические свойства критерия согласия~$\chi^2$ час\-то зависят от
выбора числа интервалов группирования при вычислении статистики~$\chi^2$, 
а также от 
 способа разбиения на эти интервалы. 
 
 Описанные выше результаты можно объяснить тем, что использованное при
вычислении статистики критерия разбиение на интервалы не учитывало
особенности структуры рассматриваемых объектов. Оказалось, что
специфика исследуемых объектов такова, что от 50\% до~70\% (в
зависимости от параметра~$\tau$) всех вершин графа занимают вершины
степени~1, вершин степени 2~--- от~14\% до~16\%, вершин степени~3~--- 
от~5\% до~8\%, вершин степени~4~--- от~2\% до~4\%, вершин степени~5~--- от 1\%
до~3\% и~т.\,д., что соответствует распределению~(\ref{eq1}). На рис.~1 
приведены доли вершин, имеющих степени от~1 до~5, для девяти
рассматриваемых значений параметра~$\tau$.

\begin{figure*} %fig1
\vspace*{1pt}
\begin{center}
\mbox{%
\epsfxsize=105.049mm
\epsfbox{ler-1.eps}
}
\end{center}
\vspace*{-11pt}
\Caption{Доля вершин, имеющих степени от 1 до~5}
\end{figure*}

Следовательно, при использовании критерия~$\chi^2$ достаточно
использовать небольшое число интервалов группирования. 

В~табл.~3 и~4 
даны результаты экспериментов с разбиением на 3~интервала
группирования.
%
В~этом случае число отвергнутых правильных гипотез оказалось в
среднем равным~5, что согласуется с заданным уровнем значимости
критерия. Таким образом, в дальнейшем при вычислении статистики
критерия~$\chi^2$ область значений случайной величины, равной
степени вершины графа, разбивалась на три интервала.

%\pagebreak

%\begin{table*}\small %tabl3
\noindent
{{\tablename~3}\ \ \small{Число отвергнутых гипотез при $\tau = \tau^*$
(разбиение на 3 интервала группирования)}}
%\Caption{Число отвергнутых гипотез при $\tau = \tau^*$
%(разбиение на 3 интервала группирования)
%}
\vspace*{6pt}

\begin{center}
{\small 
\tabcolsep=4.4pt
\begin{tabular}{|r|c|c|c|c|c|c|c|c|c|}
\hline
\multicolumn{1}{|c|}{\raisebox{-6pt}[0pt][0pt]{$N$}} &\multicolumn{9}{c|}{$\tau^*$}\\
\cline{2-10}
& 1,1 & 1,2 & 1,3 & 1,4 & 1,5 & 1,6 & 1,7 & 1,8 & 1,9 \\ 
\hline
1\,000 & $7$ & $4$ & $5$ & $5$ & $8$ & $9$ & $8$ & $6$ & $6$ \\
5\,000 & $7$ & $5$ & $6$ & $7$ & $3$ & $2$ & $6$ & $7$ & $5$ \\
10\,000 & $5$ & $5$ & $5$ & $6$ & $5$ & $2$ & $5$ & $7$ & $3$ \\
25\,000 & $6$ & $5$ & $2$ & $4$ & $2$ & $6$ & $3$ & $2$ & $3$ \\
50\,000 & $6$ & $4$ & $10$\hphantom{9} & $7$ & $6$ & $5$ & $4$ & $4$ & $5$ \\
75\,000 & $5$ & $8$ & $7$ & $7$ & $5$ & $3$ & $9$ & $2$ & $4$ \\
100\,000 & $6$ & $10$\hphantom{9} & $6$ & $3$ & $7$ & $1$ & $2$ & $2$ & $8$ \\ \hline
\end{tabular}
}
\end{center}

\vspace*{6pt}
%\end{table*}

%\begin{table*}\small %tabl4
\noindent
{{\tablename~4}\ \ \small{Число отвергнутых гипотез при $\tau$, оцени\-ва\-емом
по выборке (разбиение на 3 интервала группирования)}}

\vspace*{6pt}

\begin{center}
{\small 
\tabcolsep=4.4pt
\begin{tabular}{|r|c|c|c|c|c|c|c|c|c|}
\hline
\multicolumn{1}{|c|}{\raisebox{-6pt}[0pt][0pt]{$N$}} &\multicolumn{9}{c|}{$\tau^*$}\\
\cline{2-10}
& 1,1 & 1,2 & 1,3 & 1,4 & 1,5 & 1,6 & 1,7 & 1,8 & 1,9 \\ 
\hline
1\,000 & $11$\hphantom{9} & $5$ & $2$ & $2$ & $9$ & $8$ & $8$ & $6$ & $4$ \\
5\,000 & $9$ & $3$ & $5$ & $5$ & $4$ & $2$ & $5$ & $4$ & $3$ \\
10\,000 & $2$ & $5$ & $3$ & $4$ & $4$ & $3$ & $5$ & $4$ & $5$ \\
25\,000 & $7$ & $8$ & $1$ & $6$ & $4$ & $9$ & $4$ & $0$ & $1$ \\
50\,000 & $7$ & $2$ & $8$ & $4$ & $6$ & $7$ & $2$ & $5$ & $4$ \\
75\,000 & $6$ & $7$ & $7$ & $7$ & $4$ & $5$ & $9$ & $5$ & $6$ \\
100\,000 & $6$ & $12$\hphantom{9} & $5$ & $0$ & $7$ & $2$ & $5$ & $4$ & $8$\\ 
\hline
\end{tabular}}
\end{center}
%\end{table*}



\addtocounter{table}{2}
%\begin{table*}\small %tabl5
\noindent
{{\tablename~5}\ \ \small{Число отвергнутых нулевых гипотез при $\tau^*=1{,}1$}}

\vspace*{6pt}

\begin{center}
{\small 
\tabcolsep=4.8pt
\begin{tabular}{|r|c|c|c|c|c|c|c|c|}
\hline
\multicolumn{1}{|c|}{\raisebox{-6pt}[0pt][0pt]{$N$}} &\multicolumn{8}{c|}{$\tau^*$}\\
\cline{2-9}
& 1,2 & 1,3 & 1,4 & 1,5 & 1,6 & 1,7 & 1,8 & 1,9 \\ 
\hline
1\,000 & \hphantom{9}$48$ & \hphantom{9}$97$ & $100$ & $100$ & $100$ & $100$ & $100$ & $100$ \\
5\,000 & \hphantom{9}$97$ & $100$ & $100$ & $100$ & $100$ & $100$ & $100$ & $100$ \\
10\,000 & $100$ & $100$ & $100$ & $100$ & $100$ & $100$ & $100$ & $100$ \\
25\,000 & $100$ & $100$ & $100$ & $100$ & $100$ & $100$ & $100$ & $100$ \\
50\,000 & $100$ & $100$ & $100$ & $100$ & $100$ & $100$ & $100$ & $100$ \\
75\,000 & $100$ & $100$ & $100$ & $100$ & $100$ & $100$ & $100$ & $100$ \\
100\,000 & $100$ & $100$ & $100$ & $100$ & $100$ & $100$ & $100$ & $100$ \\ \hline
\end{tabular}}
\end{center}
%\end{table*}

\vspace*{6pt}

%\begin{table*}\small %tabl6
\noindent
{{\tablename~6}\ \ \small{Число отвергнутых нулевых гипотез при $\tau^*=1{,}5$\newline}}

\vspace*{6pt}

\begin{center}
{\small
\tabcolsep=4.8pt
\begin{tabular}{|r|c|c|c|c|c|c|c|c|}
\hline
\multicolumn{1}{|c|}{\raisebox{-6pt}[0pt][0pt]{$N$}} &\multicolumn{8}{c|}{$\tau^*$}\\
\cline{2-9}
& 1,1 & 1,2 & 1,3 & 1,4 & 1,6 & 1,7 & 1,8 & 1,9 \\ 
\hline
1\,000 & $100$ & $100$ & \hphantom{9}$80$ & \hphantom{9}$22$ & \hphantom{9}$27$ & \hphantom{9}$75$ & $100$ & $100$ \\
5\,000 & $100$ & $100$ & $100$ & \hphantom{9}$69$ & \hphantom{9}$89$ & $100$ & $100$ & $100$ \\
10\,000 & $100$ & $100$ & $100$ & \hphantom{9}$94$ & $100$ & $100$ & $100$ & $100$ \\
25\,000 & $100$ & $100$ & $100$ & $100$ & $100$ & $100$ & $100$ & $100$ \\
50\,000 & $100$ & $100$ & $100$ & $100$ & $100$ & $100$ & $100$ & $100$ \\
75\,000 & $100$ & $100$ & $100$ & $100$ & $100$ & $100$ & $100$ & $100$ \\
100\,000 & $100$ & $100$ & $100$ & $100$ & $100$ & $100$ & $100$ & $100$ \\ \hline
\end{tabular}
}
\end{center}
%\end{table*}


%\vspace*{6pt}

\addtocounter{table}{3}



\end{multicols}

\vspace*{1pt}


\begin{multicols}{2}

%\vspace*{3pt}

\section{Исследование мощности критерия}

Далее было проведено исследование мощ\-ности рассматриваемого
критерия. В~соответствии с описанными выше планами экспериментов
были\linebreak
построены случайные графы заданного объема, распределение
степеней вершин которых соответствовало распределению~(\ref{eq1})
с заданным па\-ра\-мет-\linebreak
\noindent
ром~$\tau^*$. Для каждого графа проверялись
нулевые гипотезы с параметрами~$\tau$, отличными от~$\tau^*$.
Доля экспериментов, в которых нулевая гипотеза отвергалась,
рассматривалась в качестве оценки значения функции мощности
критерия согласия~$\chi^2$ при заданных характеристиках графа~$N$
и~$\tau^*$.

Таблицы~5--7 содержат примеры результатов проверки с помощью
критерия~$\chi^2$ нулевой гипотезы при различных фиксированных~$\tau$, 
отличных от $\tau^*$, для графов всех заданных объемов с
пара\-мет\-рами $\tau^*\hm= 1{,}1$, 1,5 и 1,9.



%\begin{table*}\small %tabl7
\noindent
{{\tablename~7}\ \ \small{Число отвергнутых нулевых гипотез при $\tau^*=1{,}9$}}

\vspace*{6pt}

\begin{center}
{\small 
\tabcolsep=4.8pt
\begin{tabular}{|r|c|c|c|c|c|c|c|c|}
\hline
\multicolumn{1}{|c|}{\raisebox{-6pt}[0pt][0pt]{$N$}} &\multicolumn{8}{c|}{$\tau^*$}\\
\cline{2-9}
& 1,1 & ,1,2 & 1,3 & 1,4 & 1,5 & 1,6 & 1,7 & 1,8 \\ 
\hline
1\,000 & $100$ & $100$ & $100$ & $100$ & $100$ & \hphantom{9}$98$ & \hphantom{9}$56$ & \hphantom{9}$15$ \\
5\,000 & $100$ & $100$ & $100$ & $100$ & $100$ & $100$ & $100$ & \hphantom{9}$44$ \\
10\,000 & $100$ & $100$ & $100$ & $100$ & $100$ & $100$ & $100$ & \hphantom{9}$78$ \\
25\,000 & $100$ & $100$ & $100$ & $100$ & $100$ & $100$ & $100$ & $100$ \\
50\,000 & $100$ & $100$ & $100$ & $100$ & $100$ & $100$ & $100$ & $100$ \\
75\,000 & $100$ & $100$ & $100$ & $100$ & $100$ & $100$ & $100$ & $100$ \\
100\,000 & $100$ & $100$ & $100$ & $100$ & $100$ & $100$ & $100$ & $100$ \\ \hline
\end{tabular}
}
\end{center}
%\end{table*}

\vspace*{12pt}


Через $\mu$ обозначим оценку мощности критерия $\chi^2$, а через
$\delta$~--- величину, равную $|\tau-\tau^*|$. Из
 табл.~5--7 можно
заметить, что при фиксированном~$\delta$ начиная с определенного
объема графа для всех рассматриваемых $\tau^*$ оценка мощности
$\mu$ приближается к~1. Так, при $\delta=0{,}1$ начиная с $N=10^4$
оценка мощности близка к~1, а при $\delta=0{,}2$ аналогичный
результат достигается уже при $N=5\cdot10^3$. Для более
детального изучения этого вопроса были получены оценки мощности
критерия~$\chi^2$ для значений~$\tau$ из интервала
$(\tau^*-0{,}1,\,\tau^*+0{,}1)$ с шагом 0,01. Примеры полученных
результатов при $\tau^*\hm= 1{,}1$, 1,5 и 1,9 приведены в табл.~8--10.


По полученным данным с помощью методов регрессионного
анализа была построена сле\-ду\-ющая модель, оценивающая  мощность
критерия в зависимости от~$N$ и~$\delta$:
\begin{equation*}
\mu =
\begin{cases}
0{,}12\ln N + 0{,}19\ln\delta + 0{,}05 \\
\hspace{5mm}\mbox{при}\ 0{,}01 \leqslant \delta \leqslant \min\{0{,}8;148{,}4 N^{-0{,}632}\};\\
1 \ \hspace*{2.5mm}\mbox{при}\ \delta > \min\{0{,}8;148{,}4 N^{-0{,}632}\}.
\end{cases}
\end{equation*}
Значимость модели и ее коэффициентов соответствует
уровню 0,05, а оценка коэффициента множественной корреляции
равна 0,83. На рис.~2 отражены зависимости мощности критерия~$\chi^2$ 
от~$\delta$ для пяти объемов Интернет-графов.


%\pagebreak
\addtocounter{figure}{1}
\begin{figure*}[b] %fig3
\vspace*{6pt}
\begin{center}
\mbox{%
\epsfxsize=163.741mm
\epsfbox{ler-3.eps}
}
\end{center}
\vspace*{-6pt}
\Caption{Графики зависимости мощности критерия
от числа интервалов при $(p_1,p_2)=(50\%,50\%)$:
(\textit{а})~$N=5\,000$; (\textit{б})~$N=100\,000$
}
\end{figure*}
\end{multicols}

\begin{table}[b]\small %tabl8
\vspace*{-24pt}
\begin{center}
\Caption{Число отвергнутых нулевых гипотез при $\tau^*=1{,}1$}
\vspace*{2ex}

\tabcolsep=3.7pt
\begin{tabular}{|r|r|r|r|r|r|r|r|r|r|r|r|r|r|r|r|r|r|r|}
\hline
\multicolumn{1}{|c|}{\raisebox{-6pt}[0pt][0pt]{$N$}}& \multicolumn{18}{c|}{$\tau\neq\tau^*$} \\
\cline{2-19}
& 1,01 & 1,02 & 1,03 & 1,04 & ,1,05 & 1,06 & 1,07 & 1,08 & 1,09
& 1,11 & 1,12 & 1,13 & 1,14 & 1,15 & 1,16 & 1,17 & 1,18 & 1,19 \\ 
\hline
1\,000 & $49$ & $42$ & $35$ & $27$ & $22$ & $15$ & $10$ & $9$ & $9$
& $8$ & $8$ & $12$ & $17$ & $23$ & $29$ & $35$ & $44$ & $52$  \\
5\,000 & $100$ & $97$ & $92$ & $78$ & $62$ & $52$ & $36$ & $21$ & $11$
& $11$ & $21$ & $38$ & $51$ & $62$ & $71$ & $88$ & $94$ & $100$ \\
10\,000 & $100$ & $100$ & $100$ & $98$ & $95$ & $78$ & $52$ & $18$ & $9$
& $9$ & $23$ & $47$ & $76$ & $91$ & $98$ & $100$ & $100$ & $100$ \\
25\,000 & $100$ & $100$ & $100$ & $100$ & $100$ & $100$ & $88$ & $47$ & $16$
& $20$ & $61$ & $90$ & $99$ & $100$ & $100$ & $100$ & $100$ & $100$ \\
50\,000 & $100$ & $100$ & $100$ & $100$ & $100$ & $100$ & $100$ & $87$ & $35$
& $30$ & $83$ & $100$ & $100$ & $100$ & $100$ & $100$ & $100$ & $100$ \\
75\,000 & $100$ & $100$ & $100$ & $100$ & $100$ & $100$ & $100$ & $97$ & $41$
& $51$ & $98$ & $100$ & $100$ & $100$ & $100$ & $100$ & $100$ & $100$ \\
100\,000 & $100$ & $100$ & $100$ & $100$ & $100$ & $100$ & $100$ & $98$ & $63$
& $51$ & $100$ & $100$ & $100$ & $100$ & $100$ & $100$ & $100$ & $100$ \\ \hline
\end{tabular}
\end{center}
%\end{table}
%\begin{table}\small %tabl9
\begin{center}
\Caption{Число отвергнутых нулевых гипотез при $\tau^*=1{,}5$}
\vspace*{2ex}

\tabcolsep=3.8pt
\begin{tabular}{|r|r|r|r|r|r|r|r|r|r|r|r|r|r|r|r|r|r|r|}
\hline
\multicolumn{1}{|c|}{\raisebox{-6pt}[0pt][0pt]{$N$}}& \multicolumn{18}{c|}{$\tau\neq\tau^*$} \\
\cline{2-19}
 & 1,41 & 1,42 & 1,43 & 1,44 & 1,45 & 1,46 & 1,47 & 1,48 & 1,49
& 1,51 & 1,52 & 1,53 & 1,54 & 1,55 & 1,56 & 1,57 & 1,58 & 1,59 \\ 
\hline
1\,000 & $38$ & $35$ & $26$ & $19$ & $15$ & $10$ & $6$ & $6$ & $7$
& $8$ & $8$ & $6$ & $6$ & $8$ & $9$ & $16$ & $22$ & $27$  \\
5\,000 & $93$ & $83$ & $75$ & $65$ & $51$ & $34$ & $20$ & $9$ & $3$
& $6$ & $10$ & $19$ & $25$ & $38$ & $55$ & $65$ & $80$ & $92$ \\
10\,000 & $100$ & $99$ & $94$ & $81$ & $67$ & $51$ & $27$ & $13$ & $4$
& $14$ & $24$ & $40$ & $56$ & $76$ & $89$ & $96$ & $99$ & $100$ \\
25\,000 & $100$ & $100$ & $100$ & $100$ & $100$ & $96$ & $67$ & $33$ & $8$
& $9$ & $36$ & $73$ & $95$ & $100$ & $100$ & $100$ & $100$ & $100$ \\
50\,000 & $100$ & $100$ & $100$ & $100$ & $100$ & $98$ & $97$ & $64$ & $18$
& $16$ & $64$ & $94$ & $100$ & $100$ & $100$ & $100$ & $100$ & $100$ \\
75\,000 & $100$ & $100$ & $100$ & $100$ & $100$ & $100$ & $99$ & $83$ & $30$
& $26$ & $79$ & $100$ & $100$ & $100$ & $100$ & $100$ & $100$ & $100$ \\
100\,000 & $100$ & $100$ & $100$ & $100$ & $100$ & $100$ & $100$ & $90$ & $35$
& $40$ & $96$ & $100$ & $100$ & $100$ & $100$ & $100$ & $100$ & $100$ \\ \hline
\end{tabular}
\end{center}
%\end{table}
%\begin{table}\small
\begin{center}
\Caption{Число отвергнутых нулевых гипотез при $\tau^*=1{,}9$}
\vspace*{2ex}

\tabcolsep=3.8pt
\begin{tabular}{|r|r|r|r|r|r|r|r|r|r|r|r|r|r|r|r|r|r|r|}
\hline
\multicolumn{1}{|c|}{\raisebox{-6pt}[0pt][0pt]{$N$}}& \multicolumn{18}{c|}{$\tau\neq\tau^*$} \\
\cline{2-19}
 & 1,81 & 1,82 & 1,83 & 1,84 & 1,85 & 1,86 & 1,87 & 1,88 & 1,89
& 1,91 & 1,92 & 1,93 & 1,94 & 1,95 & 1,96 & 1,97 & 1,98 & 1,99 \\ 
\hline
1\,000 & $21$ & $18$ & $13$ & $12$ & $10$ & $7$ & $5$ & $5$ & $5$
& $5$ & $5$ & $5$ & $7$ & $6$ & $9$ & $11$ & $12$ & $12$  \\
5\,000 & $73$ & $62$ & $49$ & $40$ & $26$ & $21$ & $8$ & $5$ & $5$
& $4$ & $5$ & $8$ & $16$ & $32$ & $43$ & $51$ & $68$ & $82$ \\
10\,000 & $95$ & $89$ & $80$ & $62$ & $46$ & $34$ & $16$ & $10$ & $4$
& $10$ & $22$ & $27$ & $45$ & $60$ & $76$ & $89$ & $95$ & $97$ \\
25\,000 & $100$ & $100$ & $100$ & $98$ & $89$ & $69$ & $44$ & $16$ & $2$
& $9$ & $26$ & $51$ & $81$ & $94$ & $99$ & $100$ & $100$ & $100$ \\
50\,000 & $100$ & $100$ & $100$ & $100$ & $100$ & $98$ & $81$ & $37$ & $11$
& $13$ & $43$ & $82$ & $95$ & $100$ & $100$ & $100$ & $100$ & $100$ \\
75\,000 & $100$ & $100$ & $100$ & $100$ & $100$ & $100$ & $95$ & $69$ & $18$
& $18$ & $64$ & $95$ & $100$ & $100$ & $100$ & $100$ & $100$ & $100$ \\
100\,000 & $100$ & $100$ & $100$ & $100$ & $100$ & $100$ & $98$ & $78$ & $27$
& $21$ & $75$ & $98$ & $100$ & $100$ & $100$ & $100$ & $100$ & $100$ \\ \hline
\end{tabular}
\end{center}
\end{table}


\begin{multicols}{2}


\begin{center} %fig2
\vspace*{-1pt}
\mbox{%
\epsfxsize=79.389mm
\epsfbox{ler-2.eps}
}
%\end{center}
\vspace*{6pt}
\end{center}
{{\figurename~2}\ \ \small{График зависимости мощности критерия
от $\delta$ при разных объемах графа $N$:
\textit{1}~--- 1\,000; \textit{2}~--- 5\,000; \textit{3}~--- 10\,000;
\textit{4}~--- 50\,000; \textit{5}~--- 100\,000}}
%\end{center}
\vspace*{18pt}

%\smallskip
%\addtocounter{figure}{1}



Таким образом, ясно, что мощность критерия возрастает
как с ростом объема графа, так и с увеличением значения $\delta$.
Это означает, что при использовании критерия $\chi^2$ для проверки
нулевой гипотезы при объемах графа $N\geqslant 10^4$ и при отклонении
значения $\tau$ от~$\tau^*$ более чем на 0,09 мощность критерия
будет не менее~0,7.

Во второй серии вычислительных экспериментов рассматривались
случайные графы, состоящие из двух типов вершин, распределения
которых имеют вид~(\ref{eq1}) с параметрами $\tau_1^*$ и
$\tau_2^*$ соответственно. Здесь и далее будем полагать, что
$\tau_1^*<\tau_2^*$. Через $p_1$ и $p_2$ обозначим доли вершин,
распределения которых имеют параметры $\tau_1^*$ и $\tau_2^*$
соответственно. Также обозначим $\delta^*=|\tau_1^*-\tau_2^*|$.

План эксперимента состоял в следующем. Пары $(p_1,p_2)$ принимали
значения $(50\%,50\%)$, $(20\%,80\%)$, $(80\%,20\%)$,
$(2\%,98\%)$, $(98\%,2\%)$. Для каждого из восьми значений
$\delta^*\in [0{,}1,\,0{,}8]$, взятых с шагом 0,1, параметры
распределения степеней вершин $\tau_1^*$ и $\tau_2^*$ принимали
по 3~пары значений из интервала $(1,2)$ (например, для
$\delta^*=0{,}1$ пары $(\tau_1^*,\tau_2^*)$ принимали значения
$(1{,}1, 1{,}2)$, $(1{,}4, 1{,}5)$, $(1{,}8, 1{,}9)$). Рассматривались графы
следующих объемов $N$: $5\cdot10^3$, $10^4$, $5\cdot10^4$,
$10^5$. С~целью получения статистических данных для каждого
значения~$N$ и каждой пары $(\tau_1^*,\tau_2^*)$ было
сгенерировано по 100 случайных графов.



Учитывая свойства критерия $\chi^2$, рассмотрим зависимость
функции мощности критерия от числа интервалов группирования. Для
каждого графа проверялась нулевая гипотеза для разного числа
интервалов~$m$ (от~3 до~25) и в каждом случае подсчитывалась
доля экспериментов, в которых нулевая гипотеза отвергалась.

На рис.~3 приведены примеры графиков экспериментальных значений
оценок мощности~$\mu$ в зависимости от числа интервалов
группирования~$m$ и всех рассмотренных значений $\delta^*$ для
пары $(p_1,p_2)=(50\%,50\%)$ при $N= 5\,000$ и $N\hm =
100\,000$.



На рис.~4 приведены аналогичные результаты для пары
$(p_1,p_2)=(80\%,20\%)$.

\begin{figure*} %fig4
\vspace*{1pt}
\begin{center}
\mbox{%
\epsfxsize=164.399mm
\epsfbox{ler-5.eps}
}
\end{center}
\vspace*{-6pt}
\Caption{График зависимости мощности критерия
от числа интервалов при $(p_1,p_2)=(80\%,20\%)$: (\textit{а})~$N=5\,000$;
(\textit{б})~$N=100\,000$
}
\vspace*{9pt}
\end{figure*}


Результаты экспериментов показали, что с рос\-том числа интервалов
значения оценок мощности критерия не превосходят оценки мощности
при $m=3$. Поэтому можно сделать вывод, что и в этом случае при
проверке гипотезы согласия распределения степеней вершин
случайных графов Ин\-тер\-нет-ти\-па со степенным законом распределения
по критерию~$\chi^2$ достаточно использовать три интервала
группирования.

Были построены регрессионные модели зависимости функции мощности~$\mu$ критерия~$\chi^2$ 
при проверке нулевой гипотезы от объема
графа~$N$ и значения~$\delta^*$ для всех рассмотренных пар
$(p_1,p_2)$:
\begin{itemize}
\item 
$(p_1,p_2)=(50\%,50\%)$:
\begin{multline*}
\ln\mu ={}\\
{}=
\begin{cases}
0{,}33\ln N + 1{,}3\ln\delta^* -3{,}24 \\
\hspace*{2.5mm}\mbox{при}\ 0{,}1 < \delta^* < \min\{0{,}8;12{,}06 N^{-0{,}25}\};\hspace*{-5.27403pt}\\
0 \ \mbox{при}\ \delta^* \geqslant \min\{0{,}8;12{,}06 N^{-0{,}25}\}; %\hspace*{-5.27403pt}
\end{cases}
\end{multline*}
\item
$(p_1,p_2)=(20\%,80\%)$:
\begin{multline*}
\ln\mu ={}\\
\!\!\!\!\!\!{}=\begin{cases}
0{,}37\ln N + 3{,}84\delta^* - 6{,}74 \\
\hspace*{2.5mm}\mbox{при}\ 0{,}1 < \delta^* < \min\{0{,}8;1{,}76-0{,}1\ln N\};\hspace*{-5.85742pt}\\
0 \ \mbox{при}\ \delta^* \geqslant \min\{0{,}8;1{,}76-0{,}1\ln N\};
\end{cases}
\end{multline*}
\item
\noindent
$(p_1,p_2)=(80\%,20\%)$:
\begin{multline*}
\ln\mu ={}\\
\!\!\!\!\!\!\!{}=\begin{cases}\!
0{,}42\ln N + 2{,}83\delta^* - 7{,}13 \\
\hspace*{2.5mm}\mbox{при}\, 0{,}1 < \delta^* < \min\{0{,}8;2{,}52-0{,}15\ln N\};\hspace*{-8.52745pt}\\
0 \ \mbox{при}\, \delta^* \geqslant \min\{0{,}8;2{,}52-0{,}15\ln N\};
\end{cases}
\end{multline*}
\item
$(p_1,p_2)=(2\%,98\%)$:
\begin{multline*}
\sqrt\mu = 0{,}04\ln N - 0{,}44\delta^* + 0{,}72(\delta^*)^2 - 0{,}1 \\ \mbox{при}\ 0{,}1 < \delta^* < 0{,}8;
\end{multline*}
\item
$(p_1,p_2)=(98\%,2\%)$:
\begin{multline*}
\sqrt\mu = 0{,}02\ln N + 0{,}1(\delta^*)^2 + 0{,}1 \\
\mbox{при}\ 0{,}1 < \delta^* < 0{,}8.
\end{multline*}
\end{itemize}

\noindent
Оценки коэффициентов множественной корреляции полученных
моделей равны соответственно 0,91, 0,91, 0,9, 0,71 и
0,5, и соответствуют уровню значимости~0,05.

\section{Заключение}

Анализируя полученные данные, несложно заметить, что хотя мощность
критерия в целом увеличивается как с ростом объема графа, так и с
увеличением~$\delta^*$, однако значение оценки мощ\-ности при
небольших~$N$ мало. 

Следует отметить, что в случае, когда
величины $p_1$ и~$p_2$ принимали значения 50\% и~50\%, оценка
мощности критерия с ростом объема графа возрастала быстрее, чем в
случае $(p_1,p_2)=(20\%,80\%)$. 

При $(p_1,p_2)\hm=(80\%,20\%)$ 
увеличение эмпирических значений функции мощности
происходит медленнее, чем в случае $(20\%,80\%)$. Очевидно, что
данный факт связан с долей содержания в графе вершин
второго типа: эксперименты показали, что чем больше граф содержит
вершин с большим значением параметра, тем мощность критерия выше.
Это также подтверждается и для пар $(p_1,p_2)=(2\%,98\%)$ и
$(p_1,p_2)=(98\%,2\%)$, хотя в этих случаях оценка мощности
рассматри\-ва\-емого критерия не превышает соответственно 0,16 и~0,23. 
{\looseness=1

}

Таким образом, чем меньше разница между долями вершин
разных типов, тем оценка мощности критерия~$\chi^2$ выше. Кроме
того, мощность критерия увеличивается как с ростом разности между
истинным (неизвестным) значением параметра распределения~(\ref{eq1}) 
и проверяемым значением~$\tau$, так и с увеличением
объема графа. 

Заметим, что если нулевая гипотеза не отвергается,
то это не исключает того, что граф может содержать вершины разных
типов, но доля вершин одного из них не\-зна\-чи\-тельна.

На основании проведенных исследований можно сделать следующие
выводы. При проверке гипотезы о согласии распределения степеней
вершин случайного Ин\-тер\-нет-гра\-фа с распределением~(\ref{eq1})\linebreak\vspace*{-12pt}

\pagebreak

\noindent
 по
критерию~$\chi^2$ достаточно разбивать область значений
рассматриваемой случайной величины на три интервала группирования.
Для обеспечения достаточно высокой мощности критерия желательно
также, чтобы рассматриваемый граф содержал не менее 10000~вершин.
При проверке нулевой гипотезы на графах меньшего объема необходимо
учитывать полученную выше зависимость функции мощности от объема
графа.

{\small\frenchspacing
{%\baselineskip=10.8pt
\addcontentsline{toc}{section}{Литература}
\begin{thebibliography}{99}

\bibitem{Aie} %1
\Au{Aiello W., Chung F., Lu~L.} A random graph
model for massive graphs~// 32nd Annual ACM Symposium
on Theory of Computing Proceedings.~--- New York: ACM, 2000. P.~171--180.


\bibitem{New} %2
\Au{Newman M.\,E.\,Y., Strogatz S.\,H., Watts~D.\,J.}
Random graphs with arbitrary degree distribution and their
applications~// Phys. Rev. E, 2001. Vol.~64. P.~026118-1--026118-17.

\bibitem{RN} %3
\Au{Reittu H., Norros I.}  
On the power-law random
graph model of massive data networks~// Performance Evaluation,
2004. Vol.~55. P.~3--23.

\bibitem{Dur} %4
\Au{Durrett R.} Random graph dynamics.~--- Cambridge: Cambridge Univ.
Press, 2007.

\bibitem{Fa} %5
\Au{Faloutsos C., Faloutsos P., Faloutsos~M.}
On power-law relationships of the Internet topology~// Computer
Communications Rev., 1999. Vol.~29. P.~251--262.

\bibitem{Pav} %6
\Au{Павлов Ю.\,Л.} Предельное распределение объема
гигантской компоненты в случайном графе Интернет-ти\-па~// Дискретная
математика, 2007. Т.~19. Вып.~3. С.~22--34.

\bibitem{Tang}
\Au{Tangmunarunkit H., Govindan R., Jamin~S.,
Shenker~S., Willinger~W.} 
Network topology generators: Degree-based
vs. structural~// SIGCOMM'02 Proceedings.~--- Pittsburgh, USA,
2002. P.~147--159.

\bibitem{Ler}
\Au{Лери М.\,М.} Моделирование случайных
графов Ин\-тер\-нет-ти\-па~// Обозрение прикладной и промышленной
математики, 2009. Т.~16. Вып.~5. С.~737--744.

\bibitem{Claus} 
\Au{Clauset A., Shalizi C. R., Newman~M.\,E.\,J.}
Power-law distributions in empirical data~// SIAM Rev., 2009.
Vol.~51. No.\,4. P.~661--703.

\bibitem{Aivaz}
\Au{Айвазян С.\,А., Енюков И.\,С., Мешалкин~Л.\,Д.} 
Прикладная статистика. Т.~1: Основы моделирования и первичная
обработка данных.~--- М.: Финансы и статистика, 1983. 471~с.

\label{end\stat}

\bibitem{Lem}
\Au{Лемешко Б.\,Ю., Чимитова Е.\,В.} О~выборе числа
интервалов в критериях согласия типа $\chi^2$~// Заводская
лаборатория. Диагностика материалов, 2003. Т.~69. С.~61--67.
 \end{thebibliography}
}
}


\end{multicols}       