
%\newcommand{\vp}{{\mathbf {p}}
%\newcommand{\A}{{\mathbf A}}

\def\stat{zeifman}

\def\tit{ОЦЕНКИ УСТОЙЧИВОСТИ  ДЛЯ НЕКОТОРЫХ СИСТЕМ ОБСЛУЖИВАНИЯ С КАТАСТРОФАМИ$^*$}

\def\titkol{Оценки устойчивости  для некоторых систем обслуживания с катастрофами}

\def\autkol{А.\,И.~Зейфман, А.\,В.~Коротышева, Т.\,Л.~Панфилова, С.\,Я.~Шоргин}
\def\aut{А.\,И.~Зейфман$^1$, А.\,В.~Коротышева$^2$, Т.\,Л.~Панфилова$^3$, С.\,Я.~Шоргин$^4$}

\titel{\tit}{\aut}{\autkol}{\titkol}

{\renewcommand{\thefootnote}{\fnsymbol{footnote}}\footnotetext[1]
{Исследование поддержано грантами РФФИ 11-07-00112-а и 11-01-12026-офи-м.}}

\renewcommand{\thefootnote}{\arabic{footnote}}
\footnotetext[1]{Вологодский государственный педагогический университет;  
Институт проблем информатики Российской академии наук; Институт 
со\-ци\-аль\-но-эко\-но\-ми\-че\-ско\-го развития территорий Российской 
академии наук,  a\_zeifman@mail.ru}
\footnotetext[2]{Вологодский государственный педагогический
университет,  a\_korotysheva@mail.ru}
\footnotetext[3]{Вологодский государственный педагогический университет, ptl-70@mail.ru}
\footnotetext[4]{Институт проблем информатики Российской академии наук, SShorgin@ipiran.ru}

\Abst{Рассматриваются модели обслуживания, описываемые
марковскими цепями с непрерывным временем в случае наличия 
катастроф. Получены оценки устойчивости
различных характеристик таких систем. Рассмотрен пример конкретной
системы обслуживания.}

\KW{нестационарные системы обслуживания;
марковские модели с катастрофами; оценки устойчивости; аппроксимация предельных характеристик}

  \vskip 14pt plus 9pt minus 6pt

      \thispagestyle{headings}

      \begin{multicols}{2}
      
            \label{st\stat}

\section{Введение}

Системы массового обслуживания с катастрофами (СМО c катастрофическими сбоями, 
queues with disasters, queues with catastrophes) в разных ситуациях и при разных 
предположениях изучались во многих работах 
(см., например,~[1--10]).

Устойчивость нестационарных марковских цепей  с непрерывным временем изучалась начиная 
с~\cite{z85}, затем в работах~\cite{z98, ae}, а для систем обслуживания, описываемых 
нестационарными процессами рождения и гибели  
с катастрофами~в случае, когда интенсивность катастрофы не 
зави\-сит от числа требований в системе и является существенной, рассмотрена в 
нашей предыдущей работе~\cite{z10}.
{\looseness=-1

}

Здесь будет изучена более общая ситуация сис\-те\-мы обслуживания, чис\-ло 
требований в которой описывается марковской цепью с непрерывным временем 
и дискретным пространством состояний в той ситуации, когда интенсивности 
катастроф  зависят от числа требований в системе, но являются существенными.

Пусть $X=X(t)$, $t\geq 0$,~--- чис\-ло требований в сис\-те\-ме обслуживания.

Обозначим через $p_{ij}(s,t)\hm=\mathrm{Pr}\left\{ X(t)\hm=j\vert X(s)=\right.$\linebreak $\left.=i
\right\}$, $i,j \ge 0$, $0\leq s\leq t$, переходные вероятности
процесса $X=X(t)$, а через  $p_i(t)=\mathrm{Pr}\left\{ X(t) \hm=i \right\}$~---
его вероятности со\-сто\-яний, через $\xi_i(t)$~--- интенсивность катастрофы при наличии 
$i$ требований в сис\-те\-ме, через $a_{i+k,i}(t)$~--- интенсивность поступления 
$k$~требований в сис\-те\-му обслуживания, в которой уже есть  $i$~требований, и, 
наконец, через  $a_{i-k,i}(t)$~--- интенсивность обслуживания $k$ требований в сис\-те\-ме 
обслуживания, в которой имеется  $i$~требований.

Тогда при выполнении естественных дополнительных условий (см., например,~\cite{z08b}) 
прямую сис\-те\-му Колмогорова для вероятностей состояний
\begin{equation}
 \left.
\begin{array}{rl}
\!\!\!\!\fr{dp_0}{dt} &= \displaystyle a_{00} (t)p_0 +\sum\limits_{i \ge 1} \left(a_{0i}(t)+ \xi_i(t)\right)p_i;  \\[9pt]
\!\!\!\!\fr{dp_k}{dt} &= \displaystyle\left(a_{kk}(t)-\xi_k(t)\right)p_{k} +\sum\limits_{i \neq k} a_{ki}(t)p_i,\  k \ge 1,
\end{array}
\right\} \!\!
\label{eq111}
\end{equation}
можно записать в виде дифференциального уравнения
\begin{equation}
\fr{d\mathbf{p}}{dt}=A \left( t\right) \mathbf {p}
\label{eq112}
\end{equation}
в пространстве последовательностей~$l_1$, где  
$\mathbf {p}(t)\hm=\left(p_0(t),p_1(t),\dots\right)^{\mathrm{T}}$~---
век\-тор-стол\-бец вероятностей со\-сто\-яний, а операторная функция~$A(t)$ 
локально интегрируема на $[0,\infty)$  и ограничена почти при всех $t \ge 0$ 
(см.\  подробное рассмотрение в \cite{z06}).
Обозначим через $\Omega\hm=\left\{{\bf x}: \: {\bf x}\geq 0,\: \|{\bf
x}\|_1\hm=1\right\}$ множество всех стохастических векторов.
Тогда
\begin{equation*}
\|A(t)\|_1  = 2 \sup\limits_{i}\left(|a_{ii}(t)|+ \xi_i(t)\right) < \infty 
%\label{eq112bb}
\end{equation*}
почти при всех $t \ge 0$, а значит, задача Коши для
уравнения~(\ref{eq112}) с начальным условием~$\mathbf {p}(0)$ имеет
единственное решение
\begin{equation*}
{\bf p}(t) =  U(t){\bf p}(0)\,, 
%\label{eq112bc}
\end{equation*}
\noindent где $U(t,s)$~--- оператор Коши уравнения~(\ref{eq112}). При
этом если ${\bf p}(s) \in \Omega$, то и ${\bf p}(t) \in \Omega$ при
любом  $t \ge s$.

Рассмотрим теперь <<возмущенный>> процесс обслуживания  $\bar{X}\hm=\bar{X}(t)$, $t\geq 0$, 
обозначая все его соответствующие характеристики так же, как и у 
невозмущенного процесса, с дополнительной верхней чертой. Положим 
$\hat{A} (t) \hm= {A} (t) \hm- \bar{A} (t)$ и для простоты записи 
оценок будем предполагать, что возмущения <<равномерно малы>>, т.\,е.\ 
почти при всех $t \ge 0$ выполняется неравенство
\begin{equation*}
\left\vert\hat{A}(t)\right\vert \le \varepsilon\,.
%\label{p1}
\end{equation*}

\section{Оценки для вектора состояний}

\noindent
\textbf{Теорема 1.} \textit{Пусть $\xi_i(t) \hm\ge \xi(t) \ge b \hm>0$ при всех 
$i \hm\ge 0$ и почти всех $t \hm\ge 0$. Тогда при любых начальных условиях 
${\bf p}(0)$ и $\bar{\bf p}(0)$ справедлива следующая оценка устойчивости:}
\begin{equation*}
\overline{\lim_{t \to \infty}}\left\|\mathbf {p}(t)-{\bar{\mathbf {p}}}(t)\right\|\le \fr{\varepsilon}{b}\,.
%\label{201}
\end{equation*}

\medskip


\noindent
Д\,о\,к\,а\,з\,а\,т\,е\,л\,ь\,с\,т\,в\,о\,.\ Перепишем первое уравнение сис\-те\-мы~(\ref{eq111}) 
в следующем виде:
\begin{multline*}
\fr{dp_0}{dt} = \left(a_{00}(t)-\xi(t)\right) p_0 +
\sum\limits_{i \ge 1} \left(a_{0i}(t)+{}\right.\\
\left.{}+ \xi_i(t)-\xi(t)\right)p_i +\xi(t) \,.  
%\label{202}
\end{multline*}

Теперь уравнение~(\ref{eq112}) можно записать так:
\begin{equation}
\fr{d\mathbf {p}}{dt}=B\left( t\right) \mathbf {p} + {\bf f}(t)\,,
\label{203}
\end{equation}
причем матрица $B\left( t\right)$ отличается только первой строкой 
очевидным образом от матрицы $A\left( t\right)$, а 
${\bf f}(t) = \left(\xi(t),0,\dots\right)^{\mathrm{T}}$.

Тогда
\begin{equation}
\mathbf {p}(t)=V(t)\mathbf {p}(0)+\int\limits_0^t V(t,\tau){\bf f}(\tau) \, d\tau\,, 
\label{204}
\end{equation}
где  $V(t,s)$~--- оператор Коши уравнения~(\ref{204}).

Далее, оценивая логарифмическую норму оператора~$B(t)$ в пространстве~$l_1$ 
(см., например, подробное рассмотрение в~\cite{z08b, z95b, dzp}), получаем
\begin{multline*}
\gamma \left(B(t)\right)_{1} = 
\max\left(a_{00}(t)-\xi(t) + \sum_{i \ge 1} a_{i0}(t),\right.  \\
\sup\limits_{i\ge 1}\left(
\vphantom{\sum_{j\neq i,j\ge 1}}a_{ii}(t)-\xi_i(t)+ a_{0i}(t) + \xi_i(t) - \xi(t) +{}\right.\\
\left.\left.{}+ \sum_{j\neq i,j\ge 1} a_{ji}(t)\right)\right) = -\xi(t)\,.
%\label{205}
\end{multline*}
Тогда
\begin{equation*}
\|V(t,s)\| \le  e^{-\int\limits_s^t \xi(\tau)\, d\tau}
%\label{206}
\end{equation*}
для всех $0 \le s \le t$.

А следовательно, при любых начальных условиях 
${\bf p^{*}}(s) \hm\in \Omega$,  ${\bf p^{**}}(s) \hm\in \Omega$ и 
любых $s \hm\ge 0$, $t \hm\ge s$ справедлива оценка
\begin{multline}
\|{\bf p^{*}}(t) - {\bf p^{**}}(t)\| \le  
e^{-\int\limits_s^t \xi(\tau)\, d\tau}\|{\bf p^{*}}(s) - {\bf p^{**}}(s)\| \le{}\\
{}\le 2e^{-\int\limits_s^t \xi(\tau)\, d\tau} \le 2 e^{-b(t-s)}\,.
\label{207}
\end{multline}
Теперь можно слегка модифицировать подход, предложенный в~\cite{mit03}.

Положим
\begin{multline*} 
\beta (t, s)=\sup\limits_{ \| {\bf v} \| =1, \sum v_i=0} {\|U(t){\bf v}\|} ={} \\
{}= \fr{1}{2} \,\sup\limits_{i,j} \sum\limits_k {|p_{ik}(t, s)-p_{jk}(t, s)|}\,.
%\label{208}
\end{multline*}
Тогда
\begin{multline*}
\|{\bf p}(t)-{\bf \bar{p}}(t)\| \le \beta(t,s)\|{\bf p}(s)-{\bf \bar{p}}(s)\| +{}\\
{}+ \int\limits_s^t \|\hat{A}(u)\| \beta(u, s)\, du\,,
%\label{209}
\end{multline*}
причем
\begin{equation*}
\beta (t,s) \le 1\,, \enskip \beta (t,s) \le \fr{ce^{-b(t-s)}}{2}\,, \enskip 0 \le s \le t\,,
%\label{210}
\end{equation*}
где $c$~--- константа, принимающая в~(\ref{207}) значение~2.

В результате (при произвольном $c\ge 2$) получаем неравенство:
\begin{multline} 
\|{\bf p}(t)-\bar{\bf p}(t)\| \le{} \\
\!\!\!{}\le
\begin{cases}
\|{\bf p}(s)-{\bf \bar{p}}(s)\|+ (t-s)\varepsilon , &\!\!\!\!\!\!\!\!\!\!\!\!\!\!\!\!\!\!\! 0<t< b^{-1} \ln {\fr{c}{2}}; \\
b^{-1}\left(\ln {\fr{c}{2}} +1-ce^{-b(t-s)}\right)\varepsilon +{}&\\
{}+ \fr{c}{2}e^{-b(t-s)} \|{\bf p}(s)-{\bf \bar{p}}(s)\|, &\!\!\!\!\!\!\!\!\!\!\!\!\! 
t\ge b^{-1}\ln {\fr{c}{2}}\,.
\end{cases}\!\!
\label{211}
\end{multline}

Значит, в рассматриваемой ситуации (при $c=2$) из~(\ref{211}) при всех $t \hm\ge s$ 
вытекает оценка
\begin{multline*}
\|{\bf p}(t)-\bar{\bf p}(t)\| \le
b^{-1}(1-2e^{-b(t-s)})\varepsilon +{}\\
{}+ e^{-b(t-s)} \|{\bf p}(s)-{\bf \bar{p}}(s)\|\,,
%\label{212}
\end{multline*}
из которой и следует утверждение теоремы.

\bigskip

\noindent
\textbf{Теорема 2.} \textit{Пусть все интенсивности 1-пе\-рио\-дич\-ны, 
$\xi_i(t) \hm\ge \xi(t)$ при всех $i \hm\ge 0$ и почти всех $t \hm\in [0,1]$, 
а $\int\limits_0^1 \xi(t) \, dt \hm= \theta \hm> 0$. 
Тогда для любых начальных условий ${\bf p}(0)$ и $\bar{\bf p}(0)$
справедливо неравенство}
\begin{equation*}
\overline{\lim\limits_{t \to \infty}} \|{\bf p}(t) - \bar { \bf p}(t)\| \le
\fr{\varepsilon \left( 1+\theta\right)}{\theta}\,.
%\label{213}
\end{equation*}

\medskip
Для доказательства отметим, что теперь вмес\-то~(\ref{207}) справедлива оценка скорости сходимости
\begin{equation*}
\|{\bf p^{*}}(t) - {\bf p^{**}}(t)\| \le   
2e^{-\int\limits_s^t \xi(\tau)\, d\tau} \le 2 e^{\theta}e^{-\theta(t-s)}
%\label{214}
\end{equation*}
и, следовательно, из~(\ref{211}) вытекает при $t \hm\ge 1$  неравенство
\begin{multline*}
\|{\bf p}(t)-\bar{\bf p}(t)\| \le{}\\
{}\le 
\theta^{-1}(\theta +1-2 e^{\theta}e^{-\theta(t-s)})\varepsilon 
+ e^{\theta}e^{-\theta(t-s)} \|{\bf p}(s)-{\bf \bar{p}}(s)\|,\hspace*{-4.68616pt}
%\label{215}
\end{multline*}
а с ним и требуемая оценка.

\medskip

\noindent
\textbf{Замечание 1.} Отметим, что выписанные в теоремах~1 и~2 
оценки устойчивости справедливы, разумеется, и в случае конечного пространства состояний.

\section{Оценки для среднего}

Обозначим через $E_k(t) \hm= E\left\{X(t)\left|X(0)\hm=k\right.\right\}$ 
математическое ожидание процесса в момент~$t$ при условии, что в нулевой 
момент времени он находится в состоянии~$k$, а через $E_{\bf p}(t)$ обозначим 
математическое ожидание процесса в момент~$t$ при начальном распределении 
вероятностей состояний ${\bf p}(0) = {\bf p}$.

\medskip

Легко видеть, что тогда $\left|E_{\bf p}(t)\hm- \bar{E}_{\bf
\bar{p}}(t)\right| \hm\le \sum\limits_k k|p_k(t)-\bar{p}_k(t)|$, а значит,
если пространство состояний системы конечно (общее число требований
в системе обслуживания не превосходит $S \hm< \infty$), то
$\left|E_{\bf p}(t)\hm- \bar{E}_{\bf \bar{p}}(t)\right| \hm\le S\|{\bf
p}(t)-\bar{\bf p}(t)\|$.

А тогда  из теорем~1 и~2 сразу вытекают соответствующие оценки устойчивости средних:
\begin{equation}
\overline{\lim_{t \to \infty}} \left|E_{\bf p}(t)- \bar{E}_{\bf \bar{p}}(t)\right| \le
\fr{S\varepsilon }{b}
\label{301}
\end{equation}
и
\begin{equation}
\overline{\lim_{t \to \infty}} \left|E_{\bf p}(t)- \bar{E}_{\bf \bar{p}}(t)\right| \le
\fr{S\varepsilon \left( 1+\theta\right)}{\theta}
\label{302}
\end{equation}
соответственно.

\medskip

Основные трудности в этом параграфе связаны с ситуацией, когда количество возможных 
состояний системы очень велико или бесконечно. В~этом случае оценки~(\ref{301}) 
и~(\ref{302}) становятся неэффективными.

Для получения оценок устойчивости среднего здесь приходится использовать 
специальные перенормировки.

Перепишем уравнение~(\ref{203}) в следующем виде:
\begin{equation*}
\fr{d\mathbf {p}}{dt}= \bar{B}\left( t\right) \mathbf {p} + {\bf f}(t) +\hat{B}\left( t\right) \mathbf {p}\,.
%\label{303}
\end{equation*}
Тогда
\begin{align*}
\mathbf {p}(t)&=\bar{V}(t)\mathbf {p}(0)+\int\limits_0^t \bar{V}(t,\tau){\bf f}(\tau) \, d\tau+{}\notag\\
&\hspace*{20mm}{}+\int\limits_0^t \bar{V}(t,\tau)\hat{B}(\tau){\bf p}(\tau)\, d\tau\,;
%\label{304} 
\\
\bar{\mathbf {p}}(t)&=\bar{V}(t)\bar{\mathbf {p}}(0)+\int\limits_0^t \bar{V}(t,\tau)\bar{\bf f}(\tau) \, d\tau 
%\label{305} 
\end{align*}
и в {\it любой} норме при одинаковых начальных условиях справедлива оценка
\begin{multline}
\left\|\mathbf {p}(t)-\bar{\mathbf {p}}(t)\right\|\le {}\\
{}\le\int\limits_0^t \|\bar{V}(t,\tau)\|
\left(\|\hat{B}(\tau)\|\|{\bf p}(\tau)\| + \|\hat{\bf f}(\tau)\|\right)\, d\tau\,.
\label{306}
\end{multline}

Будем теперь дополнительно предполагать, что существуют числа $K,\,N$ такие, что
\begin{multline}
\xi(t) \le K < \infty,\quad \sup_i |a_{ii}(t)| \le K
\\  \mbox { почти при всех } t \ge 0\,;
\label{307}
\end{multline}

\vspace*{-6pt}

\noindent
\begin{multline}
a_{i+k,i}(t) = 0  \mbox { при всех } k \ge N\\
\mbox { и почти при всех } t \ge 0.
\label{308}
\end{multline}

\smallskip

\noindent
\textbf{Теорема 3.} \textit{Пусть выполнены условия теоремы~$1$, 
а также}~(\ref{307}) \textit{и}~(\ref{308}). \textit{Пусть существует воз\-рас\-та\-ющая 
последовательность положительных чисел $\{d_i\}$ такая, что} 
(\textit{a})~$\inf_{k \hm\ge 1}({d_k}/{k})\hm=w>0$; 
(\textit{б})~$\sup\limits_{k}({d_{k+1}}/{d_k}) \hm=m \hm< \infty$;
(\textit{в})~$b \hm- \left(m^N-1\right)K \hm>0$. 
\textit{Тогда при любых начальных условиях~${\bf p}(0)$ 
и $\bar{\bf p}(0)$ справедлива сле\-ду\-ющая оценка устойчивости среднего:}
\begin{multline*}
\overline{\lim\limits_{t \to \infty}} \left|E_{\bf p}(t)- \bar{E}_{\bf \bar{p}}(t)\right| \le{}\\
{}\le  \fr{\varepsilon\left(b+K\right)}{w\left(b - \left(m^N-1\right)K\right)\left(b - \left(m^N-1\right)K- m^N\varepsilon\right)}.
\hspace*{-1.62909pt}
%\label{309}
\end{multline*}

%\pagebreak

\noindent
Д\,о\,к\,а\,з\,а\,т\,е\,л\,ь\,с\,т\,в\,о\,.  
Положим $d_0=1$. Рассмотрим диагональную матрицу
%\noindent
\begin{equation*}
D=\mathrm{diag}\,\left(d_0, d_1, d_2, \dots \right)  
%\label{310}
\end{equation*}
и соответствующее пространство последовательностей 
$l_{1D}=\left\{{\bf z} \hm=(p_0,p_1,p_2,\ldots)^{\mathrm{T}}\right\}$ таких, 
что $\|{\bf z}\|_{1D}\hm=\|D {\bf z}\|_1 \hm<\infty$. Тогда имеем 
$w\|{\bf z}\|_{1E} \hm= w\sum\limits_k k|p_k| \hm\le \|{\bf z}\|_{1D}$.

Оценим теперь логарифмическую норму $\gamma(B(t))_{1D}$:
\begin{multline*}
\gamma(B(t))_{1D} = \gamma(DB(t)D^{-1})_{1} = {}\\
{}=
\max\left(a_{00}(t)-\xi(t) + \sum\limits_{i \ge 1} \fr{d_i}{d_0}\,a_{i0}(t),\right.  \\\
\left.\sup_{i\ge 1}\left(
\vphantom{\sum_{j\neq i,j\ge 1}}
a_{ii}(t)-\xi_i(t)+ \fr{d_0}{d_i}\left(a_{0i}(t) + \xi_i(t) 
- \xi(t)\right) + {}\right.\right.\\
\left.\left.{}+\sum_{j\neq i,j\ge 1} \fr{d_j}{d_i}\,a_{ji}(t)\right)\right) \le {}\\
{}\le -\xi(t) + \left(m^N-1\right)\sup\limits_{i}|a_{ii}(t)| \le{}\\
{}\le  -b + \left(m^N-1\right)K\,. 
%\label{205}
\end{multline*}
%\label{311}

Далее
\begin{equation*}
\|\hat{B}(t)\|_{1D}=\|D\hat{B}(t)D^{-1}\|_{1} \le  m^N\varepsilon\,.
%\label{312}
\end{equation*}

А тогда
\begin{multline*}
\gamma(\bar{B}(t))_{1D} \le \gamma(B(t))_{1D}+\|\hat{B}(t)\|_{1D}  \le{}\\
{}\le -b + \left(m^N-1\right)K + m^N\varepsilon\,.
%\label{313}
\end{multline*}

Оценим теперь
\begin{multline*}
\|{\bf p}(t)\|_{1D} \le
\|V(t){\bf p}(0) \|_{1D} +{}\\
{}+
 \int\limits_0^t \| V(t,\tau){\bf f}(\tau)\, d\tau \|_{1D } \le {} \\
{}\le e^{-\left(b - \left(m^N-1\right)K\right)t}\|{\bf p}(0) \|_{1D} +  
\fr{K}{b - \left(m^N-1\right)K}\,.
%\label{314}
\end{multline*}

Легко видеть, что $\|{\bf \hat{f}}(t)\| \hm\le \varepsilon$ почти при всех $t\ge 0$.

Тогда с учетом~(\ref{306}) имеем
\begin{multline*}
\left\|\mathbf {p}(t)-\bar{\mathbf {p}}(t)\right\|_{1D} \le \int\limits_0^t e^{-\left(b - \left(m^N-1\right)K - m^N\varepsilon\right)\left(t-\tau\right)}\times \\
{}\times\left(
\vphantom{\fr{K}{b - \left(m^N-1\right)K}}
m^N\varepsilon\left(\vphantom{\fr{K}{b - \left(m^N-1\right)K}}
e^{-\left(b - \left(m^N-1\right)K\right)\tau}\|{\bf p}(0) 
\|_{1D} + {}\right.\right.
\end{multline*}

\noindent
\begin{multline*}
\left.\left.{}+ \fr{K}{b - \left(m^N-1\right)K}\right) + \varepsilon\right)\, d\tau  \le {}\\ 
{}\le\mbox {o} \left(1\right)+ \frac{\varepsilon\left(1+  {Km^N}/({b - \left(m^N-1\right)K})\right)}{b - \left(m^N-1\right)K - m^N\varepsilon}\,.
%\label{315}
\end{multline*}
А тогда
\begin{multline*}
\overline{\lim_{t \to \infty}} \left\|\mathbf {p}(t)-\bar{\mathbf {p}}(t)\right\|_{1D} \le{}\\
\!\!\!\!{}\le\! \fr{\varepsilon\left(b+K\right)}{\left(b - \left(m^N-1\right)K\right)\left(b - \left(m^N-1\right)K- m^N\varepsilon\right)},\!\!\!\!
%\label{316}
\end{multline*}
откуда и получаем требуемую оценку.

\bigskip

Рассуждая так же, как при доказательстве теоремы~2, получаем следующее утверждение.

\bigskip

\noindent
\textbf{Теорема 4.} \textit{Пусть выполнены условия теоремы~2, а также}~(\ref{307}) 
\textit{и}~(\ref{308}). 
\textit{Пусть существует возрастающая последовательность положительных чисел~$\{d_i\}$ такая, 
что} (\textit{a})~$\inf\limits_{k \ge 1}({d_k}/{k})=w\hm>0$; 
(\textit{б})~$\sup\limits_{k}({d_{k+1}}/{d_k})\hm =m \hm< \infty$;
(\textit{в})~$\theta \hm- \left(m^N-1\right)K \hm>0$.  
\textit{Тогда при любых начальных условиях~${\bf p}(0)$ и~$\bar{\bf p}(0)$ справедлива сле\-ду\-ющая 
оценка устойчивости среднего:}
\begin{multline*}
\overline{\lim_{t \to \infty}} \left|E_{\bf p}(t)- \bar{E}_{\bf \bar{p}}(t)\right| \le{}\\
{}\le  \fr{\varepsilon e^{\theta}\left(\theta + K + Km^N\left(e^{\theta}-1\right)\right)}{w\left(\theta - \left(m^N-1\right)K\right)\left(\theta - 
\left(m^N-1\right)K- m^N\varepsilon\right)}.\hspace*{-2.9923pt}
%\label{309}
\end{multline*}


\section{Пример}

Рассмотрим здесь простейшую систему обслуживания с групповым поступлением  и 
обслуживанием требований с катастрофами. Будем предполагать, что требования 
поступают группами не более трех одновременно с одинаковыми интенсивностями 
$a_{i+k,i}(t) \hm= \lambda(t) \hm= 2 \hm+ \sin 2\pi t, \ 1 \hm\le k \hm\le 3$, 
одновременно обслуживается одно или два требования также с одинаковыми интенсивностями  
$a_{i-k,i}(t)\hm = \mu (t) \hm= 1 + \cos 2\pi t$, $1 \le k \le 2$, 
а интенсивность катастрофы при наличии $k$ требований в системе есть 
$\xi_k(t)\hm=5 \hm+ \sin 2 \pi k t \hm+ \cos 2 \pi t, k \ge 1$.  Тогда 
$\xi (t)\hm=4 \hm+ \cos 2 \pi t$, $b\hm=3$, $\theta\hm=4$, $K\hm=13$. Положим 
$\varepsilon\hm=10^{-3}$. Тогда получаем следующие оценки устойчивости:
\begin{itemize}
\item по теореме~1
\begin{equation*}
\overline{\lim\limits_{t \to \infty}}  \| {\bf p}(t)-\bar{{\bf p}}(t)\| \le 0{,}333\cdot 10^{-3}\,;
\end{equation*}
\item
по теореме 2
\begin{equation*}
\overline{\lim\limits_{t \to \infty}}  \| {\bf p}(t)-\bar{{\bf p}}(t)\| \le 1{,}25 \cdot 10^{-3}\,.
\end{equation*}
\end{itemize}

Далее, для получения оценок устойчивости среднего положим $d_i\hm=1{,}1^i$, 
имеем тогда  $K\hm=\sup\limits_{t,i}\xi_i(t) \hm= 7$,  $\omega\hm=0{,}259$, $m\hm=1{,}1$.

Применяя подход теоремы~3 с учетом структуры инфинитизимальной матрицы процесса, 
получаем следующие оценки:
\begin{itemize}
\item по теореме 3
\begin{equation*}
\overline{\lim\limits_{t \to \infty}}  {\left| E_{\bf p}(t)-\bar{E}_{\bar{\bf p}}(t)\right|} \le 0{,}053\,;
\end{equation*}

%\vspace*{-15pt}

\item
по теореме~4
\begin{equation*}
\overline{\lim\limits_{t \to \infty}} {\left| E_{\bf p}(t)-\bar{E}_{\bar{\bf p}}(t)\right|} \le 4{,}32\,.
\end{equation*}

%\vspace*{-15pt}
\end{itemize}

%\bigskip

Положим
\begin{equation*}
\omega_n^1=\sup\limits_{k \ge n-2}\fr{1}{d_k}; \quad \omega_n^2=\sup\limits_{k \ge n-2}\fr{k}{d_k}\,.
\end{equation*}

Рассмотрим семейство <<усеченных>> процессов $X_n(t)$ с
фазовыми пространствами $E_n \hm= \{0,1,\dots,n\}$, теми же
интенсивностями при  $k \hm\le n$ и матрицами интенсивностей~$A_n(t)$.

\medskip

\noindent
\textbf{Теорема 5.} \textit{Пусть выполняются условия теоремы}~3 \textit{и $\xi_i(t) \le B$, 
тогда
\begin{align*}
\|{\bf p} (t) - {\bf p}_n(t)\| &\le  \fr{3(K+B)\omega_n^1 Kt }{b-(m^N-1)K}\,;
\\
\left|E_{\bf p}(t)- {E}_{\bf p_n}(t)\right| &\le \fr{ 9(K+B)\omega_n^2 K t}{b-(m^N-1)K}
\end{align*}
при всех $t \ge 0$, любом $n \ge N$ и  начальных условиях ${\bf p}(0)={\bf p}_n(0)=0$.
}

\medskip

\noindent
Д\,о\,к\,а\,з\,а\,т\,е\,л\,ь\,с\,т\,в\,о\,.\ Будем отождествлять векторы
$\left(x_1,\dots,x_n,0,0,\dots\right)^{\mathrm{T}}$ и $\left(x_1,\dots,x_n
\right)^{\mathrm{T}}$.  Рас\-смот\-рим прямую систему Колмогорова   для
исходного процесса  в следующей форме:

\noindent
\begin{equation*}
\fr{d\mathbf{p}}{dt}=A_n(t) \mathbf{p}  +\left(A(t) -
A_n(t) \right) \mathbf{p}\,, 
%\label{cat06}
\end{equation*}
а также соответствующую систему
\begin{equation*}
\fr{d\mathbf{p_n}}{dt}=A_n(t) \mathbf{p_n}
%\label{cat061}
\end{equation*}
для усеченного процесса.
\smallskip

Имеем
$ %\begin{equation*}
{\bf p}_n (t)= U_n(t){\bf p} (0)
$ %\end{equation*}
при ${\bf p} (0) = {\bf p}_n (0)$ и
\begin{multline*}
{\bf p} (t)= U_n \left(t\right) {\bf p} (0) +{}\\
{}+ \int\limits_0^t U_n \left(t,
\tau\right) \left(A(\tau) - A_n(\tau) \right) {\bf p} (\tau)\,
d\tau\,. \\[-15pt]
%\label{eq3316}
\end{multline*}
Тогда (в любой норме) получаем:
\begin{multline*}
\left\|{\bf p} (t) - {\bf p}_n (t)\right\| ={}\\
{}= \left\|\int\limits_0^t
U_n \left(t,
 \tau\right) \left(A(\tau) - A_n(\tau) \right) {\bf p} (\tau)\, d\tau \right\|\,.\\[-15pt]
%\label{cat063}
\end{multline*}

Рассмотрим матрицу Коши:
\begin{equation*}
U_n =\begin{pmatrix}
  u_{00}^n & . & . & u_{0n}^n  & 0 & 0 & \cdots \\
u_{10}^n & . & . & u_{1n}^n  & 0 & 0 & \cdots \\
\cdots \\
u_{n0}^n & . & . & u_{nn}^n  & 0 & 0 & \cdots \\
0 & . & . & 0 & 1 & 0 & \cdots \\
0 & . & . & 0 & 0 & 1 & \cdots \\
\cdots
\end{pmatrix}\,.
%\label{18}
\end{equation*}

\vspace*{-13pt}

\noindent
Тогда\\[-17pt]
\begin{multline*}
\left(A -A_n\right) {\bf p} ={}
\\
{}
%{\scriptsize 
= \left(
0,\dots, \left(p_{n-2}+p_{n-1} +p_n\right)\lambda(t)+\left(p_{n+2}+{}\right.\right.\\
\left.\left.{}+p_{n+3}\right)\mu(t)-
\left(\xi_{n+1}+2\mu(t)+3\lambda(t)\right)p_{n+1}, \dots \right)^{\mathrm{T}}
\end{multline*}
%}
и, следовательно,
\end{multicols}

\hrule

%\vspace*{-24pt}

%\vspace*{2pt}

{%\scriptsize 
%\noindent
\begin{equation*}
U_n\left(A -A_n\right) {\bf p} =\left(\begin{array}{c}
0\\
\vdots \\
0 \\
\left(p_{n-2}+p_{n-1} +p_{n}\right)\lambda(t)+\left(p_{n+2}+p_{n+3}\right)\mu(t)-\left(\xi_{n+1}+2\mu(t)+3\lambda(t)\right)p_{n+1} \\
\left(p_{n-1}+p_n +p_{n+1}\right)\lambda(t)+\left(p_{n+3}+p_{n+4}\right)\mu(t)-\left(\xi_{n+2}+2\mu(t)+3\lambda(t)\right)p_{n+2} \\ \vdots
\end{array}\right)\,;
%\label{21}
\end{equation*}}



%\smallskip

\noindent
\begin{multline*}
\|U_n \left(A -A_n \right) {\bf p}\| =
\sum\limits_{k \ge 0}  \left| \left(p_{n+k}+p_{n+k-1} +p_{n+k-2}\right)\lambda(t)+{}\right.
{}+\left(p_{n+k+2}+p_{n+k+3}\right)\mu(t)-{}\\
\left.{}-  \left(\xi_{n+k+1}+2\mu(t)+3\lambda(t)\right)p_{n+k+1} \right|\le 
3\lambda(t)\sum\limits_{k \ge -2}{p_{n+k}}+2\mu(t)\sum\limits_{k \ge 1}{p_{n+k}}+
(K+B)\times{}\\
{}\times \sum\limits_{k \ge 1}{p_{n+k}}\le 
3(K+B)\sum\limits_{k \ge -2}{p_{n+k}} 
\le 3(K+B)\omega_n^1 
\sum\limits_{k \ge -2}{d_{n+k}p_{n+k}} 
\le   3(K+B)\omega_n^1 \| {\bf p}(t)\|_{1D} \,;
\end{multline*}

%\hrule



\begin{figure} %fig1
\vspace*{1pt}
\begin{center}
\mbox{%
\epsfxsize=163.547mm
\epsfbox{zei-1-2.eps}
}
\end{center}
\vspace*{-3pt}
\begin{minipage}[t]{78mm}
%\includegraphics[width=80mm]{p0.jpg}
\Caption{Вероятность $Pr\left\{ \bar{X}(t) =0 \right\}$}
\end{minipage}
\hfill
\begin{minipage}[t]{78mm}
\Caption{Предельное среднее $\bar{\phi}(t)$}
\end{minipage}
\vspace*{12pt}
\end{figure}

\begin{multicols}{2}


%\vspace*{-9pt}

\noindent
\begin{multline*}
\|U_n \left(A -A_n \right) {\bf p}\|_{1E} = {}\\[2pt]
{}=
 \sum\limits_{k  \ge 0} \left| \left(p_{n+k}+p_{n+k-1} +{}\right.\right.\\[2pt]
\left. {}+p_{n+k-2}\right)\lambda(t)+
 \left(p_{n+k+2}+p_{n+k+3}\right)\mu(t) -{}   \\[2pt]
 \left.{}- \left(\xi_{n+k+1}+2\mu(t)+3\lambda(t)\right)p_{n+k+1} \right|  (n+k+1) \le{}\\[2pt]
{}\le (K+B) \sum\limits_{k\ge n-2}{\left(6k+3\right)p_k}\le {}\\[2pt]
{}\le 9(K+B)\omega_n^2 \sum\limits_{k \ge -2}{d_{n+k}p_{n+k}} \le{}\\[2pt]
{}\le  9(K+B)\omega_n^2 \| {\bf p}(t)\|_{1D}\, .
\end{multline*}




Отсюда и вытекают требуемые оценки.
\bigskip

С учетом оценок скорости сходимости получаем, что для построения с точностью~$\varepsilon$ 
предельных характеристик исходного процесса достаточно брать $t \ge 9$; а используя теорему~5, 
находим, что для достижения нужной точности усечения при $t \le 10$ достаточно выбрать $n=285$.


На рис.~1 приближенно (с
точностью до~$2\varepsilon$) построена предельная характеристика
$\bar{p}_0(t)$~--- вероятность отсутствия требований в возмущенной
системе обслуживания, а на рис.~2 (с той же точ\-ностью)~--- предельное
число требований  в возмущенной сис\-те\-ме обслуживания
$\bar{\phi}(t)$.



%\begin{figure*} %fig2
%\Caption{Предельное среднее $\bar{\phi}(t)$}
%\end{figure*}

{\small\frenchspacing
{%\baselineskip=10.8pt
\addcontentsline{toc}{section}{Литература}
\begin{thebibliography}{99}


\bibitem{du1} %1
\Au{Dudin~A., Nishimura~S.}   
A~BMAP/SM/1 queueing system with Markovian arrival input of disasters~//
J.~Appl. Probab., 1999. Vol.~36. P.~868--881.

\bibitem{KK} %2
\Au{Krishna~Kumar~B., Arivudainambi~D.} 
Transient solution of an
$M/M/1$ queue with catastrophes // Comput. Math. Appl., 2000. Vol.~40. P.~1233--1240.

\bibitem{du2} %3
\Au{Dudin~A., Karolik~A.} 
BMAP/SM/1 queue with Markovian input of disasters and non-instantaneous recovery~//
Perform. Eval., 2001. Vol.~45. P.~19--32.

\bibitem{Di} %4
\Au{Di~Crescenzo~A., Giorno~V., Nobile~A.\,G., Ricciardi~L.\,M.} 
On the $M/M/1$ queue with catastrophes and its continuous
approximation~// Queueing Syst., 2003. Vol.~43. P.~329--347.

\bibitem{du3} %5
\Au{Dudin~A., Semenova~O.} 
Stable algorithm for stationary distribution calculation for a BMAP/SM/1 queueing system 
with Markovian input of disasters~//      J.~Appl. Prob.,  2004.  Vol.~42. No.\,2.
P.~547--556.

\bibitem{Di08}  %6
\Au{Di~Crescenzo~A., Giorno~V., Nobile~A.\,G., Ricciardi~L.\,M.}
A~note on birth--death processes with catastrophes~//  Statist.
Probab. Lett., 2008. Vol.~78.  P.~2248--2257.

\bibitem{z08} %7
\Au{Zeifman~A., Satin~Ya., Chegodaev~A., Bening~V., Shorgin~V.}
Some bounds for $M(t)/M(t)/S$ queue with catastrophes~// 
4th  Conference (International) on Performance Evaluation
Methodologies and Tools Proceedings (Athens, Greece, October 20--24, 2008).~---
ACM digital library. DOI:10.4108/ICST.VALUETOOLS2008.4270.


\bibitem{z09a} %8
\Au{Зейфман А.\,И., Сатин~Я.\,А., Чегодаев~А.\,В.} 
О~нестационарных системах обслуживания с катастрофами~// Информатика и её применения, 2009. 
Т.~3. Вып.~1. С.~47--54.

\bibitem{z09b} %9
 \Au{Зейфман А.\,И., Сатин Я.\,А., Коротышева~А.\,В., Терешина~Н.\,А.} 
О~предельных характеристиках системы обслуживания $M(t)/M(t)/S$ с катастрофами~// 
Информатика и её применения, 2009. Т.~3. Вып.~3. С.~16--22.

\bibitem{z09c}  %10
\Au{Zeifman A., Satin Ya., Shorgin S., Bening~V.} 
On $M_n(t)/M_n(t)/S$ queues with catastrophes~//  4th 
 Conference (International) on Performance Evaluation Methodologies and Tools Proceedings 
(Pisa, Italy    October 19--23, 2009).~---     ACM digital library.
DOI:10.4108/ICST.VALUETOOLS2009.7442.

\bibitem{z85}  %11
\Au{Zeifman A.\,I.} Stability for contionuous-time
nonhomogeneous Markov chains~// Lect. Notes Math.,  1985. Vol.~1155. P.~401--414.

\bibitem{z98} %12
\Au{Zeifman A.} 
Stability of birth and death processes~// J.~Math. Sci., 1998. Vol.~91. P.~3023--3031.

\bibitem{ae} %13
\Au{Андреев Д., Елесин М., Кузнецов~А., Крылов~Е., Зейфман~А.} 
Эргодичность и устойчивость нестационарных систем обслуживания~// Теория вероятностей 
и математическая статистика, 2003. Т.~68. С.~1--11.

\bibitem{z10}  %14
\Au{Зейфман А.\,И., Коротышева А.\,В., Сатин~Я.\,А., Шоргин~С.\,Я.} 
Об устойчивости  нестационарных систем обслуживания с катастрофами~//  
Информатика и её применения, 2010. Т.~4. Вып.~3. С.~9--15.

\bibitem{z08b} %15
\Au{Зейфман А.\,И., Бенинг В.\,Е., Соколов~И.\,А.} 
Марковские цепи и модели с непрерывным временем.~--- М.: Элекс-КМ, 2008.

\bibitem{z06} %16
\Au{Zeifman~A., Leorato~S., Orsingher~E., Satin~Ya., Shilova~G.}
Some universal limits for nonhomogeneous birth and death
processes~// Queueing Syst., 2006. Vol.~52. P.~139--151.

\bibitem{z95b} %17
  \Au{Zeifman A.\,I.} Upper and lower bounds on the rate of
convergence for nonhomogeneous birth and death processes~//  Stoch. Proc. Appl., 1995. 
Vol.~59. P.~157--173.

\bibitem{dzp} %18
\Au{Van Doorn~E.\,A., Zeifman~A.\,I., Panfilova~T.\,L.}  
Bounds and asymptotics for the rate of convergence of birth--death processes~//  
Theor. Prob. Appl., 2010. Vol.~54. P.~97--113.

\label{end\stat}

\bibitem{mit03} %19
\Au{Mitrophanov~A.\,Yu.} Stability and exponential 
convergence of continuous-time Markov chains~//  J.~Appl. Prob., 2003. Vol. 40.
P. 970--979.

 \end{thebibliography}
}
}


\end{multicols}       