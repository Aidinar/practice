\def\stat{chupr}

\def\tit{УСИЛЕННЫЕ ЗАКОНЫ БОЛЬШИХ ЧИСЕЛ ДЛЯ~ЧИСЛА БЕЗОШИБОЧНЫХ
БЛОКОВ ПРИ~ПОМЕХОУСТОЙЧИВОМ КОДИРОВАНИИ}

\def\titkol{Усиленные законы больших чисел для числа безошибочных
блоков при помехоустойчивом кодировании}

\def\autkol{А.\,Н.~Чупрунов, И.~Фазекаш}
\def\aut{А.\,Н.~Чупрунов$^1$, И.~Фазекаш$^2$}

\titel{\tit}{\aut}{\autkol}{\titkol}

%{\renewcommand{\thefootnote}{\fnsymbol{footnote}}\footnotetext[1]
%{Работа выполнена при финансовой поддержке РФФИ (грант 11-01-00515).}}

\renewcommand{\thefootnote}{\arabic{footnote}}
\footnotetext[1]{Научно-исследовательский институт 
математики и механики им.\ Н.\,Г.~Чеботарева, achuprunov@mail.ru}
\footnotetext[2]{Дебреценский университет, fazekas.istvan@inf.unideb.hu}

\vspace*{2pt}

\Abst{Рассматриваются  сообщения, состоящие из блоков.
Каждый блок кодируется помехоустойчивым кодом, который может
исправить не более  $r$~ошибок. При этом предполагается, что
количество ошибок в блоке~--- независимая пуассоновская величина с
параметром~$\lambda$.  Кроме того, предполагается, что число ошибок
в сообщении принадлежит некоторому  подмножеству множества
неотрицательных целых чисел. В~работе получены усиленные законы
больших чисел для случайной величины~--- числа безошибочных блоков в
сообщении.}

\vspace*{2pt}

\KW{схема размещения; условная вероятность; закон
больших чисел; код БЧХ}

  \vskip 18pt plus 9pt minus 6pt

      \thispagestyle{headings}

      \begin{multicols}{2}
      
            \label{st\stat}

  

\section{Введение и~основные результаты}

Будем рассматривать код, который позволяет исправить не больше 
$r$~ошибок типа замещения. Частным случаем такого кода является код Боу\-за--Чоуд\-ху\-ри--Хок\-вин\-ге\-ма
(БЧХ) (о кодах  БЧХ см., 
например, в~[1]). Работа посвящена изучению
асимптотического поведения  случайной величины~$S_{nN}$~--- чис\-ла
безошибочных блоков в сообщении, состоящем из $N$~блоков, причем
каждый блок подвергается помехоустойчивому кодированию, а число
ошибок в сообщении принадлежит некоторому конечному
 подмножеству~$M_n$, $n\in\mathbf{N}$, множества неотрицательных целых чисел~$\mathbf{M}$.

Обозначим через~$\pi_{\lambda}$ пуассоновскую случайную величину с
параметром~$\lambda$; $\Phi$~--- функцию распределения стандартной
гауссовской случайной величины; $\stackrel{d}{=}$~--- равенство
распределений случайных величин.
 Будем предполагать, что все рассматриваемые случайные величины определены на
вероятностном пространстве $(\Omega,
\mathfrak{A}, \mathbf{P})$.


 Рассмотрим  сообщение, состоящее из $N$~блоков. Пусть случайная величина $\xi_{Nj}$~--- 
 количество ошибок в $j$-м блоке. Будем предполагать, что  $\xi_{Nj}$, $1\le
j\le N$,~--- независимые пуассоновские случайные величины с
параметром~ $\lambda$.
  Тогда  число безошибочных  блоков в сообщении~--- случайная величина
$$
S_{nN}=\sum\limits_{i=1}^NI_{nNi}\,,
$$
где $I_{nNi}$~--- индикатор события $A_{nNi}$, состоящего в том, что
$i$-й блок  сообщения имеет не более $r$~ошибок. Заметим, что
событие
\begin{multline*} 
A_{nNi}=\{\xi_{Ni}\le r\, |\, \xi_{N1}+\xi_{N2}+ \dots +\xi_{NN}\in
M_n\}={}\\
{}=\cup_{l=0}^r A_{nNil}\,,
\end{multline*}
где   события
$$ 
A_{nNil}=\{\xi_{Ni}=l\, |\,
 \xi_{N1}+\xi_{N2}+ \dots +\xi_{NN}\in M_n\}\,.
$$

Обозначим $n'_n=\sup\{n':\ \, n'\in M_n\}$, $\alpha_{nN}\hm={n'_n}/{N}$.

Если множество $M_n=\{n\}$ состоит из одного элемента, то события~$A_{nNi}$ 
являются событиями теории размещения различимых частиц по
различным ячейкам и не зависят от~$\lambda$~[2]. Подробное изложение
этой теории  можно найти в монографии~[3]. В~[4] получены усиленные
законы больших чисел для событий теории размещения различимых частиц
по различным ячейкам. В~част\-ности, в~[4] доказана

\medskip

\noindent
\textbf{Теорема \emph{A}}. \textit{Пусть $M_n=\{n\}$. Пусть $n,
N\hm\to\infty$ так, что $\alpha_{nN}\hm\to\alpha$, где $0\hm<\alpha\hm<\infty$.
Тогда
$$
\lim\limits_{n, N\to\infty}\fr{1}{N}S_{nN}=e^{-\alpha}\sum\limits_{k=0}^r\fr{\alpha^k}{k!}
$$
почти наверное.}

\pagebreak

%\medskip

Если множества $M_n=\mathbf{M}$, то индикаторы
$I_{nNi}=I_{\{\xi_{Ni}\hm\le r\}}$, $1\hm\le i\hm\le N$, независимы. Поэтому
справедлива

\medskip

\noindent
\textbf{Теорема \emph{Б}}. \textit{Пусть $M_n\hm=\mathbf{M}$.  Тогда
$$
\lim\limits_{N\to\infty}\fr{1}{N}S_{nN}=e^{-\lambda}\sum\limits_{k=0}^r\fr{\lambda^k}{k!}
$$
почти наверное.}

\medskip

Основными результатами статьи являются следующие теоремы.

\medskip

\noindent
\textbf{Теорема 1}. \textit{Предположим,  что $n, N\hm\to\infty$ так, что
$\alpha_{nN}\hm\to\alpha$, где $0\hm<\alpha\hm<\lambda$. Пусть $M_n\hm\subset
\mathbf{M}$~--- такие конечные подмножества, что
\begin{align}
\lim_{n, N\to\infty}\fr{\sum\limits_{k+l\in
M_n}\left({\alpha}/{\lambda}\right)^{n'_n-k-l}}{\sum\limits_{k\in
M_n}\left({\alpha}/{\lambda}\right)^{n'_n-k}}&=B_l \,;
\label{e1-chu}\\
\lim_{n, N\to\infty}\fr{\sum\limits_{k+2l\in
M_n}\left({\alpha}/{\lambda}\right)^{n'_n-k-2l}}{\sum\limits_{k\in
M_n}\left({\alpha}/{\lambda}\right)^{n'_n-k}}&=(B_l)^2\,,
\label{e2-chu}
\end{align}
где $B_l<\infty$, $1\hm\le l\hm\le r$. Положим $B_0\hm=1$.  Тогда
\begin{equation}
\lim_{n, N\to\infty}\fr{1}{N}S_{nN}=e^{-\alpha}\sum\limits_{k=0}^r\fr{\alpha^k}{k!}B_k
\label{e3-chu}
\end{equation}
в $L^2(\Omega, \mathfrak{A}, \mathbf{P})$.}


\medskip

\noindent
\textbf{Следствие}. \textit{Пусть $M_n$~--- такие конечные подмножества~$\mathbf{M}$ , 
что $M_n\hm\subset M_{n+1}$, $n\hm\in \mathbf{N}$, и
$\cup_{n=1}^{\infty}M_n \hm=\mathbf{M}$. Предположим, что $n, N\hm\to\infty$
так, что $\alpha_{nN}\hm\to\alpha$, где $0\hm<\alpha\hm<\lambda$.  Тогда
$$
\lim_{n,N\to\infty}\fr{1}{N}S_{nN}=e^{-\alpha}\sum\limits_{k=0}^r\fr{\alpha^k}{k!}
$$
в $L^2(\Omega, \mathfrak{A}, \mathbf{P})$.}


\medskip

\noindent
\textbf{Теорема 2}. \textit{Пусть множества  $M_n=\{0, 1,\dots, n\}$,
$n\in\mathbf{N}$, $\alpha >\lambda $.  Тогда
\begin{equation*}
\lim_{n, N\to\infty,
\lambda\le\alpha_{nN}\le\alpha}\fr{1}{N}S_{nN}=e^{-\lambda}\sum\limits_{k=0}^r\fr{\lambda^k}{k!}
%\label{e4-chu}
\end{equation*}
почти наверное.}

\medskip

Из теорем~1 и~2 вытекает следующая теорема. В~ней показано что,
если  множества $M_n\hm=\{0, 1,\dots, n\}$, то для <<маленьких>>~ $\alpha_{nN}$ 
справедлив аналог теоремы~${\it A}$, а для <<больших>>~$\alpha_{nN}$~--- 
аналог теоремы~${\it Б}$.

\medskip

\noindent
\textbf{Теорема 3}. \textit{Пусть $M_n=\{0, 1,\dots, n\}$, $n\in\mathbf{N}$.  
Предположим, что $n, N\to\infty$ так, что
$\alpha_{nN}\hm\to\alpha$, где $0\hm<\alpha<\infty$.}


(A)~\textit{Пусть $\alpha<\lambda$.  Тогда
$$
\lim_{n,
N\to\infty}\fr{1}{N}S_{nN}=e^{-\alpha}\sum_{k=0}^r\fr{\alpha^k}{k!}
$$
по вероятности.}

(B)~\textit{Пусть $\alpha > \lambda $.  Тогда
$$
\lim_{n,
N\to\infty}\fr{1}{N}S_{nN}=e^{-\lambda}\sum\limits_{k=0}^r\fr{\lambda^k}{k!}
$$
почти наверное. }


\medskip

\noindent
\textbf{Замечание 1.} Заметим, что $\left|({1}/{N})S_{nN}\right|\hm\le
1$. Поэтому по теореме Лебега в условиях теоремы~1
$$
\lim_{n,N\to\infty}\fr{1}{N}S_{nN}=e^{-\alpha}\sum\limits_{k=0}^r\fr{\alpha^k}{k!}B_k
$$
в пространстве $L^p(\Omega, \mathfrak{A}, \mathbf{P})$ для любого
$0\hm<p\hm<\infty$, а в условиях теоремы~2
$$
\lim_{n,N\to\infty}\fr{1}{N}S_{nN}=e^{-\lambda}\sum\limits_{k=0}^r\fr{\lambda^k}{k!}
$$
в пространстве $L^p(\Omega, \mathfrak{A}, \mathbf{P})$ для любого
$0\hm<p<\infty$.


%\bigskip

\section{ Доказательства}

%\bigskip

Для доказательства теоремы~1  потребуется следующая лемма.

\medskip

\noindent
\textbf{Лемма 1}. \textit{Пусть $0<\lambda_1<1$, $M$~--- конечное
подмножество~$\mathbf{M}$. Обозначим  $n'=\sup\{n': n'\in M\}$. Тогда
\begin{equation}
\fr{\mathbf{P}\{\pi_{\lambda}\in M\}}{\mathbf{P}\{\pi_{\lambda}=n'\}}=o(1)+\sum\limits_{k\in
M}\left(\fr{n'}{\lambda}\right)^{n'-k} \label{e5-chu}
\end{equation}
 равномерно при $n', \lambda\to\infty$ так, что
 ${n'}/{\lambda}\le\lambda_1$.}

\medskip


\noindent
Д\,о\,к\,а\,з\,а\,т\,е\,л\,ь\,с\,т\,в\,о\,.\ Имеем
\begin{multline*}
\mathbf{P}\{\pi_{\lambda}\in M\}={\mathbf{P}\{\pi_{\lambda}=n'\}}\sum\limits_{k\in
M}\fr{\lambda^kn'!}{k!\lambda^{n'}}={}\\
{}=\mathbf{P}\{\pi_{\lambda}=n'\}\sum\limits_{k\in
M}\fr{n'(n'-1)\cdots(n'-k+1)}{\lambda^{n'-k}}\,.
\end{multline*}
Так как
\begin{multline*}
\fr{n'(n'-1)\cdots
(n'-k+1)}{\lambda^{n'-k}}-\left(\fr{n'}{\lambda}\right)^{n'-k}\ge{}\\
{}\ge
-\frac{k^2}{2\lambda}\left(\fr{n'}{\lambda}\right)^{n'-k-1}\,,\enskip k<n'\,,
\end{multline*}
то
\begin{multline*}
\sum\limits_{k\in M}\left(\fr{n'}{\lambda}\right)^{n'-k}\ge\fr{\mathbf{P}
(\pi_{\lambda}\in M)}{\mathbf{P}(\pi_{\lambda}=n')}\ge\sum\limits_{k\in
M}\left(\fr{n'}{\lambda}\right)^{n'-k}-{}\\
{}-\fr{(k_0)^2}{2\lambda}\sum\limits_{k=0}^{k_0}(\lambda_1)^k
-\sum\limits_{k=k_0+1}^{\infty}(\lambda_1)^k\,,\enskip k_0<n'\,.
\end{multline*}
Это неравенство и доказывает  лемму~1.

\medskip


\noindent
\textbf{Следствие 1}. \textit{Пусть $0<\lambda_1<1$,  $M_n\subset \mathbf{M}$~-- 
такие конечные подмножества, что $M_n\subset M_{n+1}$, $n\in \mathbf{N}$, 
и $\cup_{n=1}^{\infty}M_n =\mathbf{M}$.  Тогда
\begin{equation}
\fr{\mathbf{P}\{\pi_{\lambda}\in M_n\}}{\mathbf{P}\{\pi_{\lambda}=n'_n\}}=
\fr{1}{1-{n'_n}/{\lambda}}+o(1)
\label{e6-chu}
\end{equation}
 равномерно при $n, \lambda\to\infty$ так, что
 ${n'}/{\lambda}\le\lambda_1$.}

\medskip

\noindent
Д\,о\,к\,а\,з\,а\,т\,е\,л\,ь\,с\,т\,в\,о\,.\ Так как $I_{M_n}\to I_{\mathbf{M}}$ при
 $n\to\infty$ поточечно, то 
\begin{multline*}
 \sum\limits_{k\in M_n}\left(\fr{n'_n}{\lambda}\right)^{n'-k}=\sum\limits_{k\in \mathbf{M}}
 \left(\fr{n'_n}{\lambda}\right)^{k}+
o(1)={}\\
{}=\fr{1}{1-{n'_n}/{\lambda}}+o(1)\,.
\end{multline*} 
Применяя эту оценку к
левой части~(\ref{e5-chu}), получаем~(\ref{e6-chu}). Следствие доказано.

\medskip

При $M_n=\{0, 1,\dots n\}$ из следствия 1 вытекает

\medskip

\noindent
\textbf{Следствие 2}. \textit{Пусть $0<\lambda_1<1$. Тогда
$$
\fr{\mathbf{P}\{\pi_{\lambda}\le n\}}{\mathbf{P}\{\pi_{\lambda}=n\}}=\fr{1}{1-{n}/{\lambda}}+
o(1)
$$
 равномерно при $n, \lambda\hm\to\infty$ так, что
 ${n}/{\lambda}\hm\le\lambda_1$.}

\medskip

\noindent
Д\,о\,к\,а\,з\,а\,т\,е\,л\,ь\,с\,т\,в\,о\ теоремы~1.
Пусть $0\hm\le l\hm\le r$.\linebreak
Так как $\xi_{N2}\hm+\xi_{N3}\hm+ \dots +\xi_{NN}$~---
пуассоновская случайная величина с параметром $(N-1)\lambda$, а
$\xi_{N1}\hm+\xi_{N2}\hm+ \dots \hm+\xi_{NN}$~--- пуассоновская случайная
величина с параметром~$N\lambda$, в силу леммы~1 имеем
\begin{multline*}
\mathbf{E}\fr{1}{N}S_{nN}=\mathbf{E} I_{A_{nNil}}=\mathbf{P}\{A_{nNil}\} =
\mathbf{P}\{\xi_{N1}=l\}\times{}\\[2pt]
{}\times \fr{\mathbf{P}\{
 \xi_{N2}+\xi_{N3}+ \dots +\xi_{NN}\in M_n-l\}}{\mathbf{P}\{
 \xi_{N1}+\xi_{N2}+ \dots +\xi_{NN}\in M_n\}}={}\\[2pt]
  {}=
e^{-\lambda}\fr{\lambda^l}{l!}\fr{\mathbf{P}\{\pi_{(N-1)\lambda}\in
M_n-l\}}{\mathbf{P}\{\pi_{N\lambda}\in M_n\}} ={}\\[2pt]
{}=e^{-\lambda}\fr{\lambda^l}{l!}\fr{\mathbf{P}\{\pi_{(N-1)\lambda}=
n'-l\}}{\mathbf{P}\{\pi_{N\lambda}=n'\}}\times{}
 \end{multline*}
 \begin{multline*}
\times\fr {o(1)+\sum\limits_{k\in
M_n-l}\left({n'_n}/{((N-1)\lambda)}\right)^{n'_n-k-l}}{
o(1)+\sum\limits_{k\in M_n}\left({n'_n}/{(N\lambda)}\right)^{n'_n-k}} ={}\\[1pt]
{}=\fr{n'_n!}{l!((n'_n-l)!}\left(\fr{N-1}{N}\right)^{n'_n-l}\left(\fr{1}{N}\right)^{l}\times{}\\[1pt]
{}\times
\fr {o(1)+\sum\limits_{k\in M_n-l}\left({n'_n}/{((N-1)\lambda)}\right)^{n'_n-k-l}}{
o(1)+\sum\limits_{k\in M_n}\left({n'_n}/{(N\lambda)}\right)^{n'_n-k}}\,;
\end{multline*}

\vspace*{-6pt}

\noindent
\begin{multline*}
\mathbf{E} I_{A_{nNil}}I_{A_{nNjl}}={}\\[1pt]
{}=\mathbf{P}(A_{nNil}\cap A_{nNjl}) =
\left(\mathbf{P}\{\xi_{N1}=l\}\right)^2\times{}\\[1pt]
{}\times \fr{\mathbf{P}\{
 \xi_{N3}+\xi_{N4}+ \dots +\xi_{NN}\in M_n-2l\}}{\mathbf{P}\{
 \xi_{N1}+\xi_{N2}+ \dots +\xi_{NN}\in M_n\}}={}\\[1pt]
{}=
\left(e^{-\lambda}\fr{\lambda^l}{l!}\right)^2\fr{\mathbf{P}\{
\pi_{(N-2)\lambda}\in M_n-2l\}}{\mathbf{P}\{\pi_{N\lambda}\in M_n\}} ={}\\[1pt]
{}=
\left(e^{-\lambda}\fr{\lambda^l}{l!}\right)^2\fr{\mathbf{P}\{
\pi_{(N-2)\lambda}=n'-2l\}}{\mathbf{P}\{\pi_{N\lambda}=n'_n\}}\times{}\\[1pt]
{}\times
\fr {o(1)+\sum\limits_{k\in M_n-2l}\left({n'_n}/{((N-1)\lambda)}\right)^{n'_n-k-2l}}{
o(1)+\sum\limits_{k\in M_n}\left({n'_n}/{(N\lambda)}\right)^{n'_n-k}} ={}\\[1pt]
{}
=\fr{n'_n!}{(l!)^2((n'_n-2l)!}\left(\fr{N-2}{N}\right)^{n'_n-2l}\left(\fr{1}{N}\right)^{2l}\times{}\\[1pt]
{}\times
\fr{o(1)+\sum\limits_{k\in M_n-2l}\left({n'_n}/{((N-2)\lambda)}\right)^{n'_n-k-2l}}{
o(1)+\sum\limits_{k\in M_n}\left({n'_n}/{(N\lambda)}\right)^{n'_n-k}}\,,\enskip i\ne j.
\end{multline*}
Следовательно,
\begin{align}
\mathbf{P}\{A_{nNil}\} &\to e^{-\alpha}\fr{\alpha^l}{l!}B_l \,;
\label{e7-chu}\\
\mathbf{P}(A_{nNil}\cap
A_{nNjl})&\to\left(e^{-\alpha}\fr{\alpha^l}{l!}B_l\right)^2\,,\enskip i\ne j\,, 
\label{e8-chu}
\end{align}
при $n, N\to\infty$ так, что $\alpha_{nN}\hm\to\alpha$. Заметим, что
\begin{multline*}
\mathbf{E}\left(\fr{1}{N}S_{nN}-\mathbf{E}\fr{1}{N}S_{nN}\right)^2={}\\
{}=
\fr{\mathrm{P}\{A_{nN1l}\}}{N}+\mathbf{P}\{A_{nN1l}\cap A_{nN2l}\}-{}\\
{}-\fr{\mathbf{P}\{A_{nN1l}\cap
A_{nN2l}\}}{N}-\left(\mathbf{P}\{A_{nN1l}\}\right)^2\,.
\end{multline*}
Поэтому из~(\ref{e7-chu}) и~(\ref{e8-chu}) следует, что
\begin{equation}
\mathbf{E}\left(\fr{1}{N}S_{nN}-\mathbf{E}\fr{1}{N}S_{nN}\right)^2\to 0
\label{e9-chu}
\end{equation}
при $n, N\to\infty$ так, что $\alpha_{nN}\to\alpha$. Так как 
$$
\mathbf{E} \fr{1}{N}S_{nN}=\mathbf{P}\{A_{nNil}\}\,,
$$
условия~(\ref{e9-chu}) и~(\ref{e7-chu}) влекут~(\ref{e3-chu}). 
Теорема доказана.

\medskip

\noindent
Д\,о\,к\,а\,з\,а\,т\,е\,л\,ь\,с\,т\,в\,о\ следствия теоремы~1. Пусть $0\hm\le l\hm\le r$. По
следствию~1 леммы~1
\begin{align*}
\lim_{\substack{{n, N\to\infty}\\ {\alpha_{nN}\to\alpha}}}\fr{\sum\limits_{k+l\in
M_n}\left({\alpha}/{\lambda}\right)^{n'_n-k-l}}{\sum\limits_{k\in
M_n}\left({\alpha}/{\lambda}\right)^{n'_n-k}}&=
\fr{1-{\alpha}/{\lambda}}{1-{\alpha}/{\lambda}}=1\,;
\\
\lim_{\substack{{n, N\to\infty}\\ {\alpha_{nN}\to\alpha}}}\fr{\sum\limits_{k+2l\in
M_n}\left({\alpha}/{\lambda}\right)^{n'_n-k-2l}}{\sum\limits_{k\in
M_n}\left({\alpha}/{\lambda}\right)^{n'_n-k}}&=\fr{1-{\alpha}/{\lambda}}
{1-{\alpha}/{\lambda}}=1\,.
\end{align*}
Поэтому условия~(\ref{e1-chu}) и~(\ref{e2-chu}) выполнены. Следовательно, применима
теорема~1. Доказательство закончено.

\medskip

При доказательстве теоремы~2 будем использовать следующие леммы.

\medskip

\noindent
\textbf{Лемма 2}. (A)~\textit{Справедлива оценка
\begin{multline}
\!\!\!\mathbf{P}\{\pi_{\lambda}>n\}\le\left(\!1+O\left(\fr{1}{n}\right)\!\right)e^{-n\sum_{k=3}^{\infty}
({1}/{k})\left(1-{\lambda}/{n}\right)^k}\times{}\\
{}\times
\left(\Phi(\sqrt{n})-\Phi\left(\sqrt{n}\left(1-\fr{\lambda}{n}\right)\right)\right)
\label{e10-chu}
\end{multline}
при   ${\lambda}/{n}\le 1$.}

\smallskip

(Б)~\textit{Справедлива оценка
\begin{multline}
0\le \mathbf{P}\{\pi_{N\lambda}>n\}-\mathbf{P}\{\pi_{(N-l)\lambda}>n\}\le{}\\
{}\le
\left(1+O\left(
\fr{1}{n}\right)\right)e^{-n\sum\limits_{k=2}^{\infty}(1/k)\left(1-{N\lambda}/{n}\right)^k}
\fr{\lambda l}{\sqrt{2\pi n}} 
\label{e11-chu}
\end{multline}
при   ${N\lambda}/{n}\le 1$.}

\medskip

\noindent
Д\,о\,к\,а\,з\,а\,т\,е\,л\,ь\,с\,т\,в\,о\,.\ (A)~В~силу формулы Тейлора с остаточным
членом в интегральной форме (см., например,~[5, c.~161]) для функции
$f(x)=e^x$ с последующей заменой  $t\hm=\lambda-y$  получаем
пред\-став\-ле\-ние

\noindent
\begin{multline}
\mathbf{P}\{\pi_{\lambda}>n\}=e^{-\lambda}\fr{1}{n!}
\int\limits_0^{\lambda}e^y(\lambda-y)^n\,dy={}\\
{}=
\fr{1}{n!}\int\limits_0^{\lambda}e^{-(\lambda-y)}(\lambda-y)^n\,dy=
\fr{1}{n!}\int\limits_0^{\lambda}e^{-t}t^n\,dt.\label{e12-chu}
\end{multline}
 Поэтому, используя замену
$t\hm=ny$, а затем замену $x\hm=\sqrt{n}(1-y)$, формулу Стирлинга и
разложение логарифма в ряд Тейлора при оценке правой час\-ти~(\ref{e12-chu}),
получаем

\noindent
\begin{multline*}
\mathbf{P}\{\pi_{\lambda}>n\}=\fr{1}{n!}\int\limits_0^{\lambda}e^{-t}t^n\,dt={}\\
{}=
\left(1+O\left(\fr{1}{n}\right)\right)
\fr{e^n}{\sqrt{2\pi n}n^n}\int\limits_0^{{\lambda}/{n}}e^{-ny}(ny)^n\,d(ny)={}\\
{}=
\left(1+O\left(\fr{1}{n}\right)\right) \sqrt{\fr{n}{2\pi}}
\int_0^{{\lambda}/{n}}e^{n(1-y)}y^n\,dy={}\\
{}=\left(1+O\left(\fr{1}{n}\right)\right)
\sqrt{\fr{n}{2\pi}}
\int\limits_0^{{\lambda}/{n}}e^{n(1-y)}e^{n\ln(y)}\,dy={}\\
{}=
\left(1+O\left(\fr{1}{n}\right)\right) \sqrt{\fr{n}{2\pi}}\times{}\\
{}\times
\int_0^{{\lambda}/{n}}e^{n(1-y)}e^{n\ln(1-(1-y))}\,dy={}\\
{}=
\left(1+O\left(\fr{1}{n}\right)\right) \sqrt{\fr{n}{2\pi}}\times{}\\
{}\times
\int\limits_0^{{\lambda}/{n}}e^{-n{(1-y)^2}/2}
e^{-n\sum\limits_{k=3}^{\infty}{(1-y)^k}/{k}}\,dy
\le {}\\
{}\le-\left(1+O\left(\fr{1}{n}\right)\right)
\fr{1}{\sqrt{2\pi}}e^{-n\sum\limits_{k=3}^{\infty}(1/k)\left(1-{\lambda}/{n}\right)^k}\times{}\\
{}\times
\int\limits_{\sqrt{n}}^{\sqrt{n}\left(1-{\lambda}/{n}\right)}e^{-{x^2}/{2}}\,dx={}\\
{}=
\left(1+O\left(\fr{1}{n}\right)\right)e^{-n\sum\limits_{k=3}^{\infty}({1}/{k})\left(1-{\lambda}/{n}
\right)^k}\times{}\\
{}\times
\left(\Phi\left(\sqrt{n}\right)-\Phi\left(\sqrt{n}\left(1-\fr{\lambda}{n}\right)\right)\right)\,.
\end{multline*}
Поэтому справедливо~(\ref{e10-chu}). Пункт~(A) доказан.

Доказательство п.~(Б) повторяет доказательство п.~(А) с
той лишь разницей, что вместо оценивания интеграла
$\int\limits_0^{\lambda}e^{-t}t^n\,dt$  оценивается интеграл
$\int\limits_{(N-1)\lambda}^{N\lambda}e^{-t}t^n\,dt$.

В частности, при $\lambda=n$ получаем

\smallskip

\noindent
\textbf{Следствие 1}. \textit{Справедливо неравенство 
$$
\lim_{n\to\infty}{\bf P}\{\pi_{n}>n\}\le\fr{1}{2}\,.
$$}


\medskip

При доказательстве теоремы~2  будем использовать следующее обобщение
следствия~1:

\smallskip

\noindent
\textbf{Следствие 2}. \textit{Справедливо неравенство 
$$
\limsup\limits_{\lambda,n\to\infty, {\lambda}/{n}\le 1}\mathbf{P}\{
\pi_{\lambda}>n\}\le\fr{1}{2}\,.
$$
}

\smallskip


\noindent
\textbf{Замечание~2.} В~работах В.\,М.~Круглова~\cite{7-chu, 8-chu} получены оценки
хвостов безгранично делимых и пуассоновских распределений. Оценки,
полученные в лемме~2, можно рассматривать как уточнение оценок,
полученных Кругловым для пуассоновских распределений. Приведем
некоторые аналоги неравенств~(\ref{e10-chu}) и~(\ref{e11-chu}).

(В) Используя элементарные неравенства
\begin{align*}
-\sum\limits_{k=3}^{\infty}\fr{1}{k}\,(1-x)^k&\le \fr{1}{3}\,(1-x)^2\ln(x)\,;\\
-\sum\limits_{k=2}^{\infty}\fr{1}{k}\,(1-x)^k&\le \fr{1}{2}\,(1-x)\ln(x)
\end{align*}
при оценке правых частей в~(\ref{e10-chu}) и~(\ref{e11-chu}) соответственно, получаем
\begin{multline*}
\!\!\!\mathbf{P}\{\pi_{\lambda}>n\}\le\left(1+O\left(\fr{1}{n}\right)\right)
e^{({1}/{3})n\left(1-{\lambda}/{n}\right)^2\ln\left(\lambda/n\right)}\times{}\\
{}\times
\left(\Phi(\sqrt{n})-\Phi\left(\sqrt{n}\left(1-\fr{\lambda}{n}\right)\right)\right)
\end{multline*}
при   ${\lambda}/{n}\le 1$;
\begin{multline*}
0\le \mathbf{P}\{\pi_{N\lambda}>n\}-\mathbf{P}\{\pi_{(N-l)\lambda}>n\}\le{}\\
{}\le
\left(1+O\left(\fr{1}{n}\right)\right)e^{({1}/{2})n\left(1-{N\lambda}/{n}\right)
\ln\left({N\lambda}/{n}\right)}
\fr{\lambda l}{\sqrt{2\pi n}}
\end{multline*}
при   ${N\lambda}/{n}\le 1$.

(Г)~Функция $y\hm=e^{-t}t^n$ возрастает на интервале $(0, n)$ и
убывает на интервале  $(n, \infty)$. Заметим, что
$\lambda=n({\lambda}/{n})\le n$ при ${\lambda}/{n}\le 1$.
Поэтому, используя формулу Стирлинга и элементарное неравенство $e x
e^{-x}\le e^{-{(1-x)^2}/{2}}$, $0\hm\le x\hm\le 1$, при
$x\hm={\lambda}/{n}$ по\-лу\-чаем
\begin{multline*}
\mathbf{P}\{\pi_{\lambda}>n\}=\fr{1}{n!}\int\limits_0^{\lambda}e^{-t}t^n\le{}\\
{}\le\fr{1}{\sqrt{2\pi
n}}\,\fr{e^n}{n^n}\,e^{-\lambda}\lambda^n\lambda\left(1+O\left(\fr{1}{n}\right)\right)={}\\
{}=\fr{1}{\sqrt{2\pi}}\,\sqrt{n}\,\fr{\lambda}{n}\left(e\fr{\lambda}{n}e^{-{\lambda}/{n}}\right)^n
\left(1+O\left(\fr{1}{n}\right)\right)
\le{}\\
{}\le \fr{1}{\sqrt{2\pi}}\, \sqrt{n}\,\fr{\lambda}{n}e^{-({n}/{2})\left(1-{\lambda}/{n}\right)^2}
\!\left(\!1+O\left(\fr{1}{n}\right)\right),\
\fr{\lambda}{n}\le 1.\hspace*{-4.80296pt}
\end{multline*}


\smallskip

\noindent
\textbf{Замечание 3.} Пусть случайные величины $\pi_{ni}$, $1\hm\le i\hm\le
n$, $n\in\mathbf{N}$,~--- независимые копии случайной величины~$\pi_{\lambda}$. Так как
\begin{equation}
\fr{1}{n}\sum_{i=1}^n\pi_{ni}\stackrel{d}{=}\fr{\pi_{n\lambda}}{n}\,,
\label{e13-chu}
\end{equation}
то, используя представление четвертого момента в виде четвертой
производной от характеристической функции, получаем
\begin{multline*}
\mathbf{E}\left|\fr{1}{n}\sum_{i=1}^n\pi_{ni}-\lambda\right|^4={}\\
{}=
\left.\left(e^{-n\lambda\left(1+i({t}/{n})-e^{i({t}/{n})}\right)}\right)^{(4)}
\right\vert_{t=0}=\fr{3n^2\lambda^2+n\lambda}{n^4}\,.
\end{multline*}
Поэтому, используя неравенство чебышевского типа для четвертых
моментов, приходим к
\begin{multline*}
\sum\limits_{n=1}^{\infty}\mathbf{P}\left\{\left|
\fr{1}{n}\sum_{i=1}^n\pi_{ni}-\lambda\right|>\varepsilon\right\}\le{}\\
{}\le
\fr{1}{\varepsilon^4}\sum\limits_{n=1}^{\infty}\mathbf{E}
\left\vert\fr{1}{n}\sum\limits_{i=1}^n\pi_{ni}-\lambda\right\vert^4<\infty
\,\,\,\forall\,\,\, \varepsilon>0\,.
\end{multline*}
Следовательно,
$$
\fr{1}{n}\sum\limits_{i=1}^n\pi_{ni}\to\lambda\,,\quad
n\to\infty\,,
$$
почти наверное. Равенство~(\ref{e13-chu}) позволяет использовать  леммы~1 и~2
для оценки  скорости сходимости в этом усиленном законе больших
чисел. Пусть случайная величина $X_{\lambda}\hm\equiv\lambda$. При
$k\in\mathbf{N}$, $k\hm\ne \lambda$, имеем
\begin{multline}
\Bigg|\mathbf{P}\left\{\frac{1}{n}\sum_{i=1}^n\pi_{ni}\le k\Bigg\}-\mathbf{P}
\Bigg\{X_{\lambda}\le k\Bigg\} \right|\le{}\\
\le
\begin{cases}
\fr{1}{\sqrt{2\pi nk}}\left(e\fr{\lambda}{k}\,e^{-{\lambda}/{k}}\right)^{nk}
\left(1-\fr{k}{\lambda}\right)^{-1}\times{}\\
\hspace{10mm}{}\times(1+o(1)) \quad \mbox{\ при\ } 0<k<\lambda\,;\\
\fr{1}{\sqrt{2\pi}} \sqrt{nk}\,\fr{\lambda}{k}\left(e\fr{\lambda}{k}\,e^{-{\lambda}/{k}}\right)^{nk}\times{}\\
\hspace*{7mm}{}\times \left(1+O\left(\fr{1}{nk}\right)\right)\quad
\mbox{\ при\ } k>\lambda\,.
\end{cases}
\label{e14-chu}
\end{multline}
Оценка~(\ref{e14-chu}) при $0\hm<k\hm<\lambda$ вытекает из следствия~2 леммы~1, в
котором вместо~$\lambda$ используется $n\lambda$, а вместо~$n$
используется $kn$, с последующей оценкой $(kn)!$ с помощью формулы
Стирлинга. Оценка~(\ref{e14-chu}) при $\lambda\hm<k$ вытекает из замечания~2(Б). 
Заметим, что
$e({\lambda}/{k})e^{-{\lambda}/{k}}\hm<1$, $k\hm\ne\lambda$. Поэтому
правая часть в~(\ref{e14-chu}) стремится к нулю при $n\to\infty$ при любом
$k\hm\ne\lambda$.

\medskip

\noindent
\textbf{Лемма 3}. \textit{Пусть множества $M_n=\{0, 1,\dots, n\}$,
$n\hm\in\mathbf{N}$, $\alpha \hm >\lambda $,
$p_r\hm=e^{-\lambda}\sum\limits_{k=0}^r({\lambda^k}/{k!}) $.  Тогда
$$
\limsup\limits_{n, N\to\infty,
\lambda\le\alpha_{nN}\le\alpha}\fr{|S_{nN}-
\mathbf{E}S_{nN}|}{\sqrt{N\ln(N)}}\le 16\sqrt{2}\,p_r(1-p_r)
$$
почти наверное.}

\medskip

\noindent
Д\,о\,к\,а\,з\,а\,т\,е\,л\,ь\,с\,т\,в\,о\,.\ Так как $\xi_{N1}\hm+\xi_{N2}\hm+ \dots +\xi_{NN}$~--- 
пуассоновская случайная величина с параметром $N\lambda$  и
${N\lambda}/{n}\hm\le 1 $ при $\lambda\hm\le\alpha_{nN}\hm\le\alpha$, по
следствию~2 леммы~2
$$
\liminf_{n, N\to\infty,
\lambda\le\alpha_{nN}\le\alpha} \mathbf{P}\{
 \xi_{N1}+\xi_{N2}+ \dots +\xi_{NN}\in M_n\}\ge\fr{1}{2}\,.
$$
Поэтому доказательство леммы~3 повторяет доказательство теоремы~2 из~\cite{6-chu}.

\medskip

\noindent
Д\,о\,к\,а\,з\,а\,т\,е\,л\,ь\,с\,т\,в\,о\ теоремы~2. Согласно лемме~3, 
при $n, N\to\infty$ так, что
$\lambda\hm\le\alpha_{nN}\hm\le \alpha$,
 последовательность $({S_{nN}-\mathbf{E}S_{nN}})/{\sqrt{N\ln(N)}}$
ограничена почти наверное. Поэтому $({S_{nN}-\mathbf{E}S_{nN}})/N\hm\to 0$ 
при $n, N\hm\to\infty$ так, что
$\lambda\hm\le\alpha_{nN}\hm\le\alpha$ почти наверное. Так как
\begin{align*}
\fr{1}{N}\,S_{nN}&=\fr{S_{nN}-\mathbf{E}S_{nN}}{N}+\fr{1}{N}\,\mathbf{E}S_{nN}\,;\\
\fr{1}{N}\,\mathbf{E}S_{nN}&=\mathbf{P}(A_{nNi})\,,
\end{align*}
 то для завершения
доказательства теоремы достаточно показать, что
\begin{equation}
\lim_{n, N\to\infty,
\lambda\le\alpha_{nN}\le\alpha}\mathbf{P}(A_{nNi})=
e^{-\lambda}\sum_{k=0}^r\fr{\lambda^k}{k!}\,. \label{e15-chu}
\end{equation}

Пусть $0\le l\le r$. Заметим, что
\begin{multline}
\mathbf{P}\{A_{nNil}\} ={}\\
{}=\fr{\mathbf{P}\{\xi_{N1}=l\}\mathbf{P}\{
 \xi_{N2}+\xi_{N3}+ \dots +\xi_{NN}\le n-l\}}{\mathbf{P}\{
 \xi_{N1}+\xi_{N2}+ \dots +\xi_{NN}\le n\}}={}\\
 {}=
e^{-\lambda}\fr{\lambda^l}{l!}\frac{1-\mathbf{P}
\{\pi_{(N-l)\lambda}>n\}}{1-\mathbf{P}\{\pi_{N\lambda}>n\}}={}\\
{}= 
e^{-\lambda}\fr{\lambda^l}{l!}\left(1+\fr{\mathbf{P}
\{\pi_{N\lambda}>n\}-\mathbf{P}\{\pi_{(N-1)\lambda}>n\}}{1-\mathbf{P}
\{\pi_{N\lambda}>n\}}+{}\right.\\
\left.{}+\fr{\mathbf{P}\{\pi_{(N-1)\lambda}>n\}-\mathbf{P}
\{\pi_{(N-1)\lambda}>n-l\}}{1-\mathbf{P}\{\pi_{N\lambda}>n\}}\right)\,.
\label{e16-chu}
\end{multline}
Используя формулу   Стирлинга и элементарное неравенство
$exe^{-x}\le 1$, $0\le x<\infty$, получаем
\begin{multline*}
0\le \mathbf{P}\{\pi_{(N-1)\lambda}>n-l\}-\mathbf{P}\{
\pi_{(N-1)\lambda}>n\}\le{}\\
{}\le
 e^{-\lambda} l\left(\fr{\alpha}{\lambda}\right)^l\mathbf{P}\{\pi_{N\lambda}=n\}={}\hspace*{10mm}
\end{multline*}
 
\noindent
\begin{multline}
{}=
\left(1+O\left(\frac{1}{n}\right)\right) e^{-\lambda}
l\left(\fr{\alpha}{\lambda}\right)^l\left(e
e^{-{N\lambda}/{n}}\fr{N\lambda}{n}\right)^n\times{}
\\
{}\times
\fr{1}{\sqrt{2\pi n}}\le
\left(1+O\left(\fr{1}{n}\right)\right)
e^{-\lambda} l\left(\fr{\alpha}{\lambda}\right)^l
\fr{1}{\sqrt{2\pi n}}\,. \label{e17-chu}
\end{multline}
Из~(\ref{e16-chu}) и~(\ref{e17-chu})
  по  лемме~2 и ее следствию~2 вытекает, что  
  $ \mathbf{P}\{A_{nNil}\}\to e^{-\lambda} ({\lambda^l}/{l!})$ при $n,
N\hm\to\infty$ так, что $\lambda\le\alpha_{nN}\hm\le\alpha$. Условие~(\ref{e15-chu})
выполнено. Это завершает доказательство теоремы.

\medskip

\noindent
Д\,о\,к\,а\,з\,а\,т\,е\,л\,ь\,с\,т\,в\,о\  теоремы~3. (А)~Так как из сходимости в
среднем квадратичном следует сходимость по вероятности и множества
$M_n\hm=\{0, 1,\dots, n\}$, $n\in\mathbf{N}$, удовлетворяют условиям
следствия теоремы~1, то п.~(А) теоремы~3 вытекает из следствия
теоремы~1.

(Б) Пусть $0<\varepsilon\hm<\alpha\hm-\lambda$. Можно считать, что
найдется $n_0\hm\in\mathbf{N}$ со следующим свойством:
$|\alpha_{nN}\hm-\alpha|<\varepsilon$ при $N,n>n_0$. Тогда  при
$N,n\hm>n_0$ справедливо неравенство
$\lambda\hm\le\alpha_{nN}\hm\le\alpha\hm+\varepsilon$. Следовательно,
применима теорема~2. Доказательство п.~(Б) закончено.

\bigskip

Авторы благодарят профессора В.\,М.~Круглова за ценную информацию и
профессора В.\,Ф.~Колчина за ценное замечание.

  {\small\frenchspacing
{%\baselineskip=10.8pt
\addcontentsline{toc}{section}{Литература}
\begin{thebibliography}{9}

\bibitem{1-chu}
\Au{Питерсон У., Уэлдон Э.} Коды, исправляющие ошибки.~--- М.:
Мир, 1976. 596~с.

\bibitem{2-chu}
\Au{Колчин В.\,Ф. } Один класс предельных теорем для условных
распределений~// Литовский математический сборник, 1968. T.~8. №\,1. C.~53--63.


\bibitem{3-chu}
\Au{Колчин В.\,Ф., Севастьянов Б.\,А., Чистяков~В.\,П.} Случайные
размещения.~--- М.: Физматгиз, 1976.  223~с.

\bibitem{4-chu}
\Au{Chuprunov A.\,N., Fazekas~I.}  Inequality and strong law of
large numbers for random allocations~// Acta Math. Hungar., 2005.
Vol.~109. No.\,1--2. P.~163--182.

\bibitem{5-chu}
\Au{Фихтенгольц Г.\,М.} Курс дифференциального и интегрального
исчисления.~---  М.: Физматлит, 2006.  864~с.


\bibitem{7-chu} %6
\Au{Круглов В.\,М.}  Характеризация одного класса безгранично
делимых распределений~// Матем. заметки, 1974. T.~16. №\,5. C.~777--782.


\bibitem{8-chu} %7
\Au{Круглов В.\,М.}  Новая характеризация пуассоновских
распределений~// Матем. заметки, 1976. T.~20. №\,6. C.~879--882.

  \label{end\stat}

\bibitem{6-chu} %8
\Au{Чупрунов А.\,Н., Фазекаш~И.} Законы повторного логарифма для
числа безошибочных блоков при помехоустойчивом кодировании~// Информатика и её 
применения, 2010. T.~4. №\,3. C.~42--46.

 \end{thebibliography}
}
}


\end{multicols}       