
\def\stat{korolev}

\def\tit{О НЕРАВЕНСТВАХ ТИПА БЕРРИ--ЭССЕЕНА ДЛЯ~ПУАССОНОВСКИХ
СЛУЧАЙНЫХ СУММ$^*$}

\def\titkol{О неравенствах типа Берри--Эссеена для пуассоновских
случайных сумм}

\def\autkol{В.\,Ю. Королев, И.\,Г.~Шевцова, С.\,Я. Шоргин}
\def\aut{В.\,Ю. Королев$^1$, И.\,Г.~Шевцова$^2$, С.\,Я. Шоргин$^3$}

\titel{\tit}{\aut}{\autkol}{\titkol}

{\renewcommand{\thefootnote}{\fnsymbol{footnote}}\footnotetext[1]
{Работа поддержана Российским фондом
фундаментальных исследований (проекты 11-01-00515а, 11-07-00112а и
11-01-12026-офи-м),
Федеральной целевой программой <<Научные и научно-педагогические
кадры инновационной России на 2009--2013~годы>> и грантом Президента
РФ МК--581.2010.1.}}

\renewcommand{\thefootnote}{\arabic{footnote}}
\footnotetext[1]{Московский государственный 
университет им.\ М.\,В.~Ломоносова, факультет вычислительной математики и кибернетики; Институт проблем информатики Российской
академии наук, vkorolev@cs.msu.su}
\footnotetext[2]{Московский государственный 
университет им.\ М.\,В.~Ломоносова, 
факультет вычислительной математики и кибернетики; Институт проблем информатики
Российской академии наук, ishevtsova@cs.msu.su}
\footnotetext[3]{Институт проблем информатики Российской академии наук,
sshorgin@ipiran.ru}

\Abst{Для равномерного расстояния между функциями
распределения~$\Phi(x)$ стандартной нормальной случайной величины и~$F_\lambda(x)$ 
пуассоновской случайной суммы независимых одинаково
распределенных случайных величин $X_1,X_2,\ldots$ с конечным третьим
абсолютным моментом, где $\lambda>0$~--- параметр пуассоновского
индекса, доказано неравенство
\begin{equation*}
\sup\limits_{x}|F_\lambda(x)-\Phi(x)|\leqslant 0{,}4532\fr{\e|X_1-\e
X_1|^3}{(\D X_1)^{3/2}\sqrt{\lambda}}\,,\quad \la>0\,,
\end{equation*}
типа оценки Берри--Эссеена, использующее центральные моменты, в
отличие от ранее известных аналогичных неравенств, использующих
начальные моменты.}

\KW{пуассоновская случайная сумма; центральная
предельная теорема; оценка скорости сходимости; неравенство
Берри--Эссеена; абсолютная константа}

  \vskip 10pt plus 9pt minus 6pt

      \thispagestyle{headings}

      \begin{multicols}{2}
      
            \label{st\stat}


\section{Введение}

Пусть $X_1,X_2,\ldots$~--- независимые одинаково распределенные
случайные величины (с.\,в.)~с $\e|X_1|^3\hm <\infty$. Пусть $N_\la$~---
с.\,в., имеющая пуассоновское распределение с параметром $\la>0$.
Предположим, что при каждом $\lambda\hm>0$ с.\,в.\ $N_\la,X_1,X_2$
независимы. Случайная величина
\begin{equation*}
S_\la = X_1+\ldots+X_{N_\la}
\end{equation*}
называется пуассоновской случайной суммой (для определенности
полагаем $S_\la=0$, если $N_\la=0$). Пуассоновские случайные суммы
$S_{\lambda}$ являются весьма популярными математическими моделями
многих реальных объектов. В~частности, в страховой математике
величина~$S_{\lambda}$ описывает суммарное страховое требование в
классическом процессе риска в <<динамическом>> случае. Многие примеры
прикладных задач из самых разнообразных областей, в которых
используются пуассоновские случайные суммы, приведены, скажем, в
книгах~\cite{GnedenkoKorolev1996, BeningKorolev2002}.

Как известно, при указанных выше условиях на моменты слагаемых
распределения пуассоновских случайных сумм~$S_\la$ асимптотически
нормальны, что обусловливает большую важность задачи изуче\-ния
точности нормальной аппроксимации для распределений пуассоновских
случайных сумм.

Функцию распределения (ф.\,р.)\ стан\-дар\-ти\-зо\-ван\-ной пуассоновской
случайной суммы
\begin{equation*}
\widetilde S_\la=\fr{S_\la-\e S_\la}{\sqrt{\D S_\la}}=
\fr{S_\la- \la\e X_1}{\sqrt{\la\e X_1^2}}
\end{equation*}
обозначим $F_\la(x)$. Также введем обозначения:
\begin{align*}
L_0&=L_0(X_1)=\fr{\e|X_1-\e X_1|^3}{(\D X_1^2)^{3/2}}\,;\\
L_1&=L_1(X_1)=\fr{\e|X_1|^3}{(\e X_1^2)^{3/2}}\,.
\end{align*}
Величины $L_0$ и $L_1$ называются соответственно центральной и
нецентральной ляпуновскими дробями. Пусть $\Phi(x)$~--- ф.\,р.\
стандартного нормального закона.

Как известно, изначально оценки точности нормальной аппроксимации для
распределений пуассоновских случайных сумм доказывались в терминах
центральных ляпуновских дробей~$L_0$ по\linebreak аналогии с оценками для сумм
неслучайного числа независимых с.\,в. Результаты такого типа можно
найти, например, в~\cite{KruglovKorolev1990}. Однако, как было
показано позднее, при изучении нормальной аппроксимации для
распределений пуассоновских случайных сумм более естественно
использовать нецентральные ляпуновские дроби~$L_1$.

В частности, в работе~\cite{Michel1993} показано, что если для
абсолютной константы~$C$ в классическом неравенстве Берри--Эссеена
известна оценка $C\leqslant M$, то та же самая оценка справедлива для
абсолютной константы~$C_1$ в аналоге неравенства Берри--Эссеена для
пуассоновских случайных сумм, использующем нецентральную ляпуновскую
дробь:
\begin{equation}
\Delta_\la\equiv \sup\limits_x|F_\la(x)-\Phi(x)|\leqslant
C_1\fr{L_1(X_1)}{\sqrt{\la}}\,,\quad \la>0\,,
\end{equation}
при этом $C_1\leqslant M$. Тот же результат был независимо получен в
работе~\cite{KorolevShorgin1997}, но с более точной оценкой~$M$. В~2010~г.\ 
в работе~\cite{KorolevShevtsova2010DAN} было показано, что
привяз-\linebreak ка константы $C_1$ к классическому неравенству Бер\-ри--Эс\-се\-ена
на самом деле менее жесткая, и было впервые продемонстрировано, что
верхняя оценка константы~$C_1$ меньше теоретически наименьшего
возможного значения $(\sqrt{10}+3)/(6\sqrt{2\pi})=0{,}4097\ldots$\
абсолютной константы~$C$ в классическом неравенстве Бер\-ри--Эссеена.
Позднее с помощью модификации метода, использованного
в~\cite{KorolevShevtsova2010DAN}, авторы указанной статьи показали,
что $C_1\leqslant0{,}3041$~\cite{KorolevShevtsova2010OPPM, KorolevShevtsova2010SAJ}.

Необходимо отметить, что, несмотря на то что верхние оценки константы
$C_1$ в~(1) изучаются уже более 30~лет (см.\ исторические обзоры
в~\cite{KorolevShevtsova2010OPPM, KorolevShevtsova2010SAJ}), нижние
оценки для $C_1$ получены лишь недавно в
работе~\cite{NefedovaShevtsova2010}, где, в частности, было показано,
что $C_1\geqslant0{,}2344$.

В 1996~г.\ в работе~\cite{Shorgin1996} было показано, что для любой
с.\,в.~$X$ с $\e|X|^3<\infty$ имеет место соотношение:
\begin{equation*}
\fr{L_1(X)}{L_0(X)}\leqslant 2\sqrt{2}<2{,}8285\,,
\end{equation*}
откуда с учетом результатов~\cite{KorolevShevtsova2010OPPM, KorolevShevtsova2010SAJ} 
вытекает, что для абсолютной константы $C_0$
в аналоге неравенства Берри--Эссеена для пуассоновских случайных
сумм, использующем {\it центральные} ляпуновские дроби,
\begin{equation}
\Delta_\la \leqslant C_0\fr{L_0(X_1)}{\sqrt{\la}}\,,\quad \la>0\,,
\end{equation}
справедлива оценка $C_0\leqslant 0{,}3041\cdot2\sqrt{2}<0{,}8602$.

Более того, в 2001~г.\ в работе~[11] было высказано
предположение, что
\begin{equation*}
\sup_X \fr{L_1(X)}{L_0(X)}=
\fr{\sqrt{17+7\sqrt{7}}}{4}={1{,}48997\ldots} <1{,}49\,,
\end{equation*}
где супремум берется по всем распределениям с.\,в.~$X$ с
${\e|X|^3<\infty}$, и было описано гипотетическое экстремальное
распределение с.\,в.~$X$.

В данной работе будет показано, что $L_1(X)/L_0(X)\leqslant 1{,}49$ для любой
с.\,в.~$X$ с $\e|X|^3<\infty$, откуда вытекает, что
\begin{equation*}
C_0<0{,}3041\cdot1{,}49<0{,}4532\,,
\end{equation*}
что строго меньше наилучшей известной верхней оценки $C\leqslant0{,}4784$
абсолютной константы в классическом неравенстве Берри--Эссеена
(см.~\cite{KorolevShevtsova2010OPPM, KorolevShevtsova2010SAJ}).

\section{Основные результаты}

\noindent
\textbf{Теорема 1.}
\textit{Для любой с.\,в.~$X$ с $\e|X|^3<\infty$ справедливо неравенство}
\begin{equation*}
L_1(X)\leqslant 1{,}49 L_0(X)\,.
\end{equation*}

\smallskip

\noindent
Д\,о\,к\,а\,з\,а\,т\,е\,л\,ь\,с\,т\,в\,о\,.\ 
Обозначим
\begin{equation*}
J(X)=\fr{L_1(X)}{L_0(X)}\,.
\end{equation*}
Тогда утверждение теоремы эквивалентно тому, что $\sup\limits_X
J(X)\leqslant1{,}49$, где супремум берется по всем с.\,в.~$X$ с
$\e|X|^3<\infty$. Если $\e X=0$, то, очевидно, $L_0(X)=L_1(X)$ и 
утверждение теоремы верно. Пусть теперь $\e
X\neq0$. С учетом инвариантности ляпуновских дробей $L_0(X)$ и
$L_1(X)$ относительно преобразований масштаба имеем
\begin{multline*}
\sup\limits_{X}J(X) = \sup\limits_{a\neq0}\ \sup\limits_{X\colon\e X=a}J(X) =
\sup\limits_{X\colon \e X=1}J(X)={}\\
{}=\sup\limits_{X\colon \e X=0}\fr{L_1(X+1)}{L_0(X)}={}\\
{}=\sup\limits_{b>0}\
\sup\limits_{X\colon \e X=0,\,\e X^2=b^2} \fr{\e|X+1|^3}
{(1+b^{-2})^{3/2}\e|X|^3}\,.
\end{multline*}
Таким образом, утверждение теоремы эквивалентно тому, что
\begin{equation*}
\sup\limits_{b>0}\sup\limits_{X\colon \e X=0,\,\e X^2=b^2}J_b(X)\leqslant0\,,
\end{equation*}
где
\begin{equation*}
J_b(X)={\e|X+1|^3} -1{,}49(1+b^{-2})^{3/2}\e|X|^3\,.
\end{equation*}
Из результатов работ~[12--14]
вытекает, что экстремум функционала моментного типа, линейного по
функции распределения случайной величины~$X$, при двух линейных
ограничениях $\e X=0$ и $\e X^2=b^2$ моментного типа достигается на
некотором трехточечном распределении. Пусть
\begin{align*}
{\sf P}(X=x)&=p\,;\\
{\sf P}(X=y)&=q\,;\\
{\sf P}(X=z)&=1-p-q\,,
\end{align*}
где $p,q\geqslant0$, $p+q\leqslant1$, $x\leqslant y\leqslant z$. Вычисляем

\pagebreak

\noindent
\begin{align*}
\e X&= p(x-z)+q(z-y)+z\,;
\\[9pt]
\e X^2&=p(x^2-z^2)+q(y^2-z^2)+z^2\equiv{}\\[3pt]
&\hspace*{25mm}{}\equiv g_1(x,y,z,p,q)\,;
\\[9pt]
\e|X|^3&= p(|x|^3-|z|^3)+q(|y|^3-|z|^3)+|z|^3\equiv{}\\[6pt]
&\hspace*{25mm} {}\equiv g_2(x,y,z,p,q)\,;\\[9pt]
\D X&=p(1-p)(z-x)^2+q(1-q)(z-y)^2-{}\\[6pt]
&{}-2pq(z-x)(z-y)\equiv g_3(x,y,z,p,q)\,;
\\[9pt]
\e|X-\e X|^3&=-p(q(z-y)-(1-p)(z-x))^3+{}\\[6pt]
&\hspace*{11mm}{}+q|p(z-x)-(1-q)(z-y)|^3+{}
\\[6pt]
&\hspace*{3mm}{}+ (1-p-q)(p(z-x)-q(z-y))^3\equiv{}\\[6pt]
&\hspace*{30mm}{}\equiv g_4(x,y,z,p,q)\,;
\end{align*}

\vspace*{-9pt}

\noindent
\begin{multline*}
J(X)=\fr{g_2(x,y,z,p,q)}{g_4(x,y,z,p,q)}
\left(\fr{g_3(x,y,z,p,q)}{g_1(x,y,z,p,q)}\right)^{3/2}\equiv{}\\
\equiv g(x,y,z,p,q)\,,
\end{multline*}
причем в силу инвариантности $J(X)$ относительного масштабного
преобразования~$X$ без ограничения общности можно считать, что $-1\leqslant
x\leqslant y\leqslant z\leqslant 1$. Таким образом,
\begin{multline*}
\sup\limits_{X}J(X)=\sup\left.\{g(x,y,z,p,q)\colon p,q\geqslant0\,,\right.\\
\left.p+q\leqslant1,\ -1\leqslant x\leqslant
y\leqslant z\leqslant 1 \right\}\,.
\end{multline*}
Численная оптимизация непрерывной функции $g(x,y,z,p,q)$ пяти
аргументов показывает, что ее максимальное значение на описанном
компакте не превосходит 1,49, что и доказывает теорему.\hfill$\Box$


%\bigskip
\bigskip
\bigskip

\noindent
\textbf{Теормема 2.}
\textit{При условиях, сформулированных выше, для любого $\la>0$
справедливо неравенство}
\begin{equation}
\Delta_\la\leqslant 0{,}4532 \fr{L_0}{\sqrt{\la}}\,.
\end{equation}

\bigskip
\bigskip

\noindent
Д\,о\,к\,а\,з\,а\,т\,е\,л\,ь\,с\,т\,в\,о\,.
Учитывая результат работы~\cite{KorolevShevtsova2010SAJ} и
теорему~1, получаем
\begin{multline*}
\Delta_\la\leqslant 0{,}3041 \fr{L_1(X_1)}{\sqrt{\la}}\leqslant{}\\[6pt]
{}\leqslant
0{,}3041
\sup\limits_X\fr{L_1(X)}{L_0(X)}\,\fr{L_0(X_1)}{\sqrt{\la}}\leqslant{}\\[6pt]
{}\leqslant
 0{,}3041 \cdot 1{,}49\fr{L_0(X_1)}{\sqrt{\la}} <
0{,}4532\fr{L_0(X_1)}{\sqrt{\la}}\,.
\end{multline*}

%\columnbreak


{\small\frenchspacing
{%\baselineskip=10.8pt
\addcontentsline{toc}{section}{Литература}
\begin{thebibliography}{99}

\bibitem{GnedenkoKorolev1996} %1
\Au{Gnedenko B.\,V., Korolev~V.\,Yu.} Random summation: Limit theorems
and applications.~--- Boca Raton: CRC Press, 1996.


\bibitem{BeningKorolev2002} %2
\Au{Bening V., Korolev~V.} Generalized Poisson models and their
applications in insurance and finance.~--- Utrecht: VSP, 2002.

\bibitem{KruglovKorolev1990} %3
\Au{Круглов В.\,М., Королев~В.\,Ю.} Предельные теоремы для
случайных сумм.~--- М.: МГУ, 1990.

\bibitem{Michel1993} %4
\Au{Michel R.} On Berry--Esseen results for the compound Poisson
distribution~// Insurance: Mathematics and Economics, 1993. Vol.~13.
No.\,1. P.~35--37.

\bibitem{KorolevShorgin1997} %5
\Au{Korolev V.\,Yu., Shorgin S.\,Ya.} On the absolute constant in the
remainder term estimate in the central limit theorem for Poisson
random sums~// Probabilistic Methods in Discrete Mathematics:
4th  Petrozavodsk Conference (International) Proceedings.~---
Utrecht: VSP, 1997. P.~305--308.


\bibitem{KorolevShevtsova2010DAN} %6
\Au{Королев В.\,Ю., Шевцова~И.\,Г.} Уточнение верхней оценки
абсолютной постоянной в неравенстве Бер\-ри--Эссеена для смешанных
пуассоновских случайных сумм~// Докл. РАН, 2010. Т.~431. Вып.~1 С.~16--19.

\bibitem{KorolevShevtsova2010OPPM} %7
\Au{Королев В.\,Ю., Шевцова~И.\,Г.} Уточнение неравенства
Бер\-ри--Эссеена с приложениями к пуассоновским и смешанным
пуассоновским случайным суммам~// Обозрение прикладной и
промышленной математики, 2010. Т.~17. Вып.~1. С.~25--56.

\bibitem{KorolevShevtsova2010SAJ} %8
\Au{Korolev V., Shevtsova~I.} An impovement of the Berry--Esseen
inequality with applications to Poisson and mixed Poisson random
sums~// Scandinavian Actuarial J. Online first:
{\sf http://www.informaworld.com/10.1080/ 03461238.2010.485370}. June 04,
2010.


\bibitem{NefedovaShevtsova2010} %9
\Au{Нефедова Ю.\,С., Шевцова И.\,Г.} О~точности нормальной
аппроксимации для распределений пуассоновских случайных сумм~//
Информатика и её применения, 2010. Т.~5. Вып.~1. С.~39--45.

\bibitem{Shorgin1996} %10
{\it Шоргин С.\,Я.} О~точности нормальной аппроксимации для
распределений случайных сумм с безгранично делимыми индексами~//
Теория вероятностей и ее применения, 1996. Т.~41. Вып.~4. С.~920--926.


\bibitem{Shorgin2001} %11
\Au{Shorgin S.\,Ya.} Approximation of generalized Poisson
distributions: Comparison of Lyapunov fractions~// 21st Seminar on
Stability Problems for Stochastic Models (January~28\,--\,February~3,
2001, Eger, Hungary): Abstracts.~--- Publishing House of University
of Debrecen, 2001. P.~166--167.

\bibitem{Hoeffding1948} %12
\Au{Hoeffding W.} The extrema of the expected value of a function of
independent random variables~// Ann. Math. Statist., 1948. Vol.~19.
P.~239--325.



\label{end\stat}

\bibitem{Zolotarev1986} %13
\Au{Золотарев В.\,М.} Современная теория суммирования независимых
случайных величин.~--- М.: Наука, 1986.

\bibitem{Tyurin2010} %14
\Au{Тюрин И.\,С.} О~скорости сходимости в теореме Ляпунова~//
Теория вероятностей и ее применения, 2010. Т.~55. Вып.~2. С.~250--270.

 \end{thebibliography}
}
}


\end{multicols}       