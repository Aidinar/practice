
\def\stat{koltsov}

\def\tit{ИСПОЛЬЗОВАНИЕ МЕТРИК ПРИ~СРАВНИТЕЛЬНОМ 
ИССЛЕДОВАНИИ КАЧЕСТВА РАБОТЫ АЛГОРИТМОВ 
СЕГМЕНТАЦИИ ИЗОБРАЖЕНИЙ}

\def\titkol{Использование метрик при~сравнительном 
исследовании качества работы алгоритмов 
сегментации изображений}

\def\autkol{П.\,П.~Кольцов}
\def\aut{П.\,П.~Кольцов$^1$}

\titel{\tit}{\aut}{\autkol}{\titkol}

%{\renewcommand{\thefootnote}{\fnsymbol{footnote}}\footnotetext[1]
%{Работа выполнена при финансовой поддержке РФФИ (грант 11-01-00515).}}

\renewcommand{\thefootnote}{\arabic{footnote}}
\footnotetext[1]{Научно-исследовательский институт системных исследований Российской академии наук, 
koltsov@niisi.msk.ru}
 
\vspace*{-6pt}

  \Abst{Изучается качество работы четырех известных алгоритмов цифровой 
сегментации изображений. Исследование проводится на совокупности искусственных 
тестовых изображений, подвергаемых контролируемым искажениям при априори известном 
эталонном \textit{ground truth} изображении. Результат работы алгоритмов сегментации 
сравнивается с эталонным изображением с помощью метрик, облада\-ющих различными 
свойствами. Использование различных метрик для оценки качества работы алгоритмов 
сегментации и сравнение полученных при этом результатов позволяют более точно выяснить 
особенности каждого из исследуемых алгоритмов.}
  
  \KW{обработка изображений; оценка качества обработки изображений; сегментация 
изображений; выделение границ; энергетические методы}

  \vskip 12pt plus 9pt minus 6pt

      \thispagestyle{headings}

      \begin{multicols}{2}
      
            \label{st\stat}

  
  \section{Введение}
  
  В данной статье под сегментацией подразумевается разбиение изображения 
на совокупность непересекающихся связных областей, для которых характерно 
повышенное сходство между элементами одной и той же области по сравнению 
с прилежащим фоном или соседними областями. В~ряде случаев, например при 
решении задач текстурного анализа, таким областям могут отвечать 
определенные объекты или их части. Методы, предназначенные для решения 
такого класса задач, будем называть методами сегментации, а их программные 
реализации~--- алгоритмами сегментации. Таким образом, в статье под задачей 
сегментации подразумевается задача разбиения исходного изображения на 
вышеуказанные области.
  
  В~настоящее время разработка методов сегментации является одним из 
быстроразвивающихся направлений в области обработки изображений,\linebreak
широко 
востребованным в практической деятель\-ности. Однако значительное число 
существующих в настоящее время как методов сегментации, так и их 
программно-алгоритмических реализаций, порождает проблему выбора 
алгоритма, наиболее подходящего для решаемой конкретной задачи.
  
  Очевидно, что при выборе того или иного алгоритма необходимо 
руководствоваться его свойствами, позволяющими судить об ожидаемом 
качестве решения этим алгоритмом задачи сегментации. Качество решения 
задачи сегментации в статье будет определяться через оценку точности работы 
алгоритма сегментации.
  
  В статье сравнительное исследование точности работы различных 
алгоритмов сегментации проводится на наборе эта\-лон\-ных/тес\-то\-вых 
изображений, подвергаемых контролируемым искажениям. В~предположении, 
что наиболее существенные свойства тестируемых алгоритмов проявляются 
при обработке типичных и трудных для метода ситуаций, необходимым 
требованием к тестовым изоб\-ра\-же\-ни\-ям является возможность формирования\linebreak 
таких ситуаций с достаточной полнотой. Определение ситуаций, трудных и 
вместе с тем типичных для изучаемых алгоритмов сегментации, является 
содержательной задачей. Обычным методом решения задач такого рода служит 
обращение к опыту исследователя. Искомые ситуации определяются в процессе 
анализа большого числа примеров сегментации изображений различными 
методами и их реализациями. 

На основе полученных результатов анализа и 
создается набор искусственных тестовых изображений, которые в явном виде и 
с некоторой полнотой моделируют трудные ситуации.
  
  Для более полной оценки качества работы алгоритмов сегментации на наборе 
тестовых изображений они подвергаются контролируемым искажениям. 
Количественная оценка качества работы алгоритмов осуществляется с 
помощью различных метрик. Такой подход позволяет выявить особенности 
работы алгоритмов, определить границы их применимости и динамику качества 
работы. По результатам сравнительного исследования построены графики, 
отображающие количественное со\-по\-став\-ле\-ние изучаемых алгоритмов 
сегментации.

\begin{figure*}[b] %fig1
\vspace*{-7pt}
\begin{center}
\mbox{%
\epsfxsize=162.622mm
\epsfbox{kol-1.eps}
}
\end{center}
\vspace*{-9pt}
\Caption{Исходное изображение~(\textit{а}); сегментная карта~(\textit{б}); исходное 
изображение с регионами и границами~(\textit{в})
}
\end{figure*}
  
  Сравнительное исследование алгоритмов сегментации выполнено на 
реализациях четырех известных методов сегментации, основанных на\linebreak решении 
минимизационной задачи для функционала, обычно называемого в 
соответствии со своим содержательным смыслом энергетическим.
  
  Статья имеет следующую структуру.
  
  В разд.~2 кратко описана и проиллюстрирована примерами задача 
сегментации изображений.
  
  В разд.~3 описывается использованный подход к оценке качества 
сегментации.
  
  В разд.~4 приводятся наборы искусственных изоб\-ра\-же\-ний, использованных 
для тестирования алгоритмов сегментации.
  
  В разд.~5 излагаются методики анализа результатов сегментации, кратко 
описаны применяемые метрики.
  
  В разд.~6 ссылочно описаны алгоритмы сегментации, подвергнутые 
сравнительному исследованию.
  
  В разд.~7 приводятся результаты исследования.
  
  Раздел~8 посвящен выводам.
  
  \vspace*{-6pt}
  
  \section{Задача сегментации изображений}
  
  Основными понятиями, используемыми при решении задач сегментации 
изображений на основе некоторой однородности являются сегмент, регион, 
сегментная карта, границы. Результатом работы алгоритма сегментации (так 
называемого сегментатора) над исходным изображением является новое 
изображение~--- сегментная карта, содержащая области равномерной закраски. 
На рис.~1, взятом из статьи~[1], приведены примеры изображений сегментной 
карты и регионов. Исходное изображение показано на рис.~1,\,\textit{а}. 
Результатом применения алгоритма сегментации является изображение 
рис.~1,\,\textit{б}~--- сегментная карта. На ней можно видеть однотонно 
закрашенные области. Такие области на сегментной карте называются 
сегментами. При наложении сегментной карты на исходное изображение 
границы ее сегментов оконтуривают на нем области, которые называются 
регионами. Границами регионов служат границы сегментов. Для удобства 
рассмотрения на рис.~1,\,\textit{в} границы регионов нанесены на исходное 
изображение.


  Таким образом, алгоритм сегментации строит по исходному изображению 
сегментную карту и выполняет разбиение исходного изображения на регионы. 
Создание сегментной карты на основе исходного изображения является общим 
свойством всех сегментаторов, которые будут рассмотрены в статье.
  
  Количество сегментов, получаемых на сегментной карте, определяется 
особенностями конкретного сегментатора. В~наиболее простых алгоритмах 
число сегментов задается априорно, в более сложных оно определяется 
автоматически. В~некоторых случаях вводится ограничение (обычно сверху) на 
число сегментов.
  
  Сегментаторы, базирующиеся на энергетическом функционале, строят как 
монохромные, так и цветные сегментные карты. Цвета и уровни яр\-кости 
сегментов, назначаемые сегментатором, являются условными и могут быть 
достаточно далеки от цветов и яркостей соответствующих регионов исходного 
изображении. В~сущности, яркость или цвет, который сегментатор назначает 
данному сегменту, является просто номером этого сегмента на сегментной 
карте.
  
  Обычно сегменты строятся в зависимости от уровня яркости областей 
изображения, их цвета, текстуры или размера. В~основном исследователей 
интересуют те классы задач, в которых сегментация приводит к выделению 
значимых смысловых объектов.

  \vspace*{-6pt}
  
  \section{Подход к оценке качества сегментации}

 \begin{figure*}[b] %fig2
\vspace*{9pt}
\begin{center}
\mbox{%
\epsfxsize=141.364mm
\epsfbox{kol-2.eps}
}
\end{center}
\vspace*{-6pt}
\Caption{Пример сегментации изображения с линейно изменяющейся яркостью}
\end{figure*}

  Подход, развиваемый в данной статье, состоит в следующем.
  \begin{enumerate}[1.]
\item Качество сегментаторов оценивается по результатам их работы на 
искусственных тестовых изображениях, специально сконструированных 
таким образом, чтобы моделировать трудные ситуации, в которых 
сегментаторы допускают много ошибок. При этом имеется заданный априори 
эталонный результат <<хорошей>> сегментации.
\item Для сравнения результата сегментации с эталоном используются 
различные метрики. В~описываемой работе используются две метрики. Они 
обладают различными свойствами, и соответственно результаты, 
получаемые с помощью этих метрик, оказываются разными. Именно 
благодаря этому обстоятельству, сравнивая результаты измерения с 
помощью двух разных мет\-рик, можно более полно установить свойства 
сегментаторов и более точно оценить качество сегментации.
\item Динамика качества работы сегментаторов определяется путем 
сравнения результата сегментации на серии тестовых изображений с 
вносимыми искажениями относительно априори заданного результата 
сегментации исходного, неискаженного изображения. В~качестве искажения 
изображений в статье рассматривается их зашумление и размытие.
   \end{enumerate}
   
   Таким образом, описанный выше подход позволяет не только выполнить 
сравнительную оценку качества работы различных сегментаторов, но и 
выявить особенности поведения такой оценки при последовательном 
искажении входного изображения.
  
  \section{Наборы искусственных изображений}
  
  Вначале рассмотрим приведенный на рис.~2 пример достаточно простого 
изображения, но пред\-став\-ля\-юще\-го при этом сложности для работы 
сегментаторов.
        

На рис.~2,\,\textit{а} показано изображение, на котором по вертикали яркость 
неизменна. Вдоль горизонтального направления яркость изменяется от нуля на 
левом крае до~200 на правом (максимум яркости в данном формате 
изображения равен~255). Возрастание яркости происходит линейно с 
точ\-ностью до дискретизации.
  
  При правильной сегментации такого изображения должен получаться один 
сегмент. Однако практика показывает, что алгоритмы сегментации создают в 
таких областях с медленно меняющейся яркостью несколько сегментов. 
Типичный пример такой сегментации показан на рис.~2,\,\textit{б}. На рисунке не 
показаны сами сегменты, а только их границы, которые традиционно 
называются <<ложными границами>>. Приведенный пример иллюстрирует 
трудность полутоновых изображений, в которых имеет место плавное 
изменение яркости, для правильной работы сегментаторов.
  
  Основой совокупности тестовых изображений в данной статье является база 
данных системы \mbox{PICASSO}~[2], в которой аккумулированы типичные 
ситуации, присутствующие на реальных изоб\-ра\-же\-ни\-ях и представляющие 
трудность для работы различных методов обработки изображений. В~данной 
статье будет рассмотрено два класса тес\-то\-вых изоб\-ра\-же\-ний, взятых из базы 
данных сис-\linebreak\vspace*{-12pt}
\pagebreak

\end{multicols}

\begin{figure} %fig3
\vspace*{1pt}
\begin{center}
\mbox{%
\epsfxsize=141.365mm
\epsfbox{kol-3.eps}
}
\end{center}
\vspace*{-11pt}
\Caption{\textit{Круг} яркости~150 на фоне~30~(\textit{а}) и \textit{Угол} 80$^\circ$ 
яркости~50 на  фоне~200~(\textit{б})
}
%\end{figure*}
%\begin{figure*} %fig4
\vspace*{9pt}
\begin{center}
\mbox{%
\epsfxsize=163.994mm
\epsfbox{kol-4.eps}
}
\end{center}
\vspace*{-11pt}
\Caption{Изображения \textit{Step}~(\textit{а}), \textit{Junction}~(\textit{б}), 
\textit{Snail}~(\textit{в}) и \textit{Roof}~(\textit{г})
}
%\end{figure*}
%\begin{figure*} %fig5
\vspace*{9pt}
\begin{center}
\mbox{%
\epsfxsize=163.994mm
\epsfbox{kol-5.eps}
}
\end{center}
\vspace*{-11pt}
\Caption{Сегментация при гауссовом зашумлении, девиация зашумления~4
}
\end{figure}


\begin{multicols}{2}

\noindent 
те\-мы \mbox{PICASSO}, которые можно условно назвать
<<простыми>> и <<сложными>>. В~простых изображениях нет плавных 
изменений яркости, все границы резко очерчены, число регионов~--- два. 
Сложные изображения являются полутоновыми.
  
  На рис.~3 приведены примеры простых изображений, использованных в 
статье~--- \textit{Круг} и \textit{Угол}.
      
  На рис.~4 приведены примеры сложных изображений, использованных в 
статье,~--- \textit{Step}, \textit{Junction},  \textit{Snail}  и~\textit{Roof}.
   Для каждого тестового изображения из обоих классов было построено по два 
одно\-па\-ра\-мет\-ри\-че\-ских семейства новых изображений. 

Первое семейство 
строилось путем добавления к исходным изображениям гауссова шума. 
Значение девиации шума~$\sigma$ служило параметром семейства. Второе 
семейство строилось путем гауссова размытия тех же изображений. 
Параметром семейства служил радиус окна размытия~$r$.
  
  В качестве иллюстрации на рис.~5 показан типичный пример работы 
сегментатора на зашумленных сложных изображениях. Сегментные карты на 
рисунке опущены, показаны только границы полученных сегментов.



  \section{Методики анализа результатов сегментации}
  
  Поскольку различные сегментаторы по-разному закрашивают сегменты, а 
даже один и тот же сегментатор может изменять закраску сегментов при 
внесении искажений в изображения, то для сравнения результатов сегментации 
необходимо перейти к более универсальным объектам, чем сегмент. Очевидно, 
что таким наиболее естественным объектом является граница сегмента, которая 
представляет собой набор точек, никак не зависящих от закраски сегментов. 
Именно границы сегментов и будут использоваться далее при сравнительном 
исследовании алгоритмов сегментации. Отметим, что гомогенность сегментов, 
получаемых при обработке изображений исследуемым классом сегментаторов, 
оказывается очень удобной для выполнения процедуры выделения границ, а 
именно:
  \begin{itemize}
\item задача выделения границ сегментов и, соответственно, регионов 
практически тривиальна, если получена сегментная карта;
\item будучи границами двумерных однородных областей, такие границы 
являются непрерывными;
\item двумерная однородная область обладает ориентацией. Это означает, 
что для нее можно определить направление обхода граничного контура. 
После этого можно достаточно легко отследить все ее граничные точки, 
организуя их в одномерную кривую. При этом граница трактуется как 
упорядоченный массив.
\end{itemize}

  В~статье сравнение результатов сегментации основывается на измерении 
расстояний между кривыми. Для этого используется среднее расстояние~$d$ и 
хаусдорфово расстояние~$\chi$.
  
  Напомним, что среднее расстояние $d(X, Y)$ и хаусдорфово расстояние 
$\chi(X, Y)$ между множествами~$X$ и~$Y$ определяются следующим 
образом:
  \begin{align*}
  d(X,Y) &=\fr{1}{2}\left[ 
\fr{1}{N_X}\sum\limits_{x\in X}\rho(x,Y)+\fr{1}{N_Y}\sum\limits_{y\in 
Y}\rho(y,X)\right]\,;\\
  \chi(X,Y) &= \max\left[\max\limits_{x\in X} 
\rho(x,Y),\,\max\limits_{y\in Y}\rho(y,X)\right]\,.
  \end{align*}
    Здесь $\rho(x, Y)$~--- расстояние от точки $x\in X$ до множества~$Y$: 
$\rho(x,Y)\hm=\min\limits_{y\in Y}\rho(x,y)$, а $\rho(x,y)$~--- обычное 
  евклидово расстояние между точками~$x$ и~$y$, $N_X$ и $N_Y$~--- чис\-ло 
точек в~$X$ и~$Y$.
  
  Если величина $\chi(X, Y)$ дает скорее максимальное расстояние между 
точками множеств, то $d(X, Y)$ дает некоторое среднее расстояние и является 
тем, что принято называть термином <<мера различия>> (discrepancy 
measure). Очевидно, $d(X, Y) \hm\geq 0$, $d(X, X) \hm=0$ и $d(X, Y) \hm= d(Y, 
X)$.
  
  Рассмотрим подробнее свойства расстояния $d(X, Y)$ и его сходство и 
различие с расстоянием $\chi(X, Y)$.
  
  Если $X$ и $Y$~--- одноточечные множества на плоскости, то $d(X, Y)$ 
совпадает с обычным евклидовым расстоянием между точками~$X$ и~$Y$. 
Если~$X$ и~$Y$~--- отрезки, являющиеся противоположными сторонами 
прямоугольника, то $d(X, Y)$ совпадает с обычным расстоянием между этими 
сторонами.
  
  Пусть множество~$X$ состоит из одной точки~$x$, а $Y$~--- из двух точек: 
$x$ и $y$, причем евклидово расстояние между $x$ и $y$ равно~$r$. В~этом 
случае $\chi(X, Y) = r$.
  
  Найдем среднее расстояние. Поскольку $\rho(x, Y) \hm= 0$ в силу того, что $x\in 
Y$, а $\rho(y, X) = r$, то очевидно, что $d(X, Y) \hm= r/4$. Здесь учтено, что $N_Y \hm= 
2$. Если бы точка~$x$ не принадлежала множеству~$Y$, то среднее 
расстояние, как и хаусдорфово, было бы равно~$r$. Таким образом, наличие во 
множестве~$Y$ удаленной точки~$y$ не так сильно сказывается на среднем 
расстоянии, как на хаусдорфовом.
  
  Сходство и различие между $\chi$ и~$d$ проиллюстрировано на рис.~6. 
Здесь множество $X$~--- отрезок~$AB$, а множество~$Y$~--- отрезок~$CD$.

  Отметим, что если множества~$X$ и~$Y$~--- точки граничных кривых двух 
разных изображений и координаты точек задаются как координаты пикселов, 
то и расстояния~$d$ и~$\chi$ также измеряются в пикселах. Поскольку в 
процессе нахождения этих расстояний используется евклидова метрика с 
вы-\linebreak

\begin{center} %fig6
\vspace*{9pt}
\mbox{%
\epsfxsize=67.361mm
\epsfbox{kol-6.eps}
}
\end{center}
\begin{center}
%\vspace*{6pt}
{{\figurename~6}\ \ \small{Сходство и различие между $\chi$ и~$d$}}
\end{center}
%\vspace*{9pt}

%\smallskip
\addtocounter{figure}{1}



\noindent
числением квадратного корня, то величины~$d$ и~$\chi$ могут быть 
нецелыми числами.
  
  Далее будем считать, что $X$~--- это множество наперед заданных 
граничных точек на эталонном (ground truth) изображении, а $Y$~--- 
множество граничных точек, которые получены после работы сегментатора. 
В~идеальном случае множество~$X$ должно совпадать с~$Y$, но в типичном 
случае множество~$Y$ хотя и близко к~$X$, но имеет некоторое количество 
точек, удаленных от~$X$.
  
  Из рассмотренных выше примеров можно сделать следующие 
предположения относительно качества работы сегментатора:
  \begin{enumerate}[1.]
  \item  Если обе величины $\chi(X, Y)$ и $d(X, Y)$ в некотором естественном 
смысле малы, то полученные сегментатором границы близки к границам на 
эталонном изображении, причем на сегментной карте, построенной 
сегментатором, практически нет ни ложных границ, ни прочих посторонних 
сегментов, отсутствующих на эталонном изображении.
  \item Если значение $\chi(X, Y)$ велико, а $d(X, Y)$~--- мало, то на 
сегментной карте имеются посторонние сегменты, например шумовые пятна, в 
том числе удаленные от границ на эталонном изображении, но их размер 
невелик.
  \item  Если и $\chi(X, Y)$, и $d(X, Y)$ велико, то либо размер посторонних 
сегментов большой, либо граничные кривые значительно удалены друг от 
друга.
  \item Ситуация, когда среднее расстояние велико, а хаусдорфово~--- мало, по 
всей видимости, невозможна.
  \end{enumerate}
  
  В качестве иллюстрации приведем результаты типичной работы 
сегментатора на простом изображении \textit{Круг}, подвергнутом гауссову 
зашумлению. На рис.~7 представлены только границы полученных сегментов. 
Как видно, из-за наличия зашумления появились мелкие посторонние 
сегменты. Идеальной же границей эталонного изображения в данном случае 
является окружность.


  Отметим, что в хаусдорфовой метрике расстояние между фигурой на рис.~7 
и окружностью может быть велико, поскольку посторонние объекты 
значительно удалены от окружности. Однако среднее расстояние оказывается 
небольшим вследствие усреднения, поскольку размер посторонних сегментов 
мал.
  
  Таким образом, по паре расстояний~$\chi$ и~$d$ можно более полно судить 
о качестве проведенной сегментации, чем только по одному из них.
  
  На основании вышесказанного была использована следующая методика 
сравнительного исследования сегментаторов. Пусть $I$~--- некоторое исход\-ное 
изображение, а $I_\sigma$~--- изображение с до\-бав\-лен\-ным гауссовым шумом с 
девиацией~$\sigma$, $\{I_\sigma\}$~--- множество изображений~$I_\sigma$ с 
различными значениями девиации~$\sigma$, принадлежащими некоторому 
конечному набору. Пусть в результате обработки сегментатором множества 
$\{I_\sigma\}$ получается множество сегментных карт~$S\{I_\sigma\}$. Находя 
на каждой сегментной карте из~$S\{I_\sigma\}$ граничные точки, получаем 
множество граничных точек сегментов~$B(S\{I_\sigma\})$. Обозначим через 
$B(I)$ множество граничных точек исходного эталонного изображения~$I$. 
Вычисляем среднее расстояние $d = d(B(I), B(S\{I_\sigma\})$ и хаусдорфово 
$\chi = \chi(B(I), B(S\{I_\sigma\})$ для каждого значения девиации~$\sigma$. 
Полученные значения $d$ и~$\chi$ как функции~$\sigma$ могут быть 
изображены в виде графиков для последующего анализа.
  
  Если при сравнительном исследовании используется несколько исходных 
изображений, то функции $d = d(\sigma)$ и $\chi = \chi(\sigma)$ вычисляются 
отдельно для каждого из них и усредняются для каждого из значений 
девиации~$\sigma$ по всем исходным изображениям. Такое усреднение 
способствует более объективному отражению особенностей исследуемого 
сегментатора.
  
  Методика сравнительного исследования сегментаторов при размытии 
изображений совпадает с вышеизложенной для зашумления, в которой вмес\-то 
девиации шума~$\sigma$ рассматривается радиус окна гауссова размытия~$r$.


  
  Рассмотрим отдельно вопрос о выборе множества граничных точек~$B(I)$ 
исходного эталонного изображения~$I$. В~случае, когда $I$ относится к 
классу простых изображений, например \textit{Угол} или \textit{Круг}, для него 
легко построить идеальные граничные линии эталонного изображения и 
определить их в качестве~$B(I)$. Но для изображений из класса\linebreak
\begin{center} %fig7
\vspace*{1pt}
\mbox{%
\epsfxsize=53.905mm
\epsfbox{kol-7.eps}
}
\end{center}
%\begin{center}
\vspace*{6pt}
{{\figurename~7}\ \ \small{Пример сегментации простого изображения \textit{Круг} при зашумлении}}
%\end{center}
%\vspace*{9pt}

%\smallskip
\addtocounter{figure}{1}

\noindent
 сложных, 
таких как \textit{Step}, \textit{Junction}, \textit{Snail} и \textit{Roof}, даже при 
обработке неискаженного изображения все алгоритмы сегментации дают 
границы, сильно отличающиеся от предполагаемых идеальных. Причина 
состоит в наличии областей с медленно изменяющейся яркостью, обработка 
которых создает для всех методов сегментации значительные трудности. 
Поэтому для данного класса тестовых изображений в работе, представленной в 
статье, исследуется вопрос, насколько результат сегментации искаженного 
сложного изображения отличается от результата сегментации исходного 
изображения. Иными словами, на изображениях \textit{Step}, \textit{Junction}, 
\textit{Snail} и \textit{Roof} изучается устойчивость сегментатора. В~этом случае 
в качестве~$B(I)$ берутся границы, полученные при сегментации исходного 
изображения~$I$. Таким образом, методика тестирования на простых и 
сложных изображениях отличается, и поэтому графики для изображений 
\textit{Круг} и \textit{Угол} и для изображений \textit{Step}, \textit{Junction}, 
\textit{Snail} и \textit{Roof} будут приведены отдельно.
  
  Необходимо отметить, что эксперименты с оценкой качества работы 
сегментаторов показали, что они обладают определенной нестабильностью 
работы. Небольшие изменения при переходе от одного тестового изображения 
к последующему могут приводить к заметным изменениям всего результата 
сегментации. Например, при линейном увеличении максимальной яркости 
левого изображения на рис.~2 число ложных границ может в некоторых 
пределах скачкообразно и увеличиваться, и уменьшаться. Аналогичное явление 
происходит и при сегментации зашумленных и размытых изоб\-ра\-же\-ний. Такая 
ситуация приводит к тому, что графики $d(\sigma)$ и $\chi(\sigma)$ получаются 
негладкими и неудобными для анализа. Поэтому в качестве графиков строились 
более информативные линии тренда, пред\-став\-ля\-ющие собой стандартное 
полиномиальное приближение исходных точек.
  
  \section{Тестируемые методы сегментации}
  
  Изложенная выше методика исследования качества работы сегментаторов 
была опробована на примере хорошо известных реализаций четырех методов 
сегментации, базирующихся на разных видах энергетического функционала, 
описание которых можно найти в~[1--8].
  
  \subsection{Сегментатор JSEG} %6.1
  
  Сегментатор JSEG~[9] ориентирован на автоматическую сегментацию изображений и 
видео, которые могут содержать цветные регионы и текстуры. Обработка 
изображения состоит из двух независимых шагов: цветовой квантизации и 
пространственной сегментации. Собственно сегментация выполняется с 
использованием метода растущих областей. В~чис\-ле опций сегментатора 
имеются опции обработки полутоновых изображений, в том числе и 
бестекстурных, что и было использовано. Рабочие параметры сегментатора 
могут устанавливаться самим сегментатором. В~ходе исследования качества 
работы этого сегментатора такие па\-ра\-мет\-ры и были использованы.
  
  \subsection{Сегментатор EDISON} %6.2
  
  Сегментатор EDISON~\cite{10-kol} выполняет сегментацию изображений, выделение границ, а 
также фильтрацию шума, сохраняющую резкие перепады яркости изоб\-ра\-же\-ния. 
Алгоритм, реализованный в сегментаторе, определяет границы на изображении 
и использует их в процессе сегментации. Одним из основных параметров 
сегментатора является минимальный размер региона в пикселах, который 
может создать данный метод. При исследовании значения этого параметра 
брались равными~100 и~1000. Соответственно, при исследовании 
сегментаторы обозначались как {EDISON~100} и {EDISON~1000}. 
Остальные рабочие параметры фиксировались так же, как и у авторов 
сегментатора.
  
  \subsection{Сегментатор EDGEFLOW} %6.3
  
  Для сегментации и выделения границ изображения сегментатор EDGEFLOW~\cite{11-kol}
  реализует 
метод потока граничных точек. Суть его состоит в том, что в каждой точке 
изображения вычисляется направление изменения яркости, цвета или текстуры. 
Это позволяет формировать векторное поле потока граничных точек. 
Интегральные кривые этого поля, проходя через области изображения, 
образуют гомогенные регионы, сталкиваются друг с другом и, стабилизируясь, 
формируют границы регионов. Сегментатор имеет параметр, существенно 
влияющий на его работу~--- так называемое смещение. При исследовании 
качества работы этого сегментатора значения параметра брались равными~10 
и~26. Соответственно при исследовании сегментаторы обозначались как 
{EDGEFLOW~10} и {EDGEFLOW~26}.

  \begin{figure*}[b] %fig8
  \vspace*{1pt}
\begin{center}
\mbox{%
\epsfxsize=162.93mm
\epsfbox{kol-8.eps}
}
\end{center}
\vspace*{-6pt}
  \Caption{Зашумление простых изображений: \textit{1}~--- {JSEG}; \textit{2}~--- 
{EDGEFLOW~26}; \textit{3}~--- {EDISON~1000}; \textit{4}~--- 
{EDGEFLOW~10}; \textit{5}~--- {EDISON~100}; \textit{6}~--- 
{MULTISCALE}
  }
  \end{figure*}
  
  \subsection{Сегментатор MULTISCALE} %6.4
  
  При обработке сегментатором MULTISCALE~\cite{12-kol}
  изображение вначале анализируется в более 
грубом масштабе, а затем~--- в более мелком. При рассмотрении изоб\-ра\-же\-ния в 
грубом масштабе шум и помехи мало заметны. При рассмотрении изображения 
в более мелком масштабе лучше заметны детали объектов. Объединение этих 
двух подходов позволяет фильт\-ро\-вать шум и сохранять важные детали 
изображения. Сегментатор имеет большое число рабочих параметров. 
В~соответствии с рекомендациями, приведенными в описании сегментатора, 
при исследовании качества его работы были взяты значения, описанные как 
безопасные.
  
  \section{Результаты тестирования}
  
  \subsection{Зашумление изображений} %7.1
  
  Для тестирования качества работы сегментаторов на простых и сложных 
изображениях к ним аддитивно был добавлен гауссов шум с девиацией от~0 
до~30 с шагом~1. Таким образом было получено множество тестовых 
изображений.
  
  На рис.~8 приведены графики~$d(\sigma)$ и~$\chi(\sigma)$ по всем 
исследованным сегментаторам при обработке простых изображений 
\textit{Круг} и \textit{Угол} (см.\ рис.~3). Граничные кривые, полученные после 
сегментации данных изображений, сравнивались с границами эталонных 
изображений, которые очевидны и не приводятся. Здесь и далее графики 
представлены как линии тренда результатов измерений. Количество маркеров 
на графиках уменьшено для большей разборчивости.
  
  Из анализа графиков можно видеть, что на данных изображениях 
сегментатор {JSEG} резко выделяется среди остальных. Графики для 
{JSEG} практически лежат на горизонтальной оси, так что в данном 
диапазоне зашумлений этот метод работает очень хорошо по сравнению со 
всеми остальными. Кроме этого, оба расстояния~$d$ и~$\chi$ для {JSEG} 
малы. Поскольку полученные при работе этого сегментатора границы 
сравниваются с идеальными границами эталонных изображений, можно 
сделать вывод, что сегментатор {JSEG} хорошо сохраняет форму границ.
  
  В целом при увеличении зашумления как среднее расстояние~$d$, так и 
хаусдорфово расстояние~$\chi$ растут для всех исследуемых сегментаторов. 
Спад некоторых графиков при больших значениях девиации шума не должен 
вводить в заблуждение, поскольку граничные кривые очень сильно искажены, 
что делает результаты сегментации недостоверными. 

Можно заключить, что 
сегментатор {EDISON} лучше использовать на простых изображениях при 
значениях девиации не более 15--20, а сегментатор {EDGEFLOW}~--- при 
значениях девиации не более 3--5. Сегментатор {MULTISCALE} 
достаточно устойчив к зашумлению простых изображений: замедление роста 
графика среднего расстояния и некоторая тенденция к спаду начинается при 
значениях девиации шума 20--25.

  
  На рис.~9 приведены графики~$d(\sigma)$ и $\chi(\sigma)$ по всем 
исследованным сегментаторам при обработке сложных изображений 
\textit{Step}, \textit{Junction}, \textit{Snail} и \textit{Roof} (см.\ рис.~4). Как уже 
было сказано ранее, ground truth изображения, полученные после 
сегментации исходных изображений, в данном случае являлись эталоном для 
нахождения среднего и хаусдорфова расстояния.

  \begin{figure*} %fig9
  \vspace*{1pt}
\begin{center}
\mbox{%
\epsfxsize=162.93mm
\epsfbox{kol-9.eps}
}
\end{center}
\vspace*{-6pt}
  \Caption{Зашумление сложных изображений:
  \textit{1}~--- {JSEG};
    \textit{2}~--- {EDGEFLOW~26};
  \textit{3}~--- {EDISON~1000}; 
\textit{4}~--- {EDGEFLOW~10}; 
 \textit{5}~--- {EDISON~100}; 
\textit{6}~--- {MULTISCALE}
  }
%\vspace*{6pt}
  \end{figure*}
  
  \begin{figure*}[b] %fig10
\vspace*{1pt}
\begin{center}
\mbox{%
\epsfxsize=162.93mm
\epsfbox{kol-10.eps}
}
\end{center}
\vspace*{-6pt}
\Caption{Размытие простых изображений:
\textit{1}~--- {JSEG};
\textit{2}~--- {EDGEFLOW~26}; 
  \textit{3}~--- {EDISON~1000}; 
   \textit{4}~--- {EDGEFLOW~10};
  \textit{5}~--- {MULTISCALE}
  }
  \end{figure*}
  
  Из анализа графиков на рис.~9 видно, что наибольшие значения расстояний 
получились для сегментатора {EDGEFLOW~26}, причем графики\linebreak 
начинают спадать при значениях девиации 7--8. Это,
по-видимому, и есть тот 
уровень зашумления, до которого целесообразно использовать данный 
сег\-мен\-та\-тор. Величины расстояний для \mbox{EDGEFLOW}~10 меньше, однако 
графики перестают расти и начинают спадать приблизительно при той же 
величине шума. Сегментатор {MULTISCALE} несколько более устойчив к 
зашумлению: спад графиков начинается при значениях девиации порядка~20. 
Но большие в целом значения расстояний~$d$ и~$\chi$ говорят о том, что при 
увеличении зашумления изображения этот сегментатор создает много лишних 
крупных сегментов. Сегментаторы {EDISON} и {JSEG} показали 
наилучшие результаты. Их графики лежат ниже графиков остальных методов. 
Графики сегментатора {EDISON} начинают спадать при значениях 
девиации 15--20. По-видимому, это предельное зашумление, до которого 
целесообразно использовать этот сегментатор. У~сегментатора {JSEG} 
показатели несколько лучше.
  
  Сопоставим теперь графики для простых изоб\-ра\-же\-ний и сложных. 
  %
  Нетрудно 
видеть, что в обоих случаях лучшие показатели у сегментатора {JSEG}, к 
которым приближаются показатели сегментатора {EDISON}. Оценки для 
предельных уровней шума, с которым методы еще можно использовать, тоже 
примерно одинаковы.

\vspace*{-6pt}
  
  \subsection{Размытие изображений}
  
  \vspace*{-2pt}
  
  Для тестирования качества работы сегментаторов при размытии 
изображений был создан набор простых изображений, для которых выполнено 
гауссово размытие с радиусом окна от~0 до 12~пикселов с шагом 0,4~пиксела и 
набор сложных изображений, для которых выполнено гауссово размытие с 
радиусом окна от~0 до 3~пикселов с шагом 0,1~пиксела.
  
  На рис.~10 приведены графики $d(r)$ и~$\chi(r)$ по всем 
исследуемым алгоритмам при сегментации простых изображений \textit{Круг} 
и \textit{Угол}.

  \begin{figure*} %fig11
  \vspace*{1pt}
\begin{center}
\mbox{%
\epsfxsize=161.93mm
\epsfbox{kol-11.eps}
}
\end{center}
\vspace*{-12pt}
  \Caption{Размытие сложных изображений: 
  \textit{1}~--- \textit{JSEG};
   \textit{2}~--- {EDGEFLOW~26};  \textit{3}~--- {EDISON~1000};
 \textit{4}~--- {EDGEFLOW~10};  
 \textit{5}~--- {EDISON~100};
\textit{6}~--- {MULTISCALE}
  }
    \vspace*{-6pt}
  \end{figure*}

  Как видно из анализа полученных графиков, наилучшее качество показал 
сегментатор {EDISON}: у него значения как среднего, так и хаусдорфова 
расстояния наименьшие по сравнению с другими методами. Это означает 
точное сохранение формы границ при размытии исходного изображения. 
Сегментатор {EDGEFLOW} продемонстрировал приемлемое качество при 
значениях радиуса окна размытия не более 3--4, после чего на сегментной карте 
возникало значительное количество лишних сегментов. На это обстоятельство 
указывают большие значения как среднего, так и хаусдорфова расстояния. 
Сегментатор {JSEG} хорошо работает до значений радиуса 8--9. 
Примерно такие же показатели у сегментатора {MULTISCALE}. Однако 
следует еще раз подчеркнуть, что все эти результаты относятся к простым 
изображениям, на которых не создаются ложные границы.
  
  На рис.~11 приведены графики $d(r)$ и $\chi(r)$ при обработке 
сложных изображений \textit{Step}, \textit{Junction}, \textit{Snail} и \textit{Roof}, 
содержащих области с медленно изменяющейся яркостью. Как видно, для всех 
исследуемых алгоритмов сегментация сложных изоб\-ра\-же\-ний оказалась более 
трудной задачей. Обратим внимание на диапазоны значений радиуса окна 
размытия на графиках рис.~10 и~11. Если на рис.~10 радиус изменялся от~0 до 
12~пикселов, то для получения приемлемого качества сегментации сложных 
изображений (см.\ рис.~11) пришлось ограничиться максимальным значением 
радиуса в 3~пиксела. Отметим, что при радиусе размытия в 2--3~пиксела 
значения всех расстояний, приведенных для сегментаторов на рис.~10, в 
несколько раз меньше расстояний, приведенных на рис.~11.
 
  Анализ графиков на рис.~11 показывает, что при тестировании на размытых 
сложных изображениях наилучшее качество показал сегментатор {JSEG}: 
у него значения как среднего, так и хаусдорфова расстояния наименьшие по 
сравнению с другими сегментаторами. Это означает более точное сохранение 
сегментатором формы границ при размытии исходного изображения. 
Сегментатор {EDGEFLOW} может успешно работать лишь при значениях 
радиуса окна размытия не более 1--1,5~пиксела. При больших значениях 
радиуса размытия начинает значительно изменяться количество и форма 
сегментов. Сегментаторы {EDISON} и {MULTISCALE} близки по 
своим показателям. Однако большие значения хаусдорфова расстояния при 
существенно меньших (в 10--20~раз) значениях среднего расстояния указывают 
на большое число мелких искажений (изломов) граничных линий. Сохранение 
границ ухудшается при значениях радиуса окна размытия, уже больших 
1~пиксела.
 
  \section{Выводы}
  
  \noindent
  \begin{enumerate}[1.]
  \item Применение для измерения результатов сегментации двух метрик с 
различными свойствами~--- среднего и хаусдорфова расстояния~--- позволяет 
более точно оценить качество\linebreak
работы сегментаторов. Подчеркнем, что при этом 
не идет речь об определении лучшей из двух метрик. Существенно то, что обе 
метрики используются одновременно и результаты измерения сопоставляются.
  \item  Кроме хаусдорфова и среднего расстояния, могут быть и другие 
способы сравнения граничных кривых. Например, подсчет числа точек на 
кривых, количества точек ветвления, гистограммы распределения кривизны 
кривых и~т.\,п. Каждый из таких способов дает некоторую меру различия двух 
наборов кривых (граничных линий сегментов). Численное значение каждой 
такой меры является характеристикой пары изображений, а их совокупность 
задает вектор характеристик. Для такого многомерного вектора характеристик 
можно выполнить исследование, аналогичное вышеизложенному, или 
воспользоваться иными оригинальными методами. Использование таких 
методов (которые можно назвать мультиметрическими) может позволить еще 
точнее оценить свойства сегментаторов.
  \item  При тестировании конкретных сегментаторов оказалось, что одной из 
наиболее важных проблем исследованных методов сегментации является 
проблема создания ложных границ на изображениях с медленно изменяющейся 
яркостью. В~этой связи разработка методов и их программных реализаций, 
сводящих к минимуму количество ложных границ, представляется актуальной.
  \item  Исследование качества работы ряда популярных сегментаторов 
показало, что они ведут себя неустойчиво при зашумлении и размытии 
изоб\-ра\-же\-ния. Другими словами, результат сегментации даже слегка 
зашумленного или размытого изображения может существенно отличаться от 
результата сегментации исходного изображения. Таким образом, можно 
заключить, что целесообразно до процедуры сегментации выполнить очистку 
изображения от шума и повысить его четкость.
  \item  В статье приведены численные оценки степени искажения 
изображений, при которой работа сегментаторов остается удовлетворительной. 
В~этой связи становится актуальной\linebreak
задача оценивания (параметрического для 
рассмотренных в статье случаев) искажений для изображения, подвергаемого 
сегментации.
  \end{enumerate}
  

  
  {\small\frenchspacing
{%\baselineskip=10.8pt
\addcontentsline{toc}{section}{Литература}
\begin{thebibliography}{99}
  
  \bibitem{1-kol}
  \Au{Deng Y., Manjunath B.\,S.}
  Unsupervised segmentation of color-texture regions in images and video~// IEEE 
Transactions on Pattern Analysis and Machine Intelligence (PAMI'01), 2001. 
Vol.~23. No.\,8. P.~800--810.
  
  \bibitem{2-kol}
  \Au{Gribkov I.\,V., Koltsov P.\,P., Kotovich~N.\,V., Kravchenko~A.\,A., 
Kutsaev~A.\,S., Nikolaev~V.\,K., Zakharov~A.\,V.}
  PICASSO~--- a system for evaluating edge detection algorithms~// Pattern 
Recognition and Image Analysis, 2003. Vol.~13. No.\,4. P.~617--622.

  \bibitem{5-kol} %3
  \Au{Mumford D.}
  The Bayesian rationale for energy functionals~// Geometry driven diffusion in 
computer vision~/ Ed. B.~Romeny.~--- Dordrecht: Kluwer Academic, 1994. 
P.~141--153.
  
  \bibitem{6-kol} %4
  \Au{Kervrann C., Hoebeke M., Trubuil~A.}
  A level line selection approach for object boundary estimation~// 7th IEEE 
Conference (International) on Computer Vision, ICCV'99.~--- Kerkyra: IEEE 
Computer Society Press, 1999. P.~963--968.

  \bibitem{4-kol} %5
  \Au{Ma W.-Y., Manjunath B.\,S.}
  EdgeFlow: A technique for boundary detection and image segmentation~// IEEE 
Transactions on Image Processing, 2000. Vol.~9. P.~1375--1388.

  \bibitem{3-kol} %6
  \Au{Meyer F., Vachier C.}
  Image segmentation based on viscous flooding simulation~//  ISMM'02 Proceedings.~---  
Sydney: CSIRO, 2002. P.~69--77.
  
  \bibitem{7-kol}
  \Au{Christoudias C.\,M., Georgescu~B., Meer~P.}
  Synergism in low level vision~// 16th Conference (International ) on Pattern 
Recognition.~--- Quebec City: IEEE Computer Society Press, 2002. Vol.~4. 
  P.~150--155.
  
  \bibitem{8-kol}
  \Au{Sumengen B., Manjunath~B.\,S.}
  Multi-scale edge detection and image segmentation~//  European Signal Processing 
Conference (EUSIPCO) Proceedings.~--- Antalya,\linebreak 2005. {\sf 
http://vision.ece.ucsb.edu/publications/\linebreak 05eusipcoBarisMultiscale.pdf}.
  
  \bibitem{9-kol}
  Сегментатор JSEG. {\sf http://vision.ece.ucsb.edu/\linebreak segmentation/jseg/}.
  
  \label{end\stat}
  
  \bibitem{10-kol}
  Сегментатор EDISON. {\sf http://www.caip.rutgers.edu/\linebreak riul/research/code/EDISON/}.
  
  \bibitem{11-kol}
  Сегментатор EDGEFLOW. {\sf http://vision.ece.ucsb.edu/\linebreak segmentation/edgeflow}.
  
  \bibitem{12-kol}
  Сегментатор MULTISCALE. {\sf http://barissumengen.\linebreak com/seg/}.
 \end{thebibliography}
}
}


\end{multicols}       