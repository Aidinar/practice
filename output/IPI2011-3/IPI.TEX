\documentclass[10pt]{book}
\usepackage[utf8]{inputenc}

\usepackage{latexsym,amssymb,amsfonts,amsmath,indentfirst,shapepar,%fleqn,%
picinpar,shadow,floatflt,enumerate,multicol,colortbl,ipi}

\usepackage{rotating}
\usepackage{mathrsfs}

\input{epsf}

%\nofiles

%\includeonly{avtor} %,avtor-eng}
%\includeonly{avtor-eng}
%\includeonly{pred}      %+pdf
%\includeonly{podgot-1str}  %+
%\includeonly{ocherk} %+

%\includeonly{korgor}    %1pdf+
%\includeonly{ushakov}   %2pdf+
%\includeonly{yakovenko} %3pdf
%\includeonly{zeifman}   %4pdf+
%\includeonly{leri}      %5+pdf
%\includeonly{razum}     %6pdf
%\includeonly{kudr}      %7+pdf
%\includeonly{koltsov}   %8+pdf
%\includeonly{korolev}   %9pdf+
%\includeonly{ushakov1}  %10pdf+
%\includeonly{shest}     %11pdf+
%\includeonly{chuprinov} %12pdf+
%\includeonly{ubilei}    %13pdf+

%\includeonly{toc-rus, toc-en}
%\includeonly{toc-en}

%\includeonly{obchak}
%\includeonly{reshal}
%\includeonly{eng-index}
%\includeonly{cover3}

\usepackage{acad}
\usepackage{courier}
\usepackage{decor}
\usepackage{newton}
\usepackage{pragmatica}
\usepackage{zapfchan}
\usepackage{petrotex}
\usepackage{bm}                     % полужирные греческие буквы
\usepackage{upgreek}                % прямые греческие буквы
\usepackage{eufrak}
%\usepackage{verbatim}

\renewcommand{\bottomfraction}{0.99}
\renewcommand{\topfraction}{0.99}
\renewcommand{\textfraction}{0.01}

\setcounter{secnumdepth}{1} %здесь - 3 + chapter = 4

\arraycolsep=1.5pt

%\usepackage[pdftex]{graphicx}

%\usepackage{oz}

%NEW COMMANDS


\renewcommand*{\hm}[1]{#1\nobreak\discretionary{}%
            {\hbox{$\mathsurround=0pt #1$}}{}} %% Дублирует знаки операций
                               %при переносе в формуле (перед знаком, который 
                               %надо продублировать ставится команда \hm)


\renewcommand{\r}{\mathbb{R}}
\newcommand{\I}{{\rm I\hspace{-0.7mm}I}}
\newcommand{\Ik}{\mbox{{\small \tt {1}}\hspace{-1.5mm}{\tt 1}}}

\def\vrp{\varphi}
\def\prt{\partial}
\def\mm{{\rm M}}

\newcommand{\il}[2]{\int\limits_{#1}^{#2}}%интеграл с пределами #1 и #2

\def\sss{\sum\limits}
\def\tr{\,,\,\ldots\,,\,}
\def\rk{\,\right]}
\def\lk{\left[\,}
\def\rf{\right\}}
\def\lf{\left\{}

\def\ee{{\cal E}}
\def\ww{{\cal W}}
\def\yy{{\cal Y}}
\def\vv{{\cal V}}


\newcommand{\h}{{\bf H}}
\newcommand{\p}{{\sf P}}  % вероятность
\newcommand{\e}{{\sf E}}  % мат. ожидание
\newcommand{\D}{{\sf D}}  % дисперсия
\newcommand{\eps}{\varepsilon}
\newcommand{\vp}{\mathrm{v.p.}}
\newcommand{\F}{{\mathcal F}}
%\def\iint{\int\limits_{-\infty}^{\infty}}
\newcommand{\abs}[1]{\left\vert#1\right\vert}
\def\w{\omega}
\def\W{\Omega}
\def\iii{\int\limits}

\DeclareMathOperator{\sign}{sign}

%\newcommand{\gr}{{\geqslant}}

\newcommand{\g}{\mbox{\textit{g}}}

\renewcommand{\la}{\lambda}
\newcommand{\si}{\sigma}
\newcommand{\alp}{\alpha}

%\newcommand{\pto}{\stackrel{P}{\longrightarrow}} % сходимость по веpоятности

%\newcommand{\eqd}{\stackrel{d}{=}} % равенство по pаспpеделению

%\newcommand{\kp}{\kappa}
%\def\Q{{\cal Q}} \def\H{{\cal H}}
%\newcommand{\bet}{\beta_{2+\delta}}


%\newtheorem{definition}{Определение}
%\renewcommand{\thedefinition}{\arabic{definition}.}
%END NEW COMMANDS

%\renewcommand{\baselinestretch}{1.2}

%\pagestyle{myheadings}

\setlength{\textwidth}{167mm}      % 122mm
\setlength{\textheight}{658pt}
%\setlength{\textheight}{635.6pt}
\setlength{\columnsep}{4.5mm}

\setcounter{secnumdepth}{4}

%\addtolength{\headheight}{2pt}
%\addtolength{\headsep}{-2mm}

%\addtolength{\topmargin}{-20mm}  % for printing


\hoffset=-30mm  % From Yap
%\hoffset=-20mm  % From Acrobat

%\voffset=0mm % From Yap
%\voffset=-15mm   % From Acrobat

\addtolength{\evensidemargin}{-9.5mm} % for printing
\addtolength{\oddsidemargin}{9.5mm}  % for printing

%\renewcommand{\thefootnote}{\fnsymbol{footnote}}
%\renewcommand{\thefootnote}{\arabic{footnote}}
\renewcommand{\figurename}{\protect\bf Рис.}
\renewcommand{\tablename}{\protect\bf Таблица}

\newcommand{\Caption}[1]{\caption{\protect\small %\baselineskip=2.5ex
#1}}

\renewcommand{\thefigure}{\arabic{figure}}
\renewcommand{\thetable}{\arabic{table}}
\renewcommand{\theequation}{\arabic{equation}}
\renewcommand{\thesection}{\arabic{section}}

\renewcommand{\contentsname}{СОДЕРЖАНИЕ}
\newcommand{\fr}[2]{\displaystyle\frac{\displaystyle #1\mathstrut}{\displaystyle #2\mathstrut}}

%\renewcommand{\thefootnote}{\fnsymbol{footnote}}
%\newcommand{\g}{\mbox{\textit{g}}}

%\newcommand{\Caption}[1]{\caption{\protect\small\baselineskip=2ex #1}}
\newcounter{razdel}
\setcounter{razdel}{0}


\newcommand{\titel}[4]{%
\

\vspace*{5pt}

\ifodd\therazdel {\raggedright\noindent\Large\textrm\textbf
 \lineskip .75em
  \baselineskip=3.2ex #1 \par}
\vskip 1em {\noindent\large\textrm\textbf #2 \par}
\addcontentsline{toc}{subsection}{{\textrm\textbf #3}\protect\newline #1}
\def\rightheadline{\underline{\noindent\hbox to \textwidth{\hfill\small\textrm{#4}
%\hfill \large\bf\thepage
}}}
\def\leftheadline{\underline{\noindent\parbox{\textwidth}{
%\raggedleft\large\bf\thepage \hfill
\small\textit{#3}\hfill}}}
\def\leftfootline{\small{\textbf{\thepage}
\hfill ИНФОРМАТИКА И ЕЁ ПРИМЕНЕНИЯ\ \ \ том~5\ \ \ выпуск 3\ \ \ 2011}
}%
 \def\rightfootline{\small{ИНФОРМАТИКА И ЕЁ ПРИМЕНЕНИЯ\ \ \ том~5\ \ \ выпуск~3\ \ \ 2011
\hfill \textbf{\thepage}}} 
\vskip 2em \setcounter{figure}{0}
\setcounter{table}{0} 
\setcounter{equation}{0} 
\setcounter{section}{0}
\setcounter{subsection}{0} 
\setcounter{subsubsection}{0}
\setcounter{footnote}{0} 
\setcounter{razdel}{0}
%\end{flushleft}
\else {
 \raggedright\noindent\Large\textrm\textbf
 \lineskip .75em
\baselineskip=3.2ex #1 \par} \vskip 1em
%\begin{flushleft}
{\noindent\large\textrm\textbf #2 \par}
\addcontentsline{toc}{subsection}{{\textrm\textbf #3}\protect\newline #1}
\def\rightheadline{\underline{\noindent\hbox to \textwidth{\hfill\small\textrm{#4}
%\hfill \large\bf\thepage
}}}
\def\leftheadline{\underline{\noindent\parbox{\textwidth}{%\raggedleft\large\bf\thepage \hfill
\small\textit{#3}\hfill}}}
\def\leftfootline{\small{\textbf{\thepage}
\hfill ИНФОРМАТИКА И ЕЁ ПРИМЕНЕНИЯ\ \ \ том~5\ \ \ выпуск~3\ \ \ 2011}
}%
 \def\rightfootline{\small{ИНФОРМАТИКА И ЕЁ ПРИМЕНЕНИЯ\ \ \ том~5\ \ \ выпуск~3\ \ \ 2011
\hfill \textbf{\thepage}}} \vskip 2em \setcounter{figure}{0}
\setcounter{table}{0} \setcounter{equation}{0} \setcounter{section}{0}
\setcounter{subsection}{0} \setcounter{subsubsection}{0}
\setcounter{footnote}{0}
%\end{flushleft}
\fi}

\newcommand{\titelr}[2]{%
\

\vspace*{5pt}

\ifodd\therazdel {\raggedright\noindent\large\textrm\textbf
 \lineskip .75em
  \baselineskip=3.2ex #1 \par}
\vskip 1em {\noindent\normalsize\textrm\textbf #2 \par}
\else {
 \raggedright\noindent\large\textrm\textbf
 \lineskip .75em
\baselineskip=3.2ex #1 \par} \vskip 1em
%\begin{flushleft}
{\noindent\normalsize\textrm\textbf #2 \par}
\fi}

\newcommand{\titele}[5]{%
\

%\vspace*{5pt}

\ifodd\therazdel {\raggedright\noindent%\large
\textrm\textbf
 \lineskip .75em
%  \baselineskip=3.2ex
#1 \par}
\vskip .5em {\noindent\large\textrm\textbf #2 \par}
\vskip .5em
 {\noindent\textrm #3 \par}
\addcontentsline{toc}{subsection}{{\textrm\textbf #1}\protect\newline #2}
\def\rightheadline{\underline{\noindent\hbox to \textwidth{\hfill\small\textrm{#4}
%\hfill \large\bf\thepage
}}}
\def\leftheadline{\underline{\noindent\parbox{\textwidth}{
%\raggedleft\large\bf\thepage \hfill
\small\textrm{#5}\hfill}}}
\def\leftfootline{\small{\textbf{\thepage}
\hfill ИНФОРМАТИКА И ЕЁ ПРИМЕНЕНИЯ\ \ \ том~5\ \ \ выпуск~3\ \ \ 2011}
}%
 \def\rightfootline{\small{ИНФОРМАТИКА И ЕЁ ПРИМЕНЕНИЯ\ \ \ том~5\ \ \ выпуск~3\ \ \ 2011
\hfill \textbf{\thepage}}} \vskip 1em \setcounter{figure}{0}
\setcounter{table}{0} \setcounter{equation}{0} \setcounter{section}{0}
\setcounter{subsection}{0} \setcounter{subsubsection}{0}
\setcounter{footnote}{0} \setcounter{razdel}{0}
%\end{flushleft}
\else {
 \raggedright\noindent%\large
 \textrm\textbf
 \lineskip .75em
%\baselineskip=3.2ex
#1 \par} \vskip .5em
%\begin{flushleft}
{\noindent\large\textrm\textbf #2 \par} \vskip .5em
 {\noindent\textrm #3 \par}
\addcontentsline{toc}{subsection}{{\textrm\textbf #1}\protect\newline #2}
\def\rightheadline{\underline{\noindent\hbox to \textwidth{\hfill\small\textrm{#4}
%\hfill \large\bf\thepage
}}}
\def\leftheadline{\underline{\noindent\parbox{\textwidth}{%\raggedleft\large\bf\thepage \hfill
\small\textrm{#5}\hfill}}}
\def\leftfootline{\small{\textbf{\thepage}
\hfill ИНФОРМАТИКА И ЕЁ ПРИМЕНЕНИЯ\ \ \ том~5\ \ \ выпуск~3\ \ \ 2011}
}%
 \def\rightfootline{\small{ИНФОРМАТИКА И ЕЁ ПРИМЕНЕНИЯ\ \ \ том~5\ \ \ выпуск~3\ \ \ 2011
\hfill \textbf{\thepage}}} \vskip 1em \setcounter{figure}{0}
\setcounter{table}{0} \setcounter{equation}{0} \setcounter{section}{0}
\setcounter{subsection}{0} \setcounter{subsubsection}{0}
\setcounter{footnote}{0}
%\end{flushleft}
\fi}

\def\Abst#1{
\begin{center}\small\nwt
\parbox{150mm}{%\baselineskip=2.5ex
\textbf{Аннотация:}\ \
%\hspace*{\parindent}
#1}
\end{center}}
\def\Abste#1{
\begin{center}\small\nwt
\parbox{150mm}{%\baselineskip=2.5ex
\textbf{Abstract:}\ \
%\hspace*{\parindent}
#1}
\end{center}}

\def\KW#1{
\begin{center}\small\nwt
\parbox{150mm}{%\baselineskip=2.5ex
\textbf{Ключевые слова:}\ \ #1}
\end{center}}

\def\KWE#1{
\begin{center}\small\nwt
\parbox{150mm}{%\baselineskip=2.5ex
\textbf{Keywords:}\ \ #1}
\end{center}}


\def\KWN#1{
%\begin{center}
%\small
%\parbox{150mm}\end{center}
}

\renewcommand{\thesubsection}{\thesection.\arabic{subsection}\hspace*{-5pt}}
\renewcommand{\thesubsubsection}{\thesubsection\hspace*{5pt}.\arabic{subsubsection}\hspace*{-3pt}}

\begin{document}
\Rus

\nwt
%\ptb

%\renewcommand{\contentsname}{\protect\Large\bf Содержание}

\setcounter{tocdepth}{2}

%\tableofcontents

\renewcommand{\bibname}{\protect\rmfamily Литература}
  \def\Au#1{{\it #1}}

%\newcommand{\No}{№}
  \newcommand{\tg}{\,\mathrm{tg}\,}
    \newcommand{\ctg}{\,\mathrm{ctg}\,}
  \newcommand{\arctg}{\,\mathrm{arctg}\,}
  
\def\forallb{\mathop{\forall}}
\def\existsb{\mathop{\exists}}

\setcounter{page}{1}

\newpage
\addtocounter{razdel}{1}
%\def\razd{РЕГУЛИРУЕМЫЙ ЭЛЕКТРОПРИВОД ДЛЯ ЭЛЕКТРОЭНЕРГЕТИКИ}
%\newpage
%\def\stat{zakh}
\def\tit{СРЕДСТВА ОБЕСПЕЧЕНИЯ ОТКАЗОУСТОЙЧИВОСТИ ПРИЛОЖЕНИЙ}
\def\titkol{Средства обеспечения отказоустойчивости приложений}

\def\aut{В.\,Н.~Захаров$^1$, В.\,А.~Козмидиади$^2$}
\titel{\razd}{\tit}{\aut}{\titkol}


\Abst{Рассмотрены проблемы построения отказоустойчивых серверов, возникающие в связи с недетерминированностью поведения приложений. Предложена формальная модель, описывающая поведение приложения, основными объектами которой являются ресурсы и события. Предложены алгоритмы протоколирования работы приложения на резервном узле кластера, а также восстановления и продолжения его работы при отказе основного узла. При этом для клиентов сбой остается незаметным, за исключением некоторого увеличения времени обслуживания.}

\KW{сервер приложений $\bullet$ прозрачная отказоустойчивость $\diamond$
 процесс $\diamond$ ресурс $\diamond$ событие $\diamond$ контрольная точка
$\bullet$ детерминированность}

\vskip 12pt plus 6pt minus 3pt

\begin{multicols}{2}

\section*{ВВЕДЕНИЕ}

Средства вычислительной техники стали использоваться в областях,
требующих безотказной работы систем в течение многих лет (24 часа
в сутки, 365 дней в году).

\label{st\stat}

\footnotetext{$^1$ФГУП Центральный институт авиационного моторостроения
им. П.И. Баранова, Москва, Россия}
\footnotetext{$^2$ФГУП Центральный институт авиационного моторостроения
им. П.И. Баранова, Москва, Россия}

К таким областям относятся, например, центры хранения и обработки данных  в сетях (системы резервирования билетов, биллинговые,  банковские и т.д.), массированные распределенные вычисления (GRID-вычисления) и другие.

\thispagestyle{headings}

Обычно в подобных системах применяются частные решения, ориентированные в основном на обеспечение надежного хранения данных (например, файловые серверы, использующие для хранения RAID-контроллеры) и корректного их состояния при отказах (серверы баз данных с транзакционным выполнением запросов). Однако большинство приложений не гарантируют, что не произойдет потери части данных при отказе системы. Обычно предполагается, что клиентские средства должны повторять запросы после восстановления серверов, для того, чтобы данные не были потеряны, или что можно сделать возврат по времени на некоторое время назад и повторить работу с этого места. Однако далеко не все клиентские средства и условия применения приложений допускают это.

Отказоустойчивые системы для критически важных приложений, корректно решающие проблемы восстановления после сбоев,   предлагаемые ведущими производителями, как правило, дороги. Кроме того, они включают специфические серверные и клиентские приложения, не совместимые со стандартными приложениями, не обеспечивающими отказоустойчивость. Примером такого подхода к решению проблемы отказоустойчивости  хранения данных являются системы NetApp FAS компании Network Appliance, работающие на базе специализированной операционной системы Data ONTAP [1].

Построение отказоустойчивых систем, использующих серверы со стандартными приложениями, в свете вышесказанного, является актуальной проблемой, вызывающей значительный интерес. Рассмотрение методов достижения прозрачной отказоустойчивости таких систем и является предметом статьи.
\begin{figure*} %fig1
\vspace*{1pt}
\begin{center}
\mbox{%
\epsfxsize=1.6in
\epsfxsize=100mm
\epsfbox{BbR-1.eps}
}
\end{center}
\vspace*{-9pt}
\Caption{Базовый вариант трубы с разными выходными устройствами
(цилиндрическое, расширяющееся и сужающееся сопло)
\label{f1bab}}
\vspace*{-3pt}
\end{figure*}


\section{ОСНОВНЫЕ ПОНЯТИЯ И ПОДХОДЫ}

Под сервером в данной работе понимается вычислительный центр
(отдельный компьютер или кластер) в сети, предоставляющий клиентам
(пользователям, клиентским компьютерам) определенные услуги, разделяя
между ними свои ресурсы. Подобные серверы названы серверами приложений.
Широко распространенным примером сервера такого типа является файловый сервер, обеспечивающий удаленный коллективный доступ к файловой системе. Часто используются вычислительные серверы, предоставляющие клиентам возможность выполнять на них свои программы (например, в центрах коллективного пользования).


Обычно приложение представляет собой программу или группу программ, работающих в операционной среде, создаваемой операционной системой (в другой терминологии - один или несколько взаимодействующих процессов или потоков (threads)), которые реализуют функциональность сервера. Для построения отказоустойчивых серверов приложений широко используется кластерная технология. Следуя [2], кластером, названа разновидность параллельной или распределенной системы, которая:
\begin{itemize}
\item состоит из нескольких компьютеров (узлов кластера), связанных как минимум необходимыми коммуникационными каналами;
\item используется как единый, унифицированный компьютерный ресурс.
\end{itemize}

Прозрачная отказоустойчивость (Transparent Fault Tolerance, TFT) сервера приложений - это такое его поведение при возникновении аппаратных или программных отказов либо отказов в сети, при котором:
\begin{itemize}
\item отказ не вызывает потери или искажения данных, находящихся в базе данных сервера;
\item сервер продолжает нормально функционировать, несмотря на имевшие место отказы.
\end{itemize}

Клиенты сервера "не замечают" произошедших отказов. Единственным\footnote{допустимым
отклонением сервера от нормального поведения с точки зрения клиента является
некоторое увеличение времени обслуживания} (на несколько секунд или десятков секунд).

Обычно приложения, работающие на серверах приложений, не ориентированы на прозрачную отказоустойчивость. Они "заботятся" лишь о собственной целостности (например, состояния файловой системы или базы данных). Восстановление работоспособности сервера приводит к разрыву соединений с клиентами и потере их запросов. Это замечают клиенты - запросы следует повторять, на что клиентские приложения далеко не всегда рассчитаны. В данной работе предполагается, что приложения (прикладные программные средства), выполняемые на сервере, являются стандартными, то есть не имеют специальных средств, обеспечивающих отказоустойчивость.
\begin{figure*}[b] %fig1
\vspace*{1pt}
\begin{center}
\mbox{%
\epsfxsize=1.6in
\epsfxsize=100mm
\epsfbox{BbR-1.eps}
}
\end{center}
\vspace*{-9pt}
\Caption{Базовый вариант трубы с разными выходными устройствами
(цилиндрическое, расширяющееся и сужающееся сопло)
\label{f1bab}}
\vspace*{-3pt}
\end{figure*}

Серьезные исследования в области обеспечения отказоустойчивости серверов были развернуты после создания вычислительных серверов, предназначенных для решения задач, требующих больших вычислительных ресурсов. Решение этих задач выполняется на суперкомпьютерах, обеспечивающих массово-параллельные вычисления и представляющих собой кластеры из сотен и тысяч узлов (процессоров). Однако даже на этих "монстрах" решение может требовать десятков или сотен часов, и одиночный сбой, если не предприняты специальные меры, может привести к необходимости начинать работу сначала. Обычно решение вычислительной задачи в таких случаях осуществляется в модели относительно редко взаимодействующих между собой процессов, выполняемых на разных узлах кластера. Эти взаимодействия нужны для координации работы процессов, в частности, для обмена данными и промежуточными результатами. Взаимодействия опираются на специальный протокол, называемый MPI (Message-Passing Interface) и представляющий собой стандарт "de facto" [3].

Для преодоления последствий сбоя достаточно давно была разработана и широко применяется технология, опирающаяся на механизм контрольных точек (checkpoints) [4-6]. По этой технологии система должна иметь стабильную память, которая не меняется при отказах. Соответствующие программные средства периодически сохраняют информацию о состоянии процессов приложения в стабильной памяти. Все процессы также имеют доступ к устройству стабильной памяти.  В случае отказа или сбоя, записанная в стабильную память информация используется для повторения вычисления с момента, когда была записана эта информация, то есть выполняется откат назад по времени. Данные, сохранение которых позволяет выполнить откат, называются контрольной точкой. В качестве устройства стабильной памяти может использоваться дисковый том, энергонезависимая оперативная память, память другого узла или узлов кластера. В последнем случае узел, которому требуется сохранить информацию, пересылает ее через быстрый канал связи на другой узел. Стабильная память после отказа одного из узлов должна быть доступной узлу, на котором делается повтор.

Однако решение, опирающееся только на контрольные точки, не является прозрачным, поскольку не скрывает от клиентов факт отказа системы и требует от них выполнения определенных действий. Так как при работе процессы обмениваются сообщениями, возможны два варианта решения проблемы. Первый - все процессы выполняют записи контрольных точек одновременно, что затруднительно. Второй вариант, при несоблюдении синхронности, - возврат в каждом процессе к такому скоординированному набору контрольных точек, при котором невозможна противоречивая ситуация. Такая ситуация возникает, когда один процесс вернулся к контрольной точке, после которой он должен получить сообщение от другого процесса, а этот другой процесс вернулся к точке, которая следует за выдачей этого сообщения. Однако при повторе ожидаемое первым процессом сообщение не поступит. В этом случае  возможен эффект домино, в результате процессы оказываются отброшены как угодно далеко назад.

В этом состоит первая проблема, которую необходимо преодолеть.

Если нужно, чтобы последствия отказа узла не были видны клиенту,  это означает:
\begin{itemize}
\item клиент не должен терять и потом восстанавливать соединения с сервером;
\item клиент не должен повторять свои запросы;
\item клиент не должен повторно получать сообщения, которые он уже получил.
\end{itemize}

Вторая проблема, которую надо решать, связана с недетерминированностью поведения сервера приложений. Приведем пример.  Пусть имеется система продажи билетов на самолеты. Два клиента одновременно обратились к системе с запросом билета на один и тот же рейс. Клиентам безразлично, какие места им зарезервирует система. Система выполняет запросы клиентов параллельно, поэтому в какой-то момент между процессами, обрабатывающими эти запросы, может возникнуть конкуренция за ресурс - в данном случае, скажем, рейс. Один из процессов захватывает ресурс первым, резервирует место и освобождает ресурс. Потом второй процесс проделывает то же самое.

Порядок, в котором в этом примере процессы захватили ресурс, зависит от многих факторов и, в конечном счете, случаен. Однако  это не мешает правильному функционированию системы, поскольку клиентам важно одно - получить билеты, причем на разные места. Однако отсутствие детерминизма в поведении приложения приводит к тому, что при повторном выполнении могут быть получены другие результаты: например, клиенту уже сообщено, что ему зарезервировано место №5, а при повторе может получиться, что зарезервировано место №6. Система должна устранить это несоответствие и сделать его невидимым для клиента.
\begin{figure*} %fig1
\vspace*{1pt}
\begin{center}
\mbox{%
\epsfxsize=1.6in
\epsfxsize=100mm
\epsfbox{BbR-1.eps}
}
\end{center}
\vspace*{-9pt}
\Caption{Базовый вариант трубы с разными выходными устройствами
(цилиндрическое, расширяющееся и сужающееся сопло)
\label{f1bab}}
\vspace*{-3pt}
\end{figure*}

Недетерминированность поведения системы это следствие, по крайней мере, двух обстоятельств. Во-первых, это присущая системам с разделением времени неопределенность в порядке выполнения процессов. Во-вторых, это конкуренция процессов за общие ресурсы. Перечислим некоторые причины недетерминированного поведения приложений:
\begin{itemize}
\item синхронизация процессов с помощью семафоров или атомарных операций над операндами в общей памяти процессов;
\item зависимость от порядка получения клиентских запросов;
\item время, затраченное процессом на обработку полученного запроса;
\item генераторы случайных чисел;
\item системное управление процессами и потоками;
\item локальные таймеры;
\item доступ к реальному времени.
\end{itemize}

По различным  причинам время, которое тратится на выполнение вычислительной задачи с одними и теми же исходными данными, не является константой, то есть повторное выполнение может дать другое время. Процессы используют общие ресурсы, обращение к которым требует организации очередности выполнения (сериализации) - первым пришел, первым захватил. И, наконец,  результат работы процесса может зависеть от состояния ресурса, а это состояние может изменить другой процесс, ранее захвативший ресурс. Все это создает значительные трудности при попытках воспроизведения поведения процессов с сохраненной контрольной точки.

Прозрачная отказоустойчивость серверов приложений обычно осуществляется переносом приложения на другой узел кластера, идентичный первому по конфигурации аппаратных средств и операционной среды. Это делается методом, называемым snapshot/restore. На основном узле (оригинале)  периодически фиксируется состояние приложения на этом узле кластера (так называемый снимок или snapshot). После отказа оригинала на резервном узле (копии) делается восстановление (restore), то есть восстанавливается последнее зафиксированное состояние приложения. Операционная среда при этом приводится в состояние, которое соответствует моменту изготовления снимка. После этого узел-копия продолжает работу с зафиксированного места. Сравнение метода  snapshot/restore с другими подходами приведено в [7].

Ниже рассматриваются информационные  технологии, позволяющие решить ряд принципиальных вопросов, связанных с реализацией прозрачной отказоустойчивости серверов приложений. Ими являются:
\begin{itemize}
\item виртуализация операционной среды, в которой работает серверное приложение;
\item отказоустойчивая реализация протокола TCP;
\item создание контрольных точек состояния приложения и файловой системы, которые делаются внешним по отношению к приложению образом;
\item восстановление серверного приложения на основании контрольной точки.
\end{itemize}
\begin{figure*} %fig1
\vspace*{1pt}
\begin{center}
\mbox{%
\epsfxsize=1.6in
\epsfxsize=100mm
\epsfbox{BbR-1.eps}
}
\end{center}
\vspace*{-9pt}
\Caption{Базовый вариант трубы с разными выходными устройствами
(цилиндрическое, расширяющееся и сужающееся сопло)
\label{f1bab}}
\vspace*{-3pt}
\end{figure*}

\section{МОДЕЛЬ ОПИСАНИЯ ПОВЕДЕНИЯ ПРИЛОЖЕНИЯ}

Предлагаемый подход опирается на построение модели вычислений, связанной с использованием понятия времени в многопроцессных приложениях. Впервые подобные проблемы были изучены в классической работе Л. Лампорта [8].

Многопроцессными приложения называются потому, что в них параллельно работают несколько процессов. Процесс ведет себя детерминированно, пока в предписанном кодом порядке выполняет процессорные инструкции. Конечно, его работа может быть прервана практически в любой момент и процессор передан другому процессу или ядру. Поэтому абсолютное время, которое затрачивает процесс на выполнение определенной работы, не  константа, а случайная  величина. То же  относится к относительному времени, то есть времени, которое процесс занимал процессор,  поскольку одни и те же обращения к операционной среде могут вызвать работы разной длительности, а значит потребовать разное время на свое выполнение.

Кэшированность инструкций и данных, а также длина хэш-списков влияют на действительное время пребывания в операционной среде. Утрачивает смысл понятие одновременность действий, поскольку  нельзя установить, выполнили ли два разных процесса какие-либо действия одновременно или одно из них предшествовало другому. Таким образом, с процессом можно связать только его локальное время, которое линейно упорядочивает события,  происходившие в этом процессе.  Глобальное время, линейно упорядочивающее действия во всех процессах, отсутствует. Расстояние (в этом качестве используется время) между действиями оказывается случайной величиной.

Эти соображения важны, поскольку процессы в интересующих нас приложениях взаимодействуют и используют общие ресурсы. Для взаимодействия они используют средства синхронизации, предоставляемые операционной средой - например, наборы семафоров SVR4 (System V Release 4), POSIX-семафоры, бинарные семафоры и другие примитивы взаимного исключения (POSIX- mutual exclusion locks) и т.д. Подобные средства операционной среды, которые позволяют процессам синхронизировать свою деятельность друг с другом или сериализовать обращения к совместно используемым объектам,  будут ниже  называться ресурсами.

С каждым ресурсом связано свое локальное время, линейно упорядочивающее события в жизни ресурса. Например, в случае двоичных семафоров это создание семафора, а также его захват и освобождение процессом. Заметим, что событие - это не намерение процесса (например, захватить бинарный семафор), а сам факт захвата семафора процессом (т.е. успешное выполнение намерения). От изъявления намерения до его осуществления может многое произойти. Например, семафор, который хочет захватить рассматриваемый процесс, принадлежал другому процессу, потом тот процесс его освободил, но семафор был сначала передан операционной средой третьему процессу, который также на него претендовал, и т.д. Поведение рассматриваемого процесса в это время нас не интересует - он ресурсом еще не овладел, а только его захват определяет его дальнейшее поведение. По причинам,  изложенным выше, расстояние между двумя событиями - случайная величина. Однако, события замечательны тем, что они одновременно присутствуют и в локальном времени процесса, и в локальном времени ресурса. Поэтому все, что произошло в истории процесса или/и ресурса до этого события, предшествует ему. Далее  будет считаться, что истории и ресурсов и процессов состоят только из событий, причем между двумя последовательными событиями в жизни процесса последний ведет себя детерминированно.

Это означает, что на  поведении процесса сказывается только его предыдущая история, то есть состояние ресурсов, с которыми он взаимодействовал. Это свойство процессов ниже будет называться локальной детерминированностью. Этим свойством не обладают ресурсы, поскольку - следующее событие в истории ресурса не определяется однозначно по его предыдущей истории. Утверждение, касающееся детерминированного поведения процессов, неявно опирается на предположение,  что учтены все ресурсы, которые могут привести к  недетерминированности процессов.

Таким образом, описанное нами очень неформально время в многопроцессном комплексе представляет собой отношение частичного порядка, введенное на множестве событий. Зная полное состояние комплекса в некоторый момент времени,  нельзя однозначно определить, какое событие в истории ресурса наступит следующим. Можно говорить только о вероятности наступления того или иного события. Недетерминированность поведения есть следствие двух обстоятельств. Во-первых, это неопределенность времени, которое тратит процесс на переход от одного события к другому. Во-вторых, конкуренция процессов за общие ресурсы.

Выполнение приложения, на множестве событий которого введена частичная упорядоченность, можно описать направленным ациклическим графом выполнения. Вершинами этого графа являются события, с каждым  из которых связаны две входящие в него дуги. Одна дуга начинается в событии, которое непосредственно предшествует данному событию в истории процесса, другая - в истории ресурса.

Построение средств обеспечения прозрачной отказоустойчивости приложений опирается на следующее утверждение: для восстановления работы приложения после отказа достаточно располагать:
\begin{itemize}
\item контрольной точкой, которая отражает на некоторый момент времени состояния процессов и других ресурсов, образующих приложение;
\item графом выполнения приложения, который описывает работу приложения, начинающуюся с контрольной точки и заканчивающуюся отказом. Данные, которые нужны для построения графа выполнения, далее называются протоколом.
\end{itemize}
\begin{figure*} %fig1
\vspace*{1pt}
\begin{center}
\mbox{%
\epsfxsize=1.6in
\epsfxsize=100mm
\epsfbox{BbR-1.eps}
}
\end{center}
\vspace*{-9pt}
\Caption{Базовый вариант трубы с разными выходными устройствами
(цилиндрическое, расширяющееся и сужающееся сопло)
\label{f1bab}}
\vspace*{-3pt}
\end{figure*}

Вся эта информация должна находиться в стабильной памяти, не разрушающейся при отказе.

Ниже неформально описан алгоритм восстановления работы приложения после отказа, который опирается на наличие контрольной точки и графа выполнения. Будем считать, что в распоряжении имеются средства, позволяющие остановить процесс в тот момент, когда он намерен совершить некоторую операцию над ресурсом. Заметим, что событие в графе выполнения соответствует не изъявлению намерения, а его удовлетворению, то есть завершению выполнения операции.

Предварительно сделаем следующее:
\begin{itemize}
\item используя контрольную точку, приведем приложение в состояние, соответствующее этой контрольной точке;
\item в графе выполнения пометим все вершины (события) как "не наступившие". У некоторых вершин графа отсутствуют им непосредственно предшествующие; соответствующие события наступили сразу же после создания контрольной точки. Для каждой такой вершины включим в граф дополнительную вершину, ей предшествующую в истории процесса, и отметим эту дополнительную вершину как "наступившую";
\item разрешим процессам приложения выполняться.
\end{itemize}

Пусть некоторый процесс проявляет намерение выполнить операцию над каким-либо ресурсом. Отыщем для этого процесса в его истории последнее наступившее событие. Следующее в его истории событие - это то, которое соответствует требуемой операции. Посмотрим, наступило ли событие в истории ресурса, которое ему предшествует. Если нет, переведем процесс в состояния ожидания, отметив в предшествующем событии, что данный процесс ожидает его наступления. Если да, разрешим процессу выполняться, то есть выполнить операцию над ресурсом.

Пусть некоторый процесс объявляет, что он выполнил операцию над каким-либо ресурсом (это соответствует моменту протоколирования при оригинальном выполнении). Отыщем для этого процесса в его истории последнее наступившее событие и перейдем к следующему событию в его истории. Это опять то событие, которое мы рассматриваем. Отметим его как "наступившее". Если наступления этого события ожидал какой-нибудь процесс, выведем этот процесс из состояния ожидания. Наконец, разрешим процессу, выполнившему операцию, продолжаться дальше.

Когда выясняется, что наступили все события графа выполнения, повторное выполнение считается законченным.

Естественным следствием из сказанного является следующее утверждение: для того, чтобы размер протокола не рос неограниченно, нужно периодически создавать контрольные точки, очищая при этом протокол.

\section{ФОРМАЛЬНОЕ ОПИСАНИЕ МОДЕЛИ ПОВЕДЕНИЯ МНОГОПРОЦЕССНОГО ПРИЛОЖЕНИЯ}
\begin{figure*} %fig1
\vspace*{1pt}
\begin{center}
\mbox{%
\epsfxsize=1.6in
\epsfxsize=100mm
\epsfbox{BbR-1.eps}
}
\end{center}
\vspace*{-9pt}
\Caption{Базовый вариант трубы с разными выходными устройствами
(цилиндрическое, расширяющееся и сужающееся сопло)
\label{f1bab}}
\vspace*{-3pt}
\end{figure*}

Опишем формально поведение приложения, неформальное описание которого было приведено выше. Рассматриваются два типа объектов:
\begin{itemize}
\item ресурсы (r), например, наборы семафоров (POSIX- или SVR4-семафоры), бинарные семафоры (POSIX-mutex's), таймер реального времени, сокеты (sockets), то есть двусторонние виртуальные соединения с внешним миром;
\item процессы (p), например, процессы или потоки (threads) пользователя.
\end{itemize}

\end{multicols}

\label{end\stat}

%\def\stat{batr}

\def\tit{НОВЫЙ МЕТОД ВЕРОЯТНОСТНО-СТАТИСТИЧЕСКОГО\newline
АНАЛИЗА ИНФОРМАЦИОННЫХ ПОТОКОВ
В~ТЕЛЕКОММУНИКАЦИОННЫХ СЕТЯХ$^*$}
\def\titkol{Новый метод вероятностно-статистического
анализа информационных потоков
в~телекоммуникационных сетях}
\def\autkol{Д.\,А.~Батракова, В.\,Ю.~Королев, С.\,Я.~Шоргин}
\def\aut{Д.\,А.~Батракова$^1$, В.\,Ю.~Королев$^2$, С.\,Я.~Шоргин$^3$}

\titel{\tit}{\aut}{\autkol}{\titkol}

{\renewcommand{\thefootnote}{\fnsymbol{footnote}}\footnotetext[1]{Работа 
выполнена при поддержке РФФИ, проекты №№\,04-01-00671, 05-07-90103.} 
\renewcommand{\thefootnote}{\arabic{footnote}}}
 \footnotetext[1]{ИПИ РАН, 
daria.batrakova@gmail.com} \footnotetext[2]{Факультет вычислительной математики 
и кибернетики МГУ им.~М.\,В.~Ломоносова, ИПИ РАН, vkorolev@comtv.ru} 
\footnotetext[3]{ИПИ РАН, sshorgin@ipiran.ru}



\Abst{В данной работе предлагается метод исследования стохастической структуры
хаотических информационных потоков в сложных телекоммуникационных
сетях. Предлагаемый метод основан на стохастической модели
телекоммуникационной сети, в рамках которой она представляется в виде
суперпозиции некоторых простых последовательно-параллельных структур.
Эта модель естественно порождает смеси гамма-распределений для времени
выполнения (обработки) запроса сетью. Параметры получаемой смеси
гамма-распределений характеризуют стохастическую структуру
информационных потоков в сети. Для решения задачи статистического
оценивания параметров смесей экспоненциальных и гамма-распределений
(задачи разделения смесей) используется ЕМ-алгоритм. Чтобы проследить
изменение стохастической структуры информационных потоков во времени,
ЕМ-алгоритм применяется в режиме скользящего окна. Описывается
программный инструментарий для применения полученного решения к
реальным статистическим данным. Приводится интерпретация результатов.}

\KW{телекоммуникационные сети; информационные потоки;
разделение смесей  распределений;
метод скользящего окна;  программа для разделения смесей}

\vskip 24pt plus 9pt minus 6pt

\thispagestyle{headings}

\begin{multicols}{2}


\label{st\stat}

\section{Введение}

Развитие телекоммуникационных сетей, их усложнение поставило перед
инженерами важную прикладную задачу исследования характеристик
информационных потоков, возникающих в этих сетях. Здесь под
информационным потоком мы будем понимать упорядоченное движение
любого вида информации по сети.

Если на заре эры телекоммуникаций, в эпоху первых телефонных линий и
телеграфа эта проблема не была столь насущной, то со временем, при
постепенном охвате мирового пространства сетями возникла необходимость в
построении и исследовании адекватных моделей сетей и процессов,
происходящих в них.

\thispagestyle{headings}


Современные сети~--- это \textit{конвергентные} сети, т.е.\ совокупность крайне
разнородных как по топологии, так и по физической архитектуре сетей, которые
предлагают конечному пользователю самые разнообразные сервисы. Это~--- огромное
виртуальное и физическое пространство, состоящее из миллионов процессоров,
операционных платформ, линий передачи данных и стыковочных узлов.
%
Существует множество классификаций телекоммуникационных сетей по различным
признакам:
\begin{itemize}
\item масштабу (локальные сети~--- LAN, масштаба города~---
MAN, широкого масштаба~--- WAN);
\item топологии, или логической организации (<<звезда>>,
<<кольцо>>, <<шина>>);
\item физической организации (оптические сети, радио);
\item предлагаемым услугам (сотовые сети, для доступа в
Интернет);
\item назначению (военные, гражданские) и~др.
\end{itemize}


Конвергентная сеть входит во все эти классы, причем, как правило,
обладает всеми этими признаками. Поэтому построение модели для ее анализа
является и очень важной, и очень сложной задачей.

Существуют достаточно многочисленные математические методы, ориентированные на
моделирование и анализ телекоммуникационных сетей. В~большинстве своем они
основываются на теории массового обслуживания, то есть разделе теории
вероятностей, посвященном описанию функционирования сложных систем обслуживания
(в том чис\-ле телекоммуникационных сетей и систем) с помощью стохастических
процессов особого вида и анализу таких процессов. Указанная теория является
весьма развитой и широко применяется на практике. Тем не менее, ее применимость
ограничена~--- во-первых, все возрастающей сложностью структур и дисциплин
обслуживания в рас\-смат\-ри\-ва\-емых реальных сетях. Эта сложность во многих
случаях принципиально не может найти адекватного отображения в моделях
массового обслуживания, даже несмотря на постоянно растущую сложность самих
этих моделей. В результате даже модели, допускающие точный математический
анализ, дают возможность расчета всего лишь приближенных значений характеристик
реальных сетей, ибо предположения, принимаемые при построении моделей, во
многих случаях не соответствуют практике. Во-вторых, для описания
телекоммуникационной сети в виде сети массового обслуживания исследователь
должен располагать детальным описанием структуры сети, что далеко не всегда
имеет мес\-то на практике. В-третьих, разработано крайне мало моделей массового
обслуживания, в которых используется в качестве входной информация о
наблюдаемых (статистических) показателях функционирования сети; в то же время,
такая информация очень часто доступна исследователю, и ее использование при
анализе сети весьма целесообразно.

В данной работе предлагается в определенной степени альтернативный подход к
моделированию сложных телекоммуникационных сетей. Строится и исследуется
вероятностная модель сложной телекоммуникационной сети как суперпозиции
достаточно простых структур. При этом практически никакая априорная информация
о структуре исследуемой сети не используется~--- наоборот, в результате
исследования модели исследователь получает приближенное представление об этой
структуре. Характеристики типовых простых структур, составляющих в совокупности
модель сети, и сети в целом при этом подходе описываются
гам\-ма-рас\-пре\-де\-ле\-ни\-я\-ми. Ставится задача оценки параметров модели
на основе статистических данных о функционировании сети, а также предлагается
математическое решение этой задачи. В статье описан также созданный на основе
разработанных математических методов программный инструментарий и приведены
результаты расчетов для реального трафика. {\looseness=-1

}

\section{Математическая модель и~постановка задачи}

\subsection{Логическая модель сети}
 %1.1

Рассмотрим абстрактную \textit{конвергентную телекоммуникационную
сеть}. Это может быть как крупномасштабная транспортная сеть (WAN), сеть
отдельного оператора масштаба города (MAN) с различными сервисами, так и
локальная сеть (LAN).

Любой из этих случаев можно описать как ($p,\,q$)-\textit{сеть}.

\medskip
\textbf{Определение 1.} В теории графов и сетей под ($p,\,q)$-сетью понимается
набор вида $S =$\linebreak $=(G,\,V^\prime ,\,V^{\prime\prime})$, где $G$~---
граф, а $V^\prime$ и $V^{\prime\prime}$~--- выборки из множества $V(G)$ (вершин
графа) длины~$p$ и $q$ соответственно. При этом выборка $V^\prime$
($V^{\prime\prime}$) считается \textit{входной} (\textit{выходной}) выборкой, а
ее $i$-я вершина называется $i$-\textit{м} \textit{входным} (\textit{выходным})
\textit{полюсом} или, иначе, $i$-\textit{м} \textit{входом} (\textit{выходом})
сети~$S$. Вершины, не участвующие во входной и выходной выборках сети,
считаются ее внутренними вершинами~\cite{1bat}.

Сеть $S$ (рис.~\ref{f1bat}) имеет $p$ точек входа~--- точек соединения
с внешней средой (это могут быть точки стыковки разнородных сетей, сетей
различных операторов, физические подключения к интерфейсам
маршрутизаторов и~т.п.). Под \textit{внешней средой} будем понимать другие
сети, которые передают данные в сеть~$S$. Отдельные <<единицы>> данных
(кадры, сообщения, датаграммы, пакеты) поступают на входы сети~$S$,
обрабатываются и подаются на каждый из $q$ выходов, которые могут быть
соединены как с конечными пользователями, так и с другими сетями.
\begin{figure*} %fig1
\vspace*{1pt}
\begin{center}
\mbox{%
\epsfxsize=139.7mm \epsfbox{bat-1.eps}
%\epsfxsize=139.698mm
%\epsfbox{bek-3.eps}
}
\end{center}
\vspace*{-9pt} \Caption{Абстрактная телекоммуникационная сеть \label{f1bat}}
\end{figure*}

Следует отметить, что структура сложных телекоммуникационных сетей обладает
свойством некоторого самоподобия, т.е.\ на каком бы уровне сетевой архитектуры
мы ни рассматривали поведение информационных потоков, мы можем выделить
некоторые базовые структуры, субпотоки, суперпозицией которых мы можем получить
модель конкретной сети, какой бы уровень <<детализации>> сегментов сети мы ни
взяли. Так, например, физические подключения к интерфейсам оптического
кросс-коннекта в этом смысле подобны <<виртуальным>> подключениям к портам TCP
на сервере приложений.

Итак, независимо от уровня сетевой архитектуры мы можем
рассматривать некоторую величину, характеризующую количество каких-либо
ресурсов сети~$S$, занимаемых в процессе передачи и обработки данных.  Это
могут быть ресурсы, относящиеся как к <<объему>> (памяти сетевого
устройства, количеству занятых линий, размеру пакета), так и ко <<времени>>
(времени обслуживания заявки, времени простоя). Эта величина случайна, т.к.\
мы не можем абсолютно точно сказать для сложной телекоммуникационной
сети, какое сообщение на какой из входов поступит и какого размера оно будет.
Таким образом, случайный характер данной величины определяется
случайностью поведения внешней среды.

Пусть $R$~--- это описанная выше случайная величина, $R>0$. Далее, не
ограничивая общности, будем подразумевать под ней время, необходимое для
какой-либо операции сети (обработки <<единицы>> данных), предполагая, что
время обработки прямо зависит от объема сообщения.

\subsection{Вероятностная модель сети} %1.2.

Даже не зная реальной топологии сети, мы можем описать
функционирование некоторых ее участков как процесс выполнения операций
(задач сети) в последовательном  порядке (например, если доступен только
один канал данных) или как процесс одновременного выполнения субопераций
(когда доступно более одного пути выполнения). Это значит, что мы можем
представить функционирование сложной телекоммуникационной сети как
\textit{суперпозицию} таких <<последовательных>> и <<параллельных>>
блоков.

Для построения вероятностной модели распределения~$R$ используется
комбинация асимптотического подхода, основанного на предельных теоремах
теории вероятностей, и принципа максимальной неопределенности (энтропии).

Рассмотрим следующую модель. Предположим, что мы можем разделить
сеть~$S$ на несколько сегментов $S_i$. Пусть $T$~--- случайная величина,
время выполнения операции в отдельно взятом блоке $S_i$ (сегменте сети).

Если операции выполняются \textit{параллельно}, то время, необходимое
для их выполнения~--- это максимальное время, затрачиваемое на какую-либо
субоперацию:
$$
T = \underset{i}{\max}\, T_i\,,
$$
где $T_i$~--- случайные величины для со\-от\-вет\-ст\-ву\-ющих субопераций.
Модель такого выполнения пред\-став\-ле\-на на рис.~\ref{f2bat}.

\begin{figure*} %fig2
\vspace*{1pt}
\begin{center}
\mbox{%
\epsfxsize=117.271mm
\epsfbox{bat-2.eps}
}
\end{center}
\vspace*{-9pt}
\Caption{Параллельное выполнение
\label{f2bat}}
\end{figure*}

Известно, что предельное распределение экстремальных значений для
выборок ~--- это экспоненциальное распределение с плотностью~\cite{2bat}
$$
f(x) =
\begin{cases}
\lambda e^{-\lambda x}\,, & x>0\,,\\
0\,, & x\leq 0\,,
\end{cases}
$$
где $\lambda >0$~--- параметр масштаба.

 Учитывая также энтропийный подход, естественно будет считать
распределение $T$ экспоненциальным, т.к.\ экспоненциальное распределение
обладает наибольшей энтропией среди всех распределений с $x>0$.

Если же операции сети выполняются \textit{последовательно}, то величина
$T$~--- это сумма времен $T_i$, необходимых для выполнения каждой
субоперации:
$$
T = \sum\limits_i T_i\,,
$$
где $T_i$~--- случайные величины для со\-от\-вет\-ст\-ву\-ющих субопераций.
%
Такая модель представлена на рис.~\ref{f3bat}.

\begin{figure*} %fig3
\vspace*{1pt}
\begin{center}
\mbox{%
\epsfxsize=139.592mm
\epsfbox{bat-3.eps}
}
\end{center}
\vspace*{-9pt}
\Caption{Последовательное  выполнение
\label{f3bat}}
\end{figure*}

Это значит, что распределение общей длительности $T$ выполнения
блока~--- это свертка распределений <<элементарных>> времен $T_i$
(экспоненциально распределенных).

Известно, что результатом свертки экспоненциальных распределений
является гамма-распределение, определяемое плотностью
$$
\g_{\lambda , \alpha} (x) =
\begin{cases}
\fr{\lambda_0^{\alpha_0}}{\Gamma (\alpha_0 )}\,x^{\alpha_0-1}
e^{\lambda_0 x}\,, & x>0\,,\\
0,\, & x\leq 0\,,
\end{cases}
$$
где $\alpha >0$~--- параметр формы,  $\lambda >0$  параметр масштаба, а
$\Gamma (z)$~--- гамма-функция Эйлера:
$$
\Gamma (z) = \int\limits_0^\infty x^{z-1} e^{-x}\,dx\,.
$$

\begin{figure*} %fig4
\vspace*{1pt}
\begin{center}
\mbox{%
\epsfxsize=120.831mm
\epsfbox{bat-4.eps}
}
\end{center}
\vspace*{-9pt}
\Caption{Модель пути  обработки сообщения сетью~$S$
\label{f4bat}}
\end{figure*}

Известно~\cite{2bat}, что класс гамма-распределений замкнут над операцией
свертки, поэтому ре\-зуль\-ти\-ру\-ющее распределение случайной величины~$R$
будет также гамма-распределением
$$
\g_{\lambda , \alpha} (x) =
\begin{cases}
\fr{\lambda^{\alpha}}{\Gamma (\alpha )}\,x^{\alpha -1} e^{-\lambda x}\,, &
x>0\,,\\
0,\, & x\leq 0\,.
\end{cases}
$$

В силу случайного характера ввода данных в сеть~$S$ из внешней среды маршрут
передачи данных становится случайным, что представлено на рис.~\ref{f4bat}. Это
означает, что параметры ре\-зуль\-ти\-ру\-юще\-го распределения~$R$ тоже
случайны. Отсюда имеем следующую модель \textit{смеси
гам\-ма-рас\-пре\-де\-ле\-ний}, определяемой плотностью

\begin{equation} %1
p(x) = \iint \g_{\lambda , \alpha}(x)\,dH (\lambda ,\,\alpha )\,,
\end{equation}
где $H(\lambda , \alpha )$~--- смешивающая функция, функция распределения
входных параметров.

Поясним понятие \textit{смеси распределений}.

\medskip
\textbf{Определение~2.} Пусть имеется двух\-па\-ра\-мет\-ри\-че\-ское
семейство $n$-мерных плотностей  распределения
\begin{equation}
F = \{ f_\omega (x;\, \theta (\omega ))\}\,,
\end{equation}
где одномерный (целочисленный или непрерывный) параметр $\omega$ в
качестве нижнего индекса функции $f$ определяет специфику общего вида
каж\-до\-го компонента~--- распределения смеси, а в качестве аргумента при
многомерном, вообще говоря, параметре $\theta$ определяет зависимость
значений хотя бы части компонентов этого параметра от того, в каком именно
составляющем распределении $f_\omega$ он присутствует. Кроме того, пусть
$P = \{P(\omega )\}$~--- \textit{семейство смешивающих функций}
распределения.

Функция плотности распределения
\begin{equation}
f(x) = \int f_\omega (x;\,\theta(\omega ))\,dP (\omega )
\end{equation}
называется $P$-\textit{смесью} (или просто \textit{смесью})
\textit{распределений} семейства~$F$,  интеграл в~(3) понимается в смысле
Лебега--Стильтьеса~\cite{3bat}.

\medskip
\textbf{Определение 3.} Семейство смесей~(3) называется
\textit{идентифицируемым} (\textit{различимым}), если из равенства
$$
\int f_\omega (x;\,\theta(\omega ))\,dP (\omega ) =\int f_\omega
(x,\,\theta(\omega )) dP^*(\omega )
$$
следует, что $P(\omega ) \equiv P^*(\omega )$ для всех $P \in P(\omega
)$~\cite{3bat}.

\subsection{Постановка задачи} %1.3.

Перед нами встает задача \textit{разделения} такой смеси. Вообще говоря,
задача разделения смесей распределений со смешивающими функциями
общего вида является \textit{некорректно поставленной}, т.к.\ она допускает
существование нескольких решений. Поэтому будем искать решение в классе
\textit{конечных идентифицируемых смесей распределений}, где смешивающая
функция дискретна.

Для этого сузим данное выше определение и будем рассматривать в дальнейшем лишь 
случай конечного числа $k$ возможных значений па\-ра\-мет\-ра~$\omega$, что 
соответствует конечному числу скачков смешивающих функций $P(\omega )$.  
Величины этих скачков как раз и будут играть роль \textit{удельных весов} 
(\textit{априорных вероятностей}) $p_j$ компонентов смеси. Более того, в нашем 
случае мы постулируем также однотипность компонентов распределений $f_j$, т.е.\ 
принадлежность всех $f_j$ к одному общему па\-ра\-мет\-ри\-че\-ско\-му 
семейству $\{ f(X;\,\theta )\}$, где $\theta$~--- многомерный, вообще говоря, 
параметр. Так что~(3) в этом случае может быть записано в виде
\begin{equation} %4
p(x) = \sum\limits^k_{j=1} p_j f_j (x;\,\theta_j )\,.
\end{equation}

Переформулируем понятие идентифицируемости (различимости) смесей
специально применительно к такому виду смесей.

\medskip
\textbf{Определение 4.} \textit{Конечная смесь}~(3) называется
\textit{идентифицируемой} (\textit{различимой}), если из равенства
$$
\sum\limits_{j=1}^k p_j f_j (x;\,\theta_j ) = \sum\limits_{l=1}^{k^*} p_l^* f_l
(x;\,\theta_l^* )
$$
следует, что $k=k^*$ и для любого $j$ ($1\leq j \leq k$) найдется такое $l$ 
($1\leq l \leq k^*$), что $p_j = p_l^*$ и $f_j (x;\,\theta_j ) = f_l 
(x;\,\theta_l^* )$~\cite{3bat}.

Решить эту задачу в выборочном варианте~--- значит по выборке
классифицируемых наблюдений
$X_1,\ldots , X_n, $ извлеченной из генеральной совокупности, яв\-ля\-ющей\-ся смесью~(3)
генеральных совокупностей типа~(2) (при заданном общем виде составляющих
смесь функций $f_j (x;\,\theta_j )$), построить статистические оценки для числа
компонентов смеси~$k$, их удельных весов $p_j$ и, главное, для каждого из
компонентов %f_j (x;\,\theta_j )$ анализируемой смеси. Далее будет считать, что
функции $f_j$ однозначно определяются своими параметрами $\theta_j$: $f_j
=f(x;\,\theta_j)$.

Однако не следует ставить знак тождества между задачей разделения смеси
и задачей статистического оценивания параметров в модели~(4) по выборке $
X_1,\ldots , X_n$, поскольку задача разделения сохраняет смысл и
применительно к генеральным совокупностям, т.е.\ в теоретическом
варианте~\cite{3bat}.

Итак, для статистического анализа на основе реальных данных мы
аппроксимируем нашу общую модель~(1) следующей:
$$
p(x) \approx \hat{p}(x) = \sum\limits_{j=1}^k p_j \g_{\lambda_j , \alpha_j}
(x)\,,
$$
где $p_j$~--- дискретные смешивающие параметры, $\g_{\lambda_j , \alpha_j}
(x)$~--- плотности гамма-распределений.

Такая аппроксимация не только позволяет решить поставленную статистическую
задачу, но и полу\-чить наглядную визуализацию результатов. Существуют
достаточно эффективные методики разделения смесей распределений, среди них~---
семейство так называемых \textit{ЕМ-алгоритмов}
(\textit{Expectation-Maximization Algorithms}).

Полученные результаты могут быть достаточно просто интерпретированы. Число
компонентов смеси символизирует число типичных параллельных или
последовательных структур. Значения параметров составляющих смесь
гам\-ма-рас\-пре\-де\-ле\-ний показывают <<степень параллелизма>>
соответствующей структуры: чем ближе параметр формы к~1, тем выше эта
<<степень>>. И наоборот, чем дальше значение параметра формы от~1, тем больше
последовательных операций выполняется в соответствующем блоке.

Веса компонентов характеризуют примерную долю использования
ресурсов для сообщений, соответствующих каждому распределению входных
данных.

Итак, предложенный подход позволяет получить представление о
стохастической структуре телекоммуникационной сети.

\section{ЕМ-алгоритм разделения смесей распределений}

\subsection{Описание алгоритма} %2.1.

Определяемый ниже итерационный алгоритм решения поставленной в
предыдущем разделе задачи относится к процедурам, базирующимся на
\textit{методе максимального правдоподобия}.

Этот алгоритм позволяет находить максимум логарифмической функции
правдоподобия по параметрам $p_1,\,p_2,\ldots ,\,p_k$, $\theta_1 ,\,\theta_2,\ldots ,\,
\theta_k$ при фиксированном $k$ по выборке $X_1, \ldots , X_n$, т.е.\ решение
оптимизационной задачи вида

\begin{equation} \sum\limits_{i=1}^n \ln \left ( \sum\limits_{j=1}^k p_j f_j
(X_i;\,\theta_j )\right ) \rightarrow \underset{p_j,\,\theta_j}{\max}\,.
\end{equation}

Конкретные алгоритмы, построенные по этой схеме, часто называют
\textit{алгоритмами типа ЕМ}, поскольку в каждом из них можно выделить два
этапа, находящихся по отношению друг к другу в последовательности
итерационного взаимодействия: \textit{оценивание} (\textit{Estimation}) и
\textit{максимизация} (\textit{Maximization})~\cite{4bat}.

Введем в рассмотрение так называемые апостериорные вероятности
$\g_{ij}$ принадлежности наблюдения $X_i$ к $j$-му классу:
\begin{equation} %6
\g_{ij} = \fr{p_j f(X_i;\,\theta_j )}{\sum\limits_{l=1}^k p_l f(X_i;\,\theta_l 
)} \ (i=1,\ldots , n;\ j=1,\ldots ,k)\,.\!\!\end{equation} 
Очевидно, что для 
всех $i=1,\ldots ,n$ и $j=1,\ldots ,k$
$$
\g_{ij} \geq 0,\quad \sum_{j=1}^k \g_{ij} =1\,.
$$


Далее обозначим $\Theta = (p_1,\ldots p_k,\,\theta_1,\ldots ,\theta_k )$ и
представим анализируемую логарифмическую функцию правдоподобия
$$
\ln L(\Theta ) = \sum\limits_{i=1}^n \ln \left (\sum\limits_{j=1}^k p_j f_j
(X_i;\,\theta_j )\right )
$$
в виде
\begin{multline}
\ln L (\Theta ) = \sum\limits_{j=1}^k\sum\limits_{i=1}^n \g_{ij} \ln p_j+{}\\
{}+\sum\limits_{j=1}^k\sum\limits_{i=1}^n \g_{ij} f(X_i;\,\theta_j)-
\sum\limits_{j=1}^k\sum\limits_{i=1}^n \g_{ij} \ln \g_{ij}\,.
\end{multline}

Справедливость этого тождества легко проверяется с учетом
$$
\sum\limits_{j=1}^k \g_{ij} =1\,.
$$

Далее идея построения итерационного алгоритма вычисления оценок
$\hat{\Theta} = (\hat{p}_1,\ldots , \hat{p}_k,\
\hat{\theta}_1,\ldots , \hat{\theta}_k)$
для параметров $\Theta = (p_1,\ldots , p_k,\ \theta_1,\ldots , \theta_k)$ состоит в
следующем:
\begin{enumerate}[1.]
\item Выбирается некоторое \textit{начальное приближение}~$\hat{\Theta}^0$.
\item \textbf{E-step:} вычисляются по формулам~(6) начальные приближения
$\g_{ij}^0$ для апостериорных вероятностей $\g_{ij}$~--- \textit{этап
оценивания}.
\item \textbf{M-step:} затем, возвращаясь к~(7), при вычисленных значениях
$\g^0_{ij}$ следует определить значения $\hat{\Theta}^1$ из условия
максимизации отдельно каждого из первых двух слагаемых правой
части~(7), поскольку первое слагаемое
$$
\sum_{j=1}^k \sum_{i=1}^n \g_{ij} \ln p_j
$$
зависит только от параметров $p_j$, а второе слагаемое
$$
\sum_{j=1}^k \sum_{i=1}^n \g_{ij} f(X_i;\,\theta_j )
$$
зависит только от параметров $\theta_j$~--- \textit{этап максимизации}.
\item Проверяется \textit{условие останова}:
$$
\parallel \Theta^{(t)} - \Theta^{t-1}\parallel <\varepsilon\,,
$$
где $t$~--- номер итерации, а
$\parallel\bullet\parallel$~--- евклидова норма, для некоторого $\varepsilon
>0$.
\end{enumerate}

Очевидно, решение оптимизационной задачи
$$
\sum\limits_{j=1}^k\sum\limits_{i=1}^n \g_{ij}^{(t)}\ln p_j \rightarrow
\underset{p_j}{\max}
$$
дается выражением (с учетом $\sum_{j=1}^k p_j =1$):
$$
p_{ij}^{(t+1)} =\fr{1}{n}\,\sum\limits_{i=1}^n \g_{ij}^{(t)}\,,
$$
где $t$~--- номер итерации, $t = 0$, 1, 2,\,\ldots

Решение оптимизационной задачи
$$
\sum\limits_{j=1}^k \sum\limits_{i=1}^n \g_{ij}^{(t)} f(X_i;\,\theta_j )
\rightarrow \underset{\theta_j}{\max}
$$
получить намного проще решения задачи~(5): выражение для $\theta_j$
записывается с учетом знания конкретного вида функций
$f(X,\,\theta)$~\cite{3bat}.

\subsection{О сходимости алгоритма} %2.2.

В работе М.\,И.~Шлезингера~\cite{5bat}, где эта схема (позднее названная
ЕМ-схемой) впервые предложена, установлены и основные свойства
реа\-ли\-зу\-ющих ее алгоритмов. В частности, было доказано, что при достаточно
широких предположениях \textit{предельные точки} всякой последовательности,
порожденной итерациями ЕМ-алгоритма, являются стационарными точками
оптимизируемой логарифмической функции правдоподобия $\ln L(\Theta )$ и что
найдется неподвижная точка алгоритма, к которой будет сходиться каждая из таких
последовательностей. Если дополнительно потребовать положительной
определенности информационной мат\-ри\-цы Фишера для $\ln L(\Theta )$ при
истинных зна\-че\-ни\-ях па\-ра\-мет\-ра $\Theta$, то можно показать, что
асимптотически по $n\rightarrow\infty$ (т.е.\ при больших выборках) существует
единственное сходящееся (по веро\-ят\-но\-сти) решение $\hat{\Theta}(n)$
уравнений метода максимального правдоподобия и, кроме того, существует в
пространстве параметров $\Theta$ норма, в которой последовательность
$\Theta^{(t)}(n)$, порожденная ЕМ-ал\-го\-рит\-мом, сходится к $\hat{\Theta}
(n)$, если только начальная аппроксимация $\hat{\Theta}^0$ не была слишком
далека от~$\hat{\Theta} (n)$. {%\looseness=1

}

Таким образом, результаты исследования свойств ЕМ-алгоритмов метода
максимального правдоподобия разделения смеси и их практическое
использование показали, что они являются достаточно работоспособными (при
известном чис\-ле компонентов смеси) даже при большом чис\-ле $k$ компонентов и
при высоких размерностях анализируемого признака~$X$~\cite{3bat}.

\subsection{Уравнения для смеси экспоненциальных распределений}
%2.3.

Применим описанный выше алгоритм к разделению смеси
экспоненциальных распределений:
$$
p(x) = \sum\limits_{j=1}^k p_j \lambda_j e^{-\lambda_j x}\,.
$$
Получаем следующие итерационные уравнения:
\begin{align*}
\g_{ij}^{(t+1)} & = \fr{p_j^{(t)}\lambda_j^{(t)}e^{-
\lambda_j^{(t)}X_i}}{\sum\limits_{l=1}^k p_l^{(t)}\lambda_l^{(t)}
e^{-\lambda_l^{(t)}X_i}}\,,\\
p_j^{(t+1)} & = \fr{1}{n}\,\sum\limits_{i=1}^n \g_{ij}^{(t)}\,.
\end{align*}

Чтобы найти  оценки $\lambda_j$, подсчитаем первую производную функции
$$\sum_{j=1}^k\sum_{i=1}^n \g_{ij}^{(t)} \ln (\lambda_j e^{-\lambda_j X_i}):$$
\vspace*{-8pt}
\begin{multline*}
\left ( \sum\limits_{j=1}^k \sum\limits_{i=1}^n
\g_{ij}^{(t)}\ln \left ( \lambda_j
e^{-\lambda_j X_i} \right ) \right )^\prime \lambda_j =\\[-3pt]
{}= \left (
\sum\limits_{j=1}^k\sum\limits_{i=1}^n \g_{ij}^{(t)}\ln (\lambda_j -\lambda_j X_i )
\right )^\prime \lambda_j =\\[-3pt]
{}= \sum\limits_{i=1}^n \g_{ij}^{(t)}\left (
\fr{1}{\lambda_j} - X_i \right )\,.
\end{multline*}

Разрешая уравнение
$$
\sum\limits_{i=1}^n \g_{ij}^{(t)}\left ( \fr{1}{\lambda_j} -X_i\right ) =0
$$
относительно $\lambda_j$, получаем следующее итерационное уравнение:
$$
\lambda_j^{(t+1)} = \fr{\sum\limits_{i=1}^n
\g_{ij}^{(t)}}{\sum\limits_{i=1}^n \g_{ij}^{(t)} X_i}\,.
$$

\subsection{Уравнения для смеси гамма-распределений } %2.4.

Применим теперь ЕМ-алгоритм к смеси гам\-ма-рас\-пре\-де\-ле\-ний вида
$$
p(x) = \sum\limits_{j=1}^k p_j \fr{\alpha_j^{\alpha_j} x^{\alpha_j -
1}}{\lambda_j^{\alpha_j} \Gamma (\alpha_j )}\,e^{-(\alpha_j / \lambda_j)x}\,.
$$

Такая параметризация удобна для нахождения
оценок~$\alpha_j$~\cite{6bat}.

Аналогичным способом выписываются итерационные уравнения:
\begin{align*}
\g_{ij}^{(t+1)} & =   \fr{p_j^{(t)}\fr{(\alpha_j^{\alpha_j} )^{(t)}
x^{\alpha_j - 1}}{(\lambda_j^{\alpha_j} )^{(t)}\Gamma (\alpha_j)}\,
e^{-(\alpha_j /\gamma_j)^{(t)}x}}{\sum\limits_{l=1}^k
p_l^{(t)}\fr{(\alpha_l^{\alpha_l})^{(t)} x^{\alpha_l -
1}}{(\lambda_l^{\alpha_l})^{(t)}\Gamma (\alpha_l )}\,
e^{-(\alpha_l /\lambda_l)^{(t)} x}}\,,\\
p_j^{(t+1)} & = \fr{1}{n}\,\sum\limits_{i=1}^n \g_{ij}^{(t)}\,.
\end{align*}

Далее найдем оценки $\lambda_j$ для данного случая, приравнивая
производную
\begin{equation} %8
\sum\limits_{j=1}^k \sum\limits_{i=1}^n \g_{ij}^{(t)} \ln \left (
\fr{\alpha_j^{\alpha_j} x^{\alpha_j -1}}{\lambda_j^{\alpha_j}\Gamma
(\alpha_j)}\,e^{-(\alpha_j /\lambda_j) x}\right )
\end{equation}
по $\lambda_j$ к нулю и разрешая относительно~$\lambda_j$ уравнение:
$$
\sum\limits_{i=1}^n \g_{ij}^{(t+1)}\left ( \fr{\alpha_j^{(t)}}{\lambda_j^{(t)}}
- \fr{\alpha_j^{(t)}X_i}{\left ( \lambda_j^{(t)}\right )^2}\right ) =0 \,.
$$
Получаем
$$
\lambda_j^{(t+1)} = \fr{\sum\limits_{i=1}^n \g_{ij}^{(t)}
X_i}{\sum\limits_{i=1}^n \g_{ij}^{(t)}}\,.
$$

Для того чтобы получить итерационные уравнения для $\alpha_j$, найдем
первую производную~(8):
\begin{multline*}
\left ( \sum\limits_{j=1}^k\sum\limits_{i=1}^n \g_{ij}^{(t)}\ln \left (
\fr{\alpha_j^{\alpha_j} x^{\alpha_j -1}}{\lambda_j^{\alpha_j}\Gamma (\alpha_j
)}\,e^{-(\alpha_j /\lambda_j ) x} \right ) \right )^\prime \alpha_j ={}\\[-3pt]
{}=\left ( \sum\limits_{j=1}^k\sum\limits_{i=1}^n \g_{ij}^{(t)}\ln \left (
\fr{\alpha_j^{\alpha_j}}{\lambda_j^{\alpha_j}}\right ) - \ln \Gamma (\alpha_j )+{} \right.\\[-3pt]
{}+\left.
(\alpha_j -1 )\ln X_i - \fr{\alpha_j}{\lambda_j}\,X_i \right )^\prime \alpha_j =\\[-3pt]
{}=\sum\limits_{i=1}^n \g_{ij}^{(t)} \left ( \ln \alpha_j +1-\ln \lambda_j -
\fr{\Gamma^\prime (\alpha_j )}{\Gamma (\alpha_j)}\right.+\\[-3pt]
{}+\left. \ln X_i - \fr{X_i}{\lambda_j}\right )\,;
\end{multline*}
\begin{multline*}
\sum\limits_{i=1}^n \g_{ij}^{(t)} \left(  \ln \alpha_j +1 -\ln \lambda_j -{}\right. \\[-3pt]
\left. {}-\fr{\Gamma^\prime (\alpha_j )}{\Gamma (\alpha_j )}+\ln X_i 
-\fr{X_i}{\lambda_j} \right) =0\,;
\end{multline*}
\begin{multline}
\fr{\Gamma^\prime (\alpha_j )}{\Gamma (\alpha_j )} ={}\\[-3pt]
{}= \fr{\sum\limits_{i=1}^n \g_{ij}^{(t)} \left ( \ln \alpha_j +1-\ln\lambda_j 
+\ln X_i -\fr{X_i}{\lambda_j} \right )}{\sum\limits_{i=1}^n \g_{ij}^{(t)}}\,.
\end{multline}
%
Здесь $\Gamma^\prime (\alpha_j ) / \Gamma (\alpha_j )$~--- это
\textit{логарифмическая производная гамма-функции}. Для нее существует так
называемое \textit{разложение Абрамовитца}--\textit{Стигана}~\cite{4bat}:
$$
\fr{\Gamma^\prime (\alpha ) }{ \Gamma (\alpha )} = \mathrm{log}\,\alpha -
\fr{1}{2\alpha }-\fr{1}{12\alpha^2 }+\ldots
$$

Подставим первые три члена разложения в~(9) и разрешим это уравнение
относительно~$\alpha_j$:
$$
\alpha_{ij}^{(t+1)} = \fr{\sum\limits_{i=1}^n
\g_{ij}^{(t+1)}}{2\sum\limits_{i=1}^n \g_{ij}^{(t +1)}\left ( \fr{X_i}{\lambda_j^{(t)}} -
\ln \fr{X_i}{\lambda_j^{(t)}} -1\right )}\,.
$$
В итоге получаем итерационные уравнения для ~$\alpha_j$.

\section{Описание программного обеспечения (программа~ЕМ)}

\subsection{Назначение программы} %3.1.

Разработанная авторами статьи программа ЕМ предназначена для решения задачи
разделения смесей экспоненциальных и гамма-распределений, поставленной в
разд.~2, с использованием ЕМ-ал\-го\-рит\-ма и формул, описанных в разд.~3.

\subsection{Инструменты разработки} %3.2.

Для создания программы была использована среда разработки Microsoft
Visual Studio .NET 2005 и объектно-ориентированный язык C\#. Для
визуализации результатов была использована свободно распространяемая
графическая библиотека ZedGraph~\cite{7bat}.


\subsection{Возможности  программы} %3.3.

\noindent
\begin{itemize}
\item Загрузка выборочных данных из текстового файла
\item Оценивание по выборке параметров смеси экспоненциальных
распределений
\item Оценивание по выборке параметров смеси гамма-распределений
\item Отслеживание изменений параметров смесей распределений во
времени в режиме <<скользящего окна>>
\item Построение гистограммы по выборке
\end{itemize}

\subsection{Входные и выходные данные. Функционирование
программы} %3.4.

В качестве \textit{входных данных} программа ЕМ получает:
\begin{itemize}
\item выборочные данные из текстового файла;
\item число компонентов смеси;
\item размер <<скользящего окна>>;
\item размер класса гистограммы.
\end{itemize}

На \textit{выходе} мы получаем:
\begin{itemize}
\item точечные оценки параметров смеси экспоненциальных
распределений;
\item точечные оценки параметров смеси гамма-распределений;
\item графическое изображение результирующей смеси распределения;
\item графическое изображение компонентов каж\-дой смеси;
\item графическое изображение того, как меняются параметры смесей
распределений с течением времени в режиме <<скользящего окна>>;
\item гистограмма, построенная по выборке;
\item значение статистического теста.
\end{itemize}

Выборочные данные загружаются из текстового файла в память программы и подаются
на вход двум независимо работающим реализациям ЕМ-алгоритма~--- для
идентификации смеси экспоненциальных распределений и для идентификации смеси
гамма-распределений. Результатом их работы являются наборы значений оцениваемых
параметров модели, предложенной в разд.~2. Кроме того, результирующие
распределения визуализируются в виде графиков. В программе можно запустить
режим <<скользящего окна>>, который для всех подвыборок заданного
размера с помощью ЕМ-алгоритма оценивает параметры смесей распределений этих
подвыборок. Все действия программы документируются в окне информации.

\section{Описание тестовых расчетов}

С использованием разработанной программы были проведены тестовые
расчеты на выборочных данных реального сетевого трафика.

На вход программы EM были поданы выборки трафика:
\begin{enumerate}[I]
\item Между лабораторией Lawrence Berkeley (Berkeley, California) и
внешним миром размера примерно 7000~\cite{8bat}~--- \textit{выборка~1}.
\item
Сети радиодоступа ЗАО <<Синтерра>> размера примерно 1000~\cite{9bat}~---
 \textit{выборка~2}.
\end{enumerate}

\subsection{Выборка 1 ``Berkeley''} %5.1.

При числе компонентов смеси~5 и случайном начальном приближении
были получены результаты, представленные в табл.~\ref{t1bat}.


Данные результаты иллюстрирует рис.~\ref{f5bat}.

Гистограмма  на рис.~\ref{f6bat} показывает статистическую значимость
полученных результатов.

Данная выборка обладает той особенностью, что она собиралась в течение
достаточно длительного времени и в ней агрегирован самый разнородный
трафик. Поэтому в ней присутствует не только большое количество
<<коротких>> сообщений (что обычно для выборок из телетрафика), но и
некоторый массив сообщений средней длины, а также определенный
<<выброс>> больших сообщений. Это свидетельствует о \textit{пиковости}
телетрафика на довольно больших промежутках времени.

Как мы видим, ЕМ-алгоритм удачно справился с задачей идентификации
смеси.

\subsection{Выборка~2 ``Synterra''} %5.2.

Результаты применения ЕМ-алгоритма к выборке ``Synterra''
представлены в табл.~\ref{t2bat}.
\begin{table*}\small
\begin{minipage}[t]{76mm}
\begin{center}
\Caption{Результаты применения ЕМ-алго\-рит\-ма к выборке~1 ``Berkeley'' 
\label{t1bat}} \vspace*{2ex}

\tabcolsep=8.7pt
\begin{tabular}{|c|c|c|}
\hline
№&Начальное приближение&Результат\\
\hline
\multicolumn{3}{|c|}{$P$}\\
\hline
0&0,2&0,1896\\
1&0,2&0,1858\\
2&0,2&0,1830\\
3&0,2&0,2259\\
4&0,2&0,2154\\
\hline
\multicolumn{3}{|c|}{$\alpha$}\\
\hline
0&2,7028&10,9783\hphantom{9}\\
1&3,6273&5,8621 \\
2&5,7598&2,7092\\
3&0,2315&1,0235\\
4&0,9110&0,4772\\
\hline
\multicolumn{3}{|c|}{$\lambda$}\\
\hline
0&85,2066&137,1714  \\
1&23,9592&136,7349\\
2&63,8425&132,6482\\
3&\hphantom{9}1,8026&116,7317\\
4&98,3882&102,5278\\
\hline
\end{tabular}
\end{center}
\end{minipage}\hfill
\begin{minipage}[t]{76mm}
%\end{table*}
%\begin{table*}\small
\begin{center}
\Caption{Результаты применения ЕМ-алго\-рит\-ма к выборке~2 ``Synterra'' 
\label{t2bat}} \vspace*{2ex}

\tabcolsep=8.7pt
\begin{tabular}{|c|c|c|}
\hline
№&Начальное приближение&Результат\\
\hline
\multicolumn{3}{|c|}{$P$}\\
\hline
0&0,2&$0{,}3815\hphantom{{}\cdot 10^{-9}}$\\
1&0,2&$0{,}3594\hphantom{{}\cdot 10^{-9}}$\\
2&0,2&$0{,}2589\hphantom{{}\cdot 10^{-9}}$\\
3&0,2&$0{,}4401\cdot 10^{-9}$\\
4&0,2&$0{,}0\hphantom{{}\cdot 10^{-9}999}$\\
\hline
\multicolumn{3}{|c|}{$\alpha$}\\
\hline
0&6,0804&1,5833\\
1&3,1838&0,8554\\
2&1,4886&0,4557\\
3&4,6407&0,2278\\
4&3,7843&0,1139\\
\hline
\multicolumn{3}{|c|}{$\lambda$}\\
\hline
0&17,3387&15,8682\\
1&47,8294&16,9150\\
2&54,1984&19,2866\\
3&\hphantom{1}8,6254&19,2866\\
4&\hphantom{1}5,7252&19,2866\\
\hline
\end{tabular}
\end{center}
\end{minipage}
\end{table*}


Данные результаты иллюстрируют рис.~\ref{f7bat}.


Эти результаты также отражают действительную картину, как показано на
рис.~\ref{f8bat}.


Этот трафик был снят с базовой станции <<Лукойл-Юго-Запад>> сети
широкополосного радиодоступа ЗАО <<Синтерра>>. Сеть радиодоступа
является реализацией так называемой <<последней мили>>, переносящей два
разных вида трафика: данные (Ethernet пакеты) и голос (IP-телефония, VoIP).
Поэтому здесь присутствуют в качестве основной массы короткие, но
интенсивные сообщения (пакеты SIP и голосовые фреймы), а также длинные
сообщения, содержащие данные.

Как мы видим, программная реализация ЕМ-ал\-го\-рит\-ма успешно справилась с
задачей разделения смесей распределений для этих двух выборок, что делает
данную программу удобным инструментом построения стохастической картины
конкретной сети. По полученным данным, используя метод интерпретации,
предложенный в разд.~2, можно получить представление о количестве
последовательных и параллельных структур вероятностной модели сети.

\subsection{Режим <<скользящего окна>>} %5.3.

Результаты для выборки
``Berkeley'' в режиме <<скользящего окна>>  представлены
на рис.~\ref{f9bat}.


Данные графики показывают изменение параметров распределений подвыборок выборки 
``Berkeley''. Видно, что параметры распределений подвыборок не остаются 
неизменными во времени, наоборот, они имеют внешне случайный характер. На 
рис.~\ref{f9bat},\,\textit{в} видна даже своеобразная пульсация первой 
компоненты.
%
На основании расчетов можно сделать вывод о том, что пиковость трафика
обусловливается как формой, так и интенсивностью сообщений.

\section{Заключение}

В данной работе исследована вероятностная модель  информационных потоков,
возникающих в сложных телекоммуникационных конвергентных сетях, построенная с
помощью асимптотического и энтропийного подходов. Эта модель предполагает, что
функционирование сложной телекоммуникационной сети можно представить в виде
суперпозиции довольно простых стохастических структур~--- последовательных и
параллельных, которые по\-рож\-да\-ют смеси гамма-распределений для случайной
величины времени обработки и передачи сообщений в сети. Предложена простая
интерпретация параметров данной модели.
\begin{figure*} %fig5
\vspace*{1pt}
\begin{center}
\mbox{%
\epsfxsize=130mm %145.109mm 
\epsfbox{bat-5.eps} }
\end{center}
\vspace*{-13pt} \Caption{Компоненты смеси начального приближения~(\textit{а}) и 
результата~(\textit{б}) для выборки~1 ``Berkeley'' \label{f5bat}}
%\end{figure*}
%\begin{figure*} %fig6
\vspace*{12pt}
\begin{center}
\mbox{%
\epsfxsize=130mm %148.256mm 
\epsfbox{bat-7.eps} }
\end{center}
\vspace*{-13pt} \Caption{График смеси распределений~(\textit{1}) и гистограмма 
для выборки~1 ``Berkeley''~(\textit{2}) \label{f6bat}}
\end{figure*}



\begin{figure*} %fig7
\vspace*{1pt}
\begin{center}
\mbox{%
\epsfxsize=130mm %144.283mm 
\epsfbox{bat-8.eps} }
\end{center}
\vspace*{-16pt} \Caption{Компоненты смеси начального приближения~(\textit{а}) и 
результата~(\textit{б}) для выборки~2 ``Synterra'' \label{f7bat}}
%\end{figure*}
%\begin{figure*} %fig8
\vspace*{12pt}
\begin{center}
\mbox{%
\epsfxsize=130mm %148.256mm 
\epsfbox{bat-10.eps} }
\end{center}
\vspace*{-11pt} \Caption{График смеси распределений~(\textit{1}) и гистограмма
для выборки~2 ``Synterra''~(\textit{2}) \label{f8bat}}
\end{figure*}

\begin{figure*} %fig9
\vspace*{1pt}
\begin{center}
\mbox{%
\epsfxsize=119.041mm
\epsfbox{bat-11.eps} }
\end{center}
\vspace*{-9pt} \Caption{Изменение  смешивающих параметров~(\textit{а}), 
параметров формы~(\textit{б}) и параметров масштаба~(\textit{в}) во времени для 
выборки~1 ``Berkeley'' \label{f9bat}}
\end{figure*}

Для решения вытекающей из модели задачи предложен итерационный алгоритм,
базирующийся на методе максимального правдоподобия~--- ЕМ-ал\-го\-ритм, для
которого получены формулы для конкретного вида смесей~--- экспоненциальных и
гамма-распределений.
%
Кроме того, разработан программный инструментарий для оценки параметров 
предложенной модели на выборках из реальных трафиковых данных. Проведены 
исследования, которые подтвердили предположения вероятностной модели. 


Получение информации о стохастической структуре
телекоммуникационных сетей и наличие программных инструментов для
выявления более или менее стабильных структур позволит понять причины
возникновения неожиданных больших нагрузок, предотвратить такие нагрузки,
а также поможет в будущем в проектировании надежных, оптимальных по
стоимости и уровню сервиса телекоммуникационных сетей нового поколения.

%\vspace*{-15pt} 
{\small\frenchspacing
{%\baselineskip=10.8pt
\addcontentsline{toc}{section}{Литература}
\begin{thebibliography}{9}
\bibitem{1bat}
Teletraffic Engeneering Handbook. International Telecommunication Union, 
Geneva, 2005 {\sf http://www.itu.int}. \vspace*{5pt} 
\bibitem{2bat}
\Au{Севастьянов~Б.\,А.} Курс теории вероятностей и математической статистики. 
М., 2004. \vspace*{5pt} 
\bibitem{3bat}
\Au{Айвазян~C.\,А., Бухштабер~В.\,М., Енюков~И.\,С, Мешалкин~Л.\,Д.} Прикладная 
статистика. Классификация и снижение размерности~// Финансы и статистика. М., 
1989. \vspace*{5pt} 
\bibitem{4bat}
\Au{Bilmes~J.\,A.} A gentle tutorial of the EM algorithm and its application to 
parameter estimation for Gaussian mixture and hidden Markov models. Berkeley, 
CA, USA: International Computer Science Institute,  1998. \vspace*{5pt} 
\bibitem{5bat}
\Au{Шлезингер~М.\,И.} О самопроизвольном различении образов~// Шлезингер~М.\,И. 
Читающие. автоматы. Киев: Наукова думка, 1965. С.~38--45. \vspace*{5pt} 
\bibitem{6bat}
\Au{Hsiao~I.-T., Rangarajan~A., Gindi~G.}. Joint-MAP 
reconstruction/segmentation for transmission tomography using mixture-models as 
priors. Yale University, 1998. \vspace*{5pt} 
\bibitem{7bat}
{\sf http://zedgraph.org}. \vspace*{4pt} 
\bibitem{8bat}
{\sf http://ita.ee.lbl.gov/html/contrib/LBL-PKT.html}. \vspace*{5pt} 
\bibitem{9bat}
{\sf http://www.synterra.ru}.
\end{thebibliography}

} } \label{end\stat}
\end{multicols}


%\addtocounter{razdel}{1}
%\def\razd{НЕРЕГУЛИРУЕМЫЙ ЭЛЕКТРОПРИВОД ДЛЯ ЭЛЕКТРОЭНЕРГЕТИКИ}

\setcounter{page}{2}

   { %\Large  
   { %\baselineskip=16.6pt
   
   \vspace*{-48pt}
   \begin{center}\LARGE
   \textit{Предисловие}
   \end{center}
   
   %\vspace*{2.5mm}
   
   \vspace*{25mm}
   
   \thispagestyle{empty}
   
   { %\small 

    
Вниманию читателей журнала <<Информатика и её применения>> предлагается 
очередной тематический выпуск <<Вероятностно-статистические методы и 
задачи информатики и информационных технологий>>. Предыдущие тематические 
выпуски журнала по данному направлению вышли в 2008~г.\ (т.~2, вып.~2), 
в 2009~г.\ (т.~3, вып.~3) и в 2010~г.\ (т.~4, вып.~2). 

Статьи, собранные в данном журнале, посвящены разработке новых вероятностно-статистических 
методов, ориентированных на применение к решению конкретных задач информатики и информационных 
технологий, а также~--- в ряде случаев~--- и других прикладных задач. Проблематика, охватываемая 
публикуемыми работами, развивается в рамках научного сотрудничества между Институтом проблем 
информатики Российской академии наук (ИПИ РАН) и Факультетом вычислительной математики и 
кибернетики Московского государственного университета им.\ М.\,В.~Ломоносова в ходе работ 
над совместными научными проектами (в том числе в рамках функционирования 
Научно-образовательного центра <<Вероятностно-статистические методы анализа рисков>>). 
Многие из авторов статей, включенных в данный номер журнала, являются активными участниками 
традиционного международного семинара по проблемам устойчивости стохастических моделей, 
руководимого В.\,М.~Золотаревым и В.\,Ю.~Королевым; регулярные сессии этого семинара 
проводятся под эгидой МГУ и ИПИ РАН (в 2011~г.\ указанный семинар проводится в октябре 
в Калининградской области РФ). 

Наряду с представителями ИПИ РАН и МГУ в число авторов данного выпуска журнала входят 
ученые из Научно-исследовательского института системных исследований РАН, Института 
проблем технологии микроэлектроники и особочистых материалов РАН, Института 
прикладных математических исследований Карельского НЦ РАН, Московского 
авиационного института, Вологодского государственного педагогического университета, 
НИИММ им.\ Н.\,Г.~Чеботарева, Казанского государственного университета, Дебреценского 
университета (Венгрия).

Несколько статей выпуска посвящено разработке и применению стохастических методов и 
информационных технологий для решения различных прикладных задач. В~работе В.\,Г.~Ушакова 
и О.\,В.~Шестакова рассмотрена задача определения вероятностных характеристик случайных 
функций по распределениям интегральных преобразований, возникающих в задачах эмиссионной 
томографии. В~статье Д.\,О.~Яковенко и М.\,А.~Целищева рассмотрены некоторые вопросы 
математической теории риска и предложен новый подход к диверсификации инвестиционных 
портфелей. Работа И.\,А.~Кудрявцевой и А.\,В.~Пантелеева посвящена построению и 
исследованию математической модели, описывающей динамику сильноионизованной плазмы. 
В~статье П.\,П.~Кольцова изучается качество работы ряда алгоритмов сегментации изображений. 
Статья А.\,Н.~Чупрунова и И.~Фазекаша посвящена вероятностному анализу числа без\-оши\-бочных 
блоков при помехоустойчивом кодировании; получены усиленные законы больших чисел для указанных 
величин.

В данном выпуске традиционно присутствует тематика, весьма активно разрабатываемая в течение 
многих лет специалистами ИПИ РАН и МГУ,~--- методы моделирования и управления для 
информационно-телекоммуникационных и вычислительных систем, в частности методы 
теории массового обслуживания. В~статье А.\,И.~Зейфмана с соавторами рассматриваются 
модели обслуживания, описываемые марковскими цепями с непрерывным временем в случае 
наличия катастроф. В~работе М.\,М.~Лери и И.\,А.~Чеплюковой рассматриваются случайные 
графы Интернет-типа, т.\,е.\ графы, степени вершин которых имеют степенные распределения; 
такие задачи находят применение при исследовании глобальных сетей передачи данных. 
Работа Р.\,В.~Разумчика посвящена исследованию систем массового обслуживания специального 
вида~--- с отрицательными заявками и хранением вытесненных заявок.

Ряд статей посвящен развитию перспективных теоретических 
вероятностно-статистических методов, которые находят широкое применение в различных 
задачах информатики и информационных технологий. В~работе В.\,Е.~Бенинга, А.\,К.~Горшенина 
и В.\,Ю.~Королева рассмотрена задача статистической проверки гипотез о числе компонент 
смеси вероятностных распределений, приводится конструкция асимптотически наиболее мощного 
критерия. Результаты этой работы найдут применение в ряде прикладных задач, использующих 
математическую модель смеси вероятностных распределений (в информатике, моделировании 
финансовых рынков, физике турбулентной плазмы и~т.\,д.). В~статье В.\,Ю.~Королева, 
И.\,Г.~Шевцовой и С.\,Я.~Шоргина строится новая, улучшенная оценка точности нормальной 
аппроксимации для пуассоновских случайных сумм; как известно, указанные случайные суммы 
широко используются в качестве моделей многих реальных объектов, в том числе в информатике, 
физике и других прикладных областях. Работа В.\,Г.~Ушакова и Н.\,Г.~Ушакова посвящена 
исследованию ядерной оценки плотности распределения; эти результаты могут применяться, 
в част\-ности, при анализе трафика в телекоммуникационных системах. Серьезные приложения 
в статистике могут получить результаты работы О.\,В.~Шестакова, в которой доказаны оценки 
скорости сходимости распределения выборочного абсолютного медианного отклонения к нормальному 
закону. 

\smallskip

Редакционная коллегия журнала выражает надежду, что данный тематический  выпуск 
будет интересен специалистам в области теории вероятностей и математической статистики 
и их применения к решению задач информатики и информационных технологий.
     
     %\vfill 
     \vspace*{20mm}
     \noindent
     Заместитель главного редактора журнала <<Информатика и её 
применения>>,\\
     директор ИПИ РАН, академик  \hfill
     \textit{И.\,А.~Соколов}\\
     
     \noindent
     Редактор-составитель тематического выпуска,\\
     профессор кафедры математической статистики факультета\\
      вычислительной математики и кибернетики МГУ им.\ М.\,В.~Ломоносова,\\
     ведущий научный сотрудник ИПИ РАН,\\ 
доктор физико-математических наук \hfill
      \textit{В.\,Ю.~Королев}
     
     } }
     }



%   { %\Large  
   { %\baselineskip=16.6pt
   
   \vspace*{-48pt}
   \begin{center}\LARGE
   \textit{Предисловие}
   \end{center}
   
   %\vspace*{2.5mm}
   
   \vspace*{25mm}
   
   \thispagestyle{empty}
   
   { %\small 

    
Вниманию читателей журнала <<Информатика и её применения>> предлагается 
очередной тематический выпуск <<Вероятностно-статистические методы и 
задачи информатики и информационных технологий>>. Предыдущие тематические 
выпуски журнала по данному направлению вышли в 2008~г.\ (т.~2, вып.~2), 
в 2009~г.\ (т.~3, вып.~3) и в 2010~г.\ (т.~4, вып.~2). 

Статьи, собранные в данном журнале, посвящены разработке новых вероятностно-статистических 
методов, ориентированных на применение к решению конкретных задач информатики и информационных 
технологий, а также~--- в ряде случаев~--- и других прикладных задач. Проблематика, охватываемая 
публикуемыми работами, развивается в рамках научного сотрудничества между Институтом проблем 
информатики Российской академии наук (ИПИ РАН) и Факультетом вычислительной математики и 
кибернетики Московского государственного университета им.\ М.\,В.~Ломоносова в ходе работ 
над совместными научными проектами (в том числе в рамках функционирования 
Научно-образовательного центра <<Вероятностно-статистические методы анализа рисков>>). 
Многие из авторов статей, включенных в данный номер журнала, являются активными участниками 
традиционного международного семинара по проблемам устойчивости стохастических моделей, 
руководимого В.\,М.~Золотаревым и В.\,Ю.~Королевым; регулярные сессии этого семинара 
проводятся под эгидой МГУ и ИПИ РАН (в 2011~г.\ указанный семинар проводится в октябре 
в Калининградской области РФ). 

Наряду с представителями ИПИ РАН и МГУ в число авторов данного выпуска журнала входят 
ученые из Научно-исследовательского института системных исследований РАН, Института 
проблем технологии микроэлектроники и особочистых материалов РАН, Института 
прикладных математических исследований Карельского НЦ РАН, Московского 
авиационного института, Вологодского государственного педагогического университета, 
НИИММ им.\ Н.\,Г.~Чеботарева, Казанского государственного университета, Дебреценского 
университета (Венгрия).

Несколько статей выпуска посвящено разработке и применению стохастических методов и 
информационных технологий для решения различных прикладных задач. В~работе В.\,Г.~Ушакова 
и О.\,В.~Шестакова рассмотрена задача определения вероятностных характеристик случайных 
функций по распределениям интегральных преобразований, возникающих в задачах эмиссионной 
томографии. В~статье Д.\,О.~Яковенко и М.\,А.~Целищева рассмотрены некоторые вопросы 
математической теории риска и предложен новый подход к диверсификации инвестиционных 
портфелей. Работа И.\,А.~Кудрявцевой и А.\,В.~Пантелеева посвящена построению и 
исследованию математической модели, описывающей динамику сильноионизованной плазмы. 
В~статье П.\,П.~Кольцова изучается качество работы ряда алгоритмов сегментации изображений. 
Статья А.\,Н.~Чупрунова и И.~Фазекаша посвящена вероятностному анализу числа без\-оши\-бочных 
блоков при помехоустойчивом кодировании; получены усиленные законы больших чисел для указанных 
величин.

В данном выпуске традиционно присутствует тематика, весьма активно разрабатываемая в течение 
многих лет специалистами ИПИ РАН и МГУ,~--- методы моделирования и управления для 
информационно-телекоммуникационных и вычислительных систем, в частности методы 
теории массового обслуживания. В~статье А.\,И.~Зейфмана с соавторами рассматриваются 
модели обслуживания, описываемые марковскими цепями с непрерывным временем в случае 
наличия катастроф. В~работе М.\,М.~Лери и И.\,А.~Чеплюковой рассматриваются случайные 
графы Интернет-типа, т.\,е.\ графы, степени вершин которых имеют степенные распределения; 
такие задачи находят применение при исследовании глобальных сетей передачи данных. 
Работа Р.\,В.~Разумчика посвящена исследованию систем массового обслуживания специального 
вида~--- с отрицательными заявками и хранением вытесненных заявок.

Ряд статей посвящен развитию перспективных теоретических 
вероятностно-статистических методов, которые находят широкое применение в различных 
задачах информатики и информационных технологий. В~работе В.\,Е.~Бенинга, А.\,К.~Горшенина 
и В.\,Ю.~Королева рассмотрена задача статистической проверки гипотез о числе компонент 
смеси вероятностных распределений, приводится конструкция асимптотически наиболее мощного 
критерия. Результаты этой работы найдут применение в ряде прикладных задач, использующих 
математическую модель смеси вероятностных распределений (в информатике, моделировании 
финансовых рынков, физике турбулентной плазмы и~т.\,д.). В~статье В.\,Ю.~Королева, 
И.\,Г.~Шевцовой и С.\,Я.~Шоргина строится новая, улучшенная оценка точности нормальной 
аппроксимации для пуассоновских случайных сумм; как известно, указанные случайные суммы 
широко используются в качестве моделей многих реальных объектов, в том числе в информатике, 
физике и других прикладных областях. Работа В.\,Г.~Ушакова и Н.\,Г.~Ушакова посвящена 
исследованию ядерной оценки плотности распределения; эти результаты могут применяться, 
в част\-ности, при анализе трафика в телекоммуникационных системах. Серьезные приложения 
в статистике могут получить результаты работы О.\,В.~Шестакова, в которой доказаны оценки 
скорости сходимости распределения выборочного абсолютного медианного отклонения к нормальному 
закону. 

\smallskip

Редакционная коллегия журнала выражает надежду, что данный тематический  выпуск 
будет интересен специалистам в области теории вероятностей и математической статистики 
и их применения к решению задач информатики и информационных технологий.
     
     %\vfill 
     \vspace*{20mm}
     \noindent
     Заместитель главного редактора журнала <<Информатика и её 
применения>>,\\
     директор ИПИ РАН, академик  \hfill
     \textit{И.\,А.~Соколов}\\
     
     \noindent
     Редактор-составитель тематического выпуска,\\
     профессор кафедры математической статистики факультета\\
      вычислительной математики и кибернетики МГУ им.\ М.\,В.~Ломоносова,\\
     ведущий научный сотрудник ИПИ РАН,\\ 
доктор физико-математических наук \hfill
      \textit{В.\,Ю.~Королев}
     
     } }
     }

\renewcommand*{\Pr}{\mathbb P}
\newcommand*{\E}{\mathbb E}
\newcommand*{\Dd}{\mathbb D}
%\newcommand*{\R}{\mathbb R}
\newcommand*{\Fd}{\mathfrak F}
\newcommand*{\N}{\mathbb N}
\newcommand*{\Id}{\mathcal I}

\def\stat{korgor}

\def\tit{АСИМПТОТИЧЕСКИ
ОПТИМАЛЬНЫЙ КРИТЕРИЙ ПРОВЕРКИ ГИПОТЕЗ О ЧИСЛЕ КОМПОНЕНТ СМЕСИ
ВЕРОЯТНОСТНЫХ РАСПРЕДЕЛЕНИЙ$^*$}

\def\titkol{Асимптотически
оптимальный критерий проверки гипотез о числе компонент смеси
вероятностных распределений}

\def\autkol{В.\,Е.~Бенинг, А.\,К.~Горшенин, В.\,Ю.~Королев}
\def\aut{В.\,Е.~Бенинг$^1$, А.\,К.~Горшенин$^2$, В.\,Ю.~Королев$^3$}

\titel{\tit}{\aut}{\autkol}{\titkol}

{\renewcommand{\thefootnote}{\fnsymbol{footnote}}\footnotetext[1]
{Работа поддержана Российским
фондом фундаментальных исследований (проекты 09-07-12032-офи-м,
11-07-00112а, 11-01-00515а и 11-01-12026-офи-м), а также Министерством образования и
науки РФ в рамках ФЦП <<Научные и научно-педагогические кадры
инновационной России на 2009--2013 годы>>.}}

\renewcommand{\thefootnote}{\arabic{footnote}}
\footnotetext[1]{Московский государственный
университет им.\ М.\,В.~Ломоносова, факультет вычислительной математики и кибернетики;
Институт проблем информатики Российской академии наук,
 bening@yandex.ru}
 \footnotetext[2]{Московский государственный
университет им.\ М.\,В.~Ломоносова, факультет вычислительной математики и кибернетики;
Институт проблем информатики Российской академии наук,
a.k.gorshenin@gmail.com}
  \footnotetext[3]{Московский государственный
университет им.\ М.\,В.~Ломоносова, факультет вычислительной математики и кибернетики;
Институт проблем информатики Российской академии наук,
vkorolev@cs.msu.su}
 
 \vspace*{4pt}

\Abst{Рассмотрена задача статистической
проверки гипотез о числе компонент смеси вероятностных
распределений. Приведен асимптотически наиболее мощный критерий.
При выполнении достаточно слабых условий найдены предельные
распределения, потеря мощности и асимптотический дефект. Подробно
рассмотрено применение данного критерия к проверке гипотез о числе
компонент смесей равномерных, нормальных и гам\-ма-рас\-пре\-де\-лений.}

\vspace*{2pt}

\KW{смеси вероятностных распределений;
асимптотически наиболее мощный критерий; потеря мощности;
асимптотический дефект}

\vspace*{4pt}

  \vskip 14pt plus 9pt minus 6pt

      \thispagestyle{headings}

      \begin{multicols}{2}
      
            \label{st\stat}




\section{Введение}

В некоторых конкретных прикладных задачах, использующих
математическую модель смеси ве\-роятностных распределений, крайне
важна кор\-ректная интерпретация полученных результатов\linebreak (например, в
физике турбулентной плазмы необходимо соотносить полученные
компоненты смеси с наблюдаемыми в плазме процессами). Поэтому при
выборе модели смеси важное значение имеет выбор не только типа
смеси (сдвиговая, масштабная или сдвиг-масштабная) и ядер
(смешиваемых распределений), но и числа компонент в подгоняемой к
реальным данным модели. С увеличением числа параметров возрастает
и согласие модели с данными. Однако известны примеры, когда
использование моделей с б$\acute{\mbox{o}}$льшим числом параметров не только
усложняет практические вычисления, но и приводит к неадекватной
интерпретации (более подробно об этом см.~\cite{Korolev2007, Korolev2010}).

Многие популярные алгоритмы (EM, SEM, MCEM-ал\-го\-рит\-мы и их
всевозможные модификации) для статистической декомпозиции смесей
используют заданное число компонент и в процессе итерационной
процедуры практически не могут менять это \textit{заранее} заданное
число. При этом в большинстве прикладных задач число компонент смеси
является неизвестным, и в лучшем случае возможно задать лишь
некоторую оценку сверху большим значением, что, как было
отмечено ранее, может приводить к неадекватным результатам. Известны
так называемые информационные критерии Акаике~\cite{Akaike1973} и
байесовский~\cite{Schwartz1978}, основанные на функции
правдоподобия. Однако в ряде практически значимых моделей нарушаются
условия регулярности, в частности в случае конечной смеси
нормальных законов, вообще говоря, функция правдоподобия не является
ограниченной, что приводит к необходимости использовать другие
критерии, а также накладывать дополнительные искусственные
технические условия для гарантии соблюдения формальных условий
регулярности.

В частности, в работах~\cite{Lo2001, Lo2005} для проверки гипотезы о
том, что $k_0$-ком\-по\-нент\-ная смесь нормальных распределений и
$k_1$-ком\-по\-нент\-ная смесь нормальных распределений одинаково близки к
истинному распределению, т.\,е.
$$
\E_h\left\{\log f(x;\,\theta^*)\right\}=\E_h\left\{\log g(x;\,\gamma^*)\right\}\,,
$$
против альтернативы, что одна из смесей приближает лучше, т.\,е.
$$
\E_h\left\{\log f(x;\,\theta^*)\right\}>\E_h\left\{\log g(x;\,\gamma^*)\right\}\,,
$$
предложено использовать статистику
\begin{equation}
\label{LR}
LR=LR(\Hat\theta,\Hat\gamma;\,x)=\sum\limits_{j=1}^n\log\fr{f(X_j;\,\Hat\theta)}
{g(X_j;\,\Hat\gamma)}\,.
\end{equation}

В качестве меры расстояния между данным и истинным распределением в
указанных работах используется информационный критерий
Куль\-ба\-ка--Лейб\-ле\-ра (KLIC)
\begin{multline*}
I(h:f;\,\theta)=\E_h\left\{\log\fr{h(X;\,\vartheta)}{f(X;\,\theta)}\right\}={}\\
\!{}=\int\log h(x;\,\vartheta)\,dH(x;\,\vartheta)-\int\log f(x;\,\theta)\,dH(x;\,\vartheta)\,.
\end{multline*}
Здесь $h(\cdot)$~--- истинная плотность, $f(\cdot)$~---
$k_1$-ком\-по\-нент\-ная смесь нормальных распределений, $g(\cdot)$~---
$k_0$-ком\-по\-нент\-ная смесь нормальных распределений, а символ $\E_h$
обозначает математическое\linebreak ожидание относительно~$h(\cdot)$;
$\theta^*$ и $\gamma^*$ минимизируют $I(h:f;\,\theta)$ и
$I(h:g;\,\gamma)$ соответственно, а величины~$\Hat\theta$ 
и~$\Hat\gamma$ являются оценками максимального правдоподобия для~$\theta^*$ 
и~$\gamma^*$ соответственно.

При этом для смесей нормальных законов в предположении
справедливости некоторых условий регулярности (а также
дополнительных предположений, позволяющих избегать неограниченности
функции правдоподобия) в работах~\cite{Lo2001, Lo2005} отмечено, что
статистика~\eqref{LR} в случае спра\-вед\-ли\-вости нулевой гипотезы имеет
асимптотическое распределение, являющееся взвешенной суммой
$\chi^2$-рас\-пре\-де\-ле\-ний. Но данный критерий не является оптимальным
хотя бы в каком-то смысле. Однако в работе~\cite{Vuong1989}
показано, что для произвольной смеси вероятностных распределений,
удовлетворяющих условиям регулярности, в случае справедливости
альтернативы значение статистики стремится к бесконечности, т.\,е.\
такой критерий является состоятельным.

В данной работе предлагается альтернативный асимптотически
оптимальный критерий проверки гипотез о числе компонент смеси
вероятностных распределений в смысле максимизации предельной
мощности критерия (см., например, книгу~\cite{Bening2000}).

\section{Постановка задачи}

Предположим, что каждое из независимых наблюдений
$\textbf{X}_n\hm=(X_1,\ldots,X_n)$ имеет плотность, представимую в виде
конечной $K$-ком\-по\-нент\-ной смеси плотностей некоторых законов
распределения $\psi_i(x),\,i\hm=1,\ldots,K,$ вида
$$
\sum_{i=1}^{K}p_i\psi_i(x),\,\sum_{i=1}^{K}p_i=1,\,p_i\geqslant0\,, \,i=1,\ldots,K\,.
$$

Отметим, что всюду предполагается, что смесь идентифицируема (если
для конкретных распределений для этого требуются дополнительные
условия, они оговариваются отдельно). Пусть $k$~--- некоторое
известное натуральное число. Требуется проверить гипотезу
$$
H_0:\ K=k
$$
против альтернативы
$$
H_1: K=k+1\,.
$$
Другими словами, требуется проверить значимость $(k+1)$-й компоненты
(например, если веса $p_i,\,i\hm=1,\ldots,k+1,$ упорядочены по
убыванию). Такая задача довольно типична и возникает, когда нужно
убедиться в значимости компоненты с малым весом или отбросить ее без
значимой потери информативности модели. Подобные проблемы особенно
существенны для так называемых сеточных методов разделения смесей
(см., например,~\cite{Korolev2007, Korolev2010}).

Для удобства асимптотического анализа предлагаемых критериев сведем
описанную выше задачу проверки гипотез о значении \textit{дискретного}
па\-ра\-мет\-ра~$K$ к задаче проверки гипотез о значении
\textit{непрерывного} параметра. С~этой целью для некоторого
$\theta\in[0,1]$ будем считать, что $X_1$ имеет плотность
\begin{equation}
\label{p} p(x,\theta)=(1-\theta) f(x)+\theta  g(x)\,,
\end{equation}
где
$$
f(x)=\sum\limits_{i=1}^{k}p_i\psi_i\,;\quad 
\sum\limits_{i=1}^{k}p_i=1\,;\ \  g(x)=\psi_{k+1}\,.
$$

Требуется проверить простую гипотезу
$$
H_0:\,\theta=0\\
$$
против последовательности сложных альтернатив вида (так
как равномерно наиболее мощного критерия для проверки простой
гипотезы против сложной, как правило, не существует)
$$
H_{n,1}:\,\theta=\fr{t}{\sqrt{n}}\,,\enskip 0<t\leqslant C\,,\  C>0\,,
$$
где параметр $t$ неизвестен. Фактически  осуществляется проверка
гипотезы о том, является ли рас\-смат\-ри\-ва\-емая смесь $k$-ком\-по\-нент\-ной
(при спра\-вед\-ли\-вости нулевой гипотезы~$H_0$) или $(k+1)$-ком\-по\-нент\-ной
(при справедливости альтернативы~$H_{n,1}$).
{\looseness=1

}

\section{Асимптотически
наиболее мощный критерий проверки гипотез о~числе компонент смеси}

Используем асимптотический подход, подробно описанный, например, в
книге~\cite{Bening2000}. Согласно лемме Ней\-ма\-на--Пир\-со\-на, для любого
фиксированного $t\in(0,C]$ наилучший критерий проверки гипотезы~$H_0$ 
против простой альтернативы
\begin{equation*}
H_{n,1}:\theta=\fr{t}{\sqrt{n}} 
\end{equation*}
основан на логарифме отношения правдоподобия
\begin{equation}
\label{Lambda}
\Lambda_n(t)=\sum\limits_{i=1}^{n}\left(l(X_i,tn^{-1/2})-l(X_i,0)\right)\,,
\end{equation}
где
$$
l(x,\theta)=\log p(x,\theta)\,.
$$

Мощность такого критерия уровня $\alpha\in(0,1)$ обозначим через
$\beta^*_n(t)$. Хотя статистика $\Lambda_n(t)$ не может быть
использована для построения критерия проверки гипотезы~$H_0$ против
альтернативы $H_{n,1}$ из-за того, что $t$ неизвестно, однако
$\beta^*_n(t)$ задает верхнюю границу для мощности любого критерия
при проверке гипотезы~$H_0$ против фиксированной альтернативы~$H_{n,1}$, $t>0$.

В дальнейшем понадобятся следующие функции:
$$
l^{(j)}(x)=\fr{\partial^{j}}{\partial\theta^{j}}\,l(x,\theta)\biggl|_{\theta=0},
\enskip j=1,2,\ldots\,,
$$
т.\,е.\
\begin{equation}
\left.
\begin{array}{rl}
\!\!\!l^{(1)}(x)&=\fr{g(x)-f(x)}{(1-\theta)f(x)+\theta g(x)}\biggl|_{\theta=0}=
\fr{g(x)}{f(x)}-1\,;\\[12pt]
\!\!\!l^{(2)}(x)&=-\fr{(g(x)-f(x))^2}{((1-\theta)f(x)+\theta
g(x))^2}\biggl|_{\theta=0}={}\\[12pt]
&\hspace*{30mm}=-\left(\fr{g(x)}{f(x)}-1\right)^2\,;\\[9pt]
l^{(j)}(x)&={}\\[9pt]
&\!\!\!\!\!\!\!\!\!\!\!\!\!\!\!\!\!\!\!\!\!\!\!\!{}=(-1)^{j+1}(j-1)!\fr{(g(x)-f(x))^j}{((1-\theta)f(x)+\theta
g(x))^j}\biggl|_{\theta=0}={}\\[9pt]
&\!\!\!\!\!\!\!\!\!\!\!\!\!\!\!\!\!\!\!\!\!\!{}=(-1)^{j+1}(j-1)!\left(\fr{g(x)}{f(x)}-1\right)^j,\,j\geqslant1\,.
\end{array}\!
\right\}\!\!
\label{lj(x)}
\end{equation}

Фишеровская информация имеет вид:
\begin{multline}
\label{I}
I=\E_0\left(l^{(1)}(X_1)\right)^2=
\int\limits_{-\infty}^\infty\left(\fr{g(x)}{f(x)}-1\right)^2
f(x)\,dx={}\\
{}=\int\limits_{-\infty}^\infty
f(x)\,dx-2\int\limits_{-\infty}^\infty
g(x)\,dx+\int\limits_{-\infty}^\infty
\fr{g^2(x)}{f(x)}\,dx={}\\
{}=\int\limits_{-\infty}^\infty
\fr{g^2(x)}{f(x)}\,dx-1\,.
\end{multline}
В дальнейшем будем пользоваться известными соотношениями:
\begin{equation}
\label{Fisher} \E_0 l^{(1)}(X_1)=0,\quad\E_0 l^{(2)}(X_1)=-I\,.
\end{equation}
Запишем разложение в ряд Тейлора логарифмической
производной из выражения~\eqref{Lambda}:
\begin{multline*}
 l(X_i,tn^{-1/2})-l(X_i,0)=\fr
{t}{\sqrt{n}}l^{(1)}(X_i)+{}\\
{}+\fr{t^2}{2n}l^{(2)}(X_i)+\cdots
\end{multline*}
Тогда
\begin{equation}
\label{Lambda_expansion} 
\Lambda_n(t)=tL_n^{(1)}-\fr{t^2I}{2}+\fr {t^2}{2\sqrt{n}} L_n^{(2)}+o(n^{-3/2})\,,%\cdots
\end{equation}
где
\begin{equation*}
\label{Ln1}
L_n^{(j)}=\fr{1}{\sqrt{n}}\sum\limits_{i=1}^{n}(l^{(j)}(X_i)-\E_0
l^{(j)}(X_i))\,,\enskip j=1,2,\ldots
\end{equation*}

Для получения формулы~\eqref{Lambda_expansion} были использованы
соотношения~\eqref{Fisher}. Критерий, основанный на статистике
$\Lambda_n(t)$, отвергает гипотезу~$H_0$ в пользу альтернативы~$H_{n,1}$, если
\begin{equation*}
\Lambda_n(t)>c_{n,t},
\end{equation*}
где критическое значение $c_{n,t}$ выбирается из условия
$$
\Pr_{n,0}(\Lambda_n(t)>c_{n,t})=\alpha\,,
$$
где символом $\Pr_{n,\theta}$ обозначено распределение~$\textbf{X}_n$ при $\theta\geqslant0$.

Так как $\Lambda_n(t)$ представляет собой сумму независимых
одинаково распределенных случайных величин, то из центральной
предельной теоремы (ЦПТ) при $n\rightarrow\infty$ следует, что
$\Lambda_n(t)$ имеет асимптотически нормальное распределение вида
\begin{align}
\label{DistrLn1} 
\mathfrak{L}(\Lambda_n(t)\mid H_0)&\rightarrow
N(-\fr{1}{2}\,t^2I,t^2I)\,;\\
\label{DistrLn2}
\mathfrak{L}(\Lambda_n(t)\mid H_{n,1})&\rightarrow
N(\fr{1}{2}\,t^2I,t^2I)\,.
\end{align}
Отсюда несложно получить предельное значение мощности (см.,
например, книгу~\cite{Bening2000}):

\noindent
$$
\beta^*_n(t)\rightarrow\beta^*(t)=\Phi(t\sqrt{I}-u_\alpha)\,,
$$
где $\Phi(u_\alpha)\hm=1\hm-\alpha$, символ $\Phi(\cdot)$ обозначает
функцию распределения стандартного нормального закона.

В качестве критерия с предельной мощностью $\beta^*(t)$ для
различения гипотез о числе компонент рассмотрим критерий, основанный
на статистике $L_n^{(1)}$, т.\,е.\
\begin{equation}
\label{L_n^1}L_n^{(1)}=\fr{1}{\sqrt{n}}\sum\limits_{i=1}^{n}\left(\fr{g(X_i)}{f(X_i)}-1\right)
\end{equation}

Рассмотрим достаточные условия, при которых логарифм отношения
правдоподобия асимптотически нормален, т.\,е.\ выполнены
условия~\eqref{DistrLn1} и~\eqref{DistrLn2} при $n\hm\to\infty$,
$0\hm<t\hm\leqslant C$. Это так называемое условие локальной асимптотической
нор\-маль\-ности, точнее возможность представить логарифм отношения
правдоподобия $\Lambda_n(t)$ в виде
\begin{equation}
\label{LAN} \Lambda_n(t)=t L_n^{(1)}-\fr{t^2 I}{2}+\xi_n(t)\,,
\end{equation}
где остаточный член $\xi_n(t)$ стремится к нулю по вероятности при
гипотезе~$H_0$ при $n\to\infty$.

Хорошо известен следующий результат (см., например,
статью~\cite{Hajek1962}). Предположим, что плотность $p(x,\theta)$
удовлетворяет следующим условиям:
\begin{description}
\item{{(A)}} При каждом $x\in\r$ плотность $p(x,\theta)$
абсолютно непрерывна по~$\theta$ в некоторой окрестности точки
$\theta=0$.
\item{{(B)}} При каждом $\theta$ из этой окрестности производная
$\partial p(x,\theta)/\partial\theta$ существует при почти
всех (по мере Лебега) $x\in\r$.
\item{{(C)}} Функция 
$$
I(\theta)\equiv\E_\theta\left(\fr{\partial
\log p(x,\theta)}{\partial\theta}\right)^2<\infty
$$ 
положительна и непрерывна в этой окрест\-ности.
\end{description}
Тогда выполнено условие~\eqref{LAN}.

Проверим выполнение этих условий для плотности $p(x,\theta)$ из
равенства~\eqref{p} и сформулируем результат в следующем виде.

\medskip

\noindent
\textbf{Лемма~1.} \textit{Пусть фишеровская информация~$I$ из
соотношения}~\eqref{I} \textit{конечна. Тогда для плотности $p(x,\theta)$ из
равенства}~\eqref{p} \textit{выполнено соотношение}~\eqref{LAN}.

\medskip

\noindent
Д\,о\,к\,а\,з\,а\,т\,е\,л\,ь\,с\,т\,в\,о\ будет заключаться в последовательной проверке условий
$(A)$--$(C)$. Обозначим через~$\delta$ необходимую правую окрестность
точки $\theta\hm=0$. Всюду в дальнейшем подразумевается выполнение
условия $0\leqslant\theta<\delta$.

\noindent {(\textit{A})} Очевидно, что линейная функция является абсолютно
непрерывной, поскольку если
\begin{equation*}
\sum\limits_i(b_i-a_i)<\delta_1\,,
\end{equation*}
где $(a_i,b_i)$~--- произвольная система попарно непересекающихся
интервалов, то для произвольной линейной функции вида $y(x)=ax+b$, $a$ и $b$~--- 
конечные фиксированные числа, получим
\begin{equation*}
\sum\limits_i|y(b_i)-y(a_i)|=|a|\sum\limits_i(b_i-a_i)<|a|\delta_1=\varepsilon
\end{equation*}
при соответствующем выборе~$\delta_1$. Плотность $p(x,\theta)$
представима в виде:
$
\theta (g(x)-f(x))+f(x)$,
т.\,е.\ при каждом фиксированном $x\in\r$
является линейной функцией по~$\theta$, а значит, является абсолютно
непрерывной по~$\theta$ из правой $\delta$-окрест\-ности нуля.

\noindent {(\textit{B})} Найдем производную
\begin{equation*}
\fr{\partial p(x,\theta)}{\partial\theta}=\fr{\partial
((1-\theta) f(x)+\theta g(x))}{\partial\theta}=g(x)-f(x)\,.
\end{equation*}
Очевидно, что данная производная существует при почти всех (по мере
Лебега) $x\in\r$ для любого~$\theta$ из правой $\delta$-окрест\-ности
нуля.

\noindent {(\textit{C})} Запишем выражение для функции $I(\theta)$ более
подробно:
\begin{multline*}
I(\theta)=\E_\theta\left(\fr{\partial \log
p(x,\theta)}{\partial\theta}\right)^2={}\\
{}=\E_\theta\left(\fr{g(x)-f(x)}{(1-\theta)
f(x)+\theta g(x)}\right)^2\,.
\end{multline*}
Функция $g(x)-f(x)\neq0$ почти наверное, а в силу известного
свойства интеграла Лебега это означает, что функция $I(\theta)\hm>0$ (в
силу неотри\-ца\-тель\-ности подынтегрального выражения условие
$I(\theta)\hm\neq 0$ эквивалентно $I(\theta)\hm>0$) почти наверное для
любого~$\theta$ из правой $\delta$-окрест\-ности нуля (включая и
значение $\theta=0$).

Далее, используя тот факт, что $0\hm\leqslant\theta\hm<\delta$, получаем оценку
для подынтегральной функции
\begin{multline}
\label{Ival} 
\!\!\!\fr{(g(x)-f(x))^2}{(1-\theta) f(x)+\theta
g(x)}\leqslant\fr{(g(x)-f(x))^2}{(1-\theta) f(x)+\theta
g(x)}\;\leqslant\\
{}\leqslant \fr{1}{(1-\delta)}\fr{(g(x)-f(x))^2}{f(x)}\,.
\end{multline}

Функция, стоящая в правой части~\eqref{Ival}, пред\-став\-ля\-ет собой
подынтегральное выражение для интеграла в фишеровской информации~$I$ 
из соотношения~\eqref{I}, который конечен по предположению леммы.

Воспользовавшись теоремой Лебега о мажорируемой сходимости (см.,
например, книгу~\cite{KF1976}), получим:
\begin{multline*}
\lim\limits_{\theta\to\theta_0}\int\fr{(g(x)-f(x))^2}{(1-\theta)
f(x)+\theta
g(x)}\,dx={}\\
{}=\int\fr{(g(x)-f(x))^2}{(1-\theta_0)
f(x)+\theta_0 g(x)}\,dx\,.
\end{multline*}
Данное соотношение означает непрерывность функции $I(\theta)$ в
правой $\delta$-окрест\-ности нуля (включая и значение $\theta=0$).\hfill$\square$

\medskip

Лемма~1 означает, что критерий $L_n^{(1)}$ из
равенства~\eqref{L_n^1} является асимптотически наиболее мощным.
Согласно ЦПТ, $L_n^{(1)}$ при $n\hm\rightarrow\infty$ имеет нормальное
распределение с параметрами~$0$ и~$I$ (при справедливости нулевой
гипотезы). Тогда критическое значение~$c^{(1)}_n$ может быть найдено
из соотношений:
$$
\Pr_{n,0}(L_n^{(1)}>c^{(1)}_n)=\alpha\,;\enskip 
c^{(1)}_n=\sqrt{I}u_\alpha+o(1)\,.
$$

Для отыскания точного распределения $L_n^{(1)}$ выпишем его
характеристическую функцию, пользуясь тем, что элементы выборки~---
независимые одинаково распределенные случайные величины
\begin{multline}
\label{Xf}\phi_{L_n^{(1)}}(z)=\left(e^{-iz/\sqrt{n}}\phi_\xi\left(\fr{z}{\sqrt
n}\right)\right)^n={}\\
{}=e^{-iz\sqrt{n}}\left(\phi_\xi\left(
\fr {z}{\sqrt n}\right)\right)^n,\,z\in\r\,;
\end{multline}

\vspace*{-6pt}

\noindent
\begin{multline*}
\xi=\fr{g(X_1)}{f(X_1)}\,;\ \ \ \ \phi_\xi(\fr {z}{\sqrt
n})=\E_\theta \exp{\left\{i\fr{z}{\sqrt
n}\,\xi\right\}}={}\\
{}=\int\limits_{-\infty}^\infty\exp{\left\{\fr {iz}{\sqrt
n}\,\fr{g(x)}{f(x)}\right\}}((1-\theta) f(x)+\theta
 g(x))\,dx\,.
\end{multline*}
Аналогично, используя соотношения
\begin{align*}
L_n^{(1)}&=\fr{1}{\sqrt{n}}\,\sum_{i=1}^n\fr{g(X_i)}{f(X_i)}-\sqrt{n}\,;
\\
\fr{g(X_i)}{f(X_i)}&\geqslant 0\,,\enskip  i=1,\ldots,n\,,
\end{align*}
можно применить аппарат преобразования Лапласа и записать
преобразование Лапласа случайной величины
$$
\eta_n=\fr{1}{\sqrt{n}}\sum_{i=1}^n\fr{g(X_i)}{f(X_i)}
$$
в виде
$$
\Phi_{\eta_n}(s)=\left(\Phi_\xi\left(\fr {s}{\sqrt
n}\right)\right)^n\,,
$$
где

\noindent
\begin{multline*}
\Phi_\xi\left(s\right)=\E_\theta
\exp{\left\{-s\xi\right\}}={}\\
{}=\int\limits_{-\infty}^\infty\exp{\left\{-s\fr{g(x)}{f(x)}\right\}}((1-\theta)
f(x)+\theta  g(x))\,dx\,.
\end{multline*}
Теперь, так как распределение неотрицательной случайной величины~$\eta_n$ 
однозначно определяется ее преобразованием Лапласа (см.,
например,~\cite{Feller2010}), то для практического определения
распределения случайной величины $L_n^{(1)}$ при конкретных~$f(x)$ и~$g(x)$ 
мож\-но использовать процедуры численного обращения
преобразования Лапласа.

\subsection{Асимптотическое поведение разности мощностей}

В работе~\cite{Bening2000} для нормированного предела раз\-ности
мощностей (также называемого \textit{потерей мощ\-ности}) для критерия,
основанного на статистике~\eqref{L_n^1}, получено выражение:

\noindent
\begin{multline}
\label{r(t)}
r(t)=\fr{t^3}{8\sqrt{I}}\varphi\left(u_\alpha-t\sqrt{I}\right)\left[\Dd_0
l^{(2)}(X_1)-{}\right.\\
\left.{}-I^{-1}\E_0^2l^{(1)}(X_1)l^{(2)}(X_1)\right]\,,
\end{multline}
где символ $\varphi(\cdot)$ обозначает плотность стандартного
нормального распределения. Введем обозначение для моментов порядка~$s$ 
случайной величины $\xi={{g(X_1)}/{f(X_1)}}$:

\noindent
\begin{multline}
\label{Psis}
\Psi_s=\E_0\xi=\E_0\left(\fr{g(X_1)}{f(X_1)}\right)^s=
\int\limits_{-\infty}^{+\infty}\fr{g^s(x)}{f^{s-1}(x)}\,dx,
\\ \ s=2,3,4\,.
\end{multline}
В этих обозначениях фишеровская информация~\eqref{I} равна

\noindent
\begin{equation}
\label{PsiI}
I=\Psi_2-1\,.
\end{equation}
Запишем величины, входящие в формулу~\eqref{r(t)}, с учетом
обозначений~\eqref{Psis} и формулы~\eqref{lj(x)} в виде

\noindent
\begin{align*}
\!\E_0 l^{(1)}(X_1)l^{(2)}(X_1)&=-\int\limits_{-\infty}^{+\infty}
\left(\fr{g(x)}{f(x)}-1\right)^3f(x)\,dx={}\hspace*{-1.25272pt}\\
&\hspace*{-27mm}{}=-\int\limits_{-\infty}^{+\infty}\left(\fr{g^3(x)}{f^3(x)}-3\fr{g^2(x)}{f^2(x)}
+3\fr{g(x)}{f(x)}-1\right)f(x)\,dx={}\\
&{}=-\Psi_3+3\Psi_2-2\,;\\
\E_0^2l^{(1)}(X_1)l^{(2)}(X_1)&=\left(-\Psi_3+3\Psi_2-2\right)^2={}\\
&\hspace*{-17mm}{}=\Psi_3^2-6\Psi_2\Psi_3+4\Psi_3+9\Psi_2^2-12\Psi_2+4\,.
\end{align*}


\noindent
С учетом формул~\eqref{Fisher} и~\eqref{PsiI} имеем:
\begin{multline*}
\Dd_0
l^{(2)}(X_1)=\E_0\left(l^{(2)}(X_1)\right)^2-\left(\E_0l^{(2)}(X_1)\right)^2={}\\
{}=\E_0\left(l^{(2)}(X_1)\right)^2-\left(\Psi_2-1\right)^2\,;
\end{multline*}

\vspace*{-6pt}

\noindent
\begin{multline*}
\E_0\left(l^{(2)}(X_1)\right)^2=\int\limits_{-\infty}^{+\infty}\left(
\fr{g(x)}{f(x)}-1\right)^4f(x)\,dx={}\\
{}=\int\limits_{-\infty}^{+\infty}\left(\fr{g^4(x)}{f^4(x)}-4\fr{g^3(x)}{f^3(x)}
+6\fr{g^2(x)}{f^2(x)}-4\fr{g(x)}{f(x)}+{}\right.\\
\left.{}+1
\vphantom{\fr{g^2(x)}{f^2(x)}}\right)f(x)\,dx
=\Psi_4-4\Psi_3+6\Psi_2-3\,;
\end{multline*}

\vspace*{-6pt}
\noindent
$$
\Dd_0 l^{(2)}(X_1)=\Psi_4-4\Psi_3+6\Psi_2-3-\left(\Psi_2-1\right)^2\,.
$$
Окончательно получаем для величины $r(t)$ из формулы~\eqref{r(t)}
следующее соотношение:
\begin{multline}
r(t)=\fr{t^3}{8\sqrt{\Psi_2-1}}\varphi\left(u_\alpha-t\sqrt{\Psi_2-1}\right)\times{}\\
{}\times\left(\Dd_0
l^{(2)}(X_1)-I^{-1}\E_0^2 l^{(1)}(X_1)l^{(2)}(X_1)\right)={}\\
{}=\fr{t^3}{8\sqrt{\Psi_2-1}}\varphi\left(u_\alpha-t\sqrt{\Psi_2-1}\right)\times
\\
{}\times\left(\Psi_4-4\Psi_3+6\Psi_2-3-\Psi^2_2+2\Psi_2-1+{}\right.\\
\left.{}+6\Psi_3-9\Psi_2+3
-\fr{\left(\Psi_3-1\right)^2}{\Psi_2-1}\right)={}\\
{}=
\fr{t^3}{8\sqrt{\Psi_2-1}}\,\varphi\left(u_\alpha-t\sqrt{\Psi_2-1}\right)\times{}\\
{}\times
\left(\Psi_4+2\Psi_3-\Psi^2_2-\Psi_2-\fr{\left(\Psi_3-1\right)^2}{\Psi_2-1}-1\right)\,.
\label{r(t)Ln}
\end{multline}
С помощью величины~$r(t)$ можно найти асимптотический дефект, так
как (см., например, книгу~\cite{Bening2000}) с учетом формул~\eqref{PsiI}
и~\eqref{r(t)Ln}
%\end{multicols}
%\hrule
\begin{multline}
d=\lim\limits_{n\to\infty}
d_n\equiv\lim\limits_{n\to\infty}(k_n-n)=
\fr{2r(t)}{t\sqrt{I}\varphi(t\sqrt{I}-u_\alpha)}={}\\
{}=2\fr{t^3}{8\sqrt{\Psi_2-1}}\,\varphi
\left(u_\alpha-t\sqrt{\Psi_2-1}\right)\times\\
{}\times
\left(\Psi_4+2\Psi_3-\Psi^2_2-\Psi_2-\fr{\left(\Psi_3-1\right)^2}{\Psi_2-1}-1\right)\Bigg /{}\\
{}
\left(t\sqrt{\Psi_2-1}\varphi(t\sqrt{\Psi_2-1}-u_\alpha)\right)={}\\
{}=\fr{t^2}{4\left(\Psi_2-1\right)}
\left(\vphantom{\fr{\left(\Psi_3-1\right)^2}{\Psi_2-1}}
\Psi_4+2\Psi_3-\Psi^2_2-\Psi_2-{}\right.\\
\left.{}-\fr{\left(\Psi_3-1\right)^2}{\Psi_2-1}-1\right)\,.
\label{d}
\end{multline}
Здесь через $d_n$ обозначен дефект, $k_n$~--- число наблюдений,
необходимых критерию, основанному на статистике $L_n^{(1)}$ из
формулы~\eqref{L_n^1}, для достижения той же мощности, что и
критерию, основанному на статистике $\Lambda_n(t)$ из
формулы~\eqref{Lambda}, при альтернативах вида $t/\sqrt{n}$. Первое
равенство в соотношении~\eqref{d} понимается в том смысле, что если
предел существует и конечен, то он, по определению, называется
асимптотическим дефектом.

\subsection{Условия сходимости моментных характеристик~$\Psi_s$}

Рассмотрим условия, гарантирующие конеч\-ность моментных
характеристик~\eqref{Psis} для не\-ко\-торых частных случаев смесей
распределений. Приведем подробный вывод для фишеровской\linebreak
информации~\eqref{I}, т.\,е.\ для моментной характеристики~$\Psi_2$.
Для остальных условия получаются аналогично с учетом того, что
\begin{align*}
\left(\sum\limits_i a_i\right)^2&=\sum\limits_i
a_i^2+2\sum\limits_{i\neq j} a_i a_j\,;\\
\left(\sum\limits_{i=1}^k
a_i\right)^3&=\sum\limits_i a_i^3+3\sum\limits_{i\neq j} a_i
a_j^2+k\prod\limits_i a_i\,,
\end{align*}
и для любых $a>0$, $b\geqslant0$ справедливо неравенство
$$
\fr{1}{a+b}\leqslant\fr{1}{a}\,.
$$

\textbf{Нормальное распределение.} В~этом случае
\begin{align*}
f(x)&=\sum\limits_{i=1}^{k}p_i\fr{1}{\sigma_i\sqrt{2\pi}}
\exp{\left\{-\fr{(x-a_i)^2}{2\sigma_i^2}\right\}};\
\sum\limits_{i=1}^{k}p_i=1\,,\\
g(x)&=\fr{1}{\sigma_{k+1}\sqrt{2\pi}}\exp{\left\{-\fr{(x-a_{k+1})^2}{2\sigma_{k+1}^2}\right\}}\,.
\end{align*}
Поэтому
\begin{multline*}
\fr{g^2(x)}{f(x)}=
{\exp\left\{-\fr{(x-a_{k+1})^2}{\sigma_{k+1}^2}\right\}}\Bigg/\\[3pt]
\left({\sqrt{2\pi}\sigma^2_{k+1}\sum_{j=1}^k\fr{p_j}{\sigma_j}
\exp\left\{-\fr{(x-a_j)^2}{2\sigma_j^2}\right\}}\right)={}\\[3pt]
{}={\exp\left\{-\fr{1}{\sigma_{k+1}^2}(x^2-2a_{k+1}x+a_{k+1}^2)\right\}}\Bigg /\\[3pt]
\!\!\!\!\left(\!{\sqrt{2\pi}\sigma^2_{k+1}
\sum_{j=1}^k\fr{p_j}{\sigma_j}
\exp\!\left\{\!-\fr{1}{2\sigma_j^2}(x^2-2a_jx+a_j^2)\!\right\}}\!\right)={}
\end{multline*}


\noindent
\begin{multline}
{}=\left(
\sqrt{2\pi}\sigma^2_{k+1}\sum_{j=1}^k\fr{p_j}{\sigma_j}
\exp\left\{x^2\left(\fr{1}{\sigma_{k+1}^2}-\fr{1}{2\sigma_j^2}\right)
+{}\right.\right.\\
\!\!\!\left.\left.{}+2x\left(\fr{a_j}{2\sigma_j^2}-\frac{a_{k+1}}{\sigma_{k+1}^2}\right)
+\left(\fr{a_{k+1}^2}{\sigma_{k+1}^2}-\fr{a_j^2}{2\sigma_j^2}\right)\right\}
\vphantom{\left(
\sqrt{2\pi}\sigma^2_{k+1}\sum_{j=1}^k\fr{p_j}{\sigma_j}
\exp\left\{x^2\left(\fr{1}{\sigma_{k+1}^2}-\fr{1}{2\sigma_j^2}\right)
+{}\right.\right.}
\right)^{-1}\!\!\!.\!\!\!
\label{inv}
\end{multline}
При $|x|\to\infty$ парабола относительно~$x$ вида

\noindent
\begin{multline*}
y_j(x)=x^2\left(\fr{1}{\sigma_{k+1}^2}-\fr{1}{2\sigma_j^2}\right)
+2x\left(\fr{a_j}{2\sigma_j^2}-\frac{a_{k+1}}
{\sigma_{k+1}^2}\right)+{}\\
{}+\left(\fr{a_{k+1}^2}{\sigma_{k+1}^2}-\fr{a_j^2}{2\sigma_j^2}\right)
\end{multline*}
может иметь пределом либо $+\infty$, либо $-\infty$. Первая
ситуация, очевидно, имеет место, если
$$
\fr{1}{\sigma_{k+1}^2}-\fr{1}{2\sigma_j^2}=
\fr{2\sigma_j^2-\sigma_{k+1}^2}{2\sigma_j^2\sigma_{k+1}^2}>0\,,
$$
что выполнено тогда и только тогда, когда
$$
\sigma_{k+1}^2<2\sigma_j^2\,.
$$
Поэтому, если выполнено условие
\begin{equation}
\label{CondSigmaNorm}
\sigma_{k+1}^2<2\max_{1\le j\le k}\sigma_j^2\,,
\end{equation}
то функция в знаменателе соотношения~\eqref{inv} неограниченно
возрастает при $|x|\hm\to\infty$, являясь при этом величиной порядка
$O\left(e^{\alpha x^2}\right)$ при некотором $\alpha\hm>0$. Обозначим
$\sigma_{j_0}^2=\max_{1\le j\le k}\sigma_j^2$. В~таком случае,
поскольку $\sigma_{k+1}^{-2}-({1}/{2})\sigma_{j_0}^{-2}>0$ и
\begin{multline*}
\fr{1}{\sigma_{k+1}^2}-\frac{1}{2\sigma_{j_0}^2}-
\left(\fr{1}{\sigma_{k+1}^2}-\fr{1}{2\sigma_i^2}\right)={}\\
{}=
\fr{1}{2}\left(\fr{1}{\sigma_{j_0}^2}-\fr{1}{\sigma_i^2}\right)<0\,,\enskip i=1,\ldots,k\,,
\end{multline*}

\vspace*{-12pt}

\noindent
\begin{multline*}
\fr{g^2(x)}{f(x)}=\bigg(C\exp\left\{x^2
\left(\fr{1}{\sigma_{k+1}^2}-\fr{1}{2\sigma_{j_0}^2}\right)
+{}\right.\\
\left.{}+2x\left(\fr{a_{j_0}}{2\sigma_{j_0}^2}-\fr{a_{k+1}}{\sigma_{k+1}^2}\right)
+\left(\fr{a_{k+1}^2}{\sigma_{k+1}^2}-\fr{a_{j_0}^2}{2\sigma_{j_0}^2}\right)\right\}
\times{}\\
{}\times\left(1+O\left(e^{-x^2}\right)\right)\bigg)^{-1}={}\\
{}=\left({O\left(e^{\alpha x^2}\right)\left(1+O\left(e^{-x^2}\right)\right)}\right)^{-1}=
O\left(e^{-\alpha x^2}\right)\,,\\
\alpha>0\,,\enskip C>0\,,\enskip  |x|\to\infty\,,
\end{multline*}
и интеграл в соотношении~\eqref{I} конечен.
%\columnbreak

Пусть вместо условия~\eqref{CondSigmaNorm} выполнено соотношение
$$
\sigma_{k+1}^2=2\sigma_{j_0}^2\,,
$$

\columnbreak


\noindent
Тогда для любого $\alpha$
\begin{multline*}
\fr{g^2(x)}{f(x)} =\fr{1}{e^{\alpha x+C}
\left(1+O\left(e^{-x^2}\right)\right)}={}\\
{}=\fr{1}{O\left(e^{\alpha
x}\right)\left(1+O\left(e^{-x^2}\right)\right)}=O\left(e^{-\alpha
x}\right)\,,\ |x|\to\infty.\hspace*{-2.57826pt}
\end{multline*}

Таким образом, можно заключить, что для конечности фишеровской
информации из формулы~\eqref{I} достаточно, чтобы было выполнено
условие~\eqref{CondSigmaNorm}.

Предположим теперь, что
\begin{equation}
\label{invCondSigmaNorm}
\sigma_{k+1}^2>2\max_{1\le j\le
k}\sigma_j^2\,,
\end{equation}
а интеграл из соотношения~\eqref{I} сходится. Из
неравенства~\eqref{invCondSigmaNorm} следует, что все коэффициенты
при $x^2$ в равенстве~\eqref{inv} отрицательны:
\begin{multline*}
\fr{1}{\sigma_{k+1}^2}-\fr{1}{2\sigma_{j_0}^2}-
\left(\fr{1}{\sigma_{k+1}^2}-\frac{1}{2\sigma_i^2}\right)={}\\
{}=
\fr{1}{2}\left(\fr{1}{\sigma_{j_0}^2}-\fr{1}{\sigma_i^2}\right)<0\,,\enskip i=1,\ldots,k\,.
\end{multline*}
Поэтому
\begin{multline*}
\fr{g^2(x)}{f(x)}=\Bigg( C\exp\left\{x^2
\left(\fr{1}{\sigma_{k+1}^2}-\fr{1}{2\sigma_{j_0}^2}\right)
+{}\right.\\
\left.{}+2x\left(\fr{a_{j_0}}{2\sigma_{j_0}^2}-\fr{a_{k+1}}{\sigma_{k+1}^2}\right)
+\left(\fr{a_{k+1}^2}{\sigma_{k+1}^2}-\fr{a_{j_0}^2}{2\sigma_{j_0}^2}\right)\right\}
\times{}\\
{}\times\sqrt{2\pi}\sigma^2_{k+1}\cdot
\left(1+O\left(e^{-x^2}\right)\right)\Bigg)^{-1}={}\\
{}=\left({O\left(e^{-\alpha
x^2}\right)\left(1+O\left(e^{-x^2}\right)\right)}\right)^{-1}=
O\left(e^{\alpha x^2}\right)\,,\\ \alpha>0\,,\enskip C>0\,,\enskip    |x|\to\infty\,,
\end{multline*}
поэтому интеграл из равенства~\eqref{I} расходится (аналогично
вычислению интеграла $\int\limits_{-\infty}^\infty e^{-x^2}\,dx$ переходим
к полярным координатам и получаем расходящийся интеграл
$\int\limits_0^\infty e^t\,dt$). Полученное противоречие доказывает, что
условие~\eqref{CondSigmaNorm} является необходимым и достаточным для
существования фишеровской информации в данном случае.

Вообще, достаточным условием сходимости моментных характеристик,
определяемых соотношением~\eqref{Psis}, в случае конечной смеси
нормальных законов является условие:

\noindent
$$
\sigma_{k+1}^2<\fr{s}{s-1}\max_{1\le j\le
k}\sigma_j^2\,,\enskip s=2,3,4\,,
$$
для каждой из моментных характеристик $\Psi_s$. Таким образом, для
конечности моментных характеристик~\eqref{Psis} сразу для всех
$s\hm=2,3,4$ достаточно выполнения условия:
\begin{equation}
\label{CondPsiNorm} 
\sigma_{k+1}^2<\fr{4}{3}\,\max_{1\le j\le
k}\sigma_j^2\,.
\end{equation}

\textbf{Гамма-распределение.} В~этом случае
\begin{gather*}
f(x)=\sum\limits_{i=1}^{k}p_i\fr{\alpha_i^{\beta_i}}
{\Gamma(\beta_i)}x^{\beta_i-1}e^{-\alpha_ix}\,;
\quad\sum\limits_{i=1}^{k}p_i=1\,;\\
g(x)=\fr{\alpha_{k+1}^{\beta_{k+1}}}
{\Gamma(\beta_{k+1})}x^{\beta_{k+1}-1}e^{-\alpha_{k+1}x},\,x\geqslant0\,.
\end{gather*}
Имеем
\begin{multline*}
\fr{g^2(x)}{f(x)}=\fr{\alpha_{k+1}^{2\beta_{k+1}}}
{\Gamma^2(\beta_{k+1})}\,x^{2\beta_{k+1}-2}e^{-2\alpha_{k+1}x}\Bigg/\\
{\sum\limits_{i=1}^{k}p_i \fr{\alpha_i^{\beta_i}}
{\Gamma(\beta_i)}x^{\beta_i-1}e^{-\alpha_ix}}={}\\
\!{}=\fr{\alpha_{k+1}^{2\beta_{k+1}}}{\Gamma^2(\beta_{k+1})}\,
\left({\sum\limits_{i=1}^{k}\fr{p_i\alpha_i^{\beta_i}}{\Gamma(\beta_i)}
{x^{\beta_i-2\beta_{k+1}+1}
e^{\left(2\alpha_{k+1}-\alpha_i\right)x}}}\right)^{-1}\!\!\!.\hspace*{-7.35753pt}
\end{multline*}
Сходимость интеграла из равенства~\eqref{I} зависит от показателей
функций в знаменателе. Отметим, что любое слагаемое, содержащее
множитель вида~$x^{1-\nu}$, $\nu\hm>1,$ имеет особенность в нуле, а
любое слагаемое, содержащее множитель вида~$e^{\mu x}$, $\mu\hm>0,$
имеет особенность на~$+\infty$. Таким образом, для сходимости
подынтегральной функции в~$0$ и на~$+\infty$ в знаменателе должны
быть слагаемые с подобными множителями. Итак, для ограниченности
подынтегральной функции хотя бы для одного номера~$i$ должно
существовать $\nu_i\hm>1$ и хотя бы для одного номера~$j$ должно
существовать $\mu_j>0$:
\begin{equation*}
\fr{g^2(x)}{f(x)}=O\left(x^{\nu_i-1}e^{-\mu_j x}\right) A(x)\,,
\end{equation*}
где $A(x)$~--- ограниченная функция, не имеющая особенностей в~0 
и на~$+\infty$, поэтому можно считать, что $A(x)\leqslant C$ и
\begin{equation*}
\fr{g^2(x)}{f(x)}=O\left(x^{\nu_i-1}e^{-\mu_j x}\right)\,,\enskip x\to0,+\infty\,.
\end{equation*}

Учитывая вид гамма-функ\-ции и сказанное ранее, получаем, что для
интегрируемости необходимо, чтобы неравенства $\nu_i\hm>1$, $\mu_j\hm>0$
выполнялись хотя бы для одного номера~$i$ и~$j$ соответственно.
Получим
\begin{align*}
\nu_i=2\beta_{k+1}-\beta_i>1&\Rightarrow\beta_{k+1}>\frac12\min\limits_{1\le
i\le k}
{(\beta_i+1)}\,;\\
\mu_j=2\alpha_{k+1}-\alpha_j>0&\Rightarrow\alpha_{k+1}>\frac12\min\limits_{1\le
j\le k}{\alpha_j}\,.
\end{align*}

Отдельно рассмотрим случай наличия  в знаменателе слагаемых вида~$e^{\mu_j x}$ 
(т.\,е.\ случай, когда $\nu_i\hm=1$ для некоторого номера~$i$). 
Из очевидного неравенства 
$$
\fr{1}{e^{\mu x}+b}\leqslant\fr{1}{e^{\mu x}}\,,\quad b\geqslant0\,,
$$ 
следует, что интеграл из соотношения~\eqref{I} сходится при $\nu_i\hm=1$, если $\mu_j\hm>0$. Если
же $\mu_j\hm\leqslant0$, то выводы о сходимости интеграла делаются на
основании поведения функций в сумме, для которой предполагается
отсутствие в ней слагаемых вида~$e^{\mu_j x}$, а этот случай был
рассмотрен ранее. Итак, интеграл сходится в случае, когда выполнены
условия
$$ %\label{CondSigmaGamma}
\beta_{k+1}\geqslant\fr{1}{2}\min\limits_{1\le i\le
k}{(\beta_i+1)}\,;\enskip \alpha_{k+1}>\fr{1}{2}\min\limits_{1\le j\le k}{\alpha_j}\,.
$$

Вообще, достаточным условием сходимости моментных характеристик в
соотношении~\eqref{Psis} в случае конечной смеси гам\-ма-рас\-пре\-де\-ле\-ний
являются условия
\begin{align*}
\beta_{k+1}&\geqslant s^{-1}\min\limits_{1\le i\le
k}{((s-1)\beta_i+1)}\,;\\
\alpha_{k+1}&>\fr{s-1}s\min\limits_{1\le j\le k}{\alpha_j}\,,\enskip s=2,3,4\,,
\end{align*}
для каждой из моментных характеристик~$\Psi_s$. Заметим, что при
$s\hm\geqslant2$, $\beta_0\hm=\min\limits_{1\le i\le k}{\beta_i}$ неравенство
$$
\fr{1}{2}\,\beta_0+\fr{1}{2}\leqslant\fr{s-1}{s}\,\beta_0+\fr{1}{s}
$$
справедливо только при условии $\beta_0\hm>1$. Таким образом, для
конечности моментных характеристик~\eqref{Psis} сразу для всех
$s\hm=2,3,4$ достаточные условия приобретают вид:
\begin{equation}
\left.
\begin{array}{rl}
\beta_{k+1}&\geqslant\\
&\!\!\!\!\!\!\!\!\!\!\!\!\!\!\!\!\!\geqslant\max\left\{\fr{1}{4}\min\limits_{1\le
i\le k}{(3\beta_i+1)},\fr{1}{2}\min\limits_{1\le i\le
k}{(\beta_i+1)}\right\}\,;\\[9pt]
\alpha_{k+1}&>\fr{3}{4}\min\limits_{1\le j\le
k}{\alpha_j}\,.
\end{array}\!
\right\}\!\!
\label{CondPsiGamma}
\end{equation}

\medskip

\noindent
\textbf{Замечание 1.} Предположим, что при проверке гипотез на реальных
данных добавляется компонента, для которой условия
вида~\eqref{CondPsiNorm} или~\eqref{CondPsiGamma} не выполнены.
Приведем алгоритм, позволяющий считать данные условия выполненными
без потери общности. Будем последовательно из плотности~$f(x)$
исключать компоненты, для которых условия
вида~\eqref{CondPsiNorm},~\eqref{CondPsiGamma}
выполнены. Вместо этой компоненты в~$f(x)$ добавим компоненту~$g(x)$
и перенормируем веса так, чтобы они в сумме равнялись единице. Тогда
мы получим некоторую смесь
\begin{equation*}
p_i(x,\theta)=(1-\theta) f_i(x)+\theta g_i(x)\,,
\end{equation*}
где
$$
f(x)=\sum\limits_{j\neq i}p_j\psi_j;\quad \sum\limits_{j\neq i}p_j=1;\  g_i(x)=\psi_i\,.
$$
В таких обозначениях условия вида~\eqref{CondPsiNorm}
или~\eqref{CondPsiGamma} уже будут выполнены. А~значит,
последовательно перебирая все номера~$i$, для которых указанные
условия будут выполнены, можно проверить, является ли смесь $k$- или
$(k+1)$-ком\-по\-нент\-ной. Если хотя бы в одном случае получится, что
верна нулевая гипотеза, значит, смесь является $k$-ком\-по\-нентной.

Итак, из вышесказанного получаем следующую теорему.

\medskip

\noindent
\textbf{Теорема 1.}
\textit{Пусть моментные характеристики $\Psi_s$, $s=2,3,4$, из
соотношения}~\eqref{Psis} \textit{конечны. Тогда для проверки гипотез о числе
компонент идентифицируемой смеси вероятностных распределений может
быть использован критерий, основанный на статистике
\begin{equation*}
L_n^{(1)}=\fr{1}{\sqrt{n}}\sum\limits_{i=1}^{n}\left(\fr{g(X_i)}{f(X_i)}-1\right)
\end{equation*}
и обладающий следующими свойствами:}
\begin{enumerate}
\item \textit{При справедливости нулевой гипотезы эта статистика имеет
нормальное распределение с па\-ра\-мет\-ра\-ми~$0$ и $\Psi_2-1$:}
\begin{equation*}
\mathfrak{L}(L_n^{(1)}\mid H_0)\rightarrow N(0,\Psi_2-1)\,.
\end{equation*}
\item \textit{При справедливости альтернативы эта статистика имеет
нормальное распределение с па\-ра\-мет\-ра\-ми $t\left(\Psi_2-1\right)$ и
$\Psi_2-1$:}
\begin{equation*}
\mathfrak{L}(L_n^{(1)}\mid H_{n,1})\rightarrow
N(t\left(\Psi_2-1\right),\Psi_2-1)\,.
\end{equation*}
\item \textit{Данный критерий является асимптотически наиболее мощным критерием с предельной мощностью
\emph{(}для заданного уровня $\alpha\in(0,1)$\emph{)} вида}
\begin{equation*}
\beta^*(t)=\Phi(t\sqrt{\Psi_2-1}-u_\alpha)\,.
\end{equation*}
\item \textit{Потеря мощности этого критерия равна}
\begin{multline*}
r(t)=\fr{t^3}{8\sqrt{\Psi_2-1}}\varphi\left(u_\alpha-t\sqrt{\Psi_2-1}\right)
\left(\vphantom{\fr{\left(\Psi_3-1\right)^2}{\Psi_2-1}-1}\Psi_4+{}\right.\\
\left.{}+2\Psi_3-\Psi^2_2-\Psi_2-\fr{\left(\Psi_3-1\right)^2}{\Psi_2-1}-1\right)\,.
\end{multline*}
\item \textit{Асимптотический дефект этого критерия равен}
\begin{multline*}
d=\fr{t^2}{4\left(\Psi_2-1\right)}
\left(\vphantom{\fr{\left(\Psi_3-1\right)^2}{\Psi_2-1}}
\Psi_4+2\Psi_3-\Psi^2_2-\Psi_2-{}\right.\\
\left.{}-\fr{\left(\Psi_3-1\right)^2}{\Psi_2-1}-1\right)\,.
\end{multline*}
\end{enumerate}


%\medskip

\noindent
\textbf{Замечание 2.}
В~случае рассмотрения конечной смеси нормальных
законов для конечности моментных характеристик~$\Psi_s$ достаточно
потребовать выполнение условий~\eqref{CondPsiNorm}, а в случае
рассмотрения конечной смеси гам\-ма-рас\-пре\-де\-ле\-ний~---
условий~\eqref{CondPsiGamma}.

\medskip

\noindent
\textbf{Замечание 3.}
Отметим, что в теореме~1 подразумевается
выполнение условия $\Psi_2>1$. Его справедливость в терминах
фишеровской информации была доказана при проверке условия~$(C)$
леммы~1.


\section{Примеры конкретных смесей вероятностных распределений}

В этом разделе рассмотрим частные случаи смесей, для которых в явном
виде можно выписать выражения для интегралов~\eqref{Psis}. Всюду
далее предполагается, что рассматриваются идентифицируемые смеси
(для нормального и гам\-ма-рас\-пре\-де\-ле\-ний это условие конечности чис\-ла
компонент смеси, что следует из результатов
работы~\cite{Teicher1963}). Для равномерного распределения
воспользуемся следующим утверждением.

\medskip

\noindent
\textbf{Утверждение 1.} \textit{Пусть $A(M)=\bigcup\limits_{i\in M}\left[a_i,b_i\right]$, 
где $M$~--- некоторое подмножество номеров. Обозначим семейство конечных смесей
равномерных распределений через
$$
H=\left\{F(x)=\sum\limits_{i=1}^k p_iF_i(x),\,
\sum\limits_{i=1}^k p_i=1,\,F_i\in\Fd\right\}\,,
$$ 
где
$\Fd=\{F(x,a_i,b_i),\,x\hm\in\r$, $-\infty<a_i<b_i<\infty$, $i\hm\in\N\}$~--- 
некоторое множество функций распределения равномерных законов
(возможно, конечное). Семейство~$H$ идентифицируемо тогда и только
тогда, когда
\begin{equation}
\label{Uident} A(M_1)\diagdown A(M_2)\neq\varnothing
\end{equation}
для всех возможных различных $M_1$ и $M_2$, $M_i\subseteq\N$.}

%\pagebreak
%\medskip

\noindent
Д\,о\,к\,а\,з\,а\,т\,е\,л\,ь\,с\,т\,в\,о\,.\ \ 
Для доказательства утверж\-де\-ния нужно показать, основываясь на
результатах работы~\cite{Yakowitz1968}, что условие~\eqref{Uident}
является необходимым и достаточным для линейной независимости
множества~$\Fd$ над полем действительных чисел.

Будем пользоваться тем фактом, что функции распределения линейно
зависимы тогда и только тогда, когда линейно зависимы их плотности. 
В~этом просто убедиться, дифференцируя
\begin{equation}
\label{sumFi}
\sum\limits_{i=1}^N \alpha_iF_i=0\,,\enskip\forall x\in\r\,,
\end{equation}
где $\alpha_i$~--- действительные числа, одновременно не
равные нулю. Для плотностей имеем соответственно тождество

\begin{equation}
\label{sumfi}
\sum\limits_{i=1}^N \alpha_if_i=0\,.
\end{equation}
Очевидно, что интегрирование тождества~\eqref{sumfi} приводит к
соотношению~\eqref{sumFi}.

\smallskip

\textit{Необходимость.} Покажем, что если $\Fd$ является линейно
независимым множеством над~$\r$, то условие~\eqref{Uident}
выполнено. Предположим, что это не так. Тогда
\begin{equation*}
\sum\limits_{i=1}^N \alpha_iF_i=0
\end{equation*}
только для тривиального набора чисел $\{\alpha_i\}$ и
условие~\eqref{Uident} не выполняется, т.\,е.\ для некоторых параметров
$a<b<c<d$ (без ограничения общности рассмотрим случай совпадения
числа элементов в $M_1$ и $M_2$; сказанное останется верным и для
разного числа элементов в сумме в равенстве~\eqref{sumFi})
$$
[a,b]\cup[b,d]=[a,c]\cup[c,d]\,.
$$
Но
\begin{multline*}
\alpha_1\fr{1}{b-a}\Id([a,b])+\alpha_2\fr{1}{d-b}\Id([b,d])+{}\\
{}+
\alpha_3\fr{1}{c-a}\Id([a,c])+\alpha_4\fr{1}{d-c}\Id([c,d])=0
\end{multline*}
не только для тривиального набора коэффициентов. Так, в качестве
такого набора можно рассмотреть $(b-a,d-b,-c+a,-d+c)$. Получаем
противоречие с линейной независимостью множества~$\Fd$.

\smallskip

\textit{Достаточность.} Теперь предположим, что условие~\eqref{Uident}
выполнено, а $\Fd$ является линейно зависимым множеством над~$\r$.
Справедливость условия~\eqref{Uident} означает, что существует
отрезок $[c_k,d_k]$ такой, что
$$
[c_k,d_k]\subseteq[a_k, b_k]:[c_k,d_k]\nsubseteq\left( A(M_1)\cap
A(M_2)\right)\,.
$$
Теперь, рассматривая линейно зависимые плот\-ности в~\eqref{sumfi} при
$x\in[c_k,d_k]$, получаем
%\noindent
$$
\alpha_k C\equiv0\,,\enskip C>0\,,
$$
поэтому $\alpha_k=0$, причем равенство нулю коэффициента справедливо
для всех $x\hm\in\r$, так как коэффициенты в сумме в
тождестве~\eqref{sumfi} от~$x$ не зависят и тождество~\eqref{sumfi}
должно выполняться сразу для всех~$x$. Таким образом, коэффициент
перед плотностью равномерного на сегменте $[a_k, b_k]$ распределения
в~\eqref{sumfi} равен нулю, при этом условие~\eqref{Uident}
выполняется, а $\Fd$~--- линейно зависимое множество. Далее, поступая
описанным выше способом, получаем, что все коэффициенты при
плотностях равны нулю. Получаем противоречие с предположением о
линейной зависимости. Значит, множество~$\Fd$ над~$\r$ не может быть
линейно зависимым при выполнении условия~\eqref{Uident}.\hfill$\square$

\smallskip

\textbf{Равномерное распределение.} В~данном случае

\noindent
\begin{gather*}
f(x)=\sum\limits_{i=1}^{k}p_i\fr{1}{b_i-a_i}\Id([a_i,b_i])\,;\quad 
\sum\limits_{i=1}^{k}p_i=1;\,\\
g(x)=\fr{1}{b_{k+1}-a_{k+1}}\Id([a_{k+1},b_{k+1}])\,,
\end{gather*}
где символом $\Id(\cdot)$ обозначен индикатор соответствующего
множества. Предположим, что
%\noindent
$$
[a_{k+1},b_{k+1}]\bigcap\left(\bigcup\limits_{i=1}^k[a_i,b_i]\right)\neq\emptyset\,.
$$
Тогда моменты~\eqref{Psis}, которые могут быть использованы для
нахождения величины $r(t)$ из равенства~\eqref{r(t)Ln}, равны
(отметим, что в данном случае в силу определения равномерного
распределения интегрирование ведется по ограниченному множеству, а
подынтегральная функция не имеет особенностей)

\noindent
\begin{multline*}
\Psi_s=\int\limits_{-\infty}^\infty\fr{g^s(x)}{f^s(x)}f(x)\,dx=
\fr{1}{(b_{k+1}-a_{k+1})^s}\times\\
{}\times
\sum\limits_{j=1}^k\fr{1}{b_j-a_j}
\int\limits_{a_j}^{b_j}\Id_{k+1}([a_j,b_j])
\bigg(\fr{1}{b_j-a_j}+{}\\
{}+\sum_{\substack{i=1,\ldots,k\\i\neq
j}}\fr{\Id_i([a_j,b_j])}{b_i-a_i}\bigg)^{-s}\,dx=
\fr{1}{(b_{k+1}-a_{k+1})^s}\times{}\\
\!\!\!{}\times
\sum\limits_{j=1}^k\fr{1}{b_j-a_j}
\int\limits_{A_j}\!\!\bigg(\fr{1}{b_j-a_j}+\!\!\sum_{\substack{i=1,\ldots,k\\i\neq
j}}\!\!\!\fr{\Id_i([a_j,b_j])}{b_i-a_i}\bigg)^{-s}\!\!dx={}\hspace*{-3.11491pt}\\
{}=\fr{1}{(b_{k+1}-a_{k+1})^s}\sum\limits_{j=1}^k\fr{1}{b_j-a_j}
\sum\limits_{t=1}^{m^j-1}(c_{t+1}^j-c_t^j)\times{}
\\
\!\!\!\!{}\times
\bigg(\fr{1}{b_j-a_j}+\!\!\sum_{\substack{i=1,\ldots,k\\i\neq
j}}\fr{\Id_i([c_t^j,c_{t+1}^j])}{b_i-a_i}\bigg)^{-s},\
s=2,3,4,
\end{multline*}

\end{multicols}

%\hrule

\noindent
где
\begin{equation*}
\Id_j([c_i,c_{i+1}])=
\begin{cases}
1,&\mbox{\ если }a_j\leqslant c_i\leqslant c_{i+1}\leqslant b_j\,;\\
0,&\ [c_i,c_{i+1}]\notin[a_j,b_j]\,;
\end{cases}
\end{equation*}

\noindent
\begin{equation*}
A_j=[a_j,b_j]\cap[a_{k+1},b_{k+1}]\,;
\end{equation*}
набор $c_1^j,\ldots,c_{m^j}^j$ составляет разбиение множества~$A_j$
на непересекающиеся (за исключением границ) сегменты
$[c_t^j,c_{t+1}^j])$. При этом считаем, что если $A_j=\emptyset$, то
$m^j\hm=1$ и все $c_i^j\equiv0$.

Теперь выпишем характеристическую функцию из соотношения~\eqref{Xf}
для данного случая:
\begin{multline*}
\phi_{L_n^{(1)}}(z)=\left(e^{-i{z}/{\sqrt{n}}}\phi_\xi\left(\fr
{z}{\sqrt n}\right)\right)^n=e^{-iz\sqrt{n}}\left(\phi_\xi\left(\fr {z}{\sqrt n}\right)\right)^n={}\\
{}= 
e^{-iz\sqrt{n}}\bigg(\int\limits_{-\infty}^\infty\exp{\bigg\{\fr
{iz}{\sqrt n}\,\fr{g(x)}{f(x)}\bigg\}}((1-\theta)
f(x)+\theta  g(x))\,dx\bigg)^{n}={}\\
{}=e^{-iz\sqrt{n}}\bigg((1-\theta)\sum\limits_{j=1}^k\fr{1}{b_j-a_j}
\int\limits_{a_j}^{b_j}\exp{\left\{\fr {iz}{\sqrt n}
\fr{{\Id_{k+1}([a_j,b_j])}/({b_{k+1}-a_{k+1}})}
{(b_j-a_j)^{-1}+\sum_{\substack{i=1,\ldots,k\\i\neq
j}}{\Id_i([a_j,b_j])}/(b_i-a_i)}\right\}}\,dx+{}\\
{}+\theta\fr{1}{b_{k+1}-a_{k+1}}
\int\limits_{a_{k+1}}^{b_{k+1}}\exp\left\{\fr {iz}{\sqrt n}
\left({(b_{k+1}-
a_{k+1})\sum\limits_{i=1}^k\fr{\Id_i([a_{k+1},b_{k+1}])}
{b_i-a_i}}\right)^{-1}\right\}\,dx\bigg)^{n}={}\\
{}=e^{-iz\sqrt{n}}\bigg((1-\theta)\sum\limits_{j=1}^k\fr{1}{b_j-a_j}
\sum\limits_{t=1}^{m^j-1}(c_{t+1}^j-c_t^j)\times{}\\
{}\times\exp{\bigg\{\fr {iz}{\sqrt n}\fr{1}{b_{k+1}-a_{k+1}}
\bigg(\fr{1}{b_j-a_j}+\sum_{\substack{i=1,\ldots,k\\i\neq
j}}\frac{\Id_i([a_j,b_j])}{b_i-a_i}\bigg)^{-1}\bigg\}}+{}\\
{}+\theta\fr{1}{b_{k+1}-a_{k+1}}
\sum\limits_{t=1}^{l-1}(c_{t+1}^{k+1}-c_t^{k+1})\exp{\bigg\{\fr
{iz}{\sqrt n}
\bigg({(b_{k+1}-a_{k+1})\sum\limits_{i=1}^k\fr{\Id_i([a_{k+1},b_{k+1}])}
{b_i-a_i}}\bigg)^{-1}\bigg\}}\bigg)^{n}\,.
\end{multline*}

Аналогично получается выражение для преобразования Лапласа:
\begin{multline*}
\Phi_{L_n^{(1)}}
(s)=e^{s\sqrt{n}}\bigg((1-\theta)\sum\limits_{j=1}^k\fr{1}{b_j-a_j}
\sum\limits_{t=1}^{m^j-1}(c_{t+1}^j-c_t^j)\times{}\\
{}\times\exp{\bigg\{-\fr{s}{\sqrt n}\,\fr{1}{b_{k+1}-a_{k+1}}
\bigg(\fr{1}{b_j-a_j}+\sum_{\substack{i=1,\ldots,k\\i\neq
j}}\fr{\Id_i([a_j,b_j])}{b_i-a_i}\bigg)^{-1}\bigg\}}+{}\\
{}+\theta\fr{1}{b_{k+1}-a_{k+1}}
\sum\limits_{t=1}^{l-1}(c_{t+1}^{k+1}-c_t^{k+1})\exp{\bigg\{-\fr
{s}{\sqrt n}
\,\bigg({(b_{k+1}-a_{k+1})\sum\limits_{i=1}^k\fr{\Id_i([a_{k+1},b_{k+1}])}
{b_i-a_i}}\bigg)^{-1}\bigg\}}\bigg)^{n}\,,\enskip
s>0\notag\,.
\end{multline*}

\textbf{Нормальное распределение.} Пусть выполнены
условия~\eqref{CondPsiNorm}. Проверяется гипотеза о том, что
плотность каждого наблюдения является нормальным законом, против
альтернативы, что плотность представляет собой смесь двух нормальных
законов. В~данном случае

\noindent
\begin{align*}
f(x)&=\fr{1}{\sigma_1\sqrt{2\pi}}\exp{\left\{-\fr{(x-a_1)^2}{2\sigma_1^2}\right\}}\,;\\
g(x)&=\fr{1}{\sigma_2\sqrt{2\pi}}\exp{\left\{-\fr{(x-a_2)^2}{2\sigma_2^2}\right\}}\,.
\end{align*}
Моменты, определенные в соотношении~\eqref{Psis}, которые входят в
величину $r(t)$ (см.~\eqref{r(t)Ln}), равны
\begin{multline*}
\Psi_s=\fr{\sigma^{s-1}_1}{\sigma_2^s\sqrt{2\pi}}\int\limits_{-\infty}^\infty
\exp{\left\{-s\fr{(x-a_2)^2}{2\sigma_2^2}+(s-1)\fr{(x-a_1)^2}{2\sigma_1^2}\right\}}\,dx={}\\[3pt]
{}=\fr{\sigma^{s-1}_1}{\sigma_2^s\sqrt{2\pi}}\int\limits_{-\infty}^\infty
\exp{\left\{-x^2\left(\fr{s}{2\sigma_2^2}-\fr{s-1}{2\sigma_1^2}\right)+x\left(
\fr{sa_2}{\sigma_2^2}-\fr{(s-1)a_1}{\sigma_1^2}
\right)+\fr{(s-1)a_1^2}{2\sigma_1^2}-\fr{s a_2^2}{2\sigma_2^2}\right\}}\,dx={}\\[3pt]
{}=\fr{\sigma^{s-1}_1}{\sigma_2^s\sqrt{2\pi}}
\sqrt{\fr\pi{{s}/({2\sigma_2^2})-(s-1)/(2\sigma_1^2)}}\times{}\\[3pt]
{}\times\exp{\left\{\left(\fr{s a_2}{\sigma_2^2}-
\fr{(s-1)a_1}{\sigma_1^2}
\right)^2\biggl{/}\left(4\left(\fr{s}{2\sigma_2^2}-\fr{s-1}{2\sigma_1^2}\right)\right)
+\fr{(s-1)a_1^2}{2\sigma_1^2}-\fr{s a_2^2}{2\sigma_2^2}\right\}}={}\\[3pt]
{}=\fr{\sigma_1^s}{\sigma_2^{s-1}\sqrt{s\sigma_1^2-(s-1)\sigma_2^2}}\times{}\\[3pt]
{}\times\exp{\left\{\left(\fr{s a_2}{\sigma_2^2}-\fr{(s-1)a_1}{\sigma_1^2}
\right)^2\biggl{/}\left(2\left(\fr{s}{\sigma_2^2}-\fr{s-1}{\sigma_1^2}\right)\right)
+\fr{(s-1)a_1^2}{2\sigma_1^2}-\fr{s a_2^2}{2\sigma_2^2}\right\}}={}\\[3pt]
{}=\fr{\sigma_1^s}{\sigma_2^{s-1}\sqrt{s\sigma_1^2-(s-1)\sigma_2^2}}\times{}\\[3pt]
{}\times\exp{\left\{\left(s a_2 \fr{\sigma_1}{\sigma_2}-(s-1)a_1
\fr{\sigma_2}{\sigma_1}\right)^2
\biggl{/}\left(2\left(s\sigma_1^2-(s-1)\sigma_2^2\right)\right)
+\fr{(s-1)a_1^2}{2\sigma_1^2}-\fr{s a_2^2}{2\sigma_2^2}\right\}}\,,\enskip
s=2,3,4\,.
\end{multline*}

Отметим, что при нахождении значения интеграла использовалось
следующее соотношение:
$$
\int\limits_{-\infty}^{+\infty}
\exp{\left\{-ax^2+bx+c\right\}}\,dx=\sqrt{\fr\pi
a}\exp{\left\{\fr {b^2}{4a}+c\right\}}\,.
$$

%\smallskip

\textbf{Гамма-распределение.} Пусть выполнены
условия~\eqref{CondPsiGamma}. Проверяем гипотезу о том, что
плот\-ность каждого наблюдения определяется гам\-ма-рас\-пре\-де\-ле\-ни\-ем,
против альтернативы, что плот\-ность представляет собой смесь двух
гамма-распределений. В~данном случае
\begin{align*}
f(x)&=\fr{\alpha_1^{\beta_1}}{\Gamma(\beta_1)}x^{\beta_1-1}e^{-\alpha_1x}\,;\\[6pt]
g(x)&=\fr{\alpha_2^{\beta_2}}{\Gamma(\beta_2)}x^{\beta_2-1}e^{-\alpha_2x}\,,\enskip x\geqslant0\,.
\end{align*}
Моменты, определенные в соотношении~\eqref{Psis}, которые входят в
величину~$r(t)$ (см.~\eqref{r(t)Ln}), равны:
\begin{multline*}
\Psi_s=\fr{\Gamma^{s-1}(\beta_1)\alpha_2^{s\beta_2}}{\Gamma^s(\beta_2)\alpha_1^{(s-1)\beta_1}}
\int\limits_0^\infty x^{s\beta_2-(s-1)\beta_1-1}e^{-x(s\alpha_2-(s-1)\alpha_1)}\,dx={}\\[3pt]
{}=\fr{\alpha_2^{s\beta_2}}{\alpha_1^{(s-1)\beta_1}}\,
\fr{1}{(s\alpha_2-(s-1)\alpha_1)^{s\beta_2-(s-1)\beta_1}}
\fr{\Gamma^{s-1}(\beta_1)\Gamma(s\beta_2-(s-1)\beta_1)}{\Gamma^s(\beta_2)}\,,\enskip
s=2,3,4\,.
\end{multline*}

\pagebreak


%\hrule

\begin{multicols}{2}
%\vspace*{-12pt}

{\small\frenchspacing
{%\baselineskip=10.8pt
\addcontentsline{toc}{section}{Литература}
\begin{thebibliography}{99}

\bibitem{Korolev2007}
\Au{Королев~В.\,Ю.} Ве\-ро\-ят\-ност\-но-ста\-ти\-сти\-че\-ский
анализ хаотических процессов с помощью смешанных гауссовских моделей.
Декомпозиция волатильности финансовых индексов и турбулентной
плазмы.~--- М.: ИПИ РАН, 2007. 363~c.

\bibitem{Korolev2010}\Au{Королев~В.\,Ю.} Ве\-ро\-ят\-ност\-но-ста\-ти\-сти\-че\-ские
методы декомпозиции волатильности хаотических процессов.~--- М.: МГУ,
2011. 510~c.

\bibitem{Akaike1973}\Au{Akaike~H.} Information theory and an extension
of the maximum likelihood principle~// 2nd Symposium (International) 
on Information Theory~/ Eds. B.\,N.~Petrov, F.~Csake.~--- Budapest, 1973. P.~267--281.

\bibitem{Schwartz1978}\Au{Schwartz~G.} Estimating the
dimension of a model~// The Annals of Statistics, 1978. Vol.~6.
P.~461--464.

\bibitem{Lo2001}\Au{Lo~Y., Mendell~N.\,R., Rubin~D.\,B.}
Testing the number of components in a normal mixture~// Biometrika,
2001. Vol.~88. No.\,3. P.~767--778.

\bibitem{Lo2005}\Au{Lo~Y.}  Likelihood ratio tests of the number of
components in a normal mixture with unequal variances~// Statistics
and Probability Lett., 2005. Vol.~71. P.~225--235.

\bibitem{Vuong1989}\Au{Vuong Q.\,H.} Likelihood ratio tests for model
selection and non-nested hypotheses~//~Econometrica, 1989. Vol.~57. Iss.~2. P.~307--333.

\bibitem{Bening2000}\Au{Bening~V.~E.} Asymptotic theory of testing
statistical hypothesis: Efficient statistics, optimality, power loss
and deficiency.~--- Untrecht: VSP, 2000. 277~p.

\bibitem{Hajek1962}\Au{H$\acute{\mbox{a}}$jek J.} Asymptotically most powerful
rank-order tests~// Ann. Math. Statist., 1962. Vol.~33. P.~1124--1147.

\bibitem{KF1976}\Au{Колмогоров~А.\,Н., Фомин~С.\,В.}
Элементы теории функций и функционального анализа.~--- 4-е изд.~---
М.:~Наука, 1976.

\bibitem{Feller2010}\Au{Феллер В.} Введение в теорию
вероятностей и ее приложения. Т.~2.~--- М.: Либроком, 2010. 766~с.

\bibitem{Teicher1963} \Au{Teicher~H.} Identifiability of finite mixtures~// The
Annals of Statistics, 1963. Vol.~34. No.\,4. P.~1265--1269.

\label{end\stat}

\bibitem{Yakowitz1968} \Au{Yakowitz~S.\,J., Spragins~J.\,D.} On the
identifiability of finite mixtures~// The Annals of Statistics,
1968. Vol.~39. No.\,1. P.~209--214.
 \end{thebibliography}
}
}


\end{multicols}           %1
\def\stat{kondranin+ushakov}

\def\tit{СИСТЕМА ОБСЛУЖИВАНИЯ С~ОТНОСИТЕЛЬНЫМ ПРИОРИТЕТОМ  И~ПРОФИЛАКТИКАМИ ПРИБОРА$^*$}

\def\titkol{Система обслуживания с~относительным приоритетом  и~профилактиками прибора}

\def\aut{Е.\,С.~Кондранин$^1$,  В.\,Г.~Ушаков$^2$}

\def\autkol{Е.\,С.~Кондранин,  В.\,Г.~Ушаков}

\titel{\tit}{\aut}{\autkol}{\titkol}

\index{Кондранин Е.\,С.}
\index{Ушаков В.\,Г.}
\index{Kondranin E.\,S.}
\index{Ushakov V.\,G.}




{\renewcommand{\thefootnote}{\fnsymbol{footnote}} \footnotetext[1]
{Работа выполнена при финансовой поддержке РФФИ (проект 18-07-00678).}}


\renewcommand{\thefootnote}{\arabic{footnote}}
\footnotetext[1]{Факультет вычислительной математики и~кибернетики Московского государственного 
университета им.\ М.\,В.~Ломоносова, \mbox{ekondranin@yandex.ru}}
\footnotetext[2]{Факультет вычислительной математики и~кибернетики
Московского государственного университета им.\ М.\,В.~Ломоносова;
Институт проб\-лем информатики Федерального исследовательского
центра <<Информатика и~управ\-ле\-ние>> Российской академии наук,
\mbox{vgushakov@mail.ru}}

\vspace*{-10pt}




\Abst{Изучена одноканальная система
массового обслуживания с~двумя типами требований, бесконечным
числом мест для ожидания, гиперэкспоненциальным входящим потоком 
и~профилактиками обслуживающего прибора при освобождении системы.
Тип  требования определяется случайно с~заданными вероятностями 
в~момент его поступления в~систему обслуживания. Требования первого
типа имеют относительный приоритет перед требованиями второго
типа. Найдено нестационарное совместное распределение числа
требований каждого типа в~системе. Профилактики прибора
заключаются в~том, что в~момент освобождения системы от требований
прибор на случайное время с~заданным распределением становится
недоступным для обслуживания. Если за время профилактики поступает
хотя бы одно требование, то начинается нормальное функционирование
системы. Если требования не поступают, то прибор отправляется на
новую профилактику. Такие системы хорошо описывают
функционирование большого числа реальных вычислительных и~информационных систем.}

\KW{гиперэкспоненциальный поток; профилактики
обслуживающего прибора; одноканальная система; относительный
приоритет; длина очереди}

\DOI{10.14357/19922264180405}
  
%\vspace*{4pt}


\vskip 10pt plus 9pt minus 6pt

\thispagestyle{headings}

\begin{multicols}{2}

\label{st\stat}

\section{Введение}

В классической системе массового обслуживания ожидание требований
в очереди связано только с~занятостью обслуживающего прибора. В~то
же время в~реальных системах сам  прибор может пребывать как 
в~активном, так и~в~неактивном состоянии. Такое неактивное
состояние прибора (в~литературе на английском языке используется
термин vacation, а~на русском~--- профилактика или прогулка) может
быть связано со многими причинами. В~част\-ности, сис\-те\-мы
обслуживания с~профилактиками прибора хорошо описывают
функционирование  реальных вычислительных и~информационных систем,
в которых наряду с~основными требованиями имеются второстепенные.
Второстепенные требования всегда присутствуют в~сис\-те\-ме, а~их
обслуживание может проводиться только тогда, когда нет основных,
т.\,е.\ в~фоновом режиме.

С точки зрения самого процесса профилактики прибора существует
несколько ее разновидностей. Во-пер\-вых, могут быть разными
правила, задающие условия начала профилактики: прибор может брать
перерыв только при  полном исчерпании требований в~очереди
(exhaustive service) либо при наличии определенного их числа
(nonexhaustive service). Во-вто\-рых, могут быть разными правила
возвращения прибора в~работу. С~этой точки зрения различают случаи
однократного (single vacation) и~многократного (multiple vacation)
перерыва в~работе. В~первом случае ушедший на профилактику прибор
после ее окончания находится в~рабочем состоянии независимо от
наличия требований в~системе. Во втором случае прибор, не
обнаружив новых требований в~очереди, уходит на новую
профилактику.


В работах~[1--4] можно найти обзор известных результатов, большое
число постановок задач, описание различных приложений и~обширную
библиографию по анализу систем с~профилактиками обслуживающего
прибора.


В настоящей работе исследуется совместное распределение длин
очередей в~нестационарном режиме в~однолинейной системе 
с~ожиданием, гиперэкспоненциальным входящим потоком, двумя типами
требований и~относительным приоритетом. Аналогичная неприоритетная
система обслуживания исследована в~[5].

\vspace*{-6pt}

\section{Описание модели}

Рассматривается однолинейная система массового обслуживания 
с~двумя приоритетными классами требований. Входящий поток~---
гиперэкспоненциальный с~функцией распределения интервалов между
поступлениями требований вида:
\begin{multline*}
A(t)=\sum\limits_{i=1}^kc_i\left(1-e^{-a_it}\right),\enskip t>0,\enskip
a_i>0,\enskip c_i>0,\\
a_i\ne a_j\,,\enskip i\ne j\,,\enskip  \sum\limits_{i=1}^k c_i=1\,.
\end{multline*}

Каждое поступившее требование направляется в~первый класс 
с~вероятностью~$p$ и~во второй класс с~вероятностью $1\hm-p$
независимо от остальных требований. Требования первого класса
обладают относительным приоритетом перед требованиями второго
класса. Длительности обслуживания требований $i$-го приоритетного
класса~--- независимые в~совокупности и~не зависящие от входящего
потока случайные величины с~функцией распределения~$B_i(x)$,
$i\hm=1,2.$
 Если в~некоторый момент времени система освободилась от требований, 
 то обслуживающий прибор
 отправляется на профилактику, которая длится случайное время с~функцией 
 распределения~$C(x).$
 Не ограничивая общности, будем считать, что $B_i(x)\hm<1$
 и~$C(x)\hm<1$  для любого~$x$ 
 и~существуют плотности
 распределения~$b_i(x)$ и~$c(x).$
  Обозначим:
$$
 \beta_i(s)=\int\limits_0^{\infty}e^{-sx}b_i(x)\,dx\,;\enskip 
  \gamma(s)=\int\limits_0^{\infty}e^{-sx}c(x)\,dx\,.
$$
Пока прибор находится на профилактике, он не доступен для
обслуживания. Если за время профилактики поступают требования,
после ее завершения начинается их обслуживание. Если ни одно
требование не поступает, то прибор отправляется на новую
профилактику. Длительности различных профилактик являются
независимыми случайными величинами 
и~не зависят от входящего потока и~времен обслуживания.

\section{Вспомогательные результаты}

  Рассмотрим многочлен по $\mu$ степени $k$ вида:
\begin{multline}
\label{1}
\prod\limits_{i=1}^k\left(\mu+a_i\right)-{}\\
{}-
\left(pz_1+(1-p)z_2\right)\sum\limits_{j=1}^kc_ja_j\prod\limits_{i\ne
j}\left(\mu+a_i\right)\,.
\end{multline}
Занумеруем его корни $\mu_1(z_1,z_2),\ldots,\mu_k(z_1,z_2)$ таким образом,
чтобы они были непрерывными функциями и~$\mu_1(1,1)\hm=0.$ Тогда
$\mathrm{Re}\, \mu_j\left(z_1,z_2\right)\hm<0$, $|z_1|\hm<1$, 
$|z_2|\hm<1,$ $\mu_i(z_1,z_2)\hm\ne \mu_j(z_1,z_2),$ $ i\hm\ne j$,
$j\hm=1,\ldots,k.$ Обозначим:
$$
\alpha_m(z_1,z_2)=\prod\limits_{j\ne m}\left(\mu_m\left(z_1,z_2\right)-
\mu_j\left(z_1,z_2\right)\right)\,.
$$
Справедливы следующие леммы.

\smallskip

\noindent
\textbf{Лемма~1.}\
\textit{Для любого $l=1,\ldots,\:k$ система уравнений}
$$
z_j=\beta_j(s-\mu_l(z_1,z_2)),\ \ j=1,2,
$$
\textit{имеет единственное решение $z_i=z_{il}(s)$ такое, 
что $|z_{il}(s)|\hm<1$ при $l\hm=2,\ldots, k,$ $\mathrm{Re}\, s\hm\geqslant 0,$ 
а~$z_{i1}(0)\hm=1$, $|z_{i1}(s)|\hm<1$ при} $\mathrm{Re}\, s\hm> 0$, $i\hm=1,2.$

\smallskip

\noindent
\textbf{Лемма~2.}\
\textit{При каждом $l\hm=1,\ldots,k$ уравнение}
$$
z_1=\beta_1\left(s-\mu_l(z_1,z_2)\right)
$$
\textit{имеет единственное решение $z_1\hm=z_{1l}(z_2,s),$ 
аналитическое в~области $\mathrm{Re}\, s\hm>0$, $|z_2|\hm<1.$
}

\smallskip

Положим
$$
\lambda_l(s)=\mu_l\left(z_{1l}(s),z_{2l}(s)\right)\,.
$$




\section{Распределение длины очереди}

  Гиперэкспоненциальный поток можно рас\-смат\-ри\-вать как
пуассоновский поток со случайной интен\-сив\-ностью~$a,$ которая
принимает $k$ различных значений $a_1,\ldots,a_k$  с~вероятностями
$c_1,\ldots,c_k.$ Текущее значение~$a$ разыгрывается в~момент
поступления требования и~не меняется между двумя соседними
поступлениями. Введем случайный процесс~$j(t)$ такой, что если
$a\hm=a_j$ в~момент времени $t,$ то $j(t)\hm=j.$

Целью работы является нахождение распределения случайного процесса
$\left(L_1(t),L_2(t)\right),$ где $L_i(t)$~--- число требований из
$i$-го приоритетного класса, находящихся в~системе в~момент
времени~$t.$

При сделанных предположениях относительно параметров изучаемой
системы обслуживания\linebreak процесс $\left(L_1(t),L_2(t)\right)$ не
является, вообще говоря, марковским. Пусть $i(t)=i$, $i\hm=1,2,$ если
в~момент времени~$t$ обслуживается требование из $i$-го
приоритетного класса, и~$i(t)\hm=0,$ если в~момент времени~$t$ прибор
находится на профилактике. Случайный процесс~$x(t)$ определим
следующим образом. Если $i(t)\hm\ne 0,$ то $x(t)$ есть
время, прошедшее с~начала обслуживания требования, находящегося на
приборе, до момента~$t.$ Если $i(t)\hm=0,$ то $x(t)$ есть время,
прошедшее с~начала профилактики прибора до момента~$t.$ Случайный
процесс $\left(L_1(t),L_2(t),i(t),j(t),x(t)\right)$ является
однородным марковским процессом. Положим
\begin{multline*}
P_{ij}(n_1,n_2,x,t)=\fr{\partial}{\partial x}
\mathbf{P}\left(L_1(t)=n_1,L_2(t)=n_2,\right.\\
\left. i(t)=i,j(t)=j,x(t)<x
\vphantom{L_1}\right)\,,\enskip 
 x\geqslant 0,\\ 
 j=1,\ldots,k,\enskip i=0,1,2;
\end{multline*}
\begin{gather*}
\eta_i(x)=\fr{b_i(x)}{1-B_i(x)},\ i=1,2;\enskip 
\eta_0(x)=\fr{c(x)}{1-C(x)}\,;\\
\delta_{i,j}=\begin{cases}
1,&\ i=j;\\ 
0,&\ i\ne j\,.
\end{cases}
\end{gather*}
Функции $P_{ij}(n_1,n_2,x,t)$  удовлетворяют при $x\hm>0$
системам дифференциальных уравнений:
\begin{multline}
\label{3}
\fr{\partial P_{ij}(n_1,n_2,x,t)}{\partial t}+\fr{\partial
P_{ij}(n_1,n_2,x,t)}{\partial
x}={}\\
{}=-(a_j+\eta_i(x))P_{ij}(n_1,n_2,x,t)+ {}\\
{}+
c_j\sum\limits_{l=1}^ka_l\left(p\:P_{il}(n_1-1,n_2,x,t)+{}\right.\\
\left.{}+
(1-p)P_{il}(n_1,n_2-1,x,t)\right)
\end{multline}
и краевым условиям при $x\hm=0$:
\begin{multline}
\label{5}
P_{0j}(n_1,n_2,0,t)=0,\ n_1+n_2>0;\\
P_{0j}(0,0,0,t)=\int\limits_0^{\infty}P_{0j}(0,0,x,t)\eta_0(x)\,dx+{}\\
 {}+\int\limits_0^{\infty}P_{1j}(1,0,x,t)\eta_1(x)dx+{}\\
 {}+
\int\limits_0^{\infty}P_{2j}(0,1,x,t)\eta_2(x)\,dx\,;
\end{multline}

\vspace*{-12pt}

\noindent
\begin{multline}
\label{6}
P_{1j}(n_1,n_2,0,t)+P_{2j}(n_1,n_2,0,t)={}\\
{}=\int\limits_0^{\infty}P_{1j}(n_1+1,n_2,x,t)\eta_1(x)\,dx+{}\\
{}+
\int\limits_0^{\infty}P_{2j}(n_1,n_2+1,x,t)\eta_2(x)\,dx+{}\\
{}+\int\limits_0^{\infty}P_{0j}(n_1,n_2,0,t)\eta_0(x)\,dx\,.
\end{multline}

Будем предполагать, что в~начальный момент времени $t\hm=0$ система
свободна от требований, а~с~начала профилактики прибора прошло
случайное время с~заданным распределением с~плотностью $d(x).$
Таким образом,
\begin{align*}
P_{ij}\left(n_1,n_2,x,0\right)&=0,\ i=1,2;
\\
P_{0j}\left(n_1,n_2,x,0\right)&=c_jd(x)\delta_{n_1+n_2,0},\ \
j=1,\ldots,k\,.
\end{align*}
Положим
\begin{multline*}
p_{ij}\left(z_1,z_2,x,s\right)={}\\
{}=\sum\limits_{n_1=0}^{\infty}
\sum\limits_{n_2=0}^{\infty}z_1^{n_1}z_2^{n_2}\!
\int\limits_0^{\infty}e^{-st}P_{ij}(n_1,n_2,x,t)\,dt\,;
\end{multline*}
$$
  \psi(s)=\int\limits_0^{\infty}e^{-sx}\,dx
  \int\limits_0^{\infty}\fr{c(u+x)d(u)}{1-C(u)}\,du\,.
$$
Тогда, учитывая начальные условия,  из \eqref{3}
получаем:
\begin{multline}
\label{7} 
\fr{\partial p_{ij}(z_1,z_2,x,s)}{\partial x}={}\\
{}=-\left(s+a_j+\eta_i(x)\right)p_{ij}
\left(z_1,z_2,x,s\right)+{}\\
{}+c_j\left(pz_1+(1-p)z_2\right)
\sum\limits_{l=1}^ka_lp_{il}\left(z_1,z_2,x,s\right),\\ 
i=1,2;
\end{multline}

\vspace*{-12pt}

\noindent
\begin{multline}
\label{8} 
\fr{\partial p_{0j}(z_1,z_2,x,s)}{\partial x}={}\\
{}=-\left(s+a_j+\eta_0(x)\right)p_{0j}\left(z_1,z_2,x,s\right)+{}\\
{}+c_j\left(pz_1+(1-p)z_2\right)\sum\limits_{l=1}^ka_lp_{0l}\left(z_1,z_2,x,s\right)+{}\\
{}+ c_jd(x).
\end{multline}
Решения \eqref{7} и~\eqref{8} имеют вид:
\begin{multline}
\label{9}
p_{ij}\left(z_1,z_2,x,s\right)=\left(1-B_i(x)\right)c_j\times{}\\
{}\times \sum\limits_{m=1}^k\fr{\gamma_i^{(m)}(z_1,z_2,s)}{\mu_m(z_1,z_2)+a_j}\,
e^{-(s-\mu_m(z_1,z_2))x}\,,\\
 i=1,2\,,
\end{multline}
\vspace*{-12pt}

\noindent
\begin{multline}
\label{10}
p_{0j}\left(z_1,z_2,x,s\right)={}\\
{}=\left(1-C(x)\right)
c_j\!\!\sum\limits_{m=1}^k\!\! e^{-(s-\mu_m(z_1,z_2))x}\!
\!\left(\!
\vphantom{\int\limits_{l=1}^k}
\delta^{(m)}\left(z_1,z_2,s\right)+{}\right.\\
%\left.
{}+\alpha_m^{-1}\left(z_1,z_2\right)
\prod\limits_{l=1}^k
\left(\mu_m\left(z_1,z_2\right)+a_l\right)\times{}\\
\left.{}\times \int\limits_0^x\!
e^{(s-\mu_m(z_1,z_2))u}
\fr{d(u)}{1-C(u)}\,du
\right)
\!\Bigg/ \!\left(\mu_m\left(z_1,z_2\right)+{}\right.\\
\left.{}+a_j\right)\,,
\end{multline}
где функции $\gamma_i^{(m)}(z_1,z_2,s)$  и~$\delta^{(m)}(z_1,z_2,s)$ являются
произвольными функциями указанных переменных и~определяются из
краевых условий. Из~\eqref{5} и~\eqref{6} получаем:
\begin{multline}
\label{11}
p_{1j}\left(z_1,z_2,0,s\right)+p_{2j}\left(z_1,z_2,0,s\right)={}\\
{}=z_1^{-1}\int\limits_0^{\infty}p_{1j}\left(z_1,z_2,x,s\right)\eta_1(x)\,dx+{}
\\
+z_2^{-1}\int\limits_0^{\infty}p_{2j}\left(z_1,z_2,x,s\right)\eta_2(x)\,dx+{}\\
{}+
\int\limits_0^{\infty}p_{0j}\left(z_1,z_2,x,s\right)\eta_0(x)\,dx
-p_{0j}\left(z_1,z_2,0,s\right)\,.
\end{multline}
Заметим, что $p_{0j}(z_1,z_2,0,s)$ не зависит от $z_1$ и~$z_2,$ т.\,е.\
$p_{0j}(z_1,z_2,0,s)\hm=q_j(s).$ 
Подставляя~\eqref{9} и~\eqref{10} в~\eqref{11}, получаем:
\begin{multline}
\label{12}
\gamma_1^{(m)}\left(z_1,z_2,s\right)\left(1-z_1^{-1}\beta_1(s-\mu_m(z_1,z_2))\right)+{}\\
{}+
\gamma_2^{(m)}(z_1,z_2,s)\left(1-z_2^{-1}\beta_2(s-\mu_m(z_1,z_2))\right)={}\\
{} =
\delta^{(m)}\left(z_1,z_2,s\right)\left(\gamma\left(s-\mu_m\left(z_1,z_2\right)\right)-1\right)+{}\\
{}+
\alpha_m^{-1}\left(z_1,z_2\right)\prod\limits_{l=1}^k
\left(\mu_m\left(z_1,z_2\right)+a_l\right)\psi\left(s-\mu_m(z_1,z_2)\right),\\
j=1,\ldots,k.
\end{multline}
В силу леммы~1 левая часть~\eqref{12} обращается в~0 при
$z_1\hm=z_{1m}(s)$ и~$z_2\hm=z_{2m}(s)$, $m\hm=1,\ldots,k.$ Следовательно,
\begin{multline}
\label{13}
\delta^{(m)}\left(z_{1m}(s),z_{2m}(s),s\right)={}\\
{}=\fr{\psi(s-\lambda_m(s))}{\alpha_m(z_{1m}(s),z_{2m}(s))
(1-\gamma(s-\lambda_m(s)))}\times{}\\
{}\times \prod\limits_{l=1}^k\left(\lambda_m(s)+a_l\right).
\end{multline}
Из \eqref{10} следует, что
$$
q_j(s)=c_j\sum\limits_{m=1}^k\fr{\delta^{(m)}(z_1,z_2,s)}{\mu_m(z_1,z_2)+a_j},\
j=1,\ldots,k .
$$
Решая эту систему уравнений относительно
$\delta^{(m)}(z_1,z_2,s),$ получаем:
\begin{multline}
\label{n1}
\delta^{(m)}(z_1,z_2,s)=\left(pz_1+(1-p)z_2\right)\times{}\\
{}\times
\fr{\prod\nolimits_{j=1}^k(\mu_m(z_1,z_2)+a_j)}
{\alpha_m(z_1,z_2)}\sum\limits_{l=1}^k\frac{a_lq_l(s)}{\mu_m(z_1,z_2)+a_l}.
\end{multline}
Подставляя в~\eqref{n1} $z_1\hm=z_{1m}(s)$ и~$z_2\hm=z_{2m}(s),$ имеем:
\begin{multline}
\label{14}
\delta^{(m)}\left(z_{1m}(s),z_{1m}(s),s\right)={}\\
{}=
\left(pz_{1m}(s)+(1-p)z_{2m}(s)\right)\times{}\\
{}\times
\fr{\prod\nolimits_{j=1}^k
(\lambda_m(s)+a_j)}{\alpha_m(z_{1m}(s),z_{1m}(s))}
\sum\limits_{l=1}^k\fr{a_lq_l(s)}{\lambda_m(s)+a_l}\,.
\end{multline}
Сравнивая два представления~\eqref{13} в~\eqref{14} для
$\delta^{(m)}(z_m(s),s),$ получаем систему уравнений для~$q_l(s)$:
\begin{multline*}
\sum\limits_{l=1}^k\fr{a_lq_l(s)}{\lambda_m(s)+a_l}={}\\
{}=\fr{\psi(s-\lambda_m(s))}{(pz_{1m}(s)+(1-p)z_{2m}(s))
(1-\gamma(s-\lambda_m(s)))},\\
m=1,\ldots,k\,,
\end{multline*}
из которой находим
\begin{multline}
\hspace*{-3pt}q_l(s)=c_l\prod\limits_{j=1}^k
\left(\lambda_l(s)+a_j\right) 
\sum\limits_{m=1}^k
%\fr
\psi(s-\lambda_m(s))\!\Bigg/ \!
\Bigg(\left(1-{}\right.\\
\left.
{}-\gamma\left(s-\lambda_m(s)\right)\right)(\lambda_m(s)+a_l)\times{}\\
{}\times \prod\limits_{n\ne m}(\lambda_m(s)-\lambda_n(s))\!\Bigg).
\label{15}
\end{multline}
Подставляя \eqref{15} в~\eqref{n1} и~учитывая~\eqref{1}, получаем:
\begin{multline*}
\delta^{(m)}(z_1,z_2,s)=\fr{(pz_1+(1-p)z_2)}{\alpha_m(z_1,z_2)}\times
\\
\times\sum\limits_{j=1}^k
\fr{\psi(s-\lambda_j(s))\prod\nolimits_{l=1}^k(\lambda_j(s)+a_l)}
{(pz_{1j}(s)+(1-p)z_{2j}(s))(1-\gamma(s-\lambda_j(s)))}\times{}\\
{}\times\prod\limits_{\nu\ne j}
\fr{\mu_m(z_1,z_2)-\lambda_{\nu}(s)}{\lambda_j(s)-\lambda_{\nu}(s)}\,.
\end{multline*}
Положим
$$
\lambda_m(z_2,s)=\mu_m\left(z_{1m}(z_2,s),z_2\right),\enskip m=1,\ldots,k\,.
$$
Подставляя в~\eqref{12} $z_1\hm=z_{1m}(z_2,s)$, имеем:
\begin{multline}
\label{1q}
\gamma_2^{(m)}\left(z_{1m}(z_2,s),z_2,s\right)={}\\
{}=\fr{\delta^{(m)}(z_{1m}(z_2,s),z_2,s)(\gamma_m(s-\lambda_m(z_2,s))-1)}
{1-z_2^{-1}\beta_2(s-\lambda_m(z_2,s))}+{}
\\
{}+\alpha_m^{-1}(z_{1m}(z_2,s),z_2)\psi(s-\lambda_m(z_2,s))
\prod\limits_{l=1}^k\left(\lambda_m(z_2,s)+{}\right.\\
\left.{}+a_l\right)\!\Bigg/\!
\left(
1-z_2^{-1}\beta_2(s-\lambda_m(z_2,s))\right).
\end{multline}
Далее, из~\eqref{9} следует:
$$
p_{2j}(z_1,z_2,0,s)=c_j\sum\limits_{m=1}^k
\fr{\gamma_2^{(m)}(z_1,z_2,s)}{\mu_m(z_1,z_2)+a_j}\,.
$$
Отсюда
\begin{multline}
\label{2q}
\gamma_2^{(m)}(z_1,z_2,s)=\fr{pz_1+(1-p)z_2}{\alpha_m(z_1,z_2)}\times{}\\
{}\times
\prod\limits_{j=1}^k(\mu_m(z_1,z_2)+a_j)
\sum\limits_{l=1}^k\fr{a_lp_{2l}(z_1,z_2,0,s)}{\mu_m(z_1,z_2)+a_l}\,.
\end{multline}
Так как $p_{2j}(z_1,z_2,0,s)$ не зависит от $z_1$, то
\begin{multline}
\label{3q}
p_{2j}\left(z_1,z_2,0,s\right)={}\\
{}=c_j
\sum\limits_{m=1}^k\fr{\gamma_2^{(m)}\left(z_{1m}(z_2,s),z_2,s\right)}{\lambda_m(z_2,s)+a_j}\,.
\end{multline}
Таким образом, соотношения~\eqref{1q}--\eqref{3q} полностью
определяют $\gamma_2^{(m)}(z_1,z_2,s)$ при любых $z_1$ и~$z_2$.
Теперь из~\eqref{12} можно найти $\gamma_2^{(m)}(z_1,z_2,s)$.

Все функции, необходимые для вычисления $p_{ij}(z_1,z_2,x,s)$,
$i\hm=0,1,2$, $j\hm=1,\ldots,k,$ найде-\linebreak\vspace*{-12pt}

\columnbreak

\noindent
ны. Искомая производящая функция
процесса $(L_1(t),L_2(t))$ равна:

\noindent
\begin{multline*}
\int\limits_0^{\infty}e^{-st}\mathbf{E}
z_1^{L_1(t)} z_2^{L_2(t)}\,dt={}\\
{}=
\sum\limits_{i=0}^2\sum\limits_{j=1}^k\int\limits_0^{\infty}p_{ij}
\left(z_1,z_2,x,s\right)\,dx\,.
\end{multline*}

\vspace*{-18pt}

{\small\frenchspacing
 {%\baselineskip=10.8pt
 \addcontentsline{toc}{section}{References}
 \begin{thebibliography}{9}
\bibitem{1-u}
\Au{Doshi B.\,T.} Queueing systems with vacations~--- a~survey~// 
Queueing Syst., 1986. Vol.~1.  P.~29--66.
\bibitem{2-u}
\Au{Takagi H.} Time-dependent analysis of $M\vert G\vert 1$ vacation models 
with exhaustive service~// Queueing Syst.,
1990. Vol.~6.  P.~369--390.
\bibitem{3-u}
\Au{Li J., Tian N., Zhang~Z.\,G. , Luh~H.\,P.} 
Analysis of the $M\vert G\vert 1$ queue with exponentially working vacations~--- 
a~matrix analytic approach~// Queueing Syst., 2009. Vol.~61.
P.~139--166.
\bibitem{4-u}
\Au{Bouman N., Borst S.\,C., Boxma~O.\,J., Leeuwaarden~J.\,S.\,H.} 
Queues with random back-offs~// Queueing Syst.,
2014. Vol.~77. P.~33--74.
\bibitem{5-u}
\Au{Ушаков~В.\,Г.} Система обслуживания с~гиперэкспоненциальным входящим потоком 
и~профилактиками прибора~// Информатика и~её применения, 2016. Т.~10. 
Вып.~2. С.~93--98.
 \end{thebibliography}

 }
 }

\end{multicols}

\vspace*{-9pt}

\hfill{\small\textit{Поступила в~редакцию 11.05.18}}

\vspace*{6pt}

%\pagebreak

%\newpage

%\vspace*{-28pt}

\hrule

\vspace*{2pt}

\hrule

%\vspace*{-2pt}

\def\tit{A~HEAD OF~THE~LINE PRIORITY QUEUE\\ WITH~WORKING VACATIONS}

\def\titkol{A head of the line priority queue with working vacations}

\def\aut{E.\,S.~Kondranin$^1$ and~V.\,G.~Ushakov$^{1,2}$}

\def\autkol{E.\,S.~Kondranin and~V.\,G.~Ushakov}

\titel{\tit}{\aut}{\autkol}{\titkol}

\vspace*{-11pt}


\noindent
$^1$Department of 
Mathematical Statistics, Faculty of Computational Mathematics and Cybernetics, 
M.\,V.~Lo\-mo-\linebreak
$\hphantom{^1}$no\-sov Moscow State University, 1-52~Leninskiye Gory, 
Moscow 119991, GSP-1, Russian Federation

\noindent
$^2$Institute of Informatics Problems, Federal Research Center 
``Computer Science and Control'' of the Russian\linebreak
$\hphantom{^1}$Academy of Sciences,  44-2~Vavilov Str., Moscow 119333, Russian Federation

\def\leftfootline{\small{\textbf{\thepage}
\hfill INFORMATIKA I EE PRIMENENIYA~--- INFORMATICS AND
APPLICATIONS\ \ \ 2018\ \ \ volume~12\ \ \ issue\ 4}
}%
 \def\rightfootline{\small{INFORMATIKA I EE PRIMENENIYA~---
INFORMATICS AND APPLICATIONS\ \ \ 2018\ \ \ volume~12\ \ \ issue\ 4
\hfill \textbf{\thepage}}}

\vspace*{3pt}



\Abste{The authors analyze the single-server queueing system with 
two types of customers, head of the line priority, hyperexponential 
input stream, and working vacations. The authors obtain the Laplace 
transform (with respect to an arbitrary point in time) of the joint 
distribution of server state, queue size, and elapsed time in that state. 
The authors restrict themselves to a~system with exhaustive service (the 
queue must be empty when the server starts a vacation) and multiple vacations. 
The queueing systems with vacations have been well studied because of their 
applications in modeling computer networks, communication, and manufacturing 
systems. For example, in many digital systems, the processor is multiplexed 
among a~number of jobs and, hence, is not available all the time to handle one job type. 
Besides such an application, theoretical interest in vacation models 
has been aroused with respect to their relationship with polling models.}

\KWE{hyperexponential input stream; working vacations; single server; 
head of the line priority; queue length}



\DOI{10.14357/19922264180405}

\vspace*{-14pt}

\Ack
\noindent
This work was supported by the Russian Foundation for Basic Research 
(project 18-07-00678).


%\vspace*{6pt}

  \begin{multicols}{2}

\renewcommand{\bibname}{\protect\rmfamily References}
%\renewcommand{\bibname}{\large\protect\rm References}

{\small\frenchspacing
 {%\baselineskip=10.8pt
 \addcontentsline{toc}{section}{References}
 \begin{thebibliography}{9}
\bibitem{1-u-1}
\Aue{Doshi, B.\,T.} 1986. Queueing systems with vacations~--- a~survey. 
\textit{Queueing Syst.} 1:29--66.
\bibitem{2-u-1}
\Aue{Takagi, H.} 1990. Time-dependent analysis of $M\vert G\vert M\vert 1$ 
vacation models with exhaustive service. \textit{Queueing Syst.} 6:369--390.
\bibitem{3-u-1}
\Aue{Li, J., N. Tian, Z.\,G.~Zhang,  and H.\,P.~Luh.} 2009. Analysis of the 
$M\vert G\vert 1$ queue with exponentially working vacations~--- 
a~matrix analytic approach. \textit{Queueing Syst.} 61:139--166.
{\looseness=1

}
\bibitem{4-u-1}
\Aue{Bouman, N., S.\,C.~Borst, O.\,J.~Boxma, and J.\,S.\,H.~Leeuwaarden.} 
2014. Queues with random back-offs. \textit{Queueing Syst.} 77:33--74.
\bibitem{5-u-1}
\Aue{Ushakov, V.\,G.} 2016. Sistema obsluzhivaniya s~gipereksponentsialnym 
vkhodyashchim potokom i~profilaktikami\linebreak pribora [Queueing system with working 
vacations and hyperexponential input stream]. 
\textit{Informatika i~ee Primeneniya~--- Inform. Appl.} 10(2):93--98.
\end{thebibliography}

 }
 }

\end{multicols}

\vspace*{-6pt}

\hfill{\small\textit{Received May 11, 2018}}

%\pagebreak

%\vspace*{-18pt}

\Contr

\noindent
\textbf{Kondranin Egor S.} (b.\ 1995)~---  MSc student, Department of 
Mathematical Statistics, Faculty of Computational Mathematics and Cybernetics, 
M.\,V.~Lomonosov Moscow State University, 1-52~Leninskiye Gory, 
Moscow 119991, GSP-1, Russian Federation; \mbox{ekondranin@yandex.ru}

\vspace*{6pt}

\noindent
\textbf{Ushakov Vladimir G.} (b.\ 1952)~--- 
Doctor of Science in physics and mathematics, professor, Department of Mathematical 
Statistics, Faculty of Computational Mathematics and Cybernetics, 
M.\,V.~Lomonosov Moscow State University, 1-52~Leninskiye Gory, Moscow 119991, 
GSP-1, Russian Federation; 
senior scientist, Institute of Informatics Problems, Federal Research Center 
``Computer Science and Control'' of the Russian Academy of Sciences, 
44-2~Vavilov Str., Moscow 119333, Russian Federation; \mbox{vgushakov@mail.ru}
\label{end\stat}

\renewcommand{\bibname}{\protect\rm Литература}           %2
%\def\mytheorem#1#2{\begin{my_the} #1 \end{my_the} \begin{proof} #2 \end{proof} }
%\def\mylemma#1#2{\begin{my_lem} #1 \end{my_lem} \begin{proof} #2 \end{proof} }
%\def\mystatement#1#2{\begin{my_state} #1 \end{my_state} \begin{proof} #2 \end{proof} }
%\def\mydefinition#1{\begin{my_def} #1 \end{my_def}}
%\def\mysample#1{\begin{my_sam} #1 \end{my_sam}}


\def\Set#1#2{\{{#1\ \colon\ #2 \}}}
\def\SetA#1#2{\{{#1\ \mid\ #2 \}}}

\def\sigmaF{\mathscr{F}}
\def\sigmaB{\mathscr{B}}
\def\classT{\mathbb {T}}
\def\divers{\succcurlyeq}


\newcommand{\Expect}{\mathsf{E}}

%\renewcommand{\ge}{\geqslant}
%\renewcommand{\le}{\leqslant}

\def\stat{yakovenko}

\def\tit{ДИВЕРСИФИКАЦИЯ И ЕЕ СВЯЗЬ С МЕРАМИ РИСКА}

\def\titkol{Диверсификация и ее связь с мерами риска}

\def\autkol{Д.\,О.~Яковенко, М.\,А.~Целищев}
\def\aut{Д.\,О.~Яковенко$^1$, М.\,А.~Целищев$^2$}

\titel{\tit}{\aut}{\autkol}{\titkol}

%{\renewcommand{\thefootnote}{\fnsymbol{footnote}}\footnotetext[1]
%{Работа выполнена при финансовой поддержке РФФИ (гранты 11-01-00515а и 11-01-12026-офи-м).}}

\renewcommand{\thefootnote}{\arabic{footnote}}
\footnotetext[1]{Гроссмейстер ФИДЕ, ms@cs.msu.su}
\footnotetext[2]{Московский государственный университет им.\ М.\,В.~Ломоносова, 
кафедра математической статистики факультета вычислительной математики и кибернетики; 
ms@cs.msu.su}

\Abst{Предложен новый подход к понятию
диверсификации инвестиционных портфелей, которое определяется как
бинарное отношение во множестве портфелей с конечным первым
моментом. Показано, что это бинарное отношение является (в
определенном смысле) частичной упорядоченностью. Рассмотрены важные
свойства этого определения, а также необходимое и достаточное
условия срав\-ни\-мости портфелей, важнейшую роль в которых играет
когерентная мера риска \textit{Expected Shortfall} (ожидаемый
дефицит). В~качестве примера приводится интерпретация диверсификации
информационного риска.}

\KW{диверсификация; инвестиционные портфели;
сравнение портфелей; когерентные меры риска; \textit{Expected
Shortfall}; информационный риск}

  \vskip 14pt plus 9pt minus 6pt

      \thispagestyle{headings}

      \begin{multicols}{2}
      
            \label{st\stat}

\section{Введение}

Диверсификация является одним из основных способов управления
рисками практически во всех областях экономической деятельности. 
В~основе диверсификации лежит идея распределения риска по различным
источникам, недопущение ситуации, когда одно неблагоприятное событие~--- 
резкое изменение цены какого-либо товара, нарушение контрагентом
графика поставок и~т.\,п. --- может привести к катастрофическим
последствиям. В~статье предложено формальное определение
диверсификации и рассмотрены свойства этого определения. В~качестве
примера рассматривается диверсификация информационных сис\-тем как
способ управления информационными рисками, под которыми понимаются
риски возникновения убытков из-за неправильной организации или
умышленного нарушения информационных потоков и сис\-тем организации.


\section{Формализация понятия диверсификации}

Введем вероятностное пространство $(\Omega,\sigmaF,\mu)$, где
$\Omega=[0;1)$, $\sigmaF$~--- сиг\-ма-ал\-геб\-ра борелевских множеств на
$[0;1)$, $\mu$ --- мера Лебега. $\Omega$~интерпретируется как
пространство возможных вариантов развития событий (траекторий) на
рынке. Под инвестиционным портфелем в этой статье понимается не
набор ценных бумаг, а некая стратегия, которой придерживается
инвестор. Стратегия представляет собой случайную величину~$\xi$,
определенную на введенном выше вероятностном пространстве и такую,
что на траектории $\omega \hm\in \Omega$ инвестор \textbf{несет
потери}, равные $\xi(\omega)$ (если $\xi(\omega) \hm< 0 $, то инвестор
получает прибыль в размере $ - \xi(\omega)$). Класс всех
рассматриваемых стратегий полагаем таким: $V\hm=\Set{\xi}{\exists\
\Expect|\xi|< \infty}$.

Будем использовать следующие обозначения:
\begin{enumerate}
\item $\xi_1 \sim \xi_2 \Leftrightarrow F_{\xi_1}(x) \equiv F_{\xi_2}(x)$, т.\,е.\ 
$\xi_1$ и $\xi_2$ одинаково распределены.
\item $\xi^{(\alpha)} = \inf \Set{x}{F_{\xi}(x) \hm> \alpha}$, $\alpha \hm\in [0,1]$,~--- 
верхняя квантиль порядка~$\alpha$.
\end{enumerate}

В дальнейшем потребуются две операции над портфелями: сложение двух
портфелей и умножение портфеля на константу. Под суммой двух
портфелей~$\xi_1$ и~$\xi_2$ понимается портфель, который на
траектории $\omega \hm\in \Omega$ дает потери
$\xi_1(\omega)\hm+\xi_2(\omega)$, а под умножением на константу $\alpha
\hm\in [0,1]$ (другие константы не понадобятся) портфеля~$\xi$
понимается портфель, который на траектории $\omega \hm\in \Omega$ дает
потери $\alpha \xi(\omega)$. С~экономической точки зрения стратегия
$\xi_1\hm+\xi_2$ подразумевает параллельное исполнение стратегий~$\xi_1$ и~$\xi_2$, 
а стратегия $\alpha \xi$ заключается в выполнении
всех финансовых операций стратегии~$\xi$, но только <<в доле>> с
другими инвесторами (так, чтобы собственная доля во всех операциях
составляла~$\alpha$).

\medskip

\noindent
\textbf{Определение 1.} Будем говорить, что портфель~$\xi_2 \hm\in V$ является
\textbf{результатом диверсификации} портфеля~$\xi_1 \hm\in V$
(обозначается~$\xi_2 \divers \xi_1$), если $\forall\eps\hm>0$
существуют портфели $\eta_0, \ldots , \eta_n, \eta_0', \ldots,
\eta_{n-1}'\hm \in V$ и числа $\alpha_1, \ldots, \alpha_n, 0 \hm\le
\alpha_i \hm\le 1$, такие~что $\eta_0 \hm= \xi_1$, $\eta_i \hm\sim \eta_i'$,
$\eta_i \hm= \alpha_i \eta_{i-1} \hm+ (1 \hm- \alpha_i)\eta_{i-1}'$ и $\eta_n
\hm\ge \xi_2 \hm- \eps$ почти всюду (п.\,в).

\medskip

\noindent
\textbf{Пример 1.} Пусть $X_1$, $X_2$, $X_3$~--- независимые одинаково
распределенные случайные величины (портфели). Рассмотрим портфель
вида $\xi_1 \hm= p_1 X_1 \hm+ p_2 X_2 \hm+ p_3 X_3$, $p_i \hm\ge 0$,
$p_1\hm+p_2\hm+p_3\hm=1$, и покажем, что портфель $\xi_2 \hm= (1/3)X_1 \hm+
(1/3) X_2 \hm+ (1/3) X_3$ является результатом диверсификации
портфеля~$\xi_1$. Без ограничения общности считаем, что $0 \hm\le p_1
\hm\le p_2 \hm\le p_3 \hm\le 1$ и $p_1 \hm< 1/3$ (случай, когда все $p_i \hm=
1/3$, не интересен в силу его тривиальности). Определим $\eta_0 \hm=
\xi_1 \hm= p_1 X_1 \hm+ p_2 X_2 \hm+ p_3 X_3$ , $\eta_0' \hm= p_3 X_1 \hm+ p_2 X_2
\hm+ p_1 X_3$ ($\eta_0 \hm\sim \eta_0'$, так как $X_1$, $X_2$, $X_3$
независимы и одинаково распределены), $\alpha_1 \hm= (p_3 \hm-
1/3)/(p_3 \hm- p_1)$ ($0 \hm< \alpha_1 \hm< 1$ в силу предположений на
$p_1$, $p_2$, $p_3$). Тогда 

\noindent
$$
\eta_1 = \alpha_1 \eta_0 + (1 - \alpha_1) \eta_0' 
= \fr{1}{3}\, X_1 + p_2 X_2 + \left(\fr{2}{3}- p_2\right) X_3\,.
$$ 
Теперь определим 
$$
\eta_1' = \fr{1}{3}\, X_1 + \left(\fr{2}{3}- p_2\right) X_2 + p_2 X_3
$$ и
$\alpha_2 \hm= 1/2 $. 
Тогда 



\noindent
\begin{multline*}
\eta_2 = \alpha_2 \eta_1 + (1 - \alpha_2) \eta_1' = 
\fr{1}{3}\, X_1 + \fr{1}{3}\, X_2 + \fr{1}{3}\, X_3 ={}\\
{}=
    \xi_2 > \xi_2 - \eps\,,\enskip \forall \eps\hm>0\,.
    \end{multline*} 
    Тем самым, согласно введенному определению, $\xi_2 \divers \xi_1$.

\vspace*{-3pt}

\section{Свойства введенного определения}

\vspace*{-1pt}

Введем класс отображений $ \classT \hm= \Set {T}{\Omega \rightarrow
\Omega} $, так чтобы $T \hm\in \classT $ тогда и только тогда, когда
$\exists A,B \hm\in \sigmaF$, $\mu(A)\hm=\mu(B)=1$, $\exists T' \colon B
\hm\rightarrow A$ такое, что $T'$ измеримо, обратимо и $\forall
G\hm\in\sigmaF\cap B$ $\exists \mu(T'(G)) \hm= \mu(G)$, причем
$T(\omega)\hm=T'(\omega)$  для  $\omega \hm\in B$ (на множестве
нулевой меры $\Omega \backslash B$ отображение~$T$ принимает
произвольные значения из~$\Omega$).

\medskip

\noindent
\textbf{Лемма~1.} \textit{$\forall \xi \in V $, $\forall T \in \classT $
выполняется: $\xi \sim \xi(T) $.} 

\smallskip

\noindent
Д\,о\,к\,а\,з\,а\,т\,е,л\,ь\,с\,т\,в\,о.\  
Пусть $\ T \hm\in \classT$.
Тогда $\forall G \hm\in \sigmaB$ ($\sigmaB$~--- сиг\-ма-ал\-геб\-ра
борелевских множеств на прямой) в силу того, что
$\mu(A)\hm=\mu(B)\hm=1$, имеем:

\noindent
\begin{multline*}
\p(\xi(T) \in G) = \mu( \Set{\omega}{T(\omega) \in \xi^{-1}(G)} ) = {}\\
{}=\mu(\Set{\omega}{T(\omega) \in \xi^{-1}(G)} \cap B)={} \\
{}=\mu(\Set{\omega}{T(\omega) \in \xi^{-1}(G)\cap A} \cap B)={} \\
{}=\mu(\Set{\omega \in B}{T'(\omega) \in \xi^{-1}(G)\cap A})={} \\
{}=\mu(B\cap (T')^{-1}(\xi^{-1}(G)\cap A))={}\\
{}= \mu((T')^{-1}(\xi^{-1}(G)\cap A))={}\\
{}= \mu(\xi^{-1}(G)\cap A)= \mu(\xi^{-1}(G)) = \p(\xi \in G)\,.
\end{multline*}
Лемма доказана.\hfill$\square$

\columnbreak
%\medskip

\noindent

\textbf{Лемма~2.}
\textit{Пусть $\xi_1, \xi_2 \hm\in V $ и $\xi_1 \hm\sim \xi_2$. Тогда
$ \forall \eps \hm> 0$ $\exists  T \hm\in \classT$ такое, что $
|\xi_1(T)\hm-\xi_2|\hm<\eps $ почти наверное.} 

\medskip

\noindent
Д\,о\,к\,а\,з\,а\,т\,е\,л\,ь\,с\,т\,в\,о\,.\ Фиксируем произвольное
$\eps \hm> 0 $. Обозначим
$$
A_i = \xi_1^{-1}\left(  \left[  i \eps ; (i+1) \eps \right) \right )\,,  \\
B_i = \xi_2^{-1}\left( \left [  i \eps ; (i+1) \eps \right ) \right )\,.
$$
Поскольку $\xi_1 \hm\sim \xi_2$, то $\forall i$  $\mu(A_i) \hm=
\mu(B_i)$. Согласно~[1, с. 74], $\forall i$ $\exists
A_i'$, $B_i' \hm\in \sigmaF$, $A_i' \subseteq A_i$,
$B_i' \hm\subseteq B_i$, $\mu(A_i') \hm= \mu(A_i)$, $\mu(B_i')\hm =
\mu(B_i)$ и $\exists T_i \colon B_i' \hm\to A_i'$ такое, что $T_i$
измеримо, $\exists T_i^{-1}$ и $\forall G \hm\in \sigmaF \cap B_i': 
\mu(T_i(G)) \hm= \mu(G)$. Определим $T(\omega) \hm= T_i(\omega)$ для 
$\omega \hm\in B_i'$. На множестве нулевой меры $\Omega
\backslash (\bigsqcup\limits_{-\infty}^\infty B_i')$ доопределим~$T$ 
произвольно. Обозначим $B \hm= \bigsqcup\limits_{-\infty}^\infty
B_i'$, $A \hm= \bigsqcup\limits_{-\infty}^\infty A_i'$. В~этом случае
$\mu(A) = \mu(B) = 1$. Таким образом, $T \hm\in \classT$ и $\forall i
\ \forall \omega \hm\in B_i' $:
\begin{multline*}
| \xi_1(T(\omega)) - \xi_2(\omega) | = | \xi_1(T_i(\omega)) -
\xi_2(\omega) | <{}\\
{}< (i+1) \varepsilon - i \varepsilon =
\varepsilon\,.
\end{multline*} 
Лемма доказана.\hfill$\square$ 

\medskip

\noindent
\textbf{Определение 2.}
\textbf{Оператором диверсификации} назовем оператор
$D_{\alpha,T} : V \hm\rightarrow V$, $\alpha \hm\in [0;1]$, $T \hm\in \classT$,
такой что $D_{\alpha,T}(\xi) \hm= \alpha \xi \hm+ ( 1 \hm- \alpha ) \xi(T)$.

Важные свойства операторов диверсификации:
\begin{enumerate}
\item Если $\xi_1 \hm= \xi_2$ ($\xi_1 \hm\le \xi_2$) почти наверное, то
$D_{\alpha,T}(\xi_1) \hm= D_{\alpha,T}(\xi_2)$ 
($D_{\alpha,T}(\xi_1) \hm\le D_{\alpha,T}(\xi_2)$  соответственно)
 почти наверное.
\item Если $\xi,\eta\hm\in\,V$, $\eta\hm=const$ п.\,в., 
то $D_{\alpha,T}(\xi+\eta) \hm= D_{\alpha,T}(\xi) \hm+ \eta$.
\end{enumerate}

\medskip

\noindent
\textbf{Лемма 3.} \textit{$\xi_2 \divers \xi_1$ тогда и только тогда, когда $
\forall \eps\hm>0 $ существует последовательность операторов
диверсификации $ D_1=D_{\alpha_1,T_1},\, \ldots\, ,
D_n\hm=D_{\alpha_n,T_n} $ такая, что $ D_n(\ldots(D_1(\xi_1))\ldots)
\ge \xi_2 \hm- \eps $ почти наверное.}

\medskip

\noindent
Д\,о\,к\,а\,з\,а\,т\,е\,л\,ь\,с\,т\,в\,о\,.\
%$\phantom{hello,world}$\\

\noindent
\fbox{$\Leftarrow$} Непосредственно вытекает из определений~1, 2 и леммы~1.

\noindent
\fbox{$\Rightarrow$} Пусть $\xi_2 \divers \xi_1$. Фиксируем $\eps
\hm> 0$ и строим $\eta_0, \ldots , \eta_n, \eta_0', \ldots,
\eta_{n-1}', \alpha_1, \ldots, \alpha_n$ из определения~1, так чтобы
$ \eta_n \hm\ge \xi_2 \hm- \eps/2 $ почти наверное. Пользуясь
леммой~2, построим $ T_i \hm\in \classT $ так, что $ | \eta_{i-1}' \hm-
\eta_{i-1}(T_i) | \hm\le \eps/(2n) $ почти наверное. Определим
$D_i \hm= D_{\alpha_i,T_i}$, $i \hm=1,\ldots,n,$ и покажем, что эта
последовательность операторов диверсификации будет искомой.
Определим $\zeta_0 \hm= \xi_1$, $\zeta_k \hm= D_k(\zeta_{k-1})$, $k \hm=
1,\ldots,n$. Если доказать, что $ | \zeta_n \hm- \eta_n | \hm\le
\eps/2$ почти наверное, то в силу того, что $ \eta_n \hm\ge
\xi_2 \hm- \eps/2 $ почти наверное, получим $ \zeta_n \hm\ge \xi_2
\hm- \eps $ почти наверное, и лемма будет доказана. Обоснуем
необходимое неравенство по индукции: будем доказывать, что $\forall
k=0,\dots,n$ $| \zeta_k \hm- \eta_k | \hm\le k \eps/(2n)$ почти
наверное. При $k\hm=0$ утверждение справедливо: $|\zeta_0 \hm- \eta_0| \hm=
|\xi_1 \hm- \xi_1| \hm= 0 $. Пусть утверждение справедливо для некоторого
$k<n$, докажем его для $k+1$:
\begin{multline*}
\left\vert \zeta_{k+1} - \eta_{k+1} \right\vert = 
\left\vert\alpha_{k+1}\zeta_k + (1-\alpha_{k+1})\zeta_k(T_{k+1}) -{}\right.\\
\left.{}-  \alpha_{k+1}\eta_k -                  (1-\alpha_{k+1})\eta_k' \right\vert \le \\
\le \alpha_{k+1} \left\vert \zeta_k - \eta_k \right\vert 
+ \left(1-\alpha_{k+1}\right) \left\vert\zeta_k(T_{k+1}) - \eta_k'\right\vert \le {}\\
\!{}\le \alpha_{k+1} \left| \zeta_k - \eta_k \right| + \left(1-\alpha_{k+1}\right) 
\left\vert\zeta_k(T_{k+1}) - \eta_k(T_{k+1})\right\vert +{}\\
{}+    \left(1-\alpha_{k+1}\right) \left\vert\eta_k\left(T_{k+1}\right) - \eta_k'\right\vert \le{} \\
{}\le \fr{(k+1)\eps}{2n} 
\end{multline*}
почти наверное.

Тем самым лемма доказана.\hfill$\square$


\section{Необходимое условие диверсифицируемости}

В теории управления рисками хорошо известна когерентная мера риска
под названием \textit{Expected Shortfall} (ожидаемый дефицит):

\noindent
$$
{ES}_\gamma(\xi) = \fr{1}{\gamma} \int\limits_{1-\gamma}^1 \xi^{(t)} \,dt\,,\enskip
\gamma \in  (0,1]\,.
$$
Согласно~\cite{es}, \textit{Expected Shortfall} обладает следующими
свойствами:
\begin{enumerate}
\item $\xi\in V$, $\xi\hm\le0$  \textit{почти наверное} ${}\Rightarrow {ES}_\gamma(\xi)\hm\le0$, 
$\forall \gamma\hm \in (0;1]$;
\item ${ES}_\gamma(\xi_1+\xi_2) \le {ES}_\gamma(\xi_1) \hm+ {ES}_\gamma(\xi_2)$, 
$\forall \xi_1,\xi_2\hm\in V$;
\item ${ES}_\gamma(a\xi) = a {ES}_\gamma(\xi)$, $\forall \xi\hm\in V$, $a\hm\ge 0$;
\item ${ES}_\gamma(\xi+a) = {ES}_\gamma(\xi)\hm+a$, $\forall \xi\in V$, $a\hm\in\mathbb{R}$;
\item $\xi_1,\xi_2\in V$, $\xi_1\hm\sim\xi_2 \hm\Rightarrow{ES}_\gamma(\xi_1)\hm={ES}_\gamma(\xi_2)$.
\end{enumerate}

Простым следствием из свойств~1 и~2 является свойство монотонности:
\begin{enumerate}[6.]
\item $\xi_1,\xi_2\in V$, $\xi_1 \hm\le \xi_2$ \textit{почти наверное} $
{}\Rightarrow {ES}_\gamma(\xi_1) \hm\le {ES}_\gamma(\xi_2)$, $\forall \gamma \in (0;1]$.
\end{enumerate}

\medskip

\noindent
\textbf{Утверждение 1.} \textit{$\forall \xi \in V$ $\forall \alpha \in [0;1] 
\forall T \hm\in \classT$ справедливо неравенство 
${ES}_\gamma ( D_{\alpha , T}(\xi)) \hm\le {ES}_\gamma (\xi)$  для $\forall
\gamma \hm\in (0;1]$. }

\smallskip

\noindent
Д\,о\,к\,а\,з\,а\,т\,е\,л\,ь\,с\,т\,в\,о\,.\ В~силу леммы~1 \ $\xi \hm\sim \xi(T)$. По
указанным выше свойствам \textit{Expected Shortfall} имеем

\noindent
\begin{multline*}
{ES}_\gamma ( D_{\alpha , T}(\xi)) =
{ES}_\gamma ( \alpha \xi + (1 - \alpha) \xi(T)) \stackrel{\text{св.2}}{\le}{} \\
{}\le {ES}_\gamma ( \alpha \xi ) + {ES}_\gamma  ( (1 - \alpha) \xi(T)) \stackrel{\text{св.3}}{=}{}\\
{}=\alpha {ES}_\gamma ( \xi ) + (1 - \alpha) {ES}_\gamma ( \xi(T)) \stackrel{\text{св.5}}{=}  \\
= \alpha {ES}_\gamma ( \xi ) + (1 - \alpha) {ES}_\gamma ( \xi ) =
{ES}_\gamma ( \xi )\,.
\end{multline*}
Утверждение доказано.\hfill$\square$

\columnbreak 

%\medskip

\noindent
\textbf{Теорема 1.}
\textit{Если $\xi_2 \divers \xi_1 $, то ${ES}_\gamma(\xi_2) \hm\le
{ES}_\gamma(\xi_1)$ для всех $ \gamma \hm\in (0;1] $.} 

\medskip

\noindent
Д\,о\,к\,а\,з\,а\,т\,е\,л\,ь\,с\,т\,в\,о\,.\ По лемме~3
для любого $\eps\hm>0$ существует последовательность операторов
диверсификации $ D_1,\, \ldots\, , D_n $ такая, что
$D_n(\ldots(D_1(\xi_1))\ldots) \hm\ge \xi_2 \hm- \eps$ почти наверное.
Используя свойства \textit{Expected Shortfall} и утверждение~1, получим
цепочку неравенств
\begin{multline*}
{ES}_\gamma (\xi_2) - \eps \stackrel{\text{св.4}}{=} {ES}_\gamma (\xi_2 - \eps)
 \stackrel{\text{св.6}}{\le}{}\\
 {}\le {ES}_\gamma ( D_n(\ldots(D_1(\xi_1))\ldots) ) \stackrel{\text{утв.1}}{\le}{} \\
{}\le {ES}_\gamma( D_{n-1}(\ldots(D_1(\xi_1))\ldots) ) \stackrel{\text{утв.1}}{\le}\! \ldots
\! \stackrel{\text{утв.1}}{\le} {ES}_\gamma (\xi_1).\hspace*{-4.50763pt}
\end{multline*}
Так как это справедливо для всех $\eps \hm> 0$, то, переходя к пределу
при $\eps \hm\to +0$, получим ${ES}_\gamma(\xi_2) \hm\le
{ES}_\gamma(\xi_1)$. Теорема доказана.\hfill$\square$

\medskip

\noindent
\textbf{Теорема~2.}
\textit{Бинарное отношение <<$\divers$>> является час\-тич\-ной
упорядоченностью на множестве~$V$ (антисимметричность при этом
понимается в том смысле, что из $\xi_1 \divers \xi_2$, $\xi_2
\divers \xi_1$ следует, что $\xi_1 \hm\sim \xi_2$).} 

\medskip


\noindent
Д\,о\,к\,а\,з\,а\,т\,е\,л\,ь\,с\,т\,в\,о\,.\ 
\textit{Рефлексивность} очевидна.

\textit{Транзитивность.} Пусть $\xi_2 \divers \xi_1$, $\xi_3 \divers
\xi_2$. По лемме~3 существуют последовательности операторов
диверсификации $D_1, \ldots , D_n, D_1', \ldots, D_m'$ такие, что
\begin{align*}
D_n(\ldots(D_1(\xi_1))\ldots) &\ge \xi_2 - \fr{\eps}{2}\ \mbox{п.\,в.}\,;
\\
D_m'(\ldots(D_1'(\xi_2))\ldots) &\ge \xi_3 - \fr{\eps}{2}\ \mbox{п.\,в.}
\end{align*}
Тогда 
\begin{multline*}
D_m'(\ldots(D_1'(   D_n(\ldots(D_1(\xi_1))\ldots) ))\ldots)
\ge{}\\
{}\ge D_m'(\ldots\left(D_1'\left(  \xi_2 - \fr{\eps}{ 2}\right )\right)\ldots) ={}\\
{}=
D_m'(\ldots(D_1'(  \xi_2 ))\ldots) - \fr{\eps}{2}\ge  \xi_3 - \fr{\eps}{2}
- \fr{\eps}{2}= \xi_3 - \eps\,. 
\end{multline*}
По лемме~3 $\xi_3 \divers \xi_1$.

\textit{Антисимметричность.} Пусть $\xi_1 \divers \xi_2$, $\xi_2
\hm\divers \xi_1$. Тогда по теореме~1 $\forall \gamma \hm\in (0,1]$
справедливо ${ES}_\gamma(\xi_1)\hm = {ES}_\gamma(\xi_2)$, т.\,е.\
$$
\fr{1}{\gamma} \int\limits_{1-\gamma}^1 \xi_1^{(t)} \,dt = \fr{1}{\gamma}
\int\limits_{1-\gamma}^1 \xi_2^{(t)} \,dt\,.
$$ 
Отсюда вытекает, что
$\xi_1^{(t)} \hm= \xi_2^{(t)}$ почти наверное и, следовательно, $\xi_1
\hm\sim \xi_2$. Теорема доказана.\hfill$\square$


\section{Достаточное условие диверсифицируемости}

\noindent
\textbf{Лемма~4.} \textit{Пусть $[a;b),[c;d) \subset \Omega$,
$[a;b)\cap[c;d)=\varnothing$, $b-a\hm=d\hm-c$; $\xi\hm\in V$,
$\xi(\omega) \hm= y_1$  при $\omega \hm\in [a;b)$, $\xi(\omega)\hm =
y_2$ при $\omega \in [c;d)$; числа $y_1'$, $y_2'$ таковы, что
$0 \hm\le y_1 \hm\le y_1' \hm\le y_2' \hm\le y_2$ и $y_1\hm+y_2 \hm= y_1'\hm+
y_2'$. Тогда существует оператор диверсификации $D \hm= D_{\alpha,T}$
такой, что $D(\xi) \hm= y_1'$ при $\omega \hm\in [a;b)$, $D(\xi) \hm= y_2'$
при $\omega \hm\in [c;d)$,  $D(\xi) \hm= \xi$ при остальных~$\omega $.}

\medskip

\noindent
Д\,о\,к\,а\,з\,а\,т\,е\,л\,ь\,с\,т\,в\,о\,.\ 
Построим~$T(\omega)$:  $T(\omega) \hm= c \hm+ (\omega \hm- a)$ при $\omega
\hm\in [a;b)$, $T(\omega) \hm= a \hm+ (\omega \hm- c)$ при $\omega \hm\in [c;d)$,
$T(\omega) \hm= \omega$ при остальных~$\omega$. Ясно, что $T \hm\in
\classT$. Возьмем $\alpha \hm= (y_2-y_1')/(y_2-y_1)$. Оператор
$D_{\alpha,T}$ и будет искомым. Лемма доказана.\hfill$\square$

\medskip

\noindent
\textbf{Лемма~5.} \textit{Пусть $[a;b),[c;d) \subset \Omega$,
$[a;b)\cap[c;d)\hm=\varnothing$; $\xi\hm\in V$, $\xi(\omega) \hm= y_1$ при
$\omega\hm\in[a;b)$, $\xi(\omega)\hm = y_2$ при $\omega\hm\in[c;d)$. Тогда
если $ 0 \hm\le y_1 \hm\le y_1' \hm\le y_2' \hm\le y_2 $ и 
$(y_2 \hm- y_2')(d-c) \hm > (y_1' \hm- y_1)(b-a)$, 
то существует конечная последовательность
операторов диверсификации $D_n,\,\ldots\,,D_1$ такая, что
$\xi'(\omega) \hm= D_n(\ldots(D_1(\xi(\omega)))\ldots) \hm= y_1'$ при  
$\omega \hm\in [a;b)$, $\xi'(\omega) \hm\ge y_2'$  при 
$\omega \hm\in [c;d)$ и  $\xi'(\omega) \hm= \xi(\omega)$ при
остальных~$\omega$.} 

\medskip

\noindent
Д\,о\,к\,а\,з\,а\,т\,е\,л\,ь\,с\,т\,в\,о\,.\  Так как 
$(y_2 \hm- y_2')(d-c) \hm> (y_1' \hm- y_1)(b-a)$, то существует $d' \hm\in [c,d)$ такое, что 
$(y_2 \hm- y_2')(d'-c)\hm>(y_1' \hm- y_1)(b-a)$ и $(d'-c)/(b-a)$ рационально, т.\,е.\ 
$(d'-c)/(b-a) \hm=m/n$, $m,n \hm\in \mathbb{N}$.
Разделим полуинтервалы $[a;b)$ и $[c;d')$ на $n$ и~$m$ равных
полуинтервалов, т.\,е.\
$a\hm=a_0\hm<a_1<\ldots$\linebreak $\ldots<a_n\hm=b$; $c\hm=c_0\hm<c_1\hm<\ldots\hm<c_m\hm=b$ и 
$\forall i,j:
a_i\hm-a_{i-1}\hm=c_j\hm-c_{j-1}$.
Укажем теперь алгоритм построения искомой последовательности операторов диверсификации.

\smallskip

\noindent
\fbox{Шаг 1.} Пусть $y_2-y_2' \hm\ge y_1'-y_1$ (если это не так, то
переходим к следующему шагу). Тогда по предыдущей лемме существует~$D_1$ 
такой, что $\xi_1 \hm\equiv D_1(\xi) \hm= y_1'$ при $\omega \hm\in [a;a_1)$, 
$\xi_1 \hm= z \hm= y_2\hm-(y_1'\hm-y_1) \hm\ge y_2'$ при $\omega \hm\in [c;c_1)$, 
$\xi_1 \hm= \xi$ при остальных~$\omega$. Далее, если
$z-y_2' \hm\ge y_1'\hm-y_1$, то таким же образом строим (с заменой~$y_2$
на~$z$ и $[a;a_1)$ на $[a_1;a_2)$) оператор~$D_2$, получим случайную величину
$\xi_2 \hm\equiv D_2(\xi_1)$, которая равна~$y_1'$ уже на полуинтервале
$[a;a_2)$. Затем перейдем к полуинтервалу $[a_2;a_3)$ и~т.\,д.

\smallskip

\noindent
\fbox{Шаг 2.} Когда же выполнится неравенство $z-y_2' \hm< y_1'\hm-y_1$
(пусть это произойдет после применения оператора~$D_k$), построим
оператор~$D_{k+1}$ такой, что случайная величина $\xi_{k+1}(\omega)
\hm\equiv D_{k+1}(\xi_k(\omega)) \hm= y_2'$ при $\omega \hm\in [c;c_1)$,
$\xi_{k+1}(\omega) \hm= y_1\hm+(z\hm-y_2') \hm< y_1'$ при $\omega \hm\in [a_k;a_{k+1})$, 
$\xi_{k+1}(\omega) \hm= \xi_k(\omega)$ при остальных~$\omega$ 
($\xi_{k+1}\hm=y_1'$ на $[a;a_k)$). Теперь переходим от
$[c;c_1)$ к $[c_1;c_2)$ и продолжаем процесс c шага~1 (с заменой~$y_1$ 
на $y_1+(z-y_2')$, $[a;a_1)$ на $[a_k;a_{k+1})$, $[c;c_1)$ на
$[c_1;c_2)$) и~т.\,д.

Так как $(y_2 \hm- y_2')(d'\hm-c)>(y_1'\hm -
y_1)(b-a)$, или $(y_2 \hm- y_2')m>(y_1' \hm- y_1)n$, то через $l \hm\le m+n$
шагов получим случайную величину~$\xi_l$ такую, что
$\xi_l(\omega)\hm=y_1'$ при $\omega \hm\in [a;b)$, $\xi_l(\omega) \hm\ge
y_2'$ при $\omega \hm\in [c;d)$ и $\xi_l(\omega)\hm=\xi(\omega)$ при
остальных~$\omega$. Лемма доказана.\hfill$\square$

\smallskip

\noindent
\textbf{Лемма~6.}
\textit{Пусть $[a;b)\in \Omega$, $C = \bigsqcup\limits_{i=1}^n
[c_i;d_i) \hm\in \Omega$, $[a;b) \cap C=\varnothing$
$\xi\hm\in V$, $\xi(\omega)=x$ при $\omega \hm\in [a;b)$,
$\xi(\omega)\hm=y_i$  при  $\omega \hm\in
[c_i;d_i)$, $i\hm=\overline{1,n}$. Пусть чис\-ла~$x'$ и
$y'_i$, $i=\overline{1,n},$ таковы, что $\forall i$ $0 \hm\le x \hm\le x'
\hm\le y_i' \hm\le y_i$ и $\sum\limits_{i=1}^n(d_i-c_i)(y_i-y_i') \hm>
(b-a)(x'-x)$. Тогда существует последовательность операторов
диверсификации $D_n,\ldots,D_1$ такая, что $\xi'(\omega) \hm\equiv
D_n(\ldots(D_1(\xi(\omega)))\ldots) \hm= x'$ при $\omega \hm\in [a;b)$,
$\xi'(\omega) \hm\ge y_i'$ при $\omega \hm\in
[c_i;d_i)$, $i\hm=\overline{1,n}$, и $\xi'(\omega) \hm= \xi(\omega)$ при
остальных~$\omega$.} 

\smallskip

\noindent
Д\,о\,к\,а\,з\,а\,т\,е\,л\,ь\,с\,т\,в\,о\,.\ 
Обозначим
$\delta\hm=\sum\limits_{i=1}^n(d_i\hm-c_i)(y_i-y_i')\hm-(b\hm-a)(x'\hm-x) \hm>0$.
Построим последовательность $x_0\hm=x$, $x_i\hm=x_{i-1} \hm+((d_i-c_i)(y_i-y_i')\hm-\delta / n)/
(b-a)$, $i\hm=\overline{1,n}$.

Тогда $x_n\hm=x'$ и $(d_i\hm-c_i)(y_i\hm-y_i') \hm> (x_i\hm-x_{i-1})(b\hm-a)$. По
предыдущей лемме существуют последовательности операторов
диверсификации $D_{ik_i},\ldots,D_{i1}$, $i\hm=\overline{1,n},$ и
случайные величины $\xi_i\hm=D_{ik_i}(\ldots(D_{i1}(\xi_{i-1}))\ldots)$
такие, что $\xi_i(\omega) \hm= x_i$ при $\omega \hm\in [a;b)$,
$\xi_i(\omega) \hm\ge y_i'$ при $\omega \hm\in [c_i;d_i)$,
$\xi_i(\omega) \hm= \xi_{i-1}(\omega)$ при остальных~$\omega$,
$i\hm=\overline{1,n}$. Последовательность
$D_{nk_n},\ldots,D_{n1},\ldots,D_{1k_1},\ldots,D_{11}$ и будет
искомой. Лемма доказана.\hfill$\square$

\medskip

\noindent
\textbf{Теорема~3.}
\textit{Пусть $\xi_1,\xi_2 \in V$~--- почти наверное
ограниченные случайные величины. Тогда если $\forall
\gamma \hm \in (0;1]$ выполняется ${ES}_\gamma(\xi_2) \hm\le
{ES}_\gamma(\xi_1)$, то $\xi_2 \divers \xi_1$.}

\smallskip

\noindent
Д\,о\,к\,а\,з\,а\,т\,е\,л\,ь\,с\,т\,в\,о\,.\  \fbox{I}~Сначала
докажем теорему для простых, монотонно неубывающих и непрерывных
справа случайных величин~$\xi_1$ и~$\xi_2$. В~этом случае $\forall
\omega \hm\in \Omega\hm=[0;1)$ имеем
$\xi_i(\omega)\hm\equiv\xi_i^{(\omega)}$, $i\hm=1,2,$ и

\noindent
$$
{ES}_\gamma(\xi_i) = \fr{1}{\gamma} 
\int\limits_{1-\gamma}^1{\xi_i(\omega)\,d\omega}\,,\enskip \gamma\in (0;1]\,.
$$
Разобьем множество $\Omega\hm=[0;1)$ на конечное число непересекающихся
интервалов, на каждом из которых~$\xi_1$ и~$\xi_2$ постоянны:
$0{=}x_0\hm<x_1\hm<\ldots\hm<x_n{=}1$;
$\xi_1(\omega)\hm=a_i$, $\xi_2(\omega)\hm=b_i$ при
$\omega\hm\in [x_{i-1};x_i)$; $a_i\hm\le a_{i+1}$, $b_i \hm\le b_{i+1}$.
Пусть $1\le k_m \hm< \ldots \hm< k_1\le n$~--- номера, такие что $a_{k_i}
\hm< b_{k_i}$ (если таких номеров нет, то $\xi_1\hm\ge\xi_2$ и теорема
верна). Обозначим $p_i\hm=x_i\hm-x_{i-1}$,
$A\hm=\Set{\omega\hm\in\Omega}{\xi_1(\omega)\ge\xi_2(\omega)}\hm=\Omega
\backslash \bigsqcup\limits_{i=1}^m [x_{k_i-1};x_{k_i})$.
Множество~$A$ можно представить в виде объединения непересекающихся полуинтервалов.

По условию
\begin{multline*}
{ES}_{1-\gamma}(\xi_1) \ge {ES}_{1-\gamma}(\xi_2) \forall \gamma\in [0;1) \Rightarrow{}\\
{}\Rightarrow
\int\limits_\gamma^1 \xi_1(\omega)\,d\omega \ge \int\limits_\gamma^1 \xi_2(\omega)\,d\omega \Rightarrow {}\\
{}
\Rightarrow \int\limits_\gamma^1 (\xi_1(\omega)-\xi_2(\omega))\,d\omega \ge 0 \; \ \ \forall \gamma \in [0;1)\,.
\end{multline*}
Подставляя в последнее неравенство $x_{k_i-1}$ вмес\-то~$\gamma$ и используя равенство
$$
[x_{k_i-1};1) \backslash A = \bigsqcup\limits_{j=1}^{i} [x_{k_j-1};x_{k_j})\,,\enskip
i=\overline{1,m}\,,
$$
получим, что $\forall i \hm\in \overline{1,m}$ верно:
\begin{multline*}
\int\limits_{[x_{k_i-1};1)\cap A}(\xi_1(\omega)-\xi_2(\omega))\,d\omega \ge{}\\
{}\ge
-\!\!\! \int\limits_{[x_{k_i-1};1)\backslash A}\!\!\!(\xi_1(\omega)-\xi_2(\omega))\,d\omega =
\sum\limits_{j=1}^{i} p_{k_j}(b_{k_j} - a_{k_j})\,.
\end{multline*}
Отсюда по индукции легко получить, что существует
последовательность чисел
$0\hm \le c_m \hm\le \ldots \hm\le c_1 \hm\le c_0 \hm= 1$ такая, что $c_i\hm\ge
x_{k_i}$ и 
$$
\int\limits_{[c_i;c_{i-1})\cap A}(\xi_1(\omega)-\xi_2(\omega))\,d\omega
=p_{k_i}(b_{k_i} - a_{k_i})\,.
$$

Обозначим $A_i \hm= [c_i;c_{i-1}) \cap A$. Заметим, что
$A_i\cap A_j \hm= \varnothing$ при $i\hm\neq j$ и каждое из множеств~$A_i$ 
можно представить в виде объединения конечного чис\-ла
непересекающихся полуинтервалов, на каждом из которых~$\xi_1$ и~$\xi_2$ постоянны.

Фиксируем произвольное $\eps\hm>0$. Обозначим $\eta_0 \hm\equiv \xi_1$. Поскольку
$$
\int\limits_{[c_i;c_{i-1})\cap A}(\eta_0(\omega)-(\xi_2(\omega)-\eps))\,d\omega>p_{k_i}( (b_{k_i} - \eps) - a_{k_i})\,,
$$
то согласно лемме~6 существуют последовательности операторов
диверсификации $D_{il_i},\ldots,D_{i1}$ (обозначим через~$D_i'$ их
суперпозицию), $i\hm=\overline{1,m}$, и случайные величины
$\eta_i\hm=D_i'(\eta_{i-1})$, $i\hm=\overline{1,m}$, такие, что
$\eta_i(\omega)\hm=b_{k_i}\hm-\eps\hm=\xi_2(\omega)\hm-\eps$ при
$\omega \hm\in [x_{k_i-1};x_{k_i})$,
$\eta_i(\omega)\hm\ge\xi_2(\omega)\hm-\eps$ при $\omega \hm\in A_i$,
$\eta_i(\omega)\hm=\eta_{i-1}(\omega)$ при остальных~$\omega$. Легко
видеть, что построена последовательность операторов диверсификации
такая, что $\eta_m\hm\ge\xi_2\hm-\eps$. По лемме~3 $\xi_2 \divers \xi_1$.

\smallskip

\noindent
\fbox{II} Теперь докажем теорему для ограниченных, монотонно
неубывающих и непрерывных справа~$\xi_1$ и~$\xi_2$. Пусть $M\hm>0$
таково, что $|\xi_1| \hm< M$, $|\xi_2|\hm<M$. Фиксируем произвольные
$\eps\hm>0$ и $n\hm\in\mathbb{N}$ такие, что $n\hm\ge 8M/\eps$.
Разобьем отрезок $[-M;M]$ на $n$~равных частей точками
$-M\hm=a_0\hm<a_1\hm<\ldots$\linebreak $\ldots<a_n\hm=M$. Введем новые случайные величины
$\xi_1'(\omega)\hm=a_i$, если $a_{i-1}\hm\le\xi_1(\omega)\hm<a_i$,
$\xi_2'(\omega)\hm=a_{i-1}$, если $a_{i-1}\hm\le\xi_2(\omega)\hm<a_i$.
$\xi_1'$, $\xi_2'$~--- простые монотонно неубывающие случайные величины,
причем $\xi_1'\hm\ge\xi_1$, $\xi_2'\hm\le\xi_2$. Поэтому
${ES}_\gamma(\xi_2')\hm\le{ES}_\gamma(\xi_2)\hm\le{ES}_\gamma(\xi_1)\hm
\le{ES}_\gamma(\xi_1')$
$\forall \gamma\hm\in(0;1]$. По построению~$\xi_1'$ и~$\xi_2'$
непрерывны справа. Следовательно, по первой части доказательства,
существует последовательность операторов диверсификации
$D_n,\ldots,D_1$ такая, что $D_n(\ldots(D_1(\xi_1'))\ldots) \hm\ge
\xi_2' \hm- \eps / 2$. В~силу того что $n\hm\ge 8M/\eps$, имеем:
$|\xi_1'\hm-\xi_1|\hm\le M/n \hm\le \eps/4$ и аналогично $|\xi_2'\hm-\xi_2|\hm\le
\eps/4$. Таким образом, $D_n(\ldots(D_1(\xi_1))\ldots) \hm\ge \xi_2 \hm- \eps $. 
По лемме~3 $\xi_2 \divers \xi_1$.

\smallskip

\noindent
\fbox{III} Докажем теорему в общем виде. Пусть $\xi_1$, $\xi_2$~---
почти наверное ограниченные случайные величины и
${ES}_\gamma(\xi_2)\hm\le{ES}_\gamma(\xi_1)\ \forall
\gamma\hm\in(0;1]$. Фиксируем произвольное $\eps\hm>0$. Введем
$\xi_i'(\omega) \hm\equiv \xi_i^{(\omega)}$, $i\hm=1,2$. $\xi_i'$~---
ограниченные монотонно неубывающие непрерывные справа случайные
величины. В~то же время $\xi_i'\hm\sim\xi_i$, поэтому
${ES}_\gamma(\xi_2') \hm= {ES}_\gamma(\xi_2) \hm\le {ES}_\gamma(\xi_1) \hm=
{ES}_\gamma(\xi_1')$. Согласно второму пункту доказательства
существует последовательность операторов диверсификации
$D_n,\ldots,D_1$ такая, что
$D_n(\ldots(D_1(\xi_1'))\ldots)\hm\ge\xi_2'\hm-\eps/2$. Так как
$\xi_i'\hm\sim\xi_i$, то по лемме~2 существуют
$T_0,T_{n+1} \hm\in \classT$ такие, что $|\xi_1'\hm-\xi_1(T_0)| \hm\le
\eps/4$ почти наверное и $|\xi_2'\hm-\xi_2(T_{n+1})| \hm\le \eps/4$ почти
наверное. Обозначим $D_0\hm=D_{0,T_0}$, $D_{n+1}\hm=D_{0,T_{n+1}}$. Тогда
$D_{n+1}(D_n(\ldots(D_0(\xi_1))\ldots)) \hm\ge \xi_2 \hm- \eps$ и по лемме~3 
$\xi_2 \divers \xi_1$. Теорема дока-\linebreak зана.\hfill$\square$

\smallskip

Последняя теорема дает возможность про любые два портфеля с
ограниченными возможными потерями и прибылями сказать, является ли
один из них результатом диверсификации другого. Заметим, что на
практике большинство финансовых инструментов предполагает
неограниченную возможность потерь.


Определим последовательности

\noindent
$$
x_n=(n+1)!\,;\quad y_n=(n+2)n!\,;\quad
\mu_n=\fr{\nu_n}{\sum\limits_{i=1}^\infty \nu_i}\,,
$$ 
где 

\noindent
$$
\nu_n=\fr{\delta^{n^2}}{(n+2)!}\,,\quad \delta\in (0;1)\,.
$$
Разобьем множество $\Omega\hm=[0;1)$ на полуинтервалы точками
$0\hm=a_1\hm<a_2\hm=b_1\hm<b_2\hm=a_3\hm<a_4\hm=b_3\hm<\ldots$, так что
$a_{2i}\hm-a_{2i-1}\hm=b_{2i}\hm-b_{2i-1}\hm=\mu_i/2$, $i\hm\ge1$. Пронумеровав
произвольным способом все рациональные числа из интервала (0;\,0,5),
получим последовательность $\hat{\alpha}_1,\ldots$ Определим теперь
$\xi(\omega)\hm=x_i$ при $\omega\hm\in[a_{2i-1};\,a_{2i})$,
                                 $\xi(\omega)\hm=y_i$ при $\omega\hm\in[b_{2i-1};b_{2i})$,
                                 $\xi'(\omega)\hm=x_i+\hat{\alpha}_i(y_i-x_i)$ при $\omega\hm\in[a_{2i-1};a_{2i})$,
                                 $\xi'(\omega)\hm=y_i-\hat{\alpha}_i(y_i-x_i)$ при $\omega\hm\in[b_{2i-1};b_{2i})$.
Можно показать, что ${ES}_\gamma(\xi') \hm\le {ES}_\gamma(\xi)$,
$\forall \gamma \hm\in(0;1]$, но в то же время~$\xi'$ не является
результатом диверсификации портфеля~$\xi$. Таким образом,
полностью отказаться от требования ограниченности портфелей в
теореме~3 нельзя.


\section{Диверсификация и~информационные риски}

Информационным риск~--- риск возникновения убытков из-за неправильной
организации или умышленного нарушения информационных потоков и
систем организации. Можно применить предложенное понятие
диверсификации к информационным рискам. Под информационной системой
предприятия будем понимать систему взаимосвязанных информационных
объектов, которые реализуют информационный процесс в целях
эффективного функционирования предприятия. Будем рассматривать ее
как набор неких компонент, подсистем, каждая из которых
характеризуется случайной величиной $\xi(\omega)$ ($\Expect|\xi|\hm<
\infty$), определенной на вероятностном пространстве
$(\Omega=[0;1),\sigmaF,\mu)$ и отражающей убытки, которые несет
подсистема на траектории $\omega \hm\in \Omega$. Под сложением двух
подсистем $\xi_1(\omega)\hm+\xi_2(\omega)$ понимается система,
состоящая из совокупности подсистем~$\xi_1$ и~$\xi_2$, а под
умножением подсистемы~$\xi$ на константу $a\hm\in [0;1]$ понимается
информационная система, использующая $a\cdot 100\%$ ресурсов
подсистемы~$\xi$. Информационную систему~$\xi_2$ будем называть
результатом диверсификации системы~$\xi_1$, если систему~$\xi_1$
можно <<пересобрать>> (согласно правилам, указанным в определении~1)
так, чтобы $\forall \eps \hm> 0$ \  $\eta_n \hm\ge \xi_2 \hm- \eps$ почти
наверное.

Преобразуем пример~1 для случая информационного риска. Пусть имеется
три вычислительных машины, характеризующихся независимыми одинаково
распределенными убытками $X_1$, $X_2$, $X_3$, и имеется возможность
построить систему вида $p_1 X_1 \hm+ p_2 X_2 \hm+ p_3 X_3$, $p_i \hm\ge 0$,
$p_1\hm+p_2\hm+p_3\hm=1$ (это можно интерпретировать как факт огра\-ни\-чен\-ности
суммарной мощности системы). Рассуждения, приведенные в примере~1,
показывают, что система $(1/3) X_1 + (1/3) X_2 + (1/3) X_3$
является результатом диверсификации любой системы указанного выше
вида.


{\small\frenchspacing
{%\baselineskip=10.8pt
\addcontentsline{toc}{section}{Литература}
\begin{thebibliography}{9}

%\bibitem{halmosh}
%\Au{Halmos P.\,R.} Lectures on ergodic theory.~---- N.Y.: Chelsea Publishing Company, 1956.

\label{end\stat}

\bibitem{es}
\Au{Acerbi C., Tasche D.} On the coherence of expected Shortfall~//
J.~Banking Finance, 2002. Vol.~26. No.\,7. P.~1487--1503.
%{\sf http://www-m4.ma.tum.de/pers/ tasche/shortfall.pdf}.
 \end{thebibliography}
}
}


\end{multicols}         %3
%\newcommand{\A}{{\mathbf A}}
%\newcommand{\B}{{\mathbf B}}
%\newcommand{\la}{{\lambda}}
%\newcommand{\be}{\begin{equation}}
%\newcommand{\ee}{\end{equation}}
%\newcommand{\ber}{\begin{eqnarray}}
%\newcommand{\eer}{\end{eqnarray}}

%\newcommand{\nin}{\noindent}
%\newcommand{\non}{\nonumber}
%\newcommand{\half}{\frac{1}{2}}
%\newcommand{\quarter}{\frac{1}{4}}

\def\stat{zeifman}

\def\tit{ОБ ОДНОМ КЛАССЕ МАРКОВСКИХ СИСТЕМ ОБСЛУЖИВАНИЯ$^*$}

\def\titkol{Об одном классе марковских систем обслуживания}

\def\autkol{Я.\,А.~Сатин, А.\,И.~Зейфман, А.\,В.~Коротышева, С.\,Я.~Шоргин}
\def\aut{Я.\,А.~Сатин$^1$, А.\,И.~Зейфман$^2$, А.\,В.~Коротышева$^3$, С.\,Я.~Шоргин$^4$}

\titel{\tit}{\aut}{\autkol}{\titkol}

{\renewcommand{\thefootnote}{\fnsymbol{footnote}}\footnotetext[1]
{Исследование поддержано РФФИ, гранты 11-07-00112-а и 11-01-12026-офи-м.}}


\renewcommand{\thefootnote}{\arabic{footnote}}
\footnotetext[1]{Вологодский государственный педагогический
университет, yacovi@mail.ru}
\footnotetext[2]{Вологодский государственный педагогический университет;  
Институт проблем информатики Российской академии наук; 
Институт социально-экономического развития территорий Российской академии наук,  a\_zeifman@mail.ru}
\footnotetext[3]{Вологодский государственный педагогический
университет,  a\_korotysheva@mail.ru}
\footnotetext[4]{Институт проблем информатики Российской академии наук, SShorgin@ipiran.ru}


\Abst{Рассматриваются модели обслуживания, описываемые конечными марковскими 
цепями с непрерывным временем. При этом предполагается,  что интенсивности 
поступления и обслуживания требований не зависят от числа требований в сис\-те\-ме. 
Получены оценки скорости сходимости и устойчивости различных характеристик таких сис\-тем.}

\KW{нестационарные марковские системы
обслуживания; скорость сходимости; устойчивость; оценки}

 \vskip 14pt plus 9pt minus 6pt

      \thispagestyle{headings}

      \begin{multicols}{2}
      
            \label{st\stat}

\section{Введение}

Классы систем массового обслуживания, описываемых процессами
рождения и гибели (стационарными и нестационарными, с катастрофами)
изучались начиная с 1970-х~гг.\ многими авторами
(см., например,~[1--6]). С~помощью методов,
разработанных одним из авторов настоящей \mbox{статьи}\linebreak (подробное изложение
этих методов приведено в~[7--9]), для таких сис\-тем
удалось получить точные оценки скорости сходимости и устойчивости.

Оказывается, этот же подход можно применить и к существенно более 
общему классу систем обслуживания.

Рассмотрим систему массового обслуживания, число требований в которой 
описывается нестационарной марковской цепью с непрерывным временем и 
конечным пространством состояний, причем требования могут поступать и 
обслуживаться группами.

Пусть $X=X(t)$, $t\geq 0$,~--- число требований в системе обслуживания ($0 \hm\le X(t) \hm\le r$).

Обозначим через 
\begin{gather*}
p_{ij}(s,t)=\mathrm{Pr}\left\{ X(t)=j\left| X(s)=i\right.
\right\}\,,\\
i,j \ge 0\,,\ 0\leq s\leq t\,,
\end{gather*}
переходные вероятности
процесса $X\hm=X(t)$, а через  $p_i(t)\hm=\mathrm{Pr}\left\{ X(t) \hm=i \right\}$~---
его вероятности состояний.

Будем предполагать, что интенсивности поступления и обслуживания $k$ требований в 
момент~$t$ в сис\-те\-ме об\-слу\-жи\-ва\-ния ($\lambda_{k}(t)$ и  $\mu_{k}(t)$ соответственно)  
не зависят от числа требований, находящихся в системе в момент~$t$, являются локально 
интегрируемыми на $[0,\infty)$ функциями времени~$t$ и, кроме того, 
$\lambda_{k+1}(t) \hm\le \lambda_{k}(t)$ и  $\mu_{k+1}(t) \hm\le \mu_{k}(t)$ при всех~$k$ 
и почти при всех $t \hm\ge 0$.

Тогда для описания вероятностной динамики процесса получаем прямую систему Колмогорова в виде
\begin{equation} 
\fr{d\vp}{dt}=A(t)\vp(t)\,,
\label{ur_1}
\end{equation}
 где
 {\footnotesize
\begin{multline*}
A(t)={}\\
{}=
\begin{pmatrix}
a_{00}(t) & \mu_1(t)  & \mu_2(t)   & \mu_3(t)  & \mu_4(t) & \cdots & \mu_r(t) \\
\la_1(t)   & a_{11}(t)  & \mu_1(t)  & \mu_2(t)   & \mu_3(t)  & \cdots & \mu_{r-1}(t) \\
\la_2(t)  & \la_1(t)    & a_{22}(t)& \mu_1(t)  & \mu2(t)    &  \cdots & \mu_{r-2}(t) \\
\cdots&\cdots&\cdots&\cdots&\cdots&\cdots&\cdots \\
\la_r(t) & \la_{r-1}(t) & \la_{r-2}(t) & \cdots & \la_2(t)  & \la_1 (t)   &  a_{rr}(t)
\end{pmatrix}\,,
\end{multline*}}
причем  
$$
a_{ii}(t)=-\sum\limits_{k=1}^{i}\mu_k(t) - \sum\limits_{k=1}^{r-i} \la_{r-k}(t)\,.
$$

Далее будем обозначать через $\|\bullet\|$  $l_1$-нор\-му, т.\,е.\ 
$\|{\vx}\|\hm=\sum|x_i|$, а $\|B\| \hm= \max\limits_j \sum\limits_i |b_{ij}|$, 
если $B \hm= (b_{ij})_{i,j=0}^{r}$.
%
Тогда, в частности, имеем 
$$
\|A(t)\| \le 2\sum\limits_{k=1}^{r}(\la_{k}(t)+ \mu_k(t))
$$ 
при  всех $t \hm\ge 0$.

Через 
$$
E(t,k) = E\left\{X(t)\left|X(0)\hm=k\right.\right\}
$$ 
будем далее обозначать математическое ожидание процесса (среднее число требований) в момент~$t$ 
при условии, что в нулевой момент времени он находится в состоянии~$k$, 
а через $E_{\bf p}(t)$ обозначим математическое ожидание процесса в момент~$t$ 
при начальном распределении вероятностей состояний $\mathbf{p}(0) \hm= \mathbf{p}$.

\section{Оценки скорости сходимости}

Рассмотрим вспомогательную последовательность положительных чисел $\{d_i\}$, $i\hm=1, \dots,r$.

Положим
\begin{equation*}
d=\min\limits_{1 \le i \le r} d_i\,; \enskip 
G=\sum\limits_{i=1}^r d_i\,; \enskip W=\min\limits_k \fr{d_k}{k}\,.
%\label{2.01}
\end{equation*}

Рассмотрим величины
\begin{multline*}
\alpha_i(t)= -a_{ii}(t)+\la_{r-i+1}(t)-\sum\limits_{k=1}^{i-1}(\mu_{i-k}(t)-{}\\
{}-
\mu_i(t))\fr{d_k}{d_i}-\sum\limits_{k=1}^{r-i}(\la_k(t)-\la_{i+r-1}(t))\fr{d_{k+i}}{d_i}\,,
%\label{2.02}
\end{multline*}

\noindent
\begin{equation*}
\alpha(t)=\min\limits_{1 \le i \le r}\alpha_i(t)\,.
%\label{2.03}
\end{equation*}

\smallskip

\noindent
\textbf{Теорема~1.} \textit{Пусть существует последовательность положительных 
чисел  $\{d_j\}$ такая, что}
\begin{equation}
\int\limits_0^{\infty} \alpha(t)\, dt = + \infty\,.
\label{2.031}
\end{equation}
\textit{Тогда $X(t)$ слабо эргодичен, при
любых начальных условиях} $\mathbf{p}^*(s)$, $\mathbf{p}^{**}(s)$ 
\textit{и любых $s$, $t$, $0\le s\le t$, справедлива оценка
\begin{equation} 
\label{2.04}
\|\vp^*(t)-\vp^{**}(t)\| \le \fr{8G}{d}\,e^{-\int\limits_s^t {\alpha(u)\,du}}\,.
\end{equation}
Кроме того,  $X(t)$ имеет предельное среднее $\phi(t)$ и при любых~$k$ и $t \hm\ge 0$ справедливо неравенство}:
\begin{equation}
\label{2.05}
|E(t,k)-\phi(t)|\le \fr{4G}{W}\,e^{-\int\limits_0^t {\alpha(u)\,du}}\,.
\end{equation}


\smallskip


\noindent
Д\,о\,к\,а\,з\,а\,т\,е\,л\,ь\,с\,т\,в\,о\,.\

Пользуясь предложенным в предыдущих работах способом, 
выразим 
$$
p_0=1-\sum\limits_{1\le i \le r}{p_i}\,.
$$

Тогда получим неоднородное уравнение:
\begin{equation} 
\label{ur_per}
\fr{d\vz}{dt}= B(t)\vz(t)+\vf(t)\,, 
%\label{2.06}
\end{equation}
\noindent
где $\vf(t)=\left(\la_1, \la_2,\cdots,\la_r \right)^{\mathrm{T}}$;

\end{multicols}


\hrule

\vspace*{6pt}

\begin{equation*}
B = \left(
\begin{array}{cccccccc}
a_{11}- \la_1   & \mu_1 - \la_1   & \mu_2 - \la_1   & \mu_3 -\la_1   & \cdots& \cdots & \mu_{r-1}- \la_1  \\
\la_1 -\la_2    & a_{22} -\la_2  & \mu_1-\la_2   & \mu_2 -\la_2     & \cdots&  \cdots & \mu_{r-2} -\la_2 \\
\la_2 -\la_3    & \la_1 -\la_3   & a_{33} -\la_3  & \mu_1-\la_2   & \cdots&  \cdots & \mu_{r-3} -\la_3 \\
\cdots&\cdots&\cdots&\cdots&\cdots&\cdots&\cdots \\
\la_{r-1} -\la_r  &\la_{r-2} -\la_r & \cdots & \cdots & \la_2 -\la_r   & \la_1 -\la_r     &  a_{rr} -\la_r
\end{array}
\right)\,.
%\label{2.07}
\end{equation*}

Рассмотрим треугольную матрицу
\begin{equation*}
D=\begin{pmatrix}
d_1   & d_1 & d_1 & \cdots & d_1 \\
0   & d_2  & d_2  &   \cdots & d_2 \\
\cdots&\cdots&\cdots&\cdots&\cdots \\
0  & 0 & \cdots & 0 &  d_r
\end{pmatrix}
%\label{2.08}
\end{equation*}
и соответствующую норму $\|{\bf z}\|_{D}\hm=\|D {\bf z}\|_1$.

Тогда имеем:
\begin{equation*}
 D BD^{-1}=\left(
\begin{array}{ccccccc}
a_{11}-\la_r  &  (\mu_1-\mu_2) \fr{d_1}{d_2}  & (\mu_2-\mu_3)\fr{d_1}{d_3}  & \cdots &  (\mu_{r-1}-\mu_r)\fr{d_1}{d_r} \\
(\la_1-\la_r) \fr{d_2}{d_1} &  a_{22}-\la_{r-1}  &(\mu_1-\mu_3)\fr{d_2}{d_3}  & \cdots &  (\mu_{r-2}-\mu_r)\fr{d_2}{d_r} \\
(\la_2-\la_r) \fr{d_3}{d_1} &  (\la_1-\la_{r-1})\fr{d_3}{d_2}   &a_{33}-\la_{r-2}   & \cdots &  (\mu_{r-3}-\mu_r)
\fr{d_3}{d_r}  \\
\cdots&\cdots&\cdots&\cdots&\cdots \\
(\la_{r-1} -\la_r) \fr{d_r}{d_1} & (\la_{r-2} -\la_{r-1}) \fr{d_r}{d_2}  & (\la_{r-3} -\la_{r-2}) \fr{d_r}{d_3}  & \cdots & a_{rr}-\la_1 \\
\end{array}
\right)\,.
%\label{2.09}
\end{equation*}


\begin{multicols}{2}


Далее, оценивая логарифмическую норму оператора~$B(t)$ (см., например, 
подробное рассмотрение в~[8--10]), получаем
\begin{multline*}
\gamma \left(B(t)\right)_{1D} = \gamma \left(DB(t)D^{-1}\right)_{1}={}\\
{}=
\max \left(\vphantom{\sum\limits_{k=1}^{i-1}}
a_{ii}(t) - \la_{r-i+1}(t) + \sum\limits_{k=1}^{i-1}\left(\mu_{i-k}(t)-{}\right.\right.\\
\left.\left.{}-\mu_i(t)\right)
\fr{d_k}{d_i} +
\sum\limits_{k=1}^{r-i}(\la_k(t)-\la_{i+r-1}(t))\fr{d_{k+i}}{d_i}\right) ={}\\
{}=
 - \min \alpha_i(t) = - \alpha(t)\,.
% \label{2.10}
\end{multline*}
Тогда\\[-7.9pt]
\begin{equation*}
\|\vz^*(t)-\vz^{**}(t)\|_{1D}\le  e^{-\int\limits_s^t {\alpha(u)du}}\|\vz^*(s)-\vz^{**}(s)\|_{1D}
%\label{2.11}
\end{equation*}
для всех $0 \le s \le t$ и любых начальных условий $\vz^*(s)$, $\vz^{**}(s)$.

Теперь, учитывая оценки для сравнения норм (см., например,~\cite{z08b}), получаем:
\begin{multline*}
\|\vp^*(t)-\vp^{**}(t)\| \le 2\|\vz^*(t)-\vz^{**}(t)\| \le{}\\
{}\le  \fr{4}{d}\|\vz^*(t)-\vz^{**}(t)\|_{1D}\le{} \\
{} \le \fr{4}{d}\,e^{-\int\limits_s^t {\alpha(u)\,du}}\|\vz^*(s)-\vz^{**}(s)\|_{1D} 
\le{}\\
{}\le
 \fr{4G}{d}\,e^{-\int\limits_s^t {\alpha(u)\,du}}\|\vz^*(s)-\vz^{**}(s)\| \le{} \\
{} \le  \fr{4G}{d}\,e^{-\int\limits_s^t {\alpha(u)\,du}}\|\vp^*(s)-\vp^{**}(s)\| \le 
\fr{8G}{d}\,e^{-\int\limits_s^t {\alpha(u)\,du}} 
%\label{2.11-a}
\end{multline*}
для любых начальных условий ${\bf p^*}(s)$, ${\bf p^{**}}(s)$ и любых $s,t$, $0\hm\le s\hm\le t$.

Из слабой эргодичности процесса с конечным пространством состояний 
вытекает существование предельного среднего, начальные условия для которого можно 
в общем случае выбрать произвольно.
Для оценки средних воспользуемся неравенством, приведенным в параграфе~2.3 из~\cite{z08b}:
\begin{multline*}
\|{\bf z}\|_{1D} = d_0 \left|\sum\limits_{i=1}^{\infty} p_i \right|
+ d_1 \left|\sum\limits_{i=2}^{\infty} p_i \right| + \dots \ge{}\\
{}\ge 
 W \sum\limits_{k \ge 1} k \left|\sum\limits_{i \ge k} p_i\right| \ge \fr{W}{2}
\sum\limits_{k \ge 1} k \left|p_k\right|\,.  
%\label{2.12}
\end{multline*}
Получаем теперь
\begin{multline*}
|E(t,k)-\phi(t)|\le \fr{2}{W}\,\|\vp^*(t)-\vp^{**}(t)\|_{1D}\le {} \\
{}\le\fr{2}{W}\,e^{-\int\limits_0^t {\alpha(u)\,du}}\|{\bf e}_k -
\vp^{**}(0)\|_{1D} \le \frac{4G}{W}e^{-\int\limits_0^t
{\alpha(u)\,du}}\,,
%\label{2.13}
\end{multline*}
что и требовалось доказать.
\columnbreak

%\smallskip

\noindent
\textbf{Замечание~1.} {Положим в условиях теоремы~1 
$$
\beta(t)=\max\limits_{1 \le i \le r}\alpha_i(t)\,.
$$ 
Тогда, пользуясь внедиагональной неотрицательностью матрицы $DB(t)D^{-1}$ 
с помощью методики, описанной в~\cite{z08b, z95b}, получаем справедливость неравенства

\noindent
\begin{equation*} 
%\label{2.14}
\|\vp^*(t)-\vp^{**}(t)\| \ge \fr{d}{8G}\,e^{-\int\limits_s^t {\beta(u)\,du}}
\end{equation*}
при любых $s$, $t$, $0\le s\le t$ и уже не при любых начальных условиях~${\bf p^*}(s)$, 
${\bf p^{**}}(s)$, а таких, что  $D\left({\bf p^*}(s) \hm-{\bf p^{**}}(s)\right) \hm\ge 0.$ 
Следовательно, оценки тео\-ре\-мы~1 будут заведомо иметь точный по времени порядок, если удастся 
выбрать вспомогательную последовательность $\{d_i\}$ так, что $\alpha(t)\hm=\beta(t)$, т.\,е.\ 
все $\alpha_i(t)$ одинаковы (не зависят от индекса~$i$)}.



\smallskip

Введем теперь в рассмотрение величины

\vspace*{-1pt}

\noindent
\begin{multline*}
\zeta_i(t)= -a_{ii}(t)+\la_{r-i+1}(t)+{}\\
{}+\sum\limits_{k=1}^{i-1}\left(\mu_{i-k}(t)-
\mu_i(t)\right) \fr{d_k}{d_i}+{}\\
{}+\sum\limits_{k=1}^{r-i}\left(\la_k(t)-\la_{i+r-1}(t)\right)\fr{d_{k+i}}{d_i}\,;
%\label{2.0211}
\end{multline*}
\begin{equation*}
\chi(t)=\max\limits_{1 \le i \le r}\zeta_i(t)\,.
%\label{2.0311}
\end{equation*}

\noindent
\textbf{Замечание 2.} {В условиях теоремы~1 при любых начальных условиях 
${\bf p^*}(s)$, ${\bf p^{**}}(s)$ и любых $s,t$,  $0\le s\le t$, 
справедлива следующая двухсторонняя оценка скорости сходимости:

\vspace*{-1pt}

\noindent
\begin{multline*} 
%\label{2.041}
\!\!\!\fr{d}{4G}\,e^{-\int\limits_s^t {\chi(u)\,du}}\|\vp^*(s)-\vp^{**}(s)\| \le
 \|\vp^*(t)-\vp^{**}(t)\| \le {}\\
 {}\le\fr{4G}{d}\,e^{-\int\limits_s^t {\alpha(u)\,du}}\|\vp^*(s)-\vp^{**}(s)\|.
\end{multline*}
Таким образом, можно оценить и сверху и снизу время  вхождения 
сис\-те\-мы обслуживания в предельный режим. Более подробно о получении 
нижних оценок см., например, в~\cite{z95b, gz05}.}

\smallskip

Рассмотрим два частных случая теоремы.

\smallskip

\noindent
\textbf{Следствие 1}. \textit{Пусть при выполнении остальных условий теоремы~1 
вместо}~(\ref{2.031}) \textit{выполняется условие $\alpha(t) \hm\ge \alpha \hm> 0$ 
почти при всех $t \hm\ge 0$. Тогда вместо}~(\ref{2.04}) \textit{и}~(\ref{2.05}) 
\textit{справедливы оценки}:

\vspace*{-1pt}

\noindent
\begin{align*} 
%\label{2.15}
\|\vp^*(t)-\vp^{**}(t)\| &\le \fr{8G}{d}\,e^{-\alpha \left(t-s\right)}\,;
\\
%\label{2.16}
|E(t,k)-\phi(t)|&\le \fr{4G}{W}\,e^{- \alpha t}\,.
\end{align*}

\pagebreak

%\smallskip

Положим 
\begin{gather*}
M_0=\max\limits_{|t-s|\le 1}\int\limits_s^t \alpha(u)\,du;\\
\alpha^* = \int\limits_0^1 \alpha(t)\, dt\,; \quad
M=e^{M_0+\alpha^*}\,.
\end{gather*}
С учетом неравенства 
$$
e^{-\int\limits_s^t {\alpha(u)\,du}} \hm\le M e^{-\alpha^* (t-s)}
$$ 
получаем следующее утверждение.

\smallskip

\noindent
\textbf{Следствие~2.} \textit{Пусть все $\lambda_k(t)$ и $\mu_k(t)$ 1-пе\-ри\-одич\-ны,  
а при выполнении остальных условий теоремы~1 вместо}~(\ref{2.031}) 
\textit{выполняется условие  $\alpha^* \hm> 0$.  Тогда предельный режим (скажем, $\vp^*(t)$) 
и соответствующее ему предельное среднее $\phi^*(t)$ можно выбрать 
1-пе\-ри\-оди\-че\-ски\-ми, а вместо}~(\ref{2.04}) \textit{и}~(\ref{2.05}) 
\textit{справедливы оценки}:
\begin{equation*} 
%\label{2.17}
\|\vp(t) - \vp^*(t)\| \le \fr{8GM}{d}\,e^{-\alpha^*t}
\end{equation*}
\textit{и, кроме того,}
\begin{equation*}
|E(t,k)-\phi^*(t)|\le \fr{4GM}{W}\,e^{-\alpha^*t}
%\label{2.18}
\end{equation*}
\textit{при любом $k$ и $t \ge 0$}.



\section{Устойчивость}

Рассмотрим также <<возмущенный>> процесс обслуживания $\bar{X}\hm=\bar{X}(t)$, $t\hm\geq 0$, 
в котором интенсивности поступления и обслуживания требований также не зависят от чис\-ла 
требований в системе, обозначая его соответствующие характеристики теми же буквами с 
чертой сверху. Для прос\-то\-ты записи оценок будем предполагать, что возмущения 
<<равномерно малы>>, т.\,е.\ выполняется неравенство $\| A(t)-\bar{A}(t)\| \hm\le \varepsilon$. 
Первые результаты для нестационарных цепей с непрерывным временем получены в~\cite{z85}, 
а детальное рассмотрение для более общего случая неравномерных оценок можно без труда 
провести так же, как это сделано в~\cite{z98, ae}. Для получения требуемых равномерных 
оценок устойчивости необходима экспоненциальная эргодичность соответствующего процесса, 
т.\,е.\ существование положительных констант $N$, $a$ таких, что  для правой части~(\ref{2.04}) 
справедливо неравенство:
\begin{equation}
e^{-\int\limits_s^t {\alpha(u)\,du}} \le Ne^{-a\left(t-s\right)}\,.
\label{3.01}
\end{equation}
Оценка~(\ref{3.01}) заведомо имеет место, в частности, если выполнены условия одного из следствий 
предыду\-ще\-го параграфа.

\smallskip

\noindent
\textbf{Теорема~2.}
\textit{Пусть выполнены условия теоремы~1 и}~(\ref{3.01}). \textit{Тогда при
 любых начальных условиях ${\bf p}(s)$ и ${\bar{\bf p}}(s)$ для процессов~$X(t)$ 
 и $\bar{X}(t)$ соответственно справедливы следующие оценки устойчивости:}
\begin{align*} 
%\label{3.02}
\limsup_{t \to \infty}  \|{\bf p}(t)- \bar{\bf p}(t)\| &\le
\fr{\varepsilon(1+\ln(4GN/d))}{a}\,;
\\
% \label{3.03}
\limsup\limits_{t \to \infty}   |E_{\bf p}(t)- \bar{E}_{\bar{\bf p}(t)}|&\le 
\fr{r \varepsilon(1+\ln(4GN/d))}{a}\,.
\end{align*}


\smallskip

\noindent
Д\,о\,к\,а\,з\,а\,т\,е\,л\,ь\,с\,т\,в\,о\ основано на подходе, 
введенном для стационарных процессов в~\cite{mit03} и описанном для нестационарной 
ситуации в~\cite{z11}.
Если  при любых начальных условиях для исходного процесса справедлива оценка
\begin{equation*} 
%\label{3.04}
\|\vp(t) - \vp^*(t)\| \le ce^{-b\left(t-s\right)}\,,
\end{equation*}
то, полагая
\begin{multline*}
\beta (t, s)=\sup\limits_{ \| {\bf v} \| =1, \sum {v_i}=0}
{\|V(t,s){\bf v}(t,s)\|} ={}\\
{}= \fr{1}{2} \max_{i,j} \sum\limits_k {|p_{ik}(t,
s)-p_{jk}(t, s)|}\,, 
\end{multline*}
где $V(t, s)$~--- матрица Коши
уравнения~(\ref{ur_1}), получаем в итоге следующее неравенство:
\begin{equation*}
\|{\bf p}(t)-\bar{\bf p}(t)\| \le{}
\begin{cases}
\|{\bf p}(s)-{\bf \bar{p}}(s)\|+ (t-s)\varepsilon \,, &\\
&\hspace*{-35mm} 0<t< b^{-1} \ln \left(\fr{c}{2}\right)\,; \\
b^{-1}\left(\ln \fr{c}{2} +1-\fr{c}{2}\,e^{-b(t-s)}\right)\varepsilon +{}&\\
{}+
\fr{c}{2}\,e^{-b(t-s)} \|{\bf p}(s)-{\bf \bar{p}}(s)\|\,, &\\
&\hspace*{-30mm}t\ge b^{-1}\ln \left(\fr{c}{2}\right)
\end{cases}
%\label{3.05}
\end{equation*}
для любых начальных условий ${\bf p}(s)$ и $\bar{\bf p}(s)$.
Из неравенств~(\ref{2.04}) и~(\ref{3.01}) вытекает, что $b=a$, $c={8GN}/{d}$.  
Устремив $t \hm\to \infty$ и взяв $s\hm=0$, получаем требуемые оценки.


\smallskip

\noindent
\textbf{Замечание~3.} 
В полученную оценку устойчивости для математического ожидания процесса 
в качестве множителя входит размерность~$r$, поэтому иногда лучший результат 
удается получить при помощи другого подхода, описанного в работе~\cite{z11}.

\smallskip

Положим 
$$
S=\max\limits_{{1 \le i, j \le r}} \fr{d_i}{d_j}\,,
$$ 
и пусть числа $K, L$ таковы, что 

\noindent
$$
d_1\la_1(t) + (d_1+d_2)\la_2(t) + \dots + 
\left(\sum\limits_{1 \le i \le r}d_i\right) \la_r(t) \le K\,,
$$ 
а 

\noindent
\begin{multline*}
d_1(\la_1(t)-\bar{\la}_1(t)) + (d_1+d_2)(\la_2(t)-\bar{\la}_2(t)) + \dots\\
\dots + 
\left(\sum\limits_{1 \le i \le r}d_i\right) (\la_r(t)-\bar{\la}_r(t)) \le 
L\varepsilon
\end{multline*} 
почти при всех $t \ge 0.$

\smallskip

\noindent
\textbf{Теорема~3.}
\textit{Пусть  выполнены условия теоремы~2 и, кроме того, при всех~$k$ 
и почти всех $t \hm\ge 0$ $\la_k(t) \hm< \infty$. Тогда при любых начальных условиях 
${\bf p}(s)$ и ${\bar{\bf p}}(s)$ для процессов $X(t)$ и $\bar{X}(t)$ 
соответственно справедливо неравенство}

\noindent
\begin{equation*}
\limsup\limits_{t \to \infty}   |E_{\bf p}(t)- \bar{E}_{\bar{\bf p}(t)}|\le 
\fr{ N\varepsilon\left(L a+ 2KNS\right)}{W a \left(a-2\varepsilon S\right)}\,.
\end{equation*}


\smallskip

\noindent
Д\,о\,к\,а\,з\,а\,т\,е\,л\,ь\,с\,т\,в\,о.\
 Перепишем исходную систему~(\ref{ur_per}) для невозмущенного процесса в следующем виде:
 \noindent
 
\begin{equation*}
\fr{d\vp}{dt}=\bar{B}(t)\vp(t) + {\bf f}(t)+\left(B(t)-\bar{B}(t)\right)\vp(t)\,.
%\label{eq112-n}
\end{equation*}
Тогда

\noindent
\begin{multline*}
\vp(t)=\bar{U}(t,0)\vp(0)+\int\limits_0^t \bar{U}(t,\tau){\bf{f}}(\tau) \, d\tau+{}\\
{}+\int\limits_0^t \bar{U}(t,\tau) \left(B(\tau)-\bar{B}(\tau)\right)\vp(\tau)\, d\tau\,;
\end{multline*}

\vspace*{-9pt}

\begin{equation*}
\hspace*{-15mm}\bar{\vp}(t)=\bar{U}(t,0)\bar{\vp}(0)+\int\limits_0^t \bar{U}(t,\tau){\bf{f}}(\tau) \, d\tau,
\end{equation*}
где $U(t,s)$~--- матрица Коши для уравнения~(\ref{ur_per}).
В любой норме при одинаковых начальных условиях получаем следующую оценку:
%\noindent
\begin{multline}
 \label{3000}
\!\!\!\!\!\!\left\|\vp(t)-\bar{\vp}(t)\right\|\le \!\!\int\limits_0^t \!\!\|\bar{U}(t,\tau)\|
\left(\| B(\tau)-\bar{B}(\tau)\| \|\vp(\tau)\| +\right.\\
\left.{}+ \| \vf(\tau)-\bar{\vf}(\tau)\|\right)\,d\tau\,.\!
\end{multline}
Имеем почти при всех $t \ge 0$:
\begin{equation*}
\|B(t)-\bar{B}(t)\|_{1D}=\|D(B(t)-\bar{B}(t))D^{-1}\| \le 2S\varepsilon\,;
%\label{3002}
\end{equation*}
%
%\vspace*{-14pt}
%
%\noindent
\begin{multline*}
\|{\bf f}(t)\|_{1D} \le d_1\la_1(t) + (d_1+d_2)\la_2(t) + \dots + {}\\
{}+
\left(\sum\limits_{1 \le i \le r}d_i\right) \la_r(t) \le K\,, 
\quad \|\vf(\tau)-\bar{\vf}(\tau)\|_{1D} \le L\varepsilon\,.
%\label{3002-a}
\end{multline*}
А тогда
\begin{multline*}
\gamma(\bar{B}(t))_{1D} \le \gamma(DB(t)D^{-1})+\|B(t)-\bar{B}(t)\|_{1D} \le  {}\\
{}\le -
\alpha(t)+2S \varepsilon \,.
% \label{3003}
\end{multline*}

Оценим теперь
\begin{multline*} 
%\label{8402}
\!\|{\bf p}(t)\|_{1D} \le
\|U(t){\bf p}(0) \|_{1D} +
 \int\limits_0^t \!\!\| U(t,\tau){\bf f}(\tau)\, d\tau \|_{1D} \le {}\\
 {}\le
 N e^{-a t} \| \vp(0)\|_{1D}  + \fr{K N}{a}.
\end{multline*}

 Тогда с учетом~(\ref{3000}) получаем:
\begin{multline*} 
%\label{3004}
\left\|\vp(t)-\bar{\vp}(t)\right\|_{1D}\le N\int\limits_0^t e^{-(a - 2\varepsilon S)(t-\tau)}\times{}\\
{}\times
\left(2S\varepsilon (N e^{-a \tau} \| \vp(0)\|_{1D}  + \fr{K N}{a}) +  L\varepsilon \right)\, d\tau  \le {} \\
{}\le  o(1)+\fr{ N\varepsilon(L+{2KNS}/{a})}{a-2\varepsilon S}\,. 
\end{multline*}

\vspace*{-9pt}

\section{Примеры}

\noindent
\textbf{Пример 1.}

Рассмотрим исходный процесс обслуживания с интенсивностями 
$\la_1(t)\hm=\la_2(t)\hm=\la_3(t)\hm=\la(t) \hm= 3\hm+\sin{2\pi t}$, 
$\mu_1(t)\hm=\mu_2(t)\hm=  \mu(t) \hm= 2\hm+\cos{2\pi t}$, 
$\la_4(t)=\ldots=\la_r(t)\hm=\mu_3(t)=\ldots=\mu_r(t)\hm=0$. Выберем последовательность  
$d_k\hm=h^k$, где $0{,}82 \hm< h \hm<1$. Тогда имеем
$$
d=h^r\,; \quad G \le \fr{h}{1-h}\,; \quad W=\fr{h^r}{r}\,.
$$

Будем предполагать, что возмущенный процесс имеет такую же структуру 
мат\-ри\-цы интенсивностей, причем $|\la(t)\hm-\bar{\la}(t)| \hm\le \varepsilon$ 
и  $|\mu(t)\hm-\bar{\mu}(t)| \hm\le \varepsilon$ почти при всех $t \hm\ge 0$. 
Отметим кстати, что при этом $\| A(t)\hm-\bar{A}(t)\| \hm\le 10 \varepsilon$ почти при 
всех $t \hm\ge 0$. Рассмотрим дальнейшие оценки:
$$
S=\fr{1}{h^2}\,; \ K=4 \left(3h+2h^2+h^3\right)\,; \ L=3h+2h^2+h^3\,;
$$
$$
\alpha(t) \ge \la(t)\left(3 - h - h^2 -h^3\right)-\mu(t)\left(\fr{1}{h^2}+\fr{1}{h}-2\right)\,;
$$
$$
\alpha^*= 3\left(3 - h - h^2 -h^3\right)-2\left(\fr{1}{h^2}+\fr{1}{h}-2\right)\,;
$$


\noindent
\begin{multline*}
M_0 \le \int\limits_0^1 |\alpha(t)|\, dt \le 4\left(3 - h - h^2 -h^3\right)+{}\\
{}+
3\left(\fr{1}{h^2}+\fr{1}{h}-2\right)\,;
\end{multline*}

\vspace*{-9pt}

\noindent
$$
M=e^{\alpha^*+M_0}\,.
$$

Если, например, взять 
$h\hm=0{,}9$, то $\alpha^*\hm=0{,}992$, $M_0\hm=3{,}281$, $M\hm=71{,}737$.

Тогда получаем следующие оценки.

По следствию~2
\begin{align*}
 \|{\bf p}(t)- {\bf p^{*}}(t)\| &\le \fr{8Me^{-\alpha^*t}}{h^{r-1}(1-h)}\,;\\
|E_{\bf p}(t)-\phi^*(t)| &\le  \fr{4Mre^{-\alpha^*t}}{h^{r-1}(1-h)}\,.
\end{align*}

По теореме~2 ($N=M$, $a=\alpha^*$) с использованием оценок следствия~2
\begin{align*}
\limsup\limits_{t \to \infty} \|{\bf p}(t)- \bar{\bf p}(t)\| &\le{} \notag\\
&\hspace*{-15mm}{}\le \fr{\varepsilon(1+\ln({4M}/({h^{r-1}(1-h)})))}{\alpha^*}\,;\\
\limsup\limits_{t \to \infty}   |E_{\bf p}(t)- \bar{E}_{\bar{\bf p}(t)}| &\le \notag\\
&\hspace*{-15mm}{}\le\fr{r\varepsilon(1+\ln(4M/(h^{r-1}(1-h))))}{\alpha^*}\,.
\end{align*}

По теореме~3 с использованием оценок следствия~2
\begin{multline*}
\limsup\limits_{t \to \infty}   |E_{\bf p}(t)- \bar{E}_{\bar{\bf p}(t)}| \le {}\\
{}\le
\fr{rM\varepsilon(3h+2h^2+h^3)(\alpha^* h^2+8M)}{h^r\alpha^*(\alpha^* h^2-2\varepsilon)}\,.
\end{multline*}

\noindent

\textbf{Пример 2.}

Рассмотрим процесс с интенсивностями 
$\la_1(t)\hm=\la_2(t)\hm=\ldots=\la_r(t) \hm= \la(t) \hm= 3\hm+\sin{2\pi t}$; 
$\mu_1(t)\hm=\mu_2(t)\hm= \mu(t) \hm= 2+\cos{2\pi t}$;
$\mu_3(t)=\ldots=\mu_r(t)=0$.

Будем предполагать, что возмущенный процесс имеет такую же структуру 
мат\-ри\-цы интен\-сив\-ностей, причем $|\la(t)-\bar{\la}(t)| \hm\le \varepsilon$ и  
$|\mu(t)-\bar{\mu}(t)| \hm\le \varepsilon$ почти при всех $t \hm\ge 0$. 
При этом будем иметь $\| A(t)\hm-\bar{A}(t)\| \hm\le 2r \varepsilon$ почти при всех $t \hm\ge 0$.

Выберем последовательность $d_k\hm=1$. Тогда  
\begin{gather*}
d=1\,; \enskip G=r\,; \enskip W=\fr{1}{r}\,; \enskip S=1\,; \\
K=\fr{4r(1+r)}{2}\,; \quad L=\fr{r(1+r)}{2}\,;
\\
\alpha(t)=\la(t)\,; \ \alpha=2\,; \ \alpha^*=3\,; M_0 \le 4\,; \ M \le  e^{7}\,.
\end{gather*}

И получаем следующие оценки.

\columnbreak

По следствию~1
\begin{align*}
 \|{\bf p^*}(t)- {\bf p^{**}}(t)\| &\le 8re^{-2t}\,;\\
|E_{\bf p}(t)- \phi(t)|&\le  4r^2 e^{-2t}\,.
\end{align*}

По следствию~2
\begin{align*}
\|{\bf p}(t)- {\bf p^{*}}(t)\| &\le 8re^{7-3t}\,;
\\[6pt]
|E_{\bf p}(t)- \phi^*(t)| &\le 4r^2 e^{7-3t}\,.
\end{align*}

По теореме~2 ($N=1$, $a=\alpha$) с учетом оценок следствия~1
\begin{align*}
\limsup\limits_{t \to \infty} \|{\bf p}(t)- \bar{\bf p}(t)\| &\le 
\fr{\varepsilon(1+\ln{4r})}{2}\,;
\\[6pt]
\limsup\limits_{t \to \infty}   |E_{\bf p}(t)- \bar{E}_{\bar{\bf p}(t)}|
&\le \fr{r\varepsilon(1+\ln{4r})}{2}\,.
\end{align*}

По теореме~2 ($N=M$, $a=\alpha^*$) с учетом оценок следствия~2
\begin{align*}
\limsup\limits_{t \to \infty} \|{\bf p}(t)- \bar{\bf p}(t)\| &\le 
\fr{\varepsilon(8+\ln{4r})}{3}\,;
\\
\limsup\limits_{t \to \infty}   \left|E_{\bf p}(t)- \bar{E}_{\bar{\bf p}(t)}\right| &\le 
\fr{r\varepsilon(8 + \ln{4r})}{3}\,.
\end{align*}

По теореме~3 с учетом оценок следствия~1
\begin{equation*}
\limsup\limits_{t \to \infty}   \left|E_{\bf p}(t)- \bar{E}_{\bar{\bf p}(t)}\right| \le 
\fr{5 \varepsilon r^2 (1+r)}{4(1- \varepsilon)}\,.
\end{equation*}

По теореме~3 с учетом оценок следствия~2
\begin{equation*}
\limsup\limits_{t \to \infty}   \left|E_{\bf p}(t)- \bar{E}_{\bar{\bf p}(t)}\right| \le 
\fr{\varepsilon e^{7} r^2 (1+r) (3+8e^{7})}{6(3-2\varepsilon)}\,.
\end{equation*}

{\small\frenchspacing
{%\baselineskip=10.8pt
\addcontentsline{toc}{section}{Литература}
\begin{thebibliography}{99}

 \bibitem{b} %1
\Au{Баруча-Рид~А.\,Т.} Элементы теории марковских процессов и их
приложения.~--- М.: Наука, 1969.

\bibitem{gm}  %2
\Au{Гнеденко~Б.\,В., Макаров~И.\,П.} Свойства решений задачи с потерями
в случае периодических интенсивностей~// Дифф. уравнения, 1971.
Вып.~7. С.~1696--1698.

\bibitem{g1}   %3
\Au{Gnedenko~D.\,B.} On a generalization of Erlang formulae~// 
Zastosow. Mat., 1971. Vol.~12. P.~239--242.

\bibitem{S}  %4
\Au{Саати~Т.\,Л.} Элементы теории массового обслуживания
 и ее приложения.~--- М.: Сов. радио, 1971.

\bibitem{g}  %5
\Au{Gnedenko~B., Soloviev~A.} On the conditions of the
existence of final probabilities for a Markov process~// Math.
Operations. Stat., 1973. P.~379--390.

\bibitem{gk} %6
\Au{Гнеденко~Б.\,В., Коваленко~И.\,Н.} Введение в теорию массового
обслуживания.~--- М.: Наука, 1987.
\pagebreak

\bibitem{gz00}   %7
\Au{Granovsky~B.\,L., Zeifman~A.\,I.}  The N-limit of spectral gap of 
a class of birth-death Markov chains~//
 Appl. Stoch. Models Business Ind., 2000. Vol.~16. P.~235--248.

\bibitem{z08b}  %8
\Au{Зейфман~А.\,И., Бенинг~В.\,Е., Соколов~И.\,А.} 
Марковские цепи и модели с непрерывным временем.~--- М.: Элекс-КМ, 2008.

\bibitem{dzp} %9
\Au{Van Doorn~E.\,A., Zeifman~A.\,I., Panfilova~T.\,L.}  
Bounds and asymptotics for the rate of convergence of birth-death processes~//  
Th. Prob. Appl., 2010. Vol.~54. P.~97--113.

\bibitem{z95b}   %10
\Au{Zeifman~A.\,I.} Upper and lower bounds on the rate of
convergence for nonhomogeneous birth and death processes~//  Stoch.
Proc. Appl., 1995. Vol.~59. P.~157--173.

\bibitem{gz05}  %11
\Au{Granovsky~B.\,L., Zeifman~A.\,I.} On the lower bound of the spectrum
 of some mean-field models~// Theory Prob. Appl., 2005. Vol.~49. P.~148--155.
 
\bibitem{z85}  %12
\Au{Zeifman~A.\,I.} Stability for contionuous-time
nonhomogeneous Markov chains~// Lect. Notes Math.,  1985. Vol.~1155.
P.~401--414.

\bibitem{z98} %13
\Au{Zeifman~A.} Stability of birth and death processes~// 
J.~Math. Sci., 1998. Vol.~91. P.~3023--3031.

\bibitem{ae} %14
\Au{Андреев~Д., Елесин~М., Кузнецов~А., Крылов~Е., Зейфман~А.}
Эргодичность и устойчивость нестационарных систем обслуживания~//
Теория вероятностей и математическая статистика, 2003. Т.~68.
С.~1--11.

\bibitem{mit03} %15
\Au{Mitrophanov~A.\,Yu.} Stability and exponential convergence of continuous-time 
Markov chains~//  J. Appl. Prob., 2003. Vol.~40. P.~970--979.

\label{end\stat} 

\bibitem{z11} %16
\Au{Зейфман~А.\,И., Коротышева~А.\,В., Панфилова~Т.\,Л., Шоргин~С.\,Я.} 
Оценки устойчивости  для некоторых систем обслуживания с катастрофами~//  
Информатика и её применения, 2011. Т.~5. Вып.~3. С.~27--33.
 \end{thebibliography}
}
}


\end{multicols}         %4
\def\stat{leri}

\def\tit{СРЕДНЕЕ РАССТОЯНИЕ В~КОНФИГУРАЦИОННЫХ ГРАФАХ СО~СТЕПЕННЫМ РАСПРЕДЕЛЕНИЕМ$^*$}

\def\titkol{Среднее расстояние в~конфигурационных графах со~степенным распределением}

\def\aut{М.\,М.~Лери$^1$}

\def\autkol{М.\,М.~Лери}

\titel{\tit}{\aut}{\autkol}{\titkol}

\index{Лери М.\,М.}
\index{Leri M.\,M.}


{\renewcommand{\thefootnote}{\fnsymbol{footnote}} \footnotetext[1]
{Финансовое обеспечение исследований осуществлялось из средств федерального
бюджета на выполнение государственного задания Карельского научного центра Российской академии наук
(Институт прикладных математических исследований КарНЦ РАН).}}


\renewcommand{\thefootnote}{\arabic{footnote}}
\footnotetext[1]{Институт прикладных математических исследований Карельского научного центра
Российской академии наук, \mbox{leri@krc.karelia.ru}}

%\vspace*{-2pt}








\Abst{В случайных конфигурационных графах с~дискретным степенным распределением степеней вершин
с фиксированным параметром рассматривается среднее расстояние в~графе, которое вычисляется
как среднее арифметическое расстояний между всеми парами вершин графа.
Эта характеристика оценивается с~по\-мощью методов имитационного моделирования. В~силу вычислительных
ограничений рассматриваются графы в~доасимптотической области (в~настоящей работе это графы объемом
до 7000~вершин). По\-стро\-ены модели зависимостей сред\-не\-го рас\-сто\-яния от объема графа и~па\-ра\-мет\-ра распределения степеней вершин.
Проведено сравнение полученных результатов с~результатами тео\-ре\-ти\-че\-ских исследований типичного расстояния
в графе в~асимп\-то\-ти\-ке (т.\,е.\ когда число вершин графа стремится к~бес\-ко\-неч\-ности), приведенными в~работах
Р.~Хофстада.}

\KW{конфигурационные графы; степенное распределение;
сред\-нее рас\-сто\-яние в~графе; имитационное моделирование}

\DOI{10.14357/19922264230104} 
  
\vspace*{-6pt}


\vskip 10pt plus 9pt minus 6pt

\thispagestyle{headings}

\begin{multicols}{2}

\label{st\stat}

\section{Введение}

\vspace*{-1pt}

Изучение структуры и~функционирования сложных сетей продолжает оставаться одним из важных
направлений исследований в~науке и~технике~\cite{Dur,Hof1}. Примерами таких сетей, окружающих
нас в~повседневной жизни, служат интернет, электрические и~телекоммуникационные сети, сети
социальных отношений, соавторства и~цитирования и~др.
Их быст\-рое и~динамичное развитие и~нарастающая популярность легли в~основу многих фундаментальных
исследований в~об\-ласти топологии таких сетей (см., например,~\cite{Dur,Hof1,Hof2,New1,New2}).
В~качестве моделей слож\-ных сетей широко используются случайные графы, причем их разнообразие
касается как определения степеней вершин графа, так и~уста\-нов\-ле\-ния связей между этими вершинами.
В~частности, было показано (см., например,~\cite{Fa,RN}), что модели случайных графов с~независимыми
одинаково распределенными степенями вершин с~общим дискретным законом распределения подходят для
моделирования сети Интернет в~случае, когда в~качестве узлов сети рассматриваются автономные системы.

Увеличение размеров сетей и~изменчивость сетевой структуры дают понять, что для адекватного отражения
их топологии и~функционирования в~ходе по\-стро\-ения их математических моделей необходимо учитывать не
только распределение степеней вершин в~со\-от\-вет\-ст\-ву\-ющей модели случайного графа, но также
принимать во внимание и~другие не менее важ\-ные характеристики исследуемых сетей~\cite{Hof1, New1}.
В связи с~этим различные структурные характеристики слож\-ных сетей пред\-став\-ля\-ют определенный интерес
как при моделировании их топологии, так и~при изучении динамических процессов, происходящих в~таких сетях
по мере их рос\-та или под внешними воздействиями. Одна из таких характеристик~---
рас\-сто\-яние между двумя произвольными вершинами графа~\cite{Dur, Hof2, New1}.

 Мож\-но рас\-смат\-ри\-вать различные
виды расстояний в~графе: расстояние между двумя заданными вершинами, рас\-сто\-яние между произвольно\linebreak
выбранными вершинами, все расстояния между каж\-дой парой вершин, наименьшее воз\-мож\-ное {рас\-сто\-яние},
наибольшее, или диаметр графа, и~т.\,д. 

В~на\-сто\-ящей работе рассматривается среднее расстояние в~графе,
которое вы\-чис\-ля\-ет\-ся как среднее арифметическое расстояний между всеми парами вершин графа.
Цель работы со\-сто\-яла в~на\-хож\-де\-нии зависимостей среднего расстояния в~графе от числа его вершин и~па\-ра\-мет\-ра 
распределения степеней вершин, а~также в~сравнении результатов работы с~результатами тео\-ре\-ти\-че\-ских
исследований рас\-сто\-яния в~графе, приведенными в~\cite{Hof2}.
Исследование проводилось по\-средст\-вом методов имитационного моделирования с~по\-сле\-ду\-ющей статистической
обработкой данных с~помощью программного обеспечения Statistica 10 и~Wolfram Mathematica 9.0.

\section{Описание модели}

Рассматривается случайный граф, со\-сто\-ящий из~$N$~вершин. Через $\xi_1,\xi_2,\ldots,\xi_N$ обозначим
степени вершин, которые являются независимыми одинаково распределенными случайными величинами
со сле\-ду\-ющим дискретным степенным распределением~\cite{RN}:
\begin{equation}
\label{eq1}
{\bf P}\{\xi = k\} = k^{-\tau} - (k+1)^{-\tau}, \quad k=1,2,\ldots,
\end{equation}
с фиксированным параметром~$\tau\hm>1$. Легко показать, что математическое ожидание
распределения~(\ref{eq1}) рав\-но ${\bf E}\xi\hm=\zeta(\tau)$, где $\zeta(\tau)$~--- значение дзе\-та-функ\-ции
Римана в~точке~$\tau$. Что касается дисперсии, то при $\tau\hm>2$ она конечна, а при
$\tau\hm\in(1,2]$~-- бесконечна.
В~работе рассматриваются конфигурационные графы~\cite{Bol}, построение которых происходит сле\-ду\-ющим
образом.
Для каждой из $N$ вершин графа задаются степени в~соответствии  с~распределением~(\ref{eq1}) с~выбранным
значением па\-ра\-мет\-ра~$\tau$. Степени определяют чис\-ло различимых полуребер~\cite{RN} (под полуребром понимают
ребро, инцидентное данной вершине графа, для которого смежная вершина еще не определена), занумерованных в~произвольном порядке. 
Для формирования ребер все полуребра попарно и~равновероятно соединяют между собой.
Сумма степеней вершин рас\-смат\-ри\-ва\-емо\-го графа является случайной величиной. Очевидно, что она должна быть чет\-ной,
поэтому, если это не так, для построения недостающего ребра к~равновероятно выбранной
вершине добавляется одно полуребро, увеличивая степень этой вершины на~1. Граф, по\-стро\-ен\-ный таким
образом, может иметь пет\-ли, цик\-лы и~кратные \mbox{ребра}.

Известно (см., например,~\cite{Dur,Hof1,RN}), что степенной конфигурационный граф, значение па\-ра\-мет\-ра
распределения степеней вершин которого $\tau\hm>1$, асимптотически почти наверное содержит больше одной
компоненты связности, причем при $\tau\hm\in(1,2)$ в~таком графе существует, и~она единственна, так
называемая гигантская компонента связ\-ности, чис\-ло вершин в~которой пропорционально~$N$
при $N\hm\rightarrow\infty$, а~объем любой другой компоненты такого графа бесконечно мал по
сравнению с~объемом гигантской компоненты.


\vspace*{-6pt}

\section{Среднее расстояние в~графе}

\vspace*{-2pt}

Расстояния между узлами сложной сети служат важными числовыми характеристиками сетевой
топологии (см., например,~\cite{Hof2, Chu}).
Пусть $G \hm= (V,E)$~--- неориентированный граф, в~котором~$V$~--- множество вершин, а~$E$~--- множество ребер.
Обозначим через~$l(v,u)$ чис\-ло ребер простой цепи, со\-еди\-ня\-ющей вершины~$v$ и~$u$ графа~$G$
($v,u\hm\in V$ и~$v\hm\neq u$). Если вершины~$v$ и~$u$ принадлежат разным компонентам связ\-ности, то $l(v,u)$
полагают равным~$\infty$. Длину цепи от вершины~$v$ до вершины~$u$ наименьшей длины называют расстоянием
между этими вершинами:
$$
d(v,u)=\min\limits_{l(v,u)\neq\infty}l(v,u).
$$

Пусть $k$~--- чис\-ло рас\-сто\-яний $d(v,u)\hm\neq\infty$ между всеми парами вершин~$v$ и~$u$ ($v\hm\neq u$).
Среднее расстояние вы\-чис\-ля\-ет\-ся как среднее арифметическое всех рас\-сто\-яний $d(v,u)$ графа~$G$:
\begin{equation*}
\mathrm{dist} = \mathrm{dist}\,(G) = \fr{\sum\nolimits_{v,u\in V(v\neq u)}d(v,u)}{k}.
\end{equation*}

Из теорем~7.2 и~7.1 в~\cite{Hof2} следует, что при $N\hm\rightarrow\infty$ 
\begin{equation}
\label{eq2}
d(v,u)\sim\fr{2\ln\ln N}{|\ln(\tau-1)|}\,,
\end{equation}
если $1\hm<\tau\hm<2$,
и
\begin{equation}
\label{eq3}
d(v,u)\sim\fr{\ln N}{\ln\nu}\,,
\end{equation}
если $\tau>2$, 
где $\nu={\bf E}\xi(\xi-1)/({\bf E}\xi)$ и~$\nu\hm>1$.
Выражения~(\ref{eq2}) и~(\ref{eq3}) носят асимптотический характер и~получены для <<типичных рас\-сто\-яний>>
(где под типичным рас\-сто\-яни\-ем понимается математическое ожидание рас\-сто\-яния)~\cite{Hof2} в~конфигурационных
графах с~бесконечной и~с~конечной дис\-пер\-си\-ями соответственно.
Легко показать, что для конфигурационных графов с~распределением~(\ref{eq1})
$$
\nu=\fr{2\zeta(\tau-1)}{\zeta(\tau)}-2\,,
$$
причем $\nu>1$ при $2\hm<\tau\hm\leq 2,8106\ldots$

Для нахождения зависимости среднего расстояния от числа вершин графа~$N$ и~па\-ра\-мет\-ра распределения
степеней вершин~$\tau$ были получены оценки средних расстояний в~конфигурационных графах различных
размерностей с~разными па\-ра\-мет\-ра\-ми распределения степеней вершин. По полученным результатам были
построены зависимости $\mathrm{dist}$ от чис\-ла вершин графа~$N$ при конкретных значениях па\-ра\-мет\-ра распределения
степеней вершин~$\tau$, а~также зависимости $\mathrm{dist}$ от~$N$ и~$\tau$ на интервалах $\tau\hm\in(1,2)$ и~$\tau\hm\in(2,\,2,8]$.
Рассматривались конфигурационные графы сле\-ду\-ющих размерностей:
$10\hm\leq N\hm\leq 100$ с~шагом~$10$, $100\hm\leq N\hm\leq 1000$ с~шагом~$50$, $1000\hm\leq N\hm\leq 7000$ с~шагом~$500$.
Значения параметра~$\tau$ изменялись с~шагом $0{,}1$ в~двух интервалах: $1{,}1\hm\leq\tau\hm<2$ и~$2\hm<\tau\hm\leq 2{,}8$,
а~так\-же были взяты два дополнительных значения: $\tau\hm=1{,}99$ и~$2{,}01$. Для
каждой пары значений $(N,\tau)$
генерировалось\linebreak\vspace*{-12pt}

\pagebreak

\noindent
  по~$100$~графов, т.\,е.\ $40\,000$ и~$36\,000$ графов на интервалах $1{,}1\hm\leq\tau\hm\leq 1{,}99$
и~$2{,}01\hm\leq\tau\hm\leq 2{,}8$ соответственно.
Расстояния в~графе находились с~применением алгоритма Дейкстры~\cite{Dijk}.

\subsection{Результаты для интервала $\tau\in(1,2)$}

Сначала для рассматриваемых в~настоящей работе графов были по\-стро\-ены зависимости среднего рас\-сто\-яния $\mathrm{dist}$ от
чис\-ла вершин $N$ при фиксированных значениях па\-ра\-мет\-ра~$\tau$, т.\,е.\ для каждого из рас\-смот\-рен\-ных значений
$1{,}1\hm\leq\tau\hm\leq 1{,}99$ были получены регрессионные зависимости вида
\begin{equation}
\label{eq4}
\mathrm{dist} = a \ln\ln N + b\,.
\end{equation}
Здесь и~далее коэффициенты всех регрессионных уравнений находили по\-средст\-вом метода наименьших
квадратов, зна\-чи\-мость коэффициентов проверяли с~помощью критерия Стьюдента. Для оценки степени подгонки
регрессионной модели к~данным вы\-чис\-ля\-ли коэффициент детерминации этой модели и~с~по\-мощью критерия Фишера
проверяли гипотезу $H_0: R^2\hm=0$. Проверка всех статистических гипотез осуществлялась на 5\%-ном уровне
зна\-чи\-мости.

Обозначим $a_f={2}/{|\ln(\tau-1)|}$ и~сравним эти значения с~коэффициентами~$a$ уравнений~(\ref{eq4})
для каждого~$\tau$.
В табл.~1 приведены значения~$a_f$, коэффициенты~$a$ и~$b$ регрессионных урав\-не\-ний вида~(\ref{eq4})
и~соответствующие коэффициенты детерминации~$R^2$ этих уравнений. Все коэффициенты~$a$ и~$b$ в~табл.~1
значимы, а~гипотезы $H_0: R^2\hm=0$ отвергаются для всех уравнений.




Таким образом, при фиксированных~$\tau$ сред\-нее\linebreak рас\-сто\-яние в~графе с~рос\-том чис\-ла его вершин\linebreak
рас\-тет как $\ln\ln N$, так же как и~расстояние в~асимп\-то\-ти\-ке~(\ref{eq2}). Из табл.~1 видно,
что значения коэффициента~$a$ ниже значений~$a_f$ на всем
интервале\linebreak  изменения~$\tau$, однако эта разница не остается неизменной,
а~возрастает с~рос\-том~$\tau$.
Более того, сам угловой коэффициент~$a$ возрастает с~рос\-том
па\-ра-\linebreak\vspace*{-12pt}

%\begin{table*}\small %tabl1
\begin{center}

\vspace*{6pt}

\noindent
\parbox{64mm}{{{\tablename~1}\ \ \small{Значения $a_f$, коэффициенты $a$ и~$b$ зависимостей вида~(\ref{eq4})
и коэффициенты детерминации $R^2$ этих уравнений
}}}


\vspace*{6pt}


{\small 
\begin{tabular}{|c|c|c|c|c|}
\hline
&&&&\\[-10pt]
$\tau$ & $a_f$ & $a$ & $b$ & $R^2$ \\ 
\hline
1,1 & 0,869 &   0,745 & \hphantom{$-$}1,702 & 0,88 \\
1,2 & 1,243 &   0,975 & \hphantom{$-$}1,513 & 0,91 \\
1,3 & 1,661 &   1,255 & \hphantom{$-$}1,244 & 0,94 \\
1,4 & 2,183 &   1,596 & \hphantom{$-$}0,869 & 0,96 \\
1,5 & 2,885 &   1,959 & \hphantom{$-$}0,494 & 0,96 \\
1,6 & 3,915 &   2,421 & $-$0,065 & 0,98 \\
1,7 & 5,607 &   3,048 & $-$0,909 & 0,98 \\
1,8 & 8,963 &   3,668 & $-$1,710 & 0,98 \\
1,9 & 18,982\hphantom{9} & 4,417 & $-$2,806 & 0,98 \\
\hphantom{9}1,99 & 198,998\hphantom{99} & 5,262 & $-$4,045 & 0,96\\
\hline
\end{tabular}
}
\end{center}
%\end{table*}

{ \begin{center}  %fig1
 \vspace*{-2pt}
    \mbox{%
\epsfxsize=78.504mm
\epsfbox{ler-1.eps}
}

\end{center}



\noindent
{{\figurename~1}\ \ \small{График экспериментальных
значений $\mathrm{dist}$ от $N$ и~$1{,}1\hm\leq\tau\hm<2$
}}}

\vspace*{8pt}

\addtocounter{figure}{1}
\addtocounter{table}{1}



\noindent
мет\-ра~$\tau$,
 и~это  на\-гляд\-но видно
на рис.~1, где показана за\-ви\-си\-мость
экспериментальных значений $\mathrm{dist}$ от~$N$ и~$\tau$.


Далее задача состояла в~том, чтобы найти за\-ви\-си\-мость сред\-не\-го рас\-сто\-яния $\mathrm{dist}$ от обеих переменных:
$N$ и~$\tau$.
Сначала была построена за\-ви\-си\-мость в~виде $\mathrm{dist}\hm={2\ln\ln N}/({|\ln(\tau-1)|})\hm+b$. Получено
сле\-ду\-ющее регрессионное уравнение:
\begin{equation*}
\mathrm{dist} = \fr{2\ln\ln N}{|\ln(\tau-1)|}-39{,}604\,.
\end{equation*}
К сожалению, коэффициент детерминации полученной за\-ви\-си\-мости оказался очень низ\-ким
($R^2\hm=0{,}01$). Гипотеза о~равенстве~$R^2$ нулю не отвергается, коэффициент~$b$
статистически не значим, поэтому такую модель использовать для прогноза не имеет смысла.



\begin{figure*} %fig2
\vspace*{1pt}
\begin{minipage}[t]{80mm}
\begin{center}
   \mbox{%
\epsfxsize=79mm
\epsfbox{ler-2-a.eps}
}
\end{center}
\vspace*{-9pt}
\Caption{Регрессионная зависимость~(\ref{eq6}) среднего расстояния $\mathrm{dist}$ от $N$ при фиксированных
значениях $1{,}1\hm\leq\tau\hm<2$: \textit{1}~--- $\tau\hm=1{,}1$;
\textit{2}~--- 1,3; \textit{3}~--- 1,5; \textit{4}~--- 1,7; \textit{5}~--- $\tau\hm= 1{,}99$
}
\end{minipage}
%\end{figure*}
\hfill
%\begin{figure*} %fig3
\vspace*{1pt}
\begin{minipage}[t]{80mm}
\begin{center}
   \mbox{%
\epsfxsize=77.81mm
\epsfbox{ler-2-b.eps}
}
\end{center}
\vspace*{-9pt}
\Caption{Регрессионная зависимость~(\ref{eq6}) среднего расстояния $\mathrm{dist}$
от~$\tau$ при фиксированных значениях $10\hm\leq N\hm\leq 7000$:) \textit{1}~--- $N\hm=10$; \textit{2}~--- 100; \textit{3}~--- 1000;
\textit{4}~--- 5000; \textit{5}~--- $N\hm=7000$}
\end{minipage}
\vspace*{-4pt}
\end{figure*}



Далее была построена регрессия вида $\mathrm{dist}\hm={a\ln\ln N}/({|\ln(\tau-1)|})\hm+b$ и~получена
за\-ви\-си\-мость
\begin{equation}
\label{eq5}
\mathrm{dist} = \fr{0{,}0104\ln\ln N}{|\ln(\tau-1)|}+3{,}922
\end{equation}
с~коэффициентом детерминации $R^2\hm=0{,}21$. В~данном случае гипотеза $H_0: R^2\hm=0$ отвергается,
а~что касается коэффициентов~$a$ и~$b$ регрессионного уравнения~(\ref{eq5}), то коэффициент~$a$
оказался статистически значим, а~коэффициент~$b$ нет.
Поиск наилучшей регрессионной за\-ви\-си\-мости был продолжен и~привел к~получению сле\-ду\-ющей
модели за\-ви\-си\-мости сред\-не\-го рас\-сто\-яния конфигурационного графа $\mathrm{dist}$ от объема графа~$N$ 
и~па\-ра\-мет\-ра распределения степеней вершин~$\tau$:
\begin{equation}\label{eq6}
\mathrm{dist} = \fr{2(4{,}488-3{,}077\tau+0{,}417\tau^2)\ln\ln N}{|\ln(\tau-1)|}
\end{equation}
с коэффициентом детерминации $R^2\hm=0{,}88$ и~значимыми коэффициентами регрессии. Графически
за\-ви\-си\-мость~(\ref{eq6}) пред\-став\-ле\-на на рис.~2 и~3.

Оценка значимости различия между коэффициентами множественной корреляции $r\hm=\sqrt{R^2}$ регрессионных моделей~(\ref{eq5}) и~(\ref{eq6})
 на уровне зна\-чи\-мости~0,05 показала, что нулевая гипотеза $H_0:\linebreak r_{(5)}\hm=r_{(6)}$
($r_{(5)}$ и~$r_{(6)}$~--- коэффициенты множественной корреляции зависимостей~(\ref{eq5}) и~(\ref{eq6})
соответственно) отвергается; следовательно, различие между коэффициентами корреляции значимо.
Остатки обеих моделей распределены нормально, но сравнение сумм квад\-ра\-тов
остатков $\mathrm{SSR}_{(5)}\hm=62951{,}1$ и~$\mathrm{SSR}_{(6)}\hm=24786{,}9$ показывает, что $\mathrm{SSR}_{(5)}\hm>\mathrm{SSR}_{(6)}$, т.\,е.\ модель~(\ref{eq6}) 
<<лучше>> в~смыс\-ле описания изуча\-емо\-го явления и~для прогнозирования.
Таким образом, в~качестве наиболее под\-хо\-дя\-щей модели за\-ви\-си\-мости $\mathrm{dist}$ от~$N$ и~$\tau$ для до\-асимп\-то\-ти\-че\-ской
об\-ласти предлагается за\-ви\-си\-мость, описываемая уравнением~(\ref{eq6}).



На рис.~2 и~3 линии внут\-ри затененных областей соответствуют зависимостям $\mathrm{dist}$ от~$N$ (см.\ рис.~2)
и~от~$\tau$ (см.\ рис.~3) при некоторых (отраженных в~легендах) значениях па\-ра\-мет\-ра~$\tau$
или объема графа $N$ соответственно. Заметим, что кривые зависимостей $\mathrm{dist}$ от~$N$ на рис.~2
расположены одна над другой по мере роста значения параметра $1,1\leq\tau<2$ в~пределах его граничных значений.
Аналогично кривые зависимостей $\mathrm{dist}$ от~$\tau$ на рис.~3 также расположены друг над другом по
мере воз\-рас\-та\-ния чис\-ла вершин графа $10\hm\leq N\hm\leq 7000$.


\vspace*{-6pt}

\subsection{Результаты для интервала $\tau\in(2,\,2{,}8]$}

\vspace*{-2pt}


Исследование зависимости сред\-не\-го рас\-сто\-яния от~$N$ и~$\tau\hm\in(2,\,2{,}8]$ в~до\-асимп\-то\-ти\-че\-ской об\-ласти
было проведено аналогично предыду\-ще\-му исследованию для $\tau\hm\in(1,2)$.
Сначала для фиксированных значений па\-ра\-мет\-ра $2{,}01\hm\leq\tau\hm\leq 2{,}8$ были построены зависимости сред\-не\-го
рас\-сто\-яния $\mathrm{dist}$ от чис\-ла вершин графа~$N$ сле\-ду\-юще\-го вида:
\begin{equation}
\label{eq7}
\mathrm{dist} = a \ln N + b\,.
\end{equation}

Обозначим 
$$
a_f=\fr{1}{\ln\left({2\zeta(\tau-1)}/{\zeta(\tau)}-2\right)}\,.
$$
 Для сравнения этих
значений с~коэффициентами~$a$ уравнений~(\ref{eq7}) для каждого~$\tau$ все они приведены
в~табл.~2 наряду с~коэффициентами~$b$ и~со\-от\-вет\-ст\-ву\-ющи\-ми коэффициентами детерминации
$R^2$ этих уравнений. Все коэффициенты~$a$ и~$b$ значимы, а~гипотезы о~ра\-венст\-ве нулю коэффициентов
детерминации полученных моделей отвергаются.


\setcounter{figure}{4}
\begin{figure*}[b] %fig5
\vspace*{1pt}
\begin{minipage}[t]{81mm}
\begin{center}
   \mbox{%
\epsfxsize=80mm
\epsfbox{ler-4-a.eps}
}
\end{center}
\vspace*{-13pt}
\Caption{Регрессионная зависимость~(\ref{eq9}) среднего рас\-сто\-яния $\mathrm{dist}$ от~$N$ при фиксированных
значениях $2\hm<\tau\hm\leq 2{,}8$:
\textit{1}~--- $\tau\hm= 2{,}01$; \textit{2}~--- 2,2288\ldots; \textit{3}~--- 2,4; \textit{4}~--- 2,6; \textit{5}~--- $\tau\hm= 2{,}8$}
%\end{figure*}
\end{minipage}
\hfill
%\begin{figure*} %fig6
\vspace*{1pt}
\begin{minipage}[t]{79.94mm}
\begin{center}
   \mbox{%
\epsfxsize=78.94mm
\epsfbox{ler-4-b.eps}
}
\end{center}
\vspace*{-13pt}
\Caption{Регрессионная зависимость~(\ref{eq9}) среднего рас\-сто\-яния $\mathrm{dist}$ 
от~$\tau$ при фиксированных значениях $10\hm\leq N\hm\leq 7000$:
\textit{1}~--- $N\hm=10$; \textit{2}~--- 100; \textit{3}~--- 1000;
\textit{4}~--- 5000; \textit{5}~--- $N\hm=7000$}
\end{minipage}
\end{figure*}

Таким образом, при фиксированных~$\tau$ из интервала $(2,\,2{,}8]$ сред\-нее рас\-сто\-яние в~графе
возрастает логарифмически с~рос\-том чис\-ла его вершин $N$, так
же как и~рас\-сто\-яние в~асимп\-то\-ти\-ке
(см.\ выраже-\linebreak\vspace*{-12pt}

%\begin{table*}\small  %tabl2
\begin{center}

\vspace*{6pt}

\noindent
\parbox{62mm}{{{\tablename~2}\ \ \small{Значения $a_f$, коэффициенты~$a$ и~$b$ зависимостей вида~(\ref{eq7})
и~коэффициенты детерминации $R^2$ этих уравнений
}}
}

\vspace*{6pt}


{\small \begin{tabular}{|c|c|c|c|c|}
\hline
&&&&\\[-10pt]
$\tau$ & $a_f$ & $a$ & $b$ & $R^2$ \\
 \hline
\hphantom{9}2,01 & 0,209 & 0,967 & $-$0,684 & 0,96 \\
2,1 &   0,408 & 1,151 & $-$1,661 & 0,98 \\
2,2 &   0,586 & 1,344 & $-$2,777 & 0,98 \\
2,3 &   0,800 & 1,541 & $-$3,994 & 0,95 \\
2,4 &   1,096 & 1,573 & $-$4,428 & 0,91 \\
2,5 &   1,565 & 1,430 & $-$4,078 & 0,88 \\
2,6 &   2,459 & 1,076 & $-$2,673 & 0,89 \\
2,7 &   4,942 & 0,760 & $-$1,387 & 0,91 \\
2,8 &   53,870\hphantom{9} & 0,507 & $-$0,352 & 0,94\\
\hline
\end{tabular}
}
\vspace*{3pt}
\end{center}
%\end{table*}

%\vspace*{3pt}

{ \begin{center}  %fig4
 \vspace*{-2pt}
   \mbox{%
\epsfxsize=78.504mm
\epsfbox{ler-3.eps}
}

\end{center}

\noindent
{{\figurename~4}\ \ \small{График экспериментальных
значений $\mathrm{dist}$ от $N$ и~$2\hm<\tau\hm\leq 2{,}8$
}}}

\vspace*{6pt}

\addtocounter{figure}{1}
\addtocounter{table}{1}


\noindent
 ние~(\ref{eq3})).
Сравнение значений коэффициентов~$a$ с~$a_f$ показывает, что для $2{,}01\leq\tau\leq 2{,}4$ значения~$a$ 
выше значений $a_f$, а~при $2{,}5\hm\leq\tau\hm\leq 2{,}8$ ниже (см.\ табл.~2).
Изменение углового коэффициента~$a$ в~данном случае показывает, что сред\-нее расстояние в~графе
с~рос\-том значения~$\tau$ сначала возрастает, достигая максимума в~промежутке от~2,2 до~2,4, 
а~затем убывает. На\-гляд\-но это мож\-но видеть на рис.~4, где показана за\-ви\-си\-мость
экспериментальных значений $\mathrm{dist}$ от~$N$ и~$\tau$.





Поиск зависимости среднего расстояния $\mathrm{dist}$ от переменных $N$ и~$\tau$ 
проходил по аналогии с~предыду\-щим интервалом изменения параметра распределения степеней вершин.
Сначала была по\-стро\-ена за\-ви\-си\-мость в~виде 
$$
\mathrm{dist}=\fr{\ln N}{\ln\left({2\zeta(\tau-1)}/{\zeta(\tau)}-2\right)}+b
$$
и получено сле\-ду\-ющее регрессионное уравнение:
\begin{equation*}
\mathrm{dist} = \fr{\ln N}{\ln\left({2\zeta(\tau-1)}/{\zeta(\tau)}-2\right)}-40{,}936\,.
\end{equation*}
К сожалению, на этом интервале коэффициент детерминации полученной за\-ви\-си\-мости оказался очень низким
($R^2\hm=0{,}0005$), гипотеза о~равенстве~$R^2$ нулю не отвергается и~коэффициент~$b$ не значим.
Следовательно, такую модель не имеет смыс\-ла использовать для прогноза.
Поэтому была осуществлена попытка по\-стро\-ить регрессию вида
$$
\mathrm{dist}=\fr{a\ln N}{\ln\left({2\zeta(\tau-1)}/{\zeta(\tau)}-2\right)}+b
$$ 
и~была получена зависимость
\begin{equation}
\label{eq8}
\mathrm{dist} = 4{,}962 - \fr{0{,}005\ln N}{\ln\left({2\zeta(\tau-1)}/{\zeta(\tau)}-2\right)}
\end{equation}
с коэффициентом детерминации $R^2\hm=0{,}06$. Несмотря на столь низ\-кое значение~$R^2$, гипотеза о~его
равенстве нулю отвергается, однако оценка зна\-чи\-мости коэффициентов~$a$ и~$b$
регрессионного уравнения~(\ref{eq8}) показала, что коэффициент~$a$ статистически значим, тогда как~$b$~-- нет.
Дальнейший поиск наилучшей регрессии привел к~получению сле\-ду\-ющей за\-ви\-си\-мости
сред\-не\-го рас\-сто\-яния конфигурационного графа $\mathrm{dist}$ от $N$ и~$\tau$:
\begin{equation}
\label{eq9}
\mathrm{dist} = \fr{(31{,}706-22{,}076\tau+3{,}841\tau^2)\ln N}{\ln\left({2\zeta(\tau-1)}/{\zeta(\tau)}-2\right)}\,,
\end{equation}
где все коэффициенты модели значимы, а $R^2\hm=0{,}74$. Зависимость~(\ref{eq9}) отражена графически на рис.~5 и~6.

Для моделей~(\ref{eq8}) и~(\ref{eq9}) была оценена зна\-чи\-мость различия между коэффициентами множественной
корреляции этих моделей при 5\%-ном уровне зна\-чи\-мости. В~результате $H_0:r_{(8)}=r_{(9)}$ была отвергнута, т.\,е.\
различие между коэффициентами корреляции оказалось значимым.
Проверка остатков регрессий~(\ref{eq8}) и~(\ref{eq9}) на нормальность показала, что нормальное распределение
имеют только остатки модели~(\ref{eq9}). Кроме того, остаточная сумма квад\-ра\-тов модели~(\ref{eq8})
$\mathrm{SSR}_{(8)}\hm=245793{,}9$ больше, чем $\mathrm{SSR}_{(9)}\hm=102369{,}9$. Поэтому мож\-но сделать вывод о~том, что модель~(\ref{eq9})
лучше подходит для прогноза, чем модель~(\ref{eq8}).
Таким образом, при значениях па\-ра\-мет\-ра $2\hm<\tau\hm\leq 2{,}8$ в~качестве наиболее подходящей модели за\-ви\-си\-мости
среднего рас\-сто\-яния $\mathrm{dist}$ от~$N$ и~$\tau$ в~до\-асимп\-то\-ти\-че\-ской об\-ласти предлагается за\-ви\-си\-мость, опи\-сы\-ва\-емая
уравнением~(\ref{eq9}).



На рис.~5 и~6 линии, находящиеся внут\-ри затененных областей и~отраженные в~легендах, соответствуют
зависимостям $\mathrm{dist}$ от~$N$ (см.\ рис.~5) и~от~$\tau$ (см.\ рис.~6) при некоторых
значениях па\-ра\-мет\-ра $\tau$ или объема графа~$N$ соответственно.
На рис.~5 ниж\-няя граница об\-ласти соответствует $\tau\hm=2{,}01$, верхняя~--- максимуму функции~(\ref{eq9}) 
по параметру $\tau$: $\tau^*\hm=2{,}2288\ldots$, а~кривые зависимостей $\mathrm{dist}$ от~$N$ внут\-ри затененной
об\-ласти расположены сле\-ду\-ющим образом: по воз\-рас\-та\-нию значений $\mathrm{dist}$ при увеличении значений~$\tau$ от~1,1
до~$\tau^*$ и~по убыванию $\mathrm{dist}$ при рос\-те~$\tau$ от~$\tau^*$ до~2,8. А~на рис.~6 кривые
зависимостей $\mathrm{dist}$ от~$\tau$ расположены друг над другом по мере воз\-рас\-та\-ния чис\-ла вершин графа
$10\hm\leq N\hm\leq 7000$ в~пределах граничных значений.

\vspace*{-9pt}


\subsection{Результаты при $\tau=2$}

\vspace*{-2pt}

Заметим, что модели~(\ref{eq6}) и~(\ref{eq9}) не охватывают значение па\-ра\-мет\-ра распределения степеней
вершин $\tau\hm=2$. Однако по экспериментальным данным была по\-стро\-ена сле\-ду\-ющая регрессионная за\-ви\-си\-мость
$\mathrm{dist}$ от~$N$ при фиксированном $\tau\hm=2$ (все коэффициенты модели значимы):
\begin{equation}
\label{eq10}
\mathrm{dist} = 5{,}262\ln\ln N - 4{,}045 \quad \left(R^2=0{,}73\right).
\end{equation}

\vspace*{-14.5pt}

\section{Выводы}

\vspace*{-2.5pt}

Итак, экспериментальные результаты на степенных конфигурационных графах с~фиксированным
па\-ра\-мет\-ром~$\tau$ распределения~(\ref{eq1}) степеней вершин показывают, что на интервале $(1,2)$ 
с~рос\-том объема~$N$ среднее расстояние $\mathrm{dist}$ в~графе рас\-тет как $\ln\ln N$, а~на интервале $2\hm<\tau\hm\leq 2{,}8$
рас\-тет логарифмически в~доасимптотической об\-ласти (при $N\hm\leq 7000$) так же, как это было показано
Р.~Хофстадом~\cite{Hof2} для типичного расстояния в~графах при $N\hm\rightarrow\infty$.
Однако что касается за\-ви\-си\-мости среднего расстояния от переменных~$N$ и~$\tau$, то при малых
объемах графа предлагается использовать модели~(\ref{eq6}) и~(\ref{eq9}) в~соответствующих интервалах
изменения параметра~$\tau$, так как они лучше описывают данную за\-ви\-си\-мость, что было под\-тверж\-де\-но
в~настоящей работе с~по\-мощью методов статистического анализа
и предложена модель~(\ref{eq10}) за\-ви\-си\-мости сред\-не\-го рас\-сто\-яния от чис\-ла вершин~$N$ при $\tau\hm=2$
так\-же для графов в~до\-асимп\-то\-ти\-че\-ской об\-ласти.

{\small\frenchspacing
 {\baselineskip=10.7pt
 %\addcontentsline{toc}{section}{References}
 \begin{thebibliography}{99}
 
 %\vspace*{-6pt}
 
\bibitem{Dur} %1
\Au{Durrett~R.} Random graph dynamics.~--- Cambridge: Cambridge University
Press, 2007. 221~p. doi: 10.1017/ CBO9780511546594.

\bibitem{Hof1} %2
\Au{Hofstad~R.} Random graphs and complex networks.~--- Cambridge:
Cambridge University Press, 2017.  Vol.~1. 337~p. doi: 10.1017/9781316779422.



\bibitem{New1} %3
\Au{Newman~M.\,E.\,J.} Networks. An introduction.~--- Oxford: Oxford University Press, 2010. 772~p.
doi: 10.1093/ acprof:oso/9780199206650.001.0001.

\bibitem{New2} %4
\textit{Newman~M.\,E.\,J.} Networks.~--- 2nd ed.~--- Oxford: Oxford University Press, 2018. 800~p.
doi: 10.1093/oso/ 9780198805090.001.0001.

\bibitem{Hof2} %5
\textit{Hofstad~R.} Random graphs and complex networks~// Notes RGCNII, 2020.  Vol.~2.
314~p. {\sf https://www.win. tue.nl/$\sim$rhofstad/NotesRGCNII.pdf.}

\bibitem{Fa} %6
\Au{Faloutsos~C., Faloutsos~P., Faloutsos~M.} On power-law relationships of
the internet topology~// Comput. Commun. Rev., 1999. Vol.~29. P.~251--262.
doi: 10.1145/ 316194.316229.

\bibitem{RN} %7
\Au{Reittu~H., Norros~I.} On the power-law random graph model of massive
data networks~// Perform. Evaluation, 2004. Vol.~55. Iss.~1-2. P.~3--23.
doi: 10.1016/S0166-5316(03)00097-X.

\bibitem{Bol}
\Au{Bollobas~B.} A~probabilistic proof of an asymptotic formula for the number
of labelled regular graphs~// Eur. J.~Combin., 1980. Vol.~1.
Iss.~4. P.~311--316. doi: 10.1016/S0195-6698(80)80030-8.



\bibitem{Chu} %9
\Au{Chung~F., Lu~L.} The average distances in random graphs with given expected degrees~// 
P.~Natl. Acad. Sci. USA, 2002. Vol.~99. Iss.~25. P.~15879--15882.
doi: 10.1073/pnas.252631999.

\bibitem{Dijk}
\Au{Dijkstra~E.\,W.} A~note on two problems in connexion with graphs~// 
Numer. Math., 1959. Vol.~1. Iss.~1. P.~269--271. doi: 10.1007/BF01386390.
\end{thebibliography}

 }
 }

\end{multicols}

\vspace*{-7pt}

\hfill{\small\textit{Поступила в~редакцию 21.03.22}}

%\vspace*{8pt}

%\pagebreak

\newpage

\vspace*{-28pt}

%\hrule

%\vspace*{2pt}

%\hrule

%\vspace*{-2pt}

\def\tit{AN AVERAGE DISTANCE IN~THE~POWER-LAW CONFIGURATION GRAPHS}


\def\titkol{An average distance in~the~power-law configuration graphs}


\def\aut{M.\,M.~Leri}

\def\autkol{M.\,M.~Leri}

\titel{\tit}{\aut}{\autkol}{\titkol}

\vspace*{-8pt}


\noindent
Institute of Applied Mathematical Research of the Karelian Research Center of the Russian Academy of Sciences, 
11~Pushkinskaya Str., Petrozavodsk 185910, Russian Federation

\def\leftfootline{\small{\textbf{\thepage}
\hfill INFORMATIKA I EE PRIMENENIYA~--- INFORMATICS AND
APPLICATIONS\ \ \ 2023\ \ \ volume~17\ \ \ issue\ 1}
}%
 \def\rightfootline{\small{INFORMATIKA I EE PRIMENENIYA~---
INFORMATICS AND APPLICATIONS\ \ \ 2023\ \ \ volume~17\ \ \ issue\ 1
\hfill \textbf{\thepage}}}

\vspace*{3pt} 


\Abste{In random configuration graphs with a~discrete power-law vertex degree distribution with a~fixed parameter, 
the average distance in the graph is considered, i.\,e., the arithmetic mean of distances between all pairs of graph nodes. 
This characteristic is estimated using simulation methods. Due to computational constraints, the author considers graphs
 in the pre-asymptotic domain (in this paper, these are the graphs up to 7000~nodes). The models of dependencies of the average distance on 
 the graph size and the parameter of vertex degree distribution are reseived. The obtained results are compared with the results 
 of theoretical studies of the typical distance in a graph in the asymptotics (i.\,e., when the number of graph vertices tends to infinity), 
 given in the works by R.~Hofstad.}

\KWE{configuration graph; power-law distribution;
average distance in a graph; simulation}



\DOI{10.14357/19922264230104} 

\vspace*{-16pt}

\Ack
\noindent
The study was carried out under state order to the Karelian Research Center 
of the Russian Academy of Sciences (Institute of Applied Mathematical Research KarRC RAS).

\vspace*{6pt}

  \begin{multicols}{2}

\renewcommand{\bibname}{\protect\rmfamily References}
%\renewcommand{\bibname}{\large\protect\rm References}

{\small\frenchspacing
 {%\baselineskip=10.8pt
 \addcontentsline{toc}{section}{References}
 \begin{thebibliography}{99} 
\bibitem{1-leri-1}
\Aue{Durrett, R.} 2007. \textit{Random graph dynamics.} Cambridge: Cambridge University
Press. 221~p. doi: 10.1017/ CBO9780511546594.

\bibitem{2-leri-1}
\Aue{Hofstad, R.} 2017. \textit{Random graphs and complex networks.} Cambridge:
Cambridge University Press.   Vol.~1. 337~p. doi: 10.1017/9781316779422.

\bibitem{4-leri-1} %3
\Aue{Newman, M.\,E.\,J.} 2010. \textit{Networks. An introduction.} Oxford: Oxford
University Press. 772~p. doi: 10.1093/ acprof:oso/9780199206650.001.0001.

\bibitem{5-leri-1} %4
\Aue{Newman, M.\,E.\,J.} 2018. \textit{Networks.} 2nd ed. Oxford: Oxford
University Press. 800~p. doi: 10.1093/oso/ 9780198805090.001.0001.

\bibitem{3-leri-1} %5
\Aue{Hofstad, R.} 2020. Random graphs and complex networks.  \textit{Notes RGCNII}. Vol.~2. 314~p.
Available at: {\sf https:// www.win.tue.nl/$\sim$rhofstad/NotesRGCNII.pdf} (accessed January~18, 2023)



\bibitem{6-leri-1}
\Aue{Faloutsos, C., P.~Faloutsos, and M.~Faloutsos.} 1999. On power-law relationships of
the internet topology. \textit{Comput. Commun. Rev.} 29:251--262.
doi: 10.1145/316194.316229.

\bibitem{7-leri-1}
\Aue{Reittu, H., and I.~Norros.} 2004. On the power-law random graph model of massive data
networks. \textit{Perform. Evaluation} 5(1-2)5:3--23.
doi: 10.1016/S0166-5316(03)00097-X.

\bibitem{8-leri-1}
\Aue{Bollobas, B.} 1980. A~probabilistic proof of an asymptotic formula for the number
of labelled regular graphs. \textit{Eur. J.~Combin.} 1(4):311--316.
doi: 10.1016/S0195-6698(80)80030-8.

\bibitem{9-leri-1}
\Aue{Chung, F., and L.~Lu.} 2002. The average distances in random graphs with given expected
degrees. \textit{P.~Natl. Acad. Sci. USA} 99(25):15879--15882.
doi: 10.1073/pnas.252631999.

\bibitem{10-leri-1}
\Aue{Dijkstra, E.\,W.} 1959. A~note on two problems in connexion with graphs.
\textit{Numer. Math.} 1(1):269--271. doi: 10.1007/BF01386390.
\end{thebibliography}

 }
 }

\end{multicols}

\vspace*{-6pt}

\hfill{\small\textit{Received March 21, 2022}}


\Contrl

\noindent
\textbf{Leri Marina M.} (b.\ 1969)~--- Candidate of Science (PhD) in technology, scientist,
Institute of Applied Mathematical Research of the Karelian Research Center of the Russian Academy of Sciences,
11~Pushkinskaya Str., Petrozavodsk 185910, Russian Federation; \mbox{leri@krc.karelia.ru}


\label{end\stat}

\renewcommand{\bibname}{\protect\rm Литература}  %5

\def\stat{razum}

\def\tit{СИСТЕМА МАССОВОГО ОБСЛУЖИВАНИЯ С~ОТРИЦАТЕЛЬНЫМИ ЗАЯВКАМИ,
БУНКЕРОМ ДЛЯ ВЫТЕСНЕННЫХ ЗАЯВОК И~РАЗЛИЧНЫМИ ИНТЕНСИВНОСТЯМИ
ОБСЛУЖИВАНИЯ$^*$}

\def\titkol{СМО с~отрицательными заявками,
бункером для вытесненных заявок и~различными интенсивностями
обслуживания}

\def\autkol{Р.\,В.~Разумчик}
\def\aut{Р.\,В.~Разумчик$^1$}

\titel{\tit}{\aut}{\autkol}{\titkol}

{\renewcommand{\thefootnote}{\fnsymbol{footnote}}\footnotetext[1]
{Работа выполнена при финансовой поддержке Российского фонда
фундаментальных исследований (проект №\,11-07-00112).}}

\renewcommand{\thefootnote}{\arabic{footnote}}
\footnotetext[1]{Институт проблем информатики
Российской академии наук, rrazumchik@ieee.org}


\Abst{Рассмотрена система массового обслуживания (СМО), в которую поступают пуассоновские
потоки положительных и отрицательных заявок. Для положительных заявок имеется
накопитель неограниченной емкости. Отрицательная заявка, поступающая в систему,
выбивает положительную заявку из очереди в накопителе и перемещает ее в другой накопитель
неограниченной емкости (бункер). Если накопитель
пуст, отрицательная заявка покидает систему, не оказывая на нее никакого воздействия.
После окончания обслуживания очередной заявки на прибор поступает заявка из накопителя
или, если накопитель пуст, из бункера. Длительности обслуживания заявок
из накопителя и бункера имеют экспоненциальные распределения с различными
параметрами. Получены соотношения, позволяющие вычислять стационарные
распределения очередей в накопителе и бункере.}

\KW{система массового обслуживания; отрицательные заявки; бункер; различные интенсивности
обслуживания}

  \vskip 14pt plus 9pt minus 6pt

      \thispagestyle{headings}

      \begin{multicols}{2}
      
            \label{st\stat}



\section{Введение}

В~настоящее время изучению систем и сетей массового обслуживания
с отрицательными заявками по-прежнему уделяется значительное внимание.
Об этом свидетельствует большое число работ в данной области, которые
публикуются каждый год. Подробный обзор публикаций до 2003~г.\
приведен в~\cite{bib0}.
Среди недавних работ можно отметить~[2--9].
В~настоящей статье, которая является продолжением работы~\cite{mandzo},
рассматривается другой, отличный от классического, вид отрицательных заявок.
Поступающие отрицательные
заявки не разрушают заявки, ожидающие в очереди, а перемещают
их в дополнительную очередь, откуда те
обслуживаются с относительным приоритетом.

Рассмотрим однолинейную СМО, в
которую поступает пуассоновский поток заявок интен\-сив\-ности~$\lambda$. 
Заявки этого потока, как и в~\cite{mandzo}, будем называть
положительными.
Для положительных заявок имеется накопитель неограниченной ем\-кости.

Помимо положительных заявок в
систему поступает пуассоновский поток отрицательных заявок
интенсивности~$\lambda^-$. Отрицательная заявка, поступающая в
систему, вытесняет одну (положительную) заявку из конца очереди в накопителе и
перемещает ее в накопитель для вытесненных заявок или бункер,
который также имеет неограниченную емкость.

Если в момент поступления отрицательной заявки в накопителе нет
положительных заявок, а на приборе обслуживается заявка, то
отрицательная заявка, не прерывая обслуживания на приборе,
покидает систему, не оказывая на нее никакого воздействия.
То же самое происходит и в случае, когда в момент поступления
отрицательной заявки накопитель и обслуживающий прибор пусты.

Выбор заявок на обслуживание производится следующим образом.
После окончания обслуживания очередной заявки на прибор
становится заявка из накопителя.
Если же накопитель пуст, на прибор поступает заявка из бункера.
Обслуживание заявок не прерывается новыми как положительными, так
и отрицательными заявками.

Длительности обслуживания заявок из накопителя имеют экспоненциальное
распределение с параметром~$\mu_1$, а из бункера~--- экспоненциальное
распределение с параметром~$\mu_2$.

\section{Система уравнений равновесия}

Обозначим через $\nu(t)$ число заявок, находящихся в накопителе
в момент времени~$t$, через $\eta(t)$~---
число заявок в бункере, а через $\psi(t)$~--- тип заявки,
которую обслуживает прибор в момент~$t$.\linebreak
Положим $X(t)=(\nu(t),\eta(t),\psi(t))$. Случайный процесс
$\{X(t),\ \ t\ge 0\}$, описывающий стохастическое поведение
рассматриваемой СМО во времени, является марковским процессом
с непрерывным временем и дискретным множеством состояний.
Множество состояний процесса $\{X(t), \ \ t\ge 0\}$ имеет вид
$\mathcal{X}\hm= \mathcal{X}_0 \cup \mathcal{X}_1$,
где $\mathcal{X}_0\hm=\{0\}$, 
${\mathcal{X}_1}\hm=\{(i,j,k), \ i \ge 0, \ j \ge 0, \ k=\overline{0,1} \}$.
Состояние~$(0)$ соответствует полностью свободной системе;
состояние~$(i,j,k)$
означает, что в момент времени~$t$ в
накопителе находится $i$~заявок, в бункере ожидают $j$~заявок, вытесненных из накопителя, и на приборе обслуживается
заявка либо из накопителя (при $k=0$), либо из бункера (при $k=1$).

Обозначим через $p_{i,j,k}$ стационарную вероятность состояния $(i,j,k)$, а через $p_0$~--- 
стационарную вероятность состояния~$(0)$.
При некотором условии, о котором будет сказано позже, стационарное распределение
существует и удовлетворяет сле\-ду\-ющей системе уравнений равновесия:
\begin{gather}
\label{e1-r}
\lambda p_0 = \mu_1 p_{0,0,0}+\mu_2 p_{0,0,1}\,;\\[4pt]
\label{e2-r}
(\lambda +\mu_1)p_{0,0,0}=\lambda p_0 + \mu_1 p_{1,0,0} + 
\mu_2 p_{1,0,1}\,;
\\[4pt]
\label{e3-r}
(\lambda +\mu_1+\lambda^-)p_{i,0,0}=\lambda p_{i-1,0,0} + \mu_1 p_{i+1,0,0} +{}\notag\\ 
{}+\mu_2 p_{i+1,0,1}\,; \quad
i \ge 1\,, \\[4pt]
\label{e4-r}
(\lambda+\mu_1)p_{0,j,0}=\lambda^- p_{1,j-1,0} + \mu_1 p_{1,j,0} + \mu_2 p_{1,j,1}\,,\notag\\
\hspace*{50mm}j \ge 1\,;
\end{gather}

\vspace*{-9pt}

\noindent
\begin{multline}
\label{e5-r}
(\lambda+\mu_1+\lambda^-) p_{i,j,0}=\lambda p_{i-1,j,0} +  \lambda^- p_{i+1,j-1,0}  +{}\\
{}+ \mu_1 p_{i+1,j,0} + \mu_2 p_{i+1,j,1},\ i \ge 1\,, j \ge 1\,;
\end{multline}

\vspace*{-6pt}

\noindent
\begin{gather}
\label{e6-r}
(\lambda+\mu_2)p_{0,0,1}= \mu_1 p_{0,1,0} + \mu_2 p_{0,1,1}\,;
\\[4pt]
\label{e7-r}
(\lambda+\mu_2+\lambda^-)p_{i,0,1}=\lambda p_{i-1,0,1}\,,\ \ i \ge 1\,;
\\[4pt]
\label{e8-r}
(\lambda+\mu_2)p_{0,j,1}=\lambda^- p_{1,j-1,1}+ \mu_1 p_{0,j+1,0} +{}\notag\\
\hspace*{5mm}{}+ \mu_2 p_{0,j+1,1}\,,\ \ j \ge 1\,;
\\[4pt]
(\lambda+\mu_2+\lambda^-)p_{i,j,1}=\lambda^- p_{i+1,j-1,1} + \lambda 
p_{i-1,j,1}\,,\notag\\ 
\hspace*{30mm} i \ge 1, \ j \ge 1\,. \label{e9-r}
\end{gather}
К этим уравнениям необходимо добавить условие нормировки
\begin{equation}
\label{e10-r}
p_{0} + \sum\limits_{i=0}^{\infty} \sum\limits_{j=0}^{\infty}
\sum\limits_{k=0}^1 p_{i,j,k} = 1\,.
\end{equation}

\section{Совместное распределение числа заявок в~накопителе и~бункере}

Найдем совместное стационарное распределение
числа заявок в накопителе и бункере.
Для этого введем две производящие функции (ПФ):

\noindent
\begin{gather*}
%\label{e11-r}
P(u,v) = \sum\limits_{i=0}^{\infty}
\sum\limits_{j=0}^{\infty} p_{i,j,0} u^i v^j\,,\\
N(u,v) = \sum\limits_{i=0}^{\infty} \sum\limits_{j=0}^{\infty}
p_{i,j,1} u^i v^j\,,
\ \ 0 \le u \le 1\,, \ \ 0 \le v \le 1\,.
\end{gather*}

Сначала найдем выражение для $N(u,v)$. Умножая уравнения~(\ref{e6-r})---(\ref{e9-r})
на~$u^i$ и~$v^j$ и суммируя по всем значениям $i=0,1,\dots$ и $j=0,1,\dots$,
после элементарных выкладок получаем:
\begin{multline}
N(u,v)={}\\
{}=\left((\lambda^- v^2 - \lambda^- uv -  \mu_2 u ) S_1(v) - \mu_1 u S_0(v) + {}\right.\\
\!\!\!\!\!\left.{}+ \lambda u p_0
\vphantom{v^2}\right)\Big /
\left({\lambda u^2v - (\lambda + \mu_2 + \lambda^-)uv + \lambda^- v^2
}\right),\!\!
\label{e12-r}
\end{multline}
где
$$
S_0(v)=\sum\limits_{j=0}^{\infty}
p_{0,j,0} v^j\,; \quad
S_1(v)=\sum\limits_{j=0}^{\infty}
p_{0,j,1} v^j.
$$

Знаменатель в выражении~(\ref{e12-r}) представляет собой квадратный
трехчлен по~$u$, корни которого имеют вид:
\begin{multline*}
%\label{e13-r}
u_{1,2}=u_{1,2}(v)={}\\
{}=\fr{\lambda + \mu_2 + \lambda^-
\pm
\sqrt{(\lambda + \mu_2 + \lambda^-)^2-4\lambda \lambda^- v}
}{2 \lambda}\,,
\end{multline*}
причем $0<u_2<1<u_1$ при $0<v\le 1$. Поскольку ПФ $N(u,v)$ является непрерывной
функцией в области $\{0\le u \le 1,\ 0\le v \le 1\}$, то в точке $(u_2,v)$ вместе
со знаменателем должен обращаться в нуль и числитель. Отсюда приходим к равенству
\begin{multline}
\label{e14-r}
(\lambda^- v^2 - \lambda^- u_2 v -  \mu_2 u_2 )
S_1(v)-\mu_1 u_2 S_0(v)+ {}\\
{}+\lambda u_2 p_0 =0\,.
\end{multline}

Для нахождения выражения для $P(u,v)$ умножим уравнения~(\ref{e2-r})--(\ref{e5-r})
на~$u^i$ и~$v^j$ и просуммируем по всем значениям $i=0,1,\dots$ и $j=0,1,\dots$.
В~итоге получаем:
\begin{multline}
\label{e12-rp}
P(u,v)=\left((\lambda^- v + \mu_1 - \lambda^- u )
S_0(v) -\mu_2 N(u,v)
+{}\right.\\
\left.{}+\mu_2 S_1(v) -  \lambda u p_0\right)\Big /\left(
\lambda u^2 - (\lambda + \mu_1 + \lambda^-)u +{}\right.\\
\left.{}+ \mu_1 + \lambda^- v\right)\,.
\end{multline}

Корни квадратного трехчлена по~$u$ в знаменателе~(\ref{e12-rp})
имеют вид:
\begin{multline*}
%\label{e13-rp)}
u_{3,4}=u_{3,4}(v)={}\\
\!{}=\fr{\lambda + \mu_1 + \lambda^-\!\pm\!\sqrt{(\lambda + \mu_1 + \lambda^-)^2-4\lambda 
(\mu_1+\lambda^- v)}}{2 \lambda},\hspace*{-5pt}
\end{multline*}
причем $0<u_4<1<u_3$ при $0\le v < 1$. Поскольку ПФ $P(u,v)$ является непрерывной
функцией в области $\{0\le u \le 1\,,\ 0\le v \le 1\}$, то в точке $(u_4,v)$ вместе
со знаменателем должен обращаться в нуль и числитель, т.\,е.\
\begin{multline*}
%\label{e14-rp}
(\lambda^- v + \mu_1 - \lambda^- u_4 )S_0(v)
-\mu_2 N(u_4,v)+\mu_2 S_1(v) - {}\\
{}- \lambda u_4 p_0 =0\,.
\end{multline*}

Подставляя в последнее равенство вид $N(u_4,v)$ из~(\ref{e12-r}), 
после арифметических преобразований получаем:
\begin{multline}
\label{e15-rp}
\mu_2 (\lambda v u_4 - (\lambda +\mu_2)v + \mu_2)
S_1(v)-{}\\
{}-\lambda p_0 (\lambda v (u_4-u_2)(u_4-u_1)+\mu_2)-{}\\
{}- 
(\lambda v (u_4-u_2)(u_4-u_1)(\lambda^- - \lambda u_3)-{}\\
{}- \mu_1 \mu_2) S_0(v) = 0\,.
\end{multline}
Напомним, что здесь $u_i=u_i(v)$, $i =\overline{1,4}$.
Решая систему из двух уравнений~(\ref{e14-r}) и~(\ref{e15-rp}),
находим выражения для~$S_0(v)$ и~$S_1(v)$ с точностью до
вероятности~$p_0$:
\begin{multline}
\label{e15-rp1}
S_0(v) = \lambda p_0 \left ( \vphantom{\lambda^-_2}
\mu_2 u_2 (\mu_2 + (\lambda u_4 - \lambda - \mu_2)v)
- {}\right.\\
\left.{}-
(\mu_2+\lambda v (u_4-u_2)(u_4-u_1)) \left(\lambda^- v (u_2 -v) +{}\right.\right.\\
\left.\left.{}+ \mu_2 u_2\right)
\right )
\left (
\left(\lambda v (u_4-u_2)(u_4-u_1)(\lambda^- - \lambda u_3) - {}\right.\right.\\
\left.{}-\mu_1 \mu_2\right)
(\lambda^- v (u_2 -v) + \mu_2 u_2)
+{}
\\
\left.{}+
\mu_1 \mu_2 u_2 (\mu_2 + (\lambda u_4 - \lambda - \mu_2)v)
\vphantom{\lambda_2^-}
\right )^{-1}\,;
\end{multline}

\vspace*{-6pt}

\noindent
\begin{multline}
\label{e15-rp2}
S_1(v) = \fr{ \mu_1 u_2}{\lambda^- v^2 - \lambda^- u_2 v -  \mu_2 u_2}\,
S_0(v) -{}\\
{}- \fr{\lambda u_2}{\lambda^- v^2 - \lambda^- u_2 v -  \mu_2 u_2}\,
p_0 \,.
\end{multline}

Подставляя выражения для $S_0(v)$ и~$S_1(v)$ в~(\ref{e12-r})
и~(\ref{e12-rp}), получаем выражения для ПФ $N(u,v)$ и~$P(u,v)$
с точностью до вероятности~$p_0$, которая будет определена далее.

Для нахождения вероятности~$p_0$ сделаем ряд предварительных выкладок.
Введем следующие обозначения:
\begin{align*}
%\label{e17-r}
\tilde{p}_{n,0}&=\sum_{i+j=n-1} p_{i,j,0}\,, \quad n \ge 1\,; \ \\
\tilde{p}_{n,1}&= \sum_{i+j=n-1} p_{i,j,1}\,, \quad n \ge 1\,.
\end{align*}

Очевидно, что вероятность $\tilde{p}_{n,0}$ ($\tilde{p}_{n,1}$)
обозначает вероятность того, что общее число заявок в системе равно~$n$ 
и на приборе обслуживается заявка из накопителя (бункера).
Суммируя последовательно для каждого $n=1,2,\dots$ соответствующие
уравнения системы уравнений равновесия при $i+j=n-1$, приходим к следующей системе уравнений:
\begin{equation}
\left.
\begin{array}{rlr}
\!\!\!\!\!\!\lambda p_0 &= \mu_1 \tilde{p}_{1,0} + \mu_2 \tilde{p}_{1,1}\,, & n =0\,;\\
\!\!\!\!\!\!\lambda (\tilde{p}_{n,0} + \tilde{p}_{n,1}) &=&\\
&\!\!\!\!{}= \mu_1 \tilde{p}_{n+1,0} + \mu_2 \tilde{p}_{n+1,1}\,, & \ \ n \ge 1\,.\!   
\end{array}
\right\}\!\!
\label{e18-r}
\end{equation}

Если теперь просуммировать все уравнения~(\ref{e18-r}), получается
\begin{equation*}
\label{e19-r}
\lambda = \mu_1 \tilde{p}_{\cdot,0} + \mu_2 \tilde{p}_{\cdot,1},
\end{equation*}
которое с учетом того, что $\tilde{p}_{\cdot,0}=p_{\cdot, \cdot, 0}$ и
$\tilde{p}_{\cdot,1}=p_{\cdot, \cdot, 1}$, примет вид:
\begin{equation}
\label{e20-r}
\lambda = \mu_1 p_{\cdot, \cdot,0} + \mu_2 p_{\cdot, \cdot,1}\,,
\end{equation}
где символ <<$\cdot$>> обозначает суммирование по
всем значениям дискретного аргумента.

Это равенство свидетельствует о том, что, как и должно быть, в стационарном
режиме среднее число заявок, принятых в систему за единицу времени,
равно среднему числу заявок, обслуженных системой в единицу времени.

Заметим, что для системы $M|H_2|1|\infty$ равенство,
связывающее интенсивность принятого в систему потока и интенсивность
обслуженного потока полностью повторяет равенство~(\ref{e20-r})
(см., например,~[11, с.~176]).
Отличие состоит только в том, что для системы $M|H_2|1|\infty$
в равенстве, аналогичном~(\ref{e20-r}),
вместо вероятностей того, что прибор занят обслуживанием заявки
из накопителя и из бункера, стоят вероятности того, что прибор обслуживает
заявку на первой и на второй фазе прибора.
Отсюда следует, что соответствующие вероятности совпадают.

Более того, для рассматриваемой системы и для системы $M|H_2|1|\infty$
совпадают вероятности простоя, поскольку равенство~(\ref{e20-r})
с помощью условия нормировки~(\ref{e10-r}) приводится к виду:
\begin{equation}
\label{e21n-r}
\lambda= \mu_1 (1-p_0) + (\mu_2-\mu_1) p_{\cdot, \cdot, 1}\,.
\end{equation}

Таким образом, вероятность простоя~$p_0$ для рассматриваемой системы
находится по той же формуле, что и вероятность простоя для СМО $M|H_2|1|\infty$,
т.\,е.\
\begin{equation}
\label{e22-r}
p_0= 1-\beta_1 \fr{\lambda}{\mu_1} - \beta_2 \fr{\lambda}{\mu_2}\,,
\end{equation}
где $\beta_1$~--- вероятность того, что прибор занят обслуживанием
заявки из накопителя, а $\beta_2=1-\beta_1$~--- вероятность того, что прибор
занят обслуживанием заявки из бункера.

Покажем, как найти эти вероятности. Вероятность $\beta_2$
совпадает с вероятностью того, что поступившая в систему заявка
будет перемещена в бункер. Поступающая с интенсивностью~$\lambda^-$ отрицательная заявка
<<убивает>> заявку в накопителе, если он не пуст, что
происходит с вероятностью $(1-p_0-p_{0,\cdot,0} - p_{0,\cdot,1})$.
Учитывая, что в единицу времени в систему поступает $\lambda$ заявок, имеем
\begin{equation*}
%\label{e21-r}
\beta_2= \fr{\lambda^- (1-p_0-p_{0,\cdot,0} - p_{0,\cdot,1})}{\lambda}\,.
\end{equation*}

Подставляя в~(\ref{e22-r}) выражение для~$\beta_{1}$ и~$\beta_{2}$,
получаем
\begin{multline}
\label{e23-r}
p_0 \left ( 1+ \fr{\lambda^- }{\mu_1} - \fr{\lambda^-}{\mu_2}
\right ) =
1-\fr{\lambda}{\mu_1} +{}\\
{}+ \fr{\lambda^- (\mu_2 - \mu_1)}{\mu_1 \mu_2} (1- p_{0,\cdot,0} - p_{0,\cdot,1})\,.
\end{multline}

Система из четырех уравнений: (\ref{e12-r}) при $u=v\hm=1$, (\ref{e15-rp2}) при $v\hm=1$,
(\ref{e21n-r}) и (\ref{e23-r})~--- является не\-вы\-рож\-ден\-ной, и ее решение имеет вид
\begin{multline*}
%\label{e24-r}
p_0 =1-\fr{ \lambda}{\mu_1}+{}\\
{}+
\lambda^2 \lambda^- \left(\mu_2 - \mu_1\right)\!\Bigg /\!\left(\mu_1^2 \mu_2
\left(
\left(\mu_2 - \mu_1\right) \fr{u_2(1)}{(u_2(1) - 1)} +{}\right.\right.\\
\!\!\!\left.\left.{}+\lambda^-
\vphantom{\fr{u_2(1)}{(u_2(1) - 1)}}\right)
+\mu_1\left(\mu_1^2 \mu_2 + \lambda^- (\mu_2 - \mu_1) (\mu_1 + \lambda)\right)
\vphantom{\fr{u_2(1)}{(u_2(1) - 1)}}\right)\!\!
\,;
\end{multline*}

\vspace*{-6pt}

\noindent
\begin{multline*}
%\label{e25-r}
p_{0, \cdot, 0}
=
\fr{1}{\lambda^- (\mu_2 - \mu_1)
(\lambda^- - (\mu_2 - \mu_1 + \lambda^- ) u_2(1))
}
\times{}
\\
{}\times
\left (
p_0 \lambda^- (\mu_2 - \mu_1) ((\mu_2 + \lambda + \lambda^- ) u_2(1)- \lambda^- )
+{}\right.\\
+
(
\lambda^- u_2(1) + \mu_2 u_2(1) - \lambda^- )\left(
p_0 \mu_1 \mu_2 + \lambda \mu_2 - {}\right.\\
\left.\left.{}-\mu_1 \mu_2 -
 \lambda^- (\mu_2 - \mu_1)\right)
\right)
\,;
\end{multline*}

\vspace*{-6pt}

\noindent
\begin{multline*}
%\label{e15-r}
p_{0, \cdot,1} = \fr{ \mu_1 u_2(1)}
{\lambda^- - \lambda^- u_2(1) -  \mu_2 u_2(1)} p_{0, \cdot,0} -{}\\
{}- \fr{\lambda u_2(1)}
{\lambda^- - \lambda^- u_2(1) -  \mu_2 u_2(1) } p_0\,,
\end{multline*}

\noindent
\begin{equation*}
\label{e16-r}
p_{\cdot, \cdot,1} = p_{0, \cdot,1}
+\fr{\mu_1}{\mu_2} p_{0, \cdot,0} - \fr{\lambda}{\mu_2} p_0\,.
\end{equation*}

Нетрудно видеть, что если $\mu_1=\mu_2=\mu$, т.\,е.\
имеет место одинаковая интенсивность обслуживания заявок как из накопителя, так и из бункера,
то вероятность простоя $p_0=1- \lambda / \mu$, что совпадает с вероятностью простоя
 в системе из~\cite{mandzo}.

Зная вероятность~$p_0$, можно выписать алгоритм
расчета совместного стационарного распределения $p_{i,j,k}$:
\begin{itemize}
%\setcounter{cyritem}{0}
\item сначала находится вероятность $p_{0,0,0}$ по
формуле, которая следует из~(\ref{e15-rp1}) при $v = 0$:

\noindent
\begin{multline*}
\!\!\!\!\!p_{0,0,0} \!=\! \left(\!\lambda p_0 \left( u^{'}_2(0) \left[
\vphantom{\lambda_2^-}
 \lambda u_4(0) ( 1 - u_4(0) + u_1(0) )
 -{}\right.\right.\right.\hphantom{-4.68806pt}\\
\!\!\!\!\!\left.\left.\left.{}-\lambda - \mu_2 -  \lambda^- \right] + \lambda^-
\vphantom{u_2^{'}}\right)\right)\!\!\Big /\!\!\left( u^{'}_2(0)
\left[ \vphantom{\lambda^-}
\lambda u_4(0) \left(u_4(0) -{}\right.\right.\right.
\end{multline*}

\noindent
\begin{multline*}
\left.\left.\left.{}- u_1(0)\right) \left(\lambda^- - \lambda u_3(0)\right)
+
\mu_1 \left(\lambda u_4(0) - \lambda -{}\right.\right.\right.\\
\left.\left.\left.{}- \mu_2 -  \lambda^- \right)
\right] +
\mu_1 \lambda^- 
\vphantom{u_2^{'}}\right)\,;
\end{multline*}
\item затем, используя~(\ref{e1-r}), находится вероятность $p_{0,0,1}$:
$$ p_{0,0,1}
= \fr{\lambda p_0 - \mu_1 p_{0,0,0}
}{\mu_2 }\,;
$$
\item далее из~(\ref{e7-r}) находятся вероятности $p_{i,0,1}$, $i \ge 1$:
$$
p_{i,0,1}= \delta^i p_{0,0,1}\,; \enskip \delta=\fr{\lambda}{\lambda+\mu_2+\lambda^- }\,;
$$
\item используя~(\ref{e1-r}), (\ref{e2-r}) и~(\ref{e7-r}), находится
вероятность~$p_{1,0,0}$:
$$
p_{1,0,0}= \fr{\lambda p_{0,0,0} - \mu_2 (1+\delta) p_{0,0,1}}{\mu_1}\,;
$$
\item из (\ref{e3-r}) вычисляются вероятности $p_{i,0,0}$, $i \ge 2$:
\begin{multline*}
p_{i,0,0}= \left ( \fr{ \lambda + \mu_1 + \lambda^-}{\mu_1}
\right )
p_{i-1,0,0}
-{}\\
{}-\fr{\lambda}{\mu_1} p_{i-2,0,0} 
-
\fr{\mu_2 \delta^i p_{0,0,1}}{\mu_1}\,;
\end{multline*}
\item далее из~(\ref{e15-rp1}) и~(\ref{e15-rp2}) находятся
вероятности $p_{0,j,0}$,  $j \ge 1$, и $p_{0,j,1}$, $ j \ge 1$,
по формулам:
\begin{align*}
p_{0,j,0} &= \fr{1}{j!} 
\fr{d^{j}S_0(v) }{d v^{j}}
\Big|_{v=0}
\,; \\
p_{0,j,1}&= \fr{1}{j!}
\fr{ d^{j}S_1(v)}{d v^{j}}
\Big|_{v=0}\,;
\end{align*}
\item затем из~(\ref{e9-r}) последовательно
для каждого $j\hm=1, 2, \dots$ вычисляются вероятности $p_{i,j,1}$, $i \hm\ge 1$:
\begin{multline*}
p_{i,j,1}=\fr{\lambda^- }{\lambda+\mu_2+\lambda^-}\, p_{i+1,j-1,1} + {}\\
{}+
\fr{\lambda}{(\lambda+\mu_2+\lambda^-)} \,p_{i-1,j,1}\,;
\end{multline*}
\item из~(\ref{e4-r}) находятся вероятности $p_{1,j,0}$, $j \ge 1$:
$$
p_{1,j,0}= \fr{\lambda+\mu_1}{\mu_1}\, p_{0,j,0} - \fr{\lambda^-}
{\mu_1}\, p_{1,j-1,0} - \fr{\mu_2}{\mu_1} \,p_{1,j,1}\,;
$$
\item последними, из~(\ref{e5-r}), последовательно
для каж\-до\-го $j=1, 2, \dots$ находятся вероятности $p_{i,j,0}$, $i \ge 2$:
\begin{multline*}
p_{i,j,0}= \left((\lambda+\mu_1+\lambda^-) p_{i-1,j,0} - \lambda p_{i-2,j,0} - {}\right.\\
\left.{}-
\lambda^- p_{i,j-1,0} - \mu_2 p_{i,j,1}\right)\big /\mu_1\,.
\end{multline*}
\end{itemize}

Таким образом, совместное стационарное распределение
найдено. Поскольку заявки поступают в систему с интенсивностью~$\lambda$,
а обслуживаются с интенсивностью
$\mu^{*}= \left ( {\beta_1}/{\mu_1} + {\beta_2}/{\mu_2} \right )^{-1}$,
неравенство $\lambda \hm< \mu^{*}$ является необходимым и достаточным условием
для его существования.

\section{Заключение}

В статье представлен анализ СМО с отрицательными заявками
и бункером для вытесненных заявок, в которой заявки из накопителя
и бункера обслуживаются с различными интенсивностями.
Найдено совместное стационарное распределение числа
заявок в накопителе и бункере как в терминах производящих функций,
так и в терминах вычислительных алгоритмов.

С помощью программных средств GPSS 
была разработана имитационная модель системы. Результаты моделирования показали
хорошие совпадения с результатами численных расчетов, проведенных по полученным формулам.

\bigskip
Автор глубоко признателен профессору А.\,В.~Печинкину за постановку задачи, 
ряд ценных замечаний и помощь при оформлении статьи.

{\small\frenchspacing
{%\baselineskip=10.8pt
\addcontentsline{toc}{section}{Литература}
\begin{thebibliography}{99}

\bibitem{bib0}
\Au{Бочаров П.\,П., Вишневский В.\,М. }
G-сети: развитие теории мультипликативных сетей~//
Автоматика и телемеханика, 2003. №\,5. C.~70--74.


\bibitem{bib1}
\Au{Muthu Ganapathi Subramanian A., Ayyappan~G., Gopal Sekar}.
$M|M|1$ retrial queueing system with negative arrival under
non-pre-emptive priority service~// J.~Fundamental Sciences,
2009. Vol.~5. No.\,2. P.~129--145.

\bibitem{bib2}
\Au{D'Apice C., Manzo R., Pechinkin~A., Shorgin~S.}
Queueing network with negative customers and the route change~//
 Conference (International) on Ultra Modern Telecommunications Proceedings, 2009. P.~1--5.

\bibitem{bib3}
\Au{Pechinkin A., Razumchik~R.}
A queueing system with negative claims and a bunker for superseded claims in discrete time~//
Automation and Remote Control, 2009. Vol.~70. No.\,12. P.~109--120.

\bibitem{bib4} %5
\Au{Ayyappan G., Gopal Sekar, Muthu Ganapthi Subramanian~A.}
$M|M|1$ retrial queueing system with negative
arrival under erlang-k service by matrix
geometric method~//
Appl. Math. Sci., 2010. Vol.~4. No.\,48. P.~2355--2367.

\bibitem{bib6} %6
\Au{Pechinkin A.\,V., Razumchik~R.\,V.}
Waiting characteristics of queueing system $Geo|Geo|1$ with negative claims and a bunker for superseded claims in discrete 
time~//
Conference (International) on Ultra Modern Telecommunications Proceedings, 2010. P.~1051--1055.

\bibitem{bib5} %7
\Au{Krishna Kumar B., Pavai Madheswari~S., Anantha Lakshmi~S.\,R.}
An $M/G/1$ Bernoulli feedback retrial queueing system with negative customers~//
Operational Res., 2011. Vol.~1. P.~1--24.

\bibitem{bib7}
\Au{Songfang Jia, Yanheng Chen}.
A discrete time queueing system
with negative customers and single working vacation~// 3rd
 Conference (International) on Computer Research and Development
(ICCRD) Proceedings, 2011. Vol.~4. P.~15--19.

\bibitem{bib8}
\Au{Tien Van Do}. A new solution for a queueing model of a
manufacturing cell with negative customers under a rotation rule~//
J.~Performance Evaluation, 2011. Vol.~68. Issue~4. P.~330--337.

\bibitem{mandzo}
\Au{Мандзо~Р., Касконе~Н., Разумчик Р.\,В.}
Экспоненциальная система массового обслуживания с отрицательными заявками и
бункером для вытесненных заявок~// Автоматика и телемеханика, 2008. №\,9. C.~103--113.

\label{end\stat}

\bibitem{tmo}
\Au{Бочаров П.\,П., Печинкин А.\,В.}
Теория массового обслуживания.~--- М.: РУДН, 1995. 529~с.




 \end{thebibliography}
}
}


\end{multicols}           %6
\def\stat{kudr}

\def\tit{ПРИБЛИЖЕННЫЕ МЕТОДЫ РЕШЕНИЯ ЗАДАЧИ ДИАГНОСТИКИ ПЛОСКИМ 
ЗОНДОМ СИЛЬНОИОНИЗОВАННОЙ ПЛАЗМЫ С~УЧЕТОМ КУЛОНОВСКИХ 
СТОЛКНОВЕНИЙ}

\def\titkol{Приближенные методы решения задачи диагностики плоским 
зондом сильноионизованной плазмы} %с~учетом Кулоновских  столкновений}

\def\autkol{И.\,А.~Кудрявцева, А.\,В.~Пантелеев}
\def\aut{И.\,А.~Кудрявцева$^1$, А.\,В.~Пантелеев$^2$}

\titel{\tit}{\aut}{\autkol}{\titkol}

%{\renewcommand{\thefootnote}{\fnsymbol{footnote}}\footnotetext[1]
%{Работа поддержана Российским фондом фундаментальных исследований
%(проекты 11-01-00515а и 11-07-00112а), а также Министерством
%образования и науки РФ в рамках ФЦП <<Научные и
%научно-педагогические кадры инновационной России на 2009--2013~годы>>.}}


\renewcommand{\thefootnote}{\arabic{footnote}}
\footnotetext[1]{Московский авиационный институт, irina.home.mail@mail.ru}
\footnotetext[2]{Московский авиационный институт, avpanteleev@inbox.ru}

\vspace*{-2pt}

\Abst{Сформирована математическая модель, описывающая динамику сильноионизованной 
плазмы с учетом столкновений заряженных частиц вблизи плоского зонда. Модель включает уравнение 
Фоккера--Планка и уравнение Пуассона. Предложено два подхода к решению задачи: на основе метода 
статистических испытаний Мон\-те-Кар\-ло и на основе композиции метода крупных частиц и метода 
расщепления.} 

\vspace*{-2pt}

\KW{телекоммуникационные системы; метод Монте-Карло; метод крупных частиц; метод 
расщепления; зонд; уравнение Фоккера--Планка; уравнение Пуассона} 

\vspace*{-4pt}

 \vskip 8pt plus 9pt minus 6pt

      \thispagestyle{headings}

      \begin{multicols}{2}
      
            \label{st\stat}

\section{Введение}

В настоящее время в области телекоммуникаций все более востребованными становятся 
информационные технологии, основанные на использовании математических моделей и численных 
методов физики плазмы. Поэтому особенно актуальным является решение разнообразных задач анализа 
поведения плазмы, включающих в себя формирование новых моделей и методов их исследования. 
Помимо этого, в разработке телекоммуникационного оборудования эффективно используются 
собственно физические свойства плазмы. В~частности, изготовлена антенна, работа которой основана 
на газовом разряде низкотемпературной плазмы~[1], интенсивно ведутся разработки по созданию и 
усовершенствованию источников бесперебойного питания на основе плазменных элементов~[2, 3]. 
      
      Одним из наиболее перспективных направлений для построения систем оптической 
беспроводной связи является использование лазеров~\cite{4-k, 5-k}. В~этой связи большое внимание 
уделяется использованию плазмы при разработке импульсных сильноточных коммутаторов~\cite{6-k}, 
так как практическое применение подобных разработок требует повышения уровня надежности и 
быстродействия лазерных систем.
      
      Исследования низкотемпературной плазмы также связаны с разработками в области дальней 
космической связи, так как моделирование процессов взаимодействия заряженного тела с верхними 
слоями атмосферы позволяет предлагать способы улучшения существующих систем радиосвязи с 
космическими летательными аппаратами~\cite{7-k}. 
      
      Наряду с этим актуальными также являются задачи диагностики плазмы, поскольку перспективы 
ее использования в области телекоммуникаций после более полного изучения физических свойств 
могут значительно расшириться. 

Для диагностики плазмы применяют зондовые методы исследования~[8--11]. Эти методы относятся к 
классу контактных методов; как следствие, возникает сложность в исследовании пристеночной области 
вблизи зонда, которая характеризуется достаточно сложным распределением потенциала и функциями 
распределения, отличными от максвелловских. 

Данная работа посвящена исследованию переходного режима обтекания заряженного тела плазмой. Для 
переходного режима выполняется следующее условие: длина свободного пробега иона до столкновения 
с нейтральным атомом или другим ионом невелика по сравнению с характерными размерами тела. 
В~этом случае возникает необходимость учета столкновений заряженных частиц с нейтральными 
атомами и кулоновских столкновений. В~работах~\cite{10-k, 11-k} подробно рассмотрена модель с 
учетом столкновений заряженных частиц с нейтральными атомами. В~настоящей статье представлена 
теоретическая модель, описывающая влияния ион-ионных и ион-элек\-т\-рон\-ных столкновений на 
измеряемые характеристики плазмы, что ранее детально не исследовалось.
      
      В~рамках данной работы предлагается модель, описывающая динамику сильноионизованной 
плазмы с учетом кулоновских столкновений. Эта модель учитывает такие процессы взаимодействия, 
как перенос частиц и столкновения между заряженными частицами типа <<ион--ион>> и 
      <<ион--электрон>> под влиянием макроскопического электрического поля. Перечисленные 
процессы описываются самосогласованной системой уравнений, включающей уравнение 
      Фок\-ке\-ра--План\-ка и уравнение Пуассона~[12].
      
      Вычислительная модель задачи строится на основе двух методов: метода статистических 
испытаний Мон\-те-Кар\-ло и композиции метода крупных частиц и метода расщепления. Приведены 
результаты численного моделирования, полученные с использованием вышеперечисленных методов.

\vspace*{-4pt}

\section{Постановка задачи}

\vspace*{-2pt}

Рассматривается следующая физическая постановка зондовой задачи~[11]. В~невозмущенную 
бесконечно протяженную плазму, состоящую из электронов и однозарядных ионов, внесена большая\linebreak 
заряженная до потенциала $\varphi_p$ плоскость. Плоскость, расположенная поперек потока плазмы, 
является идеально поглощающей для электронов. Ионы при ударе о плоскость нейтрализуются. 
Предполагается, что частицы в плазме движутся под действием внешнего электрического поля, 
магнитное поле отсутствует. Концентрации ионов $n_{i\infty}$ и электронов $n_{e\infty}$, а также 
температуры данных час\-тиц~$T_{i\infty}$ 
и~$T_{e\infty}$ в невозмущенной плазме заданы. За начальные 
функции распределения обоих типов час\-тиц принимаются функции распределения Максвелла. 
      
      Требуется с учетом столкновений между заряженными частицами найти напряженность 
самосогласованного электрического поля $\vec{E}(\vec{r},t)$, функции распределения однозарядных 
ионов $f_i(\vec{r}, \vec{v}, t)$ и электронов $f_e(\vec{r}, \vec{v}, t)$, 
а также их моменты (плотности 
токов ионов и электронов  $j_i(\vec{r},t)\hm
=q\int f_i(\vec{r}, \vec{v}, t)\vec{v}\,d\vec{v}$, $j_e(\vec{r},t) 
\hm={\sf e}\int f_e(\vec{r},\vec{v},t)\vec{v}\,d\vec{v}$, где $q=Z_i{\sf e}$, $Z_i=1$~--- заряд иона, ${\sf 
e}$~--- заряд электрона; концентрации ионов и электронов $n_i(\vec{r},t)\hm=\int 
f_i(\vec{r},\vec{v},t)\,d\vec{v}$, $n_e(\vec{r},t)\hm=\int f_e(\vec{r},\vec{v}, t)\,d\vec{v}$). 
Поведение частиц во 
времени~$t$ характеризуется ра\-ди\-ус-век\-то\-ром~$\vec{r}$ и вектором скорости~$\vec{v}$.
      
      Математическая модель, соответствующая данной физической постановке задачи, имеет 
вид~\cite{11-k, 13-k}:

\noindent
      \begin{equation}
      \left.
      \begin{array}{c}
      \fr{\partial f_\alpha (\vec{r},\vec{v},t)}{\partial t}+
      \vec{v}\fr{\partial f_\alpha (\vec{r},\vec{v},t)}{ 
\partial \vec{r}}+
\fr{\vec{F}_\alpha(\vec{r},t)}{m_\alpha}\times{}\\[4pt]
{}\times\fr{\partial f_\alpha(\vec{r},\vec{v},t)}{ \partial 
\vec{v}}=
\left(\fr{\partial f_\alpha(\vec{r},\vec{v},t)}{ \partial t}\right)_{\mathrm{с}}+S_\alpha 
(\vec{r},\vec{v},t)\,;\\[6pt]
      \Delta\varphi(\vec{r},t)=-\fr{{\sf e}}{\varepsilon_0}\left( n_i(\vec{r},t)-n_e(\vec{r},t)\right)\,;\\[6pt]
      \vec{E}(\vec{r},t)=-\nabla \varphi(\vec{r},t)\,.
      \end{array}\!\!
      \right\}\!\!
      \label{e1-k}
      \end{equation}
Здесь первое уравнение~--- уравнение Фок\-ке\-ра--План\-ка для частиц сорта~$\alpha$ ($\alpha=i,e$), 
второе~--- уравнение Пуассона для самосогласованного электрического поля; 
$f_\alpha(\vec{r},\vec{v},t)$~--- функция\linebreak
распределения час\-тиц сорта~$\alpha$; $(\partial 
f_\alpha(\vec{r},\vec{v},t)/\partial t)_{\mathrm{с}}$~--- 
оператор столкновений Фок\-ке\-ра--План\-ка; 
функция~$S_\alpha(\vec{r},\vec{v},t)$ описывает источники или стоки\linebreak
 час\-тиц; 
$\vec{F}_\alpha(\vec{r},t)=q_\alpha\vec{E}(\vec{r},t)$, где $\vec{E}(\vec{r},t)$~--- напряженность 
самосогласованного электрического поля, 
$$
q_\alpha =
\begin{cases}
-{\sf e}\,, & \alpha=e\,,\\
{\sf e}\,, & \alpha=i\,;
\end{cases}
$$
$\varphi(\vec{r},t)$~--- потенциал самосогласованного электрического поля; $n_\alpha(\vec{r},t)$ ($\alpha 
\hm=i,e$)~--- концентрация частиц сорта~$\alpha$; $m_\alpha$~--- масса частицы сорта~$\alpha$; 
$\varepsilon_0$~--- электрическая постоянная. 

Оператор столкновений Фок\-ке\-ра--План\-ка имеет вид~\cite{13-k, 14-k}
\begin{multline*}
\fr{1}{\Gamma_\alpha}\left( \fr{\partial f_\alpha}{\partial t}\right)_{\mathrm{с}} 
=\fr{1}{2}\,\nabla_v\nabla_v:\left(f_\alpha\nabla_v\nabla_vg_\alpha(\vec{r},\vec{v},t)\right)-{}\\
{}-
\nabla_v\cdot\left(f_\alpha\nabla_v h_\alpha\right)\,,
\end{multline*}
где $\nabla_v\nabla_v g_\alpha(\vec{r},\vec{v},t)$~--- ковариантная тензорная производная второго ранга, 
знак двоеточия ($:$) обозначает операцию двойного суммирования:
\begin{gather*}
\Gamma_\alpha=\fr{Z_\alpha^4 {\sf e}^4}{4\pi \varepsilon_0^2 m^2_\alpha}\,\ln D_\alpha\,;
\\
D_\alpha =\fr{12\pi\varepsilon_0 kT_{\alpha\infty}}{Z_\alpha^2 {\sf e}^2}\left( \fr{\varepsilon_0 k 
T_{e\infty}}{n_{e\infty} {\sf e}^2}\right)^{1/2}\,;\\
g_\alpha (\vec{r},\vec{v},t)=\sum\limits_{b=i,e}\left( \fr{Z_b}{Z_\alpha}\right) \int f_b 
(\vec{r},{\vec{v}}^{\,\prime},t)\left\vert \vec{v}-{\vec{v}}^{\,\prime}\right\vert\,d\vec{v}^{\,\prime}\,;\\
h_\alpha (\vec{r},\vec{v},t)=\sum\limits_{b=i,e} \fr{m_\alpha+m_b}{m_b} 
\left(\fr{Z_b}{Z_\alpha}\right)
\int
\fr{f_b(\vec{r},{\vec{v}}^{\,\prime}, t)}{\vert \vec{v}-{\vec{v}}^{\,\prime}\vert}
\,d{\vec{v}}^{\,\prime}\,;\\
Z_\alpha =1\,, \quad \alpha=i,e\,.
\end{gather*}
 
К системе уравнений~(\ref{e1-k}) необходимо добавить начальные и краевые условия:
\begin{equation}
\!\left.
\begin{array}{rrl}
t=0:\ & f_\alpha(\vec{r},\vec{v},0)&=f_\alpha^{\mathrm{maksv}}\,,\enskip \alpha=i,e;\\[9pt]
\vec{r}\in \Omega_p:\ & f_\alpha(\vec{r},\vec{v},t)\big\vert_{\vec{r}\in\Omega_p}&=0\,,\enskip \alpha=i,e\,;\\[9pt]
&\varphi(\vec{r},t)\big\vert_{\vec{r}\in\Omega_p}&=\varphi_p\,;\\[9pt]
\vec{r}\in\Omega_\infty:\ & 
f_\alpha(\vec{r},\vec{v},t)\big\vert_{\vec{r}\in\Omega_\infty}&= %{}\\[9pt]
f_\alpha^{\mathrm{maksv}}\,,\enskip \alpha=i,e\,;\\[9pt]
&\varphi(\vec{r},t)\big\vert_{\vec{r}\in\Omega_\infty}&=0\,,
\end{array}\!\!
\right\}\!\!\!\!
\label{e2-k}
\end{equation}
    где 
    
    \noindent
    \begin{multline*}
    f_\alpha^{\mathrm{maksv}}=n_{\alpha\infty}\left(\fr{m_\alpha}{2k\pi T_{\alpha\infty}}\right)^{3/2}\times{}\\
    {}\times
    \exp\left( -
\fr{m_\alpha}{2kT_{\alpha\infty}}\left\vert\vec{v}-\vec{v}_\infty\right\vert^2\right)\,,
\enskip \alpha=i, e\,;
\end{multline*} 
$\Omega_p$ и $\Omega_\infty$~--- множество радиус-векторов час\-тиц, концы которых принадлежат плоскости зонда и 
границе возмущенной зоны соответственно.

Для решения поставленной задачи введем декартову систему координат таким образом, чтобы 
заряженная плоскость совпала с плоскостью~$0xz$. Тогда положение частицы в пространстве будет 
определяться координатами $x,y,z$, а скорость~--- координатами $v_x, v_y, v_z$. В~силу того что 
плоскость является бесконечно большой в сравнении с характерным размером задачи, функции 
распределения частиц будут зависеть только от переменных $y, v_y, t$.

Поставленную задачу предлагается решать независимо двумя методами. Первый метод основывается на 
методе статистических испытаний Мон\-те-Кар\-ло, второй метод является композицией метода 
расщепления и метода крупных частиц.

\section{Применение метода Монте-Карло}

Запишем самосогласованную систему уравнений~(\ref{e1-k}) и~(\ref{e2-k}) в декартовой системе 
координат с учетом сделанных предположений:
\begin{equation}
\left.
\begin{array}{l}
\fr{\partial f_\alpha}{\partial t}+
v_y\fr{\partial f_\alpha}{\partial y}+\fr{F_y^\alpha}{m_\alpha}\,\fr{\partial 
f_\alpha}{\partial v_y}=\fr{1}{2}\,\fr{\partial^2 }{\partial [v_y]^2}\times{}\\
{}\times \left( 
f_\alpha\fr{\partial^2 g_\alpha  }{\partial [v_y]^2}\right) -
\fr{\partial}{\partial v_y}\left( f_\alpha\fr{\partial h_\alpha}{\partial v_y}\right)\,,
\enskip \alpha=i,e\,;\\[6pt]
    \fr{\partial^2\varphi}{\partial y^2} =-\fr{{\sf e}}{\varepsilon_0}\left(n_i-n_e\right)\,;
    \enskip E_y=-
\fr{\partial\varphi}{\partial y}\,;\\[6pt]
\hspace*{3.1mm}    t=0:\  \hspace*{2.6mm}f_\alpha(y,v_y,0)=f_\alpha^{\mathrm{maksv}}\,,\ \alpha=i,e\,;\\[9pt]
\hspace*{2.9mm} y=0:\ \hspace*{2.8mm}f_\alpha(0,v_y,t)=0\,,\ \alpha=i,e\,;\\[9pt]
\hspace*{24.3mm}\varphi(0,t)=\varphi_p\,;\\[9pt]
y=y_\infty:\ f_\alpha(y_\infty, v_y, t)=f_\alpha^{\mathrm{maksv}}\,,\ \alpha=i,e\,;\\[9pt]
\hspace*{21.5mm}\varphi(y_\infty, t)=0\,.
\end{array}
\right \}
\label{e3-k}
\end{equation}

В полученной системе уравнений~(\ref{e3-k}) перейдем к безразмерным величинам, применив 
соотношение $X=M_X \hat{X}$, где $M_X$~--- масштаб размерной величины~$X$, $\hat{X}$~--- 
безразмерная величина~$X$. В~качестве используемых масштабов были взяты следующие: радиус 
Дебая, скорость теплового движения частиц, концентрация частиц в невозмущенной плазме, потенциал, 
возникающий при разделении зарядов в дебаевской сфере, и производные от них величины.

Система безразмерных уравнений имеет следующий вид:
%\noindent
\begin{equation}
\left.
\begin{array}{l}
\fr{\partial 
\hat{f}_\alpha}{\partial\hat{t}}+A_\alpha\fr{\partial\hat{f}_\alpha}{\partial\hat{y}}+
B_\alpha\hat{E}_y\fr{\partial\hat{f}_\alpha}{\partial \hat{v}_y}={}\\
\!{}=
\fr{\partial^2}{\partial[\hat{v}_y]^2}\left(D_\alpha 
\hat{f}_\alpha\right)-\fr{\partial}{\partial\hat{v}_y}\left(K_\alpha \hat{f}_\alpha\right),\enskip 
\alpha=i,e;\\[9pt]
\fr{\partial^2\hat{\varphi}}{\partial\hat{y}^2}=-\left(\hat{n}_i-\hat{n}_e\right)\,;\enskip \hat{e}_y=-
\fr{\partial\hat\varphi}{\partial\hat{y}}\,;\\[9pt]
\hspace*{3.1mm}\hat{t}=0:\ \hspace*{2.6mm}\hat{f}_\alpha(\hat{y},\hat{v}_y,0)=\hat{f}_\alpha^{\mathrm{maksv}}\,,\enskip \alpha-i,e\,;\\[9pt]
\hspace*{2.9mm}\hat{y}=0:\ \hspace*{2.8mm}\hat{f}_\alpha(0,\hat{v}_y,\hat{t})=0\,,\enskip \alpha=i,e\,;\\[9pt]
\hspace*{24.3mm}\hat\varphi(0,\hat{t})=\hat{\varphi}_p\,;\\[9pt]
\hat{y}=\hat{y}_\infty:\ \hat{f}_\alpha(\hat{y}_\infty, \hat{v}_y, \hat{t})=\hat{f}^{\mathrm{maksv}}_\alpha\,,\enskip 
\alpha=i,e\,;\\[9pt]
\hspace*{21.5mm}\hat\varphi(\hat{y}_\infty,\hat{t})=0\,.
\end{array}
\right\}
\label{e4-k}
\end{equation}
Здесь 

\vspace*{-2pt}

\noindent
\begin{gather*}
A_\alpha=\sqrt{\delta_\alpha }\,\hat{v}_y\,;\enskip 
B_\alpha=\sqrt{\delta_\alpha}\,\fr{z_\alpha}{2\varepsilon_\alpha}\,;\\
\delta_\alpha=\fr{\varepsilon_\alpha}{\mu_\alpha}\,;\enskip 
\varepsilon_\alpha=\fr{T_{\alpha\infty}}{T_{i\infty}}\,;\\
\mu_\alpha=\fr{m_\alpha}{m_i}\,;\enskip 
D_\alpha=A_g^\alpha\fr{\partial^2\hat{g}_\alpha}{\partial  [\hat{v}_y]^2}\,;\\
K_\alpha=A_h^\alpha \fr{\partial \hat{h}_\alpha}{\partial \hat{v}_y}\,,\enskip \alpha=i,e\,,
\end{gather*}
где $A_g^\alpha$ и $A_h^\alpha$~--- коэффициенты, определяемые характерными параметрами 
задачи~\cite{15-k}.

Поиск решения самосогласованной системы уравнений~(\ref{e4-k}) осуществляется по следующей 
схе-\linebreak ме. Вначале находятся значения напряженности\linebreak
 электрического поля по значениям потенциала, 
полученным из граничной задачи для уравнения Пуассона. Далее, используя найденные значения 
напряженности, решается уравнение Фок\-ке\-ра--План\-ка путем перехода к стохастическому 
дифференциальному уравнению (СДУ) Ито:

\noindent
\begin{multline*}
d\Theta_\alpha(\hat{t}) = a_\alpha \left(\hat{t},\Theta_\alpha(\hat{t})\right)+{}\\
{}+\sigma\left(
\hat{t},\Theta_\alpha(\hat{t})\right)\,dW(\hat{t})\,,\quad \alpha=i,e\,,
%\label{e5-k}
\end{multline*}
где 

\noindent
\begin{align*}
\Theta_\alpha(\hat{t})&=\begin{bmatrix}
\hat{y}(\hat{t})\\ \hat{v}_y(\hat{t})
\end{bmatrix}\,;\\
a_\alpha\left(\hat{t},\Theta_\alpha(\hat{t})\right)&=\begin{bmatrix}
-A_\alpha\\ -K_\alpha -B_\alpha \hat{E}_y
\end{bmatrix}\,;\\
\sigma_\alpha\left(\hat{t},\Theta_\alpha(\hat{t})\right)\sigma_\alpha^{\mathrm{T}}\left( 
\hat{t},\Theta_\alpha(\hat{t})\right)&=D_\alpha\,,\enskip \alpha=i,e\,;
\end{align*} 
$W(\hat{t})$~--- стандартный винеровский случайный процесс.
\pagebreak

Для нахождения значений вектора состояния~$\Theta_\alpha(\hat{t})$ применим явную разностную 
схему стохастического метода Эйлера~\cite{16-k}:
\begin{multline*}
\Theta_\alpha^{n+1}=\Theta_\alpha^n +h_\tau a_\alpha \left( \hat{t}_n, \Theta_\alpha^n\right)+\sigma_\alpha 
\left( \hat{t}_n, \Theta_\alpha^n\right)\Delta W_n\,,\\ 
n=0,\ldots , N\,,\ \alpha=i,e\,,
%\label{e6-k}
\end{multline*}
где $\Theta_\alpha^n$, $n=0,\ldots , N$,~--- приближенное значение вектора 
состояния~$\Theta_\alpha(\hat{t})$, $\alpha=i,e$, в момент времени $\hat{t}\hm=\hat{t}_n$, 
$\hat{t}_n\hm=n h_\tau$, $n=0,\ldots , N$; $h_\tau$~--- достаточно малый шаг интегрирования; $\Delta 
W_n$, $n=0,\ldots ,N$,~--- величина приращения винеровского процесса~$W(\hat{t})$ на отрезке $\left[ 
\hat{t}_n,\,\hat{t}_{n+1}\right]$, по определению независимая от~$\Theta_\alpha^0$, 
$\Delta W_0,\ldots , 
\Delta W_{n-1}$: $\Delta W_n\hm=W(\hat{t}_{n-1})\hm-W(\hat{t}_n)$; $\Delta W_n\hm\sim N(0,\,h_\tau)$, 
т.\,е.\ $\Delta W_n$ представляют собой гауссовские случайные величины с нулевыми математическими 
ожиданиями и дисперсиями, равными шагу интегрирования; $\Theta_\alpha^0$~--- значение вектора 
состояния $\Theta_\alpha(\hat{t})$, $\alpha\hm=i,e$, в момент времени $\hat{t}=0$, 
$\Theta_\alpha^0\hm\sim \hat{f}_\alpha^{\mathrm{maksv}}$. 

Частные производные $\partial^2\hat{g}_\alpha/\partial[\hat{v}_y]^2$ и $\partial \hat{h}_\alpha/\partial 
\hat{v}_y$, являющиеся составляющими матрицы $\sigma_\alpha (\hat{t}_n, 
\Theta_\alpha^n)\sigma_\alpha^{\mathrm{T}}(\hat{t}_n,\Theta_\alpha^n)$ и вектора $a_\alpha(\hat{t}_n, 
\Theta_\alpha^n)$ соответственно, аппроксимируются со вторым порядком точности на трехточечном 
шаблоне на основе значений~$\hat{g}_\alpha$ и~$\hat{h}_\alpha$~\cite{17-k}.
      
      В выражения для функций~$\hat{g}_\alpha$ и~$\hat{h}_\alpha$ входят интегралы, которые 
вычисляются методом Мон\-те-Кар\-ло с использованием набора значений скоростной компоненты 
вектора состояния~$\hat{v}_y$, полученных из решения СДУ Ито:
      \begin{equation*}
      \int \hat{f}_\alpha \left\vert \hat{v}_y-
\hat{v}_y^\prime\right\vert\,dv_y^\prime=M\left(\zeta\left(\hat{V}_y\right)\right)\,,
\end{equation*}
где
$$
      \zeta\left(\hat{V}_y\right)=\left\vert \hat{v}_y-\hat{V}_y\right\vert\,,\enskip \hat{V}_y\sim 
\hat{f}_\alpha\,.
  $$
      
      Для вычисления напряженности самосогласованного электрического поля $\hat{E}_y=-
\partial\hat{\varphi}/\partial\hat{y}$, входящей в вектор $a_\alpha(\hat{t}_n, \Theta_\alpha^n)$, необходимо 
аналогично аппроксимировать со вторым порядком точности производную 
$\partial\hat{\varphi}/\partial\hat{y}$ на трехточечном шаблоне с использованием значений 
потенциала~$\hat{\varphi}$~\cite{17-k}. Значения потенциала~$\hat\varphi$ находятся из решения 
уравнения Пуассона. 
      
      Граничную задачу для уравнения Пуассона 
      \begin{align*}
      \fr{\partial^2 \hat\varphi}{\partial \hat{y}^2} & = -\left(\hat{n}_i-\hat{n}_e\right)\,;\\
      \hat{\varphi}\big|_{\hat{y}=0} &=\hat{\varphi}_p\,;\\
      \hat{\varphi}\big|_{\hat{y}_\infty=0} &=0
      \end{align*}
    предлагается решать путем перехода к конечно-разностной системе с последующим ее решением 
методом прогонки~\cite{17-k}:

\noindent
\begin{gather*}
\hat{\varphi}^n_{l-1}+2\hat{\varphi}_l^n+\hat{\varphi}^n_{l+1}=
h_y\hat{\delta}_l^n\,,\enskip l=1,\ldots , 
N_y\,;\\
\hat{\delta}_l^n=-\left( \hat{n}^n_{i,l}-\hat{n}^n_{e,l}\right)\,;\enskip 
\hat{\varphi}_0=\hat{\varphi}_p\,;\enskip \hat{\varphi}_{N_y}=0\,,
\end{gather*}
где $N_y$~--- число шагов по переменной~$\hat{y}$, $h_y$~--- величина шагов разбиения по~$\hat{y}$. 
      
      Концентрации $\hat{n}_\alpha$, $\alpha=i,e$, и плотности токов частиц на зонд~$\hat{f}_\alpha$, 
$\alpha=i,e$, вычисляются согласно описанному выше методу Мон\-те-Карло.

\section{Применение метода расщепления и~метода крупных~частиц}

Решение задачи в данном случае предлагается начать с записи правой части уравнения 
Фок\-ке\-ра--План\-ка в декартовой системе координат в виде:
$$
\mathbf{Q} f_\alpha = \fr{1}{2}\,\fr{\partial^2 f_\alpha}{\partial [v_y]^2}\,\fr{\partial^2 g_\alpha}{\partial 
[v_y]^2}+\fr{\partial f_\alpha}{\partial v_y}\,\fr{\partial C_\alpha}{\partial v_y}+H_\alpha\,,\enskip 
\alpha=i,e\,,
$$  
где 
\begin{align*}
C_\alpha(\vec{r},\vec{v},t)&=
\begin{cases}
\fr{1-\gamma}{Z_i^2}\int\fr{f_e(\vec{r},{\vec{v}}^{\,\prime},t)}{|\vec{v}-{\vec{v}}^{\,\prime} |}\,d{\vec{v}}^{\,\prime}\,, 
&\alpha=i\,;\\[9pt]
\fr{Z_i^2(\gamma-1)}{\gamma}\int \fr{f_i(\vec{r},{\vec{v}}^{\,\prime}, t)}
{|\vec{v}-{\vec{v}}^{\,\prime} 
|}\,d{\vec{v}}^{\,\prime}\,, &\alpha=e\,;
\end{cases} 
\\
H_\alpha&=
\begin{cases}
4\pi \left( \fr{\gamma f_e}{Z_i^2}+f_i\right)f_i\,, & \alpha=i\,;\\[9pt]
4\pi\left(\fr{Z_i^2 f_i}{\gamma}+f_e\right)f_e\,, &\alpha=e\,.
\end{cases}
\end{align*}
Тогда при переходе к безразмерным величинам (см.\ разд.~3) система~(\ref{e1-k}) запишется 
следующим образом:
      \begin{equation}
      \left.
\!\!\begin{array}{l}
      \fr{\partial 
\hat{f}_\alpha}{\partial\hat{t}}+A_\alpha\fr{\partial\hat{f}_\alpha}{\partial\hat{y}}+
B_\alpha  \hat{E}_y
\fr{\partial\hat{f}_\alpha}{\partial\hat{v}_\alpha}=\tilde{\mathbf{Q}}\hat{f}_\alpha\,,\enskip 
\alpha=i,e;\\[9pt]
      \fr{\partial^2\hat{\varphi}}{\partial\hat{y}^2}=-\left( \hat{n}_i-\hat{n}_e\right)\,,\enskip \hat{E}_y=-
\fr{\partial\hat\varphi}{\partial\hat{y}}\,,\\[9pt]
\hspace*{3.1mm}\hat{t}=0:\ \hspace*{2.6mm}\hat{f}_\alpha(\hat{y},\hat{v}_y, 0)=\hat{f}_\alpha^{\mathrm{maksv}}\,,\enskip \alpha=i,e\,,\\[9pt]
\hspace*{2.9mm} \hat{y}=0:\ \hspace*{2.8mm}\hat{f}_\alpha(0,\hat{v}_y,\hat{t})=0\,,\enskip \alpha=i,e\,;\\[9pt]
\hspace*{24.3mm}\hat\varphi(0,\hat{t})=\hat{\varphi}_p\,;\\[9pt]
      \hat{y}=\hat{y}_\infty:\ \hat{f}_\alpha(\hat{y}_\infty, 
\hat{v}_y,\hat{t})=\hat{f}_\alpha^{\mathrm{maksv}}\,,\enskip \alpha=i,e\,;\\[9pt]
\hspace*{21.5mm}\hat{\varphi}(\hat{y}_\infty,\hat{t})=0\,,\\[9pt]
    \end{array}
\right\}\!\!
\label{e7-k}
\end{equation}
где 
\begin{gather*}
\tilde{\mathbf{Q}} \hat{f}_\alpha=D_\alpha\fr{\partial^2\hat{f}_\alpha}{\partial 
[\hat{v}_y]^2}+K_\alpha\fr{\partial\hat{f}_\alpha}{\partial\hat{v}_y}+H_\alpha\,;\\
D_\alpha=A_g^\alpha\fr{\partial^2\hat{g}_\alpha}{\partial [\hat{v}_y]^2}\,;\enskip 
K_\alpha=A_h^\alpha \fr{\partial \hat{h}_\alpha}{\partial\hat{v}_y}\,,\ \alpha=i,e\,.
\end{gather*}

Для решения системы уравнений~(\ref{e7-k}) применяется модификация метода 
расщепления~\cite{17-k}, согласно которой исходная задача разбивается на две вспомогательные. Такое 
разбиение можно осуществить, переписав уравнение Фок\-ке\-ра--План\-ка в следующем виде:
$$
\fr{\partial\hat{f}_\alpha}{\partial\hat{t}} =
\tilde{\mathbf{Q}}_1\hat{f}_\alpha+\tilde{\mathbf{Q}}_2\hat{f}_\alpha\,,
$$
где 
\begin{align*}
\tilde{\mathbf{Q}}_1\hat{f}_\alpha &=-
\left(A_\alpha\fr{\partial\hat{f}_\alpha}{\partial\hat{y}}+
B_\alpha\fr{\partial\hat{f}_\alpha}{\partial\hat{y}}
\right)\,;\\
\tilde{\mathbf{Q}}_2\hat{f}_\alpha 
&=\left(D_\alpha\fr{\partial^2\hat{f}_\alpha}{\partial[\hat{v}_y]^2}+K_\alpha\fr{\partial 
\hat{f}_\alpha}{\partial\hat{v}_y}+H_\alpha\right)\,.
\end{align*}

      Правая часть уравнения Фок\-ке\-ра--План\-ка представляет собой сумму двух операторов, 
первый из которых отвечает за перенос частиц, второй~--- за столкновения заряженных частиц. 
В~результате образуются следующие задачи, которые решаются последовательно:
      \begin{itemize}
\item первая задача:
\begin{align*}
&\fr{\partial w_\alpha(\hat{y},\hat{v}_y,\hat{t})}{\partial\hat{t}} =\mathbf{Q}_1 
w_\alpha(\hat{y},\hat{v}_y,\hat{t})\,,\enskip \alpha=i,e\,;\\[9pt]
&\fr{\partial^2\hat\varphi}{\partial\hat{y}^2}=-\left(\hat{n}_i-\hat{n}_e\right)\,;\enskip
\hat{E}_y=-
\fr{\partial\hat\varphi}{\partial\hat{y}}\,;\\[9pt]
&w_\alpha(\hat{y},\hat{v}_y,\hat{t}^n)=\hat{f}_\alpha(\hat{y},\hat{v}_y,\hat{t}^n)\,,\enskip n=0,\ldots ,N-
1\,;\\[9pt]
&\hspace{2.9mm}\hat{y}=0:\ \hspace*{2.9mm}w_\alpha(0,\hat{v}_y,\hat{t})=0\,,\enskip \alpha=i,e\,;\\[9pt]
&\hspace*{25.1mm}\hat\varphi(0,\hat{t})=\hat{\varphi}_p\,;\\[9pt]
&\hat{y}=\hat{y}_\infty:\ w_\alpha(\hat{y}_\infty, \hat{v}_y, \hat{t})=
\hat{f}_\alpha^{\mathrm{maksv}}\,,\enskip 
\alpha=i,e\,;\\[9pt]
&\hspace*{22.5mm}\hat\varphi(\hat{y}_\infty,\hat{t})=0\,;
\end{align*}
\item вторая задача:
\begin{align*}
\!\!\!\!\!\!\!\fr{\partial s_\alpha(\hat{y},\hat{v}_y,\hat{t})}{\partial \hat{t}} &=\mathbf{Q}_2 
s_\alpha(\hat{y},\hat{v}_y,\hat{t})\,, & \alpha&=i,e\,;\\
\!\!\!\!\!\!\!s_\alpha (\hat{y},\hat{v}_y,\hat{t}^n) &=w_\alpha (\hat{y},\hat{v}_y, \hat{t}^{n+1}),& n&=0,\ldots ,N-
1.
\end{align*}
\end{itemize}

Первая задача представляет собой систему безразмерных уравнений Вла\-со\-ва--Пуас\-со\-на. Для ее 
решения применяется метод крупных частиц~\cite{18-k}. Согласно этому методу решение задачи 
осуществляется путем расщепления на два этапа: на первом этапе не учитываются конвективные члены 
и решение получается обычным интегрированием на неподвижной эйлеровой сетке, а на втором этапе 
рассматривается система, которая описывает перенос частиц в лагранжевой системе координат. Кроме 
того, на первом этапе необходимо решить уравнение Пуассона для получения значений потенциала 
самосогласованного электрического поля. Для этого применяется метод, описанный в разд.~3. 

Вторая задача решается путем перехода к ко\-неч\-но-раз\-ност\-ной сис\-те\-ме. При этом частные 
производные $\partial^2\hat{g}_\alpha/\partial[\hat{v}_y]^2$ и $\partial\hat{h}_\alpha/\partial\hat{v}_y$ 
аппроксимируются со вторым порядком точности с использованием трехточечного шаблона, а 
производная $\partial s_\alpha/\partial\hat{t}$ аппроксимируется на двухточечном шаблоне с первым 
порядком точности~\cite{16-k}. К~полученной системе разностных уравнений предлагается применить 
один из классических методов решения систем линейных уравнений, например метод 
Гаусса~\cite{19-k}.
      
      Решением первой задачи является функция $w_\alpha(\hat{y}, \hat{v}_y, \hat{t}^n)$, 
$n\hm=0,\ldots ,N$, , которая дает начальное условие для второй задачи. Решая вторую задачу, находим 
функцию $s_\alpha(\hat{y},\hat{v}_y,\hat{t}^n)\hm=\hat{f}_\alpha(\hat{y},\hat{v}_y,\hat{t}^n)$, 
$n=1,\ldots ,N$, $\alpha=i,e$, которая определяет решение $\hat{f}_\alpha(\hat{y},\hat{v}_y,\hat{t}^n)$, 
$\alpha=i,e$, исходной системы~(\ref{e7-k}) для рассматриваемых моментов времени $n=1,\ldots ,N$.

Моменты функций распределения $\hat{f}_\alpha$, $\alpha=i,e$, находятся с помощью методов 
численного интегрирования, например метода трапеций~\cite{19-k}.

\section{Результаты численного моделирования}

Для двух описанных выше методов реализованы две отдельные программы в среде {Matlab~7.0}. 
Эти программы позволяют по заданным значениям концентраций и температур частиц $n_{i\infty}$, 
$n_{e\infty}$, $T_{i\infty}$ и~$T_{e\infty}$ в невозмущенной плазме, а также потенциала~$\varphi_p$, 
подаваемого на зонд, изучить эволюцию во времени плотностей тока частиц~$j_i$ и~$j_e$, концентраций 
частиц~$n_i$  и~$n_e$ в произвольной точке пространства в возмущенной зоне, а также динамику 
изменения напряженности~$E_y$ самосогласованного электрического поля во времени и пространстве.

С использованием разработанных программ проведены серии расчетных экспериментов, в которых 
значение концентраций варьировалось в пределах $n_{i\infty} \hm = n_{e\infty}\hm =10^{18}\div 
10^{22}$~м$^{-3}$. Значение температур было выбрано неизменным и равным $T_{i\infty}\hm = 
T_{e\infty}\hm=3000$~K, а значения потенциала, подаваемого на зонд, изменялись в пределах 
$\varphi_p\hm=0\div 2{,}6$~В.

На рис.~1  и~2 приведены графики изменения напряженности самосогласованного электрического
 поля (см.\ рис.~1) и плотности токов ионов (см.\linebreak\vspace*{-12pt}

\pagebreak

\end{multicols}

\begin{figure} %fig1
\vspace*{1pt}
\begin{center}
\mbox{%
\epsfxsize=162.594mm
\epsfbox{kud-1.eps}
}
\end{center}
\vspace*{-9pt}
\Caption{Динамика изменения плотности тока ионов во времени в фиксированной точке возмущенной 
зоны для значений потенциала: \textit{1}~--- $\varphi_p=-6$; 
\textit{2}~--- $\varphi_p=-16$; \textit{3}~--- $\varphi_p=- 30$ 
в случае применения методов Монте-Карло~(\textit{а}) 
и крупных частиц~(\textit{б})}
\end{figure}

\begin{figure} %fig2
\vspace*{1pt}
\begin{center}
\mbox{%
\epsfxsize=162.713mm
\epsfbox{kud-2.eps}
}
\end{center}
\vspace*{-9pt}
\Caption{Динамика изменения напряженности электрического поля во времени в фиксированной точке 
возмущенной зоны для значений потенциала: 
\textit{1}~--- $\varphi_p=-6$; \textit{2}~--- $\varphi_p=-16$; 
\textit{3}~--- $\varphi_p=-30$ в случае применения методов Монте-Карло~(\textit{а}) и
крупных частиц~(\textit{б})
}
\end{figure}

\begin{multicols}{2}

\noindent
 рис.~2) во времени в фиксированной точке пространства 
возмущенной зоны в случае применения обоих разработанных алгоритмов.


На основании полученных результатов можно отметить похожее поведение зависимостей 
напряженности электрического поля и плотности тока от времени в двух рассматриваемых случаях. 
Графики кривых сначала убывают, затем начинают возрастать, выходя в некоторый момент 
времени~$t^\prime$ (момент установления) на стационарные значения. 

Одинаковое поведение 
напряженности и плот\-ности тока можно объяснить из следующих соображений: плотность тока ионов в 
данной области пространства равна произведению концентрации ионов на их направленную скорость и 
на заряд иона. Скорость ионов, в свою очередь, зависит от заряда, массы и напряженности 
электрического поля. 
%\columnbreak

При внесении в плазму отрицательно заряженного зонда возникает электрическое поле, которое 
нарушает квазинейтральность плазмы. Для того чтобы компенсировать действие внешнего 
электрического поля, ионы устремляются к зонду, а электроны~--- от зонда. Это приводит к дисбалансу 
концентраций вблизи зонда и, как следствие, к увеличению разности потенциалов; график 
напряженности электрического поля убывает. Вскоре разделение зарядов компенсирует внешнее 
электрическое поле; график выходит на стационарное значение. 

Также можно отметить, что значения 
напряженности электрического поля и плотности тока частиц на зонд в момент установления для двух 
методов совпадают. 

Момент установления~$t^\prime$ зависит от при\-ме\-ня\-емо\-го метода решения. В~случае метода 
Мон\-те-Кар\-ло $t^\prime=3{,}5\div 4$~ед., а для метода крупных частиц совместно с методом 
расщепления $t^\prime\hm=5\div 5{,}5$~ед. Используя ко\-неч\-но-раз\-ност\-ный метод, можно 
получить динамику изменения функций распределения частиц~$f_\alpha$, $\alpha=i,e$, во времени и 
пространстве. Функции распределения позволяют наглядно представить влияние на картину 
распределения частиц вблизи зонда самой поверхности зонда и электрического поля.

\section{Заключение}
      
      В работе найдено решение задачи диагностики плоским зондом сильноионизованной плазмы с 
учетом столкновений заряженных частиц. Разработана математическая модель исследуемого явления, 
описываемая уравнениями Фок\-ке\-ра--План\-ка и Пуассона. Решение получено двумя методами:\linebreak 
статистическим и ко\-неч\-но-раз\-ност\-ным на основе\linebreak сформированных алгоритмов. Приведены 
резуль-\linebreak таты численного моделирования при различных\linebreak характерных параметрах задачи.
 Из  проведенных 
вычислительных экспериментов вытекает, что искомые величины: напряженность 
электрического поля, плотности токов частиц на зонд, концентрации частиц вблизи зонда~--- как по 
характеру зависимости, так и по числовым значениям совпадают. При применении метода 
      Мон\-те-Кар\-ло момент установления наступает быстрее по сравнению с конечно-разностным 
методом, однако конечно-разностный метод позволяет получить более наглядные результаты.

{\small\frenchspacing
{%\baselineskip=10.8pt
\addcontentsline{toc}{section}{Литература}
\begin{thebibliography}{99}

\bibitem{1-k}
\Au{Alexeff I., Anderson T.}
Experimental and theoretical results with plasma antenna~// IEEE Trans. Plasma Sci., 2006. Vol.~34. 
No.\,2. P.~166--172.

\bibitem{2-k}
\Au{Сысун В.\,И.}
Сильноионизованная низкотемпературная плазма в приборах электронной техники: Методы 
исследования, свойства, применение. Дисс. \ldots д-ра физ.-мат. наук в форме науч. докл.: 
01.04.08.~--- Пет\-ро\-за\-водск, 1996.

\bibitem{3-k}
\Au{Тухас В.\,А.}
Методология создания средств измерений и испытаний на устойчивость к кондуктивным помехам~// 
Мат-лы VI Междунар. симп. по электромагнитной совместимости и 
электромагнитной экологии.~--- СПб., 2005. С.~231--234.

\bibitem{4-k}
\Au{Гудзенко Л.\,И., Яковленко С.\,И.}
Плазменные лазеры.~--- М.: Атомиздат, 1978.  256~с.

\bibitem{5-k}
\Au{Звелто О.}
Принципы лазеров.~--- М.: Мир, 1990.  560~с.

\bibitem{6-k}
\Au{Сысун В.\,И., Хромой Ю.\,Д.}
Расширение канала мощного импульсного разряда в парах ртути~// Электронная техника, 1974. 
Сер.~4. Вып.~10. С.~80--85. 

\bibitem{7-k}
\Au{Винклер Дж.\,Р.}
Искусственные пучки частиц в космической плазме.~--- М.: Мир, 1985.  451~с.

\bibitem{8-k}
\Au{Bernstein I.\,B., Rabinowitz I.\,N.}
Theory of electrostatic probes in low-density plasma~// Phys. Fluids, 1959. Vol.~2. No.\,2. P.~112--121. 

\bibitem{9-k}
\Au{Альперт Я.\,Л., Гуревич А.\,В., Питаевский~Л.\,П.}
Искусственные спутники в разреженной плазме.~--- М.: Наука, 1964.  282~с.

\bibitem{10-k}
\Au{Чан П., Тэлбот Л., Турян~К.}
Электрические зонды в неподвижной и движущейся плазме.~--- М.: Мир, 1978.  202~с.

\bibitem{11-k}
\Au{Алексеев Б.\,В., Котельников В.\,А.}
Зондовый метод диагностики плазмы.~--- М.: Энергоатомиздат, 1989.  240~с.

\bibitem{12-k}
\Au{Пантелеев А.\,В., Кудрявцева И.\,А.}
Формирование математической модели двухкомпонентной плазмы с учетом столкновений 
заряженных частиц в случае плоского зонда~// Теоретические вопросы вычислительной техники и 
программного обеспечения: Межвузовский сб. научн. тр.~--- М.: МИРЭА, 2006. С.~11--21.

\bibitem{13-k}
\Au{Олдер Б.}
Вычислительные методы в физике плазмы.~--- М.: Мир, 1974.  111~с.

\bibitem{14-k}
\Au{Montgomery D.\,C., Tidman D.\,A.}
Plasma kinetic theory.~--- New York, 1964. 

\bibitem{15-k}
\Au{Кудрявцева И.\,А., Пантелеев А.\,В.}
Применение метода Мон\-те-Кар\-ло для анализа поведения двухкомпонентной плазмы с учетом 
столкновений между заряженными частицами~// Теоретические вопросы\linebreak
вычислительной техники и 
программного обеспечения: Межвузовский сб. научн. тр.~--- М.: МИРЭА, 2008. С.~122--128. 

\bibitem{16-k}
\Au{Семенов В.\,В., Пантелеев А.\,В., Руденко~Е.\,А., Бор\-та\-ков\-ский~А.\,С.}
Методы описания, анализа и синтеза нелинейных систем управления.~--- М.: МАИ, 1993.  312~с.

\bibitem{17-k}
\Au{Киреев В.\,И., Пантелеев А.\,В.}
Численные методы в примерах и задачах.~--- М.: Высшая школа, 2006.  480~с.

\bibitem{18-k}
\Au{Белоцерковский О.\,М., Давыдов~Ю.\,М.}
Метод крупных частиц в газовой динамике. Вычислительный эксперимент.~--- М.: Наука, 
Физматгиз, 1982.

\label{end\stat}

\bibitem{19-k}
\Au{Вержбицкий В.\,М.}
Основы численных методов.~--- М.: Высшая школа, 2002.  840~с.
 \end{thebibliography}
}
}


\end{multicols}         %7

\def\stat{koltsov}

\def\tit{ИСПОЛЬЗОВАНИЕ МЕТРИК ПРИ~СРАВНИТЕЛЬНОМ 
ИССЛЕДОВАНИИ КАЧЕСТВА РАБОТЫ АЛГОРИТМОВ 
СЕГМЕНТАЦИИ ИЗОБРАЖЕНИЙ}

\def\titkol{Использование метрик при~сравнительном 
исследовании качества работы алгоритмов 
сегментации изображений}

\def\autkol{П.\,П.~Кольцов}
\def\aut{П.\,П.~Кольцов$^1$}

\titel{\tit}{\aut}{\autkol}{\titkol}

%{\renewcommand{\thefootnote}{\fnsymbol{footnote}}\footnotetext[1]
%{Работа выполнена при финансовой поддержке РФФИ (грант 11-01-00515).}}

\renewcommand{\thefootnote}{\arabic{footnote}}
\footnotetext[1]{Научно-исследовательский институт системных исследований Российской академии наук, 
koltsov@niisi.msk.ru}
 
\vspace*{-6pt}

  \Abst{Изучается качество работы четырех известных алгоритмов цифровой 
сегментации изображений. Исследование проводится на совокупности искусственных 
тестовых изображений, подвергаемых контролируемым искажениям при априори известном 
эталонном \textit{ground truth} изображении. Результат работы алгоритмов сегментации 
сравнивается с эталонным изображением с помощью метрик, облада\-ющих различными 
свойствами. Использование различных метрик для оценки качества работы алгоритмов 
сегментации и сравнение полученных при этом результатов позволяют более точно выяснить 
особенности каждого из исследуемых алгоритмов.}
  
  \KW{обработка изображений; оценка качества обработки изображений; сегментация 
изображений; выделение границ; энергетические методы}

  \vskip 12pt plus 9pt minus 6pt

      \thispagestyle{headings}

      \begin{multicols}{2}
      
            \label{st\stat}

  
  \section{Введение}
  
  В данной статье под сегментацией подразумевается разбиение изображения 
на совокупность непересекающихся связных областей, для которых характерно 
повышенное сходство между элементами одной и той же области по сравнению 
с прилежащим фоном или соседними областями. В~ряде случаев, например при 
решении задач текстурного анализа, таким областям могут отвечать 
определенные объекты или их части. Методы, предназначенные для решения 
такого класса задач, будем называть методами сегментации, а их программные 
реализации~--- алгоритмами сегментации. Таким образом, в статье под задачей 
сегментации подразумевается задача разбиения исходного изображения на 
вышеуказанные области.
  
  В~настоящее время разработка методов сегментации является одним из 
быстроразвивающихся направлений в области обработки изображений,\linebreak
широко 
востребованным в практической деятель\-ности. Однако значительное число 
существующих в настоящее время как методов сегментации, так и их 
программно-алгоритмических реализаций, порождает проблему выбора 
алгоритма, наиболее подходящего для решаемой конкретной задачи.
  
  Очевидно, что при выборе того или иного алгоритма необходимо 
руководствоваться его свойствами, позволяющими судить об ожидаемом 
качестве решения этим алгоритмом задачи сегментации. Качество решения 
задачи сегментации в статье будет определяться через оценку точности работы 
алгоритма сегментации.
  
  В статье сравнительное исследование точности работы различных 
алгоритмов сегментации проводится на наборе эта\-лон\-ных/тес\-то\-вых 
изображений, подвергаемых контролируемым искажениям. В~предположении, 
что наиболее существенные свойства тестируемых алгоритмов проявляются 
при обработке типичных и трудных для метода ситуаций, необходимым 
требованием к тестовым изоб\-ра\-же\-ни\-ям является возможность формирования\linebreak 
таких ситуаций с достаточной полнотой. Определение ситуаций, трудных и 
вместе с тем типичных для изучаемых алгоритмов сегментации, является 
содержательной задачей. Обычным методом решения задач такого рода служит 
обращение к опыту исследователя. Искомые ситуации определяются в процессе 
анализа большого числа примеров сегментации изображений различными 
методами и их реализациями. 

На основе полученных результатов анализа и 
создается набор искусственных тестовых изображений, которые в явном виде и 
с некоторой полнотой моделируют трудные ситуации.
  
  Для более полной оценки качества работы алгоритмов сегментации на наборе 
тестовых изображений они подвергаются контролируемым искажениям. 
Количественная оценка качества работы алгоритмов осуществляется с 
помощью различных метрик. Такой подход позволяет выявить особенности 
работы алгоритмов, определить границы их применимости и динамику качества 
работы. По результатам сравнительного исследования построены графики, 
отображающие количественное со\-по\-став\-ле\-ние изучаемых алгоритмов 
сегментации.

\begin{figure*}[b] %fig1
\vspace*{-7pt}
\begin{center}
\mbox{%
\epsfxsize=162.622mm
\epsfbox{kol-1.eps}
}
\end{center}
\vspace*{-9pt}
\Caption{Исходное изображение~(\textit{а}); сегментная карта~(\textit{б}); исходное 
изображение с регионами и границами~(\textit{в})
}
\end{figure*}
  
  Сравнительное исследование алгоритмов сегментации выполнено на 
реализациях четырех известных методов сегментации, основанных на\linebreak решении 
минимизационной задачи для функционала, обычно называемого в 
соответствии со своим содержательным смыслом энергетическим.
  
  Статья имеет следующую структуру.
  
  В разд.~2 кратко описана и проиллюстрирована примерами задача 
сегментации изображений.
  
  В разд.~3 описывается использованный подход к оценке качества 
сегментации.
  
  В разд.~4 приводятся наборы искусственных изоб\-ра\-же\-ний, использованных 
для тестирования алгоритмов сегментации.
  
  В разд.~5 излагаются методики анализа результатов сегментации, кратко 
описаны применяемые метрики.
  
  В разд.~6 ссылочно описаны алгоритмы сегментации, подвергнутые 
сравнительному исследованию.
  
  В разд.~7 приводятся результаты исследования.
  
  Раздел~8 посвящен выводам.
  
  \vspace*{-6pt}
  
  \section{Задача сегментации изображений}
  
  Основными понятиями, используемыми при решении задач сегментации 
изображений на основе некоторой однородности являются сегмент, регион, 
сегментная карта, границы. Результатом работы алгоритма сегментации (так 
называемого сегментатора) над исходным изображением является новое 
изображение~--- сегментная карта, содержащая области равномерной закраски. 
На рис.~1, взятом из статьи~[1], приведены примеры изображений сегментной 
карты и регионов. Исходное изображение показано на рис.~1,\,\textit{а}. 
Результатом применения алгоритма сегментации является изображение 
рис.~1,\,\textit{б}~--- сегментная карта. На ней можно видеть однотонно 
закрашенные области. Такие области на сегментной карте называются 
сегментами. При наложении сегментной карты на исходное изображение 
границы ее сегментов оконтуривают на нем области, которые называются 
регионами. Границами регионов служат границы сегментов. Для удобства 
рассмотрения на рис.~1,\,\textit{в} границы регионов нанесены на исходное 
изображение.


  Таким образом, алгоритм сегментации строит по исходному изображению 
сегментную карту и выполняет разбиение исходного изображения на регионы. 
Создание сегментной карты на основе исходного изображения является общим 
свойством всех сегментаторов, которые будут рассмотрены в статье.
  
  Количество сегментов, получаемых на сегментной карте, определяется 
особенностями конкретного сегментатора. В~наиболее простых алгоритмах 
число сегментов задается априорно, в более сложных оно определяется 
автоматически. В~некоторых случаях вводится ограничение (обычно сверху) на 
число сегментов.
  
  Сегментаторы, базирующиеся на энергетическом функционале, строят как 
монохромные, так и цветные сегментные карты. Цвета и уровни яр\-кости 
сегментов, назначаемые сегментатором, являются условными и могут быть 
достаточно далеки от цветов и яркостей соответствующих регионов исходного 
изображении. В~сущности, яркость или цвет, который сегментатор назначает 
данному сегменту, является просто номером этого сегмента на сегментной 
карте.
  
  Обычно сегменты строятся в зависимости от уровня яркости областей 
изображения, их цвета, текстуры или размера. В~основном исследователей 
интересуют те классы задач, в которых сегментация приводит к выделению 
значимых смысловых объектов.

  \vspace*{-6pt}
  
  \section{Подход к оценке качества сегментации}

 \begin{figure*}[b] %fig2
\vspace*{9pt}
\begin{center}
\mbox{%
\epsfxsize=141.364mm
\epsfbox{kol-2.eps}
}
\end{center}
\vspace*{-6pt}
\Caption{Пример сегментации изображения с линейно изменяющейся яркостью}
\end{figure*}

  Подход, развиваемый в данной статье, состоит в следующем.
  \begin{enumerate}[1.]
\item Качество сегментаторов оценивается по результатам их работы на 
искусственных тестовых изображениях, специально сконструированных 
таким образом, чтобы моделировать трудные ситуации, в которых 
сегментаторы допускают много ошибок. При этом имеется заданный априори 
эталонный результат <<хорошей>> сегментации.
\item Для сравнения результата сегментации с эталоном используются 
различные метрики. В~описываемой работе используются две метрики. Они 
обладают различными свойствами, и соответственно результаты, 
получаемые с помощью этих метрик, оказываются разными. Именно 
благодаря этому обстоятельству, сравнивая результаты измерения с 
помощью двух разных мет\-рик, можно более полно установить свойства 
сегментаторов и более точно оценить качество сегментации.
\item Динамика качества работы сегментаторов определяется путем 
сравнения результата сегментации на серии тестовых изображений с 
вносимыми искажениями относительно априори заданного результата 
сегментации исходного, неискаженного изображения. В~качестве искажения 
изображений в статье рассматривается их зашумление и размытие.
   \end{enumerate}
   
   Таким образом, описанный выше подход позволяет не только выполнить 
сравнительную оценку качества работы различных сегментаторов, но и 
выявить особенности поведения такой оценки при последовательном 
искажении входного изображения.
  
  \section{Наборы искусственных изображений}
  
  Вначале рассмотрим приведенный на рис.~2 пример достаточно простого 
изображения, но пред\-став\-ля\-юще\-го при этом сложности для работы 
сегментаторов.
        

На рис.~2,\,\textit{а} показано изображение, на котором по вертикали яркость 
неизменна. Вдоль горизонтального направления яркость изменяется от нуля на 
левом крае до~200 на правом (максимум яркости в данном формате 
изображения равен~255). Возрастание яркости происходит линейно с 
точ\-ностью до дискретизации.
  
  При правильной сегментации такого изображения должен получаться один 
сегмент. Однако практика показывает, что алгоритмы сегментации создают в 
таких областях с медленно меняющейся яркостью несколько сегментов. 
Типичный пример такой сегментации показан на рис.~2,\,\textit{б}. На рисунке не 
показаны сами сегменты, а только их границы, которые традиционно 
называются <<ложными границами>>. Приведенный пример иллюстрирует 
трудность полутоновых изображений, в которых имеет место плавное 
изменение яркости, для правильной работы сегментаторов.
  
  Основой совокупности тестовых изображений в данной статье является база 
данных системы \mbox{PICASSO}~[2], в которой аккумулированы типичные 
ситуации, присутствующие на реальных изоб\-ра\-же\-ни\-ях и представляющие 
трудность для работы различных методов обработки изображений. В~данной 
статье будет рассмотрено два класса тес\-то\-вых изоб\-ра\-же\-ний, взятых из базы 
данных сис-\linebreak\vspace*{-12pt}
\pagebreak

\end{multicols}

\begin{figure} %fig3
\vspace*{1pt}
\begin{center}
\mbox{%
\epsfxsize=141.365mm
\epsfbox{kol-3.eps}
}
\end{center}
\vspace*{-11pt}
\Caption{\textit{Круг} яркости~150 на фоне~30~(\textit{а}) и \textit{Угол} 80$^\circ$ 
яркости~50 на  фоне~200~(\textit{б})
}
%\end{figure*}
%\begin{figure*} %fig4
\vspace*{9pt}
\begin{center}
\mbox{%
\epsfxsize=163.994mm
\epsfbox{kol-4.eps}
}
\end{center}
\vspace*{-11pt}
\Caption{Изображения \textit{Step}~(\textit{а}), \textit{Junction}~(\textit{б}), 
\textit{Snail}~(\textit{в}) и \textit{Roof}~(\textit{г})
}
%\end{figure*}
%\begin{figure*} %fig5
\vspace*{9pt}
\begin{center}
\mbox{%
\epsfxsize=163.994mm
\epsfbox{kol-5.eps}
}
\end{center}
\vspace*{-11pt}
\Caption{Сегментация при гауссовом зашумлении, девиация зашумления~4
}
\end{figure}


\begin{multicols}{2}

\noindent 
те\-мы \mbox{PICASSO}, которые можно условно назвать
<<простыми>> и <<сложными>>. В~простых изображениях нет плавных 
изменений яркости, все границы резко очерчены, число регионов~--- два. 
Сложные изображения являются полутоновыми.
  
  На рис.~3 приведены примеры простых изображений, использованных в 
статье~--- \textit{Круг} и \textit{Угол}.
      
  На рис.~4 приведены примеры сложных изображений, использованных в 
статье,~--- \textit{Step}, \textit{Junction},  \textit{Snail}  и~\textit{Roof}.
   Для каждого тестового изображения из обоих классов было построено по два 
одно\-па\-ра\-мет\-ри\-че\-ских семейства новых изображений. 

Первое семейство 
строилось путем добавления к исходным изображениям гауссова шума. 
Значение девиации шума~$\sigma$ служило параметром семейства. Второе 
семейство строилось путем гауссова размытия тех же изображений. 
Параметром семейства служил радиус окна размытия~$r$.
  
  В качестве иллюстрации на рис.~5 показан типичный пример работы 
сегментатора на зашумленных сложных изображениях. Сегментные карты на 
рисунке опущены, показаны только границы полученных сегментов.



  \section{Методики анализа результатов сегментации}
  
  Поскольку различные сегментаторы по-разному закрашивают сегменты, а 
даже один и тот же сегментатор может изменять закраску сегментов при 
внесении искажений в изображения, то для сравнения результатов сегментации 
необходимо перейти к более универсальным объектам, чем сегмент. Очевидно, 
что таким наиболее естественным объектом является граница сегмента, которая 
представляет собой набор точек, никак не зависящих от закраски сегментов. 
Именно границы сегментов и будут использоваться далее при сравнительном 
исследовании алгоритмов сегментации. Отметим, что гомогенность сегментов, 
получаемых при обработке изображений исследуемым классом сегментаторов, 
оказывается очень удобной для выполнения процедуры выделения границ, а 
именно:
  \begin{itemize}
\item задача выделения границ сегментов и, соответственно, регионов 
практически тривиальна, если получена сегментная карта;
\item будучи границами двумерных однородных областей, такие границы 
являются непрерывными;
\item двумерная однородная область обладает ориентацией. Это означает, 
что для нее можно определить направление обхода граничного контура. 
После этого можно достаточно легко отследить все ее граничные точки, 
организуя их в одномерную кривую. При этом граница трактуется как 
упорядоченный массив.
\end{itemize}

  В~статье сравнение результатов сегментации основывается на измерении 
расстояний между кривыми. Для этого используется среднее расстояние~$d$ и 
хаусдорфово расстояние~$\chi$.
  
  Напомним, что среднее расстояние $d(X, Y)$ и хаусдорфово расстояние 
$\chi(X, Y)$ между множествами~$X$ и~$Y$ определяются следующим 
образом:
  \begin{align*}
  d(X,Y) &=\fr{1}{2}\left[ 
\fr{1}{N_X}\sum\limits_{x\in X}\rho(x,Y)+\fr{1}{N_Y}\sum\limits_{y\in 
Y}\rho(y,X)\right]\,;\\
  \chi(X,Y) &= \max\left[\max\limits_{x\in X} 
\rho(x,Y),\,\max\limits_{y\in Y}\rho(y,X)\right]\,.
  \end{align*}
    Здесь $\rho(x, Y)$~--- расстояние от точки $x\in X$ до множества~$Y$: 
$\rho(x,Y)\hm=\min\limits_{y\in Y}\rho(x,y)$, а $\rho(x,y)$~--- обычное 
  евклидово расстояние между точками~$x$ и~$y$, $N_X$ и $N_Y$~--- чис\-ло 
точек в~$X$ и~$Y$.
  
  Если величина $\chi(X, Y)$ дает скорее максимальное расстояние между 
точками множеств, то $d(X, Y)$ дает некоторое среднее расстояние и является 
тем, что принято называть термином <<мера различия>> (discrepancy 
measure). Очевидно, $d(X, Y) \hm\geq 0$, $d(X, X) \hm=0$ и $d(X, Y) \hm= d(Y, 
X)$.
  
  Рассмотрим подробнее свойства расстояния $d(X, Y)$ и его сходство и 
различие с расстоянием $\chi(X, Y)$.
  
  Если $X$ и $Y$~--- одноточечные множества на плоскости, то $d(X, Y)$ 
совпадает с обычным евклидовым расстоянием между точками~$X$ и~$Y$. 
Если~$X$ и~$Y$~--- отрезки, являющиеся противоположными сторонами 
прямоугольника, то $d(X, Y)$ совпадает с обычным расстоянием между этими 
сторонами.
  
  Пусть множество~$X$ состоит из одной точки~$x$, а $Y$~--- из двух точек: 
$x$ и $y$, причем евклидово расстояние между $x$ и $y$ равно~$r$. В~этом 
случае $\chi(X, Y) = r$.
  
  Найдем среднее расстояние. Поскольку $\rho(x, Y) \hm= 0$ в силу того, что $x\in 
Y$, а $\rho(y, X) = r$, то очевидно, что $d(X, Y) \hm= r/4$. Здесь учтено, что $N_Y \hm= 
2$. Если бы точка~$x$ не принадлежала множеству~$Y$, то среднее 
расстояние, как и хаусдорфово, было бы равно~$r$. Таким образом, наличие во 
множестве~$Y$ удаленной точки~$y$ не так сильно сказывается на среднем 
расстоянии, как на хаусдорфовом.
  
  Сходство и различие между $\chi$ и~$d$ проиллюстрировано на рис.~6. 
Здесь множество $X$~--- отрезок~$AB$, а множество~$Y$~--- отрезок~$CD$.

  Отметим, что если множества~$X$ и~$Y$~--- точки граничных кривых двух 
разных изображений и координаты точек задаются как координаты пикселов, 
то и расстояния~$d$ и~$\chi$ также измеряются в пикселах. Поскольку в 
процессе нахождения этих расстояний используется евклидова метрика с 
вы-\linebreak

\begin{center} %fig6
\vspace*{9pt}
\mbox{%
\epsfxsize=67.361mm
\epsfbox{kol-6.eps}
}
\end{center}
\begin{center}
%\vspace*{6pt}
{{\figurename~6}\ \ \small{Сходство и различие между $\chi$ и~$d$}}
\end{center}
%\vspace*{9pt}

%\smallskip
\addtocounter{figure}{1}



\noindent
числением квадратного корня, то величины~$d$ и~$\chi$ могут быть 
нецелыми числами.
  
  Далее будем считать, что $X$~--- это множество наперед заданных 
граничных точек на эталонном (ground truth) изображении, а $Y$~--- 
множество граничных точек, которые получены после работы сегментатора. 
В~идеальном случае множество~$X$ должно совпадать с~$Y$, но в типичном 
случае множество~$Y$ хотя и близко к~$X$, но имеет некоторое количество 
точек, удаленных от~$X$.
  
  Из рассмотренных выше примеров можно сделать следующие 
предположения относительно качества работы сегментатора:
  \begin{enumerate}[1.]
  \item  Если обе величины $\chi(X, Y)$ и $d(X, Y)$ в некотором естественном 
смысле малы, то полученные сегментатором границы близки к границам на 
эталонном изображении, причем на сегментной карте, построенной 
сегментатором, практически нет ни ложных границ, ни прочих посторонних 
сегментов, отсутствующих на эталонном изображении.
  \item Если значение $\chi(X, Y)$ велико, а $d(X, Y)$~--- мало, то на 
сегментной карте имеются посторонние сегменты, например шумовые пятна, в 
том числе удаленные от границ на эталонном изображении, но их размер 
невелик.
  \item  Если и $\chi(X, Y)$, и $d(X, Y)$ велико, то либо размер посторонних 
сегментов большой, либо граничные кривые значительно удалены друг от 
друга.
  \item Ситуация, когда среднее расстояние велико, а хаусдорфово~--- мало, по 
всей видимости, невозможна.
  \end{enumerate}
  
  В качестве иллюстрации приведем результаты типичной работы 
сегментатора на простом изображении \textit{Круг}, подвергнутом гауссову 
зашумлению. На рис.~7 представлены только границы полученных сегментов. 
Как видно, из-за наличия зашумления появились мелкие посторонние 
сегменты. Идеальной же границей эталонного изображения в данном случае 
является окружность.


  Отметим, что в хаусдорфовой метрике расстояние между фигурой на рис.~7 
и окружностью может быть велико, поскольку посторонние объекты 
значительно удалены от окружности. Однако среднее расстояние оказывается 
небольшим вследствие усреднения, поскольку размер посторонних сегментов 
мал.
  
  Таким образом, по паре расстояний~$\chi$ и~$d$ можно более полно судить 
о качестве проведенной сегментации, чем только по одному из них.
  
  На основании вышесказанного была использована следующая методика 
сравнительного исследования сегментаторов. Пусть $I$~--- некоторое исход\-ное 
изображение, а $I_\sigma$~--- изображение с до\-бав\-лен\-ным гауссовым шумом с 
девиацией~$\sigma$, $\{I_\sigma\}$~--- множество изображений~$I_\sigma$ с 
различными значениями девиации~$\sigma$, принадлежащими некоторому 
конечному набору. Пусть в результате обработки сегментатором множества 
$\{I_\sigma\}$ получается множество сегментных карт~$S\{I_\sigma\}$. Находя 
на каждой сегментной карте из~$S\{I_\sigma\}$ граничные точки, получаем 
множество граничных точек сегментов~$B(S\{I_\sigma\})$. Обозначим через 
$B(I)$ множество граничных точек исходного эталонного изображения~$I$. 
Вычисляем среднее расстояние $d = d(B(I), B(S\{I_\sigma\})$ и хаусдорфово 
$\chi = \chi(B(I), B(S\{I_\sigma\})$ для каждого значения девиации~$\sigma$. 
Полученные значения $d$ и~$\chi$ как функции~$\sigma$ могут быть 
изображены в виде графиков для последующего анализа.
  
  Если при сравнительном исследовании используется несколько исходных 
изображений, то функции $d = d(\sigma)$ и $\chi = \chi(\sigma)$ вычисляются 
отдельно для каждого из них и усредняются для каждого из значений 
девиации~$\sigma$ по всем исходным изображениям. Такое усреднение 
способствует более объективному отражению особенностей исследуемого 
сегментатора.
  
  Методика сравнительного исследования сегментаторов при размытии 
изображений совпадает с вышеизложенной для зашумления, в которой вмес\-то 
девиации шума~$\sigma$ рассматривается радиус окна гауссова размытия~$r$.


  
  Рассмотрим отдельно вопрос о выборе множества граничных точек~$B(I)$ 
исходного эталонного изображения~$I$. В~случае, когда $I$ относится к 
классу простых изображений, например \textit{Угол} или \textit{Круг}, для него 
легко построить идеальные граничные линии эталонного изображения и 
определить их в качестве~$B(I)$. Но для изображений из класса\linebreak
\begin{center} %fig7
\vspace*{1pt}
\mbox{%
\epsfxsize=53.905mm
\epsfbox{kol-7.eps}
}
\end{center}
%\begin{center}
\vspace*{6pt}
{{\figurename~7}\ \ \small{Пример сегментации простого изображения \textit{Круг} при зашумлении}}
%\end{center}
%\vspace*{9pt}

%\smallskip
\addtocounter{figure}{1}

\noindent
 сложных, 
таких как \textit{Step}, \textit{Junction}, \textit{Snail} и \textit{Roof}, даже при 
обработке неискаженного изображения все алгоритмы сегментации дают 
границы, сильно отличающиеся от предполагаемых идеальных. Причина 
состоит в наличии областей с медленно изменяющейся яркостью, обработка 
которых создает для всех методов сегментации значительные трудности. 
Поэтому для данного класса тестовых изображений в работе, представленной в 
статье, исследуется вопрос, насколько результат сегментации искаженного 
сложного изображения отличается от результата сегментации исходного 
изображения. Иными словами, на изображениях \textit{Step}, \textit{Junction}, 
\textit{Snail} и \textit{Roof} изучается устойчивость сегментатора. В~этом случае 
в качестве~$B(I)$ берутся границы, полученные при сегментации исходного 
изображения~$I$. Таким образом, методика тестирования на простых и 
сложных изображениях отличается, и поэтому графики для изображений 
\textit{Круг} и \textit{Угол} и для изображений \textit{Step}, \textit{Junction}, 
\textit{Snail} и \textit{Roof} будут приведены отдельно.
  
  Необходимо отметить, что эксперименты с оценкой качества работы 
сегментаторов показали, что они обладают определенной нестабильностью 
работы. Небольшие изменения при переходе от одного тестового изображения 
к последующему могут приводить к заметным изменениям всего результата 
сегментации. Например, при линейном увеличении максимальной яркости 
левого изображения на рис.~2 число ложных границ может в некоторых 
пределах скачкообразно и увеличиваться, и уменьшаться. Аналогичное явление 
происходит и при сегментации зашумленных и размытых изоб\-ра\-же\-ний. Такая 
ситуация приводит к тому, что графики $d(\sigma)$ и $\chi(\sigma)$ получаются 
негладкими и неудобными для анализа. Поэтому в качестве графиков строились 
более информативные линии тренда, пред\-став\-ля\-ющие собой стандартное 
полиномиальное приближение исходных точек.
  
  \section{Тестируемые методы сегментации}
  
  Изложенная выше методика исследования качества работы сегментаторов 
была опробована на примере хорошо известных реализаций четырех методов 
сегментации, базирующихся на разных видах энергетического функционала, 
описание которых можно найти в~[1--8].
  
  \subsection{Сегментатор JSEG} %6.1
  
  Сегментатор JSEG~[9] ориентирован на автоматическую сегментацию изображений и 
видео, которые могут содержать цветные регионы и текстуры. Обработка 
изображения состоит из двух независимых шагов: цветовой квантизации и 
пространственной сегментации. Собственно сегментация выполняется с 
использованием метода растущих областей. В~чис\-ле опций сегментатора 
имеются опции обработки полутоновых изображений, в том числе и 
бестекстурных, что и было использовано. Рабочие параметры сегментатора 
могут устанавливаться самим сегментатором. В~ходе исследования качества 
работы этого сегментатора такие па\-ра\-мет\-ры и были использованы.
  
  \subsection{Сегментатор EDISON} %6.2
  
  Сегментатор EDISON~\cite{10-kol} выполняет сегментацию изображений, выделение границ, а 
также фильтрацию шума, сохраняющую резкие перепады яркости изоб\-ра\-же\-ния. 
Алгоритм, реализованный в сегментаторе, определяет границы на изображении 
и использует их в процессе сегментации. Одним из основных параметров 
сегментатора является минимальный размер региона в пикселах, который 
может создать данный метод. При исследовании значения этого параметра 
брались равными~100 и~1000. Соответственно, при исследовании 
сегментаторы обозначались как {EDISON~100} и {EDISON~1000}. 
Остальные рабочие параметры фиксировались так же, как и у авторов 
сегментатора.
  
  \subsection{Сегментатор EDGEFLOW} %6.3
  
  Для сегментации и выделения границ изображения сегментатор EDGEFLOW~\cite{11-kol}
  реализует 
метод потока граничных точек. Суть его состоит в том, что в каждой точке 
изображения вычисляется направление изменения яркости, цвета или текстуры. 
Это позволяет формировать векторное поле потока граничных точек. 
Интегральные кривые этого поля, проходя через области изображения, 
образуют гомогенные регионы, сталкиваются друг с другом и, стабилизируясь, 
формируют границы регионов. Сегментатор имеет параметр, существенно 
влияющий на его работу~--- так называемое смещение. При исследовании 
качества работы этого сегментатора значения параметра брались равными~10 
и~26. Соответственно при исследовании сегментаторы обозначались как 
{EDGEFLOW~10} и {EDGEFLOW~26}.

  \begin{figure*}[b] %fig8
  \vspace*{1pt}
\begin{center}
\mbox{%
\epsfxsize=162.93mm
\epsfbox{kol-8.eps}
}
\end{center}
\vspace*{-6pt}
  \Caption{Зашумление простых изображений: \textit{1}~--- {JSEG}; \textit{2}~--- 
{EDGEFLOW~26}; \textit{3}~--- {EDISON~1000}; \textit{4}~--- 
{EDGEFLOW~10}; \textit{5}~--- {EDISON~100}; \textit{6}~--- 
{MULTISCALE}
  }
  \end{figure*}
  
  \subsection{Сегментатор MULTISCALE} %6.4
  
  При обработке сегментатором MULTISCALE~\cite{12-kol}
  изображение вначале анализируется в более 
грубом масштабе, а затем~--- в более мелком. При рассмотрении изоб\-ра\-же\-ния в 
грубом масштабе шум и помехи мало заметны. При рассмотрении изображения 
в более мелком масштабе лучше заметны детали объектов. Объединение этих 
двух подходов позволяет фильт\-ро\-вать шум и сохранять важные детали 
изображения. Сегментатор имеет большое число рабочих параметров. 
В~соответствии с рекомендациями, приведенными в описании сегментатора, 
при исследовании качества его работы были взяты значения, описанные как 
безопасные.
  
  \section{Результаты тестирования}
  
  \subsection{Зашумление изображений} %7.1
  
  Для тестирования качества работы сегментаторов на простых и сложных 
изображениях к ним аддитивно был добавлен гауссов шум с девиацией от~0 
до~30 с шагом~1. Таким образом было получено множество тестовых 
изображений.
  
  На рис.~8 приведены графики~$d(\sigma)$ и~$\chi(\sigma)$ по всем 
исследованным сегментаторам при обработке простых изображений 
\textit{Круг} и \textit{Угол} (см.\ рис.~3). Граничные кривые, полученные после 
сегментации данных изображений, сравнивались с границами эталонных 
изображений, которые очевидны и не приводятся. Здесь и далее графики 
представлены как линии тренда результатов измерений. Количество маркеров 
на графиках уменьшено для большей разборчивости.
  
  Из анализа графиков можно видеть, что на данных изображениях 
сегментатор {JSEG} резко выделяется среди остальных. Графики для 
{JSEG} практически лежат на горизонтальной оси, так что в данном 
диапазоне зашумлений этот метод работает очень хорошо по сравнению со 
всеми остальными. Кроме этого, оба расстояния~$d$ и~$\chi$ для {JSEG} 
малы. Поскольку полученные при работе этого сегментатора границы 
сравниваются с идеальными границами эталонных изображений, можно 
сделать вывод, что сегментатор {JSEG} хорошо сохраняет форму границ.
  
  В целом при увеличении зашумления как среднее расстояние~$d$, так и 
хаусдорфово расстояние~$\chi$ растут для всех исследуемых сегментаторов. 
Спад некоторых графиков при больших значениях девиации шума не должен 
вводить в заблуждение, поскольку граничные кривые очень сильно искажены, 
что делает результаты сегментации недостоверными. 

Можно заключить, что 
сегментатор {EDISON} лучше использовать на простых изображениях при 
значениях девиации не более 15--20, а сегментатор {EDGEFLOW}~--- при 
значениях девиации не более 3--5. Сегментатор {MULTISCALE} 
достаточно устойчив к зашумлению простых изображений: замедление роста 
графика среднего расстояния и некоторая тенденция к спаду начинается при 
значениях девиации шума 20--25.

  
  На рис.~9 приведены графики~$d(\sigma)$ и $\chi(\sigma)$ по всем 
исследованным сегментаторам при обработке сложных изображений 
\textit{Step}, \textit{Junction}, \textit{Snail} и \textit{Roof} (см.\ рис.~4). Как уже 
было сказано ранее, ground truth изображения, полученные после 
сегментации исходных изображений, в данном случае являлись эталоном для 
нахождения среднего и хаусдорфова расстояния.

  \begin{figure*} %fig9
  \vspace*{1pt}
\begin{center}
\mbox{%
\epsfxsize=162.93mm
\epsfbox{kol-9.eps}
}
\end{center}
\vspace*{-6pt}
  \Caption{Зашумление сложных изображений:
  \textit{1}~--- {JSEG};
    \textit{2}~--- {EDGEFLOW~26};
  \textit{3}~--- {EDISON~1000}; 
\textit{4}~--- {EDGEFLOW~10}; 
 \textit{5}~--- {EDISON~100}; 
\textit{6}~--- {MULTISCALE}
  }
%\vspace*{6pt}
  \end{figure*}
  
  \begin{figure*}[b] %fig10
\vspace*{1pt}
\begin{center}
\mbox{%
\epsfxsize=162.93mm
\epsfbox{kol-10.eps}
}
\end{center}
\vspace*{-6pt}
\Caption{Размытие простых изображений:
\textit{1}~--- {JSEG};
\textit{2}~--- {EDGEFLOW~26}; 
  \textit{3}~--- {EDISON~1000}; 
   \textit{4}~--- {EDGEFLOW~10};
  \textit{5}~--- {MULTISCALE}
  }
  \end{figure*}
  
  Из анализа графиков на рис.~9 видно, что наибольшие значения расстояний 
получились для сегментатора {EDGEFLOW~26}, причем графики\linebreak 
начинают спадать при значениях девиации 7--8. Это,
по-видимому, и есть тот 
уровень зашумления, до которого целесообразно использовать данный 
сег\-мен\-та\-тор. Величины расстояний для \mbox{EDGEFLOW}~10 меньше, однако 
графики перестают расти и начинают спадать приблизительно при той же 
величине шума. Сегментатор {MULTISCALE} несколько более устойчив к 
зашумлению: спад графиков начинается при значениях девиации порядка~20. 
Но большие в целом значения расстояний~$d$ и~$\chi$ говорят о том, что при 
увеличении зашумления изображения этот сегментатор создает много лишних 
крупных сегментов. Сегментаторы {EDISON} и {JSEG} показали 
наилучшие результаты. Их графики лежат ниже графиков остальных методов. 
Графики сегментатора {EDISON} начинают спадать при значениях 
девиации 15--20. По-видимому, это предельное зашумление, до которого 
целесообразно использовать этот сегментатор. У~сегментатора {JSEG} 
показатели несколько лучше.
  
  Сопоставим теперь графики для простых изоб\-ра\-же\-ний и сложных. 
  %
  Нетрудно 
видеть, что в обоих случаях лучшие показатели у сегментатора {JSEG}, к 
которым приближаются показатели сегментатора {EDISON}. Оценки для 
предельных уровней шума, с которым методы еще можно использовать, тоже 
примерно одинаковы.

\vspace*{-6pt}
  
  \subsection{Размытие изображений}
  
  \vspace*{-2pt}
  
  Для тестирования качества работы сегментаторов при размытии 
изображений был создан набор простых изображений, для которых выполнено 
гауссово размытие с радиусом окна от~0 до 12~пикселов с шагом 0,4~пиксела и 
набор сложных изображений, для которых выполнено гауссово размытие с 
радиусом окна от~0 до 3~пикселов с шагом 0,1~пиксела.
  
  На рис.~10 приведены графики $d(r)$ и~$\chi(r)$ по всем 
исследуемым алгоритмам при сегментации простых изображений \textit{Круг} 
и \textit{Угол}.

  \begin{figure*} %fig11
  \vspace*{1pt}
\begin{center}
\mbox{%
\epsfxsize=161.93mm
\epsfbox{kol-11.eps}
}
\end{center}
\vspace*{-12pt}
  \Caption{Размытие сложных изображений: 
  \textit{1}~--- \textit{JSEG};
   \textit{2}~--- {EDGEFLOW~26};  \textit{3}~--- {EDISON~1000};
 \textit{4}~--- {EDGEFLOW~10};  
 \textit{5}~--- {EDISON~100};
\textit{6}~--- {MULTISCALE}
  }
    \vspace*{-6pt}
  \end{figure*}

  Как видно из анализа полученных графиков, наилучшее качество показал 
сегментатор {EDISON}: у него значения как среднего, так и хаусдорфова 
расстояния наименьшие по сравнению с другими методами. Это означает 
точное сохранение формы границ при размытии исходного изображения. 
Сегментатор {EDGEFLOW} продемонстрировал приемлемое качество при 
значениях радиуса окна размытия не более 3--4, после чего на сегментной карте 
возникало значительное количество лишних сегментов. На это обстоятельство 
указывают большие значения как среднего, так и хаусдорфова расстояния. 
Сегментатор {JSEG} хорошо работает до значений радиуса 8--9. 
Примерно такие же показатели у сегментатора {MULTISCALE}. Однако 
следует еще раз подчеркнуть, что все эти результаты относятся к простым 
изображениям, на которых не создаются ложные границы.
  
  На рис.~11 приведены графики $d(r)$ и $\chi(r)$ при обработке 
сложных изображений \textit{Step}, \textit{Junction}, \textit{Snail} и \textit{Roof}, 
содержащих области с медленно изменяющейся яркостью. Как видно, для всех 
исследуемых алгоритмов сегментация сложных изоб\-ра\-же\-ний оказалась более 
трудной задачей. Обратим внимание на диапазоны значений радиуса окна 
размытия на графиках рис.~10 и~11. Если на рис.~10 радиус изменялся от~0 до 
12~пикселов, то для получения приемлемого качества сегментации сложных 
изображений (см.\ рис.~11) пришлось ограничиться максимальным значением 
радиуса в 3~пиксела. Отметим, что при радиусе размытия в 2--3~пиксела 
значения всех расстояний, приведенных для сегментаторов на рис.~10, в 
несколько раз меньше расстояний, приведенных на рис.~11.
 
  Анализ графиков на рис.~11 показывает, что при тестировании на размытых 
сложных изображениях наилучшее качество показал сегментатор {JSEG}: 
у него значения как среднего, так и хаусдорфова расстояния наименьшие по 
сравнению с другими сегментаторами. Это означает более точное сохранение 
сегментатором формы границ при размытии исходного изображения. 
Сегментатор {EDGEFLOW} может успешно работать лишь при значениях 
радиуса окна размытия не более 1--1,5~пиксела. При больших значениях 
радиуса размытия начинает значительно изменяться количество и форма 
сегментов. Сегментаторы {EDISON} и {MULTISCALE} близки по 
своим показателям. Однако большие значения хаусдорфова расстояния при 
существенно меньших (в 10--20~раз) значениях среднего расстояния указывают 
на большое число мелких искажений (изломов) граничных линий. Сохранение 
границ ухудшается при значениях радиуса окна размытия, уже больших 
1~пиксела.
 
  \section{Выводы}
  
  \noindent
  \begin{enumerate}[1.]
  \item Применение для измерения результатов сегментации двух метрик с 
различными свойствами~--- среднего и хаусдорфова расстояния~--- позволяет 
более точно оценить качество\linebreak
работы сегментаторов. Подчеркнем, что при этом 
не идет речь об определении лучшей из двух метрик. Существенно то, что обе 
метрики используются одновременно и результаты измерения сопоставляются.
  \item  Кроме хаусдорфова и среднего расстояния, могут быть и другие 
способы сравнения граничных кривых. Например, подсчет числа точек на 
кривых, количества точек ветвления, гистограммы распределения кривизны 
кривых и~т.\,п. Каждый из таких способов дает некоторую меру различия двух 
наборов кривых (граничных линий сегментов). Численное значение каждой 
такой меры является характеристикой пары изображений, а их совокупность 
задает вектор характеристик. Для такого многомерного вектора характеристик 
можно выполнить исследование, аналогичное вышеизложенному, или 
воспользоваться иными оригинальными методами. Использование таких 
методов (которые можно назвать мультиметрическими) может позволить еще 
точнее оценить свойства сегментаторов.
  \item  При тестировании конкретных сегментаторов оказалось, что одной из 
наиболее важных проблем исследованных методов сегментации является 
проблема создания ложных границ на изображениях с медленно изменяющейся 
яркостью. В~этой связи разработка методов и их программных реализаций, 
сводящих к минимуму количество ложных границ, представляется актуальной.
  \item  Исследование качества работы ряда популярных сегментаторов 
показало, что они ведут себя неустойчиво при зашумлении и размытии 
изоб\-ра\-же\-ния. Другими словами, результат сегментации даже слегка 
зашумленного или размытого изображения может существенно отличаться от 
результата сегментации исходного изображения. Таким образом, можно 
заключить, что целесообразно до процедуры сегментации выполнить очистку 
изображения от шума и повысить его четкость.
  \item  В статье приведены численные оценки степени искажения 
изображений, при которой работа сегментаторов остается удовлетворительной. 
В~этой связи становится актуальной\linebreak
задача оценивания (параметрического для 
рассмотренных в статье случаев) искажений для изображения, подвергаемого 
сегментации.
  \end{enumerate}
  

  
  {\small\frenchspacing
{%\baselineskip=10.8pt
\addcontentsline{toc}{section}{Литература}
\begin{thebibliography}{99}
  
  \bibitem{1-kol}
  \Au{Deng Y., Manjunath B.\,S.}
  Unsupervised segmentation of color-texture regions in images and video~// IEEE 
Transactions on Pattern Analysis and Machine Intelligence (PAMI'01), 2001. 
Vol.~23. No.\,8. P.~800--810.
  
  \bibitem{2-kol}
  \Au{Gribkov I.\,V., Koltsov P.\,P., Kotovich~N.\,V., Kravchenko~A.\,A., 
Kutsaev~A.\,S., Nikolaev~V.\,K., Zakharov~A.\,V.}
  PICASSO~--- a system for evaluating edge detection algorithms~// Pattern 
Recognition and Image Analysis, 2003. Vol.~13. No.\,4. P.~617--622.

  \bibitem{5-kol} %3
  \Au{Mumford D.}
  The Bayesian rationale for energy functionals~// Geometry driven diffusion in 
computer vision~/ Ed. B.~Romeny.~--- Dordrecht: Kluwer Academic, 1994. 
P.~141--153.
  
  \bibitem{6-kol} %4
  \Au{Kervrann C., Hoebeke M., Trubuil~A.}
  A level line selection approach for object boundary estimation~// 7th IEEE 
Conference (International) on Computer Vision, ICCV'99.~--- Kerkyra: IEEE 
Computer Society Press, 1999. P.~963--968.

  \bibitem{4-kol} %5
  \Au{Ma W.-Y., Manjunath B.\,S.}
  EdgeFlow: A technique for boundary detection and image segmentation~// IEEE 
Transactions on Image Processing, 2000. Vol.~9. P.~1375--1388.

  \bibitem{3-kol} %6
  \Au{Meyer F., Vachier C.}
  Image segmentation based on viscous flooding simulation~//  ISMM'02 Proceedings.~---  
Sydney: CSIRO, 2002. P.~69--77.
  
  \bibitem{7-kol}
  \Au{Christoudias C.\,M., Georgescu~B., Meer~P.}
  Synergism in low level vision~// 16th Conference (International ) on Pattern 
Recognition.~--- Quebec City: IEEE Computer Society Press, 2002. Vol.~4. 
  P.~150--155.
  
  \bibitem{8-kol}
  \Au{Sumengen B., Manjunath~B.\,S.}
  Multi-scale edge detection and image segmentation~//  European Signal Processing 
Conference (EUSIPCO) Proceedings.~--- Antalya,\linebreak 2005. {\sf 
http://vision.ece.ucsb.edu/publications/\linebreak 05eusipcoBarisMultiscale.pdf}.
  
  \bibitem{9-kol}
  Сегментатор JSEG. {\sf http://vision.ece.ucsb.edu/\linebreak segmentation/jseg/}.
  
  \label{end\stat}
  
  \bibitem{10-kol}
  Сегментатор EDISON. {\sf http://www.caip.rutgers.edu/\linebreak riul/research/code/EDISON/}.
  
  \bibitem{11-kol}
  Сегментатор EDGEFLOW. {\sf http://vision.ece.ucsb.edu/\linebreak segmentation/edgeflow}.
  
  \bibitem{12-kol}
  Сегментатор MULTISCALE. {\sf http://barissumengen.\linebreak com/seg/}.
 \end{thebibliography}
}
}


\end{multicols}         %8
\def\stat{kor-kor}



\def\tit{МОДИФИЦИРОВАННЫЙ СЕТОЧНЫЙ МЕТОД РАЗДЕЛЕНИЯ ДИСПЕРСИОННО-СДВИГОВЫХ
СМЕСЕЙ НОРМАЛЬНЫХ ЗАКОНОВ$^*$}



\def\titkol{Модифицированный сеточный метод разделения дисперсионно-сдвиговых
смесей нормальных законов}

\def\aut{В.\,Ю.~Королев$^1$,  А.\,Ю.~Корчагин$^2$}

\def\autkol{В.\,Ю.~Королев,  А.\,Ю.~Корчагин}

\titel{\tit}{\aut}{\autkol}{\titkol}

{\renewcommand{\thefootnote}{\fnsymbol{footnote}} \footnotetext[1]
{Работа поддержана Российским научным фондом (проект 14-11-00364).}}


\renewcommand{\thefootnote}{\arabic{footnote}}
\footnotetext[1]{Факультет
вычислительной математики и кибернетики Московского государственного
университета им.\ М.\,В.~Ломоносова; Институт проблем информатики
Российской академии наук; victoryukorolev@yandex.ru}
\footnotetext[2]{Факультет вычислительной математики и кибернетики
Московского государственного университета им.\ М.\,В.~Ломоносова;
sasha.korchagin@gmail.com}

%\vspace*{2pt}



\Abst{Описывается модифицированный двухэтапный
сеточный метод разделения дис\-пер\-си\-он\-но-сдви\-го\-вых смесей нормальных
законов, представляющий собой альтернативу чистому ЕМ (expectation-maximization)
ал\-го\-рит\-му. На
первом этапе этого алгоритма строится дискретная аппроксимация для
смешивающего распределения, на втором этапе подбирается абсолютно
непрерывное распределение из заранее заданного семейства, например,
обобщенных обратных гауссовских законов, ближайшее к~дискретному
распределению, полученному на первом этапе. Обсуждаются вопросы
сходимости этого двухэтапного алгоритма. Доказана монотонность
сеточного итерационного метода, используемого на первом этапе.
Подробно обсуждается вопрос оптимального выбора параметров метода,
прежде всего сетки, накидываемой на носитель смешивающего
распределения. С~этой целью предложены статистические оценки
квантилей смешивающего распределения. Эффективность метода
иллюстрируется примерами конкретных вычислений оценок параметров
обобщенных гиперболических распределений.}

\KW{смесь распределений вероятностей;
дис\-пер\-си\-он\-но-сдви\-го\-вая смесь нормальных законов; обобщенное
гиперболическое распределение; ЕМ-ал\-го\-ритм; сеточный метод
разделения смесей}

\vspace*{1pt}

%\vspace*{2pt}

\DOI{10.14357/19922264140402}


\vskip 12pt plus 9pt minus 6pt

\thispagestyle{headings}

\begin{multicols}{2}

\label{st\stat}

\section{Введение}

При {\it практическом} решении задачи моделирования и исследования
волатильности (изменчивости) хаотических стохастических процессов
ключевым этапом является статистическое разделение смесей
вероятностных распределений. Задача разделения смесей~---
статистического оценивания параметров смесей вероятностных
распределений~--- в~деталях разобрана, например, в~книге~\cite{k2011}.

Для решения задачи разделения смесей вероятностных распределений
традиционно используются итерационные процедуры типа ЕМ-ал\-го\-рит\-ма.
К~сожалению, классический ЕМ-ал\-го\-ритм обладает рядом серьезных
недостатков при его применении к~смесям нормальных законов, а~именно:
он демонстрирует крайнюю неустойчивость по отношению к~исходным
данным и~начальным приближениям.

Для преодоления этих недостатков
предложено много модификаций ЕМ-ал\-го\-рит\-ма (см., например,~\cite{k2011}).
Вместе с тем в~указанной книге предложен и~исследован
принципиально новый~--- сеточный~--- метод приближенного решения
задачи разделения смесей. В~работе~\cite{n2013} подробно исследованы
вопросы сходимости сеточных методов разделения смесей.

В соответствии с подходом к~статистическому анализу хаотических
стохастических процессов, в~частности к~решению задачи декомпозиции
волатильности таких процессов, развитом в~книге~\cite{k2011},
в~общем случае на практике приходится решать задачу разделения
конечных смесей нормальных законов с~произвольно большим числом
неизвестных параметров (параметров компонент и~их весов).
И~хотя в~большинстве приложений возникают смеси не более чем с~пятью--семью
компонентами, даже при использовании таких смесей, скажем, в~задачах
анализа и~прогнозирования финансовых рисков приходится моделировать
траекторию движения точки в~пространствах, размерность которых
соответственно лежит в~пределах от~14 (для пятикомпонентных смесей)
до~20 (для семикомпонентных смесей), что существенно увеличивает
вычислительные и~временн$\acute{\mbox{ы}}$е ресурсы, необходимые для практического
решения указанных задач.

Поскольку во многих ситуациях (например,
при прогнозировании на основе высокочастотных данных) эти задачи
необходимо решать в~режиме, близком к~реальному времени, для
создания эффективных методов статистического анализа на основе
смешанных моделей на первый план выходит проб\-ле\-ма снижения
размерности решаемой задачи, т.\,е.\ параметрического пространства.

Одним из возможных подходов к~снижению размерности является
априорное сужение классов допусти\-мых смесей. К~примеру, при решении
многих задач, связанных с~анализом процессов атмосферной или
плазменной турбулентности, а~так\-же процессов, описывающих эволюцию
различных финансовых индексов, высочайшую адекватность
продемонстрировали модели, основанные на дис\-пер\-си\-он\-но-сдви\-го\-вых
смесях нормальных законов. Класс таких смесей очень обширен
и,~в~част\-ности, включает в~себя обобщенные гиперболические распределения,
которые были введены О.-Е.~Барн\-дорфф-Ниль\-се\-ном в~1977--1978~гг.\ как
класс специальных сдвиг-мас\-штаб\-ных смесей нормальных законов~\cite{BN1977, BN1978}.
Пусть $\alpha\hm\in\r$, $\beta\hm\in\r$. Если
функцию распределения обобщенного гиперболического закона
с~параметрами~$\alpha$, $\beta$, $\nu$, $\mu$, $\lambda$ обозначить
$P_{GH}(x;\alpha,\beta,\nu,\mu,\lambda)$, то по определению
\begin{multline}
P_{GH}(x;\alpha,\beta,\nu,\mu,\lambda)={}\\
{}=
\int\limits_{0}^{\infty}\Phi\left(\fr{x-\beta-\alpha
z}{\sqrt{z}}\right)\,p_{GIG}(z;\nu,\mu,\lambda)\,dz\,,\\
x\in\r\,,
\label{e1-kor}
\end{multline}
где $\Phi(x)$~--- стандартная нормальная функция распределения:
$$
\Phi(x)=\int\limits_{-\infty}^{x}\varphi(z)\,dz\,,\enskip
\varphi(x)=\fr{1}{\sqrt{2\pi}}e^{-x^2/2}\,,\enskip  x\in\mathbb{R}\,;
$$
$p_{GIG}(x;\nu,\mu,\lambda)$~--- плот\-ность обобщенного обратного
гауссовского распределения:
\begin{multline*}
p_{GIG}(x;\nu,\mu,\lambda)={}\\
{}=\fr{\lambda^{\nu/2}}{2\mu^{\nu/2}
K_{\nu}\left(\sqrt{\mu\lambda}\right)}\,
x^{\nu-1}\exp\left\{-\fr{1}{2}\left(\fr{\mu}{x}+\lambda
x\right)\right\}\,,\\ x>0\,.
\end{multline*}
Здесь $\nu\in\r$;
$$
\begin{array}{lll}
\mu>0\,, & \lambda\geqslant0\,, & \mbox{если }\nu<0\,;\\[6pt]
\mu>0\,, & \lambda>0\,, & \mbox{если }\nu=0\,;\\[6pt]
\mu\geqslant0\,, & \lambda>0\,, & \mbox{если }\nu>0\,;
\end{array}
$$
$K_{\nu}(z)$~--- модифицированная бесселева функция третьего рода
порядка~$\nu$:

\noindent
\begin{multline*}
K_{\nu}(z)=\fr{1}{2}\int\limits_{0}^{\infty}y^{\nu-1}\exp
\left\{-\fr{z}{2}\left(y+\fr{1}{y}\right)\right\}\,dy\,,\\
z\in\mathbb{C}\,,\enskip \mathrm{Re}\,z>0\,.
\end{multline*}
Обратим внимание, что в~(1) смешивание происходит одновременно и~по
параметру сдвига, и~по параметру масштаба, но так как эти параметры
в~(1)  связаны жесткой зависимостью, так что параметр сдвига
смешиваемого распределения пропорционален его дисперсии, то
фактически смесь~(1) является {\it однопараметрической} и~поэтому
называется {\it дис\-пер\-си\-он\-но-сдви\-го\-вой} (см., например,~\cite{BN1982}).

Другим примером дис\-пер\-си\-он\-но-сдви\-го\-вых смесей нормальных законов
являются обобщенные дисперсионные гам\-ма-рас\-пре\-де\-ле\-ния, в~которых
смешивающими являются обобщенные гам\-ма-рас\-пре\-де\-ле\-ния~\cite{ks2012, zk2013}.

В указанных семействах смесей число неизвестных параметров равно
пяти или шести (если\linebreak учитывать неслучайный сдвиг). Вместе
с~тем у~подоб\-ных моделей имеются довольно серьезные тео\-ре\-ти\-че\-ские
обоснования: в~работах~\cite{zk2013, k2013} показано, что указанные
модели являются асимптотическими аппроксимациями в~простой
предельной схеме случайного суммирования и~потому могут успешно
применяться для анализа процессов типа остановленных случайных
блужданий. Эти выводы подтверждены статистическим анализом
вы\-со\-ко\-час\-тот\-ных финансовых данных, в~результате которого выявлен
синхронизированный характер изменения интенсивностей потоков заявок
в~сис\-те\-мах электронных торгов, что естественно приводит к~синхронизированному
поведению па\-ра\-мет\-ров сдвига и~диффузии в~соответствующих моделях вида смесей
нормальных законов~\cite{kckg2013}.

\section{Описание моди\-фи\-ци\-ро\-ван\-но\-го
сеточного ме\-то\-да разделения дисперсионно-сдвиговых смесей
нормальных законов и~его свойства}

Оказывается, что сеточные методы разделения смесей довольно
эффективны не только при разделении конечных смесей нормальных
законов, но и~при разделении произвольных дис\-пер\-си\-он\-но-сдви\-го\-вых
смесей нормальных законов. Поясним сказанное на примере задачи
оценивания па\-ра\-мет\-ров обобщенных гиперболических распределений.

Для решения задачи оценивания параметров обобщенных гиперболических
распределений традиционно используется метод, предложенный в~статье~\cite{p2004}
и~по сути являющийся классическим ЕМ-ал\-го\-рит\-мом,
приспособленным к~конкретной задаче, и,~соответственно, наследующий
присущие ЕМ-ал\-го\-рит\-мам недостатки.

Рассмотрим следующий альтернативный двухэтапный метод. На первом
этапе на поло\-жи\-тельной полупрямой выделим основную часть носителя
смешивающего распределения, т.\,е.\ \mbox{ограниченный} интервал,
вероятность которого, вычисленная в~соответствии со смешивающим
распределением, практически равна единице. На этот интервал накинем
конечную сетку, содержащую, возможно, очень много {\it известных}
узлов $u_1,\ldots,u_K$. Считая параметр сдвига~$\beta$ равным нулю,
приблизим искомое обобщенное гиперболическое распределение конечной
смесью нормальных законов:

\noindent
\begin{multline}
P_{GH}(x;\,\alpha,0,\nu,\mu,\lambda)\approx{}\\
{}\approx \sum\limits_{i=1}^K
p_i\Phi\left(\fr{x-\alpha u_i}{\sqrt{u_i}}\right)\,,\enskip
x\in\mathbb{R}\,.\label{e2-kor}
\end{multline}
В смеси, стоящей в~правой части соотношения~(2), неизвестными
являются только параметры $p_1,\ldots,p_{K-1}$ и~$\alpha$. Пусть
$x_1,\ldots,x_n$~--- анализируемая выборка значений случайной
величины с~оцениваемым обобщенным гиперболическим распределением.
Итерационный процесс, определяющий сеточный ЕМ-ал\-го\-ритм для данной
задачи, задается следующим образом. Пусть
$p_1^{(m)},\ldots,p_{K-1}^{(m)}$ и~$\alpha^{(m)}$~--- оценки параметров
$p_1,\ldots,p_{K-1}$ и~$\alpha$ на $m$-й итерации,
$p_K^{(m)}\hm=1\hm-p_1^{(m)}-\cdots-p_{K-1}^{(m)}$. Обозначим

\noindent
\begin{align*}
\varphi_{ij}^{(m)}&=\fr{1}{\sqrt{u_i}}\varphi\left(\fr{x_j-\alpha^{(m)}u_i}{\sqrt{u_i}}\right)\,;
\\
g_{ij}^{(m)}&=\fr{p_i^{(m)}\varphi_{ij}^{(m)}}{\sum\limits_{r=1}^K
p_r^{(m)}\varphi_{rj}^{(m)}}\,,\\
&\hspace*{14mm}i=1,\ldots,K\,;\enskip j=1,\ldots,n\,.
\end{align*}
Тогда, используя стандартные рассуждения, определяющие
вычислительные формулы EM-ал\-го\-рит\-ма для параметров конечной смеси
нормальных законов (см, например,~[1, разд.~5.3.7--5.3.8]),
следует положить

\noindent
\begin{equation}
p_i^{(m+1)}=\fr{1}{n}\sum\limits_{j=1}^n g_{ij}^{(m)}\,, \enskip
i=1,\ldots,K\,.\label{e3-kor}
\end{equation}
Обозначим $\overline{x}=(1/n)\sum\limits_{j=1}^nx_j$. Используя
соотношение~(5.3.24) в~\cite{k2011}, с~учетом очевидного равенства
$\sum\limits_{i=1}^K g_{ij}^{(m)}\hm=1$ можно заметить, что уточненная
оценка параметра~$\alpha$ имеет вид:

\columnbreak

\noindent
\begin{equation}
\alpha^{(m+1)}=\fr{\overline{x}}{\sum\limits_{i=1}^K u_ip_i^{(m+1)}}\,,
\label{e4-kor}
\end{equation}
т.\,е.\ равна отношению генерального выборочного среднего и~текущего
эмпирического среднего смешивающего распределения, что вполне
согласуется с~тем, что в~соответствии с~приводимым ниже соотношением~(\ref{e5-kor})
в~данном случае ${\sf E}X\hm=\alpha{\sf E}U$.

В силу монотонности классического ЕМ-ал\-го\-рит\-ма справедливо следующее
утверждение.

\smallskip

\noindent
\textbf{Теорема~1.} {\it Пусть узлы $u_1,\ldots,u_K$ сетки различны,
неотрицательны и~известны. Тогда итерационный процесс $(3)$--$(4)$
является монотонным, т.\,е.\ каждая его итерация не уменьшает
целевую сеточную функцию правдоподобия}
\begin{multline*}
L(p_1,\ldots,p_K,\alpha;x_1,\ldots,x_n)={}\\
{}=
\prod\nolimits_{j=1}^n\left[\sum\nolimits_{i=1}^K
\fr{p_i}{\sqrt{u_i}}\,\varphi\left(\fr{x_j-\alpha^{(m)}u_i}{\sqrt{u_i}}\right)\right].
\end{multline*}

\smallskip

\noindent
\textbf{Замечание~1.} В~разд.~5.7.4 книги~\cite{k2011} показано, что
при каждом фиксированном значении параметра~$\alpha$ сеточная
функция правдоподобия\linebreak
$L(p_1,\ldots,p_{K-1},\alpha;\,x_1,\ldots,x_n)$ вогнута по
аргументам $p_1,\ldots,p_{K-1}$. Поэтому на каждом шаге
итерационного процесса вместо соотношения~(3) можно\linebreak использо\-вать
любой более быстрый алгоритм максимизации функции
$L(p_1,\ldots,p_{K-1},\alpha^{(m)};\,x_1,\ldots$\linebreak $\ldots,x_n)$ по переменным
$p_1,\ldots,p_{K-1}$. Например, оценки весов $p_1,\ldots,p_K$ можно
искать методом условного градиента~\cite{k2011, kn2010}.

\smallskip

Таким образом, на первом этапе получаются оценки параметра~$\alpha$
и~весов всех узлов~$u_i$ конечной сетки, накинутой на носитель
смешивающего обобщенного обратного гауссовского распределения
$P_{\mathrm{GIG}}(z;\,\nu,\mu,\lambda)$.

На втором этапе остается применить ка\-кой-ли\-бо стандартный метод
подгонки обобщенного обратного гауссовского распределения
$P_{\mathrm{GIG}}(z;\,\nu,\mu,\lambda)$ к~эмпирическим данным типа
гистограммы $(u_1, p_1),\ldots, (u_K, p_K)$. Например, параметры~$\nu$,
$\mu$ и~$\lambda$ можно оценить, минимизируя соответствующую
статистику хи-квад\-рат. Или же, например, можно решить задачу
наименьших квад\-ратов:
\begin{multline*}
(\nu^*,\mu^*,\lambda^*)={}\\
{}=\arg\min\limits_{\nu,\mu,\lambda}\sum\limits_{i=1}^K
\left[p_i- \!\!\!\!\!
\int\limits_{(1/2)\left(u_{i-1}+u_i\right)}^{(1/2)(u_i+u_{i+1})}\!\!\!\!\!\!\!\!\!\!\!\!\!\!\!
p_{GIG}(u;\,\nu,\mu,\lambda)\,du\right]^2,
\end{multline*}
где $u_0=0$; $u_{K+1}\hm=\infty$.

На практике хорошие результаты показал подход с решением задачи
наименьших квадратов. Для поиска параметров использовался алгоритм
ns2sol, описанный в~книге~\cite{DSch1983}. Указанный алгоритм
доступен во многих статистических пакетах, отличается высоким
быстродействием и~возможностью при желании задавать разумные
интервалы для поиска параметров.

%\vspace*{-9pt}

\section{О практическом выборе сетки
на~первом этапе моди\-фи\-ци\-ро\-ван\-но\-го
сеточного метода разделения дисперсионно-сдвиговых смесей нормальных
законов}

Естественно, что при использовании указанного двухэтапного метода
в~динамическом режиме крайне важным становится вопрос о~выборе
наиболее эффективных и~быстродействующих численных процедур и~их
параметров. В~частности, исключительную важность приобретает
правильный выбор сетки на первом этапе. Рассмотрим этот вопрос
подробнее.

Формально рассматриваемая задача выглядит так: по наблюдаемым
значениям $x_1,\ldots,x_n$ требуется построить статистическую оценку
верхней границы квантилей заданного порядка сме\-ши\-ва\-юще\-го закона так,
чтобы как можно точнее оценить носитель смешивающего распределения.

В дальнейшем будем считать, что $x_1,\ldots,x_n$~--- независимые
реализации случайной величины $X\hm=Y\sqrt{U}+\alpha U$, где $Y$~---
случайная величина со стандартным нормальным распределением, а~$U$~---
независимая от нее случайная величина с~обобщенным обратным
гауссовским распределением. Тогда, очевидно, распределение случайной
величины~$X$ имеет вид~(1). Предположим, что у~случайной величины~$U$
существуют моменты первых двух порядков. Тогда, как несложно видеть,
\begin{equation}
{\sf E}X={\sf E}Y\cdot{\sf E}\sqrt{U}+\alpha{\sf E}U=\alpha{\sf
E}U\,.\label{e5-kor}
\end{equation}
При этом по усиленному закону больших чисел с~вероятностью единица
$\overline x\hm\longrightarrow {\sf E}X$ $(n\hm\to\infty)$, так что при
больших~$n$ справедливо приближенное равенство ${\sf E}X\hm\approx\overline x$
и~с учетом~(\ref{e5-kor})
\begin{equation}
{\sf E}U\approx\fr{\overline x}{\alpha}\,.\label{e6-kor}
\end{equation}
Далее, очевидно,

\columnbreak

\noindent
\begin{multline}
{\sf E}X^2={\sf E}Y^2\cdot{\sf E}U+2\alpha{\sf E}X\cdot{\sf E}U^{3/2}+{}\\
{}+
\alpha^2{\sf E}U^2={\sf E}U+\alpha^2{\sf E}U^2\,.
\label{e7-kor}
\end{multline}

\noindent
Поэтому, обозначив
$$
m^2=\fr{1}{n}\sum\limits_{i=1}^nx_i^2\,,
$$
получаем приближенное равенство ${\sf E}X^2\hm\approx m^2$, так что
с~учетом~(\ref{e6-kor}) и~(\ref{e7-kor}) имеем:
\begin{equation}
{\sf E}U^2\approx\fr{1}{\alpha^2}\left(m^2-\fr{\overline
x}{\alpha}\right)\,.\label{e8-kor}
\end{equation}
Если параметр~$\alpha$ известен, то для определения верхней границы~$u^*$
сетки, накидываемой на носитель распределения случайной
величины~$U$, можно задать малое положительное число~$\varepsilon$
и~воспользоваться требованием
\begin{equation}
{\sf P}(U\geqslant u^*)\leqslant\varepsilon\,.\label{e9-kor}
\end{equation}
А~для гарантированного выполнения требования~(\ref{e9-kor}) можно использовать
неравенство Маркова:
$$
{\sf P}(U\geqslant u^*)\leqslant\fr{{\sf E}U^2}{(u^*)^2}\leqslant \varepsilon\,,
$$
откуда с учетом~(\ref{e8-kor})
$$
(u^*)^2\geqslant\fr{{\sf E}U^2}{\varepsilon}\approx
\fr{1}{\alpha^2\varepsilon}\left( m^2-\fr{\overline x}{\alpha}\right)
$$
или
\begin{equation}
u^*\approx\fr{1}{\alpha\sqrt{\varepsilon}}\sqrt{m^2-
\fr{\overline x}{\alpha}}\,.\label{e10-kor}
\end{equation}

\begin{figure*}[b] %fig1
\vspace*{1pt}
 \begin{center}
 \mbox{%
 \epsfxsize=161.718mm
 \epsfbox{kor-1.eps}
 }
 \end{center}
 \vspace*{-9pt}
\Caption{Примеры применения модифицированного двухэтапного сеточного
ЕМ-ал\-го\-рит\-ма для подгонки обобщенного гиперболического распределения
к искусственным данным, $\beta\hm=0$: (\textit{a})~$n\hm=1000$, $\alpha\hm=0{,}3$,
$\nu\hm=1{,}3$, $\mu\hm=1{,}6$, $\lambda\hm=0{,}2$;
(\textit{б})~$n\hm=1000$, $\alpha\hm=0{,}5$, $\nu\hm=1$, $\mu\hm=1$,
$\lambda\hm=3$;
(\textit{в})~$n\hm=1000$, $\alpha\hm=3$,
 $\nu\hm=1{,}3$, $\mu\hm=1{,}6$, $\lambda\hm=2$;
(\textit{г})~$n\hm=10\,000$,
$\alpha\hm=0{,}3$, $\nu\hm=1{,}3$, $\mu\hm=1{,}6$, $\lambda\hm=0{,}2$}
\end{figure*}


Если же параметр~$\alpha$, определяющий асим\-мет\-рию распределения
случайной величины~$X$, неизвестен, то можно воспользоваться
следующими рассуждениями. Обозначим
$$
q_n=\fr{1}{n}\sum\limits_{i=1}^n{\bf 1}(x_i<0)\,,
$$
где ${\bf 1}(A)$~--- индикаторная функция множества (события)~$A$.
При этом по усиленному закону больших чисел с~вероятностью единица
$q_n\hm\longrightarrow {\sf P}(X\hm<0)$ $(n\hm\to\infty)$, так что при
больших~$n$ справедливо приближенное равенство
\begin{equation}
q_n\approx{\sf P}(X<0)\,.\label{e11-kor}
\end{equation}
Но
\begin{multline}
{\sf P}(X<0)=\int\limits_{0}^{\infty}\Phi
\left(-\alpha\sqrt{u}\right) p_{\mathrm{GIG}}(u;\nu,\mu,\lambda)\,du={}\\
{}=
{\sf E}\Phi\left(-\alpha\sqrt{U}\right)\,.\label{e12-kor}
\end{multline}

\pagebreak

\noindent
Предположим сначала, что $q_n\hm<1/2$. Если~$n$ достаточно велико,
то можно с~большой степенью
 уверенности утверж\-дать, что тогда
$\overline x\hm>0$ и~$-\alpha\hm<0$, т.\,е.
 $\alpha\hm>0$ и,~стало быть, на
положительной полуоси значений аргумента~$u$ функция $\Phi(\alpha u)$
вогнута, т.\,е.\ выпукла вверх. Тогда из~(\ref{e11-kor}) и~(\ref{e12-kor}), дважды
применяя неравенство Иенсена, в~силу монотонности функции~$\Phi$
получаем:
\begin{multline}
1-q_n\approx 1-{\sf E}\Phi\left(-\alpha\sqrt{U}\right)=
          {\sf E}\Phi\left(\alpha\sqrt{U}\right)\leqslant{}\\
          {}\leqslant\Phi
          \left(\alpha{\sf E}\sqrt{U}\right)\leqslant
          \Phi\left(\alpha\sqrt{{\sf E}U}\right)\,.\label{e13-kor}
\end{multline}
Если теперь для $t\hm\in(0,1)$ символом~$v_t$ обозначить $t$-кван\-тиль
стандартного нормального закона, то из~(\ref{e13-kor}) и~(\ref{e6-kor}) вытекает
<<приближенное неравенство>>
$$
v_{1-q_n}\hm\leqslant \alpha\sqrt{{\sf E}U}\,,
$$
т.\,е.
$$
\alpha\geqslant\fr{v_{1-q_n}}{\sqrt{{\sf E}U}}\approx
\fr{v_{1-q_n}\sqrt{\alpha}}{\sqrt{\overline x}}\,,
$$
откуда получаем, что при достаточно больших~$n$
\begin{equation}
\alpha\geqslant\fr{v_{1-q_n}^2}{\overline x}\,.\label{e14-kor}
\end{equation}
Если теперь задать малое положительное число~$\varepsilon$, то
для определения верхней границы~$u^*$ сетки, накидываемой на
носитель распределения случайной величины~$U$, можно воспользоваться
требованием~(\ref{e9-kor}), для гарантированного выполнения которого
с~учетом~(\ref{e6-kor}) и~(\ref{e14-kor}) можно использовать неравенство Маркова:
$$
{\sf P}(U\geqslant u^*)\leqslant \fr{{\sf E}U}{u^*}\approx\fr{\overline
x}{\alpha u^*}\leqslant \fr{(\overline x)^2}{v_{1-q_n}^2 u^*}\leqslant
\varepsilon\,,
$$
откуда окончательно вытекает оценка
\begin{equation}
u^*\approx\fr{(\overline x)^2}{v_{1-q_n}^2 \varepsilon}\,.\label{e15-kor}
\end{equation}

\begin{figure*}[b] %fig2
\vspace*{18pt}
 \begin{center}
 \mbox{%
 \epsfxsize=162.433mm
 \epsfbox{kor-3.eps}
 }
 \end{center}
 \vspace*{-9pt}
\Caption{Примеры применения модифицированного двухэтапного
сеточного ЕМ-ал\-го\-рит\-ма для подгонки обобщенного гиперболического
распределения к~искусственным данным, $n=10\,000$, $\beta\hm=0$:
(\textit{а})~$\alpha\hm=0{,}3$,
$\nu\hm=2$, $\mu\hm=2$, $\lambda\hm=2{,}5$;
(\textit{б})~$\alpha\hm=0{,}5$,  $\nu\hm=1$, $\mu\hm=1$, $\lambda\hm=3$;
(\textit{в})~$\alpha\hm=0{,}8$,
$\nu\hm=1{,}3$, $\mu\hm=1{,}6$, $\lambda\hm=2$;
(\textit{г})~$\alpha\hm=1{,}3$, $\nu\hm=2$, $\mu\hm=2$, $\lambda\hm=2{,}5$}
\end{figure*}



В случае $q_n\hm\geqslant1/2$, если $n$ достаточно велико, то можно
с~большой степенью уверенности утверж\-дать, что $\overline x\hm\leqslant 0$
и~$-\alpha\hm\geqslant 0$, т.\,е.\ на положительной\linebreak\vspace*{-12pt}

\pagebreak

%\end{multicols}


%\begin{multicols}{2}

\noindent
 полуоси значений аргумента~$u$
функция $\Phi(-\alpha u)$ вогнута, т.\,е.\ выпукла вверх. Тогда
из~(\ref{e11-kor}) и~(\ref{e12-kor}), дважды применяя неравенство Иенсена, в~силу
монотонности функции~$\Phi$ получаем
$$
q_n\approx {\sf E}\Phi\left(-\alpha\sqrt{U}\right)\leqslant
\Phi\left(-\alpha\sqrt{{\sf E}U}\right)\,,
$$
откуда вытекает <<приближенное неравенство>> $v_{q_n}\hm \leqslant
-\alpha\sqrt{{\sf E}U}$,
т.\,е.
$$
-\alpha\geqslant\fr{v_{q_n}}{\sqrt{{\sf E}U}}\approx
\fr{v_{q_n}\sqrt{|\alpha|}}{\sqrt{|\overline x|}}
$$
и при достаточно больших~$n$
\begin{equation}
|\alpha|\geqslant\fr{v_{q_n}^2}{|\overline x|}\,.\label{e16-kor}
\end{equation}
Для определения верхней границы~$u^*$ сетки, накидываемой на
носитель распределения случайной величины~$U$, снова зададим малое
положительное число~$\varepsilon$ и~потребуем, чтобы было
справедливо условие~(\ref{e9-kor}), для гарантированного выполнения которого
с~учетом~(\ref{e6-kor}) и~(\ref{e16-kor}) используем неравенство Маркова и~тот факт, что
$\mathrm{sign}\, \overline x\hm=\mathrm{sign}\,\alpha$ при достаточно
больших~$n$:
\begin{multline}
{\sf P}(U\geqslant u^*)\leqslant \fr{{\sf E}U}{u^*}\approx
\fr{\overline x}{\alpha u^*}=
\fr{|\overline x|}{|\alpha| u^*} \leqslant{}\\
{}\leqslant
\fr{(\overline x)^2}{v_{q_n}^2 u^*}\leqslant
\varepsilon\,.\label{e17-kor}
\end{multline}
В силу симметричности нормального распределения $v_{t}\hm=-v_{1-t}$ для
любого $t\hm\in(0,1)$, поэтому $v_{q_n}^2\hm=v_{1-q_n}^2$ и~в~случае
$q_n\hm\geqslant1/2$ соотношение~(\ref{e17-kor}) снова приводит к~оценке~(\ref{e15-kor}).

Справедливости ради необходимо отметить, что оценки~(\ref{e10-kor}) и~(\ref{e15-kor})
являются завышенными, но они гарантируют, что
$(1-\varepsilon)$-почти-весь носитель распределения случайной
величины~$U$ будет лежать внутри интервала $[0, u^*]$.

\section{Результаты численных экспериментов}

Приводимые в~данном разделе графики иллюстрируют качество работы
модифицированного сеточного метода разделения дис\-пер\-си\-он\-но-сдви\-го\-вых
смесей нормальных законов на примере его\linebreak применения к~оцениванию
параметров обоб\-щенных гиперболических распределений с~ис\-поль\-зованием
указанного алгоритма выбора сетки\linebreak с~умеренным чис\-лом узлов $K\hm=40$.
Для вы\-чис\-ле\-ний использовались искусственно сгенерированные выборки
объемов $n\hm=1000$ и~$n\hm=10\,000$ с~разными наборами параметров, значения
которых указаны на рисунках. На рис.~1 и~2 изображены гистограммы
(серые столбики) и~графики
истинной плот\-ности (штриховые линии), промежуточной
оценки, полученной сеточным ЕМ-ал\-го\-рит\-мом (пунктирные линии)
и~итоговой оценки (непрерывные линии). На рис.~1 и~2 так\-же указаны
значения полученных оценок параметров. Как видно из приводимых
рисунков, параметры~$\alpha$ оцениваются очень точно. Точность
оценок остальных параметров удовлетворительная и~может быть повышена
за счет использования более частых сеток и~более чувствительных
критериев остановки ЕМ-ал\-го\-рит\-ма на первом этапе. Следует отметить,
что даже в~тех случаях, в~которых наблюдаются заметные расхождения
оценок параметров и~их точных значений, оценки самих плотностей
довольно \mbox{точны}.




{\small\frenchspacing
 {%\baselineskip=10.8pt
 \addcontentsline{toc}{section}{References}
 \begin{thebibliography}{99}
\bibitem{k2011}
\Au{Королев В.\,Ю.} Ве\-ро\-ят\-но\-ст\-но-ста\-ти\-сти\-че\-ские методы
декомпозиции волатильности хаотических процессов.~--- М.: Изд-во
Московского ун-та, 2011.

\bibitem{n2013}
\Au{Назаров А.\,Л.} Приближенные методы разделения смесей
вероятностных распределений: Дисс.\ \ldots\  канд. физ.-мат. наук.~--- М.:
МГУ им.\ М.\,В.~Ломоносова, 2013.

\bibitem{BN1977}
\Au{Barndorff-Nielsen~O.-E.} Exponentially decreasing distributions
for the logarithm of particle size~// Proc. Roy. Soc. Lond.~A,
1977. Vol.~353. P.~401--419.

\bibitem{BN1978}
\Au{Barndorff-Nielsen~O.-E.} Hyperbolic distributions and
distributions of hyperbolae~// Scand. J. Statist., 1978. Vol.~5.
P.~151--157.

\bibitem{BN1982}
\Au{Barndorff-Nielsen~O.-E., Kent~J., S\!{\!\ptb{\!\o}}\,rensen~M.} Normal
variance-mean mixtures and $z$-distributions~// Int. Statist. Rev.,
1982. Vol.~50. No.\,2. P.~145--159.

\bibitem{ks2012}
\Aue{Королев В.\,Ю., Соколов И.\,А.} Скошенные распределения
Стьюдента, дисперсионные гам\-ма-рас\-пре\-де\-ле\-ния и~их обобщения как
асимптотические аппроксимации~// Информатика и~её применения, 2012.
Т.~6. Вып.~1. С.~2--10.

\bibitem{zk2013}
\Au{Закс Л.\,М., Королев В.\,Ю.} Обобщенные дисперсионные
гам\-ма-рас\-пре\-де\-ле\-ния как предельные для случайных сумм~// Информатика
и её применения, 2013. Т.~7. Вып.~1. С.~105--115.

\bibitem{k2013}
\Au{Королев В.\,Ю.} Обобщенные гиперболические
распределения как предельные для случайных сумм~// Тео\-рия
вероятностей и~ее применения, 2013. Т.~58. Вып.~1. С.~117--132.

\bibitem{kckg2013}
\Au{Королев В.\,Ю., Черток А.\,В., Корчагин~А.\,Ю.,
Горшенин~А.\,К.} Ве\-ро\-ят\-но\-ст\-но-ста\-ти\-сти\-че\-ское моделирование
информационных потоков в~сложных финансовых системах на основе
высокочастотных данных~// Информатика и~её применения, 2013. Т.~7.
Вып.~1. С.~12--21.

\bibitem{p2004}
\Au{Protassov R.\,S.} EM-based maximum likelihood parameter
estimation for a~multivariate generalized hyperbolic distribution
with fixed~$\lambda$~// Statistics Computing, 2004. Vol.~14.
P.~67--77.

\bibitem{kn2010}
\Au{Королев В.\,Ю., Назаров А.\,Л.} Разделение смесей
вероятностных распределений при помощи сеточных методов моментов и~максимального правдоподобия~//
Автоматика и~телемеханика, 2010. Вып.~3. С.~98--116.

\bibitem{DSch1983}
\Au{Dennis J.\,E., Schnabel R.\,B.} Numerical methods for
unconstrained optimization and nonlinear equations.~--- Englewood
Cliffs: Prentice-Hall, 1983. 378~p.
 \end{thebibliography}

 }
 }

\end{multicols}

\vspace*{-6pt}

\hfill{\small\textit{Поступила в редакцию 01.10.14}}

\newpage

%\vspace*{12pt}

%\hrule

%\vspace*{2pt}

%\hrule

%\vspace*{12pt}

\def\tit{A MODIFIED GRID METHOD FOR~STATISTICAL SEPARATION
OF~NORMAL VARIANCE-MEAN MIXTURES}

\def\titkol{A modified grid method for statistical separation
of~normal variance-mean mixtures}

\def\aut{V.\,Yu.~Korolev$^{1,2}$ and~A.\,Yu.~Korchagin$^1$}

\def\autkol{V.\,Yu.~Korolev and~A.\,Yu.~Korchagin}

\titel{\tit}{\aut}{\autkol}{\titkol}

\vspace*{-9pt}


\noindent
$^1$Faculty of Computational Mathematics and Cybernetics,
M.\,V.~Lomonosov Moscow State University,\linebreak
$\hphantom{^1}$1-52 Leninskiye Gory, GSP-1, Moscow 119991, Russian Federation


\noindent
$^2$Institute of Informatics Problems, Russian Academy of Sciences,
44-2~Vavilov Str., Moscow 119333, Russian\linebreak
$\hphantom{^1}$Federation

\def\leftfootline{\small{\textbf{\thepage}
\hfill INFORMATIKA I EE PRIMENENIYA~--- INFORMATICS AND
APPLICATIONS\ \ \ 2014\ \ \ volume~8\ \ \ issue\ 4}
}%
 \def\rightfootline{\small{INFORMATIKA I EE PRIMENENIYA~---
INFORMATICS AND APPLICATIONS\ \ \ 2014\ \ \ volume~8\ \ \ issue\ 4
\hfill \textbf{\thepage}}}

\vspace*{3pt}

\Abste{A~modified two-stage grid method for
statistical separation of normal variance-mean mixtures is described
as an alternative to a pure EM (expectation-maximization) algorithm.
At the first stage of this
algorithm, a~discrete approximation is constructed to the mixing
distribution. At the second stage, the obtained discrete
distribution is approximated by an absolutely continuous
distribution from a~predetermined family, say, by a generalized
inverse Gaussian distribution. The convergence of this two-stage
procedure is discussed. The monotonicity of the grid procedure used
at the first stage is proved. The problem of the optimal choice of
the parameters of the method is discussed in detail. First of all,
the problem of the optimal choice of the grid thrown on the support
of the mixing distribution is considered. Statistical estimators are
proposed for the quantiles of the mixing law. The efficiency of the
method is illustrated by examples of its application to the
estimation of the parameters of generalized hyperbolic
distributions.}

\smallskip

\KWE{mixture of probability distributions; normal
variance-mean mixture; generalized hyperbolic distribution;
EM-algorithm; grid method of separation of mixtures}

\DOI{10.14357/19922264140402}

\Ack
\noindent
The research was supported by the Russian Science Foundation (project 14-11-00364).

%\vspace*{3pt}

  \begin{multicols}{2}

\renewcommand{\bibname}{\protect\rmfamily References}
%\renewcommand{\bibname}{\large\protect\rm References}



{\small\frenchspacing
 {%\baselineskip=10.8pt
 \addcontentsline{toc}{section}{References}
 \begin{thebibliography}{99}
 \bibitem{k2011eng}
 \Aue{Korolev, V.\,Yu.} 2011.
\textit{Veroyatnostno-statisticheskie metody dekompozitsii
volatil'nosti khaoticheskikh protsessov}
[Probabilistic and statistical methods for the decomposition of volatility
of chaotic processes].
Moscow: Moscow University Press. 510~p.

\bibitem{n2013eng}
\Aue{Nazarov, A.\,L.} 2013.
{Priblizhennye metody razdeleniya smesey veroyatnostnykh raspredeleniy}
[Approximate methods for the decomposition of volatility of chaotic processes].
Ph.D. Thesis. Moscow: Moscow State University.

\bibitem{BN1977eng}
\Aue{Barndorff-Nielsen, O.\,E.} 1977.
Exponentially decreasing distributions for the logarithm of particle size.
\textit{Proc. Roy. Soc. Lond. A} 353:401--419.

\bibitem{BN1978eng}
\Aue{Barndorff-Nielsen, O.\,E.} 1978.
Hyperbolic distributions and distributions of hyperbolae.
\textit{Scand. J. Statist.} 5:151--157.

\bibitem{BN1982eng}
\Aue{Barndorff-Nielsen, O.\,E., J.~Kent, and M.~S\!{\ptb{\o}}rensen}. 1982.
Normal variance-mean mixtures and $z$-distributions.
\textit{Int. Statist. Rev.} 50(2):145--159.

\bibitem{ks2012eng}
\Aue{Korolev, V.\,Yu., and I.\,A. Sokolov}. 2012.
{Skoshennye raspredeleniya St'yudenta, dispersionnye
gam\-ma-ras\-pre\-de\-le\-niya i~ikh obobshcheniya kak asimptoticheskie
approksimatsii}
[Skewed Student's distributions, variance gamma distributions, and their
generalizations as asymptotic approximations].
\textit{Informatika i ee Primeneniya}~--- \textit{Inform. Appl.} 6(1):2--10.

\bibitem{zk2013eng}
\Aue{Korolev, V.\,Yu., and L.\,M.~Zaks}. 2013.
{Obobshchennye dispersionnye gam\-ma-ras\-pre\-de\-le\-niya kak
predel'nye dlya sluchaynykh summ}
[Generalized variance gamma distributions as limiting for random sums].
\textit{Informatika i ee Primeneniya}~--- \textit{Inform. Appl.} 7(1):105--115.

\bibitem{k2013eng} \Aue{Korolev, V.\,Yu.} 2013.
{Obobshchennye giperbolicheskie raspredeleniya kak predel'nye dlya sluchaynykh summ}
[Generalized hyperbolic distributions as limiting for random sums]
\textit{Theory Probab. Appl.} 58(1):117--132.

\bibitem{kckg2013eng}
\Aue{Korolev, V.\,Yu., A.\,V. Chertok, A.\,Yu.~Korchagin, and A.\,K.~Gorshenin}.
2013. {Ve\-ro\-yat\-no\-st\-no-sta\-ti\-sti\-che\-skoe
mo\-de\-li\-ro\-va\-nie informatsionnykh potokov v~slozhnykh finansovykh sistemakh
na osnove vysokochastotnykh dannykh}
[Probability and statistical modeling of information flows in complex
financial systems from high-frequency data].
\textit{Informatika i~ee Primeneniya}~--- \textit{Inform.  Appl.} 7(1):12--21.

\bibitem{p2004eng-1}
\Aue{Protassov, R.\,S.} 2004.
EM-based maximum likelihood parameter estimation for a multivariate
generalized hyperbolic distribution with fixed~$\lambda$.
\textit{Statistics Computing} 14:67--77.

\bibitem{kn2010eng-1}
\Aue{Korolev, V.\,Yu., and A.\,L.~Nazarov}. 2010.
{Razdelenie smesey veroyatnostnykh raspredeleniy pri pomoshchi
setochnykh metodov momentov i~maksimal'nogo pravdopodobiya}
[Separation of mixtures using grid moment-based methods and maximum likelihood].
\textit{Avtomatika i~Telemekhanika} [Automatics and Telemechanics] 3:98--116.

\bibitem{DSch1983eng}
\Aue{Dennis, J.\,E., and R.\,B.~Schnabel}. 1983.
\textit{Numerical methods for unconstrained optimization and nonlinear equations}.
Englewood Cliffs: Prentice-Hall. 378~p.


\end{thebibliography}

 }
 }

\end{multicols}

\vspace*{-6pt}

\hfill{\small\textit{Received October 01, 2014}}

\vspace*{-18pt}

\Contr

\noindent
\textbf{Korolev Victor Yu.} (b.\ 1954)~---
Doctor of Science in physics and mathematics, professor,
Department of Mathematical Statistics, Faculty of Computational Mathematics
and Cybernetics, M.\,V.~Lomonosov Moscow State University,
1-52 Leninskiye Gory, GSP-1, Moscow 119991, Russian Federation;
leading scientist, Institute of Informatics Problems,
Russian Academy of Sciences, 44-2~Vavilov Str., Moscow 119333, Russian
Federation; victoryukorolev@yandex.ru

\vspace*{3pt}

\noindent
\textbf{Korchagin Alexander Yu.} (b.\ 1989)~---
PhD student, Faculty of Computational Mathematics and Cybernetics,
M.\,V.~Lomonosov Moscow State University,
1-52 Leninskiye Gory, GSP-1, Moscow 119991, Russian Federation;
sasha.korchagin@gmail.com


\label{end\stat}

\renewcommand{\bibname}{\protect\rm Литература}  %9

\def\stat{ush1+ush}

\def\tit{ОБ ОДНОЙ ЯДЕРНОЙ ОЦЕНКЕ ПЛОТНОСТИ$^*$}

\def\titkol{Об одной ядерной оценке плотности}

\def\autkol{В.\,Г.~Ушаков, Н.\,Г.~Ушаков}
\def\aut{В.\,Г.~Ушаков$^1$, Н.\,Г.~Ушаков$^2$}

\titel{\tit}{\aut}{\autkol}{\titkol}

{\renewcommand{\thefootnote}{\fnsymbol{footnote}}\footnotetext[1]
{Работа поддержана Российским фондом фундаментальных исследований
(проекты 11-01-00515а и 11-07-00112а), а также Министерством
образования и науки РФ в рамках ФЦП <<Научные и
научно-педагогические кадры инновационной России на 2009--2013~годы>>.}}


\renewcommand{\thefootnote}{\arabic{footnote}}
\footnotetext[1]{Московский государственный 
университет им.\ М.\,В.~Ломоносова, факультет вычислительной математики и кибернетики; Институт проблем информатики Российской
академии наук, vgushakov@mail.ru}
\footnotetext[2]{Институт проблем технологии
микроэлектроники и особочистых материалов Российской академии наук, ushakov@math.ntnu.no}

\Abst{Исследуется ядерная оценка плотности
распределения, основанная на так называемом синк-ядре.
Основное внимание уделено анализу среднеквадратической ошибки оценки
при конечных объемах выборки.
Рассмотрены проблемы оценивания моды и производных плотности.}

\KW{непараметрическое оценивание
плотности; ядерная оценка; ядро бесконечного порядка}

  \vskip 14pt plus 9pt minus 6pt

      \thispagestyle{headings}

      \begin{multicols}{2}
      
            \label{st\stat}

\section{Введение}

Непараметрическое оценивание плотности вероятностных распределений с помощью 
ядерных оценок является одной из важнейших задач интеллектуального
анализа данных, в частности при анализе трафика в телекоммуникационных системах.

Рассмотрим выборку $X_1,\ldots,X_n$, состоящую из $n$ независимых наблюдений,
имеющих одинаковые распределения. Всюду в настоящей работе будем
предполагать, что $X_k$ имеет абсолютно непрерывное распределение.
Обозначим функцию распределения, плотность распределения и характеристическую
функцию соответственно~$F(x)$, $f(x)$ и $\varphi(t)$. Ядерной оценкой
плотности распределения~$f(x)$, построенной по выборке $X_1,\ldots,X_n$,
называется случайная функция
\begin{equation*}
\hat f_n(x;h)=\fr{1}{nh}\sum\limits_{k=1}^nK\left(\fr{x-X_k}{h}\right)\,,
%\label{e1-u1}
\end{equation*}
где $K(x)$~--- некоторая функция, называемая \mbox{ядром}; $h$~--- параметр
сглаживания (неотрицательное\linebreak чис\-ло).

Обычно в качестве ядра выбирается плотность распределения вероятностей, т.\,е.\
неотрицательная интегрируемая функция такая, что
$\int\limits_{-\infty}^\infty K(x)\,dx=1$. В~данной работе, однако, будем иметь дело
с нестандартным ядром, а именно:
\begin{equation}
K(x)=\fr{\sin x}{\pi x}\,,
\label{e2-u1}
\end{equation}
чье преобразование Фурье (характеристическая функция) равно

\noindent
$$
\psi(t)=\begin{cases}
1\,,& \mbox{\ если\ } |t|\le 1\,;\\
0\,,& \mbox{\ если\ } |t|>1\,.
\end{cases}
$$
Это так называемое синк-ядро. Всюду в данной работе будем обозначать ядерную оценку с
ядром~(\ref{e2-u1}) как~$f_n(x;h)$. Оценка $f_n(x;h)$ изучалась в работах~[1, 2]. 
В~[1] она исследовалась для класса плотностей, чьи
характеристические функции убывают регулярно при $|t|\to\infty$.
Было показано, что при экспоненциальном убывании скорость сходимости
к нулю среднеквадратической ошибки отличается от $1/n$ лишь медленно
меняющейся функцией, т.\,е.\ ядро~(\ref{e2-u1}) имеет бесконечный порядок. Если
$|\varphi(t)|$ убывает <<алгебраически>>: $|\varphi(t)|\sim
1/{t^p}$, $t\to\infty$, то, как было показано в~[1], при $p>5$
оценка $f_n(x;h)$ имеет лучший порядок состоятельности, чем ядерные
оценки с ядрами, являющимися плотностями распределения. В~[2]
доказано, что оценка $f_n(x;h)$ имеет оптимальный порядок состоятельности.

Несмотря на очень хорошие асимптотические свойства, оценка с ядром~(\ref{e2-u1})
не нашла пока еще широкого применения. Одной из причин является то,
что, поскольку ядро принимает отрицательные значения и
неинтегрируемо, реализации соответствующей ядерной оценки не
являются плотностями распределения. Этот дефект, однако, легко может быть
исправлен без потери точности~[3]. Другая причина: не вполне понятно,
при каких объемах выборки проявляются асимптотические преимущества
оценки $f_n(x;h)$. Многие считают, что эти объемы должны быть очень велики.
В настоящей работе проводится более углубленное исследование оценки с ядром~(\ref{e2-u1}).
Получены неравенства для интегральной среднеквадратической ошибки,
позволяющие оценить величину ошибки при конечных объемах выборки
(в том числе в случае оценивания производных плотности).
Кроме того, исследуется равномерная сходимость оценки и
проблема оценивания моды плотности.

Пусть $\hat f(x)$~--- какая-либо оценка плотности~$f(x)$.
В качестве меры точности оценки будем использовать интегральную
среднеквадратическую ошибку, которую будем обозначать $J(\hat f)$ и
которая определяется следующим образом:
$$
J(\hat f)={\rm E}\int\limits_{-\infty}^\infty\left[\hat f(x)-f(x)\right]^2\,dx\,.
$$
Для ядерной оценки $f_n(x;h)$ с ядром~(\ref{e2-u1}) $J(f_n)$ может быть записана
как (см.~[2])

\noindent
\begin{multline}
J(f_n)=\fr{1}{2\pi}\int\limits_{|t|>1/h}|\varphi(t)|^2\,dt+{}\\
{}+
\fr{1}{n}\,\fr{1}{2\pi}
\int\limits_{-1/h}^{1/h}(1-|\varphi(t)|^2)\,dt\,.
\label{e3-u1}
\end{multline}
Отметим, что первое слагаемое в правой части является интегрированным
квадратом смещения оценки, а второе~--- интегрированной дисперсией.
Всюду в данной работе будем предполагать, что квадрат оцениваемой
плотности интегрируем. В~противном случае $J(f_n)=\infty$ и исследование
оценки на основе интегральной среднеквадратической ошибки становится бессмысленным.

Ядра, являющиеся плотностями распределения вероятностей, будем для
краткости называть стандартными ядрами.

\vspace*{-3pt}

\section{Неравенства}

\vspace*{-1pt}

Пусть $\tilde f(x)$~--- какая-либо оценка плотности~$f(x)$. 
Обозначим интегрированный квадрат смещения и интегрированную
дисперсию соответственно ${\rm B}(\tilde f)$ и ${\rm V}(\tilde f)$, т.\,е.\

\noindent
\begin{align*}
{\rm B}(\tilde f)&=\int\limits_{-\infty}^\infty\left[{\rm E}\tilde f(x)-f(x)\right]^2\,dx\,;\\
{\rm V}(\tilde f)&=
\int\limits_{-\infty}^\infty\left[\tilde f(x)-{\rm E}\tilde f(x)\right]^2\,dx\,.
\end{align*}
В этом параграфе будут получены верхние оценки интегральной среднеквадратической
ошибки, которые позволяют оценить реальный уровень погрешности
при конечных объемах выборки.
Определим нулевую производную функции как саму функцию. Ниже будем
использовать следующий вариант равенства Парсеваля.
Предположим, что плотность распределения~$f(x)$ $m$~раз дифференци-\linebreak\vspace*{-12pt}
\columnbreak

\noindent
руема,
$m\ge0$, а квадрат $m$-й производной интегрируем.
Пусть $\varphi(t)$~--- соответствующая характеристическая функция.
Тогда
\begin{equation}
\int\limits_{-\infty}^\infty(f^{(m)}(x))^2\,dx=
\fr{1}{2\pi}\int\limits_{-\infty}^\infty t^{2m}|\varphi(t)|^2\,dt\,.
\label{e4-u1}
\end{equation}

Пусть $g(x)$~--- некоторая функция, квадрат которой интегрируем. Обозначим
$R(g)=\int\limits_{-\infty}^\infty g^2(x)\,dx$.

\medskip

\noindent
\textbf{Теорема 1.} \textit{Пусть $m$-я производная плотности~$f(x)$
существует, а ее квадрат интегрируем. Тогда}
\begin{equation}
J(f_n)<\varepsilon(h)h^{2m}R(f^{(m)})+\fr{1}{\pi nh}\,,
\label{e5-u1}
\end{equation}
\textit{где $\varepsilon(h)\le1$ для всех~$h$ и $\varepsilon(h)\to0$ при $h\to0$.}

\medskip

\noindent
Д\,о\,к\,а\,з\,а\,т\,е\,л\,ь\,с\,т\,в\,о\,.\ 
Оценим сначала первое слагаемое в правой части~(\ref{e3-u1}).
Применяя~(\ref{e4-u1}), получаем:
\begin{multline*}
\fr{1}{2\pi}\int\limits_{|t|>1/h}|\varphi(t)|^2\,dt={}\\
{}=
h^{2m}\fr{1}{2\pi}\int\limits_{|t|>1/h}(1/h)^{2m}|\varphi(t)|^2\,dt\le{}\\
{}\le h^{2m}\fr{1}{2\pi}\int\limits_{|t|>1/h}t^{2m}|\varphi(t)|^2\,dt={}\\
{}=
h^{2m}\fr{1}{2\pi}\int\limits_{-\infty}^\infty t^{2m}|\varphi(t)|^2\,dt-{}\\
{}-
h^{2m}\fr{1}{2\pi}\int\limits_{-1/h}^{1/h} t^{2m}|\varphi(t)|^2\,dt
={}\\
{}=
h^{2m}\fr{1}{2\pi}\int\limits_{-\infty}^\infty t^{2m}|\varphi(t)|^2\,dt\times{}\\
{}\times
\left(1-\fr{\int\limits_{-1/h}^{1/h}
t^{2m}|\varphi(t)|^2\,dt}{\int\limits_{-\infty}^\infty t^{2m}|\varphi(t)|^2dt}
\right)={}\\
{}=\varepsilon(h)h^{2m}\int\limits_{-\infty}^\infty(f^{(m)}(x))^2\,dx=
\varepsilon(h)h^{2m}R(f^{(m)})\,,
\end{multline*}
где
$$
\varepsilon(h)= 1-\fr{\int\limits_{-1/h}^{1/h}
t^{2m}|\varphi(t)|^2\,dt}{\int\limits_{-\infty}^\infty t^{2m}|\varphi(t)|^2\,dt}
$$
удовлетворяет условиям теоремы:
$\varepsilon(h)\le1$ и $\varepsilon(h)\to0$ при $h\to0$.

Для второго слагаемого в правой части~(\ref{e3-u1}) имеем
$$
\fr{1}{n}\,\fr{1}{2\pi}\int\limits_{-1/h}^{1/h}(1-|\varphi(t)|^2)\,dt<
\fr{1}{n}\,\fr{1}{2\pi}\int\limits_{-1/h}^{1/h}\,dt=\fr{1}{\pi nh}\,.
$$
Таким образом, получаем~(\ref{e5-u1}).

\medskip

\noindent
\textbf{Следствие 1.} \textit{Пусть выполнены условия теоремы~$1$. Тогда}
\begin{equation}
J(f_n)<h^{2m}R(f^{(m)})+\fr{1}{\pi nh}\,.
\label{e6-u1}
\end{equation}

Полагая в~(\ref{e6-u1})
$$
h=\left[\fr{1}{2\pi nmR(f^{(m)})}\right]^{1/(2m+1)}
$$
(это значение $h$ минимизирует правую часть~(\ref{e6-u1})), получаем

\smallskip

\noindent
\textbf{Следствие 2.} \textit{Пусть выполнены условия теоремы~$1$. Тогда}
\begin{multline*}
\inf\limits_{h>0}J(f_n)<{}\\
{}< \fr{1+2m}{(2\pi m)^{2m/(2m+1)}}
R(f^{(m)})^{1/(2m+1)}n^{-2m/(2m+1)}\,.
\end{multline*}

\smallskip

\noindent
\textbf{Следствие 3.} \textit{Пусть выполнены условия теоремы~$1$. Тогда}
$$
\inf\limits_{h>0}J(f_n)=o\left(n^{-2m/(2m+1)}\right)\,,\quad n\to\infty\,.
$$

\smallskip

\noindent
\textbf{Следствие 4.} \textit{Если~$f(x)$ дважды дифференцируема и квадрат
второй производной интегрируем, то}
$$
\inf\limits_{h>0}J(f_n)=o(n^{-4/5})\,,\quad  n\to\infty\,,
$$
и
$$
\inf\limits_{h>0}J(f_n)<\fr{5}{4\pi}\left(4\pi R(f'')\right)^{1/5}n^{-4/5}\,.
$$

\smallskip

Чтобы получить более точные оценки, необходима дополнительная информация
об оцениваемой плотности. Пусть $g(x)$~--- некоторая функция. Обозначим~$Vr(g)$ 
ее полную вариацию.

\medskip

\noindent
\textbf{Теорема 2.} \textit{Если $f(x)$ $m$ раз дифференцируема ($m\ge0$),
и ее $m$-я производная имеет ограниченную полную вариацию, то}
\begin{equation}
J(f_n)\le h^{2m+1}\fr{Vr(f^{(m)})^2}{(2m+1)\pi}+\fr{1}{\pi nh}\,.
\label{e7-u1}
\end{equation}

\medskip

\noindent
Д\,о\,к\,а\,з\,а\,т\,е\,л\,ь\,с\,т\,в\,о\,.\ Для всех~$t$~[4]
$$
\left\vert\varphi(t)\right\vert\le \fr{Vr(f^{(m)})}{|t|^{m+1}}\,.
$$
Применяя это неравенство, оценим первое слагаемое в правой части~(\ref{e3-u1}):
\begin{multline*}
\fr{1}{2\pi}\int\limits_{|t|>1/h}|\varphi(t)|^2\,dt\le
\fr{Vr(f^{(m)})^2}{\pi}\int\limits_{1/h}^\infty \fr{dt}{t^{2m+2}}
={}\\
{}= \fr{Vr(f^{(m)})^2}{(2m+1)\pi}\,h^{2m+1}\,.
\end{multline*}
Для второго слагаемого в правой части~(\ref{e3-u1})
имеем (см.\ доказательство теоремы~1)
$$
\fr{1}{n}\,\fr{1}{2\pi}\int\limits_{-1/h}^{1/h}(1-|\varphi(t)|^2)\,dt<
\fr{1}{\pi nh}\,.
$$
Таким образом, получаем~(\ref{e7-u1}).

\smallskip

\noindent
\textbf{Следствие.} \textit{Если выполнены условия теоремы~$2$, то}
\begin{multline*}
\inf\limits_{h>0}J(f_n)\le{}\\
{}\le \fr{2(m+1)}{(2m+1)\pi}\,Vr(f^{(m)})^{1/(m+1)}n^{-(2m+1)/(2m+2)}\,.
\end{multline*}

\smallskip
Например, если $m=2$, получаем
$$
\inf\limits_{h>0}J(f_n)\le\fr{6}{5\pi}\,Vr(f'')^{1/3}n^{-5/6}\,,
$$
что по порядку лучше, чем в случае стандартных оценок ($n^{-4/5}$).

Следуя Ватсону и Лидбеттеру~[5] и Дэвис~[1] будем говорить, что
характеристическая функция~$\varphi(t)$ убывает экспоненциально с показателем~$\alpha$ 
и коэффициентом~$\rho$ ($\rho>0$, $0<\alpha\le2$), если
\begin{equation}
\left\vert\varphi(t)\right\vert \le Ae^{-\rho|t|^\alpha}\,,
\label{e8-u1}
\end{equation}
где $A$~--- постоянная. В~[1] доказано (теорема~4.1), что если для
характеристической функции оцениваемой плотности выполняется~(\ref{e8-u1}),
то
$$
\lim\limits_{n\to\infty}he^{\rho/h^\alpha}\left\vert{\rm B}(f_n)\right\vert=0\,.
$$
Теорема~3 уточняет это утверждение.

\smallskip

\noindent
\textbf{Теорема 3.} \textit{Если}
$$
\fr{1}{2\pi}\int\limits_{-\infty}^\infty e^{\rho|t|^\alpha}\left\vert\varphi(t)\right\vert^2\,dt
=C<\infty\,,
$$
\textit{то}
\begin{equation}
J(f_n)\le\varepsilon(h)Ce^{-\rho/h^\alpha}+\fr{1}{\pi nh}\,,
\label{e9-u1}
\end{equation}
\textit{где $0<\varepsilon(h)<1$ и $\varepsilon(h)\to0$ при $h\to0$}.
%\pagebreak


%\medskip

\noindent
Д\,о\,к\,а\,з\,а\,т\,е\,л\,ь\,с\,т\,в\,о\ аналогично доказательству\linebreak
теоремы~1.
Для первого слагаемого в правой час\-ти~(\ref{e3-u1}) имеем
\begin{multline*}
\fr{1}{2\pi}\int\limits_{|t|>1/h}|\varphi(t)|^2\,dt<{}\\
{}<
e^{-\rho/h^\alpha}\fr{1}{2\pi}
\int\limits_{|t|>1/h}e^{\rho|t|^\alpha}|\varphi(t)|^2\,dt
=\varepsilon(h)Ce^{-\rho/h^r}\,,
\end{multline*}
где
$$
\varepsilon(h)= 1-\fr{\int\limits_{-1/h}^{1/h}e^{\rho|t|^\alpha}\left\vert\varphi(t)\right\vert^2\,dt}
{\int e^{\rho|t|^\alpha}|\varphi(t)|^2\,dt}\,.
$$
Второе слагаемое оцениваем так же, как в доказательстве теоремы~1.

Трудно найти в явном виде значение~$h$, минимизирующее правую часть~(\ref{e9-u1}), 
поэтому возьмем такое~$h$, при котором правая часть~(\ref{e9-u1}) имеет
простой вид, а именно:
$$
h=\left(\fr{1}{\rho}\,\ln n\right)^{-1/\alpha}\,.
$$
Тогда
\begin{multline*}
J(f_n)< \left(C+\fr{(\ln n)^{1/\alpha}}{\pi\rho^{1/\alpha}}\right)\fr{1}{n}<{}\\
{}<
\left(C+\fr{1}{\pi\rho^{1/\alpha}}\right)\fr{(\ln n)^{1/\alpha}}{n}
\end{multline*}
при условии, что $n>2$. Если, например, $f(x)$~--- стандартная нормальная
плотность распределения, то
$$
J(f_n)< \left(\fr{1}{\sqrt{2\pi}}+ \fr{\sqrt 2}{\pi}\right)\fr{\sqrt{\ln
n}}{n}\,.
$$

\medskip

\noindent
\textbf{Следствие.} \textit{Пусть характеристическая функция~$\varphi(t)$ 
убывает экспоненциально с показателем~$\alpha$ и коэффициентом~$\rho$.
Тогда}
$$
\lim\limits_{n\to\infty}e^{c\rho/h^\alpha}\left\vert{\rm B}(f_n)\right\vert=0
$$
\textit{для любого $c<2$}.

\section {Оценивание производных}

Те преимущества, которые имеет оценка $f_n(x;h)$ по сравнению со стандартными
ядерными оценками, особенно проявляются в случае, когда оценивается не
сама плотность, а ее производная. Предположим, что~$f(x)$ является 
$r$~раз дифференцируемой и необходимо оценить
$r$-ю производную $f^{(r)}(x)$. Естественной оценкой является
$r$-я производная ядерной оценки плотности~$f(x)$
(при условии, что ядро $r$~раз дифференцируемо). Далее, пусть
$$
\hat f_n(x;h)=\fr{1}{nh}\sum\limits_{j=1}^nK\left(\fr{x-X_j}{h}\right)
$$
есть ядерная оценка плотности~$f(x)$ и существует производная
$K^{(r)}(x)$. Тогда в качестве оценки производной $f^{(r)}(x)$ берется
$$
\hat f_n^{(r)}(x;h)= \fr{1}{nh^{r+1}}\sum\limits_{j=1}^nK^{(r)}\left(
\fr{x-X_j}{h}\right)\,.
$$
Пусть $J(\hat f_n^{(r)})$~--- интегральная среднеквадратическая ошибка
$\hat f_n^{(r)}(x)$ как оценки производной $f^{(r)}(x)$, т.\,е.\
$$
J(\hat f_n^{(r)})={\rm E}\int\limits_{-\infty}^\infty
\left[\hat f_n^{(r)}(x)-f^{(r)}(x)\right]^2\,dx\,.
$$
Если $K(x)$~--- стандартное ядро, то интегральная среднеквадратическая
ошибка оценки $\hat f_n^{(r)}(x;h)$
может быть записана следующим образом (при условии, что $f(x)$ имеет
$r+2$ производные и дисперсия ядра конечна):
\begin{multline*}
J(\hat f_n^{(r)})=\fr{1}{4}\,h^4\mu_2^2R(f^{(r+2)})+
\fr{1}{nh^{2r+1}}\,R(K^{(r)})+{}\\
{}+
o\left(h^4+\fr{1}{nh^{2r+1}}\right)\,,\quad h\to0\,,
\end{multline*}
где $\mu_2=\int\limits_{-\infty}^\infty x^2K(x)\,dx$.
То есть при оптимальном выборе параметра сглаживания порядок ошибки
равен $n^{-4/(2r+5)}$ и он существенно ухудшается с ростом~$r$.
В~данном параграфе будет показано, что оценка $f_n^{(r)}(x;h)$, основанная
на ядре~(\ref{e2-u1}), при определенных условиях почти свободна от этого недостатка
или во всяком случае в гораздо меньшей степени подвержена ему.

\medskip

\noindent
\textbf{Лемма~1.} \textit{Для синк-оценки}
\begin{multline}
J(f_n^{(r)})=\fr{1}{2\pi}\int\limits_{|t|>1/h}t^{2r}\left\vert\varphi(t)\right\vert^2\,dt+{}\\
{}+
\fr{1}{n}\,\fr{1}{2\pi}\int\limits_{-1/h}^{1/h}t^{2r}\left(1-
\left\vert\varphi(t)\right\vert^2\right)\,dt\,.
\label{e10-u1}
\end{multline}

\medskip

\noindent
Д\,о\,к\,а\,з\,а\,т\,е\,л\,ь\,с\,т\,в\,о\,.\ 
Обозначим $\varphi_n(t)$ выборочную характеристическую
функцию, т.\,е.\
$$
\varphi_n(t)=\fr{1}{n}\sum\limits_{j=1}^ne^{itX_j}\,.
$$
Применяя равенство Парсеваля, получаем
\begin{multline}
J(f_n^{(r)})={\rm E}\int\limits_{-\infty}^\infty(f_n^{(r)}(x;h)-f^{(r)}(x))^2\,dx={}\\
{}=
\fr{1}{2\pi}\,
{\rm E}\int\limits_{-\infty}^\infty
t^{2r}\left\vert\varphi_n(t)I_{[-1/h,1/h]}(t)-\varphi(t)\right\vert^2\,dt={}\\
{}=
\fr{1}{2\pi}\int\limits_{-\infty}^\infty
{\rm E}\left[
\vphantom{\overline{\varphi(t)}}
\left(\varphi_n(t)I_{[-1/h,1/h]}(t)-{}\right.\right.\\
\left.\left.{}-\varphi(t)\right)
(\overline{\varphi_n(t)}\ I_{[-1/h,1/h]}(t)-
\overline{\varphi(t)})\right]\,dt={}\\
{}=\fr{1}{2\pi}\int\limits_{-1/h}^{1/h} t^{2r}{\rm E}|\varphi_n(t)|^2\,dt-{}\\
{}-
\fr{1}{2\pi}\int\limits_{-1/h}^{1/h} t^{2r}
\left(\varphi(t){\rm E}\overline{\varphi_n(t)}+
\overline{\varphi(t)}{\rm E}\varphi_n(t)\right)\,dt+{}\\
{}+
\fr{1}{2\pi}\int\limits_{-\infty}^\infty t^{2r}\left\vert\varphi(t)\right\vert^2\,dt\,.
\label{e11-u1}
\end{multline}
Легко видеть, что
\begin{align}
{\rm E}\varphi_n(t)&=\varphi(t)\,;\label{e12-u1}\\
{\rm E}\overline{\varphi_n(t)}&=\overline{\varphi(t)}\,;\label{e13-u1}\\
{\rm E}\left\vert\varphi_n(t)\right\vert^2&=
{\rm E}\left\vert\fr{1}{n}\sum\limits_{j=1}^ne^{itX_j}\right\vert^2
={}\notag\\
&{}={\rm E}\left[\fr{1}{n}\sum\limits_{j=1}^ne^{itX_j}\cdot\fr{1}{n}
\sum\limits_{k=1}^ne^{-itX_k}\right]={}\notag
\end{align}

\noindent
\begin{align}
&{}=\fr{1}{n^2}\left[n+\sum\limits_{j\not=k}e^{it(X_j-X_k)}\right]={}\notag\\
&\hspace*{15mm}{}=
\fr{1}{n}+\left(1-\fr{1}{n}\right)\left\vert\varphi(t)\right\vert^2\,.
\label{e14-u1}
\end{align}
Подставляя~(\ref{e12-u1})--(\ref{e14-u1}) в правую часть~(\ref{e11-u1}), получаем~(\ref{e10-u1}).

\smallskip

Применяя лемму~1, получаем аналоги теорем предыдущего параграфа для
случая оценивания производных.

\medskip

\noindent
\textbf{Теорема 4.} \textit{Если $f(x)$ $r+m$ раз дифференцируема и квадрат ее
$(r+m)$-й производной интегрируем, то}
$$
J(f_n^{(r)})< \varepsilon(h)h^{2m}R(f^{(r+m)})+\fr{1}{\pi(2r+1)nh^{2r+1}}\,,
$$
\textit{где $\varepsilon(h)\le1$ для всех $h$ и $\varepsilon(h)\to0$ при $h\to0$}.


\medskip

\noindent
\textbf{Следствие.} \textit{Пусть выполнены условия теоремы~$4$. Тогда}
\begin{multline*}
\inf\limits_{h>0}J(f_n^{(r)})
\le {}\\
{}\le C_{m,r}R(f^{(r+m)})^{(2r+1)/(2r+2m+1)}n^{-2m/(2r+2m+1)},\hspace*{-1.15225pt}
\end{multline*}
\textit{где}
\begin{multline*}
C_{m,r}=(2\pi m)^{-2m/(2r+2m+1)}+{}\\
{}+
\fr{(2\pi m)^{(2r+1)/(2r+2m+1)}}{\pi(2r+1)}\,.
\end{multline*}


\medskip

\noindent
\textbf{Теорема 5.} \textit{Пусть}
$$
\fr{1}{2\pi}\int\limits_{-\infty}^\infty
t^{2r}e^{\rho|t|^\alpha}\left\vert\varphi(t)\right\vert^2\,dt=C<\infty\,.
$$
\textit{Тогда}
$$
J(f_n^{(r)})\le\varepsilon(h)
Ce^{-\rho/h^\alpha}+\fr{1}{\pi nh^{2r+1}}\,,
$$
\textit{где $0<\varepsilon(h)<1$ и $\varepsilon(h)\to0$ при $h\to0$}.


\medskip

\noindent
\textbf{Следствие.} \textit{Пусть выполнены условия теоремы~$5$. Тогда}
$$
\inf\limits_{h>0}J(f_n^{(r)})< \left(
C+\fr{(\ln n)^{(2r+1)/\alpha}}{\pi\rho^{(2r+1)/\alpha}}\right)
\fr{1}{n}\,.
$$

\medskip

\noindent
\textbf{Теорема~6.} \textit{Пусть характеристическая функция $\varphi(t)$
плотности~$f(x)$ удовлетворяет условию: существует $T>0$ такое, что
$f(t)=0$ при $|t|>T$. Тогда если}
$$h\le \fr{1}{T}\,,$$
\textit{то}
$$
J(f_n^{(r)})\le\fr{1}{\pi nh^{2r+1}}\,.
$$
\textit{В частности, если $h=const=1/T$, то}
$$
J(f_n^{(r)})\le\fr{T^{2r+1}}{\pi n}\,.
$$

\smallskip

Доказательства теорем~4--6 сходны с доказательствами предыдущего
параграфа, и авторы оставляют их читателю.


\section{Равномерная состоятельность и~оценивание моды}

В данном параграфе будет доказано, что оценка $f_n(x;h)$ равномерно
состоятельна:
она сходится по вероятности к оцениваемой плотности распределения
равномерно на всей действительной прямой. Кроме того, докажем,
что мода оценки является состоятельной оценкой моды оцениваемой
плотности.

Пусть $K(x)$~--- симметричное и дифферен\-ци\-ру\-емое стандартное ядро
(т.\,е.\ ядро, являющееся плотностью распределения вероятностей)
с конечной дисперсией~$\sigma^2$. Предположим, что производная
ядра имеет конечную полную вариацию, которую обозначим~$v$.
Характеристическую функцию ядра и ядерную оценку, основанную
на этом ядре, обозначим соответственно~$\psi(t)$ и~$\hat f_n(x;h)$.

\medskip

\noindent
\textbf{Лемма~2.} \textit{Если характеристическая функция $\varphi(t)$
интегрируема:}
$$
\int\limits_{-\infty}^\infty\left\vert\varphi(t)\right\vert\,dt<\infty\,,
$$
\textit{то}
$$\sup\limits_x|f_n(x;h)-\hat f_n(x;h)|\xrightarrow{a.s.} 0$$
\textit{при $n\hm\to\infty$, $h\hm\to0$, $nh\hm\to\infty$}.

\medskip

\noindent
Д\,о\,к\,а\,з\,а\,т\,е\,л\,ь\,с\,т\,в\,о\,.\
\begin{multline*}
\sup\limits_x\left\vert f_n(x;h)-\hat f_n(x;h)\right\vert\le{}\\
{}\le\fr{1}{2\pi}
\left[ \vphantom{\int\limits_{-1/h}^{1/h}}
\int\limits_{-\infty}^\infty|\psi(ht)|\cdot|\varphi_n(t)-\varphi(t)|\,dt+{}\right.\\
\left.{}+
\int\limits_{-\infty}^\infty|\varphi(t)|\cdot|\psi(ht)-I_{[-1/h,1/h]}(t)|\,dt+{}\right.\\
\left.{}+\int\limits_{-1/h}^{1/h}\left\vert \varphi_n(t)-\varphi(t)\right\vert\,dt
\right]\,.
\end{multline*}
Докажем, что каждый из трех интегралов в правой части сходится к нулю
при $n\hm\to\infty$, $h\hm\to0$, $nh\hm\to\infty$. Обозначим эти интегралы соответственно
$I_1$, $I_2$ и~$I_3$. Тогда

\noindent
$$
I_1\le\int\limits_{|t|\le n^2}\left\vert \varphi_n(t)-\varphi(t)\right\vert\,dt+
4\int\limits_{n^2}^\infty|\psi(ht)|\,dt\,.
$$
Первый интеграл в правой части сходится почти наверное к нулю при
$n\hm\to\infty$ в силу теоремы~1 из~[6]. Чтобы оценить второй интеграл,
применим неравенство

\noindent
$$
|\psi(t)|\le\fr{v}{|t|^2}\,,
$$
справедливое для всех~$t$~[4]. Применяя это неравенство, получаем

\noindent
$$
\int\limits_{n^2}^\infty|\psi(ht)|\,dt\le\fr{v}{n^2h^2}\to 0
$$
при $nh\to\infty$.
Таким образом, $I_1\xrightarrow{a.s.} 0$ при
$n\hm\to\infty$, $nh\hm\to\infty$;

\noindent
$$
I_2\le2\int\limits_0^{1/\sqrt h}\left[1-\psi(ht)\right]\,dt+
4\int\limits_{1/\sqrt h}^\infty|\varphi(t)|\,dt\,.
$$
Второй интеграл в правой части стремится к нулю при $h\hm\to 0,$ поскольку
функция $|\varphi(t)|$ интег\-ри\-ру\-ема.
Чтобы оценить первый интеграл, используем неравенство

\noindent
$$
\psi(t)\ge 1-\fr{\sigma^2t^2}{2}\,,
$$
справедливое для всех~$t$ (см., например,~[7]).
Применяя это неравенство, получаем

\noindent
$$
\int\limits_0^{1/\sqrt h}\left[1-\psi(ht)\right]\,dt\le
\fr{\sigma^2}{6}\,\sqrt h\to 0
$$
при $h\to 0$.
Таким образом, $I_2\hm\to0$ при $h\hm\to0$.

Наконец, если $nh\hm\to\infty$, то $1/h\hm\le cn$, где $c$~--- некоторая постоянная,
и, следовательно,

\noindent
$$
I_3\le\int\limits_{|t|\le cn}\left\vert \varphi_n(t)-\varphi(t)\right\vert\,dt
\xrightarrow{a.s.} 0
$$ 
при $n\to\infty$ в силу
указанной выше теоремы~1 из~[6]. Таким образом, все три интеграла
стремятся к нулю, что доказывает лемму.

\smallskip

\noindent
\textbf{Замечание.} Из условия леммы об интегрируемости функции~$\varphi$
следует, что плотность~$f(x)$ равномерно
непрерывна, однако оно несколько более ограничительно.
Выполняется, например, в том случае,
когда $f(x)$ дифференцируема и ее
производная $f'(x)$ имеет ограниченную вариацию.
\pagebreak

%\medskip

\noindent
\textbf{Теорема 6.} \textit{Если $\varphi(t)$ интегрируема, то}
\begin{equation}
\sup\limits_x|f_n(x;h)-f(x)|\xrightarrow{P} 0
\label{e15-u1}
\end{equation}
\textit{при $n\hm\to\infty$, $h\hm\to0, nh^2\hm\to\infty$}.

\medskip

\noindent
Д\,о\,к\,а\,з\,а\,т\,е\,л\,ь\,с\,т\,в\,о\,.\ 
Пусть $K(x)$~--- произвольное стандартное ядро,
удовлетворяющее условиям леммы~2 и теоремы~3A  работы~[8].
Тогда в силу указанной теоремы~3A
\begin{equation}
\sup\limits_x \left\vert\hat f_n(x;h)-f(x)\right\vert
\xrightarrow{P} 0
\label{e16-u1}
\end{equation}
при $n\hm\to\infty$, $h\hm\to0$, $nh^2\hm\to\infty$
и в силу леммы~$2$
\begin{equation}
\sup\limits_x|f_n(x;h)-\hat f_n(x;h)|\xrightarrow{P} 0
\label{e17-u1}
\end{equation}
при $n\hm\to\infty$, $h\hm\to0$, $nh^2\hm\to\infty$.
Из~(\ref{e16-u1}) и~(\ref{e17-u1}) очевидным образом следует~(\ref{e15-u1}).

\smallskip

Обозначим $\theta$ моду плотности распределения $f(x)$.
Предположим, что она единственна. Пусть $\theta_n$~---
мода оценки~$f_n(x;h)$.

\medskip

\noindent
\textbf{Теорема 7.} \textit{Если $\varphi(t)$ интегрируема, то}
$$
\theta_n\xrightarrow{P}\theta
$$
\textit{при $n\hm\to\infty$, $h\hm\to0$, $nh^2\hm\to\infty$}.


\medskip

\noindent
Д\,о\,к\,а\,з\,а\,т\,е\,л\,ь\,с\,т\,в\,о\ 
теоремы аналогично доказательству второй части теоремы~3A из~[8].


{\small\frenchspacing
{%\baselineskip=10.8pt
\addcontentsline{toc}{section}{Литература}
\begin{thebibliography}{9}

\bibitem{1-u1}
\Au{Davis~K.\,B.} Mean square error properties of density
estimates~// Ann. Statist., 1975. Vol.~3. No.\,4. P.~1025--1030.

\bibitem{2-u1}
\Au{Davis~K.\,B.} Mean integrated square error properties of
density estimates~// Ann. Statist., 1977. Vol.~5. No.\,3. P.~530--535.

\bibitem{3-u1}
\Au{Glad~I.\,K., Hjort~N.\,L., Ushakov~N.\,G.} Correction of
density estimators that are not densities~// Scand. J.~Statist.,
2003. Vol.~30. No.\,2. P.~415--427.

\bibitem{4-u1}
\Au{Ушаков~В.\,Г., Ушаков~Н.\,Г.} Некоторые неравенства для
характеристических функций плотностей с ограниченной вариацией~//
Вестн. Моск. ун-та. Сер.~15. Вычисл. матем. и киберн., 2000. №\,3. С.~40--45.

\bibitem{5-u1}
\Au{Watson~G.\,S., Leadbetter~M.\,R.} On the estimation of the
probability density~I~//  Ann. Math. Statist., 1963. Vol.~34. P.~480--491.


\bibitem{6-u1}
\Au{Cs$\ddot{\mbox{o}}$rg$\mbox{\H{o}}$ S., Totik~V.} On how long interval is the
empirical characteristic function uniformly consistent?~// Acta Sci.
Math., 1983. Vol.~45. P.~141--149.


\bibitem{7-u1}
\Au{Ushakov~N.\,G.} Selected topics in characteristic functions.~--- Utrecht: VSP, 1999.

\label{end\stat}

\bibitem{8-u12}
\Au{Parzen~E.} On estimation of a probability density function
and its mode~// Ann. Math. Statist., 1962. Vol.~33. No.\,3. P.~1065--1076.
 \end{thebibliography}
}
}


\end{multicols}           %10
\def\stat{shestakov}

\def\tit{ОБРАЩЕНИЕ ОДНОРОДНЫХ ОПЕРАТОРОВ С~ПОМОЩЬЮ
СТАБИЛИЗИРОВАННОЙ ЖЕСТКОЙ ПОРОГОВОЙ ОБРАБОТКИ
ПРИ~НЕИЗВЕСТНОЙ ДИСПЕРСИИ ШУМА$^*$}

\def\titkol{Обращение однородных операторов с~помощью
стабилизированной жесткой пороговой обработки}
%при~неизвестной дисперсии шума}

\def\aut{О.\,В.~Шестаков$^1$}

\def\autkol{О.\,В.~Шестаков}

\titel{\tit}{\aut}{\autkol}{\titkol}

\index{Шестаков О.\,В.}
\index{Shestakov O.\,V.}


{\renewcommand{\thefootnote}{\fnsymbol{footnote}} \footnotetext[1]
{Работа выполнена при частичной финансовой поддержке РФФИ (проект 19-07-00352).}}


\renewcommand{\thefootnote}{\arabic{footnote}}
\footnotetext[1]{Московский государственный университет им.\ М.\,В.~Ломоносова, 
кафедра математической статистики факультета вычислительной математики и~кибернетики; 
Институт проб\-лем информатики Федерального исследовательского центра 
<<Информатика и~управ\-ле\-ние>> Российской академии наук, \mbox{oshestakov@cs.msu.su}}


\vspace*{-6pt}


\Abst{При обращении линейных однородных операторов обычно необходимо использовать 
методы регуляризации, поскольку наблюдаемые данные, как правило, зашумлены. 
Для подавления шума часто используется пороговая обработка 
вейвлет-ко\-эф\-фи\-ци\-ен\-тов функции наблюдаемого сигнала. 
Пороговая обработка стала популярным инструментом подавления 
шума благодаря своей простоте, вы\-чис\-ли\-тель\-ной эффективности и~воз\-мож\-ности 
адаптации к~функциям, имеющим на разных участках разную степень регулярности. 
Рассматривается предложенный недавно стабилизированный метод жесткой 
пороговой обработки, в~котором устранены основные недостатки мягкой и~жесткой 
пороговой обработки, и~исследуются статистические свойства этого метода. 
В~модели данных с~аддитивным гауссовским шумом с~неизвестной дисперсией 
проведен анализ несмещенной оценки среднеквадратичного риска и~показано, 
что при определенных условиях данная оценка является асимптотически нормальной, 
при этом дисперсия предельного распределения зависит от способа оценивания 
дисперсии шума.}

\KW{вейвлеты; пороговая обработка; несмещенная оценка риска; 
асимптотическая нормальность; сильная состоятельность}

\DOI{10.14357/19922264190107}
  
%\vspace*{4pt}


\vskip 10pt plus 9pt minus 6pt

\thispagestyle{headings}

\begin{multicols}{2}

\label{st\stat}

\section{Введение}

В медицинских, физических, астрономических и~других научных проблемах часто 
возникает задача получить представление об объекте, который описывается 
некоторой функцией~$f$, имея возможность наблюдать только функцию~$Kf$, где~$K$~--- 
некоторый линейный оператор. При этом часто нельзя просто применить 
к~наблюдаемым данным обратный оператор~$K^{-1}$, поскольку эти данные, как правило, 
содержат шум и~задача обращения оператора~$K$ некорректно поставлена. 
К~тому же обычно дис\-пер\-сия шума неизвестна и~ее необходимо оценивать 
по наблюдаемым данным. 

Одним из популярных инструментов при регуляризации 
процедуры обращения служит вейв\-лет-раз\-ло\-же\-ние с~последующей 
пороговой обработкой вейв\-лет-ко\-эф\-фи\-ци\-ен\-тов. Наиболее распростра\-нен\-ные 
виды пороговой обработки~--- жесткая и~мягкая. В~работе~\cite{HL10} 
был предложен метод стабилизированной жесткой пороговой обработки, который 
объединяет в~себе преимущества этих двух видов. 
В~ситуации, когда дисперсия шума предполагается известной, в~работе~\cite{SH18} 
доказана асимптотическая нормальность оценки среднеквадратичного риска пороговой 
обработки. 

В~данной работе исследуется влияние способов оценивания дисперсии шума 
на характеристики предельного распределения оценки среднеквадратичного риска. 
Для метода мягкой пороговой обработки подобные исследования проводились 
в~работах~\cite{KS11-1, KS11-2}.

\section{Обращение линейных однородных операторов с~помощью вейглет-вейвлет-разложения}

В данной работе рассматривается метод обращения линейных однородных операторов, 
основанный на вейг\-лет-вейв\-лет-раз\-ло\-же\-нии~\cite{AS98}. Линейный оператор~$K$ 
называется однородным, если
$$
K\left[f\left(a\left(x-x_0\right)\right)\right]=a^{-\alpha}(Kf)\left[a\left(x-x_0\right)\right]
$$
для любого $x_0$ и~любого $a\hm>0$. Параметр~$\alpha$ называется показателем 
однородности. Примерами линейных однородных операторов служат оператор 
интегрирования, преобразование Гильберта и~преобразование Абеля.

Относительно наблюдаемой функции~$Kf$ будем предполагать, что она определена на 
конечном отрезке и~равномерно регулярна по Липшицу с~некоторым показателем $\gamma\hm>0$. 
Вейв\-лет-разложение~$Kf$ представляет собой ряд по ортонормированному базису
\begin{equation}
\label{wavelet_decomp}
Kf = \sum\limits_{j,k \in Z} \langle Kf,\psi_{j,k} \rangle \psi_{j,k}\,,
\end{equation}
где $\psi(t)$~--- некоторая материнская вейв\-лет-функ\-ция, 
а~$\psi_{j,k}(t) \hm= 2^{j/2}\psi(2^jt \hm- k)$. Индекс~$j$ в~(\ref{wavelet_decomp}) 
называется масштабом, а~индекс~$k$~--- сдвигом. Если вейв\-лет-функ\-ция 
обладает определенными свойствами регулярности~\cite{Mal99}, 
то для коэффициентов разложения в~(\ref{wavelet_decomp}) справедливо
\begin{equation}
\label{wavelet_decay}
\abs{\langle Kf, \psi_{j,k} \rangle} \leqslant \fr{C_f}
{2^{j \left( \gamma + 1/2 \right)}}\,,
\end{equation}
где $C_f$~--- некоторая положительная константа.

Поскольку оператор~$K$ линеен и~однороден, существуют такие функции~$u_{j,k}$, 
что $\langle f,u_{j,k}\rangle\hm=\langle Kf,\psi_{j,k}\rangle$. При этом функция~$f$ 
представляется в~виде ряда
\begin{equation}
\label{VWD}
f = \sum\limits_{j,k \in Z}\beta_{j,k}\langle Kf,\psi_{j,k}\rangle u_{j,k},
\end{equation}
где $u_{j,k} = K^{-1}\psi_{j,k}/\beta_{j,k}$, $\beta_{j,k}\hm=2^{\alpha j}\beta_{00}$, 
$\beta_{00} \hm= \norm{K^{-1}\psi}$ (функции~$u_{j,k}$, как и~$\psi_{j,k}$, 
представляют собой сдвиги и~растяжения одной материнской функции~$u$ и~называются 
вейглетами). При соответствующем выборе~$\psi(t)$ последовательность~$\{u_{j,k}\}$ 
образует устойчивый базис~\cite{L97}. Формула~(\ref{VWD}) и~есть основа метода 
вейг\-лет-вейв\-лет-раз\-ло\-же\-ния.

\section*{Пороговая обработка эмпирических коэффициентов}

При фактических измерениях значения функции сигнала регистрируются 
в~дискретных отсчетах, при этом такие значения, как правило, зашумлены. 
Рассмотрим сле\-ду\-ющую модель данных \mbox{с~шумом}:
\begin{equation*}
%\label{Data_Model}
X_i = (Kf)_i + \epsilon_i\,, \enskip i = 1, \dots, 2^J\,, %\notag
\end{equation*}
где $2^J$~--- число отсчетов; $(Kf)_i$~--- незашумленные значения функции сигнала; 
$\epsilon_i$~--- независимые нормально распределенные случайные величины с~нулевым 
средним и~дисперсией~$\sigma^2$.
После применения дискретного вейв\-лет-пре\-об\-ра\-зо\-ва\-ния 
получается следующая модель зашумленных вейв\-лет-ко\-эф\-фи\-ци\-ен\-тов:
\begin{equation*}
Y_{j,k}=\mu_{j,k}+\epsilon^W_{j,k},\enskip 
j=0,\ldots,J-1,\ k=0,\ldots,2^{j}-1\,,
\end{equation*}
где $\epsilon^W_{j,k}$ независимы и~распределены так же, как и~$\epsilon_i$, 
а~$\mu_{j,k}\hm= 2^{J/2}\langle Kf,\psi_{j,k}\rangle$~\cite{Mal99}.

Для подавления шума и~построения оценки функции сигнала к~коэффициентам~$Y_{j,k}$ 
обычно применяется функция жесткой пороговой обработки 
$\rho_{H}(y,T)\hm=x\textbf{I}(\abs{y}>T)$ или мягкой пороговой 
обработки $\rho_{S}(y,T)\hm=\textbf{sgn}(x)\left(\abs{y}-T\right)_{+}$ 
с~порогом~$T$. При таком подходе обнуляются коэффициенты, абсолютная величина 
которых ниже порога, так как в~силу~(\ref{wavelet_decay}) основная часть
 полезного сигнала содержится в~относительно небольшом числе больших по 
 модулю коэффициентов.

Каждому из этих видов пороговой обработки присущи свои недостатки. 
Жесткая пороговая функция разрывна, и~это приводит к~отсутствию устойчивости 
при выборе порога~\cite{B96} и~невозможности построения несмещенной оценки 
среднеквадратичного риска~\cite{J01}. При мягкой пороговой обработке в~оценке 
функции появляется дополнительное смещение. Чтобы частично избежать этих недостатков, 
в~работе~\cite{HL10} был предложен новый вид пороговой обработки, представляющий 
собой сглаженный (стабилизированный) аналог жесткой пороговой обработки. 
В~этом методе оценки~$\mu_{j,k}$ вычисляются по формулам:
\begin{equation*}
\widehat{\mu}_{j,k}=\Expect 
\left[\rho_{H}(Y_{j,k}+\lambda\xi_{j,k},T_j)|Y_{j,k}\right], %\notag
\end{equation*}
где случайные величины~$\xi_{j,k}$ имеют стандартное нормальное распределение и~не 
зависят от~$Y_{j,k}$, а~$\lambda\hm>0$~--- 
параметр стабилизации, отвечающий за степень сглаживания. Вычисляя математическое 
ожидание, получаем:
\begin{multline*}
\hspace*{-8.37947pt}\widehat{\mu}_{j,k}=Y_{j,k}\left[\Phi\!\left(-\fr{T_j+Y_{j,k}}
{\lambda}\right)+1-\Phi\left(\fr{T_j-Y_{j,k}}{\lambda}\right)\!\right]+{}\\
{}+
\lambda\left[\phi\left(\fr{T_j-Y_{j,k}}{\lambda}\right)-
\phi\left(\fr{T_j+Y_{j,k}}{\lambda}\right)\right]. %\notag
\end{multline*}
Достоинством такого метода является бесконечная дифференцируемость~$\widehat{\mu}_{j,k}$ 
по~$Y_{j,k}$, что приводит к~более робастным оценкам~\cite{HL10}. Заметим также, 
что при $\lambda\hm\to0$ получается обычный метод жесткой пороговой обработки. 
В~данной работе параметр~$\lambda$ предполагается фиксированным, а~в~качестве~$T_j$ 
для каждого масштаба~$j$ выбирается порог $T_j\hm=\sigma\sqrt{2\ln 2^j}$. 
Такой порог получил название <<универсальный>>, так как он не зависит 
от наблюдаемых данных. И~при жесткой, и~при мягкой пороговой обработке этот 
порог обеспечивает близость среднеквадратичного риска к~минимальному~\cite{Mal99}.

\section{Несмещенная оценка среднеквадратичного риска}

Среднеквадратичный риск метода пороговой обработки определяется по формуле:
\begin{equation}
\label{Risk}
R_J(\sigma)=\sum\limits_{j=0}^{J-1}\sum\limits_{k=0}^{2^j-1}\beta^2_{j,k}
\Expect\left(\widehat{\mu}_{j,k}(\sigma)-\mu_{j,k}\right)^2.
\end{equation}
В~\cite{HL10} показано, что при стабилизированной жесткой пороговой обработке
\begin{multline*}
\Expect\left(\widehat{\mu}_{j,k}(\sigma)-\mu_{j,k}\right)^2={}\\
{}=
\Expect\left[(Y_{j,k}-\widehat{\mu}_{j,k}(\sigma))^2+
2\sigma^2\fr{\partial}{\partial Y_{j,k}}\,\widehat{\mu}_{j,k}(\sigma)\right]-
\sigma^2, %\notag
\end{multline*}
где
\begin{multline*}
\fr{\partial}{\partial Y_{j,k}}\widehat{\mu}_{j,k}(\sigma)={}\\
{}=\Phi\left(-\fr{T_j+Y_{j,k}}{\lambda}\right)+1-
\Phi\left(\fr{T_j-Y_{j,k}}{\lambda}\right)+{}\\
{}+
\fr{T_j}{\lambda}\left[\phi\left(\fr{T_j-Y_{j,k}}{\lambda}\right)+
\phi\left(\fr{T_j+Y_{j,k}}{\lambda}\right)\right]. %\notag
\end{multline*}
Таким образом, величина
\begin{multline}
\label{Risk_Estimate}
\widehat{R}_J(\sigma)=\sum\limits_{j=0}^{J-1}\sum\limits_{k=0}^{2^j-1}
\beta^2_{j,k}
\Bigg[
\left(
Y_{j,k}-
\widehat{\mu}_{j,k}(\sigma)\right)^2+{}\\
{}+2\sigma^2\fr{\partial}{\partial Y_{j,k}}\,\widehat{\mu}_{j,k}(\sigma)-
\sigma^2
\Bigg]
\end{multline}
является несмещенной оценкой~$R_J$, не зависящей от ненаблюдаемых значений~$\mu_{j,k}$.

В работе~\cite{SH18} доказано следующее утверждение, устанавливающее 
асимптотическую нормальность оценки~(\ref{Risk_Estimate}) и~позволяющее строить 
асимптотические доверительные интервалы для риска~(\ref{Risk}).

\smallskip

\noindent
\textbf{Теорема 1.} 
\textit{Пусть $K$~--- линейный однородный оператор с~показателем 
однородности $\alpha\hm>0$, а~$Kf$ задана на конечном отрезке и~равномерно 
регулярна по Липшицу с~показателем $\gamma\hm>0$. Тогда}
\begin{equation*}
%\label{Normality}
{\sf P}\left(\fr{\widehat{R}_J(\sigma)-
R_J(\sigma)}{D_J}<x\right)\Rightarrow\Phi(x)\,, %\notag
\end{equation*}
\textit{где}
$$
D^2_J=\fr{2\sigma^4\beta_{0,0}^4}{2^{4\alpha+1}-1}2^{(4\alpha+1)J}\,.
$$

\section{Виды оценок дисперсии шума}

Как правило, дисперсия~$\sigma^2$ неизвестна и~вместо ее точного значения 
необходимо использовать некоторую оценку~$\hat{\sigma}^2$, которая обычно 
строится по половине всех вейв\-лет-ко\-эф\-фи\-ци\-ен\-тов для $j\hm=J\hm-1$, 
так как в~силу~(\ref{wavelet_decay}) эти коэффициенты фактически содержат только шум. 
При этом порог вычисляется по формуле $\hat{T}_j\hm=\hat{\sigma}\sqrt{2\ln 2^j}$.

В качестве оценки~$\sigma^2$ (или $\sigma$) в~данной работе 
рассматривается выборочная дисперсия
\begin{equation}
\label{SampleVarianceDef}
\widehat{\sigma}_S^2=\fr{1}{2^{J-1}}
\sum\limits_{k=0}^{2^{J-1}-1}Y_{J-1,k}^2-\overline{Y}^2,
\end{equation}
где
\begin{equation*}
\overline{Y}=\fr{1}{2^{J-1}}\sum\limits_{k=0}^{2^{J-1}-1}Y_{J-1,k}\,,
\end{equation*}
а также соответствующим образом нормированный выборочный интерквартильный 
размах~$\widehat{\sigma}_{R}$ и~выборочное абсолютное медианное 
отклонение~$\widehat{\sigma}_{M}$, которые определяются сле\-ду\-ющим образом:
\begin{align}
\widehat{\sigma}_{R}&=\fr{Y_{(J-1,3/4)}-Y_{(J-1,1/4)}}{2\xi_{3/4}}\,;
\label{IQR_Definition}
\\
\widehat{\sigma}_{M}&=\fr{\mathop{\mbox{med}}\limits_{0\leqslant k\leqslant 2^{J-1}-1}|Y_{J-1,k}-\mathop{\mbox{med}}\limits_{0\leqslant l\leqslant 2^{J-1}-1} Y_{J-1,l}|}{\xi_{3/4}}\,.
\label{MAD_Definition}
\end{align}
Здесь $Y_{(J-1,1/4)}$ и~$Y_{(J-1,3/4)}$~--- выборочные квантили порядка~$1/4$ и~$3/4$, 
построенные по выборке из половины всех вейв\-лет-ко\-эф\-фи\-ци\-ен\-тов при 
$j\hm=J\hm-1$; $\xi_{3/4}$~--- теоретическая квантиль порядка~$3/4$ 
стандартного нормального распределения ($\xi_{3/4}\hm\approx0,6745$); $\mbox{med}$ 
обозначает выборочную медиану.

Выборочная дисперсия служит самой популярной оценкой величины~$\sigma^2$, и~в~случае 
отсутствия выбросов она наиболее предпочтительна. Однако в~случае, когда 
оценка дисперсии строится по выборке сигнала, естественно ожидать, 
что выборка не будет однородной. Преимущество использования последних 
двух оценок заключается в~их ро\-баст\-ности, т.\,е.\ нечувствительности к~выбросам.

\section{Предельная дисперсия оценки среднеквадратичного риска}

Способ оценивания дисперсии шума влияет на вид предельной дисперсии 
оценки среднеквадратичного риска. Подобный эффект наблюдается и~при 
мягкой пороговой обработке~[4].

\noindent
\textbf{Теорема~2.}\ \textit{Пусть $Kf$ задана на конечном отрезке и~равномерно 
регулярна по Липшицу с~показателем $\gamma\hm>1/4$, а оценка дисперсии 
шума задана соотношением}~\eqref{SampleVarianceDef}. \textit{Тогда}
\begin{equation}
\label{CLT_Operator_SampleVar_Sigma}
\mathsf{P}\left(\frac{\widehat{R}_J(\widehat{\sigma}_S)-R_J(\sigma)}{D_J}<x\right)
\Rightarrow \Phi_{\Upsilon_1}(x),\notag
\end{equation}
\textit{где $\Phi_{\Upsilon_1}(x)$~--- функция распределения нормального 
закона с~нулевым средним и~дисперсией}
$$
\Upsilon_1^2=\fr{1}{2^{4\alpha+1}}+
\fr{2^{4\alpha+1}-1}{2^{4\alpha+1}\left(2^{2\alpha+1}-1\right)^2}\,.
$$

\noindent
Д\,о\,к\,а\,з\,а\,т\,е\,л\,ь\,с\,т\,в\,о\,.\ \ Обозначим
\begin{multline*}
\widehat{U}_J(\sigma)=\sum\limits_{j=0}^{J-1}\sum\limits_{k=0}^{2^j-1}
\beta^2_{j,k}\Bigg[
\left(Y_{j,k}-\widehat{\mu}_{j,k}(\sigma)\right)^2+{}\\
{}+2\sigma^2\fr{\partial}{\partial Y_{j,k}}\widehat{\mu}_{j,k}(\sigma)\Bigg] %\notag
\end{multline*}
и запишем $\widehat{R}_J(\hat{\sigma}_S)-R_J(\sigma)$ в~виде
\begin{multline*}
%\label{Three_Sums}
\widehat{R}_J(\hat{\sigma}_S)-R_J(\sigma)={}\\
{}=\left[\widehat{U}_J(\hat{\sigma}_S)-\widehat{U}_J(\sigma)\right]+
\left[\widehat{R}_J(\sigma)-R_J(\sigma)\right]+{}\\
{}+
\fr{2^{(2\alpha+1)J}-1}{2^{2\alpha+1}-1}(\sigma^2-\hat{\sigma}^2_S)
\equiv S_1+S_2+S_3\,.
\end{multline*}

Повторяя рассуждения из работ~\cite{KS11-1, KS11-2} и~учитывая, что если $\gamma\hm>1/4$, 
то выполнено $2^{J/2}\overline{Y}^2\stackrel{{\sf P}}{\to} 0$ при 
$J\hm\rightarrow\infty$~\cite{KS11-2}, можно показать, что
\begin{equation*}
{\sf P}\left(\fr{S_2+S_3}{D_J}<x\right)\Rightarrow\Phi_{\Upsilon_1}(x)\,.%\notag
\end{equation*}
% на самом деле с~условием Линдеберга чуть по-другому (без ограниченности слагаемых). Но дисперсия равномерно ограничена -- значит выполнено.

Докажем, что $D_J^{-1}S_1\stackrel{{\sf P}}{\to}0$ при $J\hm\rightarrow\infty$. 
Пусть $C_\delta\hm>0$~--- некоторая константа, а $\delta_J\hm=C_\delta J^{1/2}2^{-J/2}$. 
Запишем
\begin{multline*}
S_1=\mathbf{1}\left(\abs{\sigma^2-\hat{\sigma}^2_S}>\delta_J\right)S_1+{}\\
{}+
\mathbf{1}\left(\abs{\sigma^2-\hat{\sigma}^2_S}\leqslant\delta_J\right)
S_1\equiv S'_1+S''_1. %\notag
\end{multline*}
Для произвольного $\varepsilon\hm>0$
\begin{equation*}
{\sf P}\left(S'_1>\varepsilon\right)\leqslant{\sf P}
\left(\abs{\sigma^2-\hat{\sigma}^2_S}>\delta_J\right). %\notag
\end{equation*}
При выполнении условий теоремы, если константа~$C_\delta$ достаточно велика, 
то найдется константа~$\tilde{C}_\delta>0$ такая, что~\cite{KS11-2}
\begin{equation*}
{\sf P}\left(\abs{\sigma^2-\hat{\sigma}^2_S}>\delta_J\right)
\leqslant\tilde{C}_\delta2^{-J/2}. %\notag
\end{equation*}
%% комментарии по поводу этого неравенства и~загрязнения выборки есть в~диссертации
Следовательно, $S'_1\stackrel{P}{\to}0$ при $J\hm\rightarrow\infty$.

Обозначим слагаемые в~сумме~$S''_1$ через~$F_{j,k}(\hat{\sigma}_S)$. Пусть 
$A_j\hm=\sqrt{A\ln 2^j}$, где $0\hm<A\hm<2(\sigma^2\hm-\delta_J)$. Имеем:

\noindent
\begin{multline*}
\hspace*{-9.9pt}\sum\limits_{j=0}^{J-1}\sum\limits_{k=0}^{2^j-1}F_{j,k}\left(\hat{\sigma}_S\right)=
\sum\limits_{j=0}^{J-1}\sum\limits_{k=0}^{2^j-1}
\mathbf{1}(\abs{Y_{j,k}}\leqslant A_j)F_{j,k}(\hat{\sigma}_S)+{}\\
{}+
\sum\limits_{j=0}^{J-1}\sum\limits_{k=0}^{2^j-1}
\mathbf{1}\left(\abs{Y_{j,k}}>A_j\right)F_{j,k}(\hat{\sigma}_S)
\equiv  W_1+W_2. %\notag
\end{multline*}
Рассмотрим $W_1$. Учитывая определения $\widehat{\mu}_{j,k}(\sigma)$, 
$({\partial}/{\partial Y_{j,k}})\widehat{\mu}_{j,k}(\sigma)$ и~$A_j$, 
можно убедиться, что найдут\-ся константы $C_1\hm>0$ и~$\theta\hm>0$ такие, что
\begin{equation*}
\abs{\mathbf{1}\left(\abs{Y_{j,k}}\leqslant A_J\right)
F_{j,k}(\hat{\sigma}_S)}\leqslant C_1 
J^{5/2}2^{(2\alpha-\theta)j-J/2}\;\;\mbox{п.в.} %\notag
\end{equation*}
% поскольку выполнено \mathbf{1}(\abs{\sigma^2-\hat{\sigma}^2_S}\leqslant\delta_J). В логарифме степень: от Y идет 1, от T идет 1, от \delta_J идет 1/2 но для J, а не для j, поэтому берем для всех J^{5/2}. В степени 2: 2\alpha от \beta{j,k}, \theta из-за выбора A, J/2 от \delta_J
Следовательно, $D_J^{-1}W_1\hm\rightarrow 0$ п.в.\ при $J\hm\rightarrow\infty$.

Далее для слагаемых~$W_2$ имеем:
\begin{multline*}
\left\vert \mathbf{1}\left(
\left\vert Y_{j,k}\right\vert
> A_J\right)F_{j,k}
\left(\hat{\sigma}_S\right)\right\vert
\leqslant{}\\
{}\leqslant C_2 J^{3/2}2^{2\alpha j-J/2} 
\mathbf{1}\left( \left\vert Y_{j,k}\right\vert > A_J\right) 
\left\vert Y_{j,k}\right\vert^2\;\;\mbox{п.в.},
%\notag
\end{multline*}
% поскольку выполнено \mathbf{1}(\abs{\sigma^2-\hat{\sigma}^2_S}\leqslant\delta_J). В логарифме от T идет 1, от \delta_J идет 1/2.
где $C_2>0$~--- некоторая константа. Учитывая распределение~$Y_{j,k}$, 
нетрудно убедиться, что
\begin{equation*}
\Expect\frac{1}{D_J} \sum\limits_{j=0}^{J-1}
\sum\limits_{k=0}^{2^j-1} J^{3/2}2^{2\alpha j-J/2} 
\mathbf{1}\left(\abs{Y_{j,k}}> A_j\right)
\abs{Y_{j,k}}^2\to 0
\end{equation*}
при $J\rightarrow\infty$. %\notag
Следовательно, используя неравенство Маркова, получаем, что
\begin{equation*}
D_J^{-1}W_2\stackrel{{\sf P}}{\to}0\;\;\mbox{при}\;J\rightarrow\infty\,. %\notag
\end{equation*}
Таким образом, $D_J^{-1}S_1\stackrel{{\sf P}}{\to}0$ при $J\hm\rightarrow\infty$.

Теорема доказана.

\smallskip

Рассмотрим теперь ситуацию, когда в~качестве оценки~$\sigma$ используется 
величина~$\widehat{\sigma}_{R}$ или~$\widehat{\sigma}_{M}$. 
В~этом случае повышаются требования к~гладкости функции сигнала.

\smallskip

\noindent
\textbf{Теорема~3.}\
\textit{Пусть~$Kf$ задана на конечном отрезке и~равномерно регулярна по 
Липшицу с~показателем $\gamma\hm>1/2$, а оценка дисперсии шума~$\hat{\sigma}$ 
задана соотношением}~\eqref{IQR_Definition} 
\textit{или соотношением}~\eqref{MAD_Definition}. \textit{Тогда}
\begin{equation*}
\label{CLT_Operator_RobVar_Sigma}
\mathsf{P}\left(\fr{\widehat{R}_J(\widehat{\sigma})-R_J(\sigma)}{D_J}<x\right)
\Rightarrow \Phi_{\Upsilon_2}(x)\,, %\notag
\end{equation*}
где $\Phi_{\Upsilon_2}(x)$~--- функция распределения нормального закона 
с~нулевым средним и~дисперсией
\begin{multline*}
\Upsilon_2^2=1+\fr{2^{4\alpha+1}-1}{4(2^{2\alpha+1}-1)^2
\xi_{3/4}^2(\phi(\xi_{3/4}))^2}-{}\\
{}-
\fr{2^{4\alpha+1}-1 }{2^{2\alpha-1}(2^{2\alpha+1}-1)}\,.
\end{multline*}

\noindent
Д\,о\,к\,а\,з\,а\,т\,е\,л\,ь\,с\,т\,в\,о\,.\ \
Как и~в~предыдущей теореме, запишем
$\widehat{R}_J(\hat{\sigma})\hm-R_J(\sigma)\hm=S_1\hm+S_2\hm+S_3.$
Учитывая,\linebreak\vspace*{-12pt}

\pagebreak

\noindent
 что $\gamma\hm>1/2$, и~поступая, как в~работах~\cite{SH18, KS11-2, SH12}, 
с~использованием разложения Бахадура для выборочных квантилей~\cite{S80} и~выборочного 
абсолютного медианного отклонения~\cite{SM09}, можно показать, что
\begin{equation*}
{\sf P}\left(\fr{S_2+S_3}{D_J}<x\right)\Rightarrow\Phi_{\Upsilon_2}(x)\,. %\notag
\end{equation*}
% на самом деле с~условием Линдеберга чуть по-другому (без ограниченности слагаемых). Но дисперсия равномерно ограничена -- значит выполнено.

Используя экспоненциальные неравенства для выборочных квантилей~\cite{S80} 
и~выборочного абсолютного медианного отклонения~\cite{SM09}, получаем, что при 
выполнении условий теоремы найдется такая константа $C_\delta\hm>0$, что при 
$\delta_J\hm=C_\delta J^{1/2}2^{-J/2}$ для некоторой константы~$\widetilde{C}_\delta>0$ 
выполнено:
\begin{align*}
\mathsf{P}\left(\abs{\widehat{\sigma}_{R}-\sigma}>\delta_J\right)
&\leqslant\widetilde{C}_\delta2^{-J/2}\,;
\\
\mathsf{P}\left(\abs{\widehat{\sigma}_{M}-\sigma}>\delta_J\right)
&\leqslant\widetilde{C}_\delta2^{-J/2}\,. %\notag
\end{align*}
%% комментарии по поводу этого неравенства и~загрязнения выборки есть в~диссертации
Далее, повторяя рассуждения предыдущей теоремы, заключаем, что 
$D_J^{-1}S_1\stackrel{{\sf P}}{\to}0$ при $J\hm\rightarrow\infty$.


Теорема доказана.



{\small\frenchspacing
 {%\baselineskip=10.8pt
 \addcontentsline{toc}{section}{References}
 \begin{thebibliography}{99}

\bibitem{HL10}
\Au{Huang H.-C., Lee~T.\,C.\,M.} 
Stabilized thresholding with generalized sure for image denoising~// 
IEEE 17th  Conference (International) on Image Processing
Proceedings.~--- IEEE, 2010. P.~1881--1884.

\bibitem{SH18}
\Au{Shestakov O.\,V.} 
Nonlinear regularization of inverse problems for linear homogeneous transforms 
by the stabilized hard thresholding~// J.~Math. Sci., 2018. Vol.~234. No.\,6. P.~780--785.

\bibitem{KS11-1}
\Au{Кудрявцев А.\,А., Шестаков~О.\,В.} 
Асимптотика оценки риска при вейг\-лет-вейв\-лет разложении наблюдаемого сигнала~// 
T-Comm~--- телекоммуникации и~транспорт, 2011. №\,2. С.~54--57.

\bibitem{KS11-2}
\Au{Кудрявцев А.\,А., Шестаков~О.\,В.} 
Асимптотическое распределение оценки риска пороговой обработки 
вейг\-лет-ко\-эф\-фи\-ци\-ен\-тов сигнала при неизвестном уровне шума~// 
T-Comm~--- телекоммуникации и~транспорт, 2011. №\,5. С.~24--30.

\bibitem{AS98}
\Au{Abramovich F., Silverman~B.\,W.} 
Wavelet decomposition approaches to statistical inverse problems~// 
Biometrika, 1998. Vol.~85. No.\,1. P. 115--129.

\bibitem{Mal99}
\Au{Mallat S.} A~Wavelet tour of signal processing.~--- 
New York, NY, USA: Academic Press, 1999. 857~p.

\bibitem{L97}
\Au{Lee N.} Wavelet-vaguelette decompositions and homogenous equations.~--- 
West Lafayette, IN, USA: Purdue University, 1997.  PhD Thesis. 103~p.

\bibitem{B96}
\Au{Breiman L.} Heuristics of instability and stabilization in model selection~// 
Ann. Stat., 1996. Vol.~24. No.\,6. P.~2350--2383.

\bibitem{J01}
\Au{Jansen M.} Noise reduction by wavelet thresholding.~--- 
Lecture notes in statistics ser.~--- New York, NY, USA: Springer Verlag,
2001. Vol.~161. 196~p.

\bibitem{SH12}
\Au{Шестаков О.\,В.} О~скорости сходимости оценки риска пороговой обработки 
вейв\-лет-ко\-эф\-фи\-ци\-ен\-тов к~нормальному закону при использовании 
робастных оценок дисперсии~// Информатика и~её применения, 2012. Т.~6. Вып.~2. 
С.~122--128.

\bibitem{S80}
\Au{Serfling R.} Approximation theorems of mathematical statistics.~--- 
New York, NY, USA: John Wiley \& Sons, 1980. 371~p.

\bibitem{SM09}
\Au{Serfling R., Mazumder~S.} 
Exponential probability inequality and convergence results for the median 
absolute deviation and its modifications~// Stat. Probabil. Lett., 2009. 
Vol.~79. No.\,16. P.~1767--1773.
 \end{thebibliography}

 }
 }

\end{multicols}

\vspace*{-3pt}

\hfill{\small\textit{Поступила в~редакцию 14.12.18}}

\vspace*{8pt}

%\pagebreak

%\newpage

%\vspace*{-28pt}

\hrule

\vspace*{2pt}

\hrule

%\vspace*{-2pt}

\def\tit{INVERSION OF~HOMOGENEOUS OPERATORS USING~STABILIZED HARD THRESHOLDING 
WITH~UNKNOWN NOISE VARIANCE}

\def\titkol{Inversion of~homogeneous operators using~stabilized hard thresholding 
with~unknown noise variance}

\def\aut{O.\,V.~Shestakov}

\def\autkol{O.\,V.~Shestakov}

\titel{\tit}{\aut}{\autkol}{\titkol}

\vspace*{-11pt}


\noindent
Department of Mathematical Statistics, Faculty of Computational Mathematics and Cybernetics, M.V. Lomonosov Moscow State University, 1-52 Leninskiye Gory, GSP-1, Moscow 119991, Russian Federation
Institute of Informatics Problems, Federal Research Center 
``Computer Science and Control'' of the Russian Academy of Sciences, 44-2~Vavilov Str., 
Moscow 119333, Russian Federation

\def\leftfootline{\small{\textbf{\thepage}
\hfill INFORMATIKA I EE PRIMENENIYA~--- INFORMATICS AND
APPLICATIONS\ \ \ 2019\ \ \ volume~13\ \ \ issue\ 1}
}%
 \def\rightfootline{\small{INFORMATIKA I EE PRIMENENIYA~---
INFORMATICS AND APPLICATIONS\ \ \ 2019\ \ \ volume~13\ \ \ issue\ 1
\hfill \textbf{\thepage}}}

\vspace*{6pt}



\Abste{When inverting linear homogeneous operators, it is necessary to use 
regularization methods, since observed data are usually noisy. For noise suppression, 
threshold processing of  wavelet coefficients of the observed signal function 
is often used. Threshold processing has become a~popular noise suppression tool 
due to its simplicity, computational efficiency, and ability to adapt to functions 
that have different degrees of regularity at different domains. The paper 
discusses the recently proposed stabilized hard thresholding method that eliminates 
the main
drawbacks of soft and hard thresholding methods and studies statistical 
properties of this method. In the data model\linebreak\vspace*{-12pt}}

\Abstend{with an additive Gaussian noise with 
unknown variance, an unbiased estimate of the mean square risk is analyzed and it 
is shown that under certain conditions, this estimate is asymptotically normal and 
the variance of the limit distribution depends on the type of estimate of noise variance.}


\KWE{wavelets; threshold processing; unbiased risk estimate; asymptotic normality;
strong consistency}




\DOI{10.14357/19922264190107}

%\vspace*{-14pt}

\Ack
\noindent
This research was partly supported by the Russian  
Foundation for Basic Research (project No.\,19-07-00352).




%\vspace*{6pt}

  \begin{multicols}{2}

\renewcommand{\bibname}{\protect\rmfamily References}
%\renewcommand{\bibname}{\large\protect\rm References}

{\small\frenchspacing
 {%\baselineskip=10.8pt
 \addcontentsline{toc}{section}{References}
 \begin{thebibliography}{99}
\bibitem{1-sh-1}
\Aue{Huang, H.-C., and T.\,C.\,M.~Lee.} 2010. 
Stabilized thresholding with generalized sure for image denoising. 
\textit{IEEE 17th Conference (International) on Image Processing}. IEEE. 1881--1884.

 

\bibitem{2-sh-1}
\Aue{Shestakov, O.\,V.} 2018. 
Nonlinear regularization of inverse problems for linear homogeneous transforms 
by the stabilized hard thresholding. 
\textit{J.~Math. Sci.} 234(6):780--785.

\bibitem{3-sh-1}
\Aue{Kudryavtsev, A.\,A., and O.\,V.~Shestakov.} 2011. Аsimptotika otsenki riska pri 
veyglet-veyvlet razlozhenii nablyuda\-emo\-go signala [The average risk assessment 
of the wavelet decomposition of the signal].
\textit{T-Comm~--- Telecommunications and Their Application in
Transport Industry} 2:54--57.

\bibitem{4-sh-1}
\Aue{Kudryavtsev, A.\,A., and O.\,V.~Shestakov.} 2011. Аsimptoticheskoe raspredelenie 
otsenki riska porogovoy ob\-ra\-bot\-ki veyglet-koeffitsientov signala pri 
neizvestnom urovne shuma [Asymptotic distribution of the risk estimate of 
the signal vaguelette coefficients thresholding at the unknown noise level]. 
\textit{T-Comm~--- Telecommunications and Their Application in
Transport Industry} 5:24--30.

\bibitem{5-sh-1}
\Aue{Abramovich, F., and B.\,W.~Silverman.} 1998. Wavelet 
decomposition approaches to statistical inverse problems. 
\textit{Biometrika} 85(1):115--129.

\bibitem{6-sh-1}
\Aue{Mallat, S.} 1999. \textit{A~wavelet tour of signal processing.} New York, NY: 
Academic Press. 857 p.

\bibitem{7-sh-1}
\Aue{Lee, N.} 1997. Wavelet-vaguelette decompositions and homogenous equations. 
 West Lafayette, IN: Purdue University. PhD Thesis. 103~p.

\bibitem{8-sh-1}
\Aue{Breiman, L.} 1996. 
Heuristics of instability and stabilization in model selection. 
\textit{Ann. Stat.} 24(6):2350--2383.

\bibitem{9-sh-1}
\Aue{Jansen, M.} 2001. \textit{Noise reduction by wavelet thresholding.} 
Lecture notes in statistics ser.
New York, NY: Springer Verlag.  Vol.~161. 196~p.

\bibitem{10-sh-1}
\Aue{Shestakov, O.\,V.} 2012. O~skorosti skhodimosti otsenki riska porogovoy 
obrabotki veyvlet-koeffitsientov k~nor\-mal'\-no\-mu zakonu pri ispol'zovanii robastnykh 
otsenok dispersii [On the rate of convergence to the normal law of risk estimate for 
wavelet coefficients thresholding when using robust variance estimates]. 
\textit{Informatika i~ee Primeneniya~--- Inform. Appl.}  6(2):122--128.

\bibitem{11-sh-1}
\Aue{Serfling, R.} 1980. \textit{Approximation theorems of mathematical statistics}.
New York, NY: John Wiley \& Sons. 371~p.

\bibitem{12-sh-1}
\Aue{Serfling, R., and S.~Mazumder.} 2009. Exponential probability inequality 
and convergence results for the median absolute deviation and its modifications. 
\textit{Stat. Probabil. Lett.} 79(16):1767--1773.
\end{thebibliography}

 }
 }

\end{multicols}

\vspace*{-6pt}

\hfill{\small\textit{Received December 14, 2018}}

%\pagebreak

%\vspace*{-18pt}  

\Contrl

\noindent
\textbf{Shestakov Oleg V.} (b.\ 1976)~--- 
Doctor of Science in physics and mathematics, professor, Department of 
Mathematical Statistics, Faculty of Computational Mathematics and Cybernetics, 
M.\,V.~Lomonosov Moscow State University, 1-52~Leninskiye Gory, GSP-1, Moscow 119991, 
Russian Federation; senior scientist, Institute of Informatics Problems, 
Federal Research Center ``Computer Science and Control'' 
of the Russian Academy of Sciences, 44-2~Vavilov Str., Moscow 119333, 
Russian Federation; \mbox{oshestakov@cs.msu.su}
\label{end\stat}

\renewcommand{\bibname}{\protect\rm Литература} 
      %11
\def\stat{chupr}

\def\tit{УСИЛЕННЫЕ ЗАКОНЫ БОЛЬШИХ ЧИСЕЛ ДЛЯ~ЧИСЛА БЕЗОШИБОЧНЫХ
БЛОКОВ ПРИ~ПОМЕХОУСТОЙЧИВОМ КОДИРОВАНИИ}

\def\titkol{Усиленные законы больших чисел для числа безошибочных
блоков при помехоустойчивом кодировании}

\def\autkol{А.\,Н.~Чупрунов, И.~Фазекаш}
\def\aut{А.\,Н.~Чупрунов$^1$, И.~Фазекаш$^2$}

\titel{\tit}{\aut}{\autkol}{\titkol}

%{\renewcommand{\thefootnote}{\fnsymbol{footnote}}\footnotetext[1]
%{Работа выполнена при финансовой поддержке РФФИ (грант 11-01-00515).}}

\renewcommand{\thefootnote}{\arabic{footnote}}
\footnotetext[1]{Научно-исследовательский институт 
математики и механики им.\ Н.\,Г.~Чеботарева, achuprunov@mail.ru}
\footnotetext[2]{Дебреценский университет, fazekas.istvan@inf.unideb.hu}

\vspace*{2pt}

\Abst{Рассматриваются  сообщения, состоящие из блоков.
Каждый блок кодируется помехоустойчивым кодом, который может
исправить не более  $r$~ошибок. При этом предполагается, что
количество ошибок в блоке~--- независимая пуассоновская величина с
параметром~$\lambda$.  Кроме того, предполагается, что число ошибок
в сообщении принадлежит некоторому  подмножеству множества
неотрицательных целых чисел. В~работе получены усиленные законы
больших чисел для случайной величины~--- числа безошибочных блоков в
сообщении.}

\vspace*{2pt}

\KW{схема размещения; условная вероятность; закон
больших чисел; код БЧХ}

  \vskip 18pt plus 9pt minus 6pt

      \thispagestyle{headings}

      \begin{multicols}{2}
      
            \label{st\stat}

  

\section{Введение и~основные результаты}

Будем рассматривать код, который позволяет исправить не больше 
$r$~ошибок типа замещения. Частным случаем такого кода является код Боу\-за--Чоуд\-ху\-ри--Хок\-вин\-ге\-ма
(БЧХ) (о кодах  БЧХ см., 
например, в~[1]). Работа посвящена изучению
асимптотического поведения  случайной величины~$S_{nN}$~--- чис\-ла
безошибочных блоков в сообщении, состоящем из $N$~блоков, причем
каждый блок подвергается помехоустойчивому кодированию, а число
ошибок в сообщении принадлежит некоторому конечному
 подмножеству~$M_n$, $n\in\mathbf{N}$, множества неотрицательных целых чисел~$\mathbf{M}$.

Обозначим через~$\pi_{\lambda}$ пуассоновскую случайную величину с
параметром~$\lambda$; $\Phi$~--- функцию распределения стандартной
гауссовской случайной величины; $\stackrel{d}{=}$~--- равенство
распределений случайных величин.
 Будем предполагать, что все рассматриваемые случайные величины определены на
вероятностном пространстве $(\Omega,
\mathfrak{A}, \mathbf{P})$.


 Рассмотрим  сообщение, состоящее из $N$~блоков. Пусть случайная величина $\xi_{Nj}$~--- 
 количество ошибок в $j$-м блоке. Будем предполагать, что  $\xi_{Nj}$, $1\le
j\le N$,~--- независимые пуассоновские случайные величины с
параметром~ $\lambda$.
  Тогда  число безошибочных  блоков в сообщении~--- случайная величина
$$
S_{nN}=\sum\limits_{i=1}^NI_{nNi}\,,
$$
где $I_{nNi}$~--- индикатор события $A_{nNi}$, состоящего в том, что
$i$-й блок  сообщения имеет не более $r$~ошибок. Заметим, что
событие
\begin{multline*} 
A_{nNi}=\{\xi_{Ni}\le r\, |\, \xi_{N1}+\xi_{N2}+ \dots +\xi_{NN}\in
M_n\}={}\\
{}=\cup_{l=0}^r A_{nNil}\,,
\end{multline*}
где   события
$$ 
A_{nNil}=\{\xi_{Ni}=l\, |\,
 \xi_{N1}+\xi_{N2}+ \dots +\xi_{NN}\in M_n\}\,.
$$

Обозначим $n'_n=\sup\{n':\ \, n'\in M_n\}$, $\alpha_{nN}\hm={n'_n}/{N}$.

Если множество $M_n=\{n\}$ состоит из одного элемента, то события~$A_{nNi}$ 
являются событиями теории размещения различимых частиц по
различным ячейкам и не зависят от~$\lambda$~[2]. Подробное изложение
этой теории  можно найти в монографии~[3]. В~[4] получены усиленные
законы больших чисел для событий теории размещения различимых частиц
по различным ячейкам. В~част\-ности, в~[4] доказана

\medskip

\noindent
\textbf{Теорема \emph{A}}. \textit{Пусть $M_n=\{n\}$. Пусть $n,
N\hm\to\infty$ так, что $\alpha_{nN}\hm\to\alpha$, где $0\hm<\alpha\hm<\infty$.
Тогда
$$
\lim\limits_{n, N\to\infty}\fr{1}{N}S_{nN}=e^{-\alpha}\sum\limits_{k=0}^r\fr{\alpha^k}{k!}
$$
почти наверное.}

\pagebreak

%\medskip

Если множества $M_n=\mathbf{M}$, то индикаторы
$I_{nNi}=I_{\{\xi_{Ni}\hm\le r\}}$, $1\hm\le i\hm\le N$, независимы. Поэтому
справедлива

\medskip

\noindent
\textbf{Теорема \emph{Б}}. \textit{Пусть $M_n\hm=\mathbf{M}$.  Тогда
$$
\lim\limits_{N\to\infty}\fr{1}{N}S_{nN}=e^{-\lambda}\sum\limits_{k=0}^r\fr{\lambda^k}{k!}
$$
почти наверное.}

\medskip

Основными результатами статьи являются следующие теоремы.

\medskip

\noindent
\textbf{Теорема 1}. \textit{Предположим,  что $n, N\hm\to\infty$ так, что
$\alpha_{nN}\hm\to\alpha$, где $0\hm<\alpha\hm<\lambda$. Пусть $M_n\hm\subset
\mathbf{M}$~--- такие конечные подмножества, что
\begin{align}
\lim_{n, N\to\infty}\fr{\sum\limits_{k+l\in
M_n}\left({\alpha}/{\lambda}\right)^{n'_n-k-l}}{\sum\limits_{k\in
M_n}\left({\alpha}/{\lambda}\right)^{n'_n-k}}&=B_l \,;
\label{e1-chu}\\
\lim_{n, N\to\infty}\fr{\sum\limits_{k+2l\in
M_n}\left({\alpha}/{\lambda}\right)^{n'_n-k-2l}}{\sum\limits_{k\in
M_n}\left({\alpha}/{\lambda}\right)^{n'_n-k}}&=(B_l)^2\,,
\label{e2-chu}
\end{align}
где $B_l<\infty$, $1\hm\le l\hm\le r$. Положим $B_0\hm=1$.  Тогда
\begin{equation}
\lim_{n, N\to\infty}\fr{1}{N}S_{nN}=e^{-\alpha}\sum\limits_{k=0}^r\fr{\alpha^k}{k!}B_k
\label{e3-chu}
\end{equation}
в $L^2(\Omega, \mathfrak{A}, \mathbf{P})$.}


\medskip

\noindent
\textbf{Следствие}. \textit{Пусть $M_n$~--- такие конечные подмножества~$\mathbf{M}$ , 
что $M_n\hm\subset M_{n+1}$, $n\hm\in \mathbf{N}$, и
$\cup_{n=1}^{\infty}M_n \hm=\mathbf{M}$. Предположим, что $n, N\hm\to\infty$
так, что $\alpha_{nN}\hm\to\alpha$, где $0\hm<\alpha\hm<\lambda$.  Тогда
$$
\lim_{n,N\to\infty}\fr{1}{N}S_{nN}=e^{-\alpha}\sum\limits_{k=0}^r\fr{\alpha^k}{k!}
$$
в $L^2(\Omega, \mathfrak{A}, \mathbf{P})$.}


\medskip

\noindent
\textbf{Теорема 2}. \textit{Пусть множества  $M_n=\{0, 1,\dots, n\}$,
$n\in\mathbf{N}$, $\alpha >\lambda $.  Тогда
\begin{equation*}
\lim_{n, N\to\infty,
\lambda\le\alpha_{nN}\le\alpha}\fr{1}{N}S_{nN}=e^{-\lambda}\sum\limits_{k=0}^r\fr{\lambda^k}{k!}
%\label{e4-chu}
\end{equation*}
почти наверное.}

\medskip

Из теорем~1 и~2 вытекает следующая теорема. В~ней показано что,
если  множества $M_n\hm=\{0, 1,\dots, n\}$, то для <<маленьких>>~ $\alpha_{nN}$ 
справедлив аналог теоремы~${\it A}$, а для <<больших>>~$\alpha_{nN}$~--- 
аналог теоремы~${\it Б}$.

\medskip

\noindent
\textbf{Теорема 3}. \textit{Пусть $M_n=\{0, 1,\dots, n\}$, $n\in\mathbf{N}$.  
Предположим, что $n, N\to\infty$ так, что
$\alpha_{nN}\hm\to\alpha$, где $0\hm<\alpha<\infty$.}


(A)~\textit{Пусть $\alpha<\lambda$.  Тогда
$$
\lim_{n,
N\to\infty}\fr{1}{N}S_{nN}=e^{-\alpha}\sum_{k=0}^r\fr{\alpha^k}{k!}
$$
по вероятности.}

(B)~\textit{Пусть $\alpha > \lambda $.  Тогда
$$
\lim_{n,
N\to\infty}\fr{1}{N}S_{nN}=e^{-\lambda}\sum\limits_{k=0}^r\fr{\lambda^k}{k!}
$$
почти наверное. }


\medskip

\noindent
\textbf{Замечание 1.} Заметим, что $\left|({1}/{N})S_{nN}\right|\hm\le
1$. Поэтому по теореме Лебега в условиях теоремы~1
$$
\lim_{n,N\to\infty}\fr{1}{N}S_{nN}=e^{-\alpha}\sum\limits_{k=0}^r\fr{\alpha^k}{k!}B_k
$$
в пространстве $L^p(\Omega, \mathfrak{A}, \mathbf{P})$ для любого
$0\hm<p\hm<\infty$, а в условиях теоремы~2
$$
\lim_{n,N\to\infty}\fr{1}{N}S_{nN}=e^{-\lambda}\sum\limits_{k=0}^r\fr{\lambda^k}{k!}
$$
в пространстве $L^p(\Omega, \mathfrak{A}, \mathbf{P})$ для любого
$0\hm<p<\infty$.


%\bigskip

\section{ Доказательства}

%\bigskip

Для доказательства теоремы~1  потребуется следующая лемма.

\medskip

\noindent
\textbf{Лемма 1}. \textit{Пусть $0<\lambda_1<1$, $M$~--- конечное
подмножество~$\mathbf{M}$. Обозначим  $n'=\sup\{n': n'\in M\}$. Тогда
\begin{equation}
\fr{\mathbf{P}\{\pi_{\lambda}\in M\}}{\mathbf{P}\{\pi_{\lambda}=n'\}}=o(1)+\sum\limits_{k\in
M}\left(\fr{n'}{\lambda}\right)^{n'-k} \label{e5-chu}
\end{equation}
 равномерно при $n', \lambda\to\infty$ так, что
 ${n'}/{\lambda}\le\lambda_1$.}

\medskip


\noindent
Д\,о\,к\,а\,з\,а\,т\,е\,л\,ь\,с\,т\,в\,о\,.\ Имеем
\begin{multline*}
\mathbf{P}\{\pi_{\lambda}\in M\}={\mathbf{P}\{\pi_{\lambda}=n'\}}\sum\limits_{k\in
M}\fr{\lambda^kn'!}{k!\lambda^{n'}}={}\\
{}=\mathbf{P}\{\pi_{\lambda}=n'\}\sum\limits_{k\in
M}\fr{n'(n'-1)\cdots(n'-k+1)}{\lambda^{n'-k}}\,.
\end{multline*}
Так как
\begin{multline*}
\fr{n'(n'-1)\cdots
(n'-k+1)}{\lambda^{n'-k}}-\left(\fr{n'}{\lambda}\right)^{n'-k}\ge{}\\
{}\ge
-\frac{k^2}{2\lambda}\left(\fr{n'}{\lambda}\right)^{n'-k-1}\,,\enskip k<n'\,,
\end{multline*}
то
\begin{multline*}
\sum\limits_{k\in M}\left(\fr{n'}{\lambda}\right)^{n'-k}\ge\fr{\mathbf{P}
(\pi_{\lambda}\in M)}{\mathbf{P}(\pi_{\lambda}=n')}\ge\sum\limits_{k\in
M}\left(\fr{n'}{\lambda}\right)^{n'-k}-{}\\
{}-\fr{(k_0)^2}{2\lambda}\sum\limits_{k=0}^{k_0}(\lambda_1)^k
-\sum\limits_{k=k_0+1}^{\infty}(\lambda_1)^k\,,\enskip k_0<n'\,.
\end{multline*}
Это неравенство и доказывает  лемму~1.

\medskip


\noindent
\textbf{Следствие 1}. \textit{Пусть $0<\lambda_1<1$,  $M_n\subset \mathbf{M}$~-- 
такие конечные подмножества, что $M_n\subset M_{n+1}$, $n\in \mathbf{N}$, 
и $\cup_{n=1}^{\infty}M_n =\mathbf{M}$.  Тогда
\begin{equation}
\fr{\mathbf{P}\{\pi_{\lambda}\in M_n\}}{\mathbf{P}\{\pi_{\lambda}=n'_n\}}=
\fr{1}{1-{n'_n}/{\lambda}}+o(1)
\label{e6-chu}
\end{equation}
 равномерно при $n, \lambda\to\infty$ так, что
 ${n'}/{\lambda}\le\lambda_1$.}

\medskip

\noindent
Д\,о\,к\,а\,з\,а\,т\,е\,л\,ь\,с\,т\,в\,о\,.\ Так как $I_{M_n}\to I_{\mathbf{M}}$ при
 $n\to\infty$ поточечно, то 
\begin{multline*}
 \sum\limits_{k\in M_n}\left(\fr{n'_n}{\lambda}\right)^{n'-k}=\sum\limits_{k\in \mathbf{M}}
 \left(\fr{n'_n}{\lambda}\right)^{k}+
o(1)={}\\
{}=\fr{1}{1-{n'_n}/{\lambda}}+o(1)\,.
\end{multline*} 
Применяя эту оценку к
левой части~(\ref{e5-chu}), получаем~(\ref{e6-chu}). Следствие доказано.

\medskip

При $M_n=\{0, 1,\dots n\}$ из следствия 1 вытекает

\medskip

\noindent
\textbf{Следствие 2}. \textit{Пусть $0<\lambda_1<1$. Тогда
$$
\fr{\mathbf{P}\{\pi_{\lambda}\le n\}}{\mathbf{P}\{\pi_{\lambda}=n\}}=\fr{1}{1-{n}/{\lambda}}+
o(1)
$$
 равномерно при $n, \lambda\hm\to\infty$ так, что
 ${n}/{\lambda}\hm\le\lambda_1$.}

\medskip

\noindent
Д\,о\,к\,а\,з\,а\,т\,е\,л\,ь\,с\,т\,в\,о\ теоремы~1.
Пусть $0\hm\le l\hm\le r$.\linebreak
Так как $\xi_{N2}\hm+\xi_{N3}\hm+ \dots +\xi_{NN}$~---
пуассоновская случайная величина с параметром $(N-1)\lambda$, а
$\xi_{N1}\hm+\xi_{N2}\hm+ \dots \hm+\xi_{NN}$~--- пуассоновская случайная
величина с параметром~$N\lambda$, в силу леммы~1 имеем
\begin{multline*}
\mathbf{E}\fr{1}{N}S_{nN}=\mathbf{E} I_{A_{nNil}}=\mathbf{P}\{A_{nNil}\} =
\mathbf{P}\{\xi_{N1}=l\}\times{}\\[2pt]
{}\times \fr{\mathbf{P}\{
 \xi_{N2}+\xi_{N3}+ \dots +\xi_{NN}\in M_n-l\}}{\mathbf{P}\{
 \xi_{N1}+\xi_{N2}+ \dots +\xi_{NN}\in M_n\}}={}\\[2pt]
  {}=
e^{-\lambda}\fr{\lambda^l}{l!}\fr{\mathbf{P}\{\pi_{(N-1)\lambda}\in
M_n-l\}}{\mathbf{P}\{\pi_{N\lambda}\in M_n\}} ={}\\[2pt]
{}=e^{-\lambda}\fr{\lambda^l}{l!}\fr{\mathbf{P}\{\pi_{(N-1)\lambda}=
n'-l\}}{\mathbf{P}\{\pi_{N\lambda}=n'\}}\times{}
 \end{multline*}
 \begin{multline*}
\times\fr {o(1)+\sum\limits_{k\in
M_n-l}\left({n'_n}/{((N-1)\lambda)}\right)^{n'_n-k-l}}{
o(1)+\sum\limits_{k\in M_n}\left({n'_n}/{(N\lambda)}\right)^{n'_n-k}} ={}\\[1pt]
{}=\fr{n'_n!}{l!((n'_n-l)!}\left(\fr{N-1}{N}\right)^{n'_n-l}\left(\fr{1}{N}\right)^{l}\times{}\\[1pt]
{}\times
\fr {o(1)+\sum\limits_{k\in M_n-l}\left({n'_n}/{((N-1)\lambda)}\right)^{n'_n-k-l}}{
o(1)+\sum\limits_{k\in M_n}\left({n'_n}/{(N\lambda)}\right)^{n'_n-k}}\,;
\end{multline*}

\vspace*{-6pt}

\noindent
\begin{multline*}
\mathbf{E} I_{A_{nNil}}I_{A_{nNjl}}={}\\[1pt]
{}=\mathbf{P}(A_{nNil}\cap A_{nNjl}) =
\left(\mathbf{P}\{\xi_{N1}=l\}\right)^2\times{}\\[1pt]
{}\times \fr{\mathbf{P}\{
 \xi_{N3}+\xi_{N4}+ \dots +\xi_{NN}\in M_n-2l\}}{\mathbf{P}\{
 \xi_{N1}+\xi_{N2}+ \dots +\xi_{NN}\in M_n\}}={}\\[1pt]
{}=
\left(e^{-\lambda}\fr{\lambda^l}{l!}\right)^2\fr{\mathbf{P}\{
\pi_{(N-2)\lambda}\in M_n-2l\}}{\mathbf{P}\{\pi_{N\lambda}\in M_n\}} ={}\\[1pt]
{}=
\left(e^{-\lambda}\fr{\lambda^l}{l!}\right)^2\fr{\mathbf{P}\{
\pi_{(N-2)\lambda}=n'-2l\}}{\mathbf{P}\{\pi_{N\lambda}=n'_n\}}\times{}\\[1pt]
{}\times
\fr {o(1)+\sum\limits_{k\in M_n-2l}\left({n'_n}/{((N-1)\lambda)}\right)^{n'_n-k-2l}}{
o(1)+\sum\limits_{k\in M_n}\left({n'_n}/{(N\lambda)}\right)^{n'_n-k}} ={}\\[1pt]
{}
=\fr{n'_n!}{(l!)^2((n'_n-2l)!}\left(\fr{N-2}{N}\right)^{n'_n-2l}\left(\fr{1}{N}\right)^{2l}\times{}\\[1pt]
{}\times
\fr{o(1)+\sum\limits_{k\in M_n-2l}\left({n'_n}/{((N-2)\lambda)}\right)^{n'_n-k-2l}}{
o(1)+\sum\limits_{k\in M_n}\left({n'_n}/{(N\lambda)}\right)^{n'_n-k}}\,,\enskip i\ne j.
\end{multline*}
Следовательно,
\begin{align}
\mathbf{P}\{A_{nNil}\} &\to e^{-\alpha}\fr{\alpha^l}{l!}B_l \,;
\label{e7-chu}\\
\mathbf{P}(A_{nNil}\cap
A_{nNjl})&\to\left(e^{-\alpha}\fr{\alpha^l}{l!}B_l\right)^2\,,\enskip i\ne j\,, 
\label{e8-chu}
\end{align}
при $n, N\to\infty$ так, что $\alpha_{nN}\hm\to\alpha$. Заметим, что
\begin{multline*}
\mathbf{E}\left(\fr{1}{N}S_{nN}-\mathbf{E}\fr{1}{N}S_{nN}\right)^2={}\\
{}=
\fr{\mathrm{P}\{A_{nN1l}\}}{N}+\mathbf{P}\{A_{nN1l}\cap A_{nN2l}\}-{}\\
{}-\fr{\mathbf{P}\{A_{nN1l}\cap
A_{nN2l}\}}{N}-\left(\mathbf{P}\{A_{nN1l}\}\right)^2\,.
\end{multline*}
Поэтому из~(\ref{e7-chu}) и~(\ref{e8-chu}) следует, что
\begin{equation}
\mathbf{E}\left(\fr{1}{N}S_{nN}-\mathbf{E}\fr{1}{N}S_{nN}\right)^2\to 0
\label{e9-chu}
\end{equation}
при $n, N\to\infty$ так, что $\alpha_{nN}\to\alpha$. Так как 
$$
\mathbf{E} \fr{1}{N}S_{nN}=\mathbf{P}\{A_{nNil}\}\,,
$$
условия~(\ref{e9-chu}) и~(\ref{e7-chu}) влекут~(\ref{e3-chu}). 
Теорема доказана.

\medskip

\noindent
Д\,о\,к\,а\,з\,а\,т\,е\,л\,ь\,с\,т\,в\,о\ следствия теоремы~1. Пусть $0\hm\le l\hm\le r$. По
следствию~1 леммы~1
\begin{align*}
\lim_{\substack{{n, N\to\infty}\\ {\alpha_{nN}\to\alpha}}}\fr{\sum\limits_{k+l\in
M_n}\left({\alpha}/{\lambda}\right)^{n'_n-k-l}}{\sum\limits_{k\in
M_n}\left({\alpha}/{\lambda}\right)^{n'_n-k}}&=
\fr{1-{\alpha}/{\lambda}}{1-{\alpha}/{\lambda}}=1\,;
\\
\lim_{\substack{{n, N\to\infty}\\ {\alpha_{nN}\to\alpha}}}\fr{\sum\limits_{k+2l\in
M_n}\left({\alpha}/{\lambda}\right)^{n'_n-k-2l}}{\sum\limits_{k\in
M_n}\left({\alpha}/{\lambda}\right)^{n'_n-k}}&=\fr{1-{\alpha}/{\lambda}}
{1-{\alpha}/{\lambda}}=1\,.
\end{align*}
Поэтому условия~(\ref{e1-chu}) и~(\ref{e2-chu}) выполнены. Следовательно, применима
теорема~1. Доказательство закончено.

\medskip

При доказательстве теоремы~2 будем использовать следующие леммы.

\medskip

\noindent
\textbf{Лемма 2}. (A)~\textit{Справедлива оценка
\begin{multline}
\!\!\!\mathbf{P}\{\pi_{\lambda}>n\}\le\left(\!1+O\left(\fr{1}{n}\right)\!\right)e^{-n\sum_{k=3}^{\infty}
({1}/{k})\left(1-{\lambda}/{n}\right)^k}\times{}\\
{}\times
\left(\Phi(\sqrt{n})-\Phi\left(\sqrt{n}\left(1-\fr{\lambda}{n}\right)\right)\right)
\label{e10-chu}
\end{multline}
при   ${\lambda}/{n}\le 1$.}

\smallskip

(Б)~\textit{Справедлива оценка
\begin{multline}
0\le \mathbf{P}\{\pi_{N\lambda}>n\}-\mathbf{P}\{\pi_{(N-l)\lambda}>n\}\le{}\\
{}\le
\left(1+O\left(
\fr{1}{n}\right)\right)e^{-n\sum\limits_{k=2}^{\infty}(1/k)\left(1-{N\lambda}/{n}\right)^k}
\fr{\lambda l}{\sqrt{2\pi n}} 
\label{e11-chu}
\end{multline}
при   ${N\lambda}/{n}\le 1$.}

\medskip

\noindent
Д\,о\,к\,а\,з\,а\,т\,е\,л\,ь\,с\,т\,в\,о\,.\ (A)~В~силу формулы Тейлора с остаточным
членом в интегральной форме (см., например,~[5, c.~161]) для функции
$f(x)=e^x$ с последующей заменой  $t\hm=\lambda-y$  получаем
пред\-став\-ле\-ние

\noindent
\begin{multline}
\mathbf{P}\{\pi_{\lambda}>n\}=e^{-\lambda}\fr{1}{n!}
\int\limits_0^{\lambda}e^y(\lambda-y)^n\,dy={}\\
{}=
\fr{1}{n!}\int\limits_0^{\lambda}e^{-(\lambda-y)}(\lambda-y)^n\,dy=
\fr{1}{n!}\int\limits_0^{\lambda}e^{-t}t^n\,dt.\label{e12-chu}
\end{multline}
 Поэтому, используя замену
$t\hm=ny$, а затем замену $x\hm=\sqrt{n}(1-y)$, формулу Стирлинга и
разложение логарифма в ряд Тейлора при оценке правой час\-ти~(\ref{e12-chu}),
получаем

\noindent
\begin{multline*}
\mathbf{P}\{\pi_{\lambda}>n\}=\fr{1}{n!}\int\limits_0^{\lambda}e^{-t}t^n\,dt={}\\
{}=
\left(1+O\left(\fr{1}{n}\right)\right)
\fr{e^n}{\sqrt{2\pi n}n^n}\int\limits_0^{{\lambda}/{n}}e^{-ny}(ny)^n\,d(ny)={}\\
{}=
\left(1+O\left(\fr{1}{n}\right)\right) \sqrt{\fr{n}{2\pi}}
\int_0^{{\lambda}/{n}}e^{n(1-y)}y^n\,dy={}\\
{}=\left(1+O\left(\fr{1}{n}\right)\right)
\sqrt{\fr{n}{2\pi}}
\int\limits_0^{{\lambda}/{n}}e^{n(1-y)}e^{n\ln(y)}\,dy={}\\
{}=
\left(1+O\left(\fr{1}{n}\right)\right) \sqrt{\fr{n}{2\pi}}\times{}\\
{}\times
\int_0^{{\lambda}/{n}}e^{n(1-y)}e^{n\ln(1-(1-y))}\,dy={}\\
{}=
\left(1+O\left(\fr{1}{n}\right)\right) \sqrt{\fr{n}{2\pi}}\times{}\\
{}\times
\int\limits_0^{{\lambda}/{n}}e^{-n{(1-y)^2}/2}
e^{-n\sum\limits_{k=3}^{\infty}{(1-y)^k}/{k}}\,dy
\le {}\\
{}\le-\left(1+O\left(\fr{1}{n}\right)\right)
\fr{1}{\sqrt{2\pi}}e^{-n\sum\limits_{k=3}^{\infty}(1/k)\left(1-{\lambda}/{n}\right)^k}\times{}\\
{}\times
\int\limits_{\sqrt{n}}^{\sqrt{n}\left(1-{\lambda}/{n}\right)}e^{-{x^2}/{2}}\,dx={}\\
{}=
\left(1+O\left(\fr{1}{n}\right)\right)e^{-n\sum\limits_{k=3}^{\infty}({1}/{k})\left(1-{\lambda}/{n}
\right)^k}\times{}\\
{}\times
\left(\Phi\left(\sqrt{n}\right)-\Phi\left(\sqrt{n}\left(1-\fr{\lambda}{n}\right)\right)\right)\,.
\end{multline*}
Поэтому справедливо~(\ref{e10-chu}). Пункт~(A) доказан.

Доказательство п.~(Б) повторяет доказательство п.~(А) с
той лишь разницей, что вместо оценивания интеграла
$\int\limits_0^{\lambda}e^{-t}t^n\,dt$  оценивается интеграл
$\int\limits_{(N-1)\lambda}^{N\lambda}e^{-t}t^n\,dt$.

В частности, при $\lambda=n$ получаем

\smallskip

\noindent
\textbf{Следствие 1}. \textit{Справедливо неравенство 
$$
\lim_{n\to\infty}{\bf P}\{\pi_{n}>n\}\le\fr{1}{2}\,.
$$}


\medskip

При доказательстве теоремы~2  будем использовать следующее обобщение
следствия~1:

\smallskip

\noindent
\textbf{Следствие 2}. \textit{Справедливо неравенство 
$$
\limsup\limits_{\lambda,n\to\infty, {\lambda}/{n}\le 1}\mathbf{P}\{
\pi_{\lambda}>n\}\le\fr{1}{2}\,.
$$
}

\smallskip


\noindent
\textbf{Замечание~2.} В~работах В.\,М.~Круглова~\cite{7-chu, 8-chu} получены оценки
хвостов безгранично делимых и пуассоновских распределений. Оценки,
полученные в лемме~2, можно рассматривать как уточнение оценок,
полученных Кругловым для пуассоновских распределений. Приведем
некоторые аналоги неравенств~(\ref{e10-chu}) и~(\ref{e11-chu}).

(В) Используя элементарные неравенства
\begin{align*}
-\sum\limits_{k=3}^{\infty}\fr{1}{k}\,(1-x)^k&\le \fr{1}{3}\,(1-x)^2\ln(x)\,;\\
-\sum\limits_{k=2}^{\infty}\fr{1}{k}\,(1-x)^k&\le \fr{1}{2}\,(1-x)\ln(x)
\end{align*}
при оценке правых частей в~(\ref{e10-chu}) и~(\ref{e11-chu}) соответственно, получаем
\begin{multline*}
\!\!\!\mathbf{P}\{\pi_{\lambda}>n\}\le\left(1+O\left(\fr{1}{n}\right)\right)
e^{({1}/{3})n\left(1-{\lambda}/{n}\right)^2\ln\left(\lambda/n\right)}\times{}\\
{}\times
\left(\Phi(\sqrt{n})-\Phi\left(\sqrt{n}\left(1-\fr{\lambda}{n}\right)\right)\right)
\end{multline*}
при   ${\lambda}/{n}\le 1$;
\begin{multline*}
0\le \mathbf{P}\{\pi_{N\lambda}>n\}-\mathbf{P}\{\pi_{(N-l)\lambda}>n\}\le{}\\
{}\le
\left(1+O\left(\fr{1}{n}\right)\right)e^{({1}/{2})n\left(1-{N\lambda}/{n}\right)
\ln\left({N\lambda}/{n}\right)}
\fr{\lambda l}{\sqrt{2\pi n}}
\end{multline*}
при   ${N\lambda}/{n}\le 1$.

(Г)~Функция $y\hm=e^{-t}t^n$ возрастает на интервале $(0, n)$ и
убывает на интервале  $(n, \infty)$. Заметим, что
$\lambda=n({\lambda}/{n})\le n$ при ${\lambda}/{n}\le 1$.
Поэтому, используя формулу Стирлинга и элементарное неравенство $e x
e^{-x}\le e^{-{(1-x)^2}/{2}}$, $0\hm\le x\hm\le 1$, при
$x\hm={\lambda}/{n}$ по\-лу\-чаем
\begin{multline*}
\mathbf{P}\{\pi_{\lambda}>n\}=\fr{1}{n!}\int\limits_0^{\lambda}e^{-t}t^n\le{}\\
{}\le\fr{1}{\sqrt{2\pi
n}}\,\fr{e^n}{n^n}\,e^{-\lambda}\lambda^n\lambda\left(1+O\left(\fr{1}{n}\right)\right)={}\\
{}=\fr{1}{\sqrt{2\pi}}\,\sqrt{n}\,\fr{\lambda}{n}\left(e\fr{\lambda}{n}e^{-{\lambda}/{n}}\right)^n
\left(1+O\left(\fr{1}{n}\right)\right)
\le{}\\
{}\le \fr{1}{\sqrt{2\pi}}\, \sqrt{n}\,\fr{\lambda}{n}e^{-({n}/{2})\left(1-{\lambda}/{n}\right)^2}
\!\left(\!1+O\left(\fr{1}{n}\right)\right),\
\fr{\lambda}{n}\le 1.\hspace*{-4.80296pt}
\end{multline*}


\smallskip

\noindent
\textbf{Замечание 3.} Пусть случайные величины $\pi_{ni}$, $1\hm\le i\hm\le
n$, $n\in\mathbf{N}$,~--- независимые копии случайной величины~$\pi_{\lambda}$. Так как
\begin{equation}
\fr{1}{n}\sum_{i=1}^n\pi_{ni}\stackrel{d}{=}\fr{\pi_{n\lambda}}{n}\,,
\label{e13-chu}
\end{equation}
то, используя представление четвертого момента в виде четвертой
производной от характеристической функции, получаем
\begin{multline*}
\mathbf{E}\left|\fr{1}{n}\sum_{i=1}^n\pi_{ni}-\lambda\right|^4={}\\
{}=
\left.\left(e^{-n\lambda\left(1+i({t}/{n})-e^{i({t}/{n})}\right)}\right)^{(4)}
\right\vert_{t=0}=\fr{3n^2\lambda^2+n\lambda}{n^4}\,.
\end{multline*}
Поэтому, используя неравенство чебышевского типа для четвертых
моментов, приходим к
\begin{multline*}
\sum\limits_{n=1}^{\infty}\mathbf{P}\left\{\left|
\fr{1}{n}\sum_{i=1}^n\pi_{ni}-\lambda\right|>\varepsilon\right\}\le{}\\
{}\le
\fr{1}{\varepsilon^4}\sum\limits_{n=1}^{\infty}\mathbf{E}
\left\vert\fr{1}{n}\sum\limits_{i=1}^n\pi_{ni}-\lambda\right\vert^4<\infty
\,\,\,\forall\,\,\, \varepsilon>0\,.
\end{multline*}
Следовательно,
$$
\fr{1}{n}\sum\limits_{i=1}^n\pi_{ni}\to\lambda\,,\quad
n\to\infty\,,
$$
почти наверное. Равенство~(\ref{e13-chu}) позволяет использовать  леммы~1 и~2
для оценки  скорости сходимости в этом усиленном законе больших
чисел. Пусть случайная величина $X_{\lambda}\hm\equiv\lambda$. При
$k\in\mathbf{N}$, $k\hm\ne \lambda$, имеем
\begin{multline}
\Bigg|\mathbf{P}\left\{\frac{1}{n}\sum_{i=1}^n\pi_{ni}\le k\Bigg\}-\mathbf{P}
\Bigg\{X_{\lambda}\le k\Bigg\} \right|\le{}\\
\le
\begin{cases}
\fr{1}{\sqrt{2\pi nk}}\left(e\fr{\lambda}{k}\,e^{-{\lambda}/{k}}\right)^{nk}
\left(1-\fr{k}{\lambda}\right)^{-1}\times{}\\
\hspace{10mm}{}\times(1+o(1)) \quad \mbox{\ при\ } 0<k<\lambda\,;\\
\fr{1}{\sqrt{2\pi}} \sqrt{nk}\,\fr{\lambda}{k}\left(e\fr{\lambda}{k}\,e^{-{\lambda}/{k}}\right)^{nk}\times{}\\
\hspace*{7mm}{}\times \left(1+O\left(\fr{1}{nk}\right)\right)\quad
\mbox{\ при\ } k>\lambda\,.
\end{cases}
\label{e14-chu}
\end{multline}
Оценка~(\ref{e14-chu}) при $0\hm<k\hm<\lambda$ вытекает из следствия~2 леммы~1, в
котором вместо~$\lambda$ используется $n\lambda$, а вместо~$n$
используется $kn$, с последующей оценкой $(kn)!$ с помощью формулы
Стирлинга. Оценка~(\ref{e14-chu}) при $\lambda\hm<k$ вытекает из замечания~2(Б). 
Заметим, что
$e({\lambda}/{k})e^{-{\lambda}/{k}}\hm<1$, $k\hm\ne\lambda$. Поэтому
правая часть в~(\ref{e14-chu}) стремится к нулю при $n\to\infty$ при любом
$k\hm\ne\lambda$.

\medskip

\noindent
\textbf{Лемма 3}. \textit{Пусть множества $M_n=\{0, 1,\dots, n\}$,
$n\hm\in\mathbf{N}$, $\alpha \hm >\lambda $,
$p_r\hm=e^{-\lambda}\sum\limits_{k=0}^r({\lambda^k}/{k!}) $.  Тогда
$$
\limsup\limits_{n, N\to\infty,
\lambda\le\alpha_{nN}\le\alpha}\fr{|S_{nN}-
\mathbf{E}S_{nN}|}{\sqrt{N\ln(N)}}\le 16\sqrt{2}\,p_r(1-p_r)
$$
почти наверное.}

\medskip

\noindent
Д\,о\,к\,а\,з\,а\,т\,е\,л\,ь\,с\,т\,в\,о\,.\ Так как $\xi_{N1}\hm+\xi_{N2}\hm+ \dots +\xi_{NN}$~--- 
пуассоновская случайная величина с параметром $N\lambda$  и
${N\lambda}/{n}\hm\le 1 $ при $\lambda\hm\le\alpha_{nN}\hm\le\alpha$, по
следствию~2 леммы~2
$$
\liminf_{n, N\to\infty,
\lambda\le\alpha_{nN}\le\alpha} \mathbf{P}\{
 \xi_{N1}+\xi_{N2}+ \dots +\xi_{NN}\in M_n\}\ge\fr{1}{2}\,.
$$
Поэтому доказательство леммы~3 повторяет доказательство теоремы~2 из~\cite{6-chu}.

\medskip

\noindent
Д\,о\,к\,а\,з\,а\,т\,е\,л\,ь\,с\,т\,в\,о\ теоремы~2. Согласно лемме~3, 
при $n, N\to\infty$ так, что
$\lambda\hm\le\alpha_{nN}\hm\le \alpha$,
 последовательность $({S_{nN}-\mathbf{E}S_{nN}})/{\sqrt{N\ln(N)}}$
ограничена почти наверное. Поэтому $({S_{nN}-\mathbf{E}S_{nN}})/N\hm\to 0$ 
при $n, N\hm\to\infty$ так, что
$\lambda\hm\le\alpha_{nN}\hm\le\alpha$ почти наверное. Так как
\begin{align*}
\fr{1}{N}\,S_{nN}&=\fr{S_{nN}-\mathbf{E}S_{nN}}{N}+\fr{1}{N}\,\mathbf{E}S_{nN}\,;\\
\fr{1}{N}\,\mathbf{E}S_{nN}&=\mathbf{P}(A_{nNi})\,,
\end{align*}
 то для завершения
доказательства теоремы достаточно показать, что
\begin{equation}
\lim_{n, N\to\infty,
\lambda\le\alpha_{nN}\le\alpha}\mathbf{P}(A_{nNi})=
e^{-\lambda}\sum_{k=0}^r\fr{\lambda^k}{k!}\,. \label{e15-chu}
\end{equation}

Пусть $0\le l\le r$. Заметим, что
\begin{multline}
\mathbf{P}\{A_{nNil}\} ={}\\
{}=\fr{\mathbf{P}\{\xi_{N1}=l\}\mathbf{P}\{
 \xi_{N2}+\xi_{N3}+ \dots +\xi_{NN}\le n-l\}}{\mathbf{P}\{
 \xi_{N1}+\xi_{N2}+ \dots +\xi_{NN}\le n\}}={}\\
 {}=
e^{-\lambda}\fr{\lambda^l}{l!}\frac{1-\mathbf{P}
\{\pi_{(N-l)\lambda}>n\}}{1-\mathbf{P}\{\pi_{N\lambda}>n\}}={}\\
{}= 
e^{-\lambda}\fr{\lambda^l}{l!}\left(1+\fr{\mathbf{P}
\{\pi_{N\lambda}>n\}-\mathbf{P}\{\pi_{(N-1)\lambda}>n\}}{1-\mathbf{P}
\{\pi_{N\lambda}>n\}}+{}\right.\\
\left.{}+\fr{\mathbf{P}\{\pi_{(N-1)\lambda}>n\}-\mathbf{P}
\{\pi_{(N-1)\lambda}>n-l\}}{1-\mathbf{P}\{\pi_{N\lambda}>n\}}\right)\,.
\label{e16-chu}
\end{multline}
Используя формулу   Стирлинга и элементарное неравенство
$exe^{-x}\le 1$, $0\le x<\infty$, получаем
\begin{multline*}
0\le \mathbf{P}\{\pi_{(N-1)\lambda}>n-l\}-\mathbf{P}\{
\pi_{(N-1)\lambda}>n\}\le{}\\
{}\le
 e^{-\lambda} l\left(\fr{\alpha}{\lambda}\right)^l\mathbf{P}\{\pi_{N\lambda}=n\}={}\hspace*{10mm}
\end{multline*}
 
\noindent
\begin{multline}
{}=
\left(1+O\left(\frac{1}{n}\right)\right) e^{-\lambda}
l\left(\fr{\alpha}{\lambda}\right)^l\left(e
e^{-{N\lambda}/{n}}\fr{N\lambda}{n}\right)^n\times{}
\\
{}\times
\fr{1}{\sqrt{2\pi n}}\le
\left(1+O\left(\fr{1}{n}\right)\right)
e^{-\lambda} l\left(\fr{\alpha}{\lambda}\right)^l
\fr{1}{\sqrt{2\pi n}}\,. \label{e17-chu}
\end{multline}
Из~(\ref{e16-chu}) и~(\ref{e17-chu})
  по  лемме~2 и ее следствию~2 вытекает, что  
  $ \mathbf{P}\{A_{nNil}\}\to e^{-\lambda} ({\lambda^l}/{l!})$ при $n,
N\hm\to\infty$ так, что $\lambda\le\alpha_{nN}\hm\le\alpha$. Условие~(\ref{e15-chu})
выполнено. Это завершает доказательство теоремы.

\medskip

\noindent
Д\,о\,к\,а\,з\,а\,т\,е\,л\,ь\,с\,т\,в\,о\  теоремы~3. (А)~Так как из сходимости в
среднем квадратичном следует сходимость по вероятности и множества
$M_n\hm=\{0, 1,\dots, n\}$, $n\in\mathbf{N}$, удовлетворяют условиям
следствия теоремы~1, то п.~(А) теоремы~3 вытекает из следствия
теоремы~1.

(Б) Пусть $0<\varepsilon\hm<\alpha\hm-\lambda$. Можно считать, что
найдется $n_0\hm\in\mathbf{N}$ со следующим свойством:
$|\alpha_{nN}\hm-\alpha|<\varepsilon$ при $N,n>n_0$. Тогда  при
$N,n\hm>n_0$ справедливо неравенство
$\lambda\hm\le\alpha_{nN}\hm\le\alpha\hm+\varepsilon$. Следовательно,
применима теорема~2. Доказательство п.~(Б) закончено.

\bigskip

Авторы благодарят профессора В.\,М.~Круглова за ценную информацию и
профессора В.\,Ф.~Колчина за ценное замечание.

  {\small\frenchspacing
{%\baselineskip=10.8pt
\addcontentsline{toc}{section}{Литература}
\begin{thebibliography}{9}

\bibitem{1-chu}
\Au{Питерсон У., Уэлдон Э.} Коды, исправляющие ошибки.~--- М.:
Мир, 1976. 596~с.

\bibitem{2-chu}
\Au{Колчин В.\,Ф. } Один класс предельных теорем для условных
распределений~// Литовский математический сборник, 1968. T.~8. №\,1. C.~53--63.


\bibitem{3-chu}
\Au{Колчин В.\,Ф., Севастьянов Б.\,А., Чистяков~В.\,П.} Случайные
размещения.~--- М.: Физматгиз, 1976.  223~с.

\bibitem{4-chu}
\Au{Chuprunov A.\,N., Fazekas~I.}  Inequality and strong law of
large numbers for random allocations~// Acta Math. Hungar., 2005.
Vol.~109. No.\,1--2. P.~163--182.

\bibitem{5-chu}
\Au{Фихтенгольц Г.\,М.} Курс дифференциального и интегрального
исчисления.~---  М.: Физматлит, 2006.  864~с.


\bibitem{7-chu} %6
\Au{Круглов В.\,М.}  Характеризация одного класса безгранично
делимых распределений~// Матем. заметки, 1974. T.~16. №\,5. C.~777--782.


\bibitem{8-chu} %7
\Au{Круглов В.\,М.}  Новая характеризация пуассоновских
распределений~// Матем. заметки, 1976. T.~20. №\,6. C.~879--882.

  \label{end\stat}

\bibitem{6-chu} %8
\Au{Чупрунов А.\,Н., Фазекаш~И.} Законы повторного логарифма для
числа безошибочных блоков при помехоустойчивом кодировании~// Информатика и её 
применения, 2010. T.~4. №\,3. C.~42--46.

 \end{thebibliography}
}
}


\end{multicols}           %12
\def\stat{sokolov}

\def\tit{О РАБОТАХ ЗАСЛУЖЕННОГО ДЕЯТЕЛЯ НАУКИ 
РОССИЙСКОЙ ФЕДЕРАЦИИ И.\,Н.~СИНИЦЫНА В ОБЛАСТИ 
ИНФОРМАЦИОННЫХ ТЕХНОЛОГИЙ И АВТОМАТИЗАЦИИ\\
(к 70-летию со дня рождения)}

\def\titkol{О работах заслуженного деятеля науки 
РФ И.\,Н.~Синицына в области 
информационных технологий и автоматизации
%(к семидесятилетию со дня рождения)
}

\def\autkol{И.\,А.~Соколов}
\def\aut{И.\,А.~Соколов$^1$}

\titel{\tit}{\aut}{\autkol}{\titkol}

%{\renewcommand{\thefootnote}{\fnsymbol{footnote}}\footnotetext[1]
%{Работа поддерживается РФФИ, грант  10-07-00017.}}

\renewcommand{\thefootnote}{\arabic{footnote}}
\footnotetext[1]{Институт проблем информатики Российской академии наук, isokolov@ipiran.ru}


\bigskip



%\bigskip

       \vskip 14pt plus 9pt minus 6pt

      \thispagestyle{headings}

      \begin{multicols}{2}

      \label{st\stat}
      
 \begin{center}
\mbox{%
\epsfxsize=50mm
\epsfbox{sok-1.eps}
}
\end{center}
\vspace*{4pt}
%\begin{center}

\bigskip



      14~августа 2010~г.\ исполнилось 70~лет Игорю Николаевичу Синицыну~--- члену 
редколлегии журнала <<Информатика и её применения>>, крупному ученому в области 
прикладной механики и управ\-ле\-ния, прикладной математики и информатики, основателю 
научной школы в области стохастических информационных технологий.
      
      И.\,Н.~Синицын родился в Москве. Высшее образование получил в МВТУ им.\ 
Н.\,Э.~Баумана и МГУ им.\ М.\,В.~Ломоносова. Одновременно с учебой в МГУ начал работать 
в известном ра\-кет\-но-кос\-ми\-че\-ском НИИ, ныне Институте прикладной механики им.\ 
В.\,И.~Кузнецова (НИИПМ). Инженерную и научную деятельность в НИИПМ в области 
разработки и испытаний гироскопических командных приборов и 
      ин\-фор\-ма\-ци\-он\-но-из\-ме\-ри\-тель\-ных систем (1960--1983~гг.), он совмещал с 
преподавательской работой сначала в МВТУ им.\ Н.\,Э.~Баумана, затем в 
      Воен\-но-воз\-душ\-ной инженерной академии им.\ профессора Н.\,Е.~Жуковского 
(ВВИА).
      
      Начиная с 1974~г.\ И.\,Н.~Синицын работал на факультете авиационного вооружения 
ВВИА. Занимался подготовкой авиационных инженеров, принимал участие в разработке и 
испытаниях\linebreak
 специальной техники, участвовал в подготовке первых космонавтов СССР.
      
      Для организации работ в области специальных применений ЭВМ новых поколений 
И.\,Н.~Синицын в 1984~г.\ переводится в только что организованный Институт проблем 
информатики АН СССР (ныне ИПИ РАН).
      
      В настоящее время И.\,Н.~Синицын работает заведующим отделом стохастических 
проблем информатики и управления ИПИ РАН, много внимания уделяет подготовке научных 
кадров. Он руководит специальной секцией ученого совета ИПИ РАН, комиссией Минобрнауки 
по информатике в военных вузах, является членом экспертного совета РФФИ, заместителем 
главных редакторов журналов <<Наукоемкие технологии>> и <<Системы высокой 
доступности>>, членом редколлегий журналов <<Pattern Recognition and Image Analysis>>, 
<<Информатика и её применения>>. С~1987~г.\ И.\,Н.~Синицын~--- профессор МАИ, читает 
лекции по теории и практике информационных технологий в инженерном деле. 
     
     В разные годы И.\,Н.~Синицын был заместителем генерального конструктора и главным 
конструктором ряда автоматизированных и информационных систем специального назначения. 
     
     В 2001~г.\ И.\,Н.~Синицыну присвоено почетное звание Заслуженного деятеля науки 
Российской Федерации.
     
     И.\,Н.~Синицын имеет большой опыт работы в промышленности и высших технических 
учебных заведениях. Он автор более 500~научных трудов, свыше 50 книг, монографий и 
30~изобретений. Его основные научные труды относятся к следующим областям:
     \begin{itemize}
     \item статистическая теория информационных технологий и автоматизированных систем;
     \item прецизионные ин\-фор\-ма\-ци\-он\-но-из\-ме\-ри\-тель\-ные технологии и системы для научных 
исследований и специального назначения;
     \item
      информационно-аналитические технологии и системы поддержки принятия решений для 
информатизации высших органов государственной власти РФ, федеральных ведомств и~др.
     \end{itemize}
     
     И.\,Н.~Синицыну принадлежат фундаментальные результаты по теории канонических 
представлений случайных функций в сложных стохастических системах (СтС), в том числе СтС 
с\linebreak
 распределенными параметрами и случайной структурой. Методы теории СтС им 
распространены на\linebreak
 СтС, описываемые дифференциальными уравнениями со случайными 
функциями состояния,\linebreak уравнениями в гильбертовых и банаховых пространствах. Им 
разработаны эффективные вычислительные методы нахождения распределений, основанные на 
параметризации, позволяющие радикально сократить число уравнений для па\-ра\-мет\-ров 
распределений, а также новые вычислительные методы статистического анализа и синтеза,\linebreak 
допускающие эффективное оценивание точности и ориентированные на параллельные 
статистические вычисления. 
     
     И.\,Н.~Синицын разработал методы нахождения точных выражений для распределений с 
инвариантной мерой, обнаружил ряд новых классов точных распределений. Им получены 
фундаментальные результаты в области нелинейной условно оптимальной и субоптимальной 
фильтрации в реальном масштабе времени. Важные результаты получены И.\,Н.~Синицыным в 
области тео\-ре\-ти\-ко-груп\-по\-вых методов анализа и синтеза автоматизированных систем. 
Им разработана статистическая теория катастрофоустойчивости автоматизированных сис\-тем 
высокой точности и доступности. 
     
     И.\,Н.~Синицын~--- основоположник стохастических информационных технологий 
оперативной обработки информации, контроля и мониторинга автоматизированных систем, а 
также\linebreak стохастического управления информационными\linebreak активами, моделирования и синтеза 
систем: проб\-лем\-но-ориен\-ти\-ро\-ван\-ных диалоговых систем и\linebreak библиотек 
     <<СтС-Ана\-лиз>>, <<СтС-Фильтр>>, <<СтС-Мо\-дель>>, Nailb, <<TransStatLib>>, 
<<Безопасность и надежность>>, <<Здоровье РФ>> и~др. В~последние годы им разработаны 
эффективные символьные методы анализа и синтеза СтС. Создано и внедрено 
специализированное программное обеспечение СтС-СМА и СтС-\mbox{ИТКР}. 
     
     Его книги~--- <<Стохастические дифференциальные системы. Анализ и фильтрация>>, 
<<Лекции по функциональному анализу и его приложениям>>, <<Теория стохастических 
систем>> (совместно с В.\,С.~Пугачевым), а также <<Фильтры Калмана и Пугачева>> и 
<<Канонические представления случайных функций и их применение в задачах компьютерной 
поддержки научных исследований>>~--- широко известны в России и за рубежом.
     
     И.\,Н.~Синицыным впервые разработана теория ряда 
     ин\-фор\-ма\-ци\-он\-но-из\-ме\-ри\-тель\-ных систем в условиях случайных динамических 
возмущений, открыт ряд новых статистических динамических эффектов (выбросы разных 
типов, флуктуационные уходы, накопление возмущений и~др.). Ему принадлежат первые 
работы по статистической динамике командно-измерительных гироскопических приборов, 
акселерометров, градиентометров и метрологических систем высочайшей точности, 
информационной теории и методам измерений, калибровок, ускоренных испытаний в 
экстремальных условиях, а также статистического и полунатурного моделирования. Под его 
руководством и при его непосредственном участии разработано и внедрено несколько 
поколений серийных систем, обла\-да\-ющих уникальными характеристиками. И.\,Н.~Синицын 
принимал непосредственное участие в определении технической политики в области новой 
специальной техники. 
     
     С именем И.\,Н.~Синицына связано создание концепций автоматизации научных 
исследований в РФ, в первую очередь основанных на средствах массовой вычислительной 
техники. Под его руководством и при участии сформулированы принципы создания 
микровидеосистем, создан ряд базовых персональных микровидеосистем. Они внедрены в МВД 
и Минздраве РФ. Достигнутые результаты получили развитие в автоматизированных системах 
метрологического обеспечения, видеоконтроля и биометрических системах. 
     
     В последние годы под руководством И.\,Н.~Синицына разработаны принципы построения 
и архитектуры вычислительных систем командных пунктов, а также новые методы и алгоритмы 
быстрой обработки изображений, обладающих сильной про\-стран\-ст\-вен\-но-вре\-мен\-н\'{о}й 
деформацией. В~целях\linebreak автоматизации астрометрических научных исследований по 
фундаментальной проблеме <<Статистическая динамика вращения Земли>> был создан 
комплекс моделей, алгоритмов и специального\linebreak программного обеспечения и информационных 
ресурсов для нестандартной интегрированной обработки параметров вращения Земли. 
И.\,Н.~Синицын впервые обнаружил ряд новых эффектов: автоколебания полюса Земли на 
чандлеровской частоте, параметрическую стабилизацию чандлеровских колебаний, нелинейные 
флуктуационные дрейфы нестабильности вращения Земли и~др.
     
     В области информационно-аналитических технологий и автоматизированных систем 
поддержки принятия решений для информатизации высших органов государственной власти, 
федеральных ведомств и~др.\ под руководством И.\,Н.~Синицына был разработан и внедрен 
ряд базовых информационных технологий (обработка информации от независимых источников, 
формирование и хранение больших баз электронных образов, управления информационными 
активами и~др.). Сформулированы принципы и разработаны базовые системотехнические 
решения для ряда крупномасштабных автоматизированных информационных и 
     ин\-фор\-ма\-ци\-он\-но-управ\-ля\-ющих систем специального назначения, высокой 
точности и доступности. 
     
     И.\,Н.~Синицын пользуется широким международным авторитетом: его книги и работы 
изданы на английском, французском, испанском и китайском языках; в 1990--1994~гг.\ он был 
директором Рос\-сий\-ско-фран\-цуз\-ско\-го центра <<Эвклид>>, с~1983~г.~--- членом 
программных комитетов многих международных конференций; в период 1985--2006~гг.\ был 
экспертом фонда INTAS.
     
     И.\,Н.~Синицын ведет активную работу по подготовке научно-педагогических кадров. 
Под его руководством выполнено свыше 25~кандидатских и докторских диссертаций. Он 
состоит членом ряда специализированных диссертационных советов; в течение многих лет был 
членом экспертного совета ВАК России. И.\,Н.~Синицын~--- член редколлегии нашего 
журнала, он не толко публикуется в нем, но и ведет большую редакционную работу.

\label{end\stat}
     
     \bigskip
     Редколлегия журнала сердечно поздравляет И.\,Н.~Синицына с юбилеем и желает ему 
здо\-ровья, счастья, новых творческих успехов.


\end{multicols}

%   { %\Large  
   { %\baselineskip=16.6pt
   
   \vspace*{-48pt}
   \begin{center}\LARGE
   \textit{Предисловие}
   \end{center}
   
   %\vspace*{2.5mm}
   
   \vspace*{25mm}
   
   \thispagestyle{empty}
   
   { %\small 

    
Вниманию читателей журнала <<Информатика и её применения>> предлагается 
очередной тематический выпуск <<Вероятностно-статистические методы и 
задачи информатики и информационных технологий>>. Предыдущие тематические 
выпуски журнала по данному направлению вышли в 2008~г.\ (т.~2, вып.~2), 
в 2009~г.\ (т.~3, вып.~3) и в 2010~г.\ (т.~4, вып.~2). 

Статьи, собранные в данном журнале, посвящены разработке новых вероятностно-статистических 
методов, ориентированных на применение к решению конкретных задач информатики и информационных 
технологий, а также~--- в ряде случаев~--- и других прикладных задач. Проблематика, охватываемая 
публикуемыми работами, развивается в рамках научного сотрудничества между Институтом проблем 
информатики Российской академии наук (ИПИ РАН) и Факультетом вычислительной математики и 
кибернетики Московского государственного университета им.\ М.\,В.~Ломоносова в ходе работ 
над совместными научными проектами (в том числе в рамках функционирования 
Научно-образовательного центра <<Вероятностно-статистические методы анализа рисков>>). 
Многие из авторов статей, включенных в данный номер журнала, являются активными участниками 
традиционного международного семинара по проблемам устойчивости стохастических моделей, 
руководимого В.\,М.~Золотаревым и В.\,Ю.~Королевым; регулярные сессии этого семинара 
проводятся под эгидой МГУ и ИПИ РАН (в 2011~г.\ указанный семинар проводится в октябре 
в Калининградской области РФ). 

Наряду с представителями ИПИ РАН и МГУ в число авторов данного выпуска журнала входят 
ученые из Научно-исследовательского института системных исследований РАН, Института 
проблем технологии микроэлектроники и особочистых материалов РАН, Института 
прикладных математических исследований Карельского НЦ РАН, Московского 
авиационного института, Вологодского государственного педагогического университета, 
НИИММ им.\ Н.\,Г.~Чеботарева, Казанского государственного университета, Дебреценского 
университета (Венгрия).

Несколько статей выпуска посвящено разработке и применению стохастических методов и 
информационных технологий для решения различных прикладных задач. В~работе В.\,Г.~Ушакова 
и О.\,В.~Шестакова рассмотрена задача определения вероятностных характеристик случайных 
функций по распределениям интегральных преобразований, возникающих в задачах эмиссионной 
томографии. В~статье Д.\,О.~Яковенко и М.\,А.~Целищева рассмотрены некоторые вопросы 
математической теории риска и предложен новый подход к диверсификации инвестиционных 
портфелей. Работа И.\,А.~Кудрявцевой и А.\,В.~Пантелеева посвящена построению и 
исследованию математической модели, описывающей динамику сильноионизованной плазмы. 
В~статье П.\,П.~Кольцова изучается качество работы ряда алгоритмов сегментации изображений. 
Статья А.\,Н.~Чупрунова и И.~Фазекаша посвящена вероятностному анализу числа без\-оши\-бочных 
блоков при помехоустойчивом кодировании; получены усиленные законы больших чисел для указанных 
величин.

В данном выпуске традиционно присутствует тематика, весьма активно разрабатываемая в течение 
многих лет специалистами ИПИ РАН и МГУ,~--- методы моделирования и управления для 
информационно-телекоммуникационных и вычислительных систем, в частности методы 
теории массового обслуживания. В~статье А.\,И.~Зейфмана с соавторами рассматриваются 
модели обслуживания, описываемые марковскими цепями с непрерывным временем в случае 
наличия катастроф. В~работе М.\,М.~Лери и И.\,А.~Чеплюковой рассматриваются случайные 
графы Интернет-типа, т.\,е.\ графы, степени вершин которых имеют степенные распределения; 
такие задачи находят применение при исследовании глобальных сетей передачи данных. 
Работа Р.\,В.~Разумчика посвящена исследованию систем массового обслуживания специального 
вида~--- с отрицательными заявками и хранением вытесненных заявок.

Ряд статей посвящен развитию перспективных теоретических 
вероятностно-статистических методов, которые находят широкое применение в различных 
задачах информатики и информационных технологий. В~работе В.\,Е.~Бенинга, А.\,К.~Горшенина 
и В.\,Ю.~Королева рассмотрена задача статистической проверки гипотез о числе компонент 
смеси вероятностных распределений, приводится конструкция асимптотически наиболее мощного 
критерия. Результаты этой работы найдут применение в ряде прикладных задач, использующих 
математическую модель смеси вероятностных распределений (в информатике, моделировании 
финансовых рынков, физике турбулентной плазмы и~т.\,д.). В~статье В.\,Ю.~Королева, 
И.\,Г.~Шевцовой и С.\,Я.~Шоргина строится новая, улучшенная оценка точности нормальной 
аппроксимации для пуассоновских случайных сумм; как известно, указанные случайные суммы 
широко используются в качестве моделей многих реальных объектов, в том числе в информатике, 
физике и других прикладных областях. Работа В.\,Г.~Ушакова и Н.\,Г.~Ушакова посвящена 
исследованию ядерной оценки плотности распределения; эти результаты могут применяться, 
в част\-ности, при анализе трафика в телекоммуникационных системах. Серьезные приложения 
в статистике могут получить результаты работы О.\,В.~Шестакова, в которой доказаны оценки 
скорости сходимости распределения выборочного абсолютного медианного отклонения к нормальному 
закону. 

\smallskip

Редакционная коллегия журнала выражает надежду, что данный тематический  выпуск 
будет интересен специалистам в области теории вероятностей и математической статистики 
и их применения к решению задач информатики и информационных технологий.
     
     %\vfill 
     \vspace*{20mm}
     \noindent
     Заместитель главного редактора журнала <<Информатика и её 
применения>>,\\
     директор ИПИ РАН, академик  \hfill
     \textit{И.\,А.~Соколов}\\
     
     \noindent
     Редактор-составитель тематического выпуска,\\
     профессор кафедры математической статистики факультета\\
      вычислительной математики и кибернетики МГУ им.\ М.\,В.~Ломоносова,\\
     ведущий научный сотрудник ИПИ РАН,\\ 
доктор физико-математических наук \hfill
      \textit{В.\,Ю.~Королев}
     
     } }
     }

%%%%%%%%%%%%%%%%%%%%%%%%%%%%%%%%%%%%%%%%%%%%%%%


                       
%\end{document}

%\def\stat{rez}
{%\hrule\par
%\vskip 7pt % 7pt
\raggedleft\Large \bf%\baselineskip=3.2ex
Р\,Е\,Ц\,Е\,Н\,З\,И\,И \vskip 17pt
    \hrule
    \par
\vskip 6pt plus 6pt minus 3pt }

%\thispagestyle{headings} %с верхним колонтитулом
%\thispagestyle{myheadings} %с нижним колонтитулом, но в верхнем РЕЦЕНЗИИ

\def\tit{НОВАЯ КНИГА И.\,Н.~СИНИЦЫНА, А.\,С.~ШАЛАМОВА <<ЛЕКЦИИ ПО ТЕОРИИ 
ИНТЕГРИРОВАННОЙ ЛОГИСТИЧЕСКОЙ ПОДДЕРЖКИ>> (М.: ТОРУС ПРЕСС, 2012. 624~с.)}

%1
\def\aut{Д.ф.-м.н., профессор С.\,Я.~Шоргин}

\def\auf{\ }

\def\leftkol{\ % РЕЦЕНЗИИ
}

\def\rightkol{ \ } 

%\def\leftkol{\ } % ENGLISH ABSTRACTS}

%\def\rightkol{\ } %ENGLISH ABSTRACTS}

%\def\leftkol{РЕЦЕНЗИИ}

%\def\rightkol{РЕЦЕНЗИИ}

\titele{\tit}{\aut}{\auf}{\leftkol}{\rightkol}
\vspace*{-18pt}


     \label{st\stat}

     \begin{multicols}{2}
     {\small
     {\baselineskip=10.1pt
     

      В книге представлено системное изложение теоретических основ одного из новейших 
направлений в \mbox{об\-ласти} экономики послепродажного обслуживания изделий наукоемкой 
продукции (ИНП) длительного пользования~--- интегрированной логистической поддержки
(ИЛП). 
{\looseness=1

}

Приведены также результаты новых работ, выполненных в Институте проблем информатики 
Российской академии наук в рамках научного направления <<Информационные технологии и 
анализ сложных сис\-тем>>.
 {%\looseness=1

}
     
      Излагаемые в книге научные подходы позво\-ляют карди\-наль\-но реформировать 
существующие системы производства и эксплуатации ИНП путем создания и внед\-ре\-ния 
методов рационального и оптимального управ\-ле\-ния процессами расходования 
вре\-мен\-н$\acute{\mbox{ы}}$х, 
мате\-ри\-аль\-ных, трудовых и других ресурсов на всех стадиях жизненного цикла изделий (ЖЦИ) по 
критериям экономической целесообразности и эф\-фек\-тив\-ности.
  {\looseness=1

}
    
      В книге приведен краткий обзор причин возник\-новения и
      развития CALS-методологии как основы 
современных международных стандартов по созданию и функционированию глобальных 
ин\-фор\-ма\-ци\-он\-но-ком\-му\-ни\-ка\-ци\-он\-ных систем, ее ключевых возможностей и эффективности 
результатов ее использования. 
Авторы %\linebreak 
предлагают ряд научных обоснований для разработки 
единой теории проектирования и управления систем ИЛП для полноценного использования 
преимуществ %\linebreak
 суще\-ст\-ву\-ющей методологии, определяют \mbox{общую} структурную схему 
комплексной системы <<ИНП-СППО>> и необходимость разработки для ее описания 
гибридных стохастических моделей.
{%\looseness=1

}

%\columnbreak
      
      Книга состоит из пяти частей, где последовательно излагается материал по каждой из 
следующих тем: <<Интегрированная логистическая поддержка>>, <<Теория гибридных 
стохастических систем и компьютерная поддержка исследований и разработок>>, <<Основы 
математического моделирования, анализа и синтеза систем послепродажного обслуживания>>, 
<<Определение и анализ показателей экспортного потенциала ИНП при проектировании>>, 
<<Задачи управления поддержкой послепродажного обслуживания>>, а также 
<<Моделирование инвестиционных процессов ИЛП в условиях неравновесных финансовых 
рынков>>. 
   
      В конце каждой главы приведены выводы и даны вопросы и задания для 
самоконтроля. В~приложениях содержатся основные определения по программам работ по 
анализу ИЛП, логистическим базам данных и компьютерным решениям, эквивалентной статистической 
линеаризации нелинейных преобразований ИЛП, справочный материал, а также развернутые 
уравнения для вероятностных характеристик.


      \def\leftkol{РЕЦЕНЗИИ}

\def\rightkol{РЕЦЕНЗИИ} 

      
      Книга заинтересует широкий круг специалистов и может быть использована научными 
проектными организациями в сфере промышленного производства ИНП. Большое количество 
иллюстраций, примеров и вопросов, обращенных к читателю, позволяет использовать книгу 
также в качестве учебного пособия для студентов и аспирантов машиностроительных, 
транспортных и~других специальностей, а также для самостоятельного изучения. 
{%\looseness=-1

}

Книга 
представляет несомненный интерес для специалистов и студентов в области прикладной 
математики и информатики.
    

}

}
\end{multicols}

%\newpage

\include{obchak}



\def\stat{authorsrus}
{%\hrule\par
%\vskip 7pt % 7pt
\raggedleft\Large \bf%\baselineskip=3.2ex
О\,Б\ \ А\,В\,Т\,О\,Р\,А\,Х \vskip 17pt
    \hrule
    \par
\vskip 21pt plus 8pt minus 4pt }


\def\tit{\ }

\def\aut{\ }

\def\auf{\ }

\def\leftkol{\ } % ENGLISH ABSTRACTS}

\def\rightkol{ОБ АВТОРАХ} %ENGLISH ABSTRACTS}

\titele{\tit}{\aut}{\auf}{\leftkol}{\rightkol}
      
            \label{st\stat}



\vspace*{24pt}

\begin{multicols}{2}




\noindent
\textbf{Архипов Олег Петрович} (р.\ 1948)~---
кандидат технических наук, директор Орловского филиала Института проб\-лем информатики
Российской академии наук
%302025, г.Орел, Московское шоссе, д.137

\vspace*{3pt}

\noindent
\textbf{Бирюкова Татьяна Константиновна} (р.\ 1968)~---
кандидат фи\-зи\-ко-ма\-те\-ма\-ти\-че\-ских наук, старший научный сотрудник Института проб\-лем информатики
Российской академии наук

\vspace*{3pt}

\noindent 
\textbf{Бобков  Сергей Геннадьевич} (р.\ 1955)~---
доктор технических наук,  заведующий отделением На\-уч\-но-ис\-сле\-до\-ва\-тель\-ско\-го 
института системных исследований Российской академии наук
%117218, Москва, Нахимовский просп., 36, к.1 

\vspace*{3pt}

\noindent \textbf{Васильев Николай Семенович} (р.\ 1952)~--- доктор 
фи\-зи\-ко-ма\-те\-ма\-ти\-че\-ских наук, профессор, 
МГТУ им.\ Н.\,Э.~Баумана 
%, Москва 105005, 2-я Бауманская ул., д.~5,

\vspace*{3pt}

\noindent
\textbf{Гершкович Максим Михайлович} (р.\ 1968)~---
старший научный сотрудник Института проб\-лем информатики
Российской академии наук

\vspace*{3pt}

\noindent 
\textbf{Дьяченко Юрий Георгиевич} (р.\ 1958)~--- кандидат технических наук, 
старший научный сотрудник Института проб\-лем информатики
Российской академии наук

\vspace*{3pt}

\noindent 
\textbf{Ерошенко Александр Андреевич} (р.\ 1989)~--- аспирант кафедры 
математической статистики факультета вычисли\-тельной математики и кибернетики 
Московского государственного университета им.\ М.\,В.~Ломоносова
%119991, Москва ГСП-1, Ленинские горы, д.\ 1, стр. 52

\vspace*{3pt}
 
\noindent 
\textbf{Захаров Виктор Николаевич} (р.\ 1948)~--- 
доктор технических наук, доцент, ученый секретарь Института проб\-лем информатики
Российской академии наук

\vspace*{3pt}

\noindent
\textbf{Зейфман Александр Израилевич} (р.\ 1954)~---
доктор фи\-зи\-ко-ма\-те\-ма\-ти\-че\-ских наук, профессор, 
заведующий кафедрой Вологодского государственного университета; 
старший научный сотрудник Института проб\-лем информатики
Российской академии наук; главный научный сотрудник ИСЭРТ Российской академии наук

\vspace*{3pt}

\noindent
\textbf{Зыкин Сергей Владимирович} (р.\ 1959)~--- 
доктор технических наук, профессор, заведующий лабораторией Института математики 
им.\ С.\,Л.~Соболева Сибирского отделения Российской академии наук, Новосибирск 
%630090, пр.\ ак.\ Коптюга, 4 

\vspace*{4pt}

\noindent
\textbf{Киреев Владимир Иванович} (р.\ 1938)~---
доктор фи\-зи\-ко-ма\-те\-ма\-ти\-че\-ских наук, профессор Московского 
государственного горного университета
%Адрес: Россия, 119991, г. Москва, Ленинский проспект, д. 6

%\columnbreak

\vspace*{4pt}

\noindent
\textbf{Козеренко Елена Борисовна} (р.\ 1959)~---
кандидат филологических наук, заведующая лабораторией Института проб\-лем информатики
Российской академии наук

\vspace*{4pt}

\noindent
\textbf{Королев Виктор Юрьевич} (р.\ 1954)~--- доктор
фи\-зи\-ко-ма\-те\-ма\-ти\-че\-ских наук, профессор кафедры математической 
статистики факультета вычисли\-тельной математики и кибернетики 
Московского государственного университета; 
ведущий научный сотрудник Института проб\-лем информатики
Российской академии наук

\vspace*{4pt}

\noindent
\textbf{Коротышева Анна Владимировна} (р.\ 1988)~---
старший преподаватель Вологодского государственного университета

\vspace*{4pt}

\noindent 
\textbf{Кун Де Турк} (р.\ 1981)~--- научный сотрудник 
исследовательской группы SMACS факультета телекоммуникаций и обработки информации
Университета Гента, Бельгия
%В-9000 Гент, Бельгия

\vspace*{4pt}

\noindent
\textbf{Лупенцов Олег Сергеевич} (р.\ 1986)~---
аспирант Омского государственного института сервиса
%Омск 644043, ул.\ Певцова 13

\vspace*{4pt}

\noindent
\textbf{Лучко Олег Николаевич} (р.\ 1961)~---
кандидат педагогических наук, профессор, заведующий кафедрой 
Омского государственного института сервиса
%Омск 644043, ул.\ Певцова 13

\vspace*{4pt}

\noindent
\textbf{Малашенко Юрий Евгеньевич} (р.\ 1946)~---
доктор фи\-зи\-ко-ма\-те\-ма\-ти\-че\-ских наук, заведующий сектором 
Вычислительного центра им.\ А.\,А.~Дородницына Российской академии наук
%Адрес: 119333, Москва, ул. Вавилова, 40,

\vspace*{4pt}

\noindent
\textbf{Маньяков Юрий Анатольевич} (р.\ 1984)~---
кандидат технических наук, научный сотрудник Орловского филиала Института проб\-лем информатики
Российской академии наук
%302025, г.Орел, Московское шоссе, д.137

\vspace*{4pt}

\noindent
\textbf{Маренко Валентина Афанасьевна} (р.\ 1951)~---
кандидат технических наук, доцент, старший научный сотрудник 
Института математики им.\ С.\,Л.~Соболева Сибирского отделения Российской академии наук
%Новосибирск 630090, пр. ак. Коптюга, 4 

\vspace*{3pt}

\noindent 
\textbf{Морозов Евсей Викторович} (р.\ 1947)~--- доктор 
фи\-зи\-ко-ма\-те\-ма\-ти\-че\-ских, профессор, ведущий научный сотрудник 
Института прикладных математических исследований Карельского научного центра Российской
академии наук; 
%%185910 Россия, Республика Карелия, г.\ Петрозаводск, ул.\ Пушкинская, 11
профессор Петрозаводского государственного университета, Петрозаводск
%185910 Россия, Республика Карелия, г.\ Петрозаводск, пр.\ Ленина, 33

%\pagebreak

\vspace*{3pt}

\noindent
\textbf{Назарова Ирина Александровна} (р.\ 1966)~---
кандидат фи\-зи\-ко-ма\-те\-ма\-ти\-че\-ских наук, 
научный сотрудник Вычислительного центра им.\ А.\,А.~Дородницына Российской академии наук 
%Адрес: 119333, Москва, ул. Вавилова, 40

\vspace*{3pt}

\noindent
\textbf{Павлов Игорь Валерианович} (р.\ 1945)~--- 
доктор фи\-зи\-ко-ма\-те\-ма\-ти\-че\-ских наук, профессор МГТУ им.\ Н.\,Э.~Баумана 
%Москва 105005, 2-я Бауманская ул., д.~5 

%\pagebreak

\vspace*{3pt}

\noindent 
\textbf{Потахина Любовь Викторовна} (р.\ 1989)~--- аспирантка
Института прикладных математических исследований Карельского научного центра
Российской академии наук; 
%%185910 Россия, Республика Карелия, г.\ Петрозаводск, ул.\ Пушкинская, 11
инженер Петрозаводского государственного университета, Петрозаводск
%185910 Россия, Республика Карелия, г.\ Петрозаводск, пр.\ Ленина, 33

\vspace*{3pt}

\noindent 
\textbf{Рождественский Юрий Владимирович} (р.\ 1952)~--- 
кандидат технических наук, заведующий сектором Института проб\-лем информатики
Российской академии наук

\vspace*{3pt}

\noindent 
\textbf{Синицын Игорь Николаевич} (р.\ 1940)~--- доктор технических наук,
профессор, заслуженный деятель\linebreak\vspace*{-12pt}

\columnbreak

\noindent
 науки РФ, заведующий отделом Института проб\-лем информатики
Российской академии наук

\vspace*{7pt}


\noindent
\textbf{Сиротинин Денис Олегович} (р.\ 1984)~---
кандидат технических наук, научный сотрудник Орловского филиала Института проб\-лем информатики
Российской академии наук
%302025, г.Орел, Московское шоссе, д.137

\vspace*{7pt}

%\columnbreak

\noindent 
\textbf{Соколов  Игорь Анатольевич} (р.\ 1954)~--- академик (действительный член) Российской 
академии наук, доктор технических наук, директор Института проб\-лем информатики
Российской академии наук

\vspace*{7pt}

\noindent
\textbf{Степченков Юрий Афанасьевич} (р.\ 1951)~---
кандидат технических наук, заведующий отделом Института проб\-лем информатики
Российской академии наук

\vspace*{7pt}

\noindent
\textbf{Сурков Алексей Викторович} (р.\ 1978)~--- 
старший научный сотрудник На\-уч\-но-ис\-сле\-до\-ва\-тель\-ско\-го 
института системных исследований Российской академии наук
%117218, Москва, Нахимовский просп., 36, к.1 

\vspace*{7pt}

\noindent 
\textbf{Шестаков Олег Владимирович} (р.\ 1976)~--- доктор 
фи\-зи\-ко-ма\-те\-ма\-ти\-че\-ских, доцент кафедры математической статистики 
факультета вычисли\-тельной математики и кибернетики Московского 
государственного университета им.\ М.\,В.~Ломоносова; 
%119991, Москва ГСП-1, Ленинские горы, д.\ 1, стр. 52
старший научный сотрудник Института проб\-лем информатики
Российской академии наук
%, Москва 119333, ул. Вавилова, д.~44, корп.~2

\vspace*{7pt}

\noindent 
\textbf{Шоргин Сергей Яковлевич} (р.\ 1952.)~--- доктор
фи\-зи\-ко-ма\-те\-ма\-ти\-че\-ских наук, профессор, заместитель директора Института 
проб\-лем информатики Российской академии наук





%%%%%%%%%%%%%%%%%%%%%%%%%%%%%%%%%%%%%%%%%%%%%%%%%%%%%%%%%%%%%%%%%%%%%%%%%%%%%%%




%\def\rightkol{ОБ АВТОРАХ}
%\def\leftkol{ОБ АВТОРАХ}

 \label{end\stat}





%\def\leftfootline{\small{\textbf{\thepage}
%\hfill ИНФОРМАТИКА И ЕЁ ПРИМЕНЕНИЯ\ \ \ том~7\ \ \ выпуск~1\ \ \ 2013}
%}%
% \def\rightfootline{\small{ИНФОРМАТИКА И ЕЁ ПРИМЕНЕНИЯ\ \ \ том~7\ \ \ выпуск~1\ \ \ 2013
%\hfill \textbf{\thepage}}}


%\thispagestyle{myheadings}



\end{multicols}

\newpage


%%\vspace*{-48pt}
\begin{center}\LARGE
\textit{About Authors}
\end{center}

\thispagestyle{empty}
\def\tit{\ }

\def\aut{\ }

\def\auf{\ }


\def\leftkol{ABOUT AUTHORS}

\def\rightkol{ABOUT AUTHORS}

\vspace*{-18pt}

\titele{\tit}{\aut}{\auf}{\leftkol}{\rightkol}

%\vspace*{36pt}

\def\rightmark{{\noindent\hbox to \textwidth{\hfill\small ABOUT AUTHORS
%\hfill \large\bf\thepage
}}}
\def\leftmark{{\noindent\parbox{\textwidth}{
%\raggedleft\large\bf\thepage \hfill
\small\textrm{ABOUT AUTHORS}\hfill}}}


\def\leftfootline{\small{\textbf{\thepage}
\hfill ИНФОРМАТИКА И ЕЁ ПРИМЕНЕНИЯ\ \ \ том~6\ \ \ выпуск~2\ \ \ 2012}
}%
 \def\rightfootline{\small{ИНФОРМАТИКА И ЕЁ ПРИМЕНЕНИЯ\ \ \ том~6\ \ \ выпуск~2\ \ \ 2012
\hfill \textbf{\thepage}}}


\begin{multicols}{2}

\noindent
\textbf{Agalarov Yaver M.} (b.\ 1952)~--- Candidate of Science (PhD)
in technology, 
leading scientist, Institute of Informatics Problems, Russian Academy of Sciences

\vspace*{5pt}


  \noindent
\textbf{Bosov Alexey V.} (b.\ 1969)~--- Doctor of Science in technology, Head of
Laboratory, Institute of Informatics Problems, Russian Academy of Sciences

\vspace*{5pt}


\noindent
\textbf{Dulin Sergey K.} (b.\ 1950)~--- Doctor of Science in technology, 
professor, senior scientist, Institute of Informatics Problems, Russian Academy of Sciences

\vspace*{5pt}

\noindent
\textbf{Gorshenin Andrey K.}~--- (b.\ 1986)~--- Candidate of Science (PhD)
in physics and mathematics,
senior scientist, Institute of Informatics Problems, Russian Academy of Sciences

\vspace*{5pt}

\noindent
\textbf{Kalenov Nikolay E.}  (b.\ 1945)~--- Doctor of Science in technology,
professor, Director, Library for Natural Sciences,  Russian Academy of Sciences 

\vspace*{5pt}

\noindent
\textbf{Kalinichenko Leonid A.} (b.\ 1937)~--- Doctor of Science in physics and mathematics, 
professor, Honored scientist of RF, 
Head of Laboratory, Institute of Informatics Problems, Russian Academy of Sciences 

\vspace*{5pt}

\noindent
\textbf{Karpov Alexey A.} (b.\ 1978)~--- Candidate of Science (PhD) in technology, 
senior scientist, St.\ Petersburg Institute for
Informatics and Automation,  Russian Academy of Sciences

\vspace*{5pt}

\noindent
\textbf{Kuznetsov Igor P.} (b.\ 1938)~--- Doctor of Science in technology, 
professor, principal scientist, Institute of Informatics Problems, Russian Academy of Sciences

\vspace*{5pt}


\noindent
\textbf{Markova Natalia A.} (b.\ 1950)~--- Candidate of Science (PhD) in
physics and mathematics, leading scientist,  
Institute of Informatics Problems, Russian Academy of Sciences

\vspace*{5pt}

\noindent
\textbf{Nikolaev Andrey V.} (b.\ 1985)~--- Candidate of Science (PhD) in technology, 
senior lecturer, Tchaikovsky Technological Institute, Branch of the Izhevsk State Technical 
University

\vspace*{6pt}

\noindent
\textbf{Pavlov Igor V.} (b.\ 1945)~---  Doctor of Science in physics and mathematics,
professor, Bauman Moscow State Technical University

\vspace*{6pt}

%\columnbreak

\noindent
\textbf{Rozenberg Igor N.} (b.\ 1965)~--- Doctor of Science in technology, 
First Deputy Director General, Research \& Design Institute for Information 
Technology, Signalling and Telecommunications on Railway Transport (JSC NIIAS)

\vspace*{6pt}


\noindent
\textbf{Semenov Konstantin K.} (b.\ 1986)~--- MPhil, 
associate professor, St.\ Petersburg State Polytechnical University

\vspace*{6pt}

\noindent
\textbf{Sharnin Mikhail M.} (b.\ 1959)~--- Candidate of Science (PhD) 
in technology, senior scientist, Institute of Informatics Problems, Russian Academy of Sciences

\vspace*{6pt}

\noindent 
\textbf{Shestakov Oleg V.} (b.\ 1976)~--- Candidate of Science (PhD) in physics and mathematics,
associate professor, Department of Mathematical Statistics, Faculty of Computational Mathematics and Cybernetics,
M.\,V.~Lomonosov Moscow State University; senior scientist, Institute of Informatics Problems, 
Russian Academy of Sciences

\vspace*{6pt}

\noindent
\textbf{Stupnikov Sergey A.} (b.\ 1978)~--- Candidate of Science (PhD) in technology, 
senior scientist, Institute of Informatics Problems, Russian Academy of Sciences 

\vspace*{6pt}

\noindent
\textbf{Umansky Vladimir I.} (b.\ 1954)~--- Candidate of Science (PhD) in technology, 
Director General, ``IntechGeoTrans'' Closed Joint Stock Company

\vspace*{6pt}

\noindent
\textbf{Zhevnerchuk Dmitry V.} (b.\ 1978)~--- Candidate of Science (PhD) in technology, 
associate professor, Tchaikovsky Technological Institute, Branch of the Izhevsk State 
Technical University

%\vspace*{6pt}

\def\leftfootline{\small{\textbf{\thepage}
\hfill ИНФОРМАТИКА И ЕЁ ПРИМЕНЕНИЯ\ \ \ том~6\ \ \ выпуск~2\ \ \ 2012}
}%
 \def\rightfootline{\small{ИНФОРМАТИКА И ЕЁ ПРИМЕНЕНИЯ\ \ \ том~6\ \ \ выпуск~2\ \ \ 2012
\hfill \textbf{\thepage}}}



%\thispagestyle{myheadings}

\end{multicols}
\newpage



%\include{IPPM-25}

%\def\stat{cont}
{%\hrule\par
%\vskip 7pt % 7pt
\raggedleft\Large \bf%\baselineskip=3.2ex
А\,В\,Т\,О\,Р\,С\,К\,И\,Й\ \ У\,К\,А\,З\,А\,Т\,Е\,Л\,Ь\ \ З\,А\ \ 2\,0\,1\,0 г. \vskip 17pt
    \hrule
    \par
\vskip 21pt plus 6pt minus 3pt }

\label{st\stat}

\def\tit{\ }

\def\aut{\ }
\def\auf{\ }

\def\leftkol{\ } % ENGLISH ABSTRACTS}

\def\rightkol{\ } %АВТОРСКИЙ УКАЗАТЕЛЬ ЗА 2010 г.} %ENGLISH ABSTRACTS}

\titele{\tit}{\aut}{\auf}{\leftkol}{\rightkol}

\vspace*{-12pt}

{\tabcolsep=3pt
\begin{tabular}{p{388pt}rr}
&\textbf{Выпуск} & \textbf{Стр.}\\[6pt]
\hangindent=23pt\noindent\textbf{Арутюнян~А.\,Р.} Моделирование влияния деформаций отпечатков пальцев на 
точность\linebreak
\vspace*{-12pt}\\
\hspace*{23pt}дактилоскопической идентификации$\dotfill$&1&51\\
\hangindent=23pt\noindent\textbf{Архипов~О.\,П., Зыкова~З.\,П.} Интеграция гетерогенной информации о цветных 
пикселях\linebreak
\vspace*{-12pt}\\
\hspace*{23pt}и их цветовосприятии$\dotfill$&4&15\\
\hangindent=23pt\noindent\textbf{Баранов~С.\,И., Френкель~С.\,Л., Захаров~В.\,Н.} Полуформальная верификация 
цифрового устройства с конвейером, основанная на использовании алгоритмических машин\linebreak
\vspace*{-12pt}\\
\hspace*{23pt}состояния$\dotfill$&4&49\\
\textbf{Бекетова~И.\,В.} см.~Каратеев~С.\,Л.&&\\
\textbf{Белоусов~В.\,В.} см.~Синицын~И.\,Н.&&\\
\hangindent=23pt\noindent\textbf{Бенинг~В.\,Е., Королев~Р.\,А.} О предельном поведении мощностей критериев в 
случае\linebreak
\vspace*{-12pt}\\
\hspace*{23pt}распределения Лапласа$\dotfill$&2&63\\
\hangindent=23pt\noindent\textbf{Бенинг~В.\,Е., Сипина~А.\,В.} Асимптотическое разложение для мощности 
критерия,\linebreak
\vspace*{-12pt}\\
\hspace*{23pt}основанного на выборочной медиане, в случае распределения Лапласа$\dotfill$&1&18\\
\textbf{Бондаренко~А.\,В.} см.~Каратеев~С.\,Л.&&\\
\hangindent=23pt\noindent\textbf{Бородина~А.\,В., Морозов~Е.\,В.} Об оценивании асимптотики вероятности 
большого\linebreak
\vspace*{-12pt}\\
\hspace*{23pt}уклонения стационарной регенеративной очереди с одним прибором$\dotfill$&3&29\\
\hangindent=23pt\noindent\textbf{Бунтман~Н.\,В., Минель~Ж.-Л., Ле~Пезан~Д., Зацман~И.\,М.} Типология и 
компьютерное\linebreak
\vspace*{-12pt}\\
\hspace*{23pt}моделирование трудностей перевода$\dotfill$&3&77\\
\textbf{Визильтер~Ю.\,В.} см.~Каратеев~С.\,Л.&&\\
\hangindent=23pt\noindent\textbf{Гавриленко~С.\,В.} Оценки скорости сходимости распределений случайных сумм с 
безгранично делимыми индексами к нормальному закону$\dotfill$&4&81\\
\hangindent=23pt\noindent\textbf{Григорьева~М.\,Е., Шевцова~И.\,Г.} Уточнение неравенства 
Каца--Берри--Эссеена$\dotfill$&2&75\\
\hangindent=23pt\noindent\textbf{Грушо~А.\,А., Грушо~Н.\,А., Тимонина~Е.\,Е.} Поиск конфликтов в политиках 
безопасности: модель случайных графов$\dotfill$&3&38\\
\textbf{Грушо~Н.\,А.} см.~Грушо~А.\,А.&&\\
\hangindent=23pt\noindent\textbf{Гудков~В.\,Ю.} Математические модели изображения отпечатка пальца на основе 
описания линий$\dotfill$&1&58\\
\textbf{Гуртов~А.\,В.} см.~Лукьяненко~А.\,С.&&\\
\textbf{Желтов~С.\,Ю.} см.~Каратеев~С.\,Л.&&\\
\hangindent=23pt\noindent\textbf{Захаров~А.\,А., Серебряков~В.\,А.} Система управления электронной библиотекой 
LibMeta$\dotfill$&4&2\\
\textbf{Захаров~В.\,Н.} см.~Баранов~С.\,И.&&\\
\textbf{Захарова~Т.\,В.} см.~Матвеева~С.\,С.&&\\
\hangindent=23pt\noindent\textbf{Зацаринный~А.\,А., Чупраков~К.\,Г.} Некоторые аспекты выбора технологии для 
постро-\linebreak
\vspace*{-12pt}\\
\hspace*{23pt}ения систем отображения информации ситуационного центра$\dotfill$&3&59\\
\textbf{Зацман~И.\,М.} см.~Бунтман~Н.\,В.&&\\
\hangindent=23pt\noindent\textbf{Зейфман~А.\,И., Коротышева~А.\,В., Сатин~Я.\,А., Шоргин~С.\,Я.} Об 
устойчивости нестаци-\linebreak
\vspace*{-12pt}\\
\hspace*{23pt}онарных систем обслуживания с катастрофами$\dotfill$&3&9\\
\textbf{Зыкова~З.\,П.} см.~Архипов~О.\,П.&&\\
\hangindent=23pt\noindent\textbf{Илюшин~Г.\,Я., Соколов~И.\,А.} Организация управляемого доступа пользователей 
к\linebreak
\vspace*{-12pt}\\
\hspace*{23pt}разнородным ведомственным информационным ресурсам$\dotfill$&1&24\\
\hangindent=23pt\noindent\textbf{Кавагучи~Ю., Ульянов~В.\,В., Фуджикоши~Я.} Приближения для статистик, 
описывающих\linebreak
\vspace*{-12pt}\\
\hspace*{23pt}геометрические свойства данных большой размерности, с оценками 
ошибок$\dotfill$&1&12\\
\hangindent=23pt\noindent\textbf{Каратеев~С.\,Л., Бекетова~И.\,В., Ососков~М.\,В., Князь~В.\,А., 
Визильтер~Ю.\,В., Бондаренко~А.\,В., Желтов~С.\,Ю.} Автоматизированный контроль 
качества цифровых\linebreak
\vspace*{-12pt}\\
\hspace*{23pt}изображений для персональных документов$\dotfill$&1&65\\
\end{tabular}
}

\pagebreak

\def\leftkol{АВТОРСКИЙ УКАЗАТЕЛЬ ЗА 2010 г.} % ENGLISH ABSTRACTS}

\def\rightkol{АВТОРСКИЙ УКАЗАТЕЛЬ ЗА 2010 г.} %ENGLISH ABSTRACTS}

{\tabcolsep=3pt
\begin{tabular}{p{388pt}rr}
&\textbf{Выпуск} & \textbf{Стр.}\\[3pt]
\hangindent=23pt\noindent\textbf{Козеренко~Е.\,Б.} Лингвистические фильтры в статистических моделях машинного\linebreak
\vspace*{-12pt}\\
\hspace*{23pt}перевода$\dotfill$&2&83\\
\hangindent=23pt\noindent\textbf{Козеренко~Е.\,Б., Кузнецов~И.\,П.} Когнитивно-лингвистические представления в 
систе-\linebreak
\vspace*{-12pt}\\
\hspace*{23pt}мах обработки текстов$\dotfill$&3&69\\
\textbf{Князь~В.\,А.} см.~Каратеев~С.\,Л.&&\\
\hangindent=23pt\noindent\textbf{Колесников~А.\,В., Солдатов~С.\,А.} Алгоритм координации для гибридной 
интеллектуальной системы решения сложной задачи оперативно-производственного\linebreak
\vspace*{-12pt}\\
\hspace*{23pt}планирования$\dotfill$&4&61\\
\hangindent=23pt\noindent\textbf{Коновалов~М.\,Г.} О планировании потоков в системах вычислительных 
ресурсов$\dotfill$&2&3\\
\textbf{Конушин~А.\,С.} см.~Конушин~В.\,С.&&\\
\hangindent=23pt\noindent\textbf{Конушин~В.\,С., Кривовязь~Г.\,Р., Конушин~А.\,С.} Алгоритм распознавания людей 
в видео-\linebreak
\vspace*{-12pt}\\
\hspace*{23pt}последовательности по одежде$\dotfill$&1&74\\
\textbf{Корепанов~Э.\, Р.} см.~Синицын~И.\,Н.&&\\
\textbf{Королев~В.\,Ю.} см.~Соколов~И.\,А.&&\\
\textbf{Королев~Р.\,А.} см.~Бенинг~В.\,Е.&&\\
\textbf{Коротышева~А.\,В.} см.~Зейфман~А.\,И.&&\\
\hangindent=23pt\noindent\textbf{Кривенко~М.\,П.} Непараметрическое оценивание элементов байесовского 
клас\-си-\linebreak
\vspace*{-12pt}\\
\hspace*{23pt}фикатора$\dotfill$&2&13\\
\textbf{Кривовязь~Г.\,Р.} см.~Конушин~В.\,С.&&\\
\textbf{Крылов~А.\,С.} см.~Павельева~Е.\,А.&&\\
\hangindent=23pt\noindent\textbf{Крылов~В.\,А.} Моделирование и классификация многоканальных дистанционных\linebreak
\vspace*{-12pt}\\
\hspace*{23pt}изображений с использованием копул$\dotfill$&4&34\\
\hangindent=23pt\noindent\textbf{Крючин~О.\,В.} Разработка параллельных эвристических алгоритмов подбора 
весовых\linebreak
\vspace*{-12pt}\\
\hspace*{23pt}коэффициентов искусственной нейтронной сети$\dotfill$&2&53\\
\hangindent=23pt\noindent\textbf{Кудрявцев~А.\,А., Шоргин~С.\,Я.} Байесовские модели массового обслуживания и 
надеж-\linebreak
\vspace*{-12pt}\\
\hspace*{23pt}ности: характеристики среднего числа заявок в системе $M\vert M \vert 1\vert 
\infty$$\dotfill$&3&16\\
\hangindent=23pt\noindent\textbf{Кузнецов~А.\,А.} Связь между временными и структурно-топологическими 
характери-\linebreak
\vspace*{-12pt}\\
\hspace*{23pt}стиками диаграмм ритма сердца здоровых людей$\dotfill$&4&39\\
\textbf{Кузнецов~И.\,П.} см.~Козеренко~Е.\,Б.&&\\
\textbf{Ле~Пезан~Д.} см.~Бунтман~Н.\,В.&&\\
\hangindent=23pt\noindent\textbf{Лукьяненко~А.\,С., Морозов~Е.\,В., Гуртов~А.\,В.} Анализ сетевого протокола с общей 
функ-\linebreak
\vspace*{-12pt}\\
\hspace*{23pt}цией расширения окна передачи сообщения при конфликтах$\dotfill$&2&46\\
\hangindent=23pt\noindent\textbf{Лямин~О.\,О.} О предельном поведении мощностей критериев в случае обобщенного\linebreak
\vspace*{-12pt}\\
\hspace*{23pt}распределения Лапласа$\dotfill$&3&47\\
\hangindent=23pt\noindent\textbf{Маркин~А.\,В., Шестаков~О.\,В.} Асимптотики оценки риска при пороговой 
обработке\linebreak
\vspace*{-12pt}\\
\hspace*{23pt}вейвлет-вейглет коэффициентов в задаче томографии$\dotfill$&2&36\\
\hangindent=23pt\noindent\textbf{Матвеева~С.\,С., Захарова~Т.\,В.} Сети массового обслуживания с наименьшей 
длиной\linebreak
\vspace*{-12pt}\\
\hspace*{23pt}очереди$\dotfill$&3&22\\
\hangindent=23pt\noindent\textbf{Матюшенко~С.\,И.} Стационарные характеристики двухканальной системы 
обслужива-\linebreak
\vspace*{-12pt}\\
\hspace*{23pt}ния с переупорядочиванием заявок и распределениями фазового типа$\dotfill$&4&68\\
\textbf{Минель~Ж.-Л.} см.~Бунтман~Н.\,В.&&\\
\textbf{Морозов~Е.\,В.} см.~Бородина~А.\,В.&&\\
\textbf{Морозов~Е.\,В.} см.~Лукьяненко~А.\,С.&&\\
\textbf{Ососков~М.\,В.} см.~Каратеев~С.\,Л.&&\\
\hangindent=23pt\noindent\textbf{Павельева~Е.\,А., Крылов~А.\,С.} Поиск и анализ ключевых точек радужной 
оболочки\linebreak
\vspace*{-12pt}\\
\hspace*{23pt}глаза методом преобразования Эрмита$\dotfill$&1&79\\
\textbf{Печинкин~А.\,В.} см.~Френкель~С.\,Л.,&&\\
\hangindent=23pt\noindent\textbf{Протасов~В.\,И.} Составление субъективного портрета с использованием 
эволюционно-\linebreak
\vspace*{-12pt}\\
\hspace*{23pt}го морфинга и квалиметрия метода$\dotfill$&1&83\\
\hangindent=23pt\noindent\textbf{Рудаков~К.\,В., Торшин~И.\,Ю.} Вопросы разрешимости задачи распознавания 
вторичной\linebreak
\vspace*{-12pt}\\
\hspace*{23pt}структуры белка$\dotfill$&2&25\\
\textbf{Сатин~Я.\,А.} см.~Зейфман~А.\,И.&&\\
\hangindent=23pt\noindent\textbf{Сейфуль-Мулюков~Р.\,Б.} Нефть как носитель информации о своем 
происхождении,\linebreak
\vspace*{-12pt}\\
\hspace*{23pt}структуре и эволюции$\dotfill$&1&41\\
\end{tabular}
}

{\tabcolsep=3pt
\begin{tabular}{p{388pt}rr}
&\textbf{Выпуск} & \textbf{Стр.}\\[6pt]
\textbf{Семендяев~Н.\,Н.} см.~Синицын~И.\,Н.&&\\
\textbf{Серебряков~В.\,А.} см.~Захаров~А.\,А.&&\\
\textbf{Синицын~В.\,И.} см.~Синицын~И.\,Н.&&\\
\hangindent=23pt\noindent\textbf{Синицын~И.\,Н., Синицын~В.\,И., Корепанов~Э.\, Р., Белоусов~В.\,В., 
Семендяев~Н.\,Н.} Оперативное построение информационных моделей движения полюса 
Земли\linebreak
\vspace*{-12pt}\\
\hspace*{23pt}методами линейных и линеаризованных фильтров$\dotfill$&1&2\\
\textbf{Сипина~А.\,В.} см.~Бенинг~В.\,Е.&&\\
\hangindent=23pt\noindent\textbf{Соколов~И.\,А.} О работах заслуженного деятеля науки Российской Федерации 
И.\,Н.~Синицына в области информационных технологий и автоматизации (к 70-летию\linebreak
\vspace*{-12pt}\\
\hspace*{23pt}со дня рождения)$\dotfill$&3&84\\
\textbf{Соколов~И.\,А.} см.~Илюшин~Г.\,Я.&&\\
\hangindent=23pt\noindent\textbf{Соколов~И.\,А., Королев~В.\,Ю.} Предисловие$\dotfill$&2&2\\
\textbf{Солдатов~С.\,А.} см.~Колесников~А.\,В.&&\\
\hangindent=23pt\noindent\textbf{Степанов~С.\,Ю.} Использование координатного метода фрагментации 
коммутаторной\linebreak
\vspace*{-12pt}\\
\hspace*{23pt}нейронной сети для сокращения трафика$\dotfill$&2&57\\
\textbf{Тимонина~Е.\,Е.} см.~Грушо~А.\,А.&&\\
\textbf{Торшин~И.\,Ю.} см.~Рудаков~К.\,В.&&\\
\textbf{Ульянов~В.\,В.} см.~Кавагучи~Ю.&&\\
\textbf{Фазекаш~И.} см.~Чупрунов~А.\,Н.&&\\
\textbf{Френкель~С.\,Л.} см.~Баранов~С.\,И.&&\\
\hangindent=23pt\noindent\textbf{Френкель~С.\,Л., Печинкин~А.\,В.} Оценка времени самовосстановления в 
цифровых\linebreak
\vspace*{-12pt}\\
\hspace*{23pt}системах после сбоев, вызываемых переходными помехами$\dotfill$&3&2\\
\textbf{Фуджикоши~Я.} см.~Кавагучи~Ю.&&\\
\hangindent=23pt\noindent\textbf{Цискаридзе~А.\,К.} Математическая модель и метод восстановления позы человека 
по\linebreak
\vspace*{-12pt}\\
\hspace*{23pt}стереопаре силуэтных изображений$\dotfill$&4&27\\
\hangindent=23pt\noindent\textbf{Чупраков~К.\,Г.} К вопросу о размещении коллективных средств отображения в 
ситуа-\linebreak
\vspace*{-12pt}\\
\hspace*{23pt}ционном зале с заданными параметрами$\dotfill$&4&89\\
\textbf{Чупраков~К.\,Г.} см.~Зацаринный~А.\,А.&&\\
\hangindent=23pt\noindent\textbf{Чупрунов~А.\,Н., Фазекаш~И.} Законы повторного логарифма для числа 
безошибочных\linebreak
\vspace*{-12pt}\\
\hspace*{23pt}блоков при помехоустойчивом кодировании$\dotfill$&3&42\\
\textbf{Шевцова~И.\,Г.} см.~Григорьева~М.\,Е.&&\\
\hangindent=23pt\noindent\textbf{Шестаков~О.\,В.} Аппроксимация распределения оценки риска пороговой 
обработки вейвлет-коэффициентов нормальным распределением при использовании 
выбо-\linebreak
\vspace*{-12pt}\\
\hspace*{23pt}рочной дисперсии$\dotfill$&4&73\\
\textbf{Шестаков~О.\,В.} см.~Маркин~А.\,В.&&\\
\textbf{Шоргин~С.\,Я.} см.~Зейфман~А.\,И.&&\\
\textbf{Шоргин~С.\,Я.} см.~Кудрявцев~А.\,А.&&\\
\end{tabular}
}

%\thispagestyle{myheadings}
\def\leftfootline{\small{\textbf{\thepage}
\hfill ИНФОРМАТИКА И ЕЁ ПРИМЕНЕНИЯ\ \ \ том~4\ \ \ выпуск~4\ \ \ 2010}
}%
 \def\rightfootline{\small{ИНФОРМАТИКА И ЕЁ ПРИМЕНЕНИЯ\ \ \ том~4\ \ \ выпуск~4\ \ \ 2010
 \hfill \textbf{\thepage}}}
 \label{end\stat}

%
%Том 10 Выпуск 1-4 Год 2016

\def\stat{cont-e}
{%\hrule\par
%\vskip 7pt % 7pt
\raggedleft\Large \bf%\baselineskip=3.2ex
2\,0\,1\,6\ \ A\,U\,T\,H\,O\,R\ \ I\,N\,D\,E\,X \vskip 17pt
 \hrule
 \par
\vskip 21pt plus 6pt minus 3pt }

\label{st\stat}

\def\tit{\ }

\def\aut{\ }
\def\auf{\ }

\def\leftkol{\ } %2016 AUTHOR INDEX} % ENGLISH ABSTRACTS}

\def\rightkol{\ } %2016 AUTHOR INDEX} %ENGLISH ABSTRACTS}

\titele{\tit}{\aut}{\auf}{\leftkol}{\rightkol}

\def\leftfootline{\small{\textbf{\thepage}
\hfill INFORMATIKA I EE PRIMENENIYA~--- INFORMATICS AND APPLICATIONS\ \ \ 2016\
\ \ volume~10\ \ \ issue\ 4}
}%
 \def\rightfootline{\small{INFORMATIKA I EE PRIMENENIYA~--- INFORMATICS AND APPLICATIONS\ \ \ 2016\ \ \ volume~10\ \ \ issue\ 4
\hfill \textbf{\thepage}}}

\vspace*{-12pt}
\vspace*{-18pt}

{\tabcolsep=2.8pt
\begin{tabular}{p{382pt}cc}
&\textbf{Issue} & \textbf{Page}\\[6pt]
\Avtors{Agalarov~M.\,Ya.} see~Agalarov~Ya.\,M.&&\\
\Avtors{Agalarov~Ya.\,M., Agalarov~M.\,Ya., and
Shorgin~V.\,S.} About the optimal threshold of queue\linebreak
\\[-12pt]
\hspace*{23pt}length in a~particular problem of profit maximization
in the $M/G/1$ queuing system&2&70--79\\
\Avtors{Alexeyevsky~D.\,A.} BioNLP ontology extraction from 
a~restricted language corpus with\linebreak
\\[-12pt]
\hspace*{23pt}context-free grammars&1&119--128\\
\Avtors{Andreev~S.\,D.} see~Gaidamaka~Yu.\,V.&&\\
\Avtors{Andreev~S.\,D.} see~Ometov~A.\,Ya.&&\\
\Avtors{Arkhipov~O.\,P., Arkhipov~P.\,O., and Sidorkin~I.\,I.} The
option to create a~local coordinate\linebreak
\\[-12pt]
\hspace*{23pt}system for synchronization of selected images&3&91--97\\
\Avtors{Arkhipov~P.\,O.} see~Arkhipov~O.\,P.&&\\
\Avtors{Belousov~V.\,V.} see~Shnurkov~P.\,V.&&\\
\Avtors{Belousov~V.\,V.} see~Shnurkov~P.\,V.&&\\
\Avtors{Bening~V.\,E.} Calculation of~the~asymptotic deficiency
of~some statistical procedures based\linebreak
\\[-12pt]
\hspace*{23pt}on~samples with~random sizes&4&34--45\\
\Avtors{Borisov~A.\,V., Bosov~A.\,V., and Miller~G.\,B.} Modeling and
monitoring of VoIP connection&2&\hphantom{1}2--13\\
\Avtors{Bosov~A.\,V.} see~Borisov~A.\,V.&&\\
\Avtors{Briukhov~D.\,O.} see~Stupnikov~S.\,A.&&\\
\Avtors{Callaos~N.\,K.\ and Seyful-Mulyukov~R.\,B.} Complexity and
its information content&1&129--139\\
\Avtors{Chertok~A.\,V., Kadaner~A.\,I., Khazeeva~G.\,T., and
Sokolov~I.\,A.} Regime switching detection\linebreak
\\[-12pt]
\hspace*{23pt}for~the~Levy driven
Ornstein--Uhlenbeck process using CUSUM methods&4&46--56\\
\Avtors{Chichagov~V.\,V.} Asymptotic expansions of mean absolute
error of uniformly minimum variance unbiased and maximum likelihood
estimators on the one-parameter exponential\linebreak
\\[-12pt]
\hspace*{23pt}family model of lattice distributions&3&66--76\\
\Avtors{Danishevsky~V.\,I.} see~Kolesnikov A.\,V.&&\\
\Avtors{Fazliev~A.\,Z.} see~Kalinichenko~L.\,A.&&\\
\Avtors{Fedoseev~A.\,A.} What is behind the concept of ``knowledge in
small packages''&3&105--110\\
\Avtors{Gaidamaka~Yu.\,V., Andreev~S.\,D., Sopin~E.\,S.,
Samouylov~K.\,E., and Shorgin~S.\,Ya.} Interference analysis
of~the~device-to-device communications model with~regard to~a~signal\linebreak
\\[-12pt]
\hspace*{23pt}propagation environment&4&\hphantom{1}2--10\\
\Avtors{Gasilov~A.\,V.} see~Yakovlev~O.\,A.&&\\
\Avtors{Goncharov~A.\,V.\ and Strijov~V.\,V.} Metric time series
classification using weighted dynamic\linebreak
\\[-12pt]
\hspace*{23pt}warping relative to centroids of classes&2&36--47\\
\Avtors{Gordov~E.\,P.} see~Kalinichenko~L.\,A.&&\\
\Avtors{Gorshenin~A.\,K.} Concept of online service for stochastic
modeling of real processes&1&72--81\\
\Avtors{Gorshenin~A.\,K.} see~Shnurkov~P.\,V.&&\\
\Avtors{Gorshenin~A.\,K.} see~Shnurkov~P.\,V.&&\\
\Avtors{Grusho~A.\,A., Grusho~N.\,A., Zabezhailo~M.\,I., and
Timonina~E.\,E.} Integration of statistical and\linebreak
\\[-12pt]
\hspace*{23pt}deterministic methods for
analysis of information security&3&2--8\\
\Avtors{Grusho~A.\,A., Zabezhailo~M.\,I., and Zatsarinny~A.\,A.} On
the advanced procedure to reduce\linebreak
\\[-12pt]
\hspace*{23pt}calculation of Galois closures&4&\hphantom{1}96--104\\
\Avtors{Grusho~N.\,A.} see~Grusho~A.\,A.&&\\
\Avtors{Havanskov~V.\,A.} see~Minin~V.\,A.&&\\
\Avtors{Inkova~O.\,Yu.} see~Zatsman~I.\,M.&&\\
\Avtors{Isachenko~R.\,V.\ and Strijov~V.\,V.} Metric learning in
multiclass time series classification\linebreak
\\[-12pt]
\hspace*{23pt}problem&2&48--57\\
\end{tabular}
}
\pagebreak

\def\leftfootline{\small{\textbf{\thepage}
\hfill INFORMATIKA I EE PRIMENENIYA~--- INFORMATICS AND APPLICATIONS\ \ \ 2016\
\ \ volume~10\ \ \ issue\ 4}
}%
 \def\rightfootline{\small{INFORMATIKA I EE PRIMENENIYA~---
INFORMATICS AND APPLICATIONS\ \ \ 2016\ \ \ volume~10\ \ \ issue\ 4
\hfill \textbf{\thepage}}}

\def\leftkol{2016 AUTHOR INDEX} % ENGLISH ABSTRACTS}

\def\rightkol{2016 AUTHOR INDEX} %ENGLISH ABSTRACTS}


{\tabcolsep=2.83pt
\begin{tabular}{p{382pt}cc}
&\textbf{Issue} & \textbf{Page}\\[6pt]
\Avtors{Kadaner~A.\,I.} see~Chertok~A.\,V.&&\\[.255pt]
\Avtors{Kalinichenko~L.\,A., Volnova~A.\,A., Gordov~E.\,P.,
Kiselyova~N.\,N., Kovaleva~D.\,A., Malkov~O.\,Yu., Okladnikov~I.\,G.,
Podkolodnyy~N.\,L., Pozanenko~A.\,S., Ponomareva~N.\,V.,
Stupnikov~S.\,A.,} \textbf{and Fazliev~A.\,Z.} Data access challenges for data
intensive\linebreak
\\[-12pt]
\hspace*{23pt}research in Russia&1& 2--22\\[.255pt]
\Avtors{Karasikov~M.\,E.\ and Strijov~V.\,V.} Feature-based
time-series classification&4&121--131\\[.255pt]
\Avtors{Khazeeva~G.\,T.} see~Chertok~A.\,V.&&\\[.255pt]
\Avtors{Khokhlov~Yu.\,S.} Multivariate fractional Levy motion and its
applications&2&\hphantom{1}98--106\\[.255pt]
\Avtors{Kirikov~I.\,A., Kolesnikov~A.\,V., Listopad~S.\,V., and
Rumovskaya~S.\,B.} Fine-grained hybrid\linebreak
\\[-12pt]
\hspace*{23pt}intelligent systems. Part 2:
Bidirectional hybridization&1&\hphantom{1}96--105\\[.255pt]
\Avtors{Kirikov~I.\,A., Kolesnikov~A.\,V., Listopad~S.\,V., and
Rumovskaya~S.\,B.} ``Virtual council''~---\linebreak
\\[-12pt]
\hspace*{23pt}source environment
supporting complex diagnostic decision making&3&81--90\\[.255pt]
\Avtors{Kiselyova~N.\,N.} see~Kalinichenko~L.\,A.&&\\[.255pt]
\Avtors{Kolesnikov A.\,V., Listopad~S.\,V., Rumovskaya~S.\,B., and
Danishevsky~V.\,I.} Informal axiomatic\linebreak
\\[-12pt]
\hspace*{23pt}theory of~the~role visual models&4&114--120\\[.255pt]
\Avtors{Kolesnikov~A.\,V.} see~Kirikov~I.\,A.&&\\[.255pt]
\Avtors{Kolesnikov~A.\,V.} see~Kirikov~I.\,A.&&\\[.255pt]
\Avtors{Kolin~K.\,K.} Humanitarian aspects of information
security&3&111--121\\[.255pt]
\Avtors{Konovalov~M.\,G.\ and Razumchik~R.\,V.} Dispatching
to~two parallel nonobservable queues using\linebreak
\\[-12pt]
\hspace*{23pt}only static
information&4&57--67\\[.255pt]
\Avtors{Korchagin~A.\,Yu.} see~Korolev~V.\,Yu.&&\\[.255pt]
\Avtors{Korchagin~A.\,Yu.} see~Korolev~V.\,Yu.&&\\[.255pt]
\Avtors{Korepanov~E.\,R.} see~Sinitsyn~I.\,N.&&\\[.255pt]
\Avtors{Korepanov~E.\,R.} see~Sinitsyn~I.\,N.&&\\[.255pt]
\Avtors{Korolev~V.\,Yu., Korchagin~A.\,Yu., and Zeifman~A.\,I.} The
Poisson theorem for Bernoulli trials\linebreak
\\[-12pt]
\hspace*{23pt}with~a~random probability
of~success and~a~discrete analog of~the~Weibull distribution&4&11--20\\[.255pt]
\Avtors{Korolev~V.\,Yu., Zeifman~A.\,I., and Korchagin~A.\,Yu.}
Asymmetric Linnik distributions as~limit\linebreak
\\[-12pt]
\hspace*{23pt}laws for~random sums
of~independent random variables with~finite variances&4&21--33\\[.255pt]
\Avtors{Koucheryavy~E.\,A.} see~Ometov~A.\,Ya.&&\\[.255pt]
\Avtors{Kovaleva~D.\,A.} see~Kalinichenko~L.\,A.&&\\[.255pt]
\Avtors{Kovalyov~S.\,P.} Metaprogramming to increase
manufacturability of large-scale software-\linebreak
\\[-12pt]
\hspace*{23pt}intensive systems&1&56--66\\[.255pt]
\Avtors{Krivenko~M.\,P.} Significance tests of feature selection for
classification&3&32--40\\[.255pt]
\Avtors{Kruzhkov~M.\,G.} see~Zalizniak~Anna~A.&&\\[.255pt]
\Avtors{Kruzhkov~M.\,G.} see~Zatsman~I.\,M.&&\\[.255pt]
\Avtors{Kudryavtsev~A.\,A.} Bayesian queueing and reliability models:
\textit{A~priori} distributions with\linebreak
\\[-12pt]
\hspace*{23pt}compact support&1&67--71\\[.255pt]
\Avtors{Kudryavtsev~A.\,A.} Characteristics dependent on the balance
coefficient in Bayesian models\linebreak
\\[-12pt]
\hspace*{23pt}with compact support of \textit{a priori}
distributions&3&77--80\\[.255pt]
\Avtors{Kudryavtsev~A.\,A.\ and Palionnaia~S.\,I.} Bayesian recurrent
model of reliability growth:\linebreak
\\[-12pt]
\hspace*{23pt}Parabolic distribution of parameters&2&80--83\\[.255pt]
\Avtors{Kudryavtsev~A.\,A.\ and Titova~A.\,I.} Bayesian queuing
and~reliability models: Degenerate-\linebreak
\\[-12pt]
\hspace*{23pt}Weibull case&4&68--71\\[.255pt]
\Avtors{Leontyev~N.\,D.\ and Ushakov~V.\,G.} Analysis of a queueing
system with autoregressive arrivals\linebreak
\\[-12pt]
\hspace*{23pt}and nonpreemptive priority&3&15--22\\[.255pt]
\Avtors{Listopad~S.\,V.} see~Kirikov~I.\,A.&&\\[.255pt]
\Avtors{Listopad~S.\,V.} see~Kirikov~I.\,A.&&\\[.255pt]
\Avtors{Listopad~S.\,V.} see~Kolesnikov A.\,V.&&\\[.255pt]
\Avtors{Malkov~O.\,Yu.} see~Kalinichenko~L.\,A.&&\\[.255pt]
\Avtors{Markov~A.\,S., Monakhov~M.\,M., and
Ulyanov~V.\,V.} Generalized Cornish--Fisher expansions\linebreak
\\[-12pt]
\hspace*{23pt}for distributions of statistics based on samples
of random size&2&84--91\\[.255pt]
\Avtors{Melnikov~A.\,K.\ and Ronzhin~A.\,F.} Generalized statistical
method of~text analysis based\linebreak
\\[-12pt]
\hspace*{23pt}on~calculation of~probability distributions
of~statistical values&4&89--95\\
\end{tabular}
}
\pagebreak

\def\leftfootline{\small{\textbf{\thepage}
\hfill INFORMATIKA I EE PRIMENENIYA~--- INFORMATICS AND APPLICATIONS\ \ \ 2016\
\ \ volume~10\ \ \ issue\ 4}
}%
 \def\rightfootline{\small{INFORMATIKA I EE PRIMENENIYA~---
INFORMATICS AND APPLICATIONS\ \ \ 2016\ \ \ volume~10\ \ \ issue\ 4
\hfill \textbf{\thepage}}}

\def\leftkol{2016 AUTHOR INDEX} % ENGLISH ABSTRACTS}

\def\rightkol{2016 AUTHOR INDEX} %ENGLISH ABSTRACTS}


{\tabcolsep=3pt
\begin{tabular}{p{381pt}cc}
&\textbf{Issue} & \textbf{Page}\\[6pt]
\Avtors{Meykhanadzhyan~L.\,A.} Stationary characteristics of the finite
capacity queueing system with\linebreak
\\[-12pt]
\hspace*{23pt}inverse service order and generalized
probabilistic priority&2&123--131\\[.23pt]
\Avtors{Miller~G.\,B.} see~Borisov~A.\,V.&&\\[.23pt]
\Avtors{Minin~V.\,A., Zatsman~I.\,M., Havanskov~V.\,A., and
Shubnikov~S.\,K.} Intensity of citation of scientific publications in
inventions on information and computer technologies patented\linebreak
\\[-12pt]
\hspace*{23pt}in Russia by domestic and foreign applicants&2&107--122\\[.23pt]
\Avtors{Monakhov~M.\,M.} see~Markov~A.\,S.&&\\[.23pt]
\Avtors{Naumov~V.\,A.\ and Samouylov~K.\,E.} On relationship
between queuing systems with resources\linebreak
\\[-12pt]
\hspace*{23pt}and Erlang networks&3&\hphantom{1}9--14\\[.23pt]
\Avtors{Okladnikov~I.\,G.} see~Kalinichenko~L.\,A.&&\\[.23pt]
\Avtors{Ometov~A.\,Ya., Andreev~S.\,D., Turlikov~A.\,M., and
Koucheryavy~E.\,A.} Performance analysis of\linebreak
\\[-12pt]
\hspace*{23pt}a wireless data
aggregation system with contention for contemporary sensor
networks&3&23--31\\[.23pt]
\Avtors{Palionnaia~S.\,I.} see~Kudryavtsev~A.\,A.&&\\[.23pt]
\Avtors{Podkolodnyy~N.\,L.} see~Kalinichenko~L.\,A.&&\\[.23pt]
\Avtors{Ponomareva~N.\,V.} see~Kalinichenko~L.\,A.&&\\[.23pt]
\Avtors{Popkova~N.\,A.} see~Zatsman~I.\,M.&&\\[.23pt]
\Avtors{Pozanenko~A.\,S.} see~Kalinichenko~L.\,A.&&\\[.23pt]
\Avtors{Razumchik~R.\,V.} see~Konovalov~M.\,G.&&\\[.23pt]
\Avtors{Ronzhin~A.\,F.} see~Melnikov~A.\,K.&&\\[.23pt]
\Avtors{Rumovskaya~S.\,B.} see~Kirikov~I.\,A.&&\\[.23pt]
\Avtors{Rumovskaya~S.\,B.} see~Kirikov~I.\,A.&&\\[.23pt]
\Avtors{Rumovskaya~S.\,B.} see~Kolesnikov A.\,V.&&\\[.23pt]
\Avtors{Samouylov~K.\,E.} see~Gaidamaka~Yu.\,V.&&\\[.23pt]
\Avtors{Samouylov~K.\,E.} see~Naumov~V.\,A.&&\\[.23pt]
\Avtors{Serebryanskii~S.\,M.} see~Tyrsin~A.\,N.&&\\[.23pt]
\Avtors{Seyful-Mulyukov~R.\,B.} see~Callaos~N.\,K.&&\\[.23pt]
\Avtors{Shestakov~O.\,V.} Statistical properties of the denoising method
based on the stabilized hard\linebreak
\\[-12pt]
\hspace*{23pt}thresholding&2&65--69\\[.23pt]
\Avtors{Shestakov~O.\,V.} The strong law of large numbers for the risk
estimate in the problem of\linebreak
\\[-12pt]
\hspace*{23pt}tomographic image reconstruction from
projections with a correlated noise&3&41--45\\[.23pt]
\Avtors{Shestakov~O.\,V.} see~Zakharova~T.\,V.&&\\[.23pt]
\Avtors{Shnurkov~P.\,V., Gorshenin~A.\,K., and Belousov~V.\,V.}
Analytical solution of~the~optimal control\linebreak
\\[-12pt]
\hspace*{23pt}task of~a~semi-Markov
process with~finite set of~states&4&72--88\\[.23pt]
\Avtors{Shnurkov~P.\,V., Zasypko~V.\,V., Belousov~V.\,V., and
Gorshenin~A.\,K.} Development of the algorithm of numerical solution
of the optimal investment control problem\linebreak
\\[-12pt]
\hspace*{23pt}in the closed dynamical model of three-sector economy&1&82--95\\[.23pt]
\Avtors{Shorgin~S.\,Ya.} see~Gaidamaka~Yu.\,V.&&\\[.23pt]
\Avtors{Shorgin~V.\,S.} see~Agalarov~Ya.\,M.&&\\[.23pt]
\Avtors{Shubnikov~S.\,K.} see~Minin~V.\,A.&&\\[.23pt]
\Avtors{Sidorkin~I.\,I.} see~Arkhipov~O.\,P.&&\\[.23pt]
\Avtors{Sinitsyn~I.\,N.} Analytical modeling of processes in stochastic
systems with complex fractional\linebreak
\\[-12pt]
\hspace*{23pt}order Bessel nonlinearities&3&55--65\\[.23pt]
\Avtors{Sinitsyn~I.\,N.} Orthogonal supoptimal filters for nonlinear
stochastic systems on manifolds&1&34--44\\[.23pt]
\Avtors{Sinitsyn~I.\,N.\ and Korepanov~E.\,R.} Normal Pugachev
conditionally-optimal filters and extra-\linebreak
\\[-12pt]
\hspace*{23pt}polators for state linear stochastic systems&2&14--23\\[.23pt]
\Avtors{Sinitsyn~I.\,N.\ and Sinitsyn~V.\,I.} Analytical modeling of
distributions in stochastic systems on\linebreak
\\[-12pt]
\hspace*{23pt}manifolds based on ellipsoidal approximation&1&45--55\\[.23pt]
\Avtors{Sinitsyn~I.\,N., Sinitsyn~V.\,I., and
Korepanov~E.\,R.} Ellipsoidal suboptimal filters for nonlinear\linebreak
\\[-12pt]
\hspace*{23pt}stochastic systems on manifolds&2&24--35\\[.23pt]
\Avtors{Sinitsyn~V.\,I.} see~Sinitsyn~I.\,N.&&\\[.23pt]
\Avtors{Sinitsyn~V.\,I.} see~Sinitsyn~I.\,N.&&\\[.23pt]
\Avtors{Skvortsov~N.\,A.} see~Stupnikov~S.\,A.&&\\[.23pt]
\Avtors{Sokolov~I.\,A.} see~Chertok~A.\,V.&&\\
\end{tabular}
}
\pagebreak

\def\leftfootline{\small{\textbf{\thepage}
\hfill INFORMATIKA I EE PRIMENENIYA~--- INFORMATICS AND APPLICATIONS\ \ \ 2016\
\ \ volume~10\ \ \ issue\ 4}
}%
 \def\rightfootline{\small{INFORMATIKA I EE PRIMENENIYA~---
INFORMATICS AND APPLICATIONS\ \ \ 2016\ \ \ volume~10\ \ \ issue\ 4
\hfill \textbf{\thepage}}}

\def\leftkol{2016 AUTHOR INDEX} % ENGLISH ABSTRACTS}

\def\rightkol{2016 AUTHOR INDEX} %ENGLISH ABSTRACTS}


{\tabcolsep=3pt
\begin{tabular}{p{382pt}cc}
&\textbf{Issue} & \textbf{Page}\\[6pt]
\Avtors{Sopin~E.\,S.} see~Gaidamaka~Yu.\,V.&&\\
\Avtors{Strijov~V.\,V.} see~Goncharov~A.\,V.&&\\
\Avtors{Strijov~V.\,V.} see~Isachenko~R.\,V.&&\\
\Avtors{Strijov~V.\,V.} see~Karasikov~M.\,E.&&\\
\Avtors{Stupnikov~S.\,A., Briukhov~D.\,O., and Skvortsov~N.\,A.}
Co-lending systemic risk analysis over\linebreak
\\[-12pt]
\hspace*{23pt}heterogeneous data collections&1&23--33\\
\Avtors{Stupnikov~S.\,A.} see~Kalinichenko~L.\,A.&&\\
\Avtors{Suchkov~A.\,P.} see~Zatsarinny~A.\,A.&&\\
\Avtors{Timonina~E.\,E.} see~Grusho~A.\,A.&&\\
\Avtors{Titova~A.\,I.} see~Kudryavtsev~A.\,A.&&\\
\Avtors{Turlikov~A.\,M.} see~Ometov~A.\,Ya.&&\\
\Avtors{Tyrsin~A.\,N.\ and Serebryanskii~S.\,M.} Recognition of
dependences on the basis of inverse\linebreak
\\[-12pt]
\hspace*{23pt}mapping&2&58--64\\
\Avtors{Ulyanov~V.\,V.} see~Markov~A.\,S.&&\\
\Avtors{Ushakov~V.\,G.} Queueing system with working vacations and
hyperexponential input stream&2&92--97\\
\Avtors{Ushakov~V.\,G.} see~Leontyev~N.\,D.&&\\
\Avtors{Volnova~A.\,A.} see~Kalinichenko~L.\,A.&&\\
\Avtors{Yakovlev~O.\,A.\ and Gasilov~A.\,V.} Speeded-up stereo
matching using geodesic support weights&3&\hphantom{1}98--104\\
\Avtors{Zabezhailo~M.\,I.} see~Grusho~A.\,A.&&\\
\Avtors{Zabezhailo~M.\,I.} see~Grusho~A.\,A.&&\\
\Avtors{Zakharova~T.\,V.\ and Shestakov~O.\,V.} Precision analysis of
wavelet processing of aerodynamic\linebreak
\\[-12pt]
\hspace*{23pt}flow patterns&3&46--54\\
\Avtors{Zalizniak~Anna~A.\ and Kruzhkov~M.\,G.} Database
of~Russian impersonal verbal constructions&4&132--141\\
\Avtors{Zasypko~V.\,V.} see~Shnurkov~P.\,V.&&\\
\Avtors{Zatsarinny~A.\,A.\ and Suchkov~A.\,P.} Systems engineering
approaches to~the~establishment of\linebreak
\\[-12pt]
\hspace*{23pt}a~system for~decision support based
on~situational analysis&4&105--113\\
\Avtors{Zatsarinny~A.\,A.} see~Grusho~A.\,A.&&\\
\Avtors{Zatsman~I.\,M., Inkova~O.\,Yu., Kruzhkov~M.\,G., and
Popkova~N.\,A.} Representation of cross-\linebreak
\\[-12pt]
\hspace*{23pt}lingual knowledge about
connectors in supracorpora databases&1&106--118\\
\Avtors{Zatsman~I.\,M.} see~Minin~V.\,A.&&\\
\Avtors{Zeifman~A.\,I.} see~Korolev~V.\,Yu.&&\\
\Avtors{Zeifman~A.\,I.} see~Korolev~V.\,Yu.&&\\
\end{tabular}
}

%\thispagestyle{myheadings}
\def\leftfootline{\small{\textbf{\thepage}
\hfill INFORMATIKA I EE PRIMENENIYA~--- INFORMATICS AND APPLICATIONS\ \ \ 2016\
\ \ volume~10\ \ \ issue\ 4}
}%
 \def\rightfootline{\small{INFORMATIKA I EE PRIMENENIYA~---
INFORMATICS AND APPLICATIONS\ \ \ 2016\ \ \ volume~10\ \ \ issue\ 4
\hfill \textbf{\thepage}}}

 \label{end\stat}

\newpage


\vspace*{-60pt} {\small
{\baselineskip=9.1pt
\section*{Правила подготовки рукописей статей для публикации в журнале
<<Информатика и её применения>>}

\thispagestyle{empty}

 Журнал <<Информатика и её применения>> публикует
теоретические, обзорные и дискуссионные статьи, посвященные научным
исследованиям и разработкам в области информатики и ее приложений. Журнал
издается на русском языке. По специальному решению редколлегии отдельные статьи,
в виде исключения, могут печататься на английском языке.
Тематика журнала охватывает следующие направления:
\begin{itemize}
\item теоретические основы информатики; %\\[-13.5pt]
\item математические методы исследования сложных систем и процессов; %\\[-13.5pt]
\item информационные системы и сети; %\\[-13.5pt]
\item информационные технологии; %\\[-13.5pt]
\item архитектура и программное
обеспечение вычислительных комплексов и сетей.
\end{itemize}
\begin{enumerate}
\item В журнале печатаются результаты, ранее не
опубликованные и не предназначенные к одновременной публикации в других
изданиях. Публикация не должна нарушать закон об авторских правах. Направляя
свою рукопись в редакцию, авторы автоматически передают учредителям и
редколлегии неисключительные права на издание данной статьи на русском языке и
на ее распространение в России и за рубежом. При этом за авторами сохраняются
все права как собственников данной рукописи. В связи с этим авторами должно
быть представлено в редакцию письмо в следующей форме:
Соглашение о передаче права на публикацию:

\textit{<<Мы, нижеподписавшиеся, авторы рукописи <<$\qquad\qquad$>>, передаем
учредителям и редколлегии журнала <<Информатика и её применения>>
неисключительное право опубликовать данную рукопись статьи на русском языке как
в печатной, так и в электронной версиях журнала. Мы подтверждаем, что данная
публикация не нарушает авторского права других лиц или организаций. Подписи
авторов: (ф.\,и.\,о., дата, адрес)>>.}

Указанное соглашение может быть представлено 
как в бумажном виде, так и в виде отсканированной копии (с подписями авторов).


Редколлегия вправе запросить у авторов экспертное заключение о возможности
опубликования представленной статьи в открытой печати. %\\[-13.5pt]
\item Статья
подписывается всеми авторами. На отдельном листе представляются данные автора
(или всех авторов): фамилия, полные имя и отчество, телефон, факс, e-mail,
почтовый адрес. Если работа выполнена несколькими авторами, указывается фамилия
одного из них, ответственного за переписку с редакцией. %\\[-13.5pt]
\item Редакция журнала
осуществляет самостоятельную экспертизу присланных статей. Возвращение рукописи
на доработку не означает, что статья уже принята к печати. Доработанный вариант
с ответом на замечания рецензента необходимо прислать в редакцию. %\\[-13.5pt]
\item Решение
редакционной коллегии о принятии статьи к печати или ее отклонении сообщается
авторам. Редколлегия не обязуется направлять рецензию авторам отклоненной
статьи. %\\[-13.5pt]
\item Корректура статей высылается авторам для просмотра. Редакция
просит авторов присылать свои замечания в кратчайшие сроки. %\\[-13.5pt]
\item При
подготовке рукописи в MS Word рекомендуется использовать следующие настройки.
Параметры страницы: формат~--- А4; ориентация~--- книжная; поля (см): внутри~---
2,5, снаружи~--- 1,5, сверху~--- 2, снизу~--- 2, от края до нижнего
колонтитула~--- 1,3. Основной текст: стиль~--- <<Обычный>>: шрифт Times New
Roman, размер 14~пунктов, абзацный отступ~--- 0,5~см, 1,5 интервала,
выравнивание~--- по ширине. Рекомендуемый объем рукописи~--- не свыше
25~страниц указанного формата. Ознакомиться с шаблонами, содержащими примеры
оформления, можно по адресу в Интернете:
\textsf{http://www.ipiran.ru/journal/template.doc}.
\item К рукописи, предоставляемой в 2-х
экземплярах, обязательно прилагается электронная версия статьи (как правило, в
форматах MS WORD (.doc) или \LaTeX\ (.tex), а также~--- дополнительно~--- в
формате .pdf) на дискете, лазерном диске или по электронной почте. Сокращения
слов, кроме стандартных, не применяются. Все страницы рукописи должны быть
пронумерованы. %\\[-13.5pt]
\item Статья должна содержать следующую информацию на русском и
английском языках: название, Ф.И.О. авторов, места работы авторов и их
электронные адреса, подробные сведения об авторах, оформленные в соответствии с форматом, 
определяемым файлами {\sf http://www.ipiran.ru/journal/issues/2011\_05\_01/authors.asp} и 
{\sf http://www.ipiran.ru/journal/issues/2011\_01\_eng/authors.asp},
аннотация (не более 100~слов), ключевые слова. Ссылки на
литературу в тексте статьи нумеруются (в квадратных скобках) и располагаются в
порядке их первого упоминания. В~списке литературы не должно быть позиций, на которые нет ссылки в тексте статьи.
Все фамилии авторов, заглавия статей, названия
книг, конференций и~т.\,п.\ даются на языке оригинала, если этот язык
использует кириллический или латинский алфавит. %\\[-13.5pt]
\item Присланные в редакцию материалы авторам не возвращаются.
\item При отправке файлов по электронной
почте просим придерживаться следующих правил:
\begin{itemize}
\item указывать в поле subject (тема) название журнала и фамилию автора; %\\[-13.5pt]
\item использовать attach (присоединение); %\\[-13.5pt]
\item в случае больших объемов информации возможно
использование общеизвестных архиваторов (ZIP, RAR); %\\[-13.5pt]
\item в состав электронной версии статьи должны входить: файл, содержащий текст статьи, и файл(ы),
содержащий(е) иллюстрации. %\\[-13.5pt]
\end{itemize}
\item Журнал <<Информатика и её применения>> является некоммерческим изданием. 
Плата за публикацию с авторов не взимается, гонорар авторам не выплачивается.
\end{enumerate}
\thispagestyle{empty}
\textbf{Адрес редакции:} Москва 119333,
ул.~Вавилова, д.~44, корп.~2, ИПИ РАН\\
\hphantom{\textbf{Адрес редакции:} }Тел.: +7 (499) 135-86-92\ \
Факс:  +7 (495) 930-45-05\ \  E-mail:   rust@ipiran.ru }
}

\end{document}


%\tableofcontents

%\end{document}





%\def\stat{cont}
{%\hrule\par
%\vskip 7pt % 7pt
\raggedleft\Large \bf%\baselineskip=3.2ex
А\,В\,Т\,О\,Р\,С\,К\,И\,Й\ \ У\,К\,А\,З\,А\,Т\,Е\,Л\,Ь\ \ З\,А\ \ 2\,0\,0\,7 г. \vskip 17pt
    \hrule
    \par
\vskip 21pt plus 6pt minus 3pt }

\label{st\stat}

\def\tit{\ }

\def\aut{\ }
\def\auf{\ }

\def\leftkol{\ } % ENGLISH ABSTRACTS}

\def\rightkol{\ } %ENGLISH ABSTRACTS}

\titele{\tit}{\aut}{\auf}{\leftkol}{\rightkol}


\contentsline {chapter}{\ }{Выпуск \quad Стр.} 
\contentsline {section}{\textbf{Батракова Д.\,А., Королев В.\,Ю., Шоргин С.\,Я.}\ \ Новый метод вероятностно-ста\-ти\-сти\-че\-ско\-го анализа информационных потоков в\nobreakspace {}телекоммуникационных сетях}{\qquad 1 \qquad 40} 
\contentsline {section}{\textbf{Борисов А.\,В.}\ \ Байесовское оценивание в системах наблюдения с\nobreakspace {}марковскими скачкообразными процессами: игровой подход}{\qquad 2 \qquad 65}
\contentsline {section}{\textbf{Босов А.\,В., Иванов А.\,В.}\ \ Программная инфраструктура информационного Web-пор\-тала}{\qquad 2 \qquad 50}
\contentsline {section}{\textbf{Захаров В.\,Н., Калиниченко Л.\,А., Соколов И.\,А., Ступников С.\,А.}\ \ Конструирование канонических информационных моделей для интегрированных информационных систем}{\qquad 2 \qquad 15}
\contentsline {section}{\textbf{Захаров В.\,Н., Козмидиади В.\,А.}\ \ Средства обеспечения отказоустойчивости при\-ло\-жений}{\qquad 1 \qquad 14} 
\contentsline {section}{\textbf{Иванов А.\,В.}\ \ см. Босов А.\,В.\hfill\hfill\hfill\hfill\hfill\hfill\hfill\hfill\hfill\hfill\hfill\hfill\hfill\hfill\hfill\hfill\hfill\hfill\hfill\hfill\hfill\hfill\hfill\hfill\hfill\hfill\hfill\hfill\hfill\hfill\hfill\hfill\hfill\hfill\hfill}{\ }
\contentsline {section}{\textbf{Ильин В.\,Д., Соколов И.\,А.}\ \ Символьная модель системы знаний информатики в\nobreakspace {}че\-ло\-ве\-ко-автоматной среде}{\qquad 1 \qquad 66} 
\contentsline {section}{\textbf{Калиниченко Л.\,А.}\ \ см. Захаров В.\,Н.\hfill\hfill\hfill\hfill\hfill\hfill\hfill\hfill\hfill\hfill\hfill\hfill\hfill\hfill\hfill\hfill\hfill\hfill\hfill\hfill\hfill\hfill\hfill\hfill\hfill\hfill\hfill\hfill\hfill\hfill\hfill\hfill\hfill\hfill\hfill}{\ }
\contentsline {section}{\textbf{Козеренко Е.\,Б.}\ \ Лингвистическое моделирование для систем машинного перевода и обработки знаний}{\qquad 1 \qquad 54} 
\contentsline {section}{\textbf{Козмидиади В.\,А.}\ \ см. Захаров В.\,Н.\hfill\hfill\hfill\hfill\hfill\hfill\hfill\hfill\hfill\hfill\hfill\hfill\hfill\hfill\hfill\hfill\hfill\hfill\hfill\hfill\hfill\hfill\hfill\hfill\hfill\hfill\hfill\hfill\hfill\hfill\hfill\hfill\hfill\hfill\hfill }{\ } 
\contentsline {section}{\textbf{Королев В.\,Ю.}\ \ см. Батракова Д.\,А.\hfill\hfill\hfill\hfill\hfill\hfill\hfill\hfill\hfill\hfill\hfill\hfill\hfill\hfill\hfill\hfill\hfill\hfill\hfill\hfill\hfill\hfill\hfill\hfill\hfill\hfill\hfill\hfill\hfill\hfill\hfill\hfill\hfill\hfill\hfill}{\ } 
\contentsline {section}{\textbf{Кудрявцев А.\,А., Шоргин С.\,Я.}\ \ Байесовский подход к\nobreakspace {}анализу систем массового обслуживания и\nobreakspace {}показателей надежности}{\qquad 2 \qquad 76}
\contentsline {section}{\textbf{Печинкин А.\,В., Соколов И.\,А., Чаплыгин В.\,В.}\ \ Многолинейная система массового обслуживания с конечным накопителем и ненадежными приборами}{\qquad 1 \qquad 27} 
\contentsline {section}{\textbf{Печинкин А.\,В., Соколов И.\,А., Чаплыгин В.\,В.}\ \ Стационарные характеристики многолинейной\nobreakspace {}системы массового обслуживания с\nobreakspace {}одновременными отказами приборов}{\qquad 2 \qquad 39}
\contentsline {section}{\textbf{Синицын И.\,Н.}\ \ Корреляционные методы построения аналитических информационных моделей флуктуаций полюса Земли по априорным данным}{\qquad 2 \qquad \hphantom{9}2}
\contentsline {section}{\textbf{Синицын И.\,Н.}\ \ Развитие теории фильтров Пугачева для оперативной обработки информации в стохастических системах}{{\qquad 1 \qquad \hphantom{9}3}} 
\contentsline {section}{\textbf{Соколов И.\,А.}\ \ см. Захаров В.\,Н.\hfill\hfill\hfill\hfill\hfill\hfill\hfill\hfill\hfill\hfill\hfill\hfill\hfill\hfill\hfill\hfill\hfill\hfill\hfill\hfill\hfill\hfill\hfill\hfill\hfill\hfill\hfill\hfill\hfill\hfill\hfill\hfill\hfill\hfill\hfill}{\ }
\contentsline {section}{\textbf{Соколов И.\,А.}\ \ см. Ильин В.\,Д.\hfill\hfill\hfill\hfill\hfill\hfill\hfill\hfill\hfill\hfill\hfill\hfill\hfill\hfill\hfill\hfill\hfill\hfill\hfill\hfill\hfill\hfill\hfill\hfill\hfill\hfill\hfill\hfill\hfill\hfill\hfill\hfill\hfill\hfill\hfill}{\ } 
\contentsline {section}{\textbf{Соколов И.\,А.}\ \ см. Печинкин А.\,В.\hfill\hfill\hfill\hfill\hfill\hfill\hfill\hfill\hfill\hfill\hfill\hfill\hfill\hfill\hfill\hfill\hfill\hfill\hfill\hfill\hfill\hfill\hfill\hfill\hfill\hfill\hfill\hfill\hfill\hfill\hfill\hfill\hfill\hfill\hfill}{\ } 
\contentsline {section}{\textbf{Соколов И.\,А.}\ \ см. Печинкин А.\,В.\hfill\hfill\hfill\hfill\hfill\hfill\hfill\hfill\hfill\hfill\hfill\hfill\hfill\hfill\hfill\hfill\hfill\hfill\hfill\hfill\hfill\hfill\hfill\hfill\hfill\hfill\hfill\hfill\hfill\hfill\hfill\hfill\hfill\hfill\hfill}{\ }
\contentsline {section}{\textbf{Ступников С.\,А.}\ \ см. Захаров В.\,Н.\hfill\hfill\hfill\hfill\hfill\hfill\hfill\hfill\hfill\hfill\hfill\hfill\hfill\hfill\hfill\hfill\hfill\hfill\hfill\hfill\hfill\hfill\hfill\hfill\hfill\hfill\hfill\hfill\hfill\hfill\hfill\hfill\hfill\hfill\hfill}{\ }
\contentsline {section}{\textbf{Чаплыгин В.\,В.}\ \ см. Печинкин А.\,В.\hfill\hfill\hfill\hfill\hfill\hfill\hfill\hfill\hfill\hfill\hfill\hfill\hfill\hfill\hfill\hfill\hfill\hfill\hfill\hfill\hfill\hfill\hfill\hfill\hfill\hfill\hfill\hfill\hfill\hfill\hfill\hfill\hfill\hfill\hfill}{\ } 
\contentsline {section}{\textbf{Чаплыгин В.\,В.}\ \ см. Печинкин А.\,В.\hfill\hfill\hfill\hfill\hfill\hfill\hfill\hfill\hfill\hfill\hfill\hfill\hfill\hfill\hfill\hfill\hfill\hfill\hfill\hfill\hfill\hfill\hfill\hfill\hfill\hfill\hfill\hfill\hfill\hfill\hfill\hfill\hfill\hfill\hfill}{\ }
\contentsline {section}{\textbf{Шоргин С.\,Я.}\ \ см. Батракова Д.\,А.\hfill\hfill\hfill\hfill\hfill\hfill\hfill\hfill\hfill\hfill\hfill\hfill\hfill\hfill\hfill\hfill\hfill\hfill\hfill\hfill\hfill\hfill\hfill\hfill\hfill\hfill\hfill\hfill\hfill\hfill\hfill\hfill\hfill\hfill\hfill}{\ } 
\contentsline {section}{\textbf{Шоргин С.\,Я.}\ \ см. Кудрявцев А.\,А.\hfill\hfill\hfill\hfill\hfill\hfill\hfill\hfill\hfill\hfill\hfill\hfill\hfill\hfill\hfill\hfill\hfill\hfill\hfill\hfill\hfill\hfill\hfill\hfill\hfill\hfill\hfill\hfill\hfill\hfill\hfill\hfill\hfill\hfill\hfill}{\ }
%\thispagestyle{myheadings}
\def\leftfootline{\small{\textbf{\thepage}
\hfill ИНФОРМАТИКА И ЕЁ ПРИМЕНЕНИЯ\ \ \ том~1\ \ \ выпуск~2\ \ \ 2007}
}%
 \def\rightfootline{\small{ИНФОРМАТИКА И ЕЁ ПРИМЕНЕНИЯ\ \ \ том~1\ \ \ выпуск~2\ \ \ 2007
 \hfill \textbf{\thepage}}}
 \label{end\stat}

%\def\stat{cont-e}
{%\hrule\par
%\vskip 7pt % 7pt
\raggedleft\Large \bf%\baselineskip=3.2ex
2\,0\,0\,7\ \ A\,U\,T\,H\,O\,R\ \ I\,N\,D\,E\,X \vskip 17pt
    \hrule
    \par
\vskip 21pt plus 6pt minus 3pt }

\label{st\stat}

\def\tit{\ }

\def\aut{\ }
\def\auf{\ }

\def\leftkol{\ } % ENGLISH ABSTRACTS}

\def\rightkol{\ } %ENGLISH ABSTRACTS}

\titele{\tit}{\aut}{\auf}{\leftkol}{\rightkol}


\contentsline {chapter}{\ }{Issue \quad Page} 
\contentsline {subsection}{\textbf{Batrakova D.\,A., Korolev V.\,Yu., Shorgin S.\,Ya.}\ \ A New Method for the Probabilistic and Statistical Analysis of Information Flows in Telecommunication Networks}{\qquad 1 \qquad 40} 
\contentsline {subsection}{\textbf{Borisov A.\,V.}\ \ Bayesian Estimation in\nobreakspace {}Observation Systems with\nobreakspace {}Markov Jump Processes: Game-Theoretic Approach}{\qquad 2 \qquad 65} 
\contentsline {subsection}{\textbf{Bosov A.\,V., Ivanov A.\,V.}\ \ Linguistic Simulation for Machine Translation and Knowledge Management Systems}{\qquad 2 \qquad 50} 
\contentsline {subsection}{\textbf{Chaplygin V.\,V.} see Pechinkin A.\,V.\hfill\hfill\hfill\hfill\hfill\hfill\hfill\hfill\hfill\hfill\hfill\hfill\hfill\hfill\hfill\hfill\hfill\hfill\hfill\hfill\hfill\hfill\hfill\hfill\hfill\hfill\hfill\hfill\hfill\hfill\hfill\hfill\hfill\hfill\hfill}{\ }
\contentsline {subsection}{\textbf{Chaplygin V.\,V.} see Pechinkin A.\,V.\hfill\hfill\hfill\hfill\hfill\hfill\hfill\hfill\hfill\hfill\hfill\hfill\hfill\hfill\hfill\hfill\hfill\hfill\hfill\hfill\hfill\hfill\hfill\hfill\hfill\hfill\hfill\hfill\hfill\hfill\hfill\hfill\hfill\hfill\hfill}{\ }
\contentsline {subsection}{\textbf{Ilyin V.\,D., Sokolov I.\,A.}\ \ The Symbol Model of Informatics Knowledge System in Human-Automaton Environment}{\qquad 1 \qquad 66} 
\contentsline {subsection}{\textbf{Ivanov A.\,V.} see Bosov A.\,V.\hfill\hfill\hfill\hfill\hfill\hfill\hfill\hfill\hfill\hfill\hfill\hfill\hfill\hfill\hfill\hfill\hfill\hfill\hfill\hfill\hfill\hfill\hfill\hfill\hfill\hfill\hfill\hfill\hfill\hfill\hfill\hfill\hfill\hfill\hfill}{\ }
\contentsline {subsection}{\textbf{Kalinichenko L.\,A.} see Zakharov V.\,N.\hfill\hfill\hfill\hfill\hfill\hfill\hfill\hfill\hfill\hfill\hfill\hfill\hfill\hfill\hfill\hfill\hfill\hfill\hfill\hfill\hfill\hfill\hfill\hfill\hfill\hfill\hfill\hfill\hfill\hfill\hfill\hfill\hfill\hfill\hfill}{\ }
\contentsline {subsection}{\textbf{Korolev V.\,Yu.} see Batrakova D.\,A.\hfill\hfill\hfill\hfill\hfill\hfill\hfill\hfill\hfill\hfill\hfill\hfill\hfill\hfill\hfill\hfill\hfill\hfill\hfill\hfill\hfill\hfill\hfill\hfill\hfill\hfill\hfill\hfill\hfill\hfill\hfill\hfill\hfill\hfill\hfill}{\ }
\contentsline {subsection}{\textbf{Kozerenko E.\,B.}\ \ Linguistic Simulation for Machine Translation and Knowledge Management Systems}{\qquad 1 \qquad 54} 
\contentsline {subsection}{\textbf{Kozmidiady V.\,A.} see Zakharov V.\,N.\hfill\hfill\hfill\hfill\hfill\hfill\hfill\hfill\hfill\hfill\hfill\hfill\hfill\hfill\hfill\hfill\hfill\hfill\hfill\hfill\hfill\hfill\hfill\hfill\hfill\hfill\hfill\hfill\hfill\hfill\hfill\hfill\hfill\hfill\hfill}{\ }
\contentsline {subsection}{\textbf{Kudryavtsev A.\,A., Shorgin S.\,Ya.}\ \ Bayesian Approach to Queueing Systems and Reliability Characteristics}{\qquad 2 \qquad 76} 
\contentsline {subsection}{\textbf{Pechinkin A.\,V., Sokolov I.\,A., Chaplygin V.\,V.}\ \ Multichannel Queuing System with Finite Buffer and Unreliable Servers}{\qquad 1 \qquad 27} 
\contentsline {subsection}{\textbf{Pechinkin A.\,V., Sokolov I.\,A., Chaplygin V.\,V.}\ \ Stationary Characteristics of a Multichannel Queueing System with\nobreakspace {}Simultaneous Refusals of Servers}{\qquad 2 \qquad 39} 
\contentsline {subsection}{\textbf{Shorgin S.\,Ya.} see Batrakova D.\,A.\hfill\hfill\hfill\hfill\hfill\hfill\hfill\hfill\hfill\hfill\hfill\hfill\hfill\hfill\hfill\hfill\hfill\hfill\hfill\hfill\hfill\hfill\hfill\hfill\hfill\hfill\hfill\hfill\hfill\hfill\hfill\hfill\hfill\hfill\hfill}{\ }
\contentsline {subsection}{\textbf{Shorgin S.\,Ya.} see Kudryavtsev A.\,A.\hfill\hfill\hfill\hfill\hfill\hfill\hfill\hfill\hfill\hfill\hfill\hfill\hfill\hfill\hfill\hfill\hfill\hfill\hfill\hfill\hfill\hfill\hfill\hfill\hfill\hfill\hfill\hfill\hfill\hfill\hfill\hfill\hfill\hfill\hfill}{\ }
\contentsline {subsection}{\textbf{Sinitsyn I.\,N.}\ \ Correlational Methods for Analytical Informational Models of the Earth Pole Fluctuations Design Based on a priori Data}{\qquad 2 \qquad \hphantom{9}2}
\contentsline {subsection}{\textbf{Sinitsyn I.\,N.}\ \ Development of Pugachev Filtering for Stochastic Systems}{\qquad 1 \qquad \hphantom{9}3}
\contentsline {subsection}{\textbf{Sokolov I.\,A.} see Ilyin V.\,D.\hfill\hfill\hfill\hfill\hfill\hfill\hfill\hfill\hfill\hfill\hfill\hfill\hfill\hfill\hfill\hfill\hfill\hfill\hfill\hfill\hfill\hfill\hfill\hfill\hfill\hfill\hfill\hfill\hfill\hfill\hfill\hfill\hfill\hfill\hfill}{\ }
\contentsline {subsection}{\textbf{Sokolov I.\,A.} see Pechinkin A.\,V.\hfill\hfill\hfill\hfill\hfill\hfill\hfill\hfill\hfill\hfill\hfill\hfill\hfill\hfill\hfill\hfill\hfill\hfill\hfill\hfill\hfill\hfill\hfill\hfill\hfill\hfill\hfill\hfill\hfill\hfill\hfill\hfill\hfill\hfill\hfill}{\ }
\contentsline {subsection}{\textbf{Sokolov I.\,A.} see Pechinkin A.\,V.\hfill\hfill\hfill\hfill\hfill\hfill\hfill\hfill\hfill\hfill\hfill\hfill\hfill\hfill\hfill\hfill\hfill\hfill\hfill\hfill\hfill\hfill\hfill\hfill\hfill\hfill\hfill\hfill\hfill\hfill\hfill\hfill\hfill\hfill\hfill}{\ }
\contentsline {subsection}{\textbf{Sokolov I.\,A.} see Zakharov V.\,N.\hfill\hfill\hfill\hfill\hfill\hfill\hfill\hfill\hfill\hfill\hfill\hfill\hfill\hfill\hfill\hfill\hfill\hfill\hfill\hfill\hfill\hfill\hfill\hfill\hfill\hfill\hfill\hfill\hfill\hfill\hfill\hfill\hfill\hfill\hfill}{\ }
\contentsline {subsection}{\textbf{Stupnikov S.\,A.} see Zakharov V.\,N.\hfill\hfill\hfill\hfill\hfill\hfill\hfill\hfill\hfill\hfill\hfill\hfill\hfill\hfill\hfill\hfill\hfill\hfill\hfill\hfill\hfill\hfill\hfill\hfill\hfill\hfill\hfill\hfill\hfill\hfill\hfill\hfill\hfill\hfill\hfill}{\ }
\contentsline {subsection}{\textbf{Zakharov V.\,N., Kalinichenko L.\,A., Sokolov I.\,A., Stupnikov S.\,A.}\ \ Development of Canonical Information Models for Integrated Information Systems}{\qquad 2 \qquad 15} 
\contentsline {subsection}{\textbf{Zakharov V.\,N., Kozmidiady V.\,A.}\ \ Means Providing Applications Fault Tolerance}{\qquad 1 \qquad 14} 
\def\leftfootline{\small{\textbf{\thepage}
\hfill ИНФОРМАТИКА И ЕЁ ПРИМЕНЕНИЯ\ \ \ том~1\ \ \ выпуск~2\ \ \ 2007}
}%
 \def\rightfootline{\small{ИНФОРМАТИКА И ЕЁ ПРИМЕНЕНИЯ\ \ \ том~1\ \ \ выпуск~2\ \ \ 2007
 \hfill \textbf{\thepage}}}
 \label{end\stat}


%\tableofcontents


\end{document}