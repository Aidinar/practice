%\newcommand{\om}{\omega}


\def\stat{art}

\def\tit{НЕОДНОРОДНЫЕ РЕКУРРЕНТНЫЕ МОДЕЛИ ИЗМЕНЕНИЯ НАДЕЖНОСТИ МОДИФИЦИРУЕМЫХ СИСТЕМ.
НЕПРЕРЫВНОЕ~ВРЕМЯ$^*$}
\def\titkol{Неоднородные рекуррентные модели изменения надежности модифицируемых систем.
Непрерывное время}

\def\autkol{С.\,В.~Артюхов, В.\,Ю. Королёв}
\def\aut{С.\,В.~Артюхов$^1$, В.\,Ю. Королёв$^2$}

\titel{\tit}{\aut}{\autkol}{\titkol}

{\renewcommand{\thefootnote}{\fnsymbol{footnote}}\footnotetext[1]
{Работа выполнена при поддержке
Российского фонда фундаментальных исследований,
гранты 08--01--00345, 08--01--00363, 08--07--00152.}}

\renewcommand{\thefootnote}{\arabic{footnote}}
\footnotetext[1]{ООО <<Внешпромбанк>>, ArtyuhovSV@yandex.ru}
\footnotetext[2]{Московский
государственный университет, факультет ВМиК; Институт проблем
информатики Российской академии наук, vkorolev@comtv.ru}

\Abst{В статье с помощью оценок скорости сходимости в
предельных теоремах для обобщенных дважды стохастических
пуассоновских процессов (обобщенных процессов Кокса) уточняется
асимптотическое поведение надежности сложных модифицируемых
технических и информационных систем в рамках неоднородных
рекуррентных моделей изменения надежности с непрерывным временем.}

\KW{обобщенный дважды стохастический
пуассоновский процесс; обобщенный процесс Кокса; модель изменения
надежности; коэффициент готовности; гарантированные доверительные
границы}

      \vskip 36pt plus 9pt minus 6pt

      \thispagestyle{headings}

      \begin{multicols}{2}

      \label{st\stat}

\section{Введение}

Одним из факторов, оказывающих существенное воздействие на
качество функционирования технической или информационной системы
(ТИС), является модификация отдельных компонентов (узлов) системы.
Такая модификация может быть обусловлена как необходимостью замены
вышедшего из строя узла, так и желанием улучшить характеристики
функционирования системы за счет замены узла устаревшего образца
более современным (более совершенным или надежным). Учет этого
фактора при анализе надежности сложных ТИС очень важен, поскольку во многих случаях
после ремонта или замены отказавшего элемента ТИС необходимо
отказаться от предположения о том, что после ремонта вся система
останется такой же, как и до ремонта. Только в этом случае можно
получить оценки надежности ТИС, приемлемые по точ\-ности для
практических нужд службы технической эксплуатации ТИС. Однако при
отказе от предположения об идентичности системы до и после ее
модернизации становятся неприемлемыми традиционные методы анализа
характеристик надежности восстанавливаемых систем (см.\
ГОСТ~27.002--89). Поэтому целью данной работы является описание метода
построения оценок показателей надежности ТИС с помощью
математических моделей, учитывающих возможное изменение
характеристик ТИС после каждой модификации.
{\looseness=1

}

%\end{multicols}
%\end{document}

При изучении надежностных характеристик модифицируемых систем
возможны как минимум два подхода. Первый из них заключается в том,
что параметры распределения времени безотказной работы считаются
функциями номера модификации (отказа, восстановления), что
приводит к моделям с дискретным временем (дискретным моделям
изменения надежности). В рамках второго подхода параметры
распределения времени безотказной работы считаются функциями
астрономического времени, что приводит к моделям с непрерывным
временем (непрерывным моделям изменения надежности).

Результаты данной работы дополняют основы математической теории
роста надежности модифицируемых систем, изложенные в~[1--3].
%(Gnedenko and Korolev, 1996), (Королев, 1997), (Королев и Соколов, 2006).
В последней из упомянутых книг был введен класс так на\-зы\-ва\-емых
{\it рекуррентных} моделей изменения надежности модифицируемых
систем (также см.\ гл.~2 в~\cite{4art}) %(Бенинг, Королев, Соколов и Шоргин, 2007))
и изучены некоторые аналитические свойства этих моделей. В
настоящей работе эти результаты будут обобщены и уточнены.

При анализе надежности ТИС довольно часто вместо такой характеристики 
надежности, как время безотказной работы системы, удобно иметь дело 
непосредственно с интегральным параметром, интерпретируемым как текущая 
надежность системы. В качестве такого интегрального показателя надежности ТИС в 
данной работе будет рассматриваться нестационарный аналог коэффициента 
го\-тов\-ности. Рассмотрим произвольную систему, на вход которой подаются 
некоторые сигналы (например, команды оператора или внешние воздействия). 
Реакция системы на поданные сигналы может быть либо правильной (корректной), 
либо неправильной (некорректной). В каждый момент времени $t$ надежность 
системы можно характеризовать параметром $P(t)$~--- вероятностью того, что на 
сигнал, поданный на вход системы в момент $t$, сис\-те\-ма отреагирует правильно. 
По смыслу такая характеристика надежности ближе всего к традиционно 
используемому {\it коэффициенту готовности}. В случайные моменты времени 
$Y_0=0\le Y_1\le Y_2\le\ldots$ сис\-те\-ма подвергается (мгновенной) модификации, в 
результате чего изменяется параметр $P(t)$. Предположим, что траектории 
процесса $P(t)$ непрерывны справа и кусочно-постоянны, так что $P(t)=P(Y_j)$ 
при $t\in [Y_j,\,Y_{j+1})$, $j\ge1$.

Задача прогнозирования поведения процесса~$P(t)$ чрезвычайно
важна. Описанная выше очень общая схема может быть
переформулирована в терминах, традиционных для столь разных
областей знания, как медицина, программирование или менеджмент.
Например, в программировании параметр $P(t)$ можно рассматривать
как надежность программного обеспечения, в которое по ходу отладки
в моменты $Y_0=0\le Y_1\le Y_2\le\ldots$ вносятся изменения для
исправления замеченных ошибок. Оценивание $P(t)$ и прогнозирование
поведения этого параметра здесь важно как для оценивания
надежности всего комплекса, составной частью которого является
программное обеспечение, так и для прогнозирования
продолжительности отладки (более подробно об этом см.\ в книгах~[1--3]).
В медицине параметр $1-P(t)$ (называемый индексом
летальности) характеризует вероятность летального исхода в момент
времени $t$ для пациента, организм которого в моменты $Y_0=0\le
Y_1\le Y_2\le\ldots$ подвергается какому-либо медицинскому
вмешательству (операции, инъекции, приему лекарств и~т.\,п.). Здесь
прогнозирование $P(t)$ чрезвычайно важно с точки зрения принятия
решений о стратегии лечения. Наконец, в менеджменте параметр
$P(t)$ может характеризовать надежность и дееспособность
коллектива, организации или предприятия, структура которых в
моменты времени $Y_0=0\le Y_1\le Y_2\le\ldots$ претерпевает
изменения. Будем предполагать, что в результате каждой модификации
системы параметр $P(t)$ изменяется случайным (не\-пред\-ска\-зу\-емым)
образом.

В данной статье будет рассмотрена зависимость надежности
модифицируемой системы не от номера модификации, а от реального
времени. Такие задачи представляют собой, пожалуй, больший
практический интерес, нежели задачи с дискретным временем. Однако
в таком случае необходимо сделать некоторые предположения о
распределении случайных моментов времени
$\{Y_0=0,Y_1,Y_2,\ldots\}$, в которые осуществляются модификации
системы. Для начала предположим, что последовательность
$\{Y_0=0,Y_1,Y_2,\ldots\}$ представляет собой однородный
пуассоновский точечный процесс с некоторой интенсивностью
$\lambda>0$. Такое предположение типично, если считать, что
моменты модификаций абсолютно хаотично рассредоточены на временной
оси (хорошо известно, что совместное условное распределение точек
пуассоновского потока на некотором интервале $[a,\,b]$ при условии,
что на этом интервале осуществилось ровно $n$ событий потока,
совпадает с распределением вариационного ряда, построенного по
выборке объема $n$ из равномерного на $[a,\,b]$ распределения).
Более того, случайные длины интервалов времени между точками
скачков пуассоновского процесса независимы и имеют одно и то же
экспоненциальное распределение, которое, как известно, обладает
максимальной дифференциальной энтропией среди всех распределений,
сосредоточенных на неотрицательной полуоси и имеющих конечное
математическое ожидание. Дифференциальная же энтропия является
вполне адекватной мерой неопределенности. Таким образом,
пуассоновский процесс в наибольшей мере соответствует общепринятым
представлениям о наиболее непредсказуемом, хаотическом размещении
точек на вещественной прямой. Однако однородный пуассоновский
процесс характеризуется постоянной интенсивностью~--- средним
количеством случайных точек, попавших в интервал времени единичной
длины.

Однако на практике существенно чаще встречаются ситуации, в
которых интенсивность процесса модификации системы непостоянна.

Предположим, что точки $\{Y_0=0,Y_1,Y_2,\ldots\}$, в которые
осуществляются модификации системы, образуют дважды стохастический
пуассоновский точечный процесс (иначе называемый процессом Кокса),
управляемый некоторым процессом $\Lambda(t)$. А именно, если
$
N(t)=\max\{j:\ Y_j\le t\}$, $t\ge0$~--- общее
число модификаций до момента $t$, то в рассматриваемом случае
$$
N(t)=N_1(\Lambda(t))\,,
$$
где $N_1$~--- стандартный пуассоновский
процесс (однородный пуассоновский процесс с единичной
интен\-сив\-ностью), независимый от процесса $\Lambda(t)$, траектории
которого выходят из нуля, не убывают, %\linebreak
непрерывны справа и почти
наверное конечны. Всюду в дальнейшем будет предполагаться, что
${\sf E}\Lambda(t)\equiv t$. Это предположение можно
интерпретировать и как то, что управляющий процесс в среднем
пропорционален времени, и (что существенно важно для построения
предельных аппроксимаций) как то, что задача параметризована
математическим ожиданием управляющего процесса.

Заметим, что в используемых обозначениях
$$
P(Y_{j+1}-0)=P(Y_j)\,,\ \ \ j\ge1\,,
$$
и
\begin{equation}
P(t)=P(Y_{N(t)})\,,\ \ \ t>0\,.
\label{e1art}
\end{equation}

\section{Неоднородные экспоненциальные модели с~непрерывным временем}

Пусть $\{(\theta_j,\eta_j)\}_{j\ge1}$~--- последовательность
независимых одинаково распределенных двумерных случайных векторов
таких, что
$$
0\le\theta_1\le 1\,,\ \ 0\le\eta_1\le 1\ \ {\mbox{почти \ наверное}}\,.
$$
Отметим, что независимость $\theta_j$ и $\eta_j$ внутри каждой
пары, равно как и совпадение распределений $\theta_j$ и $\eta_j$
внутри каждой пары, не предполагается. Однако будем считать, что
последовательность пар $\{(\theta_j,\eta_j)\}_{j\ge1}$
стохастически независима от точечного случайного процесса
$Y_1,Y_2,\ldots$

Задав начальную надежность $p_0$, рассмотрим модель, определяемую
рекуррентным соотноше\-нием
\begin{equation}
P(Y_{j+1})=\theta_{j+1}P(Y_j)+\eta_{j+1}\big(1-P(Y_j)\big)\,,\ \ j\ge0\,.
\label{e2art}
\end{equation}
Эту модель назовем неоднородной {\it экспоненциальной} моделью с
непрерывным временем. В такой модели случайные величины $\theta_j$
описывают возможное уменьшение надежности из-за некачественных
модификаций, в ходе которых вместо исправления существующих
дефектов в систему могут быть внесены новые, в то время как
величины $\eta_j$ описывают повышение надежности за счет
исправления дефектов. Частные случаи модели~(\ref{e2art}) с двухточечными
распределениями случайных величин $\theta_j$ и $\eta_j$ и
дискретным временем рассматривались в~\cite{5art, 6art}. В свою
очередь, эти частные случаи представляют собой переформулировку в
терминах теории надежности одной модели обучаемости, рассмотренной
в~\cite{7art}. Таким образом, можно пополнить список различных
приложений моделей, рассматриваемых в данной статье, еще и теорией
обучаемости.

Обозначим ${\sf E}\theta_1=1-a$, ${\sf E}\eta_1=b$.

\smallskip

\noindent
\textbf{Теорема 1.} {\it Для любого $t>0$}
$$
{\sf E}P(t)=\fr{b}{a+b}+\Big(p_0-\fr{b}{a+b}\Big){\sf
E}e^{-(a+b)\Lambda(t)}.
$$

\smallskip

\noindent
Д\,о\,к\,а\,з\,а\,т\,е\,л\,ь\,с\,т\,в\,о.\ $\,$ Сначала заметим, что
\begin{multline}
{\sf E}P(Y_j)=\fr{b}{a+b}+{}\\
{}+\left (p_0-\fr{b}{a+b}\right )
(1-a-b)^j\,,\ \ j\ge1\,.
\label{e3art}
\end{multline}
Действительно, взяв математические ожидания от обеих частей
соотношения~(\ref{e2art}) c учетом независимости векторов
$(\theta_j,\eta_j)$, получим
\begin{equation}
{\sf E}P(Y_{j+1})={\sf E}P(Y_j)(1-a-b)+b\,.
\label{e4art}
\end{equation}
Разрешая рекурсию~(\ref{e4art}), получаем соотношение~(\ref{e3art}).

Теперь, воспользовавшись соотношениями~(\ref{e1art}) и~(\ref{e3art}),
по формуле полной вероятности получим
\begin{multline*}
{\sf E}P(t)=\sum_{j=0}^{\infty}{\sf E}\big[P(t)|N(t)=j\big]{\sf
P}\big(N(t)=j\big)={}\\
{}=\sum_{j=0}^{\infty}{\sf E}P(Y_j){\sf
P}\big(N(t)=j\big)={}\\
{}=\sum_{j=0}^{\infty}{\sf
E}P(Y_j)\int\limits_{0}^{\infty}e^{-\lambda}\fr{\lambda^j}{j!}\,d{\sf
P}\big(\Lambda(t)<\lambda)=
{}\\
{}=
\int\limits_{0}^{\infty}e^{-\lambda}\sum_{j=0}^{\infty}\fr{\lambda^j}{j!}\Big[
\fr{b}{a+b}+{}\\
{}+ \Big(p_0-\fr{b}{a+b}\Big)
(1-a-b)^j\Big]\,d{\sf P}\big(\Lambda(t)<\lambda)=
{}\\
{}
=\fr{b}{a+b}+\Big(p_0-\fr{b}{a+b}\Big)\times{}\\
{}\times \int\limits_{0}^{\infty}\bigg(e^{-\lambda}
\sum_{j=0}^{\infty}\fr{[\lambda(1-a-b)]^j}{j!}\bigg)\,d{\sf
P}\big(\Lambda(t)<\lambda)=
{}\\
{}
=\fr{b}{a+b}+\Big(p_0-\fr{b}{a+b}\Big)\int\limits_{0}^{\infty}e^{-(a+b)\lambda}\,d{\sf
P}\big(\Lambda(t)<\lambda)={}\\
{}
=\fr{b}{a+b}+\Big(p_0-\fr{b}{a+b}\Big){\sf
E}e^{-(a+b)\Lambda(t)}\,.
\end{multline*}
Теорема доказана.

\smallskip

Из теоремы~1 вытекает, что если ${\sf E}e^{-(a+b)\Lambda(t)}\to 0$
при $t\to\infty$ (что может иметь место, если, к примеру,
$\Lambda(t)\to\infty$ по вероятности при $t\to\infty$), то в
зависимости от знака величины $c=(a+b)p_0-b$ ожидаемая надежность
системы либо возрастает (если $c<0$), либо убывает (если $c>0$).
Однако в любом случае справедливо

\smallskip

\noindent
{\textbf{Следствие 1.}} {\it Пусть $\Lambda(t)\to\infty$ по вероятности
при $t\to\infty$. Тогда}
$$
\lim_{t\to\infty}{\sf E}P(t)=\fr{b}{a+b}\,.
$$

\smallskip

В силу неотрицательности и ограниченности случайных величин
$\theta_j$ из следствия~1, в свою очередь, вытекает

\smallskip

\noindent
{\textbf{Следствие 2.}} {\it Пусть $\Lambda(t)\to\infty$ по вероятности
при $t\to\infty$. Соотношение
$$
\lim_{t\to\infty}{\sf E}P(t)=1
$$
имеет место в том и только в том случае, когда ${\sf
P}(\theta_1=1)$.}

\smallskip

Другими словами, в рамках экспоненциальной модели абсолютная
надежность может быть достигнута только за счет идеально
правильных модификаций, в ходе которых полностью исключена
возможность внесения каких-либо новых дефектов.

\smallskip

\noindent
{\textbf{Следствие 3.}} {\it Пусть $\Lambda(t)\equiv t$, т.\,е.\ $N(t)$~--- стандартный пуассоновский процесс. Тогда для любого $t>0$}
$$
{\sf E}P(t)=\fr{b}{a+b}+\Big(p_0-\fr{b}{a+b}\Big)e^{-(a+b)t}\,.
$$

\smallskip

Рассмотрим ситуацию, когда $\theta_j=1$ почти наверное, более
подробно. В этом случае соотношение~(\ref{e2art}) можно переписать в виде
\begin{equation}
1-P(Y_{j+1})=(1-\eta_{j+1})\big(1-P(Y_j)\big)\,,\ \ j\ge 1\,.
\label{e5art}
\end{equation}
Обозначим $\log (1-\eta_j)=\zeta_j$. Тогда из~(\ref{e5art}) получаем
\begin{equation}
\log \big(1-P(t)\big)-\log
(1-p_0)=\sum_{k=1}^{N(t)}\zeta_k\,.\label{e6art}
\end{equation}
В правой части представления~(\ref{e6art}) стоит обобщенный дважды
стохастический пуассоновский процесс (обобщенный процесс Кокса).
Это обстоятельство позволяет воспользоваться предельными теоремами
для обобщенных процессов Кокса (см., например,~\cite{8art} или~\cite{9art}) для
уточнения асимптотического поведения надежности системы в рамках
неоднородной экспоненциальной модели с непрерывным временем.

Предположим, что $0<{\sf D}\zeta_j=\sigma^2<\infty$, и обозначим
${\sf E}\zeta_j=\alpha$. Заметим, что $\alpha\le0$, так как $0\le\eta_j\le1$. 
Сходимость по распределению (слабую сходимость)
будем обозначать символом $\Longrightarrow$. Пусть $\Phi(x)$~---
стандартная нормальная функция распределения. В дополнение к
условию ${\sf E}\Lambda(t)\equiv t$, введенному выше, предположим,
что ${\sf D}\Lambda(t)\equiv s^2t$ для некоторого $s\in[0,\infty)$.
Справедлив следующий результат.

\smallskip

\noindent
{\textbf{Теорема 2.}} {\it Предположим, что $\alpha\neq 0$, ${\sf
E}\Lambda(t) \equiv t$,  ${\sf D}\Lambda(t)\equiv s^2t$ для
некоторого $s\in[0,\,\infty)$ и $\Lambda(t)\pto\infty$ при $t\to\infty$.
Тогда одномерные распределения неслучайно центрированного и
нормированного случайного процесса $P(t)$ слабо сходятся к
распределению некоторой случайной величины $Z$ при $t\to\infty$,
т.\,е.\
$$
\fr{\log \big(1-P(t)\big)-\log (1-p_0)-\alpha
t}{\sqrt{[\alpha^2(1+s^2)+\sigma^2]t}}\Longrightarrow Z\ \ \
(t\to\infty)\,,
$$
тогда и только тогда, когда существует случайная величина $V$
такая, что
$$
\fr{\Lambda(t)-t}{s\sqrt t}\Longrightarrow V\ \ \ (t\to\infty)\,.
$$
При этом}
\begin{multline*}
{\sf P}(Z<x)={}\\
{}={\sf
E}\Phi\left(x\sqrt{1+\fr{\alpha^2s^2}{\alpha^2+\sigma^2}}-\fr{\alpha
sV}{\sqrt{\si^2+\alpha^2}}\right )\,,\ \ \ x\in\r\,.
\end{multline*}

\smallskip

Несложно видеть, что предельная случайная величина $Z$ допускает
представление
$$
Z\eqd
\bigg[1+\fr{\alpha^2s^2}{\alpha^2+\sigma^2}\bigg]^{-1/2}
X+\fr{\alpha s}{\sqrt{a^2(1+s^2)+\sigma^2}}\, V\,,
$$
где $X$~--- случайная величина со стандартным нормальным
распределением, независимая от случайной величины $V$.

Теорема~2 является просто иной формой записи теоремы~9.2.2 из~\cite{8art}.

\smallskip

Для построения <<гарантированно доверительных>> границ для
надежности системы в рамках неоднородной экспоненциальной модели
можно воспользоваться оценками скорости сходимости в теореме~9.2.2,
полученными в статье~\cite{10art}.

Дополнительно предположим, во-первых, что $\e|\zeta_1|^3<\infty$,
и обозначим
$$
\beta^3 = \e|\zeta_1|^3\,,\ \ \ L_3 =
\fr{\beta^3}{(\alpha^2+\sigma^2)^{3/2}}\,.
$$
Во-вторых, предположим, что случайная величина~$V$, фигурирующая в
теореме~2, имеет стандартное нормальное распределение, причем
выполнено соотношение
\begin{equation}
\sup_x\bigg|{\sf P}\bigg(\fr{\Lambda(t)-t}{s\sqrt
t}<x\bigg)-\Phi(x)\bigg|\le\fr{\kappa}{\sqrt{t}}
\label{e7art}
\end{equation}
при некотором $\kappa\in(0,\infty)$. Соотношение~(\ref{e7art}) выполнено,
например, если $\Lambda(t)$ имеет гамма-рас\-пре\-де\-ление с параметром
масштаба, равным $s$, и параметром формы, равным $t$ (заметим, что
при этом случайная величина $N(t)$ имеет отрицательное
биномиальное распределение).

Для такого случая распределение случайной величины $Z$ является
стандартным нормальным и оценка, полученная в работе~\cite{10art}, имеет
вид
\begin{multline}
\sup_x\bigg|{\sf P}\bigg(\fr{\log \big(1-P(t)\big)-\log
(1-p_0)-\alpha
t}{\sqrt{[\alpha^2(1+s^2)+\sigma^2]t}}<x\bigg)-{}\\
{}- \Phi(x)\bigg|\le\fr{1}{\sqrt{t}}\Bigg[
\kappa+{}\\
{}+\inf_{\epsilon\in(0,1)}\left\{\fr{C_0L_3}{\sqrt{1-\epsilon}}+
\fr{s\sqrt{2}}{\sqrt{\pi}\epsilon}+sM(\epsilon)\right\}\Bigg]\,,
\label{e8art}
\end{multline}
где $C_0$~--- абсолютная постоянная в неравенстве Берри--Эссеена,
$C_0\le 0{,}7005$ (см.\ работу~\cite{11art}),
$$
M(\epsilon)=\max\bigg\{\fr{1}{\epsilon},\,\fr{\sqrt{1+\epsilon}}{\big(1+\sqrt{1-\epsilon}\big)\sqrt{2\pi
e(1-\epsilon)}}\bigg\}\,.
$$

\smallskip

Обозначим
\begin{multline}
K=K(L_3,s^2,\kappa)={}\\
{}=\kappa+\inf_{\epsilon\in(0,\,1)}\left\{
\fr{C_0L_3}{\sqrt{1-\epsilon}}+
\fr{s\sqrt{2}}{\sqrt{\pi}\epsilon}+sM(\epsilon)\right\}\,.
\label{e9art}
\end{multline}
Тогда из~(\ref{e8art}) для $z\in(0,\,1)$ вытекает справедливость неравенств
\begin{multline*}
\Phi\bigg(\fr{\log(1-z)-\log(1-p_0)-\alpha
t}{\sqrt{t[\alpha^2(1+s^2)+\sigma^2]}}\bigg)-
\fr{K}{\sqrt{t}}\le{}\\
{}\le
{\sf P}\big(P(t)>z\big)\le{}\\
{}
\le\Phi\bigg(\fr{\log(1-z)-\log(1-p_0)-\alpha
t}{\sqrt{t[\alpha^2(1+s^2)+\sigma^2]}}\bigg)+\fr{K}{\sqrt{t}}\,.
\end{multline*}

Зададим коэффициент доверия $\gamma\in(1/2,\,1)$ и решим
относительно $z$ уравнение
$$
\Phi\bigg(\fr{\log(1-z)-\log(1-p_0)-\alpha
t}{\sqrt{t[\alpha^2(1+s^2)+\sigma^2]}}\bigg)-\fr{K}{\sqrt{t}}=\gamma\,.
$$
Получим нижнюю гарантированную доверительную границу $\underline
z_{\gamma}(t)$ для $P(t)$ с коэффициентом доверия $\gamma$:
\begin{multline*}
\underline z_{\gamma}(t)=1-\exp\left \{\vphantom{\fr{K}{\sqrt{t}}}
\alpha t+{}\right.\\
\left.{}+\sqrt{t[\alpha^2(1+s^2)+\sigma^2]}
u\left(\gamma+\fr{K}{\sqrt{t}}\right)+\log(1-p_0)\right\}\,,\hspace*{-8.64pt}
\end{multline*}
где для $v\in(0,\,1)$ символом $u(v)$ обозначается $v$-кван\-тиль
стандартного нормального распределения. При этом для каждого $t>0$
$$
{\sf P}\big(P(t)>\underline z_{\gamma}(t)\big)\ge\gamma\,.
$$

Чтобы построить двусторонние границы для~$P(t)$, заметим, что из~(\ref{e8art})
вытекает неравенство
\begin{multline*}
\sup_x\bigg|{\sf P}\bigg(\bigg|\fr{\log \big(1-P(t)\big)-\log
(1-p_0)-\alpha
t}{\sqrt{[\alpha^2(1+s^2)+\sigma^2]t}}\bigg|<x\bigg)-{}\\
{}-2\Phi(x)+1\bigg|\le
\fr{2K}{\sqrt{t}}\,.
\end{multline*}
Отсюда получается гарантированная доверительная полоса
$\big\{\big(z^{(1)}_{\gamma}(t),\,z^{(2)}_{\gamma}(t)\big):\,t>0\big\}$
для $P(t)$ с коэффициентом доверия $\gamma$, где
\begin{multline*}
z^{(1)}_{\gamma}(t)=1-\exp\left\{\alpha
t+u %\left(
\fr{1}{2}\left( \vphantom{\fr{K}{\sqrt{t}}}
\gamma+1+{}\right. \right.
\\
\left. \left.
{} +2\fr{K}{\sqrt{t}}\right )%\right)
\sqrt{t[\alpha^2(1+s^2)+
\sigma^2]}+\log(1-p_0)\right\}\,;
\end{multline*}
\begin{multline*}
z^{(2)}_{\gamma}(t)=1-\exp\left\{\alpha
t-u %\left (
\fr{1}{2}\left( \vphantom{\fr{K}{\sqrt{t}}}
\gamma+1+{}\right. \right.
\\
\left. \left.
{}+2\fr{K}{\sqrt{t}}\right) %\right)
\sqrt{t[\alpha^2(1+s^2)+
\sigma^2]}+\log(1-p_0)\right\}.
\end{multline*}
При этом для каждого $t>0$
$$
{\sf P}\big(z^{(1)}_{\gamma}(t)\le P(t)\le
z^{(2)}_{\gamma}(t)\big)\ge\gamma\,.
$$

\section{Неоднородные логистические модели с непрерывным временем}

Обозначим
$$
Q(Y_j)=\fr{P(Y_j)}{1-P(Y_j)}\,,\ \ j\ge1\,.
$$
Пусть $\theta_1,\theta_2,\ldots$~--- независимые одинаково
распределенные случайные величины. Будем считать, что
последовательность $\{\theta_j\}_{j\ge1}$ стохастически независима
от точечного случайного процесса $Y_1,Y_2,\ldots$

Предположим, что
\begin{equation}
Q(Y_{j+1})=\theta_{j+1}Q(Y_j)\,,\ \ j\ge0\,.
\label{e10art}
\end{equation}
Эту модель изменения надежности, называемую {\it логистической},
можно интерпретировать следующим образом. Если $p_j=P(Y_j)$~---
вероятность успеха в последовательности испытаний Бернулли, то
величина $q_j=p_j/(1-p_j)$ характеризует ожидаемый номер
испытания, заканчивающегося первой неудачей в этой
последовательности испытаний Бернулли. Таким образом, величина
$Q(Y_j)=q_j$ характеризует ожида\-емое время жизни (безотказной
работы) системы после $j$-й модификации, как если бы после момента
$Y_j$ в нее не вносились бы никакие изменения. Следовательно,
соотношение~(\ref{e10art}) можно интерпретировать как формализацию того, что
каждая модификация системы изменяет ожи\-да\-емое время безотказной
работы после модификации в случайное число раз. Положим
$$
Q(t)=Q(Y_{N(t)})\,,\ \ \ \ t>0\,.
$$
Обозначим ${\sf E}\theta_1=a$. Пусть $p_0$~--- надежность системы в
момент $t=0$. Обозначим
$$
q_0=\fr{p_0}{1-p_0}\,.
$$


\smallskip

\noindent
{\textbf{Теорема 3.}} {\it В рамках неоднородной логистической модели
справедливо соотношение}
$$
{\sf E}Q(t)=q_0{\sf E}e^{(a-1)\Lambda(t)}\,,\ \ \ t>0\,.
$$

\smallskip

\noindent
Д\,о\,к\,а\,з\,а\,т\,е\,л\,ь\,с\,т\,в\,о.\ $\,$ Из соотношения~(\ref{e10art}) почти
очевидно, что
$$
{\sf E}Q(Y_j)=q_0a^j\,,\ \ \ j\ge1\,.
$$
Поэтому по формуле полной вероятности
\begin{multline*}
{\sf E}Q(t)=\sum_{j=0}^{\infty}{\sf E}Q(Y_j){\sf
P}\big(N(t)=j\big)={}\\
{}=\sum_{j=0}^{\infty}\int\limits_{0}^{\infty}e^{-\lambda}
\fr{\lambda^j}{j!}a^jq_0\,d{\sf
P}\big(\Lambda(t)<\lambda\big)=
{}\\
{}
=q_0\int\limits_{0}^{\infty}e^{-\lambda}\bigg(\sum_{j=0}^{\infty}
\fr{(a\lambda)^j}{j!}\bigg)\,d{\sf
P}\big(\Lambda(t)<\lambda\big)={}\\
{}=q_0\int\limits_{0}^{\infty}e^{\lambda(a-1)}\,d{\sf
P}\big(\Lambda(t)<\lambda\big)=q_0{\sf E}e^{(a-1)\Lambda(t)}\,.
\end{multline*}
Теорема доказана.

\smallskip

Так как функция
$$
g(x)=\fr{x}{1-x}
$$
выпукла при $0\le x<1$, то с помощью неравенства Иенсена из
теоремы~3 легко получить

\smallskip

\noindent
{\textbf{Следствие 4.}} {\it В рамках неоднородной логистической модели
справедливо соотношение}
$$
{\sf E}P(t)\le\fr{q_0{\sf E}e^{(a-1)\Lambda(t)}}{1+q_0{\sf
E}e^{(a-1)\Lambda(t)}}\,,\ \ \ t>0\,.
$$

\smallskip

\noindent
{\textbf{Следствие 5.}} {\it Если $N(t)$~--- стандартный пуассоновский
процесс, то в рамках неоднородной логистической модели при любом
$t>0$ справедливы соотношения}
$$
{\sf E}Q(t)=q_0{\sf E}e^{(a-1)t}\,,\ \ \ \ {\sf
E}P(t)\le\fr{q_0e^{(a-1)t}}{1+q_0e^{(a-1)t}}\,.
$$

\smallskip

Обозначим $\log \theta_j=\chi_j$. Тогда из~(\ref{e10art}) получаем
соотношение
$$
\log Q(Y_j)-\log q_0=\sum_{k=1}^{j}\chi_k\,,
$$
откуда вытекает, что
$$
\log Q(t)-\log q_0=\sum_{k=1}^{N(t)}\chi_k\,.
$$
Таким образом, по аналогии с теоремой~2 легко получить следующее
утверждение. Предположим, что $0<{\sf
D}\chi_j\equiv\sigma^2<\infty$, и обозначим ${\sf E}\chi_j=\alpha$.

\smallskip

\noindent
{\textbf{Теорема 4.}} {\it Предположим, что $\alpha\neq 0$, ${\sf
E}\Lambda(t) \equiv t$,  ${\sf D}\Lambda(t)\equiv s^2t$ для
некоторого $s\in[0,\, \infty)$ и $\Lambda(t)\pto\infty$ при $t\to\infty$.
Тогда одномерные распределения неслучайно центрированного и
нормированного случайного процесса $Q(t)$ слабо сходятся к
распределению некоторой случайной величины $Z$ при $t\to\infty$,
т.\,е.\
$$
\fr{\log Q(t)-\log q_0-\alpha
t}{\sqrt{[\alpha^2(1+s^2)+\sigma^2]t}}\Longrightarrow Z\ \ \
(t\to\infty)\,,
$$
тогда и только тогда, когда существует случайная величина $V$
такая, что
$$
\fr{\Lambda(t)-t}{s\sqrt t}\Longrightarrow V\ \ \ (t\to\infty)\,.
$$
При этом}
\begin{multline*}
{\sf P}(Z<x)={}\\
{}={\sf
E}\Phi\left(x\sqrt{1+\fr{\alpha^2s^2}{\alpha^2+\sigma^2}}-\fr{\alpha
sV}{\sqrt{\si^2+\alpha^2}}\right)\,,\ \ \ x\in\r\,.
\end{multline*}

\smallskip

Снова предположим, что имеет место соотношение~(\ref{e7art}). Для унификации
обозначений предположим, что $\e|\chi_1|^3<\infty$, и обозначим
$$
\beta^3 = \e|\chi_1|^3\,,\ \ \ L_3 =
\frac{\beta^3}{(\alpha^2+\sigma^2)^{3/2}}\,.
$$
Из результатов работы~\cite{10art} вытекает, что в сделанных
предположениях справедлива оценка
\begin{multline}
\sup_x\bigg|{\sf P}\bigg(\fr{\log Q(t)-\log q_0-\alpha
t}{\sqrt{[\alpha^2(1+s^2)+\sigma^2]t}}<x\bigg)-\Phi(x)\bigg|\le{}\\
{}\le \fr{K}{\sqrt{t}}\,,
\label{e11art}
\end{multline}
где величина $K$ определена в предыдущем разделе (см.~(\ref{e9art})). Эта
оценка позволяет получить гарантированную нижнюю доверительную
границу для надежности системы $P(t)$ в рамках модели~(\ref{e10art}),
которая для заданного коэффициента доверия $\gamma\in[1/2,\,1)$
имеет вид
$$
\underline z_{\gamma}(t)=\fr{\exp\{\underline
x_{\gamma}(t)\}}{1+\exp\{\underline x_{\gamma}(t)\}}\,,
$$
где
\begin{multline*}
\!\underline x_{\gamma}(t)=\alpha t+{}\\
{}+ u\left (1-\gamma-\fr{K}{\sqrt{t}}\right )\sqrt{t[\alpha^2(1+s^2)+\sigma^2]}+\log q_0\,.
\end{multline*}
При этом для каждого $t>0$
$$
{\sf P}\big(P(t)>\underline z_{\gamma}(t)\big)\ge\gamma\,.
$$

Чтобы получить двусторонние гарантированные границы для надежности
системы $P(t)$ в рамках модели~(\ref{e10art}), заметим, что из оценки~(\ref{e11art})
вытекает неравенство
\begin{multline*}
\sup_x\Bigg|{\sf P}\left(\left|\fr{\log Q(t)-\log q_0-\alpha
t}{\sqrt{[\alpha^2(1+s^2)+\sigma^2]t}}\right|<x\right)-{}\\
{}-2\Phi(x)+1\Bigg|\le
\fr{2K}{\sqrt{t}}\,.
\end{multline*}
Из этого неравенства следует, что гарантированная доверительная
полоса
$$
\big\{\big(z^{(1)}_{\gamma}(t),\,z^{(2)}_{\gamma}(t)\big):\,t>0\big\}
$$
для $P(t)$ с коэффициентом доверия $\gamma$ в рамках логистической
модели~(\ref{e10art}) имеет вид
\end{multicols}
$$
z^{(1)}_{\gamma}(t)=\fr{\exp\big\{\alpha
t-u\left ((1/2)(\gamma+1+2K/\sqrt{t})\right)
\sqrt{t[\alpha^2(1+s^2)+\sigma^2]}+\log
q_0\big\}}{1+\exp\big\{\alpha
t-u\left((1/2)(\gamma+1+2K/\sqrt{t})\right )
\sqrt{t[\alpha^2(1+s^2)+\sigma^2]}+\log q_0\big\}}\,,
$$
$$
z^{(2)}_{\gamma}(t)=\fr{\exp\big\{\alpha
t+u\left ((1/2)(\gamma+1+2K/\sqrt{t})\right)
\sqrt{t[\alpha^2(1+s^2)+\sigma^2]}+\log
q_0\big\}}{1+\exp\big\{\alpha
t+u\left ((1/2)(\gamma+1+2K/\sqrt{t})\right )
\sqrt{t[\alpha^2(1+s^2)+\sigma^2]}+\log q_0\big\}}\,.
$$
При этом для каждого $t>0$
$$
{\sf P}\big(z^{(1)}_{\gamma}(t)\le P(t)\le
z^{(2)}_{\gamma}(t)\big)\ge\gamma\,.
$$

\begin{multicols}{2}
\section{Неоднородные гиперболические модели с~непрерывным временем}

В исходных обозначениях и предположениях предыдущего раздела
рассмотрим модель
\begin{equation}
Q(Y_{j+1})=Q(Y_j)+\theta_{j+1}\,,\ \ j\ge0\,.\label{e12art}
\end{equation}
Эту модель изменения надежности, называемую {\it гиперболической},
можно интерпретировать как формализацию того, что каждая
модификация системы изменяет ожидаемое время ее безотказной работы
после модификации на случайное время. Как и ранее, полагаем
$$
Q(t)=Q(Y_{N(t)})\,,\ \ \ \ t>0\,.
$$
Обозначим ${\sf E}\theta_1=\alpha$. Пусть, как и ранее, $p_0$~---
надежность системы в момент $t=0$, $q_0=p_0/(1-p_0)$.

\smallskip

Из соотношения~(\ref{e12art}) очевидным образом вытекает, что в рамках
неоднородной гиперболической модели
$$
{\sf E}Q(t)=q_0+\alpha t\,,\ \ \ t>0\,.
$$
Отсюда с помощью неравенства Иенсена получаем неравенство
$$
{\sf E}P(t)\le\fr{q_0+\alpha t}{1+q_0+\alpha t}\,,
$$
справедливое при любом $t>0$.

\smallskip

Предположим теперь, что $0<{\sf D}\theta_1\equiv\sigma^2<\infty$.
Из рекуррентного соотношения~(\ref{e12art}) следует, что
$$
Q(t)=q_0+\sum_{j=0}^{N(t)}\theta_j\,, \ \ \ t>0\,.
$$
Таким образом, для уточнения асимптотики надежности $P(t)$ системы
в рамках гиперболической модели опять можно воспользоваться
предельными теоремами для обобщенных процессов \mbox{Кокса.}
{\looseness=1

}

\smallskip

\noindent
{\textbf{Теорема 5.}} {\it Предположим, что $\alpha\neq 0$, ${\sf
E}\Lambda(t) \equiv t$,  ${\sf D}\Lambda(t)\equiv s^2t$ для
некоторого $s\in[0,\,\infty)$ и $\Lambda(t)\pto\infty$ при $t\to\infty$.
Тогда одномерные распределения неслучайно центрированного и
нормированного случайного процесса $Q(t)$ слабо сходятся к
распределению некоторой случайной величины $Z$ при $t\to\infty$,
т.\,е.\
$$
\fr{Q(t)-q_0-\alpha
t}{\sqrt{[\alpha^2(1+s^2)+\sigma^2]t}}\Longrightarrow Z\ \ \
(t\to\infty)\,,
$$
тогда и только тогда, когда существует случайная величина $V$
такая, что
$$
\fr{\Lambda(t)-t}{s\sqrt t}\Longrightarrow V\ \ \ (t\to\infty)\,.
$$
При этом}
\begin{multline*}
{\sf P}(Z<x)={}\\
{}={\sf
E}\Phi\left(x\sqrt{1+\fr{\alpha^2s^2}{\alpha^2+\sigma^2}}-\fr{\alpha
sV}{\sqrt{\si^2+\alpha^2}}\right)\,,\ \ \ x\in\r\,.
\end{multline*}

\smallskip

Снова предположим, что имеет место соотношение~(\ref{e7art}). Для унификации
обозначений предположим, что $\e|\theta_1|^3<\infty$, и обозначим
$$
\beta^3 = \e|\theta_1|^3\,,\ \ \ L_3 =
\fr{\beta^3}{(\alpha^2+\sigma^2)^{3/2}}\,.
$$
Из результатов работы~\cite{10art} вытекает, что в сделанных
предположениях справедлива оценка
\begin{multline}
\sup_x\bigg|{\sf P}\bigg(\fr{Q(t)-q_0-\alpha
t}{\sqrt{[\alpha^2(1+s^2)+\sigma^2]t}}<x\bigg)-\Phi(x)\bigg|\le{}\\
{}\le
\fr{K}{\sqrt{t}}\,,
\label{e13art}
\end{multline}
где величина $K$ определена в соотношении~(\ref{e9art}). Эта оценка
позволяет получить гарантированную нижнюю доверительную границу
для надежности системы $P(t)$ в рамках модели~(\ref{e12art}), которая для
заданного коэффициента доверия $\gamma\in [1/2,\,1)$ имеет вид
$$
\underline z_{\gamma}(t)=\fr{\underline
x_{\gamma}(t)}{1+\underline x_{\gamma}(t)}\,,
$$
где
\begin{multline*}
\!\underline x_{\gamma}(t)={}\\
{}=\alpha
t+u\left(1-\gamma-\fr{K}{\sqrt{t}}\right)\sqrt{t[\alpha^2(1+s^2)+\sigma^2]}+
q_0\,.
\end{multline*}
При этом для каждого $t>0$
$$
{\sf P}\big(P(t)>\underline z_{\gamma}(t)\big)\ge\gamma\,.
$$

Чтобы получить двусторонние гарантированные границы для надежности
системы $P(t)$ в рамках модели~(\ref{e12art}), заметим, что из оценки~(\ref{e13art})
вытекает неравенство
\begin{multline*}
\sup_x\Bigg|{\sf P}\left(\left|\fr{Q(t)-q_0-\alpha
t}{\sqrt{[\alpha^2(1+s^2)+\sigma^2]t}}\right|<x\right)-{}\\
{}-2\Phi(x)+1\Bigg|\le
\fr{2K}{\sqrt{t}}\,.
\end{multline*}
Из этого неравенства следует, что гарантированная доверительная
полоса
$$
\big\{\big(z^{(1)}_{\gamma}(t),\,z^{(2)}_{\gamma}(t)\big):\,t>0\big\}
$$
для $P(t)$ с коэффициентом доверия $\gamma$ в рамках
гиперболической модели~(\ref{e12art}) имеет вид
\end{multicols}
$$
z^{(1)}_{\gamma}(t)=\fr{\alpha
t-u\left ((1/2)(\gamma+1+2K/\sqrt{t})\right)
\sqrt{t[\alpha^2(1+s^2)+\sigma^2]}+q_0}{1+\alpha
t-u\left((1/2)(\gamma+1+2K/\sqrt{t})\right)
\sqrt{t[\alpha^2(1+s^2)+\sigma^2]}+q_0}\,,
$$
$$
z^{(2)}_{\gamma}(t)=\fr{\alpha
t+u\left((1/2)(\gamma+1+2K/\sqrt{t})\right)
\sqrt{t[\alpha^2(1+s^2)+\sigma^2]}+q_0}{1+\alpha
t+u\left((1/2)(\gamma+1+2K/\sqrt{t})\right)
\sqrt{t[\alpha^2(1+s^2)+\sigma^2]}+q_0}\,.
$$


\noindent
При этом для каждого $t>0$
$$
{\sf P}\big(z^{(1)}_{\gamma}(t)\le P(t)\le
z^{(2)}_{\gamma}(t)\big)\ge\gamma\,.
$$
\pagebreak

\begin{multicols}{2}

{\small\frenchspacing
{%\baselineskip=10.8pt
\addcontentsline{toc}{section}{Литература}
\begin{thebibliography}{99}

\bibitem{1art}
\Au{Gnedenko B.\,V., V.\,Yu.~Korolev}.
Random summation: Limit theorems
and applications.~--- Boca Raton: CRC Press, 1996. %[1]

\bibitem{2art}
\Au{Коpолёв В.\,Ю.}
Пpикладные задачи теоpии веpоятностей: модели pоста
надежности модифициpуемых сис\-тем.~--- М.: Диалог-МГУ,
1997. %[2]

\bibitem{3art}
\Au{Королёв~В.\,Ю., Соколов~И.\,А.}
Основы математической теории надежности
модифицируемых систем.~--- М.: Изд-во ИПИРАН, 2006. %[3]

\bibitem{4art}
\Au{Бенинг В.\,Е., Королёв В.\,Ю., Соколов~И.\,А., Шоргин~С.\,Я.}
Рандомизированные модели и методы теории надежности информационных
и технических систем.~--- М.: Торус Пресс, 2007. %[4]

\bibitem{5art}
\Au{Волков Л.\,И., Шишкевич~А.\,М.}
Надежность летательных
аппаpатов.~--- М.: Высшая школа, 1975. %[5]

\bibitem{6art}
\Au{Волков Л.\,И.}
Упpавление эксплуатацией летательных
комплексов.~--- М.: Высшая школа, 1981. %[6]

\bibitem{7art}
\Au{Буш~Р., Мостеллеp~Ф.}
Стохастические модели обучаемости.~--- М.: ГИФМЛ, 1962. %[7]

\bibitem{8art}
\Au{Bening V.\,E., Korolev~V.\,Yu.}
Generalized Poisson models and
their applications in insurance and finance.~--- Utrecht: VSP, 2002. %[8]

\bibitem{9art}
\Au{Бенинг В.\,Е., Королёв~В.\,Ю., Шоргин~С.\,Я.}
Математические основы теории
риска.~--- М.: Физматлит, 2007. %[9]

\bibitem{10art}
\Au{Артюхов С.\,В., Королёв~В.\,Ю.}
Оценки скорости сходимости распределений
обобщенных дважды стохастических пуассоновских процессов с
ненулевым средним~// Обозрение промышленной и прикладной %\linebreak
мате\-ма\-ти\-ки, 2008 (в печати). %[10]

\label{end\stat}

\bibitem{11art}
\Au{Шевцова И.\,Г.}
Об абсолютной постоянной в неравенстве
Берри--Эссеена~// Сб.\ статей молодых ученых факультета ВМиК
МГУ.~--- М.: Изд-во факультета ВМиК МГУ, 2008. Вып.~5. С.~101--110. %[11]
\end{thebibliography}
}
}
\end{multicols}