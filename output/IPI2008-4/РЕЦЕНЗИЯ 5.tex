     монография А.\,И.~Зейфмана, В.\,Е.~Бенинга, И.\,А.~Соколова
     <<Марковские цепи и модели с непрерывным временем>> (М.: ЭЛЕКС-КМ, 2008. 
168~с.)
     
     В 2008~г.\ в издательстве <<ЭЛЕКС-КМ>> вышла монография <<Марковские цепи 
и модели с непрерывным временем>>. Ее авторы~--- А.\,И.~Зейфман, профессор, доктор 
физико-математических наук, декан факультета прикладной математики и компьютерных 
технологий Вологодского государственного педагогического университета, старший 
научный сотрудник директора Института проблем информатики РАН; В.\,Е.~Бенинг, 
профессор, доктор физико-математических наук, профессор факультета вычислительной 
математики и кибернетики МГУ им.\ М.\,В.~Ломоносова, старший научный сотрудник 
Института проблем информатики РАН; И.\,А.~Соколов, академик, директор Института 
проблем информатики РАН. 
     
     Как известно, получение явных выражений для вероятностей состояний 
стохастических моделей возможно лишь в исключительных случаях. В связи с этим одной 
из важнейших  задач при исследовании таких моделей давно является исследование 
поведения модели при стремлении времени к бесконечности и в частности, скорости 
сходимости к предельному режиму и связанных с этим функционалов. В рецензируемой 
книге работе изучаются вопросы, связанные с получением точных оценок скорости к 
предельному режиму и устойчивости для марковских цепей с непрерывным временем 
(стационарных и нестационарных), а также приложение методов и результатов к 
изучению некоторых  конкретных моделей, описываемых такими цепями, и в первую 
очередь, для  нестационарных марковских моделей систем массового обслуживания.
     
     Книга, несомненно, вызовет интерес у научных работников, инженеров, аспирантов, 
студентов и преподавателей вузов, интересующихся современным состоянием 
исследований в области теории вероятностей и ее приложений.

     Доктор физико-математических наук, профессор С.\,Я.~Шоргин 
