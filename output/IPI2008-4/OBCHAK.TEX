\def\stat{abstr}
{%\hrule\par
%\vskip 7pt % 7pt
\raggedleft\Large \bf%\baselineskip=3.2ex
A\,B\,S\,T\,R\,A\,C\,T\,S \vskip 17pt
    \hrule
    \par
\vskip 21pt plus 6pt minus 3pt }


%1
\def\tit{A DISINTEGRATED PACKET SWITCHING ARCHITECTURE}

\def\aut{I.\,A.~Sokolov$^1$ and V.\,B.~Egorov$^2$}
\def\auf{$^1$IPI RAN, isokolov@ipiran.ru\\[1pt]
$^2$IPI RAN, vegorov@ipiran.ru
}

\def\leftkol{\ } %ENGLISH ABSTRACTS}

\def\rightkol{\ } %ENGLISH ABSTRACTS}


\titele{\tit}{\aut}{\auf}{\leftkol}{\rightkol}

\noindent
The proposed disintegrated switching architecture enables the designers to create 
conventional and routing packet switches featuring enlarged performance.

\KWN{packet switch; integrated communication microcontroller; QUICC; PowerQUICC
}

\vskip 14pt plus 6pt minus 3pt

%2
\def\tit{MEDIAN MODIFICATION OF EM- AND SEM-ALGORITHMS FOR SEPARATION 
OF~MIXTURES OF~PROBABILITY DISTRIBUTIONS AND THEIR APPLICATION TO THE DECOMPOSITION 
OF~VOLATILITY OF FINANCIAL TIME SERIES}


\def\aut{A.\,K.~Gorshenin$^1$,  V.\,Yu.~Korolev$^2$, and A.\,M.~Tursunbayev$^3$}
\def\auf{$^1$Department of Mathematical
Statistics, Faculty of Computational Mathematics and Cybernetics,\\ 
$\hphantom{^1}$M.\,V.~Lomonosov Moscow State University, andygorshenin@gmail.com\\[1pt]
$^2$Department of Mathematical
Statistics, Faculty of Computational Mathematics and Cybernetics,\\
$\hphantom{^2}$M.\,V.~Lomonosov Moscow State University, vkorolev@comtv.ru\\[1pt]
$^3$Department of Mathematical
Statistics, Faculty of Computational Mathematics and Cybernetics,\\ 
$\hphantom{^3}$M.\,V.~Lomonosov Moscow State University
}

%\def\leftkol{\ } %ENGLISH ABSTRACTS}

%\def\rightkol{\ } %ENGLISH ABSTRACTS}


\titele{\tit}{\aut}{\auf}{\leftkol}{\rightkol}

\noindent
Median modofications of EM- and SEM-algorithms are proposed for separation of mixtures of normal distributions.
The advantages of the proposed algorithms over standard methods are illustrated by the numerical solution of
the problem of decomposition  of volatility of financial indices.

\KWN{separation mixtures of probability distributions; robustness; efficiency; EM-algorithm;
SEM-algorithm; volatility
}


\vskip 12pt plus 6pt minus 3pt


%3
\def\tit{SPLITTING OF DISTRIBUTION MIXTURE IN TWO COMPONENTS}

\def\aut{M.\,P. Krivenko}
\def\auf{IPI RAN, mkrivenko@ipiran.ru}

%\def\leftkol{\ } %ENGLISH ABSTRACTS}

%\def\rightkol{\ } %ENGLISH ABSTRACTS}

\titele{\tit}{\aut}{\auf}{\leftkol}{\rightkol}

\noindent
The problems of splitting the distribution mixture
in two components and of estimating  mixture parameters are examined 
if samples from distribution mixture and from one of components are
presented. Two methods of parameters estimation are suggested, as well
as corresponding algorithms are constructed and examined.


\KWN{mixture of normal distributions;
splitting of distribution mixture; EM-algorithm}
%\pagebreak


%\vfil
% \vskip 18pt plus 6pt minus 3pt
% \vskip 24pt plus 9pt minus 6pt

\pagebreak


\vskip 12pt plus 6pt minus 3pt


%3
\def\tit{CONTINUOUS-TIME NON-HOMOGENEOUS RECURRENT RELIABILITY VARIATION MODELS FOR 
MODIFIABLE SYSTEMS}


\def\aut{S.\,V. Artyukhov$^1$ and V.\,Yu.~Korolev$^2$}
\def\auf{$^1$Vneshprombank, ArtyuhovSV@yandex.ru\\[1pt]
$^2$Department of Mathematical Statistics, Faculty of Computational 
Mathematics and Cybernetics,\\
$\hphantom{^1}$M.\,V.~Lomonosov Moscow State University; IPI RAN, vkorolev@comtv.ru}

%\def\leftkol{\ } %ENGLISH ABSTRACTS}

%\def\rightkol{\ } %ENGLISH ABSTRACTS}

\titele{\tit}{\aut}{\auf}{\leftkol}{\rightkol}

\noindent
Estimates of convergence rate in limit theorems for compound doubly stochastic Poisson processes (compound
Cox processes) are used for a more accurate description of the behavior of reliability of complex modifiable
technical and information systems within the framework of continuous-time non-homogeneous recurrent
reliability variation models

\KWN{compound doubly stochastic Poisson process; compound Cox process; reliability variation model;
availability function; guaranteed confidence bounds}
%\pagebreak

%\vful

 \vskip 12pt plus 6pt minus 3pt

%5
\def\tit{INFORMATION TECHNOLOGY OF USING FACIAL BIOMETRICS 
FOR~INCREASING~AFIS~THROUGHPUT}

\def\aut{O.\,S.~Ushmaev}

\def\auf{IPI RAN, oushmaev@ipiran.ru}

\def\leftkol{ENGLISH ABSTRACTS}
%
%\def\rightkol{ENGLISH ABSTRACTS}

%\def\leftkol{\ } %ENGLISH ABSTRACTS}

%\def\rightkol{\ } %ENGLISH ABSTRACTS}

\titele{\tit}{\aut}{\auf}{\leftkol}{\rightkol}

\noindent
Nowadays, multimodal biometrics is rapidly replacing tedious procedures of identification.
Particularly operating and perspective civil ID systems use multimodal approach. The formal method
for designing high-speed multibiometric technologies and systems
is suggested. The effectiveness
of the approach is shown by an example of developed experimental software with service-oriented architecture.

\KWN{biometric identification; multimodal biometrics; platform independent; service-oriented
architecture}

%\vskip 18pt plus 6pt minus 3pt

 \vskip 12pt plus 6pt minus 3pt


\def\tit{TIME INTERVALS AS OBJECTS OF A GENERAL-PURPOSE OPERATING SYSTEM}

%6
\def\aut{V.~Yegorov$^1$ and E.~Matveev$^2$}

\def\auf{$^1$Penza State University; CryptoSoft Corp., vec@cryptosoft.ru\\[1pt]
$^2$CryptoSoft Corp., eugene@cryptosoft.ru}

%\def\leftkol{\ } % ENGLISH ABSTRACTS}

%\def\rightkol{\ } %ENGLISH ABSTRACTS}

\titele{\tit}{\aut}{\auf}{\leftkol}{\rightkol}


\noindent
The main goal of this paper is to describe a new type of the general-purpose
operating system object~--- the time interval object, which allows a
general-purpose operating system to act as a real-time operating system.

\label{st\stat}

 \KWN{operating system; real time; time interval;
extension of C language; hardware interrupt controller; multiprocessor system}

 \vskip 12pt plus 6pt minus 3pt
 

\def\tit{EUROWORDNET: OBJECTIVES, STRUCTURE, AND RELATIONSHIPS}

%7
\def\aut{O.~Kozhunova}

\def\auf{$^1$IPI RAN, okozhunova@ipiran.ru}

\def\leftkol{ENGLISH ABSTRACTS}

\def\rightkol{ENGLISH ABSTRACTS}

\titele{\tit}{\aut}{\auf}{\leftkol}{\rightkol}

\noindent
In the review, brief desciption of the EuroWordNet tool, history of its
creation, examples of the similar resources, its objectives, structure, and relationships
are given.
%\label{st\stat}

\KWN{lexical and semantic resource EuroWordNet; WordNet dictionaries;
thesaurus; sinset; language relationships; OnterLingual Index ILI
}

%\pagebreak

% \thispagestyle{headings}

%\vskip 14pt plus 6pt minus 3pt

%\vfil

% \vskip 24pt plus 9pt minus 6pt
%\vskip 6pt plus 3pt minus 3pt
%\vspace*{12pt}


% \pagebreak

 \label{end\stat}
 
 