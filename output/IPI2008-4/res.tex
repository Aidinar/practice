     РЕЦЕНЗИЯ
     на монографию В.\,Е.~Бенинга, В.\,Ю.~Королева, И.\,А.~Соколова, С.\,Я.~Шоргина
     <<Рандомизированные модели и методы теории надежности информационных и 
технических систем>>
     
     В 2007~г.\ в издательстве <<ТОРУС ПРЕСС>> вышла монография <<Рандомизированные 
модели и методы теории надежности информационных и технических систем>>. Ее авторы~--- 
В.\,Е.~Бенинг, профессор, доктор физико-математических наук, профессор факультета 
вычислительной математики и кибернетики МГУ им. М.\,В.~Ломоносова, старший научный 
сотрудник Института проблем информатики РАН; В.\,Ю.~Королёв, профессор, доктор 
     физико-математических наук, профессор факультета вычислительной математики и 
кибернетики МГУ им.\ М.\,В.~Ломоносова, ведущий научный сотрудник Института проблем 
информатики РАН; И.\,А.~Соколов, академик, директор Института проблем информатики 
РАН; С.\,Я.~Шоргин, профессор, доктор физико-математических наук, заместитель директора 
Института проблем информатики РАН. 
     
     Для исследования надежности информационных и технических систем (ИиТС), 
подверженных влиянию случайных и нестационарно изменяющихся факторов, в книге 
предложены альтернативные классическим рандомизированные математические модели и 
методы. Большой интерес представляет непараметрический подход к оцениванию 
коэффициента готовности, для получения интервальных оценок используются новейшие 
оценки скорости сходимости в центральной предельной теореме теории вероятностей. В 
частности, в книге рассматривается ситуация, в которой учитываются возможные изменения 
надежности восстанавливаемой ИиТС вследствие ее модификаций или ремонтов. Отдельная 
глава посвящена прямому применению байесовской идеологии при анализе надежности и 
эффективности ИиТС. 
     
     Большая часть книги посвящена учету влияния неоднородности интенсивности потока 
информативных событий, в результате которых накапливается статистическая информация, на 
итоговые статистические выводы о параметрах ИиТС. Существенно развита  общая теория 
статистического вывода на основе выборок случайного объема. В книге показано, что при 
замене объема выборки случайной величиной заметно увеличиваются вероятности 
критических значений тех или иных статистических критериев или уменьшаются 
доверительные вероятности по сравнению с классической ситуацией. Детально рассмотрены 
методы анализа надежности, основанные на предельных теоремах для порядковых статистик в 
выборках случайного объема. Исследована возможность использования распределения 
Стьюдента в качестве альтернативы нормальному закону в статистических методах анализа 
надежности ИиТС. 
     
     Книга может быть полезна для специалистов в области применения методов теории 
вероятностей и математической статистики к анализу надежности ИиТС, а также для 
аспирантов и студентов старших курсов, обучающихся по специальностям <<информатика>> и 
<<прикладная математика>>.
     
     
Д.ф.-м.н., профессор							А.\,В.~Печинкин
     
     
     РЕЦЕНЗИЯ
     на монографию В.\,Ю.~Королева и И.\,А.~Соколова
     <<Математические модели неоднородных потоков экстремальных событий>>
     
     В 2008~г.\ в издательстве <<ТОРУС ПРЕСС>> вышла монография <<Математические 
модели неоднородных потоков экстремальных событий>>. Ее авторы~--- В.\,Ю.~Королёв, 
профессор, доктор физико-математических наук, профессор факультета вычислительной 
математики и кибернетики МГУ им.\ М.\,В.~Ломоносова, ведущий научный сотрудник 
Института проблем информатики РАН и И.\,А.~Соколов, академик, директор Института 
проблем информатики РАН. 
     
     В книге рассмотрены математические модели вероятностно-статистических 
характеристик катастроф в неоднородных потоках экстремальных событий. Сформулированы 
задачи моделирования катастрофических событий, связанных как с критическими 
превышениями уровня процессом, описывающим накопленные эффекты неблагоприятных 
факторов, так и с однократными шоковыми воздействиями. В качестве основных 
математических моделей при решении указанных задач рассматриваются экстремумы 
обобщенных дважды стохастических пуассоновских процессов и макс-обобщенные дважды 
стохастические пуассоновские процессы. Для таких процессов доказан ряд предельных теорем. 
Возникающие в этих теоремах предельные распределения вероятностей предлагаются в 
качестве аппроксимаций для вероятностно-статистических закономерностей, присущих 
потокам экстремальных (катастрофических) событий. Большой интерес представляет анализ 
временных характеристик глобальных катастроф, вызванных столкновением Земли с 
потенциально опасными небесными телами (астероидами, кометами). На примере этого 
анализа описаны конкретные процедуры для вычисления вероятностных характеристик 
катастроф, в частности <<ожидаемого времени>> катастрофы и продолжительности периода, в 
течение которого вероятность катастроф пренебрежимо мала.
     
     Книга представляет значительный интерес для специалистов в области применения 
методов теории вероятностей и математической статистики к анализу рисков, связанных с 
чрезвычайными ситуациями и катастрофами, и надежности информационных и технических 
систем. Она также будет полезна аспирантам и студентам старших курсов, обучающимся по 
специальностям <<информатика>> и <<прикладная математика>>. 
     
Д.ф.-м.н., профессор							А.\,В.~Печинкин
     