\def\stat{ushmaev}

\def\tit{ИНФОРМАЦИОННАЯ ТЕХНОЛОГИЯ ИНТЕГРАЦИИ ИДЕНТИФИКАЦИИ
ПО~ИЗОБРАЖЕНИЮ ЛИЦА ДЛЯ~УСКОРЕНИЯ~АВТОМАТИЧЕСКОЙ
ДАКТИЛОСКОПИЧЕСКОЙ~ИДЕНТИФИКАЦИИ}
\def\titkol{Информационная технология интеграции идентификации
по~изображению лица} % для~ускорения автоматической
%дактилоскопической идентификации}
\def\autkol{О.\,С. Ушмаев}
\def\aut{О.\,С. Ушмаев$^1$}

\titel{\tit}{\aut}{\autkol}{\titkol}

%{\renewcommand{\thefootnote}{\fnsymbol{footnote}}\footnotetext[1]
%{Работа выполнена при поддержке РФФИ, грант
%08-01-00567.}}

\renewcommand{\thefootnote}{\arabic{footnote}}
\footnotetext[1]{Институт проблем информатики Российской академии наук, oushmaev@ipiran.ru
}

%\vspace*{-6pt}

\Abst{Рассмотрена проблема синтеза мультибиометрических систем распознавания по
отпечаткам пальцев и изображению лица. Основное внимание уделено производительности.
Предложены методы организации вычислений, оптимизирующие скорость сравнения
биометрических образцов. Проведенные эксперименты показывают эффективность
разработанных методов.}

\KW{биометрические технологии; мультибиометрическая идентификация; идентификация по
отпечаткам пальцев; идентификация по изображению лица; оптимизация производительности}

      \vskip 36pt plus 9pt minus 6pt

      \thispagestyle{headings}

      \begin{multicols}{2}

      \label{st\stat}


\section{Введение}

     К настоящему времени накоплен значительный опыт создания
крупномасштабных био\-мет\-риче\-ских систем. Большинство из них являются %\linebreak
автоматизированными дактилоскопическими сис\-те\-ма\-ми (АДИС). Выбор
отпечатков пальцев обуслов\-лен множеством факторов. Основным из них является
традиционное использование отпечатков пальцев в криминальном учете и высокий
потенциал дактилоскопии с точки зрения ошибок распознавания. Использование для
распознавания всех 10~отпечатков пальцев достаточно для практически
безошибочного распознавания людей в масштабах на\-се\-ле\-ния страны.  При
использовании меньшего \mbox{числа} отпечатков соотношение ошибок 1-го (FRR~---
False Rejection Rate) и 2-го (FAR~--- False Acсeptance Rate) родов является
удовлетворительных во многих приложениях (рис.~\ref{f1ush})~\cite{1ush}. Однако
использование дактилоскопии имеет недостатки. Накопленный опыт реализации АДИС
для криминальной и гражданской идентификации позволяет выделить сле\-ду\-ющие
направления развития~[2--4]:
\begin{itemize}
\item идентификация людей, не обладающих пригодными для распознавания
отпечатками пальцев (инвалиды, плохое состояние кожи);
\item увеличение производительности.
\end{itemize}

     Первая задача возникает из непосредственных требований к биометрическим
системам, а именно она должна автоматически идентифицировать личность по
предъявляемым биометрическим образцам. Соответственно люди с плохим качеством
отпечатков пальцев не могут надежно идентифицироваться средствами АДИС.

     Актуальность увеличения производительности в первую очередь связана с тем,
что в настоящее время при создании крупномасштабных биометрических систем более
75\% средств затрачиваются на аппаратные средства  вычислительных узлов. При
полномасштабном внедрении систем гражданской идентификации таких, как
пас\-порт\-но-ви\-зо\-вые %\linebreak
документы нового поколения, биометрические массивы вырастут
многократно, поэтому эффективное решение задачи увеличения про\-из\-во\-ди\-тель\-ности
может значительно повысить эффективность внед\-ре\-ния биометрии.

     Обе указанные задачи могут быть решены путем добавления
дополнительного биометрического идентификатора~\cite{5ush}. Наиболее доступной и
удобной дополнительной биометрикой является изображение лица. В частности, в
крупномасштабных системах дактилоскопической идентификации (криминальные
учеты, паспортно-визовые системы) помимо отпечатков пальцев доступна фотография
как традиционный способ идентификации личности.

\begin{figure*} %fig1
\vspace*{1pt}
\begin{center}
\mbox{%
\epsfxsize=104.24mm
\epsfbox{ush-1.eps}
}
\end{center}
\vspace*{-9pt}
\Caption{Ошибки 1-го и 2-го рода распознавания по нескольким отпечаткам пальцев (NIST
SD14)~\cite{4ush}: \textit{1}~--- 4~отпечатка, \textit{2}~--- 2~отпечатка (указательные),
\textit{3}~--- 2~отпечатка (большие),
\textit{4}~--- правый указательный, \textit{5}~--- правый большой,
 \textit{6}~--- левый указательный, \textit{7}~--- левый большой
\label{f1ush}}
\end{figure*}

     С технической точки зрения, лицевая био\-мет\-рия потенциально может быть
использована для решения задачи увеличения производительности АДИС, поскольку
скорость идентификации по изоб\-ра\-же\-нию лица многократно выше скорости сравнения
по отпечаткам пальцев. Данное обстоятельство позволяет использовать результаты
сравнения по изображению лица для грубого <<отсева>> части субъектов, что сократит
нагрузку на вы\-чис\-ли\-тель\-ные узлы АДИС.

     Далее в статье рассмотрена задача увеличения производительности АДИС без
потери качества распознавания за счет добавления лицевой биометрии. В разделе~2
дана методология оценки производительности биометрической идентификации.
Алгоритмы интеграции идентификации по изображению лица в АДИС изложены в
разделе~3. В разделе~4 представлены теоретические оценки производительности
интегрированной мультибиометрической системы распознавания по отпечаткам
пальцев и изображению лица. В разделе~5
приведены результаты экспериментов по
моделированию изменения производительности. В разделе~ 6 представлены основные
выводы и заклю\-чение.
{\looseness=1

}

\section{Методология оценки производительности}

     Основным показателем производительности биометрической системы является
проектное время ожидание отклика. В случае крупномасштабных систем, таких как
паспортно-визовые системы или системы криминального учета, время отклика обычно
устанавливается в 24~ч для суточного цик\-ла (в редких случаях~--- 168~ч для
недельного цик\-ла) функционирования системы. Соответственно, биометрическая
система должна успевать обрабатывать заявки на идентификацию, поступающие в
течение суток. Это условие слабее требования реального времени. Его использование
связано с тем, что поток заявок имеет прогнозируемую
неравномерную структуру. Примерный вид графика интенсивности запроса приведен
на рис.~2. Максимальный участок соответствует времени, когда
функционирует большинство пунктов сбора биометрической информации во всех
часовых поясах.


     В таком случае проектное время обработки заявок, поступающих в течение суток,
рассчитывается по следующей формуле:
     \begin{equation*}
     T_{\mathrm{сут}} = r_{\mathrm{сут}} t\,.
%     \label{e1ush}
     \end{equation*}

     \medskip

%\end{multicols}
%\begin{figure*} %fig2
\vspace*{1pt}
\begin{center}
\mbox{%
\epsfxsize=79mm
\epsfbox{ush-2.eps}
}
\end{center}
\vspace*{3pt}
%\Caption{
\centerline{{\figurename~2}\ \ {\small Примерный график интенсивности запросов}}
%\label{f1s}}
%\end{figure*}
\bigskip
\addtocounter{figure}{1}

\noindent
Здесь $ r_{\mathrm{сут}}$~--- максимальный проектный поток заявок в течение суток;
$t$~--- среднее время идентификации по биометрической базе, в случае АДИС (или
другой однофакторной биометрии) обычно линейно пропорционально количеству
записей в базе данных (БД), поскольку в ходе идентификации предъявляемые образцы
последовательно сравниваются с каждым хранимым, т.\,е.\ в большинстве приложений
\begin{equation}
t=\fr{N t_{\mathrm{ср}}}{W}\,,
\label{e2ush}
\end{equation}
где $N$~--- число записей в БД; $ t_{\mathrm{ср}}$~--- среднее время сравнения пары
биометрических образцов на едини\-цу мощности вычислительных средств (с); %\linebreak
$W$~---
мощность вычислительных средств, нормированная на единичную номинальную
мощность (например, на один процессор с тактовой час\-то\-той~1~ГГц) .

     В случае биометрической идентификации потери, связанные с
распараллеливанием вычислений, минимальны, поэтому при грубой оценке
производительности данным фактором можно пренебречь.

     Чтобы система справлялась с потоком заявок в течение суток, накладывается
ограничение $ T_{\mathrm{сут}} <$\linebreak $<\;24$~ч. Резерв $R$ системы (избыточность)
определяется как
     \begin{multline*}
     R = \fr{24  - T_{\mathrm{сут}}}{24} = {}\\
     {}=1-\fr{N
t_{\mathrm{ср}}}{W}\,\fr{ r_{\mathrm{сут}}}{24} =1-
\fr{Nt_{\mathrm{ср}}}{W}\,r = 1-tr\,,
%     \label{e3ush}
     \end{multline*}
     где $r$~--- интенсивность потока заявок.

     Избыточность в основном необходима в сле\-ду\-ющих случаях:
     \begin{enumerate}[(1)]
\item сезонные колебания и резкие скачки нагрузки на биометрические серверы;
\item выход из строя части вычислительных мощностей;
\item плановый профилактический вывод из эксплуатации части вычислительных
мощностей;
\item сбой системы, приводящий к необходимости повторной обработки запросов.
\end{enumerate}

     В случае систем оперативной идентификации, где время ожидания ограничено
минутами, помимо средней способности системы обеспечить обработку потока заявок
требуется ограничить дисперсию времени ожидания, чтобы в моменты пиковой
загрузки вычислительных мощностей проектное время ожидания не было превышено.
В таком случае требуется проводить специальное исследование структуры потока
заявок, чтобы определить максимально возможную интенсивность.

     Как видно из формулы~(\ref{e2ush}), основным фак\-то\-ром,
ограничивающим производительность био\-мет\-ри\-че\-ской системы, является скорость
срав\-не\-ния био\-мет\-ри\-че\-ских образцов. Уменьшение %\linebreak 
времени сравнения положительно
сказывается на производительности системы.
{\looseness=-1

}

\vspace*{3pt}

\section{Алгоритмы интеграции}

     В случае одномодальной биометрической сис\-те\-мы время сравнения и ошибки
распознавания можно уменьшить только доработкой ис\-поль\-зу\-емо\-го
специализированного биометрического программного обеспечения. Для
мультибиометрических систем, в частности комбинации отпечатков пальцев и
изображения лица, возможны несколько вариантов реализации биометрической
идентификации, позволяющих корректировать эксплуатационные показатели.

     При реализации технологии одновременной идентификации по отпечаткам
пальцев и изображению лица возможны две основные схемы интеграции:
     \begin{enumerate}[(1)]
\item процессы сравнения отпечатков и изображения лица независимы
(рис.~\ref{f3ush},\,\textit{а});
\item процессы сравнения зависимы (рис.~\ref{f3ush},\,\textit{б}).
\end{enumerate}

     В~[5--9] разработана методология интеграции биометрических систем в случае
независимого сравнения. В таком случае достигаются минимальные возможные
ошибки распознавания. Однако это приводит к потерям про\-из\-во\-ди\-тель\-ности. Во
многих задачах, решаемых современными АДИС, качество идентификации в терминах
ошибок распознавания является приемлемым. В то же время производительность
остается достаточно низкой. Поэтому далее в статье мы сосредоточим
основное внимание на схеме интеграции с зависимыми процессами идентификации,
которая позволяет достичь прироста в производительности без потерь в качестве
распознавания.

     Рассмотрим реализацию идентификации более детально. На вход функции
одномодальной био\-мет\-ри\-че\-ской идентификации поступают предъ\-явля\-емая
биометрическая запись или образец и био\-мет\-ри\-че\-ская БД или линейный
список био\-мет\-ри\-че\-ских записей. На выходе мы получаем меры сходства
предъявляемой и хранимых в БД записей. На основе этой информации принимается
решение, принадлежат ли записи одному человеку или нет. Большинство систем
идентификации по отпечаткам пальцев и изображению лица используют пороговые
методы принятия решения.

\end{multicols}

     \begin{figure*} %fig3
\vspace*{1pt}
\begin{center}
\mbox{%
\epsfxsize=135.116mm
\epsfbox{ush-3.eps}
}
\end{center}
\vspace*{-9pt}
     \Caption{Независимые~(\textit{а}) и зависимые~(\textit{б}) процессы идентификации
      \label{f3ush}}
      \end{figure*}

\begin{multicols}{2}

     Увеличения производительности АДИС можно достичь, если по результатам
идентификации по изображению лица принимать решения о целесообразности
дальнейшего поиска по отпечаткам пальцев (схема реализации функции сравнения
отпечатков пальцев и изображения лица приведена на рис.~\ref{f5ush}).

\begin{figure*} %fig5
\vspace*{1pt}
\begin{center}
\mbox{%
\epsfxsize=146.454mm
\epsfbox{ush-5.eps}
}
\end{center}
\vspace*{-9pt}
\Caption{Реализации функции мультибиометрического сравнения отпечатков пальцев и изображения
лица
\label{f5ush}}
\end{figure*}

     Как видно из рис.~\ref{f5ush}, в реализации функции сравнения биометрических
образцов есть четыре терминальных состояния:
     \begin{enumerate}[(1)]
\item  2~--- при сравнении изображений лица принято решение об идентичности
образцов (Accept), так как результат сравнения $m_1$ превышает определенный
порог~$A_1$;
\item 4~--- при сравнении изображений лица принято решение о различности
образцов (Reject), так как мера сходства $m_1$ меньше некоторого минимального
порога~$R_1$;
\item 5~--- после сравнения отпечатков пальцев принято решение об идентичности
образцов, суммарная мера сходства $m_2$ больше порога  заданного
порога~$A_2$ (проблемы построения интегральной меры сходства при
мультибиометрической идентификации изложены в~[5, 6]);
\item 6~--- по результатам сравнения принято решение о различности образцов.
\end{enumerate}

     Оценим статистические характеристики временных показателей выполнения
функции мультибиометрического сравнения. Следует разделить следующие два случая:
     \begin{enumerate}[(1)]
\item образцы принадлежат одному человеку (обозначим среднее время сравнения через~$t^g$);
\item образцы принадлежат разным людям ($t^i$).
\end{enumerate}

     В первом случае вероятности $m_1\geq A_1$ и $m_1 <$\linebreak $<\;R_1$ являются
стандартными показателями качества распознавания и обозначаются TAR($A_1$), True
Acceptance Rate, и $\mathrm{FRR}(R_1)=1 - \mathrm{TAR}(R_1)$, 
ошибка первого рода. Соответственно среднее время сравнения <<своих>> будет
складываться из двух слагаемых: время выполнения сравнения по изображению лица и
времени сравнения по отпечаткам пальцев. Причем сравнение отпечатков пальцев
будет проводиться только в случае $R_1 \leq m_1 < A_1$. Соответственно среднее
суммарное время определяется по формуле:
\begin{multline}
 t^g = t^g_{\mathrm{face}} +\left ( 1-\mathrm{TAR}_{\mathrm{face}} (A_1) -{}\right.\\
 {}-\left.
\mathrm{FRR}_{\mathrm{face}} (R_1 )\right )  t^g_{\mathrm{finger}} ={}\\
     {}= t^g_{\mathrm{face}} +\left ( \mathrm{FRR}_{\mathrm{face}} (A_1) -
\mathrm{FRR}_{\mathrm{face}} (R_1)\right ) t^g_{\mathrm{finger}}\,,
     \label{e4ush}
     \end{multline}
где $t^g_{\mathrm{face}}$~--- среднее время выполнения <<своих>> сравнений для
изображения лица; $t^g_{\mathrm{finger}}$~---  среднее время выполнения <<своих>>
сравнений для отпечатков пальцев.


     Во втором случае вероятности $m_1 \geq A_1$ и $m_1<$\linebreak $<\;R_1$ выражаются
аналогичным образом \mbox{через} ошибку второго рода $\mathrm{FAR}(A_1)$ и $\mathrm{TRR}(R_1)=1-$\linebreak
$-\;\mathrm{FAR}(R_1)$, True Rejection Rate:

\noindent
\begin{multline}
     t^i =  t^i_{\mathrm{face}} +\left (1-\mathrm{FAR}_{\mathrm{face}}(A_1)-{}\right.\\
\left. {}-\mathrm{TRR}_{\mathrm{face}}(R_1)\right ) t^i_{\mathrm{finger}} ={}\\
     {}=
     t^i_{\mathrm{face}} + \left ( \mathrm{FAR}_{\mathrm{face}}(R_1)-
\mathrm{FAR}_{\mathrm{face}}(A_1)\right ) t^i_{\mathrm{finger}}\,,
     \label{e5ush}
     \end{multline}
где $t^i_{\mathrm{face}}$~--- среднее время выполнения <<чужих>> сравнений для
изображения лица; $t^i_{\mathrm{finger}}$~--- среднее время выполнения <<чужих>>
сравнений для отпечатков пальцев.

\section{Оценка производительности}

     Сосредоточим внимание на решении задачи оценки производительности в задаче
массовой идентификации. Массовость сравнения позволяет при оценке
производительности в значительной степени ориентироваться на
формулы~(\ref{e4ush}) и~(\ref{e5ush}), так как при сравнении по большой базе
входящие в формулу вероятности дают достаточно точную оценку времени
идентификации.

      При выполнении операции массовой идентификации доминируют операции
сравнения <<чужих>>. Так, если в базе зарегистрированы по одному образцу для $N$
субъектов, то в процессе идентификации будет выполнено $N$ <<чужих>> и одно <<свое>>
сравнение. Соответственно при большом $N$ и сравнимых по масштабу $t^g$ и $t^i$
доля времени $t^g$ в процессе идентификации ничтожно мала, ей можно пренебречь.
Поэтому при использовании одного вычислительного узла изменение $a(A_1,R_1)$
производительности АДИС за счет добавления идентификации по изображению лица с
выбранными порогами $A_1$ и $R_1$ можно вычислить как 
отношение~(\ref{e5ush}) к времени идентификации $t^i_{\mathrm{finger}}$ только по
отпечаткам пальцев:
     \begin{multline}
     a(A_1,R_1) = \fr{t^i}{t^i_{\mathrm{finger}}}
=\fr{t^i_{\mathrm{face}}}{t^i_{\mathrm{finger}}}+{}\\
     {}+ \left ( \mathrm{FAR}_{\mathrm{face}}(R_1) -
\mathrm{FAR}_{\mathrm{face}}(A_1)\right )\,.
     \label{e6ush}
     \end{multline}

     Во многих случаях для сравнения лица потребуется выделение отдельных
вычислительных средств (обозначим их долю через $g$) за счет дактилоскопического
вычислительного узла, в котором остает\-ся $h =1-g$ исходных вычислительных
мощ\-ностей. Если пренебречь эффектом потерь рас\-па\-рал\-ле\-ли\-ва\-ния (которые достаточно
малы при биометрической идентификации), требуется отдельно вы\-чис\-лять скорости
сравнения с учетом изменения %\linebreak 
вычис\-ли\-тель\-ной мощности. Обозначим среднее время
биометрических сравнений на вычислительном узле единичной эталонной
про\-из\-во\-ди\-тель\-ности через $\tau^i_{\mathrm{face}}$ и $\tau^i_{\mathrm{finger}}$
соответственно, тогда выражение~(\ref{e6ush}) модифицируется следующим образом:
     \begin{multline}
     a(g,A_1,R_1) ={}\\
     {}= \fr{\tau^i_{\mathrm{face}}}{g\tau^i_{\mathrm{finger}}}+
     \fr{ \mathrm{FAR}_{\mathrm{face}}(R_1) -
\mathrm{FAR}_{\mathrm{face}}(A_1)}{h}\,,
     \label{e7ush}
     \end{multline}
где $ a(g,A_1,R_1)$~--- изменение производительности с поправкой на
перераспределение  $g$  вычислительных мощностей.

     Учитывая реальные требования к современным системам биометрической
идентификации,  вероятность перехода в узел~\textit{2} (см.\ рис.~\ref{f5ush}) будет очень
редким событием, поскольку на ошибку второго рода накладываются серьезные
ограничения $\mathrm{FAR}_{\mathrm{face}} (A_1)< 10^{-3}\div 10^{-8}$, из которого
определяется допустимое значение порога~$A_1$. Соответственно в чужих сравнениях
вероятность перехода практически нулевая и не влияет на среднее время. Поэтому
следует отметить, что в большинстве приложений при оптимизации
производительности придется отказаться от возможности принятия положительного
решения на основе идентификации только по изображению лица, так как прирост
минимальный, а риски неправильной оценки $\mathrm{FAR}_{\mathrm{face}}$~---
высоки. В то же время порог $R_1$ будет регулироваться другим ограничением:
максимально допустимым уровнем ошибки 1-го рода,
$\mathrm{FRR}_{\mathrm{face}}(R_1)\approx 0$.


     При определении распределения вычислительных мощностей наибольший
прирост производительности наблюдается в точке~$g_{\mathrm{opt}}$
минимума~(\ref{e7ush}), которая при фиксированных $A_1$ и $R_1$ может быть
найдена из соотношения
     \begin{equation}
     \fr{\partial a}{\partial g}\left ( g_{\mathrm{opt}} \right ) =0\,.
     \label{e8ush}
     \end{equation}

     Преобразуя~(\ref{e8ush}), получаем следующее уравнение для $g_{\mathrm{opt}}$:
     
     \noindent
     \begin{multline*}
     \fr{\partial a}{\partial g}\left (g_{\mathrm{opt}}\right ) =
 -
\fr{\tau^i_{\mathrm{face}}}{g^2_{\mathrm{opt}} \tau^i_{\mathrm{finger}}}+{}\\
{}+\fr{
\mathrm{FAR}_{\mathrm{face}}(R_1)-\mathrm{FAR}_{\mathrm{face}}(A_1)}
{h^2_{\mathrm{opt}}} =0\,.
%     \label{e9ush}
     \end{multline*}

     Соответственно получаем следующее условие на $g_{\mathrm{opt}}$:
     \begin{multline}
     \fr{\tau^i_{\mathrm{finger}}\left ( \mathrm{FAR}_{\mathrm{face}} (R_1)-
\mathrm{FAR}_{\mathrm{face}}(A_1)\right )}{ \tau^i_{\mathrm{face}}} ={}\\
{}= \fr{(1-
g_{\mathrm{opt}})^2}{g^2_{\mathrm{opt}}} = \left ( \fr{1}{g_{\mathrm{opt}}} -1\right )^2\,.
     \label{e10ush}
     \end{multline}
     Так как $g_{\mathrm{opt}}$ является долей вычислительных средств,
выделенных на идентификацию лица, значение $g_{\mathrm{opt}}$ находится в
интервале от 0 до 1. Поэтому из~(\ref{e10ush}) находим единственное решение:
     \begin{gather}
     \fr{1}{g_{\mathrm{opt}}}  ={}\hspace*{68mm}\notag\\
     {}=  1+\sqrt{\fr{\tau^i_{\mathrm{finger}}
(\mathrm{FAR}_{\mathrm{face}}(R_1)-
\mathrm{FAR}_{\mathrm{face}}(A_1))}{t^i_{\mathrm{face}}}}\,;\label{e11ush}\\
     \sqrt{\fr{\tau^i_{\mathrm{finger}} (\mathrm{FAR}_{\mathrm{face}}(R_1)-
\mathrm{FAR}_{\mathrm{face}}(A_1))}{t^i_{\mathrm{face}}}}  =
\fr{h_{\mathrm{opt}}}{g_{\mathrm{opt}}}\,.\label{e12ush}
     \end{gather}
     Так как на краях интервала
     $$
     \underset{g\rightarrow 0}{\lim} a(g,A_1,R_1)=+\infty$$
     и
     $$ \underset{g\rightarrow
1}{\lim} a(g,A_1,R_1)=+\infty\,,
     $$
      единственная точка экстремума является искомым минимумом.
Подставляя~(\ref{e11ush}) и~(\ref{e12ush}) в~(\ref{e7ush}), получаем следующее
     минимальное значение функции~$a$:
     \begin{multline*}
     a(g_{\mathrm{opt}}, A_1 , R_1) = \fr{\tau^i_{\mathrm{face}}}{g_{\mathrm{opt}}
t^i_{\mathrm{finger}}}+{}\\
{}+
\fr{\sqrt{\mathrm{FAR}_{\mathrm{face}}(R_1)-
\mathrm{FAR}_{\mathrm{face}}(A_1)}}{g_{\mathrm{opt}}}\sqrt{\fr{\tau^i_{\mathrm{face}}}{\tau^i_{\mathrm{finger}}}}=
 \left ( \vphantom{\sqrt{\fr{\tau^i_{\mathrm{finger}} (\mathrm{FAR}_{\mathrm{face}}
(R_1)-\mathrm{FAR}_{\mathrm{face}}(A_1))}{\tau^i_{\mathrm{face}}}}}
 1+{}\right.\\
 {}+\left.\sqrt{\fr{\tau^i_{\mathrm{finger}} (\mathrm{FAR}_{\mathrm{face}}
(R_1)-\mathrm{FAR}_{\mathrm{face}}(A_1))}{\tau^i_{\mathrm{face}}}} \right )\!
 \left (
\fr{\tau^i_{\mathrm{face}}}{\tau^i_{\mathrm{finger}}}+{}\right.\\
{}+\left.\sqrt{\mathrm{FAR}_{\mathrm{face}}(R_1)-
\mathrm{FAR}_{\mathrm{face}}(A_1)}\sqrt{\fr{\tau^i_{\mathrm{face}}}{\tau^i_{\mathrm{finger}}}}\,\right )\,.
%     \label{e13ush}
     \end{multline*}
     
\begin{figure*} %fig5
\vspace*{1pt}
\begin{center}
\mbox{%
\epsfxsize=166.305mm
\epsfbox{ush-6.eps}
}
\end{center}
\vspace*{-12pt}
\Caption{Ошибки распознавания по изображению лица (внутри помещения, съем биометрии в разные дни)
по данным~[10]: (\textit{а})~Fig.~L.18, стандартное
разрешение; (\textit{б})~Fig.~L.48, высокое разрешение. 
Производители систем распознавания по изображению лица:
\textit{1}~--- Eyematic; \textit{2}~--- AcSys; \textit{3}~--- Cognitec; \textit{4}~--- C-VIS; \textit{5}~---
DreamMIRH; \textit{6}~--- Iconquest; \textit{7}~--- Identix; \textit{8}~--- Imagis; \textit{9}~--- Visage;
\textit{10}~--- VisionSphere
\label{f6ush}}
\vspace*{12pt}
\end{figure*}

     Обозначая 
     \begin{align*}
     c^2 &= \fr{\tau^i_{\mathrm{face}}}{\tau^i_{\mathrm{finger}}}\,;\\[3pt]
     d^2(A_1, R_1) &= \mathrm{FAR}_{\mathrm{face}} (R_1) -\mathrm{FAR}_{\mathrm{face}}(A_1)
     \end{align*}
получаем следующее выражение для оптимального значения изменения
производительности:
     \begin{multline}
     a(g_{\mathrm{opt}}, A_1, R_1 ) = \left ( 1+\fr{d}{c}\right ) \left ( c^2+dc\right ) ={}\\[3pt]
     {}=
\left ( c+d\right )^2\,.
     \label{e14ush}
     \end{multline}

     \section{Моделирование}

     Для оценки возможного прироста производительности исследуем статистику
тестирований биометрических технологий NIST FRVT2006, \mbox{NIST} \mbox{FRVT2002} (лицевая
биометрия) и \mbox{NIST} \mbox{FpVTE} (отпечатки пальцев)~[10--12]. Согласно протоколам
тес\-ти\-ро\-ва\-ний, производительность современных сис\-тем распознавания фронтальных
лиц %\linebreak
 колеб\-лет\-ся в интервале от 50\,000 до 330\,000 сравнений в секунду в расчете на
одноядерный процессор тактовой частоты 2,5~ГГц. Аналогичная характеристика для
отпечатков пальцев составляет примерно 5000~сравнений в секунду. Соответственно
на основании этих данных мы имеем следующую оценку для параметра $c$ изменения
про\-из\-во\-ди\-тель\-ности~(\ref{e14ush}):

\noindent
     \begin{align*}
     c^2 & = \fr{\tau^i_{\mathrm{face}}}{\tau^i_{\mathrm{finger}}} = 0{,}03\div
0{,}1\,;\\[2pt]
     c & =  0{,}123\div 0{,}316\,.
     \end{align*}

     Для определения возможных порогов $A_1$ и $R_1$, влияющих на оценку
параметра $d$, проанализируем динамику зависимости FAR и FRR (рис.~\ref{f6ush}) 
от порога. Напомним, что порог $R_1$ определяется из
условия $\mathrm{FRR}_{\mathrm{face}}(R_1)\approx 0$, порог~$A_1$~--- из условия 
$\mathrm{FAR}_{\mathrm{face}}(A_1) < 10^{-3}\div 10^{-8}$ соответственно.



     На основании приведенных графиков получаем, что предложенная схема
ускорения дает следующее улучшение производительности:
     \begin{align*}
     d^2 & = \mathrm{FAR}_{\mathrm{face}}(R_1) -
\mathrm{FAR}_{\mathrm{face}}(A_1) =0{,}05\div 0{,}2\,;\\[2pt]
     d & = 0{,}223\div 0{,}447\,;\\[2pt]
     c+d & = 0{,}346\div 0{,}763\,;\\[2pt]
     (c+d)^2 & = 0{,}119\div 0{,}582\,.
     \end{align*}

     Таким образом, получаем, что среднее время сравнения интегрированной
идентификационной системы составляет $0{,}119\div 0{,}582$ от АДИС, построенной
на вычислительных средствах той же производительности. Соответственно скорость
идентификации возрастает в 2--10~раз в зависимости от качества используемой
лицевой биометрии.

     С точки зрения перспективы соотношение $c=$\linebreak $=\;0{,}123\div 0{,}316$ вряд ли
улучшится, при этом прирост качества автоматического распознавания лиц может
довести $d^2$ до уровня в 1\%, что дает предел ускорения примерно в 20~раз.

\vspace*{-6pt}

\section{Заключение}

     Предложенный метод интеграции АДИС и лицевой биометрии позволяет
значительно увеличить скорость мультибиометрической идентификации.
     В качестве преимуществ разработанных методов отметим следующие:
     \begin{itemize}
\item при современных технологиях распознавания разработанная информационная
технология позволяет ускорить идентификацию в 2--10 раз;
\item алгоритмы интеграции не зависят от специфики используемых технологий
распознавания по отпечаткам пальцев и лицу;
\item алгоритмы интеграции могут быть реализованы на <<верхнем>> уровне
биометрического API (Application Program Interface~--- прикладной программный
интерфейс).
\end{itemize}

     В качестве направления дальнейших работ можно выделить обобщение
разработанных подходов на общий случай мультибиометрической идентифи\-кации.

\vspace*{-6pt}

{\small\frenchspacing
{%\baselineskip=10.8pt
\addcontentsline{toc}{section}{Литература}
\begin{thebibliography}{99}

\bibitem{1ush}
\Au{Ushmaev~O.\,S., Novikov~S.\,O.}
Integral criteria for large-scale multiple fingerprint solutions~//
Biometric technology for human identification~/ Eds. A.\,K.~Jain,
N.\,K.~Ratha.  SPIE Proceedings, 2004.
Vol.~5404. P.~534--543.

\bibitem{4ush} %2
\Au{Dizard III, Wilson~P.}
FBI plans major database upgrade~// Government Computer News. 
{\sf http://www.gcn.com/print/25\_26/41792-1.} {\sf html?page=1}.


\bibitem{3ush}
\Au{Болл~Р.\,М., Коннел~Дж.\,Х., Панканти~Ш., Ратха~Н.\,К., Сеньор~Э.\,У.}
 Руководство по биометрии.~--- М.: Техносфера, 2007.
 
 \bibitem{2ush} %4
\Au{Ушмаев~О.\,С.,  Босов~А.\,В.}
Реализация концепции многофакторной биометрической
идентификации в интегрированных аналитических системах~//
Бизнес и безопас\-ность в России,  2008. №\,49. С.~104--105.

\bibitem{5ush}
\Au{Ушмаев~О.\,С., Синицын~И.\,Н.}
Опыт проектирования многофакторных биометрических систем~//
Тр.\ \mbox{VIII} международной научно-технической конференции
<<Кибернетика и высокие
технологии XXI века>>. Т.~1. С.~17--28.

\bibitem{6ush}
\Au{Синицын И.\,Н., Новиков~С.\,О., Ушмаев~О.\,С.}
Развитие технологий интеграции биометрической
информации~// Системы и средства информатики, 2004. Вып.~14. С.~5--36.

\bibitem{7ush}
\Au{Ushmaev~O., Novikov~S.}
Biometric fusion: Robust Approach~// MMUA 06 Proceedings. Toulouse, France,
May 11--12, 2006. 

\bibitem{8ush}
\Au{Ушмаев~О.\,С., Босов~А.\,В.}
Реализация концепции многофакторной биометрической
идентификации в интегрированных аналитических системах~ //
Системы высокой доступности, 2007. 4,  Т.~3. С.~13--23.

\bibitem{9ush}
\Au{Ушмаев~О.\,С., Синицын~И.\,Н.}
Программная реализация мультибиометрической идентификации в
интегрированных аналитических приложениях~ // Тр. IX международной научно-технической
конференции <<Кибернетика и высокие технологии XXI~века>>.
Воронеж, 13--15 мая 2008. Т.~2. С.~735--746.

\bibitem{10ush}
NIST FRVT2002. Evaluation Report. {\sf
http://www.} {\sf frvt.org/FRVT2002/}.

\bibitem{11ush}
NIST FRVT2006. Evaluation Report. {\sf
http://www.} {\sf frvt.org/FRVT2006/}.

\label{end\stat}

\bibitem{12ush}
NIST FpVTE. Evaluation Report.  {\sf
http://fpvte.nist.gov}.
\end{thebibliography}

}
}
\end{multicols}