
\def\stat{kozh}

\def\tit{EUROWORDNET: ЗАДАЧИ, СТРУКТУРА И ОТНОШЕНИЯ}
\def\titkol{EuroWordNet: задачи, структура и отношения}

\def\autkol{О.\,С.~Кожунова}
\def\aut{О.\,С.~Кожунова}

\titel{\tit}{\aut}{\autkol}{\titkol}

%{\renewcommand{\thefootnote}{\fnsymbol{footnote}}\footnotetext[1]
%{Работа выполнена при поддержке
%Российского фонда фундаментальных исследований,
%гранты 06-07-89056 и 08-07-00152.}}

\renewcommand{\thefootnote}{\arabic{footnote}}
\footnotetext[1]{Институт проблем
информатики Российской академии наук, okozhunova@ipiran.ru}

\Abst{В обзоре приведено краткое описание ресурса EuroWordNet, история его создания, примеры аналогичных ресурсов,
а также его задачи, структура и отношения, реализованные в его инструментарии.}

\KW{лексико-семантический ресурс EuroWordNet; <<Ворднет-словари>> (WordNet); 
тезаурус; синсет; языковые отношения; межъязыковой индекс ILI}

      \vskip 33pt plus 9pt minus 6pt

      \thispagestyle{headings}

      \begin{multicols}{2}

      \label{st\stat}

\section{Введение}
     
     Сегодня для решения задач компьютерной лингвистики привлекается 
большое количество лексических ресурсов, используемых в сфере 
информационных технологий.
     
     Одним из наиболее распространенных типов таких ресурсов являются 
автоматизированные словари, построенные по модели <<Ворднет>> (WordNet)~\cite{1koz}. 
Словари типа <<Ворднет>> объединяют в себе черты справочной 
системы и инструмента для проведения лингвистических исследований.
     
     В частности, при проведении информационного поиска 
     <<Ворд\-нет-сло\-ва\-ри>> удобно использовать для расширения запроса 
пользователя за счет парадигматически и синтагматически связанных слов, 
например компонентов синсета (множества синонимов, объединенных в набор) 
вместе с его гипо\-ни\-ма\-ми и согипонимами или %\linebreak 
связей типа <<гла\-гол--ак\-тант>>, 
которые дают возможность осуществлять контекстный поиск. Данные о 
синтагматических отношениях слов позволяют применять 
     <<Ворд\-нет-сло\-ва\-ри>> для решения %\linebreak 
     задачи снятия неоднозначности 
смысла слова. <<Ворднет>> можно использовать для вычисления смысловой 
близости текстов на основе гиперонимических отношений. 
     <<Ворд\-нет-сло\-ва\-ри>> могут служить лексиконом для формальных 
грамматик. Формат <<Ворднет>> является удобным формализмом для 
представления соста\-ва и структуры %\linebreak 
лексики специальных подъязыков 
(например, медицинских, экономических тер\-минов). 
{\looseness=1

}
     
     <<Ворд\-нет-сло\-ва\-ри>> являются удобным инструментом для проведения 
исследований в об\-ласти лексической семантики. Например, гипонимические 
отношения в <<Ворд\-нет-сло\-ва\-рях>> позволяют определять направление 
метонимических переносов и прогнозировать появление новых лек\-си\-ко-се\-ман\-ти\-че\-ских вариантов~\cite{2koz}.
     
     Проект по разработке словаря Princeton WordNet (PWN) английского 
языка в Принстонском университете (США) стартовал в первой половине 
1980-х~гг.\  и продолжается по сей день. Сейчас уже доступна версия 
WordNet~2.0. Существующая версия охватывает более 120~тыс.\ слов 
общеупотребительной лексики современного английского языка~\cite{3koz}. 
     
     За период с марта 1996~г.\ по сентябрь 1999~г.\ при финансировании 
Европейской комиссии был создан многоязычный вариант WordNet~--- 
EuroWordNet~\cite{4koz}, что стало новым этапом в эволюции 
     <<Ворд\-нет-сло\-ва\-рей>>. В рамках европейского проекта \mbox{было} создано не 
только несколько тезаурусов для европейских языков (голландского, 
испанского, итальянского, немецкого, французского, %\linebreak
 чешского и эстонского), 
но и впервые была реализована идея об объединении отдельных 
     <<Ворд\-нет-пред\-став\-ле\-ний>> в общую систему. Все компоненты 
EuroWordNet были построены по единой модели, что, однако, не предполагало 
прямого перевода английского варианта WordNet~1.5, перед разработчиками 
стояла задача~--- отразить все особенности лексических систем национальных 
языков. Со\-вмес\-ти\-мость компонентов EuroWordNet была обеспечена единством 
принципов и заданным набором общих понятий (Basic Concepts), на основе 
которых определялась система межъязыковых отсылок (Inter-Lingual-Index, ILI), 
дающих возможность переходить от лексикализованных значений одного 
языка к сходным, но не обязательно тождественным значениям в другом языке. 
Данный индекс позволяет использовать EuroWordNet не только для 
информационного поиска в рамках одного языка, но и для многоязычного 
поиска.
     
     В рамках проекта EuroWordNet первоначальная структура словаря 
претерпела серьезные изменения. Был расширен набор семантических 
отношений за счет парадигматических отношений, связывающих слова разных 
час\-тей речи (например, XPOS\_NEAR\_SYNONYMY: dead~--- death; 
XPOS\_HYPERONYMY: to love~--- emotion; XPOS\_ANTONYMY: to live~--- 
dead) и синтагматических отношений между глаголами и 
     ак\-тан\-та\-ми-су\-ще\-ст\-ви\-тель\-ны\-ми (например, ROLE\_INSTRUMENT: to 
write~--- pencil). Был сформирован новый подход к построению 
     <<Ворд\-нет-сло\-ва\-рей>>: с опорой на использование лек\-си\-ко\-гра\-фи\-че\-ских 
источников (толковых, переводных и синонимических словарей) и результатов 
обработки корпусов современных текстов~\cite{3koz}. 
     
     Успешное завершение проекта EuroWordNet послужило толчком к 
созданию большого числа <<Ворд\-нет-пред\-став\-ле\-ний>> для языков разных 
типов (например, венгерского, турецкого, арабского, тамильского, китайского 
и~пр.), а также многоязычных ресурсов типа EuroWordNet (например, проект 
BalkaNet нацелен на объединение греческого, румынского, болгарского, 
сербского, турецкого и чешского <<Ворд\-нет-сло\-ва\-рей>>). В 2001~г.\ была 
создана Всемирная Ассоциация WordNet (Global WordNet Association), которая 
ставит целью объединение уже существующих и только раз\-ви\-ва\-ющих\-ся 
национальных ресурсов этого типа, усовершенствование системы 
межъязыковых индексов и %\linebreak 
разработку общих стандартов, позволяющих 
использовать модель <<Ворднет>> для языков разных типов~\cite{5koz}.
{\looseness=1

}

     
     С 1999~г.\ на кафедре математической лингвистики СПбГУ 
исследовательская группа под руководством И.\,В.~Азаровой 
(О.\,А.~Митрофанова, А.\,А.~Синопальникова и~др.) ведет работы по проекту 
RussNet~--- созданию русской версии компьютерного словаря типа WordNet. В 
задачи проекта входит построение лек\-си\-ко-се\-ман\-ти\-че\-ско\-го %\linebreak 
ресурса для 
отражения организации лексической сис\-те\-мы русского языка в целом, для 
представления ядра его общеупотребительной лексики и фиксации 
семантических, се\-ман\-ти\-ко-грам\-ма\-ти\-че\-ских и се\-ман\-ти\-ко-де\-ри\-ва\-ци\-он\-ных 
отношений русского языка~\cite{3koz}. Кроме того, в настоящее время в 
Петербургском университете путей сообщения разрабатывается проект 
русской версии WordNet под руководством С.\,А.~Яблонского и 
А.\,М.~Сухоногова~\cite{6koz}. 
     
     В данной статье рассматривается именно лексико-се\-ман\-ти\-че\-ский ресурс 
EuroWordNet, поскольку, в отличие от его главной исходной версии 
WordNet~1.5 и аналогичных отдельных моделей в других языках, в нем 
реализована идея объединения отдельных тезаурусов в единую систему, 
разработан индекс перехода с одного языка на другой с учетом особенностей 
понятийной системы каждого европейского языка (т.\,е.\ осуществляется 
возможность перехода от лексикализованных значений одного языка к 
аналогичным, но не обязательно тождественным понятиям другого языка). 
Такие особенности EuroWordNet наделяют этот словарь качественно иной 
функциональностью, что обусловлено спецификой его структуры и системы 
отношений.

\section{Общая характеристика проекта EuroWordNet}

     Цель проекта EuroWordNet, как было сказано во введении, состоит в 
построении многоязычной лексической базы данных с <<Ворднетами>> для 
нескольких европейских языков, которые структурированы согласно 
принципам Princeton WordNet (далее~--- WordNet)~\cite{4koz}. WordNet 
содержит информацию о существительных, глаголах, прилагательных и 
наречиях английского языка. Одним из его базовых понятий является синсет 
(synset). Синсет представляет собой набор слов, принадлежащих к одному типу 
частей речи, которые взаимозаменямы в определенном контексте.
     
     Например, слова из множества \{тачка; авто; автомобиль; машина; 
автомашина\} формируют синсет, поскольку их можно использовать для 
обращения к одному и тому же понятию. Далее часто приводится толкование 
синсета: <<Четырехколесный; обычно работающий от двигателя внутреннего 
сгорания>>. Синсеты могут быть связаны друг с другом семантическими 
отношениями, такими как гипонимия (между конкретными и более общими 
понятиями), меронимия (между частями и целыми), причинно-следственные 
отношения и~т.\,д., как показано на рис.~\ref{f1koz}. В этом примере, который 
заимствован из WordNet~1.5~\cite{1koz, 2koz}, синсет \{тачка; авто; 
автомобиль; машина; автомашина\} связан с: 
\begin{enumerate}[(1)]
\item более общими понятиями или синсетом гиперонимов: 
\{автомобили; механическое транспортное средство\};
\item более конкретными понятиями или синсетом гипонимов: 
например, \{патруль; полицейская автомашина; патрульная 
спецмашина; полицейский автомобиль; патрульный полицейский 
автомобиль общего назначения\} и \{кеб; такси; машина для извоза; 
такси-кеб\};
\item частями, из которых он состоит: например, \{бампер\}, \{дверца 
автомобиля\}, \{автомобильное зеркало\} и \{окно автомобиля\}.
\end{enumerate}

\begin{figure*} %fig1
\vspace*{1pt}
\begin{center}
\mbox{%
\epsfxsize=164.817mm
\epsfbox{koz-1.eps}
}
\end{center}
\vspace*{-9pt}
\Caption{Синсеты, связанные с понятие <<машина>> в его первом смысле в WordNet 1.5
\label{f1koz}}
\end{figure*}

     Каждый из этих синсетов далее связан с другими синсетами. Это хорошо 
видно из примера, где синсет \{автомобили; транспортное механическое 
средство\} связан с синсетом \{транспортные средства\}, а синсет \{дверца авт.\}\ 
связан с синсетами, содержащими упоминания других частей автомобиля: 
\{подвеска\}, \{подлокотник\}, \{дв.\ замок\}~\cite{1koz}.
     
     Посредством этих и других семантических (концептуальных) отношений 
могут быть установлены связи между всевозможными значениями, 
со\-став\-ля\-ющи\-ми при этом огромную сеть или <<Ворднет>>.
     
     Такой <<Ворднет>> может быть использован для формирования 
суждений о значениях слов (какие именно значения могут быть 
интерпретированы как <<транспортные средства>>) и для осуществления 
информационного поиска. В частности, для поиска альтернативных выражений 
или формулировок или просто с целью расширения словарных множеств до 
наборов семантически связанных или близких слов. Кроме того, 
семантические сети дают информацию о лексикализованных шаблонах, о 
концептуальной плотности областей словарей и о распределении 
семантических различий или отношений по разным областям словарей 
различных языков~\cite{4koz}.
     
     Ресурсы европейских <<Ворднетов>> будут храниться в центральной 
лексической системе баз данных, причем каждое лексическое значение будет 
связано с наиболее близким синсетом принстонского WordNet~1.5, формируя 
таким образом многоязычную базу данных~\cite{1koz, 4koz}.
     
     В такой базе данных станет возможным переход от одного значения из 
какого-либо <<Ворднета>> к значению в другом <<Ворднете>>, который 
связан с аналогичным понятием из WordNet~1.5. Эта многоязычная база 
данных может стать полезной для кроссязыкового поиска информации, для 
передачи информации из одного ресурса в другой или для простого сравнения 
различных <<Ворднетов>>~\cite{7koz}. Сравнение может дать информацию о 
непротиворечивости отношений в <<Ворднетах>>, различия в которых могут 
указывать на противоречия или на специфичные для некоторого языка 
свойства ресурсов, а также на свойства самого языка~\cite{4koz}.
     
     Таким образом, базу данных можно также рассматривать как мощное 
средство для изучения лексических семантических ресурсов и их языковой 
специфики~\cite{8koz}.
     
В EuroWordNet ее разработчики~[1, 4, 7--10] первоначально имели дело с 
четырехязыковой мо\-делью. Первыми языками, для которых были разработаны 
вышеописанные ресурсы, стали голландский, итальянский, испанский и 
английский. %\linebreak
 Размер каж\-до\-го из этих <<Ворднетов>>, за исключением 
<<Ворднета>> английского языка, составляет около~30\,000~сравнимых 
синсетов, что приблизительно соответствует~50\,000~лексических значений. 
Для сравнения: размер WordNet~1.5 составляет~91\,591~синсетов и~168\,217~значений слов. 
В~качестве развития проекта его база данных была расширена 
соответствующими ресурсами немецкого, французского, эстонского и 
чешского языков. Размер этих <<Ворднетов>> колеблется в диапазоне от~7\,500 до~15\,000~синсетов.
     
     Состав <<Ворднетов>> ограничен существительными и глаголами, хотя 
и прилагательные, и наре\-чия включены при существовании связи с 
существительными и глаголами. В словарь системы %\linebreak
 \mbox{будут} включены все 
общие и основные слова языков. Таким образом, он будет включать все значения и 
понятия, которые необходимы для связывания более конкретных значений и 
всех слов, наиболее часто встречающихся в общих текстовых кор\-пу\-сах. 
Для одной из областей будет добавлен вспомогательный словарь для 
иллюстрации возможности объединения различной терминологии в таком 
универсальном словаре.
     
\vspace*{-3pt}
\section{Комплексная модель базы данных EuroWordNet}
\vspace*{-2pt}
     
     Разработка базовых семантических ресурсов представляет собой 
нетривиальную задачу. Смысл и его толкование изучаются множеством 
дис\-цип\-лин и парадигм с различными точками зрения и подходами. Широко 
распространено мнение, что роль общей семантики, или изучения процессов 
понимания языка, все еще чрезмерно сложна для ее описания, адекватного 
современным технологиям. Поэтому цель проекта EuroWordNet состоит не  в 
том, чтобы разрабатывать полные семантические словари, которые используют 
сложные языковые представления и механизмы логического вывода, а в том, 
чтобы свести определенную языковую информацию и механизмы к тем 
базовым семантическим отношениям между словами, которые хорошо 
изучены~[1, 4, 10].
     
Семантические отношения, встроенные в WordNet~1.5 признаны во 
многих научных областях: формальной семантике, искусственном интеллекте, %\linebreak 
когни\-тив\-ной лингвистике, лексикологии, информатике и 
математике~\cite{1koz}. Кроме того, отношения не основываются на 
     ка\-ком-ли\-бо специфичном формализме представления знаний. 
Предполагается, что они будут основой любой системы знаний 
будущего~\cite{4koz}. Даже при том, что реляционный подход к смыслу в 
WordNet  и EuroWordNet нельзя рас\-смат\-ри\-вать как полное описание смысла 
слова, тем не менее в этих ресурсах предусмотрена возможность расширения 
такой базовой информации более исчерпывающими описаниями с частичным %\linebreak 
использованием кор\-пус\-но-текс\-то\-вых технологий~[7--9]. Более того, в работе 
Sanfilippo~\cite{11koz} продемонстрировано, что доступность 
     <<Ворд\-нет-струк\-тур>> очень полезна для автоматического извлечения 
такой информации из корпусов текстов.
     
     Таким образом, проект EuroWordNet-database~--- это первый проект из 
всевозможных аналогичных разработок, базирующихся на структуре PWN, 
в частности на версии WordNet~1.5. Из этой версии для проекта 
EuroWordNet были заимствованы понятие синсета и базовые семантические 
отношения~\cite{4koz}.
     
     Однако в проект базы данных были внесены определенные изменения, 
которые в основном мотивированы следующими факторами~\cite{4koz}:
     \begin{enumerate}[(1)]
     \item  стремлением создать многоязычную базу данных;
\item потребностью в поддержании ориентированных на конкретный 
язык отношений в <<Ворднет>>;
\item стремлением достигнуть максимальной со\-вмес\-ти\-мости с другими 
ресурсами;
\item необходимостью в построении <<Ворднетов>> с использованием 
существующих ресурсов в относительно независимом режиме.
     \end{enumerate}
     Поэтому важнейшим отличием EuroWordNet от WordNet  является его 
многоязычность, в связи с которой, тем не менее, также возникают некоторые 
принципиальные вопросы в отношении статуса моноязычной информации, 
отраженной в <<Ворднетах>>~\cite{7koz}. Многоязычность достигается, 
главным образом, добавлением отношения эквивалентности для каждого 
синсета некоторого языка с наиболее близким ему синсетом из WordNet~1.5. 
Предполагается, что синсеты, связанные с одним и тем же синсетом из 
WordNet~1.5, эквивалентны или близки по смыслу, а следовательно, сравнимы. 
Однако возникает проблема различий <<Ворднетов>>. Если 
<<эквиалентные>> слова по-разному связаны в различных ресурсах, то 
необходимо принять решение о законности этих различий~\cite{4koz}. 
Например, в голландском <<Ворднете>> hond (``dog'' в WordNet~1.5) 
классифицируется и как huisdier (``pet'' в WordNet~1.5), и как zoogdier 
(``mammal'' в WordNet~1.5). Тем не менее в итальянском языке нет никакого 
эквивалента для <<домашнего животного>> (``pet'' в WordNet~1.5), а слово 
``cane'', которое связано с тем же самым синсетом ``dog'', в итальянском 
<<Ворднете>> классифицируется только как ``mammal''~\cite{4koz}.
     
     Разработчики EuroWordNet придерживаются мнения, что необходимо 
сделать все возможное для отражения подобных различий в лексических 
семантических отношениях. <<Ворднеты>> скорее рассматриваются в 
качестве лингвистических онтологий, чем только как онтологии для 
производства выводов~\cite{1koz, 8koz}. В онтологии, основанной на выводах, 
может быть так, что для достижения лучшего управления, или 
результативности, или компактной и последовательной структуры необходим 
определенный подход к структурированию. В связи с этим может возникнуть 
необходимость во введении искусственных уровней для понятий, которые не 
выражены словами в языке (например, природные объекты, внешние части 
тела), или в исключении уровней (например, сторожевой пес), которые 
представлены словами языка, но нерелевантны целям онтологии. С другой 
стороны, лингвистическая онтология точно отражает лексикализацию\linebreak\vspace*{-12pt}
\pagebreak
\end{multicols}

\begin{figure} %fig2
\vspace*{1pt}
\begin{center}
\mbox{%
\epsfxsize=157.23mm
\epsfbox{koz-2.eps}
}
\end{center}
\vspace*{-12pt}
\Caption{Лексикализованные и нелексикализованные уровни в <<Ворднетах>>
\label{f2koz}}
\vspace*{9pt}
\end{figure}

\begin{multicols}{2}

\noindent
 (т.\,е.\ 
представление понятий словами) и отношения между словами в языке. Это 
<<Ворднет>> в истинном смысле этого слова и, следовательно, охватывает 
ценную информацию о представлениях, которые лексикализованы в языке, а 
именно информацию о том, каков объем доступных слов и выражений 
некоторого языка~\cite{4koz}. 

    
     Данное отличие проиллюстрировано на рис.~\ref{f2koz}, где 
гипонимическая структура WordNet~1.5 содержит сочетание 
лексикализованных и нелексикализованных категорий, в то время как 
голландский <<Ворднет>> содержит только лексикализованные категории 
языка~\cite{1koz, 4koz, 11koz}. 
     
     
     На примере фрагмента WordNet~1.5, приведенного на рис.~\ref{f2koz}, 
видно, что синсет для понятия <<объект>> (object) сначала подразделяется на 
два подкласса <<артефакт>> (artefact) и <<природный объект>> %\linebreak
 (natu\-ral object), 
из которых последний является нелексикализованным выражением в 
английском языке (которое должно быть в словаре языка в качестве заглавного 
слова), а скорее представляет собой построенное в соответствии с правилами 
языка выражение~\cite{4koz}. У класса <<артефакт>> (artefact) есть важный 
подкласс <<инструментарий>> (instrumentality), который используется для 
группировки таких взаимосвязанных синсетов, как <<орудие>> (implement), 
<<устройство>> (device), <<инструмент>> (tool) и <<средство>> (instrument). 
Создается впечатление, что такая группировка может быть полезной для 
организации иерархии и прогнозирования функциональности подклассов. Тем 
не менее она не обеспечивает корректными прогнозами о вза\-и\-мо\-за\-ме\-ня\-емости 
существительных, т.\,е.\ нель\-зя обратиться к существительным 
<<контейнеры>> (containers), <<коробки>> (boxes), <<ложки>> (spoons) и 
<<мешки>> (bags), используя существительное <<инструментарий>> 
(instrumentality) английского языка~\cite{1koz, 4koz}. 
{\looseness=1

}

     
     На примере фрагмента голландской иерархии, приведенного на 
рис.~\ref{f2koz}, видно, что такие искусственные уровни, как <<природный 
объект>> (natural object) и <<инструментарий>> (instrumentality) использованы 
не были. Кроме того, точных эквивалентов для английских ``artifact''  и 
``container'' в голландском языке нет. В результате получается гораздо более 
плоская иерархия, в которой такие специфичные свойства, как 
<<натуральный>> (natural), <<искусственный>> (artificial) и 
<<функциональность>> (functionality) не могут быть выведены из отношений 
гипонимии~\cite{1koz, 4koz}. 
%\pagebreak
     
     С другой стороны, такая сеть предоставляет корректные прогнозы 
выразительных возможностей словаря голландского языка, поскольку 
включает в себя только допустимые для языка слова (и составные выражения). 
Можно было бы построить новые классы и выражения голландского языка, 
чтобы охватить различные обобщения в языке, но априорного критерия для 
определения полезных или ненужных классов не существует~\cite{4koz}. 
Можно было бы, наконец, добавить в такую сеть какое-либо потенциальное 
семантическое свойство в виде нового класса с целью создания детальных 
структур наследования или  заимствовать всевозможные классификации из 
всех других <<Ворднетов>>. Однако подобные модификации уничтожили бы 
данный <<Ворднет>> как сеть допустимых выражений языка. Кроме того, эти 
изменения не гарантируют наличия хорошей концептуальной онтологии для 
наследования свойств~\cite{4koz}. 
     
     Далее можно расширять базу данных посредством отдельных, 
нейтральных в плане используемых языков или специализированных 
онтологий, которые содержат адекватные механизмы вывода и хорошо 
спроектированы для этой цели. Если такая онтология привязана к базе данных 
EuroWordNet, все <<Ворднеты>> могут получать доступ к этим 
классификациям с целью поиска правильных выводов для их синсетов. В 
таком случае <<Ворднеты>> обеспечивают точное отображение 
ориентированного на конкретный язык словаря этой онтологии~\cite{4koz, 
8koz}. 
{\looseness=1

}
     
     Для поддержания целостности ориентированных на конкретный язык 
структур и отдельной модификации независимых ресурсов разработчики 
EuroWordNet проводят различие между ориентированными на конкретный 
язык модулями и отдель\-ным независимым от языка модулем.  Каж\-дый \mbox{модуль} 
языка представляет собой автономную и %\linebreak
уникальную, ориентированную на 
конкретный язык систему внутриязыковых отношений между синсетами. 
Отношения эквивалентности между синсетами различных языков и синсетами 
WordNet~1.5 предельно четко выражены при помощи так называемого 
межъязыкового индекса (ILI). %\linebreak 
Каж\-дый синсет в 
моноязычных <<Ворднетах>> будет \mbox{связан,} по крайней мере, одним 
отношением эквивалентности с записью данного ILI. Поэтому 
ориентированные на конкретный язык синсеты, которые связаны с той же 
самой записью ILI, должны быть эквивалентными для всех языков. Это 
проиллюстрировано на рис.~\ref{f3koz} примером для ориентированных на 
конкретный язык синсетов, связанных с записью ILI ``drive'' (ехать, 
вести)~[4, 7--9]. 
     
     На рис.~\ref{f3koz} приведено схематическое представление различных 
модулей и отношений между ними. В середине рисунка показаны модули, 
общие для всех языков: ILI (межъязыковой индекс), Онтология предметной 
области и Онтология понятий верхнего уровня. ILI состоит из списка так 
называемых записей ILI (ILIRs), которые связаны со значениями слов из 
модулей конкретных языков, (возможно) с одним или более понятий верхнего 
уровня и (возможно) с предметными областями. Таким образом, модули 
конкретных языков состоят из схемы лексических единиц этих языков, 
связанных индексами с наборами синсетов, между которыми построены 
внутриязыковые отношения~\cite{4koz, 7koz}. 
     
Индекс ILI представляет собой неструктурированный список значений, 
заимствованный в основном из WordNet~1.5, где каждая запись ILI %\linebreak
 со\-стоит из 
некоторого синсета, толкования на английском языке, определяющего 
конкретное значение, и ссылку на его источник. Единственная цель ILI 
заключается в связывании синсетов из ориентированных на конкретный язык 
<<Ворднетов>>. Поэтому между записями ILI  никаких отношений как 
таковых не предусмотрено~\cite{4koz}. 
     
     Некоторые из процедур структурирования ILI, независимого от выбора 
языка, обеспечиваются двумя отдельными онтологиями, которые могут быть 
связаны с записями ILI~\cite{4koz, 9koz}: 
     \begin{itemize}
\item онтологией понятий верхнего уровня, которая представляет собой 
иерархию независимых от выбора языка понятий, отражая важные 
семантические различия (например,  различия между понятиями 
<<объект>> и <<вещество>>, <<динамический>> и <<статический>>); 
\item иерархией наименований предметных областей, которые являются 
структурами знаний, группирующими значения в терминах различных 
сфер или ситуаций (например, <<движение>>, <<дорожное 
движение>>, <<воздушное движение>>, <<спорт>>, <<больница>>, 
<<ресторан>>).
     \end{itemize}
     
     Как понятия верхнего уровня, так и  наименования предметных областей,  
могут быть переданы при помощи отношений эквивалентности между 
записями ILI и языковыми значениями конкретного языка, что 
проиллюстрировано на рис.~\ref{f3koz}. 


\begin{figure*} %fig3
\vspace*{1pt}
\begin{center}
\mbox{%
\epsfxsize=163.942mm
\epsfbox{koz-3.eps}
}
\end{center}
\vspace*{-9pt}
\Caption{Архитектура базы данных EuroWordNet
\label{f3koz}}
\end{figure*}

     Например, понятия верхнего уровня <<позиция, местоположение>> 
(location) и <<динамический>> (dynamic) непосредственно связаны с записью 
ILI <<ехать, везти>> (drive). Следовательно, они также имеют косвенное 
отношение ко всем понятиям конкретных языков, связанных с этой записью 
ILI. Далее посредством внутриязыковых отношений понятия верхнего уровня 
могут наследоваться всеми другими соответствующими понятиями 
конкретных языков~\cite{4koz}. 
     
     Основная цель онтологии  верхнего уровня состо\-ит в формировании 
общей структуры для важней\-ших понятий всевозможных <<Ворднетов>>. %\linebreak 
Данная онтология содержит 63 основных семантических различия, которые 
классифицируют множество 1024 записей ILI, представляющих важнейшие 
понятия из различных <<Ворднетов>>~\cite{9koz}. 
     
     Наименования предметных областей могут быть применены 
непосредственно при информационном поиске (а также в инструментариях для 
изучения языка и при публикации словарей) к групповым понятиям, 
основанным скорее на предписаниях, чем на классификациях. Предметные 
области также могут быть использованы для разделения словарей общего 
назначения и словарей конкретных областей. Это играет важную роль при 
решении проблемы двусмысленности при обработке естест\-вен\-но-язы\-ко\-вых 
текстов~\cite{7koz}. 
     
     В проекте EuroWordNet онтология предметной области будет построена 
только для небольшого фрагмента словаря  с целью иллюстрации. Тем не 
менее пользователи базы данных проекта смогут свободно добавлять 
наименования предметных областей к межъязыковому индексу ILI или 
уточнять понятия онтологии верхнего уровня без доступа или рассмотрения 
внутриязыковых отношений каждого <<Ворднета>>. Таким же образом 
возможно расширение этой базы данных при помощи других онтологий при 
условии, что они определены в соответствии с форматом EuroWordNet и 
содержат надлежащую ссылку на межъязыковой индекс ILI~\cite{9koz}. 

     
     \section{Заключение}
     
     Описанный в статье модульный многоязычный проект EuroWordNet 
имеет следующие достоинства~\cite{4koz}:
\begin{itemize}
\item возможность использования базы данных для многоязычного 
поиска информации посредством расширения набора слов одного языка 
соответствующими словами другого языка через межъязыковой индекс 
ILI; 
\item различные <<Ворднеты>> могут быть подвергнуты сравнению и 
кросслингвистической проверке, что сделает их более совместимыми; 
\item особенности конкретных языков можно реализовать в отдельных 
<<Ворднетах>>; 
\item возможность разработки <<Ворднетов>> на различных сайтах 
Интернета в относительно независимом режиме; 

\item независимая от выбора исходного языка информация, такая как 
толкования, знания предметной области и аналитические понятия 
верхнего уровня, может быть сохранена только один раз и может стать 
доступной для всех модулей, предназначенных для определенных 
языков, посредством поддержания межъязыковых отношений; 
\item база данных проекта может быть адаптирована к потребностям 
пользователей при помощи модификации понятий верхнего уровня, 
наименований предметных областей или их сущностей (например, 
добавлением семантических признаков) без необходимости доступа к 
конкретным языковым <<Ворднетам>>. 
\end{itemize}

     Помимо многоязычного дизайна базы данных проекта были 
осуществлены некоторые изменения во внутриязыковых отношениях по 
отношению к исходному проекту WordNet~1.5, а именно~\cite{4koz, 7koz}:
     \begin{enumerate}[(1)]
     \item
использование наименований отношений, что делает семантические 
следования более явными и точными; 
\item введение общих отношений между частями речи, так что может 
быть найдено соответствие между другими поверхностными 
реализациями подобных понятий в самом языке и его пересечениях с 
другими языками; 
\item добавление некоторых новых отношений с \mbox{целью} поиска 
различений определенных поверхностных иерархий. 
\end{enumerate}

{\small\frenchspacing
{%\baselineskip=10.8pt
\addcontentsline{toc}{section}{Литература}
\begin{thebibliography}{99}

\bibitem{1koz}
\Au{Fellbaum C.}
WordNet: An electronic lexical database.~--- Cambridge, 1998. 

\bibitem{2koz}
\Au{Miller G., Beckwith R., Fellbaum C., Gross D., Miller K.}
Five papers on WordNet ~// CSL Report 43.~--- Princeton University, Cognitive Science Laboratory, 1990. 

\bibitem{3koz}
\Au{Азарова И.\,В., Митрофанова О.\,А., Синопальникова~А.\,А., 
Ушакова~А.\,А., Яворская~М.\,В.}
Разработка компьютерного тезауруса русского языка типа WordNet~// Доклады 
научной конференции <<Корпусная лингвистика и лингвистические базы 
данных>>~/ Под ред.\ А.\,С.~Герда.~--- СПб., 2002. С.~6--18.

\bibitem{4koz}
\Au{Vossen P.}
Introduction to EuroWordNet~// Com\-puters and the Humanities, Special Issue on EuroWordNet, 1998. Vol.~32. 
No.\,2--3.  P.~73--89. 

\bibitem{5koz}
\Au{Азарова И.\,В., Синопальникова~А.\,А., Яворская~М.\,В.}
Принципы построения wordnet-тезауруса RussNet~// Материалы конференции 
Диалог-2004.~--- М.: Наука, 2004.

\bibitem{6koz}
\Au{Сухоногов А.\,М., Яблонский~С.\,А.}
Словари типа WordNet в технологиях Semantic Web~// Девятая Национальная 
конференция по искусственному интеллекту с международным участием 
КИИ-2004. Тр.\ конференции в 3-х т.~--- М.: Физматлит, 2004. Т.~2. 
С.~557--564.

\bibitem{7koz}
\Au{Gonzalo J., Verdejo~F., Peters C., Calzolari N.}
Applying EuroWordNet to cross-language text retrieval~// Com\-puters and the 
Humanities, Special Issue on EuroWordNet, 1998. Vol.~32. No.\,2--3.  P.~185--207.

\bibitem{8koz}
\Au{Alonge A., Calzolari~N., Vossen P., Bloksma L., Castellon~I., Marti T., Peters W.}
The linguistic design of the EuroWordNet Database~// Com\-puters and the 
Humanities, Special Issue on EuroWordNet, 1998. Vol.~32. No.\,2--3.  P.~91--115.

\bibitem{9koz}
\Au{Rodriguez H., Climent~S., Vossen P., Bloksma L., Peters~W., Roventini A., Bertagna F., Alonge A.}
The top-down strategy for building EuroWordNet: Vocabulary coverage, base 
concepts and top ontology~// Com\-puters and the Humanities, Special Issue on EuroWordNet, 1998. Vol.~32. 
No.\,2--3.  P.~117--152.

\label{end\stat}

\bibitem{10koz}
\Au{Vossen P., Bloksma~L., Peters C., Alonge A., Roventini A., Marinai E., Castellon I., Marti T., Rigau G.} 
Compatibility in interpretation of relations in EuroWordNet~// Com\-puters and the 
Humanities, Special Issue on EuroWordNet, 1998. Vol.~32. No.\,2--3.  P.~153--184.

\bibitem{11koz}
\Au{Sanfilippo~A.} 
Using semantic similarity to acquire co-occurrence restrictions from corpora~// 
ACL/EACL'97 Workshop on Automatic Information Extraction 
and Building of Lexical Semantic Resources for NLP Applications Proceedings~/ 
Eds. P.~Vossen, N. Calzolari, G.~Adriaens, A. Sanfilippo, Y. Wilks.
Madrid, 1997.
\end{thebibliography}
}
}
\end{multicols}






 
 
 
 