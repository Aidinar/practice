\def\stat{sokolov}

\def\tit{ДЕЗИНТЕГРИРОВАННАЯ АРХИТЕКТУРА ПАКЕТНОЙ КОММУТАЦИИ}
\def\titkol{Дезинтегрированная архитектура пакетной коммутации}

\def\autkol{И.\,А.~Соколов, В.\,Б.~Егоров}
\def\aut{И.\,А.~Соколов$^1$, В.\,Б.~Егоров$^2$}

\titel{\tit}{\aut}{\autkol}{\titkol}

%{\renewcommand{\thefootnote}{\fnsymbol{footnote}}\footnotetext[1]
%{Работа выполнена при поддержке
%Российского фонда фундаментальных исследований,
%гранты 06-07-89056 и 08-07-00152.}}

\renewcommand{\thefootnote}{\arabic{footnote}}
\footnotetext[1]{Институт проблем информатики Российской академии наук, isokolov@ipiran.ru}
\footnotetext[2]{Институт проблем информатики Российской академии наук, vegorov@ipiran.ru}

 
\Abst{Предложена дезинтегрированная архитектура пакетной коммутации,
позволяющая создавать простые и маршрутизирующие пакетные коммутаторы с
широкими функциональными возможностями без использования
высокоинтегрированных коммуникационных микроконтроллеров.}

\KW{пакетный коммутатор; интегрированный коммуникационный микроконтроллер;
ИКМ; QUICC; PowerQUICC}

      \vskip 36pt plus 9pt minus 6pt

      \thispagestyle{headings}

      \begin{multicols}{2}

      \label{st\stat}
      
\section{Введение}

    В настоящее время у разработчиков телекоммуникационной аппаратуры, в частности
устройств пакетной коммутации, большой популярностью пользуются интегрированные
коммуникационные микроконтроллеры (ИКМ)~\cite{1sok}. Широко известные ИКМ семейств
QUICC (QUad Integrated Communications
Controller) и PowerQUICC впервые были разработаны компанией ``Motorola'', а в настоящее время
выпускаются ее преемницей на рынке микроэлектроники компанией
``Freescale Semiconductor''~[2--4]. Основная причина успеха ИКМ заключается в высокой
степени интеграции как аппаратуры, так и функциональных возможностей на одном кристалле.
Широкий набор этих возможностей уже обеспечил большое разнообразие областей применения
ИКМ. Но в ряде приложений, особенно специального назначения, применение ИКМ может
оказаться нежелательным по целому ряду причин, в том числе совершенно не технического
характера. В этих ситуациях их приходится заменять менее интегрированными компонентами,
т.\,е.\ как бы дезинтегрировать ИКМ, проигрывая при этом в объеме аппаратуры, а значит,
надежности, потребляемой мощности и цене изделия. Следствием такой дезинтеграции может
стать и снижение производительности разрабатываемого устройства.

    При замене ИКМ менее интегрированными компонентами важно не просто уменьшить
потери от дезинтеграции, но и добиться при этом каких-то ощутимых выигрышей, например,
расширением функциональных возможностей устройства или повышением его пропускной
спо\-соб\-ности. Один из возможных путей такой <<ком\-пен\-си\-ру\-ющей>> дезинтеграции был
предложен в~\cite{5sok}. Суть его заключается в том, чтобы расширить <<узкие места>>
архитектуры ИКМ, в частности распараллелить функции RISC-про\-цес\-со\-ра коммуникационного
модуля, распределив их между множеством простых мик\-ро\-кон\-т\-рол\-ле\-ров, и разгрузить
основную системную \mbox{шину,} %\linebreak 
разде\-лив потоки инструкций программируемого ядра и
коммутируемых данных.

    Предлагаемая далее дезинтегрированная архитектура маршрутизирующего пакетного
коммутатора практически реализует намеченные в~\cite{5sok} пути дезинтеграции ИКМ.


\section{Основные компоненты дезинтегрированной архитектуры}

  Основные компоненты предлагаемой дезинтегрированной архитектуры универсального
пакетного коммутатора и их взаимосвязь проиллюстрированы на рис.~\ref{f1sok}, где в качестве
типичных примеров внешних портов коммутатора представлены следующие:
    \begin{itemize}
\item низкоскоростной с интерфейсом E1;
\item среднескоростной с интерфейсом Fast Ethernet (FE);
\item высокоскоростной с интерфейсом Gigabit Ethernet (GE) .
\end{itemize}

    В общем для предлагаемой архитектуры случае каждый внешний порт включает
микропроцессор порта и два адаптера, которые могут быть реализованы на одной, как показано
на рисунке, или разных программируемых логических интегральных\linebreak\vspace*{-12pt}
\pagebreak
\end{multicols}

\begin{figure} %fig1
\vspace*{1pt}
\begin{center}
\mbox{%
\epsfxsize=157.246mm
\epsfbox{sok-1.eps}
}
\end{center}
\vspace*{-9pt}
\Caption{Графическая иллюстрация дезинтегрированной архитектуры
\label{f1sok}}
\end{figure}

\begin{multicols}{2}


\noindent
схемах (ПЛИС). В последнем случае адаптер шин может
быть унифицирован, и
его варианты для различных портов и микропроцессоров будут отличаться лишь разрядностью
внутренней шины данных.

Коммуникационные интерфейсы внешних портов коммутатора обслуживаются
соответствующими адаптерами интерфейсов, которые обеспечивают необходимую аппаратную
поддержку уровня MAC (Media Access Control) подключенного интерфейса. В част\-ности, тракт передачи адаптера
интерфейса должен оформлять и передавать, а тракт приема принимать, разграничивать и
квалифицировать входящие блоки данных\footnote{В дальнейшем тексте для общности используется
термин \textit{блок данных}, но все сказанное в отношении них применимо к кадрам, ячейкам и т. п. конкретных
MAC-интерфейсов.}. 

На рис.~\ref{f1sok} не показаны компоненты физического уровня внешних
интерфейсов, трудно реализуемые или совсем не реализуемые на типовых ПЛИС:
приемопередатчики, импульсные трансформаторы, аналоговые усилители, пиковые детекторы,
эквалайзеры и~пр. Эти компоненты, обычно объединяемые в некие интегрированные
трансиверы конкретных интерфейсов, являются внешними по отношению к предлагаемой
архитектуре и несущественными для нее.

    Каждый адаптер интерфейса обслуживается своим микропроцессором, производительность
и реактивность которого должны быть достаточными, чтобы в совокупности с адаптером
интерфейса как минимум отработать протокол MAC-уровня подключенного интерфейса при
принятых на том скоростях передачи данных. 

Кроме того, существенной особенностью работы
микропроцессора в предлагаемой дезинтегрированной архитектуре является необходимость
отделе\-ния заголовков от тел входящих пакетов и %\linebreak
 до\-бав\-ле\-ние заголовков к телам исходящих
пакетов. Отделяться и присоединяться могут заголовки или даже целые стеки заголовков любой
длины, начиная от заголовков MAC-уров\-ня и кончая целыми цепочками (стеками) заголовков,
включающими заголовки сетевого, транспортного и даже сеансового уровней. Содержание
отделяемых и присоединяемых заголовков и стеков заголовков не имеет значения для
микропроцессора порта до тех пор, пока речь идет не об обработке, а о формальном их
отделении и присоединении. Но в каких-то конкретных случаях микропроцессор порта может,
не противореча описываемой архитектурной концепции, выполнять некую содержательную
работу над отделяемыми заголовками, например фильтровать входящие блоки данных по 
MAC-ад\-ре\-сам или меткам частных локальных сетей. Соответственно, требования к
производительности микропроцессора порта оказываются весьма разными. Чтобы обеспечить
требуемую производительность в %\linebreak
 зави\-си\-мости от скорости передачи данных на порте и
совокупной сложности выполняемых функций, микропроцессор порта может быть выбран 8-,
16- или 32-раз\-ряд\-ным, с различными рабочими час\-то\-та\-ми, объемами памяти и другими
особен\-но\-стями. 
{\looseness=1

}

Существенно, что предлагаемая дезинтегрированная архитектура сама по себе
толерантна к типам и характеристикам микропроцессоров портов.

    Если адаптеры интерфейсов и микропроцессоры портов могут варьироваться в самых
широких пределах, то следующий компонент внешнего порта~--- адаптер шин~--- более
универсален. Он обеспечивает сопряжение внешнего порта с двумя %\linebreak
архитектурными шинами:
<<узкой>> H-ши\-ной заголовков (headers) и <<широкой>> B-ши\-ной тел (bodies) блоков данных.

Принцип работы обеих шин одинаков, различаются они только шириной тракта данных и,
соответственно, пропускной способностью. Поскольку по H-шине передаются относительно
короткие заголовки, ее пропускная спо\-собность может быть меньше при небольшой ширине, в
типичном случае~--- один байт. Ширина B-шины непосредственно определяет пропускную
способность коммутатора, поэтому чем более высокая требуется пропускная способность, тем
шире должна быть B-шина, в типичных случаях~--- от 32 до 128~разрядов. 

Если у коммутатора
имеется один или несколько сверхскоростных портов, например порт GE, то для адаптеров шин
таких портов может потребоваться дополнительное локальное расширение B-шины (на
рис.~\ref{f1sok} показано пунктиром как B$^+$-шина). Архитектурные шины могут
функционировать и на традиционном принципе временного слотового разделения (time-division
mode)~\cite{6sok}, и на принципе блочных передач, как, например, это было предложено
в~\cite{7sok}. 

В более сложных случаях для обеспечения пропускной способности,
пре\-вы\-ша\-ющей возможности даже блочной шины, может быть использована та или иная
коммутационная структура (switch fabric), что в целом не противоречит концепции
предлагаемой архитектуры.

    Входящие блоки данных разделяются аппаратурой тракта приема адаптера интерфейса и
программным обеспечением микропроцессора порта на две части: H-блок, включающий
заголовок или стек заголовков, и B-блок, содержащий остальную часть (тело) принятого блока
данных. H-блок формируется немедленно после окончания приема отделяемого заголовка или
стека заголовков и тут же отправляется по H-шине в память заголовков~--- H-память. B-блок
формируется в процессе приема тела блока данных и по ходу приема отсылается по B-шине в
память тел~--- B-память. В начало H- и B-блоков перед собственно данными вставляется
начальный адрес буфера соответствующей памяти, в который должен быть помещен данный
блок\footnote{Структура H- и B-блоков напоминает, например, структуру записываемого в память блока данных
на шине PCI, где каждому такому блоку предшествует адрес этой памяти.}. Возможно также явное указание
действительной длины содержимого блока. Ссылка на буфер для тела блока данных в B-памяти
также дублируется в H-блоке.

    Очевидно, что B-память должна иметь объем и быстродействие, достаточные для
складирования тел всех блоков данных, входящих со всех внешних портов, и их хранения до
дальнейшей отправки. Например, если коммутатор имеет 20~портов и для каждого порта
обеспечивается хранение до 100~пакетов длиной по 1,5~кбайт (типичная длина IP-па\-ке\-та), то
требуемый объем B-памяти составит 3~Мбайт. При ширине B-шины в 64~разряда и рабочей
частоте 66~МГц, что типично для шин типа PCI (Peripheral Component Interconnect)
  и заведомо не превышает возможностей
синхронной динамической памяти SDRAM (Synchronous Dynamic Random Access Memory), 
общая пропускная способность B-па\-мя\-ти, она же
предельная пропускная способность всего коммутатора, равна приблизительно 2~Гбит/с. Если в
коммутаторе требуются более высокие пропускные способности, например для обслуживания
нескольких портов GE, то для такого случая предусматривается возможность локального
расширения B$^+$-ши\-ной. Пропускная способность архитектуры может быть повышена в
несколько раз использованием B-па\-мя\-ти типа DDR (Double Data Rate)
SDRAM и, сверх того, удвоена переходом к
памяти QDR SRAM (Quad Data Rate Static Random
Access Memory)
с физическим разделением B-шины на два сепаратных тракта: тракта
записи в память и тракта чтения из памяти. Наконец, при относительно небольшом числе
высокоскоростных портов B-шина может быть заменена звездообразными дуплексными
трактами, соединяющими контроллер B-памяти со всеми адаптерами шин.

    Распределение ресурса B-шины между ее абонентами, т.\,е.\ портовыми адаптерами
шины, выполняет контроллер B-шины, который может быть реализован отдельно или на одной
ПЛИС совместно с контролером B-памяти (ПЛИС <<B>> в примере на рис.~\ref{f1sok}).

    Отделяемые заголовки входящих пакетов должны содержать информацию, достаточную
для принятия решения по коммутации со\-от\-вет\-ст\-ву\-юще\-го блока данных. В классической
трактовке 7-уров\-не\-вой модели ISO/OSI (International Standards Organization Open Systems
Interconnection)
для реализации простого коммутатора L2 достаточно
отделять заголовки MAC-уров\-ня. Включение в отделяемую часть заголовков сетевого и
транспортного уровней предо\-став\-ля\-ет возможности более <<интеллектуальной>> коммутации
L3 и~L4.

    Сразу после получения из тракта приема адаптера интерфейса заранее оговоренной
необходимой для коммутации порции информации микропроцессор порта формирует
H$^{\mathrm{I}}$-блок, куда, помимо принятых данных, включает также адрес буфера в H-памяти для
хранения этого H$^{\mathrm{I}}$-блока и описатель буфера B-памяти, где будет храниться тело
принимаемого <<обезглавленного>> блока данных. По мере формирования H$^{\mathrm{I}}$-блоки
отправляются по H-шине в двупортовую H-память, где они становятся доступными через ее
второй порт коммутирующему процессору. При максимальном размере H$^{\mathrm{I}}$-блока порядка
100~байт и принятых выше условиях по количеству внешних портов и числу складируемых
блоков данных на порт минимальный требуемый объем H-памяти для хранения H$^{\mathrm{I}}$-блоков
равен приблизительно 1~Мбайт.

    Аналогично B-шине, распределение ресурса H-ши\-ны между абонентами выполняет
контроллер H-ши\-ны, который может быть реализован на одной ПЛИС (ПЛИС <<H>> на
рис.~\ref{f1sok}) вместе с контролером H-памяти или совмещен с контроллером B-шины.

    Коммутирующий процессор принимает решения по коммутации блоков данных на
основании информации, получаемой в H$^{\mathrm{I}}$-блоках, и формирует H$^{\mathrm{O}}$-блоки, содержащие
заголовки (стеки заголовков) для исходящих блоков данных и дескрипторы буферов с их
телами в B-памяти. Эти H$^{\mathrm{O}}$-блоки он возвращает обратно в H-память, откуда они по H-шине
доставляются в микропроцессор порта назначения. Если полагать, что число исходящих из
коммутатора пакетов примерно равно числу входящих, то и число проходящих через H-память
H$^{\mathrm{O}}$-блоков должно быть примерно равно числу хранящихся там же H$^{\mathrm{I}}$-блоков.
Соответственно, общий объем H-памяти при оговоренных выше условиях должен быть равен
приблизительно 2~Мбайт.

    Заметим, что поскольку тела складируемых блоков данных не попадают в оперативную
память коммутирующего процессора, ее объем может быть относительно небольшим и целиком
определяться нуждами программного обеспечения. Этот объем может оказаться совсем
скромным, если коммутирующий процессор имеет гарвардскую архитектуру с отдельной
памятью программ (на рис.~\ref{f1sok} показана пунктиром).

    Микропроцессор порта получает в H$^{\mathrm{O}}$-блоке комплект заголовков исходящего блока
данных и дескриптор буфера в B-памяти, где хранится тело этого блока. Это тело
микропроцессор извлекает по B-шине, приклеивает к нему новые заголовки непосредственно из
H$^{\mathrm{O}}$-блока и передает вновь сформированный блок данных адаптеру интерфейса, тракт
передачи которого оформляет блок надлежащим образом, сопровождая его преамбулой,
флагами или символами SYNC, а также контрольной суммой и~т.\,п.

    Таким образом, предлагаемая дезинтегрированная архитектура расширяет сразу два <<узких
места>> архитектуры ИКМ. Во-первых, вместо одного процессора коммуникационного модуля,
обслу\-жи\-ва\-юще\-го все внешние порты ИКМ, она предполагает множество отдельных
микропроцессоров, в результате чего на каждом порте может быть получена любая требуемая
производительность по отработке канальных протоколов и, если требуется, протоколов более
высокого уровня. При этом, в отличие от ИКМ, пропускная способность отдельного порта не
зависит от загрузки других портов коммутатора. Во-вторых, в предлагаемой архитектуре
предусмот\-ре\-на отдельная память для хранения тел блоков данных и отдельная шина для их
пересылки в эту память, вследствие чего два наиболее интенсивных потока информации в
коммутаторе~--- поток инструкций коммутирующего процессора и поток тел коммутируемых
пакетов~--- никак не конфликтуют между собой.

    Однако изоляция тел блоков данных в отдельном хранилище создает определенную
проблему с административными блоками данных и блоками данных, инкапсулирующими пакеты
коммутационных протоколов, содержимое которых должно быть доступно центральному ядру
коммутатора (C-яд\-ру на рис.~\ref{f1sok}). Это ядро в общем случае может включать, помимо
коммутирующего процессора, системный диспетчер общего управления коммутатором и, в
случае маршрутизирующего коммутатора, отдельный маршрутизирующий процессор. 
В~частном случае функции ка\-кой-ли\-бо пары или всех трех перечисленных компонентов может
выполнять один единственный универсальный процессор с достаточно высокой
производительностью. Для решения проблемы доступа к B-па\-мя\-ти всех про\-цессо\-ров
центрального ядра в предлагаемую ар\-хи\-тектуру приходится вводить дополнительный адаптер
шин, обеспечивающий сопряжение C- и B-ши\-ны. Если этот дополнительный адаптер\linebreak
реализован анало\-гич\-но адаптерам шин внешних портов, то процессоры центрального ядра,
помимо доступа к %\linebreak
\mbox{B-шине,} <<автоматически>> получают доступ и к H-ши\-не. Эта
дополнительная возможность, как будет видно далее, оказывается полезной при инициализации
коммутатора системным диспетчером, а также для тестовых и кон\-т\-роль\-но-диа\-гно\-сти\-че\-ских
\mbox{целей.}
{ %\looseness=1

}


\section{Организация H- и B-шины}

  Особенность H- и B-шин состоит в том, что они должны обеспечить гарантированную
полосу пропускания каждому порту коммутатора. При этом в зависимости от особенностей
внешних интерфейсов и частных требований к коммутатору речь может идти как о среднем
значении за какой-то период времени, так и о гарантии пропускания шиной очередного блока
данных в очень жесткие интервалы времени, определяемые физической скоростью передачи
данных на том или ином внешнем интерфейсе. Жесткость требований может быть в любой
степени смягчена буферами FIFO (First In, First Out)
соответствующего объема между адаптерами шин и
микропроцессором порта, показанными на рис.~\ref{f1sok}, но принципиально это картины не
меняет. С учетом сказанного для H- и B-шины можно предложить два решения:
    \begin{enumerate}[(1)]
\item слотовая шина с жестким разделением временных слотов между портами коммутатора,
как, например, это описано в~\cite{6sok};
\item блочная шина, работающая по принципу сверх\-ло\-каль\-ной сети, в частности
эффективное решение, предложенное в~\cite{7sok}.
\end{enumerate}

    Слотовая шина, во-первых, способна обеспечить абонентам регулярный гарантированный
доступ к памяти, а во-вторых, относительно проста и удобна в реализации. Однако, поскольку
слотовая шина фактически реализует принцип коммутации каналов, ей свойственно присущее
самому принципу недоиспользование потенциальной полосы пропускания. С этой точки зрения
предпочтительнее выглядит блочная шина, реализующая, по существу, принцип пакетной
коммутации; к тому же она лучше сопрягается с синхронной динамической памятью любого
типа от простой SDRAM до QDR SRAM. Однако шина с блочной организацией заметно
сложнее слотовой в реализации. Поэтому не стоит пренебрегать возможностью улучшить
использование ресурса слотовой шины, варьируя в ней доли отдельных внешних портов в
соответствии с максимальными скоростями передачи данных на них. Ниже кратко рассмотрен
простой путь улучшения использования полосы пропускания простейшей слотовой шины.

    Пусть слотовая B-шина, обслуживающая тех же трех абонентов, что показаны на
рис.~\ref{f1sok}, имеет ширину 64~разряда\footnote{В предположении, что все внешние порты, как и в
нашем примере, дуплексные, с точки зрения дальнейших оценок не принципиально, рассматривается ли единая
    64-разрядная шина или две сепаратные 32-разрядные шины, отдельно для чтения из B-памяти и записи в нее.},
работает на частоте 50~МГц и разделена на 16~временных слотов\footnote{Длительность
временного слота шины желательно иметь кратной длине блока (burst length) SDRAM~--- в типичном минимальном
варианте четырем тактам шины.}. При этих условиях пропускная способность одного временного
слота составит 200~Мбит/с (25~Мбайт/с), а общая пропускная способность всей шины будет
равна 3,2~Гбит/с (400~Мбайт/с). Три показанных на рис.~\ref{f1sok} внешних порта
поддерживают следующие максимальные (пиковые) дуплексные скорости передачи данных:
%    \begin{itemize}
\begin{center}
\begin{tabular}{lp{0.5mm}l}
%\item[\ ]  
порт E1 &&$2\times 2048$~кбит/с (512~кбайт/с);\\
%\item[\ ] 
порт FE && $2 \times 100$~Мбит/с (25~Мбайт/с);\\
%\item[\ ] 
порт GE && $2 \times 1$~Гбит/с   (250~Мбайт/с).
%\end{itemize}
\end{tabular}
\end{center}

    Тогда ресурс нашей гипотетической слотовой шины 3,2~ Гбит/с можно поделить между
внешними портами, например, следующим образом (см.\ рис.~\ref{f2sok}):
%    \begin{itemize}
\begin{center}
\begin{tabular}{lp{0.5mm}l}
порт GE && 12~слотов (300~кбайт/с);\\
порт FE &&   2~слота (50~Мбайт/с);\\
порт E1 &&  1~слот   (25~Мбайт/с).
\end{tabular}
\end{center}


\begin{figure*} %fig2
\vspace*{1pt}
\begin{center}
\mbox{%
\epsfxsize=159.819mm
\epsfbox{sok-2.eps}
}
\end{center}
\vspace*{-9pt}
\Caption{Пример распределения ресурсов слотовой шины
  \label{f2sok}}
    \end{figure*}
    \begin{figure*} %fig3
\vspace*{1pt}
\begin{center}
\mbox{%
\epsfxsize=164.843mm
\epsfbox{sok-3.eps}
}
\end{center}
\vspace*{-9pt}
\Caption{Пример структуры слотов H- и B-шины
  \label{f3sok}}
    \end{figure*}

    Пример демонстрирует трудности с выделением временных слотов и относительную
неэффективность слотовой шины. Для гарантии доступа и максимальной равномерности
предоставления шины 12~слотов вместо минимально достаточных 10 выделено порту GE. По
тем же причинам двойная полоса~--- два слота вместо минимально достаточного одного~---
выделена порту FE. Огромна и неизбежна при принятых условиях избыточность полосы для
порта~E1. В итоге на шине остался лишь один слот (слот~13), который может быть выделен для
адаптера шины центрального ядра.

    При небольшом числе внешних портов и ограниченных скоростях передачи на них
отмеченная избыточность никому не мешает. Однако при большой потенциальной загрузке
шины возникнет необходимость раздавать ее ресурсы более экономно. Уменьшать
избыточность выделяемых полос можно простым уменьшением шага квантования, т.\,е.\
увеличением числа слотов в цикле шины. 

Например, если бы цикл нашей шины состоял не из
16, а из 128~слотов, то минимальная выделяемая на порт полоса пропускания~--- квант полосы
шины~--- уменьшилась бы с~25 до~3~Мбайт/с, что было бы достаточно для порта~E1.
Если в коммутаторе соседствуют низкоскоростные и высокоскоростные порты, то нужного
эффекта можно достичь применением иерархических цикловых структур вроде суперциклов.
Для нашего примера суперцикл из 32~циклов при 32~слотах в цикле уменьшил бы квант
полосы шины до 40~кбайт/с.

    Предлагаемая дезинтегрированная архитектура не накладывает жестких ограничений на
структуру слотов H- и B-шины. Один из возможных примеров слотовой структуры со слотами
размером в 8~тактов шины показан рис.~\ref{f3sok}.

    В приведенном примере слот делится на два полуслота по 4~такта каждый. Первый
полуслот (такты R$_1$--R$_4$) отводится на чтение данных адап\-те\-ром шины соответствующего
порта из H- или B-памяти, а второй полуслот (такты W$_1$--W$_4$)~--- на запись в H- или
B-память. Таким образом, содержание передаваемой информации и направление ее передачи
зависят от позиции полуслота. Кроме позиции определим еще два типа полуслотов:
    адресный~--- $a$-тип и информационный~--- $d$-тип с соответствующими индексами. Тип
полуслота задается текущим мастером шины, в качестве которого выступает адаптер шины,
контролирующий шину в данном слоте. В полуслоте чтения $a$-типа  мастер шины передает в
контроллер памяти адрес буфера, из которого следует извлекать следующий H$^{\mathrm{O}}$- или B-блок
(слот~$i$ на рис.~\ref{f3sok}), а в полуслоте  чтения $d$-типа  контроллер памяти выдает
мастеру шины очередную порцию H$^{\mathrm{O}}$- или B-блока (слот $i+1$ на рис.~\ref{f3sok}). В
полуслотах записи направление передачи всегда от мастера шины к контроллеру памяти: в
полуслоте $a$-типа  передается адрес буфера, в котором следует сохранять следующий H$^{\mathrm{I}}$-
или B-блок, а в полуслоте $d$-типа~--- очередная порция H$^{\mathrm{I}}$- или B-блока.

    На любой слотовой шине не каждый выделенный порту временной слот будет
использоваться для передачи полезной информации. Пустые полуслоты неизбежны как из-за
избыточности выделенных портам полос пропускания вследствие их квантования, так и просто
из-за перерывов трафика на конкретном порте. Поэтому слотовая шина должна иметь
отдельный маркер занятости полуслотов (на рис.~\ref{f3sok} полуслот занят~--- <<$\bullet$>>,
пуст~--- <<$\circ$>>). Маркер занятости устанавливается, раздельно для H- и B-шины, мастером
шины, который тем самым получает возможность регулировать темп передачи данных
независимо по обеим шинам, выдерживая при необходимости паузы как между H- и B-блоком,
так и внутри них.

\section{Организация буферов и~адресация H- и~B-памяти}

  Использовать абсолютные адреса на H- и B-шине неразумно. Дело не только в их
разрядности. В целях взаимной защиты ресурсов, выделяемых для буферирования различным
портам, желательно, чтобы каждый микропроцессор порта пользовался лишь относительными
адресами внутри выделенных ему адресных пространств (пулов) H- и B-памяти. Выделение
этих пространств должно быть прерогативой системного диспетчера C-ядра. Кроме того,
начальные адреса буферов с разумными относительными потерями памяти могут задаваться на
границах блоков размером 2$^M$~байт, вследствие чего микропроцессор порта адресует
буферы в категориях этих блоков, укорачивая тем самым на $M$ разрядов адреса, передаваемые
по H- и B-шине. Разумные значения $M$ лежат в диапазоне 2--4 для H-памяти и, при типичной
длине пакетов 1,5~кбайт, 4--8 для B-памяти.


    В целом организация буферов в H- и B-памяти и их адресация со стороны микропроцессора могли бы
быть следующими.

    Системный диспетчер (или универсальный процессор) C-ядра в процессе инициализации
коммутатора выделяет каждому внешнему порту в памяти некие пулы: по одному пулу в
B-памяти~--- для B-блоков; и по два пула в H-памяти~--- для H$^{\mathrm{I}}$- и H$^{\mathrm{O}}$-блоков. Размер
пула в B-памяти, вы\-де\-ля\-емо\-го внешнему порту, должен быть, как правило, пропорционален
общему объему проходящих через этот порт данных, т.\,е., в конечном счете, скорости
передачи данных на порте. Размер пулов в H-памяти определяется выбранным размером пула в
B-памяти с учетом типичного соотношения размеров H- и B-блоков для коммуникационных
протоколов, принятых на данном порте.

    Установочные параметры пулов системный диспетчер раздает микропроцессорам внешних
портов через B-шину в форме неких S-блоков, по одному блоку на каждый внешний порт.
S-блок включает базовые адреса и размеры всех выделенных для порта пулов, а также
назначаемые порту временные слоты на H- и B-шине. Для передачи установочных параметров
пулов и назначения слотов шины могут быть введены дополнительные типы (полу-) слотов,
соответственно пуловые и слотовые. Как альтернатива для передачи установочных параметров
могут использоваться адресные и информационные полуслоты, по умолчанию трактуемые
иначе во время инициализации системы.

    Мастером шины (шин) на все время инициализации во всех слотах является адаптер шин
C-ядра. Поскольку передача S-блоков осуществляется по шине (шинам), слоты которой
(которых) на данный момент еще не получили своего назначения, для инициализации системы
должно использоваться некое исходное фиксированное (по умолчанию) соответствие слотов
внешним портам. Простая и широко практикуемая основа установления такого фиксированного
соответствия~--- <<географическая>> или любая другая физическая нумерация портов. В
дальнейшем для определенности будет использоваться номер внешнего порта.

    Пусть три внешних порта, показанных на рис.~\ref{f1sok}, имеют номера: порт~E1~--- 1,
порт~FE~--- 2, а порт~GE~--- 3. На рис.~\ref{f4sok} показан пример возможного использования
H-шины во время инициализации коммутатора для назначения внешним портам слотов шины в
соответствии с их распределением, показанным ранее на рис.~\ref{f2sok}.

    Выдача системным диспетчером в слоте нуля означает, что этот слот либо не используется,
либо резервируется для себя самим системным диспетчером. Ненулевой код в некотором слоте
указывает номер внешнего порта, которому выделяется данный слот. Пример передачи портам
установочных параметров пулов по B-шине приведен на рис.~\ref{f5sok}.

\begin{figure*} %fig4
\vspace*{1pt}
\begin{center}
\mbox{%
\epsfxsize=159.819mm
\epsfbox{sok-4.eps}
}
\end{center}
\vspace*{-9pt}
\Caption{Пример использования H-шины для назначения слотов внешним портам
\label{f4sok}}
\vspace*{6pt}
\end{figure*}
\begin{figure*} %fig5
\vspace*{1pt}
\begin{center}
\mbox{%
\epsfxsize=166.218mm
\epsfbox{sok-5.eps}
}
\end{center}
\vspace*{-9pt}
\Caption{Пример использования B-шины для передачи установочных параметров
  \label{f5sok}}
    \end{figure*}

    Первый такт каждого слота B-шины, такт P, несет номер внешнего порта, которому
назначается данный слот, т.\,е.\ выполняет ту же роль, что и H-шина на рис.~\ref{f4sok} (если, в
частности, H-шина не задействована в процессе инициализации коммутатора). Информация в
остальных тактах слота относится к внешнему порту, номер которого равен номеру данного
слота, и включает базовые адреса и длины пулов, выделенные этому порту.

    После рассылки установочных параметров коммутатор переходит в рабочий режим, при
котором используется только относительная адресация пулов, организованных как
циркулярные буферы. Для организации одного циркулярного буфера на одного абонента шины
(т.\,е.\ на внешний порт или на сис\-тем\-но\-го диспетчера) контроллер со\-от\-вет\-ст\-ву\-ющей памяти
должен иметь следующий комплект оборудования: регистр адресной базы, \mbox{регистр} размера
пула, счетчик рабочих (относительных) адресов до\-сту\-па, сумматор выходов счетчика с
адресной базой пула и компаратор этих выходов с размером пула. Для обслуживания $N$
абонентов шин ($N-1$ внешних портов плюс сис\-тем\-ный диспетчер) контроллер B-памяти
должен иметь $N$ таких комплектов оборудования, а контроллер H-па\-мя\-ти~--- $2N$
комплектов, хотя и несколько меньшей раз\-ряд\-ности. Разумно однотипные элементы этих
комплектов объединить внутри контроллера в блоки памяти с произвольным доступом объемом~$N$~слов 
каж\-дый, унифицированно адресуемые номером текущего абонента. После
объединения регистр адресной базы превращается в память адресных баз, регистр размера
пула~--- в память размеров пулов, а счетчик относительных адресов доступа~--- в память
относительных адресов. Общими остаются сумматор выходов счетчика с адресной базой пула и
компаратор этих выходов с размером пула. Пример структуры контроллера B-па\-мя\-ти показан
на рис.~\ref{f6sok}.
{\looseness=1

}

\begin{figure*} %fig6
\vspace*{1pt}
\begin{center}
\mbox{%
\epsfxsize=158.157mm
\epsfbox{sok-6.eps}
}
\end{center}
\vspace*{-9pt}
\Caption{Пример структуры контроллера B-памяти
  \label{f6sok}}
    \end{figure*}

    Память рабочих адресов загружается из полуслотов $a$-типа некими начальными адресами
или просто нулями, которые в дальнейшем инкрементируются после каждого обращения к
памяти, т.\,е.\ после каждого непустого полуслота $d$-типа данного абонента.
Исполнительный адрес обращения к H- или B-памяти получается суммированием содержимого
адресного счетчика, извлекаемого из памяти рабочих адресов, с базой пула, получаемой из
памяти адресных баз. Для обеспечения циркулярности пулов контроллер каждой памяти
должен в процессе инкрементации адресов сравнивать рабочий адрес с размером выделенного
абоненту пула. Как только адрес после инкремента адресного счетчика переходит верхнюю
границу пула, счетчик принудительно обнуляется, возвращая тем самым относительный адрес к
началу пула.

    В остальные памяти во время инициализации записываются соответственно номера
внешних портов, адресные базы и длины пулов. Запись производится соответственно в тактах
P, B$_{\mathrm{B}}$ и B$_{\mathrm{S}}$ B-шины (см.\ рис.~\ref{f5sok}). Адресом записи служит текущий номер слота 
B-ши\-ны. В рабочем режиме записанные в память назначения слотов номера внешних портов
служат адресами для чтения адресных баз и размеров пулов, а также чтения и записи рабочих
адресов. Поэтому в каждом слоте $d$-типа из всех памятей считывается информация,
относящаяся к порту, которому был назначен данный слот. То же самое справедливо для
стартовых адресов, заносимых в память рабочих адресов в полуслотах $a$-типа.
{\looseness=1

}

    Практически удобно организовывать циркулярные пулы размером равным степеням
двойки. В~этом случае контроллеры шин могут обойтись без сумматоров для вычисления
очередного адреса обращения и компараторов для контроля границ пулов. Адресная база пула
значительно укорачивается и превращается в простой префикс для рабочего адреса, причем
разрядность последнего уста\-нав\-ли\-ва\-ет\-ся численно равной той самой назначенной для данного
пула степени двойки. Префиксация заменяет суммирование адреса и базы, а переполнение
счетчика автоматически возвращает текущий адрес на начало пула.
\section{Заключение}

  Описанная дезинтегрированная архитектура пакетной коммутации реализует общую
концепцию, предложенную в~\cite{5sok}. Она позволяет не только создавать пакетные
коммутаторы без использования ИКМ, но и получать при этом ряд существенных преимуществ.
В обсуждаемой дезинтегрированной архитектуре:
    \begin{itemize}
\item снимаются присущие ИКМ серьезные и принципиальные ограничения на число
внешних портов;
\item устраняется зависимость пропускной способности отдельного внешнего порта от
интенсивности трафиков на других портах;
\item исключаются конфликты потоков инструкций и данных, присущие единой системной
шине ИКМ;
\item создаются предпосылки радикального повышения общей пропускной способности
коммутатора и снижения времени коммутации пакетов.
\end{itemize}

  Ограничение числа внешних портов в ИКМ является следствием нескольких причин. 
  
  Во-пер\-вых, всегда имеет место физическое ограничение числа выводов корпуса ИКМ.
Дезинтеграция автоматически снимает эту проблему. 

Во-вторых, в ИКМ все внешние порты
обслуживаются одним коммуникационным процессором, реактивности которого недостаточно
для обслуживания большого числа внешних портов~\cite{7sok}. По этой же причине в ИКМ
неизбежна зависимость пропуск\-ной способности отдельного внешнего порта от ин\-тен\-сив\-ности
потоков ком\-му\-ти\-ру\-емых данных на других портах. 
Несколько улучшает %\linebreak
 ситу\-а\-цию применение
для процессоров ИКМ специальных архитектур высокой реактивности~\cite{8sok, 9sok}, но
лишь предлагаемая дезинтегрированная архитектура комплексно и радикально устраняет все
отмеченные ограничения. 

В-третьих, косвенно число обслуживаемых внешних портов зависит
от общей пропускной способности коммутатора, а та, в свою очередь, от пропускной
способности внутренних путей данных. 
Пространственно разделяя потоки инструкций
ком\-му\-ти\-ру\-юще\-го процессора и коммутиру\-емых им блоков данных, дезинтегрированная
архитектура снимает и эту зависимость, открывая возможность найти более рациональное
решение <<по месту>> для каждой специализированной шины или коммутационной структуры.
В результате устраняются излишние задержки блоков данных внутри коммутатора и
сокращается время коммутации.

    Перечисленные преимущества обеспечивают дезинтегрированной архитектуре качественно
новый уровень общей пропускной способности по сравнению с ИКМ. При этом уникальная
пропускная способность с одинаковым успехом может быть реализована как в коммутаторах с
небольшим чис\-лом высокоскоростных внешних портов, так и с большим чис\-лом
низкоскоростных.

    Предложенные в статье решения, безусловно, нельзя рассматривать как спецификацию
новой архитектуры. В частности, не затронуты многие важные вопросы управления
коммутатором в целом, а также ряд частных проблем коммутации и адап\-та\-ции, таких как
сегментация и сборка/разборка блоков данных, а также инкапсуляция пакетов, которые еще
требуют своего решения. Но для поиска таких решений в рамках предлагаемой
дезинтегрированной архитектуры на данный момент не видно никаких принципиальных
препятствий.


{\small\frenchspacing
{%\baselineskip=10.8pt
\addcontentsline{toc}{section}{Литература}
\begin{thebibliography}{9}


\bibitem{1sok}
\Au{Шагурин И., Белецкий~В.}
Микроконтроллеры, интегрированные процессоры и гибридные DSP-процессоры компании
FreeScale Semiconductors (SPS~--- Motorola)~// Электронные компоненты, 2004. №\,7.

\bibitem{4sok} %2
\Au{Егоров В.\,Б.}
Принципы создания коммутационной аппаратуры на основе специализированных
микроконтроллеров~// Системы и средства информатики.~--- М.: Наука, 1999. Вып.~9.
С.~44--55.

\bibitem{2sok} %3
\Au{Егоров В.\,Б.}
Новое поколение коммуникационных мик\-ро\-контроллеров компании ``Freescale
Semi\-con\-ductor''~// Chip News, 2007. No.\,3. P.~14--18.

\bibitem{3sok} %4
\Au{Егоров В.\,Б.}
Интегрированные коммуникационные процессоры компании ``Freescale Semiconductor''~//
Электронные компоненты, 2007. №\,8. С.~85--89.

\bibitem{5sok}
\Au{Соколов И.\,А., Егоров В.\,Б.}
Дезинтеграционный подход к архитектуре универсального процессора коммутации пакетов~//
Информационные технологии и вычислительные системы, 2005. №\,2. С.~76--85.

\bibitem{6sok}
\Au{Егоров В.\,Б.}
 Способ увеличения количества портов пакетного коммутатора с помощью слотовой шины~//
Информационные технологии и вычислительные сис\-те\-мы, 2006. №\,2. С.~16--21.

\bibitem{7sok}
\Au{Егоров В.\,Б., Полухин~А.\,Н.}
Принципы создания сис\-тем\-ной шины многопортовых пакетных коммутаторов~// Сис\-те\-мы и
средства информатики.~--- М.: Наука, 2000. Вып.~10. С.~80--90.

\bibitem{8sok}
\Au{Егоров В.\,Б.}
Способ повышения реактивности процессора~// Информационные технологии и
вычислительные системы, 2006. №\,4. С.~3--15.

\label{end\stat}

\bibitem{9sok}
\Au{Егоров В.\,Б.}
<<Многоэтажная>> архитектура процессора~ // Информационные технологии и
вычислительные системы, 2007. №\,3. С.~79--87.
\end{thebibliography}
}
}
\end{multicols}