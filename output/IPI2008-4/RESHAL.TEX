\def\stat{rez}
{%\hrule\par
%\vskip 7pt % 7pt
\raggedleft\Large \bf%\baselineskip=3.2ex
Р\,Е\,Ц\,Е\,Н\,З\,И\,И \vskip 17pt
    \hrule
    \par
\vskip 24pt plus 6pt minus 3pt }


\def\tit{МОНОГРАФИЯ А.\,С.~ШАЛАМОВА <<ИНТЕГРИРОВАННАЯ ЛОГИСТИЧЕСКАЯ ПОДДЕРЖКА НАУКОЕМКОЙ ПРОДУКЦИИ>>
(М.: Университетская книга, 2008. 464~с.)}

%1
\def\aut{Заслуженный деятель науки РФ, д.т.н., профессор И.\,Н.~Синицын}

\def\auf{\ }

\def\leftkol{\ } % ENGLISH ABSTRACTS}

\def\rightkol{\ } %ENGLISH ABSTRACTS}

\titele{\tit}{\aut}{\auf}{\leftkol}{\rightkol}



%     РЕЦЕНЗИЯ на монографию А.\,С.~Шаламова      <<Интегрированная логистическая поддержка наукоемкой продукции>>

     \label{st\stat}

     \begin{multicols}{2}
     {\small

     В июне текущего 2008~г.\ в издательстве <<Университетская книга>> вышла монография
<<Интегрированная логистическая поддержка наукоемкой продукции>>.

     Ее автор Шаламов~А.\,С., профессор, доктор технических наук, заместитель
руководителя Департамента послепродажного обслуживания авиационной техники ОАО
<<Российская самолетостроительная корпорация <<МиГ>>, возглавляет направление работ по
интегрированной логистической поддержке (ИЛП).

Интегрированная логистическая поддержка~---
ор\-га\-ни\-за\-ци\-он\-но-тех\-но\-ло\-ги\-че\-ский и
про\-грам\-мно-тех\-ни\-ческий комплекс \textit{управления послепродажным
обслуживанием} наукоемкой продукции (НП), направленный на %\linebreak
со\-зда\-ние и
обеспечение функционирования интегрированной информационной среды, объединяющей
всех участников жизненного цикла (ЖЦ) НП в единое виртуальное предприятие, с
целью \textit{минимизации финансовых затрат при заданном уровне технической
(эксплуатационной) готовности} парка изделий, что и является главным принципом
ИЛП.
 {\looseness=-1

 }

     Помимо информационного обеспечения весьма важными являются задачи
\textit{интеллектуальной поддержки} управления послепродажным обслуживанием НП, на
которую и возлагается реализация указанного принципа. Недостаточный уровень существующих
методов математического моделирования не позволяет пока осуществить это в полной мере.
Основной целью рецензируемой монографии как раз и является разработка теоретических и
практических подходов к решению подобных задач.

     В первых двух частях книги после краткого описания сис\-те\-мы
     послепродажного обслуживания (СППО) как объекта моделирования
и управ\-ле\-ния рассматриваются различные вопросы~--- от процессов управления
конфигурацией изделий, СППО на  этапах жизненного цик\-ла и моделирования
стоимости жизненного цик\-ла НП, до информационного моделирования основных
логистических процессов с использованием современных международных стандартов, включая
демонстрацию программных решений при создании интегрированной логистической базы
данных.

     В третьей, основной части работы, излагаются современные инновационные технологии в
области математического моделирования и оптимизации ор\-га\-ни\-за\-ци\-он\-но-тех\-ни\-че\-ских и
ор\-га\-ни\-за\-ци\-он\-но-эко\-но\-ми\-че\-ских систем (ОТС-ОЭС), имеющие российский приоритет.
Приведенные здесь научные методы разработаны впервые. С~опорой на известные результаты
теории не\-пре\-рыв\-но-дис\-крет\-ных марковских процессов получено
     ин\-тег\-ро-диф\-фе\-рен\-ци\-аль\-но-раз\-но\-ст\-ное уравнение типа Кол\-мо\-го\-ро\-ва--Фел\-ле\-ра для
плотности ве\-ро\-ят\-ности фа\-зо\-во\-го\linebreak вектора применительно к СППО с конечным множеством
дискретных состояний объектов об\-слу\-живания, изменя\-ющих\-ся в течение
     не\-пре\-рыв\-но-дис\-крет\-но\-го време\-ни.  Получено также соответствующее уравне\-ние
Пуга\-чё\-ва для характеристической функции фа\-зово\-го векто\-ра \mbox{СППО}, используемое для
вывода диф\-фе\-рен\-ци\-аль\-ных уравнений, определяющих математическое ожида\-ние и
ковариационную матрицу вектора фазовых координат, а также вероятность эффективной
работы сис\-те\-мы. В~основе уравнений лежит предложенный автором способ формализации
ор\-га\-ни\-за\-ци\-он\-но-тех\-но\-ло\-ги\-че\-ской структуры СППО как сети систем
 массового обслуживания с
нелинейными и нестационарными функциями производительности (пропускной способности).
Достаточно большое место отводится вопросам адаптации существующих методов
оптимизации динамических систем применительно к нелинейной \mbox{СППО}   в условиях
нестационарного стохастического характера ее непрерывно-дискретных процессов и влияния
внешней среды.
{\looseness=-1

}

     Для большей конкретизации можно указать на теоретические результаты в области
моделирования вероятностной динамики СППО с различными стратегиями расходования,
восстановления и пополнения ресурсов в комплексе с другими процессами виртуального
предприятия, а также методы  оптимизации как параметров СППО на этапе проектирования
НП, так и процессов управления при ее эксплуатации. Интерес представляют также
результаты в области систем с ограниченной об\-ластью функционирования (изменения
фазового вектора). Выход одной или нескольких фазовых координат на границы допустимой
области здесь может означать прекращение функционирования системы. В~работе предложена
модель такой системы, в основе которой (модели) лежит элемент поглощения реализаций
фазового вектора на границе области.

     В практическом плане на основе разработанных тео\-ре\-ти\-че\-ских методов решен ряд задач
интеллектуальной поддержки проектирования НП по критерию кон\-курентоспособности
(рыночного потенциала), опти\-мального планирования потребностей заказчика в ресурсах,
необходимых для эксплуатации НП на предстоящем временн$\acute{\mbox{о}}$м периоде, а также по
определению %\linebreak
оптимальных параметров политики поставок для пополнения складов как
потребителя, так и поставщика, что нашло отражение в двух последних частях моно\-графии.
{\looseness=1

}

     Предложенные в работе теоретические методы и проведенные прикладные исследования
являются основой при создании современных программных комплексов для
автоматизированных систем управления широким классом ОТС-ОЭС, включающих в себя
практически весь набор подсистем, предназначенных для обслуживания и сопровождения как
изделий наукоемкой продукции военного и гражданского назначения, так и объектов
     материальной инфраструктуры, таких как, например, технические комплексы и объекты
машиностроения, энергетики и~др.

Кроме того, полученные результаты применимы для
создания эффективных сис\-тем моделирования процессов, прогнозирования результатов
деятельности и оптимального управления в банковской, страховой, лизинговой и других
сферах экономики.

     Также одной из важнейших проблем по реализации современных методов управления
СППО является необходимость  организации системы  фундаментальной подготовки кадров в
этой весьма специфической области знаний, находящейся на стыке многих наук, таких, по
крайней мере, как техническая эксплуатация НП,  прикладная математика и информатика.
Монография Шаламова~А.\,С., являясь своего рода учебным пособием, содержит основы для
решения и этой задачи, включая подготовку аспирантов и докторантов с целью обеспечения
притока вузовских кадров (педагогов и ученых).
%\vspace*{12pt}
}

\end{multicols}



\vspace*{9pt}

\hrule

\vspace*{3pt}

\hrule


%\hfill Заслуженный деятель науки РФ

%\hfill Д.т.н., профессор

%\hfill  И.\,Н.~Синицын


%\end{multicols}
\vskip 18pt plus 6pt minus 3pt

%2

\def\tit{МОНОГРАФИЯ И.\,Н.~СИНИЦЫНА <<ФИЛЬТРЫ КАЛМАНА И ПУГАЧЕВА>>
  (Изд. 2-е перераб. и доп., М.: Университетская книга, ЛОГОС, 2007.
776~с.)}

%1
\def\aut{Член-корреспондент РАН А.\,П.~Реутов}

\def\auf{\ }

%\def\leftkol{\ } % ENGLISH ABSTRACTS}

%\def\rightkol{\ } %ENGLISH ABSTRACTS}

\def\leftkol{РЕЦЕНЗИИ}

\def\rightkol{РЕЦЕНЗИИ}

\titele{\tit}{\aut}{\auf}{\leftkol}{\rightkol}

\begin{multicols}{2}

 {\small

     В книге дано систематическое изложение теории фильтров Калмана и
Пугачёва для обработки информации в сложных стохастических системах, а также
приведены новые результаты фундаментальных работ, выполненных в Институте
проблем информатики Российской академии наук в рамках научного направления
<<Стохастические системы и стохастические информационные технологии>>.

     Во втором издании книга подверглась существенной переработке с целью
ориентации на читателя, знакомого только с элементарной теорией вероятностей и
математической статистики. В отдельные главы выделены собственно теория
фильтров Калмана и Пугачёва, а также некоторые прикладные задачи оценивания,
распознавания и идентификации сигналов и параметров на основе фильтров Калмана
и Пугачёва.

     Глава~1 содержит необходимые сведения по тео\-рии случайных величин и
функций. Изложены основы стохастического анализа.

     В гл.~2 приведены необходимые сведения по моделям непрерывных и
дискретных стохастических систем (СтС). Рассмотрена теория одно- и многомерных
распределений процессов в СтС. Описаны элементы теории оценивания в
непрерывных и дискретных СтС.

     В гл.~3 рассмотрен фильтр Калмана для непрерывных и дискретных
линейных СтС. Изложены элементы линейного стохастического анализа
непрерывных СтС. Выведены основные уравнения оптимальной фильтрации в
гауссовских непрерывных СтС. Особое внимание уделено уравнениям, линейным
относительно вектора состояния. Рассмотрена теория непрерывных фильтров и
экстраполяторов. Даны обобщения калмановской теории фильтрации на случай
автокоррелированной помехи в наблюдениях. Специальный раздел посвящен
вопросам устойчивости фильтра Калмана--Бьюси. Изложена теория дискретного
фильтра Калмана.

     Глава~4 посвящена теории приближенных (субоптимальных) методов
оценивания состояния и параметров в нелинейных СтС, основанная на теории
нелинейного оценивания. Приведены элементы нелинейного стохастического
анализа непрерывных СтС, основанные на методах нормальной аппроксимации
(МНА), эквивалентной линеаризации, а также методах параметризации
распределений. Дана краткая характеристика субоптимальных методов оценивания
для диф\-фе\-рен\-циаль\-ных СтС. Подробно рассмотрены МНА апос\-те\-ри\-ор\-но\-го
распределения и метод статистической ли\-не\-а\-ри\-за\-ции (МСЛ). Описан
модифицированный МНА, %\linebreak
основанный на использовании ненормированных
распределений. Особое внимание уделено квазилинейным субоптимальным
фильтрам, основанным на МСЛ. Приведены методы моментов, семиинвариантов,
ортогональных разложений и квазимоментов для приближенного решения
фильтрационных уравнений, а также модифицированные версии методов,
основанные на использовании ненормированных распределений. Подробно
рассмотрены квазилинейные субоптимальные методы оценивания, основанные на
методах параметризации распределений. Специальный раздел отведен %\linebreak
 методам
субоптимального оценивания, основанным на упрощении уравнений оптимальной
фильтрации.  Большое внимание уделено непрерывному обобщенному фильтру
Калмана (ОФК), а также дискретному ОФК. Рассмотрены дискретные
субоптимальные фильтры, основанные как на приближенном решении
фильтрационных уравнений, так и на их упрощениях.
{\looseness=-1

}

     Глава~5 содержит систематическое изложение теории фильтра В.\,С.~Пугачёва.
Изложен принцип условно оптимальной фильтрации и постановки основных задач.
Дано решение задач условно оптимальной фильтрации, экстраполяции и
интерполяции. Рассмотрены фильтрация при автокоррелированной помехе в
наблюдениях, линейная фильтрация Пугачёва.

     В гл.~6 рассмотрены некоторые прикладные задачи оценивания,
распознавания и идентификации на основе фильтров Калмана и Пугачёва, фильтры
Пугачёва для линейных СтС с параметрическими шумами, фильтры Калмана и
Пугачёва по бейесовым и сложно статистическим критериям. Излагаются элементы
эллипсоидального анализа распределений в СтС, а также теория субоптимальных и
условно оптимальных фильтров для задач фильтрации, распознавания и
идентификации сигналов и параметров в нелинейных СтС. Рас\-смот\-ре\-но применение
фильтров Калмана и Пугачёва в задачах совместной фильтрации, распознавания и
идентификации.

     В приложениях 1--5 содержатся сведения о полиномах Эрмита,
     $\chi^2$-распределении и полиномах, ортогональных к
     $\chi^2$-распределению, функции Лап\-ла\-са и ее производных, а также формулы
для статистической и эллипсоидальной линеаризации. В~приложении~6 приведены
сведения по известному программному обеспечению фильтров Калмана и Пугачёва, а
также примеры его использования.

Биографические замечания и список
литературных источников даны в конце книги. Автор в конце биографических
замечаний счел необходимым привести портреты и краткие биографические сведения
о Р.\,Э.~Калмане (р.~1930) и В.\,С.~Пугачёве (1911--1998).

     Книга предназначена для научных работников и инженеров в области
прикладной математики и информатики, системного анализа, теории управ\-ле\-ния, а
также в  других областях науки и техники, связанных с обработкой информации в
системах, поведение которых описывается стохастическими дифференциальными,
интегральными, интегродифференциальными, разностными и другими уравнениями
(стохастические системы). Книга может представлять интерес для математиков,
спе\-циа\-ли\-зи\-ру\-ющих\-ся в области стохастических уравнений и их приложений. Она
может быть полезна студентам высших учебных заведений, обуча\-ющих\-ся по
специальности <<Прикладная математика и информатика>>.
Единая методика,
тщательный подбор примеров и задач (их свыше~500) позволяют использовать книгу
широкому кругу студентов, аспирантов и преподавателей.
   %  \vspace*{12pt}

    % \hfill Член-корреспондент РАН А.\,П.~Реутов

}

\end{multicols}

\vspace*{9pt}

\hrule

\vspace*{3pt}

\hrule


\vskip 18pt plus 6pt minus 3pt

%3

\def\tit{МОНОГРАФИЯ В.\,Е.~БЕНИНГА, В.\,Ю.~КОРОЛЕВА, И.\,А.~СОКОЛОВА, С.\,Я.~ШОРГИНА
<<РАНДОМИЗИРОВАННЫЕ МОДЕЛИ И МЕТОДЫ ТЕОРИИ НАДЕЖНОСТИ ИНФОРМАЦИОННЫХ И
ТЕХНИЧЕСКИХ СИСТЕМ>> (М.: ТОРУС ПРЕСС, 2007. 256~с.)}

%1
\def\aut{Д.ф.-м.н., профессор  А.\,В.~Печинкин}

\def\auf{\ }

%\def\leftkol{\ } % ENGLISH ABSTRACTS}

%\def\rightkol{\ } %ENGLISH ABSTRACTS}

\def\leftkol{РЕЦЕНЗИИ}

\def\rightkol{РЕЦЕНЗИИ}

\titele{\tit}{\aut}{\auf}{\leftkol}{\rightkol}

\begin{multicols}{2}

 {\small

     В 2007~г.\ в издательстве <<ТОРУС ПРЕСС>> вышла монография <<Рандомизированные
модели и методы тео\-рии надежности информационных и технических сис\-тем>>. Ее авторы~---
В.\,Е.~Бенинг, профессор, доктор физико-математических наук, профессор факультета
вычислительной математики и кибернетики МГУ им. М.\,В.~Ломоносова, старший научный
сотрудник Института проблем информатики РАН; В.\,Ю.~Королёв, профес-\linebreak сор, доктор
     физико-математических наук, профессор факультета вычислительной математики и
кибернетики МГУ им.\ М.\,В.~Ломоносова, ведущий научный сотрудник Института проблем
информатики РАН; И.\,А.~Со\-колов, академик, директор Института проблем ин\-форматики
РАН; С.\,Я.~Шоргин, профессор, доктор\linebreak физико-математических наук, заместитель директора
Института проблем информатики РАН.

     Для исследования надежности информационных и технических систем (ИиТС),
подверженных влиянию случайных и нестационарно изменяющихся факторов, в книге
предложены альтернативные классическим рандомизированные математические модели и
методы. Большой интерес представляет непараметрический подход к оцениванию
коэффициента готовности, для получения интервальных оценок используются новейшие
оценки скорости сходимости в центральной предельной теореме теории вероятностей. В
частности, в книге рассматривается ситуация, в которой учитываются возможные изменения
надежности восстанавливаемой ИиТС вследствие ее модификаций или ремонтов. Отдельная
глава посвящена прямому применению байесовской идеологии при анализе надежности и
эффективности ИиТС.

     Большая часть книги посвящена учету влияния не\-однородности интенсивности потока
информативных событий, в результате которых накапливается статистическая информация, на
итоговые статистические выводы о параметрах ИиТС. Существенно развита  общая теория
статистического вывода на основе выборок случайного объема. В книге показано, что при
замене объема выборки случайной величиной заметно увеличиваются вероятности
критических значений тех или иных статистических критериев или уменьшаются
доверительные\linebreak вероятности по сравнению с классической ситуацией. Детально рассмотрены
методы анализа на\-деж\-ности, основанные на предельных теоремах для порядковых статистик в
выборках случайного объема. Исследована возможность использования распределения
Стьюдента в качестве альтернативы нормальному закону в статистических методах анализа
надежности ИиТС.

Книга может быть полезна для специалистов в об\-ласти применения методов теории
вероятностей и математической статистики к анализу надежности ИиТС, а также для
аспирантов и студентов старших курсов, обучающихся по специальностям <<информатика>> и
<<прикладная математика>>.

}
\end{multicols}

\vspace*{6pt}

\hrule

\vspace*{3pt}

\hrule


\vskip 16pt plus 6pt minus 3pt

%3

\def\tit{МОНОГРАФИЯ В.\,Ю.~КОРОЛЕВА И И.\,А.~СОКОЛОВА
     <<МАТЕМАТИЧЕСКИЕ МОДЕЛИ НЕОДНОРОДНЫХ ПОТОКОВ ЭКСТРЕМАЛЬНЫХ СОБЫТИЙ>>
(М.: ТОРУС ПРЕСС, 2008. 192~с.)}

%1
\def\aut{Д.ф.-м.н., профессор  А.\,В.~Печинкин}

\def\auf{\ }

%\def\leftkol{\ } % ENGLISH ABSTRACTS}

%\def\rightkol{\ } %ENGLISH ABSTRACTS}

\def\leftkol{РЕЦЕНЗИИ}

\def\rightkol{РЕЦЕНЗИИ}

\titele{\tit}{\aut}{\auf}{\leftkol}{\rightkol}

\vspace*{-24pt}

\begin{multicols}{2}
 {\small
     В 2008~г.\ в издательстве <<ТОРУС ПРЕСС>> вышла монография <<Математические
модели неоднородных потоков экстремальных событий>>. Ее авторы~--- В.\,Ю.~Королёв,
профессор, доктор физико-математических\linebreak наук, профессор факультета вычислительной
математики и кибернетики МГУ им.\ М.\,В.~Ломоносова, ведущий научный сотрудник
Института проблем информатики РАН и И.\,А.~Соколов, академик, директор Института
проблем информатики РАН.

     В книге рассмотрены математические модели вероятностно-статистических
характеристик катастроф в неоднородных потоках экстремальных событий. Сформулированы
задачи моделирования катастрофических событий, связанных как с критическими
превышениями уровня процессом, описывающим накопленные эффекты неблагоприятных
факторов, так и с однократными шоковыми воздействиями. В качестве основных
математических моделей при решении указанных задач рассматриваются экстремумы
обобщенных дважды стохастических пуассоновских процессов и макс-обобщенные дважды
стохастические пуассоновские процессы. Для таких процессов доказан ряд предельных теорем.
Возникающие в этих теоремах предельные распределения вероятностей предлагаются в
качестве аппроксимаций для вероятностно-статистических закономерностей, присущих
потокам экстремальных (катастрофических) событий. Большой интерес представляет анализ
временных характеристик глобальных катастроф, вызванных столкновением Земли с
потенциально опасными небесными телами (астероидами, кометами). На примере этого
анализа описаны конкретные процедуры для вычисления вероятностных характеристик
катастроф, в частности <<ожидаемого времени>> катастрофы и продолжительности периода, в
течение которого вероятность катастроф пренебрежимо мала.

     Книга представляет значительный интерес для специалистов в области применения
методов теории вероятностей и математической статистики к анализу рисков, связанных с
чрезвычайными ситуациями и катастрофами, и надежности информационных и технических
сис\-тем. Она также будет полезна аспирантам и студентам старших курсов, обучающимся по
специальностям <<информатика>> и <<прикладная математика>>.
}

\end{multicols}


\vspace*{9pt}

\hrule

\vspace*{3pt}

\hrule


\vskip 12pt plus 6pt minus 3pt

%3

\def\tit{МОНОГРАФИЯ А.\,И.~ЗЕЙФМАНА, В.\,Е.~БЕНИНГА, И.\,А.~СОКОЛОВА
<<МАРКОВСКИЕ ЦЕПИ И~МОДЕЛИ С НЕПРЕРЫВНЫМ ВРЕМЕНЕМ>> (М.: ЭЛЕКС-КМ, 2008. 
168~с.)}

%1
\def\aut{Д.ф.-м.н., профессор  С.\,Я.~Шоргин}


\def\auf{\ }



%\def\leftkol{\ } % ENGLISH ABSTRACTS}

%\def\rightkol{\ } %ENGLISH ABSTRACTS}

\def\leftkol{РЕЦЕНЗИИ}

\def\rightkol{РЕЦЕНЗИИ}

\titele{\tit}{\aut}{\auf}{\leftkol}{\rightkol}


\vspace*{-24pt}

\begin{multicols}{2}
 {\small
В 2008~г.\ в издательстве <<ЭЛЕКС-КМ>> вышла монография <<Марковские цепи 
и модели с непрерывным временем>>. Ее авторы~--- А.\,И.~Зейфман, профессор, доктор 
физико-математических наук, декан факультета прикладной математики и компьютерных 
технологий Вологодского государственного педагогического университета, старший 
научный сотрудник директора Института проблем информатики РАН; В.\,Е.~Бенинг, 
профессор, доктор физико-математических наук, профессор факультета вычислительной 
математики и кибернетики МГУ им.\ М.\,В.~Ломоносова, старший научный сотрудник 
Института проблем информатики РАН; И.\,А.~Соколов, академик, директор Института 
проблем информатики РАН. 
     
     Как известно, получение явных выражений для вероятностей состояний 
стохастических моделей возможно лишь в исключительных случаях. В связи с этим одной 
из важнейших  задач при исследовании таких моделей давно является исследование 
поведения модели при стремлении времени к бесконечности и в частности, скорости 
сходимости к предельному режиму и связанных с этим функционалов. В рецензируемой 
книге работе изучаются вопросы, связанные с получением точных оценок скорости к 
предельному режиму и устойчивости для марковских цепей с непрерывным временем 
(стационарных и нестационарных), а также приложение методов и результатов к 
изучению некоторых  конкретных моделей, описываемых такими цепями, и в первую 
очередь, для  нестационарных марковских моделей систем массового обслуживания.
     
     Книга, несомненно, вызовет интерес у научных работников, инженеров, аспирантов, 
студентов и преподавателей вузов, интересующихся современным состоянием 
исследований в области теории вероятностей и ее приложений.

}
\end{multicols}