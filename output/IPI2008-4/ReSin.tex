    Р Е Ц Е Н З И Я
     на книгу И.\,Н.~Синицына <<Фильтры Калмана и Пугачева>>: 
      (Монография. Изд. 2-е перераб. и доп. М.: Университетская книга, Логос, 2007. 
776~с.)
       
       
     В книге дается систематическое изложение теории фильтров Калмана и 
Пугачева для обработки информации в сложных стохастических системах, а также 
приводятся новые результаты фундаментальных работ, выполненных в Институте 
проблем информатики Российской академии наук в рамках научного направления 
<<Стохастические системы и стохастические информационные технологии>>. 
     
     Во втором издании книга подверглась существенной переработке с целью 
ориентации на читателя, знакомого только с элементарной теорией вероятностей и 
математической статистики. В отдельные главы выделены собственно теория 
фильтров Калмана и Пугачева, а также некоторые прикладные задачи оценивания, 
распознавания и идентификации сигналов и параметров на основе фильтров Калмана 
и Пугачева. 
     
     Глава~1 содержит необходимые сведения по теории случайных величин и 
функций. Изложены основы стохастического анализа.
     
     В главе~2 приведены необходимые сведения по моделям непрерывных и 
дискретных стохастических систем (СтС). Рассмотрена теория одно- и многомерных 
распределений процессов в СтС. Описаны элементы теории оценивания в 
непрерывных и дискретных СтС.
     
     В главе 3 рассматривается фильтр Калмана для непрерывных и дискретных 
линейных СтС. Изложены элементы линейного стохастического анализа 
непрерывных СтС. Выведены основные уравнения оптимальной фильтрации в 
гауссовских непрерывных СтС. Особое внимание уделено уравнениям, линейным 
относительно вектора состояния. Рассмотрена теория непрерывных фильтров и 
экстраполяторов. Даны обобщения калмановской теории фильтрации на случай 
автокоррелированной помехи в наблюдениях. Специальный раздел посвящен 
вопросам устойчивости фильтра Калмана--Бьюси. Изложена теория дискретного 
фильтра Калмана.
     
     Глава 4 посвящена теории приближенных (субоптимальных) методов 
оценивания состояния и параметров в нелинейных СтС, основанная на теории 
нелинейного оценивания. Приведены элементы нелинейного стохастического 
анализа непрерывных СтС, основанные на методах нормальной аппроксимации 
(МНА), эквивалентной линеаризации, а также методах параметризации 
распределений. Дана краткая характеристика субоптимальных методов оценивания 
для дифференциальных СтС. Подробно рассмотрены МНА апостериорного 
распределения и метод статистической линеаризации (МСЛ). Описан 
модифицированный МНА, основанный на использовании ненормированных 
распределений. Особое внимание уделено квазилинейным субоптимальным 
фильтрам, основанным на МСЛ. Приведены методы моментов, семиинвариантов, 
ортогональных разложений и квазимоментов для приближенного решения 
фильтрационных уравнений, а также модифицированные версии методов, 
основанные на использовании ненормированных распределений. Подробно 
рассмотрены квазилинейные субоптимальные методы оценивания, основанные на 
методах параметризации распределений. Специальный раздел отведен методам 
субоптимального оценивания, основанным на упрощении уравнений оптимальной 
фильтрации.  Большое внимание уделено непрерывному обобщенному фильтру 
Калмана (ОФК), а также дискретному ОФК. Рассматрены дискретные 
субоптимальные фильтры, основанные как на приближенном решении 
фильтрационных уравнений, так и на их упрощениях.
     
     Глава 5 содержит систематическое изложение теории фильтра В.\,С.~Пугачева. 
Изложен принцип условно оптимальной фильтрации и постановки основных задач. 
Дано решение задач условно оптимальной фильтрации, экстраполяции и 
интерполяции. Рассмотрены фильтрация при автокоррелированной помехе в 
наблюдениях, линейная фильтрация Пугачева. 
     
     В главе 6 рассмотрены некоторые прикладные задачи оценивания, 
распознавания и идентификации на основе фильтров Калмана и Пугачева, фильтры 
Пугачева для линейных СтС с параметрическими шумами, фильтры Калмана и 
Пугачева по бейесовым и сложно статистическим критериям. Излагаются элементы 
эллипсоидального анализа распределений в СтС, а также теория субоптимальных и 
условно оптимальных фильтров для задач фильтрации, распознавания и 
идентификации сигналов и параметров в нелинейных СтС. Рассмотрено применение 
фильтров Калмана и Пугачева в задачах совместной фильтрации, распознавания и 
идентификации.
     
     В приложениях 1--5 содержатся сведения о полиномах Эрмита, 
     $\chi^2$-распределении и полиномах, ортогональных к 
     $\chi^2$-распределению, функции Лапласа и ее производных, а также формулы 
для статистической и эллипсоидальной линеаризации. В приложении~6 приведены 
сведения по известному программному обеспечению фильтров Калмана и Пугачева, а 
также примеры его использования. Биографические замечания и список 
литературных источников даны в конце книги. Автор в конце биографических 
замечаний счел необходимым привести портреты и краткие биографические сведения 
о Р.\,Э.~Калмане (р.~1930) и В.\,С.~Пугачеве (1911--1998).
     
     Книга предназначена для научных работников и инженеров в области 
прикладной математики и информатики, системного анализа, теории управления, а 
также в  других областях науки и техники, связанных с обработкой информации в 
системах, поведение которых описывается стохастическими дифференциальными, 
интегральными, интегродифференциальными, разностными и другими уравнениями 
(стохастические системы). Книга может представлять интерес для математиков, 
специализирующихся в области стохастических уравнений и их приложений. Она 
может быть полезна студентам высших учебных заведений, обучающихся по 
специальности <<Прикладная математика и информатика>>. Единая методика, 
тщательный подбор примеров и задач (их свыше~500) позволяют использовать книгу 
широкому кругу студентов, аспирантов и преподавателей.
     \vspace*{12pt}
     
     \hfill Член-корреспондент РАН А.\,П.~Реутов