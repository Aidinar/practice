
\def\stat{gorsh}

\def\tit{МЕДИАННЫЕ МОДИФИКАЦИИ EM- И SEM-АЛГОРИТМОВ ДЛЯ~РАЗДЕЛЕНИЯ СМЕСЕЙ ВЕРОЯТНОСТНЫХ РАСПРЕДЕЛЕНИЙ И~ИХ~ПРИМЕНЕНИЕ
К~ДЕКОМПОЗИЦИИ ВОЛАТИЛЬНОСТИ ФИНАНСОВЫХ ВРЕМЕННЫХ РЯДОВ$^*$}
\def\titkol{Медианные модификации EM- и SEM-алгоритмов для
разделения смесей вероятностных распределений} %и их применение к декомпозиции волатильности финансовых временных рядов}

\def\autkol{А.\,К.~Горшенин,  В.\,Ю.~Королёв, А.\,М.~Турсунбаев}
\def\aut{А.\,К.~Горшенин$^1$,  В.\,Ю.~Королёв$^2$, А.\,М.~Турсунбаев$^3$}

\titel{\tit}{\aut}{\autkol}{\titkol}

{\renewcommand{\thefootnote}{\fnsymbol{footnote}}\footnotetext[1]
{Работа выполнена при поддержке РФФИ, гранты
08-01-00345, 08-01-00363, 08-07-00152.}}

\renewcommand{\thefootnote}{\arabic{footnote}}
\footnotetext[1]{Московский государственный
университет, факультет ВМиК, andygorshenin@gmail.com}
\footnotetext[2]{Московский государственный университет, факультет
ВМиК, Институт проблем информатики РАН, vkorolev@comtv.ru}
\footnotetext[3]{Московский государственный университет, факультет ВМиК}

\Abst{Предложены медианные модификации ЕМ- и
SEM-алгоритмов и на примере численного решения задачи декомпозиции
волатильности финансовых индексов демонстрируются их преимущества
по сравнению с классическими методами. Приведены примеры
декомпозиции волатильности различных финансовых временных рядов.}


\KW{разделение смесей вероятностных распределений; робастность; эффективность; ЕМ-алгоритм; SЕМ-алгоритм; волатильность}

      \vskip 24pt plus 9pt minus 6pt

      \thispagestyle{headings}

      \begin{multicols}{2}

      \label{st\stat}


\section{Введение}

Для численного решения задачи разделения конечных смесей
вероятностных распределений (т.\,е.\ задачи отыскания
статистических оценок весов компонент смеси и параметров компонент
смеси) при относительно большом числе компонент традиционно
применяется ЕМ-алгоритм. Если функция правдоподобия регулярна, то
этот метод, как правило, находит наиболее правдоподобные оценки
параметров смеси. Однако если функция правдоподобия нерегулярна,
имеет много локальных максимумов (возможно, к тому же
бесконечных), то ЕМ-алгоритм становится крайне неустойчивым. К
сожалению, последнее обстоятельство является серьезным
препятствием при интерпретации результатов применения ЕМ-алгоритма
к разделению конечных смесей нормальных законов. Именно такие
смеси повсеместно применяются при математическом моделировании
многих явлений в самых разных областях~--- от биологии до экономики
и от физики до финансового анализа.

В частности, как было экспериментально установлено, ЕМ-алгоритм
обладает сильной неустойчивостью по начальным данным. Например, в
случае четырехкомпонентной смеси нормальных\linebreak законов при объеме
выборки 200--300 наблюдений замена лишь одного наблюдения другим
может кардинально изменить итоговые оценки, полученные с помощью
ЕМ-алгоритма~\cite{Korolev2007b}.

По-видимому, эта неустойчивость обусловлена тем, что стандартные
(наиболее правдоподобные для случая нормального распределения)
оценки математического ожидания и дисперсии (среднее
арифметическое и выборочная дисперсия) при <<засорении>> ({\it
контаминации}) выборки <<посторонними>> или <<паразитными>>
наблюдениями становятся заметно менее эффективными по сравнению
со, скажем, выборочной медианой. Этот эффект обнаружен
Дж.~Тьюки~\cite{Tukey1960} и описан, например, в~\cite{Ayvazyan1983, Korolev2006}.
Формально модель контаминации Тьюки сводится
к тому, что вместо <<чистого>> модельного распределения,
интерпретируемого как {\it однородная} модель, в качестве
модельного распределения рассматривается неоднородная модель,
имеющая вид смеси исходного <<чистого>> распределения и некоторого
другого закона, описывающего <<за\-соря\-ющие>> наблюдения. В задаче
разделения смесей по сам\'{о}й сути модели, когда оцениваются
параметры одной компоненты смеси, наблюдения с распределениями,
соответствующими другим компонентам, являются <<загрязняющими>>.
Это обстоятельство\linebreak может сыграть особенно важную роль при
реали\-зации SEM-алгоритма, описываемого ниже. В~данной статье в
развитие методов, описанных в работе~\cite{GKT2008}, предлагаются
медианные модификации\linebreak ЕМ- и SEM-алгоритмов и на примере численного
решения задачи декомпозиции волатильности финансовых индексов
демонстрируются их преимущества по сравнению с классическими
методами.

\section{EM-алгоритм для разделения смесей вероятностных распределений}

Пусть $\emph{\textbf{x}}=(x_1,\ldots,x_n)$~--- наблюдаемое значение
случайной выборки $\emph{\textbf{X}}=(X_1,\ldots,X_n)$, в которой
$X_1,\ldots,X_n$~--- независимые случайные величины с одинаковой
функцией распределения
\begin{equation}
F(x)=\sum\limits_{i=1}^kp_i\Phi\left(\fr{x-a_i}{\sigma_i}\right)\,,\quad
x\in\mathbb{R}\,,
\end{equation}
где $a_i\in\mathbb{R}$,
$\sigma_i>0$, $p_i\ge0$, $i=1,\ldots,k$, $p_1+\ldots+p_k=$\linebreak $=\;1$,
$\Phi(x)$~--- стандартная нормальная функция рас\-пре\-де\-ле\-ния.


ЕМ-алгоритмом принято называть итерационную процедуру поиска
оценок максимального правдоподобия вектора $\theta$ параметров
$$
\theta=(p_1,\ldots,p_k,a_1,\ldots,a_k,\sigma_1,\ldots,\sigma_k)\,.
$$
Применительно к смесям нормальных законов вида~(1) ЕМ-алгоритм
определяется следующим образом (см., например,~\cite{Korolev2007b}). Пусть значение
$$
\theta^{(m)}\!=\!\left ( p_1^{(m)},\ldots,p_k^{(m)},a_1^{(m)},\ldots,a_k^{(m)},\sigma_1^{(m)},\ldots,\sigma_k^{(m)}\!\right )
$$
параметра $\theta$ на $m$-й итерации ЕМ-алгоритма известно
($m\ge0$). Обозначим
\begin{multline*}
g_{ij}^{(m)}=\fr{\fr{p_i^{(m)}}{\sigma_i^{(m)}}\,\phi\left(\fr{x_j-a_i^{(m)}}{\sigma_i^{(m)}}\right)}
{\sum\limits_{r=1}^k\fr{p_r^{(m)}}{\sigma_r^{(m)}}\,\phi\left(\fr{x_j-a_r^{(m)}}{\sigma_r^{(m)}}\right)}={}\\
{}=
\fr{\fr{p_i^{(m)}}{\sigma_i^{(m)}}\exp\left\{-\fr{1}{2}\left(\fr{x_j-a_i^{(m)}}{\sigma_i^{(m)}}\right)^2\right\}}
{\sum\limits_{r=1}^k{\displaystyle{\fr{p_r^{(m)}}{\sigma_r^{(m)}}\exp\left\{-\fr{1}{2}\left(
\fr{x_j-a_r^{(m)}}{\sigma_r^{(m)}}\right)^2\right\}}}}\,.
\end{multline*}
Величину $g_{ij}^{(m)}$ можно интерпретировать как статистическую
оценку апостериорной вероятности того, что элемент $X_j$ выборки
сгенерирован в соответствии с $i$-й компонентой смеси (1) (т.\,е.\
$g_{ij}^{(m)}$ является <<апостериорной вероятностью>> того, что
распределением случайной величины $X_j$ является
$\Phi\left(\left(x-a_i^{(m)}\right)\bigg /\sigma_i^{(m)}\right)$. Тогда значения
параметров $p_i$, $a_i$ и $\sigma_i$ на $(m+1)$-й итерации
ЕМ-алгоритма соответственно определяются как

\noindent
\begin{align}
p_i^{(m+1)}&=\fr{1}{n}\sum\limits_{j=1}^ng_{ij}^{(m)}\,;\notag \\
a_{i}^{(m+1)}      & = \fr{1}{\sum\limits_{j=1}^{n}g_{ij}^{(m)}}
\sum\limits_{j=1}^{n}g_{ij}^{(m)}x_{j}\,;    \\
\sigma_{i}^{(m+1)}  & =
 \left[ \fr{1}{\sum\limits_{j=1}^{n}g_{ij}^{(m)}}
\sum\limits_{j=1}^{n}g_{ij}^{(m)}\left(x_{j}-a_{i}^{(m+1)}\right)^2\right]^{1/2}\,,\notag\\\
&\ \ \ \ \ \ \ \ \  \ \ \ \ \ \ \ \ \ \ \ \ \ \ \ \ \ \ \ \ \ \ \ \ \ \ \ \  \ \ \ \ \ \ \ \ \ \ \ \ \ \ \ \ \ \ \
i=1,\ldots,k\,.\notag
\end{align}
Обратим внимание на то, что $a_{i}^{(m+1)}$ является <<выборочным
средним>>, построенным по реализации
$\emph{\textbf{x}}=(x_1,\ldots,x_n)$ выборки
$\emph{\textbf{X}}=(X_1,\ldots,X_n)$, как если бы распределение
каждого ее элемента задавалось вероятностями
$
g_{ij}^{(m)}/\sum\limits_{j=1}^{n}g_{ij}^{(m)}$, $i=1,\ldots,k$.
ЕМ-ал\-го\-ритм довольно сильно зависит от начального приближения.
Будучи алгоритмом проксимального типа~\cite{Korolev2007b, Vasilyev2002},
он находит лишь локальный максимум функции
правдоподобия. Для борьбы с этим недостатком предназначена, в
частности, модификация ЕМ-алгоритма, называемая стохастическим
ЕМ-алгоритмом или SEM-алгоритмом. Описание этого алгоритма будет
специально приведено\linebreak ниже.

ЕМ-алгоритм также проявляет сильную неустойчивость по отношению к
начальным данным. Для противодействия этому предназначены {\it
медианные} модификации EM- и SEM-алгоритмов, которым, собственно,
и посвящена данная статья. Строгое описание этих модификаций
необходимо предварить обсуждением целесообразности применения в
разных случаях разных~--- моментных и медианных~--- оценок
параметров положения компонент смесей вида~(1).

\section{Относительная эффективность выборочного среднего и~выборочной
медианы при~оценивании параметров положения компонент конечных смесей нормальных~законов}

\label{efficency}

Предположим, что в выборке $\emph{\textbf{X}}=(X_1,\ldots,X_n)$
все элементы независимы и имеют одну и ту же непрерывную плотность
распределения $f(x)$. Обозначим $m={\mbox{med}}X_1$. Предположим,
что $f(m)>0$. Пусть $\overline m_n$~--- выборочная медиана,
построенная по выборке $X_1,\ldots,X_n$. Еще в 1931~г.\ А.\,Н.~Колмогоров~\cite{Kolmogorov1931}
(см.\ также с.~111--114 в~\cite{Kolmogorov1986}) показал, что при $n\to\infty$
$$
{\sf P}\big(\sqrt{n}(\overline m_n-m)<x\big)\longrightarrow
\Phi\big(2f(m)x\big)\,,
$$
где, как обычно, $\Phi(y)$~--- стандартная нормальная функция
распределения, так что $\Phi\big(2f(m)x\big)$~--- функция
распределения нормально распределенной случайной величины с
дисперсией $\big(2f(m)\big)^{-2}$. Таким образом, для
рассмотренного выше критерия качества выборочной медианы при
больших $n$ имеем
$$
{\sf E}(\overline m_n-m)^2=\fr{1}{n}{\sf
E}\left[\sqrt{n} (\overline
m_n-m)\right]^2\approx\fr{1}{4n\left(f(m)\right)^2}\,.
$$
Если дополнительно обозначить $a={\sf E}X_1$, $\overline X=$\linebreak
$=\;(1/n)\sum\limits_{j=1}^nX_j$, то согласно центральной предельной
теореме
$$
{\sf P}\left(\sqrt{n}\left( \overline X_n-a\right ) <x\right)\longrightarrow
\Phi\left ( \fr{x}{\sqrt{{\sf D}X_1}}\right )\,,
$$
т.\,е.\ при больших $n$
$$
{\sf E}(\overline X_n-a)^2=\fr{1}{n}{\sf
E}\left[\sqrt{n} (\overline X_n-a)\right]^2\approx\fr{{\sf
D}X_1}{n}\,.
$$
Таким образом, ответ на вопрос о том, какая из оценок~---
выборочное среднее или выборочная медиана~--- лучше, можно
получить, скажем, вычислив отношение
$$
\fr{{\sf E}(\overline X_n-a)^2}{{\sf E}(\overline
m_n-m)^2}\approx 4\left(f(m)\right)^2{\sf D}X_1
$$
(относительную эффективность оценок $\overline X_n$ и $\overline
m_n$).

В частности, если
$f(x)$~--- плотность нормального распределения со средним $a$ и дисперсией
${\sf D}X_1$:
$$
f(x)=\fr{1}{\sqrt{2\pi{\sf D}X_1}}\exp\left\{-\fr{(x-a)^2}{2
{\sf D}X_1}\right\}\,,
$$
то, во-первых, $a=m$ и, во-вторых, $f(m)=$\linebreak $=1/
\sqrt{2\pi{\sf D}X_1}$, так что
$$
\fr{{\sf E}(\overline X_n-a)^2}{{\sf E}(\overline
m_n-m)^2}\approx \fr{2}{\pi}\,.
$$
Очевидно, что если в нормальном случае для оценивания параметра
положения использовать выборочную медиану, то для того, чтобы
достичь той же точности, что при использования выборочного
среднего, понадобится в $\pi/2\approx 1.57$ раз больше наблюдений,
т.\,е.\ в таком случае выборочная медиана примерно в полтора раза
менее эффективна, нежели выборочное среднее.


\smallskip
При использовании выборочного среднего и выборочной медианы в
качестве статистических оценок параметра, характеризующего
<<центр>> распределения, следует заметить, что выборочная медиана
обладает большей устойчивостью к присутствию в выборке так
называемых <<загрязняющих>> наблюдений. Действительно, если
выборка $X_1,\ldots,X_n$ в некотором смысле не является
однородной, т.\,е. наряду с наблюдениями, имеющими функцию
распределения $F(x)$, в ней присутствуют наблюдения с какой-то
другой функцией распределения, то в выборочное среднее наряду с
<<правильными>> наблюдениями войдут {\it значения}
<<загрязняющих>> наблюдений. При этом если значения
<<за\-гряз\-ня\-ющих>> наблюдений велики, то их присутствие,
естественно, сильно смажет итоговую картину. В то же время
отклонения выборочной медианы от ее <<правильного>> значения
зависят не столько от значений <<загрязняющих>> наблюдений,
сколько от их числа. Такое свойство выборочной медианы, как
известно, называется робастностью.

\smallskip

Вышеупомянутое свойство робастности выборочной медианы хорошо
иллюстрируется на примере следующей ситуации. Предположим, что в
независимой выборке $X_1,\ldots,X_n$ все элементы имеют одну и ту
же плотность распределения
$$
f(x)=\sum\limits_{i=1}^k\fr{p_i}{\sqrt{2\pi}\cdot\sigma_i}\exp\bigg\{-\fr{(x-a)^2}{2\sigma_i^2}\bigg\}\,,
$$
где $0<p_i<1$, $i=\overline{1,k}$, $p_1 + \ldots + p_k=1$ и
$\sigma_i^2>0$. Эту ситуацию можно интерпретировать как наличие в
выборке примерно $p_i\cdot 100\%$ наблюдений с нормальным
распределением, имеющим параметры $a$ и $\sigma_i^2$,
$i=\overline{1,k}$, т.\,е.\ изучаемая популяция (генеральная
совокупность) является {\it смесью} $k$ популяций, каждая из
которых нормально распределена с параметрами~$a$ и~$\sigma_i^2$,
причем доли этих $k$ субпопуляций ({\it компонент смеси})
составляют соответственно $p_i\cdot 100\%$, $i=\overline{1,k}$.
Если при этом какое-либо из значений $p_i$ близко к единице, то
говорят, что выборка из $i$-й субпопуляции {\it загрязнена}
объектами (наблюдениями) из других субпопуляций. Заметим, что
па\-ра\-мет\-ры <<центра>> у всех компонент смеси одинаковы. Легко
видеть, что $a=m$ и
$$
f(m)=\fr{1}{\sqrt{2\pi}}\sum\limits_{i=1}^k\fr{p_i}{\sigma_i}.
$$
Далее,
\begin{multline*}
{\sf
D}X_1={}\\
{}=\int\limits_{-\infty}^{\infty}(x-a)^2\sum\limits_{i=1}^k\fr{p_i}{\sqrt{2\pi}\cdot\sigma_i}\exp\left\{
-\fr{(x-a)^2}{2\sigma_i^2}\right\}\,dx={}
\\{}
=\sum\limits_{i=1}^k\fr{p_i}{\sqrt{2\pi}\sigma_i}\int\limits_{-\infty}^{\infty}(x-a)^2
\exp\left\{-\fr{(x-a)^2}{2\sigma_i^2}\right\}dx={}\\
{}= \sum\limits_{i=1}^kp_i\sigma_i^2\,.
\end{multline*}
Вычислим асимптотическую относительную эффективность выборочного
среднего и выборочной медианы:
\begin{multline*}
\lim_{n\to\infty}\fr{{\sf E}\left(\overline X_n-a\right)^2}{{\sf
E}\left(\overline m_n-m\right)^2}= 4\left(f(m)\right)^2{\sf D}X_1={}\\
{}=\fr{2}{\pi}\left(\sum\limits_{i=1}^k\fr{p_i}{\sigma_i}\right)^2
\sum\limits_{i=1}^kp_i\sigma_i^2\,.
\end{multline*}
Несложно видеть, что если зафиксировать все\linebreak параметры
$p_1,\ldots,p_k,\sigma_1,\ldots,\sigma_k$, кроме одного,
скажем~$\sigma_{i_0}$, то правая часть последнего соотношения
неограниченно возрастает при неограниченном увеличении
$\sigma_{i_0}$. Действительно,
\begin{multline}
\fr{\pi}{2}\lim_{n\to\infty}\fr{{\sf E}(\overline X_n-a)^2}{{\sf
E}(\overline
m_n-m)^2}=\bigg(\fr{p_{i_0}}{\sigma_{i_0}}+A{\bigg)\!}^2\big(p_{i_0}\sigma_{i_0}^2+B\big)={}
\\{}
=A^2p_{i_0}\sigma_{i_0}^2+2Ap_{i_0}^2\sigma_{i_0}+2AB\fr{p_{i_0}}{\sigma_{i_0}}+
B\fr{p_{i_0}^2}{\sigma_{i_0}^2}+{}\\
{}+ p_{i_0}^3+A^2B\,,
%\label(e3)
\end{multline}
где
$$
A=\sum\limits_{\substack{{1\le i\le k}\\{i\neq i_0}}}\fr{p_i}{\sigma_i}\,;\ \ \ \
B=\sum\limits_{\substack{{1\le i\le k}\\{i\neq i_0}}}p_i\sigma_i^2
$$
и первые два слагаемых в правой части (3) неограниченно возрастают
при $\sigma_{i_0}\to\infty$, в то время как остальные слагаемые
стремятся к $p_{i_0}^3+A^2B$.

К примеру, если $k=2$, $\sigma_1=1$, $p_1=0{,}01$, $p_2=0{,}99$, то
выборочная медиана эффективнее выборочного среднего для
$\sigma_2^2>61$. Если же $p_1=0{,}05$, то выборочная медиана
эффективнее выборочного среднего для $\sigma_2^2>14$. Наконец,
если доля <<загрязняющих>> наблюдений составляет 10\%, то
выборочная медиана эффективнее выборочного среднего уже для
$\sigma_2^2>9{,}1$.

\section{Медианные модификации ЕМ-алгоритма}

Как было экспериментально установлено, ЕМ-алгоритм обладает
сильной неустойчивостью по начальным данным. Например, в случае
четырехкомпонентной смеси нормальных законов при объеме выборки
200--300 наблюдений замена лишь одного наблюдения другим может
кардинально изменить итоговые оценки, полученные с помощью
ЕМ-алгоритма.

Для противодействия указанной неустой\-чи\-вости ЕМ-алгоритма можно
использовать его медианные модификации. Смысл этих модификаций в
том, что наиболее <<неустойчи\-вые>> этапы выполнения ЕМ-алгоритма
заменяются более устойчивыми. В частности, на М-этапе неустойчивые
моментные оценки наибольшего правдоподобия (которые для нормальных
компонент минимизируют квадратичный риск) заменяются более
устойчивыми (робастными) оценками медианного типа, оптимальными в
смысле среднего абсолютного отклонения. Более того, в задачах
разделения смесей вероятностных распределений выборочные медианы
при определенных соотношениях между параметрами смеси оказываются
более эффективными, нежели оценки максимального правдоподобия типа
выборочных моментов, т.\,е.\ иногда выборочные медианы оптимальны
не только в смысле среднего абсолютного отклонения, но и в
традиционном смысле квадратичного риска.

Опишем две возможные медианные модификации М-этапа ЕМ-алгоритма. В
рамках этих модификаций параметры $a_i$ оцениваются одинаково.
Различными являются лишь оценки параметров $\sigma_i$.

Пусть числа $g_{ij}^{(m)}$ известны. По числам $g_{ij}^{(m)}$
определим <<вероятности>> $p_{ij}^{(m)}$ по правилу
$$
p_{ij}^{(m)}=g_{ij}^{(m)}\bigg(\sum\limits_{j=1}^{n}g_{ij}^{(m)}\bigg)^{-1}\,,\  \ i=1,\ldots,k;\ j=1,\ldots,n$$
($n$~--- объем выборки, $k$~--- число компонент смеси). Пусть
$\emph{\textbf{x}} =(x_1,\ldots,x_n)$~--- выборка. Тогда число
$p_{ij}^{(m)}$ можно интерпретировать как вероятность того, что
наблюдение $x_j$ имеет распределение, определяемое $i$-й
компонентой смеси.

Введем <<фиктивные>> случайные величины $\xi_i^{(m)}$,
$i=1,\ldots,k$, которые соответственно принимают значение $x_j$ с
вероятностями $p_{ij}^{(m)}$, $i=1,\ldots,k$, $j=1,\ldots,n$
(несложно видеть, что $\sum\limits_{j=1}^{n}p_{ij}^{(m)}=1$). При этом
оценка параметра сдвига $i$-й компоненты смеси на $(m+1)$-й
итерации, приведенная в предыду\-щем разделе, оказывается в точности
равной математическому ожиданию случайной величины $\xi_i^{(m)}$:
\begin{multline*}
a_{i}^{(m+1)}=\fr{1}{\sum\limits_{j=1}^{n}g_{ij}^{(m)}}
\sum\limits_{j=1}^{n}g_{ij}^{(m)}x_{j}={}\\
{}=\sum\limits_{j=1}^{n}p_{ij}^{(m)}x_{j}={\sf
E}_{\theta^{(m)}}\xi_i^{(m)}\,.
\end{multline*}

Для того чтобы построить модификацию ЕМ-алгоритма, более
устойчивую по отношению к наличию <<засоряющих>> наблюдений (а при
оценивании параметров какой-либо компоненты смеси\linebreak
наблюдения,
распределения которых соответствуют другим компонентам, неизбежно
будут <<засоряющими>> по отношению к оцениваемой компоненте), в
качестве оценки параметра $a_i$ на $(m+1)$-й итерации предлагается
взять медиану $\mathrm{med}\,\xi_i^{(m)}$ случайной величины
$\xi_i^{(m)}$, которую можно вычислить так. Переупорядочим
значения $x_1,\ldots,x_n$ случайной величины $\xi_i^{(m)}$ по
неубыванию. Получим вариационный ряд $x_{(1)},\ldots,x_{(n)}$.
Ясно, что одно и то же переупорядочение имеет место для значений
всех случайных величин $\xi_i^{(m)}$. Одновременно переставятся и
вероятности $p_{ij}^{(m)}$, соответствующие значениям каждой
случайной величины $\xi_i^{(m)}$. Пусть $\widehat p_{ij}^{(m)}$~---
это та из вероятностей $p_{ij}^{(m)}$, которая соответствует
значению~$x_{(j)}$ случайной величины $\xi_i^{(m)}$. Положим
$$
J_i=\min \left \{ j:\, \widehat p_{i1}^{(m)}+\widehat p_{i2}^{(m)}+\ldots+\widehat
p_{ij}^{(m)}\ge \fr{1}{2}\right \}\,.
$$
Тогда
\begin{equation}
a_i^{(m+1)}=\mathrm {med}\,\xi_i^{(m)}=x_{(J_i)}\,.
%\label(e4)
\end{equation}

Для оценивания параметра $\sigma_i$ на $(m+1)$-й итерации сначала
по указанной выше схеме вычислим медиану случайной величины
$\left | \xi_i^{(m)}-a_i^{(m+1)}\right |$,
$$
\widehat m_i^{(m+1)}=\mathrm{med}\left |\xi_i^{(m)}-a_i^{(m+1)}\right |\,.
$$
Затем введем <<фиктивную>> случайную величину $\zeta_i^{(m+1)}$ с
функцией распределения
$$
{\sf P}_{\theta^{(m+1)}}\left ( \zeta_i^{(m+1)}<x\right ) =\Phi\left(\fr{x-a_i^{(m+1)}}{\sigma_i^{(m+1)}}\right)\,,
$$
т.\,е.\ распределение случайной величины $\zeta_i^{(m+1)}$
является $i$-й компонентой смеси. <<Эмпирическим>> аналогом
случайной величины $\zeta_i^{(m+1)}$ является случайная величина
$\xi_i^{(m)}$, введенная ранее. В идеале (при достаточно большом
$m$ и при большом~$n$) должно быть справедливо приближенное
равенство ${\sf P}_{\theta^{( m+1)}}\left(\zeta_i^{(m+1)}<x\right )\approx{\sf
P}_{\theta^{(m+1)}}\left ( \xi_i^{(m)}<x\right )$, $-\infty<$\linebreak $<\;x<+\infty$.

Таким образом, отыскав эмпирическую медиану $\widehat m_i^{(m+1)}$
(т.\,е.\ медиану случайной величины
$\left |\xi_i^{(m)}-a_i^{(m+1)}\right |$), в соответствии с идеологией
метода моментов можно сказать, что она близка к медиане
$\mu_i^{(m+1)}$ случайной величины
$\left |\zeta_i^{(m+1)}-a_i^{(m+1)}\right |$.

Медиана $\mu_i^{(m+1)}$ случайной величины
$\bigg|\zeta_i^{(m+1)}-$\linebreak $-\;a_i^{(m+1)}\bigg|$ определяется из условия
$$
{\sf
P}_{\theta^{(m+1)}}\left (\left | \zeta_i^{(m+1)}-a_i^{(m+1)}\right |\le\mu_i^{(m+1)}\right )=\fr{1}{2}\,.
$$
Но
\begin{multline*}
{\sf P}_{\theta^{(m+1)}}\left(\left |\zeta_i^{(m+1)}-a_i^{(m+1)}\right |\le\mu_i^{(m+1)}\right)={}
\\
{}={\sf
P}_{\theta^{(m+1)}}\left(-\mu_i^{(m+1)}\le\zeta_i^{(m+1)}-a_i^{(m+1)}\le{}\right.\\
\left.{}\le \mu_i^{(m+1)}\right)=
{\sf
P}_{\theta^{(m+1)}}\left(a_i^{(m+1)}-\mu_i^{(m+1)}\le{}\right.\\
\left.{}\le
\zeta_i^{(m+1)}\le a_i^{(m+1)}+\mu_i^{(m+1)}\right)= {}
\\
{}=\Phi\left(\fr{\left (a_i^{(m+1)}+\mu_i^{(m+1)}\right )-a_i^{(m+1)}}
{\sigma_i^{(m+1)}}\right)-{}\\
{}-\Phi\left(\fr{\left( a_i^{(m+1)}-\mu_i^{(m+1)}\right )-a_i^{(m+1)}}{\sigma_i^{(m+1)}}\right)={}
\\
{}=
\Phi\left(\fr{\mu_i^{(m+1)}}{\sigma_i^{(m+1)}}\right)-\Phi\left(-\fr{\mu_i^{(m+1)}}{\sigma_i^{(m+1)}}\right)={}\\
{}=
2\Phi\left(\fr{\mu_i^{(m+1)}}{\sigma_i^{(m+1)}}\right)-1\,.
\end{multline*}
Следовательно, справедливо соотношение
$$
2\Phi\left(\fr{\mu_i^{(m+1)}}{\sigma_i^{(m+1)}}\right)-1=\fr{1}{2}\,,
$$
т.\,е.\
$$
\Phi\left(\fr{\mu_i^{(m+1)}}{\sigma_i^{(m+1)}}\right)=\fr{3}{4}\,,
$$
что эквивалентно соотношению
$$
\fr{\mu_i^{(m+1)}}{\sigma_i^{(m+1)}}=u_{3/4}\,,
$$
где $u_{3/4}$~---
квантиль порядка $3/4$ стандартного нормального закона. В таблицах
находим $u_{3/4}\approx 0{,}6745$. Следуя идеологии метода моментов,
приравняем эмпирическую медиану $\widehat m_i^{(m+1)}$
теоретической медиане $\mu_i^{(m+1)}$ и окончательно получим
уравнение для оценки параметра $\sigma_i$ на $(m+1)$-й итерации:
\begin{equation}
\sigma_i^{(m+1)}=\fr{\widehat
m_i^{(m+1)}}{u_{3/4}}=1{,}4826\,\widehat m_i^{(m+1)}\,.
%\eqno(5)
\end{equation}

Оценки $p_i^{(m+1)}$ весов $p_i$ в модели~(1) ищутся, как и ранее,
по формулам~(2). Числа же $g_{ij}^{(m+1)}$ на каждой итерации
переназначаются так же, как и ранее, а именно
\begin{multline}
g_{ij}^{(m+1)}={}\\
{}=\fr{\fr{p_i^{(m+1)}}{\sigma_i^{(m+1)}}\exp\bigg\{-\fr{1}{2}\left(
\fr{x_j-a_i^{(m+1)}}{\sigma_i^{(m+1)}}\right)^2\bigg\}}
{{\displaystyle{\sum\limits_{r=1}^k\fr{p_r^{(m+1)}}{\sigma_r^{(m+1)}}\exp\left\{
-\fr{1}{2}\left(\fr{x_j-a_r^{(m+1)}}{\sigma_r^{(m+1)}}\right)^2\right\}}}}
\,.
%\eqno(6)
\end{multline}

Итак, соотношения~(2), (4)--(6) определяют первую медианную
модификацию ЕМ-алгоритма.

Вторая медианная модификация ЕМ-ал\-го\-рит\-ма отличается от первой
лишь способом оценивания параметров $\sigma_i$. А именно: вычислим\linebreak
${\sf E}_{\theta^{(m+1)}}\left |\zeta_i^{(m+1)}-a_i^{(m+1)}\right|$.
Имеем
\begin{multline*}
{\sf E}_{\theta^{(m+1)}}\left |\zeta_i^{(m+1)}-
a_i^{(m+1)}\right |={}\\
{}=
\int\limits\limits_{-\infty}^{\infty}\left |x-a_i^{(m+1)}\right |\,d_x\Phi\left(
\fr{x-a_i^{(m+1)}}{\sigma_i^{(m+1)}}\right)={}
\\
{}=2\int\limits_{0}^{\infty}xd_x\Phi\left(\fr{x}{\sigma_i^{(m+1)}}\right)=\sigma_i^{(m+1)}\sqrt{\fr{2}{\pi}}\,.
\end{multline*}
Эмпирическим аналогом величины\linebreak
${\sf E}_{\theta^{(m+1)}}\left|\zeta_i^{(m+1)}-a_i^{(m+1)}\right|$ является
величина
\begin{multline*}
s_i^{(m+1)}={\sf E}_{\theta^{(m)}}\left |\xi_i^{(m)}-a_i^{(m+1)}\right |={}\\
{}=
\sum\limits_{j=1}^np_{ij}^{(m)}\left |x_j-a_i^{(m+1)}\right |\,.
\end{multline*}
Реализуя метод моментов и приравнивая величину ${\sf
E}_{\theta^{(m+1)}}\left |\zeta_i^{(m+1)}-a_i^{(m+1)}\right |$ ее
эмпирическому аналогу, получаем еще одну оценку для параметра
$\sigma_i$ на $(m+1)$-й итерации:
\begin{equation}
\sigma_i^{(m+1)}=\sqrt{\fr{\pi}{2}}\cdot s_i^{(m+1)}=1{,}2533 s_i^{(m+1)}\,.
%\eqno(7)
\end{equation}
Таким образом, вторая медианная модификация ЕМ-алгоритма
определяется соотношениями~(2), (4), (7) и~(6).

Заметим, что вторая модификация более соответствует духу так
называемой $L_1$-теории устойчивого оценивания в силу
известного свойства
\begin{multline*}
\arg\min_{a}{\sf E}_{\theta^{(m+1)}}\left |\zeta_i^{(m+1)}-a\right |=
\mathrm{med}\,\zeta_i^{(m+1)}\\
\ \ \ \ \  \ \left( \approx \mathrm{med}\,\xi_i^{(m)}=a_i^{(m+1)}\right)\,.
\end{multline*}

\section{SEM-алгоритм}

Классический EM-алгоритм выбирает первый попавшийся локальный
максимум, т.\,е., являясь методом локальной оптимизации, он
приводит не к глобальному максимуму функции правдоподобия, а к тому
локальному максимуму, который является ближайшим к начальному
приближению.

Самый простой способ противодействия этому свойству заключается в
том, чтобы, не ограничиваясь единственным начальным приближением
и, соответственно, единственной траекторией EM-алгоритма,
реализовать несколько траекторий, задавая (например, случайно)
несколько различных начальных приближений, а затем выбрать тот из
результатов, для которого правдоподобие является наибольшим среди
всех реализованных траекторий EM-алгоритма. Однако при таком
подходе остается неясным ответ на вопрос о том, каким механизмом
разумнее всего пользоваться при переходе от одного начального
приближения к другому. В~частности, когда начальное приближение
задается случайно, без дополнительной информации нельзя
исчерпывающим образом определить распределение вероятностей, в
соответствии с которым следует генерировать очередное начальное
приближение.

Другой, оказавшийся весьма эффективным, способ заключается как бы
в случайном <<встряхивании>> наблюдений (выборки) на каждой
итерации. Этот способ лежит в основе SEM-алгоритма, название
которого является аббревиатурой термина {\it Stochastic
EM-algorithm} (стохастический (или
случайный) EM-алгоритм)~\cite{Korolev2007b}.

Чтобы описать SEM-алгоритм, представим ненаблюдаемую информацию в
иной форме (однако, по сути, эквивалентной старой форме). А именно:
будем считать, что каждому наблюдению $x_{j}$ соответствует
вектор $\vec{y}_{j}=(y_{1j},y_{2j}%
,\ldots,y_{kj})$, $j=1,\ldots,n$, где $k$~--- число компонент смеси,
$n$~--- объем
выборки. При этом%
\begin{equation*}
y_{ij}=\begin{cases}
1\,, &\mbox{если
наблюдение}\ x_{j}\\
& \mbox{порождено }i\mbox{-й компонентой
смеси}\,;\\
0\,, &\mbox{в противном случае.}
\end{cases}
\end{equation*}


При каждом $j$ только одна из компонент вектора
$\vec{y}_{j}$ равна единице, остальные компоненты этого вектора равны нулю.

В терминах величин $\mathbf{y}=\{
\vec{y}_{j}=(y_{1j},y_{2j},\ldots,y_{kj})$, $j=1,\ldots,n\}  $
логарифм полной функции правдоподобия для модели~$(1)$
принимает вид%
\begin{multline}
\log L(\theta;x,y)=
\sum\limits_{j=1}^{n}
\sum\limits_{i=1}^{k}
y_{ij}\log\left [ p_{i}\psi_{i}(x_{j};t_{i})\right]={}\\
{}=
\sum\limits_{i=1}^{k}
\log p_{i}
\sum\limits_{j=1}^{n}
y_{ij}+%
\sum\limits_{i=1}^{k}
\sum\limits_{j=1}^{n}
y_{ij}\log\psi_{i}(x_{j};t_{i})\,.
\end{multline}


Векторы $\vec{y}_{j}=(y_{1j}%
,y_{2j},\ldots,y_{kj})$, $j=1,\ldots,n$, разбивают исходную
наблюдаемую выборку $\mathbf{x}$ на $k$~классов (кластеров)
$K_{1},\ldots,K_{k}$:
$$
\mathbf{x}=K_{1}\cup\ldots\cup K_{k}\,.
$$

Для каждого $i=1,\ldots,k$ с формальной точки зрения $K_{i}$~---
это множество тех наблюдений $x_{j}$, каж\-до\-му из которых
соответствует $y_{ij}=1$. При этом каждое наблюдение $x_{j}$
входит ровно в один кластер, т.\,е.\ $K_{i}\cap K_{j}=\varnothing$
при $i\neq j.$ Пусть $\upsilon_{i}$~--- это число наблюдений,
попавших в кластер $K_{i}$, $i=1,\ldots,k$,
$$
\upsilon_{i}=%
%TCIMACRO{\dsum \limits_{j=1}^{n}}%
%BeginExpansion
\sum\limits_{j=1}^{n}
%EndExpansion
y_{ij}\,.
$$


Очевидно, что $\upsilon_{1}+\ldots+\upsilon_{k}=n$. Тогда,
продолжая~$(8)$, для логарифма полной функции правдоподобия в
модели~$(1)$ получаем представление
\begin{multline}
\log L(\theta;x,y)=
\sum\limits_{i=1}^{k}
\upsilon_{i}\log p_{i}+{}
\\
{}+
\sum\limits_{i=1}^{k}
\sum\limits_{j:\text{ }x_{j}\in K_{i}}
\log\psi_{i}(x_{j};t_{i})\,.
%\eqno(9)
\end{multline}


Если бы величины $y_{ij}$ были известны, то искать значение
$\theta$, максимизирующее функцию
правдоподобия~$(9)$, можно было бы, максимизируя по~$\theta$ каждое из слагаемых в
правой части~$(9)$, поскольку эти слагаемые зависят только от
<<своих>> групп параметров. А именно: с помощью метода
неопределенных множителей Лагранжа несложно убедиться, что
максимум первого слагаемого по набору\linebreak $p_{1},\ldots,p_{k}$ при
очевидном ограничении $p_{1}+\ldots$\linebreak $\ldots +p_{k}$ $=1$
достигается при
\begin{equation}
p_{i}^{\ast}=\fr{\upsilon_{i}}{n}\,.
%\eqno(10)
\end{equation}


Далее заметим, что
\begin{multline*}
\sum\limits_{j:\text{ }x_{j}\in K_{i}}
\log\psi_{i}(x_{j};t_{i})={}
\\
{}=\log
\prod\limits_{j:\text{ }x_{j}\in K_{i}}
\psi_{i}(x_{j};t_{i})\equiv\log L_{i}(t_{i};K_{i}),
\end{multline*}
где $L_{i}(t_{i};K_{i})$~--- это функция правдоподобия параметра
$t_{i}$, построенная по подвыборке (кластеру) $K_{i}$ в
предположении, что каждый элемент подвыборки имеет плотность
распределения $\psi_{i}(x_{j};t_{i})$. Отсюда видно, что значения
\begin{equation}
t_{i}^{\ast}=\arg\max L_{i}(t_{i};K_{i}),\ i=1,\ldots,k\,,
%\eqno(11)
\end{equation}
доставляют максимум второму слагаемому в правой части ~$(9)$.
Легко видеть, что соотношение~$(11)$ определяет обычные оценки
наибольшего правдоподобия для параметров $i$-й компоненты
смеси~$(1)$, построенные по подвыборке наблюдений, распределение
которых равно этой компоненте, т.\,е.\ по кластеру $K_{i}$.

Таким образом, если бы величины $y_{ij}$ были известны, то оценки
наибольшего правдоподобия параметров модели~$(1)$ определялись бы
соотношениями~$(10)$ и~$(11)$. Однако на практике величины~$y_{ij}$
неизвестны. Идея SEM-алгоритма заключается в том, что эти
величины определяются с помощью специального имитационного
моделирования.

Итерационный SEM-алгоритм определяется так. Предположим, что
известны значения $g_{ij}^{(m)}$ апостериорных вероятностей
принадлежности наблюдения $x_{j}$ к кластеру $K_{i}$,
$i=1,\ldots,k$; $j=1,\ldots,n$; $m$~--- номер итерации
(отметим, что
$
\sum\limits_{i=1}^{k}
g_{ij}^{(m)}=1
$
для каждого $j$ и при каждом $m$).

На первом этапе SEM-алгоритма ($S$\textit{-этапе}, от слов
$Stochastic$ или $Simulation$) для каждого $j=1,\ldots,n$
генерируются
векторы $\vec{y}_{j}^{(m+1)}%
=$\linebreak $\;=\left ( y_{1j}^{(m+1)},y_{2j}^{(m+1)},\ldots,y_{kj}^{(m+1)}\right )$ как
реализации\linebreak
 случайных векторов с полиномиальным распределением с
параметрами~1 и $g_{1j}^{(m)},\ldots,g_{kj}^{(m)}$ ($g_{ij}^{(m)}$~---
это вероятность того, что $y_{ij}^{(m+1)}=1$). По векторам $\vec{y}_{j}^{(m+1)}$ определяется разбиение
выборки $\mathbf{x}=(x_{1}%
,\ldots,x_{n})$ на
кластеры\ $K_{1}^{(m+1)},\ldots,K_{k}^{(m+1)}$ и соответствующие числа
$\upsilon_{1}^{(m+1)},\ldots,\upsilon _{k}^{(m+1)}$ (численности
кластеров) на $(m+1)$-й итерации. (Можно сказать, что на $S$-этапе
реализуется случайное <<встряхивание>> исходной выборки, о котором
говорилось выше.)

На втором
этапе ($M$\textit{-этапе}), этапе
\textit{максимизации}, в соответствии с формулами ~$(10)$ и ~$(11)$ вычисляются оценки
максимального правдоподобия компонент
параметра $\theta$:
\begin{align}
p_{i}^{(m+1)}&=\fr{\upsilon_{i}^{(m+1)}}{n}\,;\\
%\eqno(12)\\
t_{i}^{(m+1)}&=\arg\max\limits_{t}L_{i}\left (t;K_{i}^{(m+1)}\right ),i=1,\ldots
,k\,.
%\eqno(13)
\end{align}


Наконец, на третьем
этапе ($E$\textit{-этапе}), переназначаются вероятности $g_{ij}$. Название
этого этапа восходит к слову $Expectation$. Это обусловлено тем,
что если
$\vec{Y}_{j}^{(m+1)}=\left ( Y_{1j}^{(m+1)},Y_{2j}^{(m+1)},\ldots,Y_{kj}^{(m+1)}\right )$~---
это случайный вектор, реализацией которого является вектор
$\vec{y}_{j}^{(m+1)},$ а $\vec{X}=(X_{1},\ldots,X_{n})$~--- это
случайный вектор, реализацией которого является\linebreak
выборка $\mathbf{x}=(x_{1}%
,\ldots,x_{n})$, то по определению%
$$
g_{ij}^{(m+1)}=E_{\theta^{(m+1)}}\left ( Y_{ij}^{(m+1)}|X_{j}\right )
$$
($Y_{ij}^{(m+1)}$~--- это индикатор (случайного) события $\left \{
X_{j}\in K_{i}^{(m+1)}\right \}$, а математическое ожидание
индикатора случайного события равно вероятности этого события).
При известном значении $X_{j}=x_{j}$ имеем
\begin{equation}
g_{ij}^{(m+1)}=\fr{p_{i}^{(m+1)}\psi_{i}\left (x_{j};t_{i}^{(m+1)}\right )}%
{\sum\limits_{r=1}^{k}p_{r}^{(m+1)}\psi_{r}\left(x_{j};t_{r}^{(m+1)}\right)}%
\,.
%\eqno(14)
\end{equation}

Для случая смеси нормальных распределений,
в которой%
$$
\psi_{i}(x;t_{i})=\fr{1}{\sigma_{i}}\,\phi\left(  \fr{x-a_{i}}{\sigma_{i}%
}\right)\  ,x\in%
\mathbb{R} \,,
$$
SEM-алгоритм выглядит так. Соотношение~$(12)$ остается без
изменений, соотношение~$(13)$ трансформируется
в два соотношения:
\begin{multline}
a_{i}^{(m+1)}=\fr{1}{\upsilon_{i}^{(m+1)}}\sum\limits_{j=1}^{n}
y_{ij}^{(m+1)}x_{j}={}\\
{}=\fr{1}{\upsilon_{i}^{(m+1)}}\sum\limits_{j:\text{ }%
x_{j}\in K_{i}^{(m+1)}}x_{j}\,;
%\eqno(15)
\end{multline}
\begin{multline*}
\sigma_{i}^{(m+1)}={}\\
{}=\left[  \fr{1}{\upsilon_{i}^{(m+1)}}\sum\limits_{j=1}%
^{n}y_{ij}^{(m+1)}\left(  x_{j}-a_{i}^{(m+1)}\right)  ^{2}\right]^{1/2}={}
\\
{}
=\left[ \fr{1}{\upsilon_{i}^{(m+1)}}\sum\limits_{j:\text{
}x_{j}\in
K_{i}^{(m+1)}}\left(  x_{j}-a_{i}^{(m+1)}\right)  ^{2}\right]^{1/2}\,.
%\eqno(16)
\end{multline*}
Соотношение же~$(14)$ примет вид
\begin{multline*}
g_{ij}^{(m+1)}={}\\
{}=\fr{\fr{p_{i}^{(m+1)}}{\sigma_{i}^{(m+1)}}\exp\left\{
-\fr{1}{2}\left(
\fr{x_{j}-a_{i}^{(m+1)}}{\sigma_{i}^{(m+1)}}\right)^{2}\right\}  }
{\sum\limits_{r=1}^{k}\fr{p_{r}^{(m+1)}}{\sigma_{r}^{(m+1)}%
}\exp\left\{  -\fr{1}{2}\left(  \fr{x_{j}-a_{r}^{(m+1)}}{\sigma
_{r}^{(m+1)}}\right)^{2}\right\}  }\,.
%\eqno(17)
\end{multline*}


\section{Медианная модификация SEM-алгоритма}

Так как на каждой итерации SEM-алгоритма в каждый из
кластеров $K_{1}^{(m+1)},\ldots,K_{k}^{(m+1)}$ могут попасть <<лишние>> наблюдения,
фактически распределенные в соответствии с другими компонентами
смеси, то можно рассмотреть устойчивые медианные модификации.

Медианные модификации SEM-алгоритма определяются следующим образом.
Упорядочим элементы
выборки $\mathbf{x}=(x_{1}%
,\ldots,x_{n})$, попавшие в кластер $K_{i}^{(m+1)}$, по неубыванию.
Полученный в результате набор
обозначим
$$
K_{i}^{(m+1)}=\left\{
x_{i,1}^{(m+1)},\ldots,x_{i,\upsilon_{i}}^{(m+1)}\right\}\,.
$$
В этом
случае для оценки параметров используется выборочная медиана для
каждого кластера.

В случае смеси нормальных компонент вмес\-то~$(15)$ можно использовать
более устойчивую \mbox{оценку}
\begin{equation*}
a_{i}^{(m+1)}=
\begin{cases}
&\fr{1}{2}\left(
x_{i,\upsilon_{i}^{(m+1)}/2}^{(m+1)}+x_{i,\upsilon_{i}^{(m+1)}/2+1}^{(m+1)}\right)\,, \\
&\ \ \ \ \ \ \ \ \ \ \ \ \ \ \ \ \mbox{если }\upsilon_{i}^{(m+1)}\mbox{ --- четное\,;}\\
&x_{i,\left[  \upsilon_{i}^{(m+1)}/2\right]  +1}^{(m+1)}\,, \\
&\ \ \ \ \ \ \ \ \ \ \ \ \ \ \ \ \mbox{ если }\upsilon_{i}^{(m+1)}\mbox{ ---\
нечетное\,,}
\end{cases}
\end{equation*}
где символ $[z]$ обозначает целую часть числа $z$. Другими
словами, в качестве оценки параметра $a_{i}$ на $(m+1)$-й итерации
SEM-алгоритма можно использовать выборочную медиану кластера
$K_{i}^{(m+1)}$.

В качестве оценки параметра $\sigma_{i}$ на $(m+1)$-й итерации
SEM-алгоритма можно взять
$$
\sigma_{i}^{(m+1)}=\sqrt{\fr{\pi}{2}}\cdot S_{i}^{(m+1)}\,,
$$
где $S_{i}^{(m+1)}$--- выборочное среднее абсолютное отклонение, вычисленное для
кластера $K_{i}^{(m+1)}$:
$$
S_{i}^{(m+1)}=\fr{1}{\upsilon_{i}^{(m+1)}}\sum\limits_{j=1}^{\upsilon_{i}^{(m+1)}}
\left\vert x_{i,j}^{(m+1)}-a_{i}^{(m+1)}\right\vert\,.
$$


Таким образом, SEM-алгоритм и его медианная модификация
представляют собой методы для оценивания неизвестных параметров
компонент смеси без каких-либо дополнительных предположений об
этих параметрах (например, предположения о равенстве нулю параметра
$a_{i}$ для каждой компо\-ненты).

\section{Выбор точности приближений}

Свойства SEM-алгоритма были подвергнуты исследованию в~\cite{Celeux, Ip}. В частности, в этих работах для
многих достаточно общих конкретных случаев отмечено, что построенная SEM-алгоритмом последовательность $\left\{
\theta^{(m)}\right\}  _{m\geq1}$, вообще говоря, не сходится с вероятностью единица, но образует цепь Маркова, которая при
некоторых дополнительных условиях регулярности довольно быстро
сходится к стационарному распределению. Стационарность достигается после довольно продолжительного\linebreak
 периода <<приработки>>  алгоритма.
При этом получаемые с помощью SEM-алгоритма оценки па\-ра\-мет\-ров смеси являются асимптотически несмещенными в том смысле, что
оценка \mbox{максимального} правдоподобия параметров смеси является асимптотически эквивалентной математическому ожиданию
$\theta^{(m)}$ относительно стационарного распределения. Поэтому в
качестве <<окончательной>> оценки~$\widetilde{\theta}^{(m)}$
параметра $\theta$ после $m$ итераций SEM-алгоритма в упомянутых
работах предлагается использовать <<выборочное среднее>>
$$
\widetilde{\theta}^{(m)}=\widetilde{\theta}^{(m)}(m_{0})=\fr{1}{m-m_{0}}
\sum\limits_{r=m_{0}+1}^{m}\theta^{(m)}\,,
$$
где $m_{0}$~--- настолько большое число, что при $r>m_{0}$ цепь
Маркова $\theta^{(r)}$ близка к стационарному режиму.

Многочисленные реализации SEM-алгоритма показали, что он работает
относительно быстро по сравнению с другими методами, результаты
его работы практически не зависят от начального приближения,
он позволяет избегать выхода на неустойчивые локальные максимумы
анализируемой функции правдоподобия за счет постоянного случайного
<<встряхивания>> выборки и, более того, как правило, приводит к
глобальному максимуму этой функции. Кроме того, SEM-алгоритм легко
модифицировать с целью отыскания числа $k$ компонент смеси, если
оно заранее неизвестно (отметим, что в реализованных алгоритмах
фактически используется данная идея: если в кластер попадает менее
двух элементов, он считается пустым и удаляется).


\section{Декомпозиция волатильности с~помощью метода скользящего
разделения смесей}

Возможности описанных выше медианных модификаций EM- и
SEM-алгоритмов для разделения конечных смесей нормальных законов
будут проиллюстрированы на примере решения задачи декомпозиции
волатильности (т.\,е.\ задачи разложения волатильности на
компоненты) некоторых финансовых индексов.

Статистическое оценивание параметров конечных смесей нормальных
законов является ядром метода скользящего разделения смесей
(СРС-ме\-то\-да), предназначенного для исследования стохастической
структуры хаотических процессов и, в частности, для исследования
волатильности финансовых индексов и других показателей.

Теоретические основы СРС-метода можно кратко описать следующим
образом (см.~\cite{Korolev2007c, Korolev2007a}).

%\smallskip
%\noindent
\begin{enumerate}[(1)]
\item %(1)~
Асимптотический подход, основанный на предельных
теоремах для обобщенных дважды стохастических пуассоновских
процессов как моделей неоднородных хаотических случайных
блужданий, естественно приводит к заключению о том, что
аппроксимации для распределений (логарифмов) приращений процессов
эволюции финансовых индексов на интервалах времени умеренной длины
следует искать в виде общих сдвиг-масштабных смесей нормальных
законов, в которых смешивающий закон определяется накопленной
(интегральной) интенсивностью потоков со\-от\-вет\-ст\-ву\-ющих
информативных событий (элементарных скачков, <<тиков>>).

\item Проблема статистической реконструкции распределений
приращений упомянутых процессов (или их логарифмов) сводится к
задаче\linebreak статистического оценивания смешивающего\linebreak распределения,
которое является параметром этой задачи.

\item В самой общей постановке задача статистического
оценивания смешивающего распределения является некорректной, так
как общие сдвиг-масштабные смеси нормальных законов не являются
идентифицируемыми. Таким образом, в рамках общего принципа
регуляризации некорректных задач исходная проблема заменяется
задачей отыскания решения, наиболее близкого к истинному в классе
конечных дискретных сдвиг-масштабных смесей нормальных законов.
Эта <<редуцированная>> задача уже является корректной и имеет
единственное решение, так как семейство конечных дискретных
сдвиг-масштабных смесей нормальных законов идентифицируемо.
Поскольку сдвиг-масштабные смеси нормальных законов обладают
свойством устойчивости относительно смешивающего закона, эта
замена оправданна и регулярна. При этом, зная оценки устойчивости,
можно вычислить погрешности, образующиеся при замене исходной
задачи редуцированной. При упомянутой регуляризации происходит
автоматическое выделение типичных или более-менее устойчивых
структур в эволюции рассматриваемых сложных систем.

\item Представление распределений (логарифмов) приращений
процессов эволюции финансо\-вых индексов в виде конечных
сдвиг-мас\-штаб\-ных смесей нормальных законов естественно\linebreak приводит к
многомерной интерпретации волатильности рассматриваемого процесса
и к возможности разложения волатильности на динамическую и
диффузионную компоненты.\linebreak
 Действительно, если функция распределения
логарифмического приращения $Z$ некоторого финансового индекса
имеет вид~(1), то для нее справедливо представление
\begin{multline*}
F(x)={\sf
P}(Z<x)=\sum\limits_{j=1}^kp_j\Phi\left(\fr{x-a_j}{\sigma_j}\right)={}\\
{}={\sf
E}\Phi\left(\fr{x-V}{U}\right)\,,
\end{multline*}
где пара случайных величин $U,V$ имеет дискретное распределение
$$
{\sf P}((U,V)=(\sigma_j,a_j))=p_j\,,\ \ \ \ j=1,\ldots,k\,.
$$
Так что, как продемонстрировано в книге~\cite{Korolev2007a},
\begin{equation}
{\sf D}Z={\sf D}V+{\sf E}U^2\,,
%\eqno(8)
\end{equation}
причем
\begin{equation}
{\sf D}V=\sum\limits_{j=1}^k(a_j-\overline a)^2p_j\,,\ \ \ \ {\sf
E}U^2=\sum\limits_{j=1}^kp_j\sigma_j^2\,,
%\eqno(9)
\end{equation}
где
$$
\overline a=\sum\limits_{j=1}^ka_jp_j\,.
$$
Волатильность индекса естественно отождествить с величиной ${\sf
D}Z$ (или $\sqrt{{\sf D}Z}$). При этом первое выражение в~(17)
зависит только от весов $p_j$ и параметров положения (сдвига)~$a_j$
компонент, и потому характеризует ту часть волатильности,
которая обусловлена наличием\linebreak
 локальных трендов, т.\,е.\ <<динамическую>> компоненту волатильности, тогда как второе
выражение в~(17) зависит только от весов $p_j$ и па\-ра\-мет\-ров масштаба (<<коэффициентов диффузии>>) $\sigma_j$ компонент и
потому характеризует <<чисто диффузионную>> компоненту волатильности.

Если вспомнить традиционное одномерное представление о волатильности как о стандартном отклонении приращения процесса, то
можно заметить, что разложение~(16) уточняет это представление:
волатильность процесса представляет собой корень квадратный из суммы двух компонент, первая из которых является характеристикой
разбросанности локальных трендов, а вторая характеризует диффузию
процесса. Если локальные тренды отсутствуют, то классическая волатильность равна корню квадратному из взвешенной суммы
квадратов волатильностей компонент, причем веса компонент показывают важность соответствующей диффузионной компоненты.

\item Статистические закономерности поведения
рассматриваемых процессов, формализованные в пункте~(1),
изменяются во времени, результатом чего является отсутствие {\it
универсального} сме\-ши\-ва\-юще\-го закона. Таким обра-\linebreak зом, чтобы изучить
динамику изменения статистических закономерностей в поведении\linebreak
ис\-сле\-ду\-емо\-го хаотического процесса, задача статистического
разделения конечных смесей нормальных законов должна быть
последовательно решена на интервалах времени, постоянно
сдви\-га\-ющих\-ся в направлении <<астрономического>> времени. Тем самым
параметры смесей (параметры сдвига (дрейфа), масштаба (диффузии),
а также соответствующие веса) оцениваются как функции времени. При
этом естественно возникают задачи, связанные как с выбором
подходящих методов оценивания параметров сдвига и масштаба, так и
с выбором оптимальных па\-ра\-мет\-ров вычислительных процедур,
реализующих эти методы: начального приближения, ширины скользящего
интервала (окна), правила остановки и~др.

\item Наконец, для адекватной интерпретации результатов и
для идентификации феноменологически выделенных (статистически
оцененных) компонент, т.\,е.\ для адекватного сопоставления
статистически оцененных компонент с реальными процессами или
явлениями, необходимо из многих возможных моделей выбрать наиболее
адекватную, например проверить, является выделенная динамическая
компонента волатильности статистически значимой или нет.
\end{enumerate}

\section{Диффузионный спектр и~предполагаемый диффузионный
спектр}

Сказанное в пункте~(4) предыдущего раздела позволяет предложить
новую точку зрения на природу волатильности.

Чтобы дать более полный ответ на вопрос о том, что такое
волатильность, помня тем не менее о том, что среди финансовых
аналитиков <<волатильность>>~--- это ныне скорее обыденное понятие,
нежели математический термин, необходимо заметить, что
волатильность процесса (т.\,е.\ его изменчивость) обусловлена как
минимум двумя типами факторов. Первый тип факторов может быть
условно назван \textit{динамическим}. Влияние факторов такого типа
проявляется в том, что процесс изменяется из-за наличия некоторого
тренда или комбинации, взаимодействия трендов, отражающих интересы
некоторых (нескольких) групп участников рынка (простейшим примером
таких групп являются <<быки>>, играющие на повышение, и
<<медведи>>, играющие на понижение). Упрощенная физическая
интерпретация действия этой группы факторов может быть
проиллюстрирована примером движения, скажем, щепки в горной реке,
где имеются ярко выраженные течения или комбинации течений,
перемещающие щепку. Следует отметить, что обычно при описании
методов анализа эволюции финансовых индексов понятие тренда в
некотором смыс\-ле противопоставляется понятию во\-ла\-тиль\-ности и оба
эти понятия изучаются раздельно. Согласно сказанному в пункте~(4)
предыдущего раздела, СРС-метод показывает, что на самом деле между
этими понятиями имеется глубокая и непростая взаимосвязь, которую,
к счастью, можно довольно разумно описать, используя аппарат
теории вероятностей и математической статистики.

Факторы второго типа могут быть названы \textit{стохастическими}
или \textit{диффузионными}. Число факторов, влияющих на финансовые
рынки (параметры и интересы участников, новости и~т.\,п.), огромно,
так что на практике нереально учесть влияние каждого из них в
отдельности. Физическим примером действия факторов диффузионного
типа может служить суммарное воздействие молекул среды на частицу,
испытывающую броуновское (тепловое) движение.

Конечно, эта классификация факторов скорее условна, нежели строго
определена. Однако СРС-ме\-тод спонтанно и довольно естественно
выделяет факторы указанных типов и автоматически дает возможность
проследить их взаимосвязь (см.\ соотношения~(16) и~(17)). В
соответствии со сказанным выше в рамках основной модели,
используемой в данной работе, волатильность довольно естественно
раскладывается на \textit{динамическую} и \textit{диффузионную}
составляющие. В свою очередь, диффузионная составляющая
раскладывается на несколько разных компонент, каждая из которых
имеет свое собственное происхождение. Таким образом, в данной
работе рассматривается метод статистического разложения
волатильности \textit{скалярного} процесса в нетривиальную и
существенно \textit{многомерную} информативную картину.
Наблюдаемое значение финансового индекса в каждый момент времени
является интегральным (суммарным, усредненным) результатом
взаимодействия интересов и стратегий участников рынка. Подобно
тому, как Исаак Ньютон с помощью призмы разложил белый свет на
со\-став\-ля\-ющие его цвета радуги, с помощью СРС-метода наблюдаемый
интегральный индекс, характеризующий состояние рынка, можно в
определенном смысле разложить на составляющие его компоненты,
отражающие текущие интересы и стратегии характерных групп
участников рынка.

Как будет показано на примере исследования конкретных временн$\acute{\mbox{ы}}$х
рядов, описывающих эволюцию реальных финансовых индексов,
динамическая и диффузионные составляющие вносят в итоговую
волатильность примерно одинаковый вклад, причем в разные моменты
времени доминировать могут разные компоненты.

Если модели типа масштабных смесей нормальных законов оказываются
адекватными для распределений логарифмических приращений
финансовых индексов, то смешивающее распределение можно
интерпретировать как {\it диффузионный спектр} волатильности. Если
такие модели в целом оказываются неадекватными по какому-либо
критерию (например, адекватной является сдвиг-масштабная смесь или
вообще иная модель, не имеющая вид смеси нормальных законов), то
смешивающее распределение в наиболее правдоподобной модели типа
чисто масштабной смеси нормальных законов можно интерпретировать
как {\it предполагаемый диффузионный спектр} волатильности. В
последней ситуации (возможное) наличие динамической (трендовой)
составляющей волатильности игнорируется и все параметры положения
$a_j$ компонент смеси полагаются равными нулю.

Такой подход, в определенном смысле упрощающий изучаемую ситуацию
и, возможно, ее искажающий, тем не менее может оказаться полезным,
поскольку модели, имеющие вид чисто\linebreak
масштабных смесей, и модели
типа сдвиг-мас\-штаб\-ных смесей по-разному реагируют на их
<<уточнение>> с помощью добавления дополнительных компонент. Для
чисто масштабных смесей наблюдается эффект <<насыщения>>, когда
введение новых компонент сверх какого-то уровня (обычно
четыре--пять компонент) практически не меняет модель, в то время
как модели типа сдвиг-масштабных смесей при добавлении новых
компонент изменяются довольно существенно, превращаясь в своего
рода аналоги ядерных оценок плотности. Эффект <<насыщения>> ставит
естественный предел степени детализации модели, имеющей вид чисто
масштабной смеси нормальных законов.

При этом некоторые компоненты пред\-по\-ла\-га\-емо\-го диффузионного
спектра могут быть интерпретированы как компоненты шума, которым
объявляется невязка между моделью, имеющей вид чисто масштабной
смеси нормальных законов, и <<истинной>> (адекватной) моделью,
возможно, имеющей более сложный вид. В частности, если
<<истинная>> модель имеет вид сдвиг-масштабной смеси нормальных
законов, то к <<зашумляющей>> невязке при таком подходе относится
динамическая компонента волатильности.

Критерии адекватности или неадекватности моделей весьма
относительны. Например, известно довольно много информационных
критериев выбора моделей типа критерия Акаике, и в силу самог$\acute{\mbox{о}}$
их определения они могут приводить к разным решениям относительно
адекватности или неадекватности чисто масшабно-смешанных моделей.
Поэтому понятие предполагаемого диффузионного спектра оказывается
полезным в любом случае.

Будучи смешивающим распределением, диффузионный спектр может
оказаться как непрерывным (абсолютно непрерывным), так и
дискретным. Диффузионный спектр (предполагаемый диффузионный
спектр) характеризует распределение волатильности (отождествляемой,
возможно, с многомерным коэффициентом диффузии) по уровням в
каждый момент времени. Тем самым изменение диффузионного спектра
во времени характеризует перераспределение волатильности по
уровням. При этом множество допустимых уровней волатильности может
быть как континуальным (что\linebreak
соответствует непрерывному
предполагаемому диффузионному спектру), так и счетным (что
соответствует дискретному предполагаемому диффузионному спектру).

Наиболее приемлемым с точки зрения возможности удобной
интерпретации результатов является дискретный (конечный)
предполагаемый диффузионный спектр.

Рассматриваемые далее конкретные примеры проиллюстрируют
сказанное.

\section{Применение медианных модификаций EM-алгоритма для разделения
конечных смесей нормальных законов к~декомпозиции волатильности
конкретных финансовых~индексов}

\label{em}

В качестве исходных данных использовались минутные логарифмические
приращения четырех биржевых индексов: AMEX, CAC~40, Nasdaq~100 и
Nikkei. При этом ширина скользящего окна $n$ взята равной $300$
отсчетам (что соответствует пяти часам). Задача оценивания
параметров смесей последовательно решается для каждого положения
скользящего окна по выборке (отрезку исходного ряда),
соответствующей данному положению окна. С целью получения довольно
точных результатов использовался весьма строгий критерий остановки
итерационных процедур, согласно которому евклидово расстояние
между векторами значений оцениваемых параметров на
последовательных итерациях должно быть меньше $10^{-8}$. Число
возможных компонент смеси полагалось равным шести.

Результаты представлены на рис.~\ref{f1gr}--\ref{f8gr}. На каж\-дом графике
горизонтальная ось~--- это ось времени (каждая точка на
горизонтальной оси соответствует конкретному значению правого
конца скользящего интервала времени, по которому (т.\,е.\ по
наблюдениям, попавшим в который) вычисляются оценки параметров.
Вертикальная ось~--- это ось значений параметров $a_j=a_j(t)$ для
портретов динамических (трендовых) компонент во\-ла\-тиль\-ности или
$\sigma_j=\sigma_j(t)$ для портретов диффузионных компонент
волатильности. Веса компонент смеси, соответствующих конкретным
значениям па\-ра\-мет\-ров $a_j$ и~$\sigma_j$, показаны оттенками серого
цвета. Чем линия темнее, тем вес больше.

\begin{figure*} %fig1
\vspace*{1pt}
\begin{center}
\mbox{%
\epsfxsize=100.655mm
\epsfbox{gor-1.eps}
}
\end{center}
\vspace*{-9pt}
%\includegraphics[width=5.5in,height=2.5in]{AMEX_em_diffus(dyn=0)_bw.jpg}
%\includegraphics[width=5.5in,height=2.5in]{AMEX_em_diffus_bw.jpg}
%\includegraphics[width=5.5in,height=2.5in]{AMEX_em_dynamic_bw.jpg}
\Caption{Портреты волатильности индекса AMEX, полученные
СРС-методом с использованием ЕМ-алгоритма: предполагаемая
диффузионная волатильность~(\textit{а}); диффузионная компонента
волатильности~(\textit{б}); динамическая компонента волатильности~(\textit{в})
\label{f1gr}}
\end{figure*}

\begin{figure*} %fig2
\vspace*{1pt}
\begin{center}
\mbox{%
\epsfxsize=118.368mm
\epsfbox{gor-2.eps}
}
\end{center}
\vspace*{-9pt}
%\includegraphics[width=5.5in,height=2.5in]{AMEX_medem_diffus(dyn=0)_bw.jpg}
%\includegraphics[width=5.5in,height=2.5in]{AMEX_medem_diffus_bw.jpg}
%\includegraphics[width=5.5in,height=2.5in]{AMEX_medem_dynamic_bw.jpg}
\Caption{Портреты волатильности индекса AMEX, полученные
СРС-методом с использованием медианной модификации ЕМ-алгоритма:
предполагаемая диффузионная волатильность~(\textit{а}); диффузионная
компонента волатильности~(\textit{б}); динамическая компонента
волатильности~(\textit{в})
\label{f2gr}}
\end{figure*}

\begin{figure*} %fig3
\vspace*{1pt}
\begin{center}
\mbox{%
\epsfxsize=102.14mm
\epsfbox{gor-3.eps}
}
\end{center}
\vspace*{-9pt}
%\includegraphics[width=5.5in,height=2.5in]{CAC40_em_diffus(dyn=0)_bw.jpg}
%\includegraphics[width=5.5in,height=2.5in]{CAC40_em_diffus_bw.jpg}
%\includegraphics[width=5.5in,height=2.5in]{CAC40_em_dynamic_bw.jpg}
\Caption{Портреты волатильности индекса CAC 40, полученные
СРС-методом с использованием ЕМ-алгоритма: предполагаемая
диффузионная волатильность~(\textit{а}); диффузионная компонента
волатильности~(\textit{б}); динамическая компонента волатильности~(\textit{в})
\label{f3gr}}
\end{figure*}


\begin{figure*} %fig4
\vspace*{1pt}
\begin{center}
\mbox{%
\epsfxsize=117.509mm
\epsfbox{gor-4.eps}
}
\end{center}
\vspace*{-9pt}
%\includegraphics[width=5.5in,height=2.5in]{CAC40_medem_diffus(dyn=0)_bw.jpg}
%\includegraphics[width=5.5in,height=2.5in]{CAC40_medem_diffus_bw.jpg}
%\includegraphics[width=5.5in,height=2.5in]{CAC40_medem_dynamic_bw.jpg}
\Caption{Портреты волатильности индекса CAC~40, полученные
СРС-методом с использованием медианной модификации ЕМ-алгоритма:
предполагаемая диффузионная волатильность~(\textit{а}); диффузионная
компонента волатильности~(\textit{б}); динамическая компонента
волатильности~(\textit{в})
\label{f4gr}}
\end{figure*}

\begin{figure*} %fig5
\vspace*{1pt}
\begin{center}
\mbox{%
\epsfxsize=101.986mm
\epsfbox{gor-5.eps}
}
\end{center}
\vspace*{-9pt}
%\includegraphics[width=5.5in,height=2.5in]{Nasdaq100_em_diffus(dyn=0)_bw.jpg}
%\includegraphics[width=5.5in,height=2.5in]{Nasdaq100_em_diffus_bw.jpg}
%\includegraphics[width=5.5in,height=2.5in]{Nasdaq100_em_dynamic_bw.jpg}
\Caption{Портреты волатильности индекса Nasdaq~100, полученные
СРС-методом с использованием ЕМ-алгоритма: предполагаемая
диффузионная волатильность~(\textit{а}); диффузионная компонента
волатильности~(\textit{б}); динамическая компонента волатильности~(\textit{в})
\label{f5gr}}
\end{figure*}


\begin{figure*} %fig6
\vspace*{1pt}
\begin{center}
\mbox{%
\epsfxsize=116.941mm
\epsfbox{gor-6.eps}
}
\end{center}
\vspace*{-9pt}
%\includegraphics[width=5.5in,height=2.5in]{Nasdaq100_medem_diffus(dyn=0)_bw.jpg}
%\includegraphics[width=5.5in,height=2.5in]{Nasdaq100_medem_diffus_bw.jpg}
%\includegraphics[width=5.5in,height=2.5in]{Nasdaq100_medem_dynamic_bw.jpg}
\Caption{Портреты волатильности индекса Nasdaq 100, полученные
СРС-методом с использованием медианной модификации ЕМ-алгоритма:
предполагаемая диффузионная волатильность~(\textit{а}); диффузионная
компонента волатильности~(\textit{б}); динамическая компонента
волатильности~(\textit{в})
\label{f6gr}}
\end{figure*}

\begin{figure*} %fig7
\vspace*{1pt}
\begin{center}
\mbox{%
\epsfxsize=101.246mm
\epsfbox{gor-7.eps}
}
\end{center}
\vspace*{-9pt}
%\includegraphics[width=5.5in,height=2.5in]{Nikkei_em_diffus(dyn=0)_bw.jpg}
%\includegraphics[width=5.5in,height=2.5in]{Nikkei_em_diffus_bw.jpg}
%\includegraphics[width=5.5in,height=2.5in]{Nikkei_em_dynamic_bw.jpg}
\Caption{Портреты волатильности индекса Nikkei, полученные
СРС-методом с использованием ЕМ-алгоритма: предполагаемая
диффузионная волатильность~(\textit{а}); диффузионная компонента
волатильности~(\textit{б}); динамическая компонента волатильности~(\textit{в})
\label{f7gr}}
\end{figure*}

\begin{figure*} %fig8
\vspace*{1pt}
\begin{center}
\mbox{%
\epsfxsize=116.842mm
\epsfbox{gor-8.eps}
}
\end{center}
\vspace*{-9pt}
%\includegraphics[width=5.5in,height=2.5in]{Nikkei_medem_diffus(dyn=0)_bw.jpg}
%\includegraphics[width=5.5in,height=2.5in]{Nikkei_medem_diffus_bw.jpg}
%\includegraphics[width=5.5in,height=2.5in]{Nikkei_medem_dynamic_bw.jpg}
\Caption{Портреты волатильности индекса Nikkei, полученные
СРС-методом с использованием медианной модификации ЕМ-алгоритма:
предполагаемая диффузионная волатильность~(\textit{а}); диффузионная
компонента волатильности~(\textit{б}); динамическая компонента
волатильности~(\textit{в})
\label{f8gr}}
\end{figure*}

Для сравнения на рисунках также представлены результаты решения
аналогичной задачи с по\-мощью ЕМ-алгоритма. Этот алгоритм
реализован программой ZHPlot, разработанной Ю.\,В.~Жуковым.

На рис.~\ref{f1gr} и~\ref{f2gr} представлены результаты анализа волатильности
индекса AMEX с помощью СРС-ме\-то\-да. По рис.~\ref{f1gr} видно, что
игнорирование фактически ненулевой динамической (трендовой)
составляющей волатильности приводит к тому, что портрет
предполагаемой диффузионной волатильности (\textit{а}) оказывается
довольно <<мохнатым>>, на нем трудно выделить явные компоненты.
Однако как только наличие нетривиальных динамических компонент
признается и они автоматически оцениваются, картина становится
намного более четкой. При этом как в диффузионной, так и в
динамической составляющих волатильности выделяются по две
компоненты, причем в каждой паре одна имеет явно выраженный
периодический характер. Этот эффект значительно более наглядно
проявляется на рис.~\ref{f2gr}. Если в портрете предполагаемой диффузионной
волатильности (\textit{а}) присутствуют элементы <<хаотических
шумов>>, хотя и меньших по сравнению с портретом, получаемым
обычным ЕМ-ал\-го\-рит\-мом (рис.~\ref{f1gr},\,\textit{а}), то рис.~\ref{f1gr},\,\textit{б} и~\textit{в}
объясняют наличие этого <<шума>>: он оказывается обусловленным
наличием явно выраженных трендовых со\-став\-ля\-ющих. При этом обе пары
диффузионных и динамических компонент выделяются медианными
модификациями ЕМ-алгоритма намного более четко, нежели обычным
ЕМ-алгоритмом.

Точно такой же эффект наблюдается на рис.~\ref{f3gr} и~\ref{f4gr}, где приведены
результаты анализа волатильности индекса CAC~40 с помощью
СРС-метода с использованием обычного ЕМ-алгоритма (см.\ рис.~\ref{f3gr}) и
медианной модификации ЕМ-алгоритма (см.\ рис.~\ref{f4gr}). При этом на рис.~\ref{f3gr} с
трудом угадываются две компоненты предполагаемой диффузионной
волатильности, которые довольно хорошо выделяются (с небольшими
зашумлениями) медианной версией ЕМ-алгоритма на рис.~\ref{f4gr}.

Рисунки~\ref{f5gr} и~\ref{f6gr} великолепно иллюстрируют как разницу между
предполагаемой диффузионной\linebreak
 волатильностью и самой диффузионной
волатильностью, так и неоспоримые преимущества медианных версий
ЕМ-алгоритма перед обычным ЕМ-ал\-го\-рит\-мом при их использовании для
разделения смесей нормальных законов. На рис.~\ref{f6gr} очень хорошо
видно, что <<мнимые>> компоненты предполагаемой диффузионной
волатильности, <<за\-шум\-ля\-ющие>> верхний рисунок, полностью
обусловлены наличием динамической компоненты вола\-тиль\-ности: на
рис.~\ref{f6gr},\,\textit{б} в отличие от рис.~\ref{f6gr},\,\textit{а}, присутствует лишь одна
компонента.

На рис.~\ref{f7gr} и~\ref{f8gr} наблюдается картина, аналогичная приведенным на рис.~\ref{f1gr}--\ref{f4gr}.

\section{Применение медианных модификаций SEM-алгоритма для разделения
конечных смесей нормальных законов к~декомпозиции волатильности
конкретных финансовых~индексов}

В качестве исходных данных использовались минутные логарифмические
приращения биржевых индексов AMEX, CAC~40, Nasdaq~100, Nikkei,
S\&P500 и РТС, а также контракты на золото и специально
смоделированная выборка с известными параметрами. При этом ширина
скользящего окна~$n$ взята равной $200$ отсчетам. Задача
оценивания параметров смесей последовательно решалась для каждого
положения скользящего окна по выборке (отрезку исходного ряда),
соответствующей данному положению окна.

Эмпирически было установлено, что необходимая <<приработка>>
достигается при точности приближения $10^{-5}$--$10^{-6}$. Стоит
отметить, что хорошая скорость сходимости достигается и при
точности $10^{-12}.$ Однако в силу некоторых особенностей
алгоритма расчеты производились именно для точности~$\varepsilon$,
равной $10^{-5}$--$10^{-6}$. В качестве критерия останова
использовалось соотношение
$$
\max\left\vert \theta^{(m)}-\theta^{(m-1)}\right\vert
<\varepsilon\,,
$$
где $\theta^{(m)}$~--- вектор всех оцениваемых
параметров на $m$-м итерационном шаге, а $\varepsilon$~--- указанная выше точность.
Число возможных компонент смеси полагалось равным шести.

Результаты представлены на рис.~\ref{f9gr}--\ref{f20gr}. Как и на предыдущих
рисунках, на каждом графике любая точка на горизонтальной оси
соответствует конкретному значению правого конца скользящего
интервала времени, по которому (т.\,е.\ по наблюдениям, попавшим в
который) вычисляются оценки параметров. Вертикальная ось~--- это
ось значений параметров $a_j=a_j(t)$ для портретов динамических
(трендовых) компонент волатильности или $\sigma_j=\sigma_j(t)$ для
портретов диффузионных компонент волатильности. Веса компонент
смеси, соответствующих конкретным значениям па\-ра\-мет\-ров $a_j$
и~$\sigma_j$ показаны оттенками серого цвета. Чем линия темнее, тем
вес больше.

%fig9
\begin{figure*}
\vspace*{1pt}
\begin{center}
\mbox{%
\epsfxsize=117.439mm
\epsfbox{gor-9.eps}
}
\end{center}
\vspace*{-9pt}
%\includegraphics[width=5in,height=2.5in]{AMEX_em_diffus_bw.jpg}
%\includegraphics[width=5.3in,height=2.5in]{AMEX_sem_diffus.jpg}
%\includegraphics[width=5.3in,height=2.5in]{AMEX_medsem_diffus.jpg}
\Caption{Портреты диффузионной волатильности индекса AMEX,
полученные: ЕМ-алгоритмом~(\textit{а}); SЕМ-ал\-го\-рит\-мом~(\textit{б});
медианной модификацией SЕМ-алгоритма~(\textit{в})
\label{f9gr}}
\end{figure*}


%fig10
\begin{figure*}
\vspace*{1pt}
\begin{center}
\mbox{%
\epsfxsize=116.934mm
\epsfbox{gor-10.eps}
}
\end{center}
\vspace*{-9pt}
%\includegraphics[width=5in,height=2.5in]{CAC40_em_diffus_bw.jpg}
%\includegraphics[width=5.3in,height=2.5in]{CAC40_sem_diffus.jpg}
%\includegraphics[width=5.3in,height=2.5in]{CAC40_medsem_diffus.jpg}
\Caption{Портреты диффузионной волатильности индекса CAC~40,
полученные: ЕМ-алгоритмом~(\textit{а}); SЕМ-ал\-го\-рит\-мом~(\textit{б});
медианной модификацией SЕМ-алгоритма~(\textit{в})
\label{f10gr}}
\end{figure*}

%fig11
\begin{figure*}
\vspace*{1pt}
\begin{center}
\mbox{%
\epsfxsize=117.071mm
\epsfbox{gor-11.eps}
}
\end{center}
\vspace*{-9pt}
%\includegraphics[width=5in,height=2.5in]{Nasdaq100_em_diffus_bw.jpg}
%\includegraphics[width=5.3in,height=2.5in]{Nasdaq100_sem_diffus.jpg}
%\includegraphics[width=5.3in,height=2.5in]{Nasdaq100_medsem_diffus.jpg}
\Caption{Портреты диффузионной волатильности индекса Nasdaq~100,
полученные: ЕМ-алгоритмом~(\textit{а}); SЕМ-ал\-го\-рит\-мом~(\textit{б});
медианной модификацией SЕМ-алгоритма~(\textit{в})
\label{f11gr}}
\end{figure*}

%fig12
\begin{figure*}
\vspace*{1pt}
\begin{center}
\mbox{%
\epsfxsize=117.103mm
\epsfbox{gor-12.eps}
}
\end{center}
\vspace*{-9pt}
%\includegraphics[width=5in,height=2.5in]{Nikkei_em_diffus_bw.jpg}
%\includegraphics[width=5.3in,height=2.5in]{Nikkei_sem_diffus.jpg}
%\includegraphics[width=5.3in,height=2.5in]{Nikkei_medsem_diffus.jpg}
\Caption{Портреты диффузионной волатильности индекса Nikkei,
полученные: ЕМ-алгоритмом~(\textit{а}); SЕМ-ал\-го\-рит\-мом~(\textit{б});
медианной модификацией SЕМ-алгоритма~(\textit{в})
\label{f12gr}}
\end{figure*}
Для сравнения на рисунках также представлены результаты решения
аналогичной задачи с по\-мощью ЕМ-алгоритма (при точности
приближения, равной $10^{-8}$).

%fig13
\begin{figure*}
\vspace*{1pt}
\begin{center}
\mbox{%
\epsfxsize=117.417mm
\epsfbox{gor-13.eps}
}
\end{center}
\vspace*{-9pt}
%\includegraphics[width=5in,height=2.5in]{Nasdaq_em_diffus.jpg}
%\includegraphics[width=5.3in,height=2.5in]{Nasdaq_sem_diffus.jpg}
%\includegraphics[width=5.3in,height=2.5in]{Nasdaq_medsem_diffus.jpg}
\Caption{Портреты диффузионной волатильности индекса NASDAQ,
полученные: ЕМ-алгоритмом~(\textit{а}); SЕМ-ал\-го\-рит\-мом~(\textit{б});
медианной модификацией SЕМ-алгоритма~(\textit{в})
\label{f13gr}}
\end{figure*}

%fig14
\begin{figure*}
\vspace*{1pt}
\begin{center}
\mbox{%
\epsfxsize=116.988mm
\epsfbox{gor-14.eps}
}
\end{center}
\vspace*{-9pt}
%\includegraphics[width=5in,height=2.5in]{SP500_em_diffus.jpg}
%\includegraphics[width=5.3in,height=2.5in]{SP500_sem_diffus.jpg}
%\includegraphics[width=5.3in,height=2.5in]{SP500_medsem_diffus.jpg}
\Caption{Портреты диффузионной волатильности индекса S\&P500,
полученные: ЕМ-алгоритмом~(\textit{а}); SЕМ-ал\-го\-рит\-мом~(\textit{б});
медианной модификацией SЕМ-алгоритма~(\textit{в})
\label{f14gr}}
\end{figure*}

%fig15
\begin{figure*}
\vspace*{1pt}
\begin{center}
\mbox{%
\epsfxsize=116.884mm
\epsfbox{gor-15.eps}
}
\end{center}
\vspace*{-9pt}
%\includegraphics[width=5in,height=2.5in]{PTC_em_diffus.jpg}
%\includegraphics[width=5.3in,height=2.5in]{PTC_sem_diffus.jpg}
%\includegraphics[width=5.3in,height=2.5in]{PTC_medsem_diffus.jpg}
\Caption{Портреты диффузионной волатильности индекса PTC,
полученные: ЕМ-алгоритмом~(\textit{а}); SЕМ-ал\-го\-рит\-мом~(\textit{б});
медианной модификацией SЕМ-алгоритма~(\textit{в})
\label{f15gr}}
\end{figure*}

%fig16
\begin{figure*}
\vspace*{1pt}
\begin{center}
\mbox{%
\epsfxsize=118.405mm
\epsfbox{gor-16.eps}
}
\end{center}
\vspace*{-9pt}
%\includegraphics[width=5in,height=2.5in]{PTC_em_dynamic.jpg}
%\includegraphics[width=5.3in,height=2.5in]{PTC_sem_dynamic.jpg}
%\includegraphics[width=5.3in,height=2.5in]{PTC_medsem_dynamic.jpg}
\Caption{Портреты динамической волатильности индекса PTC,
полученные: ЕМ-алгоритмом~(\textit{а}); SЕМ-ал\-го\-рит\-мом~(\textit{б});
медианной модификацией SЕМ-алгоритма~(\textit{в})
\label{f16gr}}
\end{figure*}


На рис.~\ref{f9gr}--\ref{f12gr} представлены результаты анализа диффузионной
волатильности индексов, рас\-смот\-рен\-ных в разд.~10. Так как
используется меньшая точность итерационных приближений, результаты
получаются более зашумленными, однако характерное поведение компонент (а также их число) сохраняется.

Рассмотрим ряд других индексов. На рис.~\ref{f13gr} представлен анализ
диффузионной компоненты волатильности для индекса NASDAQ за три торговых дня.

На рис.~\ref{f13gr},\,\textit{а} изображены результаты применения EM-алгоритма
(окно~$200$, точность $10^{-8}$). Результат достаточно сильно
<<зашумлен>>, однако можно говорить о наличии одной компоненты
смеси. На рис.~\ref{f13gr},\,\textit{б} приводятся результаты применения
SEM-алгоритма (окно~$200$, точность $10^{-6}$). На данном графике
четко просматривается наличие одной компоненты с большим весом
(весьма близким к~1), а также наличие низковолатильной компоненты
с весом, близким к~0. Отметим наличие изломов на графике
компоненты. При этом график рис.~\ref{f13gr},\,\textit{в} результатов применения
медианной модификации SEM-алгоритма (окно~$200$, точность
$10^{-6}$) получается практически гладким. Приведенный ряд
интересен тем, что удалось обнаружить наличие единственной
компоненты, которая, пусть и на протяжении не очень длительного
периода времени, оказывала определяющее влияние на поведение
индекса NASDAQ.

Похожая картина (с соответствующими выводами по каждому методу)
наблюдается и на рис.~\ref{f14gr} анализа диффузионной компоненты
во\-ла\-тиль\-ности для индекса S\&P500: значительная <<зашумленность>>
результатов EM-алгоритма (окно~$200$, точность $10^{-8}$), большая
четкость результатов SEM-ал\-го\-рит\-ма (окно~$200$, точность
$10^{-6}$) и, наконец, большая гладкость и отсутствие ложного
дробления компонент на графике результатов применения медианной
модификации SEM-алгоритма (окно~$200$, точность $10^{-6}$).

На рис.~\ref{f15gr} приведен анализ диффузионной компоненты волатильности для
индекса РТС. Отметим значительно более высокую наглядность графика
для медианной модификации SEM-алгоритма (окно~$200$, точность
$10^{-5}$), на котором все компоненты видны максимально четко.

Как отмечалось выше, SEM-алгоритм и его медианные модификации
предназначены для оценки всех параметров смеси без каких-либо
предварительных предположений об их значениях. Ранее
рассматривались только оценки параметра масштаба. Теперь
рассмотрим и динамику параметров сдвига~--- проанализируем
динамические (трендовые) компоненты волатильности.

На рис.~\ref{f16gr} приведен результат анализа динамической компоненты
волатильности для индекса РТС. Все три сравниваемых метода
выделяют похожую картину волатильности, на графике SEM-ал\-го\-рит\-ма
(окно~$200$, точность $10^{-5}$) она видна более четко, а на
графике медианной модификации SEM-алгоритма (окно~$200$, точность
$10^{-5}$) еще и практически избавлена от шумов. Для данного ряда
результаты оценивания динамической компоненты волатильности
повторяют результаты, полученные для диффузионной компоненты:
лучшая визуализация достигается применением медианной модификации
SEM-алгоритма. Однако этот вывод справедлив для динамической
компоненты далеко не всегда.

%fig17
\begin{figure*}
\vspace*{1pt}
\begin{center}
\mbox{%
\epsfxsize=117.633mm
\epsfbox{gor-17.eps}
}
\end{center}
\vspace*{-9pt}
%\includegraphics[width=5in,height=2.5in]{Comex_em_diffus.jpg}
%\includegraphics[width=5.3in,height=2.5in]{Comex_sem_diffus.jpg}
%\includegraphics[width=5.3in,height=2.5in]{Comex_medsem_diffus.jpg}
\Caption{Портреты динамической волатильности для золота,
полученные: ЕМ-алгоритмом~(\textit{а}); SЕМ-ал\-го\-рит\-мом~(\textit{б});
медианной модификацией SЕМ-алгоритма~(\textit{в})
\label{f17gr}}
\end{figure*}

На рис.~\ref{f17gr} приведен анализ диффузионной компоненты волатильности
для золота. Результаты аналогичны тем, что получены для индексов.
Обратим внимание на заметную периодичность компоненты с наибольшим
весом. Период составляет около~250~ша\-гов, что согласуется с
дневной активностью на рынках (данные минутные, торговый день
составляет 480~минут~--- 240~шагов в пятиминутных данных).
Таким образом, SEM-алгоритм и его медианная модификация выявили
периодичность в активности торгов (что хорошо согласуется с
ситуацией на рынке), при этом видны 7~периодов, данные
представлены за 8~торговых дней (но не полных). EM-алгоритм
данную периодичность выявить не смог.

В разд.~\ref{efficency} приводились обоснования эффективности
использования медианных оценок. Однако было отмечено, что
необходимые для этого условия выполняются не всегда. В качестве
примера рассмотрим рис.~\ref{f18gr}, на котором приведены результаты
анализа динамической компоненты волатильности для золота.

%fig18
\begin{figure*}
\vspace*{1pt}
\begin{center}
\mbox{%
\epsfxsize=115.404mm
\epsfbox{gor-18.eps}
}
\end{center}
\vspace*{-9pt}
%\includegraphics[width=5in,height=2.5in]{Comex_em_dynamic.jpg}
%\includegraphics[width=5.3in,height=2.5in]{Comex_sem_dynamic.jpg}
%\includegraphics[width=5.3in,height=2.5in]{Comex_medsem_dynamic.jpg}
\Caption{Портреты динамической волатильности для золота,
полученные: ЕМ-алгоритмом~(\textit{а}); SЕМ-ал\-го\-рит\-мом~(\textit{б});
медианной модификацией SЕМ-алгоритма~(\textit{в})
\label{f18gr}}
\end{figure*}

На рис.~\ref{f18gr},\,\textit{а}, где приведены результаты применения
EM-алгоритма (окно~$200$, точность $10^{-8}$), компоненты не
выделяются. На рис.~\ref{f18gr},\,\textit{б}, где приведены результаты
применения SEM-ал\-го\-рит\-ма, четко различима трендовая компонента. На
рис.~\ref{f18gr},\,\textit{в}, где приведены результаты применения медианной
модификации SEM-алгоритма, тренда нет совсем~--- компоненты
преимущественно оцениваются нулем. Обратим внимание на вид
результатов медианной модификации SEM-алгоритма: график
представляет собой набор <<полос>>. Такой вид объясняется тем, что
в качестве оценки параметра сдвига используется выборочная
медиана. Благодаря тому что она по определению пред\-став\-ля\-ет собой
либо элемент выборки, либо полусумму двух элементов выборки, на
каждом следующем шаге выборка изменяется ровно на один элемент, и
поскольку медиана является робастной оценкой, получаются
горизонтальные участки оценок параметров, которые на графике
представляют собой <<полосы>>. Расстояние между ними объясняется
различием в точности, с которой получены оценки (фактически
пред\-став\-ля\-ющих собой элементы выборки) и с которой заданы исходные
данные. Так, в данном примере выборка задана с точностью порядка
$10^{-3}$, в то время как точность алгоритма составляет $10^{-6}$.

Таким образом, получается, что медианная модификация SEM-алгоритма
эффективна для оценки диффузионных компонент (результаты почти\linebreak
всегда значительно лучше аналогичных для обычного SEM-алгоритма).
Однако при оценке динамической компоненты встречаются ситуации,
когда разделение компонент SEM-алгоритмом\linebreak
проводится правильно, но
результаты, получаемые медианной модификацией SEM-алгоритма, с
трудом поддаются анализу.

%fig.19
\begin{figure*}
\vspace*{1pt}
\begin{center}
\mbox{%
\epsfxsize=117.019mm
\epsfbox{gor-19.eps}
}
\end{center}
\vspace*{-9pt}
%\includegraphics[width=5in,height=2.5in]{Test_em_diffus.jpg}
%\includegraphics[width=5.3in,height=2.5in]{Test_sem_diffus.jpg}
%\includegraphics[width=5.3in,height=2.5in]{Test_medsem_diffus.jpg}
\Caption{Портреты диффузионной волатильности тестовой выборки,
полученные: ЕМ-алгоритмом~(\textit{а}); SЕМ-алгоритмом~(\textit{б});
медианной модификацией SЕМ-алгоритма~(\textit{в})
\label{f19gr}}
\end{figure*}


Наконец, рассмотрим тестовую выборку. Моделируем выборку из смеси
нормальных распределений с весами 1/3, 1/3 и
1/3 с параметрами $(0,\left( 0{,}001\right)^{2})$,
$(0,\left( 0{,}005\right)^{2})$ и $(0,\left( 0{,}004\right)^{2})$,
всего $1000$ элементов. Сравним результаты алгоритмов, примененных
к этому ряду. Сначала проанализируем диффузионную компоненту
волатильности.  На рис.~\ref{f19gr},\,\textit{а} изображены результаты
применения EM-ал\-го\-рит\-ма (окно~$200$, точность $10^{-6}$).
Определить число компонент достаточно сложно~--- возможны две или
более компонент. При этом ни одна из дисперсий исходных компонент
не оценена правильно. На рис.~\ref{f19gr},\,\textit{б} приводятся
результаты применения SEM-ал\-го\-рит\-ма (окно~$200$, точность
$10^{-6}$). Компонента, соответствующая параметру
$\left(0{,}005\right)^{2},$ плавно переходит в компоненту,
соответствующую $\left(0{,}004\right)^{2}$. Данный алгоритм
позволяет визуально обнаружить только две компоненты. А вот на
графике рис.~\ref{f19gr},\,\textit{в} результатов применения медианной
модификации SEM-ал\-го\-рит\-ма (окно~$200$, точность~$10^{-6}$)
компоненты уже четко разделены~--- видны правильные оценки трех(!)
компонент. Итак, EM-ал\-го\-ритм определил число компонент
неправильно, также неверно оценены параметры компонент: как
дисперсия, так и математическое ожидание. SEM-алгоритм обнаружил
только две компоненты, хотя оценки параметров, по сути, верны для
всех трех компонент. Наконец, медианная модификация SEM-алгоритма
оценивает параметры правильно и обнаруживает все компоненты.
Несмотря на то что оценки параметров при этом весьма близки,
компоненты не сливаются в одну.

%fig.20
\begin{figure*}
\vspace*{1pt}
\begin{center}
\mbox{%
\epsfxsize=115.858mm
\epsfbox{gor-20.eps}
}
\end{center}
\vspace*{-9pt}
%\includegraphics[width=5in,height=2.5in]{Test_em_dynamic.jpg}
%\includegraphics[width=5.3in,height=2.5in]{Test_sem_dynamic.jpg}
%\includegraphics[width=5.3in,height=2.5in]{Test_medsem_dynamic.jpg}
\Caption{Портреты динамической волатильности тестовой выборки,
полученные: ЕМ-алгоритмом~(\textit{а}); SЕМ-алгоритмом~(\textit{б});
медианной модификацией SЕМ-алгоритма~(\textit{в})
\label{f20gr}}
\end{figure*}


На рис.~\ref{f20gr} приведен анализ динамической компоненты волатильности для
тестовой выборки. Видно, что результаты EM-алгоритма сложны для
интерпретации. В то же время результаты SEM-ал\-го\-рит\-ма и его
медианной модификации похожи, что говорит о выполнении условий
эффективности медианных оценок.

\section{Заключение}

\begin{enumerate}[1.]
\item Медианные модификации ЕМ-алгоритма продемонстрировали
намного большую пригодность к использованию для решения задачи
чис\-лен\-но\-го (статистического) разделения смесей нормальных законов
по сравнению с обычным ЕМ-алгоритмом. Портреты волатильности
финансовых индексов, получаемые с помощью медианных модификаций
ЕМ-алгоритма, отличаются большей четкостью, гладкостью и,
следовательно, более наглядны и удобны для интерпретации.
\item Статистический анализ данных о поведении финансовых индексов
свидетельствует в пользу наличия нетривиальных динамических
компонент, возможность существования которых вытекает из базовой
модели в рамках метода скользящего разделения смесей.
\item В пользу адекватности базовой модели типа конечной смеси
нормальных законов свидетельствует и то, что при максимально
возможных шести компонентах смеси практически значимыми
оказывались лишь 1--3~компоненты.
\item Результаты, получаемые медианными мо\-ди\-фи\-кациями SEM-алгоритма, не требуют
никаких дополнительных предположений о начальных приближениях (что
существенно используется в данной реализации медианного
EM-ал\-го\-рит\-ма). При этом медианные модификации SEM-ал\-го\-рит\-ма
позволяют получать весьма хорошо интерпретируемые результаты с
меньшей точностью приближений, что позволяет затрачивать меньшее
время на вычисления. Медианный EM-алгоритм без учета
дополнительных предположений зачастую дает худшие результаты (даже
на большей точности), сравнимые с результатами применения
стандартного EM-ал\-го\-рит\-ма.
\end{enumerate}
\vspace*{-12pt}

\pagebreak

{\small\frenchspacing
{%\baselineskip=10.8pt
\addcontentsline{toc}{section}{Литература}
\begin{thebibliography}{99}

\bibitem{Korolev2007b}  %1
\Au{Королёв В.\,Ю.}
ЕМ-алгоритм, его модификации и их
применение к задаче разделения смесей вероятностных распределений.
Теоретический обзор.~--- М.: ИПИРАН, 2007.

\bibitem{Tukey1960} %2
\Au{Tukey J.\,W.}
A survey of sampling from contaminated distributions~//  Contributions to probability and
statistics. Essays in Honor of Harold Hotelling~/ Eds. I.~Olkin, S.~G.~Ghurye, W.~Hoeffding,
W.\,G.~Madow, H.\,B.~Mann.--- Stanford:
Stanford University Press, 1960. P.~448--485.

\bibitem{Ayvazyan1983} %3
\Au{Айвазян С.\,А., Енюков~И.\,С., Мешалкин~Л.\,Д.}
Прикладная статистика. Основы
моделирования и первичная обработка данных.~--- М.: Финансы и
статистика, 1983.

\bibitem{Korolev2006}  %4
\Au{Королёв В.\,Ю.}
Теория вероятностей и математическая статистика.~--- М.: Проспект, 2006.

\bibitem{GKT2008}  %5
\Au{Горшенин А.\,К., Королёв В.\,Ю., Турсунбаев~А.\,М.}
Медианные модификации EM-алгоритма для разделения смесей вероятностных распределений и их применение к декомпозиции
волатильности финансовых индексов~// Статистические методы оценивания и проверки гипотез.~--- Пермь: Изд-во Пермского
университета, 2008 (в печати).

\bibitem{Vasilyev2002} %6
\Au{Васильев Ф.\,П.}
Методы оптимизации.~--- М.: Факториал Пресс, 2002.


\bibitem{Kolmogorov1931}  %7
\Au{Колмогоров А.\,Н.}
Метод медианы в теории ошибок~//
Матем. сборник, 1931. Т.~38, №\,3/4. С.~47--50.

\bibitem{Kolmogorov1986}  %8
\Au{Колмогоров А.\,Н.}
Теория вероятностей и математическая статистика. Сб.\ статей.~--- М.: Наука, 1986.

\bibitem{Celeux} %9
\Au{Diebolt~J., Celeux~G.}
Asymptotic properties of a stochastic EM algorithm for estimating mixing
proportions~// Communications in Statistics B: Stochastic Models,
1993. Vol.~9, No.\,4. P.~599--613.

\bibitem{Ip} %10
\Au{Diebolt J., Ip~E.\,H.\,S.}
Stochastic EM: Method and application~// Markov сhain Monte Carlo in practice~/
Eds. W.\,R.~Gilks, S.~Richardson,
D.\,J.~Spiegelhalter.~--- London: Chapman and Hall, 1996.

\bibitem{Korolev2007c}  %11
\Au{Королёв В.\,Ю.} Статистическая декомпозиция
волатильности~// Статистические методы оценивания и проверки гипотез.~--- Пермь: Изд-во Пермского университета, 2007.
С.~170--206.

\label{end\stat}

\bibitem{Korolev2007a}  %12
\Au{Королёв В.\,Ю.}
Вероятностно-статистический анализ хаотических процессов с помощью смешанных гауссовских моделей.
Декомпозиция волатильности финансовых индексов и турбулентной
плазмы.~--- М.: ИПИРАН, 2007.


\end{thebibliography}
}
}
\end{multicols}