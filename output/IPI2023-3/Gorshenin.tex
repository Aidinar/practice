\def\stat{gorshenin}

\def\tit{О КЛАСТЕРИЗАЦИИ ОБЪЕКТОВ СЕТЕВОЙ ВЫЧИСЛИТЕЛЬНОЙ ИНФРАСТРУКТУРЫ НА~ОСНОВЕ АНАЛИЗА СТАТИСТИЧЕСКИХ АНОМАЛИЙ В~ТРАФИКЕ$^*$}

\def\titkol{О кластеризации объектов сетевой вычислительной инфраструктуры на 
основе анализа статистических аномалий} %в~трафике}

\def\aut{А.\,К.~Горшенин$^1$, С.\,А.~Горбунов$^2$, Д.\,Ю.~Волканов$^3$}

\def\autkol{А.\,К.~Горшенин, С.\,А.~Горбунов, Д.\,Ю.~Волканов}

\titel{\tit}{\aut}{\autkol}{\titkol}

\index{Горшенин А.\,К.}
\index{Горбунов С.\,А.}
\index{Волканов Д.\,Ю.}
\index{Gorshenin A.\,K.}
\index{Gorbunov S.\,A.}
\index{Volkanov D.\,Yu.}


{\renewcommand{\thefootnote}{\fnsymbol{footnote}} \footnotetext[1]
{Работа выполнена при 
поддержке Программы развития МГУ, проект №\,23-Ш03-03. При обработке и~анализе 
трафика использовалась инфраструктура Центра коллективного пользования 
<<Высокопроизводительные вы\-чис\-ле\-ния и~большие данные>> (ЦКП <<Информатика>>) ФИЦ 
ИУ РАН (г.~Москва).}}


\renewcommand{\thefootnote}{\arabic{footnote}}
\footnotetext[1]{Федеральный исследовательский
центр <<Информатика и~управление>> Российской академии наук; Московский 
государственный университет имени М.\,В.~Ломоносова, \mbox{agorshenin@frccsc.ru}}
\footnotetext[2]{Московский государственный университет имени М.\,В.~Ломоносова; 
Московский центр фундаментальной и~при\-клад\-ной математики, 
\mbox{s.gorbunov.cmc@gmail.com}}
\footnotetext[3]{Московский государственный университет имени М.\,В.~Ломоносова, 
\mbox{volkanov@asvk.cs.msu.ru}}

%\vspace*{-12pt}

\Abst{Рассматривается задача выявления статистических аномалий (т.\,е.\ существенных превышений от 
типичных значений полученного и~исходящего трафика) на\-груз\-ки на узлы сетевой 
вы\-чис\-ли\-тель\-ной инфраструктуры. Рост на\-груз\-ки 
в~реальных сис\-те\-мах ведет к~не\-об\-хо\-ди\-мости регулярного мас\-шта\-би\-ро\-ва\-ния
 вы\-чис\-ли\-тель\-ных ресурсов и~хранилищ, 
а~так\-же перенаправления потоков данных. Предложена процедура выявления 
ста\-ти\-сти\-че\-ских аномалий в~сетевом трафике с~использованием ап\-прок\-си\-ма\-ции 
наблюдений обоб\-щен\-ным гам\-ма-рас\-пре\-де\-ле\-ни\-ем для дальнейшей клас\-те\-ри\-за\-ции объектов 
сетевой вы\-чис\-ли\-тель\-ной инфраструктуры с~\mbox{целью} оцен\-ки по\-треб\-ности в~ресурсах. Все 
вычислительные ста\-ти\-сти\-че\-ские процедуры, описанные в~\mbox{статье}, реализованы 
с~использованием языка программирования~R и~применены к~сетевому трафику, 
полученному в~рамках моделирования на специализированном ар\-хи\-тек\-тур\-но-про\-грам\-мном 
стенде. Предложенные под\-хо\-ды могут быть использованы и~для более 
широкого класса телекоммуникационных задач.}

\KW{сетевая инфраструктура; сетевой трафик; обобщенное гам\-ма-рас\-пре\-де\-ле\-ние; вы\-чис\-ли\-тель\-ная ста\-ти\-сти\-ка; 
проверка ста\-ти\-сти\-че\-ских гипотез; вы\-яв\-ле\-ние аномалий; клас\-те\-ри\-зация}

\DOI{10.14357/19922264230311}{XHTMVI}
  
%\vspace*{2pt}


\vskip 10pt plus 9pt minus 6pt

\thispagestyle{headings}

\begin{multicols}{2}

\label{st\stat}

\section{Введение}

В современной сетевой вы\-чис\-ли\-тель\-ной 
инфраструктуре по мере развития 
информационных ресурсов рас\-тет на\-груз\-ка на вы\-чис\-ли\-тель\-ные ресурсы 
инфраструктуры~\cite{Smeliansky2023,Smeliansky2019}. Этот рост периодически вызывает не\-об\-хо\-ди\-мость мас\-шта\-би\-ро\-ва\-ния 
вы\-чис\-ли\-тель\-ных ресурсов и~ресурсов хранилищ данных на узлах сетевой 
вы\-чис\-ли\-тель\-ной инфраструктуры или же пе\-ре\-на\-прав\-ле\-ния потоков данных. На 
се\-год\-няш\-ний день эта проб\-ле\-ма чаще всего решается в~<<руч\-ном>> режиме на основе 
опыта сетевых и~сис\-тем\-ных администраторов. Для 
автоматизации процессов 
управ\-ле\-ния мас\-шта\-би\-ро\-ва\-ни\-ем ресурсов на вы\-чис\-ли\-тель\-ных узлах сетевой 
вы\-чис\-ли\-тель\-ной инфраструктуры необходимы методы объективного вы\-яв\-ле\-ния аномалий 
в~на\-груз\-ке на такие узлы и~оцен\-ки размера аномалий для определения объемов 
необходимых ресурсов~\cite{Rossem2017,Malak2021,Mesodiakaki2022,Zhang2022}. Под аномалиями 
в~данной работе понимаются существенные превышения 
ти\-пич\-ных значений полученного и~исходящего трафика. Они могут быть вызва\-ны, 
например, работой несанкционированного программного обеспечения на узле или 
не\-хват\-кой ресурсов для обработки пользовательского трафика. Один из воз\-мож\-ных 
путей решения этой проб\-ле\-мы заключается в~мониторинге и~анализе сетевого трафика в~сетевой вы\-чис\-ли\-тель\-ной 
инфраструктуре и~выявления в~нем ста\-ти\-сти\-че\-ских 
аномалий, воз\-ни\-ка\-ющих в~том чис\-ле в~результате влияния набора случайных 
факторов~\cite{Hosseinzadeh2020,Abood2021}.

\begin{figure*}[b] %fig1
  \vspace*{1pt}
\begin{center}
   \mbox{%
\epsfxsize=151.725mm 
\epsfbox{gor-1.eps}
}
\end{center}
\vspace*{-9pt}
    \Caption{Временной ряд полученного трафика для одного объекта}
    \label{some_machine_example}
\end{figure*}

Хорошо известна воз\-мож\-ность ста\-ти\-сти\-че\-ско\-го описания процессов 
в~телекоммуникационных сетях с~использованием различных семейств 
гам\-ма-рас\-пре\-де\-ле\-ний: классических гамма~--- для распределений времени 
обслуживания~\cite{Parulekar1997}; гам\-ма--гам\-ма~--- для аппроксимации некоторых 
характеристик\linebreak в~сетях сотовой связи~\cite{Tabassum2014}, подводных 
коммуникационных сис\-те\-мах~\cite{Noor2022}, глобальных вы\-чис\-ли\-тель\-ных 
сетях~\cite{Padhan2023}; конечных гам\-ма-сме\-сей~--- для\linebreak описания тон\-кой структуры 
информационных потоков~\cite{Gorshenin2013a,Gorshenin2013b,Gorshenin2015,Gorshenin2016a,Gorshen
in2017a}; обобщенных гамма  (Generalized Gamma, GG)~--- для распределений времени 
пребывания пользователя в~ячейке сотовой сети~\cite{Zonoozi1996}.

\begin{figure*} %fig2
\vspace*{1pt}
\begin{center}
   \mbox{%
\epsfxsize=151.725mm 
\epsfbox{gor-2.eps}
}
\end{center}
\vspace*{-6pt}
    \Caption{Выборка полученного трафика за одну отметку времени для всех объектов}
    \label{some_timestamp_example}
\end{figure*}

В работе предложен ста\-ти\-сти\-че\-ский подход к~решению задачи клас\-те\-ри\-за\-ции объектов 
сетевой вычис\-ли\-тель\-ной инфраструктуры по уровню на\-гру\-жен\-ности с~точ\-ки зрения 
информационного обмена (передаваемого и~по\-лу\-ча\-емо\-го трафика) на основе процедуры 
вы\-яв\-ле\-ния аномальных наблюдений в~сетевом трафике с~использованием специального 
GG-тес\-та. Продемонстрированы примеры применения разработанной методики для 
данных сетевого трафика, полученных в~рамках моделирования на специализированном 
ар\-хи\-тек\-тур\-но-про\-грам\-мном стенде ка\-фед\-ры автоматизации сис\-тем вы\-чис\-ли\-тель\-ных 
комплексов факультета вы\-чис\-ли\-тель\-ной математики и~кибернетики МГУ имени 
М.\,В.~Ло\-мо\-но\-сова. 

\vspace*{-4pt}

\section{Статистическая модель трафика на~основе обобщенного гамма-распределения}

\vspace*{-2pt}

В качестве анализируемых данных в~\mbox{статье} используются наборы, полученные на 
специализированном стен\-де,
на котором моделируются раз\-лич\-ные сценарии реальных сетевых взаимодействий. 
Рас\-смат\-ри\-ва\-ют\-ся суточные данные (с~пятиминутной агрегацией). В~каж\-дый момент 
времени известно суммарное чис\-ло отправленных и~полученных бит (для удобства 
дальнейшего анализа данные нормированы значением~$2^{20}$~--- около 131~Кбайт). 
За указанный промежуток времени моделировалось взаимодействие для 1\,920  
объектов в~сети.
На рис.~\ref{some_machine_example} приведен пример данных входящего трафика для 
одного из объектов за все время наблюдений, а~на 
рис.~\ref{some_timestamp_example}~--- данные сразу для всех объектов, но в~некоторый фиксированный момент вре\-мени. 

Ранее авторами было установлено высокое ста\-ти\-сти\-че\-ское согласие данных 
мобильного трафика сотового оператора с~семейством обоб\-щен\-ных 
гам\-ма-рас\-пре\-де\-ле\-ний~\cite{Stacy1962}~--- это со\-сре\-до\-то\-чен\-ное на положительной 
полупрямой трех\-па\-ра\-мет\-ри\-че\-ское семейство вероятностных распределений, 
опре\-де\-ля\-емое плот\-ностью вида
$$f(x;r,\gamma,\mu)\hm=\fr{|\gamma|\mu^r}{\Gamma(r)}\,x^{\gamma r - 1} e^{-\mu x^{\gamma}}\,,$$ 
где $x\hm>0$, $r\hm>0$, $\mu\hm > 0$, $\gamma\hm\in\mathbb{R}\setminus\{0\}$.
Его выбор может объясняться и~тем фактом, что указанное семейство содержит 
практически все самые популярные абсолютно непрерывные рас\-пре\-де\-ле\-ния, 
со\-сре\-до\-то\-чен\-ные на положительной полупрямой, в~том чис\-ле рас\-пре\-де\-ле\-ния 
с~тяжелыми хвос\-та\-ми (некоторые примеры приведены в~таб\-ли\-це). %~\ref{tab:GG}).
Естественным образом возникает идея использования по\-доб\-ной ста\-ти\-сти\-че\-ской модели 
для иных информационных сис\-тем.

\begin{table*}[b]\small
%\vspace*{-6pt}
% \caption{Примеры непрерывных распределений}
 %Некоторые важные частные случаи GG-распределения}
%\vspace*{2ex}
\begin{center}
\begin{tabular}{|l|c|}
\multicolumn{2}{c}{Некоторые частные случаи обобщенного гамма-распределения}\\[2ex]
\hline
\multicolumn{1}{|c|}{\bf{Семейство}} & \bf{Значения параметров} \\
\hline
Гамма-распределение & $\gamma=1$\\ 
Обратное гамма-распределение& $\gamma=-1$\\
Распределение Леви & $\gamma=-1,\ r=0{,}5$\\
Показательное распределение & $\gamma=1,\ r=1$\\
Распределение Эрланга & $\gamma=1,\ r \in \mathbb{N}$\\
Распределение хи-квадрат & $\gamma=1,\ \mu=0{,}5$\\
Распределение Накагами & $\gamma=2$\\
Полунормальное распределение & $\gamma=2,\ r=0{,}5$\\
Распределение Рэлея & $\gamma=2,\ r=1$\\
Хи-распределение & $\gamma=2,\ \mu=1/\sqrt{2}$\\
Распределение Максвелла &$\gamma=2,\ r=1{,}5$\\
Распределение Фреше (распределение экстремальных значений II типа) &$r=1,\ \gamma<0$\\
Распределение Вейбулла--Гнеденко (распределение экстремальных значений III типа)& $r=1,\ \gamma>0$\\
\hline           
\end{tabular}
\end{center}
% \label{tab:GG}
\end{table*}

Для оценивания па\-ра\-мет\-ров GG-рас\-пре\-де\-ле\-ния мож\-но применять различные методы, 
начиная от классического метода максимального прав\-до\-по\-до\-бия. Однако в~этом 
случае для ана\-ли\-зи\-ру\-емых данных наблюдались существенные ошиб\-ки при 
ап\-прок\-си\-ма\-ции хвос\-тов распределения. Поэтому в~качестве вы\-чис\-ли\-тель\-ной процедуры 
был реализован алгоритм на основе минимизации $l^2$-но\-рмы меж\-ду значениями 
эмпирической и~тео\-ре\-ти\-че\-ской функций рас\-пре\-де\-ле\-ния, вы\-чис\-лен\-ны\-ми в~узлах 
некоторой сет\-ки $y \hm= (y_1,\ldots, y_m)$. Для оценки па\-ра\-мет\-ров по выборке 
$\mathbb{X} \hm= (X_1, \ldots, X_n)$ решается оптимизационная задача сле\-ду\-юще\-го 
\mbox{вида}:
{\looseness=-1

}


\noindent
\begin{multline*}
    \left(\hat{r}, \hat{\gamma}, \hat{\mu}\right)={}\\
    {}=\argmin\limits_{r>0, \mu > 0, 
\gamma\in\mathbb{R}\setminus\{0\}} \sum\limits_{i=1}^m \Bigg[\int\limits_0^{y_i} 
\fr{|\gamma|\mu^r}{\Gamma(r)}\,x^{\gamma r - 1} e^{-\mu x^{\gamma}}\,dx - {}\\
{}-
\fr{1}{n} \sum\limits_{i=1}^n \mathbb{I}({X_i < y_i})\Bigg]^2.
\end{multline*}
 \vspace*{-15pt}


Типичный пример функций, полученных в~результате применения различных методов, 
продемонстрирован на рис.~3. %\ref{mle_vs_cdf}. 
Оптимизационный под\-ход поз\-во\-лил не 
толь\-ко точ\-нее, но и~примерно в~30~раз быст\-рее
оценивать па\-ра\-мет\-ры для всех ана\-ли\-зи\-ру\-емых данных по срав\-не\-нию с~методом 
максимального прав\-до\-по\-до\-бия. 

{ \begin{center}  %fig3
 \vspace*{-6pt}
    \mbox{%
 \epsfxsize=79mm 
 \epsfbox{gor-3.eps}
 }

\end{center}
\vspace*{-3pt}

\noindent
{{\figurename~3}\ \ \small{Сравнение эмпирической функции распределения~(\textit{1}) 
и~функций, полученных двумя методами оценки па\-ра\-мет\-ров~--- максимального 
прав\-до\-по\-до\-бия~(\textit{2}) и~функциональной оптимизации~(\textit{3})}
}}

\vspace*{-6pt}

\addtocounter{figure}{1}

\section{Метод кластеризации объектов сетевой вычислительной инфраструктуры на~основе анализа аномалий трафика}

В реальных задачах ано\-маль\-ность трафика определяется не только его абсолютным 
значением, но и~соотношением с~другими наблюдениями. В~данном разделе рас\-смот\-рим 
процедуру, которая поз\-во\-ля\-ет ста\-ти\-сти\-че\-ски корректно учитывать это.

Сначала рас\-смот\-рим ста\-ти\-сти\-че\-скую процедуру выявления аномальных наблюдений для 
данных, опи\-сы\-ва\-емых обобщенным гам\-ма-рас\-пре\-де\-ле\-ни\-ем. Впервые такого рода 
статистический тест был пред\-ло\-жен в~статье~\cite{Gorshenin2020} для 
метеорологических рядов (осад\-ков). Учитывая особенности ана\-ли\-зи\-ру\-емых данных 
(па\-ра\-метр~$\gamma$ для трафика может иметь как положительные, так 
и~отрицательные значения), обоб\-щим описанную в~упомянутой \mbox{статье} про\-це\-дуру.

Пусть $V_1,\ldots,V_m$~--- независимая выборка из обоб\-щен\-но\-го 
гам\-ма-рас\-пре\-де\-ле\-ния с~некоторыми па\-ра\-мет\-ра\-ми $r\hm>0$, $\gamma \hm\neq 0$, $\mu \hm>0$, 
$V_1 \hm\geqslant V_j,$ $\forall j \hm\geqslant 2$. Рас\-смот\-рим статистику вида
$$\hat{\mathcal{R}} = \left(\fr{(m-1)V_1^\gamma}{V_2^\gamma+\cdots+V_m^\gamma}\right)^{\mathrm{sgn}\,(\gamma)}\,.$$
% \end{equation*}
Тогда при условии, что вер\-на гипотеза $H_0$: <<значение~$V_1$ не является 
аномально большим>>, 
$$\hat{\mathcal{R}} \sim 
\begin{cases}
F(r,\ (m-1)r) \enskip \mbox{для~~} \gamma \hm> 0;\\
F((m-1)r,\ r) \enskip \mbox{для~~} \gamma\hm < 0,
\end{cases}
$$
 \vspace*{-6pt}

\noindent
где через~$F$ обозначено распределение 
Сне\-де\-ко\-ра--Фи\-ше\-ра (\mbox{F-рас}\-пре\-де\-ле\-ние) с~со\-от\-вет\-ст\-ву\-ющи\-ми па\-ра\-мет\-рами.

Для того чтобы выявить аномальные наблюдения в~выборке из обоб\-щен\-но\-го 
гам\-ма-рас\-пре\-де\-ле\-ния $\mathbb{V} \hm= \{V_1, \ldots, V_m\}$ с~па\-ра\-мет\-ра\-ми $(r, \gamma, \mu)$, 
необходимо зафиксировать уровень зна\-чи\-мости~$\alpha$, для каждого 
$V_i$, $i \hm\in \{1, \ldots, m\}$, рас\-счи\-тать значение статистики~$\hat{\mathcal{R}}$ 
и~срав\-нить это значение с~$\alpha$-кван\-ти\-ля\-ми 
$F$-рас\-пре\-де\-ле\-ния с~со\-от\-вет\-ст\-ву\-ющи\-ми па\-ра\-мет\-ра\-ми. Наблюдение признается 
аномальным тогда и~только тогда, когда наблюд\-аемое значение ста\-ти\-сти\-ки больше 
кван\-ти\-ля $F$-рас\-пре\-де\-ле\-ния.

На практике для выборок большого объема под\-счет знаменателя ста\-ти\-сти\-ки~$\hat{\mathcal{R}}$ 
для каж\-до\-го наблюдения может за\-мет\-но увеличить время работы 
алгоритма, поэтому рациональнее будет один раз по\-счи\-тать $S \hm=  V_1^\gamma+\cdots+V_m^\gamma$, 
а~затем для $i$-го наблюдения вы\-чис\-лять значение 
статистики сле\-ду\-ющим образом:
\vspace*{-3pt}

\noindent 
$$\hat{\mathcal{R}} = 
\left(\fr{(m-1)V_i^\gamma}{S - V_i^\gamma}\right)^{\mathrm{sgn}\,(\gamma)}\,.$$

% \vspace*{-3pt}

На рис.~\ref{bytime_ex1} % и~\ref{bytime_ex2}
приведены примеры выявления аномальных наблюдений при уровне зна\-чи\-мости\linebreak $\alpha \hm= 0{,}05$ 
для фиксированного момента времени (выделены маркерами) с~по\-мощью 
описанного ме\-тода.

Для кластеризации мож\-но использовать такую раз\-мет\-ку наблюдений для выборок за 
каж\-дый момент времени. Причем формировать вы\-бор\-ки можно как на не\-пе\-ре\-се\-ка\-ющих\-ся 
по\-сле\-до\-ва\-тель\-ных интервалах времени, так и~воспользовавшись методом сколь\-зя\-ще\-го 
\mbox{окна}.

Сначала разметим аномальные наблюдения описанным выше GG-тес\-том, формируя 
выборки не\-пе\-ре\-се\-ка\-ющи\-ми\-ся часовыми окнами, и~нанесем по\-лу\-чив\-шу\-юся раз\-мет\-ку на 
временн$\acute{\mbox{ы}}$е ряды по кон\-крет\-ным объектам.
Если рас\-смат\-ри\-вать разметку GG-тес\-та на выборках, по которым он обучал\-ся 
(рис.~\ref{bytime_60m}), то мож\-но заметить, что существует некоторое пороговое 
значение, раз\-де\-ля\-ющее аномальные и~обыч\-ные зна\-че\-ния.



Однако при рас\-смот\-ре\-нии той же раз\-мет\-ки для конкретного объекта 
(рис.~\ref{bymachine_60m}) такого порога может и~не оказаться. Так происходит из-за 
того, что вы\-бор\-ки, на которых раз\-ме\-ча\-ют\-ся аномалии, формируются окнами по 
времени, поэтому раз\-мет\-ка содержит в~себе информацию о~сред\-нем тренде по всем 
объектам (рис.~7). %\ref{netin_percentiles}). 
На примере объекта 
с~риc.~\ref{bymachine_60m} мож\-но выделить две зоны с~большим чис\-лом аномалий: 
в~районе $22{:}00$, где трафик рос быст\-рее сред\-не\-го по всем объектам, и~в~районе 
$14{:}00$, где трафик конкретного объекта до\-стиг максимума, а~пер\-цен\-ти\-ли имели 
нисходящий тренд.
Имея раз\-мет\-ку по каж\-до\-му объекту, мож\-но провести бинарную клас\-те\-ри\-за\-цию, задавая 
некоторый порог доли данных, которые были размечены ано\-маль\-ными.

\end{multicols}

\begin{figure*} %fig4
\vspace*{-1pt}
\begin{center}
   \mbox{%
\epsfxsize=151.725mm 
\epsfbox{gor-4.eps}
}
\end{center}
\vspace*{-11pt}
   \Caption{Аномалии в~полученном~(\textit{а}) и~от\-прав\-лен\-ном~(\textit{б}) тра\-фике}
 \label{bytime_ex1}
\vspace*{-7pt} 
\end{figure*}

\begin{multicols}{2}


Для визуализации полученных результатов со\-ста\-вим при\-зна\-ко\-вое описание объектов: 
каж\-дый объект опишем медианой, квар\-ти\-ля\-ми и~ин\-тер\-квар\-тиль\-ным размахом по 
полученному и~от\-прав\-лен\-ному трафику по всем име\-ющим\-ся наблю\-де\-ниям. 
{\looseness=-1

}

Применим 
метод главных компонент, чтобы\linebreak пе\-ре\-вес\-ти при\-зна\-ко\-вое описание на двумерную 
плос\-кость. Уста\-нов\-ле\-но, что пер\-вые две глав\-ные компоненты суммарно описывают 
$95{,}8\%$ вариации, причем одна приписывает  больший вес перценти-\linebreak
\vspace*{-11pt}

\columnbreak

\noindent
лям (см.\ 
ось абсцисс на рис.~\ref{disjoint_window_60m}), а~другая~--ин\-тер\-квар\-тиль\-но\-му 
раз\-ма\-ху (см.\ ось ординат на рис.~\ref{disjoint_window_60m}). 
%Цветами (красный, синий, сиреневый) 
На рис.~\ref{disjoint_window_60m} отмечены объекты, при\-знан\-ные 
аномальными по полученному~(\textit{1}), отправленному~(\textit{2}) и~обоим видам трафика~(\textit{3}).

По оси абсцисс на рис.~\ref{disjoint_window_60m} отложены значения пер\-вой 
глав\-ной компоненты, которая имеет вы\-ра\-жен\-ную положительную корреляцию 
с~пер\-цен\-ти\-ля\-ми, по оси ординат~--- значения второй главной компоненты  
с~выраженной отрицательной корре\-ляцией с~раз\-ма\-ха\-ми. При фиксированном уров-\linebreak
\vspace*{-12pt}


\end{multicols}

\begin{figure*} %fig5
\vspace*{-1pt}
\begin{center}
   \mbox{%
\epsfxsize=151.725mm 
\epsfbox{gor-5.eps}
}
\end{center}
\vspace*{-12pt}
    \Caption{Аномалии на часовом горизонте наблюдения}
    \label{bytime_60m}
%\end{figure*}
%\begin{figure*} %fig6
\vspace*{10pt}
\begin{center}
   \mbox{%
\epsfxsize=148.725mm 
\epsfbox{gor-6.eps}
}
\end{center}
\vspace*{-12pt}
    \Caption{Аномалии на временном ряде конкретного объ\-екта}
    \label{bymachine_60m}
\vspace*{-6pt}
\end{figure*}

\begin{multicols}{2}

\noindent
не 
пер\-цен\-ти\-лей (пер\-вой компоненты) аномальны
объекты, у~которых меньше размах, и,~наоборот, 
при фиксированном уровне второй компоненты~--- у~которых больше 
значения пер\-цен\-ти\-лей.


Недостаток метода не\-пе\-ре\-се\-ка\-ющих\-ся окон заключается в~том, что каж\-дое наблюдение 
попадает только в~одно окно, поэтому наблюдение для текущего окна может быть 
при\-зна\-но аномальным, хотя при увеличении интервала времени оно таковым уже не 
является (в~част\-ности, см.\ об\-суж\-де\-ние под\-хо\-да в~статье~\cite{Gorshenin2020}) 
(см.\ рис.~\ref{bytime_60m}).


\addtocounter{figure}{1}

\begin{figure*}[b] %fig8
\vspace*{-3pt}
\begin{center}
   \mbox{%
\epsfxsize=148.67mm 
\epsfbox{gor-8.eps}
}
\end{center}
\vspace*{-12pt}
    \Caption{Кластеризация объектов на группы с~высоким входящим, исходящим и~общим трафиком:
    \textit{1}~--- аномален по полученному трафику;
\textit{2}~--- аномален по отправленному трафику;
\textit{3}~--- аномален по обоим типам трафика}
    \label{disjoint_window_60m}
\end{figure*}


Если же использовать скользящее окно некоторой ширины~$w$, то каж\-дое наблюдение 
(за исключением расположенных на краях) попадет в~$w$~окон.\linebreak
\vspace*{-12pt} 

{ \begin{center}  %fig7
 \vspace*{-6pt}
    \mbox{%
 \epsfxsize=79mm 
 \epsfbox{gor-7.eps}
 }

\end{center}
 \vspace*{-3pt}

\noindent
{{\figurename~7}\ \ \small{ Медиана~(\textit{1}) и~верхний квар\-тиль полученного 
трафика для всех объектов: \textit{2}~--- 75\% пер\-цен\-тиль}
\label{netin_percentiles}

}}

\vspace*{6pt}

%\addtocounter{figure}{2}



\noindent
Таким образом, 
наблюдение может быть признано аномальным от~0 до~$w$~раз и~мож\-но говорить 
о~некотором со\-по\-став\-ле\-нии уровней ано\-маль\-ности
конкретного наблюдения относительно 
других. Будем считать аномальное наблюдение:

\noindent
\begin{itemize}
    \item абсолютно аномальным (признано аномальным ров\-но $w$~раз);
    \item относительно аномальным (признано аномальным от $\lceil {w}/{2} 
\rceil$ до $w\hm - 1$~раз);
    \item промежуточно аномальным (признано аномальным от~1 до $\lceil 
{w}/{2} \rceil \hm- 1$~раз).
\end{itemize}


\noindent
На рис.~\ref{bymachine_moving_ex1} 
приведены примеры разметки некоторых объектов в~режиме сколь\-зя\-ще\-го окна шириной 
1~ч.
{\looseness=-1

}
%абсолютно аномальные наблюдения помечены крас\-ным цветом, относительно 
%аномальные~--- сиреневым и~промежуточно аномальные~--- синим.

\begin{figure*} %fig9
\vspace*{1pt}
\begin{center}
   \mbox{%
\epsfxsize=148.725mm 
\epsfbox{gor-9.eps}
}
\end{center}
\vspace*{-9pt}
 \Caption{Примеры разметки временных рядов трафика, скользящее окно: \textit{1}~--- абсолютно аномальные; 
\textit{2}~--- относительно аномальные; \textit{3}~--- промежуточно аномальные}
\label{bymachine_moving_ex1}
\end{figure*}

В этом случае наибольшие по абсолютной величине значения не обязательно 
признаются ано-\linebreak
\vspace*{-12pt}

\noindent
мальными, так как они могут соответствовать периодам об\-щей высокой 
на\-груз\-ки на сеть. Для каж\-до\-го наблюдения~$V_j$ по каж\-до\-му объекту имеется 
некоторый уровень ано\-маль\-ности $z_j \hm\in \{0,\ldots,w\}$, и~в~качестве правила 
клас\-те\-ри\-за\-ции мож\-но использовать некоторую функ\-цию от $\{z_1,\ldots,z_n\}$. 
Например, на рис.~\ref{moving_window_60m} пред\-став\-ле\-на клас\-те\-ри\-за\-ция на основе 
ре\-ша\-юще\-го правила $\mathbb{I}\left( \left\{({1}/{n}) \sum\nolimits_i z_i\hm > 0{,}5\right \}\right)$. 

\begin{figure*} %fig10
\vspace*{-3pt}
\begin{center}
   \mbox{%
\epsfxsize=148.67mm 
\epsfbox{gor-10.eps}
}
\end{center}
\vspace*{-12pt}
        \Caption{Кластеризация объектов на группы с~высоким входящим, исходящим 
        и~общим трафиком с~использованием скользящего окна: \textit{1}~--- аномален по полученному трафику;
\textit{2}~--- аномален по отправленному трафику;
\textit{3}~--- аномален по обоим типам трафика}
    \label{moving_window_60m}
\vspace*{-6pt}
\end{figure*}

На рис.~\ref{disjoint_window_60m} не были отображены~3~объекта с~наибольшими 
значениями глав\-ных компонент, рас\-по\-ла\-га\-ющи\-еся в~правом верхнем углу на 
рис.~\ref{moving_window_60m}, так как подход на основе не\-пе\-ре\-се\-ка\-ющих\-ся окон 
отнес к~аномальным по общему трафику~37~объектов, большая часть из которых 
продемонстрирована. Метод скользящего окна в~качестве 
аномальных сразу по обоим типам трафика отметил только 
эти~3~наблюдения, что действительно поз\-во\-ля\-ет говорить об их существенных 
отличиях от других объектов.

\vspace*{-12pt}

\section{Заключение}
\vspace*{-2pt}

В работе рас\-смот\-рен статистический под\-ход к~выявлению аномальных на\-гру\-зок на 
узлах сетевой вы\-чис\-ли\-тель\-ной архитектуры. С~использованием языка 
программирования~R реализована процедура определения аномальных наблюдений 
в~трафике в~рамках хорошо со\-от\-вет\-ст\-ву\-юще\-го реальным данным предположения 
о~воз\-мож\-ности описания наблюдений обоб\-щен\-ным гам\-ма-рас\-пре\-де\-ле\-ни\-ем. На основе 
анализа по\-яв\-ле\-ния таких наблюдений в~узлах может быть реализована процедура 
клас\-те\-ри\-за\-ции объектов сети. Дальнейшие на\-прав\-ле\-ния исследований в~данной 
об\-ласти могут быть связаны с~по\-стро\-ени\-ем ста\-ти\-сти\-че\-ских моделей для иных 
характеристик вы\-чис\-ли\-тель\-ных узлов (например, за\-груз\-ки процессора или 
ис\-поль\-зу\-емой памяти) и~решение описанной задачи клас\-те\-ри\-за\-ции в~расширенном 
при\-зна\-ко\-вом пространстве. 

Предложенная методика имеет определенный потенциал и~для других задач в~об\-ласти 
телекоммуникаций, например связанных с~рас\-пре\-де\-ле\-ни\-ем ресурсов для виртуальных 
машин~\cite{Huang2016,Tian2022}. Алгоритмы такого рода могут быть реализованы 
в~виде сервисов отдельных аналитических сис\-тем~\cite{Gorshenin2016b,Gorshenin2017b} или в~рамках циф\-ро\-вых 
плат\-форм~\cite{Gorshenin2018}.

\medskip

Авторы выражают признательность чле\-ну-кор\-рес\-пон\-ден\-ту РАН Р.\,Л.~Сме\-лян\-ско\-му за 
ценные\linebreak советы, ка\-са\-ющи\-еся телекоммуникационной со\-став\-ля\-ющей \mbox{статьи}, а~так\-же 
профессору В.\,Ю.~Королеву за пло\-до\-твор\-ные об\-суж\-де\-ния вопросов моделирования 
реальных данных с~\mbox{использованием} различных семейств вероятностных распределений.
{\looseness=-1

}

\vspace*{-6pt}

{\small\frenchspacing
 { %\baselineskip=12pt
 %\addcontentsline{toc}{section}{References}
 \begin{thebibliography}{99}
\vspace*{-3pt}

\bibitem{Smeliansky2019}  %2
\Au{Смелянский~Р.\,Л.} Иерархические периферийные 
вы\-чис\-ле\-ния~// Моделирование и~анализ информационных сис\-тем, 2019. Т.~26. Вып.~1.  C.~146--169.
doi: 10.18255/1818-1015-2019-1-146-169.

\bibitem{Smeliansky2023} 
\Au{Smeliansky~R.} Network powered by computing~// 
Edge computing~--- technology, management and integration.~--- IntechOpen, 2023. 21~p.
doi: 10.5772/\mbox{intechopen}. 110178.

\bibitem{Rossem2017} %3
\Au{Rossem~S.V., Tavernier~W., Colle~D., Pickavet~M., 
Demeester~P.} Automated monitoring and detection of resource-limited NFV-based 
services~// Conference on Network Softwarization: Softwarization 
Sustaining a~Hyper-Connected World: en Route to 5G, NetSoft 2017.~--- Piscataway, 
NJ, USA: IEEE, 2017. Art.~8004220. doi: 10.1109/NETSOFT.2017.8004220.

\bibitem{Malak2021} %4
\Au{Malak~D., Medard~M., Andrews~J.\,G.} 
Spatial concentration of caching in wireless heterogeneous networks~// 
IEEE T. Wirel. Commun., 2021. Vol.~20. Iss.~6. P.~3397--3414. 
doi: 10.1109/TWC.2021.3049812.

\bibitem{Zhang2022} %5
\Au{Zhang~Z., Lu~J., Chen~H.}
Controller robust placement with dynamic traffic in software-defined networking~// Comput. Commun. 2022. Vol.~194. P.~458--467. 
doi: 10.1016/j.comcom.2022.07.018.

\pagebreak

\bibitem{Mesodiakaki2022} %6
\Au{Mesodiakaki~A., Zola~E., Kassler~A.} Robust and energy-efficient user association and traffic routing in B5G HetNets~//
Comput. Netw., 2022. Vol.~217. Art.~109305. 
doi: 10.1016/j.comnet.2022.109305.

\bibitem{Hosseinzadeh2020} %7
\Au{Hosseinzadeh~S.,
Amirmazlaghani~M., Shajari~M.} An aggregated statistical approach for network 
flood detection using gamma-normal mixture modeling~// Comput. Commun., 2020. Vol.~152. P.~137--148. 
doi: 10.1016/j.comcom.2020.01.028.

\bibitem{Abood2021} %8
\Au{Abood~M.\,S., Mustafa~A.\,S., Mahdi~H.\,F., 
Mohammed~A.-F.\,A., Hamdi~M.\,M., Hussein~N.\,A.} The analysis of teletraffic 
and handover performance in cellular system~// 3rd  
Congress (International) on Human--Computer Interaction, Optimization and Robotic Applications 
Proceedings.~--- Piscataway, NJ, USA: IEEE, 2021. P.~1--5. 
doi: 10.1109/HORA52670.2021.9461300.

\bibitem{Parulekar1997} %9
\Au{Parulekar~M., Makowski~A.\,M.} $M|G|\infty$ input 
processes: A~versatile class of models for network traffic~// IEEE INFOCOM Proceedings.~--- Piscataway, NJ, USA: IEEE, 1997. Vol.~2. P.~419--426.
doi: 10.1109/INFCOM. 1997.644490.

\bibitem{Tabassum2014} %10
\Au{Tabassum~H., Dawy~Z., Hossain~E., Alouini~M.-S.} 
Interference statistics and capacity analysis for uplink transmission in two-tier 
small cell networks: A~geometric probability approach~//
IEEE T. Wirel. Commun., 2014. Vol.~13. Iss.~7. P.~3837--3852. doi: 10.1109/TWC.2014.2314101.

\bibitem{Noor2022} %11
\Au{Noor~K., Shahid~H., Obaid~H.\,M., Rauf~A., Yousaf~A., Shahid~A.}
Hybrid underwater intelligent communication system //
Wireless Pers. Commun., 2022. Vol.~125. Iss.~3. P.~2219--2238. doi: 10.1007/s11277-022-09653-7.

\bibitem{Padhan2023} %12
\Au{Padhan~A.\,K., Kumar~S.\,H., Sahu~P.\,R., Samantaray~S.\,R.} 
Performance analysis of smart grid wide area network with RIS 
assisted three hop system~// IEEE Transactions Signal Information  Processing Networks, 2023. Vol.~9. P.~48--59.
doi: 10.1109/TSIPN.2023.3239652.

\bibitem{Gorshenin2013b} %14
\Au{Gorshenin~A., Korolev~V., Kuzmin~V., Zeif\-man~A.} 
Coordinate-wise versions of the grid method for the analysis of intensities of 
non-stationary information flows by moving separation of mixtures of gamma-distribution~// 
27th European Conference on Modelling and Simulation Proceedings~/
  Eds. W.~Rekdalsbakken, R.\,T.~Bye, and H.~Zhang.~--- 
Dudweiler, Germany: Digitaldruck Pirrot GmbHP, 2013. P.~565--568. 
doi: 10.7148/2013-0565.

\bibitem{Gorshenin2013a} %13
\Au{Gorshenin~A., Korolev~V.} Modelling of 
statistical fluctuations of information flows by mixtures of gamma 
distributions~// 27th European Conference on Modelling and Simulation Proceedings~/ 
Eds. W.~Rekdalsbakken, R.\,T.~Bye, and H.~Zhang.~--- Dudweiler, Germany: Digitaldruck Pirrot GmbHP, 2013. P.~569--572. 
doi: 10.7148/2013-0569.

\bibitem{Gorshenin2015} %15
\Au{Gorshenin~A., Kuzmin~V.} Online system for the 
construction of structural models of information flows~// 7th 
 Congress (International) on Ultra Modern Telecommunications and Control Systems  and 
Workshops Proceedings.~--- Piscataway, NJ, USA: IEEE, 2015. P.~216--219. doi: 10.1109/ICUMT.2015.7382430.

\bibitem{Gorshenin2016a} %16
\Au{Gorshenin~A., Kuzmin~V.} On an interface of the 
online system for a~stochastic analysis of the varied information flows~// AIP 
Conf. Proc., 2016. Vol.~1738. Art.~220009. 4~p. doi: 10.1063/1.4952008.

\bibitem{Gorshenin2017a} %17
\Au{Горшенин~А.\,К.} О некоторых математических и~программных методах по\-стро\-ения структурных моделей информационных 
потоков~// Информатика и~её применения, 2017. Т.~11. Вып.~1. C.~58--68. doi: 10.14357/ 19922264170105.

\bibitem{Zonoozi1996} %18
\Au{Zonoozi~M.\,M., Dassanayake~P., Faulkner~M.} 
Teletraffic modelling of cellular mobile networks~// IEEE VTC~P., 1996. 
Vol.~2. P.~1274--1277. doi: 10.1109/VETEC.1996.501517.

\bibitem{Stacy1962} %19
\Au{Stacy~E.\,W.} A~generalization of the gamma 
distribution~// Ann. Math. Stat., 1962. Vol.~33. Iss.~3. 
P.~1187--1192. doi: 10.1214/aoms/1177704481.

\bibitem{Gorshenin2020} %20
\Au{Korolev~V.\,Yu., Gor\-she\-nin~A.\,K.} Probability 
models and statistical tests for extreme precipitation based on generalized 
negative binomial distributions~// Mathematics, 2020. Vol.\,8. Iss.\,4. 
Art.~604. doi: 10.3390/math8040604.

\bibitem{Huang2016} %21
\Au{Huang~B., Chen~J., He~Q., Wang~B., Liu~Z., Cheng~Y.} 
HASO: A~hot-page aware scheduling optimization method in virtualized NUMA 
systems~// 7th  Conference (International) on Information and Communication 
Systems Proceedings.~--- Piscataway, NJ, USA: IEEE, 2016. P.~68--73. doi: 10.1109/IACS.2016.7476088.

\bibitem{Tian2022} %22
\Au{Tian~H., Li~S., Wang~A., Wang~W., Wu~T., Yang~H.} 
Owl:  Performance-aware scheduling for resource-efficient function-as-a-service cloud~// 
13th Symposium on Cloud Computing Proceedings.~--- New York, NY, USA: 
Association for Computing Machinery, 2022. P.~78--93. doi: 10.1145/3542929.3563470.

\bibitem{Gorshenin2016b} %23
\Au{Горшенин~А.\,К.} Концепция он\-лайн-комп\-лек\-са для 
сто\-ха\-сти\-че\-ско\-го моделирования реальных процессов~// Информатика и~её 
применения, 2016. Т.~10. Вып.~1. C.~72--81.  doi: 10.14357/19922264160107.

\bibitem{Gorshenin2017b} %24
\Au{Gorshenin~A.\,K., Kuz\-min~V.\,Yu.} Research 
support system for stochastic data processing~// Pattern Recognition  Image 
Analysis, 2017. Vol.~27. No.~3. P.~518--524.  doi: 10.1134/S1054661817030117.

\bibitem{Gorshenin2018} %25
 \Au{Gorshenin~A.} Toward modern educational IT-ecosystems: From learning management systems to digital platforms~// 
10th  Congress (International) on Ultra Modern Telecommunications and 
Control Systems and Workshops Proceedings.~--- Piscataway, NJ, USA: IEEE, 2018. 
P.~329--333. doi: 10.1109/ICUMT.2018.8631229.
\end{thebibliography}

 }
 }

\end{multicols}

\vspace*{-6pt}

\hfill{\small\textit{Поступила в~редакцию 15.07.23}}

\vspace*{8pt}

%\pagebreak

\newpage

\vspace*{-28pt}


\def\tit{TOWARD CLUSTERING OF~NETWORK COMPUTING INFRASTRUCTURE OBJECTS BASED ON ANALYSIS OF~STATISTICAL ANOMALIES IN~NETWORK TRAFFIC}


\def\titkol{Toward clustering of~network computing infrastructure objects based on analysis of~statistical anomalies in~network traffic}


\def\aut{A.\,K.~Gorshenin$^{1,2}$, S.\,A.~Gorbunov$^{2,3}$, and~D.\,Yu.~Volkanov$^2$}

\def\autkol{A.\,K.~Gorshenin, S.\,A.~Gorbunov, and~D.\,Yu.~Volkanov}

\titel{\tit}{\aut}{\autkol}{\titkol}

\vspace*{-10pt}

\noindent
$^{1}$Federal Research Center ``Computer Science and Control'' of the Russian Academy of Sciences, 44-2~Vavilov\linebreak
$\hphantom{^1}$Str., Moscow 119133, Russian Federation

\noindent
$^{2}$M.\,V.~Lomonosov Moscow State University, 1~Leninskie Gory, GSP-1, Moscow 119991, Russian Federation

\noindent
$^{3}$Moscow Center for Fundamental and Applied Mathematics, M.\,V.~Lomonosov Moscow State University,\linebreak
$\hphantom{^1}$1-52~Leninskie Gory, GSP-1, Moscow 119991, Russian Federation



\def\leftfootline{\small{\textbf{\thepage}
\hfill INFORMATIKA I EE PRIMENENIYA~--- INFORMATICS AND
APPLICATIONS\ \ \ 2023\ \ \ volume~17\ \ \ issue\ 3}
}%
 \def\rightfootline{\small{INFORMATIKA I EE PRIMENENIYA~---
INFORMATICS AND APPLICATIONS\ \ \ 2023\ \ \ volume~17\ \ \ issue\ 3
\hfill \textbf{\thepage}}}

\vspace*{3pt}



\Abste{The problem of detecting statistical anomalies (that is, outliers in relation to the typical values of upload and download traffic) 
of the load on the nodes of the network computing infrastructure is considered. The regular scaling in computing resources and storage 
as well as redirection of data flows is needed due to the increase of load in real systems. The procedure for detecting statistical
 anomalies in network traffic is proposed using the approximation of observations by the generalized gamma distribution for further 
 clustering of network computing infrastructure objects in order to evaluate resource need. All computational statistical 
 procedures described in the paper are implemented using the~R programming language and they are applied for network traffic, 
 simulated using a~specialized architectural and software stand. The proposed approaches can also be used for a~wider class of telecommunication problems.}

\KWE{network  infrastructure; network traffic; generalized gamma distribution; computational statistics; statistical hypothesis testing; anomaly detection; 
clustering}



\DOI{10.14357/19922264230311}{XHTMVI}

\vspace*{-12pt}

\Ack
\noindent
This work was done with the support of MSU Program of Development, Project No.\,23-SCH03-03.
The research was carried out using the infrastructure of the Shared Research Facilities ``High Performance Computing and Big Data'' (CKP ``Informatics'') 
of FRC CSC RAS (Moscow).

%\vspace*{8pt}

  \begin{multicols}{2}

\renewcommand{\bibname}{\protect\rmfamily References}
%\renewcommand{\bibname}{\large\protect\rm References}

{\small\frenchspacing
 {%\baselineskip=10.8pt
 \addcontentsline{toc}{section}{References}
 \begin{thebibliography}{99} 

\bibitem{Smeliansky2019-1} %2
\Aue{Smelyansky, R.\,L.} 
2019. Ierar\-khi\-che\-skie pe\-ri\-fe\-riy\-nye vy\-chis\-le\-niya [Hierarchical edge computing]. 
\textit{Mo\-de\-li\-ro\-va\-nie i~ana\-liz in\-for\-ma\-tsi\-on\-nykh sis\-tem} 
[Modeling and Analysis of Information Systems] 26(1):146--169. doi: 10.18255/1818-1015-2019-1-146-169.

\bibitem{Smeliansky2023-1} 
\Aue{Smeliansky, R.} 2023. 
Network powered by computing. \textit{Edge computing~--- technology, management and integration}.  
IntechOpen. 21~p. doi: 10.5772/intechopen.110178.

\bibitem{Rossem2017-1} %3
\Aue{Rossem, S.\,V., W.~Tavernier, D.~Colle, M.~Pickavet, and P.~Demeester.}
 2017. Automated monitoring and detection of resource-limited NFV-based services. 
 \textit{Conference on Network Softwarization: Softwarization Sustaining a~Hyper-Connected World: en Route to 5G}. 
Piscataway, NJ: IEEE. 8004220. doi: 10.1109/NETSOFT.2017.8004220.

\bibitem{Malak2021-1} %4
\Aue{Malak, D., M.~Medard, and J.\,G.~Andrews.}
 2021. Spatial concentration of caching in wireless heterogeneous networks. \textit{IEEE T. Wirel. Commun.} 
 20(6):3397--3414. doi: 10.1109/TWC.2021.3049812.

\bibitem{Zhang2022-1}  %5
\Aue{Zhang, Z., J.~Lu, and H.~Chen.}
 2022. Controller robust placement with dynamic traffic in software-defined networking. \textit{Comput. Commun.} 194:458--467. 
 doi: 10.1016/ j.comcom.2022.07.018.

\bibitem{Mesodiakaki2022-1} %6
\Aue{Mesodiakaki, A., E.~Zola, and A.~Kassler.} 2022. Robust and energy-efficient 
user association and traffic routing in B5G HetNets. 
\textit{Comput. Netw.} 217:109305. doi: 10.1016/j.comnet.2022.109305.

\bibitem{Hosseinzadeh2020-1} %7
\Aue{Hosseinzadeh, S., M.~Amirmazlaghani, and M.~Shajari.}
 2020. An aggregated statistical approach for network flood detection using 
 gamma-normal mixture modeling. \textit{Comput. Commun.}
  152:137--148. doi: 10.1016/ j.comcom.2020.01.028.

\bibitem{Abood2021-1} %8
\Aue{Abood, M.\,S., A.\,S. Mustafa, H.\,F. Mahdi, A.-F.\,A.~Mohammed, M.\,M.~Hamdi, and N.\,A.~Hussein.}
 2021. The analysis of teletraffic and handover performance in cellular system. 
 \textit{3rd Congress (International) on Human--Computer Interaction, Optimization and Robotic Applications Proceedings.} 
 Piscataway, NJ: IEEE. 1--5. doi: 10.1109/\linebreak HORA52670.2021.9461300.

\bibitem{Parulekar1997-1} %9
\Aue{Parulekar, M., and A.\,M.~Makowski.} 1997. 
$M|G|\infty$ input processes: A~versatile class of models for network traffic. 
\textit{IEEE INFOCOM Proceedings}. Piscataway, NJ: IEEE. 2:419--426. doi: 10.1109/INFCOM.1997.644490.

\bibitem{Tabassum2014-1} %10
\Aue{Tabassum, H., Z.~Dawy, E.~Hossain, and M.-S.~Alouini.}
 2014. Interference statistics and capacity analysis for uplink transmission in two-tier small cell networks: A~geometric probability approach. 
 \textit{IEEE T. Wirel. Commun.} 13(7):3837--3852. doi: 10.1109/TWC.2014.2314101.

\bibitem{Noor2022-1} %11
\Aue{Noor, K., H.~Shahid, H.\,M.~Obaid, A.~Rauf, A.~Yousaf, and  A.~Shahid.} 2022. 
Hybrid underwater intelligent communication system. \textit{Wireless Pers. Commun.} 125(3):2219--2238. doi: 10.1007/s11277-022-09653-7.

\bibitem{Padhan2023-1} %12
\Aue{Padhan, A.\,K., S.\,H.~Kumar, P.\,R.~Sahu and S.\,R.~Samantaray.}
 2023.  Performance analysis of smart grid wide area network with RIS assisted three hop system. 
 \textit{IEEE Transactions Signal Information Processing Networks} 9:48--59. doi: 10.1109/TSIPN.2023.3239652.

\bibitem{Gorshenin2013b-1} %14
\Aue{Gorshenin, A., V.~Korolev, V.~Kuz\-min, and A.~Zeif\-man.}
 2013. Coordinate-wise versions of the grid method for the analysis of intensities of non-stationary information flows by moving separation of mixtures
  of gamma-distribution. \textit{27th European Conference on Modelling and Simulation Proceedings}. 
  Eds. W.~Rekdalsbakken, R.\,T.~Bye, and H.~Zhang. Dudweiler, Germany: Digitaldruck Pirrot GmbHP. 565--568. doi: 10.7148/2013-0565.

\bibitem{Gorshenin2013a-1} %13 
\Aue{Gorshenin, A., and V.~Ko\-ro\-lev.}
 2013. Modelling of statistical fluctuations of information flows by mixtures of gamma distributions. 
 \textit{27th European Conference on Modelling and Simulation Proceedings}. Eds.\ W.~Rekdalsbakken, R.\,T.~Bye, and H.~Zhang. 
 Dudweiler, Germany: Digitaldruck Pirrot GmbHP. 569--572. doi: 10.7148/2013-0569.

\bibitem{Gorshenin2015-1} %15
\Aue{Gorshenin, A.\,K., and V.~Kuz\-min.}
 2015. Online system for the construction of structural models of information flows. 
 \textit{7th Congress (International) on Ultra Modern Telecommunications and Control Systems and Workshops Proceedings}. 
 Piscataway, NJ: IEEE. 216--219. doi: 10.1109/ICUMT.2015.7382430.

\bibitem{Gorshenin2016a-1} %16
\Aue{Gorshenin, A., and V.~Kuz\-min.}
 2016. On an interface of the online system for a~stochastic analysis of the varied information flows. 
 \textit{AIP Conference Proceedings} 1738(1):220009. 4~p. doi: 10.1063/1.4952008.

\bibitem{Gorshenin2017a-1} %17
\Aue{Gorshenin, A.\,K.} 2017. O~ne\-ko\-to\-rykh ma\-te\-ma\-ti\-che\-skikh i~prog\-ram\-mnykh me\-to\-dakh po\-stro\-eniya struk\-tur\-nykh mo\-de\-ley 
in\-for\-ma\-tsi\-on\-nykh po\-to\-kov [On some mathematical and programming methods for construction of structural models of information flows]. 
\textit{Informatika i~ee Primeneniya~--- Inform. Appl.} 11(1):58--68. doi: 10.14357/ 19922264170105.

\bibitem{Zonoozi1996-1} %18
\Aue{Zonoozi, M.\,M., P.~Dassanayake, and M.~Faulkner.}
 1996. Teletraffic modelling of cellular mobile networks. 
 \textit{IEEE VTC~P.} 2:1274--1277. doi: 10.1109/VETEC.1996.501517.

\bibitem{Stacy1962-1} %19
\Aue{Stacy, E.\,W.} 1962. A~generalization of the gamma distribution. \textit{Ann. Math. Stat.} 33(3):1187--1192. doi: 10.1214/aoms/1177704481.

\bibitem{Gorshenin2020-1} %20
\Aue{Korolev, V.\,Yu., and A.\,K.~Gor\-she\-nin.}
 2020. Probability models and statistical tests for extreme precipitation based on generalized negative binomial distributions. 
 \textit{Mathematics} 8(4):604. doi: 10.3390/math8040604.


\bibitem{Huang2016-1} %21
\Aue{Huang, B., J.~Chen, Q.~He, B.~Wang, Z.~Liu, and Y.~Cheng.}
 2016. HASO: A~hot-page aware scheduling optimization method in virtualized NUMA systems. 
 \textit{7th  Conference (International) on Information and Communication Systems  Proceedings}. 
 Piscataway, NJ: IEEE. 68--73. doi: 10.1109/IACS.2016.7476088.

\bibitem{Tian2022-1} %22
\Aue{Tian, H., S.~Li, A.~Wang, W.~Wang, T.~Wu, and H.~Yang.}
 2022. Owl: Performance-aware scheduling for resource-efficient function-as-a-service cloud. 
 \textit{13th Symposium on Cloud Computing  Proceedings}. 
 New York, NY: Association for Computing Machinery. 78--93. 
doi: 10.1145/3542929.3563470.

\bibitem{Gorshenin2016b-1} %23
\Aue{Gorshenin, A.\,K.} 2016. 
Kon\-tsep\-tsiya onlayn-kompleksa dlya sto\-kha\-sti\-che\-sko\-go mo\-de\-li\-ro\-va\-niya real'\-nykh pro\-tses\-sov 
[Concept of online service for stochastic modeling of real processes]. \textit{Informatika i~ee Primeneniya~--- Inform. Appl.} 10(1):72--81. 
doi: 10.14357/19922264160107.

\bibitem{Gorshenin2017b-1} %24
\Aue{Gorshenin, A.\,K., and V.\,Yu.~Kuz\-min.} 
2017. Research support system for stochastic data processing. 
\textit{Pattern Recognition Image Analysis} 27(3):518--524. doi: 10.1134/ S1054661817030117.

\bibitem{Gorshenin2018-1} %25
\Aue{Gorshenin, A.} 2018. Toward modern educational IT-ecosystems: From learning management systems to digital platforms. 
\textit{10th  Congress (International) on Ultra Modern Telecommunications and Control Systems and Workshops Proceedings}. 
Piscataway, NJ: IEEE. 329--333. doi: 10.1109/\mbox{ICUMT}.2018.8631229.

\end{thebibliography}

 }
 }

\end{multicols}

\vspace*{-6pt}

\hfill{\small\textit{Received July 15, 2023}} 

\vspace*{-12pt}

\Contr

\noindent
\textbf{Gorshenin Andrey K.} (b.\ 1986)~--- Doctor of Science in physics and mathematics, 
associate professor, principal scientist, head of department, 
Federal Research Center ``Computer Science and Control'' of the Russian Academy of Sciences, 44-2~Vavilov Str., 
Moscow 119333, Russian Federation; associate professor, Department of Mathematical Statistics, 
Faculty of Computational Mathematics and Cybernetics, M.\,V.~Lomonosov Moscow State University, 
1-52~Leninskie Gory, GSP-1, Moscow 119991, Russian Federation; \mbox{agorshenin@frccsc.ru}

\vspace*{3pt}

\noindent
\textbf{Gorbunov Sergei A.} (b.\ 2000)~--- master student, Faculty of Computational Mathematics and Cybernetics, 
M.\,V.~Lomonosov Moscow State University, 1-52~Leninskie Gory, GSP-1, Moscow 119991, Russian Federation; 
mathematician, Moscow Center for Fundamental and Applied Mathematics, M.\,V.~Lomonosov Moscow State University, 
1-52~Leninskie Gory, GSP-1, Moscow 119991, Russian Federation; \mbox{shadesilent@yandex.ru}

\vspace*{3pt}

\noindent
\textbf{Volkanov Dmitrii Yu.} (b.\ 1979)~--- Candidate of Science (PhD) in physics and mathematics, associate 
professor, Faculty of Computational Mathematics and Cybernetics, M.\,V.~Lomonosov Moscow State University, 
1-52~Leninskie Gory, GSP-1, Moscow 119991, Russian Federation; \mbox{volkanov@asvk.cs.msu.ru}



\label{end\stat}

\renewcommand{\bibname}{\protect\rm Литература} 