\def\stat{zeifman}

\def\tit{О СКОРОСТИ СХОДИМОСТИ И~ПРЕДЕЛЬНЫХ ХАРАКТЕРИСТИКАХ 
ДЛЯ~ОДНОГО ОБОБЩЕННОГО ПРОЦЕССА~РОЖДЕНИЯ И~ГИБЕЛИ}

\def\titkol{О скорости сходимости и~предельных характеристиках 
для~одного обобщенного процесса рождения и~гибели}

\def\aut{И.\,А.~Усов$^1$, Я.\,А.~Сатин$^2$, А.\,И.~Зейфман$^3$}

\def\autkol{И.\,А.~Усов, Я.\,А.~Сатин, А.\,И.~Зейфман}

\titel{\tit}{\aut}{\autkol}{\titkol}

\index{Усов И.\,А.}
\index{Сатин Я.\,А.}
\index{Зейфман А.\,И.}
\index{Usov I.\,A.}
\index{Satin Y.\,A.}
\index{Zeifman A.\,I.}


%{\renewcommand{\thefootnote}{\fnsymbol{footnote}} \footnotetext[1]
%{Работа выполнялась с~использованием инфраструктуры Центра коллективного пользования 
%<<Высокопроизводительные вычисления и~большие данные>> (ЦКП <<Информатика>>) ФИЦ 
%ИУ РАН (г.~Моск\-ва).}}


\renewcommand{\thefootnote}{\arabic{footnote}}
\footnotetext[1]{Вологодский государственный  университет, iusov35@yandex.ru}
\footnotetext[2]{Вологодский государственный  университет, yacovi@mail.ru}
\footnotetext[3]{Вологодский государственный университет; 
Федеральный исследовательский центр <<Информатика и~управ\-ле\-ние>> Российской академии наук; 
Вологодский научный центр Российской академии наук; 
Московский центр фундаментальной и~при\-клад\-ной математики; 
Московский  государственный университет имени М.\,В.~Ломоносова, \mbox{a\_zeifman@mail.ru}}

\vspace*{-12pt}


       
       \Abst{Рассмотрена система обслуживания с~одним сервером и~разными вариантами 
ремонта и~отказов, чис\-ло требований в~которой опи\-сы\-ва\-ет\-ся неоднородным 
обобщенным процессом рож\-де\-ния и~гибели (ПРГ) (т.\,е.\ процессом, в~котором 
интенсивности переходов не константы, а~функ\-ции времени). Для обоснованного 
на\-хож\-де\-ния предельных вероятностных характеристик сис\-те\-мы изучается ско\-рость 
схо\-ди\-мости к~ним (т.\,е.\ ско\-рость, с~которой <<забываются>> начальные условия 
сис\-те\-мы). Для исследования ско\-рости схо\-ди\-мо\-сти к~предельному режиму применен 
недавно разработанный вариант подхода, основанного на понятии логарифмической 
нор\-мы операторной функции, со\-от\-вет\-ст\-ву\-ющей оценке нор\-мы мат\-ри\-цы Коши, а~так\-же 
модернизированного специального преобразования прямой сис\-те\-мы Колмогорова. 
Рас\-смот\-рен чис\-лен\-ный пример, в~котором детально показано получение оценок 
ско\-рости схо\-ди\-мости  и~основанное на этих оценках по\-стро\-ение некоторых 
предельных характеристик модели.}
    
    
\KW{обобщенный процесс рождения и~гибели; 
ско\-рость схо\-ди\-мости; эр\-го\-дич\-ность; логарифмическая нор\-ма; сис\-те\-мы массового 
обслуживания}

\DOI{10.14357/19922264230307}{RBKZJI}
  
%\vspace*{-4pt}


\vskip 10pt plus 9pt minus 6pt

\thispagestyle{headings}

\begin{multicols}{2}

\label{st\stat}
    
    
    \section{Введение}
    
    Изучение обобщенных ПРГ 
играет важную роль в~слож\-ных телекоммуникационных сис\-те\-мах, био\-ло\-гии 
и~радиотехнике. В~со\-вре\-мен\-ной разработке программного обеспечения все чаще 
применяется мик\-ро\-сер\-вис\-ная архитектура и~облач\-ная инфраструктура, где прием 
и~обслуживание за\-про\-сов имеют свои особенности и~очереди.
    
    Обобщенные ПРГ имеют различные приложения, например при моделировании 
сис\-тем сборки на заказ~\cite{ILS2010}, производственных линий~\cite{FY2002}, 
беспроводных сетей~\cite{KL1999}. Ранее многие исследователи изуча\-ли 
многосерверные сис\-те\-мы с~отказами и~ремонтами~\cite{W1994, JM2004, JG1997, JM2003, KW2012}, 
а~проб\-ле\-ма\-ми тео\-рии на\-деж\-ности 
и~исследованием некоторых сис\-тем массового обслуживания (СМО) ак\-тив\-но занимались 
отечественные авторы (см., например, работы~\cite{VD2017, VSB2020, DN2015}).
{\looseness=1

}
    
    В \cite{WK2014} была рассмотрена модель с~различными вариантами ремонта 
    и~использован мат\-рич\-но-ана\-ли\-ти\-че\-ский метод для получения стационарной ве\-ро\-ят\-ности 
и~исследования раз\-лич\-ных показателей эф\-фек\-тив\-ности сис\-темы.
    

 При исследовании СМО, чис\-ло требований в~которых описывается неоднородной 
марковской цепью с~не\-пре\-рыв\-ным временем, очень важ\-ную роль играют предельные 
вероятностные характеристики (в~однородном случае этому соответствует 
стационарное распределение). Для обосно\-ван\-но\-го их по\-стро\-ения необходимо оценить 
ско\-рость схо\-ди\-мости к~предельному режиму. Основ\-ные методы, поз\-во\-ля\-ющие получить 
такие оценки, описаны в~\cite{ZSKRKS2020}. Однако для обоб\-щен\-ных 
ПРГ эти методы применить не удается.
    
    В данной статье рассмотрен метод, основанный на понятии ло\-га\-риф\-ми\-че\-ской 
нор\-мы и~использовании мат\-ри\-цы~$C(t)$ для перехода к~редуцированной прямой 
сис\-те\-ме Колмогорова в~форме 
$$
\fr{d\mathbf{z}(t) }{dt}=W(t)\mathbf{z}\left(t\right)
$$ 
при  $t\hm\ge 0$ (по\-дроб\-нее см.\ в~разд.~4). Данный метод  без  
по\-дроб\-но\-стей был описан в~работе~\cite{Satin2021}, а~в~работе~\cite{Razumchik2022} была впер\-вые показана воз\-мож\-ность его применения 
к~обобщенному ПРГ, опи\-сы\-ва\-юще\-му один класс суперкомпьютерных 
сис\-тем. По\-дроб\-ное описание данного подхода было сделано в~\cite{KSSZ2022}. 
В~недавней работе~\cite{USZ2023} с~по\-мощью этого метода была 
исследована модель массового обслуживания $\mathrm{PH}/M/1$ с~одним сервером.
    
    Модель, ис\-сле\-ду\-емая в~текущей работе, в~однородном случае была введена 
    и~изуче\-на в~\cite{L2021}. С~по\-мощью метода Лап\-ла\-са--Стилть\-еса были получены оценки 
для раз\-лич\-ных показателей на\-деж\-ности сис\-те\-мы и~исследованы част\-ные случаи. 

В~на\-сто\-ящей работе исследован нестационарный случай (т.\,е.\ ситуация, в~которой   
интенсивности процесса, опи\-сы\-ва\-юще\-го чис\-ло требований в~сис\-те\-ме, зависят от 
времени), получена оцен\-ка ско\-рости схо\-ди\-мости к~предельному режиму 
и~рас\-смот\-рен чис\-лен\-ный пример.
    
\vspace*{-7pt}   

    \section{Основные понятия}
    
    \vspace*{-1pt}
    
        Обозначим через $ \|\cdot\| $ $l_1$-нор\-му век\-то\-ра, 
$\|\mathbf{x}\|\hm=\sum|x_i|$,
    $\|H\| \hm= \sup_j \sum_{i \geq 0} |h_{ij}|$, если $H \hm= 
(h_{ij})_{i,j=0}^{\infty}$, и~обозначим через~$\Omega $ множество всех векторов 
из~$l_1$ с~не\-от\-ри\-ца\-тель\-ны\-ми координатами и~единичной нормой. Пусть $A(t) \hm= 
(a_{ij}(t))_{i,j=0}^{\infty}$~--- транс\-по\-ни\-ро\-ван\-ная мат\-ри\-ца 
интенсивностей переходов некоторого марковского
    процесса, причем $|\alpha_{ii}| \hm< L \hm< \infty$ при всех~$i$ и~почти всех $t \hm\geq 0$. 
    Тогда имеем 
    $$
    \|A(t)\| = 2\sup_{i}\left|a_{ii}\left(t\right)\right| \le 2L
    $$ 
    почти для всех $t \hm\ge 0$.
    
    Можем рассматривать прямую сис\-те\-му Колмогорова
             $$
             \fr{d\mathbf{p}(t)}{dt}= A(t)\mathbf{p}(t)
             $$
     как дифференциальное урав\-не\-ние в~про\-стран\-ст\-ве по\-сле\-до\-ва\-тель\-но\-стей~$l_1$, 
где $A\left (t\right)$~--- ограниченный почти для всех $ t \hm\ge 0 $ линейный 
оператор из~$l_1$ в~\mbox{себя}.
    
    
    Напомним, что марковская цепь~$X\left(t\right)$ называется \textit{слабо эргодичной}, если
  $$ 
\lim\limits_{t\rightarrow \infty }\left\| \mathbf{p}^1\left( 
t\right)-\mathbf{p}^2\left( t\right) \right\| =0 
$$ 
для любых начальных условий 
$$
\mathbf{p}^1\left( 0\right) =\mathbf{p}^1\in \Omega ,\enskip \mathbf{p}^2\left( 
0\right) =\mathbf{p}^2\in \Omega\,,
$$ 

  \vspace*{-2pt}
    
    \noindent
где $\mathbf{p}^1(t)$ 
и~$\mathbf{p}^2(t)$~--- это решения прямой сис\-те\-мы Колмогорова, опи\-сы\-ва\-ющей 
некоторый марковский про\-цесс.
    
    Напомним также, что логарифмическая норма операторной функции в~$l_1$ 
вы\-чис\-ля\-ет\-ся по фор\-муле:
    \begin{equation*}
        \gamma \left( H\left(t\right) \right)_{1} = \sup\limits_i 
\left(h_{ii}\left(t\right) + \sum\limits_{j\neq i} |h_{ji}\left(t\right)|\right).
       % \label{k11}
    \end{equation*}
    
    \vspace*{-2pt}
    
    \noindent
Оператор Коши со\-от\-вет\-ст\-ву\-юще\-го дифференциального урав\-не\-ния 
$$
\fr{d\mathbf{x}(t)}{dt}=H(t)\mathbf{x}(t)
$$
 име\-ет~вид:
 
 \columnbreak
 
 \noindent
\begin{multline*}
    U\left(t,s\right) = I + \int\limits_{s}^{t}H(t_1)\,dt_1 + {}\\[-1pt]
    {}+ 
\int\limits_{s}^{t}H(t_1)\int\limits_{s}^{t_1}H(t_2)\,dt_2dt_1 + \cdots, 
\end{multline*}

\vspace*{-3pt}

\noindent
и~справедлива сле\-ду\-ющая оцен\-ка (по\-дроб\-нее см.\ в~\cite{ZKKS2016}):
    \begin{equation*}
        \|U\left(t,s\right)\| \le e^{\int\nolimits_s^t \gamma\left(H(\tau)\right)\, d\tau}.     
%\label{k12}
    \end{equation*}
    
    \vspace*{-22pt}
    
    \section{Описание модели}
    
    \vspace*{-1pt}
    
    Рассмотрим классическую СМО с~конечным чис\-лом источников, т.\,е.\ из $N$ приборов и~одного сервера. Каж\-дый прибор под\-вер\-жен по\-лом\-кам по 
независимому пуассоновскому процессу с~ин\-тен\-сив\-ностью~$\lambda(t)$. Когда прибор 
ломается, за\-яв\-ка немедленно обслуживается сервером, если сервер до\-сту\-пен. 
В~противном случае
аварийные приборы долж\-ны ждать в~очереди на обслуживание сервером. 

Будем 
предполагать, что время обслуживания для аварийного прибора имеет 
экспоненциальное распределение с~ин\-тен\-сив\-ностью~$\mu(t)$. Сервер
    может выйти из строя в~любой момент, время до отказа имеет 
экспоненциальное распределение с~разными вариантами отказов: $\zeta_1(t)$~--- 
в~простое и~$\zeta_2(t)$~--- во время за\-ня\-тости. Когда сервер выходит из строя, он 
отправляется в~ремонт, а~время ремонта имеет экспоненциальное распределение; 
$\eta_1(t)$~--- ин\-тен\-сив\-ность ремонта, когда все приборы исправны, и~$\eta_2(t)$~--- 
когда хотя бы один прибор вышел из строя. Все интенсивности предполагаются 
не\-от\-ри\-ца\-тель\-ны\-ми, локально ин\-тег\-ри\-ру\-емы\-ми на $[0,\infty)$, детерминированными 
функциями вре\-мени.  

    
    Рассмотрим двумерный марковский процесс $\left(X(t), Y(t)\right)$, где $X(t)$~--- чис\-ло до\-ступ\-ных серверов в~момент времени~$t$, 
$Y(t)$~--- чис\-ло аварийных приборов в~момент времени~$t$, его  пространство 
со\-сто\-яний есть $\{(i, j) \vert  i \hm= 0, 1; j \hm= 0, 1, 2, \ldots, N\}$.  

Обозначим через 
$p_{i, j}(t)$ ве\-ро\-ят\-ность того, что сис\-те\-ма находится в~со\-сто\-янии $(i, j)$ при  $t \hm\geq 0$.
    \mbox{Тогда}
    
    \vspace*{-12pt}
  \begin{align*}
        \fr{dp_{0,0}(t)}{dt} &= p_{1,0}(t)\zeta_1(t)-p_{0,0}(t)\left(\eta_1(t)+N\lambda(t)\right);
   \\
    \fr{dp_{1,0}(t)}{dt}& = p_{0,0}(t)\eta_1(t)-p_{1,0}(t)\left(\zeta_1(t)+N\lambda(t)\right) +{}\\
&\hspace*{40mm}{}+ p_{1,1}(t)\mu(t);\\
\fr{dp_{0,j}(t)}{dt} &= p_{1,j}(t)\zeta_2(t)-p_{0,j}(t)(\eta_2(t)+{}\\
&    {}+(N-j)\lambda(t)) + p_{0,j-1}(t)(N-j+1)\lambda(t),\\
&\hspace*{45mm}0<j<N;
   \end{align*}
    
\noindent
     \begin{align*}
           \fr{dp_{1,j}(t)}{dt}& = p_{0,j}(t)\eta_2(t)-p_{0,j}(t)(\zeta_2(t)+{}\\[1pt]
   &\hspace*{-10mm}{}+(N-j)\lambda(t)) + p_{1,j-1}(t)(N-j+1)\lambda(t) +{}\\[1pt]
&   \hspace*{25mm}{}+ p_{1, j+1}\mu(t),\enskip 0<j<N;\\[1pt]
    \fr{dp_{0,N}(t)}{dt} &= p_{1,N}(t)\zeta_2(t)-p_{0,N}(t)\eta_2(t)+ {}\\[1pt]
    &\hspace*{38mm}{}+ p_{0,N-1}(t)\lambda(t);
    \\[1pt]
    \fr{dp_{1,N}(t)}{dt}& = p_{0,N}(t)\eta_2(t)-p_{1,N}(t)(\zeta_2(t)+\mu(t))+{}\\[1pt]
&\hspace*{38mm}{}+     p_{0,N-1}(t)\lambda(t).
    \end{align*}
    

    Будем считать, что со\-сто\-яния имеют сле\-ду\-ющий порядок: $\{(0,0), (1,0), 
(0,1), (1, 1), \ldots$\linebreak $\ldots, (0, N), (1, N)\}$, а~со\-от\-вет\-ст\-ву\-ющий одномерный процесс 
обозначим $\chi(t)$. Тогда $\mathbf{p}(t)\hm=(p_{0,0}(t),$\linebreak $p_{1,0}(t), p_{0,1}(t), 
p_{1,1}(t), \ldots, p_{0,N}(t), p_{1,N}(t))^T$~--- век\-тор рас\-пре\-де\-ле\-ния 
вероятностей со\-сто\-яний~$\chi(t)$ в~момент времени~$t$, а~его мат\-ри\-ца 
интенсивностей~$Q(t)$ пред\-ста\-ви\-ма в~сле\-ду\-ющем \mbox{виде}:
\begin{equation*} 
Q =
                \begin{pmatrix}
                        A_0(t)     & C_0(t)  &  0            & \cdots            & 0  \\[1pt]
                        B_1(t) & A_1(t)     & C_1(t)  & \cdots       & 0  \\[1pt]
                        0      &   \ddots       & \ddots    & \ddots  & 0  \\[1pt]
                        \vdots& \ddots& \ddots&\ddots& \vdots\\[1pt]
                        0      & \ddots            &  B_{N-1}(t)      & A_{N-1}(t)  & C_{N-1}(t)  \\[1pt]
                        0& \cdots        & 0       &B_{N}(t)       & A_{N}(t)  \\[1pt]
                \end{pmatrix},
               % \label{1}
            \end{equation*}
где

\noindent
\begin{align*}
B_1(t)&=\cdots= B_N(t) =
\begin{pmatrix}
    0  & 0  \\
    0  & \mu(t) \\
\end{pmatrix};
\\[3pt]
C_j(t) &=
\begin{pmatrix}
    (N-j)\lambda(t)  & 0  \\
    0  & (N-j)\lambda(t) \\
\end{pmatrix},\\
&\hspace*{35mm} j = 0, 1, \ldots, N-1;
\\[3pt]
A_0(t) &=
\begin{pmatrix}
    -\eta_1(t) - N\lambda(t)  & \eta_1(t)  \\
    -\zeta_1(t)  & -\zeta_1(t) - N\lambda(t) \\
\end{pmatrix};
\\[3pt]
A_j(t) &={}\\
&\hspace*{-11mm}{}=\!
\begin{pmatrix}
    -\eta_2(t)\! -\! (N-j)\lambda(t)  & \eta_2(t)  \\
    -\zeta_2(t)  & -\zeta_2(t) \!-\! \mu(t) \!-\! (N\!-\!j)\lambda(t) \\
\end{pmatrix},\\
&\hspace*{36mm} j = 0, 1, \ldots, N-1;
\\[3pt]
A_N(t)& =
\begin{pmatrix}
    -\eta_2(t)  & \eta_2(t)  \\
    \zeta_2(t)  & -\zeta_2(t) - \mu(t)
\end{pmatrix}.
\end{align*}


Процесс $\chi(t)$ описывается прямой сис\-те\-мой Колмогорова
\begin{equation}
    \fr{d\mathbf{p}(t)}{dt} = A(t)\mathbf{p}(t),
    \label{3}
\end{equation}
где мат\-ри\-ца $A(t) \hm= Q^{\mathrm{T}}(t)$.

%\vspace*{-7pt}

\section{Оценки скорости сходимости}

\vspace*{-1pt}

Для получения оценки ско\-рости схо\-ди\-мости к~предельному режиму будем использовать 
подход из~\cite{KSSZ2022}. Пусть $ \mathbf{p}^*(t) $  и~$ \mathbf{p}^{**}(t) $~--- решения 
прямой сис\-те\-мы Колмогорова~(\ref{3}) с~со\-от\-вет\-ст\-ву\-ющи\-ми (различными)
начальными условиями $\mathbf{p}^*(0) \hm\in \Omega $ и~$ \mathbf{p}^{**}(0) \hm\in \Omega $. Тогда 
их раз\-ность $\mathbf{z}(t) \hm= \mathbf{p}^*(t) \hm- \mathbf{p}^{**}(t)$ так\-же будет решением сис\-те\-мы~(\ref{3}), 
но при этом сумма всех координат вектора~$\mathbf{z}(t)$ рав\-на нулю при всех 
$t \hm\ge 0$, $\sum_{i\ge 0} z_i(t) \hm= 0$.
Пусть $c(t)$~--- некоторая положительная при всех $t\ge 0$ 
функция. Вычтем из правой час\-ти уравнения 
$$
z'_0(t) = \sum\limits_{j \ge 0} 
a_{0j}(t)z_j(t)
$$ 
выражение  
$$
 c(t) \sum\limits_{j \ge 0} z_j(t) = 0\,.
 $$
Запишем по\-лу\-чив\-шу\-юся при этом сис\-те\-му в~\mbox{виде}:
$$
\fr{d\mathbf{z}(t)}{dt}=W(t)\mathbf{z} \left(t\right), \quad t\ge 0\,,
 \label{5}
$$
 где $ W(t) \hm= A\left( t\right)\hm-C\left( t\right)$ 
и~у~мат\-ри\-цы~$C(t)$ нулевая строка со\-сто\-ит из элементов~$c(t)$, а~все остальные строки со\-сто\-ят из ну\-лей.
    
Рассмотрим по\-сле\-до\-ва\-тель\-ность положительных чисел~$d_k$ при $0\hm \leq k \hm\leq K$ 
и~диагональную мат\-ри\-цу 
$$
D = \mathrm{diag}\left(d_0, d_1, d_2, \dots, d_K \right),
$$ 
где $K \hm= 2N\hm+1$. Обозначим 
$$
\|\mathbf{z}(t)\|_{1D} = \|D\mathbf{z}(t)\|_1.
$$

\smallskip

\noindent
\textbf{Теорема~1.}
%\label{t1} 
\textit{Пусть существует по\-сле\-до\-ва\-тель\-ность положительных 
чисел~$d_k$, $0 \hm\leq k \hm\leq K$, и~функция~$c(t)$ при $t \hm> 0$, такая что $c(t) 
\hm\le \zeta_1(t)$. Тогда процесс, опи\-сы\-ва\-ющий чис\-ло требований в~сис\-те\-ме, слабо  
эргодичен и~справедливо сле\-ду\-ющее не\-ра\-вен\-ство}:
    $$
    %\label{7}
    \|\mathbf{p}^*(t) - \mathbf{p}^{**}(t)\| \leq  \fr{2\max\nolimits_k d_k}{\min\nolimits_k d_k}\,e^{-\int\nolimits_0^t\beta_{*}(\tau)\, d\tau}
    $$  
\textit{для любых начальных условий} $ \mathbf{p}^{*}(0), \mathbf{p}^{**}(0)$ и~$\beta_{*}(t) 
\hm= \min_k a_k(t)$, \textit{если} $a_k(t) \hm> 0$,  $0 \hm\leq k \hm\leq K$.

\smallskip      


\noindent
Д\,о\,к\,а\,з\,а\,т\,е\,л\,ь\,с\,т\,в\,о\,.\ \
Положим
\begin{multline*}
    \gamma\left(W(t)\right)_{1\textsf{D}} = \gamma \left({D} 
W(t){D}^{-1}\right)  ={}\\
{}= \min\limits_i \left(|w_{ii}(t)| + \sum\limits_{j\neq i}\fr{d_j}{d_i}\left\vert w_{ji}(t)\right\vert \right) =  - \beta_{*}(t),
    %\label{f10}
\end{multline*}
где матрицу $DW(t)D^{-1}$ мож\-но записать в~сле\-ду\-ющем \mbox{виде}:

\end{multicols}

\noindent
{ %\small
\begin{multline*}
            DW(t)D^{-1} ={}\\[10.0pt]
            {}=\left(
            \begin{matrix}
                -N\lambda(t) - \eta_1(t)-c(t)   & \fr{d_0}{d_1}(\zeta_1(t)-c(t))           &  -\fr{d_0}{d_2}\,c(t) & -\fr{d_0}{d_3}\,c(t)\\[10.0pt]
\fr{d_1}{d_0}\,\eta_1(t)               & -N\lambda(t)-\zeta_1(t) &  0                    & \fr{d_1}{d_3}\,\mu(t)\\[10.0pt]
 \fr{d_2}{d_0}N\,\lambda(t)             & 0                 &  -\lambda(t)(N-1)-\eta_2(t) & \fr{d_2}{d_3}\,\zeta_2(t)\\[10.0pt]
0   & \fr{d_3}{d_1}\,N\lambda(t)  &\fr{d_3}{d_2}\,\eta_2(t)  & -\mu(t)-\lambda(t)(N-1)-\zeta_2(t)\\[10.0pt]
0   & 0    &  \fr{d_4}{d_2}\,\lambda(t)(N-1)  & 0\\ [10.0pt]
0   & 0    &  0  & \fr{d_5}{d_3}\,\lambda(t)(N-1)\\[10.0pt]
0   & 0    &  0  & 0\\[10.0pt]
0   & 0    &  0  & 0\\[10.0pt]
0   & 0    &  0  & 0\\[10.0pt]
\vdots     & \vdots            &  \vdots  &  \vdots \\[10.0pt]
0   & 0    & 0   &0\\[10.0pt]
            \end{matrix}\right.\\[10.0pt]
\left.                        \begin{matrix}
             -\fr{d_0}{d_4}\,c(t)&-\fr{d_0}{d_5}\,c(t)&\cdots  & -\fr{d_0}{d_{K-1}}\,c(t)& -\fr{d_0}{d_{K}}\,c(t)  \\[10.0pt]
              0&0&\cdots  & 0& 0  \\[10.0pt]
               0&0&\cdots  & 0& 0  \\[10.0pt]
                0&\fr{d_3}{d_5}\,\mu(t)&\cdots  & 0& 0  \\[10.0pt]
                 -\lambda(t)(N-2)-\eta_2(t)&\fr{d_4}{d_5}\zeta_2(t)&\cdots  & 0& 0  \\[10.0pt]
                  \fr{d_5}{d_4}\,\eta_2(t)&-\mu(t)-\lambda(t)(N-2)-\zeta_2(t)&\cdots  & 0& 0  \\[10.0pt]
                   \fr{d_6}{d_4}\,\lambda(t)(N-2)&0&\cdots  & 0& 0  \\[10.0pt]
                    0&\fr{d_7}{d_5}\,\lambda(t)(N-2)&\cdots  & 0& 0  \\[10.0pt]
                     0&0&\cdots  & 0& \fr{d_{K-2}}{d_K}\,\mu  \\[10.0pt]
                      \vdots &\vdots & \ddots  & -\eta_2& \fr{d_{K-1}}{d_K}\,\zeta_2(t)  \\[10.0pt]
                      0&  0  & \cdots  & \fr{d_{K}}{d_{K-1}}\,\eta_2& -\mu-\zeta_2(t)  
                      \end{matrix}\right);
                 \end{multline*}
    }
    
    \begin{multicols}{2}

\noindent
$\beta_{*}(t) = \min \alpha_{k}$, где

\vspace*{-10pt}

\noindent
\begin{multline}
    \alpha_{k}\left( t\right) ={}\\
    \!\!{}=\!
    \begin{cases}
        \left(1\!-\!\fr{d_{2}}{d_{0}}\right) N\lambda(t) \!+\! \left(1\!-\!\fr{d_{1}}{d_{0}}\right)\eta_1(t)\! +\! c(t), &\\
        & \hspace*{-20mm}k = 0;\\
                \left(1\!-\!\fr{d_{3}}{d_{1}}\right) N\lambda(t)\! +\! \left(1\!-\!\fr{d_{0}}{d_{1}}\right)\zeta_1(t) \!+ \!c(t), &\\
                & \hspace*{-20mm}k = 1;\\
                \left(1\!-\!\fr{d_{k+2}}{d_{k}}\right)\left(N \!-\! \fr{k}{2}\right)\lambda(t) \!+{}&\\[9pt]
\hspace*{10mm}{}+\!\left(1\!-\!\fr{d_{k+1}}{d_{k}}\right)\eta_2(t)\!-\! c(t)\fr{d_{0}}{d_{k}}, &\\
& \hspace*{-50mm}k\ \mbox{---\ четное},\ 2 \leq k \leq 2N;\\[3pt]
                \left(1\!-\!\fr{d_{k+2}}{d_{k}}\right)\left(N \!-\! \fr{k-1}{2}\right)\lambda(t)\! +{}&\\[9pt]
\hspace*{4mm}{}+\!\left(1\!-\!\fr{d_{k-2}}{d_{k}}\right)\mu(t)\! +\! \left(1\!-\!\fr{d_{k-1}}{d_{k}}\right)\zeta_2(t) \!-{}&\\[3pt]
{}-\! c(t)\fr{d_{0}}{d_{k}}, &  \hspace*{-53mm}k\ \mbox{--- нечетное},\ 3 \leq k \leq 2N+1.
    \end{cases}\!\!\!\!
\label{13}
\end{multline}

\vspace*{-4pt}

Выберем положительные чис\-ла $a$ и~$b$ такие, что $a \hm> b$, 
и~по\-сле\-до\-ва\-тель\-ность~$d_k$ при $0 \hm\leq k \hm\leq K$ сле\-ду\-ющим об\-ра\-зом:

\noindent
\begin{equation*}
    d_{k} =
    \begin{cases}
        1, & k = 0;\\
        \fr{k}{2} + a, & k\ \mbox{---\ четное},\ 2 \leq k \leq 2N,\ a > 0;\\
       \fr{k-1}{2} + b, & k\ \mbox{---\ нечетное},\ 1 \leq k \leq 2N+1,\\
       &\hspace*{40mm} b > 0.
    \end{cases}
\end{equation*}

\vspace*{-3pt}

Подставляем данную по\-сле\-до\-ва\-тель\-ность в~(\ref{13}) и~по\-лу\-чаем:

\vspace*{-6pt}

\noindent
\begin{multline}
    \alpha_{k}\left( t\right) ={}\\
    \hspace*{-1.5mm}{}=\!
    \begin{cases}
        \left(1-(1 + a)\right) N\lambda(t) + \left(1-b\right)\eta_1(t) +c(t), &\\
        &\hspace*{-15mm} k = 0;\\
        \left(1\!-\!\fr{1+b}{b}\right) N\lambda(t) \!+\! \left(1\!-\!\fr{1}{b}\right)\zeta_1(t) \!+\! c(t), &\\
        &\hspace*{-15mm} k = 1;\\
       \left(1-\fr{k+2a+2}{k+2a}\right)\left(N - \fr{k}{2}\right)\lambda(t)
        +{}&\\[7pt]
       \hspace*{5mm}{}+\left(1-\fr{k+2b}{k+2a}\right)\eta_2(t)- c(t)\fr{2}{k+2a}, &\\
       & \hspace*{-45mm}k\ \mbox{---\ четное},\ 2 \leq k \leq 2N;\\[2pt]
     \left(1-\fr{k+2b+1}{k+2b-1}\right)\left(N - \fr{k-1}{2}\right)\lambda(t) +{}&\\[7pt]
     \hspace*{15mm}{}+\left(1-\fr{k+2b-3}{k+2b-1}\right)\mu(t) + {}&\\[7pt]
{}+\!\left(1\!-\!\fr{k+2a-1}{k+2b-1}\right)\zeta_2(t) \!-\! c(t)\fr{2}{k+2b-1}, &\\[2pt]
         & \hspace*{-50mm}k\ \mbox{---\ нечетное},\ 3 \leq k \leq 2N+1.
    \end{cases}\!\!\!
\label{14}
\end{multline}


Заметим, что $\|D\| = \max_k d_k$;  $\|D^{-1}\| \hm= {1}/(\min_k d_k)$. Таким 
образом, по\-лу\-чаем

\columnbreak

\noindent
\begin{gather*}
%\label{101}
\|\mathbf{z}(t)\|_{1D} \le \max_k d_k \|\mathbf{z}(t)\|_1; \\
\|\mathbf{z}(t)\|_{1} \le \fr{1}{\min\nolimits_k d_k}\, \|\mathbf{z}(t)\|_{1D};
%\end{align*}
%и
\\
\|\mathbf{z}^*(t) - \mathbf{z}^{**}(t)\|_{1D} \le  e^{-\int\nolimits_0^t\beta_{*}(\tau) \,d\tau}
\|\mathbf{z}^*(0) - \mathbf{z}^{**}(0)\|_{1D},
\end{gather*}

\vspace*{-2pt}

\noindent
откуда и~следует утверж\-де\-ние тео\-ремы.

\vspace*{-9pt}

\section{Численный пример}

\vspace*{-1pt}

В качестве иллюстрации полученных результатов рас\-смот\-рим получение 
оцен\-ки ско\-рости схо\-ди\-мости, поз\-во\-ля\-ющей про\-вес\-ти по\-стро\-ение предельных 
вероятностных характеристик в~случае \mbox{периодических} интенсивностей (с~пе\-ри\-о\-дом~$1$).

В данном примере будем полагать, что $a \hm> b$ и~$a \hm= 4$, $b \hm= 3$, $N \hm= 25$, $K\hm=51$, 
а~интен\-сив\-ности имеют вид:

\vspace*{-8pt}

\noindent
\begin{align*}
\lambda(t) &= 0{,}3 + 0{,}1\sin(2\pi t);
\\
\mu(t) &= 850 + 4{,}5\cos(2\pi t);
\\
\zeta_1(t) &= 75 + 0{,}1\cos(2\pi t);
\\
\zeta_2(t) &= 1 + 0{,}1\cos(2\pi t);
\\
\eta_1(t) &= 6 + \sin(2\pi t);
\\
\eta_2(t) &= 750 + \cos(2\pi t);
\\
c(t) &=\zeta_1(t).
\end{align*}

\vspace*{-4pt}

Последовательность~$d_k$ выберем сле\-ду\-ющим обра\-зом:

\begin{figure*}[b] %fig1
 \vspace*{1pt}
\begin{center}
   \mbox{%
\epsfxsize=162mm 
\epsfbox{zei-1.eps}
}
\end{center}
\vspace*{-9pt}
    \Caption{Вероятность пус\-той СМО для $ t \hm\in 
[0,4]$~(\textit{а})  и~$ t \hm\in 
[4,5]$~(\textit{б}): \textit{1}~--- $\chi(0) \hm= 0$; \textit{2}~---$\chi(0) \hm= 52$}
\label{fig1}
%\end{figure*}
%\begin{figure*}[b] %fig2
 \vspace*{9pt}
\begin{center}
   \mbox{%
\epsfxsize=162.346mm 
\epsfbox{zei-3.eps}
}
\end{center}
\vspace*{-9pt}
    \Caption{Вероятность $p_{1}(t)$ для $ t \hm\in [0,4]$~(\textit{а}) и~$ t \hm\in [4,5]$~(\textit{б}):
\textit{1}~---    $\chi(0) \hm= 0$; \textit{2}~--- $\chi(0) \hm= 52$}
    \label{fig3} 
\end{figure*}

\vspace*{1pt}

\noindent
\begin{equation*}
    d_{k}\left( t\right) \!=\!
    \begin{cases}
        1, & \!\!k = 0;\\
        \fr{k}{2} + 4, & \!\!k\ \mbox{---\ четное},\ 2 \leq k \leq 2N\,;\\
        \fr{k-1}{2} + 3, & \!\!k\ \mbox{---\ нечетное},\ 1 \leq k \leq 2N\!+\!1\,.\hspace*{-2.9128pt}
    \end{cases}
\end{equation*}

\vspace*{-3pt}

\noindent
    Теперь подставим по\-сле\-до\-ва\-тель\-ность~$d_k$, чис\-ла~$N$, $a$, $b$ и~заданные 
функции интенсивностей в~(\ref{14}) и~по\-лу\-чим:

\vspace*{-9pt}

\noindent
    \begin{multline*}
        \alpha_{k}\left( t\right) ={}\\[-2pt]
        {}=\!
        \begin{cases}
            -4 N\lambda(t) -2\eta_1(t) + c(t), & \hspace*{-18mm}k = 0;\\ %[3pt]
            \left(1-\fr{4}{3}\right) N\lambda(t) + \left(1-\fr{1}{3}\right)\zeta_1(t) + c(t), &\\[-3pt]
            & \hspace*{-18mm}k = 1;\\[-5pt]
            \left(1-\fr{k+10}{k+8}\right)\left(N - \fr{k}{2}\right)\lambda(t) +{}&\\%[2pt]
            \hspace*{5mm}{}+\left(1-\fr{k+6}{k+8}\right)\eta_2(t)- c(t)\fr{2}{k+8}, &\\
            &\hspace*{-45mm}k\ \mbox{---\ четное},\ 2 \leq k \leq 2N;\\ %[1pt]
            \left(1-\fr{k+7}{k+5}\right)\left(N - \fr{k-1}{2}\right)\lambda(t) +{}&\\%[2pt]
            {}+\left(1-\fr{k+3}{k+5}\right)\mu(t) + \left(1-\fr{k+7}{k+5}\right)\zeta_2(t) -{}&\\%[2pt]
            \hspace*{3mm}{}- c(t)\fr{2}{k+5}, & \hspace*{-45mm}k\ \mbox{---\ нечетное},\\[-1pt]
            & \hspace*{-31mm}3  \leq k \leq 2N+1,
        \end{cases}
        \!\!\!\!\!\geq\hspace*{-4.63823pt}
        \end{multline*}
        
        \noindent
        \begin{multline*} 
        \!\!\!\!\!{}\geq 
            \begin{cases}
            -4 N\lambda(t) -2\eta_1(t) + c(t), & \hspace*{-20mm}k = 0;\\
            -\fr{1}{3}\, N\lambda(t) + \fr{2}{3}\,\zeta_1(t) + c(t), & \hspace*{-20mm}k = 1;\\
           \fr{1}{5}\left(N - 1\right)\lambda(t) +\left(1-\fr{2N+6}{2N+8}\right)\eta_2(t)-{}&\\
\hspace*{10mm}{}- \fr{1}{5}\,c(t), & \hspace*{-35mm}k\ \mbox{---\ четное},\ 2 \leq k \leq 2N;\\
             \fr{1}{4}\left(N - 1\right)\lambda(t) +\left(1-\fr{2N+4}{2N+6}\right)\mu(t) -{}&\\
            \hspace*{5mm}{}- \fr{1}{4}\,\zeta_2(t) -              \fr{1}{4}\,c(t), & \hspace*{-25mm}k\ \mbox{---\ нечетное},\\
            &\hspace*{-22mm} 3 \leq k \leq 2N+1.
        \end{cases}
    \end{multline*}



Таким образом, окончательно по\-лу\-чаем:
\begin{align*}
    \alpha_k\left( t\right)& =
    \begin{cases}
        20{,}99, & \!\!k = 0;\\
        121{,}65, & \!\!k = 1;\\
        8{,}941, & \!\!k\ \mbox{---\ четное},\ 2 \leq k \leq 2N\,;\\
        8{,}947, & \!\!k\ \mbox{---\ нечетное},\ 2 \leq k \leq 2N+1\,;
    \end{cases}
\\
\beta_*(t) &= \min\limits_k(\alpha_k) = 8{,}941.
\end{align*}





Из теоремы~1 следует оцен\-ка скорости схо\-ди\-мости:
\begin{equation}
\left \|\mathbf{p}^*(t) - \mathbf{p}^{**}(t)\right\| \leq  58e^{-8{,}941t}.
\label{15}
\end{equation}

Далее, применяя метод Рун\-ге--Кут\-та чет\-вер\-то\-го порядка, чис\-лен\-но 
решаем прямую сис\-те\-му Колмогорова, опи\-сы\-ва\-ющую данный марковский процесс, 
с~заданными функциями интенсивностей и~стро\-им сле\-ду\-ющие гра\-фики:
\begin{itemize}
    \item на рис.~\ref{fig1},\,\textit{а} пред\-став\-лен график ве\-ро\-ят\-ности пус\-той СМО при 
$\chi(0) \hm= 0$~(\textit{1}) и~$52$~(\textit{2}) и~$t \hm\in 
[0,4]$. Заметим, что кривые на графике сходятся достаточно \mbox{быстро};
    \item на рис.~1,\,\textit{б} про\-ил\-люст\-ри\-ро\-ва\-но поведение кривых (вероятности 
пус\-той СМО) на $t \hm\in [4,5]$;
    \item на рис.~\ref{fig3},\,\textit{а} пред\-став\-лен график ве\-ро\-ят\-ности $p_1(t)$ при  
$\chi(0) \hm= 0$~(\textit{1}) и~$52$~(\textit{2}) и~$t \hm\in 
[0,4]$. Так\-же заметим, что кривые на графике сходятся достаточно \mbox{быстро};
    \item на рис.~\ref{fig3},\,\textit{б} про\-ил\-люст\-ри\-ро\-ва\-но поведение кривых (вероятности~$p_1(t)$) на $t \hm\in [4,5]$.
\end{itemize}


\vspace*{-7pt}


\section{Заключение}

\vspace*{-1pt}

В данной работе сформулирована и~доказана тео\-ре\-ма, поз\-во\-ля\-ющая получать оценки 
ско\-рости схо\-ди\-мости к~предельному режиму неоднородного обобщенного 
ПРГ. Для исследования данной модели методы треугольного 
и~диагональных преобразований не применимы, а~метод дифференциальных неравенств 
трудоемок, поэтому был применен метод, основанный на понятии логарифмической 
нор\-мы операторной функции. Так\-же был рас\-смот\-рен чис\-лен\-ный пример, в~котором 
была получена оценка ско\-рости схо\-ди\-мо\-сти и~построены со\-от\-вет\-ст\-ву\-ющие графики. 
Точ\-ность оцен\-ки зависит от выбора по\-сле\-до\-ва\-тель\-ности~$d_k$. Данным методом, по-ви\-ди\-мо\-му, 
мож\-но получать оцен\-ки предельных характеристик для моделей $M/\mathrm{PH}/c$ 
или $\mathrm{PH}/\mathrm{PH}/c$, что  может стать пред\-ме\-том дальнейших исследований.


{\small\frenchspacing
 { %\baselineskip=12pt
 %\addcontentsline{toc}{section}{References}
 \begin{thebibliography}{99}   
\bibitem{ILS2010} 
\Au{Irvani~S.\,M.\,R., Luangkesorn~K.\,L., Simchi-Levi~D.} 
On assemble to order systems with flexible customers~// IIE Trans., 2010. Vol.~35. P.~389--403.
doi: 10.1080/ 07408170390184107.

%===============2=======================
\bibitem{FY2002}
\Au{Fadiloglu M.\,M., Yeralan~S.} Models of production lines 
as quasi-birth-death processes~// Math. Comput. Model., 2002. Vol.~35. P.~913--930.
doi: 10.1016/S0895-7177(02)00059-6.

%===============3=======================
\bibitem{KL1999} 
\Au{Kim Y.\,Y., Li~S}. Performance evaluation of packet data 
services over cellular voice networks~// Wirel. Netw., 1999. Vol.~5. P. 211--219. doi: 10.1023/A:1019198927563.


%===============4=======================    
\bibitem{W1994} 
\Au{Wang K.} Profit analysis of the $M/M/R$ machine repair 
problem with spares and server breakdowns~// J.~Oper. Res. Soc., 1994. Vol.~45.  P.~539--548.
doi: 10.1057/jors.1994.81.



%===============5=======================
\bibitem{JG1997} 
\Au{Jain M., Ghimire~R.\,P.} Machine repair queueing system 
with non-reliable server and heterogeneous service discipline~// J.~M.A.C.T., 1997. Vol.~30. P.~105--115.

%===============6=======================
\bibitem{JM2003} 
\Au{Jain M., Maheshwari S.} Transient analysis of a redundant 
system with additional repairmen~// Am. J. Math.~--- S., 2003. Vol.~23. 
P.~347--382. doi: 10.1080/ 01966324.2003.10737618.

%===============7=======================
\bibitem{JM2004} 
\Au{Jain M., Kulshrestha~R., Maheshwari~S.} N-policy for a machine repair 
system with spares and reneging~// Appl. Math. Model., 2004. Vol.~28. P.~513--531.
doi: 10.1016/ j.apm.2003.10.013.

%===============8=======================
\bibitem{KW2012} 
\Au{Ke J.\,C., Wu~C.\,H.} Multi-server machine repair model 
with standbys and synchronous multiple vacation~// Comput. Ind. Eng., 2012. Vol.~62. P.~296--305.
doi: 10.1016/ j.cie.2011.09.017.

\bibitem{DN2015} %9
\Au{Дудин А.\,Н., Назаров~А.\,А.} Сис\-те\-ма обслуживания $\mathrm{MMAP}/M/R/0$ 
с~резервированием приборов, функционирующая в~случайной среде~// Проблемы передачи 
информации, 2015. Т.~51. Вып.~3. С.~93--104.



\bibitem{VD2017} %10
\Au{Вишневский В.\,М., Дудин~А.\,Н.} Сис\-те\-мы массового 
обслуживания с~коррелированными входными потоками и~их применение для 
моделирования телекоммуникационных сетей~// Автоматика и~телемеханика, 2017. Вып.~8. С.~3--59.

\bibitem{VSB2020} %11
\Au{Вишневский~В.\,М., Семенова~О.\,В., Буй~З.\,Т.} Исследование сис\-те\-мы  поллинга
с~адап\-тив\-ным циклическим опро\-сом и~ее применение для \mbox{проектирования}
широкополосных беспроводных сетей~// Проблемы управ\-ле\-ния, 2020. Вып.~5. С.~50--55.
doi: 10.25728/ pu.2020.5.6.



%===============12=======================
\bibitem{WK2014} 
\Au{Wu C.\,H., Ke~J.\,C.} 
Multi-server machine repair 
problems under a $(V,R)$ synchronous single vacation policy~// Appl. Math. Model., 2014. Vol.~38. P.~2180--2189.
doi: 10.1016/j.apm.2013.10.045.

\bibitem{ZSKRKS2020} %13
\Au{Zeifman A., Satin~Y., Kryukova~A., Ra\-zum\-chik~R., Ki\-se\-le\-va~K.,  Shi\-lo\-va~G.} On three methods for bounding the rate of convergence 
for some continuous--time Markov chains~// Int. J. Appl. Math. Comp., 2020. Vol.~30. P.~251--266. doi: 10.34768/amcs-2020-0020.

%===============14=======================
\bibitem{Satin2021}
\Au{Сатин Я.\,А.} Исследование модели типа $M_t/M_t/1 $ с~двумя различными 
классами требований~// Сис\-те\-мы и~средства информатики, 2021. Т.~31. №\,1. С.~17--27.
doi: 10.14357/08696527210102.

%===============15=======================
\bibitem{Razumchik2022}
\Au{Razumchik R., Rumyantsev~A.} Some ergodicity and truncation bounds for a~small scale Markovian supercomputer model~// 
36th ECMS Conference (International) on Modelling and Simulation Proceedings.~--- Saarbrucken-Dudweiler,
Germany: Digitaldruck Pirrot GmbH, 2022. 
P.~324--330.  doi: 10.7148/2022-0324.

%===============16=======================
\bibitem{KSSZ2022}
\Au{Ковалёв И.\,А., Сатин~Я.\,А., Си\-ни\-ци\-на~А.\,В., Зейф\-ман~А.\,И.} Об одном 
подходе к~оцениванию ско\-рости схо\-ди\-мости нестационарных марковских моделей 
сис\-тем обслуживания~// Информатика и~её применения, 2022. Т.~16. №\,3. С.~75---82.   
doi: 10.14357/ 19922264220310. 

%===============17=======================
\bibitem{USZ2023} 
\Au{Usov I., Satin Y., Zeifman~A.} Estimating the rate of 
convergence of the $\mathrm{PH}/M/1$ model by reducing to quasi-birth--death processes~// 
Mathematics, 2023. Vol.~11. No.\,6. Art.~1494. doi: 10.3390/math11061494.

%===============18=======================
\bibitem{L2021} 
\Au{Lv~S.} Multi-machine repairable system with one
unreliable server and variable repair rate~// Mathematics, 2021. Vol.~9. No.\,11. Art.~1299.  
doi: 10.3390/math9111299. 


\bibitem{ZKKS2016} %19
\Au{Зейфман А.\,И., Королев~В.\,Ю.,  Ко\-ро\-ты\-ше\-ва~А.\,В.,  Са\-тин~Я.\,А.} 
Оценки для неоднородных марковских сис\-тем обслуживания с~особенностями в~нуле.~--- М.: ФИЦ ИУ РАН, 2016. 56~с.
\end{thebibliography}

 }
 }

\end{multicols}

\vspace*{-6pt}

\hfill{\small\textit{Поступила в~редакцию 14.04.23}}

%\vspace*{8pt}

%\pagebreak

\newpage

\vspace*{-28pt}

%\hrule

%\vspace*{2pt}

%\hrule



\def\tit{ON THE~RATE OF~CONVERGENCE AND~LIMITING CHARACTERISTICS FOR~ONE QUASI-BIRTH--DEATH PROCESS}


\def\titkol{On the~rate of~convergence and~limiting characteristics for one quasi-birth--death process}


\def\aut{I.\,A.~Usov$^{1}$, Y.\,A.~Satin$^{1}$, and A.\,I.~Zeifman$^{1,2,3,4}$}

\def\autkol{I.\,A.~Usov, Y.\,A.~Satin, and A.\,I.~Zeifman}

\titel{\tit}{\aut}{\autkol}{\titkol}

\vspace*{-10pt}


\noindent
$^{1}$Vologda State University, 15~Lenin Str., Vologda 160000, Russian Federation

\noindent
$^{2}$Moscow Center for Fundamental and Applied Mathematics, M.\,V.~Lomonosov Moscow State University,\linebreak
$\hphantom{^1}$1-52~Leninskie Gory, GSP-1, Moscow 119991, Russian Federation

\noindent
$^{3}$Federal Research Center ``Computer Science and Control'' of the Russian Academy of Sciences, 44-2~Vavilov\linebreak
$\hphantom{^1}$Str., Moscow 119333, Russian Federation

\noindent
$^{4}$Vologda Research Center of the Russian Academy of Sciences, 56A~Gorky Str., Vologda 160014, Russian\linebreak
$\hphantom{^1}$Federation


\def\leftfootline{\small{\textbf{\thepage}
\hfill INFORMATIKA I EE PRIMENENIYA~--- INFORMATICS AND
APPLICATIONS\ \ \ 2023\ \ \ volume~17\ \ \ issue\ 3}
}%
 \def\rightfootline{\small{INFORMATIKA I EE PRIMENENIYA~---
INFORMATICS AND APPLICATIONS\ \ \ 2023\ \ \ volume~17\ \ \ issue\ 3
\hfill \textbf{\thepage}}}

\vspace*{3pt}




\Abste{A queuing system with one server and different repair and failure options is considered, the number of requirements in which is 
described by a~quasi-birth-death process. To reasonably find the limiting probabilistic characteristics of the system, the rate 
of convergence to them is studied (that is, the rate at which the initial conditions of the system are ``forgotten''). 
To study the rate of convergence to the limiting regime, a~recently developed version of the approach based on the concept of the 
logarithmic norm of the operator function corresponding to the estimate of the norm of the Cauchy matrix as well as a~modernized special 
transformation of the forward Kolmogorov system was applied. A~numerical example is considered for which the estimation of the rate of 
convergence is shown in detail as well as the construction of some limiting characteristics of the model based on these estimates.}


\KWE{quasi-birth--death processes; rate of convergence; ergodicity bounds; logarithmic norm; queuing systems}


\DOI{10.14357/19922264230307}{RBKZJI}


% \Ack
 %     \noindent
  


  \begin{multicols}{2}

\renewcommand{\bibname}{\protect\rmfamily References}
%\renewcommand{\bibname}{\large\protect\rm References}

{\small\frenchspacing
 {%\baselineskip=10.8pt
 \addcontentsline{toc}{section}{References}
 \begin{thebibliography}{99} 

\bibitem{ILS2010-1}  %1
\Aue{Irvani, S.\,M.\,R., K.\,L.~Luangkesorn, and D.~Simchi-Levi.} 2003. On assemble
to order systems with flexible customers. \textit{IIE Trans.} 35(5):389--403. doi: 10.1080/ 07408170390184107.
% Published online: 29 Oct 2010. doi: 10.1080/07408170304392.

%===============2=======================
\bibitem{FY2002-1} 
\Aue{Fadiloglu, M.\,M., and S.~Yeralan.}
 2002. Models of production lines as quasi-birth-death processes. \textit{Math. Comput. Model.} 35(7-8):913--930. doi: 10.1016/S0895-7177(02)00059-6.

%===============3=======================
\bibitem{KL1999-1} 
\Aue{Kim, Y.\,Y., and S.~Li.} 1999. Performance evaluation of packet data services over cellular voice networks. 
\textit{Wirel. Netw.} 5:211--219. doi: 10.1023/A:1019198927563.


\bibitem{W1994-1} %4
\Aue{Wang, K.} 1994. Profit analysis of the $M/M/R$ machine repair problem with spares and server breakdowns. 
\textit{J.~Oper. Res. Soc.} 45:539--548. doi: 10.1057/jors.1994.81.





\bibitem{JG1997-1} %5
\Aue{Jain, M., and R.\,P.~Ghimire.} 1997. Machine repair queueing system with non-reliable server and heterogeneous service discipline. 
\textit{J.~M.A.C.T.} 30:105--115.

\bibitem{JM2003-1}  %6
\Aue{Jain, M., and S.~Maheshwari.}
 2003. Transient analysis of a~redundant system with additional repairmen. \textit{Am. J. Math.~--- S.} 23(3-4):347--382. doi: 10.1080/01966324. 2003.10737618.


\bibitem{JM2004-1} %7
\Aue{Jain, M., R.~Kulshrestha, and S.~Maheshwari.} 2004. 
\mbox{N-policy} for a~machine repair system with spares and reneging. \textit{Appl. Math. Model.} 28(6):513--531. doi: 10.1016/j.apm.2003.10.013.

%===============8=======================
\bibitem{KW2012-1} 
\Aue{Ke, J.\,C., and C.\,H.~Wu.} 2012. Multi-server machine repair model with standbys and synchronous multiple vacation. 
\textit{Comput. Ind. Eng.} 62(1):296--305. doi: 10.1016/ j.cie.2011.09.017.

\bibitem{DN2015-1} %9
\Aue{Dudin, A.\,N., and A.\,A.~Nazarov.} 2015. The $\mathrm{MMAP}/M/R/0$ queueing system with reservation of servers operating in a~random environment. 
\textit{Probl. Inf. Transm.} 51(3):289--298. doi: 10.1134/ S0032946015030060.


\bibitem{VD2017-1} %10
\Aue{Vishnevskii, V.\,M., and A.\,N.~Dudin.}
 2017. Queueing systems with correlated arrival flows and their applications to modeling telecommunication networks. 
 \textit{Automat. Rem. Contr.} 78(8):1361--1403. doi: 10.1134/ S000511791708001X.

\bibitem{VSB2020-1} %11
\Aue{Vishnevskiy, V.\,M., O.\,V.~Semenova, and Z.\,T.~Buy.}
 2020. Issledovanie sis\-te\-my pol\-lin\-ga s~adap\-tiv\-nym tsik\-li\-che\-skim op\-ro\-som i~ee 
 pri\-me\-ne\-nie dlya pro\-ek\-ti\-ro\-va\-niya shi\-ro\-ko\-po\-los\-nykh bes\-pro\-vod\-nykh se\-tey 
 [Investigation of the stochastic polling system and its applications in broadband wireless networks]. \textit{Control Sciences} 
 5:50--55. doi: 10.25728/pu.2020.5.6.



%===============12=======================
\bibitem{WK2014-1} 
\Aue{Wu, C.\,H., and J.\,C.~Ke.}
 2014. Multi-server machine repair problems under a~$(V,R)$ synchronous single vacation policy. \textit{Appl. Math. Model.} 38:2180--2189. 
 doi: 10.1016/j.apm.2013.10.045.

\bibitem{ZSKRKS2020-1}  %13
\Aue{Zeifman, A., Y.~Satin, A.~Kryukova, R.~Razumchik, K.~Kiseleva, and G.~Shilova.}
 2020. On the three methods for bounding the rate of convergence for some continuous--time Markov chains. 
 \textit{Int. J. Appl. Math. Comp.} 30(2):251--266. 
 doi: 10.34768/amcs-2020-0020.

%===============14=======================
\bibitem{Satin2021-1} 
\Aue{Satin, Y.\,A.} 2021. Is\-sle\-do\-va\-nie mo\-de\-li ti\-pa $M_t/M_t/1$ s~dvu\-mya raz\-lich\-ny\-mi klas\-sa\-mi tre\-bo\-va\-niy 
[On the bounds of the rate of convergence for $M_t/M_t/1$ model with two different requests]. \textit{Sis\-te\-my i~Sredstva Informatiki~--- 
Systems and Means of Informatics} 31(1):17--27.  doi: 10.14357/08696527210102.

%===============15=======================
\bibitem{Razumchik2022-1}
\Aue{Razumchik, R., and A.~Rumyantsev.}
 2022. Some ergodicity and truncation bounds for a~small scale Markovian supercomputer model. 
 \textit{36th ECMS Conference (International) on Modelling and Simulation Proceedings}.\linebreak Saarbrucken-Dudweiler, Germany: Digitaldruck Pirrot GmbH. 324--330. 
 doi: 10.7148/2022-0324.
 
 

%===============16=======================
\bibitem{KSSZ2022-1} 
\Aue{Kovalev, I.\,A., Y.\,A.~Satin, A.\,V.~Si\-ni\-tci\-na, and A.\,I.~Zeif\-man.}
 2022. Ob od\-nom pod\-kho\-de k~otse\-ni\-va\-niyu sko\-rosti skho\-di\-mosti ne\-sta\-tsi\-o\-nar\-nykh mar\-kov\-skikh mo\-de\-ley sis\-tem 
 ob\-slu\-zhi\-va\-niya [On an approach for estimating the rate of convergence for nonstationary Markov models of queueing systems]. 
 \textit{Informatika i~ee Primeneniya~--- Inform. Appl.} 16(3):75--82.  doi: 10.14357/19922264220310.


%===============17=======================
\bibitem{USZ2023-1} 
\Aue{Usov, I., Y.~Satin, and A.~Zeifman.}
 2023. Estimating the rate of convergence of the $\mathrm{PH}/M/1$ model by reducing to quasi-birth--death processes. 
 \textit{Mathematics} 11(6):1494. doi: 10.3390/math11061494.

%===============18=======================
\bibitem{L2021-1} 
\Aue{Lv, S.} 2021. Multi-machine repairable system with one unreliable server and variable repair rate. 
\textit{Mathematics} 9(11):1299. doi: 10.3390/math9111299.


\bibitem{ZKKS2016-1} %19
\Aue{Zeifman, A.\,I., V.\,Yu.~Korolev, A.\,V.~Ko\-ro\-ty\-she\-va, and Ya.\,A.~Sa\-tin.}
 2016. \textit{Otsen\-ki dlya ne\-od\-no\-rod\-nykh mar\-kov\-skikh sis\-tem ob\-slu\-zhi\-va\-niya s~oso\-ben\-no\-stya\-mi v~nu\-le} 
 [Bounds for inhomogeneous Markov service systems with singularities at zero]. Moscow: FRC CSC RAS. 56~p.
\end{thebibliography}

 }
 }

\end{multicols}

\vspace*{-6pt}

\hfill{\small\textit{Received April 14, 2023}} 

\vspace*{-12pt}


\Contr

\noindent
\textbf{Usov Ilya A.} (b.\ 1996)~--- PhD student, Department
of Applied Mathematics, Vologda State University, 15~Lenin Str., Vologda 160000, Russian Federation; 
\mbox{iusov35@yandex.ru}

\vspace*{3pt}

\noindent
\textbf{Satin Yacov A.} (b.\ 1978)~--- Candidate of Science (PhD) in physics and mathematics, associate professor, Department
of Applied Mathematics, Vologda State University, 15~Lenin Str., Vologda 160000; \mbox{yacovi@mail.ru}

\vspace*{3pt}

\noindent
\textbf{Zeifman Alexander I.} (b.\ 1954)~--- Doctor of Science in physics and mathematics, professor, head of department,
Department
of Applied Mathematics,
Vologda State University, 15~Lenin Str., Vologda 160000, Russian Federation; senior scientist, Institute of
Informatics Problems, Federal Research Center ``Computer Science and Control'' of the Russian Academy of
Sciences, 44-2~Vavilov Str., Moscow 119133, Russian Federation; principal scientist, Vologda Research Center of
the Russian Academy of Sciences, 56A~Gorky Str., Vologda 160014, Russian Federation; senior scientist, Moscow
Center for Fundamental and Applied Mathematics, M.\,V.~Lomonosov Moscow State University, 1-52~Leninskie
Gory, GSP-1, Moscow 119991, Russian Federation; \mbox{a\_zeifman@mail.ru}




\label{end\stat}

\renewcommand{\bibname}{\protect\rm Литература} 