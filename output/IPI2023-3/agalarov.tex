\def\stat{agalarov}

\def\tit{ОПТИМИЗАЦИЯ СХЕМЫ РАСПРЕДЕЛЕНИЯ\\ БУФЕРНОЙ ПАМЯТИ УЗЛА ПАКЕТНОЙ 
КОММУТАЦИИ}

\def\titkol{Оптимизация схемы распределения буферной памяти узла пакетной 
коммутации}

\def\aut{Я.\,М.~Агаларов$^1$}

\def\autkol{Я.\,М.~Агаларов}

\titel{\tit}{\aut}{\autkol}{\titkol}

\index{Агаларов Я.\,М.}
\index{Agalarov Ya.\,M.}


%{\renewcommand{\thefootnote}{\fnsymbol{footnote}} \footnotetext[1]
%{Работа выполнялась с~использованием инфраструктуры Центра коллективного пользования 
%<<Высокопроизводительные вычисления и~большие данные>> (ЦКП <<Информатика>>) ФИЦ 
%ИУ РАН (г.~Моск\-ва).}}


\renewcommand{\thefootnote}{\arabic{footnote}}
\footnotetext[1]{Федеральный исследовательский центр <<Информатика и~управление>> Российской академии наук, 
\mbox{agglar@yandex.ru}}

\vspace*{-10pt}
 
\Abst{Рассматривается буфер узла коммутации (УК) пакетов, совместно ис\-поль\-зу\-емый 
несколькими выходными линиями связи. Совместное использование буферной памяти (БП)
несколькими пользователями поз\-во\-ля\-ет уменьшить объем памяти, необходимый для 
удовле\-тво\-ре\-ния требований к~задержке и~ве\-ро\-ят\-ности потерь пакетов. Однако возникает 
проб\-ле\-ма распределения БП между пользователями, поскольку отдельные 
пользователи, заняв всю память, могут ограничить (или закрыть) доступ к~линиям связи 
другим пользователям, что может значительно снизить про\-из\-во\-ди\-тель\-ность УК
в~целом. Существует множество различных схем распределения БП, одна из 
которых, на\-зы\-ва\-емая SMA (Sharing with Minimum Allocation), исследуется в~данной работе 
с~целью снижения за\-трат, связанных с~отклонением и~за\-держ\-кой пакетов и~эксплуатацией 
накопителя и~линий связи. В~качестве модели УК используется 
многопотоковая сис\-те\-ма массового обслуживания (СМО) с~параллельными приборами типа 
$M/M/s/K$ с~совместно ис\-поль\-зу\-емым по схеме SMA буфером с~фиксированным чис\-лом 
зарезервированных за каж\-дым прибором мест хранения. Сформулирована математическая 
по\-ста\-нов\-ка задачи оптимизации схемы SMA по объему общедоступных мест буфера с~\mbox{целью} 
минимизации потерь сис\-те\-мы, воз\-ни\-ка\-ющих из-за отклонения заявок, за\-держ\-ки заявок 
в~очереди и~эксплуатации буфера и~приборов. Доказана тео\-ре\-ма о~границах об\-ласти, 
содержащей точ\-ку глобального оптимума. Приведен так\-же ряд вы\-те\-ка\-ющих из тео\-ре\-мы 
утверждений о~точ\-ке глобального оптимума целевой функции для других моделей 
УК и~част\-ных случаев SMA.} 

\KW{узел коммутации; распределение буферной памяти; оптимизация; сис\-те\-ма массового 
обслуживания}

\DOI{10.14357/19922264230306}{QLXCKV}
  
\vspace*{-6pt}


\vskip 10pt plus 9pt minus 6pt

\thispagestyle{headings}

\begin{multicols}{2}

\label{st\stat}

  \section{Введение }
  
  \vspace*{-4pt}
  
  В сетях связи с~коммутацией пакетов УК имеют 
БП, в~которой пакеты, по\-сту\-па\-ющие со всех входных линий 
связи, сохраняются в~ожидании, преж\-де чем передаются на со\-от\-вет\-ст\-ву\-ющие 
выходы. Качество обслуживания в~таких сетях в~значительной степени зависит 
от объема памяти УК. Возникает эта за\-ви\-си\-мость из-за того, что при 
ограниченной про\-пуск\-ной спо\-соб\-ности каналов связи и~небольшом объеме 
памяти могут быть недопустимо большие потери пакетов или рост чис\-ла 
по\-втор\-ных передач пакетов и,~как следствие, недопустимо силь\-ное снижение 
про\-из\-во\-ди\-тель\-ности сети, а~при большом объеме памяти воз\-мож\-ны длинные 
очереди и,~как следствие, не\-до\-пус\-ти\-мо большие за\-держ\-ки пакетов в~УК. 
Совместное использование общей памяти в~такой многопользовательской 
сис\-те\-ме может значительно повысить про\-из\-во\-ди\-тель\-ность УК, но со\-зда\-ет 
новую проб\-ле\-му~--- требует управ\-ле\-ние БП~[1]. Отсюда и~возникает задача 
выбора оптимального объема и~схемы распределения БП УК с~целью 
выполнения определенных требований к~качеству обслуживания пакетов 
и~за\-тра\-там, связанным со сто\-и\-мостью оборудования и~технической 
эксплуатации УК~[1].
  
  Сразу же после появления сетей связи с~КП был опуб\-ли\-ко\-ван ряд работ, 
в~которых рас\-смат\-ри\-ва\-лась проб\-ле\-ма совместного использования БП.\linebreak 
Например, в~[2] была предложена схема совместного использования БП, 
которая использует пороговое правило, близ\-кое к~оптимальному, 
а~в~работе~[3] было исследовано несколько схем \mbox{совместного} использования, 
а~именно: CS (Complete Sharing), при которой по\-сту\-па\-ющий пакет 
принимается, если до\-ступ\-но ка\-кое-ли\-бо место для хранения; CP (Complete 
Partitioning), при которой все хранилище по\-сто\-ян\-но распределяется между 
выходными линиями; SMQ (Sharing with Maximum Allocation), при которой 
ограничивается чис\-ло мест хранения, выделенных каж\-дой выходной линии; 
SMA, при которой для каж\-дой выходной 
линии всегда зарезервировано минимальное чис\-ло мест хранения, а~остальные 
одинаково до\-ступ\-ны всем выходным линиям; SMQMA (Sharing with Maximum 
Queue Length and Minimum Allocation), которая пред\-став\-ля\-ет собой 
комбинацию схем SMQ и~SMA.
  
  Наряду с~выбором эффективной схемы важ\-но еще оптимизировать 
па\-ра\-мет\-ры этой схемы, в~част\-ности общий объ\-ем памяти, что, за исключением 
единичных част\-ных случаев (ниже на некоторые из них сделаны ссылки), 
остается задачей открытой (в~смыс\-ле ее точ\-но\-го решения) из-за ее слож\-ности. 
Как правило, ра\-зум\-но ожидать рос\-та про\-пуск\-ной спо\-соб\-ности (здесь и~ниже 
про\-пуск\-ная спо\-соб\-ность определяется как ин\-тен\-сив\-ность потока пакетов на 
выходе сис\-те\-мы) с~увеличением ем\-кости буфера. Это проверено для многих 
стандартных СМО, таких как очередь $M/M/s/K$ 
(см., например,~[4]), и~в~[5] показано, что результат спра\-вед\-лив и~при более 
общих условиях. Однако в~целом результат неверен (см., например,~[6]). 
  
  Интерес к~управлению БП исследователи проявляют и~в~на\-сто\-ящее время, 
что вид\-но по множеству научных статьей в~отечественных и~за\-ру\-беж\-ных 
на\-уч\-ных журналах, опуб\-ли\-ко\-ван\-ных за по\-след\-нее время (см., например, [7--17]). 
В~основном они по\-свя\-ще\-ны анализу раз\-лич\-ных схем рас\-пре\-де\-ле\-ния БП, 
однако работ, в~которых получены точ\-ные решения по оптимизации схем, 
в~литературе встречается мало. Ниже приводится об\-зор некоторых из них, 
в~которых проводились исследования за\-ви\-си\-мости показателей качества работы 
УК от типа схемы рас\-пре\-де\-ле\-ния БП и~значений ее па\-ра\-мет\-ров. В~\cite{7-ag} 
авторы рас\-смат\-ри\-ва\-ют очередь $M/G/1/K\mbox{-}\mathrm{PS}$, выводят выражения для 
раз\-лич\-ных показателей про\-из\-во\-ди\-тель\-ности, связанных с~про\-пуск\-ной 
спо\-соб\-ностью и~сред\-ней ско\-ростью передачи и~приводят несколько 
структурных результатов о~взаимосвязях меж\-ду этими показателями. Показано, 
что изменения ем\-кости буфера оказывают наиболее значительное влияние, 
когда на\-груз\-ка на сис\-те\-му не слиш\-ком низ\-кая и~не слиш\-ком высокая. Для 
систем $M/M/1/K$ и~$M/G/1/K\mbox{-}\mathrm{PS}$ показано, что увеличение 
про\-пуск\-ной спо\-соб\-ности (как номинальное, так и~относительное), которое было 
бы получено за счет до\-бав\-ле\-ния дополнительного буферного пространства, 
унимодально по на\-груз\-ке. В~работе~\cite{8-ag} предложен эвристический 
алгоритм решения задачи совместного управ\-ле\-ния объемом БП
и~ско\-ростью передачи линий связи, который рас\-смат\-ри\-ва\-ет\-ся в~сле\-ду\-ющих 
аспектах: 
{\looseness=1

}

\noindent
\begin{enumerate}[(1)]
\item при заданной степени за\-груз\-ки каналов и~заданной ско\-рости 
передачи найти необходимый объем БП УК, 
обес\-пе\-чи\-ва\-ющий заданную сред\-нюю за\-держ\-ку в~сети и~требуемую ве\-ро\-ят\-ность 
потери ячеек вследствие переполнения буферов; 
\item при заданной степени 
за\-груз\-ки каналов и~известных объ\-емах БП найти необходимую 
ско\-рость передачи, обес\-пе\-чи\-ва\-ющую заданную сред\-нюю задержку в~сети 
и~тре\-бу\-емую ве\-ро\-ят\-ность потери ячеек вследствие переполнения 
буфера.
\end{enumerate}
 
 В~работе~\cite{9-ag} авторы решили задачу выбора объема БП для 
СМО типа $M/M/1/K$, сформулировав ее как задачу нелинейного 
программирования:
\begin{enumerate}[(1)]
\item для фиксированной вход\-ной на\-груз\-ки найти объем 
накопителя (БП), при котором сред\-няя за\-держ\-ка достигает минимума, 
а~ин\-тен\-сив\-ность потерь не превышает заданной величины; 
\item для 
фиксированного объема накопителя (БП) найти значение входной на\-груз\-ки, 
при которой сред\-няя задержка достигает минимума, а~ин\-тен\-сив\-ность 
потерь не превышает заданной величины.
\end{enumerate}

 В~работе~\cite{10-ag} 
в~предположении пуассоновских входящих потоков, экспоненциального времени 
обслуживания и~одноканальных линий передачи рас\-смот\-ре\-но динамическое 
распределение (управ\-ле\-ние) БП и~получен для случая трех каналов вид 
до\-пус\-ти\-мо\-го пространства со\-сто\-яний, со\-от\-вет\-ст\-ву\-ющий оптимальному 
решению. Схема динамического совместного использования БП, которая 
поз\-во\-ля\-ет каж\-дой очереди увеличиваться до динамически на\-зна\-ча\-емо\-го порога, 
предложена в~работе~\cite{11-ag}. Этот порог вы\-чис\-ля\-ет\-ся как произведение 
остав\-ше\-го\-ся буфера с~предопределенным па\-ра\-мет\-ром. В~работе~\cite{12-ag} 
пред\-ло\-же\-на схема (названная drop-on-demand, или сокращенно DoD) 
совместного использования, которая до\-пус\-ка\-ет отбрасывание принятых 
пакетов и,~следовательно, не относится к~классу упомянутых выше схем. 
Согласно этой схеме по\-сту\-па\-ющий пакет всегда принимается, если имеется 
пус\-той буфер. Если по\-сту\-па\-ет пакет, пред\-на\-зна\-чен\-ный для выходной 
линии~$i$, и~обнаруживается, что буфер заполнен, а~вы\-ход\-ная линия~$I$ 
содержит больше пакетов в~общей памяти, чем любые другие пор\-ты, 
выполняется сле\-ду\-ющее действие: 
\begin{itemize}
\item если $i\hm=I$, то по\-сту\-па\-ющий пакет 
отбрасывается; 
\item если $i\not= I$, то по\-сту\-па\-ющий пакет принимается в~буфер 
и~один пакет к~линии~$I$ отбрасывается.
\end{itemize}
 Были приведены чис\-лен\-ные 
примеры, по\-ка\-зы\-ва\-ющие, что схема DoD обеспечивает луч\-шую пропускную 
спо\-соб\-ность, чем схемы CS и~CP. В~работе~\cite{13-ag} на примере 
двухканальной СМО показано, что эта политика оптимальна только для 
сим\-мет\-рич\-ных сис\-тем. Там же утверж\-да\-ет\-ся, что динамические схемы 
рас\-пре\-де\-ле\-ния БП труд\-но ре\-а\-ли\-зу\-емы по срав\-не\-нию со статическими. Для 
СМО с~конечной совместно ис\-поль\-зу\-емой БП и~несколькими очередями 
пакетов в~\cite{14-ag} рас\-смот\-ре\-на схема с~динамическими индивидуальными 
потолками, зависящими от чис\-ла свободных мест в~БП (названная FAB~--- Flow 
Aware Buffer). Приведены результаты чис\-лен\-но\-го срав\-ни\-тель\-но\-го анализа схем 
пол\-но\-го разделения, полного совместного использования, динамического 
совместного использования и~FAB на примерах модели УК в~виде 
параллельных очередей $M/G/1/K$ с~ограниченным накопителем.  
В~\cite{15-ag} сформулирована задача оптимизации общих стоимостных затрат 
(сто\-и\-мость потери клиента, сто\-и\-мость хранения буфера и~эксплуатационные 
расходы) СМО типа $M/G/1/K$ с~несколькими отпусками и~чис\-лен\-но 
исследовано влияние па\-ра\-мет\-ров сис\-те\-мы, таких как раз\-мер буфера, 
про\-дол\-жи\-тель\-ность отпуска и~рас\-пре\-де\-ле\-ние времени обслуживания, на 
показатели про\-из\-во\-ди\-тель\-ности и~общую сто\-и\-мость затрат. Алгоритм 
оптимизации схемы СР при целевой функции дохода, учи\-ты\-ва\-ющей сред\-нюю 
за\-держ\-ку и~ве\-ро\-ят\-ность потерь пакетов, пред\-ло\-жен в~\cite{16-ag} в~рамках 
модели УК, пред\-став\-лен\-ной параллельными СМО типа $M/G/1/K$. В~рамках 
этой модели УК доказана уни\-мо\-даль\-ность целевой функции по объему БП. 
Аналогичное утверж\-де\-ние для схемы СР доказано и~в~случае УК, 
пред\-став\-лен\-но\-го параллельными СМО типа $G/M/1/K$ с~резервными 
каналами~\cite{17-ag}. 
  
  Ниже исследуется задача оптимизации схемы SMA при стоимостной целевой 
функции, учи\-ты\-ва\-ющей стоимостные за\-тра\-ты из-за потерь пакетов, за\-держ\-ки 
пакетов в~очереди и~эксплуатации накопителя и~при\-бо\-ров.
  
  \section{Модель узла коммутации и~постановка задачи}
  
  В качестве модели УК рассматривается многопотоковая и~многоканальная 
СМО с~общим накопителем (БП) ем\-кости~$N$, на которую 
по\-сту\-па\-ет~$n$~пуассоновских потоков заявок (пакетов) с~\mbox{интенсивностями} 
$\lambda_j\hm>0$, $j\hm= 1,\ldots , n$, в~которой к~каж\-до\-му $j$-му потоку для 
обслуживания заявок прикреплена со\-от\-вет\-ст\-ву\-ющая $j$-я линия из~$s_j$ 
однотипных приборов (каналов). При этом каж\-дая заявка может занимать 
только од\-но свободное мес\-то (ячейку) в~накопителе, на $j$-ли\-нию по\-сту\-па\-ют 
толь\-ко заявки $j$-по\-тока. 
  
  Введем обозначения: $k_j$~--- чис\-ло $j$-за\-явок в~накопителе, $\bar{k}\hm= 
(k_1, \ldots , k_n)$~--- вектор со\-сто\-яния сис\-те\-мы; $\bar{a}\hm= (a_1, \ldots , 
a_n)$, $a_j\hm\geq 0$,~--- чис\-ло закрепленных за $j$-за\-яв\-ка\-ми мест 
в~накопителе (которые могут занять только $j$-за\-яв\-ки), 
$\sum\nolimits^n_{j=1} a_j \hm \leq N$; $L$~--- чис\-ло общедоступных мест 
в~накопителе:
$$
L= N- \sum\limits^n_{j=1} a_j\,;
$$
 $K^L\hm= \{ \bar{k}: 
\sum\nolimits^n_{j=1} k_j \hm\leq \sum\nolimits^n_{j=1} a_j\hm+L\}$~--- 
множество всех воз\-мож\-ных со\-сто\-яний сис\-те\-мы; $\bar{K}_j^L \hm= \{ 
\bar{k}\in K^L: k_j\hm\geq a_j\ \mbox{и}\ \sum\nolimits^n_{j=1} (k_j\hm- a_j) 
\hm= L\}$~--- множество со\-сто\-яний, при которых $j$-за\-яв\-ка не\linebreak
 допускается в~накопитель (теряется); $K_j^L\hm= \{ \bar{k}\hm\in K^L: k_j <a_j\ \mbox{или}\ 
\sum\nolimits^n_{j=1} (k_j\hm- a_j)^+ \hm< L\}$~--- множество со\-сто\-яний, при 
которых $j$-за\-яв\-ке до\-ступ\-но мес\-то в~накопителе, где 
$$
(k_j-a_j)^+ =\begin{cases}
k_j-a_j & \mbox{при}\ k_j>a_j\,;\\
0 & \mbox{при}\ k_j\leq a_j\,. 
\end{cases}
$$
  
  Согласно введенным обозначениям заявки до\-пус\-ка\-ют\-ся в~накопитель по 
схеме SMA~\cite{3-ag}: $j$-за\-яв\-ка допускается в~накопитель, если 
выполняется условие $\bar{k}\hm\in K_j^L$, а~при $\bar{k}\hm\in \bar{K}_j^L$ 
отклоняется (те\-ря\-ется).
  
  Каждая поступившая $j$-за\-яв\-ка занимает одно из до\-ступ\-ных и~свободных 
мест в~накопителе и~один свободный прибор в~$j$-ли\-нии сразу, если $k_j\hm< 
s_j$, или после осво\-бож\-де\-ния, если $k_j\hm \geq s_j$, а~после завершения 
обслуживания покидает сис\-те\-му, освободив одновременно прибор и~мес\-то 
в~накопителе. Время обслуживания $j$-за\-яв\-ки~--- экспоненциальная 
случайная величина с~заданным па\-ра\-мет\-ром~$\mu_j$, $j\hm= 1,\ldots , n$.
  
  Предположим, что $j$-за\-яв\-ки в~первую очередь занимают закрепленные за 
ними мес\-та и,~если есть\linebreak об\-ще\-до\-ступ\-ное мес\-то, занятое $j$-за\-яв\-кой, то 
осво\-бо\-див\-ше\-еся от $j$-за\-яв\-ки за\-креп\-лен\-ное мес\-то становится 
общедоступным, а~это занятое об\-ще\-до\-ступ\-ное мес\-то становится 
за\-креп\-лен\-ным. Таким \mbox{образом}, чис\-ло закрепленных за каж\-дым потоком мест 
остается постоянным. Если вновь по\-сту\-пив\-шая заявка за\-ста\-ет все до\-ступ\-ные ей 
мес\-та в~накопителе занятыми, то она теряется без\-воз\-вратно. 
  
   Процесс перехода описанной СМО из со\-сто\-яния в~со\-сто\-яние~--- марковский 
и~имеет сле\-ду\-ющее распределение стационарных вероятностей 
со\-сто\-яний~\cite{18-ag}:

\vspace*{-1pt}

\noindent
  \begin{equation}
  \pi_{\bar{k}} (L) =\pi_{\bar{0}}(L) \prod\limits^n_{j=1} Z_j(k_j)\,.
  \label{e1-ag}
  \end{equation}
  Здесь 
  
  \vspace*{-4pt}
  
  \noindent
\begin{multline*}
   \pi_{\bar{0}}(L) =\Bigg[ \sum\limits_{\bar{k}\in K} \prod\limits^n_{j=1} Z_j(k_j)\Bigg]^{-1},\\
   \bar{0} =\left\{ \bar{k}: k_i=0,\ i=1,\ldots , n\right\};
   \end{multline*}
  
  \vspace*{-9pt}
  
  \noindent
  $$
  Z_j(k_j) =\begin{cases}
  \fr{\rho_j^{k_j}}{k_j!}\,, & \ 0\leq k_j \leq s_j\,;\\[12pt]
  \fr{\rho_j^{s_j}}{s_j!} \left( \fr{\rho_j}{s_j}\right)^{k_j-s_j}\,, & s_j< k_j \leq 
a_j+L\,,
  \end{cases}
  $$
  где
  $$
   \rho_j=\fr{\lambda_j}{\mu_j}\,.
   $$

%\vspace*{-12pt}


   
   \vspace*{-5pt}
  
  В дальнейшем сис\-те\-му рас\-смат\-ри\-ва\-ем только в~стационарном режиме 
работы и~считаем, что значения па\-ра\-мет\-ров~$a_j=s_j$, $j\hm= 1, \ldots , n$, 
и~являются по\-сто\-ян\-ны\-ми ве\-ли\-чи\-нами. 
  
  В качестве показателя эф\-фек\-тив\-ности сис\-те\-мы используется функция 
предельного дохода в~единицу времени, учи\-ты\-ва\-ющая пла\-ту за обслуживание, 
потери из-за ожидания в~очереди, отклонения заявок и~технического 
обслуживания сис\-те\-мы. Считается, что пла\-ту за обслуживание сис\-те\-ма 
получает в~момент приема заявки в~накопитель. Доход\linebreak измеряется 
в~стоимостных единицах и~зависит так\-же от сле\-ду\-ющих стоимостных 
па\-ра\-мет\-ров:\linebreak $C_{0,i}\hm\geq 0$~--- плата, по\-лу\-ча\-емая сис\-те\-мой, если 
по\-сту\-пив\-шая $i$-за\-яв\-ка будет обслужена сис\-те\-мой;  
$C_{1,i}\hm\geq 0$~--- штраф за отклонение по\-сту\-пив\-шей $i$-за\-яв\-ки; 
$C_{2,i}\hm\geq0$~--- штраф за единицу времени ожидания\linebreak $i$-за\-яв\-ки 
в~очереди; $C_3\hm\geq 0$~--- за\-тра\-ты сис\-те\-мы в~единицу времени на 
техническое обслуживание одного мес\-та в~накопителе; $C_4\hm\geq0$~---  
за\-тра\-ты сис\-те\-мы в~единицу времени на техническое обслуживание всех 
приборов сис\-темы.
  
  Доход выражается функцией сле\-ду\-юще\-го вида: 
  \begin{equation*}
  Q(L)= \lambda \sum\limits_{\bar{k}\in K} \pi_{\bar{k}} (L) 
q_{\bar{k}}\,,\enskip \lambda=\sum\limits^n_{j=1} \lambda_j\,,
  %\label{e2-ag}
  \end{equation*}
где $L$~--- переменная величина; $q_{\bar{k}}$~--- сред\-ний доход, по\-лу\-ча\-емый 
сис\-те\-мой за период между соседними моментами по\-ступ\-ле\-ния извне заявок, 
если в~момент по\-ступ\-ле\-ния в~начале периода сис\-те\-ма оказалась  
в~со\-сто\-янии~$\overline{k}$.

  Задача оптимизации объема об\-ще\-до\-ступ\-ных мест в~накопителе 
сформулирована в~виде математической за\-дачи
  \begin{equation*}
  L^*=\argmax\limits_{L\geq0} Q(L)\,.
  %\label{e3-ag}
  \end{equation*}
  
  \vspace*{-6pt}
  
  \section{Метод решения}
  
  Введем обозначения:
\begin{multline*}
  \bar{K}^L_{j,m}=\left\{
  \vphantom{\sum\limits^n_{i=1,i\not= j}}
   \bar{k}\in\bar{K}_j^L: k_j=m\geq a_j,\right.\\ 
\left.\sum\limits^n_{i=1,i\not= j} (k_i-a_i)^+ =L- (m-a_j) \right\}\,;
  \end{multline*}
  
  \vspace*{-12pt}
  
  \noindent
  \begin{multline*}
  K^L_{j,m} = \left\{
  \vphantom{\sum\limits^n_{i=1,i\not= j}}
   \bar{k}\in K_j^L: k_j=m,\right.\\
   \sum\limits^n_{i=1,i\not=j} (k_i-a_i)^+ < L- (k_j-a_j)^+{}\\
 \left. \mbox{или}\ k_j=m<a_j
   \vphantom{\sum\limits^n_{i=1,i\not= j}}
\right\};
\end{multline*}
  
  \noindent
  $\pi^-_{j,m}(L) = \sum\nolimits_{\bar{k}\in K^L_{j,m}} \pi_{\bar{k}} (L)$~--- 
стационарная ве\-ро\-ят\-ность того, что $\bar{k}\hm\in K^L_{j,m}$;
  $\pi^+_{j,m}(L)= \sum\nolimits_{\bar{k}\in \bar{K}^L_{j,m}}  
 \pi_{\bar{k}}(L)$~--- стационарная ве\-ро\-ят\-ность того, что $\bar{k}\hm\in 
\bar{K}^L_{j,m}$;
$\pi_{j,m}(L)= \left( \pi^-_{j,m}(L)\hm+ \pi^+_{j,m}(L)\right)$~--- стационарная 
ве\-ро\-ят\-ность того, что $k_j\hm=m$;
 $\chi_j(\bar{k})$~--- функция Хевисайда:
$$
\chi_j(\bar{k}) =\begin{cases}
1, & \bar{k}\in K_j^L\,;\\
0, & \bar{k}\in \bar{K}^L_j\,;
\end{cases}
$$

\noindent
$d_{j,m}$~--- стоимость средних суммарных потерь из-за ожидания 
и~отклонения $j$-за\-явок за период меж\-ду со\-сед\-ни\-ми моментами поступления 
из\-вне заявок, если в~начале периода чис\-ло $j$-за\-явок в~накопителе было 
рав\-но~$m$.

\smallskip

%\columnbreak

\noindent
  \textbf{Теорема~1.} \textit{Функцию $Q(L)$ мож\-но записать в~\mbox{виде}}:
  
    \vspace*{-6pt}
  
  \noindent
  \begin{multline}
  Q(L)= \sum\limits^n_{j=1} \left[ (\lambda-\lambda_j) Q_j^+(L) + \lambda_j Q_j^-(L)\right] -{}\\
  {}-C_3 N-C_4\,,
  \label{e4-ag}
  \end{multline}
\textit{где} $Q_j^+(L)$ \textit{и}~$Q_j^-(L)$, $j\hm= 1,\ldots , n$,~--- \textit{уни\-мо\-даль\-ные 
по $L\hm \geq 0$ функ\-ции}:

  \vspace*{-6pt}
  
  \noindent
  \begin{align*}
Q_j^+(L) &= \sum\limits_{m=0}^{a_j+L-1} d_{j,m}\pi^-_{j,m}(L) 
+\sum\limits_{m=0}^{a_j+L} d_{j,m} \pi^+_{j,m}(L)\,;\\
Q_j^-(L) &= \sum\limits_{m=0}^{a_j+L-1} \left(d_{j,m+1} +C_{0,j}\right)  
\pi^-_{j,m}(L) +{}&\\
& \hspace*{20mm}{}+ \sum\limits_{m=0}^{a_j+L} \left( d_{j,m}-C_{1,j}\right) 
\pi^+_{j,m}(L)\,.
\end{align*}
  
  
  \noindent
  Д\,о\,к\,а\,з\,а\,т\,е\,л\,ь\,с\,т\,в\,о\,.\ \  Лег\-ко видеть, что для 
дохода~$q_{\bar{k}}$ и~величин~$d_{j,m}$, $j\hm= 1,\ldots, n$, $m\hm= s_j\hm-
1, \ldots$\linebreak $\ldots ,  a_j\hm+L$ имеет мес\-то сле\-ду\-ющее соотношение:
  $$
  q_{\bar{k}} =\begin{cases}
  \displaystyle \sum\limits^n_{\substack{{l=1,}\\ {l\not=j}}} d_{l,k_l}+d_{j,m+1} +C_{0,j}-
\fr{C_3N+C_4}{\lambda}\,, &\\
&\hspace*{-62mm}\mbox{если\ извне\ поступила\ $j$-заявка\ и}\ 
\bar{k}\in K^L_{j,m}\,;\\
  \displaystyle \sum\limits_{l=1}^n d_{l,k_l} %@@@
  -C_{1,j} -\fr{C_3 N+C_4}{\lambda}\,, &  
\hspace*{-15mm}\mbox{если\ извне}\\
&\hspace*{-43mm}\mbox{поступила\ $j$-заявка\ и}\ \bar{k}\in \bar{K}^L_{j,m}.
  \end{cases}
  $$
    Тогда для целевой функции справедливо вы\-ра\-жение:
  
  \vspace*{-3pt}
  
  \noindent
  \begin{multline*}
  Q(L)= \lambda\sum\limits_{\bar{k}\in K^L} \pi_{\bar{k}} (L) 
\sum\limits^n_{j=1} \fr{\lambda_j}{\lambda} \Bigg[ \sum\limits^n_{{\substack{{l=1,}\\ {l\not=j}}}} 
d_{l,k_l} +{}\\[1pt]
{}+\left( d_{j,k_j+1} +C_{0,j}\right) \chi_j(\bar{k})+{}\\[1pt]
  {}+ \left( d_{j,k_j} -C_{1,j}\right) \left[1-\chi_j (\bar{k})\right] -\fr{C_3 N+C_4}{\lambda}\Bigg] ={}\\[1pt]
   {}=
  \sum\limits^n_{j=1} \lambda_j \Bigg[ \sum\limits^n_{l=1} 
\sum\limits_{m=0}^{a_l+L} \pi_{l,m}(L) d_{l,m} -{}\\[1pt]
{}- \sum\limits_{m=0}^{a_j+L} \pi_{j,m}(L) d_{j,m} +\sum\limits_{m=0}^{a_l+L-1} \pi^-_{j,m}(L) d_{l,m+1}+{}\\[1pt]
{}+
  \sum\limits_{m=0}^{a_j+L} \pi^+_{j,m}(L) d_{j,m} +C_{0,j} 
\sum\limits_{m=0}^{a_j+L-1} \pi^-_{j,m}(L) -{}\\[1pt]
{}- C_{1,j}\sum\limits_{m=0}^{a_j+L} \pi^+_{j,m}(L) \Bigg] -C_3 N-C_4={}
\\[1pt]
     {}=
  \sum\limits^n_{j=1} (\lambda -\lambda_j) \Bigg[ \sum\limits_{m=0}^{a_j+L-1} 
d_{j,m} \pi^-_{j,m}(L) +{}\\[1pt]
{}+ \sum\limits_{m=0}^{a_j+L} d_{j,m} \pi^+_{j,m}(L)\Bigg]+{}\\[1pt]
  {}+
  \sum\limits_{j=1}^n \lambda_j \Bigg[ \sum\limits_{m=0}^{a_j+L-1} \left( 
d_{j,m+1} +C_{0,j}\right) \pi^-_{j,m}(L) +{}\\[1pt]
{}+
\sum\limits_{m=0}^{a_j+L} \left( 
d_{j,m}-C_{1,j}\right) \pi^+_{j,m}(L) \Bigg]-C_3N -C_4 \,.
  \end{multline*}
Следовательно, равенство~(\ref{e4-ag}) вы\-пол\-ня\-ется.

  Приведем аналитические выражения для па\-ра\-мет\-ра~$d_{j,m}$. 
Вос\-поль\-зу\-ем\-ся результатами работы~\cite{7-ag}. Со\-глас\-но этой работе (см.\ 
вывод формулы~(5) в~\cite{7-ag}),
  \begin{multline*}
  d_{j,m}={}\\
  \!\!{}= \begin{cases}
  0\,, & \hspace*{-40mm}m\leq s_j\,;\\[6pt]
  -\fr{C_{2,j}}{2\mu_j s_j} \Bigg[ \displaystyle\sum\limits_{l=1}^{m+1-s_j} \!\!\! l(l+2s_j-2m-1) 
r_{j,l}-{}&\\[6pt]
  {}- (m-s_j)(m+1-s_j) \displaystyle\sum\limits^\infty_{l=m+2-s_j} \!\!\! r_{j,l}\Bigg] , &\\[6pt]
  & \hspace*{-30mm}s_j\leq m\leq a_j+L\,;\\[6pt]
  d_{j,m-1}, &\hspace*{-40mm} m=a_j+L\,,
  \end{cases}\!\!
 % \label{e5-ag}
  \end{multline*}
где $r_{j,l}$~--- вероятность того, что за период между соседними 
по\-ступ\-ле\-ни\-ями заявок на $j$-ли\-нии завершат обслуживание  
ров\-но~$l$~за\-явок при условии, что в~начале периода в~очереди не 
менее~$l$~за-\linebreak явок:

\noindent
$$
r_{j,l}= \lambda \int\limits_0^\infty \fr{(\mu_j s_j t)^l}{l!}\,e^{-(\mu_js_j+\lambda)t}\,dt\ \mbox{при}\ l \geq 0\,.
$$
\vspace*{-4pt}
  
  Согласно работе~\cite{7-ag}, справедливо так\-же со\-от\-но\-шение:

\vspace*{-9pt}

\noindent
  \begin{multline}
  d_{j,m} =d_{j,m+1} -\fr{C_{2,j}}{\mu_j s_j} \sum\limits_{l=1}^{m+1-s_j} \!\!
lr_{j,l} -{}\\
{}- \fr{C_{2,j}(m+1-s_j)}{\mu_j s_j}\sum\limits^\infty_{l=m+2-s_j} \!\! r_{j,l},\\
  s_j\leq m\leq a_j+L-1\,.
  \label{e6-ag}
  \end{multline}
  
  \noindent
  \textbf{Лемма~1.}\ \textit{Для вероятностей $\pi^-_{j,m}(L)$ 
и~$\pi^+_{j,m}(L)$, $j\hm= 1,\ldots, n$, $m\hm= s_j\hm-1, \ldots , a_j\hm+L\hm-1$, 
справедливы ра\-вен\-ства}:
\vspace*{2pt}

\noindent
  \begin{equation}
  \left.
  \begin{array}{rl}
  \pi^-_{j,m+1} (L+1)&=  \pi^-_{j,m}(L) A_j(L+1)\,;\\[6pt]
  \pi^+_{j,m+1} (L+1) &= \pi^+_{j,m}(L) A_j(L+1)\,,
  \end{array}
  \right\}
  \label{e7-ag}
  \end{equation}
  

\noindent
\textit{где}

\noindent
\begin{equation*}
A_j(L+1) \!=\!\fr{1-P_{j,s_j-1}(L+1)}{1-P_{j,s_j-2}(L)}\,,\  
P_{j,m}(L)\!=\!\sum\limits^m_{l=0} \pi_{i,j}(L).
\end{equation*}
  
  \noindent
  Д\,о\,к\,а\,з\,а\,т\,е\,л\,ь\,с\,т\,в\,о\,.\ \ Из~(\ref{e1-ag}) и~равенства 
$K^L_{j,m}\hm= K_{j,m+1}^{L+1}$ сле\-дует 

\vspace*{-4pt}

\noindent
\begin{multline*}
  \pi^-_{j,m+1} (L+1) -\pi^-_{j,m}(L)= {}\\
  {}= \pi_{\bar{0}} (L+1) \fr{\rho_j^{s_j}}{s_j!} 
\left( \fr{\rho_j}{s_j}\right)^{m+1-s_j} \!\!\!\!\!\!\!\!\! \sum\limits_{\bar{k}\in K^{L+1}_{j,m+1}} 
\prod\limits^n_{{\substack{{l=1,}\\ {l\not=j}}}} \! Z_j(k_j)-{}\\
  {}-
   \pi_{\bar{0}}(L) \fr{\rho_j^{s_j}}{s_j!}\left( \fr{\rho_j}{s_j}\right)^{m-s_j} \sum\limits_{\bar{k}\in K^L_{j,m}} \prod\limits^n_{{\substack{{l=1,}\\ {l\not=j}}}}\! Z_j(k_j)={}\\
   {}=
   \pi_{\bar{0}} (L+1) \pi_{\bar{0}}(L) \fr{\rho_j^{s_j}}{s_j!} \left( \fr{\rho_j}{s_j}\right)^{\!m-s_j} \!\!\!\!\!\!
\sum\limits_{\bar{k}\in K^L_{j,m}} \prod\limits^n_{{\substack{{l=1,}\\ {l\not=j}}}} \!  Z_j(k_j)\!\times{}\\[-1pt]
   {}\times 
   \Bigg[ \fr{\rho_j}{s_j} \Bigg[ \sum\limits_{m=0}^{s_j-1} \fr{\rho_j^m}{m!} \sum\limits_{\bar{k}\in K^L_{j,m}}
    \prod\limits^n_{{\substack{{l=1,}\\ {l\not=j}}}} Z_j(k_j)-{}\\
    {}-\fr{\rho_j^{s_j}}{s_j!} \sum\limits_{m=s_j}^{a_j+L} \left( 
\fr{\rho_j}{s_j}\right)^{m-s_j} \sum\limits_{\bar{k}\in K^L_{j,m}} \prod\limits^n_{{\substack{{l=1,}\\ {l\not=j}}}} Z_j(k_j)\Bigg]-{}\\
{}- 
  \sum\limits_{m=0}^{s_j-1} \fr{\rho_j^m}{m!} \sum\limits_{\bar{k}\in 
K^{L+1}_{j,m+1}} \prod\limits^n_{{\substack{{l=1,}\\ {l\not=j}}}} Z_j(k_j) +{}\\[-1pt]
{}+  \fr{\rho_j^{s_j}}{s_j!} 
\sum\limits_{m=s_j}^{a_j+L+1} \left( \fr{\rho_j}{s_j}\right)^{m-s_j} \!\!\!
\sum\limits_{\bar{k}\in K^{L+1}_{j,m+1}} \prod\limits^n_{{\substack{{l=1,}\\ {l\not=j}}}} 
Z_j(k_j)\Bigg]={}
\end{multline*}

\noindent
\begin{multline*}
  {}= 
  \pi_{\bar{0}} (L+1) \fr{\rho_j^{s_j}}{s_j!} \left( \fr{\rho_j}{s_j}\right)^{m+1-s_j}\times{}\\
  {}\times  \sum\limits_{\bar{k}\in K^{L+1}_{j,m+1}}\prod\limits^n_{{\substack{{l=1,}\\ {l\not=j}}}} Z_j(k_j)\pi_{\bar{0}}(L) 
\sum\limits_{m=0}^{s_j-2} \fr{\rho_j^m}{m!}\times{}\\
  {}\times 
  \sum\limits_{\bar{k}\in K^L_{j,m}} \prod\limits^n_{{\substack{{l=1,}\\ {l\not=j}}}} Z_j(k_j) -
\pi_{\bar{0}}(L) \fr{\rho_j^{s_j}}{s_j!} \left( \fr{\rho_j}{s_j}\right)^{m-s_j} \times{}\\
{}\times \sum\limits_{\bar{k}\in K^L_{j,m} }\prod\limits^n_{{\substack{{l=1,}\\ {l\not=j}}}} \! Z_j(k_j)\times{}\\
{}\times 
\pi_{\bar{0}}(L+1)  \sum\limits_{m=0}^{s_j-1} \fr{\rho_j^m}{m!} \sum\limits_{\bar{k}\in K^{L+1}_{j,m+1}} \prod\limits^n_{{\substack{{l=1,}\\ {l\not=j}}}} \! Z_j(k_j)={}\\
  {}=
  \pi^-_{j,m+1}(L+1) P_{j,s_j-2}(L) -\pi^-_{j,m}(L) P_{j,s_j-1}(L+1).
  \end{multline*}
Следовательно, первое равенство в~(\ref{e7-ag}) выполняется. Точ\-но так же 
доказывается второе равенство в~(\ref{e7-ag}).

%\pagebreak
  
  \noindent
  \textbf{Лемма~2.} \textit{Для функций $Q_j^+(L)$ и~$Q_j^-(L)$ спра\-вед\-ливы 
со\-от\-но\-ше\-ния}:
  \begin{equation}
  \left.
  \begin{array}{rl}
  \hspace*{-5mm}Q_j^-(L) -Q_j^-(L+1) &={}\\[6pt]
  &\hspace*{-25mm}{}= \left[ 1-A_j(L+1)\right] \left[ Q_j^-(L) -G_j^-(L)\right]\,;\\[6pt]
 \hspace*{-5mm} Q_j^+(L) -Q_j^+(L+1) &={}\\[6pt]
  &\hspace*{-25mm}{}= \left[ 1-A_j(L+1)\right] \left[ Q_j^+(L) -G_j^+(L)\right],
  \end{array}
  \right\}
  \label{e8-ag}
  \end{equation}
  
  \vspace*{-3pt}
  
  \noindent
    \textit{где} 
  
  \vspace*{-12pt}
  
  \noindent
  \begin{multline*}
  G_j^-(L) =C_{0,j}+ \fr{C_{2,j}}{1-A_j(L+1)}\Bigg[ \sum\limits_{m=s_j}^{a_j+L} 
\Bigg[ \fr{1}{\lambda} -{}\\
{}- \fr{1}{\mu_j s_j} \sum\limits^\infty_{l=m+2-s_j} (l-m-1+s_j) 
r_{j,l}\Bigg] \pi^-_{j,m}(L+1)+{}\\
  {}+
  \sum\limits_{m=s_j}^{a_j+L+1} \Bigg[ \fr{1}{\lambda} -{}\\
  {}- \fr{1}{\mu_j s_j} 
\sum\limits^\infty_{l=m+1-s_j} (l-m+s_j) r_{j,l}\Bigg] \pi^+_{j,m}(L+1)\Bigg];
    \end{multline*}
  
  \vspace*{-12pt}
  
  \noindent
  \begin{multline*}
  G_j^+(L)=  \fr{C_{2,j}}{1-A_j(L+1)} \Bigg[ \sum\limits_{m=s_j}^{a_j+L} \Bigg[ 
\fr{1}{\lambda}-{}\\
{}-\fr{1}{\mu_j s_j} \sum\limits^\infty_{l=m+1-s_j} (l-m+s_j) 
r_{j,l}\Bigg] \pi^-_{j,m}(L+1)+{}\\
  {}+\sum\limits_{m=s_j}^{a_j+L+1} \Bigg[ \fr{1}{\lambda}-{}\\
  {}-\fr{1}{\mu_j s_j} 
\sum\limits^\infty_{l=m+1-s_j} (l-m+s_j) r_{j,l} \Bigg] \pi^+_{j,m}(L+1)\Bigg].
  \end{multline*}
  
  \noindent
  Д\,о\,к\,а\,з\,а\,т\,е\,л\,ь\,с\,т\,в\,о\,.\ \  Докажем первое равенство 
в~(\ref{e8-ag}). Обозначим $\Delta_{j,m+1}\hm= d_{j, m+1} \hm- d_{j,m}$. 
Из~(\ref{e4-ag}) и~(\ref{e7-ag}) после не\-слож\-ных преобразований получим  
ра\-вен\-ства: 

\vspace*{-4pt}

\noindent
  \begin{multline*}
  Q_j^-(L) -Q_j^-(L+1)=
   \sum\limits_{m=0}^{s_j-2} (d_{j,m+1} +C_{0,j})\pi^-_{j,m}(L)- {}\\[1pt]
   {}-\sum\limits_{m=0}^{s_j-1} (d_{j,m+1}+C_{0,j}) \pi^-_{j,m}(L+1)+{}\\[1pt]
   {}+
  \sum\limits_{m=s_j-1}^{a_j+L-1} (d_{j,m+1} +C_{0,j})\pi^-_{j,m}(L) -{}\\[1pt]
  {}-
\sum\limits_{m=s_j}^{a_j+L} (d_{j,m+1}+C_{0,j})\pi^-_{j,m}(L+1)+{}\\[1pt]
{}+
  \sum\limits_{m=0}^{s_j-2} (d_{j,m}-C_{1,j}) \pi^+_{j,m}(L) -{}
  \\[1pt]
     {}-
\sum\limits_{m=0}^{s_j-1} (d_{j,m}-C_{1,j})\pi^+_{j,m}(L+1)+{}\\[1pt]
    {}+
  \sum\limits^{a_j+L}_{m=s_j-1} (d_{j,m}-C_{1,j})\pi^+_{j,m}(L)- {}\\[1pt]
  {}-
\sum\limits_{m=s_j}^{a_j+L+1} (d_{j,m}-C_{1,j})\pi^+_{j,m}(L+1)={}\\[1pt]
  {}=
  C_{0,j} \left[ P_{j,s_j-2}(L) -P_{j,s_j-1}(L+1)\right] +{}\\[1pt]
  {}+\sum\limits_{m=s_j-1}^{a_j+L-1} (d_{j,m+1} +C_{0,j}) \pi^-_{j,m}(L)-{}\\[1pt]
  {}-
   A_j(L+1) \sum\limits_{m=s_j-1}^{a_j+L-1} (d_{j,m+1} +\Delta_{j,m+2} +C_{0,j})\pi^-_{j,m}(L)+{}\\[1pt]
   {}+
   \sum\limits_{m=s_j-1}^{a_j+L} (d_{j,m} -C_{1,j}) \pi^+_{j,m}(L) -{}\\[1pt]
   {}- A_j(L+1) \sum\limits_{m=s_j-1}^{a_j+L} (d_{j,m}+\Delta_{j,m+1} -C_{1,j})\pi^+_{j,m}(L)={}\\[1pt]
   {}= 
  -C_{0,j} [1-A_j(L+1)] +[1-A_j(L+1)] Q_j^-(L)-{}\\[1pt]
  {}-
  A_j(L+1) \sum\limits_{m=s_j-1}^{a_j+L-1} \Delta_{j,m+2} \pi^-_{j,m}(L) -{}\\[1pt]
  {}-A_j(L+1)  \sum\limits_{m=s_j-1}^{a_j+L} \Delta_{j,m+1} \pi^+_{j,m}(L)={}\\[1pt]
  \hspace*{-10mm}{}= 
  [1-A_j(L+1)] \Bigg[ Q_j^-(L) - C_{0,j} -{}
  \end{multline*}
  
  \noindent
  \begin{multline*}
  {}- \fr{1}{1-A_j(L+1)} \Bigg[ 
\sum\limits_{m=s_j}^{a_j+L} \Delta_{j,m+1}\pi^-_{j,m}(L+1)+{}\\
  {}+
  \sum\limits_{m=s_j}^{a_j+L+1} \Delta_{j,m}\pi^+_{j,m}(L+1)\Bigg] \Bigg].
  \end{multline*}
  
  \vspace*{-6pt}
  
  \noindent
  Применив формулу~(\ref{e6-ag}), по\-лучим
  
  \vspace*{-10pt}
  
  \noindent
  \begin{multline}
  Q_j^-(L)-Q_j^-(L+1)= \left[ 1-A_j(L+1)\right] \Bigg[ Q_j^-(L)-{}\\
  {}-C_{0,j}- \fr{C_{2,j}}{1-A_j(L+1)} \Bigg[ \sum\limits_{m=s_j}^{a_j+L} \Bigg[ 
\fr{1}{\lambda} -{}\\
{}- \fr{1}{\mu_j s_j} \sum\limits^\infty_{l=m+2-s_j} (l-m-1+s_j) r_{j,l} \Bigg] \pi^-_{j,m}(L+1)+{}
\\[-2pt]
  {}+\sum\limits_{m=s_j}^{a_j+L+1} \Bigg[ \fr{1}{\lambda} -{}\\[-2pt]
  {}- \fr{1}{\mu_j s_j} 
\sum\limits^\infty_{l=m+1-s_j} \!\!\!\! (l-m+s_j) r_{j,l}\Bigg] \pi^+_{j,m}(L+1)\Bigg] 
\Bigg].
  \label{e9-ag}
  \end{multline}
  
  Следовательно, первое равенство в~(\ref{e8-ag}) выполняется. Точ\-но так же 
для второго равенства в~(\ref{e8-ag}) по\-лучим:

  \vspace*{-6pt}
  
  \noindent
  \begin{multline}
  Q_j^+(L) -Q_j^+(L+1) =\left[ 1-A_j(L+1)\right[ \Bigg[ Q_j^+(L)-{}\\
  {}- \fr{C_{2,j}}{1-A_j(L+1)} \Bigg[ \sum\limits_{m=s_j}^{a_j+L} \Bigg[
  \fr{1}{\lambda} -{}\\
  {}-\fr{1}{\mu_j s_j} \sum\limits^\infty_{l=m+1-s_j} (l-m+s_j) 
r_{j,l}\Bigg] \pi^-_{j,m} (L+1)+{}\\
  {}+\sum\limits_{m=s_j}^{a_j+L+1} \Bigg[ \fr{1}{\lambda} -{}\\
 \!\! {}-\fr{1}{\mu_j s_j} 
\sum\limits^\infty_{l=m+1-s_j} \!\!\!\!(l-m+s_j) r_{j,l}\Bigg] \pi^+_{j,m} (L+1)\Bigg] 
\Bigg].
  \label{e10-ag}
  \end{multline}
Следовательно, выполняется и~второе равенство в~(\ref{e8-ag}).

  \smallskip
  
  \noindent
  \textbf{Лемма~3.} \textit{Функции $G_j^-(L)$ и~$G_j^+(L)$~--- 
не\-воз\-рас\-та\-ющие функ\-ции}.
  
  \smallskip
  
  \noindent
  Д\,о\,к\,а\,з\,а\,т\,е\,л\,ь\,с\,т\,в\,о\,.\ \ Обратим внимание на то, что 
сумма во внеш\-ней квад\-рат\-ной скоб\-ке в~(\ref{e9-ag}) выражает сред\-нее время 
на\-хож\-де\-ния $j$-ли\-нии в~со\-сто\-янии с~пол\-ностью занятыми приборами за 
период меж\-ду соседними по\-ступ\-ле\-ни\-ями заявок в~сис\-те\-ме 
с~об\-ще\-до\-ступ\-ны\-ми местами в~количестве $L\hm+1$ при условии, что 
по\-сту\-пив\-шая из\-вне за\-яв\-ка принадлежит $j$-по\-то\-ку, а~сумма во внеш\-ней 
квад\-рат\-ной скоб\-ке в~(\ref{e10-ag}) выражает то же самое для $j$-ли\-нии при 
условии, что по\-сту\-пив\-шая заявка не принадлежит $j$-по\-то\-ку. Поэтому, 
очевидно, выражения в~указанных скоб\-ках~--- воз\-рас\-та\-ющие по~$L$ 
функ\-ции. 
  
  Рассмотрим разность
  
    %\vspace*{-3pt}
  
  \noindent
  \begin{multline}
  A_j(L+2) -A_j(L+1) ={}\\
  {}=\fr{1-P_{j,s_j-1}(L+2)}{1-P_{j,s_j-2}(L+1)}-
   \fr{1-P_{j,s_j-1}(L+1)}{1-P_{j,s_j-2}(L)}\,.
  \label{e11-ag}
  \end{multline}
  
    \vspace*{-3pt}
  
  \noindent
  Справедливы ра\-вен\-ства:
  
  \vspace*{-3pt}
  
  \noindent
  \begin{multline*}
  \fr{\pi_{\bar{0}}(L+1)}{ \pi_{\bar{0}}(L+2)} [ 1-P_{j,s_j-1}(L+2)] [1-P_{j,s_j-2}(L)]={}\\
  {}=
  \fr{\pi_{\bar{0}}(L)}{ \pi_{\bar{0}}(L+1)} [1-P_{j,s_j-1} (L+1)]^2={}\\
  {}=
  \Bigg[ \sum\limits_{\bar{k}\in K^{L+1}_{j,s_j}\cup \bar{K}^{L+1}_{j,s_j}} 
\prod\limits^n_{{\substack{{l=1,}\\ {l\not=j}}}} Z_j(k_j)+{}\\
{}+\sum\limits^{a_j+L+1}_{m=s_j+1} \left( 
\fr{\rho_j}{s_j}\right)^{m-s_j} \sum\limits_{\bar{k}\in K^{L+1}_{j,m} \cup 
\bar{K}^{L+1}_{j,m}} \prod\limits^n_{{\substack{{l=1,}\\ {l\not=j}}}} Z_j(k_j)\Bigg] \times{}
\\
  {}\times
  \Bigg[ \sum\limits_{\bar{k}\in K^{L+2}_{j,s_j}} \prod\limits^n_{{\substack{{l=1,}\\ {l\not=j}}}} 
Z_j(k_j) +{}\\
{}+\sum\limits_{m=s_j+1}^{a_j+L+2} \left( \fr{\rho_j}{s_j}\right)^{m-s_j} 
\sum\limits_{\bar{k}\in K^{L+2}_{j,m}\cup \bar{K}^{L+2}_{j,m}} 
\prod\limits^n_{{\substack{{l=1,}\\ {l\not=j}}}} Z_j(k_j)\Bigg].
  \end{multline*}
    Отсюда следует, что знак раз\-ности~(\ref{e11-ag}) совпадает со знаком  
раз\-ности 
  ${\pi_{\bar{0}}(L+2)}/{\pi_{\bar{0}}(L+1)} \hm - {\pi_{\bar{0}}(L+1)}/{\pi_{\bar{0}}(L)}.$
  
  Для рассматриваемой сис\-те\-мы в~случае накопителя с~($L\hm+1$) 
общедоступными мес\-та\-ми величина $\pi_{\bar{0}}(L+1)/ \pi_{\bar{0}} 
(L)$~--- ве\-ро\-ят\-ность того, что в~накопителе есть хотя бы одно свободное  
мес\-то, и,~очевидно, эта величина с~увеличением~$L$ воз\-рас\-та\-ет. Отсюда 
следует, что указанная выше раз\-ность для всех $L\hm\geq 0$ является 
положительной величиной. Следовательно, $G_j^-(L)$~---  
не\-воз\-рас\-та\-ющая функция. Точ\-но так же доказывается лемма и~в~случае 
функ\-ции~$G_j^+(L)$. 
  
   Из лемм 1 и~2 следует, что функции $Q_j^-(L)$ и~$Q_j^+(L)$  
удовле\-тво\-ря\-ют всем условиям тео\-ре\-мы в~работе~\cite{19-ag}, откуда 
вытекает спра\-вед\-ли\-вость утверж\-де\-ний тео\-ре\-мы~1 об  
уни\-мо\-даль\-ности функций $Q_j^-(L)$ и~$Q_j^+(L)$. Следовательно,  
тео\-ре\-ма~1 до\-ка\-зана.
  
  \smallskip
  
  \noindent
  \textbf{Следствие~1.}\  Пусть $L_j^-$ и~$L_j^+$~--- точ\-ки максимума 
со\-от\-вет\-ст\-ву\-ющих функций $Q_j^-(L)$ и~$Q_j^+(L)$, $j\hm=1,\ldots , n$. Тогда 
значение~$L^*$~--- точ\-ка глобального максимума функции $D(L)$~--- 
удовлетворяет условию: $L_1^*\hm\leq L^*\hm\leq L_2^*$, где $L_1^*\hm= \min 
\{ L_j^-, L_j^+,\ j\hm= 1,\ldots, n\}$; $L_2^*\hm= \max\{ L_j^-, L_j^+,\ j\hm= 
1,\ldots , n\}$.\linebreak\vspace*{-12pt}

{ \begin{center}  %fig1
 \vspace*{-3pt}
   \mbox{%
\epsfxsize=78.231mm 
\epsfbox{aga-1.eps}
}

\end{center}

\vspace*{-2pt}

\noindent
\small{Зависимость $F_j(L)$ от $L$: \textit{1}~--- $Q(L)$; \textit{2}~--- 
$F_1(L)$; \textit{3}~--- $F_2(L)\hm= F_3(L)$; \textit{4}~--- максимальные 
значения функций $F_j(L)$ и~$Q(L)$
}
}

\vspace*{9pt}

\noindent
 В~част\-ности, при одинаковых для всех $j\hm= 1, \ldots, n$ 
значениях величин~$\mu_j$, $\lambda_j$, $s_j$ и~$a_j$ функ\-ция~$Q(L)$ 
уни\-мо\-дальна. 
  
 % \smallskip
  
  На рисунке приведены графики функций $Q(L)$ и~$F_j(L)\hm= (1\hm- 
\lambda_j/\lambda)Q_j^+(L)\hm+ (\lambda_j/\lambda)Q_j^-(L)$, $j\hm= 1,2,3$, 
де\-мон\-ст\-ри\-ру\-ющие результаты тео\-ре\-мы~1, при сле\-ду\-ющих па\-ра\-мет\-рах: 
$\lambda_1\hm= 3$; $\lambda_2\hm=\lambda_3=3/2$; $\mu\hm= 2$; 
$s_1\hm=s_2\hm=s_3\hm=2$; $C_0\hm=20$; $C_{0j}\hm=20$; $C_{1j}\hm=10$; 
$C_{2j}\hm= 3$; $C_3\hm= 0$; $C_4\hm= 0{,}01$.

  \vspace*{-6pt}
   
  \section{Заключение}
  
  Основной результат статьи~--- доказательство утверж\-де\-ния о~том, что 
функция дохода рас\-смот\-рен\-ной СМО пред\-став\-ля\-ет собой линейную 
комбинацию унимодальных функций. Прак\-ти\-че\-ская важ\-ность результата 
заключается в~том, что для такой функции границами об\-ласти, где 
гарантированно лежит точ\-ка глобального максимума, служат минимальное 
и~максимальное значения на множестве глобальных максимумов указанных 
унимодальных функций. 
  

  Для рассмотренной СМО аналогично тео\-ре\-ме~1 доказывается так\-же 
сле\-ду\-ющее утверж\-де\-ние: в~случае схемы CP (схема SMA при $L\hm=0$) 
если~$N$~--- фиксированная величина, а~$a_j$~--- переменные величины, 
$j\hm= 1, \ldots , n$, то функция дохода есть сумма $\sum\nolimits^n_{j=1} 
f_j(a_j)$ унимодальных на отрезке $[1,N]$ функций $f_j(a_j)$ таких, что 
$f_j(a_j)$~--- вы\-пук\-лая по~$a_j$ на со\-от\-вет\-ст\-ву\-ющем отрезке $[1,a_j^*]$, где 
$(a_1^*, \ldots , a_n^*)$~--- точ\-ка глобального максимума функции дохода на 
множестве наборов $(a_1, \ldots , a_n)$, $\sum\nolimits^n_{j=1} a_j \hm\leq N$, 
$a_j\hm> 0$, $j\hm=1, \ldots , n$. Из этого утверж\-де\-ния следует прос\-тое правило 
поиска точ\-ки глобального максимума функции дохода при схеме СР: пока 
$\sum\nolimits^n_{j=1} a_j\hm<N$ по\-сле\-до\-ва\-тель\-но на каж\-дом шаге 
увеличиваем на~1 одну из переменных~$a_j$ с~индексом~$j^*$ таким, что 
при\-ра\-щение 
 \begin{multline*}
  f_{j^*} \left( a_{j^*}+1\right) - f_{j^*}\left(a_{j^*}\right) ={}\\
  {}=\max\limits_j \left\{ 
f_j\left( a_j+1\right) -f_j\left(a_j\right)>0\,,\ j=1,\ldots , n\right\}.
  \end{multline*}
  
  Результаты работы могут быть использованы при разработке и~эксплуатации 
информационных и~производственных потоковых сис\-тем с~накопителями 
ограниченной ем\-кости для повышения их эф\-фек\-тив\-ности. 

\vspace*{-6pt}
  
{\small\frenchspacing
 { %\baselineskip=12pt
 %\addcontentsline{toc}{section}{References}
 \begin{thebibliography}{99}
\bibitem{1-ag}
\Au{Клейнрок Л.} Вы\-чис\-ли\-тель\-ные сис\-те\-мы с~очередями~/ Пер.\ с~англ.~--- М.: Мир, 
1979. 600~с. (\Au{Kleinrock~L.} Queueing systems.~--- New York, NY, USA: 
Wiley, 1976. Vol.~2. 549~p.)
\bibitem{2-ag}
\Au{Irland M.} Buffer management in a~packet switch~// IEEE T. Commun., 1978. Vol.~26. 
No.\,3. P.~328--337. doi: 10.1109/TCOM.1978.1094076.
\bibitem{3-ag}
\Au{Kamoun F., Kleinrock~L.} Analysis of shared finite storage in a~computer networks node 
environment under general traffic conditions~// IEEE T. Commun., 1980. Vol.~28. No.\,7. P.~992--
1003. doi: 10.1109/TCOM.1980.1094756.
\bibitem{4-ag}
\Au{K$\ddot{\mbox{o}}$chel P.} Finite queueing systems~--- structural investigations and optimal 
design~// Int. J. Prod. Econ., 2004. Vol.~88. P.~157--171. doi: 10.1016/j.ijpe.2003.11.005.
\bibitem{5-ag}
\Au{Sonderman D.} Comparing multi-server queues with finite waiting rooms, I: Same number of 
servers~// Adv. Appl. Probab., 1979. Vol.~11. P.~439--447. doi: 10.2307/1426848.
\bibitem{6-ag}
\Au{Whitt W.} Counterexamples for comparisons of queues with finite waiting rooms~// Queueing 
Syst., 1992. Vol.~10. P.~271--278. doi: 10.1007/BF01159210.

\bibitem{12-ag} %7
\Au{Wei S.\,X., Coyle~E.\,J., Hsiao~M.-T.\,T.} An optimal buffer management policy for  
high-performance packet switching~// Global Telecommunications Conference ``Countdown to the New Millennium'' Proceedings.~--- Phoenix, AZ, USA: 
IEEE, 1991. Vol.~2. P.~924--928. doi: 10.1109/\mbox{GLOCOM}.1991.188515.
\bibitem{13-ag} %8
\Au{Cidon I., Georgiadis~L., Gubrin~R., Khamisy~A.} Optimal buffer sharing~// IEEE J.~Sel. 
Area. Comm., 1995. Vol.~13. No.\,7. P.~1229--1240. doi: 10.1109/ 49.414642.

\bibitem{11-ag} %9
\Au{Choudhury A.\,K., Hahne~E.\,L.} Dynamic queue length thresholds for shared-memory packet 
switches~// IEEE ACM T. Network., 1998. Vol.~6. No.\,2. P.~130--140. 
\bibitem{7-ag} %10
\Au{Ziya S.} On the relationships among traffc load, capacity, and throughput for the $M/M/1/m$, 
$M/G/1/m\mbox{-}\mathrm{PS}$, and $M/G/c/c$ queues~// IEEE T. Automat. Contr., 2008. Vol.~53. 
P.~2696--2701. doi: 10.1109/TAC.2008.2007173.
\bibitem{8-ag} %11
\Au{Линец Г.\,И.} Управ\-ле\-ние объемом буферной памяти и~про\-пуск\-ной спо\-соб\-ностью 
каналов в~муль\-ти\-сер\-вис\-ных сетях~// Инфокоммуникационные технологии, 2008. Т.~6. №\,2. 
С.~62--64. 
\bibitem{9-ag} %12
\Au{Жерновый Ю.\,В.} Решение задач оптимального син\-те\-за для некоторых марковских 
моделей обслуживания~// Информационные процессы, 2010. Т.~10. №\,3. C.~257--274.
\bibitem{10-ag} %13
\Au{Михеев П.\,А.} Анализ стратегий разделения конечной буферной памяти маршрутизатора 
меж\-ду выходными каналами~// Автоматика и~телемеханика, 2014. №\,10. С.~125--128. 
\bibitem{16-ag} %14
\Au{Агаларов Я.\,М.} Оптимизация объема буферной памяти уз\-ла коммутации при схеме 
пол\-но\-го раз\-де\-ле\-ния памяти~// Информатика и~её применения, 2018. Т.~12. Вып.~4.  
С.~25--32. doi: 10.14357/19922264180404.

\bibitem{14-ag} %15
\Au{Apostolaki M., Vanbever~L., Ghobadi~M.} FAB: Toward flow-aware buffer sharing on 
programmable switches~// ACM Workshop on Buffer Sizing, 2019. 6~p. doi: 10.1145/3375235.3375237.  
{\sf https://people.csail.mit.edu/\linebreak  ghobadi/papers/fab\_buffer\_2019.pdf}.
\bibitem{15-ag} %16
\Au{Kim K.} Numerical study of optimal buffer size and\linebreak vacation length in $M/G/1/K$ queues with 
multiple vacations~// Int. J. Engineering Technologies Management Research, 2019. Vol.~6. 
No.\,2. P.~1--13. doi: 10.29121/ ijetmr.v6.i2.2019.350.

\bibitem{17-ag}
\Au{Агаларов Я.\,М.} Об оптимизации работы резервного прибора в~многолинейной сис\-те\-ме 
массового обслуживания~// Информатика и~её применения, 2023. Т.~17. Вып.~1. С.~89--95. 
doi: 10.14357/19922264230112.
\bibitem{18-ag}
\Au{Башарин Г.\,П., Са\-муй\-лов~К.\,Е.} Об оптимальной структуре буферной памяти в~сетях 
передачи данных с~коммутацией пакетов.~--- М., 1982. Препринт АН \mbox{СССР}. 70~с. 
\bibitem{19-ag}
\Au{Агаларов Я.\,М.} Признак уни\-мо\-даль\-ности це\-ло\-чис\-лен\-ной функ\-ции одной переменной~// 
Обо\-зре\-ние при\-клад\-ной и~промышленной математики, 2019. Т.~26. Вып.~1. С.~65--66.

\end{thebibliography}

 }
 }

\end{multicols}

\vspace*{-9pt}

\hfill{\small\textit{Поступила в~редакцию 01.06.23}}

\vspace*{6pt}

%\pagebreak

%\newpage

%\vspace*{-28pt}

\hrule

\vspace*{2pt}

\hrule


\def\tit{OPTIMIZATION OF THE BUFFER MEMORY ALLOCATION SCHEME 
OF~THE~PACKET SWITCHING NODE}


\def\titkol{Optimization of the buffer memory allocation scheme 
of~the~packet switching node}


\def\aut{Ya.\,M.~Agalarov}

\def\autkol{Ya.\,M.~Agalarov}

\titel{\tit}{\aut}{\autkol}{\titkol}

\vspace*{-15pt}


\noindent
Federal Research Center ``Computer Science and Control'' of the Russian Academy 
of Sciences, 44-2~Vavilov Str., Moscow 119333, Russian Federation


\def\leftfootline{\small{\textbf{\thepage}
\hfill INFORMATIKA I EE PRIMENENIYA~--- INFORMATICS AND
APPLICATIONS\ \ \ 2023\ \ \ volume~17\ \ \ issue\ 3}
}%
 \def\rightfootline{\small{INFORMATIKA I EE PRIMENENIYA~---
INFORMATICS AND APPLICATIONS\ \ \ 2023\ \ \ volume~17\ \ \ issue\ 3
\hfill \textbf{\thepage}}}

\vspace*{9pt}

\Abste{The buffer of the packet switching node shared by several output communication lines is 
considered. Sharing buffer memory by multiple users reduces the amount of memory needed to meet latency 
requirements and the likelihood of packet loss. However, there is a problem of allocating buffer memory 
between different users, since individual users, having occupied all the memory, can restrict (or close) access 
to communication lines to other users which can significantly reduce the performance of the switching node 
as a whole. There are many different buffer memory allocation schemes, one of which, called SMA (Sharing 
with Minimum Allocation), is being investigated in this paper in order to reduce the costs associated with 
packet rejection and delay and the operation of the drive and communication lines. A~multithreaded queuing 
system with parallel devices of the $M/M/s/K$ type with a buffer shared according to the SMA scheme with 
a~fixed number of storage locations reserved for each device is used as a model of the switching node. The 
mathematical formulation of the problem of optimizing the SMA scheme in terms of the volume of publicly 
accessible buffer locations is formulated in order to minimize system losses arising from rejection of 
applications, delay of applications in the queue, and operation of the buffer and devices. The theorem on the 
boundaries of the domain containing the point of the global optimum is proved. A~number of statements are 
also given which are the consequences of the theorem about the point of the global optimum of the objective 
function for other models of the switching node and special cases of SMA.}

\KWE{switching node; buffer memory allocation; optimization; queuing system}

\DOI{10.14357/19922264230306}{QLXCKV}

%\vspace*{-20pt}

 %\Ack
 %     \noindent
  

\vspace*{4pt}

  \begin{multicols}{2}

\renewcommand{\bibname}{\protect\rmfamily References}
%\renewcommand{\bibname}{\large\protect\rm References}

{\small\frenchspacing
 {%\baselineskip=10.8pt
 \addcontentsline{toc}{section}{References}
 \begin{thebibliography}{99} 
 
 \vspace*{-2pt}
 
\bibitem{1-ag-1}
\Aue{Kleinrock, L.} 1976. \textit{Queueing systems}. New York, NY: 
Wiley. Vol.~2. 549~p.
\bibitem{2-ag-1}
\Aue{Irland, M.} 1978. Buffer management in a~packet switch. \textit{IEEE T. Commun.} 26(3):328--337. 
doi: 10.1109/TCOM. 1978.1094076.
\bibitem{3-ag-1}
\Aue{Kamoun, F., and L.~Kleinrock.} 1980. Analysis of shared finite storage in a computer networks node 
environment under general traffic conditions. \textit{IEEE T. Commun.} 28(7):992--1003. doi: 
10.1109/TCOM.1980.1094756.
\bibitem{4-ag-1}
\Aue{K$\ddot{\mbox{o}}$chel, P.} 2004. Finite queueing systems~--- structural investigations and optimal 
design.  \textit{Int. J. Prod. Econ.} 88(2):157--171. doi: 10.1016/j.ijpe.2003.11.005.
\bibitem{5-ag-1}
\Aue{Sonderman, D.} 1979. Comparing multi-server queues with finite waiting rooms, I: Same number of 
servers. \textit{Adv. Appl. Probab.} 11:439--447. doi: 10.2307/1426848.
\bibitem{6-ag-1}
\Aue{Whitt, W.} 1992. Counterexamples for comparisons of queues with finite waiting rooms. 
\textit{Queueing Syst.} 10:271--278. doi: 10.1007/BF01159210.

\bibitem{12-ag-1} %7
\Aue{Wei, S.\,X., E.\,J.~Coyle, and M.-T.\,T.~Hsiao.} 1991. An optimal buffer management policy for high-performance 
packet switching. \textit{Global Telecommunications Conference ``Countdown to the New Millennium'' Proceedings}. 
Phoenix, AZ: IEEE. 2:924--928. doi: 10.1109/\mbox{GLOCOM}. 1991.188515.
\bibitem{13-ag-1} %8
\Aue{Cidon, I., L.~Georgiadis, R.~Gubrin, and A.~Khamisy.} 1995. Optimal buffer sharing. \textit{IEEE J. 
Sel. Area. Comm.} 13(7):1229--1240. doi: 10.1109/49.414642.

\bibitem{11-ag-1} %9
\Aue{Choudhury, A.\,K., and E.\,L.~Hahne.} 1998. Dynamic queue length thresholds for shared-memory 
packet switches. \textit{IEEE ACM T. Network}. 6(2):130--140. doi: 10.1109/90. 664262.

\bibitem{7-ag-1} %10
\Aue{Ziya, S.} 2008. On the relationships among traffic load, capacity, and throughput for the $M/M/1/m$, 
$M/G/1/m\mbox{-}\mathrm{PS}$, and $M/G/c/c$ queues. \textit{IEEE T. Automat. Contr.} 53(11):2696--2701. doi: 
10.1109/TAC.2008. 2007173.
\bibitem{8-ag-1} %11
\Aue{Linets, G.\,I.} 2010. Uprav\-le\-nie ob''\-emom bu\-fer\-noy pa\-mya\-ti i~pro\-pusk\-noy  
spo\-sob\-nost'yu ka\-na\-lov v~mul'\-ti\-ser\-vis\-nykh se\-tyakh [Volume management of buffer memory 
and throughput of channels in multiservice networks]. \textit{In\-fo\-kom\-mu\-ni\-ka\-tsi\-on\-nye  
tekh\-no\-lo\-gii} [Information Communication Technologies] 6(2):62--64.
\bibitem{9-ag-1} %12
\Aue{Zhernovyy, Yu.\,V.} 2010. Re\-she\-nie za\-dach op\-ti\-mal'\-no\-go sin\-te\-za dlya ne\-ko\-to\-rykh 
mar\-kov\-skikh mo\-de\-ley ob\-slu\-zhi\-va\-niya [Solution of optimum synthesis problem for some Markov 
models of service]. \textit{In\-for\-ma\-tsi\-on\-nye pro\-tses\-sy} [Information Processses] 10(3):257--274.
\bibitem{10-ag-1} %13
\Aue{Mikheev, P.\,A.} 2014. Analyzing sharing strategies for finite buffer memory in a router among 
outgoing channels. \textit{Automat. Rem. Contr.} 75(10):1814--1825. doi: 10.1134/ S0005117914100087.



\bibitem{16-ag-1} %14
\Aue{Agalarov, Ya.\,M.} 2018. Op\-ti\-mi\-za\-tsiya ob''\-ema bu\-fer\-noy pa\-mya\-ti uz\-la  
kom\-mu\-ta\-tsii pri skhe\-me pol\-no\-go raz\-de\-le\-niya pa\-mya\-ti [Optimization of buffer memory size 
of switching node in mode of full memory sharing]. \textit{Informatika i~ee Primeneniya~--- Inform. Appl.} 
12(4):25--32. doi: 10.14357/19922264180404.
\bibitem{14-ag-1} %15
\Aue{Apostolaki, M., L.~Vanbever, and M.~Ghobadi.} 2019. FAB: Toward flow-aware buffer sharing on 
programmable switches. \textit{ACM Workshop on Buffer Sizing}. 6~p. doi: 10.1145/3375235.3375237. 
Available at:  {\sf https://people. csail.mit.edu/ghobadi/papers/fab\_buffer\_2019.pdf} (accessed July~12, 2023).
\bibitem{15-ag-1} %16
\Aue{Kim, K.} 2019. Numerical study of optimal buffer size and vacation length in $M/G/1/K$ queues with 
multiple vacations. \textit{Int. J. Engineering Technologies Management Research} 6(2):1--13. doi: 
10.29121/ijetmr.v6.i2.2019.350.

\bibitem{17-ag-1}
\Aue{Agalarov, Ya.\,M.} 2023. Ob op\-ti\-mi\-za\-tsii ra\-bo\-ty re\-zerv\-no\-go pri\-bo\-ra  
v~mno\-go\-li\-ney\-noy sis\-te\-me mas\-so\-vo\-go ob\-slu\-zhi\-va\-niya [Optimization of a queue-length 
dependent additional server in the multiserver queue]. \textit{Informatika i~ee Primeneniya~--- Inform. 
Appl.} 17(1):89--95. doi: 10.14357/ 19922264230112.
\bibitem{18-ag-1}
\Aue{Basharin, G.\,P., and K.\,E.~Sa\-mui\-lov.} 1982. \textit{Ob op\-ti\-mal'\-noy struk\-tu\-re bu\-fer\-noy 
pa\-mya\-ti v~se\-tyakh pe\-re\-da\-chi dan\-nykh s~kom\-mu\-ta\-tsiey pa\-ke\-tov} [On the optimal structure 
of buffer memory in data transmission networks with packet commutation]. Moscow. USSR AS Preprint. 70~p.
\bibitem{19-ag-1}
\Aue{Agalarov, Ya.\,M.} 2019. Pri\-znak uni\-mo\-dal'\-nosti tse\-lo\-chis\-len\-noy funk\-tsii  
od\-noy pe\-re\-men\-noy [A~sign of unimodality of an integer function of one variable]. \textit{Obozrenie 
pri\-klad\-noy i~pro\-mysh\-len\-noy ma\-te\-ma\-ti\-ki} [Surveys Applied and Industrial Mathematics] 
26(1):65--66.

\end{thebibliography}

 }
 }

\end{multicols}

\vspace*{-6pt}

\hfill{\small\textit{Received June 1, 2023}} 

\vspace*{-12pt}


\Contrl

\noindent
\textbf{Agalarov Yaver M.} (b.\ 1952)~--- Candidate of Science (PhD) in technology, associate professor, 
leading scientist, Institute of Informatics Problems, Federal Research Center ``Computer Science and 
Control'' of the Russian Academy of Sciences, 44-2~Vavilov Str., Moscow 119333, Russian Federation; 
\mbox{agglar@yandex.ru}
  




\label{end\stat}

\renewcommand{\bibname}{\protect\rm Литература} 