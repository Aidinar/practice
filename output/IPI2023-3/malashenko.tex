\def\stat{malashenko}

\def\tit{АНАЛИЗ ЗАГРУЗКИ МНОГОПОЛЬЗОВАТЕЛЬСКОЙ СЕТИ  ПРИ~РАСЩЕПЛЕНИИ ПОТОКОВ 
ПО~КРАТЧАЙШИМ МАРШРУТАМ}

\def\titkol{Анализ загрузки многопользовательской сети  при~расщеплении потоков 
по кратчайшим маршрутам}

\def\aut{Ю.\,Е.~Малашенко$^1$, И.\,А.~Назарова$^2$}

\def\autkol{Ю.\,Е.~Малашенко, И.\,А.~Назарова}

\titel{\tit}{\aut}{\autkol}{\titkol}

\index{Малашенко Ю.\,Е.}
\index{Назарова И.\,А.}
\index{Malashenko Yu.\,E.}
\index{Nazarova I.\,A.}


%{\renewcommand{\thefootnote}{\fnsymbol{footnote}} \footnotetext[1]
%{Работа выполнена с~использованием инфраструктуры Центра коллективного пользования <<Высокопроизводительные вы\-чис\-ле\-ния и~большие данные>> 
%(ЦКП <<Информатика>>) ФИЦ ИУ РАН (г.~Москва).}}


\renewcommand{\thefootnote}{\arabic{footnote}}
\footnotetext[1]{Федеральный исследовательский центр <<Информатика и~управление>> Российской академии наук, \mbox{malash09@ccas.ru}}
\footnotetext[2]{Федеральный исследовательский центр <<Информатика и~управление>> Российской академии наук, \mbox{irina-nazar@yandex.ru}}


\vspace*{-12pt}


\Abst{В~рамках вычислительных экспериментов на многопродуктовой 
сетевой модели изуча\-ют\-ся два способа передачи потоков различных видов  по 
крат\-чай\-шим марш\-ру\-там. В~первом случае пе\-ре\-да\-ва\-емые межузловые потоки рав\-ны по 
величине.  В~другом~---  определяется не\-диск\-ри\-ми\-ни\-ру\-ющее распределение,   при 
котором  всем  парам корреспондентов  выделяются одинаковые ресурсы. Суммарная  
за\-груз\-ка ребер сети, воз\-ни\-ка\-ющая при одновременной передаче  всех  межузловых 
информационных потоков, считается заданной.  Предложенный  метод позволяет 
получить  гарантированные оценки  удель\-ных за\-трат  ресурсов сети  и~предельно 
до\-пус\-ти\-мых  за\-гру\-зок  ребер  при одновременной   передаче рас\-щеп\-лен\-ных   
межузловых потоков по всем  найденным крат\-чай\-шим марш\-ру\-там.
Приводятся результаты сравнительного анализа урав\-ни\-тель\-но\-го  рас\-пре\-де\-ле\-ния 
потоков и~ресурсов  в~сетях  с~различными структурными  особенностями. 
Ал\-го\-рит\-ми\-че\-ская  схема имеет полиномиальную оценку  тре\-бу\-емо\-го чис\-ла операций.}


\KW{многопродуктовая потоковая модель; распределение 
межузловых потоков и~на\-гру\-зок; предельная за\-груз\-ка сети}

\DOI{10.14357/19922264230305}{NLUSQJ}
  
%\vspace*{-4pt}


\vskip 10pt plus 9pt minus 6pt

\thispagestyle{headings}

\begin{multicols}{2}

\label{st\stat}

\section{Введение}

Данная работа продолжает исследование проб\-лем управ\-ле\-ния потоками 
в~телекоммуникационных территориально рас\-пре\-де\-лен\-ных сис\-те\-мах \mbox{связи}~[1--3]. 
В~рамках вы\-чис\-ли\-тель\-ных экспериментов на многопродуктовой сетевой модели 
анализируются ре\-зуль\-ти\-ру\-ющие за\-груз\-ки ребер при урав\-ни\-тель\-ном способе 
распределения потоков. Под межузловой на\-груз\-кой понимается суммарная \mbox{величина} 
про\-пуск\-ных способностей, вы\-де\-ля\-емых в~сети
для обеспечения передачи информационного потока определенного вида. Сумма всех 
дуговых потоков раз\-ных видов, проходящих по некоторому реб\-ру, трактуется как 
\textit{реберная загрузка}. Общая до\-пус\-ти\-мая за\-груз\-ка ребер сети, воз\-ни\-ка\-ющая при 
одновременной передаче межузловых информационных потоков для всех пар 
корреспондентов, считается заданной.

Анализируются две стратегии управ\-ле\-ния и~диспетчерские правила марш\-ру\-ти\-за\-ции. 
В~первом \mbox{случае} осуществляется распределение межузловых потоков, рав\-ных друг 
другу,  во втором~--- определяется не\-диск\-ри\-ми\-ни\-ру\-ющее распределение, при котором  
всем  парам корреспондентов  выделяются одинаковые ресурсы. При передаче каж\-до\-го 
вида потока по всем крат\-чай\-шим марш\-ру\-там проводится оценка удельных за\-трат. 
Анализ результатов экспериментов поз\-во\-ля\-ет оценить распределение \textit{реберных 
загрузок} для описанных стратегий управ\-ле\-ния в~многопользовательских сетях с~различными структурными особенностями.

При создании, развитии и~эксплуатации те\-ле\-ком\-му\-ни\-ка\-ци\-он\-ных сис\-тем
в~на\-сто\-ящее время используются потоковые модели~[4]. В~част\-ности, указанные 
модели применяются для поиска стратегий управ\-ле\-ния и~по\-стро\-ения диспетчерских 
правил рас\-пре\-де\-ле\-ния потоков, на\-гру\-зок и~ресурсов в~многопользовательских сетях~[5, 6]. 
В~русле исследований по разработке SPLIT-ме\-то\-дов~[7--10] лежат 
пред\-став\-лен\-ные в~разд.~2--4 ал\-го\-рит\-ми\-че\-ские схемы вы\-рав\-ни\-ва\-ния нагрузок 
и~получения урав\-ни\-тель\-ных распределений межузловых потоков по раз\-лич\-ным марш\-ру\-там. 
В~данной работе используются процедуры поиска всех крат\-чай\-ших путей 
с~полиномиальными оцен\-ка\-ми вы\-чис\-ли\-тель\-ных за\-трат~[11]. 

\section{Математическая модель}

Для описания многопользовательской сетевой сис\-те\-мы связи  воспользуемся 
сле\-ду\-ющей математической за\-писью модели передачи многопродуктового потока.
Сеть~$G$ задается \mbox{множествами}  $\langle V, R,  U, P \rangle$:
узлов (вершин) сети  $V \hm= \{ v_1, v_2, \ldots, v_n, \ldots, v_N \}$;
неориентированных ребер $R \hm= \{ r_1, r_2,  \ldots, r_k,  \ldots, r_E \}$;
ориентированных дуг  $U = \{ u_1, u_2, \ldots, u_k, \ldots, u_{2E}\}$;
пар уз\-лов-кор\-рес\-пон\-ден\-тов $P \hm= \{ p_1, p_2,  \ldots, p_M\}$.
Предполагается, что в~сети отсутствуют пет\-ли и~сдво\-ен\-ные реб\-ра.
В~многопользовательской сети~$G$ рас\-смат\-ри\-ва\-ет\-ся $M \hm= N (N\hm-1)$ независимых, 
не\-взаи\-мо\-за\-ме\-ня\-емых и~рав\-но\-прав\-ных межузловых потоков различных ви\-дов.

Ребро $r_k \hm\in R$ со\-еди\-ня\-ет \textit{смеж\-ные} вершины~$v_{n_k}$ и~$v_{j_k}$.
Каж\-до\-му реб\-ру~$r_k$ ставятся в~соответствие две ориентированные дуги~$u_k$ 
и~$u_{k+E}$ из множества~$U$.
Дуги $\{u_k, u_{k+E}\}$ определяют прямое и~обрат\-ное на\-прав\-ле\-ние передачи потока 
по  реб\-ру~$r_k$ меж\-ду концевыми вершинами $v_{n_k}$ и~$v_{j_k}$.

Каждой паре узлов-кор\-рес\-пон\-ден\-тов~$p_m$ из множества~$P$ ставятся в~соответствие:
вер\-ши\-на-ис\-точ\-ник с~номером~$s_m$,  из~$s_m$  входной поток $m$-го вида по\-сту\-па\-ет в~сеть;
вер\-ши\-на-при\-ем\-ник с~номером~${t_m}$, из~${t_m}$ поток $m$-го вида покидает сеть.
Обозначим через~$z_m$ величину \textit{межузлового} потока $m$-го вида, по\-сту\-па\-юще\-го в~сеть через 
узел с~номером~$s_m$ и~по\-ки\-да\-юще\-го сеть из узла с~номером~$t_m$;
$x_{mk}$ и~$x_{m(k + E)}$~---  поток $m$-го вида, который передается по дугам~$u_k$ и~$u_{k + E}$ со\-глас\-но на\-прав\-ле\-нию передачи, 
$x_{mk} \hm\ge 0$, $x_{m(k + E)}\hm\ge 0$, $m \hm= \overline{1, M}$, $k \hm= \overline {1, E}$;
$S(v_n)$~--- множество номеров исходящих дуг, по ним поток покидает узел~$v_n$;
$T(v_n)$~--- множество номеров входящих дуг, по ним поток по\-сту\-па\-ет в~узел~$v_n$.
Со\-став множеств~$S(v_n)$ и~$T(v_n)$ однозначно формируется в~ходе выполнения 
сле\-ду\-ющей процедуры. Пусть некоторое реб\-ро $r_k \hm\in R$ соединяет вершины 
с~номерами~$n$ и~$j$, такими что $n\hm < j$. Тогда ориентированная дуга $u_k \hm= (v_n, v_j)$, 
на\-прав\-лен\-ная из вершины~$v_n$ в~$v_j$, считается \textit{исходящей} из 
вершины~$v_{n}$ и~ее номер~$k$ заносится в~множество~$S(v_n)$, а~дуга~$u_{k+E}$, 
на\-прав\-лен\-ная из~$v_j$ в~$v_n$,~--- \textit{входящей} для~$v_{n}$ и~ее номер $k\hm+E$ помещается в~список~$T(v_n)$.
Дуга~$u_k$ является \textit{входящей} для~$v_j$, и~ее номер~$k$ попадает 
в~$T(v_j)$, а~дуга~$u_{k+E}$~--- \textit{исходящей}, и~номер $k+E$ вносится в~список ис\-хо\-дя\-щих дуг~$S(v_j)$.

Во всех узлах сети $v_n \in V$, $n = \overline{1,N}$,  для каж\-до\-го вида потока долж\-ны 
выполняться условия со\-хра\-не\-ния потоков:
\begin{multline}
\sum\limits_{i \in S(v_n)}{x_{mi}} - \sum\limits_{i \in T(v_n)}{x_{mi}} ={}\\
{}=
\begin{cases}
 z_m, & \mbox{если\ } v_n = v_{s_m}; \\
- z_m, & \mbox{если\ } v_n = v_{t_m}; \\
 0 & \mbox{в\ остальных\ случаях,}
\end{cases}
\\
n = \overline {1, N}\,, \enskip m = \overline {1, M}\,, \enskip x_{mi} \ge 0\,, \enskip z_m \ge 0\,.
\label{e1-ms}
\end{multline}
Величина $z_m$ равна входному межузловому потоку $m$-го вида, проходящему от 
источника к~приемнику пары~$p_m$ при рас\-пре\-де\-ле\-нии  потоков~$x_{mi}$ по дугам 
\mbox{сети}.

Каждому ребру $r_k \in R$ приписывается не\-от\-ри\-ца\-тель\-ное чис\-ло~$d_k$, 
опре\-де\-ля\-ющее суммарный предельно до\-пус\-ти\-мый поток, который мож\-но передать по 
реб\-ру~$r_k$ в~обоих на\-прав\-ле\-ни\-ях. Компоненты век\-то\-ра про\-пуск\-ных способностей   
$\mathbf{d} \hm= (d_1, d_2,  \ldots, d_k,  \ldots, d_E)$~--- положительные чис\-ла $d_k \hm> 0$.  
Век\-тор~$\mathbf{d}$ задает сле\-ду\-ющие ограничения на сумму потоков всех видов, 
пе\-ре\-да\-ва\-емых по реб\-ру~$r_k$ од\-но\-вре\-менно:
\begin{multline}
\sum\limits_{m=1}^{M} {(x_{mk}+ x_{m(k+E)})} \le d_k,  \\
 x_{mk} \ge 0\,, \enskip 
x_{m(k+E)} \ge 0\,, \enskip  k =\overline{1, E}\,. 
\label{e2-ms}
\end{multline}
Ограничения~(\ref{e1-ms}) и~(\ref{e2-ms}) задают вы\-пук\-лое многогранное множество  до\-пус\-ти\-мых значений 
компонент вектора межузловых по\-токов
$\mathbf{z} \hm= (z_1, z_2,  \ldots, z_m,  \ldots, z_M)$:
\begin{multline*}
\mathcal{Z}(\mathbf{d}) ={}\\
{}= \{\mathbf{z} \ge 0 \ |\  \exists\, \mathbf{x} \ge 0: \ 
(\mathbf{z}, \mathbf{x})  \mbox{ удовлетворяют}\ (1), (2)\}.
\end{multline*}

В рамках данной модели про\-пуск\-ная спо\-соб\-ность ребер сети измеряется в~условных 
единицах потока и~трак\-ту\-ет\-ся как \textit{ресурсное} огра\-ни\-че\-ние. Суммарное значение 
про\-пуск\-ной спо\-соб\-ности сети
$ D(0) \hm= \sum\nolimits_{k=1}^{E} d_k$ считается заданным.
Для каждой пары уз\-лов-кор\-рес\-пон\-ден\-тов $p_m \hm\in P$, для некоторого до\-пус\-ти\-мо\-го 
межузлового потока~$\tilde{z}_m$ и~со\-от\-вет\-ст\-ву\-ющих дуговых потоков~$\tilde{x}_{mk}$, 
 $k \hm= \overline{1, 2E}$, ве\-ли\-чина
$$
\tilde{y}_m = \sum\limits_{i=1}^{2E} \tilde{x}_{mi}, \quad m = \overline{1, M}\,,
$$
характеризует \textit{нагрузку} на сеть  при передаче  межузлового потока величиной~$\tilde{z}_m$ 
из уз\-ла-ис\-точ\-ни\-ка~$s_m$  в~узел-при\-ем\-ник~$t_m$. 
Величина~$\tilde{y}_m$ показывает, какая суммарная про\-пуск\-ная спо\-соб\-ность сети 
по\-тре\-бу\-ет\-ся для передачи дуговых потоков~$\tilde{x}_{mk}$. В~рамках модели 
отношение реберных и~межузловых потоков
$ \tilde{w}_m \hm= {\tilde{y}_m}/{\tilde{z}_m}$,  $m \hm= \overline{1, M},$
мож\-но трак\-то\-вать как удельные \textit{затраты}  ресурсов сети при передаче 
единичного   потока $m$-го вида меж\-ду узлами~$s_m$ и~$t_m$ при  дуговых потоках~$\tilde{x}_{mi}$.
Величины $\overline {z}_m \hm= \tilde{z}_m/\tilde{y}_m$ и~$\overline{x}_{mi} \hm= \tilde{x}_{mi}/ \tilde{y}_m$, 
 $m \hm= \overline{1, M}$,  $i \hm= \overline{1, E}$, соответствуют межузловому 
потоку при единичной на\-груз\-ке для пары~$p_m$.

Вводится величина $\tilde{\Delta}_k$, ха\-рак\-те\-ри\-зу\-ющая ре\-бер\-ную суммарную 
\textit{загрузку} (РС-за\-груз\-ку) $k$-го ребра при одновременной передаче всех межузловых 
потоков~$\tilde{z}_m$:
$$
\tilde{\Delta}_k = \sum\limits_{m=1}^{M} \left(\tilde{x}_{mk}+ \tilde{x}_{m(k+E)}\right),  \quad  k =\overline{1, E}\,.  
$$

\section{Распределение межузловых потоков по~кратчайшим маршрутам}

Для оценки минимальных удельных затрат при различных урав\-ни\-тель\-ных стратегиях 
управ\-ле\-ния использовалась SFSR-про\-це\-ду\-ра (\textit{англ}.\ split flow shortest 
route~--- рас\-щеп\-ле\-ние потока по всем крат\-чай\-шим маршрутам), которая
поз\-во\-ля\-ет распределять межузловые потоки по марш\-ру\-там, со\-сто\-ящим из минимального 
чис\-ла ре\-бер.

При проведении эксперимента на пер\-вом этапе для каж\-дой пары узлов $p_m \hm= (s_m, t_m)$ в~сети~$G(0)$ 
определяется множество всех крат\-чай\-ших путей, которые далее 
используются как марш\-ру\-ты передачи $m$-го вида потока
$$
H^0(m) = \left\{ h^1(m), h^2(m), \ldots, h^j(m), \ldots, h^{J(m)}(m)\right\}\!.
$$ 

\vspace*{-3pt}

\noindent
Здесь $h^j(m)\hm = 
\{k_1^j(m), \ k_2^j(m),  \ldots, k_{\mu(m)}^j(m)\}$~--- список номеров ребер в~$j$-м крат\-чай\-шем пути меж\-ду узлами~$s_m$ и~$t_m$,  
где $\mu(m)$~--- чис\-ло ребер в~кратчайшем маршруте~$h^j(m)$; $J(m)$~--- чис\-ло 
крат\-чай\-ших марш\-ру\-тов для $m$-й \mbox{пары}.

Для оценки возможности <<расщепления>>  потока по раз\-лич\-ным марш\-ру\-там на пер\-вом 
этапе для каж\-дой пары $p_m \hm\in P$ по каж\-до\-му маршруту~$h^j(m)$ из~$H^0(m)$ 
передается межузловой поток $z_m^j \hm= 1$. Под\-счи\-ты\-ва\-ет\-ся величина~$z_m^0$, 
чис\-лен\-но рав\-ная сумме единичных потоков, которые одновременно передаются по всем 
крат\-чай\-шим маршрутам из узла~$s_m$ в~узел~$t_m$:  
$$
z_m^0 = \sum\limits_{j=1}^{J(m)} z_m^j = J(m)\,.
$$

\vspace*{-3pt}


Вычисляется нормирующий коэффициент 
$$
\xi_m^0 = \fr{1}{z_{m}^0}\,,\enskip z_{m}^0 \not = 0\,,\enskip m = \overline{1, M}\,,
$$

\vspace*{-3pt}

\noindent
 значения индикаторной \mbox{функции}
$$ 
\eta_k^j(m) = \begin{cases}
 1, & \mbox{если}\ r_k \in R, \enskip k \in h^j(m)\,; \\
 0 & \mbox{в\ остальных\ случаях}
\end{cases} 
$$

\vspace*{-3pt}

\noindent
и подсчитываются реберные за\-грузки
$$
\Delta_k^0(m) = \xi_m^0 \sum\limits_{j=1}^{J(m)} \eta_k^j(m), \enskip  m = \overline{1, M}\,, \enskip k =\overline{1, E}\,,
$$

\vspace*{-3pt}

\noindent
которые возникнут при передаче единичного межузлового потока из узла~$s_m$ в~узел~$t_m$ одновременно по всем крат\-чай\-шим марш\-ру\-там из~$H^0(m)$.

В ходе вы\-чис\-ли\-тель\-ных экспериментов предполагалось, что суммарная за\-груз\-ка всех ребер сети не может превышать~$D(0)$. Для 
заданного~$D(0)$ под\-счи\-ты\-ва\-ют\-ся предельно до\-пус\-ти\-мые РС-за\-грузки:
$$
\alpha^{**}(0) \sum\limits_{m=1}^{M} \Delta_k^0(m)  = \Delta_k^{*}(z), \quad
\sum\limits_{k=1}^{E} \Delta_k^{*}(z) = D(0). 
$$

Реберные суммарные загрузки~$\Delta_k^{*}(z)$ возникают в~сети при одновременной передаче  межузловых потоков
$ z_m^{**} \hm= \alpha^{**}(0)$ для всех пар $p_m \hm\in P$ и~суммарного межузлового 
потока $\sum\nolimits_{m=1}^{M} z_m^{**} \hm= M \alpha^{**}(0)$.

Для оценки РС-за\-гру\-зок при \textit{уравнительном } рас\-пре\-де\-ле\-нии ре\-зуль\-ти\-ру\-ющих 
межузловых на\-гру\-зок для всех пар уз\-лов $p_m \hm\in P$  так\-же используется SFSR-про\-це\-ду\-ра. 
На первом этапе определяются ре\-бер\-ные за\-груз\-ки для $m\hm = \overline{1, M}$ и~$k \hm= \overline{1, E}$ 
при передаче единичных потоков по всем крат\-чай\-шим марш\-ру\-там из~$H^0(m)$, $m \hm= \overline{1, M}$. 
Для каж\-дой пары $p_m \hm\in P$ вы\-чис\-ля\-ют\-ся на\-грузки
$$ 
y_m^0 = \sum\limits_{k=1}^{E} \sum\limits_{j=1}^{J(m)} \eta_k^j(m) = \mu(m)J(m), \quad m = \overline{1, M}\,, 
$$
нормирующие коэффициенты
$$
 \theta_m^0 = \fr{1}{y_m^0}\,,\enskip  y_m^0 \not = 0\,,\enskip  m = \overline{1, M}\,,
 $$
и \textit{загрузки}
$$
\Delta_k^0(m) = \theta_m^0 \sum\limits_{j=1}^{J(m)} \eta_k^j(m), \enskip  m = \overline{1, M}\,, \enskip 
k =\overline{1, E}\,,
$$
при которых по всем марш\-ру\-там~$H^0(m)$, $m \hm= \overline{1, M}$, ре\-зуль\-ти\-ру\-ющие 
межузловые нагрузки $y_m^1 \hm= \theta_m^0 y_m^0$ будут рав\-ны единице. Определяются РС-за\-грузки
$$
\beta^{**}  \sum\limits_{j=1}^{M} \Delta_k^0(m) = \Delta_k^{*}(y), \quad
 \sum\limits_{k=1}^{E}  \Delta_k^{*}(y) = D(0) 
 $$
и межузловые потоки
$$
z_m\left(y^{**}\right) = \beta^{**}\theta_m^0 z_{m}^0,\enskip m = \overline{1, M}\,.
$$ 



   
\section{Вычислительный эксперимент}

Вычислительный эксперимент проводился на моделях сетевых сис\-тем, пред\-став\-лен\-ных 
на рис.~1. В~каж\-дой сети~69~узлов. В~ходе вы\-чис\-ли\-тель\-но\-го эксперимента 
проводилась нормировка, и~суммарная про\-пуск\-ная спо\-соб\-ность в~обеих сетях была 
оди\-на\-кова:
$$
\sum\limits_{k=1}^{E} \Delta_k^*(\cdot) = D(0)= 68\,256\,.
$$

Результаты вычислительных экспериментов с~использованием SFSR-про\-це\-ду\-ры 
пред\-став\-ле\-ны\linebreak\vspace*{-12pt}

\pagebreak

\end{multicols}

\begin{figure*} %fig1
\vspace*{1pt}
\begin{center}
   \mbox{%
\epsfxsize=153.408mm 
\epsfbox{mash-1.eps}
}
\end{center}
\vspace*{-9pt}
\Caption{Результирующие реберные за\-груз\-ки в~базовой~(\textit{а}) и~в~кольцевой~сети~(\textit{б})}
%\end{figure*}
%\begin{figure*} %fig2
\vspace*{6pt}
\begin{center}
   \mbox{%
\epsfxsize=163mm 
\epsfbox{mash-3.eps}
}
\end{center}
\vspace*{-9pt}
\Caption{Распределение ре\-бер\-ных за\-гру\-зок в~базовой~(\textit{а}) и~в~кольцевой~сети~(\textit{б})}
\end{figure*}

\begin{table*}\small
\vspace*{6pt}
\begin{center}
\begin{tabular}{|l|c|c|c|c|} 
\multicolumn{5}{p{98mm}}{Значения потоков и загрузок в базовой и кольцевой сети (SFSR-cтра\-те\-гия)}\\
\multicolumn{5}{c}{\ }\\[-6pt]
\hline
 \multicolumn{1}{|c|}{Сеть} &  \tabcolsep=0pt\begin{tabular}{c} Медиана\\  потоков\end{tabular} & 
   \tabcolsep=0pt\begin{tabular}{c} Сумма\\ межузловых\\ потоков\end{tabular} & 
   \tabcolsep=0pt\begin{tabular}{c}Удельные\\ затраты\end{tabular} &  
 \tabcolsep=0pt\begin{tabular}{c}   Норма вектора\\ РС-загрузок\end{tabular}\\
    \hline
    Базовая & &&&\\
      \hspace*{5mm}$z^{**}$   & 1,85& \hphantom{9}8\,670& 7,9& 9210\\
      \hspace*{5mm}$z(y^{**})$ & 1,82& 12\,810& 5,3& 9000\\
      \hline
     Кольцевая &&&&\\
     \hspace*{5mm}$z^{**}$   &  2,5\hphantom{9}& 11\,750& 5,8& 9520\\
         \hspace*{5mm}$z(y^{**})$  & 2,4\hphantom{9}& 15\,650& 4,4& 9160\\
        \hline
    \end{tabular}
    \end{center}
    \vspace*{-9pt}
    \end{table*}


\begin{multicols}{2}


\noindent
 на рис.~1 и~2. В~таб\-ли\-це и~на рис.~1 значения потоков и~за\-гру\-зок 
указаны в~единицах потока, а~на рис.~2~--- в~относительных единицах. По 
завершении экспериментов на осно\-ве полученных значений $\Delta_k^*(z)$ 
(за\-груз\-ки сети при урав\-ни\-тель\-ном рас\-пре\-де\-ле\-нии межузловых потоков) 
и~$\Delta_k^*(y)$ (загрузок при урав\-ни\-тель\-ном распределении межузловых на\-гру\-зок) вы\-чис\-ля\-ют\-ся

 \vspace*{-4pt}
 
 \noindent
\begin{multline*}
\Delta_{\min}(z) = \min_k\{\Delta_k^*(z)\}, \enskip
 \Delta_{\min}(y) =  \min_k\{\Delta_k^*(y)\},\\
    k= \overline{1, E}\,;
\end{multline*}

\vspace*{-15pt}

\columnbreak

\noindent
\begin{multline*}
\delta_k(z) = \fr{\Delta_k^*(z)}{\Delta_{\min}(z)}, \enskip
 \delta_k(y) = \fr{\Delta_k^*(y)}{\Delta_{\min}(y)}, \\[4pt]
  k= \overline{1, E},\enskip \Delta_{\min}(\cdot) \not = 0\,.
\end{multline*}

 %\vspace*{-22pt}

 


Результирующие диаграммы пред\-став\-ле\-ны на рис.~2. На диаграммах по 
вертикальной оси откладываются значения~$\delta_k(\cdot)$ (упорядоченные по
не\-воз\-рас\-та\-нию), а~по горизонтальной ука\-зы\-ва\-ют\-ся порядковые номера ре\-бер в~данной 
упорядоченной по\-сле\-до\-ва\-тель\-ности $\pi(k) \hm= k/E$  для $k \hm= \overline{1, E}$, где $E \hm= 
72$~--- чис\-ло ребер для базовой сети (см.\ рис.~2,\,\textit{а}) и~$E \hm= 80$~--- для кольцевой (см.\ рис.~2,\,\textit{б}). 
Значения $\delta_k(\cdot) \hm= 1$ в~правой час\-ти диаграммы относятся к~реб\-рам, 
инцидентным висячим узлам, а~в~левой час\-ти диаграммы~--- к~реб\-рам с~максимальной за\-груз\-кой.
{\looseness=1

}

Схематичные изображения сетей на рис.~1 соответствуют рас\-пре\-де\-ле\-нию 
относительных за\-гру\-зок~$\delta_k(\cdot)$ на реб\-рах сети. В~базовой сети (см.\ рис.~1,\,\textit{а}) 
наибольшая за\-груз\-ка на коль\-це и~реб\-рах, исходящих из цент\-ра сети. В~кольцевой 
сети (см.\ рис.~1,\,\textit{б}) максимальная за\-груз\-ка приходится на дополнительные реб\-ра 
внут\-рен\-не\-го коль\-ца и~на реб\-ра, со\-еди\-ня\-ющие внеш\-нее кольцо с~внут\-рен\-ним. В~обеих 
сетях единичная относительная за\-груз\-ка на висячих реб\-рах соответствует 
собственным информационным потокам при отсутствии тран\-зит\-ных.
{\looseness=1

}


Результаты, собранные в~таб\-ли\-це, свидетельствуют о~том, что как суммарное, так 
и~медианное значение межузловых потоков значительно больше в~коль\-це\-вой сети при 
рав\-ной сум\-мар\-ной допустимой за\-груз\-ке в~обеих сетях.
Удельные за\-тра\-ты на передачу потоков при рав\-ных межузловых на\-груз\-ках достигают 
минимального значения в~кольцевой сети. Медианные значения реберной за\-груз\-ки 
в~базовой сети на~50\% больше, чем в~кольцевой. 


%\vspace*{-6pt}

\section{Заключение}

%\vspace*{-6pt}

В работе анализируются распределения реберных за\-гру\-зок и~проводится оценка 
удельных за-\linebreak трат ресурсов при одновременной передаче межузловых потоков разных 
видов и~урав\-ни\-тель\-ном\linebreak способе управ\-ле\-ния. Полученные значения медиан  
распределений межузловых потоков мож\-но интерпретировать как пред\-ста\-ви\-тель\-ные 
оценки функциональных возможностей сети в~условиях не\-опре\-де\-лен\-ности, 
не\-вза\-и\-мо\-за\-ме\-ня\-е\-мости потоков и~\mbox{рав\-но\-прав\-ности} корреспондентов. Все межузловые 
потоки, не пре\-вы\-ша\-ющие медианных значений,
мож\-но передать в~сети одновременно. В~кольцевой сети удельные за\-тра\-ты на 
передачу потоков и~медианные значения ре\-бер\-ных загрузок значительно меньше, чем 
в~базовой. Сред\-ние показатели удельных за\-трат позволяют
оценивать эф\-фек\-тив\-ность использования ресурсов при изменении структуры \mbox{сети}.

Предложенный метод расщепления потоков по всем крат\-чай\-шим марш\-ру\-там дает 
воз\-мож\-ность анализировать за\-груз\-ку сети с~учетом рас\-щеп\-ле\-ния каж\-до\-го вида 
потока. Полученные агрегированные показатели для узлов сети могут быть 
использованы при по\-стро\-ении марш\-рут\-но-ад\-рес\-ных таб\-лиц, подготовке графиков 
обходов и~замен для работы в~аварийных режимах и~при критически опас\-ных 
по\-вреж\-де\-ниях.

Описанный метод позволяет проводить пред\-ва\-ри\-тель\-ную оцен\-ку проекта сети, 
по\-стро\-ен\-ной на базе арендованных каналов связи при со\-хра\-не\-нии их общего чис\-ла. 
Вы\-чис\-ли\-тель\-ные эксперименты показали, что изменение структуры и~за\-груз\-ки ребер 
при переходе от базовой к~кольцевой поз\-во\-ля\-ет значительно увеличить межузловые 
потоки, хотя общее чис\-ло арен\-ду\-емых каналов совпадает в~обеих се\-тях. 

%\vspace*{-6pt}

{\small\frenchspacing
 { %\baselineskip=12pt
 %\addcontentsline{toc}{section}{References}
 \begin{thebibliography}{99}

    
    \bibitem{Mal22-5} %1
    \Au{Малашенко~Ю.\,Е., Назарова~И.\,А.} Оцен\-ка 
предельных рас\-пре\-де\-ле\-ний  пропускной спо\-соб\-ности в~многопользовательской сети 
при  передаче  межузловых потоков  по кратчайшим  марш\-ру\-там~// Известия РАН. 
Тео\-рия и~сис\-те\-мы управ\-ле\-ния, 2022. №\,5. С.~79--89.

\bibitem{Mal22-3}  %2
\Au{Малашенко~Ю.\,Е., На\-за\-ро\-ва~И.\,А.} Анализ 
рас\-пре\-де\-ле\-ния на\-груз\-ки и~межузловых потоков при различных стратегиях  
марш\-ру\-ти\-за\-ции  в~многопользовательской сети~// Известия РАН. Тео\-рия и~сис\-те\-мы 
управ\-ле\-ния, 2022. №\,6. С.~112--122.
    
    \bibitem{Mal23-2} %3
    \Au{Малашенко~Ю.\,Е., На\-за\-ро\-ва~И.\,А.} 
Срав\-ни\-тель\-ный анализ стратегий  управ\-ле\-ния потоками в~многопользовательской 
телекоммуникационной сети~// Известия РАН. Тео\-рия и~сис\-те\-мы управ\-ле\-ния, 2023.  
№~2. С.~164--176.
    
    \bibitem{Salimifard2020}  %4
    \Au{Salimifard~K., Bigharaz~S.} The 
multicommodity network flow problem: State of the art classification, 
applications, and solution methods~// Operational Research, 2020. Vol.~22. Iss.~1.  P.~1--47.
doi: 10.1007/s12351-020-00564-8.
    
    \bibitem{Luss2012}  %5
    \Au{Luss H.} Equitable resource allocation: 
Models, algorithms, and applications.~--- Hoboken, NJ, USA: John Wiley \& Sons, 
2012. 376~p.
    
    \bibitem{Balakrishnan2017}   %6
    \Au{Balakrishnan~A., Li~G., Mirchandani~P.}  
    Optimal network design with end-to-end service requirements~// Oper. Res., 
2017. Vol.~65. Iss.~3. P.~729--750. doi: 10.1287/opre. 2016.1579.
    
    
    
    \bibitem{Caramia10}  %7
    \Au{Caramia M., Sgalambro~A.} A~fast heuristic 
algorithm for the maximum concurrent k-splittable flow problem~// Optim. Lett., 
2010. Vol.~4. Iss.~1. P.~37--55. doi: 10.1007/s11590-009-0147-4.


    
    \bibitem{Kabadurmus2016}   %8
    \Au{Kabadurmus~O., Smith~A.\,E.}  
Multicommodity k-splittable survivable network design problems with relays~// 
Telecommun. Syst., 2016. Vol.~62. Iss.~1. P.~123--133. doi: 10.1007/s11235-015-0067-9.

\bibitem{Bialon2017}  %9
    \Au{Bialon P.} A~randomized rounding approach 
to a~\mbox{k-splittable} multicommodity flow problem with lower path flow bounds 
affording solution quality guarantees~// Telecommun. Syst., 2017. Vol.~64. 
Iss.~3. P.~525--542.  doi: 10.1007/s11235-016-0190-2.

\bibitem{Melchiori20}   %10
\Au{ Melchiori~A., Sgalambro~A.} A~branch and price 
algorithm to solve the quickest multicommodity k-splittable flow problem~// 
Eur. J. Oper. Res., 2020. Vol.~282. Iss.~3. P.~846--857. doi: 10.1016/j.ejor.2019.10.016.
    
    \bibitem{Kormen} %11
     \Au{Кормен~Т.\,Х., Лей\-зер\-сон~Ч.\,И., Ри\-вест~Р.\,Л., Штайн~К.} 
    Ал\-го\-рит\-мы: по\-стро\-ение и~анализ~/ Пер. с~англ.~--- М.: Вильямс, 2005. 1296~с.
    (\Au{Cormen~T.\,H., Leiserson~C.\,E., Rivest~R.\,L., Stein~C.}
{Introduction to algorithms}.~--- 2nd ed.~--- Cambridge, London: The MIT Press,  2001.\linebreak  1180~p.)
    \end{thebibliography}

 }
 }

\end{multicols}

\vspace*{-11pt}

\hfill{\small\textit{Поступила в~редакцию 31.05.23}}

\vspace*{6pt}


%\newpage

%\vspace*{-28pt}

\hrule

\vspace*{2pt}

\hrule

\vspace*{-2pt}

\def\tit{MULTIUSER NETWORK LOAD ANALYSIS BY~SPLITTING FLOWS ALONG~THE~SHORTEST ROUTES\\[-4pt]}


\def\titkol{Multiuser network load analysis by~splitting flows along~the~shortest routes}


\def\aut{Yu.\,E.~Malashenko and I.\,A.~Nazarova}

\def\autkol{Yu.\,E.~Malashenko and I.\,A.~Nazarova}

\titel{\tit}{\aut}{\autkol}{\titkol}

\vspace*{-14pt}


\noindent
Federal Research Center ``Computer Science and Control'' of the Russian Academy of Sciences, 44-2~Vavilov
Str., Moscow 119333, Russian Federation



\def\leftfootline{\small{\textbf{\thepage}
\hfill INFORMATIKA I EE PRIMENENIYA~--- INFORMATICS AND
APPLICATIONS\ \ \ 2023\ \ \ volume~17\ \ \ issue\ 3}
}%
 \def\rightfootline{\small{INFORMATIKA I EE PRIMENENIYA~---
INFORMATICS AND APPLICATIONS\ \ \ 2023\ \ \ volume~17\ \ \ issue\ 3
\hfill \textbf{\thepage}}}

\vspace*{2pt}





\Abste{In computational experiments on a~multicommodity network model, two ways of transmitting flows of different types along shortest routes
 are investigated. In the first case, the transmitted internodal flows are equal in magnitude.  In the other~--- 
 a~nondiscriminatory distribution is defined in which all pairs of nodes are distributed the same resources. 
 The total load of the network edges resulting from the simultaneous transmission of all internodal flows is considered to be given. 
 The proposed method allows one to obtain guaranteed estimates of the specific resource costs of the network and the maximum feasible load 
 of the edges while transmitting split internodal flows along the shortest routes found. The results of a comparative analysis of 
 the equalization distribution of flows and resources in networks with different structural 
features are given. The algorithmic scheme has a~polynomial estimate of the required number of operations. } 


\KWE{multicommodity flow model; distribution of internodal flows and loads; network peak load}

\DOI{10.14357/19922264230305}{NLUSQJ}

%\vspace*{-20pt}

%\Ack
%\noindent

  

\vspace*{-6pt}

  \begin{multicols}{2}

\renewcommand{\bibname}{\protect\rmfamily References}
%\renewcommand{\bibname}{\large\protect\rm References}

{\small\frenchspacing
 {\baselineskip=10.5pt
 \addcontentsline{toc}{section}{References}
 \begin{thebibliography}{99} 
 
 \bibitem{Mal22-5-1} %1
\Aue{Malashenko, Y.\,E., and I.\,A.~Nazarova.}
 2022. Estimate of resource distribution with the shortest paths in the multiuser network. 
 \textit{J.~Comput. Sys. Sc. Int.} 61(4):599--610. doi: 10.1134/S106423072204013X.

\bibitem{Mal22-3-1}    %2
\Aue{Malashenko, Y.\,E., and I.\,A.~Nazarova.}
 2022. Analysis of the load distribution and internodal flows under different routing strategies in a~multiuser network. 
 \textit{J.~Comput. Sys. Sc. Int.} 61(6): 970--980. doi: 10.1134/ S1064230722060132.
    

    
\bibitem{Mal23-2-1} %3
\Aue{Malashenko, Y.\,E., and I.\,A.~Nazarova.}
 2023. Quantitative analysis of flow distributions in a~multi-user telecommunication network. 
 \textit{J.~Comput. Sys. Sc. Int.} 62(2):\linebreak 324--335.
    
\bibitem{Salimifard2020-1}  %4
\Aue{Salimifard, K., and S.~Bigharaz.}
  2020. The multicommodity network flow problem: State of the art classification, applications, and solution methods.
   \textit{Operational Research} 22(1):1--47. doi: 10.1007/s12351-020-00564-8.
    
\bibitem{Luss2012-1}   %5
\Aue{Luss, H.} 
2012. \textit{Equitable resource allocation: Models, algorithms, and applications}. Hoboken, NJ: John Wiley \&~Sons. 376~p.
    
\bibitem{Balakrishnan2017-1}   %6
\Aue{Balakrishnan, A., G.~Li, and P.~Mirchandani.}
 2017. Optimal network design with end-to-end service requirements. \textit{Oper. Res.} 65(3):729--750. doi: 10.1287/opre.2016.1579.
    

    
\bibitem{Caramia10-1} %7
\Aue{Caramia, M., and A.~Sgalambro.} 
2010. A~fast heuristic algorithm for the maximum concurrent k-splittable flow problem. 
\textit{Optim. Lett.} 4(1):37--55. doi: 10.1007/s11590-009-0147-4.


    
\bibitem{Kabadurmus2016-1} %8
  \Aue{Kabadurmus, O., and A.\,E.~Smith.}
   2016. Multicommodity k-splittable survivable network design problems with relays. 
   \textit{Telecommun. Syst.} 62(1):123--133. doi: 10.1007/ s11235-015-0067-9.
   
   \bibitem{Bialon2017-1} %9 
\Aue{Bialon, P.} 2017. A~randomized rounding approach to a~k-splittable multicommodity flow problem 
with lower path flow bounds affording solution quality guarantees.
 \textit{Telecommun. Syst.} 64(3):525--542. doi: 10.1007/s11235-016-0190-2.
   
   \bibitem{Melchiori20-1} %10
\Aue{Melchiori, A., and A.~Sgalambro.} 
2020. A~branch and price algorithm to solve the quickest multicommodity \mbox{k-splittable} flow problem. 
\textit{Eur. J. Oper. Res.} 282(3):846--857. doi: 10.1016/j.ejor.2019.10.016.
    
\bibitem{Kormen-1}   %11
\Aue{Cormen, T.\,H., C.\,E.~Leiserson, R.\,L.~Rivest, and C.~Stein.}
 2001. \textit{Introduction to algorithms}. 2nd ed. Cambridge, London: The MIT Press.  1180~p.
\end{thebibliography}

 }
 }

\end{multicols}

\vspace*{-9pt}

\hfill{\small\textit{Received May 31, 2023}} 

\vspace*{-22pt}
  


\Contr

\vspace*{-4pt}

\noindent
\textbf{Malashenko Yuri E.} (b.\ 1946)~--- 
Doctor of Science in physics and mathematics, principal scientist, Federal Research Center ``Computer Science and Control'' 
of the Russian Academy of Sciences, 44-2~Vavilov Str., Moscow 119333, Russian Federation; \mbox{malash09@ccas.ru} 

%\vspace*{3pt}

\noindent
\textbf{Nazarova Irina A.} (b.\ 1966)~--- Candidate of Science (PhD) in physics and mathematics, scientist, 
Federal Research Center ``Computer Science and Control'' of the Russian Academy of Sciences, 44-2~Vavilov Str., Moscow 119333, Russian Federation; 
\mbox{irina-nazar@yandex.ru}


\label{end\stat}

\renewcommand{\bibname}{\protect\rm Литература} 