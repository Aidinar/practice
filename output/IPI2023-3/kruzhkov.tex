\def\stat{inkova}

\def\tit{КРИТЕРИИ ОПРЕДЕЛЕНИЯ СЕМАНТИЧЕСКОЙ БЛИЗОСТИ ДИСКУРСИВНЫХ 
ОТНОШЕНИЙ$^*$}

\def\titkol{Критерии определения семантической близости дискурсивных 
отношений}

\def\aut{О.\,Ю.~Инькова$^1$, М.\,Г.~Кружков$^2$}

\def\autkol{О.\,Ю.~Инькова, М.\,Г.~Кружков}

\titel{\tit}{\aut}{\autkol}{\titkol}

\index{Инькова О.\,Ю.}
\index{Кружков М.\,Г.}
\index{Inkova O.\,Yu.}
\index{Kruzhkov M.\,G.}


{\renewcommand{\thefootnote}{\fnsymbol{footnote}} \footnotetext[1]
{Работа выполнялась с~использованием инфраструктуры Центра коллективного пользования <<Высокопроизводительные вы\-чис\-ле\-ния и~большие данные>> 
(ЦКП <<Информатика>>) ФИЦ ИУ РАН (г.~Москва).}}


\renewcommand{\thefootnote}{\arabic{footnote}}
\footnotetext[1]{Федеральный исследовательский центр <<Информатика и~управ\-ле\-ние>> Российской академии наук; Женевский 
университет, \mbox{olyainkova@yandex.ru}}
\footnotetext[2]{Федеральный исследовательский центр <<Информатика и~управ\-ле\-ние>> Российской академии наук, 
\mbox{magnit75@yandex.ru}}


\vspace*{-12pt}

  \Abst{Работа посвящена результатам разработки структурированных определений 
дискурсивных отношений на осно\-ве их классификации, а~так\-же критериям, поз\-во\-ля\-ющим 
определить их семантическую бли\-зость. Авторы показывают недостатки су\-щест\-ву\-ющих 
подходов, которые приводят к~противоречивым или час\-то необоснованным результатам, 
а~так\-же раскрывают преимущества альтернативного решения: классификации дискурсивных 
отношений на осно\-ве их структурированных определений. Приводятся примеры таких 
определений, сформированных в~Надкорпусной базе данных коннекторов (НБДК), а~так\-же 
критерии, поз\-во\-ля\-ющие определить семантическую бли\-зость дискурсивных отношений. 
Поскольку структурированные определения пред\-став\-ля\-ют собой набор различительных 
признаков, авторы об\-суж\-да\-ют проб\-ле\-му при\-сво\-ения коэффициента бли\-зости каж\-до\-му из 
признаков. Полученные данные, в~том чис\-ле количественные, поз\-во\-ля\-ют вы\-дви\-нуть 
гипотезу, со\-глас\-но которой из трех групп признаков: <<Уровень>>, <<Базовая 
операция>> и~<<Семейство признаков>>~--- наибольший вес имеет по\-след\-няя. 
Предлагаются пути дальнейшего исследования этой проб\-ле\-мы, в~част\-ности с~учетом таких 
факторов, как данные по со\-че\-та\-е\-мости дискурсивных отношений, по соответствиям 
дискурсивных отношений и~их показателей в~текс\-те оригинала и~в~текс\-те перевода, а~так\-же 
тех случаев, когда один показатель может выражать несколько дискурсивных отношений.} 
  
  \KW{надкорпусная база данных; ло\-ги\-ко-се\-ман\-ти\-че\-ские отношения; коннекторы; 
аннотирование; фасетная классификация}

\DOI{10.14357/19922264230314}{UJZJZI}
  
%\vspace*{-4pt}


\vskip 10pt plus 9pt minus 6pt

\thispagestyle{headings}

\begin{multicols}{2}

\label{st\stat}

\section{Определения дискурсивных отношений и~вопросы их~семантической близости}


  Вопросы, связанные с~изучением дискурсивных\linebreak отношений и~их показателей, 
коннекторов, не те\-ря\-ют своей ак\-ту\-аль\-ности в~силу того, что эти от\-но\-ше\-ния 
играют значительную роль в~обеспечении связ\-ности текс\-та~[1--3], а~список их 
показателей постоянно расширяется (ср.\ для русского языка, например,~[4, 
с.~663--686; 5] и~др.), что за\-труд\-ня\-ет автоматическую обработку текс\-та. Кроме 
того, в~су\-щест\-ву\-ющих подходах дискурсивные отношения определены, как 
правило, довольно противоречиво, по\-сколь\-ку нет чет\-ких критериев для их 
выделения (см., например, наиболее известную в~данной об\-ласти тео\-рию 
риторических отношений~[6], а~так\-же~[7--9]). 

В~некоторых 
классификациях~[10, 11] отношения оформлены в~виде иерархической 
структуры, однако критерии объ\-еди\-не\-ния отношения в~группы и,~в~част\-ности, 
критерий выделения того или иного чис\-ла выс\-ших иерархических уров\-ней, 
отсутствуют. Так, в~классификации, ис\-поль\-зу\-емой в~Пенсильванском 
аннотированном корпусе (Penn Discourse Treebank, RDTB)~[10], выс\-ших клас\-си\-фи\-ци\-ру\-ющих уров\-ней четыре: 
Temporal, Contingency, Comparison, Expansion. Если основания для выделения 
уров\-ней Temporal и~Contingency интуитивно ясны, то основания для 
объединения отношений в~классы Comparison и~Expansion не\-оче\-вид\-ны. 

Вызывает, например, воз\-ра\-же\-ние отнесение к~группе Comparison 
уступительных отношений (\textit{Хотя на улице дождь, Петя не захотел 
взять зонтик}), в~основе интерпретации которых лежит не сравнение, 
а~импликация (Если на улице идет дождь, то мы, как правило, берем зонтик), 
а~значит, им мес\-то в~классе Contingency вмес\-те с~причинными и~условными 
отношениями (подробнее см.~[12]). 

В~результате в~одном классе оказываются 
семантически далекие отношения: например, в~классе Expansion оказывается 
отношение альтернативы и~спецификации; ср.\ \textit{Она учится 
в~университете или работает?} с~отношением альтернативы и~\textit{Подари 
ей четвероного друга, например хомяка} с~отношением спецификации. 

И~наоборот: семантически близ\-кие отношения оказываются в~раз\-ных группах, 
как в~случае усту\-питель\-ных и~услов\-ных отношений. Отсутствие критериев для 
выделения иерархических уровней и~отнесения к~ним того или иного набора 
ло\-ги\-ко-се\-ман\-ти\-че\-ских отношений (ЛСО) приводит так\-же к~тому, что 
классификация, ис\-поль\-зу\-емая в~PDTB, 
претерпевает по\-сто\-ян\-ные изменения; ср.\ две ее версии, пред\-став\-лен\-ные в~[10] 
и~[13].
  
  Основные проблемы классификации дискурсивных отношений описаны 
в~предыду\-щей работе авторов~[14], где пред\-став\-ле\-ны так\-же пер\-вые результаты 
ее альтернативного решения, в~част\-ности классификация, ис\-поль\-зу\-емая 
в~%Надкорпусной базе данных коннекторов (
НБДК, разработанной в~ИПИ ФИЦ 
ИУ РАН\footnote{Подробнее о структуре НБД, ее возможностях и~результатах, 
полученных с~ее использованием, см., например, [15--17]. Пред\-ста\-ви\-тель\-ный фрагмент НБД 
до\-сту\-пен по адресу: {\sf http://a179.frccsc.ru/RFH41002/main.aspx}.}, и~созданные на ее основе 
структурированные определения дис\-кур\-сив\-ных отношений, или, в~терминологии авторов, 
ЛСО. Коротко пе\-ре\-чис\-лим основные положения данного под\-хода.
  
  В основе классификации лежат четыре базовые семантические операции, на 
которые опирается то или иное ЛСО: импликация; расположение на шкале 
времени; срав\-не\-ние; соотнесение част\-но\-го и~общего или элемента и~множества. 
Классификация различает так\-же уровни, на которых может быть уста\-нов\-ле\-но 
ЛСО: пропозициональный уровень, уровень вы\-ска\-зы\-ва\-ния (иллокутивный), 
метаязыковой (по\-дроб\-нее см.~[12]). Каждое ЛСО определяется, следовательно, 
на основе этих двух критериев, к~которым до\-бав\-ля\-ет\-ся критерий, 
характеризующий ЛСО, основанные на импликации и~сравнении: по\-ляр\-ность, 
т.\,е.\ уста\-нав\-ли\-ва\-ет\-ся ли ЛСО непосредственно между положениями 
вещей~$p$ и~$q$, описанными в~свя\-зы\-ва\-емых им фрагментах текс\-та, или же 
при его интерпретации долж\-ны быть учтены также их отрицательные 
корреляты $\neg p$ и~$\neg q$. 
Учитываются и~семантические, 
и~праг\-ма\-ти\-че\-ские характеристики кон\-текста.
  
\section{Структурированные определения логико-семантических отношений}

  Разработанная концепция классификации дает воз\-мож\-ность описывать ЛСО 
при помощи структурированных определений, пред\-став\-ля\-ющих собой набор 
раз\-ли\-чи\-тель\-ных признаков. На момент на\-писания \mbox{статьи} в~НБДК 
описаны~26~ЛСО. Определения ЛСО пропозициональной альтернативы 
и~спецификации приведены в~табл.~1 (другие определения см.\ в~[14, 18] 
и~ниже).
  
  
  Структурированные определения ЛСО пропозициональной альтернативы 
и~спецификации\linebreak показывают, что они имеют лишь один общий 
различительный признак: оба они уста\-нов\-ле\-ны на пропозициональном уров\-не. 
Этого недостаточно для того, чтобы отнести их к~единому иерархическому 
уров\-ню, как предлагается в~PDTB (см.\ обоснование в~разд.~3). Напротив, 
у~уступительных и~условных ЛСО, ока\-зы\-ва\-ющих\-ся в~PDTB на раз\-ных уров\-нях, 
общих признаков больше (табл.~2). В~част\-ности, общим для них является помимо 
пропозиционального уров\-ня базовая операция.
  
  
 В этой связи следует заметить, что если некоторые различительные при\-зна\-ки, 
например <<пропозициональный уровень>> или <<операция импликации>>, 
характеризуют несколько ЛСО, то другие\linebreak высту\-па\-ют уникальными свойствами 
того или иного ЛСО, поз\-во\-ля\-ющи\-ми его идентифицировать и~отличать от 
других. К~ним относится, например, при\-знак <<$Y$ содержит более част\-ное 
понятие~$q$,\linebreak су\-жа\-ющее экстенсионал~$p$>>\footnote{Строчные буквы $p$ и~$q$ 
обозначают положение дел, прописные~$P$ и~$Q$~--- акты высказывания, прописные~$X$ 
и~$Y$~--- фрагменты текста.}, ха\-рак\-те\-ри\-зу\-ющий ЛСО 
спецификации. В~табл.~3 
приведены данные по использованию различительных при\-зна\-ков для ЛСО, 
получивших определения в~НБД. 

\end{multicols}


\begin{table*}[h]\small %tabl1
\vspace*{-15pt}

  \begin{center}
  \Caption{Примеры структурированных определений ЛСО}
  \vspace*{2ex}
  
\begin{tabular}{|l|l|l|l|}
\hline
\multicolumn{1}{|c|}{ЛСО}&
\multicolumn{1}{c|}{\tabcolsep=0pt\begin{tabular}{c}Базовая\\ семантическая операция\end{tabular}}&
\multicolumn{1}{c|}{Уровень}&
\multicolumn{1}{c|}{\tabcolsep=0pt\begin{tabular}{c}Дополнительные\\ характеристики\end{tabular}}\\
\hline
\tabcolsep=0pt\begin{tabular}{l}Пропозицио-\\ нальная\\ альтернатива\end{tabular}&
\tabcolsep=0pt\begin{tabular}{l}$\bullet$~Операция сравнения, устанав-\\ \hphantom{$\bullet$~}ливающая сходство~$p$ и~$q$\end{tabular}&
\tabcolsep=0pt\begin{tabular}{l}$\bullet$~Пропозицио-\\ \hphantom{$\bullet$~}нальный\end{tabular}&
\tabcolsep=0pt\begin{tabular}{l}$\bullet$~$p$ и~$q$~---  положения вещей, имеющие\\ \hphantom{$\bullet$~}статус гипотезы;\\
$\bullet$~говорящий предлагает сделать выбор\\ \hphantom{$\bullet$~}между~$p$ и~$q$\end{tabular}\\
\hline
Спецификация&
\tabcolsep=0pt\begin{tabular}{l}$\bullet$~Операция соотнесения общего\\ \hphantom{$\bullet$~}и~частного\end{tabular}&
\tabcolsep=0pt\begin{tabular}{l}$\bullet$~Пропозицио-\\ \hphantom{$\bullet$~}нальный\end{tabular} &
\tabcolsep=0pt\begin{tabular}{l}$\bullet$~$X$ содержит обобщенное понятие или\\ \hphantom{$\bullet$~}положение вещей~$p$;\\
$\bullet$~$Y$ содержит более частное понятие~$q$,\\ \hphantom{$\bullet$~}сужающее экстенсионал~$p$\end{tabular}\\
\hline
\end{tabular}
\end{center}
\vspace*{-6pt}
\end{table*}

 \begin{table*}\small %tabl2
\begin{center}
\Caption{Структурированные определения уступительных и~условных ЛСО}
\vspace*{2ex}

\begin{tabular}{|l|l|l|l|}
\hline
\multicolumn{1}{|c|}{ЛСО}&
\multicolumn{1}{c|}{\tabcolsep=0pt\begin{tabular}{c}Базовая\\ семантическая операция\end{tabular}}&
\multicolumn{1}{c|}{Уровень}&
\multicolumn{1}{c|}{\tabcolsep=0pt\begin{tabular}{c}Дополнительные\\ характеристики\end{tabular}}\\
\hline
Уступительные&
\tabcolsep=0pt\begin{tabular}{l}$\bullet$~Операция импликации\end{tabular}&
\tabcolsep=0pt\begin{tabular}{l}$\bullet$~Пропозицио-\\ \hphantom{$\bullet$~}нальный\end{tabular}&
\tabcolsep=0pt\begin{tabular}{l}$\bullet$~$p$ и~$q$~--- положения вещей;\\
$\bullet$~как правило, если имеет место~$q$, то не имеет\\ \hphantom{$\bullet$~}места~$p$\end{tabular}\\
\hline
Условные&
\tabcolsep=0pt\begin{tabular}{l}$\bullet$~Операция импликации\end{tabular} &
\tabcolsep=0pt\begin{tabular}{l}$\bullet$~Пропозицио-\\ \hphantom{$\bullet$~}нальный\end{tabular}&
\tabcolsep=0pt\begin{tabular}{l}$\bullet$~$p$ и~$q$~--- 
положения вещей, имеющие статус\\ \hphantom{$\bullet$~}гипотезы или неосуществившегося положе-\\ \hphantom{$\bullet$~}ния вещей;\\
$\bullet$~если имеет место~$p$, то имеет место~$q$\end{tabular}\\
\hline
\end{tabular}
\end{center}
\vspace*{-6pt}
\end{table*}
  
\begin{table*}\small %tabl3
\begin{center}
\Caption{Использование признаков в~описаниях ЛСО}
\vspace*{2ex}

%\tabcolsep=3pt
\begin{tabular}{|l|c|}
\hline
\multicolumn{1}{|c|}{Признак ЛСО}&\tabcolsep=0pt\begin{tabular}{c}Количество\\ ЛСО\end{tabular}\\
\hline
Пропозициональный уровень&14\hphantom{9}\\
\hline
Операция сравнения, устанавливающая несходство~$p$ и~$q$&12\hphantom{9}\\
\hline
Метаязыковой уровень&6\\
\hline
Уровень высказывания&6\\
\hline
$p$ отвергается, принимается~$q$&5\\
\hline
Операция сравнения, устанавливающая сходство между $p$ и~$q$&5\\
\hline
Говорящий предлагает сделать выбор между $p$ и~$q$&5\\
\hline
\multicolumn{1}{|c|}{$\cdots$}&$\cdots$\\
\hline
Как правило, если имеет место $q$, то не имеет места~$p$&1\\
\hline
$p$ и~$q$~--- положения вещей, не связанные никаким другим ЛСО&1\\
\hline
$p$ верно, только если исключить осуществление~$q$&1\\
\hline
Положение вещей $q$ служит аргументом в~пользу ожидаемого вывода не-$r$&1\\
\hline
Положение вещей р служит аргументом в~пользу ожидаемого вывода~$r$&1\\
\hline
$p$ и~$q$ имеют одинаковый экстенсионал&1\\
\hline
$q$~--- возможное описание того же положения вещей~$r$&1\\
\hline
\tabcolsep=0pt\begin{tabular}{l}$q$~--- обобщенное (<<без частностей>>) представление положения вещей,\\ сделанное на 
основании свойств~$p$\end{tabular}&1\\
\hline
\multicolumn{1}{|c|}{$\cdots$}&$\cdots$\\
\hline
\end{tabular}
\end{center}
\vspace*{-6pt}
\end{table*}




\begin{multicols}{2}
 
  Из 52 использованных в~НБДК различительных при\-зна\-ков~19~характеризуют 
более одного ЛСО, а~33~уникальны. Тем не менее эти уникальные при\-зна\-ки 
могут быть объединены в~семейства, фик\-си\-ру\-ющие концептуальную бли\-зость 
при\-зна\-ков, не\-смот\-ря на их формальные раз\-ли\-чия. Например, семейст\-во 
при\-зна\-ков <<используется отрицательный коррелят $p/P$>> включает в~себя 
сле\-ду\-ющий набор признаков: 
\begin{enumerate}[(1)]
\item %1)~
$P$ отвергается, принимается~$q$; 
\item %2)~
$q$ имеет мес\-то, если не имеет мес\-та~$p$; 
\item %3)~
$q$ имеет мес\-то, если не имеет мес\-та положение вещей, описанное в~$P$; 
\item %4)~
$p$ отвергается, принимается~$q$; 
\item %5)~
как правило, если имеет мес\-то~$q$, то не имеет мес\-та~$p$.
\end{enumerate}
  
\begin{table*}\small %tabl4
\vspace*{-3pt}
\begin{center}
\Caption{Структурированные определения ЛСО од\-но\-вре\-мен\-ности, со\-пут\-ст\-во\-ва\-ния и~со\-по\-став\-ления}
\vspace*{2ex}

\tabcolsep=4.3pt
\begin{tabular}{|l|l|l|l|}
\hline
\multicolumn{1}{|c|}{ЛСО}&
\multicolumn{1}{c|}{\tabcolsep=0pt\begin{tabular}{c}Базовая\\ семантическая операция\end{tabular}}&
\multicolumn{1}{c|}{Уровень}&
\multicolumn{1}{c|}{\tabcolsep=0pt\begin{tabular}{c}Дополнительные\\ характеристики\end{tabular}}\\
\hline
Одновременность&
\tabcolsep=0pt\begin{tabular}{l}$\bullet$~Расположение на оси времени\end{tabular} &
\tabcolsep=0pt\begin{tabular}{l}$\bullet$~Пропозицио-\\ \hphantom{$\bullet$~}нальный\end{tabular}&
\tabcolsep=0pt\begin{tabular}{l}$\bullet$~$p$ и~$q$~--- положения вещей;\\
$\bullet$~$Tp$ включает в~себя~$Tq$\end{tabular}\\
\hline
Сопутствование&
\tabcolsep=0pt\begin{tabular}{l}$\bullet$~Операция сравнения, устанав-\\ \hphantom{$\bullet$~}ливающая сходство между~$p$ и~$q$\end{tabular} &
\tabcolsep=0pt\begin{tabular}{l}$\bullet$~Пропозицио-\\ \hphantom{$\bullet$~}нальный\end{tabular}&
\tabcolsep=0pt\begin{tabular}{l}$\bullet$~$q$~--- положение вещей, зависимое от~$p$;\\
$\bullet$~$Tp$ включает в~себя $Tq$\end{tabular}\\
\hline
Сопоставление&
\tabcolsep=0pt\begin{tabular}{l}$\bullet$~Операция сравнения, устанав-\\ \hphantom{$\bullet$~}ливающая несходство~$p$ и~$q$\end{tabular}&
\tabcolsep=0pt\begin{tabular}{l}$\bullet$~Пропозицио-\\ \hphantom{$\bullet$~}нальный\end{tabular}&
\tabcolsep=0pt\begin{tabular}{l}$\bullet$~$p$ и~$q$ актуальны для говорящего\\ \hphantom{$\bullet$~}в~момент речи~$Td$;\\
$\bullet$~сходство~$p$ и~$q$ относительно некоторого\\ \hphantom{$\bullet$~}<<общего знаменателя>>\end{tabular}\\
\hline
\end{tabular}
\end{center}
\vspace*{-9pt}
\end{table*}

\begin{table*}[b]\small %tabl5
\vspace*{-12pt}
\begin{center}
\Caption{Распределение аннотаций с~ЛСО альтернативы и~коррекции по уровням}
\vspace*{2ex}

\tabcolsep=10pt
\begin{tabular}{|l|c|c|c|}  
\hline
&\multicolumn{3}{c|}{Уровень}\\
\cline{2-4}
\multicolumn{1}{|c|}{\raisebox{6pt}[0pt][0pt]{ЛСО}}&Пропозициональный&Иллокутивный&Метаязыковой\\
\hline
%&&\multicolumn{1}{p{80pt}|}{\hphantom{80pt}}&\multicolumn{1}{p{80pt}|}{\hphantom{80pt}}\\[-12pt]
Альтернатива&2138&15&144\\
Коррекция&\hphantom{9}231&36&\hphantom{9}36\\
\hline
\end{tabular}
\end{center}
\vspace*{-6pt}
\end{table*}

\section{Критерии определения степени семантической близости 
логико-семантических отношений }

  Структурированные определения ЛСО поз\-во\-ля\-ют создать их 
непротиворечивую классификацию, обосновав чис\-ло высших иерархических 
уров\-ней и~объединение ЛСО в~семантические классы общ\-ностью лежащей в~их 
основе семантической операции. Общ\-ность этого признака не всегда, однако, 
служит необходимым и~достаточным условием для того, чтобы считать ЛСО 
семантически близ\-ки\-ми. Было замечено, что, с~одной стороны, показателю ЛСО 
в~одном языке может соответствовать в~переводе показатель ЛСО, 
относящийся к~другому иерархическому уров\-ню, причем как при наличии сис\-тем\-но\-го 
переводного эквивалента у~данного показателя~(a), так и~при его 
отсутствии~(б), и~такие случаи не единичны. Например, в~высказывании~(а) 
итальянский коннектор \textit{intanto} `меж\-ду тем' переведен русским 
со\-по\-ста\-ви\-тель\-ным союзом~\textit{а}:\\[-15pt] 
  \begin{itemize}
  \item[(a)] Si volse, e prese ad arrampicarsi di traverso lungo la proda assolata 
dell'argine. Si aiutava con la mano destra, afferrandosi ai ciuffi dell'erba; 
\textit{intanto}, la sinistra levata all'altezza del capo, veniva togliendosi 
e~rimettendosi il cerchietto ferma-capelli.~--- Она повернулась и~стала 
карабкаться по залитому солнцем спус\-ку, хватаясь правой рукой за траву, 
\textit{а}~левой, поднятой над головой, по\-прав\-ля\-ла обруч на волосах. [Giorgio 
Bassani. Il giardino dei Finzi-Contini (1962), перевод И.~Соболева (2008)]. 
  \end{itemize}

  В~(б) тот же итальянский коннектор переводит русский показатель ЛСО 
со\-пут\-ст\-во\-ва\-ния:\\[-15pt] 
  \begin{itemize}
  \item[(б)] Он говорил громко \textit{и при этом} делал такие удив\-ленные 
глаза, что мож\-но было подумать, что он лгал.~--- Egli parlava ad alta voce 
\textit{e~intanto} faceva degli occhi cos$\grave{\mbox{\!\ptb{\i}}}$ meravigliati che 
si pensava ch'egli mentisse. [А.\,П.~Чехов. Палата №\,6 (1892), перевод 
F.~Malcovati]. 
  \end{itemize}
  
  С другой стороны, известно, что показатели ЛСО многозначны: на это 
указывают, в~част\-ности, словари и~грамматики. Так, для русского коннектора 
\textit{между тем}, вы\-сту\-па\-юще\-го сис\-тем\-ным эквивалентом итальянского 
\textit{intanto}, словарь~[19] дает три значения: од\-но\-вре\-мен\-ность, 
со\-по\-став\-ле\-ние и~про\-ти\-ви\-тель\-ность (ЛСО <<вопреки ожи\-да\-емо\-му>>).
  
  Если, однако, сравнить определения ЛСО со\-пут\-ст\-во\-ва\-ния, со\-по\-став\-ле\-ния 
  и~од\-но\-вре\-мен\-ности, то увидим, что они, не\-смот\-ря на то что в~их основе лежат 
разные семантические операции, имеют общие различительные признаки (табл.~4). 

  
  Общим для трех отношений является при\-знак <<пропозициональный 
уровень>>, для ЛСО од\-но\-вре\-мен\-ности и~со\-пут\-ст\-во\-ва\-ния~--- при\-знак <<$Tp$ 
включает в~себя $Tq$>>, который входит в~то же семейство признаков 
(<<единство временн$\acute{\mbox{o}}$го интервала>>), что и~при\-знак <<$p$ и~$q$ актуальны для 
говорящего в~момент речи $Td$>>, ха\-рак\-те\-ри\-зу\-ющий ЛСО со\-по\-став\-ле\-ния. 
Таким образом, с~одной стороны, в~качестве переводного эквивалента 
показателя некоторого ЛСО выбираются показатели, раз\-де\-ля\-ющие с~ним 
различительные признаки, а~с~другой~--- набор значений, вы\-ра\-жа\-емых 
коннектором, та\-кже не является произвольным, а~определяется семантической 
бли\-зостью ЛСО.
  
  Однако при определении семантической близости ЛСО не все 
различительные при\-зна\-ки имеют одинаковый вес, или <<коэффициент  
бли\-зости>>. Так, признак <<пропозициональный уровень>> не может иметь 
высокий коэффициент, поскольку не является дис\-кри\-ми\-ни\-ру\-ющим. Он входит 
в~группу при\-зна\-ков (<<уровень, на котором установлено ЛСО>>), 
вклю\-ча\-ющую всего три элемента, а~значит, он становится общим для большого 
чис\-ла ЛСО, которые в~по\-дав\-ля\-ющем большинстве уста\-нав\-ли\-ва\-ют\-ся именно 
между пропозициями. Кроме того, ЛСО, если оно может уста\-нав\-ли\-вать\-ся на 
всех трех уров\-нях, в~большинстве случаев уста\-нав\-ли\-ва\-ет\-ся именно на уров\-не 
пропозиций. В~табл.~5 приводятся данные для ЛСО альтернативы и~коррекции, 
для которых в~НБДК сделана сплош\-ная вы\-борка.

  
  Признаки из группы <<Базовая семантическая операция>> так\-же, по 
име\-ющим\-ся данным, не имеют ре\-ша\-юще\-го значения для определения 
семантической бли\-зости ЛСО, хотя их коэффициенты, по-ви\-ди\-мо\-му, 
долж\-ны быть выше, чем у~признаков группы <<Уровень>>, поскольку они 
являются классифицирующими и~служат для обосно\-ва\-ния того, почему ЛСО 
объединяются в~семантические \mbox{классы}. 
{\looseness=-1

}
  
  Проанализированные данные поз\-во\-ля\-ют предположить, что на\-и\-выс\-ший 
коэффициент дол\-жен быть присвоен тем случаям, когда различительные 
при\-зна\-ки ЛСО относятся к~одному семейству признаков. Это хорошо вид\-но на 
примере ЛСО од\-но\-вре\-мен\-ности, со\-по\-став\-ле\-ния и~со\-пут\-ст\-во\-ва\-ния (см.\ табл.~4), 
а~так\-же на примере других ЛСО, получивших определения в~НБДК. 
Дальнейшее исследование поз\-во\-лит присвоить чис\-ло\-вые па\-ра\-мет\-ры 
коэффициентам бли\-зости для различительных при\-знаков.
{\looseness=-1

}

\vspace*{-6pt}

\section{Заключение}

  Разработанная классификация ЛСО и~сформированные на ее основе в~НБДК 
структурированные определения поз\-во\-ля\-ют не только избежать противоречий 
или необоснованных решений в~классификации ЛСО, но и~определить степень 
бли\-зости ЛСО с~учетом общ\-ности их раз\-ли\-чи\-тель\-ных при\-зна\-ков. 
В~дальнейшем при определении коэффициентов бли\-зости, при\-сва\-и\-ва\-емых 
при\-зна\-кам, мож\-но учитывать данные по со\-че\-та\-е\-мости ЛСО~[20], по 
соответствиям ЛСО в~оригинальном и~переводном текс\-те, а~так\-же те случаи, 
когда раз\-ные ЛСО выражаются одним и~тем же показателем. Полученные 
результаты поз\-во\-лят улучшить автоматическую обработку текс\-та, а~так\-же 
качество машинного пе\-ре\-вода.

\vspace*{-6pt}
  
{\small\frenchspacing
 { %\baselineskip=12pt
 %\addcontentsline{toc}{section}{References}
 \begin{thebibliography}{99}

\bibitem{1-kr}
\Au{Hobbs J.\,R.} A~computational approach to discourse analysis.~--- 
New York, NY, USA: Department of Computer Science, City College, City University of New 
York, 1976. Research Report~76-2. P.~28--38.
\bibitem{2-kr}
\Au{Hobbs J.\,R.} Why is discourse coherent?~--- Menlo Park, CA, 
USA: SRI International, 1978.  SRI Technical Note~176. 44~p.
\bibitem{3-kr}
\Au{Hobbs J.\,R.} Coherence and coreference~// Cognitive Sci., 1979. Vol.~3. No.\,1.  
P.~67--90. 
\bibitem{4-kr}
Русская грамматика.~/ Под ред. Н.\,Ю.~Шведовой.~--- М.: Наука, 1980. Т.~2. 714~с.
\bibitem{5-kr}
Лексикографические порт\-ре\-ты служебных слов~/ Под ред. Е.\,А.~Ста\-ро\-ду\-мо\-вой, 
Е.\,С.~Ше\-ре\-меть\-евой, В.\,Н.~Завья\-ло\-ва.~--- Владивосток: ДВФУ, 2022. 322~с.
\bibitem{6-kr}
\Au{Mann W.\,C., Thompson S.\,A.} Rhetorical structure theory: Towards a functional theory of text 
organization~// Text, 1988. Vol.~8. No.\,3. P.~243--281. doi: 10.1515/text.1. 1988.8.3.243.
\bibitem{7-kr}
\Au{Knott A., Dale~R.} Using linguistic phenomena to motivate a~set of coherence relations~// 
Discourse Process., 1994. Vol.~18. No.\,1. P.~35--62. doi: 10.1080/ 01638539409544883.
\bibitem{8-kr}
\Au{Rudolph E.} Contrast: Adversative and concessive expressions on sentence and text  
level.~--- Berlin/Boston: Walter de Gruyter, 1996. 564~p.
\bibitem{9-kr}
\Au{Fraser B.} An account of discourse markers~// International Review Pragmatics, 2009. Vol.~1. 
No.\,2. P.~293--320. doi: 10.1163/187730909X12538045489818.
\bibitem{10-kr}
PDTB Research Group. The Penn Discourse Treebank~2.0 annotation manual.~--- Philadelphia, PA, USA: 
Institute for Research in Cognitive Science, University of Pennsylvania, 2008.  Technical Report 
IRCS-08-01. 99~p. {\sf https://www.cis.upenn.edu/$\sim$elenimi/pdtb-manual.pdf}.
\bibitem{11-kr}
\Au{Breindl E., Volodina~A., \mbox{Wa\!{\ptb{\!\ss}}\,ner}~U.\,H.} Handbuch der deutschen Konnektoren~2. 
Semantik der deutschen Satzverkn$\ddot{\mbox{u}}$pfer.~--- Berlin: Walter de Gruyter, 2014. 1327~p.
\bibitem{12-kr}
\Au{Инькова О.\,Ю.} Ло\-ги\-ко-се\-ман\-ти\-че\-ские отношения: проблемы 
классификации~// Связ\-ность текста: мереологические ло\-ги\-ко-се\-ман\-ти\-че\-ские 
отношения.~--- М.: ЯСК, 2019. С.~11--98.
\bibitem{13-kr}
\Au{Webber B., Prasad~R., Lee~A., Joshi~A.} The Penn Discourse Treebank~3.0 annotation 
manual, 2019. 81~p. {\sf  
https:// catalog.ldc.upenn.edu/docs/LDC2019T05/ PDTB3-Annotation-Manual.pdf}.
\bibitem{14-kr}
\Au{Инькова О.\,Ю., Кружков М.\,Г.} Структурированные определения дискурсивных 
отношений в~Надкорпусной базе данных коннекторов~// Информатика и~её применения, 
2021. Т.~15. Вып.~4. С.~27--32. doi: 10.14357/19922264210404.
\bibitem{15-kr}
Семантика коннекторов: контрастивное исследование~/ Под. ред. О.\,Ю.~Иньковой.~--- 
М.: ТОРУС ПРЕСС, 2018. 368~с.
\bibitem{16-kr}
Структура коннекторов и~методы ее описания~/ Под. ред. О.\,Ю.~Иньковой.~--- М.: 
ТОРУС ПРЕСС, 2019. 316~с.
\bibitem{17-kr}
\Au{Кружков М.\,Г.} Концепция по\-стро\-ения надкорпусных баз данных~// Сис\-те\-мы 
и~средства информатики, 2021. T.~31. №\,3. С.~101--112. doi: 
10.14357/ 08696527210309.
\bibitem{18-kr}
\Au{Инькова О.\,Ю.} Определения дискурсивных отно\-шений: опыт Надкорпусной базы 
данных коннекторов~// Компьютерная линг\-ви\-сти\-ка и~интеллектуальные технологии: По 
мат-лам ежегодной\linebreak Междунар. конф. <<Диалог>>.~--- М.: РГГУ, 2021. Вып.~20(27). 
С.~328--338.
\bibitem{19-kr}
Словарь структурных слов русского языка~/ Под ред. В.\,В.~Морковкина.~--- М.: Лазурь, 
1997. 422~с.
\bibitem{20-kr}
\Au{Инькова О.\,Ю., Кружков~М.\,Г.} Со\-че\-та\-емость ло\-ги\-ко-се\-ман\-ти\-че\-ских 
отношений: количественные методы анализа~// Информатика и~её применения, 2019. Т.~13. 
Вып.~2. С.~83--91. doi: 10.14357/19922264190212.

\end{thebibliography}

 }
 }

\end{multicols}

\vspace*{-8pt}

\hfill{\small\textit{Поступила в~редакцию 10.07.23}}

%\vspace*{8pt}

%\pagebreak

\newpage

\vspace*{-28pt}



\def\tit{EVALUATION CRITERIA FOR~DISCOURSE~RELATIONS~SEMANTIC~AFFINITY}


\def\titkol{Evaluation criteria for~discourse relations semantic 
affinity}


\def\aut{O.\,Yu.~Inkova$^{1,2}$ and~M.\,G.~Kruzhkov$^1$}

\def\autkol{O.\,Yu.~Inkova and~M.\,G.~Kruzhkov}

\titel{\tit}{\aut}{\autkol}{\titkol}

\vspace*{-10pt}


\noindent
$^1$Federal Research Center ``Computer Science and Control'' of the Russian 
Academy of Sciences, 44-2~Vavilov\linebreak
$\hphantom{^1}$Str., Moscow 119333, Russian Federation

\noindent
$^2$University of Geneva, 22 Bd des Philosophes, CH-1205 Geneva 4, Switzerland

\def\leftfootline{\small{\textbf{\thepage}
\hfill INFORMATIKA I EE PRIMENENIYA~--- INFORMATICS AND
APPLICATIONS\ \ \ 2023\ \ \ volume~17\ \ \ issue\ 3}
}%
 \def\rightfootline{\small{INFORMATIKA I EE PRIMENENIYA~---
INFORMATICS AND APPLICATIONS\ \ \ 2023\ \ \ volume~17\ \ \ issue\ 3
\hfill \textbf{\thepage}}}

\vspace*{3pt}



\Abste{The paper presents an overview of structured definitions of discourse 
relations created based on classification principles and criteria for evaluating their 
semantic affinity. The authors point out the shortcomings of existing classification 
approaches that are sometimes inconsistent or contradictory and outline the benefits 
of an alternative approach to classification of discourse relations which is based on 
their structured definition. The paper provides examples of such definitions created 
within the Supracorpora Database of Connectives and discusses the criteria 
for evaluating their semantic closeness. As the structured definitions are represented 
by sets of distinguishing features, the authors discuss the problem of identifying 
proximity factors for each of these features. The gathered data suggest a~hypothesis 
that among the three groups of features: ``Level,'' ``Basic operation,'' and ``Feature 
family'', it is the last one that should have the most impact. Finally, directions for 
further research of this problem are considered, namely, the option of taking into 
account such factors as compatibility of discourse relations, correspondence of 
relations between the source text and its translation, and such cases where certain 
relation markers may express different discourse relations in various contexts.}

\KWE{supracorpora database; logical-semantic relations; connectives; 
annotation; faceted 
classification}



\DOI{10.14357/19922264230314}{UJZJZI}

%\vspace*{-20pt}

 \Ack
\noindent
The research was carried out using the infrastructure of the Shared Research 
Facilities ``High Performance 
Computing and Big Data'' (CKP ``Informatics'') of FRC CSC RAS (Moscow).

%\vspace*{6pt}

  \begin{multicols}{2}

\renewcommand{\bibname}{\protect\rmfamily References}
%\renewcommand{\bibname}{\large\protect\rm References}

{\small\frenchspacing
 {%\baselineskip=10.8pt
 \addcontentsline{toc}{section}{References}
 \begin{thebibliography}{99} 
\bibitem{1-kr-1}
\Aue{Hobbs, J.\,R.} 1976. A~computational approach to discourse analyses. New 
York, NY: Department of Computer Science, City College, City University of New 
York. Research Report~76-2. 28--38.
\bibitem{2-kr-1}
\Aue{Hobbs, J.\,R.} 1978. Why is discourse coherent? Menlo Park, CA: SRI 
International. SRI Technical Note~176. 44~p.
\bibitem{3-kr-1}
\Aue{Hobbs, J.\,R.} 1979. Coherence and coreference. \textit{Cognitive Sci.} 
3(1):67--90.
\bibitem{4-kr-1}
Shvedova, N.\,Yu., ed. 1980. \textit{Rus\-skaya gram\-ma\-ti\-ka.} 
[Russian grammar]. Moscow: Nauka. Vol.~2. 714~p.
\bibitem{5-kr-1}
Starodumova, E.\,A., E.\,S.~She\-re\-met'\-eva, and V.\,N.~Zav'ya\-lov, eds. 2022.  
\textit{Lek\-si\-ko\-gra\-fi\-che\-skie port\-re\-ty slu\-zheb\-nykh slov} 
[Lexicographic portraits of auxiliary words]. Vladivostok: FEFU. 322~p.
\bibitem{6-kr-1}
\Aue{Mann, W.\,C., and S.\,A.~Thomp\-son}. 1988. Rhetorical structure theory: 
Towards a~functional theory of text organization. \textit{Text} 8(3):243--281. doi: 
10.1515/text.1.1988.8.3.243.
\bibitem{7-kr-1}
\Aue{Knott, A., and R.~Dale.} 1994. Using linguistic phenomena to motivate a~set 
of coherence relations. \textit{Discourse Process.} 18(1):35--62. doi: 10.1080/01638539409544883.
\bibitem{8-kr-1}
\Aue{Rudolph, E.} 1996. \textit{Contrast: Adversative and concessive expressions on 
sentence and text level}. Berlin/Boston: Walter de Gruyter. 564~p.
\bibitem{9-kr-1}
\Aue{Fraser, B.} 2009. An account of discourse markers. \textit{International Review Pragmatics} 1(2):293--320. doi: 10.1163/187730909X12538045489818.
\bibitem{10-kr-1}
PDTB Research Group. 2008. The Penn Discourse Treebank~2.0 annotation manual. 
Philadelphia, PA: Institute for Research in Cognitive 
Science, University of Pennsylvania. Technical Report  IRCS-08-01. 99~p.\linebreak Available at: {\sf 
https://www.cis.upenn.edu/$\sim$elenimi/\linebreak pdtb-manual.pdf} (accessed August~1, 
2023).
\bibitem{11-kr-1}
\Aue{Breindl, E., A.~Vo\-lo\-di\-na, and U.\,H.~Wa{\ptb{\!\ss}}ner.} 2014. 
\textit{Handbuch der deutschen Konnektoren~2. Semantik der deutschen Satzverkn$\ddot{\mbox{u}}$pfer}. Berlin: Walter de Gruyter. 
1327~p.
\bibitem{12-kr-1}
\Aue{Inkova, O.\,Yu.} 2019. Logiko-semanticheskie otno\-she\-niya: prob\-le\-my klas\-si\-fi\-ka\-tsii 
[Logical-semantic relations: Classification problems]. \textit{Svyaz\-nost' teks\-ta: 
me\-reo\-lo\-gi\-che\-skie  logiko-semanticheskie ot\-no\-she\-niya} [Text Coherence: Mereological Logical Semantic 
Relations]. Moscow: LRC 
Publs. 11--98.
\bibitem{13-kr-1}
\Aue{Webber, B., R.~Prasad, A.~Lee, and A.~Joshi.} 2019. The Penn Discourse Treebank~3.0 annotation 
manual. 81~p. Available at: {\sf https://catalog.ldc.upenn.edu/docs/
LDC2019T05/PDTB3-Annotation-Manual.pdf} 
(accessed August~1, 2023).
\bibitem{14-kr-1}
\Aue{Inkova, O.\,Yu., and M.\,G.~Kruzh\-kov.} 2021. Struk\-tu\-ri\-ro\-van\-nye opre\-de\-le\-niya  
dis\-kur\-siv\-nykh ot\-no\-she\-niy v~Nad\-kor\-pus\-noy ba\-ze dan\-nykh  
kon\-nek\-to\-rov [Structured definitions of discourse relations in the Supracorpora Database of Connectives]. 
\textit{In\-for\-ma\-ti\-ka i~ee Pri\-me\-ne\-niya~--- Inform. Appl.} 15(4):27--32. doi: 10.14357/19922264210404.
\bibitem{15-kr-1}
Inkova, O.\,Yu., ed. 2018. \textit{Se\-man\-ti\-ka kon\-nek\-to\-rov:  
Kont\-ras\-tiv\-noe is\-sle\-do\-va\-nie} [Semantics of connectives: Contrastive study]. Moscow: TORUS PRESS. 368~p.
\bibitem{16-kr-1}
Inkova, O.\,Yu., ed. 2019. \textit{Struk\-tu\-ra kon\-nek\-to\-rov i~me\-to\-dy ee  
opi\-sa\-niya} [Structure of connectors and methods of its description]. Moscow: TORUS PRESS. 316~p.
\bibitem{17-kr-1}
\Aue{Kruzhkov, M.\,G.} 2021. Kon\-tsep\-tsiya po\-stro\-eniya nad\-kor\-pus\-nykh 
baz dan\-nykh [Conceptual framework for supracorpora databases]. 
 \textit{Sis\-te\-my i~Sred\-st\-va In\-for\-ma\-ti\-ki~--- Systems and Means of Informatics} 31(3):101--112.
  doi: 10.14357/08696527210309.
\bibitem{18-kr-1}
\Aue{Inkova, O.\,Yu.} 2021. Opre\-de\-le\-niya dis\-kur\-siv\-nykh ot\-no\-she\-niy: 
opyt Nad\-kor\-pus\-noy bazy dan\-nykh kon\-nek\-to\-rov [Definition of discursive relations: The experience 
of the supracorpora database of connectors]. \textit{Komp'yu\-ter\-naya 
 ling\-vi\-sti\-ka i~in\-tel\-lek\-tu\-al'\-nye tekh\-no\-lo\-gii: 
Po mat-m ezhegodnoy Mezhdunar.  konf. ``Dialog''} [Computational Linguistics and 
Intellectual Technologies: Papers from the Annual Conference (International) 
``Dialogue'']. Moscow: RGGU. 20(27):328--338.
\bibitem{19-kr-1}
Morkovkin, V.\,V., ed. 1997. \textit{Slo\-var' struk\-tur\-nykh slov rus\-sko\-go 
yazyka} [Dictionary of structural words of the Russian language]. Moscow: Lazur'. 422~p. 
\bibitem{20-kr-1}
\Aue{Inkova, O.\,Yu., and M.\,G.~Kruzh\-kov.} 2019. So\-che\-ta\-e\-most'  
logiko-semanticheskikh ot\-no\-she\-niy: ko\-li\-chest\-ven\-nye me\-to\-dy ana\-li\-za 
[Compatibility of logical semantic relations: Methods of quantitative analysis]. \textit{In\-for\-ma\-ti\-ka i~ee  
Pri\-me\-ne\-niya~--- Inform. Appl.} 
 13(2):83--91. doi: 10.14357/ 19922264190212.
 
 \end{thebibliography}

 }
 }

\end{multicols}

\vspace*{-6pt}

\hfill{\small\textit{Received July 10, 2023}} 

\vspace*{-18pt}

\Contr

\vspace*{-4pt}

\noindent
\textbf{Inkova Olga Yu.} (b.\ 1965)~--- Doctor of Science in philology, senior 
scientist, Institute of 
Informatics Problems, Federal Research Center ``Computer Science and Control'' of 
the Russian Academy of 
Sciences, 44-2~Vavilov Str., Moscow 119333, Russian Federation; faculty member, 
University of Geneva, 22~Bd des Philosophes, CH-1205 Geneva~4, Switzerland; 
\mbox{olyainkova@yandex.ru}

\vspace*{3pt}

\noindent
\textbf{Kruzhkov Mikhail G.} (b.\ 1975)~--- senior scientist, Institute of Informatics 
Problems, Federal Research Center ``Computer Science and Control'' of the Russian Academy of 
Sciences, 44-2~Vavilov Str., 
Moscow 119333, Russian Federation; \mbox{magnit75@yandex.ru}





\label{end\stat}

\renewcommand{\bibname}{\protect\rm Литература} 