\def\stat{vakulenko}

\def\tit{ФОРМАЛИЗОВАННОЕ ОПИСАНИЕ СТАТИСТИЧЕСКОЙ ОБРАБОТКИ ИНФОРМАЦИИ В~БАЗАХ 
ДАННЫХ$^*$}

\def\titkol{Формализованное описание статистической обработки информации в~базах 
данных}

\def\aut{В.\,В.~Вакуленко$^1$, И.\,М.~Зацман$^2$}

\def\autkol{В.\,В.~Вакуленко, И.\,М.~Зацман}

\titel{\tit}{\aut}{\autkol}{\titkol}

\index{Вакуленко В.\,В.}
\index{Зацман И.\,М.}
\index{Vakulenko V.\,V.}
\index{Zatsman I.\,M.}


{\renewcommand{\thefootnote}{\fnsymbol{footnote}} \footnotetext[1]
{Работа выполнялась с~использованием инфраструктуры Центра коллективного пользования <<Высокопроизводительные вы\-чис\-ле\-ния и~большие данные>> 
(ЦКП <<Информатика>>) ФИЦ ИУ РАН (г.~Москва).}}


\renewcommand{\thefootnote}{\arabic{footnote}}
\footnotetext[1]{Федеральный исследовательский центр <<Информатика и~управление>> Российской академии наук;
\mbox{vvak@pm.me}}
\footnotetext[2]{Федеральный исследовательский центр <<Информатика и~управление>> Российской академии наук, 
\mbox{izatsman@yandex.ru}}

\vspace*{-13pt}

  

   \Abst{Рассматривается последовательность этапов ста\-ти\-сти\-че\-ской обработки текс\-то\-вой 
информации, начиная с~конкретных информационных объектов (КИО) баз данных (БД) и~заканчивая 
значениями чис\-ло\-вых характеристик множеств этих объектов. Например, если в~БД
хранятся описания пол\-но\-текс\-то\-вых научных ста\-тей, то они считаются 
КИО. При со\-от\-вет\-ст\-ву\-ющем наполнении такой БД
многоэтапный процесс их обработки позволяет определить значения чис\-ло\-вых характеристик 
пуб\-ли\-ка\-ци\-он\-ной ак\-тив\-ности исследователя, научного подразделения или научной организации в~целом. 
Такие процессы начинаются с~обработки КИО
и~завершаются вы\-чис\-ле\-ни\-ем значений характеристик множеств этих объектов. На 
промежуточных этапах обработки могут формироваться таб\-ли\-цы и~другие  
вер\-баль\-но-чис\-ло\-вые объекты. Если этапы ста\-ти\-сти\-че\-ской обработки спроектированы как 
обратимые и~в~БД реализована функция верификации значений чис\-ло\-вых 
характеристик, то процесс их проверки начинается со значений характеристик и~завершается 
до\-сту\-пом к~КИО, которые были использованы для 
вы\-чис\-ле\-ния этих значений. Предлагается формализованное описание этапов статистической 
обработки текс\-то\-вой информации в~БД. Такую ее трансформацию в~чис\-ло\-вые 
значения предлагается назвать ин\-фор\-ма\-ци\-он\-но-ма\-те\-ма\-ти\-че\-ской  
(ИМ-транс\-фор\-ма\-ция). Она сочетает обработку КИО, 
формирование вер\-баль\-но-чис\-ло\-вых объектов и~математические вы\-чис\-ле\-ния значений 
чис\-ло\-вых характеристик. Такая трансформация текс\-то\-вой информации может на отдельных 
этапах включать математические преобразования, но в~целом она к~ним не сводится. Цель 
\mbox{статьи}~--- предложить принципы формализованного описания ИМ-транс\-фор\-ма\-ции текс\-тов 
в~БД. В~качестве ее ил\-люст\-ра\-ции рас\-смот\-рен пример формализации процесса 
определения чис\-ла вариантов перевода коннекторов, вы\-ра\-жа\-ющих внут\-ри\-текс\-то\-вые отношения 
между текс\-то\-вы\-ми фрагментами в~надкорпусной БД (НБД) коннекторов, созданной в~ФИЦ 
ИУ РАН.}
   
   \KW{информационно-математическая трансформация; текс\-то\-вая информация; 
ста\-ти\-сти\-че\-ская обработка текс\-то\-вой информации; надкорпусная база данных}

\DOI{10.14357/19922264230313}{TYCAEX}
  
\vspace*{-6pt}


\vskip 10pt plus 9pt minus 6pt

\thispagestyle{headings}

\begin{multicols}{2}

\label{st\stat}
   

\section{Введение}

\vspace*{-3pt}

  Процесс статистической обработки текс\-то\-вой информации в~НБД
параллельных текс\-тов начинается с~\textit{этапа поиска} вхож\-де\-ний 
ис\-сле\-ду\-емой языковой единицы (ЯЕ) в~параллельных текс\-тах. Одна из целей 
статистической обработки со\-сто\-ит в~определении чис\-ла вариантов перевода 
исследуемой ЯЕ, которая может быть словом, словосочетанием, грамматической 
категорией (например, время глагола) или знаком препинания\footnote[3]{Знаку 
препинания в~переводе может соответствовать слово или словосочетание.}.
  
  Например, если исследуются варианты перевода на французский язык 
коннектора\footnote[4]{Коннекторами называются лексические единицы, основная функция которых состоит 
в~выражении одного или нескольких внут\-ри\-текс\-то\-вых отношений между фрагментами текс\-та, например 
отношения причины~\cite{1-zac}.} \textit{а~то} в~рус\-ско-фран\-цуз\-ских параллельных 
текстах художественных произведений, то на первом этапе из текс\-тов 
НБД извлекаются фраг\-мен\-ты (большей частью пред\-ло\-же\-ния), 
содержащие этот коннектор, и~их переводы на французский язык (см.\ три примера 
найденных пар фрагментов оригинала и~их переводов в~табл.~1). Всего в~НБД 
коннекторов, созданной в~ФИЦ ИУ РАН, на 08.05.2023 было~144\,088 таких пар, 
каждая из которых~--- это конкретный информационный объект НБД. Из 
них 311~КИО содержат коннектор \textit{а~то} в~русском оригинале, включая три 
примера, приведенных в~табл.~1. Методы поиска ис\-сле\-ду\-емых ЯЕ в~НБД описаны в~работах~\cite{2-zac, 3-zac, 4-zac}.

   
  На \textit{втором этапе} для найденных пар выполняется формирование 
аннотируемых переводных соответствий (АПС), каж\-дое из которых включает 
структурированное описание коннектора \textit{а~то}, его кон\-текс\-та, перевода 
этого коннектора на французский язык и~его кон\-текс\-та со\-глас\-но общим 
принципам аннотирования~\cite{5-zac} и~разработанной на их основе 
методологии~\cite{3-zac}.
  
  Всего на 08.05.2023 для~220 из~311~найденных пар были сформированы 
220~АПС с~описанием\linebreak\vspace*{-12pt}

\pagebreak

\end{multicols}

\begin{table*}\small %tabl1
\begin{center}
\Caption{Примеры результатов поиска фрагментов текста с~коннектором \textit{а~то} и~их 
переводов на французский язык}
\vspace*{2ex}

\begin{tabular}{|p{79mm}|p{79mm}|}
\hline
\multicolumn{1}{|c|}{\textbf{Фрагмент с~коннектором в~оригинале}}& \multicolumn{1}{c|}{\textbf{Перевод}}\\
\hline
Я вот теперь выехала, чтоб узнать, \textbf{а~то} засяду уже безвыездно, если, избави Бог, 
скарлатина.\newline
[Л.\,Н.~Толстой. Анна Каренина  
(1873--1877)]&Je suis sortie aujourd'hui pour savoir o$\grave{\mbox{u}}$ vous en 
$\acute{\mbox{e}}$tiez, \textbf{car} j'ai peur de ne plus pouvoir bouger de longtemps.\newline
[Перевод Анри Монго (1952)]\\
\hline
Я нарочно прибрала, чтобы кто не поднял, \textbf{а~то} потом поминай как звали!\newline
[М.\,А.~Булгаков. Мастер и~Маргарита (1929--1940)]&Je l'ai trouv$\acute{\mbox{e}}$, dans une 
serviette, et justement je l'avais mis de c$\hat{\mbox{o}}$t$\acute{\mbox{e}}$ pour que personne ne 
la ramasse, \textbf{sinon}, hein, adieu la valise!\newline
[Перевод Клода Линьи (1968)]\\
\hline
--- Он женится! Хочешь об заклад, что не женится?~--- возразил он. --- Да ему Захар и~спать-то 
помогает, \textbf{а~то} жениться!\newline
[И.\,А.~Гончаров. Обломов (1848--1859)]&--- Lui, se marier ! je parie qu'il ne se mariera pas ! 
r$\acute{\mbox{e}}$pliqua-t-il. Il a~besoin de Zakhar m$\hat{\mbox{e}}$me pour dormir, \textbf{et} 
tu veux qu'il se marie!\newline
[Перевод Артюра Адамова (1959)]\\
\hline
\end{tabular}
\end{center}
%\end{table*}
 %  \begin{table*}[b]\small %tabl2
   \vspace*{-3pt}
   \begin{center}
   \Caption{Примеры АПС с~коннектором \textit{а~то}}
   \vspace*{2ex}
   
   \begin{tabular}{|p{44mm}|p{30mm}|p{44mm}|p{30mm}|}
   \hline
\multicolumn{1}{|c|}{\tabcolsep=0pt\begin{tabular}{c}\textbf{Фрагмент с~коннектором}\\ \textbf{в~оригинале}\end{tabular}}&
\multicolumn{1}{c|}{\tabcolsep=0pt\begin{tabular}{c}\textbf{Коннектор}\\
\textbf{и~внутритекстовое}\\
\textbf{отношение}\end{tabular}}&\multicolumn{1}{c|}{\textbf{Перевод}}&\multicolumn{1}{c|}{\tabcolsep=0pt\begin{tabular}{c}\textbf{Коннектор}\\
\textbf{и~внутритекстовое}\\ \textbf{отношение}\end{tabular}}\\
\hline
Я вот теперь выехала, чтоб узнать, \textbf{а~то} засяду уже безвыездно, \textit{если}, избави Бог, 
скарлатина.&\textbf{а то}\newline
$\langle$отношение причины$\rangle$&Je suis sortie aujourd'hui pour savoir o$\grave{\mbox{u}}$ 
vous en $\acute{\mbox{e}}$tiez, \textbf{car} j'ai peur de ne plus pouvoir bouger de 
longtemps.&\textbf{car}\newline
$\langle$отношение причины$\rangle$\\
\hline
Я нарочно прибрала, чтобы кто не поднял, \textbf{а~то} потом поминай как звали!&\textbf{а 
то}\newline
$\langle$отношение\newline отрицательной\newline альтернативы$\rangle$&Je l'ai trouv$\acute{\mbox{e}}$, dans 
une serviette, et justement je l'avais mis de c$\hat{\mbox{o}}$t$\acute{\mbox{e}}$ pour que personne 
ne la ramasse, \textbf{sinon}, hein, adieu la valise!&\textbf{sinon}\newline
$\langle$отношение\newline отрицательной\newline альтернативы$\rangle$\\
\hline
--- Он женится! 
Хочешь об за\-клад, что не женится?~--- воз\-ра\-зил он. --- Да ему Захар и~спать-то помогает, 
\textbf{а~то} жениться!&\textbf{а то}\newline
$\langle$отношение несоответствия$\rangle$&--- Lui, se marier!
Qu'est-ce que tu paries qu'il ne se mariera pas? Il ne peut m$\hat{\mbox{e}}$me pas s'endormir sans 
Zakhar, \textbf{et} tu veux qu'il se marie!&\textbf{et}\newline
$\langle$соединительное отношение$\rangle$\\
\hline
\end{tabular}
\end{center}
\end{table*}


\begin{multicols}{2}

\noindent
 коннектора \textit{а~то} и~его переводов (см.\ в~табл.~2 три
 примера АПС, со\-от\-вет\-ст\-ву\-ющих трем парам фрагментов оригинала и~их 
переводов из табл.~1). Отметим, что в~этих трех АПС коннектор \textit{а~то} 
выражает три раз\-ных внут\-ри\-текс\-то\-вых отношения: причина, отрицательная 
альтернатива, несоответствие. При этом внут\-ри\-текс\-то\-вое 
отношение в~оригинале и~переводе может быть раз\-ным (см.\ ниж\-нюю строку табл.~2).
  

   
  Сформированные АПС дают воз\-мож\-ность определить на \textit{треть\-ем 
этапе} чис\-ло вариантов перевода коннектора \textit{а~то} на французский язык. 
Рас\-смот\-рен\-ные три этапа процесса ста\-ти\-сти\-че\-ской обработки текс\-то\-вой 
информации описаны вербально (словами естественного языка) на примере 
перевода коннектора \textit{а~то} в~НБД.
  
  В статье предлагается подход к~фор\-ма\-ли\-зо\-ванному описанию статистической 
обработки \mbox{текс\-товой} информации в~БД. Ее трансформацию 
в~чис\-ло\-вые значения предлагается назвать  
ИМ-транс\-фор\-ма\-ци\-ей. Она 
сочетает обработку конкретных текс\-тов, формирование в~результате обработки 
вер\-баль\-но-чис\-ло\-вых объектов и~вы\-чис\-ле\-ние значений чис\-ло\-вых 
характеристик. Такая трансформация текс\-то\-вой информации в~чис\-ло\-вые 
характеристики может на отдельных этапах включать \mbox{математические} 
преобразования, но в~целом она к~ним не сво\-дится.
  
  Цель статьи~--- предложить принципы формализованного описания  
ИМ-транс\-фор\-ма\-ции текс\-тов в~БД и~ввести понятия: (1)~пол\-ностью 
обратимая, (2)~час\-ти\-чно обратимая и~(3)~необратимая статистическая обработка 
текс\-то\-вой ин\-фор\-ма\-ции.

\vspace*{-9pt}

\section{Обратимость статистической обработки}

\vspace*{-2pt}

  Три понятия (полностью обратимая, час\-тич\-но обратимая и~необратимая 
ста\-ти\-сти\-че\-ская обработка) опишем на примере определения чис\-ла вариантов 
перевода коннектора \textit{а~то}. В~предыдущем разделе были рас\-смот\-ре\-ны три 
этапа ста\-ти\-сти\-че\-ской обработки текс\-то\-вой информации, которые сформулируем 
в~общем \mbox{виде}:
  \begin{enumerate}[(1)]
\item поиск вхождений ис\-сле\-ду\-емой ЯЕ в~текс\-то\-вой ин\-фор\-ма\-ции~БД;
   \item формирование АПС для ис\-сле\-ду\-емых~ЯЕ;
   \item определение чис\-ло\-вых ста\-ти\-сти\-че\-ских характеристик информации, 
хранимой в~БД.
   \end{enumerate}
   
  В рассмотренном примере на первом этапе были найдены 311~КИО. Из них для 
220~КИО на втором этапе были сформированы 220~АПС. Сейчас более по\-дроб\-но 
рас\-смот\-рим третий этап. В~НБД структура АПС включает два отдельных поля: 
одно для ис\-сле\-ду\-емо\-го коннектора \textit{а~то} (см.\ второй стол\-бец табл.~2) 
и~одно для его перевода, например \textit{car} (см.\ четвертый столбец табл.~2; если 
коннектор не был переведен, то ставится специальная мет\-ка). В~результате 
со\-по\-став\-ле\-ния значений этих двух полей в~220~АПС была сформирована табл.~3 
с~чис\-лом вариантов перевода коннектора \textit{а~то} и~чис\-лом АПС, 
со\-от\-вет\-ст\-ву\-ющих каж\-до\-му варианту. Всего эти АПС содержат 40~вариантов 
перевода. Девять вариантов с~наибольшим чис\-лом АПС, включая вариант 
отсутствия перевода, включены в~первые девять строк табл.~3. Сумма АПС для 
этих девяти вариантов рав\-на~174. Для остальных вариантов (их в~НБД~31) 
отведена одна десятая строка. Сумма АПС для 31~варианта рав\-на~46.



  
  В НБД этапы статистической обработки информации спроектированы как 
пол\-ностью обратимые, что дает воз\-мож\-ность проверки результатов 
ста\-ти-\linebreak\vspace*{-12pt}

\vspace*{12pt}

% \begin{table*}\small %tabl3
   \begin{center}
   \noindent
\parbox{228pt}{{{\tablename~3}\ \ \small{Варианты перевода коннектора \textit{а~то} на французский язык 
(\textit{согласно НБД коннекторов ФИЦ ИУ РАН по состоянию на 08.05.2023})
}}
}
  % \Caption{}
   \vspace*{2ex}
   
 {\small
  \tabcolsep=4pt
   \begin{tabular}{|c|l|c|}
   \hline
№&\multicolumn{1}{c|}{\tabcolsep=0pt\begin{tabular}{c}Вариант перевода\\ коннектора \textit{а~то}\end{tabular}}&
\tabcolsep=0pt\begin{tabular}{c}Число\\ АПС\\ варианта\end{tabular}\\
\hline
1&sinon&83\\
\hline
2&\textit{коннектор в~переводе отсутствует}&43\\
\hline
3&et&11\\
\hline
4&car&11\\
\hline
5&mais&\hphantom{9}9\\
\hline
6&si&\hphantom{9}5\\
\hline
7&sans quoi&\hphantom{9}4\\
\hline
8&sans cela&\hphantom{9}4\\
\hline
9&ou&\hphantom{9}4\\
\hline
10\hphantom{9}&\tabcolsep=0pt\begin{tabular}{l}31 вариант перевода с~числом\\  АПС~3 и~меньше\end{tabular}&46\\
\hline
Итого&40 моделей перевода&220\\
\hline
\end{tabular}
}

\vspace*{3pt}
\end{center}

\columnbreak


%\end{table*}

\noindent
сти\-че\-ской обработки, начиная с~чис\-ло\-вых значений
 (см.\ табл.~3) и~заканчивая 
получением спис\-ка тех АПС, которые были использованы для вы\-чис\-ле\-ния этих 
зна\-че\-ний\footnote{Аннотируемые переводные соответствия могут редактироваться. Поэтому после завершения статистической 
обработки отобранных АПС воз\-мож\-ность их редактирования долж\-на быть от\-клю\-чена.}.
   

  
  Например, число~11 из четвертой строки табл.~3 в~НБД служит гиперссылкой 
на список из~11~АПС, включая первое АПС из табл.~2, каж\-дое из которых имеет 
гиперссылку на со\-от\-вет\-ст\-ву\-ющий КИО, т.\,е.\ пару текс\-то\-во\-го оригинала 
с~коннектором \textit{а~то} и~перевода с~французским коннектором \textit{car}, 
включая пер\-вую пару из табл.~1. Если бы в~АПС эта гиперссылка отсутствовала, 
то ста\-ти\-сти\-че\-ская обработка была бы час\-тич\-но об\-ра\-ти\-мой. А~если и~чис\-ло\-вые 
значения в~табл.~3 не служили бы гиперссылками на списки АПС, то 
ста\-ти\-сти\-че\-ская обработка стала бы пол\-ностью необратимой.

  
\section{Описание этапов статистической обработки}

%\vspace*{-6pt}

  Статистический анализ текстовой информации~--- рас\-про\-стра\-нен\-ный метод 
исследований в~компьютерной линг\-ви\-сти\-ке, ис\-поль\-зу\-ющий БД
и~корпусы текс\-тов, включая параллельные~\cite{6-zac, 7-zac, 8-zac, 9-zac}. 
Вер\-баль\-ное описание этапов ста\-ти\-сти\-че\-ско\-го анализа текс\-то\-вой информации 
рас\-смот\-рим на примере конт\-рас\-тив\-но\-го исследования пары предлогов 
\textit{from}/\textit{fra} в~английском и~норвежском языке (пер\-вый предлог~--- 
английский, второй~--- нор\-веж\-ский)~\cite{10-zac} с~использованием текс\-тов  
анг\-ло-нор\-веж\-ско\-го параллельного корпуса университета Осло~\cite{11-zac}.
  
  Статистические характеристики этой пары предлогов определялись для трех 
видов переводных соответствий\footnote[2]{Эти три 
вида переводного соответствия введены в~работе~\cite{11-zac} для исследования переводов значений 
английского предлога at на араб\-ский язык.}: пред\-лог \textit{from} переведен предлогом 
\textit{fra}, предлог \textit{from} переведен любым другим предлогом, кроме 
\textit{fra}, и~предлог в~переводе не использовался~\cite{12-zac}. \mbox{Целью} работы~\cite{10-zac} ставилось 
вы\-чис\-ле\-ние час\-тот\-ности каждого вида переводного соответствия для каж\-до\-го 
смыс\-ло\-во\-го значения предлога \textit{from}. Приведем крат\-кое описание этапов их 
вы\-чис\-ле\-ния из этой ра\-боты.
  
  Для определения час\-тот\-ности сначала был сформирован массив текс\-тов, 
отобранных из анг\-ло-нор\-веж\-ско\-го параллельного корпуса. В~мас\-сив во\-шли 
только художественные текс\-ты. В~нем был выполнен поиск пар предложений, 
содержащих пред\-лог \textit{from} в~английском ори\-ги\-нале.
  
  Затем был выполнен семантический анализ найден\-ных пар предложений, 
в~результате которого были экспертно определены кон\-текст\-ные значения 
предлога \textit{from} и~про\-став\-ле\-ны виды переводных соответствий для 
\textit{from} (этот пред\-лог переведен предлогом \textit{fra}, другим предлогом или 
никакой пред\-лог в~переводе не использовался). После определения кон\-текст\-ных 
значений предлога \textit{from} и~простановки одного из трех видов были 
определены их час\-тот\-но\-сти для каж\-до\-го кон\-текст\-но\-го значения 
в~сформированном массиве текс\-тов. Например, проведенное исследование 
показало: если пред\-лог \textit{from} в~оригинальных английских предложениях 
имел значение <<причина>> (suffered from high blood pressure), то в~их переводах 
норвежский пред\-лог \textit{fra} не использовался (нулевая час\-тот\-ность). Если 
предлог \textit{from} имел значение <<источник>> (lights from the cars in the road), 
то в~50\% переводов использовался пред\-лог \textit{fra}, в~38\% использовались 
другие предлоги и~в~12\% переводов при переводе \textit{from} пред\-ло\-ги не 
использовались. Вы\-чис\-лен\-ные частотности каж\-до\-го вида переводного 
соответствия для каждого смыс\-ло\-во\-го значения пред\-ло\-га \textit{from} приведены 
в~табл.~4.8 на с.~63 работы~\cite{10-zac}.
  
  Таким образом, в~этой работе дано вербальное описание всех этапов 
ста\-ти\-сти\-че\-ской обработки параллельных текс\-тов, вклю\-ча\-ющее таб\-ли\-цу 
частотностей. В~отличие от ста\-ти\-сти\-че\-ской обработки информации НБД 
коннекторов, в~работе~\cite{10-zac} не ставилась задача обеспечить об\-ра\-ти\-мость 
этапов обработки, т.\,е.\ <<пройти>> от час\-тот\-ности до исходной текс\-то\-вой 
информации.
  
\section{Принципы информационно-математической трансформации}

   Сформулируем предлагаемые принципы формализованного описания 
многоэтапного процесса ста\-ти\-сти\-че\-ской обработки текс\-то\-вой информации в~БД. 
Сначала определяется список этапов ста\-ти\-сти\-че\-ской обработки и~схема связей 
меж\-ду ними. Затем для каж\-до\-го этапа опре\-де\-ля\-ются:
   \begin{itemize}
   \item множества объектов (например, текстовых, вер\-баль\-но-чис\-ло\-вых 
и/или чис\-ло\-вых) на входе и~выходе \mbox{этапа};
   \item идентификаторы для доступа в~БД к~объектам входных и~выходных 
множеств этапа или способ(ы) их опре\-де\-ле\-ния;
   \item процессы преобразования входных множеств объектов в~выходные;
   \item классификационные сис\-те\-мы, ис\-поль\-зу\-емые в~процессе пре\-обра\-зо\-ва\-ния;
   \item возможность до\-сту\-па к~входным множествам объектов этапа либо 
в~процессе обратного преобразования выходных множеств объектов во входные 
(обратимое преобразование), либо сохранение в~БД вход\-ных множеств этапа 
(обратимый этап).
   \end{itemize}
   
   Используя перечисленные принципы, сформулируем описание процесса 
определения чис\-ла вариантов (моделей) перевода \textit{а~то} в~НБД 
коннекторов, рас\-смот\-рен\-но\-го в~начале \mbox{статьи}. Этот процесс включает три этапа, 
которые выполняются по\-сле\-до\-ва\-тельно.
   
   На первом этапе есть только одно входное множество (обозначим его как Input 
Set~1~--- IS1), которое содержит параллельные текс\-ты НБД коннекторов. Эти 
текс\-ты со\-ст\-оят из пар, вклю\-ча\-ющих русское предложение и~его перевод на 
французский\linebreak язык. Каждая пара имеет уникальный идентификатор в~НБД. На 
08.05.2023 было 144\,088 идентификаторов таких пар вход\-но\-го множества IS1\linebreak 
(список идентификаторов обозначим как Input List~1~--- IL1). Выходное 
множество на этом этапе формируется так\-же одно (обозначим его как Output 
Set~1~--- OS1), и~оно содержит предложения русского языка с~коннектором 
\textit{а~то} и~их переводы на французский язык (см.\ табл.~1). На 08.05.2023 
было 311~идентификаторов тех пар вы\-ход\-но\-го множества~OS1, рус\-ские 
предложения которого содержат коннектор \textit{а~то} (список 
идентификаторов элементов множества~OS1 обозначим как OL1).
   
   Преобразование входного множества IS1(IL1) в~выходное~OS1(OL1) 
выполняется с~по\-мощью поиска тех пар в~IS1, которые содержат коннектор 
\textit{а~то}. Его обозначим как R1(\textit{а~то}):\ IS1(IL1)\;$\to$\linebreak $\to$\;OS1(OL1). При 
выполнении этого преобразования классификационные сис\-те\-мы не используются. 
Первый этап обратимый, что обозначим как RR1(\textit{а~то}), так как в~НБД 
коннекторов со\-хра\-ня\-ет\-ся входное множество этого \mbox{этапа}.
   
   Входным множеством IS2 на втором этапе служит выходное множество OS1 
первого этапа, идентификаторы элементов (т.\,е.\ 311~пар, рус\-ские предложения 
которых содержат коннектор \textit{а~то}) \mbox{которого} образуют список~OL1. На 
этом этапе формируются АПС с~использованием методологии 
аннотирования~\cite{3-zac}. Выходное множество (обозначим его как OS2) 
содержит сформированные АПС (см.\ табл.~2), идентификаторы которых 
образуют список~OL2. Преобразование входного мно\-жест\-ва~IS2(OL1) в~вы\-ход\-ное~OS2(OL2) 
выполняется с~по\-мощью аннотирования тех пар, которые содержат 
коннектор \textit{а~то}. Его обозначим как  
A2(\textit{а~то}):\ IS2(OL1)\;$\to$\;OS2(OL2)/CS(LSR), где CS(LSR)~--- это 
классификационная сис\-те\-ма ло\-ги\-ко-се\-ман\-ти\-че\-ских отношений, четыре 
рубрики которой были использованы в~трех АПС в~табл.~2 (отношения причины, 
отрицательной альтернативы, несоответствия и~соединительное отношение). 
Второй этап тоже обратимый, что обозначим как~AR2(\textit{а~то}), так как 
в~НБД коннекторов со\-хра\-ня\-ет\-ся входное множество этого \mbox{этапа}.
   
   Входным множеством IS3 третьего этапа служит множество~OS2, которое 
содержит OL2 сформированных АПС. На этом этапе вы\-чис\-ля\-ет\-ся количество 
вариантов перевода коннектора \textit{а~то} на французский язык (обозначим как 
OL3) и~определяется множество~OS3, каж\-дый элемент которого~--- это число 
АПС для каждого варианта в~НБД коннекторов. Преобразование входного 
мно\-жест\-ва~IS3(OL2) в~вы\-ход\-ное~OS3(OL3) выполняется с~по\-мощью сравнения 
содержимого двух полей АПС (для ис\-сле\-ду\-емо\-го коннектора и~его перевода) 
и~определения чис\-ла пар этих полей, содержимое которых совпадает. Его 
обозначим как М3(\textit{а~то}):\ IS3(OL2)\;$\to$\;OS3(OL3). Преобразование  
М2(\textit{а~то}) третьего этапа обратимо, что обозначим как MR3(\textit{а~то}), 
так как чис\-ло АПС для каж\-до\-го варианта перевода в~НБД коннекторов служит 
гиперссылкой на список этих АПС. Классификационные сис\-те\-мы на этом этапе не 
ис\-поль\-зу\-ются.
   
   Таким образом, на основе сформулированных принципов  
ИМ-транс\-фор\-ма\-ции было получено сле\-ду\-ющее формализованное описание 
трех этапов ста\-ти\-сти\-че\-ской обработки текс\-то\-вой информации в~НБД кон\-нек\-то\-ров:
   \begin{itemize}
   \item RR1(\textit{а~то}):\ IS1(IL1)\;$\to$\;OS1(OL1);
   \item AR2(\textit{а~то}):\ IS2(OL1) \;$\to$\;OS2(OL2)/CS(LSR);
   \item МR3(\textit{а~то}):\ IS3(OL2) \;$\to$\;OS3(OL3).
   \end{itemize}
   
\section{Заключение}
   
   Предлагаемые принципы описания ИМ-транс\-фор\-ма\-ции час\-тич\-но уже 
использовались при проектировании БД нау\-ко\-мет\-ри\-че\-ской  
информации~\cite{13-zac, 14-zac, 15-zac, 16-zac}, которые применялись для 
\mbox{информационного} мониторинга при решении задачи про\-грам\-мно-це-\linebreak ле\-во\-го 
управ\-ле\-ния~\cite{17-zac}. Эти принципы дают воз-\linebreak мож\-ность получать 
формализованное описание ста\-ти\-сти\-че\-ской обработки информации в~БД до начала 
ее проектирования. Это описание может быть использовано как условия, которые 
необходимо выполнить при проектировании БД, если ста\-ти\-сти\-че\-ская обработка 
информации в~ней долж\-на быть обратимой. Их выполнение поз\-во\-лит лицам, 
при\-ни\-ма\-ющим решения, верифицировать с~по\-мощью БД ис\-поль\-зу\-емые ими 
результаты ста\-ти\-сти\-че\-ской обработки. Для проектировщиков БД 
формализованное описание, вклю\-ча\-ющее спецификацию преобразований каж\-до\-го 
этапа, которая в~\mbox{статье} не рас\-смат\-ри\-ва\-ет\-ся, задает требование к~по\-стро\-ению 
необходимых классификационных сис\-тем, а~так\-же разграничивает обратимые 
этапы и~обратимые преобразования в~БД.
   
   В заключение отметим, что необходимость в~пред\-ла\-га\-емых принципах 
описания ИМ-транс\-фор\-ма\-ции, воз\-мож\-но, обуслов\-ле\-на теми же причинами, 
которые Д.\,А.~Пос\-пе\-лов сформулировал при по\-стро\-ении  
ло\-ги\-ко-линг\-ви\-сти\-че\-ских моделей~\cite{18-zac}. Со\-по\-ста\-ви\-тель\-ный 
анализ пред\-ла\-га\-емых принципов и~причин создания этих моделей в~будущем 
поз\-во\-лил бы проверить эту ги\-по\-тезу.
   
{\small\frenchspacing
 { %\baselineskip=12pt
 %\addcontentsline{toc}{section}{References}
 \begin{thebibliography}{99}
\bibitem{1-zac}
\Au{Гончаров А.\,А., Инькова~О.\,Ю.} Методика поиска имплицитных ло\-ги\-ко-се\-ман\-ти\-че\-ских 
отношений в~текс\-те~// Информатика и~её применения, 2019. Т.~13. Вып.~3. С.~97--104.
doi: 10.14357/19922264190314.
\bibitem{2-zac}
\Au{Зализняк Анна А., Круж\-ков~М.\,Г.} База данных безличных глагольных конструкций 
русского языка~// Информатика и~её применения, 2016. Т.~10. Вып.~4. С.~132--141.
doi: 10.14357/19922264160414.
\bibitem{3-zac}
\Au{Гончаров А.\,А., Инькова~О.\,Ю., Круж\-ков~М.\,Г.} Методология аннотирования 
в~надкорпусных базах данных~// Сис\-те\-мы и~средства информатики, 2019. Т.~29. №\,2.  
С.~148--160. doi: 10.14357/08696527190213.
\bibitem{4-zac}
\Au{Нуриев В.\,А., Кружков~М.\,Г.}  Корпусные данные при конт\-рас\-тив\-ном изуче\-нии 
пунк\-ту\-ации~// Сис\-те\-мы и~средства информатики, 2023. Т.~33. №\,1. С.~14--23.
doi: 10.14357/08696527230102.
\bibitem{5-zac}
Handbook of linguistic annotation~/ Eds. N.~Ide, J.~Pustejovsky.~--- Dordrecht, The Netherlands: 
Springer Science\;+\;Business Media, 2017. 1468~p.
\bibitem{6-zac}
Parallel corpora for contrastive and translation studies: New resources and applications~/ Eds. 
I.~Doval, M.\,T.~S$\acute{\mbox{a}}$nchez Nieto.~--- Amsterdam, Philadelphia: John Benjamins, 
2019. 301~p.
\bibitem{7-zac}
Translating and comparing languages: Corpus-based insights~/ Eds. S.~Granger, M.-A.~Lefer.~--- 
Louvain-la-Neuve, Belgique: Presses universitaires de Louvain, 2020. 298~p.
\bibitem{8-zac}
Corpora in translation and contrastive research in the digital age: Recent advances and explorations~/ 
Eds. J.~Lavid-L$\acute{\mbox{o}}$pez, C.~\mbox{Ma$\acute{\mbox{\!\ptb{\i}}}$\,z}-Ar$\acute{\mbox{e}}$valo, 
J.\,R.~Zamorano-Mansilla.~--- Amsterdam, Philadelphia: John Benjamins, 2021. 351~p.
\bibitem{9-zac}
Extending the scope of corpus-based translation studies~/ Eds. S.~Granger, M.-A.~Lefer.~--- London: 
Bloomsbury Academic, 2022. 288~p.
\bibitem{10-zac}
\Au{Savchenko E.} A~contrastive study of the English and Norwegian cognates from and fra: Master's 
Thesis.~--- Oslo, Norway: University of Oslo, 2013. 120~p. {\sf  
https:// www.duo.uio.no/handle/10852/37026}.
\bibitem{11-zac}
The English--Norwegian Parallel Corpus (\mbox{ENPC}). {\sf  
https://www.hf.uio.no/ilos/english/services/\linebreak knowledge-resources/omc/enpc}.
\bibitem{12-zac}
\Au{Hasan A.\,A., Abdullah~I.\,H.} A~cross mapping of temporal at-ba ``Forward and backward 
translation''~// English Language Teaching, 2009. Vol.~2. Iss.~1. P.~80--84. doi: 10.5539/elt.v2n1p80.
\bibitem{13-zac}
\Au{Зацман И.\,М.} Полидоменные модели в~сис\-те\-мах\linebreak оценки инновационного потенциала 
и~ре\-зуль\-та\-тивности научных исследований~// Компьютерная лингвисти\-ка и~интеллектуальные 
технологии: Тр. Междунар. конф. Диалог-2006.~--- М.: РГГУ, 2006. С.~178--183.
\bibitem{14-zac}
\Au{Зацман И.\,М.} Полидоменные модели электронных биб\-лио\-тек сис\-тем мониторинга сферы 
нау\-ки~// Электронные биб\-лио\-те\-ки: перспективные методы и~технологии, электронные 
коллекции: Тр.\ 8-й Всеросс. научн. конф.~--- Ярославль: ЯрГУ им.\ П.\,Г.~Демидова, 2006. 
С.~75--81.
\bibitem{15-zac}
\Au{Зацман И.\,М., Веревкин~Г.\,Ф., Дры\-но\-ва~И.\,В., Кур\-ча\-во\-ва~О.\,А., Ла\-рин~Н.\,В., 
Но\-ре\-кян~Т.\,П.} Моделирование сис\-тем информационного мониторинга как проб\-ле\-ма 
информатики~// Сис\-те\-мы и~средства информатики, 2006. Т.~16. №\,3. С.~257--278.
\bibitem{16-zac}
\Au{Зацман И.\,М., Веревкин~Г.\,Ф., Шубников~С.\,К.} Моделирование сис\-тем мониторинга.~--- 
М.: ИПИ РАН, 2008. 115~с.
\bibitem{17-zac}
\Au{Зацман И.\,М., Веревкин~Г.\,Ф.} Информационный мониторинг сферы нау\-ки в~задачах  
про\-грам\-мно-це\-ле\-во\-го управ\-ле\-ния~// Сис\-те\-мы и~средства информатики, 2006. 
Т.~16. №\,1. С.~185--210.
\bibitem{18-zac}
\Au{Поспелов~Д.\,А.} Ло\-ги\-ко-линг\-ви\-сти\-че\-ские модели в~сис\-те\-мах управ\-ле\-ния.~--- 
М.: Энергоиздат, 1981. 231~с.
\end{thebibliography}

 }
 }

\end{multicols}

\vspace*{-12pt}

\hfill{\small\textit{Поступила в~редакцию 20.06.23}}

\vspace*{6pt}

%\pagebreak

%\newpage

%\vspace*{-28pt}

\hrule

\vspace*{2pt}

\hrule



\def\tit{FORMALIZED DESCRIPTION OF~STATISTICAL INFORMATION PROCESSING IN~DATABASES}


\def\titkol{Formalized description of~statistical information processing in~databases}


\def\aut{V.\,V.~Vakulenko and~I.\,M.~Zatsman}

\def\autkol{V.\,V.~Vakulenko and~I.\,M.~Zatsman}

\titel{\tit}{\aut}{\autkol}{\titkol}

\vspace*{-13pt}


\noindent
Federal Research Center ``Computer Science and Control'' of the Russian Academy 
of Sciences, 44-2~Vavilov Str., Moscow 119333, Russian Federation


\def\leftfootline{\small{\textbf{\thepage}
\hfill INFORMATIKA I EE PRIMENENIYA~--- INFORMATICS AND
APPLICATIONS\ \ \ 2023\ \ \ volume~17\ \ \ issue\ 3}
}%
 \def\rightfootline{\small{INFORMATIKA I EE PRIMENENIYA~---
INFORMATICS AND APPLICATIONS\ \ \ 2023\ \ \ volume~17\ \ \ issue\ 3
\hfill \textbf{\thepage}}}

\vspace*{3pt}




\Abste{The paper presents an overview of the stages of statistical processing of text data, 
from specific informational objects in databases to the values of the numerical characteristics of these objects. 
For example, if the database contains the descriptions of full-text research articles, then they represent specific 
informational objects. With the appropriate population of such a database, the multistage procedure of their processing 
makes it possible to determine the values of the numerical characteristics of the publication activity of a~researcher, 
a~scientific division, and a~scientific organization as a~whole. Such procedures begin with the processing of specific informational objects
 and end with computing of the values of the numerical characteristics of these objects. 
 At intermediate stages, tables and other both verbal and numerical objects may form. 
 If the stages of the statistical processing are designed to be reversible and the database implements the function 
 of verifying the values of the numerical characteristics, then the procedure of their verification begins with 
 the values of the characteristics and ends with access to specific informational objects that were used 
 to compute these values. The paper proposes a~formalized description of the stages of statistical processing 
 of text data in databases. Informational-mathematic transformation (IM-transformation) is the proposed name 
 for such transformation of text data into numerical values. It combines the processing of specific informational 
 objects, the formation of verbal and numerical objects, and the mathematical computation of the values of numerical 
 characteristics. Such transformation of text data may include mathematic processes at certain stages; however, 
 it does not completely reverse back to them. The goal of the paper is to propose the principles of formalized description 
 of IM-transformation of texts in databases. To illustrate this, the paper provides the example of formalizing the
  process of determining the frequency of translation variants of connectives expressing intertextual 
  relations between text fragments in the supracorpora database of connectives developed in the FRC CSC RAS.}

\KWE{informational-mathematic transformation; text information; statistical processing of text 
information; supracorpora database}



\DOI{10.14357/19922264230313}{TYCAEX}

\vspace*{-18pt}

 
     \Ack
     
     \vspace*{-2pt}
     
\noindent
The research was carried out using the infrastructure of the Shared Research Facilities ``High 
Performance Computing and Big Data'' (CKP ``Informatics'') of FRC CSC RAS (Moscow). 
  

%\vspace*{6pt}

  \begin{multicols}{2}

\renewcommand{\bibname}{\protect\rmfamily References}
%\renewcommand{\bibname}{\large\protect\rm References}

{\small\frenchspacing
 {%\baselineskip=10.8pt
 \addcontentsline{toc}{section}{References}
 \begin{thebibliography}{99} 
\bibitem{1-zac-1}
\Aue{Goncharov, A.\,A., and O.\,Yu.~In\-ko\-va.} 2019. Me\-to\-di\-ka po\-iska imp\-li\-tsit\-nykh  
logiko-semanticheskikh ot\-no\-she\-niy v~teks\-te [Methods for identification of implicit logical-semantic 
relations in texts]. \textit{Informatika i~ee Primeneniya~--- Inform. Appl.} 13(3):97--104. 
doi: 10.14357/ 19922264190314.
\bibitem{2-zac-1}
\Aue{Zalizniak, Anna~A., and M.\,G.~Kruzh\-kov.} 2016. Baza dan\-nykh bez\-lich\-nykh  
gla\-gol'\-nykh kon\-struk\-tsiy rus\-sko\-go yazy\-ka [Database of Russian impersonal verbal 
constructions]. \textit{Informatika i~ee Primeneniya~--- Inform. Appl}. 10(4):132--141. doi: 
10.14357/19922264160414.
\bibitem{3-zac-1}
\Aue{Goncharov, A.\,A., O.\,Yu.~Inkova, and M.\,G.~Kruzh\-kov.} 2019. Me\-to\-do\-lo\-giya  
an\-no\-ti\-ro\-va\-niya v~nad\-kor\-pus\-nykh ba\-zakh dan\-nykh [Annotation methodology of 
\mbox{supracorpora} databases]. \textit{Sistemy i~Sredstva Informatiki~--- Systems and Means of Informatics} 
29(2):148--160. doi: 10.14357/ 08696527190213.
\bibitem{4-zac-1}
\Aue{Nuriev, V.\,A., and M.\,G.~Kruzh\-kov.} 2023. Korpusnye dannye pri kontrastivnom izuchenii 
punktuatsii [The parallel corpora perspective on studying contrastive punctuation]. \textit{Sis\-te\-my 
i~Sredstva Informatiki~--- Systems and Means of Informatics} 33(1):14--23. doi: 
10.14357/08696527230102.
\bibitem{5-zac-1}
Ide, N., and J.~Pustejovsky, eds. 2017. \textit{Handbook of linguistic annotation}. Dordrecht, The 
Netherlands: Springer Science\;+\;Business Media. 1468~p.
\bibitem{6-zac-1}
Doval, I., and M.\,T.~Sanchez Nieto, eds. 2019. \textit{Parallel corpora for contrastive and translation 
studies: New resources and applications}. Amsterdam, Philadelphia: John Benjamins. 310~p.
\bibitem{7-zac-1}
Granger, S., and M.-A.~Lefer, eds. 2020. Translating and comparing languages: Corpus-based 
insights. Louvain-la-Neuve, Belgique: Presses universitaires de Louvain. 298~p.
\bibitem{8-zac-1}
Lavid-L$\acute{\mbox{o}}$pez, J., C.~\mbox{Ma$\acute{\mbox{\ptb{\!\i}}}$\,z}-Ar$\acute{\mbox{e}}$valo, and J.\,R.~Zamorano-Mansilla, eds. 2021. \textit{Corpora in 
translation and contrastive research in the digital age: Recent advances and explorations}. 
Amsterdam, Philadelphia: John Benjamins. 351~p.
\bibitem{9-zac-1}
Granger, S., and M.-A.~Lefer, eds. 2022. \textit{Extending the scope of corpus-based translation 
studies}. London: Bloomsbury Academic. 288~p.
\bibitem{10-zac-1}
\Aue{Savchenko, E.} 2013. \textit{A~contrastive study of the English and Norwegian cognates from 
and fra}.  Oslo, Norway: University of Oslo. Master's Thesis. 120~p. Available at:  {\sf 
https:// www.duo.uio.no/handle/10852/37026} (accessed July~10, 2023).
\bibitem{11-zac-1}
The English--Norwegian parallel corpus (\mbox{ENPC}). Available at:   {\sf 
https://www.hf.uio.no/ilos/english/services/\linebreak  knowledge-resources/omc/enpc} (accessed July~10, 2023).
\bibitem{12-zac-1}
\Aue{Hasan, A.\,A., and I.\,H.~Ab\-dullah.} 2009. A~cross mapping of temporal at-ba ``Forward and 
backward translation.'' \textit{English Language Teaching} 2(1):80--84. doi: 10.5539/elt. v2n1p80.
\bibitem{13-zac-1}
\Aue{Zatsman, I.\,M.} 2006. Po\-li\-do\-men\-nye modeli v~sis\-te\-makh otsen\-ki  
in\-no\-va\-tsi\-on\-no\-go po\-ten\-tsi\-a\-la i~re\-zul'\-ta\-tiv\-nosti na\-uch\-nykh is\-sle\-do\-va\-niy 
[Polydomain models for evaluation systems of innovative potential and performance of researches]. 
\textit{Computational Linguistics and Intellectual\linebreak Technologies:   Conference (International) ``Dialog 2006'' 
Proceedings}. Moscow: RGGU. 178--183.
\bibitem{14-zac-1}
\Aue{Zatsman, I.\,M.} 2006. Po\-li\-do\-men\-nye mo\-de\-li elekt\-ron\-nykh bib\-lio\-tek  
sis\-tem mo\-ni\-to\-rin\-ga sfe\-ry nau\-ki [Polydomain models for digital libraries of monitoring 
systems in scientific sphere]. \textit{Elektronnyye biblioteki: perspektivnye metody i~tekhnologii, 
elektronnye kollektsii: Tr. 8-y Vseross. nauchn. konf.} [Digital Libraries: Advanced Methods and 
Technologies, Digital Collectons. 8th All-Russian Research Conference Proceedings]. Yaroslavl': P.\,G.~Demidov Yaroslavl' State University. 
75--81.
\bibitem{15-zac-1}
\Aue{Zatsman, I.\,M., G.\,F.~Ve\-rev\-kin, I.\,V.~Dry\-no\-va, O.\,A.~Kur\-cha\-vo\-va, N.\,V.~La\-rin, and 
T.\,P.~No\-re\-kyan.} 2006. Mo\-de\-li\-ro\-va\-nie sis\-tem in\-for\-ma\-tsi\-on\-no\-go mo\-ni\-to\-rin\-ga 
kak prob\-le\-ma in\-for\-ma\-ti\-ki [Model for systems of information monitoring as a~problem of 
informatics]. \textit{Sis\-te\-my i~Sredstva Informatiki~--- Systems and Means of Informatics}  
16(3):257--278.
\bibitem{16-zac-1}
\Aue{Zatsman, I.\,M., G.\,F.~Ve\-rev\-kin, and S.\,K.~Shub\-ni\-kov}. 2008. \textit{Mo\-de\-li\-ro\-va\-nie 
sis\-tem mo\-ni\-to\-rin\-ga} [Modeling of monitoring systems]. Moscow: IPI RAN. 115~p.
\bibitem{17-zac-1}
\Aue{Zatsman I.\,M., and G.\,F.~Ve\-rev\-kin.} 2006. In\-for\-ma\-tsi\-on\-nyy mo\-ni\-to\-ring sfe\-ry 
nau\-ki v~za\-da\-chakh programmno-tselevogo uprav\-le\-niya [Information monitoring for scientific 
sphere in programme budgeting problems]. \textit{Sis\-te\-my i~Sredstva Informatiki~--- Systems and 
Means of Informatics} 16(1):185--210.
\bibitem{18-zac-1}
\Aue{Pospelov, D.\,A.} 1981. \textit{Logiko-lingvisticheskie mo\-de\-li v~sis\-te\-makh  
uprav\-le\-niya} [Logical-linguistic models in control systems]. Moscow: Energoizdat. 231~p.

\end{thebibliography}

 }
 }

\end{multicols}

\vspace*{-7pt}

\hfill{\small\textit{Received June 20, 2023}} 

\vspace*{-20pt}

\Contr


\vspace*{-2pt}

\noindent
\textbf{Vakulenko Vasily V.} (b.\ 1995)~--- engineer-researcher, Institute of Informatics Problems, 
Federal Research Center ``Computer Science and Control'' of the Russian Academy of Sciences,  
44-2~Vavilov Str., Moscow 119333, Russian Federation; \mbox{vvak@pm.me}

\vspace*{3pt}

\noindent
\textbf{Zatsman Igor M.} (b.\ 1952)~--- Doctor of Science in technology, head of department, 
Institute of Informatics Problems, Federal Research Center ``Computer Science and Control'' of the 
Russian Academy of Sciences, 44-2~Vavilov Str., Moscow 119333, Russian Federation; 
\mbox{izatsman@yandex.ru}



\label{end\stat}

\renewcommand{\bibname}{\protect\rm Литература} 