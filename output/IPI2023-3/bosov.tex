\def\stat{bosov}

\def\tit{ОПТИМАЛЬНАЯ ФИЛЬТРАЦИЯ СОСТОЯНИЯ НЕЛИНЕЙНОЙ ДИНАМИЧЕСКОЙ 
СИСТЕМЫ ПО НАБЛЮДЕНИЯМ СО~СЛУЧАЙНЫМИ ЗАПАЗДЫВАНИЯМИ$^*$}

\def\titkol{Оптимальная фильтрация состояния нелинейной динамической 
системы по наблюдениям} % со случайными запаздываниями}

\def\aut{А.\,В.~Босов$^1$}

\def\autkol{А.\,В.~Босов}

\titel{\tit}{\aut}{\autkol}{\titkol}

\index{Босов А.\,В.}
\index{Bosov A.\,V.}


{\renewcommand{\thefootnote}{\fnsymbol{footnote}} \footnotetext[1]
{Работа выполнялась с~использованием инфраструктуры Центра коллективного пользования 
<<Высокопроизводительные вы\-чис\-ле\-ния и~большие данные>> (ЦКП <<Информатика>>) ФИЦ 
ИУ РАН (г.~Моск\-ва).}}


\renewcommand{\thefootnote}{\arabic{footnote}}
\footnotetext[1]{Федеральный исследовательский центр <<Информатика и~управление>> Российской академии наук;
\mbox{ABosov@frccsc.ru}}

\vspace*{-10pt}




\Abst{Изучается математическая модель нелинейной динамической системы наблюдения 
с~дискретным временем, позволяющая учитывать за\-ви\-си\-мость времени получения наблюдений от 
со\-сто\-яния наблюдаемого объекта. Модель реализует предположение о~том, что время между 
моментом, когда формируется измерение со\-сто\-яния, и~моментом получения измеренного 
со\-сто\-яния наблюдателем зависит случайным образом от положения движущегося объекта. 
Источником такого предположения вы\-сту\-па\-ет процесс наблюдения стационарными средствами за 
автономным подводным аппаратом, в~котором время получения актуальных данных зависит от 
неизвестного рас\-сто\-яния меж\-ду объектом и~наблюдателем. В~отличие от детерминированных 
задержек, фор\-ми\-ру\-емых известным со\-сто\-яни\-ем среды наблюдения, для учета за\-ви\-си\-мости 
временных задержек от неизвестного со\-сто\-яния объекта наблюдения требуется использовать для 
их описания случайные функции. Основным результатом исследования предложенной модели 
стало решение задачи оптимальной фильт\-ра\-ции. Для этого получены рекуррентные байесовские 
соотношения, опи\-сы\-ва\-ющие эволюцию апостериорной плот\-ности ве\-ро\-ят\-ности. Использование 
полученного фильт\-ра для практических целей не пред\-став\-ля\-ет\-ся возможным из-за 
вы\-чис\-ли\-тел\-ьной слож\-ности. Предложенная модель про\-ил\-люст\-ри\-ро\-ва\-на практическим примером 
задачи слежения за движущимся подводным объектом по результатам измерений, вы\-пол\-ня\-емых 
типовыми акус\-ти\-че\-ски\-ми сен\-со\-ра\-ми. Предполагается, что объект движется под водой в~плос\-кости 
с~известной средней ско\-ростью, по\-сто\-ян\-но выполняет хаотические маневры и~наблюдается двумя 
независимыми комплексами акус\-ти\-че\-ских сенсоров, из\-ме\-ря\-ющи\-ми дальности до объекта 
и~на\-прав\-ля\-ющие косинусы. Слож\-ность определения положения такого объекта иллюстрируется 
прос\-тым фильт\-ром, ис\-поль\-зу\-ющим гео\-мет\-ри\-че\-ские свойства измеряемых величин.}

\KW{стохастическая динамическая сис\-те\-ма наблюдения; фильт\-ра\-ция со\-сто\-яния; оптимальный 
байесовский фильтр; сред\-не\-квад\-ра\-тич\-ный критерий оценивания; автономный подводный аппарат; 
акус\-ти\-че\-ский сенсор; слежение за целью}

 \DOI{10.14357/19922264230302}{CFVYJM}
  
%\vspace*{-4pt}


\vskip 10pt plus 9pt minus 6pt

\thispagestyle{headings}

\begin{multicols}{2}

\label{st\stat}

\section{Введение}

     Задача слежения за маневрирующим объектом по косвенным 
наблюдениям с~ошибками пред\-став\-ля\-ет собой типичное приложение 
тео\-рии и~методов сто\-ха\-сти\-че\-ской фильт\-ра\-ции~[1]. Предложение учитывать в~модели 
наблюдений случайные за\-держ\-ки времени их получения от со\-сто\-яния 
наблюдаемого объекта исходит от популярной современной при\-клад\-ной 
об\-ласти автономных подводных аппаратов (Autonomous Underwater Vehicle, 
AUV)~[2]. Приложениям AUV посвящено значительное чис\-ло работ, в~большинстве 
связанных с~задачами управ\-ле\-ния (см., например, специальный 
выпуск~[3]). 

Вмес\-те с~тем и~традиционные задачи слежения за 
ма\-нев\-ри\-ру\-ющим объектом сохраняют ак\-ту\-аль\-ность~[4--11], а~их решение 
обеспечивают методы сто\-ха\-сти\-че\-ской фильт\-ра\-ции. 

Наибольшее 
практическое рас\-про\-стра\-не\-ние здесь, как и~в других приложениях, получили 
модель и~алгоритм фильт\-ра\-ции Калмана~[12] и~основанные на нем 
различные субоптимальные фильт\-ры, начиная с~расширенного фильт\-ра 
Калмана~[13] и~\mbox{вплоть} до концепций дезодорированного или кубатурного 
фильт\-ров Калмана~[14--16]. Исследования и~усовершенствования таких 
субоптимальных фильт\-ров продолжаются и~по на\-сто\-ящее время, в~том чис\-ле 
с~применением к~слежению за AUV~[17]. 

Однако принципиально новых 
идей уже не предлагается, а~изначально имев\-ши\-еся недостатки так 
и~остались. Суть этих недостатков со\-сто\-ит в~том, что ни для какого фильт\-ра 
нельзя гарантировать точ\-ност\-ные характеристики для сколь-ни\-будь 
широкого класса моделей, т.\,е.\ ни несмещенности оценки, ни 
ограниченности ошиб\-ки хотя бы дис\-пер\-си\-ей оце\-ни\-ва\-емо\-го со\-сто\-яния 
гарантировать нельзя, а~значит, всегда есть опас\-ность, что в~другой модели 
фильтр будет неустойчивым и~оценка будет расходиться. Единственным 
под\-тверж\-де\-ни\-ем свойств субоптимального фильт\-ра был и~остается 
практический эксперимент с~конкретной мо\-делью и~фиксированным набором 
па\-ра\-мет\-ров.
     
     Надо отметить, что в~обсуждаемой прикладной об\-ласти и~проб\-ле\-ма 
за\-паз\-ды\-ва\-ющих наблюдений хорошо известна. Временн$\acute{\mbox{ы}}$е за\-держ\-ки обязаны\linebreak 
известному свойству акус\-ти\-че\-ских измерителей, а~именно: за\-ви\-си\-мости 
ско\-рости распространения акус\-ти\-че\-ской волны от температуры, со\-ле\-ности\linebreak 
и~дав\-ле\-ния воды~\cite{18-bos}. Именно это при\-влек\-ло внимание в~\cite{11-bos}, 
и~решение было найде\-но путем объединения данных измерений 
акус\-ти\-че\-ских датчиков с~информацией от других датчиков в~\mbox{бортовой} 
инерциальной навигационной сис\-те\-ме. Более типичный вариант борьбы 
с~за\-паз\-ды\-ва\-ющи\-ми наблюдениями со\-сто\-ит в~том, чтобы оценивать время 
за\-паз\-ды\-ва\-ния и~учитывать его, корректируя обыч\-ный фильтр~[19]. 
Интересно, что за\-паз\-ды\-ва\-ние сигнала объясняется не только природой 
подводных наблюдений, но присутствует и~в~химических процессах~[20].
     
     Следует учитывать, что модели с~детерминированным временем 
за\-паз\-ды\-ва\-ния имеют ограниченное применение в~связи с~тем, что 
в~ре\-аль\-ности время за\-паз\-ды\-ва\-ния случайно и~существенно зависит от 
условий наблюдения, а~главное, оно меняется со временем. Эти изменения 
могут быть существенны. В~модели, которая используется в~статье, время 
за\-паз\-ды\-ва\-ния наблюдений описывается случайным процессом, который 
является известной функцией со\-сто\-яния сис\-те\-мы. Это делает модель 
максимально адекватной при описании результатов работы акус\-ти\-че\-ских 
сенсоров. При этом формально модель сис\-те\-мы наблюдения остается 
мо\-делью марковского процесса с~дискретным временем, к~ней может быть 
применена классическая процедура байесовской фильт\-ра\-ции~[21], т.\,е.\ 
записаны рекуррентные соотношения для апостериорной плот\-ности 
ве\-ро\-ят\-ности. Эти соотношения со\-став\-ля\-ют основной тео\-ре\-ти\-че\-ский 
результат \mbox{статьи}. Но этот формальный успех оказывается полезным лишь для 
того, чтобы обосновать не\-воз\-мож\-ность практической реализации ни самого 
оптимального фильт\-ра, ни его субоптимальных упрощений. Причины этого 
со\-сто\-ят в~том, что формальное приведение урав\-не\-ний сис\-те\-мы наблюдения 
к~традиционному виду чу\-до\-вищ\-но увеличивает раз\-мер\-ность. Поэтому 
полученные ниже урав\-не\-ния оптимальной фильт\-ра\-ции следует 
рас\-смат\-ри\-вать только как однозначное осно\-ва\-ние для поиска других, более 
ориентированных на реализацию методов. Убедиться в~том, что такие 
методы действительно нуж\-ны, поз\-во\-ля\-ет описанный в~по\-след\-нем разделе 
статьи чис\-лен\-ный эксперимент. Прос\-тей\-шая модель движения AUV 
в~плос\-кости с~известной средней ско\-ростью и~постоянными хаотическими 
маневрами плюс типовые наблюдатели, из\-ме\-ря\-ющие дальности  
и~на\-прав\-ля\-ющие косинусы до объекта, поз\-во\-ля\-ют без труда записать 
<<естественный>> фильтр, используя гео\-мет\-ри\-че\-скую связь из\-ме\-ря\-емых 
величин и~координат объекта. Такой фильтр в~предложенной модели 
приводит к~не\-при\-ем\-ле\-мо\-му падению качества оценивания по сравнению 
с~оцениванием в~сис\-те\-ме, не пред\-по\-ла\-га\-ющей временн$\acute{\mbox{ы}}$е за\-держки. 

\section{Система наблюдения со~случайными временными задержками}

     В модели используется дискретное время. Предполагается, что процесс 
фильт\-ра\-ции начинается в~момент $t\hm=0$, т.\,е.\ в~этот момент вы\-чис\-ля\-ет\-ся 
пер\-вая оцен\-ка со\-сто\-яния. При этом наблюдения по\-сту\-па\-ют с~временн$\acute{\mbox{о}}$й 
за\-держ\-кой, максимально воз\-мож\-ная за\-держ\-ка известна и~рав\-на~$T$, поэтому 
со\-сто\-яние сис\-те\-мы начинает формироваться в~моменты $t\hm= -T, -T+1, 
\ldots , 0, 1, \ldots$ и~начальное со\-сто\-яние задается в~момент $-T\hm-1$.
     
     Уравнения состояния и~наблюдений записываются в~самой ти\-пич\-ной 
форме с~аддитивными, не зависящими от со\-сто\-яния шумами. Именно такую 
модель сис\-те\-мы наблюдения использует большинство субоптимальных 
фильт\-ров. Отличие от канонической записи со\-сто\-ит только в~описании 
момента времени, в~который было сформировано со\-сто\-яние, измеренное 
текущим наблюде\-нием:
{\looseness=1

}


\vspace*{8pt}

\noindent
     \begin{equation}
     \left.
     \begin{array}{rlrl}
     x_t&= \varphi_t\left(x_{t-1}\right)+w_t\,, &\quad x_{-T-1} &=\eta\,;\\[6pt]
     y_t &= \psi_t\left(x_{t-\tau_t}\right) +v_t\,,&\enskip \tau_t&=\theta_t(x_t)\,.
     \end{array}
     \right\}
     \label{e1-bos}
     \end{equation}
     Здесь $x_t\in \mathbb{R}^{p_x}$~--- век\-тор со\-сто\-яния сис\-те\-мы; 
$w_t\hm\in \mathbb{R}^{p_x}$~--- дискретный белый шум, мо\-де\-ли\-ру\-ющий 
возмущения; $\eta\hm\in \mathbb{R}^{p_x}$~--- век\-тор начальных усло\-вий; 
$y_t\hm\in \mathbb{R}^{q_y}$~--- век\-тор косвенных наблюдений; $v_t\hm\in 
\mathbb{R}^{q_y}$~--- дискретный белый шум, мо\-де\-ли\-ру\-ющий ошиб\-ки 
измерений; век\-то\-ры~$\eta$, $w_t$ и~$v_t$ предполагаются независимыми 
в~со\-во\-куп\-ности.
     
     Процесс~$\tau_t$ моделирует за\-держ\-ки наблюдений, описывая их 
известной функцией~$\theta_t$ со\-сто\-яния~$x_t$. Элементы вектора 
$\tau_t\hm\in \mathbb{R}^{q_y}$ задают временн$\acute{\mbox{ы}}$е\linebreak за\-держ\-ки для 
со\-от\-вет\-ст\-ву\-ющих компонентов вектора~$y_t$, каж\-дый компонент 
моделируется случайной по\-сле\-до\-ва\-тель\-ностью, элементы которой~--- 
дискретные случайные величины со значениями в~множестве $\{0,1,\ldots , 
T\}$, т.\,е.\ $(\tau_t)_i\hm\in \{0,1,\ldots, T\}$, $i\hm=1,\ldots , q_y$. Чтобы это 
определение было корректным, будем считать, что в~(\ref{e1-bos}) и~всюду 
далее через~$x_{t-\tau_t}$ обозначен со\-став\-ной век\-тор, содержащий 
со\-сто\-яния~$x_t$, со всеми сдвигами~$(\tau_t)_i$, т.\,е.\ 
$$
x_{t-\tau_t}\hm= 
\left( x^\prime_{t-(\tau_t)_1}, \ldots , x^\prime_{t-(\tau_t)_{q_y}}\right)^\prime 
\hm\in \mathbb{R}^{q_y p_x}.
$$
     
     Рассматривается задача оценивания со\-сто\-яния~$x_t$ по наблюдениям 
$y_s$, $s\hm= 0,1,\ldots , t$, критерий точ\-ности оцен\-ки~$\tilde{x}_t$ 
сред\-не\-квад\-ра\-тич\-ный: ${\sf E}\left\{ \| x_t\hm- \tilde{x}_t\|^2\right\}$, 
${\sf E}\{x\}$~--- математическое ожидание~$x$, $\| x\|$~--- обыч\-ная 
евклидова норма век\-то\-ра~$x$. Здесь использовано обозначение 
<<${}^\prime$>> для операции транс\-по\-ни\-ро\-ва\-ния.
     
     Таким образом, единственное отличие по\-став\-лен\-ной задачи от 
традиционной задачи фильт\-ра\-ции со\-сто\-яния сис\-те\-мы наблюдения 
в~дискретном времени со\-сто\-ит в~запаздывании наблюдений на случайное 
время~$\tau_t$. Формально даже этого отличия нет, потому что в~сле\-ду\-ющем 
разделе показано, как при\-вес\-ти сис\-те\-му~(\ref{e1-bos}) к~традиционной 
марковской фор\-ме записи. Поэтому оптимальное решение $\hat{x}_t\hm= 
\argmin_{\tilde{x}_t} {\sf E} \left\{ \| x_t\hm- \tilde{x}_t\|^2\right\}$ является 
условным\linebreak математическим ожиданием~$x_t$ относительно 
наблюдений~$y_s$, $s\hm=0,1,\ldots , t$, а~значит, для его вы\-чис\-ле\-ния 
достаточно знать апостериорную плот\-ность ве\-ро\-ят\-ности~$x_t$ 
относительно~$y_s$, $s\hm= 0,1,\ldots , t$. Далее будут выписаны 
апостериорные плотности для сис\-те\-мы~(\ref{e1-bos}) и~$t\hm= 0,1,\ldots$ 
в~форме рекуррентных байесовских соотношений~\cite{21-bos} в~условиях, 
когда нуж\-ные плотности ве\-ро\-ят\-ности существуют, т.\,е.\ при определенных 
ограничениях на не\-пре\-рыв\-ность нелинейных функций и~возмущений 
в~(\ref{e1-bos}).
     
\section{Оптимальная фильтрация}

     Немарковская система наблюдения~(\ref{e1-bos}) может быть записана 
в~форме марковского процесса, име\-юще\-го расширенный век\-тор со\-сто\-яния. 
Формально этот расширенный процесс будет иметь тот же вид~(\ref{e1-bos}), 
но с~$\tau_t\hm=0$.
     
     Расширенный вектор со\-сто\-яния далее обозначается $\mathbf{x}_t 
\hm\in \mathbb{R}^{(T+1)p_x}$ и~пред\-став\-ля\-ет собой со\-став\-ной век\-тор, 
вклю\-ча\-ющий все со\-сто\-яния сис\-те\-мы от момента времени $t\hm- T$ до 
текущего момента~$t$, т.\,е.\ $\mathbf{x}_t \hm= \left( x^\prime_{t-T}, \ldots , 
x^\prime_{t-1}, x_t^\prime\right)^\prime$.  Отметим, что этот век\-тор 
не нуж\-но путать с~со\-став\-ным вектором $x_{t-\tau_t} \hm\in \mathbb{R}^{q_y 
p_x}$, использованным в~записи~(\ref{e1-bos}). В~$x_{t-\tau_t}$ объединены 
те со\-сто\-яния, которые сдвинуты на фактические за\-держ\-ки~$\tau_t$ для 
каж\-до\-го элемента измерений, $\mathbf{x}_t$ объединяет со\-сто\-яния, 
сдвинутые на все воз\-мож\-ные значения $0,1,\ldots , T$ за\-дер\-жек.  Урав\-не\-ния 
для~$\mathbf{x}_t$ име\-ют~вид:

\noindent
     \begin{equation*}
     \begin{matrix}
     \left( \mathbf{x}_t\right)_1^{p_x} =\left( \mathbf{x}_{t-
1}\right)^{2p_x}_{p_x+1}\,;\\[6pt]
     \cdots\\[6pt]
     \left( \mathbf{x}_t\right)^{Tp_x}_{(T-1)p_x+1} = \left( \mathbf{x}_{t-
1}\right)^{(T+1)p_x}_{Tp_x+1}\,;\\[6pt]
        \hspace*{-19mm} \left( \mathbf{x}_t\right)_{Tp_x+1}^{(T+1)p_x} ={}\\[6pt]
{} =\varphi_t\left( \left( 
\mathbf{x}_{t-1}\right)_{Tp_x+1}^{(T+1)p_x}\right)+w_t\,;
     \end{matrix}\,
     \left\{
     \vphantom{\begin{matrix}
     \left( \mathbf{x}_t\right)_1^{p_x} =\left( \mathbf{x}_{t-
1}\right)^{2p_x}_{p_x+1}\,;\\[6pt]
     \cdots\\[6pt]
     \left( \mathbf{x}_t\right)^{Tp_x}_{(T-1)p_x+1} = \left( \mathbf{x}_{t-
1}\right)^{(T+1)p_x}_{Tp_x+1}\,;\\[12pt]
        \hspace*{-19mm} \left( \mathbf{x}_t\right)_{Tp_x+1}^{(T+1)p_x} ={}\\[6pt]
{} =\varphi_t\left( \left( 
\mathbf{x}_{t-1}\right)_{Tp_x+1}^{(T+1)p_x}\right)+w_t\,,
     \end{matrix}}
          \begin{matrix}
          \\[-16pt]
     x_{t-T}\\[6pt]
      \cdots\\[10pt] 
      x_{t-1}\\[10pt]
    \hspace*{-19mm} x_t=\\[6pt]
     =\varphi_t( x_{t-1}) +w_t
     \end{matrix}
     \right\}\!,\hspace*{-0.7pt}
     \end{equation*}
где через $( \mathbf{x})_i^j$ обозна\-чен под\-век\-тор век\-то\-ра~$\mathbf{x}$ 
с~элементами от $i$-го до $j$-го.

     Обозначив такое преобразование век\-то\-ра со\-сто\-яния через~$\Phi_t$, 
а~со\-от\-вет\-ст\-ву\-ющий ад\-ди\-тив\-ный шум~--- через $\mathbf{w}_t\hm= \left( 
0^\prime, \ldots, 0^\prime, w_t^\prime\right)^\prime$, получаем урав\-не\-ние со\-сто\-яния 
в~\mbox{виде}:
     \begin{equation}
     \mathbf{x}_t=\Phi_t(\mathbf{x}_t) +\mathbf{w}_t\,.
     \label{e2-bos}
     \end{equation}
     
     Для записи соотношения для наблюдателя определим мат\-рич\-ную 
функцию $\Psi_t(\mathbf{x}_t) \hm\in \mathbb{R}^{(T+1) %\times 
q_y}$ сле\-ду\-ющим обра\-зом:
     $$
     \Psi_t(\mathbf{x}_t) =\begin{pmatrix}
     \psi_t^\prime \left( (\mathbf{x}_t)_1^{p_x}\right)\\[3pt]
     \psi_t^\prime \left( (\mathbf{x}_t)_{(T-1)p_x+1}^{Tp_x}\right)\\[3pt]
     \cdots\\[3pt]
     \psi_t^\prime \left( (\mathbf{x}_t)_{Tp_x+1}^{(T+1)p_x} \right)
     \end{pmatrix} = \left(
     \vphantom{\begin{pmatrix}
     \psi_t^\prime \left( (\mathbf{x}_t)_1^{p_x}\right)\\[3pt]
     \psi_t^\prime \left( (\mathbf{x}_t)_{(T-1)p_x+1}^{Tp_x}\right)\\[6pt]
     \cdots\\[3pt]
     \psi_t^\prime \left( (\mathbf{x}_t)_{Tp_x+1}^{(T+1)p_x} \right)
     \end{pmatrix}}
     \begin{array}{c}
     \\[-17pt]
     \psi_t^\prime (x_{t-T})\\[9pt]
     \psi_t^\prime (x_{t-T+1})\\[1pt]
     \cdots\\[6pt]
     \psi_t^\prime(x_t)
     \end{array}
     \right).
     $$
     
Таким образом, в~строках этой мат\-ри\-цы собраны все воз\-мож\-ные в~момент~$t$ 
из-за задержек~$\tau_t$ наблюдения без шума. Для описания 
самих задержек так\-же введем мат\-рич\-ную функцию $\Theta_t (\mathbf{x}_t) 
\hm\in \mathbb{R}^{(T+1)q_y%\times 
}$, при\-ни\-ма\-ющую значения 
в~соответствии со сле\-ду\-ющим правилом:
     \begin{multline*}
     \left( \Theta_t(\mathbf{x}_t)\right)_{i,j} =\begin{cases} 
     1, & \mbox{если } \left( \theta_t(x_t)\right)_i=T-j+1\,;\\
     0, &\mbox{в\ противном\ случае},
     \end{cases}\\
     i=1,\ldots , q_y,\ j=1,\ldots , T+1\,.
     \end{multline*}
     Таким образом, элемент $(i,j)$ мат\-ри\-цы $\Theta_t(\mathbf{x}_t)$ 
принимает значение~1, если наблюдению $(y_t)_i$ соответствует за\-держ\-ка 
$(\tau_t)_i\hm= T\hm- j\hm+1$. 
     
     Эти два обозначения поз\-во\-ля\-ют записать уравнение наблюдений в~\mbox{виде}:
     \begin{equation}
     y_t=\Theta_t(\mathbf{x}_t)\Psi_t(\mathbf{x}_t) +v_t\,.
     \label{e3-bos}
     \end{equation}
     
     Таким образом, (\ref{e2-bos}), (\ref{e3-bos})~--- это каноническая форма 
марковской сис\-те\-мы наблюдения с~дискретным временем и~аддитивными 
независимыми шу\-мами.
     
     Значения, которые может принимать 
функция~$\Theta_t(\mathbf{x}_t)$, обозначим~$\left[ \Theta_t\right]_k$, 
$k\hm=1,\ldots , (T\hm+1)q_y$. Порядковый номер~$k$ мож\-но задать, 
по\-ложив
     \begin{equation}
     k=j_{k,1} +(T+1) j_{k,2} +\cdots + (T+1)^{q_y-1} j_{k,q_y}\,,
     \label{e4-bos}
     \end{equation}
     
     \vspace*{-3pt}
     
     \pagebreak
     
     \noindent
где $j_{k,i}$~--- номер строки $i$-го столбца мат\-ри\-цы~$\left[ 
\Theta_t\right]_k$, в~котором сто\-ит~1.

     Эти же обозначения дают воз\-мож\-ность пе\-ре\-счи\-тать воз\-мож\-ные 
значения функции наблюдателя $\Theta_t(\mathbf{x}_t)\Psi_t (\mathbf{x}_t)$. 
Обозначим значение этой функ\-ции с~тем же номером~$k$ из~(\ref{e4-bos}) 
через~$\left[ \Theta_t \Psi_t\right]_k$. \mbox{Тогда}
     $$
     \left[ \Theta_t \Psi_t\right]_k= \left[ \Theta_t\Psi_t(\mathbf{x}_t)\right]_k= 
\begin{pmatrix}
     \left( \psi_t(x_{t-j_{k,1}})\right)_1\\[3pt]
     \left( \psi_t(x_{t-j_{k,2}})\right)_2\\[3pt]
     \cdots\\[3pt]
     \left( \psi_t(x_{t-j_{k,1}})\right)_{q_y}
     \end{pmatrix}.
     $$
     Здесь $(\psi_t)_i$~--- $i$-я координата вектора~$\psi_t$. Таким образом, 
$\left[ \Theta_t \Psi_t\right]_k$ со\-сто\-ит из наблюдений без шу-\linebreak ма, смещенных на 
величины задержек, и~соответствует порядковому номеру~$k$, заданному 
для мат\-ри\-цы~$\left[ \Theta_t\right]_k$ соотношением~(\ref{e4-bos}). Кроме 
того, пронумерованы воз\-мож\-ные значения вектора \mbox{за\-паз\-ды\-ва\-ний}~$\tau_t$:
     \begin{multline*}
     [\tau_t]_k =\left[ \theta_t(x_t)\right]_k, \enskip
     \left( [\tau_t]_k\right)_i=T-j_{k,i} +1\,,\\
     i=1,\ldots , q_y\,.
     \end{multline*}
     
     Сделанные обозначения несколько громоздки, но они нуж\-ны, чтобы 
пред\-ста\-вить оптимальный байесовский фильтр. С~этой \mbox{целью} введем 
сле\-ду\-ющие правила именования вероятностных характеристик. Для 
случайных векторов $x\hm\in \mathbb{R}^p$, $y\hm\in \mathbb{R}^q$ через 
$(x^\prime, y^\prime)^\prime \hm\in \mathbb{R}^{p+q}$ обозначим со\-став\-ной 
вектор. %с~использованием обозначения <<${}^\prime$>> %(транспонирование). 
Ис\-поль\-зу\-емые в~соотношениях ниже плот\-ности ве\-ро\-ят\-ности будем 
обозначать так: $f_x(X)$~--- маргинальная плот\-ность~$x$,  
$f_{x,y}(X,Y)$~--- плот\-ность со\-став\-но\-го вектора $(x^\prime, 
y^\prime)^\prime$, $f_{x\vert y} (X\vert Y)$~--- услов\-ная плот\-ность~$x$ 
относительно~$y$ при дополнительном предположении $f_y(Y)\hm>0$. 
Дополнительно от\-ме\-тим,~что
     \begin{itemize}
\item символы~$x$ используются для обозначения случайных век\-то\-ров 
в~нотациях исходной сис\-те\-мы наблюдения~(\ref{e1-bos});
\item символы~$X$ используются для обозначения аргументов плотностей 
вероятности, со\-от\-вет\-ст\-ву\-ющих~$x$;
\item символы $\mathbf{x}$ используются для обозначения случайных 
векторов в~нотации расширенного со\-сто\-яния в~сис\-те\-ме 
 наблюдения~(\ref{e2-bos}), (\ref{e3-bos});
\item символы $\mathbf{X}$ используются для обозна\-че\-ния аргументов 
плотностей ве\-ро\-ят\-ности, со\-от\-вет\-ст\-ву\-ющих~$\mathbf{x}$;
\item обозначения $y^t$ используются для век\-то\-ра всех наблюдений до 
момента~$t$ включительно, т.\,е.\ $y^t\hm= \left( y_0^\prime, \ldots , 
y_t^\prime\right)^\prime$, и~$Y^t$~--- для со\-от\-вет\-ст\-ву\-юще\-го аргумента  
плот\-ности ве\-ро\-ят\-ности.
\end{itemize}

     Для условного распределения со\-сто\-яния~$\mathbf{x}_t$ 
сис\-те\-мы~(\ref{e2-bos}) относительно наблюдений~$y_s$, $s\hm= 0,\ldots , t$, 
(\ref{e3-bos}) мож\-но формально записать рекуррентные байесовские 
соотношения для апостериорной плот\-ности ве\-ро\-ят\-ности~\cite{21-bos}:
     \begin{multline*}
     f_{\mathbf{x}_t\vert y^t} \left( \mathbf{X}_t\vert Y^t\right) = {}\\
    \! {}=\!\!
     \Bigg(\!\!\int \!\! f_{\mathbf{x}_t\vert \mathbf{x}_{t-1}} (\mathbf{X}_t\vert 
\mathbf{X}_{t-1}) f_{\mathbf{x}_{t-1}\vert y^{t-1}} (\mathbf{X}_{t-1}\vert 
Y^{t-1}) d\mathbf{X}_{t-1} \times{}\hspace*{-1.10406pt}\\
{}\times  f_{y_t\vert \mathbf{x}_t} (Y_t\vert 
\mathbf{X}_t)\!\Bigg) \!\Bigg/\!  \Bigg (\! \iint\!\! f_{\mathbf{x}_t\vert \mathbf{x}_{t-1}} (\mathbf{X}_t\vert 
\mathbf{X}_{t-1})\times{}\\
{}\times f_{\mathbf{x}_{t-1}\vert y^{t-1}} (\mathbf{X}_{t-1}\vert 
Y^{t-1})\,d\mathbf{X}_{t-1}\cdot f_{y_t\vert \mathbf{x}_t} (Y_t\vert 
\mathbf{X}_t)\,d\mathbf{X}_t \Bigg).
     \end{multline*}
Это выражение мож\-но использовать при условии, что все плотности 
ве\-ро\-ят\-ности в~нем существуют. Для сис\-те\-мы~(\ref{e2-bos}), (\ref{e3-bos}) это 
не так для переходной плот\-ности $f_{\mathbf{x}_t\vert \mathbf{x}_{t-1}} 
\left( \mathbf{X}_t\vert \mathbf{X}_{t-1}\right)$. Кроме того, нуж\-но получить 
явный вид плот\-ности $f_{y_t\vert\mathbf{x}_t}\left( Y_t\vert \mathbf{X}_t\right)$.
        
     Выполненные в~следующей тео\-ре\-ме преобразования поз\-во\-ля\-ют 
вмес\-то переходной плот\-ности $f_{\mathbf{x}_t\vert\mathbf{x}_{t-1}}\left( 
\mathbf{X}_t\vert\mathbf{X}_{t-1}\right)$ расширенного со\-сто\-яния 
использовать переходную плот\-ность $f_{x_t\vert x_{t-1}}\left( X_t\vert X_{t-1}\right)$ 
исходного со\-сто\-яния и,~соответственно, вмес\-то~$\int\cdot d\mathbf{X}_{t-1}$, 
который вы\-чис\-ля\-ет\-ся по пространству~$\mathbb{R}^{(T+1)p_x}$, считать интеграл~$\int\cdot dX_{t-T-1}$ по 
пространству~$\mathbb{R}^{p_x}$. Кроме того, записывается явный вид 
плот\-ности $f_{y_t\vert \mathbf{x}_t} \left( Y_t\vert \mathbf{X}_t\right)$.
     
     \smallskip
     
     \noindent 
     \textbf{Теорема.}\ \textit{Пусть для сис\-те\-мы}~(\ref{e1-bos}) 
\textit{задано $T\hm>1$ и~вы\-пол\-нено}:
     \begin{itemize}
\item \textit{плотность ве\-ро\-ят\-ности возмущений $f_{w_t}(W_t)$, $t\hm= -T, -T+1, \ldots$, 
непрерывна и~векторы~$w_t$ имеют конечные вторые мо\-менты};
\item \textit{плотность ве\-ро\-ят\-ности ошибок наблюдений $f_{v_t}(V_t)$, 
$t\hm= 0,1,\ldots$, непрерывна, не\-от\-ри\-ца\-тель\-на и~векторы~$v_t$ имеют конечные вторые мо\-менты};
\item \textit{плотность ве\-ро\-ят\-ности начального условия $f_\eta(X_{-T-1})$ 
не\-пре\-рыв\-на и~вектор~$\eta$ имеет конечный второй мо\-мент};
\item \textit{функции $\varphi_t(x)$ и~$\psi_t(x)$ не\-пре\-рыв\-ны и~удовле\-тво\-ря\-ют 
условию линейного рос\-та, т.\,е.\ существует константа}~$C$: 
$$
\| \varphi_t(x)\|^2\hm+ \| \psi_t(x)\|^2\hm< C\left(1\hm+ \| x\|^2\right),\ x\in \mathbb{R}^{p_x}.
$$
\end{itemize}
     
     \textit{Тогда для апостериорной плот\-ности ве\-ро\-ят\-ности 
$\rho_t\hm= \rho_t\left(\bm{X}_t\vert Y^t\right)$ расширенного со\-сто\-яния 
$\mathbf{x}_t\hm= \left( x^\prime_{t-T}, \ldots , %x^\prime_{t-1}, 
x_t^\prime\right)^\prime$, $t\hm= 0,1,\ldots$, сис\-те\-мы, заданного 
соотношением}~(\ref{e2-bos}), \textit{относительно} \mbox{\textit{наблюдений}} $y^t\hm= 
\left( y_0^\prime, \ldots , y_t^\prime\right)^\prime$, $t\hm= 0,1,\ldots$, 
 (\ref{e3-bos}) \textit{и оценки оптимального байесовского фильт\-ра 
$\hat{x}_t\hm= {\sf E} \{ x_t\vert Y^t\}$  
со\-сто\-яния~$x_t$ выполнены рекуррентные ра\-вен\-ства}:

\pagebreak

\noindent
     \begin{equation}
     \left.
    \hspace*{-2mm} \begin{array}{l}
     \rho_t= \Bigg( f_{w_t}(X_t-\varphi_t(X_{t-1}))  \displaystyle\sum\limits_k I 
\left(\Theta_t(X_t)={}\right.\\[6pt]
\hspace*{10mm}\left.{}= [\tau_t]_k\right) f_{v_t} \left(Y_t - [\Theta_t\Psi_t(\bm{X}_t)]_k\right)\times{}\\[6pt]
\hspace*{10mm}{}\times \displaystyle\int \rho_{t-1} \,dX_{t-T-1} \Bigg) \!\Bigg/  \!
\Bigg( \int \Bigg( f_{w_t} 
\left(X_t -{}\right.\\[6pt]
\left.{}-\varphi_t\left(X_{t-1}\right)\right) \displaystyle \sum\limits_k I \left(\theta_t(X_t) =[\tau_t]_k\right) f_{v_t} 
\left(Y_t -{}\right.\\[6pt]
\hspace*{4mm}\left.{}-\left[\Theta_t \Psi_t (\bm{X}_t)\right]_k \right)\displaystyle\int \rho_{t-1} dX_{t-T-1}\Bigg) \,d\bm{X}_t
\Bigg);
\\[6pt]
     \hat{\mathbf{x}}_t= \displaystyle \int \bm{X}_t \rho_t \left( \bm{X}_t\vert 
Y^t\right) d\bm{X}_t\,,\ \hat{x}_t=\left( 
\hat{\mathbf{x}}_t\right)_{Tp_x+1}^{(T+1)p_x}
     \end{array}
\!\!     \right\}
     \label{e5-bos}
     \end{equation}
\textit{с начальным условием}
\begin{multline*}
\rho_{-1} \left( \bm{X}_{-1}\vert Y^{-1}\right) ={}\\
{}=\rho_{-1}\left( \bm{X}_{-1}\right) 
=\rho_{-1} \left( X_{-T-1}, \ldots , X_{-1}\right)={}\\
{}= f_\eta \left( X_{-T-1}\right) f_{w-T}\left( X_{-T} -\varphi_{-T} \left( X_{-T-1}\right)\right) \cdots\\
\cdots f_{w_{-1}} \left( X_{-1} -\varphi_{-T}\left( X_{-2}\right)\right)
\end{multline*}
\textit{и правилом нумерации $k\hm= 1,\ldots , q_y$, определенным  в}~(\ref{e4-bos}).

\smallskip

     В рекуррентных соотношениях~(\ref{e5-bos}) опущены аргументы 
и~упрощена запись суммы. Отметим, что не от\-но\-ся\-щи\-еся к~необходимым 
условия огра\-ни\-чен\-ности вторых моментов наблюдений~$y_t$ 
сформулированы для того, чтобы в~модели можно было использовать 
помимо байесовского фильт\-ра другие, неоптимальные оцен\-ки. Также 
заметим, что нет формальных причин, пре\-пят\-ст\-ву\-ющих компьютерной 
реализации этих формул. Однако отсутствует ре\-а\-ли\-стич\-ность выполнения 
такого компьютерного расчета из-за рос\-та раз\-мер\-ности, мас\-штаб которого 
определяется значением~$T$. В~чис\-лен\-ном эксперименте, об\-суж\-да\-емом 
далее в~\mbox{статье}, $T\hm=50$. Значит, даже при небольших $p_x\hm=2$ 
и~$q_y\hm=4$ интегралы в~(\ref{e5-bos}) будут по 
пространству~$\mathbb{R}^{100}$, а~сла\-га\-емых будет~404. При таких 
па\-ра\-мет\-рах расчеты со\-глас\-но~(\ref{e5-bos}) невозможно выполнить на 
практике. Именно этот тезис, т.\,е.\ отсутствие практических перспектив 
использования в~рас\-смат\-ри\-ва\-емой задаче классических алгоритмов 
фильт\-ра\-ции, хотя они и~могут быть записаны в~окончательном замкнутом 
виде, со\-став\-ля\-ет формальный итог исследования.
     
     \noindent
     Д\,о\,к\,а\,з\,а\,т\,е\,л\,ь\,с\,т\,в\,о\,.\ \ Представим искомую плот\-ность 
ве\-ро\-ят\-ности в~виде:
     \begin{multline*}
     f_{\mathbf{x}_t\vert y^t} \left( \mathbf{X}_t\vert Y^t\right) 
=\fr{f_{\mathbf{x}_t, y^t} (\mathbf{X}_t, Y^t)}{f_{y^t}(Y^t)} = {}\\
{}=
\fr{f_{\mathbf{x}_t, y^{t-1}, y_t} (\mathbf{X}_t, Y^{t-1}, Y^t)}{ f_{y^t}(Y^t)} 
={}\\
{}= \fr{f_{\mathbf{x}_t, y^{t-1}} (\mathbf{X}_t, Y^{t-1}) 
f_{y_t\vert\mathbf{x}_t} (Y_t\vert \mathbf{X}_t)}{ f_{y^t}(Y^t)}\,.
     \end{multline*}
     
\noindent
     Далее для первого множителя $f_{\mathbf{x}_t,y^{t-1}} \left( 
\mathbf{X}_t,Y^{t-1}\right)$ имеем:
     \begin{multline*}
     f_{\mathbf{x}_t,y^{t-1}} \left( \mathbf{X}_t, Y^{t-1}\right) ={}\\
     {}=
    f_{x_{t-T},\ldots , x_t, y^{t-1}} \left( X_{t-T},\ldots , X_t, Y^{t-1}\right)={}\\
     {}= \int f_{x_{t-T-1}, x_{t-T}, \ldots, x_t, y^{t-1}} \left( X_{t-T-1}, X_{t-T}, \ldots\right.\\
    \left. \ldots , X_t, Y^{t-1}\right) dX_{t-T-1}={}\\
     {}=\int f_{x_t\vert\mathbf{x}_{t-1}, y^{t-1}} \left( X_t\vert \mathbf{X}_{t-1}, Y^{t-1}\right)\times{}\\
     {}\times
     f_{\mathbf{x}_{t-1}, y^{t-1}}\left( \mathbf{X}_{t-1}, Y^{t-1}\right) dX_{t-T-1}={}\\
    {}=\int f_{x_t\vert x_{t-1}} \left( X_t\vert X_{t-1}\right)\times\\
    {}\times
     f_{\mathbf{x}_{t-1}\vert y^{t-1}}\left( \mathbf{X}_{t-1}\vert Y^{t-1}\right) dX_{t-T-1}\cdot
   f_{y^{t-1}}\left( Y^{t-1}\right)={}\\
  {}=
 \int f_{\mathbf{x}_{t-1}\vert y^{t-1}} \left( \mathbf{X}_{t-1}\vert Y^{t-1}\right) dX_{t-T-1}\times{}\\
 {}\times 
     f_{x_t\vert x_{t-1}}\left( X_t\vert X_{t-1}\right) f_{y^{t-1}} \left( Y^{t-1}\right)={}\\
     {}=
    \int f_{\mathbf{x}_{t-1}\vert y^{t-1}} \left( \mathbf{X}_{t-1}\vert Y^{t-1}\right) dX_{t-T-1}\times{}\\
    {}\times 
 f_{w_t} \left( X_t-\varphi_t\left( X_{t-1}\right)\right) f_{y^{t-1}}\left( Y^{t-1}\right).
     \end{multline*}
 
 \noindent    
     Для второго множителя $f_{y_t\vert \mathbf{x}_t} \left( Y_t\vert 
\mathbf{X}_t\right)$:
     \begin{multline*}
     f_{y_t\vert\mathbf{x}_t}\left( Y_t\vert \mathbf{X}_t\right) = 
\sum\limits_{k=0}^{(T+1) q_y} I\left( \Theta_t(\mathbf{X}_t)=[\Theta_t]_k\right) \times{}\\
{}\times
f_{y_t\vert \mathbf{x}_t} \left( Y_t\vert \mathbf{X}_t, \Theta_t(\mathbf{X}_t)=[\Theta_t]_k\right)={}\\
     {}=
     \sum\limits_{k=0}^{(T+1)q_y} I\left( \theta_t(X_t)=[\tau_t]_k\right) \times{}\\
     {}\times
f_{y_t\vert\mathbf{x}_t, \tau_t=[\tau_t]_k} \left( Y_t\vert \mathbf{X}_t, 
\tau_t=\left[\tau_t\right]_k\right)={}\\
     {}=\!\!\!\sum\limits_{k=0}^{(T+1)q_y} \!\! I\left( \theta_t(X_t) = [\tau_t]_k\right) 
f_{v_t} \left( Y_t -[\Theta_t \Psi_t(\mathbf{X}_t)]_k \right).
     \end{multline*}
     
 \noindent
     И окончательно получаем
     \begin{multline*}
     f_{\mathbf{x}_t\vert y^t} = \fr{f_{y^{t-1}}(Y^{t-1})}{f_{y^t}(Y^t)} \times{}\\
     {}\times
    \Bigg( \!\int\!\! f_{\mathbf{x}_{t-1}\vert y^{t-1}} 
dX_{t-T-1}\cdot f_{w_t} (X_t -\varphi_t(X_{t-1})) \times{}\\
{}\times \sum\limits_k 
I(\theta_t(X_t)=[\tau_t]_k) f_{v_t} \left(Y_t -[\Theta_t \Psi_t 
(\mathbf{X}_t)]_k\right)\Bigg) ={} \\
     {}=
     \Bigg(f_{w_t}(X_t-\varphi_t(X_{t-1}))\sum\limits_k 
I(\theta_t(X_t)=[\tau_t]_k)\times{}\\
{}\times
 f_{v_t}(Y_t-[\Theta_t\Psi_t(\mathbf{X}_t)]_k) \int 
f_{\mathbf{x}_{t-1}\vert y^{t-1}} dX_{t-T-1}\Bigg)\times{}
\end{multline*}

\noindent
\begin{multline*}
{}\times 
\Bigg(\int f_{w_t} (X_t -
\varphi_t(X_{t-1}))\sum\limits_k I(\theta_t(X_t) =[\tau_t]_k) \times{}\\
{}\times 
f_{v_t} (Y_t -
[\Theta_t\Psi_t(\mathbf{X}_t)]_k) \!\int\! f_{\mathbf{x}_{t-1}\vert y^{t-1}} dX_{t-T-1} d\mathbf{X}_t\Bigg)^{-1}\!.\hspace*{-4.87433pt}
     \end{multline*}
     
   \noindent
     В последнем равенстве опущены аргументы у~плотностей 
$f_{\mathbf{x}_t\vert y^t} (\mathbf{X}_t\vert Y^t)$  
и~$f_{\mathbf{x}_{t-1}\vert y^{t-1}} (\mathbf{X}_{t-1}\vert Y^{t-1})$, 
а~так\-же суммирование $\sum\nolimits_{k=0}^{(T+1)q_y}$ обозначено 
как~$\sum\nolimits_k$ и~учтено, что коэффициент $f_{y^t} (Y^t)/f_{y^{t-1}} 
(Y^{t-1})$~--- нор\-ми\-ру\-ющий множитель.
     
     Условия ограниченности вторых моментов и~линейного рос\-та функций 
сис\-те\-мы очевидным образом достаточны для существования вторых 
моментов~$x_t$ и~$y_t$: ${\sf E} \left\{ \| x_t\|^2\right\} \hm+ {\sf E} \left\{ \| y_t\|^2\right\} \hm<\infty$ 
и~\mbox{оп\-ти\-маль\-ности} оценки~${\sf E} \left\{ x_t\vert Y^t\right\}$ в~сред\-не\-квад\-ра\-тич\-ном, что завершает доказательство.


     
\section{Компьютерное моделирование}

     Отсутствие возможности даже в~упрощенном модельном эксперименте 
проводить слож\-ные вы\-чис\-ле\-ния~(\ref{e5-bos}) не означает, что для 
рас\-смат\-ри\-ва\-емой модели требуются специальные алгоритмы оценивания. 
Впол\-не возможно, что существенного влияния временн$\acute{\mbox{а}}$я за\-держ\-ка на 
качество оценивания не оказывает, так что в~практических задачах ее можно 
прос\-то проигнорировать. В~данном разделе \mbox{статьи} приводится прос\-той, но 
практически со\-сто\-ятель\-ный пример движения объекта в~водной среде 
и~наблюдения за ним и~иллюстрируется низ\-кое качество прос\-то\-го алгоритма 
фильт\-ра\-ции, не учи\-ты\-ва\-юще\-го задержки. Предполагается, что AUV 
движется на глубине в~горизонтальной плос\-кости~$Oxy$. В~каж\-дый момент 
времени к~текущей ско\-рости добавляется случайный шум, ими\-ти\-ру\-ющий 
хаотичное маневрирование. Координаты траектории AUV обозначаются, как 
принято, $x(t)$ и~$y(t)$. Обратим внимание на то, что эти обозначения 
отличны от обозначений вектора со\-сто\-яния~$x_t$ и~наблюдений~$y_t$ 
в~исходной модели~(\ref{e1-bos}). Единицей измерений положения остаются 
ки\-ло\-мет\-ры, скорости измеряются в~километрах в~час, время~--- в~часах. Модель 
непрерывного движения дискретизована с~шагом $h\hm= 0{,}0001$~ч, что 
соответствует час\-то\-те около трех измерений в~секунду. Для удобства 
графического пред\-став\-ле\-ния результатов считается, что движение 
начинается в~момент времени $t\hm=0$ и~завершается в~момент $t\hm= 
1000$, т.\,е.\ движение продолжается~6~мин. Сис\-те\-ма\-ти\-че\-ские по\-сто\-ян\-ные 
значения ско\-рости со\-став\-ляют $v_x\hm= 25$~км/ч и~$v_y\hm= 12{,}5$~км/ч. 
За время движения AUV в~сред\-нем перемещается на~3~км.
     
     Начальное положение AUV задано гауссовским вектором $(x(0), 
y(0))^\prime$, который имеет среднее $(6{,}25; 12{,}5)^\prime$ и~ковариацию 
$\mathrm{diag}\left\{ 2{,}5^2; 5^2\right\}$. Однотипные наблюдатели 
расположены в~двух точках на той же плос\-кости $Oxy$: первый имеет 
координаты $(0,l_y)$, $l_y\hm= 6{,}25$, т.\,е.\ расположен в~6,25~км от начала 
координат по оси~$Oy$. Координаты второго~--- $(l_x,0)$, $l_x\hm= 12{,}5$, 
т.\,е.\ он расположен в~12,5~км\linebreak
 от начала координат по оси~$Ox$. 
В~предположении, что для наблюдения используются акус\-ти\-че\-ские сонары и~ско\-рость звука 
в~воде рав\-на $v_s\hm= 5400$~км/ч (1500~м/с), мож\-но 
определить \mbox{максимально} воз\-мож\-ную величину за\-держ\-ки наблюдений 
$T\hm=50$,\linebreak т.\,е.\ максимальная величина, на которую могут за\-паз\-ды\-вать 
наблюдения, со\-став\-ля\-ет 18~с. Соответственно, процесс фильт\-ра\-ции и~пер\-вая 
оцен\-ка появляются в~момент~$T$, через~18~с после \mbox{начала} движения 
(мож\-но считать, после обнаружения цели). Каж\-дый наблюдатель измеряет 
даль\-ность и~на\-прав\-ля\-ющий косинус до цели с~ад\-ди\-тив\-ной ошиб\-кой. Век\-тор 
ошибок предполагается гауссовским со сред\-ним $(0,0,0,0)^\prime$ 
и~ковариацией $\mathrm{diag}\,\left\{ 0{,}001^2, 0{,}005^2, 0{,}001^2, 
0{,}005^2\right\}$. 
     
     Таким образом, имеется сле\-ду\-ющая сис\-те\-ма наблюде\-ния. Век\-тор 
со\-сто\-яния име\-ет~вид:
     \begin{equation}
     \left.
     \begin{array}{rl}
     x(t)&= x(t-1)+hv_x+w_x(t)\,;\\[6pt]
     y(t)&= y(t-1)+hv_y +w_y(t)\,.
     \end{array}
     \right\}
     \label{e6-bos}
     \end{equation}
     
     Описывающий хаотичные изменения ско\-рости вектор $\left( w_x(t), w_y(t)\right)^\prime$ 
     предполагается гауссовским со сред\-ним $(0,0)^\prime$ 
и~ковариацией $\mathrm{diag}\left\{ 0{,}01^2; 0{,}01^2\right\}$. Таким 
образом, AUV может менять ско\-рость на величину по\-ряд\-ка~100~км/ч, 
т.\,е.\ это очень быст\-ро маневрирующая цель. Но она интересна для рас\-че\-та 
тем, что дает очень разнообразные тра\-ек\-то\-рии.
{\looseness=1

}
     
     Получаемые измерения обозначаются $d_1(t)$ и~$d_2(t)$ (даль\-ности, 
измеренные пер\-вым и~вторым наблюдателем) и~$c_1(t)$ и~$c_2(t)$ 
(на\-прав\-ля\-ющие косинусы). Век\-тор ошибок измерений обозначен 
     $\left( v_{d_1}(t), v_{d_2}(t), v_{c_1}(t), v_{c_2}(t)\right)^\prime$. 
Таким образом, урав\-не\-ния наблюдений при\-ни\-ма\-ют~вид:
     \begin{equation}
     \left.
     \begin{array}{rl}
     d_1(t)&= {}\\[6pt]
& \hspace*{-10mm}    {}=\sqrt{(x(t-(\tau_t)_1))^2 +(y(t-(\tau_t)_1)-l_y)^2} +{}\\[6pt]
&\hspace*{45mm}{}+v_{d_1}(t)\,;\\[6pt]
c_1(t)&={}\\[6pt]
& \hspace*{-10mm}     {}= \fr{y(t-(\tau_t)_2)-l_y}{\sqrt{(x(t-(\tau_t)_2))^2 +(y(t-(\tau_t)_2)-l_y)^2}}+{}\\[6pt]
&\hspace*{45mm}{}+v_{c_1(t)}\,;
\end{array}
\right\}
\label{e7-1-bos}
\end{equation}

\noindent
 \begin{equation}
     \left.
     \begin{array}{rl}
          d_2(t)&= {}\\[6pt]
     & \hspace*{-10mm}{}=\sqrt{(x(t-(\tau_t)_3)-l_x)^2 + (y(t-(\tau_t)_3))^2}+{}\\[6pt]
&    \hspace*{45mm} {}+v_{d_2}(t)\,;\\[6pt]
     c_2(t) &={}\\[6pt]
     & \hspace*{-10mm}{}= \fr{x(t-(\tau_t)_4)-l_x} {\sqrt{(x(t-(\tau_t)_4)-l_x)^2 +(y(t-(\tau_t)_4))^2}}+{}\\[6pt]
     &\hspace*{45mm}{}+v_{c_2}(t)\,.
     \end{array}
     \right\}
     \label{e7-bos}
     \end{equation}
     
     С учетом выбранного шага дискретизации~$h$ и~ско\-рости звука 
в~воде~$v_s$ элементы~$\tau_t$ выражаются через даль\-ности наблюдателей 
до AUV, т.\,е.
     \begin{equation}
     \left.
     \begin{array}{rl}
     (\tau_t)_1 &= (\tau_t)_2={}\\[6pt]
     &\hspace*{-5mm}{}=\min \left\{ T, \left[\fr{\sqrt{(x(t))^2+(y(t)-l_y)^2}}{hv_s}\,\right]\right\};\\[9pt]
     (\tau_t)_3&=(\tau_t)_4= {}\\[6pt]
&     \hspace*{-5mm}{}=\min\left\{ T, \left[\fr{\sqrt{(x(t)-l_x)^2 +(y(t))^2}}{hv_s}\,\right]\right\}.
     \end{array}
     \right\}
     \label{e8-bos}
     \end{equation}
     
     В~(\ref{e8-bos}) использовано обозначение~$[x]$ для целой час\-ти~$x$ 
и~учте\-на потенциальная воз\-мож\-ность удаления объекта на дис\-тан\-цию, для 
которой величина за\-держ\-ки становится больше заданного мак\-си\-му\-ма~$T$.
     
     Для реализации <<естественного>> фильт\-ра учтен  
гео\-мет\-ри\-че\-ский смысл наблюдений, а~имен\-но: если предположить, 
что в~(\ref{e7-1-bos}) и~(\ref{e7-bos}) отсутствует шум, то мож\-но оче\-вид\-ным образом 
определить положение AUV, пе\-ре\-счи\-тав дальности и~косинусы в~декартовы 
координаты. Такое преобразование дает две независимые оцен\-ки положения 
AUV $\left( \tilde{x}_1(t),\tilde{y}_1(t)\right)$ и~$\left( \tilde{x}_2(t), 
\tilde{y}_2(t)\right)$, вы\-чис\-лен\-ные на основании измерений, выполненных 
первым и~вторым наблюдателем, без учета шумов и~временн$\acute{\mbox{о}}$й за\-держки:
     $$
     \begin{pmatrix}
     \tilde{x}_1(t)\\[3pt]
      \tilde{y}_1(t)\\[3pt]
      \tilde{x}_2(t)\\[3pt] 
      \tilde{y}_2(t)
     \end{pmatrix} = 
     \begin{pmatrix}
     \sqrt{(d_1(t))^2 -(c_1(t)d_1(t))^2}\\[3pt]
     c_2(t)d_1(t)+l_y\\[3pt]
     c_2(t) d_2(t)+l_x\\[3pt]
     \sqrt{(d_2(t))^2 -(c_2(t) d_2(t))^2}
     \end{pmatrix}.
     $$
     
     \noindent
     С этими выражениями очевидным образом могут быть использованы 
равенства~(\ref{e8-bos}) для получения оцен\-ки~$\tilde{\tau}_t$:
     \begin{align*}
     \left( \tilde{\tau}_t\right)_1 &= \left( \tilde{\tau}_t\right)_2 
=\fr{\sqrt{(\tilde{x}(t))^2+\left(\tilde{y}(t)-l_y\right)^2}}{hv_s}\,;\\
     \left( \tilde{\tau}_t\right)_3 &= \left( \tilde{\tau}_t\right)_4 
=\fr{\sqrt{\left(\tilde{x}(t)-l_x\right)^2 +\left(\tilde{y}(t)\right)^2}}{hv_s}\,.
     \end{align*}
     
     Далее в~оценках надо учесть изменение положения в~силу динамики 
движения~(\ref{e6-bos}) за время за\-держки

\vspace*{-3pt}

\noindent
     $$
     \begin{pmatrix}
     \hat{x}_1(t)\\[3pt]
     \hat{y}_1(t)\\[3pt]
     \hat{x}_2(t)\\[3pt]
      \hat{y}_2(t)
     \end{pmatrix}= \begin{pmatrix}
     \tilde{x}_1(t)+\left( \tilde{\tau}_t\right)_1 v_x\\[3pt]
     \tilde{y}_1(t)+\left( \tilde{\tau}_t\right)_2 v_y\\[3pt]
     \tilde{x}_2(t)+\left( \tilde{\tau}_t\right)_3 v_x\\[3pt]
     \tilde{y}_2(t)+\left( \tilde{\tau}_t\right)_4 v_y
     \end{pmatrix}
     $$
     
     \vspace*{-1pt}
     
\noindent
     и окончательно скомбинировать оценки двух наблюдателей, учитывая 
их одинаковую точ\-ность: 
     $$
     \hat{x}(t)=\fr{\hat{x}_1(t)+\hat{x}_2(t)}{2}\,;\enskip \hat{y}(t)= 
\fr{\hat{y}_1(t)+\hat{y}_2(t)}{2}\,.
     $$
     
          \vspace*{-1pt}
     
     В выполненном компьютерном расчете моделировался пучок из 
100\,000~траекторий сис\-те\-мы наблюдения~(\ref{e6-bos})--(\ref{e8-bos}) 
и~оценки $\left( \hat{x}(t), \hat{y}(t)\right)$. Анализировались точности 
оценивания координат положения AUV, опре\-де\-ля\-емые 
сред\-не\-квад\-ра\-тич\-ны\-ми отклонениями ошибок оценок $\sigma_{\hat{x}}(t)$ 
и~$\sigma_{\hat{y}}(t)$.


     
     На рис.~1 представлены примеры траектории AUV $\left( x(t), 
y(t)\right)^\prime$ и~со\-от\-вет\-ст\-ву\-ющих оценок $\left( \hat{x}(t), 
\hat{y}(t)\right)^\prime$. Этот рисунок дает качественное пред\-став\-ле\-ние 
о~результатах оценивания: фильтр работает, отслеживая цель, но отклонения 
от истинной траектории кажутся значительными. Под\-тверж\-да\-ет это рис.~2 
с~формальными оценками каче-\linebreak\vspace*{-12pt}



{ \begin{center}  %fig1
 \vspace*{6pt}
    \mbox{%
\epsfxsize=79mm 
\epsfbox{bos-1.eps}
}

\end{center}

\vspace*{-4pt}

\noindent
{{\figurename~1}\ \ \small{Пример траекторий: \textit{1}~--- оценка фильт\-ра\-ции $\left( \hat{x}(t), 
\hat{y}(t)\right)^\prime$; \textit{2}~--- истинная траектория AUV $\left( x(t), 
y(t)\right)^\prime$
}}}

%\vspace*{6pt}

\addtocounter{figure}{1}
     
{ \begin{center}  %fig2
 \vspace*{6pt}
     \mbox{%
\epsfxsize=79mm 
\epsfbox{bos-2.eps}
}

\end{center}


\vspace*{-4pt}

\noindent
{{\figurename~2}\ \ \small{Статистические оценки качества фильт\-ра\-ции: \textit{1}~--- $\sigma_{\hat{x}}(t)$; 
\textit{2}~--- $\sigma_{\hat{y}}(t)$
}}}

%\vspace*{6pt}

\addtocounter{figure}{1}

\pagebreak

\noindent
ства фильт\-ра\-ции~$\sigma_{\hat{x}}(t)$ 
и~$\sigma_{\hat{y}}(t)$, вы\-чис\-лен\-ны\-ми путем осред\-не\-ния по 
смоделированному \mbox{пучку}.
     
   \vspace*{-6pt}   

\section{Заключение}

     Рассмотренная модель динамической сис\-те\-мы со случайными 
задержками наблюдений пред\-став\-ля\-ет вызов с~точ\-ки зрения 
традиционной задачи фильт\-ра\-ции. Обосновывают это выполненные 
формальные преобразования модели к~традиционной марковской сис\-те\-ме 
и~полученные урав\-не\-ния оптимальной байесовской фильт\-ра\-ции. Также 
под\-тверж\-да\-ет этот тезис пред\-став\-лен\-ный эксперимент с~мо\-делью, 
близкой к~ре\-аль\-ности. К~качеству <<естественного>> фильт\-ра,  
ис\-поль\-зу\-юще\-го гео\-мет\-ри\-че\-ские свойства измерений, есть 
вопросы. Порядок (сред\-не\-квад\-ра\-тич\-ное отклонение) ошиб\-ки оценки 
координат AUV составляет~70~м для~$x(t)$ и~150~м для~$y(t)$. 
Эксперименты с~услов\-но-оп\-ти\-маль\-ным фильт\-ром  
Пугачева~\cite{22-bos, 23-bos} показывают, что в~данном примере при 
$T\hm=0$ можно добиться точ\-ности оценок каждой координаты порядка~20~м. 
Если учесть, что за максимальное время задержки $T\hm=50$ AUV 
в~сред\-нем изменяет положение примерно на~15~м в~известном  
на\-прав\-ле\-нии, то придется при\-знать, что оценка <<естественного>> 
фильт\-ра имеет неприемлемо низ\-кую точ\-ность. При этом исследованная 
оценка $\left( \hat{x}(t), \hat{y}(t)\right)^\prime$ не так плоха. Она гораздо 
лучше тривиальной оцен\-ки (безуслов\-но\-го математического ожидания) 
и~обладает устой\-чи\-востью, которой нет у~большинства известных 
субоптимальных фильт\-ров. Прямое применение  
услов\-но-оп\-ти\-маль\-но\-го фильт\-ра так\-же вызывает трудности, но 
перспективы успеха использования данной концепции к~предложенной 
модели представляются весьма хорошими. В~этом на\-прав\-ле\-нии 
предполагаются будущие ис\-сле\-до\-ва\-ния.

   \vspace*{-6pt}
     
{\small\frenchspacing
 { %\baselineskip=12pt
 %\addcontentsline{toc}{section}{References}
 \begin{thebibliography}{99}
\bibitem{1-bos}
\Au{Bar-Shalom Y., Li~X.\,R., Kirubarajan~T.} Estimation with applications to tracking and 
navigation: Theory, algorithms, and software.~--- New York, NY, USA: J.~Wiley \&~Sons, 
2001. 584~p.
\bibitem{2-bos}
Autonomous underwater vehicles: Design and practice (radar, sonar \& navigation)~/ Ed. 
F.~Ehlers.~--- London, U.K.: SciTech Publishing, 2020. 592~p.
\bibitem{3-bos}
Advances in marine vehicles, automation and robotics~//  J.~Marine 
Science Engineering. Special Issue. {\sf 
www.mdpi.\linebreak com/journal/jmse/special\_issues/advances\_in\_marine\_ vehicles\_automation\_and\_robotics}.

\bibitem{9-bos} %4
      \Au{Groen J., Beerens~S.\,P., Been~R., Doisy~Y., Noutary~E.} Adaptive port-starboard 
beamforming of triplet sonar arrays~// IEEE J. Oceanic Eng., 2005. Vol.~30. P.~348--359. doi: 
10.1109/JOE.2005.850880.
\bibitem{4-bos} %5
\Au{Luo J., Han~Y., Fan~L.} Underwater acoustic target tracking: A~review~// Sensors, 2018. 
Vol.~18. No.\,1. Art.~112. doi: 10.3390/s18010112.
\bibitem{5-bos} %6
      \Au{Ghafoor H., Noh~Y.} An overview of next-generation underwater target detection 
and tracking: An integrated underwater architecture~// IEEE Access, 2019. Vol.~7.  
P.~98841--98853. doi: 10.1109/ACCESS.2019.2929932.

\bibitem{8-bos} %7
      \Au{Wolek A., Dzikowicz~B.\,R., McMahon~J., Houston~B.\,H.} At-Sea evaluation of an 
underwater vehicle behavior for passive target tracking~// IEEE J. Oceanic Eng., 2019. Vol.~44. 
P.~514--523. doi: 10.1109/JOE.2018.2817268.
\bibitem{6-bos} %8
\Au{Su X., Ullah I., Liu~X., Choi~D.} A~review of underwater localization techniques, 
algorithms, and challenges~// J.~Sensors, 2020. Vol.~2020. Art.~6403161. doi: 
10.1155/2020/6403161.



\bibitem{10-bos} %9
\Au{Borisov A., Bosov~A., Miller~B., Miller~G.} Passive underwater target tracking: 
Conditionally minimax nonlinear filtering with bearing-Doppler observations~// Sensors, 2020. 
Vol.~20. No.\,8. Art.~2257. doi: 10.3390/s20082257.

\bibitem{7-bos} %10
\Au{Kumar M., Mondal~S.} Recent developments on target tracking problems: A~review~// 
Ocean Eng., 2021. No.\,236. Art.~109558. 20~p. doi: 10.1016/j.oceaneng. 2021.109558.

\bibitem{11-bos} %11
\Au{Miller A., Miller~B., Miller~G.} Navigation of underwater drones and integration of 
acoustic sensing with onboard inertial navigation system~// Drones, 2021. Vol.~5. No.\,3. 
Art.~83. doi: 10.3390/drones5030083.
\bibitem{12-bos}
\Au{Kalman R.\,E.} A~new approach to linear filtering and prediction problems~// J.~Basic Eng.~--- T.~ASME, 1960. Vol.~82. No.\,1. P.~35--45. 
      doi: 10.1115/1.3662552.
\bibitem{13-bos}
\Au{Bernstein I., Friedland~B.} Estimation of the state of a~nonlinear process in the presence of 
nongaussian noise and disturbances~// J. Frankl. Inst., 1966. Vol.~281. No.\,6. P.~455--480.
      doi: 10.1016/0016-0032(66)90434-0.
\bibitem{14-bos}
\Au{Julier S.\,J., Uhlmann J.\,K.} New extension of the Kalman filter to nonlinear systems~// 
Proc. SPIE, 1997. Vol.~3068. P.~182--193. doi: 10.1117/12.280797.

\bibitem{15-bos}
      \Au{Julier S.\,J., Uhlmann~J.\,K.} Unscented filtering and nonlinear estimation~// P.~IEEE, 2004. Vol.~92. No.\,3. P.~401--422. doi: 10.1109/JPROC.2003.823141.
\bibitem{16-bos}
      \Au{Arasaratnam I., Haykin~S.} Cubature Kalman filters~// IEEE T. Automat. Contr., 
2009. Vol.~54. No.\,6. P.~1254--1269. doi: 10.1109/TAC.2009.2019800.
\bibitem{17-bos}
\Au{Wang T., Zhang~L., Liu~S.} Improved robust high-degree cubature Kalman filter based on 
novel cubature formula and maximum correntropy criterion with application to surface target 
tracking~// J.~Marine Science Engineering,  2022. Vol.~10. No.\,8. Art.~1070. doi: 10.3390/jmse10081070.
\bibitem{18-bos}
\Au{Christ R.\,D., Wernli~R.\,L.} The ROV manual: A~user guide for remotely operated 
vehicles.~--- 2nd ed.~--- Oxford, U.K.: Butterworth-Heinemann, 2013. 712~p.
\bibitem{19-bos}
\Au{Li L., Li~Y., Zhang~Y., Xu~G., Zeng~J., Feng~X.} Formation control of multiple 
autonomous underwater vehicles under communication delay, packet discreteness and dropout~// 
J.~Marine Science Engineering, 2022. Vol.~10. No.\,7. Art.~920. doi: 10.3390/jmse10070920.
\bibitem{20-bos}
\Au{Zhao L., Wang~J., Yu~T., Chen~K., Su~A.} Incorporating delayed measurements in an 
improved high-degree cubature Kalman filter for the nonlinear state estimation of chemical 
processes~// ISA~T., 2019. Vol.~86. P.~122--133. doi: 10.1016/j.isatra.2018.11.004.
\bibitem{21-bos}
\Au{Bertsekas D.\,P., Shreve~S.\,E.} Stochastic optimal control: The discrete-time case.~--- New 
York, NY, USA: Academic Press, 1978. 330~p.

\columnbreak


\bibitem{22-bos}
\Au{Пугачев В.\,С., Синицын~И.\,Н.} Стохастические дифференциальные системы. Анализ 
и~фильт\-ра\-ция.~---  М.: Наука, 1990. 632~с.
\bibitem{23-bos}
\Au{Синицын И.\,Н., Корепанов~Э.\,Р.} Нормальные услов\-но-оп\-ти\-маль\-ные  
фильт\-ры Пугачёва для дифференциальных стохастических сис\-тем, линейных 
относительно со\-сто\-яния~// Информатика и~её \mbox{применения}, 2015. Т.~9. Вып.~2. С.~30--38. 
doi: 10.14357/ 19922264150204.
\end{thebibliography}

 }
 }

\end{multicols}

\vspace*{-6pt}

\hfill{\small\textit{Поступила в~редакцию 26.05.23}}

\vspace*{8pt}

%\pagebreak

%\newpage

%\vspace*{-28pt}

\hrule

\vspace*{2pt}

\hrule



\def\tit{NONLINEAR DYNAMIC SYSTEM STATE OPTIMAL FILTERING 
BY~OBSERVATIONS WITH~RANDOM DELAYS}


\def\titkol{Nonlinear dynamic system state optimal filtering 
by~observations with~random delays}


\def\aut{A.\,V.~Bosov}

\def\autkol{A.\,V.~Bosov}

\titel{\tit}{\aut}{\autkol}{\titkol}

\vspace*{-10pt}


\noindent
Federal Research Center ``Computer Science and Control'' of the Russian Academy 
of Sciences, 44-2~Vavilov Str., Moscow 119333, Russian Federation


\def\leftfootline{\small{\textbf{\thepage}
\hfill INFORMATIKA I EE PRIMENENIYA~--- INFORMATICS AND
APPLICATIONS\ \ \ 2023\ \ \ volume~17\ \ \ issue\ 3}
}%
 \def\rightfootline{\small{INFORMATIKA I EE PRIMENENIYA~---
INFORMATICS AND APPLICATIONS\ \ \ 2023\ \ \ volume~17\ \ \ issue\ 3
\hfill \textbf{\thepage}}}

\vspace*{3pt}
      
     
      
      \Abste{A~mathematical model of a nonlinear dynamic observation system with a discrete 
time which allows taking into account the dependence of the time of receiving observations on the state 
of the observed object is proposed. The model implements the assumption that the time between the 
moment when the measurement of the state is formed and the moment when the measured state is 
received by the observer depends randomly on the position of the moving object. Such an assumption 
source is the process of observation by stationary means of an autonomous underwater apparatus in 
which the time of obtaining up-to-date data depends on the unknown distance between the object and the 
observer. Unlike deterministic delays formed by the known state of the observation environment, to 
account for the dependence of time delays on the unknown state of the object of observation, it is required 
to use random functions to describe them. The main result of the study of the proposed model is the 
solution of the optimal filtering problem. For this purpose, recurrent Bayesian relations describing the 
evolution of the a~posteriori probability density are obtained. The difficulties of using a~semifinished 
filter for practical purposes are discussed. The proposed model is illustrated by a practical example of the 
task of tracking a moving underwater object based on the results of measurements performed by typical 
acoustic sensors. It is assumed that the object moves under the water in a~plane with a~known average 
speed, constantly performs chaotic maneuvers, and is observed by two independent complexes of acoustic 
sensors measuring the distances to the object and the guiding cosines. The complexity of determining the 
position of such an object is illustrated by a~simple filter using the geometric properties of the measured 
quantities and the least squares method.}
      
      \KWE{stochastic dynamic observation system; state filtering; optimal Bayesian filter; mean 
square evaluation criterion; autonomous underwater vehicle; acoustic sensor; target tracking}
      
     
      
 \DOI{10.14357/19922264230302}{CFVYJM}

\vspace*{-14pt}

 \Ack
 
 \vspace*{-2pt}
 
 
      \noindent
      The research was carried out using the infrastructure of the Shared Research Facilities ``High 
Performance Computing and Big Data'' (CKP ``Informatics'') of FRC CSC RAS (Moscow). 
  

\vspace*{9pt}

  \begin{multicols}{2}

\renewcommand{\bibname}{\protect\rmfamily References}
%\renewcommand{\bibname}{\large\protect\rm References}

{\small\frenchspacing
 {%\baselineskip=10.8pt
 \addcontentsline{toc}{section}{References}
 \begin{thebibliography}{99} 
      \bibitem{1-bos-1}
      \Aue{Bar-Shalom, Y., X.\,R.~Li, and T.~Kirubarajan.} 2004. \textit{Estimation with applications 
to tracking and navigation: Theory, algorithms and software}. New York, NY: John Wiley \&~Sons. 
548~p.
      \bibitem{2-bos-1}
      Ehlers, F., ed. 2020. \textit{Autonomous underwater vehicles: Design and practice  (radar, sonar 
\& navigation)}. London, U.K.: SciTech Publishing. 592~p.
      \bibitem{3-bos-1}
      Advances in marine vehicles, automation and robotics.  \textit{J.~Marine 
Science Engineering}. Special Issue. Available at: {\sf 
www.mdpi.com/journal/jmse/special\_issues/\linebreak advances\_in\_marine\_vehicles\_automation\_and\_robotics} 
(accessed June~27, 2023).

\bibitem{9-bos-1} %4
      \Aue{Groen, J., S.\,P.~Beerens, R.~Been, Y.~Doisy, and E.~Noutary.} 2005. Adaptive  
port-starboard beamforming of triplet sonar arrays. \textit{IEEE J. Oceanic Eng.} 30:348--359. doi: 
10.1109/JOE.2005.850880.

      \bibitem{4-bos-1} %5
      \Aue{Luo, J., Y. Han, and L.~Fan.} 2018. Underwater acoustic target tracking: A~review. 
\textit{Sensors} 18(1):112. doi: 10.3390/ s18010112.
      \bibitem{5-bos-1} %6
      \Aue{Ghafoor, H., and Y.~Noh.} 2019. An overview of next-generation underwater target 
detection and tracking: An integrated underwater architecture. \textit{IEEE Access} 7:98841--98853. doi: 
10.1109/ACCESS.2019.2929932.

\bibitem{8-bos-1} %7
      \Aue{Wolek, A., B.\,R.~Dzikowicz, J.~McMahon, and B.\,H.~Houston.} 2019. At-Sea 
evaluation of an under-water vehicle behavior for passive target tracking. \textit{IEEE J. Oceanic Eng.} 
44:514--523. doi: 10.1109/JOE.2018.2817268.
      \bibitem{6-bos-1} %8
      \Aue{Su, X., I.~Ullah, X.~Liu, and D.~Choi.} 2020. A~review of underwater localization 
techniques, algorithms, and challenges. \textit{J.~Sensors} 2020:6403161. doi: 10.1155/ 2020/6403161.
     
      
      
      \bibitem{10-bos-1} %9
      \Aue{Borisov, A., A.~Bosov, B.~Miller, and G.~Miller.} 2020. Passive underwater target 
tracking: Conditionally minimax nonlinear filtering with bearing-Doppler observations. \textit{Sensors} 
20(8):2257. doi: 10.3390/s20082257.

 \bibitem{7-bos-1} %10
      \Aue{Kumar, M., and S.~Mondal.} 2021. Recent developments on target tracking problems: 
A~review. \textit{Ocean Eng.} 236:109558. 20~p. doi: 10.1016/j.oceaneng.2021.109558.
      \bibitem{11-bos-1}
      \Aue{Miller, A., B.~Miller, and G.~Miller.} 2021. Navigation of underwater drones and 
integration of acoustic sensing with onboard inertial navigation system. \textit{Drones} 5(3):83. doi: 
10.3390/drones5030083.
      \bibitem{12-bos-1}
      \Aue{Kalman, R.\,E.} 1960. A~new approach to linear filtering and prediction problems. 
\textit{J.~Basic Eng.~--- T.~ASME} 82(1):35--45. doi: 10.1115/1.3662552.
      \bibitem{13-bos-1}
      \Aue{Bernstein, I., and B.~Friedland.} 1966. Estimation of the state of a nonlinear process in the 
presence of nongaussian noise and disturbances. \textit{J.~Frankl. Inst.} 281(6):455--480. doi: 
10.1016/0016-0032(66)90434-0.
      \bibitem{14-bos-1}
      \Aue{Julier, S.\,J., and J.\,K.~Uhlmann.} 1997. New extension of the Kalman filter to nonlinear 
systems. \textit{Proc. SPIE}  
3068:182--193. doi: 10.1117/12.280797.
      \bibitem{15-bos-1}
      \Aue{Julier, S.\,J., and J.\,K.~Uhlmann.} 2004. Unscented filtering and nonlinear estimation. 
\textit{P.~IEEE} 92(3):401--422. doi: 10.1109/JPROC.2003.823141.
      \bibitem{16-bos-1}
      \Aue{Arasaratnam, I., and S.~Haykin.} 2009. Cubature Kalman filters. \textit{IEEE T. Automat. 
Contr.} 54(6):1254--1269. doi: 10.1109/TAC.2009.2019800.
      \bibitem{17-bos-1}
      \Aue{Wang, T., L.~Zhang, and S.~Liu.} 2022. Improved robust high-degree cubature Kalman 
filter based on novel cubature formula and maximum correntropy criterion with application to surface 
target tracking. \textit{J.~Marine Science Engineering} 10(8):1070. doi: 10.3390/jmse10081070.
      \bibitem{18-bos-1}
      \Aue{Christ, R.\,D., and R.\,L.~Wernli.} 2013. \textit{The ROV manual: A~user guide for 
remotely operated vehicles}. 2nd ed. Oxford, U.K.: Butterworth-Heinemann. 712~p.
      \bibitem{19-bos-1}
      \Aue{Li, L., Y.~Li, Y.~Zhang, G.~Xu, J.~Zeng, and X.~Feng.} 2022. Formation control of 
multiple autonomous underwater vehicles under communication delay, packet discreteness and dropout. 
\textit{J.~Marine Science Engineering} 10(7):920. doi: 10.3390/jmse10070920.
      \bibitem{20-bos-1}
      \Aue{Zhao, L., J.~Wang, T.~Yu, K.~Chen, and A.~Su.} 2019. Incorporating delayed 
measurements in an im-proved high-degree cubature Kalman filter for the nonlinear state estimation of 
chemical processes. \textit{ISA~T.} 86:122--133. doi: 10.1016/j.isatra.2018.11.004.
      \bibitem{21-bos-1}
      \Aue{Bertsekas, D.\,P., and S.\,E.~Shreve.} 1978. \textit{Stochastic optimal control: The 
discrete-time case.} New York, NY: Academic Press. 330~p.
      \bibitem{22-bos-1}
      \Aue{Pugachev, V.\,S., and I.\,N.~Sinitsyn.} 1990. \textit{Sto\-kha\-sti\-che\-skie 
 dif\-fe\-ren\-tsi\-al'\-nye sis\-te\-my. Ana\-liz i~fil't\-ra\-tsiya} [Stochastic differential systems. Analysis 
and filtering]. Moscow: Nauka. 632~p.
      \bibitem{23-bos-1}
      \Aue{Sinitsyn, I.\,N., and E.\,R.~Korepanov.} 2015. Nor\-mal'\-nye uslovno-optimal-nye fil't\-ry 
Pu\-ga\-che\-va dlya dif\-fe\-ren\-tsi\-al'\-nykh sto\-kha\-sti\-che\-skikh sis\-tem, li\-ney\-nykh  
ot\-no\-si\-tel'\-no so\-sto\-yaniya [Normal Pugachev filters for state linear stochastic systems]. 
\textit{Informatika i~ee Primeneniya~--- Inform Appl.} 9(2):30--38. doi: 10.14357/19922264150204.

\end{thebibliography}

 }
 }

\end{multicols}

\vspace*{-6pt}

\hfill{\small\textit{Received May 26, 2023}} 

\vspace*{-20pt}
      
      \Contrl
      
      \noindent
      \textbf{Bosov Alexey V.} (b.\ 1969)~--- Doctor of Science in technology, principal scientist, 
Institute of Informatics Problems, Federal Research Center ``Computer Science and Control'' of the 
Russian Academy of Sciences, 44-2~Vavilov Str., Moscow 119333, Russian Federation; 
\mbox{AVBosov@ipiran.ru}
      
      


\label{end\stat}

\renewcommand{\bibname}{\protect\rm Литература} 