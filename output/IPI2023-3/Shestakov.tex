\newcommand{\correl}{{\sf corr}}
\newcommand{\covrel}{{\sf cov}}



\def\stat{shestakov}

\def\tit{МЕТОД ОЦЕНИВАНИЯ ПАРАМЕТРОВ ГАММА-ЭКСПОНЕНЦИАЛЬНОГО РАСПРЕДЕЛЕНИЯ ПО~ВЫБОРКЕ СО~СЛАБО ЗАВИСИМЫМИ КОМПОНЕНТАМИ$^*$}

\def\titkol{Метод оценивания параметров %гамма-экспоненциального 
распределения по выборке со слабо зависимыми компонентами}

\def\aut{А.\,А.~Кудрявцев$^1$, О.\,В.~Шестаков$^2$}

\def\autkol{А.\,А.~Кудрявцев, О.\,В.~Шестаков}

\titel{\tit}{\aut}{\autkol}{\titkol}

\index{Кудрявцев А.\,А.}
\index{Шестаков О.\,В.}
\index{Kudryavtsev A.\,A.}
\index{Shestakov O.\,V.}


{\renewcommand{\thefootnote}{\fnsymbol{footnote}} \footnotetext[1]
{Работа выполнена при финансовой поддержке РНФ (проект №\,22-11-00212).}}

%Исследование выполнено при 
%поддержке Междисциплинарной на\-уч\-но-обра\-зо\-ва\-тель\-ной школы Московского 
%университета <<Мозг, когнитивные сис\-те\-мы, искусственный интеллект>>.}}


\renewcommand{\thefootnote}{\arabic{footnote}}
\footnotetext[1]{Московский государственный университет 
имени М.\,В.~Ломоносова, факультет вы\-чис\-ли\-тель\-ной математики и~кибернетики; 
Московский центр фундаментальной и~при\-клад\-ной математики, 
\mbox{aakudryavtsev@cs.msu.ru}}
\footnotetext[2]{Московский государственный университет 
имени М.\,В.~Ломоносова, факультет вы\-чис\-ли\-тель\-ной математики и~кибернетики; 
Московский центр фундаментальной и~при\-клад\-ной математики; Федеральный 
исследовательский центр <<Информатика и~управ\-ле\-ние>> Российской академии наук, 
\mbox{oshestakov@cs.msu.ru}}

\vspace*{-9pt}


\Abst{Доказывается асимптотическая нормальность оценок па\-ра\-мет\-ров гамма-экс\-по\-нен\-ци\-аль\-но\-го 
распределения, полученных при помощи модифицированного метода моментов, в~случае 
слабой за\-ви\-си\-мости компонент выборки. Для оценок параметров изгиба и~масштаба 
гамма-экспоненциального распределения при фиксированных параметрах формы и~концентрации 
доказана центральная предельная теорема в~случае, когда максимальный коэффициент корреляции 
между элементами выборки стремится к~нулю. Метод доказательства основан на исследовании 
спектральной плотности выборки и~результатах теории стационарных случайных 
по\-сле\-до\-ва\-тель\-нос\-тей. Результаты статьи могут быть использованы 
для обоснования асимптотической нормальности оценок параметров ди\-гам\-ма-рас\-пре\-де\-ле\-ния, 
к~частным видам которого относятся обобщенное гамма-распределение и~обобщенное бета-распределение 
второго рода, возникающие при описании процессов, для моделирования которых используются 
распределения с~неотрицательным неограниченным носителем.}

\KW{слабая зависимость; оценивание параметров; гамма-экс\-по\-нен\-ци\-аль\-ное распределение; смешанные распределения; метод моментов; 
асимп\-то\-ти\-че\-ская нор\-маль\-ность}


\DOI{10.14357/19922264230308}{PEXTVK}
  
\vspace*{2pt}


\vskip 10pt plus 9pt minus 6pt

\thispagestyle{headings}

\begin{multicols}{2}

\label{st\stat}

\section{Введение}

%\vspace*{-2pt}

Подавляющее большинство моделей, ис\-поль\-зу\-ющих\-ся для описания случайных 
характеристик, име\-ющих непрерывные не\-от\-ри\-ца\-тель\-ные носители, оперируют 
распределениями из гамма- и~бета-клас\-сов. При этом необходимо возникает задача 
ста\-ти\-сти\-че\-ско\-го оценивания неизвестных па\-ра\-мет\-ров при\-ме\-ня\-емо\-го распределения. 
В~статье рас\-смат\-ри\-ва\-ет\-ся предложенное в~\cite{Ku2019_2} распределение, тес\-но 
связанное как с~обобщенным гам\-ма-рас\-пре\-де\-ле\-ни\-ем~\cite{KrMe1946,KrMe1948}, так 
и~с~обобщенным бе\-та-рас\-пре\-де\-ле\-ни\-ем второго рода~\cite{McDonald1984}.

\smallskip

\noindent
\textbf{Определение.}\ 
Будем говорить, что случайная величина~$\zeta$ имеет гам\-ма-экс\-по\-нен\-ци\-аль\-ное 
распределение GE$(r,\nu,s,t,\delta)$ с~па\-ра\-мет\-ра\-ми изгиба $0\hm\le r\hm<1$, формы 
$\nu\hm\neq0$, концентрации $s,t\hm>0$ и~мас\-шта\-ба $\delta\hm>0$, если ее плот\-ность при 
$x\hm>0$ задается со\-от\-но\-ше\-нием
\begin{equation}
\label{GED}
g_E(x) =
\fr{|\nu|x^{t\nu-1}}{\delta^{t\nu}\Gamma(s)\Gamma(t)}
   \,{\sf Ge}_{r,\, tr+s}\left(-\left(\fr{x}{\delta} \right)^{\nu}\right),
\end{equation}
где $E=(r,\nu,s,t,\delta)$; ${\sf Ge}_{\alpha,\, \beta} (x)$~--- гамма-экс\-по\-нен\-ци\-аль\-ная функция~\cite{KuTi2017}:

\noindent
\begin{multline}
\label{GEF}
{\sf Ge}_{\alpha, \beta} (x) = 
\sum\limits_{k=0}^{\infty}\fr{x^k}{k!}\, \Gamma(\alpha k + \beta), \\
 x\in\mathbb{R}\,, \enskip 0\le\alpha<1\,, \enskip \beta> 0\,.
\end{multline}

\vspace*{-3pt}

В работе~\cite{Ku2019_2} было показано, что рас\-пре\-де\-ле\-ние~(\ref{GED}) адекватно 
описывает байесовские модели ба\-лан\-са~\cite{Ku2018}. Это преж\-де всего вызвано 
тем, что распределение с~плот\-ностью~(\ref{GED}) может быть пред\-став\-ле\-но
 как мас\-штаб\-ная смесь двух случайных ве\-ли\-чин, име\-ющих обобщенное 
гам\-ма-рас\-пре\-де\-ле\-ние~\cite{Ku2019_2}.

Проблема оценивания па\-ра\-мет\-ров обобщенного гамма-рас\-пре\-де\-ле\-ния, его част\-ных 
видов и~смесей имеет богатую историю и~до сих пор актуальна~\cite{IrVaGoGo2020,RiBaGaGo2020,CoNg2021,SaGoBaGo2022}.

Ввиду представления плот\-ности~$g_E(x)$ в~терминах специальной гам\-ма-экс\-по\-нен\-ци\-аль\-ной функции~(\ref{GEF})
 метод максимального прав\-до\-по\-до\-бия 
пред\-став\-ля\-ет\-ся за\-труд\-ни\-тель\-ным. То же мож\-но сказать и~о~прямом методе моментов. 
По этой причине в~работе~\cite{KuShe2020} было предложено оценивать па\-ра\-мет\-ры 
гам\-ма-экс\-по\-нен\-ци\-аль\-но\-го рас\-пре\-де\-ле\-ния при помощи модифицированного метода, 
осно\-ван\-но\-го на логарифмических моментах.

\pagebreak

Кроме того, в~ряде задач приходится рас\-смат\-ри\-вать выборку из зависимых 
компонент. Такие ситуации воз\-ни\-ка\-ют, например, при анализе раз\-лич\-ных 
геофизических процессов и~анализе помех в~коммуникационных каналах.

Пусть $\{X_i\}_{i\in\mathbb{Z}}$~--- стационарная в~строгом смыс\-ле 
по\-сле\-до\-ва\-тель\-ность случайных величин, определенных на одном вероятностном 
про\-стран\-ст\-ве~$(\Omega, \mathfrak{F}, {\sf P})$, и~$\mathfrak{F}_k^m$~--- $\sigma$-ал\-геб\-ра, 
по\-рож\-ден\-ная \mbox{случай\-ны\-ми} величинами~$X_i$, $k\hm\leq i\hm\leq m$. Обозначим 
через ${\mathcal{L}}^2(\mathfrak{F}_k^m)$ со\-во\-куп\-ность всех $\mathfrak{F}_k^m$-из\-ме\-ри\-мых 
случайных величин с~конечными вторыми моментами. Определим 
максимальный коэффициент кор\-ре\-ляции:
\begin{multline*}
\label{corr}
\rho(n)=\sup\limits_{f,g}\abs{\correl(f,g)},\\
 f\in {\cal{L}}^2(\mathfrak{F}_{-\infty}^0),\enskip g\in {\cal{L}}^2(\mathfrak{F}_n^{\infty}), \enskip  n>0\,.
\end{multline*}
Условие $\rho(n)\longrightarrow0$ при $n\hm\to\infty$ означает ослабевание 
зависимости в~по\-сле\-до\-ва\-тель\-ности $\{X_i\}_{i\in\mathbb{Z}}$ между <<прош\-лым>> 
и~<<будущим>>.

В статье показывается асимп\-то\-ти\-че\-ская нор\-маль\-ность оценок па\-ра\-мет\-ров гам\-ма-экс\-по\-нен\-ци\-аль\-но\-го 
рас\-пре\-де\-ле\-ния, полученных при помощи модифицированного метода 
моментов. При этом предполагается, что компоненты вы\-бор\-ки, по которой строятся 
оцен\-ки, слабо за\-ви\-симы.


\section{Структурные характеристики распределения выборки}

В дальнейших рас\-суж\-де\-ни\-ях понадобятся сле\-ду\-ющие мо\-мент\-ные характеристики 
гам\-ма-экс\-по\-нен\-ци\-аль\-но\-го рас\-пре\-де\-ле\-ния~(\ref{GED}), полученные в~работе~\cite{KuShe2020}:
\begin{equation}
\left.
\begin{array}{rl}
\mu_1&\equiv\e\ln \zeta=\fr{\nu\ln\delta+\psi(t)-r\psi(s)}{\nu};
\\[6pt]
\mu_2&\equiv\e\ln^2 \zeta=
\fr{\left[\nu\ln\delta+\psi(t)-r\psi(s)\right]^2}{\nu^2}+{}\\
&\hspace*{30mm}{}+\fr{\psi'(t)+r^2\psi'(s)}{\nu^2}.
\end{array}
\right\}
\label{E_ln}
\end{equation}

Обозначим
\begin{multline*}
R_{k,l}(m)=\covrel\left(\ln^kX_n,\ln^lX_{n+m}\right)={}\\
{}=\e\ln^kX_n\ln^lX_{n+m}-\mu_k\mu_l,\enskip k,l=1,2.
\end{multline*}

Всюду далее предполагается, что ряд $\sum\nolimits_{n=1}^\infty\rho(n)$ сходится. В~этом 
случае существует положительно по\-лу\-опре\-де\-лен\-ная спект\-раль\-ная мат\-ри\-ца~\cite{JeWa1972, Ibragimov1975}
\begin{equation}
\label{Sigma}
\Sigma_\lambda=2\pi\begin{pmatrix}
f_{1,1}(\lambda)
&
f_{1,2}(\lambda)
\\[3pt]
f_{1,2}(\lambda)
&
f_{2,2}(\lambda)
\end{pmatrix},
\end{equation}
где
$$
f_{k,l}(\lambda)=\fr{1}{2\pi}\sum\limits_me^{-im\lambda}R_{k,l}(m),\enskip k,l=1,2\,,
$$
непрерывны и~ограничены. Ввиду чет\-ности~$R_{k,l}(m)$
\begin{multline*}
f_{k,l}(\lambda)=\fr{1}{2\pi}\sum\limits_me^{-im\lambda}R_{k,l}(m)={}\\
{}=
\fr{1}{2\pi}\sum\limits_me^{im\lambda}R_{k,l}(m)=
\fr{1}{2\pi}\sum\limits_m\cos(m\lambda)R_{k,l}(m)={}\\
{}=
\fr{1}{2\pi}\left(R_{k,l}(0)+2\sum\limits_{m=1}^\infty\cos(m\lambda)R_{k,l}(m)\right).
\end{multline*}

\section{Оценки параметров гамма-экспоненциального распределения}

Введем обозначение для выборочных логарифмических моментов случайной величины~$\zeta$:
\begin{equation*}
\label{L_m}
L_k(\mathbb{X})=\fr{1}{n}\sum\limits_{i=1}^n\ln^k X_i,
\end{equation*}
где $\mathbb{X}=(X_1,\ldots,X_n)$~--- выборка из рас\-пре\-де\-ле\-ния $\zeta\hm\sim \mathrm{GE}(r,\nu,s,t,\delta)$.

Метод построения рас\-смат\-ри\-ва\-емых оценок основан на решении сис\-те\-мы урав\-нений
$$
\mu_k=L_k(\mathbb{X}),\enskip k=1,2\,,
$$
причем предполагается, что па\-ра\-мет\-ры фор\-мы~$\nu$ и~кон\-цент\-ра\-ции~$s$ и~$t$ 
фик\-си\-ро\-ваны.

Приведем известные оценки па\-ра\-мет\-ров из\-ги\-ба~$r$ и~мас\-шта\-ба~$\delta$ гам\-ма-экс\-по\-нен\-ци\-аль\-но\-го 
рас\-пре\-де\-ле\-ния. Для этого введем в~рас\-смот\-ре\-ние ди\-гам\-ма-функ\-цию 
$ \psi(z)\hm={\Gamma'(z)}/{\Gamma(z)}$ и~функции
\begin{equation}
\left.
\begin{array}{rl}
\!\!\!\!R(l_1,l_2)&=
\sqrt{\fr{\nu^2(l_2-l_1^2)-{\psi'(t)}}{\psi'(s)}}\,;
\\[6pt]
\!\!\!\!D(l_1,l_2)&=
\exp\left\{
\vphantom{\sqrt{\fr{\nu^2(l_2\!-\!l_1^2)\!-\!{\psi'(t)}}{\psi'(s)}}^2}
l_1\!-\!
\fr{\psi(t)}{\nu}+{}\right.\\[6pt]
&\left.{}+\fr{\psi(s)}{\nu}\sqrt{\fr{\nu^2(l_2-l_1^2)-{\psi'(t)}}{\psi'(s)}}
\right\}.
\end{array}\!\!
\right\}\!
\label{R_D}
\end{equation}

В силу состоятельности выборочных логарифмических моментов для оценок па\-ра\-мет\-ров~$r$ и~$\delta$, 
полученных в~работе~\cite{KuShe2020}, справедливо сле\-ду\-ющее 
утверж\-де\-ние.

\smallskip

\noindent
\textbf{Лемма~1.}\
%\label{r_delta_est}
\textit{При фиксированных па\-ра\-мет\-рах~$\nu$, $s$ и~$t$ рас\-пре\-де\-ле\-ния 
$\mathrm{GE}(r,\nu,s,t,\delta)$ оценки}

\pagebreak

\noindent
\begin{align}
{\hat r}(\mathbb{X}) &=R(L_1(\mathbb{X}),L_2(\mathbb{X}));
\label{hat_r}\\
%\textit{и}
{\hat \delta}(\mathbb{X})&=D(L_1(\mathbb{X}),L_2(\mathbb{X}))
\label{hat_delta_r}
\end{align}
\textit{параметров~$r$ и~$\delta$ обладают свойством со\-сто\-я\-тель\-ности}.



\section{Асимптотическая нормальность оценок параметров гамма-экспоненциального 
распределения}

Утверждения об асимп\-то\-ти\-че\-ской нор\-маль\-ности оценок~(\ref{hat_r}) и~(\ref{hat_delta_r}) 
базируются на сле\-ду\-ющих леммах~\cite{Serfling2002}.

\smallskip

\noindent
\textbf{Лемма~2.}\
%\label{Serfling_Rn}
\textit{В $\mathbb{R}^n$ случайный вектор~$X_n$ сходится по рас\-пре\-де\-ле\-нию к~случайному 
век\-то\-ру~$X$, если и~только если каж\-дая линейная комбинация компонент~$X_n$ 
сходится по распределению к~такой же линейной комбинации компонент~$X$.}


\smallskip

\noindent
\textbf{Лемма~3.}\
%\label{Serfling_nd}
\textit{Предположим, что в~$\mathbb{R}^k$ при $n\hm\to\infty$}
$$
\sqrt{n}(T_{n1},\ldots,T_{nk})\Longrightarrow N\left(\mu, \Sigma\right),
$$
\textit{где $\Sigma$~--- ковариационная мат\-ри\-ца. Пусть действительная функция 
$g(t)\hm=g(t_1,\ldots,t_k)$ имеет от\-лич\-ный от нуля дифференциал в~точ\-ке $t\hm=\mu$. 
По\-ложим}
$$
d=\left(\fr{\partial g}{\partial t_1}\Big|_{t=\mu},\ldots,
\fr{\partial g}{\partial t_k}\Big|_{t=\mu}\right).
$$
\textit{Тогда $\sqrt{n}g(T_{n1},\ldots,T_{nk})\Longrightarrow N(g(\mu), d\Sigma d^{\mathrm{T}})$}.

\smallskip


Ввиду леммы~3 понадобятся производные функ\-ций~(\ref{R_D}):
\begin{align*}
\fr{\partial R}{\partial l_1}(l_1,l_2)&=
\fr{\nu^2}{2 \psi'(s) R(l_1,l_2)};
\\ %[9pt]
\fr{\partial R}{\partial l_2}(l_1,l_2)&=
-\fr{l_2 \nu^2}{\psi'(s) R(l_1,l_2)};
\\ %[9pt]
\fr{\partial D}{\partial l_1}(l_1,l_2)&=
\left(\fr{\nu\psi(s) }{2 \psi'(s) R(l_1,l_2)}+1\right)D(l_1,l_2);
\\ %[9pt]
\fr{\partial D}{\partial l_2}(l_1,l_2)&=
-\fr{\nu\psi(s)l_2D(l_1,l_2)}{\psi'(s) R(l_1,l_2)}.
%\label{d_function_d_l}
\end{align*}

Обозначим
\begin{multline}
d_{R}=\left(\fr{\partial R}{\partial l_1}(l_1,l_2)\Big|_{(l_1,l_2)=(\mu_1,\mu_2)},\right.\\
\left. \fr{\partial R}{\partial l_2}(l_1,l_2)\Big|_{(l_1,l_2)=(\mu_1,\mu_2)}\right);
\label{d_R_pm}
\end{multline}

\vspace*{-12pt}

\noindent
\begin{multline}
d_{D}=\left(\fr{\partial D}{\partial 
l_1}(l_1,l_2)\Big|_{(l_1,l_2)=(\mu_1,\mu_2)},\right.\\
\left.\fr{\partial D}{\partial l_2}(l_1,l_2)\Big|_{(l_1,l_2)=(\mu_1,\mu_2)}\right),
\label{d_D_pm}
\end{multline}

\columnbreak

\noindent
где математические ожидания $\mu_k$ определены\linebreak
 в~(\ref{E_ln}).

Сформулируем утверждения об асимптотической нор\-маль\-ности оценок~(\ref{hat_r}) и~(\ref{hat_delta_r})
 при фиксированных па\-ра\-мет\-рах формы~$\nu$ и~кон\-цент\-ра\-ции~$s$ и~$t$.

\smallskip

\noindent
\textbf{Теорема~1.}\
\textit{Предположим, что спект\-раль\-ная мат\-ри\-ца~$\Sigma_0$ в}~(\ref{Sigma}) \textit{положительно 
определена. При $n\hm\to\infty$ оцен\-ка}~(\ref{hat_r}) \textit{неизвестного па\-ра\-мет\-ра~$r$ 
асимп\-то\-ти\-чески нормальна}:
$$
\sqrt{n}\fr{{\hat r}(\mathbb{X})-r}{\sqrt{
d_{R}\Sigma_0 d_{R}^{\mathrm{T}}}}\Longrightarrow N(0,1),
$$
\textit{где мат\-ри\-ца~$\Sigma_0$ и~вектор $d_R$ задаются соотношениями}~(\ref{Sigma}) \textit{и}~(\ref{d_R_pm}).

\smallskip

\noindent
Д\,о\,к\,а\,з\,а\,т\,е\,л\,ь\,с\,т\,в\,о\,.\ \
Покажем асимптотическую нор\-маль\-ность вектора из выборочных логарифмических 
моментов $(L_1(\mathbb{X}), L_2(\mathbb{X}))$. Для этого введем в~рас\-смот\-ре\-ние 
линейную комбинацию 
$$Z_i(\alpha_1,\alpha_2)\hm=\alpha_1\ln X_i\hm+\alpha_2\ln^2 X_i,\enskip
\alpha_1^2\hm+\alpha_2^2\hm>0\,.$$
 Найдем дис\-пер\-сию нормированной суммы таких линейных 
комбинаций. \mbox{Имеем}
\begin{multline*}
\D\fr{1}{n}\sum\limits_{i=1}^nZ_i(\alpha_1,\alpha_2)=
\D(\alpha_1L_1(\mathbb{X})+\alpha_2L_2(\mathbb{X}))={}\\
{}=
(\alpha_1,\alpha_2)\Sigma'(\alpha_1,\alpha_2)^{\mathrm{T}},
\end{multline*}
где
\begin{multline*}
\label{Sigma_bis}
\Sigma' =\begin{pmatrix}
\sigma_{1,1}(n)
&
\sigma_{1,2}(n)
\\[3pt]
\sigma_{1,2}(n)
&
\sigma_{2,2}(n)
\end{pmatrix}
\equiv{}\\
{}\equiv
\begin{pmatrix}
\D L_1(\mathbb{X})
&
\covrel\left(L_1(\mathbb{X}),L_2(\mathbb{X})\right)
\\[3pt]
\covrel\left(L_1(\mathbb{X}),L_2(\mathbb{X})\right)
&
\D L_2(\mathbb{X})
\end{pmatrix}.
\end{multline*}

При $n\to\infty$~\cite{Ibragimov1975}
$$
n\sigma_{k,l}(n)
\longrightarrow
2\pi f_{k,l}(0),\enskip k,l=1,2\,.
$$

Таким образом,
$$
n\D\fr{1}{n}\sum\limits_{i=1}^nZ_i(\alpha_1,\alpha_2)\longrightarrow
(\alpha_1,\alpha_2)\Sigma_0(\alpha_1,\alpha_2)^{\mathrm{T}}.
$$

Поскольку матрица~$\Sigma_0$ положительно определена и~ряд 
$\sum\nolimits_{n=1}^\infty\rho(n)$ сходится, имеем~\cite{Ibragimov1975}
\begin{multline*}
\left(\fr{1}{n}\sum\nolimits_{i=1}^n \!Z_i(\alpha_1,\alpha_2)-
\e \fr{1}{n}\sum\nolimits_{i=1}^n \! Z_i(\alpha_1,\alpha_2)\right) \times{}\\
{}\times \left( %\sqrt
{\D \fr{1}{n}\sum\nolimits_{i=1}^n 
Z_i(\alpha_1,\alpha_2)}\right)^{-1/2}\Longrightarrow%\hspace*{-0.771pt}
%\\
%\Longrightarrow 
N(0,1).
\end{multline*}

Следовательно, по лемме~2 получаем
$$
\sqrt{n}\,\left(L_1(\mathbb{X}),L_2(\mathbb{X})\right)\Longrightarrow 
N(\mu,\Sigma_0),\enskip n\to\infty\,.
$$

Кроме того, в~условиях тео\-ре\-мы компоненты вектора~$d_{R}$, определенного в~(\ref{d_R_pm}), 
конечны и~функция~$R(l_1,l_2)$  имеет в~точ\-ке $(\mu_1,\mu_2)$ 
отличный от нуля дифференциал.

Таким образом, выполнены все условия лем\-мы~3. Сле\-до\-ва\-тельно,
$$
\sqrt{n}R(L_1(\mathbb{X}),L_2(\mathbb{X}))\Longrightarrow N\left(R(\mu_1,\mu_2), 
d_{R}\Sigma_0 d_{R}^{\mathrm{T}}\right).
$$

Теорема доказана.

\smallskip

Полностью аналогично доказывается тео\-ре\-ма об асимп\-то\-ти\-че\-ской нор\-маль\-ности оцен\-ки 
${\hat \delta}(\mathbb{X})$ па\-ра\-мет\-ра мас\-шта\-ба~$\delta$.

\smallskip

\noindent
\textbf{Теорема~2.}\
\textit{Предположим, что спект\-раль\-ная мат\-ри\-ца $\Sigma_0$ в}~(\ref{Sigma}) \textit{положительно 
определена. При $n\hm\to\infty$ оценка}~(\ref{hat_delta_r}) \textit{неизвестного па\-ра\-мет\-ра~$\delta$ асимпто\-ти\-чески нор\-мальна}:
$$
\sqrt{n}\,\fr{{\hat \delta}(\mathbb{X})-\delta}{\sqrt{d_{D}\Sigma_0 d_{D}^{\mathrm{T}}}}\Longrightarrow N(0,1),
$$
\textit{где матрица~$\Sigma_0$ и~вектор~$d_D$ задаются соотношениями}~(\ref{Sigma}) \textit{и}~(\ref{d_D_pm}).

\section{Заключение}


В работе показана асимптотическая нор\-маль\-ность по\-стро\-ен\-ных по выборке со слабо 
зависимыми компонентами оценок па\-ра\-мет\-ров изгиба\linebreak и~масштаба гам\-ма-экс\-по\-нен\-ци\-аль\-но\-го распре\-де\-ления при фиксированных 
па\-ра\-мет\-рах фор\-мы 
и~концентрации. Результаты работы автоматически переносятся на пару оценок 
па\-ра\-мет\-ров формы и~мас\-шта\-ба при фиксированных па\-ра\-мет\-рах изгиба и~концентрации. 
Предложенный метод доказательства асимп\-то\-ти\-че\-ской нор\-маль\-ности применим в~более 
широком классе задач, в~част\-ности при доказательстве цент\-раль\-ной предельной 
тео\-ре\-мы для оценок па\-ра\-мет\-ров ди\-гам\-ма-рас\-пре\-де\-ле\-ния, 
обобщающего гам\-ма-экс\-по\-нен\-ци\-аль\-ное рас\-пре\-де\-ле\-ние. Решение по\-доб\-ных задач пред\-став\-ля\-ет собой 
на\-прав\-ле\-ние дальнейших исследований авторов.

{\small\frenchspacing
 { %\baselineskip=12pt
 %\addcontentsline{toc}{section}{References}
 \begin{thebibliography}{99}
\bibitem{Ku2019_2}
\Au{Кудрявцев~А.\,А.}
О~представлении гам\-ма-экс\-по\-нен\-ци\-аль\-но\-го и~обобщенного отрицательного 
биномиаль-\linebreak
\vspace*{-12pt}

\columnbreak

\noindent
ного распределений~// Информатика и~её применения, 2019. Т.~13. Вып.~4. С.~76--80.
doi: 10.14357/ 19922264190412.

\bibitem{KrMe1946} %2
\Au{Крицкий~С.\,Н., Менкель~М.\,Ф.}
О~приемах исследования случайных колебаний реч\-но\-го стока~// Труды НИУ ГУГМС. Сер.~IV, 1946. Вып.~29. С.~3--32. 

\bibitem{KrMe1948} %3
\Au{Крицкий~С.\,Н., Менкель~М.\,Ф.}
Выбор кривых рас\-пре\-де\-ле\-ния вероятностей для расчетов реч\-но\-го стока~// Известия АН СССР. Отд. техн. наук, 1948. №\,6. С.~15--21.

\bibitem{McDonald1984} %4
\Au{McDonald~J.\,B.}
Some generalized functions for the size distribution of income~// Econometrica, 1984. Vol.~52. No.\,3. P.~647--665.

\bibitem{KuTi2017} %5
\Au{Кудрявцев~А.\,А., Титова~А.\,И.}
Гам\-ма-экс\-по\-нен\-ци\-аль\-ная функция в~байесовских моделях массового обслуживания~// Информатика и~её применения, 2017. Т.~11. Вып.~4. С.~104--108.
doi: 10.14357/ 19922264170413. 

\bibitem{Ku2018} %6
\Au{Кудрявцев~А.\,А.}
Байесовские модели баланса~// Информатика и~её применения, 2018. Т.~12. Вып.~3. С.~18--27.
doi: 10.14357/19922264180303.

\bibitem{IrVaGoGo2020} %7
\Au{Iriarte~Y.\,A., Varela~H., G$\acute{\mbox{o}}$mez~H.\,J., G$\acute{\mbox{o}}$mez~H.\,W.}
A~gamma-type distribution with applications~// Symmetry, 2020. Vol.~12. Iss.~5. Art.~870. 15~p. doi: 10.3390/ sym12050870.

\bibitem{RiBaGaGo2020} %8
\Au{Rivera~P.\,A., Barranco-Chamorro~I., Gallardo~D.\,I., G$\acute{\mbox{o}}$mez~H.\,W.}
Scale mixture of Rayleigh distribution~// Mathematics, 2020. Vol.~8. Iss.~10. Art.~1842. 22~p. doi: 10.3390/math8101842.

\bibitem{CoNg2021} %9
\Au{Combes~C., Ng~H.\,K.\,T.}
On parameter estimation for Amoroso family of distributions~// Math. Comput. Simulat., 2021. Vol.~191. P.~309--327.

\bibitem{SaGoBaGo2022} %10
\Au{Santoro~K.\,I., G$\acute{\mbox{o}}$mez~H.\,J., Barranco-Chamorro~I., G$\acute{\mbox{o}}$mez~H.\,W.}
Extended half-power exponential distribution with applications to COVID-19 data~// Mathematics, 2022. Vol.~10. Iss.~6. Art.~942. 
16~p. doi: 10.3390/ math10060942.

\bibitem{KuShe2020} %11
\Au{Кудрявцев~А.\,А., Шестаков~О.\,В.}
Метод логарифмических моментов для оценивания па\-ра\-мет\-ров гам\-ма-экс\-по\-нен\-ци\-аль\-но\-го рас\-пре\-де\-ле\-ния~// 
Информатика и~её применения, 2020. Т.~14. Вып.~3. С.~49--54. doi: 10.14357/19922264200307.

\bibitem{JeWa1972} %12
\Au{Дженкинс~Г., Ваттс~Д.} Спект\-раль\-ный анализ и~его приложения / Пер. с англ.~--- М.: Мир, 1972. 285~с. (\Au{Jenkins G.\,M., Watts D.\,G.}
 Spectral analysis and its applications.~--- San Francisco, CA, USA: Holden-Day, 1968. 552~p.)



\bibitem{Ibragimov1975} %13
\Au{Ибрагимов~И.\,А.} 
Замечание о~цент\-раль\-ной предельной тео\-ре\-ме для зависимых случайных величин~// Тео\-рия вероятностей и~её применения, 1975. Т.~20. Вып.~1. С.~134--140.

\bibitem{Serfling2002} %14
\Au{Serfling~R.\,J.}
Approximation theorems of mathematical statistics.~--- New York, NY, USA: John Wiley \& Sons, 2002. 371~p.
\end{thebibliography}

 }
 }

\end{multicols}

\vspace*{-6pt}

\hfill{\small\textit{Поступила в~редакцию 03.07.23}}

%\vspace*{8pt}

%\pagebreak

\newpage

\vspace*{-28pt}

%\hrule

%\vspace*{2pt}

%\hrule



\def\tit{A METHOD FOR ESTIMATING PARAMETERS OF~THE~GAMMA-EXPONENTIAL DISTRIBUTION FROM~A~SAMPLE~WITH~WEAKLY DEPENDENT COMPONENTS}


\def\titkol{A method for estimating parameters of~the~gamma-exponential distribution from~a~sample with weakly dependent components}


\def\aut{A.\,A.~Kudryavtsev$^{1,2}$ and~O.\,V.~Shestakov$^{1,2,3}$}

\def\autkol{A.\,A.~Kudryavtsev and~O.\,V.~Shestakov}

\titel{\tit}{\aut}{\autkol}{\titkol}

\vspace*{-10pt}

\noindent
$^{1}$Department of Mathematical Statistics, Faculty of Computational Mathematics and Cybernetics, M.\,V.~Lomo-\linebreak
$\hphantom{^1}$nosov Moscow State University,  1-52~Leninskie Gory, GSP-1, Moscow 119991, Russian Federation

\noindent
$^{2}$Moscow Center for Fundamental and Applied Mathematics, M.\,V.~Lomonosov Moscow State University,\linebreak
$\hphantom{^1}$1~Leninskie Gory, GSP-1, Moscow 119991, Russian Federation


\noindent
$^3$Federal Research Center ``Computer Science and Control'' of the Russian Academy of Sciences,  
44-2~Vavilov\linebreak
$\hphantom{^1}$Str., Moscow 119133, Russian Federation



\def\leftfootline{\small{\textbf{\thepage}
\hfill INFORMATIKA I EE PRIMENENIYA~--- INFORMATICS AND
APPLICATIONS\ \ \ 2023\ \ \ volume~17\ \ \ issue\ 3}
}%
 \def\rightfootline{\small{INFORMATIKA I EE PRIMENENIYA~---
INFORMATICS AND APPLICATIONS\ \ \ 2023\ \ \ volume~17\ \ \ issue\ 3
\hfill \textbf{\thepage}}}

\vspace*{9pt}


\Abste{The article proves the asymptotic normality of the estimators for 
the gamma-exponential distribution parameters obtained using the modified 
method of moments in the case of a weak dependence of the sample components. 
For the estimators of the bent and scale parameters of the gamma-exponential 
distribution with fixed shape and concentration parameters, the central limit 
theorem is proved in the case when the maximum correlation coefficient between 
the sample elements tends to zero. The proof is based on the study of the sample 
spectral density and the results of the theory of stationary random sequences. 
The results of the article can be used to substantiate the asymptotic normality 
of the estimators for the parameters of the digamma distribution, the particular 
types of which include the generalized gamma distribution and the generalized beta 
distribution of the second kind that arise when describing processes modeled 
with distributions having a~nonnegative unbounded support.}


\KWE{weak dependence; parameter estimation; gamma-exponential distribution; mixed distributions; method of moments; asymptotic normality}

\DOI{10.14357/19922264230308}{PEXTVK}

\vspace*{-12pt}

\Ack
\noindent
The research was supported by the Russian Science Foundation, project No.\,22-11-00212.

%This research has been supported by the Interdisciplinary Scientific 
%and Educational School of Moscow University ``Brain, Cognitive Systems, Artificial Intelligence.''

\vspace*{8pt}

  \begin{multicols}{2}

\renewcommand{\bibname}{\protect\rmfamily References}
%\renewcommand{\bibname}{\large\protect\rm References}

{\small\frenchspacing
 {%\baselineskip=10.8pt
 \addcontentsline{toc}{section}{References}
 \begin{thebibliography}{99} 

\bibitem{Ku2019_2-1}
\Aue{Kudryavtsev, A.\,A.} 2019. O~pred\-stav\-le\-nii gamma-eksponentsial'nogo i~obob\-shchen\-no\-go ot\-ri\-tsa\-tel'\-no\-go 
bi\-no\-mi\-al'\-no\-go ras\-pre\-de\-le\-niy 
[On the representation of gamma-exponential and generalized negative binomial distributions]. \textit{Informatika i~ee Primeneniya~--- Inform. Appl.} 13(4):76--80. 
doi: 10.14357/19922264190412.

\bibitem{KrMe1946-1} %2
\Aue{Kritsky, S.\,N., and M.\,F.~Menkel.}
 1946. O~pri\-emakh is\-sle\-do\-va\-niya slu\-chay\-nykh ko\-le\-ba\-niy rech\-no\-go sto\-ka 
 [Methods of investigation of random fluctuations of river flow]. \textit{Trudy NIU GUGMS. Ser.~IV} [Proceedings of GUGMS research institutions. Ser. IV] 29:3--32.

\bibitem{KrMe1948-1} %3
\Aue{Kritsky, S.\,N., and M.\,F.~Menkel.}
 1948. Vy\-bor kri\-vykh ras\-pre\-de\-le\-niya ve\-ro\-yat\-no\-stey dlya ra\-sche\-tov rech\-no\-go sto\-ka 
 [Selection of probability distribution curves for river flow calculations]. \textit{Izvestiya AN SSSR. Otd. tekhn. nauk} 
 [Herald of the Russian Academy of Sciences. Technical Sciences] 6:15--21.

\bibitem{McDonald1984-1} %4
\Aue{McDonald, J.\,B.} 1984. Some generalized functions for the size distribution of income. \textit{Econometrica} 52(3):647--665.

\bibitem{KuTi2017-1} %5
\Aue{Kudryavtsev, A.\,A., and A.\,I.~Titova.}
 2017. Gamma-eksponentsial'naya funk\-tsiya v~baye\-sov\-skikh mo\-de\-lyakh mas\-so\-vo\-go ob\-slu\-zhi\-va\-niya 
 [Gamma-exponential function in Bayesian queuing models]. \textit{Informatika i~ee Primeneniya~--- Inform. Appl.} 11(4):104--108. doi: 10.14357/ 19922264170413. 

\bibitem{Ku2018-1} %6
\Aue{Kudryavtsev, A.\,A.} 2018. Baye\-sov\-skie mo\-de\-li ba\-lan\-sa [Bayesian balance models]. \textit{Informatika i~ee Primeneniya~--- Inform. Appl.} 
12(3):18--27. doi: 10.14357/ 19922264180303.

\bibitem{IrVaGoGo2020-1} %7
\Aue{Iriarte, Y.\,A., H.~Varela, H.\,J.~G$\acute{\mbox{o}}$mez, and H.\,W.~G$\acute{\mbox{o}}$mez.}
 2020. A~gamma-type distribution with applications. \textit{Symmetry} 12(5):870. 15~p. doi: 10.3390/sym12050870.

\bibitem{RiBaGaGo2020-1} %8
\Aue{Rivera, P.\,A., I.~Barranco-Chamorro, D.\,I.~Gallardo, and H.\,W.~G$\acute{\mbox{o}}$mez.}
 2020. Scale mixture of Rayleigh distribution. \textit{Mathematics} 8(10):1842. 22~p. doi: 10.3390/ math8101842.

\bibitem{CoNg2021-1} %9
\Aue{Combes, C., and H.\,K.\,T.~Ng.} 2021. On parameter estimation for Amoroso family of distributions.
\textit{Math. Comput. Simulat.} 191:309--327. doi: 10.1016/j.matcom. 2021.07.004

\bibitem{SaGoBaGo2022-1} %10
\Aue{Santoro, K.\,I., H.\,J.~G$\acute{\mbox{o}}$mez, I.~Barranco-Chamorro, and H.\,W.~G$\acute{\mbox{o}}$mez.} 2022.
Extended half-power exponential\linebreak
\vspace*{-12pt}

\pagebreak

\noindent
distribution with applications to COVID-19 data.
\textit{Mathematics} 10(6):942. 16~p. doi:~10.3390/math10060942.

\bibitem{KuShe2020-1}
 \Aue{Kudryavtsev, A.\,A., and O.\,V.~Shestakov.} %11
 2020. Me\-tod lo\-ga\-rif\-mi\-che\-skikh mo\-men\-tov dlya otse\-ni\-va\-niya pa\-ra\-met\-rov gamma-eksponentsialnogo ras\-pre\-de\-le\-niya
[Method of logarithmic moments for estimating the gamma-exponential distribution parameters]. \textit{Informatika i~ee Primeneniya~--- Inform. Appl.} 
 14(3):49--54. doi: 10.14357/ 19922264200307.
 
 \bibitem{JeWa1972-1}
\Aue{Jenkins, G.\,M., and D.\,G.~Watts.}
 1968. \textit{Spectral analysis and its applications}. San Francisco, CA: Holden-Day. 552~p.
 
\bibitem{Ibragimov1975-1} %13
\Aue{Ibragimov, I.\,A.} 1975. A~note on the central limit theorems for dependent random variables. 
\textit{Theor. Probab. Appl.} 20(1):135--141. doi: 10.1137/1120011.

\bibitem{Serfling2002-1} %14
\Aue{Serfling, R.\,J.} 2002. 
\textit{Approximation theorems of mathematical statistics}.
New York, NY: John Wiley \& Sons. 371~p.
\end{thebibliography}

 }
 }

\end{multicols}

\vspace*{-6pt}

\hfill{\small\textit{Received July 3, 2023}} 

\vspace*{-12pt}


\Contr

\noindent
\textbf{Kudryavtsev Alexey A.}(b.\ 1978)~--- 
Candidate of Science (PhD) in physics and mathematics, associate professor, Department of Mathematical Statistics, Faculty of
 Computational Mathematics and Cybernetics, M.\,V.~Lomonosov Moscow State University, 1-52~Leninskie Gory, GSP-1, Moscow 119991, Russian Federation; 
 senior scientist, Moscow Center for Fundamental and Applied Mathematics, M.\,V.~Lomonosov Moscow State University,
1~Leninskie Gory, GSP-1, Moscow 119991, Russian Federation; \mbox{aakudryavtsev@cs.msu.ru}


\vspace*{3pt}

\noindent
\textbf{Shestakov Oleg V.} (b.\ 1976)~--- Doctor of Science in physics and mathematics, professor, Department of Mathematical Statistics, 
Faculty of Computational Mathematics and Cybernetics, M.\,V.~Lomonosov Moscow State University, 1-52~Leninskie Gory, GSP-1, Moscow 119991, Russian Federation; 
senior scientist, Institute of Informatics Problems, Federal Research Center ``Computer Science and Control'' 
of the Russian Academy of Sciences, 44-2~Vavilov Str., Moscow 119333, Russian Federation; 
leading scientist, Moscow Center for Fundamental and Applied Mathematics, M.\,V.~Lomonosov Moscow State University, 1~Leninskie Gory, GSP-1, 
Moscow 119991, Russian Federation; \mbox{oshestakov@cs.msu.su}



\label{end\stat}

\renewcommand{\bibname}{\protect\rm Литература} 