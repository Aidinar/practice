\newcommand{\me}[2]{\mathsf{E}_{ #1 }\left\{ \mathop{#2} \right\} }

\def\stat{borisov}

\def\tit{РЫНОК С МАРКОВСКОЙ СКАЧКООБРАЗНОЙ ВОЛАТИЛЬНОСТЬЮ II: АЛГОРИТМ ВЫЧИСЛЕНИЯ СПРАВЕДЛИВОЙ ЦЕНЫ ДЕРИВАТИВОВ$^*$}

\def\titkol{Рынок с~марковской скачкообразной волатильностью II: алгоритм вычисления справедливой цены деривативов}

\def\aut{А.\,В.~Борисов$^1$}

\def\autkol{А.\,В.~Борисов}

\titel{\tit}{\aut}{\autkol}{\titkol}

\index{Борисов А.\,В.}
\index{Borisov A.\,V.}


{\renewcommand{\thefootnote}{\fnsymbol{footnote}} \footnotetext[1]
{Работа выполнялась с~использованием инфраструктуры Центра коллективного пользования <<Высокопроизводительные вы\-чис\-ле\-ния и~большие данные>> 
(ЦКП <<Информатика>>) ФИЦ ИУ РАН (г.~Москва).}}


\renewcommand{\thefootnote}{\arabic{footnote}}
\footnotetext[1]{Федеральный исследовательский центр <<Информатика и~управление>> Российской академии наук;
\mbox{aborisov@frccsc.ru}}

\vspace*{-8pt}




\Abst{Вторая часть цикла посвящена численной реализации задачи моделирования справедливой цены производных финансовых инструментов (деривативов) в~модели 
неполного рынка с~мар\-ков\-ской скач\-ко\-об\-раз\-ной во\-ла\-тиль\-ностью. Концепция рыночной цены рис\-ка, рас\-про\-стра\-нен\-ная 
в~Runggaldier (2004) на класс рисковых базовых 
активов, поз\-во\-ли\-ла в~первой части цикла получить сис\-те\-му дифференциальных уравнений в~част\-ных производных, опи\-сы\-ва\-ющих временную эволюцию цены 
деривативов как функцию текущей цены базового актива и~скрытой во\-ла\-тиль\-ности~--- обобщение классического уравнения Блэ\-ка--Шоул\-за. В~отличие от 
по\-след\-не\-го, полученная сис\-те\-ма не допускает аналитического решения. Для этого в~работе предложено использовать при\-бли\-жен\-но-ана\-ли\-ти\-че\-ский метод 
дроб\-ных шагов. Временная шкала разбивается сеткой, и~иско\-мое решение ап\-прок\-си\-ми\-ру\-ет\-ся комбинацией решений классического уравнения 
теп\-ло\-про\-вод\-ности и~сис\-те\-мы обыкновенных линейных дифференциальных уравнений. Свойства полученных решений уравнений и~смоделированных с~их по\-мощью 
цен деривативов про\-ил\-люст\-ри\-ро\-ва\-ны комплексом чис\-лен\-ных экспериментов.}

\KW{марковский скачкообразный процесс; оптимальная фильт\-ра\-ция; стохастическая во\-ла\-тиль\-ность; рыночная цена риска; 
пре\-об\-ла\-да\-ющая мартингальная мера}

 \DOI{10.14357/19922264230303}{DNXJGB}
  
\vspace*{-4pt}


\vskip 10pt plus 9pt minus 6pt

\thispagestyle{headings}

\begin{multicols}{2}

\label{st\stat}

\section{Необходимые сведения и~постановка задачи}

В цикле рассматривается модель неполного финансового рынка со сто\-ха\-сти\-че\-ской во\-ла\-тиль\-ностью, со\-сто\-яще\-го из банковского вкла\-да 
с~детерминированной став\-кой~$r_t$,  $N$ базовых финансовых инструментов~$S$ и~$M$ производных \mbox{финансовых} инструментов (деривативов)~$F$, 
опре\-де\-ля\-емых платежным требованием~$H(S)$. Сто\-ха\-сти\-че\-ская во\-ла\-тиль\-ность определяется внеш\-ним скрытым марковским скачкообразным процессом (МСП) 

\noindent
$$
Z_t \triangleq \mathrm{col}\,(Z_t^1, \ldots, Z_t^L) \in \{e_1,\ldots, e_L\}
$$ 
с~мат\-рич\-но\-знач\-ной функцией
 ин\-тен\-сив\-ности переходов $\Lambda(\cdot)$ и~начальным распределением $\pi_0^Z$.
Этот процесс является единственным сильным решением линейной сто\-ха\-сти\-че\-ской дифференциальной сис\-те\-мы (СДС) с~мартингалом $M_t$ в~правой \mbox{части}:

\noindent
\begin{equation}
  dZ_t = \Lambda^{\top}(t)Z_t\,dt + dM_t,\enskip t \in (0,T], \enskip Z_0 \sim \pi_0^Z.
  \label{eq:Z_sys}
  \end{equation}

Эволюция цен базовых финансовых активов $S_t \hm\triangleq \mathrm{col}\,(S_t^1,\ldots,S_t^N)$  задается единственным силь\-ным
решением~СДС

\columnbreak

\noindent
  \begin{multline}
  dS_t = \mathrm{diag}\, S_t a(t,Z_t)\,dt + \mathrm{diag}\,S_t \sigma(t,Z_t)\,dw_t,\\
  t \in (0,T], \quad S_0 \sim \pi_0^{S},
  \label{eq:S_sys_P}
  \end{multline}
  где $w_t \triangleq \mathrm{col}\left (w_t^1,\ldots,w_t^K\right)$~--- $K$-мер\-ный стандартный винеровский процесс ($K \hm\geqslant N$), 
  а~случайные функции мгновенной процентной став\-ки~$a$ и~внут\-рен\-ней во\-ла\-тиль\-ности~$\sigma$ <<модулируются>> МСП~$Z_t$~как
  
  \noindent
  $$
  a(t,Z_t) = \sum\limits_{\ell=1}^L Z_t^{\ell} a^{\ell}(t); \enskip  \sigma(t,Z_t) = \sum\limits_{\ell=1}^L Z_t^{\ell}  \sigma^{\ell}(t)
$$
с известными наборами детерминированных функций $\{a^{\ell}(t)\}_{\ell=\overline{1,L}}$  и~$\{\sigma^{\ell}(t)\}_{\ell=\overline{1,L}}$.

Срок погашения всех деривативов совпадает с~$T$, погашение происходит в~соответствии с~пла\-теж\-ным требованием 
$$
H(S_T) \triangleq \mathrm{col}\,(H^1(S_T), \ldots, H^M(S_T)).
$$

Концепция рыночной цены риска, рас\-про\-стра\-нен\-ная в~[1] на класс рис\-ко\-вых базовых активов, поз\-во\-ли\-ла в~\cite{B_23_1_IA} доказать, что в~условиях 
без\-ар\-бит\-раж\-ности данного рынка при наличии мартингальной меры спра\-вед\-ли\-вая цена~$F^m$ $m$-го дериватива определяется фор\-мулой:

\pagebreak

\noindent
\begin{equation}
F^m = F^m \left(t, S_t, Z_t\right) = \sum\limits_{\ell=1}^L Z_t^{\ell} F^{m\ell} \left(t, S_t\right),
\label{eq:F_m_expl}
\end{equation}
а функции $\{F^{m\ell} (t, s)\}_{\ell = \overline{1,L}}$ являются решением задачи Коши для сле\-ду\-ющей 
сис\-те\-мы дифференциальных уравнений в~част\-ных про\-из\-вод\-ных:
\begin{equation}
\left.
\begin{array}{l}
\displaystyle F_t^{m\ell} = rF^{m\ell} - \sum\limits_{j=1}^L \Lambda^{\ell j}F^{mj} - {}\\[6pt]
\displaystyle {}-\sum\limits_{n=1}^N F_{s^n}^{m\ell}s^n(r-a_n^{\ell} ) - \fr{1}{2} \sum\limits_{i,j=1}^N s^is^j F_{s^i,s^j}^{m\ell}B_{ij}^{\ell},\\[9pt]
  \hspace*{19mm}\ell=\overline{1,M}, \enskip m=\overline{1,M}, \enskip t \in [0,T];\\[6pt]
F^{m\ell}(T,s) = H^m(s),
\label{eq:F_m_eq}
\end{array}
\right\}
\end{equation}
где 
$$
B^{\ell} = \left\|B_{ij}^{\ell}\right\|_{i,j=\overline{1,N}} \hm\triangleq \sigma^{\ell}\left( \sigma^{\ell} \right)^{\top}.$$
Эти сис\-те\-мы могут трактоваться как некоторые обобщения классического уравнения
Блэ\-ка--Шоул\-за. При этом цены~$F^m$ допускают сле\-ду\-ющий сто\-ха\-сти\-че\-ский диф\-фе\-рен\-циал:
  \begin{multline}
dF^{m} =
 \sum\limits_{\ell=1}^L Z^{\ell}
 \Bigg[
 r F^{m\ell} + \sum\limits_{n=1}^N F_{s^n}^{m\ell} S^n(a_n^{\ell} - r)
 \Bigg]dt + {}\\
 {}+  \sum\limits_{\ell=1}^L F^{m\ell}\,dM^{\ell} +
 \sum\limits_{\ell=1}^L Z^{\ell} \sum\limits_{n=1}^N \sum\limits_{k=1}^K F_{s^n}^{m\ell} S^n \sigma_{nk}^{\ell}\,dw^k, \\
 m=\overline{1,M}, \enskip
 {\ell}=\overline{1,L}, \enskip
 F^{m\ell}= F^{m\ell}(t,S_t).
\label{eq:F_m_obs}
\end{multline}
Задача данной статьи заключается в~разработке алгоритма моделирования цены деривативов. Сто\-ха\-сти\-че\-ский дифференциал~(\ref{eq:F_m_obs}) 
не подходит для этой цели: он необходим при решении задачи оценивания со\-сто\-яния скрытого МСП по наблюдениям~$S$ и~$F$, 
что станет предметом исследований по\-сле\-ду\-ющих час\-тей цикла. Генерация траекторий деривативов выполняется только с~по\-мощью~(\ref{eq:F_m_expl}) 
и~(\ref{eq:F_m_eq}), поэтому решение сис\-те\-мы~(\ref{eq:F_m_eq}) пред\-став\-ля\-ет\-ся актуальной задачей.

\section{Применение алгоритма дробных шагов при~решении обобщения уравнения Блэка--Шоулза}

Вообще говоря, (\ref{eq:F_m_eq}) пред\-став\-ля\-ет собой со\-во\-куп\-ность~$M$ независимых под\-сис\-тем с~$L$~уравнениями в~каж\-дой, 
причем внут\-ри каждой  под\-сис\-те\-мы уравнения не могут быть разделены. При этом все урав\-не\-ния относятся к~классу урав\-не\-ний теп\-ло\-про\-вод\-ности.

Для упрощения пред\-став\-ле\-ния алгоритма решения~(\ref{eq:F_m_eq}) методом дроб\-ных шагов сделаем сле\-ду\-ющие до\-пу\-щения.
\begin{enumerate}
\item
Матрица интенсивностей переходов~$\Lambda$,
банковский процент~$r$, мгновенные процентные став\-ки по базовым активам~$a$ и~внут\-рен\-ние волатильности~$\sigma$ не зависят от времени:
$$
\Lambda(t) \equiv \Lambda\,;\enskip r_t \equiv r\,;\enskip a^{\ell}(t) \equiv a^{\ell};\enskip \sigma^{\ell}(t) \equiv  \sigma^{\ell}.
$$
\item
Платежные требования~$H^m(s^1,\ldots,s^N)$, со\-от\-вет\-ст\-ву\-ющие отдельным деривативам, определяются одним базовым активом, т.\,е.\
 для любого $1 \hm\leqslant m \hm \leqslant M$ существует такой номер $1 \hm\leqslant n_m \hm\leqslant N$, что $H^m(s)\hm = H^{m}(s^{n_m})$.
\end{enumerate}
Сделанные предположения не слишком обременительны. Во-пер\-вых, постоянство коэффициентов требуется лишь на отрезках дискретизации по времени. 
Порядок длины этого отрезка для расчетов не превышает~1~ч (т.\,е.~0,001 при нормировании к~году), 
и~на нем указанные характеристики весьма близ\-ки к~константам. Во-вто\-рых, большинство реальных деривативов, например европейских опционов, 
строятся на отдельных базовых активах, а~не на их комбинациях.

Рассмотрим один дериватив и~по\-рож\-да\-ющий его базовый актив. 
Для прос\-то\-ты записи будем опускать за\-ви\-си\-мость от индексов: номера дериватива~$m$ и~базового актива~$n$. Используя обо\-зна\-чения
\begin{gather*}
\overline{F} \triangleq \mathrm{col}\left(F^1(t,s), \ldots, F^L(t,s)\right); \\
\overline{F}(t) \triangleq \mathrm{col}\left (F^1(t,\cdot), \ldots, F^L(t,\cdot)\right); \\
\overline{H} \triangleq \mathrm{col}\left(H(\cdot), \ldots, H(\cdot)\right);
\\
A \triangleq \mathrm{diag}\left (a^1,\ldots, a^L\right), \enskip B \triangleq \mathrm{diag}\left(B^1,\ldots, B^L\right);
\\
\overline{F}_t \triangleq \fr{\partial}{\partial t} \overline{F} (t,s); \\
\mathcal{L}_1\overline{F} \triangleq
\left[
r - (rI-A)s\fr{\partial}{\partial s} - \fr{1}{2}\, B s^2\fr{\partial^2}{\partial s^2}
\right] \overline{F}\,; \\
 \mathcal{L}_2\overline{F} \triangleq - \Lambda \overline{F},
\end{gather*}
однородную задачу Коши~(\ref{eq:F_m_eq}) мож\-но записать в~\mbox{виде}
\begin{equation*}
\overline{F}_t = \left( \mathcal{L}_1 + \mathcal{L}_2 \right)\overline{F}, \enskip
t \in [0,T), \enskip
\overline{F}(T,s) =\overline{H}(s) ,
%\label{eq:Cauchy}
\end{equation*}
и ее решение будет пред\-ста\-ви\-мо в~\mbox{виде}
\begin{equation*}
\overline{F}(t) = \mathfrak{L}(t,T)\overline{H}\,,
%\label{eq:Cauchy_sol}
\end{equation*}
где $\mathfrak{L}(t,T)$~--- линейный оператор перехода на промежутке $[t,T]$, со\-от\-вет\-ст\-ву\-ющий дифференциальному оператору $ \left( \mathcal{L}_1 
\hm+ \mathcal{L}_2 \right)$. Для данной задачи Коши выполнен прин\-цип Ада\-мара:
\begin{equation*}
\overline{F}(t) =  \mathfrak{L}(t,\tau) \overline{F}(\tau) \ \mbox{для любых} \ 0 < t \leqslant \tau \leqslant T\,,
%\label{eq: Hadamard}
\end{equation*}
который можно положить в~основу рекуррентной процедуры вы\-чис\-ле\-ния решения~$\overline{F}$ на временн$\acute{\mbox{о}}$й сетке 
$\{t_j\}_{i=\overline{1,J}}$: $t_j \hm= j h$, $h = {T}/{J}$:
\begin{multline}
\overline{F}(t_{j-1}) =  \mathfrak{L}(t_{j-1},t_j) \overline{F}(t_j), \\[2pt]
 j=\overline{1,J}\,, \enskip \overline{F}(t_J,s) = \overline{H}(s).
\label{eq:grid_L}
\end{multline}
Поиск оператора~$\mathfrak{L}(t,T)$ пред\-став\-ля\-ет\-ся не\-три\-ви\-аль\-ной задачей, и~исход\-ную задачу Коши мож\-но решать 
с~по\-мощью различных вариантов метода конечных разностей~\cite{TS_04}. В~данной работе предлагается решать ее методом дроб\-ных шагов~\cite{Ya_67}. 
Для этого рас\-смот\-рим две вспомогательные од\-но\-род\-ные задачи \mbox{Коши}:
\begin{alignat}{3}
\overline{R}_t &= \mathcal{L}_1 \overline{R}, &\quad t &\in [0,T), &\quad
\overline{R}(T,s) &=\overline{H}_R(s);
\label{eq:Cauchy_1}
\\[2pt]
\overline{Q}_t &=  \mathcal{L}_2 \overline{Q}, &\quad
t &\in [0,T), &\quad
\overline{Q}(T,s) &=\overline{H}_Q(s),
\label{eq:Cauchy_2}
\end{alignat}
решения которых так\-же могут быть выражены через со\-от\-вет\-ст\-ву\-ющие операторы перехода~$\mathfrak{L}_1$ и~$\mathfrak{L}_2$ 
(ниже они также пред\-став\-ле\-ны в~форме принципа Ада\-мара):
\begin{equation*}
\overline{R}(t) =  \mathfrak{L}_1(t,\tau) \overline{R}(\tau)\,;\enskip
\overline{Q}(t) =  \mathfrak{L}_2(t,\tau) \overline{Q}(\tau)
%\label{eq: Hadamard}
\end{equation*}
для любых $0 < t \leqslant \tau \leqslant T$.

Решение~(\ref{eq:grid_L}) $\{\overline{F}(t_j,s)\}_{j=\overline{0,J}}$ предлагается ап\-прок\-си\-ми\-ро\-вать, используя вмес\-то~$\mathfrak{L}$ композицию 
переходных операторов~$\mathfrak{L}_1$ и~$\mathfrak{L}_2$:
\begin{multline}
\widetilde{F}(t_{j-1}) =  \mathfrak{L}_2(t_{j-1},t_j) \mathfrak{L}_1(t_{j-1},t_j)
\widetilde{F}(t_j), \\[2pt]
 j=\overline{1,J}, \quad \widetilde{F}(t_J) = \overline{H}\,.
\label{eq:grid_L_app}
\end{multline}
Операторы $\mathfrak{L}_1$ и~$\mathfrak{L}_2$ могут быть легко найдены.

Все уравнения сис\-те\-мы~(\ref{eq:Cauchy_1}) независимы, и~они могут быть преобразованы к~стандартному уравнению теп\-ло\-про\-вод\-ности так же, 
как это сделано для классического урав\-не\-ния Блэ\-ка--Шоул\-за~\cite{Sh_98}. Рас\-смот\-рим одну из таких задач Коши (индекс~$\ell$ опус\-тим для 
прос\-то\-ты за\-писи)
\begin{multline}
R_t(t,s) = rR(t,s) - (r - a)sR_s(t,s) - \fr{b}{2}\,s^2R_{ss}(t,s), \\[2pt]
 0 \leqslant t < T\,,
\quad R(T,s) = H_R(s),
\label{eq:Cauchy_3}
\end{multline}
и решим ее с~по\-мощью сле\-ду\-ющей по\-сле\-до\-ва\-тель\-ности за\-мен.
\begin{enumerate}[1.]
\item
Искомая функция $R$ заменяется на~$V$:
$V(t,s) \triangleq$\linebreak $\triangleq e^{r(T-t)} R(t,s)$.
Задача Коши~(\ref{eq:Cauchy_3}) для новой функции~$V$ при\-ни\-ма\-ет~вид:
\begin{multline}
V_t(t,s) =  - (r - a)sV_s(t,s) - \fr{b}{2} \,s^2V_{ss}(t,s), \\[2pt]
 0 \leqslant t < T\,,
\quad V(T,s) = H_R(s).\\[-30pt]
\label{eq:Cauchy_4}
\end{multline}
\end{enumerate}

\columnbreak

\begin{enumerate}[1.]
\setcounter{enumi}{1}
\item
Вводится новая переменная $\tau(t) \hm= T\hm-t$ и~функция $U(\tau(t),s)\hm \triangleq V(t,s)$. Задача Коши~(\ref{eq:Cauchy_4}) для~$U$ при\-ни\-ма\-ет~вид:
\begin{multline}
U_{\tau}(\tau,s) =  (r - a)sU_s(\tau,s) + \fr{b}{2}\,s^2U_{ss}(\tau,s), \\
 0 < \tau \leqslant T\,,
\quad U(0,s) = H_R(s).
\label{eq:Cauchy_5}
\end{multline}
\item
Вводится новая переменная $x(\tau,s) \hm= \ln s\hm + (r-a-({b}/{2}))\tau$ и~функция $G(\tau,x(\tau,s)) \hm\triangleq U(\tau,s)$. 
Задача Коши~(\ref{eq:Cauchy_5}) для~$G$ при\-ни\-ма\-ет~вид:

\vspace*{-3pt}

\noindent
\begin{multline}
G_{\tau}(\tau,x) =  \fr{b}{2}\,x^2G_{xx}(\tau,x), \\
 0 < \tau \leqslant T\,,
\quad G(0,x(0,s)) = H_R(s).
\label{eq:Cauchy_5-1}
\end{multline}

\vspace*{-3pt}

\noindent
Решение~(\ref{eq:Cauchy_5-1}) из\-вестно:
\begin{equation*}
G(\tau,x) =  \int\limits_{-\infty}^{\infty} \Phi (\tau, x-y)H_R(y)\, dy\,,
%\label{eq:Cauchy_5_sol}
\end{equation*}

\vspace*{-3pt}

\noindent
где
$$
\Phi (u,x) \triangleq \fr{1}{\sqrt{2\pi b u}} \exp \left(- \fr{x^2}{2bu} \right).
$$
\end{enumerate}

\vspace*{-3pt}

\noindent
Производя обратные под\-ста\-нов\-ки, мож\-но получить окончательный покомпонентный вид переходного оператора $\mathfrak{L}_1
\hm= \mathrm{col}\,(\mathfrak{L}_1^1,\ldots,\mathfrak{L}_1^L)$ на одном временн$\acute{\mbox{о}}$м \mbox{шаге}:

\vspace*{-3pt}

\noindent
\begin{multline*}
\mathfrak{L}_1^{\ell}(t_{j-1},t_j) R^{\ell} \triangleq{}\\
{}\triangleq
e^{-rh} \int\limits_0^{\infty}
\fr{\Phi(h, \ln {s}/{y}  +
(r- a^{\ell} - {B^{\ell}}/{2} )h)}{y}\,R^{\ell}(y)\,dy\,, \\
 \ell = \overline{1,L}\,.
%\label{eq:L1_expl}
\end{multline*}

\vspace*{-3pt}

Система~(\ref{eq:Cauchy_2}) со\-сто\-ит из однородных обыкновенных линейных дифференциальных урав\-не\-ний,
оператор перехода~$\mathfrak{L}_2$ для нее определен сле\-ду\-ющим об\-ра\-зом:
\begin{equation*}
\mathfrak{L}_2(t_{j-1},t_j)Q = e^ {-h \Lambda }Q 
%\label{eq:L2_expl}
\end{equation*}
\mbox{для любых} $0 \leqslant  t  \leqslant \tau  \leqslant T$.

Окончательно рекуррентная процедура~(\ref{eq:grid_L_app}) послойного вы\-чис\-ле\-ния при\-бли\-жен\-но\-го решения~$\overline{F}$ при\-ни\-ма\-ет~вид

\vspace*{-3pt}

\noindent
\begin{multline}
\widetilde{F}(t_{j-1},s) =
e^{-h(rI+\Lambda)} \int\limits_0^{\infty}\Psi(s,y)
\widetilde{F}(t_j,y)\,dy, \\ 
j=\overline{1,J}\,, \quad \widetilde{F}(t_J,s) = \overline{H}(s)\,,
\label{eq:grid_L_app_expl}
\end{multline}


\noindent
где

\columnbreak

%\vspace*{-6pt}

\noindent
\begin{multline*}
\Psi(s,y) \triangleq
\mathrm{diag} 
\left(\fr{1}{y\sqrt{2\pi B^1 h}}\times{}\right.\\
{}\times e^{- ({\left( \ln {s}/{y}  +
(r- a^1 - {B^1}/{2} )h\right)^2})/({2B^1h})},\ldots \\
\left.\ldots,
\fr{1}{y\sqrt{2\pi B^L h}}\, e^{- ({\left( \ln {s}/{y}  +
(r- a^L - {B^L}/{2} )h\right)^2})/(2B^Lh) }
\right)\!.\hspace*{-1.81146pt}
\end{multline*}
Интеграл (\ref{eq:grid_L_app_expl}) не может быть вы\-чис\-лен аналитически и~заменяется некоторой аппроксимацией.
В~данном цик\-ле работ используется метод трапеций с~рав\-но\-мер\-ной сеткой с~шагом~$\Delta$. 
Об\-ласть интегрирования~--- положительная полуось~--- заменяется конечным отрезком $[\underline{s},\overline{s}]$, который выбирается, исходя из условия
$
{\sf E}\left\{S_T (1 - \mathbf{I}_{[\underline{s},\overline{s}]}(S_T))\right\} \hm\leqslant \Delta^2$. 
В~этом случае аппроксимация~$\widetilde{F}$ обеспечивает на каж\-дом шаге точ\-ность вы\-чис\-ле\-ния слоя $O(h^2\hm+ \Delta^2)$~\cite{Ya_67}. 
Сле\-ду\-ющий раздел содержит результаты чис\-лен\-но\-го эксперимента, ил\-люст\-ри\-ру\-юще\-го качество предложенного алгоритма 
вы\-чис\-ле\-ния справедливой цены дериватива и~его совместное поведение с~базовым ак\-тивом.

\vspace*{-2pt}

\section{Комплекс численных экспериментов}

\vspace*{-2pt}


Первый эксперимент посвящен чис\-лен\-но\-му анализу эволюции  во времени функций~$\{F^{\ell}\}$.
Па\-ра\-мет\-ры финансовой сис\-те\-мы в~данном примере искусственно выбраны так, 
чтобы на\-гляд\-но продемонстрировать разницу~$F^{\ell}$ для разных значений~$\ell$.

На отрезке времени $[0,1]$ (1~год, 250~торговых дней по~8~ч каж\-дый) моделируется поведение одного базового актива~(\ref{eq:S_sys_P})
($N\hm=1$), сто\-ха\-сти\-че\-ская во\-ла\-тиль\-ность которого пред\-став\-ля\-ет собой МСП~(\ref{eq:Z_sys}) 
с~че\-тырь\-мя воз\-мож\-ны\-ми со\-сто\-яни\-ями ($L \hm= 4$).
Став\-ка банковского вклада $r \hm= 10\%$ годовых, на рын\-ке присутствует один дериватив ($M$\hm=1)~--- 
\textit{call}-оп\-ци\-он с~ценой исполнения $K\hm=1$. Для сис\-те\-мы~(\ref{eq:S_sys_P}), (\ref{eq:Z_sys}) 
были выбраны сле\-ду\-ющие па\-ра\-метры:

\vspace*{-3pt}

\noindent
\begin{gather*}
S_0 \equiv 1\,; \ a = (0{,}8; 0{,}6; 0{,}4; 0{,}2); \ \sigma = (0{,}2; 0{,}4; 0{,}6; 0{,}8);
\\
\Lambda = \left[
\begin{array}{cccc}
-15 & 5 & 5 & 5 \\
5 & -15 & 5 & 5 \\
5 & 5 & -15 & 5 \\
5 & 5 & 5 & -15 \\
\end{array}
\right]; \enskip
\pi_0^Z = \left[
\begin{array}{c}
\fr{1}{4}  \\
\fr{1}{4}  \\
\fr{1}{4}  \\
\fr{1}{4}  \\
\end{array}
\right].
\end{gather*}

\vspace*{-3pt}

\noindent
Интегрирование СДС~(\ref{eq:S_sys_P}) выполнялось методом Эй\-ле\-ра--Ма\-ру\-ямы~\cite{KP_92}, адап\-ти\-ро\-ван\-ным к~скач\-кам~\cite{PB_10}, 
с~основ\-ным шагом $\tau \hm= 10^{-7}$,
интегрирование сис\-те\-мы уравнений в~част\-ных производных~(\ref{eq:F_m_eq})\linebreak\vspace*{-12pt}

{ \begin{center}  %fig1
 \vspace*{-5pt}
    \mbox{%
\epsfxsize=79mm 
\epsfbox{bor-1.eps}
}

\end{center}

\vspace*{-3pt}

\noindent
{{\figurename~1}\ \ \small{Зависимость $F^{\ell}(t,s)$ от цены~$s$ для некоторых значений времени~$t$:
черные кривые~--- $t\hm=0$; серые кривые~--- $t\hm=0{,}5$; \textit{1}~--- $\ell\hm=1$;
\textit{2}~--- 2; 
\textit{3}~--- 3; 
\textit{4}~--- $\ell\hm=4$
}}}

\vspace*{14pt}

\addtocounter{figure}{1}

\noindent
 выполнялось c~шагом по времени 
$h\hm = 0{,}0001$ (шаг соответствует 6~мин торгового времени) и~шагом по координате $\delta \hm= 0{,}002$ (шаг соответствует $0{,}2$~процентным 
пунк\-там начальной цены базового актива).
{\looseness=-1

}



 
На рис.~1 пред\-став\-ле\-ны сечения решений $\{F^{\ell}(t,s)\}_{\ell\hm=\overline{1,4}}$ сис\-те\-мы~(\ref{eq:F_m_eq}) 
при фиксированных\linebreak значениях времени: $t\hm=0$~--- начало года и~$t\hm=0{,}5$~--- 
полгода до погашения. Для срав\-не\-ния на рисунке пред\-став\-ле\-на и~функция платежного требования~$H(s)$, 
с~которой совпадают все функции~$F^{\ell}(t,s)$  в~момент погашения $t\hm=1$.
Действительно, при  разных значениях~$\ell$ и~фиксированных моментах~$t$ графики~$F^{\ell}(\cdot,s)$ значительно отличаются друг от друга. 
Помимо этого рисунок также демонстрирует изменение функций~$F^{\ell}(t,\cdot)$ с~течением времени.

 \setcounter{figure}{2}
\begin{figure*}[b] %fig3
  \vspace*{-6pt}
\begin{center}
   \mbox{%
\epsfxsize=161.344mm 
\epsfbox{bor-3.eps}
}
\end{center}
\vspace*{-11pt}
\Caption{Скрытый марковский скачкообразный процесс и~цены базового актива~$S_t$~(\textit{1}) и~\textit{call}-оп\-ци\-о\-нов~$F_t(K)$ 
в~за\-ви\-си\-мости от раз\-ных цен исполнения~$K$: \textit{2}~--- $K\hm=0{,}9$; \textit{3}~--- 0,95; \textit{4}~--- 1; \textit{5}~--- 1,05;
\textit{6}~--- 1,1; \textit{7}~--- 1,15; \textit{8}~--- 1,2; \textit{9}~--- 1,4; \textit{10}~--- 1,5; \textit{11}~--- $K\hm= 1{,}6$} 
\label{pic:pic3}
\end{figure*}

Второй численный эксперимент призван продемонстрировать сущ\-ность совместной генерации 
цены базового актива и~его производного финансового инструмента.
Слу\-чай\-ность цены опциона~$F_t$~--- это результат влияния двух случайных факторов: текущей цены базового актива~$S_t$ 
и~скрытого со\-сто\-яния рынка~$Z_t$~(\ref{eq:F_m_expl}):

\vspace*{4pt}

\noindent
$$
F_t = \overline{F}\left(t, S_t\right)Z_t.
$$

\vspace*{-3pt}

\noindent
Графическая иллюстрация этой комбинации пред\-став\-ле\-на на рис.~2. 
Трехмерная по\-верх\-ность пред\-став\-ля\-ет собой произведение~$\overline{F}(t, s)Z_t$,  
и~на ней\linebreak траектория базового актива <<вы\-чер\-чи\-ва\-ет>> про\-стран\-ст\-вен\-ную кривую, 
аппликата которой и~пред\-став\-ля\-ет собой цену дериватива. 
Следует отметить, что эта по\-верх\-ность не непрерывна: ее разрывы вызваны скачками МСП~$Z_t$. 
Эти разрывы  также по\-рож\-да\-ют скач\-ки дериватива~$F$.

Третий эксперимент иллюстрирует совместное поведение базового актива и~различных деривативов~--- \textit{call}-оп\-ци\-о\-нов 
с~раз\-ны\-ми ценами ис\-пол\-не-\linebreak\vspace*{-12pt}



{ \begin{center}  %fig2
 \vspace*{-4pt}
    \mbox{%
\epsfxsize=78.561mm 
\epsfbox{bor-2.eps}
}

\end{center}



\noindent
{{\figurename~2}\ \ \small{Зависимость $F^{\ell}(t,s)$ от цены~$s$ для некоторых значений времени~$t$
}}}

\vspace*{6pt}

%\addtocounter{figure}{1}

 
\setcounter{figure}{3}
 \begin{figure*} %fig4
  \vspace*{1pt}
\begin{center}
   \mbox{%
\epsfxsize=162.931mm 
\epsfbox{bor-4.eps}
}
\end{center}
\vspace*{-12pt}
\Caption{Скрытый марковский скачкообразный процесс и~цены базового актива~$S_t$~(\textit{1}) и~\textit{call}-оп\-ци\-о\-нов~$F_t(K)$ 
в~за\-ви\-си\-мости от разных значений цены исполнения~$K$: \textit{2}~--- $K\hm=0{,}9$; \textit{3}~--- 0,95; \textit{4}~--- 1; \textit{5}~--- 1,05;
\textit{6}~--- 1,1; \textit{7}~--- 1,15; \textit{8}~--- 1,2; \textit{9}~--- 1,4; \textit{10}~--- 1,5; \textit{11}~--- $K\hm= 1{,}6$. Увеличенный фрагмент со скач\-ком} 
\label{pic:pic4}
\vspace*{-3pt}
\end{figure*}





\noindent
ния~$K$. Колебания цен деривативов, представлен\-ные на рис.~\ref{pic:pic3}, похожи на 
колебания котировок базового актива, однако не прямо пропорциональны им. Особенно на\-гляд\-но это вид\-но в~случае с~опционами, 
цена исполнения которых выше текущей цены базового актива (так на\-зы\-ва\-емые опционы \textit{out of the money}).

Формула~(\ref{eq:F_m_expl}), определяющая цену деривативов, подразумевает также наличие ее скачков, 
совпадающих по времени со скачками внеш\-не\-го МСП~$Z_t$. 
На рис.~\ref{pic:pic4} пред\-став\-лен увеличенный фрагмент рис.~\ref{pic:pic3}: на нем на интервале $[0{,}22; 0{,}225]$ виден скачок МСП 
и~по\-рож\-ден\-ные им скачки цен деривативов. Примечательно, что не для всех цен исполнения скач\-ки имеют явно выраженный характер. 
Причину этому легко понять, анализируя рис.~1. Дело в~том, что амплитуда скач\-ка зависит от того, насколько отличаются значения 
функций~$F^{\ell}(\cdot,\cdot)$ для различных~$\ell$. Из рис.~4 можно сделать вывод, что это различие 
незначительно при достаточно малых значениях аргумента~$s$ 
и~исче\-за\-юще мало при $s \hm\to +\infty$. Таким образом, различия между~$F^{\ell}(\cdot,\cdot)$, 
а~значит, и~величины скачков цены дериватива 
существенны, когда цена базового актива близ\-ка к~цене исполнения (опционы с~такими страйками называются \textit{at the money}).

\vspace*{-6pt}

\section{Промежуточные выводы}

\vspace*{-3pt}

Вторая часть цикла статей пред\-став\-ля\-ет алгоритм вы\-чис\-ле\-ния справедливой цены  производных финансовых инструментов в~моделях рынка с~марковской 
скачкообразной во\-ла\-тиль\-ностью. Во-пер\-вых, данный алгоритм необходим при решении широкого класса задач анализа цен деривативов в~предложенной 
модели финансовой сис\-те\-мы. Во-вто\-рых, он становится ключевым при совместном чис\-лен\-ном моделировании эволюции цен базовых финансовых инструментов и~их 
деривативов. Эта воз\-мож\-ность нуж\-на для проведения сравнительного чис\-лен\-но\-го анализа решения задачи мониторинга рыночной цены рис\-ка как 
част\-но\-го случая задачи фильт\-ра\-ции МСП по име\-ющим\-ся наблюдениям цен бумаг. 
В~по\-сле\-ду\-ющих час\-тях цик\-ла будут пред\-став\-ле\-ны алгоритмы решения этой задачи оценивания в~за\-ви\-си\-мости от структуры име\-ющих\-ся наблюдений: 
до\-ступ\-ны\-ми будут цены инструментов, дискретизованные по времени с~известным детерминированным шагом~\cite{B_20_2_IA, B_20_3_IA},
 либо вы\-со\-ко\-час\-тот\-ные потоки наблюдений цен в~случайные моменты времени~\cite{B_20_3_ARC}.

\vspace*{-12pt}


{\small\frenchspacing
 {\baselineskip=10.6pt
 %\addcontentsline{toc}{section}{References}
 \begin{thebibliography}{99}
   

\bibitem{R_04}
\Au{Runggaldier W.} Estimation via stochastic filtering in financial market models~// Contemp. Math., 2004. Vol.~351.  P.~309--318.
doi: 10.1090/conm/351/06412.

\bibitem{B_23_1_IA}
\Au{Борисов~A.} 
Рынок с~марковской скачкообразной во\-ла\-тиль\-ностью I: мониторинг цены риска как задача оптимальной 
фильт\-ра\-ции~// Информатика и~её применения, 2023. Т.~17. Вып.~2. 
С.~27--33. doi:  10.14357/ 19922264230204. EDN: GAXCHQ.

   \bibitem{TS_04}
\Au{Тихонов А., Самарский~А.} Уравнения математической физики.~--- М.: Наука, 2004. 798~с.

   \bibitem{Ya_67}
\Au{Яненко Н.} Метод дроб\-ных шагов решения многомерных задач математической физики.~--- Новосибирск: Наука, 1967. 197~с.

\bibitem{Sh_98} %5
\Au{Shiryayev A.} Essentials of stochastic finance: Facts, models, theory.~--- Hackensack, NJ, USA: World Scientific, 1999. 834~p.

   \bibitem{KP_92}
\Au{Kloeden P., Platen~E.} Numerical solution of stochastic differential equations.~--- Berlin: Springer, 1992. 636~p.

   \bibitem{PB_10}
\Au{Platen E., Bruti-Liberati~N.} Numerical solution of stochastic differential equations with jumps in finance.~--- Berlin: Springer, 2010. 884~p.

    \bibitem{B_20_2_IA}
\Au{Борисов A.} Численные схемы фильт\-ра\-ции марковских скачкообразных процессов по дискретизованным наблюдениям II: случай ад\-ди\-тив\-ных шумов~//
 Информатика и~её применения, 2020. Т.~14. Вып.~1. С.~17--23. doi: 10.14357/19922264200103.

  \bibitem{B_20_3_IA}
\Au{Борисов A.} Чис\-лен\-ные схемы фильт\-ра\-ции марковских скачкообразных процессов 
по дискретизованным наблюдениям III: случай муль\-ти\-пли\-ка\-тив\-ных шумов~//
 Информатика и~её применения, 2020. Т.~14. Вып.~2. С.~10--18. doi: 10.14357/19922264200202.

\bibitem{B_20_3_ARC}
\Au{Борисов A.} Алгоритм робастной фильт\-ра\-ции марковских скачкообразных процессов по вы\-со\-ко\-час\-тот\-ным счи\-та\-ющим наблюдениям~// 
Автоматика и~телемеханика, 2020. Вып.~4. С.~3--20.
\end{thebibliography}

 }
 }

\end{multicols}

\vspace*{-10pt}

\hfill{\small\textit{Поступила в~редакцию 29.12.22}}

%\vspace*{8pt}

%\pagebreak

\newpage

\vspace*{-28pt}

%\hrule

%\vspace*{2pt}

%\hrule



\def\tit{MARKET WITH MARKOV JUMP VOLATILITY II:\\ ALGORITHM OF~DERIVATIVE FAIR PRICE CALCULATION}


\def\titkol{Market with Markov jump volatility II: Algorithm of derivative fair price calculation}


\def\aut{A.\,V.~Borisov}

\def\autkol{A.\,V.~Borisov}

\titel{\tit}{\aut}{\autkol}{\titkol}

\vspace*{-10pt}


\noindent
Federal Research Center ``Computer Science and Control'' of the Russian Academy 
of Sciences, 44-2~Vavilov Str., Moscow 119333, Russian Federation


\def\leftfootline{\small{\textbf{\thepage}
\hfill INFORMATIKA I EE PRIMENENIYA~--- INFORMATICS AND
APPLICATIONS\ \ \ 2023\ \ \ volume~17\ \ \ issue\ 3}
}%
 \def\rightfootline{\small{INFORMATIKA I EE PRIMENENIYA~---
INFORMATICS AND APPLICATIONS\ \ \ 2023\ \ \ volume~17\ \ \ issue\ 3
\hfill \textbf{\thepage}}}

\vspace*{3pt}



\Abste{The second part of the series is devoted to the numerical realization of the fair derivative price in the case 
of the incomplete financial market
with a~Markov jump stochastic volatility. The concept of the market price
of risk applied to this model by Runggaldier (2004) 
allows derivation of a~system of partial differential equations describing
the temporal evolution of the derivative price as 
a~function of the current underlying price and the implied stochastic volatility. This system represents a~generalization of the classic 
Black--Sholes equation. By contrast with this classic version, the proposed system of equations does not permit an analytical solution. 
The paper presents an approximate analytical method of the fractional steps. The author equips the temporal axis with a~grid, 
then the author approximates the required solution as a~combination of the solution to the classical heat equation and the system of 
the ordinary linear differential equations. The paper contains the results of the numerical experiments illustrating the properties of 
both the Black--Sholes generalization and the joint evolution of the derivative and corresponding underlying prices.}

\KWE{Markov jump process; optimal filtering; stochastic volatility; market price of risk; prevailing martingale measure}


 \DOI{10.14357/19922264230303}{DNXJGB}

\vspace*{-20pt}

\Ack

\vspace*{-3pt}

\noindent
The research was carried out using the infrastructure of the Shared Research Facilities 
``High Performance Computing and Big Data'' (CKP ``Informatics'') of FRC CSC RAS (Moscow).
  

%\vspace*{6pt}

  \begin{multicols}{2}

\renewcommand{\bibname}{\protect\rmfamily References}
%\renewcommand{\bibname}{\large\protect\rm References}

{\small\frenchspacing
 {%\baselineskip=10.8pt
 \addcontentsline{toc}{section}{References}
 \begin{thebibliography}{99} 

\bibitem{R_04-1} 
\Aue{Runggaldier, W.} 2004. Estimation via stochastic filtering in financial market models. \textit{Contemp. Math.} 351:309--318.
doi: 10.1090/conm/351/06412.

\bibitem{B_23_1_IA-1} 
\Aue{Borisov, A.} 2023. Ry\-nok s~mar\-kov\-skoy skach\-ko\-ob\-raz\-noy vo\-la\-til'\-nost'yu I: mo\-ni\-to\-ring tse\-ny ris\-ka kak za\-da\-cha 
op\-ti\-mal'\-noy fil't\-ra\-tsii 
[Market with Markov jump volatility I: Price of risk monitoring as an optimal filtering problem]. \textit{Informatika i~ee Primeneniya~--- Inform. Appl.} 17(2):~27--33. 
doi:   10.14357/19922264230204. EDN: GAXCHQ.



\bibitem{TS_04-1} 
\Aue{Tikhonov, A., and A.~Samarskiy.} 2004.  \textit{Urav\-ne\-niya ma\-te\-ma\-ti\-che\-skoy fi\-zi\-ki} 
[Equations of mathematical physics]. Moscow: Nauka. 798~p.

\bibitem{Ya_67-1} 
\Aue{Yanenko, N.} 1967. \textit{Me\-tod drob\-nykh sha\-gov re\-she\-niya mno\-go\-mer\-nykh za\-dach ma\-te\-ma\-ti\-che\-skoy fi\-zi\-ki}
 [The fractional steps method for solving multidimensional problems of mathematical physics]. Novosibirsk: Nauka. 197~p.

\bibitem{Sh_98-1}
\Aue{Shiryayev, A.} 1999. \textit{Essentials of stochastic finance: Facts, models, theory}. Hackensack, NJ: World Scientific. 834~p.

\bibitem{KP_92-1} 
\Aue{Kloeden, P., and E.~Platen.} 1992. \textit{Numerical solution of stochastic differential equations}. Berlin: Springer. 636~p.

\bibitem{PB_10-1} 
\Aue{Platen, E., and N.~Bruti-Liberati.} 2010. \textit{Numerical solution of stochastic differential equations with jumps in finance}.
 Berlin: Springer. 884~p.

\bibitem{B_20_2_IA-1} 
\Aue{Borisov, A.} 2020. Chis\-len\-nye skhe\-my fil't\-ra\-tsii mar\-kov\-skikh skach\-ko\-ob\-raz\-nykh pro\-tses\-sov po 
dis\-kre\-ti\-zo\-van\-nym nab\-lyu\-de\-niyam II: slu\-chay ad\-di\-tiv\-nykh shu\-mov
 [Numerical schemes of Markov jump process filtering given discretized observations II: Additive noises case]. 
 \textit{Informatika i~ee primeneniya~---  Inform. Appl.} 14(1):~17--23. doi: 10.14357/19922264200103.


\bibitem{B_20_3_IA-1} 
\Aue{Borisov, A.} 2020. Chis\-len\-nye skhe\-my fil't\-ra\-tsii mar\-kov\-skikh skach\-ko\-ob\-raz\-nykh pro\-tses\-sov 
po dis\-kre\-ti\-zo\-van\-nym nablyu\-de\-ni\-yam III: 
slu\-chay mul'\-ti\-pli\-ka\-tiv\-nykh shu\-mov [Numerical schemes of Markov jump process filtering given discretized 
observations III: Multiplicative noises case]. 
\textit{Informatika i~ee primeneniya~--- Inform. Appl.} 14(2):~10--18. doi: 10.14357/19922264200202.

\bibitem{B_20_3_ARC-1} 
\Aue{Borisov, A.} 2020. Robust filtering algorithm for Markov jump processes with high-frequency counting observations. \textit{Automat. Rem. Contr.} 81(4):575--588. 
doi: 10.1134/ S0005117920040013.
\end{thebibliography}

 }
 }

\end{multicols}

\vspace*{-6pt}

\hfill{\small\textit{Received December 29, 2022}} 

\vspace*{-20pt}

\Contrl

\vspace*{-4pt}

\noindent
\textbf{Borisov Andrey V.} (b.\ 1965)~--- 
Doctor of Science in physics and mathematics, principal scientist, Federal Research Center ``Computer Science and Control''
 of the Russian Academy of Sciences, 44-2~Vavilov Str., Moscow 119333, Russian Federation; \mbox{aborisov@frccsc.ru}




\label{end\stat}

\renewcommand{\bibname}{\protect\rm Литература} 