\def\stat{dukovi}

\def\tit{ЛОГИЧЕСКИЕ МЕТОДЫ КОРРЕКТНОЙ КЛАССИФИКАЦИИ ДАННЫХ}

\def\titkol{Логические методы корректной классификации данных}

\def\aut{Е.\,В.~Дюкова$^1$, Г.\,О.~Масляков$^2$, А.\,П.~Дюкова$^3$}

\def\autkol{Е.\,В.~Дюкова, Г.\,О.~Масляков, А.\,П.~Дюкова}

\titel{\tit}{\aut}{\autkol}{\titkol}

\index{Дюкова Е.\,В.}
\index{Масляков Г.\,О.}
\index{Дюкова А.\,П.}
\index{Djukova E.\,V.}
\index{Masliakov G.\,O.}
\index{Djukova A.\,P.}


%{\renewcommand{\thefootnote}{\fnsymbol{footnote}} \footnotetext[1]
%{Работа выполнена с~использованием инфраструктуры Центра коллективного пользования <<Высокопроизводительные вы\-чис\-ле\-ния и~большие данные>> 
%(ЦКП <<Информатика>>) ФИЦ ИУ РАН (г.~Москва).}}


\renewcommand{\thefootnote}{\arabic{footnote}}
\footnotetext[1]{Федеральный исследовательский центр <<Информатика и~управление>> Российской академии наук, \mbox{edjukova@mail.ru}}
\footnotetext[2]{Федеральный исследовательский центр <<Информатика и~управление>> Российской академии наук, \mbox{gleb-mas@mail.ru}}
\footnotetext[3]{Федеральный исследовательский центр <<Информатика и~управление>> Российской академии наук, 
\mbox{anastasia.d.95@gmail.com}}


\vspace*{-12pt}

 

\Abst{Работа посвящена вопросам применения дискретного аппарата (логических методов 
анализа це\-ло\-чис\-лен\-ных данных) для задачи классификации по прецедентам. 
Рас\-смат\-ри\-ва\-ют\-ся три  на\-прав\-ле\-ния логической классификации: Correct Voting 
Procedures (CVP), Logical Analysis of Data (LAD) и~Formal Concept Analysis (FCA). С~использованием 
терминологии на\-прав\-ле\-ния CVP приводятся основные понятия, 
ис\-поль\-зу\-емые в~LAD и~FCA. Описывается общая схема 
работы логического классификатора, со\-глас\-но которой каж\-дый логический классификатор 
на этапе обуче\-ния задает некоторый час\-тич\-ный порядок на специальном множестве 
фрагментов описаний прецедентов и~ищет максимальные относительно заданного порядка 
элементы этого множества. Подобные исследования важ\-ны для создания общей тео\-рии 
корректной классификации по прецедентам на осно\-ве применения дискретного аппарата.}
 
\KW{классификация на основе прецедентов; логический классификатор; процедуры 
корректного голосования; логический анализ данных; анализ формальных понятий; 
тупиковый пред\-ста\-ви\-тель\-ный элементарный классификатор; силь\-ная логическая 
за\-ко\-но\-мер\-ность; ДСМ-ги\-по\-те\-за; час\-тич\-ный порядок}

\DOI{10.14357/19922264230309}{OZHXOX}
  
\vspace*{-2pt}


\vskip 10pt plus 9pt minus 6pt

\thispagestyle{headings}

\begin{multicols}{2}

\label{st\stat}

\section{Введение}

%\vspace*{-4pt}

  Задача классификации на основе прецедентов рас\-смат\-ри\-ва\-ет\-ся в~сле\-ду\-ющей 
по\-ста\-новке.
  
  Исследуется множество объектов~$M$. Каждый объект из~$M$ может быть 
пред\-став\-лен в~виде чис\-ло\-во\-го набора, полученного на основе наблюдения или 
измерения ряда его характеристик. Такие характеристики называют 
при\-зна\-ка\-ми. Предполагается, что каж\-дый при\-знак имеет ограниченное 
множество до\-пус\-ти\-мых значений, которые кодируются целыми чис\-ла\-ми. 
Известно, что~$M$ пред\-ста\-ви\-мо в~виде объединения не\-пе\-ре\-се\-ка\-ющих\-ся 
подмножеств, на\-зы\-ва\-емых классами. Имеется конечный набор объектов 
множества~$M$, про каж\-дый из которых известно, какому классу он 
принадлежит. Это прецеденты, или обуча\-ющие объекты. Требуется по 
предъявленному набору значений признаков, опи\-сы\-ва\-юще\-му некоторый объект 
из~$M$, о~котором, вообще говоря, неизвестно, какому классу он 
принадлежит, определить (рас\-по\-знать) этот класс. 
  
  Логический, или дискретный, подход к~задаче классификации возник в~связи 
с~не\-об\-хо\-ди\-мостью прогнозировать ред\-кие события, для которых нет 
достаточного статистического материала. Обучение классификатора сводится 
к~поиску в~исходных данных определенных закономерностей. Найденные 
закономерности поз\-во\-ля\-ют различать объекты из разных классов и,~как 
правило, имеют содержательное описание в~терминах той при\-клад\-ной  
об\-ласти, в~которой решается задача. По их наличию или, наоборот, 
отсутствию в~описании рас\-по\-зна\-ва\-емо\-го объекта решается вопрос о~его 
классификации. При этом большое внимание уделяется корректному обуче\-нию, 
поз\-во\-ля\-юще\-му безошибочно классифицировать материал обуче\-ния. Особенно 
эффективен логический подход в~случае це\-ло\-чис\-лен\-ной информации низ\-кой 
знач\-ности, например бинарной. Однако в~случае большого чис\-ла признаков при 
поиске закономерностей в~описаниях прецедентов необходимо решать слож\-ные 
дис\-крет\-ные задачи. 
  
  Фундаментальную роль в~создании отечественных методов логической 
классификации сыгра\-ли работы чле\-на-кор\-рес\-пон\-ден\-та РАН С.\,В.~Яблонского, 
в~которых введено хорошо известное в~дискретной математике понятие тес\-та, 
и~работы академика РАН Ю.\,И.~Журавлёва, опубликованные в~1970-х и~1980-х~гг. Понятие тес\-та (введено в~[1]), первоначально 
при\-ме\-няв\-ше\-еся в~задачах конт\-ро\-ля управ\-ля\-ющих сис\-тем, стало основным для 
конструирования одной из первых моделей логических классификаторов 
на\-прав\-ле\-ния, име\-ну\-емо\-го далее CVP. Основы 
проб\-ле\-ма\-ти\-ки были заложены так\-же в~стать\-ях российских ученых 
М.\,М.~Бонгарда (1967~г.)\ и~М.\,Н.~Вайнцвайга (1973~г.). В~дальнейшем 
на\-прав\-ле\-ние CVP развивалось в~работах отечественных и~зарубежных авторов 
и~наиболее существенное развитие получило в~стать\-ях~[2--8]. В~част\-ности, 
в~\cite{6-duk, 7-duk} впер\-вые по\-стро\-ены логические классификаторы, 
ориентированные на задание час\-тич\-ных порядков на множествах значений 
при\-зна\-ков. 
  
  Зарубежные исследования в~об\-ласти логической классификации 
пред\-став\-ле\-ны методами на\-прав\-ле\-ний LAD и~FCA. Осно\-во\-по\-ла\-га\-ющие идеи LAD и~FCA принадлежат 
соответственно П.~Хам\-ме\-ру~[9] и~Р.~Вил\-ле~[10]. 
  
  В России методы LAD предложены практически параллельно с~зарубежными 
авторами и,~в~основ\-ном, получили развитие в~работах школы 
Ю.\,И.~Жу\-рав\-лё\-ва (см., например,~[11]). Следует отметить, что в~пуб\-ли\-ка\-ци\-ях 
И.\,С.~Масича развиваются идеи за\-ру\-беж\-ных пред\-ста\-ви\-те\-лей LAD~[12]. 
Методы FCA пред\-став\-ле\-ны работами В.\,К.~Фин\-на, С.\,О.~Куз\-не\-цо\-ва, 
М.\,И.~За\-бе\-жай\-ло, Д.\,И.~Иг\-на\-то\-ва и~Д.\,В.~Ви\-но\-гра\-до\-ва~[13--17].  
В~\cite{13-duk} предложен так на\-зы\-ва\-емый метод автоматического по\-рож\-де\-ния 
гипотез (или ДСМ-ме\-тод), который позднее в~1990-х~гг.\ был адап\-ти\-ро\-ван 
В.\,К.~Фин\-ном и~его учениками для задач машинного обуче\-ния. 
  
  Все три названных на\-прав\-ле\-ния имеют много общего. С~другой стороны, 
каж\-дый из них использует свою терминологию и~демонстрирует некоторую 
ори\-ги\-наль\-ность. В~разд.~2--4 на\-сто\-ящей работы приводятся основные понятия, 
ис\-поль\-зу\-емые при описании классификаторов на\-прав\-ле\-ний CVP, LAD и~FCA, 
а~так\-же принципиальные схемы работы этих классификаторов. В~разд.~5 
предлагается общая схема логической классификации це\-ло\-чис\-лен\-ных данных 
с~использованием понятий на\-прав\-ле\-ния CVP. 
  
\section{Направление CVP}

  Введем основные понятия, ис\-поль\-зу\-емые при описании классификаторов 
на\-прав\-ле\-ния CVP. 
  
  Пусть $H\hm= \{ x_{j_1}, \ldots , x_{j_r}\}$~--- набор из~$r$ раз\-лич\-ных 
при\-зна\-ков, $\sigma\hm= (\sigma_1, \ldots , \sigma_r)$~--- набор, 
в~котором~$\sigma_i$~--- до\-пус\-ти\-мое значение при\-зна\-ка~$x_{j_i}$, 
$i\hm=1,2,\ldots , r$. Пара $(\sigma, H)$ называется элементарным 
классификатором (ЭК).
  
  Пусть $S\in M$, $S\hm= (a_1, \ldots , a_n)$, где~$a_j$, $j\hm\in \{1,2,\ldots,  
n\}$,~--- значение при\-зна\-ка~$x_j$ для объекта~$S$. Бли\-зость объекта~$S$ и~ЭК 
$(\sigma, H)$ оценивается величиной $B(S, \sigma, H)$:
$$
B(S, \sigma, H)=\begin{cases}
1, &\mbox{если}\ a_{j_t}=\sigma_t\ \mbox{при}\ t= 1,2,\ldots , r;\\
0 & \mbox{в\ противном\ случае}.
\end{cases}
$$
 Объект~$S$ \textit{содержит} ЭК $(\sigma, H)$, если $B(S,\sigma, 
H)\hm=1$.
  
  Элементарный  классификатор $(\sigma, H)$ называется \textit{корректным для класса}~$K$, если для 
любой пары прецедентов $S\hm\in  K$ и~$S^\prime\notin K$ не выполнено
  $B(S, \sigma, H)\hm= B(S^\prime, \sigma, H)\hm=1$. Корректный ЭК $(\sigma, 
H)$ класса~$K$ называется тупиковым, если любой ЭК $(\sigma^\prime, 
H^\prime)$ такой, что $\sigma^\prime\hm \subset \sigma$, $H^\prime \subset H$, 
не является корректным для~$K$.
  
  Множество прецедентов класса~$K$ обо\-зна\-ча\-ет\-ся через $R(K)$. Элементарный  классификатор $(\sigma, 
H)$~--- (\textit{тупиковый}) \textit{пред\-ста\-ви\-тель\-ный} для класса~$K$, если 
$(\sigma, H)$~--- (тупиковый) корректный ЭК для~$K$ и~хотя бы один объект 
из $R(K)$ содержит $(\sigma, H)$. 
  
  Положим $R_K(\sigma, H) \hm= \{S\hm\in R(K): B(S,\sigma, H)\hm=1\}$, $\vert 
R_K(\sigma, H)\vert$~--- мощ\-ность множества~$R_K(\sigma, H)$. 
  
  На этапе обучения классифицирующий алгоритм строит для каж\-до\-го 
класса~$K$ некоторое множество $P_1(K)$ тупиковых пред\-ста\-ви\-тель\-ных ЭК. 
Оценка за при\-над\-леж\-ность объекта~$S$ классу~$K$ вы\-чис\-ля\-ет\-ся на основе 
суммирования величин $\vert R_K(\sigma, H)\vert \times B(S,\sigma, H)$, 
$(\sigma, H)\hm\in P_1(K)$. Объект~$S$ относится к~классу с~наибольшей 
оцен\-кой. Если максимальную оцен\-ку получают несколько классов, то 
происходит отказ от классификации. При по\-стро\-ении~$P_1(K)$ возникают 
труд\-но\-ре\-ша\-емые дис\-крет\-ные пе\-ре\-чис\-ли\-тель\-ные задачи, среди которых 
цент\-раль\-ное мес\-то принадлежит задаче монотонной дуализации~\cite{6-duk}.
  
\section{Направление LAD}

  При описании методов на\-прав\-ле\-ния LAD, как правило, рас\-смат\-ри\-ва\-ют\-ся 
задачи с~бинарными при\-зна\-ка\-ми. Переход от це\-ло\-чис\-лен\-ных признаков 
к~бинарным может быть осуществлен по\-средст\-вом описанного ниже  
\textit{one-hot} преобразования (см.\ замечание~1). 
  
  Отметим, что в~случае бинарных при\-зна\-ков ЭК~$(\sigma, H)$, $H\hm= \{ 
x_{j_1}, \ldots , x_{j_r}\}$, $\sigma\hm= (\sigma_1, \ldots , \sigma_r)$, 
$\sigma_i\hm\in \{0,1\}$, $i\hm=1,2,\ldots , r$,~--- это элементарная 
конъюнкция~$B$ над переменными $x_1, \ldots , x_n$ вида 
$x^{\sigma_1}_{j_1}\wedge \cdots \wedge x^{\sigma_r}_{j_r}$, которая 
обращается в~1 на описании объекта~$S$, если объект~$S$ содержит ЭК~$(\sigma, H)$, 
т.\,е.\ если описание объекта~$S$ принадлежит интервалу 
ис\-тин\-ности~$N_B$ конъюнкции~$B$.
  
  Пусть $f_K(x_1, \ldots , x_n)$~--- час\-тич\-ная булева функция, определенная на 
бинарных описаниях объектов обуча\-ющей выборки и~при\-ни\-ма\-ющая 
значение~1 и~0 соответственно на описаниях прецедентов из класса~$K$ и~не 
из~$K$. Через~$N_{f_K}$ и~$N_{\bar{f}_K}$ обозначим соответственно 
множество описаний прецедентов из класса~$K$ и~множество описаний 
прецедентов, не принадлежащих клас\-су~$K$. 
  
  Элементарная конъюнкция~$B$ называется \textit{до\-пус\-ти\-мой} 
для~$f_K$, если $N_B\cup N_{\bar{f}_K}\hm= \emptyset$ и~$N_B\cup 
N_{f_K}\not=\emptyset$. До\-пус\-ти\-мая для~$f_K$ конъюнкция~$B$ называется 
\textit{максимальной} для~$f_K$, если не существует до\-пус\-ти\-мой для~$f_K$ 
конъюнкции~$B^\prime$ такой, что $N_{B^\prime} \hm\supset N_B$. Нетрудно 
видеть, что в~терминах тео\-рии булевых функций пред\-ста\-ви\-тель\-ный ЭК 
класса~$K$~--- это до\-пус\-ти\-мая конъюнкция для функции~$f_K$, а~тупиковый 
представительный ЭК класса~$K$~--- это максимальная конъюнкция 
для~$f_K$. В~LAD до\-пус\-ти\-мая конъюнкция для~$f_K$ называется 
\textit{логической за\-ко\-но\-мер\-ностью} (ЛЗ) класса~$K$. Таким образом, 
определение ЛЗ эквивалентно определению 
представительного~ЭК. 
  
  Пусть ЭК $(\sigma, H)$~--- представительный ЭК для класса~$K$. Элементарный 
классификатор~$(\sigma, H)$ называется \textit{наибольшей} ЛЗ (\textit{maximum pattern}), если 
  $\vert R_K (\sigma, H) \vert \hm\geq \vert R_K(\sigma^\prime, H^\prime)\vert$ 
для любого другого пред\-ста\-ви\-тель\-но\-го ЭК $(\sigma^\prime, H^\prime)$ 
класса~$K$. Элементарный 
классификатор~$(\sigma, H)$ называется \textit{сильной} ЛЗ (\textit{strong 
pattern}), если не существует другого представительного ЭК $(\sigma^\prime, 
H^\prime)$ класса~$K$ такого, что $ R_K(\sigma, H) \hm\subset 
R_K(\sigma^\prime, H^\prime)$ ~\cite{18-duk}.
  
  В LAD схема работы классифицирующего алгоритма пол\-ностью аналогична 
схеме работы алгоритма CVP. На этапе обуче\-ния для каж\-до\-го класса~$K$ 
строится некоторое множество $P_2(K)$ ЛЗ, 
например строятся сильные ЛЗ. При этом решаются 
слож\-ные в~вы\-чис\-ли\-тель\-ном плане оптимизационные задачи линейного 
программирования. На следующем этапе каж\-дый элемент множества $P_2(K)$ 
<<голосует>> за отнесение объекта~$S$ классу~$K$. Для оцен\-ки 
при\-над\-леж\-ности рас\-по\-зна\-ва\-емо\-го объекта~$S$ к~классу~$K$ суммируются 
величины $B(S,\sigma, H)$, $(\sigma, H)\hm\in P_2(K)$. 
  
\smallskip

\noindent
\textbf{Замечание~1.} Опишем процедуру бинаризации исходных це\-ло\-чис\-лен\-ных данных. 
Рас\-смот\-рим признак~$x_j$, $j\in \{1,2,\ldots , n\}$, и~множество $N_j\hm= \{a_{j1}, \ldots, 
a_{jk_j}\}$ до\-пус\-ти\-мых значений этого при\-зна\-ка. Пусть $a,b\hm\in N_j$. Положим 
$$
\delta(a,b)=\begin{cases}
1, & \mbox{если}\ a=b\,;\\
0 & \mbox{иначе}.
\end{cases}
$$
 Тогда бинарное описание 
объекта $S\hm= (a_1, \ldots , a_n)$, $a_j\hm\in N_j$, $j\hm=1,2,\ldots , n$, полученное из 
исходного на основе \textit{one-hot} преобразования, будет иметь вид: $(\delta(a_{11}, a_1), 
\ldots , \delta(a_{1 k_1}, a_1) , \ldots , \delta(a_{n1}, a_n), \ldots$\linebreak $\ldots , \delta(a_{n k_n},a_n))$.
  
  Таким образом, каждый объект может быть описан при помощи $k_1+\cdots +k_n$ 
бинарных признаков. Не\-труд\-но заметить, что признаку~$x_j$, $j\hm=1,2,\ldots , n$, 
и~до\-пус\-ти\-мо\-му значению $a_{ji}\hm\in N_j$, $i\hm=1,2,\ldots, k_j$, признака~$x_j$ 
соответствует бинарный при\-знак~$\tilde{x}_t$, $t\hm= k_1+\cdots + k_{j-1} \hm+i$, 
причем~$\tilde{x}_t$ принимает значение~1 тогда и~только тогда, когда при\-знак~$x_j$ 
принимает значение~$a_{ji}$. Бинарный при\-знак~$\tilde{x}_t$ и~пару $(a_{ji}, x_j)$ назовем 
\textit{родственными}. 
  
\smallskip

\noindent
\textbf{Замечание~2.} В задачах с~вещественными признаками, как правило, при\-зна\-ки сначала 
преобразуют в~це\-ло\-чис\-лен\-ные при\-зна\-ки (см., например,~\cite{19-duk}), а~затем проводится 
процесс бинаризации.
  
\section{Направление FCA}

  Опишем основные понятия из FCA. Отметим, что обычно методы FCA, как 
и~методы LAD, работают с~бинарной информацией. Однако one-hot 
преобразование (см.\ замечание~1 в~разд.~3) применяется в~FCA и~в~этом 
случае. 
  
  Пусть $\tilde{X}$~--- множество бинарных признаков мощ\-ности~$n^\prime$, 
полученное в~результате one-hot преобразования множества~$X$. Говорят, что 
объект $S\hm= (a_1, \ldots , a_{n^\prime})$ \textit{обладает при\-зна\-ком} 
$\tilde{x}_j\hm\in \tilde{X}$, $j\hm\in \{1,2,\ldots , n^\prime\}$, если $a_j\hm=1$. 
  
  \textit{Формальным контекстом} называется трой\-ка вида $C\hm= (R, 
\tilde{X}, I)$, где~$R$~--- некоторое множество объектов; $I$~--- бинарное 
отношение меж\-ду множествами~$R$ и~$\tilde{X}$. Иначе говоря, формальный 
кон\-текст пред\-став\-ля\-ет собой булеву мат\-ри\-цу~$L$, строками которой служат 
описания объектов из~$R$ по\-сред\-ст\-вом признаков из~$\tilde{X}$. Далее запись 
$(S, \tilde{x})\hm\in I$ означает, что объект $S\hm\in R$ обладает при\-зна\-ком 
$\tilde{x}\hm\in \tilde{X}$.
  
  \textit{Операторами вывода} для формального кон\-текс\-та $C\hm= 
(R,\tilde{X},I)$ и~множеств $A\hm\subseteq R$, $B\hm\subseteq \tilde{X}$ 
называются соответственно множества $A^*\hm= \{ \tilde{x}\hm\in 
\tilde{X}\vert (S,\tilde{x})\hm\in I\ \forall\,S \hm\in A\}$ и~$B^*\hm= \{ S\hm\in 
R\vert (S,\tilde{x})\hm\in I\ \forall\, \tilde{x}\hm\in \tilde{X}\}$.
  
  Таким образом, $A^*$~--- это оператор, воз\-вра\-ща\-ющий столб\-цы  
мат\-ри\-цы~$L$, которые в~пересечении с~заданными строками образуют 
<<максимальную>> под\-мат\-ри\-цу, име\-ющую~$\vert A\vert$ строк и~не 
име\-ющую элементов, рав\-ных~0. Аналогично, $B^*$~--- это оператор, 
воз\-вра\-ща\-ющий строки мат\-ри\-цы~$L$, которые в~пересечении с~заданными 
столб\-ца\-ми образуют <<максимальную>> под\-мат\-ри\-цу, име\-ющую $\vert 
B\vert$ столб\-цов и~не име\-ющую элементов, рав\-ных~0.
  
  \textit{Формальным понятием} формального кон\-текс\-та $C\hm= 
(R,\tilde{X}, I)$ называется пара вида $(A,B)$, $A\hm\subseteq R$, 
$B\hm\subseteq \tilde{X}$, такая что $A^*\hm= B$, $B^*\hm= A$. Заметим, что 
паре $(A,B)$ соответствует <<максимальная>> под\-мат\-рица мат\-ри\-цы~$L$, 
со\-сто\-ящая из единичных элементов.
  
  Рассмотрим формальные кон\-текс\-ты $C_K\hm= (R(K),\tilde{X},I)$ 
и~$C_{\bar{K}}\hm= (R(\bar{K}),\tilde{X},I)$. Введем обозначения: $H^+$~--- 
оператор вывода для формального кон\-текс\-та~$C_K$ и~множества 
при\-зна\-ков~$H$; $H^-$~--- оператор вывода для формального 
кон\-текс\-та~$C_{\bar{K}}$ и~множества при\-зна\-ков~$H$. 
  
  Набор признаков $H\hm\subseteq \tilde{X}$ называется 
\textit{положительной ДСМ-ги\-по\-те\-зой} для класса~$K$, если пара   $(H^+, H)$
является формальным понятием формального кон\-текс\-та~$C_K$ и~$H^-\hm= 
\emptyset$.
  
  Положительной ДСМ-ги\-по\-те\-зе $H\hm= \{ \tilde{x}_{j_1}, \ldots$\linebreak $\ldots , 
\tilde{x}_{j_r}\}$ класса~$K$ по\-ста\-вим в~соответствие ЭК $(\sigma, H^\prime)$, 
$\sigma\hm= (\sigma_1, \ldots, \sigma_r)$, $H^\prime\hm= 
  \{ x_{t_1}, \ldots , x_{t_r}\}$, $H^\prime\hm\subseteq X$, такой что при любом 
$i\hm\in \{1,2,\ldots , r\}$ пара $(\sigma_i, x_{t_i})$ и~при\-знак~$\tilde{x}_{j_i}$ 
являются родственными (см.\ замечание~1 в~разд.~3). Будем говорить, 
что~$H$ по\-рож\-да\-ет ЭК $(\sigma, H^\prime)$. Не\-труд\-но видеть, что $(\sigma, 
H^\prime)$~--- пред\-ста\-ви\-тель\-ный ЭК класса~$K$. 
  
  \smallskip
  
  \textbf{Пример.} Рассмотрим задачу классификации с~тремя прецедентами $S_1\hm= (1,1,1)$, 
$S_2\hm= (0,1,1)$ и~$S_3\hm= (1,0,0)$. При этом $S_1, S_2\hm\in K_1$, $S_3\hm\in K_2$. 
После one-hot преобразования прецеденты будут иметь соответственно описания 
$(0,1,0,1,0,1)$, $(1,0,0,1,0,1)$, $(0,1,1,0,1,0)$. В~данном примере ЭК $((1,1), \{ x_2, x_3\})$ 
по\-рож\-ден положительной для класса~$K_1$ 
 ДСМ-ги\-по\-те\-зой $H\hm= (\tilde{x}_4, \tilde{x}_6\}$. Этот ЭК не является тупиковым 
представительным ЭК класса~$K_1$, по\-сколь\-ку $((1), \{x_2\})$~--- пред\-ста\-ви\-тель\-ный ЭК 
для~$K_1$. Отметим, что тупиковый пред\-ста\-ви\-тель\-ный ЭК $((1), \{x_2\})$ класса~$K_1$ не 
по\-рож\-да\-ет\-ся ни одной положительной ДСМ-ги\-по\-те\-зой класса~$K_1$. 
  
  ДСМ-классификатор действует более строго по сравнению 
с~классификаторами из CVP и~LAD. На пер\-вом этапе для каж\-до\-го класса~$K$ 
строится некоторое множество $P_3(K)$ положительных для класса~$K$ 
гипотез. Объект~$S$ относится к~классу~$K$, если~$S$ содержит хотя бы один 
ЭК, по\-рож\-да\-емый гипотезой из $P_3(K)$, и~не содержит ни одного ЭК, 
порождаемого гипотезами из $P_3(K^\prime)$, $K^\prime \not= K$. 
В~про\-тив\-ном случае происходит отказ от классификации. При 
по\-стро\-ении~$P_3(K)$ возникают дискретные пе\-ре\-чис\-ли\-тель\-ные задачи, 
которые алгоритмически менее слож\-ны, чем задача монотонной 
дуа\-ли\-за\-ции~\cite{20-duk}.
  
\section{Схема обучения логического классификатора}

  Предлагаемое описание общей схемы обуче\-ния алгоритмов логической 
классификации основано на приводимых ниже утверж\-де\-ни\-ях~1--4. 
  
  Пусть~$\mathcal{P}(K)$~--- множество всех пред\-ста\-ви\-тель\-ных ЭК 
класса~$K$. Бинарное отношение $\preceq$ на~$\mathcal{P}(K)$ называется 
час\-тич\-ным предпорядком, если это отношение реф\-лек\-сив\-но ($(\sigma, H) 
\preceq (\sigma, H)$ для любого $(\sigma, H)\hm\in \mathcal{P}(K)$ ) 
и~транзитивно ($(\sigma, H)\preceq (\sigma^\prime, H^\prime)$ и~$(\sigma^\prime, 
H^\prime) \preceq (\sigma^{\prime\prime}, H^{\prime\prime})$  влечет $(\sigma, 
H)\preceq (\sigma^{\prime\prime}, H^{\prime\prime})$). Час\-тич\-ный предпорядок 
называется \textit{частичным порядком}, если он ан\-ти\-сим\-мет\-ри\-чен $((\sigma, 
H)\preceq (\sigma^\prime, H^\prime)$ и~$(\sigma^\prime, H^\prime) \preceq 
(\sigma, H)$ влечет $(\sigma, H)\hm= (\sigma^\prime, H^\prime))$. Запись 
$(\sigma, H) \prec (\sigma^\prime, H^\prime)$ озна\-ча\-ет, что $(\sigma, H) \preceq 
(\sigma^\prime, H^\prime)$ и~$(\sigma^\prime, H^\prime)\not\preceq (\sigma, H)$.
Элементарный 
классификатор $(\sigma, H) \hm\in \mathcal{P}(K)$ называется \textit{максимальным 
элементом} в~$\mathcal{P}(K)$ относительно час\-тич\-но\-го 
(пред)порядка~$\preceq$, если не существует ЭК $(\sigma^\prime, H^\prime) 
\hm\in \mathcal{P}(K)$ такого, что $(\sigma, H) \prec (\sigma^\prime, H^\prime)$.
  
  Зададим частичный порядок $\preceq_1$ на множестве~$\mathcal{P}(K)$. 
Будем считать, что $(\sigma_2, H_2)\hm\in \mathcal{P}(K)$ следует за $(\sigma_1, 
H_1) \hm\in \mathcal{P}(K)$, если $H_2\hm\subseteq H_1$ 
и~$\sigma_2\hm\subseteq \sigma_1$. Спра\-вед\-ливо
  
  \smallskip
  
  \noindent
  \textbf{Утверждение~1.} \textit{Элементарный 
классификатор~$(\sigma, H)$ будет тупиковым 
пред\-ста\-ви\-тель\-ным для класса~$K$ тогда и~только тогда, когда  
$(\sigma, H)$~--- максимальный относительно час\-тич\-но\-го 
порядка~$\preceq_1$ элемент множества~$\mathcal{P}(K)$}.
  
  \smallskip
  
  Справедливость утверж\-де\-ния~1 следует непосредственно из определения 
понятия тупикового пред\-ста\-ви\-тель\-но\-го ЭК, приведенного в~разд.~2.
  
  Зададим на множестве~$\mathcal{P}(K)$ отношение час\-тич\-но\-го предпорядка~$\preceq_2$. 
  Будем считать, что $(\sigma_2, H_2)\hm\in \mathcal{P}(K)$ следует 
за $(\sigma_1, H_1)\hm\in \mathcal{P}(K)$, если $\vert R_K(\sigma_1, H_1)\vert 
\hm\leq \vert R_K(\sigma_2, H_2)\vert$. Спра\-вед\-ливо
  
  \smallskip
  
  \noindent
  \textbf{Утверждение~2.}\ \textit{Элементарный 
классификатор $(\sigma, H)$ будет наибольшей ЛЗ 
класса~$K$ тогда и~только тогда, когда $(\sigma, H)$~--- максимальный 
относительно час\-тич\-но\-го предпорядка~$\preceq_2$ элемент 
множества~$\mathcal{P}(K)$}.
  
  \smallskip
  
  Справедливость утверж\-де\-ния~2 следует непосредственно из определения 
понятия наибольшей ЛЗ, приведенного в~разд.~3.
  
  Зададим на множестве~$\mathcal{P}(K)$ отношение час\-тич\-но\-го 
предпорядка~$\preceq_3$, предложенного в~\cite{21-duk}. Будем считать, что 
$(\sigma_2, H_2)\hm\in \mathcal{P}(K)$ следует за $(\sigma_1, H_1) \hm\in 
\mathcal{P}(K)$, если $R_K(\sigma_1, H_1)\hm\subseteq R_K(\sigma_2, H_2)$. 
Спра\-вед\-ливо
  
  \smallskip
  
  \noindent
  \textbf{Утверждение~3.}\ \textit{Элементарный 
классификатор $(\sigma, H)$ будет сильной ЛЗ 
класса~$K$ тогда и~только тогда, когда $(\sigma, H)$~--- максимальный 
относительно час\-тич\-но\-го предпорядка~$\preceq_3$ элемент множества~$\mathcal{P}(K)$}.
  
  \smallskip
  
  Справедливость утверж\-де\-ния~3 следует непосредственно из определения 
понятия сильной ЛЗ, приведенного в~разд.~3.
  
  Зададим на множестве $\mathcal{P}(K)$ отношение час\-тич\-но\-го по\-ряд\-ка~$\preceq_4$. 
  Будем считать, что ЭК $(\sigma_2, H_2)\hm\in \mathcal{P}(K)$ 
следует за $(\sigma_1, H_1) \hm\in \mathcal{P}(K)$, если $R_K(\sigma_1, H_1) 
\hm\subseteq R_K(\sigma_2, H_2)$ и~$H_1\hm\subseteq H_2$. Спра\-вед\-ливо
  
  \smallskip
  
  \noindent
  \textbf{Утверждение~4.}\ \textit{Элементарный 
классификатор $(\sigma, H)$ по\-рож\-да\-ет\-ся 
положительной ДСМ-ги\-по\-те\-зой для класса~$K$ тогда и~только тогда, 
когда $(\sigma, H)$~--- максимальный относительно час\-тич\-но\-го по\-ряд\-ка~$\preceq_4$ 
элемент множества~$\mathcal{P}(K)$}. 
  
  \smallskip
  
  \noindent
  Д\,о\,к\,а\,з\,а\,т\,е\,л\,ь\,с\,т\,в\,о\,.\ \ \textit{Достаточность}. Пусть ЭК 
$(\sigma, H)\hm\in \mathcal{P}(K)$~--- максимальный элемент 
в~$\mathcal{P}(K)$ относительно порядка $\preceq_4$. Рас\-смот\-рим 
формальный кон\-текст $C_K\hm= (R(K), \tilde{X}, I)$, а~так\-же множество 
при\-зна\-ков~$\tilde{H}$ из~$\tilde{X}$, по\-рож\-да\-ющее ЭК $(\sigma, H)$. По 
определению $\tilde{H}^+\hm= R_K(\sigma, H)$. Покажем, что $R_K(\sigma, 
H)^+\hm= \tilde{H}$.
  
  Предположим противное. Пусть $R_K(\sigma, H)^+ \hm= H^\prime$, 
$\tilde{H}\hm\subset H^\prime$. Это означает, что существуют при\-знак 
$x_i\hm\in X$, $x_i\not\in H$, и~его до\-пус\-ти\-мое значение~$\sigma^\prime$, 
такие что при\-знак~$x_i$ принимает значение~$\sigma^\prime$ на всех объектах 
из $R_K(\sigma, H)$. Заметим, что ЭК $((\sigma_1, \ldots , \sigma_r, 
\sigma^\prime), H\cup x_i)$~--- корректный для класса~$K$, так как $(\sigma, 
H)$ не содержится в~прецедентах не из класса~$K$. Рас\-смат\-ри\-ва\-емый ЭК 
содержится в~прецедентах из класса~$K$, поскольку он содержится в~объектах 
из множества $R_K(\sigma, H)$. Следовательно, $((\sigma_1, \ldots , \sigma_r, 
\sigma^\prime), H\cup x_i)$~--- пред\-ста\-ви\-тель\-ный ЭК для клас\-са~$K$. 
  
  Нетрудно видеть, что $R_K((\sigma_1, \ldots , \sigma_r, \sigma^\prime), H\hm\cup x_i)\hm= R_K(\sigma, H)$. 
  Это значит, что $(\sigma, H) \prec_4$\linebreak $\prec_4 ((\sigma_1, \ldots , \sigma_r, \sigma^\prime), H\hm\cup x_i)$. Противоречие. 
  
  \textit{Необходимость} очевидна. Утверж\-де\-ние до\-ка\-зано.
  
 \vspace*{-4pt}
  
\section{Заключение}

 \vspace*{-2pt}

  В работе приведена общая схема работы алгоритмов логической 
классификации. В~рамках данной схемы в~единой терминологии дано 
оригинальное описание основ\-ных моделей логических\linebreak \mbox{классификаторов} 
на\-прав\-ле\-ний CVP, LAD и~FCA. Приведены утверж\-де\-ния, по\-ка\-зы\-ва\-ющие, что 
каж\-дое из рас\-смот\-рен\-ных на\-прав\-ле\-ний логической классификации на этапе 
обуче\-ния строит для каж\-до\-го класса~$K$ специальное подмножество 
множества~$\mathcal{P}(K)$ пред\-ста\-ви\-тель\-ных ЭК этого класса. Данное 
подмножество определяется заданием час\-тич\-но\-го порядка или предпорядка на 
множестве $\mathcal{P}(K)$ и~образуется максимальными относительно 
заданного порядка элементами. При этом каждое из на\-прав\-ле\-ний 
ориентируется на <<свой>> порядок. Результаты \mbox{статьи} имеют значение для 
создания общей тео\-рии логической классификации и~разработки более 
совершенных моделей классификаторов. Например, идеи, лежащие в~основе 
методов анализа час\-ти\-чно упорядоченных данных, разработанных в~последнее 
время в~на\-прав\-ле\-нии CVP, могут быть успеш\-но применены и~для по\-стро\-ения 
классификаторов более общего вида в~двух других на\-прав\-ле\-ни\-ях логической 
клас\-си\-фи\-ка\-ции. 

  \vspace*{-4pt}
  
{\small\frenchspacing
 { %\baselineskip=12pt
 %\addcontentsline{toc}{section}{References}
 \begin{thebibliography}{99}
 
  \vspace*{-2pt}
  
\bibitem{1-duk}
\Au{Чегис И.\,А., Яблонский~С.\,В.} Логические способы конт\-ро\-ля электрических схем~// 
Труды Математического института имени В.\,А.~Стеклова АН СССР, 1958. Т.~51.  
С.~270--360. 
\bibitem{2-duk}
\Au{Баскакова Л.\,В., Журавлёв~Ю.\,И.} Модель рас\-по\-зна\-ющих алгоритмов 
с~пред\-ста\-ви\-тель\-ны\-ми наборами и~сис\-те\-ма\-ми опор\-ных множеств~// Ж.~вычисл. матем. 
и~матем. физ., 1981. Т.~21. №\,5. С.~1264--1275.

\bibitem{4-duk} %3
\Au{Дюкова Е.\,В., Журавлёв~Ю.\,И., Рудаков~К.\,В.} Об ал\-геб\-ра\-и\-че\-ском синтезе  
кор\-рек\-ти\-ру\-ющих процедур рас\-по\-зна\-ва\-ния на базе элементарных алгоритмов~//\linebreak Ж.~вычисл. матем. и~матем. физ., 1996. Т.~36. №\,8. С.~215--223.

\bibitem{3-duk} %4
\Au{Дюкова Е.\,В., Журавлёв~Ю.\,И.} Дискретный анализ признаковых описаний в~задачах 
рас\-по\-зна\-ва\-ния большой раз\-мер\-ности~// Ж.~вычисл. матем. и~матем. физ., 2000. Т.~40. 
№\,8. С.~1264--1278.

\bibitem{5-duk}
\Au{Дюкова Е.\,В., Песков~Н.\,В.} Поиск информативных фрагментов описаний объектов 
в~дис\-крет\-ных процедурах распознавания~// Ж. вычисл. матем. и~матем. физ., 2002. Т.~42. 
№\,5. С.~741--753.
\bibitem{6-duk}
\Au{Дюкова~Е.\,В., Масляков~Г.\,О., Прокофьев~П.\,А.} О~логическом анализе данных 
с~час\-тич\-ны\-ми порядками в~задаче классификации по прецедентам~// Ж. вычисл. матем. и~матем. физ., 2019. Т.~59. №\,9. С.~1605--1616.
\bibitem{7-duk}
\Au{Дюкова Е.\,В., Масляков~Г.\,О.} О~выборе час\-тич\-ных порядков на множествах 
значений при\-зна\-ков в~задаче классификации~// Информатика и~её применения, 2021. Т.~15. 
Вып.~4. С.~72--78. doi: 10.14357/ 19922264210410.
\bibitem{8-duk}
\Au{Dragunov N.\,A., Djukova~E.\,V., Dju\-ko\-va~A.\,P.} Supervised classification and finding 
frequent elements in data~// 8th  Conference (International) on Information Technology and 
Nanotechnology Proceedings.~--- Piscataway, NJ, USA: IEEE, 2022. Art.~9848521. 5~p.
\bibitem{9-duk}
\Au{Crama, Y., Hammer~P.\,L., Ibaraki~T.} Cause-effect relationships and partially defined 
Boolean functions~// Ann. Oper. Res., 1988. Vol.~16. Iss.~1. P.~299--325. doi: 
10.1007/ BF02283750.

\bibitem{10-duk}
\Au{Wille R.} Restructuring lattice theory: An approach based on hierarchies of concepts~// 
Ordered sets~/ Ed. I.~Rival.~--- NATO Advanced Study Institutes ser.~--- Netherlands: 
Springer, 1981. Vol.~83. P.~445--470. doi: 10.1007/978-94-009-7798-3\_15.
\bibitem{11-duk}
\Au{Журавлёв Ю.\,И., Рязанов~В.\,В., Сень\-ко~О.\,В.} Рас\-по\-зна\-ва\-ние. Математические 
методы. Программная сис\-те\-ма. Практические применения.~--- М.: ФАЗИС, 2006. 159~с. 
\bibitem{12-duk}
\Au{Масич И.\,С.} Метод оптимальных логических ре\-ша\-ющих правил для задач 
рас\-по\-зна\-ва\-ния и~прогнозирования~// Сис\-те\-мы управ\-ле\-ния и~информационные 
технологии, 2019. Т.~75. №\,1. С.~31--37.
\bibitem{13-duk}
\Au{Финн В.\,К.} О~возможности формализации прав\-до\-по\-доб\-ных рас\-суж\-де\-ний 
средствами многозначных логик~// Всесоюзн. симпозиум по логике и~методологии науки.~--- 
Киев: Наукова думка, 1976. С.~82--83.
\bibitem{14-duk}
\Au{Kuznetsov S.\,O.} Mathematical aspects of concept analysis~// J.~Mathematical Sciences, 1996. Vol.~80. 
Iss.~2. P.~1654--1698. doi: 10.1007/BF02362847.
\bibitem{15-duk}
\Au{Gnatyshak D.\,V., Ignatov~D.\,I., Kuz\-ne\-tsov~S.\,O.} Triadic formal concept analysis and 
triclustering: Searching for opti-\linebreak\vspace*{-12pt}

\pagebreak

\noindent
mal patterns~// Mach. Learn., 2015. Vol.~101. P.~271--302.
doi: 10.1007/s10994-015-5487-y.
\bibitem{16-duk}
\Au{Забежайло М.\,И.} О~некоторых оценках слож\-ности вы\-чис\-ле\-ний  
в~ДСМ-рас\-суж\-де\-ни\-ях~// Искусственный интеллект и~принятие решений, 2015. 
Часть~I: №\,1. С.~3--17. Часть~II: №\,2. С.~3--17.
\bibitem{17-duk}
\Au{Виноградов Д.\,В.} О~пред\-став\-ле\-нии объектов битовыми строками для  
ВКФ-ме\-то\-да~// Научная и~техническая информация. Сер.~2, 2018. №\,5. С.~1--4.
\bibitem{18-duk}
\Au{Bonates T.\,O., Hammer~P.\,L., Kogan~A.} Maximum patterns in datasets~// Discrete Appl. 
Math., 2008. Vol. 156. Iss.~6. P.~846--861. doi: 10.1016/j.dam.2007.06.004.
\bibitem{19-duk}
\Au{Eckstein J., Hammer~P.\,L., Liu~Y., Nediak~M., Simeone~B.} The maximum box problem and 
its application to data analysis~// Comput. Optim. Appl., 2002. Vol.~23. P.~285--298. doi: 
10.1023/A:1020546910706.
\bibitem{20-duk}
\Au{Kuznetsov S.\,O., Obiedkov~S.\,A.} Comparing performance of algorithms for generating 
concept lattice~// J.~Exp. Theor. Artif. In., 2002. Vol.~14. Iss.~2-3. P.~189--216. doi: 
10.1080/09528130210164170.
\bibitem{21-duk}
\Au{Hammer P.\,L., Kogan~A., Simeone~A., Szedm$\acute{\mbox{a}}$k~B.} Pareto-optimal 
patterns in logical analysis of data~// Discrete Appl. Math., 2004. Vol.~144. Iss.~1-2. P.~79--102.  
doi: 10.1016/j.dam.2003.08.013.
\end{thebibliography}

 }
 }

\end{multicols}

\vspace*{-6pt}

\hfill{\small\textit{Поступила в~редакцию 11.01.23}}

\vspace*{8pt}

%\pagebreak

%\newpage

%\vspace*{-28pt}

\hrule

\vspace*{2pt}

\hrule



\def\tit{LOGICAL METHODS OF~CORRECT DATA CLASSIFICATION}


\def\titkol{Logical methods of~correct data classification}


\def\aut{E.\,V.~Djukova, G.\,O.~Masliakov, and~A.\,P.~Djukova}

\def\autkol{E.\,V.~Djukova, G.\,O.~Masliakov, and~A.\,P.~Djukova}

\titel{\tit}{\aut}{\autkol}{\titkol}

\vspace*{-10pt}


\noindent
Federal Research Center ``Computer Science and Control'' of the Russian Academy of Sciences,  
44-2~Vavilov Str., Moscow 119133, Russian Federation



\def\leftfootline{\small{\textbf{\thepage}
\hfill INFORMATIKA I EE PRIMENENIYA~--- INFORMATICS AND
APPLICATIONS\ \ \ 2023\ \ \ volume~17\ \ \ issue\ 3}
}%
 \def\rightfootline{\small{INFORMATIKA I EE PRIMENENIYA~---
INFORMATICS AND APPLICATIONS\ \ \ 2023\ \ \ volume~17\ \ \ issue\ 3
\hfill \textbf{\thepage}}}

\vspace*{3pt}



\Abste{The work is devoted to issues of the application of discrete apparatus (logical methods 
of integer data analysis) for the supervised classification problem. Three main directions of 
logical classification are considered: Correct Voting Procedures (CVP), Logical Analysis of Data (LAD), and 
Formal Concept Analysis (FCA). Using the terminology of the CVP direction, the 
basic concepts applied in LAD and FCA are presented. 
The general scheme of the logical classifier is described, according to which each logical 
classifier at the training stage sets some partial order on a~special set of fragments of precedents 
descriptions and searches for the maximum elements of this set relative to the given order. Such 
studies are important for creating a~general theory of correct classification according to 
precedents based on the use of a~discrete apparatus.}

\KWE{supervised classification; logical classifier; Correct Voting Procedures; Logical 
Analysis of Data; Formal Concept Analysis; irredundant representative elementary classifier; 
 strong pattern; JSM-hypothesis; partial order}

\DOI{10.14357/19922264230309}{OZHXOX}

%\vspace*{-20pt}

%\Ack
%\noindent

  

%\vspace*{6pt}

  \begin{multicols}{2}

\renewcommand{\bibname}{\protect\rmfamily References}
%\renewcommand{\bibname}{\large\protect\rm References}

{\small\frenchspacing
 {%\baselineskip=10.8pt
 \addcontentsline{toc}{section}{References}
 \begin{thebibliography}{99} 
\bibitem{1-duk-1}
\Aue{Chegis, I.\,A., and S.\,V.~Yablonskiy.} 1958. Lo\-gi\-che\-skie spo\-so\-by kont\-ro\-lya  
elekt\-ri\-che\-skikh skhem [Logical methods of control of work of electric schemes]. 
\textit{Trudy Ma\-te\-ma\-ti\-che\-sko\-go instituta imeni V.\,A.~Steklova AN SSSR} [Proceedings of the 
Steklov Institute of Mathematics of the USSR Academy of Sciences] 51:270--360. 
\bibitem{2-duk-1}
\Aue{Baskakova, L., and Yu.~Zhuravlev.} 1981. A~model of recognition algorithms with 
representative sampls and systems of supporting sets. \textit{USSR Comp. Math. Math.} 
21(5):189--199. doi: 10.1016/0041-5553(81)90109-9.

\bibitem{4-duk-1}
\Aue{Djukova, E.\,V., Yu.\,I.~Zhuravlev, and K.\,V.~Ru\-da\-kov.} 1996. Algebraic-logic synthesis of 
correct recognition procedures based on elementary algorithms. \textit{Comput. Math. Math. 
Phys.} 36(8):1161--1167.

\bibitem{3-duk-1}
\Aue{Djukova, E., and Yu.~Zhuravlev.} 2000. Discrete analysis of feature descriptions in 
recognition problems of high dimensionality. \textit{Comp. Math. Math. Phys.} 
40(8):1214--1227.

\bibitem{5-duk-1}
\Aue{Djukova, E., and N.~Peskov.} 2002. Search for informative fragments of object descriptions 
in discrete recognition procedures. \textit{Comp. Math. Math. Phys.} 42(5):711--723.
\bibitem{6-duk-1}
\Aue{Djukova, E.\,V., G.\,O.~Maslyakov, and P.\,A.~Pro\-kofyev.} 2019. On the logical analysis of 
partially ordered data in the supervised classification problem. \textit{Comput. Math. Math. 
Phys.} 59(9):1542--1552.
\bibitem{7-duk-1}
\Aue{Djukova, E.\,V., and G.\,O.~Maslyakov.} 2021. O~vy\-bo\-re chas\-tic\-hnykh po\-ryad\-kov 
na mno\-zhest\-vakh zna\-che\-niy priz\-na\-kov v~za\-da\-che klas\-si\-fi\-ka\-tsii [On the choice 
of partial orders on feature values sets in the supervised classification problem]. 
\textit{Informatika i~ee Primeneniya~--- Inform. Appl.} 15(4):72--78. doi: 
10.14357/19922264210410.

\bibitem{8-duk-1}
\Aue{Dragunov, N.\,A., E.\,V.~Djukova, and A.\,P.~Dju\-ko\-va.} 2022. Supervised classification and 
finding frequent elements in data. \textit{8th Conference (International) on Information 
Technology and Nanotechnology Proceedings}. 9848521. 5~p.
\bibitem{9-duk-1}
\Aue{Crama, Y., P.\,L.~Hammer, and T.~Ibaraki.} 1988. Cause-effect relationships and partially 
defined Boolean functions. \textit{Ann. Oper. Res.} 16(1):299--325. doi: 10.1007/ BF02283750.
\bibitem{10-duk-1}
\Aue{Wille, R.} 1982. Restructuring lattice theory: An approach based on hierarchies of concepts. 
\textit{Ordered sets}. Ed. I.~Rival. NATO Advanced Study Institutes ser. Netherlands: Springer. 
83:445--470. doi: 10.1007/978-94-009-7798-3\_15.
\bibitem{11-duk-1}
\Aue{Zhuravlev, Yu.\,I., V.\,V.~Ryazanov, and O.\,V.~Sen\-ko.} 2006. \textit{Ras\-po\-zna\-va\-nie.  
Ma\-te\-ma\-ti\-che\-skie me\-to\-dy. Programmnaya sis\-te\-ma. Prak\-ti\-che\-skie  
pri\-me\-ne\-niya} [Recognition. Mathematical methods. Software system. Practical 
applications]. Moscow: FAZIS. 159~p.
\bibitem{12-duk-1}
\Aue{Masich, I.\,S.} 2019. Me\-tod op\-ti\-mal'\-nykh lo\-gi\-che\-skikh re\-sha\-yushchikh pra\-vil 
dlya za\-dach ras\-po\-zna\-va\-niya i~prog\-no\-zi\-ro\-va\-niya [The method of optimal logical 
decision rules for recognition and prediction problems]. \textit{Sistemy upravleniya 
i~informatsionnye tekhnologii} [Control Systems and Information Technologies] 75(1):31--37.
\bibitem{13-duk-1}
\Aue{Finn, V.\,K.} 1976. O~voz\-mozh\-nosti for\-ma\-li\-za\-tsii prav\-do\-po\-dob\-nykh  
ras\-suzh\-de\-niy sred\-st\-va\-mi mno\-go\-znach\-nykh lo\-gik [On the possibility of 
formalizing plausible reasoning by means of multivalued logics]. \textit{Vsesoyuzn. simpozium 
po logike i~metodologii nauki} [All-Union Symposium on the Logic and Methodology of 
Science]. Kiev: Naukova dumka. 82--83.
\bibitem{14-duk-1}
\Aue{Kuznetsov, S.\,O.} 1996. Mathematical aspects of concept analysis. \textit{J.~Mathematical Sciences} 
80(2):1654--1698. doi: 10.1007/BF02362847.
\bibitem{15-duk-1}
\Aue{Ignatov, D.\,I., D.\,V.~Gnatyshak, S.\,O.~Kuz\-ne\-tsov, and B.\,G.~Mir\-kin.} 2015. Triadic 
formal concept analysis and triclustering: Searching for optimal patterns. \textit{Mach. Learn}. 
101:271--302. doi: 10.1007/s10994-015-5487-y.
\bibitem{16-duk-1}
\Aue{Zabezhailo, M.\,I.} 2015. Some capabilities of enumeration control in the DSM method. 
\textit{Scientific and Technical Information Processing}. Part One: 41(6):335--347. doi: 
10.3103/S0147688214060082. Part~Two: 41(6):348--361. doi: 10.3103/S0147688214060094.
\bibitem{17-duk-1}
\Aue{Vinogradov, D.\,V.} 2018. On object representation by bit strings for the VKF-method. 
\textit{Automatic Documentation Mathematical Linguistics} 52(3):113--116. doi: 
10.3103/ S0005105518030044.
\bibitem{18-duk-1}
\Aue{Bonates, T.\,O., P.\,L.~Hammer, and A.~Kogan.} 2008. Maximum patterns in datasets. 
\textit{Discrete Appl. Math.} 156(6):846--861. doi: 10.1016/j.dam.2007.06.004.
\bibitem{19-duk-1}
\Aue{Eckstein, J., P.\,L.~Hammer, Y.~Liu, M.~Nediak, and B.~Simeone.} 2002. The maximum 
box problem and its application to data analysis. \textit{Comput. Optim. Appl.} 23:285--298. doi: 
10.1023/A:1020546910706.
\bibitem{20-duk-1}
\Aue{Kuznetsov, S.\,O., and S.\,A.~Obied\-kov.} 2002. Comparing performance of algorithms for 
generating concept lattice. \textit{J.~Exp. Theor. Artif. In.} 14(2-3):189--216. doi: 
10.1080/ 09528130210164170.
\bibitem{21-duk-1}
\Aue{Hammer, P.\,L., A.~Kogan, A.~Simeone, and B.~Szedm$\acute{\mbox{a}}$k.} 
2004. Pareto-optimal patterns in logical analysis of data. \textit{Discrete Appl. Math.} 
 144(1-2):79--102. doi: 10.1016/ j.dam.2003.08.013.
 \end{thebibliography}

 }
 }

\end{multicols}

\vspace*{-6pt}

\hfill{\small\textit{Received January 11, 2023}} 

\vspace*{-12pt}

\Contr

\noindent
\textbf{Djukova Elena V.} (b.\ 1945)~--- Doctor of Science in physics and mathematics, principal 
scientist, Federal Research Center ``Computer Science and Control'' of the Russian Academy of 
Sciences; 44-2~Vavilov Str., Moscow 119333, Russian Federation; \mbox{edjukova@mail.ru}

\vspace*{3pt}

\noindent
\textbf{Masliakov Gleb O.} (b.\ 1996)~--- PhD student, Federal Research Center ``Computer 
Science and Control'' of the Russian Academy of Sciences, 44-2~Vavilov Str., Moscow 119333, 
Russian Federation; \mbox{gleb-mas@mail.ru}


\vspace*{3pt}

\noindent
\textbf{Djukova Anastasia P.} (b.\ 1995)~--- PhD student, Federal Research Center ``Computer 
Science and Control'' of the Russian Academy of Sciences, 44-2~Vavilov Str., Moscow 119333, 
Russian Federation; \mbox{anastasia.d.95@gmail.com}



\label{end\stat}

\renewcommand{\bibname}{\protect\rm Литература} 