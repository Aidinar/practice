\def\stat{sopin}

\def\tit{МОДЕЛИРОВАНИЕ НАСТОЙЧИВОГО ПОВЕДЕНИЯ ПОЛЬЗОВАТЕЛЕЙ В~СЕТЯХ 5G NR С~АДАПТАЦИЕЙ~СКОРОСТИ~И~БЛОКИРОВКАМИ$^*$}

\def\titkol{Моделирование настойчивого поведения пользователей в~сетях 5G NR с~адаптацией скорости и~блокировками}

\def\aut{Э.\,С.~Сопин$^1$, А.\,Р.~Маслов$^2$, В.\,С.~Шоргин$^3$, В.\,О.~Бегишев$^4$}

\def\autkol{Э.\,С.~Сопин, А.\,Р.~Маслов, В.\,С.~Шоргин, В.\,О.~Бегишев}

\titel{\tit}{\aut}{\autkol}{\titkol}

\index{Сопин Э.\,С.}
\index{Маслов А.\,Р.}
\index{Шоргин В.\,С.}
\index{Бегишев В.\,О.}
\index{Sopin E.\,S.}
\index{Maslov A.\,R.}
\index{Shorgin V.\,S.}
\index{Begishev V.\,O.}


{\renewcommand{\thefootnote}{\fnsymbol{footnote}} \footnotetext[1]
{Исследование выполнено за счет гранта 
Российского научного фонда №\,22-79-10128.}}


\renewcommand{\thefootnote}{\arabic{footnote}}
\footnotetext[1]{Российский университет дружбы народов им.\ 
Патриса Лумумбы; Федеральный исследовательский центр <<Информатика и~управление>> Российской академии наук, \mbox{sopin-es@rudn.ru}}
\footnotetext[2]{Российский университет дружбы народов им.\ 
Патриса Лумумбы,\mbox{maslov-ar@rudn.ru}}
\footnotetext[3]{Федеральный исследовательский центр 
<<Информатика и~управление>> Российской академии наук, \mbox{vshorgin@ipiran.ru}}
\footnotetext[4]{Российский университет дружбы народов им.\ 
Патриса Лумумбы, \mbox{begishev-vo@rudn.ru}}

\vspace*{-8pt}



\Abst{Технология радиодоступа 5G NR (New Radio), работающая в~диапазоне 
мил\-ли\-мет\-ро\-вых волн, и~будущие терагерцевые сис\-те\-мы~6G предназначены для 
приложений, чувствительных к~ско\-рости. Такие приложения характеризуются 
адап\-тив\-ностью, поз\-во\-ля\-ющей снизить ско\-рость передачи в~соответствии с~текущими 
условиями сети. Не\-на\-деж\-ный характер сетей~5G/6G может вызывать повторные попытки 
продолжить обслуживание. Предложена модель обслуживания абонентов 
с~нетерпеливым поведением на основе ре\-сурс\-ной сис\-те\-мы массового обслуживания (РеСМО)
с~орбитой. В~качестве характеристик обслуживания рас\-смат\-ри\-ва\-ют\-ся вероятности 
блокировки до\-сту\-па на обслуживание и~прерывания за\-яв\-ки, а~так\-же коэффициент 
использования ресурсов сис\-те\-мы. Показано, что на\-стой\-чи\-вость пользователей 
поз\-во\-ля\-ет понизить рас\-смат\-ри\-ва\-емые вероятности: выполнение в~сред\-нем двух 
по\-втор\-ных попыток снижает оба показателя на 20\%--70\%. На\-стой\-чи\-вость увеличивает 
использование сис\-тем\-ных ресурсов на 20\%--40$\%$ и~снижает долю по\-тра\-чен\-ных 
впус\-тую ресурсов в~2--3~\mbox{раза}.}


\KW{5G; NR (New Radio); ресурсная СМО; повторные 
вызовы; блокировки распространения; прерывание об\-слу\-жи\-вания}

\DOI{10.14357/19922264230304}{ENSHKV}
  
\vspace*{-2pt}


\vskip 10pt plus 9pt minus 6pt

\thispagestyle{headings}

\begin{multicols}{2}

\label{st\stat}
    
\section{Введение}
%\label{sect:00}

%\vspace*{-4pt}


Технология 5G NR, ра\-бо\-та\-ющая в~диапазоне мил\-ли\-мет\-ро\-вых 
волн (28--100 ГГц), разработана для удовле\-тво\-ре\-ния потребностей мультимедийных 
приложений. В~будущих сис\-те\-мах~6G эти сис\-те\-мы будут дополнены технологиями 
радиодоступа, ра\-бо\-та\-ющи\-ми в~диапазоне час\-тот~100--300~ГГц и~об\-ла\-да\-ющи\-ми схожими 
характеристиками распространения ра\-дио\-волн.

Мультимедийные приложения требуют услуги поддержания по\-сто\-ян\-ной ско\-рости 
передачи данных на радиоинтерфейсе. Однако даже эти приложения час\-то снабжены 
возможностями адап\-та\-ции\linebreak ско\-рости. Приложение может отслеживать со\-сто\-яние 
со\-еди\-не\-ния и~реагировать на снижение пропускной спо\-соб\-ности сети и/или временную 
потерю соединения путем снижения тре\-бу\-емой \mbox{ско\-рости}. В~сис\-те\-мах связи 
с~прерываниями активного со\-еди\-не\-ния, таких как 5G/6G сети, пользователи будут 
воспринимать услугу с~пониженным качеством.

Принципиальное различие между микроволновыми технологиями, такими как LTE (long-term evolution), 
и~сис\-те\-ма\-ми мил\-ли\-мет\-ро\-вых волн и~терагерцевого диапазона час\-тот заключается 
в~не\-на\-деж\-ном характере беспроводного канала вследствие блокировки путей 
рас\-про\-стра\-не\-ния радиоволн, что приводит к~потерям соединения~\cite{begishev2021joint}. 
На ранних этапах развертывания потери со\-еди\-не\-ния будут 
приводить к~<<нетерпеливому>> поведению пользователей~--- попыткам вос\-ста\-нов\-ле\-ния 
активной сессии передачи данных. 

\begin{figure*}[b] %fig1
 \vspace*{1pt}
\begin{center}
   \mbox{%
\epsfxsize=163mm 
\epsfbox{sop-1.eps}
}
\end{center}
\vspace*{-9pt}
\Caption{Рассматриваемый сценарий развертывания}
\label{fig:deployment}
\end{figure*}

Цель работы состоит в~анализе эффектов влияния нетерпеливого поведения 
пользователей в~случае потери связи на характеристики обслуживания абонентов 
в~сетях радиодоступа 5G/6G в~условиях динамической адап\-та\-ции ско\-рости передачи 
и~динамической блокировки. Предложена математическая модель процесса обслуживания 
абонентов на основе РеСМО с~орбитой 
и~по\-втор\-ны\-ми вызовами. 

В~отличие от похожей модели с~неоднородными заявками 
и~орбитой~\cite{vlaskina2022}, в~данной работе модель учитывает специфику 
мил\-ли\-мет\-ро\-вых волн\,/\,те\-ра\-гер\-це\-вых час\-тот, вклю\-чая модели распространения,
 антенны и~блокировки пря\-мой ви\-ди\-мости. В~качестве характеристик обслуживания 
рассматриваются ве\-ро\-ят\-ность блокировки до\-сту\-па на обслуживание, ве\-ро\-ят\-ность 
прерывания сессии, а~так\-же коэффициент использования ресурсов сис\-темы.

\section{Системная модель}
%\label{sys}


Рассматривается базовая станция (БС) мил\-ли\-мет\-ро\-во\-го/те\-ра\-гер\-це\-во\-го диапазона 
с~радиусом по\-кры\-тия~$R_M$ (рис.~\ref{fig:deployment}). Зона покрытия соты рав\-на~$d_E$, 
зависит от ис\-поль\-зу\-емой схемы модуляции и~кодирования и~рас\-счи\-ты\-ва\-ет\-ся 
с~использованием модели рас\-про\-стра\-не\-ния 3GPP (3rd Generation Partnership Project)~\cite{kovalchukov2019accurate}. 
До\-ступ\-ная полоса про\-пус\-ка\-ния~--- $B$ МГц. В~зоне обслуживания БС находятся 
пешеходы с~плот\-ностью~$\lambda_B$~пе\-ше\-ход/км$^2$, дви\-жу\-щи\-еся согласно модели 
случайных на\-прав\-ле\-ний (RDM~--- random direction model)~\cite{nain2005properties}. Базовая станция обслуживает 
абонентские устройства (АУ), ас\-со\-ци\-иро\-ван\-ные с~пешеходами. Высоты БС и~АУ равны~$h_A$ и~$h_U$ 
соответственно. Высота пешеходов рав\-на~$h_P$, $h_P\hm>h_U$.


Пусть $\lambda_{U}$~--- интенсивность по\-ступ\-ле\-ния заявок от одного АУ. 
Предполагая, что АУ генерируют запросы на уста\-нов\-ле\-ние сессий независимо, 
процесс по\-ступ\-ле\-ния мож\-но считать пуассоновским с~ин\-тен\-сив\-ностью
$\lambda_A\hm=\lambda_{U}\pi{}d_{E}^2$. Пред\-по\-ла\-га\-ет\-ся рав\-но\-мер\-ное рас\-пре\-де\-ле\-ние 
положения пешеходов в~радиусе обслуживания \mbox{соты}.


Рассматриваются услуги с~высокой ско\-ростью передачи данных, соз\-да\-ющие 
неэластичный, но адап\-ти\-ру\-ющий\-ся к~до\-ступ\-ной ско\-рости трафик.\linebreak Каж\-дая за\-яв\-ка 
запрашивает по\-сто\-ян\-ную ско\-рость~$C_{\max}$. В~случае прерывания обслуживания, 
вызванного блокировкой, механизм адап\-та\-ции снижает ско\-рость до~$C_{\min}$, 
$C_{\min}\hm<C_{\max}$. \mbox{Вследствие} случайного расположения АУ в~зоне обслуживания 
фактический объем ресурса, тре\-бу\-емый для достижения скоростей~$C_{\min}$ и~$C_{\max}$, считается случайным. Предполагается, что время обслуживания запроса 
на уста\-нов\-ле\-ние сес\-сии подчиняется экспоненциальному распределению с~па\-ра\-мет\-ром~$\mu$.

В работе учитывается специфика мил\-ли\-мет\-ро\-во\-го/те\-ра\-гер\-це\-во\-го диапазонов, включая 
модели распространения, антенны и~блокировки, аналогичные рас\-смот\-рен\-ным в~\cite{petrov2017interference, gapeyenko2017temporal, begishev2019estimate}.

\section{Анализ математической модели}


\subsection{Описание модели}
%\label{sect:rsmo_model}


Рассматривается РеСМО с~$N$ приборами, $R$ единицами ресурса и~орбитой ем\-костью 
$M$ заявок. По\-сту\-па\-ющий поток заявок пуассоновский с~ин\-тен\-сив\-ностью~$\lambda$, 
время их обслуживания распределено экспоненциально с~па\-ра\-мет\-ром~$\mu$. Для 
каж\-дой по\-сту\-па\-ющей в~сис\-те\-му заявки требуется прибор и~случайный объем ресурса, 
опре\-де\-ля\-емый в~соответствии с~распределением $\{p_{1,j}\}$, $j\hm=1,2,\dots, R$. 
По\-сту\-па\-ющая заявка принимается на обслуживание, если в~сис\-те\-ме есть свободный 
прибор и~достаточный объем ресурса. Если хотя бы одно из условий не выполняется, 
то заявка либо уходит на орбиту с~вероятностью $\theta$, либо покидает сис\-те\-му с~ве\-ро\-ят\-ностью $1\hm-\theta$. Заявки проводят на орбите экспоненциально 
распределенное время с~ин\-тен\-сив\-ностью~$\alpha$ и~затем пытаются повторно 
по\-сту\-пить на обслуживание с~новыми требованиями к~ресурсу в~соответствии с~распределением $\{p_{2,j}\}$,\linebreak
\vspace*{-12pt}

\pagebreak

\noindent
 $j\hm=1,2,\ldots, R$, что отражает снижение ско\-рости 
переда\-чи при неудачной попытке продолжения сес\-сии. Далее вновь поступающие 
заявки будем называть первичными, а~по\-сту\-па\-ющие с~орбиты~--- по\-втор\-ны\-ми за\-яв\-ками.


Каждая обслуживаемая заявка порождает пуассоновский поток сигналов 
ин\-тен\-сив\-ностью~$\gamma$, при по\-ступ\-ле\-нии которых заявка освобождает ранее 
занятый объем ресурса, формирует новые требования согласно тому же распределению и~пытается занять новый объем. Если свободного ресурса достаточно, то 
продолжается ее обслуживание. В~противном случае она либо уходит на орбиту с~ве\-ро\-ят\-ностью~$\theta$, 
либо покидает сис\-те\-му с~дополнительной ве\-ро\-ят\-ностью $1\hm-\theta$.

Поведение системы описывается трехмерным случайным процессом $X(t)\hm=\left( 
\xi(t), \delta(t), \varphi(t) \right)$, где $\xi (t)$~--- чис\-ло заявок на 
обслуживании; $\delta(t)$~--- общее чис\-ло занятых ими единиц ресурса; 
$\varphi(t)$~--- чис\-ло заявок на орбите. В~этом случае нельзя точ\-но определить 
объем высвободившегося ресурса в~момент ухода заявки, но его можно 
аппроксимировать условным вероятностным распределением~\cite{resmo, tutorial2022}.

Пространство со\-сто\-яний случайного процесса~$X(t)$ име\-ет~вид:

\vspace*{-6pt}

\noindent
\begin{multline*}
*\label{eqn:stateSpace}
     S=\mathop{\bigcup}\limits_{n=0}^N S_n, \
     S_n=\Bigg\{ (n,r,m): m=0,1,\ldots\\
     \ldots ,M, \sum\limits_{i=0}^n 
\sum\limits_{j=0}^r p_{1,j}^{(i )} p_{2,r-j}^{(n-i)}>0 \Bigg\},
\end{multline*}

\vspace*{-3pt}

\noindent
где распределение $\{ p_{s,j}^{(i)} \}$, $j\hm=0,1,\ldots ,R$, $s\hm=\{1,2\}$, есть $i$-крат\-ная 
сверт\-ка распределения $\{p_{s,j}\}$, $j\hm=1,2,\ldots ,R$, а~$p_{s,j}^{(i)}$ 
обозначает ве\-ро\-ят\-ность того, что~$i$~заявок типа~$s$ суммарно занимают~$j$ 
единиц ре\-сурса.

Предполагается, что требования заявок к~ресурсу определяются согласно 
распределению $\{p_{3,j}\}$, $j\hm=1,2,\ldots ,R$, пред\-став\-ля\-юще\-му собой смесь 
распределений~$\{p_ {1,j}\}$ и~$\{p_{2,j}\}$:
\begin{equation*} 
%\label{eqn:mix}
    p_{3,j}=\bar{N_1} p_{1,j} + \bar{N_2} p_{2,j},
\end{equation*}



\noindent
где $\bar{N_1}$ и~$\bar{N_2}$~--- доли первичных и~по\-втор\-ных заявок на 
обслуживании.


Пусть $q(n,r,m)$~--- стационарная ве\-ро\-ят\-ность со\-сто\-яния $(n,r,m)$ процесса~$X(t)$. 
Для рассматриваемой РеСМО мож\-но вывести сис\-те\-му уравнений рав\-но\-ве\-сия:
\begin{multline}
 (\lambda + m\alpha)q(0,0,m)=\mu \!\!\!\! \sum\limits_{j: (1,j,m) \in S_1} \!\!\!\!\!\! q\left(1,j,m\right), \\
  0  \leq m \leq M; 
  \label{eqn:sur}
\end{multline}

\vspace*{-15pt}
\columnbreak
\

\vspace*{-18pt}

\noindent
\begin{multline}
q(n,r,m) \Bigg[ \lambda \Bigg( \sum\limits_{j=0}^{R-r} p_{1,j} + u(M-m) \theta 
\!\! \!\sum\limits_{j=R-r+1}^R \!\!\!\! p_{1,j} \Bigg) +{}\\
{}+n\mu  + n\gamma + 
m\alpha \Bigg( \sum\limits_{j=1}^{R-r} p_{2,j} +(1-\theta) \!\! \sum\limits_{j=R-
r+1}^R \!\!\! p_{2,j} \Bigg) \Bigg] = {} \\
{}  = \lambda \!\! \!\!\! \sum\limits_{j<r, (n-1,j,m) \in S_{n-1}} \!\!\!\!\!\!\!\!\!\!\!\!\! q(n-
1,j,m)p_{1,r-j} + {}\\
{}+u(m) \lambda\, \theta\, q(n,r,m-1) \!\! \sum\limits_{j=R-r+1}^R \!\!
p_{1,j} +{} \\
{}+ (n+1)\mu \!\!\!\!\! \sum\limits_{j>r, (n+1,j,m) \in S_{n+1}} \!\!\!\!\!\!\!\!\!\!\!\!\!\!
q(n+1,j,m) \fr{p_{3,j-r} p_{3,r}^{(n)}}{ p_{3,j}^{(n+1)} } +{}\\
{}+  n\gamma \!\!\!\!\!\!
 \sum\limits_{(n,j,m) \in S_n} \!\!\!\!\!\! q(n,j,m) \!\!\! \sum\limits_{s=\max(0,j-r)}^j\!\! 
 \fr{p_{3,s} p_{3,j-s}^{(n-1)} }{ p_{3,j}^{(n)} }\,\! p_{3,r-j+s} + {} \\
{}+ (n+1)\gamma \Bigg( \sum\limits_{j>r, (n+1,j,m) \in S_{n+1}} \!\!\!\!\!\!\!\!\!\!\!\!\! \left( 
u(m)\,\theta \,q(n+1,j,m-1) +{}\right.\\
\left.{}+(1-\theta)q(n+1,j,m) \right) \fr{ p_{3,j-r}p_{3,r}^{(n)} }{ p_{3,j}^{(n+1)} } 
\sum\limits_{s=R-r+1}^R p_{3,s}  \Bigg) + {} \\
{}+ u(M-m)(m+1)\alpha \Bigg( (1-\theta)q(n,r,m+1) \!\!\!\sum\limits_{j=R-r+1}^R \!\! p_{2,j} + {}\\
{}+ \!\! \sum\limits_{j<r, (n-1,j,m+1) \in S_{n-1}} \!\! \!\!\!\!\!\!\!\!\!\! \!\! \!\! \!\!q(n-1,j,m+1)p_{2,r-j} \Bigg) ,  \\
0<n<N, \enskip 0 \leq m \leq M, \enskip (n,r,m) \in S; 
\end{multline}

\vspace*{-12pt}

\noindent
\begin{multline}
 q(N,r,m) \left[ u(M-m)\theta\lambda +N\mu + N\gamma +{}\right.\\
 \left.{}+ m\alpha(1-\theta) 
\right] = \lambda \!\!\!\!\!\! \sum\limits_{j<r, (N-1,j,m) \in S_{N-1}} \!\!\!\!\!\!\!\!\!\!\!\!\!\!\!\!\!
q(N-1,j,m)p_{1,r-j} + {} \\
{}+u(m) \lambda \theta q(N,r,m-1) + {}\\
{}+ N\gamma \!\!\!\!\!\! \sum\limits_{(N,j,m) \in S_N} 
\!\!\!\!\!\!\!\! q(N,j,m) \!\!\!\sum\limits_{s=\max(0,j-r)}^j \! \!\!\!\!\fr{ p_{3,s} p_{3,j-s}^{(N-1)} }{ 
p_{3,j}^{(N)} } \!p_{3,r-j+s} + {}\\
u(M-m)(m+1)\alpha
\Bigg( (1-\theta)q(N,r,m+1) + {}\\
{}+\!\!\! \sum\limits_{j<r, (N-1,j,m+1) \in 
S_{N-1}} \!\!\!\!\!\!\!\!\!\!\!\!\!\!\!\!\!\! q(N-1,j,m+1)p_{2,r-j} \Bigg),\\
 0 \leq m \leq M, \enskip (N,r,m) \in S_N,
\end{multline}


\noindent
где $u(\cdot)$~--- 
функ\-ция Хевисайда. Для вы\-чис\-ле\-ния 
стационарного распределения сис\-те\-ма уравнений~(1)--(3) решается чис\-лен\-но 
с~использованием условия нормировки. При этом для весов смешанного распределения 
требований к~ресурсу $\{ p_{3 ,j}\}$, $j\hm=1,2,\ldots ,R$, предлагается использовать 
метод\linebreak
\vspace*{-12pt}

\pagebreak
\noindent
прос\-тых итераций с~начальными значениями \mbox{$\bar{N}_1\hm=1$}g, $\bar{N}_2\hm=0$.


\subsection{Вероятностные характеристики} %\label{sect:metrics}

Время пребывания заявки в~сис\-те\-ме может включать множество периодов обслуживания
 и~ожидания. Под периодом обслуживания понимается интервал времени, в~течение 
которого за\-яв\-ка непрерывно обслуживается и~по окончании либо покидает сис\-те\-му, 
либо уходит на орбиту. Аналогично период ожидания~--- это временной интервал, 
который за\-яв\-ка непрерывно проводит на орбите, в~конце которого она либо покидает 
сис\-те\-му, либо вновь принимается на обслуживание.


Начнем с~ве\-ро\-ят\-ности блокировки доступа на обслуживание~$\pi_N$, которая 
пред\-став\-ля\-ет собой ве\-ро\-ят\-ность того, что за\-яв\-ка ни разу не попала на 
обслуживание и~в конечном итоге покинула сис\-те\-му. Для этого заявка либо 
сбрасывается сразу в~момент пер\-во\-го поступления с~ве\-ро\-ят\-ностью~$\pi_1$, либо 
сначала переходит на орбиту c~ве\-ро\-ят\-ностью~$\pi_2$, а~в~конце периода ожидания 
сбрасывается. Вероятности~$\pi_1$ и~$\pi_2$ име\-ют~вид:
\begin{align*} 
%\label{eqn:pi1}
   \pi_1 &= \sum\limits_{r=1}^R \left( q(N,r,M)+ (1-\theta) \sum\limits_{m=0}^{M-1} q(N,r,m) 
\right) + {}\\
   & {}+ \sum\limits_{n=1}^{N-1} \sum\limits_{r=1}^R \left(q(n,r,M)+ (1-\theta) \!\! \sum\limits_{m=0}^{M-1} q(n,r,m) \right) \times{}\\
   &\hspace*{45mm}{}\times \sum\limits_{j=R-r+1}^R p_{1,j}; \\
   \pi_2 &= \theta \sum\limits_{n=1}^{N-1}\sum\limits_{r=1}^R  \sum\limits_{m=0}^{M-1} q(n,r,m) 
\sum\limits_{j=R-r+1}^R p_{1,j} +{}\\
&\hspace*{35mm}{}+
   \theta \sum\limits_{r=1}^R \sum\limits_{m=0}^{M-1} q(N,r,m).
\end{align*}
Отметим, что дополнительная ве\-ро\-ят\-ность $1\hm-\pi_ 1\hm-\pi_ 2$ имеет смысл 
ве\-ро\-ят\-ности того, что первичная заявка принимается на об\-слу\-жи\-ва\-ние.

При попытке попасть на обслуживание с~орбиты (по\-ступ\-ле\-ние повторной заявки) 
заявка может быть сброшена с~ве\-ро\-ят\-ностью~$\pi_3$, может вернуться на орбиту 
с~ве\-ро\-ят\-ностью $\pi_4$ либо быть принята на обслуживание с~ве\-ро\-ят\-ностью $1\hm-\pi_3\hm-\pi_4$. Эти вероятности будем определять как отношения 
со\-от\-вет\-ст\-ву\-ющих 
интенсивностей. Так, сред\-няя ин\-тен\-сив\-ность по\-ступ\-ле\-ния по\-втор\-ных заявок равна 
$\alpha \bar{M}$, где $\bar{M}$~--- сред\-нее чис\-ло заявок на орбите:
$$ 
\bar{M}=\sum\limits_{n=0}^N \sum\limits_{r=0}^R \sum\limits_{m=1}^M m q(n,r,m).
$$


Тогда вероятность $\pi_3$ есть отношение сред\-ней ин\-тен\-сив\-ности~$\nu_3$ сброса 
повторных заявок к~сред\-ней ин\-тен\-сив\-ности их по\-ступ\-ле\-ния: 
$$
\pi_3=\fr{{\nu_3}}{\alpha \bar{M}}\,,
$$ где
\begin{multline*} 
%\label{eqn:pi3}
   \nu_3=(1-\theta) \alpha  \Bigg(  \sum\limits_{n=1}^{N-1}\sum\limits_{r=1}^R \sum\limits_{m=1}^M m q(n,r,m) \!\!\!\!\sum\limits_{j=R-r+1}^R 
\!\!\! \!p_{2,j} +{}\\
   {}+\!\! \sum\limits_{r=1}^R \sum\limits_{m=1}^M\! mq(N,r,m)  \Bigg).
\end{multline*}

Аналогично вероятность~$\pi_4$ рав\-на отношению сред\-ней ин\-тен\-сив\-ности~$\nu_4$ 
возвратов на орбиту по\-втор\-ных заявок к~средней ин\-тен\-сив\-ности их по\-ступ\-ле\-ния. 
Поскольку~$\nu_4$ отличается от~$\nu_3$ только множителем, получаем 
$$
\pi_4=\fr{\theta}{1-\theta}\, \pi_3\,.
$$
Тогда ве\-ро\-ят\-ность блокировки до\-сту\-па на обслуживание~$\pi_N$ име\-ет~вид:
\begin{equation*} 
%\label{eqn:piN}
    \pi_N=\pi_1 + \pi_2 \left( \sum\limits_{k=0}^{\infty} \pi_4^k \right) 
\pi_3=\pi_1+\fr{\pi_2 \pi_3}{1-\pi_4}\,.
\end{equation*}

Перейдем к~анализу ве\-ро\-ят\-ности прерывания~$\pi_O$. По завершении периода 
обслуживания заявка либо прерывается с~ве\-ро\-ят\-ностью~$\pi_5$, либо переходит на 
орбиту с~ве\-ро\-ят\-ностью~$\pi_6$, либо успешно завершается ее обслуживание 
с~ве\-ро\-ят\-ностью $1\hm-\pi_5\hm- \pi_6$.  Ве\-ро\-ят\-ность~$\pi_{5}$ определяется как отношение 
средней ин\-тен\-сив\-ности~$\nu_5$ прерываний заявок к~сред\-ней ин\-тен\-сив\-ности приема 
на обслуживание:
$$
\pi_5=\fr{\nu_5}{\lambda(1-\pi_1-\pi_2)+ \alpha \bar{M} (1- \pi_3\hm-\pi_4)}\,,
$$
 где
\begin{multline*}
   \nu_5=\gamma \sum\limits_{n=2}^N \sum\limits_{r=1}^R n \Bigg( q(n,r,M) + {}\\
   {}+\!
\sum\limits_{m=0}^{M-1} (1-\theta)q(n,r,m) \! \Bigg)\! \sum\limits_{j<r, r-j+s>R} \!\!\!\!\!\!\! p_{3,s} \fr{p_{3,j} p_{3,r-j}^{(n-1)}}{p_r^{(n)}} \,.
\end{multline*}

\begin{figure*}[b] %fig2
 \vspace*{6pt}
\begin{center}
   \mbox{%
\epsfxsize=163mm 
\epsfbox{sop-2.eps}
}
\end{center}
\vspace*{-9pt}
\Caption{Основные характеристики обслуживания абонентов:
(\textit{а})~ве\-ро\-ят\-ность блокировки до\-сту\-па на обслуживание;
(\textit{б})~ве\-ро\-ят\-ность прерывания заявки: серые кривые~--- $\lambda_B\hm= 0{,}01$~бл./м$^2$;
черные кривые~--- $\lambda_B\hm= 0{,}1$~бл./м$^2$; \textit{1}~---  $\theta\hm=0$;
\textit{2}~--- $0{,}5$;
\textit{3}~---  $\theta\hm= 0{,}9$
%\textit{4}~---  $0$;
%\textit{5}~---  $0{,}5$;
%\textit{6}~---  $\theta\hm= 0{,}9$
}
\label{fig:basic}
\end{figure*}


Аналогично $\pi_6$ определяется как отношение сред\-ней ин\-тен\-сив\-ности~$\nu_6$ 
перехода заявок с~обслуживания на орбиту к~сред\-ней ин\-тен\-сив\-ности приема на 
обслуживание:
$$
\pi_6=\fr{\nu_6}{\lambda(1\hm-\pi_1\hm-\pi_2) \hm+ \alpha \bar{M} (1\hm-
\pi_3\hm-\pi_4)}\,,
$$
 где
\begin{multline*}
   \nu_6=\theta \gamma \sum\limits_{n=2}^N \sum\limits_{r=1}^R n \sum\limits_{m=0}^{M-1} q(n,r,m) 
\sum\limits_{j=1}^r \fr{p_{3,j} p_{3,r-j}^{(n-1)}}{p_r^{(n)}} \times{}\\
{}\times \sum\limits_{s=R-r+j+1}^R \!\!\!\! 
p_{3,s}.
\end{multline*}


Период ожидания заявки завершается либо прерыванием с~ве\-ро\-ят\-ностью~$\pi_7$, либо 
возвратом на обслуживание с~ве\-ро\-ят\-ностью $1\hm-\pi_7$. \mbox{Тогда}
\begin{equation*}
 %\label{eqn:pi7}
\pi_7=\sum\limits_{k=0}^{\infty} \pi_4^k \pi_3 = \fr{\pi_3}{1-\pi_4}\,.
\end{equation*}

Заявка прерывается, если она получила хотя бы один период обслуживания 
(с~ве\-ро\-ят\-ностью $1\hm-\pi_N$), затем получает $k \hm\geq 0$ периодов ожидания 
и~обслуживания, а~затем либо прерывается в~конце по\-след\-не\-го периода обслуживания 
с~ве\-ро\-ят\-ностью~$\pi_5$, либо уходит на орбиту и~прерывается в~конце периода 
ожидания с~ве\-ро\-ят\-ностью~$\pi_6 \pi_7$. \mbox{Тогда}
\begin{multline*} 
%\label{eqn:piO}
    \pi_O= (1  - \pi_N) \sum\limits_{k=0}^{\infty} \left( 
\pi_6 (1-\pi_7)\right)^k (\pi_5 + \pi_6 \pi_7) = {}\\
{}=(1-\pi_N) \fr{\pi_5 + \pi_6 
\pi_7}{1-\pi_6(1-\pi_7)}\,.
\end{multline*}

Средняя доля занятых ресурсов вы\-чис\-ля\-ет\-ся по фор\-муле:
\begin{equation*} 
%\label{eqn:u}
U=\fr{1}{R} \sum\limits_{n=1}^N \sum\limits_{r=1}^R \sum\limits_{m=0}^M r q(n,r,m).
\end{equation*}

Важным показателем для систем 5G/6G с~ненадежным обслуживанием служит доля 
ресурсов~$U_W$, пред\-остав\-лен\-ных сессиям, которые в~конечном итоге были прерваны:
\begin{equation*} 
%\label{eqn:uw}
U_W=\fr{\pi_O}{1-\pi_N}\, U.
\end{equation*}


\section{Численные результаты}


% Вероятность блокировки поступающей заявки

В качестве численного исследования рас\-смот\-рим основные характеристики 
обслуживания абонентов~--- ве\-ро\-ят\-ность блокировки до\-сту\-па на обслуживание 
и~ве\-ро\-ят\-ность прерывания заявки,\linebreak пред\-став\-лен\-ные на рис.~\ref{fig:basic}, где 
сред\-нее время пребывания на орбите со\-став\-ля\-ет $1/\alpha\hm=1$~с. \mbox{Методы} расчета 
ин\-тен\-сив\-ности сигналов и~распределения требований первичных и~повторных заявок 
к~ресурсу пред\-став\-ле\-ны в~\cite{begishev2019estimate}. Анализируя \mbox{ве\-ро\-ят\-ность} 
блокировки до\-сту\-па на обслуживание, отметим, что она рас\-тет по мере увеличения 
ин\-тен\-сив\-ности по\-ступ\-ле\-ния заявок. Отметим, что\linebreak увеличение плот\-ности блокеров 
приводит к~уменьшению ве\-ро\-ят\-ности блокировки до\-сту\-па на обслуживание для всех 
значений коэффициента на\-стой\-чи\-вости пользователей. Примечательно, что 
\mbox{на\-стой\-чи\-вость} пользователя поз\-во\-ля\-ет значительно уменьшить ис\-сле\-ду\-емую 
ве\-ро\-ят\-ность. В~част\-ности, при $\lambda_B\hm=0{,}01$~бл./м$^2$ и~$\theta\hm=0$ 
ве\-ро\-ят\-ность блокировки до\-сту\-па на обслуживание рав\-на~0,17 и~уменьшается до~0,1 при 
совершении двух повторных попыток в~сред\-нем.


Рассмотрим теперь ве\-ро\-ят\-ность прерывания сессий (см.\ рис.~\ref{fig:basic},\,\textit{б}). 
Здесь наблюдается рез\-кое\linebreak\vspace*{-12pt}

{ \begin{center}  %fig3
 \vspace*{-4pt}
    \mbox{%
\epsfxsize=78.875mm 
\epsfbox{sop-3.eps}
}

\end{center}

\vspace*{-3pt}

\noindent
{{\figurename~3}\ \ \small{Коэффициент использования ресурса сис\-те\-мы: серые кривые~--- $\lambda_B\hm= 0{,}01$~бл./м$^2$;
черные кривые~--- $\lambda_B\hm= 0{,}1$~бл./м$^2$;
\textit{1}~--- использование ресурсов сис\-те\-мы;
\textit{2}~--- доля потраченного впус\-тую ресурса}}}

\vspace*{12pt}

\noindent
 различие меж\-ду кривыми, со\-от\-вет\-ст\-ву\-ющи\-ми 
$\lambda_B\hm=0{,}1$ и~$0{,}01$~бл./м$^2$, до\-сти\-га\-ющее~0,2--0,3. 
Увеличение плот\-ности блокеров приводит к~рос\-ту ис\-сле\-ду\-емой ве\-ро\-ят\-ности 
и~снижению ве\-ро\-ят\-ности\linebreak блокировки до\-сту\-па на обслуживание (см.\ рис.~\ref{fig:basic},\,\textit{а}). 
В~целом при рас\-смат\-ри\-ва\-емых значениях ин\-тен\-сив\-ности 
по\-ступ\-ле\-ния большая часть заявок, принятых к~обслуживанию, в~итоге теряется. 
На\-стой\-чи\-вость пользователя значительно повышает ве\-ро\-ят\-ность успеш\-но\-го 
обслуживания. В~част\-ности, при $\lambda\hm=0{,}1$~з./с, $\lambda_B\hm=0{,}1$~бл./м$^2$ 
на\-стой\-чи\-вость снижает ве\-ро\-ят\-ность прерывания с~0,4 до~0,1. 
{\looseness=1

}

% Коэффициент использования ресурсов системы

Наконец, рассмотрим коэффициент использования ресурса сис\-те\-мы и~долю ресурса, 
по\-тра\-чен\-но\-го сис\-те\-мой впус\-тую, пред\-став\-лен\-ные на рис.~3.\linebreak Анализируя 
пред\-став\-лен\-ные данные,  мож\-но заметить, что на\-стой\-чи\-вость пользователей 
положительно влияет на обе мет\-ри\-ки, поз\-во\-ляя использовать больше сис\-тем\-но\-го 
ресурса и~делать это\linebreak более эффективно. В~част\-ности, при $\lambda\hm=0{,}1$~бл./м$^2$ 
и~$\theta\hm=0$ около~40\% ресурса из занятых $55$\% тратится впустую. При 
$\theta\hm=0{,}9$ сис\-те\-ма не только использует~80\% ресурса, но и~только~15\% из 
этого чис\-ла тратится на заявки, которые не будут успеш\-но об\-слу\-жены. 





\section{Заключение}
%\label{conclus}

В статье предложена модель процесса обслуживания пользователей на БС с~учетом 
не\-на\-деж\-но\-го характера мил\-ли\-мет\-ро\-вых и~субтерагерцевых сред передачи сис\-тем 
5G/6G, которые подвержены прерываниям со\-еди\-не\-ния, вызванным блокировкой, и~настойчивому поведению абонентов. Предложенная модель описана в~виде РеСМО 
с~орбитой и~поз\-во\-ля\-ет получить основные характеристики обслуживания абонентов 
и~про\-из\-во\-ди\-тель\-ности сис\-те\-мы~--- вероятности блокировки до\-сту\-па на обслуживание 
и~прерывания, а~так\-же коэффициент использования ресурса сис\-темы.
{\looseness=1

}




{\small\frenchspacing
 {\baselineskip=12pt
 %\addcontentsline{toc}{section}{References}
 \begin{thebibliography}{9}
\bibitem{begishev2021joint} %1
\Au{Begishev V., Sopin~E., Moltchanov~D., Kovalchukov~R.,
  Samuylov~A., Andreev~S., Koucheryavy~Y., Samouylov~K.} Joint use of guard 
capacity and multiconnectivity for improved session continuity in millimeter-wave 5G NR systems~// 
IEEE T. Veh. Technol., 2021. Vol.~70. 
Iss.~3. P.~2657--2672. doi: 10.1109/TVT.2021.3061906.

\bibitem{vlaskina2022} %2
\Au{Власкина А.\,С., Бурцева~С.\,А., Ко\-чет\-ко\-ва~И.\,А., Шор\-гин~С.\,Я.}
Управляемая сис\-те\-ма массового обслуживания с~элас\-тич\-ным трафиком и~сигналами для 
анализа нарезки ресурсов в~сети радиодоступа~// Информатика и~её применения, 
2022. Т.~16. Вып.~3. С.~90--96. doi: 10.14357/19922264220312.

\bibitem{kovalchukov2019accurate} %3
\Au{Kovalchukov R., Moltchanov~D., Gai\-da\-ma\-ka~Y., Bob\-ri\-ko\-va~E.} An accurate 
approximation of resource request distributions in millimeter wave 3GPP new 
radio systems~// Internet of things, smart spaces, and next generation networks 
and systems~/ Eds. O.~Galinina, S.~Andreev, S.~Balandin, Y.~Koucheryavy.~--- 
Lecture notes in computer science ser.~--- Cham: Springer, 2019. Vol.~11660. 
P.~572--585. doi: 10.1007/978-3-030-30859-9\_50.

\bibitem{nain2005properties} %4
\Au{Nain P., Towsley~D., Liu~B., Liu~Z.} Properties of random direction 
models~//  24th Annual Joint Conference of the IEEE Computer and 
Communications Societies Proceedings.~--- Piscataway, NJ, USA: IEEE, 2005. Vol.~3. P.~1897--1907. 
doi: 10.1109/INFCOM.2005.1498468. 

\bibitem{petrov2017interference} %5
\Au{Petrov V., Komarov~M., Mol\-tcha\-nov~D., Jornet~J., Kou\-che\-rya\-vy~Y.}
\newblock Interference and SINR in millimeter wave and terahertz communication 
systems with blocking and directional antennas~// IEEE T. Wirel. 
Commun., 2017. Vol.~16. Iss.~3. P.~1791--1808. doi: 
10.1109/TWC. 2017.2654351.

\bibitem{gapeyenko2017temporal} %6
\Au{Gapeyenko M., Samuylov~A., Ge\-ra\-si\-men\-ko~M., Mol\-tcha\-nov~D., Singh~S., 
Akdeniz~M., Aryafar~E., Himayat~N., And\-re\-ev~S., Kou\-che\-rya\-vy~E.} On the temporal 
effects of mobile blockers in urban millimeter-wave cellular scenarios~// IEEE 
T. Veh. Technol., 2017. Vol.~66. Iss.~11. P.~10124--10138. 
doi: 10.1109/TVT.2017.2754543.

\bibitem{begishev2019estimate} %7
\Au{Бегишев В.\,О., Сопин~Э.\,С., Мол\-ча\-нов~Д.\,А., Са\-муй\-лов~А.\,К., 
Гай\-да\-ма\-ка~Ю.\,В.,  Са\-муй\-лов~К. Е.} Оценка эф\-фек\-тив\-ности механизма резервирования 
полосы про\-пус\-ка\-ния для технологии mmWave в~сетях связи пятого поколения~// 
Ин\-фор\-ма\-ци\-он\-но-управ\-ля\-ющие сис\-те\-мы, 2019. №\,5(102). C.~51--63. doi: 
10.31799/1684-8853-2019-5-51-63.

\pagebreak

\bibitem{resmo} %8
\Au{Горбунова А.\,В., Наумов В.\,А., Гай\-да\-ма\-ка Ю.\,В., Са\-муй\-лов К.\,Е.} 
Ресурсные
сис\-те\-мы массового обслуживания как модели беспроводных сис\-тем связи~// 
Информатика и~её применения, 2018. Т.~12. Вып.~3. С.~48--55. doi: 
10.14357/19922264180307.

\bibitem{tutorial2022} %9
\Au{Moltchanov D., Sopin E., Be\-gi\-shev~V., Sa\-muy\-lov~A., Kou\-che\-rya\-vy~Y., 
Sa\-mouy\-lov~K.} A~tutorial on mathematical modeling of 5G/6G millimeter wave and 
terahertz cellular systems~// IEEE Commun. Surv. Tut., 2022. 
Vol.~24. Iss.~2. P.~1072--1116. doi: 10.1109/ COMST.2022.3156207.


\end{thebibliography}

 }
 }

\end{multicols}

\vspace*{-6pt}

\hfill{\small\textit{Поступила в~редакцию 08.02.23}}

\vspace*{8pt}

%\pagebreak

%\newpage

%\vspace*{-28pt}

\hrule

\vspace*{2pt}

\hrule

%\vspace*{-2pt}

\def\tit{MODELING INSISTENT USER BEHAVIOR IN~5G NEW RADIO NETWORKS WITH~RATE ADAPTATION AND~BLOCKAGE\\[-4pt]}


\def\titkol{Modeling insistent user behavior in~5G New Radio networks with~rate adaptation and~blockage}


\def\aut{E.\,S.~Sopin$^{1,2}$, A.\,R.~Maslov$^1$, V.\,S.~Shorgin$^2$, and~V.\,O.~Begishev$^1$}

\def\autkol{E.\,S.~Sopin, A.\,R.~Maslov, V.\,S.~Shorgin, and~V.\,O.~Begishev}

\titel{\tit}{\aut}{\autkol}{\titkol}

\vspace*{-13pt}


\noindent 
$^{1}$RUDN University, 6 Miklukho-Maklaya Str., Moscow 117198, Russian Federation

\noindent
$^{2}$Federal Research Center ``Computer Science and Control'' of the Russian Academy of Sciences, 44-2~Vavilov\linebreak
$\hphantom{^1}$Str., Moscow 119333, Russian Federation


\def\leftfootline{\small{\textbf{\thepage}
\hfill INFORMATIKA I EE PRIMENENIYA~--- INFORMATICS AND
APPLICATIONS\ \ \ 2023\ \ \ volume~17\ \ \ issue\ 3}
}%
 \def\rightfootline{\small{INFORMATIKA I EE PRIMENENIYA~---
INFORMATICS AND APPLICATIONS\ \ \ 2023\ \ \ volume~17\ \ \ issue\ 3
\hfill \textbf{\thepage}}}

\vspace*{2pt}




\Abste{Millimeter wave 5G New Radio access technology and future 6G terahertz systems are designed for rate-demanding 
multimedia applications. Such applications are adaptive to the network transmission rate. The unreliable nature of 5G/6G communications networks, subject 
to loss of connection due to dynamic blockage of radio propagation paths, can cause impatient user behavior leading to retries to continue service.
 For such systems, the authors propose a~service model with impatient behavior and service interruption based on 
 a~resource queuing system with an orbit. The authors derive the new and ongoing session loss probabilities as well as 
 characterize the efficiency of resource usage. Numerical results show that user persistence reduces the probabilities under 
 consideration: performing an average of two retries reduces both indicators by 20\%--70\% while with an average of~10~attempts, the gain can reach 200\%--300\%. 
 Persistence not only increases the use of system resources 
by~20\%--40\% but also allows them to be used efficiently reducing the percentage of wasted resources by two to three times.}

\KWE{5G; New Radio; resource queuing system; callbacks; propagation blocking; service interruption}




\DOI{10.14357/19922264230304}{ENSHKV}

\vspace*{-12pt}

\Ack

\vspace*{-4pt}

\noindent
The research was funded by the Russian Science Foundation, project No.\,22-79-10128.

\vspace*{2pt}

  \begin{multicols}{2}

\renewcommand{\bibname}{\protect\rmfamily References}
%\renewcommand{\bibname}{\large\protect\rm References}

{\small\frenchspacing
 {\baselineskip=10.8pt
 \addcontentsline{toc}{section}{References}
 \begin{thebibliography}{9} 
\bibitem{begishev2021joint-1}
\Aue{Begishev, V., E.~Sopin, D.~Moltchanov, R.~Ko\-val\-chu\-kov, A.~Sa\-muy\-lov, S.~And\-re\-ev, Y.~Kou\-che\-ry\-avy, and K.~Sa\-mouy\-lov.}
 2021. Joint use of guard capacity and multiconnectivity for improved session continuity in millimeter-wave 5G NR systems. \textit{IEEE T.~Veh. Technol.}
  70(3):~2657--2672. doi: 10.1109/TVT.2021.3061906.

\bibitem{vlaskina2022-1}
\Aue{Vlaskina, A.\,S., S.\,A.~Burtseva, I.\,A.~Ko\-chet\-ko\-va, and S.\,Ya.~Shor\-gin.} 2022. 
Up\-rav\-lya\-emaya sis\-te\-ma mas\-so\-vo\-go ob\-slu\-zhi\-va\-niya s~elas\-tich\-nym tra\-fi\-kom 
i~sig\-na\-la\-mi dlya ana\-li\-za na\-rez\-ki re\-sur\-sov v~se\-ti ra\-dio\-do\-stu\-pa 
[Controllable queuing system with elastic traffic and signals for analyzing network slicing]. \textit{Informatika i~ee Primeneniya~--- Inform. Appl.} 
16(3):~90--96. doi: 10.14357/19922264220312.

\bibitem{kovalchukov2019accurate-1}
\Aue{Kovalchukov, R., D.~Moltchanov, Y.~Gai\-da\-ma\-ka, and E.~Bob\-ri\-ko\-va.}
 2019. An accurate approximation of resource request distributions in millimeter wave 3GPP new radio systems. 
 \textit{Internet of things, smart spaces, and next generation networks and systems}. Eds. O.~Galinina, S.~Andreev, S.~Balandin, and Y.~Koucheryavy. 
Lecture notes in computer science ser. Cham: Springer. 11660:572--585. doi: 10.1007/978-3-030-30859-9\_50.

\bibitem{nain2005properties-1} %4
\Aue{Nain, P., D.~Towsley, B.~Liu, and Z.~Liu.}
 2005. Properties of random direction models. \textit{24th Annual Joint Conference of the IEEE Computer and Communications Societies Proceedings}. 
 Piscataway, NJ: IEEE. 3:1897--1907. doi: 10.1109/\linebreak INFCOM.2005.1498468.
 
 

\bibitem{petrov2017interference-1} %5
\Aue{Petrov, V., M.~Komarov, D.~Mol\-tcha\-nov, J.~Jornet, and Y.~Kou\-che\-rya\-vy.}
 2017. Interference and SINR in millimeter wave and terahertz communication systems with blocking and directional antennas. 
 \textit{IEEE T. Wirel. Commun.} 16:1791--1808. doi: 10.1109/TWC.2017.2654351.

\bibitem{gapeyenko2017temporal-1} %6
\Aue{Gapeyenko, M., A.~Samuylov, M.~Gerasimenko, D.~Mol\-tcha\-nov, S.~Singh, M.~Riza, E.~Aryafar, N.~Himayat, S.~And\-re\-ev, and Y.~Kou\-che\-rya\-vy.}
 2017. On the temporal effects of mobile blockers in urban millimeter-wave cellular scenarios. \textit{IEEE T. Veh. Technol.} 66(11):10124--10138. 
 doi: 10.1109/TVT.2017.2754543.
 
 \bibitem{begishev2019estimate-1} %7
\Aue{Begishev, V.\,O., E.\,S.~Sopin, D.\,A.~Moltchanov, A.\,K.~Sa\-muy\-lov, Yu.\,V.~Gaydamaka, and K.\,E.~Sa\-mouy\-lov.}
 2019. Otsen\-ka ef\-fek\-tiv\-nosti me\-kha\-niz\-ma re\-zer\-vi\-ro\-va\-niya po\-lo\-sy 
 pro\-pus\-ka\-niya dlya tekh\-no\-lo\-gii mmWave v~se\-tyakh svya\-zi pya\-to\-go po\-ko\-le\-niya 
 [Performance evaluation of bandwidth reservation for mmWave in 5G NR systems]. \textit{Informatsionno-upravlyayushchie sis\-te\-my} 
 [Information and Control Systems] 5(102):51--63. doi: 10.31799/1684-8853-2019-5-\mbox{51-63}.

\bibitem{resmo-1} %8
\Aue{Gorbunova, A.\,V., V.\,A.~Naumov, Yu.\,V.~Gay\-da\-ma\-ka, and K.\,E.~Sa\-muy\-lov.} 2018. 
Re\-surs\-nye sis\-te\-my mas\-so\-vo\-go ob\-slu\-zhi\-va\-niya kak mo\-de\-li bes\-pro\-vod\-nykh sis\-tem svya\-zi 
[Resource queuing systems as models of wireless communication systems]. 
 \textit{Informatika i~ee Primeneniya~--- Inform. Appl.} 12(3):48--55. doi: 10.14357/19922264180307.

\bibitem{tutorial2022-1} %9
\Aue{Moltchanov, D., E.~Sopin, V.~Begishev, A.~Samuylov, Y.~Kou\-che\-rya\-vy, and K.~Samouylov.}
 2022. A~tutorial on mathematical modeling of 5G/6G millimeter wave and terahertz cellular systems. \textit{IEEE Commun. Surv. Tut.} 24(2):1072--1116. 
 doi: 10.1109/COMST.2022.3156207.


\end{thebibliography}

 }
 }

\end{multicols}

\vspace*{-9pt}

\hfill{\small\textit{Received February 8, 2023}} 

\Contr


\noindent
\textbf{Sopin Eduard S.} (b.\ 1987)~--- 
Candidate of Science (PhD) in physics and mathematics, associate professor, Department of Applied Probability and Informatics, 
RUDN University, 6~Miklukho-Maklaya Str., Moscow 117198, Russian Federation; senior scientist, Institute of Informatics Problems, 
Federal Research Center ``Computer Science and Control'' of the Russian Academy of Sciences, 44-2~Vavilov Str., Moscow 119333, Russian Federation; 
\mbox{sopin-es@rudn.ru}

\vspace*{3pt}

\noindent
\textbf{Maslov Alexander R.} (b.\ 1998)~--- PhD student, Department of Applied Probability and Informatics, 
RUDN University, 6~Miklukho-Maklaya Str., Moscow 117198, Russian Federation; 
\mbox{maslov-ar@rudn.ru}

\vspace*{3pt}

\noindent
\textbf{Shorgin Vsevolod S.} (b.\ 1978)~--- Candidate of Science (PhD) in technology, senior scientist, Institute of Informatics Problems,
 Federal Research Center ``Computer Science and Control'' of the Russian Academy of Sciences, 44-2~Vavilov Str., Moscow 119333, Russian Federation; 
 \mbox{vshorgin@ipiran.ru}
 
 \vspace*{3pt}

\noindent
\textbf{Begishev Vyacheslav O.} (b.\ 1988)~--- 
Candidate of Science (PhD) in physics and mathematics, associate professor, Department of Applied Probability and Informatics, 
RUDN University, 6~Miklukho-Maklaya Str., Moscow 117198, Russian Federation; \mbox{begishevu-vo@rudn.ru}



\label{end\stat}

\renewcommand{\bibname}{\protect\rm Литература} 