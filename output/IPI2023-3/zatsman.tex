
\def\stat{zatsman}

\def\tit{ТРАНСФОРМАЦИЯ ИЕРАРХИИ АКОФФА В~НАУЧНОЙ~ПАРАДИГМЕ ИНФОРМАТИКИ$^*$}

\def\titkol{Трансформация иерархии Акоффа в~научной парадигме  информатики}

\def\aut{И.\,М.~Зацман$^1$}

\def\autkol{И.\,М.~Зацман}

\titel{\tit}{\aut}{\autkol}{\titkol}

\index{Зацман И.\,М.}
\index{Zatsman I.\,M.}


{\renewcommand{\thefootnote}{\fnsymbol{footnote}} \footnotetext[1]
{Работа выполнялась с~использованием инфраструктуры Центра коллективного пользования <<Высокопроизводительные вы\-чис\-ле\-ния и~большие данные>> 
(ЦКП <<Информатика>>) ФИЦ ИУ РАН (г.~Москва).}}


\renewcommand{\thefootnote}{\arabic{footnote}}
\footnotetext[1]{Федеральный исследовательский центр <<Информатика и~управление>> Российской академии наук;
\mbox{izatsman@yandex.ru}}

\vspace*{-13pt}



  \Abst{Рассматривается иерархия DIKW (data, information, knowledge, wisdom~--- данные, 
информация, знания, мудрость), которая была опуб\-ли\-ко\-ва\-на в~1989~г.\ Расселом Акоффом. 
В~ней муд\-рость находится на вершине иерархии, затем следуют знание, информация 
и,~в~самом низу, данные. Первоначально предполагалось, что иерархию DIKW мож\-но будет 
использовать для описания отношений между ее четырьмя компонентами. Однако проб\-ле\-му 
описания взаимных преобразований двух со\-сед\-них компонентов, особенно для знания 
и~информации, оказалось весьма слож\-но решить в~рамках иерархии DIKW. Слож\-ность ее 
решения за\-клю\-ча\-ет\-ся в~том, что иерархия DIKW под\-ра\-зу\-ме\-ва\-ет генерацию знания 
в~результате процесса фильт\-ра\-ции со\-сед\-ней с~ней информации, но средства реализации 
этого процесса не были определены Акоффом. Не удается также описать смысловую 
интерпретацию данных, так как они непосредственно не примыкают к~знанию в~иерархии 
DIKW, которая подразумевает наличие отношений только меж\-ду соседними компонентами. 
Цель статьи со\-сто\-ит в~трансформации иерархии DIKW в~рамках на\-уч\-ной па\-ра\-диг\-мы 
информатики, осно\-ван\-ной на средовом делении ее предметной об\-ласти на 
ментальную, информационную, циф\-ро\-вую и~ряд других сред. В~то время как Акофф 
использовал принцип вертикального размещения компонентов иерархии, в~данной статье 
вместо этого предлагается соотнести ис\-поль\-зу\-емые в~информатике интерфейсы и~знаковые 
сис\-те\-мы с~отношениями между тремя компонентами иерархии: данными, информацией 
и~знанием. Если использовать принцип вертикального размещения не компонентов, а~сред 
предметной об\-ласти информатики, то тогда мож\-но предложить подход к~решению проб\-ле\-мы 
описания взаимных преобразований трех компонентов иерархии, со\-по\-ста\-вив их 
с~интерфейсами информатики и~знаковыми сис\-те\-ма\-ми. Такое со\-по\-став\-ле\-ние даст воз\-мож\-ность 
увидеть те пары компонентов, для которых интерфейсы в~на\-сто\-ящее время не 
формализованы, не имеют компьютерной реализации и~выполняются экспертами. В~статье 
приводится пример технологии извлечения знания, со\-че\-та\-ющей автоматические 
и~экспертные (неформализованные) технологические этапы.}
   
\KW{научная парадигма информатики; данные; информация; знание; иерархия DIKW; 
технологии извлечения знания}

\DOI{10.14357/19922264230315}{UMVRRV}
  
\vspace*{-6pt}


\vskip 10pt plus 9pt minus 6pt

\thispagestyle{headings}

\begin{multicols}{2}

\label{st\stat}

\section{Введение}

\vspace*{-3pt}

  Графически иерархию DIKW, которая со\-сто\-ит из четырех компонентов 
(данные, информация, знания, муд\-рость), час\-то пред\-став\-ля\-ют в~форме 
пирамиды (рис.~1~[1]). Такое визуальное пред\-став\-ле\-ние впол\-не соответствует 
первоначальному ее описанию: <<Муд\-рость находится на вершине иерархии$\ldots$ 
и,~в~самом низу, данные. Каж\-дое из пе\-ре\-чис\-лен\-ных понятий [кроме данных] 
содержит в~себе нижестоящие$\ldots$>>~[2]. Утверж\-де\-ние о~том, что 
\textit{пе\-ре\-чис\-лен\-ные понятия содержат в~себе ни\-же\-сто\-ящие}, Акофф 
комментирует сле\-ду\-ющим образом. Данные он трактует как наборы 
символов, которые \textit{характеризуют свойства объектов и~событий, 
а~также их окружение}. Эти наборы формируются в~процессе наблюдения 
или зондирования. Информация определяется как \textit{результат анализа 
данных}. Знание рас\-смат\-ри\-ва\-ет\-ся им в~кон\-текс\-те 
управ\-ле\-ния организационными сис\-те\-ма\-ми в~сфере экономики:\linebreak\vspace*{-12pt}

{ \begin{center}  %fig1
 %\vspace*{6pt}
    \mbox{%
\epsfxsize=79.014mm 
\epsfbox{zac-1.eps}
}

\end{center}

\vspace*{-3pt}

\noindent
{{\figurename~1}\ \ \small{Иерархия DIKW (данные, информация, знание, муд\-рость)
}}}

\vspace*{9pt}

\noindent

 это то, что делает 
воз\-мож\-ным \textit{преобразование информации в~инструкции}, и~это делает 
воз\-мож\-ным управ\-ле\-ние такими сис\-те\-ма\-ми~[2].



  Процесс анализа данных и~получение информации на их основе, как правило, 
предполагают их ментальное понимание человеком, формирование концептов 
(со\-став\-ля\-ющих знания) и~их пред\-став\-ле\-ние в~виде текс\-та, диаграмм, графиков 
и~т.\,д. Таким образом, до получения информации в~результате анализа данных 
сначала формируется знание, но оно в~иерархии не имеет общей границы 
с~дан\-ными.
  
  Согласно Дэвиду Вайнбергеру, переход от информации к~знаниям в~иерархии 
DIKW оказался еще более проб\-ле\-ма\-ти\-чен, чем переход от данных 
к~информации. Ее недостатки он описал так: <<знание~--- это не прос\-то 
результат фильт\-ра\-ции [компонентов]$\ldots$ Оно~--- результат гораздо более 
слож\-но\-го процесса, который является \textit{социальным}, 
\textit{це\-ле\-на\-прав\-лен\-ным}, \textit{контекстно} и~\textit{культурно 
обуслов\-лен\-ным}$\ldots$ Наиболее важ\-ным в~этом отношении является то, что 
там, где решения слож\-ные и~знание [необходимое для решения] получить 
труд\-но, \textit{оно не определяется информацией}, поскольку именно процесс 
по\-зна\-ния в~пер\-вую очередь определяет, какая информация необходима и~как ее 
следует использовать$\ldots$ Представление о~том, что знание (а~тем более 
муд\-рость) является результатом применения фильт\-ров на каж\-дом уров\-не, 
рисует не\-вер\-ную картину (курсив мой.~--- И.\,З.)>>~[3].
  
  В данной статье предлагается изменить принцип по\-стро\-ения иерархии 
и~соотнести используемые в~информатике интерфейсы и~знаковые сис\-те\-мы 
с~отношениями между тремя компонентами иерархии: данными, информацией и~знанием. 
Акофф использовал принцип вертикального размещения 
компонентов иерархии, которому свойственны недостатки, 
проанализированные, в~част\-ности, в~работах~\cite{1-zatsm, 3-zatsm}. 
  
  Цель статьи состоит в~трансформации иерархии DIKW в~рамках научной 
парадигмы информатики, основанной на средов$\acute{\mbox{о}}$м делении 
ее предметной об\-ласти на ментальную, информационную, циф\-ро\-вую и~ряд 
других сред. Если использовать принцип вертикального размещения не 
компонентов, а~сред предметной об\-ласти информатики, то тогда мож\-но 
пред\-ло\-жить подход к~решению проб\-ле\-мы описания взаимных 
преобразований трех компонентов иерархии (кроме муд\-рости), со\-по\-ста\-вив их 
с~интерфейсами информатики и~знаковыми сис\-те\-ма\-ми. Такое 
со\-по\-став\-ле\-ние даст воз\-мож\-ность увидеть те пары компонентов, для которых 
интерфейсы в~на\-сто\-ящее время не формализованы и~не имеют компьютерной 
реализации (т.\,е.\ преобразования таких компонентов выполняются 
экс\-пер\-та\-ми).

\vspace*{-3pt}
  
\section{Среды и~объекты предметной области информатики}

\vspace*{-3pt}
  
  В работе~\cite{4-zatsm} было начато описание научной парадигмы 
информатики, основанной на средов$\acute{\mbox{о}}$м 
 делении ее 
предметной об\-ласти. Было описано основание для по\-стро\-ения верх\-не\-го уров\-ня 
классификации сущностей ее предметной об\-ласти (по\-зи\-ци\-о\-ни\-ру\-емых как 
феномены раз\-ной природы) и~даны определения сле\-ду\-ющих пяти ее сред, 
каж\-дая из которых включает сущности одной и~той же при\-роды:
\begin{enumerate}[(1)]
\item \textit{ментальная среда}~--- это со\-во\-куп\-ность когнитивных 
феноменов, фор\-ми\-ру\-емых в~процессах познания, происходящих в~сознании 
людей (далее~--- концепты как со\-став\-ля\-ющие знания человека);
\item \textit{информационная среда}~--- это со\-во\-куп\-ность сенсорно 
вос\-при\-ни\-ма\-емых феноменов, находящихся вне со\-зна\-ния;
\item \textit{цифровая среда}~--- это со\-во\-куп\-ность компьютерных кодов;
\item \textit{нейросреда}~--- это электрические потенциалы и~магнитные 
поля, генерируемые мозгом, которые используются в~информационных 
технологиях (ИТ) управ\-ле\-ния роботизированной рукой~\cite{5-zatsm} 
и~в~других ИТ, при\-ме\-ня\-ющих интерфейсы <<мозг--ком\-пью\-тер>>;
\item \textit{ДНК-среда}~--- это со\-во\-куп\-ность цепочек РНК и~ДНК.
\end{enumerate}

  В соответствии с~пе\-ре\-чис\-лен\-ны\-ми средами верх\-ний уровень классификации 
сущностей предметной об\-ласти информатики включает как минимум пять 
классов, каж\-дый из которых содержит объекты одной среды: ментальной, 
информационной, циф\-ро\-вой, нейросреды или ДНК-сре\-ды. При этом  
с~рос\-том разнообразия природы сущностей верх\-ний уровень классификации 
может пополняться новыми классами, если при проектировании ИТ 
обнаружатся сущности, которые по своей природе не относятся ни к~одной из 
ранее уже выделенных сред~[6, 7].

\setcounter{figure}{1}
\begin{figure*} %fig2
\vspace*{1pt}
\begin{center}
   \mbox{%
\epsfxsize=110mm 
\epsfbox{zac-2.eps}
}
\end{center}
\vspace*{-10pt}
\Caption{Средовая версия иерархия DIKW как результат трансформации}
\vspace*{-3pt}
\end{figure*}

\begin{figure*}[b] %fig3
\vspace*{-6pt}
\begin{center}
   \mbox{%
\epsfxsize=110mm 
\epsfbox{zac-3.eps}
}
\end{center}
\vspace*{-10pt}
\Caption{Средовая версия иерархии DIKW: знаковые сис\-те\-мы и~кодовые таб\-лицы}
\end{figure*}

\vspace*{-3pt}


\section{Трансформация иерархии DIKW}

\vspace*{-3pt}

  Согласно используемой парадигме информатики и~определению верх\-не\-го 
уров\-ня классификации сущностей ее предметной об\-ласти, \textit{ментальная 
среда} охватывает концепты как со\-став\-ля\-ющие знания. \textit{Информационная 
среда} содержит как минимум сен\-сор\-но вос\-при\-ни\-ма\-емые данные и~знаковую 
информацию. При их компьютерном кодировании получаем соответственно 
циф\-ро\-вые данные и~циф\-ро\-вую информацию, которые в~па\-ра\-диг\-ме 
информатики позиционируются как две принципиально раз\-ные сущности 
\textit{цифровой среды}. Нейросреда и~ДНК-сре\-да в~трансформации иерархии 
\mbox{DIKW} в~этой статье не рас\-смат\-ри\-ва\-ются.
  
  Если использовать принцип вертикального размещения не компонентов, 
а~сред пред\-мет\-ной об\-ласти информатики, то получим новый вариант 
иерархии \mbox{DIKW} (рис.~2), который назовем \textit{средов$\acute{\mbox{о}}$й 
версией} иерархии \mbox{DIKW}. Она включает три среды и~как минимум восемь сущностей: 
\begin{enumerate}[(1)]
\item мудрость, знание, ментальные образы данных  и~концепты ментальной \mbox{среды};\\[-15pt]
\item знаковую информацию и~сенсорно вос\-при\-ни\-ма\-емые данные 
информационной \mbox{среды};
\item цифровые данные и~циф\-ро\-вую информацию циф\-ро\-вой \mbox{среды}.
\end{enumerate}


  Сопоставим известные интерфейсы информатики, ре\-а\-ли\-зу\-емые  
с~по\-мощью кодовых таб\-лиц, и~знаковые сис\-темы с~двумя следующими границами 
средов$\acute{\mbox{о}}$й версии иерархии \mbox{DIKW}:
  \begin{enumerate}[(1)]
\item между информационной и~циф\-ро\-вой сре\-дами;
\item между информационной и~ментальной сре\-дами.
\end{enumerate}

\begin{figure*}[b] %fig4
\vspace*{9pt}
\begin{center}
   \mbox{%
\epsfxsize=163.149mm 
\epsfbox{zac-4.eps}
}
\end{center}
\vspace*{-7pt}
\Caption{Модель ИТ извлечения знания}
\end{figure*}

  В результате сопоставления можно увидеть (рис.~3), что первой границе 
в~информатике со\-от\-вет\-ст\-ву\-ют кодовые таб\-ли\-цы, если данные могут быть 
пред\-став\-ле\-ны в~символьной фор\-ме. Что касается второй границы, то для 
пред\-став\-ле\-ния концептов знания широко используются знаковые сис\-темы.
  
  Наименее исследованными остаются отношения между сенсорно 
вос\-при\-ни\-ма\-емы\-ми данными и~их ментальными образами. Однако для 
проектирования в~информатике одного из ключевых этапов 
ИТ извлечения нового знания из данных необходимо тем или иным 
способом реализовать эти отношения. Отметим, что их исследование 
и~описание относятся к~когнитивным наукам, а~не к~предметной об\-ласти 
ин\-фор\-ма\-тики.
  
  В современных ИТ визуальной аналитики~[8, 9] 
и~извлечения нового знания из данных~[10, 11] для их смыс\-ло\-вой 
\textit{интерпретации} привлекаются, как правило, эксперты, об\-ла\-да\-ющие 
опытом в~со\-от\-вет\-ст\-ву\-ющей предметной об\-ласти.
  
\section{Модель технологии извлечения знания}

\vspace*{-15pt}

  Для применения средов$\acute{\mbox{о}}$й версии иерархии \mbox{DIKW} 
в~процессе проектирования ИТ извлечения нового 
знания из данных и/или текс\-то\-вой информации необходимо разделить 
индивидуальные знания экспертов и~их коллективные знания~[12--14]. Вариант 
средов$\acute{\mbox{о}}$й версии (см.\ рис.~3), дополненный разделением 
на индивидуальное и~коллективное знание экспертов, по\-дроб\-но описан 
в~работе~[15]. В~этой же работе определены основные процессы и~стадии 
извлечения нового знания, а~так\-же предложена модель технологии извлечения 
знания на основе спиральной модели генерации знания, которую создал 
Икуджиро Нонака~\cite{12-zatsm}.
  
  Основные отличия предлагаемой модели от ранее созданной спиральной 
модели со\-сто\-ят в~сле\-ду\-ющем. Во-пер\-вых, пред\-ла\-га\-емая модель включает 
в~себя потенциальные источники нового знания, которые долж\-ны 
соответствовать цели его генерации. Во-вто\-рых, она охватывает сущности 
трех сред разной природы: ментальной, информационной и~циф\-ро\-вой, 
а~спиральная модель подразумевает использование только ментальной 
и~информационной сред. В-треть\-их, она включает восемь процессов 
извлечения знания (рис.~4), а~спиральная модель~--- четыре процесса 
(интернализация, социализация, экстернализация и~комбинирование, которые 
по\-дроб\-но описаны в~работах~[12--14]). Пред\-ла\-га\-емая модель кроме этих 
четырех процессов спиральной модели включает еще четыре: \textit{поиск} 
потенциальных источников нового знания, их \textit{визуализация}, 
\textit{интерпретация} экспертами и~\textit{кодирование} индивидуальных 
и~коллективных концентов знания экс\-пер\-тов.
{ %\looseness=-1

}
  
  В предлагаемой модели процессы могут со\-сто\-ять из одной, четырех, пяти 
или шести стадий, которые описаны в~работе~[15]. Верх\-няя часть этой модели 
на рис.~4 практически пол\-ностью совпадает со спиральной моделью. Ниж\-няя 
ее часть начинается с~поиска потенциальных источников нового знания в~базе 
данных, со\-от\-вет\-ст\-ву\-ющих цели его извлечения. Например, в~задачах 
медицинской информатики это могут собранные клинические данные 
о~течении некоторого ис\-сле\-ду\-емо\-го заболевания.
  
  Найденный источник визуализируется для его по\-сле\-ду\-ющей интерпретации 
экспертом. На рис.~4 изображен случай, когда в~результате интерпретации 
сенсорно воспринимаемых данных (информационная среда) сформированы 
концепты нового знания (ментальная среда) и~они пред\-став\-ле\-ны как 
индивидуальная информация (информационная среда) этого эксперта. 
По\-сле\-ду\-ющие процессы (комбинирование, интернализация, социализация 
и~экстернализация) соответствуют спиральной мо\-дели.
{\looseness=-1

}
  
  Если вход и~выход процесса принадлежат одной среде, то на рис.~4 она 
размещается именно в~ней (см.\ процессы социализации, комбинирования 
и~интерпретации). Если вход и~выход процесса принадлежат двум разным 
средам, то она размещается на границе этих сред (см.\ процессы 
интернализации, экстернализации, визуализации и~кодирования).
  
  Принципиальное отличие пред\-ла\-га\-емой модели от спиральной со\-сто\-ит 
в~итерационном пополнении баз индивидуальных и~коллективных знаний. 
Пред\-став\-ле\-ние ее процессов и~их стадий в~средов$\acute{\mbox{о}}$й версии 
иерархии \mbox{DIKW} детально описано в~работе~[15].
{\looseness=-1

}

\vspace*{-6pt}
  
\section{Заключение}

\vspace*{-3pt}

  Рассмотренный вариант трансформации иерархии \mbox{DIKW} выполнен в~рамках 
научной па\-ра\-диг\-мы информатики, в~которой ее предметная об\-ласть делится на 
ментальную, информационную, циф\-ро\-вую, нейро- и~ДНК-сре\-ду. 
В~\mbox{статье} рас\-смат\-ри\-ва\-лись толь\-ко первые три среды, поэтому 
интерфейс <<мозг--ком\-пью\-тер>> отсутствует в~рас\-смот\-рен\-ном варианте 
сре\-до\-в$\acute{\mbox{о}}$й версии иерархии \mbox{DIKW}. Описание ее варианта 
с~позиционированием интерфейса <<мозг--ком\-пью\-тер>> заслуживает 
отдельной \mbox{статьи}.
  
  Переход к~принципу вертикального размещения сред поз\-во\-лил со\-по\-ставить 
отдельные известные интерфейсы и~перечень отношений между компонентами 
иерархии \mbox{DIKW}. Такое со\-по\-став\-ле\-ние дало воз\-мож\-ность увидеть 
мес\-то\-по\-ло\-же\-ние отношений между сенсорно вос\-при\-ни\-ма\-емы\-ми данными 
(информационная среда) и~их ментальными образами (ментальная среда), 
которые не формализованы и~не имеют компьютерной реализации (они 
обозначены знаком~<<?>> на рис.~3).
  
  Согласно Харелу, реализация таких отношений относится к~задачам 
когнитивной слож\-ности. Основной вопрос при решении по\-доб\-ных задач 
со\-сто\-ит в~том, чтобы описать отношения меж\-ду ментальными образами 
и~знанием человека так, чтобы его пред\-став\-ле\-ние под\-да\-ва\-лось 
алгоритмической обработке, применению сис\-тем и~средств 
информатики~\cite[с.~402]{16-zatsm}.

\vspace*{-3pt}
  
{\small\frenchspacing
 { %\baselineskip=12pt
 %\addcontentsline{toc}{section}{References}
 \begin{thebibliography}{99}
\bibitem{1-zatsm}
\Au{Rowley J.} The wisdom hierarchy: Representations of the DIKW hierarchy~// J.~Inf. Sci., 2007. Vol.~33. 
No.\,2. P.~163--180. doi: 10.1177/0165551506070706.
\bibitem{2-zatsm}
\Au{Ackoff R.} From data to wisdom~// J.~Appl. Systems Analysis, 1989. Vol.~16. P.~3--9.
\bibitem{3-zatsm}
\Au{Weinberger D.} The problem with the data--information--knowledge--wisdom hierarchy~// 
Harvard Bus. Rev., 2010. {\sf https://hbr.org/2010/02/data-is-to-info-as-info-is-not}.
\bibitem{4-zatsm}
\Au{Зацман И.\,М.} Тео\-ре\-ти\-че\-ские осно\-ва\-ния компьютерного образования: среды 
пред\-мет\-ной об\-ласти информатики как основание классификации ее объектов~// Сис\-те\-мы 
и~средства информатики, 2022. Т.~32. №\,4. С.~77--89. doi: 10.14357/08696527220408.
\bibitem{5-zatsm}
\Au{Зацман И.\,М.} Интерфейсы треть\-его порядка в~информатике~// Информатика и~её 
применения, 2019. Т.~13. Вып.~3. С.~82--89. doi: 10.14357/19922264190312.

\bibitem{7-zatsm}
\Au{Зацман И.\,М.} Таб\-ли\-ца интерфейсов информатики как ин\-фор\-ма\-ци\-он\-но-ком\-пью\-тер\-ной 
науки~// На\-уч\-но-тех\-ни\-че\-ская информация. Сер.~1: Организация и~методика 
информационной работы, 2014. №\,11. С.~1--15.

\bibitem{6-zatsm}
\Au{Зацман И.\,М.} О~научной па\-ра\-диг\-ме информатики: верх\-ний уровень классификации 
объ\-ек\-тов ее предметной об\-ласти~// Информатика и~её применения, 2022. Т.~16. Вып.~4. 
С.~108--114. doi: 10.14357/ 19922264220411.
\bibitem{8-zatsm}
\Au{Chen M., Ebert~D., Hagen~H., Laramee~R., Van Liere~R., Ma~K.-L., Ribarsky~W., 
Scheuermann~G., Silver~D.} Data, information, and knowledge in visualization~// IEEE Comput. 
Graph., 2009. Vol.~29. No.\,1. P.~12--19. doi: 10.1109/MCG.2009.6.
\bibitem{9-zatsm}
\Au{Federico P., Wagner~M., Rind~A., Amor-Amoros~A., Miksch~S., Aigner~W.} The role of 
explicit knowledge: A conceptual model of knowledge-assisted visual analytics~// 
Conference on Visual Analytics Science and Technology Proceedings.~--- New York, NY, USA: 
IEEE, 2017. P.~92--103. doi: 10.1109/VAST.2017.8585498.
\bibitem{10-zatsm}
\Au{Zatsman I.} A~model of goal-oriented knowledge discovery based on human--computer 
symbiosis~// 16th Forum (International) on Knowledge Asset Dynamics Proceedings.~--- Matera, 
Italy: Arts for Business Institute, 2021. P.~297--312.
\bibitem{11-zatsm}
\Au{Zatsman I., Khakimova~A.} New knowledge discovery for creating terminological profiles of 
diseases~// 22nd European Conference on Knowledge Management Proceedings.~--- Reading, 
U.K.: Academic Publishing International Ltd., 2021. P.~837--846. doi: 10.34190/EKM.21.041.
\bibitem{12-zatsm}
\Au{Nonaka I}. The knowledge-creating company~// Harvard Bus. Rev., 1991. Vol.~69. No.\,6. 
P.~96--104.
\bibitem{13-zatsm}
\Au{Nonaka I.} A~dynamic theory of organizational knowledge creation~// Organ. Sci., 1994. 
Vol.~5. No.\,1. P.~14--37. doi: 10.1287/orsc.5.1.14.
\bibitem{14-zatsm}
\Au{Нонака И., Такеучи~Х.} Компания~--- создатель знания~/ Пер. c~англ.~---  
М.: Олимп-биз\-нес, 2003. 384~с. (\Au{Nonaka~I., Takeuchi~H.} The knowledge-creating 
company.~--- Oxford, NY, USA: Oxford University Press, 1995. 284~p.)
\bibitem{15-zatsm}
\Au{Zatsman I.} Digital spiral model of knowledge creation and encoding its dynamics~// 18th 
Forum (International) on Knowledge Asset Dynamics Proceedings.~--- Matera, Italy: Arts for 
Business Institute, 2023. P.~581--596.
\bibitem{16-zatsm}
\Au{Harel D.} Algorithmics~--- the spirit of computing.~--- Reading, MA, USA: Addison-Wesley, 
1987. 514~p.
\end{thebibliography}

 }
 }

\end{multicols}

\vspace*{-8pt}

\hfill{\small\textit{Поступила в~редакцию 21.06.23}}

\vspace*{6pt}

%\pagebreak

%\newpage

%\vspace*{-28pt}

\hrule

\vspace*{2pt}

\hrule

\vspace*{-1pt}

\def\tit{TRANSFORMATION OF~THE~ACKOFF'S HIERARCHY IN~THE~SCIENTIFIC 
PARADIGM OF~INFORMATICS}


\def\titkol{Transformation of~the~Ackoff's hierarchy in~the~scientific 
paradigm of~informatics}


\def\aut{I.\,M.~Zatsman}

\def\autkol{I.\,M.~Zatsman}

\titel{\tit}{\aut}{\autkol}{\titkol}

\vspace*{-13pt}


\noindent
Federal Research Center ``Computer Science and Control'' of the Russian Academy 
of Sciences, 44-2~Vavilov Str., Moscow 119333, Russian Federation


\def\leftfootline{\small{\textbf{\thepage}
\hfill INFORMATIKA I EE PRIMENENIYA~--- INFORMATICS AND
APPLICATIONS\ \ \ 2023\ \ \ volume~17\ \ \ issue\ 3}
}%
 \def\rightfootline{\small{INFORMATIKA I EE PRIMENENIYA~---
INFORMATICS AND APPLICATIONS\ \ \ 2023\ \ \ volume~17\ \ \ issue\ 3
\hfill \textbf{\thepage}}}

\vspace*{2pt}




\Abste{The DIKW (data, information, knowledge, and wisdom) hierarchy, which was published in 1989 by Russell Ackoff, is considered. In it, 
wisdom is at the top of the hierarchy followed by knowledge, information, and, at the very bottom, data. It was originally intended 
that the DIKW hierarchy could be used to describe the relationships between its four components. 
However, the problem of describing the mutual transformations of two neighboring components, especially for knowledge and information, 
turned out to be very difficult to solve within the DIKW hierarchy. The complexity of its solution lies in the fact that the DIKW 
hierarchy implies the generation of knowledge as an outcome of the process of filtering neighboring information 
but the means of implementing this process were not defined by Ackoff. It is also impossible to describe the semantic 
interpretation of data, since they are not directly adjacent to knowledge in the DIKW hierarchy which implies the existence of 
relations between neighboring components only. The aim of the paper is to transform the DIKW hierarchy within the framework of 
the scientific paradigm of informatics which is based on the medium division of its subject domain into mental, 
informational, digital, and a~number of other media. While Ackoff used the principle of vertical placement of the components, 
this paper instead proposes to correlate the interfaces and sign systems used in informatics with the relationships between 
the three components of the hierarchy: data, information, and knowledge. If one uses the principle of vertical placement not of 
the components but of the media of the subject domain of informatics, then it can be proposed an approach to solving the problem 
of describing the mutual transformations of the three components of the hierarchy comparing them with informatics interfaces 
and sign systems. Such a~comparison will make it possible to see those pairs of components for which the interfaces are not currently formalized,
 do not have a~computer implementation, and are performed by experts. The paper provides the 
example of a~knowledge discovery technology that combines automatic and expert (nonformalized) technological stages.}

\KWE{scientific paradigm of informatics; data; information; knowledge; DIKW hierarchy; knowledge discovery technology}



\DOI{10.14357/19922264230315}{UMVRRV}

\vspace*{-18pt}

\Ack

\vspace*{-3pt}

\noindent
The research was carried out using the infrastructure of the Shared Research Facilities ``High Performance 
Computing and Big Data'' (CKP ``Informatics'') of FRC CSC RAS (Moscow).
  

%\vspace*{6pt}

  \begin{multicols}{2}

\renewcommand{\bibname}{\protect\rmfamily References}
%\renewcommand{\bibname}{\large\protect\rm References}

{\small\frenchspacing
 {%\baselineskip=10.8pt
 \addcontentsline{toc}{section}{References}
 \begin{thebibliography}{99} 
\bibitem{1-zatsm-1}
\Aue{Rowley, J.} 2007. The wisdom hierarchy: Representations of the DIKW hierarchy. \textit{J.~Inf. Sci.} 
33(2):163--180. doi: 10.1177/0165551506070706.
\bibitem{2-zatsm-1}
\Aue{Ackoff, R.} 1989. From data to wisdom. \textit{J.~Appl. Systems Analysis} 16(1):3--9.
\bibitem{3-zatsm-1}
\Aue{Weinberger, D.} 2010. The problem with the data-information-knowledge-wisdom hierarchy. 
\textit{Harvard Bus. Rev.} Available at: {\sf https://hbr.org/2010/02/data-is-to-info-as-info-is-not} (accessed 
July~11, 2023). 

\bibitem{4-zatsm-1}
\Aue{Zatsman, I.\,M.} 2022. Teo\-re\-ti\-che\-skie osno\-va\-niya komp'yu\-ter\-no\-go ob\-ra\-zo\-va\-niya: 
sre\-dy pred\-met\-noy ob\-lasti in\-for\-ma\-ti\-ki kak osno\-va\-nie klas\-si\-fi\-ka\-tsii ee ob''\-ek\-tov 
[Theoretical foundations of digital education: Subject domain media of informatics as the base of its objects' 
classification]. \textit{Sistemy i~Sredstva Informatiki~--- Systems and Means of Informatics} 32(4):77--89. 
doi: 10.14357/ 08696527220408.
\bibitem{5-zatsm-1}
\Aue{Zatsman, I.\,M.} 2019. Inter\-fey\-sy tret'\-ego po\-ryad\-ka v~in\-for\-ma\-ti\-ke [Third-order interfaces 
in informatics]. \textit{Informatika i~ee Primeneniya~--- Inform. Appl.} 13(3):82--89. doi: 
10.14357/19922264190312.

\bibitem{7-zatsm-1}
\Aue{Zatsman, I.\,M.} 2014. A~table of interfaces of informatics as computer and information science. 
\textit{Scientific Technical Information Processing} 41(4):230--243. doi: 10.3103/S014768821404008X.

\bibitem{6-zatsm-1}
\Aue{Zatsman, I.} 2022. O~na\-uch\-noy pa\-ra\-dig\-me in\-for\-ma\-ti\-ki: verkh\-niy uro\-ven'  
klas\-si\-fi\-ka\-tsii ob''\-ek\-tov ee pred\-met\-noy ob\-lasti [On the scientific paradigm of informatics: The 
classification high level of its objects]. \textit{Informatika i~ee Primeneniya~--- Inform. Appl.} 16(4):73--79. 
doi: 10.14357/ 19922264220411.

\bibitem{8-zatsm-1}
\Aue{Chen, M., D.~Ebert, H.~Hagen, R.~Laramee, R.~van Liere, K.-L.~Ma, W.~Ribarsky, 
G.~Scheuermann, and D.~Silver.} 2009. Data, information, and knowledge in visualization. \textit{IEEE 
Comput. Graph.} 29(1):12--19. doi: 10.1109/MCG.2009.6.


\bibitem{9-zatsm-1}
\Aue{Federico, P., M.~Wagner, A.~Rind, A.~Amor-Amoros, S.~Miksch, and W.~Aigner.} 2017. The role of 
explicit knowledge: A~conceptual model of knowledge-assisted visual analytics. \textit{Conference on 
Visual Analytics Science and Technology Proceedings}. New York, NY: IEEE. 92--103. doi: 
10.1109/VAST.2017.8585498.
\bibitem{10-zatsm-1}
\Aue{Zatsman, I.} 2021. A~model of goal-oriented knowledge discovery based on human--computer 
symbiosis. \textit{16th Forum (International) on Knowledge Asset Dynamics Proceedings}. Matera, Italy: Arts for Business Institute. 297--312.
\bibitem{11-zatsm-1}
\Aue{Zatsman, I., and A.~Khakimova.} 2021. New knowledge discovery for creating terminological profiles 
of diseases. \textit{22nd European Conference on Knowledge Management Proceedings}. Reading, U.K.: 
Academic Publishing International Ltd. 837--846. doi: 10.34190/EKM.21.041.
\bibitem{12-zatsm-1}
\Aue{Nonaka, I.} 1991. The knowledge-creating company. \textit{Harvard Bus. Rev.} 69(6):96--104.
\bibitem{13-zatsm-1}
\Aue{Nonaka, I.} 1994. A~dynamic theory of organizational knowledge creation. \textit{Organ. Sci.} 
5(1):14--37. doi: 10.1287/ orsc.5.1.14.
\bibitem{14-zatsm-1}
\Aue{Nonaka, I., and H.~Takeuchi.} 1995. \textit{The knowledge-creating company}. Oxford, NY: Oxford 
University Press. 284~p.
\bibitem{15-zatsm-1}
\Aue{Zatsman, I.} 2023. Digital spiral model of knowledge creation and encoding its dynamics. \textit{18th 
Forum (International) on Knowledge Asset Dynamics Proceedings}. Matera, Italy: Arts for Business Institute. 581--596.
\bibitem{16-zatsm-1}
\Aue{Harel, D.} 1987. \textit{Algorithmics~--- the spirit of computing}. Reading, MA: Addison-Wesley. 514~p.

\end{thebibliography}

 }
 }

\end{multicols}

\vspace*{-6pt}

\hfill{\small\textit{Received June 21, 2023}} 

\vspace*{-12pt}
      
\Contrl

\vspace*{-2pt}

\noindent
\textbf{Zatsman Igor M.} (b.\ 1952)~--- Doctor of Science in technology, head of department, Institute of 
Informatics Problems, Federal Research Center ``Computer Science and Control'' of the Russian Academy of 
Sciences, 44-2~Vavilov Str., Moscow 119333, Russian Federation; \mbox{izatsman@yandex.ru}



\label{end\stat}

\renewcommand{\bibname}{\protect\rm Литература} 