\def\stat{torshin}

\def\tit{О ФОРМИРОВАНИИ МНОЖЕСТВ ПРЕЦЕДЕНТОВ НА~ОСНОВЕ ТАБЛИЦ 
РАЗНОРОДНЫХ ПРИЗНАКОВЫХ ОПИСАНИЙ МЕТОДАМИ ТОПОЛОГИЧЕСКОЙ 
ТЕОРИИ АНАЛИЗА ДАННЫХ$^*$}

\def\titkol{О формировании множеств прецедентов на~основе таблиц 
разнородных признаковых описаний} % методами топологической  теории анализа данных}

\def\aut{И.\,Ю.~Торшин$^1$}

\def\autkol{И.\,Ю.~Торшин}

\titel{\tit}{\aut}{\autkol}{\titkol}

\index{Торшин И.\,Ю.}
\index{Torshin I.\,Yu.}


{\renewcommand{\thefootnote}{\fnsymbol{footnote}} \footnotetext[1]
{Работа выполнена при поддержке гранта РНФ (проект 23-21-00154) с~использованием инфраструктуры 
Центра коллективного пользования <<Высокопроизводительные вычисления и~большие данные>> (ЦКП 
<<Информатика>>) ФИЦ ИУ РАН (г.~Москва).}}


\renewcommand{\thefootnote}{\arabic{footnote}}
\footnotetext[1]{Федеральный исследовательский центр <<Информатика и~управление>> Российской академии наук, 
\mbox{tiy135@yahoo.com}}

\vspace*{-8pt}



\Abst{Факторизация вкладов различных переменных при анализе раз\-но\-род\-ных 
признаковых описаний~--- насущная задача интеллектуального анализа слож\-ных данных. 
В~работе пред\-ло\-же\-но развитие решеточного формализма топологической тео\-рии анализа 
данных, в~рам\-ках которого получены новые способы по\-рож\-де\-ния па\-ра\-мет\-ри\-че\-ских 
оценок и~мет\-рик на решетках, образованных над топологиями мно\-жеств объектов. 
Формализм был апро\-би\-ро\-ван на задаче формирования множеств прецедентов для 
проведения хемомикробиомного анализа. Тогда как по\-рож\-де\-ние множества исходных 
информаций на основе регрессионных коэффициентов и~раз\-ности значений материала 
обуче\-ния соответствовало крайне низ\-кой обобщающей спо\-соб\-ности на\-стра\-и\-ва\-емых 
алгоритмов (коэффициент корреляции на контроле~--- $0{,}32\hm\pm 0{,}20$), 
использование пред\-ла\-га\-емых оценок для по\-рож\-де\-ния множеств прецедентов в~задачах 
хе\-мо\-мик\-ро\-био\-ми\-ки позволило существенно повысить обоб\-ща\-ющую спо\-соб\-ность 
со\-от\-вет\-ст\-ву\-ющих алгоритмов (коэффициент корреляции на контроле~--- $0{,}79\hm\pm 
0{,}21$).}

\KW{топологический анализ данных; тео\-рия решеток; па\-ра\-мет\-ри\-за\-ция решеточных 
тер\-мов; мик\-ро\-би\-ом человека; фармакоинформатика, ал\-геб\-ра\-и\-че\-ский подход 
Ю.\,И.~Жу\-рав\-лёва}

 \DOI{10.14357/19922264230301}{AQEUYO}
  
%\vspace*{-2pt}


\vskip 10pt plus 9pt minus 6pt

\thispagestyle{headings}

\begin{multicols}{2}

\label{st\stat}

\section{Введение}

     В биомедицинских исследованиях объектом служит формализованное 
описание со\-сто\-яния пациента, вклю\-ча\-ющее булевы (диагнозы, прием 
лекарств и~др.), чис\-ло\-вые (лабораторные анализы) и~категорные (показатели 
демографии и~др.)\ переменные, графы (формулы лекарств), временн$\acute{\mbox{ы}}$е ряды 
и~изображения (результаты обследования пациента аппаратными методами). 
При формализации таких задач важ\-но выделить независимые вклады 
пе\-ре\-мен\-ных-при\-зна\-ков в~таргетную переменную (отклик) таким образом, 
чтобы получить\linebreak высокое качество работы алгоритмов 
рас\-по\-зна\-ва\-ния/клас\-си\-фи\-ка\-ции. Топологический анализ данных, раз\-ви\-ва\-емый 
в~рамках ал\-геб\-ра\-и\-че\-ско\-го \mbox{подхода} к~распознаванию научной школы 
Ю.\,И.~Журавлёва и~К.\,В.~Рудакова~[1, 2], поз\-во\-ля\-ет сис\-те\-ма\-ти\-че\-ски 
исследовать воз\-мож\-ные решения этой за\-дачи. 
     
     В настоящей работе данный подход апробирован на задаче 
формирования множества прецедентов для проведения хе\-мо\-мик\-ро\-би\-ом\-но\-го 
анализа лекарств~[3]. В~этой при\-клад\-ной задаче (оцен\-ка вли\-яния лекарств на 
микробиоту) факт использования лекарства, как правило, описан булевым 
при\-зна\-ком, а~от\-кли\-ком служит уровень той или иной бактерии мик\-ро\-био\-ма 
(чис\-ло\-вая переменная).\linebreak Для выделения независимых вкла\-дов булевых 
переменных в~чис\-ло\-вые таргетные переменные предложен сис\-тем\-ный 
подход к~по\-рож\-де\-нию па\-ра\-мет\-ри\-зо\-ван\-ных оценок на решетке (решеточных 
термов) и~со\-от\-вет\-ст\-ву\-ющих мет\-рик. Такой подход пред\-став\-ля\-ет собой 
тео\-ре\-ти\-че\-ское обобщение <<рас\-щеп\-ле\-ния>> эмпирической функции 
распределения булевым признаком~\cite{2-tor} и~тесно связан 
с~по\-рож\-де\-ни\-ем мет\-ри\-че\-ских функций рас\-сто\-яния и~с~проб\-ле\-ма\-ти\-кой так 
на\-зы\-ва\-емых <<оценок ин\-фор\-ма\-тив\-ности>> (точ\-нее, весовых функций 
при\-зна\-ков), ис\-поль\-зу\-емых в~комбинаторной тео\-рии раз\-ре\-ши\-мости~[4].

\vspace*{-6pt}
     
\section{Основные понятия}

     Основы формализма изложены в~\cite{2-tor, 5-tor}. Задано множество 
ис\-ход\-ных описаний объектов $\mathbf{X}\hm= \{x_1, \ldots , x_{N_0}\}$, 
$\mathbf{X}\hm\subseteq {S}$, множества значений признаков $I_k\hm= \{ 
\lambda_{k_1}, \lambda_{k_2}, \ldots , \lambda_{k_b}, \ldots , 
\lambda_{k_{\vert I_k\vert -1}}, \Delta\}$,\linebreak $b\hm= 1,\ldots , \vert I_k\vert$, 
функции $\Gamma_k: S \hm\to I_k$, $k\hm=1,\ldots ,n \hm+l$\linebreak 
 ($n$~--- чис\-ло признаков; $l$~--- чис\-ло таргетных (про\-гно\-зи\-ру\-емых) 
переменных; $\Delta$~--- не\-опре\-де\-лен\-ность). Определено пространство 
до\-пус\-ти\-мых при\-зна\-ко\-вых описаний объектов 
$J_{\mathrm{ob}}\hm\subseteq I_i\times I_f$ ($I_i\hm\subseteq I_1\times 
\cdots\times I_n$, $I_f\hm\subseteq I_{n+1} \times\cdots\times I_{n+1}$), 
функции $D: S\hm\to J_{\mathrm{ob}}$, $D(x_\alpha)\hm=(\Gamma_1(x_\alpha)\times \cdots$\linebreak $\cdots\times 
\Gamma_k (x_\alpha)\times\cdots\times \Gamma_{n+1}(x_\alpha))_\Delta$ 
и~$\varphi(\mathbf{X}) \hm= \{ D(x_\alpha)\vert x_\alpha \hm\in \mathbf{X}\}$. 
Принимается, что множества~$\mathbf{X}$ и~$Q\hm= \varphi(\mathbf{X})$, 
$\vert Q\vert \hm= N$, \textit{регулярны}, т.\,е.\ $N_0\hm= N$ и~$\exists\,  
D^{-1}: \forall x\hm\in \mathbf{X}$, $x\hm= D^{-1}(D(x))$~\cite{5-tor}. 
Множество $U(\mathbf{X})\hm=\{\Gamma_k^{-1}(\lambda_{k_b})\}$, 
образованное функциями пол\-ных прообразов значений 
при\-зна\-ков~$\lambda_{k_b}$, рас\-смат\-ри\-ва\-ет\-ся как пред\-ба\-за 
\textit{топологии} $T(\mathbf{X})\hm=\{\varnothing, I, a\cup b, a\cap b: a,b 
\hm\in U(\mathbf{X})\}$, где $I\hm=\{\mathbf{X}\}$. 
Топологии~$T(\mathbf{X})$ соответствует \textit{решетка} 
$L(T(\mathbf{X}))\hm=\{a\vee b, a\wedge b: a,b\hm\in T(\mathbf{X})\}$. При 
ре\-гу\-ляр\-ности~$\mathbf{X}$ и~$Q$ ре\-шет\-ка $L(T(\mathbf{X}))$~--- 
булева~\cite{2-tor}.

\smallskip

\noindent
\textbf{Теорема~1.}\ \textit{Произвольной цепи $\langle u_1, \ldots, 
u_m\rangle$ решетки $L(T(\mathbf{X}))$ мож\-но сопоставить чис\-ло\-вой 
при\-знак с~множеством значений} $I_t\hm= (\lambda_{t_1}, \ldots, 
\lambda_{t_b}, \ldots , \lambda_m)$, $\lambda_{t_{i-1}}\hm\leq \lambda_{t_1} 
\hm\leq \lambda_{t_{i+1}}$. 
     
     \smallskip
     
     Справедливость тео\-ре\-мы следует из того, что линейный порядок любой 
конечной цепи $\langle u_1\hm \supseteq u_2\cdots \supseteq u_i\cdots \supseteq 
u_m\rangle$ изоморфен линейному порядку на конечном множестве чисел 
$\{\lambda_{t_i}\}$, $\lambda_{t_{i-1}} \hm\leq \lambda_{t_i}$, $i\hm=  
2, \ldots , m$.
     
     \smallskip
     
     \noindent
     \textbf{Следствие~1.}\ \textit{Любая цепь~$A_t$ в~$L(T(\mathbf{X}))$ 
пред\-ста\-ви\-ма в~виде $A_t\hm= \langle u(\lambda_{t_1}), \ldots , 
u(\lambda_{t_i}),\ldots , u(\lambda_{t_m})\rangle$, $u(\lambda_{t_i})\hm= 
\mathop{\bigcup}_{\beta=1}^i \Gamma^{-1}_t(\lambda_{t_\beta})$, где 
$I_t\hm= (\lambda_{t_1}, \ldots , \lambda_m)$ монотонна}.
     
     \smallskip
     
     \noindent
     \textbf{Следствие~2.} \textit{При дополнении наблюда\-емой 
об\-ласти~$I_t$ значений $t$-го чис\-ло\-во\-го при\-зна\-ка 
не\-опре\-де\-лен\-ностью~$\Delta$ и~принятии со\-гла\-ше\-ния 
$\Gamma_t(\varnothing)\hm=\Delta$ со\-от\-вет\-ст\-ву\-ющая цепь~$A_t$~--- 
максимальная цепь решетки}~$L(T(\mathbf{X}))$. 
     
     \smallskip
     
     \noindent
\textbf{Следствие~3.} \textit{Для произвольной цепи} $t$\textit{-го чис\-ло\-во\-го 
при\-зна\-ка определена эмпирическая функция распределения}  
$\mathrm{cdf}\,(\lambda, A_k(\mathbf{X})) \hm= \vert u(\lambda_{k_b})\vert /N \big\vert \lambda_{k_{b-1}} \hm\leq \lambda\hm\leq \lambda_{k_b}$. 

\section{Решеточные термы и~операции над~ними}

     В топологическом анализе данных булевой решетке   
со\-по\-став\-ля\-ет\-ся \textit{мет\-ри\-че\-ское пространство значений при\-зна\-ков} 
$M_L(L(T(\mathbf{X})), \rho_L)$ с~мет\-ри\-кой $\rho_L: L^2\hm\to 
R^+$~\cite{2-tor}. Метрики~$\rho_L$ могут определяться на основании 
формализма оценок (см.\ ниже) или же на основе известных подходов 
(расстояния Танимото, Амана, Твер\-ско\-го, Со\-ка\-ла--Сни\-са,\linebreak  
Гоу\-эра--Ле\-жанд\-ра и~др.)~\cite{6-tor}. При заданной~$\rho_L$ 
\textit{расстояние} $\rho_A: \mathbf{A}(\mathbf{X})^2\hm\to R^+$ 
\textit{меж\-ду цепями} $a\hm= \langle a_1, \ldots , a_i,\ldots\rangle$ и~$b\hm= 
\langle b_1, \ldots , b_j,\ldots\rangle$ определено так, что существует 
однозначное соответствие элементов цепей~$a$ и~$b$ (например, 
$\rho_A(a,b)\hm= \min ( \sum_{i=1,\vert a\vert} \rho_L(a_i, \argmin_{b_j\in b} 
\rho_L(a_i, b_j)),$\linebreak $\sum_{i=1,\vert b\vert} \rho_L(b_j, \argmin_{a_i\in a} \rho_L 
(b_j, a_i))$. При заданных~$\rho_L$, $\rho_A$, $A_k(\mathbf{X})$ 
и~множества до\-пус\-ти\-мых цепей $A(\mathbf{X})_{1,n}$ (содержит цепи 
с~длинами от~1 до~$n$) иско\-мый ($\varepsilon$-кор\-рект\-ный) алгоритм 
со\-от\-вет\-ст\-ву\-ет решению задачи комбинаторной оптимизации~\cite{5-tor}: $$
    % aa= 
    \argmin\limits_{a\in \mathbf{A}(\mathbf{X})_{1,n}} \rho_A \left( 
A_k(\mathbf{X}), a\right)\vert A(\mathbf{X})_{1,n}\hm\subset 
\mathbf{A}(\mathbf{X}).
     $$
При порождении мет\-ри\-че\-ских пространств над произвольной решеткой   
вводится понятие решеточного тер\-ма или \textit{оценки} $v: L\hm\to R^+$, 
для которой выполнено \textit{условие оценки} (\textbf{yO}: $\forall_L a,b: 
v[a]\hm+v[b]\hm= v[a\wedge b]\hm+ v[a\vee b]$) и~\textit{условие 
изотонности} (\textbf{уИ}: $\forall_L a,b: a\hm\supseteq b \hm \Rightarrow 
v[a]\hm\geq v[b]$). Изо\-тон\-ность оцен\-ки $v[\,]$ важ\-на потому, что поз\-во\-ля\-ет 
определить мет\-ри\-ку $\rho(x,y) \hm= v[x\vee y] \hm- v[x\wedge y]$~\cite{2-tor}.

\smallskip

\noindent
\textbf{Теорема~2.}\ \textit{Линейная комбинация $v[\,]\hm= \sum \omega_i 
v_i[\,]$, $i\hm=1, \ldots, m,$ произвольного чис\-ла изотонных  
оценок $v_i[\,]$~--- изотонная оцен\-ка при $v[\,]\hm\geq0$.}

\smallskip
 
     Теорема доказывается через линейные преобразований условий уО$_i$ 
и~уИ$_i$ для $m$ оценок~$v_i[\,]$.
     
     
     \smallskip
     
     \noindent
\textbf{Следствие~1.}\ \textit{Сумма произвольного чис\-ла (изотонных) 
оценок~--- оценка}.

     \smallskip
     
     \noindent
\textbf{Следствие~2.} \textit{Раз\-ность (изотонных) оценок становится 
оцен\-кой только при условии положительной опре\-де\-лен\-ности}.

     \smallskip
     
     \noindent
\textbf{Следствие~3.} \textit{Пусть индекс~$i$ в~выражении 
для~$v[\,]$ изменяется от $i\hm=0$, $v_0\hm\equiv 1$. Тогда мож\-но 
по\-до\-брать такое значение~$\omega_0$, что} $v[\,]\hm\geq 0$. 
    
\smallskip

При произвольном нелинейном преобразовании $f: R^+\hm\to R^+$, 
при\-ме\-ня\-емом к~(изотонной) оцен\-ке~$v[\,]$, вы\-пол\-ни\-мость уО неочевидна. 
Если $v[\,]$ изотонна, то $f(v[\,])$ может быть изотонна только при условии 
мо\-но\-тон\-ности~$f$ (сигмоида, степенная функция и~др.). Для оценки 
вы\-пол\-ни\-мости уО/уИ $f(v[\,])$ на подмножестве решетки~$L$ воз\-мож\-но 
применение аналитических или комбинаторных подходов. Например, лег\-ко 
показать, что для $f(x)\hm= x^\alpha$ при $\alpha\hm=2$ или~0,5 выражение 
$f(v[\,])$ служит оцен\-кой только на элементах произвольной цепи решетки~$L(T(\mathbf{X}))$. 

\section{Параметрические оценки на~основе <<опорного 
элемента>>} 

Выберем подмножество объектов~$\mathbf{X}$, $\alpha \hm\in 
L(T(\mathbf{X}))$. Подмножество~$\alpha$ может соответствовать 
$k^\prime$-му булеву при\-зна\-ку в~исходном при\-зна\-ко\-вом описании (тогда 
$\alpha\hm= \Gamma^{-1}_{k^\prime} (1)$) или синтетическому булеву 
при\-зна\-ку, что разбивает~$\mathbf{X}$ на~$\alpha$ и~$\overline{\alpha}\hm= 
\mathbf{X}\backslash \alpha$ и~определяет $\nu_\alpha \hm= \vert \alpha \vert / 
\vert \mathbf{X}\vert \hm= \vert\alpha\vert / N$ 
и~$\nu_{\overline{\alpha}}\hm=1\hm- \nu_\alpha$. 
     
     <<Опираясь>> на множество~$\alpha$ как параметр, мож\-но породить 
несколько изотонных решеточных оценок на основе базовой оцен\-ки, рав\-ной 
высоте элемента~$u_i$ в~$L(T(\mathbf{X}))$, $h[u_i] \hm= \vert u_i\vert$, 
час\-тот~$\nu_\alpha$ и~$\nu_{\overline{\alpha}}$. Оцен\-ка $v_\alpha^+ [u_i] 
\hm= \vert u_i \cap \alpha \vert / \vert\alpha\vert$ соответствует час\-то\-те 
встре\-ча\-емости элементов из $u_i\hm\subseteq \mathbf{X}$ в~$\alpha \subset 
\mathbf{X}$; оцен\-ка $v_\alpha^-[u_i]\hm= \vert u_i \cap \overline{\alpha}\vert / 
\vert \overline{\alpha}\vert \hm= \vert u_i \backslash\alpha\vert / (N\hm- 
\vert\alpha\vert)$~--- встре\-ча\-емость элементов~$u_i$ в~$\overline{\alpha}$. 
В~соответствии с~тео\-ре\-мой~2 линейные комбинации оценок~$v_\alpha^+$ 
и~$v_\alpha^-$ так\-же изотонны при условии положительной опре\-де\-лен\-ности. 
     
     \medskip
     
     \noindent
\textbf{Теорема~3.} \textit{Изотонная оцен\-ка $d_\alpha [\,]\hm= 
v_\alpha^+[\,] \hm- v_\alpha^-[\,]$ существует, когда объекты из 
множества~$\alpha$ встречаются в~оце\-ни\-ва\-емых множествах~$u_i$ не 
реже, чем в~сред\-нем по множеству всех объектов}~$\mathbf{X}$.

\medskip

     Рассмотрим функционал $\mathrm{d}_\alpha [u_i] \hm= v_\alpha^+[u_i] 
\hm+ \omega v_\alpha^- [u_i]$, $\omega\hm\in R$, и~покажем, что 
$d_\alpha[\,]$~--- линейная комбинация мет\-ри\-ки $\rho(u_i,\alpha)$ 
и~оценки $h[u_i]\hm=\vert u_i\vert$. Приведем~$v^+_\alpha$ и~$v_\alpha^-$ 
к~общему знаменателю, рав\-но\-му $\vert \alpha \vert (N\hm- \vert\alpha\vert)$. 
Подставляя 
$$
\vert u_i \cap \alpha \vert = 0{,}5 (\vert u_i\vert + \vert\alpha\vert - \rho(u_i,\alpha)),
$$
 после упрощения получим (для 
произвольного~$u_i$), что 
$$
{d}_\alpha [u_i] = k_\alpha \vert\alpha\vert + b_\alpha\vert u_i\vert - k_\alpha \rho(u_i,\alpha)\,,
$$
 где 
\begin{align*}
b_\alpha&= \fr{1/v_\alpha+\omega-1}{2(N-\vert\alpha\vert)}\,;\\
k_\alpha&= \fr{1/v_\alpha -\omega-1}{2(N-\vert\alpha\vert)}\,.
\end{align*}

 Рас\-смот\-рим 
функционал~${d}_\alpha[\,]$ (который служит изотонной оцен\-кой 
для всех $\omega\hm>0$) в~случае $\omega\hm<0$ и~запишем его в~виде 
${d}_\alpha[\,] \hm= v_\alpha^+ [\,] \hm- \vert\omega\vert v_\alpha^-[\,]$. 
Условию ${d}_\alpha [u_i] \hm\geq 0$ со\-от\-вет\-ст\-вует 
     $$
     \fr{1/v_\alpha+\vert\omega\vert-1}{N-\vert\alpha\vert}\,\vert u_i\bigcap \alpha 
\vert \geq \fr{\vert\omega\vert \vert u_i\vert}{N-\vert\alpha\vert}\,,
     $$ 
     т.\,е.\
     $$
     \fr{\vert i_i\cap \alpha\vert}{\vert u_i\vert} \geq 
\fr{\vert\omega\vert}{1/v_\alpha +\vert\omega\vert-1}\,.
     $$
      Это условие выделяет подмножество $\{u_i\}\hm\subset 
L(T(\mathbf{X}))$, где ${d}_\alpha[\,]$ изотонна. При 
$\vert\omega\vert \hm\sim 1$ по\-лу\-чаем 
     $\vert u_i\cap \alpha\vert/\vert u_i\vert\hm\geq \nu_\alpha,$
      соответствующее условию тео\-ре\-мы. Тео\-ре\-ма до\-ка\-зана.

     \medskip
     
     \noindent
\textbf{Следствие~1.}\ 
$$
{d}_\alpha^\prime [u_i] =\begin{cases}
v_\alpha^\prime [u_i]-v_\alpha^\prime[u_i] & \mbox{\textit{при}\ } \fr{\vert u_i\cap \alpha \vert}{\vert u_i\vert} \geq \nu_\alpha\,;\\
0 & \mbox{\textit{в\ противном\ случае}}
\end{cases}
$$
\textit{есть изотонная оценка}.
    
     \medskip

Функционал $d_\alpha[u_i]$ интересен тем, что может быть увязан 
с~конструктами не\-па\-ра\-мет\-ри\-че\-ской ста\-ти\-сти\-ки, разработанными в~научной 
школе А.\,Н.~Колмогорова~\cite{7-tor}. Как известно, для 
не\-па\-ра\-мет\-ри\-че\-ских тес\-тов используется так на\-зы\-ва\-емое максимальное 
уклонение~$D$, рав\-ное супремуму модуля раз\-ности значений двух функций 
распределения (тео\-ре\-ти\-че\-ской и~эмпирической или двух эмпирических 
функций распределения) по всей их об\-ласти опре\-де\-ле\-ния.

     \medskip
     
     \noindent
\textbf{Теорема~4.}\ \textit{Для <<опор\-но\-го>> множества~$\alpha$ 
и~произвольной цепи $U\hm= \langle u_1, \ldots , u_i, \ldots\rangle$ 
$$
D=\mathrm{sup}\,\vert d_\alpha[u_i]\vert
$$ 
при $\omega\hm= -1$~--- максимальное уклонение Кол\-мо\-го\-рова.}

     \medskip

     \noindent
     Доказательство проводится исходя из тео\-ре\-мы~1 для любой цепи 
$A\hm= \langle u_i\rangle$, $u_i\hm= u(\lambda_{k_i})$ и~анализа чис\-ло\-во\-го 
при\-зна\-ка с~множеством значений~$I_k$ в~двух цепях, образованных 
пересечением каж\-до\-го элемента~$u_i$ цепи~$A$ с~множествами~$\alpha$ 
и~$\overline{\alpha}$ со\-от\-вет\-ст\-венно.

     \medskip
     
     \noindent
\textbf{Следствие~1.} \textit{Статистическая до\-сто\-вер\-ность значения~$D$ 
оценивается по критерию Кол\-мо\-го\-ро\-ва--Смир\-нова}.

     \medskip

    Оценки $v_\alpha^+$, $v_\alpha^-$, $d_\alpha$ и~др.\ могут строиться 
для набора <<опор\-ных>> множеств. Пусть задан 
набор $\bm{\alpha}\hm\subset L(T(\mathbf{X}))$ подмножеств объектов 
(т.\,е.\ элементов решетки): $\bm{\alpha}\hm= \{\alpha_1, \alpha_2, \ldots , 
\alpha_i\ldots\}$. Каж\-дое из множеств в~наборе~$\bm{\alpha}$ по\-рож\-да\-ет 
(изотонную) па\-ра\-мет\-ри\-че\-скую оцен\-ку~$v_{\alpha_i}$ (например, 
$v^+_{\alpha_i}$, $v^-_{\alpha_i}$, их линейные комбинации). Тогда 
в~соответствии с~тео\-ре\-мой~2 может быть по\-рож\-де\-на и~на\-стра\-и\-ва\-емая 
изотонная оцен\-ка 
$$
v_\alpha= \sum\limits_{i=0,\vert \bm{\alpha}\vert} \!\omega_i 
v_{\alpha_i},\quad v_{\alpha_0} \equiv 1\,.
$$
 Альтернативно на основании 
набора~$\bm{\alpha}$ для каж\-до\-го $a\hm\in L(T(\mathbf{X}))$ может быть 
вы\-чис\-лен вектор оценок $(v_{\alpha_1}[a], v_{\alpha_2}[a], \ldots , 
v_{\alpha_i}[a], \ldots)$, $v_{\alpha_i}[a]\hm\in R^+$, и~введены мет\-ри\-ки уже на 
про\-стран\-ст\-ве таких векторов ($l_p$-мет\-ри\-ки, рас\-сто\-яния Пен\-роу\-за, 
Мотыки, Брея--Кур\-ти\-са, кореляционные  
рас\-сто\-яния)~\cite{6-tor}. Формирование наборов~$\bm{\alpha}$ может 
проводиться на основе <<ин\-фор\-ма\-тив\-ности>> со\-от\-вет\-ст\-ву\-ющих 
$\alpha_i\hm\in \bm{\alpha}$ методами мет\-ри\-че\-ско\-го анализа данных 
и~др.~\cite{4-tor}. 

\begin{table*}[b]\small
\vspace*{6pt}

\begin{center}
\begin{tabular}{|l|c|c|c|}
\multicolumn{4}{p{148mm}}{Результаты вычислительных экспериментов на 
вы\-бор\-ке~156\,327~измерений хе\-мо\-мик\-ро\-би\-ом\-ной био\-ло\-ги\-че\-ской ак\-тив\-ности 
(2173~пациентов). Оценены эффекты~122~лекарств. Приведены значения коэффициентов 
ранговой корреляции для различных моделей формирования множеств пре\-це\-ден\-тов. Эксперименты проводились 
в~рамках \mbox{кросс-ва}\-ли\-да\-ци\-он\-но\-го дизайна (10~разбиений в~соотношении  
<<слу\-чай--конт\-роль>> $6:1$). \mbox{Поиск} оптимальных значений па\-ра\-мет\-ров проводился 
муль\-ти\-стар\-то\-вой сто\-ха\-сти\-че\-ской оптимизацией}\\
\multicolumn{4}{c}{\ }\\[-6pt]
\hline 
\multicolumn{1}{|c|}{Вычислительный эксперимент}&$n_{\mathrm{акт}}$&$r$&$r_c$\\
\hline
Коэффициент многопараметрической регрессии, $p(D)=0{,}20$&1250&$0{,}72\pm0{,}23$&$0{,}32\pm 
0{,}20$\\
Коэффициент многопараметрической регрессии, $p(D)=0{,}05$&\hphantom{9}730&$0{,}73\pm 
0{,}39$&$0{,}38\pm 0{,}18$\\
Разность значений, $p(D)=0{,}20$&1192&$0{,}74\pm 0{,}19$&$0{,}34\pm 0{,}22$\\
Разность значений, $p(D)=0{,}05$&\hphantom{9}787&$0{,}78\pm 0{,}36$&$0{,}39\pm 
0{,}25$\\
Доля разности значений, $p(D)=0{,}20$&1292&$0{,}80\pm 0{,}22$&$0{,}46\pm 0{,}14$\\
Доля разности значений, $p(D)=0{,}10$&1291&$0{,}80\pm 0{,}27$&$0{,}62\pm 0{,}18$\\
\textbf{Доля разности значений}, 
{\boldmath{$p(D)=0{,}05$}}&\hphantom{9}\textbf{868}&{\boldmath{$0{,}76\pm 0{,}37$}}&{\boldmath{$0{,}77\pm 0{,}22$}}\\
Уклонение $D$ со знаком, $p(D)=0{,}20$&1286&$0{,}81\pm 0{,}21$&$0{,}48\pm 0{,}15$\\
Уклонение $D$ со знаком, $p(D)=0{,}10$&1306&$0{,}80\pm 0{,}26$&$0{,}63\pm 0{,}18$\\
\textbf{Уклонение $\bm D$ со знаком}, 
{\boldmath{$p(D)=0{,}05$}}&\hphantom{9}\textbf{836}&{\boldmath{$0{,}80\pm 0{,}38$}}&{\boldmath{$0{,}79\pm 0{,}21$}}\\
\hline
\multicolumn{4}{p{148mm}}{\footnotesize \hspace*{3mm}\textbf{Обозначения:} $r$ и~$r_c$~--- коэффициенты ранговой корреляции на обуче\-нии и~на
контроле соответственно;  $n_{\mathrm{акт}}$~--- число различных типов хе\-мо\-мик\-ро\-биом\-ной ак\-тив\-ности, по 
которым проводилось усред\-не\-ние~$r$ и~$r_c$; $p(D)$~--- верхний порог статистической 
зна\-чи\-мости по тес\-ту Кол\-мо\-го\-ро\-ва--Смир\-но\-ва.}
\end{tabular}
\end{center}
\end{table*}

\section{О параметрических оценках на~основе <<опорной цепи>>}

     Крайне перспективным пред\-став\-ля\-ет\-ся на\-прав\-ле\-ние по\-рож\-де\-ния 
оценок, в~котором заданная максимальная цепь~$A_t$ используется 
в~качестве <<опор\-ной>>, т.\,е.\ для по\-рож\-де\-ния па\-ра\-мет\-ри\-че\-ских \mbox{оценок}. 
Данное на\-прав\-ле\-ние будет по\-дроб\-но рас\-смот\-ре\-но в~отдельной работе. 
Вкратце: произвольная цепь $A\hm= \langle u_i\rangle$ пред\-ста\-ви\-ма в~виде 
$\langle u(\lambda_{t_i})\rangle$ для некоторого упорядоченного множества 
чисел~$I_t$ (тео\-ре\-ма~1). Значение функции~$\Gamma_t$ (включая 
не\-опре\-де\-лен\-ность), вы\-чис\-ли\-мое для любого объекта в~$\mathbf{X}$, 
рав\-но~$\Gamma_t(q)$ для каж\-до\-го решеточного атома $\{q\}\hm\in 
L(T(\mathbf{X}))$, ($h[\{q\}] \hm\equiv \vert \{q\}\vert \hm\equiv 1$), так что 
любому $u\hm\in L(T(\mathbf{X}))$ соответствует множество 
$\bm{\Gamma}_t (u)\hm= \{ \Gamma_t(q), q\hm\in u\}$. Определим 
\textit{оператор} $\hat{\phi}(x)$ для \textit{формирования эмпирической 
функции распределения}  конечного множества $A\hm= \{a_1, a_2, \ldots , a_i, 
\ldots , a_n\}$, $a_i\hm\in R$, как 
$$
\hat{\phi}(x) A=\mathrm{sup}\,\fr{\vert \{ 
B\hm\subseteq A\vert \forall\ a\in B: a\hm\leq x\}\vert }{\vert A\vert},\enskip x\in R\,,
$$
 и~\textit{оператор вы\-чис\-ле\-ния математического ожидания}~$\hat{\mu}$. 
     
     Итак, при задании опорной цепи~$A_t$ (т.\,е.\ опорного чис\-ло\-во\-го 
при\-зна\-ка с~об\-ластью значений~$I_t$) любому элементу решетки $u\hm\in 
L(T(\mathbf{X}))$ со\-по\-став\-ле\-но множество чисел~$\bm{\Gamma}_t(u)$, 
числовая функция $\hat{\phi}(x)\bm{\Gamma}_t(u)$ одной переменной 
$x\hm\in R$, не\-три\-ви\-аль\-но определенная в~каж\-дой точке отрезка 
$[\lambda_{t_1}, \lambda_{t_{\vert I_t\vert-1}}]$, и~ряд функционалов 
(в~част\-ности, $\hat{\mu}\hat{\phi}(x)\bm{\Gamma}_t(u)$). При выполнении 
условия ре\-гу\-ляр\-ности для $\mathbf{X}/Q$ решетка $L(T(\mathbf{X}))$ 
однозначно со\-по\-став\-ле\-на решетке, образованной чис\-ло\-вы\-ми множествами 
$\bm{\Gamma}_t(u)$, для каж\-до\-го элемента которой вы\-чис\-ли\-ма функция 
$\hat{\phi}\bm{\Gamma}_t(u)$. Множества~$\bm{\Gamma}_t(u)$, функции 
$\hat{\phi}\bm{\Gamma}_t(u)$ и~функционалы наподобие 
$\hat{\mu}\hat{\phi}\bm{\Gamma}_t(u)$ могут быть использованы для 
определения оценок в~решетке $L(T(\mathbf{X}))$, по\-рож\-да\-емых на осно\-ве 
выбранного опорного при\-знака. 

\section{Экспериментальная апробация формализма}

     Формализм был апробирован на задаче формирования множеств 
прецедентов для проведения хе\-мо\-мик\-ро\-биом\-но\-го анализа~\cite{3-tor, 8-tor}. 
В~исходной таб\-ли\-це признаковых описаний пред\-став\-ле\-ны булевы 
переменные, со\-от\-вет\-ст\-ву\-ющие~122~лекарствам, которые влияют 
на~1533~чис\-ло\-вые переменные, со\-от\-вет\-ст\-ву\-ющие уров\-ням отдельных 
бактерий или их групп. 
     
     Для квантификации вклада лекарства~$\alpha$ в~изменение уров\-ней $t$-й 
бактерии среди разработанных оценок использовалось уклонение~$D$ со 
знаком (тео\-ре\-мы~3 и~4). Для учета на\-прав\-ле\-ния изменения уров\-ней 
бактерий уклонение~$D$ необходимо до\-мно\-жить на знак d$_\alpha [\,]$. По 
тес\-ту Кол\-мо\-го\-ро\-ва--Смир\-но\-ва вы\-чис\-ля\-лось значение 
ста\-ти\-сти\-че\-ской до\-сто\-вер\-ности $p(D)$ для отсечения наименее до\-сто\-вер\-ных 
ассоциаций <<ле\-карст\-во--эф\-фект>>. Так\-же использовались более 
очевидные подходы: коэффициенты линейной мно\-го\-па\-ра\-мет\-ри\-че\-ской 
регрессии (с~отбором признаков по модели <<лассо>>), раз\-ность значений 
     $\hat{\mu}\hat{\phi}\bm{\Gamma}_t(\alpha) \hm- \hat{\mu} \hat{\phi} 
\bm{\Gamma}_t(\overline{\alpha})$, доля раз\-ности $1\hm- \hat{\mu} \hat{\phi} 
\bm{\Gamma}_t(\overline{\alpha})/ \hat{\mu} \hat{\phi} 
\bm{\Gamma}_t(\alpha)$. 
     
     Лекарства с~известной химической структурой описывались на осно\-ве 
хемогр$\acute{\mbox{а}}$фов~$G_j$, а~в~качестве множества начальных 
информаций~$I_i$ использовалось множество хемоинвариантов над 
алфавитом элементных меток. \mbox{Алгоритмы}\linebreak $f_{\theta_k}: I_i\hm\to R$ 
строились в~виде композиций вло\-жен\-ных кор\-рек\-ти\-ру\-ющих функций 
ниж\-не\-го уров\-ня (т.\,е.\ по\-рож\-де\-ния синтетических при\-зна\-ков) для 
фиксированного чис\-ла моделей  $n_{\mathrm{mod}}: f_{\theta_k} \hm=$
$= g 
(f_1(\sum \omega_k^j x_k), \ldots , f_l(\sum \omega_k^j x_k),\ldots)$, $l\hm=1, 
\ldots$\linebreak $\ldots , n_{\mathrm{mod}}$, как в~работе~\cite{5-tor}. Тес\-ти\-ро\-ва\-ние 
ал\-го\-рит\-мов~$f_{\theta_k}$ проводилось на выборке данных о~пациентах 
(см.\ таб\-лицу).
     

    
Результаты экспериментов показывают, что по\-рож\-де\-ние множества 
исходных информаций на основе регрессионных коэффициентов и~раз\-ности 
значений материала обуче\-ния соответствует крайне низкой обоб\-ща\-ющей 
спо\-соб\-ности на\-стра\-и\-ва\-емых алгоритмов. При этом ужес\-то\-че\-ние порога на 
значение $p(D)$, давая  снижение чис\-ла раз\-лич\-ных типов 
хе\-мо\-мик\-ро\-биом\-ной ак\-тив\-ности, не приводило к~увеличению качества 
ал\-го\-рит\-мов. Наилучший результат был получен при использовании 
в~качестве исходной информации значения уклонения~$D$ со знаком 
($r\hm=0{,}80\pm0{,}38$, $r_c\hm=0{,}79\hm\pm 0{,}21$), а~более низ\-кие 
значения порога $p(D)$ приводили к~повышению качества распознавания. 
Отметим не\-пло\-хой результат и~для такого прос\-то\-го функционала, как доля 
раз\-ности значений (см.\ таб\-ли\-цу).

\vspace*{-5pt}

\section{Заключение}

\vspace*{-1pt}

    В работе впервые проведено сис\-те\-ма\-ти\-че\-ское рас\-смот\-ре\-ние способов 
введения оценок на решетке (высота элемента, оцен\-ки на основании булевых 
и~чис\-ло\-вых переменных, линейные комбинации вы\-ше\-пе\-ре\-чис\-лен\-ных 
оценок~и~др.). Введение оценок на~$L(X)$, по аналогии с~понятием меры 
в~функциональном анализе, поз\-во\-ля\-ет по\-рож\-дать па\-ра\-мет\-ри\-че\-ские 
решеточные оценки и~затем вво\-дить проб\-лем\-но-ори\-ен\-ти\-ро\-ван\-ные мет\-ри\-ки, 
оце\-ни\-ва\-ющие рас\-сто\-яние между вершинами решетки. Разработанный 
формализм был применен для до\-сти\-же\-ния практической цели на\-сто\-ящей 
статьи~--- на\-хож\-де\-ния оптимального способа оцен\-ки вкладов переменных 
при анализе слож\-ных данных хе\-мо\-мик\-ро\-би\-ом\-ных исследований. 
Разработанный формализм пред\-остав\-ля\-ет инструментарий для поиска 
аде\-кват\-ной формализации задач классификации и~прогнозирования.

\vspace*{-5pt}

{\small\frenchspacing
 { %\baselineskip=12pt
 %\addcontentsline{toc}{section}{References}
 \begin{thebibliography}{9}
 
 \vspace*{-1pt}
 
\bibitem{1-tor}
\Au{Журавлёв Ю.\,И.} Избранные научные труды.~--- М.: Магистр, 1998. 420~с. 
\bibitem{2-tor}
\Au{Torshin I.\,Yu., Rudakov~K.\,V.} On the theoretical basis of metric analysis of poorly 
formalized problems of recognition and classification~// Pattern Recognition Image Analysis, 2015. 
Vol.~25. No.\,4. P.~577--587. doi: 10.1134/ S1054661815040252.
\bibitem{3-tor}
\Au{Торшин И.\,Ю., Громова~О.\,А., Захарова~И.\,Н., Мак\-си\-мов~В.\,А.} 
Хе\-мо\-мик\-ро\-би\-ом\-ный ана\-лиз Лак\-ти\-то\-ла~// Экспериментальная и~клиническая 
гаст\-ро\-эн\-те\-ро\-ло\-гия, 2019. Т.~164. №\,4. С.~111--121. doi:  
10.31146/1682-8658-ecg-164-4-111-121.
\bibitem{4-tor}
\Au{Рудаков К.\,В., Торшин~И.\,Ю.} Анализ информативности мотивов на основе 
критерия раз\-ре\-ши\-мости в~задаче рас\-по\-зна\-ва\-ния вторичной структуры белка~// 
Информатика и~её применения, 2012. Т.~6. Вып.~1. С.~79--90.
\bibitem{5-tor}
\Au{Торшин И.\,Ю.} О~задачах оптимизации, воз\-ни\-ка\-ющих при применении 
топологического анализа данных к~поиску алгоритмов прогнозирования 
с~фиксированными корректорами~// Информатика и~её применения, 2023. Т.~17. Вып.~2. 
С.~2--10.  doi: 10.14357/ 19922264230201. EDN: IGSPEW.
\bibitem{6-tor}
\Au{Деза Е.\,И., Деза~М.\,М.} Энциклопедический словарь рас\-сто\-яний~/
Пер.\ с~англ.~--- М.: Наука, 2008. 444~с.
(\Au{Deza~E., Deza~M.-M.} Dictionary of distances.~---  Elsevier B.V., 2006. 412~p.)
\bibitem{7-tor}
\Au{Колмогоров A.\,H., Фомин~С.\,В.} Элементы тео\-рии функций и~функционального 
анализа.~--- М.: Наука, 1989. 624~с.
\bibitem{8-tor}
\Au{Forslund~S.\,K., Chakaroun~R., Stumvoll~M., Bork~P.} Combinatorial, additive and  
dose-dependent drug-microbiome associations~// Nature, 2021. Vol.~600. No.\,7889.  
P.~500--505. doi: 10.1038/s41586-021-04177-9.

\end{thebibliography}

 }
 }

\end{multicols}

\vspace*{-8pt}

\hfill{\small\textit{Поступила в~редакцию 02.02.23}}

\vspace*{6pt}

%\pagebreak

%\newpage

%\vspace*{-28pt}

\hrule

\vspace*{2pt}

\hrule

%\vspace*{-2pt}

\def\tit{ON THE FORMATION OF SETS OF~PRECEDENTS BASED ON~TABLES 
OF~HETEROGENEOUS FEATURE DESCRIPTIONS BY~METHODS 
OF~TOPOLOGICAL THEORY OF~DATA ANALYSIS}


\def\titkol{On the formation of sets of~precedents based on~tables 
of~heterogeneous feature descriptions} % by~methods  of~topological theory of~data analysis}


\def\aut{I.\,Yu.~Torshin}

\def\autkol{I.\,Yu.~Torshin}

\titel{\tit}{\aut}{\autkol}{\titkol}

\vspace*{-14pt}


\noindent
Federal Research Center ``Computer Science and Control'' of the Russian Academy 
of Sciences, 44-2~Vavilov Str., Moscow 119333, Russian Federation


\def\leftfootline{\small{\textbf{\thepage}
\hfill INFORMATIKA I EE PRIMENENIYA~--- INFORMATICS AND
APPLICATIONS\ \ \ 2023\ \ \ volume~17\ \ \ issue\ 3}
}%
 \def\rightfootline{\small{INFORMATIKA I EE PRIMENENIYA~---
INFORMATICS AND APPLICATIONS\ \ \ 2023\ \ \ volume~17\ \ \ issue\ 3
\hfill \textbf{\thepage}}}

\vspace*{3pt}



\Abste{Factorization of the contributions of various variables in the analysis of 
heterogeneous feature descriptions is an urgent task of complex data mining. The 
paper proposes the development of the lattice formalism of the\linebreak\vspace*{-12pt}}

\Abstend{ topological theory 
of data analysis, within which new methods for generating parametric estimates 
and metrics on lattices formed over the topologies of sets of objects are obtained. 
The formalism was tested on the problem of
forming sets of precedents for 
conducting chemomicrobiome analysis. Whereas the generation of a~set of initial 
information based on regression coefficients and the difference in the values of the 
learning material corresponded to an extremely low generalizing ability of custom 
algorithms (correlation coefficient in the control $0.32 \pm 0.20$), the use of the 
proposed estimates for generating sets of precedents in chemomicrobiomics 
problems made it possible 
to significantly increase the generalizing ability of the 
corresponding algorithms (correlation coefficient in control $0.79\pm 0.21$).}

\KWE{topological data analysis; lattice theory; parametrization of lattice terms; human 
microbiome; pharmacoinformatics, algebraic approach of Yu.\,I.~Zhu\-rav\-lev.}



 \DOI{10.14357/19922264230301}{AQEUYO}

\vspace*{-18pt}

\Ack
\noindent
The research was funded by the Russian Science Foundation, project No.\,23-21-00154. The 
research was carried out using the infrastructure of the Shared Research Facilities ``High 
Performance Computing and Big Data'' (CKP ``Informatics'') of FRC CSC RAS (Moscow).
  

\vspace*{6pt}

  \begin{multicols}{2}

\renewcommand{\bibname}{\protect\rmfamily References}
%\renewcommand{\bibname}{\large\protect\rm References}

{\small\frenchspacing
 {%\baselineskip=10.8pt
 \addcontentsline{toc}{section}{References}
 \begin{thebibliography}{9} 
\bibitem{1-tor-1}
\Aue{Zhuravlev, Yu.\,I.} 1998. \textit{Iz\-bran\-nye na\-uch\-nye trudy} [Selected scientific works]. 
Moscow: Magistr. 420~p.
\bibitem{2-tor-1}
\Aue{Torshin, I.\,Y., and K.\,V.~Rudakov.} 2015. On the theoretical basis of metric analysis of 
poorly formalized problems of recognition and classification. \textit{Pattern Recognition Image 
Analysis} 25(4):577--587. doi: 10.1134/S1054661815040252.
\bibitem{3-tor-1}
\Aue{Torshin, I.\,Yu., O.\,A.~Gromova, I.\,N.~Za\-kha\-rova, and V.\,A.~Mak\-si\-mov.} 2019. 
 Khe\-mo\-mik\-ro\-biom\-nyy ana\-liz Lak\-ti\-to\-la [Hemomikrobiomny lactitol analysis]. 
\textit{Eks\-pe\-ri\-men\-tal'\-naya i~kli\-ni\-che\-skaya gast\-ro\-en\-te\-ro\-lo\-giya} [Experimental and Clinical 
Gastroenterology]. 164(4):111--121. doi: 10.31146/1682-8658-ecg-164-4-111-121.
\bibitem{4-tor-1}
\Aue{Rudakov, K.\,V., and I.\,Yu.~Tor\-shin.} 2012. Ana\-liz in\-for\-ma\-tiv\-nosti mo\-ti\-vov 
na osno\-ve kri\-te\-riya raz\-re\-shi\-mosti v~za\-da\-che ras\-po\-zna\-va\-niya vto\-rich\-noy 
struk\-tu\-ry bel\-ka [Analysis of the informativeness of motives based on the criterion of 
solvability in the problem of recognizing the secondary structure\linebreak\vspace*{-12pt}

\columnbreak

\noindent
 of a~protein]. 
\textit{Informatika i~ee Primeneniya~--- Inform Appl.} 6(1):79--90.
\bibitem{5-tor-1}
\Aue{Torshin, I.\,Yu.} 2023. O~za\-da\-chakh op\-ti\-mi\-za\-tsii, voz\-ni\-ka\-yushchikh  
pri pri\-me\-ne\-nii to\-po\-lo\-gi\-che\-sko\-go ana\-li\-za dan\-nykh k~pois\-ku al\-go\-rit\-mov 
prog\-no\-zi\-ro\-va\-niya s~fik\-si\-ro\-van\-ny\-mi kor\-rek\-to\-ra\-mi [On optimization 
problems arising from the application of topological data analysis to the search for forecasting 
algorithms with fixed correctors]. \textit{Informatika i~ee Primeneniya~--- Inform Appl.} 
17(2):2--10. doi: 10.14357/19922264230201. EDN: IGSPEW.
\bibitem{6-tor-1}
\Aue{Deza, E., and M.-M.~Deza.} 2006. \textit{Dictionary of distances}.  Elsevier B.V. 412~p.
\bibitem{7-tor-1}
\Aue{Kolmogorov, A.\,N., and S.\,V.~Fo\-min.} 1989. \textit{Ele\-men\-ty teo\-rii funk\-tsiy 
i~funk\-tsi\-o\-nal'\-no\-go ana\-li\-za} [Elements of the theory of functions and functional 
analysis]. Moscow: Nauka. %\linebreak 
624~p.
\bibitem{8-tor-1}
\Aue{Forslund, S.\,K., R.~Chakaroun, M.~Stumvoll, and P.~Bork.} 2021. Combinatorial, 
additive and dose-dependent drug-microbiome associations. \textit{Nature} 600(7889):500--505. 
doi: 10.1038/s41586-021-04177-9.

\end{thebibliography}

 }
 }

\end{multicols}

\vspace*{-6pt}

\hfill{\small\textit{Received February 2, 2023}} 

  \Contrl
  
  \noindent
  \textbf{Torshin Ivan Y.} (b.\ 1972)~--- Candidate of Science (PhD) in physics and mathematics, 
Candidate of Science (PhD) in chemistry, senior scientist, A.\,A.~Dorodnicyn Computing Center, 
Federal Research Center ``Computer Science and Control'' of the Russian Academy of Sciences, 
40~Vavilov Str., Moscow 119333, Russian Federation; \mbox{tiy135@yahoo.com}




\label{end\stat}

\renewcommand{\bibname}{\protect\rm Литература} 