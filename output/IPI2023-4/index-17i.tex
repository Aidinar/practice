
\def\stat{cont}
{%\hrule\par
%\vskip 7pt % 7pt
\raggedleft\Large \bf%\baselineskip=3.2ex
А\,В\,Т\,О\,Р\,С\,К\,И\,Й\ \ У\,К\,А\,З\,А\,Т\,Е\,Л\,Ь\ \ З\,А\ \ 2\,0\,2\,3 г. \vskip 17pt
 \hrule
 \par
\vskip 21pt plus 6pt minus 3pt }

\label{st\stat}

\def\tit{\ }

\def\aut{\ }
\def\auf{\ }

\def\leftkol{\ } % ENGLISH ABSTRACTS}

\def\rightkol{\ } %АВТОРСКИЙ УКАЗАТЕЛЬ ЗА 2021 г.} %ENGLISH ABSTRACTS}

\titele{\tit}{\aut}{\auf}{\leftkol}{\rightkol}
\addcontentsline{toc}{subsection}{\textrm\textbf Авторский указатель за 2023 г.}

\vspace*{-24pt}

\noindent
{\tabcolsep=3pt
\begin{tabular}{p{397pt}cc}
&\textbf{Вып.} & \textbf{Стр.}\\[6pt]
\Avtors{Агаларов~Я.\,М.} Об оптимизации работы резервного прибора в~многолинейной 
системе массового обслуживания&\raisebox{-12pt}[0pt][0pt]{1}&\raisebox{-12pt}[0pt][0pt]{89--95}\\
\Avtors{Агаларов~Я.\,М.} Оптимизация схемы распределения буферной памяти узла 
пакетной коммутации&\raisebox{-12pt}[0pt][0pt]{3}&\raisebox{-12pt}[0pt][0pt]{39--48}\\
\Avtors{Агасандян~Г.\,А.} Многомерные баттерфляи в~задачах оптимизации по CC-VaR&1&107--115\\
\Avtors{Аду~К.\,И.\,Б., Маркова~Е.\,В., Гайдамака~Ю.\,В., Шоргин~С.\,Я.} Анализ схемы 
доступа с~прерыванием при нарезке радиоресурсов сети пятого 
поколения&\raisebox{-12pt}[0pt][0pt]{1}&\raisebox{-12pt}[0pt][0pt]{\hphantom{1}96--106}\\
\Avtors{Архипов~П.\,О., Филиппских~С.\,Л., Цуканов~М.\,В.} Разработка новой модели 
ступенчатой сверточной нейронной сети для классификации аномалий на панорамах&\raisebox{-12pt}[0pt][0pt]{1}&\raisebox{-12pt}[0pt][0pt]{50--56}\\
\Avtors{Бегишев~В.\,О.} см.\ Сопин~Э.\,С.&&\\
\Avtors{Берговин~А.\,К., Ушаков~В.\,Г.} Исследование систем обслуживания со 
смешанными приоритетами&\raisebox{-12pt}[0pt][0pt]{2}&\raisebox{-12pt}[0pt][0pt]{57--61}\\
\Avtors{Борисов~А.\,В.} Рынок с~марковской скачкообразной волатильностью 
I:~мониторинг цены риска как задача оптимальной фильтрации&\raisebox{-12pt}[0pt][0pt]{2}&\raisebox{-12pt}[0pt][0pt]{27--33}\\
\Avtors{Борисов~А.\,В.} Рынок с~марковской скачкообразной волатильностью~II: алгоритм 
вы\-чис\-ле\-ния справедливой цены деривативов&\raisebox{-12pt}[0pt][0pt]{3}&\raisebox{-12pt}[0pt][0pt]{18--24}\\
\Avtors{Борисов А.\,В.} Рынок с марковской скачкообразной волатильностью III:  алгоритм 
мониторинга цены риска по дискретным наблюдениям цен активов&\raisebox{-12pt}[0pt][0pt]{4}&\raisebox{-12pt}[0pt][0pt]{\hphantom{9}9--16}\\
\Avtors{Босов~А.\,В.} Исследование робастности численных аппроксимаций фильтра 
Вонэма&2&41--49\\
\Avtors{Босов~А.\,В.} Оптимальная фильтрация состояния нелинейной динамической 
системы по наблюдениям со случайными запаздываниями&\raisebox{-12pt}[0pt][0pt]{3}&\raisebox{-12pt}[0pt][0pt]{\hphantom{1}8--17}\\
\Avtors{Босов~А.\,В., Иванов~А.\,В.} Технология многофакторной классификации 
математического контента электронной системы обучения&\raisebox{-12pt}[0pt][0pt]{4}&\raisebox{-12pt}[0pt][0pt]{32--41}\\
\Avtors{Босов~А.\,В., Игнатов~А.\,Н.} О~задаче оценки и~анализа риска транспортных 
происшествий на рельсовом транспорте&\raisebox{-12pt}[0pt][0pt]{1}&\raisebox{-12pt}[0pt][0pt]{73--82}\\
\Avtors{Вакуленко~В.\,В., Зацман~И.\,М.} Формализованное описание статистической 
обработки информации в~базах данных&\raisebox{-12pt}[0pt][0pt]{3}&\raisebox{-12pt}[0pt][0pt]{93--99}\\
\Avtors{Васильев~Н.\,С.} Композициональное представление структуры игры многих лиц 
в~моноидальной категории бинарных отношений&\raisebox{-12pt}[0pt][0pt]{2}&\raisebox{-12pt}[0pt][0pt]{18--26}\\
\Avtors{Волканов~Д.\,Ю.} см.\ Горшенин~А.\,К.&&\\
\Avtors{Воронцов~М.\,О., Шестаков~О.\,В.} Среднеквадратичный риск FDR-процедуры 
в~условиях слабой зависимости&\raisebox{-12pt}[0pt][0pt]{2}&\raisebox{-12pt}[0pt][0pt]{34--40}\\
\Avtors{Гайдамака~Ю.\,В.} см.\ Аду~К.\,И.\,Б.&&\\
\Avtors{Гайдамака~Ю.\,В.} см.\ Иванова Д.\,В.&&\\
\Avtors{Гайдамака~Ю.\,В.} см.\ Самуйлов~А.\,К.&&\\
\Avtors{Гаримелла~Р.\,М.} см.\ Разумчик~Р.\,В.&&\\
\Avtors{Гончаров~А.\,А.} Аннотирование параллельных корпусов: подходы и направления 
развития&4&81--87\\
\Avtors{Горбунов~С.\,А.} см.\ Горшенин~А.\,К.&&\\
\Avtors{Горшенин~А.\,К., Горбунов~С.\,А., Волканов~Д.\,Ю.} О~кластеризации объектов 
сетевой вы\-чис\-ли\-тель\-ной инфраструктуры на основе анализа статистических аномалий 
в~трафике&\raisebox{-12pt}[0pt][0pt]{3}&\raisebox{-12pt}[0pt][0pt]{76--87}\\
\Avtors{Грушо~А.\,А., Грушо~Н.\,А., Забежайло~М.\,И., Кульченков~В.\,В., 
Тимонина~Е.\,Е., Шоргин~С.\,Я.} Причинно-следственные связи в~задачах 
классификации&\raisebox{-12pt}[0pt][0pt]{1}&\raisebox{-12pt}[0pt][0pt]{43--49}\\
\Avtors{Грушо~А.\,А., Грушо~Н.\,А., Забежайло~М.\,И., Смирнов~Д.\,В., Тимонина~Е.\,Е.} 
Классификация с~помощью причинно-следственных связей&\raisebox{-12pt}[0pt][0pt]{3}&\raisebox{-12pt}[0pt][0pt]{71--75}\\
\Avtors{Грушо~А.\,А., Грушо~Н.\,А., Забежайло~М.\,И., Тимонина~Е.\,Е., Шоргин~С.\,Я.} 
Сложные причинно-следственные связи&\raisebox{-12pt}[0pt][0pt]{2}&\raisebox{-12pt}[0pt][0pt]{84--89}\\
\end{tabular}
}

\pagebreak

\def\leftkol{АВТОРСКИЙ УКАЗАТЕЛЬ ЗА 2023 г.} % ENGLISH ABSTRACTS}

\def\rightkol{АВТОРСКИЙ УКАЗАТЕЛЬ ЗА 2023 г.} %ENGLISH ABSTRACTS}

%\thispagestyle{myheadings}
\def\leftfootline{\small{\textbf{\thepage}
\hfill ИНФОРМАТИКА И ЕЁ ПРИМЕНЕНИЯ\ \ \ том~17\ \ \ выпуск~4\ \ \ 2023}
}%
 \def\rightfootline{\small{ИНФОРМАТИКА И ЕЁ ПРИМЕНЕНИЯ\ \ \ том~17\ \ \ выпуск~4\ \ \ 2023
 \hfill \textbf{\thepage}}}


\noindent
{\tabcolsep=3pt
\begin{tabular}{p{394pt}cc}
&\textbf{Вып.} & \textbf{Стр.}\\[3pt]
\Avtors{Грушо~Н.\,А.} см.\ Грушо~А.\,А.&&\\
\Avtors{Грушо~Н.\,А.} см.\ Грушо~А.\,А.&&\\
\Avtors{Грушо~Н.\,А.} см.\ Грушо~А.\,А.&&\\
\Avtors{Дулин~С.\,К.} см.\ Розенберг~И.\,Н.&&\\
\Avtors{Дулина~Н.\,Г.} см.\ Розенберг~И.\,Н.&&\\
\Avtors{Дюкова~А.\,П.} см.\ Дюкова~Е.\,В.&&\\
\Avtors{Дюкова~Е.\,В., Масляков~Г.\,О., Дюкова~А.\,П.} Логические методы корректной 
классификации данных&\raisebox{-12pt}[0pt][0pt]{3}&\raisebox{-12pt}[0pt][0pt]{64--70}\\
\Avtors{Забежайло~М.\,И.} см.\ Грушо~А.\,А&&\\
\Avtors{Забежайло~М.\,И.} см.\ Грушо~А.\,А.&&\\
\Avtors{Забежайло~М.\,И.} см.\ Грушо~А.\,А.&&\\
\Avtors{Захаров~В.\,Н.} см.\ Сазонтьев В.\,В.&&\\
\Avtors{Захаров В.\,Н.} см.\ Френкель С.\,Л.&&\\
\Avtors{Зацман~И.\,М.} Данные, информация и~знание в~научной парадигме 
информатики&1&116--125\\
\Avtors{Зацман И.\,М.} Научная парадигма информатики: классификация объектов 
предметной области&\raisebox{-12pt}[0pt][0pt]{4}&\raisebox{-12pt}[0pt][0pt]{\hphantom{9}96--103}\\
\Avtors{Зацман~И.\,М.} Трансформация иерархии Акоффа в~научной парадигме 
информатики&3&107--113\\
\Avtors{Зацман~И.\,М.} см.\ Вакуленко~В.\,В.&&\\
\Avtors{Зейфман~А.\,И.} см.\ Усов~И.\,А.&&\\
\Avtors{Иванов~А.\,В.} см.\ Босов~А.\,В.&&\\
\Avtors{Иванова Д.\,В., Маркова Е.\,В., Шоргин~С.\,Я., Гайдамака~Ю.\,В.} Модели 
совместного обслуживания трафика eMBB и URLLC на основе приоритетов в 
промышленных развертываниях 5G NR&\raisebox{-24pt}[0pt][0pt]{4}&\raisebox{-24pt}[0pt][0pt]{64--70}\\
\Avtors{Игнатов~А.\,Н.} см.\ Босов~А.\,В.&&\\
\Avtors{Инькова~О.\,Ю., Кружков~М.\,Г.} Критерии определения семантической близости 
дискурсивных отношений&\raisebox{-12pt}[0pt][0pt]{3}&\raisebox{-12pt}[0pt][0pt]{100--106}\\
\Avtors{Инькова О.\,Ю., Кружков~М.\,Г.} Степень семантической близости дискурсивных 
отношений:  методы и инструменты расчета&\raisebox{-12pt}[0pt][0pt]{4}&\raisebox{-12pt}[0pt][0pt]{88--95}\\
\Avtors{Кабанов~Ю.\,М., Сидоренко~А.\,П.} Аксиоматический взгляд на модели системного 
риска Роджерса--Вераарт и~Судзуки--Эльсингера&\raisebox{-12pt}[0pt][0pt]{1}&\raisebox{-12pt}[0pt][0pt]{11--17}\\
\Avtors{Карпов~В.\,И.} см.\ Нуриев~В.\,А.&&\\
\Avtors{Кириков~И.\,А.} см.\ Листопад~С.\,В.&&\\
\Avtors{Ковалёв~С.\,П.} Монада диаграмм как математическая метамодель системной 
инженерии&2&11--17\\
\Avtors{Королев~Д.\,О., Малеев~О.\,Г.} Исследование эффективности применения бинарных 
нейронных сетей при детектировании объекта на изображении&\raisebox{-12pt}[0pt][0pt]{3}&\raisebox{-12pt}[0pt][0pt]{88--92}\\
\Avtors{Кривенко~М.\,П.} Критерии выбора размерности модели факторизации&2&50--56\\
\Avtors{Кружков~М.\,Г.} см.\ Инькова О.\,Ю.&&\\
\Avtors{Кружков~М.\,Г.} см.\ Инькова~О.\,Ю.&&\\
\Avtors{Кудрявцев~А.\,А., Шестаков~О.\,В.} Метод оценивания параметров 
гамма-экс\-по\-нен\-ци\-аль\-но\-го распределения по выборке со слабо зависимыми компонентами&\raisebox{-12pt}[0pt][0pt]{3}&\raisebox{-12pt}[0pt][0pt]{58--62}\\
\Avtors{Кульченков~В.\,В.} см.\ Грушо~А.\,А.&&\\
\Avtors{Лапко~А.\,В.} см.\ Тубольцев~В.\,П.&&\\
\Avtors{Лапко~В.\,А.} см.\ Тубольцев~В.\,П.&&\\
\Avtors{Лери~М.\,М.} Среднее расстояние в~конфигурационных графах со степенным 
распределением&\raisebox{-12pt}[0pt][0pt]{1}&\raisebox{-12pt}[0pt][0pt]{28--34}\\
\Avtors{Листопад~С.\,В., Кириков~И.\,А.} Метод на основе нечетких правил для 
управления конфликтами агентов в~гибридных интеллектуальных многоагентных 
системах&\raisebox{-12pt}[0pt][0pt]{1}&\raisebox{-12pt}[0pt][0pt]{66--72}\\
\Avtors{Малашенко~Ю.\,Е., Назарова~И.\,А.} Анализ загрузки многопользовательской сети 
при расщеплении потоков по кратчайшим маршрутам&\raisebox{-12pt}[0pt][0pt]{3}&\raisebox{-12pt}[0pt][0pt]{33--38}\\
\Avtors{Малашенко~Ю.\,Е., Назарова~И.\,А.} Оценки распределения ресурсов 
в~многопользовательской сети при равных межузловых нагрузках&\raisebox{-12pt}[0pt][0pt]{1}&\raisebox{-12pt}[0pt][0pt]{83--88}\\
\Avtors{Малеев~О.\,Г.} см.\ Королев~Д.\,О.&&\\
\Avtors{Маркова~Е.\,В.} см.\ Аду~К.\,И.\,Б.&&\\
\Avtors{Маркова Е.\,В.} см.\ Иванова Д.\,В.&&\\
\end{tabular}
}

\pagebreak

\def\leftkol{АВТОРСКИЙ УКАЗАТЕЛЬ ЗА 2023 г.} % ENGLISH ABSTRACTS}

\def\rightkol{АВТОРСКИЙ УКАЗАТЕЛЬ ЗА 2023 г.} %ENGLISH ABSTRACTS}

%\thispagestyle{myheadings}
\def\leftfootline{\small{\textbf{\thepage}
\hfill ИНФОРМАТИКА И ЕЁ ПРИМЕНЕНИЯ\ \ \ том~17\ \ \ выпуск~4\ \ \ 2023}
}%
 \def\rightfootline{\small{ИНФОРМАТИКА И ЕЁ ПРИМЕНЕНИЯ\ \ \ том~17\ \ \ выпуск~4\ \ \ 2023
 \hfill \textbf{\thepage}}}


\noindent
{\tabcolsep=3pt
\begin{tabular}{p{394pt}cc}
&\textbf{Вып.} & \textbf{Стр.}\\[3pt]
\Avtors{Маслов~А.\,Р.} см.\ Сопин~Э.\,С&&\\
\Avtors{Масляков~Г.\,О.} см.\ Дюкова~Е.\,В.&&\\
\Avtors{Мелехин~В.\,Б., Хачумов~В.\,М., Хачумов~М.\,В.} Самообучение автономных 
интеллектуальных роботов в~процессе поисково-исследовательской деятельности&\raisebox{-12pt}[0pt][0pt]{2}&\raisebox{-12pt}[0pt][0pt]{78--83}\\
\Avtors{Назарова~И.\,А.} см.\ Малашенко~Ю.\,Е.&&\\
\Avtors{Назарова~И.\,А.} см.\ Малашенко~Ю.\,Е.&&\\
\Avtors{Нейчев~Р.\,Г., Шибаев~И.\,А., Стрижов~В.\,В.} Восстановление матрицы 
суперпозиции в~задаче символьной регрессии&\raisebox{-12pt}[0pt][0pt]{1}&\raisebox{-12pt}[0pt][0pt]{35--42}\\
\Avtors{Нуриев~В.\,А., Карпов~В.\,И.} Методология корпусно-ориентированного 
исследования в~области контрастивной пунктуации&\raisebox{-12pt}[0pt][0pt]{2}&\raisebox{-12pt}[0pt][0pt]{90--95}\\
\Avtors{Пешкова И.\,В.} Границы незавершенной работы в системе с повторными вызовами 
разных классов и показательным временем обслуживания&\raisebox{-12pt}[0pt][0pt]{4}&\raisebox{-12pt}[0pt][0pt]{57--63}\\
\Avtors{Платонова~А.\,А.} см.\ Самуйлов~А.\,К.&&\\
\Avtors{Рабинович Я.\,И.} Процедура построения множества Парето для дифференцируемых 
критериальных функций&\raisebox{-12pt}[0pt][0pt]{4}&\raisebox{-12pt}[0pt][0pt]{17--22}\\
\Avtors{Разумчик~Р.\,В., Румянцев~А.\,С., Гаримелла~Р.\,М.} Вероятностная модель для 
оценки основных характеристик производительности марковской модели 
суперкомпьютера&\raisebox{-24pt}[0pt][0pt]{2}&\raisebox{-24pt}[0pt][0pt]{62--70}\\
\Avtors{Розенберг~И.\,Н., Дулин~С.\,К., Дулина~Н.\,Г.} Моделирование структуры 
интероперабельности средствами структурной согласованности&\raisebox{-12pt}[0pt][0pt]{1}&\raisebox{-12pt}[0pt][0pt]{57--65}\\
\Avtors{Румовская~С.\,Б.} Подходы к~подбору специалистов при организации 
коллективного решения проблем&\raisebox{-12pt}[0pt][0pt]{2}&\raisebox{-12pt}[0pt][0pt]{\hphantom{1}96--103}\\
\Avtors{Румянцев~А.\,С.} см.\ Разумчик~Р.\,В.&&\\
\Avtors{Сазонтьев В.\,В., Ступников~С.\,А., Захаров~В.\,Н.} Расширяемый подход к слиянию 
данных в распределенных вычислительных средах&\raisebox{-12pt}[0pt][0pt]{4}&\raisebox{-12pt}[0pt][0pt]{42--47}\\
\Avtors{Самуйлов~А.\,К., Платонова~А.\,А., Шоргин~В.\,С., Гайдамака~Ю.\,В.} 
К~моделированию эффектов обслуживания многоадресного трафика в~сетях 5G~NR&\raisebox{-12pt}[0pt][0pt]{2}&\raisebox{-12pt}[0pt][0pt]{71--77}\\
\Avtors{Сатин~Я.\,А.} см.\ Усов~И.\,А.&&\\
\Avtors{Сидоренко~А.\,П.} см.\ Кабанов~Ю.\,М.&&\\
\Avtors{Синицын~И.\,Н.} Аналитическое моделирование распределений с~инвариантной 
мерой в~стохастических системах, не разрешенных относительно 
производных&\raisebox{-12pt}[0pt][0pt]{1}&\raisebox{-12pt}[0pt][0pt]{\hphantom{1}2--10}\\
\Avtors{Смирнов~Д.\,В.} см.\ Грушо~А.\,А.&&\\
\Avtors{Сопин~Э.\,С., Маслов~А.\,Р., Шоргин~В.\,С., Бегишев~В.\,О.} Моделирование 
настойчивого поведения пользователей в~сетях 5G NR с~адаптацией скорости 
и~блокировками&\raisebox{-12pt}[0pt][0pt]{3}&\raisebox{-12pt}[0pt][0pt]{25--32}\\
\Avtors{Степанов~Е.\,П.} см.\ Шестаков~О.\,В.&&\\
\Avtors{Стрижов~В.\,В.} см.\ Нейчев~Р.\,Г.&&\\
\Avtors{Ступников~С.\,А.} см.\ Сазонтьев В.\,В.&&\\
\Avtors{Тимонина~Е.\,Е.} см.\ Грушо~А.\,А.&&\\
\Avtors{Тимонина~Е.\,Е.} см.\ Грушо~А.\,А.&&\\
\Avtors{Тимонина~Е.\,Е.} см.\ Грушо~А.\,А.&&\\
\Avtors{Торшин~И.\,Ю.} О~задачах оптимизации, возникающих при применении 
топологического анализа данных к~поиску алгоритмов прогнозирования с~фиксированными 
корректорами&\raisebox{-24pt}[0pt][0pt]{2}&\raisebox{-24pt}[0pt][0pt]{\hphantom{1}2--10}\\
\Avtors{Торшин~И.\,Ю.} О~формировании множеств прецедентов на основе таблиц 
разнородных признаковых описаний методами топологической теории анализа 
данных&\raisebox{-12pt}[0pt][0pt]{3}&\raisebox{-12pt}[0pt][0pt]{2--7}\\
\Avtors{Тубольцев~В.\,П., Лапко~А.\,В., Лапко~В.\,А.} Непараметрический алгоритм 
автоматической классификации данных дистанционного зондирования&\raisebox{-12pt}[0pt][0pt]{4}&\raisebox{-12pt}[0pt][0pt]{23--31}\\
\Avtors{Усов~И.\,А., Сатин~Я.\,А., Зейфман~А.\,И.} О~скорости сходимости и~предельных 
характеристиках для одного обобщенного процесса рождения и~гибели&\raisebox{-12pt}[0pt][0pt]{3}&\raisebox{-12pt}[0pt][0pt]{49--57}\\
\Avtors{Ушаков~В.\,Г., Ушаков~Н.\,Г.} Критерии нормальности вероятностного 
распределения при округленных данных&\raisebox{-12pt}[0pt][0pt]{1}&\raisebox{-12pt}[0pt][0pt]{18--27}\\
\Avtors{Ушаков~В.\,Г.} см.\ Берговин~А.\,К.&&\\
\Avtors{Ушаков~Н.\,Г.} см.\ Ушаков~В.\,Г.&&\\
\Avtors{Филиппских~С.\,Л.} см.\ Архипов~П.\,О.&&\\
\end{tabular}
}

\pagebreak

\def\leftkol{АВТОРСКИЙ УКАЗАТЕЛЬ ЗА 2023 г.} % ENGLISH ABSTRACTS}

\def\rightkol{АВТОРСКИЙ УКАЗАТЕЛЬ ЗА 2023 г.} %ENGLISH ABSTRACTS}

%\thispagestyle{myheadings}
\def\leftfootline{\small{\textbf{\thepage}
\hfill ИНФОРМАТИКА И ЕЁ ПРИМЕНЕНИЯ\ \ \ том~17\ \ \ выпуск~4\ \ \ 2023}
}%
 \def\rightfootline{\small{ИНФОРМАТИКА И ЕЁ ПРИМЕНЕНИЯ\ \ \ том~17\ \ \ выпуск~4\ \ \ 2023
 \hfill \textbf{\thepage}}}


\noindent
{\tabcolsep=3pt
\begin{tabular}{p{394pt}cc}
&\textbf{Вып.} & \textbf{Стр.}\\[3pt]
\Avtors{Френкель С.\,Л., Захаров В.\,Н.} Модели учета влияния статистических 
характеристик трафика вычислительных сетей на эффективность прогнозирования 
средствами машинного обучения&\raisebox{-24pt}[0pt][0pt]{4}&\raisebox{-24pt}[0pt][0pt]{71--80}\\
\Avtors{Хачумов~В.\,М.} см.\ Мелехин~В.\,Б.&&\\
\Avtors{Хачумов~М.\,В.} см.\ Мелехин~В.\,Б.&&\\
\Avtors{Цуканов~М.\,В.} см.\ Архипов~П.\,О.&&\\
\Avtors{Шестаков~О.\,В., Степанов~Е.\,П.} Нелинейная регуляризация обращения линейных 
однородных операторов с помощью метода блочной пороговой обработки&\raisebox{-12pt}[0pt][0pt]{4}&\raisebox{-12pt}[0pt][0pt]{2--8}\\
\Avtors{Шестаков~О.\,В.} см.\ Воронцов~М.\,О.&&\\
\Avtors{Шестаков~О.\,В.} см.\ Кудрявцев~А.\,А.&&\\
\Avtors{Шибаев~И.\,А.} см.\ Нейчев~Р.\,Г.&&\\
\Avtors{Шнурков П.\,В.} Решение задачи оптимального управления запасом непрерывного 
продукта в стохастической модели регенерации со случайными стоимостными 
характеристиками&\raisebox{-24pt}[0pt][0pt]{4}&\raisebox{-24pt}[0pt][0pt]{48--56}\\
\Avtors{Шоргин~В.\,С.} см.\ Самуйлов~А.\,К.&&\\
\Avtors{Шоргин~В.\,С.} см.\ Сопин~Э.\,С.&&\\
\Avtors{Шоргин~С.\,Я.} см.\ Аду~К.\,И.\,Б.&&\\
\Avtors{Шоргин~С.\,Я.} см.\ Грушо~А.\,А.&&\\
\Avtors{Шоргин~С.\,Я.} см.\ Грушо~А.\,А.&&\\
\Avtors{Шоргин~С.\,Я.} см.\ Иванова Д.\,В.&&\\


\end{tabular}
}

%\thispagestyle{myheadings}
\def\leftfootline{\small{\textbf{\thepage}
\hfill ИНФОРМАТИКА И ЕЁ ПРИМЕНЕНИЯ\ \ \ том~17\ \ \ выпуск~4\ \ \ 2023}
}%
 \def\rightfootline{\small{ИНФОРМАТИКА И ЕЁ ПРИМЕНЕНИЯ\ \ \ том~17\ \ \ выпуск~4\ \ \ 2023
 \hfill \textbf{\thepage}}}

 \label{end\stat}

\newpage

\def\stat{cont-e}
{%\hrule\par
%\vskip 7pt % 7pt
\raggedleft\Large \bf%\baselineskip=3.2ex
2\,0\,2\,3\ \ A\,U\,T\,H\,O\,R\ \ I\,N\,D\,E\,X \vskip 17pt
 \hrule
 \par
\vskip 21pt plus 6pt minus 3pt }

\label{st\stat}

\def\tit{\ }

\def\aut{\ }
\def\auf{\ }

\def\leftkol{\ } %2021 AUTHOR INDEX} % ENGLISH ABSTRACTS}

\def\rightkol{\ } %2021 AUTHOR INDEX} %ENGLISH ABSTRACTS}

\titele{\tit}{\aut}{\auf}{\leftkol}{\rightkol}
\addcontentsline{toc}{subsection}{\textrm\textbf 2023 Author Index}

\def\leftfootline{\small{\textbf{\thepage}
\hfill INFORMATIKA I EE PRIMENENIYA~--- INFORMATICS AND APPLICATIONS\ \ \ 2023\
\ \ volume~17\ \ \ issue\ 4}
}%
 \def\rightfootline{\small{INFORMATIKA I EE PRIMENENIYA~--- INFORMATICS AND APPLICATIONS\ \ \ 2023\ \ \ volume~17\ \ \ issue\ 4
\hfill \textbf{\thepage}}}

\vspace*{-24pt}

\noindent
{\tabcolsep=3pt
\begin{tabular}{p{395.89pt}cc}
&\textbf{Issue} & \textbf{Page}\\[6pt]
\Avtors{Adou~K.\,Y.\,B., Markova~E.\,V., Gaidamaka~Yu.\,V., and~Shorgin~S.\,Ya.} 
Preemption-based prioritization scheme for network resources slicing in 5G 
systems&\raisebox{-12pt}[0pt][0pt]{1}&\raisebox{-12pt}[0pt][0pt]{\hphantom{1}96--106}\\
\Avtors{Agalarov~Ya.\,M.} Optimization of a queue-length dependent additional server in the 
multiserver queue&\raisebox{-12pt}[0pt][0pt]{1}&\raisebox{-12pt}[0pt][0pt]{89--95}\\
\Avtors{Agalarov~Ya.\,M.} Optimization of the buffer memory allocation scheme of the packet 
switching node&\raisebox{-12pt}[0pt][0pt]{3}&\raisebox{-12pt}[0pt][0pt]{39--48}\\
\Avtors{Agasandyan~G.\,A.} Multidimensional butterflies in problems of optimization on CC-VaR&1&107--115\\
\Avtors{Arkhipov~P.\,O., Philippskih~S.\,L., and~Tsukanov~M.\,V.} Development of a~new model 
of a~step convolutional neural network for classification of anomalies on panoramas&\raisebox{-12pt}[0pt][0pt]{1}&\raisebox{-12pt}[0pt][0pt]{50--56}\\
\Avtors{Begishev~V.\,O.} see Sopin~E.\,S.&&\\
\Avtors{Bergovin~A.\,K. and~Ushakov~V.\,G.} Analysis of the queueing systems with mixed 
priorities&2&57--61\\
\Avtors{Borisov~A.\,V.} Market with Markov jump volatility I: Price of risk monitoring as an 
optimal filtering problem&\raisebox{-12pt}[0pt][0pt]{2}&\raisebox{-12pt}[0pt][0pt]{27--33}\\
\Avtors{Borisov~A.\,V.} Market with Markov jump volatility~II: Algorithm of derivative fair price 
calculation&3&18--24\\
\Avtors{Borisov A.\,V.} Market with Markov jump volatility III: Price of risk monitoring algorithm 
given discrete-time observations of asset prices&\raisebox{-12pt}[0pt][0pt]{4}&\raisebox{-12pt}[0pt][0pt]{\hphantom{9}9--16}\\
\Avtors{Bosov~A.\,V.} Nonlinear dynamic system state optimal filtering by observations with 
random delays&\raisebox{-12pt}[0pt][0pt]{3}&\raisebox{-12pt}[0pt][0pt]{\hphantom{1}8--17}\\
\Avtors{Bosov~A.\,V.} Robustness investigation of the numerical approximation of the Wonham 
filter&2&41--49\\
\Avtors{Bosov~A.\,V. and~Ignatov~A.\,N.} On the problem of assessing and analyzing traffic 
accidents risk on the rail transport&\raisebox{-12pt}[0pt][0pt]{1}&\raisebox{-12pt}[0pt][0pt]{73--82}\\
\Avtors{Bosov~A.\,V. and Ivanov~A.\,V.} Multifactor classification technology of mathematical 
content of e-learning system&\raisebox{-12pt}[0pt][0pt]{4}&\raisebox{-12pt}[0pt][0pt]{32--41}\\
\Avtors{Djukova~A.\,P.} see Djukova~E.\,V.&&\\
\Avtors{Djukova~E.\,V., Masliakov~G.\,O., and Djukova~A.\,P.} Logical methods of correct data 
classification&3&64--70\\
\Avtors{Dulin~S.\,K.} see Rozenberg~I.\,N.&&\\
\Avtors{Dulina~N.\,G.} see Rozenberg~I.\,N.&&\\
\Avtors{Frenkel~S.\,L. and Zakharov~V.\,N.} Models for study of the influence of statistical 
characteristics of computer networks traffic on the efficiency of prediction by machine learning 
tools&\raisebox{-12pt}[0pt][0pt]{4}&\raisebox{-12pt}[0pt][0pt]{71--80}\\
\Avtors{Gaidamaka~Yu.\,V.} see Adou~K.\,Y.\,B.&&\\
\Avtors{Gaidamaka~Yu.\,V.} see Ivanova~D.\,V.&&\\
\Avtors{Gaidamaka~Yu.\,V.} see Samouylov~A.\,K.&&\\
\Avtors{Garimella~R.\,M.} see Razumchik~R.\,V.&&\\
\Avtors{Goncharov~A.\,A.} Parallel corpus annotation: Approaches and directions for 
development&4&81--87\\
\Avtors{Gorbunov~S.\,A.} see Gorshenin~A.\,K.&&\\
\Avtors{Gorshenin~A.\,K., Gorbunov~S.\,A., and Volkanov~D.\,Yu.} Toward clustering of 
network computing infrastructure objects based on analysis of statistical anomalies in network 
traffic&\raisebox{-12pt}[0pt][0pt]{3}&\raisebox{-12pt}[0pt][0pt]{76--87}\\
\Avtors{Grusho~A.\,A., Grusho~N.\,A., Zabezhailo~M.\,I., Kulchenkov~V.\,V., Timonina~E.\,E., 
and~Shorgin~S.\,Ya.} Causal relationships in classification problems&\raisebox{-12pt}[0pt][0pt]{1}&\raisebox{-12pt}[0pt][0pt]{43--49}\\
\Avtors{Grusho~A.\,A., Grusho~N.\,A., Zabezhailo~M.\,I., Smirnov~D.\,V., and Timonina~E.\,E.} 
Classification by cause-and-effect relationships&\raisebox{-12pt}[0pt][0pt]{3}&\raisebox{-12pt}[0pt][0pt]{71--75}\\
\Avtors{Grusho~A.\,A., Grusho~N.\,A., Zabezhailo~M.\,I., Timonina~E.\,E., 
and~Shorgin~S.\,Ya.} Complex cause-and-effect relationships&\raisebox{-12pt}[0pt][0pt]{2}&\raisebox{-12pt}[0pt][0pt]{84--89}\\
\Avtors{Grusho~N.\,A.} see Grusho~A.\,A.&&\\
\Avtors{Grusho~N.\,A.} see Grusho~A.\,A.&&\\
\Avtors{Grusho~N.\,A.} see Grusho~A.\,A.&&\\
\end{tabular}
}
\pagebreak

\def\leftfootline{\small{\textbf{\thepage}
\hfill INFORMATIKA I EE PRIMENENIYA~--- INFORMATICS AND APPLICATIONS\ \ \ 2023\
\ \ volume~17\ \ \ issue\ 4}
}%
 \def\rightfootline{\small{INFORMATIKA I EE PRIMENENIYA~---
INFORMATICS AND APPLICATIONS\ \ \ 2023\ \ \ volume~17\ \ \ issue\ 4
\hfill \textbf{\thepage}}}

\def\leftkol{2023 AUTHOR INDEX} % ENGLISH ABSTRACTS}

\def\rightkol{2023 AUTHOR INDEX} %ENGLISH ABSTRACTS}


\noindent
{\tabcolsep=3pt
\begin{tabular}{p{395.5pt}cc}
&\textbf{Issue} & \textbf{Page}\\[6pt]
\Avtors{Ignatov~A.\,N.} see Bosov~A.\,V.&&\\
\Avtors{Inkova O.\,Yu. and Kruzhkov M.\,G.} Evaluating the degree of discourse relations 
semantic affinity: Methods and instruments&\raisebox{-12pt}[0pt][0pt]{4}&\raisebox{-12pt}[0pt][0pt]{88--95}\\
\Avtors{Inkova~O.\,Yu. and Kruzhkov~M.\,G.} Evaluation criteria for discourse relations semantic 
affinity&3&100--106\\
\Avtors{Kruzhkov~M.\,G.} see Inkova~O.\,Yu.&&\\
\Avtors{Ivanov~A.\,V.} see Bosov~A.\,V.&&\\
\Avtors{Ivanova~D.\,V., Markova~E.\,V., Shorgin~S.\,Ya., and~Gaidamaka~Yu.\,V.} Priority-based 
eMBB and URLLC traffic coexistence models in 5G NR industrial deployments&\raisebox{-12pt}[0pt][0pt]{4}&\raisebox{-12pt}[0pt][0pt]{64--70}\\
\Avtors{Kabanov~Yu.\,M. and~Sidorenko~A.\,P.} An axiomatic viewpoint on the Rogers--Veraart 
and Suzuki--Elsinger models of systemic risk&\raisebox{-12pt}[0pt][0pt]{1}&\raisebox{-12pt}[0pt][0pt]{11--17}\\
\Avtors{Karpov~V.\,I.} see Nuriev~V.\,A.&&\\
\Avtors{Khachumov~M.\,V.} see Melekhin~V.\,B.&&\\
\Avtors{Khachumov~V.\,M.} see Melekhin~V.\,B.&&\\
\Avtors{Kirikov~I.\,A.} see Listopad~S.\,V.&&\\
\Avtors{Korolev~D.\,O. and Maleev~O.\,G.} Efficiency of binary neural networks for object 
detection on an image&\raisebox{-12pt}[0pt][0pt]{3}&\raisebox{-12pt}[0pt][0pt]{88--92}\\
\Avtors{Kovalyov~S.\,P.} The monad of diagrams as a mathematical metamodel of systems 
engineering&2&11--17\\
\Avtors{Krivenko~M.\,P.} Criteria for choosing the factorization model dimensionality&2&50--56\\
\Avtors{Kruzhkov M.\,G.} see Inkova O.\,Yu.&&\\
\Avtors{Kudryavtsev~A.\,A. and Shestakov~O.\,V.} A~method for estimating parameters of the 
gamma-exponential distribution from a~sample with weakly dependent components&\raisebox{-12pt}[0pt][0pt]{3}&\raisebox{-12pt}[0pt][0pt]{58--63}\\
\Avtors{Kulchenkov~V.\,V.} see Grusho~A.\,A.&&\\
\Avtors{Lapko~A.\,V.} see Tuboltsev V.\,P.&&\\
\Avtors{Lapko~V.\,A.} see Tuboltsev V.\,P.&&\\
\Avtors{Leri~M.\,M.} An average distance in the power-law configuration graphs&1&28--34\\
\Avtors{Listopad~S.\,V. and~Kirikov~I.\,A.} Fuzzy rules based method for agent conflict 
management in hybrid intelligent multiagent systems&\raisebox{-12pt}[0pt][0pt]{1}&\raisebox{-12pt}[0pt][0pt]{66--72}\\
\Avtors{Malashenko~Yu.\,E. and~Nazarova~I.\,A.} Estimates of the resource distribution in the 
multiuser network with equal internodal loads&\raisebox{-12pt}[0pt][0pt]{1}&\raisebox{-12pt}[0pt][0pt]{83--88}\\
\Avtors{Malashenko~Yu.\,E. and Nazarova~I.\,A.} Multiuser network load analysis by splitting 
flows along the shortest routes&\raisebox{-12pt}[0pt][0pt]{3}&\raisebox{-12pt}[0pt][0pt]{33--38}\\
\Avtors{Maleev~O.\,G.} see Korolev~D.\,O.&&\\
\Avtors{Markova~E.\,V.} see Adou~K.\,Y.\,B.&&\\
\Avtors{Markova~E.\,V.} see Ivanova~D.\,V.&&\\
\Avtors{Masliakov~G.\,O.} see Djukova~E.\,V.&&\\
\Avtors{Maslov~A.\,R.} see Sopin~E.\,S.&&\\
\Avtors{Melekhin~V.\,B., Khachumov~V.\,M., and~Khachumov~M.\,V.} Self-learning of 
autonomous intelligent robots in the process of search and explore activities&\raisebox{-12pt}[0pt][0pt]{2}&\raisebox{-12pt}[0pt][0pt]{78--83}\\
\Avtors{Nazarova~I.\,A.} see Malashenko~Yu.\,E.&&\\
\Avtors{Nazarova~I.\,A.} see Malashenko~Yu.\,E.&&\\
\Avtors{Neychev~R.\,G., Shibaev~I.\,A., and~Strijov~V.\,V.} Optimal spanning tree reconstruction 
in symbolic regression&\raisebox{-12pt}[0pt][0pt]{1}&\raisebox{-12pt}[0pt][0pt]{35--42}\\
\Avtors{Nuriev~V.\,A. and~Karpov~V.\,I.} Methodology of the corpus-based studies in the field of 
contrastive punctuation&\raisebox{-12pt}[0pt][0pt]{2}&\raisebox{-12pt}[0pt][0pt]{90--95}\\
\Avtors{Peshkova~I.\,V.} Bounds of the workload in a~multiclass retrial queue with exponential 
services&\raisebox{-12pt}[0pt][0pt]{4}&\raisebox{-12pt}[0pt][0pt]{57--63}\\
\Avtors{Philippskih~S.\,L.} see Arkhipov~P.\,O.&&\\
\Avtors{Platonova~A.\,A.} see Samouylov~A.\,K.&&\\
\Avtors{Rabinovich Ya.\,I.} Procedure of constructing a~Pareto set for differentiable criteria 
functions&4&17--22\\
\Avtors{Razumchik~R.\,V., Rumyantsev~A.\,S., and~Garimella~R.\,M.} A~queueing system for 
performance evaluation of a~Markovian supercomputer model&\raisebox{-12pt}[0pt][0pt]{2}&\raisebox{-12pt}[0pt][0pt]{62--70}\\
\Avtors{Rozenberg~I.\,N., Dulin~S.\,K., and~Dulina~N.\,G.} Modeling the structure of 
interoperability by means of structural consistency&\raisebox{-12pt}[0pt][0pt]{1}&\raisebox{-12pt}[0pt][0pt]{57--65}\\
\Avtors{Rumovskaya~S.\,B.} Selection of specialists in the organization of collective solving 
problems&2&\hphantom{1}96--103\\
\end{tabular}
}
\pagebreak

\def\leftfootline{\small{\textbf{\thepage}
\hfill INFORMATIKA I EE PRIMENENIYA~--- INFORMATICS AND APPLICATIONS\ \ \ 2023\
\ \ volume~17\ \ \ issue\ 4}
}%
 \def\rightfootline{\small{INFORMATIKA I EE PRIMENENIYA~---
INFORMATICS AND APPLICATIONS\ \ \ 2023\ \ \ volume~17\ \ \ issue\ 4
\hfill \textbf{\thepage}}}

\def\leftkol{2023 AUTHOR INDEX} % ENGLISH ABSTRACTS}

\def\rightkol{2023 AUTHOR INDEX} %ENGLISH ABSTRACTS}


\noindent
{\tabcolsep=3pt
\begin{tabular}{p{395.5pt}cc}
&\textbf{Issue} & \textbf{Page}\\[6pt]
\Avtors{Rumyantsev~A.\,S.} see Razumchik~R.\,V.&&\\
\Avtors{Samouylov~A.\,K., Platonova~A.\,A., Shorgin~V.\,S., and~Gaidamaka~Yu.\,V.} On 
modeling the effects of multicast traffic servicing in 5G NR networks&\raisebox{-12pt}[0pt][0pt]{2}&\raisebox{-12pt}[0pt][0pt]{71--77}\\
\Avtors{Satin~Y.\,A.} see Usov~I.\,A.&&\\
\Avtors{Sazontev V.\,V., Stupnikov~S.\,A., and~Zakharov~V.\,N.} An extensible approach to data 
fusion in~distributed computing environments&\raisebox{-12pt}[0pt][0pt]{4}&\raisebox{-12pt}[0pt][0pt]{42--47}\\
\Avtors{Shestakov~O.\,V. and Stepanov~E.\,P.} Nonlinear regularization of the inversion of linear 
homogeneous operators using the block thresholding method&\raisebox{-12pt}[0pt][0pt]{4}&\raisebox{-12pt}[0pt][0pt]{2--8}\\
\Avtors{Shestakov~O.\,V.} see Kudryavtsev~A.\,A.&&\\
\Avtors{Shestakov~O.\,V.} see Vorontsov~M.\,O.&&\\
\Avtors{Shibaev~I.\,A.} see Neychev~R.\,G.&&\\
\Avtors{Shnurkov P.\,V.} Solution of the problem of~optimal control of~the stock of a~continuous 
product in a~stochastic model of regeneration with random cost characteristics&\raisebox{-12pt}[0pt][0pt]{4}&\raisebox{-12pt}[0pt][0pt]{48--56}\\
\Avtors{Shorgin~S.\,Ya.} see Adou~K.\,Y.\,B.&&\\
\Avtors{Shorgin~S.\,Ya.} see Grusho~A.\,A.&&\\
\Avtors{Shorgin~S.\,Ya.} see Grusho~A.\,A.&&\\
\Avtors{Shorgin~S.\,Ya.} see Ivanova~D.\,V.&&\\
\Avtors{Shorgin~V.\,S.} see Samouylov~A.\,K.&&\\
\Avtors{Shorgin~V.\,S.} see Sopin~E.\,S.&&\\
\Avtors{Sidorenko~A.\,P.} see Kabanov~Yu.\,M.&&\\
\Avtors{Sinitsyn~I.\,N.} Analytical modeling of distributions with invariant measure in stochastic 
systems with unsolved derivatives&\raisebox{-12pt}[0pt][0pt]{1}&\raisebox{-12pt}[0pt][0pt]{\hphantom{1}2--10}\\
\Avtors{Smirnov~D.\,V.} see Grusho~A.\,A.&&\\
\Avtors{Sopin~E.\,S., Maslov~A.\,R., Shorgin~V.\,S., and Begishev~V.\,O.} Modeling insistent 
user behavior in~5G New Radio networks with rate adaptation and blockage&\raisebox{-12pt}[0pt][0pt]{3}&\raisebox{-12pt}[0pt][0pt]{25--32}\\
\Avtors{Stepanov~E.\,P.} see Shestakov~O.\,V.&&\\
\Avtors{Strijov~V.\,V.} see Neychev~R.\,G.&&\\
\Avtors{Stupnikov~S.\,A.} see Sazontev V.\,V.&&\\
\Avtors{Timonina~E.\,E.} see Grusho~A.\,A.&&\\
\Avtors{Timonina~E.\,E.} see Grusho~A.\,A.&&\\
\Avtors{Timonina~E.\,E.} see Grusho~A.\,A.&&\\
\Avtors{Torshin~I.\,Yu.} On optimization problems arising from the application of topological data 
analysis to the search for forecasting algorithms with fixed correctors&\raisebox{-12pt}[0pt][0pt]{2}&\raisebox{-12pt}[0pt][0pt]{\hphantom{1}2--10}\\
\Avtors{Torshin~I.\,Yu.} On the formation of sets of precedents based on tables of heterogeneous 
feature descriptions by methods of topological theory of data analysis&\raisebox{-12pt}[0pt][0pt]{3}&\raisebox{-12pt}[0pt][0pt]{2--7}\\
\Avtors{Tsukanov~M.\,V.} see Arkhipov~P.\,O.&&\\
\Avtors{Tuboltsev V.\,P., Lapko~A.\,V., and~Lapko~V.\,A.} Nonparametric algorithm for 
automatic classification of remote sensing data&\raisebox{-12pt}[0pt][0pt]{4}&\raisebox{-12pt}[0pt][0pt]{23--31}\\
\Avtors{Ushakov~N.\,G.} see Ushakov~V.\,G.&&\\
\Avtors{Ushakov~V.\,G. and~Ushakov~N.\,G.} Tests for normality of the probabilistic distribution 
when data are rounded&\raisebox{-12pt}[0pt][0pt]{1}&\raisebox{-12pt}[0pt][0pt]{18--27}\\
\Avtors{Ushakov~V.\,G.} see Bergovin~A.\,K.&&\\
\Avtors{Usov~I.\,A., Satin~Y.\,A., and Zeifman~A.\,I.} On the rate of convergence and limiting 
characteristics for one quasi-birth--death process&\raisebox{-12pt}[0pt][0pt]{3}&\raisebox{-12pt}[0pt][0pt]{49--57}\\
\Avtors{Vakulenko~V.\,V. and Zatsman~I.\,M.} Formalized description of statistical information 
processing in databases&\raisebox{-12pt}[0pt][0pt]{3}&\raisebox{-12pt}[0pt][0pt]{93--99}\\
\Avtors{Vasilyev~N.\,S.} Multiplayers' games compositional structure in the monoidal category of 
binary relations&\raisebox{-12pt}[0pt][0pt]{2}&\raisebox{-12pt}[0pt][0pt]{18--26}\\
\Avtors{Volkanov~D.\,Yu.} see Gorshenin~A.\,K.&&\\
\Avtors{Vorontsov~M.\,O. and~Shestakov~O.\,V.} Mean-square risk of the FDR procedure under 
weak dependence&\raisebox{-12pt}[0pt][0pt]{2}&\raisebox{-12pt}[0pt][0pt]{34--40}\\
\Avtors{Zabezhailo~M.\,I.} see Grusho~A.\,A.&&\\
\Avtors{Zabezhailo~M.\,I.} see Grusho~A.\,A.&&\\
\end{tabular}
}
\pagebreak

\def\leftfootline{\small{\textbf{\thepage}
\hfill INFORMATIKA I EE PRIMENENIYA~--- INFORMATICS AND APPLICATIONS\ \ \ 2023\
\ \ volume~17\ \ \ issue\ 4}
}%
 \def\rightfootline{\small{INFORMATIKA I EE PRIMENENIYA~---
INFORMATICS AND APPLICATIONS\ \ \ 2023\ \ \ volume~17\ \ \ issue\ 4
\hfill \textbf{\thepage}}}

\def\leftkol{2023 AUTHOR INDEX} % ENGLISH ABSTRACTS}

\def\rightkol{2023 AUTHOR INDEX} %ENGLISH ABSTRACTS}


\noindent
{\tabcolsep=3pt
\begin{tabular}{p{395.5pt}cc}
&\textbf{Issue} & \textbf{Page}\\[6pt]
\Avtors{Zabezhailo~M.\,I.} see Grusho~A.\,A.&&\\
\Avtors{Zakharov~V.\,N.} see Frenkel~S.\,L.&&\\
\Avtors{Zakharov~V.\,N.} see Sazontev V.\,V.&&\\
\Avtors{Zatsman~I.\,M.} On the scientific paradigm of informatics: Data, information, and 
knowledge&1&116--125\\
\Avtors{Zatsman I.\,M.} Scientific paradigm of informatics: Classification of domain 
objects&4&\hphantom{9}96--103\\
\Avtors{Zatsman~I.\,M.} Transformation of the Ackoff's hierarchy in the scientific paradigm of 
informatics&3&107--113\\
\Avtors{Zatsman~I.\,M.} see Vakulenko~V.\,V.&&\\
\Avtors{Zeifman~A.\,I.} see Usov~I.\,A.&&\\
\end{tabular}
}

%\thispagestyle{myheadings}
\def\leftfootline{\small{\textbf{\thepage}
\hfill INFORMATIKA I EE PRIMENENIYA~--- INFORMATICS AND APPLICATIONS\ \ \ 2023\
\ \ volume~17\ \ \ issue\ 4}
}%
 \def\rightfootline{\small{INFORMATIKA I EE PRIMENENIYA~---
INFORMATICS AND APPLICATIONS\ \ \ 2023\ \ \ volume~17\ \ \ issue\ 4
\hfill \textbf{\thepage}}}

 \label{end\stat}

\newpage