\def\stat{rabinovich}

\def\tit{ПРОЦЕДУРА ПОСТРОЕНИЯ МНОЖЕСТВА ПАРЕТО ДЛЯ~ДИФФЕРЕНЦИРУЕМЫХ 
КРИТЕРИАЛЬНЫХ ФУНКЦИЙ}

\def\titkol{Процедура построения множества Парето для~дифференцируемых 
критериальных функций}

\def\aut{Я.\,И.~Рабинович$^1$}

\def\autkol{Я.\,И.~Рабинович}

\titel{\tit}{\aut}{\autkol}{\titkol}

\index{Рабинович Я.\,И.}
\index{Rabinovich Ya.\,I.}


%{\renewcommand{\thefootnote}{\fnsymbol{footnote}} \footnotetext[1]
%{Работа выполнялась с~использованием инфраструктуры Центра коллективного пользования <<Высокопроизводительные вы\-чис\-ле\-ния и~большие данные>> 
%(ЦКП <<Информатика>>) ФИЦ ИУ РАН (г.~Москва).}}


\renewcommand{\thefootnote}{\arabic{footnote}}
\footnotetext[1]{Федеральный исследовательский центр <<Информатика и~управление>> Российской академии наук;
\mbox{jacrabin@rambler.ru}}

\vspace*{-12pt}



\Abst{Универсальная вычислительная процедура многокритериальной 
оптимизации позволяет аппроксимировать множество Парето при предъявлении 
различных требований к~вектору частных критериев эффективности и~множеству 
допустимых решений. В~настоящей работе предполагается, что частные критерии 
эффективности псевдовогнуты в~открытой окрестности компактного выпуклого 
множества допустимых решений, которое может быть задано дифференцируемыми 
функциональными ограничениями. Для построения на основе универсальной 
процедуры конкретных численных методов аппроксимации множества Парето 
предлагается правило выбора начального приближения и~правило перехода от 
текущего опорного решения к~последующему.}

\KW{многокритериальная оптимизация; множество Парето; численные методы 
аппроксимации; универсальная процедура}

\DOI{10.14357/19922264230403}{NEZRGD}
  
\vspace*{-4pt}


\vskip 10pt plus 9pt minus 6pt

\thispagestyle{headings}

\begin{multicols}{2}

\label{st\stat}

\section{Введение}

\vspace*{-3pt}

  Задача построения множества эффективных векторных оценок (множества 
Парето) чрезвычайно актуальна в~самых разнообразных отраслях 
деятельности, таких как разработка сложных сис\-тем технического 
назначения, перспективное планирование или социология. В~настоящей 
работе в~предположении дифференцируемости критериальных функций на 
основе универсальной процедуры~[1] предложен конкретный численный 
метод аппроксимации множества Парето.

\vspace*{-9pt}
  
\section{Постановка задачи}

\vspace*{-3pt}

  Пусть в~$s$-мерном евклидовом пространстве~$\mathbb{R}^s$ задана 
  $m$-мер\-ная непрерывная век\-тор-функ\-ция
  \begin{equation}
  w(x)\in \mathbb{R}^m\,,
  \label{e1-r}
  \end{equation}
образующая вектор частных критериев эффективности, принимающий на 
непустом компактном множестве допустимых решений 
\begin{equation}
X\subset \mathbb{R}^s
\label{e2-r}
\end{equation}
положительные значения $w(x)\hm>0$, удовлетворяя соотношениям:
\begin{equation}
\left.
\begin{array}{c}
w(x)\in w(X)\subset \mathrm{int}\, \mathbb{R}_+^m\,;\\[3pt]
w(X)=\left\{ u\in \mathbb{R}^m\vert u=w(x),\ x\in X\right\}\,;\\[3pt]
\mathbb{R}_+^m =\left\{ u\in \mathbb{R}^m\vert u\geq 0\right\}\,,
\end{array}
\right\}
\label{e3-r}
\end{equation}
так что множество достижимых векторных оценок $w(X)$ из~(\ref{e3-r}) 
принадлежит внутренности $\mathrm{int}\, \mathbb{R}_+^m$ не\-от\-ри\-ца\-тель\-но\-го 
ортанта~$\mathbb{R}_+^m$. Каждый критерий набора
%\begin{equation}
$\left\{ w_k(x)\right\}_{k\in I}$, $I\hm=\left\{ k\vert 1\leq k\leq m\right\}$,
%\label{e4-r}
%\end{equation}
желательно увеличивать на множестве допустимых решений~(2).
% $X\hm\subset  \mathbb{R}^s$.

\smallskip

\noindent
  \textbf{Определение~1.} Векторная оценка $w\hm\in w(X)$ 
\textit{эффективна (слабо эффективна)}, если для всякой векторной оценки 
$u\hm\in w(X)$ сис\-те\-ма неравенств $u\hm\geq w$ несовместна при 
условии, что хотя бы одно неравенство строгое (все неравенства строгие).
  
  Векторная оценка $w\hm\in w(X)$ \textit{доминируема} или 
\textit{определенно неэффективна}, если существует векторная оценка 
$u\hm\in w(X)$, $u\hm> w$.
  
  Всякое допустимое решение $x\hm\in X$, доставляющее эффективное 
(слабо эффективное, доминируемое) значение вектора~$w(x)$, называется 
эффективным (слабо эффективным, доминируемым) \mbox{решением}.
  
  В согласии с~определением~1 множества эффективных 
($X_{e}$), слабо эффективных ($X_0$) и~доминируемых 
($X_{\mathrm{д}}$) решений из множества допустимых решений ($X$) 
подчиняются соотношениям:
  $$
  X_{e}\subset X_0\subset X\,;\ X_0\cap 
X_{\mathrm{д}}=\emptyset\,;\ X=X_0\cup X_{\mathrm{д}}\,,
  $$
где $\emptyset$~--- пустое множество; $\cup (\cap)$~--- символ объединения 
(пересечения) множеств. Соответственно множества 
  эффективных $w(X_{e}$), слабо 
эффективных $w(X_0)$, доминируемых $w(X_{\mathrm{д}})$ и~достижимых  $w(X)$ векторных оценок 
удовлетворяют соотношениям:

\noindent
\begin{multline*}
 w(X_{e}) \subset  w(X_{0}) \subset w(X);\
w(X_0)\cap  w(X_{\mathrm{д}})=\emptyset\,;\\
w(X)=w(X_0)\cup w (X_{\mathrm{д}}).
\end{multline*}
  
  В настоящей работе рассматриваются методы аппроксимации множества 
Парето на основе универсальной вычислительной процедуры~[1]. Введем
  
\pagebreak
  
  \noindent
  \textbf{Определение~2.} Если $\| v\|$~--- норма вектора $v\hm\in 
\mathbb{R}^m$, то величина
  $$
  D(W,U)=\sup\limits_{w\in W} \mathop{\mathrm{inf}}\limits_{u\in U} \| w-
u\|,\enskip \emptyset \not= W,\ U\subset \mathbf{R}^m,
  $$
называется \textit{отклонением} множества~$W$ от множества~$U$, 
а~величина
\begin{multline*}
\Delta (W,U) =\max\left\{ D(W,U), D(U,W)\right\}\,,\\
 \emptyset\not= W,\ U\subset \mathbb{R}^m\,,
\end{multline*}
называется \textit{расстоянием по Хаусдорфу} между множествами~$W$ и~$U$.

\smallskip
  
  \noindent
  \textbf{Определение~3.} Дифференцируемая на открытом множестве 
$A\hm\subset \mathbb{R}^s$ функция~$f$ называется \textit{псевдовогнутой} 
на $A\not= \emptyset$, если для любых~$x,y \hm\in A$ неравенство $\langle 
\nabla f(x), y\hm- x\rangle \hm\leq 0$ влечет неравенство $f(x)\hm\geq f(y)$, где 
$\nabla f$~--- градиент функции~$f$; $\langle \bullet,\bullet\rangle$~--- 
скалярное произведение векторов.
  
  Функция~$f$ называется \textit{псевдовыпуклой}, если функция $-f$ 
псевдовогнута.
  \smallskip
  
  \noindent
  \textbf{Следствие~1.} 
  \begin{enumerate}[1.]
  \item  Функция $\sum\nolimits_{k=1}^m f_k$ псевдовогнута, если ее 
слагаемые псевдовогнуты.
  \item  Точка локального максимума псевдовогнутой на выпуклом 
множестве $A\hm\subset \mathbb{R}^s$ функции~$f$ является точкой ее 
глобального максимума.
  \item Пусть функция~$f$ псевдовогнута на выпуклом множестве 
$A\hm\subset \mathbb{R}^s$ и~положительно определена на~$A$, так что 
$f(x)\hm>0$, $x\hm\in A$. Тогда
  \begin{itemize}
  \item[(а)] функция~$f^n$ псевдовогнута на~$A$, если целая степень 
$n\hm\geq 1$;
  \item[(б)] функция~$f^n$ псевдовыпукла на~$A$, если целая степень 
$n\hm\leq -1$.
  \end{itemize}
  \end{enumerate}
  
  \noindent
  \textbf{Определение~4.} Если непустое множество $X\hm\subset 
\mathbb{R}^s$ задано функциональными ограничениями
\begin{multline*}
%\begin{equation*}
  X=\left\{ x\in \mathbb{R}^s\vert v_k(x)\geq 0,\ k\in K\right\},\\
  K=\{ k\vert 1\leq k\leq n\},
%\end{equation*}
\end{multline*}
левые части которых дифференцируемы, то в~точке $x\hm\in X$ множество
$$
H(x)=\left\{ h\in \mathbb{R}^s\vert \langle \nabla v_k(x),h\rangle>0, \ k\in 
K(x)\right\},
$$
называется \textit{конусом внутренних направлений}, где $K(x)\hm= \{k\hm\in K\vert 
v_k(x)\hm=0\}$~--- множество индексов <<активных>> в~точке $x\hm\in X$ 
функциональных ограничений.

\smallskip

\noindent
\textbf{Следствие~2.} Если $X\hm\subset \mathbb{R}^s$~--- непустой 
выпуклый компакт, то выполняется включение $X\hm\subset \{x\}\hm+ 
\overline{H}(x)$, где многогранный конус~$\overline{H}(x)$~--- замыкание 
конуса внутренних направлений; $A\hm+B$~--- векторная сумма множеств.

  Укажем также известное полезное свойство псевдовогнутых функций~[2].
  
  \smallskip
  \noindent
  \textbf{Лемма~1.}\ \textit{Псевдовогнутая на выпуклом многограннике 
$A\hm= \mathrm{conv} \{a^q\}^r_{q=1}$ функция~$f$ достигает минимума в~крайней 
точке}:
$$
\min\limits_{x\in A} f(x)\hm= \min\limits_{1\leq q\leq r} f(a^q).
$$

\vspace*{-14pt}
   
\section{Универсальная процедура}

\vspace*{-3pt}

  Дадим в~согласии с~[1] строгое описание универсальной процедуры 
аппроксимация множества Парето.
  
  На непустом компактном множестве допустимых решений $X\hm\subset 
\mathbb{R}^s$ строится последовательность множеств 
$\{X_t\}^\infty_{t=1}\hm\subset X$. Если $X_1\hm\subset X$~--- произвольное 
начальное приближение и~известно множество~$X_t$, $t\hm\geq 1$, то 
следующее за ним множество~$X_{t+1}$ подчиняется соотношениям:
  \begin{multline}
  X_{t+1} =\bigcup\limits_{x\in X_t} X_{t+1}(x)\,,\\
  X_{t+1}(x)=\{x\} +\bigcup\limits_{J\in M_t(x)} \{ h(x,J)\} \subset X\,.
   \label{e6-r}
  \end{multline}
  
  В согласии с~(\ref{e6-r}) всякая опорная точка $x\hm\in X_t$ порождает на 
следующем ($t\hm+1$)-м уровне не\-пус\-тую векторную сумму множеств 
$X_{t+1}(x)$, причем на\-прав\-ле\-ние $h(x,J)$ перехода из опорной точки 
$x\hm\in X_t$ в~следующую точку $y\hm= x\hm+ h(x,J)\hm\in X_{t+1}(x)$, 
$J\hm\in M_t(x)$, удовлетворяет условиям:
  \begin{multline}
  h(x,\emptyset) =0\,,\enskip h(x,J)\not= 0\,,\\
   \emptyset\not= J\subset I 
=\{k\vert 1\leq k\leq m\},
  \label{e7-r}
  \end{multline}
где ненулевые направления $h(x,J)\not= 0$ подчиняются соотношениям
\begin{equation}
\left.
\begin{array}{c}
\!\!\!y=x+h(x,J)\in X\cap Y_J(x,\varepsilon) \cap Y(x,\varepsilon)\,,\\[3pt]
\emptyset \not= J\subset I= \{k\vert 1\leq k\leq m\}\,,\\[3pt]
\!\!\!Y_J(x,\varepsilon) =\left\{ y\in \mathbb{R}^s \left\vert \fr{w_k(y)}{w_k(x)} \right. 
\geq 1+\sigma\varepsilon^2\,,\right.\\[3pt]
\!\!\!\left.k\in J\,,\ \fr{w_k(y)}{w_k(x)}\geq 1-\sigma\varepsilon,\ k\in I\backslash 
J\right\}\,;\\[3pt]
\!\!\!Y(x,\varepsilon) =\left\{ y\in \mathbb{R}^s\left\vert \displaystyle\sum\limits_{k\in I} \right. 
\fr{w_k(y)^2}{w_k(x)^2}\leq (1+\sigma\varepsilon)^2\right\},\\[4pt]
\!\!\!\varepsilon=\varepsilon_t(x)\in (0,1),\\[3pt]
\!\!\! \!\!\!\sigma \!=\!\left[ 2\max\limits_{z\in X} 
\sum\limits_{k\in I} w_k(z)\right]^{-1}  \!\!\!\!\min\limits_{z\in X} \min\limits_{k\in I} 
w_k(z)\in \left(\! 0, \fr{1}{2}\right].
\end{array}\!
\right\}\!\!
\label{e8-r}
\end{equation}
  
  Последовательность множеств~(\ref{e6-r}) в~каждой опорной точке 
$x\hm\in X_t$ ветвится, причем степень ее ветвления $\vert 
\mathbf{M}_t(x)\vert$ определяет множество не вложенных друг в~друга 
подмножеств $\mathbf{M}_t(x)\hm\subset 2^I$, заданное соотношениями:
\begin{equation}
\left.
\begin{array}{rl}
\!\!\!\!\!\mathbf{M}_t(x) &=\begin{cases}
\mathbf{N}_t(x), &\mbox{если } \mathbf{N}_t(x)\not=\emptyset\,;\\[3pt]
\{\emptyset\}, &\mbox{если } \mathbf{N}_t(x)=\emptyset\,;
\end{cases}\\[12pt]
\!\!\!\!\!\mathbf{N}_t(x) &= {}\\
&\hspace*{-10mm}{}=\left\{
J\subset I\left\vert\begin{array}{l}
X\cap Y_J(x,\varepsilon) \cap Y(x,\varepsilon) \not= \emptyset\,;\\[3pt]
 X\cap Y_M(x,\varepsilon) \cap Y(x,\varepsilon) 
\not= \emptyset,\\[3pt]
 \hspace*{15mm}M\not= J\subset M\subset I.
 \end{array}
 \right.
 \right\}
\end{array}
\right\}
\label{e9-r}
\end{equation}
  
  Наибольшая степень ветвления последовательности~(\ref{e6-r}) совпадает 
с~наибольшим числом не вложенных друг в~друга подмножеств множества 
\mbox{номеров} частных критериев эффективности $I\hm= \{ k\vert 1\hm\leq k\hm\leq 
m\}$, так что
$$
\vert \mathbf{M}_t(x)\vert \leq \fr{m!}{\lfloor m/2\rfloor! \left(m-\lfloor m/2\rfloor\right)!} \,,
$$
где $\vert A\vert$~--- число элементов в~конечном множестве~$A$; $\lfloor 
z\rfloor$~--- целая часть чис\-ла~$z$.
  Соотношения~(\ref{e8-r}) и~(\ref{e9-r}) включают величину $\varepsilon\hm= 
\varepsilon_t(x)\hm\in (0,1)$~--- параметр возмущения, который в~начальной 
точке~$x^1$ и~в~любых последующих точках~$x^t\hm\in X_t$, 
$x^{t+1}\hm\in X_{t+1}(x^t)$ последовательности~(\ref{e6-r}) таких, что
  \begin{equation}
  x^{t+1}=x^t+h(x^t, J_t),\enskip J_t\in M_t(x)\,,
  \label{e10-r}
  \end{equation}
  
  \vspace*{-4pt}
  
  \noindent
удовлетворяет условиям:

\vspace*{-4pt}

\noindent
\begin{multline}
\varepsilon_1\left(x^1\right) =\kappa\in (0,1),\\
\varepsilon_{t+1}\left(x^{t+1}\right) =\begin{cases}
\kappa \varepsilon_t(x^t), &\mbox{если } Q_t=\emptyset\,;\\
\varepsilon_t(x^t), &\mbox{если } Q_t\not= \emptyset\,,
\end{cases}\\
Q_t=\bigcap\limits_{q\leq t},\ \varepsilon_q(x^q)=\varepsilon_t(x^t)J_q\,,
\label{e11-r}
\end{multline}

\vspace*{-4pt}

\noindent
где величина~$\kappa$ определяет степень дробления параметра 
возмущения. Эта величина может быть выбрана любой на интервале $(0,1)$, 
но остается фиксированной на протяжении всей вычислительной процедуры.

\smallskip

\noindent
\textbf{Замечание~1.} Комментируя процедуру~(\ref{e6-r})--(\ref{e11-r}), 
следует подчеркнуть:
\begin{itemize}
\item включение $y\hm\in Y_J(x,\varepsilon)$ доставляет относительное 
(порядка~$\varepsilon^2$) \textit{увеличение} значений критериев из 
множества $\{ w_k\}_{k\in J}$ за счет возможного относительного 
\textit{уменьшения} (на величину порядка~$\varepsilon$) остальных 
критериев;
\item включение $y\hm\in Y(x,\varepsilon)$ препятствует слишком резкому 
росту значений частных критериев, что на поздних этапах вычислительной 
процедуры не позволяет переходить от одной эффективной векторной оценки 
к~другой и~возвращаться обратно через все множество достижимых 
векторных оценок;
\item пересечение $X\cap Y_J (x,\varepsilon)$~--- выпуклый компакт, 
поскольку~$X$~--- выпуклый компакт, $ Y_J(x,\varepsilon)$~--- замкнуто 
и~выпукло в~согласии с~(\ref{e8-r}) и~следствием~1;
\item пересечение множеств $X\cap Y(x,\varepsilon)$ компактно, но не 
выпукло, поскольку согласно~(\ref{e8-r}) множество $ Y(x,\varepsilon)$ 
выпукло лишь в~некоторых частных случаях (например, если псевдовогнуты 
компоненты век\-тор-функ\-ций $\pm w(x)$).
\end{itemize}
  
  В согласии с~доказательством, предложенным в~работе~[1], заданная 
соотношениями~(\ref{e6-r})--(\ref{e11-r}) последовательность множеств $\{ 
w(X_t)\}^\infty_{t=1}$ аппроксимирует множество Парето в~следующем 
смысле.
  
  \smallskip
  
  \noindent
  \textbf{Теорема~1.} \textit{Пусть в}~(\ref{e1-r}) \textit{компоненты  
век\-тор-функ\-ции $w\hm\in \mathbb{R}^m$ положительно определены 
и~псевдовогнуты в~открытой окрестности непустого выпуклого 
компакта~$X$. Тогда отклонение множества Парето $w(X_{e})$ 
от аппроксимирующего множества~$w(X_t)$ и~отклонение 
аппроксимирующего множества~$w(X_t)$ от множества 
Слейтера~$w(X_0)$ стремятся к~нулю с~рос\-том номера 
аппроксимации}~$t$:
\begin{multline*}
  \lim\limits_{t\to\infty} D\left( w(X_{\mathrm{е}}), 
w(X_t)\right)={}\\
{}=\lim\limits_{t\to\infty} D\left( w(X_t), w(X_0)\right)=0\,.
\end{multline*}

\section{Численные методы на~основе~процедуры}

  Пусть условия теоремы~1 выполняются. Универсальная  
процедура~(\ref{e6-r})--(\ref{e11-r}) не содержит однозначного рецепта 
построения последовательности $\{ w(X_t)^\infty_{t=1}$, 
аппроксимирующей множество Парето, поскольку не дает прямого указания, 
в~какую именно новую опорную точку $y\hm= x\hm+ h(x,J)\hm\in X\cap 
Y_J(x,\varepsilon)\cap Y(x,\varepsilon)$ следует переходить из исходной 
опорной точки $x\hm\in X_t$.
  
  Началу работы вычислительной процедуры предшествует процедура 
выбора начального приближения. В~согласии с~утверж\-де\-ни\-ем  
тео\-ре\-мы~1, сходимость процедуры~(\ref{e6-r})--(\ref{e11-r}) обеспечена 
при старте из любых точек~$X$. Вмес\-те с~тем эффективность процедуры 
существенным образом зависит от выбора начальной точки (либо состава 
начального множества $X_1\hm\subset X$).
  
  Процедура~(\ref{e6-r})--(\ref{e11-r}) становится реальной вы\-чис\-ли\-тель\-ной 
процедурой (численным методом), если она дополнена \textit{правилом 
выбора}:
  \begin{itemize}
  \item[(а)] начальной точки $\{x^1\}\hm\subset X$ (множества начальных 
точек $X_1\hm\subset X$, $\vert X_1\vert \hm>1$, если в~работе численного 
метода предусмотрен мультистарт);
  \item[(б)] каждой последующей опорной точки $y\hm \in X\hm\cap 
Y_J(x,\varepsilon) \hm\cap Y(x,\varepsilon)$.
  \end{itemize}
  
\section{Выбор начального приближения}

  Согласно замечанию~1, для формирования начального 
приближения~$X_1$ можно использовать упрощенный вариант 
процедуры~(\ref{e6-r})--(\ref{e11-r}), где параметр возмущения 
$\varepsilon\hm\in (0,1)$ фиксирован, ограничения сверху на приращения 
частных критериев $\{w_k\}_{k\in I}$ отменяются~--- условие~(\ref{e8-r}) 
заменятся требованием
  \begin{equation}
  \left.
  \begin{array}{c}
 \!\! \!\!\!\! y=x+h(x,J) \in X \cap Y_J(x,\varepsilon),\\[3pt]
  \!\! \!\!\!\! \emptyset \not= J\subset I= \{k\vert 1\leq k\leq m\}\,;\\[3pt]
  \!\! \!\!\!\!Y_J(x,\varepsilon) =\left\{ y\in \mathbb{R}^s\left\vert 
\fr{w_k(y)}{w_k(x)}\right. \geq 1+\sigma\varepsilon^2\,,\right.\\[3pt]
 \!\!  \!\!\!\!\left. k\in J\,,\ \fr{w_k(y)}{w_k(x)}\geq 1-\sigma\varepsilon,\ k\in I\backslash 
J\right\}\!,\\[3pt]
 \!\!\!\!\!\! \sigma\!=\! \left[ 2\max\limits_{z\in X}\displaystyle \sum\limits_{k\in I}\! w_k(z)\right]^{-1}
  \!\!\!\!\!\min\limits_{z\in X} \min\limits_{k\in I} w_k(z)\in \left(0, \fr{1}{2}\right]\!,
  \end{array}\!\!
  \right\}\!\!\!
  \label{e12-r}
  \end{equation}
тогда как множество подмножеств в~(\ref{e9-r}) приобретает вид:
\begin{equation}
\mathbf{M}_t(x)=\left\{ J\subset I\vert \emptyset \not= J,\ X\cap 
Y_J(x,\varepsilon)\not= \emptyset\right\}
\label{e13-r}
\end{equation}
и содержит все такие подмножества $J\hm\subset I\hm= \{ k\vert 1\hm\leq 
k\hm\leq m\}$, что каждый из критериев $\{ w_k\}_{k\in J}$ допускает 
относительное (порядка~$\varepsilon^2$) увеличение.
  
  Сформулируем на основе упрощенной процедуры~(\ref{e6-r}), (\ref{e7-r}), 
(\ref{e12-r}) и~(\ref{e13-r}) правило выбора начального приближения.
  
  \smallskip
  
  \noindent
  \textbf{Правило~А.} В~качестве решения $y^J\hm\in X\cap 
Y_J(x,\varepsilon)$ принимается проекция внешней точки $x\hm\in X$, 
$x\not\in Y_J(x,\varepsilon)$, на выпуклый компакт $X\cap Y_J(x,\varepsilon)$:
  \begin{equation}
  y^J=\argmin\limits_{y\in X\cap Y_J(x,\varepsilon)} \| x-y\|,\enskip J\in M_t(x),
  \label{e14-r}
  \end{equation}
для чего следует решить задачу выпуклого программирования 
(минимизировать квадратичную функцию $\| x\hm- y\|^2$ на выпуклом 
компакте).

  Если компактное выпуклое допустимое множество~$X$ задано 
дифференцируемыми функциональными ограничениями, приближенное 
решение задачи~(\ref{e14-r}) может быть получено следующим образом.
  
  Согласно определению~4 и~следствию~2, в~каж\-дой опорной точкой 
$x\hm\in X$ задана прямоугольная мат\-ри\-ца~$V(x)$ такая, что замыкание 
со\-от\-вет\-ст\-ву\-юще\-го конуса внут\-рен\-них на\-прав\-ле\-ний~$H(x)$ удовле\-тво\-ря\-ет 
соотношениям:
  \begin{multline}
  \overline{H}(x)=\left\{ h\in \mathbb{R}^s\vert V(x)h\geq 0\right\}\,,\
  X\subset \{x\} +\overline{H}(x),\\
  X\backslash \left\{ \{x\} +\overline{H}(x)\right\}=\emptyset\,.
  \label{e15-r}
  \end{multline}
  
  Определим многогранный конус
  \begin{equation}
  H_J(x)=\left\{ h\in \mathbb{R}^s\vert W_J(x) h\geq 0\right\},
  \label{e16-r}
  \end{equation}
где строками матрицы $W_J(x)$ служат векторы $\nabla w_k(x)$, $k\hm\in J$. 
Согласно~(\ref{e15-r}), (\ref{e16-r}) мож\-но утверж\-дать, что
\begin{equation}
X\cap Y_J(x,\varepsilon) \subset \{x\} +\overline{H}(x)\cap H_J(x),
\label{e17-r}
\end{equation}
поскольку для всякого вектора $y\hm= x\hm+ h \hm\in X$, удовле\-тво\-ря\-юще\-го 
при некотором номере $k\hm\in J$ неравенству $\langle \nabla w_k(x),h\rangle 
\hm<0$, выполняется, согласно определению~3, условие $w_k(x)\hm\geq 
w_k(y)$, что ввиду~(\ref{e12-r}) влечет $y\not\in Y_J(x,\varepsilon)$.

  Многогранный конус $\overline{H}(x)\cap H_J(x)$ можно представить 
в~виде конической оболочки некоторого \textit{остова}~$B(x)$: 

\vspace*{-6pt}

\noindent
  \begin{multline}
 \! \!\!\overline{H}(x)\cap H_J(x)\!=\!\left\{ h\in \mathbb{R}^s\vert V(x)h\geq 0,\ 
W_J(x)h\geq 0\right\}\!={}\\
  {}= \left\{ h\in \mathbb{R}^s\vert h=B(x)\lambda,\ \lambda\geq0\right\},
  \label{e18-r}
  \end{multline}
  
  \vspace*{-4pt}

\noindent
где для составленной из единичных век\-то\-ров-столб\-цов матрицы 

\vspace*{-6pt}

\noindent
\begin{multline}
B(x)=\left[ b^1(x),b^2(x),\ldots , b^r(x)\right],\\
\| b^j(x)\|=1\,,\enskip 1\leq j\leq r\,,
\label{e19-r}
\end{multline}

\vspace*{-4pt}

\noindent
существуют конструктивные методы по\-стро\-ения~\cite{3-r}.

  Согласно включениям~(\ref{e15-r}), (\ref{e17-r}), пересечение 
  $\{X \hm- \{x\}\}\cap H_J(x)$ можно аппроксимировать, выбрав из конуса 
$\overline{H}(x)\cap H_J(x)$ конечный <<пучок>> лучей.
  
  \smallskip
  
  \noindent
  \textbf{Лемма~2.} \textit{Если для малого $\rho\hm>0$ выполняются 
соотношения}

\vspace*{-6pt}

\noindent
  \begin{multline}
%  \left.
%  \begin{array}{rl}
  \Delta(\Lambda_r,\Lambda) =D(\Lambda_r,\Lambda)=\sup\limits_{\alpha\in 
\Lambda_r} \mathop{\mathrm{inf}}\limits_{\beta\in \Lambda} \| \alpha-\beta\|\leq 
\rho\,,\\
  \Lambda_r=\left\{ \lambda\in \mathbb{R}^s \left\vert \sum\limits^r_{j=1} \right.
\lambda_j=1\,,\ \lambda\geq 0\right\},\\
 \Lambda =\{ \lambda^q\}^Q_{q=1} \subset \Lambda_r\,,
    \label{e20-r}
  \end{multline}
  
  \vspace*{-4pt}

\noindent
\textit{где конечное множество~$\Lambda$ является $\rho$-ап\-прок\-си\-ма\-ци\-ей $(r\hm-1)$-мер\-но\-го стандартного симплекса~$\Lambda_r$, то 
ввиду}~(\ref{e18-r})--(\ref{e20-r}) \textit{вложенный <<пучок>> лучей}

\vspace*{-6pt}

\noindent
\begin{multline}
L_q(x)=\left\{ q\in \mathbb{R}^s\left\vert  g=\gamma\sum\limits^r_{j=1} \lambda_j^q 
b^j(x), \gamma\geq 0\right.\right\}\subset{}\\
{}\subset \overline{H}(x)\cap H_J(x),\enskip 1\leq q\leq Q\,,
\label{e21-r}
\end{multline}

\vspace*{-4pt}

\noindent
\textit{аппроксимирует выпуклый компакт $\{X\hm -\{x\}\} \hm\cap H_J(x)$ 
в~следующем смысле}.

\textit{Если можно указать величину $\delta\hm>0$ такую, что ненулевой 
вектор $h\hm\in \mathbb{R}^s$ содержится в~множестве $\{ X\hm- \{x\} \} 
\cap H_J(x)$ вместе с~шаром}
$$
U_\delta(h) =\left\{ q\in \mathbb{R}^s\vert \| h-g\| \leq\delta\right\}\subset \left\{ 
X-\{x\}\right\} \cap H_J(x),
$$
\textit{то в}~(\ref{e21-r}) \textit{найдется луч, удовлетворяющий 
соотношению $L_q(x)\cap U_\delta(h)\bot=\emptyset$ при условии $\rho r 
d_X\hm\leq \delta$, где $d_X\hm= \max\nolimits_{x^1, x^2\in X} \| x^1\hm- 
x^2\|$~--- диаметр компакта~$X$}.


  Следовательно, если <<мишень>> $X\cap Y_J(x,\varepsilon)$ конечного 
размера, то один из заданных соотношением~(\ref{e21-r}) лучей $\{x\} \hm+ 
L_1(x)$ в~нее попадет. Строгое утверждение состоит в~следующем: при 
условии $int \{ X\cap Y_J(x,\varepsilon)\}\not= \emptyset$ из 
включения~(\ref{e17-r}) по лемме~2 следует, что в~конечном множестве 
лучей~(\ref{e21-r}) всегда можно указать луч~$L_q(x)$ такой, что
  \begin{equation}
  \left\{ \{x\} +L_q(x)\right\} \cap X\cap Y_J(x,\varepsilon)\not= \emptyset\,,
  \label{e22-r}
  \end{equation}
если конечное множество $\{\lambda^q\}^Q_{q=1}$ аппроксимирует 
стандартный симплекс~$\Lambda_r$ с~тре\-бу\-емой точ\-ностью. Метод 
равномерной аппроксимации стандартного симплекса конечным множеством 
точек предложен в~работе~\cite{4-r}.

  Но тогда хотя бы одна из проекций
  \begin{equation}
  p^q=\argmin\limits_{y\in \{ \{x\} +L_q(x)\} \cap Y_J(x,\varepsilon)} \| x-
y\|\,,\enskip 1\leq q\leq Q\,,
 \label{e23-r}
  \end{equation}
внешней точки $x\hm\in X\backslash Y_J(x,\varepsilon)$ на пересечение $\{ \{ 
x\} \hm+ L_q(x)\} \cap Y_J(x,\varepsilon)$ удовлетворяет включению 
$p^q\hm\in X\cap Y_J(x,\varepsilon)$, поскольку включения $x\hm\in X$ 
и~$p^q\hm\in Y_J(x,\varepsilon)$ выполняются, так что отрицание $p^q\not\in 
X$ противоречит утверждению~(\ref{e22-r}) ввиду выпуклости 
множества~$X$. Следовательно, по правилу~А выбор опорной точки 
$y^J\hm= p^q\hm\in X\cap Y_J(x,\varepsilon)$ требует решения не более 
чем~$Q$~задач одномерного квадратичного программирования~(\ref{e23-r}).

  Если применение упрощенной процедуры на этом заканчивается, 
множество $\{ y^J\}_{J\in \mathbf{M}_t(x)}$, составленное из векторов, 
удовлетворяющих условиям~(\ref{e12-r}), (\ref{e13-r}), может быть принято в~качестве начального множества~$X_1$. Тем самым согласно правилу~А по 
упрощенной процедуре через несколько шагов\linebreak (число которых определяется 
спецификой ре\-ша\-емой задачи) будет построено начальное приближение 
$X_1\hm\subset \mathbb{R}^s$, образ которого~--- множество 
$w(X_1)\hm\subset \mathbb{R}^m$~--- может представлять собой грубую 
\mbox{аппроксимацию} множества Парето по всему <<фронту>>.

\vspace*{-6pt}

  
\section{Правило выбора шага универсальной процедуры}

\vspace*{-3pt}


  Если задано начальное множество~$X_1$, в~согласии с~результатом 
теоремы~1 универсальная процедура~(\ref{e6-r})--(\ref{e11-r}) обеспечивает 
построение последовательности множеств $\{ w(X_t)\}^\infty_{t=1}$, 
аппроксимирующих множество Парето. Следует лишь принять правило 
выбора шага процедуры, для чего в~соотношениях~(\ref{e8-r}), (\ref{e9-r}) 
требуется установить конкретное правило перехода от опорного решения 
$x\hm\in X_t$ к~новым опорным решениям
  \begin{equation}
  y^J\in X\cap Y_J(x,\varepsilon) \cap Y(x,\varepsilon),\ J\in M_t(x).
  \label{e24-r}
  \end{equation}
  
  Сформулируем конструктивный способ выбора опорного 
решения~(\ref{e24-r}).
  
  \smallskip
  
  \noindent
  \textbf{Правило~Б.} Пусть множество допустимых решений~--- выпуклый 
компакт~$X$~--- задано дифференцируемыми функциональными 
ограничениями. Согласно определению~4 и~следствию~2, в~каждой опорной 
точке $x\hm\in X$ известна прямоугольная мат\-ри\-ца~$V(x)$ такая, что 
замыкание со\-от\-вет\-ст\-ву\-юще\-го конуса внутренних на\-прав\-ле\-ний~$H(x)$ 
удовлетворяет условиям~(\ref{e15-r}), а~его пересечение с~многогранным 
конусом~(\ref{e16-r}) удовле\-тво\-ря\-ет условиям~(\ref{e17-r}).
  
  Если в~правой части включения~(\ref{e24-r}) пересечение множеств имеет 
внутренность $\mathrm{int}\, \{ X\hm\cap Y_J(x,\varepsilon) \hm\cap Y(x,\varepsilon)\not= 
\emptyset$, из включения~(\ref{e17-r}) по лемме~2 следует, что в~конечном 
множестве лучей~(\ref{e21-r}) можно указать луч~$L_q(x)$ такой, что луч 
$\{x\}\hm+ L_q(x)$ содержит решение~(\ref{e24-r}):
  \begin{equation}
  \!Z_J\!=\! \left\{ \{x\} +L_q(x)\right\}\cap X\cap Y_J(x,\varepsilon) \cap 
Y(x,\varepsilon) \not= \emptyset\,.
  \label{e25-r}
  \end{equation}
  
  Найти опорное решение~(\ref{e24-r}) на множестве~(\ref{e25-r}) помогает
  
  \smallskip
  
  \noindent
  \textbf{Лемма~3.} \textit{Условие}~(\ref{e25-r}) \textit{в соответствии 
с~леммой~$1$ влечет утверждение}

\vspace*{-6pt}

\noindent
  \begin{multline*}
  \left\{ p^q,g^q\right\}\cap X\cap Y_J(x,\varepsilon)\cap 
Y(x,\varepsilon)\not=\emptyset\,,\\
  p^q=\argmin\limits_{y\in \{\{x\}+L_q(x)\} \cap Y_J(x,\varepsilon)} \| x-
y\|,\\
  g^q=\argmax\limits_{y\in\{\{x\}+L_q(x)\}\cap X} \| x-y\|.
%  \end{array}
 % \right\}
%  \label{e26-r}
  \end{multline*}
  
  \vspace*{-4pt}
  
  Тем самым по правилу~Б для определения опорной точки~(\ref{e24-r}) 
достаточно проверить выполнение включения~(\ref{e24-r}) не более чем для 
$2Q$ точек, выбранных на множестве, состоящем из~$Q$~лучей:

\vspace*{-6pt}

\noindent
\begin{multline}
%\left.
%  \begin{array}{c}
   p^q,g^q \in \{x\}+L_q(x),\ 1\leq q\leq Q\,,\\
  p^q=\argmin\limits_{y\in \{\{x\}+L_q(x)\} \cap Y_J(x,\varepsilon)} \| x-
y\|,\\
  g^q=\argmax\limits_{y\in\{\{x\}+L_q(x)\}\cap X} \| x-y\|,
%  \end{array}
%  \right\}
  \label{e27-r}
  \end{multline}
  
  \vspace*{-4pt}
  
  \noindent
где согласно соотношению~(\ref{e8-r}) точка~$p^q$ есть ближайшая 
к~опорной точке~$x$ точка отрезка $\{x\}\hm+ L_q(x)\cap 
Y_J(x,\varepsilon)$, тогда как~$q^q$~--- наиболее удаленная от~$x$ точка 
отрезка $\{x\} \hm+ L_q(x)\cap X$.

\vspace*{-9pt}

\section{Заключение}

\vspace*{-3pt}

  Задача синтеза численных методов аппроксимации множества Парето 
исследовалась на основе предложенной в~работе~[1] универсальной 
вычисли-\linebreak\vspace*{-12pt}

\pagebreak

\noindent
тельной процедуры при следующих предположениях: множество 
допустимых решений~$X$~--- непустой выпуклый компакт (заданный, если 
потребуется, дифференцируемыми функциональными ограничениями), 
компонентами вектора частных критериев эффективности служат функции, 
псевдовогнутые в~некоторой открытой окрестности~$X$.
  
  Разработанный численный метод аппроксимации множества Парето 
определяется: правилом~А выбора начального приближения и~правилом~Б 
выбора каждого последующего опорного решения.
  
  Построение каждой из точек начального приближения сводится по 
правилу~А к~решению одномерной задачи квадратичного программирования 
на каждом луче из конечного <<пучка>> лучей, аппроксимирующих 
многогранный конус, вложенный в~конус возможных направлений. 
  
  По правилу~Б на каждом этапе вычислительной процедуры переход от 
текущего опорного решения к~последующему предполагает аппроксимацию 
(согласно включению~(\ref{e21-r})) многогранного конуса конечным 
множеством лучей $L_q(x)$, $1\hm\leq q\hm\leq Q$, и~выбор опорного 
решения среди конечного чис\-ла пар решений $\{ p^q, g^q\}\hm\subset \{x\} 
\hm+ L_q(x)$ в~согласии с~соотношениями~(\ref{e27-r}).

\vspace*{-6pt}

{\small\frenchspacing
 {\baselineskip=10.6pt
 %\addcontentsline{toc}{section}{References}
 \begin{thebibliography}{9}
\bibitem{1-r}
\Au{Рабинович Я.\,И.} Универсальная процедура построения множества Парето~// 
Ж.~вычисл. матем. матем. физ., 2017. Т.~57. №\,1. С.~28--47.
\bibitem{2-r}
\Au{Базара М., Шетти~К.} Нелинейное программирование. Теория и~алгоритмы~/ Пер. 
с~англ.~--- М.: Мир, 1986. 583~с. (\Au{Bazaraa~M.\,S., Shetty~C.\,M.}  
Nonlinear programming: Theory and algorithms.~--- New York, NY, USA: 
Wiley, 1979. 872~p.)
\bibitem{3-r}
Линейные неравенства и~смежные вопросы~/ Под ред. Г.\,У.~Куна, 
А.\,У.~Таккера; пер. с~англ.~--- Annals of mathematics studies ser.~--- М.: ИЛ, 1959. 470~с.
(Linear inequalities and 
related systems~/ Eds. H.\,W.~Kuhn, A.\,W.~Tucker.~--- Annals of mathematics studies ser.~---  Princeton, NJ, USA: Princeton University Press, 1956. 
322~p.)
\bibitem{4-r}
\Au{Рабинович Я.\,И.} Численные методы оценивания приближенных решений в~задачах 
многокритериальной оптимизации~// Докл.\ Акад.\ наук, 2015. Т.~462. №\,2. С.~151--153.

\end{thebibliography}

 }
 }

\end{multicols}

\vspace*{-10pt}

\hfill{\small\textit{Поступила в~редакцию 21.10.22}}

\vspace*{6pt}

%\pagebreak

%\newpage

%\vspace*{-28pt}

\hrule

\vspace*{2pt}

\hrule



\def\tit{PROCEDURE OF~CONSTRUCTING A~PARETO SET 
FOR~DIFFERENTIABLE CRITERIA FUNCTIONS}


\def\titkol{Procedure of~constructing a~Pareto set 
for~differentiable criteria functions}


\def\aut{Ya.\,I.~Rabinovich}

\def\autkol{Ya.\,I.~Rabinovich}

\titel{\tit}{\aut}{\autkol}{\titkol}

\vspace*{-14pt}


\noindent
Federal Research Center ``Computer Science and Control'' of the Russian Academy 
of Sciences, 44-2~Vavilov Str., Moscow 119333, Russian Federation


\def\leftfootline{\small{\textbf{\thepage}
\hfill INFORMATIKA I EE PRIMENENIYA~--- INFORMATICS AND
APPLICATIONS\ \ \ 2023\ \ \ volume~17\ \ \ issue\ 4}
}%
 \def\rightfootline{\small{INFORMATIKA I EE PRIMENENIYA~---
INFORMATICS AND APPLICATIONS\ \ \ 2023\ \ \ volume~17\ \ \ issue\ 4
\hfill \textbf{\thepage}}}

\vspace*{2pt}
      
 


\Abste{A ubiquitous computational procedure for the multicriteria optimization 
allows one to approximate the Pareto set under different requirements to the vector 
of particular efficiency criteria and the set of feasible solutions. In the paper, it is 
assumed that particular efficiency criteria are pseudoconcave in an open 
neighborhood of a compact convex set of feasible solutions which can be given by 
differentiable functional constraints. To build specific numerical methods for 
approximating the Pareto set, a rule for choosing the initial approximation and 
a~rule for moving from the current reference solution to the next one are 
proposed.}


\KWE{multicriteria optimization; Pareto set; numerical methods of 
approximation; universal procedure}

\DOI{10.14357/19922264230403}{NEZRGD}

%\vspace*{-20pt}

%\Ack
%\vspace*{-3pt}
%\noindent
  

\vspace*{-5pt}

  \begin{multicols}{2}

\renewcommand{\bibname}{\protect\rmfamily References}
%\renewcommand{\bibname}{\large\protect\rm References}

{\small\frenchspacing
 {%\baselineskip=10.8pt
 \addcontentsline{toc}{section}{References}
 \begin{thebibliography}{9} 
 
 \vspace*{-1pt}
 
\bibitem{1-r-1}
\Aue{Rabinovich, Ya.\,I.} 2017. Universal procedure for constructing a~Pareto set. 
\textit{Comp. Math. Math. Phys.} 57(1):45--63. doi: 
10.1134/S0965542517010122.
\bibitem{2-r-1}
\Aue{Bazaraa, M.\,S., and C.\,M.~Shetty.} 1979. \textit{Nonlinear programming: 
Theory and algorithms}. New York, NY: Wiley. 872~p.
\bibitem{3-r-1}
Kuhn, H.\,W., and A.\,W.~Tucker, eds. 1956. \textit{Linear inequalities and 
related systems}. Annals of mathematics studies ser.  Princeton, NJ: Princeton University Press. 
322~p.
\bibitem{4-r-1}
\Aue{Rabinovich, Ya.\,I.} 2015. Numerical methods for estimating approximate 
solutions of multicriteria optimization problems. \textit{Dokl. Math.} 
91(3):384--386. doi: 10.1134/ S1064562415030114. EDN: UFAQXF.

\end{thebibliography}

 }
 }

\end{multicols}

\vspace*{-8pt}

\hfill{\small\textit{Received October 21, 2022}} 

\vspace*{-18pt}

\Contrl

\vspace*{-4pt}

\noindent
\textbf{Rabinovich Yaacov I.} (b.\ 1947)~--- Candidate of Science (PhD) in 
physics and mathematics, senior scientist, Federal Research Center ``Computer 
Science and Control'' of the Russian Academy of Sciences, 44-2~Vavilov Str., 
Moscow 119333, Russian Federation; \mbox{jacrabin@rambler.ru}

      





\label{end\stat}

\renewcommand{\bibname}{\protect\rm Литература} 