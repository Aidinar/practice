\def\stat{shnurkov}

\def\tit{РЕШЕНИЕ ЗАДАЧИ ОПТИМАЛЬНОГО УПРАВЛЕНИЯ ЗАПАСОМ НЕПРЕРЫВНОГО ПРОДУКТА 
В~СТОХАСТИЧЕСКОЙ МОДЕЛИ РЕГЕНЕРАЦИИ СО СЛУЧАЙНЫМИ СТОИМОСТНЫМИ ХАРАКТЕРИСТИКАМИ}

\def\titkol{Решение задачи оптимального управления запасом непрерывного продукта 
в~стохастической модели регенерации} % со случайными стоимостными характеристиками}

\def\aut{П.\,В.~Шнурков$^1$}

\def\autkol{П.\,В.~Шнурков}

\titel{\tit}{\aut}{\autkol}{\titkol}

\index{Шнурков П.\,В.}
\index{Shnurkov P.\,V.}


%{\renewcommand{\thefootnote}{\fnsymbol{footnote}} \footnotetext[1]
%{Работа выполнено с~использованием инфраструктуры Центра коллективного пользования <<Высокопроизводительные вы\-чис\-ле\-ния и~большие данные>> 
%(ЦКП <<Информатика>>) ФИЦ ИУ РАН (г.~Москва).}}


\renewcommand{\thefootnote}{\arabic{footnote}}
\footnotetext[1]{Национальный исследовательский университет 
<<Высшая школа экономики>>, \mbox{pshnurkov@hse.ru}}

\vspace*{-14pt}



\Abst{Работа посвящена исследованию проблемы управления запасом определенного 
непрерывного продукта, эволюция объема которого описывается регенерирующим 
случайным процессом. Основная особенность рассматриваемой математической модели 
заключается в~том, что стоимостные характеристики, определяющие цену поставки 
продукта потребителю и~затраты, связанные с обеспечением функционирования 
системы, зависят от случайных внешних факторов. Случайный параметр управления 
представляет собой время от момента очередного пополнения запаса до момента 
последующего заказа на пополнение. Доказано, что стационарный стоимостный 
показатель эффективности управ\-ле\-ния в~рас\-смат\-ри\-ва\-емой задаче оптимизации по 
своей аналитической структуре пред\-став\-ля\-ет собой дроб\-но-ли\-ней\-ный интегральный 
функционал, зависящий от функции распределения параметра управления. 
Теоретическое решение задачи оптимизации основано на использовании тео\-ре\-мы об 
экстремуме дроб\-но-ли\-ней\-ных интегральных функционалов.}

\KW{проблема управления запасом непрерывного продукта; 
случайные стоимостные характеристики сис\-те\-мы; управ\-ля\-емые ре\-ге\-не\-ри\-ру\-ющие 
случайные процессы; дроб\-но-ли\-ней\-ные интегральные функционалы в~задачах 
стохастического оптимального управ\-ления}

\DOI{10.14357/19922264230407}{WHZRNY}
  
\vspace*{-4pt}


\vskip 10pt plus 9pt minus 6pt

\thispagestyle{headings}

\begin{multicols}{2}

\label{st\stat}

\section{Введение}

\vspace*{-2pt}

Применение теории регенерирующих случайных процессов и~процессов восстановления 
для создания и~исследования моделей управления запасами известно достаточно 
давно. Соответствующие динамические стохастические модели в~той или иной форме 
рассматриваются не только в~классических изданиях прошлых лет, но 
и~в~современных научных изданиях, посвященных теории логистики и~управления 
запасами. Упомянем в~этой связи работы~[1, 2].  Ряд стохастических моделей 
управ\-ле\-ния запасами, рас\-смот\-рен\-ных в~этих изданиях, имеет характерные признаки 
ре\-ге\-не\-ри\-ру\-ющих и~полумарковских процессов. К~сожалению, даже в~упомянутых 
фундаментальных и~весьма содержательных изданиях отсутствуют специальные 
разделы, посвященные применению ре\-ге\-не\-ри\-ру\-ющих и~полумарковских случайных 
процессов в~тео\-рии управ\-ле\-ния запасами.

Среди специальных научных исследований, посвященных применению ре\-ге\-не\-ри\-ру\-ющих 
процессов в~тео\-рии управ\-ле\-ния запасами, выделим работы П. Айзигера и~его 
соавторов~[3, 4]. В~част\-ности, в~работе~[3] построена общая теоретическая 
конструкция использования ре\-ге\-не\-ри\-ру\-юще\-го процесса для создания моделей 
управления запасами и~определения их вероятностных характеристик. В~работе~[4] 
рассмотрены примеры и~численные иллюстрации применения полученных теоретических 
результатов.

В своих предшествующих исследованиях автор настоящей работы также получил ряд 
результатов, связанных с применением ре\-ге\-не\-ри\-ру\-ющих процессов к~задачам 
управления запасами в~различных конкретных стохастических моделях~[5--8].

Основная отличительная особенность данного исследования заключается 
в~стохастическом характере стоимостных характеристик модели, определяющих затраты и~доходы, возникающие в~ходе функционирования системы. По своей аналитической 
структуре все эти характеристики пред\-став\-ля\-ют собой суммы заданных 
детерминированных функций, которые можно назвать трендами, или направлениями 
эволюции, и~случайных уклонений от трендов, которые в~каждый момент времени 
имеют усеченные нормальные распределения с~заданными па\-ра\-мет\-ра\-ми. Проводя 
аналогию с~известными физическими явлениями, можно сказать, что в~данной модели 
стоимостные характеристики представляют собой случайные флуктуации, 
эволюционирующие в~заданных границах, которые зависят от времени, прошедшего 
с~момента начала очередного периода регенерации и~объема\linebreak запаса в~эти моменты 
времени. В~работе получены аналитические пред\-став\-ле\-ния для классического 
стационарного стоимостного показателя эф\-фек\-тив\-ности, который по своему 
экономическому \mbox{содержанию} пред\-став\-ля\-ет собой сред\-нюю удель\-ную прибыль. 
Установлено, что этот показатель имеет структуру дроб\-но-ли\-ней\-но\-го интегрального 
функционала, зависящего от распределения случайного па\-ра\-мет\-ра управ\-ле\-ния. Как 
и~в~пред\-шест\-ву\-ющих работах~[5--8], этот параметр управления совпадает 
с~дли\-тель\-ностью периода времени от момента очередного пополнения запаса (момента 
регенерации основного процесса) до момента сле\-ду\-юще\-го заказа на пополнение 
запаса. Тео\-ре\-ти\-че\-ское решение по\-став\-лен\-ной задачи оптимизации основывается на 
утверждении об экстремуме дроб\-но-ли\-ней\-но\-го интегрального функционала.

\vspace*{-6pt}

\section{Описание математической модели исследуемой системы}

\vspace*{-3pt}

В настоящем исследовании рассматривается стохастическая модель некоторой 
системы, предназначенной для хранения и~поставки потребителю определенного 
продукта. Объем продукта измеряется вещественными числами и~может принимать 
значения из множества $(-\infty;\tau]$, где $\tau$~--- максимально допустимый 
объем продукта, который может находиться в~сис\-те\-ме. Следуя традиции, сложившейся в~математической теории управления запасами, будем считать, что положительные 
значения объема продукта соответствуют реальному запасу, а~отрицательные 
значения означают наличие неудовлетворенного спроса, или дефицита продукта.

Введем основное вероятностное пространство $(\Omega,\mathcal{A},\mathsf{P})$, на 
котором будут определены все рас\-смат\-ри\-ва\-емые в~дальнейшем стохастические 
объекты. Это пространство будет формально описывать реальный случайный 
эксперимент, в~ходе которого реализуется исследуемая вероятностная модель 
управления запасом.

Обозначим через $x(t)$ объем запаса продукта в~системе в~момент времени~$t$. 
Переменная величина $x(t)\hm=x(\omega,t)$, $\omega\hm\in\Omega$, $t\hm\in[0,\infty)$, 
представляет собой случайный процесс с непрерывным временем и~множеством 
состояний $(-\infty;\tau]$, свойства которого будут установлены в~дальнейшем.

Спрос на продукт детерминирован. Товар реализуется со склада с постоянной 
ско\-ростью $\alpha\hm>0$. Через время ${\tau}/{\alpha}$ после начала 
потребления, когда уровень запаса продукта на складе достигает нуля, реализация 
товара продолжается с запасного хранилища с той же постоянной ско\-ростью~$\alpha$.

Система начинает свое функционирование при максимальном объеме продукта: 
$x(0)\hm=\tau$. В~начальный момент работы системы принимается решение, когда 
провести заказ на следующую поставку продукта. Промежуток времени с момента 
начала работы до заказа на поставку описывается случайной величиной~$\eta$ с 
функцией распределения $G(u)\hm=\mathsf{P}\left(\eta\hm<u\right)$. После проведенного 
заказа продукт проступает в~систему через случайное время~$\delta$, имеющее 
распределение $H(z)\hm=\mathsf{P}(\delta\hm<z)$.

Пополнение запаса осуществляется мгновенно. После пополнения запаса до 
максимального уровня~$\tau$ эволюция системы происходит по закономерностям, 
описанным выше, независимо от прош\-лого.

Рассмотрим случайные моменты времени 
$t_n\hm=t_n(\omega)$, $\omega\hm\in\Omega$, $n\hm=0,1,2\ldots,t_0\hm=0$, в~которые происходит 
пополнение запаса до максимального уровня~$\tau$. Зафиксируем некоторый 
момент~$t_n$. Период времени от момента пополнения запаса~$t_n$ до следующего заказа на 
пополнение является случайной величиной~$\eta_n$ с функцией распределения 
$G(u)\hm=\mathsf{P}\left(\eta_n\hm<u\right)$.

Рассмотрим далее случайную величину $t_n^\prime\hm=t_n\hm+\eta_n$, которая 
представляет собой момент заказа на поставку продукции. В~момент~$t_n^\prime$ 
начинается\linebreak
 так называемый период задержки поставки. Обозначим случайную 
длительность этого периода через~$\delta_n$. Следующее пополнение запаса до 
максимального уровня~$\tau$ осуществляется в~момент \mbox{времени} 
$t_{n+1}\hm=t_n^\prime\hm+\delta_n\hm=t_n\hm+\eta_n\hm+\delta_n$. Далее эволюция процесса~$x(t)$ 
происходит по тем же закономерностям, независимо от прошлого. Отсюда 
следует, что случайный процесс~$x(t)$ относится к~виду ре\-ге\-не\-ри\-ру\-ющих процессов, 
а моменты~$t_n$, $n\hm=0,1,2,\dots$, представляют собой моменты его регенерации. 
Интервал времени $\left[t_n;t_{n+1}\right)$ называется периодом регенерации 
данного процесса.

Траектория процесса $x(t)$ на интервале времени $\left[t_n;t_{n+1}\right)$ 
определяется соотношением:
$$  
x(t)=x(t_n)-\alpha\left(t-t_n\right),\ t_n\le t<t_{n+1}.
$$


Управление процессом (решение) принимается в~начальный момент каждого периода 
регенерации $[t_n,t_{n+1})$, т.\,е.\ в~момент времени~$t_n$. Управление является 
случайной величиной $u_n\hm=u_n(\omega)$, $\omega\hm\in\Omega$, принимающей значения из 
множества допустимых управлений $U\hm=[0,\infty)$, и~совпадает со случайной 
величиной~$\eta_n$, т.\,е.\ выполняется соотношение $\mathsf{P}(\omega\in\Omega:\ u_n\hm=\eta_n)\hm=1$. Функция распределения $G(u)\hm=\mathsf{P}\left(
\eta_n\hm<u\right)$ называется управляющим распределением.

Задача оптимального управления ре\-ге\-не\-ри\-ру\-ющим процессом~$x(t)$ формально заключается 
в~на\-хож\-де\-нии вероятностного распределения~$G^\ast(u)$, которое бы доставляло 
глобальный экстремум стоимостному показателю средней удельной прибыли 
вида~$I(G)$. Аналитическое представление для него будет получено в~дальнейшем.

\vspace*{-6pt}

\section{Стоимостные характеристики стохастической модели}

\vspace*{-3pt}

Предположим, что с каждым периодом регенерации  $[t_n,t_{n+1})$ связаны 
семейства случайных величин 
$s_1^{(n)}(t),s_2^{(n)}(t),r_1^{(n)}(t),r_2^{(n)}(t),~n=0,1,2,\dots$, заданных 
на исходном вероятностном пространстве $(\Omega,\mathcal{A},\mathsf{P})$, зависящих 
от параметра времени~$t$, которые описывают внешние факторы, влияющие на 
стоимостные характеристики модели. Предположим также, что указанные факторы 
удовлетворяют следующим условиям.
\begin{enumerate}[1.]
\item Время эволюции каждого фактора с номером~$n$ связано с соответствующим 
периодом регенерации $[t_n,t_{n+1})$ и~изменяется от 0 до случайной длительности 
этого периода $\Delta t_n\hm=t_{n+1}\hm-t_n$.

\item Все рассматриваемые стохастические семейства 
$\{s_1^{(n)}(t)$, $n\hm=0,1,2,\dots\},$  $\{s_2^{(n)}(t)$, $n\hm=0,1,2,\dots\}$; 
$\{r_1^{(n)}(t)$, $n\hm=0,1,2,\dots\}$ и~$\{r_2^{(n)}(t)$, $n\hm=0,1,2,\dots\}$ состоят из 
случайных процессов, которые независимы при различных значениях~$n$ и~имеют 
одинаковые конечномерные распределения. Таким образом, указанные случайные 
факторы действуют независимо на различных периодах регенерации.
\end{enumerate}

Заметим, что все введенные факторы могут совпадать. Тогда стоимостные 
характеристики модели будут зависеть от одного случайного фактора, описывающего 
влияние внешней среды.

\begin{figure*}[b] %fig1
  \vspace*{1pt}
\begin{center}
   \mbox{%
\epsfxsize=163mm 
\epsfbox{shn-1.eps}
}
\end{center}
\vspace*{-9pt}
\Caption{Пример~1. Возможные траектории изменения 
цены единицы продукта на периоде регенерации для реализаций~1~(\textit{а}) и~2~(\textit{б}):
\textit{1}~--- минимальные значения; \textit{2}~--- максимальные значения; \textit{3}~--- тренд; \textit{4}~--- истинные значения}
\end{figure*}

Будем предполагать, что стоимостные характеристики модели зависят от указанных 
внешних факторов. Введем следующие обозначения для стоимостных характеристик:
\begin{description}
\item[\,] $c_1(x,s_1)$~--- затраты в~единицу времени, связанные с хранение запаса объема 
$x\geq0$ при условии, что внешний фактор~$s_1(t)$ принимает значение~$s_1$;
\item[\,] 
$c_2(x,s_2)$~--- затраты (штрафы) в~единицу времени, связанные с реализацией 
продукции объема~$|x|$ с запасного склада, $x\hm<0$, при условии, что внешний 
фактор~$s_2(t)$ принимает значение~$s_2$;
\item[\,] 
$d_1(x,r_1)$~--- цена, которую платит потребитель в~единицу времени за единицу 
продукта при условиях, что объем запаса принимает значение $x\hm>0$, а~внешний 
фактор~$r_1(t)$ принял значение~$r_1$;
\item[\,] 
$d_2(x,r_2)$~--- цена, которую платит потребитель в~единицу времени за единицу 
продукта в~период дефицита, т.\,е.\ при условиях, что объем запаса принимает 
отрицательное значение $x\hm<0$, а~внешний фактор~$r_2(t)$ принял значение~$r_2$.
\end{description}

Теперь перейдем к~описанию конкретного влияния случайных факторов на стоимостные 
характеристики. Будем предполагать, что в~рас\-смат\-ри\-ва\-емой сто\-ха\-сти\-че\-ской модели 
все стоимостные\linebreak \mbox{характеристики} пред\-став\-ля\-ют\-ся в~виде суммы детерминированной 
со\-став\-ля\-ющей, или тренда, и~случайной со\-став\-ля\-ющей, или флуктуации, опи\-сы\-ва\-ющей 
отклонения от тренда.

Пусть заданы следующие детерминированные функции: $c_1^{\max}(x(t))$, 
$c_1^{\min}(x(t))$, $c_1^{\mathrm{trend}}(x(t))$, $c_2^{\max}(x(t))$, $c_2^{\min}(x(t))$, 
$c_2^{\mathrm{trend}}(x(t))$, $d_1^{\max}(x(t))$, $d_1^{\min}(x(t))$, $d_1^{\mathrm{trend}}(x(t))$, 
$d_2^{\max}(x(t))$, $d_2^{\min}(x(t))$, $d_2^{\mathrm{trend}}(x(t))$, которые определены 
при всех значениях аргумента $x(t)\hm\in\left(-\infty;\tau\right]$, принимают 
только неотрицательные значения и~для которых справедливы следующие соотношения:

\noindent
\begin{align*}
c_1^{\min}(x(t))&<c_1^{\mathrm{trend}}(x(t))<c_1^{\max}(x(t));\\
c_2^{\min}(x(t))&<c_2^{\mathrm{trend}}(x(t))<c_2^{\max}(x(t));\\
d_1^{\min}(x(t))&<d_1^{\mathrm{trend}}(x(t))<d_1^{\max}(x(t));\\
d_2^{\min}(x(t))&<d_2^{\mathrm{trend}}(x(t))<d_2^{\max}(x(t)),
\end{align*}

\vspace*{-3pt}

\noindent
где $x(t)$~--- объем запаса в~системе через время~$t$ после начала очередного 
периода регенерации: 
$$
x(t)=\tau\hm-\alpha t,\enskip t\hm\geq 0\,.
$$

\vspace*{-6pt}

Указанные функции определяют переменные границы областей, в~которых могут 
принимать значения стоимостные характеристики модели, а также линии трендов этих 
характеристик.

Обозначим:

\noindent
\begin{align*}
\Delta c_1^{(+)}(x(t))&=c_1^{\max}(x(t))-c_1^{\mathrm{trend}}(x(t));\\
\Delta c_1^{(-)}(x(t))&=c_1^{\mathrm{trend}}(x(t))-c_1^{\min}(x(t)).
\end{align*}

\vspace*{-6pt}

Соответствующие функции $\Delta c_2^{(+)}(x(t))$, $\Delta c_2^{(-)}(x(t))$, 
$\Delta d_1^{(+)}(x(t))$, $\Delta d_1^{(-)}(x(t))$, $\Delta d_2^{(+)}(x(t))$ 
и~$\Delta d_2^{(-)}(x(t))$ определяются аналогично.

Будем предполагать, что в~любой момент времени, отстоящий на величину $t\geq0$ 
от начала очередного периода регенерации, случайные стоимостные характеристики 
определяются следующими соотношениями:

\noindent
\begin{align}
c_1(x(t),s_1^{(n)}(t))&=c_1^{\mathrm{trend}}(x(t))+\Delta\beta_1(x(t));  \label{eq1}\\
c_2(x(t),s_2^{(n)}(t))&=c_2^{\mathrm{trend}}(x(t))+\Delta\beta_2(x(t)); \label{eq2}
\\
d_1(x(t),r_1^{(n)}(t))&=d_1^{\mathrm{trend}}(x(t))+\Delta\gamma_1(x(t));\label{eq3}\\
d_2(x(t),r_2^{(n)}(t))&=d_2^{\mathrm{trend}}(x(t))+\Delta\gamma_2(x(t)).
 \label{eq4}
\end{align}

Случайные функции 
$\Delta\beta_1(x(t))$, $\Delta\beta_2(x(t))$, $\Delta\gamma_1(x(t))$ и~$\Delta\gamma_2(x(t))$,
 фигурирующие в~правых частях соотношений~(1)--(4), представляют собой 
уклонения значений стоимостных характеристик от линий трендов. Предполагается,\linebreak 
что при каждом значении~$t$ эти величины имеют усеченные нормальные 
распределения на отрезках $[-\Delta c_k^{(-)}(x(t));\Delta 
c_k^{(+)}(x(t))]$, $k\hm=1,2,$ и~$[-\Delta d_l^{(-)}(x(t));\Delta 
d_l^{(+)}(x(t))]$, $l\hm=1,2$, соответственно. Параметры этих распределений считаются 
заданными, причем 
\begin{gather*}
\mathsf{E}\,\Delta\beta_1(x(t))=a_1^{(1)}(t);\enskip \mathsf{E}\,\Delta\beta_2(x(t))\hm=
a_2^{(1)}(t);\\
 \mathsf{E}\,\Delta\gamma_1(x(t))\hm=a_1^{(2)}(t);\enskip 
\mathsf{E}\,\Delta\gamma_2(x(t))=a_2^{(2)}(t).
\end{gather*}


\smallskip

\noindent
\textbf{Замечание 1.} Случайные отклонения от линий трендов могут иметь 
произвольные усеченные распределения, сосредоточенные на конечных интервалах с 
переменными границами, для которых известны их вероятностные характеристики.

При сделанных предположениях математические ожидания стоимостных характеристик 
задаются равенствами:
\begin{align}
{\overline{c}}_1(x(t))&=\mathsf{E} c_1(x(t),s_1^{(n)}(t))={}\notag \\
&{}= c_1^{\mathrm{trend}}(x(t))+a_1^{(1)}(t),\enskip x(t)\geq 0;
\label{e5-sh}\\
{\overline{c}}_2(x(t))&=\mathsf{E} c_2(x(t),s_2^{(n)}(t))={}\notag\\
&{}=c_2^{\mathrm{trend}}(x(t))+a_2^{(1)}(t),\enskip x(t)< 0; \label{e6-sh}
\\
{\overline{d}}_1(x(t))&=\mathsf{E} 
 d_1(x(t),r_1^{(n)}(t))={}\notag\\
& {}=d_1^{\mathrm{trend}}(x(t))+ a_1^{(2)}(t),\enskip x(t)\geq 0;
\label{e7-sh}\\
 {\overline{d}}_2(x(t))&=\mathsf{E}  d_2(x(t),r_2^{(n)}(t))={}\notag\\
 &{}=d_2^{\mathrm{trend}}(x(t))+a_2^{(2)}(t),~x(t)< 0.
 \label{e8-sh}
\end{align} 

\section{Численное моделирование стоимостных характеристик}

В данном разделе рассмотрены два примера наборов стоимостных характеристик 
построенной модели. В~первом примере математические ожидания случайных 
флуктуаций равны нулю и~математические ожидания стоимостных характеристик 
совпадают с линиями трендов. Во втором примере параметры случайных флуктуаций 
сконструированы таким образом, что математические ожидания стоимостных 
характеристик смещены относительно линий трендов.

\begin{figure*} %fig2
 \vspace*{1pt}
\begin{center}
   \mbox{%
\epsfxsize=161.435mm 
\epsfbox{shn-2.eps}
}
\end{center}
\vspace*{-9pt}
\Caption{Пример~1. Возможные траектории изменения 
затрат на хранение и~штрафов, связанных с дефицитом единицы продукта для реализаций~1~(\textit{а}) и~2~(\textit{б}):
\textit{1}~--- минимальные значения; \textit{2}~--- максимальные значения; \textit{3}~--- тренд; \textit{4}~--- истинные значения}
\end{figure*}

\begin{figure*}[b] %fig3
 \vspace*{1pt}
\begin{center}
   \mbox{%
\epsfxsize=163mm 
\epsfbox{shn-3.eps}
}
\end{center}
\vspace*{-9pt}
\Caption{Пример~2. Возможные траектории цены единицы 
товара на периоде регенерации для реализаций~1~(\textit{а}) и~2~(\textit{б}):
\textit{1}~--- минимальные значения; \textit{2}~--- максимальные значения; \textit{3}~--- тренд; \textit{4}~--- истинные значения}
\end{figure*}

\smallskip

\noindent
\textbf{Пример 1.} 
Зададим функции, определяющие цену продукта:
\begin{align*}
d_1^{\max}(x)&=\begin{cases}
   100 -\fr{x}{100},  &x\geq0;\\
   0,  &x<0;
   \end{cases}
\\
d_1^{\min}(x)&= \begin{cases}
  90 -\fr{x}{100}, & x\geq0;\\
 0, &x<0;
 \end{cases}
\\
d_1^{\mathrm{trend}}(x)&=\begin{cases}
   95 -\fr{x}{100}, & x\geq0;\\
   0,  &x<0;
   \end{cases}
   \\
d_2^{\max}(x)&=    \begin{cases}
0, &x \geq0;\\
100e^{{x}/{100}}, &x<0;
\end{cases}
\\
d_2^{\min}(x)&=   \begin{cases}
0, & x\hm\geq0;\\
 90e^{{x}/{100}}, &x<0;
 \end{cases}
\\
d_2^{\mathrm{trend}}(x)&= 
\begin{cases}
0, &x\hm\geq0;\\
95e^{{x}/{100}},  &x<0.
\end{cases}
\end{align*}

Зададим далее функции, определяющие величины за\-трат, связанных 
с~функционированием сис\-те\-мы, при помощи сле\-ду\-ющих соотношений:
\begin{align*}
c_1^{\max}(x)&=\begin{cases}
   310,  &x\geq0;\\
   0, & x<0;
   \end{cases}
  \\[3pt]
c_1^{\min}(x)&=\begin{cases}
   290, & x\hm\geq0;\\
   0,  & x<0;
   \end{cases}
   \\[3pt]
c_1^{\mathrm{trend}}(x)&=
\begin{cases}
 300, &x\geq0;\\
  0, & x<0;
  \end{cases}
  \\[3pt]
c_2^{\max}(x)&= \begin{cases}
  0, & x\geq0;\\
  310 \hm+ \ln{(1-x)}, & x<0;
  \end{cases}
 \\[3pt]
c_2^{\min}(x)&=\begin{cases}
    0, & x\hm\geq0; \\
    290 + \ln{(1-x)}, & x\hm<0;
    \end{cases}
    \end{align*}
    
  \noindent
    \begin{align*}
c_2^{\mathrm{trend}}(x)&= \begin{cases}
  0, & x\hm\geq0;\\
  300 + \ln{(1-x)}, & x\hm<0.
  \end{cases}
\end{align*}


Предположим также, что в~данном примере математические ожидания уклонений от 
линии трендов равны~0:
$$
a_1^{(1)}(t)=a_2^{(1)}(t)=a_1^{(2)}(t)=a_2^{(2)}(t)=0\,,
$$
а соответствующие дисперсии равны~1.

Траектории (реализации) случайной цены представлены на рис.~1, траектории 
случайных затрат на рис.~2. Заметим, что для каждой из указанных случайных 
функций приведены по две возможные реализации.


\begin{figure*} %fig4
 \vspace*{1pt}
\begin{center}
   \mbox{%
\epsfxsize=161.435mm 
\epsfbox{shn-4.eps}
}
\end{center}
\vspace*{-12pt}
\Caption{Пример~2. Возможные траектории изменения 
затрат на периоде регенерации для реализаций~1~(\textit{а}) и~2~(\textit{б}):
\textit{1}~--- минимальные значения; \textit{2}~--- максимальные значения; \textit{3}~--- тренд; \textit{4}~--- истинные значения}
\vspace*{-6pt}
\end{figure*}
%

%\smallskip


\noindent
\textbf{Пример~2.} Предположим, что функции, определяющие границы областей 
изменения и~тренды для стоимостных характеристик, совпадают с теми, которые были 
заданы в~примере~1. В~то же время математические ожидания случайных отклонений 
от трендов (флуктуаций) задаются сле\-ду\-ющим образом:

\vspace*{-6pt}

\noindent
\begin{align*}
a_1^{(1)}(t)&=\Delta c_1^{(+)}-\fr{\Delta c_1^{(+)}+\Delta c_1^{(-)}}{3}\,;\\[-1pt]
a_2^{(1)}(t)&=\Delta c_2^{(+)}-\fr{\Delta c_2^{(+)}+\Delta c_2^{(-)}}{3}\,;
\\[-1pt]
a_1^{(2)}(t)&=\Delta d_1^{(-)}+\fr{\Delta d_1^{(+)}+\Delta d_1^{(-)}}{4}\,;\\[-1pt]
a_2^{(2)}(t)&=\Delta d_2^{(-)}+\fr{\Delta d_2^{(+)}+\Delta d_2^{(-)}}{4}\,.
\end{align*}

\vspace*{-6pt}

Дисперсии соответствующих отклонений предполагаются равными единице.


В данном варианте траектории случайной цены, образующиеся при заданных 
характеристиках модели, представлены на рис.~3, а~траектории случайных затрат~--- 
на рис.~4. Как и~в~примере~1, для каждой из указанных случайных функций 
приведены по две возможные реализации.

\vspace*{-6pt}


\section{Аналитическое представление стационарного стоимостного показателя 
эффективности}

\vspace*{-3pt}

В данном разделе будет получено явное аналитическое представление для 
стационарного стоимостного показателя эффективности управления~--- средней 
удельной прибыли на периоде \mbox{регенерации}. Данный показатель представим в~виде 
дроб\-но-ли\-ней\-но\-го интегрального функционала от вероятностного распределения 
$G(\cdot)$, характеризующего случайный параметр управления.

Приведем сначала вспомогательный результат, связанный с представлением 
математического ожидания случайной прибыли, образующейся на произвольном 
фиксированном конечном интервале времени, вложенном в~некоторый период 
регенерации.

Как и~ранее, обозначим через~$t_n$, $n\hm=0,1,2,\dots$, $t_0\hm=0,$~--- случайные моменты 
пополнения запаса (моменты регенерации), $\Delta t_n\hm=t_{n+1}\hm-
t_n$, $n\hm=0,1,2,\dots,$~--- случайные длительности периодов регенерации. Зафиксируем 
произвольные чис\-ла $t^{(0)}$ и~$\Delta^{(0)}$, для которых $0\hm \leq t^{(0)} \hm< 
t^{(0)} + \Delta^{(0)}$. Предположим, что для некоторого периода регенерации 
$[t_n, t_{n+1}]$ выполняется условие 
$$
t^{(0)} + \Delta^{(0)} <\Delta t_n.
$$
 
Обозначим через $\tilde{S}(t^{(0)}, t^{(0)} + \Delta^{(0)})$ случайную прибыль, 
связанную с рассматриваемой моделью и~образующуюся на интервале времени 
$[t_n\hm+t^{(0)}, t_n\hm + t^{(0)} \hm+ \Delta^{(0)}]$.

Обозначим также 
$$
S(t^{(0)}, t^{(0)} + \Delta^{(0)}) = {\sf E}\tilde{S}(t^{(0)}, 
t^{(0)} +\Delta^{(0)}).
$$


\smallskip

\noindent
\textbf{Лемма 1.} \textit{Математическое ожидание случайной прибыли, образующейся 
на заданном конечном интервале времени, вложенном в~некоторый период 
регенерации, может быть аналитически выражено сле\-ду\-ющим образом}:

\vspace*{-6pt}

\noindent
\begin{multline*}
S(t^{(0)},t^{(0)}+\Delta^{(0)})={}\\
{}=\int\limits_{t^{(0)}}^{t^{(0)}+\Delta^{(0)}}
\left[\alpha\bar{d}_1(\tau-\alpha t)-\bar{c}_1(\tau-\alpha t)\right]dt,\\
0\leq t^{(0)}<t^{(0)}+\Delta^{(0)}\leq\dfrac{\tau}{\alpha};
\end{multline*}

%\vspace*{-12pt}

\noindent
\begin{multline*}
    S(t^{(0)},t^{(0)}+\Delta^{(0)})=\!\!\int\limits_{t^{(0)}}^{\tau/\alpha}[\alpha\bar{d
}_1(\tau-\alpha t)-\bar{c}_1(\tau-\alpha t)]\, dt+ {}\\
{}+
\int\limits_{\tau/\alpha}^{t^{(0)}+\Delta^{(0)}}[\alpha\bar{d}_2(\tau-\alpha t)-
\bar{c}_2(\tau-\alpha t)]\,dt,\\
0\leq t^{(0)}<\fr{\tau}{\alpha}<t^{(0)}+\Delta^{(0)}.
\end{multline*}

\vspace*{-6pt}

\noindent
Д\,о\,к\,а\,з\,а\,т\,е\,л\,ь\,с\,т\,в\,о\ леммы~1 основано на использовании утверждения о~математическом 
ожидании интеграла от случайной функции, приведенного в~монографии Липцера и~Ширяева~[9, гл.~1, \S~1], и~интегральном представлении для 
величины прибыли, образующейся на заданном конечном интервале времени.

Теперь возникает возможность найти явное представление для стационарного 
показателя эффективности управления~--- средней удельной прибыли. Соответствующий 
результат можно сформулировать в~виде тео\-ремы.

\smallskip

\noindent
\textbf{Теорема 1.} \textit{Предположим, что для любого управ\-ля\-юще\-го 
вероятностного распределения $G(\cdot )$ математическое ожидание длительности 
периода регенерации строго положительно. Тогда стационарный показатель средней 
удельной прибыли в~рас\-смат\-ри\-ва\-емой сто\-ха\-сти\-че\-ской модели регенерации 
пред\-став\-ля\-ет\-ся в~виде дроб\-но-ли\-ней\-но\-го интегрального функционала от 
распределения $G(\cdot)$, который имеет вид}:
\begin{equation}
    I(G)=\fr{\int\nolimits_0^\infty A(u)\, dG(u)}{\int\nolimits_0^\infty 
B(u)\, dG(u)}\,. \label{e11-sh}
\end{equation}
\textit{Здесь подынтегральные функции числителя и~знаменателя задаются следующими 
формулами}:

\vspace*{-6pt}

\noindent
\begin{multline}
        A(u)={}\\
        {}=\int\limits_0^{\tau/\alpha-u}\left[\int\limits_0^{u+z}[\alpha\bar{d}_1(\tau-\alpha t)-\bar{c}_1(\tau-\alpha 
t)] \,dt\right] dH(z)+{}\\
{}+\int\limits_{\tau/\alpha-u}^\infty\left[\int\limits_0^{\tau/\alpha}[\alpha\bar{d}_1(\tau-\alpha t)-
\bar{c}_1(\tau-\alpha t)] \,dt+{}\right.\\
\left.{}+\int\limits_{\tau/\alpha}^{u+z}[\alpha\bar{d}_2(\tau-\alpha t)-\bar{c}_2(\tau-
\alpha t)]\,dt\right] dH(z),\\
0\leq u\leq\fr{\tau}{\alpha};
\label{e12-sh}
\end{multline}

%\vspace*{-12pt}

\noindent
\begin{multline}
A(u)=\int\limits_0^\infty\left[\int\limits_0^{\tau/\alpha}[\alpha\bar{d}_1(\tau
-\alpha t)-\bar{c}_1(\tau-\alpha t)] \,dt+{}\right.\\
\left.{}+\int\limits_{\tau/\alpha}^{u+z}[\alpha\bar{d}_2(\tau-\alpha t)-\bar{c}_2(\tau-\alpha t)]\,dt\right]dH(z),\\
u> \fr{\tau}{\alpha};
\label{e13-sh}
\end{multline}

\vspace*{-12pt}

\noindent
\begin{equation}
    B(u)=u+\int\limits_0^\infty z\, dH(z)=u+T_0,\quad u\geq 0,
\label{e14-sh}
\end{equation}

\noindent
\textit{где $T_0=\mathrm{E}\,\delta$~--- математическое ожидание случайной длительности 
задержки поставки (заданная величина).}

\smallskip

\noindent
Д\,о\,к\,а\,з\,а\,т\,е\,л\,ь\,с\,т\,в\,о\ \ тео\-ре\-мы~1 основывается на использовании известного теоретического 
утверж\-де\-ния для регенерирующих процессов, а~именно: эргодической тео\-ре\-мы для 
стоимостных аддитивных функционалов, связанных с этими процессами. Кроме того, 
при доказательстве используется утверж\-де\-ние леммы~1 и~интегральные пред\-став\-ле\-ния 
для величины прибыли при различных соотношениях между параметрами.

\smallskip

\noindent
\textbf{Замечание 2.} Стоимостные характеристики модели $\bar{c}_1(x)$, 
$\bar{d}_1(x)$, $x\hm\geq0$, и~$\bar{c}_2(x)$, $\bar{d}_2(x)$, $x\hm<0$, фи\-гу\-ри\-ру\-ющие 
в~правых частях соотношений~(\ref{e12-sh}) и~(\ref{e13-sh}), определяются равенствами~(\ref{e5-sh})--(\ref{e8-sh}). При 
заданных вероятностных характеристиках случайных цен и~затрат эти функции 
становятся известными.

\vspace*{-6pt}

\section{Общее решение задачи оптимального управления
запасом}

\vspace*{-3pt}

Проблема оптимального управления запасом в~рассматриваемой стохастической модели 
регенерации может быть сформулирована как задача безусловного экстремума для 
дроб\-но-ли\-ней\-но\-го интегрального функционала~(\ref{e11-sh}):
\begin{equation}
    I(G)\underset{G\in\Gamma}{\rightarrow}\max\,,
    \label{e15-sh}
\end{equation}
где $\Gamma$~--- множество функций распределения неотрицательных случайных 
величин, принимающих значения из множества допустимых управлений $U \hm= [0, \infty)$.

Для решения экстремальной задачи~(\ref{e15-sh}) при\-меним теорему о~безуслов\-ном экстремуме 
дроб\-но-ли\-ней\-но\-го интегрального функционала, сформулированную и~доказанную 
в~работах автора \mbox{настоящего} исследования~[10, 11].
Заметим, что условия упо\-мя\-ну\-той тео\-ре\-мы выполнены. Действительно, если $T_0 \hm> 
0$, то име\-ют мес\-то соотношения:
\begin{gather*}
    B(u)=u+T_0>0,\qquad u\in U;\\
    \int\limits_U B(u)\,dG(u)=\int\limits_U[u+T_0]\,dG(u)>0,\enskip G\in\Gamma.
\end{gather*}
Согласно основному утверждению указанной тео\-ре\-мы, если основная функция данного 
дроб\-но-ли\-ней\-но\-го интегрального функционала
\begin{equation}
    C(u)=\fr{A(u)}{B(u)}\,,\qquad u\in U,\label{e16-sh}
\end{equation}
достигает своего глобального максимума на множестве $U\hm=[0, \infty)$ в~некоторой 
фиксированной точке $u_\ast\hm\in U$, то решение экстремальной задачи~(\ref{e15-sh}) 
существует и~достигается на вырожденном вероятностном распределении 
$G^\ast_{u_\ast}(u)$, сосредоточенном в~точке~$u_\ast$:

\noindent
\begin{equation*}
    G^\ast_{u_\ast}(u)=
    \begin{cases}0,&\enskip u\leq u_\ast;\\
    1,&\enskip 
u>u_\ast.\end{cases}
\end{equation*}

Таким образом, проблема оптимального управ\-ле\-ния в~данной стохастической модели 
регенерации сводится к~исследованию на глобальный максимум функции $C(u)$ вида~(\ref{e16-sh}), 
где функции $A(u)$ и~$B(u)$ заданы явными аналитическими формулами~(\ref{e12-sh})--(\ref{e14-sh}).

\vspace*{-6pt}

\section{Заключение}

Подводя итог проведенного исследования проб\-ле\-мы оптимального управления запасом 
непрерывного продукта в~стохастической модели регенерации со случайными 
стоимостными \mbox{характеристиками}, мож\-но утверж\-дать, что, как и~в~пред\-шест\-ву\-ющих 
исследованиях проб\-лем управ\-ле\-ния запасом в~различных регенерационных моделях~[5--8], 
решение задачи оптимального\linebreak управ\-ле\-ния сводится к~исследованию на 
глобальный экстремум так на\-зы\-ва\-емой основной функции дроб\-но-ли\-ней\-но\-го 
интегрального функционала, вы\-ра\-жа\-юще\-го стационарный стоимостный \mbox{показатель} 
эф\-фек\-тив\-ности управ\-ле\-ния. При этом указанная основная функция допускает явное 
аналитическое пред\-став\-ле\-ние через математические ожидания исходных стоимостных 
характеристик модели.

\vspace*{-6pt}


{\small\frenchspacing
 {\baselineskip=10.5pt
 %\addcontentsline{toc}{section}{References}
 \begin{thebibliography}{99}


\bibitem{163} 
\Au{Porteus E.\,L.} Foundations of stochastic inventory 
theory.~--- Stanford, CA, USA: Stanford Business Books, 2002. 320~p.

\bibitem{109} 
\Au{Axs$\ddot{\mbox{a}}$ter S.} Inventory control.~---   International ser. 
in operations research \&~management science.~--- Springer, 2015. Vol.~225. 281~p.
doi: 10.1007/978-3-319-15729-0.

\bibitem{iseg2} 
\Au{Iseger P., Oldenkamp E., Frenk~J.\,B.\,G.}  Inventory 
control and regenerative processes: Theory~// SSRN Electronic~J., 1999. 
23~p. doi: 10.2139/ssrn.1014782.

\bibitem{iseg1} 
\Au{Iseger P., Oldenkamp E., Frenk~J.\,B.\,G.}  Inventory 
control and regenerative processes: Computations~// SSRN Electronic~J., 
1999. 25~p. doi: 10.2139/ssrn.1014783.


\bibitem{103} 
\Au{Шнурков П.\,В., Мельников~Р.\,В.} Исследование проб\-ле\-мы 
управ\-ле\-ния запасом непрерывного продукта при детерминированной задержке по\-став\-ки~//  Автоматика и~телемеханика, 2008. №\,10. С.~93--113.

\bibitem{oyznp} 
\Au{Шнурков П.\,В., Пименова~Е.\,Ю.} Оптимальное 
управ\-ле\-ние запасом непрерывного продукта в~схеме регенерации с детерминированной 
за\-держ\-кой по\-став\-ки и~периодом реального пополнения~//  Системы и~средства 
информатики, 2017. Т.~27. №\,4. С.~80--94. doi: 10.14357/08696527170406. EDN: ZSUSNP.

\bibitem{94} 
\Au{Шнурков П.\,В., Вахтанов~Н.\,А.} Исследование проб\-ле\-мы 
оптимального управ\-ле\-ния запасом дискретного продукта в~стохастической модели 
регенерации с непрерывно происходящим по\-треб\-ле\-ни\-ем и~случайной за\-держ\-кой 
по\-став\-ки~//  Информатика и~её применения, 2019. Т.~13. Вып.~2.  С.~54--61. doi: 10.14357/19922264190208. EDN: ZTINLF.

\bibitem{95} 
\Au{Шнурков П.\,В., Вахтанов~Н.\,А.}  О~решении проблемы 
оптимального управления запасом дискретного продукта в~стохастической модели 
регенерации с непрерывно происходящим потреблением~//  Информатика и~её 
применения, 2019. Т.~13. Вып.~3. C.~50--57.  doi: 10.14357/19922264190308. EDN: BPKRGC.
 

\bibitem{liptser} 
\Au{Липцер Р.\,Ш., Ширяев~А.\,Н.} Статистика случайных 
процессов (нелинейная фильтрация и~смежные вопросы).~--- М.: Наука, 1974. 696~с.

\bibitem{83} 
\Au{Шнурков П.\,В.} О~решении задачи безусловного экстремума 
для дроб\-но-ли\-ней\-но\-го интегрального функционала на множестве вероятностных мер~//  Докл. Акад. наук. Сер.: Математика, 2016. Т.~470. №\,4. С.~387--392.
 doi: 10.7868/S0869565216280045. EDN: WLNMTH.
 

 


\bibitem{97}
 \Au{Шнурков~П.\,В., Горшенин~А.\,К., Белоусов~В.\,В.} 
Аналитическое решение задачи оптимального управления полумарковским процессом 
с~конечным множеством состояний~//  Информатика и~её применения, 2016. Т.~10. 
Вып.~4.  С.~72--88. doi: 10.14357/19922264160408. EDN: XGSITZ.
\end{thebibliography}

 }
 }

\end{multicols}

\vspace*{-12pt}

\hfill{\small\textit{Поступила в~редакцию 24.09.23}}

\vspace*{6pt}

%\pagebreak

\newpage

\vspace*{-28pt}

%\hrule

%\vspace*{2pt}

%\hrule



\def\tit{SOLUTION OF THE PROBLEM OF~OPTIMAL CONTROL OF~THE~STOCK OF~A~CONTINUOUS PRODUCT\\ IN~A~STOCHASTIC MODEL 
OF~REGENERATION\\ WITH~RANDOM COST CHARACTERISTICS\\[-7pt]}


\def\titkol{Solution of the problem of~optimal control of~the~stock of~a~continuous product in~a~stochastic model 
of~regeneration} % with~random cost characteristics}


\def\aut{P.\,V.~Shnurkov}

\def\autkol{P.\,V.~Shnurkov}

\titel{\tit}{\aut}{\autkol}{\titkol}

\vspace*{-16pt}


\noindent 
National Research University Higher School of Economics, 34 Tallinskaya Str., Moscow 123458, Russian Federation


\def\leftfootline{\small{\textbf{\thepage}
\hfill INFORMATIKA I EE PRIMENENIYA~--- INFORMATICS AND
APPLICATIONS\ \ \ 2023\ \ \ volume~17\ \ \ issue\ 4}
}%
 \def\rightfootline{\small{INFORMATIKA I EE PRIMENENIYA~---
INFORMATICS AND APPLICATIONS\ \ \ 2023\ \ \ volume~17\ \ \ issue\ 4
\hfill \textbf{\thepage}}}

%\vspace*{1pt}




\Abste{The work is devoted to the study of the problem of managing the stock of a~certain
continuous product, the evolution of the volume of which is described by a~regenerating stochastic process. 
The main feature of the
considered mathematical model is that the cost characteristics that determine the price of supplying the product
 to the consumer and the costs associated with ensuring the functioning of the system depend on random external factors. 
 The random control parameter is the time from the moment of the next replenishment of the stock to the moment 
 of the next order for replenishment. It is proved that the stationary cost indicator of control efficiency in the
  optimization problem under consideration in its analytical structure is 
  a~fractional-linear integral functional depending on the distribution function of the control parameter. The theoretical 
  solution of the optimization problem is based on the use of the extremum theorem for linear-fractional integral functionals.}

\KWE{continuous product inventory control problem; random cost characteristics of the system; controlled regenerative stochastic 
processes; linear-fractional integral functionals in problems of stochastic optimal control}

\DOI{10.14357/19922264230407}{WHZRNY}

%\vspace*{-16pt}

%\Ack
%\vspace*{-4pt}
%
%\noindent

  

\vspace*{-10pt}

  \begin{multicols}{2}

\renewcommand{\bibname}{\protect\rmfamily References}
%\renewcommand{\bibname}{\large\protect\rm References}

{\small\frenchspacing
 {\baselineskip=10.4pt
 \addcontentsline{toc}{section}{References}
 \begin{thebibliography}{99} 
 
 \vspace*{-4pt}
 
 \bibitem{163-1} 
\Aue{Porteus, E.\,L.} 2002. \textit{Foundations of stochastic inventory theory}. Stanford, CA: Stanford Business Book. 320~p.

\bibitem{109-1} 
\Aue{Axs$\ddot{\mbox{a}}$ter, S.} 2015. \textit{Inventory control}. 
International ser. in operations research \&~management science. Springer. Vol.~225. 281~p. doi: 10.1007/978-3-319-15729-0.



\bibitem{iseg2-1} 
\Aue{Iseger, P., E.~Oldenkamp, and J.\,B.\,G.~Frenk.}
 1999. Inventory control and regenerative processes: Theory. \textit{SSRN Electronic J}. 23~p. doi: 10.2139/ssrn.1014782.

\bibitem{iseg1-1} 
\Aue{Iseger, P., E.~Oldenkamp, and J.\,B.\,G.~Frenk.}
 1999. Inventory control and regenerative processes: Computations. \textit{SSRN Electronic J}. 25~p. doi: 10.2139/ssrn.1014783.

\bibitem{103-1} 
\Aue{Shnurkov, P.\,V., and R.\,V.~Mel'nikov.}
 2008. Analysis of the problem of continuous-product inventory control under deterministic lead time. 
  \textit{Automat. Rem. Contr.} 69(10):1734--1751. doi: 10.1134/S0005117908100081.

\bibitem{oyznp-1} 
\Aue{Shnurkov, P.\,V., and E.\,Yu.~Pimenova.}
 2017. Optimal'noe upravleniye zapasom nepreryvnogo produkta v~skheme regeneratsii s~determinirovannoy zaderzhkoy postavki i~periodom real'nogo popolneniya  
 [Optimal inventory control of continuous product in regeneration theory with determinate delay of the delivery and the period of real replenishment]. 
 \textit{Sistemy i~Sredstva Informatiki~--- Systems and Means of Informatics} 27(4):80--94. doi: 10.14357/08696527170406. EDN: ZSUSNP.

\bibitem{94-1} 
\Aue{Shnurkov, P.\,V., and N.\,A.~Vakhtanov.}
 2019. Issledovanie problemy optimal'nogo upravleniya zapasom diskretnogo produkta 
 v~stokhasticheskoy modeli regeneratsii s~nep\-re\-ryv\-no proiskhodyashchim potrebleniem i~sluchaynoy za\-derzh\-koy postavki 
 [Research of the optimal con-\linebreak\vspace*{-12pt}
 
 \columnbreak
 
 \noindent
 trol problem of inventory of a~discrete product in the stochastic regeneration model with continuously occuring 
 consumption and random delivery delay]. \textit{Informatika i~ee Primeneniya~--- Inform. Appl.} 13(2):54--61. doi: 10.14357/19922264190208. EDN: ZTINLF.

\bibitem{95-1}
\Aue{Shnurkov, P.\,V., and N.\,A.~Vakhtanov.}
 2019. O~reshenii problemy optimal'nogo upravleniya zapasom diskretnogo produkta v~stokhasticheskoy modeli regeneratsii 
 s~nepreryvno proiskhodyashchim potrebleniem [On the solution of the optimal control problem of inventory of a~discrete product in the stochastic model 
 of regeneration with continuously occuring consumption]. \textit{Informatika i~ee Primeneniya~--- Inform. Appl.} 13(3):50--57.
doi: 10.14357/19922264190308. EDN: BPKRGC.

\bibitem{liptser-1} 
\Aue{Liptser, R.\,S., and A.\,N.~Shiryaev.}
 1974. \textit{Statistika sluchaynykh protsessov (nelineynaya fil'tratsiya i~smezhnye voprosy)}
  [Statistics of random processes (nonlinear filtering and related problems)]. Moscow: Nauka. 696~p.

\bibitem{83-1}
\Aue{Shnurkov, P.\,V.} 2016. Solution of the unconditional extremum problem for a linear-fractional integral functional on 
a~set of probability measures. \textit{Dokl. Math.} 94(2):550--554.
doi: 10.1134/S1064562416050161. EDN: XFUJKH.

\bibitem{97-1} 
\Aue{Shnurkov, P.\,V., A.\,K.~Gorshenin, and V.\,V.~Belousov.}
 2016. Analiticheskoe reshenie zadachi optimal'nogo upravleniya polumarkovskim protsessom 
 s~konechnym mnozhestvom sostoyaniy [Analytical solution of the optimal control task
of a~semi-Markov process with finite set of states]. \textit{Informatika i~ee Primeneniya~--- Inform. \mbox{Appl.}} 10(4):72--88. 
doi: 10.14357/19922264160408. EDN: XGSITZ.

\end{thebibliography}

 }
 }

\end{multicols}

\vspace*{-14pt}

\hfill{\small\textit{Received September 24, 2023}} 

\vspace*{-26pt}

\Contrl

\vspace*{-6pt}


\noindent
\textbf{Shnurkov Peter V.} (b.\ 1953)~--- 
Candidate of Science (PhD) in physics and mathematics, associate professor,
 National Research University Higher School of Economics, 34~Tallinskaya Str., Moscow 123458, Russian Federation; \mbox{pshnurkov@hse.ru}


\label{end\stat}

\renewcommand{\bibname}{\protect\rm Литература} 