
%\def\F{{\rm I\!F}}
\def\P{{\rm I\!P}}

\def\stat{peshkova}

\def\tit{ГРАНИЦЫ НЕЗАВЕРШЕННОЙ РАБОТЫ В~СИСТЕМЕ С~ПОВТОРНЫМИ ВЫЗОВАМИ РАЗНЫХ КЛАССОВ 
И~ПОКАЗАТЕЛЬНЫМ ВРЕМЕНЕМ ОБСЛУЖИВАНИЯ$^*$}

\def\titkol{Границы незавершенной работы в~системе с~повторными вызовами разных классов 
и~показательным временем} % обслуживания}

\def\aut{И.\,В.~Пешкова$^1$}

\def\autkol{И.\,В.~Пешкова}

\titel{\tit}{\aut}{\autkol}{\titkol}

\index{Пешкова И.\,В.}
\index{Peshkova I.\,V.}


{\renewcommand{\thefootnote}{\fnsymbol{footnote}} \footnotetext[1]
{Работа выполнена при финансовой поддержке РНФ (проект 21-71-10135).}}


\renewcommand{\thefootnote}{\arabic{footnote}}
\footnotetext[1]{Петрозаводский государственный университет; 
Институт прикладных математических исследований Карельского 
научного центра Российской академии наук, \mbox{iaminova@petrsu.ru}}


%\vspace*{-12pt}



\Abst{Исследуется односерверная система 
с~повторными вызовами и~пуассоновским входным потоком, в~которую поступает~$M$ 
классов заявок.
Для системы с~временами обслуживания, имеющими показательное распределение, 
получены верхняя и~нижняя границы для стационарной незавершенной работы. 
В~качестве нижней границы выступает   стационарная незавершенная работа 
в~классической сис\-те\-ме  $M/H_M/1$ с~гиперэкспоненциальным временем обслуживания. 
Верхней границей служит незавершенная работа в~сис\-те\-ме, в~которой к~времени 
обслуживания добавляется дополнительное время, равное интервалу между попытками 
попасть на сервер с~самой <<медленной орбиты>>. Полученные результаты численного 
моделирования подтверждают теоретические выводы.}


\KW{система с~повторными вызовами; незавершенная 
работа; стохастическая упо\-ря\-до\-чен\-ность} 

\DOI{10.14357/19922264230408}{UOKQRD}
  
%\vspace*{-6pt}


\vskip 10pt plus 9pt minus 6pt

\thispagestyle{headings}

\begin{multicols}{2}

\label{st\stat}

\section{Введение}

Модели систем с~повторными вызовами широко используются для моделирования  
телефонных станций, кол-цент\-ров, сис\-тем связи, телекоммуникационных сетей. 
В~работах~\cite{Ar1, Ar3} \mbox{изложены}  приложения и~математические методы анализа 
сис\-тем c~повторными вызовами, включая сис\-те\-мы с~постоянной интенсивностью 
повторов. В~работе~\cite{F86}   телефонная станция была смоделирована  с~по\-мощью 
такой сис\-те\-мы. Аналогичная модель используется для описания широкого класса 
протоколов множественного доступа~\cite {CSA92, CRP93}. В~част\-ности, в~работе~\cite{BG92} 
показано, что постоянная интенсивность повторных вызовов  снижает 
интенсивность попыток (при незапланированном множественном доступе) обратно 
пропорционально числу задержанных пакетов. В~результате общая ско\-рость повторной 
обработки становится нечувствительной к~виртуальному <<размеру орбиты>> (числу 
отложенных пакетов). Более того, сис\-те\-мы с~повторными вызовами с~постоянной 
интенсивностью вызовов использовались для описания TCP-тра\-фи\-ка с~короткими HTTP-со\-еди\-не\-ни\-ями~\cite{AY08,AY10} 
и~оп\-ти\-ко-элект\-ри\-че\-ской гиб\-рид\-ной схемой разрешения 
конфликтов~\cite{Wongetal09,Yaoetal02}.

 Большинство же современных моделей повторных вызовов имеют сложную природу или 
конфигурацию, и~поэтому для их исследования применяются численные методы или 
имитационное моделирование.

Ранее в~работе~\cite{mathematics2022} была доказана тео\-ре\-ма о~верх\-ней и~ниж\-ней 
границах стационарной незавершенной работы  для исходной сис\-те\-мы с~повторными 
вызовами с~постоянной интенсивностью вызовов  (см.\ тео\-ре\-му~1 ниже). Эта тео\-ре\-ма 
стала основой анализа, развитого в~данной \mbox{статье}.

В данном исследовании рассматривается частный случай  односерверной сис\-те\-мы  
с~повторными вызовами с~пуассоновским входным потоком и~показательным 
распределением времени обслуживания, при этом  время обслуживания  и~время 
нахождения на орбите зависят от класса заявки~$k$.
%
Для такой системы предлагается строить две классические сис\-те\-мы с~неограниченной 
очередью (с~ожиданием) типа $M/G/1$: в~первой (минорантной) сис\-те\-ме время 
обслуживания пред\-став\-ля\-ет собой конечную смесь времен обслуживания заявок всех 
классов (т.\,е.\ имеет гиперэкспоненциальное распределение), во второй 
(мажорантной) сис\-те\-ме ко времени обслуживания первой сис\-те\-мы добавляется 
дополнительное время, равное  интервалу между вызовами  с~самой <<медленной 
орбиты>>.   Более того, для минорантной сис\-те\-мы получено распределение 
стационарного времени ожидания в~явном виде для трех классов ($M\hm=3$).
Сис\-те\-мы, в~которых  время обслуживания задано в~виде конечной смеси 
распределений, обсуждались ранее  в~работах~\cite{pesh-mor2022, pesh2022}.
 
Структура статьи следующая. В~разд.~2 приведено описание модели сис\-те\-мы 
с~повторными вызовами, минорантной и~мажорантной сис\-тем, а~также основная тео\-ре\-ма, 
полученная авторами в~работе~\cite{mathematics2022}.
В~разд.~3 получены коэффициенты загрузки, математические ожидания 
незавершенной нагрузки, а~также преобразования Лап\-ла\-са--Стилть\-еса для 
незавершенной нагрузки в~минорантной и~мажорантной сис\-те\-мах с~показательным 
распределением времени обслуживания. В~разд.~4 приведены результаты численного 
эксперимента для случая трех классов. При этом параметры для минорантной системы 
были использованы такие же, как в~работе~\cite{rego}, в~которой получена  
функция распределения  стационарного времени пребывания. Отметим, что  в~работе~\cite{rego} 
неверно указано, что полученное распределение~--- это распределение 
стационарного времени ожидания. В~статье получена функция распределения  
стационарного времени ожидания для минорантной сис\-те\-мы  в~явном  виде. Для 
исходной сис\-те\-мы с~повторными вызовами и~мажорантной  сис\-те\-мы проведены 
численные эксперименты и~построены эмпирические функции распределения. 
Полученные результаты численного эксперимента иллюстрируют доказанную 
стохастическую упо\-ря\-до\-чен\-ность стационарной незавершенной работы в~рассмотренных 
сис\-те\-мах.

\section{Описание модели}

Рассмотрим односерверную систему с~повторными вызовами~$\Sigma$, в~которой 
обслуживаются~$M$~классов заявок. Заявки $k$-го класса поступают в~сис\-те\-му в~соответствии 
с~пуассоновским потоком с~па\-ра\-мет\-ром~$\lambda_k$, $k\hm=1,\ldots,M$. 
Если заявка застает сервер пустым, то она немедленно обслуживается, в~противном 
случае, если сервер занят,  заявка уходит на $k$-ю орбиту бесконечного объема, 
образуя очередь в~порядке поступления на орбиту, $k\hm=1,\ldots,M$. Первая 
в~очереди на $k$-й орбите заявка производит независимые попытки присоединиться 
к~обслуживанию на сервере через экспоненциальное  время~$\eta_k$.
Интенсивность вызовов с~орбиты не зависит от размера орбиты (т.\,е.\ от числа 
заявок на орбите), но может зависеть от класса орбиты~$k$. Такие сис\-те\-мы 
называют сис\-те\-ма\-ми  с~постоянной ин\-тен\-сив\-ностью вызовов.

Обозначим через $t_n$ момент прихода $n$-й заявки в~общий пуассоновский входной  
поток (образованный суперпозицией~$M$~входных потоков, $t_1\hm=0$),   $S_n{(k)}$~--- 
время обслуживания  $n$-й заявки  $k$-го класса,  $k\hm=1,\ldots,M$, $n\hm\ge1$. Пусть 
последовательность независимых одинаково распределенных (н.\,о.\,р.)\ интервалов 
между приходами заявок  $\{T_n:=t_{n+1}\hm-t_n,\ n\hm\ge 1\}$ и~последовательность 
времен обслуживания  $\{S_n{(k)},\ n\hm\ge1\}$ независимы для каждого  $k$-го 
класса.
Предположим, что интервалы между приходами заявок с~(непустой)  $k$-й орбиты~$\eta_k$ 
распределены показательно и~не зависят от размера орбиты  (числа заявок 
на $k$-й орбите). Время обслуживания заявок $k$-го класса  $S(k)$ имеет 
произвольное распределение  с~функцией распределения (ф.~р.)\ $F_{S(k)}$, 
$k\hm=1,\ldots, M$. (Далее в~обозначениях  опускаем индекс номера заявки.)  Обозначим
\begin{equation*}
%\label{rates}
\lambda=\sum\limits_{k=1}^M\lambda_k ;\ \ \ \rho_k=\lambda_k\mathbb{E} S{(k)}; \enskip 
\rho=\sum\limits_{k=1}^M \rho_k.
\end{equation*}
Пусть $W(t)$ есть незавершенная работа в~момент времени~$t^-$, и~предположим, 
что система в~начальный момент времени пуста: $W(0)\hm=0$. Обозначим 
$W_n=W(t_n)$, $n\hm\ge1$.
Известно~\cite{Morozov2019}, что неравенство
 \begin{equation}
 \label{stability}
 \rho + \max\limits_{k=1,\ldots, M} \fr{\lambda}{\lambda+\eta_k} < 1
 \end{equation}
служит достаточным условием стационарности сис\-те\-мы.  При  таком условии 
существует  предел
$$
W_n \Rightarrow W\,,\quad n\to\infty
$$
(где обозначим $\Rightarrow$ схо\-ди\-мость по распределению). Функция распределения~$F_W$ 
стационарной незавершенной работы~$W$ для исходной сис\-те\-мы~$\Sigma$ неизвестна. 
С~другой стороны, $W$ служит важной характеристикой качества обслуживания 
сис\-те\-мы. Ниже мы построим верхнюю и~нижнюю границы~$F_W$, используя 
классические  $M/G/1$ сис\-те\-мы с~неограниченной оче\-редью, в~которых время 
обслуживания пред\-став\-ля\-ет\-ся конечной смесью распределений.


Рассмотрим две новые системы: \textit{минорантную сис\-те\-му}~$\Sigma^{(1)}$ 
и~\textit{мажорантную сис\-те\-му}~$\Sigma^{(2)}$. (Далее индекс~$(i)$ означает номер 
сис\-те\-мы.)
Входной поток во все три сис\-те\-мы~--- пуассоновский  с~параметром~$\lambda$ (это 
суперпозиция входных потоков, образованных заявками разных классов).

Пусть в~минорантной сис\-те\-ме~$\Sigma^{(1)}$ время обслуживания $S^{(1)}\hm=S$ задано 
конечной смесью распределений вида
\begin{equation}
\label{mixture}
S=\sum\limits_{k=1}^M I(k) S(k), \enskip n\ge1\,,
\end{equation}
где  $I(k)$~--- индикатор, такой что  $\mathbb{E} I(k)\hm=p_k=\lambda_k/\lambda$; $S(k)$~--- время  обслуживания заявки $k$-го класса.
Заметим, что сис\-те\-ма~$\Sigma^{(1)}$ пред\-став\-ля\-ет собой классическую 
(с~дисциплиной обслуживания первым при\-шел\,--\,пер\-вым обслужен) односерверную сис\-те\-му 
типа $M/G/1$ с~неограниченной очередью, в~которой время обслуживания~(\ref{mixture})
является конечной смесью времен обслуживания заявок всех классов исходной 
сис\-те\-мы.

В мажорантной системе~$\Sigma^{(2)}$~--- классической односерверной сис\-те\-ме 
типа  $M/G/1$ с~неограниченной оче\-редью~--- каждая заявка обслуживается на 
сервере в~течение времени~$S$, заданного соотношением~\eqref{mixture},  плюс 
время~$\xi_0$, имеющее показательное распределение с~па\-ра\-мет\-ром  
$$
\mu_0=\min\limits_{1\le k\le M} (\lambda\hm+\eta_k),
$$
 т.\,е.
%\begin{equation*}
%\label{sums2}
$S^{(2)} \hm= S\hm +\xi_0$.
%\end{equation*}
Таким образом, случайная величина (с.\,в.)~$\xi_0$ соответствует самой 
<<медленной>> орбите (с наибольшими интервалами между попытками). Заметим, что 
мажорантная сис\-те\-ма~$\Sigma^{(2)}$ имеет другой коэффициент загрузки,
\begin{equation*}
%\label{rho2def}
 \rho^{(2)}=\lambda \mathbb{E} S+\fr{\lambda}{\mu_0}=\rho+\fr{\lambda}{\mu_0}\,,
\end{equation*}
и условие стационарности~\eqref{stability} для нее принимает вид:
$$
\rho^{(2)}<1.
$$

В работе~\cite{mathematics2022}  доказана сле\-ду\-ющая тео\-ре\-ма, в~которой даны 
верхняя и~нижняя границы незавершенной работы~$W$ в~исходной сис\-те\-ме 
с~повторными вызовами~$\Sigma$.

\smallskip

\noindent
\textbf{Теорема~1.}
\textit{Пусть сис\-те\-мы  $\Sigma^{(1)}$, $\Sigma^{(2)}$ и~$\Sigma$ в~начальный момент времени 
пустые, т.\,е.}
 \begin{equation*}
W_1^{(1)}=W_1=W_1^{(2)}=0\,.
 \end{equation*}
\textit{Тогда при  выполнении условия}~\eqref{stability} \textit{стационарные времена 
незавершенной работы стохастически упорядочены}:
 \begin{equation}
 \label{theor1-1}
 W^{(1)}\underset{\mathrm{st}}\le W \underset{\mathrm{st}}\le W^{(2)},
 \end{equation}
\textit{где $W^{(1)}\le_{\mathrm{st}} W$ означает $\overline F_{W^{(1)}} (x) \hm\le 
\overline F_{W} (x) $ для любого $x\hm\ge 0$, $\overline F_{W^{(1)}}  (x)\hm= 1\hm-  
F_{W^{(1)}} (x)$}.


\smallskip

В следующем разделе применим данный результат для сис\-те\-мы, в~которой~$M$~классов 
заявок, име\-ющих  показательное распределение времени обслуживания.

\section{Границы незавершенной работы~$W$ в~системе с~показательным 
обслуживанием разных классов }

Пусть в~исходной сис\-те\-ме с~повторными вызовами~$\Sigma$ времена обслуживания 
$k$-го класса~$S(k)$ имеют показательное распределение с~ф.\,р.
\begin{equation}
\label{hyperexp}
F_{S(k)}(x)= 1- e^{-\mu_k x}, \enskip x\ge 0\,, \ \mu_k >0\,.
\end{equation}

В качестве минорантной сис\-те\-мы~$\Sigma^{(1)}$ рассмотрим сис\-те\-му 
с~неограниченной  очередью  $M/H_M/1$, в~которой времена обслуживания $S^{(1)}\hm=S$ 
имеют гиперэкспоненциальное распределение (пред\-став\-ля\-ют\-ся $M$-ком\-по\-нент\-ной 
смесью показательно распределенных с.\,в.~$S(k)$) с~ф.\,р.
\begin{multline*}
F_{S^{(1)}}(x) = 1 -  \sum\limits_{k=1}^M p_k e^{-\mu_k x}, \enskip \mu_k > 0\,, \\ 
p_k\ge 0\,,\enskip k=1,\ldots,M, \enskip \sum\limits_{i=k}^M p_k=1\,.
\end{multline*}

Обозначим коэффициент загрузки  $\rho^{(1)} \hm=\sum\nolimits_{k=1}^M \lambda 
p_k/\mu_k$ в~сис\-те\-ме~$\Sigma^{(1)}$. Поскольку
$$
\rho^{(1)} \le \rho + \fr{\lambda}{\mu_0 }\,,
$$
то, если условие стационарности~\eqref{stability} выполнено,    сис\-те\-ма~$\Sigma^{(1)}$ также стационарна.

Рассмотрим преобразование Лап\-ла\-са--Стилть\-еса:
\begin{equation*}
%\label{lstdef}
 \psi_{S_e}(z)=\int\limits_0^\infty e^{-zt} \,dF_{S_e}(t),
\end{equation*}
где $F_{S_e}$ -- так называемый \textit{интегрированный хвост
распределения} с~плот\-ностью
\begin{equation*}
%\label{fequilibr}
f_{S_e}(x)=\fr{1}{\mathbb{E} S}\, \overline F_S(x),\enskip x\ge0\,.
% f_{S_e}(x)=\mu \overline{F}_S(x),\enskip x\ge0\,.
\end{equation*}

Распределение $F_{S_e}$ соответствует распределению стационарного перескока 
процесса вос\-ста\-нов\-ле\-ния, фор\-ми\-ру\-емо\-го по\-сле\-до\-ва\-тель\-ностью н.\,о.\,р.\ времен 
обслуживания~$\{S_n\}$~\cite{Asmus}.

В работе~\cite{mathematics2022} доказано, что преобразование\linebreak Лап\-ла\-са--Стилть\-еса 
стационарной незавершенной работы~$W^{(1)}$ выражается через преобразования 
Лап\-ла\-са--Стилть\-еса компонент смеси времен обслуживания в~сле\-ду\-ющем виде:
\begin{multline}
\label{lstformixture}
\psi_{W^{(1)}}(z)=\fr{1-\rho}{z\left(1-\rho\sum\nolimits_{k=1}^M 
(\rho_k/{\rho}) \psi_{S_e(k)}(z)\right)}={}\\
{}=\fr{1-\rho}{z\left(1-
\sum\nolimits_{k=1}^M \rho_k \psi_{S_e(k)}(z)\right)}.
\end{multline}

Преобразование Лапласа--Стилть\-еса для показательного распределения хорошо 
известно: 
$$
\psi_{S_e(k)}(z) = \fr{\mu_k}{\mu_k +z}\,.
$$ 
Подставляя его в~соотношение~\eqref{lstformixture},  получим
$$
\psi_{W^{(1)}}(z)=\fr{1-\sum\nolimits_{k=1}^M \lambda_k/\mu_k}{z\left(1-
\sum\nolimits_{k=1}^M  {\lambda_k}/(\mu_k +z)   
\right)}\,.
$$

Применяя формулу По\-ла\-чи\-ка--Хин\-чи\-на, получим среднюю стационарную незавершенную 
работу  в~сис\-те\-ме~$\Sigma^{(1)}$ в~виде:
\begin{equation}
\label{ew5}
\mathbb{E}  W^{(1)} = \fr{\lambda \mathbb{E} ( S^{(1)})^2}{2(1-\rho^{(1)})} = 
\fr{\sum\nolimits_{k=1}^M \rho_k^2 +\rho^2}{2\lambda (1-\rho)}\,.
\end{equation}


Рассмотрим теперь мажорантную сис\-те\-му~$\Sigma^{(2)}$,  время обслуживания 
в~которой равно сумме с.\,в.~$S$ с~гиперэкспоненциальным распределением~\eqref{hyperexp} и~с.~в.~$\xi_0$, т.\,е.\
 $\ S^{(2)}\hm=S \hm+ \xi_0$. Обозначим для простоты 
$\mu_0\hm=\min\nolimits_{1\le k\le M} (\lambda\hm+\eta_k)$, и~пусть с.\,в.~$\xi_0$ имеет 
показательное распределение  с~па\-ра\-мет\-ром~$\mu_0$. Условие стационарности 
в~такой сис\-те\-ме совпадает с~\eqref{stability}.

Известно \cite{mathematics2022}, что в~сис\-те\-ме~$\Sigma^{(2)}$ преобразование 
Лап\-ла\-са--Стилть\-еса для стационарной незавершенной работы~$W^{(2)}$ имеет 
сле\-ду\-ющий вид:
\begin{multline}
\label{reslemma2}
 \psi_{W^{(2)}} (z)=
 \left(1-\rho- \fr{\lambda}{\mu_0}\right) \Bigg/
\left(  z\left(
\vphantom{\left(\sum\limits_{k=1}^M\right)}
1-{}\right.\right.\\
\left.\left.{}-\fr{\mu_0}{\mu_0+z}\left(\sum\limits_{k=1}^M \rho_k \psi_{S_e(k)}(z) +
\fr{\lambda}{\mu_0}\right)\right)\right)\,.
\end{multline}
Подставив $\psi_{S_e(k)}(z) = \mu_k/(\mu_k \hm+z)$ в~\eqref{reslemma2}, получим
\begin{multline*}
\psi_{W^{(2)}} (z)=
\left(1-\sum\limits_{k=1}^M \fr{\lambda_k}{\mu_k}-\fr{\lambda}{\mu_0}\right) \Bigg/
\left(z\left(
\vphantom{\left(\sum\limits_{k=1}^M\right)}
1-{}\right.\right.\\
\left.\left.{}-\fr{\mu_0}{\mu_0+z}\left(\sum\limits_{k=1}^M \fr{\lambda_k}{\mu_k+z}  +\fr{\lambda}{\mu_0}\right)\right)\right)\,.
\end{multline*}
Аналогично формуле~\eqref{ew5} можно получить среднюю стационарную незавершенную 
работу
\begin{multline}
\label{ew6}
\mathbb{E}  W^{(2)} = \fr{\lambda \mathbb{E} \left( S^{(2)}\right)^2}{2\left(1-\rho^{(2)}\right)} = {}\\
{}=
\fr{\mu_0^2\left(\rho^2 + \sum\nolimits_{k=1}^M \rho_k^2 \right) +2 \lambda (\lambda +\rho \mu_0)}{2\lambda \mu_0 (\mu_0 -\rho \mu_0 -\lambda)}.
\end{multline}



Из теоремы~1 следует, что
стационарные времена незавершенной работы стохастически упорядочены:
 $$
 W^{(1)}\underset{\mathrm{st}}\le W \underset{\mathrm{st}}\le W^{(2)},
 $$
а следовательно,  их математические ожидания также упорядочены~\cite{Ross}:
$$
 \mathbb{E} W^{(1)}\le \mathbb{E} W \le \mathbb{E} W^{(2)}.
$$

Действительно, легко проверить, что
\begin{multline*}
\mathbb{E}  W^{(2)} = \fr{\lambda \mathbb{E} \left( S^{(2)}\right)^2}{2(1-\rho^{(2)})} = {}\\
{}=
\fr{\mu_0^2 \lambda \mathbb{E} ( S^{(1)})^2 + 2 \lambda (\lambda +\rho \mu_0)} {\mu_0^2 2(1-\rho^{(1)}) - 2\lambda^2 \mu_0 } 
\ge \mathbb{E}  W^{(1)}.
\end{multline*}


\section{Численный эксперимент}

В качестве примера рассмотрим систему с~повторными вызовами~$\Sigma$ с~тремя 
классами заявок ($M\hm=3$), в~которую поступает пуассоновский поток 
с~ин\-тен\-сив\-ностью $\lambda\hm=10$.
Пусть
$p_1\hm=1/2$, $p_2\hm=1/3$, $p_3\hm=1/6$, $\mu_1\hm=10$, $\mu_2\hm=30$ и~$\mu_3\hm=60$.
Будем полагать, что~$\eta_k$ принимают значения
$\eta_1\hm=50$, $\eta_2\hm=100$ и~$\eta_3\hm=150.$
В~этом случае $\mu_0\hm=\lambda\hm+\eta_1\hm=60$,  коэффициенты загрузки \mbox{равны}
$$
\rho^{(1)}=\rho= \sum\limits_{k=1}^3 \fr{\lambda p_k}{\mu_k} = \fr{23}{36}\,; \enskip  
\rho^{(2)} = \rho+\fr{\lambda}{\mu_0} = \fr{29}{36} < 1
$$
и условие стационарности~\eqref{stability} выполнено. По 
формулам~\eqref{ew5}--\eqref{ew6} находим математические ожидания $\mathbb{E} W^{(1)} \hm\approx 0{,}093$ 
и~$\mathbb{E} W^{(2)} \hm\approx 0{,}106$.

Рассмотрим минорантную сис\-те\-му~$\Sigma^{(1)}$, в~которой время обслуживания 
имеет гиперэкспоненциальное распределение
$$
\overline F_S(x)=\sum\limits_{k=1}^3 p_k e^{-\mu_k x}.
$$
%
Для такой классической системы $M/G/1$ в~работе~\cite{rego} получено 
распределение числа клиентов в~сис\-те\-ме в~стационарном режиме~$N^{(1)}$ в~сле\-ду\-ющем виде:
$$
\pi_n^\ast=\mathbb{P}\left\{N^{(1)}=n\right\}=\sum\limits_{k=1}^3 \beta_k (r_k)^n,
$$
где параметры  $\beta_k$ и~$r_k$ вычислены и~равны
\begin{alignat*}{3}
    \beta_1&=0{,}040;&\enskip \beta_2&=0{,}075;&\enskip \beta_3&=0{,}245; \\
    r_1&=0{,}146;&\enskip r_2&=0{,}268;&\enskip r_3&=0{,}711.
\end{alignat*}
В работе~\cite{rego} с~помощью результата~\cite{haji}  получено стационарное 
распределение  \textit{времени пребывания} клиента в~сис\-те\-ме~$V^{(1)}$ с~хвостом 
ф.\,р.\ вида
$$
\overline F_{V^{(1)}}(x)=\sum\limits_{k=1}^3 \gamma_k e^{-\theta_k x},
$$
где коэффициенты $\gamma_k$  и~па\-ра\-мет\-ры~$\theta_k$  для исходных параметров 
\mbox{равны}
\begin{alignat*}{3}
\gamma_1&=0{,}047;&\quad \gamma_2&=0{,}103; &\quad \gamma_3&=0{,}850; \\
\theta_1&=58{,}633; &\quad \theta_2&=27{,}307;&\quad \theta_3&= 4{,}059.
\end{alignat*}
При этом в~утверждении теоремы~3 из работы ~\cite{rego} было  ошибочно указано, 
что $\sum\nolimits_{k=1}^3 \gamma_k\hm=1\hm-\rho^{(1)}$ (что было бы верно, если бы 
распределение~$F_V^{(1)}$ соответствовало  \textit{времени ожидания}). На самом 
деле легко проверить, что $\sum\nolimits_{k=1}^3 \gamma_k\hm=1$. Для исправления этой 
неточности повторим вывод, получив выражение для стационарного времени 
\textit{ожидания} в~сис\-те\-ме (что  соответствует незавершенной работе в~сис\-те\-ме 
в~момент прихода клиента). Заметим, что~$\pi_{n+1}^\ast$ есть стационарная 
вероятность наблюдать~$n$~клиентов в~очереди, т.\,е.
$$
\mathbb{P}\{Q^{(1)}=n\}=\pi_{n+1}^\ast,
$$
где $Q^{(1)}$ есть число клиентов \textit{в очереди} в~стационарном режиме.

Вычислив производящую функцию вероятностей~$\pi (z)$ для~$Q^{(1)}$, получим
\begin{multline*}
    \pi (z)=\sum\limits_{n=0}^\infty z^n \pi_{n+1}^\ast = \sum\limits_{k=1}^3
    \fr{\beta_k }{z}\sum\limits_{n=1}^\infty (r_k z)^n={}\\
    {}=\sum\limits_{k=1}^3 \fr{\beta_k }{z}\left(\fr{1}{1-r_k z}-1\right)=\sum\limits_{k=1}^3 \fr{\beta_k r_k}{1-r_k z}\,.
\end{multline*}

C другой стороны,  производящая функция стационарной очереди~$\pi(z)$ и~преобразование Лап\-ла\-са--Стилть\-еса для стационарного времени 
ожидания~$\psi_{W^{(1)}}(z)$ связаны формулой:
\begin{multline*}
\pi(z)= \sum\limits_{n=0}^{\infty}  \int\limits_{0^-}^{\infty} z^n e^{-\lambda x} \fr{(\lambda x )^n}{n!}\, dF_{W^{(1)}} (x) ={}\\
{}=
\int\limits_{0^-}^{\infty} e^{-(\lambda-\lambda z) x} \, d F_{W^{(1)}} (x) ={}\\
{}=\psi_{W^{(1)}} (\lambda - \lambda z) + \left(1-\rho^{(1)}\right),
\end{multline*}
 где $F_{W^{(1)}}(0) = (1-\rho^{(1)})$~--- скачок  ф.~р.\ в~нуле. Сделав замену 
переменной $s\hm=\lambda\hm-\lambda z$,  получим
$$
\psi_{W^{(1)}} (s)=\sum_{k=1}^3 \fr{\beta_k r_k}{1-r_k(1-
s/\lambda)}=\sum\limits_{k=1}^3 \fr{\beta_k r_k}{1-r_k}\,\fr{\theta_k}{\theta_k+ s}\,,
$$
где, как и~в~работе~\cite{rego},
$$
\theta_k=\fr{\lambda(1-r_k)}{r_k}\,.
$$

{ \begin{center}  %fig1
 \vspace*{-1pt}
     \mbox{%
\epsfxsize=79mm 
\epsfbox{pes-1.eps}
}

\end{center}



\noindent
{\small{Функции распределения в~нижней~$\Sigma^{(1)}$~(\textit{1}), исходной $\Sigma$~(\textit{2}) 
и~верх\-ней~$\Sigma^{(2)}$~(\textit{3}) сис\-те\-мах при $\lambda \hm= 10$, $p_1\hm= 1/2$, $p_2\hm= 1/3$, $p_3\hm=1/6$, 
$\mu_1\hm=10$, $\mu_2\hm=30$, $\mu_3\hm= 60$, $\eta_1\hm=50$, $\eta_2\hm=100$ и~$\eta_3\hm=150$}}}

\vspace*{12pt}

\noindent
Таким образом,~$\psi_{W^{(1)}}$ соответствует взвешенной  сумме показательных 
распределений. Отметим при этом, что, в~отличие от~\cite{rego}, коэффициенты 
смеси имеют вид:
$$
\hat\gamma_k=\fr{\beta_k r_k}{1-r_k}\,.
$$
Таким образом,
\begin{equation}
\label{fwlower}
\overline F_{W^{(1)}}(x)=\sum\limits_{k=1}^3 \hat\gamma_k e^{-\theta_k x} + \left(1- \rho^{(1)}\right),
\end{equation}
где
$\hat\gamma_1=0{,}007$, $\hat\gamma_2\hm=0{,}027$ и~$\hat\gamma_3\hm=0{,}604$. Заметим, что 
$\sum\nolimits_{k=1}^3 \hat\gamma_k\hm=\rho^{(1)}\hm\approx 0{,}638\hm <1.$

Воспользуемся выражением~\eqref{fwlower} для по\-стро\-ения ф.\,р.\ 
в~сис\-те\-ме~$\Sigma^{(1)}$, а~для по\-стро\-ения оценок в~исходной~$\Sigma$ и~верхней~$\Sigma^{(2)}$ 
сис\-те\-мах воспользуемся имитационным моделированием.
Построим графики (эмпирических) ф.~р.\ для незавершенной 
работы в~трех сис\-те\-мах. Как видно на рисунке, стохастический 
порядок~\eqref{theor1-1} для стационарных времен ожидания выполнен, что 
и~следовало ожидать.






\section{Заключение}

В работе показано, что для исходной системы с~повторными вызовами можно 
построить минорантную и~мажорантную системы так, что стационарная незавершенная 
нагрузка во всех трех сис\-те\-мах будет стохастически упорядочена. Численный 
эксперимент для сис\-те\-мы с~показательными временами обслуживания подтверждает 
теоретические выводы. При этом в~качестве примера рас\-смот\-ре\-ны такие па\-ра\-мет\-ры 
(как в~работе~\cite{rego}), для которых получена ф.\,р.\ 
стационарного времени ожидания в~явном виде в~минорантной сис\-теме.


{\small\frenchspacing
 { %\baselineskip=10.6pt
 %\addcontentsline{toc}{section}{References}
 \begin{thebibliography}{99}

\bibitem{Ar1}
\Au{Artalejo J.\,R.} {Accessible bibliography on retrial queues}~// Math. 
Comput. Model., 1999. Vol.~30. Iss.~3-4. P.~1--6. doi: 10.1016/S0895-7177(99)00128-4.


\bibitem{Ar3}
\Au{Artalejo J.,   Gomez-Corral~A.}
{Retrial queueing systems: A~computational approach}.~---   Springer, 2008. 318~p.
doi: 10.1007/978-3-540-78725-9.


\bibitem{F86} 
\Au{Fayolle G.}
A~simple telephone exchange with delayed feedbacks~// 
Seminar (International) on Teletraffic Analysis and Computer Performance 
Evaluation Proceedings.~--- Elseiver Science, 1986. P.~245--253.

\bibitem{CSA92}
\Au{Choi~B.\,D.,  Shin~Y.\,W.,  Ahn~W.\,C.}
Retrial queues with collision arising from unslotted {CSMA/CD} protocol~//
Queueing Syst., 1992.  Vol.~11. P.~335--356. doi: 10.1007/ BF01163860.

\bibitem{CRP93}
\Au{Choi B.\,D., Rhee~K.\,H., Park~K.\,K.} {The $M/G/1$ retrial queue with
retrial rate control policy}~//
Probab.  Eng. Inform. Sc., 1993.  Vol.~7. P.~29--46. doi: 10.1017/ S0269964800002771.

\bibitem{BG92}
\Au{Bertsekas D., Gallager~R.}
{Data networks}.~--- Athena Scientific, 2021.  570~p.

\bibitem{AY08}
\Au{Avrachenkov K., Yechiali~U.}
Retrial networks with finite buffers and their application to Internet data 
traffic~//  Probab. Eng. Inform. Sc., 2008. 
Vol.~22. P.~519--536. doi: 10.1017/S0269964808000314.

\bibitem{AY10} %8
\Au{Avrachenkov K., Yechiali~U.}
{On tandem blocking queues with a~common retrial queue}~// Comput.  
Oper. Res., 2010. Vol.~37. Iss.~7. P.~1174--1180. doi: 10.1016/j.cor.2009. 10.004.



\bibitem{Yaoetal02} %9
\Au{Yao S., Xue~F.,  Mukherjee~B.,  Yoo~S.\,J.\,B., Dixit~S.}
{Electrical ingress buffering and traffic aggregation for optical packet 
switching and their effect on TCP-level performance in optical mesh networks}~//
IEEE Commun. Mag., 2002.
Vol.~40. Iss.~9. P.~66--72. doi: 10.1109/MCOM. 2002.1031831.

\bibitem{Wongetal09} %10
\Au{Wong E.\,W.\,M.,  Andrew L.\,L.\,H.,  Cui~T.,  Moran~B.,  Zalesky~A., Tucker~R.\,S., Zukerman~M.}
{Towards a~bufferless optical internet}~//
J.~Lightwave Technol., 2009. Vol.~27. P.~2817--2833. doi: 10.1109/JLT.2009.2017211.

\bibitem{mathematics2022} %11
\Au{Morozov E.\,V., Peshkova~I.\,V., Rumyantsev~A.\,S.} Bounds and maxima for the 
workload in a~multiclass orbit queue~// Mathematics, 2023. Vol.~11. Iss.~3. 
Art.~564. doi: 10.3390/math11030564.

\bibitem{pesh-mor2022}  %12
\Au{Peshkova I., Morozov~E.} On comparison of 
multiserver systems with multicomponent mixture distributions~// J.~Math. Sci., 2022. Vol.~267. No.\,2. P.~260--272.
doi: 10.1007/ s10958-022-06132-z. 

\bibitem{pesh2022} %13
\Au{Пешкова И.\,В.} 
Границы экстремального индекса времени ожидания в~системе
$M/G/1$ с~распределением времени обслуживания в~виде конечной
смеси~// Информатика и~её применения, 2022.
Т.~16. Вып.~2. С.~26--33. doi: 10.14357/19922264220405. EDN: VFKRKT.



\bibitem{rego} %14
\Au{Rego V.}
Some explicit formulas for mixed exponential service systems~//
Computers Operations Research, 1988. Vol.~15. Iss.~6. P.~509--520. doi: 
{10.1016/0305-0548(88)90047-0}.

\bibitem{Morozov2019}  %15
\Au{Morozov E.\,V.,   Rumyantsev~A.\,S., Dey~S.,  Deepak~T.\,G.}
Performance analysis and stability of multiclass orbit queue with constant 
retrial rates and balking~//
 Perform. Evaluation, 2019.  Vol.~134. Art.~102005. doi: 
10.1016/ J.PEVA.2019.102005.

\bibitem{Asmus} %16
\Au{Asmussen S.} Applied probability and queues. Stochastic modelling and 
applied probability.~--- New York, NY, USA: Springer-Verlag, 2003. 438~p.

\bibitem{Ross} %17
\Au{Ross S., Shanthikumar~J., Zhu~Z.}  On increasing-failure-rate random 
variables~// J.~Appl. Probab., 2005. Vol.~42. P.~797--809. doi: 
10.1239/jap/1127322028.

\bibitem{haji} %18
\Au{Haji R.,  Newell~G.\,F.}  A~relation between stationary queue and waiting 
time distributions~// J.~Appl. Probab., 1971. Vol.~8. P.~617--620. doi: 10.2307/3212186.




\end{thebibliography}

 }
 }

\end{multicols}

\vspace*{-10pt}

\hfill{\small\textit{Поступила в~редакцию 26.08.23}}

\vspace*{8pt}

%\pagebreak

%\newpage

%\vspace*{-28pt}

\hrule

\vspace*{2pt}

\hrule



\def\tit{BOUNDS OF THE WORKLOAD IN~A~MULTICLASS RETRIAL QUEUE WITH~EXPONENTIAL SERVICES}


\def\titkol{Bounds of the workload in~a~multiclass retrial queue with~exponential services}


\def\aut{I.\,V.~Peshkova$^{1,2}$}

\def\autkol{I.\,V.~Peshkova}

\titel{\tit}{\aut}{\autkol}{\titkol}

\vspace*{-10pt}


\noindent 
$^1$Petrozavodsk State University, 33~Lenina Pr., Petrozavodsk 185910, Russian Federation

\noindent 
$^2$Karelian Research Center of
the Russian Academy of Sciences, 11~Pushkinskaya Str., Petrozavodsk 185910,\linebreak
$\hphantom{^1}$Russian Federation 

\def\leftfootline{\small{\textbf{\thepage}
\hfill INFORMATIKA I EE PRIMENENIYA~--- INFORMATICS AND
APPLICATIONS\ \ \ 2023\ \ \ volume~17\ \ \ issue\ 4}
}%
 \def\rightfootline{\small{INFORMATIKA I EE PRIMENENIYA~---
INFORMATICS AND APPLICATIONS\ \ \ 2023\ \ \ volume~17\ \ \ issue\ 4
\hfill \textbf{\thepage}}}

\vspace*{3pt}

 


\Abste{A~multiclass retrial queue with Poisson input and $M$ classes of customers is investigated. 
For the given retrial system with exponential service times, the lower and upper bounds of the workload are derived. 
It is shown that the workload in the classical system $M/H_M/1$ with hyperexponential service times is the lower bound for the workload of the given retrial system. 
The upper bound is the workload in the classical $M/G/1$ system where each customer occupies the server for the given service time and additional
 time corresponding to the inter-retrial time from the ``slowest'' orbit. 
The presented simulation results confirm the theoretical conclusions.}


\KWE{retrial queue; workload; stochastic ordering}  




\DOI{10.14357/19922264230408}{UOKQRD}

\vspace*{-12pt}

\Ack

\vspace*{-4pt}

\noindent
The research has been prepared with the support of the Russian Science Foundation according to
the research project No.\,21-71-10135. 



  \begin{multicols}{2}

\renewcommand{\bibname}{\protect\rmfamily References}
%\renewcommand{\bibname}{\large\protect\rm References}

{\small\frenchspacing
 {%\baselineskip=10.8pt
 \addcontentsline{toc}{section}{References}
 \begin{thebibliography}{99} 
%1
\bibitem{Ar1-1}
\Aue{Artalejo, J.\,R.} 1999. Accessible bibliography on retrial queues. \textit{Math.
Comput. Model.} 30(3-4):1--6. doi: 10.1016/S0895-7177(99)00128-4.
%2
\bibitem{Ar3-1}
\Aue{Artalejo, J., and A.~Gomez-Corral.} 2008. 
\textit{Retrial queueing systems: A computational approach}. Springer. 318~p.
doi: 10.1007/978-3-540-78725-9.
%3
\bibitem{F86-1} 
\Aue{Fayolle, G.} 1986. 
A simple telephone exchange with delayed feedbacks. \textit{Seminar (International) on Teletraffic Analysis and Computer Performance Evaluation Proceedings}.
Elseiver Science. 245--253.
%4
\bibitem{CSA92-1}
\Aue{Choi, B.\,D., Y.\,W.~Shin, and W.\,C.~Ahn.} 1992. 
Retrial queues with collision arising from unslotted \mbox{CSMA}/CD protocol. \textit{Queueing Syst.} 11:335--356.  
doi: 10.1007/ BF01163860.
%5
\bibitem{CRP93-1}
\Aue{Choi, B.\,D., K.\,H.~Rhee, and K.\,K.~Park.} 1993. 
The $M/G/1$ retrial queue with retrial rate control policy.
\textit{Probab. Eng. Inform. Sc.} 7(1):29--46.
doi: 10.1017/ S0269964800002771.
%6
\bibitem{BG92-1}
\Aue{Bertsekas, D., and R.~Gallager.} 2021.
\textit{Data networks}. Athena Scientific. 570~p.
%7
\bibitem{AY08-1}
\Aue{Avrachenkov, K., and U.~Yechiali.} 2008.
Retrial networks with finite buffers and their application to Internet data traffic. \textit{Probab. Eng. Inform. Sc.} 22(4):519--536.
doi: 10.1017/S0269964808000314.
%8
\bibitem{AY10-1} 
\Aue{Avrachenkov, K., and U.~Yechiali.} 2010.
On tandem blocking queues with a~common retrial queue. \textit{Comput. Oper. Res.} 37(7):1174--1180.
doi: 10.1016/j.cor.2009.10.004.

%9
\bibitem{Yaoetal02-1}
\Aue{Yao, S., F.~Xue, B.~Mukherjee, S.\,J.\,B.~Yoo, and S.~Dixit.} 2002.
Electrical ingress buffering and traffic aggregation for optical packet switching and their
effect on TCP-level performance in optical mesh networks.
\textit{IEEE Commun. Mag.} 40(9):66--72. doi: 10.1109/MCOM.2002.1031831.

%10
\bibitem{Wongetal09-1}
\Aue{Wong, E.\,W.\,M., L.\,L.\,H.~Andrew, T.~Cui, B.~Moran, A.~Zalesky, R.\,S.~Tucker, and M.~Zukerman.} 2009.
Towards a~bufferless optical internet.
\textit{J.~Lightwave Technol.} 27(14):2817--2833. doi: 10.1109/JLT.2009.2017211.

%11
\bibitem{mathematics2022-1}
\Aue{Morozov, E.\,V., I.\,V.~Peshkova, and A.\,S.~Rumyantsev.}
 2023. Bounds and maxima for the workload in a~multiclass orbit queue. \textit{Mathematics} 11(3):564. doi: 10.3390/ math11030564.

%12
\bibitem{pesh-mor2022-1} 
\Aue{Peshkova, I., and E.~Morozov.} 2022. On comparison of multiserver systems with multicomponent mixture distributions. 
\textit{J.~Math. Sci.} 267(2):260--272. doi: 10.1007/ s10958-022-06132-z.
%13
\bibitem{pesh2022-1}
\Aue{Peshkova, I.\,V.} 2022. Granitsy ekstremal'nogo in\-dek\-sa vre\-me\-ni ozhi\-da\-niya v~sis\-te\-me $M/G/1$ 
s~raspredeleniem vremeni obsluzhivaniya v~vide konechnoy
smesi [On bounds of the stationary waiting time extremal index in $M/G/1$
system with mixture service times]. \textit{Informatika i~ee Primeneniya~--- Inform. Appl.} 16(4):26--33. doi: 10.14357/19922264220405. EDN: VFKRKT.

%14
\bibitem{rego-1}
\Aue{Rego, V.} 1988. 
Some explicit formulas for mixed exponential service systems. 
\textit{Comput. Oper. Res.} 15(6):509--520. doi: 10.1016/0305-0548(88)90047-0.
%15
\bibitem{Morozov2019-1} 
\Aue{Morozov, E.\,V., A.\,S.~Rumyantsev, S.~Dey, and T.\,G.~Deepak.} 2019.
Performance analysis and stability of multiclass orbit queue with constant retrial rates and balking.
\textit{Perform. Evaluation} 134:102005. doi: 10.1016/ J.PEVA.2019.102005.
%16
\bibitem{Asmus-1}
\Aue{Asmussen, S.} 2003. \textit{Applied probability and queues. Stochastic modelling and 
applied probability.} New York, NY: Springer. 438~p.

%17
\bibitem{Ross-1}
\Aue{Ross, S., J.~Shanthikumar, and Z.~Zhu.}
 2005. On increasing-failure-rate random variables. \textit{J.~Appl. Probab.} 42(3):797--809. doi: 10.1239/jap/1127322028.
 
 %18
\bibitem{haji-1}
\Aue{Haji, R., and G.\,F.~Newell.} 1971. A~relation between stationary queue and waiting time distributions. \textit{J. Appl. Probab.} 8(3):617--620.
doi: 10.2307/3212186.

\end{thebibliography}

 }
 }

\end{multicols}

\vspace*{-6pt}

\hfill{\small\textit{Received August 26, 2023}} 

%\vspace*{-18pt}

\Contrl

\vspace*{-4pt}

\noindent
\textbf{Peshkova Irina V.} (b.\ 1975)~--- 
Candidate of Science (PhD) in physics and mathematics, associate professor, Petrozavodsk State University, 33~Lenina Pr., Petrozavodsk 185910, 
Russian Federation; senior scientist, Karelian Research Center of the Russian Academy of Sciences, 
11~Pushkinskaya Str., Petrozavodsk 185910, Russian Federation; \mbox{iaminova@petrsu.ru}


\label{end\stat}

\renewcommand{\bibname}{\protect\rm Литература} 