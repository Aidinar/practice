\def\stat{goncharov}

\def\tit{АННОТИРОВАНИЕ ПАРАЛЛЕЛЬНЫХ КОРПУСОВ:\\ ПОДХОДЫ И~НАПРАВЛЕНИЯ 
РАЗВИТИЯ$^*$}

\def\titkol{Аннотирование параллельных корпусов: подходы и~направления 
развития}

\def\aut{А.\,А.~Гончаров$^1$}

\def\autkol{А.\,А.~Гончаров}

\titel{\tit}{\aut}{\autkol}{\titkol}

\index{Гончаров А.\,А.}
\index{Goncharov A.\,A.}


{\renewcommand{\thefootnote}{\fnsymbol{footnote}} \footnotetext[1]
{Работа выполнено с~использованием инфраструктуры Центра коллективного пользования <<Высокопроизводительные вы\-чис\-ле\-ния и~большие данные>> 
(ЦКП <<Информатика>>) ФИЦ ИУ РАН (г.~Москва).}}


\renewcommand{\thefootnote}{\arabic{footnote}}
\footnotetext[1]{Федеральный исследовательский центр <<Информатика и~управление>> Российской академии наук, 
\mbox{a.gonch48@gmail.com}}

\vspace*{-12pt}


  
  \Abst{Представлены возможные направления развития инструментов для 
аннотирования параллельных корпусов с~учетом актуального положения дел в~этой сфере. 
Рассмотрены основные подходы к~проведению исследований на корпусном материале~--- 
(1)~кор\-пус\-но-ори\-ен\-ти\-ро\-ван\-ный, (2)~кор\-пус\-но-управ\-ля\-емый  
и~(3)~использующий корпус как источник иллюстративного материала~--- и~кратко описаны 
различия между ними. Показано, что, несмотря на обилие инструментов для 
аннотирования корпусов, подавляющее большинство из них предназначено для работы с~моноязычными корпусами и/или поддерживает очень узкий спектр функций по 
аннотированию текстовых данных. Наибольшее число функций предоставляют 
надкорпусные базы данных (НБД) и~веб-при\-ло\-же\-ния для доступа к~ним, разрабатываемые в~ФИЦ ИУ РАН: (1)~формирование блоков текста оригинала и~перевода, необходимых 
и~достаточных для анализа вхождения изучаемой языковой единицы и~варианта ее 
перевода; (2)~выявление вхождения изуча\-емой языковой единицы и~варианта ее перевода; 
(3)~выбор признаков, характеризующих упо\-треб\-ле\-ние изуча\-емой языковой единицы и~варианта ее перевода; (4)~выбор признаков, ха\-рак\-те\-ри\-зу\-ющих переводное соответствие. 
Такой спектр функций позволяет решать значительную часть исследовательских задач, 
однако он может быть расширен. Предлагаются три направления развития имеющегося 
функционала, способные обеспечить более детализированное описание языкового 
материала.}
  
  \KW{параллельный корпус; корпусная лингвистика; аннотирование корпуса; 
лингвистическое аннотирование}

  \DOI{10.14357/19922264230411}{GDKDOZ}
  
%\vspace*{-4pt}


\vskip 10pt plus 9pt minus 6pt

\thispagestyle{headings}

\begin{multicols}{2}

\label{st\stat}
  
\section{Введение}

На сегодняшний день обращение к~корпусам в~лингвистических 
исследованиях становится скорее стандартом, чем исключением: по частоте 
использования и~значимости эти информационные ресурсы встают в~один 
ряд со словарями. Еще в~2005~г.\ В.\,А.~Плунгян отмечал: <<Вполне 
возможно, что в~недалеком будущем без корпуса изучаемого языка 
лингвисту будет так же невозможно обходиться, как, например, без словаря 
этого языка. Более того, корпус, словарь и~грамматика, скорее всего, 
соединятся в~один электронный ресурс~--- или базу данных, на основании 
которой и~можно будет изучать язык>>~\cite[с.~14]{1-gon}. Действительно, 
во многом развитие идет именно в~этом направлении, а данные корпусов 
вновь и~вновь заставляют ставить под сомнение или корректировать 
утверждения лингвистов, сделанные без использования корпусного 
материала (см.\ примеры в~[2--4]).

Ввиду столь широкого распространения корпусных исследований перед 
специалистами по информатике и~компьютерной лингвистике встает вопрос 
о том, каким условиям должны удовлетворять инструменты, используемые 
при проведении этих исследований. Цель статьи состоит в~том, чтобы 
(с~учетом текущего положения вещей) обозначить некоторые возможные 
направления развития таких инструментов на примере работы 
с~параллельным корпусом.

\section{Подходы к~проведению корпусных исследований}

Прежде чем перейти к~рассмотрению собственно инструментов, стоит 
отметить, что исследования, выполняемые в~русле корпусной лингвистики,\linebreak 
различаются в~зависимости от подхода. В~подавляющем большинстве работ 
таких подходов выделяется два: (1)~кор\-пус\-но-ори\-ен\-ти\-ро\-ван\-ный 
и~(2)~кор\-пус\-но-управ\-ля\-емый~[5--8], к~которым иногда \mbox{добавляется} 
(3)~подход, использующий корпус как источник иллюстративного 
материала~[9]. Для русскоязычных обозначений этих подходов характерна 
вариативность. Так, в~работе~[9] подход~1 назван как <<основанный на 
корпусе>>, а~подход~2~--- <<направляемый корпусом>>; в~настоящей 
статье для их именования используются термины из~\cite[с.~14]{8-gon}. Что 
касается подхода~3, то он в~работе~[9] назван как <<использующий 
корпус>>. Это представляется не вполне удачным, так как любой из 
подходов к~работе с~корпусом по определению подразумевает его 
использование; в~настоящей статье за основу взято русскоязычное 
обозначение этого подхода из работы~\cite[с.~399]{10-gon}.

Границы перечисленных подходов размыты, а~некоторые исследователи 
и~вовсе ставят под сомнение необходимость подобного разделения. Например, в~работе~[11] с~говорящим названием <<Corpus-based and corpus-driven 
approaches to linguistic analysis: One and the same?>> ставится под сомнение 
необходимость противопоставления подходов к~проведению корпусных 
исследований в~пользу общего обозначения \textit{corpus approach}. 
В~работе~[12]\linebreak также утверждается, что четкое разграничение между 
подходами преувеличено. Автор работы~[13],\linebreak напротив, считает, что это 
разграничение преуменьшено, так как исследование, выполненное в~строгом 
соответствии с~кор\-пус\-но-управ\-ля\-емым подходом, по его мнению, не 
что иное, как миф: <<[the distinction between corpus-based and corpus-driven 
approaches] is understated given that \textit{truly} corpus-driven work seems 
a~myth at best>>.

Кратко укажем, в~чем состоят основные характеристики каждого из 
подходов.

\textbf{1.~Корпусно-ориентированный подход} (\textit{corpus-based}). 
Корпусные данные анализируются качественно и~количественно, а~исходной 
точкой исследования оказывается сформулированная \mbox{заранее} гипотеза и/или 
допущение (т.\,е.\ движение осуществляется от теории к~данным).

\textbf{2.~Корпусно-управляемый подход} (\textit{corpus-driven}). 
Корпусные данные анализируются качественно и~количественно, причем они 
сами же служат исходной точкой исследования. Теоретические положения 
заранее не формулируются или это делается в~минимальной степени (т.\,е.\ 
движение осуществляется от данных к~теории). Говоря об этом подходе, 
Дж.~Синклер заявляет: <<В~кор\-пус\-но-управ\-ля\-емом лингвистическом 
исследовании вы не используете заранее размеченный текст, а~работаете 
непосредственно с~текстом как он есть, и~только тогда вы можете увидеть, 
как устроен этот текст, не содержащий посторонних 
примесей>>~\cite[c.~191]{14-gon}\footnote{Перевод мой~--- А.\,Г. Оригинальная 
цитата: <<In corpus-driven linguistics you do not use pre-tagged text, but you process the raw text 
directly and then the patterns of this uncontaminated text are able to be observed>>.}.

\textbf{3.\ Подход, использующий корпус как источник иллюстративного 
материала} (\textit{corpus-informed/corpus-illustrated}). Корпус служит лишь 
для поиска примеров на естественном языке, количественный анализ данных 
не проводится.

Значительная часть публикуемых исследований, содержащих отсылки к~корпусу, относится именно к~третьему подходу. Безусловно, эти 
исследования гораздо менее субъективны по сравнению с~теми, где языковые 
примеры конструируются лингвистом. Однако~--- и~в~пользу этого 
свидетельствует в~том числе то, что третий подход даже не упоминается 
в~абсолютном большинстве работ по корпусной лингвистике,~--- 
исследования такого типа, вероятно, не могут считаться по-на\-сто\-яще\-му 
корпусными, так как корпус в~них выступает лишь\linebreak в~качестве удобного 
инструмента для быстрого поиска иллюстративного материала. В~рамках 
настоящей статьи такие исследования не рассматриваются, поскольку они не 
подразумевают \mbox{ка\-ко\-го-ли\-бо} аннотирования, а~задействуют лишь 
поисковый функционал корпусного менеджера.

\section{Инструменты для~аннотирования параллельных корпусов}

Если для проведения исследований, где корпус используется как источник 
иллюстративного материала, необходим лишь инструмент для поиска 
релевантных примеров в~текстах корпуса, то такие исследования, которые 
выполняются в~рамках кор\-пус\-но-ори\-ен\-ти\-ро\-ван\-но\-го или  
кор\-пус\-но-управ\-ля\-емо\-го подхода, требуют широкого применения 
информационных технологий. Так, чтобы коллектив исследователей 
(особенно если он распределенный) мог работать над изучением  
ка\-кой-ли\-бо категории языковых единиц на корпусном материале в~рамках 
продолжительного научного проекта, необходимо обеспечить поддержку 
многопользовательского режима работы с~данными. Этот режим работы 
может обеспечиваться при помощи веб-ин\-тер\-фейса.

Однако, если работа с~текстами проводится только в~режиме чтения, это не 
позволяет последовательно анализировать примеры упо\-треб\-ле\-ния 
исследуемых языковых единиц и~сохранять результаты анализа в~форме 
аннотированных переводных соответствий (в~случае моноязычного 
корпуса~--- аннотированных блоков текста). Традиционно лингвисты 
составляли картотеки примеров, на смену которым сегодня пришли 
электронные документы, создаваемые при помощи текстовых или таб\-лич\-ных 
процессоров, входящих в~со\-став офисных пакетов приложений. Однако 
подобное использование электронных документов, как правило, снова 
переводит работу с~данными в~однопользовательский режим, а~также 
за\-труд\-ня\-ет как сам процесс исследования, так и~получение 
непротиворечивых (\textit{consistent}) и~лег\-ко про\-ве\-ря\-емых результатов.

Ввиду этого на протяжении последних десятилетий непрерывно создаются 
новые программные продукты для создания корпусов и~последующей работы с~ними, в~том числе аннотирования. Так, на постоянно обновляемом  
ин\-тер\-нет-ре\-сур\-се Tools for Corpus Linguistics~[15] по состоянию 
на~10.10.2023 содержится информация о~277~инструментах для работы 
с~корпусами. Каждый инструмент описан при помощи набора меток, 
которые можно использовать для поиска инструментов,  
удовле\-тво\-ря\-ющих заданным критериям. Шестьдесят пять из них 
доступны через интернет (метка <<Web>>), но если добавить критерий 
поддержки параллельных корпусов (метка <<multilingual>>), то в~поисковой 
выдаче останутся лишь две записи: ACTRES Corpus Manager~[16] 
и~корпусный менеджер Sketch Engine~[17]. Впрочем, даже это не говорит 
о~том, что упомянутые инструменты предоставляют возможность 
аннотирования найденных в~них примеров в~соответствии с~целями 
исследования: автору настоящей статьи не удалось обнаружить подобного 
функционала (доступны лишь функции формирования параллельных 
корпусов и~последующего поиска по ним).

В этом отношении, по всей видимости, уникальным инструментом 
оказываются НБД и~веб-при\-ло\-же\-ния для 
работы с~ними, раз\-ра\-ба\-ты\-ва\-емые в~ФИЦ ИУ РАН с~2013~г.~[18, 19]. Помимо 
разнообразных функций поиска по параллельным текс\-там (см., например,~[20, 
21]) они дают возможность многоаспектного аннотирования переводных 
соответствий, содержащих вхождения ис\-сле\-ду\-емой языковой единицы. 
Именно это сочетание характеристик~--- под\-держ\-ка аннотирования 
примеров из параллельных текс\-тов с~использованием  
веб-при\-ло\-же\-ния~--- отличает НБД как от параллельных корпусов (в~них 
отсутствует функционал для аннотирования), так и~от других известных 
автору инструментов для аннотирования (они не поз\-во\-ля\-ют работать 
с~параллельными текс\-тами).

\section{Аннотирование параллельных корпусов с~использованием 
надкорпусных баз данных}

Функционал для аннотирования примеров употребления изучаемой языковой 
единицы с~использованием НБД предоставляет 
следующие возможности\footnote{Для упрощения изложения здесь и~далее исходим из 
того, что объект изучения~--- языковая единица, содержащаяся в~текстах оригинала. По этой 
причине при перечислении возможностей аннотирования говорится об изучаемой языковой 
единице и~варианте ее перевода. Объектом изучения может быть и~языковая единица, 
содержащаяся в~текстах перевода. В~таком случае та языковая единица, которая соответствует ей в~оригинале, может быть названа стимулом перевода.}.

\textbf{1.\ Формирование блоков текста оригинала и~перевода}, 
необходимых и~достаточных для анализа вхож\-де\-ния изучаемой языковой 
единицы и~варианта ее перевода. Одной пары фрагментов параллельного текста 
далеко не всегда достаточно для анализа вхож\-де\-ния изучаемой языковой 
единицы. Так, в~примере~1 содержимое пары не позволяет определить 
семантику изучаемой языковой единицы~--- формы \textit{soll} (букв.\ 
`должен'), а~также объяснить\linebreak причины, которые заставили переводчика 
использовать в~качестве варианта ее перевода форму \textit{хочешь}.

\begin{itemize}
\item[\,] \textit{Пример}~1. \textbf{Soll} ich? [H.~B$\ddot{\mbox{o}}$ll. Ansichten 
eines Clowns (1963)] \textbf{Хочешь}? [Глазами клоуна (пер.\ Л.~Черная, 1964)]
\end{itemize}
Поэтому при аннотировании блок текста должен быть расширен за счет 
содержимого предыдущей пары фрагментов (пример~2).

\begin{itemize}
\item[\,] \textit{Пример}~2. Ich k$\ddot{\mbox{o}}$nnte jetzt vor deinen Augen von 
hier bis zur T$\ddot{\mbox{u}}$r humpeln, \mbox{da{\!\ptb{\!\ss}}} du vor Schmerz und Mitleid 
aufschreien und sofort einen Arzt anrufen w$\ddot{\mbox{u}}$rdest, den besten Chirurgen der Welt, 
Fretzer. \textbf{Soll} ich? [H.~B$\ddot{\mbox{o}}$ll. Ansichten eines Clowns (1963)] 
Хочешь, я~проковыляю сейчас до двери так, что ты закричишь от боли и~жалости и~кинешься звонить врачу, самому лучшему в~мире хирургу, Фретцеру? 
\textbf{Хочешь}? [Глазами клоуна (пер.\ Л.~Черная, 1964)]
\end{itemize}

В случае если пара, содержащая вхождение изуча\-емой языковой единицы, 
напротив, включает слишком длинные предложения, то при формировании 
блоков текста оригинала и~перевода они могут быть сокращены. Так, 
в~примере~2 нет необходимости включать в~состав блоков текста всю 
предыдущую пару, поскольку в~таком случае объем блоков текста будет хотя и~достаточным для анализа вхождения языковой единицы и~варианта ее 
перевода, но не необходимым. Поэтому часть первого предложения в~оригинале и~в переводе может быть опущена (пример~3), что обозначено при 
помощи <<[$\ldots$]>>.

\begin{itemize}
\item[\,] \textit{Пример}~3. Ich k$\ddot{\mbox{o}}$nnte jetzt vor deinen Augen von hier 
bis zur T$\ddot{\mbox{u}}$r humpeln [$\ldots$]. \textbf{Soll} ich? 
[H.~B$\ddot{\mbox{o}}$ll. Ansichten eines Clowns (1963)] Хочешь, я~проковыляю 
сейчас до двери [$\ldots$]? \textbf{Хочешь}? [Глазами клоуна (пер.\ Л.~Черная, 
1964)]
\end{itemize}

\textbf{2.\ Выявление вхождения изучаемой языковой единицы 
и~варианта ее перевода}. Не всегда объектом изучения оказываются 
лексические единицы: это могут быть, например, глагольные формы. 
И~тогда особенно актуальным становится выявление тех слов, которые 
представляют собой вхождение изуча\-емой формы, их графическое 
выделение и~выбор подходящей метки. Так, в~примере~4 в~русском языке 
исследуемая языковая единица~--- \textit{видал} (глагол несовершенного 
вида в~форме прошедшего времени изъявительного наклонения 
действительного залога), которой во французском переводе соответствует 
неоднословная единица \textit{ai vu} (глагол в~форме прошедшего сложного 
времени изъявительного наклонения~--- \textit{pass$\acute{\mbox{e}}$ 
compos$\acute{\mbox{e}}$}~--- действительного залога). При аннотировании 
эти словоформы графически выделяются (в интерфейсе НБД для этого 
используется полужирный шрифт), после чего добавляются метки, 
указывающие на то, к~какой языковой единице они относятся: <<Past-IPF 
(ind, act)>> для русского и~<<Pass$\acute{\mbox{e}}$Comp (ind, act)>> для 
французского.

\begin{itemize}
\item[\,] \textit{Пример}~4.~--- Какое письмо? Я~никакого письма не \textbf{видал},~--- 
сказал Захар [И.~Гончаров. Обломов (1848--1859)].~--- Quelle lettre, je n'\textbf{ai} pas 
\textbf{vu} de lettre, dit Zakhar [Oblomov (пер.\ L.~Jurgenson, 1988)].
\end{itemize}

\textbf{3.\ Выбор признаков, характеризующих употребление изучаемой 
языковой единицы и~варианта ее перевода}. В~зависимости от класса 
изучаемых языковых единиц при аннотировании примеров из корпуса 
необходимо добавлять сведения о тех признаках употребления этих единиц, 
которые значимы для их анализа.

Так, при анализе вхождений глагольных форм необходимо фиксировать, 
в~частности, то, не сопровождается ли анализируемая форма отрицанием 
и~не употреблена ли она в~составе прямой речи. В~примере~4 это именно 
так (и~в~оригинале, и~в~переводе), поэтому аннотатор должен добавить 
метки <<Neg>> для отрицания и~<<DialRepl>> для прямой речи.

\textbf{4.\ Выбор признаков, характеризующих переводное соответствие}. 
В~то время как признаки, рассматриваемые в~п.~3, характеризуют либо 
оригинал, либо перевод, иногда возникает необходимость добавления меток, 
опи\-сы\-ва\-ющих переводное соответствие в~целом. Так, в~примере~5 
переводчик переформулировал исходный текст (не искажая его смысла): 
в~оригинале буквально сказано `Я~хотел бы поговорить с~госпожой Шнир'. 
Поскольку подобные изменения влияют на выбор варианта перевода, 
переводное соответствие следует снабдить меткой <<Paraphr>> для 
переформулирования.

\begin{itemize}
\item[\,]
\textit{Пример}~5. ``Ich m$\ddot{\mbox{o}}$chte Frau Schnier sprechen'', sagte ich. 
[H.~B$\ddot{\mbox{o}}$ll. Ansichten eines Clowns (1963)]~--- Позовите, 
пожалуйста, госпожу Шнир,~--- сказал~я. [Глазами клоуна (пер.\ Л.~Черная, 1964)]
\end{itemize}

Таким образом, НБД позволяет работать с~че\-тырь\-мя основными видами разметки 
примеров, найден\-ных в~параллельном корпусе.

\section{Направления развития функционала для~аннотирования 
параллельных корпусов}

Несмотря на возможности аннотирования, поддержку которых обеспечивают 
НБД и~веб-при\-ло\-же\-ние для работы с~ней, в~ходе выполнения научных 
проектов, предполагающих аннотирование параллельных текстов, стало 
ясно, что функционал существующих инструментов аннотирования 
необходимо расширить.

\textbf{1.\ Должна быть предусмотрена возможность} не только 
обрабатывать найденные в~корпусе релевантные примеры (фиксируя 
результаты обработки в~форме аннотированных переводных \mbox{соответствий}), 
но и~\textbf{снабжать специальными метками те примеры, которые 
нерелевантны поисковому запросу, но попали в~выдачу} (так называемый 
<<шум>> в~данных). Как правило, такие примеры могут попадать 
в~поисковую выдачу либо из-за неснятой омонимии, либо при выполнении 
запросов на двух языках, когда искомые единицы оригинала и~перевода 
содержатся в~одной и~той же паре, но не соответствуют друг другу.

Так, пример~6 иллюстрирует ситуацию, когда при осуществлении поиска 
глагола \textit{faire} `делать' во всех формах во фран\-цуз\-ско-рус\-ском 
корпусе в~поисковую выдачу попала пара, содержащая в~оригинале 
словоформу \textit{faits} `факты', омонимичную словоформе \textit{faits} 
`сделавшие'.

\begin{itemize}
\item[\,] \textit{Пример}~6. ``Les $\underline{\mbox{faits}}$ sont 
avou$\acute{\mbox{e}}$s?'' dit le juge [J.~Verne. Le tour du monde en quatre-vingts 
jours (1872)].~--- Признаете ли вы факт преступления?~--- спросил судья [Вокруг 
света за восемьдесят дней (пер.\ Н.~Габинский, 1939)].
\end{itemize}

Пример~7, найденный в~анг\-ло-рус\-ском корпусе по запросу <<\textit{but} 
в~оригинале, \textit{однако} в~переводе>>, иллюстрирует ситуацию, когда 
при поиске на двух языках в~выдачу попадает пара, которая, хотя и~содержит 
искомые единицы, не иллюстрирует нужное переводное соответствие: в~этом 
примере союз \textit{but} переведен при помощи \textit{но}, а~слову \textit{однако} 
в~текс\-те оригинала ничего не соответствует.
%осталось без перевода.

\begin{itemize}
\item[\,]
\textit{Пример}~7. ``I'm getting nervous,'' said Kemp. $\underline{\mbox{But}}$ it was 
five minutes before he went to the window again. ``It must have been a sparrow,'' he said 
[H.\,G.~Wells. The Invisible Man (1897)].~--- $\underline{\mbox{Однако}}$ нервы 
у~меня расходились,~--- проговорил он про себя, $\underline{\mbox{но}}$ добрых 
пять минут не решался подойти к~окну.~--- Воробей, должно быть,~--- сказал он. 
[Че\-ло\-век-не\-ви\-дим\-ка (пер.\ Д.~Вейс, 1935)].
\end{itemize}

Подобные примеры необходимо снабжать соответствующими метками, для 
того чтобы иметь возможность получить (1)~общее чис\-ло пар, найденных по 
некоторому запросу в~корпусе, (2)~чис\-ло пар, содержащих релевантные 
соответствия, (3)~чис\-ло пар, содержащих нерелевантные соответствия. Все 
это~--- неотъемлемая часть кор\-пус\-но-ори\-ен\-ти\-ро\-ван\-но\-го 
исследования.

\textbf{2.~Необходимо предоставить более широкие возможности 
разметки блоков текста}, содержащих вхож\-де\-ние изуча\-емой языковой 
единицы. Например, при аннотировании союза \textit{потому что} 
в~примере 8~может возникнуть по\-треб\-ность в~выделении границ 
фрагментов, связанных при помощи этого союза (выделены одинарным 
и~двойным подчеркиванием).

\begin{itemize}
\item[\,]
\textit{Пример}~8. $\underline{\mbox{Обломов\ всегда\ ходил\ дома\ без\ гал-}}$\linebreak 
$\underline{\mbox{стука\ и~без\ жилета}}$, $\underline{\underline{\mbox{\textbf{потому\ что}\ любил\ 
простор}}}$\linebreak  $\underline{\underline{\mbox{и~приволье}}}$ [И.~Гончаров. Обломов (1848--1859)]. 
$\underline{\mbox{Chez\ lui,\ Oblomov\ ne\ portait\ jamais\ ni}}$\linebreak $\underline{\mbox{cravate\ ni\ gilet,}}$ 
$\underline{\underline{\mbox{\textbf{car}\ il\ aimait\ la\ libert$\acute{\mbox{e}}$\ et\ 
l'espace}}}$ [Oblomov (пер.\ A.~Adamov, 1959)].
\end{itemize}


Еще более ценной такая разметка станет в~том случае, если в~переводе союз 
никак не переведен, но смысл предложения не искажен. Так, в~примере~9 
в~оригинале имеется союз \textit{car}, выражающий  
ло\-ги\-ко-се\-ман\-ти\-че\-ское отношение причины, а~в~переводе это 
отношение оказывается имплицитным.

\begin{itemize}
\item[\,]
\textit{Пример}~9. $\underline{\mbox{J'$\acute{\mbox{e}}$tais\ 
tr$\grave{\mbox{e}}$s\ soucieux}}$ $\underline{\underline{\mbox{\textbf{car}\ ma\ 
panne}}}$\linebreak 
$\underline{\underline{\mbox{commen{\hspace*{-1pt}\ptb{\!\c{c}}}ait\  de\  \ m'appara$\hat{\mbox{\ptb{\!\hspace*{-1pt}\i}}}$tre\ comme\
tr$\grave{\mbox{e}}$s\  grave,\ \ et}}}$\linebreak $\underline{\underline{\mbox{l'eau\ \ $\grave{\mbox{a}}$~boire\  \ qui\ 
s'$\acute{\mbox{e}}$puisait\ \ me\ \ faisait\ \ craindre\ \ le}}}$\linebreak
 $\underline{\underline{\mbox{pire}}}$ 
[A.~de Saint-Exup$\acute{\mbox{e}}$ry. Le petit prince (1942)]. $\underline{\mbox{Мне\ было\ не\ 
по\ себе,}}$ $\underline{\underline{\mbox{положение\ становилось}}}$\linebreak
 $\underline{\underline{\mbox{серьезным,\ \ 
воды\ \,почти\ не\ осталось,\ \ и\ я\ начал}}}$\linebreak
$\underline{\underline{\mbox{бояться,\ \,что\ моя\ вынужденная\ \,посадка\ \
плохо}}}$\linebreak $\underline{\underline{\mbox{кончится}}}$ [Маленький принц (пер.\ Н.~Галь, 1959)].
\end{itemize}

Разметка фрагментов текста может быть актуальна и~при изучении языковых 
единиц других классов (например, для демонстрации сферы действия отрицания, 
модальной лексики и~проч.).

\textbf{3.\ Необходимо дать возможность указывать, по какой причине 
был выбран тот или иной признак употребления изучаемой языковой 
единицы или варианта ее перевода}.

В разд.~4 при описании примера~4 было отмечено, что этот пример 
следует снабдить меткой, указывающей на наличие отрицания. Однако 
информация о том, по какой причине аннотатор решил добавить эту метку, 
никак не зафиксирована, хотя он, вероятно, принял такое решение из-за 
наличия каких-либо языковых единиц. Для большей полноты описания, 
а~также для его про\-ве\-ря\-емости следовало бы фиксировать при 
аннотировании и~эту информацию, в~явном виде указывая, что до\-бав\-ле\-ние 
метки <<Neg>> в~оригинале связано с~наличием \textit{не}, а~в~переводе~--- 
с~наличием \textit{n}$^\prime$ (усеченная форма слова \textit{ne}) 
и~\textit{pas}.

\section{Заключение}

В статье показано, что, несмотря на обилие инструментов для работы 
с~корпусами текстов, функционал подавляющего большинства этих 
инструментов весьма ограничен. С~этой точки зрения разработанные в~ФИЦ 
ИУ РАН надкорпусные базы данных и~веб-при\-ло\-же\-ния для работы 
с~ними, вероятно, не имеют отечественных и~зарубежных аналогов. По этой 
причине особенно важно анализировать опыт их использования в~сфере 
научных исследований и~образования как с~целью их дальнейшего 
совершенствования, так и~для разработки новых инструментов 
аннотирования параллельных корпусов.

{\small\frenchspacing
 {\baselineskip=10.6pt
 %\addcontentsline{toc}{section}{References}
 \begin{thebibliography}{99}
\bibitem{1-gon}
\Au{Плунгян В.\,А.} Зачем нужен Национальный корпус русского языка? Неформальное 
введение~// Национальный корпус русского языка: 2003--2005. Результаты и~перспективы.~--- М.: Индрик, 2005.  
С.~6--20. EDN: PXFYCP.
\bibitem{2-gon}
\Au{Перцов~Н.\,В.} О~роли корпусов в~лингвистических исследованиях~// Корпусная 
лингвистика: Труды Междунар. конф.~--- СПб: Изд-во С.-Петерб. ун-та, 2006. 
С.~318--331. EDN: RGQPTB.
\bibitem{3-gon}
\Au{Перцов Н.\,В.} К~суждениям о фактах русского языка в~свете корпусных данных~// 
Русский язык в~научном освещении, 2006. №\,1(11). С.~227--245. EDN: \mbox{PVNQUT}.
\bibitem{4-gon}
\Au{Плунгян В.\,А.} Корпус как инструмент и~как идеология: о~некоторых уроках 
современной корпусной лингвистики~// Русский язык в~научном освещении, 2008. 
№\,2(16). С.~7--20. EDN: MTBALV.
\bibitem{5-gon}
\Au{Tognini-Bonelli~E.} Corpus linguistics at work.~--- Amsterdam/Philadelphia: John 
Benjamins Publishing Co., 2001.  235~p.
\bibitem{6-gon}
\Au{Baker~P., Hardie~A., McEnery~T.} A~glossary of corpus linguistics.~--- Edinburgh: 
Edinburgh University Press, 2006. 187~p.
\bibitem{7-gon}
\Au{McEnery T., Hardie~A.} Corpus linguistics: Method, theory and practice.~---  Cambridge: 
Cambridge University Press, 2012. 310~p.
\bibitem{8-gon}
\Au{Захаров В.\,П., Богданова~С.\,Ю.} Корпусная лингвистика.~--- 3-е изд.~--- СПб: Изд-во С.-Петерб. ун-та, 2020. 234~с.
\bibitem{9-gon}
\Au{Копотев М.\,В.} Введение в~корпусную лингвистику.~--- Прага: Animedia 
Co., 2014. %Электр. изд.
\bibitem{10-gon}
\Au{Добровольский Д.\,О.} Корпусный подход к~исследованию фразеологии: новые 
результаты по данным параллельных корпусов~// Вестник Санкт-Пе\-тер\-бург\-ско\-го 
университета. Язык и~литература, 2020. Т.~17. №\,3. С.~398--411. doi: 
10.21638/spbu09.2020.303. EDN: QZIAAB.
\bibitem{11-gon}
\Au{Meyer~Ch.\,F.} Corpus-based and corpus-driven approaches to linguistic analysis: One and 
the same?~// Developments in English. Expanding electronic evidence.~--- Cambridge: 
Cambridge University Press, 2015. P.~14--28. doi: 10.1017/CBO9781139833882.004.
\bibitem{12-gon}
\Au{Xiao R.} Theory-driven corpus research: Using corpora to inform aspect theory~// Corpus 
linguistics: An international handbook~/ Eds. A.~L$\ddot{\mbox{u}}$deling, 
M.~Kyt$\ddot{\mbox{o}}$.~--- Berlin/New York: Walter de Gruyter, 2009.  Vol.~2. P.~987--1008. doi: 
10.1515/9783110213881.2.987.
\bibitem{13-gon}
\Au{Gries St.\,Th.} Corpus linguistics and theoretical linguistics. A~love--hate relationship? Not 
necessarily$\ldots$~// Int. J.~Corpus Linguis., 2010. Vol.~15. Iss.~3. P.~327--343. doi: 
10.1075/\mbox{IJCL}.15.3.02GRI.
\bibitem{14-gon}
\Au{Sinclair~J.} Trust the text: Language, corpus and discourse.~--- London/New York: 
Routledge, 2004. 224~p.
\bibitem{15-gon}
Tools for corpus linguistics. {\sf https://corpus-analysis.com}.
\bibitem{16-gon}
ACTRES corpus manager. {\sf https://actres.unileon.\linebreak es/ACM2.0/home}.
\bibitem{17-gon}
Sketch engine. {\sf https://www.sketchengine.eu}.
\bibitem{18-gon}
\Au{Зацман~И., Кружков~М., Лощилова~Е.} Методы и~средства информатики для 
описания структуры неоднословных коннекторов~/ Под ред.\ О.\,Ю.~Иньковой~// 
Структура коннекторов и~методы ее \mbox{описания}.~--- М.: ТОРУС ПРЕСС, 2019. С.~205--230. 
doi: 10.30826/SEMANTICS19-06. EDN: YVAJWN.

\bibitem{19-gon}
  \Au{Кружков М.\,Г.} Концепция построения надкорпусных баз данных~// Системы 
и~средства информатики, 2021. Т.~31. №\,3. С.~101--112. doi: 10.14357/08696527210309. 
EDN: UMWNIU.
\bibitem{20-gon}
\Au{Гончаров~А.\,А., Инькова~О.\,Ю., Кружков~М.\,Г.} Методология аннотирования 
в~надкорпусных базах данных~// Системы и~средства информатики, 2019. Т.~29. №\,2. 
С.~148--160. doi: 10.14357/08696527190213. EDN: GNDCJE.


\bibitem{21-gon}
\Au{Гончаров А.\,А.} Поиск с~исключением в~параллельных текстах~// Системы и~средства 
информатики, 2023. Т.~33. №\,4. С.~102--114.
doi: 10.14357\!/\!08696527230410.  EDN: CVPFDV.
\end{thebibliography}

 }
 }

\end{multicols}

\vspace*{-10pt}

\hfill{\small\textit{Поступила в~редакцию 15.10.23}}

\vspace*{6pt}

%\pagebreak

%\newpage

%\vspace*{-28pt}

\hrule

\vspace*{2pt}

\hrule



\def\tit{PARALLEL CORPUS ANNOTATION: APPROACHES AND~DIRECTIONS FOR~DEVELOPMENT}


\def\titkol{Parallel corpus annotation: Approaches and~directions for development}


\def\aut{A.\,A.~Goncharov}

\def\autkol{A.\,A.~Goncharov}

\titel{\tit}{\aut}{\autkol}{\titkol}

\vspace*{-12pt}


\noindent
Federal Research Center ``Computer Science and Control'' of the Russian Academy 
of Sciences, 44-2~Vavilov Str., Moscow 119333, Russian Federation


\def\leftfootline{\small{\textbf{\thepage}
\hfill INFORMATIKA I EE PRIMENENIYA~--- INFORMATICS AND
APPLICATIONS\ \ \ 2023\ \ \ volume~17\ \ \ issue\ 4}
}%
 \def\rightfootline{\small{INFORMATIKA I EE PRIMENENIYA~---
INFORMATICS AND APPLICATIONS\ \ \ 2023\ \ \ volume~17\ \ \ issue\ 4
\hfill \textbf{\thepage}}}

\vspace*{3pt}


\Abste{Possible directions for the development of parallel corpus annotation tools are presented 
 considering the actual situation in this area. The main approaches to conducting research on corpus material~--- ($i$)~corpus-based; 
 ($ii$)~corpus-driven; and
 ($iii$)~corpus-illustrated~--- are considered and the differences between them are briefly described. It is demonstrated that despite the abundance of corpus annotation tools, 
 the vast majority of them  are designed to deal with monolingual corpora and/or support a~very limited functionality for annotating textual data. 
 The largest number of functions are provided by supracorpora databases and web applications to access them which are being
 developed at FRC CSC RAS: 
 ($i$)~forming of original and translated text blocks necessary and sufficient for analyzing the occurrence of the studied language unit and its 
 translation variant; ($ii$)~identification of the occurrence of the studied language unit and its translation variant; 
 ($iii$)~selection of features characterizing the use of the studied language unit and its translation variant; 
 and ($i\nu$)~selection of features characterizing the translation correspondence. This set of functions provides solutions to a~significant part of research problems 
 but it can be extended. Three directions for the development 
of the existing functionality  are suggested which can provide a~more detailed description of linguistic material.}

\KWE{parallel corpus; corpus linguistics; corpus annotation; linguistic annotation}



  \DOI{10.14357/19922264230411}{GDKDOZ}

\vspace*{-16pt}

\Ack

\vspace*{-4pt}

\noindent
The research was carried out using the infrastructure of the Shared Research Facilities ``High 
Performance Computing and Big Data'' (CKP ``Informatics'') of FRC CSC RAS (Moscow).
  

%\vspace*{-5pt}

  \begin{multicols}{2}

\renewcommand{\bibname}{\protect\rmfamily References}
%\renewcommand{\bibname}{\large\protect\rm References}

{\small\frenchspacing
 {%\baselineskip=10.8pt
 \addcontentsline{toc}{section}{References}
 \begin{thebibliography}{99} 
\bibitem{1-gon-1}
\Aue{Plungyan, V.\,A.} 2005. Zachem nuzhen Natsional'nyy kor\-pus russkogo yazyka? 
Neformal'noe vvedenie [What the Russian National Corpus is for? Informal introduction]. 
\textit{Natsional'nyy korpus russkogo yazyka: 2003--2005. Rezul'taty i~perspektivy} [Russian 
National Corpus: 2003--2005. Results and prospects].  Moscow: Indrik. 6--20. EDN: PXFYCP.
\bibitem{2-gon-1}
\Aue{Pertsov, N.\,V.} 2006. O~roli korpusov v~lingvisticheskikh issledovaniyakh [On the role 
of corpora in linguistic research]. \textit{Scientific Conference (International) ``Corpus Linguistics'' Proceedings}. St.\ 
Petersburg: St. Petersburg University Press. 318--331. EDN: RGQPTB.
\bibitem{3-gon-1}
\Aue{Pertsov, N.\,V.} 2006. K~suzhdeniyam o~faktakh russkogo yazyka v~svete korpusnykh 
dannykh [Toward judgments about Russian language facts in the light of corpus data]. 
\textit{Russkiy yazyk v~nauchnom osveshchenii} [Russian Language and Linguistic Theory] 
1(11):227--245. EDN: PVNQUT.
\bibitem{4-gon-1}
\Aue{Plungyan, V.\,A.} 2008. Korpus kak instrument i kak ideologiya: o~nekotorykh urokakh 
sovremennoy korpusnoy lingvistiki [The corpus as tool and as ideology: On some lessons from 
modern corpus linguistics]. \textit{Russkiy yazyk v~nauchnom osveshchenii} [Russian Language 
and Linguistic Theory] 2(16):7--20. EDN: MTBALV.
\bibitem{5-gon-1}
\Aue{Tognini-Bonelli, E.} 2001. \textit{Corpus linguistics at work}. Amsterdam/Philadelphia: 
John Benjamins Publishing Co. 235~p.
\bibitem{6-gon-1}
\Aue{Baker,~P., A.~Hardie, and T.~McEnery.} 2006. \textit{A~glossary of corpus linguistics}. 
Edinburgh: Edinburgh University Press. 187~p.
\bibitem{7-gon-1}
\Aue{McEnery, T., and A.~Hardie.} 2012. \textit{Corpus linguistics: Method, theory and 
practice}. Cambridge: Cambridge University Press. 310~p.
\bibitem{8-gon-1}
\Aue{Zakharov, V.\,P., and S.\,Yu.~Bogdanova.} 2020. \textit{Korpusnaya lingvistika} [Corpus linguistics].
3rd ed.\ St.\ Petersburg: 
St. Petersburg University Press. 234~p.
\bibitem{9-gon-1}
\Aue{Kopotev, M.\,V.} 2014. \textit{Vvedenie v~korpusnuyu lingvistiku} [Introduction to 
corpus linguistics]. Prague: Animedia Co. 
\bibitem{10-gon-1}
\Aue{Dobrovol'skiy, D.\,O.} 2020. Korpusnyy podkhod k~issledovaniyu frazeologii: novye 
rezul'taty po dannym parallel'nykh korpusov [Corpus-based approach to phraseology research: 
New evidence from parallel corpora].\linebreak \textit{Vestnik Sankt-Peterburgskogo universiteta. Yazyk 
i~literatura} [Vestnik of Saint Petersburg University. Language and Literature] 17(3):398--411. 
doi: 10.21638/spbu09.2020.303. EDN: QZIAAB.
\bibitem{11-gon-1}
\Aue{Meyer, Ch.\,F.} 2015. Corpus-based and corpus-driven approaches to linguistic analysis: 
One and the same? \textit{Developments in English. Expanding electronic evidence}. 
Cambridge: Cambridge University Press. 14--28. doi: 10.1017/CBO9781139833882.004.
\bibitem{12-gon-1}
\Aue{Xiao, R.} 2009. Theory-driven corpus research: Using corpora to inform aspect theory. 
\textit{Corpus linguistics: An international handbook}.  Eds. 
A.~L$\ddot{\mbox{u}}$deling and M.~Kyt$\ddot{\mbox{o}}$. Berlin/New York: Walter de 
Gruyter. 2:987--1008. doi: 10.1515/9783110213881.2.987.
\bibitem{13-gon-1}
\Aue{Gries, St.\,Th.} 2010. Corpus linguistics and theoretical linguistics. A~love--hate 
relationship? Not necessarily$\ldots$ \textit{Int. J. Corpus Linguist.} 15(3):327--343. doi: 
10.1075/\linebreak IJCL.15.3.02GRI.
\bibitem{14-gon-1}
\Aue{Sinclair, J.} 2004. \textit{Trust the text: Language, corpus and discourse}. London/New 
York: Routledge. 224~p.
\bibitem{15-gon-1}
Tools for corpus linguistics. Available at: {\sf https://corpus-analysis.com} (accessed 
November~27, 2023).
\bibitem{16-gon-1}
ACTRES corpus manager. Available at: {\sf https://actres.\linebreak unileon.es/ACM2.0/home} (accessed 
November~27, 2023).
\bibitem{17-gon-1}
Sketch engine. Available at: {\sf https://www.sketchengine.\linebreak eu} (accessed November~27, 2023).
\bibitem{18-gon-1}
\Aue{Zatsman, I., M.~Kruzhkov, and E.~Loshchilova.} 2019. Metody i~sredstva informatiki 
dlya opisaniya struktury neodnoslovnykh konnektorov [Methods and means of informatics for 
multiword connectives structure description]. \textit{Struktura konnektorov i~metody ee 
opisaniya} [Con-nectives structure and methods of its description]. Ed. O.~Yu. Inkova. Moscow: 
TORUS PRESS. 205--230. doi: 10.30826/SEMANTICS19-06. EDN: YVAJWN.
\bibitem{19-gon-1}
\Aue{Kruzhkov, M.} 2021. Kontseptsiya postroeniya nadkorpusnykh baz dannykh [Conceptual 
framework for supracorpora databases]. \textit{Sistemy i~Sredstva Informatiki~--- Systems and 
Means of Informatics} 31(3):101--112. doi: 10.14357/08696527210309. EDN: UMWNIU.
\bibitem{20-gon-1}
\Aue{Goncharov, A.\,A., O.\,Yu.~Inkova, and M.~Kruzhkov.} 2019. Metodologiya 
annotirovaniya v~nadkorpusnykh ba\-zakh dan\-nykh [Annotation methodology of supracorpora 
databases]. \textit{Sistemy i~Sredstva Informatiki~--- Systems and Means of Informatics} 
29(2):148--160. doi: 10.14357/ 08696527190213. EDN: GNDCJE.
\bibitem{21-gon-1}
\Aue{Goncharov, A.\,A.} 2023. Poisk s~isklyucheniem v~parallel'nykh teks\-takh [Search with 
exclusion in parallel texts]. \textit{Sistemy i~Sredstva Informatiki~--- Systems and Means of 
Informatics} 33(4):102--114. doi: 10.14357\!/\!08696527230410.  EDN: CVPFDV.

\end{thebibliography}

 }
 }

\end{multicols}

\vspace*{-8pt}

\hfill{\small\textit{Received October 15, 2023}} 

\vspace*{-18pt}

\Contrl

\vspace*{-4pt}

\noindent
\textbf{Goncharov Alexander A.} (b.\ 1994)~--- scientist, Federal Research Center ``Computer 
Science and Control'' of the Russian Academy of Sciences, 44-2~Vavilov Str., Moscow 119333, 
Russian Federation; \mbox{a.gonch48@gmail.com}

\label{end\stat}

\renewcommand{\bibname}{\protect\rm Литература} 