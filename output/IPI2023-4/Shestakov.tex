\def\stat{shestakov}

\def\tit{НЕЛИНЕЙНАЯ РЕГУЛЯРИЗАЦИЯ ОБРАЩЕНИЯ ЛИНЕЙНЫХ
ОДНОРОДНЫХ ОПЕРАТОРОВ С~ПОМОЩЬЮ МЕТОДА\\ БЛОЧНОЙ
ПОРОГОВОЙ ОБРАБОТКИ$^*$}

\def\titkol{Нелинейная регуляризация обращения линейных
однородных операторов с~помощью метода %блочной
пороговой обработки}

\def\aut{О.\,В.~Шестаков$^1$, Е.\,П.~Степанов$^2$}

\def\autkol{О.\,В.~Шестаков, Е.\,П.~Степанов}

\titel{\tit}{\aut}{\autkol}{\titkol}

\index{Шестаков О.\,В.}
\index{Степанов Е.\,П.}
\index{Shestakov O.\,V.}
\index{Stepanov E.\,P.}


{\renewcommand{\thefootnote}{\fnsymbol{footnote}} \footnotetext[1]
{Работа выполнена при поддержке программы 
развития Московского университета, проект №\,23-Ш03-03.}}


\renewcommand{\thefootnote}{\arabic{footnote}}
\footnotetext[1]{Московский государственный университет 
им.~М.~В.~Ломоносова, факультет вычислительной математики и~кибернетики; 
Мос\-ков\-ский центр фундаментальной и~прикладной математики; Федеральный 
исследовательский центр <<Информатика и~управление>> Российской академии наук, 
\mbox{oshestakov@cs.msu.ru}}
\footnotetext[2]{Московский государственный университет 
им.~М.~В.~Ломоносова, факультет вычислительной математики и~кибернетики, 
\mbox{estepanov@lvk.cs.msu.ru}}


\vspace*{-12pt}



\Abst{Методы пороговой обработки коэффициентов вейв\-лет-раз\-ло\-же\-ний 
стали популярным инструментом для регуляризации обратных статистических задач 
благодаря своей простоте, вычислительной эффективности и~возможности адаптации 
как к~виду обращаемых операторов, так и~к особенностям исследуемой функции. 
Наиболее плодотворным данный подход оказался при обращении линейных однородных 
операторов, возникающих в~некоторых задачах обработки сигналов и~изображений. 
В~работе рассматривается метод блочной пороговой обработки, в~котором коэффициенты 
разложения обрабатываются группами, что позволяет учитывать информацию 
о~соседних коэффициентах. В~модели данных с~аддитивным гауссовским шумом 
проводится анализ несмещенной оценки среднеквадратичного риска и~показывается, 
что при определенных условиях данная оценка становится сильно состоятельной 
и~асимптотически нормальной. Данные свойства позволяют строить асимптотические 
доверительные интервалы для теоретического среднеквадратичного риска 
рассматриваемого метода.}


\KW{линейный однородный оператор; вейвлеты; блочная 
пороговая обработка; несмещенная оценка риска; асимптотическая нормальность; 
сильная состоятельность}

\DOI{10.14357/19922264230401}{PGKKYE}
  
\vspace*{-6pt}


\vskip 10pt plus 9pt minus 6pt

\thispagestyle{headings}

\begin{multicols}{2}

\label{st\stat}



\section{Введение}

При решении обратных статистических задач возникает проблема обращения 
некоторого оператора, и~если в~наблюдаемых данных содержится шум, то необходимо 
применять методы регуляризации.
Примером может служить восстановление сигнала в~беспроводных каналах, 
зашумленного за счет гео- и~гелиомагнитных условий, состояния ионосферы Земли.
Во многих ситуациях рассматриваемый оператор оказывается линейным и~однородным. 
В таких случаях часто применяются методы подавления шума с~помощью вейв\-лет-раз\-ло\-же\-ния и~процедур пороговой обработки. Они приобрели свою популярность 
благодаря адаптивности, вычислительной эффективности и~асимптотической 
оптимальности. 

Чаще всего процедуры пороговой обработки применяются отдельно 
к~каждому коэффициенту вейв\-лет-раз\-ло\-же\-ния. Коэффициент сравнивается с~пороговым 
значением, и~если его абсолютная величина оказывается меньше этого значения, то 
он обнуляется. Среди таких процедур наиболее распространены процедуры жест\-кой 
и~мягкой пороговой обработки. Однако они имеют свои недостатки и~часто не 
достигают оптимальных результатов. В~работе~\cite{HKP99} рассмотрен метод 
блочной пороговой обработки, при котором коэффициенты обрабатываются не 
отдельно, а~группами. Цель такого подхода заключается в~использовании информации 
о~соседних коэффициентах. По\-лу\-ча\-емые оценки имеют оптимальный (в~минимаксном 
смысле) порядок среднеквадратичного риска для различных классов функций~\cite{Cai99}.

Для практического анализа погрешности данного метода можно использовать 
несмещенную оценку среднеквадратичного риска~\cite{St81}, которая дает 
возможность оценивать качество обработанного сигнала на основе только 
наблюдаемых данных. В~данной работе исследуются статистические свойства этой 
оценки и~показывается, что для довольно широкого класса пространств Бесова она 
является асимптотически нормальной и~сильно состоятельной. В~работе~\cite{SH21} 
подобные исследования проведены для задачи непараметрического оценивания функции 
сигнала. 

Для методов пороговой обработки отдельных коэффициентов свойства оценки 
среднеквадратичного риска при обращении линейных однородных операторов 
исследованы в~работах~\cite{KS11, SH12, SH16, SH18} в~предположении 
о~принадлежности исследуемой функции сигнала классу равномерно регулярных по 
Липшицу функций.

\vspace*{-6pt}

\section{Обращение линейных однородных операторов с~помощью вейг\-лет-вейв\-лет-разложения}

\vspace*{-3pt}

Линейный оператор $K$ называется однородным, если
$$
K[f(a(x-x_0))]=a^{-\alpha}(Kf)[a(x-x_0)]
$$
для любого $x_0\hm\in\mathbb{R}$ и~любого $a\hm>0$. Параметр~$\alpha$ называется 
показателем однородности. Примерами\linebreak однородных линейных операторов служат 
операторы дифференцирования и~интегрирования, преобразование Гильберта, 
преобразование Абеля и~некоторые операторы свертки. Во многих \mbox{прикладных} задачах 
наблюдаются значения функции~$Kf$, а~интерес представляет функция~$f$, т.\,е.\ 
необходимо решать задачу обращения оператора~$K$. Для этого воспользуемся 
методом вейг\-лет-вейв\-лет-раз\-ло\-же\-ния.

Вейвлет-разложение~$Kf$ представляет собой ряд
\begin{equation}
\label{wavelet_decomp}
Kf = \sum\limits_{j,k \in Z} \langle Kf,\psi_{j,k} \rangle \psi_{j,k},
\end{equation}
где $\psi$~--- некоторая материнская вейв\-лет-функ\-ция, а~$\psi_{j,k}(t) \hm= 
2^{j/2}\psi(2^jt - k)$ (семейство $\{\psi_{j,k}\}_{j,k \in Z}$ образует 
ортонормированный базис в~$L^2(\mathbb{R})$). Индекс~$j$ в~(\ref{wavelet_decomp}) называется масштабом, а~индекс~$k$~--- сдвигом.

Поскольку оператор $K$ линеен и~однороден, существуют такие функции~$u_{jk}$, 
что $\langle f,u_{jk}\rangle\hm=\langle Kf,\psi_{jk}\rangle$. При этом функция~$f$ 
представляется в~виде ряда
\begin{equation}
\label{VWD}
f = \sum_{j,k \in Z}\beta_{j}\langle Kf,\psi_{j,k}\rangle u_{j,k},
\end{equation}
где $u_{j,k} = K^{-1}\psi_{j,k} / \beta_{jk}$, $\beta_{j}\hm=2^{\alpha j}\beta_{0}$, а~$\beta_{0} \hm= \norm{K^{-1}\psi}$ 
(функции~$u_{j,k}$  называются 
вейглетами). Формула~(\ref{VWD}) и~есть основа метода вейг\-лет-вейв\-лет-раз\-ло\-же\-ния~\cite{AS98}. 
При соответствующем выборе вейв\-лет-функ\-ции~$\psi$ 
последовательность~$\{u_{jk}\}$ образует устойчивый базис~\cite{Lee97}. Также, 
если~$\psi$ имеет~$r$~непрерывных производных и~$r$ нулевых моментов, определим 
при $0\hm<\gamma\hm<r$ и~$1\hm\leqslant p,q\hm\leqslant\infty$ полунорму последовательности вейв\-лет-ко\-эф\-фи\-ци\-ен\-тов выражением

\noindent
$$
\abs{Kf}_{{B}^{\gamma}_{p,q}}=\!\left(\sum\limits_{j=0}^{\infty}\left(2^{sj
}\left(\sum\limits_{k}\abs{\langle 
Kf,\psi_{j,k}\rangle}^p\right)^{\!1/p}\right)^q\right)^{\!1/q}\!,
$$
где $s=\gamma+1/2-1/p$. Далее будем считать, что~$\psi$ удовле\-тво\-ря\-ет всем 
необходимым требованиям, а~$Kf$ задана на конечном отрезке и~принадлежит 
пространству Бесова ${B}^{\gamma}_{p,q}(A)$ ($A\hm>0$), т.\,е.\ 
$\abs{Kf}_{{B}^{\gamma}_{p,q}}\hm\leqslant A$.

\vspace*{-6pt}

\section{Модель данных}

\vspace*{-3pt}

При практических измерениях значения функции~$Kf$ регистрируются в~дискретных 
отсчетах. К~тому же эти значения, как правило, содержат шум. В~данной работе 
рассматривается следующая модель данных:
\begin{equation*}
%\label{Data_Model}
X_i = (Kf)_i + \epsilon_i, \enskip i =\overline{1, 2^J}\,,
\end{equation*}
где $2^J$~--- число отсчетов функции сигнала; $(Kf)_i$~--- незашумленные значения 
функции~$Kf$; $\epsilon_i$~--- независимые нормально распределенные случайные 
величины с~нулевым средним и~дисперсией~$\sigma^2$.
После применения дискретного вейв\-лет-пре\-обра\-зо\-ва\-ния получается следующая модель 
зашумленных вейв\-лет-ко\-эф\-фи\-ци\-ен\-тов:
\begin{equation*}
Y_{j,k}=\mu_{j,k}+\epsilon^W_{j,k},\enskip j=\overline{0,J-1},\enskip k=\overline{0,2^{j}-1}\,,
\end{equation*}
где $\epsilon^W_{j,k}$ независимы и~распределены так же, как и~$\epsilon_i$, 
а~$\mu_{j,k}\hm= 2^{J/2}\langle Kf,\psi_{j,k}\rangle$~\cite{Mall99}.

\vspace*{-6pt}


\section{Блочная пороговая обработка}

\vspace*{-3pt}

Для подавления шума коэффициенты~$Y_{j,k}$ подвергаются некоторой обработке. 
Наибольшее распространение получили методы жесткой и~мягкой пороговой обработки 
и~их модификации~\cite{DonJ94, DonJ95, DonJ98, G98, PK05, LC10, HL10, HX15, ZC15}. При использовании этих методов происходит сравнение абсолютной величины 
каждого коэффициента с~некоторым порогом, и~если это значение\linebreak оказывается меньше 
порога, то коэффициент считается шумом и~обнуляется. Такие методы обрабатывают 
каждый коэффициент отдельно, не используя информацию о~других коэффициентах. 
\mbox{Блочная} пороговая обработка применяется к~группам соседних коэффициентов, т.\,е.\ 
решение об обнулении принимается сразу ко всем коэффициентам из группы.

Пусть $B_{j,1},\ldots,B_{j,M_j}$~--- разбиение множества индексов 
$\{0,\ldots,2^j-1\}$ на блоки одинаковой длины~$L$ (для удобства предположим, 
что~$2^j$ делится на~$L$). Пусть $S^2_{j,m}\hm=\sum\nolimits_{k\in 
B_{j,m}}Y^2_{j,k}$. Оценки коэффициентов~$\mu_{j,k}$ вычисляются по правилу:
%\noindent
\begin{equation*}
\widehat{\mu}_{j,k}=\left(1-\fr{TL\sigma^2}{S^2_{j,m}}\right)_+Y_{j,k},\enskip
j=\overline{0,J-1},\ \ k\in B_m,
\end{equation*}

\noindent
т.\,е.\ если величина $\sum\nolimits_{k\in B_{j,m}}Y^2_{j,k}$ меньше порога 
$TL\sigma^2$, то все коэффициенты в~рассматриваемом блоке обнуляются.

На качество оценок, получаемых с~помощью блочной пороговой обработки, 
естественно, влияют размер блока $L$ и~значение порога~$T$. В~работе~\cite{Cai99} показано, что при $L\hm=\log 2^j$ достигается баланс между локальной 
и~глобальной адап\-тив\-ностью метода блочной пороговой обработки, и~если при этом 
$T^*\hm\approx4{,}50524$ ($T^*$~--- это корень уравнения $T\hm-\log  T \hm-3\hm=0$), то 
сред\-не\-квад\-ра\-тич\-ный риск оказывается (почти) оптимальным (в~минимаксном смыс\-ле). 
В~данной работе рас\-смат\-ри\-ва\-ют\-ся именно такие значения~$L$ и~$T$.

Одним из самых распространенных критериев качества описанного метода служит 
среднеквадратичный риск, который определяется по формуле:
\begin{equation*}
%\label{Risk}
R_J(T)=\sum\limits_{j=0}^{J-1}\sum\limits_{k=0}^{2^j-1}\beta^2_{j}{\sf E}\left(\widehat{\mu}_{j,k}-\mu_{j,k}\right)^2.
\end{equation*}
Вычислить его значение на практике нельзя, поскольку~$R_J(T)$ зависит от 
ненаблюдаемых <<чистых>> коэффициентов~$\mu_{j,k}$. Однако его можно оценить, 
используя наблюдаемые значения~$Y_{j,k}$. Далее будут исследованы свойства такой 
оценки.

\vspace*{-6pt}


\section{Асимптотические свойства несмещенной оценки риска}

\vspace*{-3pt}

В~\cite{Cai99} показано, что
\begin{equation*}
\sum\limits_{k=0}^{2^j-1}{\sf E}\left(\widehat{\mu}_{j,k}-
\mu_{j,k}\right)^2=\sum\limits_{m=1}^{M_j}{\sf E} F_m[T,L], %\notag
\end{equation*}
где
\begin{multline*}
F_m(T,L)=\left[
\vphantom{\fr{T^2L^2\sigma^4-2TL\sigma^4(L-2)}{S_{j,m}^2}}
L\sigma^2+{}\right.\\
{}+\fr{T^2L^2\sigma^4-2TL\sigma^4(L-2)}{S_{j,m}^2}
\cdot \Ik\left(S^2_{j,m}>TL\sigma^2\right)+{}\\
\left.{}+\left(S^2_{j,m}-2L\sigma^2\right)\cdot \Ik\left(S^2_{j,m}\leqslant 
TL\sigma^2\right)
\vphantom{\fr{T^2L^2\sigma^4-2TL\sigma^4(L-2)}{S_{j,m}^2}}
\right].
\end{multline*}
Таким образом, величина
\begin{equation}
\label{Risk_Estimate}
\widehat{R}_J(T)=\sum\limits_{j=0}^{J-1}\beta^2_{j}\sum\limits_{m=1}^{M_j} F_m(T,L)
\end{equation}
служит несмещенной оценкой~$R_J(T)$, не зависящей от~$\mu_{j,k}$.

Покажем, что оценка~(\ref{Risk_Estimate}) асимптотически нормальна.

\columnbreak

\noindent
\textbf{Теорема~1.}\ \textit{Пусть $K$~--- линейный однородный оператор с~показателем 
однородности $\alpha\hm>0$, а~$Kf\hm\in B^{\gamma}_{p,q}(A)$ и~задана на 
конечном отрезке. Если $1\hm\leqslant p,q\hm\leqslant\infty$ и~$r\hm>\gamma\hm>\max(0,1/p-1/2)$, то 
при} $J\hm\to\infty$
\begin{equation}
\label{Normality}
{\sf P}\left(\fr{\widehat{R}_J(T^*)-R_J(T^*)}{D_J}<x\right)\to\Phi(x),
\end{equation}
\textit{где}
$$
D^2_J=\fr{2\sigma^4\beta_{0}^4}{2^{4\alpha+1}-1}\cdot 2^{(4\alpha+1)J}\,,
$$
\textit{а $\Phi(x)$~--- функция распределения стандартного нормального закона}.

\smallskip


\noindent
Д\,о\,к\,а\,з\,а\,т\,е\,л\,ь\,с\,т\,в\,о\,: \ 
Выберем $0\hm<d\hm<1$ и~запишем:
\begin{multline}
\label{Two_Sums}
\widehat{R}_J(T^*)-R_J(T^*)={}\\
{}=\sum\limits_{j=0}^{[dJ]}\beta^2_{j}\sum\limits_{m=1}^{M_j} 
\left[F_m(T^*,L)-{\sf E} F_m(T^*,L)\right]+{}\\
{}+\sum\limits_{j=[dJ]+1}^{J-1}\beta^2_{j}\sum\limits_{m=1}^{M_j} 
\left[F_m(T^*,L)-{\sf E} F_m(T^*,L)\right].
\end{multline}
 Число слагаемых в~первой сумме не превосходит $2^{[dJ]\hm+1}$. Кроме того, 
существует такая константа $C_F\hm>0$, что
\begin{equation}
\label{Bounds}
\abs{F_m(T^*,L)-{\sf E} F_m(T^*,L)}\leqslant C_FT^*L\;\;\mbox{п.~в.} % см. формулу 9.3 в~статье Cai
\end{equation}

Применяя неравенство Хеффдинга, получаем, что для любого $\delta\hm>0$ найдется 
константа $C_\delta\hm>0$ такая, что
\begin{multline}
{\sf P}\left(\fr{1}{D_J}\left\vert \sum\limits_{j=0}^{[dJ]}\beta^2_{j}\sum\limits_{m=1}^{M_j} 
\left[F_m(T^*,L)-{}\right.\right.\right.\\
\left.\left.\left.{}-{\sf E} F_m(T^*,L)\right]
\vphantom{\sum\limits_{j=0}^{[dJ]}}
\right\vert>\delta\right)\leqslant{}
\\
{}\leqslant\sum\limits_{j=0}^{[dJ]}{\sf P}\left( \fr{\beta^2_{j}}{D_J}
\sum\limits_{m=1}
^{M_j} \left[F_m(T^*,L)-{}\right.\right.\\
\!\!\!\left.\left.{}-{\sf E} F_m(T^*,L)\right]\!>\!J^{-1}\delta
\vphantom{\sum\limits_{j=0}^{[dJ]}}
\!\right)\!\leqslant\! % {}\\
%{}\leqslant
J\exp\left\{-C_\delta\fr{2^{J-[dJ]}}{J^2}\!\right\}\!,
\!\!\label{Hoeff}\!\!
\end{multline} 
т.\,е.\
\begin{multline}
\label{First_Sum}
\fr{\sum\nolimits_{j=0}^{[dJ]}\beta^2_{j}\sum\nolimits_{m=1}^{M_j} \left[F_m(T^*,L)-
{\sf E} F_m(T^*,L)\right]}{D_J}\stackrel{{\sf P}}{\to}0\\
\mbox{при}\; J\to\infty.
\end{multline}

При $p_1\leqslant p_2$ справедливы неравенства~\cite{Cai99}
\begin{multline}
\label{Norms}
\left(\sum\limits_{k=0}^{2^j-1}\abs{\mu_{j,k}}^{p_2}\right)^{1/p_2}\leqslant\left(\sum\limits_{k=0}^{2^j-1}
\abs{\mu_{j,k}}^{p_1}\right)^{1/p_1}\leqslant{}\\
{}\leqslant 2^{j(1/p_1-1/p_2)}\left(\sum\limits_{k=0}^{2^j-1}
\abs{\mu_{j,k}}^{p_2}\right)^{1/p_2}.
\end{multline}
Так как $f\in B^{\gamma}_{p,q}(A)$, то 
$$
2^{js}\left(\sum\limits_{k=0}^{2^j-1}\abs{\mu_{j,k}}^p\right)^{1/p}\hm\leqslant 
A2^{J/2}
$$ 
и~из неравенств~\eqref{Norms} следует, что при $p\hm\geqslant2$
\begin{equation}
\label{Besov2}
\sum\limits_{k=0}^{2^j-1}\mu_{j,k}^2 =\sum\limits_{m=1}^{M_j}\sum\limits_{k\in 
B_{j,m}}\mu_{j,k}^2\leqslant A^22^{-2\gamma j+J}.
\end{equation}
Для произвольного $\varepsilon\hm>0$ при всех $j\hm>dJ$ существуют не более $A^22^{J-2\gamma 
j\hm+\varepsilon j}$ слагаемых в~\eqref{Besov2} таких, что 
$$
\sum\limits_{k\in B_{j,m}}\mu_{j,k}^2\hm>2^{-\varepsilon j}.
$$
Выделяя эти слагаемые из второй суммы в~(\ref{Two_Sums}) в~отдельную сумму~$S'$ и~применяя к~$S'$ неравенство Хеффдинга, аналогичное~\eqref{Hoeff}, получаем, 
что
$$
\fr{S'}{\sigma^2\sqrt{2^{J+1}}}\stackrel{{\sf P}}{\to}0
$$ 
при $J\hm\to\infty$.
Таким образом, без ограничения общности можно считать, что во второй сумме в~(\ref{Two_Sums}) 
$$
\sum\limits_{k\in B_{j,m}}\mu_{j,k}^2\hm\to0
$$ 
при $J\hm\to\infty$
для всех $m\hm=1,\ldots,M_j$ и~$j\hm>dJ$.

Если $p<2$ и~$\gamma>1/p-1/2$, то $s\hm>0$ и~из неравенств~\eqref{Norms} следует, 
что
\begin{equation}
\label{Besovp}
\sum\limits_{k=0}^{2^j-1}\mu_{j,k}^2 \leqslant A^2\cdot 2^{-2s j+J}.
\end{equation}
Рассуждая аналогично случаю $p\hm\geqslant2$, заключаем, что без ограничения общности 
можно считать, что во второй сумме в~(\ref{Two_Sums}) также $\sum\nolimits_{k\in B_{j,m}}\mu_{j,k}^2\hm\to0$ при $J\hm\to\infty$ для всех $m\hm=1,\ldots,M_j$ и~$j\hm>dJ$.

Рассмотрим дисперсии слагаемых во второй сумме в~(\ref{Two_Sums}). Имеем 
${\sf D} F_m(T^*,L)\hm={\sf D} \left[S^2_{j,m}\hm-L\sigma^2\hm+U_{j,m}(T^*,L)\right]$, где
\begin{multline*}
U_{j,m}(T^*,L)=\left[\fr{(T^*L)^2\sigma^4-2T^*L\sigma^4(L-2)}{S_{j,m}^2}-{}\right.\\
\left.{}-
S_{j,m}^2 +2L\sigma^2
\vphantom{\fr{(T^*L)^2\sigma^4-2T^*L\sigma^4(L-2)}{S_{j,m}^2}}
\right]\cdot \Ik\left(S^2_{j,m}>T^*L\sigma^2\right).
\end{multline*}
Пусть $X_{L}$~--- случайная величина, имеющая распределение~$\chi^2_{L}$ с~$L$ 
степенями свободы, и~$C_1$ и~$C_2$~--- некоторые положительные константы.
Тогда\linebreak в~силу~(\ref{Besov2}) или~(\ref{Besovp}) для некоторой константы 
$1\hm<C_\chi\hm<1\hm+\delta_\chi$ ($0\hm<\delta_\chi\hm<1$)

\vspace*{-6pt}

\noindent
\begin{multline*}
{\sf D}\left[S_{j,m}^2\cdot \Ik(S^2_{j,m}>T^*L\sigma^2)\right]\leqslant{}\\
{}\leqslant {\sf E} 
\left[\left(S_{j,m}^2\right)^2\cdot \Ik(S^2_{j,m}>T^*L\sigma^2)\right]\leqslant{}\\
{}\leqslant C^2_\chi{\sf E}\left[X^2_{L}\cdot \Ik(C_\chi X_{L}>T^*L)\right]\leqslant{}\\
{}\leqslant 
C_1(T^*L)^2{\sf P}\left(C_\chi X_{L}>T^*L\right)
\end{multline*}
и аналогично

\vspace*{-6pt}

\noindent
\begin{multline*}
{\sf D}\left(\left[\fr{(T^*L)^2\sigma^4-2T^*L\sigma^4(L-2)}
{S_{j,m}^2}+{}\right.\right.\\
\left.\left.{}+2L\sigma^2
\vphantom{\fr{(T^*L)^2\sigma^4-2T^*L\sigma^4(L-2)}{S_{j,m}^2}}
\right]\cdot \Ik(S^2_{j,m}>T^*L\sigma^2)\right)\leqslant {}\\
{}\leqslant
C_2(T^*L)^2{\sf P}\left(C_\chi X_{L}>T^*L\right).
\end{multline*}

\vspace*{-6pt}

\noindent
В силу леммы~2 из работы~\cite{Cai99} и~вида~$T^*$ и~$L$
$$
\left(T^*L\right)^2{\sf P}\left(C_\chi X_{L}\hm>T^*L\right)\to0
$$
при $J\hm\to\infty$.

Таким образом, учитывая очевидное соотношение для дисперсии разности случайных 
величин $X\hm-Y$: 
$$
(\sqrt{{\sf D} X}\hm-\sqrt{{\sf D} Y})^2\hm\leqslant {\sf D}(X\hm-
Y)\leqslant(\sqrt{{\sf D} X}+\sqrt{{\sf D} Y})^2,
$$
 получаем
\begin{equation}
\label{Var_Lim1}
\lim\limits_{J\to\infty}\fr{{\sf D}\sum\nolimits_{j=[dJ]+1}^{J-1}\beta^2_{j}\sum\nolimits_{m=1}^{M_j} 
F_m(T^*,L)}{{\sf D}\sum\nolimits_{j=[dJ]+1}^{J-1}\sum\nolimits_{k=0}^{2^j-
1}\beta^2_{0}2^{2\alpha j}Y^2_{j,k}}=1\,.
\end{equation}
Кроме того, поскольку $Y_{j,k}$ независимы и~${\sf D} Y^2_{j,k}\hm=2\sigma^4\hm+4\sigma^2\mu^2_{j,k}$, получаем:
\begin{equation}
\label{Var_Lim2}
\lim\limits_{J\to\infty}\fr{{\sf D}\sum\nolimits_{j=[dJ]+1}^{J-1}\sum\nolimits_{k=0}^{2^j-1}\beta^2_{0}2^{2\alpha j}Y^2_{j,k}}{D^2_J}=1.
\end{equation}

Наконец, выполнено условие Линдеберга: для любого $\epsilon\hm>0$ при $J\hm\to\infty$

\vspace*{-6pt}

\noindent
\begin{multline}
\label{Norm_Cond}
\fr{1}{D^2_J}\sum\limits_{j=[dJ]+1}^{J-1}\sum\limits_{m=1}^{M_j} {\sf E}
\left[\beta^4_{j}\left( F_m(T^*,L)-{}\right.\right.\\
\left.{}-{\sf E} F_m\left(T^*,L\right)\right)^2\cdot \Ik\left( 
\beta^2_{j}\left\vert F_m(T^*,L)-{}\right.\right.\\
\left.\left.\left.{}-{\sf E} F_m\left(T^*,L\right)\right\vert >\epsilon D_J\right)\right]\rightarrow 0\,.
\end{multline}

\vspace*{-6pt}

\noindent
Действительно, в~силу~(\ref{Bounds}), (\ref{Var_Lim1}) и~(\ref{Var_Lim2}) 
начиная с~некоторого~$J$ все индикаторы в~(\ref{Norm_Cond}) обращаются в~нуль. Объединяя~(\ref{First_Sum}), 
(\ref{Var_Lim2}) и~(\ref{Norm_Cond}), получаем~(\ref{Normality}). Теорема 
доказана.


\pagebreak



%\smallskip

Докажем теперь свойство сильной состоятельности оценки~(\ref{Risk_Estimate}), 
справедливое при более слабых ограничениях.

\smallskip

\noindent
\textbf{Теорема~2.}\ \textit{Пусть $K$~--- линейный однородный оператор с~показателем 
однородности $\alpha\hm>0$, а $Kf\hm\in  L^2(\mathbb{R})$ и~задана на конечном 
отрезке. Тогда при любом $\theta\hm>1/2\hm+2\alpha$ имеет место сходимость}
\begin{equation}
\label{Strong_Con}
\fr{\widehat{R}_J(T^*)-R_J(T^*)}{2^{\theta J}}\rightarrow 0 \;\mbox{ п.~в.\ при\ } J\rightarrow\infty.
\end{equation}

\noindent
Д\,о\,к\,а\,з\,а\,т\,е\,л\,ь\,с\,т\,в\,о\,:\ \  Используя неравенство Хеффдинга, с~учетом~(\ref{Bounds})
 получаем, что для любого $\delta\hm>0$ найдется константа 
$C_\delta\hm>0$ такая, что

\vspace*{-6pt}

\noindent
\begin{multline*}
p_J={\sf P}\left(\fr{\abs{\widehat{R}_J(T^*)-R_J(T^*)}}{2^{\theta 
J}}>\delta\right)\leqslant {}\\
{}\leqslant J\exp\left\{-C_\delta\fr{2^{(2\theta-1-4\alpha) 
J}}{J^2}\right\},
\end{multline*}

\vspace*{-6pt}

\noindent
и, поскольку
\begin{equation*}
\sum\limits_{J=1}^{\infty}p_J<\infty\,,
\end{equation*}

\vspace*{-3pt}

\noindent
в силу леммы Бореля--Кан\-тел\-ли выполнено~(\ref{Strong_Con}). Тео\-ре\-ма доказана.

Теоремы 1 и~2 дают теоретическое обоснование использования значения 
$\widehat{R}_J(T^*)$ в~качестве оценки неизвестной величины риска (по\-греш\-ности) 
$R_J(T^*)$, а~так\-же поз\-во\-ля\-ют строить асимптотические доверительные интервалы 
для~$R_J(T^*)$.

\vspace*{-9pt}

{\small\frenchspacing
 {\baselineskip=10.6pt
 %\addcontentsline{toc}{section}{References}
 \begin{thebibliography}{99}
 
 \vspace*{-3pt}

\bibitem{HKP99}
\Au{Hall P., Kerkyacharian~G., Picard~D.} On the minimax optimality of block 
thresholded wavelet estimators~// Stat. Sinica, 1999. Vol.~9. P.~33--50.

\bibitem{Cai99}
\Au{Cai T.} Adaptive wavelet estimation: A~block thresholding and oracle 
inequality approach~// Ann. Stat., 1999. Vol.~28. No.\,3. P.~898--924.
doi: 10.1214/aos/1018031262.

\bibitem{St81} %3
\Au{Stein C.} Estimation of the mean of a~multivariate normal distribution~// Ann. Stat., 1981. Vol.~9. No.\,6. P.~1135--1151.
doi: 10.1214/aos/1176345632.

\bibitem{SH21} %4
\Au{Шестаков О.\,В.} Анализ несмещенной оценки среднеквадратичного риска 
метода блочной пороговой обработки~// Информатика и~её применения, 2021.  Т.~15. 
Вып.~2. С.~30--35. doi: 10.14357/19922264210205. EDN: DSQQAU.

\bibitem{KS11} %5
\Au{Кудрявцев А.\,А., Шестаков~О.\,В.} Асимптотика оценки риска при вейг\-лет-вейв\-лет 
разложении наблюдаемого сигнала~// T-Comm~--- Телекоммуникации 
и~транспорт, 2011. №\,2. С.~54--57.

\bibitem{SH12} %6
\Au{Шестаков О.\,В.}  О~свойствах оценки среднеквадратичного риска при 
регуляризации обращения линейного однородного оператора с~по\-мощью адаптивной 
пороговой обработки коэффициентов вейг\-лет-вейв\-лет разложения~// Вестн. Тверск. 
ун-та. Сер. Прикладная математика, 2012. №\,1. С.~117--130. EDN: PEZOJH.

\bibitem{SH16} %7
\Au{Шестаков О.\,В.} Статистические свойства метода подавления шума, 
основанного на стабилизированной жесткой пороговой обработке~// Информатика и~её 
применения, 2016.  Т.~10. Вып.~2. С.~65--69. doi: 10.14357/19922264160207. EDN: WCBWWP.

\bibitem{SH18} %8
\Au{Shestakov O.\,V.} Nonlinear regularization of inverse problems for linear 
homogeneous transforms by stabilized hard thresholding~// J.~Math. Sci., 2018. 
Vol.~234. No.\,6. P.~780--785. doi: 10.1007/s10958-018-4044-1.

\bibitem{AS98} %9
\Au{Abramovich F., Silverman~B.\,W.} Wavelet decomposition approaches to 
statistical inverse problems~// Biometrika, 1998. Vol.~85. No.\,1. P. 115--129.

\bibitem{Lee97}
\Au{Lee N.} Wavelet-vaguelette decompositions and homogenous equations: PhD 
Thesis.~--- West Lafayette, IN, USA: Purdue University, 1997. 93~p.

\bibitem{Mall99}
\Au{Mallat S.} A~wavelet tour of signal processing.~--- New York, NY, USA: 
Academic Press, 1999. 857~p.

\bibitem{DonJ94} %12
\Au{Donoho D., Johnstone~I.\,M.} Ideal spatial adaptation via wavelet 
shrinkage~// Biometrika, 1994. Vol.~81. No.\,3.
P.~425--455. doi: 10.2307/2337118.

\bibitem{DonJ95} %13
\Au{Donoho D., Johnstone~I.\,M.} Adapting to unknown smoothness via wavelet 
shrinkage~// J.~Am. Stat. Assoc., 1995. Vol.~90. P.~1200--1224. doi: 10.2307/2291512.

\bibitem{DonJ98} %14
\Au{Donoho D., Johnstone~I.\,M.} Minimax estimation via wavelet shrinkage~// 
Ann. Stat., 1998. Vol.~26. No.\,3. P.~879--921.

\bibitem{G98}
\Au{Gao H.-Y.} Wavelet shrinkage denoising using the non-negative garrote~// 
J.~Comput. Graph. Stat., 1998. Vol.~7. No.\,4. P.~469--488.

\bibitem{PK05} %16
\Au{Poornachandra S., Kumaravel~N., Saravanan~T.\,K., Somaskandan~R.} 
WaveShrink using modified hyper-shrinkage function // 27th Annual Conference (International) 
 of the IEEE Engineering in Medicine and Biology Society Proceedings.~--- Piscataway, NJ, USA: IEEE, 2005. P.~30--32.

\bibitem{LC10} %17
\Au{Lin Y., Cai~J.} A~new threshold function for signal denoising based on 
wavelet transform~// Conference (International) on Measuring Technology and Mechatronics Automation Proceedings.~--- 
Piscataway, NJ, USA: IEEE, 2010. P.~200--203.

\bibitem{HL10} %18
\Au{Huang H.-C., Lee~T.\,C.\,M.} Stabilized thresholding with generalized sure 
for image denoising~// 17th Conference (International) on Image Processing Proceedings.~--- Piscataway, NJ, USA: IEEE, 2010. P.~1881--1884.

\bibitem{HX15}
\Au{He C., Xing~J., Li~J., Yang~Q., Wang~R.} A~new wavelet thresholding 
function based on hyperbolic tangent function~// Math. Probl. 
Eng., 2015. Vol.~2015. Art.~528656. 10~p.

\bibitem{ZC15}
\Au{Zhao R.-M., Cui H.-M.} Improved threshold denoising method based on 
wavelet transform~// 7th  Conference (International) on 
Modelling, Identification and Control Proceedings.~--- Piscataway, NJ, USA: 
IEEE, 2015. Art.~7409352. 4~p. doi: 10.1109/ICMIC.2015.7409352.
\end{thebibliography}

 }
 }

\end{multicols}

\vspace*{-10pt}

\hfill{\small\textit{Поступила в~редакцию 07.09.23}}

%\vspace*{8pt}

%\pagebreak

\newpage

\vspace*{-28pt}

%\hrule

%\vspace*{2pt}

%\hrule



\def\tit{NONLINEAR REGULARIZATION OF~THE~INVERSION\\ OF~LINEAR
HOMOGENEOUS OPERATORS\\ USING~THE~BLOCK THRESHOLDING
METHOD}


\def\titkol{Nonlinear regularization of~the~inversion of~linear
homogeneous operators using~the~block thresholding
method}


\def\aut{O.\,V.~Shestakov$^{1,2,3}$ and~E.\,P.~Stepanov$^{1}$}

\def\autkol{O.\,V.~Shestakov and~E.\,P.~Stepanov}

\titel{\tit}{\aut}{\autkol}{\titkol}

\vspace*{-10pt}


\noindent
$^1$Department of Mathematical Statistics, Faculty of Computational Mathematics and Cybernetics, M.\,V.~Lomo-\linebreak
$\hphantom{^1}$nosov Moscow State University, 1-52~Leninskie 
Gory, GSP-1, Moscow 119991, Russian Federation

\noindent
$^2$Moscow Center for Fundamental and Applied Mathematics, M.\,V.~Lomonosov Moscow State University,
1~Lenin-\linebreak
$\hphantom{^1}$skie Gory, GSP-1, Moscow 119991, Russian Federation

\noindent
$^3$Federal Research Center ``Computer Science and Control'' of the Russian Academy of Sciences, 44-2~Vavilov\linebreak
$\hphantom{^1}$Str., Moscow 119333, Russian Federation


\def\leftfootline{\small{\textbf{\thepage}
\hfill INFORMATIKA I EE PRIMENENIYA~--- INFORMATICS AND
APPLICATIONS\ \ \ 2023\ \ \ volume~17\ \ \ issue\ 4}
}%
 \def\rightfootline{\small{INFORMATIKA I EE PRIMENENIYA~---
INFORMATICS AND APPLICATIONS\ \ \ 2023\ \ \ volume~17\ \ \ issue\ 4
\hfill \textbf{\thepage}}}

\vspace*{3pt}




\Abste{The methods of thresholding the coefficients of wavelet expansions have
become a popular tool for regularization of inverse statistical problems due to their
simplicity, computational efficiency, and the ability to adapt both to the type of operators
and to the features of the function under study. This approach proved to be the most
fruitful for inversion of linear homogeneous operators arising in some signal and image
processing problems. The paper considers the block thresholding method in which the
decomposition coefficients are processed in groups that allows taking into account
information about neighboring coefficients. In a~data model with an additive Gaussian
noise, an unbiased estimate of the mean-square risk is analyzed and it is shown that under
certain conditions, this estimate is strongly consistent and asymptotically normal. These
properties allow constructing asymptotic confidence intervals for the theoretical mean-square risk of the method under consideration.}


\KWE{linear homogeneous operator; wavelets; block thresholding; unbiased risk
estimate; asymptotic normality; strong consistency}



\DOI{10.14357/19922264230401}{PGKKYE}

\vspace*{-20pt}

\Ack

\vspace*{-4pt}

\noindent
The work was done with the support of MSU Program of Development, Project No.\,23-SCH03-03.



\vspace*{-1pt}

  \begin{multicols}{2}

\renewcommand{\bibname}{\protect\rmfamily References}
%\renewcommand{\bibname}{\large\protect\rm References}

{\small\frenchspacing
 {%\baselineskip=10.8pt
 \addcontentsline{toc}{section}{References}
 \begin{thebibliography}{99} 
 
 \vspace*{-5pt}
%1
\bibitem{HKP99-1} 
\Aue{Hall, P., G.~Kerkyacharian, and D.~Picard.}
 1999. On the minimax optimality of block thresholded wavelet estimators. \textit{Stat. Sinica} 9:33--50.

%2
\bibitem{Cai99-1}
\Aue{Cai, T.} 1999. Adaptive wavelet estimation: A~block thresholding and oracle inequality approach. 
\textit{Ann. Stat.} 28(3):898--924. doi: 10.1214/aos/1018031262.

%3
\bibitem{St81-1}
\Aue{Stein, C.} 
1981. Estimation of the mean of a multivariate normal distribution. \textit{Ann. Stat.} 9(6):1135--1151. doi: 10.1214/aos/1176345632.

%4
\bibitem{SH21-1}
\Aue{Shestakov, O.\,V.} 2021. 
Analiz nesmeshchennoy otsenki srednekvadratichnogo riska metoda blochnoy po\-ro\-go\-voy obrabotki 
[Analysis of the unbiased mean-square risk estimate of the block thresholding method]. 
\textit{Informatika i~ee Primeneniya~--- Inform. Appl.} 15(2):30--35. doi: 10.14357/19922264210205. EDN: DSQQAU.

%5
\bibitem{KS11-1}
\Aue{Kudryavtsev, A.\,A., and O.\,V.~Shestakov.}
 2011. Asimptotika otsenki riska pri veyglet-veyvlet razlozhenii nablyuda\-emo\-go signala [The average risk assessment of the wavelet decomposition of the signal]. 
 \textit{T-Comm~--- Telekommunikatsii i~transport} [T-Comm~--- Telecommunications and Transport] 2:54--57.

%6
\bibitem{SH12-1}
\Aue{Shestakov, O.\,V.} 2012. 
O~svoystvakh otsenki sred\-ne\-kvad\-ra\-tich\-no\-go riska pri regulyarizatsii obrashcheniya lineynogo odnorodnogo operatora 
s~pomoshch'yu adaptivnoy porogovoy obrabotki koeffitsientov veyglet-veyvlet raz\-lo\-zhe\-niya 
[On the properties of estimating the mean-square risk in the regularization of the inversion of a~linear homogeneous operator using adaptive 
thresholding of the coefficients of the vaguelette-wavelet decomposition]. 
\textit{Vestn. Tversk. un-ta. Ser. Prikladnaya matematika} [Herald of Tver State University. Series: Applied Mathematics] 1:117--130. EDN: PEZOJH.

%7
\bibitem{SH16-1}
\Aue{Shestakov, O.\,V.}
 2016. Statisticheskie svoystva metoda podavleniya schuma, osnovannogo na stabilizirovannoy zhest\-koy porogovoy obrabotke 
 [Statistical properties of the denoising method based on the stabilized hard thresholding]. 
 \textit{Informatika i~ee Primeneniya~--- Inform. Appl.} 10(2):65--69. 
doi: 10.14357/19922264160207. EDN: WCBWWP.

%8
\bibitem{SH18-1}
\Aue{Shestakov, O.\,V.}
 2018. Nonlinear regularization of inverse problems for linear homogeneous transforms by stabilized hard thresholding. 
 \textit{J.~Math. Sci.} 234(6):780--785. doi: 10.1007/s10958-018-4044-1.

%9
\bibitem{AS98-1}
\Aue{Abramovich, F., and B.\,W.~Silverman.}
 1998. Wavelet decomposition approaches to statistical inverse problems. \textit{Biometrika} 85(1):115--129.

%10
\bibitem{Lee97-1}
\Aue{Lee, N.} 1997. Wavelet-vaguelette decompositions and homogeneous equations.  West Lafayette, IN: Purdue University. PhD Thesis. 93~p.

%11
\bibitem{Mall99-1}
\Aue{Mallat, S.} 1999. \textit{A wavelet tour of signal processing}. New York, NY: Academic Press. 857~p.

%12
\bibitem{DonJ94-1}
\Aue{Donoho, D., and I.\,M.~Johnstone.} 1994. Ideal spatial adaptation via wavelet shrinkage. \textit{Biometrika} 81(3):425--455. doi: 10.2307/2337118.

%13
\bibitem{DonJ95-1}
\Aue{Donoho, D., and I.\,M.~Johnstone.} 1995. Adapting to unknown smoothness via wavelet shrinkage. \textit{J.~Am. Stat. Assoc.} 90:1200--1224. 
doi: 10.2307/2291512.

%14
\bibitem{DonJ98-1}
\Aue{Donoho, D., and I.\,M.~Johnstone.} 1998. Minimax estimation via wavelet shrinkage. \textit{Ann. Stat.} 26(3):879--921.

%15
\bibitem{G98-1}
\Aue{Gao, H.-Y.} 1998. Wavelet shrinkage denoising using the non-negative garrote. \textit{J.~Comput. Graph. Stat.} 7(4):469--488.

%16
\bibitem{PK05-1}
\Aue{Poornachandra, S., N.~Kumaravel, T.\,K.~Saravanan, and R.~Somaskandan.}
 2005. WaveShrink using modified hyper-shrinkage function. \textit{27th Annual Conference (International) 
 of the IEEE Engineering in Medicine and Biology Society Proceedings}. Piscataway, NJ: IEEE. 30--32.

%17
\bibitem{LC10-1}
\Aue{Lin, Y, and J.~Cai.} 2010. A~new threshold function for signal denoising based on wavelet transform. 
\textit{Conference (International) on Measuring Technology and Mechatronics Automation Proceedings}. Piscataway, NJ: IEEE.\linebreak 200--203.

%18
\bibitem{HL10-1}
\Aue{Huang, H.-C., and T.\,C.\,M.~Lee.}
 2010. Stabilized thresholding with generalized sure for image denoising. 
 \textit{17th Conference (International) on Image Processing Proceedings}. Piscataway, NJ: IEEE. 1881--1884.

%19
\bibitem{HX15-1}
\Aue{He, C., J.~Xing, J.~Li, Q.~Yang, and R.~Wang.} 2015. 
A~new wavelet thresholding function based on hyperbolic tangent function. \textit{Math. Probl. Eng.} 2015:528656. 10~p.

%20
\bibitem{ZC15-1}
\Aue{Zhao, R.-M., and H.-M.~Cui.}
 2015. Improved threshold denoising method based on wavelet transform. 
 \textit{7th Conference (International) on Modelling, Identification and Control Proceedings}. Piscataway, NJ: IEEE. Art. ID.~7409352. 4~p. 
 doi: 10.1109/ICMIC.2015.7409352.
\end{thebibliography}

 }
 }

\end{multicols}

\vspace*{-6pt}

\hfill{\small\textit{Received September 7, 2023}} 

%\vspace*{-18pt}

\Contr

\vspace*{-4pt}

\noindent
\textbf{Shestakov Oleg V.} (b.\ 1976)~--- Doctor of Science in physics and mathematics, professor, Department of Mathematical Statistics, 
Faculty of Computational Mathematics and Cybernetics, M.\,V.~Lomonosov Moscow State University, 1-52~Leninskie Gory, GSP-1, 
Moscow 119991, Russian Federation; senior scientist, Federal Research Center ``Computer Science and Control'' 
of the Russian Academy of Sciences, 44-2~Vavilov Str., Moscow 119333, Russian Federation; leading scientist, Moscow Center 
for Fundamental and Applied Mathematics, M.\,V.~Lomonosov Moscow State University, 1~Leninskie Gory, GSP-1, Moscow 119991, Russian Federation; 
\mbox{oshestakov@cs.msu.su}

\vspace*{3pt}

\noindent
\textbf{Stepanov Evgeniy P.} (b.\ 1994)~--- Candidate of Science (PhD) in physics and
mathematics, junior scientist, Department of Computing Systems and Automation, Faculty of
Computational Mathematics and Cybernetics, M.\,V.~Lomonosov Moscow State
University, 1-52~Leninskie Gory, GSP-1, Moscow 119991, Russian Federation;
\mbox{estepanov@lvk.cs.msu.ru}


\label{end\stat}

\renewcommand{\bibname}{\protect\rm Литература} 