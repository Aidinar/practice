\def\stat{zatsman}

\def\tit{НАУЧНАЯ ПАРАДИГМА ИНФОРМАТИКИ:\\ КЛАССИФИКАЦИЯ ОБЪЕКТОВ 
ПРЕДМЕТНОЙ ОБЛАСТИ$^*$}

\def\titkol{Научная парадигма информатики: классификация объектов 
предметной области}

\def\aut{И.\,М.~Зацман$^1$}

\def\autkol{И.\,М.~Зацман}

\titel{\tit}{\aut}{\autkol}{\titkol}

\index{Зацман И.\,М.}
\index{Zatsman I.\,M.}


{\renewcommand{\thefootnote}{\fnsymbol{footnote}} \footnotetext[1]
{Исследование выполнено с~использованием инфраструктуры Цент\-ра коллективного пользования 
<<Высокопроизводительные вы\-чис\-ле\-ния и~большие данные>> (ЦКП <<Информатика>>) ФИЦ ИУ РАН 
(г.~Москва).}}


\renewcommand{\thefootnote}{\arabic{footnote}}
\footnotetext[1]{Федеральный исследовательский центр <<Информатика и~управление>> Российской академии наук;
\mbox{izatsman@yandex.ru}}

\vspace*{-12pt}




  \Abst{Дано описание трех верхних уровней классификации объектов предметной области 
информатики, которая позиционируется как составная часть системы научного знания, 
охватывающая широкий спектр информационных и~компьютерных наук. Границы 
предметной области при таком позиционировании значительно расширяются и~во многом 
соответствуют концепции полиадического компьютинга Пола Розенблума. В~предлагаемой 
научной парадигме информатики все сущности ее предметной области разделены на два 
глобальных класса: объекты и~их трансформации. Для каждого такого класса в~процессе 
создания парадигмы конструируется своя классификация. С~формирования этих 
классификаций и~началось построение парадигмы. В~данной статье рассматриваются три 
верхних уровня классификации объектов предметной области информатики. Основанием для 
построения первого уровня (самого верхнего) служит предложенное ранее деление 
предметной области информатики на среды: ментальную, сенсорно воспринимаемую, 
цифровую и~ряд других сред. Основанием для построения второго уровня служит разделение 
сенсорно воспринимаемых объектов информатики на данные и~знаковую информацию как 
результат преобразования в~знаковую форму когнитивных структур человека. Основанием 
для построения третьего уровня служит типология знаковых систем А.~Соломоника. Цель 
статьи состоит в~описании подхода к~построению трех верхних уровней классификации 
объектов информатики и~его сопоставлении с~ранее использованными подходами 
к~описанию основных сущностей ее предметной области. Также на основе предлагаемого 
подхода отчасти сформулированы ответы на те вопросы Томаса Куна об основных 
сущностях предметной области, которые должна содержать парадигма любой науки, а~не 
только информатики.}
   
\KW{научная парадигма; классификация объектов предметной об\-ласти информатики; 
основания классификации; среды предметной об\-ласти}

  \DOI{10.14357/19922264230413}{FIUQAT}
  
%\vspace*{-4pt}


\vskip 10pt plus 9pt minus 6pt

\thispagestyle{headings}

\begin{multicols}{2}

\label{st\stat}

\section{Введение}

  Вопросы позиционирования информатики в~сис\-те\-ме научного знания 
дебатируются на протяжении нескольких десятилетий. Они обсуждаются, 
например, в~работах К.~Нюгора, Д.~Харела, К.~Колина, Р.~Гиляревского,    
П.~Деннинга, П.~Розенблума, М.~Касперсена и~его соавторов~[1--8]. Эти 
работы были опубликованы в~период с~1986 по 2019~гг. Сегодня эти вопросы  
по-преж\-не\-му сохраняют свою актуальность и~активно обсуждаются, в~том 
чис\-ле с~позиции поиска необходимой совокупности теоретических оснований 
для разработки стратегии компьютерного образования. В~[9] М.~Тедре 
и~Д.~Паюнен не только рассмотрели камни преткновения и~проблемные места 
в~поиске теоретических оснований и~в~определении предметной области 
информатики в~интересах компьютерного образования. Они также 
констатировали, что наибольшим препятствием выступает не что иное, как 
\textit{отсутствие конвенциональной научной парадигмы информатики 
в~трактовке Т.~Куна}~[10].
  
  Необходимость создания именно куновской парадигмы в~интересах 
компьютерного образования отмечалась и~раньше~[11]. Неослабевающий 
интерес именно к~парадигмальности теоретических оснований обусловлен тем, 
что с~точки зрения Т.~Куна научная парадигма любой науки, чтобы обеспечить 
эффективные научные исследования, должна содержать или позволять 
получать ответы на следующие вопросы~[10, 11]:
\begin{itemize}
\item Из каких основных сущностей состоит ее предметная область?\\[-14pt]
  \item Какие трансформации описывают взаимодействие сущностей друг 
с~другом и~какие сущности взаимодействуют с~органами чувств?\\[-14pt]
  \item Какие вопросы допустимо задать об основных ее сущностях и~какие 
методы используются для поиска решений ее задач?
  \end{itemize}
  
  В~настоящее время отсутствует конвенциональная парадигма информатики, 
которая отвечала бы на эти вопросы. Структура предлагаемого варианта 
научной парадигмы, первоначальная версия\linebreak\vspace*{-12pt}

\pagebreak

\noindent
 которого отчасти позволяет 
ответить на них, содержит восемь компонентов, перечисленных 
и~специфицированных в~работах А.~Соломоника~[12, 13]. Согласно 
Соломонику, научная парадигма <<зрелой>> науки состоит из следующих 
компонентов, которые могут разрабатываться отдельно, но объединяются 
в~единую и~цельную конструкцию: (1)~философские основы; (2)~предмет 
изучения; (3)~методы изучения; (4)~аксиоматика; (5)~классификации;  
(6)~сис\-те\-ма терминов; (7)~языки [знаковые системы] науки; (8)~методы 
верификации результатов. Сам термин <<научная парадигма>> трактуется им 
в~соответствии с~теорией Т.~Куна, которая описывает процесс смены научных 
парадигм~[10]. При этом А.~Соломоник отмечает тот факт, что в~книге Куна 
нет ответа на вопрос: <<Из чего должна состоять парадигма любой ``зрелой'' 
науки?>>~\cite[с.~23]{14-zac}. Отметим, что в~данной статье рассматривается 
пятый компонент парадигмы (классификации сущностей предметной об\-ласти 
информатики) и~основное внимание уделяется только классификации объектов 
ее предметной об\-ласти.
  
  Цель статьи состоит в~описании подхода к~построению трех верхних уровней 
классификации объектов и~его сопоставлении с~ранее использованными 
подходами к~описанию основных сущностей предметной области информатики 
(объектный, трансформационный и~синтетический). Предлагаемый подход 
и~нижеизложенные результаты первой стадии его реализации уже позволяют 
отчасти ответить на первые два из трех вопросов, сформулированных Куном. 
Отметим, что построению верхних уровней классификации 
\textit{трансформаций объектов предметной области информатики} 
посвящена работа~[15].

\vspace*{-6pt}
  
\section{Три подхода к~описанию предметной области 
информатики}

\vspace*{-3pt}

  Все сущности информатики в~предлагаемой научной парадигме 
информатики разделены на два глобальных класса: объекты и~их 
трансформации.\linebreak Для каждого такого клас\-са в~процессе создания па\-ра\-диг\-мы 
конструируется своя классификация. Рас\-смот\-рим три основных подхода, 
которые ранее использовались в~информатике для описания \mbox{сущностей} ее 
предметной об\-ласти: объектный, трансформационный и~синтетический. 
Примерами первого подхода, в~рамках которого основное внимание уделяется 
объектам предметной об\-ласти информатики (например, данные, информация 
и/или знание), могут служить работы~[16--18]; примерами второго подхода, 
в~рамках которого основное внимание уделяется трансформациям и~в~меньшей 
степени трансформируемым объектам~--- работы~\cite{6-zac, 19-zac}; 
примерами треть\-его, синтетического, подхода, в~котором уделяется внимание 
и~объектам предметной об\-ласти информатики, и~их трансформациям, могут 
служить работы~[20--24].
  
  Для построения двух классификаций (объектов и~их трансформаций) 
наибольший интерес пред\-став\-ля\-ют трансформационный и~синтетический 
подходы. В~2009~г.\ П.~Деннинг и~П.~Розенблум сформулировали суть 
информатики как компьютинга следующим образом: <<\ldots информатика~--- 
это не просто алгоритмы и~структуры данных; это трансформации 
представлений>>~\cite{6-zac}. Чуть позже, в~контексте краткого описания 
парадигмы информатики как компьютинга, П.~Деннинг и~П.~Фриман изменили 
эту формулировку на такую: <<\textit{Центральный объект} внимания 
в~информатике можно определить как информационные процессы~--- 
естественные или искусственные процессы, \textit{преобразующие 
информацию} (курсив мой~--- И.\,З.)>>~\cite{19-zac}.
  
  На синтетическом подходе остановимся чуть подроб\-нее, так как именно его 
обобщение использовалось в~процессе построения обеих классификаций 
в~процессе создания научной парадигмы информатики. Согласно 
Р.~Гиляревскому, <<если \textit{данные воспринимаются и~интерпретируются} человеком, то они становятся для него 
\textit{информацией}>> (курсив мой~--- И.\,З.)~\cite[с.~10]{20-zac}. 
В~работах~\cite{21-zac, 22-zac, 23-zac, 24-zac} внимание также уделяется не 
только таким объектам предметной области, как информация и~знание, но и~их 
трансформациям. Согласно Ю.\,А.~Шрейдеру,  
<<ин\-фор\-ма\-ци\-он\-но-зна\-ни\-евые процессы включают преобразование 
человеческих (в~значительной мере личностных) знаний, существующих 
``здесь'' и~``теперь'', в~социальную информацию, доступную ``везде'' и~``всегда'' 
и~\textit{гарантирующую возможность извлечения из нее знаний} (курсив 
мой~--- И.\,З.)>>~\cite{21-zac}.

  \begin{figure*} %fig1
     \vspace*{1pt}
\begin{center}
   \mbox{%
\epsfxsize=150.639mm 
\epsfbox{zac-1.eps}
}

\vspace*{4pt}

     {\small Три верхних уровня классификации объектов предметной области 
информатики}
\end{center}
\vspace*{-6pt}
     \end{figure*}
  
  Главный вывод П.~Ингверсена состоит в~том, что понятие информации 
должно удовлетворять двум требованиям. С~одной стороны, информация 
является результатом трансформации в~знаковую форму когнитивных структур 
человека. С~другой стороны, информация~--- это то, что трансформирует 
(может трансформировать) знания человека, который ее получает~\cite{22-zac}. 
В~описании предметной области информатики как информационной науки 
Дж.~Фаррадейн сопоставляет знание и~информацию, которые он 
позиционирует как центральные понятия ее предметной области, а также 
рассматривает трансформацию информации в~знания~\cite{23-zac}. По мнению 
В.~Хьорланда, в~определении информационной науки в~явном виде должна 
быть выражена соотнесенность информации со знанием~\cite{24-zac}. 
Перечисленные положения из работ~[20--24] и~стали предпосылками 
формирования двух классификаций сущностей предметной области 
информатики: ее объектов (см.\ рисунок) и~их трансформаций  
(см.\ рисунок на с.~130 в~\cite{15-zac}). 
  
  В~использовании этих положений можно увидеть авторскую позицию при 
построении обеих классификаций: по максимуму интегрировать уже известные в~информатике положения об объектах ее предметной области и~их 
трансформациях, которые не противоречат друг другу.

\vspace*{-6pt}
  
\section{Среды и~объекты предметной области информатики}

\vspace*{-3pt}

  Как отмечено выше, для каждого из двух глобальных классов сущностей 
предметной области информатики (объекты и~их трансформации) в~процессе 
создания ее парадигмы конструируется своя классификация. С~описания 
оснований построения этих классификаций и~началось по\-стро\-ение 
парадигмы~\cite{25-zac, 26-zac}. В~этих работах было дано описание 
основания для по\-стро\-ения верх\-не\-го уровня классификации объектов 
пред\-мет\-ной об\-ласти, основанной на ее \textit{средов$\acute{\mbox{о}}$м 
делении}. Были даны определения сле\-ду\-ющих пяти сред пред\-мет\-ной об\-ласти, 
каж\-дая из которых включает объекты одной и~той же  
природы~\cite{25-zac, 26-zac}:
  \begin{enumerate}[(1)]
\item  \textit{ментальная среда}~--- это совокупность когнитивных феноменов и~их структур, ис\-поль\-зу\-емых и/или фор\-ми\-ру\-емых в~процессе по\-зна\-ния, 
происходящего в~сознании людей (феномены, фор\-ми\-ру\-емые с~использованием 
знаковых сис\-тем в~процессах по\-зна\-ния как смыс\-ло\-вые элементы знания, будем 
называть \mbox{\textit{концептами}});
\item \textit{сенсорно воспринимаемая среда}, которую для краткости иногда 
будем называть \textit{информационной}~--- это совокупность сенсорно 
воспринимаемых объектов, находящихся вне сознания, но взаимодействующих с~когнитивными феноменами и~их структурами;
\item  цифровая среда~--- это совокупность компьютерных кодов;
\item нейросреда~--- это электрические потенциалы и~магнитные поля, 
генерируемые мозгом, которые используются, например, в~информационных 
технологиях (ИТ) управ\-ле\-ния роботизированной рукой~\cite{27-zac} 
и~в~других ИТ, применяющих интерфейсы <<мозг--компь\-ютер>>;
\item ДНК-среда~--- это совокупность цепочек РНК и~ДНК\footnote{Например, 
модели трансляции естественных ДНК, созданные микробиологами, используются в~информатике 
при разработке методов записи данных большого объема с~применением синтезированных цепочек 
ДНК.}.
\end{enumerate}

    Еще в~1986~г.\ Кристен Нюгор имплицитно включил в~предметную 
область информатики объекты ментальной среды~\cite{3-zac}, что фактически 
можно рассматривать как ее добавление к~сенсорно воспринимаемой 
и~цифровой средам. В~научной парадигме информатики число сред 
предлагается увеличить. С~одной стороны, добавляются объекты еще двух сред 
(нейро- и~ДНК-сре\-ды), с~другой стороны, учитывается и~возможность 
включения в~будущем объектов тех сред, которые не используются сейчас 
в~ИТ, системах и~средствах информатики. Поэтому в~предлагаемой научной 
парадигме информатики такая возможность увеличения числа сред 
устанавливается с~самого начала, что в~будущем при построении четвертого 
компонента парадигмы (аксиоматика) планируется сформулировать как одну из 
аксиом информатики.

\vspace*{-6pt}

\section{Верхние уровни классификации объектов}

\vspace*{-3pt}

    В~соответствии с~перечисленными пятью средами верхний уровень 
классификации будет включать как минимум пять классов объектов. Каждый 
из этих классов содержит объекты только одной среды предметной области: 
ментальной, информационной, цифровой, нейро- или ДНК-сре\-ды. При этом 
с~ростом разнообразия природы объектов верхний уровень классификации 
может пополняться новыми классами, включающими объекты, природа 
которых отличается от природы сред, ранее включенных в~верхний  
уровень~\cite{26-zac}. Это может произойти, например, в~том случае, когда при 
проектировании будущих ИТ встретятся объекты, которые по своей природе не 
относятся ни к~одной из сред текущей версии классификации~\cite{28-zac}. 
Таким образом, в~предлагаемом варианте парадигмы информатики 
классификацию объектов ее предметной области предлагается сделать 
открытой, что обусловлено возможным включением в~будущем в~нее объектов 
ранее не рассматривавшейся природы.
    
Перечень классов объектов верхнего уровня (см.\ рисунок) 
сформирован согласно чис\-лу сред раз\-ной природы (объекты ней\-ро\-сре\-ды, 
ментальной, сенсорно вос\-при\-ни\-ма\-емой, циф\-ро\-вой, ДНК-сре\-ды и~будущих 
сред, обозначенных на верх\-нем уровне многоточием)~\cite{25-zac}. Далее 
ограничимся детализацией объектов только трех сред: ментальной, сенсорно 
вос\-при\-ни\-ма\-емой и~циф\-ро\-вой. 
    
   
     
    В~каждом из трех классов объектов (ментальных, сенсорно 
воспринимаемых и~цифровых) выделим два подкласса, разместив их на 
следующем, втором, уровне классификации. Перечислим шесть выделенных 
подклассов: 
\begin{enumerate}[(1)]
\item ментальные образы данных; 
\item знание и~составляющие его 
концепты; 
\item сенсорно воспринимаемые данные; 
\item знаковая информация;
\item цифровые данные;
\item цифровая информация.
\end{enumerate}
 Оставшиеся три части от 
выделения этих подклассов, обозначенные на рисунке знаком~<<?>>, в~статье 
не рассматриваются. Предлагаемое выделение шес\-ти подклассов соответствует 
ранее предложенному определению сред предметной области 
информатики~\cite{25-zac}. Необходимость выделения как минимум двух 
подклассов в~каждой из трех сред при решении прикладных задач 
проиллюстрируем примером кардиограммы, сформированной на основе 
электрических полей, образующихся при регистрации работы сердца, 
и~сохраненной в~компьютере в~цифровой форме.
    
    Кардиограмма на экране компьютера до начала ее интерпретации 
кардиологом~--- это \textit{сенсорно воспринимаемые данные}. Информация 
заключения кардиолога, созданная им как результат содержательного анализа 
данных, является \textit{знаковой} и~по определению~\cite{25-zac} 
принадлежит к~сенсорно воспринимаемой среде. При компьютерном 
кодировании (на границе меж\-ду сенсорно вос\-при\-ни\-ма\-емой и~циф\-ро\-вой 
средами) знаковой информации и~сенсорно вос\-при\-ни\-ма\-емых данных получаем, 
соответственно, циф\-ро\-вую информацию и~циф\-ро\-вые данные. Это два 
принципиально разных подкласса объектов циф\-ро\-вой среды, так как цифровая 
информация всегда есть результат кодирования знаковой информации, а~для 
циф\-ро\-вых данных это не так: они могут быть и~результатом кодирования 
сенсорно вос\-при\-ни\-ма\-емых данных, и~результатом их непосредственной 
генерации в~циф\-ро\-вой среде (например, файл циф\-ро\-вых данных, полученный  
с~по\-мощью маг\-нит\-но-ре\-зо\-нанс\-ной томографии).
    
    Процесс содержательного анализа и~интерпретации данных кардиограммы, 
а также подготовки заключения состоит из нескольких этапов трансформаций 
объектов трех сред: ментальной, сенсорно воспринимаемой и~цифровой. 
Сначала кардиолог сенсорно воспринимает данные кардиограммы, 
предварительно сохраненной в~компьютере (циф\-ро\-вая среда), а~затем 
отобра\-жен\-ной на экране (сенсорно вос\-при\-ни\-ма\-емая среда). На сле\-ду\-ющем 
этапе появляются \textit{ментальные образы данных} в~сознании кардиолога 
как результат их сенсорного восприятия и~затем генерируется \textit{знание} 
(ментальная среда) кардиолога о~наличии или отсутствии нарушений в~работе 
серд\-ца пациента как результат креативного процесса, вклю\-ча\-юще\-го понимание 
кардиологом ментальных образов данных.
    
    Далее следует этап подготовки заключения на некотором естественном 
языке, включающий генерацию \textit{концептов} (ментальная среда) как 
результат\linebreak деления полученного знания и~выражение сге\-не\-рированных 
концептов словами этого языка в~сен\-сорно воспринимаемой среде. Если 
готовится двуязычное заключение, например на русском языке и~в~\mbox{переводе} на 
английский язык, то деление знания на концепты выполняется по-раз\-но\-му  
в~сис\-те\-мах двух языков. Причина разного деления одного и~того же знания 
на концепты состоит в~том, что объемы значений (концептов) слов в~переводе, 
как правило, отличаются от объемов значений слов в~оригинале. Таким 
образом, деление одного и~того же знания кардиолога (о~наличии или 
отсутствии нарушений в~работе сердца пациента) на концепты зависит от 
используемого естественного языка, т.\,е.\ \textit{вербальной знаковой 
системы}, используемой на границе между ментальной и~сенсорно 
воспринимаемой средами.
    
    В~результате второй верхний уровень первоначальной версии 
классификации включает сле\-ду\-ющие объекты ментальной, сенсорно 
вос\-при\-ни\-ма\-емой и~циф\-ро\-вой сред:
    \begin{itemize}
\item знание, концепты и~ментальные образы сенсорно воспринимаемых 
данных~\cite{29-zac}\footnote{Деление на личностное, коллективное, организационное и~конвенциональное знание по Ниссену~\cite{30-zac} (конвенциональное знание по Ниссену 
названо социальным у~Ю.~Шрейдера~\cite{21-zac}) в~первоначальной версии этой 
классификации не рассматривается.};
\item знаковую информацию и~сенсорно воспринимаемые данные;
\item цифровую информацию и~цифровые данные.
\end{itemize}

    При построении третьего уровня классификации объектов в~ее 
первоначальной версии ограничимся только текстовой знаковой информацией, 
ее компьютерными кодами (цифровой информацией) и~тем знанием 
ментальной среды, которое может быть представлено в~текстовой знаковой 
форме. Основанием для построения этого уровня классификации служит 
следующая типология знаковых систем А.~Соломоника~\cite[c.~131]{31-zac}: 
(1)~естественные знаковые сис\-те\-мы (например, классификатор следов 
зверей); (2)~образные (сис\-те\-ма дорожных знаков);  
(3)~естест\-вен\-но-язы\-ко\-в$\acute{\mbox{ы}}$е;  
(4)~вер\-баль\-но-не\-сло\-вес\-ные сис\-те\-мы записи\footnote{Под системой записи 
понимается знаковая система, сочетающая вербальные знаки с~несловесными (языки нотной записи, 
карт, таблиц и~др.).}; (5)~формализованные знаковые системы\footnote{В двух 
последующих монографиях А.~Соломоник делит формализованные системы на знаковые 
формализованные системы первого и~второго порядка~[12, с.~76; 13, с.~64].}, включая 
математические.
    
    Используя эту типологию, введем понятие обобщенного текста~--- это 
текст, который может быть создан с~использованием любой из пяти 
пе\-ре\-чис\-лен\-ных знаковых сис\-тем. Тогда обобщенные текс\-ты по определению 
могут быть естественными, образными, естест\-вен\-но-язы\-ко\-в$\acute{\mbox{ы}}$\-ми, 
вер\-баль\-но-не\-сло\-вес\-ны\-ми и~формализованными. Таким образом, тре\-тий уровень 
классификации в~этой статье охватывает не все виды объектов предметной 
об\-ласти информатики, а~толь\-ко~15 их видов (по пять видов для каж\-дой из трех 
сред).
    
    На рисунке приведены надписи только для пяти видов объектов цифровой 
среды, полученных при детализации циф\-ро\-вой информации. Под ними 
следующие пять видов объектов сенсорно вос\-при\-ни\-ма\-емой среды (без 
надписей на рисунке), полученные при детализации знаковой информации~--- 
это тексты в~каж\-дой из пяти видов знаковых сис\-тем. И~еще ниже пять видов 
объектов ментальной среды без надписей~--- это пять сфер знаний, для 
пред\-став\-ле\-ния которых в~сенсорно вос\-при\-ни\-ма\-емой среде применяются 
соответственно естественные, образные, 
 естест\-вен\-но-язы\-ко\-в$\acute{\mbox{ы}}$е, вер\-баль\-но-не\-сло\-вес\-ные 
или формализованные знаковые сис\-те\-мы. Отметим, что эти сферы могут 
пересекаться, так как существует такое знание, которое допускает его 
пред\-став\-ле\-ние в~сенсорно вос\-при\-ни\-ма\-емой среде с~помощью разных видов 
знаковых сис\-тем.

\vspace*{-6pt}
    
\section{Заключение}

\vspace*{-3pt}

    Описание трех верхних уровней классификации объектов, относящейся 
к~пятому компоненту (классификации) предлагаемой научной парадигмы, и~их 
визуализация на рисунке представляют собой только отдельные фрагменты 
этого компонента. Наполнение каждого из уровней классификации, 
естественно, не претендует на полноту. Но и~такое первоначальное описание 
классификации объектов уже позволяет отчасти ответить на первые два из трех 
вопросов Т.~Куна.
    
    На первый вопрос (Из каких основных сущностей состоит предметная 
область информатики?) предлагается следующий ответ:
    
    <<Все основные сущности сначала делятся на два глобальных класса: 
объекты и~их трансформации. Для каждого класса формируется своя 
классификация, и~они в~совокупности специфицируют основные сущности 
предметной области информатики>>.
    
    Ответ на второй вопрос (Какие трансформации описывают взаимодействие 
сущностей друг с~другом и~какие сущности взаимодействуют с~органами 
чувств?) состоит из сле\-ду\-ющих двух положений:
    \begin{enumerate}[(1)]
    \item с~органами чувств человека взаимодействуют объекты сенсорно 
воспринимаемой среды предметной области информатики;\\[-14pt]
    \item виды трансформаций объектов представлены в~классификации, 
первоначальное описание которой приведено в~работе~\cite{15-zac}.
    \end{enumerate}
    
  В~настоящее время создание парадигмы информатики находится на 
первоначальной стадии,\linebreak\vspace*{-12pt}

\pagebreak

\noindent
 на которой формируется только пятый из восьми ее 
компонентов~--- классификации информатики. После завершения создания 
первой версии компонента планируется сформировать второй и~третий 
компоненты парадигмы (предмет и~те методы изучения, которые свойственны 
именно информатике, включая метод  
ин\-фор\-ма\-ци\-он\-но-ма\-те\-ма\-ти\-че\-ских трансформаций~\cite{32-zac}), 
описав второй, третий и~пятый компоненты в~\textit{единой системе 
терминов} как шестого компонента парадигмы.
  
  
  В~заключение отметим, что сопоставление предлагаемого подхода 
к~построению трех верхних уровней классификации объектов предметной 
области с~ранее использованными подходами позволяет выдвинуть  
сле\-ду\-ющую гипотезу: для создания научной парадигмы информатики, 
охватывающей широкий спектр информационных и~компьютерных наук, 
необходимо выделять среды в~ее предметной области, учитывать увеличение 
числа сред в~будущем, определять и~объекты каждой среды, и~виды их 
трансформаций.

\vspace*{-6pt}
  
{\small\frenchspacing
 {\baselineskip=10.6pt
 %\addcontentsline{toc}{section}{References}
 \begin{thebibliography}{99}
 
 \vspace*{-3pt}
 
 \bibitem{3-zac} %1
\Au{Nygaard K.} Program development as a~social activity~// Information processing~/ Ed. H.-J.~Kugler.~--- North Holland: Elsevier 
Science Publs.\ B.\,V., IFIP, 1986. P.~189--198.
\bibitem{4-zac} %2
\Au{Harel D.} Algorithmics: The spirit of computing.~--- Reading, MA, USA: Addison-Wesley, 
1987. 514~p.

\bibitem{2-zac} %3
\Au{Колин К.\,К.} Становление информатики как фундаментальной науки и~комплексной 
научной проблемы~// Сис\-те\-мы и~средства информатики, 2006. Т.~16. №\,3. С.~7--58.

\bibitem{1-zac} %4
\Au{Гиляревский Р.\,С.} Информатика как наука об информации~// Системы и~средства 
информатики, 2006. Т.~16. №\,3.  
С.~59--87.

\bibitem{5-zac} %5
\Au{Denning P.} Computing is a natural science~// Commun. ACM, 2007. Vol.~50. Iss.~7. 
P.~13--18. doi: 
10.1145/ 1272516.1272529.
\bibitem{6-zac}
\Au{Denning P., Rosenbloom~P.} Computing: The fourth great domain of science~// 
Commun. ACM, 2009. Vol.~52. Iss.~9. P.~27--29. doi: 10.1145/1562164.1562176.
\bibitem{7-zac}
\Au{Rosenbloom P.\,S.} On computing: The fourth great scientific domain.~--- Cambridge, MA, USA: MIT 
Press, 2013. 307~p.
\bibitem{8-zac}
\Au{Caspersen M.\,E., Gal-Ezer~J., McGettrick~A., Nardelli~E.} Informatics as a~fundamental 
discipline for the 21st century~// Commun. ACM, 2019. Vol.~62. Iss.~4. P.~58--63. doi: 10.1145/3310330.
\bibitem{9-zac}
\Au{Tedre M., Pajunen~J.} Grand theories or design guidelines? Perspectives on the role of theory 
in computing education research~// ACM T. Comput. Educ., 2022. Vol.~23. Iss.~1. P.~1--20. 
doi: 10.1145/3487049.
\bibitem{10-zac}
\Au{Кун Т.} Структура научных революций~/ Пер. c англ.~--- М.: Прогресс, 1977. 302~с. 
(\Au{Kuhn~T.} The structure of scientific revolutions.~--- Chicago, IL, USA: University of Chicago Press, 
1962. 264~p.).
\bibitem{11-zac}
\Au{Fincher S., Tenenberg~J.} Using theory to inform capacity-building: Bootstrapping 
communities of practice in computer science education research~// J.~Eng. Educ., 2006. 
Vol.~95. Iss.~4. P.~265--277. doi: 
10.1002/j.2168-9830.2006.tb00902.x.
\bibitem{12-zac}
\Au{Соломоник А.\,Б.} Опыт современной теории познания.~--- СПб: Алетейя, 2019. 
232~с.
\bibitem{13-zac}
\Au{Solomonick A.\,B.} The modern theory of cognition.~--- Newcastle, U.K.: Cambridge 
Scholars Publishing, 2021. 242~p.
\bibitem{14-zac}
\Au{Соломоник А.\,Б.} Парадигма семиотики.~--- Минск: МЕТ, 2006. 335~с.
\bibitem{15-zac}
\Au{Зацман И.\,М.} Научная парадигма информатики: классификация трансформаций 
объектов предметной области~// Системы и~средства информатики, 2023. Т.~33. №\,4.  
С.~126--138. doi: 10.14357/08696527230412. EDN: ZIKUWO.
\bibitem{16-zac}
\Au{Ackoff R.} From data to wisdom~// J.~Applied Systems Analysis, 1989. Vol.~16. P.~3--9.
\bibitem{17-zac}
\Au{Rowley J.} The wisdom hierarchy: Representations of the DIKW hierarchy~// J.~Inf. 
Sci., 2007. Vol.~33. Iss.~2. P.~163--180. doi: 10.1177/0165551506070706.
\bibitem{18-zac}
\Au{Frick$\acute{\mbox{e}}$ M.\,H.} Data--Information--Knowledge--Wisdom (DIKW) 
pyramid, framework, continuum~// Encyclopedia of Big Data~/ Eds. L.~Schintler, 
 C.~McNeely.~--- Cham: Springer, 2018. 4~p. doi: 10.1007/978-3-319-32001-4\_\mbox{331-1}.
\bibitem{19-zac}
\Au{Denning P., Freeman~P.} Computing's paradigm~// Commun. ACM, 2009. 
Vol.~52. Iss.~12. P.~28--30.  doi: 10.1145/ 1610252.1610265.

\bibitem{23-zac} %20
\Au{Farradane J.} Knowledge, information, and information science~// J.~Inf. Sci., 
1980. Vol.~2. Iss.~2. P.~75--80. doi: 10.1177/01655515800020020.

\bibitem{21-zac} %21
\Au{Шрейдер Ю.\,А.} Информация и~знание~// Системная концепция информационных 
процессов.~--- М.: ВНИИСИ, 1988. С.~47--52.
\bibitem{22-zac} %22
\Au{Ingwersen P.} Information and information science~// Encyclopedia of library and 
information science~/ Eds.\ J.\,D.~McDonald, M.~Levine-Clark.~--- New York, NY, USA: Marcel Dekker Inc., 1995.   Vol.~56. Sup.~19.
P.~137--174.

\bibitem{24-zac} %23
\Au{Hjorland B.} Library and information science: Practice, theory, and philosophical basis~// 
Inform. Process. Manag., 2000. Vol.~36. Iss.~3. P.~501--531. doi: 10.1016/S0306-4573(99)00038-2.

\bibitem{20-zac} %24
Информатика как наука об информации: информационный, документальный, 
технологический, экономический, социальный и~организационный аспекты~/ Под ред. 
Р.\,С.~Гиляревского.~--- М.: Фаир-Пресс, 2006. 592~с.
\bibitem{25-zac}
\Au{Зацман И.\,М.} Теоретические основания компьютерного образования: среды 
предметной области информатики как основание классификации ее объектов~// Системы 
и~средства информатики, 2022. Т.~32. №\,4. С.~77--89. doi: 10.14357/08696527220408. EDN: 
SAUWDF.
\bibitem{26-zac}
\Au{Зацман И.\,М.} О научной парадигме информатики: верхний уровень классификации объектов ее 
предметной об\-ласти~// Информатика и~её применения, 2022. Т.~16. Вып.~4. С.~73--79. doi: 
10.14357/ 19922264220411. EDN: \mbox{XZNKVI}.

\bibitem{27-zac}
\Au{Зацман И.\,М.} Интерфейсы третьего порядка в~информатике~// Информатика и~её применения, 
2019. Т.~13. Вып.~3. С.~82--89. doi: 10.14357/19922264190312. EDN: EHRQLF.

\bibitem{28-zac}
\Au{Зацман И.\,М.} Таблица интерфейсов информатики как  
ин\-фор\-ма\-ци\-он\-но-компью\-тер\-ной науки~// На\-уч\-но-тех\-ни\-че\-ская информация. 
Сер.~1: Организация и~методика информационной работы, 2014. №\,11. С.~1--15. EDN: 
WEPZSZ.
\bibitem{29-zac}
\Au{Zatsman I.} Digital spiral model of knowledge creation and encoding its dynamics~// 18th 
Forum (International) on Knowledge Asset Dynamics Proceedings.~--- Matera, Italy: Arts for 
Business Institute, 2023. P.~581--596.
\bibitem{30-zac}
\Au{Nissen M.\,E.} Harnessing knowledge dynamics: Principled organizational knowing \& 
learning.~--- London: IRM Press, 2006. 278~p.
\bibitem{31-zac}
\Au{Соломоник А.\,Б.} Философия знаковых систем и~язык.~--- М.: ЛКИ, 2011. 408~с.
\bibitem{32-zac}
\Au{Вакуленко В.\,В., Зацман~И.\,М.} Формализованное описание статистической обработки 
информации в~базах данных~// Информатика и~её применения, 2023. Т.~17. Вып.~3.  
С.~93--99. doi: 10.14357/19922264230313. EDN: TYCAEX.
\end{thebibliography}

 }
 }

\end{multicols}

\vspace*{-10pt}

\hfill{\small\textit{Поступила в~редакцию 09.10.23}}

\vspace*{6pt}

%\pagebreak

%\newpage

%\vspace*{-28pt}

\hrule

\vspace*{2pt}

\hrule



\def\tit{SCIENTIFIC PARADIGM OF~INFORMATICS:\\ CLASSIFICATION OF~DOMAIN OBJECTS\\[-5pt]}


\def\titkol{Scientific paradigm of~informatics: Classification of~domain objects}


\def\aut{I.\,M.~Zatsman}

\def\autkol{I.\,M.~Zatsman}

\titel{\tit}{\aut}{\autkol}{\titkol}

\vspace*{-16pt}


\noindent
Federal Research Center ``Computer Science and Control'' of the Russian Academy 
of Sciences, 44-2~Vavilov Str., Moscow 119333, Russian Federation


\def\leftfootline{\small{\textbf{\thepage}
\hfill INFORMATIKA I EE PRIMENENIYA~--- INFORMATICS AND
APPLICATIONS\ \ \ 2023\ \ \ volume~17\ \ \ issue\ 4}
}%
 \def\rightfootline{\small{INFORMATIKA I EE PRIMENENIYA~---
INFORMATICS AND APPLICATIONS\ \ \ 2023\ \ \ volume~17\ \ \ issue\ 4
\hfill \textbf{\thepage}}}

\vspace*{2pt}


\Abste{A description is given of the three top levels of classification of domain's objects of informatics 
which is positioned as an integral part of the system of scientific knowledge that covers a~wide range of 
information and computer sciences. With such positioning, the boundaries of the domain expand 
significantly and largely correspond to the concept of polyadic computing by Paul Rosenbloom. All entities 
of informatics in the proposed scientific paradigm are divided into two global classes: objects and their 
transformations. For each class, in the process of creating the paradigm, its own classification is constructed. 
The paradigm's creation began with the formation of these classifications. The paper discusses the three top 
levels of classification of domain's objects of informatics. The basis for constructing the first (the highest) 
level is the division of the domain of informatics into the media: mental, sensory, digital, and 
a~number of other media. The basis for constructing the second level of objects' classification is the division 
of sensory perceived objects of informatics into data and sign information which is the outcome of 
transformation of human cognitive structures into a sign form. The basis for constructing the third level of 
classification of objects is the typology of sign systems by A.~Solomonick. The aim of the paper is to 
describe the approach to constructing the three top levels of classification of domain's objects of informatics 
and to compare it with the previously used approaches to describing its subject domain. Also, based on the 
proposed approach, the answers to those questions of Thomas Kuhn about the basic entities of the subject 
domain which should contain the paradigm of any science, not just informatics, are partly formulated.}

\KWE{scientific paradigm; classification of domain's objects of informatics; basis of classification; subject 
domain media}




  \DOI{10.14357/19922264230413}{FIUQAT}


\vspace*{-16pt}

\Ack

\vspace*{-4pt}

\noindent
The research was carried out using the infrastructure of the Shared Research Facilities ``High Performance 
Computing and Big Data'' (CKP ``Informatics'') of FRC CSC RAS (Moscow).
%\vspace*{6pt}

  \begin{multicols}{2}

\renewcommand{\bibname}{\protect\rmfamily References}
%\renewcommand{\bibname}{\large\protect\rm References}

{\small\frenchspacing
 {\baselineskip=10.8pt
 \addcontentsline{toc}{section}{References}
 \begin{thebibliography}{99} 
 
 \bibitem{3-zac-1} %1
\Aue{Nygaard, K.} 1986. Program development as a social activity. \textit{Information processing}. Ed. H.-J.~Kugler. North Holland: Elsevier Science 
Publs.\ B.\,V., IFIP. 189--198.
\bibitem{4-zac-1} %2
\Aue{Harel, D.} 1987. \textit{Algorithmics: The spirit of computing}. Reading, MA: Addison-Wesley. 
514~p.

\bibitem{2-zac-1} %3
\Aue{Kolin, K.\,K.} 2006. Stanovleniye informatiki kak fundamental'noy nauki i~kompleksnoy nauchnoy 
problemy [Becoming of informatics as fundamental science and the complex scientific problem]. 
\textit{Sistemy i~Sredstva Informatiki~--- Systems and Means of Informatics}  3:7--58.
\bibitem{1-zac-1} %4
\Aue{Gilyarevskiy, R.\,S.} 2006. Informatika kak nauka ob informatsii [Informatics as science of 
information]. \textit{Sistemy i~Sredstva Informatiki~--- Systems and Means of Informatics}  3:59--87.

\bibitem{5-zac-1}
\Aue{Denning, P.} 2007. Computing is a natural science. \textit{Commun. ACM} 50(7):13--18. doi: 
10.1145/1272516.1272529.
\bibitem{6-zac-1}
\Aue{Denning, P., and P.~Rosenbloom.} 2009. Computing: The fourth great domain of science. 
\textit{Commun. ACM} 52(9):27--29. doi: 10.1145/1562164.1562176.
\bibitem{7-zac-1}
\Aue{Rosenbloom, P.\,S.} 2013. \textit{On computing: The fourth great scientific domain}. Cambridge, 
MA: MIT Press. 307~p. 
\bibitem{8-zac-1}
\Aue{Caspersen, M.\,E., J.~Gal-Ezer, A.~McGettrick, and E.~Nardelli.} 2019. Informatics as a fundamental 
discipline for the 21st century. \textit{Commun. ACM} 62(4):58--63. doi: 10.1145/3310330.
\bibitem{9-zac-1}
\Aue{Tedre, M., and J.~Pajunen.} 2022. Grand theories or design guidelines? Perspectives on the role of 
theory in computing education research. \textit{ACM T. Comput. Educ.} 23(1):1--20. doi: 10.1145/3487049.
\bibitem{10-zac-1}
\Aue{Kuhn, T.} 1962. \textit{The structure of scientific revolutions}. Chicago, IL: University of Chicago 
Press. 264~p.
\bibitem{11-zac-1}
\Aue{Fincher, S., and J.~Tenenberg.} 2006. Using theory to inform capacity-building: Bootstrapping 
communities of practice in computer science education research. \textit{J.~Eng. Educ.} 95(4):265--277. doi: 
10.1002/j.2168-9830.2006.tb00902.x.
\bibitem{12-zac-1}
\Aue{Solomonick, A.\,B.} 2019. \textit{Opyt sovremennoy filosofii po\-zna\-niya} [Experience of modern 
philosophy of cognition]. Saint Petersburg: Aletheia. 232~p.
\bibitem{13-zac-1}
\Aue{Solomonick, A.\,B.} 2021. \textit{The modern theory of cognition}. Newcastle, U.K.: Cambridge 
Scholars Publishing. 242~p.
\bibitem{14-zac-1}
\Aue{Solomonick, A.} 2006. \textit{Paradigma semiotiki} [The paradigm of semiotics]. Minsk: MET Publs. 
335~p.
\bibitem{15-zac-1}
\Aue{Zatsman, I.\,M.} 2023. Nauchnaya paradigma informatiki: klassifikatsiya transformatsiy ob''ektov 
predmetnoy oblasti [Scientific paradigm of informatics: Transformations' classification of domain objects]. 
\textit{Sistemy i Sredstva Informatiki~--- Systems and Means of Informatics} 33(4):126--138. doi: 
10.14357/08696527230412. EDN: ZIKUWO.
\bibitem{16-zac-1}
\Aue{Ackoff, R.} 1989. From data to wisdom. \textit{J.~Applied Systems Analysis} 16(1):3--9.
\bibitem{17-zac-1}
\Aue{Rowley, J.} 2007. The wisdom hierarchy: Representations of the DIKW hierarchy. \textit{J.~Inf. Sci.} 
33(2):163--180. doi: 10.1177/0165551506070706.
\bibitem{18-zac-1}
\Aue{Frick$\acute{\mbox{e}}$, M.\,H.} 2018. Data--Information--Knowledge--Wisdom (DIKW) pyramid, framework, continuum. 
\textit{Encyclopedia of Big Data}. Eds. L.~Schintler and C.~\mbox{McNeely}. Cham: Springer. 4~p. doi: 
10.1007/978-3-319-32001-4\_331-1.
\bibitem{19-zac-1}
\Aue{Denning, P., and P.~Freeman.} 2009. Computing's paradigm. \textit{Commun. ACM} 52(12):28--30. 
doi: 10.1145/ 1610252.1610265.

\bibitem{23-zac-1} %20
\Aue{Farradane, J.} 1980. Knowledge, information, and information science. \textit{J.~Inf. Sci.} 
 2(2):75--80. doi: 10.1177/ 01655515800020020.

\bibitem{21-zac-1}
\Aue{Shreyder, Yu.\,A.} 1988. Informatsiya i~znanie [Information and knowledge]. \textit{Sistemnaya 
kontseptsiya in\-for\-ma\-tsi\-on\-nykh protsessov} [System concept of information processes]. Moscow: VNIISI.  
47--52.
\bibitem{22-zac-1}
\Aue{Ingwersen, P.} 1995. Information and information science. \textit{Encyclopedia of library and 
information science}. Eds.\ J.\,D.~McDonald and M.~Levine-Clark. New York, NY: Marcel Dekker Inc. 56(19):137--174.

\bibitem{24-zac-1} %23
\Aue{Hjorland, B.} 2000. Library and information science: Practice, theory, and philosophical basis. 
\textit{Inform. Process. Manag.} 36(3):501--531. doi: 10.1016/S0306-\mbox{4573(99)00038-2}.

\bibitem{20-zac-1} %24
Gilyarevskiy, R.\,S., ed. 2006. \textit{Informatika kak nauka ob informatsii: informatsionnyy, 
dokumental'nyy, tekhno-logicheskiy, ekonomicheskiy, sotsial'nyy i~organizatsionnyy aspekty} [Informatics as 
information science: Informational, documentary, technological, economic, social, and organizational 
dimensions]. Moscow: FAIR-PRESS. 592~p.

\bibitem{25-zac-1} %25
\Aue{Zatsman, I.\,M.} 2022. Teoreticheskie osnovaniya komp'yuternogo obrazovaniya: sredy predmetnoy 
oblasti informatiki kak osnovanie klassifikatsii ee ob''ektov [Theoretical foundations of digital education: 
Subject domain media of informatics as the base of its objects' classification]. \textit{Sistemy i~Sredstva 
Informatiki~--- Systems and Means of Informatics} 32(4):77--89. doi: 10.14357/ 08696527220408. EDN: 
SAUWDF.
\bibitem{26-zac-1} %26
\Aue{Zatsman, I.\,M.} 2022. O nauchnoy paradigme informatiki: verkhniy uroven' klassifikatsii ob''ektov 
ee predmetnoy oblasti [On the scientific paradigm of informatics: The classification high level of its 
objects].  \textit{Informatika i~ee Primeneniya~--- Inform. Appl.} 16(4):73--79. doi: 
10.14357/19922264220411. EDN: XZNKVI.
\bibitem{27-zac-1} %27
\Aue{Zatsman, I.\,M.} 2019. Interfeysy tret'ego poryadka v~informatike [Third-order interfaces in 
informatics].  \textit{Informatika i~ee Primeneniya~--- Inform. Appl.} 13(3):82--89. doi: 
10.14357/19922264190312. EDN: EHRQLF.
\bibitem{28-zac-1}
\Aue{Zatsman, I.} 2014. Table of interfaces of informatics as computer and information science. 
\textit{Scientific Technical Information Processing} 41(4):233--246. EDN: \mbox{WEPZSZ}.
\bibitem{29-zac-1}
\Aue{Zatsman, I.} 2023. Digital spiral model of knowledge creation and encoding its dynamics. \textit{18th 
Forum (International) on Knowledge Asset Dynamics Proceedings}. Matera, Italy: Arts for Business 
Institute.  581--596.
\bibitem{30-zac-1}
\Aue{Nissen, M.\,E.} 2006. \textit{Harnessing knowledge dynamics: Principled organizational knowing \& 
learning}. London: IRM Press. 278~p.
\bibitem{31-zac-1}
\Aue{Solomonick, A.\,B.} 2011. \textit{Filosofiya znakovykh sistem i~yazyk} [Philosophy of sign systems 
and language]. Moscow: LKI. 408~p.
\bibitem{32-zac-1}
\Aue{Vakulenko, V.\,V., and I.\,M.~Zatsman.} 2023. For\-ma\-li\-zo\-van\-noe opi\-sa\-nie statisticheskoy obrabotki 
in\-for\-ma\-tsii v~bazakh dannykh [Formalized description of statistical information processing in databases]. 
\textit{Informatika i~ee Primeneniya~--- Inform. Appl.} 17(3):93--99. doi: 10.14357/ 19922264230313. 
EDN: TYCAEX.

\end{thebibliography}

 }
 }

\end{multicols}

\vspace*{-8pt}

\hfill{\small\textit{Received October 9, 2023}} 

\vspace*{-18pt}

\Contrl

\vspace*{-4pt}

\noindent
\textbf{Zatsman Igor M.} (b.\ 1952)~--- Doctor of Science in technology, head of department, Federal 
Research Center ``Computer Science and Control'' of the Russian Academy of Sciences, 44-2~Vavilov Str., 
Moscow 119333, Russian Federation; \mbox{izatsman@yandex.ru}


\label{end\stat}

\renewcommand{\bibname}{\protect\rm Литература} 