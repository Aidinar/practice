\documentclass[10pt]{book}
\usepackage[utf8]{inputenc}

\usepackage{latexsym,amssymb,amsfonts,amsmath,amsxtra,dsfont,
indentfirst,shapepar,%fleqn,%
picinpar,shadow,floatflt,enumerate,multicol,colortbl,moreverb,cite,ipi}

\usepackage{rotating}
\usepackage{mathrsfs}
\usepackage[noend]{algorithmic}
\usepackage{ulem}
\usepackage{graphicx}
%\usepackage{algorithm2e}
\usepackage[linesnumbered,boxed,ruled]{algorithm2e}
%\usepackage{xypic}
\usepackage{oldgerm}
\usepackage{epic}
\usepackage{eepic}

\SetAlgorithmName{Algorithm}{алгоритм}{Список алгоритмов}

%из Дюковой

\newcommand{\algKeyword}[1]{{\bf #1}}
\newcommand{\Proc}[1]{\text{\tt #1}}
\def\CALL{\algKeyword{call}~}

\newenvironment{AlgProcedure}[1]
{
\small
\medskip
%    \hrule
\medskip
\algKeyword{PROCEDURE} #1
\begin{algorithmic}[1]}
{\end{algorithmic}
%    \hrule
\bigskip
}

\def\CALL{\algKeyword{call}~}

%конец для Дюковой

%\RequirePackage[ruled]{algorithm}


\input{epsf}

%\nofiles

%\includeonly{avtor}    %pdf
%\includeonly{podgot-rus-site,podgot-eng-site}  
%\includeonly{podgot-rus,podgot-eng}  
%\includeonly{ipi-ind} 
%\includeonly{index-17i}
%\includeonly{toc-rus, toc-en}
%\includeonly{toc-rus}
%\includeonly{toc-en} 
%\includeonly{popravka}



%\includeonly{rabinovich}     %pdf+авт+
%\includeonly{borisov}        %pdf+авт+
%\includeonly{stupnikov}      %pdf+авт+             
%\includeonly{zatsman}        %pdf+авт+
%\includeonly{shestakov}      %pdf+авт+
%\includeonly{lapko}          %pdf+авт повторно отпр
%\includeonly{inkova}         %pdf+авт+
%\includeonly{goncharov}      %pdf+авт+
%\includeonly{shnurkov}       %pdf+авт+
%\includeonly{bosov}          %pdf+авт+
%\includeonly{ivanova}        %pdf+авт повт отпр
%\includeonly{frenkel}        %pdf+авт+
%\includeonly{peshkova}       %pdf+авт+



%%%%%%%%%%%%%%%%%%%\includeonly{nekrolog-new}



%\includeonly{rekl}




\usepackage{acad}
%\usepackage{courier}
\usepackage{decor}
\usepackage{newton}
\usepackage{pragmatica}
\usepackage{zapfchan}
\usepackage{petrotex}
\usepackage{bm}                     % полужирные греческие буквы
\usepackage{upgreek}                % прямые греческие буквы \upalpha
\usepackage{eufrak}
\usepackage{verbatim}

\renewcommand{\bottomfraction}{0.99}
\renewcommand{\topfraction}{0.99}
\renewcommand{\textfraction}{0.01}

\setcounter{secnumdepth}{1} %здесь - 3 + chapter = 4

\arraycolsep=1.5pt

%\usepackage[pdftex]{graphicx}

%\usepackage{oz}

%NEW COMMANDS



\renewcommand*{\hm}[1]{#1\nobreak\discretionary{}%
            {\hbox{$\mathsurround=0pt #1$}}{}} %% Дублирует знаки операций
                               %при переносе в формуле (перед знаком, который
                               %надо продублировать ставится команда \hm)
                               
                               \newcommand{\PRB}{\begin{picture}(22.5,11)
      \spline(1,8)(4,10)(7,10.5)(10,11)(13,11)(16,10.5)(19,10)(22,8)
               \put(0,0){$P_{i-1}P_{t_{t-1}}$} \end{picture}}

\newcommand{\prb}{\begin{picture}(15.5,9)
      \spline(1,6)(3,8)(5,8.5)(7,9)(9,9)(11,8.5)(13,8)(15,6)
               \put(0,0){$PP_t$} \end{picture}}
               
                 \newcommand{\PRDN}{\begin{picture}(40,11)
      \spline(4,11.5)(7,10.5)(12,10)(16,9)(20,9)(24,10)(29,10.5)(32,11.5)
               \put(0,0){$P_{i-1}P_{t_{t-1}}$} \end{picture}}

\newcommand{\prdn}{\begin{picture}(18,11)
      \spline(3,10.5)(4,10)(6,9)(8,8.5)(10,8.5)(12,9)(14,10)(15,10.5)
               \put(0,0){$PP_t$} \end{picture}}




%\newcommand{\endproof}{\hfill$\Box$}
%\renewcommand{\r}{\mathbb{R}}
%\newcommand{\I}{{\rm I\hspace{-0.7mm}I}}
%\newcommand{\Ikl}{{\tt{1}}\hspace*{-1.44mm}\mathtt{1}}
%\newcommand{\Ik}{\mbox{{\small \tt {1}}\hspace{-1.3mm}{\tt 1}}}
\newcommand{\Ik}{\mbox{{{\tt 1}}\hspace{-1.3mm}{\sf 1}}}
\newcommand{\argmin}{\mathop{\mathrm{arg}\,\mathrm{min}}}
\newcommand{\argmax}{\mathop{\mathrm{arg}\,\mathrm{max}}}
%\newcommand{\capr}{\mathop{\cap\,}}
%\newcommand{\cupr}{\mathop{\cup\,}}
%\def\argmin{\mathop{arg\,min}}

\def\vrp{\varphi}
\def\prt{\partial}
\def\mm{{\sf M}}
\def\modnop#1{\mathop{#1}\limits_{n}}
\def\eam{\mathbin{{\mathop{=}\limits^{\mathrm{def}}}}}
\def\dey#1#2{#1 (#2)}
\def\deyc#1#2{#1 \cdot  #2}
\def\ra#1{\;\mathop{\to}\limits^{#1}\;}
\def\raz#1{\;\mathop{\longrightarrow}\limits^{\!\!\!#1}\;}
\def\ral#1{\;\mathop{\longrightarrow}\limits^{#1}\;}





\newcommand{\il}[2]{\int\limits_{#1}^{#2}}%интеграл с пределами #1 и #2

\def\sm2{\mathop {\sum\limits^{n^\Theta}\sum\limits^{n^\Theta}}}
\def\sss{\sum\limits}
\def\tr{,\,\ldots\,,\,}
\def\rk{\right]}
\def\lk{\left[}
\def\rf{\right\}}
\def\lf{\left\{}
\def\lv{\,\left\vert}
\def\rv{\right\vert\,}
\def\iii{\int\limits}
\def\iin{\int\limits_{-\infty}^\infty}
\def\rrv{\right\vert}


\def\ee{{\cal E}}
\def\ww{{\cal W}}
\def\yy{{\cal Y}}
\def\vv{{\cal V}}

\newcommand{\R}{\mathbb R}
\newcommand{\E}{\mathbb E}
\newcommand{\N}{\mathbb N}
\newcommand{\T}{\mathbb{T}}
\newcommand{\Z}{\mathbb{Z}}

\renewcommand{\P}{\mathbb{P}}

\newcommand{\Nor}{\mathcal{N}}

\newcommand{\h}{{\bf H}}
\newcommand{\p}{{\sf P}}  % вероятность
\newcommand{\e}{{\sf E}}  % мат. ожидание
\newcommand{\D}{{\sf D}}  % дисперсия



\newcommand{\vw}{{\mathbf w}}
\newcommand{\vp}{{\mathbf p}}
\newcommand{\vz}{{\mathbf z}}
\newcommand{\vx}{{\mathbf x}}
\newcommand{\vf}{{\mathbf f}}
\newcommand{\F}{{\mathcal F}}
\def\ap{{\mathrm{ЭР}}}
\newcommand{\ud}{\Delta_n} %uniform ditance
\newcommand{\nud}{\Delta_n(x)}
%\renewcommand{\Re}{\mathrm{Re}\,}

\newcommand{\abs}[1]{\left\vert#1\right\vert}

\newcommand{\norm}[1]{\left\Vert#1\right\Vert}
\def\da{(\Delta_t,A)}

\newcommand{\corr}{\mathrm{corr}}

\newcommand{\cov}{\mathrm{cov}}
\newcommand{\Expect}{\mathbb{E}}

\def\w{\omega}
\def\W{\Omega}


\def\inh{\int\limits_{nh}^{(n+1)h}}

\def\sumin{\sum_{i=1}^N}


\def\bxt{(Y,t)}
\def\xt{(y,t)}

\def\ovth{{\fr{\tau-nh}{h}}}
\def\ov{\overline}
\def\tm{\tilde m}
\def\tl{\tilde\lambda}
\def\tB{\widetilde B}
\def\tb{\tilde b}
\def\ld{\ldots}
\def\cd{\cdots}


\DeclareMathOperator{\sign}{sign}



\newcommand{\g}{\mbox{\textit{g}}}

\renewcommand{\la}{\lambda}
\newcommand{\si}{\sigma}
\newcommand{\eps}{\varepsilon}
\newcommand{\alp}{\alpha}

\newcommand{\pto}{\stackrel{P}{\longrightarrow}} % сходимость по веpоятности

\newcommand{\eqd}{\stackrel{\mathrm{d}}{=}} % равенство по pаспpеделению
\newcommand{\eqdelta}{\stackrel{\triangle}{=}} % равенство по pаспpеделению

\def\be#1{\begin{equation}\label{#1}}
\def\ee{\end{equation}}
\def\re#1{(\ref{#1})}

\def\bn{\begin{enumerate}}
\def\en{\end{enumerate}}
\def\bi{\begin{itemize}}
\def\ei{\end{itemize}}
%\def\i{\item}

%\newcommand{\kp}{\kappa}
%\def\Q{{\cal Q}} \def\H{{\cal H}}
%\newcommand{\bet}{\beta_{2+\delta}}




%\renewcommand{\baselinestretch}{1.2}

%\pagestyle{myheadings}

\setlength{\textwidth}{167mm}      % 122mm
\setlength{\textheight}{658pt}
%\setlength{\textheight}{635.6pt}
\setlength{\columnsep}{4.5mm}

\setcounter{secnumdepth}{4}

%\addtolength{\headheight}{2pt}
%\addtolength{\headsep}{-2mm}

\addtolength{\topmargin}{-7mm}  % for printing


%\hoffset=-30mm  % From Yap
\hoffset=-23mm  % From Acrobat

%\voffset=0mm % From Yap
\voffset=-5mm   % From Acrobat

%\addtolength{\evensidemargin}{-2.5mm} % for printing
%\addtolength{\oddsidemargin}{2.5mm}  % for printing

\addtolength{\evensidemargin}{-12mm} % for printing
\addtolength{\oddsidemargin}{8mm}  % for printing

%\renewcommand{\thefootnote}{\fnsymbol{footnote}}
%\renewcommand{\thefootnote}{\arabic{footnote}}
\renewcommand{\figurename}{\protect\bf Рис.}
\renewcommand{\tablename}{\protect\bf Таблица}

\newcommand{\Caption}[1]{\caption{\protect\small %\baselineskip=2.5ex
#1}}

\renewcommand{\thefigure}{\arabic{figure}}
\renewcommand{\thetable}{\arabic{table}}
\renewcommand{\theequation}{\arabic{equation}}
\renewcommand{\thesection}{\arabic{section}}

\renewcommand{\contentsname}{СОДЕРЖАНИЕ}
\newcommand{\fr}[2]{\displaystyle\frac{\displaystyle #1\mathstrut}{\displaystyle #2\mathstrut}}

%\renewcommand{\thefootnote}{\fnsymbol{footnote}}
%\newcommand{\g}{\mbox{\textit{g}}}

%\newcommand{\Caption}[1]{\caption{\protect\small\baselineskip=2ex #1}}
\newcounter{razdel}
\setcounter{razdel}{0}

\def\god{2023}
\def\tom{17}
\def\vyp{4}


\newcommand{\titel}[4]{%
\

\vspace*{5pt}

\ifodd\therazdel {\raggedright\noindent\Large\textrm\textbf
 \lineskip .75em
  \baselineskip=3.2ex #1 \par}
\vskip 1em {\noindent\large\textrm\textbf #2 \par}
\addcontentsline{toc}{subsection}{{\textrm\textbf #1}\protect\newline #2}
\def\rightheadline{\underline{\noindent\hbox to \textwidth{\hfill\small\textrm{#4}
%\hfill \large\bf\thepage
}}}
\def\leftheadline{\underline{\noindent\parbox{\textwidth}{
%\raggedleft\large\bf\thepage \hfill
\small\textit{#3}\hfill}}}
\def\leftfootline{\small{\textbf{\thepage}
\hfill ИНФОРМАТИКА И ЕЁ ПРИМЕНЕНИЯ\ \ \ том~\tom\ \ \ выпуск~\vyp\ \ \ \god}
}%
 \def\rightfootline{\small{ИНФОРМАТИКА И ЕЁ ПРИМЕНЕНИЯ\ \ \ том~\tom\ \ \ выпуск~\vyp\ \ \ \god
\hfill \textbf{\thepage}}}
\vskip 2em \setcounter{figure}{0}
\setcounter{table}{0}
\setcounter{equation}{0}
\setcounter{section}{0}
\setcounter{subsection}{0}
\setcounter{subsubsection}{0}
\setcounter{footnote}{0}
\setcounter{razdel}{0}
%\end{flushleft}
\else {
 \raggedright\noindent\Large\textrm\textbf
 \lineskip .75em
\baselineskip=3.2ex #1 \par} \vskip 1em
%\begin{flushleft}
{\noindent\large\textrm\textbf #2 \par}
\addcontentsline{toc}{subsection}{{\textrm\textbf #1}\protect\newline #2}
\def\rightheadline{\underline{\noindent\hbox to \textwidth{\hfill\small\textrm{#4}
%\hfill \large\bf\thepage
}}}
\def\leftheadline{\underline{\noindent\parbox{\textwidth}{%\raggedleft\large\bf\thepage \hfill
\small\textit{#3}\hfill}}}
\def\leftfootline{\small{\textbf{\thepage}
\hfill ИНФОРМАТИКА И ЕЁ ПРИМЕНЕНИЯ\ \ \ том~\tom\ \ \ выпуск~\vyp\ \ \ \god}
}%
 \def\rightfootline{\small{ИНФОРМАТИКА И ЕЁ ПРИМЕНЕНИЯ\ \ \ том~17\ \ \ выпуск~\vyp\ \ \ 2023
\hfill \textbf{\thepage}}} \vskip 2em \setcounter{figure}{0}
\setcounter{table}{0} \setcounter{equation}{0} \setcounter{section}{0}
\setcounter{subsection}{0} \setcounter{subsubsection}{0}
\setcounter{footnote}{0}
%\end{flushleft}
\fi}

\newcommand{\titelr}[2]{%
\

\vspace*{5pt}

\ifodd\therazdel {\raggedright\noindent%\Large\textrm\textbf
 \lineskip .75em
  \baselineskip=3.2ex #1 \par}
\vskip 1em {\noindent\normalsize\textrm\textbf #2 \par}
\else {
 \raggedright\noindent\Large\textrm\textbf
 \lineskip .75em
\baselineskip=3.2ex #1 \par} \vskip 1em
%\begin{flushleft}
{\noindent\large\textrm\textbf #2 \par
%\noindent\normalsize\textrm\textbf #2 \par
} \fi}

\newcommand{\titele}[5]{%
\

%\vspace*{5pt}

\ifodd\therazdel {\raggedright\noindent\large
\textrm\textbf
 \lineskip .75em
%  \baselineskip=3.2ex
#1 \par}
\vskip .5em {\noindent\large\textrm\textbf #2 \par}
\vskip .5em
 {\noindent\textrm #3 \par}
\addcontentsline{toc}{subsection}{{\textrm\textbf #1}\protect\newline #2}
\def\rightheadline{\underline{\noindent\hbox to \textwidth{\hfill\small\textrm{#4}
%\hfill \large\bf\thepage
}}}
\def\leftheadline{\underline{\noindent\parbox{\textwidth}{
%\raggedleft\large\bf\thepage \hfill
\small\textrm{#5}\hfill}}}
\def\leftfootline{\small{\textbf{\thepage}
\hfill ИНФОРМАТИКА И ЕЁ ПРИМЕНЕНИЯ\ \ \ том~17\ \ \ выпуск~4\ \ \ 2023}
}%
 \def\rightfootline{\small{ИНФОРМАТИКА И ЕЁ ПРИМЕНЕНИЯ\ \ \ том~17\ \ \ выпуск~4\ \ \ 2023
\hfill \textbf{\thepage}}} \vskip 1em \setcounter{figure}{0}
\setcounter{table}{0} \setcounter{equation}{0} \setcounter{section}{0}
\setcounter{subsection}{0} \setcounter{subsubsection}{0}
\setcounter{footnote}{0} \setcounter{razdel}{0}
%\end{flushleft}
\else {
 \raggedright\noindent\large
 \textrm\textbf
 \lineskip .75em
%\baselineskip=3.2ex
#1 \par} \vskip .5em
%\begin{flushleft}
{\noindent\large\textrm\textbf #2 \par} \vskip .5em
 {\noindent\textrm #3 \par}
\addcontentsline{toc}{subsection}{{\textrm\textbf #1}\protect\newline #2}
\def\rightheadline{\underline{\noindent\hbox to \textwidth{\hfill\small\textrm{#4}
%\hfill \large\bf\thepage
}}}
\def\leftheadline{\underline{\noindent\parbox{\textwidth}{%\raggedleft\large\bf\thepage \hfill
\small\textrm{#5}\hfill}}}
\def\leftfootline{\small{\textbf{\thepage}
\hfill ИНФОРМАТИКА И ЕЁ ПРИМЕНЕНИЯ\ \ \ том~17\ \ \ выпуск~4\ \ \ 2023}
}%
 \def\rightfootline{\small{ИНФОРМАТИКА И ЕЁ ПРИМЕНЕНИЯ\ \ \ том~17\ \ \ выпуск~4\ \ \ 2023
\hfill \textbf{\thepage}}} \vskip 1em \setcounter{figure}{0}
\setcounter{table}{0} \setcounter{equation}{0} \setcounter{section}{0}
\setcounter{subsection}{0} \setcounter{subsubsection}{0}
\setcounter{footnote}{0}
%\end{flushleft}
\fi}

\def\Abst#1{
\begin{center}\small\nwt
\parbox{150mm}{%\baselineskip=2.5ex
\textbf{Аннотация:}\ \
%\hspace*{\parindent}
#1}
\end{center}}
\def\Abste#1{
\begin{center}\small\nwt
\parbox{150mm}{%\baselineskip=2.5ex
\textbf{Abstract:}\ \
%\hspace*{\parindent}
#1}
\end{center}}

%\def\DOI#1{
%\begin{center}\small\nwt
%\parbox{150mm}{%\baselineskip=2.5ex
%\textbf{DOI:}\ \
%%\hspace*{\parindent}
%#1}
%\end{center}}

\def\Abstend#1{
\begin{center}\small\nwt
\parbox{150mm}{%\baselineskip=2.5ex
%\hspace*{\parindent}
#1}
\end{center}}

\newcommand{\DOI}[2]{\begin{center}\small\nwt
\parbox{150mm}{%\baselineskip=2.5ex
\textbf{DOI:}\ \
%\hspace*{\parindent}
#1 \hfill \textbf{EDN:}\ \
#2}
\end{center}}




\def\KW#1{
\begin{center}\small\nwt
\parbox{150mm}{%\baselineskip=2.5ex
\textbf{Ключевые слова:}\ \ #1}
\end{center}}

\def\KWE#1{
\begin{center}\small\nwt
\parbox{150mm}{%\baselineskip=2.5ex
\textbf{Keywords:}\ \ #1}
\end{center}}


\def\KWN#1{
%\begin{center}
%\small
%\parbox{150mm}\end{center}
}

\newcommand{\Avtors}[1]{%\smallskip
%\vspace*{.5pt}
\hangindent=23pt\noindent
%\nwt
{\bfseries#1}\
}


\renewcommand{\thesubsection}{\thesection.\arabic{subsection}\hspace*{-5pt}}
\renewcommand{\thesubsubsection}{\thesubsection\hspace*{5pt}.\arabic{subsubsection}\hspace*{-3pt}}

\newcommand{\Ack}{\section*{\protect\rmfamily Acknowledgments}\noindent}
\newcommand{\Contr}{\section*{\protect\rmfamily Contributors}\noindent}
\newcommand{\Contrl}{\section*{\protect\rmfamily Contributor}\noindent}

\makeindex


\begin{document}
\Rus

\nwt
%\ptb


%\renewcommand{\contentsname}{\protect\Large\bf Содержание}

\setcounter{tocdepth}{2}

%\tableofcontents

\renewcommand{\bibname}{\protect\rmfamily Литература}
  \def\Au#1{{\it #1}}
    \def\Aue#1{{#1}}

%\newcommand{\No}{№}
  \newcommand{\tg}{\,\mathrm{tg}\,}
    \newcommand{\ctg}{\,\mathrm{ctg}\,}
  \newcommand{\arctg}{\,\mathrm{arctg}\,}

\def\forallb{\mathop{\forall}}
\def\cupb{\mathop{\cup}}
\def\existsb{\mathop{\exists}}


\newpage
\addtocounter{razdel}{1}
%\def\razd{РЕГУЛИРУЕМЫЙ ЭЛЕКТРОПРИВОД ДЛЯ ЭЛЕКТРОЭНЕРГЕТИКИ}


\setcounter{page}{2}

%   { %\Large  
   { %\baselineskip=16.6pt
   
   \vspace*{-48pt}
   \begin{center}\LARGE
   \textit{Предисловие}
   \end{center}
   
   %\vspace*{2.5mm}
   
   \vspace*{25mm}
   
   \thispagestyle{empty}
   
   { %\small 

    
Вниманию читателей журнала <<Информатика и её применения>> предлагается 
очередной тематический выпуск <<Вероятностно-статистические методы и 
задачи информатики и информационных технологий>>. Предыдущие тематические 
выпуски журнала по данному направлению вышли в 2008~г.\ (т.~2, вып.~2), 
в 2009~г.\ (т.~3, вып.~3) и в 2010~г.\ (т.~4, вып.~2). 

Статьи, собранные в данном журнале, посвящены разработке новых вероятностно-статистических 
методов, ориентированных на применение к решению конкретных задач информатики и информационных 
технологий, а также~--- в ряде случаев~--- и других прикладных задач. Проблематика, охватываемая 
публикуемыми работами, развивается в рамках научного сотрудничества между Институтом проблем 
информатики Российской академии наук (ИПИ РАН) и Факультетом вычислительной математики и 
кибернетики Московского государственного университета им.\ М.\,В.~Ломоносова в ходе работ 
над совместными научными проектами (в том числе в рамках функционирования 
Научно-образовательного центра <<Вероятностно-статистические методы анализа рисков>>). 
Многие из авторов статей, включенных в данный номер журнала, являются активными участниками 
традиционного международного семинара по проблемам устойчивости стохастических моделей, 
руководимого В.\,М.~Золотаревым и В.\,Ю.~Королевым; регулярные сессии этого семинара 
проводятся под эгидой МГУ и ИПИ РАН (в 2011~г.\ указанный семинар проводится в октябре 
в Калининградской области РФ). 

Наряду с представителями ИПИ РАН и МГУ в число авторов данного выпуска журнала входят 
ученые из Научно-исследовательского института системных исследований РАН, Института 
проблем технологии микроэлектроники и особочистых материалов РАН, Института 
прикладных математических исследований Карельского НЦ РАН, Московского 
авиационного института, Вологодского государственного педагогического университета, 
НИИММ им.\ Н.\,Г.~Чеботарева, Казанского государственного университета, Дебреценского 
университета (Венгрия).

Несколько статей выпуска посвящено разработке и применению стохастических методов и 
информационных технологий для решения различных прикладных задач. В~работе В.\,Г.~Ушакова 
и О.\,В.~Шестакова рассмотрена задача определения вероятностных характеристик случайных 
функций по распределениям интегральных преобразований, возникающих в задачах эмиссионной 
томографии. В~статье Д.\,О.~Яковенко и М.\,А.~Целищева рассмотрены некоторые вопросы 
математической теории риска и предложен новый подход к диверсификации инвестиционных 
портфелей. Работа И.\,А.~Кудрявцевой и А.\,В.~Пантелеева посвящена построению и 
исследованию математической модели, описывающей динамику сильноионизованной плазмы. 
В~статье П.\,П.~Кольцова изучается качество работы ряда алгоритмов сегментации изображений. 
Статья А.\,Н.~Чупрунова и И.~Фазекаша посвящена вероятностному анализу числа без\-оши\-бочных 
блоков при помехоустойчивом кодировании; получены усиленные законы больших чисел для указанных 
величин.

В данном выпуске традиционно присутствует тематика, весьма активно разрабатываемая в течение 
многих лет специалистами ИПИ РАН и МГУ,~--- методы моделирования и управления для 
информационно-телекоммуникационных и вычислительных систем, в частности методы 
теории массового обслуживания. В~статье А.\,И.~Зейфмана с соавторами рассматриваются 
модели обслуживания, описываемые марковскими цепями с непрерывным временем в случае 
наличия катастроф. В~работе М.\,М.~Лери и И.\,А.~Чеплюковой рассматриваются случайные 
графы Интернет-типа, т.\,е.\ графы, степени вершин которых имеют степенные распределения; 
такие задачи находят применение при исследовании глобальных сетей передачи данных. 
Работа Р.\,В.~Разумчика посвящена исследованию систем массового обслуживания специального 
вида~--- с отрицательными заявками и хранением вытесненных заявок.

Ряд статей посвящен развитию перспективных теоретических 
вероятностно-статистических методов, которые находят широкое применение в различных 
задачах информатики и информационных технологий. В~работе В.\,Е.~Бенинга, А.\,К.~Горшенина 
и В.\,Ю.~Королева рассмотрена задача статистической проверки гипотез о числе компонент 
смеси вероятностных распределений, приводится конструкция асимптотически наиболее мощного 
критерия. Результаты этой работы найдут применение в ряде прикладных задач, использующих 
математическую модель смеси вероятностных распределений (в информатике, моделировании 
финансовых рынков, физике турбулентной плазмы и~т.\,д.). В~статье В.\,Ю.~Королева, 
И.\,Г.~Шевцовой и С.\,Я.~Шоргина строится новая, улучшенная оценка точности нормальной 
аппроксимации для пуассоновских случайных сумм; как известно, указанные случайные суммы 
широко используются в качестве моделей многих реальных объектов, в том числе в информатике, 
физике и других прикладных областях. Работа В.\,Г.~Ушакова и Н.\,Г.~Ушакова посвящена 
исследованию ядерной оценки плотности распределения; эти результаты могут применяться, 
в част\-ности, при анализе трафика в телекоммуникационных системах. Серьезные приложения 
в статистике могут получить результаты работы О.\,В.~Шестакова, в которой доказаны оценки 
скорости сходимости распределения выборочного абсолютного медианного отклонения к нормальному 
закону. 

\smallskip

Редакционная коллегия журнала выражает надежду, что данный тематический  выпуск 
будет интересен специалистам в области теории вероятностей и математической статистики 
и их применения к решению задач информатики и информационных технологий.
     
     %\vfill 
     \vspace*{20mm}
     \noindent
     Заместитель главного редактора журнала <<Информатика и её 
применения>>,\\
     директор ИПИ РАН, академик  \hfill
     \textit{И.\,А.~Соколов}\\
     
     \noindent
     Редактор-составитель тематического выпуска,\\
     профессор кафедры математической статистики факультета\\
      вычислительной математики и кибернетики МГУ им.\ М.\,В.~Ломоносова,\\
     ведущий научный сотрудник ИПИ РАН,\\ 
доктор физико-математических наук \hfill
      \textit{В.\,Ю.~Королев}
     
     } }
     }


\def\stat{shestakov+vor}

\def\tit{АСИМПТОТИЧЕСКАЯ НОРМАЛЬНОСТЬ И~СИЛЬНАЯ СОСТОЯТЕЛЬНОСТЬ ОЦЕНКИ РИСКА ПРИ~ИСПОЛЬЗОВАНИИ FDR-ПОРОГА В УСЛОВИЯХ СЛАБОЙ ЗАВИСИМОСТИ}

\def\titkol{Асимптотическая нормальность и~сильная состоятельность оценки риска при~использовании FDR-порога} % в~условиях слабой зависимости}

\def\aut{М.\,О.~Воронцов$^1$, О.\,В.~Шестаков$^2$}

\def\autkol{М.\,О.~Воронцов, О.\,В.~Шестаков}

\titel{\tit}{\aut}{\autkol}{\titkol}

\index{Воронцов М.\,О.}
\index{Шестаков О.\,В.}
\index{Vorontsov M.\,O.}
\index{Shestakov O.\,V.}


%{\renewcommand{\thefootnote}{\fnsymbol{footnote}} \footnotetext[1]
%{Работа 
%выполнена при поддержке Программы развития МГУ, проект №\,23-Ш03-03. При анализе 
%данных использовалась инфраструктура Центра коллективного пользования 
%<<Высокопроизводительные вычисления и~большие данные>> 
%(ЦКП <<Информатика>>) ФИЦ ИУ РАН (г.~Москва)}}


\renewcommand{\thefootnote}{\arabic{footnote}}
\footnotetext[1]{Московский государственный университет 
имени~М.\,В.~Ломоносова, факультет вычислительной математики и~кибернетики;  
Московский центр фундаментальной и~прикладной математики, \mbox{m.vtsov@mail.ru}}
\footnotetext[2]{Московский государственный университет 
имени М.\,В.~Ломоносова, факультет вычислительной математики и~кибернетики; 
Федеральный исследовательский центр <<Информатика и~управление>> Российской 
академии наук; Московский центр фундаментальной и~прикладной математики, 
\mbox{oshestakov@cs.msu.ru}}


\vspace*{-12pt}





\Abst{Рассматривается подход к~решению задачи удаления шума в~большом массиве 
разреженных данных, основанный на методе контроля средней доли ложных отклонений 
гипотез (False Discovery Rate, FDR). Данный подход эквивалентен процедурам 
пороговой обработки, обнуляющим компоненты массива, значения которых не 
превосходят некоторого заданного порога.  Наблюдения в~модели считаются слабо 
зависимыми. Для контроля степени зависимости используются ограничения на 
коэффициент сильного перемешивания и~максимальный коэффициент корреляции. 
В~качестве меры эффективности рассматриваемого подхода используется 
среднеквадратичный риск. Вычислить значение риска можно только на тестовых 
данных, поэтому в~работе рассматривается его статистическая оценка и~исследуются 
ее свойства. Показана асимптотическая нормальность и~сильная состоятельность 
оценки риска при использовании FDR-по\-ро\-га в~условиях слабой зависимости в~данных.}

\KW{пороговая обработка; множественная проверка гипотез; 
оценка риска}

\DOI{10.14357/19922264240309}{ZOQVTO}
  
%\vspace*{-6pt}


\vskip 10pt plus 9pt minus 6pt

\thispagestyle{headings}

\begin{multicols}{2}

\label{st\stat}



\section{Введение}

Во многих прикладных областях возникает задача обработки больших массивов 
зашумленных данных. Примерами служат задачи обработки изоб\-ра\-же\-ний с~высоким 
разрешением~\cite{FDRImage}, задачи множественной проверки гипотез, возникающие 
в~\mbox{исследованиях} в~об\-ласти генетики~\cite{MultipleTesting}, и~другие проб\-ле\-мы. 
В~связи с~этим рас\-смот\-рим модель
$$
x_i = \mu_i + z_i, \enskip i=\overline{1,n}\,,
$$
где $\mu_i\in\mathbb{R}$~--- <<полезные>> данные; $z_i \sim N(0,\sigma^2)$~--- 
шум. Задача заключается в~нахождении оценки неизвестного вектора $\mu \hm= 
(\mu_1,\ldots,\mu_n)$ как функции вектора $x \hm= (x_1,\ldots,x_n)$ и~может 
рассматриваться как задача множественной проверки гипотез о~равенстве нулю 
компонент вектора~$\mu$~\cite{AdaptingFDR}. При этом обычно предполагается, что 
вектор~$\mu$ имеет в~определенном смысле <<разреженную>> структуру, т.\,е.\ для 
<<полезных>> данных используется <<экономное>> представление.



В работе~\cite{AdaptingFDR} для решения рассматриваемой задачи в~условиях 
независимости компонент вектора~$x$ и~разреженности вектора~$\mu$ была 
предложена процедура построения оценки~$\hat{\mu}_F$ вектора~$\mu$, основанная 
на методе контроля средней доли ложных отклонений (FDR) 
гипотез при помощи алгоритма Бен\-жа\-ми\-ни--Хох\-бер\-га,
и~было проведено исследование асимптотики ее среднеквадратичного риска. 
В~работах~\cite{ZasShe17,Mathematics2020} была показана состоятельность 
и~асимптотическая нормальность оценки риска данной процедуры. Аналогичные 
результаты для других методов построения~$\hat{\mu}_F$ получены в~работах~\cite{Shestakov2021-1,Shestakov2021-2,Shestakov2022}.

В то же время в~определенных приложениях, например  при анализе полученных 
в~результате использования ДНК-мик\-ро\-чи\-пов данных~\cite{ResultsOnFDRUnderDependence}, исследовании геофизических процессов 
и~анализе помех\linebreak в~телекоммуникационных каналах, условие незави\-си\-мости компонент 
вектора $x$ может не выполняться. Ранее в~работах~\cite{VorontsovShestakov2023,Vorontsov2024} была \mbox{исследована} асимп\-то\-ти\-ка 
среднеквадратичного риска оценки~$\hat{\mu}_F$ \mbox{в~случае}, когда~$\mu$ принадлежит 
одному из классов разреженности
$$
l_0[\eta] = \left\{\mu\,:\, ||\mu||_0 \leq \eta n\right\}, \enskip \eta \in 
(0,1),
$$

\vspace*{-12pt}

\noindent
\begin{multline*}
m_p[\eta] \equiv{}\\
{}\equiv \left\{\mu \in \mathbb{R}^n : |\mu|_{(k)} \leq \eta n^{1/p} 
k^{-1/p},\ k=\overline{1,n}\right\}, \\
 p\in(0, 2),
\end{multline*}
а компоненты вектора~$x$ слабо зависимы~--- имеют достаточно быстро убывающий 
коэффициент сильного перемешивания~\cite{Bosq}

\noindent
\begin{multline*}
\alpha(k) = \sup\limits_{1\leq m\leq n}\alpha\left(\sigma(x_i, i\leq m), 
\sigma(x_i, i\geq m+k)\right), \\ 
k=\overline{1,n-1}\,,
\end{multline*}
где символом $\sigma(x_i, i\in I)$ обозначена сиг\-ма-ал\-геб\-ра, порожденная 
множеством случайных величин $\{x_i, i \hm\in I\}$, а~мера  $\alpha(\cdot, \cdot)$ 
близости двух сиг\-ма-ал\-гебр определяется как
$$
\alpha(\mathcal{B},\mathcal{C}) = \sup\limits_{B\in\mathcal{B}, 
C\in\mathcal{C}} \left|\p(BC)-\p(B)\p(C)\right|.
$$

В настоящей работе показана асимптотическая нормальность и~сильная 
состоятельность оценки риска при применении FDR-про\-це\-ду\-ры в~случае, когда 
компоненты вектора~$x$ слабо зависимы, а~$\mu$ принадлежит одному из классов 
раз\-ре\-жен\-ности: 
$l_0[\eta]$ или $m_p[\eta]$.


\section{Обработка вектора данных с~помощью FDR-процедуры}

Широким классом методов построения оценки~$\hat{\mu}$ стала пороговая обработка 
вектора~$x$ с~некоторым порогом~$T$. Различают жесткую пороговую обработку, при 
которой полагается
\begin{equation*}
\left(\hat{\mu}\right)_i  = p_H(x_i,T) \equiv
 \begin{cases}
   x_i, & |x_i| > T\,;\\
   0, & |x_i| \leq T\,,
 \end{cases}
\end{equation*}
и мягкую пороговую обработку, для которой
\begin{equation*}
(\hat{\mu})_i  = p_S(x_i,T) \equiv
 \begin{cases}
   x_i-T, & \hphantom{\vert\vert}x_i > T;\\
   x_i+T, & \hphantom{\vert\vert}x_i <- T;\\
   0, & |x_i| \leq T.
 \end{cases}
\end{equation*}
Среднеквадратичный риск подобных процедур определяется как
\begin{equation}
\label{riskDef}
R(T) = {\mathsf E} ||\hat{\mu}-\mu||^2 = \sum\limits_{i=1}^n {\mathsf E} \left((\hat{\mu})_i-
\mu_i\right)^2.
\end{equation}
Обозначим через~$T_m$ наилучшее значение порога:
$$
T_m : \, R(T_m) = \min\limits_{T} R(T).
$$

Предложенная в~\cite{AdaptingFDR} процедура заключается в~жесткой пороговой 
обработке компонент вектора~$x$ с~порогом $\hat{t}_F \hm= \hat{t}_F(x)$, и~ее 
результат~--- оценка $\hat{\mu}_F$ вектора~$\mu$ с~компонентами $(\hat{\mu}_F)_i  
\hm= p_H(x_i,\hat{t}_F)$, где
\begin{multline*}
\hat{t}_F = \sigma z\left(\fr{q \hat{k}_F}{2n}\right), \enskip
\hat{k}_F = \max 
\left\{k \, :\, |x|_{(k)} \geq t_k \right\}, \\
 t_k = \sigma z\left(\fr{q  k}{2n}\right);
\end{multline*}
$z(\alpha)$ --- квантиль уровня $(1\hm-\alpha)$ стандартного нормального 
распределения; $|x|_{(k)}$~--- $k$-й элемент вектора, получаемого в~результате 
упорядочения вектора~$|x|$ по невозрастанию:
$$
|x|_{(1)} \geq |x|_{(2)} \geq \cdots \geq |x|_{(n)};
$$
$q\in(0;1)$~--- управ\-ля\-ющий параметр FDR-ме\-то\-да.
Далее полагается, что $q\hm\equiv q_n$ зависит от~$n$. В~\cite{AdaptingFDR} 
показано, что эта процедура эквивалентна множественной проверке гипотез 
о~равенстве нулю компонент наблюдаемого вектора. Также показано, что с~помощью 
метода штрафных функций данную процедуру можно свести к~другим видам пороговой 
обработки, в~част\-ности к~мягкой пороговой обработке.

В работах~\cite{VorontsovShestakov2023, Vorontsov2024} была исследована 
асимптотика среднеквадратичного риска~$R(\hat{t}_F)$ описанной процедуры 
в~случае, когда компоненты вектора $x$ слабо зависимы, а $\mu$ принадлежит классу 
разреженности~$\Theta_n$, где~$\Theta_n$ есть~$l_0[\eta_n]$ или~$m_p[\eta_n]$. 
Было показано, что~$R(\hat{t}_F)$ асимптотически отличается от минимаксного 
риска
$\inf\nolimits_{\hat{\mu}\hm=\hat{\mu}(x)} \sup\nolimits_{\mu\in \Theta_n} {\mathsf E} 
||\hat{\mu}-\mu||^2$
на множитель не более чем логарифмического по\-рядка.

Отметим, что в~выражении для среднеквадратичного риска~(\ref{riskDef}) 
присутствуют неизвестные величины~$\mu_i$, а~потому вычислить~$R(T_m)$ и~$T_m$ 
не представляется возможным. На практике можно пользоваться, например, следующей 
оценкой среднеквадратичного риска~\cite{Mallat}:
$$
\hat{R}(T) = \sum\limits_{i=1}^n F[x_i, T],
$$
где  
\begin{multline*}
F[x_i, T] = {}\\[3pt]
{}=\!\begin{cases}
\left(x_i^2-\sigma^2\right) \Ik(|x_i|\leq T) + \sigma^2 \Ik\left(|x_i|>T\right) &\\[3pt]
&\hspace*{-53mm}\mbox{для\ жесткой\ пороговой\ обработки};\\[3pt]
\left(x_i^2-\sigma^2\right) \Ik\left(|x_i|\leq T\right) + (\sigma^2+T^2) 
\Ik \left(|x_i|>T\right) \hspace*{-11.21576pt}&\\[3pt]
&\hspace*{-51mm}\mbox{для\ мягкой\ пороговой\ обработки}.
\end{cases}\hspace*{-7.17859pt}
\end{multline*}


\noindent
\textbf{Замечание}.\ При пороговой обработке иногда также используется так 
называемый универсальный порог $T_U\hm = \sigma \sqrt{2\ln n}$, предложенный 
в~работе~\cite{spatialAdaptation}. Исследования в~\cite{AdaptingSURE, ExactRisk} 
показали, что порог~$T_U$ в~определенном смысле максимальный, и~рас\-смат\-ри\-вать 
пороги выше него не имеет смысла. Более того, нетрудно показать, что $t_k \hm< T_U$ 
для всех~$k$ и~всех достаточно больших~$n$, в~связи с~чем всюду далее полагаем, 
что порог~$\hat{t}_F$ выбирается на отрезке $[0; T_U]$.

\section{Вспомогательные утверждения}

Кроме коэффициента сильного перемешивания~$\alpha(\cdot)$ также понадобится 
следующее понятие~\cite{Bosq}.

\smallskip

\noindent
\textbf{Определение.} %\label{defRho}
Максимальным коэффициентом корреляции~$\rho(\cdot)$ компонент вектора~$x$ 
называется
\begin{multline*}
\rho (k) \equiv \rho_n (k) = {}\\
{}=\sup\limits_{1\leq m\leq n}\rho\left(\sigma(x_i, 
i\leq m), \sigma(x_i, i\geq m+k)\right), \\
 k=\overline{1,n-1}\,,
\end{multline*}
где мера $\rho(\cdot, \cdot)$ близости двух сиг\-ма-ал\-гебр определяется как
$$
\rho(\mathcal{B},\mathcal{C}) = \sup\limits_{\substack{\xi 
\in\mathcal{L}^2(\mathcal{B}) \\
 \eta \in\mathcal{L}^2(\mathcal{C})}} 
\left|\mathrm{corr}\,(\xi, \eta)\right|.
$$


Введем обозначения:
$$
T_1 = \sqrt{2\ln \eta_n^{-p}};  \,\gamma_n = \fr{1}{\ln\ln n}; \, \kappa_n 
= \fr{n \eta_n^p T_1^{-p}}{1 - q_n - \gamma_n}; 
$$
$$ 
\kappa_n^0 = \fr{[n \eta_n]}{1 - q_n - \gamma_n} ;\, \rho^\star (k) = 
\sup\limits_{n\geq k+1} \rho(k), k \in \mathbb{N} ;
$$
$$
t_{\kappa_n} = \sigma z\left(\fr{q_n \kappa_n }{2n}\right) , \,\, t_{\kappa_n^0} 
= \sigma z\left(\fr{q_n \kappa_n^0 }{2n}\right).
$$


Следующие два утверждения показывают, что случайный порог~$\hat{t}_F$ в~случае 
$\mu\hm\in m_p[\eta_n]$ (соответственно $\mu\hm\in l_0[\eta_n]$) с~большой 
вероятностью будет не меньше~$t_{\kappa_n}$ (соответственно~$ t_{\kappa_n^0}$). 
Их  доказательства приведены в~работах~\cite{VorontsovShestakov2023, Vorontsov2024}.

\smallskip

\noindent
%\begin{lem}\label{lem5}
\textbf{Лемма~1.}\ \textit{Пусть $n^{-\delta_1} \hm\leq \eta_n^p \hm\leq n^{-\delta_2}$, 
$0\hm<\delta_2\hm<\delta_1<1$, $\mathrm{lim\,inf} q_n \ln n \hm\geq C \hm> 0$, 
$m\hm\in[1;n/2]\cap\mathbb{N}$, а $\alpha(\cdot)$~--- коэффициент сильного 
перемешивания компонент вектора~$x$. Для некоторого $N\hm\in\mathbb{N}$ при $n \hm\geq 
N$ справедливо}
\begin{multline*}
\hspace*{-3pt}\sup\limits_{\mu\in m_p[\eta_n]} \p \left(\hat{k}_F \geq \kappa_n \right) \leq 
4 n \exp\left\{-\fr{m}{256n}  \kappa_n q_n \gamma_n^2    \right\}+{}\\
{}+ 22\left(1+\fr{8n}{\kappa_n q_n \gamma_n}\right)^{1/2} n m 
\alpha\left(\left[\fr{n}{2m}\right]\right).
\end{multline*}



\smallskip

\noindent
\textbf{Лемма 2.}\ 
%\label{lem1}
\textit{Пусть $\eta_n \hm\leq b\hm<1$, $m\in[1;n/2]\cap\mathbb{N}$, а~$\alpha(\cdot)$~--- 
коэффициент сильного перемешивания компонент вектора~$x$. Для некоторого 
$N\hm\in\mathbb{N}$ при $n \hm\geq N$ справедливо}
\begin{multline*}
\sup\limits_{\mu\in l_0[\eta_n]} \p \left(\hat{k}_F \geq \kappa_n^0 \right) 
\leq{}\\
{}\leq 4 n \exp\left\{-\fr{(1-b)m}{64n}\,  \kappa_n^0 q_n \gamma_n^2    
\right\}+{}\\
{}+ 22\left(1+\fr{4n}{(1-b)\kappa_n^0 q_n \gamma_n}\right)^{1/2} n m 
\alpha\left(\left[\fr{n}{2m}\right]\right).
\end{multline*}

Следующие два утверждения доказаны в~\cite{Bosq} и~представляют собой аналоги 
неравенств Хеффдинга и~Бернштейна для слабо зависимых случайных величин.


\smallskip

\noindent
\textbf{Лемма 3.}\
\textit{Пусть для набора действительных случайных величин $X_1, \ldots, X_n$ 
с~коэффициентом сильного перемешивания $\alpha(\cdot)$ выполняется ${\mathsf E} X_i \hm=0$, 
$|X_i|\hm\leq b$, $i\hm=\overline{1,n}$. Тогда для любого целого числа $m\hm\in[1; n/2]$ 
и~любого $\eps\hm>0$ справедливо}
\begin{multline*}
\p\left(\left|\sum\limits_{i=1}^n X_i\right| > n\eps \right) \leq 4 
\exp\left\{-\fr{\eps^2 m}{8 b^2}\right\}+ {}\\
{}+
22\left(1+\fr{4b}{\eps}\right)^{1/2} m\, 
\alpha\left(\left[\fr{n}{2m}\right]\right).
\end{multline*}


\smallskip

\noindent
\textbf{Лемма 4.}\
\textit{Пусть для набора действительных случайных величин $X_1, \ldots, X_k$ 
с~коэффициентом сильного перемешивания $\alpha(\cdot)$ выполняется ${\mathsf E} X_i \hm=0$, 
$|X_i|\hm\leq b$, $i\hm=\overline{1,k}$. Тогда для любого целого числа $m\hm\in[1; k/2]$ 
и~любого $\eps\hm>0$ справедливо}
\begin{multline*}
\p\left(\left|\sum\limits_{i=1}^k X_i\right| > \eps \right) \leq 4 
\exp\left\{-\fr{\eps^2 m}{8 v^2 k^2}\right\}+{}\\
{}+ 22\left(1+\fr{4bk}{\eps}\right)^{1/2} m\, 
\alpha\left(\left[\fr{k}{2m}\right]\right),
\end{multline*}
\textit{где $p = k/(2m)$}:
\begin{multline*}
v^2 =
 \fr{b \eps}{2k} + {}\\
 {}+\fr{2}{p^2} \,  \max\limits_{ j\in[0,\,2m-1]} 
{\mathsf E} \big( ([jp]+1-jp)X_{[jp]+1} + X_{[jp]+2}+{}\\
{}+ \cdots +  X_{[(j+1)p]} + ((j+1)p-[(j+1)p])X_{[(j+1)p+1]}\big)^2.
\end{multline*}

\noindent
\textbf{Замечание.}
Если существует такое число $S \hm> 0$, что сразу для всех $i\hm\in[1;k]$  выполняется 
${\mathsf E} X_i^2 \hm\leq S^2$, то в~качестве~$v^2$ можно взять
$$
v^2 = \fr{b \eps}{2k} + 8 S^2.
$$


Д\,о\,к\,а\,з\,а\,т\,е\,л\,ь\,с\,т\,в\,о\ \ сле\-ду\-юще\-го утверж\-де\-ния приведено в~работе~\cite{AdaptingFDR}.

\smallskip

\noindent
\textbf{Лемма 5.}\ 
\textit{Для $y\leq 0{,}01$ справедливы представления}
\begin{multline}
\label{lem1eq1}
z^2(y) = 2 \ln y^{-1} - \ln \ln y^{-1} - r_2(y), \\
 r_2(y) \in [1{,}8; 3];
\end{multline}

\noindent
\begin{equation}
\label{lem1eq2}
z(y) = \sqrt{2 \ln y^{-1}} - r_1(y), \, \, r_1(y) \in [0; 1{,}5].
\end{equation}


\section{Асимптотическая нормальность оценки риска при~применении FDR-процедуры в~условиях слабой зависимости}

Перейдем к~описанию достаточных условий для асимптотической нормальности оценки 
риска $\hat{R}(\hat{t}_F)$ в~случае $\mu \hm\in m_p[\eta_n]$.

\smallskip

\noindent
\textbf{Теорема~1.}\
\textit{Пусть $\mu \hm\in m_p[\eta_n],$ $\eta_n^p \hm\in[n^{-\delta_1}; n^{-\delta_2}],$ $1/2 \hm< 
\delta_2 \hm< \delta_1<1;$ имеются такие константы $c_1, c_2>0$, что для 
коэффициента сильного перемешивания $\alpha(\cdot)$ компонент вектора $x$ 
справедливо  $\alpha(k) \hm\leq c_1 k^{-1-(5/2)\delta_1/(1-\delta_1)-c_2},$ 
$k\hm=\overline{1,n-1};$ $q_n \hm< c_3 \hm< 1;$ $\mathrm{lim\,inf} q_n \ln n \hm= c_4 \hm> 0;$ и,~кроме того, 
для максимального коэффициента корреляции $\rho(\cdot)$ компонент вектора~$x$ 
справедливо}
$$
\sum\limits_{k = 1}^{\infty} \sup\limits_{n\geq k+1} \rho(k) \equiv 
\sum\limits_{k = 1}^{\infty}  \rho^\star (k) = c_5 < \infty. 
$$
\textit{Тогда при $n \to \infty$}
$$
\fr{\hat{R}(\hat{t}_F) - R(T_m)}{C_\rho \sqrt{2n}} \Rightarrow N(0, 1),
$$
\textit{где}
$$
C_\rho = \sigma^2\sqrt{1 +  \lim\limits_{n\to\infty} \fr{1}{n} \sum\limits_{j\neq i} \mathrm{corr}^2 (x_i, x_j)}.
$$

\noindent
Д\,о\,к\,а\,з\,а\,т\,е\,л\,ь\,с\,т\,в\,о\  \
 приводится для метода мягкой пороговой обработки; в~случае жесткой пороговой 
обработки доказательство аналогично. Обозначим
$$
U(T) = \hat{R}(T) -  \hat{R}(T_m) = \sum \limits_{i=1}^n H_i(T, T_m),
$$
где
$$
H_i(T, T_m) = F[x_i, T] - F[x_i, T_m].
$$
Имеем

\vspace*{-3pt}

\noindent
\begin{multline}
\label{D00}
\hat{R}(\hat{t}_F) - R(T_m) + \hat{R}(T_m) - \hat{R}(T_m) ={}\\
{}= \hat{R}(T_m) - 
R(T_m) + U(\hat{t}_F).
\end{multline}
Покажем, что
\begin{equation}
\label{D0}
\fr{\hat{R}(T_m) - R(T_m)}{C_\rho\sqrt{2n}} \Rightarrow N(0, 1).
\end{equation}


Повторяя рассуждения из~\cite{KuShe2016_1,KuShe2016_2,Jansen}, можно показать, 
что $T_m \hm\geq t_{\kappa_n}$. Учитывая также $T_m\hm \leq T_U$, имеем 
$$
C \sqrt{\ln n} \leq T_m \leq C^\prime \sqrt{\ln n}
$$ 
для некоторых положительных констант $C$ и~$C^\prime$.

\columnbreak

В случае мягкой пороговой обработки $\hat{R}(T_m)$ представляет собой 
несмещенную оценку~$R(T_m)$, а~при жесткой пороговой обработке и~выполнении 
условий теоремы смещение стремится к~нулю при делении на $\sqrt{n}$~\cite{Mallat}.

Для дисперсии числителя~(\ref{D0}) имеем:
\begin{multline*}
{\mathsf D} \left(\hat{R}(T_m) - R(T_m)\right) = \sum\limits_{i=1}^n {\mathsf D} F[x_i, T_m] + {}\\
{}+
\sum\limits_{i=1}^n\sum\limits_{\substack{j=1 \\  j\neq i}}^n \mathrm{cov}\left( F[x_i, T_m], F[x_j, 
T_m] \right).
\end{multline*}

Поскольку $\mu \in m_p[\eta_n]$,
\begin{equation}
\left.
\begin{array}{l}
 \displaystyle\sum\limits_{i: |\mu_i| > 1/T_1} {\mathsf D} F[x_i, T_m]  \leq{}\\
 \hspace*{15mm}{}\leq  4\left(\sigma^2 + T_m^2\right)^2 n \eta_n^p 
T_1^p = o(n);
\\[6pt]
\displaystyle \sum\limits_{\substack{{i,j: \max\{|\mu_i|, |\mu_j|\} > 1/T_1,}\\{j\neq i}}}  \hspace*{-12mm}\mathrm{cov}\,(F[x_i, 
T_m],F[x_j, T_m])  \leq{}\\
\hspace*{10mm}{}\leq 16\left(\sigma^2 + T_m^2\right)^2 n \eta_n^p T_1^p c_5 = o(n). 
\end{array}
\right\}    
\label{D2}
\end{equation}
Далее, учитывая что ${\mathsf D} x_i^2 \hm= 2\sigma^4 \hm+ 4\sigma^2 \mu_i^2$, нетрудно 
убедиться, что
\begin{multline}
\label{D3}
\sum\limits_{i: |\mu_i| \leq 1/T_1}\hspace*{-4mm} {\mathsf D} F[x_i, T_m] ={}\\
{}= \sum\limits_{i: |\mu_i| \leq 1/T_1} \hspace*{-4mm} {\mathsf D} 
x_i^2 + o(n) = 2\sigma^4 n + o(n).
\end{multline}


Введем обозначение 
$$
D_n = \left\{(i,j) : \max\left\{|\mu_i|, |\mu_j|\right\}  \leq \fr{1}{T_1}\,, \enskip j\hm\neq i\right\}.
$$
 Для суммы ковариаций аналогично~(\ref{D3}) получим
\begin{multline*}
\sum\limits_{(i,j)\in D_n} \hspace*{-2mm}\mathrm{cov}\left( F[x_i, T_m], F[x_j, T_m] \right) = {}\\
{}=
\sum\limits_{(i,j)\in D_n} \hspace*{-2mm}\mathrm{cov}\left( x_i^2, x_j^2 \right) + o(n).
\end{multline*}
Воспользуемся тождеством~\cite{Eroshenko}
$$
\mathrm{cov}\left (x_i^2, x_j^2\right) = 4 {\mathsf E} x_i {\mathsf E} x_j \mathrm{cov}\left(x_i, x_j\right) + 2 \mathrm{cov}^2 \left(x_i, x_j\right)
$$
для вектора $(x_i, x_j)$, имеющего двумерное нормальное распределение. Заметим, 
что
\begin{gather*}
 \sum\limits_{(i,j)\in D_n} 4 | {\mathsf E} x_i {\mathsf E} x_j \mathrm{cov}\left(x_i, x_j\right)| \leq 8 T_1^{-2} 
\sigma^2 n c_5 = o(n);
\\
\sum\limits_{(i,j)\in D_n} 2 \mathrm{cov}^2 (x_i, x_j)  = 2\sigma^4 \sum\limits_{(i,j)\in D_n} 
\mathrm{corr}^2 (x_i, x_j). 
\end{gather*}
Более того, поскольку  %< 4 \sigma^2 n c_5.$$
\begin{equation*}
\sum\limits_{\substack{{i,j: \max\{|\mu_i|, |\mu_j|\} > 1/T_1} \\ {j\neq i}}}
\hspace*{-10mm}\mathrm{corr}^2 (x_i, x_j)  
\leq  4 n \eta_n^p T_1^p c_5 =  o(n),
\end{equation*}
имеем
\begin{multline*}
\sum\limits_{(i,j)\in D_n} \mathrm{corr}^2 (x_i, x_j) ={}\\
{}= \sum\limits_{j\neq i} \mathrm{corr}^2 (x_i, x_j) 
+o(n)= c_6 n + o(n),
\end{multline*}
где
$$
c_6 = \lim\limits_{n\to\infty} \fr{1}{n} \sum\limits_{j\neq i} \mathrm{corr}^2 (x_i, x_j) 
\leq 2 c_5.
$$
Полагая $C_\rho \hm= \sigma^2\sqrt{1 + c_6}$, получим, наконец,
\begin{equation}
\label{D1}
{\mathsf D} \left(\hat{R}(T_m) - R(T_m)\right)  =  2 n C_\rho^2 + o(n).
\end{equation}
Заметим, что из~(\ref{D2}), (\ref{D3}) и~(\ref{D1}) следует, что
\begin{equation}
\label{D5}
\sup\limits_{n} \fr{\sum\nolimits_{i=1}^n {\mathsf D} F[x_i, T_m]}{V_n^2} < \infty\,,
\end{equation}
где 
$$
V_n^2 = {\mathsf D} \sum\limits_{i=1}^n \left(F[x_i, T_m] \hm- {\mathsf E} F[x_i, T_m]\right).
$$
Кроме того, поскольку $F[x_i, T_m]$ по модулю ограничены величиной $\sigma^2 \hm+ 
T_m^2$, выполнено условие Линдеберга: для любого $\eps\hm>0$ при $n \hm\to \infty$
\begin{multline}
\label{D6}
\!\!\!\fr{1}{V_n^2}\sum\limits_{i=1}^n {\mathsf E} \left( \!\left( F\left[x_i, T_m\right]\! -\! {\mathsf E} F\left[x_i, T_m\right]\right)^2 
\Ik \left(\vert F\left[x_i, T_m\right] -{}\right.\right.\hspace*{-2.69505pt}\\
\left.\left.{}- {\mathsf E} F\left[x_i, T_m\right]\vert >\eps V_n\right)\!
\vphantom{\left( F\left[x_i, T_m\right]\! -\! {\mathsf E} F\left[x_i, T_m\right]\right)^2}
\right) 
\to  0\,.
\end{multline}
Из~(\ref{D1})--(\ref{D6}), очевидного неравенства
$$ 
\lim\limits_{k\to\infty} \sup\limits_{n\geq k+1}\rho(k) \equiv 
\lim\limits_{k\to\infty} \rho^\star (k)  < 1
$$
 и~центральной предельной теоремы для сильно перемешанных случайных величин~\cite{Peligrad} следует~(\ref{D0}).

Перейдем к~доказательству того, что $U(\hat{t}_F) \, n^{-1/2} \overset{\, \p \, }{\to} 0$.
Всюду далее, не ограничивая общности, полагаем $\sigma=1$. 
Введем обозначения:

\noindent
\begin{align*}
S_1(T) &= \sum\limits_{i: |\mu_i| > 1/T_1} H_i(T, T_m); \\
S_2(T) &= \sum\limits_{i: |\mu_i| \leq 1/T_1} H_i(T, T_m); 
\\
N_1(a, b) &= \sum\limits_{i: |\mu_i| > 1/T_1} \Ik (a<|x_i|\leq b); \\ 
N_2(a, b) &= \sum\limits_{i: |\mu_i| \leq 1/T_1} \Ik (a<|x_i|\leq b);
\end{align*}

\noindent
\begin{align*}
Z_l(T) &= S_l(T) - {\mathsf E} S_l(T),\enskip l = 1,2\,; \\  
d_n &= \fr{T_U -  t_{\kappa_n}}{n};\\
T_j^{\prime} &= t_{\kappa_n}+j d_n,\enskip j = \overline{0,n-1}\,.
\end{align*} 

\vspace*{-3pt}

\noindent
Для произвольного $\eps>0$

\vspace*{-3pt}

\noindent
\begin{multline}
\p \left( \fr{|U(\hat{t}_F)|}{\sqrt{n}}> 4\eps \right) \leq 
\p\left(\hat{t}_F \leq t_{\kappa_n}\right) + {}\\
{}+\p \left(\fr{\sup\nolimits_{T\in 
[t_{\kappa_n}, T_U]} |U(T)|}{\sqrt{n}}>4\eps \right)\leq  {}\\
{}\leq \p\left(\hat{t}_F \leq t_{\kappa_n}\right) + \p\left(\fr{\sup\nolimits_{T\in 
[t_{\kappa_n}, T_U]} |{\mathsf E} U(T)|}{\sqrt{n}}>\eps\right)+{}\\
{}+ \p \left(\sup\limits_{T\in [t_{\kappa_n}, T_U]} |Z_1(T)| > 
\eps\sqrt{n}\right) +{}\\
{}+ \p \left(\sup\limits_{j \in [0, n-1]} |Z_2(T_j^{\prime})| > 
\eps\sqrt{n}\right) +{}\\
{}+ \p \left(\sup\limits_{\substack{j \in [0, n-1] \\
 T\in [T_j^{\prime},T_j^{\prime}+d_n]}} |Z_2(T)-Z_2(T_j^{\prime})| > \eps\sqrt{n}\right).
\label{M1}
\end{multline}
Заметим, что $\gamma_n\hm > \ln^{-1} n$, $\kappa_n\hm > n \eta_n^p \ln ^{-1} n \hm\geq 
n^{1-\delta_1} \ln ^{-1} n$ и~$q_n\hm > c_4 \ln ^{-1} n /2$ для всех достаточно 
больших~$n$.
Для первого слагаемого в~(\ref{M1}) по лемме~1 с~$m \hm= n^{\delta_1} \ln 
^7 n$ для  больших~$n$ имеем

\vspace*{-3pt}

\noindent
\begin{multline}
\label{M1next}
\p\left(\hat{t}_F \leq t_{\kappa_n}\right)  = \p \left(\hat{k}_F \geq \kappa_n 
\right) \leq 4 n e^{-\ln^2 n} + {}\\
{}+n^{1+(3/2)\,\delta_1} \ln^9 n \, 
\alpha\left(\left[\fr{n^{1-\delta_1}}{\ln^{7} n}\right]\right) = o(1)
\end{multline}
при $n\to\infty$. 
Для оценки второго слагаемого в~(\ref{M1}) заметим, что при $T \hm\in 
[t_{\kappa_n}, T_U]$ справедливо
\begin{equation}
\label{M2}
{\mathsf E} H_i(T, T_m) \leq T_U^2 + 1.
\end{equation}
Если же кроме $T \hm\in [t_{\kappa_n}, T_U]$ также выполнено $|\mu_i| \hm\leq T_1^{-1}$, то

\vspace*{-6pt}

\noindent
\begin{multline*}
|{\mathsf E} H_i (T, T_m)| \leq 2 T_U^2 \, \p \left(|x_i| > t_{\kappa_n}\right) \leq {}\\
{}\leq2 
T_U^2 \, \p \left(|x_i-\mu_i| > t_{\kappa_n}-T_1^{-1}\right) \leq{}\\
{}\leq 2 T_U^2  \exp\left\{ -\fr{1}{2} \left(t_{\kappa_n} - T_1^{-
1}\right)^2 \right\}  \leq{}\\
{}\leq
 4 (\ln n)  \exp\left\{ -\fr{1}{2} 
\left(z\left(\fr{q_n\kappa_n}{2n}\right)\right)^2 + t_{\kappa_n} T_1^{-
1}\right\},
\end{multline*}

\vspace*{-2pt}

\noindent
где использовано неравенство 

\noindent
$$
2(1-\Phi(x))\hm \leq \fr{e^{-x^2/2}}{x}
$$

\pagebreak


\noindent
 для $x\hm\geq 0$ 
($\Phi(x)$~--- функция распределения $N(0,1)$). Рас\-смот\-рим выражение 
в~экспоненте. Второе слагаемое не превышает $1\hm+o(1)$ при $n\hm\to\infty$, поскольку 
$t_{\kappa_n} \hm\leq T_1 (1+o(1))$ при $\sigma\hm=1$, что нетрудно получить из 
определения~$t_{\kappa_n}$, пред\-став\-ле\-ния~(\ref{lem1eq2}) и~ограничения на~$q_n$ 
из формулировки тео\-ре\-мы. Для первого слагаемого, используя пред\-став\-ле\-ние~(\ref{lem1eq1}) 
и~ограничения, наложенные на~$q_n$, при больших~$n$ получим
\begin{multline*}
-\fr{1}{2}\left(z\left(\fr{q_n \kappa_n}{2n}\right)\right)^2 \leq - \ln 
\fr{2n (1-q_n-\gamma_n)}{q_n n \eta_n^p T_1^{-p}} + {}\\
{}+\fr{1}{2} \ln 
\left((1+o(1)) \ln \eta_n^{-p}\right) + \fr{3}{2} \leq{}\\
{}\leq \ln \fr{c_3}{1-c_3} + \ln \eta_n^p + \ln T_1^{-p} + \ln T_1 + 
\fr{3}{2}+ o(1).
\end{multline*}
Из приведенных соотношений следует, что с~некоторой константой $c_7 = c_7(c_3, 
p, \delta_1, \delta_2, c_4)$
\begin{equation}\label{M3}
\sup\limits_{\substack{i: |\mu_i| \leq 1/T_1 \\ T\in [t_{\kappa_n}, T_U]}} |{\mathsf E} 
H_i (T, T_m)|  \leq c_7 (\ln n)^{(3-p)/2}\eta_n^p.
\end{equation}
Из (\ref{M2}) и~(\ref{M3}) с~учетом $\delta_2 \hm> 1/2$ следует
\begin{multline*}
\sup\limits_{T\in [t_{\kappa_n}, T_U]} |{\mathsf E} U(T)| \leq{}\\
{}\leq 
 n\eta_n^p T_1^p 
(T_U^2+1) + c_7 (\ln n)^{(3-p)/2} n \eta_n^p = o(\sqrt{n})
\end{multline*}
при $n\to\infty$, а следовательно, для любого $\eps\hm>0$ второе слагаемое в~(\ref{M1}) обращается в~ноль для всех достаточно больших~$n$.

Далее, поскольку при $T \hm\leq T_U$ и~$\sigma\hm=1$
$$
|H_i(T, T_m) - {\mathsf E} H_i(T, T_m)| \leq 2 (T_U^2 +2), \enskip i=\overline{1, n}\,,
$$
а число слагаемых в~$Z_1(T)$ не превосходит $n\eta_n^p T_1^p$, имеем
$$
\sup\limits_{T\in [t_{\kappa_n}, T_U]} |Z_1(T)|  \leq 2 n\eta_n^p T_1^p (T_U^2 
+2) = o(\sqrt{n})
$$
при $n\to\infty$, а следовательно, для любого $\eps\hm>0$ и~третье слагаемое в~(\ref{M1}) обращается в~ноль для всех достаточно больших~$n$.

Перейдем к~оценке четвертого слагаемого в~(\ref{M1}). Аналогично~(\ref{M3}) 
можно получить:
\begin{multline}
\label{M10}
\!\!\sup\limits_{\substack{i: |\mu_i| \leq 1/T_1 \\ T\in [t_{\kappa_n}, T_U]}} \!{\mathsf D} 
H_i (T, T_m)  \leq \!\sup\limits_{\substack{i: |\mu_i| \leq 1/T_1 \\ T\in 
[t_{\kappa_n}, T_U]}} \!{\mathsf E} \left(H_i (T, T_m)\right)^2  \leq{}\\
{}\leq 2 c_7 (\ln n)^{(5-p)/2} \eta_n^p.
\end{multline}
По лемме~4 с~$m \hm= \sqrt{n} (\ln n)^3$ и~$k \hm= n-[n\eta_n^p T_1^p]$ 
для четвертого слагаемого в~(\ref{M1}) имеем:

\noindent
\begin{multline}
\p \left(\sup\limits_{j \in [0, n-1]} |Z_2(T_j^\prime)| > \eps\sqrt{n}\right) 
\leq {}\\
{}\leq \sum\limits_{j \in [0, n-1]} \hspace*{-3mm}\p \left( |Z_2(T_j^\prime)| > \varepsilon\sqrt{n}\right)\leq{}\\
{}\leq 4 n \exp \left\{ - \fr{\eps^2 n^{3/2} (\ln n)^3}{n-[n\eta_n^p T_1^p]}\!\Bigg/\! \big( 8 (T_U^2+2)\eps\sqrt{n} +{}\right.\\
\left.{}+ 128 c_7 (\ln n)^{(5-p)/2} \eta_n^p  (n-
[n\eta_n^p T_1^p])\big) 
\vphantom{ \fr{\eps^2 n^{3/2} (\ln n)^3}{n-[n\eta_n^p T_1^p]}}
\right\} +{}\\
{}
+ 22 \left(1+\fr{8(T_U^2+2) (n-[n\eta_n^p T_1^p])}{\eps 
\sqrt{n}}\right)^{1/2}\times{}\\
{}\times n^{3/2} (\ln n)^3 \alpha\left(\left[\fr{n-[n\eta_n^p 
T_1^p]}{2 (\ln n)^3 \sqrt{n}}\right]\right).
\label{M5}
\end{multline}
Используя ограничения $n^{-\delta_1}\hm\leq \eta_n^p \leq n^{-\delta_2}$ 
и~$1/2\hm<\delta_2\hm<\delta_1\hm<1$, из~(\ref{M5}) получим для любого $\eps\hm>0$
$$
\p \left(\sup\limits_{j \in [0, n-1]} |Z_2(T_j^\prime)| > \eps\sqrt{n}\right) 
\to 0
$$
при $n \to \infty$.

Рассмотрим, наконец, пятое слагаемое в~(\ref{M1})). Заметим, что при $0\hm< a \hm< b$ 
справедливо
$$
|Z_2(b)-Z_2(a)| \leq 2 |N_2(a,b)-{\mathsf E} N_2(a,b)| + n (b^2-a^2).
$$
Полагая $a = T_j^\prime$, $b \hm= T \hm\in [T_j^\prime, T_j^\prime+d_n]$ для 
произвольного $j \hm\in [0, n-1]$ и~учитывая, что
$$
(T^2 - (T_j^\prime )^2) = (T - T_j^\prime)(T+ T_j^\prime ) \leq  2 d_n T_U < 2 
T_U^2 n^{-1}; 
$$

\vspace*{-12pt}

\noindent
\begin{multline*}
\p\left(T_j^\prime < |x_i| \leq T \right) \leq \p\left(T_j^\prime < |x_i| \leq 
T_j^\prime+d_n\right) <{}\\
{}< d_n < T_U n^{-1}, 
\end{multline*}
получим  оценку
$$
|Z_2(T)-Z_2(T_j^\prime)| \leq 2 N_2(T_j^\prime, T) +  3 T_U^2 .
$$
Далее, поскольку $N_2 (T_j^\prime, T) \hm\leq N_2 (T_j^\prime, T_j^\prime+d_n)$ и~${\mathsf E} N_2 (T_j^\prime, T_j^\prime+d_n) \hm< T_U^2$,
имеем
\begin{multline*}
\sup\limits_{T \in [T_j^\prime, T_j^\prime+d_n]} |Z_2(T)-Z_2(T_j^\prime)| \leq {}\\
{}\leq
2 \left|N_2 (T_j^\prime, T_j^\prime+d_n) - {\mathsf E} N_2 (T_j^\prime, 
T_j^\prime+d_n)\right| +  5 T_U^2 .
\end{multline*}
Аналогично~(\ref{M3}) показывается, что
\begin{multline}
\label{M11}
\sup\limits_{\substack{i : |\mu_i| \leq 1/T_1 \\ j \in [0, n-1]}} {\mathsf D} \Ik 
(T_j^\prime < |x_i| \leq T_j^\prime + d_n) <{}\\
{}< c_7 (\ln n)^{(1-p)/2} \eta_n^p.
\end{multline}
Пусть $n > N(\eps)$ настолько, что 
$$
\fr{\eps\sqrt{n} - 5 T_U^2}{2} > \fr{\eps \sqrt{n} }{4}\,.
$$
%
 Тогда для пятого слагаемого в~(\ref{M1}) по лемме~4 с~$m \hm= 
\sqrt{n} (\ln n)^2$ и~$k \hm= n\hm-[n\eta_n^p T_1^p]$ имеем
\begin{multline}
\p \left(\sup\limits_{\substack{j \in [0, n-1] \\ T\in 
[T_j^{\prime},T_j^{\prime}+d_n]}} |Z_2(T)-Z_2(T_j^{\prime})| > 
\eps\sqrt{n}\right) \leq{}\\
{}\leq  \sum\limits_{j \in [0, n-1]} \p \left(  \left|N_2 (T_j^\prime, 
T_j^\prime+d_n) -{}\right.\right.\\
\left.\left.{}- {\mathsf E} N_2 (T_j^\prime, T_j^\prime+d_n)\right| > \fr{\eps\sqrt{n}}{4} 
\right) \leq{}\\
{}\leq  4n \exp \left\{ -  \fr{\eps^2 n^{3/2} (\ln n)^2}{(n-[n\eta_n^p T_1^p])^{-1}}\Bigg/ 
\big( 16 \eps \sqrt{n} +{}\right.\\
\left.{}+ 64 c_7 (\ln n)^{(1-p)/2} \eta_n^p (n-[n\eta_n^p 
T_1^p]) \big) 
\vphantom{\fr{\eps^2 n^{3/2} (\ln n)^2}{(n-[n\eta_n^p T_1^p])^{-1}}}
\right\} +{}\\
{}+ 22 \left(1+\fr{16 (n-[n\eta_n^p T_1^p])}{\eps \sqrt{n}}\right)^{1/2}\times{}\\
{}\times 
n^{3/2} (\ln n)^2 \alpha\left(\left[\fr{n-[n\eta_n^p T_1^p]}{2 (\ln n)^2 
\sqrt{n}}\right]\right).
\label{M6}
\end{multline}
Используя ограничения $n^{-\delta_1}\hm\leq \eta_n^p\hm \leq n^{-\delta_2}$ 
и~$1/2\hm<\delta_2\hm<\delta_1<1$, из~(\ref{M6}) получим для любого $\eps\hm>0$
$$
\p \left(\sup\limits_{\substack{j \in [0, n-1] \\ T\in 
[T_j^{\prime},T_j^{\prime}+d_n]}} |Z_2(T)-Z_2(T_j^{\prime})| > 
\eps\sqrt{n}\right) \to 0
$$
при $n \to \infty$.

Таким образом, показано, что для любого $\eps>0$ все слагаемые в~(\ref{M1}) 
стремятся к~нулю при $n\to\infty$. Следовательно,
$$
\fr{|U(\hat{t}_F)|}{\sqrt{n}}  \overset{\, \p \, }{\to} 0 \,,
$$
что вместе с~(\ref{D0}) завершает доказательство тео\-ремы.~\hfill$\square$

\smallskip

Следующая теорема дает достаточные условия для асимптотической нормальности 
оценки риска $\hat{R}(\hat{t}_F)$ в~случае $\mu \hm\in l_0[\eta_n]$.

\smallskip

\noindent
\textbf{Теорема 2.}\ 
\textit{Пусть $\mu \hm\in l_0[\eta_n]$, $\eta_n\hm\in[n^{-\delta_1}, n^{-\delta_2}]$, $1/2\hm < 
\delta_2\hm < \delta_1\hm<1;$ имеются такие константы $c_1, c_2\hm>0$, что для 
коэффициента сильного перемешивания $\alpha(\cdot)$ компонент вектора~$x$ 
справедливо} 
\begin{gather*}
\alpha(k) \leq c_1 k^{-1-(5/2)\delta_1/(1\hm-\delta_1)\hm-c_2},\enskip 
k=\overline{1,n-1};\\
 q_n < c_3 < 1;\enskip \mathrm{lim\,inf} q_n \ln n = c_4 > 0;
\end{gather*}
\textit{для максимального коэффициента корреляции~$\rho(\cdot)$ компонент вектора~$x$ 
справедливо}
$$
\sum\limits_{k = 1}^{\infty} \sup\limits_{n\geq k+1} \rho(k) \equiv 
\sum\limits_{k = 1}^{\infty}  \rho^\star (k) = c_5 < \infty. 
$$
\textit{Тогда при $n \to \infty$}
$$
\fr{\hat{R}(\hat{t}_F) - R(T_m)}{C_\rho \sqrt{2n}} \Rightarrow N(0, 1),
$$
\textit{где}
$$
C_\rho = \sigma^2\sqrt{1 +   \lim\limits_{n\to\infty} \fr{1}{n} 
\sum\limits_{j\neq i} \mathrm{corr}^2 (x_i, x_j)}\,.
$$

\noindent
Д\,о\,к\,а\,з\,а\,т\,е\,л\,ь\,с\,т\,в\,о\  проводится аналогично доказательству теоремы~1. 
Переменная~$D_n$ теперь определяется как $D_n \hm= \{(i,j) : 
|\mu_i|\hm=|\mu_j|=0$, $j\hm\neq i\}$. Условия вида $|\mu_i|\hm<T_1^{-1}$ (вида 
$|\mu_i|\hm\geq T_1^{-1}$) заменяются условиями  $\mu_i\hm=0$ (соответственно 
$|\mu_i|\hm>0$).
Поскольку $\mu \hm\in l_0[\eta_n]$, количество~$i$ таких, что $|\mu_i|\hm>0$ 
(а~значит, и~число слагаемых в~$Z_1(T)$), не превышает~$[n \eta_n]$.

Для оценки первого слагаемого в~(\ref{M1}) используется лемма~2, 
в~которой можно взять, например, $b\hm=1/2$, а~для~$\kappa_n^0$ использовать оценку 
$\kappa_n^0 \hm> n \eta_n$. Формулы (\ref{M3}),  (\ref{M10}) и~(\ref{M11}) 
принимают вид соответственно
\begin{align*}
\sup\limits_{\substack{i: \mu_i =0 \\ T\in [t_{\kappa_n^0}, T_U]}} |{\mathsf E} H_i (T, 
T_m)| & \leq c_8 (\ln n)^{3/2} \eta_n ;
\\
\sup\limits_{\substack{i: \mu_i =0 \\ T\in [t_{\kappa_n^0}, T_U]}} {\mathsf D} H_i (T, 
T_m)  & \leq 2 c_8 (\ln n)^{5/2} \eta_n;
\\
\sup\limits_{\substack{i : \mu_i =0 \\ j \in [0, n-1]}} {\mathsf D} \Ik (T_j^\prime < 
|x_i| \leq T_j^\prime + d_n) &< c_8 (\ln n)^{1/2} \eta_n,
\end{align*}
где $c_8 = c_8(c_3,\delta_1, \delta_2, c_4)$. В~остальном доказательство 
аналогично.~\hfill$\square$

\section{Сильная состоятельность оценки риска при~применении FDR-процедуры 
в~условиях слабой зависимости}

Следующая теорема дает достаточные условия для сильной состоятельности оценки 
риска $\hat{R}(\hat{t}_F)$ в~случаях $\mu \hm\in m_p[\eta_n]$ и~$\mu \hm\in 
l_0[\eta_n]$.

\smallskip

\noindent
\textbf{Теорема 3.}
\textit{Пусть $\mu\hm \in m_p[\eta_n]$, $\eta_n^p\hm\in[n^{-\delta_1}, n^{-\delta_2}]$ либо 
$\mu \hm\in l_0[\eta_n]$, $\eta_n\hm\in[n^{-\delta_1}, n^{-\delta_2}]$; $0 \hm< \delta_2 
\hm< \delta_1<1$; имеются такие константы $c_1, c_2\hm>0$, что для коэффициента 
сильного перемешивания $\alpha(\cdot)$ компонент вектора~$x$ справедливо}  
$\alpha(k) \hm\leq c_1 k^{-2-(7/2)\delta_1/(1\hm-\delta_1)\hm-c_2}$, $k\hm=\overline{1,n-1}$; 
$q_n \hm< c_3 \hm< 1$; $\mathrm{lim\,inf} q_n \ln n \hm= c_4 \hm> 0$. \textit{Тогда при} $n \hm\to \infty$
$$
\fr{\hat{R}(\hat{t}_F) - R(T_m)}{n} \rightarrow 0 \, \, \,\textit{п.~в.}
$$


\noindent
Д\,о\,к\,а\,з\,а\,т\,е\,л\,ь\,с\,т\,в\,о\,.  Воспользуемся представлением~(\ref{D00}).

Покажем, что $(\hat{R}(T_m)-R(T_m))n^{-1}\hm \to 0$ п.~в.\ при $n\hm\to\infty$. 
При мягкой пороговой обработке ${\mathsf E} \hat{R}(T_m) \hm= R(T_m)$, а~при жесткой 
пороговой обработке
\begin{multline*}
\fr{\hat{R}(T_m)-R(T_m)}{n} = {}\\
{}=\fr{\hat{R}(T_m)-{\mathsf E} \hat{R}(T_m)}{n} 
+\fr{{\mathsf E}\hat{R}(T_m)-R(T_m)}{n}\,,
\end{multline*}
где второе слагаемое стремится к~нулю при $n\to\infty$ \cite{Mallat}. 
Следовательно, достаточно показать, что $(\hat{R}(T_m)\hm-{\mathsf E}\hat{R}(T_m))n^{-1} \hm\to 0$ п.~в.

Полагая в~лемме~3 $X_i \hm= F[x_i, T_m] \hm- {\mathsf E} F[x_i, T_m]$, $b \hm= 
2(\sigma^2\hm+T_m^2)$ и~$m \hm= n^{1/4}$ и~учитывая ограничения на $\alpha(\cdot)$ из 
условия, нетрудно убедиться, что для всех~$n$
$$
\p \left(\left| \fr{\hat{R}(T_m)-{\mathsf E} \hat{R}(T_m)}{n}\right| >\eps \right) 
\leq \fr{c_5}{n^{1+c_6}}\,, 
$$
где константы $c_5$, $c_6$ положительны. Отсюда
$$
\sum\limits_{n=1}^{\infty}\p \left(\left|\fr{\hat{R}(T_m)-{\mathsf E} 
\hat{R}(T_m)}{n}\right| >\eps \right) < \infty,
$$
и по теореме~1.3.4 из~\cite{Serfling2002} 
$$
\left(\hat{R}(T_m)-{\mathsf E}\hat{R}(T_m)\right)n^{-1} \to 0~\mbox{п.~в.}
$$



Покажем теперь, что  $U(\hat{t}_F) \, n^{-1}\hm \to 0$ п.~в. Доказательство 
проведено для $\mu \hm\in m_p[\eta_n]$, в~случае $\mu\hm \in l_0[\eta_n]$ 
доказательство аналогично.
Аналогично формуле~(\ref{M1}), для произвольного $\eps\hm>0$ в~терминах тео\-ре\-мы~1 имеем
\begin{multline*}
\p \left( \fr{|U(\hat{t}_F)|}{n}> 4\eps \right) \leq \p\left(\hat{t}_F 
\leq t_{\kappa_n}\right) +{}\\
{}+ \p\left(\fr{\sup\nolimits_{T\in [t_{\kappa_n}, T_U]} |{\mathsf E} 
U(T)|}{n}>\eps\right)+{}\\
{}+ \p \left(\sup\limits_{T\in [t_{\kappa_n}, T_U]} |Z_1(T)| > \eps n\right) +{}
\end{multline*}

\noindent
\begin{multline}
{}+ \p  \left(\sup\limits_{j \in [0, n-1]} |Z_2(T_j^{\prime})| > \eps n\right) +{}\\
{}+ \p \left(\sup\limits_{\substack{j \in [0, n-1] \\ T\in 
[T_j^{\prime},T_j^{\prime}+d_n]}} |Z_2(T)-Z_2(T_j^{\prime})| > \eps n\right).
\label{M1SC}
\end{multline}
Применяя рассуждения, аналогичные приведенным в~доказательстве теоремы~1, можно показать, что
$$
\sup\limits_{T\in [t_{\kappa_n}, T_U]} |{\mathsf E} U(T)| = o(n); \enskip
\sup\limits_{T\in [t_{\kappa_n}, T_U]} |Z_1(T)|  = o(n),
$$
откуда следует, что второе и~третье слагаемые в~(\ref{M1SC}) обращаются в~ноль 
для всех достаточно больших~$n$.

Для некоторых положительных констант  $c_7$ и~$c_8$ первое, четвертое и~пятое 
слагаемые  в~(\ref{M1SC}) не превышают $c_7 n^{-1-c_8}$ для всех достаточно 
боль\-ших~$n$, что можно показать с~помощью ограничения на $\alpha(\cdot)$ из 
условия и~рассуждений, аналогичных приведенным при выводе соответственно формул~(\ref{M1next}), (\ref{M5}) и~(\ref{M6}), с~тем отличием, что при применении 
леммы~4 полагается $m \hm= (\ln n)^3$.

Из доказанного следует, что
$$
\sum\limits_{n=1}^{\infty}\p \left( \fr{|U(\hat{t}_F)|}{n}> 4\eps \right) 
< \infty,
$$
и по теореме~1.3.4 из~\cite{Serfling2002} $U(\hat{t}_F) \, n^{-1} \to 0$ п.~в., 
что завершает доказательство теоремы.~\hfill$\square$



{\small\frenchspacing
 {\baselineskip=11.5pt
 %\addcontentsline{toc}{section}{References}
 \begin{thebibliography}{99}
\bibitem{FDRImage}
\Au{Krylov V.\,A., Moser~G., Serpico~S.\,B., Zerubia~J.}
False discovery rate approach to unsupervised image change detection~// IEEE 
T. Image Process., 2016. Vol.~25. No.\,10. P.~4704--4718. doi: 10.1109/TIP.2016.2593340.

\bibitem{MultipleTesting} %2
\Au{Menyhart~O., Weltz~B., Gyorffy~B.}
MultipleTesting.com: A~tool for life science researchers for multiple hypothesis 
testing correction~// PLoS One, 2021. Vol.~16. No.\,6. Art.~0245824. doi: 10.1371/journal.pone.0245824.

\bibitem{AdaptingFDR} %3
\Au{Abramovich~F., Benjamini~Y., Donoho~D., Johnstone~I.}
Adapting to unknown sparsity by controlling the false discovery rate~// Ann. Stat., 2006. Vol.~34. No.\,2. P.~584--653.
doi: 10.1214/009053606000000074.

\bibitem{ZasShe17} %4
\Au{Заспа~А.\,Ю., Шестаков~О.\,В.}
Состоятельность оценки риска при множественной проверке гипотез с~FDR-по\-ро\-гом~// 
Вестник ТвГУ. Сер. Прикладная математика, 2017. Вып.~1. С.~5--16.
doi: 10.26456/vtpmk119. EDN: YFYJXT.

\bibitem{Mathematics2020} %5
\Au{Palionnaya~S.\,I., Shestakov~O.\,V.}
Asymptotic properties of MSE estimate for the false discovery rate controlling 
procedures in multiple hypothesis testing // Mathematics, 2020. Vol.~8. No.~11. 
Art.~1913. 11~p. doi: 10.3390/ math8111913.

\bibitem{Shestakov2021-1} %6
\Au{Шестаков~О.\,В.}
Анализ несмещенной оценки среднеквадратичного риска метода блочной пороговой 
обработки~// Информатика и~её применения, 2021. Т.~15. Вып.~2. С.~30--35.
doi: 10.14357/19922264210205. EDN: DSQQAU.

\bibitem{Shestakov2021-2} %7
\Au{Шестаков~О.\,В.}
Пороговые функции в~методах подавления шума, основанных на вейв\-лет-раз\-ло\-же\-нии 
сигнала~// Информатика и~её применения, 2021. Т.~15. Вып.~3. С.~51--56.
doi: 10.14357/19922264210307. EDN: WSEAYG.

\bibitem{Shestakov2022} %8
\Au{Шестаков~О.\,В.}
Несмещенная оценка риска пороговой обработки с~двумя пороговыми значениями~// 
Информатика и~её применения, 2022. Т.~16. Вып.~4. С.~14--19.
doi: 10.14357/19922264220403. EDN: \mbox{DZBVLC}.

\bibitem{ResultsOnFDRUnderDependence} %9
\Au{Farcomeni~A.}
Some results on the control of the false discovery rate under dependence~// 
Scand. J. Stat., 2007. Vol.~34. No.\,2. P.~275--297.
doi: 10.1111/j.1467-9469.2006.00530.x.

\bibitem{VorontsovShestakov2023} %10
\Au{Воронцов~М.\,О., Шестаков~О.\,В.}
Среднеквадратичный риск FDR-про\-це\-ду\-ры в~условиях слабой за\-ви\-си\-мости~// 
Информатика и~её применения, 2023. Т.~17. Вып.~2. С.~34--40.
doi: 10.14357/19922264230205. EDN: AVJZDX.

\bibitem{Vorontsov2024} %11
\Au{Воронцов~М.\,О.}
Анализ среднеквадратичного риска при использовании методов множественной 
проверки гипотез для выбора параметров пороговой обработки в~условиях слабой 
зависимости~// Вестник Московского университета. Сер. 15: Вычислительная 
математика и~кибернетика, 2024. №\,2. С.~18--24.

\bibitem{Bosq} %12
\Au{Bosq~D.}
Nonparametric statistics for stochastic processes: Estimation and prediction.~--- 
Lecture notes in statistics ser.~--- New York, NY, USA: Springer, 1996. Vol.~110. 
188~p.

\bibitem{Mallat} %13
\Au{Mallat~S.}
A wavelet tour of signal processing.~--- New York, NY, USA: Academic Press, 1999. 
857~p.

\bibitem{spatialAdaptation} %14
\Au{Donoho~D., Johnstone~I.}
Ideal spatial adaptation via wavelet shrinkage~// Biometrika, 1994. Vol.~81. 
No.\,3. P.~425--455. doi: 10.1093/biomet/81.3.425.

\bibitem{AdaptingSURE} %15
\Au{Donoho D., Johnstone I.\,M.}
Adapting to unknown smoothness via wavelet shrinkage~// J.~Amer. Stat. Assoc., 
1995. Vol.~90. P.~1200--1224.

\bibitem{ExactRisk} %16
\Au{Marron J.\,S., Adak~S., Johnstone~I.\,M., Neumann~M.\,H., Patil~P.}
Exact risk analysis of wavelet regression~// J.~Comput. Graph. Stat., 1998. 
Vol.~7. P.~278--309. doi: 10.1080/ 10618600.1998.10474777.

\bibitem{Jansen} %17
\Au{Jansen~M.}
Noise reduction by wavelet thresholding.~-- Lecture notes in statistics ser.~--- 
New York, NY, USA: Springer, 2001. Vol.~161. 217~p.

\bibitem{KuShe2016_1} %18
\Au{Кудрявцев~А.\,А., Шестаков~О.\,В.}
Асимптотическое поведение порога, минимизирующего усредненную\linebreak вероятность ошибки 
вычисления вейв\-лет-ко\-эф\-фи\-ци\-ен\-тов~// Докл. Акад. наук, 2016. Т.~468. №\,5. 
С.~487--491.

\bibitem{KuShe2016_2} %19
\Au{Кудрявцев~А.\,А., Шестаков~О.\,В.}
Асимптотически оптимальная пороговая обработка вейв\-лет-ко\-эф\-фи\-ци\-ен\-тов в~моделях с~негауссовым распределением шума~// Докл. Акад. наук, 2016. Т.~471. №\,1. 
С.~11--15.



\bibitem{Eroshenko} %20
\Au{Ерошенко~А.\,А.}
Статистические свойства оценок сигналов и~изображений при пороговой обработке 
коэффициентов в~вейв\-лет-раз\-ло\-же\-ни\-ях: Дис.\ \ldots\ канд. физ.-мат. наук.~--- 
М.: МГУ, 2015. 82~с.

\bibitem{Peligrad} %21
\Au{Peligrad~M.}
On the asymptotic normality of sequences of weak dependent random variables~// 
J. Theor. Probab., 1996. Vol.~9. No.\,3. P.~703--715. doi: 10.1007/BF02214083.

\bibitem{Serfling2002} %22
\Au{Serfling~R.\,J.}
Approximation theorems of mathematical statistics.~--- New York, NY, USA: John Wiley \&~Sons, Inc., 2002. 371~p.

\end{thebibliography}

 }
 }

\end{multicols}

\vspace*{-6pt}

\hfill{\small\textit{Поступила в~редакцию 21.05.24}}

\vspace*{8pt}

%\pagebreak

%\newpage

%\vspace*{-28pt}

\hrule

\vspace*{2pt}

\hrule



\def\tit{ASYMPTOTIC NORMALITY AND STRONG CONSISTENCY\\ OF~RISK ESTIMATE WHEN USING THE~FDR THRESHOLD\\ UNDER WEAK DEPENDENCE CONDITION}


\def\titkol{Asymptotic normality and strong consistency of~risk estimate when using the~FDR threshold under weak dependence condition}


\def\aut{M.\,O.~Vorontsov$^{1,2}$ and~O.\,V.~Shestakov$^{1,2,3}$}

\def\autkol{M.\,O.~Vorontsov and~O.\,V.~Shestakov}

\titel{\tit}{\aut}{\autkol}{\titkol}

\vspace*{-13pt}


\noindent
$^{1}$Department of Mathematical Statistics, Faculty of Computational Mathematics and Cybernetics,
 M.\,V.~Lo\-mo-\linebreak
 $\hphantom{^1}$nosov Moscow State University, 1-52~Leninskie Gory, GSP-1, Moscow 119991, Russian Federation

\noindent
$^{2}$Moscow Center for Fundamental and Applied Mathematics, M.\,V.~Lomonosov Moscow State University,\linebreak
$\hphantom{^1}$1~Leninskie Gory, GSP-1, Moscow 119991, Russian Federation

\noindent
$^{3}$Federal Research Center ``Computer Science and Control'' of the Russian Academy of Sciences, 44-2~Vavilov\linebreak
$\hphantom{^1}$Str., Moscow 119333, Russian Federation


\def\leftfootline{\small{\textbf{\thepage}
\hfill INFORMATIKA I EE PRIMENENIYA~--- INFORMATICS AND
APPLICATIONS\ \ \ 2024\ \ \ volume~18\ \ \ issue\ 3}
}%
 \def\rightfootline{\small{INFORMATIKA I EE PRIMENENIYA~---
INFORMATICS AND APPLICATIONS\ \ \ 2024\ \ \ volume~18\ \ \ issue\ 3
\hfill \textbf{\thepage}}}

\vspace*{2pt}






\Abste{An approach to solving the problem of noise removal in a large array of sparse data is considered
 based on the method of controlling the average proportion of false hypothesis rejections (False Discovery Rate, FDR). 
 This approach is equivalent to threshold processing procedures that remove array components whose values do not exceed 
 some specified threshold. The observations in the model are considered weakly dependent. To control the\linebreak\vspace*{-12pt}}
 
 \Abstend{degree of dependence, 
 restrictions on the strong mixing coefficient and the maximum correlation coefficient are used. The mean-square risk is 
 used as a measure of the effectiveness of the considered approach. It is possible to calculate the risk value only on the test data;
  therefore, its statistical estimate is considered in the work and its properties are investigated. The asymptotic normality and
   strong consistency of the risk estimate are proved when using the FDR threshold under conditions of weak dependence in the data.}

\KWE{thresholding; multiple hypothesis testing; risk estimate}

\DOI{10.14357/19922264240309}{ZOQVTO}

%\vspace*{-12pt}


    
   %   \Ack

%\vspace*{-3pt}
%\noindent



  \begin{multicols}{2}

\renewcommand{\bibname}{\protect\rmfamily References}
%\renewcommand{\bibname}{\large\protect\rm References}

{\small\frenchspacing
 {\baselineskip=10.8pt
 \addcontentsline{toc}{section}{References}
 \begin{thebibliography}{99} 

%1
\bibitem{FDRImage-1}
\Aue{Krylov, V.\,A., G.~Moser, S.\,B.~Serpico, and J.~Zerubia.} 2016. 
False discovery rate approach to unsupervised image change detection. 
\textit{IEEE T. Image Process.} 25(10):4704--4718. doi: 10.1109/TIP.2016.2593340.

%2
\bibitem{MultipleTesting-1}
\Aue{Menyhart, O., B.~Weltz, and B.~Gyorffy.} 2021. 
MultipleTesting.com: A~tool for life science researchers for multiple hypothesis testing correction. 
\textit{PLoS One} 16(6):0245824. 
doi: 10.1371/journal.pone.0245824.

%3
\bibitem{AdaptingFDR-1}
\Aue{Abramovich, F., Y.~Benjamini, D.~Donoho, and I.\,M.~Johnstone.} 2006. 
Adapting to unknown sparsity by controlling the false discovery rate. 
\textit{Ann. Stat.} 34(2):584--653. 
doi: 10.1214/009053606000000074.


%4
\bibitem{ZasShe17-1}
\Aue{Zaspa, A.\,Yu., and O.\,V.~Shestakov.} 2017.
Sostoyatel'nost' otsenki riska pri mnozhestvennoy proverke gipotez s~FDR-porogom
 [Consistency of the risk estimate of the multiple hypothesis testing with the FDR threshold]. 
\textit{Vestnik TvGU. Ser.: Prikladnaya matematika} [Herald of Tver State University. Ser. Applied Mathematics] 1:5--16.
doi: 10.26456/vtpmk119. EDN: YFYJXT.

%5
\bibitem{Mathematics2020-1}
\Aue{Palionnaya, S.\,I., and O.\,V.~Shestakov.} 2020. 
Asymptotic properties of MSE estimate for the false discovery rate controlling procedures in multiple hypothesis testing. 
\textit{Mathematics} 8(11):1913. 11~p.
doi: 10.3390/math8111913.

%6
\bibitem{Shestakov2021-1-1}
\Aue{Shestakov, O.\,V.} 2021.
Analiz nesmeshchennoy otsenki srednekvadratichnogo riska metoda blochnoy po\-ro\-go\-voy obrabotki 
[Analysis of the unbiased mean-square risk estimate of the block thresholding method]. 
\textit{Informatika i~ee Primeneniya~--- Inform. Appl.} 15(2):30--35.
doi: 10.14357/19922264210205. EDN: DSQQAU.

%7
\bibitem{Shestakov2021-2-1}
\Aue{Shestakov, O.\,V.} 2021.
Porogovye funktsii v~metodakh podavleniya shuma, osnovannykh na veyvlet-razlozhenii signala 
[Thresholding functions in the noise suppression methods based on the wavelet expansion of the signal]. 
\textit{Informatika i~ee Primeneniya~--- Inform. Appl.} 15(3):51--56.
doi: 10.14357/19922264210307. EDN: WSEAYG.

%8
\bibitem{Shestakov2022-1}
\Aue{Shestakov, O.\,V.} 2022.
Nesmeshchennaya otsenka riska porogovoy obrabotki s dvumya porogovymi znacheniyami [Unbiased thresholding risk estimate with two threshold values]. 
\textit{Informatika i~ee Primeneniya~--- Inform. Appl.} 16(4):14--19.
doi: 10.14357/19922264220403. EDN: DZBVLC.

%9
\bibitem{ResultsOnFDRUnderDependence-1}
\Aue{Farcomeni, A.} 2007. Some results on the control of the false discovery rate under dependence. 
\textit{Scand. J. Stat.} 34(2):275--297. 
doi: 10.1111/j.1467-9469.2006.00530.x.

%10
\bibitem{VorontsovShestakov2023-1}
\Aue{Vorontsov, M.\,O., and O.\,V.~Shestakov.} 2023.
Sred\-ne\-kvad\-ra\-tich\-nyy risk FDR-protsedury v~usloviyakh slaboy za\-vi\-si\-mosti [Mean-square risk of the FDR procedure under weak dependence]. 
\textit{Informatika i~ee Primeneniya~--- Inform. Appl.} 17(2):34--40.
doi: 10.14357/19922264230205. EDN: AVJZDX.

%11
\bibitem{Vorontsov2024-1}
\Aue{Vorontsov, M.\,O.} 2024. 
RMS risk analysis when using multiple hypothesis testing select parameters of thresholding under conditions of weak dependence. 
\textit{Moscow University Computational Mathematics Cybernetics} 48:91--97. 
doi: 10.3103/S027864192470002X.

%12
\bibitem{Bosq-1}
\Aue{Bosq, D.} 1996. 
\textit{Nonparametric statistics for stochastic processes: Estimation and prediction}. 
Lecture notes in statistics ser. New York, NY: Springer Verlag. Vol.~110. 188~p.

%13
\bibitem{Mallat-1}
\Aue{Mallat, S.} 1999. 
\textit{A wavelet tour of signal processing}. New York, NY: Academic Press. 857~p.

%14
\bibitem{spatialAdaptation-1}
\Aue{Donoho, D., and I.\,M.~Johnstone.} 1994. 
Ideal spatial adaptation via wavelet shrinkage. 
\textit{Biometrika} 81(3):425--455. doi: 10.1093/biomet/81.3.425.

%15
\bibitem{AdaptingSURE-1}
\Aue{Donoho, D., and I.\,M.~Johnstone.} 1995. 
Adapting to unknown smoothness via wavelet shrinkage. 
\textit{J. Am. Stat. Assoc.} 90(432):1200--1224. doi: 10.1080/01621459. 1995.10476626.

%16
\bibitem{ExactRisk-1}
\Aue{Marron, J.\,S., S.~Adak, I.\,M.~Johnstone, M.\,H.~Neumann, and P.~Patil.} 1998. 
Exact risk analysis of wavelet regression. 
\textit{J.~Comput. Graph. Stat.} 7(3):278-309. doi: 10.1080/ 10618600.1998.10474777.

%17
\bibitem{Jansen-1}
\Aue{Jansen, M.} 2001. 
\textit{Noise reduction by wavelet thresholding}. Lecture notes in statistics ser. New York, NY: Springer Verlag. Vol.~161. 217~p.

%18
\bibitem{KuShe2016_1-1}
\Aue{Kudryavtsev, A.\,A., and O.\,V.~Shestakov.} 2016. 
Asymptotic behavior of the threshold minimizing the average probability of error in calculation of wavelet coefficients. 
\textit{Dokl. Math.} 93(3):295--299.
doi: 10.1134/S1064562416030212. EDN: WUMUEV. 

%19
\bibitem{KuShe2016_2-1}
\Aue{Kudryavtsev, A.\,A., and O.\,V.~Shestakov.} 2016. 
Asymptotically optimal wavelet thresholding in the models with non-Gaussian noise distributions. 
\textit{Dokl. Math.} 94(3):615--619.
doi: 10.1134/S1064562416060028. EDN: YUYVUP.




%20
\bibitem{Eroshenko-1}
\Aue{Eroshenko, A.\,A.} 2015. Statisticheskie svoystva otsenok signalov i~izobrazheniy pri porogovoy obrabotke ko\-ef\-fi\-tsi\-en\-tov 
v~veyvlet-razlozheniyakh 
[Statistical properties of signal and image estimates under thresholding of coefficients in wavelet decompositions]. Moscow: MSU. PhD Diss. 82~p.

%21
\bibitem{Peligrad-1}
\Aue{Peligrad, M.} 1996. 
On the asymptotic normality of sequences of weak dependent random variables. 
\textit{J. Theor. Probab.} 9(3):703--715. doi: 10.1007/BF02214083.

%22
\bibitem{Serfling2002-1}
\Aue{Serfling, R.\,J.} 2002. 
\textit{Approximation theorems of mathematical statistics}. New York, NY: John Wiley \&~Sons. 371~p.
\end{thebibliography}

 }
 }

\end{multicols}

\vspace*{-6pt}

\hfill{\small\textit{Received May 21, 2024}} 

%\vspace*{-18pt}

\Contr

\vspace*{-3pt}


\noindent
\textbf{Vorontsov Mikhail O.} (b.\ 1996)~--- PhD student, Department of Mathematical Statistics, 
Faculty of Computational Mathematics and Cybernetics, M.\,V.~Lomonosov Moscow State University, 1-52~Leninskie Gory, GSP-1, Moscow 119991, Russian Federation;  
mathematician, Moscow Center for Fundamental and Applied Mathematics, M.\,V.~Lomonosov Moscow State University, 1~Leninskie Gory, GSP-1, Moscow 119991, Russian Federation;
\mbox{m.vtsov@mail.ru}

\vspace*{6pt}

\noindent
\textbf{Shestakov Oleg V.} (b.\ 1976)~--- Doctor of Science in physics and mathematics, professor, Department of Mathematical Statistics,
 Faculty of Computational Mathematics and Cybernetics, M.\,V.~Lomonosov Moscow State University, 1-52~Leninskie Gory, GSP-1, Moscow 119991, Russian Federation; 
 senior scientist, Federal Research Center ``Computer Science and Control'' of the Russian Academy of Sciences, 44-2~Vavilov Str., Moscow 119333, 
 Russian Federation; leading scientist, Moscow Center for Fundamental and Applied Mathematics, M.\,V.~Lomonosov Moscow State University, 
 1~Leninskie Gory, GSP-1, Moscow 119991, Russian Federation; \mbox{oshestakov@cs.msu.su}


\label{end\stat}

\renewcommand{\bibname}{\protect\rm Литература}  %1
%\newcommand {\ff}{{\mathcal F}}
\newcommand {\ebd}{\triangleq}
\newcommand{\me}[2]{\mathbf{E}_{ #1 }\left\{ \mathop{#2} \right\} }



\def\stat{borisov}

\def\tit{ФИЛЬТРАЦИЯ СОСТОЯНИЙ МАРКОВСКИХ СКАЧКООБРАЗНЫХ ПРОЦЕССОВ 
ПО~ДИСКРЕТИЗОВАННЫМ НАБЛЮДЕНИЯМ$^*$}

\def\titkol{Фильтрация состояний марковских скачкообразных процессов 
по~дискретизованным наблюдениям}

\def\aut{А.\,В.~Борисов$^1$}

\def\autkol{А.\,В.~Борисов}

\titel{\tit}{\aut}{\autkol}{\titkol}

\index{Борисов А.\,В.}
\index{Borisov A.\,A.}




{\renewcommand{\thefootnote}{\fnsymbol{footnote}} \footnotetext[1]
{Работа выполнена при частичной поддержке РФФИ (проект 16-07-00677).}}


\renewcommand{\thefootnote}{\arabic{footnote}}
\footnotetext[1]{Институт проблем информатики Федерального исследовательского центра <<Информатика 
и~управление>> Российской академии наук,
\mbox{aborisov@frccsc.ru}}

%\vspace*{8pt}



\Abst{Статья посвящена решению задачи оптимальной 
фильтрации состояний однородного марковского скачкообразного процесса (МСП). 
Наблюдения представляют собой приращения случайных процессов~--- интегральных 
преобразований состояний, зашумленные винеровскими процессами, интенсивность 
которых также зависит от оцениваемого состояния. Оптимальная оценка в~моменты 
получения нового наблюдения вычисляется как функция предыдущей оценки и~новых 
наблюдений, а~между моментами наблюдений~--- простейшим прогнозом в~силу системы 
уравнений Колмогорова. Рекуррентная формула пересчета ресурсозатратна, так как 
содержит  интегралы~--- мас\-штаб\-но-сдви\-го\-вые смеси многомерных гауссиан, 
где в~качестве смешивающих выступают распределения времени пребывания 
состояния в~каждом из возможных значений. Предложены более простые аппроксимации, 
основанные на предположении об ограниченности числа скачков состояния за время между 
наблюдениями. Получены универсальные локальная и~глобальная характеристики точности 
аппроксимаций, зависящие от па\-ра\-мет\-ров оцениваемого процесса, величины 
временн$\acute{\mbox{о}}$го шага  между наблюдениями и~максимального числа учитываемых скачков.}

\KW{марковский скачкообразный процесс; оптимальная фильтрация; мультипликативные 
шумы в~наблюдениях; стохастическое дифференциальное уравнение; численная аппроксимация}

\DOI{10.14357/19922264180316}
  
%\vspace*{4pt}


\vskip 10pt plus 9pt minus 6pt

\thispagestyle{headings}

\begin{multicols}{2}

\label{st\stat}



 \section{Введение}
 
 Фильтр Вонэма~\cite{Won_65}~--- один из редких удачных случаев, когда 
 оценка оптимальной фильтрации состо\-яния стохастической системы наблюдения 
 выражается в~виде решения некоторой замк\-ну\-той\linebreak конечномерной сис\-те\-мы 
 стохастических дифференциальных уравнений. 
 
 Алгоритм данного фильт\-ра 
 позволяет вычислить оценку фильт\-ра\-ции со\-сто\-яния \textit{марковского скачкообразного 
 процесса} с~\mbox{конечным} множеством состояний по наблюдениям в~присутствии 
 аддитивных винеровских шумов. Теоретически оптимальная оценка со\-сто\-яния~--- 
 его условное распределение в~текущий момент времени~--- 
 обладает очевидными свойствами неотрицательности и~нормировки. 
 При чис\-лен\-ной реализации данного фильтра классическим методом 
 Эй\-ле\-ра--Ма\-ру\-ямы~\cite{KP_92} данные свойства могут не сохраняться и~процедура 
 вы\-чис\-ле\-ния становится неустойчивой.  В~связи с~этим обстоятельством разрабатывались 
 другие алгоритмы чис\-лен\-но\-го решения уравнения фильтра Вонэма, обладающие 
 требуемыми свойствами устойчивости (см.~\cite{YZL_04, PR_10} и~библиографию в~них). 
 В~час\-ти этих работ доказана лишь слабая сходимость пред\-ла\-га\-емых аппроксимационных 
 схем к~оценке фильт\-ра Вонэма, в~то время как ка\-кая-ли\-бо 
 характеризация точ\-ности этих приближений отсутствует.
 
 В~\cite{B_18} было представлено распространение фильт\-ра Вонэма на случай 
 наблюдений с~мультипликативными шумами. При этом уравнение обобщенного 
 фильт\-ра содержит в~правой части квадратическую характеристику шумов в~наблюдениях. 
 Данный процесс на практике никогда не наблюдается непосредственно, а~является лишь 
 некоторым нелинейным интегральным преобразованием наблюдений. Очевидно, что 
 имеющиеся в~настоящий момент времени алгоритмы приближенного вычисления оценки 
 фильтрации Вонэма для данной системы не подходят. 
 
 Целью предлагаемой работы является ис\-поль\-зование результатов оптимальной 
 фильтрации со\-стояний сис\-тем с~дискретным временем для аппроксимации решения 
 аналогичной задачи для\linebreak стохастических дифференциальных сис\-тем. 
 
 Статья организована следующим образом. Раздел~2 содержит формальную постановку 
 задачи фильт\-ра\-ции со\-сто\-яний однородного МСП с~конечным множеством со\-сто\-яний 
 по наблюдениям, полученным путем временн$\acute{\mbox{о}}$й дискретизации процессов с~непрерывным 
 временем~--- интегральных преобразований со\-сто\-яния сис\-те\-мы в~присутствии 
 мультипликативных винеровских шумов.\linebreak
  В~разд.~3 пред\-став\-ле\-но решение поставленной 
 задачи фильт\-ра\-ции: пересчет оценок со\-сто\-яний в~момент получения новых 
 дискретизованных наблюдений выполняется в~соответствии с~некоторыми\linebreak 
 рекуррентными интегральными соотношениями, в~то время как между 
 моментами наблюдений оценка корректируется в~соответствии с~прогнозом в~силу 
 сис\-те\-мы уравнений Колмогорова. Вы\-чис\-ли\-тель\-ная слож\-ность 
 упомянутых выше интегральных\linebreak 
 соотношений связана с~тем, что в~расчет принимается воз\-мож\-ность того, что между 
 моментами наблюдений оцениваемое со\-сто\-яние может совершить произвольное чис\-ло 
 скачков. В~разд.~4 пред\-став\-лен более простой алгоритм приближенного вы\-чис\-ле\-ния 
 оценки фильт\-ра\-ции, основанный на ограничении возможного числа учитываемых скачков 
 МСП. Доказана тео\-ре\-ма, опре\-де\-ля\-ющая как\linebreak
  локальную (одношаговую), так и~глобальную 
 (многошаговую) характеристики точ\-ности предложенного при\-бли\-же\-ния~--- 
 $\ell_1$-нор\-мы ошибки аппроксимации. Полученные характеристики являются\linebreak 
 универсальными, т.\,е.\ не асимптотическими по шагу дискретизации, и~зависят от характеристик 
 самого МСП, %\linebreak
  шага временн$\acute{\mbox{о}}$й дискретизации и~чис\-ла
  скачков со\-сто\-яния, учи\-ты\-ва\-емых 
 на шаге. Об\-суж\-де\-ние результатов и~заключительные комментарии пред\-став\-ле\-ны 
 в~разд.~5.
 
 \section{Постановка задачи фильтрации}
 
 На полном вероятностном пространстве с~фильт\-ра\-цией 
 $(\Omega,\mathcal{F},\mathcal{P},\{\mathcal{F}_{t}\}_{t \geqslant 0})$ рассматривается система наблюдений
\begin{equation}
 \left.
 \begin{array}{rl}
 \displaystyle X_t &=X_0 +  \displaystyle
 \int\limits_0^t \Lambda^{\top}X_{s}\,ds + \mu_s\,;  \\[6pt]
 \displaystyle Y_k &=  \displaystyle\int\limits_{t_{k-1}}^{t_k}fX_s\,ds+
 \int\limits_{t_{k-1}}^{t_k} 
 \sum\limits_{n=1}^NX_s^ng_n \,dW_s, \\[6pt]
 &\hspace*{10mm}\{t_k\}_{k \geqslant 0}: \; 0 = t_0 < t_1 < t_2\cdots,
 \end{array}
 \right\}
 \label{eq:obsys_1}
 \end{equation}
 где
  \begin{itemize}
  \item
  $X_t \ebd \mathrm{col}\left(X_t^1,\ldots,X_t^N\right) \hm\in \mathbb{S}^N$~--- 
  ненаблюда\-емое состояние системы, являющееся однородным МСП с~конечным 
  множеством состояний $ \mathbb{S}^N \ebd$\linebreak $\ebd \{e_1,\ldots,e_N\}$ ($\mathbb{S}^N$~--- 
  множество единичных векторов евклидова пространства~$\mathbb{R}^N$), 
  матрицей интенсивностей переходов~$\Lambda$ и~начальным распределением~$\pi$;
  \item
  $\mu_t \ebd \mathrm{col}\left(
  \mu_t^1,\ldots,\mu_t^N\right)\hm\in \mathbb{R}^N$~--- 
  ${\mathcal{F}}_t$-со\-гла\-со\-ван\-ный мартингал;
  \item
  $\{Y_k\}_{k \in \mathbb{N}}:\;  Y_k \ebd \mathrm{col}\left(Y_k^1,\ldots,Y_k^M\right) 
  \hm\in \mathbb{R}^M$~--- последовательность дискретизованных наблюдений, 
  доступных в~известные неслучайные  моменты времени~$\{t_k\}_{k \in \mathbb{N}}$,
в~которых $W_t \ebd$\linebreak $\ebd \mathrm{col}\left(W_t^1,\ldots,W_t^M\right) \hm\in \mathbb{R}^M$
 является ${\mathcal{F}}_t$-со\-гла\-со\-ван\-ным стандартным винеровским процессом, 
 определяющим шумы в~наблюдениях,\linebreak  $f$~--- $(M \times N)$-мер\-ная 
 мат\-ри\-ца плана наблюдений, а~набор мат\-риц~$\{g_n\}_{n=\overline{1,N}}$ 
 характеризует интенсивности шумов в~зависимости от текущего состояния~$X_t$.
  \end{itemize}
  
  Введем также в~рассмотрение неубывающие семейства $\sigma$-ал\-гебр 
  $\mathcal{O}_k \ebd \sigma\{ Y_{\ell}: \; 1 \hm\leqslant \ell \hm\leqslant k\}$ 
  и~$\mathcal{O}_t \ebd  \mathcal{O}_{k(t)}$, где 
  $k(t) \ebd \sum\nolimits_{j \in \mathbb{N}}\mathbf{I}(t-t_{j})$; 
  $\mathcal{O}_0 \ebd \{\varnothing,\; \Omega\}$.
  
   \textit{Задача оптимальной фильтрации состояния~$X$ по наблюдениям~$Y$} 
   заключается в~нахождении \textit{условного математического ожидания} (УМО)
  \begin{equation*}
  \widehat{X}_t \ebd {\sf E}\left\{X_t|\mathcal{O}_{t} \right\}\,.
 % \label{eq:fest_1}
  \end{equation*}
  
  Относительно системы~(\ref{eq:obsys_1})  сделаны следующие предположения:
   \begin{itemize}
 \item[(а)]
 ${\mathcal{F}}_t \equiv {\mathcal{F}}_{t}^X \bigvee 
 {\mathcal{F}}_{t}^W $ для любого $t \hm\geqslant 0$;
 \item[(б)]
 шумы в~наблюдениях равномерно невырожденные, т.\,е.\
  $g_ng_n^{\top} \hm\geqslant \alpha I \hm> 0$ для всех $n\hm=\overline{1,N}$ 
  и~некоторого $\alpha\hm>0$.
% \item
 % Верно неравенство
  %\begin{equation}
  %\min_{1\leqslant k \leqslant N}|\lambda_{kk}| > 0.
  %\label{eq:ineq_0}
  % \end{equation}
 %\item
 %Для любого $t \geqslant 0$ все компоненты вектора $p_t \ebd \me{}{X_t}$ строго %положительны. 
 \end{itemize} 

 \section{Уравнения оптимального фильтра} 
 
 Для получения уравнений оптимального фильт\-ра воспользуемся подходом, 
 применяемым для решения аналогичной задачи в~стохастических сис\-те\-мах 
 наблюдения с~дискретным временем~\cite{BSh_85}. 
 Воспользу\-ем\-ся методом математической индукции. 
 
 При $r=0$ 
 \begin{equation}
 \widehat{X}_{t_0}={\sf E}\{X_0|\mathcal{O}_0\}={\sf E}\{X_0\}=\pi\,.
 \label{eq:in_cond}
 \end{equation} 
 
 Пусть для некоторого $ r \hm\geqslant 0$ известна оценка оптимальной 
 фильтрации~$\widehat{X}_{t_r} \hm= {\sf E}{X_{t_r} |\mathcal{O}_r}$. 
 Определим оценку оптимальной фильтрации~$\widehat{X}_{t} $ для $t\hm \in (t_r,t_{r+1}]$. 
 
 Для произвольного момента $t \hm\in (t_r,t_{r+1})$ в~силу мартингального 
 разложения МСП~$X_t$ и~свойств УМО верна следующая цепочка равенств:
 \begin{multline*}
 \widehat{X}_{t} = {\sf E}\left\{X_t | \mathcal{O}_r\right\}={}\\
 {}=
 {\sf E}\left\{{\cal P}^{\top}(t_r,t)X_{t_r}+
 \int\limits_{t_r}^t{\cal P}^{\top}(t_r,s)\,dM_s\big\vert \mathcal{O}_r\right\} = {}
\end{multline*}

\noindent
   \begin{multline}
 \hspace*{-11.66pt}{}=\mathcal{P}^{\top}(t_r,t)\widehat{X}_{t_r} + {\sf E}\hspace*{-2pt}
 \left\{{\sf E}\hspace*{-2pt}\left\{\int\limits_{t_r}^t\hspace*{-2pt}\mathcal{P}^{\top}(t_r,s)\,dM_s |
 {\mathcal{F}}_{t_r}\right\}\!\big\vert 
 \mathcal{O}_r\!\right\} ={}\hspace*{-4.24124pt}\\
 {}=
  \mathcal{P}^{\top}(t_r,t)\widehat{X}_{t_r}\,,
 \label{eq:bw_obs}
 \end{multline}
 где $\mathcal{P}(s,t)$ $(s \hm\leqslant t)$~--- матрица переходной ве\-ро\-ят\-ности МСП 
 на промежутке $[s,t]$, являющаяся решением сис\-те\-мы дифференциальных 
 уравнений Колмогорова
 \begin{equation*}
 \mathcal{P}'_t(s,t) = \mathcal{P}(s,t) \Lambda, \enskip t > s, \enskip \mathcal{P}(s,s) = I.
 \end{equation*}
 В случае однородного МСП $\mathcal{P}(s,t) \hm= e^{(t-s)\Lambda}$.
 
 Далее необходимо определить совместное распределение $(X_{t_{r+1}},Y_{r+1})$ 
 относительно~$ \mathcal{O}_r$. Из модели наблюдений следует, что 
 распределение~$Y_{r+1}$ относительно 
 $\sigma$-ал\-геб\-ры~$\mathcal{F}^X_{t_{r+1}} \vee \mathcal{O}_r$~---
 гауссовское с~параметрами 
 \begin{align*}
{\sf E}\left\{Y_{r+1}|{\mathcal{F}}^X_{t_{r+1}}\right\}& = f \tau_{r+1}\,; \\[6pt]
 \mathrm{cov} \left(Y_{r+1},Y_{r+1}|{\mathcal{F}}^X_{t_{r+1}}\right) &= 
 \displaystyle\sum\limits_{n=1}^N \tau_{r+1}^n g_ng_n^{\top}\,,
% \label{eq:occup_1}
 \end{align*}
 где $\tau_{r+1} \hm= \tau_{r+1}(X(\omega))=
 \mathrm{col}\left(\tau_{r+1}^1,\ldots,\tau_{r+1}^N\right) \ebd$\linebreak
 $\ebd 
 \int\nolimits_{t_r}^{t_{r+1}}X_s\,ds$~--- случайный вектор, $n$-я 
 компонента которого равна времени пребывания процесса~$X$ в~со\-сто\-янии~$e_n$ 
 на  интервале времени $[t_r, t_{r+1}]$. 
 Обозначим через $\mathcal{D}_{r+1} \ebd \{u=\mathrm{col}\,(u^1,\ldots,u^N):\; 
 u_m \hm\geqslant 0,\; \sum\nolimits_{m=1}^Mu_m\hm= t_{r+1}-t_r\}$ $(M-1)$-мер\-ный 
 симплекс в~пространстве~$\mathbb{R}^M$, являющийся носителем распределения 
 вектора~$\tau_{r+1}$. Пусть $\rho^{k,\ell}_{r+1}(du)$~--- 
 распределение вектора $\tau_{r+1} X_{t_{r+1}}^{\ell}$ при условии $X_{t_r}\hm=e_k$, 
 т.\,е.\ 
 для любого $\mathcal{A} \hm\in \mathcal{B}(\mathbb{R}^M)$ верно тождество:
\begin{multline*}
 \mathbf{P}\left\{\omega: \; X_{t_{r+1}}(\omega)=e_{\ell},\right.\\
 \left. 
 \tau_{r+1}(X(\omega)) \in \mathcal{A}\;|\;X_{t_r}=e_k\right\} \equiv
   \rho^{k,\ell}_{r+1}(\mathcal{A})\,.
\end{multline*}
 
Обозначим через
\begin{multline*}
 \mathcal{N}(y,m,K) \ebd (2\pi)^{-M/2} \mathrm{ det}^{-1/2} K\times{}\\
 {}\times\exp
 \left\{ -\fr{1}{2}\left(y-m\right)^{\top}K^{-1}(y-m)\right\}
\end{multline*}
 $M$-мер\-ную плот\-ность гауссовского распределения с~математическим 
 ожиданием~$m$ и~ковариационной матрицей~$K$.
 
 Из марковского свойства  $\{X_{t_{r}},Y_{r})\}_{r \geqslant 0}$ 
 относительно~${\mathcal{F}}_{t_{r}}$~\cite{ZhSh_95} и~теоремы Фубини следует, что 
 для любого  множества $\mathcal{A} \hm\in \mathcal{B}(\mathbb{R}^M)$ 
 верна следующая цепочка равенств:
 \begin{multline*}
 {\sf E}\left\{X_{t_{r+1}}\mathbf{I}_{\mathcal{A}}
 \left(Y_{r+1}\right)\big|\mathcal{O}_r\right\}={}\\
 {}=
{\sf E}\left\{{\sf E}\left\{X_{t_{r+1}}\mathbf{I}_{\mathcal{A}}
\left(Y_{r+1}\right)\big|
\mathcal{F}^X_{t_{r+1}} \vee \mathcal{O}_r\right\}
 \big|\mathcal{O}_r\right\} = {}
\end{multline*}

\noindent
\begin{multline*}
 %{}=
% {\sf E}\left\{{\sf E}\left\{X_{t_{r+1}}\mathbf{I}_{\mathcal{A}}
% \left(Y_{r+1}\right)\vert X_{t_r}\right\}
% \vert\mathcal{O}_r\right\} = {}\\
% {}=
%{\sf E}\left\{\sum\limits_{k=1}^N {\sf E}\left\{X_{t_{r+1}}\mathbf{I}_{\mathcal{A}}
%\left(Y_{r+1}\right)  \big| X_{t_r}=e_k\right\}X_{t_r}^k
% \big|\mathcal{O}_r\right\} = {}\\ 
% {}=
% \sum\limits_{k=1}^N{\sf E}
% \left\{X_{t_{r+1}}\mathbf{I}_{\mathcal{A}}\left(Y_{r+1}\right)\bigl| X_{t_r}=e_k\right\} 
% \widehat{X}_{t_r}^k ={}\\
% {}=\!
% \sum\limits_{k=1}^N{\sf E}
% \left\{{\sf E}\left\{X_{t_{r+1}}\mathbf{I}_{\mathcal{A}}
% \left(Y_{r+1}\right)\!\bigl| {\mathcal{F}}_{t_{r+1}}\right\}\!\bigl| 
% X_{t_r}\!=e_k\right\} \widehat{X}_{t_r}^k ={}\\
% {}=
% \sum\limits_{k=1}^N {\sf E}\left\{
% \vphantom{\int\limits_A\left(\sum\limits_{p=1}^N\right)}
% X_{t_{r+1}} \times{}\right.\\
% {}\times\int\limits_{\mathcal{A}}  
% \mathcal{N}\left(y,f \tau_{r+1}(X),\sum\limits_{p=1}^N \tau_{r+1}^p(X) g_pg_p^{\top}\right)dy
% \Biggl| X_{t_r}={}\\
%\left. {}=e_k
% \vphantom{\int\limits_A\left(\sum\limits_{p=1}^N\right)}
%\right\} \widehat{X}_{t_r}^k = 
% \sum\limits_{k=1}^N \int\limits_{\mathcal{A}}{\sf E}\left\{ 
% \vphantom{\sum\limits_{p=1}^N}
% X_{t_{r+1}} \times{}\right.\\
% {}\times\mathcal{N}\left(y,f \tau_{r+1}(X),\sum\limits_{p=1}^N \tau_{r+1}^p(X) 
% g_p g_p^{\top}\right)
% \Biggl| X_{t_r}={}\\
%\left. {}=e_k
%\vphantom{\sum\limits^N_{p=1}}
%\right\} \widehat{X}_{t_r}^k\, dy
 %={}\\
 {}=
 \sum\limits_{\ell=1}^N e_{\ell} \int\limits_{\mathcal{A}} 
 \left[ \sum\limits_{k=1}^N 
 \int\limits_{\mathcal{D}_{r+1}} 
 \mathcal{N}\left(y,f u,\sum_{p=1}^N u^p g_pg_p^{\top}\right)\times{}\right.\\
\left. {}\times
 \rho^{k,\ell}_{r+1}(du)\widehat{X}_{t_r}^k
 \vphantom{\int\limits_A\sum\limits_{p=1}^N}
 \right] 
 dy,
 \end{multline*}
 из чего следует, что интегранд в~квадратных скобках в~последнем выражении 
 определяет искомое совместное распределение $(X_{t_{r+1}},Y_{r+1})$ 
 относительно~$ \mathcal{O}_r$. Оценка~$\widehat{X}_{t_{r+1}}$ покомпонентно 
 определяется~\cite{BSh_85} с~помощью обобщенного варианта формулы Байеса:
 \begin{multline}
 \widehat{X}_{t_{r+1}}^j = {}\\
 \hspace*{-1mm}{}=
 \fr{\int\nolimits_{\mathcal{D}_{r+1}}\hspace*{-6mm} 
 \mathcal{N}\left(Y_{r+1},f u,\sum\nolimits_{p=1}^N \hspace*{-2mm}
 u^p g_pg_p^{\top}\!\right)\hspace*{-1mm}
 \sum\nolimits_{k=1}^N \hspace*{-2mm}
 \widehat{X}_{t_r}^k
 \rho^{k,j}_{r+1}(du)
 }
 { \int\nolimits_{\mathcal{D}_{r+1}} \hspace*{-6mm}
 \mathcal{N}\left(Y_{r+1},f v,\sum\nolimits_{q=1}^N \hspace*{-2mm}
 v^q g_qg_q^{\top}\!\right)\hspace*{-1mm}
 \sum\nolimits_{i,\ell=1}^N \hspace*{-2mm}
 \widehat{X}_{t_r}^i
 \rho^{i,\ell}_{r+1}(dv)
  },  \\ 
  j = \overline{1,N}\,.
 \label{eq:filt_1}
 \end{multline}
 Таким образом, доказана следующая
 
 %\smallskip
 
 \noindent
 \textbf{Лемма~1.}
\textit{Если для системы наблюдения}~(\ref{eq:obsys_1}) 
\textit{верны условия~(а) и~(б), то оценка~$\widehat{X}_t$ оптимальной фильтрации 
определяется формулой}~(\ref{eq:in_cond}) 
\textit{при $t\hm=0$, рекуррентным соотношением}~(\ref{eq:filt_1})~---
\textit{в~моменты~$t_{r+1}$ получения наблюдений~$Y_{r+1}$ 
и~формулой}~(\ref{eq:bw_obs})~--- 
\textit{в~промежутках времени между моментами получения наблюдений}.


\smallskip
 

 
 Несмотря на компактную запись~(\ref{eq:filt_1}), их прямая численная реализация 
 ресурсозатратна. Во-пер\-вых, в~(\ref{eq:filt_1}) требуется вычислять 
 распределения мас\-штаб\-но-сдви\-го\-вых смесей многомерных нормальных 
 распределений, что является трудоемкой\linebreak процедурой. Во-вто\-рых, 
 распределения~$\rho^{k,j}_{r+1}$ вре-\linebreak мени пребывания представляют собой 
 сумму\linebreak бесконечного ряда, слагаемые которого вычис\-ляются с~помощью 
 некоторой рекуррентной про\-це\-дуры~\cite{S_00}. В-третьих, 
 распределения~$\rho^{k,j}_{r+1}$ не являются абсолютно непрерывными 
 относительно меры Ле\-бега.
 { %\looseness=1
 
 }
 
 Следующий раздел посвящен численной аппроксимации~(\ref{eq:filt_1}) и~исследованию 
 ее точностных характеристик.
 
 \section{Приближенное вычисление оценки фильтрации}
 
 Без ограничения общности будем считать, что сетка~$\{t_r\}_{r \geqslant 0}$ 
 является равномерной с~шагом~$\Delta$, т.\,е.\ $t_r \hm= r \Delta$ 
 и~$\mathcal{D}_r \hm\equiv \mathcal{D}$.
 Обозначим через~$N_{r+1}$ об-\linebreak\vspace*{-12pt}
 
 \pagebreak
 
 \noindent
 щее число скачков процесса~$X_t$, имевших место 
 на промежутке $(t_r,t_{r+1}]$. Тогда из формулы полной вероятности следует, 
 что~(\ref{eq:filt_1}) представима в~виде:
 \begin{multline}
 \widehat{X}_{t_{r+1}}^j =  \left(
 \int\limits_{\mathcal{D}} 
 \mathcal{N}\left(Y_{r+1},f u,\sum\limits_{p=1}^N u^p g_pg_p^{\top}\right)\times{}\right.\\
\left. {}\times
 \sum\limits_{h=0}^{\infty}\sum\limits_{k=1}^N \widehat{X}_{t_r}^k
 \rho^{k,j,h}_{r+1}(du)
 \right)\Bigg/ \\
 \left(
 \vphantom{\sum\limits_{m=0}^{\infty}
 \sum\limits_{i,\ell=1}^N \widehat{X}_{t_r}^i
 \rho^{i,\ell,m}_{r+1}(dv)}
 \int\limits_{\mathcal{D}} 
 \mathcal{N}\left(Y_{r+1},f v,\sum\limits_{q=1}^N v^q g_qg_q^{\top}\right)\times{}\right.\\
\left.{}\times \sum\limits_{m=0}^{\infty}
 \sum\limits_{i,\ell=1}^N \widehat{X}_{t_r}^i
 \rho^{i,\ell,m}_{r+1}(dv)
 \right)
  \,, \enskip j = \overline{1,N}\,,
  \label{eq:filt_1_1}
 \end{multline}
 где 
 $ \rho^{k,j,h}_{r+1}(du)$~--- распределение вектора 
 $\tau_{r+1}X_{t_{r+1}}^{j}\mathbf{I}_{\{h\}}(N_{r+1})$ при 
 условии $X_{t_r}\hm=e_k$, т.\,е.\ 
 для любого $\mathcal{A} \hm\in \mathcal{B}(\mathbb{R}^M)$ верно тождество
\begin{multline*}
 \mathbf{P}\left\{\omega: \; X_{t_{r+1}}(\omega)=e_{j}, \; N_{r+1} = h,\right.\\ 
\left. \tau_{r+1}(X(\omega)) \in \mathcal{A}\;|\;X_{t_r}=e_k\right\} \equiv
  \rho^{k,j,h}_{r+1}(\mathcal{A}).
\end{multline*}
В качестве аппроксимации оценок можно использовать  
 $\overline{X}_{t_{r+1}}^n \ebd 
 \mathrm{col}\,(\overline{X}_{t_{r+1}}^{n,1},\ldots,\overline{X}_{t_{r+1}}^{n,N})$, 
 полученные из~(\ref{eq:filt_1_1}) путем урезания сумм ряда в~числителе и~знаменателе:
 
 \noindent
 \begin{multline}
 \overline{X}_{t_{r+1}}^{n,j} = 
 \left(
 \int\limits_{\mathcal{D}} 
 \mathcal{N}\left(Y_{r+1},f u,\sum\limits_{p=1}^N u^p g_pg_p^{\top}\right)\times{}\right.\\[-1pt]
\left.{}\times \sum\limits_{h=0}^{n}\sum\limits_{k=1}^N \overline{X}_{t_r}^k
 \rho^{k,j,h}_{r+1}(du)
 \right)\Bigg/ \\[-1pt]
 \left(
 \int\limits_{\mathcal{D}} 
 \mathcal{N}\left(Y_{r+1},f v,\sum\limits_{q=1}^N v^q g_qg_q^{\top}\right)\times{}\right.\\[-1pt]
\left. {}\times
 \sum\limits_{m=0}^{n}
 \sum\limits_{i,\ell=1}^N \overline{X}_{t_r}^i
 \rho^{i,\ell,m}_{r+1}(dv)
  \right)\,, \enskip
   j = \overline{1,N}.
  \label{eq:filt_2}
 \end{multline}
 Ниже по формуле полной вероятности получены интегралы из~(\ref{eq:filt_2}) для 
 $h\hm=0,1,2$:
 
\vspace*{-3pt}

 \noindent
  \begin{multline*}
 \int\limits_{\mathcal{D}}  \mathcal{N}
 \left(Y_{r+1},f u,\sum\limits_{p=1}^N u^p g_pg_p^{\top}\right) 
 \rho^{k,j,0}_{r+1}(du) = {}\\[-1pt]
 {}=
 \delta_{kj}\mathcal{N}\left(Y_{r+1},\Delta f^j,\Delta g_jg_j^{\top}\right)
 e^{\lambda_{jj}\Delta};
 %\label{eq:h0}
\\[-1pt]
 \int\limits_{\mathcal{D}}  \mathcal{N}\left(
 Y_{r+1},f u,\sum\limits_{p=1}^N u^p g_pg_p^{\top}\right) 
 \rho^{k,j,1}_{r+1}(du) ={} 
 \end{multline*}
 
 \noindent
 \begin{multline}
 \hspace*{-6.7pt}{}=\left(1-\delta_{kj}\right)\lambda_{kj}e^{\lambda_{jj}\Delta}
\! \int\limits_0^{\Delta}\!
 e^{(\lambda_{kk}-\lambda_{jj})u^k}
 \mathcal{N}\left(Y_{r+1},u^kf^k +{}\right.\hspace*{-0.28818pt}\\[-1pt]
\hspace*{-3mm}\left. {}+ \left(\Delta - u^k\right)f^j, u^k g_kg_k^{\top}+
 \left(\Delta-u^k\right)g_jg_j^{\top}\right)\,du^k;
 \label{eq:h1}
 \end{multline}
 
 \vspace*{-12pt}
 
 \noindent
 \begin{multline}
 \int\limits_D \mathcal{N}\left( 
Y_{r+1},f u,\sum\limits_{p=1}^N u^p g_pg_p^{\top}\right)du ={}\\[-1pt]
{}=
\sum\limits_{\substack{{\ell:\ell \neq k,}\\ {\ell \neq j}}}
 \lambda_{k\ell}\lambda_{\ell j} e^{\lambda_{jj}\Delta}\times {}\\[-1pt] 
 {}\times
 \int\limits_0^{\Delta} \int\limits_0^{\Delta-u^k} \!
e^{(\lambda_{kk}-\lambda_{\ell\ell})u^k+(\lambda_{\ell\ell}-
 \lambda_{jj})u^{\ell}}\times{} \\[-1pt] 
{}  \times
 \mathcal{N}\left(Y_{r+1},u^k f^k+u^{\ell}f^{\ell}+\left(
 \Delta-u^k-u^{\ell} \right)f^j,\right.\\[-1pt]
 \hspace*{-1mm}\left.
 u^k g_kg_k^{\top}+u^{\ell}g_{\ell}g_{\ell}^{\top}+\left(
 \Delta-u^k-u^{\ell} \right)
 g_jg_j^{\top}
 \right) du^{\ell}du^{k}, \!\!
  \label{eq:h2}
 \end{multline} 
 
\vspace*{-2pt}
 
 \noindent
  где  $\delta_{ij}$~--- символ Кронекера. Интегралы для $h\hm>2$ также могут 
  быть получены в~явном виде, однако их сложность резко возрастает.
 

   Так как система~(\ref{eq:obsys_1}) является автономной, то в~качестве локальной 
   характеристики бли\-зости~$\{\overline{X}_{t_r}\}$ 
   к~$\{\widehat{X}_{t_r}\}$ может быть выбрана величина
   
\noindent
 \begin{multline*}
 \overline{\sigma}(\pi) \ebd {\sf E}\left\{
 \|\widehat{X}_{t_{1}}(\pi, Y_{1}) - \overline{X}_{t_{1}}
 \left(\pi,Y_{1}\right)\|_{1}\right\} = {}\\
 {}=
 \sum\limits_{j=1}^N{\sf E}
 \left\{\left\vert \widehat{X}^j_{t_{1}}\left(\pi, Y_{1}\right) - \overline{X}^{n,j}_{t_{1}}
 \left(\pi,Y_{1}\right)\right\vert\right\}.
 %\label{eq:prec_1}
 \end{multline*}
 При этом начальное распределение $\pi \hm\in \mathcal{D}_1 \ebd $\linebreak $\ebd
 \{\mathrm{col}\,(\pi^1,\ldots,\pi^N):\;\pi^j > 0$, 
 $\sum\nolimits_{j=1}^N\pi^j\hm=1\}$ является начальным условием применения 
 одного шага рекурсии~(\ref{eq:filt_1}) или~(\ref{eq:filt_2}) для вычисления 
 оценки~$\widehat{X}_{t_{1}}$
   или~$\overline{X}_{t_{1}}$ соответственно. Фактически, 
 характеристика~$\overline{\sigma}(\pi)$ определяет, насколько сильно 
 рекурсивные схемы~(\ref{eq:filt_1}) и~(\ref{eq:filt_2}) разойдутся за 
 один шаг, стартуя из общей точки~$\pi$.
 
 Рекуррентные схемы~(\ref{eq:filt_1}) и~(\ref{eq:filt_2}), примененные~$r$~раз, 
 позволяют вычислить оценки~$\widehat{X}_{t_r}$ и~$\overline{X}_{t_r}$ 
 в~точке~$t_r$. В~качестве характеристики точности глобальной аппроксимации в~этом 
 случае естественно рассмотреть величину
 
 \vspace*{-2pt}
 
 \noindent
 \begin{equation*}
 \overline{\Sigma}_{t_r}(\pi) \ebd {\sf E}
 \left\{\|\widehat{X}_{t_{r}} - \overline{X}_{t_{r}}\|_{1}\right\} = 
 \!\sum\limits_{j=1}^N\!{\sf E}
 \left\{\left\vert \widehat{X}^j_{t_{r}} - 
 \overline{X}^{n,j}_{t_{r}}\right\vert \right\}.
% \label{eq:prec_2}
 \end{equation*}
 
 Следующее утверждение определяет оценки локальной и~глобальной 
 точности схемы аппроксимации~(\ref{eq:filt_2}).
 
 %\smallskip
 
 \noindent
 \textbf{Теорема~1.}\
\textit{Выполняются неравенства} 

%\vspace*{-2pt}

\noindent
 \begin{equation}
 \sup_{\pi \in \mathcal{D}_1} \overline{\sigma}(\pi) 
 \leqslant 2 \fr{(\overline{\lambda}\Delta)^{n+1}}{(n+1)!}\,;
 \label{eq:prec_loc}
\end{equation}

\noindent
\begin{align}
  \sup\limits_{\pi \in \mathcal{D}_1} \overline{\Sigma}_{t_r}(\pi)
   &\leqslant 2r \fr{(\overline{\lambda}\Delta)^{n+1}}{(n+1)!} +{}\notag\\[-0.5pt]
   &\hspace*{-20mm}{}+
  r(r-1)\left(
  \fr{(\overline{\lambda}\Delta)^{n+1}}{(n+1)!}
  \right)^2
  \left(
  1-\fr{(\overline{\lambda}\Delta)^{n+1}}{(n+1)!}
  \right)^{r-2},
 \label{eq:prec_glob}
 \end{align}
 
 \vspace*{-2pt}
 
 \noindent
 \textit{где} $\overline{\lambda} \ebd \max_{1 \leqslant j \leqslant N}|\lambda_{jj}|$.


%\smallskip

 Доказательство теоремы~1 приведено в~приложении.
 
 Данное утверждение представляет полезные оценки точности. Во-пер\-вых, 
 они являются равномерными по начальному распределению $\pi \hm\in \mathcal{D}_1$. 
 Во-вто\-рых, оценки носят универсальный, а~не асимптотический характер. Это 
 существенно в~практических задачах оценивания по дискретизованным 
 наблюдениям с~физическими или алгоритмическими ограничениями на шаг 
 по времени. Например, в~случае наблюдаемого процесса восстановления в~силу 
 центральной предельной теоремы для процессов восстановления~\cite{B_80} его
  приращения можно рассматривать как гауссовские случайные величины. 
  Однако данная аппроксимация обладает удовлетворительной точностью 
  только в~случае, когда шаг дискретизации по времени достаточно большой. 
 %
 В-третьих, неравенство~(\ref{eq:prec_glob}) позволяет получить порядок 
 аппроксимации при $\Delta \hm\to 0$. Зафиксируем момент времени $t\hm=T$ и~рассмотрим 
 характеристику $\sup\nolimits_{\pi \in \mathcal{D}_1} 
 \overline{\Sigma}_{T}(\pi)$ при $r\hm={T}/{\Delta}$ и~$\Delta \hm\to 0$. 
 Как только~$\Delta$ становится настолько мало, что 
 $\max\left({(\overline{\lambda}\Delta)^{n+1}}/{(n+1)!}, 
 \Delta ({T\lambda^{n+1}}/{(n+1)!})\right)\hm< 1$, из~(\ref{eq:prec_glob}) 
 следует неравенство
  %\begin{equation}
  $\sup\nolimits_{\pi \in \mathcal{D}_1} \overline{\Sigma}_{T}(\pi) 
  \hm\leqslant  ({3\overline{\lambda}^{n+1}}/{(n+1)!}) T\Delta^n.$
 %\label{eq:prec_asympt}
 %\end{equation}
 Это значит, что с~ростом времени~$T$ 
 ошибка аппроксимации копится пропорционально~$T$ и~при этом порядок точности 
 по~$\Delta$ равен~$n$.
 
 %\vspace*{-7pt}
 
  \section{Заключение}
  
  \vspace*{-4pt}
 
  В работе решена задача оценивания состояния однородного МСП по 
  дискретизованным наблюдениям. Получено аналитическое решение и~его 
  чис\-лен\-ные аппроксимации. Локальные и~глобальные показатели точ\-ности этих 
  приближений в~статье так\-же пред\-став\-ле\-ны. Примечательно, что  част\-ный случай 
  аппроксимаций~(\ref{eq:filt_2}) при $n\hm=0$ и~$\Lambda\hm=0$ был ранее 
  пред\-став\-лен в~\cite{B_17_1,B_17_2} для решения задачи байесовской классификации 
  случайного вектора по непрерывным наблюдениям с~мультипликативными шумами. 
 % 
Алгоритм оптимальной фильт\-ра\-ции и~его субоптимальные версии могут 
рас\-смат\-ри\-вать\-ся в~качестве основы чис\-лен\-ной реализации обобщения фильт\-ра 
Вонэма для сис\-тем с~мультипликативными шумами в~наблюдениях. 
Однако для их непосредственного использования необходимо решить 
следующие проб\-ле\-мы. Во-пер\-вых, в~(\ref{eq:h1}) и~(\ref{eq:h2}) присутствуют
 многомерные интегралы. Следует выяснить, какую результирующую погрешность 
 будут вносить ошибки их вы\-чис\-ле\-ния. Во-вто\-рых, представляется интересным 
 определить характеристики точ\-ности оптимальной фильт\-ра\-ции по дискретизованным 
 наблюдениям по отношению к~оптимальной фильт\-ра\-ции по непрерывным наблюдениям: 
 каков порядок точ\-ности по шагу временной дискретизации~$\Delta$? Для случая 
 вы\-чис\-ле\-ния классического фильт\-ра Вонэма с~по\-мощью алгоритма Эй\-ле\-ра--Ма\-ру\-ямы 
 подобный результат известен: порядок глобальной ошибки равен~${1}/{2}$. 
 Перечисленные задачи являются предметом дальнейших исследований.
 
 
  \vspace*{-10pt}
 
{\small
\subsection*{\raggedleft Приложение} 

\vspace*{-2pt}


\noindent
Д\,о\,к\,а\,з\,а\,т\,е\,л\,ь\,с\,т\,в\,о\ \ теоремы~1.\ \ Введем следующие 
обозначения для случайных величин и~мат\-риц, составленных из них:
\begin{align*}
\xi^{ji}(\ell)&\ebd 
\sum\limits_{h=0}^n \int\limits_{\mathcal{D}} 
 \mathcal{N}\left(Y_{\ell},f u,\sum\limits_{p=1}^N u^p g_pg_p^{\top}\right)
 \rho^{j,i,h}_{1}(du)\,; \\
  \theta^{ji}(\ell)&\ebd 
\sum\limits_{h=n+1}^{\infty} \int\limits_{\mathcal{D}} 
 \mathcal{N}\left(Y_{\ell},f u,\sum\limits_{p=1}^N u^p g_pg_p^{\top}\right)
 \rho^{j,i,h}_{1}(du)\,;
\\
 \xi(\ell)&\ebd \|\xi^{ji}(\ell)\|_{j,i=\overline{1,N}}\,,\quad 
 \Xi(r) \ebd \xi(r) \xi(r-1)\cdots \xi(1)\,;
 \\
 \theta(\ell)&\ebd \|\theta^{ji}(\ell)\|_{j,i=\overline{1,N}}\,, \quad 
 \Theta(r) \ebd \theta(r) \theta(r-1)\cdots \theta(1)\,.
%\label{eq:not_1}
\end{align*}
 
 Рекуррентные формулы~(\ref{eq:filt_1}) и~(\ref{eq:filt_2}) можно записать в~явной 
 форме
 
 
\noindent
\begin{align*}
 \widehat{X}_{t_r}& = \left( \mathbf{1}\left(\Xi(r) + 
 \Theta(r)\right)\pi\right)^{-1} \left(\Xi(r) + \Theta(r)\right)\pi\,;
\\
 \overline{X}_{t_r} &= \left( \mathbf{1}\Xi(r)\pi\right)^{-1} \Xi(r) \pi,
\end{align*}

\vspace*{-2pt}

\noindent
где $\mathbf{1} \ebd (1,\ldots,1)$~--- век\-тор-стро\-ка 
подходящей раз\-мер\-ности, составленная из единиц.

%Далее для краткости записи зависимость от~$r$ в~обозначениях~$\Xi(r)$ 
%и~$\Theta(r)$ будет опущена. 
Верна следующая цепочка неравенств:

 \vspace*{-3pt}

\noindent
\begin{multline}
\overline{\Sigma}_{t_r}(\pi)=%
%\me{}{\left\| 
%\widehat{X}_{t_r}(\pi, Y_1,\ldots,Y_r) - \overline{X}_{t_r}(\pi, Y_1,\ldots,Y_r)
%\right\|_1} =\\=
{\sf E}\left\{\left\| 
\fr{1}{\mathbf{1}\left(\Xi(r) + \Theta(r)\right)\pi} \left(\Xi(r) +{}\right.\right.\right.\\[-1pt]
\left.\left.\left.{}+ \Theta(r)\right)\pi
- \fr{1}{\mathbf{1}\Xi(r)\pi}\,\Xi(r) \pi
\right\|_1\right\} ={} \\[-1pt]
{}=
{\sf E}\left\{\fr{1}{\mathbf{1}\left(\Xi(r) + \Theta(r)\right)\pi \mathbf{1}\Xi(r)\pi}
\left\|
 \mathbf{1}\Xi(r) \pi \Theta(r)\pi -{}\right.\right.\\[-1pt]
\left.\left. {}- \mathbf{1}\Theta(r)\pi \Xi(r) \pi
 \right\|_1
 \vphantom{\fr{1}{\mathbf{1}\left(\Xi(r) + \Theta(r)\right)\pi \mathbf{1}\Xi(r)\pi}}
\right\} \leqslant {}\\[-1pt]
{}\leqslant 
{\sf E}\left\{\fr{1}{\mathbf{1}\left(\Xi(r) + \Theta(r)\right)\pi \mathbf{1}\Xi(r)\pi}
\left(
\mathbf{1}\Xi(r)\pi \| \Theta(r)\pi \|_1 +{}\right.\right.\\[-1pt]
\left.\left.{}+ \mathbf{1}\Theta(r)\pi 
\|
\Xi(r) \pi
\|_1
\right)
 \vphantom{\fr{1}{\mathbf{1}\left(\Xi(r) + \Theta(r)\right)\pi \mathbf{1}\Xi(r)\pi}}
\right\} ={}\\[-1pt]
{}=
2\,{\sf E}\left\{\fr{1}{\mathbf{1}\left(\Xi(r) + \Theta(r)\right)\pi}\mathbf{1}\Theta(r)\pi 
\right\}.
\label{eq:ineq_1}
\end{multline}

 
 \noindent
 Рассмотрим случайные события $a_{\ell} \ebd \{\omega \in \Omega: 
 N_{\ell}(\omega) \hm\leqslant n\}$, $\ell \hm= \overline{1,r}$, и~$A_r \ebd \{
 \omega\hm \in \Omega: \max_{1 \leqslant {\ell} \leqslant r}N_{\ell}(\omega) 
 \hm\leqslant n
 \}\hm=\prod\nolimits_{\ell=1}^r a_{\ell}$ и~оценку 
 $
 \widetilde{X}_{t_r}(\pi, Y_1,\ldots,Y_r)\ebd$\linebreak $\ebd
 {\sf E}\left\{X_{t_r}(\omega)\mathbf{I}_{A_r}(\omega)|\mathcal{O}_r\right\}.
 $
 Используя введенные выше обозначе\-ния и~абстрактный вариант формулы Байеса, 
 получаем, что
 
 \noindent
\begin{align}
\widetilde{X}_{t_r}& = \fr{1}{{\mathbf{1}\left(\Xi(r) + 
 \Theta(r)\right)\pi}}\,\Xi(r)\pi\,;\notag
 \\
\widehat{X}_{t_r} - \widetilde{X}_{t_r} &=
{\sf E}\left\{X_{t_r}(\omega)\mathbf{I}_{\overline{A}_r}(\omega)|\mathcal{O}_r\right\} ={}\notag\\[-1pt]
&\hspace*{17mm}{}= 
\fr{1}{\mathbf{1}\left(\Xi(r) + \Theta(r)\right)\pi}\Theta(r)\pi\,. 
\label{eq:eq_2}
 \end{align}
 Из (\ref{eq:ineq_1}) и~(\ref{eq:eq_2}) для $r\hm=1$ следует, что
 
 \vspace*{-4pt}
 
 \noindent
 \begin{multline}
 \overline{\sigma}(\pi) \leqslant 2\,{\sf E}
 \left\{\|{\sf E}\left\{X_{t_1}(\omega)\mathbf{I}_{\overline{a}_1}(\omega)|\mathcal{O}_1
 \right\}\|_1
 \right\} ={}\\[-1.5pt]
 {}=
 2\,{\sf E}\left\{\sum\limits_{n=1}^N {\sf E}
 \left\{X^n_{t_1}(\omega)\mathbf{I}_{\overline{a}_1}
 (\omega)|\mathcal{O}_1\right\}\right\} ={} \\[-2pt] 
 {}=
  2\,{\sf E}\left\{{\sf E}\left\{\mathbf{I}_{\overline{a}_1}(\omega)|\mathcal{O}_1
  \right\}\right\} =
   2 \mathbf{P}\left\{\overline{a}_1(\omega)\right\}.
\label{eq:ineq_3}
\end{multline}

 \vspace*{-2pt}
 
 \noindent
 Процесс $N^X_t$ общего числа скачков состояния~$X_t$ является считающим, и~его
  квадратическая характеристика равна 
  
\vspace*{-2pt}
  
  \noindent
 $$
 \langle N^X, N^X\rangle_t = - \int\limits_0^t \sum\limits_{n=1}^N \lambda_{nn} X_s^n\,ds\,,
 $$
 поэтому искомая вероятность ограничена сверху:
 $$ 
 \mathbf{P}\left\{\overline{a}_1(\omega)\right\} \leqslant 
 e^{-\overline{\lambda}\Delta}\sum\limits_{k=n+1}^{\infty} 
 \fr{(\overline{\lambda}\Delta)^{k}}{k!} <
 \fr{(\overline{\lambda}\Delta)^{n+1}}{(n+1)!}.
 $$
 
  \vspace*{-2pt}
  
  \noindent
 Из последнего неравенства и~(\ref{eq:ineq_3}) следует, что  для любого 
 начального распределения~$\pi$ выполняется неравенство $\overline{\sigma}(\pi)  
 \hm< 2({(\overline{\lambda}\Delta)^{n+1}}/{(n+1)!})$, т.\,е.\ 
 локальная оценка~(\ref{eq:prec_loc}) верна.
 
 С помощью марковского свойства пары $(X_t, N^X_t)$ и~последнего 
 неравенства можно оценить сверху вероятность 
 $\mathbf{P}\left\{\overline{A}_r(\omega)\right\}$:
 
  \vspace*{-2pt}
 
 \noindent
 \begin{multline*}
 \mathbf{P}\left\{\overline{A}_r(\omega)\right\} \leqslant 1 - \left(
 1- \fr{(\overline{\lambda}\Delta)^{n+1}}{(n+1)!}
 \right)^r \leqslant r \fr{(\overline{\lambda}\Delta)^{n+1}}{(n+1)!} + {}\\[-1pt]
 {}+\left|
 \sum\limits_{k=2}^r C_r^k \left(-\fr{(\overline{\lambda}\Delta)^{n+1}}{(n+1)!}
 \right)^k
 \right| \leqslant
 r \fr{(\overline{\lambda}\Delta)^{n+1}}{(n+1)!} +{}\\[-1pt]
 {}+\fr{r(r-1)}{2}
 \left(
 \fr{(\overline{\lambda}\Delta)^{n+1}}{(n+1)!}
 \right)^2
 \left(
 1-\fr{(\overline{\lambda}\Delta)^{n+1}}{(n+1)!}
 \right)^{r-2},
 \end{multline*} 
 из чего следует истинность глобальной оценки~(\ref{eq:prec_glob}).
Теорема~1 доказана.

}

%\vspace*{-12pt}

{\small\frenchspacing
 {%\baselineskip=10.8pt
 \addcontentsline{toc}{section}{References}
 \begin{thebibliography}{99}

\bibitem{Won_65}
\Au{Wonham W.} 
Some applications of stochastic differential equations to optimal
  nonlinear filtering~//
SIAM~J.~Control, 1965. Vol.~2. P.~347--369. 

\bibitem{KP_92}
\Au{Kloeden P., Platen E.} Numerical solution of stochastic
differential equations.~--- Berlin: Springer, 1992.~636~p.

\bibitem{YZL_04}
\Au{Yin G., Zhang Q., Liu Y.} 
Discrete-time approximation of Wonham filters~//
J.~Control Theory Applications, 2004. Iss.~2. P.~1--10.

\bibitem{PR_10}
\Au{Platen E., Rendek R.}
Quasi-exact approximation of hidden Markov chain filters~//
Communicat.~Stoch.~Analys., 2010. Vol.~4. Iss.~1. P.~129--142.

\bibitem{B_18}
\Au{Борисов А.} Фильтрация Вонэма по наблюдениям с~мультипликативными шумами~// 
Автоматика и~телемеханика, 2018.
№~1. C.~52--65. 
 
  \bibitem{BSh_85} %6
\Au{Бертсекас Д., Шрив С.} Стохастическое оптимальное управление. 
Случай дискретного времени~/ Пер. с~англ.~--- М.: Наука, 1985.~280~c.
(\Au{Betsekas~D.\,P., Shreve~S.\,E.} Stochastic optimal control:
The discrete-time case.~--- Orlando, FL, USA:
Academic Press Inc., 1978. 323~p.)

  \bibitem{ZhSh_95} %7
\Au{Жакод Ж., Ширяев А.} Предельные теоремы для случайных процессов,~I.~/
Пер. с~англ.~--- 
М.: Физматлит, 1995.~544~c.
(\Au{Jacod~J., Shiryaev~A.} Limit theorems for stochastic processes.~---
Berlin: Springer, 2003. 664~p.)

\bibitem{S_00}
\Au{Sericola B.} Occupation times in Markov processes~//
Commun. Stat. Stochastic Models, 2000. Vol.~16. Iss.~5. P.~479--510. 

  \bibitem{B_80}
\Au{Боровков А.} Асимптотические методы в~тео\-рии массового обслуживания.~--- 
М.: Физматлит, 1995.~384~c.

  \bibitem{B_17_1}
\Au{Борисов А.} Классификация по непрерывным наблюдениям с~мультипликативными шумами.~I. 
Формулы байесовской оценки~// Информатика и~её применения, 2017. Т.~11. Вып.~1. C.~11--19.
doi: 10.14357/19922264170102.

  \bibitem{B_17_2}
\Au{Борисов А.} Классификация по непрерывным наблюдениям с~мультипликативными 
шумами.~II. Алгоритм численной реализации оценки~// Информатика и~её 
применения, 2017. Т.~11. Вып.~2. C.~33--41.
doi: 10.14357/19922264170204.

 \end{thebibliography}

 }
 }

\end{multicols}

\vspace*{-4pt}

\hfill{\small\textit{Поступила в~редакцию 10.07.18}}

\vspace*{6pt}

%\pagebreak

%\newpage

%\vspace*{-28pt}

\hrule

\vspace*{2pt}

\hrule

%\vspace*{-2pt}

\def\tit{FILTERING OF~MARKOV JUMP PROCESSES\\ BY~DISCRETIZED OBSERVATIONS}

\def\titkol{Filtering of Markov jump processes by discretized observations}

\def\aut{A.\,V.~Borisov}

\def\autkol{A.\,V.~Borisov}

\titel{\tit}{\aut}{\autkol}{\titkol}

\vspace*{-11pt}


\noindent
Institute of Informatics Problems, Federal Research Center ``Computer Science 
and Control'' of the Russian Academy of Sciences, 44-2~Vavilov Str., Moscow 
119333, Russian Federation


\def\leftfootline{\small{\textbf{\thepage}
\hfill INFORMATIKA I EE PRIMENENIYA~--- INFORMATICS AND
APPLICATIONS\ \ \ 2018\ \ \ volume~12\ \ \ issue\ 3}
}%
 \def\rightfootline{\small{INFORMATIKA I EE PRIMENENIYA~---
INFORMATICS AND APPLICATIONS\ \ \ 2018\ \ \ volume~12\ \ \ issue\ 3
\hfill \textbf{\thepage}}}

\vspace*{6pt}



\Abste{The article is devoted to a~solution of the optimal filtering problem 
of a~homogenous Markov
jump process state. The available observations represent 
time increments of the integral transformations of the Markov\linebreak\vspace*{-12pt}}

\Abstend{state corrupted by 
Wiener processes. The noise intensity is also state-dependent. At the instant of 
the consecutive
observation obtaining, the optimal estimate is calculated recursively 
as a~function of previous estimate and the new observation, meanwhile between 
observations the filtering estimate is a simple forecast by virtue of the Kolmogorov 
differential system. The recursion is rather expensive because of  need to calculate 
the integrals, which are the location-scale mixtures of Gaussians. The mixing 
distributions represent the occupation of the state in each of possible values 
during the mid-observation intervals. The paper contains numerically cheaper 
approximations, based on the restriction of the state transitions number between 
the observations. Both the local and global characteristics of approximation 
accuracy are obtained as functions of the dynamics parameters, mid-observation 
interval length, and upper bound of transitions number.}

\KWE{Markov jump process; optimal filtering; multiplicative observation noises; 
stochastic differential equation; numerical approximation}




\DOI{10.14357/19922264180316}

%\vspace*{-14pt}

\Ack
\noindent
The work was supported in part by the Russian Foundation
for Basic Research (Project No.\,16-07-00677).



%\vspace*{6pt}

  \begin{multicols}{2}

\renewcommand{\bibname}{\protect\rmfamily References}
%\renewcommand{\bibname}{\large\protect\rm References}

{\small\frenchspacing
 {%\baselineskip=10.8pt
 \addcontentsline{toc}{section}{References}
 \begin{thebibliography}{99}
\bibitem{Won_65-1}
\Aue{Wonham, W.} 1965.
Some applications of stochastic differential equations to optimal
  nonlinear filtering.
\textit{SIAM~J.~Control} 2:347--369. 

\bibitem{KP_92-1}
\Aue{Kloeden,~P., and E.~Platen.} 1992. \textit{Numerical solution of stochastic
differential equations.} Berlin: Springer. 636~p.

\bibitem{YZL_04-1}
\Aue{Yin,~G., Q.~Zhang, and Y.~Liu.} 2004.
Discrete-time approximation of Wonham filters.
\textit{J.~Control Theory Applications} 2:1--10.

\bibitem{PR_10-1}
\Aue{Platen, E., and R.~Rendek.} 2010.
Quasi-exact approximation of hidden Markov chain filters.
\textit{Communicat. Stoch. Analys.} 4(1):129--142.

\bibitem{B_18-1}
\Aue{Borisov, A.} 2018. Wonham filtering by observations
with multiplicative noises. \textit{Automat.~Rem.~Contr.} 79(1):39--50.  
doi: 10.1134/ S0005117918010046.
 
  \bibitem{BSh_85-1}
\Aue{Bertsekas, D., and S.~Shreve.} 1996.
\textit{Stochastic optimal control: The discrete-time case}.
Nashua, NH: Athena Scientific. 330~p.
  
  \bibitem{ZhSh_95-1}
  \Aue{Jacod,~J., and A.~Shiryaev.} 2003.
\textit{Limit theorems for stochastic processes.}
Berlin: Springer. 664~p.

\bibitem{S_00-1}
\Aue{Sericola, B.}
2000. Occupation times in Markov processes.
\textit{Commun. Stat.} 16(5):479--510. 

  \bibitem{B_80-1}
\Aue{Borovkov, A.} 1984.
 \textit{Asymptotic methods in queueing theory}. 
 Hoboken, NJ: Wiley-Blackwell.~304~p.

  \bibitem{B_17_1-1}
  \Aue{Borisov, A.} 2017. 
  Klassifikatsiya po ne\-pre\-ryv\-nym nablyu\-de\-miyam s~mul'tiplikativnymi shumami. I. 
  Formuly bayesov\-skoy otsenki [Classification by continuous-time observations
in multiplicative noise. I.~Formulae for Bayesian 
estimate]. \textit{Informatika i~ee Primeneniya~--- Inform.~Appl.}
11(1):11--19. doi: 10.14357/19922264170102.

  \bibitem{B_17_2-1}
\Aue{Borisov, A.} 2017. Klassifikatsiya po nepreryvnym nablyudemiyam 
s~mul'tiplikativnymi summami. II.~Formuly bayesovskoy otsenki 
[Classification by continuous-time observations
in multiplicative noise. II.~Numerical algorithm].
\textit{Informatika i~ee Primeneniya~--- Inform.~Appl.}
11(2):33--41. doi: 10.14357/19922264170204.

\end{thebibliography}

 }
 }

\end{multicols}

\vspace*{-6pt}

\hfill{\small\textit{Received July 10, 2018}}

%\pagebreak

%\vspace*{-18pt}

\Contrl

\noindent
\textbf{Borisov Andrey V.} (b.\ 1965)~--- 
Doctor of Science in physics and mathematics, principal scientist, Institute of
Informatics Problems, Federal Research Center ``Computer Science and Control''
 of the Russian Academy of
Sciences, 44-2 Vavilov Str., Moscow 119333, Russian Federation; 
\mbox{aborisov@frccsc.ru}
\label{end\stat}

\renewcommand{\bibname}{\protect\rm Литература}         %2
\def\stat{rabinovich}

\def\tit{ПРОЦЕДУРА ПОСТРОЕНИЯ МНОЖЕСТВА ПАРЕТО ДЛЯ~ДИФФЕРЕНЦИРУЕМЫХ 
КРИТЕРИАЛЬНЫХ ФУНКЦИЙ}

\def\titkol{Процедура построения множества Парето для~дифференцируемых 
критериальных функций}

\def\aut{Я.\,И.~Рабинович$^1$}

\def\autkol{Я.\,И.~Рабинович}

\titel{\tit}{\aut}{\autkol}{\titkol}

\index{Рабинович Я.\,И.}
\index{Rabinovich Ya.\,I.}


%{\renewcommand{\thefootnote}{\fnsymbol{footnote}} \footnotetext[1]
%{Работа выполнялась с~использованием инфраструктуры Центра коллективного пользования <<Высокопроизводительные вы\-чис\-ле\-ния и~большие данные>> 
%(ЦКП <<Информатика>>) ФИЦ ИУ РАН (г.~Москва).}}


\renewcommand{\thefootnote}{\arabic{footnote}}
\footnotetext[1]{Федеральный исследовательский центр <<Информатика и~управление>> Российской академии наук;
\mbox{jacrabin@rambler.ru}}

\vspace*{-12pt}



\Abst{Универсальная вычислительная процедура многокритериальной 
оптимизации позволяет аппроксимировать множество Парето при предъявлении 
различных требований к~вектору частных критериев эффективности и~множеству 
допустимых решений. В~настоящей работе предполагается, что частные критерии 
эффективности псевдовогнуты в~открытой окрестности компактного выпуклого 
множества допустимых решений, которое может быть задано дифференцируемыми 
функциональными ограничениями. Для построения на основе универсальной 
процедуры конкретных численных методов аппроксимации множества Парето 
предлагается правило выбора начального приближения и~правило перехода от 
текущего опорного решения к~последующему.}

\KW{многокритериальная оптимизация; множество Парето; численные методы 
аппроксимации; универсальная процедура}

\DOI{10.14357/19922264230403}{NEZRGD}
  
\vspace*{-4pt}


\vskip 10pt plus 9pt minus 6pt

\thispagestyle{headings}

\begin{multicols}{2}

\label{st\stat}

\section{Введение}

\vspace*{-3pt}

  Задача построения множества эффективных векторных оценок (множества 
Парето) чрезвычайно актуальна в~самых разнообразных отраслях 
деятельности, таких как разработка сложных сис\-тем технического 
назначения, перспективное планирование или социология. В~настоящей 
работе в~предположении дифференцируемости критериальных функций на 
основе универсальной процедуры~[1] предложен конкретный численный 
метод аппроксимации множества Парето.

\vspace*{-9pt}
  
\section{Постановка задачи}

\vspace*{-3pt}

  Пусть в~$s$-мерном евклидовом пространстве~$\mathbb{R}^s$ задана 
  $m$-мер\-ная непрерывная век\-тор-функ\-ция
  \begin{equation}
  w(x)\in \mathbb{R}^m\,,
  \label{e1-r}
  \end{equation}
образующая вектор частных критериев эффективности, принимающий на 
непустом компактном множестве допустимых решений 
\begin{equation}
X\subset \mathbb{R}^s
\label{e2-r}
\end{equation}
положительные значения $w(x)\hm>0$, удовлетворяя соотношениям:
\begin{equation}
\left.
\begin{array}{c}
w(x)\in w(X)\subset \mathrm{int}\, \mathbb{R}_+^m\,;\\[3pt]
w(X)=\left\{ u\in \mathbb{R}^m\vert u=w(x),\ x\in X\right\}\,;\\[3pt]
\mathbb{R}_+^m =\left\{ u\in \mathbb{R}^m\vert u\geq 0\right\}\,,
\end{array}
\right\}
\label{e3-r}
\end{equation}
так что множество достижимых векторных оценок $w(X)$ из~(\ref{e3-r}) 
принадлежит внутренности $\mathrm{int}\, \mathbb{R}_+^m$ не\-от\-ри\-ца\-тель\-но\-го 
ортанта~$\mathbb{R}_+^m$. Каждый критерий набора
%\begin{equation}
$\left\{ w_k(x)\right\}_{k\in I}$, $I\hm=\left\{ k\vert 1\leq k\leq m\right\}$,
%\label{e4-r}
%\end{equation}
желательно увеличивать на множестве допустимых решений~(2).
% $X\hm\subset  \mathbb{R}^s$.

\smallskip

\noindent
  \textbf{Определение~1.} Векторная оценка $w\hm\in w(X)$ 
\textit{эффективна (слабо эффективна)}, если для всякой векторной оценки 
$u\hm\in w(X)$ сис\-те\-ма неравенств $u\hm\geq w$ несовместна при 
условии, что хотя бы одно неравенство строгое (все неравенства строгие).
  
  Векторная оценка $w\hm\in w(X)$ \textit{доминируема} или 
\textit{определенно неэффективна}, если существует векторная оценка 
$u\hm\in w(X)$, $u\hm> w$.
  
  Всякое допустимое решение $x\hm\in X$, доставляющее эффективное 
(слабо эффективное, доминируемое) значение вектора~$w(x)$, называется 
эффективным (слабо эффективным, доминируемым) \mbox{решением}.
  
  В согласии с~определением~1 множества эффективных 
($X_{e}$), слабо эффективных ($X_0$) и~доминируемых 
($X_{\mathrm{д}}$) решений из множества допустимых решений ($X$) 
подчиняются соотношениям:
  $$
  X_{e}\subset X_0\subset X\,;\ X_0\cap 
X_{\mathrm{д}}=\emptyset\,;\ X=X_0\cup X_{\mathrm{д}}\,,
  $$
где $\emptyset$~--- пустое множество; $\cup (\cap)$~--- символ объединения 
(пересечения) множеств. Соответственно множества 
  эффективных $w(X_{e}$), слабо 
эффективных $w(X_0)$, доминируемых $w(X_{\mathrm{д}})$ и~достижимых  $w(X)$ векторных оценок 
удовлетворяют соотношениям:

\noindent
\begin{multline*}
 w(X_{e}) \subset  w(X_{0}) \subset w(X);\
w(X_0)\cap  w(X_{\mathrm{д}})=\emptyset\,;\\
w(X)=w(X_0)\cup w (X_{\mathrm{д}}).
\end{multline*}
  
  В настоящей работе рассматриваются методы аппроксимации множества 
Парето на основе универсальной вычислительной процедуры~[1]. Введем
  
\pagebreak
  
  \noindent
  \textbf{Определение~2.} Если $\| v\|$~--- норма вектора $v\hm\in 
\mathbb{R}^m$, то величина
  $$
  D(W,U)=\sup\limits_{w\in W} \mathop{\mathrm{inf}}\limits_{u\in U} \| w-
u\|,\enskip \emptyset \not= W,\ U\subset \mathbf{R}^m,
  $$
называется \textit{отклонением} множества~$W$ от множества~$U$, 
а~величина
\begin{multline*}
\Delta (W,U) =\max\left\{ D(W,U), D(U,W)\right\}\,,\\
 \emptyset\not= W,\ U\subset \mathbb{R}^m\,,
\end{multline*}
называется \textit{расстоянием по Хаусдорфу} между множествами~$W$ и~$U$.

\smallskip
  
  \noindent
  \textbf{Определение~3.} Дифференцируемая на открытом множестве 
$A\hm\subset \mathbb{R}^s$ функция~$f$ называется \textit{псевдовогнутой} 
на $A\not= \emptyset$, если для любых~$x,y \hm\in A$ неравенство $\langle 
\nabla f(x), y\hm- x\rangle \hm\leq 0$ влечет неравенство $f(x)\hm\geq f(y)$, где 
$\nabla f$~--- градиент функции~$f$; $\langle \bullet,\bullet\rangle$~--- 
скалярное произведение векторов.
  
  Функция~$f$ называется \textit{псевдовыпуклой}, если функция $-f$ 
псевдовогнута.
  \smallskip
  
  \noindent
  \textbf{Следствие~1.} 
  \begin{enumerate}[1.]
  \item  Функция $\sum\nolimits_{k=1}^m f_k$ псевдовогнута, если ее 
слагаемые псевдовогнуты.
  \item  Точка локального максимума псевдовогнутой на выпуклом 
множестве $A\hm\subset \mathbb{R}^s$ функции~$f$ является точкой ее 
глобального максимума.
  \item Пусть функция~$f$ псевдовогнута на выпуклом множестве 
$A\hm\subset \mathbb{R}^s$ и~положительно определена на~$A$, так что 
$f(x)\hm>0$, $x\hm\in A$. Тогда
  \begin{itemize}
  \item[(а)] функция~$f^n$ псевдовогнута на~$A$, если целая степень 
$n\hm\geq 1$;
  \item[(б)] функция~$f^n$ псевдовыпукла на~$A$, если целая степень 
$n\hm\leq -1$.
  \end{itemize}
  \end{enumerate}
  
  \noindent
  \textbf{Определение~4.} Если непустое множество $X\hm\subset 
\mathbb{R}^s$ задано функциональными ограничениями
\begin{multline*}
%\begin{equation*}
  X=\left\{ x\in \mathbb{R}^s\vert v_k(x)\geq 0,\ k\in K\right\},\\
  K=\{ k\vert 1\leq k\leq n\},
%\end{equation*}
\end{multline*}
левые части которых дифференцируемы, то в~точке $x\hm\in X$ множество
$$
H(x)=\left\{ h\in \mathbb{R}^s\vert \langle \nabla v_k(x),h\rangle>0, \ k\in 
K(x)\right\},
$$
называется \textit{конусом внутренних направлений}, где $K(x)\hm= \{k\hm\in K\vert 
v_k(x)\hm=0\}$~--- множество индексов <<активных>> в~точке $x\hm\in X$ 
функциональных ограничений.

\smallskip

\noindent
\textbf{Следствие~2.} Если $X\hm\subset \mathbb{R}^s$~--- непустой 
выпуклый компакт, то выполняется включение $X\hm\subset \{x\}\hm+ 
\overline{H}(x)$, где многогранный конус~$\overline{H}(x)$~--- замыкание 
конуса внутренних направлений; $A\hm+B$~--- векторная сумма множеств.

  Укажем также известное полезное свойство псевдовогнутых функций~[2].
  
  \smallskip
  \noindent
  \textbf{Лемма~1.}\ \textit{Псевдовогнутая на выпуклом многограннике 
$A\hm= \mathrm{conv} \{a^q\}^r_{q=1}$ функция~$f$ достигает минимума в~крайней 
точке}:
$$
\min\limits_{x\in A} f(x)\hm= \min\limits_{1\leq q\leq r} f(a^q).
$$

\vspace*{-14pt}
   
\section{Универсальная процедура}

\vspace*{-3pt}

  Дадим в~согласии с~[1] строгое описание универсальной процедуры 
аппроксимация множества Парето.
  
  На непустом компактном множестве допустимых решений $X\hm\subset 
\mathbb{R}^s$ строится последовательность множеств 
$\{X_t\}^\infty_{t=1}\hm\subset X$. Если $X_1\hm\subset X$~--- произвольное 
начальное приближение и~известно множество~$X_t$, $t\hm\geq 1$, то 
следующее за ним множество~$X_{t+1}$ подчиняется соотношениям:
  \begin{multline}
  X_{t+1} =\bigcup\limits_{x\in X_t} X_{t+1}(x)\,,\\
  X_{t+1}(x)=\{x\} +\bigcup\limits_{J\in M_t(x)} \{ h(x,J)\} \subset X\,.
   \label{e6-r}
  \end{multline}
  
  В согласии с~(\ref{e6-r}) всякая опорная точка $x\hm\in X_t$ порождает на 
следующем ($t\hm+1$)-м уровне не\-пус\-тую векторную сумму множеств 
$X_{t+1}(x)$, причем на\-прав\-ле\-ние $h(x,J)$ перехода из опорной точки 
$x\hm\in X_t$ в~следующую точку $y\hm= x\hm+ h(x,J)\hm\in X_{t+1}(x)$, 
$J\hm\in M_t(x)$, удовлетворяет условиям:
  \begin{multline}
  h(x,\emptyset) =0\,,\enskip h(x,J)\not= 0\,,\\
   \emptyset\not= J\subset I 
=\{k\vert 1\leq k\leq m\},
  \label{e7-r}
  \end{multline}
где ненулевые направления $h(x,J)\not= 0$ подчиняются соотношениям
\begin{equation}
\left.
\begin{array}{c}
\!\!\!y=x+h(x,J)\in X\cap Y_J(x,\varepsilon) \cap Y(x,\varepsilon)\,,\\[3pt]
\emptyset \not= J\subset I= \{k\vert 1\leq k\leq m\}\,,\\[3pt]
\!\!\!Y_J(x,\varepsilon) =\left\{ y\in \mathbb{R}^s \left\vert \fr{w_k(y)}{w_k(x)} \right. 
\geq 1+\sigma\varepsilon^2\,,\right.\\[3pt]
\!\!\!\left.k\in J\,,\ \fr{w_k(y)}{w_k(x)}\geq 1-\sigma\varepsilon,\ k\in I\backslash 
J\right\}\,;\\[3pt]
\!\!\!Y(x,\varepsilon) =\left\{ y\in \mathbb{R}^s\left\vert \displaystyle\sum\limits_{k\in I} \right. 
\fr{w_k(y)^2}{w_k(x)^2}\leq (1+\sigma\varepsilon)^2\right\},\\[4pt]
\!\!\!\varepsilon=\varepsilon_t(x)\in (0,1),\\[3pt]
\!\!\! \!\!\!\sigma \!=\!\left[ 2\max\limits_{z\in X} 
\sum\limits_{k\in I} w_k(z)\right]^{-1}  \!\!\!\!\min\limits_{z\in X} \min\limits_{k\in I} 
w_k(z)\in \left(\! 0, \fr{1}{2}\right].
\end{array}\!
\right\}\!\!
\label{e8-r}
\end{equation}
  
  Последовательность множеств~(\ref{e6-r}) в~каждой опорной точке 
$x\hm\in X_t$ ветвится, причем степень ее ветвления $\vert 
\mathbf{M}_t(x)\vert$ определяет множество не вложенных друг в~друга 
подмножеств $\mathbf{M}_t(x)\hm\subset 2^I$, заданное соотношениями:
\begin{equation}
\left.
\begin{array}{rl}
\!\!\!\!\!\mathbf{M}_t(x) &=\begin{cases}
\mathbf{N}_t(x), &\mbox{если } \mathbf{N}_t(x)\not=\emptyset\,;\\[3pt]
\{\emptyset\}, &\mbox{если } \mathbf{N}_t(x)=\emptyset\,;
\end{cases}\\[12pt]
\!\!\!\!\!\mathbf{N}_t(x) &= {}\\
&\hspace*{-10mm}{}=\left\{
J\subset I\left\vert\begin{array}{l}
X\cap Y_J(x,\varepsilon) \cap Y(x,\varepsilon) \not= \emptyset\,;\\[3pt]
 X\cap Y_M(x,\varepsilon) \cap Y(x,\varepsilon) 
\not= \emptyset,\\[3pt]
 \hspace*{15mm}M\not= J\subset M\subset I.
 \end{array}
 \right.
 \right\}
\end{array}
\right\}
\label{e9-r}
\end{equation}
  
  Наибольшая степень ветвления последовательности~(\ref{e6-r}) совпадает 
с~наибольшим числом не вложенных друг в~друга подмножеств множества 
\mbox{номеров} частных критериев эффективности $I\hm= \{ k\vert 1\hm\leq k\hm\leq 
m\}$, так что
$$
\vert \mathbf{M}_t(x)\vert \leq \fr{m!}{\lfloor m/2\rfloor! \left(m-\lfloor m/2\rfloor\right)!} \,,
$$
где $\vert A\vert$~--- число элементов в~конечном множестве~$A$; $\lfloor 
z\rfloor$~--- целая часть чис\-ла~$z$.
  Соотношения~(\ref{e8-r}) и~(\ref{e9-r}) включают величину $\varepsilon\hm= 
\varepsilon_t(x)\hm\in (0,1)$~--- параметр возмущения, который в~начальной 
точке~$x^1$ и~в~любых последующих точках~$x^t\hm\in X_t$, 
$x^{t+1}\hm\in X_{t+1}(x^t)$ последовательности~(\ref{e6-r}) таких, что
  \begin{equation}
  x^{t+1}=x^t+h(x^t, J_t),\enskip J_t\in M_t(x)\,,
  \label{e10-r}
  \end{equation}
  
  \vspace*{-4pt}
  
  \noindent
удовлетворяет условиям:

\vspace*{-4pt}

\noindent
\begin{multline}
\varepsilon_1\left(x^1\right) =\kappa\in (0,1),\\
\varepsilon_{t+1}\left(x^{t+1}\right) =\begin{cases}
\kappa \varepsilon_t(x^t), &\mbox{если } Q_t=\emptyset\,;\\
\varepsilon_t(x^t), &\mbox{если } Q_t\not= \emptyset\,,
\end{cases}\\
Q_t=\bigcap\limits_{q\leq t},\ \varepsilon_q(x^q)=\varepsilon_t(x^t)J_q\,,
\label{e11-r}
\end{multline}

\vspace*{-4pt}

\noindent
где величина~$\kappa$ определяет степень дробления параметра 
возмущения. Эта величина может быть выбрана любой на интервале $(0,1)$, 
но остается фиксированной на протяжении всей вычислительной процедуры.

\smallskip

\noindent
\textbf{Замечание~1.} Комментируя процедуру~(\ref{e6-r})--(\ref{e11-r}), 
следует подчеркнуть:
\begin{itemize}
\item включение $y\hm\in Y_J(x,\varepsilon)$ доставляет относительное 
(порядка~$\varepsilon^2$) \textit{увеличение} значений критериев из 
множества $\{ w_k\}_{k\in J}$ за счет возможного относительного 
\textit{уменьшения} (на величину порядка~$\varepsilon$) остальных 
критериев;
\item включение $y\hm\in Y(x,\varepsilon)$ препятствует слишком резкому 
росту значений частных критериев, что на поздних этапах вычислительной 
процедуры не позволяет переходить от одной эффективной векторной оценки 
к~другой и~возвращаться обратно через все множество достижимых 
векторных оценок;
\item пересечение $X\cap Y_J (x,\varepsilon)$~--- выпуклый компакт, 
поскольку~$X$~--- выпуклый компакт, $ Y_J(x,\varepsilon)$~--- замкнуто 
и~выпукло в~согласии с~(\ref{e8-r}) и~следствием~1;
\item пересечение множеств $X\cap Y(x,\varepsilon)$ компактно, но не 
выпукло, поскольку согласно~(\ref{e8-r}) множество $ Y(x,\varepsilon)$ 
выпукло лишь в~некоторых частных случаях (например, если псевдовогнуты 
компоненты век\-тор-функ\-ций $\pm w(x)$).
\end{itemize}
  
  В согласии с~доказательством, предложенным в~работе~[1], заданная 
соотношениями~(\ref{e6-r})--(\ref{e11-r}) последовательность множеств $\{ 
w(X_t)\}^\infty_{t=1}$ аппроксимирует множество Парето в~следующем 
смысле.
  
  \smallskip
  
  \noindent
  \textbf{Теорема~1.} \textit{Пусть в}~(\ref{e1-r}) \textit{компоненты  
век\-тор-функ\-ции $w\hm\in \mathbb{R}^m$ положительно определены 
и~псевдовогнуты в~открытой окрестности непустого выпуклого 
компакта~$X$. Тогда отклонение множества Парето $w(X_{e})$ 
от аппроксимирующего множества~$w(X_t)$ и~отклонение 
аппроксимирующего множества~$w(X_t)$ от множества 
Слейтера~$w(X_0)$ стремятся к~нулю с~рос\-том номера 
аппроксимации}~$t$:
\begin{multline*}
  \lim\limits_{t\to\infty} D\left( w(X_{\mathrm{е}}), 
w(X_t)\right)={}\\
{}=\lim\limits_{t\to\infty} D\left( w(X_t), w(X_0)\right)=0\,.
\end{multline*}

\section{Численные методы на~основе~процедуры}

  Пусть условия теоремы~1 выполняются. Универсальная  
процедура~(\ref{e6-r})--(\ref{e11-r}) не содержит однозначного рецепта 
построения последовательности $\{ w(X_t)^\infty_{t=1}$, 
аппроксимирующей множество Парето, поскольку не дает прямого указания, 
в~какую именно новую опорную точку $y\hm= x\hm+ h(x,J)\hm\in X\cap 
Y_J(x,\varepsilon)\cap Y(x,\varepsilon)$ следует переходить из исходной 
опорной точки $x\hm\in X_t$.
  
  Началу работы вычислительной процедуры предшествует процедура 
выбора начального приближения. В~согласии с~утверж\-де\-ни\-ем  
тео\-ре\-мы~1, сходимость процедуры~(\ref{e6-r})--(\ref{e11-r}) обеспечена 
при старте из любых точек~$X$. Вмес\-те с~тем эффективность процедуры 
существенным образом зависит от выбора начальной точки (либо состава 
начального множества $X_1\hm\subset X$).
  
  Процедура~(\ref{e6-r})--(\ref{e11-r}) становится реальной вы\-чис\-ли\-тель\-ной 
процедурой (численным методом), если она дополнена \textit{правилом 
выбора}:
  \begin{itemize}
  \item[(а)] начальной точки $\{x^1\}\hm\subset X$ (множества начальных 
точек $X_1\hm\subset X$, $\vert X_1\vert \hm>1$, если в~работе численного 
метода предусмотрен мультистарт);
  \item[(б)] каждой последующей опорной точки $y\hm \in X\hm\cap 
Y_J(x,\varepsilon) \hm\cap Y(x,\varepsilon)$.
  \end{itemize}
  
\section{Выбор начального приближения}

  Согласно замечанию~1, для формирования начального 
приближения~$X_1$ можно использовать упрощенный вариант 
процедуры~(\ref{e6-r})--(\ref{e11-r}), где параметр возмущения 
$\varepsilon\hm\in (0,1)$ фиксирован, ограничения сверху на приращения 
частных критериев $\{w_k\}_{k\in I}$ отменяются~--- условие~(\ref{e8-r}) 
заменятся требованием
  \begin{equation}
  \left.
  \begin{array}{c}
 \!\! \!\!\!\! y=x+h(x,J) \in X \cap Y_J(x,\varepsilon),\\[3pt]
  \!\! \!\!\!\! \emptyset \not= J\subset I= \{k\vert 1\leq k\leq m\}\,;\\[3pt]
  \!\! \!\!\!\!Y_J(x,\varepsilon) =\left\{ y\in \mathbb{R}^s\left\vert 
\fr{w_k(y)}{w_k(x)}\right. \geq 1+\sigma\varepsilon^2\,,\right.\\[3pt]
 \!\!  \!\!\!\!\left. k\in J\,,\ \fr{w_k(y)}{w_k(x)}\geq 1-\sigma\varepsilon,\ k\in I\backslash 
J\right\}\!,\\[3pt]
 \!\!\!\!\!\! \sigma\!=\! \left[ 2\max\limits_{z\in X}\displaystyle \sum\limits_{k\in I}\! w_k(z)\right]^{-1}
  \!\!\!\!\!\min\limits_{z\in X} \min\limits_{k\in I} w_k(z)\in \left(0, \fr{1}{2}\right]\!,
  \end{array}\!\!
  \right\}\!\!\!
  \label{e12-r}
  \end{equation}
тогда как множество подмножеств в~(\ref{e9-r}) приобретает вид:
\begin{equation}
\mathbf{M}_t(x)=\left\{ J\subset I\vert \emptyset \not= J,\ X\cap 
Y_J(x,\varepsilon)\not= \emptyset\right\}
\label{e13-r}
\end{equation}
и содержит все такие подмножества $J\hm\subset I\hm= \{ k\vert 1\hm\leq 
k\hm\leq m\}$, что каждый из критериев $\{ w_k\}_{k\in J}$ допускает 
относительное (порядка~$\varepsilon^2$) увеличение.
  
  Сформулируем на основе упрощенной процедуры~(\ref{e6-r}), (\ref{e7-r}), 
(\ref{e12-r}) и~(\ref{e13-r}) правило выбора начального приближения.
  
  \smallskip
  
  \noindent
  \textbf{Правило~А.} В~качестве решения $y^J\hm\in X\cap 
Y_J(x,\varepsilon)$ принимается проекция внешней точки $x\hm\in X$, 
$x\not\in Y_J(x,\varepsilon)$, на выпуклый компакт $X\cap Y_J(x,\varepsilon)$:
  \begin{equation}
  y^J=\argmin\limits_{y\in X\cap Y_J(x,\varepsilon)} \| x-y\|,\enskip J\in M_t(x),
  \label{e14-r}
  \end{equation}
для чего следует решить задачу выпуклого программирования 
(минимизировать квадратичную функцию $\| x\hm- y\|^2$ на выпуклом 
компакте).

  Если компактное выпуклое допустимое множество~$X$ задано 
дифференцируемыми функциональными ограничениями, приближенное 
решение задачи~(\ref{e14-r}) может быть получено следующим образом.
  
  Согласно определению~4 и~следствию~2, в~каж\-дой опорной точкой 
$x\hm\in X$ задана прямоугольная мат\-ри\-ца~$V(x)$ такая, что замыкание 
со\-от\-вет\-ст\-ву\-юще\-го конуса внут\-рен\-них на\-прав\-ле\-ний~$H(x)$ удовле\-тво\-ря\-ет 
соотношениям:
  \begin{multline}
  \overline{H}(x)=\left\{ h\in \mathbb{R}^s\vert V(x)h\geq 0\right\}\,,\
  X\subset \{x\} +\overline{H}(x),\\
  X\backslash \left\{ \{x\} +\overline{H}(x)\right\}=\emptyset\,.
  \label{e15-r}
  \end{multline}
  
  Определим многогранный конус
  \begin{equation}
  H_J(x)=\left\{ h\in \mathbb{R}^s\vert W_J(x) h\geq 0\right\},
  \label{e16-r}
  \end{equation}
где строками матрицы $W_J(x)$ служат векторы $\nabla w_k(x)$, $k\hm\in J$. 
Согласно~(\ref{e15-r}), (\ref{e16-r}) мож\-но утверж\-дать, что
\begin{equation}
X\cap Y_J(x,\varepsilon) \subset \{x\} +\overline{H}(x)\cap H_J(x),
\label{e17-r}
\end{equation}
поскольку для всякого вектора $y\hm= x\hm+ h \hm\in X$, удовле\-тво\-ря\-юще\-го 
при некотором номере $k\hm\in J$ неравенству $\langle \nabla w_k(x),h\rangle 
\hm<0$, выполняется, согласно определению~3, условие $w_k(x)\hm\geq 
w_k(y)$, что ввиду~(\ref{e12-r}) влечет $y\not\in Y_J(x,\varepsilon)$.

  Многогранный конус $\overline{H}(x)\cap H_J(x)$ можно представить 
в~виде конической оболочки некоторого \textit{остова}~$B(x)$: 

\vspace*{-6pt}

\noindent
  \begin{multline}
 \! \!\!\overline{H}(x)\cap H_J(x)\!=\!\left\{ h\in \mathbb{R}^s\vert V(x)h\geq 0,\ 
W_J(x)h\geq 0\right\}\!={}\\
  {}= \left\{ h\in \mathbb{R}^s\vert h=B(x)\lambda,\ \lambda\geq0\right\},
  \label{e18-r}
  \end{multline}
  
  \vspace*{-4pt}

\noindent
где для составленной из единичных век\-то\-ров-столб\-цов матрицы 

\vspace*{-6pt}

\noindent
\begin{multline}
B(x)=\left[ b^1(x),b^2(x),\ldots , b^r(x)\right],\\
\| b^j(x)\|=1\,,\enskip 1\leq j\leq r\,,
\label{e19-r}
\end{multline}

\vspace*{-4pt}

\noindent
существуют конструктивные методы по\-стро\-ения~\cite{3-r}.

  Согласно включениям~(\ref{e15-r}), (\ref{e17-r}), пересечение 
  $\{X \hm- \{x\}\}\cap H_J(x)$ можно аппроксимировать, выбрав из конуса 
$\overline{H}(x)\cap H_J(x)$ конечный <<пучок>> лучей.
  
  \smallskip
  
  \noindent
  \textbf{Лемма~2.} \textit{Если для малого $\rho\hm>0$ выполняются 
соотношения}

\vspace*{-6pt}

\noindent
  \begin{multline}
%  \left.
%  \begin{array}{rl}
  \Delta(\Lambda_r,\Lambda) =D(\Lambda_r,\Lambda)=\sup\limits_{\alpha\in 
\Lambda_r} \mathop{\mathrm{inf}}\limits_{\beta\in \Lambda} \| \alpha-\beta\|\leq 
\rho\,,\\
  \Lambda_r=\left\{ \lambda\in \mathbb{R}^s \left\vert \sum\limits^r_{j=1} \right.
\lambda_j=1\,,\ \lambda\geq 0\right\},\\
 \Lambda =\{ \lambda^q\}^Q_{q=1} \subset \Lambda_r\,,
    \label{e20-r}
  \end{multline}
  
  \vspace*{-4pt}

\noindent
\textit{где конечное множество~$\Lambda$ является $\rho$-ап\-прок\-си\-ма\-ци\-ей $(r\hm-1)$-мер\-но\-го стандартного симплекса~$\Lambda_r$, то 
ввиду}~(\ref{e18-r})--(\ref{e20-r}) \textit{вложенный <<пучок>> лучей}

\vspace*{-6pt}

\noindent
\begin{multline}
L_q(x)=\left\{ q\in \mathbb{R}^s\left\vert  g=\gamma\sum\limits^r_{j=1} \lambda_j^q 
b^j(x), \gamma\geq 0\right.\right\}\subset{}\\
{}\subset \overline{H}(x)\cap H_J(x),\enskip 1\leq q\leq Q\,,
\label{e21-r}
\end{multline}

\vspace*{-4pt}

\noindent
\textit{аппроксимирует выпуклый компакт $\{X\hm -\{x\}\} \hm\cap H_J(x)$ 
в~следующем смысле}.

\textit{Если можно указать величину $\delta\hm>0$ такую, что ненулевой 
вектор $h\hm\in \mathbb{R}^s$ содержится в~множестве $\{ X\hm- \{x\} \} 
\cap H_J(x)$ вместе с~шаром}
$$
U_\delta(h) =\left\{ q\in \mathbb{R}^s\vert \| h-g\| \leq\delta\right\}\subset \left\{ 
X-\{x\}\right\} \cap H_J(x),
$$
\textit{то в}~(\ref{e21-r}) \textit{найдется луч, удовлетворяющий 
соотношению $L_q(x)\cap U_\delta(h)\bot=\emptyset$ при условии $\rho r 
d_X\hm\leq \delta$, где $d_X\hm= \max\nolimits_{x^1, x^2\in X} \| x^1\hm- 
x^2\|$~--- диаметр компакта~$X$}.


  Следовательно, если <<мишень>> $X\cap Y_J(x,\varepsilon)$ конечного 
размера, то один из заданных соотношением~(\ref{e21-r}) лучей $\{x\} \hm+ 
L_1(x)$ в~нее попадет. Строгое утверждение состоит в~следующем: при 
условии $int \{ X\cap Y_J(x,\varepsilon)\}\not= \emptyset$ из 
включения~(\ref{e17-r}) по лемме~2 следует, что в~конечном множестве 
лучей~(\ref{e21-r}) всегда можно указать луч~$L_q(x)$ такой, что
  \begin{equation}
  \left\{ \{x\} +L_q(x)\right\} \cap X\cap Y_J(x,\varepsilon)\not= \emptyset\,,
  \label{e22-r}
  \end{equation}
если конечное множество $\{\lambda^q\}^Q_{q=1}$ аппроксимирует 
стандартный симплекс~$\Lambda_r$ с~тре\-бу\-емой точ\-ностью. Метод 
равномерной аппроксимации стандартного симплекса конечным множеством 
точек предложен в~работе~\cite{4-r}.

  Но тогда хотя бы одна из проекций
  \begin{equation}
  p^q=\argmin\limits_{y\in \{ \{x\} +L_q(x)\} \cap Y_J(x,\varepsilon)} \| x-
y\|\,,\enskip 1\leq q\leq Q\,,
 \label{e23-r}
  \end{equation}
внешней точки $x\hm\in X\backslash Y_J(x,\varepsilon)$ на пересечение $\{ \{ 
x\} \hm+ L_q(x)\} \cap Y_J(x,\varepsilon)$ удовлетворяет включению 
$p^q\hm\in X\cap Y_J(x,\varepsilon)$, поскольку включения $x\hm\in X$ 
и~$p^q\hm\in Y_J(x,\varepsilon)$ выполняются, так что отрицание $p^q\not\in 
X$ противоречит утверждению~(\ref{e22-r}) ввиду выпуклости 
множества~$X$. Следовательно, по правилу~А выбор опорной точки 
$y^J\hm= p^q\hm\in X\cap Y_J(x,\varepsilon)$ требует решения не более 
чем~$Q$~задач одномерного квадратичного программирования~(\ref{e23-r}).

  Если применение упрощенной процедуры на этом заканчивается, 
множество $\{ y^J\}_{J\in \mathbf{M}_t(x)}$, составленное из векторов, 
удовлетворяющих условиям~(\ref{e12-r}), (\ref{e13-r}), может быть принято в~качестве начального множества~$X_1$. Тем самым согласно правилу~А по 
упрощенной процедуре через несколько шагов\linebreak (число которых определяется 
спецификой ре\-ша\-емой задачи) будет построено начальное приближение 
$X_1\hm\subset \mathbb{R}^s$, образ которого~--- множество 
$w(X_1)\hm\subset \mathbb{R}^m$~--- может представлять собой грубую 
\mbox{аппроксимацию} множества Парето по всему <<фронту>>.

\vspace*{-6pt}

  
\section{Правило выбора шага универсальной процедуры}

\vspace*{-3pt}


  Если задано начальное множество~$X_1$, в~согласии с~результатом 
теоремы~1 универсальная процедура~(\ref{e6-r})--(\ref{e11-r}) обеспечивает 
построение последовательности множеств $\{ w(X_t)\}^\infty_{t=1}$, 
аппроксимирующих множество Парето. Следует лишь принять правило 
выбора шага процедуры, для чего в~соотношениях~(\ref{e8-r}), (\ref{e9-r}) 
требуется установить конкретное правило перехода от опорного решения 
$x\hm\in X_t$ к~новым опорным решениям
  \begin{equation}
  y^J\in X\cap Y_J(x,\varepsilon) \cap Y(x,\varepsilon),\ J\in M_t(x).
  \label{e24-r}
  \end{equation}
  
  Сформулируем конструктивный способ выбора опорного 
решения~(\ref{e24-r}).
  
  \smallskip
  
  \noindent
  \textbf{Правило~Б.} Пусть множество допустимых решений~--- выпуклый 
компакт~$X$~--- задано дифференцируемыми функциональными 
ограничениями. Согласно определению~4 и~следствию~2, в~каждой опорной 
точке $x\hm\in X$ известна прямоугольная мат\-ри\-ца~$V(x)$ такая, что 
замыкание со\-от\-вет\-ст\-ву\-юще\-го конуса внутренних на\-прав\-ле\-ний~$H(x)$ 
удовлетворяет условиям~(\ref{e15-r}), а~его пересечение с~многогранным 
конусом~(\ref{e16-r}) удовле\-тво\-ря\-ет условиям~(\ref{e17-r}).
  
  Если в~правой части включения~(\ref{e24-r}) пересечение множеств имеет 
внутренность $\mathrm{int}\, \{ X\hm\cap Y_J(x,\varepsilon) \hm\cap Y(x,\varepsilon)\not= 
\emptyset$, из включения~(\ref{e17-r}) по лемме~2 следует, что в~конечном 
множестве лучей~(\ref{e21-r}) можно указать луч~$L_q(x)$ такой, что луч 
$\{x\}\hm+ L_q(x)$ содержит решение~(\ref{e24-r}):
  \begin{equation}
  \!Z_J\!=\! \left\{ \{x\} +L_q(x)\right\}\cap X\cap Y_J(x,\varepsilon) \cap 
Y(x,\varepsilon) \not= \emptyset\,.
  \label{e25-r}
  \end{equation}
  
  Найти опорное решение~(\ref{e24-r}) на множестве~(\ref{e25-r}) помогает
  
  \smallskip
  
  \noindent
  \textbf{Лемма~3.} \textit{Условие}~(\ref{e25-r}) \textit{в соответствии 
с~леммой~$1$ влечет утверждение}

\vspace*{-6pt}

\noindent
  \begin{multline*}
  \left\{ p^q,g^q\right\}\cap X\cap Y_J(x,\varepsilon)\cap 
Y(x,\varepsilon)\not=\emptyset\,,\\
  p^q=\argmin\limits_{y\in \{\{x\}+L_q(x)\} \cap Y_J(x,\varepsilon)} \| x-
y\|,\\
  g^q=\argmax\limits_{y\in\{\{x\}+L_q(x)\}\cap X} \| x-y\|.
%  \end{array}
 % \right\}
%  \label{e26-r}
  \end{multline*}
  
  \vspace*{-4pt}
  
  Тем самым по правилу~Б для определения опорной точки~(\ref{e24-r}) 
достаточно проверить выполнение включения~(\ref{e24-r}) не более чем для 
$2Q$ точек, выбранных на множестве, состоящем из~$Q$~лучей:

\vspace*{-6pt}

\noindent
\begin{multline}
%\left.
%  \begin{array}{c}
   p^q,g^q \in \{x\}+L_q(x),\ 1\leq q\leq Q\,,\\
  p^q=\argmin\limits_{y\in \{\{x\}+L_q(x)\} \cap Y_J(x,\varepsilon)} \| x-
y\|,\\
  g^q=\argmax\limits_{y\in\{\{x\}+L_q(x)\}\cap X} \| x-y\|,
%  \end{array}
%  \right\}
  \label{e27-r}
  \end{multline}
  
  \vspace*{-4pt}
  
  \noindent
где согласно соотношению~(\ref{e8-r}) точка~$p^q$ есть ближайшая 
к~опорной точке~$x$ точка отрезка $\{x\}\hm+ L_q(x)\cap 
Y_J(x,\varepsilon)$, тогда как~$q^q$~--- наиболее удаленная от~$x$ точка 
отрезка $\{x\} \hm+ L_q(x)\cap X$.

\vspace*{-9pt}

\section{Заключение}

\vspace*{-3pt}

  Задача синтеза численных методов аппроксимации множества Парето 
исследовалась на основе предложенной в~работе~[1] универсальной 
вычисли-\linebreak\vspace*{-12pt}

\pagebreak

\noindent
тельной процедуры при следующих предположениях: множество 
допустимых решений~$X$~--- непустой выпуклый компакт (заданный, если 
потребуется, дифференцируемыми функциональными ограничениями), 
компонентами вектора частных критериев эффективности служат функции, 
псевдовогнутые в~некоторой открытой окрестности~$X$.
  
  Разработанный численный метод аппроксимации множества Парето 
определяется: правилом~А выбора начального приближения и~правилом~Б 
выбора каждого последующего опорного решения.
  
  Построение каждой из точек начального приближения сводится по 
правилу~А к~решению одномерной задачи квадратичного программирования 
на каждом луче из конечного <<пучка>> лучей, аппроксимирующих 
многогранный конус, вложенный в~конус возможных направлений. 
  
  По правилу~Б на каждом этапе вычислительной процедуры переход от 
текущего опорного решения к~последующему предполагает аппроксимацию 
(согласно включению~(\ref{e21-r})) многогранного конуса конечным 
множеством лучей $L_q(x)$, $1\hm\leq q\hm\leq Q$, и~выбор опорного 
решения среди конечного чис\-ла пар решений $\{ p^q, g^q\}\hm\subset \{x\} 
\hm+ L_q(x)$ в~согласии с~соотношениями~(\ref{e27-r}).

\vspace*{-6pt}

{\small\frenchspacing
 {\baselineskip=10.6pt
 %\addcontentsline{toc}{section}{References}
 \begin{thebibliography}{9}
\bibitem{1-r}
\Au{Рабинович Я.\,И.} Универсальная процедура построения множества Парето~// 
Ж.~вычисл. матем. матем. физ., 2017. Т.~57. №\,1. С.~28--47.
\bibitem{2-r}
\Au{Базара М., Шетти~К.} Нелинейное программирование. Теория и~алгоритмы~/ Пер. 
с~англ.~--- М.: Мир, 1986. 583~с. (\Au{Bazaraa~M.\,S., Shetty~C.\,M.}  
Nonlinear programming: Theory and algorithms.~--- New York, NY, USA: 
Wiley, 1979. 872~p.)
\bibitem{3-r}
Линейные неравенства и~смежные вопросы~/ Под ред. Г.\,У.~Куна, 
А.\,У.~Таккера; пер. с~англ.~--- Annals of mathematics studies ser.~--- М.: ИЛ, 1959. 470~с.
(Linear inequalities and 
related systems~/ Eds. H.\,W.~Kuhn, A.\,W.~Tucker.~--- Annals of mathematics studies ser.~---  Princeton, NJ, USA: Princeton University Press, 1956. 
322~p.)
\bibitem{4-r}
\Au{Рабинович Я.\,И.} Численные методы оценивания приближенных решений в~задачах 
многокритериальной оптимизации~// Докл.\ Акад.\ наук, 2015. Т.~462. №\,2. С.~151--153.

\end{thebibliography}

 }
 }

\end{multicols}

\vspace*{-10pt}

\hfill{\small\textit{Поступила в~редакцию 21.10.22}}

\vspace*{6pt}

%\pagebreak

%\newpage

%\vspace*{-28pt}

\hrule

\vspace*{2pt}

\hrule



\def\tit{PROCEDURE OF~CONSTRUCTING A~PARETO SET 
FOR~DIFFERENTIABLE CRITERIA FUNCTIONS}


\def\titkol{Procedure of~constructing a~Pareto set 
for~differentiable criteria functions}


\def\aut{Ya.\,I.~Rabinovich}

\def\autkol{Ya.\,I.~Rabinovich}

\titel{\tit}{\aut}{\autkol}{\titkol}

\vspace*{-14pt}


\noindent
Federal Research Center ``Computer Science and Control'' of the Russian Academy 
of Sciences, 44-2~Vavilov Str., Moscow 119333, Russian Federation


\def\leftfootline{\small{\textbf{\thepage}
\hfill INFORMATIKA I EE PRIMENENIYA~--- INFORMATICS AND
APPLICATIONS\ \ \ 2023\ \ \ volume~17\ \ \ issue\ 4}
}%
 \def\rightfootline{\small{INFORMATIKA I EE PRIMENENIYA~---
INFORMATICS AND APPLICATIONS\ \ \ 2023\ \ \ volume~17\ \ \ issue\ 4
\hfill \textbf{\thepage}}}

\vspace*{2pt}
      
 


\Abste{A ubiquitous computational procedure for the multicriteria optimization 
allows one to approximate the Pareto set under different requirements to the vector 
of particular efficiency criteria and the set of feasible solutions. In the paper, it is 
assumed that particular efficiency criteria are pseudoconcave in an open 
neighborhood of a compact convex set of feasible solutions which can be given by 
differentiable functional constraints. To build specific numerical methods for 
approximating the Pareto set, a rule for choosing the initial approximation and 
a~rule for moving from the current reference solution to the next one are 
proposed.}


\KWE{multicriteria optimization; Pareto set; numerical methods of 
approximation; universal procedure}

\DOI{10.14357/19922264230403}{NEZRGD}

%\vspace*{-20pt}

%\Ack
%\vspace*{-3pt}
%\noindent
  

\vspace*{-5pt}

  \begin{multicols}{2}

\renewcommand{\bibname}{\protect\rmfamily References}
%\renewcommand{\bibname}{\large\protect\rm References}

{\small\frenchspacing
 {%\baselineskip=10.8pt
 \addcontentsline{toc}{section}{References}
 \begin{thebibliography}{9} 
 
 \vspace*{-1pt}
 
\bibitem{1-r-1}
\Aue{Rabinovich, Ya.\,I.} 2017. Universal procedure for constructing a~Pareto set. 
\textit{Comp. Math. Math. Phys.} 57(1):45--63. doi: 
10.1134/S0965542517010122.
\bibitem{2-r-1}
\Aue{Bazaraa, M.\,S., and C.\,M.~Shetty.} 1979. \textit{Nonlinear programming: 
Theory and algorithms}. New York, NY: Wiley. 872~p.
\bibitem{3-r-1}
Kuhn, H.\,W., and A.\,W.~Tucker, eds. 1956. \textit{Linear inequalities and 
related systems}. Annals of mathematics studies ser.  Princeton, NJ: Princeton University Press. 
322~p.
\bibitem{4-r-1}
\Aue{Rabinovich, Ya.\,I.} 2015. Numerical methods for estimating approximate 
solutions of multicriteria optimization problems. \textit{Dokl. Math.} 
91(3):384--386. doi: 10.1134/ S1064562415030114. EDN: UFAQXF.

\end{thebibliography}

 }
 }

\end{multicols}

\vspace*{-8pt}

\hfill{\small\textit{Received October 21, 2022}} 

\vspace*{-18pt}

\Contrl

\vspace*{-4pt}

\noindent
\textbf{Rabinovich Yaacov I.} (b.\ 1947)~--- Candidate of Science (PhD) in 
physics and mathematics, senior scientist, Federal Research Center ``Computer 
Science and Control'' of the Russian Academy of Sciences, 44-2~Vavilov Str., 
Moscow 119333, Russian Federation; \mbox{jacrabin@rambler.ru}

      





\label{end\stat}

\renewcommand{\bibname}{\protect\rm Литература}  %3
\def\stat{lapko}

\def\tit{НЕПАРАМЕТРИЧЕСКИЙ АЛГОРИТМ АВТОМАТИЧЕСКОЙ КЛАССИФИКАЦИИ 
ДАННЫХ ДИСТАНЦИОННОГО ЗОНДИРОВАНИЯ}

\def\titkol{Непараметрический алгоритм автоматической классификации 
данных дистанционного зондирования}

\def\aut{В.\,П.~Тубольцев$^1$, А.\,В.~Лапко$^2$, В.\,А.~Лапко$^3$}

\def\autkol{В.\,П.~Тубольцев, А.\,В.~Лапко, В.\,А.~Лапко}

\titel{\tit}{\aut}{\autkol}{\titkol}

\index{Тубольцев В.\,П.}
\index{Лапко А.\,В.}
\index{Лапко В.\,А.}
\index{Tuboltsev V.\,P.}
\index{Lapko A.\,V.}
\index{Lapko V.\,A.}


%{\renewcommand{\thefootnote}{\fnsymbol{footnote}} \footnotetext[1]
%{Работа выполнялась с~использованием инфраструктуры Центра коллективного пользования 
%<<Высокопроизводительные вычисления и~большие данные>> (ЦКП <<Информатика>>) ФИЦ ИУ 
%РАН.}}


\renewcommand{\thefootnote}{\arabic{footnote}}
\footnotetext[1]{Сибирский государственный университет науки и технологий им. академика М.\,Ф.~Решетнёва, \mbox{vitalya.98@mail.ru}}
\footnotetext[2]{Сибирский государственный университет науки и технологий им.\ академика М.\,Ф.~Решетнёва; 
Институт вычислительного моделирования Сибирского отделения Российской академии наук, 
\mbox{lapko@icm.krasn.ru}}
\footnotetext[3]{Сибирский государственный университет науки и технологий им.\ академика М.\,Ф.~Решетнёва; 
Институт вычислительного моделирования Сибирского отделения Российской академии наук, 
\mbox{valapko@yandex.ru}}

\vspace*{-14pt}



  

  \Abst{Предлагается непараметрический алгоритм автоматической классификации 
статистических данных большого объема. Эта задача возникает при обработке данных 
дистанционного зондирования природных объектов. Рассматриваемый алгоритм 
предполагает сжатие исходной информации на основе декомпозиции многомерного 
пространства признаков. В~результате статистическая выборка большого объема 
преобразуется в массив данных, составленный из центров многомерных интервалов 
дискретизации и соответствующих им частот принадлежности случайных величин. 
Полученная информация используется при синтезе регрессионной оценки плотности 
вероятности. Под классом понимается компактная группа наблюдений случайной величины, 
соответствующая одномодальному фрагменту плотности вероятности. На этой основе 
разрабатывается непараметрический алгоритм автоматической классификации, который 
основан на последовательной процедуре проверки близости центров многомерных 
интервалов дискретизации и соотношений между частотами принадлежности случайных 
величин из исходной выборки этим интервалам. Для повышения вычислительной 
эффективности пред\-ла\-га\-емо\-го алгоритма автоматической классификации используется 
многопоточный метод его программной реализации. Практическая зна\-чи\-мость 
разработанного алгоритма автоматической классификации под\-тверж\-да\-ет\-ся результатами его 
применения при оценивании со\-сто\-яния лесных массивов по данным дистанционного 
зондирования.}
  
  \KW{автоматическая классификация; выборки большого объема; дискретизация области 
значений случайных величин; регрессионная оценка плотности вероятности; данные 
дистанционного зондирования}

\DOI{10.14357/19922264230404}{MPEWAW}
  
\vspace*{-4pt}


\vskip 10pt plus 9pt minus 6pt

\thispagestyle{headings}

\begin{multicols}{2}

\label{st\stat}
  
\section{Введение}

\vspace*{-4pt}

  Обнаружение компактных групп наблюдений~--- одна из основных задач 
системного анализа данных дистанционного зондирования объектов различной 
природы~[1, 2]. 

Обзор методов автоматической классификации представлен 
в~работах~[3, 4]. Особое внимание уделяется разработке алгоритмов 
автоматической классификации для обнаружения компактных\linebreak групп 
наблюдений, со\-от\-вет\-ст\-ву\-ющих одномодальным фрагментам плот\-ности 
ве\-ро\-ят\-ности признаков ис\-сле\-ду\-емых объектов. Подобное определение класса 
было введено академиком Я.\,З.~\mbox{Цыпкиным}~[5] и~развито в~работах 
профессора В.\,И.~Васильева~[6] с использованием непараметрической оценки 
плотности вероятности Ро\-зен\-блат\-та--Пар\-зе\-на~[7--10].
  
  В этих условиях обоснована возможность решения задачи автоматической 
классификации путем итерационной процедуры последовательного 
непараметрического оценивания байесовского уравнения разделяющей 
поверхности между классами~[11]. 

Предложенный подход развит при решении 
задачи автоматической классификации в условиях больших объемов 
статистических данных~[12]. Его идея состоит в сжатии исходной информации 
путем декомпозиции пространства признаков в массив данных, составленный 
из центров многомерных интервалов дискретизации и соответствующих им 
частот принадлежности значений случайных величин. Основу этого подхода 
составляет анализ многомерной гистограммы признаков классифицируемых 
объектов.
  
  Цель статьи состоит в совершенствовании непараметрического алгоритма 
автоматической классификации статистической информации большого объема 
на примере данных дистанционного зондирования путем использования 
регрессионной оценки плотности вероятности, применение которой позволяет 
сгладить многомерную гистограмму признаков классифицируемых объектов.

\section{Оценивание плотности вероятности в~условиях 
статистических данных большого объема}

Имеется выборка
\begin{equation}
V=\left( x_{vi}, \ v=\overline{1,k}\,,\ i=\overline{1,n}\right),
\label{e1-l}
\end{equation}
составленная из статистически независимых наблюдений многомерной 
случайной величины $x\hm= (x_v, \ v\hm= \overline{1,k})$ размерности~$k$ 
большого объема~$n$. Подобная информация, например, отражает данные 
дистанционного зондирования $n$ элементов земной поверхности в пространстве 
спектральных признаков размерности~$k$. Вычислительная эффективность 
непараметрических алгоритмов принятия решений ядерного типа во многом 
зависит от объема~$n$ статистических данных. При оценивании плотности 
вероятности $p(x)$ эта проблема решается путем сжатия исходной 
информации, заданной выборкой~$V$ вида~(1), что предполагает выполнение 
следующих действий~\cite{10-l}.
\begin{enumerate}[1.]
\item  Разбить область определения плотности вероятности $p(x)$ случайной 
величины~$x$ по каждой ее компоненте~$x_v$ на $N$ непересекающихся 
интервалов длиной~$2\beta_v$, $v\hm= \overline{1,k}$. Для выбора числа~$N$ 
интервалов дискретизации можно использовать результаты 
исследований~\cite{10-l, 13-l, 14-l}.
  \item Определить оценки вероятностей попадания элементов выборки~$V$ 
в~каждый $j$-й многомерный интервал
  $$
  \overline{P}_j= \fr{n(j)}{n}\,,\enskip j=\overline{1, N^k}\,,
  $$
где $n(j)$~--- число наблюдений из выборки~$V$, принадлежащих 
многомерному интервалу $D(j)\hm= [z_v\hm\pm \beta_v,\ v\hm=\overline{1,k}]$, 
а~$z_j\hm= (z_{vj}, \ v\hm= \overline{1,k})$~--- координаты центра $j$-го 
интервала.
  \item Положить, что в каждом многомерном интервале $D(j)$, $j\hm= 
\overline{1, N^k}$, случайная величина $x\hm= \left( x_v.\ 
v\hm=\overline{1,k}\right)$ имеет равномерный закон распределения. С~этих 
позиций вычислить оценки плотности вероятности
  $$
  \overline{p}(x_j) =\overline{p}_j= \fr{\overline{P}_j}{\prod\nolimits^k_{v=1} 
(2\beta_v)} \ \forall\ x\in D(j),\ j=\overline{1,N^k}\,.
  $$
  \item На основе полученной информации сформировать массив данных
  \begin{equation}
  \overline{V} =\left( z_j,\ \overline{p}_j,\ j=\overline{1,N^k}\right)
  \label{e2-l}
  \end{equation}
и осуществить синтез регрессионной оценки плотности вероятности 
$p(x)$~\cite{10-l}:
\begin{equation}
\overline{p}(x)=\fr{1}{\prod\nolimits^k_{v=1} c_v} \sum\limits_{j=1}^{N^k} 
\overline{P}_j \prod\limits^k_{v=1} \Phi \left( \fr{x_v-z_{vj}}{c_v}\right).
\label{e3-l}
\end{equation}
Здесь ядерные функции $\Phi (u_v)$ удовлетворяют условиям~\cite{8-l, 10-l}:

\vspace*{-6pt}

\noindent
\begin{multline*}
\Phi(u_v) =\Phi(-u_v),\ 0\leq \Phi(u_v)<\infty\,,\\
 \int\limits_{-\infty}^\infty 
\Phi(u_v)\,du_v=1,\ \int\limits_{-\infty}^\infty  u_v^2 \Phi(u_v)\,du_v=1\,,\\
\int\limits_{-\infty}^\infty u_v^m \Phi(u_v)\,du_v< \infty,\ 0\leq m< \infty,\ 
v=\overline{1,k}\,.
\end{multline*}
  \end{enumerate}
  
  \vspace*{-6pt}
  
  Коэффициенты размытости~$c_v$ ядерных функций $\Phi (u_v)$ 
в~статистике~(\ref{e3-l}) убывают с ростом~$N$~\cite{10-l}. Значения~$c_v$ 
зависят от длины интервала изменения случайных величин~$x_v$, $v\hm= 
\overline{1,k}$. Поэтому будем полагать, что $c_v\hm= c\overline{\sigma}_v$, 
где $\overline{\sigma}_v$~--- оценка сред\-не\-квад\-ра\-тич\-но\-го отклонения 
случайной величины~$x_v$, $v\hm=\overline{1,k}$. 
Значения~$\overline{\sigma}_v$ оцениваются по данным 
массива~$\overline{V}$.
  
  Выбор оптимального значения~$\overline{c}$ параметра~$c$ определяется из 
условия минимума сред\-не\-квад\-ра\-тич\-ной ошибки аппроксимации регрессионной 
оценкой~(\ref{e3-l}) плотности вероятности $p(x)$:
  $$
  W(c)=\fr{1}{N^k}\sum\limits_{i=1}^{N^k} \left( \overline{p}_i- 
\overline{p}(z_i)\right)^2.
  $$
  %
  При вычислении $\overline{p}(z_i)$ значение~$z_i$ исключается из 
статистики~$\overline{p}(x)$.
  
  Процедура оптимизации регрессионной оценки плотности вероятности 
случайной величины~$x$ повышает вычислительную эффективность 
процедуры выбора коэффициентов размытости ядерных функций по сравнению 
с традиционной методикой~\cite{10-l}.
  
  Традиционная методика основана на выборе коэффициента размытости 
ядерной функции из условия минимума оценки среднеквадратического 
отклонения $\overline{p}(x)$ от $p(x)$~\cite{10-l}:

\pagebreak

\noindent
\begin{multline*}
  W_1(c)=\int\limits_{-\infty}^\infty\! \cdots\! \int\limits_{-\infty}^\infty 
\overline{p}^2(x_1,\ldots , x_k)\,dx_1\cdots  dx_k -{}\\
{}-2\sum\limits^{N^k}_{i=1} 
\overline{P}_i p ( x_{1i},\ldots , x_{ki}),
  \end{multline*}
где
\begin{multline*}
\overline{p} (x_{1i},\ldots , x_{ki})={}\\  
{}= \fr{1}{\prod\nolimits_{v=1}^k c_v} \sum\limits^{N^k}_{\substack{{j=1}\\ {j\not=i}}}
\overline{P}_j \prod\limits^k_{v=1} \Phi \left( \fr{x_{vi}- z_{vj}}{c_v}\right).
\end{multline*}
  
  Нетрудно заметить, что процедура вычисления критерия~$W(c)$ более 
проста по сравнению с критерием~$W_1(c)$.
  
  Из условия минимума среднеквадратичного отклонения $\overline{p}(x)$ 
от~$p(x)$ в работах~\cite{10-l, 13-l} определено оптимальное число~$N$ 
интервалов дискретизации об\-ласти значений одномерной случайной 
величины~$x$, которое соответствует целому числу вы\-ра\-жения
\begin{equation}
 \left( n\Delta \int\limits_{-\infty}^\infty p^2(x)\,dx\right)^{1/2}.
 \label{e4-l}
\end{equation}
  
  Значение $N$ определяется видом плотности вероятности и не зависит от ее 
параметров. Для равномерного закона распределения одномерной случайной 
величины с плотностью вероятности $p(x)$ выражение для $N$ совпадает с 
формулой дискретизации Хайн\-холь\-да--Га\-е\-де~\cite{15-l}. В~этих 
условиях выражение 
$$
\Delta \int\limits_a^b p^2(x)\,dx\hm=1
$$
 не зависит от 
конечных пределов интегрирования~$a$ и~$b$ ($\Delta\hm= b\hm-a$). При 
нормальном законе распределения 
$$
\Delta \int\limits_a^b p^2(x)\,dx\hm= 
1{,}693\,,
$$
 где $a\hm= {\sf M}(x)\hm-3\sigma$; $b\hm= {\sf M}(x)\hm+3\sigma$. Здесь 
${\sf M}(x)$~--- математическое ожидание случайной величины~$a$; $\sigma$~--- 
среднеквадратичное отклонение~$x$. Вероятность попадания случайной 
величины~$x$ в~эти пределы равна~0,997.
  
  В работе~\cite{10-l} обоснована возможность оценивания 
произведения~$\Delta$ на интеграл от квадрата плотности вероятности 
одномерной случайной величины в выражении~(\ref{e4-l}) по значению 
коэффициента контрэксцесса. На этой основе предложена формула оценивания 
числа интервалов дискретизации многомерной случайной величины $x\hm= 
(x_v,\ v\hm=\overline{1,k})$.
  
  Регрессионная оценка плотности ве\-ро\-ят\-ности~(\ref{e3-l}) лежит в~ основе 
синтеза непараметрического алгоритма автоматической классификации в~условиях большого объема статистических данных.
  
\section{Алгоритм автоматической классификации}

Пусть имеется выборка многомерных статистических данных~$V$~(1) 
большого объема~$n$, распределенная с~неизвестной плот\-ностью ве\-ро\-ят\-ности 
$p(x)\hm= p(x_1, \ldots , x_k)$. Необходимо выборку~$V$ разбить на группы 
компактных наблюдений~$V_j$, $j\hm=\overline{1,M}$, чис\-ло~$M$ которых 
неизвестно. Под компактной группой наблюдений (классом) случайной 
величины~$x$ будем подразумевать об\-ласть ее значений, которая 
соответствует одномодальному фрагменту многомерной плот\-ности 
ве\-ро\-ят\-ности~$p(x)$~\cite{11-l, 12-l}.

  Следуя методике синтеза регрессионной оценки плотности вероятности, 
определим статистику $\overline{p}(x)$ по формуле~(\ref{e3-l}). Для 
обнаружения наблюдений первого класса по информации массива 
данных~$\overline{V}$ вида~(\ref{e2-l}) выберем элемент $(z_j, 
\overline{p}_j)$, который определяется условием:
  \begin{equation}
  \overline{p}_j= \max\limits_{i=\overline{1,N^k}} \overline{p}_i.
  \label{e5-l}
  \end{equation}
  %
  Этот элемент $(z_j, \overline{p}_j)$ соответствует максимальному значению 
оценки плот\-ности ве\-ро\-ят\-ности~$\overline{p}(x)$. Тогда начальный этап 
формирования элементов массива данных~$\overline{V}_1$ из~$\overline{V}$, 
принадлежащих первому классу~$\Omega_1$, определяется правилом:
  \begin{multline}
  D(j) \subset \Omega_1,\ \mbox{если } \prod\limits_{v=1}^k \Phi \left( \fr{z_{vi}-
z_{vj}}{c_v}\right) 1\left( \overline{p}_i, \overline{p}_j\right) >0\,,\\
 i=\overline{1, 
N^k},\ i\not= j\,.
  \label{e6-l}
  \end{multline}
  %
  Здесь, например, ядерная и индикаторная функции имеют вид:
  \begin{align*}
  \Phi\left( \fr{z_{vi}-z_{vj}}{c_v}\right) &= \begin{cases}
  0{,}5, &\mbox{если } \vert z_{vi}-z_{vj}\vert \leq c_v\,;\\
  0\,, &\mbox{если } \vert z_{vi}-z_{vj}\vert >c_v\,;
  \end{cases}
  \\
  1\left( \overline{p}_i, \overline{p}_j\right) &= \begin{cases}
  1\,, &\mbox{усли } \overline{p}_i\leq \overline{p}_j\,;\\
  0\,, &\mbox{если } \overline{p}_i>\overline{p}_j\,,
  \end{cases}
  \end{align*}
где $c_v\geq 2\beta_v$, так как при конкретных значениях~$c_v$ ядерные 
функции определяют элементы массива~$\overline{V}$, смежные с элементом 
$(z_j, \overline{p}_j)$. Индикаторная функция $1(\overline{p}_i,\overline{p}_j)$ 
позволяет определять элементы $(z_{i}, \overline{p}_i)$ массива 
данных~$\overline{V}$, которые характеризуются убывающими значениями 
плотности вероятности при изменении значений~$x$ в окрестности~$z_j$. На 
этом этапе индикаторная функция принимает значение $1(\overline{p}_i, 
\overline{p}_j)\hm=1$, так как выполняется условие~(\ref{e5-l}).
  
  Элементы выборки~$V$, принадлежащие интервалу $D(j)$, включаются 
в~выборку наблюдений~$V_1$ первого класса. Данное правило сохраняется 
и~для последующих этапов классификации.

 \begin{figure*} %fig1
    \vspace*{1pt}
\begin{center}
   \mbox{%
\epsfxsize=163mm 
\epsfbox{lap-1.eps}
}
\end{center}
\vspace*{-9pt}
  \Caption{Отображение исходных данных, составленных из материалов дистанционного 
зондирования космическим аппаратом Sentinel-2A за 4~августа 2018~г.~(\textit{а}) и~23~мая 
2022~г.~(\textit{б})}
\vspace*{6pt}
  \end{figure*}
  
  \begin{figure*}[b]  %fig2
 \vspace*{7pt}
\begin{center}
   \mbox{%
\epsfxsize=163mm 
\epsfbox{lap-2.eps}
}
\end{center}
\vspace*{-9pt}
\Caption{Отображение результатов классификации непараметрическим алгоритмом 
автоматической классификации исходных данных за 4~августа 2018~г.~(\textit{а}) и~23~мая 
2022~г.~(\textit{б})}
\end{figure*}
  
  Обозначим через~$I_1$ множество номеров элементов массива 
данных~$\overline{V}$, отнесенных на первом этапе автоматической 
классификации в соответствии с~правилом~(\ref{e6-l}) к первому 
классу~$\Omega_1$. Множество~$I_1$ включает номер~$j$~элемента $(z_j, 
\overline{p}_j)$.

 
  
  Второй этап классификации реализует ре\-ша\-ющее правило
  \begin{multline}
  D(i)\subset \Omega_1, \\
   \mbox{если\ }
 \! \sum\limits_{j\in I_1} \prod\limits^k_{v=1} \Phi\left( \fr{z_{vi}-
z_{vj}}{c_v}\right) 1 \left(\overline{p}_i, \overline{p}_j\right)>0\,,\\
 i\in I\backslash 
I_1,\ I=\overline{1,N^k}\,,\ j\in I_1.
  \label{e7-l}
  \end{multline}
  
  По аналогии на $t$-м этапе автоматической классификации элементы 
массива данных~$\overline{V}$, принадлежащие первому классу~$\Omega_1$, 
определяются правилом: 
  \begin{multline}
  D(i)\subset \Omega_1,\\ 
  \mbox{если\ } 
\sum\limits_{j\in I_{t-1}} \prod\limits^k_{v=1}  \Phi\left( \fr{z_{vi}-
z_{vj}}{c_v}\right) 1 \left(\overline{p}_i, \overline{p}_j\right)>0\,,\\
 i\in I\backslash 
\left(I_1\cup I_2\cup\cdots \cup I_{t-1}\right), \ j\in I_{t-1}.
\label{e8-l}
\end{multline}
  
  Последовательная процедура автоматической  
классификации~(\ref{e6-l})--(\ref{e8-l}) продолжается до выполнения условия 
$I_t\hm= \varnothing$, когда прекратится обнаружение новых элементов 
массива данных~$\overline{V}$ из класса~$\Omega_1$.
  
  Обнаруженный первый класс характеризуется регрессионной оценкой 
плотности вероятности
  $$
  \overline{p}_1(x) =\fr{1}{\prod\nolimits^k_{v=1} c_v} \sum\limits_{i\in I_{t-
1}} \overline{P}_i \prod\limits^k_{v=1} \Phi \left( \fr{x_v-z_{vi}}{c_v}\right),
  $$
которая восстанавливается по данным $\overline{V}_1$. В~этом случае 
оптимальные коэффициенты размытости~$c_v$ соответствуют значениям 
$\overline{c}_v\hm= \overline{c} \,\overline{\sigma}_v$, $v\hm=\overline{1,k}$. 
Массив данных $\overline{V}_1\hm\in \overline{V}$ определяется процедурами 
автоматической классификации, заданными  
соотношениями~(\ref{e6-l})--(\ref{e8-l}).
  
  При обнаружении элементов выборки~$V_1$ исходной статистической 
информации~$V$, принадлежащих первому классу~$\Omega_1$, используются 
результаты автоматической классификации массива данных~$\overline{V}$.
  
  Для обнаружения элементов массива $\overline{V}_2\hm\in \overline{V}$, 
принадлежащих второму классу~$\Omega_2$, из оставшихся данных 
$\overline{V}\backslash\overline{V}_1$ выбирается элемент $(z_j, 
\overline{p}_j)$ с максимальным значением~$\overline{p}_j$ и предложенная 
выше процедура автоматической классификации используется для определения 
класса~$\Omega_2$. По аналогии формируются оставшиеся классы, число 
которых априори не определено.

\vspace*{-6pt}
  
\section{Оценивание состояния лесных массивов, поврежденных 
полиграфом, по~данным дистанционного зондирования}

\vspace*{-3pt}

  Территория исследования определялась юго-за\-пад\-ной частью Дзержинского 
района Красноярского края. На этой территории преобладают пихтовые, 
кедровые и~еловые древостои, встречаются березы и осины. Высота территории 
варьирует в~диапазоне от~370 до~610~м над уровнем моря.


  
  Исходная информация формировалась по данным дистанционного 
зондирования аппаратом Европейского космического агентства Sentinel-2A за 
4~августа 2018~г.\ и 23~мая 2022~г. Сним\-ки получены с геопортала Earth 
Explorer, из которых вырезаны тестовые участки в~6000~га (рис.~1). 
Каждый из них определяется 605\,414~пикселями. Каждый пиксель 
характеризуется шестью спектральными признаками $x\hm= (x_1, \ldots , x_6)$, 
которым соответствуют длины волн (нм): 492,7~($x_1$); 559,8~($x_2$); 
664,6~($x_3$); 740,5~($x_4$); 832,8~($x_5$); 2202,4~($x_6$). На рис.~1,\,\textit{а} 
и~1,\,\textit{б} представлены RGB-изоб\-ра\-же\-ния исходных данных за 2018 
и~2022~гг. соответственно. Каналам~R, G и~B соответствуют признаки $x_6$, 
$x_5$ и~$x_3$.
  


  
  В центральной части исследуемой территории располагались лесные 
массивы усохших темнохвойных древостоев, поврежденных полиграфом 
уссурийским. Их отличительная особенность~--- куртинный характер 
повреждений. Они отображаются на рисунке ярким фиолетовым цветом 
и~имеют гладкие границы, приближенные к овалу или кругу. Ярким розовым 
цветом отображаются сплош\-ные руб\-ки лесных насаждений, которые 
отличаются правильной формой, близкой к прямоугольнику. Части 
изображения в темных оттенках фиолетового и зеленого соответствуют 
хвойным лесным насаждениям. Фрагменты рисунка зеленого цвета, от темных 
к светлым тонам, соответствуют лиственным древостоям  
и~тра\-вя\-но-кус\-тар\-ни\-ко\-вым сообществам. На изображении отчетливо 
различаются дороги. Они представляют собой линии различной толщины 
с~ярким зеленым или желтым оттенком.
  
  Для обнаружения компактных групп наблюдений (классов) в пространстве 
спектральных признаков $x\hm= (x_1,\ldots , x_6)$ использовался метод анализа 
данных ISODATA (Iterative Self-Organizing Data\linebreak Analysis) и~пред\-ла\-га\-емый не\-па\-ра\-мет\-ри\-че\-ский алгоритм 
автоматической классификации. Его программная реализация NAC~v.2.0 
поз\-во\-ля\-ет выполнять сле\-ду\-ющие функции: за\-гру\-жать \mbox{изоб\-ра\-же\-ния} в~формате 
GeoTIFF; проводить классификацию с~воз\-мож\-ностью выбора метода расчета 
чис\-ла интервалов дискретизации исходного пространства признаков; получать 
изобра\-же\-ние для дальнейшей гео\-об\-ра\-бо\-тки~\cite{16-l}. В~этой программе 
реализованы методы многопоточных вы\-чис\-ле\-ний, которые повышают ско\-рость 
обработки данных большого объема.

\begin{figure*} %fig3
 \vspace*{1pt}
\begin{center}
   \mbox{%
\epsfxsize=163mm 
\epsfbox{lap-3.eps}
}
\end{center}
\vspace*{-9pt}
\Caption{Отображение результатов автоматической классификации алгоритмом ISODATA 
спектральных данных, полученных 4~августа 2018~г.~(\textit{а}) и~23~мая 
2022~г.~(\textit{б})}
%\end{figure*}
%\begin{figure*}[b] %fig4
 \vspace*{7pt}
\begin{center}
   \mbox{%
\epsfxsize=163mm 
\epsfbox{lap-4.eps}
}
\end{center}
\vspace*{-9pt}
\Caption{Отображение классов, наиболее точно характеризующих пиксели, 
соответствующие лесным насаждениям различной степени по\-вреж\-де\-ния полиграфом 
уссурийским. Результаты классификации по данным 4~августа 2018~г.\ (левый столбец) и~23~мая 2022~г.\ (правый столбец)
пред\-ла\-га\-емым не\-па\-ра\-мет\-ри\-че\-ским алгоритмом~(\textit{а}) и~алгоритмом 
ISODATA~(\textit{б})}
\end{figure*}
  
  Применение программы NAC~v.2.0. позволило в исходных изображениях за 
2018 и 2022~гг.\ обнаружить соответственно 84 и 177~компактных групп точек 
(рис.~2). При выборе метода расчета чис\-ла интервалов 
дискретизации был установлен флажок <<агрегация>>, что позволило избежать 
в~итоговом\linebreak классифицированном изображении большого чис\-ла классов. 
С~увеличением порядкового номера компактной группы наблюдений чис\-ло 
пикселей, ей при\-над\-ле\-жа\-щих, уменьшается. Это можно\linebreak объяснить тем, что 
алгоритм основан на по\-сле\-до\-ва\-тель\-ном опре\-де\-ле\-нии класса как 
одномодального фрагмента плот\-ности ве\-ро\-ятн\-ости спект\-раль\-ных 
признаков исследуемого объекта. \mbox{Обнаруженные} классы соответствуют лес\-ным 
массивам с~различной сте\-пенью поражения полиграфом уссурийским, 
усохшим и~лиственным древостоям, тра\-вя\-но-кус\-тар\-ни\-ко\-вым 
сообществам и вырубкам различной дав\-ности.



  Применение алгоритма ISODATA, реализованного в программном продукте 
ArcGIS ArcMap, требует указания необходимого числа классов. При 
классификации исходных изображений за~2018 и~2022~гг.\ алгоритмом 
ISODATA число классов принималось равным~84 и~177 соответственно. Это 
число классов было обнаружено при использовании программы NAC~v.2.0. 
В~результате обработки данных 2018 и 2022~гг.\ алгоритмом ISODATA были
выявлены только~73 и~128~классов соответственно (рис.~3). В~обоих результатах число пикселей в~классах распределено 
равномерно и не зависит от порядкового номера класса.
  



  По результатам классификации экспертами определены классы 
(рис.~4), которые характеризуют пиксели, 
соответствующие лесным на\-саж\-де\-ни\-ям с~раз\-ной сте\-пенью по\-вреж\-де\-ния 
полиграфом уссурийским. 

Полученные классы позволили рассчитать площадь 
по\-вреж\-де\-ний полиграфом уссурийским лесных на\-саж\-де\-ний ис\-сле\-ду\-емой 
территории. Программа NAC~v.2.0 позволила определить площадь 
по\-вреж\-де\-ний в~2018 и~2022~гг.\ в~размере 503,5 и~635,0~га соответственно. При 
использовании алгоритма ISODATA по\-вреж\-де\-нные площади со\-ста\-ви\-ли~323,6 
и~795,4~га. Пространственное распределение обнаруженных классов 
рас\-смат\-ри\-ва\-емы\-ми методами автоматической классификации различается на 
участ\-ках изображения, ха\-рак\-те\-ри\-зу\-ющих вырубки, переход от поврежденных 
лес\-ных на\-саж\-де\-ний к~здоровым и~открытые поч\-вы. Например, на изображении 
2018~г.\ в~центральной час\-ти алгоритм ISODATA объединил в~один класс 
пиксели, ха\-рак\-те\-ри\-зу\-ющие все темнохвойные на\-саж\-де\-ния и~по\-вреж\-ден\-ные 
лесные массивы. В~юго-вос\-точ\-ной части изображения этот алгоритм отнес 
вы\-руб\-ку к~по\-вреж\-ден\-ным древостоям. При этом пиксели рас\-смат\-ри\-ва\-емо\-го 
клас\-са характеризуют по\-вреж\-ден\-ные насаждения полиграфом уссурийским 
в~северной час\-ти изоб\-ра\-же\-ния.
  

  
  Экспертный анализ показал, что пред\-ла\-га\-емый метод автоматической 
классификации обладает преимуществом по сравнению с~алгоритмом 
\mbox{ISODATA}, что следует из анализа рис.~2--4. Пред\-ла\-га\-емый алгоритм 
автоматической классификации достаточно пол\-но выделяет зоны 
по\-вреж\-ден\-ных древостоев, разделяет темнохвойные и~лиственные породы, 
определяет участ\-ки с~открытой почвой, территории вырубок различной 
дав\-ности.


 Алгоритм ISODATA в этих условиях показал схожий результат 
классификации. Отличия наблюдаются на участках вырубок и поврежденных 
лесных насаждений, что вызывает разницу в их итоговой расчетной площади. 

Сравниваемые алгоритмы определили одинаковые участки изображения, 
со\-от\-вет\-ст\-ву\-ющие по\-вреж\-ден\-ным лесным на\-саж\-де\-ни\-ям. По результатам 
автоматической классификации алгоритмом \mbox{ISODATA} и~не\-па\-ра\-мет\-ри\-че\-ским 
классификатором рас\-счи\-та\-ны площади территорий по\-вреж\-ден\-ных древостоев 
полиграфом уссурийским (см.\ рис.~4). По данным 2022~г.\ они со\-ста\-ви\-ли~795,4 
и~635,0~га лесных массивов соответственно, что указывает на их 
различие в~25\%.

\section{Заключение}
  
  Разработанный непараметрический алгоритм автоматической классификации 
статистических данных большого объема основан на их сжатии путем 
декомпозиции многомерного пространства признаков исследуемых объектов 
и~алгоритмизации традиционной процедуры классификации. В~пред\-ла\-га\-емом 
алгоритме автоматической классификации использование многомерной 
гистограммы заменено на анализ регрессионной оценки плот\-ности ве\-ро\-ят\-ности 
случайных величин. Его применение позволяет обнаруживать классы, 
соответствующие одномодальным фрагментам плот\-ности ве\-ро\-ят\-ности. 
Использование многопоточной технологии обработки данных поз\-во\-ля\-ет 
в~2~раза сократить время автоматической классификации, что под\-тверж\-да\-ет\-ся 
результатами обработки спект\-раль\-ных данных дистанционного зондирования 
лесных массивов. Установлены условия преимущества непараметрического 
алгоритма по сравнению с методом ISODATA.

{\small\frenchspacing
 {\baselineskip=10.6pt
 %\addcontentsline{toc}{section}{References}
 \begin{thebibliography}{99}
\bibitem{1-l}
\Au{Abbas A.\,W., Minallh~N., Ahmad~N., Abid~S.\,A.\,R., Khan~M.\,A.\,A.} K-means and 
ISODATA clustering algorithms for landcover classification using remote sensing~// Sindh 
University Research~J. (Science Series), 2016. Vol.~48. No.\,2. P.~315--318.
\bibitem{2-l}
\Au{Manthena N.\,R., Kumaran~N., Chandra~S.\,V.} Remote sensing image classification using 
CNN--LSTM model~// Revue d'Intelligence Artificielle, 2022. Vol.~36. No.\,1. P.~147--153. doi: 
10.18280/ria.360117.
\bibitem{3-l}
\Au{Дорофеюк А.\,А.} Алгоритмы автоматической классификации (обзор)~// Автоматика 
и~телемеханика, 1971. №\,12. С.~78--113.
\bibitem{4-l}
\Au{Дорофеюк А.\,А.} Методология экс\-перт\-но-клас\-си\-фи\-ка\-ци\-он\-но\-го анализа в~задачах 
управления и~обработки сложноорганизованных данных (история и перспективы 
развития)~// Проб\-ле\-мы управ\-ле\-ния, 2009. №\,3.1. С.~19--28. EDN: \mbox{KJUOIN}.
\bibitem{5-l}
\Au{Цыпкин Я.\,З.} Основы теории обучающихся систем.~--- М.: Наука, 1970. 252~c.
\bibitem{6-l}
\Au{Васильев В.\,И., Эш~С.\,Н.} Особенности алгоритмов самообучения и кластеризации~//  
Управ\-ля\-ющие сис\-те\-мы и машины, 2011. №\,3. С.~3--9.
\bibitem{7-l}
\Au{Parzen E.} On estimation of a probability density function and mode~// Ann. Math. Stat., 1962. 
Vol.~33. No.\,3. P.~1065--1076. doi: 10.1214/aoms/1177704472.
\bibitem{8-l}
\Au{Епанечников В.\,А.} Непараметрическая оценка многомерной плотности вероятности~// 
Теория вероятностей и ее применения, 1969. Т.~14. №\,1. С.~156--161.
\bibitem{9-l}
\Au{Тарарушкин Е.\,В.} Восстановление плотности распределения частиц дисперсных 
материалов методом окна Пар\-зе\-на--Ро\-зен\-блат\-та~// Вестник МГСУ, 2018. Т.~13. 
Вып.~7(118). С.~855--862. doi: 10.22227/1997-0935.2018.7.855-862. EDN: UVNCVV.
\bibitem{10-l}
\Au{Лапко А.\,В., Лапко В.\,А.} Ядерные оценки плотности вероятности и их применение.~--- 
Красноярск: СибГУ им.\ М.\,Ф.~Решетнева, 2021. 308~с.
\bibitem{11-l}
\Au{Лапко А.\,В., Лапко В.\,А.} Непараметрический алгоритм автоматической классификации 
в условиях статистических данных большого объема~// Информатика и~сис\-те\-мы  
управ\-ле\-ния, 2018. Т.~57. №\,3. С.~59--70. doi: 10.22250/isu.2018.57.59-70.
\bibitem{12-l}
\Au{Зеньков~И.\,В., Лапко~А.\,В., Лапко~В.\,А., Им~С.\,Т., Тубольцев~В.\,П., Авдеенок~В.\,Л.} 
Непараметрический алгоритм автоматической классификации многомерных статистических 
данных большого объема и его применение~// Компьютерная оптика, 2021. Т.~45. №\,2. 
С.~253--260. doi: 10.18287/2412-6179-CO-801. EDN: WUOYYA.

\bibitem{13-l}
\Au{Scott D.\,W.} Multivariate density estimation: Theory, practice, and visualization.~--- Hoboken, 
NJ, USA: John Wiley \& Sons, 2015. 384~p.
\bibitem{14-l}
\Au{Fushimi T., Saito~K., Motoda~H.} Constructing outlier-free histograms with variable bin-width 
based on distance minimization~// Intell. Data Anal., 2023. Vol.~27. No.\,1. P.~5--29.
\bibitem{15-l}
\Au{Heinhold I., Gaede~K.\,W.} Ingeniur statistic.~--- M$\ddot{\mbox{u}}$nchen, Wien: 
Springer-Verlag, 1964. 352~p.
\bibitem{16-l}
\Au{Лапко А.\,В., Лапко В.\,А., Им~С.\,Т., Тубольцев~В.\,П., Авдеенок~В.\,Л.} Программа 
автоматической классификации данных дистанционного зондирования Земли на основе 
непараметрических алгоритмов принятия решений (NAC v.~2.0). Свидетельство 
о~государственной регистрации программы для ЭВМ №\,2022619023 от 18.05.2022.
\end{thebibliography}

 }
 }

\end{multicols}

\vspace*{-10pt}

\hfill{\small\textit{Поступила в~редакцию 16.01.23}}

\vspace*{6pt}

%\pagebreak

%\newpage

%\vspace*{-28pt}

\hrule

\vspace*{2pt}

\hrule



\def\tit{NONPARAMETRIC ALGORITHM FOR~AUTOMATIC CLASSIFICATION OF~REMOTE 
SENSING DATA\\[-5pt]}


\def\titkol{Nonparametric algorithm for~automatic classification of~remote 
sensing data}


\def\aut{V.\,P.~Tuboltsev$^1$, A.\,V.~Lapko$^{1,2}$, and~V.\,A.~Lapko$^{1,2}$}

\def\autkol{V.\,P.~Tuboltsev, A.\,V.~Lapko, and~V.\,A.~Lapko}

\titel{\tit}{\aut}{\autkol}{\titkol}

\vspace*{-14pt}


\noindent
$^1$M.\,F.~Reshetnev Siberian State University of Science and Technology, 31~Krasnoyarsky Rabochy 
Av., Krasno-\linebreak
$\hphantom{^1}$yarsk 660037, Russian Federation

\noindent
$^2$Institute of Computational Modelling of the Siberian Branch of the Russian Academy of 
Sciences, 50/44~Akadem-\linebreak
$\hphantom{^1}$gorodok, Krasnoyarsk 660036, Russian Federation


\def\leftfootline{\small{\textbf{\thepage}
\hfill INFORMATIKA I EE PRIMENENIYA~--- INFORMATICS AND
APPLICATIONS\ \ \ 2023\ \ \ volume~17\ \ \ issue\ 4}
}%
 \def\rightfootline{\small{INFORMATIKA I EE PRIMENENIYA~---
INFORMATICS AND APPLICATIONS\ \ \ 2023\ \ \ volume~17\ \ \ issue\ 4
\hfill \textbf{\thepage}}}

\vspace*{2pt}




\Abste{A nonparametric algorithm for automatic classification of large-volume statistical data 
is proposed. The algorithm under consideration assumes compression of initial information 
based on decomposition of  multidimensional feature space. As a~result, a~large statistical 
sample is transformed into a~data array composed of the centers of multidimensional sampling 
intervals and their corresponding frequencies of random variables. The information obtained is used 
in the synthesis of the regression estimate of the probability density. A~class is understood as 
a~compact group of observations of a~random variable corresponding to a unimodal fragment of 
the probability density function. On this basis, a~nonparametric automatic classification algorithm is 
developed which is based on the sequential procedure for checking the proximity of the centers of 
multidimensional sampling intervals and the ratios between the frequencies of belonging of random 
variables from the original sample to these intervals. To improve the computational efficiency of 
the proposed automatic classification algorithm, a~multithreaded method of its software 
implementation is used. The practical significance of the developed algorithm for automatic 
classification is confirmed by the results of its application for assessing the state of the forests areas using 
remote sensing data.}


\KWE{automatic classification; large-volume samples; sampling of the range of values of random 
variables; regression estimation of probability density; remote sensing data}

\DOI{10.14357/19922264230404}{MPEWAW}

%\vspace*{-12pt}

%\Ack
%\noindent


\vspace*{-8pt}

  \begin{multicols}{2}

\renewcommand{\bibname}{\protect\rmfamily References}
%\renewcommand{\bibname}{\large\protect\rm References}

{\small\frenchspacing
 {%\baselineskip=10.8pt
 \addcontentsline{toc}{section}{References}
 \begin{thebibliography}{99} 
 
 \vspace*{-2pt}
 
 
\bibitem{1-l-1}
\Aue{Abbas, A.\,W., N.~Minallh, N.~Ahmad, S.\,A.\,R.~Abid, and M.\,A.\,A.~Khan.} 2016.  
K-means and ISODATA clustering algorithms for landcover classification using remote sensing. 
\textit{Sindh University Research~J. (Science Series)} 48(2):315--318.
\bibitem{2-l-1}
\Aue{Manthena, N.\,R., N.~Kumaran, and S.\,V.~Chandra.} 2022. Remote sensing image 
classification using CNN--LSTM model. \textit{Revue d'Intelligence Artificielle} 36(1):147--153. 
doi: 10.18280/ria.360117.
\bibitem{3-l-1}
\Aue{Dorofeyuk, А.\,А.} 1971. Algoritmy avtomaticheskoy klassifikatsii (obzor)
[Algorithms of automatic classification (review)]. \textit{Automat. 
Rem. Contr.} 12:78--113.
\bibitem{4-l-1}
\Aue{Dorofeyuk, А.\,А.} 2009. Metodologiya ekspertno-klassifikatsionnogo analiza v~zadachakh 
upravleniya i~obrabotki slozhnoorganizovannykh dannykh (istoriya i~perspektivy razvitiya) 
[Expert-ranging analysis methodology in complex organized data processing and control problems 
(history of development and perspectives)]. \textit{Problemy upravleniya} [Control Sciences] 
3S1:19--28. EDN: KJUOIN.
\bibitem{5-l-1}
\Aue{Tsypkin, Ya.\,Z.} 1970. \textit{Osnovy teorii obuchayushchikhsya sistem} [Foundations of the 
theory of learning systems]. Moscow: Nauka. 252~p.
\bibitem{6-l-1}
\Aue{Vasil'ev, V.\,I., and S.\,N.~Esh.} 2011. Osobennosti algoritmov samoobucheniya 
i~klasterizatsii [Features of self-learning algorithms and clustering]. \textit{Upravlyayushchie 
sistemy i~mashiny} [Control Systems and Computers] 3:3--9.
\bibitem{7-l-1}
\Aue{Parzen, E.} 1962. On estimation of a probability density function and mode. \textit{Ann. 
Math. Stat.} 33(3):1065--1076. doi: 10.1214/aoms/1177704472.
\bibitem{8-l-1}
\Aue{Epanechnikov, V.\,A.} 1969. Non-parametric estimation of a multivariate probability density. 
\textit{Theor. Probab. Appl.} 14(1):153--158. doi: 10.1137/1114019.
\bibitem{9-l-1}
\Aue{Tararushkin, E.\,V.} 2018. Vosstanovlenie plotnosti raspredeleniya chastits dispersnykh 
materialov metodom okna Parzena--Rozenblatta [Reconstructing distribution density of particles for 
disperse materials by the Parzen--Rozenblatt window method]. \textit{Vestnik MGSU} 
 13(7):855--862. doi: 10.22227/1997-0935.2018.7.855-862. EDN: UVNCVV.
\bibitem{10-l-1}
\Aue{Lapko, A.\,V., and V.\,A.~Lapko.} 2021. \textit{Yadernye otsenki plotnosti veroyatnosti i~ikh 
primenenie} [Kernel probability density estimates and their application]. Krasnoyarsk: Reshetnev 
University Publs. 308~p.
 \bibitem{11-l-1}
\Aue{Lapko, A.\,V., and V.\,A.~Lapko.} 2018. Ne\-pa\-ra\-met\-ri\-che\-skiy algoritm avtomaticheskoy 
klassifikatsii v~usloviyakh sta\-ti\-sti\-che\-skikh dannykh bol'shogo ob"ema [Nonparametric algorithm 
of automatic classification under conditions of large-scale statistical data]. \textit{Informatika 
i~sistemy upravleniya} [Information Science and Control Systems] 57(3):59--70. doi: 
10.22250/isu.2018.57.59-70. EDN: YACMRN.
\bibitem{12-l-1}
\Aue{Zenkov, I.\,V., A.\,V.~Lapko, V.\,A.~Lapko, S.\,T.~Im, V.\,P.~Tuboltsev, and 
V.\,L.~Avdeenok.} 2021. A nonparametric algorithm for automatic classification of large 
multivariate statistical data sets and its application. \textit{Computer Optics} 45(2):253--260. doi:  
10.18287/2412-6179-CO-801. EDN: WUOYYA.
\bibitem{13-l-1}
\Aue{Scott, D.\,W.} 2015. \textit{Multivariate density estimation: Theory, practice, and 
visualization}. Hoboken, NJ: John Wiley \&~Sons. 384~p.
\bibitem{14-l-1}
\Aue{Fushimi, T., K.~Saito, and H.~Motoda.} 2023. Constructing outlier-free histograms with 
variable bin-width based on distance minimization. \textit{Intell. Data Anal.} 27(1):5--29.
\bibitem{15-l-1}
\Aue{Heinhold, I., and K.\,W.~Gaede.} 1964. \textit{Ingeniur statistic}. 
M$\ddot{\mbox{u}}$nchen, Wien: Springler Verlag. 352~p.
\bibitem{61-l-1}
\Aue{Lapko, A.\,V., V.\,A.~Lapko, S.\,T.~Im, V.\,P.~Tuboltsev, and V.\,L.~Avdeenok.} 2022. 
Programma avtomaticheskoy klassifikatsii dannykh distantsionnogo zondirovaniya Zemli na 
osnove neparametricheskikh algoritmov prinyatiya resheniy (NAC v.~2.0) [The program for 
automatic classification of Earth remote sensing data based on nonparametric decision-making 
algorithms (NAC v.~2.0)]. Certificate of State Registration of the Computer Program RF 
No.\,2022619023.

\end{thebibliography}

 }
 }

\end{multicols}

\vspace*{-6pt}

\hfill{\small\textit{Received January 16, 2023}} 

%\vspace*{-18pt}

\Contr

\vspace*{-4pt}

     \noindent
     \textbf{Tuboltsev Vitaly P.} (b.\ 1998)~--- PhD student, M.\,F.~Reshetnev Siberian State University 
of Science and Technology, 31~Krasnoyarsky Rabochy Av., Krasnoyarsk 660037, Russian 
Federation; \mbox{vitalya.98@mail.ru}
     
     \vspace*{3pt}
     
     \noindent
     \textbf{Lapko Alexander V.} (b.\ 1949)~--- Doctor of Science in technology, professor, 
Honored Scientist of the RF, principal scientist, Institute of Computational Modelling 
of the Siberian Branch of the Russian Academy of Sciences, 50/44~Akademgorodok, Krasnoyarsk 
660036, Russian Federation; professor, Department of Space Facilities and Technologies, 
M.\,F.~Reshetnev Siberian State University of Science and Technology, 31~Krasnoyarsky Rabochy Av., 
Krasnoyarsk 660037, Russian Federation; \mbox{lapko@icm.krasn.ru}
     
     \noindent
     \textbf{Lapko Vasiliy A.} (b.\ 1974)~--- Doctor of Science in technology, professor, leading 
scientist, Institute of Computational Modelling of the Siberian Branch of the Russian Academy of 
Sciences, 50/44~Akademgorodok, Krasnoyarsk 660036, Russian Federation; head of 
department, M.\,F.~Reshetnev Siberian State University of Science 
and Technology, 31~Krasnoyarsky Rabochy Av., Krasnoyarsk 660037, Russian Federation; 
valapko@yandex.ru

\label{end\stat}

\renewcommand{\bibname}{\protect\rm Литература}   %4
\def\stat{bosov+stef}

\def\tit{УПРАВЛЕНИЕ ВЫХОДОМ СТОХАСТИЧЕСКОЙ ДИФФЕРЕНЦИАЛЬНОЙ СИСТЕМЫ 
ПО~КВАДРАТИЧНОМУ КРИТЕРИЮ. I.~ОПТИМАЛЬНОЕ РЕШЕНИЕ МЕТОДОМ 
ДИНАМИЧЕСКОГО ПРОГРАММИРОВАНИЯ$^*$}

\def\titkol{Управление выходом стохастической дифференциальной системы 
по~квадратичному критерию. I}
%.~Оптимальное решение методом 
%динамического программирования}

\def\aut{А.\,В.~Босов$^1$, А.\,И.~Стефанович$^2$}

\def\autkol{А.\,В.~Босов, А.\,И.~Стефанович}

\titel{\tit}{\aut}{\autkol}{\titkol}

\index{Босов А.\,В.}
\index{Стефанович А.\,И.}
\index{Bosov A.\,V.}
\index{Stefanovich A.\,I.}




{\renewcommand{\thefootnote}{\fnsymbol{footnote}} \footnotetext[1]
{Работа выполнена при частичной поддержке РФФИ (проект 16-07-00677).}}


\renewcommand{\thefootnote}{\arabic{footnote}}
\footnotetext[1]{Институт проблем информатики Федерального исследовательского центра <<Информатика 
и~управление>> Российской академии наук, \mbox{AVBosov@ipiran.ru}}
\footnotetext[2]{Институт проблем информатики Федерального исследовательского центра <<Информатика 
и~управление>> Российской академии наук, \mbox{AStefanovich@frccsc.ru}}

%\vspace*{8pt}



  
  \Abst{Решается задача оптимального управления для диффузионного процесса 
Ито и~линейного управ\-ля\-емо\-го выхода. Рассматриваемая постановка близка 
к~классической ли\-ней\-но-квад\-ра\-тич\-ной гауссовской задаче управления 
(linear-quadratic Gaussian (LQG) control). Отличия состоят в~том, что состояние описывается нелинейным 
дифференциальным уравнение Ито $dy_t\hm= A_t(y_t) \,dt\hm+ \Sigma_t(y_t)\,dv_t$ 
и~не зависит от управ\-ле\-ния~$u_t$, оптимизации подлежит управ\-ля\-емый 
линейный выход $dz_t\hm= a_t y_t\,dt\hm+ b_t z_t \,dt\hm+ c_t u_t \,dt\hm+ \sigma_t\, 
dw_t$. Дополнительные обобщения внесены в~квад\-ра\-тич\-ный критерий качества 
с~целью воз\-мож\-ности постановки таких задач, как отслеживание выходом 
состояния или управ\-ле\-ни\-ем~--- линейной комбинации состояния и~выхода. Для 
решения используется метод динамического программирования. Функцию 
Беллмана позволяет найти предположение о~ее структуре вида $V_t(y,z)\hm= 
\alpha_t z^2\hm+ \beta_t(y)z \hm+\gamma_t(y)$. Решение дают три 
дифференциальных уравнения для коэффициентов~$\alpha_t$, $\beta_t(y)$ 
и~$\gamma_t(y)$. Эти уравнения со\-став\-ля\-ют оптимальное решение 
рас\-смат\-ри\-ва\-емой задачи.}
  
  \KW{стохастическое дифференциальное уравнение; оптимальное управ\-ле\-ние; 
динамическое программирование; функция Беллмана; уравнение Риккати; 
линейные уравнения параболического типа}

\DOI{10.14357/19922264180314}
  
%\vspace*{4pt}


\vskip 10pt plus 9pt minus 6pt

\thispagestyle{headings}

\begin{multicols}{2}

\label{st\stat}

\section{Введение}

     Ключевые результаты в~области оптимизации стохастических 
динамических систем, со\-став\-ля\-ющие классическую теорию управления, 
получены более~40~лет назад (такова работа~[1] в~отношении задачи 
управ\-ле\-ния ли\-ней\-но-гаус\-сов\-ски\-ми стохастическими сис\-те\-ма\-ми по 
квад\-ра\-тич\-но\-му критерию). К~классической тео\-рии следует относить 
линейные модели стохастических сис\-тем и~квадратичный критерий качества. 
Это исходный базис, на котором основано множество успешно 
исследованных и~решенных задач стохастического управ\-ле\-ния 
и~оптимизации. 

Дальнейшее развитие~--- это новые модели и~критерии, но 
прежде всего это новые методы: от тео\-рии линейных регуляторов, метода 
динамического программирования и~принципа максимума к~адаптивному 
и~минимаксному подходу, импульсному управ\-ле\-нию и~т.\,д. Множество 
инноваций как в~час\-ти моделей, так и~в~час\-ти математического аппарата, 
имевших мес\-то в~по\-сле\-ду\-ющие годы, существенно обогатили тео\-рию 
управ\-ле\-ния. Но и~до настоящего времени линейные модели и~квадратичный 
критерий, несмотря на всю справедливую критику в~отношении их 
аде\-кват\-ности и~гиб\-кости, сохраняют исследовательский интерес и~находят 
современные области приложения.
     
     Не претендуя на сколь\-ко-ни\-будь полное обосно\-ва\-ние последнего 
тезиса, приведем несколько примеров, показавшихся наиболее ин\-те\-рес\-ными. 

Так, в~[2] решается ли\-ней\-но-квад\-ра\-тич\-ная за\-да\-ча в~игровой 
постановке с~запаздыванием. В~близ\-кой по модели работе~[3] задача 
управ\-ле\-ния ставится в~терминах $H_\infty$-ро\-баст\-ности. Точнее \mbox{называть} 
эту тематику $H_2/H_\infty$-управ\-ле\-ни\-ем, и~работ по этой теме очень 
много. Аккуратности ради следует уточнить, что под линейными 
понимаются модели с~мультипликативными по состоянию воз\-му\-ще\-ниями. 

Совсем другой класс моделей, особо популярных в~по\-след\-ние годы, 
составляют скачкообразные процессы. Например, линейные уравнения 
в~сочетании с~пуассоновскими скачками в~[4] используются в~моделях, 
описывающих различные показатели функционирования сетевых протоколов 
передачи данных транспортного уровня. Телекоммуникации представляют 
в~последние годы самый популярный прикладной материал для 
исследований, работ по этой проб\-ле\-ма\-ти\-ке множество, математические 
техники привлекаются самые разные и~самые современные, но и~линейным 
моделям место находится. Еще один любопытный пример исследования 
скачкообразного процесса и~оптимизации на основе квад\-ра\-тич\-но\-го критерия 
можно найти в~[5] применительно к~задаче инвестирования на финансовом 
рынке. Наконец, упомянем еще работу~[6], подводящую итог исследований 
в~отношении классической детерминированной  
ли\-ней\-но-квад\-ра\-тич\-ной задачи с~использованием техники матричных 
неравенств.
     
     В данной работе также эксплуатируются привлекательные свойства 
линейных моделей и~квад\-ра\-тич\-но\-го критерия, причем в~стохастической 
постановке. На\-прав\-ле\-ни\-ем для обобщения \mbox{выбрана} модель динамики 
сис\-те\-мы: основные усилия на\-прав\-ле\-ны на то, чтобы сделать ее нелинейной. 
Кроме того, пред\-став\-лен\-ная постановка может рас\-смат\-ри\-вать\-ся и~как 
обобщение ранее решенной задачи в~дискретном времени~[7, 8] на время 
непрерывное. В~упомянутых работах помимо собственно модельной 
постановки важна еще и~привлекаемая прикладная об\-ласть~--- 
функционирование сложных программных сис\-тем. Результатов, 
ориентированных непосредственно на такие приложения, к~настоящему 
времени пренебрежимо мало, поэтому~[7, 8]~--- это еще и~прикладное 
обоснование рас\-смат\-ри\-ва\-емой далее задачи.
     
     Оптимизируемая динамическая сис\-те\-ма описывается двумя 
уравнениями. Состояние задается нелинейным стохастическим 
дифференциальным уравнением Ито, не содержащим управ\-ля\-емой 
переменной. Возмущение здесь описывается стандартным винеровским 
процессом, накладываются простые условия существования 
и~един\-ст\-вен\-ности решения. Поскольку состояние не управ\-ля\-ет\-ся, то уместно 
его интерпретировать как слож\-ное внешнее возмущение. Вторая 
переменная~--- управ\-ля\-емый выход~--- задается линейным стохастическим 
дифференциальным уравнением. Цель оптимизации выхода формируется 
квадратичным критерием общего вида. Формальная постановка задачи 
приведена в~сле\-ду\-ющем разделе.
     
     Для решения задачи используется метод динамического 
программирования, решается уравнение Беллмана~[9]. Соответственно, 
в~результате получаются аналитические выражения и~для оптимального 
управ\-ле\-ния, и~для значения функционала качества. Технически 
традиционный, стандартный подход к~задаче обременен, пожалуй, 
единственной проблемой~--- поиском верного пред\-став\-ле\-ния структуры 
функции Беллмана. Справиться с~этой проблемой в~большей степени удается 
за счет результата, полученного при решении дискретного по времени 
аналога рассматриваемой постановки~\cite{8-bos}. Конечные соотношения 
для оптимального решения, как и~во всех подобных задачах, включая 
классическую ли\-ней\-но-квад\-ра\-тич\-ную, содержат решения 
определенных дифференциальных уравнений (обыкновенных и~в~частных 
производных). Вывод этих уравнений и~со\-став\-ля\-ет содержание первой час\-ти 
данной работы. Во второй части будет обсуждаться их приближенное 
чис\-лен\-ное решение и~компьютерные эксперименты.
     
     Кратко обозначим основные положения, при\-вле\-ка\-емые далее 
к~решению задачи, следуя в~основном обозначениям 
и~терминологии~\cite{9-bos}, а~именно: будем рассматривать задачу 
оптимального управления в~стохастической динамической сис\-те\-ме по полной 
информации, применяя метод динамического программирования. В~качестве 
целевого функционала, опре\-де\-ля\-юще\-го качество управ\-ле\-ния $U_0^T\hm= \{ 
u_t,\ 0\leq t\leq T\}$, выступает
     \begin{equation}
     J\left(U_0^T\right)={\sf E}\left\{ \int\limits_0^T L_t \left(x_t, u_t\right)\,dt+ 
l\left(x_T\right)\right\}\,.
     \label{e1-bos}
     \end{equation}
Здесь ${\sf E}\{\cdot\}$~--- оператор математического ожидания; $x_t$~--- 
случайный процесс, описываемый стохастическим дифференциальным 
уравнением Ито
     \begin{equation}
     dx_t=m_t\left( x_t, u_t\right) dt+ \sigma_t\left( x_t\right)dW_t\,,\enskip 
x_0=X\,,
     \label{e2-bos}
     \end{equation}
где $W_t$~--- стандартный винеровский процесс подходящей раз\-мер\-ности; 
$X$~--- случайный вектор.

     $U_0^T$ будем выбирать из класса допустимых неупреждающих (по 
отношению к~$W_t$) управлений~\cite{9-bos}. Соответственно, 
относительно функций сноса и~диффузии~$m_t$ и~$\sigma_t$  
в~(\ref{e2-bos}) будем предполагать выполненными ка\-кие-ли\-бо условия 
существования сильного решения для заданного до\-пус\-ти\-мо\-го управ\-ле\-ния. 
Например, для управ\-ле\-ния с~обратной связью $u_t\hm= u_t(x_t)$ будем 
считать, что $m_t(x,u_t(x))$ и~$\sigma_t(x)$ удовлетворяют условию 
линейного рос\-та и~локальному условию Липшица по~$x$ равномерно 
по~$t$ (т.\,е.\ условиям Ито).
     
     Для поиска оптимального управления, минимизирующего $J(U_0^T)$, 
рас\-смат\-ри\-ва\-ет\-ся функция Беллмана
     \begin{equation}
     V_t(x)=\left.\mathop{\mathrm{inf}}\limits_{U_t^T} {\sf E} \left\{ \int\limits_t^T 
L_t \left( x_t, u_t\right)\,dt+l\left( x_T\right) \right\vert \mathcal{F}_t^x\right\}\,,
     \label{e3-bos}
     \end{equation}
где $\mathcal{F}_t^x$~--- $\sigma$-ал\-геб\-ра, по\-рож\-ден\-ная~$x_\tau$, 
$0\hm\leq \tau\hm\leq t$, ${\sf E}\{\cdot\vert \mathcal{F}\}$~--- оператор условного 
математического ожидания относительно~$\mathcal{F}$. Соответственно, 
в~качестве достаточного условия оп\-ти\-маль\-ности воспользуемся уравнением 
динамического программирования
\begin{multline}
\fr{\partial V_t(x)}{\partial t} +\fr{1}{2}\sum\limits^n_{i,j=1} \sigma^2_{t_{ij}}
\fr{\partial^2 V_t(x)}{\partial x_i \partial x_j}+{}\\
{}+\min\limits_u\left[  
\sum\limits^n_{i=1} m_{t_i} \fr{\partial V_t(x)}{\partial x_i} + L_t(x,u)\right] 
=0\,,\\
V_T(x)=l(x)\,,
\label{e4-bos}
\end{multline}
где $m_{t_i}$~--- $i$-й элемент век\-тор-функ\-ции~$m_t(x,u)$; 
$\sigma^2_{t_{ij}} \hm= \sum\nolimits^m_{k=1} 
\sigma_{t_{ik}}\sigma_{t_{ki}}$, $\sigma_{t_{ij}}$~--- $i$-й по строке, $j$-й 
по столб\-цу элемент мат\-рич\-ной функции~$\sigma_t(x)$; $n$ и~$m$~--- 
размерности~$x_t$ и~$W_t$ соответственно.

     Традиционно в~рамках применения метода динамического 
программирования будем предполагать, что функции~$L_t$, $l$, $m_t$ 
и~$\sigma_t$ обеспечивают существование хотя бы одного решения 
уравнения~(\ref{e4-bos}), а~следовательно, и~оптимального 
управления~$u_t^*$, $0\hm\leq t\hm\leq T$, до\-став\-ля\-юще\-го минимум 
целевому функционалу~(\ref{e1-bos}). Задача оптимизации далее получается 
путем указания конкретных выражений для~$L_t$, $l$, $m_t$ и~$\sigma_t$.

\section{Постановка задачи управления выходом}

     Рассматриваемые далее случайные функции будут предполагаться 
скалярными. Такое упрощение позволит разгрузить выкладки и~итоговые 
выражения от не самых существенных деталей.
     
     Рассмотрим стохастическую дифференциальную сис\-те\-му, со\-сто\-яние 
которой представляет диффузи\-он\-ный процесс~$y_t$, описываемый 
нелинейным стохастическим дифференциальным уравнением Ито
     \begin{equation}
     dy_t=A_t\left( y_t\right) dt +\Sigma_t \left( y_t\right) dv_t\,,\enskip 
y_0=Y\,,
     \label{e5-bos}
     \end{equation}
где $v_t$~--- стандартный (одномерный) винеровский процесс; $Y$~--- 
случайная величина с~конечным вторым моментом; функции~$A_t$ 
и~$\Sigma_t$ удовлетворяют условиям Ито:
\begin{equation*}
\left\vert A_t(y)\right\vert +\left\vert \Sigma_t(y)\right\vert \leq C(1+\vert y\vert )\ 
\mbox{для\ всех } 0\leq t\leq T\,;
\end{equation*}

\vspace*{-12pt}

\noindent
\begin{multline*}
\hspace*{-2.10051pt}\left\vert A_t\left(y_1\right) -A_t \left( y_2\right) \right\vert +\left\vert 
\Sigma_t\left( y_1\right) -\Sigma_t \left(y_2\right)\right\vert \leq
C\left\vert y_1-y_2\right\vert\\
 \mbox{для\ всех\ } 0\leq t\leq T\ \mbox{и } 
y_1,y_2\in \mathbb{R}^1\,,
\end{multline*}
обеспечивающим существование единственного сильного (потраекторного) 
решения уравнения.
     
     Будем считать, что~$y_t$ описывает состояние некоторой 
динамической системы. Соответственно, поведение этой сис\-те\-мы опишем 
выходом, линейно связанным с~со\-сто\-янием:
     \begin{equation}
     dz_t=a_t y_t \,dt+ b_t z_t \,dt+ c_t u_t \,dt+\sigma_t \,dw_t\,,\enskip
     z_0=Z\,.
     \label{e6-bos}
     \end{equation}
Здесь $w_t$~--- не зависящий от~$v_t$, $Y$ и~$Z$ стандартный (одномерный) 
винеровский процесс; $Z$~--- случайная величина с~конечным вторым 
моментом; $u_t$~--- допустимое неупреждающее управ\-ле\-ние, качество 
которого определяется целевым функционалом следующего вида:
\begin{multline}
\!\hspace*{-3.98538pt}J\left( U_0^T\right) ={\sf E}\left\{ \int\limits_0^T \!\left( S_t\left( s_ty_t-g_t z_t -h_t 
u_t\right)^2 +G_t z_t^2+{}\right.\right.\\
\left.\left.{}+ H_t u_t^2
\vphantom{S_t\left( s_ty_t-g_t z_t -h_t 
u_t\right)^2}
\right) dt+S_T\left( s_T y_T -g_T 
z_T\right)^2+G_T z_T^2
\vphantom{\int\limits_0^T}\right\}\,,
\label{e7-bos}
\end{multline}
где $S_t$, $G_t$ и~$H_t$~--- неотрицательные функции\linebreak
$0\hm\leq t\hm\leq T$. 
Такой критерий отражает физический смысл задачи распределения ресурсов 
со\-глас\-но аналогичной~(\ref{e5-bos})--(\ref{e7-bos}) задаче для дис\-крет\-но\-го 
времени, рас\-смот\-рен\-ной в~\cite{7-bos}. В~част\-ности,  
функци\-онал~(\ref{e7-bos}) поз\-во\-ля\-ет ставить задачи отслеживания
 выходом 
со\-сто\-яния сис\-те\-мы, используя сла\-га\-емое $(y_t\hm- z_t)^2$, или 
управлением~--- линейной комбинации со\-сто\-яния и~выхода, сла\-га\-емое типа\linebreak 
$(y_t\hm+ z_t\hm- u_t)^2$. Поскольку задача формулируется 
в~предположении наличия пол\-ной информации о~со\-сто\-янии~$y_t$ 
и~выходе~$z_t$ (соответствующую $\sigma$-ал\-геб\-ру 
обозначим~$\mathcal{F}_t^{y,z}$), то допустимое управ\-ле\-ние ищется 
в~классе~$\mathcal{F}_t^{y,z}$-из\-ме\-ри\-мых неупреждающих функций 
(и,~как будет показано далее, оказывается управ\-ле\-ни\-ем с~обратной связью).

     Функции~$a_t$, $b_t$, $c_t$ и~$\sigma_t$ будем предполагать 
ограниченными: $\vert a_t\vert \hm+ \vert b_t\vert \hm+\vert c_t\vert \hm+ \vert 
\sigma_t \vert \hm\leq C$ для всех $0\hm\leq t\hm\leq T$, процесс  
управления~--- допустимым не\-упреж\-да\-ющим~\cite{9-bos}, обеспечивая, 
таким образом, существование сильного решения урав\-не\-ния~(\ref{e6-bos}) 
для любого допустимого управ\-ления.
     
     Задачу составляет поиск~$u_t^*$~--- допустимого управ\-ле\-ния, 
доставляющего минимум квад\-ра\-тич\-но\-му функционалу~$J(U_0^T)$.
      
     Поставленная задача очевидным образом формулируется в~терминах 
введенных выше в~(\ref{e1-bos})--(\ref{e3-bos}) обозначений, а~именно: 
     требуется обозначить
     \begin{gather*}
      x_t=\begin{pmatrix}
     y_t\\ z_t\end{pmatrix};\quad  m_t(x_t, u_t)=\begin{pmatrix}
     A_t(y_t)\\ a_t y_t +b_t z_t +c_t u_t\end{pmatrix};\\
     \sigma_t(x_t)= \begin{pmatrix}
     \Sigma_t(y_t)& 0\\
     0& \sigma_t\end{pmatrix};\quad W_t=\begin{pmatrix}
     v_t \\ w_t\end{pmatrix}
     %     \label{e8-bos}
     \end{gather*}
для записи уравнения со\-сто\-яния типа~(\ref{e2-bos}) и
\begin{align*}
L_t(x,u)&= L_t(y,z,u) ={}\\
&\hspace*{3mm}{}=S_t\left( s_t y-g_t z -h_t u\right)^2 +G_t z^2 +H_t  u^2\,;\\
l(x)&= l(y,z) =S_T \left( S_T y-g_T z\right)^2 +G_T z^2
%\label{e9-bos}
\end{align*}
для записи целевого функционала в~виде~(\ref{e1-bos}).

     Функция Беллмана~(\ref{e3-bos}) принимает вид 
     $V_t(x)\hm= V_t(y,z)$. Для записи со\-от\-вет\-ст\-ву\-юще\-го~(\ref{e4-bos}) 
уравнения Беллмана для~$V_t(y,z)$ заметим, что
     $$
     \left( \sigma^2_{t_{ij}}\right)_{i,j=1,2}= \begin{pmatrix}
     \Sigma_t^2(y) & 0\\
     0 & \sigma_t^2\end{pmatrix}\,.
     $$
     
     С~учетом перечисленных обозначений урав\-не\-ние динамического 
программирования~(\ref{e4-bos}) принимает вид:
     \begin{multline}
     \fr{\partial V_t(y,z)}{\partial t} +\fr{1}{2}\left( \Sigma_t^2(y) \fr{\partial^2 
V_t(y,z)} {\partial y^2}+\sigma_t^2\fr{\partial^2 V_t(y,z)} {\partial 
z^2}\right)+{}\\
    {}+\min\limits_u\! \left[ A_t(y) \fr{\partial V_t(y,z)}{\partial y}+\left( a_t 
y+b_t z+c_t u\right) \fr{\partial V_t(y,z)}{\partial z} +{}\right.\hspace*{-3pt}\\
\left.{}+ S_t\left( s_t y-g_t z-h_t 
u\right)^2+G_t z^2+H_t u^2
     \vphantom{\fr{\partial V_t(y,z)}{\partial y}}\right] =0\,,\\
     V_T(y,z)=S_T\left( s_T y-g_T z\right)^2+G_T z^2\,.
     \label{e10-bos}
     \end{multline}
     Это и~есть то самое уравнение, которое требуется решить: 
существование решения данного урав\-не\-ния суть достаточное условие 
оптимальности; оптимальное управ\-ле\-ние при этом~--- точ\-ка минимума 
со\-от\-вет\-ст\-ву\-юще\-го сла\-га\-емого.
     
\section{Динамическое программирование и~оптимальное 
управление}

     В рассматриваемой постановке линейность\linebreak выхода и~квадратичность 
критерия дают те же преимущества, что и~в~классической  
ли\-ней\-но-квад\-ра\-тич\-ной задаче управ\-ле\-ния~\cite{1-bos}, а~именно: 
позволяют сразу определить вид оптимального управ\-ле\-ния и~фактические 
условия его существования. Действительно, со\-хра\-няя в~(\ref{e10-bos}) под 
знаком $\min\nolimits_u$ только члены, зависящие от~$u$, получаем
     \begin{multline*}
     \fr{\partial V_t(y,z)}{\partial t} +\fr{1}{2}\left( \Sigma_t^2(y) \fr{\partial^2 
V_t(y,z)} {\partial y^2}+\sigma_t^2\fr{\partial^2 V_t(y,z)} {\partial 
z^2}\right)+{}\\
     {}+A_t(y)\fr{\partial V_t(y,z)}{\partial y}+\left( a_t y+b_t z\right) 
\fr{\partial V_t(y,z)}{\partial z}+{}\\
{}+S_t\left( s_t y-g_t z\right)^2 +G_t z^2+{}
\end{multline*}

\noindent
\begin{multline*}
     {}+\min\limits_u \left[ \left( c_t \fr{\partial V_t(y,z)}{\partial z}-2S_t \left( 
s_t y-g_t z\right) h_t\right)u +{}\right.\\
\left.{}+\left( S_t h_t^2+H_t\right) u^2
\vphantom{\fr{\partial V_t(y,z)}{\partial z}}
\right]=0\,,
     %\label{e11-bos}
     \end{multline*}
откуда в~предположении $S_t h_t^2\hm+ H_t\hm>0$ следует, что существует 
оптимальное управ\-ле\-ние, которое определяется равенством
\begin{multline}
u_t^* = u_t^*(y,z)=-\fr{1}{2}\left( S_t h_t^2 +H_t\right)^{-1} \left( c_t 
\fr{\partial V_t(y,z)}{\partial z}-{}\right.\\
\left.{}-2S_t\left( s_t y-g_t z\right) h_t
\vphantom{\fr{\partial V_t(y,z)}{\partial z}}
\right)
\label{e12-bos}
\end{multline}
и доставляет минимум соответствующему сла\-га\-емо\-му в~урав\-не\-нии Беллмана, 
равный
$-\left( S_t h_t^2\hm+\right.$\linebreak
$\left.{}+H_t\right)^{-1} \left( c_t 
{\partial V_t(y,z)}/{\partial 
z}\hm-2S_t\left( s_t y \hm-g_t z\right) h_t \right)^2/4.
$ 
     
     Отметим, что, как и~в~классической ли\-ней\-но-квад\-ра\-тич\-ной 
задаче, управ\-ле\-ние из класса до\-пус\-ти\-мых не\-упреж\-да\-ющих получилось 
управ\-ле\-ни\-ем с~обратной связью.
     
     Таким образом, функция Беллмана описывается сле\-ду\-ющим 
дифференциальным уравнением:
     \begin{multline}
     \fr{\partial V_t(y,z)}{\partial t} +\fr{1}{2}\left( \Sigma_t^2(y) \fr{\partial^2 
V_t(y,z)} {\partial y^2}+\sigma_t^2\fr{\partial^2 V_t(y,z)} {\partial 
z^2}\right)+{}\\
     {}+ A_t(y) \fr{\partial V_t(y,z)}{\partial y}+\left( a_t y+b_t z\right) 
\fr{\partial V_t(y,z)}{\partial z}+{}\\
{}+ S_t \left( s_t y- g_t z\right)^2 +G_t z^2-
 \fr{1}{4}\left( S_t h_t^2+H_t\right)^{-1}\times{}\\
 {}\times \left( c_t \fr{\partial V_t(y,z)} 
{\partial z}-2S_t\left( s_t y -g_t z\right) h_t \right)^2=0\,.
     \label{e13-bos}
     \end{multline}
     
     Возводя в~квадрат по\-след\-нее сла\-га\-емое в~(\ref{e13-bos}), перепишем 
его в~виде:
     \begin{multline}
     \fr{\partial V_t(y,z)}{\partial t} +\fr{1}{2}\left( \Sigma_t^2(y) \fr{\partial^2 
V_t(y,z)} {\partial y^2}+\sigma_t^2\fr{\partial^2 V_t(y,z)} {\partial 
z^2}\!\right)+{}\\
{}+A_t(y) \fr{\partial V_t(y,z)}{\partial y}
+ \left( 
\vphantom{\left( S_t h_t^2 +H_t\right)^{-1}}
a_t y+b_t z+{}\right.\\
\left.{}+\left( S_t h_t^2 +H_t\right)^{-1}
 c_t S_t \left( s_t y-g_t z\right) h_t
\right) 
     \fr{\partial V_t(y,z)}{\partial z}+{}\\
     {}+\left( S_t-\left( S_t h_t^2 +H_t\right)^{-1} S_t^2 h_t^2\right)\left( s_t y -
g_t z\right)^2+{}\\
     \!\!{}+
     G_t z^2 -\fr{1}{4}\left( S_t h_t^2+H_t\right)^{-1}\! c_t^2
     \left(\fr{\partial V_t(y,z)}{\partial z}\right)^{\!2}=0\,.\!\!
     \label{e14-bos}
     \end{multline}
     
     Рассматривая полученное уравнение, заметим, что его решение может 
быть пред\-став\-ле\-но в~виде:
   \begin{equation}
     V_t(y,z)= \alpha_t z^2+\beta_t(y) z +\gamma_t(y)\,,
     \label{e15-bos}
     \end{equation}
т.\,е.\ будем искать решение при дополнительном предположении 
о~квад\-ра\-тич\-ности функции Белл\-ма\-на по переменной~$z$, и~сведем, таким 
образом, поиск оптимального решения к~уравнениям относительно функций 
$\alpha_t$, $\beta_t(y)$ и~$\gamma_t(y)$. Отметим сразу, что явный вид 
функции~$\gamma_t(y)$ для реализации оптимального управ\-ле\-ния не 
требуется, однако далее будет предложен вариант вы\-чис\-ле\-ния и~этой 
функции, что пред\-став\-ля\-ет\-ся небесполезным, поскольку позволит выполнять 
расчет минимума целевого функционала. Источником для 
предложения~(\ref{e15-bos}) является уже упоминавшаяся аналогичная 
задача для случая дис\-крет\-но\-го времени~\cite{7-bos, 8-bos}. В~той задаче 
выражение для функции Беллмана получается формально без 
дополнительных усилий. При этом форма~(\ref{e15-bos}) обнаруживается 
как свойство оптимального решения. В~рассматриваемом случае 
непрерывного времени~(\ref{e15-bos}) постулируется, а~пра\-виль\-ность 
постулата под\-тверж\-да\-ет\-ся далее ре\-зуль\-ти\-ру\-ющи\-ми уравнениями 
для~$\alpha_t$, $\beta_t(y)$ и~$\gamma_t(y)$ Кроме того, данное 
предположение пред\-став\-ля\-ет\-ся вы\-те\-ка\-ющим из линейной структуры задачи 
в~отношении переменной~$z$, в~част\-ности, тем фактом, что такой вид 
функции Беллмана обеспечивает линейность оптимального 
управ\-ле\-ния~(\ref{e12-bos}) по~$z$.

     Граничное условие при выбранном предположении~(\ref{e15-bos}) 
принимает вид:

\noindent
     \begin{multline*}
     V_T(y,z)= S_T\left( s_T y- g_T z\right)^2+G_T z^2 ={}\\[-0.5pt]
     {}=\alpha_T z^2 
+\beta_T(y) z +\gamma_T(y)\,,
    \end{multline*}
т.\,е.

\noindent
\begin{align*}
\alpha_T&= S_T g_T^2 +G_T\,;\\[-0.5pt]
\beta_T(y)&=-2S_T s_T g_T y\,;\\[-0.5pt]
\gamma_T(y)&=S_T s_T^2 y^2\,.
%\label{e16-bos}
\end{align*}
          При этом само оптимальное управ\-ле\-ние, определенное 
выражением~(\ref{e12-bos}), оказывается управ\-ле\-ни\-ем с~обратной связью 
по~$y_t$ и~$z_t$:

\noindent
     \begin{multline}
     u_t^*=u_t^*(y,z) ={}\\[-0.5pt]
     {}=
     -\fr{1}{2}\left( S_t h_t^2 +H_t\right)^{-1}
     \left( c_t \left( 2\alpha_t z +\beta_t(y)\right) +{}\right.\\[-0.5pt]
    \left. {}+2S_t\left( s_t y-g_t z\right) 
h_t\right)\,.
     \label{e17-bos}
     \end{multline}
          Подставляем $V_t(y,z)\hm= \alpha_t z^2 \hm+ \beta_t(y) 
z\hm+\gamma_t(y)$ в~(\ref{e14-bos}):

\noindent
     \begin{multline*}
     \fr{\partial \alpha_t}{\partial t}\, z^2 +
     \fr{\partial \beta_t(y)}{\partial t}\,z +
     \fr{\partial \gamma_t(y)}{\partial t}+{}\\[-0.5pt]
     {}+\fr{1}{2}\left( \Sigma_t^2(y) \left( 
\fr{\partial^2\beta_t(y)}{\partial y^2}\,z +\fr{\partial^2 \gamma_t(y)}{\partial 
y^2}\right) +2\sigma_t^2\alpha_t\right)+{}\\[-0.5pt]
 {}+A_t(y)\left(\fr{\partial \beta_t(y)}{\partial y}\,z + \fr{\partial 
\gamma_t(y)}{\partial y}\right) +{}\\[-0.5pt]
\hspace*{-0.22987pt}{}+\left( a_t y+b_t z+\left( S_t h_t^2 +H_t\right)^{-1} c_t S_t \left( s_t y-
g_t z\right) h_t\right)\times{}
\end{multline*}

\noindent
\begin{multline*}
         {}\times \left( 2\alpha_t z+\beta_t(y)\right)+{}\\
     {}+\left( S_t-\left( S_t h_t^2 +H_t\right)^{-1} S_t^2 h_t^2\right)\left( s_t y-
g_t z\right)^2+{}\\
     {}+ G_t z^2 -\fr{1}{4}\left( S_t h_t^2 +H_t\right)^{-1} c_t^2 \left( 
2\alpha_t z+\beta_t(y)\right)^2=0\,.
     \end{multline*}
          Далее выделяем слагаемые при~$z^2$, $z$ и~$z^0$
          
          \noindent
     \begin{multline*}
     \fr{\partial \alpha_t}{\partial t}\, z^2 +\fr{\partial \beta_t(y)}{\partial t}\,z +
     \fr{\partial \gamma_t(y)}{\partial 
t}+\fr{1}{2}\,\Sigma_t^2(y)\fr{\partial^2\beta_t(y)}{\partial y^2}\,z+ {}\\
{}+
\fr{1}{2}\,\Sigma_t^2(y)\fr{\partial^2\gamma_t(y)}{\partial 
y^2}+\sigma_t^2\alpha_t+A_t(y)\fr{\partial \beta_t(y)}{\partial y}\,z +{}\\
{}+A_t(y) \fr{\partial 
\gamma_t(y)}{\partial y}+{}\\
{}+ 2\alpha_t \left( b_t -\left( S_t h_t^2+H_t\right)^{-1} c_t 
S_t h_t g_t \right)z^2+{}\\
     {}+
     \left( 2\alpha_t\left( \alpha_t+\left( S_t h_t^2+H_t\right)^{-1} c_t S_t h_t 
s_t\right)y +{}\right.\\
\left.{}+\beta_t(y) \left( b_t-\left( S_t h_t^2+H_t\right)^{-1} c_t S_t h_t 
g_t\right) \right) z+{}\\
     {}+\beta_t(y)\left( a_t +\left( S_t h_t^2+H_t\right)^{-1} c_t S_t h_t s_t\right) 
y+{}\\
{}+ \left( S_t -\left( S_t h_t^2+H_t\right)^{-1} S_t^2 h_t^2\right) g_t^2 z^2-{}\\
     {}- 2\left( S_t -\left( S_t h_t^2+H_t\right)^{-1} S_t^2 h_t^2\right) s_t g_t yz 
+{}\\
{}+
     \left( S_t-\left( S_t h_t^2+H_t\right)^{-1} S_t^2 h_t^2\right) s_t^2 y^2+{}\\
     {}+G_t z^2 -\left( S_t h_t^2 +H_t\right)^{-1} c_t^2 \alpha_t^2 z^2 -{}\\
     {}-\left( 
S_t h_t^2+H_t\right)^{-1} c_t^2 \alpha_t \beta_t(y) z-{}\\
{}-
\fr{1}{4}\left( S_t h_t^2+H_t\right)^{-1}  c_t^2 \beta_t^2(y)=0\,,
     \end{multline*}
группируем их и~получаем сле\-ду\-ющие уравнения:
\begin{itemize}
\item  для~$\alpha_t$:

\noindent
\begin{multline}
\fr{\partial\alpha_t}{\partial t}+2\alpha_t\left( b_t-\left( S_t h_t^2+H_t\right)^{-1} c_t 
S_t h_t g_t\right)+{}\\
{}+ \left( S_t- \left( S_t h_t^2+H_t\right)^{-1} S_t^2 h_t^2\right) 
g_t^2+G_t-{}\\
\hspace*{-8mm}{}-\left( S_t h_t^2+H_t\right)^{-1} c_t^2 \alpha_t^2 =0\,,\enskip \alpha_T=S_T 
g_t^2+G_T\,;\!\!
\label{e18-bos}
\end{multline}
\item для $\beta_t$:

\noindent
\begin{multline}
\fr{\partial\beta_t(y)}{\partial 
t}+\fr{1}{2}\,\Sigma_t^2(y)\fr{\partial^2\beta_t(y)}{\partial y^2} 
+A_t(y)\fr{\partial \beta_t(y)}{\partial y}+{}\\
{}+ 2\alpha_t\left( a_t +\left( S_t h_t^2+H_t\right)^{-1} c_t S_t h_t s_t\right) y+{}\\
{}+
\beta_t(y)\left( b_t -\left( S_t h_t^2 +H_t\right)^{-1} c_t S_t h_t g_t\right)-{}\\
{}-2\left( S_t-\left( S_t h_t^2+H_t\right)^{-1} S_t^2 h_t^2\right) s_t g_t y-{}
\\
{}-
\left( S_t h_t^2+H_t\right)^{-1} c_t^2 \alpha_t \beta_t(y)=0\,,\\
\beta_T(y)=-2S_T s_T g_T y\,;
\label{e19-bos}
\end{multline}
\item  для $\gamma_t$:
\begin{multline}
\hspace*{-0.8pt}\fr{\partial \gamma_t(y)}{\partial t}+\fr{1}{2}\,\Sigma_t^2(y)
\fr{\partial^2 \gamma_t(y)}{\partial y^2} +\sigma_t^2 \alpha_t +A_t(y)
\fr{\partial \gamma_t(y)}{\partial y}+{}\\
{}+ \beta_t(y)\left( a_t +\left( S_t h_t^2+H_t\right)^{-1} c_t S_t h_t s_t\right) y+{}\\
{}+
\left( S_t-\left( S_t h_t^2+H_t\right)^{-1} S_t^2 h_t^2\right)  s_t^2 y^2-{}\\
{}-\fr{1}{4}\left( S_t h_t^2+H_t\right)^{-1} c_t^2 \beta_t^2(y) =0\,,\\
\gamma_T(y)=S_T s_T^2 y^2\,.
\label{e20-bos}
\end{multline}
\end{itemize}
     
     Уравнение~(\ref{e18-bos}), легко заметить, является уравнением 
Риккати, которое в~силу сформулированного выше условия   
имеет единственное неотрицательное решение для всех $0\hm\leq t\hm\leq T$. 
Этот факт требует дополнительного комментария. Для получения 
уравнения~(\ref{e18-bos}) рас\-смот\-рим исходную задачу при дополнительных 
условиях $a_t\hm=0$ и~$s_t\hm=0$ для всех $0\hm\leq t\hm\leq T$. Нетрудно 
видеть, что эти условия рассматриваемую по\-ста\-нов\-ку сводят фактически 
к~классической ли\-ней\-но-квад\-ра\-тич\-ной задаче. Имеющуюся 
в~рассматриваемой формулировке чуть более общую форму целевой 
функции (принципиального значения это обобщение, конечно, не имеет) 
сведем к~классической еще одним предположением: $S_t\hm=0$ для всех 
$0\hm\leq t\hm\leq T$. Теперь уравнение для~$\alpha_t$ принимает хорошо 
известный вид:
     \begin{equation}
     \fr{\partial \alpha_t}{\partial t}+2\alpha_t b_t +G_t- H_t^{-1} c_t^2 
\alpha_t^2=0\,,\enskip \alpha_T=G_T\,.
     \label{e21-bos}
     \end{equation}

     В таком случае, как известно~\cite{10-bos}, существует единственное 
оптимальное управление~--- линейное с~обратной связью по выходу~$z_t$, 
с~коэффициентом усиления, опи\-сы\-ва\-емым уравнением  
Риккати~(\ref{e21-bos}). Именно этот результат дают  
уравнения~(\ref{e18-bos})--(\ref{e20-bos}) и~описываемая ими функция 
Беллмана~(\ref{e15-bos}), так как из $a_t\hm=0$ и~$s_t\hm=0$ немедленно 
следует, что $\beta_t(y)\hm=0$, откуда, в~свою очередь, с~учетом 
не\-за\-ви\-си\-мости решения от~$y_t$ следует, что $\gamma_t(y)\hm=\gamma_t$, 
т.\,е.\ не зависит от~$y$ и~задается уравнением: 
     $$
     \fr{\partial \gamma_t(y)}{\partial t} +\sigma^2_t \alpha_t=0\,,\enskip 
\gamma_T=0\,.
     $$ 
     Оптимальное управ\-ле\-ние при этом 
     $$
     u_t^*= -H_t^{-1} c_t \alpha_t z_t\,,
     $$
      т.\,е.\ все полностью совпадает с~известным классическим решением.
     
     С уравнениями~(\ref{e19-bos}) и~(\ref{e20-bos}) ситуация, естественно, 
обстоит сложнее. Это линейные уравнения второго порядка параболического 
типа, поскольку\linebreak
 $\Sigma_t^2(y)\hm>0$. Фактически отсутствуют 
конструктивные условия, гарантирующие существование их\linebreak
 решений 
(требовать, чтобы все фигурирующие в~уравнениях коэффициенты были 
представлены аналитическими функциями на всем пространстве значений, 
вряд ли целесообразно), поэтому далее будем предполагать, что данные 
уравнения имеют на рас\-смат\-ри\-ва\-емом интервале $0\hm\leq t\hm\leq T$ хотя 
бы одно ограниченное решение и~именно эти условия будем рас\-смат\-ри\-вать 
как достаточные условия существования оптимального решения 
рассматриваемой задачи.
     
     Таким образом, доказана следующая тео\-рема.
     
     \smallskip
     
     \noindent
     \textbf{Теорема.}\ \textit{Пусть для диффузионного 
процесса}~(\ref{e5-bos}) \textit{выполнены условия Ито, для 
     процесса}~(\ref{e6-bos})~--- \textit{ограничены коэффициенты, 
уравнения}~(\ref{e18-bos})--(\ref{e20-bos}) \textit{имеют ограниченные 
решения для $0\hm\leq t\hm\leq T$. Тогда минимум  
функционалу}~(\ref{e7-bos}) \textit{доставляет оптимальное 
управ\-ле\-ние}~(\ref{e17-bos}), \textit{где} $y\hm= y_t$; $z\hm=z_t$.
     
\section{Заключение}

     Рассмотренная задача оптимизации в~целом близка и~по модели, и~по 
критерию к~классической ли\-ней\-но-квад\-ра\-тич\-ной постановке. 
Принципиальным отличием является нелинейная модель для описания 
со\-сто\-яния динамической сис\-те\-мы, в~которой отсутствует управ\-ля\-ющее 
воздействие.\linebreak
 Такую модель наряду с~традиционной интер\-пре\-тацией  
<<со\-сто\-яние--вы\-ход>> мож\-но понимать как\linebreak модель неконтролируемого 
слож\-но\-го внешнего воздействия. Небольшое дополнительное отличие дает 
предложенная форма квад\-ра\-тич\-но\-го критерия, поз\-во\-ля\-ющая, в~част\-ности, 
ставить такие задачи, как отслеживание выходом или управ\-ле\-ни\-ем со\-сто\-яния 
сис\-те\-мы или ее выхода.
     
     Поскольку обсуждать возможности точного решения уравнений, 
определяющих оптимальное управ\-ле\-ние, не имеет смыс\-ла, наиболее 
актуальной далее является задача их приближенного чис\-лен\-но\-го решения 
и~анализа воз\-мож\-ности практической реализации. Этому посвящена вторая 
часть данной работы, пла\-ни\-ру\-емая к~выходу в~ближайшее время.

{\small\frenchspacing
 {%\baselineskip=10.8pt
 \addcontentsline{toc}{section}{References}
 \begin{thebibliography}{99}
\bibitem{1-bos}
\Au{Athans M.} Editorial on the LQG problem~// IEEE~T. Automat. Contr., 1971. Vol.~16. 
No.\,6. P.~528--552. doi: 10.1109/TAC.1971.1099845.
\bibitem{2-bos}
\Au{Wu Z.} Forward-backward stochastic differential equations, linear quadratic stochastic 
optimal control and nonzero sum differential games~// J.~Syst. Sci. Complex., 2005. Vol.~18. 
No.\,2. P.~179--192.
\bibitem{3-bos}
\Au{Chen B.\,S., Zhang~W.} Stochastic H2/H1 control with state-dependent noise~// IEEE 
T.~Automat. Contr., 2004. Vol.~49. No.\,1. P.~45--56. doi: 10.1109/TAC.2003.821400.
\bibitem{4-bos}
\Au{Bohacek S.} A~stochastic model of TCP and fair video transmission~// IEEE 
INFOCOM, 2003. Vol.~2. P.~1134--1144. doi: 10.1109/INFCOM.2003.1208950.
\bibitem{5-bos}
\Au{Домбровский В.\,В., Объедко~Т.\,Ю.} Управление с~прогнозированием системами 
с~марковскими скачками при ограничениях и~применение к~оптимизации 
инвестиционного портфеля~// Автомат. телемех., 2011. №\,5. С.~96--112. doi: 
10.1134/S0005117911050079.
\bibitem{6-bos}
\Au{Баландин Д.\,В., Коган~М.\,М.} Оптимальное линейно-квад\-ра\-тич\-ное управление: от 
матричных уравнений к~линейным матричным неравенствам~// Автомат. телемех., 2011. 
№\,11. С.~60--69. doi: 10.1134/ S0005117911110038.
\bibitem{7-bos}
\Au{Босов А.\,В.} Обобщенная задача распределения ресурсов программной системы~// 
Информатика и~её применения, 2014. Т.~8. Вып.~2. С.~39--47. doi: 
10.14357/19922264140204.
\bibitem{8-bos}
\Au{Босов А.\,В.} Управление линейным выходом дискретной стохастической системы по 
квадратичному критерию~// Изв. РАН. Теория и~системы управления, 2016. №\,3.  
С.~19--35. doi: 10.1134/S1064230716030060.
\bibitem{9-bos}
\Au{Флеминг У., Ришел~Р.} Оптимальное управление детерминированными 
и~стохастическими системами~/ Пер. с~англ.~--- М.: Мир, 1978. 316~с. 
(\Au{Fleming~W.\,H., Rishel~R.\,W.} Deterministic and stochastic optimal control.~--- New 
York, NY, USA: Springer-Verlag, 1975. 222~p.)
\bibitem{10-bos}
\Au{Девис М.\,Х.\,А.} Линейное оценивание и~стохастическое управление~/ Пер. с~англ.~--- 
М.: Наука, 1984. 206~с. (\Au{Davis~M.\,H.\,A.} Linear estimation and stochastic control.~--- 
London: Chapman and Hall, 1977. 224~p.)

 \end{thebibliography}

 }
 }

\end{multicols}

\vspace*{-6pt}

\hfill{\small\textit{Поступила в~редакцию 30.03.18}}

\vspace*{4pt}

%\newpage

%\vspace*{-24pt}

\hrule

\vspace*{2pt}

\hrule

\vspace*{-2pt}


\def\tit{STOCHASTIC DIFFERENTIAL SYSTEM OUTPUT CONTROL 
BY~THE~QUADRATIC CRITERION.~I.~DYNAMIC\\ PROGRAMMING 
OPTIMAL SOLUTION}


\def\titkol{Stochastic differential system output control 
by~the~quadratic criterion. I.~Dynamic programming 
optimal solution}

\def\aut{A.\,V.~Bosov and~A.\,I.~Stefanovich}

\def\autkol{A.\,V.~Bosov and~A.\,I.~Stefanovich}

\titel{\tit}{\aut}{\autkol}{\titkol}

\vspace*{-11pt}


\noindent
Institute of Informatics Problems, Federal Research Center ``Computer Science 
and Control'' of the Russian Academy of Sciences, 44-2~Vavilov Str., Moscow 
119333, Russian Federation


\def\leftfootline{\small{\textbf{\thepage}
\hfill INFORMATIKA I EE PRIMENENIYA~--- INFORMATICS AND
APPLICATIONS\ \ \ 2018\ \ \ volume~12\ \ \ issue\ 3}
}%
 \def\rightfootline{\small{INFORMATIKA I EE PRIMENENIYA~---
INFORMATICS AND APPLICATIONS\ \ \ 2018\ \ \ volume~12\ \ \ issue\ 3
\hfill \textbf{\thepage}}}

\vspace*{3pt}



\Abste{The problem of optimal control for the Ito diffusion 
process and a~controlled linear output is solved. The considered 
statement is close to the classical linear-quadratic Gaussian 
control  (LQG control) problem. Differences consist in the fact 
that the state is described by the nonlinear differential Ito equation  $dy_y = A_t(y_t) 
\,dt+\Sigma_t(y_t)\,dv_t$ and does not depend on the control~$u_t$, 
optimization subject is controlled linear output 
 $dz_t=a_ty_t\,dt +b_tz_t\,dt +c_t u_t\,dt +\sigma_t \,dw_t$. 
Additional generalizations are included in the quadratic 
quality criterion for the purpose of statement such problems 
as state tracking by output or a linear combination of state 
and output tracking by control. The method of dynamic programming 
is used for the solution. 
The assumption about Bellman function in the form  $V_t(y,z)= \alpha_t 
z^2+\beta_t(y) z+\gamma_t(y)$ allows one to find it. 
Three differential equations for the coefficients $\alpha_t$,  $\beta_t(y)$,
and $\gamma_t(y)$ give the solution. 
These equations constitute the optimal solution of the problem under consideration.}

\KWE{stochastic differential equation; optimal control; dynamic programming; 
Bellman function; Riccati equation; linear differential equations of parabolic type}


\DOI{10.14357/19922264180314}

\vspace*{-12pt}

\Ack
\noindent
This work was partially supported by the Russian Science Foundation (grant  
16-07-00677).



%\vspace*{6pt}

  \begin{multicols}{2}

\renewcommand{\bibname}{\protect\rmfamily References}
%\renewcommand{\bibname}{\large\protect\rm References}

{\small\frenchspacing
 {%\baselineskip=10.8pt
 \addcontentsline{toc}{section}{References}
 \begin{thebibliography}{99}
\bibitem{1-bos-1}
\Aue{Athans, M.} 1971. Editorial on the LQG problem. \textit{IEEE~T. 
Automat. Contr.} 16(6):528--552. doi: 10.1109/ TAC.1971.1099845.
\bibitem{2-bos-1}
\Aue{Wu, Z.} 2005. Forward-backward stochastic differential equations, linear 
quadratic stochastic optimal control and\linebreak\vspace*{-12pt}

\columnbreak

\noindent
 nonzero sum differential games. 
\textit{J.~Syst. Sci. Complex.} 18(2):179--192.
\bibitem{3-bos-1}
\Aue{Chen, B.\,S. and W.~Zhang.} 2004. Stochastic H2/H1 control with  
state-dependent noise. \textit{IEEE~T. Automat. Contr.} 49(1):45--56.
doi: 10.1109/TAC.2003.821400.
\bibitem{4-bos-1}
\Aue{Bohacek, S.} 2003. A~stochastic model of TCP and fair video 
transmission. \textit{IEEE INFOCOM}. 2:1134--1144.
doi: 10.1109/INFCOM.2003.1208950.
\bibitem{5-bos-1}
\Aue{Dombrovskii, V.\,V., and T.\,Yu.~Ob''edko.} 2011. Predictive control of 
systems with Markovian jumps under constraints and its application to the 
investment portfolio optimization. \textit{Automat. Rem. Contr.}  
72(5):989--1003.
\bibitem{6-bos-1}
\Aue{Balandin, D.\,V., and M.\,M.~Kogan.} 2011. Optimal linear-quadratic 
control: From matrix equations to linear matrix inequalities. \textit{Automat. 
Rem. Contr.} 72(11):2276--2284.
\bibitem{7-bos-1}
\Aue{Bosov, A.\,V.} 2014. Obobshchennaya zadacha raspredeleniya resursov 
programmnoy sistemy [The generalized problem of software system resources 
distribution]. \textit{Informatika i~ee Primeneniya~--- Inform. Appl.}  
8(2):39--47. doi: 
10.14357/19922264140204.
\bibitem{8-bos-1}
\Aue{Bosov, A.\,V.} 2016. Discrete stochastic system linear output control 
with respect to a quadratic criterion. \textit{J.~Comput. Syst. Sc. 
Int.} 55(3):349--364.
\bibitem{9-bos-1}
\Aue{Fleming, W.\,H., and R.\,W.~Rishel.} 1975. \textit{Deterministic and 
stochastic optimal control.} New York, NY: Springer-Verlag. 222~p.
\bibitem{10-bos-1}
\Aue{Davis, M.\,H.\,A.} 1977. \textit{Linear estimation and stochastic 
control.} London: Chapman and Hall. 224~p.
\end{thebibliography}

 }
 }

\end{multicols}

\vspace*{-6pt}

\hfill{\small\textit{Received March 30, 2018}}

%\pagebreak

%\vspace*{-18pt}
     
     \Contr
     
       \noindent
       \textbf{Bosov Alexey V.} (b.\ 1969)~--- Doctor of Science in technology, 
principal scientist, Institute of Informatics Problems, Federal Research 
Center ``Computer Science and Control'' of the Russian Academy of Sciences, 
44-2~Vavilov Str., Moscow 119333, Russian Federation; 
\mbox{AVBosov@ipiran.ru}
       
       \vspace*{3pt}
       
       \noindent
       \textbf{Stefanovich Alexey I.} (b.\ 1983)~--- principal specialist, 
Institute of Informatics Problems, Federal Research Center ``Computer Science 
and Control'' of the Russian Academy of Sciences, 44-2~Vavilov Str., Moscow 
119333, Russian Federation; \mbox{AStefanovich@frccsc.ru}
\label{end\stat}

\renewcommand{\bibname}{\protect\rm Литература}       

       %5
\def\stat{stupnikov}

\def\tit{ВЕРИФИЦИРУЕМОЕ ОТОБРАЖЕНИЕ МОДЕЛИ ДАННЫХ, ОСНОВАННОЙ НА~МНОГОМЕРНЫХ МАССИВАХ, 
В~ОБЪЕКТНУЮ~МОДЕЛЬ ДАННЫХ$^*$}

\def\titkol{Верифицируемое отображение модели данных, основанной на~многомерных массивах, 
в~объектную модель данных}

\def\autkol{С.\,А.~Ступников}

\def\aut{С.\,А.~Ступников$^1$}

\titel{\tit}{\aut}{\autkol}{\titkol}

{\renewcommand{\thefootnote}{\fnsymbol{footnote}}\footnotetext[1] {Работа 
выполнена при поддержке РФФИ (проект 11-07-00402-а). Статья рекомендована к 
публикации в журнале Программным комитетом конференции <<Электронные 
библиотеки: перспективные методы и технологии, электронные коллекции>> 
(RCDL-2012).}}

\renewcommand{\thefootnote}{\arabic{footnote}}
\footnotetext[1]{Институт проблем информатики Российской академии наук, 
ssa@ipi.ac.ru}

\vspace*{-6pt}       

\Abst{Рассматривается отображение модели данных, основанной на 
многомерных мас\-си\-вах (ММ-мо\-де\-ли), в объектную модель данных. Изложены 
общие принципы отображения ММ-мо\-де\-лей в объектные модели данных. 
Рассмотрено отображение конкретной модели~--- Array Data Model (ADM), 
использующейся в системе управления базами данных (СУБД) SciDB, в язык СИНТЕЗ, 
использующийся в качестве канонической модели данных в технологии предметных 
посредников. Проиллюстрирован метод верификации отображения~--- доказательства 
сохранения информации и семантики операций при отображении. Верификация 
осуществляется при помощи формального языка спецификаций AMN. Практической 
целью работы ставилось создание базы для виртуальной или материализованной 
интеграции ресурсов, основанных на многомерных массивах.}

\vspace*{-1pt}

\KW{многомерные массивы; объектная модель данных; отображение моделей 
данных; интеграция баз данных}

\vspace*{-6pt}

\vskip 14pt plus 9pt minus 6pt

      \thispagestyle{headings}

      \begin{multicols}{2}

            \label{st\stat}
            

\section{Введение}

        Развитие науки и промышленности, широкое распространение 
информационных технологий ведет к накоплению огромных объемов данных 
как в науке, так и в бизнесе. Данные могут быть как наблюдательными, 
экспериментальными, так и полученными в ходе компьютерного 
моделирования. Данные таких масштабов (часто измеряемых уже в петабайтах) 
называются <<большими данными>> (Big Data)~\cite{1-stu}. Они плохо 
поддаются обработке и анализу в рамках широко известных технологий баз 
данных, опирающихся в основном на реляционную модель данных.
        
        Именно поэтому развиваются различные модели данных, нацеленные на 
параллельную обработку и анализ данных в распределенных средах~--- гридах 
и облаках. Важными видами таких моделей являются модели данных, 
основанные на многомерных массивах (array-based data models, или ADM) 
и называемые далее ММ-мо\-де\-ля\-ми. Родственны данным моделям 
так называемые <<кубы данных>>, используемые в 
OLAP (online analytical processing) тех\-но\-ло\-гии~[2--4]. 
Исследования ММ-мо\-де\-лей начались достаточно 
давно~\cite{4-stu, 5-stu} и продолжают развиваться. В~данной статье 
рассматривается конкретная модель, а именно модель, используемая в СУБД 
SciDB~\cite{6-stu}.
        
        История SciDB начинается с 2007~г., когда на симпозиуме по 
экстремально большим базам данных (XLDB~--- extremely large data bases) 
представителями науки и 
промышленности был сделан вывод о том, что существующие СУБД не в 
состоянии манипулировать объемами данных, которые появятся в ближайшем 
будущем. Одним из примеров поставщиков таких данных служит строящийся 
телескоп LSST (Large Synoptic Survey Telescope)~\cite{7-stu}. Был также сделан 
вывод о необходимости разработки СУБД нового поколения, которая должна 
удовлетворять, в частности, следующим требованиям~\cite{8-stu}:
        \begin{itemize}
\item модель данных основывается на многомерных массивах, а не на 
кортежах;
\item модель хранения базируется на версионности, а не на обновлении 
значений;
\item масштабируемость до сотен петабайт и высокая отказоустойчивость;
\item СУБД является свободно распространяемым программным 
обеспечением.
\end{itemize}

        Некоторое время спустя был запущен международный проект под 
руководством Майкла Стоунбрейкера, целью которого стало создание новой 
СУБД, получившей название SciDB. В~настоящее 
время свободно распространяется очередная версия системы для операционных
сис\-тем (ОС) Ubuntu и  RedHat.
        
        Целью данной статьи является исследование вопроса о верифицируемом 
отображении ММ-мо\-де\-лей, и в частности ADM~\cite{9-stu}, 
использующейся в системе SciDB, в объектные 
модели данных для виртуальной или материализованной интеграции ресурсов 
при создании федеративных баз данных или хранилищ данных. 
        
        При материализованной интеграции предполагается создание 
хранилища данных (warehouse), в которое загружаются ресурсы, подлежащие 
интеграции. В~процессе загрузки происходит преобразование данных из схемы 
ресурса в общую схему хранилища.
        
        Виртуальная же интеграция рассматривается в статье применительно к 
предметным посредникам~\cite{10-stu}. Предметные посредники представляют 
собой специальный вид программного обеспечения (ПО), образующий 
промежуточный слой между пользователем (приложением) и неоднородными 
информационными ресурсами. При этом данные из ресурсов не 
материализуются в посреднике. Федеративная схема посредника, описывающая 
некоторую предметную область, создается независимо от существующих 
ресурсов. Ресурсы, релевантные предметной области, затем регистрируются в 
посреднике~--- их схемы связываются специальными семантическими 
отображениями с федеративной схемой. Исполнительная среда посредников 
предо\-став\-ля\-ет возможность пользователям (приложениям) задавать запросы 
(программы) к посреднику в терминах федеративной схемы. Эти запросы 
переписываются в частичные запросы над информационными ресурсами, а 
затем исполняются на ресурсах. Результаты частичных запросов объединяются 
и выдаются пользователю также в терминах федеративной схемы.
        
        Важным понятием технологии систем интеграции баз данных является 
каноническая модель, служащая общим языком, унифицирующим 
разнообразные модели ресурсов.
        
        Необходимым предусловием интеграции ресурсов, основанных на 
многомерных массивах, является построение отображения соответствующей\linebreak 
ММ-мо-де\-ли в каноническую модель данных, сохраняющего информацию и 
семантику операций языка манипулирования данными (ЯМД)~\cite{11-stu}. 
Это обусловлено тем, что семантические отображения, связывающие 
федеративную схему и схемы ресурсов, нужно проводить в единой 
(канонической) модели~\cite{12-stu}. Отображение должно быть 
верифицируемым~--- доказуемо правильным. 
        
        В качестве канонической модели в данной работе рассматривается язык 
СИНТЕЗ~\cite{13-stu}~--- комбинированная слабоструктурированная и 
объектная модель данных, нацеленная на разработку предметных посредников 
для решения задач в средах неоднородных ресурсов. Разработан прототип 
программных средств для поддержки среды предметных посредников с языком 
СИНТЕЗ в роли канонической модели~\cite{14-stu}.
        
        С точки зрения предметных посредников СУБД, основанные на 
многомерных массивах, пред\-став\-ля\-ют собой новый вид ресурсов, подлежащих 
интеграции в посредниках вместе с привычными ресурсами~--- реляционными 
и объектными СУБД, веб-сер\-ви\-са\-ми и~т.\,д. 
        
        Нужно отметить, что ADM подвергается некоторой критике со стороны 
исследователей, продолжающих развитие моделей, основанных на 
многомерных массивах. Так, авторы языка SciQL~\cite{15-stu} отмечают, что 
язык ADM является смесью SQL и деревьев алгебраических операций. По их 
мнению, язык для СУБД, основанных на многомерных массивах, должен быть 
интегрирован с синтаксисом и семантикой SQL:2003. Несмотря на эти 
замечания, модель ADM представляет несомненный практический интерес для 
интеграции баз данных. SciDB используется как в научных проектах, связанных 
с LSST (предполагается после запуска телескопа) и физикой высоких энергий, 
так и в коммерческих, связанных с генетикой, страхованием, финансами. 
Сравнительное тестирование SciDB с СУБД Postgres и статистическим ПО R 
показало преимущества SciDB по производительности и масштабируемости.
        
        Статья организована следующим образом. В~разд.~2 рассмотрены и 
проиллюстрированы основные принципы отобра\-же\-ния модели данных ADM в 
язык СИНТЕЗ. Принципы обобщены на случай моделей, отличных от ADM и 
СИНТЕЗ. В~разд.~3 рассмотрен метод доказательства сохранения информации 
и семантики операций при отоб\-ра\-же\-нии моделей с использованием 
формального языка спецификаций AMN~\cite{16-stu}. Метод 
проиллюстрирован на структурах данных и операциях ЯМД моделей SciDB и 
СИНТЕЗ. В~разд.~4 рассмотрены некоторые родственные исследования и 
направления дальнейшей работы.

\vspace*{6pt}

\section{Отображение модели ADM в~язык СИНТЕЗ}

\vspace*{2pt}

        SciDB поддерживает два языка для работы с массивами: AQL (Array 
Query Language) и AFL (Array\linebreak Functional Language). Язык AQL является 
        SQL (Structured Query Language)
        по\-доб\-ным декларативным языком, включающим как операции 
языка описания данных (ЯОД), так и операции ЯМД. Язык AFL представляет собой функциональный язык 
манипулирования массивами, операции которого можно объединять в 
композиции. Допускается использование операций AFL в запросах AQL.
        
        Операции языков и отображение будут иллюстрироваться на 
адаптированных примерах из сценария применения SciDB в области 
оптической астрономии~\cite{17-stu}, а также на простых примерах из 
документации SciDB~\cite{9-stu}.

\subsection{Отображение языка определения данных}

        Отображение ЯОД в данном разделе описывается независимо от вида 
интеграции~--- виртуальной или материализованной.
        
        Основной единицей определения данных в модели ADM является 
массив, имеющий конечное количество {измерений} $d_1, d_2, \ldots , 
d_n$~[9]. Длиной измерения называется количество упорядоченных значений в 
этом измерении. По умолчанию типом измерения являются 64-бит\-ные целые 
числа. Поддерживаются также нецелочисленные измерения, например строки 
или числа с плавающей точкой. Каждая комбинация значений измерений 
соответствует ячейке массива, которая может содержать конечное количество 
значений, называемых \textit{атрибутами}. Типом атрибута может быть один 
из встроенных типов ADM~\cite{9-stu}.
        
        Основная операция ЯОД ADM~--- создание массива~--- выглядит 
следующим образом:
        \begin{verbatim}
CREATE ARRAY source
< ampExposureId: int64 NULL, 
   filterId: int8,
   apMag: double >
[ ra(double), de(double), objectId=0:*];
\end{verbatim}

        Создается массив оптических источников {\sf source}, измерениями 
которого являются координаты {\sf ra} и {\sf de} типа {\sf double} и целочисленный 
идентификатор объекта. Для целочисленного измерения указаны его нижняя (0) 
и верхняя (<<*>>, обозна\-ча\-ющая бесконечность) границы. Ячейка массива 
состоит из трех атрибутов: {\sf ampExposureId}, {\sf filterId}, 
{\sf apMag}. Указано, что 
атрибут {\sf ampExposureId} может принимать неопределенное значение {\sf NULL}. 
В~данном примере приведены только некоторые из реально используемых 
атрибутов и измерений.
        
        В языке СИНТЕЗ создание массива представляется определением 
одноименного класса:
        \begin{verbatim}
{ source; in: class;
  instance_type:{
  double ra;
  ra2long: {in: function; 
            params: {-ret/long}; };
  double de;
  de2long: {in: function; 
            params: {-ret/long}; };
  long objectId; metaslot lower: 0;  
  higher: inf; end
  objectIdBounds: {in: invariant;
    {{all s(source(s) -> s.objectId >= 0)}}
  };
  long ampExposureId;
  short filterId;
  double apMag;
  key: { unique; { ra, de, objectId } };
  definiteness: {obligatory;
    { ra, de, objectId, filterId, apMag } };
  };
}
\end{verbatim}

        Как измерения, так и атрибуты, составляющие ячейку, представляются в 
языке СИНТЕЗ атрибутами типа экземпляров ({\sf instance\_type}) класса. Между 
встроенными типами ADM ({\sf int8}, {\sf int64}, {\sf double} и~др.)\ и встроенными 
типами языка \mbox{СИНТЕЗ} ({\sf short}, {\sf long}, {\sf double}) устанавливается взаимно 
однозначное соответствие. Совокупность атрибутов, со\-от\-вет\-ст\-ву\-ющих 
измерениям, объявляется уникальной (инвариант {\sf key}, выражаемый 
встроенным утверждением {\sf unique}). Объявляется также, что атрибуты, 
соответствующие измерениям и не-{\sf NULL} атрибутам ADM, должны быть 
определены у всех экземпляров класса (инвариант {\sf definiteness}, выражаемый 
встроенным утверждением {\sf obligatory}).
        
        Таким образом обеспечивается сохранение отличи\-тель\-ных свойств 
многомерных массивов (<<кубов данных>>), существенным образом 
раз\-ли\-ча\-ющих измерения и атрибуты, со\-став\-ля\-ющие \mbox{ячейку}:
        \begin{itemize}
\item по набору значений измерений однозначно определяется набор 
значений атрибутов ячейки (уникальность измерений);
\item ячейка массива всегда определяется полным набором значений 
измерений (определенность измерений).
\end{itemize}

        Заметим также, что отсутствие в коллекции объекта с некоторым 
набором значений измерений означает \textit{пустую ячейку} в массиве.
        
        Для нецелочисленных измерений {\sf ra} и {\sf de} в языке СИНТЕЗ кроме 
атрибутов определяются функции {\sf ra2long}, {\sf de2long}, преобразующие 
нецелочисленные значения в целочисленные. Необходимость при\-вне\-се\-ния этих 
функций вызвана следующим. При попытке описать операции, характерные для 
ММ-мо\-де\-лей, в объектной модели (в частности, в языке СИНТЕЗ) 
выясняется необходимость применения принципиально различных механизмов 
работы с целочисленными и нецелочисленными измерениями. Это вызвано 
различием типов измерений, возможной неравномерностью шага измерения 
и~т.\,д.\linebreak Для того чтобы обеспечить возможность единообразного описания 
операций над цело\-чис\-лен\-ными и нецелочисленными измерениями и 
необходимы функции, приводящие нецелочисленные\linebreak измерения к 
целочисленным.
        
        Ограничения, связанные с нижними и верхними границами 
целочисленных измерений, пред\-став\-ля\-ют\-ся в языке СИНТЕЗ, во-пер\-вых, 
мета\-слотом соответствующего атрибута (например,\linebreak {\sf objectId}). В~метаслоте 
хранится метаинформация, связанная с атрибутом как с отдельной сущностью 
языка. В~данном случае метаслот включает два слота {\sf lower} и {\sf higher}, 
отвечающих соответственно верхней и нижней границе измерения. 
        Во-вто\-рых, создается инвариант (например, {\sf objectIdBounds}), 
предикативная спецификация которого устанавливает ограничения на значения 
измерения для каждого из объектов класса, отвечающего массиву. 
Спецификация инварианта имеет вид формулы первого порядка, где {\sf all}~--- 
квантор существования, <<\verb -> >> --- импликация.
        
        Необходимо отметить, что массив представляется в объектной модели 
множеством объектов класса (фактически кортежей значений атрибутов). При 
этом наблюдается некоторое противоречие со стремлением создателей 
        ММ-мо\-де\-лей \mbox{отойти} от моделей, основанных на кортежах. Однако в 
контексте интеграции ресурсов ММ-мо\-де\-ли это лишь один класс из 
большого множества разнообразных классов моделей данных. Представление 
специфических ММ-мо\-де\-лей в объектной модели является методологически 
и технически гораздо более простым и естественным, нежели использование 
многомерных массивов в качестве канонической модели.
        
        Изложенные принципы отображения ЯОД могут быть обобщены на 
случай, когда канонической является объектная или 
        объ\-ект\-но-ре\-ля\-ци\-он\-ная модель, отличная от языка СИНТЕЗ. 
Также не принципиален выбор модели данных, основанной на многомерных 
массивах. В~общем виде принципы отображения ЯОД выглядят следующим 
образом:
        \begin{itemize}
\item массив отображается в коллекцию типизированных объектов (класс) 
объектной модели;
\item измерения и атрибуты, составляющие ячейку массива, отображаются в 
атрибуты типа экземпляров класса;
\item между встроенными типами модели, основанной на многомерных 
массивах, и встроенными типами объектной модели устанавливается 
взаимно однозначное соответствие;
\item совокупность атрибутов, соответствующих измерениям, объявляется 
уникальной (при помощи механизма ключей, утверждений или 
инвариантов);
\item атрибуты, соответствующие измерениям и не-{\sf NULL} атрибутам ячейки 
массива, объявляются определенными (при помощи утверждений или 
инвариантов);
\item для нецелочисленных измерений в типе экземпляров дополнительно 
определяются методы, преобразующие нецелочисленные значения в 
целочисленные;
\item ограничения, связанные с нижними и верхними границами 
целочисленных измерений, отображаются при помощи инвариантов или 
встроенных утверждений о кардинальности соответствующих атрибутов. 
В~случае использования инвариантов при отображении границы измерений 
представляются также метаданными атрибута.
\end{itemize}

\subsection{Отображение языка манипулирования данными}

        При интеграции баз данных для установления семантических 
соотношений между схемами ресурсов и федеративной схемой необходимо 
отображение ЯОД исходной модели в каноническую. Язык манипулирования данными канонической 
модели, напротив, необходимо отображать в ЯМД исходной модели. Это 
связано с тем, что запросы к посреднику в канонической модели необходимо 
отображать в запросы к ресурсам.
        
        Отметим отличие виртуальной и материализованной интеграции. При 
виртуальной интеграции отображение ЯМД обеспечивает возможность 
трансляции программ на языке посредника в запросы на языке ресурсов. 
        
        В случае материализованной интеграции данные извлекаются из ресурса 
и представляются в хранилище в канонической модели. При этом программы 
на языке канонической модели исполняются непосредственно на данных. 
Отоб\-ра\-же\-ние\linebreak ЯМД нужно лишь для того, чтобы убедиться, что отображение 
моделей сохраняет информацию и семантику операций. Семантически 
правильное\linebreak отоб\-ра\-же\-ние служит базой для процесса 
        <<из\-вле\-че\-ния--пре\-образо\-ва\-ния--за\-груз\-ки>> (ETL), 
формирующего из данных ресурса данные хранилища:\linebreak ETL-про\-цесс может 
быть выражен только в терминах канонической модели.
        
        \smallskip
        
        Язык запросов (программ) модели СИНТЕЗ представляет собой 
        Datalog-по\-доб\-ный язык в объектной среде. Программа представляет 
собой набор конъюнктивных запросов (правил) вида 

\noindent
\begin{multline*}
        q(x/T): - C_1(x_1/T_1),\ldots , C_n(x_n/T_n), (X_1,Y_1), 
\ldots \\
\ldots F_m(X_m,Y_m), B\,.
        \end{multline*}
        Тело запроса представляет собой конъюнкцию 
        пре\-ди\-ка\-тов-кол\-лек\-ций, функциональных предикатов и 
ограничения. Здесь $C_i$~--- имена коллекций (классов), $F_i$~--- имена 
функций, $x_i$~--- имена переменных, значения которых пробегают по 
классам, $T_i$~--- типы переменных, $X_j$ и $Y_j$~--- входные и выходные 
параметры функций, $B$~--- ограничение, налагаемое на $x_i$, $X_j$, $Y_j$. 
Предикаты, находящиеся в голове правил, могут быть использованы в телах 
других правил в качестве пре\-ди\-ка\-тов-кол\-лек\-ций. 
        
        В дальнейшем будет часто использоваться запись 
        пре\-ди\-ка\-та-кол\-лек\-ции вида {\sf source([ra, de])}. Неформально это 
означает, что представляют интерес не объекты класса {\sf source} целиком, а 
лишь их атрибуты {\sf ra} и {\sf de}. Формально запись означает сокращение от 
{\sf source(\_/source.inst[ra, de])}. Здесь знак <<{\sf \_}>> обозначает анонимную 
переменную, {\sf source.inst}~--- анонимный тип экземпляров (instance) класса 
{\sf source}, а {\sf ra} и {\sf de}~--- необходимые атрибуты типа экземпляров класса.
        
        Будет также использоваться запись {\sf source([i, j, val1/val])}, означающая 
переименование атрибута {\sf val} в {\sf val1}.
        
        \medskip
        
        При отображении ЯМД будут сначала рассмотрены основные 
конструкции языка программ СИНТЕЗ, соответствующие конструкциям языка 
AQL. Затем будут рассмотрены конструкции \mbox{СИНТЕЗ}, соответствующие 
конструкциям языка AFL.
        
        Начнем рассмотрение с конструкций языка СИНТЕЗ, соответствующих 
конструкциям языка AQL, связанных с {извлечением} данных.
        
%        \smallskip
        
\paragraph*{Предикаты-классы, условия, подзапросы.} Рас\-смот\-рим 
программу, извлекающую координаты ({\sf ra}, {\sf de}) и апертурную звездную 
величину ({\sf apMag}) астрономических источников из класса  {\sf source} с 
условием на фильтр ({\sf filterId}) и апертурную звездную величину, причем 
запрос~{\sf q} использует результаты запроса~{\sf r}:
        \begin{verbatim}
q([ra,de,apMag]) :- r([ra,de,apMag]),
   filterId= #filterId.
r([ra,de,apMag]) :- source([ra,de,apMag]),
   apMag >= #apMag.
\end{verbatim}
Здесь {\sf \#filterId} и {\sf \#apMag}~--- некоторые константы 
соответствующих типов.
        
        Такая программа представляется в AQL сле\-ду\-ющим запросом:
        \begin{verbatim}
SELECT apMag FROM 
  ( SELECT apMag FROM source
    WHERE apMag >= #apMag )
WHERE filterId = #filterId;
\end{verbatim}
        
        Простые условия отображаются в AQL без изменений, рекурсивные 
запросы представляются вложенными запросами. Заметим, что координаты 
{\sf ra} и {\sf de} не указываются в секции {\sf SELECT}~--- они являются измерениями и 
извлекаются по умолчанию.
        
\paragraph*{Соединение классов.} Соединение по определенным атрибутам 
(например, {\sf objectId})
        \begin{verbatim}
q2([ra, de, filterId, uMag]) :- 
    source([ra, de, objectId, fliterId]), 
    objectSummary([objectId, uMag]).
\end{verbatim}
представляется в AQL конструкцией {\sf JOIN-ON}:
\begin{verbatim}
SELECT filterId, uMag INTO q2
FROM source
JOIN objectSummary 
ON source.objectId = objectSummary.objectId;
\end{verbatim}
где массив {\sf objectSummary} имеет вид: 
\begin{verbatim}
CREATE ARRAY objectSummary
<uMag: float NULL,  gMag: float NULL>
[ objectId=0:* ];
\end{verbatim}
        
\paragraph*{Агрегация.} Рассмотрим запрос, возвращающий объекты с 
минимальной звездной величиной {\sf uMag}:
        \begin{verbatim}
q([objectId, uMag]) :-  
  objectSummary(obj/[objectId, uMag]), 
    uMag = min(obj.uMag).
\end{verbatim}

        Запрос представляется в AQL с использованием агрегирующей функции 
того же рода:
        \begin{verbatim}
SELECT uMag
FROM source, 
 (SELECT min(uMag) AS min FROM Source)
WHERE uMag = min;
\end{verbatim}
        
\paragraph*{Группирование.} Рассмотрим запрос, возвра\-ща\-ющий среднее 
значение звездной величины {\sf uMag}, вычисленное на группе по 
идентификатору объекта {\sf filterId}:
        \begin{verbatim}
q([objectId, avgMag]) :- 
    group_by({objectId}, 
       q2(obj/[ra,de,filterId, uMag])),
    avgMag = average(obj.uMag).
\end{verbatim}

        Здесь коллекция {\sf q2}, на которой производится группирование по 
атрибуту {\sf objectId}~--- результат соединения классов {\sf source} и 
{\sf objectSummary}, рассмотренных выше.
        
        Очевидно, в AQL запрос представляется при помощи конструкции 
GROUP BY:
        \begin{verbatim}
SELECT avg(uMag) AS avgMag
FROM q2 GROUP BY objectId;
\end{verbatim}
        
        Рассмотрим конструкции языка СИНТЕЗ, соответствующие 
конструкциям языка AQL и связанные с {изменением} данных.

        
\paragraph*{Обновление.} Рассмотрим запрос, изменяющий значения в 
квадратной матрице (см.\ предыдущий пример) на значения с обратным знаком 
в том случае, если модуль значения больше~5:
        \begin{verbatim}
source(x/[i, j, val]) :- 
    source(x/[i, j, val1/val]), 
       abs(val) > 5, val = -val1.
\end{verbatim}
        
        В AQL данный запрос представляется сле\-ду\-ющим образом:
        \begin{verbatim}
UPDATE source
SET val =  -val WHERE abs(val) > 5;
\end{verbatim}


        
\paragraph*{Удаление.} Рассмотрим программу, удаляющую из базы данных 
класс и все его содержимое:
        \begin{verbatim}
-source(x) :- source(x).; 
delete_frame(source).
\end{verbatim}

        В правилах со знаком минус в голове осуществляется удаление объектов 
из коллекции. В~данном случае из коллекции удаляются все объекты. Функция 
{\sf delete\_frame} удаляет коллекцию как объект из базы данных. Операция <<{\sf ;}>> 
обозначает последовательную композицию программ. В~AQL данный запрос 
представляется при помощи операции {\sf DROP}:
\begin{verbatim}
DROP ARRAY source;
\end{verbatim}

        Рассмотрим принципы отображения конструкций языка СИНТЕЗ, 
соответствующих конструкциям AFL, на примере {расширения элементов 
мас\-си\-ва в подмассивы}. Каждый элемент массива расширя\-ется в подмассив 
определенного размера. Значения всех ячеек подмассива дублируют значение 
оригинальной ячейки. Пример программы, расширяющей каждую ячейку 
матрицы $3\times3$ в подматрицу $2\times2$:
        \begin{verbatim}
q([i,j,val]) :- {x/[i,j,val] | exists y (
  source(y/[i1/i, j1/j, val]) & 
  ( i = i1*2 & j = j1*2 | i = i1*2 +1 & 
  j = j1*2 | i= i1*2 & 
  j= j1*2 + 1 | i= i1*2 +1 & j= j1*2 +1))}.
\end{verbatim}
    Здесь выражение $\{x/T \vert F(x)\}$, где $F$~--- формула со свободной 
переменной~$x$, обозначает конструкцию выделения множества; {\sf exists} 
обозначает квантор существования. 

\columnbreak
        
        В ADM запрос представляется с использованием операции {\sf xgrid}:
        \begin{verbatim}
SELECT * FROM xgrid(source, 2, 2);
\end{verbatim}
        
        Можно заметить, что операция AFL {\sf xgrid} имеет достаточно сложно 
устроенный прообраз в канонической модели (это справедливо и для многих 
других операций). Между тем эти операции являются естественными для 
массивов. Поэтому при интеграции ресурсов, основанных на многомерных 
массивах, в канонической модели возможно использование специального 
класса {\sf array}, инкапсулирующего специфические операции, характерные для 
многомерных массивов:
        \begin{verbatim}
{ array; in: class;
  instance_type: {
  xgrid: { in: function; 
    params: {
     +dimensions/{sequence; 
      type_of_element: string;},
     +scales/{sequence; 
      type_of_element: integer;}};
  };  };
}
\end{verbatim}
        В приведенном примере рассмотрена сигнатура единственной операции 
{\sf xgrid}, параметрами которой являются последовательность имен измерений\linebreak 
{\sf dimensions} и последовательность масштабов увеличения по каждому из 
измерений {\sf scales}. Па\-ра\-мет\-ром операции по умолчанию также считается 
класс\linebreak {\sf array} как коллекция объектов. При отображении ЯОД каждый класс~--- 
образ массива (например, класс {\sf source} из подразд.~2.1) становится подклассом 
класса {\sf array}:
        \begin{verbatim}
{ source; in: class; superclass: array;
  instance_type: { ... };
}
\end{verbatim}

        Аналогично {\sf xgrid}, операциями класса {\sf array} могут быть 
представлены такие операции AFL, как {\sf substitute}, {\sf sort}, 
{\sf multiply} и~т.\,д. 
        
        Заметим, что решение о представлении операций, характерных для 
многомерных массивов, в рамках специального класса канонической модели 
допускает обобщение на объектные канонические модели, отличные от языка 
СИНТЕЗ, и модели, основанные на многомерных массивах, отличные от ADM.
        
        \smallskip
        
        Разработанные отображения ЯОД и ЯМД были частично реализованы на 
языке ATL (ATLAS\linebreak Transformation Language)~\cite{18-stu}. ATL-программы 
пред\-став\-ля\-ют собой де\-кла\-ра\-тив\-но-им\-пе\-ра\-тив\-ные трансформации, 
реализующие отображения произвольных исходных моделей уровня М1 
(согласно классификации MOF~\cite{19-stu}), конформных исходной 
метамодели уровня М2, в целевые модели уровня М1, конформные целевой 
метамодели уровня М2. Модели уровня М1 являются схемами, 
представленными в канонической модели данных или модели ADM; модели 
уровня М2 есть описание абстрактного синтаксиса канонической модели или 
модели ADM. В~качестве метамодели уровня М3, которой конформны 
метамодели уровня M2, рассматривается модель Ecore~\cite{20-stu}. Cинтаксис 
ЯОД и ЯМД ядра канонической информационной модели (языка СИНТЕЗ) и 
модели ADM был представлен в метамодели Ecore. 
        
        Было осуществлено построение ATL-транс\-фор\-ма\-ций, реализующих 
отображения подмножества ЯОД модели ADM в ЯОД канонической модели и 
подмножества ЯМД канонической модели в ЯМД модели ADM. Подмножества 
ЯМД определялись конструкциями ЯОД и ЯМД канонической модели, 
поддерживаемыми в настоящее время в исполнительной среде предметных 
посредников. Поддерживаемый язык запросов канонической модели включает 
правила, в голове которых могут быть пре\-ди\-ка\-ты-кол\-лек\-ции, а в теле~--- 
конъюнкция пре\-ди\-ка\-тов-кол\-лек\-ций, условий соединения коллекций и 
других условий на значения атрибутов типов экземпляров коллекций. 
Условием соединения может быть только равенство атрибутов. 
Поддерживаются основные арифметические предикаты и функции, а также 
термы~--- вызовы функций. 

\section{Сохранение информации и~семантики операций языка манипулирования данными 
при~отображении}
        
        Метод доказательства сохранения информации и семантики операций 
при отображении моделей данных~\cite{21-stu} основывается на представлении 
семантики моделей в формальном языке спецификаций AMN~\cite{16-stu}. 
        
        Язык AMN представляет собой тео\-ре\-ти\-ко-мо\-дель\-ную нотацию, 
основанную на теории множеств и типизированном языке первого порядка. 
Спецификации AMN называются абстрактными машинами. Язык AMN позволяет 
интегрированно рас\-смат\-ри\-вать спецификацию пространства состояний и 
поведения машины (определенного операциями на состояниях). В~AMN 
формализуется специальное отношение между спецификациями~--- 
{уточнение}. Неформально спецификация~$B$ уточняет 
спецификацию~$A$, если пользователь может использовать $B$ вместо~$A$, 
не замечая факта замены~$A$ на~$B$. 
{\looseness=1

}
        
        Идея метода заключается в следующем. Рассмотрим исходную 
модель~$S$ и целевую модель~$T$. Построим отображение~$\theta$ 
модели~$S$ в модель~$T$ (подобно изложенному в предыдущем разделе). 
Выразим семантику моделей в виде абстрактных машин AMN, построив при 
этом машины $M_S$ и $M_T$ соответственно. При этом структуры данных 
моделей~--- классы, массивы~--- представляются переменными машин, 
различные свойства структур данных представляются инвариантами машин, 
характерные операции моделей данных представляются операциями машин. 
Рассматриваемые операции исходной и целевой модели должны быть связаны 
отображением ЯМД. Отображение ЯОД представляется в виде специального 
\textit{склеивающего инварианта}~--- замкнутой формулы, связывающей 
состояния машин~$M_S$ и~$M_T$.
        
        Будем считать отображение~$\theta$ сохраняющим инфор\-ма\-цию и 
семантику операций, если машина~$M_S$, соответствующая исходной модели, 
уточняет машину~$M_T$, соответствующую целевой модели. Уточнение 
доказывается интерактивно при помощи специальных программных 
средств~\cite{22-stu}.
        
        \smallskip
        
        В качестве иллюстрации основных принципов выражения семантики 
моделей ADM и СИНТЕЗ в AMN рассмотрим частичные 
        AMN-спе\-ци\-фи\-ка\-ции, соответствующие данным моделям.
        
        Cпецификация, выражающая семантику объектной модели языка 
СИНТЕЗ, представляется в языке AMN конструкцией {\sf REFINEMENT}:
\begin{verbatim}
REFINEMENT ObjectDM
\end{verbatim}

        Переменные, составляющие пространство состояний объектной модели, 
объявлены в разделе {\sf ABSTRACT\_VARIABLES} машины {\sf ObjectDM} и 
типизируются в разделе {\sf INVARIANT}:
\begin{verbatim}
ABSTRACT_VARIABLES
    typeNames, classNames, attributeNames,
    instanceType, typeAttributes, 
      attributeType,
    unique, obligatory,
    intAttributeLowerBound, 
      intAttributeHigherBound,
    objectIDs, objectType, objectsOfClass,
    integerAttributeValue,
    adtAttributeValue
INVARIANT ...
\end{verbatim}

        Раздел {\sf INVARIANT} содержит формулу, которая состоит из предикатов, 
типизирующих переменные состояния и налагающих различные совместные 
ограничения на переменные. Предикаты соединяются операцией конъюнкции.
        
        Так, имена типов и классов представлены переменными {\sf typeNames} и 
{\sf classNames}, тип которых~--- подмножество множества строк 
({\sf STRING\_Type}):
        \begin{verbatim}
typeNames: POW(STRING_Type) &
classNames: POW(STRING_Type)
\end{verbatim}
        
        \noindent
        Здесь {\sf POW}~--- конструктор множества подмножеств.
        
        Атрибуты (переменная {\sf attributeNames}) пред\-став\-ле\-ны частичной 
функцией (знак <<\verb +-> >>), ставящей в соответствие уникальному идентификатору 
атрибута (натуральному числу из множества {\sf NAT}) имя атрибута (строку):
        \begin{verbatim}
attributeNames: NAT +-> STRING_Type
\end{verbatim}

        Типы экземпляров классов (переменная\linebreak {\sf instanceType}) представлены 
тотальной функцией (знак \verb -> ) из множества имен классов в 
множество имен типов:
        \begin{verbatim}
instanceType: classNames --> typeNames
\end{verbatim}

        Принадлежность атрибутов типам (переменная {\sf typeAttributes}) 
выражена тотальной функцией из множества имен типов в множество 
подмножеств атрибутов:
        \begin{verbatim}
typeAttributes: 
  typeNames --> POW(dom(attributeNames))
\end{verbatim}
        Здесь {\sf dom}~--- операция, возвращающая область определения 
функции, а {\sf dom(attributeNames)}~--- множество имен атрибутов.
        
        Типы значений атрибутов (переменная\linebreak {\sf attributeType}) представлены 
функцией из множества атрибутов в множество идентификаторов встроенных 
типов данных {\sf BuiltInTypes}:
        \begin{verbatim}
attributeType: 
  dom(attributeNames) +-> BuiltInTypes
\end{verbatim}

        Множества уникальных атрибутов типов {\sf unique}\linebreak и множества 
определенных атрибутов типов\linebreak {\sf obligatory} представлены тотальными 
функциями из множества имен типов в множество подмножеств атрибутов:
\begin{verbatim}
unique: 
  typeNames --> POW(dom(attributeNames))&
obligatory: 
  typeNames --> POW(dom(attributeNames))
\end{verbatim}

        Нижние границы целочисленных атрибутов (переменная 
{\sf intAttributeLowerBound}) представлены час\-тич\-ной функцией из множества 
атрибутов в множество целых чисел:
\begin{verbatim}
intAttributeLowerBound: 
  dom(attributeNames) +-> INT
\end{verbatim}

        Аналогично представляются верхние границы.
        
        Идентификаторы объектов (переменная\linebreak {\sf objectIDs}) представлены 
подмножеством натуральных чисел:
        \begin{verbatim}
objectIDs: POW(NAT)
\end{verbatim}

        Типы объектов (переменная {\sf objectType}) представлены тотальной 
функцией из множества объектных идентификаторов в множество имен типов:
\begin{verbatim}
objectType: objectIDs --> typeNames
\end{verbatim}

        Состав классов (переменная {\sf objectsOfClass}) представлен тотальной 
функцией из множества имен классов в множество подмножеств 
идентификаторов объектов:
        \begin{verbatim}
objectsOfClass: 
  classNames --> POW(objectIDs)
\end{verbatim}
        
        Значения атрибутов объектов (переменные\linebreak {\sf integerAttributeValue}, 
{\sf adtAttributeValue} и~др.)\ пред\-став\-ле\-ны функциями из множества атрибутов\linebreak 
в функции из множества идентификаторов объектов в множества значений 
атрибутов. Для простоты рассмотрены лишь функции для целочисленных 
атрибутов и атрибутов со значениями АТД\linebreak (абстрактного типа данных) (объектами):
        \begin{verbatim}
integerAttributeValue: 
 dom(attributeNames) +-> (objectIDs+->INT)& 
adtAttributeValue: 
 dom(attributeNames) +-> (objectIDs+->NAT)
\end{verbatim}
        
        Дополнительные необходимые свойства переменных состояния 
представлены конъюнктивными компонентами инварианта. Так, каждый 
атрибут является атрибутом некоторого типа:
        \begin{verbatim}
        
UNION(tt).(tt:typeNames|typeAttributes(tt))=
    dom(attributeNames)
\end{verbatim}
        Здесь {\sf UNION}~--- родовая операция объединения, в данном случае 
объединяются множества атрибутов {\sf typeAttributes(tt)} по всем именам 
типов~{\sf tt} из множества {\sf typeNames}. 
        
        Никакой атрибут не принадлежит двум типам одновременно:
        \begin{verbatim}
!(t1, t2).(t1: typeNames & t2: typeNames =>
  (typeAttributes(t1) /\ typeAttributes(t2) 
    = {}))
\end{verbatim}
   Здесь <<\verb ! >>~--- знак квантора всеобщности, <<\verb => >>~--- логическая 
импликация, <<\verb /\ >>~--- символ пересечения множеств, <<\verb {} >>~--- пустое 
множество.
        
        Уникальные и определенные атрибуты типа выбираются из множества 
атрибутов типа:
        \begin{verbatim}
!(tt).(tt: dom(unique) => unique(tt) <: 
typeAttributes(tt)) &
!(tt).(tt: dom(obligatory) => 
    obligatory(tt) <: typeAttributes(tt))
\end{verbatim}
        Здесь <<\verb <: >>~--- символ отношения мно\-жес\-во--под\-мно\-жество.
        
        Нижние и верхние границы могут быть определены только для 
целочисленных атрибутов:
        \begin{verbatim}
!(attr).(attr: dom(intAttributeLowerBound)=> 
    attributeType(attr) = Integer) 
\end{verbatim}

        Тип объекта~--- экземпляра класса есть тип экземпляров этого класса:
        \begin{verbatim}
!(cc).(cc: classNames => 
    !(oo).(oo: objectsOfClass(cc) => 
       objectType(oo) = instanceType(cc))) 
\end{verbatim}

        Для каждого атрибута определена ровно одна функция значений:
        \begin{verbatim}
dom(adtAttributeValue) /\ 
  dom(integerAttributeValue) = {} &
dom(adtAttributeValue) \/ 
  dom(integerAttributeValue) = 
    dom(attributeNames)
\end{verbatim}
   Здесь <<\verb \/ >>~--- символ объединения множеств.
        
        Для любого объекта и любого определенного атрибута типа этого 
объекта функция значений атрибута определена на объекте:
        \begin{verbatim}
!(oo, aa).(oo: dom(objectType) & 
  aa: typeAttributes(objectType(oo)) & 
  aa: obligatory(objectType(oo)) =>
      (attributeType(aa) = Integer => 
       oo: dom(integerAttributeValue(aa))) &
      (attributeType(aa) = ADT =>
       oo: dom(adtAttributeValue(aa)))) 
\end{verbatim}

        Для любого объекта и любого целочисленного атрибута типа объекта, 
определенного на объекте и для которого определена нижняя (верхняя) 
граница, значение атрибута не меньше (не больше) нижней (верхней) границы:
        \begin{verbatim}
!(oo, aa).(oo: objectIDs & 
    aa: typeAttributes(objectType(oo)) &
    oo: dom(integerAttributeValue(aa) => 
    (aa: dom(intAttributeLowerBound) =>
        (integerAttributeValue(aa)(oo) >= 
         intAttributeLowerBound(aa))) ) 
\end{verbatim}

        Объект однозначно идентифицируется набором своих уникальных 
атрибутов:
        \begin{verbatim}
!(oo1, oo2).(oo1: objectIDs & 
  oo2: objectIDs &
    objectType(oo1) = objectType(oo2) & 
    unique(objectType(oo1)) /= {} &
    !(aa).(aa: unique(objectType(oo1)) => 
      (attributeType(aa) = Integer =>
        integerAttributeValue(aa)(oo1) =
         integerAttributeValue(aa)(oo2)) &
      (attributeType(aa) = ADT =>
         adtAttributeValue(aa)(oo1) =
          adtAttributeValue(aa)(oo2)) ) => 
    oo1 = oo2 )
\end{verbatim}

        Из всего ЯМД в спецификации рассмотрена единственная операция 
{\sf update} обновления значений атрибута в объектах класса:
        \begin{verbatim}
OPERATIONS
update(cls, attr, exp, cond) =
PRE cls: classNames & 
  attr: typeAttributes(instanceType(cls)) &
  attributeType(attr) = Integer &
  exp: INT --> INT & cond: NAT --> BOOL
THEN
 integerAttributeValue := 
 integerAttributeValue <+ 
 { xx | xx: (NAT*(NAT<->INT)) &
  #(oo, val).( oo: objectsOfClass(cls) & 
  val: INT &
    xx = attr |-> ({oo |-> val}) & 
  (cond(integerAttributeValue(attr)(oo)) 
  = TRUE =>
  val=exp(integerAttributeValue(attr)(oo)))&
  (cond(integerAttributeValue(attr)(oo)) 
  = FALSE => 
  val=integerAttributeValue(attr)(oo)))}
END
\end{verbatim}

        Параметрами операции являются имя класса {\sf cls}, имя целочисленного 
атрибута {\sf attr} типа экземпляров класса, функция {\sf exp}, отвечающая за 
преобразование атрибута, и функция {\sf cond}, отвечающая условию на значение 
атрибута. Пусть {\sf o}~--- некоторый объект класса {\sf cls}, для которого определено 
значение атрибута {\sf attr}, и это значение есть~{\sf v}. Тогда операция {\sf update} 
изменяет значение атрибута на {\sf exp(v)} в случае, если выражение {\sf cond(v)} 
принимает значение <<истина>>, и оставляет значение атрибута без изменений в 
противном случае. Очевидно, такая операция {\sf update} есть обобщение примера 
обновления, рассмотренного в подразд.~2.2. Действительно, для рассмотренного 
примера {\sf cls} есть {\sf source}, {\sf attr} есть {\sf val}, 
{\sf exp(v)}\;=\;-\,{\sf v}, {\sf cond(v)}\;=\;{\sf abs(v)}.
        
        Заметим, что в рассмотренной спецификации для простоты не 
рассмотрены некоторые черты объектной модели, например отношения 
        тип--под\-тип и класс--под\-класс.
        
        \smallskip
        
        Спецификация, выражающая семантику модели ADM, представляется в 
языке AMN конструкцией
        \begin{verbatim}
REFINEMENT ArrayDM
\end{verbatim}

        Переменные, составляющие пространство состояний объектной модели, 
объявлены в разделе {\sf ABSTRACT\_VARIABLES} машины {\sf ArrayDM}:
        \begin{verbatim}
ABSTRACT_VARIABLES
    arrayNames, dimensionNames, 
    cellAttributeNames,
    arrayDimensions, arrayCellAttributes,    
    cellAtrributeType, nullable, 
    dimLowerBound, dimHigherBound,
    cells, dimensionValue, 
    integerCellAttributeValue
\end{verbatim}

        Имена массивов представлены переменной\linebreak 
{\sf arrayNames}; имена измерений~--- переменной\linebreak 
{\sf  dimensionNames}; имена атрибутов ячеек массива~--- переменной 
\mbox{{\sf cellAttributeNames}}; принадлежность измерений массивам~--- переменной 
\mbox{{\sf arrayDimensions}}; принадлежность атрибутов ячеек~--- переменной 
\mbox{{\sf arrayCellAttributes}}; 
тип атрибута ячейки~--- переменной \mbox{{\sf cellAtrributeType}}; 
атрибуты ячеек массивов, которые могут принимать неопределенные 
значения,~--- переменной \mbox{{\sf nullable}}; верхние (нижние) границы измерений~--- 
переменной \mbox{{\sf dimLowerBound}} (\mbox{{\sf dimHigherBound}}); множества 
идентификаторов ячеек массивов~--- переменной 
\mbox{{\sf cells}}, значения измерений в 
ячейках~--- переменной \mbox{{\sf dimensionValue}}; значения атрибутов ячеек~--- 
переменной \mbox{{\sf integerCellAttributeValue}}. Переменные типизируются в разделе 
\mbox{{\sf INVARIANT}} при помощи частичных и тотальных функций аналогично 
переменным, использующимся для придания семантики объектной модели:
        \begin{verbatim}
INVARIANT
   arrayNames: POW(STRING_Type) &
   dimensionNames: NAT +-> STRING_Type &
   cellAttributeNames: NAT +-> STRING_Type &
   arrayDimensions: arrayNames --> 
   POW(dom(dimensionNames)) &
   arrayCellAttributes: arrayNames --> 
     POW(dom(cellAttributeNames)) &
   cellAtrributeType: 
     dom(cellAttributeNames) --> 
       BuiltInTypes &
   nullable: 
     dom(cellAttributeNames) --> BOOL &
   dimLowerBound: 
     dom(dimensionNames) --> INT &
   dimHigherBound: 
     dom(dimensionNames) +-> INT &
   cells: arrayNames --> POW(NAT) & 
   dimensionValue: 
     NAT*dom(dimensionNames) +-> INT  &
   integerCellAttributeValue: 
     NAT*dom(cellAttributeNames) +-> INT &
\end{verbatim}
        Здесь <<\verb * >>~--- знак декартова произведения множеств.
        
        Дополнительные необходимые свойства переменных состояния 
представлены конъюнктивными компонентами инварианта. Так, любая ячейка 
любого массива однозначно идентифицируется набором значений измерений:
        \begin{verbatim}
!(arr, cell1, cell2).(arr: arrayNames & 
  cell1: cells(arr) &  cell2: cells(arr) &
  !(dim).(dim: arrayDimensions(arr) =>
    dimensionValue(cell1, dim) = 
    dimensionValue(cell2, dim)) =>
    cell1 = cell2)
        \end{verbatim}
        
                \vspace*{-6pt}
        
        Для любой ячейки любого массива определены значения всех измерений 
и значение по крайней мере одного атрибута:
        \begin{verbatim}
!(arr, cell).(arr: arrayNames & 
 cell: cells(arr) =>
  !(dim).(dim: arrayDimensions(arr) => 
   (cell |-> dim): dom(dimensionValue)) &
   #(attr).(attr: arrayCellAttributes(arr) & 
    cellAtrributeType(attr) = Integer & 
    (cell, attr): 
      dom(integerCellAttributeValue)) )
        \end{verbatim}
        
        \vspace*{-6pt}
        
        Аналогично объектной модели рассмотрена единственная операция 
ЯМД~--- операция об\-нов\-ле\-ния {\sf update}:
        \begin{verbatim}
OPERATIONS
update(cls, attr, exp, cond) =
PRE cls: arrayNames & 
 attr: arrayCellAttributes(cls) &
  cellAtrributeType(attr) = Integer &
  exp: INT --> INT & cond: NAT --> BOOL
THEN
  integerCellAttributeValue := 
  integerCellAttributeValue <+
  { yy | yy: (NAT*NAT)*INT &
    #(cell, val).(cell: cells(cls) & 
     val: INT & 
    yy = ((cell |-> attr)|-> val) &
    (cond(integerCellAttributeValue(cell, 
     attr)) = TRUE =>
      val = 
       exp(integerCellAttributeValue(cell,
        attr))) &
      (cond(integerCellAttributeValue(cell, 
       attr))= FALSE  =>
    val = 
     integerCellAttributeValue(cell,attr)))}
END   
END
        \end{verbatim}
        
                \vspace*{-6pt}
        
        Сигнатура операции совпадает с сигнатурой операции объектной 
модели. Семантика операции также аналогична: значение~{\sf v} атрибута {\sf attr} 
массива {\sf cls} заменяется на {\sf exp(v)}, если значение {\sf cond(v)} есть 
<<истина>>, и не изменяется в противном случае. 
        
        Заметим, что в данной спецификации для прос\-то\-ты не рассмотрены 
некоторые черты ADM, например нецелочисленные измерения.
        
        \smallskip
        
        Для формального доказательства того, что машина {\sf ArrayDM} уточняет 
машину {\sf ObjectDM}, необходимо построить {инвариант уточнения}, 
связы\-вающий переменные машин, и добавить его к\linebreak инварианту уточняющей 
машины. 
        
        Инвариант формализует принципы отображения ЯОД, изложенные в 
подразд.~2.1, и объединяет их в одну конъюнкцию.
        
        Так, множество имен массивов совпадает с множеством имен классов:
        \begin{verbatim}
classNames = arrayNames
\end{verbatim}

%                \vspace*{-6pt}
        
        Множество идентификаторов и имен измерений и атрибутов ячеек 
совпадает с множеством идентификаторов и имен атрибутов типов экземпляров 
классов:
        \begin{verbatim}
attributeNames = 
  dimensionNames \/ cellAttributeNames
\end{verbatim}

%                \vspace*{-6pt}

        Любому измерению любого массива соответствует атрибут типа 
экземпляра класса, соответствующего этому массиву:
        \begin{verbatim}
!(arr, dim).(arr: arrayNames & 
  dim: arrayDimensions(arr) =>
    #(attr).(attr: 
     typeAttributes(instanceType(arr)) &
          attr = dim & 
          attributeType(attr) = Integer) )s
        \end{verbatim}
        
                        \vspace*{-6pt}
        
        Любому атрибуту ячейки любого массива соответствует атрибут типа 
экземпляра класса, соответствующего этому массиву, и типы атрибутов 
совпадают:
        \begin{verbatim}
!(arr, cattr).(arr: arrayNames & 
   cattr: arrayCellAttributes(arr) =>
    #(attr).(attr: 
      typeAttributes(instanceType(arr)) & 
         attr = cattr & 
         attributeType(attr) = 
           attributeType(cattr)))
        \end{verbatim}
        
                        \vspace*{-9pt}
        
        Атрибут ячейки массива, который может принимать неопределенные 
значения, соответствует определенному ({\sf obligatory}) атрибуту типа:
        \begin{verbatim}
!(arr, cattr).(arr: arrayNames & 
  cattr /: dom(nullable) &
    cattr: arrayCellAttributes(arr) => 
    cattr: obligatory(instanceType(arr)) )
        \end{verbatim}
        
\vspace*{-9pt}

           Здесь знак <<\verb /: >> обозначает отношение непринадлежности элемента 
множеству.
        
        Измерения соответствуют уникальным атрибутам типов:
        \begin{verbatim}
!(arr, dim).(arr: arrayNames & 
    dim: arrayDimensions(arr) => 
      dim: unique(instanceType(arr)) )
        \end{verbatim}
        
                        \vspace*{-6pt}
        
        Верхние (нижние) границы измерений равны верхним (нижним) 
границам соответствующих атрибутов типов:
        \begin{verbatim}
!(dim).(dim: dom(dimLowerBound) =>
    dim: dom(intAttributeLowerBound) & 
    dimLowerBound(dim) = 
      intAttributeLowerBound(dim))
        \end{verbatim}
        
                        \vspace*{-6pt}
        
        Непустые ячейки массивов соответствуют объектам классов:
        \begin{verbatim}
cells = objectsOfClass
\end{verbatim}

%                \vspace*{-6pt}

        Для любой ячейки значения ее измерений и определенных атрибутов 
совпадают со значениями соответствующих атрибутов объекта, 
соответствующего ячейке:
        \begin{verbatim}
!(cell, dim).(cell: NAT & dim: NAT & 
  (cell |-> dim): dom(dimensionValue) =>
  cell: dom(integerAttributeValue(dim)) &
  dimensionValue(cell, dim) = 
    integerAttributeValue(dim)(cell)) &
!(cell, cattr).(cell: NAT & cattr: NAT & 
   (cell |-> cattr): 
   dom(integerCellAttributeValue) =>
   cell: dom(integerAttributeValue(cattr)) &
   integerCellAttributeValue(cell, cattr) =
     integerAttributeValue(cattr)(cell) )
        \end{verbatim}
        
                        \vspace*{-6pt}
        
        Для указания того, что машина {\sf ArrayDM} уточняет машину 
{\sf ObjectDM}, в машину {\sf ArrayDM} была добавлена директива
        \begin{verbatim}
REFINES ObjectDM
\end{verbatim}

%                \vspace*{-6pt}

        Спецификации {\sf ObjectDM} и {\sf ArrayDM} вместе с инвариантом 
уточнения были загружены в инструментальное средство 
        Atelier~B~\cite{22-stu}. Автоматически были сгенерированы теоремы, 
выражающие уточнение спецификаций. В~частности, для операции {\sf update} 
были сгенерированы 10~тео\-рем. Три из них были доказаны автоматически, 
для доказательства остальных необходимо применять интерактивные средства 
доказательства.

\vspace*{-9pt}
  
\section{Родственные исследования и~направления дальнейшей 
работы}

\vspace*{-2pt}

        Родственными данной работе следует считать исследования, связанные с 
отображением моделей, основанных на многомерных массивах, в реляционную 
модель данных. Обычно они нацелены на реализацию многомерных массивов 
при помощи реляционных СУБД. Такие работы начались одновременно с 
исследованиями моделей, основанных на многомерных массивах~\cite{5-stu}, и 
продолжаются в настоящее время~\cite{23-stu}.
        
        Основные особенности данной работы состоят в следующем. 
В~качестве исходной модели при отображении используется специфическая 
модель, основанная на многомерных массивах СУБД SciDB, язык которой 
представляет собой комбинацию декларативного SQL-по\-доб\-но\-го языка и 
функционального языка, включающего специфические\linebreak операции над 
многомерными массивами. В~качестве целевой модели используется объектная 
модель с Datalog-по\-доб\-ным языком запросов (программ)~--- язык СИНТЕЗ. 
Для отображения\linebreak обеспечивается формальное доказательство сохранения 
информации и семантики операций ЯМД.
        
        Отметим, что результаты работы могут быть с легкостью обобщены и 
использованы при интеграции в системах, использующих каноническую 
модель, отличную от языка СИНТЕЗ, например другую объектную (ODMG) 
или объект\-но-ре\-ля\-ци\-он\-ную модель (SQL:2003). Результаты также могут 
быть использованы для интеграции ресурсов, представленных в модели, 
основанной на многомерных массивах, но отличной от ADM.
        
        Некоторые вопросы отображения требуют дальнейших исследований. 
Например, следует ли иметь в канонической модели при интеграции 
        масс\-сив-ори\-ен\-ти\-ро\-ван\-ных моделей данных операции, 
связанные с размером порции (chunk size) данных в БД~\cite{9-stu}?
        
        Дальнейшую работу можно разбить на два этапа:
        \begin{enumerate}[(1)]
\item расширение инструментальных средств поддержки предметных 
посредников для виртуальной интеграции SciDB-ресурсов: 
\begin{itemize}
\item[(а)] расширение средств регистрации ресурсов в посреднике~\cite{10-stu} 
трансформацией ЯОД\ ADM в каноническую модель; 
\item[(б)] создание 
SciDB-адап\-те\-ра~--- специального ПО, связывающего исполнительную 
среду посредников с SciDB-ресурсами (составной частью адаптера является 
разработанная трансформация ЯМД);
\end{itemize}
\item применение технологии предметных посредников для решения 
научных задач в некоторой предметной области над множеством\linebreak 
неоднородных ресурсов, включающим SciDB-ре\-сурсы.
\end{enumerate}

\bigskip
        Автор выражает благодарность Л.\,А.~Калиниченко, П.\,Е.~Велихову и 
А.\,Е.~Вовченко за полезные замечания, высказанные в ходе обсуждения 
данной работы на семинарах ИПИ РАН.

\vspace*{-6pt}

{\small\frenchspacing
{%\baselineskip=10.8pt
\addcontentsline{toc}{section}{Литература}
\begin{thebibliography}{99}

\vspace*{-2pt}

\bibitem{1-stu} %1
Challenges and opportunities with big data~// A~community white paper developed 
by leading researchers across the United States, 2012. {\sf http://cra.org/ccc/docs/ init/bigdatawhitepaper.pdf}. 

\bibitem{1-2-stu} %2
\Au{Abrial J.-R.} The B-Book: Assigning programs to 
meanings.~--- Cambridge: Cambridge University Press, 1996. 

\bibitem{2-stu} %3
\Au{Vassiliadis P., Sellis T.\,K.} A~survey of logical models for OLAP databases~// SIGMOD 
Record, 1999. Vol.~28. No.\,4. P.~64--69. 

\bibitem{3-stu}
\Au{Pedersen T.\,B., Jensen C.\,S.} Multidimensional database technology~// IEEE Computer, 
2001. Vol.~34. No.\,12. P.~40--46. 

\bibitem{4-stu} %5
\Au{Libkin L., Machlin R., Wong~L.} A~query language for multidimensional arrays: Design, 
implementation, and optimization techniques.~--- SIGMOD, 1996. P.~228--239. 
\bibitem{5-stu} %6
\Au{Baumann P.} A~database array algebra for spatio-temporal data and beyond~// Next 
generation information technologies and systems. Lectures notes in computer science ser.
Springer Verlag KG, 1999. Vol.~1649. P.~76--93.
\bibitem{6-stu} %7
Overview of SciDB: Large scale array storage, processing and analysis. The SciDB development 
team.~--- SIGMOD, 2010. 
\bibitem{7-stu}
Large synoptic survey telescope. {\sf http://www.lsst.org}. 
\bibitem{8-stu}
\Au{Becla J., Lim K.-T.} Report from the First Workshop on Extremely Large Databases~// Data 
Sci.~J., 2008. Vol.~7.
\bibitem{9-stu}
SciDB User's Guide. Version~12.3, 2012. {\sf http:// www.scidb.org}.
\bibitem{10-stu}
\Au{Kalinichenko L.\,A., Briukhov D.\,O., Martynov~D.\,O., Skvortsov~N.\,A., Stupnikov~S.\,A.} 
Mediation framework for enterprise information system infrastructures~// Volume Databases and 
Information Systems Integration: 9th Conference (International) on Enterprise Information 
Systems (ICEIS 2007) Proceedings ~--- Funchal, 2007. P.~246--251.
\bibitem{11-stu}
\Au{Захаров В.\,Н., Калиниченко Л.\,А., Соколов~И.\,А., Ступников~С.\,А.} Конструирование 
канонических информационных моделей для интегрированных информационных 
сис\-тем~// Информатика и её применения, 2007. Т.~1. Вып.~2. C.~15--38. 
\bibitem{12-stu}
\Au{Kalinichenko L.\,A., Stupnikov S.\,A.} Heterogeneous information model unification as a 
prerequisite to resource schema mapping~// Information Systems: People, Organizations, 
Institutions, and Technologies: 5th Conference of the Italian Chapter of Association for 
Information Systems itAIS Proceedings.~--- Berlin--Heidelberg: Springer Physica Verlag, 2010. 
P.~373--380. 
\bibitem{13-stu}
\Au{Kalinichenko L.\,A., Stupnikov S.\,A., Martynov~D.\,O.} SYNTHESIS: A~language for 
canonical information modeling and mediator definition for problem solving in heterogeneous 
information resource environments.~--- Moscow: IPI RAN, 2007. 171~p. 
\bibitem{14-stu}
\Au{Брюхов Д.\,О., Вовченко А.\,Е., Захаров~В.\,Н., Желенкова~О.\,П., Калиниченко~Л.\,А., 
Мартынов~Д.\,О., Скворцов~Н.\,А., Ступников~С.\,А.} Архитектура промежуточного слоя 
предметных посредников для решения \mbox{задач} над множеством интегрируемых 
неоднородных распределенных информационных ресурсов в гиб\-рид\-ной 
грид-ин\-фра\-струк\-ту\-ре виртуальных обсерваторий~// Информатика и её применения, 
2008. Т.~2. Вып.~1. С.~2--34. 

\bibitem{15-stu} %16
\Au{Kersten M.\,L., Zhang~Y., Ivanova~M., Nes~N.} SciQL, a query language for science 
applications~// EDBT/ICDT~--- Workshop on Array Databases 2011 Proceedings.~--- Uppsala, 
Sweden, 2011. P.~1--12.

\bibitem{16-stu} %17
\Au{Abrial J.-R.} The B-Book: Assigning programs to meanings.~--- Cambridge: Cambridge 
University Press, 1996.
\bibitem{17-stu} %18
Astronomy in ArrayDB. 
{\sf http://trac.scidb.org/\linebreak raw-attachment/wiki/UseCases/Astronomy\%20in\%20\linebreak
ArrayDB.pdf }
\bibitem{18-stu} %19
ATL Project. {\sf http://www.eclipse.org/m2m/atl}.
\bibitem{19-stu} %20
\Au{Budinsky F., Steinberg D., Ellersick~R., Grose~T.}
Eclipse modeling framework. Ch.~5: Ecore modeling concepts.~--- Addison Wesley 
Professional, 2004.
\bibitem{20-stu} %21
Meta Object Facility (MOF) 2.0 Core Specification, 2003. 
{\sf http://www.omg.org/cgi-bin/apps/doc?ptc/\linebreak 03-10-04.pdf}. 
\bibitem{21-stu} %22
\Au{Kalinichenko L.\,A.} Method for data models integration in the common paradigm~//  1st 
East-European Symposium on Advances in Databases and Information Systems \mbox{ADBIS'97} 
Proceedings.~--- St.-Petersburg: Nevsky Dialect, 1997. Vol.~1: Regular papers. P.~275--284.
\bibitem{22-stu}
Atelier~B: The industrial tool to efficiently deploy the B Method. 
{\sf http://www.atelierb.eu/index-en.php}.

\label{end\stat}

\bibitem{23-stu} %24
\Au{Van Ballegooij A.} RAM: Array database management through relational mapping~// SIKS 
Dissertation ser. No.\,2009-25. {\sf http://oai.cwi.nl/oai/asset/14074/ 14074D.pdf}.
         
\end{thebibliography}
} }

\end{multicols} %6

\def\stat{shnurkov}

\def\tit{АНАЛИТИЧЕСКОЕ РЕШЕНИЕ ЗАДАЧИ ОПТИМАЛЬНОГО УПРАВЛЕНИЯ ПОЛУМАРКОВСКИМ ПРОЦЕССОМ\\ 
С~КОНЕЧНЫМ МНОЖЕСТВОМ СОСТОЯНИЙ$^*$}

\def\titkol{Аналитическое решение задачи оптимального управления полумарковским 
процессом} %с~конечным множеством состояний}

\def\aut{П.\,В.~Шнурков$^1$, А.\,К.~Горшенин$^2$, В.\,В.~Белоусов$^3$}

\def\autkol{П.\,В.~Шнурков, А.\,К.~Горшенин, В.\,В.~Белоусов}

\titel{\tit}{\aut}{\autkol}{\titkol}

\index{Шнурков П.\,В.}
\index{Горшенин А.\,К.}
\index{Белоусов В.\,В.}
\index{Shnurkov P.\,V.}
\index{Gorshenin A.\,K.}
\index{Belousov V.\,V.}


{\renewcommand{\thefootnote}{\fnsymbol{footnote}} \footnotetext[1]
{Работа выполнена при частичной поддержке РФФИ (проект 15-07-05316).}}


\renewcommand{\thefootnote}{\arabic{footnote}}
\footnotetext[1]{Национальный исследовательский университет <<Высшая школа экономики>>, 
\mbox{pshnurkov@hse.ru}}
\footnotetext[2]{Институт проблем информатики Федерального исследовательского центра <<Информатика 
и~управ\-ле\-ние>> Российской академии наук, \mbox{agorshenin@frccsc.ru}}
\footnotetext[3]{Институт проблем информатики Федерального исследовательского центра <<Информатика 
и~управление>> Российской академии наук, \mbox{vbelousov@ipiran.ru}}

%\vspace*{-6pt}

\Abst{Настоящее исследование посвящено теоретическому обоснованию нового метода 
нахождения оптимальной стратегии управления полумарковским процессом с~конечным 
множеством состояний. Рассматриваются марковские рандомизированные стратегии 
управления, определяемые конечным набором вероятностных мер, соответствующих 
каждому состоянию. Характеристикой качества управления служит стационарный 
стоимостной показатель. Данный показатель представляет собой дроб\-но-ли\-ней\-ный 
интегральный функционал от набора вероятностных мер, задающих стратегию управления. 
Для этого функционала известны явные аналитические представления подынтегральных 
функций числителя и~знаменателя. Дальнейшие результаты основываются на новой 
усиленной и~обобщенной форме теоремы об экстремуме дроб\-но-ли\-ней\-но\-го интегрального 
функционала. Доказывается, что проблемы существования оптимальной стратегии управления 
полумарковским процессом и~ее нахождения сводятся к~задаче численного исследования 
на глобальный экстремум заданной функции от конечного числа вещественных переменных.}

\KW{оптимальное управление полумарковским процессом; стационарный стоимостной 
показатель качества управления; дроб\-но-ли\-ней\-ный интегральный функционал}

\DOI{10.14357/19922264160408} 

\vspace*{9pt}


\vskip 10pt plus 9pt minus 6pt

\thispagestyle{headings}

\begin{multicols}{2}

\label{st\stat}

\section{Введение}

Теория оптимального управления марковскими и~полумарковскими случайными 
процессами интенсивно развивается с~начала 1960-х~гг. Еще в~первых 
основополагающих исследованиях рассматривались не только проблемы существования 
оптимальных стратегий управления, но и~способы нахождения этих стратегий. 

Для решения таких проблем, имеющих алгоритмическое содержание, использовались 
открытые незадолго до этого мощные методы прикладной математики: линейное 
программирование и~динамическое программирование. Отметим, прежде всего, 
классическую работу Р.~Ховарда~\cite{1}, в~которой метод динамического 
программирования был применен для решения проблемы оптимального управления 
марковским процессом с~непрерывным временем. В~дальнейшем В.\,В.~Рыков~\cite{2} 
доказал, что для аналогичной модели управления марковским процессом с~учетом 
переоценки оптимальной стратегией также является стационарная.

Важную роль в~развитии теории управления случайными процессами сыграла работа 
В.~Джевелла~\cite{3}, в~которой были впервые рассмотрены полумарковские модели 
управления для вариантов с~переоценкой и~без переоценки. Данная работа была 
переведена на русский язык и~послужила основой для многих последующих работ 
отечественных и~зарубежных специалистов. В~частности, Б.~Фокс показал~\cite{4}, 
что оптимальной стратегией управления полумарковским процессом в~варианте без 
переоценки является стационарная; аналогичные результаты были получены Э.~Денардо 
и~для варианта с~переоценкой~\cite{5}.

Среди последующих исследований алгоритмической направленности отметим работы 
Р.~Ховарда~\cite{6}, Б.~Фокса~\cite{4}, а также С.~Осаки и~Х.~Майна~\cite{7}. 
В~этих работах для нахождения оптимальных стратегий управления полумарковскими 
процессами использовался метод линейного программирования.

В 1970~г.\ была опубликована фундаментальная монография Х.~Майна и~С.~Осаки~\cite{8}, 
переведенная на русский язык в~1977~г., в~которой были систе\-ма\-ти\-зи\-ро\-ва\-ны и~изложены 
основные результаты по теории оптимального управления марковскими и~полумарковскими 
случайными процессами. Фактически данная книга стала итогом исследований по проблемам 
стохастического управления\linebreak
 за~10~лет. Отметим, что в~этой монографии рас\-смат\-ри\-ва\-лись 
марковские и~полумарковские модели управления с~конечными множествами состояний 
и~допустимых решений, принимаемых \mbox{в~каждом} состоянии. Были получены принципиальные 
тео\-ре\-ти\-че\-ские результаты, заключающиеся в~том, что оптимальные стратегии управ\-ле\-ния 
для основных видов рас\-смат\-ри\-ва\-емых моделей с~переоценкой и~без переоценки являются 
детерминированными и~стационарными. Были разработаны и~обоснованы процедуры нахождения 
оптимальных стратегий управления. В~частности, для модели управления полумарковским 
процессом без переоценки, когда множество со\-сто\-яний образует один эргодический класс, 
а~показатель качества управления пред\-став\-ля\-ет собой стационарный средний удельный 
доход (см.~[8, гл.~5, п.~5.5]), процедура поиска оптимальной рандомизированной 
стратегии осуществлялась методом линейного программирования. Обратим особое внимание 
на данный результат, поскольку аналогичная модель управления полумарковским 
процессом будет рассмотрена в~настоящей работе.

Принципиальную роль в~развитии теории стохастического управления сыграла 
монография И.\,И.~Гихмана и~А.\,В.~Скорохода~\cite{9}. В~этой книге были впервые 
систематически изложены основы теории оптимального управления случайными процессами 
с~дискретным и~непрерывным временем, включая теорию управления процессами, которые 
описываются стохастическими дифференциальными уравнениями. Отдельно были рас\-смот\-ре\-ны 
проблемы управления марковскими процессами с~дискретным временем и~скачкообразными 
марковскими процессами с~непрерывным временем. Роли множеств состояний и~допустимых 
управ\-ле\-ний играли пространства весьма общей структуры. Для широких классов функционалов 
качества управ\-ле\-ния (так называемых эволюционных функционалов в~марковских моделях 
с~дискретным временем и~интегральных функционалов накопления в~марковских моделях 
с~непрерывным временем) были доказаны теоремы о~существовании и~формах пред\-став\-ле\-ния 
оптимальных стратегий управ\-ле\-ния. Было установлено, что для однородных марковских 
моделей оптимальные стратегии управ\-ле\-ния существуют, являются стационарными 
и~детерминированными. Иначе говоря, такие стратегии задаются детерминированными 
функциями, аргументом которых является со\-сто\-яние сис\-те\-мы в~момент принятия решения, 
и~не зависящими от самого момента принятия решения. Что же касается важного вопроса 
о~формах представления этих функций, то их можно охарактеризовать следующим образом. 
Были найдены функциональные уравнения, осложненные условием экстремума, которым 
удовле\-тво\-ря\-ют упомянутые функции. По существу эти соотношения пред\-став\-ля\-ют собой 
уравнения Беллмана для соответствующих динамических стохастических моделей.

Особо отметим, что в~монографии~\cite{9} не рас\-смат\-ри\-ва\-лись проблемы управления 
полумарковскими процессами. Однако дальнейшее развитие общей теории управления 
такими процессами шло по пути, идейно намеченному в~указанной книге.

В последующие годы развитие теории управ\-ле\-ния полумарковскими процессами 
осуществля-\linebreak лось по направлению усложнения моделей и~обобщения исходных предположений. 
Например,\linebreak в~работах~\cite{10, 11} рассмотрены управляемые по\-лумарковские процессы при 
весьма общих предположениях относительно характера пространств состояний и~управлений. 
Проблемы управления исследовались по отношению к~различным видам целевых показателей, 
обобщающих упомянутый выше стационарный показатель средней удельной прибыли. В~этих 
работах доказывается, что оптимальная стратегия управления по отношению к~каж\-до\-му из 
показателей существует и~является одной и~той же стационарной детерминированной 
стратегией, определяемой некоторой функцией, заданной на множестве со\-сто\-яний процесса. 
Об этой функции известно лишь то, что она удовлетворяет некоторому интегральному 
уравнению, которое по содержанию пред\-став\-ля\-ет собой уравнение Бел\-лма\-на для 
соответствующей задачи управ\-ления.

Среди исследований, предшествовавших настоящему, отметим работу 
В.\,А.~Каштанова~[12, гл. 13]. В этом разделе коллективной монографии~\cite{12} 
автором была рассмотрена проблема оптимального управления полумарковским 
процессом с~конечным множеством состояний и~множеством возможных решений, 
которое представляет собой произвольный интервал множества вещественных чисел. 
Модель относится к~виду моделей без переоценки, показателем качества управления 
служит стационарное значение среднего удельного дохода, определяемое аналогично 
классическим работам~\cite{3, 8}. Рандомизированное управление в~каждом состоянии 
определяется в~соответствии с~вероятностным распределением, совокупность которых 
задает\linebreak
 стратегию управления. В.\,А.~Каш\-та\-но\-вым было\linebreak сформулировано утверждение о том, 
что стацио\-нарное значение среднего удельного дохода представляет собой 
дроб\-но-ли\-ней\-ный интегральный функционал от набора вероятностных распределений, 
образующих стратегию управления. При этом\linebreak ранее~[12, гл.~10; 13] было уста\-нов\-ле\-но, 
что дроб\-но-ли\-ней\-ный функционал достигает экстремума на вырожденных распределениях. 
Отсюда естест-\linebreak венно следует, что оптимальная стратегия управ\-ле-ния является 
детерминированной и~должна\linebreak определяться точкой экстремума функции, представляющей 
собой отношение подынтегральных функций чис\-ли\-те\-ля и~знаменателя данного 
дроб\-но-ли\-ней\-но\-го функционала. Однако в~\cite{12} не были получены явные 
представления для указан-\linebreak ных функций. Кроме того, приведенный в~гл.~10 
монографии~\cite{12} вариант теоремы об экстремуме дроб\-но-ли\-ней\-но\-го 
интегрального функционала требовал проверки выполнения условия существования 
этого экстремума. Такие условия указаны не были. В~связи с~этими обстоятельствами 
использовать полученные в~\cite{12} результаты для доказательства существования 
оптимальной детерминированной стратегии управ\-ле\-ния полумарковским процессом и~для 
строгого обоснования способа нахождения такой стратегии оказалось невозможным.

Настоящее исследование посвящено теоретическому обоснованию нового метода 
нахождения\linebreak оптимальной стратегии управления полумарковским процессом с~конечным 
множеством со\-сто\-яний. Рассматриваются марковские рандомизи\-рованные стратегии 
управления, определяемые конеч\-ным набором вероятностных мер, соответствующих 
каждому состоянию. Показателем качества управления служит уже упоминавшийся 
классический  показатель~--- стационарное значение средней удельной прибыли. 
Доказано, что этот показатель представляет собой дроб\-но-ли\-ней\-ный интегральный 
функционал от набора вероятностных мер, задающих стратегию управления. При этом, 
в~отличие от~\cite{12}, получены явные аналитические представления для подынтегральных 
функций числителя и~знаменателя этого дроб\-но-ли\-ней\-но\-го\linebreak
 функционала. Дальнейшие 
результаты основываются на новой усиленной и~обобщенной форме\linebreak
 теоремы об экстремуме 
дроб\-но-ли\-ней\-но\-го интегрального функционала, впервые опубликованной 
в~работе П.\,В.~Шнуркова~\cite{14}. Согласно\linebreak
 утверж\-де\-нию этой теоремы, если 
существует глобальный экстремум так называемой основной функции дроб\-но-ли\-ней\-но\-го 
функционала, которая пред\-став\-ля\-ет собой отношение подынтегральных функций чис\-ли\-те\-ля 
и~знаменателя, то существует безусловный экстремум самого дроб\-но-ли\-ней\-но\-го 
функционала, который достигается на наборе вырожденных вероятностных распределений, 
сосредоточенных в~точке глобального экстремума. В~этом случае оптимальная стратегия 
управ\-ле\-ния существует, является стационарной и~детерминированной и~определяется точкой, 
в~которой основная\linebreak функция достигает глобального экстремума. Таким\linebreak образом, проблемы 
существования оптимальной стратегии управ\-ле\-ния полумарковским процессом и~ее 
нахождения сводятся к~задаче чис\-лен\-но\-го исследования на глобальный экстремум 
заданной функции от конечного чис\-ла вещественных переменных.

\section{Общее описание модели управления полумарковским случайным процессом}

Построим модель управления полумарковским случайным процессом, следуя общему 
подходу, принятому в~классических работах~\cite{3, 8}. Пусть $\xi(t)$~--- 
случайный полумарковский процесс с~конечным множеством состояний
$X\hm=\{1,2,\ldots, N\}$, $N\hm< \infty$. Обозначим через~$t_n$, $n=0,1,2,\ldots$, 
$t_0\hm=0$, случайные моменты изменения состояний данного процесса, 
$\theta_n\hm=t_{n+1}-t_n$, $n\hm=0,1,2,\ldots$, $\xi_n\hm=\xi(t_n)\hm=\xi(t_n+0)$, 
$n\hm=0,1,2,\ldots$ (предполагается, что траектории процесса~$\xi(t)$ 
непрерывны справа). Случайная последовательность~$\{\xi_n\}$
образует цепь Маркова, вложенную в~полумарковский процесс~$\xi(t)$.
Зададим набор измеримых пространств\linebreak $(U_1, \mathscr{B}_1), 
(U_2, \mathscr{B}_2), \ldots, (U_N, \mathscr{B}_N)$, где $U_i$~--- 
множество возможных допустимых управ\-ле\-ний, $\mathscr{B}_i$~--- $\sigma$-ал\-геб\-ра 
подмножеств множества~$U_i$, вклю\-ча\-ющая в~себя все одноточечные подмножества\linebreak  
множества~$U_i$, т.\,е.\ если $u_i \hm\in U_i$, то $\{u_i\} \hm\in \mathscr{B}_i$, 
$i\hm=1,2,\ldots, N$.
Пусть $\Gamma_i$~--- некоторое множество всевозможных вероятностных мер $\Psi_i 
\hm \in \Gamma_i$, заданных на $\sigma$-ал\-геб\-ре~$\mathscr{B}_i$, $i\hm=1,2,\ldots,N$.

Поскольку идейное содержание и~свойства вероятностных мер существенно используются 
в~данной работе, укажем на некоторые фундаментальные издания, в~которых 
изложена соответствующая тео\-рия. Понятие и~основные свойства вероятностной 
меры определены и~подробно проанализированы в~книге А.\,Н.~Ширяева~\cite[гл.~II]{15}. 
Глубокое изложение основ теории вероятностных мер имеется также в~книге 
А.\,А.~Боровкова~\cite{16}. Заметим попутно, что в~книге~\cite{16} имеются разделы, 
посвященные изложению основ теории полумарковских и~регенерирующих случайных процессов. 
Из зарубежных изданий отметим фундаментальную работу П.~Хеннекена и~А.~Тортра~\cite{17}, 
основная часть которой посвящена изложению математических основ теории вероятностей.

Введем специальное понятие вырожденной вероятностной меры, которое будет часто 
использоваться в~дальнейшем. Пусть $(U_0, \mathscr{B}_0)$~--- некоторое измеримое 
пространство, $\mathscr{B}_0$~--- $\sigma$-ал\-геб\-ра подмножеств множества~$U_0$, 
включающая в~себя все одноточечные подмножества этого множества.

\medskip

\noindent
\textbf{Определение 1.}\ Вероятностная мера~$\Psi^*$, заданная 
на $\sigma$-ал\-геб\-рe~$\mathscr{B}_0$, называется вырожденной, если существует 
такой элемент $u^* \hm\in U_0$, для которого выполняются условия $\Psi^*(\{u^*\})\hm=
1$, $\Psi^*(U_0 \setminus \{u^*\})\hm=0$, где $\{u^*\}=u^*$~--- 
множество, состоящее из единственной точки $u^* \hm\in U_0$. Соответствующая 
точка $u^* \hm\in U_0$ будет называться точкой сосредоточения вырожденной 
вероятностной меры~$\Psi^*$.
Таким образом, всякая вырожденная вероятностная мера~$\Psi^*$ определяется 
своей точкой сосредоточения~$u^*$. В~дальнейшем будем использовать 
обозначение~$\Psi_{u^*}^{*}$, имея в~виду, что вырожденная вероятностная мера~$\Psi^*$ 
сосредоточена в~точке~$u^*$.
Отметим также, что вырожденная вероятностная мера~$\Psi_{u^*}^{*}$ соответствует 
детерминированной величине, которая принимает фиксированное значение $u\hm=u^*$ 
с~вероятностью, равной единице.

\medskip

Обозначим через $\Gamma_0$ множество всех  вероятностных мер, заданных 
на измеримом пространстве ($U_0, \mathscr{B}_0$), а через~$\Gamma_0^*$~--- 
множество всех вырожденных вероятностных мер, заданных на этом пространстве, 
$\Gamma_0^*\hm\in \Gamma_0$. Аналогичные обозначения будут использоваться 
и~в~дальнейшем. Заметим, что множество~$\Gamma_0^*$ находится во взаимно
 однозначном соответствии с~множеством точек сосредоточения вырожденных 
 вероятностных мер, т.\,е.\ с~множеством~$U_0$.

Пусть $\Gamma_i^{*}$~--- множество всех вырожденных мер, заданных на 
$\sigma$-ал\-геб\-ре~$\mathscr{B}_i$, $\Gamma_i^{*}\hm\subset \Gamma_i$.
Произвольная вероятностная мера~$\Psi_i$ описывает случайную величину, 
принимающую значения в~$U_i$, а вырожденная мера~$\Psi_i^*$, сосредоточенная 
в~точке~$u_i^*$, соответствует детерминированной величине $u_i^*\hm\in U_i$.
Предполагается, что соответствующие конструкции определены на всех измеримых 
пространствах управлений $(U_1, \mathscr{B}_1), (U_2, \mathscr{B}_2), \ldots, 
(U_N,\mathscr{B}_N)$.

Предположим, что управления случайным полумарковским процессом~$\xi(t)$ 
осуществляются в~моменты времени~$t_n,$ $n\hm=0,1,2,\ldots,$
непосредственно после изменения состояния процесса. Если\linebreak 
$\xi_n\hm=\xi(t_n)\hm=i \hm\in X$, то значение управления представляет 
собой случайную величину~$u_n$, принимающую значения в~множестве допустимых 
управ\-ле\-ний~$U_i$ и~описываемую вероятностной мерой (распределе\-ни\-ем 
вероятностей) $\Psi_i \hm\in \Gamma_i$.
Будем предполагать, что при фиксированном условии $\xi_n\hm=\xi(t_n)=i$ 
управ\-ле\-ние определяется независимо от прошлого поведения процесса~$\xi(t)$ 
и~вероятностная мера~$\Psi_i$,
описывающая стохастическое управление~$u_n$, зависит только от состояния $i\hm\in X$.
Тогда выбор управ\-ле\-ний в~моменты изменения состояний $\{t_n, n\hm=0,1,2,\ldots \}$ 
описывается набором вероятностных мер (распределений вероятностей) 
$(\Psi_1, \Psi_2,\ldots, \Psi_N)$, 
$\Psi_i \hm\in \Gamma_i$, $i\hm=1,2,\ldots,N$.
Назовем любой такой набор стратегией управ\-ле\-ния полумарковским процессом~$\xi(t)$. 
По своим свойствам такая стратегия является марковской, однородной 
и~рандомизированной.

Следуя классической монографии П.~Халмоша~\cite[гл.~VII]{18}, 
рассмотрим декартово произведение пространств $U\hm=U_1\times U_2\times \cdots\times U_N$ 
и~соответствующих $\sigma$-ал\-гебр $\mathscr{B}\hm=\mathscr{B}_1\times \mathscr{B}_2
\times \cdots \times\mathscr{B}_N$. Обозначим через $\Psi\hm=\Psi_1\times \Psi_2\times \cdots
\times \Psi_N$ вероятностную меру на~$(U,\mathscr{B})$, определяемую как 
произведение мер $\Psi_1,\Psi_2, \ldots , \Psi_N$, где $\Psi_i \hm\in \Gamma_i$, 
$i\hm=1,2,\ldots,N$. Обозначим также через~$\Gamma$ множество вероятностных мер~$\Psi$, 
заданных на~$(U,\mathscr{B})$, которые пред\-став\-ля\-ют собой произведение 
мер $\Psi_1,\Psi_2, \ldots , \Psi_N$, где $\Psi_i \hm\in \Gamma_i$, $i\hm=1,2,\ldots,N$.
Множество~$\Gamma$ можно отож\-де\-ст\-вить с~множеством всех стратегий управ\-ле\-ния 
полумарковским процессом~$\xi(t)$.

Проблема оптимального управления полумар\-ковским процессом~$\xi(t)$ будет в~дальнейшем 
сформулирована в~виде задачи безусловного экстремума некоторого функционала 
$I(\Psi)\hm=I(\Psi_1,\Psi_2, \ldots , \Psi_N)$, заданного на множестве 
допустимых стратегий управления. Содержание показателя качества управления~$I(\Psi)$, 
аналитическое представление для него, а~также описание множества допустимых 
стратегий управления будут приведены в~последующих разделах данной работы.

Для получения дальнейших результатов потребуются различные вероятностные 
характеристики управляемого полумарковского процесса~$\xi(t)$. Как известно из
 общей теории полумарковских процессов~\cite{19, 20}, 
 основной вероятностной характеристикой такого процесса является так называемая 
 полумарковская функция. Определим эту функцию для процесса с~управлением 
 (см.~\cite[гл.~5]{8}):
\begin{multline}
Q_{ij}(t,u)=
{\sf P}\left(\xi_{n+1}=j,\theta_n<t \mid \xi_n=i, u_n=u\right)\,,\\
t\in [0,\infty)\,,\ u\in U_i\,;\ i,j\in X=\{1,2,\ldots,N\}\,. \label{e1}
\end{multline}
Используя полумарковские функции, можно получить вероятности перехода 
управляемой цепи Маркова~$\{\xi_n\}$:
\begin{multline}
p_{ij}(u)={\sf P}\left(\xi_{n+1}=j \mid \xi_n=i, u_n=u\right)= {}\\
{}=
\lim\limits_{t\rightarrow\infty}Q_{ij}(t,u)\,,\enskip
u\in U_i\,;\enskip i,j\in X\,, 
\label{e2}
\end{multline}
а также функции распределения длительностей пребывания полумарковского 
процесса~$\xi(t)$ в~соответствующих состояниях:

\noindent
\begin{multline}
H_{i}(t,u)={\sf P}\left(\theta_n<t \mid \xi_n=i, u_n=u\right)={}\\
{}=
\sum\limits_{j\in X}Q_{ij}(t,u)\,,\enskip
t\in [0,\infty)\,,\  u\in U_i\,; \  i\in X\,. 
\label{e3}
\end{multline}

Обозначим через
\begin{multline}
T_{i}(u)=\mathbf{E}\left[\theta_n \mid \xi_n=i, u_n=u\right]={}\\
{}=
\int\limits_0^{\infty}\left[1-H_i(t,u)\right]\,dt\,,\enskip
u\in U_i\,,\ i\in X\,, 
\label{e4}
\end{multline}
математические ожидания длительностей пребывания полумарковского процесса~$\xi(t)$ 
в~каждом из состояний.

Введенные выше характеристики~(1)--(4) определены для случая, когда 
в~момент изменения состояния~$t_n$ процесс оказывается в~состоянии~$i$ 
и~принимается решение $u\hm\in U_i$. При заданной стратегии управления 
$\Psi\hm=\left(\Psi_1,\Psi_2, \ldots , \Psi_N\right)$ можно записать 
соответствующие вероятностные характеристики без условия на управление, а~именно:
\begin{multline*}
Q_{ij}(t)={\sf P}\left(\xi_{n+1}=j,\theta_n<t \mid \xi_n=i\right)={}\\
{}=
\int\limits_{U_i}Q_{ij}(t,u) \,d\Psi_i(u)\,,\enskip 
t\in [0,\infty)\,,\ i,j\in X\,; %\label{e5}
\end{multline*}

\vspace*{-12pt}

\noindent
\begin{multline}
p_{ij}={\sf P}\left(\xi_{n+1}=j \mid \xi_n=i\right)=
\int\limits_{U_i}p_{ij}(u)\, d\Psi_i(u)\,,\\  
i,j\in X\,; 
\label{e6}
\end{multline}

\vspace*{-9pt}

\noindent
\begin{equation}
T_{i}=\mathbf{E}\left[\theta_n \mid \xi_n=i\right]=
\int\limits_{U_i}T_{i}(u)\,d\Psi_i(u)\,,\enskip i\in X\,. 
\label{e7}
\end{equation}
В дальнейшем будем предполагать, что для рас\-смат\-ри\-ва\-емой 
полумарковской модели заданы вероятностные характеристики 
$p_{ij}(u)$, $u\hm\in U_i$, $i,j\hm\in X$, и~$T_i(u)$, $u\hm\in U_i$, $i\hm\in X$, 
определяемые соотношениями~(\ref{e2}) и~(\ref{e4}). 
Для фиксированной стратегии управления $\Psi\hm=(\Psi_1, \Psi_2,\ldots, \Psi_N)$ 
соответствующие вероятностные характеристики~$p_{ij}$ и~ $T_i$, $i,j\hm\in X,$ 
определены равенствами~(\ref{e6}) и~(\ref{e7}) без условий на управление.

\section{Стационарный стоимостной показатель качества управления}

Определим некоторый стоимостной аддитивный функционал, связанный 
с~рассматриваемым полумарковским процессом~$\xi(t)$. По содержанию этот функционал 
представляет собой случайный\linebreak доход или прибыль, накопленную за период времени $[0,t]$. 
Определения такого функционала приведены в~основополагающих работах~[3; 8, гл.~5].\linebreak 
Обозначим через $\widetilde{v}(t)$, $t\hm\geq 0$, значение этого аддитивного 
функционала в~момент времени~$t$; $\widetilde{v}_n\hm=\widetilde{v}(t_n\hm+0)$~--- 
соответствующее значение непосредственно после очередного момента изменения\linebreak 
состояния~$t_n$, $n\hm=0,1,2,\ldots$; $\widetilde{v}_0\hm=v_0$~--- 
заданное начальное значение в~момент $t\hm=0$. Рассмотрим величину
\begin{multline}
d_i(u)=\mathbf{E}\left[\widetilde{v}_{n+1}-\widetilde{v}_n \mid \xi_n=i\,, 
u_n=u\right]\,,\\
u\in U_i\,, \enskip i\in X\,, \label{e8}
\end{multline}
представляющую собой математическое ожидание приращения стоимостного 
аддитивного функционала за период времени между последовательными 
изменениями состояния полумарковского процесса~$\xi(t)$. Тогда соответствующее 
математическое ожидание, вычисляемое без условия на решение, 
принимаемое в~момент времени~$t_n$, представляется в~виде:
\begin{equation*}
d_i=\mathbf{E}\left[\widetilde{v}_{n+1}-\widetilde{v}_n \mid \xi_n=i\right]=
\!\int\limits_{U_i}\!d_i(u)\,d\Psi_i(u)\,,\ i\in X \,. %\label{e9}
\end{equation*}

Предположим, что для заданной стратегии управ\-ле\-ния 
$\Psi\hm=(\Psi_1,\Psi_2,\ldots,\Psi_N)$ вложенная цепь Маркова~$\{\xi_n\}$ 
имеет ровно один класс возвратных положительных состояний (по терминологии, 
принятой в~\cite{8}, такое множество состояний называется эргодическим классом). 
Как известно~\cite[гл.~VIII]{15}, данное условие является необходимым 
и~достаточным для существования единственного\linebreak стационарного распределения. 
Обозначим это стационарное распределение цепи Маркова~$\{\xi_n\}$ через 
$\pi\hm=(\pi_1, \pi_2,\ldots, \pi_N)$. Заметим, что данное\linebreak распределение зависит  
от стратегии управления $\Psi\hm=(\Psi_1,\Psi_2,\ldots,\Psi_N)$. При указанном 
условии имеет место следующий результат, называемый эргодической теоремой 
для аддитивного стоимостного функционала:
\begin{equation}
I=\lim\limits_{t\rightarrow\infty}\fr{\mathbf{E}\widetilde{v}(t)}{t}=
\fr{\sum\nolimits_{i=1}^N d_i\pi_i}{\sum\nolimits_{i=1}^N T_i\pi_i}\,. 
\label{e10}
\end{equation}

Соотношение~(\ref{e10}) доказано в~работе~\cite[гл.~5]{8}. Заметим, что аналогичные 
результаты имеют мес\-то для гораздо более общих полумарковских моделей~\cite{10, 11}.

По своему прикладному содержанию величина, определяемая соотношением~(\ref{e10}), 
представляет собой
среднюю удельную прибыль, связанную с~эволюцией системы в~стационарном
режиме. Кроме того, величина~$I$ представляет собой функционал от
набора вероятностных распределений~$\Psi_{i}$, $i\hm\in\lbrace 1,\ldots
,N\rbrace $, определяющих стратегию управле-\linebreak\vspace*{-12pt}

\pagebreak

\noindent
ния системой. 
В~дальнейшем будем рассматривать стационарный стоимостной функционал 
$I\hm=I(\Psi_{1},\Psi_{2},\ldots , \Psi_{N})$ как
показатель качества управ\-ле\-ния системой и~построенным полумарковским
процессом~$\xi (t)$.

\section{Представление стационарного показателя в~форме
дробно-линейного интегрального функционала}

В данном разделе будет приведено утверждение об аналитическом
представлении стационарного стоимостного функционала~(\ref{e10}), 
служащего критерием качества управления в~рассматриваемой задаче управления 
полумарковским процессом.

\smallskip

\noindent
\textbf{Теорема 1.} \textit{Стационарный стоимостной показатель, 
определяемый равенством}~(\ref{e10}), \textit{представляет собой дроб\-но-ли\-ней\-ный
функционал от вероятностных распределений~$\Psi_{i}(u_{i})$,
$i\hm\in\{1,\dots,N\}$. Данный функционал задается
аналитически следующей формулой:}
\begin{multline}
I=I(\Psi_{1},\ldots, \Psi_{N})={}\\
\hspace*{-2mm}{}=\!
\fr{\int\nolimits_{U_1}\!{\cdots\! 
\int\nolimits_{U_N}\!{A(u_{1},\ldots ,u_{N})d\Psi_{1}(u_{1})\cdots
\,d\Psi_{N}(u_{N})}}}{\int\nolimits_{U_1}{\!\cdots\! \int\nolimits_{U_N}\!{B(u_{1},\ldots ,u_{N})
\,d\Psi_{1}(u_{1})\ldots
d\Psi_{N}(u_{N})}}},\!\!\! \label{e11}
\end{multline}
\textit{где подынтегральные функции числителя и~знаменателя выражаются
соотношениями}:
\begin{align}
A(u_{1},\ldots
,u_{N})&={}\notag\\
&\hspace*{-20mm}{}=\sum\limits_{i=1}^{N}{d_{i}(u_{i})}{\widehat{D}}^{(i)}(u_{1}, \ldots
,u_{i-1},u_{i+1}, \ldots , u_{N})\,;  \label{e12}\\
 B(u_{1},\ldots
,u_{N})&={}\notag\\
&\hspace*{-20mm}{}=\sum\limits_{i=1}^{N}{T_{i}(u_{i})}{\widehat{D}}^{(i)}(u_{1}, \ldots
,u_{i-1},u_{i+1}, \ldots , u_{N})\,.  \label{e13}
\end{align}
\textit{В свою очередь, функции} ${\widehat{D}}^{(i)}(u_{1}, \ldots
,u_{i-1},u_{i+1}, \ldots$\linebreak $\ldots , u_{N})$, $i\hm\in\{1,\dots,N\}$, 
\textit{входящие в~правые части формул}~(\ref{e12}) и~(\ref{e13}), 
\textit{определяются следующим образом:}

\noindent
\begin{multline}
{\widehat{D}}^{(i)}(u_{1}, \ldots ,u_{i-1},u_{i+1}, \ldots , u_{N})={}
\\
{}=(-1)^{N+i+2}\sum\limits_{\alpha ^{(N),i}}{(-1)}^{\delta (\alpha
^{(N),i}) }{\widehat{D}}_{0}^{(i)}\left(\alpha ^{(N),i},u_{1}, \ldots\right.\\
\left.\ldots , u_{i-1},u_{i+1}, \ldots , u_{N}\right)\,. \label{e14}
\end{multline}
\textit{Здесь} $\alpha ^{(N),i}=(\alpha _{1}, \ldots , \alpha _{i-1},\alpha_{i+1}, \ldots , 
\alpha _{N})$~\textit{--- произвольная
перестановка чисел }$(1, \ldots , i-1, i+1, \ldots , N)$;
$\delta
(\alpha ^{(N),i})$~\textit{--- число инверсий в~перестановке} 
$\alpha ^{(N),i}$;

\noindent
\begin{multline}
{\widehat{D}}_{0}^{(i)}(\alpha ^{(N),i},u_{1}, \ldots ,u_{i-1},u_{i+1},
\ldots , u_{N})={}\\
{} ={\widetilde{p}}_{1,\alpha _{1}}\left(u_{1}\right)\cdots {\widetilde{p}}_{i-1,\alpha
_{i-1}}\left(u_{i-1}\right){\widetilde{p}}_{i+1,\alpha _{i+1}}\left(u_{i+1}\right)\cdots\\
\cdots
{\widetilde{p}}_{N,\alpha _{N}}\left(u_{N}\right)\,, 
\label{e15}
\end{multline}
где
\begin{multline}
 {\widetilde{p}}_{k,\alpha _{k}}(u_{k})=
\begin{cases}
p_{kk}(u_{k})-1,\  & \alpha _{k}=k\,; \\
p_{k,\alpha _{k}}(u_{k}),\  & \alpha _{k}\ne k, \\
\end{cases}\\
 k=1, \ldots , i-1, i+1, \ldots ,N\,. \label{e16}
 \end{multline}
\textit{Функции $p_{ij}(u_i)$, $T_{i}(u_{i})$ и~$d_{i}(u_{i})$,
$u_i\hm\in U_i$, $i,j\hm\in \{1,2,\ldots,N\}$, 
входящие в~соотношения}~(\ref{e12})--(\ref{e16}), 
\textit{определяются равенствами}~(\ref{e2}), (\ref{e4}) \textit{и}~(\ref{e8}) \textit{соответственно.}

\smallskip

\noindent
Д\,о\,к\,а\,з\,а\,т\,е\,л\,ь\,с\,т\,в\,о\ теоремы~1 
в~весьма сжатой форме приведено в~работе~\cite{21}. Читателю, интересующемуся 
более подробным обоснованием данного результата, порекомендуем обратиться к~тексту 
кандидатской диссертации А.\,В.~Иванова~\cite[гл.~3]{22}.

\smallskip

Итак, теорема~1 позволяет получить явное аналитическое представление 
для стационарного стоимостного показателя вида~(\ref{e10}) в~форме 
дроб\-но-ли\-ней\-но\-го интегрального функционала от набора\linebreak вероятностных мер 
$\Psi\hm=(\Psi_{1},\Psi_{2},\ldots , \Psi_{N})$, за\-да\-ющих стратегию управления 
полумарковским процессом~$\xi(t)$. При этом подынтегральные функции числителя 
и~знаменателя задаются формулами~(\ref{e12}), (\ref{e13}) 
и~вспомогательными равенствами~(\ref{e14})--(\ref{e16}). Таким образом, функция
\begin{equation}
C\left(u_1, u_2,\ldots, u_N\right)=\fr{A(u_1, u_2,\ldots, u_N)}{B(u_1, u_2,\ldots, u_N)}\,,
\label{e17}
\end{equation}
которая в~дальнейшем будет называться основной функцией дроб\-но-ли\-ней\-но\-го 
интегрального функционала~(\ref{e11}) и~которая будет играть важную роль 
в~дальнейшем исследовании, также явно определяется формулами~(\ref{e17}), 
(\ref{e12}), (\ref{e13}).

\section{Формальная постановка оптимизационной задачи 
и~условия существования оптимальной стратегии управления полумарковским процессом}

Будем рассматривать проблему управления полумарковским процессом~$\xi(t)$ в~форме 
экстремальной задачи
\begin{multline}
I(\Psi)=I\left(\Psi_1, \Psi_2,\ldots,\Psi_N\right)\rightarrow \mathrm{extr}\,,
\\
\Psi=\left(\Psi_1, \Psi_2,\ldots,\Psi_N\right)\in\Gamma\,. \label{e18}
\end{multline}
При этом показатель качества управления~$I(\Psi)$ представляет собой 
дроб\-но-ли\-ней\-ный интегральный функционал вида~(\ref{e11}).

Для решения экстремальной задачи~(\ref{e18}) воспользуемся некоторым утверждением 
об экстремуме дроб\-но-ли\-ней\-но\-го интегрального функционала. Прежде 
чем сформулировать данное утверждение, отметим, что в~теории оптимизации 
хорошо известны задачи, в~которых целевая функция представляет собой 
отношение двух линейных отображений, а имеющиеся ограничения также линейны. 
Такой раздел называется дроб\-но-ли\-ней\-ным программированием. Основные
 теоретические результаты данного направления изложены в~работе~\cite{23},
  там же приведена подробная библиография. В~дальнейшем потребуется некоторый 
  специальный результат о безусловном экстремуме дроб\-но-ли\-ней\-но\-го 
  интегрального функционала вида~(\ref{e11}), который был впервые сформулирован 
  в~работе~\cite{14}. Заметим, что для использования этого результата необходимо, 
  чтобы выполнялись некоторые предварительные условия, которые в~данном случае 
  можно сформулировать следующим образом:
\begin{enumerate}[1.]
\item Интегральные выражения
\begin{align*}
I_1(\Psi)&=I_1\left(\Psi_1,\Psi_2,\ldots,\Psi_N\right)={}&\\
&\hspace*{-13mm}{}=\int\limits_{U_1}\!\cdots\!
\int\limits_{U_N}\!\!A\left(u_1,\ldots ,u_N\right)\,
d\Psi_1\left(u_1\right) %d\Psi_2\left(u_2\right)
\cdots
 d\Psi_N\left(u_N\right)\,;
\\
I_2(\Psi)&=I_2\left(\Psi_1,\Psi_2,\ldots,\Psi_N\right)={}&\\
&\hspace*{-13mm}{}=\int\limits_{U_1}\!\cdots\!\int\limits_{U_N}\!\!
B\left(u_1,\ldots,u_N\right)\,
d\Psi_1\left(u_1\right)% d\Psi_2\left(u_2\right)\cdots\\
\cdots d\Psi_N\left(u_N\right)
\end{align*}
определены для всех стратегий управления $\Psi\hm=(\Psi_1, \ldots,\Psi_N)
\hm\in \Gamma$.

\item Функционал $I_2(\Psi)=I_2(\Psi_1, \ldots,\Psi_N)\hm\neq 0$ 
для всех $\Psi\hm=(\Psi_1, \ldots,\Psi_N)\hm\in \Gamma$.

\item Множество $\Gamma$ включает в~себя множество всех вырожденных 
вероятностных мер: $\Gamma^* \hm\subset \Gamma$.
\end{enumerate}

Сделаем несколько важных замечаний по поводу введенных предварительных условий.

\smallskip

\noindent
\textbf{Замечание~1.}\ Из условия~2 следует, что функция $B(u_1, u_2,\ldots, u_N)$ 
не может принимать значения разных знаков. С~учетом условия~3 
получаем, что указанная функция должна обладать \mbox{свойством} строгой 
знакопостоянности на всем множестве~$U$. С~другой стороны, если выполняется 
условие строгой знакопостоянности функции $B(u_1, u_2,\ldots, u_N), 
(u_1, u_2,\ldots, u_N)\hm\in U$, то условие~2 выполняется автоматически.

\smallskip

\noindent
\textbf{Замечание~2.}\ Если рассматривать в~качестве целевого функционала 
$I(\Psi_1, \Psi_2,\ldots,\Psi_N)$ экстремальной задачи~(\ref{e18}) 
стационарный стоимостной пока\-затель~(\ref{e10}), то функция $B(u_1,u_2,\ldots,u_N)$ 
имеет\linebreak следующее теоретическое содержание. Данная функция представляет собой условное 
математическое ожидание длительности периода времени между соседними моментами 
изменения со\-сто\-яния полумарковского процесса~$\xi(t)$ при условии, что стратегия 
его управ\-ле\-ния является детерминированной и~задается набором значений аргументов 
$(u_1,u_2,\ldots,u_N)$. Тогда условие строгой положительности функции 
$B(u_1,u_2,\ldots,u_N)$ при всех $(u_1,u_2,\ldots,u_N)\hm\in U$ является естественным 
и~фактически означает, что при любой заданной детерминированной стратегии 
управ\-ле\-ния процесс~$\xi(t)$ не имеет мгновенных со\-сто\-яний, длительность пребывания 
в~которых равна нулю.

\smallskip

\noindent
\textbf{Замечание~3.}\ Сделаем некоторые замечания, связан\-ные с~подынтегральной 
функцией числителя дроб\-но-ли\-ней\-но\-го интегрального функционала~(\ref{e11}). 
Как и~ранее, будем рассматривать в~качестве целевого функционала $I(\Psi_1, \Psi_2,\ldots,\Psi_N)$\linebreak 
экстремальной задачи~(\ref{e18}) стационарный стоимостной показатель~(\ref{e10}). 
Тогда для любого фиксированного набора значений аргументов $(u_1,u_2,\ldots,u_N)\hm\in U$ 
значение функции $A(u_1,u_2,\ldots\linebreak \ldots,u_N)$ представляет собой условное математическое
 ожидание приращения рассматриваемого стоимостного функционала, 
 происшедшее за время пребывания полумарковского процесса~$\xi(t)$ в~некотором 
 фиксированном  состоянии при условии, что стратегия управления является 
 детерминированной и~задается указанным набором $(u_1,u_2,\ldots,u_N)\hm\in U$. 
 Отметим, что в~теореме об экстремуме дроб\-но-ли\-ней\-но\-го интегрального 
 функционала, доказанной в~работе~\cite[гл.~10]{12}, 
 на подынтегральную функцию числителя накладываются условия ограниченности на 
 всем множестве значений аргумента. Для многих математических моделей и~связанных 
 с~ними задач оптимального управления такое условие является излишне ограничительным. 
 В~качестве примера можно привести модели оптимального управления запасом непрерывного 
 продукта, рассмотренные в~работах~\cite{27, 28}. 
 В~настоящем исследовании на функцию $A(u_1,u_2,\ldots,u_N)$ накладывается только 
 условие интегрируемости по любому заданному набору вероятностных мер 
 $\Psi\hm=(\Psi_1, \Psi_2,\ldots,\Psi_N)$, образующему стратегию управления 
 полумарковским процессом~$\xi(t)$ (условие~1 системы предварительных условий).

\smallskip

\noindent
\textbf{Замечание~4.} Условия~1--3 являются необходимыми для корректной 
постановки задачи безусловного экстремума дроб\-но-ли\-ней\-но\-го интегрального 
функционала. Если этот функционал служит показателем качества в~задаче оптимального 
управления случайным процессом, то необходимо добавить к~этим условиям дополнительное, 
связанное с~некоторой регулярностью самого управляемого процесса, а~именно: некоторый 
содержательный показатель, связанный с~поведением этого процесса, должен существовать 
и~быть представимым в~виде дроб\-но-ли\-ней\-но\-го интегрального функционала. 
Если потребовать, чтобы выполнялось эргодическое соотношение~(\ref{e10}), 
то можно использовать\linebreak теорему~1 и~сформулировать задачу оптимального управ\-ле\-ния 
в~виде~(\ref{e18}) для дроб\-но-ли\-ней\-но\-го\linebreak интегрального функционала~(\ref{e11}). 
Таким образом, необходимо ввести условие, обеспечивающее существование единственного 
стационарного распределения вложенной цепи Маркова и~выполнение\linebreak соотношения~(\ref{e10}). 
По аналогии с~[8, гл.~5] сформулируем это дополнительное условие в~следующем виде:
\begin{enumerate}
\setcounter{enumi}{3}
\item Для любой рассматриваемой стратегии управ\-ле\-ния $\Psi\hm=
(\Psi_1, \Psi_2,\ldots,\Psi_N)\hm\in \Gamma$ вложенная цепь Маркова 
полумарковского процесса $\xi(t)$ имеет ровно один класс возвратных 
положительных состояний.
\end{enumerate}

Теперь определим понятие допустимой стратегии управления полумарковским процессом 
с~конечным множеством состояний.

\smallskip

\noindent
\textbf{Определение~2.}\ Назовем стратегию управления 
$\Psi\hm=(\Psi_1, \Psi_2,\ldots,\Psi_N)$ 
допустимой в~данной задаче, если она удовлетворяет условиям~1--4.


\smallskip

\noindent
\textbf{Замечание~5.}\ Как следует из замечания~1, если потребовать, 
чтобы функция $B(u_1, u_2,\ldots,u_N)$ являлась строго знакопостоянной при 
всех $(u_1, u_2,\ldots,u_N)\hm\in U$, то можно считать допустимыми стратегии 
$(\Psi_1, \Psi_2,\ldots,\Psi_N)$, удовлетворяющие условиям~1, 3, 4. С~учетом замечания~2 
о~естественном характере условия строгой знакопостоянности функции $B(u_1,u_2,\ldots,u_N)$ 
при всех значениях аргументов $(u_1, u_2,\ldots,u_N)\hm\in U$ будем требовать 
выполнения этого условия в~формулировке приводимой в~дальнейшем основной 
теоремы об оптимальной стратегии управления полумарковским процессом.

\smallskip

\noindent
\textbf{Замечание~6.}\ Ниже будет сформулирована и~доказана основная 
теорема об оптимальной стра\-тегии управления полумарковским процессом с~конеч\-ным 
множеством состояний. Будем формулировать эту теорему по отношению к~экстремальной 
задаче~(\ref{e18}), в~которой целевой функционал $I(\Psi_1, \Psi_2,\ldots,\Psi_N)$ 
имеет вид дроб\-но-ли\-ней\-но\-го интегрального функционала. 
Это обстоятельство связано с~тем, что целевой функционал в~задаче 
оптимального управления необязательно должен иметь характер стационарного 
стоимостного показателя вида~(\ref{e10}). В~частности, еще в~1983~г.\ П.\,В.~Шнурковым 
было установлено~\cite{24}, что ряд показателей, связанных 
с~временем пребывания управляемого полумарковского процесса в~заданном конечном 
подмножестве состояний, имеет структуру дроб\-но-ли\-ней\-но\-го интегрального 
функционала от набора вероятностных мер, определяющих стратегию управления. 
Таким образом, рассматриваемая задача управления имеет более общий характер, 
чем задача, в~которой целевой функционал представляет собой стационарный 
стоимостной показатель вида~(\ref{e10}).






\smallskip

\noindent
\textbf{Замечание~7.}\ Если рассматривать задачу оптимального управления 
полумарковским процессом, в~кото\-рой целевой функционал не совпадает 
со стационарным стоимостным показателем~(\ref{e10}), то возможно, что могут 
потребоваться другие дополнительные условия, обеспечивающие существование этого 
показателя и~его представление в~форме~(\ref{e11}). В~связи с~этим в~формулировке 
основной теоремы будем использовать термин допустимые стратегии в~широком смысле, 
имея в~виду выполнение всех необходимых условий для каждого рассмат\-ри\-ва\-емо\-го 
показателя качества управления.

\smallskip


\noindent
\textbf{Замечание 8.} Множество допустимых стратегий может 
не совпадать с~множеством всех возможных стратегий управления. 
В~частности, допустимые стратегии могут состоять только из дискретных вероятностных 
мер $\Psi_1, \Psi_2,\ldots,\Psi_N$, т.\,е.\ таких, которые сосредоточены на дискретных 
множествах точек пространств $U_1, U_2,\ldots,U_N$.

\section{Теоретическое решение задачи оптимального управления}

Перейдем к~формулировке и~доказательству тео\-ре\-мы об 
оптимальной стратегии управ\-ле\-ния полумарковским процессом с~конечным 
множеством состояний.

\smallskip

\noindent
\textbf{Теорема~2.} \textit{Рассмотрим проблему оптимального управ\-ле\-ния 
полумарковским процессом~$\xi(t)$ в~виде экстремальной задачи}~(\ref{e18}), 
\textit{определенной на множестве допустимых стратегий $\Gamma$, 
для дроб\-но-ли\-ней\-но\-го 
функционала}~(\ref{e11}). \textit{Пусть функция $B(u_1,u_2,\ldots,u_N)$, 
входящая в~определение функционала}~(\ref{e11}),
\textit{является строго знакопостоянной (строго положительной или строго отрицательной) 
при всех значениях аргументов $(u_1,u_2,\ldots,u_N)\hm\in U$.
Тогда справедливы сле\-ду\-ющие утверждения}:
\begin{enumerate}[1.]
\item \textit{Если функция} $C(u_1,u_2,\ldots,u_N)\hm=A(u_1,u_2,\ldots$\linebreak
$\ldots,u_N)/{B(u_1,u_2,\ldots,u_N)}$ 
\textit{ограничена сверху или снизу и~достигает глобального экст\-ре\-му\-ма на множестве
$U\hm=U_1\times U_2\times \cdots \times U_N$ (максимума или минимума), 
то оптимальная стратегия управления полумарковским процессом~$\xi(t)$ существует, 
является детерминированной и~определяется
вырожденной вероятностной мерой $\Psi^*\hm\in \Gamma^*$, сосредоточенной в~точке, 
в~которой достига\-ет соответствующего экстремума функция $C(u_1,u_2,\ldots,u_N)$,
и~при этом выполняются соотношения}:
\begin{multline}  %{\substack{{i=\overline{1,n}}\\ {j=\overline{1,l}}}}
\max\limits_{\Psi \in \Gamma} I(\Psi)=
\max\limits_{\substack{{\Psi_i \in \Gamma_i\,,}\\ 
{i=\overline{1,N}}}}
I\left(\Psi_1,\Psi_2,\ldots,\Psi_N\right)={}\\
{}=
\max\limits_{\substack{{\Psi_i^* \in \Gamma_i^*,}\\ 
{i=\overline{1,N}}}}
 I\left(\Psi_1^*,\Psi_2^*,\ldots,\Psi_N^*\right)={}\\
{}=\max\limits_{(u_1,u_2,\ldots,u_N)\in U}\fr{A(u_1,u_2,\ldots,u_N)}
{B(u_1,u_2,\ldots,u_N)}\,; \label{e19}
\end{multline}

\vspace*{-12pt}

\noindent
\begin{multline*}
\min\limits_{\Psi \in \Gamma} I(\Psi)=
\min\limits_{\substack{{\Psi_i \in \Gamma_i\,,}\\ 
{i=\overline{1,N}}}} I\left(\Psi_1,\Psi_2,\ldots,\Psi_N\right)={}\\
{}=
\min\limits_{\substack{{\Psi_i^* \in \Gamma_i^*,}\\ 
{i=\overline{1,N}}}}
I\left(\Psi_1^*,\Psi_2^*,\ldots,\Psi_N^*\right)={}\\
{}=\min\limits_{(u_1,u_2,\ldots,u_N)\in U}\fr{A(u_1,u_2,\ldots,u_N)}
{B(u_1,u_2,\ldots,u_N)}\,. %\label{e20}
\end{multline*}
\item \textit{Если функция $C(u_1,u_2,\ldots,u_N)\hm=
{A(u_1,u_2,\ldots,u_N)}/{B(u_1,u_2,\ldots,u_N)}$ ограничена сверху или снизу, 
но не достигает глобального экстремума на множестве $U\hm=U_1\times U_2\times\cdots
\times U_N$,
то для любого $\varepsilon\hm > 0$ можно выбрать $\varepsilon$-оп\-ти\-маль\-ную 
детерминированную стратегию управления полумарковским процессом~$\xi(t)$, 
которая определяется вырожденной
вероятностной мерой $\Psi^{*(+)}(\varepsilon)\hm\in \Gamma^*$ или вырожденной
вероятностной мерой $\Psi^{*(-)}(\varepsilon)\hm\in \Gamma^*$, в~зависимости от 
вида экстремума (максимума или минимума) в~задаче}~(\ref{e18}). 
\textit{При этом вероятностная мера $\Psi^{*(+)}(\varepsilon)\hm\in \Gamma^*$ может быть 
сосредоточена в~любой точке $\left(u_1^{(+)}(\varepsilon),u_2^{(+)}(\varepsilon),\ldots,
u_N^{(+)}(\varepsilon)\right)$, удовлетворяющей соотношению}:
\begin{multline}
\sup\limits_{(u_1,u_2,\ldots,u_N) \in U}
\fr{A(u_1,u_2,\ldots,u_N)}{B(u_1,u_2,\ldots,u_N)}-\varepsilon <{}\\
{}<
\fr{A\left(u_1^{(+)}(\varepsilon),u_2^{(+)}(\varepsilon),\ldots,u_N^{(+)}
(\varepsilon)\right)}
{B\left(u_1^{(+)}(\varepsilon),u_2^{(+)}(\varepsilon),\ldots,u_N^{(+)}
(\varepsilon)\right)}<{}\\
{}<\sup\limits_{(u_1,u_2,\ldots,u_N) \in U}
\fr{A(u_1,u_2,\ldots,u_N)}{B(u_1,u_2,\ldots,u_N)}<\infty\,, 
\label{e21}
\end{multline}
\textit{если функция $C(u_1,u_2,\ldots,u_N)$ ограничена сверху 
и~экстремальная задача}~(\ref{e18}) 
\textit{представляет собой задачу на максимум. Аналогично вероятностная мера 
$\Psi^{*(-)}(\varepsilon)\hm\in \Gamma^*$ может быть сосредоточена в~любой точке 
$\left(u_1^{(-)}(\varepsilon),u_2^{(-)}(\varepsilon),\ldots,u_N^{(-)}(\varepsilon)
\right)$, удовлетворяющей соотношению}:

\noindent
\begin{multline*}
-\infty<\inf\limits_{(u_1,u_2,\ldots,u_N) \in U}\fr{A(u_1,u_2,\ldots,u_N)}
{B(u_1,u_2,\ldots,u_N)} <{}\\
{}<
\fr{A\left(u_1^{(-)}(\varepsilon),u_2^{(-)}
(\varepsilon),\ldots,u_N^{(-)}(\varepsilon)\right)}
{B\left(u_1^{(-)}(\varepsilon),u_2^{(-)}(\varepsilon),\ldots,
u_N^{(-)}(\varepsilon)\right)}<{}\\
{}<\inf\limits_{(u_1,u_2,\ldots,u_N) \in U}
\fr{A(u_1,u_2,\ldots,u_N)}{B(u_1,u_2,\ldots,u_N)}+\varepsilon\,, 
%\label{e22}
\end{multline*}
\textit{если функция $C(u_1,u_2,\ldots,u_N)$ ограничена снизу и~экстремальная 
задача}~(\ref{e18})  \textit{представляет собой задачу на минимум}.
\item \textit{Если функция $C(u_1,u_2,\ldots,u_N)\hm=
{A(u_1,u_2,\ldots,u_N)}/{B(u_1,u_2,\ldots,u_N)}$ не ограничена сверху 
или снизу, то оптимальной стратегии управления в~смысле
соответствующей экстремальной задачи не существует. 
При этом найдется такая последовательность вырожденных вероятностных
мер~$\Psi^{*(+)}(n)$, сосредоточенных в~точках 
$\left(u_1^{(+)}(n),u_2^{(+)}(n),\ldots,u_N^{(+)}(n)\right)$, $n\hm=1,2,\dots $, 
для которых выполняется соотношение}:
\begin{multline*}
I\left(\Psi^*(n)\right)={}\\
{}=
I\left(\Psi_1^{*(+)}(n),\Psi_2^{*(+)}(n),\ldots,\Psi_N^{*(+)}(n)\right)={}\\
{}=\fr{A\left(u_1^{(+)}(n),u_2^{(+)}(n),\ldots,u_N^{(+)}(n)\right)}
{B\left(u_1^{(+)}(n),u_2^{(+)}(n),\ldots,u_N^{(+)}(n)\right)}\to 
\infty\\
\mbox{при}\ n\to\infty\,, 
%\label{e23}
\end{multline*}
\textit{если функция $C(u_1,u_2,\ldots,u_N)$ не ограничена сверху. 
Аналогично найдется такая последовательность вырожденных вероятностных
мер~$\Psi^{*(-)}(n)$, сосредоточенных в~точках 
$\left(u_1^{(-)}(n),u_2^{(-)}(n),\ldots,u_N^{(-)}(n)\right)$, 
$n\hm=1,2,\dots $, для которых выполняется соотношение}:
\begin{multline*}
I\left(\Psi^{*(-)}(n)\right)={}\\
{}= I
\left(\Psi_1^{*(-)}(n),\Psi_2^{*(-)}(n),\ldots,\Psi_N^{*(-)}(n)\right)={}\\
{}=\fr{A\left(u_1^{(-)}(n),u_2^{(-)}(n),\ldots,u_N^{(-)}(n)\right)}
{B\left(u_1^{(-)}(n),u_2^{(-)}(n),\ldots,u_N^{(-)}(n)\right)}\to 
-\infty\\
\mbox{при}~~n\to\infty\,,  
%\label{e24}
\end{multline*}
\textit{если функция $C(u_1,u_2,\ldots,u_N)$ не ограничена \mbox{снизу}}.
\end{enumerate}
\textit{При этом сформулированные утверждения каждого пункта теоремы~$2$ 
могут выполняться как по отдельности, для одного из двух
видов экстремума, так и~совместно, для обоих видов экстремума.}

\smallskip

Прежде чем непосредственно доказывать теорему~2, докажем некоторые 
вспомогательные утверждения.

\smallskip

\noindent
\textbf{Лемма~1.}\ 
\textit{Рассмотрим дроб\-но-ли\-ней\-ный интегральный функционал 
$I(\Psi_1, \Psi_2,\ldots, \Psi_N)$ вида}~(\ref{e11}), 
\textit{заданный на некотором множестве наборов вероятностных мер 
$\Psi\hm=(\Psi_1, \Psi_2,\ldots, \Psi_N)\hm \in \Gamma$. Предположим, что на 
множестве~$\Gamma$ выполняется условие~$1$ из набора предварительных условий 
и~функция $B(u_1, u_2,\ldots, u_N)$  обладает свойством строгой знакопостоянности 
при всех $(u_1, u_2,\ldots, u_N) \hm\in U$. Тогда справедливы следующие утверждения}:
\begin{enumerate}[1.]
\item \textit{Если основная функция 
$C(u_1, u_2,\ldots, u_N)\hm={A(u_1, u_2,\ldots, u_N)}/{B(u_1, u_2,\ldots, u_N)}$ 
ограничена сверху, т.\,е.\ выполняется условие}
\begin{multline}
C\left(u_1, u_2,\ldots, u_N\right)=
\fr{A(u_1, u_2,\ldots, u_N)}{B(u_1, u_2,\ldots, u_N)}\leq {}\\
{}\leq
c_0^{(+)}<\infty \,, \enskip \left(u_1, u_2,\ldots, u_N\right) \in U\,, \label{e25}
\end{multline}
\textit{то имеет место неравенство}:
\begin{equation}
I\left(\Psi_1, \Psi_2,\ldots, \Psi_N\right)\leq c_0^{(+)} 
\label{e26}
\end{equation}
\textit{для всех} $(\Psi_1, \Psi_2,\ldots, \Psi_N) \in \Gamma$.
\item \textit{Если основная функция 
$C(u_1, u_2,\ldots, u_N)\hm={A(u_1, u_2,\ldots, u_N)}/{B(u_1, u_2,\ldots, u_N)}$ 
ограничена снизу, т.\,е.\ выполняется условие}
\begin{multline*}
C\left(u_1, u_2,\ldots, u_N\right)=\fr{A(u_1, u_2,\ldots, u_N)}{B(u_1, u_2,\ldots, 
u_N)}\geq{}\\
{}\geq c_0^{(-)}>-\infty \,, 
\left(u_1, u_2,\ldots, u_N\right) \in U\,, 
%\label{e27}
\end{multline*}
\textit{то имеет место неравенство}:
\begin{equation*}
I\left(\Psi_1, \Psi_2,\ldots, \Psi_N\right)\geq c_0^{(-)} 
%\label{e28}
\end{equation*}
\textit{для всех} $(\Psi_1, \Psi_2,\ldots, \Psi_N) \hm\in \Gamma$.
\end{enumerate}

\noindent
Д\,о\,к\,а\,з\,а\,т\,е\,л\,ь\,с\,т\,в\,о\ \ леммы~1.\ 
Докажем первое утверждение леммы. Предположим сначала, 
что функция $B(u_1, u_2,\ldots,  u_N)$ строго положительна:
\begin{equation}
B\left(u_1, u_2,\ldots, u_N\right)>0\,,\enskip
\left(u_1, u_2,\ldots, u_N\right)\in U\,. \label{e29}
\end{equation}
Заметим, что в~таком случае по свойству интеграла~\cite[гл.~V]{18}
\begin{multline}
\hspace*{-2mm}\int\limits_{U_1}\!\!\cdots\! \!\int\limits_{U_N}\!\!B(u_1, \ldots,u_N) \,
d\Psi_1(u_1)%d\Psi_2(u_2)\cdots\\
\cdots d\Psi_N(u_N)>0 \!\!\!\!\label{e30}
\end{multline}
для любого фиксированного набора $\Psi\hm=(\Psi_1, \ldots, \Psi_N)\hm\in \Gamma$.
Из неравенства~(\ref{e25}) с~уче\-том~(\ref{e29}) получаем:
\begin{multline}
\hspace*{-4mm}A\left(u_1,\ldots, u_N\right)\leq{}\\
\hspace*{-4mm}{}\leq c_0^{(+)} B\left(u_1, \ldots, u_N\right)\,, 
\left(u_1, \ldots, u_N\right)\in U\,. \label{e31}
\end{multline}
В свою очередь, из неравенства~(\ref{e31}) и~свойств интеграла следует:
\begin{multline}
\int\limits_{U_1}\!\!\cdots\! \!\int\limits_{U_N}\!\!A(u_1,\ldots, u_N) \,
d\Psi_1\left(u_1\right)%d\Psi_2\left(u_2\right)\cdots\\
\cdots d\Psi_N\left(u_N\right)\leq\\
\hspace*{-24pt}\leq 
c_0^{(+)}\!\!\int\limits_{U_1}\!\!\cdots\!\! \int\limits_{U_N}\!\!\!B\!\left(u_1,\ldots, u_N\right)
 d\Psi_1\!\left(u_1\right)\!%d\Psi_2\left(u_2\right)\cdots\\
 \cdots d\Psi_N\!\left(u_N\right)\!\! 
 \label{e32}
\end{multline}
для любого фиксированного набора $\Psi\hm=(\Psi_1, \ldots, \Psi_N)\hm\in \Gamma$. 
Но тогда из~(\ref{e32}) с~учетом~(\ref{e30}) получаем:
\begin{multline}
I(\Psi_1, \ldots, \Psi_N)={}\\
{}=
\fr{\int\nolimits_{U_1}\!\cdots\! \int\nolimits_{U_N}\!\!A\left(u_1, \ldots, u_N\right)\,
 d\Psi_1(u_1)\cdots d\Psi_N(u_N)}{
\int\nolimits_{U_1}\!\cdots\! \int\nolimits_{U_N}\!\!B\left(u_1, \ldots, u_N\right)\,
 d\Psi_1(u_1)
 \cdots d\Psi_N(u_N)}\leq{}\\
 {}\leq c_0^{(+)} 
 \label{e33}
\end{multline}
для любого фиксированного набора $(\Psi_1, \ldots\linebreak\ldots, \Psi_N)\hm\in \Gamma$.

Предположим теперь, что функция $B(u_1,\ldots, u_N)$ строго отрицательна:
\begin{equation}
B(u_1,\ldots, u_N)<0 \quad \left(u_1, \ldots, u_N\right)\in U\,. 
\label{e34}
\end{equation}
Тогда
\begin{multline}
\hspace*{-6pt}\int\limits_{U_1}\!\!\cdots\!\! \int\limits_{U_N}\!\!B\!\left(u_1,\ldots, u_N\right)\!
 d\Psi_1(u_1) \cdots d\Psi_N(u_N)<0 \!\!\!
 \label{e35}
\end{multline}
для любого фиксированного набора $(\Psi_1, \ldots\linebreak \ldots, \Psi_N)\hm\in \Gamma$.

Как и~ранее, будем исходить из неравенства~(\ref{e25}). 
При выполнении условий~(\ref{e34}) и~(\ref{e35}) характер неравенств~(\ref{e31}) 
и~(\ref{e32}) меняется на противоположный, но характер неравенства~(\ref{e33}) 
остается неизменным. Таким образом, для любой функции 
$B(u_1, u_2,\ldots, u_N)$, обладающей свойством строгой знакопостоянности, 
из условия~(\ref{e25}) следует выполнение неравенства~(\ref{e33}), 
которое совпадает с~(\ref{e26}). Первое утверждение леммы~1 доказано. 
Второе утверждение доказывается аналогично. Лемма~1 доказана.

\smallskip

\noindent
\textbf{Лемма 2.} \textit{Рассмотрим дроб\-но-ли\-ней\-ный интегральный функционал 
$I(\Psi_1, \Psi_2,\ldots, \Psi_N)$ вида}~(\ref{e11}), 
\textit{заданный на некотором множестве наборов вероятностных мер 
$\Psi\hm=(\Psi_1, \Psi_2,\ldots, \Psi_N)\hm\in \Gamma$. Предпо\-ложим, что на 
множестве~$\Gamma$ выполняется условие~$1$ из набора предварительных условий 
и~функция $B(u_1, u_2,\ldots, u_N)$ обладает свойством строгой знакопостоянности 
при всех $(u_1, u_2,\ldots, u_N)\hm\in U$. Тогда справедливы следующие утверждения}:
\begin{enumerate}[1.]
\item \textit{Если основная функция $C(u_1, u_2,\ldots, u_N)\hm=
{A(u_1, u_2,\ldots, u_N)}/{B(u_1, u_2,\ldots, u_N)}$ ограничена сверху, 
но не достигает своего максимального 
значения, то имеет место неравенство}:
\begin{multline}
I\left(\Psi_1, \Psi_2,\ldots, \Psi_N\right)<{}\\
{}< \sup\limits_{(u_1, u_2,\ldots, u_N)\in U}
 C\left(u_1, u_2,\ldots, u_N\right)<\infty \label{e36}
\end{multline}
\textit{для всех} $(\Psi_1, \Psi_2,\ldots, \Psi_N)\in \Gamma$.
\item \textit{Если основная функция $C(u_1, u_2,\ldots, u_N)\hm=
{A(u_1, u_2,\ldots, u_N)}/{B(u_1, u_2,\ldots, u_N)}$ ограничена снизу, 
но не достигает своего минимального значения, то имеет место неравенство}:
\begin{multline*}
I\left(\Psi_1, \Psi_2,\ldots, \Psi_N\right)>{}\\
{}> \inf\limits_{(u_1, u_2,\ldots, u_N)\in U} 
C\left(u_1, u_2,\ldots, u_N\right)>-\infty 
%\label{e37}
\end{multline*}
\textit{для всех} $(\Psi_1, \Psi_2,\ldots, \Psi_N)\hm\in \Gamma$.
\end{enumerate}

\noindent
Д\,о\,к\,а\,з\,а\,т\,е\,л\,ь\,с\,т\,в\,о\ \ леммы~2. 
Докажем первое утверждение леммы. Поскольку множество значений 
основной функции $C(u_1, u_2,\ldots, u_N)$ ограничено сверху, оно имеет конечную 
верхнюю грань:
$$
\exists \sup\limits_{(u_1, u_2,\ldots, u_N)\in U} 
C\left(u_1, u_2,\ldots, u_N\right)<\infty
$$
(см.~\cite[гл.~1, \S3, п.~3.4, теорема~1]{25}).

По условию функция $C(u_1, u_2,\ldots, u_N)$ не достигает своего максимального 
значения. Следовательно, выполняется неравенство:
\begin{multline}
C(u_1, u_2,\ldots, u_N)=\fr{A(u_1, u_2,\ldots, u_N)}{B(u_1, u_2,\ldots, u_N)}<{}\\
{}< 
\sup\limits_{(u_1, u_2,\ldots, u_N)\in U} C(u_1, u_2,\ldots, u_N)<\infty\,, 
\\
\left(u_1, u_2,\ldots, u_N\right)\in U\,.
\label{e38}
\end{multline}
Взяв за основу строгое неравенство~(\ref{e38}), проведем рассуждения, аналогичные тем, 
которые были проведены в~лемме~1 по отношению к~неравенству~(\ref{e25}). 
В~результате получим строгое неравенство~(\ref{e36}).

Второе утверждение леммы~2 доказывается аналогично. Лемма~2 доказана.

\noindent
Д\,о\,к\,а\,з\,а\,т\,е\,л\,ь\,с\,т\,в\,о\ 
\ теоремы~2.
Начнем с~доказательства утверждения~1. Предположим сначала, что основная 
функция $C(u_1, u_2,\ldots, u_N)={A(u_1, u_2,\ldots, u_N)}/{B(u_1, u_2,\ldots, u_N)}$ 
ограничена сверху и~достигает глобального максимума на множестве~$U$ 
в~некоторой точке $u^{(+)}\hm=\left(u^{(+)}_1,u^{(+)}_2,\ldots,u^{(+)}_N\right)\hm\in U$,
а~именно:
\begin{multline*}
\max\limits_{(u_1, u_2,\ldots, u_N)\in U} C\left(u_1, u_2,\ldots, u_N\right) = {}\\
{}=
C\left(u^{(+)}_1,u^{(+)}_2,\ldots,u^{(+)}_N\right)<\infty\,.
\end{multline*}
Тогда выполняется соотношение:
\begin{multline}
C(u_1, u_2,\ldots, u_N)=\fr{A(u_1, u_2,\ldots, u_N)}{B(u_1, u_2,\ldots, u_N)}
\leq{}\\
{}\leq C\left(u^{(+)}_1,u^{(+)}_2,\ldots,u^{(+)}_N\right)<\infty\,, 
\\
\left(u_1, u_2,\ldots, u_N\right)\in U\,.
\label{e39}
\end{multline}
Условия леммы~1 выполнены, и~можно воспользоваться ее утверждениями. 
Согласно первому из них, если выполняется неравенство~(\ref{e39}), 
то имеет место соотношение:
\begin{equation*}
I(\Psi_1, \Psi_2,\ldots, \Psi_N)\leq 
C\left(u^{(+)}_1,u^{(+)}_2,\ldots,u^{(+)}_N\right)<\infty 
%\label{e40}
\end{equation*}
для всех стратегий управления $\Psi\hm=(\Psi_1, \Psi_2,\ldots\linebreak
\ldots, \Psi_N)\hm\in \Gamma$.

Таким образом, множество значений дроб\-но-ли\-ней\-но\-го интегрального 
функционала $I(\Psi_1, \Psi_2,\ldots, \Psi_N)$ ограничено сверху при всех 
$\Psi\hm=(\Psi_1, \Psi_2,\ldots, \Psi_N)\hm\in \Gamma$. Тогда существует верхняя 
грань этого множества и~выполняется неравенство:
\begin{multline}
\sup\limits_{(\Psi_1, \Psi_2,\ldots, \Psi_N)\in \Gamma} 
I\left(\Psi_1, \Psi_2,\ldots, \Psi_N\right)\leq {}\\
{}\leq
C\left(u^{(+)}_1,u^{(+)}_2,\ldots,u^{(+)}_N\right). \label{e41}
\end{multline}
Рассмотрим детерминированную стратегию управ\-ле\-ния 
$\Psi^{*(+)}\hm=\left(\Psi_1^{*(+)}, \Psi_2^{*(+)},\ldots, \Psi_N^{*(+)}\right)$, 
в~которой каждая вероятностная мера~$\Psi_i^{*(+)}$ является вы\-рож\-ден\-ной 
и~сосредоточена в~точке $u_i^{(+)}$, $i\hm=\overline{1, N}$.
По свойству интеграла
\begin{multline}
I\left(\Psi_1^{*(+)}, \Psi_2^{*(+)},\ldots ,\Psi_N^{*(+)}\right)={}\\
{}=
C\left(u^{(+)}_1,u^{(+)}_2,\ldots,u^{(+)}_N\right). \label{e42}
\end{multline}
Из соотношений~(\ref{e41}) и~(\ref{e42}) получаем:
\begin{multline}
\sup\limits_{(\Psi_1, \Psi_2,\ldots, \Psi_N)\in \Gamma} 
I\left(\Psi_1, \Psi_2,\ldots, \Psi_N\right)\leq{}\\
{}\leq
 I\left(\Psi_1^{*(+)}, 
\Psi_2^{*(+)},\ldots, \Psi_N^{*(+)}\right). \label{e43}
\end{multline}
Заметим дополнительно, что выполняются отношения принадлежности:
\begin{equation}
\Psi^{*(+)}=\left(\Psi_1^{*(+)}, \Psi_2^{*(+)},\ldots, \Psi_N^{*(+)}\right) 
\in \Gamma^* \subset \Gamma\,. \label{e44}
\end{equation}
Из~(\ref{e44}) и~свойства верхней грани следует:
\begin{multline}
\sup\limits_{\left(\Psi_1^{*}, \Psi_2^{*},\ldots, \Psi_N^{*}\right) \in \Gamma^*} 
I\left(\Psi_1^{*}, \Psi_2^{*},\ldots, \Psi_N^{*}\right)\leq {}\\
{}\leq
\sup\limits_{\left(\Psi_1, \Psi_2,\ldots, \Psi_N\right) 
\in \Gamma} I\left(\Psi_1, \Psi_2,\ldots, \Psi_N\right)\,. 
\label{e45}
\end{multline}
Объединяя~(\ref{e42}), (\ref{e43}) и~(\ref{e45}), получаем соотношение:
\begin{multline}
\sup\limits_{\left(\Psi_1^{*}, \Psi_2^{*},\ldots, \Psi_N^{*}\right) 
\in \Gamma^*} I\left(\Psi_1^{*}, \Psi_2^{*},\ldots, 
\Psi_N^{*}\right)\leq{}\\
{}\leq \sup\limits_{\left(\Psi_1, \Psi_2,\ldots, \Psi_N\right) 
\in \Gamma} I\left(\Psi_1, \Psi_2,\ldots, \Psi_N\right)\leq{}\\
{}\leq I\left(\Psi_1^{*(+)}, \Psi_2^{*(+)},\ldots, \Psi_N^{*(+)}\right)={}\\
{}=
\fr{A\left(u^{(+)}_1,u^{(+)}_2,\ldots,u^{(+)}_N\right)}{B\left(u^{(+)}_1,u^{(+)}_2,
\ldots,u^{(+)}_N\right)}\,.
 \label{e46}
\end{multline}
Из соотношения~(\ref{e46}) с~учетом~(\ref{e44}) получаем, что максимум 
функционала $I(\Psi_1, \Psi_2,\ldots, \Psi_N)$ на множестве допустимых стратегий 
$\Psi\hm=(\Psi_1, \Psi_2,\ldots, \Psi_N)\hm\in \Gamma$ существует и~достигается 
на детерминированной стратегии $\left(\Psi_1^{*(+)}, \Psi_2^{*(+)},\ldots, 
\Psi_N^{*(+)}\right)$.

Кроме того, выполняются соотношения~(\ref{e19}). Таким образом, утверждение~1 
в~случае, когда основная функция $C(u_1, u_2,\ldots, u_N)$ достигает глобального 
максимума, доказано. Соответствующее утверждение в~случае, когда основная функция 
$C(u_1, u_2,\ldots, u_N)$ достигает глобального минимума, доказывается аналогично. 
При этом используется второе утверждение леммы~1.

\smallskip

Перейдем к~доказательству второго утверждения теоремы~2. Предположим, что основная 
функция $C(u_1, u_2,\ldots, u_N)\hm=A(u_1, u_2,\ldots$\linebreak
$\ldots, u_N)/{B(u_1, u_2,\ldots, u_N)}$ 
ограничена сверху, но не достигает глобального максимума на множестве 
$U \hm= U_1 \times U_2 \times \cdots \times U_N$. Тогда множество значений 
основной функции имеет конечную верхнюю грань:

\noindent
\begin{multline*}
C\left(u_1, u_2,\ldots, u_N\right)=\fr{A(u_1, u_2,\ldots, u_N)}
{B(u_1, u_2,\ldots, u_N)}<{}\\
{}<
\sup\limits_{(u_1, u_2,\ldots, u_N)\in U} \fr{A(u_1, u_2,\ldots, u_N)}
{B(u_1, u_2,\ldots, u_N)}<\infty\,, 
\\
\left(u_1, u_2,\ldots, u_N\right)\in U\,.
%\label{e47}
\end{multline*}
По определению верхней грани для любого фиксированного $\varepsilon \hm>0$ 
существует точка $(u_1^{(+)}(\varepsilon), u_2^{(+)}(\varepsilon),\ldots, 
u_N^{(+)}(\varepsilon))$ такая, что выполняется двойное неравенство~(\ref{e21}) 
(см.~\cite[гл.~1, \S\,3, п.~3.4]{25}). Иначе говоря, значение основной функции 
в~указанной точке лежит в~левой \mbox{$\varepsilon$-окрест}\-ности верхней грани. 
Рассмотрим детерминированную стратегию управления 
$\Psi^{*(+)}(\varepsilon)\hm=\!\left(\Psi_1^{*(+)}(\varepsilon), 
\Psi_2^{*(+)}(\varepsilon),\ldots, \Psi_N^{*(+)}(\varepsilon)\!\right)$, компонентами\linebreak 
которой являются вырожденные вероятностные меры $\Psi_1^{*(+)}(\varepsilon), 
\Psi_2^{*(+)}(\varepsilon),\ldots, \Psi_N^{*(+)}(\varepsilon)$, причем вырожденная 
мера~$\Psi_i^{*(+)}(\varepsilon)$ сосредоточена в~точке~$u_i^{(+)}(\varepsilon)$,
$i\hm=1,2,\ldots,N$.

По свойству интеграла
\begin{multline}
I\left(\Psi_1^{*(+)}(\varepsilon), \Psi_2^{*(+)}(\varepsilon),\ldots,
 \Psi_N^{*(+)}(\varepsilon)\right)={}\\
 {}=
 C\left(u_1^{(+)}(\varepsilon), u_2^{(+)}(\varepsilon),\ldots, 
 u_N^{(+)}(\varepsilon)\right)\,. 
 \label{e48}
\end{multline}
Из соотношения~(\ref{e48}) с~учетом указанного свойства основной функции получаем:
\begin{multline}
\sup\limits_{(u_1, u_2,\ldots, u_N)\in U} \fr{A(u_1, u_2,\ldots, u_N)}
{B(u_1, u_2,\ldots, u_N)}-\varepsilon<{}\\
{}< I\left(\Psi_1^{*(+)}(\varepsilon), 
\Psi_2^{*(+)}(\varepsilon),\ldots, \Psi_N^{*(+)}(\varepsilon)\right)<{}
\\
{}< \sup\limits_{(u_1, u_2,\ldots, u_N)\in U} \fr{A(u_1, u_2,\ldots, u_N)}
{B(u_1, u_2,\ldots, u_N)}<\infty\,. 
\label{e49}
\end{multline}
Заметим также, что в~рассматриваемом случае выполнены условия леммы~2. 
Воспользуемся первым утверждением этой леммы, а~именно соотношением~(\ref{e36}):
\begin{multline}
I(\Psi_1, \Psi_2,\ldots, \Psi_N)< {}\\
{}<\sup\limits_{(u_1, u_2,\ldots, u_N)
\in U} \fr{A(u_1, u_2,\ldots, u_N)}{B(u_1, u_2,\ldots, u_N)}<\infty 
\label{e50}
\end{multline}
для всех $(\Psi_1, \Psi_2,\ldots, \Psi_N)\in\Gamma$.

Из соотношений~(\ref{e49}) и~(\ref{e50}) следует, что детерминированная стратегия 
$\Psi^{*(+)}(\varepsilon)\hm=\left(\Psi_1^{*(+)}(\varepsilon), \Psi_2^{*(+)}(\varepsilon),
\ldots, \Psi_N^{*(+)}(\varepsilon)\right)$, опре\-де\-ля\-емая набором вырожденных 
вероятностных мер, сосредоточенных в~соответствующих точках 
$\left(u_1^{(+)}(\varepsilon), u_2^{(+)}(\varepsilon),\ldots, 
u_N^{(+)}(\varepsilon)\right)$, является $\varepsilon$-оп\-ти\-маль\-ной. 
Вторая часть утверждения~2 теоремы~2, связанная со свойствами нижней грани, 
доказывается аналогично.

Докажем третье утверждение теоремы~2. Предположим, что множество значений 
основной функции $C(u_1, u_2,\ldots, u_N)\hm=
A(u_1, u_2,\ldots$\linebreak $\ldots, u_N)/{B(u_1, u_2,\ldots, u_N)}$
не является ограниченным сверху на множестве $U\hm=U_1\times U_2 \times \cdots $\linebreak
$\cdots \times U_N$.
Тогда существует последовательность\linebreak точек $\left(u_1^{(+)}(n), u_2^{(+)}(n),
\ldots,u_N^{(+)}(n)\right)\hm\in U$, $n\hm=1,2,\ldots$, для которой
\begin{multline}
C\left(u_1^{(+)}(n), u_2^{(+)}(n),\ldots,u_N^{(+)}(n)\right)={}\\
{}=
\fr{A\left(u_1^{(+)}(n), u_2^{(+)}(n),\ldots,u_N^{(+)}(n)\right)}
{B\left(u_1^{(+)}(n), u_2^{(+)}(n),\ldots,u_N^{(+)}(n)\right)}
\longrightarrow \infty \,,\\
n\rightarrow \infty\,.
\label{e51}
\end{multline}
Зафиксируем некоторую последовательность точек $\left(u_1^{(+)}(n), u_2^{(+)}(n),
\ldots,u_N^{(+)}(n)\right)\hm\in U$, $n\hm=1,2,\ldots$, обладающих указанным свойством, 
и~рассмотрим последовательность детерминированных  стратегий управления 
$\Psi^{*(+)}(n)\hm=\left(\Psi_1^{*(+)}(n), \Psi_2^{*(+)}(n),\ldots, 
\Psi_N^{*(+)}(n)\right)$, $n\hm=1,2,\ldots$, определяемых набором вырожденных 
вероятностных мер, сосредоточенных в~соответствующих точках 
$\left(u_1^{(+)}(n), u_2^{(+)}(n),\ldots,u_N^{(+)}(n)\right)$, $n\hm=1,2,\ldots$ 
По свойству интеграла для любого фиксированного значения $n=1,2,\ldots$ 
выполняется равенство:
\begin{multline}
I \left(\Psi^{*(+)}(n)\right)={}\\
{}=I\left(\Psi_1^{*(+)}(n), \Psi_2^{*(+)}(n),\ldots,
 \Psi_N^{*(+)}(n)\right)={}\\
{}=\fr{A\left(u_1^{(+)}(n), u_2^{(+)}(n),\ldots,u_N^{(+)}(n)\right)}
{B\left(u_1^{(+)}(n), u_2^{(+)}(n),\ldots,u_N^{(+)}(n)\right)}\,. 
\label{e52}
\end{multline}
Из соотношений~(\ref{e51}) и~(\ref{e52}) следует, что
\begin{multline}
I\left(\Psi^{*(+)}(n)\right)={}\\
{}=I\left(\Psi_1^{*(+)}(n), \Psi_2^{*(+)}(n),\ldots, 
\Psi_N^{*(+)}(n)\right)\longrightarrow\infty\,,\\ 
n \rightarrow\infty\,.
 \label{e53}
\end{multline}
Соотношение~(\ref{e53}) означает, что множество значе\-ний дроб\-но-ли\-ней\-но\-го 
интегрального функциона\-ла $I(\Psi_1, \Psi_2,\ldots, \Psi_N)$ вида~(\ref{e11}) 
не ограничено сверху\linebreak на множестве наборов вырожденных вероятностных мер 
$\left(\Psi_1^{*(+)}(n), \Psi_2^{*(+)}(n),\ldots, \Psi_N^{*(+)}(n)\right)\hm\in\Gamma^*$, 
а~следовательно, и~на более широком\linebreak множестве наборов вероятностных 
мер $(\Psi_1, \Psi_2,\ldots$\linebreak $\ldots, \Psi_N)\hm\in\Gamma$. В~таком случае решения экстремальной 
задачи~(\ref{e18}) в~форме задачи на максимум не существует. Соответствующее утвержде\-ние 
для варианта, когда множество значений основной функции $C(u_1, u_2,\ldots,u_N)
\hm=A(u_1, u_2,\ldots$\linebreak $\ldots,u_N)/{B(u_1, u_2,\ldots,u_N)}$ 
не является ограниченным снизу, доказывается аналогично. Третье утверж\-де\-ние теоремы~2 
доказано. Тем самым тео\-ре\-ма~2 доказана полностью.

\smallskip

Применим теорему~2 для решения поставленной задачи оптимального управления. 
Из утверждения этой теоремы следует, что для доказательства су-\linebreak ществования 
оптимального управ\-ле\-ния и~его нахождения необходимо исследовать на 
глобальный экстремум основную функцию дроб\-но-ли\-ней\-но\-го интегрального 
функционала $C(u_1,u_2,\ldots,u_N)$, определяемую формулой~(\ref{e17}) с~учетом 
равенств~(\ref{e12})--(\ref{e16}). В~некоторых случаях, например когда основной 
процесс~$\xi(t)$ является регенерирующим, а~стоимостные характеристики 
модели задаются линейными функциями, такое исследование можно провести 
аналитически. Однако для подавляющего большинства полумарковских моделей 
для этого необходимо использовать численные методы.

\section{Заключение}

В заключительной части работы приведем \mbox{краткое} описание теоретической 
основы метода решения задачи оптимального управления полумарковским 
процессом с~конечным множеством состояний.

\begin{enumerate}[1.]
\item Исходная проблема оптимального управления формулируется в~виде 
экстремальной задачи~(\ref{e18}). Целевым показателем качества управ\-ле\-ния в~данной задаче 
служит величина~(\ref{e10}), которая имеет характер средней удельной прибыли.
\item Доказывается, что стационарный показатель~(\ref{e10}) представим в~виде 
дроб\-но-ли\-ней\-но\-го интегрального функционала~(\ref{e11}), для которого явно 
определяются подынтегральные функции числителя и~знаменателя, а~следовательно, 
и~основная функция данного функционала.
\item Используется теорема об экстремуме дроб\-но-ли\-ней\-но\-го интегрального 
функционала. На основании утверждений этой теоремы уста\-нав\-ли\-ва\-ет\-ся, что 
исходная задача оптимального управления сводится к~исследованию на глобальный 
экстремум основной функции этого функционала, для которой получено явное 
аналитическое представление.
\end{enumerate}

Заметим, что такое исследование задач оптимального управления 
стохастическими системами фактически уже было проведено в~ряде работ П.\,В.~Шнуркова 
и~его соавторов. В~частности, в~работе~\cite{26} была рассмотрена модель 
управления для обрывающегося процесса восстановления, описывающего функционирование 
некоторой технической системы. Задача управления решалась для различных показателей 
эффективности и~надежности этой системы, имеющих структуру дроб\-но-ли\-ней\-но\-го 
интегрального функционала.

В работах~\cite{27, 28} рассматривались модели регенерирующих процессов 
для исследования сис\-тем управления запасами. Различные показатели качества 
управления были представлены в~форме дроб\-но-ли\-ней\-ных интегральных функционалов. 
Основные функции этих функционалов были найде\-ны в~явной форме и~исследовались 
на глобальный экстремум. В~работах~\cite{21,29} рассматривалась достаточно 
сложная полумарковская модель с~конечным множеством состояний, описывающая 
сис\-те\-му управления запасом непрерывного продукта. Показатели качества управления в~этой 
модели также имели структуру дроб\-но-ли\-ней\-ных интегральных функционалов, 
для основных функций которых были найдены явные аналитические представления. 
Упомянем также работы~\cite{30, 31}, в~которых была исследована полумарковская 
модель с~дис\-крет\-но-не\-пре\-рыв\-ным фазовым пространством. Показатели 
качества управления в~этой  модели были найдены в~явной форме как функции от 
двух непрерывных параметров управления.

Фактически во всех упомянутых работах уже был использован метод решения задачи 
оптимального управления регенерирующим или полумарковским случайным процессом, 
основанный на исследовании экстремальных свойств основной функции соответствующего 
дроб\-но-ли\-ней\-но\-го интегрального функционала. Из соображений, изложенных 
во\linebreak введении, следует, что в~период написания и~пуб\-ли\-кации этих работ данный метод 
не имел стро\-гого обоснования. Однако после публикации\linebreak работы~\cite{14} и~настоящего 
исследования можно утверж\-дать, что полученные в~них результаты полностью теоретически 
обоснованы.

Таким образом, изложенный выше метод решения проблемы оптимального управления 
полумарковскими процессами с~конечными множествами состояний может быть успешно 
реализован для многих задач, рассматриваемых в~различных областях прикладной 
теории вероятностей.

Практическая реализация численной процедуры поиска оптимального решения на примере\linebreak 
полумарковской модели управления запасом непрерывного продукта (подробнее 
см.~\cite{21, 29}), ба\-зи\-ру\-юща\-яся на изложенных выше результатах (в~частности, 
теореме~1), была осуществлена А.\,К.~Горшениным и~соавторами 
в~статье~\cite{Gorshenin2015}. Коротко опишем наиболее важные аспекты этой работы.

Для решения поставленной задачи опти\-мального управления была создана 
специальная программа \verb"Inventory" на встроенном языке программирования 
пакета \verb"MATLAB", ее возможности\linebreak кратко представле\-ны в~упомянутой ранее 
\mbox{статье}~\cite{Gorshenin2015}. В~программе \verb"Inventory" реализованы функции 
для оценивания через заданные исходные параметры вероятностных и~стоимостных 
характеристик модели, которые в~дальнейшем используются для поиска значений 
основной функции дроб\-но-ли\-ней\-но\-го функционала~(\ref{e17}). Точка глобального 
экстремума этой функции и~определяет оптимальное управление.

В качестве начальных данных необходимо задание следующих параметров:
\begin{itemize}
\item спрос и~вместимость склада;
\item разбиение множества значений объема запаса;
\item вероятностные характеристики, описывающие модель пополнения запаса;
\item условные математические ожидания длительностей задержек пополнения запаса;
\item функции для характеризации затрат и~доходов.
\end{itemize}

По итогам работы программы \verb"Inventory" ряд вспомогательных функций 
представляется в~аналитической форме (в частности, с~использованием аппарата 
символьных вычислений  \verb"Symbolic Toolbox"\linebreak пакета \verb"MATLAB"), выводится 
точка глобального экстремума функции нескольких вещественных переменных~(\ref{e17}), 
найденная с~помощью применения численных и~при\-бли\-жен\-но-ана\-ли\-ти\-че\-ских\linebreak 
аппроксимаций. 
Также формируются графики оценок значений ве\-ро\-ят\-ност\-но-сто\-и\-мост\-ных 
характеристик 
и~основной функции дроб\-но-ли\-ней\-но\-го функционала~(\ref{e17}), либо трехмерных 
сечений в~случае наличия более трех параметров управления (переменных).

Функциональность пакета \verb"Inventory" может быть расширена для практической 
реализации метода решения задачи поиска оптимального управ\-ле\-ния полумарковскими 
процессами с~конечными множествами состояний, рассмотренного в~данной статье.


 {\small\frenchspacing
 {%\baselineskip=10.8pt
 \addcontentsline{toc}{section}{References}
 \begin{thebibliography}{99}
 \bibitem{1}
\Au{Ховард Р.} Динамическое программирование и~марковские процессы~/ 
Пер. с~англ.~--- М.: Сов. радио, 1964. 189~с.
(\Au{Howard~R.\,A.} Dynamic programming and Markov processes.~--- 
Cambridge, MA, USA: MIT Press, 1960. 136~p.)
\bibitem{2} 
\Au{Рыков В.\,В.} Управляемые марковские процессы с~конечными пространствами 
состояний и~управлений~// Теория вероятностей и~ее применения, 1966. Т.~11. 
Вып.~2. С.~343--351.
\bibitem{3} 
\Au{Джевелл В.} Управляемые полумарковские процессы~// Кибернетич. сборник.~--- 
М.: Мир, 1967. Вып.~4. С.~97--134.
%{\em Jewell W.\,S.} Markov-renewal programming~// Operations Research, 1963. Vol.~11. P.~938--971.
\bibitem{4} 
\Au{Fox B.} Markov renewal programming by linear fractional programming~// 
SIAM J.~Appl. Math., 1966. Vol.~14. P.~1418--1432.
\bibitem{5} 
\Au{Denardo E.\,V.} Contraction mappings in the theory underlying dinamic programming~// 
SIAM Rev., 1967. Vol.~9. P.~165--177.

\bibitem{6} 
\Au{Howard R.\,A.} Research in semi-Markovian decision structures~// 
J.~Oper. Res. Soc. Japan, 1963. Vol.~6. P.~163--199.
\bibitem{7} 
\Au{Osaki S., Mine H.} Linear programming algorithms for Markovian decision processes~//
 J.~Math. Anal. Appl., 1968. Vol.~22. P.~356--381.
\bibitem{8} 
\Au{Майн Х., Осаки С.} Марковские процессы принятия решений~/ Пер. с~англ.~--- 
М.: Наука, 1977. 176~с.
(\Au{Mine~H., Osaki~S.} 
Markovian decision processes.~--- New York, NY, USA: 
American Elsevier Publishing Co., 1970. 142~p.)
\bibitem{9} 
\Au{Гихман И.\,И., Скороход А.\,В.} Управляемые случайные процессы.~--- 
Киев: Наукова думка, 1977. 251~с.
\bibitem{10} 
\Au{Luque-Vasquez F., Herndndez-Lerma~О.} Semi-Markov control models with average costs~// 
Appl. Math., 1999. Vol.~26. No.\,3. P.~315--331.
\bibitem{11} 
\Au{Vega-Amaya O., Luque-Vasquez~F.} Sample-path average cost optimality for 
semi-Markov control processes on Borel spaces: Unbounded costs and mean holding times~// 
Appl. Math., 2000. Vol.~27. No.\,3. P.~343--367.
\bibitem{12} 
Вопросы математической теории надежности~/ Под ред. Б.\,В. Гнеденко.~--- 
М.: Радио и~связь, 1983. 376~с.
\bibitem{13} 
\Au{Барзилович Е.\,Ю., Каштанов~В.\,А.} Некоторые математические вопросы теории 
обслуживания сложных систем.~---  М.: Сов. радио, 1971. 272~с.
\bibitem{14} 
\Au{Шнурков П.\,В.} О~решении проблемы безусловного экстремума для 
дроб\-но-ли\-ней\-но\-го интегрального функционала на множестве вероятностных мер~// 
Докл. РАН. Сер. Математика, 2016. Т.~470. №\,4. C.~387--392.
\bibitem{15} 
\Au{Ширяев А.\,Н.}  Вероятность.~--- М.:~МЦНМО, 2011. Кн.~1. 552~с. Кн.~2. 968~с.
\bibitem{16} 
\Au{Боровков А.\,А.} Теория вероятностей.~--- М.: Либроком, 2009. 656~c.
\bibitem{17} 
\Au{Хеннекен П.\,Л., Тортра А.} Теория вероятностей 
и~некоторые ее приложения.~--- М.: Наука, 1974. 472~c.
\bibitem{18} 
\Au{Халмош П.} Теория меры~/ Пер. с~англ.~--- М.: ИЛ, 1953. 282~c.
(\Au{Halmos~P.} Measure theory.~--- Litton Educational Publishing, Inc. 1950. 304~p.)
\bibitem{19} 
\Au{Королюк В.\,С., Турбин~А.\,Ф.} Полумарковские процессы и~их приложения.~--- 
Киев:~Наукова думка, 1976. 184~с.
\bibitem{20} 
\Au{Janssen J., Manca R.} Applied semi-Markov processes.~--- New York,
NY, USA: Springer, 2006. 309~p.
\bibitem{21} 
\Au{Шнурков П.\,В., Иванов~А.\,В.} Анализ дискретной полумарковской модели
 управления запасом непрерывного продукта при периодическом прекращении потребления~// 
 Дискретная математика, 2014. Т.~26. Вып.~1. С.~143--154.
\bibitem{22} 
\Au{Иванов~А.\,В.} Анализ дискретной полумарковской модели
 управления запасом непрерывного продукта при периодическом прекращении 
 потребления.~--- М.: НИУ ВШЭ, 2014.  Дисс.\ \ldots\ канд. физ.-мат. наук. 120~с.
\bibitem{23}  %23
\Au{Bajalinov~E.\,B.} Linear-fractional programming. 
Theory, methods, applications and software.~--- 
Boston/\linebreak Dordrecht/London: Kluwer Academic Publs., 2003. 423~p.

\bibitem{27} %27
\Au{Шнурков П.\,В., Мельников~Р.\,В.} Оптимальное управление запасом 
непрерывного продукта в~модели регенерации~// Обозрение прикладной 
и~промышленной математики, 2006. Т.~13. Вып.~3. С.~434--452.
\bibitem{28} 
\Au{Шнурков П.\,В., Мельников~Р.\,В.} 
Исследование проб\-ле\-мы управления запасом непрерывного продукта при детерминированной 
задержке поставки~// Автоматика и~телемеханика, 2008. Т.~10. С.~93--113.


\bibitem{24}  %26
\Au{Шнурков П.\,В.} Методы исследования задач оптимального обслуживания 
в~математической теории надежности.~--- 
М.: МИЭМ, 1983.  Дисс.\ \ldots\ канд. физ.-мат. наук.

 \bibitem{25}  %25
\Au{Кудрявцев Л.\,Д.} Курс математического анализа. Т.~1.~--- 
М.: Дрофа, 2006. 704~с.

\bibitem{26} %24
\Au{Шнурков П.\,В.} Оптимальное обслуживание на периоде 
до первого отказа системы~// Применение аналитических методов в~вероятностных
 задачах.~--- Киев: Институт математики АН УССР, 1986. С.~121--129.

\bibitem{29} 
\Au{Шнурков П.\,В., Иванов~А.\,В.} Исследование задачи оптимизации в~дискретной 
полумарковской модели управления непрерывным запасом~// Вестник МГТУ им.\ 
Н.\,Э. Баумана. Сер.\ Естественные науки, 2013. Т.~3. Вып.~50. С.~62--87.
\bibitem{30} 
\Au{Shnourkoff P.\,V.} The two-element system with one 
restoring device optimum maintenance~// Stoch. Anal. Appl., 1997. 
Vol.~15. No.\,5. P.~823--837.
\bibitem{31} 
\Au{Shnourkoff P.\,V.} The two-element system optimum maintenance tills the first fail~// 
Stoch. Anal. Appl., 2001. Vol.~19. No.\,6. P.~1005--1024.
\bibitem{Gorshenin2015} 
\Au{Gorshenin~A.\,K., Belousov~V.\,V., Shnourkoff~P.\,V.,
Ivanov~A.\,V.} Numerical research of the optimal control problem in the semi-Markov 
inventory model~// AIP Conference Proceedings, 2015. Vol.~1648. {250007}. 4~p.
%\bibitem{33} {\em Горшенин А.\,К., Белоусов В.\,В., Шнурков П.\,В.} 2016. Система управления запасами на основе стохастических полумарковских моделей. Свидетельство о государственной регистрации программы для ЭВМ \textnumero 2016619021.
 \end{thebibliography}

 }
 }

\end{multicols}

\vspace*{-6pt}

\hfill{\small\textit{Поступила в~редакцию 15.07.16}}

%\vspace*{8pt}

\newpage

\vspace*{-24pt}

%\hrule

%\vspace*{2pt}

%\hrule

%\vspace*{8pt}


\def\tit{ANALYTICAL SOLUTION OF~THE~OPTIMAL CONTROL TASK OF~A~SEMI-MARKOV 
PROCESS WITH~FINITE SET OF~STATES}

\def\titkol{Analytical solution of~the~optimal control task of~a~semi-Markov 
process with~finite set of~states}

\def\aut{P.\,V.~Shnurkov$^{1}$, A.\,K.~Gorshenin$^{2}$, and~V.\,V.~Belousov$^{2}$}

\def\autkol{P.\,V.~Shnurkov, A.\,K.~Gorshenin, and~V.\,V.~Belousov}

\titel{\tit}{\aut}{\autkol}{\titkol}

\vspace*{-9pt}


    
\noindent
$^1$National Research University Higher School of Economics, 34~Tallinskaya Str., 
Moscow, 123458, Russian\linebreak
$\hphantom{^9}$Federation

\noindent
$^2$Institute of Informatics Problems, Federal Research Center 
``Computer Science and Control'' of the Russian\linebreak
$\hphantom{^9}$Academy of Sciences, 44-2~Vavilova Str., 
Moscow 119333, Russian Federation



\def\leftfootline{\small{\textbf{\thepage}
\hfill INFORMATIKA I EE PRIMENENIYA~--- INFORMATICS AND
APPLICATIONS\ \ \ 2016\ \ \ volume~10\ \ \ issue\ 4}
}%
 \def\rightfootline{\small{INFORMATIKA I EE PRIMENENIYA~---
INFORMATICS AND APPLICATIONS\ \ \ 2016\ \ \ volume~10\ \ \ issue\ 4
\hfill \textbf{\thepage}}}

\vspace*{3pt}


\Abste{The theoretical verification of the new method of finding 
the optimal strategy of control of a~semi-Markov process with finite set of states is 
presented. The paper considers Markov randomized strategies of control, determined by 
a~finite collection of probability measures, corresponding to each state. The quality 
characteristic is the stationary cost index. This index is a~linear-fractional integral 
functional, depending on collection of probability measures, giving the strategy of control. 
Explicit analytical forms of integrands of numerator and denominator of this 
linear-fractional integral functional are known. The basis of consequent results is 
the new generalized and strengthened form of the theorem about an extremum of 
a~linear-fractional integral functional. It is proved that problems of existence 
of an optimal control strategy of a~semi-Markov process and finding this strategy 
can be reduced to the task of numerical analysis of global extremum for 
the given function, depending on finite number of real arguments.}

\KWE{optimal control of a~semi-Markov process; stationary cost index of quality control; 
linear-fractional integral functional}




\DOI{10.14357/19922264160408} 

\vspace*{-16pt}

\Ack
\noindent
The research was partially supported by the Russian Foundation 
for Basic Research (project 15-07-05316).



%\vspace*{3pt}

  \begin{multicols}{2}

\renewcommand{\bibname}{\protect\rmfamily References}
%\renewcommand{\bibname}{\large\protect\rm References}

{\small\frenchspacing
 {%\baselineskip=10.8pt
 \addcontentsline{toc}{section}{References}
 \begin{thebibliography}{99}
\bibitem{1-1}
\Aue{Howard,~R.\,A.} 1960. \textit{Dynamic programming and Markov processes}. 
Cambridge, MA: MIT Press. 136~p.
\bibitem{2-1}
\Aue{Rykov,~V.\,V.} 1966. Upravlyaemye markovskie protsessy 
s~konechnymi prostranstvami sostoyaniy i~upravleniy 
[Controlled Markov processes with finite spaces of states and controls ]. 
\textit{Teoriya veroyatnostey i~ee primeneniya} 
[Theory of Probability and Its Applications] 11(2):343--351.
\bibitem{3-1}
\Aue{Jewell,~W.\,S.} 1963. Markov-renewal programming. 
\textit{Oper. Res.} 11:938--971.
\bibitem{4-1}
\Aue{Fox,~B.} 1966. Markov renewal programming by linear fractional programming. 
\textit{SIAM J.~Appl. Math.} 14:1418--1432.
\bibitem{5-1}
\Aue{Denardo, E.\,V.} 1967. Contraction mappings in the theory underlying dinamic 
programming. \textit{SIAM Rev.} 9:165--177.
\bibitem{6-1}
\Aue{Howard,~R.\,A.} 1963. Research in semi-Markovian decision structures. 
\textit{J.~Oper. Res. Soc. Japan} 6:163--199.
\bibitem{7-1}
\Aue{Osaki,~S., and H.~Mine.} 1968. Linear programming algorithms 
for Markovian decision processes. \textit{J.~Math. Anal. Appl.} 22:356--381.
\bibitem{8-1}
\Aue{Mine,~H., and S.~Osaki.} 1970. 
\textit{Markovian decision processes}. New York, NY: Elsevier. 142~p.
\bibitem{9-1}
\Aue{Gikhman,~I.\,I., and A.\,V.~Skorokhod.} 1977. 
\textit{Upravlyaemye sluchaynye protsessy} 
[Controlled random processes]. Kiev: Naukova Dumka. 251~p.
\bibitem{10-1}
\Aue{Luque-Vasquez,~F., and О.~Herndndez-Lerma.} 1999. 
Semi-Markov control models with average costs. \textit{Appl. Math.} 26(3):315--331.
\bibitem{11-1}
\Aue{Vega-Amaya,~O., and  F.~Luque-Vasquez.} 2000.  
Sample-path average cost optimality for semi-Markov control processes on Borel spaces: 
Unbounded costs and mean holding times. \textit{Appl. Math.} 27(3):343--367.
\bibitem{12-1}
Gnedenko,~B.~V., ed. 1983. 
\textit{Voprosy matematicheskoy teorii nadezhnosti} 
[Problems of the mathematical theory of reliability].  Moscow: Radio i~svyaz'. 376~p.
\bibitem{13-1}
\Aue{Barzilovich,~E.\,Yu., and V.\,A.~Kashtanov.} 1971. 
\textit{Nekotorye matematicheskie voprosy teorii obsluzhivaniya slozhnykh sistem}  
[Some mathematical questions in theory of complex systems maintenance]. 
Moscow: Sovetskoe radio. 272~p.
\bibitem{14-1}
\Aue{Shnurkov,~P.\,V.} 2016. Solution of the unconditional extremum problem for 
a~linear-fractional 
integral functional on a~set of probability measures. 
\textit{Dokl. Math.} 94(2):550--554.
\bibitem{15-1} %14
\Aue{Shiryaev,~A.\,N.} 2016. 
\textit{Probability-1}. Graduate texts in mathematics ser.
New York, NY: Springer. Vol.~95. 503~p.;
2017. \textit{Probability-2.} Vol.~900. 500~p.
\bibitem{16-1}
\Aue{Borovkov,~А.\,А.} 2009. 
\textit{Teoriya veroyatnostey} [Probability theory]. Moscow: Librokom. 656~p.
\bibitem{17-1}
\Aue{Khenneken,~P.\,L., and A.~Tortra.} 1974. 
\textit{Teoriya veroyatnostey i~nekotorye ee prilozheniya} 
[Probability theory and some of its applications]. Moscow: Nauka. 472~p.
\bibitem{18-1}
\Aue{Halmos,~P.} 1950. \textit{Measure theory}. Litton Educational Publishing. 304~p.
\bibitem{19-1}
\Aue{Korolyuk, V.\,S., and A.\,F.~Turbin.} 1976. 
\textit{Polumarkovskie protsessy i~ikh prilozheniya} 
[Semi-Markov processes and their applications]. Kiev: Naukova Dumka. 184~p.
\bibitem{20-1}
\Aue{Janssen,~J., and R.~Manca.} 2006. 
\textit{Applied semi-Markov processes}. New York, NY: Springer. 309~p.
\bibitem{21-1}
\Aue{Shnurkov,~P.\,V, and A.\,V~Ivanov.} 2015. Analysis of a~discrete semi-Markov model of continuous inventory 
control with periodic interruptions of consumption. 
\textit{Discrete Math. \mbox{Appl}.} 25(1):59--67.
\bibitem{22-1} %21
\Aue{Ivanov,~A.\,V.} 2014. Analiz diskretnoy polumarkovskoy modeli upravleniya 
zapasom nepreryvnogo produkta pri periodicheskom prekrashchenii potrebleniya 
[Analysis of a~discrete semi-Markov control model of continuous product inventory 
in a~periodic cessation of consumption].  
Moscow: Natsional'nyy Issledovatel'skiy Universitet ``Vysshaya Shkola Ekonomiki.'' 
PhD Thesis. 120~p.
\bibitem{23-1} %22
\Aue{Bajalinov,~E.\,B.} 2003. 
\textit{Linear-fractional programming. Theory, methods, applications and software}. 
Boston/\linebreak Dordrecht/London: Kluwer Academic Publs. 423~p.
\bibitem{26-1} %24
\Aue{Shnurkov,~P.\,V., and R.\,V.~Mel'nikov.} 2006. Optimal'noe upravlenie 
zapasom nepreryvnogo produkta v modeli regeneratsii [Optimal control of 
a~continuous product inventory in the regeneration model]. 
\textit{Obozrenie prikladnoy i~promyshlennoy matematiki} [Rev. Appl. Ind. Math.]
13(3):434--452.

\bibitem{25-1} %25
\Aue{Shnurkov,~P.\,V., and R.\,V.~Mel'nikov.} 2008. 
Analysis of the problem of continuous-product inventory control under deterministic 
lead time. \textit{Automat. Rem. Contr.} 69(10):1734--1751.

\columnbreak

\bibitem{24-1} %26
\Aue{Shnurkov,~P.\,V.} 1983. Metody issledovaniya zadach optimal'nogo obsluzhivaniya 
v~matematicheskoy teorii nadezhnosti [Research methods of optimal service problems 
in the mathematical theory of reliability].  
Moscow: Moskovskiy Institut Elektronnogo Mashinostroeniya.  PhD Thesis. 


\bibitem{27-1} %27
\Aue{Kudryavtsev,~L.\,D.} 2006. 
\textit{Kurs matematicheskogo analiza} 
[A~course of mathematical analysis]. Vol.~1. Moscow: Drofa. 704~p.

\bibitem{28-1}
\Aue{Shnurkov,~P.\,V.} 1986. Optimal'noe obsluzhivanie na periode do 
pervogo otkaza sistemy [The optimum service period until the first system failure]. 
\textit{Primenenie analiticheskikh metodov v~veroyatnostnykh zadachakh} 
[The application of analytical methods in probabilistic tasks]. Kiev:
Institute of Mathematics of the Academy of Sciences of the USSR. 121--129.

\bibitem{29-1}
\Aue{Shnurkov,~P.\,V., and A.\,V.~Ivanov.} 2013. Issledovanie zadachi optimizatsii 
v~diskretnoy polumarkovskoy modeli upravleniya nepreryvnym zapasom 
[Study of the optimization problem in discrete semi-Markov model of continuous 
inventory control]. \textit{Vestnik MGTU im.\ N.\,E.~Baumana. Ser. 
Estestvennye nauki} [Vestnik of MSTU named after N.\,E.~Bauman. Ser. Natural sciences] 
3(50):62--87.
\bibitem{30-1}
\Aue{Shnourkoff,~P.\,V.} 1997. The two-element system with one restoring device 
optimum maintenance.  \textit{Stoch. Anal. Appl.} 15(5):823--837.
\bibitem{31-1}
\Aue{Shnourkoff,~P.\,V.} 2001. The two-element system optimum maintenance tills 
the first fail. \textit{Stoch. Anal. Appl.} 19(6):1005--1024.
\bibitem{32-1}
\Aue{Gorshenin,~A.\,K., V.\,V.~Belousov, P.\,V.~Shnourkoff, and A.\,V.~Ivanov.}
2015. Numerical research of the optimal control problem in the semi-Markov 
inventory model. \textit{AIP Conference Proceedings} 1648:250007.
\end{thebibliography}

 }
 }

\end{multicols}

\vspace*{-3pt}

\hfill{\small\textit{Received July 15, 2016}}

\Contr

\noindent
\textbf{Shnurkov Peter V.} (b.\ 1953)~---
 Candidate of Science (PhD) in physics and mathematics, 
 associate professor, National Research University Higher School of Economics, 
 34~Tallinskaya Str., Moscow 123458, Russian Federation; \mbox{pshnurkov@hse.ru} 
 
 \vspace*{3pt}
 
 \noindent
\textbf{Gorshenin Andrey K.}  (b.\ 1986)~---
Candidate of Science (PhD) in physics and mathematics, leading scientist, 
Institute of Informatics Problems, Federal Research Center ``Computer Science 
and Control'' of the Russian Academy of Sciences, 44-2~Vavilov Str., Moscow 119333, 
Russian Federation; associate professor, Federal State Budget Educational 
Institution of Higher Education ``Moscow Technological University,'' 
78~Vernadskogo Ave., Moscow 119454, Russian Federation;
\mbox{agorshenin@frccsc.ru}

\vspace*{3pt}

\noindent
\textbf{Belousov Vasiliy V.} (b.\ 1977)~---
Candidate of Science (PhD) in technology, senior scientist, Institute of 
Informatics Problems, Federal Research Center ``Computer Science and Control'' 
of the Russian Academy of Sciences, 44-2~Vavilov Str., Moscow 119333, Russian 
Federation; \mbox{VBelousov@ipiran.ru}
\label{end\stat}


\renewcommand{\bibname}{\protect\rm Литература}  %7

%\def\F{{\rm I\!F}}
\def\P{{\rm I\!P}}

\def\stat{peshkova}

\def\tit{ГРАНИЦЫ НЕЗАВЕРШЕННОЙ РАБОТЫ В~СИСТЕМЕ С~ПОВТОРНЫМИ ВЫЗОВАМИ РАЗНЫХ КЛАССОВ 
И~ПОКАЗАТЕЛЬНЫМ ВРЕМЕНЕМ ОБСЛУЖИВАНИЯ$^*$}

\def\titkol{Границы незавершенной работы в~системе с~повторными вызовами разных классов 
и~показательным временем} % обслуживания}

\def\aut{И.\,В.~Пешкова$^1$}

\def\autkol{И.\,В.~Пешкова}

\titel{\tit}{\aut}{\autkol}{\titkol}

\index{Пешкова И.\,В.}
\index{Peshkova I.\,V.}


{\renewcommand{\thefootnote}{\fnsymbol{footnote}} \footnotetext[1]
{Работа выполнена при финансовой поддержке РНФ (проект 21-71-10135).}}


\renewcommand{\thefootnote}{\arabic{footnote}}
\footnotetext[1]{Петрозаводский государственный университет; 
Институт прикладных математических исследований Карельского 
научного центра Российской академии наук, \mbox{iaminova@petrsu.ru}}


%\vspace*{-12pt}



\Abst{Исследуется односерверная система 
с~повторными вызовами и~пуассоновским входным потоком, в~которую поступает~$M$ 
классов заявок.
Для системы с~временами обслуживания, имеющими показательное распределение, 
получены верхняя и~нижняя границы для стационарной незавершенной работы. 
В~качестве нижней границы выступает   стационарная незавершенная работа 
в~классической сис\-те\-ме  $M/H_M/1$ с~гиперэкспоненциальным временем обслуживания. 
Верхней границей служит незавершенная работа в~сис\-те\-ме, в~которой к~времени 
обслуживания добавляется дополнительное время, равное интервалу между попытками 
попасть на сервер с~самой <<медленной орбиты>>. Полученные результаты численного 
моделирования подтверждают теоретические выводы.}


\KW{система с~повторными вызовами; незавершенная 
работа; стохастическая упо\-ря\-до\-чен\-ность} 

\DOI{10.14357/19922264230408}{UOKQRD}
  
%\vspace*{-6pt}


\vskip 10pt plus 9pt minus 6pt

\thispagestyle{headings}

\begin{multicols}{2}

\label{st\stat}

\section{Введение}

Модели систем с~повторными вызовами широко используются для моделирования  
телефонных станций, кол-цент\-ров, сис\-тем связи, телекоммуникационных сетей. 
В~работах~\cite{Ar1, Ar3} \mbox{изложены}  приложения и~математические методы анализа 
сис\-тем c~повторными вызовами, включая сис\-те\-мы с~постоянной интенсивностью 
повторов. В~работе~\cite{F86}   телефонная станция была смоделирована  с~по\-мощью 
такой сис\-те\-мы. Аналогичная модель используется для описания широкого класса 
протоколов множественного доступа~\cite {CSA92, CRP93}. В~част\-ности, в~работе~\cite{BG92} 
показано, что постоянная интенсивность повторных вызовов  снижает 
интенсивность попыток (при незапланированном множественном доступе) обратно 
пропорционально числу задержанных пакетов. В~результате общая ско\-рость повторной 
обработки становится нечувствительной к~виртуальному <<размеру орбиты>> (числу 
отложенных пакетов). Более того, сис\-те\-мы с~повторными вызовами с~постоянной 
интенсивностью вызовов использовались для описания TCP-тра\-фи\-ка с~короткими HTTP-со\-еди\-не\-ни\-ями~\cite{AY08,AY10} 
и~оп\-ти\-ко-элект\-ри\-че\-ской гиб\-рид\-ной схемой разрешения 
конфликтов~\cite{Wongetal09,Yaoetal02}.

 Большинство же современных моделей повторных вызовов имеют сложную природу или 
конфигурацию, и~поэтому для их исследования применяются численные методы или 
имитационное моделирование.

Ранее в~работе~\cite{mathematics2022} была доказана тео\-ре\-ма о~верх\-ней и~ниж\-ней 
границах стационарной незавершенной работы  для исходной сис\-те\-мы с~повторными 
вызовами с~постоянной интенсивностью вызовов  (см.\ тео\-ре\-му~1 ниже). Эта тео\-ре\-ма 
стала основой анализа, развитого в~данной \mbox{статье}.

В данном исследовании рассматривается частный случай  односерверной сис\-те\-мы  
с~повторными вызовами с~пуассоновским входным потоком и~показательным 
распределением времени обслуживания, при этом  время обслуживания  и~время 
нахождения на орбите зависят от класса заявки~$k$.
%
Для такой системы предлагается строить две классические сис\-те\-мы с~неограниченной 
очередью (с~ожиданием) типа $M/G/1$: в~первой (минорантной) сис\-те\-ме время 
обслуживания пред\-став\-ля\-ет собой конечную смесь времен обслуживания заявок всех 
классов (т.\,е.\ имеет гиперэкспоненциальное распределение), во второй 
(мажорантной) сис\-те\-ме ко времени обслуживания первой сис\-те\-мы добавляется 
дополнительное время, равное  интервалу между вызовами  с~самой <<медленной 
орбиты>>.   Более того, для минорантной сис\-те\-мы получено распределение 
стационарного времени ожидания в~явном виде для трех классов ($M\hm=3$).
Сис\-те\-мы, в~которых  время обслуживания задано в~виде конечной смеси 
распределений, обсуждались ранее  в~работах~\cite{pesh-mor2022, pesh2022}.
 
Структура статьи следующая. В~разд.~2 приведено описание модели сис\-те\-мы 
с~повторными вызовами, минорантной и~мажорантной сис\-тем, а~также основная тео\-ре\-ма, 
полученная авторами в~работе~\cite{mathematics2022}.
В~разд.~3 получены коэффициенты загрузки, математические ожидания 
незавершенной нагрузки, а~также преобразования Лап\-ла\-са--Стилть\-еса для 
незавершенной нагрузки в~минорантной и~мажорантной сис\-те\-мах с~показательным 
распределением времени обслуживания. В~разд.~4 приведены результаты численного 
эксперимента для случая трех классов. При этом параметры для минорантной системы 
были использованы такие же, как в~работе~\cite{rego}, в~которой получена  
функция распределения  стационарного времени пребывания. Отметим, что  в~работе~\cite{rego} 
неверно указано, что полученное распределение~--- это распределение 
стационарного времени ожидания. В~статье получена функция распределения  
стационарного времени ожидания для минорантной сис\-те\-мы  в~явном  виде. Для 
исходной сис\-те\-мы с~повторными вызовами и~мажорантной  сис\-те\-мы проведены 
численные эксперименты и~построены эмпирические функции распределения. 
Полученные результаты численного эксперимента иллюстрируют доказанную 
стохастическую упо\-ря\-до\-чен\-ность стационарной незавершенной работы в~рассмотренных 
сис\-те\-мах.

\section{Описание модели}

Рассмотрим односерверную систему с~повторными вызовами~$\Sigma$, в~которой 
обслуживаются~$M$~классов заявок. Заявки $k$-го класса поступают в~сис\-те\-му в~соответствии 
с~пуассоновским потоком с~па\-ра\-мет\-ром~$\lambda_k$, $k\hm=1,\ldots,M$. 
Если заявка застает сервер пустым, то она немедленно обслуживается, в~противном 
случае, если сервер занят,  заявка уходит на $k$-ю орбиту бесконечного объема, 
образуя очередь в~порядке поступления на орбиту, $k\hm=1,\ldots,M$. Первая 
в~очереди на $k$-й орбите заявка производит независимые попытки присоединиться 
к~обслуживанию на сервере через экспоненциальное  время~$\eta_k$.
Интенсивность вызовов с~орбиты не зависит от размера орбиты (т.\,е.\ от числа 
заявок на орбите), но может зависеть от класса орбиты~$k$. Такие сис\-те\-мы 
называют сис\-те\-ма\-ми  с~постоянной ин\-тен\-сив\-ностью вызовов.

Обозначим через $t_n$ момент прихода $n$-й заявки в~общий пуассоновский входной  
поток (образованный суперпозицией~$M$~входных потоков, $t_1\hm=0$),   $S_n{(k)}$~--- 
время обслуживания  $n$-й заявки  $k$-го класса,  $k\hm=1,\ldots,M$, $n\hm\ge1$. Пусть 
последовательность независимых одинаково распределенных (н.\,о.\,р.)\ интервалов 
между приходами заявок  $\{T_n:=t_{n+1}\hm-t_n,\ n\hm\ge 1\}$ и~последовательность 
времен обслуживания  $\{S_n{(k)},\ n\hm\ge1\}$ независимы для каждого  $k$-го 
класса.
Предположим, что интервалы между приходами заявок с~(непустой)  $k$-й орбиты~$\eta_k$ 
распределены показательно и~не зависят от размера орбиты  (числа заявок 
на $k$-й орбите). Время обслуживания заявок $k$-го класса  $S(k)$ имеет 
произвольное распределение  с~функцией распределения (ф.~р.)\ $F_{S(k)}$, 
$k\hm=1,\ldots, M$. (Далее в~обозначениях  опускаем индекс номера заявки.)  Обозначим
\begin{equation*}
%\label{rates}
\lambda=\sum\limits_{k=1}^M\lambda_k ;\ \ \ \rho_k=\lambda_k\mathbb{E} S{(k)}; \enskip 
\rho=\sum\limits_{k=1}^M \rho_k.
\end{equation*}
Пусть $W(t)$ есть незавершенная работа в~момент времени~$t^-$, и~предположим, 
что система в~начальный момент времени пуста: $W(0)\hm=0$. Обозначим 
$W_n=W(t_n)$, $n\hm\ge1$.
Известно~\cite{Morozov2019}, что неравенство
 \begin{equation}
 \label{stability}
 \rho + \max\limits_{k=1,\ldots, M} \fr{\lambda}{\lambda+\eta_k} < 1
 \end{equation}
служит достаточным условием стационарности сис\-те\-мы.  При  таком условии 
существует  предел
$$
W_n \Rightarrow W\,,\quad n\to\infty
$$
(где обозначим $\Rightarrow$ схо\-ди\-мость по распределению). Функция распределения~$F_W$ 
стационарной незавершенной работы~$W$ для исходной сис\-те\-мы~$\Sigma$ неизвестна. 
С~другой стороны, $W$ служит важной характеристикой качества обслуживания 
сис\-те\-мы. Ниже мы построим верхнюю и~нижнюю границы~$F_W$, используя 
классические  $M/G/1$ сис\-те\-мы с~неограниченной оче\-редью, в~которых время 
обслуживания пред\-став\-ля\-ет\-ся конечной смесью распределений.


Рассмотрим две новые системы: \textit{минорантную сис\-те\-му}~$\Sigma^{(1)}$ 
и~\textit{мажорантную сис\-те\-му}~$\Sigma^{(2)}$. (Далее индекс~$(i)$ означает номер 
сис\-те\-мы.)
Входной поток во все три сис\-те\-мы~--- пуассоновский  с~параметром~$\lambda$ (это 
суперпозиция входных потоков, образованных заявками разных классов).

Пусть в~минорантной сис\-те\-ме~$\Sigma^{(1)}$ время обслуживания $S^{(1)}\hm=S$ задано 
конечной смесью распределений вида
\begin{equation}
\label{mixture}
S=\sum\limits_{k=1}^M I(k) S(k), \enskip n\ge1\,,
\end{equation}
где  $I(k)$~--- индикатор, такой что  $\mathbb{E} I(k)\hm=p_k=\lambda_k/\lambda$; $S(k)$~--- время  обслуживания заявки $k$-го класса.
Заметим, что сис\-те\-ма~$\Sigma^{(1)}$ пред\-став\-ля\-ет собой классическую 
(с~дисциплиной обслуживания первым при\-шел\,--\,пер\-вым обслужен) односерверную сис\-те\-му 
типа $M/G/1$ с~неограниченной очередью, в~которой время обслуживания~(\ref{mixture})
является конечной смесью времен обслуживания заявок всех классов исходной 
сис\-те\-мы.

В мажорантной системе~$\Sigma^{(2)}$~--- классической односерверной сис\-те\-ме 
типа  $M/G/1$ с~неограниченной оче\-редью~--- каждая заявка обслуживается на 
сервере в~течение времени~$S$, заданного соотношением~\eqref{mixture},  плюс 
время~$\xi_0$, имеющее показательное распределение с~па\-ра\-мет\-ром  
$$
\mu_0=\min\limits_{1\le k\le M} (\lambda\hm+\eta_k),
$$
 т.\,е.
%\begin{equation*}
%\label{sums2}
$S^{(2)} \hm= S\hm +\xi_0$.
%\end{equation*}
Таким образом, случайная величина (с.\,в.)~$\xi_0$ соответствует самой 
<<медленной>> орбите (с наибольшими интервалами между попытками). Заметим, что 
мажорантная сис\-те\-ма~$\Sigma^{(2)}$ имеет другой коэффициент загрузки,
\begin{equation*}
%\label{rho2def}
 \rho^{(2)}=\lambda \mathbb{E} S+\fr{\lambda}{\mu_0}=\rho+\fr{\lambda}{\mu_0}\,,
\end{equation*}
и условие стационарности~\eqref{stability} для нее принимает вид:
$$
\rho^{(2)}<1.
$$

В работе~\cite{mathematics2022}  доказана сле\-ду\-ющая тео\-ре\-ма, в~которой даны 
верхняя и~нижняя границы незавершенной работы~$W$ в~исходной сис\-те\-ме 
с~повторными вызовами~$\Sigma$.

\smallskip

\noindent
\textbf{Теорема~1.}
\textit{Пусть сис\-те\-мы  $\Sigma^{(1)}$, $\Sigma^{(2)}$ и~$\Sigma$ в~начальный момент времени 
пустые, т.\,е.}
 \begin{equation*}
W_1^{(1)}=W_1=W_1^{(2)}=0\,.
 \end{equation*}
\textit{Тогда при  выполнении условия}~\eqref{stability} \textit{стационарные времена 
незавершенной работы стохастически упорядочены}:
 \begin{equation}
 \label{theor1-1}
 W^{(1)}\underset{\mathrm{st}}\le W \underset{\mathrm{st}}\le W^{(2)},
 \end{equation}
\textit{где $W^{(1)}\le_{\mathrm{st}} W$ означает $\overline F_{W^{(1)}} (x) \hm\le 
\overline F_{W} (x) $ для любого $x\hm\ge 0$, $\overline F_{W^{(1)}}  (x)\hm= 1\hm-  
F_{W^{(1)}} (x)$}.


\smallskip

В следующем разделе применим данный результат для сис\-те\-мы, в~которой~$M$~классов 
заявок, име\-ющих  показательное распределение времени обслуживания.

\section{Границы незавершенной работы~$W$ в~системе с~показательным 
обслуживанием разных классов }

Пусть в~исходной сис\-те\-ме с~повторными вызовами~$\Sigma$ времена обслуживания 
$k$-го класса~$S(k)$ имеют показательное распределение с~ф.\,р.
\begin{equation}
\label{hyperexp}
F_{S(k)}(x)= 1- e^{-\mu_k x}, \enskip x\ge 0\,, \ \mu_k >0\,.
\end{equation}

В качестве минорантной сис\-те\-мы~$\Sigma^{(1)}$ рассмотрим сис\-те\-му 
с~неограниченной  очередью  $M/H_M/1$, в~которой времена обслуживания $S^{(1)}\hm=S$ 
имеют гиперэкспоненциальное распределение (пред\-став\-ля\-ют\-ся $M$-ком\-по\-нент\-ной 
смесью показательно распределенных с.\,в.~$S(k)$) с~ф.\,р.
\begin{multline*}
F_{S^{(1)}}(x) = 1 -  \sum\limits_{k=1}^M p_k e^{-\mu_k x}, \enskip \mu_k > 0\,, \\ 
p_k\ge 0\,,\enskip k=1,\ldots,M, \enskip \sum\limits_{i=k}^M p_k=1\,.
\end{multline*}

Обозначим коэффициент загрузки  $\rho^{(1)} \hm=\sum\nolimits_{k=1}^M \lambda 
p_k/\mu_k$ в~сис\-те\-ме~$\Sigma^{(1)}$. Поскольку
$$
\rho^{(1)} \le \rho + \fr{\lambda}{\mu_0 }\,,
$$
то, если условие стационарности~\eqref{stability} выполнено,    сис\-те\-ма~$\Sigma^{(1)}$ также стационарна.

Рассмотрим преобразование Лап\-ла\-са--Стилть\-еса:
\begin{equation*}
%\label{lstdef}
 \psi_{S_e}(z)=\int\limits_0^\infty e^{-zt} \,dF_{S_e}(t),
\end{equation*}
где $F_{S_e}$ -- так называемый \textit{интегрированный хвост
распределения} с~плот\-ностью
\begin{equation*}
%\label{fequilibr}
f_{S_e}(x)=\fr{1}{\mathbb{E} S}\, \overline F_S(x),\enskip x\ge0\,.
% f_{S_e}(x)=\mu \overline{F}_S(x),\enskip x\ge0\,.
\end{equation*}

Распределение $F_{S_e}$ соответствует распределению стационарного перескока 
процесса вос\-ста\-нов\-ле\-ния, фор\-ми\-ру\-емо\-го по\-сле\-до\-ва\-тель\-ностью н.\,о.\,р.\ времен 
обслуживания~$\{S_n\}$~\cite{Asmus}.

В работе~\cite{mathematics2022} доказано, что преобразование\linebreak Лап\-ла\-са--Стилть\-еса 
стационарной незавершенной работы~$W^{(1)}$ выражается через преобразования 
Лап\-ла\-са--Стилть\-еса компонент смеси времен обслуживания в~сле\-ду\-ющем виде:
\begin{multline}
\label{lstformixture}
\psi_{W^{(1)}}(z)=\fr{1-\rho}{z\left(1-\rho\sum\nolimits_{k=1}^M 
(\rho_k/{\rho}) \psi_{S_e(k)}(z)\right)}={}\\
{}=\fr{1-\rho}{z\left(1-
\sum\nolimits_{k=1}^M \rho_k \psi_{S_e(k)}(z)\right)}.
\end{multline}

Преобразование Лапласа--Стилть\-еса для показательного распределения хорошо 
известно: 
$$
\psi_{S_e(k)}(z) = \fr{\mu_k}{\mu_k +z}\,.
$$ 
Подставляя его в~соотношение~\eqref{lstformixture},  получим
$$
\psi_{W^{(1)}}(z)=\fr{1-\sum\nolimits_{k=1}^M \lambda_k/\mu_k}{z\left(1-
\sum\nolimits_{k=1}^M  {\lambda_k}/(\mu_k +z)   
\right)}\,.
$$

Применяя формулу По\-ла\-чи\-ка--Хин\-чи\-на, получим среднюю стационарную незавершенную 
работу  в~сис\-те\-ме~$\Sigma^{(1)}$ в~виде:
\begin{equation}
\label{ew5}
\mathbb{E}  W^{(1)} = \fr{\lambda \mathbb{E} ( S^{(1)})^2}{2(1-\rho^{(1)})} = 
\fr{\sum\nolimits_{k=1}^M \rho_k^2 +\rho^2}{2\lambda (1-\rho)}\,.
\end{equation}


Рассмотрим теперь мажорантную сис\-те\-му~$\Sigma^{(2)}$,  время обслуживания 
в~которой равно сумме с.\,в.~$S$ с~гиперэкспоненциальным распределением~\eqref{hyperexp} и~с.~в.~$\xi_0$, т.\,е.\
 $\ S^{(2)}\hm=S \hm+ \xi_0$. Обозначим для простоты 
$\mu_0\hm=\min\nolimits_{1\le k\le M} (\lambda\hm+\eta_k)$, и~пусть с.\,в.~$\xi_0$ имеет 
показательное распределение  с~па\-ра\-мет\-ром~$\mu_0$. Условие стационарности 
в~такой сис\-те\-ме совпадает с~\eqref{stability}.

Известно \cite{mathematics2022}, что в~сис\-те\-ме~$\Sigma^{(2)}$ преобразование 
Лап\-ла\-са--Стилть\-еса для стационарной незавершенной работы~$W^{(2)}$ имеет 
сле\-ду\-ющий вид:
\begin{multline}
\label{reslemma2}
 \psi_{W^{(2)}} (z)=
 \left(1-\rho- \fr{\lambda}{\mu_0}\right) \Bigg/
\left(  z\left(
\vphantom{\left(\sum\limits_{k=1}^M\right)}
1-{}\right.\right.\\
\left.\left.{}-\fr{\mu_0}{\mu_0+z}\left(\sum\limits_{k=1}^M \rho_k \psi_{S_e(k)}(z) +
\fr{\lambda}{\mu_0}\right)\right)\right)\,.
\end{multline}
Подставив $\psi_{S_e(k)}(z) = \mu_k/(\mu_k \hm+z)$ в~\eqref{reslemma2}, получим
\begin{multline*}
\psi_{W^{(2)}} (z)=
\left(1-\sum\limits_{k=1}^M \fr{\lambda_k}{\mu_k}-\fr{\lambda}{\mu_0}\right) \Bigg/
\left(z\left(
\vphantom{\left(\sum\limits_{k=1}^M\right)}
1-{}\right.\right.\\
\left.\left.{}-\fr{\mu_0}{\mu_0+z}\left(\sum\limits_{k=1}^M \fr{\lambda_k}{\mu_k+z}  +\fr{\lambda}{\mu_0}\right)\right)\right)\,.
\end{multline*}
Аналогично формуле~\eqref{ew5} можно получить среднюю стационарную незавершенную 
работу
\begin{multline}
\label{ew6}
\mathbb{E}  W^{(2)} = \fr{\lambda \mathbb{E} \left( S^{(2)}\right)^2}{2\left(1-\rho^{(2)}\right)} = {}\\
{}=
\fr{\mu_0^2\left(\rho^2 + \sum\nolimits_{k=1}^M \rho_k^2 \right) +2 \lambda (\lambda +\rho \mu_0)}{2\lambda \mu_0 (\mu_0 -\rho \mu_0 -\lambda)}.
\end{multline}



Из теоремы~1 следует, что
стационарные времена незавершенной работы стохастически упорядочены:
 $$
 W^{(1)}\underset{\mathrm{st}}\le W \underset{\mathrm{st}}\le W^{(2)},
 $$
а следовательно,  их математические ожидания также упорядочены~\cite{Ross}:
$$
 \mathbb{E} W^{(1)}\le \mathbb{E} W \le \mathbb{E} W^{(2)}.
$$

Действительно, легко проверить, что
\begin{multline*}
\mathbb{E}  W^{(2)} = \fr{\lambda \mathbb{E} \left( S^{(2)}\right)^2}{2(1-\rho^{(2)})} = {}\\
{}=
\fr{\mu_0^2 \lambda \mathbb{E} ( S^{(1)})^2 + 2 \lambda (\lambda +\rho \mu_0)} {\mu_0^2 2(1-\rho^{(1)}) - 2\lambda^2 \mu_0 } 
\ge \mathbb{E}  W^{(1)}.
\end{multline*}


\section{Численный эксперимент}

В качестве примера рассмотрим систему с~повторными вызовами~$\Sigma$ с~тремя 
классами заявок ($M\hm=3$), в~которую поступает пуассоновский поток 
с~ин\-тен\-сив\-ностью $\lambda\hm=10$.
Пусть
$p_1\hm=1/2$, $p_2\hm=1/3$, $p_3\hm=1/6$, $\mu_1\hm=10$, $\mu_2\hm=30$ и~$\mu_3\hm=60$.
Будем полагать, что~$\eta_k$ принимают значения
$\eta_1\hm=50$, $\eta_2\hm=100$ и~$\eta_3\hm=150.$
В~этом случае $\mu_0\hm=\lambda\hm+\eta_1\hm=60$,  коэффициенты загрузки \mbox{равны}
$$
\rho^{(1)}=\rho= \sum\limits_{k=1}^3 \fr{\lambda p_k}{\mu_k} = \fr{23}{36}\,; \enskip  
\rho^{(2)} = \rho+\fr{\lambda}{\mu_0} = \fr{29}{36} < 1
$$
и условие стационарности~\eqref{stability} выполнено. По 
формулам~\eqref{ew5}--\eqref{ew6} находим математические ожидания $\mathbb{E} W^{(1)} \hm\approx 0{,}093$ 
и~$\mathbb{E} W^{(2)} \hm\approx 0{,}106$.

Рассмотрим минорантную сис\-те\-му~$\Sigma^{(1)}$, в~которой время обслуживания 
имеет гиперэкспоненциальное распределение
$$
\overline F_S(x)=\sum\limits_{k=1}^3 p_k e^{-\mu_k x}.
$$
%
Для такой классической системы $M/G/1$ в~работе~\cite{rego} получено 
распределение числа клиентов в~сис\-те\-ме в~стационарном режиме~$N^{(1)}$ в~сле\-ду\-ющем виде:
$$
\pi_n^\ast=\mathbb{P}\left\{N^{(1)}=n\right\}=\sum\limits_{k=1}^3 \beta_k (r_k)^n,
$$
где параметры  $\beta_k$ и~$r_k$ вычислены и~равны
\begin{alignat*}{3}
    \beta_1&=0{,}040;&\enskip \beta_2&=0{,}075;&\enskip \beta_3&=0{,}245; \\
    r_1&=0{,}146;&\enskip r_2&=0{,}268;&\enskip r_3&=0{,}711.
\end{alignat*}
В работе~\cite{rego} с~помощью результата~\cite{haji}  получено стационарное 
распределение  \textit{времени пребывания} клиента в~сис\-те\-ме~$V^{(1)}$ с~хвостом 
ф.\,р.\ вида
$$
\overline F_{V^{(1)}}(x)=\sum\limits_{k=1}^3 \gamma_k e^{-\theta_k x},
$$
где коэффициенты $\gamma_k$  и~па\-ра\-мет\-ры~$\theta_k$  для исходных параметров 
\mbox{равны}
\begin{alignat*}{3}
\gamma_1&=0{,}047;&\quad \gamma_2&=0{,}103; &\quad \gamma_3&=0{,}850; \\
\theta_1&=58{,}633; &\quad \theta_2&=27{,}307;&\quad \theta_3&= 4{,}059.
\end{alignat*}
При этом в~утверждении теоремы~3 из работы ~\cite{rego} было  ошибочно указано, 
что $\sum\nolimits_{k=1}^3 \gamma_k\hm=1\hm-\rho^{(1)}$ (что было бы верно, если бы 
распределение~$F_V^{(1)}$ соответствовало  \textit{времени ожидания}). На самом 
деле легко проверить, что $\sum\nolimits_{k=1}^3 \gamma_k\hm=1$. Для исправления этой 
неточности повторим вывод, получив выражение для стационарного времени 
\textit{ожидания} в~сис\-те\-ме (что  соответствует незавершенной работе в~сис\-те\-ме 
в~момент прихода клиента). Заметим, что~$\pi_{n+1}^\ast$ есть стационарная 
вероятность наблюдать~$n$~клиентов в~очереди, т.\,е.
$$
\mathbb{P}\{Q^{(1)}=n\}=\pi_{n+1}^\ast,
$$
где $Q^{(1)}$ есть число клиентов \textit{в очереди} в~стационарном режиме.

Вычислив производящую функцию вероятностей~$\pi (z)$ для~$Q^{(1)}$, получим
\begin{multline*}
    \pi (z)=\sum\limits_{n=0}^\infty z^n \pi_{n+1}^\ast = \sum\limits_{k=1}^3
    \fr{\beta_k }{z}\sum\limits_{n=1}^\infty (r_k z)^n={}\\
    {}=\sum\limits_{k=1}^3 \fr{\beta_k }{z}\left(\fr{1}{1-r_k z}-1\right)=\sum\limits_{k=1}^3 \fr{\beta_k r_k}{1-r_k z}\,.
\end{multline*}

C другой стороны,  производящая функция стационарной очереди~$\pi(z)$ и~преобразование Лап\-ла\-са--Стилть\-еса для стационарного времени 
ожидания~$\psi_{W^{(1)}}(z)$ связаны формулой:
\begin{multline*}
\pi(z)= \sum\limits_{n=0}^{\infty}  \int\limits_{0^-}^{\infty} z^n e^{-\lambda x} \fr{(\lambda x )^n}{n!}\, dF_{W^{(1)}} (x) ={}\\
{}=
\int\limits_{0^-}^{\infty} e^{-(\lambda-\lambda z) x} \, d F_{W^{(1)}} (x) ={}\\
{}=\psi_{W^{(1)}} (\lambda - \lambda z) + \left(1-\rho^{(1)}\right),
\end{multline*}
 где $F_{W^{(1)}}(0) = (1-\rho^{(1)})$~--- скачок  ф.~р.\ в~нуле. Сделав замену 
переменной $s\hm=\lambda\hm-\lambda z$,  получим
$$
\psi_{W^{(1)}} (s)=\sum_{k=1}^3 \fr{\beta_k r_k}{1-r_k(1-
s/\lambda)}=\sum\limits_{k=1}^3 \fr{\beta_k r_k}{1-r_k}\,\fr{\theta_k}{\theta_k+ s}\,,
$$
где, как и~в~работе~\cite{rego},
$$
\theta_k=\fr{\lambda(1-r_k)}{r_k}\,.
$$

{ \begin{center}  %fig1
 \vspace*{-1pt}
     \mbox{%
\epsfxsize=79mm 
\epsfbox{pes-1.eps}
}

\end{center}



\noindent
{\small{Функции распределения в~нижней~$\Sigma^{(1)}$~(\textit{1}), исходной $\Sigma$~(\textit{2}) 
и~верх\-ней~$\Sigma^{(2)}$~(\textit{3}) сис\-те\-мах при $\lambda \hm= 10$, $p_1\hm= 1/2$, $p_2\hm= 1/3$, $p_3\hm=1/6$, 
$\mu_1\hm=10$, $\mu_2\hm=30$, $\mu_3\hm= 60$, $\eta_1\hm=50$, $\eta_2\hm=100$ и~$\eta_3\hm=150$}}}

\vspace*{12pt}

\noindent
Таким образом,~$\psi_{W^{(1)}}$ соответствует взвешенной  сумме показательных 
распределений. Отметим при этом, что, в~отличие от~\cite{rego}, коэффициенты 
смеси имеют вид:
$$
\hat\gamma_k=\fr{\beta_k r_k}{1-r_k}\,.
$$
Таким образом,
\begin{equation}
\label{fwlower}
\overline F_{W^{(1)}}(x)=\sum\limits_{k=1}^3 \hat\gamma_k e^{-\theta_k x} + \left(1- \rho^{(1)}\right),
\end{equation}
где
$\hat\gamma_1=0{,}007$, $\hat\gamma_2\hm=0{,}027$ и~$\hat\gamma_3\hm=0{,}604$. Заметим, что 
$\sum\nolimits_{k=1}^3 \hat\gamma_k\hm=\rho^{(1)}\hm\approx 0{,}638\hm <1.$

Воспользуемся выражением~\eqref{fwlower} для по\-стро\-ения ф.\,р.\ 
в~сис\-те\-ме~$\Sigma^{(1)}$, а~для по\-стро\-ения оценок в~исходной~$\Sigma$ и~верхней~$\Sigma^{(2)}$ 
сис\-те\-мах воспользуемся имитационным моделированием.
Построим графики (эмпирических) ф.~р.\ для незавершенной 
работы в~трех сис\-те\-мах. Как видно на рисунке, стохастический 
порядок~\eqref{theor1-1} для стационарных времен ожидания выполнен, что 
и~следовало ожидать.






\section{Заключение}

В работе показано, что для исходной системы с~повторными вызовами можно 
построить минорантную и~мажорантную системы так, что стационарная незавершенная 
нагрузка во всех трех сис\-те\-мах будет стохастически упорядочена. Численный 
эксперимент для сис\-те\-мы с~показательными временами обслуживания подтверждает 
теоретические выводы. При этом в~качестве примера рас\-смот\-ре\-ны такие па\-ра\-мет\-ры 
(как в~работе~\cite{rego}), для которых получена ф.\,р.\ 
стационарного времени ожидания в~явном виде в~минорантной сис\-теме.


{\small\frenchspacing
 { %\baselineskip=10.6pt
 %\addcontentsline{toc}{section}{References}
 \begin{thebibliography}{99}

\bibitem{Ar1}
\Au{Artalejo J.\,R.} {Accessible bibliography on retrial queues}~// Math. 
Comput. Model., 1999. Vol.~30. Iss.~3-4. P.~1--6. doi: 10.1016/S0895-7177(99)00128-4.


\bibitem{Ar3}
\Au{Artalejo J.,   Gomez-Corral~A.}
{Retrial queueing systems: A~computational approach}.~---   Springer, 2008. 318~p.
doi: 10.1007/978-3-540-78725-9.


\bibitem{F86} 
\Au{Fayolle G.}
A~simple telephone exchange with delayed feedbacks~// 
Seminar (International) on Teletraffic Analysis and Computer Performance 
Evaluation Proceedings.~--- Elseiver Science, 1986. P.~245--253.

\bibitem{CSA92}
\Au{Choi~B.\,D.,  Shin~Y.\,W.,  Ahn~W.\,C.}
Retrial queues with collision arising from unslotted {CSMA/CD} protocol~//
Queueing Syst., 1992.  Vol.~11. P.~335--356. doi: 10.1007/ BF01163860.

\bibitem{CRP93}
\Au{Choi B.\,D., Rhee~K.\,H., Park~K.\,K.} {The $M/G/1$ retrial queue with
retrial rate control policy}~//
Probab.  Eng. Inform. Sc., 1993.  Vol.~7. P.~29--46. doi: 10.1017/ S0269964800002771.

\bibitem{BG92}
\Au{Bertsekas D., Gallager~R.}
{Data networks}.~--- Athena Scientific, 2021.  570~p.

\bibitem{AY08}
\Au{Avrachenkov K., Yechiali~U.}
Retrial networks with finite buffers and their application to Internet data 
traffic~//  Probab. Eng. Inform. Sc., 2008. 
Vol.~22. P.~519--536. doi: 10.1017/S0269964808000314.

\bibitem{AY10} %8
\Au{Avrachenkov K., Yechiali~U.}
{On tandem blocking queues with a~common retrial queue}~// Comput.  
Oper. Res., 2010. Vol.~37. Iss.~7. P.~1174--1180. doi: 10.1016/j.cor.2009. 10.004.



\bibitem{Yaoetal02} %9
\Au{Yao S., Xue~F.,  Mukherjee~B.,  Yoo~S.\,J.\,B., Dixit~S.}
{Electrical ingress buffering and traffic aggregation for optical packet 
switching and their effect on TCP-level performance in optical mesh networks}~//
IEEE Commun. Mag., 2002.
Vol.~40. Iss.~9. P.~66--72. doi: 10.1109/MCOM. 2002.1031831.

\bibitem{Wongetal09} %10
\Au{Wong E.\,W.\,M.,  Andrew L.\,L.\,H.,  Cui~T.,  Moran~B.,  Zalesky~A., Tucker~R.\,S., Zukerman~M.}
{Towards a~bufferless optical internet}~//
J.~Lightwave Technol., 2009. Vol.~27. P.~2817--2833. doi: 10.1109/JLT.2009.2017211.

\bibitem{mathematics2022} %11
\Au{Morozov E.\,V., Peshkova~I.\,V., Rumyantsev~A.\,S.} Bounds and maxima for the 
workload in a~multiclass orbit queue~// Mathematics, 2023. Vol.~11. Iss.~3. 
Art.~564. doi: 10.3390/math11030564.

\bibitem{pesh-mor2022}  %12
\Au{Peshkova I., Morozov~E.} On comparison of 
multiserver systems with multicomponent mixture distributions~// J.~Math. Sci., 2022. Vol.~267. No.\,2. P.~260--272.
doi: 10.1007/ s10958-022-06132-z. 

\bibitem{pesh2022} %13
\Au{Пешкова И.\,В.} 
Границы экстремального индекса времени ожидания в~системе
$M/G/1$ с~распределением времени обслуживания в~виде конечной
смеси~// Информатика и~её применения, 2022.
Т.~16. Вып.~2. С.~26--33. doi: 10.14357/19922264220405. EDN: VFKRKT.



\bibitem{rego} %14
\Au{Rego V.}
Some explicit formulas for mixed exponential service systems~//
Computers Operations Research, 1988. Vol.~15. Iss.~6. P.~509--520. doi: 
{10.1016/0305-0548(88)90047-0}.

\bibitem{Morozov2019}  %15
\Au{Morozov E.\,V.,   Rumyantsev~A.\,S., Dey~S.,  Deepak~T.\,G.}
Performance analysis and stability of multiclass orbit queue with constant 
retrial rates and balking~//
 Perform. Evaluation, 2019.  Vol.~134. Art.~102005. doi: 
10.1016/ J.PEVA.2019.102005.

\bibitem{Asmus} %16
\Au{Asmussen S.} Applied probability and queues. Stochastic modelling and 
applied probability.~--- New York, NY, USA: Springer-Verlag, 2003. 438~p.

\bibitem{Ross} %17
\Au{Ross S., Shanthikumar~J., Zhu~Z.}  On increasing-failure-rate random 
variables~// J.~Appl. Probab., 2005. Vol.~42. P.~797--809. doi: 
10.1239/jap/1127322028.

\bibitem{haji} %18
\Au{Haji R.,  Newell~G.\,F.}  A~relation between stationary queue and waiting 
time distributions~// J.~Appl. Probab., 1971. Vol.~8. P.~617--620. doi: 10.2307/3212186.




\end{thebibliography}

 }
 }

\end{multicols}

\vspace*{-10pt}

\hfill{\small\textit{Поступила в~редакцию 26.08.23}}

\vspace*{8pt}

%\pagebreak

%\newpage

%\vspace*{-28pt}

\hrule

\vspace*{2pt}

\hrule



\def\tit{BOUNDS OF THE WORKLOAD IN~A~MULTICLASS RETRIAL QUEUE WITH~EXPONENTIAL SERVICES}


\def\titkol{Bounds of the workload in~a~multiclass retrial queue with~exponential services}


\def\aut{I.\,V.~Peshkova$^{1,2}$}

\def\autkol{I.\,V.~Peshkova}

\titel{\tit}{\aut}{\autkol}{\titkol}

\vspace*{-10pt}


\noindent 
$^1$Petrozavodsk State University, 33~Lenina Pr., Petrozavodsk 185910, Russian Federation

\noindent 
$^2$Karelian Research Center of
the Russian Academy of Sciences, 11~Pushkinskaya Str., Petrozavodsk 185910,\linebreak
$\hphantom{^1}$Russian Federation 

\def\leftfootline{\small{\textbf{\thepage}
\hfill INFORMATIKA I EE PRIMENENIYA~--- INFORMATICS AND
APPLICATIONS\ \ \ 2023\ \ \ volume~17\ \ \ issue\ 4}
}%
 \def\rightfootline{\small{INFORMATIKA I EE PRIMENENIYA~---
INFORMATICS AND APPLICATIONS\ \ \ 2023\ \ \ volume~17\ \ \ issue\ 4
\hfill \textbf{\thepage}}}

\vspace*{3pt}

 


\Abste{A~multiclass retrial queue with Poisson input and $M$ classes of customers is investigated. 
For the given retrial system with exponential service times, the lower and upper bounds of the workload are derived. 
It is shown that the workload in the classical system $M/H_M/1$ with hyperexponential service times is the lower bound for the workload of the given retrial system. 
The upper bound is the workload in the classical $M/G/1$ system where each customer occupies the server for the given service time and additional
 time corresponding to the inter-retrial time from the ``slowest'' orbit. 
The presented simulation results confirm the theoretical conclusions.}


\KWE{retrial queue; workload; stochastic ordering}  




\DOI{10.14357/19922264230408}{UOKQRD}

\vspace*{-12pt}

\Ack

\vspace*{-4pt}

\noindent
The research has been prepared with the support of the Russian Science Foundation according to
the research project No.\,21-71-10135. 



  \begin{multicols}{2}

\renewcommand{\bibname}{\protect\rmfamily References}
%\renewcommand{\bibname}{\large\protect\rm References}

{\small\frenchspacing
 {%\baselineskip=10.8pt
 \addcontentsline{toc}{section}{References}
 \begin{thebibliography}{99} 
%1
\bibitem{Ar1-1}
\Aue{Artalejo, J.\,R.} 1999. Accessible bibliography on retrial queues. \textit{Math.
Comput. Model.} 30(3-4):1--6. doi: 10.1016/S0895-7177(99)00128-4.
%2
\bibitem{Ar3-1}
\Aue{Artalejo, J., and A.~Gomez-Corral.} 2008. 
\textit{Retrial queueing systems: A computational approach}. Springer. 318~p.
doi: 10.1007/978-3-540-78725-9.
%3
\bibitem{F86-1} 
\Aue{Fayolle, G.} 1986. 
A simple telephone exchange with delayed feedbacks. \textit{Seminar (International) on Teletraffic Analysis and Computer Performance Evaluation Proceedings}.
Elseiver Science. 245--253.
%4
\bibitem{CSA92-1}
\Aue{Choi, B.\,D., Y.\,W.~Shin, and W.\,C.~Ahn.} 1992. 
Retrial queues with collision arising from unslotted \mbox{CSMA}/CD protocol. \textit{Queueing Syst.} 11:335--356.  
doi: 10.1007/ BF01163860.
%5
\bibitem{CRP93-1}
\Aue{Choi, B.\,D., K.\,H.~Rhee, and K.\,K.~Park.} 1993. 
The $M/G/1$ retrial queue with retrial rate control policy.
\textit{Probab. Eng. Inform. Sc.} 7(1):29--46.
doi: 10.1017/ S0269964800002771.
%6
\bibitem{BG92-1}
\Aue{Bertsekas, D., and R.~Gallager.} 2021.
\textit{Data networks}. Athena Scientific. 570~p.
%7
\bibitem{AY08-1}
\Aue{Avrachenkov, K., and U.~Yechiali.} 2008.
Retrial networks with finite buffers and their application to Internet data traffic. \textit{Probab. Eng. Inform. Sc.} 22(4):519--536.
doi: 10.1017/S0269964808000314.
%8
\bibitem{AY10-1} 
\Aue{Avrachenkov, K., and U.~Yechiali.} 2010.
On tandem blocking queues with a~common retrial queue. \textit{Comput. Oper. Res.} 37(7):1174--1180.
doi: 10.1016/j.cor.2009.10.004.

%9
\bibitem{Yaoetal02-1}
\Aue{Yao, S., F.~Xue, B.~Mukherjee, S.\,J.\,B.~Yoo, and S.~Dixit.} 2002.
Electrical ingress buffering and traffic aggregation for optical packet switching and their
effect on TCP-level performance in optical mesh networks.
\textit{IEEE Commun. Mag.} 40(9):66--72. doi: 10.1109/MCOM.2002.1031831.

%10
\bibitem{Wongetal09-1}
\Aue{Wong, E.\,W.\,M., L.\,L.\,H.~Andrew, T.~Cui, B.~Moran, A.~Zalesky, R.\,S.~Tucker, and M.~Zukerman.} 2009.
Towards a~bufferless optical internet.
\textit{J.~Lightwave Technol.} 27(14):2817--2833. doi: 10.1109/JLT.2009.2017211.

%11
\bibitem{mathematics2022-1}
\Aue{Morozov, E.\,V., I.\,V.~Peshkova, and A.\,S.~Rumyantsev.}
 2023. Bounds and maxima for the workload in a~multiclass orbit queue. \textit{Mathematics} 11(3):564. doi: 10.3390/ math11030564.

%12
\bibitem{pesh-mor2022-1} 
\Aue{Peshkova, I., and E.~Morozov.} 2022. On comparison of multiserver systems with multicomponent mixture distributions. 
\textit{J.~Math. Sci.} 267(2):260--272. doi: 10.1007/ s10958-022-06132-z.
%13
\bibitem{pesh2022-1}
\Aue{Peshkova, I.\,V.} 2022. Granitsy ekstremal'nogo in\-dek\-sa vre\-me\-ni ozhi\-da\-niya v~sis\-te\-me $M/G/1$ 
s~raspredeleniem vremeni obsluzhivaniya v~vide konechnoy
smesi [On bounds of the stationary waiting time extremal index in $M/G/1$
system with mixture service times]. \textit{Informatika i~ee Primeneniya~--- Inform. Appl.} 16(4):26--33. doi: 10.14357/19922264220405. EDN: VFKRKT.

%14
\bibitem{rego-1}
\Aue{Rego, V.} 1988. 
Some explicit formulas for mixed exponential service systems. 
\textit{Comput. Oper. Res.} 15(6):509--520. doi: 10.1016/0305-0548(88)90047-0.
%15
\bibitem{Morozov2019-1} 
\Aue{Morozov, E.\,V., A.\,S.~Rumyantsev, S.~Dey, and T.\,G.~Deepak.} 2019.
Performance analysis and stability of multiclass orbit queue with constant retrial rates and balking.
\textit{Perform. Evaluation} 134:102005. doi: 10.1016/ J.PEVA.2019.102005.
%16
\bibitem{Asmus-1}
\Aue{Asmussen, S.} 2003. \textit{Applied probability and queues. Stochastic modelling and 
applied probability.} New York, NY: Springer. 438~p.

%17
\bibitem{Ross-1}
\Aue{Ross, S., J.~Shanthikumar, and Z.~Zhu.}
 2005. On increasing-failure-rate random variables. \textit{J.~Appl. Probab.} 42(3):797--809. doi: 10.1239/jap/1127322028.
 
 %18
\bibitem{haji-1}
\Aue{Haji, R., and G.\,F.~Newell.} 1971. A~relation between stationary queue and waiting time distributions. \textit{J. Appl. Probab.} 8(3):617--620.
doi: 10.2307/3212186.

\end{thebibliography}

 }
 }

\end{multicols}

\vspace*{-6pt}

\hfill{\small\textit{Received August 26, 2023}} 

%\vspace*{-18pt}

\Contrl

\vspace*{-4pt}

\noindent
\textbf{Peshkova Irina V.} (b.\ 1975)~--- 
Candidate of Science (PhD) in physics and mathematics, associate professor, Petrozavodsk State University, 33~Lenina Pr., Petrozavodsk 185910, 
Russian Federation; senior scientist, Karelian Research Center of the Russian Academy of Sciences, 
11~Pushkinskaya Str., Petrozavodsk 185910, Russian Federation; \mbox{iaminova@petrsu.ru}


\label{end\stat}

\renewcommand{\bibname}{\protect\rm Литература}  %8
\def\stat{ivanova}

\def\tit{МОДЕЛИ СОВМЕСТНОГО ОБСЛУЖИВАНИЯ ТРАФИКА eMBB И~URLLC НА ОСНОВЕ 
ПРИОРИТЕТОВ В~ПРОМЫШЛЕННЫХ РАЗВЕРТЫВАНИЯХ 5G NR$^*$}

\def\titkol{Модели совместного обслуживания трафика eMBB и~URLLC на основе 
приоритетов в %промышленных 
развертываниях 5G NR}

\def\aut{Д.\,В.~Иванова$^1$, Е.\,В.~Маркова$^2$, С.\,Я.~Шоргин$^3$, Ю.\,В.~Гайдамака$^4$}

\def\autkol{Д.\,В.~Иванова, Е.\,В.~Маркова, С.\,Я.~Шоргин, Ю.\,В.~Гайдамака}

\titel{\tit}{\aut}{\autkol}{\titkol}

\index{Д.\,В.~Иванова$^1$, Е.\,В.~Маркова$^2$, С.\,Я.~Шоргин$^3$, Ю.\,В.~Гайдамака$^4$}
\index{Borisov A.\,V.}


{\renewcommand{\thefootnote}{\fnsymbol{footnote}} \footnotetext[1]
{Исследование выполнено за счет гранта 
Российского научного фонда №\,22-79-10053.}}


\renewcommand{\thefootnote}{\arabic{footnote}}
\footnotetext[1]{Российский университет дружбы народов имени
Патриса Лумумбы, \mbox{ivanova-dv@rudn.ru}}
\footnotetext[2]{Российский университет дружбы народов имени
Патриса Лумумбы, \mbox{markova-ev@rudn.ru}}
\footnotetext[3]{Федеральный исследовательский центр 
<<Информатика и~управление>> Российской академии наук, \mbox{sshorgin@ipiran.ru}}
\footnotetext[4]{Российский университет дружбы народов имени 
Патриса Лумумбы; Федеральный исследовательский центр <<Информатика и~управление>> 
Российской академии наук, \mbox{gaydamaka-yuv@rudn.ru}}

%\vspace*{-10pt}



\Abst{Технология радиодоступа 5G NR способна осуществлять 
одновременную поддержку как сверхнадежной доставки с~малой задержкой (Ultra-Reliable Low Latency 
Communication, URLLC), 
так и~расширенного мобильного широкополосного доступа (enhanced Mobile Broadband, eMBB). Из-за критических 
требований к~задержке и~надежности, предъявляемых при предоставлении услуг двух 
классов обслуживания, возникает необходимость введения приоритетов. В~статье 
рассмотрена промышленная среда, в~которой производственное оборудование при 
управлении движением и~синхронизации работы генерирует URLLC-тра\-фик, а при 
удаленном мониторинге~--- eMBB-тра\-фик. Предложена модель с~приоритетным 
обслуживанием на базовой станции (БС) с~прямой связью между устройствами (device-to-device, D2D) 
и~без нее. Полученные численные результаты показывают, что введение приоритетов 
позволяет эффективно изолировать трафик URLLC и~eMBB. При этом стратегия 
с~поддержкой D2D, в~которой БС явно резервирует ресурсы для прямой связи, 
значительно превосходит по показателям качества обслуживания стратегии, 
в~которых явное резервирование не используется, а также стратегию, в~которой весь 
трафик проходит через БС.}
%\end{abstract}

\KW{5G; NR (New Radio); D2D; URLLC; eMBB; управ\-ле\-ние 
ресурсами; приоритетное обслуживание}

\DOI{10.14357/19922264230409}{JXCGXQ}
  
%\vspace*{-4pt}


\vskip 10pt plus 9pt minus 6pt

\thispagestyle{headings}

\begin{multicols}{2}

\label{st\stat}

    
\section{Введение}
%\label{sect:00}

Мобильные системы пятого поколения (5G), характеризующиеся высокими 
требованиями к~качеству обслуживания, разработаны с~учетом стремительного роста 
числа новых приложений. Помимо классических сценариев использования технология 
5G NR (New Radio) обещает поддержку в~сфере промышленной автоматизации таких 
приложений, как совместное управление мобильными роботами, синхронизация, 
позиционирование, сервисы дополненной реальности, а также обслуживание на основе 
технологий телеприсутствия~\cite{ghosh2019industrial}.

Системы, управляющие движущимися элементами производственного оборудования, 
требуют сверхнадежной доставки с~низкими задержками (URLLC). В~то же время для видеонаблюдения требуется расширенный 
мобильный широкополосный доступ (eMBB). Таким 
образом, БС NR должны поддерживать предоставление услуг двух 
классов обслуживания eMBB и~URLLC одновременно. Механизмы их отдельной поддержки 
на БС NR в~миллиметровом диапазоне (mmWave) широко исследованы 
(см., например,~[2--4] для eMBB и~\cite{rao2018packet, mahmood2019resource} для 
URLLC). Однако в~об\-ласти совместной поддержки исследования практически не 
проводились.

\begin{figure*} %fig1
 \vspace*{1pt}
\begin{center}
   \mbox{%
\epsfxsize=160.748mm 
\epsfbox{iva-1.eps}
}
\end{center}
\vspace*{-3pt}
\Caption{Рассматриваемый сценарий развертывания:
\textit{1}~--- передача от камеры до БС; 
\textit{2}~--- D2D-пе\-ре\-да\-ча с~координацией через БС; 
\textit{3}~--- прямая D2D-пе\-ре\-да\-ча; \textit{4}~--- заблокированная D2D-пе\-ре\-да\-ча;
\textit{5}~--- зона покрытия рас\-смат\-ри\-ва\-емой~БС}
\label{fig:deployment}
\vspace*{9pt}
\end{figure*}

В статье исследованы три схемы (стратегии) одновременного предоставления услуг 
eMBB и~URLLC с~реализацией явного приоритета~\cite{kochetkova2021queuing}: 
(1)~базовая стратегия~--- передача трафика через БС NR; (2)~стратегия 
D2D-aware~--- использование D2D для передачи с~полной 
координацией через БС (дополнительная интерференция не создается); (3)~стратегия 
D2D-unaware~--- использование D2D без координации через БС (уменьшает задержку, 
но создает дополнительную интерференцию). Для построения и~анализа модели 
использованы методы стохастической геометрии и~теории массового обслуживания~\cite{gorbunova2018resource}.

\vspace*{-6pt}

\section{Системная модель}
%\label{sys}

\vspace*{-3pt}

Рассмотрим модель развертывания 5G NR в~промышленной среде, например на 
автоматизированном заводе с~несколькими производственными линиями (рис.~1). 
Автоматизация производства предполагает регулярный мониторинг процесса с~по\-мощью 
расположенных на станках датчиков с~контролем посредством камер видеонаблюдения. 
Предположим, что БС NR монтируются на потолке на высоте~$h_A$, образуя 
пуассоновский точечный процесс с~плотностью~$\chi$~БС/$\text{м}^2$. Пользовательское оборудование, состоящее из датчиков и~камер, 
расположено на высоте~$h_U$ в~узлах регулярной сетки с~шагом $l$~м. Ширина 
полосы пропускания каждой БС составляет $W$~Гц, что соответствует емкости соты 
сети связи~$C$ условных единиц ре\-сурса.
{\looseness=1

}



Датчики, связанные с~автоматизированным оборудованием, генерируют потоковый 
трафик, соответствующий  URLLC-услуге и~ха\-рак\-те\-ри\-зу\-ющий\-ся гарантированной 
скоростью передачи данных $c_1 \hm\geq 1$. Эластичный трафик, соответствующий eMBB, 
характеризуется минимальной скоростью $c_2^{\min} \hm\geq 1$ и~генерируется 
видеокамерами, использующимися для удаленного мониторинга. Известна средняя 
интенсивность генерации запросов на передачу данных от датчиков и~камер~$\lambda_i$, а также средняя длительность сессий по передаче данных~$\mu_i^{-1}$, 
$i \hm= 1,2$.
Объем ресурса для обслуживания запроса URLLC через БС $b_{1,B}$ и~в режиме D2D~$b_{1,D}$, а~также объем ресурса для обслуживания запроса \mbox{eMBB}~$b_2^{\min}$ 
зависит от плотности БС и~рассчитывается по формулам:
\begin{equation*}
b_2^{\min}=\fr{c_2^{\min}}{E[S_{e,B}]}\,;\enskip b_{1,B} =\fr{c_1}{E[S_{e,B}]}\,;\enskip
  b_{1,D}=\fr{c_1}{E[S_{e,D}]}\,,
 % \label{eqn:requirements}
\end{equation*}

\noindent
где $E[S_{e,B}]$~--- средняя спектральная эффективность при передаче трафика 
через БС, а~$E[S_{e,D}]$~--- при D2D-передаче.

Рассчитаем среднюю спектральную эффективность для разных схем обслуживания. 
Плотности вероятности расстояний $D$ между двумя случайно выбранными 
пользовательскими устройствами и~$B$ от случайно выбранного устройства до БС 
определяются как~\cite{moltchanov2012distance}
\begin{align*}
f_{B}(x)&=\fr{2x}{r_N}\,;\\[3pt]
f_{D}(x)&=\fr{2x}{r_N^2}\left[\fr{2}{\pi}\cos^{-1}\left(\fr{x}{2r_N}\right)-\fr{x}{r_N\pi}\sqrt{1-\fr{x^2}{4r_N^2}}\,\right].
%\label{strat:00}
\end{align*}

Далее воспользуемся моделью распространения и~блокировки сигнала, рассмотренной 
в~\cite{ivanova2022performance}.
Эффективный радиус покрытия $r_N$ определяется как $\min(r_{N,S},r_{N,V})$, где 
$r_{N,S}$~--- максимально возможное расстояние между пользовательским 
оборудованием и~БС NR; $r_{N,V}$~--- половина расстояния меж\-ду БС~NR.

Отношение сигнала к~шуму (SNR) на приемнике, расположенном на расстоянии~$x$ от~БС
\begin{equation*}
%\label{eqn:genericProp}
S(x)=\fr{P_{U}G_{A}G_{U}}{N_0W+M_I}\,x^{-\zeta},
\end{equation*}
где $P_{U}$~--- излучаемая мощность; $G_{A}$ и~$G_{U}$~--- коэффициент усиления 
антенны на базовой станции и~на пользовательском оборудовании соответственно; 
$N_0$~--- спектральная плотность мощности шума; $M_I$~--- средняя мощность помех; 
$\zeta$~--- коэффициент распространения. Тогда

\noindent
\begin{align*}
%\label{eqn:se_bs}
E[S_{e,B}]&=\int\limits_{0}^{r_N}f_{B}(x)\log_{2}[1+S(x)]\,dx\,;\\
E[S_{e,D}]&=\int\limits_{0}^{2r_N}f_{D}(x)\log_{2}[1+S(x)]\,dx\,.
\end{align*}

Для расчета вероятности блокировки D2D-пе\-ре\-да\-чи сначала найдем $p_{B,1}(x)$~--- 
вероятность того, что путь прямой видимости длиной $x$ между двумя устройствами 
перекрывается одним станком. Воспользуемся методами интегральной геометрии 
и~получим вероятность перекрытия~\cite{santalo2004integral}:
\begin{multline*}
%\label{strat:02}
p_{B,1}(x)={}\\
{}=\fr{2w(\pi{}w+4x) (1-\kappa)}{\pi(2\pi{}r_N^2-4r_N^2\sin^{-1}(x/(2r_N))-
x\sqrt{4r_N^2-x^2})},
\end{multline*}
где $\kappa$~--- прозрачность пользовательского оборудования; $w$~--- ширина 
пользовательского оборудования.
Согласно рассматриваемой модели, максимальное чис\-ло станков~$N_R$, находящихся 
в~зоне покрытия БС NR, можно найти с~по\-мощью аппроксимации задачи о~круге Гаусса.

Общее число станков имеет биномиальное распределение с~параметрами $N_R$ и~$\nu$ 
(вероятность нахождения станка в~точке сетки). Таким образом, вероятность того, 
что путь прямой видимости заблокирован, имеет вид:
\begin{multline*}
p_{B}(x)={}\\
\!{}=\sum\limits_{j=1}^{N_R}\!\begin{pmatrix}
N_R\\ j\end{pmatrix} 
\nu^{j}(1-\nu)^{N_R-j}
\left[1-(1-p_{B,1}(x))\right]^j\!. \!
%\label{strat:03}
\end{multline*}
%
Тогда искомая вероятность блокировки будет рассчитываться как
\begin{equation*}
%\label{strat:04}
p_{B}=\int\limits_{0}^{2r_N}f_{D}(x)p_{B}(x)\,dx.
\end{equation*}

\section{Математическая модель}

Функционирование рассматриваемой сис\-те\-мы описывает двумерный марковский 
случайный процесс $(N_1(t), N_2(t))$,\ $t \hm\geq 0$, где $N_i(t)$, $i \hm= 1,2$,~---
случайное чис\-ло об\-слу\-жи\-ва\-емых сис\-те\-мой запросов типа $i$ в~момент $t$. Обозначим 
максимальное чис\-ло запросов на предоставление услуг URLLC (за\-про\-сы 1-го типа) 
и~eMBB (запросы 2-го типа), которое может находиться в~сис\-те\-ме, $N_1 \hm= \lfloor 
C/b_1 \rfloor$ и~$N_2\hm = \lfloor C/b_2^{\min} \rfloor$ соответственно, тогда $n_i \hm= 0,\ldots,N_i$~--- чис\-ло 
об\-слу\-жи\-ва\-емых сис\-те\-мой за\-про\-сов типа~$i$, $i\hm = 1,2$.
Состояние сис\-те\-мы описывает двумерный век\-тор $\mathbf{n}\hm=(n_1,n_2)$ над 
пространством со\-сто\-яний


\vspace*{-6pt}

\noindent
\begin{multline*}
\mathbf{X} = {}\\[2pt]
\!{}=\left\{ (n_1,n_2): n_1 \geq 0,\ \  n_2 \geq 0,\ \ n_1 b_1 + n_2 b_2^{\min} \leq 
C \right\}.\hspace*{-5.19798pt}
%\label{eq:stateSpace}
\end{multline*}
Обозначим через 

\noindent
$$
k(n_1) = \left\lfloor \fr{C-n_1 b_1}{b_2^{\min}}\right \rfloor
$$ 
максимальное чис\-ло 
запросов на предостав\-ле\-ние услуги \mbox{eMBB}, которое может быть принято в~сис\-те\-му, 
когда в~ней уже обслуживаются~$n_1$~запросов на предостав\-ле\-ние услуги URLLC. При 
этом чис\-ло единиц ресурса, выделяемое для обслуживания каждого запроса на 
предостав\-ле\-ние услуги eMBB, может меняться в~за\-ви\-си\-мости от со\-сто\-яния сис\-темы: 

\noindent
$$
b_2\left(n_1,n_2\right) = \left\lfloor \fr{C-n_1 b_1}{n_2} \right\rfloor \geq b_2^{\min}.
$$

Сформулируем правила приема и~обслуживания запросов:
\begin{itemize}
    \item если число обслуживаемых сис\-те\-мой запросов типа $i$ меньше максимально 
возможного чис\-ла запросов данного типа~$N_i$ и~чис\-ло свободных единиц ресурса, 
доступных для данных запросов, не меньше~$b_1$ и~$b_2^{\min}$ для 1-го и~2-го 
типа соответственно, то поступающий в~сис\-те\-му запрос будет принят на 
обслуживание;
    \item если число обслуживаемых сис\-те\-мой запросов на предостав\-ле\-ние услуги 
URLLC меньше максимально возможного чис\-ла запросов данного типа~$N_1$, чис\-ло 
свободных единиц ресурса, доступных для запросов данного типа, меньше~$b_1$, 
а~чис\-ло обслуживаемых запросов на предостав\-ле\-ние услуги eMBB не меньше $1$, то 
по\-сту\-па\-ющий запрос на предостав\-ле\-ние услуги URLLC будет принят на обслуживание 
за счет прерывания обслуживания $q(n_1,n_2) \hm= \lceil (b_1\hm - C \hm+ (n_1 b_1\hm + n_2 
b_2^{\min}))/b_2^{\min} \rceil$ запросов eMBB;
    \item в~противном случае поступающие в~сис\-те\-му запросы будут заблокированы.
\end{itemize}

На основе сформулированных правил со\-ста\-вим диаграмму интенсивностей переходов 
(рис.~\ref{fig:generalView}).

\begin{figure*} %fig2
\vspace*{1pt}
\begin{center}
   \mbox{%
\epsfxsize=88.713mm 
\epsfbox{iva-2.eps}
}
\end{center}
\vspace*{-10pt}
  \Caption{Диаграмма интенсивностей переходов для центрального состояния}
  \label{fig:generalView}
  \vspace*{-5pt}
\end{figure*}

Стационарное распределение вероятностей со\-сто\-яний сис\-те\-мы $p(\mathbf{n}), 
\mathbf{n}\hm \in \mathbf{X}$, может быть получено путем чис\-лен\-но\-го решения сис\-те\-мы 
уравнений равновесия 
$$
\mathbf{p}^{\mathrm{T}} \mathbf{A} = \mathbf{0}^{\mathrm{T}};\quad 
\mathbf{p}^{\mathrm{T}} \mathbf{1} = 1\,,
$$
 где $\mathbf{A}$~--- инфинитезимальная 
мат\-ри\-ца, элементы которой определяются сле\-ду\-ющим образом:

\pagebreak

\noindent
\begin{multline*}
a\left(\mathbf{n},\mathbf{n}'\right) ={}\\
{}=
  \begin{cases}
    \lambda_1, &\mathbf{n}' = \mathbf{n}+\mathbf{e}_1,\\
    &   n_1 < N_1, b_1 (n_1+1) + b_2^{\min} n_2 \leq C,\\
               &\mbox{или\ } n'_1 = n_1 +1, n'_2 = n_2 - q(n_1,n_2),\\
               &n_1 < N_1,\enskip n_2 > 0,\\
               & b_1(n_1+1) + b_2^{\min} n_2 > C\,;\\
    \lambda_2, &\mathbf{n}' = \mathbf{n}+\mathbf{e}_2,\\
    &  n_2 < N_2, b_1 n_1 + b_2^{\min} (n_2+1) \leq C\,;\\
    n_1 \mu_1, & \mathbf{n}' = \mathbf{n}-\mathbf{e}_1,\ n_1 > 0\,;\\
    n_2 \mu_2, & \mathbf{n}' = \mathbf{n}-\mathbf{e}_2,\ n_2 > 0\,;\\
     \phi,      &\mathbf{n}' = \mathbf{n}\,;\\
    0         &\mbox{в\ ином\ случае;}
  \end{cases}
\!  
%\label{eqn:generator}
\end{multline*}

\vspace*{-13pt}

\noindent
\begin{multline*}
\phi = -\left[\lambda_1  I\{n_1 <N_1, b_1 (n_1 +1) + b_2^{\min} n_2 \leq C\} + {}\right.\\
{}+\lambda_1  I\{n_1 < N_1, n_2 > 0, b_1 (n_1 +1) + b_2^{\min} n_2 > C\} + {}\\
{}+\lambda_2  I\{n_2 < N_2, b_1 n_1 + b_2^{\min} (n_2 +1) \leq C\} + n_1 
\mu_1 +{}\\
\left.{}+ n_2 \mu_2\right],
\end{multline*}
где
$I\{x\}$~--- индикаторная функция.

Рассчитав распределение вероятностей~$p(\mathbf{n})$, можно вычислить основные 
показатели эффективности модели: ве\-ро\-ят\-ность потери URLLC-за\-просов

\vspace*{-2pt}

\noindent 
$$
B_1 = \sum\limits_{n_1=0}^{k(N_1)}{p(N_1, n_1)}\,;
$$

\vspace*{-2pt}

\noindent
 ве\-ро\-ят\-ность потери eMBB-за\-про\-сов 
 $$
 B_2 = 
\sum\limits_{n_1=0}^{N_1}{p(n_1, k(n_1))}\,;
$$
 ве\-ро\-ят\-ность прерывания обслуживания eMBB-за\-про\-са
\begin{multline*}
%\label{eq:preemptioneMBB}
\Pi = \sum\limits_{n_1=0}^{N_1 -1} \sum\limits_{\substack{ n_2 = k(n_1+1)+1 \\
    k(n_1) \ne k(n_1+1)}}^{k(n_1)}\!
\lambda_1 p(n_1,n_2)\Big/ 
\left(\lambda_1+{}\right.\\
\left.{}+\lambda_2 
I\{n_2<k(n_1)\}+n_1 \mu_1+n_2 \mu_2\right) \,.
\end{multline*}


\begin{figure*} %fig3
\vspace*{1pt}
\begin{center}
   \mbox{%
\epsfxsize=162.604mm 
\epsfbox{iva-3.eps}
}
\end{center}
\vspace*{-9pt}
\Caption{Вероятность потери запроса в~зависимости от плотности $\nu$~(\textit{a}) и~скорости $c_{2}^{\min}$~(\textit{б}):
\textit{1}~--- базовая стратегия; \textit{2}~--- стратегия D2D-aware;
\textit{3}~--- стратегия D2D-unaware; пустые значки~--- \mbox{URLLC}; залитые значки~--- \mbox{eMBB}}
\end{figure*}

\vspace*{-9pt}

\section{Численные результаты}

Для проведения численного эксперимента исследуем ве\-ро\-ят\-ность потери запроса на 
передачу данных, представленную как функцию от плот\-ности пользовательского 
оборудования (ве\-ро\-ят\-ности наличия стан\-ка в~точ\-ке сетки) $\nu$ и~от минимальной 
ско\-рости передачи данных~$c_{2}^{\min}$ на рис.~3.

На рис.~3,\,\textit{а}, где $c_1\hm = 2$~Мбит/с, $c_{2}^{\min} \hm= 1$~Мбит/с, $G_A\hm = 16\times 4$, 
$G_U \hm= 4\times4$, $\xi\hm = 5\cdot 10^{-4}$, 
$\mu_1 \hm= 10^3$ и~$\mu_2 \hm= 1/120$, мож\-но увидеть, что увеличение~$\nu$ приводит 
к~росту вероятности потери. Очевидно, что базовая стратегия, в~которой все запросы 
проходят через БС, характеризуется по\-сто\-ян\-ной ве\-ро\-ят\-ностью потери запросов на 
предостав\-ле\-ние услуг eMBB и~URLLC. В~свою очередь, стратегия D2D-aware 
характеризуется меньшей ве\-ро\-ят\-ностью потери запросов на передачу трафика URLLC, 
которая не превышает~$10^{-5}$. Рас\-смат\-ри\-вая стратегию D2D-unaware, отметим, что 
преимущества от использования прямой передачи данных незначи-\linebreak\vspace*{-12pt}

\pagebreak

\noindent
тельны, поскольку 
не\-конт\-ро\-ли\-ру\-емая интерференция отрицательно влияет на передачу в~дополнение 
к~ве\-ро\-ят\-ности потери запросов из-за нехватки ресурсов, что в~условиях высокой 
на\-груз\-ки может оказаться критичным.



Результаты показывают, что передача eMBB-тра\-фи\-ка оказывает влияние на 
обслуживание URLLC-тра\-фи\-ка. На рис.~3,\,\textit{б}, где $\nu \hm= 0{,}5$, 
проиллюстрирована за\-ви\-си\-мость ве\-ро\-ят\-ности потери запросов на передачу данных от 
минимальной тре\-бу\-емой ско\-рости передачи \mbox{eMBB}-тра\-фи\-ка. Таким образом, 
обслуживание на основе приоритетов эффективно в~том случае, когда требования 
к~запросам на предостав\-ле\-ние услуги \mbox{eMBB} растут, так как вероятность потери 
запроса на предостав\-ле\-ние услуги \mbox{URLLC} при этом сохраняется практически 
неизменной. Предлагаемый метод на основе приоритетов гарантирует, что 
обслуживание запросов на предостав\-ле\-ние услуги \mbox{URLLC} будет защищено от 
потенциально из\-ме\-ня\-юще\-йся нагрузки, создаваемой запросами на предостав\-ле\-ние 
услуги \mbox{eMBB}. В~данном случае также наблюдается 
превосходство стратегии D2D-aware над другими стратегиями для любых 
рас\-смат\-ри\-ва\-емых значений~$c_{2}^{\min}$.

\vspace*{-6pt}

\section{Заключение}
%\label{conclus}

\vspace*{-3pt}

В статье предложена модель одновременной поддержки услуг двух классов 
обслуживания~--- \mbox{eMBB} и~\mbox{URLLC}, основанная на реализации явного приоритета. Для 
трех рассмотренных схем передачи трафика на примере развертывания сис\-те\-мы на 
автоматизированном предприятии проведена оценка ключевых вероятностных 
характеристик, в~част\-ности вероятности потери запросов на передачу данных, на 
основе которой сделан вывод о~наиболее эффективной стратегии совместного 
обслуживания.

{\small\frenchspacing
 {\baselineskip=10.6pt
 %\addcontentsline{toc}{section}{References}
 \begin{thebibliography}{99}
\bibitem{ghosh2019industrial} %1
\Au{Ghosh A., Ratasuk~R., Rao~A.\,M.} Industrial IoT networks powered by 5G 
New Radio~// Microwave~J., 2019. Vol.~62. No.\,12. P.~24--40.

\bibitem{moltchanov2018improving} %2
\Au{Moltchanov D., Samuylov~A., Petrov~V., Gapeyenko~M., Himayat~N., Andreev~S., Koucheryavy~Y.} Improving session continuity with bandwidth reservation in 
mm{W}ave communications~// IEEE Wirel. Commun. Le., 2018. Vol.~8. 
No.\,1. P.~105--108. doi: 10.1109/LWC.2018.2859988.

\bibitem{begishev2019quantifying} %3
\Au{Begishev V., Moltchanov~D., Sopin~E., Samuylov~A., Andreev~S., 
Koucheryavy~Y., Samouylov~K.} Quantifying the impact of guard capacity on 
session continuity in 3GPP New Radio systems~// IEEE T. Veh. 
Technol., 2019. Vol.~68. No.\,12. P.~12345--12359. doi: 10.1109/TVT.2019.2948702.

\bibitem{samuylov2020characterizing} %4
\Au{Samuylov A., Moltchanov~D., Kovalchukov~R., Pirmagomedov~R., Gaidamaka~Y., Andreev~S., Koucheryavy~Y., Samouylov~K.}
Characterizing resource allocation trade-offs in 5G NR serving multicast and 
unicast traffic~// IEEE T. Wirel. Commun., 2020. Vol.~19. 
No.\,5. P.~3421--3434. doi: 10.1109/TWC.2020.2973375.





\bibitem{rao2018packet} %5
\Au{Rao J., Vrzic S.}
Packet duplication for URLLC in 5G: Architectural enhancements and performance 
analysis~// IEEE Network, 2018. Vol.~32. No.\,2. P.~32--40. doi: 10.1109/MNET.2018.1700227.

\bibitem{mahmood2019resource} %6
\Au{Mahmood N.\,H., Karimi~A., Berardinelli~G., Pedersen~K.\,I., Laselva~D.} 
On the resource utilization of multi-connectivity transmission for URLLC 
services in 5G New Radio~// IEEE Wireless Communications and Networking 
Conference Workshop.~--- Piscataway, NJ, USA: IEEE, 2019. Art.~8902865. 
6~p. doi: 10.1109/\mbox{WCNCW}.\linebreak 2019.8902865.

\bibitem{kochetkova2021queuing} %7
\Au{Кочеткова И.\,А., Власкина~А.\,С., Ву~Н.\,Н., Шоргин~В.\,С.} Система 
массового обслуживания с~управляемым по сигналам перераспределением приборов для 
анализа нарезки ресурсов сети 5G~// Информатика и~её применения, 2021. Т.~15. Вып.~3. 
С.~91--97. doi: 
10.14357/ 19922264210312. EDN: JJENVV.

\bibitem{gorbunova2018resource} %8
\Au{Горбунова А.\,В., Наумов~В.\,А., Гайдамака~Ю.\,В., Самуйлов~К.\,Е.} 
Ресурсные системы массового обслуживания как модели беспроводных сис\-тем связи~// 
Информатика и~её применения, 2018. Т.~12. Вып.~3. С.~48--55. doi: 
10.14357/19922264180307. EDN: YAMDIL.

\bibitem{ivanova2022performance} %9
\Au{Ivanova D., Markova~E., Moltchanov~D., Pirmagomedov~R., Koucheryavy~Y., 
Samouylov~K.} Performance of priority-based traffic coexistence strategies in 5G 
mmWave industrial deployments~// IEEE Access, 2022.  Vol.~10. P.~9241--9256. doi: 10.1109/ACCESS.2022.3143583.

\bibitem{moltchanov2012distance} %10
\Au{Moltchanov D.} Distance distributions in random networks~// Ad Hoc 
Netw., 2012. Vol.~10. No.\,6. P.~1146--1166. doi: 10.1016/j.adhoc.2012.02.005.

\bibitem{santalo2004integral} %11
\Au{Santalo L.\,A.} Integral geometry and geometric probability.~--- 
Cambridge: Cambridge University Press, 2004. 428~p.
doi: 10.1017/CBO9780511617331.

\end{thebibliography}

 }
 }

\end{multicols}

\vspace*{-6pt}

\hfill{\small\textit{Поступила в~редакцию 25.09.23}}

\vspace*{8pt}

%\pagebreak

%\newpage

%\vspace*{-28pt}

\hrule

\vspace*{2pt}

\hrule



\def\tit{PRIORITY-BASED eMBB AND~URLLC TRAFFIC COEXISTENCE MODELS 
IN~5G~NR INDUSTRIAL DEPLOYMENTS}


\def\titkol{Priority-based eMBB and~URLLC traffic coexistence models 
in~5G~NR industrial deployments}


\def\aut{D.\,V.~Ivanova$^{1}$, E.\,V.~Markova$^{1}$, S.\,Ya.~Shorgin$^{2}$, and~Yu.\,V.~Gaidamaka$^{1,2}$}

\def\autkol{D.\,V.~Ivanova, E.\,V.~Markova, S.\,Ya.~Shorgin, and~Yu.\,V.~Gaidamaka}

\titel{\tit}{\aut}{\autkol}{\titkol}

\vspace*{-9pt}


\noindent
$^1$RUDN University, 6~Miklukho-Maklaya Str., Moscow 117198, Russian 
Federation

\noindent
$^2$Federal Research Center ``Computer Science and Control'' of the Russian 
Academy of Sciences, 44-2~Vavilov\linebreak
$\hphantom{^1}$Str., Moscow 119333, Russian Federation


\def\leftfootline{\small{\textbf{\thepage}
\hfill INFORMATIKA I EE PRIMENENIYA~--- INFORMATICS AND
APPLICATIONS\ \ \ 2023\ \ \ volume~17\ \ \ issue\ 4}
}%
 \def\rightfootline{\small{INFORMATIKA I EE PRIMENENIYA~---
INFORMATICS AND APPLICATIONS\ \ \ 2023\ \ \ volume~17\ \ \ issue\ 4
\hfill \textbf{\thepage}}}

\vspace*{4pt}




\Abste{The technology 5G New Radio simultaneously supports both Ultra-Reliable Low-Latency Service (URLLC) and enhanced Mobile Broadband Service 
(eMBB). Owing to extreme latency and reliability requirements of both types of services, a~prioritization needs to be provided.
The present authors consider an industrial environment where production equipment utilizes URLLC service 
 for controlling motion and synchronous operation while eMBB service is used for remote monitoring. 
 The authors proposed the model with priority service at base station (BS) with and without direct device-to-device (D2D)
  communications. The obtained numerical results indicate that priorities allow one to isolate URLLC and eMBB traffic efficiently. 
  The D2D-aware strategy where the BS explicitly reserves resources for direct communications significantly outperforms strategies 
  where explicit reservation is not utilized as well as the strategy where all the traffic goes through the BS.}

\KWE{5G; NR (New Radio); D2D; URLLC; eMBB; resource 
allocation; priority service}




\DOI{10.14357/19922264230409}{JXCGXQ}

\vspace*{-12pt}

\Ack

\vspace*{-4pt}

\noindent
The reported study was funded by the Russian  Science Foundation, project No.\,22-79-10053.

\vspace*{6pt}

  \begin{multicols}{2}

\renewcommand{\bibname}{\protect\rmfamily References}
%\renewcommand{\bibname}{\large\protect\rm References}



{\small\frenchspacing
 {%\baselineskip=10.8pt
 \addcontentsline{toc}{section}{References}
 \begin{thebibliography}{99} 
%1
\bibitem{ghosh2019industrial-1}
\Aue{Ghosh, A., R.~Ratasuk, and A.\,M.~Rao.} 2019. Industrial IoT networks powered by 
5G New Radio. \textit{Microwave J.} 62(12):24--40.

%2
\bibitem{moltchanov2018improving-1} 
\Aue{Moltchanov, D., A.~Samuylov, V.~Pet\-rov, M. Gapeyenko, N.~Himayat, S.~And\-re\-ev, 
and Ye.~Kouc\-he\-rya\-vy.} 2019. Improving session continuity with bandwidth 
reservation in mmWave communications. \textit{IEEE Wirel. Commun. Le.} 8(1):105--108. doi: 10.1109/LWC.2018.2859988.

%3
\bibitem{begishev2019quantifying-1}
\Aue{Begishev, V., D.~Moltchanov, E.~So\-pin, A.~Samuy\-lov, S.~And\-re\-ev, Y.~Kou\-che\-rya\-vy, 
and K.~Samouylov.} 2019. Quantifying the impact of guard capacity on session 
continuity in 3GPP New Radio systems. \textit{IEEE T. Veh. Technol.} 
68(12):12345-12359. doi: 10.1109/TVT.2019.2948702.

%4
\bibitem{samuylov2020characterizing-1}
\Aue{Samuylov, A., D.~Moltchanov, R.~Ko\-val\-chu\-kov, R.~Pir\-ma\-go\-me\-dov, Y.~Gai\-da\-ma\-ka, S.~And\-re\-ev, Y.~Kou\-che\-rya\-vy, and K.~Sa\-mouy\-lov.} 
2020. Characterizing resource 
allocation trade-offs in 5G NR serving multicast and unicast traffic. 
\textit{IEEE T. Wirel. Commun.} 19(5):3421--3434. doi: 10.1109/ TWC.2020.2973375.


%5
\bibitem{rao2018packet-1}
\Aue{Rao, J., and S.~Vrzic.} 2018. 
Packet duplication for URLLC in 5G: Architectural enhancements and performance 
analysis. \textit{IEEE Network} 32(2):32--40. doi: 10.1109/ MNET.2018.1700227.
%6
\bibitem{mahmood2019resource-1}
\Aue{Mahmood, N.\,H., A.~Karimi, G.~Berardinelli, K.\,I.~Pe\-der\-sen, and D.~La\-sel\-va.} 
2019.
On the resource utilization of multi-connectivity transmission for URLLC 
services in 5G New Radio.
\textit{IEEE Wireless Communications and Networking 
Conference Workshop.} Piscataway, NJ: IEEE. 6~p. doi: 
10.1109/\mbox{WCNCW}.2019.8902865.
%7
\bibitem{kochetkova2021queuing-1}
\Aue{Kochetkova, I.\,A., A.\,S.~Vlaskina, N.\,N.~Vu, and V.\,S.~Shor\-gin.} 2021. Sistema 
mas\-so\-vo\-go ob\-slu\-zhi\-va\-niya s~uprav\-lya\-emym po sig\-na\-lam pe\-re\-ras\-pre\-de\-le\-ni\-em 
pri\-bo\-rov dlya ana\-li\-za na\-rez\-ki re\-sur\-sov se\-ti 5G [Queuing system with signals for 
dynamic resource allocation for analyzing network slicing in 5G networks]. 
\textit{Informatika i~ee Primeneniya~--- Inform. Appl.} 15(3):91--97. doi: 
10.14357/ 19922264210312. EDN: JJENVV.
%8
\bibitem{gorbunova2018resource-1}
\Aue{Gorbunova, A.\,V., V.\,A.~Naumov, Y.\,V.~Gai\-da\-ma\-ka, and K.\,E.~Sa\-mouy\-lov.} 2018. 
Re\-surs\-nye sis\-te\-my mas\-so\-vo\-go ob\-slu\-zhi\-va\-niya kak mo\-de\-li bes\-pro\-vod\-nykh sis\-tem 
svya\-zi [Resource queuing systems as models of wireless communication systems]. 
\textit{Informatika i~ee Primeneniya~--- Inform. Appl.} 12(3):48--55. doi: 
10.14357/19922264180307. EDN: YAMDIL.
%9
\bibitem{Ivanova2022-1} 
\Aue{Ivanova,~D., E.~Markova, D.~Moltchanov, R.~Pir\-ma\-go\-me\-dov, 
Y.~Kou\-che\-rya\-vy, and K.~Sa\-mouy\-lov.}
2022. Performance of priority-based traffic coexistence strategies in 5G mmWave 
industrial deployments.
\textit{IEEE Access} 10:9241--9256.
doi: 10.1109/ACCESS.2022.3143583.
%10
\bibitem{moltchanov2012distance-1}
\Aue{Moltchanov, D.} 2012. Distance distributions in random networks. \textit{AD Hoc 
Netw.} 10(6):1146--1166. doi: 10.1016/ j.adhoc.2012.02.005.
%11
\bibitem{santalo2004integral-1}
\Aue{Santalo, L.\,A.} 2004. \textit{Integral geometry and geometric probability}. 
Cambridge: Cambridge University Press. 428~p. 
doi: 10.1017/CBO9780511617331.


\end{thebibliography}

 }
 }

\end{multicols}

\vspace*{-6pt}

\hfill{\small\textit{Received September 25, 2023}} 

%\vspace*{-18pt}

\Contr

\vspace*{-4pt}

\noindent
\textbf{Ivanova Daria V.} (b.\ 1996)~--- PhD student, Department of Probability 
Theory and Cyber Security, RUDN University, 6~Miklukho-Maklaya Str., Moscow 
117198, Russian Federation; \mbox{ivanova-dv@rudn.ru}

\vspace*{3pt}

\noindent
\textbf{Markova Ekaterina V.} (b.\ 1987)~--- Candidate of Science (PhD) in physics 
and mathematics, associate professor, Department of Probability Theory and Cyber 
Security, RUDN University, 6~Miklukho-Maklaya Str., Moscow 117198, Russian 
Federation; \mbox{markova-ev@rudn.ru}

\vspace*{3pt}

\noindent
\textbf{Shorgin Sergey Ya.} (b.\ 1952)~--- Doctor of Science in physics and 
mathematics, professor, principal scientist, 
Federal Research Center ``Computer Science and Control'' of the Russian Academy 
of Sciences, 44-2~Vavilov Str., Moscow 119333, Russian Federation; 
\mbox{sshorgin@ipiran.ru}

\vspace*{3pt}

\noindent
\textbf{Gaidamaka Yuliya V.} (b.\ 1971)~--- Doctor of Science in physics and 
mathematics, professor, Department of Probability Theory and Cyber Security, 
RUDN University, 6~Miklukho-Maklaya Str., Moscow, 117198, Russian Federation; 
senior scientist, Federal Research Center ``Computer Science and Control'' of the Russian Academy 
of Sciences, 44-2~Vavilov St, Moscow, 119333, Russian Federation; 
\mbox{gaydamaka-yuv@rudn.ru}


\label{end\stat}

\renewcommand{\bibname}{\protect\rm Литература}  %9
%\newcommand{\eol}{\end{enumerate}\setlength{\itemsep}{-\parsep}}
%\newcommand{\ang}[1]{\langle{#1}\rangle}
%\newcommand{\infinity}{\infty}
%\newcommand{\mess}[1]{\mbox{\tt #1}}
%\newcommand{\var}[1]{\mbox{\it #1}}
%\newcommand{\order}[1]{\stackrel{#1}\fa}
%\newcommand{\orderr}[1]{\stackrel{#1}\Longrightarrow}
%\newcommand{\infrel}[1]{\stackrel{#1}\Longrightarrow}
%\newcommand{\prog}{\mbox{\tt Prog}}
%\newcommand{\comment}[1]{}
%\newcommand{\set}[1]{\{#1\}}
%\newcommand{\pair}[2]{\langle #1,#2 \rangle}
%\newcommand{\remove}[1]{}
%\renewcommand{\qed}{\hfill\rule{2mm}{2mm}}
%\newcommand{\bull}[1]{\begin{itemize}\item{#1}\end{itemize}}
%\newcommand{\marg}[1]{\marginpar{\small #1}}


\renewcommand{\figurename}{\protect\bf Figure}
\renewcommand{\tablename}{\protect\bf Table}

\def\stat{frenkel}


\def\tit{SEAMLESS ROUTE UPDATES IN SOFTWARE-DEFINED NETWORKING 
VIA QUALITY OF~SERVICE COMPLIANCE VERIFICATION}

\def\titkol{Seamless route updates in software-defined networking via 
quality of service compliance verification}

\def\autkol{S.\,L.~Frenkel and~D.~Khankin}

\def\aut{S.\,L.~Frenkel$^1$ and~D.~Khankin$^2$}

\titel{\tit}{\aut}{\autkol}{\titkol}

%{\renewcommand{\thefootnote}{\fnsymbol{footnote}}
%\footnotetext[1] {The 
%research of Yuri Kabanov was done under partial financial support of the grant 
%of RSF No.\,14-49-00079.}}

\renewcommand{\thefootnote}{\arabic{footnote}}
\footnotetext[1]{Institute of Informatics Problems, Federal Research 
Center ``Computer Science and Control'' of the Russian Academy of Sciences,
 44-2~Vavilov Str., Moscow 119333, Russian Federation, \mbox{fsergei51@gmail.com}}
\footnotetext[2]{Computer Science Department, Ben-Gurion University of the Negev, 
Beer-Sheva 84105, Israel, \mbox{danielkh@post.bgu.ac.il}}


\index{Frenkel S.\,L.}
\index{Khankin D.}
\index{Френкель С.}
\index{Ханкин Д.}

\def\leftfootline{\small{\textbf{\thepage}
\hfill INFORMATIKA I EE PRIMENENIYA~--- INFORMATICS AND
APPLICATIONS\ \ \ 2018\ \ \ volume~12\ \ \ issue\ 4}
}%
 \def\rightfootline{\small{INFORMATIKA I EE PRIMENENIYA~---
INFORMATICS AND APPLICATIONS\ \ \ 2018\ \ \ volume~12\ \ \ issue\ 4
\hfill \textbf{\thepage}}}

\vspace*{4pt}

\Abste{In software-defined networking (SDN), the control plane and the data 
plane are decoupled. This allows high flexibility by providing abstractions 
for network management applications and being directly programmable. 
However, reconfiguration and updates of a~network are sometimes inevitable due 
to topology changes, maintenance, or failures. In the scenario,  
a~current route~$C$ and a set of possible new routes~$\{N_i\}$, where one of the 
new routes is required to replace the current route, are given. There is a chance that 
a~new route $N_i$ is longer than a~different new route $N_j$, but $N_i$ is 
a~more reliable one and it will update faster or perform better after the update 
in terms of quality of service (QoS) demands. 
Taking into account the random nature of the network functioning, 
the present authors supplement the recently proposed algorithm by Delaet
\textit{et al}.\ for route updates with 
a~technique based on Markov chains (MCs). As such, an enhanced algorithm 
for complying QoS demands during route updates is proposed
in a~seamless fashion. First, 
an extension to the update algorithm of Delaet \textit{et al}.\ 
that describes the transmission of packets through a~chosen route and compares 
the update process for all possible alternative routes is suggested. Second, several 
methods for choosing a~combination of preferred subparts of new routes, resulting 
in an optimal, in the sense of QoS compliance, new route is provided.} 

\KWE{software-defined networking; Markov chains; quality of service}

\DOI{10.14357/19922264180408}


\vspace*{8pt}


\vskip 12pt plus 9pt minus 6pt

 \thispagestyle{myheadings}

 \begin{multicols}{2}

 \label{st\stat}

\section{Introduction}
\label{s:Intro}

\noindent
Software-defined networking is an emerging network paradigm, in which the 
control plane is decoupled from the data plane enabling centralized control 
logic. Such a~dynamic network may require frequent modifications and updates to 
the network topology and configuration. 
Also, the network topology is available to the centralized control entity, there, 
due to this flexibility, it is possible to perform offline optimized calculations.

Network functions virtualization (NFV) allows replacing traditional network 
devices with software that is running on commodity servers. This software 
implements the functionality that was previously provided by dedicated hardware. 
Network functions virtualization
 allows services to be composed of virtual network functions (VNF) hosted on 
different data centers. Software-defined networking, 
when applied to NFV, helps in addressing challenges 
of dynamic resource management and intelligent service 
orchestration~\cite{rao_sdn_2014}. Sometimes, traffic is often required to pass 
through and be processed by an ordered sequence of possibly remote 
VNFs~\cite{ghaznavi_service_2016}. For example, traffic may be required to pass 
through intrusion detection system, proxy, load balancer, or a~firewall. 
Such concatenation of services is called \textit{service function chaining} 
(SFC).

Consider, for example, two communicating parties in a~network featuring complex 
network topology (e.\,g., Small-world network), and the communication flow is 
passed over a~series of VNFs. It may be the case that the network operator is 
required to move the communicating flow to a~different path due to QoS 
requirements or other possible cost considerations. We are interested 
to model the anticipated expected number of steps until the update is complete 
given a~possible new route following the required QoS demands, e.\,g., 
delay, communication rounds, cost, etc. 

%Aforesaid dynamic networking requires frequent modifications and updates to the network. 
Let us consider a pair $(C, \{N_i\})$ where a~current route~$C$ from~$s$ to~$d$ 
is scheduled to be replaced by a new route from the set~$\{N_i\}$, each from~$s$ 
to~$d$ either. Let us model each route as an ordered list of network elements, such 
as VNFs (SFCs) or generally saying routers. Each new route~$N_i$ is constructed 
during the update process, and thus, certain delays may be introduced due to
 initial packet processing or due to possible losses. 
 %There, the eventual arrival of packets along the new route during the update process is critical for successful route update. Another possible example is when the routes are SFCs, and the requirement is to update a current chain to a new one, different service chains may exhibit different delays. 

The design goals must be achieved by constructing effective algorithms for 
efficient packet QoS routing in NFV/SDN computer network. Depending on the 
QoS metric, the lower (e.\,g., for reliability) or upper (e.\,g., for a~delay) 
constraints represent the desired bounds that the orchestration must meet. 
Since different configurations could meet these bounds, the designer should also 
optimize against a~specific metric by using these both ends of the extreme. 

Methods based on integer linear programming (ILP) were proposed in several works 
(see section~\ref{sec:related_work}). The difficulty of using tools based on ILP 
 in the operational work of an administrator is that in view of the possible 
 infeasibility of the resulting solution, it may take not a~few resources (time, efforts) 
 until acceptable QoS values can be ensured.

We consider the use of ``design via verification'' approach, suggesting a~method 
for complying QoS demands. The method is based on augmenting the update algorithm with
a~verification logic. Namely, we suggest the use of 
\textit{Probabilistic real-time Computation Tree Logic} 
(PCTL)~\cite{hansson_logic_1994} for expressing real-time and probability in systems. 
Using PCTL, we can express the probability for a~process to complete after 
a~certain number of steps along an execution path and verify the selected route 
for the update.


%Assume that packets are sent from a source node $s$ to a destination node $d$ along the current route. After the update process is finished, packets will be forwarded from $s$ to $d$ along the new route. 
Delaet \textit{et al.}\ proposed a~multicast-based scheme for seamlessly updating 
a~current route to a~new one~\cite{delaet_seamless_2015}. 
According to the multicast scheme, the controller instructs 
a~router to temporarily have two $(s,d)$ entries in the routing table. When 
a~router $r \neq d$ receives a~packet from~$s$ to~$d$, it sends the packet 
according to the forwarding instructions of all of its $(s,d)$ routing 
table entries. When two identical copies of a~packet that was multicasted 
over the current and new portion of a~route arrive, the controller can dismantle 
the current route, as the new route is ready. During the update process, packets 
should not be lost, no cycles should be formed, and communication should not 
be disrupted.

%Taking into account the random nature of the network functioning, we supplement the algorithm for route updates introduced by Delaet et al. in \cite{delaet_seamless_2015}, with a technique based on Markov chains. In our extension of the algorithm, we describe the transmission of packets through a chosen route and compare the update process for all the possible alternative routes that are candidates for replacement. 

Our contribution is a model for a successful route update, including its 
intermediate steps, as MC states, each with 
a~given probability. With our model, we are able to characterize the quality of 
an update by expected number of steps in the~MC. 
%We use Markov chains to characterize the quality of the update service, and represent the expected number of steps in the Markov chain as the quality of a successful update. While, the probability for an update event 

We suggest an enhanced update method for the network administrator to augment 
his decision regarding QoS demands in terms of various network parameters and 
possible failure of the update process. Moreover, in contrast to other works, 
we are able to provide a~version of an algorithm that can perform real-time QoS
 assessment during a~route update, for each subpart of a~route. At last, using 
 our method, it is possible that the active new route will be comprised of subparts 
 of different new routes, providing optimal route update service in regard of 
 required network QoS. 

%We assume that each new route is legal. 
%However, mixing subroutes belonging to different routes may result in inconsistent state or a cycle formed in the network. We use different 
%
%
%
%We model the update process as a service, namely as a VNF, and we use Markov chains to characterize the quality of the update service. Using the expected number of steps in the Markov chain representing the update, we abstract the quality of the update service. We calculate for each possible new (sub-)route the expected number of steps required to update an old (sub-)route successfully. Subsequently, the old route is updated to the new route which requires less number of steps with high probability. We supplement the seamless update algorithm proposed by the authors of \cite{delaet_seamless_2015} with the model in this work.

%The virtualized service implementing the update algorithm will provide a recommendation for an optimal choice of a route, based on the performed calculations. Fundamentally, we create a QoS VNF for seamlessly updating a route, regarding network parameters, and taking into consideration the complexity and possible failures of updating a route. In case there exist several alternatives for a route update, there is a chance that one of the possible new routes is much longer, however, a more reliable one, and as such will update faster. 
%
%
%One of the important requirements to modification process is that the update process should not form congestion in the network, nor result in time delays, and not lose any packets. 
%
%
%Additionally, we provide an enhanced version of an algorithm that can perform the quality of service assessment during the update process, for each subpart of the new route. 
%
%We propose a directed graph $G=(V,E)$, for representing the possible legal combinations of sub-routes. The set of common nodes to $(C, \{N_i\})$ subdivides the old route and each of the new routes to sub-routes. For two sub-routes represented by the nodes $u,v \in V$, the sub-route $v$ can be launched after $u$ if and only if there exists a directed edge $(u,v) \in E$. Otherwise, the launch of $v$ after $u$ is forbidden and can result in a cycle formed in the network.


%The results of this work helped to develop an operating strategy for a network administrator, supporting both, seamlessly updating a route, and providing QoS requirements. 

Extended abstract of this work appeared as a conference paper 
in~\cite{frenkel_predicting_2017} which presented preliminary results. 
In this work, we describe in detail the system settings and bring new results 
by providing two additional algorithms.
{\looseness=1

}

In the following section, we overview the related work. Next, we provide 
the required definitions and the system settings and describe the MC 
characterization of the network. Further, we describe different update setting, 
accordingly accompanying algorithms and data structures, used for QoS assessment 
during route updates.

\vspace*{-9pt}

\section{Related Work}
\label{sec:related_work}

\vspace*{-2pt}
%The design goals must be achieved by constructing effective algorithms for efficient packet QoS routing in NFV/SDN computer network. %These algorithms, which must enable an administrator to orchestrate the existing services exported by remote providers, were considered in \cite{martins_clickos_2014, zaalouk_orchsec:_2014}. Likewise, the functional behavior (e.g., services being deprecated by their providers), as well as changes in the non-functional behavior of the orchestrated services (e.g., an increased execution time) were also considered.

%Depending on the QoS metric, the lower (e.g., for reliability) or upper (e.g., for delay) constraints represent the desired bounds that the orchestration must meet. Since different configurations could meet these bounds, the designer must also optimize against a specific metric by using these both ends of extreme.

\noindent
Quality of service routing using multipath was proposed in~\cite{devi_approach_2015}. 
The routing algorithm, initially, eliminates all links that do not meet the 
bandwidth requirements. Then, it finds disjoint shortest paths based on 
the residual network graph in each iteration.

The work~\cite{egilmez_distributed_2012} proposed a~QoS optimized routing 
over multidomain OpenFlow networks managed by a~distributed control plane, 
where each controller performs optimal routing within its domain. 
The QoS routing problem was posed as a~constrained shortest path (CSP) problem, 
and the proposed solution computes a~near-optimal route, based on LARAC 
(Lagrange relaxation based aggregated cost)
algorithm~\cite{juttner_lagrange_2001}. The proposed algorithm is an approximation 
algorithm; it always gives a~suboptimal solution.

For traditional network architecture, a~routing strategy approach based on 
ILP was introduced in~\cite{yu_efficient_2013}.
 The main disadvantage of using ILP is that the problem is NP-hard. 
 Additionally, ILP cannot be applied to probabilistic values. 
 Using linear programming (not limited to integers) rounded to integer solutions 
 will not yield an optimal solution.
 

Route updates are extensively researched in SDN~\cite{foerster_survey_2016}, 
standing on the work by Reitblatt \textit{et al.}\ where requirements for SDN 
updates were examined. This work focused on per-packet consistency property, 
stating that packets have to be forwarded either using the initial configuration 
or the final configuration but never a~mixture of them, throughout the update 
process~\cite{reitblatt_consistent_2011}. The authors proposed 
a~2-phase commit technique which relies on packets tagging so that either of 
the rules is applied. However, such technique wastes critical network resources 
and complications are formed due to packet tagging~\cite{foerster_survey_2016}. 
Further, Delaet \textit{et al.}\ showed in~\cite{delaet_seamless_2015} 
that using a~careful multicast during route updates provides 
a~better working solution.

Hogan and Esposito propose in~\cite{hogan_stochastic_2017} the use of
 Bayesian networks for delay estimation as a~traffic engineering tool and model 
 the path selection problem using a~risk minimization technique. 
 However, the authors state that the accuracy of their model is limited by its 
 ability to correctly identify dependencies in the data. In our work, 
 we suggest a~general tool for probabilistic verification of any network parameter, 
 which does not depend on variance within the dataset.
 
 

In~\cite{mcgeer_safe_2012}, an update protocol proposed where packets are 
sent to the controller during updates; such approach adds 
a~significant cost to the control plane bandwidth~\cite{delaet_seamless_2015}. 
In~\cite{mcgeer_correct_2013}, an algorithm to find 
a~safe update sequence expressed as a~logic circuit has been proposed. 
However, the algorithm 
requires a~dedicated protocol which is not currently 
supported~\cite{foerster_survey_2016}. The authors 
of~\cite{katta_incremental_2013} propose to perform the 2-phase update 
scheme from~\cite{reitblatt_consistent_2011} incrementally, making the update longer. 
%For a thorough review of route updates, the reader is referred to \cite{foerster_survey_2016}.






Software-defined networking allows the involvement of the network administrator into the network 
management during route udpdates and, in particular, during packet transmission. 
Thus, it would be highly desirable to support the decision making process 
with the right tools. Our novelty is exactly such tool, for augmenting 
online decision making of the network administrator during network management 
in a~stochastic environment.
%In this work, we propose a technique to optimize the update process by selecting the preferred (sub-)route in order to reduce the update time. We use the expected number of steps for successfully completing the update as a QoS metric, and extend the algorithm by Delaet~et~al. with Discrete Time Markov Chains (DTMC) for finding (sub-)routes which are preferred in terms of QoS. % As such, we propose to use the route updates algorithm from \cite{delaet_seamless_2015} as a virtual service for network updates per QoS requirements.

%The interaction of software components have a greater weight in NFV context, which may lead to stochastic-like behavior 

%At present, certain routing algorithms (including $k$ Edge-Disjoint) are based on the shortest path (SP) problem solution \cite{wood_toward_2015}. However, the method proposed by Wood et al. is generic and valuable only in the case of request arrival, and do not consider certain additional important requirements, such as removal or priorities of requests. 

%Several approaches for efficient SP-based QoS routing have been recently proposed in \cite{buchbinder_improved_2006}, where the authors introduce and analyze a centralized algorithm for an online scheduling and routing of arbitrary sequence of communication requests. 

%Unsplittable (single-path) assignment for each request of QoS routing is probably competitive with the best possible splittable (multipath assignment).

The work by Delaet \textit{et al.}~[4] introduced the Make\&Activate-Before-Break 
approach for seamless
route update in SDN. The authors described in a~high-level the multicasting-based 
update, which we
employ in this work. Also, they introduced a~controller-based method for 
verifying the correctness
of a~new route before the traffic redirection. Dinitz \textit{et al.}~[16] 
extended the work~[4] to the general
case of several dependent (via shared links) routes pairs. The routes update 
problem was proved to
be NP-hard~\cite{17-aaa}. The authors of~[16] suggested the use of 
artificial intelligence (AI) methods for 
solving the problem. As a~basis for AI-based solutions, Dinitz 
\textit{et al.}\ proposed a dependence graph model describing the current
state of the problem instance at any replacement stage. 
In addition, route readiness verification similar
to that in~[4] was implemented in~[16] as a high-level network protocol.

In this work, we investigate a different problem; we consider the route updates 
problem from a~QoS
perspective and describe in high-level both the prediction and the update processes.

\vspace*{-9pt}

\section{Preliminaries and Definitions}

\vspace*{-2pt}

\noindent
The basic system settings are as follows. 
For a~(route) sequence~$X$, we denote by~$x_i$ the $i$th element in it.
In a~(directed) communication network, 
we are given a~route~$C$ from source~$s$ to destination~$d$. 
Additionally, we are given a~set of different new routes~$N_i$, each going from~$s$ 
to~$d$. We model each route as an ordered set of network nodes connected by network 
links. We assume that neither of the routes contains cycles. 
Each router in a~route matches a~packet from~$s$ to~$d$ 
and forwards the packet to the next router in order. After the update 
is complete, each router in the new route should forward the packets from~$s$ 
to~$d$ to the next router in order along the new route. 

In our work, we consider the route replacement problem as a~sequence of 
subroutes replacements.
The routes replacement subsystem was in great detail described by Dinitz 
\textit{et al.} in~\cite{dinitz_dependence_2017}. We borrow
from~[16] the relevant parts which we briefly describe here.

\smallskip

\noindent
\textbf{Definition~1.} We  define a~subset from $a\in X$ to $b\in X$ of an ordered
set~$X$, when $a$ precedes~$b$, as~a~subroute from~$a$ to~$b$, and denote such subroute by
$[a,b]$.

\smallskip

 

\textbf{Subroutes.} The current route~$C$ subdivides each new route 
to~$k$~common subroutes (a~subroute may consist of one router in the simplest case) 
and $k-1$ noncommon subroutes. 
For illustration, see Fig.~1.
In Fig.~1 and figures below, the current route is depicted
in a~light grey color full nodes, connected with
solid edges. The new route is depicted in white colored nodes, connected with
dashed edges. The common nodes are depicted as shaded. 
If there are several new
routes, the nodes of each route are filled with a~designating pattern. 
Additionally, for easier reading,
when it is possible, we denote subroutes of some route~$X$ as~$X^\prime$, $X^{\prime\prime}$, 
etc. In other cases, a~subroute~$j$
of a~new (current) route~$i$ is denoted as $N_j^i (C_i^j)$. 
Similarly, routers of some route~$X$ are denoted by~$r^\prime$,
$r^{\prime\prime}$, etc.

 { \begin{center}  %fig1
\vspace*{1pt}
 \mbox{%
 \epsfxsize=78.631mm 
 \epsfbox{fre-1.eps}
 }


\vspace*{3pt}


\noindent
{{\figurename~1}\ \ \small{Route $C$ with two possible new routes sharing a~link}}
\end{center}
}

\vspace*{6pt}






In the example in Fig.~1, 
noncommon new subroutes 
of route~$N_1$ are denoted by~$N^1_1=[s,r_2]$ and~$N^2_1=[r_2,d]$, while the noncommon new 
subroutes of~$N_2$ are denoted by~$N^1_2=[s,r_1]$, $N^2_2=[r_1,r_3]$, 
$N^3_2=[r_3,r_2]$, and~$N^4_2=[r_2,d]$. 

Note that in general, the order of common subroutes along~$C$ and along~$N$ 
can be different. See, for example, the common subroutes of~$C$ and~$N_2$ in 
%Figure \ref{fig:two_routes}.
Fig.~1.

\smallskip

\noindent
\textbf{Definition~2.} A~new noncommon subroute of~$N$ from router~$a$
to router~$b$ is legitimate for update only if~$a$ precedes~$b$ on the route~$C$.

\smallskip

Definition~2 guides us on which subroutes can be launched without creating routing cycles in the
network system. (See~[4] for details.)


When an update of a~subroute~$N^\prime$ from router~$r$ to~$r^\prime$ is finished, 
the update flow goes along~$C$ from~$s$ to~$r$, continues along~$N^\prime$ up to~$r^\prime$, 
and finishes along~$C$ from~$r^\prime$
 to~$d$. 
For illustration, see the result of launching~$N^4_2$ in Fig.~2.

 { \begin{center}  %fig2
\vspace*{-1pt}
 \mbox{%
 \epsfxsize=78.631mm 
 \epsfbox{fre-2.eps}
 }


\vspace*{3pt}


\noindent
{{\figurename~2}\ \ \small{$N^4_2$ was launched}}
\end{center}
}

\vspace*{4pt}


 

 Note that launching a~currently nonlegitimate new subroute, for example,~$N^3_2$ 
 in Fig.~1, is forbidden since it will form a~cycle 
 resulting in packets circulating and overwhelming the network. 

\textbf{Dynamics of the system.}
%\label{sec:dynamics} 
Dinitz \textit{et al.}\ performed a~detailed analysis on the dynamics of a~subroutes
system. After an update of a~subroute is complete, the set of current subroutes~$C$ 
and the set
of new subroutes~$N$ are recalculated. This may result in different system of subroutes. For example,
see Fig.~2 where after the launch of $N^4_2$ from the example in Fig.~1, 
the sets of subroutes are
recalculated. As a~result, we obtain different subroutes (for clarity, the previous labels are kept). See
also~[16] for details and extensive analysis.

\vspace*{-4pt}

\subsection{Markov chain characterization of~the~network~states}

\noindent
We characterize execution of some (sub)route in the network by 
a~packet delay time between the (sub)route's common sender and common destination 
routers as well the probability of a~packet drop. Let us for now define our 
network routing model (conceptual model) informally in the following terms. 
Delay of a~packet is obtained using a~physical delay and the total processing 
time in the router. We consider that transmission of packets in 
a~network can have a~random behavior, caused by the random character of both, 
the input, and possible loss of packets. There we are interested in 
a~probabilistic model, namely, a~Markov model. In order to fully characterize 
the network as an~MC, the internal state of each router 
(and, in particular, the buffer occupancies), as well as the characteristics
 of all flows, need to be expressed as states in the chain. 

However, such approach would result in an enormous and intractable number of states. 
Therefore, to simplify these computations, let us characterize the delay time as 
an abstract variable~$t$. This abstract variable can be interpreted in different ways, 
e.\,g., the current processing queue length and a~packet transmission rate of the link, 
or possibly a~fixed value, such as an interval between the beginning of 
a~packet transmission after being processed in some node and the end of processing 
at the next node. 

We describe the functioning of the network in the transmission of packets 
as transitions of a~discrete-time MC (DTMC). The state space corresponds to the set 
of nodes such that 
the transmission of a~packet from a~node that has finished processing the packet 
to the next node corresponds to the transition of the chain to the next state.


Discrete-time MC is defined as a~tuple $D\linebreak =(S, s_0, P)$. In the tuple, $S$ is 
the finite set of states, $s_0\in S$ is the initial
state, $P:S \times S \rightarrow [0, 1]$ is the transition probability matrix in 
which $\forall s\in S$, $\sum\nolimits_{s' \in S} P(s,s') = 1$. 
For any two states $s, s' \in S$, if $P(s,s')>0$, then~$s'$ is the successor of~$s$. 
For a~subset of states $T \subseteq S$, the probability of moving from a~state~$s$ 
to any state $t \in T$ in a~single step is denoted by $P(s, T)$ and is given by 
$P(s,T)=\sum\nolimits_{t \in T} P(s, t)$. 
%The row $P(s,:)$, in the transition matrix $P$, contains the probabilities of moving from $s$ to its successors, while the column $P(:, s)$ contains the probabilities of entering the state $s$ from any other state.

\vspace*{-6pt}

\subsection{Verification syntax}

\noindent
For implementation of our PCTL-based model, we use PRISM~--- 
probabilistic model checker~\cite{kwiatkowska_prism_2011}. There, we follow 
PRISM property specification language. Here, we briefly describe the essential 
syntax while more details can be found in~\cite{noauthor_prism_nodate}.

Given a property~$\Psi$, we say that~$\Psi$ is true with probability~$p$ 
and write that as
$P_p [ \Psi ]$. If the probability~$p$ is unknown, PRISM allows, for DTMC, 
writing properties queries of the form $P_{=?}[ \Psi ]$, meaning 
``what is the probability that~$\Psi$ is true?''. Additionally, it is possible 
to use a~time bound and write properties queries such as 
$P_{=?}[F^{\leq T} \Psi]$, meaning ``what is the probability that~$\Psi$ 
is true after less than~$T$~steps?''. At last, it is possible to compute 
properties such as expected time or expected number of steps. 
For example, $R_{=?}[F \Psi]$, meaning ``what is the expected number of 
steps until $\Psi$ is true?''. 
%\section{Model Settings}
%, and a subroute of route $X$ from router $a$ to router $b$ is specified by $[a,b]_X$

%When a new subroute of $N$ that is scheduled to update a current sub-route of $C_i$ is launched, the route $C$ is updated such that the updated sub-route is replaced by launched sub-route, and the new sub-route is now part of the current route $C$.

\setcounter{figure}{3}
\begin{figure*}[b] %fig4
\vspace*{-6pt}
 \begin{center}
 \mbox{%
 \epsfxsize=149.177mm 
 \epsfbox{fre-3.eps}
 }
 \end{center}
\vspace*{-9pt}

 \Caption{New routes~$N_1$~(\textit{a}) and $N_2$~(\textit{b}) and
 MC states for~$N_1$~(\textit{c}) 
and~$N_2$~(\textit{d})}
 \label{fig:routes_dtmc_example}
\end{figure*}



\vspace*{-6pt}

\section{Prediction of Preferred Update}
%\section{Prediction of Preferred Update}
\label{sec:dtmc}

\noindent
The states of a~DTMC describe the nodes in the new route and the transition 
probabilities in the chain represent the possible delay or 
a~packet loss in the routers along the new route. The
states are defined as 
$\{s_1, \ldots , s_n\}$ where~$n$ is the number
  of nodes in the new route. 
The network achieves the state~$s_i$ if a packet has reached the $i$th node. 
For example, in Fig.~3, the self-transition 
edge represents the probability for a~delay due to packet loss, rules installation 
at the router, or congestion on the router-controller link, while the 
forward transition edge represents the probability for 
a~successful transition to the next state. These probabilities can be estimated 
from network statistics (see, for example,~\cite{hogan_stochastic_2017}). 
The labels on edges are the probability values, when edge has no label
 means probability~1.
 
 The initial probability distribution of states is given by the vector~$P_0$ of size~$n$. 
We can determine the prob-\linebreak\vspace*{-12pt}
 
 %\linebreak\vspace*{-12pt}

{ \begin{center}  %fig3
\vspace*{-0.5pt}
  \mbox{%
 \epsfxsize=77.518mm 
 \epsfbox{fre-4.eps}
 }


\end{center}

\vspace*{-3pt}

\noindent
{{\figurename~3}\ \ \small{Probability as a~function of number of steps to update routes~$N_1$~(\textit{1})
 and~$N_2$~(\textit{2})}}
}

\vspace*{12pt}



\noindent
ability that a~particular route delays the update process 
by~$k$, that is, the number of steps required for a~successful update is given by 
$p(k)=P_0 P^k$. Using this characteristic, which is, in fact, the 
probability distribution of the number of steps $P(k < x)$, one can 
calculate various properties like average delay time for the new route, 
maximum or minimum number of steps to update, etc.
 
 Consider the example illustrated in Fig.~4. 
Figure~4\textit{a} illustrates the current route~$C$ and a candidate new route~$N_1$. 
Figure~4\textit{b} shows the same current route~$C$ with another candidate 
new route~$N_2$. 
Figures~4\textit{c} and~4\textit{d} 
show the MCs for new routes~$N_1$ and~$N_2$, accordingly, with given transition 
probabilities.

During the update process, packets are sent along the current and the new routes. 
Since the new route is\linebreak\vspace*{-9.5pt}

\columnbreak

\noindent
 not operational yet, packets can be delayed due to 
congestion on certain nodes or due to switch configurations. 
%
For example, if routing rules have not yet been installed in some switch, then an 
arriving packet is sent to the controller~\cite{onf_openflow_2015}. The controller 
then decides reactively on further actions whether to install an appropriate rule 
for the packet. Also, the controller may be busy with other work and not respond 
immediately. Those packet processing actions may delay the update process. 
In the case buffer becomes full, for example, if the network is being congested, 
packets may be dropped. There, the transition to the next state during the 
update process depends on the likelihood of a~delay or a~loss of a~packet in the 
current state. 

In the example, the number of steps required for launching~$N_2$ is smaller than 
the number of steps required for launching~$N_1$. However, due to a higher likelihood 
of delays along the route~$N_2$, it is possible that~$N_1$ is preferred having 
a~higher probability for a~successful update. The network administrator may ask 
which new route is recommended for the update process, considering the expected 
number of steps required for the update. 
%
That is, updating paths requires the operator to decide 
on the possible choice of a~subroute for the next step. 
One should consider the possibility of including a~decision tool augmenting the 
controller during route updates. 

There were many attempts to use the LP/ILP 
approach, as it was already mentioned above (see, e.\,g.,~\cite{juttner_lagrange_2001}), 
but they have encountered the same difficulties, especially when taking 
into account online implementation. We show that it is possible to describe 
the routing process as DTMC. Thus, taking into consideration~$O(n^3)$ worst case 
computation complexity, we consider using the ``design via verification'' 
mentioned above based on PCTL verification, similar to the one used in 
PRISM~\cite{kwiatkowska_prism_2011}.


We have calculated the probability for a~successful update as a~function of 
number of steps for routes~$N_1$ and~$N_2$ from the example in 
Fig.~\ref{fig:routes_dtmc_example}. See Fig.~3 
where this function is shown. Curve~\textit{1}
represents the plot for~$N_1$ and curve~\textit{2} represents
 the plot for~$N_2$. 

Observe that after~20~steps, both new routes will be launched with probability~1 
which can be written as 
$$
P_{1}\left[F^{>20}N_1\right]=P_{1}\left[F^{>20}N_2\right]=1\,.
$$
The expected number of steps required for~$N_1$ is smaller than the required for~$N_2$:
$$
R \left[F~N_1\right] < R \left[F~N_2\right]\,.
$$
However, the probability for successfully updating in less than~15~steps 
is higher for route~$N_2$ ($0.55 \pm 0.040$ for~$N_1$ and 
$0.717 \pm 0.036$ for~$N_2$, based on~99\% confidence level):
$P_{0.717 \pm 0.036}\left[F^{\leq 15} N_2 \right].$

\vspace*{-6pt}


\section{Route Updates per~Quality~of~Service}
\label{sec:updates_qos}

\vspace*{-2pt}

\noindent
In this section, we show algorithm that we propose for various settings. 
First, we show an enhancement for the sequential update algorithm 
from~\cite{delaet_seamless_2015}, which during the update process decides on 
preferred subroute from the set of possible subroutes as part of QoS requirements. 
In the multicast-based update, several methods were proposed 
in~\cite{delaet_seamless_2015} for eliminating duplicated packets. 
In the case the common destination router is not able to immediately eliminate 
duplicated packets, the algorithm begins the update from the end, 
ensuring a~correct update process~[4].



\begin{algorithm*} %alg1
 \setlength{\algowidth}{100mm}
 \setlength{\hsize}{\algowidth}
 \caption{Update per QoS Algorithm}
 \label{alg:update_per_qos}

%\hrule
%\vspace*{2pt}
%\centerline
%{\textbf{Algorithm~1:} Update per QoS Algorithm}\par

%\vspace*{2pt}

%\hrule
 \small
 
 %\Input
 {directed graph $G$} 
 
 \BlankLine
 \tcc{$A$ is a collection of nodes} $A \leftarrow$ choose nodes from $G$ with in-degree $0$ \\
 
 \Repeat {out-degree of node $N^t_i > 0$}
 {
 \ForEach{$v \in A$ \label{alg:inner_loop}}
 {
 calculate $R[F~v]$ \\
% calculate the expected QoS for this node as described in Section \ref{sec:updates_qos} \\
 }\label{alg:end_inner_loop}
 
% $N^t_i \leftarrow$ choose the node that maximizes QoS \label{alg:choose_qos}\\ 
 $N^t_i \leftarrow \argmax_{v} (R[F~v])$ \label{alg:choose_qos} \\
 launch $N^t_i$ \\
 update $C$ accordingly \\
 merge any new and common subroutes as described in section~3 \\ 
 $A \leftarrow$ choose nodes neighboring to $N^t_i$ \\ 
 }
 
 \BlankLine 
 
\end{algorithm*}





 
%The algorithm starts from any node with in-degree 0 since it means that such node has no precedence dependence. Updating is completed when the algorithm arrives to a node with out-degree zero, which would be the last subroute to launch.


After that, we show an algorithm that chooses the subroutes for update arbitrary, 
assuming that the common destination node will not leak duplicated packets. 
However, the packets sending rate along the new subroute need to be temporarily limited~[4].

At last, we present a supplementing algorithm that suggests which subroutes can 
be updated in parallel.

%The set of common nodes for each pair of routes subdivides the routes to sub-routes relatively to each other. 

\vspace*{12pt}

\subsection{Sequential update}

\noindent
Let us begin the update from the end, namely, from the last alternative 
subroute of any new route. Provably, this prevents the formation of 
cycles~\cite{delaet_seamless_2015}. In order to represent all possible choices 
of a~path from a current state of the update process to the end of the update process, 
we propose to use a directed graph which nodes are the new, legitimate for launching, 
subroutes of the network. The edges of the graph represent a~legal order of launching 
new subroutes. Each path in this graph from a~current node to the last node in 
the path represents a~legal combination of chosen subroutes. The update process is 
continued as long as there is a~possible node to transition to. 

Let us examine the two possible new routes~$N_1$ and~$N_2$ that can replace the 
current route~$C$ from the example depicted in Fig.~1. 
The new route~$N_1$ is composed of~$N^1_1$ and~$N^2_1$, while the new route~$N_2$ 
composed of~$N^1_2$, $N^2_2$, $N^3_2$, and~$N^4_2$. Starting from the end, the only 
new subroutes that are allowable to launch are~$N^2_1$ and~$N^4_2$. 
Assume that based on the DTMC calculations performed as described in section~4, 
the subroute~$N^4_2$ is chosen for update. After the update of the subroute is 
complete, the current route~$C$ is composed of not updated yet part of the old 
route and~$N^4_2$. See Fig.~2 where the change in~$C$ 
is depicted.

After the subroute~$N^4_2$ is launched, we arrive at a~smaller problem in which 
less subroutes are left to update. Due to dynamics of the system 
(see section~3), some new subroutes can merge into a~single new subroute.
See Fig.~2 where after~$N^4_2$ was launched, the 
new subroutes~$N^3_2$ and~$N^2_2$ are merged into a~single subroute. Now, one 
can launch either~$N^1_1$ or~$N^2_2$ merged with~$N^3_2$. Assume that we choose to 
launch~$N^1_1$, which launch
 finishes the update. The route~$C$ updated to~$N^1_1$ 
and~$N^4_2$. See Fig.~5 illustrating that.


Figure~6 shows the directed graph that represents 
the possible update sequences. Initially, the subroutes that %\linebreak\vspace*{-12pt}
 are legal 
for launch are~$N^2_1$ and~$N^4_2$. As such, these are
the only subroutes that
 have in-degree~0. Launching~$N^3_2$
 is forbidden; hence, there is no node in the 
 graph~$G$ that represents this subroute. After launching~$N^4_2$, we\linebreak\vspace*{-12pt}
 
 \setcounter{figure}{4}

{ \begin{center}  %fig5
\vspace*{12pt}
 \mbox{%
 \epsfxsize=78.631mm 
 \epsfbox{fre-5.eps}
 }


\vspace*{3pt}


\noindent
{{\figurename~5}\ \ \small{$N^1_1$ was launched}}
\end{center}
}

\vspace*{6pt}

{ \begin{center}  %fig6
\vspace*{1pt}
 \mbox{%
 \epsfxsize=36.428mm 
 \epsfbox{fre-6.eps}
 }


\end{center}


\noindent
{{\figurename~6}\ \ \small{Graph 
representation for possible update paths for routes update example from Fig.~1}}

}

%\vspace*{6pt}

\noindent
  can 
 proceed by launching~$N^1_1$ or~$N^2_2$. However, if~$N^2_1$ was launched first, 
 it would be forbidden to launch~$N^2_2$ since it shares a~common edge with~$N^2_1$. 
 This is reflected in the graph~$G$ by not having a~directed edge from the
  node~$N^2_1$ to the node~$N^2_2$. We finish the update process
 by arriving either 
 to~$N^1_1$ or to~$N^1_2$. Notably, these nodes have out-degree~0.

 Algorithm~1 updates subroutes according to calculated QoS for each new subroute, by
 choosing at each step the new subroute that maximizes QoS.


The algorithm starts by selecting the initial set of subroute nodes. 
These are nodes with in-degree~0. The algorithm continues traversing the graph up 
to arrival at a node with out-degree~0 which would be the last subroute to launch. 
The inner loop at lines~\ref{alg:inner_loop}--\ref{alg:end_inner_loop} 
calculates the QoS for each neighboring node. Afterward, at 
line~\ref{alg:choose_qos}, the algorithm chooses the node that maximizes QoS. 
Then launches this node and updates the route~$C$, accordingly (see 
Figs.~1--5 for illustration). 
Afterward, the algorithm selects the next neighboring nodes.

After execution of Algorithm~1, the resulting new route maximally complies QoS 
requirements.

%\vspace*{12pt}

\subsection{Arbitrary subroutes selection} 
%\label{sec:arbitrary}

%\vspace*{-12pt}

\noindent
In this subsection, we assume that immediate duplicate packets elimination is possible. 
It may be that some of the subroutes are not ready for an update yet. 
Thus, meanwhile, the administrator may want to proceed with the update process 
to other subroutes or see possible variations of the update. 
For such scenario, we provide an algorithm which can select a~subroute for 
update arbitrary and continue the update process from there. 
We create a~forest graph of all possible update combinations from which the 
desired update sequence can be chosen. 
{\looseness=1

}
 


Figure~7 shows all possible combinations from example 
in Fig.~1. Noticeable, as mentioned earlier, some\linebreak\vspace*{-12pt}

{ \begin{center}  %fig7
\vspace*{1pt}
  \mbox{%
 \epsfxsize=71.694mm 
 \epsfbox{fre-7.eps}
 }


\end{center}


\noindent
{{\figurename~7}\ \ \small{Forest graph representing execution combinations for example from 
 Fig.~1}}
}

\vspace*{12pt}


\noindent
 combinations 
exhibit fewer steps, though possible that its QoS compliance is worse than others.



Algorithm~2 starts by iterating over all roots of the forest graph and 
calculating QoS using Algorithm~1 each tree. Afterward, launch the update 
of the tree that maximizes QoS.

\begin{algorithm*} %alg2
\setlength{\algowidth}{100mm}
 \setlength{\hsize}{\algowidth}
 \caption{Arbitrary Selection Update}
 \label{alg:arbitrary_update}
 \small
 
% \Input
{directed graph $G$} 
 
 %\BlankLine
 
 $A_0 \leftarrow$ choose nodes from $G$ with in-degree $0$ \\
 $Q \leftarrow \{\}$ \\
 
 \BlankLine
 \tcc{iterate over all roots of trees in the forest $G$}
 \ForEach{$v_r \in A_0$}
 {
 $q \leftarrow$ get the expected QoS using Algorithm~1 for $v_r$ \\
 $Q \leftarrow Q \cup \{q \rightarrow \mathrm{root} \}$ \\
 }

 \BlankLine
 $q_{\max} \leftarrow \max_{\mathrm{QoS}}(Q)$ \\
 launch maximizing QoS update order in $\mathrm{root}=Q[q_{\max}]$ \\ 
 
 
\end{algorithm*}


%\columnbreak

\vspace*{12pt}





\subsection{Parallel update}

\noindent
In certain cases, it is possible to update in parallel several subroutes 
and, as such, decrease update time. However, launching subroutes in parallel 
is not always possible
 since subroute may share a~link and, thus, leads to congestion 
during the update process, close a~cycle, or lead to an inconsistent state of the 
system. In~\cite{delaet_seamless_2015}, it was shown that two new subroutes~$N'$ 
from~$a$ to~$b$ and~$N''$ from~$c$ to~$d$ can be launched in parallel only if~$c$ 
succeeds~$b$ or~$a$ succeeds~$d$.



%\begin{proposition}
% Let $N'$ from $a$ to $b$ and $N''$ from $c$ to $d$ be two legitimate new subroutes. $N'$ and $N''$ can be launched in parallel only if $c$ succeeds $b$ or $a$ succeeds $d$.
%%Two subroutes that are each legitimate can be launched in parallel only if they share at most one common subroute.
%\end{proposition}
%\begin{proof}
% \textbf{Direction}: $\Rightarrow$ Let $N'$ from router $a$ to $b$ and $N''$ from router $c$ to $d$, be two new legitimate sub-routes. The only way for them to share more than one common sub-route is if $b$ succeeds $c$ on $C$. In such case, launching $N'$ will eliminate the part of $C$ from $c$ to $b$ with no proper connection from $b$ to $c$, which leaves the system in an inconsistent state. The same occurs if $N''$ is launched. \\
% \textbf{Direction}: $\Leftarrow$ Let $N'$ from router $a$ to $b$ and $N''$ from router $c$ to $d$, be two new sub-routes, not necessary part of the same new route, such that $b$ precedes $c$ or $b=c$. If $a$ precedes $b$, than $N'$ is legal for launching independently of $N''$. Similarly, if $c$ precedes $b$, than $N''$ is legal for launching independently of $N'$. Thus, since $N'$ can be launched independently from $N''$, they can be launched in parallel. Symmetric considerations lead to same result in case $a$ succeeds $d$.
% 
%\noindent Generalization to more than two sub-routes is trivial.
%\end{proof}



\begin{algorithm*}[b] %[t] %alg3
\setlength{\algowidth}{100mm}
 \setlength{\hsize}{\algowidth}
 \caption{Parallel Update}
 \label{alg:parallel_update}
 \small
 
 %\Input
 {weighted graph $G_S$} 
 
 \BlankLine
 
 \While{there are still current subroutes to update}
 {
 $A \leftarrow$ find maximum-weight independent set in $G_S$ \\
 
 \BlankLine 
 \tcc{do in parallel} 
 \ForEach{$N^t_i \in A$} 
 { 
 launch $N^t_i$ \\
 }
 }
 
 \vspace*{6pt}
 
\end{algorithm*}

We create a supplementary graph~$G_S$, in which nodes are the new legitimate 
for launching subroutes, and edges represent restrictions on parallel 
launching of subroutes. See Fig.~8 for illustration, 
depicting subroutes from example in Fig.~1 and their parallel 
restrictions. For example, $N^4_2$ and~$N^1_2$ can be launched in parallel since 
there is no edge connecting them.

Clearly, any independent set of subroutes from the supplementary 
graph contains subroutes that can be launched in parallel. 
This can be further enhanced by setting QoS calculated values as weights 
on nodes of the graph and finding the subroutes that can be launched 
in parallel by finding a~maximum-weight independent set of the graph~$G_S$. 
Since~$G_S$ has few
 number of nodes (several tens), it is possible to find 
the
 maximum-weight independent set even by enumerating
 all possible independent 
sets~\cite{wu_review_2015} and comparing their total weights.
{\looseness=-1



{ \begin{center}  %fig8
\vspace*{12pt}
  \mbox{%
 \epsfxsize=36.666mm 
 \epsfbox{fre-8.eps}
 }


\end{center}


\noindent
{{\figurename~8}\ \ \small{Supplementary graph of the example in 
 Fig.~1, showing which subroutes cannot be run in parallel}
}}

%\vspace*{12pt}



} 



Important, the parallel method should not be launched on its own. 
For example, assume that at the first iteration of Algorithm~3, 
the independent sets of nodes are~$A_1$ and~$A_2$. Let us assume that~$A_1$ complies 
better to QoS demands than~$A_2$ and, thus, $A_1$ will be selected. 
Also, let us assume that~$B_1$ is the next independent set in the graph 
if~$A_1$ was selected and~$B_2$ if~$A_2$ was selected. 
Also, let us assume that~$B_1$ is
the next independent set in the graph if~$A_1$ was selected and~$B_2$ if~$A_2$ 
was selected.
It is possible that due to the dynamics of the system (see section~3), 
we could obtain overall higher QoS results if we initially launched the 
subroutes from the sets~$A_2$ and~$B_2$ afterwards than from the sets~$A_1$ and~$B_1$.
 

Therefore, the graph that we create in this section for parallelization constraints 
is a~supplementary graph which must be used in conjunction with the graphs from 
previous sections. Optimal results will be obtained when used in conjunction with 
the forest graph from subsection~5.2.

It is also important to note that, in the worst case, when there are 
no disjoint subroutes, the parallel method is reduced to the sequential 
method thought with a higher running time.

\vspace*{-12pt} 


\section{Implementation}

\noindent
We implemented the update algorithms from~\cite{delaet_seamless_2015} as 
services for our QoS verification module. The update algorithm itself 
was not modified. In other words, we treated the update itself as 
an atomic action. The route updates
 algorithms are implemented as 
applications interacting with the northbound interface of an SDN controller. 
We used POX~\cite{kaur_network_2014} as a~platform for controller development and 
Mininet~\cite{lantz_network_2010} for network topology emulation. 
Figure~9 depicts the schematic arrangement of the 
functional elements. 



We created networks with topology of random graph and small-world features. 
During each simulation trial, a~pair of common source and destination nodes $(s,d)$ 
were selected. A~path connecting~$s$ and~$d$ was selected as a~current route and 
a~set of~4~new routes connecting $(s,d)$, to replace the current route, were 
selected, possibly with shared links among themselves and the current route. 

We considered latency due to the formed congestion as QoS demands for the update, 
implemented by forming congestion on randomly selected subroutes. Route 
update was executed by the update algorithm from~\cite{delaet_seamless_2015} for 
each pair of current and new routes. Further, one of the enhanced versions 
was executed, updating to the
 preferred combination of subroutes, by identifying 
the congested subroutes (e.\,g., by estimating latency).

{ \begin{center}  %fig9
\vspace*{8pt}
  \mbox{%
 \epsfxsize=58.544mm 
 \epsfbox{fre-9.eps}
 }

\vspace*{3pt}


\noindent
{{\figurename~9}\ \ \small{Description of the system}
}
\end{center}}

%\vspace*{12pt}



%\vspace*{-45pt}

\section{Concluding Remarks}

\noindent
The study in this paper illustrates a~feasibility of modeling and 
designing the route update process via verification using DTMC. The goal was to 
strengthen the network administrator involvement in management and decision 
making during route update. In the present model, the network administrator is able 
to consider network parameters such as packet losses, delay, communication 
rounds, flow table updates, congestion, and other inherent unreliabilities of 
the network. 

We extended the updating algorithm with the ability to compute QoS as the 
MC characteristics, where the MC corresponds to the states 
of the update process. Using this MC computation ability, it is 
possible to predict the expected number of steps (delay time) required to 
complete the update process. These prediction results allow the administrator 
to make a~decision whether a~new route can satisfy the user requirements per QoS 
or a~more reliable route will be selected.

We provided sequential update algorithm and an arbitrary order algorithm 
when for the later, it is assumed that immediate duplicate packets elimination 
is possible. Further, we suggest a supplementary graph and algorithm for launching 
updates in parallel when it is possible.

This paper proposes a~conceptual approach. In future research, we will focus 
on optimization of predictions supplementing the network administrator with 
a~powerful tool which will be able to enhance the update process 
with fine grained analysis of the network.

\vspace*{-12pt}


\Ack
\noindent
The first author has partially been supported by the 
Russian Foundation for Basic Research under grants RFBR 18-07-00669 and 18-29-03100. 
The second author has partially been supported by the Rita Altura Trust Chair in
Computer Sciences; The Lynne and William Frankel Center for Computer
Science.

%\bigskip


The authors thank Prof.\ Shlomi Dolev 
for his valuable input and Prof.\ Yefim Dinitz for his comments.
 
\renewcommand{\bibname}{\protect\rmfamily References}

%\vspace*{-6pt}

\vspace*{-6pt}

{\small\frenchspacing
{\baselineskip=10.35pt
\begin{thebibliography}{99}



\bibitem{rao_sdn_2014}  %1
\Aue{Rao, S.\,K.} 2014. SDN and its use-cases~--- NV and NFV:
A~state-of-the-art survey. NEC Technologies India Ltd. 25~p.

\bibitem{ghaznavi_service_2016}  %2
\Aue{Ghaznavi, M., N.~Shahriar, R.~Ahmed, and R.~Boutaba}. 2016. 
Service function chaining simplified. {arXiv.org}. arXiv:1601.00751.

\bibitem{hansson_logic_1994}  %3
\Aue{Hansson, H., and B.~Jonsson}. 
1994. A~logic for reasoning about time and reliability. 
\textit{Form. Asp. Comput.} 6(5):512--535.

\bibitem{delaet_seamless_2015}  %4
\Aue{Delaet, S., S.~Dolev, D.~Khankin, S.~Tzur-David, and T.~Godinger}. 
2015. Seamless SDN route updates. \textit{IEEE 14th Symposium (International)
on Network Computing and Applications}. IEEE. 120--125.

\bibitem{frenkel_predicting_2017} 
\Aue{Frenkel, S., D.~Khankin, and A.~Kutsyy}. 
2017. Predicting and choosing alternatives of route updates per QoS VNF in SDN. 
\textit{IEEE 16th Symposium (International) on Network Computing and Applications}. 
IEEE. 1--6. 

\bibitem{devi_approach_2015} 
\Aue{Devi, G., and S.~Upadhyaya}. 2015. 
An approach to distributed multi-path QoS routing. 
\textit{Indian J.~Sci. Technol.} 8(20):1--14. 
doi: 10.17485/ijst/2015/v8i20/49253.

\bibitem{egilmez_distributed_2012} 
\Aue{Egilmez, H.\,E., S.~Civanlar, and A.\,M.~Tekalp}. 2012. 
A~distributed QoS routing architecture for scalable video streaming over multi-domain 
OpenFlow networks. \textit{19th IEEE Conference (International) on Image Processing}.
IEEE. 2237--2240.

\bibitem{juttner_lagrange_2001} 
\Aue{Juttner, A., B.~Szviatovski, I.~Mecs, and Z.~Rajko}. 2001. 
Lagrange relaxation based method
for the QoS routing problem. \textit{IEEE Conference on Computer Communications. 
20th Annual Joint Conference of the IEEE Computer and Communications Society
 Proceedings}. IEEE. 2:859--868.

\bibitem{yu_efficient_2013} %9
\Aue{Yu, Z., F.~Ma, J.~Liu, B.~Hu, and Z.~Zhang}. 2013. 
An efficient approximate algorithm for disjoint QoS routing.
\textit{Math. Probl. Eng.} 2013:489149. 9~p. 
doi: 10.1155/2013/489149.

\bibitem{foerster_survey_2016} 
\Aue{Foerster, K.-T., S.~Schmid, and S.~Vissicchio} 2016. 
A~survey of consistent network updates. \mbox{Arxiv.org}. \mbox{arXiv}:\linebreak 1609.02305.

\bibitem{reitblatt_consistent_2011} 
\Aue{Reitblatt, M., N.~Foster, J.~Rexford, and D.~Walker}. 
2011. Consistent updates for software-defined networks: Change you can believe in! 
\textit{10th ACM Workshop on Hot Topics in Networks Proceedings}.
New York, NY: ACM. Art.\ No.\,7. doi: 10.1145/2070562.2070569.

\bibitem{hogan_stochastic_2017} 
\Aue{Hogan, M., and F.~Esposito}. 
2017. Stochastic delay forecasts for edge traffic engineering via Bayesian networks. 
\textit{IEEE 16th Symposium (International) on Network Computing and Applications}. 
IEEE. 1--4.

\bibitem{mcgeer_safe_2012} %15
\Aue{McGeer, R.} 2012. A~safe, efficient Update Protocol for Openflow Networks. 
\textit{1st Workshop on Hot Topics in Software Defined Networks Proceedings}. 
New York, NY: ACM. 12:61--66.
\bibitem{mcgeer_correct_2013} 
\Aue{McGeer, R.} 2013. A~correct, zero-overhead protocol for network updates. 
\textit{2nd ACM SIGCOMM Workshop on Hot Topics in Software Defined Networking
Proceedings}. New York, NY: ACM. 13:161--162.
\bibitem{katta_incremental_2013} 
\Aue{Katta, N.\,P., J.~Rexford, and D.~Walker}. 
2013. Incremental consistent updates. \textit{2nd ACM SIGCOMM Workshop on Hot Topics 
in Software Defined Networking Proceedings}.
New York, NY: ACM. 13:49--54.

\bibitem{dinitz_dependence_2017}  %16
\Aue{Dinitz, Y., S.~Dolev, and D.~Khankin}. 
2017. Dependence graph and master switch for seamless dependent routes 
replacement in SDN. \textit{IEEE 16th Symposium 
(International) on Network Computing and Applications}. IEEE. 1--7.

\bibitem{17-aaa}
\Aue{Amiri, S.\,A., S.~Dudycz, S.~Schmid, and S.~Wiederrecht}.
2016. Congestion-free rerouting of flows
on DAGs. \mbox{ArXiv}.org. arXiv:1611.09296.
% [cs, math], Nov. 2016, arXiv: 1611.09296. [Online]. Available:
%http://arxiv.org/abs/1611.09296

\bibitem{kwiatkowska_prism_2011}  %17
\Aue{Kwiatkowska, M., G.~Norman, and D.~Parker}. 2011. 
PRISM~4.0: Verification of probabilistic real-time systems. 
\textit{Computer aided verification}.
Eds. G.~Gopalakrishnan and S.~Qadeer.
Lecture notes in computer science ser. Springer.
6806:585--591.

\bibitem{noauthor_prism_nodate}  %18
\Aue{Kwiatkowska, M., G.~Norman, and D.~Parker}. 2018. 
{PRISM manual}. Available at:
{\sf http://www.\linebreak prismmodelchecker.org/manual/}
(accessed December~10, 2018).

\bibitem{onf_openflow_2015} %19
{Open Networking Foundation}. 2015. 
OpenFlow Switch Specification Ver~1.5.1. 


\bibitem{wu_review_2015}  %20
\Aue{Wu, Q., and J.-K.~Hao}. 2015. 
A~review on algorithms for maximum clique problems. 
\textit{Eur. J.~Oper. Res.} 242(3):693--709.

\bibitem{kaur_network_2014}  %21
\Aue{Kaur, S., J.~Singh, and N.\,S.~Ghumman}. 2014. 
Network programmability using POX controller. 
\textit{Conference (International) on Communication, Computing and Systems}.
138.

\bibitem{lantz_network_2010}  %22
\Aue{Lantz, B., B.~Heller, and N.~McKeown}. 2010. 
A~network in a~laptop: Rapid prototyping for software-defined networks. 
\textit{9th ACM SIGCOMM Workshop on Hot Topics in Networks Proceedings}. 
New York, NY: ACM.  Art.\ No.\,19. doi: 10.1145/1868447.1868466.
\end{thebibliography} } }

\end{multicols}

\vspace*{-9pt}

\hfill{\small\textit{Received October 9, 2018}}

\vspace*{-22pt}

\Contr

\vspace*{-3pt}

\noindent
\textbf{Frenkel Sergey L.} (b.\ 1951)~--- 
Candidate of Science (PhD) in technology, associate professor, 
senior scientist, Institute of Informatics Problems, Federal Research Center 
``Computer Sciences and Control'' of the Russian Academy of Sciences, 
44-2~Vavilov Str., Moscow 119333, Russian Federation; \mbox{fsergei51@gmail.com}

\vspace*{1pt}

\noindent
\textbf{Khankin D.} (b.\ 1983)~--- MSc, doctorate student, Department of Computer 
Science, Ben-Gurion University of the Negev, Beer-Sheva 84105, Israel; 
\mbox{danielkh@post.bgu.ac.il}

\vspace*{4pt}

\hrule

\vspace*{2pt}

\hrule

\vspace*{-7pt}

%\newpage

%\vspace*{-28pt}

\def\tit{НЕПРЕРЫВНЫЕ ОБНОВЛЕНИЯ МАРШРУТА В~SDN С~ИСПОЛЬЗОВАНИЕМ ПРОВЕРКИ СООТВЕТСТВИЯ 
КАЧЕСТВУ~ОБСЛУЖИВАНИЯ$^*$\\[-7pt]}

\def\titkol{Непрерывные обновления маршрута в~SDN с~использованием проверки соответствия 
качеству обслуживания}

\def\aut{С.\,Л.~Френкель$^1$, Д.~Ханкин$^2$\\[-7pt]}

\def\autkol{С.\,Л.~Френкель, Д.~Ханкин}

{\renewcommand{\thefootnote}{\fnsymbol{footnote}} \footnotetext[1]
{Работа была частично поддержана РФФИ (гранты 18-07~00669 и~18-29-03100), 
а~также Rita Altura Trust Chair in
Computer Sciences; The Lynne and William Frankel Center for Computer
Science.}}



\titel{\tit}{\aut}{\autkol}{\titkol}

\vspace*{-22pt}

\noindent
$^1$Институт проблем информатики Федерального исследовательского центра 
<<Информатика и~управление>>\linebreak
$\hphantom{^1}$Российской академии наук
%, fsergei51@gmail.com 

\noindent
$^2$Университет им.\ Бен-Гуриона в Негеве, Беэр-Шева, Израиль
%, danielkh@post.bgu.ac.il 

\vspace*{1pt}

\def\leftfootline{\small{\textbf{\thepage}
\hfill ИНФОРМАТИКА И ЕЁ ПРИМЕНЕНИЯ\ \ \ том\ 12\ \ \ выпуск\ 4\ \ \ 2018}
}%
 \def\rightfootline{\small{ИНФОРМАТИКА И ЕЁ ПРИМЕНЕНИЯ\ \ \ том\ 12\ \ \ выпуск\ 4\ \ \ 2018
\hfill \textbf{\thepage}}}

\vspace*{-1pt}


 
\Abst{В программно-определяемой сети (SDN~--- software-defined networking) 
уровень управ\-ле\-ния 
и~уровень данных разделены. Это обеспечивает высокую гибкость эксплуатации, 
предоставляя абстракции для управления сетью приложений 
и~возможность непосредственного программирования маршрутов.
Однако из-за изменений топологии, процедуры обслуживания или происходящих 
сбоев иногда необходима реконфигурация и~обновление сети. 
В~предлагаемом сценарии рассматривается текущий маршрут~$C$
и~набор возможных новых маршрутов~~$\{N_i\}$, где для замены текущего 
маршрута требуется 
один из\linebreak\vspace*{-12pt}}

\Abstend{новых маршрутов. Существует вероятность того, что новый маршрут~$N_i$ 
окажется длиннее некоторого другого нового маршрута~$N_j$, но при этом~$N_i$ 
будет более надежным и~он будет обновляться быстрее или работать лучше 
после обновления с~точки зрения требований качества обслуживания (QoS~---
quality of service). Принимая 
во внимание случайный характер функционирования сети, авторы дополнили недавно 
предложенный алгоритм обновления маршрута Delaet с~соавт.\ методом оценки соблюдения 
требований QoS во время непрерывного обновления маршрута, основанным на 
использовании цепей Маркова. При этом, во-пер\-вых, предлагается расширить 
алгоритм передачи пакетов по выбранному маршруту, сравнивая процесс обновления 
для возможных альтернатив маршрута. Во-вто\-рых, предлагается несколько 
способов выбора комбинаций предпочтительных отрезков путей новых маршрутов, 
что приводит к оптимальному в~смысле соответствия QoS маршруту.}


\KW{программно-определяемые сети; цепи Маркова; качество обслуживания}

\DOI{10.14357/19922264180408}



%\vspace*{-3pt}


 \begin{multicols}{2}

\renewcommand{\bibname}{\protect\rmfamily Литература}
%\renewcommand{\bibname}{\large\protect\rm References}

{\small\frenchspacing
{\baselineskip=10.5pt
\begin{thebibliography}{99}
%\vspace*{-3pt}


\bibitem{2-fr-1}
\Au{Rao S.\,K.} SDN and its use-cases~--- NV and NFV: A~state-of-the-art survey.~--- 
NEC Technologies India Ltd., 2014. 25~p.
\bibitem{3-fr-1}
\Au{Ghaznavi M., Shahriar~N., Ahmed~R., Boutaba~R.} 
Service function chaining simplified~// Arxiv.org, 2016. \mbox{arXiv}:1601.00751cs.
\bibitem{4-fr-1}
\Au{Hansson H., Jonsson~B.} A~logic for reasoning about time and reliability~// 
Form. Asp. Comput., 1994. Vol.~6. No.\,5. P.~512--535.

\bibitem{1-fr-1} %4
\Au{Delaet S., Dolev~S., Khankin~D., Tzur-David~S., Godinger~T.}
Seamless SDN route updates~// IEEE 14th Symposium (International)
 on Network Computing and Applications.~--- IEEE, 2015. P.~120--125.
 
 
\bibitem{5-fr-1}
\Au{Frenkel S., Khankin D., Kutsyy~A.} Predicting and choosing alternatives 
of route updates per QoS VNF in SDN~// IEEE 16th Symposium (International)
on Network Computing and Applications.~--- IEEE, 2017. P.~1--6.
\bibitem{6-fr-1}
\Au{Devi G., Upadhyaya~S.} An approach to distributed multi-path QoS routing~// 
Indian J.~Sci. Technol., 2015. Vol.~8. Iss.~20. P.~1--14. 
doi: 10.17485/ijst/2015/v8i20/49253.
\bibitem{7-fr-1}
\Au{Egilmez H.\,E., Civanlar S., Tekalp~A.\,M.} 
A~distributed QoS routing architecture for scalable video streaming over multi-domain 
OpenFlow networks~// 19th IEEE Conference (International)
on Image Processing.~--- IEEE, 2012. P.~2237--2240.
\bibitem{8-fr-1}
\Au{Juttner A., Szviatovski B., Mecs~I., Rajko~Z.}
Lagrange relaxation based method for the QoS routing problem~// 
IEEE INFOCOM 2001 Conference on Computer Communications. 20th 
Annual Joint Conference of the IEEE Computer and Communications Society
Proceedings.~--- IEEE, 2001. Vol.~2. P.~859--868.
\bibitem{9-fr-1}
\Au{Yu Z., Ma F., Liu~J., Hu~B., Zhang~Z.}
An efficient approximate algorithm for disjoint QoS routing~// 
Math. Probl. Eng., 2013. Vol.~2013. Art.\ No.\,489149. 9~p. 
doi: 10.1155/2013/489149.
\bibitem{10-fr-1}
\Au{Foerster K.-T., Schmid S., Vissicchio~S.}
A~survey of consistent network updates~// Arxiv.org, 2016. arXiv:1609.02305.
\bibitem{11-fr-1}
\Au{Reitblatt M., Foster N., Rexford J., Walker~D.} 
Consistent updates for software-defined networks: Change you can believe in!~// 
10th ACM Workshop on Hot Topics in Networks Proceedings.~--- New York, NY, USA: ACM, 
2011. Art.\ No.\,7. doi: 10.1145/2070562.2070569.
\bibitem{12-fr-1}
\Au{Hogan M., Esposito F.} Stochastic delay forecasts for edge traffic engineering 
via Bayesian Networks~// IEEE 16th Symposium (International)
on Network Computing and Applications.~--- IEEE, 2017. P.~1--4.
\bibitem{13-fr-1}
\Au{McGeer R.} A~safe, efficient Update Protocol for Openflow Networks~// 
1st Workshop on Hot Topics in Software Defined Networks Proceedings.~--- 
New York, NY, USA: ACM, 2012. Vol.~12. P.~61--66.
\bibitem{14-fr-1}
\Au{McGeer R.} 2013. A~correct, zero-overhead protocol for network updates~// 
2nd Workshop on Hot Topics in Software Defined Networking Proceedings.~--- 
New York, NY, USA: ACM, 2013. Vol.~13. P.~161--162.
\bibitem{15-fr-1}
\Au{Katta N.\,P., Rexford J., Walker~D.} Incremental consistent updates~// 
2nd Workshop on Hot Topics in Software Defined Networking Proceedings.~--- 
New York, NY, USA: ACM, 2013. Vol.~13. P.~49--54.
\bibitem{16-fr-1}
\Au{Dinitz Y., Dolev S., Khankin~D.}
 Dependence graph and master switch for seamless dependent 
 routes replacement in SDN~// IEEE 16th Symposium 
 (International) on Network Computing and Applications.~--- IEEE, 2017. P.~1--7.
 \bibitem{17-aaa-1}
\Au{Amiri~S.\,A., Dudycz~S., Schmid~S., Wiederrecht~S}.
 Congestion-free rerouting of flows
on DAGs~// ArXiv.org, 2016. arXiv:1611.09296.
% [cs, math], Nov. 2016, arXiv: 1611.09296. [Online]. Available:
%http://arxiv.org/abs/1611.09296

\bibitem{17-fr-1}
\Au{Kwiatkowska M., Norman~G., Parker~D.}
 PRISM~4.0: Verification of probabilistic real-time systems~//
 Computer aided verification~/
 Eds. G.~Gopalakrishnan, S.~Qadeer.~---
Lecture notes in computer science ser.~--- Springer, 2011. 
 Vol.~6806. P.~585--591.
\bibitem{18-fr-1}
\Au{Kwiatkowska M., Norman G., Parker~D.}
 PRISM manual, 2018. 
{\sf http://www.prismmodelchecker.org/manual}.
\bibitem{19-fr-1}
Open Networking Foundation. OpenFlow Switch Specification Ver~1.5.1, 2015. 

\bibitem{21-fr-1}
\Au{Wu Q., Hao J.-K.} A~review on algorithms for maximum clique problems~// 
Eur. J.~Oper. Res., 2015. Vol.~242. No.\,3. P.~693--709.

\bibitem{20-fr-1}
\Au{Kaur S., Singh J., Ghumman~N.\,S.}
 Network programmability using POX controller~// Conference
 (International) on Communication, Computing and Systems, 2014. P.~138.
\bibitem{22-fr-1}
\Au{Lantz B., Heller B., McKeown~N.} 
A~network in a~laptop: Rapid prototyping for software-defined networks~// 
9th ACM SIGCOMM Workshop on Hot Topics in Networks Proceedings.~--- 
New York, NY, USA: ACM, 2010. Art.\ No.\,19. doi: 10.1145/1868447.1868466.
\end{thebibliography}
} }

\end{multicols}

 \label{end\stat}

 \vspace*{-9pt}

\hfill{\small\textit{Поступила в~редакцию 09.10.2018}}


%\renewcommand{\bibname}{\protect\rm Литература}
\renewcommand{\figurename}{\protect\bf Рис.}
\renewcommand{\tablename}{\protect\bf Таблица} %10
\def\stat{goncharov}

\def\tit{ВЫРАВНИВАНИЕ ДЕКАРТОВЫХ ПРОИЗВЕДЕНИЙ УПОРЯДОЧЕННЫХ МНОЖЕСТВ$^*$}

\def\titkol{Выравнивание декартовых произведений упорядоченных множеств}

\def\aut{А.\,В.~Гончаров$^1$, В.\,В.~Стрижов$^2$}

\def\autkol{А.\,В.~Гончаров, В.\,В.~Стрижов}

\titel{\tit}{\aut}{\autkol}{\titkol}

\index{Гончаров А.\,В.}
\index{Стрижов В.\,В.}
\index{Goncharov A.\,V.}
\index{Strijov V.\,V.}


{\renewcommand{\thefootnote}{\fnsymbol{footnote}} \footnotetext[1]
{Работа выполнена при частичной финансовой поддержке РФФИ 
(проекты 19-07-1155 и~19-07-00885). Настоящая статья содержит 
результаты проекта <<Статистические методы машинного обучения>>, 
выполняемого в~рамках реализации Программы Центра компетенций 
Национальной технологической инициативы <<Центр хранения 
и~анализа больших данных>>, поддерживаемого Министерством науки 
и~высшего образования Российской Федерации по договору МГУ им.\ 
М.\,В.~Ломоносова  с~Фондом поддержки проектов Национальной 
технологической инициативы от 11.12.2018 №\,13/1251/2018.}}


\renewcommand{\thefootnote}{\arabic{footnote}}
\footnotetext[1]{Московский физико-технический институт, alex.goncharov@phystech.edu}
\footnotetext[2]{Вычислительный центр им.\ А.\,А.~Дородницына Федерального исследовательского 
центра <<Информатика и~управ\-ле\-ние>> Российской академии наук; 
Московский фи\-зи\-ко-тех\-ни\-че\-ский институт, \mbox{strijov@ccas.ru}}

%\vspace*{-12pt}



\Abst{Работа посвящена исследованию метрических методов анализа 
объектов сложной структуры. Предлагается обобщить метод динамического 
выравнивания двух временных рядов на случай объектов, определенных на 
двух и~более осях времени. В~дискретном представлении такие объекты 
являются матрицами. Метод динамического выравнивания временных рядов 
обобщается как метод динамического выравнивания матриц. Предложена 
функция расстояния, устойчивая к~монотонным нелинейным деформациям 
декартова произведения двух и~более временных шкал. Определен выравнивающий 
путь между объектами. В~дальнейшем объектом называется матрица, 
в~которой строки и~столбцы соответствуют осям времени. Исследованы 
свойства предложенной функции расстояния. Для иллюстрации метода 
решаются задачи метрической классификации объектов на модельных 
данных и~данных из набора MNIST.}

\KW{функция расстояния; динамическое выравнивание; расстояние между матрицами; 
нелинейные деформации времени; про\-стран\-ст\-вен\-но-вре\-мен\-ные ряды}

\DOI{10.14357/19922264200105} 
  
\vspace*{-3pt}


\vskip 10pt plus 9pt minus 6pt

\thispagestyle{headings}

\begin{multicols}{2}

\label{st\stat}


\section{Введение}

Временн$\acute{\mbox{ы}}$е ряды представляют собой набор измерений, упорядоченных 
по оси времени. Анализ временн$\acute{\mbox{ы}}$х рядов производится при решении задач, 
связанных с~классификацией активности человека по измерениям акселерометра 
телефона, поиском паттернов в~EEG-сиг\-на\-лах (электроэнцефалограмма), 
кластеризации набора ECoG (электрокортикограмма) данных и~во многих других 
задачах~\cite{0}. Рассматриваются объекты, для которых время между измерениями 
фиксированно. В~данной работе для построения адекватной функции 
расстояния между объектами требуется учесть нелинейные деформации 
относительно оси времени: глобальные и~локальные сдвиги, растяжения 
и~сжатия~\cite{1}.

В~\cite{2} приводятся различные методы решения задач анализа 
временн$\acute{\mbox{ы}}$х рядов: классификации, детектирования паттернов, 
кластеризации и~др. В~\cite{3} описание временных рядов 
строится с~по\-мощью анализа параметров моделей, в~\cite{4} 
используется их признаковое описание, в~\cite{5} анализируется их форма. 
Комбинации этих подходов описаны в~\cite{2}.

Метрические методы находят схожие объекты в~наборе. Используются 
функции расстояния над временн$\acute{\mbox{ы}}$ми рядами: расстояние Хаусдорфа~\cite{10}, 
MODH~\cite{11}, расстояние, основанное на HMM
(hiden Markov model)~\cite{6}, евклидово расстояние 
в~исходном пространстве или в~пространстве сниженной размерности~\cite{5}, 
\mbox{LCSS} (longest common\linebreak subsequence)~\cite{7}. Показано~\cite{8}, что в~случае локальных или глобальных 
деформаций времени при решении задач, требующих анализа исходной формы 
временн$\acute{\mbox{о}}$го ряда, метод динамического выравнивания оси времени 
DTW (Dynamic Time Warping) 
превосходит другие функции расстояния~\cite{9} по качеству итогового 
решения задачи, так как при наличии смещений двух объектов относительно 
друг друга требуется выравнивать их оптимальным образом для вычисления 
расстояния между ними.

В данной работе предлагается перейти от рас\-смот\-ре\-ния объекта~$\textbf{s}(t)$, 
временн$\acute{\mbox{о}}$го ряда, к~более общему случаю $\textbf{s}(\textbf{t})$, 
в~котором компоненты вектора~$\textbf{t}$~--- оси времени. Из-за 
существенного рос\-та вы\-чис\-ли\-тель\-ной слож\-ности при увеличении чис\-ла 
осей времени предлагается рас\-смот\-реть объекты $\textbf{s}(t_1, t_2)$, 
определенные на двух осях времени. Оси времени считаются независимыми. 
В~случае единственной дискретной и~ограниченной сверху шкалы времени 
объект представим вектором фиксированной размерности. 
Аналогично объект настоящего исследования представим мат\-ри\-цей.

Вводятся ограничения на зависимости осей времени в~декартовом 
произведении для таких объектов. Определена гипотеза порождения данных: 
объекты одного класса эквивалентности получены при помощи допустимых 
преобразований, а~именно: локальных деформаций (растяжений и~сжатий) 
каждой из осей времени по отдельности. В~дискретном случае преобразование 
представимо дуп\-ли\-ци\-ро\-ва\-ни\-ем строк и~столбцов матриц. 
В~число допустимых преобразований попадают и~глобальные деформации: 
сдвиги по осям времени, представимые добавлением и~удалением крайних 
строк и~столбцов исходных матриц. Для каждой из осей времени выполняются 
свойства времени: монотонность и~непрерывность. Похожими на описанные 
свойствами обладает, например, частотный спектр сигнала, где одна ось 
определяет время, а другая~--- частоту, величину, обратную времени.


Между двумя объектами, матрицами, в~случае допустимых преобразований 
требуется определить инвариантную к~преобразованиям осей времени функцию 
расстояния, которая сможет выделить классы эквивалентности множества 
преобразованных объектов. Работа посвящена определению такой функции 
расстояния, как обобщения метода динамического выравнивания временных рядов 
DTW для матриц.

Цель данной работы~--- построение метода, основанного на динамическом 
выравнивании осей времени для матриц. Метод динамического выравнивания 
временн$\acute{\mbox{ы}}$х рядов~\cite{33} определен только для объектов с~одной осью времени, 
что делает его неприменимым для описанного случая. Однако концепции, 
используемые на каждой стадии вы\-чис\-ле\-ния оптимального выравнивания, обобщены 
на рассматриваемый случай. Работа исследует свойства предложенного 
метода и~сравнивает результаты применения метода к~задачам классификации 
изображений~\cite{12} с~результатами функции расстояния~$L_2$.

Для иллюстрации и~анализа результатов решается задача метрической 
классификации объектов (матриц низкой размерности). Используются наборы данных: 
модельные данные, которые согласуются с~выдвинутой гипотезой порождения 
данных для временн$\acute{\mbox{ы}}$х рядов, подмножество набора MNIST сниженной 
размерности и~частотный спектр сигнала.

\vspace*{-10pt}

\section{Постановка задачи построения функции расстояния}

\vspace*{-2pt}

Рассмотрим задачу построения функции расстояния между объектами. 
Функция расстояния инвариантна к~допустимым преобразованиям осей времени: 
глобальным и~локальным линейным и~нелинейным деформациям временн$\acute{\mbox{о}}$й шкалы. 
Ниже приведены две постановки задачи, с~помощью которых определены свойства 
предложенной функции расстояния, оценено ее качество и~проведено сравнение 
нескольких функций расстояния: предложенной и~$L_2$.

Первая постановка задачи использует общее свойство функций расстояния: 
объединение схожих объектов и~разделение непохожих объектов. 
Вводится определение свойства инвариантности функции расстояния к~допустимым 
преобразованиям осей времени.
Вторая постановка задачи уточняет первую и~заключается в~проведении метрической 
классификации методом ближайшего соседа.

\textbf{Постановка задачи выбора функции расстояния между двумя объектами.}
На двух временн$\acute{\mbox{ы}}$х осях заданы объекты вида 
$\textbf{A}(t_1,t_2)\hm \in \mathbb{R}^{n \times n}$. 
Функция $G_w(\textbf{A}):\mathbb{R}^{n \times n} \hm\rightarrow 
\mathbb{R}^{\hat{n} \times \hat{n}}$ задает допустимые преобразования 
исходного объекта~$\textbf{A}$: глобальные сдвиги, локальные линейные 
и~нелинейные деформации, а~именно: растяжения и~сжатия оси времени, 
сдвиги значений по оси времени. Скалярный параметр $w \hm\in \mathbb{R}^+$
 функции~$G$ фиксирует набор этих преобразований.

Допустимым элементарным преобразованием матрицы~$\textbf{A}$ назовем 
дуплицирование случайных строк и~столбцов исходной матрицы, добавление 
или удаление крайних строк и~столбцов. Допустимым преобразованием 
примем некоторую последовательность допустимых элементарных 
преобразований матрицы~$\textbf{A}$ и~обозначим как~$G_w(\textbf{A})$.

Будем называть объект~$\textbf{B} \hm\in \mathbb{R}^{\hat{n} \times \hat{n}}$ 
полученным из объекта~$\textbf{A}$ при помощи допустимых 
преобразований~$G_{\hat{w}}$, если существует $\hat{w}\hm\in \mathbb{R}^+ : 
\textbf{B} \hm= G_{\hat{w}}(\textbf{A})$.

Функцию расстояния между двумя объектами $\rho: 
\mathbb{R}^{{n} \times {n}} \times \mathbb{R}^{\hat{n} \times \hat{n}} 
\hm\rightarrow  \mathbb{R}^+$ оценим на выборке $\mathfrak{D } \hm= 
\{ \textbf{A}_i \}_{i=1}^m$ объектов вида $\textbf{A}_i \hm\in 
\mathbb{R}^{n \times n}$.

Для каждого объекта выборки~$\textbf{A}_i$ и~объекта~$\textbf{B}_j$ его 
класса эквивалентности $\{\textbf{B}_j\}_i \hm= \{  \textbf{B} 
\hm\in \mathfrak{D} | \exists w_i,w_j: G_{w_i}(\textbf{A}_i) \hm= G_{w_j}
(\textbf{B}_j)   \}$ заданы допустимые трансформации с~параметрами~$w_i$ 
и~$w_j$, такие что $G_{w_i}(\textbf{A}_i)\hm = G_{w_j}(\textbf{B}_j)$. 
Для каждого объекта выборки~$\textbf{A}_i$ и~объекта~$\textbf{C}_j$ 
из других классов эквивалентности $\{ \textbf{C}_k\}_i \hm= 
\{  \textbf{C} \hm\in \mathfrak{D} | \nexists w_i,w_k: G_{w_i}(\textbf{A}_i)
\hm = G_{w_k}(\textbf{C})   \}$ не существует таких $ w_i, w_k : G_{w_i}
(\textbf{A}_i) \hm= G_{w_k}(\textbf{C}_k)$.

Решается задача поиска функции расстояния~$\rho$, значение
 которой на паре объектов одного класса эквивалентности меньше, 
 чем на любой паре объектов из разных: для любых $i,j,k \hm\in 
 \{1,\dots,m\}$ $\quad \rho(\textbf{A}_i,\textbf{B}_j) \hm< 
 \rho(\textbf{A},\textbf{C}_k)$. Функцию расстояния, обладающую 
 таким свойством, назовем инвариантной на классах эквивалентности.

Критерием качества для функции расстояния~$\rho$ на выборке~$\mathfrak{D}$ 
примем долю объектов, для которых указанное неравенство выполняется:
$$
S_{\rho}(\mathfrak{D}) = \fr{1}{m} \sum\limits_{i=1}^m 
\prod\limits_{\{ \textbf{B}_j\}_i} 
\prod\limits_{\{ \textbf{C}_k\}_i}  
\left[  \rho(\textbf{A}_i,\textbf{B}_j) < \rho(\textbf{A}_i,\textbf{C}_k)  
 \right].
 $$
Постановка задачи выбора функции расстояния~$\rho$ 
сводится к~задаче максимизации критерия качества.

\textbf{Прикладное использование функции расстояния.}
Задана выборка $\mathfrak{D}\hm = \{(\textbf{A}_i,y_i)\}^m_{i=1}$, 
состоящая из пар объ\-ект--от\-вет. Объектами служат объекты сложной 
структуры: $\textbf{A}_i\hm \in \mathbb{R}^{n\times n}$, 
а~ответами выступают метки класса~---~$y_i\hm \in Y \hm= \{1,\ldots,E\}$, 
где $E \hm\ll m$. Выборка разделена на обучение $\mathfrak{D}_l \hm= 
\{(\textbf{A}_i,y_i)\}^{m_1}_{i=1}$ и~контроль $\mathfrak{D}_t \hm= 
\{(\textbf{A}_i,y_i)\}_{m_1}^{m_1+m_2}$.

Модель классификации~$f$ принадлежит множеству моделей метрической 
классификации 1NN, которые классифицируемому объекту ставят 
в~соответствие метку класса ближайшего объекта из обучающей 
выборки по заданной функции расстояния~$\rho$:
$$ 
\hat{y} = f(\textbf{B} | \rho) = y \argmin\limits_{i = 1,\dots, m_1} 
\rho\left(B,A_i\right)\,.
$$
Критерий качества $S$ модели~$f$ для задачи классификации~--- 
доля правильно проставленного класса на контрольной выборке:
 $$ 
 S(f | \rho) = \fr{1}{m_2}\sum\limits_{i=m_1}^{m_1+m_2} 
 \left[f(\textbf{A}_i | \rho) = y_i\right].
 $$

Требуется выбрать функцию расстояния~$\rho$ для модели 
классификации~$f:~\mathbb{R}^{n\times n} \hm\rightarrow~Y$, 
максимизируюшую критерий качества~$S$ на контрольной выборке:
\begin{equation*}
f =  \argmax\limits_{\rho \in \{\mathrm{mDTW}, L_2\}}\left(S(f | \rho)\right).
\end{equation*}

\section{Вычисление матричного расстояния mDTW}

Предлагается использовать функцию расстояния DTW, 
модифицированную для случая выравнивания двойной шкалы времени.

\smallskip

\noindent
\textbf{Определение~1.} {Даны два объекта~$\textbf{A},\textbf{B}\hm \in 
\mathbb{R}^{n\times n}$. Тензор 
невязок~$\boldsymbol{\Omega}^{n \times n \times n \times n}$~--- 
такой тензор, что его элемент~$\boldsymbol{\Omega}(i,j,k,l)$ 
равен квадрату разности между элементами~$\textbf{A}(i,j)$ и~$\textbf{B}(k,l)$:}
\begin{equation*}
\boldsymbol{\Omega}(i,j,k,l)=(\textbf{A}(i,j) - \textbf{B}(k,l))^2.
\end{equation*}

\noindent
\textbf{Определение 2.} {Путем~$\boldsymbol{\pi}$ между двумя 
объектами $\textbf{A},\textbf{B} \hm\in \mathbb{R}^{n\times n}$ 
назовем множество индексов тензора~$\boldsymbol{\Omega}$: }
$$
\boldsymbol{\pi} = \{(i,j,k,l)\},\quad i,j,k,l \in \{1,\ldots,n\} ,
$$
\textit{удовлетворяющее следующим условиям:}

{\bfseries\textit{Частичный порядок.}}
Для элементов пути~$\boldsymbol{\pi}$ с~фиксированными значениями~$i,k$ 
задан порядок: выравнивающий путь для фиксированных строк двух 
матриц упорядочен~--- $\{(i,j_r,k,l_r))\}_{r=1}^{R} \hm\subset 
\boldsymbol{\pi}$ мощностью~$R$. Аналогично для фиксированных столбцов 
с~индексами~$j,l$.

{\bfseries\textit{Граничные условия.}}
 Пусть $(i,j,k,l) \in \boldsymbol{\pi}$, тогда $(1,j,1,l) \hm\in 
 \boldsymbol{\pi}$ и~$(i,1,k,1) \hm\in \boldsymbol{\pi}$.
Путь $\boldsymbol{\pi}$ содержит элементы тензора~$\boldsymbol{\Omega}$: 
$(1,1,1,1) \hm\in \boldsymbol{\pi}$ и~$(n,n,n,n) \hm\in \boldsymbol{\pi}$.

{\bfseries\textit{Непрерывность по направлению.}}
Для упорядоченного подмножества пути $\{(i,j_r,k,l_r)\}_{r=1}^{R}
\hm\subset\boldsymbol{\pi}$ выполняется условие непрерывности:
$$
j_{r}-j_{r-1}\leq1\,,\quad l_r-l_{r-1}\leq1\,, \quad r = 2,\ldots,R\,.
$$
На~шаге пути~$\boldsymbol{\pi}$ по фиксированному направлению времени~$i,k$ 
встречаются только соседние элементы матрицы (включая соседние по диагонали). 
Аналогично для фиксированных~$j,l$.

{\bfseries\textit{Монотонность по направлению.}}
Для упорядоченного подмножества пути  $\{(i,j_r,k,l_r)\}_{r=1}^{R}
\hm\subset\boldsymbol{\pi}$ выполняется хотя бы одно из условий 
монотонности функции выравнивания времени: 
$$
j_{r}-j_{r-1}\geq1\,,\quad l_r-l_{r-1}\geq1\,, \quad r = 2,\ldots,R\,.
$$

Свойства пути между матрицами обобщают свойства пути между двумя 
временными рядами.

\smallskip

\noindent
\textbf{Определение~3.}\ {Стоимость 
$\mathrm{Cost}\,(\textbf{A},\textbf{B},{\boldsymbol{\pi}})$ пути $\boldsymbol{\pi}$ 
между объектами $\textbf{A}, \textbf{B}$:
\begin{equation*}
\mathrm{Cost}\,(\textbf{A},\textbf{B},{\boldsymbol{\pi}}) = 
\sum\limits_{(i,j,k,l) \in \boldsymbol{\pi}}{\boldsymbol{\Omega}}(i,j,k,l).
\end{equation*}}

\noindent
\textbf{Определение~4.}\ 
{Выравнивающий путь~$\hat{\boldsymbol{\pi}}$ между 
объектами $\textbf{A},\textbf{B}$~--- путь наименьшей стоимости 
среди всех возможных путей между объектами:
\begin{equation*}
\hat{\boldsymbol{\pi}} = 
\argmin\limits_{{\boldsymbol{\pi}}} \mathrm{Cost}
\left(\textbf{A},\textbf{B},{\boldsymbol{\pi}}\right).
\end{equation*}}
Функция расстояния~$\rho (\textbf{A},\textbf{B})\hm = \mathrm{mDTW}\,
(\textbf{A},\textbf{B})$ между объектами~$\textbf{A}$ и~$\textbf{B}$ 
рассчитывается как стоимость выравнивающего пути~$\hat{\boldsymbol{\pi}}$:
\begin{equation}
\mathrm{mDTW}(\textbf{A},\textbf{B}) = \mathrm{Cost}\left(\textbf{A},
\textbf{B},\hat{\boldsymbol{\pi}}\right).
\end{equation}

\setcounter{figure}{1}
\begin{figure*}[b] %fig2
{\small 
\begin{center}
\begin{tabular}{l}
\hline
DTW(\textbf{s},\textbf{c}):\\
\hspace*{3mm}$\boldsymbol{D}$(1:n+1,1:m+1) = inf;\\
\hspace*{3mm}$\boldsymbol{D}$(1,1) = 0;\\
\hspace*{3mm}for $i = 2$: $n+1$\\
\hspace*{6mm}for $j = 2$ : $m+1$\\
\hspace*{9mm}$d = (\textbf{s}(i-1)-\textbf{c}(j-1))^2$;\\
\hspace*{9mm}$\boldsymbol{D}(i,j) = d + \min( 
[ \boldsymbol{D}(i-1,j), \boldsymbol{D}(i,j-1), \boldsymbol{D}(i-1,j-1) ])$;\\
return\ sqrt$(\boldsymbol{D}(n+1,m+1))$\\
\hline
\end{tabular}
\end{center}}
\vspace*{-9pt}

\Caption{Алгоритм вычисления DTW для временных рядов
\label{ris:dtwts}}
%\end{figure*}
%\begin{figure*} %fig3
\vspace*{6pt}
{\small 
\begin{center}
\begin{tabular}{l}
\hline
\\[-9pt]
Correction $(\overline{i,j,k,l}, \boldsymbol{\pi}(\overline{i,j,k,l})):$\\
\hspace*{3mm}if $\overline{i,j,k,l} \in \{ (i-1, j, k,l)  ;  
(i, j, k-1, l)  ;  (i-1, j, k-1, l) \}$:\\
\hspace*{6mm}$ \widehat{\pi} = \{ (\overline{i}, r, \overline{k}, f) \in 
\boldsymbol{\pi}(\overline{i, j, k, l}) \vert r, f \in \mathbb{N} \}$\\
\hspace*{3mm}elif $\overline{i,j,k,l}\in \{  
(i, j-1, k, l); (i, j, k, l-1); (i, j-1, k, l-1) \}$:\\
\hspace*{6mm}$\widehat{\pi} = \{ (r, \overline{j}, f, \overline{l}) 
\in \boldsymbol{\pi}(\overline{i, j, k, l}) \vert r, f \in \mathbb{N} \}$\\
\hspace*{3mm}elif $\overline{i,j,k,l} =  i-1,j-1,k-1,l-1:$\\
\hspace*{6mm}$\widehat{\pi} = \{ (\overline{i}, r, \overline{k}, f) 
\in \boldsymbol{\pi}(\overline{i, j, k, l}) \vert r,f \in \mathbb{N} \} \cup$\\
\hspace*{6mm}$\cup \{ (r, \overline{j}, f, \overline{l}) \in \boldsymbol{\pi}
(\overline{i, j, k, l}) \vert r,f \in \mathbb{N} \}$\\
\hspace*{3mm}$\boldsymbol{d\pi} = \{ \mathrm{element} \in \widehat{\pi}: 
\mbox{произведены\ замены\ индексов } 
\overline{i} = i,\ \overline{j} = j,\ \overline{k} = k,\ \overline{l} = l \}$\\
return $\boldsymbol{d\pi}$\\
\hline
\end{tabular}
\end{center}
}
\vspace*{-9pt}

\Caption{Алгоритм вычисления поправки $\boldsymbol{d\pi}$ 
пути $\boldsymbol{\pi}$
\label{ris:codedpi}}
\end{figure*}


\textbf{Алгоритм вычисления значения расстояния~(4).}
Построение алгоритма вычисления значения функции расстояния 
между матрицами основан на алгоритме расчета функции расстояния 
между временн$\acute{\mbox{ы}}$ми рядами. В~случае выравнивания одной\linebreak\vspace*{-12pt}

{ \begin{center}  %fig1
 \vspace*{-3pt}
    \mbox{%
 \epsfxsize=79mm 
 \epsfbox{gon-1.eps}
 }


\end{center}


\noindent
{{\figurename~1}\ \ \small{Матрица стоимости оптимального выравнивания, по обеим 
осям отложены временные отсчеты}}
}

\vspace*{12pt}


\noindent 
временн$\acute{\mbox{о}}$й шкалы
 итоговая матрица расстояний~$\boldsymbol{D}$ (рис.~1) в~каждом 
 элементе~$\boldsymbol{D}(i,j)$ содержит рас\-сто\-яние между подрядом 
 первого временн$\acute{\mbox{о}}$го ряда и~подрядом второго временн$\acute{\mbox{о}}$го ряда. 
 Рас\-смот\-рим алгоритм динамического выравнивания двух временн$\acute{\mbox{ы}}$х 
 рядов $\textbf{s} \hm\in R^n$ и~$\textbf{c} \hm\in R^m$ на рис.~2.
 
 

Элемент $\boldsymbol{D}(i,j)$ матрицы~$\boldsymbol{D}$ соответствует 
стоимости выравнивающего пути между подпоследовательностями 
исходных временн$\acute{\mbox{ы}}$х рядов: $\textbf{s}(1:i) \hm= \textbf{s}(t)$, 
$t \hm= 1,\ldots,i,$ и~$\textbf{c}(1:j) \hm= \textbf{c}(t)$, $t \hm= 1,\ldots,j$. 
Алгоритм построения наилучшего выравнивания времени 
подразумевает, что выравнивающий путь между этими 
подпоследовательностями получен одним из трех способов~--- 
если стоимость выравнивающего пути между 
подпоследовательностями~$\textbf{s}(1:\overline{i}) $ 
и~$\textbf{c}(1:\overline{j})$ минимальна для~$\overline{i,j}$ из множества
$$
\overline{i,j} \in \left\{ \{i-1,j\},\{i,j-1\},\{i-1,j-1\} \right\},$$
тогда выравнивающий путь между $\textbf{s}(1:i)$ и~$\textbf{c}(1:j)$ получен добавлением пары~$(i,j)$ к~выбранному 
выравнивающему пути с~минимальной стоимостью из трех.



Предложенный алгоритм переносит эти рас\-суж\-де\-ния на случай 
выравнивания двух матриц~$\textbf{A}$ и~$\textbf{B}$. 
Элемент~$\boldsymbol{D}(i,j,k,l)$ четырехиндексного
 тензора расстояний~$\boldsymbol{D}$ соответствует стоимости выравнивающего 
 пути между $\textbf{A}(1:i,1:j) \hm= \textbf{A}(t_1,t_2)$, 
 $t_1 \hm= 1,\ldots, i$, $t_2 \hm= 1,\ldots, j,$ 
 и~$\textbf{B}(1:k,1:l) \hm= \textbf{B}(t_1,t_2)$, $t_1 \hm= 1,\ldots, k$,
 $t_2 \hm= 1,\ldots, l$. Выравнивающий путь между этими 
 подматрицами получен одним из семи способов~--- 
 если стоимость выравнивающего пути между 
 подматрицами $\textbf{A}(1:\overline{i},1:\overline{j})$ 
 и~$\textbf{B}(1:\overline{k},1:\overline{l})$ 
 минимальна для~$\overline{i,j,k,l}$ из множества
\begin{multline*} 
\overline{i,j,k,l} \in 
\left\{ \{i-1,j,k,l\},\{i,j-1,k,l\},\right.\\
\{i,j,k-1,l\},
\{i,j,k,l-1\}, \{i-1,j,k-1,l\},\\
\left.
\{i,j-1,k,l-1\},\{i-1,j-1,k-1,l-1\}\right\},
\end{multline*}

\setcounter{figure}{3}
\begin{figure*} %fig4
{\small 
\begin{center}
\begin{tabular}{l}
\hline
$\mathrm{mDTW}\left(\textbf{A},\textbf{B}\right):$\\
\hspace*{3mm}$\textbf{D}(1:n+1,1:n+1, 1:n+1, 1:n+1) = inf$;\\
\hspace*{3mm}$\textbf{D}(1,1,1,1) = 0;$\\
\hspace*{3mm}$\boldsymbol{\pi}(1,1,1,1) = ((1,1),(1,1))$\\
\hspace*{3mm}$for\ i,j,k,l  \in \mathbb{N}^{2 : n+1} \times 
\mathbb{N}^{2 : n+1} \times \mathbb{N}^{2 : n+1} \times \mathbb{N}^{2 : n+1}:$\\
\hspace*{6mm}$\overline{i,j,k,l} = \argmin($ [ \textbf{D}(i-1, j, k, l), 
\textbf{D}(i, j-1, k, l), \textbf{D}(i, j, k-1, l), 
\textbf{D}(i, j, k, l-1),    \\
\hspace*{9mm}$\textbf{D}(i-1, j, k-1, l), \textbf{D}(i, j-1, k, l-1), 
\textbf{D}(i-1, j-1, k-1, l-1) ])$;\\
\hspace*{3mm}$\boldsymbol{d \pi} = \mathrm{Correction}\,(\overline{i,j,k,l}, 
\boldsymbol{\pi}(\overline{i,j,k,l}))$\\
\hspace*{3mm}$\boldsymbol{\pi}(i, j, k, l) = \boldsymbol{d \pi} \cup 
\{(\overline{i,j,k,l})\}$\\
\hspace*{3mm}$\mathrm{cost} = (\textbf{A}(i, j)-\textbf{B}(k, l))^2 + 
\sum\nolimits_{(r,f,t,g) \in \boldsymbol{d \pi}}
(\textbf{A}(r, f)-\textbf{B}(t, g))^2$;\\
\hspace*{3mm}$\textbf{D}(i,j,k,l) = \mathrm{cost} + \textbf{D}
(\overline{i,j,k,l})$\\
return  sqrt$(\textbf{D}(n+1,n+1,n+1,n+1))$\\
\hline
\end{tabular}
\end{center}
}
\vspace*{-9pt}

\Caption{Алгоритм вычисления расстояния между матрицами
\label{ris:matrixdtw}}
\end{figure*}

\begin{table*}[b]\small
\begin{center}
\begin{tabular}{|l|c|c|c|c|}
\multicolumn{5}{c}{Снижение расстояний при выполнении преобразований 
для различных наборов данных}\\
\multicolumn{5}{c}{\ }\\[-6pt]
\hline
 &\multicolumn{4}{c|}{Метод}\\
 \cline{2-5}
\multicolumn{1}{|c|}{Данные}  & \multicolumn{2}{c|}{$L_2$} & \multicolumn{2}{c|}{MatrixDTW} \\
\cline{2-5}
& $S(f|p)$  &  $S_{\rho}(\mathfrak{D})$ &  $S(f|p)$ & $S_{\rho}(\mathfrak{D})$ \\
\hline
Модельные данные без преобразований& 92\% & 78\% & 100\%\hphantom{9} & 85\% \\
Модельные данные с~преобразованиями & 86\% & 65\% &  100\%\hphantom{9} & 82\% \\
Модельные данные с~преобразованиями и~шумом& 69\% & 61\% &  92\% & 78\% \\
MNIST без преобразований& 95\% & --- & 95\% & --- \\
MNIST с~преобразованиями & 53\% & --- & 92\% & --- \\
Спектр сигнала& 83\% & --- & 96\% & --- \\
\hline
\end{tabular}
\end{center}
\end{table*}

\noindent
то к~выравнивающему пути между этими под\-мат\-ри\-ца\-ми 
добавляется элемент пути $(i,j,k,l)$ и~поправка~$\boldsymbol{d\pi} $ 
пути~$\boldsymbol{\pi}$, алгоритм вычисления которой приведен ниже.

Обозначим выравнивающий путь между $\textbf{A}(1:i,\linebreak 1:j)$
 и~$\textbf{B}(1:k,1:l)$ как~$\boldsymbol{\pi}(i,j,k,l)$, тогда 
 поправка~$\boldsymbol{d\pi} $ пути~$\boldsymbol{\pi}(i,j,k,l)$ 
 при фиксированных~$\overline{i,j,k,l}$ вычисляется приведенным на рис.~3 
 образом.





Алгоритм динамического выравнивания двух матриц и~вычисления 
расстояния $\mathrm{mDTW}$ между ними с~учетом приведенного выше 
алгоритма примет вид, представленный на рис.~4.





\begin{figure*} %fig5
\vspace*{1pt}
    \begin{center}  
  \mbox{%
 \epsfxsize=161.412mm 
 \epsfbox{gon-5.eps}
 }
\end{center}
\vspace*{-12.5pt}
\Caption{Выравнивание модельных данных: (\textit{а})~один класс без шума; 
(\textit{б})~разные классы без шума; 
(\textit{в})~один класс с~шумом; (\textit{г})~разные классы с~шумом
\label{ris:random}}
%\end{figure*}
%\begin{figure*} %fig6
\vspace*{1pt}
    \begin{center}  
  \mbox{%
 \epsfxsize=163mm 
 \epsfbox{gon-6.eps}
 }
\end{center}
\vspace*{-12.5pt}
\Caption{Выравнивание данных MNIST: левый столбец~--- один класс; 
правый столбец~--- разные 
классы;
(\textit{а})~$\mathrm{mDTW}\hm=720{,}1$; 
(\textit{б})~948,6;
(\textit{в})~2017,0;
(\textit{г})~$\mathrm{mDTW}\hm=2071{,}4$
\label{ris:mnist}}
\end{figure*}


Следует отметить, что алгоритм~\cite{15} имеет\linebreak высокую сложность 
вычисления~--- $O(n^4)$. Предполагается ускорение метода 
с~использованием ограниче\-ния Sakoe-Chiba band, что сократит 
вычислительную сложность алгоритма до $O(n^2k^2)$, где~$k$~--- 
параметр ограничения.


\section{Вычислительный эксперимент}

Вычислительный эксперимент проведен на модельных данных с~допустимыми 
преобразованиями и~на реальных данных: объектах коллекции MNIST с~допустимыми 
преобразованиями и~на спектрограммах зашумленных сигналов.





Решается задача метрической классификации методом ближайшего соседа. В~таблице 
приведены значения критерия качества функции расстояния 
$S_{\rho}(\mathfrak{D})$ и~критерия качества метрической классификации $S(f|p)$ 
при использовании двух функций расстояния: предложенной в~работе $\mathrm{mDTW}$ 
и~$L_2$.

Модельные данные~--- это нулевые матрицы со случайными ненулевыми 
строками, столбцами, подпрямоугольниками с~наложенным шумом. 
К~ним применены допустимые преобразования, согласованные с~гипотезой 
наличия локальных и~глобальных искажений. На рис.~\ref{ris:random} 
показан пример оптимального выравнивания двух объектов. 
Линиями показаны элементы пути~$\boldsymbol{\pi}$.

Подготовлена подвыборка набора данных MNIST. Она 
состоит из~100 объектов классов 0 и~1 сниженной размерности
 с~допустимыми преобразованиями. На рис.~\ref{ris:mnist} 
 показан пример оптимального выравнивания объектов.


Аналогичный эксперимент проведен для решения задачи метрической 
классификации спектров различных сигналов, пример которых приведен на 
рис.~\ref{ris:spectr}. На рисунке показаны примеры Фурье-спект\-ров 
этих сигналов. Спектр получен путем применения быстрого преобразования 
Фурье к~исходному сигналу для различных окон с~фиксированным размером и~сдвигом. 
Исходные временн$\acute{\mbox{ы}}$е ряды обладали свойством периодичности, период выбирался 
случайным образом.



Тестирование проведено на разного рода данных: исходных 
модельных данных без наложения\linebreak\vspace*{-12pt}

\pagebreak

\end{multicols}

\begin{figure*} %fig7
\vspace*{1pt}
    \begin{center}  
  \mbox{%
 \epsfxsize=149.062mm 
 \epsfbox{gon-7.eps}
 }
\end{center}
\vspace*{-8pt}
\Caption{Данные спектров сигнала: (\textit{а})~класс~1; (\textit{б})~спектр 
класса~1; (\textit{в})~класс~2; (\textit{г})~спектр класса~2; 
(\textit{д})~класс~3; (\textit{е})~спектр класса~3
\label{ris:spectr}}
\vspace*{9pt}
\end{figure*}

\begin{multicols}{2}

\noindent допустимых преобразований, с~ними, а~также 
на модельных данных с~наложенным поверх объектов случайным шумом.



В каждом из проведенных экспериментов была продемонстрирована 
устойчивость предложенного подхода к~допустимым преобразованиям. 
Наилучшее значение критерия качества задачи классификации было 
достигнуто при использовании предложенной функции расстояния.

\vspace*{-5pt}

\section{Заключение}

В работе предложено обобщение метода динамического выравнивания
 временн$\acute{\mbox{ы}}$х рядов для случая объектов, определенных на двух осях времени. 
 Существует теоретическое обобщение предлагаемых методов на случай 
 конечного множества осей времени. Вычислительный эксперимент позволил 
 проанализировать свойства подхода: устойчивость к~допустимым 
 преобразованиям и~разделяющая способность функции расстояния как 
 на реальных, так и~на модельных данных. Качество решения задачи 
 метрической классификации выше решения, основанного на евклидовом 
 расстоянии. Вычислительная сложность метода высокая, что ограничивает 
 его применимость на объектах высокой размерности.

\vspace*{-2pt}

{\small\frenchspacing
 {%\baselineskip=10.8pt
 \addcontentsline{toc}{section}{References}
 \begin{thebibliography}{99}
%\bibitem{Karasikov2016}
%\Au{Карасиков~М.\,Е., Стрижов~В.\,В.} Классификация временных рядов 
%в~пространстве параметров по\-рож\-да\-ющих моделей~// Информатика и~её 
%применения,~2016. T.~10. Вып.~4. С.~121--131.

\bibitem{0}
\Au{Hill~N.\,J., Lal~T.\,N., Schroder~M., Hinterberger~T., 
Wilhelm~B., Nijboer~F., Mochty~U., Widman~G., Elger~C., 
Scholkopf~B., Kubler~A., Birbaumer~N.} Classifying EEG and 
ECoG signals without subject training for fast BCI implementation: 
Comparison of nonparalyzed and completely paralyzed subjects~//  
IEEE~T. Neur. Sys. Reh., 2006. Vol.~14. 
Iss.~2. P.~183--186.

\bibitem{1}
\Au{Sakoe~H., Chiba~S.} 
A~dynamic programming approach to continuous speech recognition~// 
7th  Congress (International) on Acoustics Proceedings, 1971. Vol.~3. P.~65--69.

\bibitem{2} %3
\Au{Aghabozorgi~S., Ali~S.\,S., Wah~T.\,Y.} 
Time-series clustering~--- a~decade review~// Inform. Syst., 
2015. Vol.~53. P.~16--38.

\bibitem{3} %4
\Au{Warrenliao~T.} Clustering of time series data~--- a~survey~// 
Pattern Recogn., 2005. Vol.~38. Iss.~11. P.~1857--1874.



\bibitem{4} %5
\Au{Hautamaki~V., Nykanen~P., Franti~P.} 
Time-series clustering by approximate prototypes~// 
19th  Conference (International) on Pattern Recognition Proceedings, 2008. No.\,D. 
P.~1--4.

\bibitem{5} %6
\Au{Faloutsos~C., Ranganathan~M., Manolopoulos~Y.} 
Fast subsequence matching in time-series databases~// \mbox{SIGMOD} Rec., 1994. 
Vol.~23. Iss.~2. P.~419--429.

\bibitem{10} %7
\Au{Basalto~N., Bellotti~R., Carlo~F.\,D., Facchi~P., 
Pascazio~S.} Hausdorff clustering of financial time series~// 
Physica~A, 2007. Vol.~379. Iss.~2. P.~635--644.

\bibitem{11} %8
\Au{Gorelick~L., Blank~M., Shechtman~E., Irani~M., Basri~R.} 
Actions as space-time shapes~// IEEE~T. Pattern Anal., 
2007. Vol.~29. Iss.~12. P.~2247--2253.

\bibitem{6} %9
\Au{Smyth~P.} Clustering sequences with hidden Markov models~// 
Adv. Neural In., 1997. Vol.~9. P.~648--654.

\bibitem{7} %10
\Au{Banerjee~A., Ghosh~J.} Clickstream clustering using weighted 
longest common subsequences~// 
Workshop on Web Mining, SIAM Conference on Data Mining
Proceedings, 2001. P.~33--40.

\bibitem{8} %11
\Au{Aach~J., Church~G.M.} Aligning gene expression time series
 with time warping algorithms~// Bioinformatics, 2001. Vol.~17. Iss.~6. P.~495--508.

\bibitem{9} %12
\Au{Yi~B.\,K., Faloutsos~C.} Fast time sequence indexing 
for arbitrary $\mathcal{L}_p$ norms~// 
26th  Conference (International) on Very Large Data Bases Proceedings, 2000. P.~385--394.

\bibitem{33} %13
\Au{Goncharov~A.\,V., Strijov~V.\,V.} 
Analysis of dissimilarity set between time series~// Computational 
Mathematics Modeling, 2018. Vol.~29. Iss.~3. P.~359--366.

\bibitem{12} %14
\Au{Alon~J., Athitsos~V., Sclaroff~S.}
 Online and offline character recognition using alignment to prototypes~// 
 8th  Conference (International) on Document Analysis and Recognition, 2005. 
 Vol.~2. P.~839--843.

\bibitem{15} %15
\Au{Гончаров~А.\,В.} 
Выравнивания декартовых произведений упорядоченных множеств mDTW. 
Про\-грам\-мная реализация алгоритма, 2019. 
{\sf https://github.
com/Intelligent-Systems-Phystech/PhDThesis/tree/\linebreak  master/Goncharov2019/MatrixDTW/code}.
 \end{thebibliography}

 }
 }

\end{multicols}

\vspace*{-9pt}

\hfill{\small\textit{Поступила в~редакцию 24.04.19}}

\vspace*{6pt}

%\pagebreak

%\newpage

%\vspace*{-28pt}

\hrule

\vspace*{2pt}

\hrule

\vspace*{-4pt}

\def\tit{ALIGNMENT OF~ORDERED SET CARTESIAN PRODUCT\\[-5pt]}


\def\titkol{Alignment of~ordered set cartesian product}

\def\aut{A.\,V.~Goncharov$^1$ and~V.\,V.~Strijov$^{1,2}$}

\def\autkol{A.\,V.~Goncharov and~V.\,V.~Strijov}

\titel{\tit}{\aut}{\autkol}{\titkol}

\vspace*{-13pt}


\noindent
$^1$ Moscow Institute of Physics and Technology, 
9~Institutskiy Per., Dolgoprudny, Moscow Region 141700, Russian\linebreak
$\hphantom{^1}$Federation


\noindent
$^2$A.\,A.~Dorodnicyn Computing Center, Federal Research Center 
``Computer Science and Control'' of the Russian\linebreak
$\hphantom{^1}$Academy of Sciences, 
40~Vavilov Str., Moscow 119333, Russian Federation

\def\leftfootline{\small{\textbf{\thepage}
\hfill INFORMATIKA I EE PRIMENENIYA~--- INFORMATICS AND
APPLICATIONS\ \ \ 2020\ \ \ volume~14\ \ \ issue\ 1}
}%
 \def\rightfootline{\small{INFORMATIKA I EE PRIMENENIYA~---
INFORMATICS AND APPLICATIONS\ \ \ 2020\ \ \ volume~14\ \ \ issue\ 1
\hfill \textbf{\thepage}}}

\vspace*{2pt} 



\Abste{The work is devoted to the study of metric methods for analyzing 
objects with complex structure. It proposes to generalize the dynamic 
time warping method of two time series for the case of objects defined 
on two or more time axes. Such objects are matrices in the discrete 
representation. The DTW (Dynamic Time Warping) method of time series is generalized as 
a~method of matrices dynamic alignment. The paper proposes 
a~distance function resistant to monotonic nonlinear deformations of the 
Cartesian product of two time scales. The alignment path between objects is 
defined. An object is called a~matrix in which the rows and columns correspond 
to the axes of time. The properties of the proposed distance function 
are investigated. To illustrate the method, the problems of metric 
classification of objects are solved on model data and data from the 
MNIST dataset.}

\KWE{distance function; dynamic alignment; distance between matrices; 
nonlinear time warping; space--time series}



\DOI{10.14357/19922264200105} 

%\vspace*{-14pt}

\Ack
\noindent
This work was supported by the Russian Foundation for Basic
Research (projects 19-07-1155 and 19-07-00885). 
The paper contains results of the project Statistical 
methods of machine learning, which is carried out within the 
framework of the Program ``Center of Big Data Storage and Analysis'' 
of the National Technology Initiative Competence Center. 
It is supported by the Ministry of Science and Higher Education 
of the Russian Federation according to the agreement between the
 M.\,V.~Lomonosov Moscow State University and the Foundation 
 of project support of the National Technology Initiative from 11.12.2018, 
 No.\,13/1251/2018.
 


%\vspace*{6pt}

  \begin{multicols}{2}

\renewcommand{\bibname}{\protect\rmfamily References}
%\renewcommand{\bibname}{\large\protect\rm References}

{\small\frenchspacing
 {%\baselineskip=10.8pt
 \addcontentsline{toc}{section}{References}
 \begin{thebibliography}{99}

 \bibitem{0-1}   
\Aue{Hill, N.\,J., T.\,N.~Lal, M.~Schroder, T.~Hinterberger, B.~Wilhelm, 
F.~Nijboer, U.~Mochty, G.~Widman, C.~Elger, B.~Scholkopf, A.~Kubler, and 
N.~Birbaumer.} 2006. Classifying EEG and ECoG signals without subject 
training for fast BCI implementation: Comparison of nonparalyzed and completely 
paralyzed subjects. \textit{IEEE~T. Neur. Sys. 
Reh.} 14(2):183--186.

\bibitem{1-1}   
\Aue{Sakoe, H., and S.~Chiba.} 1971. A~dynamic programming approach 
to continuous speech recognition. \textit{7th 
 Congress (International) on Acoustics Proceedings}. 3:65--69.

\bibitem{2-1}    %2
\Aue{Aghabozorgi,~S., S.\,S.~Ali, and T.\,Y.~Wah.} 2015. 
Time-series clustering~--- a~decade review.  \textit{Inform. Syst.} 
53:16--38.

\bibitem{3-1}   %4 
\Aue{Warrenliao,~T.} 2005. Clustering of time series data~--- a~survey. 
\textit{Pattern Recogn.}
38(11):1857--1874.



\bibitem{4-1}    %5
\Aue{Hautamaki,~V., P.~Nykanen, and P.~Franti.} 2008. 
Time-series clustering by approximate prototypes. 
 \textit{19th  Conference (International) on Pattern Recognition Proceedings}. 
 D:1--4.

\bibitem{5-1}    %6
\Aue{Faloutsos,~C., M.~Ranganathan, and Y.~Manolopoulos.} 1994. 
Fast subsequence matching in time-series databases.  \textit{SIGMOD Rec}. 
23(2):419--429.

\bibitem{10-1}    %7
\Aue{Basalto, N., R.~Bellotti, F.\,D.~Carlo, P.~Facchi, and S.~Pascazio.} 
2007. Hausdorff clustering of financial time series. 
\textit{Physica~A} 379(2):635--644.

\bibitem{11-1}   %8
\Aue{Gorelick, L., M.~Blank, E.~Shechtman, M.~Irani, and R.~Basri.} 
2007. Actions as space-time shapes.
\textit{IEEE~T. Pattern Anal.} 29(12):2247--2253.

\bibitem{6-1}    %9
\Aue{Smyth, P.} 1997. 
Clustering sequences with hidden Markov models. \textit{Adv. Neural In.} 9:648--654.

\bibitem{7-1}    %10
\Aue{Banerjee,~A., and J.~Ghosh.} 2001. 
Clickstream clustering using weighted longest common subsequences.  
\textit{Workshop on Web Mining, SIAM Conference 
on Data Mining Proceedings.} 33--40.

\bibitem{8-1}    %11
\Aue{Aach, J., and G.\,M.~Church.} 2001. 
Aligning gene expression time series with time warping algorithms. 
\textit{Bioinformatics} 17(6):495--508.

\bibitem{9-1}   %12
\Aue{Yi, B.\,K., and C.~Faloutsos.} 2000. 
Fast time sequence indexing for arbitrary $\mathcal{L}_p$ norms. 
\textit{26th  Conference (International) 
on Very Large Data Bases Proceedings}. 385--394.

\bibitem{33-1}   %13 
\Aue{Goncharov,~A.\,V., and V.\,V.~Strijov.} 2018. 
Analysis of dissimilarity set between time series. 
\textit{Computational Mathematics Modeling } 29(3):359--366.



\bibitem{12-1}    %14
\Aue{Alon, J., V.~Athitsos, and S.~Sclaroff.} 2005.
 Online and offline character recognition using alignment to prototypes. 
 \textit{8th  Conference (International) on Document Analysis and Recognition}. 
 2:839--843.

\bibitem{15-1}    %15
\Aue{Goncharov, A.\,V.} Alignment of 
Ordered Set Cartesian Product mDTW. Software implementation of the algorithm. 
Available at: {\sf https://github.com/Intelligent-\linebreak 
Systems-Phystech/PhDThesis/tree/master/Goncharov\linebreak 2019/MatrixDTW/code} 
(accessed December~27, 2019).
\end{thebibliography}

 }
 }

\end{multicols}

%\vspace*{-7pt}

\hfill{\small\textit{Received April 24, 2019}}

%\pagebreak

%\vspace*{-22pt}



\Contr

\noindent
\textbf{Goncharov Alexey V.} (b.\ 1995)~--- 
PhD student, Moscow Institute of Physics and Technology, 
9~Institutskiy Per., Dolgoprudny, Moscow Region 141701, 
Russian Federation; \mbox{alex.goncharov@phystech.edu}

\vspace*{3pt}

\noindent
\textbf{Strijov Vadim V.} (b.\ 1967)~--- 
Doctor of Science in physics and mathematics, leading scientist, 
A.\,A.~Dorodnicyn Computing Centre, Federal Research Center 
``Computer Science and Control'' of the Russian Academy of Sciences, 
40~Vavilov Str., Moscow 119333, Russian Federation;
 professor, Moscow Institute of Physics and Technology, 
 9~Institutskiy Per., Dolgoprudny, Moscow Region 141701, Russian Federation; 
 \mbox{strijov@ccas.ru}
\label{end\stat}

\renewcommand{\bibname}{\protect\rm Литература} %11
\def\stat{inkova}

\def\tit{СТЕПЕНЬ СЕМАНТИЧЕСКОЙ БЛИЗОСТИ ДИСКУРСИВНЫХ ОТНОШЕНИЙ: МЕТОДЫ И~ИНСТРУМЕНТЫ РАСЧЕТА$^*$}

\def\titkol{Степень семантической близости дискурсивных отношений: методы и~инструменты расчета}

\def\aut{О.\,Ю.~Инькова$^1$, М.\,Г.~Кружков$^2$}

\def\autkol{О.\,Ю.~Инькова, М.\,Г.~Кружков}

\titel{\tit}{\aut}{\autkol}{\titkol}

\index{Инькова О.\,Ю.}
\index{Кружков М.\,Г.}
\index{Inkova O.\,Yu.}
\index{Kruzhkov M.\,G.}


{\renewcommand{\thefootnote}{\fnsymbol{footnote}} \footnotetext[1]
{Работа выполнена в~Федеральном исследовательском центре <<Информатика и~управление>> Российской 
академии наук с~использованием ЦКП <<Информатика>> ФИЦ ИУ РАН.}}


\renewcommand{\thefootnote}{\arabic{footnote}}
\footnotetext[1]{Федеральный исследовательский центр <<Информатика и~управление>> Российской академии наук; 
Женевский университет, \mbox{olyainkova@yandex.ru}}
\footnotetext[2]{Федеральный исследовательский центр <<Информатика и~управление>> Российской 
академии наук, \mbox{magnit75@yandex.ru}}

%\vspace*{-14pt}


  
  \Abst{Рассматриваются методы оценки семантической близости дискурсивных 
отношений. Авторы предлагают несколько подходов к~решению этой проблемы с~применением двух информационных ресурсов: коллекции сформированных авторами 
структурированных определений ло\-ги\-ко-се\-ман\-ти\-че\-ских отношений (ЛСО) 
и~Надкорпусной базы данных коннекторов (НБДК), включающей в~себя аннотации переводных 
соответствий текстовых фрагментов с~маркерами ЛСО на русском, французском 
и~итальянском языках. Показано, что при оценке семантической близости ЛСО высокий 
приоритет будут иметь такие факторы, как принадлежность различительных признаков ЛСО к~одному семейству в~структурированных определениях отношений, соответствия между 
показателями различных ЛСО в~оригинальных и~переводных текстах, а также случаи, когда 
различные ЛСО выражаются одинаковыми показателями в~разных контекстах. Менее значим 
фактор сочетаемости различных ЛСО в~рамках одного и~того же контекста. Предполагается, 
что на основе сформулированных методов станет возможным более точно определить, какие 
различительные признаки ЛСО имеют наибольший вес при определении их семантической  
бли\-зости.}
  
  \KW{надкорпусная база данных; логико-семантические отношения; коннекторы; 
аннотирование; фасетная классификация}

  \DOI{10.14357/19922264230412}{FXTSPZ}
  
%\vspace*{-1pt}


\vskip 10pt plus 9pt minus 6pt

\thispagestyle{headings}

\begin{multicols}{2}

\label{st\stat}
  
\section{Степень семантической близости дискурсивных 
отношений}

%\vspace*{-4pt}

  Проблемы классификации дискурсивных отношений, обеспечивающих 
связность текста, занимают лингвистов и~специалистов по автоматической 
обработке текста не один десяток лет: первые исследования начались  
в~1970-х~гг.~[1, 2]. Были предложены их многочисленные классификации (ср.\ 
наиболее известные~[3--7]), однако никто, насколько известно авторам, не 
пытался определить степень семантической близости (ССБ) дискурсивных 
отношений. Это связано прежде всего с~тем, что классификации имеют, за 
редким исключением~\cite{7-in, 8-in, 9-in}, форму списка, и~этот вопрос просто 
не ставился. Однако его решение полезно не только для анализа текста, в~том 
числе автоматического, но и~для когнитивных наук и~переводоведения, 
поскольку позволяет выявить общие закономерности человеческого мышления.
  
  Кроме того, сами дискурсивные отношения определены во многом неточно 
или тавтологично\footnote[3]{См., например, определение отношения альтернативы 
(disjunction) в~теории риторической структуры: (а)~элемент пред\-став\-ля\-ет собой (не 
обязательно исключающую) альтернативу другому; (б)~слу\-ша\-ющий/чи\-та\-тель 
распознает, что связанные элементы альтернативны (см.\ {\sf http://www.sfu.ca/rst}).}, схожие 
или идентичные отношения носят даже в~англоязычных классификациях разные 
названия, а одинаковые названия описывают разную языковую реальность. 
Например, в~теории сегментированного представления дискурса (Segmented 
Discourse Representation Theory, SDRT~[10]) отношение contrast включает как 
отношения <<вопреки ожидаемому>>, так и~уступительные отношения. 
В~классификации Пенсильванского аннотированного корпуса им 
соответствуют два отношения (opposition и~contra-expectation)~\cite{7-in}, 
а~в~теории риторической структуры~--- contrast и~concession~[11] (подробнее 
см.~\cite[с.~37]{9-in}). 

\begin{table*}[b]\small %tabl1
\vspace*{-10pt}
\begin{center}
\Caption{Структурированные определения уступительных ЛСО и~ЛСО <<вопреки 
ожидаемому>>}
\vspace*{2ex}

\tabcolsep=3pt
\begin{tabular}{|l|p{40mm}|p{38mm}|p{57mm}|}
\hline
\multicolumn{1}{|c|}{\textbf{ЛСО}} & \multicolumn{1}{c|}{\tabcolsep=0pt\begin{tabular}{c}\textbf{Базовая семантическая}\\ \textbf{операция}\end{tabular}}&
\multicolumn{1}{c|}{\textbf{Уровень}} &
\multicolumn{1}{c|}{ \tabcolsep=0pt\begin{tabular}{c}\textbf{Дополнительные}\\ \textbf{характеристики}\end{tabular}}\\
\hline
&&&\\[-20pt]
\multicolumn{1}{|l|}{\raisebox{-26pt}[0pt][0pt]{\textbf{Уступительные}}}& 
%\begin{itemize}
\multicolumn{1}{l|}{\raisebox{-26pt}[0pt][0pt]{\ \ \ \  --\ \ операция импликации}}
%\end{itemize} 
& 
%\begin{itemize}
\multicolumn{1}{l|}{\raisebox{-26pt}[0pt][0pt]{\tabcolsep=0pt\begin{tabular}{l}\ \ \ \ --\ \ пропозициональный\\
\hphantom{\ \ \ \ --\ \ }уровень\end{tabular}}}
%\end{itemize}
&
\begin{itemize}
\item $p$ и~$q$~--- положения вещей;\vspace*{-3pt}
\item как правило, если имеет место $q$, то не имеет места~$p$\vspace*{-8pt}
   \end{itemize}
\\
\hline
&&&\\[-20pt]
\multicolumn{1}{|l|}{\raisebox{-48pt}[0pt][0pt]{\tabcolsep=0pt\begin{tabular}{l}\textbf{<<Вопреки}\\ \textbf{ожидаемому>>}\end{tabular} }}& 
%\begin{itemize}
\multicolumn{1}{l|}{\raisebox{-48pt}[0pt][0pt]{\tabcolsep=0pt\begin{tabular}{l}\ \ \ \  --\ \ операция сравнения,\\
 \hphantom{\ \ \ \ --\ \ }уста\-нав\-ли\-ва\-ющая не-\\
 \hphantom{\ \ \ \ --\ \ }сходство $p$ и~$q$\end{tabular}}}
%\end{itemize} 
&
%\begin{itemize}
\multicolumn{1}{l|}{\raisebox{-48pt}[0pt][0pt]{\tabcolsep=0pt\begin{tabular}{l}\ \ \ \  --\ \ пропозициональный\\ 
 \hphantom{\ \ \ \ --\ \ }уровень\end{tabular}}}
%\end{itemize} 
&
 \begin{itemize}
 \item $q$ имеет большую аргументативную\newline силу, чем~$p$;\vspace*{-3pt}
  \item положение вещей $p$ служит аргументом в~пользу ожи\-да\-емо\-го вывода~$r$;\vspace*{-3pt}
  \item положение вещей $q$ служит аргументом в~пользу ожи\-да\-емо\-го вывода не-$r$\vspace*{-8pt}
  \end{itemize}\\
\hline
\end{tabular}
\end{center}
\end{table*}
  
  В~этой связи были сделаны попытки сравнить\linebreak существующие 
классификации, чтобы понять, насколько соотносимы выделяемые в~них 
дискурсивные отношения~[12--14]. В~[14] для этого применяется 
набор различительных признаков. Этих\linebreak признаков, однако, недостаточно, чтобы 
сформулировать уникальное определение отношения, и~некоторые из них 
имеют одинаковый набор признаков. Это касается, например, четырех 
отношений (narration, precondition, background и~parallel) в~SDRT~\cite[с.~38]{14-in}. 
  
  В~работе~[15] были заложены основы для разработки структурированных 
определений дискурсивных, или в~терминологии автора  
ло\-ги\-ко-се\-ман\-ти\-че\-ских, отношений на основе применяемой 
в~НБДК классификации. Каждое 
ЛСО может быть описано набором различительных признаков (см.\ примеры 
в~\cite{16-in} и~\cite{17-in}). Некоторые признаки оказываются общими для 
нескольких ЛСО, другие~--- индивидуальны, т.\,е.\ свойственны только данному 
ЛСО. На момент написания статьи в~НБДК были описаны 26~ЛСО 
с~использованием~52~различительных признаков. Это позволяет дать каждому 
ЛСО уникальное определение (см.\ примеры в~разд.~2), а~также определить 
ССБ ЛСО. 

\vspace*{-6pt}
  
\section{Критерии, лежащие в~основе определения степени 
семантической близости логико-семантических отношений}

\vspace*{-3pt}

  В~предыдущей работе авторов~[17] показано, что не все различительные 
признаки имеют одинаковый вес при определении семантической близости 
ЛСО и~что, предположительно, наибольшее значение имеет принадлежность 
общих признаков к~одному семейству. 
  

  
  В~основе уступительных ЛСО и~ЛСО <<вопреки ожидаемому>> лежат 
разные базовые операции: импликация~--- для первого и~сравнение, 
уста\-нав\-ли\-ва\-ющее несходство $p$ и~$q$,~--- для второго (табл.~1). Это 
значит, что эти два ЛСО находятся в~разных семантических группах. Оба ЛСО 
при этом установлены на пропозициональном уровне, т.\,е.\ непосредственно 
между положениями дел $p$ и~$q$, которые они связывают, и~оба используют 
отрицательный коррелят одного из положений вещей. Иначе говоря, признаки 
<<как правило, если имеет место~$q$, то не имеет места $p$>> и~<<положение 
вещей~$q$ служит аргументом в~пользу ожидаемого вывода не-$r$>> 
принадлежат к~одному семейству. В~примере~(1) с~ЛСО <<вопреки 
ожидаемому>>: \textit{Ему [$\ldots$] очень неприятно было сталкиваться с~народом,} {\bfseries\textit{но}} \textit{он шел именно туда, где виднелось больше 
народу}. [Ф.\,М.~Достоевский. Преступление и~наказание], положение вещей 
$p$\;=\;<<ему очень неприятно было сталкиваться с~народом>> ориентирует в~пользу вывода $r$\;=\;<<он не должен был бы идти к~народу>>. Этот вывод 
опровергается непосредственно в~$q$ (=\;не-$r$)\;=\;<<он шел именно туда, где 
виднелось больше народу>>. Семантический механизм, лежащий в~основе 
уступительных отношений (их прототипическим показателем может считаться 
союз \textit{хотя}), совпадает с~этим семантическим механизмом, но 
в~зеркальном отражении: 
  \begin{gather*}
p\ \mbox{\textit{хотя}}\  q (q \to  \mbox{не-}p)\\
p \to r\ \mbox{но}\  q\ (q = \mbox{не-}r),\ \mbox{т.\,е.}\ p \to \mbox{не-}q\ 
\mbox{\textit{но}}\ q.
\end{gather*}
  %
  Отсюда необходимость при замене \textit{хотя} на \textit{но} и~наоборот 
изменить порядок следования фрагментов текста: \textit{Ему неприятно было 
сталкиваться с~народом}, {\bfseries\textit{но}} \textit{он шел туда, где виднелось 
больше народу} (ЛСО <<вопреки ожидаемому>>); \textit{Он шел туда, где 
виднелось больше народу}, {\bfseries\textit{хотя}} \textit{ему неприятно было 
сталкиваться с~народом} (ЛСО уступки)~\cite{18-in}. Это позволяет говорить 
о~семантической близости двух ЛСО и,~например, в~классификации~\cite{7-in} 
они объединены в~одну группу concession.

\begin{table*}[b]\small %tabl2
\vspace*{-6pt}
\begin{center}
\Caption{Логико-семантические отношения, соответствующие ЛСО <<вопреки ожидаемому>> в~оригинальных и~переводных текстах }
\vspace*{2ex}

\tabcolsep=4.3pt
\begin{tabular}{|c|l|c|c|c|c|c|c|}
\hline
\textbf{ЛСО1}&\multicolumn{1}{c|}{\textbf{ЛСО2}}&\textbf{1}\;+\;\textbf{2}&\textbf{1}&
\textbf{2}&\textbf{1}\;$\to$\;\textbf{2}&\textbf{2}\;$\to$\;\textbf{1}&\textbf{Сумма}\\
\hline
<<вопреки ожидаемому>>&уступительные&237\hphantom{9}&2140&853&11,07\%\hphantom{9}&27,78\%\hphantom{9}&38,86\%\hphantom{9}\\
<<вопреки ожидаемому>>&одновременность&139\hphantom{9}&2140&1268\hphantom{9}&6,50\%&10,96\%\hphantom{9}&17,46\%\hphantom{9}\\
<<вопреки ожидаемому>>&соединительные&149\hphantom{9}&2140&2088\hphantom{9}&6,96\%&7,14\%&14,10\%\hphantom{9}\\
<<вопреки ожидаемому>>&сопоставительные&78&2140&807&3,64\%&9,67\%&13,31\%\hphantom{9}\\
<<вопреки ожидаемому>>&пропозициональное 
сопутствование&39&2140&378&1,82\%&10,32\%\hphantom{9}&12,14\%\hphantom{9}\\
<<вопреки ожидаемому>>&исключение из 
рассмотрения&\hphantom{9}8&2140&\hphantom{9}90&0,37\%&8,89\%&9,26\%\\
<<вопреки ожидаемому>>&иллокутивное 
сопутствование&17&2140&471&0,79\%&3,61\%&4,40\%\\
<<вопреки ожидаемому>>&интенсиональная 
генерализация&\hphantom{9}8&2140&248&0,37\%&3,23\%&3,60\%\\
<<вопреки ожидаемому>>&замещение&\hphantom{9}7&2140&294&0,33\%&2,38\%&2,71\%\\
<<вопреки ожидаемому>>&пропозициональная 
коррекция&\hphantom{9}4&2140&165&0,19\%&2,42\%&2,61\%\\
<<вопреки ожидаемому>>&условные&12&2140&1075\hphantom{9}&0,56\%&1,12\%&1,68\%\\
<<вопреки ожидаемому>>&спецификация&11&2140&1608\hphantom{9}&0,51\%&0,68\%&1,20\%\\
<<вопреки ожидаемому>>&исключение&\hphantom{9}5&2140&615&0,23\%&0,81\%&1,05\%\\
<<вопреки ожидаемому>>&отрицательная 
альтернатива&\hphantom{9}2&2140&271&0,09\%&0,74\%&0,83\%\\
<<вопреки ожидаемому>>&оговорка&\hphantom{9}1&2140&150&0,05\%&0,67\%&0,71\%\\
<<вопреки ожидаемому>>&экстенсиональная 
генерализация&\hphantom{9}2&2140&588&0,09\%&0,34\%&0,43\%\\
<<вопреки ожидаемому>>&переформулирование&\hphantom{9}2&2140&1183\hphantom{9}&0,09\%&0,17\%&0,26\%\\
<<вопреки ожидаемому>>&пропозициональная 
альтернатива&\hphantom{9}1&2140&1238\hphantom{9}&0,05\%&0,08\%&0,13\%\\
\hline
\multicolumn{8}{p{163mm}}{\footnotesize \hspace*{3mm}Расшифровка названий столбцов: 
1\;+\;2~--- число переводных аннотаций, в~которых ЛСО1 в~тексте на одном языке 
соответствует ЛСО2 в~тексте на другом языке; 1~--- число аннотаций, в~которых в~любом из 
текстов проставлено ЛСО1; 2~--- число аннотаций, в~которых в~любом из текстов 
проставлено ЛСО2; 1\;$\to$\;2~--- процент соответствия для ЛСО1 с~ЛСО2; 2\;$\to$\;1~--- 
процент соответствия для ЛСО2 с~ЛСО1; сумма~--- сумма двух предыдущих показателей.}
\end{tabular}
\end{center}
\end{table*}

  
  
  Кроме того, сформулирована гипотеза, согласно которой при определении 
ССБ ЛСО могут учитываться также другие 
факторы:
\begin{enumerate}[(1)] 
\item соответствия ЛСО в~оригинальных и~переводных текстах; 
\item случаи, когда разные ЛСО выражаются одним и~тем же показателем; 
\item сочетаемость показателей ЛСО в~одном фрагменте текста.
\end{enumerate}
 В~НБДК для 
ЛСО, имеющих структурированные определения, были получены 
количественные данные по этим трем критериям.

  
  
\subsection{Соответствие логико-семантических отношений в~оригинальных и~переводных текстах}

  Соответствие ЛСО в~оригинальных и~переводных текстах означает, что 
некоторому ЛСО в~тексте оригинала, точнее, его показателю, соответствует 
показатель иного ЛСО в~тексте перевода. Так, если для перевода на 
французский язык коннектора \textit{но} в~примере~(1) был выбран коннектор 
\textit{mais}, также выражающий ЛСО <<вопреки ожидаемому>>: (2)~\textit{Il 
lui $\acute{\mbox{e}}$tait d$\acute{\mbox{e}}$sagr$\acute{\mbox{e}}$able, 
tr$\grave{\mbox{e}}$s d$\acute{\mbox{e}}$sagr$\acute{\mbox{e}}$able, de 
rencontrer du monde} {\bfseries\textit{mais}} \textit{il allait justement 
l$\grave{\mbox{a}}$ o$\grave{\mbox{u}}$ l'on en voyait le plus} [перевод 
$\acute{\mbox{E}}$lisabeth Guertik], то в~примере~(3) тот же коннектор 
переведен \textit{bien que}~--- показателем уступительных ЛСО: 
\textit{С~такой поправкой смысл телеграммы становился ясен,} 
{\bfseries\textit{но}}\textit{, конечно, трагичен}.~--- \textit{Ainsi 
corrig$\acute{\mbox{e}}$, le t$\acute{\mbox{e}}$l$\acute{\mbox{e}}$gramme 
prenait un sens parfaitement clair,} {\bfseries\textit{bien que}} \textit{tragique, 
naturellement}. [М.~Булгаков. Мастер и~Маргарита, перевод Claude Ligny].
  
  Количественные данные по ЛСО, соответствующим ЛСО <<вопреки 
ожидаемому>> в~оригинальных и~переводных текстах на русском, французском и~итальянском языках, приведены в~табл.~2.
  
  
  Для ЛСО <<вопреки ожидаемому>> в~НБДК сформирована 2141~двуязычная 
аннотация. В~237~случаях ему соответствует уступительное ЛСО. Это 
подтверждает важность критерия принадлежности \mbox{различительных} признаков к~одному семейству. 

Схожую картину можно наблюдать для других отношений 
(табл.~3): для сопоставительных и~соединительных ЛСО (основаны на 
общей базовой операции и~имеют общий различительный признак 
<<сходство~$p$ и~$q$ относительно некоторого ``общего\linebreak знаменателя''>>); для 
ЛСО оговорки и~пропозициональной альтернативы (они имеют общий 
различительный признак~--- <<$p$ и~$q$~--- положения вещей, име\-ющие 
статус гипотезы>>); для ЛСО \mbox{одновременности} и~со\-по\-став\-ле\-ния (их 
различительные при\-зна\-ки <<T$p$ включает в~себя T$q$>> и~<<$p$ и~$q$ 
актуальны для говорящего в~момент речи T$d$>> принадлежат к~семейству 
признаков <<Единство временного интервала>>); для ЛСО одновременности 
и~пропозиционального сопутствования (об\-щий признак <<T$p$ включает 
в~себя T$q$>>). 
  
\begin{table*}\small %tabl3
\begin{center}
\Caption{Соответствия других ЛСО }
\vspace*{2ex}

\begin{tabular}{|l|l|c|c|c|c|c|c|}
\hline
\multicolumn{1}{|c|}{\textbf{ЛСО1}}&\multicolumn{1}{c|}{\textbf{ЛСО2}}&\textbf{1}\;+\;\textbf{2}&\textbf{1}&\textbf{2}&\textbf{1}\;
$\to$\;\textbf{2}&\textbf{2}\;$\to$\;\textbf{1}&\textbf{Сумма}\\
\hline
соединительные&сопоставительные&272\hphantom{9}&2088&807&13,03\%&33,71\%&46,73\%\\
оговорка&пропозициональная альтернатива&40&\hphantom{9}150&1238\hphantom{9}&26,67\%&\hphantom{9}3,23\%&29,90\%\\
одновременность&сопоставление&180\hphantom{9}&1268&807&14,20\%&22,30\%&36,50\%\\
одновременность &пропозициональное 
сопутствование&43&1268&378&\hphantom{9}3,39\%&11,38\%&14,77\%\\
\hline
\end{tabular}
\end{center}
\vspace*{-4pt}
\end{table*}

\begin{table*}[b]\small %tabl4
\vspace*{-12pt}
\begin{center}
\Caption{Количественные данные по ЛСО, выражаемым одним показателем}
\vspace*{2ex}

\begin{tabular}{|c|l|l|c|}
\hline 
\textbf{Язык}&\multicolumn{1}{c|}{\textbf{Коннектор}}&\multicolumn{1}{c|}{\textbf{ЛСО}}&\textbf{Количество аннотаций}\\
\hline
\multicolumn{1}{|c|}{\raisebox{-11pt}[0pt][0pt]{RU}}&\multicolumn{1}{l|}{\raisebox{-11pt}[0pt][0pt]{а то}}&отрицательная альтернатива&125\hphantom{9}\\
&&пропозициональная альтернатива&12\\
&&исключение из рассмотрения&\hphantom{9}6\\
\hline
\multicolumn{1}{|c|}{\raisebox{-6pt}[0pt][0pt]{RU}}&\multicolumn{1}{l|}{\raisebox{-6pt}[0pt][0pt]{если$\|$то}}&условные&183\hphantom{9}\\
&&сопоставительные&13\\
\hline
\multicolumn{1}{|c|}{\raisebox{-6pt}[0pt][0pt]{RU}}&\multicolumn{1}{l|}{\raisebox{-6pt}[0pt][0pt]{когда}}&одновременность&13\\
&&условные&\hphantom{9}1\\
\hline
\multicolumn{1}{|c|}{\raisebox{-6pt}[0pt][0pt]{RU}}&\multicolumn{1}{l|}{\raisebox{-6pt}[0pt][0pt]{когда$\|$то}}&одновременность&38\\
&&условные&\hphantom{9}6\\
\hline
\multicolumn{1}{|c|}{\raisebox{-11pt}[0pt][0pt]{RU}}
&\multicolumn{1}{l|}{\raisebox{-11pt}[0pt][0pt]{между тем}}
&одновременность&126\hphantom{9}\\
&&<<вопреки ожидаемому>>&53\\
&&сопоставительные&11\\
\hline
\multicolumn{1}{|c|}{\raisebox{-6pt}[0pt][0pt]{RU}}&\multicolumn{1}{l|}{\raisebox{-6pt}[0pt][0pt]{между тем как}}&сопоставительные&29\\
&&одновременность&\hphantom{9}6\\
\hline
\multicolumn{1}{|c|}{\raisebox{-18pt}[0pt][0pt]{RU}}
&\multicolumn{1}{l|}{\raisebox{-18pt}[0pt][0pt]{разве}}
&оговорка&20\\
&&исключение&\hphantom{9}5\\
&&исключение из рассмотрения&\hphantom{9}4\\
&&условные&\hphantom{9}2\\
\hline
\multicolumn{1}{|c|}{\raisebox{-6pt}[0pt][0pt]{FR}}&\multicolumn{1}{l|}{\raisebox{-6pt}[0pt][0pt]{cependant}}&<<вопреки ожидаемому>>&100\hphantom{9}\\
&&одновременность&27\\
\hline
\multicolumn{1}{|c|}{\raisebox{-6pt}[0pt][0pt]{FR}}&\multicolumn{1}{l|}{\raisebox{-6pt}[0pt][0pt]{en m$\hat{\mbox{e}}$me temps}}&одновременность&29\\
&&сопоставительные&\hphantom{9}1\\
\hline
\multicolumn{1}{|c|}{\raisebox{-6pt}[0pt][0pt]{FR}}&\multicolumn{1}{l|}{\raisebox{-6pt}[0pt][0pt]{quand}}&одновременность&197\hphantom{9}\\
&&условные&10\\
\hline
\end{tabular}
\end{center}
\end{table*}

  
  Напротив, ЛСО, соответствующие ЛСО <<вопреки ожидаемому>> 
и~представленные менее чем в~1\% аннотаций (см.\ табл.~2), не имеют 
различительных признаков, принадлежащих к~одному семейству, и~выбор их 
показателей для перевода показателя ЛСО <<вопреки ожидаемому>> может 
быть квалифицирован как авторский и~контекстуальный.
  
\subsection{Разные логико-семантические отношения выражаются одним~и~тем~же~показателем}

  Известно, что коннекторы в~значительной своей части относятся 
к~многозначным языковым единицам, т.\,е.\ могут служить показателями более 
чем одного ЛСО. Так, для русского союза \textit{и} принято выделять пять 
значений: сочинительное, временного следования, добавления,  
ре\-зуль\-та\-тив\-но-след\-ст\-вен\-ное и~несоответствия; для союза 
\textit{когда}~--- два: одновременности и~условия; у~союза \textit{но} 
выделяются собственно противительное  
и~про\-ти\-ви\-тель\-но-усту\-пи\-тель\-ное значения, а~у~\textit{хотя}~--- 
уступительное и~усту\-пи\-тель\-но-про\-ти\-ви\-тель\-ное и~т.\,д.~[19--21]. Это 
отражают и~данные НБДК, причем с~указанием на частотность того или иного 
значения коннектора в~сформированных аннотациях. 

В~табл.~4 приведены 
выборочно данные для многозначных коннекторов русского и~французского 
языков.
  

  
  Приведенные данные подтверждают прежде всего положения теории 
грамматикализации, согласно которым семантическая эволюция языковых 
единиц имеет определенные закономерности.\linebreak Так, было показано, что на основе 
значения одновременности может развиваться семантика сопоставления и~противопоставления, а~также импликации~\cite{22-in}. Это хорошо видно на 
примере \mbox{коннекторов} \textit{когда}, \textit{между тем}, а~также французских 
\textit{cependant} `в~то же время, однако', \textit{en m$\hat{\mbox{e}}$me temps} 
`в~то же время' и~\textit{quand} `когда' (см.\ табл.~4). С~другой стороны, эти 
данные подтверждают гипотезу авторов о~том, что набор ЛСО, которые может 
маркировать один показатель, не случаен, а~включает семантически близкие 
ЛСО. Так, коннектор \textit{разве} зафиксирован в~НБДК как показатель ЛСО 
оговорки, исключения, исключения из рассмотрения и~условия. Эти ЛСО имеют 
общие различительные признаки. Ло\-ги\-ко-се\-ман\-ти\-че\-ские отношения оговорки и~условия~--- два признака: 
базовая операция импликации и~признаки из семейства гипотетичность; ЛСО 
условия и~исключения устанавливаются на пропозициональном уровне, а~ЛСО 
оговорки и~исключения из рас\-смот\-ре\-ния~--- на уров\-не вы\-ска\-зы\-ва\-ния; ЛСО 
оговорки, исключения и~исключения из рас\-смот\-ре\-ния обладают общими 
признаками на уровне семейства признаков (семантика исключения), а~ЛСО 
исключения и~исключения из рас\-смот\-ре\-ния осно\-ва\-ны на общей базовой 
операции (соотнесение элемента и~множества).
  
  Таким образом, данный критерий может быть полезен при определении CСБ 
ЛСО и~иметь достаточно высокий приоритет.
  
\subsection{Сочетаемость логико-семантических отношений в~рамках одного фрагмента текста}

  Третий критерий, который можно учитывать при определении ССБ ЛСО,~--- 
сочетаемость ЛСО, точнее их показателей. Здесь, однако, возникает ряд 
сложностей, связанных с~тем, что возможность сочетаемости показателей 
зависит в~первую очередь от морфологической природы показателя ЛСО. Как 
известно, коннекторы относятся к~разнообразным морфологическим классам: 
сочинительные со\-юзы (\textit{и}, \textit{а}, \textit{но}); подчинительные союзы 
(\textit{хотя}, \textit{потому что}, \textit{как}), так называемые 
<<конкретизаторы со\-юзов>>, перешедшие в~класс коннекторов, как правило, из 
наречных выражений (\textit{в~то же время}, \textit{однако}, \textit{впрочем}); 
предлоги (\textit{кроме}, \textit{после}). Союзы, например, как сочинительные, 
так и~подчинительные, не могут сочетаться между собой в~рамках единого 
фрагмента текста, и, наоборот, наибольшей легкостью в~сочетании именно с~союзами обладают <<конкретизаторы>> (\textit{но однако}, \textit{но впрочем}, 
\textit{а~между тем}, \textit{или например}, \textit{и~в~частности}). Если для 
показателей некоторых ЛСО можно выявить закономерности, то другие менее 
избирательны в~своих сочетаниях. Так, показатель ЛСО спецификации 
\textit{например} сочетается со всеми сочинительными союзами, а~показатель 
ЛСО <<вопреки ожидаемому>> \textit{впрочем} только с~союзами~\textit{а} 
и~\textit{но}, т.\,е.\ показателями близких ему (\textit{а}) или тех же (\textit{но}) 
ЛСО. Можно также учитывать двухместные реализации коннекторов, т.\,е.\ 
такие, где компоненты коннектора находятся в~каждом из соединяемых 
фрагментов текста, например \textit{хотя$\ldots$\ но}: \textit{Хотя он меня 
очень уговаривал, но я~не согласился}. Но такие сочетания возможны не для 
всех ЛСО и~сужают круг возможностей для получения адекватных 
количественных данных.
 
  В~связи с~вышесказанным при подсчете ССБ ЛСО этот критерий может 
использоваться лишь как дополнительный.
  
\section{Заключение}

  Из четырех рассмотренных критериев определения ССБ ЛСО: 
(1)~принадлежности различительных признаков ЛСО к~одному семейству, 
(2)~соответствия ЛСО в~оригинальных и~переводных \mbox{текс\-тах}, (3)~возможности 
одного показателя выражать разные ЛСО и~(4)~сочетаемости показателей ЛСО 
в~одном фрагменте текста~--- первые три могут иметь достаточно высокий 
приоритет. Четвертый признак обладает, напротив, наименьшим весом при 
определении ССБ ЛСО. 
  
  Степень детальности разметки, а следовательно, и~определений ЛСО не 
позволяет пока объяснить некоторые явления. Например, семантическую 
близость ЛСО условия и~одновременности, который подтверждается как их 
соответствиями в~оригинальных и~переводных текстах, так и~воз\-мож\-ностью 
выражаться одним показателем (\textit{когда}). Их общий признак <<T$p$ 
включает в~себя T$q$>> не входит в~определение условных ЛСО, так как 
соотношение временн$\acute{\mbox{ы}}$х планов положений вещей~$p$ и~$q$ может быть 
самым различным в~условном периоде. С~другой стороны, при ЛСО 
одновременности различным может быть их семантическое соотношение 
(семантическая независимость, противопоставленность, причина, следствие 
и~т.\,д.). Перевод показателя ЛСО одновременности показателем условных 
ЛСО наблюдается только при одновременной реализации положений 
вещей~$p$ и~$q$ и~при возможности установить между ними отношение 
импликации. Семантическая близость данных двух ЛСО может быть, 
следовательно, установлена на более низком иерархическом уровне, а~именно: 
при определении частных случаев его реализации. В~НБДК такая возможность 
предусмотрена, что позволит в~дальнейшем более детально описывать каждое 
ЛСО и~его виды, а~значит, более точно определить ССБ ЛСО.
{\looseness=1

}
  
{\small\frenchspacing
 {\baselineskip=10.6pt
 %\addcontentsline{toc}{section}{References}
 \begin{thebibliography}{99}
\bibitem{1-in}
\Au{Hobbs J.\,R.} A~computational approach to discourse analysis.~--- 
New York, NY, USA: Department of Computer Science, City College, City University of New 
York, 1976.  Research Report 76-2. P.~28--38.
\bibitem{2-in}
\Au{Hobbs J.\,R.} Why is discourse coherent?~--- Menlo Park, CA, 
USA: SRI International, 1978. SRI Technical Note 176. 44~p.
\bibitem{3-in}
\Au{Halliday M.\,A.\,K., Hasan~R.}  Cohesion in English.~--- London: Longman, 1976. 374~p.


\bibitem{5-in} %4
\Au{Mann W.\,C., Thompson~S.\,A.} Rhetorical structure theory: Towards a functional theory of 
text organization~// Text, 1988. Vol.~8. No.\,3. P.~243--281. doi: 10.1515/text.\linebreak  1.1988.8.3.243.

\bibitem{6-in} %5
\Au{Asher N.} Reference to abstract objects in discourse.~--- Dordrecht: Kluwer, 1993. 455~p.

\bibitem{4-in} %6
\Au{Halliday M.\,A.\,K.} An introduction to functional grammar.~--- 2nd ed.~--- London: 
Edward Arnold, 1994. 434~p.

\bibitem{7-in} %7
PDTB Research Group. The Penn Discourse Treebank 2.0 annotation manual.~--- Philadelphia, PA, USA: Institute for Research in Cognitive Science, University 
of Pennsylvania, 2007.  Technical Report 
IRCS-08-01. 104~p. {\sf https://www.cis.upenn.edu/$\sim$elenimi/\linebreak pdtb-manual.pdf}.
\bibitem{8-in}
\Au{Breindl E., Volodina~A., \mbox{Wa{\!\ptb{\!\ss}}\,ner}~U.\,H.} Handbuch der deutschen 
Konnektoren~2: Semantik der deutschen Satzverkn$\ddot{\mbox{u}}$pfer.~--- Berlin: Walter de Gruyter, 2014. 
1327~p.
\bibitem{9-in}
\Au{Инькова О.\,Ю.} Логико-се\-ман\-ти\-че\-ские отношения: проблемы 
классификации~// Связность текста: мереологические ло\-ги\-ко-се\-ман\-ти\-че\-ские 
отношения.~--- М.: ЯСК, 2019. С.~11--98.
\bibitem{10-in}
\Au{Asher N., Lascarides~A.} Logics of conversation.~--- Cambridge: Cambridge University 
Press, 2003. 526~p.
\bibitem{11-in}
\Au{Carlson L., Marcu D.} Discourse tagging reference manual.~--- Marina del Rey, CA, USA: Information Sciences Institute, University of Southern 
California, 2001.  Technical Report ISI-TR-545. 87~p.



\bibitem{13-in} %12
\Au{Chiarcos Ch.} Towards interoperable discourse annotation: Discourse features in the 
Ontologies of Linguistic Annotation~// 9th Conference (International) on Language Resources 
and Evaluation Proceedings~/ Eds.\ N.~Calzolari, K.~Choukri, T.~Declerck, \textit{et al.}~--- Reykjavik, Iceland: European Language Resources Association 
(ELRA), 2014. P.~4569--4577.

\bibitem{12-in} %13
\Au{Benamara F., Taboada~M.} Mapping different rhetorical relation annotations: A~proposal~// 
4th Joint Conference on Lexical and Computational Semantics  Proceedings~/ Eds.\ M.~Palmer, G.~Boleda, P.~Rosso.~--- Denver, CO, USA: 
Association for Computational Linguistics, 2015. Р.~147--152. doi: 10.18653/v1/S15-1016.

\bibitem{14-in}
\Au{Sanders T., Demberg~V., Hoek~J., Scholman~M., Asr~F.\,T., Zufferey~S., Evers-Vermeul~J.} 
Unifying dimensions in coherence relations: How various annotation frameworks are related~// 
Corpus Linguist. Ling., 2018. Vol.~17. No.\,1. P.~1--71. doi:  
10.1515/cllt-2016-0078.
\bibitem{15-in}
\Au{Инькова О.\,Ю.} Определения дискурсивных отношений: опыт Надкорпусной базы 
данных коннекторов~// Компьютерная лингвистика и~интеллектуальные технологии: По 
мат-лам ежегодной \mbox{Междунар.} конф. <<Диалог>>.~--- М.: РГГУ, 2021. Вып.~20(27). 
С.~328--338.
\bibitem{16-in}
\Au{Инькова О.\,Ю., Кружков М.\,Г.} Структурированные определения дискурсивных 
отношений в~Надкорпусной базе данных коннекторов~// Информатика и~её применения, 
2021. Т.~15. Вып.~4. С.~27--32. doi: 10.14357/19922264210404. EDN: EZJXVI.

\bibitem{17-in}
\Au{Инькова О.\,Ю., Кружков М.\,Г.} Критерии определения семантической близости 
дискурсивных отношений~// Информатика и~её применения, 2023. Т.~17. Вып.~3.  
С.~100--106. doi: 10.14357/19922264230314. EDN: UJZJZI.

\bibitem{18-in}
\Au{Инькова О.\,Ю., Нуриев В.\,А.} Насколько лингвоспецифичен союз \textit{хотя}?~// 
Компьютерная лингвистика и~интеллектуальные технологии: По мат-лам ежегодной 
Междунар. конф. <<Диалог>>.~--- М.: РГГУ, 2018. Вып.~17(24). С.~254--266.

\bibitem{20-in} %19
Словарь современного русского литературного языка: в~17~т.~/ Под ред. 
В.\,И.~Чернышева.~--- М., Л.: Изд-во Академии наук СССР~/ Наука, 1950--1965.

\bibitem{19-in} %20
Русская грамматика~/ Под ред. Н.\,Ю.~Шведовой.~--- М.: Наука, 1980.   Т.~2.
714~с.

\bibitem{21-in}
Словарь русского языка: в~4~т.~/ Под ред. А.\,П.~Ев\-гень\-евой.~--- М.: Русский язык, 
 1981--1984. 
\bibitem{22-in}
\Au{Heine B., Kuteva T.} World lexicon of grammaticalization.~--- Cambridge: Cambridge 
University Press, 2002. 387~p.
\end{thebibliography}

 }
 }

\end{multicols}

\vspace*{-10pt}

\hfill{\small\textit{Поступила в~редакцию 15.10.23}}

\vspace*{8pt}

%\pagebreak

%\newpage

%\vspace*{-28pt}

\hrule

\vspace*{2pt}

\hrule



\def\tit{EVALUATING THE DEGREE OF~DISCOURSE RELATIONS SEMANTIC AFFINITY: 
METHODS AND~INSTRUMENTS}


\def\titkol{Evaluating the degree of~discourse relations semantic affinity: 
Methods and instruments}


\def\aut{O.\,Yu.~Inkova$^{1,2}$ and~M.\,G.~Kruzhkov$^1$}

\def\autkol{O.\,Yu.~Inkova and~M.\,G.~Kruzhkov}

\titel{\tit}{\aut}{\autkol}{\titkol}

\vspace*{-14pt}


\noindent
$^1$Federal Research Center ``Computer Science and Control'' of the Russian Academy of Sciences, 
44-2~Vavilov\linebreak
$\hphantom{^1}$Str., Moscow 119333, Russian Federation

\noindent
$^2$University of Geneva, 22 Bd des Philosophes, CH-1205 Geneva 4, Switzerland


\def\leftfootline{\small{\textbf{\thepage}
\hfill INFORMATIKA I EE PRIMENENIYA~--- INFORMATICS AND
APPLICATIONS\ \ \ 2023\ \ \ volume~17\ \ \ issue\ 4}
}%
 \def\rightfootline{\small{INFORMATIKA I EE PRIMENENIYA~---
INFORMATICS AND APPLICATIONS\ \ \ 2023\ \ \ volume~17\ \ \ issue\ 4
\hfill \textbf{\thepage}}}

\vspace*{3pt}




\Abste{The methods for evaluating semantic affinity of discourse relations are examined. The 
authors propose several approaches to this problem using two information resources: 
a~collection of structured definitions of logical-semantic relations (LSRs) formed by the authors
and the Supracorpora 
Database of Connectives incorporating\linebreak\vspace*{-12pt}}

\Abstend{corpus-based annotations of translation correspondences 
that include text fragments with LSR markers in Russian,
French, and Italian. It is demonstrated that when it comes to 
assessing the semantic affinity of LSRs, the following factors will be of a~higher priority: affiliation of 
distinctive features of LSRs with the same family in the structured definitions of relations; correspondences 
between markers of different LSRs in the source and target texts; and cases when different LSRs are 
regularly expressed by the same markers in different contexts. Of a~lesser importance is the factor of 
compatibility of different LSRs within the same context. It is assumed that based on the proposed 
methods, it will become possible to specify more precisely which distinguishing features of LSRs 
have the greatest impact on their potential semantic affinity.}

\KWE{supracorpora database; logical-semantic relations; connectives; annotation; faceted 
classification}


  \DOI{10.14357/19922264230412}{FXTSPZ}

\vspace*{-16pt}

\Ack

\vspace*{-3pt}

\noindent
The research was carried out using the infrastructure of the Shared Research Facilities ``High 
Performance Computing and Big Data'' (CKP ``Informatics'') of FRC CSC RAS (Moscow).


\vspace*{6pt}

  \begin{multicols}{2}

\renewcommand{\bibname}{\protect\rmfamily References}
%\renewcommand{\bibname}{\large\protect\rm References}

{\small\frenchspacing
 {%\baselineskip=10.8pt
 \addcontentsline{toc}{section}{References}
 \begin{thebibliography}{99}
\bibitem{1-in-1}
\Aue{Hobbs, J.\,R.} 1976. A~computational approach to discourse analyses. New York, NY: 
Department of Computer Science, City College, City University of New York. Research Report  
76-2. 28--38.
\bibitem{2-in-1}
\Aue{Hobbs, J.\,R.} 1978. Why is discourse coherent? Menlo Park, CA: SRI International. SRI 
Technical Note 176. 44~p.
\bibitem{3-in-1}
\Aue{Halliday, M.\,A.\,K., and R.~Hasan.} 1976. \textit{Cohesion in English}. London: Longman. 
374~p.


\bibitem{5-in-1} %4
\Aue{Mann, W.\,C., and S.\,A.~Thompson.} 1988. Rhetorical structure theory: Towards 
a~functional theory of text organization. \textit{Text} 8(3):243--281. doi: 
10.1515/text.1.1988.8.3.243.
\bibitem{6-in-1} %5
\Aue{Asher, N.} 1993. \textit{Reference to abstract objects in discourse}. Dordrecht: Kluwer. 
455~p.
\bibitem{4-in-1} %6
\Aue{Halliday, M.\,A.\,K.} 1994. \textit{An introduction to functional grammar}. 2nd ed. London: 
Edward Arnold. 434~p.

\bibitem{7-in-1}
PDTB Research Group. 2007. The Penn Discourse Treebank 2.0 annotation manual. Philadelphia, 
PA: Institute for Research in Cognitive Science, University of Pennsylvania. Technical Report 
IRCS-08-01. 104~p. Available at: {\sf https://www.cis.upenn.edu/$\sim$elenimi/pdtb-manual.pdf} 
(accessed November~28, 2023).
\bibitem{8-in-1}
\Aue{Breindl, E., A.~Volodina, and U.\,H.~Wa{\!\ptb{\!\ss}}ner.} 2014. \textit{Handbuch der 
deutschen Konnektoren~2: Semantik der deutschen Satzverkn$\ddot{\mbox{u}}$pfer}. Berlin: Walter de Gruyter. 
1327~p.
\bibitem{9-in-1}
\Aue{Inkova, O.\,Yu.} 2019. Logiko-semanticheskie otnosheniya: problemy klassifikatsii  
[Logical-semantic relations: Classification problems]. \textit{Svyaznost' teksta: mereologicheskie 
logiko-semanticheskie otnosheniya} [Text coherence: Mereological logical semantic relations]. 
Moscow: LRC Publishing House. 11--98.
\bibitem{10-in-1}
\Aue{Asher, N., and A.~Lascarides.} 2003. \textit{Logics of conversation}. Cambridge: Cambridge 
University Press. 526~p.
\bibitem{11-in-1}
\Aue{Carlson, L., and D.~Marcu.} 2001. Discourse tagging reference manual.  Marina del Rey, CA: Information Sciences Institute, University of Southern 
California. Technical Report 
ISI-TR-545.  87~p. Available at: {\sf https://www.isi.edu/~marcu/discourse/tagging-ref-manual.pdf} 
(accessed November~28, 2023).

\bibitem{13-in-1} %12
\Aue{Chiarcos, Ch.} 2014. Towards interoperable discourse annotation: Discourse features in the 
Ontologies of Linguistic Annotation. \textit{9th Conference (International) on\linebreak Language Resources 
and Evaluation Proceedings}. Eds. N.~Calzolari, K.~Choukri, T.~Declerck, \textit{et al.} Reykjavik, Iceland: 
European Language Resources Association. 4569--4577.
{ %\looseness=1

}

\bibitem{12-in-1} %13
\Aue{Benamara, F., and M.~Taboada.} 2015. Mapping different rhetorical relation annotations: 
A~proposal. \textit{4th Joint Conference on Lexical and Computational Semantics}. Eds. 
M.~Palmer, G.~Boleda, and P.~Rosso. Denver, CO, USA: Association for Computational 
Linguistics. 147--152. doi: 10.18653/v1/S15-1016.

\bibitem{14-in-1}
\Aue{Sanders, T., V.~Demberg, J.~Hoek, M.~Scholman, F.\,T.~Asr, S.~Zufferey, and  
J.~Evers-Vermeul.} 2018. Unifying dimensions in coherence relations: How various annotation 
frameworks are related. \textit{Corpus Linguist. Ling.} 17(1):1--71. doi: 10.1515/cllt-2016-0078.
\bibitem{15-in-1}
\Aue{Inkova, O.\,Yu.} 2021. Opredeleniya diskursivnykh otnosheniy: opyt Nadkorpusnoy bazy 
dannykh konnektorov [Definition of discursive relations: The experience of the supracorpora 
database of connectors]. \textit{Komp'yuternaya lingvistika i~intellektual'nye Tekhnologii: Po 
mat-lam ezhegodnoy Mezhdunar.  konf. ``Dialog''} [Computational Linguistics 
and Intellectual Technologies: Papers from the Annual Conference (International) ``Dialogue'']. 
Moscow: RGGU. 20(27):328--338.
\bibitem{16-in-1}
\Aue{Inkova, O.\,Yu., and M.\,G.~Kruzhkov.} 2021. Strukturirovannye opredeleniya 
diskursivnykh otnosheniy v~Nadkorpusnoy baze dannykh konnektorov [Structured definitions of 
discourse relations in the Supracorpora Database of Connectives]. \textit{Informatika i~ee 
Primeneniya~--- Inform. Appl.} 15(4):27--32. doi: 10.14357/ 19922264210404. EDN: EZJXVI.
\bibitem{17-in-1}
\Aue{Inkova, O.\,Yu., and M.\,G.~Kruzhkov.} 2023. Kriterii opredeleniya semanticheskoy blizosti 
diskursivnykh otnosheniy [Evaluation criteria for discourse relations semantic affinity]. 
\textit{Informatika i~ee Primeneniya~--- Inform. Appl.} 17(3):100--106. doi: 
10.14357/19922264230314. EDN: UJZJZI.

\pagebreak


\bibitem{18-in-1}
\Aue{Inkova, O.\,Yu., and V.\,A.~Nuriev.} 2018. Naskol'ko lingvospetsifichen soyuz \textit{khotya}? [To 
what extent is the conjunction \textit{khotya} language-specific?]. \textit{Komp'yuternaya lingvistika 
i~intellektual'nye tekhnologii: Po mat-lam ezhegodnoy Mezhdunar. konf. ``Dialog''} 
[Computational Linguistics and Intellectual Technologies: Papers from the Annual Conference 
(International) ``Dialogue'']. Moscow: RGGU. 17(24):254--266. 

\bibitem{20-in-1} %19
Chernyshev, V.\,I., ed. 1950--1965. \textit{Slovar' sovremennogo russkogo literaturnogo yazyka} 
[Dictionary of modern Russian literary language]. In 17~vols. Moscow, Leningrad: USSR Academy 
of Sciences Publishing House/Nauka.

\bibitem{19-in-1} %20
Shvedova, N.\,Yu., ed. 1980. \textit{Russkaya grammatika} [Russian grammar]. Moscow: Nauka. Vol.~2. 714~p.

\bibitem{21-in-1} %21
Evgen'eva, A.\,P., ed. 1981--1984. \textit{Slovar' russkogo yazyka} [Dictionary of the Russian 
language].  Moscow: Russkiy yazyk. 4~vols.


\bibitem{22-in-1}
\Aue{Heine, B., and T.~Kuteva.} 2002. \textit{World lexicon of grammaticalization}. Cambridge: 
Cambridge University Press. 387~p.

\end{thebibliography}

 }
 }

\end{multicols}

\vspace*{-6pt}

\hfill{\small\textit{Received October 5, 2023}} 

%\vspace*{-18pt}

\Contr

\vspace*{-4pt}

\noindent
\textbf{Inkova Olga Yu.} (b.\ 1965)~--- Doctor of Science in philology, senior scientist, Federal 
Research Center ``Computer Science and Control'' of the Russian Academy of Sciences,  
44-2~Vavilov Str., Moscow 119333, Russian Federation; faculty member, University of Geneva, 
22~Bd des Philosophes, CH-1205 Geneva~4, Switzerland; \mbox{olyainkova@yandex.ru}

\vspace*{3pt}

\noindent
\textbf{Kruzhkov Mikhail G.} (b.\ 1975)~--- senior scientist, Federal Research Center ``Computer 
Science and Control'' of the Russian Academy of Sciences, 44-2~Vavilov Str., Moscow 119333, 
Russian Federation; \mbox{magnit75@yandex.ru}


\label{end\stat}

\renewcommand{\bibname}{\protect\rm Литература}  %12
\def\stat{zatsman}

\def\tit{ТРАНСФОРМАЦИИ ОБЪЕКТОВ ПЕРВОГО И~ВТОРОГО ПОРЯДКА 
В~ЛЕКСИКОГРАФИЧЕСКОЙ ИНФОРМАЦИОННОЙ СИСТЕМЕ$^*$}

\def\titkol{Трансформации объектов первого и~второго порядка 
в~лексикографической информационной системе}

\def\aut{И.\,М.~Зацман$^1$}

\def\autkol{И.\,М.~Зацман}

\titel{\tit}{\aut}{\autkol}{\titkol}

\index{Зацман И.\,М.}
\index{Zatsman I.\,M.}


{\renewcommand{\thefootnote}{\fnsymbol{footnote}} \footnotetext[1]
{Исследование выполнено в~ФИЦ ИУ РАН за счет гранта Российского научного фонда №\,24-18-00155, {\sf 
https://rscf.ru/project/24-18-00155}. Работа выполнялась с~использованием инфраструктуры Центра 
коллективного пользования <<Высокопроизводительные вычисления и~большие данные>> (ЦКП 
<<Информатика>>) ФИЦ ИУ РАН (г.\ Москва).}}


\renewcommand{\thefootnote}{\arabic{footnote}}
\footnotetext[1]{ Федеральный исследовательский центр <<Информатика и~управление>> Российской академии наук, 
\mbox{izatsman@yandex.ru}}

\vspace*{-12pt}


  
  \Abst{Рассматриваются теоретические основания проектирования информационных 
технологий (ИТ) интеграции двуязычных словарей и~параллельных корпусов. Дано описание 
первых результатов создания третьего уровня классификации трансформаций объектов 
предметной области информатики, которую предполагается использовать при создании 
концепции лексикографической информационной системы, обеспечивающей интеграцию. 
Все сущности информатики в~статье разделены на два глобальных класса: объекты и~их 
трансформации. Для каждого такого класса конструируется своя классификация. Ранее были 
описаны два верхних уровня классификации трансформаций объектов предметной области. 
В~данной статье рассматривается третий уровень этой классификации. Основанием для 
построения самого верхнего ее уровня служило деление предметной области информатики 
на среды (ментальная, сенсорно воспринимаемая, цифровая и~ряд других сред), каждая из 
которых по определению включает объекты одной природы. Основанием для построения 
второго уровня классификации трансформаций объектов служила типология знаковых  
сис\-тем А.~Соломоника. Цель статьи состоит в~систематизации трансформаций первого 
и~второго порядка объектов предметной области на третьем уровне этой классификации. 
Основанием для систематизации служит средовая версия иерархии Акоффа.}
  
  \KW{объекты предметной области; трансформации объектов; классификация; данные; 
информация; знание; лексикографическая информационная сис\-тема}

\DOI{10.14357/19922264240211}{VZTGVV}
  
\vspace*{3pt}


\vskip 10pt plus 9pt minus 6pt

\thispagestyle{headings}

\begin{multicols}{2}

\label{st\stat}
  
\section{Введение}

\vspace*{-9pt}

  Возникновение параллельных корпусов, в~которых предложениям 
оригинального текста со\-по\-став\-ле\-ны предложения его перевода, обеспечило 
возможность контрастивного лингвистического\linebreak \mbox{анализа} на принципиально 
новом уровне полноты и~точности, недостижимом в~докорпусную эпоху. 
Пионерскими в~этой области стали работы \mbox{1990-х~гг}. Стига Йоханссона  
с~анг\-ло-нор\-веж\-ским корпусом~[1]. В России параллельные корпусы стали 
формироваться в~начале XXI~века в~рамках Национального корпуса русского 
языка~[2].
  
  Создатели двуязычных словарей используют параллельные корпусы для 
сбора материала и~эмпирической проверки своих гипотез, касающихся 
межъязы\-ко\-вой эквивалентности. Ценность параллельных корпусов 
определяется тем, что в~лингвистике этап сбора исходного материала считается 
наиболее трудоемким и~наименее творческим, а~параллельные корпусы 
позволяют значительно сэкономить время и~силы для творческого этапа 
создания словарей~[3].
 % 
  При этом двуязычные словари, создаваемые на основе исходного материала, 
извлеченного из параллельных корпусов, сейчас формируются без связей с~их 
текстами. Другими словами, онлайновые связи созданных словарей 
с~параллельными корпусами, которые служили источниками исходного 
материала, отсутствуют. 

Параллельные корпусы постоянно пополняются 
новыми текстами, в~предложениях которых можно обнаружить новые значения 
слов и~устойчивых словосочетаний. Однако при этом отсутствуют методы 
и~средства оперативного обновления словарей по корпусным данным. 
В~настоящее время проблема установления связей между двуязычными 
словарями и~параллельными корпусами (далее~--- проблема интеграции) 
находится на стадии поиска концептуальных подходов к~их интеграции на 
уровне значений.
  
  Подход к~решению проблемы интеграции, предлагаемый в~статье, учитывает 
  и~появление новых значений слов и~устойчивых словосочетаний, и~динамику 
смысловых значений, которая обусловлена развитием и~пополнением знания 
лингвистов, фиксирующих эти значения в~результате семантического анализа 
пополняемых корпусных данных. Проведенные эксперименты показали, что 
обнаружение нового лингвистического знания обусловливает и~формирование 
дефиниций новых значений, и~пересмотр уже существующих дефиниций~[4, 5].
  
  Например, в~проведенных экспериментах с~использованием ЦКП 
<<Информатика>> ФИЦ ИУ РАН фиксировалась эволюция значений немецких 
модальных глаголов, исходное состояние значений которых было описано 
в~не\-мец\-ко-рус\-ском словаре. В~экспериментальном массиве текстов как 
потенциальных источниках нового знания 16\,268 предложений содержали 
немецкие модальные глаголы и~в~2041 из них встречался глагол sollen. 
В~начале эксперимента в~словаре были описаны~12~значений этого модального 
глагола. По окончании эксперимента лингвисты обнаружили два новых его 
значения, согласовали их дефиниции и~описали эволюцию дефиниций~[6, 7].
  
  Таким образом, для решения проблемы интеграции требуется фиксировать 
новое знание, обнаруженное лингвистами в~текстовых данных параллельных 
корпусов, отслеживать эволюцию знания, представленного в~виде дефиниций 
значений слов и~устойчивых словосочетаний, и,~соответственно, 
актуализировать электронные двуязычные словари. Предлагаемый 
концептуальный подход к~интеграции, который планируется реализовать 
в~процессе проектирования лексикографической информационной сис\-те\-мы, 
фиксирующей эволюцию лингвистического знания, основан на решении 
следующих задач:\\[-14pt]
  \begin{itemize}
  \item категоризация трех базовых понятий информатики, включенных 
  в~иерархию Акоффа~[8] (данные, информация, знание), на объекты 
проектируемой сис\-те\-мы, которая необходима, чтобы фиксировать 
<<кванты>> нового знания и~отслеживать его эволюцию в~этой сис\-теме;\\[-15pt]
  \item  систематизация трансформаций объектов этой сис\-темы.\\[-14pt]
  \end{itemize}
  
  Цель статьи и~состоит в~решении двух задач: категоризации трех базовых 
понятий информатики на объекты лексикографической информационной  
сис\-те\-мы и~сис\-те\-ма\-ти\-за\-ции трансформаций первого и~второго порядка 
ее объектов.
  
  Трансформациями первого порядка, о которых сказано в~формулировке цели 
статьи, называются взаимные преобразования между двумя объектами  
сис\-те\-мы одной природы. Например, перевод в~сис\-те\-ме текста с~русского 
языка на английский относится к~ним. Трансформациями второго порядка 
и~выше называются взаимные преобразования между двумя и~более объектами 
разной природы. Например, кодирование символов текс\-та компьютерными 
кодами и~их декодирование относятся по определению к~трансформациям 
второго порядка.

%\vspace*{-9pt}
  
\section{Процессы трансформаций в~информатике}

%\vspace*{-3pt}

Процессы трансформаций, рассматриваемые в~статье, относятся к~теоретическому ядру информатики, а~не 
только к~проектированию лексикографической информационной сис\-те\-мы. Например, из трех основных 
подходов к~описанию предметной об\-ласти информатики\footnote{В статье предметная область информатики 
трактуется согласно концепции полиадического компьютинга Пола Розенблума~\cite{9-zac}.} (объектный, 
трансформационный и~синтетический) сис\-те\-ма\-ти\-за\-ция трансформаций ближе всего ко второму 
подходу. Примерами первого подхода, в~рамках которого основное внимание уделяется объектам предметной 
области информатики и~в~меньшей степени отношениям\linebreak между ними, могут служить  
работы~\cite{8-zac, 10-zac, 11-zac}; \mbox{примерами} второго подхода, в~рамках которого основное внимание 
уделяется трансформациям и~в~меньшей степени трансформируемым объектам,~---  
работы~\cite{12-zac, 13-zac}; примерами третьего, синтетического подхода, в~котором уделяется внимание 
и~объектам предметной об\-ласти информатики, и~отношениям между ними, могут служить работы~\cite{14-zac, 
15-zac, 16-zac, 17-zac, 18-zac}.

  Таким образом, для описания трансформаций объектов лексикографической 
информационной\linebreak системы предпочтительнее всего трансформационный 
подход, который упоминается и~в определениях информатики. Например, 
в~2009~г.\ П.~Деннинг и~П.~Розенблум сформулировали суть \mbox{информатики} как 
компьютинга следующим образом: <<$\ldots$информатика~--- это не просто 
алгоритмы и~структуры данных; это преобразования [трансформации] 
представлений>>~\cite{12-zac}. Чуть позже, в~контексте краткого описания 
парадигмы информатики как компьютинга, П.~Деннинг и~П.~Фриман изменили 
эту формулировку на такую: <<Центральный объект внимания в~информатике 
можно определить как информационные процессы~--- \textit{естественные или 
искусственные процессы, преобразующие информацию} (курсив мой~--- 
И.\,З.)>>~\cite{13-zac}. Согласно парадигме, предлагаемой авторами этой 
статьи, на начальном этапе проектирования автоматизированных систем 
базовыми элементами моделей их функционирования служат 
\textit{информационные про\-цессы}.
  
  Однако если 15~лет назад в~формулировке из работы~\cite{13-zac} шла речь 
о~процессах, преобразующих информацию, то в~последние~10~лет в~спектр 
процессов трансформаций все чаще стали включать процессы, преобразующие 
не только информацию, но также и~другие объекты автоматизированных 
систем, в~первую очередь данные и~знания~[19--21]. Например, Виктория 
Стодден, позиционируя науку о~данных как одну из дисциплин информатики, 
говорит, что центральный объект исследований в~науке о~данных~--- это 
<<изучение обобщаемого извлечения знания из данных>>~\cite{21-zac}. 
Увеличение и~чис\-ла объектов, и~спект\-ра процессов их трансформаций 
в~автоматизированных сис\-те\-мах обуслов\-ли\-ва\-ет не\-об\-хо\-ди\-мость 
систематизации и~объектов, и~процессов их трансформаций на начальном этапе 
проектирования сис\-тем.
  
  Для создания концепции лексикографической информационной сис\-те\-мы 
и~проектирования ИТ, обеспечивающих интеграцию 
двуязычных словарей и~параллельных корпусов, сначала выполним 
категоризацию на объекты этой сис\-те\-мы трех базовых понятий информатики 
(данные, информация, знание) в~контексте построения классификаций 
сущностей ее предметной об\-ласти.
  
  Необходимость использования классификаций информатики в~процессе 
создания концепции проиллюстрируем, используя иерархию  
Акоффа~\cite{8-zac}. Он использовал принцип их вертикального размещения 
в~иерархии снизу вверх: данные, информация и~знание. Еще в~ней есть термин 
<<мудрость>>, который в~статье не рассматривается. Такое размещение Акофф 
прокомментировал так: <<Каждое из пе\-ре\-чис\-лен\-ных понятий [кроме данных] 
содержит в~себе нижестоящие$\ldots$>>~\cite{8-zac}.
  
  Этому принципу размещения и~комментарию Акоффа свойственны 
недостатки, проанализированные, в~частности, в~работе~\cite{10-zac}. Главный 
вывод, к~которому пришла Роули после изучения иерархии Акоффа, 
заключается в~следующем: <<$\ldots$информация определяется в~терминах 
данных, знание~--- в~терминах информации$\ldots$ но существует меньше 
консенсуса в~описании трансформаций, которые преобразуют сущности, 
расположенные ниже в~иерархии, в~те, которые находятся над ними, что 
приводит к~их терминологической неопределенности>>~\cite{10-zac}. Причина 
этой неопределенности, скорее всего, в~том, что базовые понятия информатики 
включены в~иерархию Акоффа изолированно от общего контекста 
классификаций сущностей ее предметной об\-ласти.

%\vspace*{-9pt}
  
\section{Классификации сущностей информатики}


%\vspace*{-2pt}

  Все сущности предметной области информатики в~работах~[22, 23] 
разделены на два глобальных класса: ее объекты и~их трансформации. Для 
каждого такого класса была предложена своя классификация. 
В~работе~\cite{22-zac} дано описание классификации объектов предметной 
области информатики, первый уровень которой содержит базовые понятия ее 
предметной области (данные, информация, знания и~др.).  
В~работе~\cite{23-zac} дано описание двух верхних уровней классификации 
трансформаций объектов предметной об\-ласти (см.\ рисунок 
в~работе~\cite{23-zac}). Основанием для построения самого верхнего ее уровня послужило деление 
предметной области информатики на среды\footnote{В~работе~\cite{24-zac} дано описание пяти сред 
предметной области информатики (ментальная; сенсорно воспринимаемая, или информационная; 
цифровая; нейро- и~ДНК-среда), каждая из которых по определению включает объекты одной и~той же 
природы.} и~степень разнообразия природы объектов, вовлеченных в~трансформации:
\begin{itemize}
\item  первый класс верхнего уровня классификации включает 
трансформации объектов в~пределах среды только одной природы 
(трансформации первого порядка);
\item  второй класс включает трансформации объектов, относящихся 
к~двум средам разной природы (трансформации второго порядка);
\item третий и~последующие классы включают трансформации объектов, 
относящихся к~трем и~более средам разной природы (трансформации 
третьего и~более высоких порядков).
\end{itemize}

  В работе~\cite{23-zac} были приведены примеры для трех первых классов 
трансформаций, включая пример трансформаций объектов, относящихся 
к~двум средам разной природы (компьютерное кодирование символов текстов 
с~по\-мощью таб\-лиц Unicode).
  
Основанием для построения второго уровня классификации трансформаций объектов послужила типология 
знаковых сис\-тем А.~Соломоника~\cite[c.~131]{25-zac}: естественные знаковые сис\-те\-мы, образные,  
ес\-тест\-вен\-но-язы\-ко\-в$\acute{\mbox{ы}}$е,  
вер\-баль\-но-не\-сло\-вес\-ные сис\-те\-мы записи\footnote{Под системой записи понимается знаковая 
система, сочетающая вербальные знаки с~несловесными (языки нотной записи, карт, таблиц и~др.).} 
и~формализованные знаковые сис\-те\-мы, включая математические. Введем понятие обобщенного текста~--- 
это текст, который может быть создан в~любой из перечисленных знаковых систем. Тогда обобщенные тексты 
могут быть естественными, образными, ес\-тест\-вен\-но-язы\-ко\-в$\acute{\mbox{ы}}$\-ми,  
вер\-баль\-но-не\-сло\-вес\-ны\-ми и~формализованными. Второй уровень классификации трансформаций 
охватывает не все виды объектов предметной  
об\-ласти информатики, а~только перечисленные~5~видов текс\-тов и~их представления, вовлеченные 
в~процессы трансформаций в~одной или более средах вместе с~данными, знанием и~его концептами.

\begin{figure*}[b] %fig1
\vspace*{6pt}
      \begin{center}
     \mbox{%
\epsfxsize=121.191mm 
\epsfbox{zac-1.eps}
}
\end{center}
\vspace*{-6pt}
\Caption{Средовая версия иерархии Акоффа}
\end{figure*}

\section{Классификация трансформаций: построение~третьего 
уровня}

  Основанием для систематизации трансформаций первого и~второго порядка 
на третьем уровне этой классификации служит иерархия Акоффа~\cite{8-zac}, 
на основе которой и~была создана ее средов$\acute{\mbox{а}}$я версия~[26, 
27]. Для создания средов$\acute{\mbox{о}}$й версии была выполнена 
категоризация трех базовых понятий информатики (данные, информация, 
знания) на объекты лексикографической информационной сис\-те\-мы 
в~процессе создания ее концепции\linebreak (рис.~1).
  


  В отличие от классической иерархии Акоффа, в~ее 
средов$\acute{\mbox{о}}$й версии различаются три вида данных: сенсорно 
воспринимаемые, цифровые и~те данные, которые генерируются 
искусственными нейронными сетями (ИНС) в~системах искусственного интеллекта 
(далее~--- ИИ-дан\-ные). Последний вид данных необходим, например, для 
различения входа и~выхода процесса применения обученной 
ИНС в~цифровой модели генерации знания, описанию которой 
посвящена работа~\cite{27-zac}.
  
  Также предлагается различать два вида информации: сенсорно 
воспринимаемая и~цифровая. Кроме знания в~средов$\acute{\mbox{у}}$ю 
версию добавлены концепты и~ментальные образы сенсорно воспринимаемых 
данных. Последние служат промежуточной сущностью между сенсорно 
воспринимаемыми данными и~генерируемым знанием при описании процессов 
извлечения знания из текстовых данных лексикографической информационной 
системы. Описание объектов средов$\acute{\mbox{о}}$й версии иерархии 
Акоффа (см.\ рис.~1) и~отношений между ними дано в~работах~\cite{26-zac, 28-zac}.
  
  В средов$\acute{\mbox{о}}$й версии число объектов равно восьми. Если 
учитывать направления трансформаций, то между восемью объектами на 
рис.~1 она включает~16 их видов (трансформации на границе между сенсорно 
воспринимаемыми данными и~информацией, обозначенные символом~<<?>>, 
в~статье не рас\-смат\-ри\-ва\-ют\-ся). В~будущем число объектов 
в~средов$\acute{\mbox{о}}$й версии, которая выбрана как основание для 
сис\-те\-ма\-ти\-за\-ции трансформаций первого и~второго порядка, может быть 
увеличено. Для построения классификации трансформаций 
важ\-но не возможное увеличение числа объектов 
и~трансформаций между ними, а то, что их виды в~средов$\acute{\mbox{о}}$й 
версии распределены между трансформациями первого и~второго порядка. Из 
16~видов на рис.~1 шесть относятся к~трансформациям первого порядка, это\linebreak 
виды с~номерами~7, 8, 13--16 (далее~--- типология трансформаций первого 
порядка), а~десять~--- к~трансформациям второго порядка, это виды 
с~\mbox{номерами}~1--6 и~9--12 (далее~--- типология трансформаций второго 
порядка). Разместим обе типологии на третьем уровне классификации (см.\ ее 
схему на рис.~2). Перечислим виды трансформаций первой типологии, вводя 
в~скобках их краткие названия, используемые ниже на рис.~3:
  \begin{description}
  \item[\,] 7~--- членение знания на концепты с~помощью одной или нескольких 
знаковых систем (далее~--- членение знания);
  \item[\,] 8~--- формирование знания на основе концептов (формирование 
знания);
  \item[\,] 13~--- обучение ИНС;
  \end{description}
  
  \vspace*{-6pt}
  
  \pagebreak
  
  \end{multicols}
  
  \begin{figure*} %fig2
\vspace*{1pt}
      \begin{center}
     \mbox{%
\epsfxsize=127.513mm 
\epsfbox{zac-2.eps}
}
\end{center}
\vspace*{-9pt}
\Caption{Схема трех верхних уровней классификации трансформаций объектов (объединены 
по три слоя и~для второго, и~для третьего уровней этой классификации)}
\end{figure*}
  
  \begin{multicols}{2}
  
  \noindent
  \begin{description}
  \item[\,] 14~--- восстановление обучающей информации на основе 
содержания обученной ИНС (обращение ИНС);
  \item[\,] 15~--- использование обученной ИНС (использование ИНС);



  \item[\,] 16~--- восстановление исходных данных, соответствующих 
полученным результатам работы обучен\-ной ИНС (восстановление исходных данных 
по результатам ИНС).
  \end{description}
  
  
  Не все виды трансформаций 13--16 поддерживаются в~конкретных системах 
искусственного интеллекта, но с~теоретической точки зрения все их 
предлагается включить в~первую типологию для полноты спектра видов 
трансформаций.
  
  Перечислим виды трансформаций второй типологии:
  \begin{description}
  \item[\,] 1~--- декодирование цифровых данных в~компьютерных системах 
(декодирование данных);
  \item[\,]  2~--- кодирование сенсорно воспринимаемых данных (кодирование 
данных);
  \item[\,] 3~--- ментальное копирование сенсорно воспринимаемых данных 
(ментальное копирование);
  \item[\,] 4~--- восстановление сенсорно воспринимаемых данных по 
ментальным образам (восстановление по образам);
  \item[\,] 5~--- смысловая интерпретация без деления на концепты ментальных 
образов сенсорно воспринимаемых данных (смысловая интерпретация);
  \item[\,] 6~--- восстановление ментальных образов (восстановление образов);
  \item[\,] 9~--- представление концептов в~виде сенсорно воспринимаемой 
информации, например текс\-та\-ми, формулами, таблицами, рисунками и~т.\,д.\ 
(представление концептов);
  \item[\,] 10~--- понимание смысла сенсорно воспринимаемой информации 
(понимание смысла);
  \item[\,] 11~--- кодирование сенсорно воспринимаемой информации 
(кодирование информации);
\end{description}

\vspace*{-6pt}

\pagebreak

\end{multicols}

\begin{figure*} %fig3
\vspace*{1pt}
      \begin{center}
     \mbox{%
\epsfxsize=163mm 
\epsfbox{zac-3.eps}
}
\end{center}
\vspace*{-9pt}
\Caption{Схема частного случая классификации трансформаций объектов (трансформации 
пронумерованы согласно рис.~1)}
\end{figure*}

\begin{multicols}{2}

\noindent
\begin{description}

  \item[\,] 12~--- декодирование цифровой информации (декодирование 
информации).
  \end{description}
  
  Отметим, что в~существующих ИТ
  и~компьютерных системах наиболее часто используются виды 
трансформаций~13 и~15 типологии первого порядка и~1, 2, 11 и~12 типологии 
второго порядка. На рис.~2 в~первом слое третьего уровня классификации 
показаны типологии первого порядка без указания числа трансформаций в~них 
и~без детализации трансформируемых объектов.
  
  Во втором слое третьего уровня классификации условно (без названий) 
показаны типологии второго порядка. Также на рис.~2 в~третьем слое третьего 
уровня классификации условно (также без названий) показаны типологии 
третьего порядка, которые планируется рассмотреть в~отдельной статье. По 
определению они должны включать трансформации между тремя объектами 
разной природы, но средов$\acute{\mbox{а}}$я версия иерархии Акоффа 
включает трансформации только между двумя объектами разной природы. 
Поэтому потребуется другое основание для их систематизации (ранее были 
рассмотрены отдельные примеры трансформаций третьего 
порядка\footnote{Далеко не всегда трансформации третьего и~более высоких порядков можно 
рассматривать как последовательность трансформаций второго порядка. Примером этого могут 
служить трансформации в~процессе обучения пациента пользованию роботизированной рукой, 
охватывающие личностные концепты пациента, релевантные его намерениям, сигналы активности 
мозга как объекты нейросреды и~компьютерные коды~\cite{29-zac}.}~\cite{29-zac}).

\section{Классификация трансформаций: частный~случай}

  Выше было отмечено, что в~будущем число объектов 
в~средов$\acute{\mbox{о}}$й версии иерархии Акоффа может быть увеличено. 
Это означает, что увеличатся и~чис\-ло объектов, и~чис\-ло трансформаций между 
ними в~классификации трансформаций, так как эта средов$\acute{\mbox{а}}$я 
версия служит по определению основанием для систематизации 
трансформаций первого и~второго порядка. Поэтому на третьем уровне рис.~2 
указаны типологии без детализации объектов и~без указания числа 
трансформаций в~каждой из них. С~одной стороны, при таком подходе 
получаем достаточно общий вид этой классификации, так как она не зависит от 
числа объектов в~том или ином варианте средов$\acute{\mbox{о}}$й версии 
(и~это существенно упрощает рис.~2). С~другой стороны, на третьем уровне 
такой общей классификации подразумевается, но не эксплицируется природа 
трансформируемых объектов и~их возможные сочетания в~трансформациях. 

При проектировании лексикографической информационной системы важно 
эксплицировать природу трансформируемых объектов и~их возможные 
сочетания.
  %
  Поэтому в~парадигму информатики~\cite{30-zac} кроме общей 
классификации трансформаций предлагается включать и~ее частные случаи, 
эксплицирующие природу трансформируемых объектов. 

В~этом разделе 
рассмотрим один частный случай, когда используются только естественные 
знаковые сис\-те\-мы из типологии А.~Соломоника~\cite{25-zac} вместе 
с~данными, знанием и~его концептами. Чис\-ло естественных языков при этом не 
ограничено. И~этот частный случай классификации включает только три 
класса природных трансформаций (первого, второго и~третьего порядка, см.\ 
схему классификации на рис.~3).
  
  Первый и~второй уровни схемы общей классификации (см.\ рис.~2) можно 
объединить в~один уровень в~этом частном случае. Ниже этого уровня 
приведено содержание типологий первого и~второго порядка без содержания 
типологий третьего по\-рядка.




  Наполнение типологий первого и~второго порядка соответствует 
средов$\acute{\mbox{о}}$й версии иерархии Акоффа на рис.~1, содержащей 
6~видов трансформаций типологии первого порядка и~10~видов 
трансформаций типологии второго порядка (на рис.~3 стрелки указывают 
направления трансформаций согласно средов$\acute{\mbox{о}}$й версии на рис.~1).
  
  Таким образом, частный случай классификации содержит для этих двух 
типологий 16~теоретически возможных трансформаций, 6 из которых 
в~настоящее время в~существующих ИТ применяются наиболее часто: виды 
трансформаций~1, 2, 11 и~12 типологии второго порядка реализуются 
с~помощью тех или иных методов ко\-ди\-ро\-ва\-ния/де\-ко\-ди\-ро\-ва\-ния 
(например, с~использованием таблиц Unicode), а~виды трансформаций~13 и~15
 в~типологии первого порядка реализуются полностью с~по\-мощью процессов 
цифровой обработки компьютерами.
  
  Остальные виды трансформаций или применяются намного реже (это 
виды~3, 5, 7, 9 и~10), или находятся в~стадии поиска и~разработки (14 и~16) или 
в~настоящее время носят только теоретический характер, обеспечивая полноту 
первой и~второй типологий (4, 6 и~8). Знаком~<<?>> обозначены те виды 
трансформаций, которые по определению не существуют в~используемой 
парадигме информатики~\cite{30-zac}. Однако возможно, что в~других 
будущих подходах к~построению ее парадигмы эти виды трансформаций будут 
существовать.
  
\section{Заключение}

  На сегодняшний день процесс построения классификаций объектов 
предметной области информатики~\cite{22-zac} и~их  
трансформаций~\cite{23-zac} еще не завершен. Однако первые результаты их 
построения уже используются для создания концепции лексикографической 
информационной сис\-те\-мы, обеспечивающей интеграцию двуязычных 
словарей и~параллельных корпусов.
  
  \bigskip
  
  
  Автор признателен рецензентам за помощь в~улучшении статьи.
  
{\small\frenchspacing
 { %\baselineskip=10.6pt
 %\addcontentsline{toc}{section}{References}
 \begin{thebibliography}{99}
\bibitem{1-zac}
\Au{Aijmer K., Altenberg~B.} Advances in corpus-based contrastive linguistics. Studies in honour 
of Stig Johansson.~--- Amsterdam: John Benjamins, 2013. 295~p.  doi: 10.1075/scl.54.
\bibitem{2-zac}
\Au{Добровольский Д.\,О., Кретов~А.\, А., Шаров~С.\,А.} Корпус параллельных текстов~// 
Научная и~техническая информация. Сер.~2: Информационные процессы и~сис\-те\-мы, 2005. 
№\,6. С.~16--27.
\bibitem{3-zac}
\Au{Добровольский Д.\,О.} Корпус параллельных текстов и~сопоставительная 
лексикология~// Труды Института русского языка им.\ В.\,В.~Виноградова, 2015. №\,6. 
С.~413--449. EDN: VJQBHP.
\bibitem{4-zac}
\Au{Гончаров А.\,А., Зацман~И.\,М., Кружков~М.\,Г.} Эволюция классификаций 
в~надкорпусных базах данных~// Информатика и~её применения, 2020. Т.~14. Вып.~4. 
С.~108--116. doi: 10.14357/19922264200415.  
EDN: \mbox{GKWBZT}.
\bibitem{5-zac}
\Au{Гончаров А.\, А., Зацман И. \,М., Кружков~М.\, Г}. Представление новых 
лексикографических знаний в~динамических классификационных сис\-те\-мах~// 
Информатика и~её применения, 2021. Т.~15. Вып.~1. С.~86--93.  doi: 10.14357/19922264210112. EDN: OPEFXW.
\bibitem{6-zac}
\Au{Zatsman I.} Finding and filling lacunas in linguistic typologies~// 15th Forum (International) 
on Knowledge Asset Dynamics Proceedings.~--- Matera, Italy: Institute of Knowledge Asset 
Management, 2020. P.~780--793.
\bibitem{7-zac}
\Au{Zatsman I.} Three-dimensional encoding of emerging meanings in AI-systems~// 21st 
European Conference on Knowledge Management Proceedings.~--- Reading, U.K.: Academic 
Publishing International Ltd., 2020. P.~878--887.
\bibitem{8-zac}
\Au{Ackoff R.} From data to wisdom~// J.~Applied Systems Analysis, 1989. Vol.~16. No.\,1. P.~3--9.
\bibitem{9-zac}
\Au{Rosenbloom P.\,S.} On computing: The fourth great scientific domain.~--- Cambridge, MA, 
USA: MIT Press, 2013. 307~p.
\bibitem{10-zac}
\Au{Rowley J.} The wisdom hierarchy: Representations of the DIKW hierarchy~// J.~Inf. 
Sci., 2007. Vol.~33. Iss.~2. P.~163--180. doi: 10.1177/0165551506070706.
\bibitem{11-zac} 
\Au{Frick$\acute{\mbox{e}}$~M.\,H.} Data--Information--Knowledge--Wisdom (DIKW) pyramid, 
framework, continuum~// Encyclopedia of big data~/ Eds. L.~Schintler, C.~McNeely.~--- Cham: 
Springer, 2018. 4~p. doi: 10.1007/978-3-319-32001-4\_331-1.
\bibitem{12-zac}
\Au{Denning P., Rosenbloom~P.} Computing: The fourth great domain of science~// Commun. 
ACM, 2009. Vol.~52. Iss.~9. P.~27--29.
\bibitem{13-zac}
\Au{Denning P., Freeman~P.} Computing's paradigm~// Commun.  ACM, 2009. Vol.~52. 
Iss.~12. P.~28--30. doi: 10.1145/ 1610252.1610265.
\bibitem{17-zac} %14
\Au{Farradane J.} Knowledge, information, and information science~// J.~Inf. Sci., 
1980. Vol.~2. Iss.~2. P.~75--80. doi: 10.1177/01655515800020020.

\bibitem{15-zac}
\Au{Шрейдер Ю.\,А.} Информация и~знание~// Сис\-тем\-ная концепция информационных 
процессов.~--- М.: ВНИИСИ, 1988. С.~47--52.
\bibitem{16-zac}
\Au{Ingwersen P.} Information and information science~// Enclyclopaedie of library and 
information science~/ Eds. J.\,D.~McDonald, 
M.~Levine-Clark.~--- New York, NY, USA: Marcel Dekker Inc., 1992. Vol.~56. Sup.~19. 
P.~137--174.

\bibitem{14-zac} %17
Информатика как наука об информации: Информационный, документальный, 
технологический, экономический, социальный и~организационный аспекты~/ Под ред. 
Р.\,С.~Гиляревского.~--- М.: Фаир-Пресс, 2006. 592~с.

\bibitem{18-zac}
\Au{Hjorland B.} Library and information science: practice, theory, and philosophical basis~// 
Inform. Process. Manag., 2000. Vol.~36. Iss.~3. P.~501--531. doi:  
10.1016/S0306-\mbox{4573(99)00038-2}.
\bibitem{19-zac}
Deep shift~--- technology tipping points and societal impact.~--- Geneva: WE Forum, 2015. 44~p. 
{\sf http://www3.weforum.org/docs/WEF\_GAC15\_ Technological\_Tipping\_Points\_report\_2015.pdf}.
\bibitem{20-zac}
\Au{Berman F., Rutenbar~R., Hailpern~B., Christensen~H., Davidson~S., Estrin~D., 
Franklin~M., Martonosi~M., Raghavan~P., Stodden~V., Szalay~A.\,S.} Realizing the potential of 
data science~// Commun.  ACM, 2018. Vol.~61. Iss.~4. P.~67--72. doi: 10.1145/3188721.

\bibitem{21-zac}
\Au{Stodden V.} The data science life cycle: A~disciplined approach to advancing data science as 
a~science~// Commun.  ACM, 2020. Vol.~63. Iss.~7. P.~58--66. doi: 10.1145/ 3360646.


\bibitem{23-zac} %22
\Au{Зацман И.\,М.} Научная парадигма информатики: классификация трансформаций 
объектов предметной об\-ласти~// Системы и~средства информатики, 2023. Т.~33. №\,4. 
С.~126--138. doi: 10.14357/08696527230412. EDN: ZIKUWO.

\bibitem{22-zac} %23
\Au{Зацман И.\,М.} Научная парадигма информатики: классификация объектов предметной  
об\-ласти~// Информатика и~её применения, 2023. Т.~17. Вып.~4. С.~96--103. doi: 
10.14357/19922264230413. EDN: FIUQAT.

\bibitem{24-zac}
\Au{Зацман И.\,М.} О~научной парадигме информатики: верхний уровень классификации 
объектов ее предметной об\-ласти~// Информатика и~её применения, 2022. Т.~16. Вып.~4. 
С.~73--79. doi: 10.14357/ 19922264220411. EDN: XZNKVI.

\bibitem{25-zac}
\Au{Соломоник А.\,Б.} Философия знаковых систем и~язык.~--- М.: ЛКИ, 2011. 408~с.
\bibitem{26-zac}
\Au{Зацман И.\,М.} Трансформация иерархии Акоффа в~научной парадигме информатики~// 
Информатика и~её применения, 2023. Т.~17. Вып.~3. С.~107--113. doi: 
10.14357/19922264230315. EDN: UMVRRV.

\bibitem{27-zac}
\Au{Zatsman I.} Building digital spiral models of knowledge generation~// 19th Forum 
(International) on Knowledge Asset Dynamics Proceedings.~--- Matera, Italy: Arts for Business 
Institute, 2024. P.~2185--2196.
\bibitem{28-zac}
\Au{Zatsman I.} Digital spiral model of knowledge creation and encoding its dynamics~// 18th 
Forum (International) on Knowledge Asset Dynamics Proceedings.~--- Matera, Italy: Arts for 
Business Institute, 2023. P.~581--596.
\bibitem{29-zac}
\Au{Зацман И.\,М.} Интерфейсы третьего порядка в~информатике~// Информатика и~её 
применения, 2019. Т.~13. Вып.~3. С.~82--89. doi: 10.14357/19922264190312. EDN: 
EHRQLF.

\bibitem{30-zac}
\Au{Зацман И.\,М.} Научная парадигма информатики как третьей культуры~//  
На\-уч\-но-тех\-ни\-че\-ская информация. Сер.~1: Организация и~методика информационной 
работы, 2023. №\,11. С.~1--14.

\end{thebibliography}

 }
 }

\end{multicols}

\vspace*{-9pt}

\hfill{\small\textit{Поступила в~редакцию 14.04.24}}

\vspace*{4pt}

%\pagebreak

%\newpage

%\vspace*{-28pt}

\hrule

\vspace*{2pt}

\hrule



\def\tit{OBJECT TRANSFORMATIONS OF~THE~FIRST AND~SECOND ORDER
IN~A~LEXICOGRAPHIC INFORMATION SYSTEM\\[-5pt]}


\def\titkol{Object transformations of~the~first and~second order
in~a~lexicographic information system}


\def\aut{I.\,M.~Zatsman}

\def\autkol{I.\,M.~Zatsman}

\titel{\tit}{\aut}{\autkol}{\titkol}

\vspace*{-13pt}


\noindent
Federal Research Center ``Computer Science and Control'' of the Russian Academy of Sciences, 
44-2~Vavilov Str., Moscow 119133, Russian Federation


\def\leftfootline{\small{\textbf{\thepage}
\hfill INFORMATIKA I EE PRIMENENIYA~--- INFORMATICS AND
APPLICATIONS\ \ \ 2024\ \ \ volume~18\ \ \ issue\ 2}
}%
 \def\rightfootline{\small{INFORMATIKA I EE PRIMENENIYA~---
INFORMATICS AND APPLICATIONS\ \ \ 2024\ \ \ volume~18\ \ \ issue\ 2
\hfill \textbf{\thepage}}}

\vspace*{2pt}



\Abste{The theoretical foundations of the design of information technologies used for 
the integration of bilingual dictionaries and parallel corpora are considered. The 
description of the first outcomes of the creation of the third\linebreak\vspace*{-12pt}}

\Abstend{ level of object 
transformations classification in the subject domain of informatics, which is supposed 
to be used
in creating the lexicographic information system providing integration, is 
given. All the entities of informatics are divided into two global classes: objects and 
their transformations. For each such class, its own classification is constructed. 
Previously, the two upper levels of the object transformation classification in the subject 
domain have been described. The present paper discusses the third level of this classification. The 
basis for the construction of its highest level was the division of the subject domain of 
informatics into media (mental, sensory, digital, and a~number of other media), each 
of which by definition includes objects of the same nature. The Solomonick's 
typology of sign systems served as the basis for constructing the second level of the 
object transformation classification. The aim of the paper is to systematize object 
transformations of the first and second orders at the third level of this classification. 
The basis for systematization is the medium version of the Ackoff's hierarchy.}

\KWE{subject domain objects; object transformations; classification; data; 
information; knowledge; lexicographic information system}


\DOI{10.14357/19922264240211}{VZTGVV}

\vspace*{-12pt}

\Ack

\vspace*{-3pt}


\noindent
The reported study was funded by the Russian Science Foundation, project  
No.\,24-18-00155, {\sf 
https://rscf.ru/project/24-18-00155}. The research was carried out using the infrastructure of the Shared 
Research Facilities ``High Performance Computing and Big Data'' (CKP 
``Informatics'') of FRC CSC RAS (Moscow) .
   


  \begin{multicols}{2}

\renewcommand{\bibname}{\protect\rmfamily References}
%\renewcommand{\bibname}{\large\protect\rm References}

{\small\frenchspacing
 {%\baselineskip=10.8pt
 \addcontentsline{toc}{section}{References}
 \begin{thebibliography}{99} 
\bibitem{1-zac-1}
\Aue{Aijmer, K., and B.~Altenberg.} 2013. \textit{Advances in corpus-based 
contrastive linguistics. Studies in honour of Stig Johansson}. Amsterdam: John 
Benjamins. 295~p. doi: 10.1075/scl.54.
\bibitem{2-zac-1}
\Aue{Dobrovolskiy, D.\,O., A.\,A.~Kretov, and S.\,A.~Sharov.} 2005. Korpus 
parallel'nykh tekstov [Corpus of parallel texts]. \textit{Nauchnaya i~tekhnicheskaya 
informatsiya. Ser. 2. Informatsionnye protsessy i~sistemy} [Scientific and Technical 
Information. Ser.~2: Information Processes and Systems] 6:16--27.
\bibitem{3-zac-1}
\Aue{Dobrovolskiy, D.\,O.} 2015. Korpus parallel'nykh tekstov i~sopostavitel'naya 
leksikologiya [The corpus of parallel texts and contrastive lexicology]. \textit{Trudy 
Instituta russkogo yazyka im. V.\,V.~Vinogradova} [Proceedings of the 
V.\,V.~Vinogradov Russian Language Institute] 6:413--449. EDN: VJQBHP.
\bibitem{4-zac-1}
\Aue{Goncharov, A.\,A., I.\,M.~Zatsman, and M.\,G.~Kruzhkov.} 2020. Evolyutsiya 
klassifikatsiy v~nadkorpusnykh ba\-zakh dannykh [Evolution of classifications in 
supracorpora databases]. \textit{Informatika i~ee Primeneniya~--- Inform. \mbox{Appl.}}  
14(4):108--116. doi: 10.14357/19922264200415.  
EDN: GKWBZT.
\bibitem{5-zac-1}
\Aue{Goncharov, A.\,A., I.\,M.~Zatsman, and M.\,G.~Kruzhkov.} 2021. 
Predstavlenie novykh leksikograficheskikh znaniy v~dinamicheskikh 
klassifikatsionnykh sistemakh [Representation of new lexicographical knowledge in 
dynamic classification systems]. \textit{Informatika i~ee Primeneniya~--- Inform. 
Appl.} 15(1):86--93. doi: 10.14357/19922264210112. EDN: OPEFXW.
\bibitem{6-zac-1}
\Aue{Zatsman, I.} 2020. Finding and filling lacunas in linguistic typologies. 
\textit{15th Forum (International) on Knowledge Asset Dynamics Proceedings}. 
Matera, Italy: Institute of Knowledge Asset Management. 780--793.
\bibitem{7-zac-1}
\Aue{Zatsman, I.} 2020. Three-dimensional encoding of emerging meanings in  
AI-systems. \textit{21st European Conference on Knowledge Management 
Proceedings}. Reading, U.K.: Academic Publishing International Ltd. 878--887.
\bibitem{8-zac-1}
\Aue{Ackoff, R.} 1989. From data to wisdom. \textit{J.~Applied Systems Analysis} 
16(1):3--9.
\bibitem{9-zac-1}
\Aue{Rosenbloom, P.\,S.} 2013. \textit{On computing: The fourth great scientific 
domain}. Cambridge, MA: MIT Press. 307~p.
\bibitem{10-zac-1}
\Aue{Rowley, J.} 2007. The wisdom hierarchy: Representations of the DIKW 
hierarchy. \textit{J.~Inf. Sci.} 33(2):163--180. doi: 10.1177/0165551506070706.
\bibitem{11-zac-1}
\Aue{Frick$\acute{\mbox{e}}$, M.\,H.} 2018.  
Data-Information-Knowledge-Wisdom (DIKW) pyramid, framework, continuum. 
\textit{Encyclopedia of big data}. Eds. L.~Schintler and C.~McNeely. Cham: 
Springer. 4~p. doi: 10.1007/978-3-319-32001- 4\_331-1.
\bibitem{12-zac-1}
\Aue{Denning, P., and P.~Rosenbloom.} 2009. Computing: The fourth great domain 
of science. \textit{Commun. ACM} 52(9):27--29.
\bibitem{13-zac-1}
\Aue{Denning, P., and P.~Freeman.} 2009. Computing's paradigm. \textit{Commun. 
ACM} 52(12):28--30. doi: 10.1145/ 1610252.1610265.

\bibitem{17-zac-1} %14
\Aue{Farradane, J.} 1980. Knowledge, information, and information science. 
\textit{J.~Inf. Sci.} 2(2):75--80. doi: 10.1177/ 01655515800020020.

\bibitem{15-zac-1}
\Aue{Shreyder, Yu.\,A.} 1988. Informatsiya i~znanie [Information and knowledge]. 
\textit{Sistemnaya kontseptsiya in\-for\-ma\-tsi\-on\-nykh protsessov} [System concept of 
information processes]. Moscow: VNIISI. 47--52.
\bibitem{16-zac-1}
\Aue{Ingwersen, P.} 1995. Information and information science. 
\textit{Encyclopedia of library and information science}. Eds. J.\,D.~McDonald and 
M.~Levine-Clark. New York, NY: Marcel Dekker Inc. 56(19):137--174.

\bibitem{14-zac-1} %17
Gilyarevskiy, R.\,S., ed. 2006. \textit{Informatika kak nauka ob informatsii: 
informatsionnyy, dokumental'nyy, tekh\-no\-lo\-gi\-che\-skiy, ekonomicheskiy, sotsial'nyy 
i~organizatsionnyy aspekty} [Informatics as information science: Informational, 
documentary, technological, economic, social, and organizational dimensions]. 
Moscow: FAIR-PRESS. 592~p.

\bibitem{18-zac-1}
\Aue{Hjorland, B.} 2000. Library and information science: Practice, theory, and 
philosophical basis. \textit{Inform. Process. Manag.} 36(3):501--531. doi:  
10.1016/S0306-\mbox{4573(99)00038-2}.
\bibitem{19-zac-1}
Deep shift~--- technology tipping points and societal impact. 2015. \textit{World Economic 
Forum}. Geneva. 44~p. Available at: {\sf 
http://www3.weforum.org/docs/WEF\_ GAC15\_Technological\_Tipping\_Points\_report\_2015.pdf} (accessed May~20, 
2024).
\bibitem{20-zac-1}
\Aue{Berman, F., R.~Rutenbar, B.~Hailpern, H.~Christensen, S.~Davidson, 
D.~Estrin, M.~Franklin, M.~Martonosi, P.~Raghavan, V.~Stodden, and 
A.\,S.~Szalay.} 2018. Realizing the potential of data science. \textit{Commun. ACM} 
61(4):67--72. doi: 10.1145/3188721.
\bibitem{21-zac-1}
\Aue{Stodden, V.} 2020. The data science life cycle: A~disciplined approach to 
advancing data science as a~science. \textit{Commun. ACM} 
 63(7):58--66. doi: 10.1145/3360646.

\bibitem{23-zac-1} %22
\Aue{Zatsman, I.\,M.} 2023. Nauchnaya paradigma informatiki: klassifikatsiya 
transformatsiy ob''ektov predmetnoy oblasti [Scientific paradigm of informatics: 
Transformation classification of domain objects]. \textit{Sistemy i~Sredstva 
Informatiki~--- Systems and Means of Informatics} 33(4):126--138. doi: 
10.14357/08696527230412. EDN: ZIKUWO.

\bibitem{22-zac-1} %23
\Aue{Zatsman, I.\,M.} 2023. Nauchnaya paradigma informatiki: klassifikatsiya 
ob''ektov predmetnoy oblasti [Scientific paradigm of informatics: Classification of 
domain objects]. \textit{Informatika i~ee Primeneniya~--- Inform. Appl.} 
 17(4):96--103. doi: 10.14357/19922264230413. EDN: FIUQAT.
 
\bibitem{24-zac-1}
\Aue{   Zatsman, I.\,M.} 2022. O nauchnoy paradigme informatiki: verkhniy uroven' 
klassifikatsii ob''ektov ee predmetnoy oblasti [On the scientific paradigm of 
informatics: The classification high level of its objects]. \textit{Informatika i~ee 
Primeneniya~--- Inform. Appl.} 16(4):73--79. doi: 10.14357/19922264220411. EDN: 
XZNKVI.
\bibitem{25-zac-1}
\Aue{Solomonick, A.\,B.} 2011. \textit{Filosofiya znakovykh system i~yazyk} 
[Philosophy of sign systems and language]. Moscow: LKI. 408~p.
\bibitem{26-zac-1}
\Aue{Zatsman, I.\,M.} 2023. Transformatsiya ierarkhii Akoffa v~nauchnoy 
paradigme informatiki [Transformation of the Ackoff's hierarchy in the scientific 
paradigm of informatics]. \textit{Informatika i~ee Primeneniya~--- Inform. \mbox{Appl.}} 
17(3):107--113. doi: 10.14357/19922264230315. EDN: UMVRRV.
\bibitem{27-zac-1}
\Aue{Zatsman, I.} 2024. Building digital spiral models of knowledge 
generation. \textit{19th Forum (International) on Knowledge Asset Dynamics 
Proceedings}. Matera, Italy: Arts for Business Institute. 2185--2196.
\bibitem{28-zac-1}
\Aue{Zatsman, I.} 2023. Digital spiral model of knowledge creation and encoding its 
dynamics. \textit{18th Forum (International) on Knowledge Asset Dynamics 
Proceedings}. Matera, Italy: Arts for Business Institute. 581--596.
\bibitem{29-zac-1}
\Aue{Zatsman, I.\,M.} 2019. Interfeysy tret'ego poryadka v~informatike 
 [Third-order interfaces in informatics]. \textit{Informatika i~ee Primeneniya~--- 
Inform. Appl.} 13(3):82--89. doi: 10.14357/19922264190312. EDN: EHRQLF.
\bibitem{30-zac-1}
\Aue{Zatsman, I.} 2023. Scientific paradigm of informatics as a~third culture. 
\textit{Scientific Technical Information Processing} 50(4):246--258. doi: 
10.3103/S0147688223040111. EDN: CKHMYS.

\end{thebibliography}

 }
 }

\end{multicols}

\vspace*{-6pt}

\hfill{\small\textit{Received April 14, 2024}} 


\vspace*{-12pt}


\Contrl

\vspace*{-3pt}

\noindent
\textbf{Zatsman Igor M.} (b.\ 1952)~--- Doctor of Science in technology, head of 
department, Federal Research Center ``Computer Science and Control'' of the 
Russian Academy of Sciences, 44-2~Vavilov Str., Moscow 119333, Russian 
Federation; \mbox{izatsman@yandex.ru}





\label{end\stat}

\renewcommand{\bibname}{\protect\rm Литература}  %13



\def\stat{authorsrus}
{%\hrule\par
%\vskip 7pt % 7pt
\raggedleft\Large \bf%\baselineskip=3.2ex
О\,Б\ \ А\,В\,Т\,О\,Р\,А\,Х \vskip 17pt
    \hrule
    \par
\vskip 21pt plus 8pt minus 4pt }


\def\tit{\ }

\def\aut{\ }

\def\auf{\ }

\def\leftkol{\ } % ENGLISH ABSTRACTS}

\def\rightkol{ОБ АВТОРАХ} %ENGLISH ABSTRACTS}

\titele{\tit}{\aut}{\auf}{\leftkol}{\rightkol}
      
            \label{st\stat}



\vspace*{24pt}

\begin{multicols}{2}




\noindent
\textbf{Архипов Олег Петрович} (р.\ 1948)~---
кандидат технических наук, директор Орловского филиала Института проб\-лем информатики
Российской академии наук
%302025, г.Орел, Московское шоссе, д.137

\vspace*{3pt}

\noindent
\textbf{Бирюкова Татьяна Константиновна} (р.\ 1968)~---
кандидат фи\-зи\-ко-ма\-те\-ма\-ти\-че\-ских наук, старший научный сотрудник Института проб\-лем информатики
Российской академии наук

\vspace*{3pt}

\noindent 
\textbf{Бобков  Сергей Геннадьевич} (р.\ 1955)~---
доктор технических наук,  заведующий отделением На\-уч\-но-ис\-сле\-до\-ва\-тель\-ско\-го 
института системных исследований Российской академии наук
%117218, Москва, Нахимовский просп., 36, к.1 

\vspace*{3pt}

\noindent \textbf{Васильев Николай Семенович} (р.\ 1952)~--- доктор 
фи\-зи\-ко-ма\-те\-ма\-ти\-че\-ских наук, профессор, 
МГТУ им.\ Н.\,Э.~Баумана 
%, Москва 105005, 2-я Бауманская ул., д.~5,

\vspace*{3pt}

\noindent
\textbf{Гершкович Максим Михайлович} (р.\ 1968)~---
старший научный сотрудник Института проб\-лем информатики
Российской академии наук

\vspace*{3pt}

\noindent 
\textbf{Дьяченко Юрий Георгиевич} (р.\ 1958)~--- кандидат технических наук, 
старший научный сотрудник Института проб\-лем информатики
Российской академии наук

\vspace*{3pt}

\noindent 
\textbf{Ерошенко Александр Андреевич} (р.\ 1989)~--- аспирант кафедры 
математической статистики факультета вычисли\-тельной математики и кибернетики 
Московского государственного университета им.\ М.\,В.~Ломоносова
%119991, Москва ГСП-1, Ленинские горы, д.\ 1, стр. 52

\vspace*{3pt}
 
\noindent 
\textbf{Захаров Виктор Николаевич} (р.\ 1948)~--- 
доктор технических наук, доцент, ученый секретарь Института проб\-лем информатики
Российской академии наук

\vspace*{3pt}

\noindent
\textbf{Зейфман Александр Израилевич} (р.\ 1954)~---
доктор фи\-зи\-ко-ма\-те\-ма\-ти\-че\-ских наук, профессор, 
заведующий кафедрой Вологодского государственного университета; 
старший научный сотрудник Института проб\-лем информатики
Российской академии наук; главный научный сотрудник ИСЭРТ Российской академии наук

\vspace*{3pt}

\noindent
\textbf{Зыкин Сергей Владимирович} (р.\ 1959)~--- 
доктор технических наук, профессор, заведующий лабораторией Института математики 
им.\ С.\,Л.~Соболева Сибирского отделения Российской академии наук, Новосибирск 
%630090, пр.\ ак.\ Коптюга, 4 

\vspace*{4pt}

\noindent
\textbf{Киреев Владимир Иванович} (р.\ 1938)~---
доктор фи\-зи\-ко-ма\-те\-ма\-ти\-че\-ских наук, профессор Московского 
государственного горного университета
%Адрес: Россия, 119991, г. Москва, Ленинский проспект, д. 6

%\columnbreak

\vspace*{4pt}

\noindent
\textbf{Козеренко Елена Борисовна} (р.\ 1959)~---
кандидат филологических наук, заведующая лабораторией Института проб\-лем информатики
Российской академии наук

\vspace*{4pt}

\noindent
\textbf{Королев Виктор Юрьевич} (р.\ 1954)~--- доктор
фи\-зи\-ко-ма\-те\-ма\-ти\-че\-ских наук, профессор кафедры математической 
статистики факультета вычисли\-тельной математики и кибернетики 
Московского государственного университета; 
ведущий научный сотрудник Института проб\-лем информатики
Российской академии наук

\vspace*{4pt}

\noindent
\textbf{Коротышева Анна Владимировна} (р.\ 1988)~---
старший преподаватель Вологодского государственного университета

\vspace*{4pt}

\noindent 
\textbf{Кун Де Турк} (р.\ 1981)~--- научный сотрудник 
исследовательской группы SMACS факультета телекоммуникаций и обработки информации
Университета Гента, Бельгия
%В-9000 Гент, Бельгия

\vspace*{4pt}

\noindent
\textbf{Лупенцов Олег Сергеевич} (р.\ 1986)~---
аспирант Омского государственного института сервиса
%Омск 644043, ул.\ Певцова 13

\vspace*{4pt}

\noindent
\textbf{Лучко Олег Николаевич} (р.\ 1961)~---
кандидат педагогических наук, профессор, заведующий кафедрой 
Омского государственного института сервиса
%Омск 644043, ул.\ Певцова 13

\vspace*{4pt}

\noindent
\textbf{Малашенко Юрий Евгеньевич} (р.\ 1946)~---
доктор фи\-зи\-ко-ма\-те\-ма\-ти\-че\-ских наук, заведующий сектором 
Вычислительного центра им.\ А.\,А.~Дородницына Российской академии наук
%Адрес: 119333, Москва, ул. Вавилова, 40,

\vspace*{4pt}

\noindent
\textbf{Маньяков Юрий Анатольевич} (р.\ 1984)~---
кандидат технических наук, научный сотрудник Орловского филиала Института проб\-лем информатики
Российской академии наук
%302025, г.Орел, Московское шоссе, д.137

\vspace*{4pt}

\noindent
\textbf{Маренко Валентина Афанасьевна} (р.\ 1951)~---
кандидат технических наук, доцент, старший научный сотрудник 
Института математики им.\ С.\,Л.~Соболева Сибирского отделения Российской академии наук
%Новосибирск 630090, пр. ак. Коптюга, 4 

\vspace*{3pt}

\noindent 
\textbf{Морозов Евсей Викторович} (р.\ 1947)~--- доктор 
фи\-зи\-ко-ма\-те\-ма\-ти\-че\-ских, профессор, ведущий научный сотрудник 
Института прикладных математических исследований Карельского научного центра Российской
академии наук; 
%%185910 Россия, Республика Карелия, г.\ Петрозаводск, ул.\ Пушкинская, 11
профессор Петрозаводского государственного университета, Петрозаводск
%185910 Россия, Республика Карелия, г.\ Петрозаводск, пр.\ Ленина, 33

%\pagebreak

\vspace*{3pt}

\noindent
\textbf{Назарова Ирина Александровна} (р.\ 1966)~---
кандидат фи\-зи\-ко-ма\-те\-ма\-ти\-че\-ских наук, 
научный сотрудник Вычислительного центра им.\ А.\,А.~Дородницына Российской академии наук 
%Адрес: 119333, Москва, ул. Вавилова, 40

\vspace*{3pt}

\noindent
\textbf{Павлов Игорь Валерианович} (р.\ 1945)~--- 
доктор фи\-зи\-ко-ма\-те\-ма\-ти\-че\-ских наук, профессор МГТУ им.\ Н.\,Э.~Баумана 
%Москва 105005, 2-я Бауманская ул., д.~5 

%\pagebreak

\vspace*{3pt}

\noindent 
\textbf{Потахина Любовь Викторовна} (р.\ 1989)~--- аспирантка
Института прикладных математических исследований Карельского научного центра
Российской академии наук; 
%%185910 Россия, Республика Карелия, г.\ Петрозаводск, ул.\ Пушкинская, 11
инженер Петрозаводского государственного университета, Петрозаводск
%185910 Россия, Республика Карелия, г.\ Петрозаводск, пр.\ Ленина, 33

\vspace*{3pt}

\noindent 
\textbf{Рождественский Юрий Владимирович} (р.\ 1952)~--- 
кандидат технических наук, заведующий сектором Института проб\-лем информатики
Российской академии наук

\vspace*{3pt}

\noindent 
\textbf{Синицын Игорь Николаевич} (р.\ 1940)~--- доктор технических наук,
профессор, заслуженный деятель\linebreak\vspace*{-12pt}

\columnbreak

\noindent
 науки РФ, заведующий отделом Института проб\-лем информатики
Российской академии наук

\vspace*{7pt}


\noindent
\textbf{Сиротинин Денис Олегович} (р.\ 1984)~---
кандидат технических наук, научный сотрудник Орловского филиала Института проб\-лем информатики
Российской академии наук
%302025, г.Орел, Московское шоссе, д.137

\vspace*{7pt}

%\columnbreak

\noindent 
\textbf{Соколов  Игорь Анатольевич} (р.\ 1954)~--- академик (действительный член) Российской 
академии наук, доктор технических наук, директор Института проб\-лем информатики
Российской академии наук

\vspace*{7pt}

\noindent
\textbf{Степченков Юрий Афанасьевич} (р.\ 1951)~---
кандидат технических наук, заведующий отделом Института проб\-лем информатики
Российской академии наук

\vspace*{7pt}

\noindent
\textbf{Сурков Алексей Викторович} (р.\ 1978)~--- 
старший научный сотрудник На\-уч\-но-ис\-сле\-до\-ва\-тель\-ско\-го 
института системных исследований Российской академии наук
%117218, Москва, Нахимовский просп., 36, к.1 

\vspace*{7pt}

\noindent 
\textbf{Шестаков Олег Владимирович} (р.\ 1976)~--- доктор 
фи\-зи\-ко-ма\-те\-ма\-ти\-че\-ских, доцент кафедры математической статистики 
факультета вычисли\-тельной математики и кибернетики Московского 
государственного университета им.\ М.\,В.~Ломоносова; 
%119991, Москва ГСП-1, Ленинские горы, д.\ 1, стр. 52
старший научный сотрудник Института проб\-лем информатики
Российской академии наук
%, Москва 119333, ул. Вавилова, д.~44, корп.~2

\vspace*{7pt}

\noindent 
\textbf{Шоргин Сергей Яковлевич} (р.\ 1952.)~--- доктор
фи\-зи\-ко-ма\-те\-ма\-ти\-че\-ских наук, профессор, заместитель директора Института 
проб\-лем информатики Российской академии наук





%%%%%%%%%%%%%%%%%%%%%%%%%%%%%%%%%%%%%%%%%%%%%%%%%%%%%%%%%%%%%%%%%%%%%%%%%%%%%%%




%\def\rightkol{ОБ АВТОРАХ}
%\def\leftkol{ОБ АВТОРАХ}

 \label{end\stat}





%\def\leftfootline{\small{\textbf{\thepage}
%\hfill ИНФОРМАТИКА И ЕЁ ПРИМЕНЕНИЯ\ \ \ том~7\ \ \ выпуск~1\ \ \ 2013}
%}%
% \def\rightfootline{\small{ИНФОРМАТИКА И ЕЁ ПРИМЕНЕНИЯ\ \ \ том~7\ \ \ выпуск~1\ \ \ 2013
%\hfill \textbf{\thepage}}}


%\thispagestyle{myheadings}



\end{multicols}

\newpage  

%\def\stat{cont}
{%\hrule\par
%\vskip 7pt % 7pt
\raggedleft\Large \bf%\baselineskip=3.2ex
А\,В\,Т\,О\,Р\,С\,К\,И\,Й\ \ У\,К\,А\,З\,А\,Т\,Е\,Л\,Ь\ \ З\,А\ \ 2\,0\,0\,7 г. \vskip 17pt
    \hrule
    \par
\vskip 21pt plus 6pt minus 3pt }

\label{st\stat}

\def\tit{\ }

\def\aut{\ }
\def\auf{\ }

\def\leftkol{\ } % ENGLISH ABSTRACTS}

\def\rightkol{\ } %ENGLISH ABSTRACTS}

\titele{\tit}{\aut}{\auf}{\leftkol}{\rightkol}


\contentsline {chapter}{\ }{Выпуск \quad Стр.} 
\contentsline {section}{\textbf{Батракова Д.\,А., Королев В.\,Ю., Шоргин С.\,Я.}\ \ Новый метод вероятностно-ста\-ти\-сти\-че\-ско\-го анализа информационных потоков в\nobreakspace {}телекоммуникационных сетях}{\qquad 1 \qquad 40} 
\contentsline {section}{\textbf{Борисов А.\,В.}\ \ Байесовское оценивание в системах наблюдения с\nobreakspace {}марковскими скачкообразными процессами: игровой подход}{\qquad 2 \qquad 65}
\contentsline {section}{\textbf{Босов А.\,В., Иванов А.\,В.}\ \ Программная инфраструктура информационного Web-пор\-тала}{\qquad 2 \qquad 50}
\contentsline {section}{\textbf{Захаров В.\,Н., Калиниченко Л.\,А., Соколов И.\,А., Ступников С.\,А.}\ \ Конструирование канонических информационных моделей для интегрированных информационных систем}{\qquad 2 \qquad 15}
\contentsline {section}{\textbf{Захаров В.\,Н., Козмидиади В.\,А.}\ \ Средства обеспечения отказоустойчивости при\-ло\-жений}{\qquad 1 \qquad 14} 
\contentsline {section}{\textbf{Иванов А.\,В.}\ \ см. Босов А.\,В.\hfill\hfill\hfill\hfill\hfill\hfill\hfill\hfill\hfill\hfill\hfill\hfill\hfill\hfill\hfill\hfill\hfill\hfill\hfill\hfill\hfill\hfill\hfill\hfill\hfill\hfill\hfill\hfill\hfill\hfill\hfill\hfill\hfill\hfill\hfill}{\ }
\contentsline {section}{\textbf{Ильин В.\,Д., Соколов И.\,А.}\ \ Символьная модель системы знаний информатики в\nobreakspace {}че\-ло\-ве\-ко-автоматной среде}{\qquad 1 \qquad 66} 
\contentsline {section}{\textbf{Калиниченко Л.\,А.}\ \ см. Захаров В.\,Н.\hfill\hfill\hfill\hfill\hfill\hfill\hfill\hfill\hfill\hfill\hfill\hfill\hfill\hfill\hfill\hfill\hfill\hfill\hfill\hfill\hfill\hfill\hfill\hfill\hfill\hfill\hfill\hfill\hfill\hfill\hfill\hfill\hfill\hfill\hfill}{\ }
\contentsline {section}{\textbf{Козеренко Е.\,Б.}\ \ Лингвистическое моделирование для систем машинного перевода и обработки знаний}{\qquad 1 \qquad 54} 
\contentsline {section}{\textbf{Козмидиади В.\,А.}\ \ см. Захаров В.\,Н.\hfill\hfill\hfill\hfill\hfill\hfill\hfill\hfill\hfill\hfill\hfill\hfill\hfill\hfill\hfill\hfill\hfill\hfill\hfill\hfill\hfill\hfill\hfill\hfill\hfill\hfill\hfill\hfill\hfill\hfill\hfill\hfill\hfill\hfill\hfill }{\ } 
\contentsline {section}{\textbf{Королев В.\,Ю.}\ \ см. Батракова Д.\,А.\hfill\hfill\hfill\hfill\hfill\hfill\hfill\hfill\hfill\hfill\hfill\hfill\hfill\hfill\hfill\hfill\hfill\hfill\hfill\hfill\hfill\hfill\hfill\hfill\hfill\hfill\hfill\hfill\hfill\hfill\hfill\hfill\hfill\hfill\hfill}{\ } 
\contentsline {section}{\textbf{Кудрявцев А.\,А., Шоргин С.\,Я.}\ \ Байесовский подход к\nobreakspace {}анализу систем массового обслуживания и\nobreakspace {}показателей надежности}{\qquad 2 \qquad 76}
\contentsline {section}{\textbf{Печинкин А.\,В., Соколов И.\,А., Чаплыгин В.\,В.}\ \ Многолинейная система массового обслуживания с конечным накопителем и ненадежными приборами}{\qquad 1 \qquad 27} 
\contentsline {section}{\textbf{Печинкин А.\,В., Соколов И.\,А., Чаплыгин В.\,В.}\ \ Стационарные характеристики многолинейной\nobreakspace {}системы массового обслуживания с\nobreakspace {}одновременными отказами приборов}{\qquad 2 \qquad 39}
\contentsline {section}{\textbf{Синицын И.\,Н.}\ \ Корреляционные методы построения аналитических информационных моделей флуктуаций полюса Земли по априорным данным}{\qquad 2 \qquad \hphantom{9}2}
\contentsline {section}{\textbf{Синицын И.\,Н.}\ \ Развитие теории фильтров Пугачева для оперативной обработки информации в стохастических системах}{{\qquad 1 \qquad \hphantom{9}3}} 
\contentsline {section}{\textbf{Соколов И.\,А.}\ \ см. Захаров В.\,Н.\hfill\hfill\hfill\hfill\hfill\hfill\hfill\hfill\hfill\hfill\hfill\hfill\hfill\hfill\hfill\hfill\hfill\hfill\hfill\hfill\hfill\hfill\hfill\hfill\hfill\hfill\hfill\hfill\hfill\hfill\hfill\hfill\hfill\hfill\hfill}{\ }
\contentsline {section}{\textbf{Соколов И.\,А.}\ \ см. Ильин В.\,Д.\hfill\hfill\hfill\hfill\hfill\hfill\hfill\hfill\hfill\hfill\hfill\hfill\hfill\hfill\hfill\hfill\hfill\hfill\hfill\hfill\hfill\hfill\hfill\hfill\hfill\hfill\hfill\hfill\hfill\hfill\hfill\hfill\hfill\hfill\hfill}{\ } 
\contentsline {section}{\textbf{Соколов И.\,А.}\ \ см. Печинкин А.\,В.\hfill\hfill\hfill\hfill\hfill\hfill\hfill\hfill\hfill\hfill\hfill\hfill\hfill\hfill\hfill\hfill\hfill\hfill\hfill\hfill\hfill\hfill\hfill\hfill\hfill\hfill\hfill\hfill\hfill\hfill\hfill\hfill\hfill\hfill\hfill}{\ } 
\contentsline {section}{\textbf{Соколов И.\,А.}\ \ см. Печинкин А.\,В.\hfill\hfill\hfill\hfill\hfill\hfill\hfill\hfill\hfill\hfill\hfill\hfill\hfill\hfill\hfill\hfill\hfill\hfill\hfill\hfill\hfill\hfill\hfill\hfill\hfill\hfill\hfill\hfill\hfill\hfill\hfill\hfill\hfill\hfill\hfill}{\ }
\contentsline {section}{\textbf{Ступников С.\,А.}\ \ см. Захаров В.\,Н.\hfill\hfill\hfill\hfill\hfill\hfill\hfill\hfill\hfill\hfill\hfill\hfill\hfill\hfill\hfill\hfill\hfill\hfill\hfill\hfill\hfill\hfill\hfill\hfill\hfill\hfill\hfill\hfill\hfill\hfill\hfill\hfill\hfill\hfill\hfill}{\ }
\contentsline {section}{\textbf{Чаплыгин В.\,В.}\ \ см. Печинкин А.\,В.\hfill\hfill\hfill\hfill\hfill\hfill\hfill\hfill\hfill\hfill\hfill\hfill\hfill\hfill\hfill\hfill\hfill\hfill\hfill\hfill\hfill\hfill\hfill\hfill\hfill\hfill\hfill\hfill\hfill\hfill\hfill\hfill\hfill\hfill\hfill}{\ } 
\contentsline {section}{\textbf{Чаплыгин В.\,В.}\ \ см. Печинкин А.\,В.\hfill\hfill\hfill\hfill\hfill\hfill\hfill\hfill\hfill\hfill\hfill\hfill\hfill\hfill\hfill\hfill\hfill\hfill\hfill\hfill\hfill\hfill\hfill\hfill\hfill\hfill\hfill\hfill\hfill\hfill\hfill\hfill\hfill\hfill\hfill}{\ }
\contentsline {section}{\textbf{Шоргин С.\,Я.}\ \ см. Батракова Д.\,А.\hfill\hfill\hfill\hfill\hfill\hfill\hfill\hfill\hfill\hfill\hfill\hfill\hfill\hfill\hfill\hfill\hfill\hfill\hfill\hfill\hfill\hfill\hfill\hfill\hfill\hfill\hfill\hfill\hfill\hfill\hfill\hfill\hfill\hfill\hfill}{\ } 
\contentsline {section}{\textbf{Шоргин С.\,Я.}\ \ см. Кудрявцев А.\,А.\hfill\hfill\hfill\hfill\hfill\hfill\hfill\hfill\hfill\hfill\hfill\hfill\hfill\hfill\hfill\hfill\hfill\hfill\hfill\hfill\hfill\hfill\hfill\hfill\hfill\hfill\hfill\hfill\hfill\hfill\hfill\hfill\hfill\hfill\hfill}{\ }
%\thispagestyle{myheadings}
\def\leftfootline{\small{\textbf{\thepage}
\hfill ИНФОРМАТИКА И ЕЁ ПРИМЕНЕНИЯ\ \ \ том~1\ \ \ выпуск~2\ \ \ 2007}
}%
 \def\rightfootline{\small{ИНФОРМАТИКА И ЕЁ ПРИМЕНЕНИЯ\ \ \ том~1\ \ \ выпуск~2\ \ \ 2007
 \hfill \textbf{\thepage}}}
 \label{end\stat} 
                     
%\def\stat{cont-e}
{%\hrule\par
%\vskip 7pt % 7pt
\raggedleft\Large \bf%\baselineskip=3.2ex
2\,0\,0\,7\ \ A\,U\,T\,H\,O\,R\ \ I\,N\,D\,E\,X \vskip 17pt
    \hrule
    \par
\vskip 21pt plus 6pt minus 3pt }

\label{st\stat}

\def\tit{\ }

\def\aut{\ }
\def\auf{\ }

\def\leftkol{\ } % ENGLISH ABSTRACTS}

\def\rightkol{\ } %ENGLISH ABSTRACTS}

\titele{\tit}{\aut}{\auf}{\leftkol}{\rightkol}


\contentsline {chapter}{\ }{Issue \quad Page} 
\contentsline {subsection}{\textbf{Batrakova D.\,A., Korolev V.\,Yu., Shorgin S.\,Ya.}\ \ A New Method for the Probabilistic and Statistical Analysis of Information Flows in Telecommunication Networks}{\qquad 1 \qquad 40} 
\contentsline {subsection}{\textbf{Borisov A.\,V.}\ \ Bayesian Estimation in\nobreakspace {}Observation Systems with\nobreakspace {}Markov Jump Processes: Game-Theoretic Approach}{\qquad 2 \qquad 65} 
\contentsline {subsection}{\textbf{Bosov A.\,V., Ivanov A.\,V.}\ \ Linguistic Simulation for Machine Translation and Knowledge Management Systems}{\qquad 2 \qquad 50} 
\contentsline {subsection}{\textbf{Chaplygin V.\,V.} see Pechinkin A.\,V.\hfill\hfill\hfill\hfill\hfill\hfill\hfill\hfill\hfill\hfill\hfill\hfill\hfill\hfill\hfill\hfill\hfill\hfill\hfill\hfill\hfill\hfill\hfill\hfill\hfill\hfill\hfill\hfill\hfill\hfill\hfill\hfill\hfill\hfill\hfill}{\ }
\contentsline {subsection}{\textbf{Chaplygin V.\,V.} see Pechinkin A.\,V.\hfill\hfill\hfill\hfill\hfill\hfill\hfill\hfill\hfill\hfill\hfill\hfill\hfill\hfill\hfill\hfill\hfill\hfill\hfill\hfill\hfill\hfill\hfill\hfill\hfill\hfill\hfill\hfill\hfill\hfill\hfill\hfill\hfill\hfill\hfill}{\ }
\contentsline {subsection}{\textbf{Ilyin V.\,D., Sokolov I.\,A.}\ \ The Symbol Model of Informatics Knowledge System in Human-Automaton Environment}{\qquad 1 \qquad 66} 
\contentsline {subsection}{\textbf{Ivanov A.\,V.} see Bosov A.\,V.\hfill\hfill\hfill\hfill\hfill\hfill\hfill\hfill\hfill\hfill\hfill\hfill\hfill\hfill\hfill\hfill\hfill\hfill\hfill\hfill\hfill\hfill\hfill\hfill\hfill\hfill\hfill\hfill\hfill\hfill\hfill\hfill\hfill\hfill\hfill}{\ }
\contentsline {subsection}{\textbf{Kalinichenko L.\,A.} see Zakharov V.\,N.\hfill\hfill\hfill\hfill\hfill\hfill\hfill\hfill\hfill\hfill\hfill\hfill\hfill\hfill\hfill\hfill\hfill\hfill\hfill\hfill\hfill\hfill\hfill\hfill\hfill\hfill\hfill\hfill\hfill\hfill\hfill\hfill\hfill\hfill\hfill}{\ }
\contentsline {subsection}{\textbf{Korolev V.\,Yu.} see Batrakova D.\,A.\hfill\hfill\hfill\hfill\hfill\hfill\hfill\hfill\hfill\hfill\hfill\hfill\hfill\hfill\hfill\hfill\hfill\hfill\hfill\hfill\hfill\hfill\hfill\hfill\hfill\hfill\hfill\hfill\hfill\hfill\hfill\hfill\hfill\hfill\hfill}{\ }
\contentsline {subsection}{\textbf{Kozerenko E.\,B.}\ \ Linguistic Simulation for Machine Translation and Knowledge Management Systems}{\qquad 1 \qquad 54} 
\contentsline {subsection}{\textbf{Kozmidiady V.\,A.} see Zakharov V.\,N.\hfill\hfill\hfill\hfill\hfill\hfill\hfill\hfill\hfill\hfill\hfill\hfill\hfill\hfill\hfill\hfill\hfill\hfill\hfill\hfill\hfill\hfill\hfill\hfill\hfill\hfill\hfill\hfill\hfill\hfill\hfill\hfill\hfill\hfill\hfill}{\ }
\contentsline {subsection}{\textbf{Kudryavtsev A.\,A., Shorgin S.\,Ya.}\ \ Bayesian Approach to Queueing Systems and Reliability Characteristics}{\qquad 2 \qquad 76} 
\contentsline {subsection}{\textbf{Pechinkin A.\,V., Sokolov I.\,A., Chaplygin V.\,V.}\ \ Multichannel Queuing System with Finite Buffer and Unreliable Servers}{\qquad 1 \qquad 27} 
\contentsline {subsection}{\textbf{Pechinkin A.\,V., Sokolov I.\,A., Chaplygin V.\,V.}\ \ Stationary Characteristics of a Multichannel Queueing System with\nobreakspace {}Simultaneous Refusals of Servers}{\qquad 2 \qquad 39} 
\contentsline {subsection}{\textbf{Shorgin S.\,Ya.} see Batrakova D.\,A.\hfill\hfill\hfill\hfill\hfill\hfill\hfill\hfill\hfill\hfill\hfill\hfill\hfill\hfill\hfill\hfill\hfill\hfill\hfill\hfill\hfill\hfill\hfill\hfill\hfill\hfill\hfill\hfill\hfill\hfill\hfill\hfill\hfill\hfill\hfill}{\ }
\contentsline {subsection}{\textbf{Shorgin S.\,Ya.} see Kudryavtsev A.\,A.\hfill\hfill\hfill\hfill\hfill\hfill\hfill\hfill\hfill\hfill\hfill\hfill\hfill\hfill\hfill\hfill\hfill\hfill\hfill\hfill\hfill\hfill\hfill\hfill\hfill\hfill\hfill\hfill\hfill\hfill\hfill\hfill\hfill\hfill\hfill}{\ }
\contentsline {subsection}{\textbf{Sinitsyn I.\,N.}\ \ Correlational Methods for Analytical Informational Models of the Earth Pole Fluctuations Design Based on a priori Data}{\qquad 2 \qquad \hphantom{9}2}
\contentsline {subsection}{\textbf{Sinitsyn I.\,N.}\ \ Development of Pugachev Filtering for Stochastic Systems}{\qquad 1 \qquad \hphantom{9}3}
\contentsline {subsection}{\textbf{Sokolov I.\,A.} see Ilyin V.\,D.\hfill\hfill\hfill\hfill\hfill\hfill\hfill\hfill\hfill\hfill\hfill\hfill\hfill\hfill\hfill\hfill\hfill\hfill\hfill\hfill\hfill\hfill\hfill\hfill\hfill\hfill\hfill\hfill\hfill\hfill\hfill\hfill\hfill\hfill\hfill}{\ }
\contentsline {subsection}{\textbf{Sokolov I.\,A.} see Pechinkin A.\,V.\hfill\hfill\hfill\hfill\hfill\hfill\hfill\hfill\hfill\hfill\hfill\hfill\hfill\hfill\hfill\hfill\hfill\hfill\hfill\hfill\hfill\hfill\hfill\hfill\hfill\hfill\hfill\hfill\hfill\hfill\hfill\hfill\hfill\hfill\hfill}{\ }
\contentsline {subsection}{\textbf{Sokolov I.\,A.} see Pechinkin A.\,V.\hfill\hfill\hfill\hfill\hfill\hfill\hfill\hfill\hfill\hfill\hfill\hfill\hfill\hfill\hfill\hfill\hfill\hfill\hfill\hfill\hfill\hfill\hfill\hfill\hfill\hfill\hfill\hfill\hfill\hfill\hfill\hfill\hfill\hfill\hfill}{\ }
\contentsline {subsection}{\textbf{Sokolov I.\,A.} see Zakharov V.\,N.\hfill\hfill\hfill\hfill\hfill\hfill\hfill\hfill\hfill\hfill\hfill\hfill\hfill\hfill\hfill\hfill\hfill\hfill\hfill\hfill\hfill\hfill\hfill\hfill\hfill\hfill\hfill\hfill\hfill\hfill\hfill\hfill\hfill\hfill\hfill}{\ }
\contentsline {subsection}{\textbf{Stupnikov S.\,A.} see Zakharov V.\,N.\hfill\hfill\hfill\hfill\hfill\hfill\hfill\hfill\hfill\hfill\hfill\hfill\hfill\hfill\hfill\hfill\hfill\hfill\hfill\hfill\hfill\hfill\hfill\hfill\hfill\hfill\hfill\hfill\hfill\hfill\hfill\hfill\hfill\hfill\hfill}{\ }
\contentsline {subsection}{\textbf{Zakharov V.\,N., Kalinichenko L.\,A., Sokolov I.\,A., Stupnikov S.\,A.}\ \ Development of Canonical Information Models for Integrated Information Systems}{\qquad 2 \qquad 15} 
\contentsline {subsection}{\textbf{Zakharov V.\,N., Kozmidiady V.\,A.}\ \ Means Providing Applications Fault Tolerance}{\qquad 1 \qquad 14} 
\def\leftfootline{\small{\textbf{\thepage}
\hfill ИНФОРМАТИКА И ЕЁ ПРИМЕНЕНИЯ\ \ \ том~1\ \ \ выпуск~2\ \ \ 2007}
}%
 \def\rightfootline{\small{ИНФОРМАТИКА И ЕЁ ПРИМЕНЕНИЯ\ \ \ том~1\ \ \ выпуск~2\ \ \ 2007
 \hfill \textbf{\thepage}}}
 \label{end\stat} 


%\end{document}

%
\def\stat{rekl}
%\label{preobr}

%\def\tit{АКАДЕМИК ПУГАЧЁВ  ВЛАДИМИР СЕМЁНОВИЧ\\
%25.03.1911--25.03.1998}


%   \vspace*{-48pt}
%   \begin{center}\LARGE
%Академик Пугачёв  Владимир Семёнович\\ (25.03.1911--25.03.1998)
%   \end{center}

   %\vspace*{2.5mm}

   \begin{center}

{\prgsh\LARGE
ЮБИЛЕИ}

\end{center}
%\hrule

\vspace*{6pt}


   \vspace*{8mm}

   \thispagestyle{empty}


%\def\stat{emel}


\section*{К 70-летию заместителя директора ИПИ РАН,\\ члена редколлегии журнала
<<Информатика и её применения>>\\ доктора технических наук В.\,И.~Будзко}

\vspace*{18pt}




          \begin{multicols}{2}

%            \label{st\stat}

\begin{center}
\vspace*{1pt}
\mbox{%
\epsfxsize=78mm
\epsfbox{bud-1.eps}
}
\end{center}

\vspace*{12pt}

      14 августа 2014~г.\ исполнилось 70~лет за\-мес\-ти\-те\-лю директора ИПИ РАН по
научной работе доктору технических наук Владимиру Игоревичу Будзко.

      Владимир Игоревич Будзко родился в г.~Москве. Высшее образование получил на факультете
элект\-рон\-но-вы\-чис\-ли\-тель\-ных устройств в Московском
ин\-же\-нер\-но-фи\-зи\-че\-ском институте
(МИФИ), который он окончил в 1968~г., после чего был на\-прав\-лен для прохождения
службы в одну из войс\-ко\-вых частей, где прошел путь от инженера до первого заместителя
командира войсковой части.

      С приходом В.\,И.~Будзко в ИПИ РАН (2001~г.)\ в институте
сформировалось новое научное на\-прав\-ле\-ние теоретических исследований~--- <<Постро\-ение
ин\-фор\-ма\-ци\-он\-но-те\-ле\-ком\-му\-ни\-ка\-ци\-он\-ных\linebreak сис\-тем
высокой до\-ступ\-ности>>. В~рамках этого
направления выполнен широкий круг фундаментальных исследований по поиску подходов и
определению принципов построения средств обеспечения доступности, конфиденциальности
и целостности современных крупномасштабных
ин\-фор\-ма\-ци\-он\-но-те\-ле\-ком\-му\-ни\-ка\-ци\-он\-ных
сис\-тем (ИТС). Разработаны основные сис\-тем\-но-тех\-ни\-че\-ские принципы и базовые
архитектурные решения построения перспективных для условий России ИТС с
централизованной обработкой и хранением информации, сочетающих в себе свойства
высокой доступности, отказо- и катастрофоустойчивости, информационной защищенности.
Определены принципы, методы и математические основы рационального построения и
оптимизации средств восстановления функционирования центров обработки данных (ЦОД)
после возникновения отказов и катастроф, передачи и хранения данных, обеспечения
информационной безопасности при достижении минимальной совокупной стоимости
владения такими системами. Результаты нашли практическое воплощение при реализации
проектов в интересах ряда отечественных государственных и негосударственных
организаций, таких как Банк России (БР), Внешторгбанк, ОАО <<ГМК <<Норильский Никель>>,
<<Газпром>>, Минэкономразвития России, Правительство Москвы, а также ряд силовых
ведомств.

      Под руководством В.\,И.~Будзко начиная с 2001~г.\ выполнен комплекс
      на\-уч\-но-ис\-сле\-до\-ва\-тель\-ских и
      опыт\-но-кон\-ст\-рук\-тор\-ских работ (свыше 100~проектов),
направленных на развитие электронной информационной технологии БР.
Разработаны концепции развития ИТС БР сначала до 2008~г., а затем до 2013~г., которые
были приняты в качестве основы проведения технической политики. За реализацию проекта
<<Катастрофоустойчивая тер\-ри\-то\-ри\-аль\-но-рас\-пре\-де\-лен\-ная
      ин\-фор\-ма\-ци\-он\-но-те\-ле\-ком\-му\-ни\-ка\-ци\-он\-ная сис\-те\-ма централизованной
обработки банковской информации>> В.\,И.~Будзко удостоен Премии Правительства РФ в
области науки и техники за 2010~г.

      В.\,И.~Будзко возглавлял и возглавляет работы по ряду других прикладных проектов,
связанных с созданием, совершенствованием и развитием крупномасштабных ИТС.

      В.\,И.~Будзко~--- генерал-майор, доктор технических наук, член-кор\-рес\-пон\-дент
Академии криптографии РФ, известный ученый в области информатики и применения
информационных технологий при построении территориально распределенных ИТС
различного назначения. Является автором свыше 250~научных работ, опубликованных в
на\-уч\-но-тех\-ни\-че\-ских и специальных изданиях.

    \thispagestyle{empty}

      В.\,И.~Будзко уделяет большое внимание подготовке научных кадров. Под его
руководством защищено 6~диссертаций на соискание ученой степени кандидата
технических наук. Свыше 30~лет он читает лекции в ИКСИ Академии ФСБ, профессор
кафедры НИЯУ МИФИ. Является членом двух диссертационных советов, главным
редактором журнала <<Системы высокой доступности>> и членом редколлегии журнала
<<Информатика и её применения>>.

      \bigskip

      Редакционный совет и Редакционная коллегия журнала <<Информатика и её
применения>> сердечно поздравляют Владимира Игоревича Будзко с 70-ле\-ти\-ем и желают
крепкого здоровья и новых научных достижений.

\end{multicols}



\def\stat{cont}
{%\hrule\par
%\vskip 7pt % 7pt
\raggedleft\Large \bf%\baselineskip=3.2ex
А\,В\,Т\,О\,Р\,С\,К\,И\,Й\ \ У\,К\,А\,З\,А\,Т\,Е\,Л\,Ь\ \ З\,А\ \ 2\,0\,2\,3 г. \vskip 17pt
 \hrule
 \par
\vskip 21pt plus 6pt minus 3pt }

\label{st\stat}

\def\tit{\ }

\def\aut{\ }
\def\auf{\ }

\def\leftkol{\ } % ENGLISH ABSTRACTS}

\def\rightkol{\ } %АВТОРСКИЙ УКАЗАТЕЛЬ ЗА 2021 г.} %ENGLISH ABSTRACTS}

\titele{\tit}{\aut}{\auf}{\leftkol}{\rightkol}
\addcontentsline{toc}{subsection}{\textrm\textbf Авторский указатель за 2023 г.}

\vspace*{-24pt}

\noindent
{\tabcolsep=3pt
\begin{tabular}{p{397pt}cc}
&\textbf{Вып.} & \textbf{Стр.}\\[6pt]
\Avtors{Агаларов~Я.\,М.} Об оптимизации работы резервного прибора в~многолинейной 
системе массового обслуживания&\raisebox{-12pt}[0pt][0pt]{1}&\raisebox{-12pt}[0pt][0pt]{89--95}\\
\Avtors{Агаларов~Я.\,М.} Оптимизация схемы распределения буферной памяти узла 
пакетной коммутации&\raisebox{-12pt}[0pt][0pt]{3}&\raisebox{-12pt}[0pt][0pt]{39--48}\\
\Avtors{Агасандян~Г.\,А.} Многомерные баттерфляи в~задачах оптимизации по CC-VaR&1&107--115\\
\Avtors{Аду~К.\,И.\,Б., Маркова~Е.\,В., Гайдамака~Ю.\,В., Шоргин~С.\,Я.} Анализ схемы 
доступа с~прерыванием при нарезке радиоресурсов сети пятого 
поколения&\raisebox{-12pt}[0pt][0pt]{1}&\raisebox{-12pt}[0pt][0pt]{\hphantom{1}96--106}\\
\Avtors{Архипов~П.\,О., Филиппских~С.\,Л., Цуканов~М.\,В.} Разработка новой модели 
ступенчатой сверточной нейронной сети для классификации аномалий на панорамах&\raisebox{-12pt}[0pt][0pt]{1}&\raisebox{-12pt}[0pt][0pt]{50--56}\\
\Avtors{Бегишев~В.\,О.} см.\ Сопин~Э.\,С.&&\\
\Avtors{Берговин~А.\,К., Ушаков~В.\,Г.} Исследование систем обслуживания со 
смешанными приоритетами&\raisebox{-12pt}[0pt][0pt]{2}&\raisebox{-12pt}[0pt][0pt]{57--61}\\
\Avtors{Борисов~А.\,В.} Рынок с~марковской скачкообразной волатильностью 
I:~мониторинг цены риска как задача оптимальной фильтрации&\raisebox{-12pt}[0pt][0pt]{2}&\raisebox{-12pt}[0pt][0pt]{27--33}\\
\Avtors{Борисов~А.\,В.} Рынок с~марковской скачкообразной волатильностью~II: алгоритм 
вы\-чис\-ле\-ния справедливой цены деривативов&\raisebox{-12pt}[0pt][0pt]{3}&\raisebox{-12pt}[0pt][0pt]{18--24}\\
\Avtors{Борисов А.\,В.} Рынок с марковской скачкообразной волатильностью III:  алгоритм 
мониторинга цены риска по дискретным наблюдениям цен активов&\raisebox{-12pt}[0pt][0pt]{4}&\raisebox{-12pt}[0pt][0pt]{\hphantom{9}9--16}\\
\Avtors{Босов~А.\,В.} Исследование робастности численных аппроксимаций фильтра 
Вонэма&2&41--49\\
\Avtors{Босов~А.\,В.} Оптимальная фильтрация состояния нелинейной динамической 
системы по наблюдениям со случайными запаздываниями&\raisebox{-12pt}[0pt][0pt]{3}&\raisebox{-12pt}[0pt][0pt]{\hphantom{1}8--17}\\
\Avtors{Босов~А.\,В., Иванов~А.\,В.} Технология многофакторной классификации 
математического контента электронной системы обучения&\raisebox{-12pt}[0pt][0pt]{4}&\raisebox{-12pt}[0pt][0pt]{32--41}\\
\Avtors{Босов~А.\,В., Игнатов~А.\,Н.} О~задаче оценки и~анализа риска транспортных 
происшествий на рельсовом транспорте&\raisebox{-12pt}[0pt][0pt]{1}&\raisebox{-12pt}[0pt][0pt]{73--82}\\
\Avtors{Вакуленко~В.\,В., Зацман~И.\,М.} Формализованное описание статистической 
обработки информации в~базах данных&\raisebox{-12pt}[0pt][0pt]{3}&\raisebox{-12pt}[0pt][0pt]{93--99}\\
\Avtors{Васильев~Н.\,С.} Композициональное представление структуры игры многих лиц 
в~моноидальной категории бинарных отношений&\raisebox{-12pt}[0pt][0pt]{2}&\raisebox{-12pt}[0pt][0pt]{18--26}\\
\Avtors{Волканов~Д.\,Ю.} см.\ Горшенин~А.\,К.&&\\
\Avtors{Воронцов~М.\,О., Шестаков~О.\,В.} Среднеквадратичный риск FDR-процедуры 
в~условиях слабой зависимости&\raisebox{-12pt}[0pt][0pt]{2}&\raisebox{-12pt}[0pt][0pt]{34--40}\\
\Avtors{Гайдамака~Ю.\,В.} см.\ Аду~К.\,И.\,Б.&&\\
\Avtors{Гайдамака~Ю.\,В.} см.\ Иванова Д.\,В.&&\\
\Avtors{Гайдамака~Ю.\,В.} см.\ Самуйлов~А.\,К.&&\\
\Avtors{Гаримелла~Р.\,М.} см.\ Разумчик~Р.\,В.&&\\
\Avtors{Гончаров~А.\,А.} Аннотирование параллельных корпусов: подходы и направления 
развития&4&81--87\\
\Avtors{Горбунов~С.\,А.} см.\ Горшенин~А.\,К.&&\\
\Avtors{Горшенин~А.\,К., Горбунов~С.\,А., Волканов~Д.\,Ю.} О~кластеризации объектов 
сетевой вы\-чис\-ли\-тель\-ной инфраструктуры на основе анализа статистических аномалий 
в~трафике&\raisebox{-12pt}[0pt][0pt]{3}&\raisebox{-12pt}[0pt][0pt]{76--87}\\
\Avtors{Грушо~А.\,А., Грушо~Н.\,А., Забежайло~М.\,И., Кульченков~В.\,В., 
Тимонина~Е.\,Е., Шоргин~С.\,Я.} Причинно-следственные связи в~задачах 
классификации&\raisebox{-12pt}[0pt][0pt]{1}&\raisebox{-12pt}[0pt][0pt]{43--49}\\
\Avtors{Грушо~А.\,А., Грушо~Н.\,А., Забежайло~М.\,И., Смирнов~Д.\,В., Тимонина~Е.\,Е.} 
Классификация с~помощью причинно-следственных связей&\raisebox{-12pt}[0pt][0pt]{3}&\raisebox{-12pt}[0pt][0pt]{71--75}\\
\Avtors{Грушо~А.\,А., Грушо~Н.\,А., Забежайло~М.\,И., Тимонина~Е.\,Е., Шоргин~С.\,Я.} 
Сложные причинно-следственные связи&\raisebox{-12pt}[0pt][0pt]{2}&\raisebox{-12pt}[0pt][0pt]{84--89}\\
\end{tabular}
}

\pagebreak

\def\leftkol{АВТОРСКИЙ УКАЗАТЕЛЬ ЗА 2023 г.} % ENGLISH ABSTRACTS}

\def\rightkol{АВТОРСКИЙ УКАЗАТЕЛЬ ЗА 2023 г.} %ENGLISH ABSTRACTS}

%\thispagestyle{myheadings}
\def\leftfootline{\small{\textbf{\thepage}
\hfill ИНФОРМАТИКА И ЕЁ ПРИМЕНЕНИЯ\ \ \ том~17\ \ \ выпуск~4\ \ \ 2023}
}%
 \def\rightfootline{\small{ИНФОРМАТИКА И ЕЁ ПРИМЕНЕНИЯ\ \ \ том~17\ \ \ выпуск~4\ \ \ 2023
 \hfill \textbf{\thepage}}}


\noindent
{\tabcolsep=3pt
\begin{tabular}{p{394pt}cc}
&\textbf{Вып.} & \textbf{Стр.}\\[3pt]
\Avtors{Грушо~Н.\,А.} см.\ Грушо~А.\,А.&&\\
\Avtors{Грушо~Н.\,А.} см.\ Грушо~А.\,А.&&\\
\Avtors{Грушо~Н.\,А.} см.\ Грушо~А.\,А.&&\\
\Avtors{Дулин~С.\,К.} см.\ Розенберг~И.\,Н.&&\\
\Avtors{Дулина~Н.\,Г.} см.\ Розенберг~И.\,Н.&&\\
\Avtors{Дюкова~А.\,П.} см.\ Дюкова~Е.\,В.&&\\
\Avtors{Дюкова~Е.\,В., Масляков~Г.\,О., Дюкова~А.\,П.} Логические методы корректной 
классификации данных&\raisebox{-12pt}[0pt][0pt]{3}&\raisebox{-12pt}[0pt][0pt]{64--70}\\
\Avtors{Забежайло~М.\,И.} см.\ Грушо~А.\,А&&\\
\Avtors{Забежайло~М.\,И.} см.\ Грушо~А.\,А.&&\\
\Avtors{Забежайло~М.\,И.} см.\ Грушо~А.\,А.&&\\
\Avtors{Захаров~В.\,Н.} см.\ Сазонтьев В.\,В.&&\\
\Avtors{Захаров В.\,Н.} см.\ Френкель С.\,Л.&&\\
\Avtors{Зацман~И.\,М.} Данные, информация и~знание в~научной парадигме 
информатики&1&116--125\\
\Avtors{Зацман И.\,М.} Научная парадигма информатики: классификация объектов 
предметной области&\raisebox{-12pt}[0pt][0pt]{4}&\raisebox{-12pt}[0pt][0pt]{\hphantom{9}96--103}\\
\Avtors{Зацман~И.\,М.} Трансформация иерархии Акоффа в~научной парадигме 
информатики&3&107--113\\
\Avtors{Зацман~И.\,М.} см.\ Вакуленко~В.\,В.&&\\
\Avtors{Зейфман~А.\,И.} см.\ Усов~И.\,А.&&\\
\Avtors{Иванов~А.\,В.} см.\ Босов~А.\,В.&&\\
\Avtors{Иванова Д.\,В., Маркова Е.\,В., Шоргин~С.\,Я., Гайдамака~Ю.\,В.} Модели 
совместного обслуживания трафика eMBB и URLLC на основе приоритетов в 
промышленных развертываниях 5G NR&\raisebox{-24pt}[0pt][0pt]{4}&\raisebox{-24pt}[0pt][0pt]{64--70}\\
\Avtors{Игнатов~А.\,Н.} см.\ Босов~А.\,В.&&\\
\Avtors{Инькова~О.\,Ю., Кружков~М.\,Г.} Критерии определения семантической близости 
дискурсивных отношений&\raisebox{-12pt}[0pt][0pt]{3}&\raisebox{-12pt}[0pt][0pt]{100--106}\\
\Avtors{Инькова О.\,Ю., Кружков~М.\,Г.} Степень семантической близости дискурсивных 
отношений:  методы и инструменты расчета&\raisebox{-12pt}[0pt][0pt]{4}&\raisebox{-12pt}[0pt][0pt]{88--95}\\
\Avtors{Кабанов~Ю.\,М., Сидоренко~А.\,П.} Аксиоматический взгляд на модели системного 
риска Роджерса--Вераарт и~Судзуки--Эльсингера&\raisebox{-12pt}[0pt][0pt]{1}&\raisebox{-12pt}[0pt][0pt]{11--17}\\
\Avtors{Карпов~В.\,И.} см.\ Нуриев~В.\,А.&&\\
\Avtors{Кириков~И.\,А.} см.\ Листопад~С.\,В.&&\\
\Avtors{Ковалёв~С.\,П.} Монада диаграмм как математическая метамодель системной 
инженерии&2&11--17\\
\Avtors{Королев~Д.\,О., Малеев~О.\,Г.} Исследование эффективности применения бинарных 
нейронных сетей при детектировании объекта на изображении&\raisebox{-12pt}[0pt][0pt]{3}&\raisebox{-12pt}[0pt][0pt]{88--92}\\
\Avtors{Кривенко~М.\,П.} Критерии выбора размерности модели факторизации&2&50--56\\
\Avtors{Кружков~М.\,Г.} см.\ Инькова О.\,Ю.&&\\
\Avtors{Кружков~М.\,Г.} см.\ Инькова~О.\,Ю.&&\\
\Avtors{Кудрявцев~А.\,А., Шестаков~О.\,В.} Метод оценивания параметров 
гамма-экс\-по\-нен\-ци\-аль\-но\-го распределения по выборке со слабо зависимыми компонентами&\raisebox{-12pt}[0pt][0pt]{3}&\raisebox{-12pt}[0pt][0pt]{58--62}\\
\Avtors{Кульченков~В.\,В.} см.\ Грушо~А.\,А.&&\\
\Avtors{Лапко~А.\,В.} см.\ Тубольцев~В.\,П.&&\\
\Avtors{Лапко~В.\,А.} см.\ Тубольцев~В.\,П.&&\\
\Avtors{Лери~М.\,М.} Среднее расстояние в~конфигурационных графах со степенным 
распределением&\raisebox{-12pt}[0pt][0pt]{1}&\raisebox{-12pt}[0pt][0pt]{28--34}\\
\Avtors{Листопад~С.\,В., Кириков~И.\,А.} Метод на основе нечетких правил для 
управления конфликтами агентов в~гибридных интеллектуальных многоагентных 
системах&\raisebox{-12pt}[0pt][0pt]{1}&\raisebox{-12pt}[0pt][0pt]{66--72}\\
\Avtors{Малашенко~Ю.\,Е., Назарова~И.\,А.} Анализ загрузки многопользовательской сети 
при расщеплении потоков по кратчайшим маршрутам&\raisebox{-12pt}[0pt][0pt]{3}&\raisebox{-12pt}[0pt][0pt]{33--38}\\
\Avtors{Малашенко~Ю.\,Е., Назарова~И.\,А.} Оценки распределения ресурсов 
в~многопользовательской сети при равных межузловых нагрузках&\raisebox{-12pt}[0pt][0pt]{1}&\raisebox{-12pt}[0pt][0pt]{83--88}\\
\Avtors{Малеев~О.\,Г.} см.\ Королев~Д.\,О.&&\\
\Avtors{Маркова~Е.\,В.} см.\ Аду~К.\,И.\,Б.&&\\
\Avtors{Маркова Е.\,В.} см.\ Иванова Д.\,В.&&\\
\end{tabular}
}

\pagebreak

\def\leftkol{АВТОРСКИЙ УКАЗАТЕЛЬ ЗА 2023 г.} % ENGLISH ABSTRACTS}

\def\rightkol{АВТОРСКИЙ УКАЗАТЕЛЬ ЗА 2023 г.} %ENGLISH ABSTRACTS}

%\thispagestyle{myheadings}
\def\leftfootline{\small{\textbf{\thepage}
\hfill ИНФОРМАТИКА И ЕЁ ПРИМЕНЕНИЯ\ \ \ том~17\ \ \ выпуск~4\ \ \ 2023}
}%
 \def\rightfootline{\small{ИНФОРМАТИКА И ЕЁ ПРИМЕНЕНИЯ\ \ \ том~17\ \ \ выпуск~4\ \ \ 2023
 \hfill \textbf{\thepage}}}


\noindent
{\tabcolsep=3pt
\begin{tabular}{p{394pt}cc}
&\textbf{Вып.} & \textbf{Стр.}\\[3pt]
\Avtors{Маслов~А.\,Р.} см.\ Сопин~Э.\,С&&\\
\Avtors{Масляков~Г.\,О.} см.\ Дюкова~Е.\,В.&&\\
\Avtors{Мелехин~В.\,Б., Хачумов~В.\,М., Хачумов~М.\,В.} Самообучение автономных 
интеллектуальных роботов в~процессе поисково-исследовательской деятельности&\raisebox{-12pt}[0pt][0pt]{2}&\raisebox{-12pt}[0pt][0pt]{78--83}\\
\Avtors{Назарова~И.\,А.} см.\ Малашенко~Ю.\,Е.&&\\
\Avtors{Назарова~И.\,А.} см.\ Малашенко~Ю.\,Е.&&\\
\Avtors{Нейчев~Р.\,Г., Шибаев~И.\,А., Стрижов~В.\,В.} Восстановление матрицы 
суперпозиции в~задаче символьной регрессии&\raisebox{-12pt}[0pt][0pt]{1}&\raisebox{-12pt}[0pt][0pt]{35--42}\\
\Avtors{Нуриев~В.\,А., Карпов~В.\,И.} Методология корпусно-ориентированного 
исследования в~области контрастивной пунктуации&\raisebox{-12pt}[0pt][0pt]{2}&\raisebox{-12pt}[0pt][0pt]{90--95}\\
\Avtors{Пешкова И.\,В.} Границы незавершенной работы в системе с повторными вызовами 
разных классов и показательным временем обслуживания&\raisebox{-12pt}[0pt][0pt]{4}&\raisebox{-12pt}[0pt][0pt]{57--63}\\
\Avtors{Платонова~А.\,А.} см.\ Самуйлов~А.\,К.&&\\
\Avtors{Рабинович Я.\,И.} Процедура построения множества Парето для дифференцируемых 
критериальных функций&\raisebox{-12pt}[0pt][0pt]{4}&\raisebox{-12pt}[0pt][0pt]{17--22}\\
\Avtors{Разумчик~Р.\,В., Румянцев~А.\,С., Гаримелла~Р.\,М.} Вероятностная модель для 
оценки основных характеристик производительности марковской модели 
суперкомпьютера&\raisebox{-24pt}[0pt][0pt]{2}&\raisebox{-24pt}[0pt][0pt]{62--70}\\
\Avtors{Розенберг~И.\,Н., Дулин~С.\,К., Дулина~Н.\,Г.} Моделирование структуры 
интероперабельности средствами структурной согласованности&\raisebox{-12pt}[0pt][0pt]{1}&\raisebox{-12pt}[0pt][0pt]{57--65}\\
\Avtors{Румовская~С.\,Б.} Подходы к~подбору специалистов при организации 
коллективного решения проблем&\raisebox{-12pt}[0pt][0pt]{2}&\raisebox{-12pt}[0pt][0pt]{\hphantom{1}96--103}\\
\Avtors{Румянцев~А.\,С.} см.\ Разумчик~Р.\,В.&&\\
\Avtors{Сазонтьев В.\,В., Ступников~С.\,А., Захаров~В.\,Н.} Расширяемый подход к слиянию 
данных в распределенных вычислительных средах&\raisebox{-12pt}[0pt][0pt]{4}&\raisebox{-12pt}[0pt][0pt]{42--47}\\
\Avtors{Самуйлов~А.\,К., Платонова~А.\,А., Шоргин~В.\,С., Гайдамака~Ю.\,В.} 
К~моделированию эффектов обслуживания многоадресного трафика в~сетях 5G~NR&\raisebox{-12pt}[0pt][0pt]{2}&\raisebox{-12pt}[0pt][0pt]{71--77}\\
\Avtors{Сатин~Я.\,А.} см.\ Усов~И.\,А.&&\\
\Avtors{Сидоренко~А.\,П.} см.\ Кабанов~Ю.\,М.&&\\
\Avtors{Синицын~И.\,Н.} Аналитическое моделирование распределений с~инвариантной 
мерой в~стохастических системах, не разрешенных относительно 
производных&\raisebox{-12pt}[0pt][0pt]{1}&\raisebox{-12pt}[0pt][0pt]{\hphantom{1}2--10}\\
\Avtors{Смирнов~Д.\,В.} см.\ Грушо~А.\,А.&&\\
\Avtors{Сопин~Э.\,С., Маслов~А.\,Р., Шоргин~В.\,С., Бегишев~В.\,О.} Моделирование 
настойчивого поведения пользователей в~сетях 5G NR с~адаптацией скорости 
и~блокировками&\raisebox{-12pt}[0pt][0pt]{3}&\raisebox{-12pt}[0pt][0pt]{25--32}\\
\Avtors{Степанов~Е.\,П.} см.\ Шестаков~О.\,В.&&\\
\Avtors{Стрижов~В.\,В.} см.\ Нейчев~Р.\,Г.&&\\
\Avtors{Ступников~С.\,А.} см.\ Сазонтьев В.\,В.&&\\
\Avtors{Тимонина~Е.\,Е.} см.\ Грушо~А.\,А.&&\\
\Avtors{Тимонина~Е.\,Е.} см.\ Грушо~А.\,А.&&\\
\Avtors{Тимонина~Е.\,Е.} см.\ Грушо~А.\,А.&&\\
\Avtors{Торшин~И.\,Ю.} О~задачах оптимизации, возникающих при применении 
топологического анализа данных к~поиску алгоритмов прогнозирования с~фиксированными 
корректорами&\raisebox{-24pt}[0pt][0pt]{2}&\raisebox{-24pt}[0pt][0pt]{\hphantom{1}2--10}\\
\Avtors{Торшин~И.\,Ю.} О~формировании множеств прецедентов на основе таблиц 
разнородных признаковых описаний методами топологической теории анализа 
данных&\raisebox{-12pt}[0pt][0pt]{3}&\raisebox{-12pt}[0pt][0pt]{2--7}\\
\Avtors{Тубольцев~В.\,П., Лапко~А.\,В., Лапко~В.\,А.} Непараметрический алгоритм 
автоматической классификации данных дистанционного зондирования&\raisebox{-12pt}[0pt][0pt]{4}&\raisebox{-12pt}[0pt][0pt]{23--31}\\
\Avtors{Усов~И.\,А., Сатин~Я.\,А., Зейфман~А.\,И.} О~скорости сходимости и~предельных 
характеристиках для одного обобщенного процесса рождения и~гибели&\raisebox{-12pt}[0pt][0pt]{3}&\raisebox{-12pt}[0pt][0pt]{49--57}\\
\Avtors{Ушаков~В.\,Г., Ушаков~Н.\,Г.} Критерии нормальности вероятностного 
распределения при округленных данных&\raisebox{-12pt}[0pt][0pt]{1}&\raisebox{-12pt}[0pt][0pt]{18--27}\\
\Avtors{Ушаков~В.\,Г.} см.\ Берговин~А.\,К.&&\\
\Avtors{Ушаков~Н.\,Г.} см.\ Ушаков~В.\,Г.&&\\
\Avtors{Филиппских~С.\,Л.} см.\ Архипов~П.\,О.&&\\
\end{tabular}
}

\pagebreak

\def\leftkol{АВТОРСКИЙ УКАЗАТЕЛЬ ЗА 2023 г.} % ENGLISH ABSTRACTS}

\def\rightkol{АВТОРСКИЙ УКАЗАТЕЛЬ ЗА 2023 г.} %ENGLISH ABSTRACTS}

%\thispagestyle{myheadings}
\def\leftfootline{\small{\textbf{\thepage}
\hfill ИНФОРМАТИКА И ЕЁ ПРИМЕНЕНИЯ\ \ \ том~17\ \ \ выпуск~4\ \ \ 2023}
}%
 \def\rightfootline{\small{ИНФОРМАТИКА И ЕЁ ПРИМЕНЕНИЯ\ \ \ том~17\ \ \ выпуск~4\ \ \ 2023
 \hfill \textbf{\thepage}}}


\noindent
{\tabcolsep=3pt
\begin{tabular}{p{394pt}cc}
&\textbf{Вып.} & \textbf{Стр.}\\[3pt]
\Avtors{Френкель С.\,Л., Захаров В.\,Н.} Модели учета влияния статистических 
характеристик трафика вычислительных сетей на эффективность прогнозирования 
средствами машинного обучения&\raisebox{-24pt}[0pt][0pt]{4}&\raisebox{-24pt}[0pt][0pt]{71--80}\\
\Avtors{Хачумов~В.\,М.} см.\ Мелехин~В.\,Б.&&\\
\Avtors{Хачумов~М.\,В.} см.\ Мелехин~В.\,Б.&&\\
\Avtors{Цуканов~М.\,В.} см.\ Архипов~П.\,О.&&\\
\Avtors{Шестаков~О.\,В., Степанов~Е.\,П.} Нелинейная регуляризация обращения линейных 
однородных операторов с помощью метода блочной пороговой обработки&\raisebox{-12pt}[0pt][0pt]{4}&\raisebox{-12pt}[0pt][0pt]{2--8}\\
\Avtors{Шестаков~О.\,В.} см.\ Воронцов~М.\,О.&&\\
\Avtors{Шестаков~О.\,В.} см.\ Кудрявцев~А.\,А.&&\\
\Avtors{Шибаев~И.\,А.} см.\ Нейчев~Р.\,Г.&&\\
\Avtors{Шнурков П.\,В.} Решение задачи оптимального управления запасом непрерывного 
продукта в стохастической модели регенерации со случайными стоимостными 
характеристиками&\raisebox{-24pt}[0pt][0pt]{4}&\raisebox{-24pt}[0pt][0pt]{48--56}\\
\Avtors{Шоргин~В.\,С.} см.\ Самуйлов~А.\,К.&&\\
\Avtors{Шоргин~В.\,С.} см.\ Сопин~Э.\,С.&&\\
\Avtors{Шоргин~С.\,Я.} см.\ Аду~К.\,И.\,Б.&&\\
\Avtors{Шоргин~С.\,Я.} см.\ Грушо~А.\,А.&&\\
\Avtors{Шоргин~С.\,Я.} см.\ Грушо~А.\,А.&&\\
\Avtors{Шоргин~С.\,Я.} см.\ Иванова Д.\,В.&&\\


\end{tabular}
}

%\thispagestyle{myheadings}
\def\leftfootline{\small{\textbf{\thepage}
\hfill ИНФОРМАТИКА И ЕЁ ПРИМЕНЕНИЯ\ \ \ том~17\ \ \ выпуск~4\ \ \ 2023}
}%
 \def\rightfootline{\small{ИНФОРМАТИКА И ЕЁ ПРИМЕНЕНИЯ\ \ \ том~17\ \ \ выпуск~4\ \ \ 2023
 \hfill \textbf{\thepage}}}

 \label{end\stat}

\newpage

\def\stat{cont-e}
{%\hrule\par
%\vskip 7pt % 7pt
\raggedleft\Large \bf%\baselineskip=3.2ex
2\,0\,2\,3\ \ A\,U\,T\,H\,O\,R\ \ I\,N\,D\,E\,X \vskip 17pt
 \hrule
 \par
\vskip 21pt plus 6pt minus 3pt }

\label{st\stat}

\def\tit{\ }

\def\aut{\ }
\def\auf{\ }

\def\leftkol{\ } %2021 AUTHOR INDEX} % ENGLISH ABSTRACTS}

\def\rightkol{\ } %2021 AUTHOR INDEX} %ENGLISH ABSTRACTS}

\titele{\tit}{\aut}{\auf}{\leftkol}{\rightkol}
\addcontentsline{toc}{subsection}{\textrm\textbf 2023 Author Index}

\def\leftfootline{\small{\textbf{\thepage}
\hfill INFORMATIKA I EE PRIMENENIYA~--- INFORMATICS AND APPLICATIONS\ \ \ 2023\
\ \ volume~17\ \ \ issue\ 4}
}%
 \def\rightfootline{\small{INFORMATIKA I EE PRIMENENIYA~--- INFORMATICS AND APPLICATIONS\ \ \ 2023\ \ \ volume~17\ \ \ issue\ 4
\hfill \textbf{\thepage}}}

\vspace*{-24pt}

\noindent
{\tabcolsep=3pt
\begin{tabular}{p{395.89pt}cc}
&\textbf{Issue} & \textbf{Page}\\[6pt]
\Avtors{Adou~K.\,Y.\,B., Markova~E.\,V., Gaidamaka~Yu.\,V., and~Shorgin~S.\,Ya.} 
Preemption-based prioritization scheme for network resources slicing in 5G 
systems&\raisebox{-12pt}[0pt][0pt]{1}&\raisebox{-12pt}[0pt][0pt]{\hphantom{1}96--106}\\
\Avtors{Agalarov~Ya.\,M.} Optimization of a queue-length dependent additional server in the 
multiserver queue&\raisebox{-12pt}[0pt][0pt]{1}&\raisebox{-12pt}[0pt][0pt]{89--95}\\
\Avtors{Agalarov~Ya.\,M.} Optimization of the buffer memory allocation scheme of the packet 
switching node&\raisebox{-12pt}[0pt][0pt]{3}&\raisebox{-12pt}[0pt][0pt]{39--48}\\
\Avtors{Agasandyan~G.\,A.} Multidimensional butterflies in problems of optimization on CC-VaR&1&107--115\\
\Avtors{Arkhipov~P.\,O., Philippskih~S.\,L., and~Tsukanov~M.\,V.} Development of a~new model 
of a~step convolutional neural network for classification of anomalies on panoramas&\raisebox{-12pt}[0pt][0pt]{1}&\raisebox{-12pt}[0pt][0pt]{50--56}\\
\Avtors{Begishev~V.\,O.} see Sopin~E.\,S.&&\\
\Avtors{Bergovin~A.\,K. and~Ushakov~V.\,G.} Analysis of the queueing systems with mixed 
priorities&2&57--61\\
\Avtors{Borisov~A.\,V.} Market with Markov jump volatility I: Price of risk monitoring as an 
optimal filtering problem&\raisebox{-12pt}[0pt][0pt]{2}&\raisebox{-12pt}[0pt][0pt]{27--33}\\
\Avtors{Borisov~A.\,V.} Market with Markov jump volatility~II: Algorithm of derivative fair price 
calculation&3&18--24\\
\Avtors{Borisov A.\,V.} Market with Markov jump volatility III: Price of risk monitoring algorithm 
given discrete-time observations of asset prices&\raisebox{-12pt}[0pt][0pt]{4}&\raisebox{-12pt}[0pt][0pt]{\hphantom{9}9--16}\\
\Avtors{Bosov~A.\,V.} Nonlinear dynamic system state optimal filtering by observations with 
random delays&\raisebox{-12pt}[0pt][0pt]{3}&\raisebox{-12pt}[0pt][0pt]{\hphantom{1}8--17}\\
\Avtors{Bosov~A.\,V.} Robustness investigation of the numerical approximation of the Wonham 
filter&2&41--49\\
\Avtors{Bosov~A.\,V. and~Ignatov~A.\,N.} On the problem of assessing and analyzing traffic 
accidents risk on the rail transport&\raisebox{-12pt}[0pt][0pt]{1}&\raisebox{-12pt}[0pt][0pt]{73--82}\\
\Avtors{Bosov~A.\,V. and Ivanov~A.\,V.} Multifactor classification technology of mathematical 
content of e-learning system&\raisebox{-12pt}[0pt][0pt]{4}&\raisebox{-12pt}[0pt][0pt]{32--41}\\
\Avtors{Djukova~A.\,P.} see Djukova~E.\,V.&&\\
\Avtors{Djukova~E.\,V., Masliakov~G.\,O., and Djukova~A.\,P.} Logical methods of correct data 
classification&3&64--70\\
\Avtors{Dulin~S.\,K.} see Rozenberg~I.\,N.&&\\
\Avtors{Dulina~N.\,G.} see Rozenberg~I.\,N.&&\\
\Avtors{Frenkel~S.\,L. and Zakharov~V.\,N.} Models for study of the influence of statistical 
characteristics of computer networks traffic on the efficiency of prediction by machine learning 
tools&\raisebox{-12pt}[0pt][0pt]{4}&\raisebox{-12pt}[0pt][0pt]{71--80}\\
\Avtors{Gaidamaka~Yu.\,V.} see Adou~K.\,Y.\,B.&&\\
\Avtors{Gaidamaka~Yu.\,V.} see Ivanova~D.\,V.&&\\
\Avtors{Gaidamaka~Yu.\,V.} see Samouylov~A.\,K.&&\\
\Avtors{Garimella~R.\,M.} see Razumchik~R.\,V.&&\\
\Avtors{Goncharov~A.\,A.} Parallel corpus annotation: Approaches and directions for 
development&4&81--87\\
\Avtors{Gorbunov~S.\,A.} see Gorshenin~A.\,K.&&\\
\Avtors{Gorshenin~A.\,K., Gorbunov~S.\,A., and Volkanov~D.\,Yu.} Toward clustering of 
network computing infrastructure objects based on analysis of statistical anomalies in network 
traffic&\raisebox{-12pt}[0pt][0pt]{3}&\raisebox{-12pt}[0pt][0pt]{76--87}\\
\Avtors{Grusho~A.\,A., Grusho~N.\,A., Zabezhailo~M.\,I., Kulchenkov~V.\,V., Timonina~E.\,E., 
and~Shorgin~S.\,Ya.} Causal relationships in classification problems&\raisebox{-12pt}[0pt][0pt]{1}&\raisebox{-12pt}[0pt][0pt]{43--49}\\
\Avtors{Grusho~A.\,A., Grusho~N.\,A., Zabezhailo~M.\,I., Smirnov~D.\,V., and Timonina~E.\,E.} 
Classification by cause-and-effect relationships&\raisebox{-12pt}[0pt][0pt]{3}&\raisebox{-12pt}[0pt][0pt]{71--75}\\
\Avtors{Grusho~A.\,A., Grusho~N.\,A., Zabezhailo~M.\,I., Timonina~E.\,E., 
and~Shorgin~S.\,Ya.} Complex cause-and-effect relationships&\raisebox{-12pt}[0pt][0pt]{2}&\raisebox{-12pt}[0pt][0pt]{84--89}\\
\Avtors{Grusho~N.\,A.} see Grusho~A.\,A.&&\\
\Avtors{Grusho~N.\,A.} see Grusho~A.\,A.&&\\
\Avtors{Grusho~N.\,A.} see Grusho~A.\,A.&&\\
\end{tabular}
}
\pagebreak

\def\leftfootline{\small{\textbf{\thepage}
\hfill INFORMATIKA I EE PRIMENENIYA~--- INFORMATICS AND APPLICATIONS\ \ \ 2023\
\ \ volume~17\ \ \ issue\ 4}
}%
 \def\rightfootline{\small{INFORMATIKA I EE PRIMENENIYA~---
INFORMATICS AND APPLICATIONS\ \ \ 2023\ \ \ volume~17\ \ \ issue\ 4
\hfill \textbf{\thepage}}}

\def\leftkol{2023 AUTHOR INDEX} % ENGLISH ABSTRACTS}

\def\rightkol{2023 AUTHOR INDEX} %ENGLISH ABSTRACTS}


\noindent
{\tabcolsep=3pt
\begin{tabular}{p{395.5pt}cc}
&\textbf{Issue} & \textbf{Page}\\[6pt]
\Avtors{Ignatov~A.\,N.} see Bosov~A.\,V.&&\\
\Avtors{Inkova O.\,Yu. and Kruzhkov M.\,G.} Evaluating the degree of discourse relations 
semantic affinity: Methods and instruments&\raisebox{-12pt}[0pt][0pt]{4}&\raisebox{-12pt}[0pt][0pt]{88--95}\\
\Avtors{Inkova~O.\,Yu. and Kruzhkov~M.\,G.} Evaluation criteria for discourse relations semantic 
affinity&3&100--106\\
\Avtors{Kruzhkov~M.\,G.} see Inkova~O.\,Yu.&&\\
\Avtors{Ivanov~A.\,V.} see Bosov~A.\,V.&&\\
\Avtors{Ivanova~D.\,V., Markova~E.\,V., Shorgin~S.\,Ya., and~Gaidamaka~Yu.\,V.} Priority-based 
eMBB and URLLC traffic coexistence models in 5G NR industrial deployments&\raisebox{-12pt}[0pt][0pt]{4}&\raisebox{-12pt}[0pt][0pt]{64--70}\\
\Avtors{Kabanov~Yu.\,M. and~Sidorenko~A.\,P.} An axiomatic viewpoint on the Rogers--Veraart 
and Suzuki--Elsinger models of systemic risk&\raisebox{-12pt}[0pt][0pt]{1}&\raisebox{-12pt}[0pt][0pt]{11--17}\\
\Avtors{Karpov~V.\,I.} see Nuriev~V.\,A.&&\\
\Avtors{Khachumov~M.\,V.} see Melekhin~V.\,B.&&\\
\Avtors{Khachumov~V.\,M.} see Melekhin~V.\,B.&&\\
\Avtors{Kirikov~I.\,A.} see Listopad~S.\,V.&&\\
\Avtors{Korolev~D.\,O. and Maleev~O.\,G.} Efficiency of binary neural networks for object 
detection on an image&\raisebox{-12pt}[0pt][0pt]{3}&\raisebox{-12pt}[0pt][0pt]{88--92}\\
\Avtors{Kovalyov~S.\,P.} The monad of diagrams as a mathematical metamodel of systems 
engineering&2&11--17\\
\Avtors{Krivenko~M.\,P.} Criteria for choosing the factorization model dimensionality&2&50--56\\
\Avtors{Kruzhkov M.\,G.} see Inkova O.\,Yu.&&\\
\Avtors{Kudryavtsev~A.\,A. and Shestakov~O.\,V.} A~method for estimating parameters of the 
gamma-exponential distribution from a~sample with weakly dependent components&\raisebox{-12pt}[0pt][0pt]{3}&\raisebox{-12pt}[0pt][0pt]{58--63}\\
\Avtors{Kulchenkov~V.\,V.} see Grusho~A.\,A.&&\\
\Avtors{Lapko~A.\,V.} see Tuboltsev V.\,P.&&\\
\Avtors{Lapko~V.\,A.} see Tuboltsev V.\,P.&&\\
\Avtors{Leri~M.\,M.} An average distance in the power-law configuration graphs&1&28--34\\
\Avtors{Listopad~S.\,V. and~Kirikov~I.\,A.} Fuzzy rules based method for agent conflict 
management in hybrid intelligent multiagent systems&\raisebox{-12pt}[0pt][0pt]{1}&\raisebox{-12pt}[0pt][0pt]{66--72}\\
\Avtors{Malashenko~Yu.\,E. and~Nazarova~I.\,A.} Estimates of the resource distribution in the 
multiuser network with equal internodal loads&\raisebox{-12pt}[0pt][0pt]{1}&\raisebox{-12pt}[0pt][0pt]{83--88}\\
\Avtors{Malashenko~Yu.\,E. and Nazarova~I.\,A.} Multiuser network load analysis by splitting 
flows along the shortest routes&\raisebox{-12pt}[0pt][0pt]{3}&\raisebox{-12pt}[0pt][0pt]{33--38}\\
\Avtors{Maleev~O.\,G.} see Korolev~D.\,O.&&\\
\Avtors{Markova~E.\,V.} see Adou~K.\,Y.\,B.&&\\
\Avtors{Markova~E.\,V.} see Ivanova~D.\,V.&&\\
\Avtors{Masliakov~G.\,O.} see Djukova~E.\,V.&&\\
\Avtors{Maslov~A.\,R.} see Sopin~E.\,S.&&\\
\Avtors{Melekhin~V.\,B., Khachumov~V.\,M., and~Khachumov~M.\,V.} Self-learning of 
autonomous intelligent robots in the process of search and explore activities&\raisebox{-12pt}[0pt][0pt]{2}&\raisebox{-12pt}[0pt][0pt]{78--83}\\
\Avtors{Nazarova~I.\,A.} see Malashenko~Yu.\,E.&&\\
\Avtors{Nazarova~I.\,A.} see Malashenko~Yu.\,E.&&\\
\Avtors{Neychev~R.\,G., Shibaev~I.\,A., and~Strijov~V.\,V.} Optimal spanning tree reconstruction 
in symbolic regression&\raisebox{-12pt}[0pt][0pt]{1}&\raisebox{-12pt}[0pt][0pt]{35--42}\\
\Avtors{Nuriev~V.\,A. and~Karpov~V.\,I.} Methodology of the corpus-based studies in the field of 
contrastive punctuation&\raisebox{-12pt}[0pt][0pt]{2}&\raisebox{-12pt}[0pt][0pt]{90--95}\\
\Avtors{Peshkova~I.\,V.} Bounds of the workload in a~multiclass retrial queue with exponential 
services&\raisebox{-12pt}[0pt][0pt]{4}&\raisebox{-12pt}[0pt][0pt]{57--63}\\
\Avtors{Philippskih~S.\,L.} see Arkhipov~P.\,O.&&\\
\Avtors{Platonova~A.\,A.} see Samouylov~A.\,K.&&\\
\Avtors{Rabinovich Ya.\,I.} Procedure of constructing a~Pareto set for differentiable criteria 
functions&4&17--22\\
\Avtors{Razumchik~R.\,V., Rumyantsev~A.\,S., and~Garimella~R.\,M.} A~queueing system for 
performance evaluation of a~Markovian supercomputer model&\raisebox{-12pt}[0pt][0pt]{2}&\raisebox{-12pt}[0pt][0pt]{62--70}\\
\Avtors{Rozenberg~I.\,N., Dulin~S.\,K., and~Dulina~N.\,G.} Modeling the structure of 
interoperability by means of structural consistency&\raisebox{-12pt}[0pt][0pt]{1}&\raisebox{-12pt}[0pt][0pt]{57--65}\\
\Avtors{Rumovskaya~S.\,B.} Selection of specialists in the organization of collective solving 
problems&2&\hphantom{1}96--103\\
\end{tabular}
}
\pagebreak

\def\leftfootline{\small{\textbf{\thepage}
\hfill INFORMATIKA I EE PRIMENENIYA~--- INFORMATICS AND APPLICATIONS\ \ \ 2023\
\ \ volume~17\ \ \ issue\ 4}
}%
 \def\rightfootline{\small{INFORMATIKA I EE PRIMENENIYA~---
INFORMATICS AND APPLICATIONS\ \ \ 2023\ \ \ volume~17\ \ \ issue\ 4
\hfill \textbf{\thepage}}}

\def\leftkol{2023 AUTHOR INDEX} % ENGLISH ABSTRACTS}

\def\rightkol{2023 AUTHOR INDEX} %ENGLISH ABSTRACTS}


\noindent
{\tabcolsep=3pt
\begin{tabular}{p{395.5pt}cc}
&\textbf{Issue} & \textbf{Page}\\[6pt]
\Avtors{Rumyantsev~A.\,S.} see Razumchik~R.\,V.&&\\
\Avtors{Samouylov~A.\,K., Platonova~A.\,A., Shorgin~V.\,S., and~Gaidamaka~Yu.\,V.} On 
modeling the effects of multicast traffic servicing in 5G NR networks&\raisebox{-12pt}[0pt][0pt]{2}&\raisebox{-12pt}[0pt][0pt]{71--77}\\
\Avtors{Satin~Y.\,A.} see Usov~I.\,A.&&\\
\Avtors{Sazontev V.\,V., Stupnikov~S.\,A., and~Zakharov~V.\,N.} An extensible approach to data 
fusion in~distributed computing environments&\raisebox{-12pt}[0pt][0pt]{4}&\raisebox{-12pt}[0pt][0pt]{42--47}\\
\Avtors{Shestakov~O.\,V. and Stepanov~E.\,P.} Nonlinear regularization of the inversion of linear 
homogeneous operators using the block thresholding method&\raisebox{-12pt}[0pt][0pt]{4}&\raisebox{-12pt}[0pt][0pt]{2--8}\\
\Avtors{Shestakov~O.\,V.} see Kudryavtsev~A.\,A.&&\\
\Avtors{Shestakov~O.\,V.} see Vorontsov~M.\,O.&&\\
\Avtors{Shibaev~I.\,A.} see Neychev~R.\,G.&&\\
\Avtors{Shnurkov P.\,V.} Solution of the problem of~optimal control of~the stock of a~continuous 
product in a~stochastic model of regeneration with random cost characteristics&\raisebox{-12pt}[0pt][0pt]{4}&\raisebox{-12pt}[0pt][0pt]{48--56}\\
\Avtors{Shorgin~S.\,Ya.} see Adou~K.\,Y.\,B.&&\\
\Avtors{Shorgin~S.\,Ya.} see Grusho~A.\,A.&&\\
\Avtors{Shorgin~S.\,Ya.} see Grusho~A.\,A.&&\\
\Avtors{Shorgin~S.\,Ya.} see Ivanova~D.\,V.&&\\
\Avtors{Shorgin~V.\,S.} see Samouylov~A.\,K.&&\\
\Avtors{Shorgin~V.\,S.} see Sopin~E.\,S.&&\\
\Avtors{Sidorenko~A.\,P.} see Kabanov~Yu.\,M.&&\\
\Avtors{Sinitsyn~I.\,N.} Analytical modeling of distributions with invariant measure in stochastic 
systems with unsolved derivatives&\raisebox{-12pt}[0pt][0pt]{1}&\raisebox{-12pt}[0pt][0pt]{\hphantom{1}2--10}\\
\Avtors{Smirnov~D.\,V.} see Grusho~A.\,A.&&\\
\Avtors{Sopin~E.\,S., Maslov~A.\,R., Shorgin~V.\,S., and Begishev~V.\,O.} Modeling insistent 
user behavior in~5G New Radio networks with rate adaptation and blockage&\raisebox{-12pt}[0pt][0pt]{3}&\raisebox{-12pt}[0pt][0pt]{25--32}\\
\Avtors{Stepanov~E.\,P.} see Shestakov~O.\,V.&&\\
\Avtors{Strijov~V.\,V.} see Neychev~R.\,G.&&\\
\Avtors{Stupnikov~S.\,A.} see Sazontev V.\,V.&&\\
\Avtors{Timonina~E.\,E.} see Grusho~A.\,A.&&\\
\Avtors{Timonina~E.\,E.} see Grusho~A.\,A.&&\\
\Avtors{Timonina~E.\,E.} see Grusho~A.\,A.&&\\
\Avtors{Torshin~I.\,Yu.} On optimization problems arising from the application of topological data 
analysis to the search for forecasting algorithms with fixed correctors&\raisebox{-12pt}[0pt][0pt]{2}&\raisebox{-12pt}[0pt][0pt]{\hphantom{1}2--10}\\
\Avtors{Torshin~I.\,Yu.} On the formation of sets of precedents based on tables of heterogeneous 
feature descriptions by methods of topological theory of data analysis&\raisebox{-12pt}[0pt][0pt]{3}&\raisebox{-12pt}[0pt][0pt]{2--7}\\
\Avtors{Tsukanov~M.\,V.} see Arkhipov~P.\,O.&&\\
\Avtors{Tuboltsev V.\,P., Lapko~A.\,V., and~Lapko~V.\,A.} Nonparametric algorithm for 
automatic classification of remote sensing data&\raisebox{-12pt}[0pt][0pt]{4}&\raisebox{-12pt}[0pt][0pt]{23--31}\\
\Avtors{Ushakov~N.\,G.} see Ushakov~V.\,G.&&\\
\Avtors{Ushakov~V.\,G. and~Ushakov~N.\,G.} Tests for normality of the probabilistic distribution 
when data are rounded&\raisebox{-12pt}[0pt][0pt]{1}&\raisebox{-12pt}[0pt][0pt]{18--27}\\
\Avtors{Ushakov~V.\,G.} see Bergovin~A.\,K.&&\\
\Avtors{Usov~I.\,A., Satin~Y.\,A., and Zeifman~A.\,I.} On the rate of convergence and limiting 
characteristics for one quasi-birth--death process&\raisebox{-12pt}[0pt][0pt]{3}&\raisebox{-12pt}[0pt][0pt]{49--57}\\
\Avtors{Vakulenko~V.\,V. and Zatsman~I.\,M.} Formalized description of statistical information 
processing in databases&\raisebox{-12pt}[0pt][0pt]{3}&\raisebox{-12pt}[0pt][0pt]{93--99}\\
\Avtors{Vasilyev~N.\,S.} Multiplayers' games compositional structure in the monoidal category of 
binary relations&\raisebox{-12pt}[0pt][0pt]{2}&\raisebox{-12pt}[0pt][0pt]{18--26}\\
\Avtors{Volkanov~D.\,Yu.} see Gorshenin~A.\,K.&&\\
\Avtors{Vorontsov~M.\,O. and~Shestakov~O.\,V.} Mean-square risk of the FDR procedure under 
weak dependence&\raisebox{-12pt}[0pt][0pt]{2}&\raisebox{-12pt}[0pt][0pt]{34--40}\\
\Avtors{Zabezhailo~M.\,I.} see Grusho~A.\,A.&&\\
\Avtors{Zabezhailo~M.\,I.} see Grusho~A.\,A.&&\\
\end{tabular}
}
\pagebreak

\def\leftfootline{\small{\textbf{\thepage}
\hfill INFORMATIKA I EE PRIMENENIYA~--- INFORMATICS AND APPLICATIONS\ \ \ 2023\
\ \ volume~17\ \ \ issue\ 4}
}%
 \def\rightfootline{\small{INFORMATIKA I EE PRIMENENIYA~---
INFORMATICS AND APPLICATIONS\ \ \ 2023\ \ \ volume~17\ \ \ issue\ 4
\hfill \textbf{\thepage}}}

\def\leftkol{2023 AUTHOR INDEX} % ENGLISH ABSTRACTS}

\def\rightkol{2023 AUTHOR INDEX} %ENGLISH ABSTRACTS}


\noindent
{\tabcolsep=3pt
\begin{tabular}{p{395.5pt}cc}
&\textbf{Issue} & \textbf{Page}\\[6pt]
\Avtors{Zabezhailo~M.\,I.} see Grusho~A.\,A.&&\\
\Avtors{Zakharov~V.\,N.} see Frenkel~S.\,L.&&\\
\Avtors{Zakharov~V.\,N.} see Sazontev V.\,V.&&\\
\Avtors{Zatsman~I.\,M.} On the scientific paradigm of informatics: Data, information, and 
knowledge&1&116--125\\
\Avtors{Zatsman I.\,M.} Scientific paradigm of informatics: Classification of domain 
objects&4&\hphantom{9}96--103\\
\Avtors{Zatsman~I.\,M.} Transformation of the Ackoff's hierarchy in the scientific paradigm of 
informatics&3&107--113\\
\Avtors{Zatsman~I.\,M.} see Vakulenko~V.\,V.&&\\
\Avtors{Zeifman~A.\,I.} see Usov~I.\,A.&&\\
\end{tabular}
}

%\thispagestyle{myheadings}
\def\leftfootline{\small{\textbf{\thepage}
\hfill INFORMATIKA I EE PRIMENENIYA~--- INFORMATICS AND APPLICATIONS\ \ \ 2023\
\ \ volume~17\ \ \ issue\ 4}
}%
 \def\rightfootline{\small{INFORMATIKA I EE PRIMENENIYA~---
INFORMATICS AND APPLICATIONS\ \ \ 2023\ \ \ volume~17\ \ \ issue\ 4
\hfill \textbf{\thepage}}}

 \label{end\stat}

\newpage

%
   \vspace*{-46pt}

\begin{center}
\vspace*{4pt}
\mbox{%

\epsfxsize=55mm %112.705
\epsfbox{zhur-2.eps}
}
%\end{center}

\vspace*{10pt} 


%   \begin{center}
\fbox{\large\textbf{Академик Юрий Иванович Журавлёв}}\\[10pt]
\textbf{\large 14.01.1935--14.01.2022}
   \end{center}


   %\vspace*{2.5mm}

   \vspace*{5mm}

   \thispagestyle{empty}

%\

%\vspace*{-12pt}
       


В январе этого года ушел из жизни главный научный сотрудник Федерального исследовательского 
центра <<Информатика и управление>> РАН, председатель Редакционного совета журнала 
<<Информатика и~её применения>> академик Юрий Иванович Журавлёв. В~его лице мировая 
наука потеряла одного из своих ярчайших представителей~--- выдающегося ученого-исследователя 
и~талантливого ученого-организатора.

Юрий Иванович родился в Воронеже в 1935~г.\ в семье ученого и врача. Среднее образование 
получил в школе №\,6 г.~Фрунзе (ныне Бишкек) Киргизской ССР. В~1952~г.\ поступил на 
ме\-ха\-ни\-ко-ма\-те\-ма\-ти\-че\-ский факультет МГУ им.\ М.\,В.~Ломоносова. В~1957~г.\ Юрий Иванович 
защищает диплом и продолжает обучение в аспирантуре Московского университета на кафедре 
вычислительной математики (возглавляемой тогда академиком С.\,Л.~Соболевым). После 
успешной защиты кандидатской диссертации (к.ф.-м.н., 1959 г., научный руководитель~--- 
А.\,А.~Ляпунов, оппоненты~--- чл.-корр.\ А.\,А.~Марков, к.ф.-м.н.\ О.\,Б.~Лупанов) и~до 
окончательного переезда в Москву в 1969~г.\ работал в Институте математики Сибирского 
отделения АН СССР, занимая в нем последовательно должности младшего научного сотрудника, 
заведующего отделом, заведующего отделением, заместителя директора по научной работе. 
В~этот период (1954--1966~гг.)\ им был опубликован цикл работ по решению задач алгебры и 
математической логики, причем полученные результаты применялись для создания эффективных 
программ для ЭВМ, конструирования схем и сетей для обработки информации. Наиболее значимый 
результат этого периода научной работы~--- обоснование нового направления исследований, 
общей теории локальных алгоритмов. В~ней были окончательно объединены топологические 
принципы и теория алгоритмов. Эта теория и легла в основу докторской диссертации Юрия 
Ивановича (д.ф.-м.н., 1965~г.)\ по еще тогда новой научной специальности <<Математическая 
кибернетика>>. Оппонировали ему как специалисты по кибернетике~--- академик 
В.\,М.~Глушков, член-корреспондент А.\,А.~Ляпунов и О.\,Б.~Лупанов, так и про\-фес\-сор-ал\-геб\-раист А.\,Д.~Тайманов. 

В 1969~г.\ Юрий Иванович переезжает в Москву и возглавляет в Вычислительном центре АН 
СССР лабораторию проблем распознавания. Впоследствии он~--- заместитель директора по 
научной работе. Научные интересы этого периода связаны с проблемами классификации или 
распознавания образов. В~1976--1978~гг.\ Юрий Иванович публикует цикл работ по ставшему 
вскоре знаменитым алгебраическому подходу к проблеме синтеза корректных алгоритмов. Эти 
работы определили современное состояние всей проблематики распознавания и многих смежных 
областей прикладной математики и информатики. В~своих основополагающих работах Юрий 
Иванович показал, что можно в явном виде строить экстремальные по качеству алгоритмы для 
решения очень широких классов плохо формализованных задач. 
{\looseness=-1

}





Научные заслуги Юрия Ивановича получили широкое признание. В~1966~г.\ он совместно с 
О.\,Б.~Лупановым и чле\-ном-кор\-рес\-пон\-ден\-том АН СССР С.\,В.~Яблонским были удостоены 
звания лауреата Ленинской премии в~об\-ласти науки и техники. В~1984~г.\ Юрий Иванович 
был избран членом-корреспондентом АН СССР (по специальности <<Информатика>>), 
а~в~1992~г.~--- академиком РАН (по той же специальности).\linebreak\vspace*{-12pt}

\pagebreak

\

\vspace*{-46pt}

\noindent
\begin{floatingfigure}{48mm}
\begin{center}
%\vspace*{6pt}
\mbox{%

\epsfxsize=46mm %112.705
\epsfbox{zhur-3.eps}
}
\end{center}
\vspace*{6pt}
\end{floatingfigure}

 \thispagestyle{empty}

\noindent
В~1986~г.\ за цикл прикладных 
работ ему и ряду его учеников была при\-суж\-де\-на премия Совета Министров СССР. Он являлся 
членом иностранных академий наук, председателем секции <<Прикладная математика
 и~информатика>> Отделения математических наук РАН, председателем экспертного совета ВАК 
России по управ\-ле\-нию и информатике, заслуженным профессором нескольких университетов, 
председателем Российской ассоциации <<Распознавание образов и обработка изображений>>, 
членом исполкома Международной ассоциации IAPR (распознавание образов и обработка 
изображений). Был награжден 8-ю орденами и медалями СССР и России.

Юрий Иванович проводил большую научно-литературную работу, являясь, в том числе, главным 
редактором международных научных журналов и членом редколлегий ряда рецензируемых 
научных журналов. 


Параллельно с активной научной деятельностью Юрий Иванович вел и преподавательскую 
работу. С~1961 по~1969~гг.~--- в Новосибирском государственном университете на кафедре 
алгебры и математической логики, которую возглавлял в то время академик А.\,И.~Мальцев. 
С~1970~г., будучи уже профессором (1967~г.),~--- в Московском физико-техническом институте 
на кафедре академика Н.\,Н.~Моисеева. В~1997~г.\ по предложению ректора МГУ им.\ 
М.\,В.~Ломоносова академика В.\,А.~Садовничего Юрий Иванович организовал на факультете 
Вычислительной математики и кибернетики новую кафедру <<Математические методы 
прогнозирования>>, которой и руководил до конца жизни. В~2008~г.\ ему была присуждена 
премия Совета Министров РФ в области образования. С~1965~г.\ Юрий Иванович периодически 
читал курсы лекций за рубежом, в университетах США, Франции, Финляндии, Швеции, Австрии, 
Польши, Болгарии, ГДР и других стран. Эта работа в существенной степени обеспечила широкое 
международное признание советской и российской науки в области дискретной математики и~распознавания образов. 

%\begin{floatingfigure}{60mm}
\begin{figure}[b]
\begin{center}
\vspace*{-6pt}
\mbox{%

\epsfxsize=112mm %90mm %112.705
\epsfbox{zhur-1.eps}
}
\end{center}
\end{figure}
%\end{floatingfigure}

Понимая важность вопроса воспитания подрастающего поколения для развития науки в стране, 
Юрий Иванович вскоре после защиты первой диссертации включился в работу по подготовке 
научных кадров. Им создана большая научная школа: под руководством Юрия Ивановича 
защищены более 100~кандидатских диссертаций по всевозможным разделам естествознания 
(математике, информатике, медицине, технике, экономике, геологии), не один десяток докторов 
наук. Он воспитал академиков и членов-корреспондентов РАН и академий государств СНГ. 
С~большим вниманием и участием Юрий Иванович относился к развитию научных школ страны 
в~об\-ласти обработки изображений, распознавания образов и компьютерной оптики. 

Для всех коллег и учеников Юрия Ивановича он останется примером замечательного человека, 
та\-лант\-ли\-во\-го педагога и выдающегося, преданного служению науке ученого. 


%\def\stat{cont}
{%\hrule\par
%\vskip 7pt % 7pt
\raggedleft\Large \bf%\baselineskip=3.2ex
А\,В\,Т\,О\,Р\,С\,К\,И\,Й\ \ У\,К\,А\,З\,А\,Т\,Е\,Л\,Ь\ \ З\,А\ \ 2\,0\,1\,0 г. \vskip 17pt
    \hrule
    \par
\vskip 21pt plus 6pt minus 3pt }

\label{st\stat}

\def\tit{\ }

\def\aut{\ }
\def\auf{\ }

\def\leftkol{\ } % ENGLISH ABSTRACTS}

\def\rightkol{\ } %АВТОРСКИЙ УКАЗАТЕЛЬ ЗА 2010 г.} %ENGLISH ABSTRACTS}

\titele{\tit}{\aut}{\auf}{\leftkol}{\rightkol}

\vspace*{-12pt}

{\tabcolsep=3pt
\begin{tabular}{p{388pt}rr}
&\textbf{Выпуск} & \textbf{Стр.}\\[6pt]
\hangindent=23pt\noindent\textbf{Арутюнян~А.\,Р.} Моделирование влияния деформаций отпечатков пальцев на 
точность\linebreak
\vspace*{-12pt}\\
\hspace*{23pt}дактилоскопической идентификации$\dotfill$&1&51\\
\hangindent=23pt\noindent\textbf{Архипов~О.\,П., Зыкова~З.\,П.} Интеграция гетерогенной информации о цветных 
пикселях\linebreak
\vspace*{-12pt}\\
\hspace*{23pt}и их цветовосприятии$\dotfill$&4&15\\
\hangindent=23pt\noindent\textbf{Баранов~С.\,И., Френкель~С.\,Л., Захаров~В.\,Н.} Полуформальная верификация 
цифрового устройства с конвейером, основанная на использовании алгоритмических машин\linebreak
\vspace*{-12pt}\\
\hspace*{23pt}состояния$\dotfill$&4&49\\
\textbf{Бекетова~И.\,В.} см.~Каратеев~С.\,Л.&&\\
\textbf{Белоусов~В.\,В.} см.~Синицын~И.\,Н.&&\\
\hangindent=23pt\noindent\textbf{Бенинг~В.\,Е., Королев~Р.\,А.} О предельном поведении мощностей критериев в 
случае\linebreak
\vspace*{-12pt}\\
\hspace*{23pt}распределения Лапласа$\dotfill$&2&63\\
\hangindent=23pt\noindent\textbf{Бенинг~В.\,Е., Сипина~А.\,В.} Асимптотическое разложение для мощности 
критерия,\linebreak
\vspace*{-12pt}\\
\hspace*{23pt}основанного на выборочной медиане, в случае распределения Лапласа$\dotfill$&1&18\\
\textbf{Бондаренко~А.\,В.} см.~Каратеев~С.\,Л.&&\\
\hangindent=23pt\noindent\textbf{Бородина~А.\,В., Морозов~Е.\,В.} Об оценивании асимптотики вероятности 
большого\linebreak
\vspace*{-12pt}\\
\hspace*{23pt}уклонения стационарной регенеративной очереди с одним прибором$\dotfill$&3&29\\
\hangindent=23pt\noindent\textbf{Бунтман~Н.\,В., Минель~Ж.-Л., Ле~Пезан~Д., Зацман~И.\,М.} Типология и 
компьютерное\linebreak
\vspace*{-12pt}\\
\hspace*{23pt}моделирование трудностей перевода$\dotfill$&3&77\\
\textbf{Визильтер~Ю.\,В.} см.~Каратеев~С.\,Л.&&\\
\hangindent=23pt\noindent\textbf{Гавриленко~С.\,В.} Оценки скорости сходимости распределений случайных сумм с 
безгранично делимыми индексами к нормальному закону$\dotfill$&4&81\\
\hangindent=23pt\noindent\textbf{Григорьева~М.\,Е., Шевцова~И.\,Г.} Уточнение неравенства 
Каца--Берри--Эссеена$\dotfill$&2&75\\
\hangindent=23pt\noindent\textbf{Грушо~А.\,А., Грушо~Н.\,А., Тимонина~Е.\,Е.} Поиск конфликтов в политиках 
безопасности: модель случайных графов$\dotfill$&3&38\\
\textbf{Грушо~Н.\,А.} см.~Грушо~А.\,А.&&\\
\hangindent=23pt\noindent\textbf{Гудков~В.\,Ю.} Математические модели изображения отпечатка пальца на основе 
описания линий$\dotfill$&1&58\\
\textbf{Гуртов~А.\,В.} см.~Лукьяненко~А.\,С.&&\\
\textbf{Желтов~С.\,Ю.} см.~Каратеев~С.\,Л.&&\\
\hangindent=23pt\noindent\textbf{Захаров~А.\,А., Серебряков~В.\,А.} Система управления электронной библиотекой 
LibMeta$\dotfill$&4&2\\
\textbf{Захаров~В.\,Н.} см.~Баранов~С.\,И.&&\\
\textbf{Захарова~Т.\,В.} см.~Матвеева~С.\,С.&&\\
\hangindent=23pt\noindent\textbf{Зацаринный~А.\,А., Чупраков~К.\,Г.} Некоторые аспекты выбора технологии для 
постро-\linebreak
\vspace*{-12pt}\\
\hspace*{23pt}ения систем отображения информации ситуационного центра$\dotfill$&3&59\\
\textbf{Зацман~И.\,М.} см.~Бунтман~Н.\,В.&&\\
\hangindent=23pt\noindent\textbf{Зейфман~А.\,И., Коротышева~А.\,В., Сатин~Я.\,А., Шоргин~С.\,Я.} Об 
устойчивости нестаци-\linebreak
\vspace*{-12pt}\\
\hspace*{23pt}онарных систем обслуживания с катастрофами$\dotfill$&3&9\\
\textbf{Зыкова~З.\,П.} см.~Архипов~О.\,П.&&\\
\hangindent=23pt\noindent\textbf{Илюшин~Г.\,Я., Соколов~И.\,А.} Организация управляемого доступа пользователей 
к\linebreak
\vspace*{-12pt}\\
\hspace*{23pt}разнородным ведомственным информационным ресурсам$\dotfill$&1&24\\
\hangindent=23pt\noindent\textbf{Кавагучи~Ю., Ульянов~В.\,В., Фуджикоши~Я.} Приближения для статистик, 
описывающих\linebreak
\vspace*{-12pt}\\
\hspace*{23pt}геометрические свойства данных большой размерности, с оценками 
ошибок$\dotfill$&1&12\\
\hangindent=23pt\noindent\textbf{Каратеев~С.\,Л., Бекетова~И.\,В., Ососков~М.\,В., Князь~В.\,А., 
Визильтер~Ю.\,В., Бондаренко~А.\,В., Желтов~С.\,Ю.} Автоматизированный контроль 
качества цифровых\linebreak
\vspace*{-12pt}\\
\hspace*{23pt}изображений для персональных документов$\dotfill$&1&65\\
\end{tabular}
}

\pagebreak

\def\leftkol{АВТОРСКИЙ УКАЗАТЕЛЬ ЗА 2010 г.} % ENGLISH ABSTRACTS}

\def\rightkol{АВТОРСКИЙ УКАЗАТЕЛЬ ЗА 2010 г.} %ENGLISH ABSTRACTS}

{\tabcolsep=3pt
\begin{tabular}{p{388pt}rr}
&\textbf{Выпуск} & \textbf{Стр.}\\[3pt]
\hangindent=23pt\noindent\textbf{Козеренко~Е.\,Б.} Лингвистические фильтры в статистических моделях машинного\linebreak
\vspace*{-12pt}\\
\hspace*{23pt}перевода$\dotfill$&2&83\\
\hangindent=23pt\noindent\textbf{Козеренко~Е.\,Б., Кузнецов~И.\,П.} Когнитивно-лингвистические представления в 
систе-\linebreak
\vspace*{-12pt}\\
\hspace*{23pt}мах обработки текстов$\dotfill$&3&69\\
\textbf{Князь~В.\,А.} см.~Каратеев~С.\,Л.&&\\
\hangindent=23pt\noindent\textbf{Колесников~А.\,В., Солдатов~С.\,А.} Алгоритм координации для гибридной 
интеллектуальной системы решения сложной задачи оперативно-производственного\linebreak
\vspace*{-12pt}\\
\hspace*{23pt}планирования$\dotfill$&4&61\\
\hangindent=23pt\noindent\textbf{Коновалов~М.\,Г.} О планировании потоков в системах вычислительных 
ресурсов$\dotfill$&2&3\\
\textbf{Конушин~А.\,С.} см.~Конушин~В.\,С.&&\\
\hangindent=23pt\noindent\textbf{Конушин~В.\,С., Кривовязь~Г.\,Р., Конушин~А.\,С.} Алгоритм распознавания людей 
в видео-\linebreak
\vspace*{-12pt}\\
\hspace*{23pt}последовательности по одежде$\dotfill$&1&74\\
\textbf{Корепанов~Э.\, Р.} см.~Синицын~И.\,Н.&&\\
\textbf{Королев~В.\,Ю.} см.~Соколов~И.\,А.&&\\
\textbf{Королев~Р.\,А.} см.~Бенинг~В.\,Е.&&\\
\textbf{Коротышева~А.\,В.} см.~Зейфман~А.\,И.&&\\
\hangindent=23pt\noindent\textbf{Кривенко~М.\,П.} Непараметрическое оценивание элементов байесовского 
клас\-си-\linebreak
\vspace*{-12pt}\\
\hspace*{23pt}фикатора$\dotfill$&2&13\\
\textbf{Кривовязь~Г.\,Р.} см.~Конушин~В.\,С.&&\\
\textbf{Крылов~А.\,С.} см.~Павельева~Е.\,А.&&\\
\hangindent=23pt\noindent\textbf{Крылов~В.\,А.} Моделирование и классификация многоканальных дистанционных\linebreak
\vspace*{-12pt}\\
\hspace*{23pt}изображений с использованием копул$\dotfill$&4&34\\
\hangindent=23pt\noindent\textbf{Крючин~О.\,В.} Разработка параллельных эвристических алгоритмов подбора 
весовых\linebreak
\vspace*{-12pt}\\
\hspace*{23pt}коэффициентов искусственной нейтронной сети$\dotfill$&2&53\\
\hangindent=23pt\noindent\textbf{Кудрявцев~А.\,А., Шоргин~С.\,Я.} Байесовские модели массового обслуживания и 
надеж-\linebreak
\vspace*{-12pt}\\
\hspace*{23pt}ности: характеристики среднего числа заявок в системе $M\vert M \vert 1\vert 
\infty$$\dotfill$&3&16\\
\hangindent=23pt\noindent\textbf{Кузнецов~А.\,А.} Связь между временными и структурно-топологическими 
характери-\linebreak
\vspace*{-12pt}\\
\hspace*{23pt}стиками диаграмм ритма сердца здоровых людей$\dotfill$&4&39\\
\textbf{Кузнецов~И.\,П.} см.~Козеренко~Е.\,Б.&&\\
\textbf{Ле~Пезан~Д.} см.~Бунтман~Н.\,В.&&\\
\hangindent=23pt\noindent\textbf{Лукьяненко~А.\,С., Морозов~Е.\,В., Гуртов~А.\,В.} Анализ сетевого протокола с общей 
функ-\linebreak
\vspace*{-12pt}\\
\hspace*{23pt}цией расширения окна передачи сообщения при конфликтах$\dotfill$&2&46\\
\hangindent=23pt\noindent\textbf{Лямин~О.\,О.} О предельном поведении мощностей критериев в случае обобщенного\linebreak
\vspace*{-12pt}\\
\hspace*{23pt}распределения Лапласа$\dotfill$&3&47\\
\hangindent=23pt\noindent\textbf{Маркин~А.\,В., Шестаков~О.\,В.} Асимптотики оценки риска при пороговой 
обработке\linebreak
\vspace*{-12pt}\\
\hspace*{23pt}вейвлет-вейглет коэффициентов в задаче томографии$\dotfill$&2&36\\
\hangindent=23pt\noindent\textbf{Матвеева~С.\,С., Захарова~Т.\,В.} Сети массового обслуживания с наименьшей 
длиной\linebreak
\vspace*{-12pt}\\
\hspace*{23pt}очереди$\dotfill$&3&22\\
\hangindent=23pt\noindent\textbf{Матюшенко~С.\,И.} Стационарные характеристики двухканальной системы 
обслужива-\linebreak
\vspace*{-12pt}\\
\hspace*{23pt}ния с переупорядочиванием заявок и распределениями фазового типа$\dotfill$&4&68\\
\textbf{Минель~Ж.-Л.} см.~Бунтман~Н.\,В.&&\\
\textbf{Морозов~Е.\,В.} см.~Бородина~А.\,В.&&\\
\textbf{Морозов~Е.\,В.} см.~Лукьяненко~А.\,С.&&\\
\textbf{Ососков~М.\,В.} см.~Каратеев~С.\,Л.&&\\
\hangindent=23pt\noindent\textbf{Павельева~Е.\,А., Крылов~А.\,С.} Поиск и анализ ключевых точек радужной 
оболочки\linebreak
\vspace*{-12pt}\\
\hspace*{23pt}глаза методом преобразования Эрмита$\dotfill$&1&79\\
\textbf{Печинкин~А.\,В.} см.~Френкель~С.\,Л.,&&\\
\hangindent=23pt\noindent\textbf{Протасов~В.\,И.} Составление субъективного портрета с использованием 
эволюционно-\linebreak
\vspace*{-12pt}\\
\hspace*{23pt}го морфинга и квалиметрия метода$\dotfill$&1&83\\
\hangindent=23pt\noindent\textbf{Рудаков~К.\,В., Торшин~И.\,Ю.} Вопросы разрешимости задачи распознавания 
вторичной\linebreak
\vspace*{-12pt}\\
\hspace*{23pt}структуры белка$\dotfill$&2&25\\
\textbf{Сатин~Я.\,А.} см.~Зейфман~А.\,И.&&\\
\hangindent=23pt\noindent\textbf{Сейфуль-Мулюков~Р.\,Б.} Нефть как носитель информации о своем 
происхождении,\linebreak
\vspace*{-12pt}\\
\hspace*{23pt}структуре и эволюции$\dotfill$&1&41\\
\end{tabular}
}

{\tabcolsep=3pt
\begin{tabular}{p{388pt}rr}
&\textbf{Выпуск} & \textbf{Стр.}\\[6pt]
\textbf{Семендяев~Н.\,Н.} см.~Синицын~И.\,Н.&&\\
\textbf{Серебряков~В.\,А.} см.~Захаров~А.\,А.&&\\
\textbf{Синицын~В.\,И.} см.~Синицын~И.\,Н.&&\\
\hangindent=23pt\noindent\textbf{Синицын~И.\,Н., Синицын~В.\,И., Корепанов~Э.\, Р., Белоусов~В.\,В., 
Семендяев~Н.\,Н.} Оперативное построение информационных моделей движения полюса 
Земли\linebreak
\vspace*{-12pt}\\
\hspace*{23pt}методами линейных и линеаризованных фильтров$\dotfill$&1&2\\
\textbf{Сипина~А.\,В.} см.~Бенинг~В.\,Е.&&\\
\hangindent=23pt\noindent\textbf{Соколов~И.\,А.} О работах заслуженного деятеля науки Российской Федерации 
И.\,Н.~Синицына в области информационных технологий и автоматизации (к 70-летию\linebreak
\vspace*{-12pt}\\
\hspace*{23pt}со дня рождения)$\dotfill$&3&84\\
\textbf{Соколов~И.\,А.} см.~Илюшин~Г.\,Я.&&\\
\hangindent=23pt\noindent\textbf{Соколов~И.\,А., Королев~В.\,Ю.} Предисловие$\dotfill$&2&2\\
\textbf{Солдатов~С.\,А.} см.~Колесников~А.\,В.&&\\
\hangindent=23pt\noindent\textbf{Степанов~С.\,Ю.} Использование координатного метода фрагментации 
коммутаторной\linebreak
\vspace*{-12pt}\\
\hspace*{23pt}нейронной сети для сокращения трафика$\dotfill$&2&57\\
\textbf{Тимонина~Е.\,Е.} см.~Грушо~А.\,А.&&\\
\textbf{Торшин~И.\,Ю.} см.~Рудаков~К.\,В.&&\\
\textbf{Ульянов~В.\,В.} см.~Кавагучи~Ю.&&\\
\textbf{Фазекаш~И.} см.~Чупрунов~А.\,Н.&&\\
\textbf{Френкель~С.\,Л.} см.~Баранов~С.\,И.&&\\
\hangindent=23pt\noindent\textbf{Френкель~С.\,Л., Печинкин~А.\,В.} Оценка времени самовосстановления в 
цифровых\linebreak
\vspace*{-12pt}\\
\hspace*{23pt}системах после сбоев, вызываемых переходными помехами$\dotfill$&3&2\\
\textbf{Фуджикоши~Я.} см.~Кавагучи~Ю.&&\\
\hangindent=23pt\noindent\textbf{Цискаридзе~А.\,К.} Математическая модель и метод восстановления позы человека 
по\linebreak
\vspace*{-12pt}\\
\hspace*{23pt}стереопаре силуэтных изображений$\dotfill$&4&27\\
\hangindent=23pt\noindent\textbf{Чупраков~К.\,Г.} К вопросу о размещении коллективных средств отображения в 
ситуа-\linebreak
\vspace*{-12pt}\\
\hspace*{23pt}ционном зале с заданными параметрами$\dotfill$&4&89\\
\textbf{Чупраков~К.\,Г.} см.~Зацаринный~А.\,А.&&\\
\hangindent=23pt\noindent\textbf{Чупрунов~А.\,Н., Фазекаш~И.} Законы повторного логарифма для числа 
безошибочных\linebreak
\vspace*{-12pt}\\
\hspace*{23pt}блоков при помехоустойчивом кодировании$\dotfill$&3&42\\
\textbf{Шевцова~И.\,Г.} см.~Григорьева~М.\,Е.&&\\
\hangindent=23pt\noindent\textbf{Шестаков~О.\,В.} Аппроксимация распределения оценки риска пороговой 
обработки вейвлет-коэффициентов нормальным распределением при использовании 
выбо-\linebreak
\vspace*{-12pt}\\
\hspace*{23pt}рочной дисперсии$\dotfill$&4&73\\
\textbf{Шестаков~О.\,В.} см.~Маркин~А.\,В.&&\\
\textbf{Шоргин~С.\,Я.} см.~Зейфман~А.\,И.&&\\
\textbf{Шоргин~С.\,Я.} см.~Кудрявцев~А.\,А.&&\\
\end{tabular}
}

%\thispagestyle{myheadings}
\def\leftfootline{\small{\textbf{\thepage}
\hfill ИНФОРМАТИКА И ЕЁ ПРИМЕНЕНИЯ\ \ \ том~4\ \ \ выпуск~4\ \ \ 2010}
}%
 \def\rightfootline{\small{ИНФОРМАТИКА И ЕЁ ПРИМЕНЕНИЯ\ \ \ том~4\ \ \ выпуск~4\ \ \ 2010
 \hfill \textbf{\thepage}}}
 \label{end\stat}
%
%Том 10 Выпуск 1-4 Год 2016

\def\stat{cont-e}
{%\hrule\par
%\vskip 7pt % 7pt
\raggedleft\Large \bf%\baselineskip=3.2ex
2\,0\,1\,6\ \ A\,U\,T\,H\,O\,R\ \ I\,N\,D\,E\,X \vskip 17pt
 \hrule
 \par
\vskip 21pt plus 6pt minus 3pt }

\label{st\stat}

\def\tit{\ }

\def\aut{\ }
\def\auf{\ }

\def\leftkol{\ } %2016 AUTHOR INDEX} % ENGLISH ABSTRACTS}

\def\rightkol{\ } %2016 AUTHOR INDEX} %ENGLISH ABSTRACTS}

\titele{\tit}{\aut}{\auf}{\leftkol}{\rightkol}

\def\leftfootline{\small{\textbf{\thepage}
\hfill INFORMATIKA I EE PRIMENENIYA~--- INFORMATICS AND APPLICATIONS\ \ \ 2016\
\ \ volume~10\ \ \ issue\ 4}
}%
 \def\rightfootline{\small{INFORMATIKA I EE PRIMENENIYA~--- INFORMATICS AND APPLICATIONS\ \ \ 2016\ \ \ volume~10\ \ \ issue\ 4
\hfill \textbf{\thepage}}}

\vspace*{-12pt}
\vspace*{-18pt}

{\tabcolsep=2.8pt
\begin{tabular}{p{382pt}cc}
&\textbf{Issue} & \textbf{Page}\\[6pt]
\Avtors{Agalarov~M.\,Ya.} see~Agalarov~Ya.\,M.&&\\
\Avtors{Agalarov~Ya.\,M., Agalarov~M.\,Ya., and
Shorgin~V.\,S.} About the optimal threshold of queue\linebreak
\\[-12pt]
\hspace*{23pt}length in a~particular problem of profit maximization
in the $M/G/1$ queuing system&2&70--79\\
\Avtors{Alexeyevsky~D.\,A.} BioNLP ontology extraction from 
a~restricted language corpus with\linebreak
\\[-12pt]
\hspace*{23pt}context-free grammars&1&119--128\\
\Avtors{Andreev~S.\,D.} see~Gaidamaka~Yu.\,V.&&\\
\Avtors{Andreev~S.\,D.} see~Ometov~A.\,Ya.&&\\
\Avtors{Arkhipov~O.\,P., Arkhipov~P.\,O., and Sidorkin~I.\,I.} The
option to create a~local coordinate\linebreak
\\[-12pt]
\hspace*{23pt}system for synchronization of selected images&3&91--97\\
\Avtors{Arkhipov~P.\,O.} see~Arkhipov~O.\,P.&&\\
\Avtors{Belousov~V.\,V.} see~Shnurkov~P.\,V.&&\\
\Avtors{Belousov~V.\,V.} see~Shnurkov~P.\,V.&&\\
\Avtors{Bening~V.\,E.} Calculation of~the~asymptotic deficiency
of~some statistical procedures based\linebreak
\\[-12pt]
\hspace*{23pt}on~samples with~random sizes&4&34--45\\
\Avtors{Borisov~A.\,V., Bosov~A.\,V., and Miller~G.\,B.} Modeling and
monitoring of VoIP connection&2&\hphantom{1}2--13\\
\Avtors{Bosov~A.\,V.} see~Borisov~A.\,V.&&\\
\Avtors{Briukhov~D.\,O.} see~Stupnikov~S.\,A.&&\\
\Avtors{Callaos~N.\,K.\ and Seyful-Mulyukov~R.\,B.} Complexity and
its information content&1&129--139\\
\Avtors{Chertok~A.\,V., Kadaner~A.\,I., Khazeeva~G.\,T., and
Sokolov~I.\,A.} Regime switching detection\linebreak
\\[-12pt]
\hspace*{23pt}for~the~Levy driven
Ornstein--Uhlenbeck process using CUSUM methods&4&46--56\\
\Avtors{Chichagov~V.\,V.} Asymptotic expansions of mean absolute
error of uniformly minimum variance unbiased and maximum likelihood
estimators on the one-parameter exponential\linebreak
\\[-12pt]
\hspace*{23pt}family model of lattice distributions&3&66--76\\
\Avtors{Danishevsky~V.\,I.} see~Kolesnikov A.\,V.&&\\
\Avtors{Fazliev~A.\,Z.} see~Kalinichenko~L.\,A.&&\\
\Avtors{Fedoseev~A.\,A.} What is behind the concept of ``knowledge in
small packages''&3&105--110\\
\Avtors{Gaidamaka~Yu.\,V., Andreev~S.\,D., Sopin~E.\,S.,
Samouylov~K.\,E., and Shorgin~S.\,Ya.} Interference analysis
of~the~device-to-device communications model with~regard to~a~signal\linebreak
\\[-12pt]
\hspace*{23pt}propagation environment&4&\hphantom{1}2--10\\
\Avtors{Gasilov~A.\,V.} see~Yakovlev~O.\,A.&&\\
\Avtors{Goncharov~A.\,V.\ and Strijov~V.\,V.} Metric time series
classification using weighted dynamic\linebreak
\\[-12pt]
\hspace*{23pt}warping relative to centroids of classes&2&36--47\\
\Avtors{Gordov~E.\,P.} see~Kalinichenko~L.\,A.&&\\
\Avtors{Gorshenin~A.\,K.} Concept of online service for stochastic
modeling of real processes&1&72--81\\
\Avtors{Gorshenin~A.\,K.} see~Shnurkov~P.\,V.&&\\
\Avtors{Gorshenin~A.\,K.} see~Shnurkov~P.\,V.&&\\
\Avtors{Grusho~A.\,A., Grusho~N.\,A., Zabezhailo~M.\,I., and
Timonina~E.\,E.} Integration of statistical and\linebreak
\\[-12pt]
\hspace*{23pt}deterministic methods for
analysis of information security&3&2--8\\
\Avtors{Grusho~A.\,A., Zabezhailo~M.\,I., and Zatsarinny~A.\,A.} On
the advanced procedure to reduce\linebreak
\\[-12pt]
\hspace*{23pt}calculation of Galois closures&4&\hphantom{1}96--104\\
\Avtors{Grusho~N.\,A.} see~Grusho~A.\,A.&&\\
\Avtors{Havanskov~V.\,A.} see~Minin~V.\,A.&&\\
\Avtors{Inkova~O.\,Yu.} see~Zatsman~I.\,M.&&\\
\Avtors{Isachenko~R.\,V.\ and Strijov~V.\,V.} Metric learning in
multiclass time series classification\linebreak
\\[-12pt]
\hspace*{23pt}problem&2&48--57\\
\end{tabular}
}
\pagebreak

\def\leftfootline{\small{\textbf{\thepage}
\hfill INFORMATIKA I EE PRIMENENIYA~--- INFORMATICS AND APPLICATIONS\ \ \ 2016\
\ \ volume~10\ \ \ issue\ 4}
}%
 \def\rightfootline{\small{INFORMATIKA I EE PRIMENENIYA~---
INFORMATICS AND APPLICATIONS\ \ \ 2016\ \ \ volume~10\ \ \ issue\ 4
\hfill \textbf{\thepage}}}

\def\leftkol{2016 AUTHOR INDEX} % ENGLISH ABSTRACTS}

\def\rightkol{2016 AUTHOR INDEX} %ENGLISH ABSTRACTS}


{\tabcolsep=2.83pt
\begin{tabular}{p{382pt}cc}
&\textbf{Issue} & \textbf{Page}\\[6pt]
\Avtors{Kadaner~A.\,I.} see~Chertok~A.\,V.&&\\[.255pt]
\Avtors{Kalinichenko~L.\,A., Volnova~A.\,A., Gordov~E.\,P.,
Kiselyova~N.\,N., Kovaleva~D.\,A., Malkov~O.\,Yu., Okladnikov~I.\,G.,
Podkolodnyy~N.\,L., Pozanenko~A.\,S., Ponomareva~N.\,V.,
Stupnikov~S.\,A.,} \textbf{and Fazliev~A.\,Z.} Data access challenges for data
intensive\linebreak
\\[-12pt]
\hspace*{23pt}research in Russia&1& 2--22\\[.255pt]
\Avtors{Karasikov~M.\,E.\ and Strijov~V.\,V.} Feature-based
time-series classification&4&121--131\\[.255pt]
\Avtors{Khazeeva~G.\,T.} see~Chertok~A.\,V.&&\\[.255pt]
\Avtors{Khokhlov~Yu.\,S.} Multivariate fractional Levy motion and its
applications&2&\hphantom{1}98--106\\[.255pt]
\Avtors{Kirikov~I.\,A., Kolesnikov~A.\,V., Listopad~S.\,V., and
Rumovskaya~S.\,B.} Fine-grained hybrid\linebreak
\\[-12pt]
\hspace*{23pt}intelligent systems. Part 2:
Bidirectional hybridization&1&\hphantom{1}96--105\\[.255pt]
\Avtors{Kirikov~I.\,A., Kolesnikov~A.\,V., Listopad~S.\,V., and
Rumovskaya~S.\,B.} ``Virtual council''~---\linebreak
\\[-12pt]
\hspace*{23pt}source environment
supporting complex diagnostic decision making&3&81--90\\[.255pt]
\Avtors{Kiselyova~N.\,N.} see~Kalinichenko~L.\,A.&&\\[.255pt]
\Avtors{Kolesnikov A.\,V., Listopad~S.\,V., Rumovskaya~S.\,B., and
Danishevsky~V.\,I.} Informal axiomatic\linebreak
\\[-12pt]
\hspace*{23pt}theory of~the~role visual models&4&114--120\\[.255pt]
\Avtors{Kolesnikov~A.\,V.} see~Kirikov~I.\,A.&&\\[.255pt]
\Avtors{Kolesnikov~A.\,V.} see~Kirikov~I.\,A.&&\\[.255pt]
\Avtors{Kolin~K.\,K.} Humanitarian aspects of information
security&3&111--121\\[.255pt]
\Avtors{Konovalov~M.\,G.\ and Razumchik~R.\,V.} Dispatching
to~two parallel nonobservable queues using\linebreak
\\[-12pt]
\hspace*{23pt}only static
information&4&57--67\\[.255pt]
\Avtors{Korchagin~A.\,Yu.} see~Korolev~V.\,Yu.&&\\[.255pt]
\Avtors{Korchagin~A.\,Yu.} see~Korolev~V.\,Yu.&&\\[.255pt]
\Avtors{Korepanov~E.\,R.} see~Sinitsyn~I.\,N.&&\\[.255pt]
\Avtors{Korepanov~E.\,R.} see~Sinitsyn~I.\,N.&&\\[.255pt]
\Avtors{Korolev~V.\,Yu., Korchagin~A.\,Yu., and Zeifman~A.\,I.} The
Poisson theorem for Bernoulli trials\linebreak
\\[-12pt]
\hspace*{23pt}with~a~random probability
of~success and~a~discrete analog of~the~Weibull distribution&4&11--20\\[.255pt]
\Avtors{Korolev~V.\,Yu., Zeifman~A.\,I., and Korchagin~A.\,Yu.}
Asymmetric Linnik distributions as~limit\linebreak
\\[-12pt]
\hspace*{23pt}laws for~random sums
of~independent random variables with~finite variances&4&21--33\\[.255pt]
\Avtors{Koucheryavy~E.\,A.} see~Ometov~A.\,Ya.&&\\[.255pt]
\Avtors{Kovaleva~D.\,A.} see~Kalinichenko~L.\,A.&&\\[.255pt]
\Avtors{Kovalyov~S.\,P.} Metaprogramming to increase
manufacturability of large-scale software-\linebreak
\\[-12pt]
\hspace*{23pt}intensive systems&1&56--66\\[.255pt]
\Avtors{Krivenko~M.\,P.} Significance tests of feature selection for
classification&3&32--40\\[.255pt]
\Avtors{Kruzhkov~M.\,G.} see~Zalizniak~Anna~A.&&\\[.255pt]
\Avtors{Kruzhkov~M.\,G.} see~Zatsman~I.\,M.&&\\[.255pt]
\Avtors{Kudryavtsev~A.\,A.} Bayesian queueing and reliability models:
\textit{A~priori} distributions with\linebreak
\\[-12pt]
\hspace*{23pt}compact support&1&67--71\\[.255pt]
\Avtors{Kudryavtsev~A.\,A.} Characteristics dependent on the balance
coefficient in Bayesian models\linebreak
\\[-12pt]
\hspace*{23pt}with compact support of \textit{a priori}
distributions&3&77--80\\[.255pt]
\Avtors{Kudryavtsev~A.\,A.\ and Palionnaia~S.\,I.} Bayesian recurrent
model of reliability growth:\linebreak
\\[-12pt]
\hspace*{23pt}Parabolic distribution of parameters&2&80--83\\[.255pt]
\Avtors{Kudryavtsev~A.\,A.\ and Titova~A.\,I.} Bayesian queuing
and~reliability models: Degenerate-\linebreak
\\[-12pt]
\hspace*{23pt}Weibull case&4&68--71\\[.255pt]
\Avtors{Leontyev~N.\,D.\ and Ushakov~V.\,G.} Analysis of a queueing
system with autoregressive arrivals\linebreak
\\[-12pt]
\hspace*{23pt}and nonpreemptive priority&3&15--22\\[.255pt]
\Avtors{Listopad~S.\,V.} see~Kirikov~I.\,A.&&\\[.255pt]
\Avtors{Listopad~S.\,V.} see~Kirikov~I.\,A.&&\\[.255pt]
\Avtors{Listopad~S.\,V.} see~Kolesnikov A.\,V.&&\\[.255pt]
\Avtors{Malkov~O.\,Yu.} see~Kalinichenko~L.\,A.&&\\[.255pt]
\Avtors{Markov~A.\,S., Monakhov~M.\,M., and
Ulyanov~V.\,V.} Generalized Cornish--Fisher expansions\linebreak
\\[-12pt]
\hspace*{23pt}for distributions of statistics based on samples
of random size&2&84--91\\[.255pt]
\Avtors{Melnikov~A.\,K.\ and Ronzhin~A.\,F.} Generalized statistical
method of~text analysis based\linebreak
\\[-12pt]
\hspace*{23pt}on~calculation of~probability distributions
of~statistical values&4&89--95\\
\end{tabular}
}
\pagebreak

\def\leftfootline{\small{\textbf{\thepage}
\hfill INFORMATIKA I EE PRIMENENIYA~--- INFORMATICS AND APPLICATIONS\ \ \ 2016\
\ \ volume~10\ \ \ issue\ 4}
}%
 \def\rightfootline{\small{INFORMATIKA I EE PRIMENENIYA~---
INFORMATICS AND APPLICATIONS\ \ \ 2016\ \ \ volume~10\ \ \ issue\ 4
\hfill \textbf{\thepage}}}

\def\leftkol{2016 AUTHOR INDEX} % ENGLISH ABSTRACTS}

\def\rightkol{2016 AUTHOR INDEX} %ENGLISH ABSTRACTS}


{\tabcolsep=3pt
\begin{tabular}{p{381pt}cc}
&\textbf{Issue} & \textbf{Page}\\[6pt]
\Avtors{Meykhanadzhyan~L.\,A.} Stationary characteristics of the finite
capacity queueing system with\linebreak
\\[-12pt]
\hspace*{23pt}inverse service order and generalized
probabilistic priority&2&123--131\\[.23pt]
\Avtors{Miller~G.\,B.} see~Borisov~A.\,V.&&\\[.23pt]
\Avtors{Minin~V.\,A., Zatsman~I.\,M., Havanskov~V.\,A., and
Shubnikov~S.\,K.} Intensity of citation of scientific publications in
inventions on information and computer technologies patented\linebreak
\\[-12pt]
\hspace*{23pt}in Russia by domestic and foreign applicants&2&107--122\\[.23pt]
\Avtors{Monakhov~M.\,M.} see~Markov~A.\,S.&&\\[.23pt]
\Avtors{Naumov~V.\,A.\ and Samouylov~K.\,E.} On relationship
between queuing systems with resources\linebreak
\\[-12pt]
\hspace*{23pt}and Erlang networks&3&\hphantom{1}9--14\\[.23pt]
\Avtors{Okladnikov~I.\,G.} see~Kalinichenko~L.\,A.&&\\[.23pt]
\Avtors{Ometov~A.\,Ya., Andreev~S.\,D., Turlikov~A.\,M., and
Koucheryavy~E.\,A.} Performance analysis of\linebreak
\\[-12pt]
\hspace*{23pt}a wireless data
aggregation system with contention for contemporary sensor
networks&3&23--31\\[.23pt]
\Avtors{Palionnaia~S.\,I.} see~Kudryavtsev~A.\,A.&&\\[.23pt]
\Avtors{Podkolodnyy~N.\,L.} see~Kalinichenko~L.\,A.&&\\[.23pt]
\Avtors{Ponomareva~N.\,V.} see~Kalinichenko~L.\,A.&&\\[.23pt]
\Avtors{Popkova~N.\,A.} see~Zatsman~I.\,M.&&\\[.23pt]
\Avtors{Pozanenko~A.\,S.} see~Kalinichenko~L.\,A.&&\\[.23pt]
\Avtors{Razumchik~R.\,V.} see~Konovalov~M.\,G.&&\\[.23pt]
\Avtors{Ronzhin~A.\,F.} see~Melnikov~A.\,K.&&\\[.23pt]
\Avtors{Rumovskaya~S.\,B.} see~Kirikov~I.\,A.&&\\[.23pt]
\Avtors{Rumovskaya~S.\,B.} see~Kirikov~I.\,A.&&\\[.23pt]
\Avtors{Rumovskaya~S.\,B.} see~Kolesnikov A.\,V.&&\\[.23pt]
\Avtors{Samouylov~K.\,E.} see~Gaidamaka~Yu.\,V.&&\\[.23pt]
\Avtors{Samouylov~K.\,E.} see~Naumov~V.\,A.&&\\[.23pt]
\Avtors{Serebryanskii~S.\,M.} see~Tyrsin~A.\,N.&&\\[.23pt]
\Avtors{Seyful-Mulyukov~R.\,B.} see~Callaos~N.\,K.&&\\[.23pt]
\Avtors{Shestakov~O.\,V.} Statistical properties of the denoising method
based on the stabilized hard\linebreak
\\[-12pt]
\hspace*{23pt}thresholding&2&65--69\\[.23pt]
\Avtors{Shestakov~O.\,V.} The strong law of large numbers for the risk
estimate in the problem of\linebreak
\\[-12pt]
\hspace*{23pt}tomographic image reconstruction from
projections with a correlated noise&3&41--45\\[.23pt]
\Avtors{Shestakov~O.\,V.} see~Zakharova~T.\,V.&&\\[.23pt]
\Avtors{Shnurkov~P.\,V., Gorshenin~A.\,K., and Belousov~V.\,V.}
Analytical solution of~the~optimal control\linebreak
\\[-12pt]
\hspace*{23pt}task of~a~semi-Markov
process with~finite set of~states&4&72--88\\[.23pt]
\Avtors{Shnurkov~P.\,V., Zasypko~V.\,V., Belousov~V.\,V., and
Gorshenin~A.\,K.} Development of the algorithm of numerical solution
of the optimal investment control problem\linebreak
\\[-12pt]
\hspace*{23pt}in the closed dynamical model of three-sector economy&1&82--95\\[.23pt]
\Avtors{Shorgin~S.\,Ya.} see~Gaidamaka~Yu.\,V.&&\\[.23pt]
\Avtors{Shorgin~V.\,S.} see~Agalarov~Ya.\,M.&&\\[.23pt]
\Avtors{Shubnikov~S.\,K.} see~Minin~V.\,A.&&\\[.23pt]
\Avtors{Sidorkin~I.\,I.} see~Arkhipov~O.\,P.&&\\[.23pt]
\Avtors{Sinitsyn~I.\,N.} Analytical modeling of processes in stochastic
systems with complex fractional\linebreak
\\[-12pt]
\hspace*{23pt}order Bessel nonlinearities&3&55--65\\[.23pt]
\Avtors{Sinitsyn~I.\,N.} Orthogonal supoptimal filters for nonlinear
stochastic systems on manifolds&1&34--44\\[.23pt]
\Avtors{Sinitsyn~I.\,N.\ and Korepanov~E.\,R.} Normal Pugachev
conditionally-optimal filters and extra-\linebreak
\\[-12pt]
\hspace*{23pt}polators for state linear stochastic systems&2&14--23\\[.23pt]
\Avtors{Sinitsyn~I.\,N.\ and Sinitsyn~V.\,I.} Analytical modeling of
distributions in stochastic systems on\linebreak
\\[-12pt]
\hspace*{23pt}manifolds based on ellipsoidal approximation&1&45--55\\[.23pt]
\Avtors{Sinitsyn~I.\,N., Sinitsyn~V.\,I., and
Korepanov~E.\,R.} Ellipsoidal suboptimal filters for nonlinear\linebreak
\\[-12pt]
\hspace*{23pt}stochastic systems on manifolds&2&24--35\\[.23pt]
\Avtors{Sinitsyn~V.\,I.} see~Sinitsyn~I.\,N.&&\\[.23pt]
\Avtors{Sinitsyn~V.\,I.} see~Sinitsyn~I.\,N.&&\\[.23pt]
\Avtors{Skvortsov~N.\,A.} see~Stupnikov~S.\,A.&&\\[.23pt]
\Avtors{Sokolov~I.\,A.} see~Chertok~A.\,V.&&\\
\end{tabular}
}
\pagebreak

\def\leftfootline{\small{\textbf{\thepage}
\hfill INFORMATIKA I EE PRIMENENIYA~--- INFORMATICS AND APPLICATIONS\ \ \ 2016\
\ \ volume~10\ \ \ issue\ 4}
}%
 \def\rightfootline{\small{INFORMATIKA I EE PRIMENENIYA~---
INFORMATICS AND APPLICATIONS\ \ \ 2016\ \ \ volume~10\ \ \ issue\ 4
\hfill \textbf{\thepage}}}

\def\leftkol{2016 AUTHOR INDEX} % ENGLISH ABSTRACTS}

\def\rightkol{2016 AUTHOR INDEX} %ENGLISH ABSTRACTS}


{\tabcolsep=3pt
\begin{tabular}{p{382pt}cc}
&\textbf{Issue} & \textbf{Page}\\[6pt]
\Avtors{Sopin~E.\,S.} see~Gaidamaka~Yu.\,V.&&\\
\Avtors{Strijov~V.\,V.} see~Goncharov~A.\,V.&&\\
\Avtors{Strijov~V.\,V.} see~Isachenko~R.\,V.&&\\
\Avtors{Strijov~V.\,V.} see~Karasikov~M.\,E.&&\\
\Avtors{Stupnikov~S.\,A., Briukhov~D.\,O., and Skvortsov~N.\,A.}
Co-lending systemic risk analysis over\linebreak
\\[-12pt]
\hspace*{23pt}heterogeneous data collections&1&23--33\\
\Avtors{Stupnikov~S.\,A.} see~Kalinichenko~L.\,A.&&\\
\Avtors{Suchkov~A.\,P.} see~Zatsarinny~A.\,A.&&\\
\Avtors{Timonina~E.\,E.} see~Grusho~A.\,A.&&\\
\Avtors{Titova~A.\,I.} see~Kudryavtsev~A.\,A.&&\\
\Avtors{Turlikov~A.\,M.} see~Ometov~A.\,Ya.&&\\
\Avtors{Tyrsin~A.\,N.\ and Serebryanskii~S.\,M.} Recognition of
dependences on the basis of inverse\linebreak
\\[-12pt]
\hspace*{23pt}mapping&2&58--64\\
\Avtors{Ulyanov~V.\,V.} see~Markov~A.\,S.&&\\
\Avtors{Ushakov~V.\,G.} Queueing system with working vacations and
hyperexponential input stream&2&92--97\\
\Avtors{Ushakov~V.\,G.} see~Leontyev~N.\,D.&&\\
\Avtors{Volnova~A.\,A.} see~Kalinichenko~L.\,A.&&\\
\Avtors{Yakovlev~O.\,A.\ and Gasilov~A.\,V.} Speeded-up stereo
matching using geodesic support weights&3&\hphantom{1}98--104\\
\Avtors{Zabezhailo~M.\,I.} see~Grusho~A.\,A.&&\\
\Avtors{Zabezhailo~M.\,I.} see~Grusho~A.\,A.&&\\
\Avtors{Zakharova~T.\,V.\ and Shestakov~O.\,V.} Precision analysis of
wavelet processing of aerodynamic\linebreak
\\[-12pt]
\hspace*{23pt}flow patterns&3&46--54\\
\Avtors{Zalizniak~Anna~A.\ and Kruzhkov~M.\,G.} Database
of~Russian impersonal verbal constructions&4&132--141\\
\Avtors{Zasypko~V.\,V.} see~Shnurkov~P.\,V.&&\\
\Avtors{Zatsarinny~A.\,A.\ and Suchkov~A.\,P.} Systems engineering
approaches to~the~establishment of\linebreak
\\[-12pt]
\hspace*{23pt}a~system for~decision support based
on~situational analysis&4&105--113\\
\Avtors{Zatsarinny~A.\,A.} see~Grusho~A.\,A.&&\\
\Avtors{Zatsman~I.\,M., Inkova~O.\,Yu., Kruzhkov~M.\,G., and
Popkova~N.\,A.} Representation of cross-\linebreak
\\[-12pt]
\hspace*{23pt}lingual knowledge about
connectors in supracorpora databases&1&106--118\\
\Avtors{Zatsman~I.\,M.} see~Minin~V.\,A.&&\\
\Avtors{Zeifman~A.\,I.} see~Korolev~V.\,Yu.&&\\
\Avtors{Zeifman~A.\,I.} see~Korolev~V.\,Yu.&&\\
\end{tabular}
}

%\thispagestyle{myheadings}
\def\leftfootline{\small{\textbf{\thepage}
\hfill INFORMATIKA I EE PRIMENENIYA~--- INFORMATICS AND APPLICATIONS\ \ \ 2016\
\ \ volume~10\ \ \ issue\ 4}
}%
 \def\rightfootline{\small{INFORMATIKA I EE PRIMENENIYA~---
INFORMATICS AND APPLICATIONS\ \ \ 2016\ \ \ volume~10\ \ \ issue\ 4
\hfill \textbf{\thepage}}}

 \label{end\stat}

\newpage

%\def\stat{rekl}
%\label{preobr}

%\def\tit{АКАДЕМИК ПУГАЧЁВ  ВЛАДИМИР СЕМЁНОВИЧ\\
%25.03.1911--25.03.1998}


%   \vspace*{-48pt}
%   \begin{center}\LARGE
%Академик Пугачёв  Владимир Семёнович\\ (25.03.1911--25.03.1998)
%   \end{center}
   
   %\vspace*{2.5mm}
   
   \begin{center}

{\prgsh\LARGE
ОБЪЯВЛЕНИЯ О КОНФЕРЕНЦИЯХ}

\end{center}
%\hrule

\vspace*{6pt}

   
   \vspace*{10mm}
   
   \thispagestyle{empty}

\noindent
\begin{tabular}{cc}
%\begin{center}
\multicolumn{1}{c}{\raisebox{-40pt}[0pt][0pt]{\mbox{%
\epsfxsize=33mm
\epsfbox{vspu.eps}
}}}
%\end{center}
&
\tabcolsep=0pt\begin{tabular}{c}
{\prg{\Large\textbf{XII Всероссийское совещание}}}\\[6pt]
{\prg{\Large\textbf{по проблемам управления}}}\\[12pt]
{\prg{\large 16--19 июня 2014~г.}}\\[6pt] 
{\prg{\large Институт проблем управления имени В.\,А.~Трапезникова РАН}}\\[6pt]
{\prg{\large Москва, Россия}}
\end{tabular}
\end{tabular}

\vspace*{60pt}

     
 { %\large    
 XII Всероссийское совещание по проблемам управления (ВСПУ XII), посвященное 75-летию 
Института проблем управления (ИПУ) имени В.\,А.~Трапезникова РАН, проводится 16--19~июня 
2014~г.\ 
в ИПУ РАН (г.~Москва, Россия). ВСПУ XII организуется ИПУ РАН при поддержке РФФИ, Отделения 
энергетики, машиностроения, механики и процессов управления Российской академии наук, 
Российского 
национального комитета по автоматическому управлению, Академии навигации и управ\-ле\-ния 
движением, 
Научного совета РАН по комплексным проблемам управления и автоматизации, Совета по 
мехатронике и робототехнике РАН. Официальный язык Совещания~--- русский.

\vspace*{24pt}
     
     \textbf{Направления работы}
     \begin{enumerate}[1.]
\item Теория систем управления
\item Управление подвижными объектами и навигация
\item Интеллектуальные системы управления
\item Управление в промышленности, транспортом и логистикой
\item Управление системами междисциплинарной природы
\item Средства измерения, вычислений и контроля в управлении
\item Системный анализ и принятие решений в задачах управления
\item Информационные технологии в управлении
\item Проблемы образования в области управления: современное содержание и технологии обучения
\end{enumerate}

\vspace*{24pt}

     Подробная информация о Совещании находится на сайте {\sf http://vspu2014.ipu.ru}. Срок 
окончательной подачи докладов через систему подачи докладов на сайте~--- \textbf{30~ноября} 
2013~г.
}


%\end{document}

%\include{nekrolog-rb}


%\end{document}

%\include{IPPM-25}

\def\stat{cont-rus}
{%\hrule\par
%\vskip 7pt % 7pt
\vspace*{-24pt}
\raggedleft\Large \bf%\baselineskip=3.2ex
Правила подготовки рукописей  для публикации в журнале
<<Информатика~и~её~применения>> \vskip 8pt
    \hrule
    \par
\vskip 14pt plus 6pt minus 3pt }

\label{st\stat}

\def\tit{\ }

\def\aut{\ }
\def\auf{\ }

\def\leftkol{\ }
% Правила подготовки рукописей  для публикации в журнале
%<<Информатика и её применения>>

\def\rightkol{\ }
%Правила подготовки рукописей  для публикации в журнале
%<<Информатика и её применения>>}


\titele{\tit}{\aut}{\auf}{\leftkol}{\rightkol}


\vspace*{-60pt}
{ %\small

Журнал <<Информатика и её применения>>
публикует теоретические, обзорные и дискуссионные статьи,
посвященные научным исследованиям и разработкам в области
информатики и ее приложений.

Журнал издается на русском языке. По специальному решению
редколлегии отдельные статьи могут печататься на английском языке.

Тематика журнала охватывает следующие направления:
\begin{itemize}
\item теоретические основы информатики;\\[-15pt]
      \item
математические методы исследования сложных систем и процессов;\\[-15pt]
           \item
информационные системы и сети;\\[-15pt]
                \item
информационные технологии;\\[-15pt]
                     \item
архитектура и программное обеспечение вычислительных комплексов и сетей.\\[-15pt]
\end{itemize}


\noindent
\begin{enumerate}[1.]
\item В журнале печатаются статьи, содержащие результаты, ранее не опубликованные и
не предназначенные к одновременной публикации в других изданиях.

%Публикация не должна нарушать закон об авторских правах.
Публикация предоставленной автором(ами) рукописи не должна нарушать 
положений глав~69, 70 раздела~VII части~IV Гражданского кодекса, 
которые определяют права на результаты интеллектуальной деятельности 
и~средства индивидуализации, в~том числе авторские права, в~РФ.

Ответственность за нарушение авторских прав, в~случае предъявления претензий к~редакции журнала,  
несут авторы статей.



Направляя рукопись в редакцию, авторы сохраняют свои права на данную
рукопись и при этом передают учредителям и редколлегии журнала неисключительные права на
издание статьи на русском языке 
(или на языке статьи, если он отличен от рус\-ско\-го) и~на перевод ее на английский
язык, а~также на
ее распространение в России и за рубежом. 
Каждый автор должен представить в~редакцию подписанный 
с~его стороны <<Лицензионный договор о~передаче неисключительных прав 
на использование произведения>>, текст которого размещен по адресу 
{\sf http://www.ipiran.ru/publications/licence.doc}. 
Этот договор может быть пред\-став\-лен в~бумажном (в~2-х экз.)\ 
или в~электронном виде (отсканированная копия заполненного и~подписанного документа).




Редколлегия вправе запросить у авторов экспертное заключение о возможности
пуб\-ли\-ка\-ции пред\-став\-лен\-ной статьи в открытой печати.\\[-13.5pt]

\item К статье прилагаются данные автора (авторов) (см.\ п.~8). При наличии нескольких
авторов указывается фамилия автора, ответственного за переписку с редакцией.\\[-13.5pt]

\item Редакция журнала осуществляет экспертизу присланных статей в соответствии с
принятой в журнале процедурой рецензирования.

Возвращение рукописи на доработку не означает ее принятия к печати.

Доработанный вариант с ответом на замечания рецензента необходимо прислать в
редакцию.\\[-13.5pt]

\item Решение редколлегии о публикации статьи или ее отклонении сообщается авторам.

Редколлегия может также направить авторам текст рецензии на их статью. Дискуссия по
поводу отклоненных статей не ведется.\\[-13.5pt]

%\pagebreak

\item Редактура статей высылается авторам для просмотра. Замечания к редактуре должны
быть присланы авторами в кратчайшие сроки.\\[-13.5pt]

\item Рукопись предоставляется в электронном виде в форматах MS WORD (.doc или
.docx) или \LaTeX\  (.tex), дополнительно~--- в формате .pdf, на дискете, лазерном диске
или электронной почтой. Предоставление бумажной рукописи необязательно.\\[-13.5pt]

\item При подготовке рукописи в MS Word рекомендуется использовать следующие
настройки.

Параметры страницы:
формат~--- А4; ориентация~--- книжная; поля (см): внутри~--- 2,5, снаружи~--- 1,5,
сверху~--- 2, снизу~--- 2, от края до нижнего колонтитула~--- 1,3.

Основной текст: стиль~--- <<Обычный>>, шрифт~--- Times New Roman, размер~---
14~пунк\-тов, абзацный отступ~--- 0,5~см, 1,5~интервала, выравнивание~--- по ширине.

\pagebreak

\def\leftkol{Правила подготовки рукописей  для публикации в журнале
<<Информатика и её применения>>}

\def\rightkol{Правила подготовки рукописей  для публикации в журнале
<<Информатика и её применения>>}



Рекомендуемый объем рукописи~--- не свыше 10~страниц указанного формата.
При превышении указанного объема редколлегия вправе потребовать от 
автора сокращения объема рукописи.


Сокращения слов, помимо стандартных, не допускаются. Допускается минимальное
количество аббревиатур.


Все страницы рукописи нумеруются.

Шаблоны оформления представлены в интернете:

\noindent
 {\sf
http://www.ipiran.ru/journal/template\_iiep\_ssi\_2024.zip}\\[-14pt]

\item Статья должна содержать следующую информацию на {\bfseries\textit{русском и
английском языках}}:\\[-16pt]

\begin{itemize}
\item название статьи;\\[-15pt]
\item Ф.И.О.\ авторов, на английском можно только имя и фамилию;\\[-15pt]
\item место работы, с указанием почтового адреса организации и электронного адреса каждого
автора;\\[-15pt]
\item сведения об авторах, в соответствии с форматом, образцы которого
представлены на страницах:



\def\leftfootline{\small{\textbf{\thepage}
\hfill ИНФОРМАТИКА И ЕЁ ПРИМЕНЕНИЯ\ \ \ том\ 18\ \ \ выпуск\ 3\ \ \ 2024}
}%
 \def\rightfootline{\small{ИНФОРМАТИКА И ЕЁ ПРИМЕНЕНИЯ\ \ \ том\ 18\ \ \ выпуск\ 3\ \ \ 2024
\hfill \textbf{\thepage}}}



{\sf http://www.ipiran.ru/journal/issues/2013\_07\_01/authors.asp} и

{\sf http://www.ipiran.ru/journal/issues/2013\_07\_01\_eng/authors.asp};
\item аннотация (не менее 100~слов на каждом из языков). Аннотация~--- это краткое
резюме работы, которое может публиковаться отдельно. Она является основным
источником информации в~ин\-фор\-ма\-ци\-он\-ных системах и базах данных. Английская
аннотация должна быть оригинальной, может не быть дословным переводом русского
текста и должна быть написана хорошим английским языком. В~аннотации не должно
быть ссылок на литературу и, по возможности, формул;\\[-15pt]
\item ключевые слова~--- желательно из принятых в мировой
на\-уч\-но-тех\-ни\-че\-ской литературе тематических тезаурусов. Предложения не
могут быть ключевыми словами;\\[-15pt]
\item источники финансирования работы (ссылки на гранты, проекты,
поддерживающие организации и~т.\,п.).
\end{itemize}



%\pagebreak

\item  Требования к спискам литературы.\\[-14pt]

Ссылки на литературу в тексте статьи нумеруются (в квадратных скобках) и
располагаются в каждом из списков литературы в порядке  первых упоминаний. Если источник имеет DOI и/или EDN,
то их необходимо указывать.

Списки литературы представляются в двух вариантах:\\[-14pt]


\noindent
\begin{enumerate}[(1)]
\item \textbf{Список литературы к русскоязычной части}. Русские и английские
работы~---  на языке и в алфавите оригинала;\\[-14.5pt]
\item  \textbf{References}. Русские работы и работы на других языках~--- в латинской
транслитерации с переводом на английский язык; английские работы и работы на других
языках~--- на языке оригинала.
\end{enumerate}

Необходимо для составления списка ``References'' пользоваться размещенной на сайте
{\sf http://www. translit.net/ru/bgn/} бесплатной программой транслитерации русского
 текста в~латиницу. %, при этом в~за\-клад\-ке <<варианты\ldots>> следует выбратьопцию BGN.

Список литературы ``References'' приводится полностью отдельным блоком, повторяя все
позиции из списка литературы к русскоязычной части, независимо от того, имеются или
нет в нем иностранные источники. Если в списке литературы к русскоязычной части есть
ссылки на иностранные публикации, набранные латиницей, они полностью повторяются в
списке ``References''.

Ниже приведены примеры ссылок на различные виды публикаций в списке ``References''.

\def\leftfootline{\small{\textbf{\thepage}
\hfill ИНФОРМАТИКА И ЕЁ ПРИМЕНЕНИЯ\ \ \ том\ 18\ \ \ выпуск\ 3\ \ \ 2024}
}%
 \def\rightfootline{\small{ИНФОРМАТИКА И ЕЁ ПРИМЕНЕНИЯ\ \ \ том\ 18\ \ \ выпуск\ 3\ \ \ 2024
\hfill \textbf{\thepage}}}

{\small

\noindent
\textbf{Описание статьи из журнала:}

\Aue{Zagurenko, A.\,G., V.\,A.~Korotovskikh, A.\,A.~Kolesnikov, A.\,V.~Timonov, and D.\,V.~Kardymon}. 2008.
Tekhniko-ekonomicheskaya optimizatsiya dizayna gidrorazryva plasta [Technical and
economic optimization of the design
of hydraulic fracturing]. \textit{Neftyanoe hozyaystvo} [\textit{Oil Industry}] 11:54--57.

\Aue{Zhang, Z., and D.~Zhu}. 2008. Experimental research on the localized
electrochemical micromachining. \textit{Russ. J.~Electrochem.}  44(8):926--930.
{\sf doi:10.1134/S1023193508080077}.

\noindent
\textbf{Описание статьи из электронного журнала:}

\Aue{Swaminathan, V., E.~Lepkoswka-White, and B.\,P.~Rao}. 1999. Browsers or buyers in cyberspace? An
investigation of electronic factors influencing electronic exchange. \textit{JCMC}
5(2). Available at: {\sf http://www.ascusc.org/jcmc/vol5/issue2/} (accessed April~28, 2011).

\def\leftkol{Правила подготовки рукописей  для публикации в журнале
<<Информатика и её применения>>}

\def\rightkol{Правила подготовки рукописей  для публикации в журнале
<<Информатика и её применения>>}


\noindent
\textbf{Описание статьи из продолжающегося издания (сборника трудов):}

\Aue{Astakhov, M.\,V., and T.\,V.~Tagantsev}. 2006. Eksperimental'noe
issledovanie prochnosti soedineniy ``stal'--kompozit'' [Experimental study of
the strength of joints ``steel--composite'']. \textit{Trudy MGTU
``Matematicheskoe modelirovanie slozhnykh tekh\-ni\-che\-skikh sistem''}
[\textit{Bauman MSTU ``Mathematical Modeling of Complex Technical
Systems'' Proceedings}]. 593:125--130.


\pagebreak



\noindent
\textbf{Описание материалов конференций:}

\Aue{Usmanov, T.\,S., A.\,A.~Gusmanov, I.\,Z.~Mullagalin, R.\,Ju.~Muhametshina, A.\,N.~Chervyakova, and
A.\,V.~Sveshnikov}. 2007. Osobennosti proektirovaniya razrabotki mestorozhdeniy
s primeneniem gidrorazryva
plasta [Features of the design of field development with the use of hydraulic fracturing].
\textit{Trudy 6-go
Mezhdu\-na\-rod\-no\-go Simpoziuma ``Novye resursosberegayushchie tekhnologii nedropol'zovaniya i povysheniya
neftegazootdachi''} [\textit{6th  Symposium (International) ``New Energy Saving Subsoil Technologies and
the Increasing of the Oil and Gas Impact'' Proceedings}]. Moscow. 267--272.



\def\leftfootline{\small{\textbf{\thepage}
\hfill ИНФОРМАТИКА И ЕЁ ПРИМЕНЕНИЯ\ \ \ том\ 18\ \ \ выпуск\ 3\ \ \ 2024}
}%
 \def\rightfootline{\small{ИНФОРМАТИКА И ЕЁ ПРИМЕНЕНИЯ\ \ \ том\ 18\ \ \ выпуск\ 3\ \ \ 2024
\hfill \textbf{\thepage}}}



\noindent
\textbf{Описание книги (монографии, сборники):}



Lindorf, L.\,S., and L.\,G.~Mamikoniants, eds. 1972.
\textit{Ekspluatatsiya turbogeneratorov s neposredstvennym
okhlazhdeniem} [\textit{Operation of turbine generators with direct cooling}].
Moscow: Energy Publs. 352~p.


\Aue{Latyshev, V.\,N.} 2009. \textit{Tribologiya rezaniya. Kn.~1: Friktsionnye protsessy
pri rezanii metallov}
[\textit{Tribology of cutting. Vol.~1: Frictional processes in metal cutting}]. Ivanovo: Ivanovskii
State Univ. 108~p.

\def\leftkol{Правила подготовки рукописей  для публикации в журнале
<<Информатика и её применения>>}

\def\rightkol{Правила подготовки рукописей  для публикации в журнале
<<Информатика и её применения>>}

\noindent
\textbf{Описание переводной книги}
(в списке литературы к русскоязычной части необходимо указать:~/ Пер.\ с англ.~---
после названия книги, а в конце ссылки указать оригинал книги в круглых скобках):
\begin{enumerate}[1.]
\item  В русскоязычной части:

\def\leftfootline{\small{\textbf{\thepage}
\hfill ИНФОРМАТИКА И ЕЁ ПРИМЕНЕНИЯ\ \ \ том\ 18\ \ \ выпуск\ 3\ \ \ 2024}
}%
 \def\rightfootline{\small{ИНФОРМАТИКА И ЕЁ ПРИМЕНЕНИЯ\ \ \ том\ 18\ \ \ выпуск\ 3\ \ \ 2024
\hfill \textbf{\thepage}}}

\Au{Тимошенко С.\,П., Янг Д.\,Х., Уивер~У.}
Колебания в инженерном деле~/ Пер.\ с англ.~--- М.: Машиностроение, 1985. 472~с.
(\Au{Timoshenko~S.\,P., Young~D.\,H., Weaver~W.}
Vibration problems in engineering.~--- 4th ed.~--- New York, NY, USA: Wiley, 1974. 521~p.)\\[-13.5pt]
\item  В англоязычной части:

\Aue{Timoshenko, S.\,P., D.\,H.~Young, and W.~Weaver}.
1974. \textit{Vibration problems in engineering}. 4th ed. New York: 
Wiley. 521~p.
\end{enumerate}

\vspace*{-3pt}


\noindent
\textbf{Описание неопубликованного документа:}


\Aue{Latypov, A.\,R., M.\,M.~Khasanov, and V.\,A.~Baikov}.
2004 (unpubl.). Geologiya i~dobycha (NGT GiD) [Geology and production (NGT GiD)]. Certificate on official registration of the computer program
No.\,2004611198. 

\noindent
\textbf{Описание интернет-ресурса:}


Pravila tsitirovaniya istochnikov [Rules for the citing of sources]. Available at: {\sf
http://www.scribd.com/doc/1034528/} (accessed February~7, 2011).

%\pagebreak

\noindent
\textbf{Описание диссертации или автореферата диссертации:}

\Aue{Semenov, V.\,I.}
2003. Matematicheskoe modelirovanie plazmy v sisteme kompaktnyy tor [Mathematical
modeling of the plasma in the compact torus].  Moscow.  D.Sc.\ Diss. 272~p.

\Aue{Kozhunova, O.\,S.} 2009. Tekhnologiya razrabotki semanticheskogo
slovarya informatsionnogo monitoringa [Technology of development of
semantic dictionary of information monitoring system].  Moscow: IPI RAN. PhD Thesis. 23~p.


\noindent
\textbf{Описание ГОСТа:}

GOST 8.586.5-2005. 2007. Metodika vypolneniya izmereniy. Izmerenie raskhoda i~kolichestva zhidkostey i~gazov
s~pomoshch'yu standartnykh suzhayushchikh ustroystv [Method of measurement.
Measurement of flow rate and volume of liquids and gases by means of orifice devices]. Moscow:
Standardinform  Publs. 10~p.

\noindent
\textbf{Описание патента:}

\Aue{Bolshakov, M.\,V., A.\,V.~Kulakov, A.\,N.~Lavrenov, and M.\,V.~Palkin}.
2006. Sposob orientirovaniya po krenu letatel'nogo
apparata s opti\-che\-skoy golovkoy
samonavedeniya [The way to orient on the roll of aircraft with optical homing head].
Patent RF No.\,2280590.
}

\item Присланные в редакцию материалы авторам не возвращаются.\\[-13.5pt]

\item При отправке файлов по электронной почте просим придерживаться следующих
правил:
\begin{itemize}
\item указывать в поле subject (тема) название журнала и фамилию автора;\\[-13.5pt]
\item указывать в тексте письма название статьи, авторов и~журнал, в~который направляется статья;\\[-13.5pt]
\item использовать attach (присоединение);\\[-13.5pt]
\item в состав электронной версии статьи должны входить: файл, содержащий текст
статьи, и файл(ы), содержащий(е) иллюстрации.\\[-13.5pt]
\end{itemize}

\item Журнал <<Информатика и её применения>> является некоммерческим изданием.
Плата за публикацию не взимается, гонорар авторам не выплачивается.
\end{enumerate}



\def\leftfootline{\small{\textbf{\thepage}
\hfill ИНФОРМАТИКА И ЕЁ ПРИМЕНЕНИЯ\ \ \ том\ 18\ \ \ выпуск\ 3\ \ \ 2024}
}%
 \def\rightfootline{\small{ИНФОРМАТИКА И ЕЁ ПРИМЕНЕНИЯ\ \ \ том\ 18\ \ \ выпуск\ 3\ \ \ 2024
\hfill \textbf{\thepage}}}


\vspace*{-1mm}

\begin{center}

\textbf{Адрес редакции журнала <<Информатика и её применения>>:} \\




Москва 119333, ул.~Вавилова, д.~44, корп.~2, ФИЦ ИУ РАН\\[-10pt]

\

Тел.: +7\,(499)\,135-86-92\ \ Факс:  +7\,(495)\,930-45-05\\[-10pt]

 \

e-mail:   {\sf iiep@frccsc.ru} (Стригина Светлана Николаевна)\\[-10pt]

\

{\sf http://www.ipiran.ru/journal/issues/}
\end{center}
}


\def\leftkol{Правила подготовки рукописей  для публикации в журнале
<<Информатика и её применения>>}

\def\rightkol{Правила подготовки рукописей  для публикации в журнале
<<Информатика и её применения>>}


\def\leftfootline{\small{\textbf{\thepage}
\hfill ИНФОРМАТИКА И ЕЁ ПРИМЕНЕНИЯ\ \ \ том\ 18\ \ \ выпуск\ 3\ \ \ 2024}
}%
 \def\rightfootline{\small{ИНФОРМАТИКА И ЕЁ ПРИМЕНЕНИЯ\ \ \ том\ 18\ \ \ выпуск\ 3\ \ \ 2024
\hfill \textbf{\thepage}}} 
\def\stat{podg-e}
{%\hrule\par
%\vskip 7pt % 7pt
\vspace*{-24pt}
\raggedleft\Large \bf%\baselineskip=3.2ex
Requirements for manuscripts submitted to Journal
``Informatics~and~Applications'' \vskip 8pt
    \hrule
    \par
\vskip 21pt plus 6pt minus 3pt }

\label{st\stat}

\def\tit{\ }

\def\aut{\ }
\def\auf{\ }

\def\leftkol{\ }

\def\rightkol{\ }
%Requirements for manuscripts submitted to Journal
%``Informatics~and~Applications''}

\titele{\tit}{\aut}{\auf}{\leftkol}{\rightkol}

\def\leftfootline{\small{\textbf{\thepage}
\hfill INFORMATIKA I EE PRIMENENIYA~--- INFORMATICS AND APPLICATIONS\ \ \ 2019\
\ \ volume~13\ \ \ issue\ 4}
}%
 \def\rightfootline{\small{INFORMATIKA I EE PRIMENENIYA~--- INFORMATICS AND APPLICATIONS\ \ \ 2019\ \ \ volume~13\ \ \ issue\ 4
\hfill \textbf{\thepage}}}

\vspace*{-60pt}

{\small

\noindent
Journal ``Informatics and Applications'' (Inform.\ Appl.)
publishes theoretical, review, and discussion
articles on the research and development in the
field of informatics and its applications.

The journal is published in Russian.
By a special decision of the editorial
board, some articles can be published in English.


The topics covered include the following areas:
\begin{itemize}
               \item
     theoretical fundamentals of informatics; \\[-14pt]
\item
mathematical methods for studying complex systems and processes; \\[-14pt]
\item
information systems and networks;\\[-14pt]
\item
information technologies; and \\[-14pt]
\item
architecture and software of computational complexes and networks. \\[-14pt]
\end{itemize}

\noindent
\begin{enumerate}[1.]
\item The Journal publishes original articles which have not been published before and are not
intended for simultaneous publication in other editions. An article submitted to the Journal must not violate the
Copyright law. Sending the manuscript to the Editorial Board, the authors retain all rights of the
owners of the manuscript and transfer the nonexclusive rights to publish the article in Russian
(or the language of the article, if not Russian) and its distribution in Russia and abroad to the
Founders and the Editorial Board. Authors should submit a letter to the Editorial Board in the
following form:

{\bfseries\textit{Agreement on the transfer of rights to publish:}}

``\textit{We, the undersigned authors of the manuscript ``\ldots'', pass to the
Founder and the Editorial Board of the Journal ``Informatics and Applications''
the nonexclusive right to publish the manuscript of the article in Russian (or
in English) in both print and electronic versions of the Journal. We affirm
that this publication does not violate the Copyright of other persons or
organizations.}

\textit{Author(s) signature(s): (name(s), address(es), date).}

This agreement should be submitted in paper form or in the form of a scanned copy (signed by
the authors).


%The Editorial Board has the right to request from the authors an official expert conclusion that
%the submitted article has no secret data prohibited for publication. \\[-13.5pt]
\item
A submitted article should be attached with \textbf{the data on the author(s)} (see item~8). If
there are several authors, the contact person should be indicated who is responsible for
correspondence with the Editorial Board and other authors about revisions and final approval
of the proofs.\\[-13.5pt]

\item The Editorial Board of the Journal examines the article according to the established
reviewing procedure. If the authors receive their article for correction after reviewing, it does not
mean that the article is approved for publication. The corrected article should be sent to the
Editorial Board for the subsequent review and approval.\\[-13.5pt]

\item The decision on the article publication or its rejection is communicated to the authors. The
Editorial Board may also send the reviews on the submitted articles to the authors. Any
discussion upon the rejected articles is not possible.\\[-13.5pt]

\item The edited articles will be sent to the authors for proofread. The comments of the authors
to the edited text of the article should be sent to the Editorial Board as soon as possible.\\[-13.5pt]

\item The manuscript of the article should be presented electronically in the MS WORD (.doc or
.docx) or \LaTeX\ (.tex) formats, and additionally in the .pdf format. All documents
 may be sent
by e-mail or provided on a CD or diskette. A~hard copy submission is not necessary.\\[-13.5pt]

\item The recommended typesetting instructions for manuscript.

Pages parameters: format A4, portrait orientation, document margins (cm): left~--- 2.5, right~---
1.5, above~--- 2.0, below~--- 2.0, footer 1.3.

Text: font~---Times New Roman, font size~--- 14, paragraph indent~--- 0.5, line spacing~--- 1.5,
justified alignment.

The recommended manuscript size: not more than 15~pages of the specified format.
If the specified size exceeded, the editorial board is entitled to require the author
to reduce the manuscript.

Use only standard abbreviations. Avoid  abbreviations in the title and
abstract. The full term for which an abbreviation stands should precede
its first use in the text unless it is a standard unit of measurement.

All pages of the manuscript should be numbered.

The templates for the manuscript typesetting are presented on site: {\sf
http://www.ipiran.ru/journal/template.doc}.\\[-13.5pt]


%\def\leftkol{Requirements for manuscripts submitted to Journal
%``Informatics~and~Applications''}

\item The articles should enclose data both in \textbf{Russian and English}:
\begin{itemize}
\item title;\\[-13.5pt]
\item author's name and surname;\\[-13.5pt]
\item affiliation~--- organization, its address with ZIP code, city, country, and
official e-mail address;\\[-13.5pt]
\item data on authors according to the format: (see site)

{\sf http://www.ipiran.ru/journal/issues/2013\_07\_01/authors.asp}  and

{\sf  http://www.ipiran.ru/journal/issues/2013\_07\_01\_eng/authors.asp};\\[-13.5pt]

\pagebreak

\def\leftfootline{\small{\textbf{\thepage}
\hfill INFORMATIKA I EE PRIMENENIYA~--- INFORMATICS AND APPLICATIONS\ \ \ 2019\
\ \ volume~13\ \ \ issue\ 4}
}%
 \def\rightfootline{\small{INFORMATIKA I EE PRIMENENIYA~--- INFORMATICS AND APPLICATIONS\ \ \ 2019\ \ \ volume~13\ \ \ issue\ 4
\hfill \textbf{\thepage}}}


%\def\leftkol{Requirements for manuscripts submitted to Journal
%``Informatics~and~Applications''}

%\def\rightkol{Requirements for manuscripts submitted to Journal
%``Informatics~and~Applications''}



\item abstract (not less than 100 words) both in Russian and in English. Abstract is a short
summary of the article that can be published separately. The abstract is the
main source of information on the article and it could be included in leading information
systems and data bases. The abstract in English has to be an original text and should
not be an exact translation of the Russian one. Good English is required.
In abstracts, avoid references and formulae;\\[-13.5pt]
\item indexing is performed on the basis of keywords. The use of keywords from the
internationally accepted thematic Thesauri is recommended.

%\def\leftkol{Requirements for manuscripts submitted to Journal
%``Informatics~and~Applications''}

%\def\rightkol{Requirements for manuscripts submitted to Journal
%``Informatics~and~Applications''}

Important! Keywords must not be sentences;
\item Acknowledgments.
\end{itemize}

\item References. Russian references have to be presented both in English translation and Latin
transliteration (refer {\sf http://www.translit.net/ru/bgn/}).

Please take into account the following examples of Russian references appearance:

\noindent
\textbf{Article in journal:}

\Aue{Zhang, Z., and D.~Zhu}. 2008. Experimental research on the localized electrochemical
micromachining.
\textit{Rus. J.~Electrochem.}  44(8):926--930. {\sf doi:10.1134/S1023193508080077}.


\noindent
\textbf{Journal article in electronic format:}

\Aue{Swaminathan, V., E.~Lepkoswka-White, and B.\,P.~Rao}. 1999. Browsers or buyers in
cyberspace? An
investigation of electronic factors influencing electronic exchange. \textit{JCMC}
5(2). Available at: {\sf http://www.ascusc.org/jcmc/vol5/issue2/} (accessed April~28, 2011).




\noindent
\textbf{Article from the continuing publication (collection of works, proceedings):}

\Aue{Astakhov, M.\,V., and T.\,V.~Tagantsev}. 2006. Eksperimental'noe
issledovanie prochnosti soedineniy ``stal'--kompozit'' [Experimental study of
the strength of joints ``steel--composite'']. \textit{Trudy MGTU
``Matematicheskoe modelirovanie slozhnykh tekh\-ni\-che\-skikh sistem''}
[\textit{Bauman MSTU ``Mathematical Modeling of Complex Technical
Systems'' Proceedings}]. 593:125--130.

\def\leftfootline{\small{\textbf{\thepage}
\hfill INFORMATIKA I EE PRIMENENIYA~--- INFORMATICS AND APPLICATIONS\ \ \ 2019\
\ \ volume~13\ \ \ issue\ 4}
}%
 \def\rightfootline{\small{INFORMATIKA I EE PRIMENENIYA~--- INFORMATICS AND APPLICATIONS\ \ \ 2019\ \ \ volume~13\ \ \ issue\ 4
\hfill \textbf{\thepage}}}

\def\leftkol{Requirements for manuscripts submitted to Journal
``Informatics~and~Applications''}

\def\rightkol{Requirements for manuscripts submitted to Journal
``Informatics~and~Applications''}

\noindent
\textbf{Conference proceedings:}

\Aue{Usmanov, T.\,S., A.\,A.~Gusmanov, I.\,Z.~Mullagalin, R.\,Ju.~Muhametshina,
A.\,N.~Chervyakova, and
A.\,V.~Sveshnikov}. 2007. Osobennosti proektirovaniya razrabotki mestorozhdeniy
s primeneniem gidrorazryva
plasta [Features of the design of field development with the use of hydraulic fracturing].
\textit{Trudy 6-go
Mezhdu\-na\-rod\-no\-go Simpoziuma ``Novye resursosberegayushchie tekhnologii
nedropol'zovaniya i povysheniya
neftegazootdachi''} [\textit{6th  Symposium (International) ``New Energy Saving Subsoil
Technologies and
the Increasing of the Oil and Gas Impact'' Proceedings}]. Moscow. 267--272.


\noindent
\textbf{Books and other monographs:}




Lindorf, L.\,S., and L.\,G.~Mamikoniants, eds. 1972.
\textit{Ekspluatatsiya turbogeneratorov s neposredstvennym
okhlazhdeniem} [\textit{Operation of turbine generators with direct cooling}].
Moscow: Energy Publs. 352~p.


%\Aue{Latyshev, V.\,N.} 2009. \textit{Tribologiya rezaniya. Kn.~1: Frikcionnye prosessy
%pri rezanii metallov}
%[\textit{Tribology of cutting. Vol.~1: Frictional processes in metal cutting}]. Ivanovo: Ivanovskii
%State Univ. 108~p.


%\noindent
%\textbf{Unpublished material:}

%\Aue{Latypov, A.\,R., M.\,M.~Khasanov, and V.\,A.~Baikov}.
%2004. Geology and production (NGT GiD). Certificate on official registration of the computer
%program
%No.\,2004611198. (In Russian, unpubl.)

%\noindent
%\textbf{Internet-source:}

%APA Style. 2011. Available at: {\sf http://www.apastyle.org/apa-style-help.aspx} (accessed
%February~5, 2011).

%Pravila citirovaniya istochnikov [Rules for the citing of sources]. Available at: {\sf
%http://www.scribd.com/doc/1034528/} (accessed February~7, 2011).


\noindent
\textbf{Dissertation and Thesis:}

%\Aue{Semenov, V.\,I.}
%2003. Matematicheskoe modelirovanie plazmy v sisteme kompaktnyy tor. [Mathematical
%modeling of the plasma in the compact torus]. D.Sc.\ Diss. Moscow. 272~p.

\Aue{Kozhunova, O.\,S.} 2009. Tekhnologiya razrabotki semanticheskogo
slovarya informatsionnogo monitoringa [Technology of development of
semantic dictionary of information monitoring system]. PhD Thesis. Moscow: IPI RAN. 23~p.


\noindent
\textbf{State standards and patents:}

GOST 8.586.5-2005. 2007. Metodika vypolneniya izmereniy. Izmerenie raskhoda i~kolichestva
zhidkostey i gazov 
s~pomoshch'yu standartnykh suzhayushchikh ustroystv [Method of measurement.
Measurement of flow rate and volume of liquids and gases by means of orifice devices]. M.:
Standardinform
Publs. 10~p.

%\noindent
%\textbf{Patent:}

\Aue{Bolshakov, M.\,V., A.\,V.~Kulakov, A.\,N.~Lavrenov, and M.\,V.~Palkin}.
2006. Sposob orientirovaniya po krenu letatel'nogo
apparata s opti\-che\-skoy golovkoy
samonavedeniya [The way to orient on the roll of aircraft with optical homing head].
Patent RF No.\,2280590.

References in Latin transcription are presented in the original language.

References in the text are numbered according to the order of their
first appearance; the number is
placed in square brackets. All items from the reference list should be
cited.\\[-13.5pt]

\item Manuscripts and additional materials are not returned to Authors by the Editorial Board.\\[-13.5pt]

\item Submissions of files by e-mail must include:\\[-13.5pt]
\begin{itemize}
\item   the journal title and author's name in the ``Subject'' field; \\[-13.5pt]
\item   an article and additional materials have to be attached using the ``attach'' function;\\[-13.5pt]
\item   an electronic version of the article should contain the file with the text and a separate file
with figures.\\[-13.5pt]
\end{itemize}

\item ``Informatics and Applications'' journal is not a profit publication. There are no
charges for the authors as well as there are no royalties.\\[-13.5pt]
\end{enumerate}

\def\leftfootline{\small{\textbf{\thepage}
\hfill INFORMATIKA I EE PRIMENENIYA~--- INFORMATICS AND APPLICATIONS\ \ \ 2019\
\ \ volume~13\ \ \ issue\ 4}
}%
 \def\rightfootline{\small{INFORMATIKA I EE PRIMENENIYA~--- INFORMATICS AND APPLICATIONS\ \ \ 2019\ \ \ volume~13\ \ \ issue\ 4
\hfill \textbf{\thepage}}}

\def\leftkol{Requirements for manuscripts submitted to Journal
``Informatics~and~Applications''}

\def\rightkol{Requirements for manuscripts submitted to Journal
``Informatics~and~Applications''}


%\vspace*{5mm}


\begin{center}
\textbf{Editorial Board address:} \\

%ABOUT AUTHORS



FRC CSC RAS, 44, block~2, Vavilov Str., Moscow 119333, Russia\\[-10pt]

\

Ph.: +7\,(499)\,135\,86\,92,\ \ Fax: +7\,(495)\,930\,45\,05\\[-10pt]

\

 e-mail: {\sf rust@ipiran.ru} (to Prof.\ Rustem Seyful-Mulyukov)\\[-10pt]

\

 {\sf http://www.ipiran.ru/english/journal.asp}
\end{center}
 }
%\thispagestyle{myheadings}

\def\leftkol{Requirements for manuscripts submitted to Journal
``Informatics~and~Applications''}

\def\rightkol{Requirements for manuscripts submitted to Journal
``Informatics~and~Applications''}

\def\leftfootline{\small{\textbf{\thepage}
\hfill INFORMATIKA I EE PRIMENENIYA~--- INFORMATICS AND APPLICATIONS\ \ \ 2019\
\ \ volume~13\ \ \ issue\ 4}
}%
 \def\rightfootline{\small{INFORMATIKA I EE PRIMENENIYA~--- INFORMATICS AND APPLICATIONS\ \ \ 2019\ \ \ volume~13\ \ \ issue\ 4
\hfill \textbf{\thepage}}}

 \label{end\stat}

\newpage



%\include{ipi-ind}

%\tableofcontents

\end{document}





%%%%%%%%%%%%%%%%%%%%%%

%\newcommand{\Ack}{\subsection*{\protect\large\bf Acknowledgments}}

%\vphantom*{\int\limits_0^T}

{ \begin{center}  %fig1
 \vspace*{6pt}
    \mbox{%
 \epsfxsize=79mm 
 \epsfbox{gru-1.eps}
 }

\end{center}



\noindent
{{\figurename~1}\ \ \small{
}}}

%\vspace*{6pt}

\addtocounter{figure}{1}