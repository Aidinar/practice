\def\stat{ivanova}

\def\tit{МОДЕЛИ СОВМЕСТНОГО ОБСЛУЖИВАНИЯ ТРАФИКА eMBB И~URLLC НА ОСНОВЕ 
ПРИОРИТЕТОВ В~ПРОМЫШЛЕННЫХ РАЗВЕРТЫВАНИЯХ 5G NR$^*$}

\def\titkol{Модели совместного обслуживания трафика eMBB и~URLLC на основе 
приоритетов в %промышленных 
развертываниях 5G NR}

\def\aut{Д.\,В.~Иванова$^1$, Е.\,В.~Маркова$^2$, С.\,Я.~Шоргин$^3$, Ю.\,В.~Гайдамака$^4$}

\def\autkol{Д.\,В.~Иванова, Е.\,В.~Маркова, С.\,Я.~Шоргин, Ю.\,В.~Гайдамака}

\titel{\tit}{\aut}{\autkol}{\titkol}

\index{Д.\,В.~Иванова$^1$, Е.\,В.~Маркова$^2$, С.\,Я.~Шоргин$^3$, Ю.\,В.~Гайдамака$^4$}
\index{Borisov A.\,V.}


{\renewcommand{\thefootnote}{\fnsymbol{footnote}} \footnotetext[1]
{Исследование выполнено за счет гранта 
Российского научного фонда №\,22-79-10053.}}


\renewcommand{\thefootnote}{\arabic{footnote}}
\footnotetext[1]{Российский университет дружбы народов имени
Патриса Лумумбы, \mbox{ivanova-dv@rudn.ru}}
\footnotetext[2]{Российский университет дружбы народов имени
Патриса Лумумбы, \mbox{markova-ev@rudn.ru}}
\footnotetext[3]{Федеральный исследовательский центр 
<<Информатика и~управление>> Российской академии наук, \mbox{sshorgin@ipiran.ru}}
\footnotetext[4]{Российский университет дружбы народов имени 
Патриса Лумумбы; Федеральный исследовательский центр <<Информатика и~управление>> 
Российской академии наук, \mbox{gaydamaka-yuv@rudn.ru}}

%\vspace*{-10pt}



\Abst{Технология радиодоступа 5G NR способна осуществлять 
одновременную поддержку как сверхнадежной доставки с~малой задержкой (Ultra-Reliable Low Latency 
Communication, URLLC), 
так и~расширенного мобильного широкополосного доступа (enhanced Mobile Broadband, eMBB). Из-за критических 
требований к~задержке и~надежности, предъявляемых при предоставлении услуг двух 
классов обслуживания, возникает необходимость введения приоритетов. В~статье 
рассмотрена промышленная среда, в~которой производственное оборудование при 
управлении движением и~синхронизации работы генерирует URLLC-тра\-фик, а при 
удаленном мониторинге~--- eMBB-тра\-фик. Предложена модель с~приоритетным 
обслуживанием на базовой станции (БС) с~прямой связью между устройствами (device-to-device, D2D) 
и~без нее. Полученные численные результаты показывают, что введение приоритетов 
позволяет эффективно изолировать трафик URLLC и~eMBB. При этом стратегия 
с~поддержкой D2D, в~которой БС явно резервирует ресурсы для прямой связи, 
значительно превосходит по показателям качества обслуживания стратегии, 
в~которых явное резервирование не используется, а также стратегию, в~которой весь 
трафик проходит через БС.}
%\end{abstract}

\KW{5G; NR (New Radio); D2D; URLLC; eMBB; управ\-ле\-ние 
ресурсами; приоритетное обслуживание}

\DOI{10.14357/19922264230409}{JXCGXQ}
  
%\vspace*{-4pt}


\vskip 10pt plus 9pt minus 6pt

\thispagestyle{headings}

\begin{multicols}{2}

\label{st\stat}

    
\section{Введение}
%\label{sect:00}

Мобильные системы пятого поколения (5G), характеризующиеся высокими 
требованиями к~качеству обслуживания, разработаны с~учетом стремительного роста 
числа новых приложений. Помимо классических сценариев использования технология 
5G NR (New Radio) обещает поддержку в~сфере промышленной автоматизации таких 
приложений, как совместное управление мобильными роботами, синхронизация, 
позиционирование, сервисы дополненной реальности, а также обслуживание на основе 
технологий телеприсутствия~\cite{ghosh2019industrial}.

Системы, управляющие движущимися элементами производственного оборудования, 
требуют сверхнадежной доставки с~низкими задержками (URLLC). В~то же время для видеонаблюдения требуется расширенный 
мобильный широкополосный доступ (eMBB). Таким 
образом, БС NR должны поддерживать предоставление услуг двух 
классов обслуживания eMBB и~URLLC одновременно. Механизмы их отдельной поддержки 
на БС NR в~миллиметровом диапазоне (mmWave) широко исследованы 
(см., например,~[2--4] для eMBB и~\cite{rao2018packet, mahmood2019resource} для 
URLLC). Однако в~об\-ласти совместной поддержки исследования практически не 
проводились.

\begin{figure*} %fig1
 \vspace*{1pt}
\begin{center}
   \mbox{%
\epsfxsize=160.748mm 
\epsfbox{iva-1.eps}
}
\end{center}
\vspace*{-3pt}
\Caption{Рассматриваемый сценарий развертывания:
\textit{1}~--- передача от камеры до БС; 
\textit{2}~--- D2D-пе\-ре\-да\-ча с~координацией через БС; 
\textit{3}~--- прямая D2D-пе\-ре\-да\-ча; \textit{4}~--- заблокированная D2D-пе\-ре\-да\-ча;
\textit{5}~--- зона покрытия рас\-смат\-ри\-ва\-емой~БС}
\label{fig:deployment}
\vspace*{9pt}
\end{figure*}

В статье исследованы три схемы (стратегии) одновременного предоставления услуг 
eMBB и~URLLC с~реализацией явного приоритета~\cite{kochetkova2021queuing}: 
(1)~базовая стратегия~--- передача трафика через БС NR; (2)~стратегия 
D2D-aware~--- использование D2D для передачи с~полной 
координацией через БС (дополнительная интерференция не создается); (3)~стратегия 
D2D-unaware~--- использование D2D без координации через БС (уменьшает задержку, 
но создает дополнительную интерференцию). Для построения и~анализа модели 
использованы методы стохастической геометрии и~теории массового обслуживания~\cite{gorbunova2018resource}.

\vspace*{-6pt}

\section{Системная модель}
%\label{sys}

\vspace*{-3pt}

Рассмотрим модель развертывания 5G NR в~промышленной среде, например на 
автоматизированном заводе с~несколькими производственными линиями (рис.~1). 
Автоматизация производства предполагает регулярный мониторинг процесса с~по\-мощью 
расположенных на станках датчиков с~контролем посредством камер видеонаблюдения. 
Предположим, что БС NR монтируются на потолке на высоте~$h_A$, образуя 
пуассоновский точечный процесс с~плотностью~$\chi$~БС/$\text{м}^2$. Пользовательское оборудование, состоящее из датчиков и~камер, 
расположено на высоте~$h_U$ в~узлах регулярной сетки с~шагом $l$~м. Ширина 
полосы пропускания каждой БС составляет $W$~Гц, что соответствует емкости соты 
сети связи~$C$ условных единиц ре\-сурса.
{\looseness=1

}



Датчики, связанные с~автоматизированным оборудованием, генерируют потоковый 
трафик, соответствующий  URLLC-услуге и~ха\-рак\-те\-ри\-зу\-ющий\-ся гарантированной 
скоростью передачи данных $c_1 \hm\geq 1$. Эластичный трафик, соответствующий eMBB, 
характеризуется минимальной скоростью $c_2^{\min} \hm\geq 1$ и~генерируется 
видеокамерами, использующимися для удаленного мониторинга. Известна средняя 
интенсивность генерации запросов на передачу данных от датчиков и~камер~$\lambda_i$, а также средняя длительность сессий по передаче данных~$\mu_i^{-1}$, 
$i \hm= 1,2$.
Объем ресурса для обслуживания запроса URLLC через БС $b_{1,B}$ и~в режиме D2D~$b_{1,D}$, а~также объем ресурса для обслуживания запроса \mbox{eMBB}~$b_2^{\min}$ 
зависит от плотности БС и~рассчитывается по формулам:
\begin{equation*}
b_2^{\min}=\fr{c_2^{\min}}{E[S_{e,B}]}\,;\enskip b_{1,B} =\fr{c_1}{E[S_{e,B}]}\,;\enskip
  b_{1,D}=\fr{c_1}{E[S_{e,D}]}\,,
 % \label{eqn:requirements}
\end{equation*}

\noindent
где $E[S_{e,B}]$~--- средняя спектральная эффективность при передаче трафика 
через БС, а~$E[S_{e,D}]$~--- при D2D-передаче.

Рассчитаем среднюю спектральную эффективность для разных схем обслуживания. 
Плотности вероятности расстояний $D$ между двумя случайно выбранными 
пользовательскими устройствами и~$B$ от случайно выбранного устройства до БС 
определяются как~\cite{moltchanov2012distance}
\begin{align*}
f_{B}(x)&=\fr{2x}{r_N}\,;\\[3pt]
f_{D}(x)&=\fr{2x}{r_N^2}\left[\fr{2}{\pi}\cos^{-1}\left(\fr{x}{2r_N}\right)-\fr{x}{r_N\pi}\sqrt{1-\fr{x^2}{4r_N^2}}\,\right].
%\label{strat:00}
\end{align*}

Далее воспользуемся моделью распространения и~блокировки сигнала, рассмотренной 
в~\cite{ivanova2022performance}.
Эффективный радиус покрытия $r_N$ определяется как $\min(r_{N,S},r_{N,V})$, где 
$r_{N,S}$~--- максимально возможное расстояние между пользовательским 
оборудованием и~БС NR; $r_{N,V}$~--- половина расстояния меж\-ду БС~NR.

Отношение сигнала к~шуму (SNR) на приемнике, расположенном на расстоянии~$x$ от~БС
\begin{equation*}
%\label{eqn:genericProp}
S(x)=\fr{P_{U}G_{A}G_{U}}{N_0W+M_I}\,x^{-\zeta},
\end{equation*}
где $P_{U}$~--- излучаемая мощность; $G_{A}$ и~$G_{U}$~--- коэффициент усиления 
антенны на базовой станции и~на пользовательском оборудовании соответственно; 
$N_0$~--- спектральная плотность мощности шума; $M_I$~--- средняя мощность помех; 
$\zeta$~--- коэффициент распространения. Тогда

\noindent
\begin{align*}
%\label{eqn:se_bs}
E[S_{e,B}]&=\int\limits_{0}^{r_N}f_{B}(x)\log_{2}[1+S(x)]\,dx\,;\\
E[S_{e,D}]&=\int\limits_{0}^{2r_N}f_{D}(x)\log_{2}[1+S(x)]\,dx\,.
\end{align*}

Для расчета вероятности блокировки D2D-пе\-ре\-да\-чи сначала найдем $p_{B,1}(x)$~--- 
вероятность того, что путь прямой видимости длиной $x$ между двумя устройствами 
перекрывается одним станком. Воспользуемся методами интегральной геометрии 
и~получим вероятность перекрытия~\cite{santalo2004integral}:
\begin{multline*}
%\label{strat:02}
p_{B,1}(x)={}\\
{}=\fr{2w(\pi{}w+4x) (1-\kappa)}{\pi(2\pi{}r_N^2-4r_N^2\sin^{-1}(x/(2r_N))-
x\sqrt{4r_N^2-x^2})},
\end{multline*}
где $\kappa$~--- прозрачность пользовательского оборудования; $w$~--- ширина 
пользовательского оборудования.
Согласно рассматриваемой модели, максимальное чис\-ло станков~$N_R$, находящихся 
в~зоне покрытия БС NR, можно найти с~по\-мощью аппроксимации задачи о~круге Гаусса.

Общее число станков имеет биномиальное распределение с~параметрами $N_R$ и~$\nu$ 
(вероятность нахождения станка в~точке сетки). Таким образом, вероятность того, 
что путь прямой видимости заблокирован, имеет вид:
\begin{multline*}
p_{B}(x)={}\\
\!{}=\sum\limits_{j=1}^{N_R}\!\begin{pmatrix}
N_R\\ j\end{pmatrix} 
\nu^{j}(1-\nu)^{N_R-j}
\left[1-(1-p_{B,1}(x))\right]^j\!. \!
%\label{strat:03}
\end{multline*}
%
Тогда искомая вероятность блокировки будет рассчитываться как
\begin{equation*}
%\label{strat:04}
p_{B}=\int\limits_{0}^{2r_N}f_{D}(x)p_{B}(x)\,dx.
\end{equation*}

\section{Математическая модель}

Функционирование рассматриваемой сис\-те\-мы описывает двумерный марковский 
случайный процесс $(N_1(t), N_2(t))$,\ $t \hm\geq 0$, где $N_i(t)$, $i \hm= 1,2$,~---
случайное чис\-ло об\-слу\-жи\-ва\-емых сис\-те\-мой запросов типа $i$ в~момент $t$. Обозначим 
максимальное чис\-ло запросов на предоставление услуг URLLC (за\-про\-сы 1-го типа) 
и~eMBB (запросы 2-го типа), которое может находиться в~сис\-те\-ме, $N_1 \hm= \lfloor 
C/b_1 \rfloor$ и~$N_2\hm = \lfloor C/b_2^{\min} \rfloor$ соответственно, тогда $n_i \hm= 0,\ldots,N_i$~--- чис\-ло 
об\-слу\-жи\-ва\-емых сис\-те\-мой за\-про\-сов типа~$i$, $i\hm = 1,2$.
Состояние сис\-те\-мы описывает двумерный век\-тор $\mathbf{n}\hm=(n_1,n_2)$ над 
пространством со\-сто\-яний


\vspace*{-6pt}

\noindent
\begin{multline*}
\mathbf{X} = {}\\[2pt]
\!{}=\left\{ (n_1,n_2): n_1 \geq 0,\ \  n_2 \geq 0,\ \ n_1 b_1 + n_2 b_2^{\min} \leq 
C \right\}.\hspace*{-5.19798pt}
%\label{eq:stateSpace}
\end{multline*}
Обозначим через 

\noindent
$$
k(n_1) = \left\lfloor \fr{C-n_1 b_1}{b_2^{\min}}\right \rfloor
$$ 
максимальное чис\-ло 
запросов на предостав\-ле\-ние услуги \mbox{eMBB}, которое может быть принято в~сис\-те\-му, 
когда в~ней уже обслуживаются~$n_1$~запросов на предостав\-ле\-ние услуги URLLC. При 
этом чис\-ло единиц ресурса, выделяемое для обслуживания каждого запроса на 
предостав\-ле\-ние услуги eMBB, может меняться в~за\-ви\-си\-мости от со\-сто\-яния сис\-темы: 

\noindent
$$
b_2\left(n_1,n_2\right) = \left\lfloor \fr{C-n_1 b_1}{n_2} \right\rfloor \geq b_2^{\min}.
$$

Сформулируем правила приема и~обслуживания запросов:
\begin{itemize}
    \item если число обслуживаемых сис\-те\-мой запросов типа $i$ меньше максимально 
возможного чис\-ла запросов данного типа~$N_i$ и~чис\-ло свободных единиц ресурса, 
доступных для данных запросов, не меньше~$b_1$ и~$b_2^{\min}$ для 1-го и~2-го 
типа соответственно, то поступающий в~сис\-те\-му запрос будет принят на 
обслуживание;
    \item если число обслуживаемых сис\-те\-мой запросов на предостав\-ле\-ние услуги 
URLLC меньше максимально возможного чис\-ла запросов данного типа~$N_1$, чис\-ло 
свободных единиц ресурса, доступных для запросов данного типа, меньше~$b_1$, 
а~чис\-ло обслуживаемых запросов на предостав\-ле\-ние услуги eMBB не меньше $1$, то 
по\-сту\-па\-ющий запрос на предостав\-ле\-ние услуги URLLC будет принят на обслуживание 
за счет прерывания обслуживания $q(n_1,n_2) \hm= \lceil (b_1\hm - C \hm+ (n_1 b_1\hm + n_2 
b_2^{\min}))/b_2^{\min} \rceil$ запросов eMBB;
    \item в~противном случае поступающие в~сис\-те\-му запросы будут заблокированы.
\end{itemize}

На основе сформулированных правил со\-ста\-вим диаграмму интенсивностей переходов 
(рис.~\ref{fig:generalView}).

\begin{figure*} %fig2
\vspace*{1pt}
\begin{center}
   \mbox{%
\epsfxsize=88.713mm 
\epsfbox{iva-2.eps}
}
\end{center}
\vspace*{-10pt}
  \Caption{Диаграмма интенсивностей переходов для центрального состояния}
  \label{fig:generalView}
  \vspace*{-5pt}
\end{figure*}

Стационарное распределение вероятностей со\-сто\-яний сис\-те\-мы $p(\mathbf{n}), 
\mathbf{n}\hm \in \mathbf{X}$, может быть получено путем чис\-лен\-но\-го решения сис\-те\-мы 
уравнений равновесия 
$$
\mathbf{p}^{\mathrm{T}} \mathbf{A} = \mathbf{0}^{\mathrm{T}};\quad 
\mathbf{p}^{\mathrm{T}} \mathbf{1} = 1\,,
$$
 где $\mathbf{A}$~--- инфинитезимальная 
мат\-ри\-ца, элементы которой определяются сле\-ду\-ющим образом:

\pagebreak

\noindent
\begin{multline*}
a\left(\mathbf{n},\mathbf{n}'\right) ={}\\
{}=
  \begin{cases}
    \lambda_1, &\mathbf{n}' = \mathbf{n}+\mathbf{e}_1,\\
    &   n_1 < N_1, b_1 (n_1+1) + b_2^{\min} n_2 \leq C,\\
               &\mbox{или\ } n'_1 = n_1 +1, n'_2 = n_2 - q(n_1,n_2),\\
               &n_1 < N_1,\enskip n_2 > 0,\\
               & b_1(n_1+1) + b_2^{\min} n_2 > C\,;\\
    \lambda_2, &\mathbf{n}' = \mathbf{n}+\mathbf{e}_2,\\
    &  n_2 < N_2, b_1 n_1 + b_2^{\min} (n_2+1) \leq C\,;\\
    n_1 \mu_1, & \mathbf{n}' = \mathbf{n}-\mathbf{e}_1,\ n_1 > 0\,;\\
    n_2 \mu_2, & \mathbf{n}' = \mathbf{n}-\mathbf{e}_2,\ n_2 > 0\,;\\
     \phi,      &\mathbf{n}' = \mathbf{n}\,;\\
    0         &\mbox{в\ ином\ случае;}
  \end{cases}
\!  
%\label{eqn:generator}
\end{multline*}

\vspace*{-13pt}

\noindent
\begin{multline*}
\phi = -\left[\lambda_1  I\{n_1 <N_1, b_1 (n_1 +1) + b_2^{\min} n_2 \leq C\} + {}\right.\\
{}+\lambda_1  I\{n_1 < N_1, n_2 > 0, b_1 (n_1 +1) + b_2^{\min} n_2 > C\} + {}\\
{}+\lambda_2  I\{n_2 < N_2, b_1 n_1 + b_2^{\min} (n_2 +1) \leq C\} + n_1 
\mu_1 +{}\\
\left.{}+ n_2 \mu_2\right],
\end{multline*}
где
$I\{x\}$~--- индикаторная функция.

Рассчитав распределение вероятностей~$p(\mathbf{n})$, можно вычислить основные 
показатели эффективности модели: ве\-ро\-ят\-ность потери URLLC-за\-просов

\vspace*{-2pt}

\noindent 
$$
B_1 = \sum\limits_{n_1=0}^{k(N_1)}{p(N_1, n_1)}\,;
$$

\vspace*{-2pt}

\noindent
 ве\-ро\-ят\-ность потери eMBB-за\-про\-сов 
 $$
 B_2 = 
\sum\limits_{n_1=0}^{N_1}{p(n_1, k(n_1))}\,;
$$
 ве\-ро\-ят\-ность прерывания обслуживания eMBB-за\-про\-са
\begin{multline*}
%\label{eq:preemptioneMBB}
\Pi = \sum\limits_{n_1=0}^{N_1 -1} \sum\limits_{\substack{ n_2 = k(n_1+1)+1 \\
    k(n_1) \ne k(n_1+1)}}^{k(n_1)}\!
\lambda_1 p(n_1,n_2)\Big/ 
\left(\lambda_1+{}\right.\\
\left.{}+\lambda_2 
I\{n_2<k(n_1)\}+n_1 \mu_1+n_2 \mu_2\right) \,.
\end{multline*}


\begin{figure*} %fig3
\vspace*{1pt}
\begin{center}
   \mbox{%
\epsfxsize=162.604mm 
\epsfbox{iva-3.eps}
}
\end{center}
\vspace*{-9pt}
\Caption{Вероятность потери запроса в~зависимости от плотности $\nu$~(\textit{a}) и~скорости $c_{2}^{\min}$~(\textit{б}):
\textit{1}~--- базовая стратегия; \textit{2}~--- стратегия D2D-aware;
\textit{3}~--- стратегия D2D-unaware; пустые значки~--- \mbox{URLLC}; залитые значки~--- \mbox{eMBB}}
\end{figure*}

\vspace*{-9pt}

\section{Численные результаты}

Для проведения численного эксперимента исследуем ве\-ро\-ят\-ность потери запроса на 
передачу данных, представленную как функцию от плот\-ности пользовательского 
оборудования (ве\-ро\-ят\-ности наличия стан\-ка в~точ\-ке сетки) $\nu$ и~от минимальной 
ско\-рости передачи данных~$c_{2}^{\min}$ на рис.~3.

На рис.~3,\,\textit{а}, где $c_1\hm = 2$~Мбит/с, $c_{2}^{\min} \hm= 1$~Мбит/с, $G_A\hm = 16\times 4$, 
$G_U \hm= 4\times4$, $\xi\hm = 5\cdot 10^{-4}$, 
$\mu_1 \hm= 10^3$ и~$\mu_2 \hm= 1/120$, мож\-но увидеть, что увеличение~$\nu$ приводит 
к~росту вероятности потери. Очевидно, что базовая стратегия, в~которой все запросы 
проходят через БС, характеризуется по\-сто\-ян\-ной ве\-ро\-ят\-ностью потери запросов на 
предостав\-ле\-ние услуг eMBB и~URLLC. В~свою очередь, стратегия D2D-aware 
характеризуется меньшей ве\-ро\-ят\-ностью потери запросов на передачу трафика URLLC, 
которая не превышает~$10^{-5}$. Рас\-смат\-ри\-вая стратегию D2D-unaware, отметим, что 
преимущества от использования прямой передачи данных незначи-\linebreak\vspace*{-12pt}

\pagebreak

\noindent
тельны, поскольку 
не\-конт\-ро\-ли\-ру\-емая интерференция отрицательно влияет на передачу в~дополнение 
к~ве\-ро\-ят\-ности потери запросов из-за нехватки ресурсов, что в~условиях высокой 
на\-груз\-ки может оказаться критичным.



Результаты показывают, что передача eMBB-тра\-фи\-ка оказывает влияние на 
обслуживание URLLC-тра\-фи\-ка. На рис.~3,\,\textit{б}, где $\nu \hm= 0{,}5$, 
проиллюстрирована за\-ви\-си\-мость ве\-ро\-ят\-ности потери запросов на передачу данных от 
минимальной тре\-бу\-емой ско\-рости передачи \mbox{eMBB}-тра\-фи\-ка. Таким образом, 
обслуживание на основе приоритетов эффективно в~том случае, когда требования 
к~запросам на предостав\-ле\-ние услуги \mbox{eMBB} растут, так как вероятность потери 
запроса на предостав\-ле\-ние услуги \mbox{URLLC} при этом сохраняется практически 
неизменной. Предлагаемый метод на основе приоритетов гарантирует, что 
обслуживание запросов на предостав\-ле\-ние услуги \mbox{URLLC} будет защищено от 
потенциально из\-ме\-ня\-юще\-йся нагрузки, создаваемой запросами на предостав\-ле\-ние 
услуги \mbox{eMBB}. В~данном случае также наблюдается 
превосходство стратегии D2D-aware над другими стратегиями для любых 
рас\-смат\-ри\-ва\-емых значений~$c_{2}^{\min}$.

\vspace*{-6pt}

\section{Заключение}
%\label{conclus}

\vspace*{-3pt}

В статье предложена модель одновременной поддержки услуг двух классов 
обслуживания~--- \mbox{eMBB} и~\mbox{URLLC}, основанная на реализации явного приоритета. Для 
трех рассмотренных схем передачи трафика на примере развертывания сис\-те\-мы на 
автоматизированном предприятии проведена оценка ключевых вероятностных 
характеристик, в~част\-ности вероятности потери запросов на передачу данных, на 
основе которой сделан вывод о~наиболее эффективной стратегии совместного 
обслуживания.

{\small\frenchspacing
 {\baselineskip=10.6pt
 %\addcontentsline{toc}{section}{References}
 \begin{thebibliography}{99}
\bibitem{ghosh2019industrial} %1
\Au{Ghosh A., Ratasuk~R., Rao~A.\,M.} Industrial IoT networks powered by 5G 
New Radio~// Microwave~J., 2019. Vol.~62. No.\,12. P.~24--40.

\bibitem{moltchanov2018improving} %2
\Au{Moltchanov D., Samuylov~A., Petrov~V., Gapeyenko~M., Himayat~N., Andreev~S., Koucheryavy~Y.} Improving session continuity with bandwidth reservation in 
mm{W}ave communications~// IEEE Wirel. Commun. Le., 2018. Vol.~8. 
No.\,1. P.~105--108. doi: 10.1109/LWC.2018.2859988.

\bibitem{begishev2019quantifying} %3
\Au{Begishev V., Moltchanov~D., Sopin~E., Samuylov~A., Andreev~S., 
Koucheryavy~Y., Samouylov~K.} Quantifying the impact of guard capacity on 
session continuity in 3GPP New Radio systems~// IEEE T. Veh. 
Technol., 2019. Vol.~68. No.\,12. P.~12345--12359. doi: 10.1109/TVT.2019.2948702.

\bibitem{samuylov2020characterizing} %4
\Au{Samuylov A., Moltchanov~D., Kovalchukov~R., Pirmagomedov~R., Gaidamaka~Y., Andreev~S., Koucheryavy~Y., Samouylov~K.}
Characterizing resource allocation trade-offs in 5G NR serving multicast and 
unicast traffic~// IEEE T. Wirel. Commun., 2020. Vol.~19. 
No.\,5. P.~3421--3434. doi: 10.1109/TWC.2020.2973375.





\bibitem{rao2018packet} %5
\Au{Rao J., Vrzic S.}
Packet duplication for URLLC in 5G: Architectural enhancements and performance 
analysis~// IEEE Network, 2018. Vol.~32. No.\,2. P.~32--40. doi: 10.1109/MNET.2018.1700227.

\bibitem{mahmood2019resource} %6
\Au{Mahmood N.\,H., Karimi~A., Berardinelli~G., Pedersen~K.\,I., Laselva~D.} 
On the resource utilization of multi-connectivity transmission for URLLC 
services in 5G New Radio~// IEEE Wireless Communications and Networking 
Conference Workshop.~--- Piscataway, NJ, USA: IEEE, 2019. Art.~8902865. 
6~p. doi: 10.1109/\mbox{WCNCW}.\linebreak 2019.8902865.

\bibitem{kochetkova2021queuing} %7
\Au{Кочеткова И.\,А., Власкина~А.\,С., Ву~Н.\,Н., Шоргин~В.\,С.} Система 
массового обслуживания с~управляемым по сигналам перераспределением приборов для 
анализа нарезки ресурсов сети 5G~// Информатика и~её применения, 2021. Т.~15. Вып.~3. 
С.~91--97. doi: 
10.14357/ 19922264210312. EDN: JJENVV.

\bibitem{gorbunova2018resource} %8
\Au{Горбунова А.\,В., Наумов~В.\,А., Гайдамака~Ю.\,В., Самуйлов~К.\,Е.} 
Ресурсные системы массового обслуживания как модели беспроводных сис\-тем связи~// 
Информатика и~её применения, 2018. Т.~12. Вып.~3. С.~48--55. doi: 
10.14357/19922264180307. EDN: YAMDIL.

\bibitem{ivanova2022performance} %9
\Au{Ivanova D., Markova~E., Moltchanov~D., Pirmagomedov~R., Koucheryavy~Y., 
Samouylov~K.} Performance of priority-based traffic coexistence strategies in 5G 
mmWave industrial deployments~// IEEE Access, 2022.  Vol.~10. P.~9241--9256. doi: 10.1109/ACCESS.2022.3143583.

\bibitem{moltchanov2012distance} %10
\Au{Moltchanov D.} Distance distributions in random networks~// Ad Hoc 
Netw., 2012. Vol.~10. No.\,6. P.~1146--1166. doi: 10.1016/j.adhoc.2012.02.005.

\bibitem{santalo2004integral} %11
\Au{Santalo L.\,A.} Integral geometry and geometric probability.~--- 
Cambridge: Cambridge University Press, 2004. 428~p.
doi: 10.1017/CBO9780511617331.

\end{thebibliography}

 }
 }

\end{multicols}

\vspace*{-6pt}

\hfill{\small\textit{Поступила в~редакцию 25.09.23}}

\vspace*{8pt}

%\pagebreak

%\newpage

%\vspace*{-28pt}

\hrule

\vspace*{2pt}

\hrule



\def\tit{PRIORITY-BASED eMBB AND~URLLC TRAFFIC COEXISTENCE MODELS 
IN~5G~NR INDUSTRIAL DEPLOYMENTS}


\def\titkol{Priority-based eMBB and~URLLC traffic coexistence models 
in~5G~NR industrial deployments}


\def\aut{D.\,V.~Ivanova$^{1}$, E.\,V.~Markova$^{1}$, S.\,Ya.~Shorgin$^{2}$, and~Yu.\,V.~Gaidamaka$^{1,2}$}

\def\autkol{D.\,V.~Ivanova, E.\,V.~Markova, S.\,Ya.~Shorgin, and~Yu.\,V.~Gaidamaka}

\titel{\tit}{\aut}{\autkol}{\titkol}

\vspace*{-9pt}


\noindent
$^1$RUDN University, 6~Miklukho-Maklaya Str., Moscow 117198, Russian 
Federation

\noindent
$^2$Federal Research Center ``Computer Science and Control'' of the Russian 
Academy of Sciences, 44-2~Vavilov\linebreak
$\hphantom{^1}$Str., Moscow 119333, Russian Federation


\def\leftfootline{\small{\textbf{\thepage}
\hfill INFORMATIKA I EE PRIMENENIYA~--- INFORMATICS AND
APPLICATIONS\ \ \ 2023\ \ \ volume~17\ \ \ issue\ 4}
}%
 \def\rightfootline{\small{INFORMATIKA I EE PRIMENENIYA~---
INFORMATICS AND APPLICATIONS\ \ \ 2023\ \ \ volume~17\ \ \ issue\ 4
\hfill \textbf{\thepage}}}

\vspace*{4pt}




\Abste{The technology 5G New Radio simultaneously supports both Ultra-Reliable Low-Latency Service (URLLC) and enhanced Mobile Broadband Service 
(eMBB). Owing to extreme latency and reliability requirements of both types of services, a~prioritization needs to be provided.
The present authors consider an industrial environment where production equipment utilizes URLLC service 
 for controlling motion and synchronous operation while eMBB service is used for remote monitoring. 
 The authors proposed the model with priority service at base station (BS) with and without direct device-to-device (D2D)
  communications. The obtained numerical results indicate that priorities allow one to isolate URLLC and eMBB traffic efficiently. 
  The D2D-aware strategy where the BS explicitly reserves resources for direct communications significantly outperforms strategies 
  where explicit reservation is not utilized as well as the strategy where all the traffic goes through the BS.}

\KWE{5G; NR (New Radio); D2D; URLLC; eMBB; resource 
allocation; priority service}




\DOI{10.14357/19922264230409}{JXCGXQ}

\vspace*{-12pt}

\Ack

\vspace*{-4pt}

\noindent
The reported study was funded by the Russian  Science Foundation, project No.\,22-79-10053.

\vspace*{6pt}

  \begin{multicols}{2}

\renewcommand{\bibname}{\protect\rmfamily References}
%\renewcommand{\bibname}{\large\protect\rm References}



{\small\frenchspacing
 {%\baselineskip=10.8pt
 \addcontentsline{toc}{section}{References}
 \begin{thebibliography}{99} 
%1
\bibitem{ghosh2019industrial-1}
\Aue{Ghosh, A., R.~Ratasuk, and A.\,M.~Rao.} 2019. Industrial IoT networks powered by 
5G New Radio. \textit{Microwave J.} 62(12):24--40.

%2
\bibitem{moltchanov2018improving-1} 
\Aue{Moltchanov, D., A.~Samuylov, V.~Pet\-rov, M. Gapeyenko, N.~Himayat, S.~And\-re\-ev, 
and Ye.~Kouc\-he\-rya\-vy.} 2019. Improving session continuity with bandwidth 
reservation in mmWave communications. \textit{IEEE Wirel. Commun. Le.} 8(1):105--108. doi: 10.1109/LWC.2018.2859988.

%3
\bibitem{begishev2019quantifying-1}
\Aue{Begishev, V., D.~Moltchanov, E.~So\-pin, A.~Samuy\-lov, S.~And\-re\-ev, Y.~Kou\-che\-rya\-vy, 
and K.~Samouylov.} 2019. Quantifying the impact of guard capacity on session 
continuity in 3GPP New Radio systems. \textit{IEEE T. Veh. Technol.} 
68(12):12345-12359. doi: 10.1109/TVT.2019.2948702.

%4
\bibitem{samuylov2020characterizing-1}
\Aue{Samuylov, A., D.~Moltchanov, R.~Ko\-val\-chu\-kov, R.~Pir\-ma\-go\-me\-dov, Y.~Gai\-da\-ma\-ka, S.~And\-re\-ev, Y.~Kou\-che\-rya\-vy, and K.~Sa\-mouy\-lov.} 
2020. Characterizing resource 
allocation trade-offs in 5G NR serving multicast and unicast traffic. 
\textit{IEEE T. Wirel. Commun.} 19(5):3421--3434. doi: 10.1109/ TWC.2020.2973375.


%5
\bibitem{rao2018packet-1}
\Aue{Rao, J., and S.~Vrzic.} 2018. 
Packet duplication for URLLC in 5G: Architectural enhancements and performance 
analysis. \textit{IEEE Network} 32(2):32--40. doi: 10.1109/ MNET.2018.1700227.
%6
\bibitem{mahmood2019resource-1}
\Aue{Mahmood, N.\,H., A.~Karimi, G.~Berardinelli, K.\,I.~Pe\-der\-sen, and D.~La\-sel\-va.} 
2019.
On the resource utilization of multi-connectivity transmission for URLLC 
services in 5G New Radio.
\textit{IEEE Wireless Communications and Networking 
Conference Workshop.} Piscataway, NJ: IEEE. 6~p. doi: 
10.1109/\mbox{WCNCW}.2019.8902865.
%7
\bibitem{kochetkova2021queuing-1}
\Aue{Kochetkova, I.\,A., A.\,S.~Vlaskina, N.\,N.~Vu, and V.\,S.~Shor\-gin.} 2021. Sistema 
mas\-so\-vo\-go ob\-slu\-zhi\-va\-niya s~uprav\-lya\-emym po sig\-na\-lam pe\-re\-ras\-pre\-de\-le\-ni\-em 
pri\-bo\-rov dlya ana\-li\-za na\-rez\-ki re\-sur\-sov se\-ti 5G [Queuing system with signals for 
dynamic resource allocation for analyzing network slicing in 5G networks]. 
\textit{Informatika i~ee Primeneniya~--- Inform. Appl.} 15(3):91--97. doi: 
10.14357/ 19922264210312. EDN: JJENVV.
%8
\bibitem{gorbunova2018resource-1}
\Aue{Gorbunova, A.\,V., V.\,A.~Naumov, Y.\,V.~Gai\-da\-ma\-ka, and K.\,E.~Sa\-mouy\-lov.} 2018. 
Re\-surs\-nye sis\-te\-my mas\-so\-vo\-go ob\-slu\-zhi\-va\-niya kak mo\-de\-li bes\-pro\-vod\-nykh sis\-tem 
svya\-zi [Resource queuing systems as models of wireless communication systems]. 
\textit{Informatika i~ee Primeneniya~--- Inform. Appl.} 12(3):48--55. doi: 
10.14357/19922264180307. EDN: YAMDIL.
%9
\bibitem{Ivanova2022-1} 
\Aue{Ivanova,~D., E.~Markova, D.~Moltchanov, R.~Pir\-ma\-go\-me\-dov, 
Y.~Kou\-che\-rya\-vy, and K.~Sa\-mouy\-lov.}
2022. Performance of priority-based traffic coexistence strategies in 5G mmWave 
industrial deployments.
\textit{IEEE Access} 10:9241--9256.
doi: 10.1109/ACCESS.2022.3143583.
%10
\bibitem{moltchanov2012distance-1}
\Aue{Moltchanov, D.} 2012. Distance distributions in random networks. \textit{AD Hoc 
Netw.} 10(6):1146--1166. doi: 10.1016/ j.adhoc.2012.02.005.
%11
\bibitem{santalo2004integral-1}
\Aue{Santalo, L.\,A.} 2004. \textit{Integral geometry and geometric probability}. 
Cambridge: Cambridge University Press. 428~p. 
doi: 10.1017/CBO9780511617331.


\end{thebibliography}

 }
 }

\end{multicols}

\vspace*{-6pt}

\hfill{\small\textit{Received September 25, 2023}} 

%\vspace*{-18pt}

\Contr

\vspace*{-4pt}

\noindent
\textbf{Ivanova Daria V.} (b.\ 1996)~--- PhD student, Department of Probability 
Theory and Cyber Security, RUDN University, 6~Miklukho-Maklaya Str., Moscow 
117198, Russian Federation; \mbox{ivanova-dv@rudn.ru}

\vspace*{3pt}

\noindent
\textbf{Markova Ekaterina V.} (b.\ 1987)~--- Candidate of Science (PhD) in physics 
and mathematics, associate professor, Department of Probability Theory and Cyber 
Security, RUDN University, 6~Miklukho-Maklaya Str., Moscow 117198, Russian 
Federation; \mbox{markova-ev@rudn.ru}

\vspace*{3pt}

\noindent
\textbf{Shorgin Sergey Ya.} (b.\ 1952)~--- Doctor of Science in physics and 
mathematics, professor, principal scientist, 
Federal Research Center ``Computer Science and Control'' of the Russian Academy 
of Sciences, 44-2~Vavilov Str., Moscow 119333, Russian Federation; 
\mbox{sshorgin@ipiran.ru}

\vspace*{3pt}

\noindent
\textbf{Gaidamaka Yuliya V.} (b.\ 1971)~--- Doctor of Science in physics and 
mathematics, professor, Department of Probability Theory and Cyber Security, 
RUDN University, 6~Miklukho-Maklaya Str., Moscow, 117198, Russian Federation; 
senior scientist, Federal Research Center ``Computer Science and Control'' of the Russian Academy 
of Sciences, 44-2~Vavilov St, Moscow, 119333, Russian Federation; 
\mbox{gaydamaka-yuv@rudn.ru}


\label{end\stat}

\renewcommand{\bibname}{\protect\rm Литература} 