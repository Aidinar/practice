\def\stat{stupnikov}

\def\tit{РАСШИРЯЕМЫЙ ПОДХОД К СЛИЯНИЮ ДАННЫХ В~РАСПРЕДЕЛЕННЫХ 
ВЫЧИСЛИТЕЛЬНЫХ СРЕДАХ$^*$}

\def\titkol{Расширяемый подход к~слиянию данных в~распределенных 
вычислительных средах}

\def\aut{В.\,В.~Сазонтьев$^1$, С.\,А.~Ступников$^2$, В.\,Н.~Захаров$^3$}

\def\autkol{В.\,В.~Сазонтьев, С.\,А.~Ступников, В.\,Н.~Захаров}

\titel{\tit}{\aut}{\autkol}{\titkol}

\index{Сазонтьев В.\,В.}
\index{Ступников С.\,А.}
\index{Захаров В.\,Н.}
\index{Sazontev V.\,V.}
\index{Stupnikov S.\,A.}
\index{Zakharov V.\,N.}


{\renewcommand{\thefootnote}{\fnsymbol{footnote}} \footnotetext[1]
{Исследование выполнено за счет гранта Российского научного фонда №\,22-21-00692 ({\sf  
https://rscf.ru/project/22-21-00692/}).}}


\renewcommand{\thefootnote}{\arabic{footnote}}
\footnotetext[1]{Федеральный исследовательский центр <<Информатика и~управление>> Российской академии наук;
\mbox{vladimirsvsite@gmail.com}}
\footnotetext[2]{Федеральный исследовательский центр <<Информатика и~управление>> Российской академии наук;
\mbox{sstupnikov@ipiran.ru}}
\footnotetext[3]{Федеральный исследовательский центр <<Информатика и~управление>> Российской академии наук;
\mbox{vzakharov@ipiran.ru}}

\vspace*{-6pt}

 
\Abst{Статья относится к~области разработки методов и~средств интеграции данных. Один 
из важнейших этапов интеграции данных~--- это слияние данных, объединение записей, 
относящихся к~одной и~той же сущности реального мира, в~одну запись с~разрешением 
конфликтов для каждого из атрибутов. Рассмотрена формальная постановка 
задачи слияния данных, приведен краткий обзор основных существующих групп методов 
слияния данных. Предложен подход к~реализации этапа слияния данных в~расширяемой 
системе интеграции разнородных источников данных в~распределенной вычислительной 
среде, рассмотрена его программная архитектура и~основные идеи реализации.}

\KW{слияние данных; распределенные вычислительные среды}

\DOI{10.14357/19922264230406}{LCQAZX}
  
%\vspace*{-4pt}


\vskip 10pt plus 9pt minus 6pt

\thispagestyle{headings}

\begin{multicols}{2}

\label{st\stat}

\section{Введение}

    Как только в~мире появились наборы данных, которые было необходимо 
связывать и~соединять, возникло понятие интеграции данных. Даже до 
появления компьютеров перед уче\-ны\-ми-ста\-ти\-сти\-ка\-ми вставали задачи 
сопоставления и~анализа данных переписей населения, собранных в~разные\linebreak 
периоды времени. Задачи интеграции данных~--- обеспечения 
унифицированного доступа к~множеству разнородных автономных 
источников данных~--- сложны не только ввиду способности \mbox{человека} 
представлять информацию о~сущностях реального мира крайне 
разнообразным образом, но и~ввиду возможности отслеживания практически 
любых событий и~взаимодействий в~цифровом мире, а также желания 
человека проводить анализ данных и~применять его результаты для принятия 
решений~[1]. 
    
    В интеграции данных можно выделить четыре основных этапа: 
извлечение данных; сопоставление схем; связывание записей; слияние 
данных~[2].

 \textit{Извлечение данных} предполагает \mbox{преобразование} 
неструктурированных данных (например, текстов) или 
полуструктурированных данных из веб-ис\-точ\-ни\-ков 
в~структурированный вид~[3]. 

\textit{Сопоставление схем} предполагает 
определение некоторой унифицированной схемы данных для предметной 
об\-ласти и~установление семантических отношений меж\-ду атрибутами 
и~сущностями схем источников и~атрибутами и~сущностями 
унифицированной схемы. 

\textit{Связывание записей} со\-сто\-ит в~группировке 
записей из разных источников, относящихся к~одной и~той же сущ\-ности 
реального мира.

 Наконец, \textit{слияние данных} означает объединение 
сгруппированных записей в~одну с~разрешением конфликтов для каж\-до\-го из 
атрибутов.
    
    Данная статья посвящена описанию подхода к~реализации этапа слияния 
данных в~раз\-ра\-ба\-ты\-ва\-емой сис\-те\-ме интеграции разнородных источников 
данных в~рас\-пре\-де\-лен\-ной вы\-чис\-ли\-тель\-ной сре\-де~[4,~5]. 

\vspace*{-7pt}
    
\section{Методы слияния данных}

\vspace*{-3pt}

    Даже когда источники данных предоставляют информацию об одном и~том же атрибуте одной и~той же сущности, значения атрибута в~разных 
источниках могут не совпадать. Эти конфликты возникают из-за 
неправильного ввода, некорректных вычислений, устаревшей информации, 
противоречивой интерпретации семантики и~т.\,д. Цель слияния данных~--- 
определить, какое значение отражает положение дел в~реальном мире.
    
    Формально задачу слияния данных можно определить следующим 
образом. Пусть $S$~--- множество источников данных, $D$~--- множество 
элементов данных. \textit{Элемент данных} представляет отдельный аспект 
сущности реального мира (например, дату рождения человека), 
а~в~реляционной базе данных элемент данных соответствует атрибуту 
кортежа из некоторого отношения. Для каждого $d\hm\in D$ источ-\linebreak\vspace*{-12pt}

\pagebreak

\noindent
ник 
$s\hm\in S$ может предоставлять значение. Значение может быть атомарным 
(дата рождения), множеством (телефонные номера), списком (авторы \mbox{статьи}). 
Среди различных значений элемента данных одно соответствует реальному 
миру (истинное значение), остальные~--- нет (ложные значения). Задача 
слияния данных~--- определить истинное значение для каждого элемента 
$d\hm\in D$~[1]. 
    
    В настоящее время разработано множество методов решения задачи 
слияния данных. Разрешение конфликтов и~удаление ошибочных значений 
опирается на коллективный <<разум>> источников данных, определение 
надежных источников, определение копирования данных между 
источниками. 
    
    При разрешении конфликтов возможно использование различных 
высокоуровневых стратегий~[6]: \textit{игнорирование конфликтов} 
(сохранение всех вариантов значений или создание всех возможных 
комбинаций значений, когда разрешение конфликта фактически остается за 
пользователем); \textit{избегание конфликтов} (когда значения берутся из 
предпочитаемого источника либо отбираются лишь непротиворечивые 
данные); \textit{разрешение конфликтов} (выбор наиболее часто 
встречающегося, случайного, среднего, наиболее актуального значения 
и~т.\,д.).
    
    Реализация стратегий возможна при помощи различных реляционных 
операций~[6]. Использование различных вариантов \textit{соединения} (join) 
отношений опирается на комбинацию кортежей из отношений при 
обращении в~истину некоторых предикатов над атрибутами. Варианты 
операции \textit{объединения} (union) предполагают добавление кортежей из 
исходных отношений в~заранее определенное отношение унифицирующей 
схемы. 
    
    Современным \textit{базовым методом определения\linebreak истинного 
значения} элемента данных обыч\-но считается голосование: значение, 
представ\-ля\-емое наибольшим чис\-лом источников, объявляется \mbox{истинным}. 
\textit{Надежность источника} определяется сте\-пенью кор\-рект\-ности 
предостав\-ля\-емых им значений. Более надежный источник обладает 
б$\acute{\mbox{о}}$льшим весом при голосовании. При обнаружении 
\textit{копирования значения} из другого источника вес этого значения при 
голосовании может быть уменьшен. К~подходам, сочетающим вы\-чис\-ле\-ние 
ве\-ро\-ят\-ности копирования, вы\-чис\-ле\-ние на\-деж\-ности источников 
и~голосование, относится, например, алгоритм \mbox{\textit{AccuCopy}} и~его 
модификации~[1]. 
    
    Другие существующие методы вычисления надежности источников и~корректности значений\linebreak
     опираются на алгоритм вычисления важности  
веб-стра\-ниц \textit{PageRank}; метрики схожести, ис\-поль\-зу\-емые в~об\-ласти 
информационного поиска; вероятностные графические модели; вы\-чис\-ле\-ние 
\mbox{точ\-ности} и~пол\-но\-ты (precision/recall)~[1].
    
    Все большее значение в~области слияния данных приобретают методы 
машинного обучения~[2]. 
    
    Многообещающим выглядит применение дискриминативных 
вероятностных моделей для слияния данных (в простейшем случае~--- 
логистической регрессии)~\cite{7-ste}. При этом процесс слияния данных 
делится на два этапа:
    \begin{enumerate}[(1)]
\item статистическое обучение для вычисления параметров модели, 
используемой для оценки надежности источников;
\item использование вероятностного вывода для предсказания значений 
элементов данных.
\end{enumerate}
    
    Для обучения моделей привлекаются признаки, характерные для 
предметной области: например, для источников научных статей характерны 
\textit{число цитирований} и~\textit{год публикации}, а для  
веб-ис\-точ\-ни\-ков~--- статистика трафика (\textit{число посещений}, 
\textit{число ежедневных просмотров} и~т.\,д.). Оптимизация при обучении 
моделей основывается на максимизации\linebreak ожидания (expectation maximization, 
EM), опи\-ра\-ющей\-ся на пересечения между значениями в~источниках 
и~сред\-нюю на\-деж\-ность источников, и~минимизации эмпирического риска 
(empirical risk \mbox{minimization}, ERM), опи\-ра\-ющей\-ся на истинные значения 
элементов данных. Для автоматизации выбора меж\-ду EM и~ERM 
применяется оптимизатор, опре\-де\-ля\-ющий, какой из алгоритмов ведет к~более 
точ\-ным результатам сли\-я\-ния данных.

\section{Подход к~реализации этапа слияния данных в~системе 
интеграции данных}

\subsection{Архитектура и~основные положения подхода}

    Предлагаемый подход предназначен для реализации в~рамках 
расширяемой системы интеграции данных в~распределенной вычислительной 
\mbox{среде}~\cite{3-ste, 4-ste, 5-ste}. Архитектура подхода (основные компоненты, 
связь с~базой данных системы интеграции) представлена на рис.~1.



     
    В системе интеграции возможно одновременное решение нескольких 
\textit{задач интеграции} в~различных предметных областях. На этапе 
слияния данных\linebreak при решении конкретной задачи интеграции предполагается, 
что этапы извлечения данных и~связывания записей уже пройдены и~их 
результаты помещены в~\textit{базу данных извлеченных записей} 
и~\textit{базу} \mbox{\textit{данных}} \textit{связанных записей} соответственно. Основные 
положения подхода к~реализации слияния данных в~системе состоят  
в~сле\-ду\-ющем:

\pagebreak

\end{multicols}

\begin{figure*} %fig1
 \vspace*{1pt}
\begin{center}
   \mbox{%
\epsfxsize=162.089mm 
\epsfbox{stu-1.eps}
}
\end{center}
\vspace*{-3pt}
\Caption{Программная архитектура компонента слияния данных}
\vspace*{7pt}
\end{figure*}

\begin{multicols}{2}

\noindent
    \begin{itemize}
\item процесс слияния данных состоит из конечного числа шагов, 
последовательность которых определяется \textit{конфигурацией} процесса 
слияния и~задается экспертом в~предопределенном формате;
\item конфигурация шага слияния включает список \textit{входных} 
атрибутов, список \textit{выходных} атрибутов, имя \textit{ключевого 
атрибута}, а~также название и~список значений параметров алгоритма 
слияния данных, применяемого на данном шаге. Выходные атрибуты могут 
быть входными для последующих шагов;
\item на каждом шаге слияния происходит применение некоторого 
алгоритма слияния на тех записях из базы данных связанных записей, 
в~которых есть входные атрибуты. Если среди входных атрибутов 
присутствуют выходные атрибуты предыдущих шагов, то они соединяются 
(используется операция реляционной алгебры \textit{соединение}~--- join) 
с~данными из базы данных связанных записей по ключевому атрибуту. 
Набор записей с~выходными атрибутами, сформированный в~результате 
работы алгоритма, сохраняется в~\textit{базе данных соединенных 
сущностей};
\item вспомогательные данные для шагов слияния сохраняются во 
\textit{вспомогательной базе данных для слияния}. К~таким данным, 
например, относятся обученные модели машинного обучения (если шаг 
предполагает использование соответствующих методов обучения на 
данных из базы данных связанных записей), которые могут быть повторно 
использованы в~других процессах слияния данных;
\item по завершении всех шагов процесса слияния данных пользователю 
предоставляется возможность задания аналитических запросов к~базе 
данных соединенных сущностей.
\end{itemize}

\begin{figure*} %fig2
      \vspace*{1pt}
\begin{center}
   \mbox{%
\epsfxsize=109.051mm 
\epsfbox{stu-2.eps}
}
\end{center}
\vspace*{-9pt}
     \Caption{Основные классы реализации слияния данных}
     \end{figure*}
    
    Подход расширяем в~следующем смысле: в~сис\-те\-му могут 
унифицированным образом встраиваться различные новые алгоритмы 
слияния данных (основанные на реляционных операциях, \mbox{определении} 
истинных значений элементов данных и~на\-деж\-ности источников, 
применении моделей машинного обуче\-ния). При этом не требуется внесение 
изменений в~общую архитектуру сис\-те\-мы. На каждом из этапов интеграции 
данных, как и~на каж\-дом из шагов слияния данных, происходят чтение 
входных данных и~запись выходных данных в~со\-от\-вет\-ст\-ву\-ющие базы данных 
без передачи данных напрямую от одного программного компонента 
к~другому. При этом реализации методов, ис\-поль\-зу\-ющих одни и~те же 
входные данные, становятся взаимозаменяемыми в~сис\-теме.

\subsection{Основные идеи программной реализации подхода}

    Рассматриваемый подход реализуется в~со\-ста\-ве прототипа сис\-те\-мы 
интеграции данных\footnote{{\sf 
https://bitbucket.org/VladimirSaz/integration\_system}.}~\cite{3-ste, 5-ste}. 
\mbox{Прототип} реализован с~использованием языков программирования Python и~Javascript в~распределенной вычислительной среде Hadoop с~использованием 
распределенной файловой сис\-те\-мы HDFS и~вы\-чис\-ли\-тель\-ной модели Spark. 
Отдельные шаги этапов интеграции данных реализуются в~виде различных 
классов языка Python. При этом для каж\-до\-го шага определен единый 
родительский класс, чтобы путем наследования от него можно было 
определять различные реализации шагов с~сохранением методов чтения 
и~записи. Это поз\-во\-ля\-ет предостав\-лять унифицированный интерфейс для 
отдельных шагов и~обеспечивает вза\-и\-мо\-за\-ме\-ня\-емость классов реализации. 
Результаты каж\-до\-го этапа и~шага сохраняются в~базе данных, развернутой 
в~распределенной фай\-ло\-вой сис\-теме. 
    
    Основные классы, используемые для реализации слияния данных,~--- 
это \textit{Processor} и~\textit{Step} из пакета \textit{data\_fusion}, а~также 
\textit{Factory} и~\textit{Algorithm} из пакета \textit{algorithm} (рис.~2). Класс 
\textit{Processor} отвечает за исполнение цик\-ла слияния данных для задачи 
системы интеграции с~идентификатором \textit{job\_id}. Значение па\-ра\-мет\-ра 
\textit{config\_path}~--- это путь в~хранилище к~файлу с~конфигурацией 
процесса слияния. Класс \textit{Step} отвечает за исполнение отдельного шага 
цикла слияния с~номером \textit{step\_num}, реализует чтение и~запись 
данных в~базу данных. 

     
    
    Класс \textit{Factory} отвечает за выбор класса алгоритма слияния 
данных для шага и~передачу настроек в~экземпляр класса алгоритма. Метод 
\textit{get} возвращает экземпляр класса с~названием \textit{className}. Класс 
\textit{Algorithm}~--- это родительский класс для всех классов, ре\-а\-ли\-зу\-ющих 
алгоритмы слияния данных. Параметр \textit{options} класса 
\textit{Algorithm}~--- это словарь па\-ра\-мет\-ров для конкретного алгоритма 
(например, для алгоритма \textit{случайного леса}~--- число деревьев 
решений, глубина дерева и~т.\,д.); параметр \textit{support\_data}~--- 
это путь к~данным конкретного шага в~хранилище (например, 
временно создаваемые файлы, готовые модели машинного обучения). 
Примеры классов, реализующих конкретные алгоритмы слияния,~--- 
\textit{AverageAllToOne} (усред\-не\-ние значений атрибутов записей), 
\textit{RandomForest} (обучение и~применение модели случайного леса для 
определения значения атрибута). 

\vspace*{-6pt}

\section{Заключение}

\vspace*{-3pt}
    
    Слияние данных~--- объединение записей, относящихся к~одной и~той 
же сущности реального мира, в~одну запись с~разрешением конфликтов для 
каждого из атрибутов~--- важ\-ный этап интеграции данных, нацеленной на 
обеспечение унифицированного до\-сту\-па к~множеству разнородных 
автономных источников данных. В~работе рас\-смот\-ре\-на формальная 
по\-ста\-нов\-ка задачи слияния данных, приведен краткий обзор основных 
сущест\-ву\-ющих групп методов слияния данных. Предложен подход 
к~реализации этапа слияния данных в~сис\-те\-ме интеграции данных, 
раз\-ра\-ба\-ты\-ва\-емой авторами, рас\-смот\-ре\-ны его программная архитектура 
и~основ\-ные идеи реализации. 
    
{\small\frenchspacing
 {\baselineskip=10.6pt
 %\addcontentsline{toc}{section}{References}
 \begin{thebibliography}{9}
\bibitem{1-ste}
\Au{Dong X.\,L., Srivastava~D.} Big Data integration.~--- Synthesis lectures on data  
management ser.~--- Morgan \& Claypool Publs., 2015. 178~p.
\bibitem{2-ste}
\Au{Dong X.\,L., Rekatsinas~T.} Data integration and machine learning: A~natural synergy~// 
Conference (International) on Management of 
Data.~--- New York, NY, USA: ACM, 2018. P.~1645--1650. doi: 
10.1145/3183713.3197387.
\bibitem{3-ste}
\Au{Sazontev V.\,V., Stupnikov S.\,A.} An extensible approach to searching and selecting data 
sources for materialized Big Data integration in distributed computing environments~// Pattern 
Recognition Image Analysis, 2023. Vol.~33. No.\,2. P.~147--156.  doi: 
10.1134/S1054661823020141. EDN: \mbox{YXUMDO}.
\bibitem{4-ste}
\Au{Sazontev V.\,V.} Methods for Big Data integration in distributed computation 
environments~// CEUR Workshop Procee. Vol.~2277. P.~238--244.
\bibitem{5-ste}
\Au{Sazontev V., Stupnikov~S.} An extensible approach for materialized Big Data integration in 
distributed computation environments~// Ivannikov Memorial Workshop.~--- 
IEEE, 2019. P.~33--38. doi: 10.1109/IVMEM.2019.00011. EDN: \mbox{BWILRL}.
\bibitem{6-ste}
\Au{Bleiholder J., Naumann~F.} Data fusion~// ACM Comput. Surv., 2008. Vol.~41. No.\,1. 
P.~1--41. doi: 10.1145/ 1456650.1456651.
\bibitem{7-ste}
\Au{Rekatsinas T., Joglekar~M., Garcia-Molina~H., Parameswaran~A., Re~C.} Slimfast: 
Guaranteed results for data fusion and source reliability~// Conference (International) on Management of Data Proceedings.~--- New York, 
NY, USA: ACM, 2017. P.~1399--1414. doi: 10.1145/ 3035918.3035951.
\end{thebibliography}

 }
 }

\end{multicols}

\vspace*{-6pt}

\hfill{\small\textit{Поступила в~редакцию 29.09.23}}

\vspace*{8pt}

%\pagebreak

%\newpage

%\vspace*{-28pt}

\hrule

\vspace*{2pt}

\hrule



\def\tit{AN EXTENSIBLE APPROACH TO~DATA FUSION\\ IN~DISTRIBUTED 
COMPUTING ENVIRONMENTS}


\def\titkol{An extensible approach to~data fusion in~distributed 
computing environments}


\def\aut{V.\,V.~Sazontev, S.\,A.~Stupnikov, and~V.\,N.~Zakharov}

\def\autkol{V.\,V.~Sazontev, S.\,A.~Stupnikov, and~V.\,N.~Zakharov}

\titel{\tit}{\aut}{\autkol}{\titkol}

\vspace*{-14pt}


\noindent
Federal Research Center ``Computer Science and Control'' of the Russian Academy 
of Sciences, 44-2~Vavilov Str., Moscow 119333, Russian Federation


\def\leftfootline{\small{\textbf{\thepage}
\hfill INFORMATIKA I EE PRIMENENIYA~--- INFORMATICS AND
APPLICATIONS\ \ \ 2023\ \ \ volume~17\ \ \ issue\ 4}
}%
 \def\rightfootline{\small{INFORMATIKA I EE PRIMENENIYA~---
INFORMATICS AND APPLICATIONS\ \ \ 2023\ \ \ volume~17\ \ \ issue\ 4
\hfill \textbf{\thepage}}}

\vspace*{2pt}





\Abste{The paper belongs to the area of development of methods and tools for data 
integration. One of the most important stages of data integration is data fusion, i.\,e., the
combination of records relating to the same real-world entity into a~single record 
with conflict resolution for each of the attributes. The paper considers the formal statement of 
the data fusion problem, provides a~brief review of major groups of data fusion 
methods. An approach for implementation of the data fusion stage within an 
extensible heterogeneous data integration system in a distributed computing 
environment is proposed. Software architecture and basic implementation ideas of 
the approach are considered.}

\KWE{data fusion; distributed computing environment}




\DOI{10.14357/19922264230406}{LCQAZX}

\vspace*{-20pt}

\Ack

\vspace*{-5pt}

\noindent
The work was supported by the Russian Science Foundation, project  
No.\,22-21-00692 ({\sf  
https://rscf.ru/project/22-21-00692/}).

\vspace*{-4pt}

  \begin{multicols}{2}

\renewcommand{\bibname}{\protect\rmfamily References}
%\renewcommand{\bibname}{\large\protect\rm References}

{\small\frenchspacing
 {%\baselineskip=10.8pt
 \addcontentsline{toc}{section}{References}
 \begin{thebibliography}{9} 
 
\vspace*{-5pt}
 
\bibitem{1-ste-1}
\Aue{Dong, X.\,L., and D.~Srivastava.} 2015. \textit{Big Data integration}.  
Synthesis lectures on data management ser. Morgan \&~Claypool Publs. 
178~p.
\bibitem{2-ste-1}
\Aue{Dong, X.\,L., and T.~Rekatsinas.} 2018. Data integration and machine 
learning: A~natural synergy. \textit{Conference (International) on Management of 
Data}. New York, NY: ACM. 1645--1650. doi: 10.1145/3183713.3197387.
\bibitem{3-ste-1}
\Aue{Sazontev, V.\,V., and S.\,A.~Stupnikov.} 2023. An extensible approach to 
searching and selecting data sources for materialized Big Data integration in 
distributed computing environments. \textit{Pattern Recognition Image Analysis} 
33(2):147--156. doi: 10.1134/S1054661823020141. EDN: YXUMDO.
\bibitem{4-ste-1}
\Aue{Sazontev, V.\,V.} 2018. Methods for Big Data integration in distributed 
computation environments. \textit{CEUR Workshop Procee.} 2277:238--244.

%\pagebreak


\bibitem{5-ste-1}
\Aue{Sazontev, V., and S.~Stupnikov.} 2019. An extensible approach for 
materialized Big Data integration in distributed
 computation environments. 
\textit{Ivannikov Memorial Workshop}. IEEE. 33--38. doi: 
10.1109/IVMEM.2019.00011. EDN: BWILRL.

\pagebreak 

\bibitem{6-ste-1}
\Aue{Bleiholder, J., and F.~Naumann.} 2008. Data fusion. \textit{ACM Comput. 
Surv.} 41(1):1--41. doi: 10.1145/1456650.1456651.
\bibitem{7-ste-1}
\Aue{Rekatsinas, T., M.~Joglekar, H.~Garcia-Molina, A.~Parameswaran, and 
C.~Re.} 2017. SLiMFast: Guar-\linebreak

\columnbreak

\noindent
anteed results for data fusion and source reliability. 
\textit{Conference (International) on Management of Data Proceedings}. 
New York, NY: ACM. 1399--1414. doi: 10.1145/ 3035918.3035951.

\end{thebibliography}

 }
 }

\end{multicols}

\vspace*{-6pt}

\hfill{\small\textit{Received September 29, 2023}} 

%\vspace*{-18pt}

\Contr

\vspace*{-4pt}

\noindent
\textbf{Sazontev Vladimir V.} (b.\ 1990)~--- programmer, Federal Research Center 
``Computer Science and Control'' of the Russian Academy of Sciences,  
44-2~Vavilov Str., Moscow 119333, Russian Federation; \mbox{vladimirsvsite@gmail.com}
    
\vspace*{3pt}
    
\noindent
    \textbf{Stupnikov Sergey A.} (b.\ 1978)~--- Candidate of Science (PhD) in 
technology, leading scientist, Federal Research Center ``Computer Science and 
Control'' of the Russian Academy of Sciences, 44-2~Vavilov Str., Moscow 
119333, Russian Federation; \mbox{sstupnikov@ipiran.ru}
    
\vspace*{3pt}

\noindent
    \textbf{Zakharov Victor N.} (b.\ 1948)~--- Doctor of Science in technology, 
associate professor, scientific secretary, Federal Research Center ``Computer 
Science and Control'' of the Russian Academy of Sciences, 44-2~Vavilov Str., 
Moscow 119333, Russian Federation; \mbox{vzakharov@ipiran.ru}





\label{end\stat}

\renewcommand{\bibname}{\protect\rm Литература} 