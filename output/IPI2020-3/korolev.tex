\newcommand{\tod}{\stackrel{d}{\longrightarrow}}


\def\stat{korolev}

\def\tit{О РАСПРЕДЕЛЕНИИ ОТНОШЕНИЯ СУММЫ ЭЛЕМЕНТОВ ВЫБОРКИ,
ПРЕВОСХОДЯЩИХ НЕКОТОРЫЙ ПОРОГ, К~СУММЕ~ВСЕХ~ЭЛЕМЕНТОВ
ВЫБОРКИ.~I$^*$}

\def\titkol{О распределении отношения суммы элементов выборки,
превосходящих некоторый порог, к~сумме всех элементов}
%выборки.~I}

\def\aut{В.\,Ю.~Королев$^1$}

\def\autkol{В.\,Ю.~Королев}

\titel{\tit}{\aut}{\autkol}{\titkol}

\index{Королев В.\,Ю.}
\index{Korolev V.\,Yu.}

 

{\renewcommand{\thefootnote}{\fnsymbol{footnote}} \footnotetext[1]
{Работа выполнена при частичной финансовой поддержке РФФИ (проект
19-07-00914) и в соответствии с программой
 Московского центра фундаментальной и~прикладной математики.}}


\renewcommand{\thefootnote}{\arabic{footnote}}
\footnotetext[1]{Факультет вычислительной математики и~кибернетики Московского 
государственного университета имени М.\,В.~Ломоносова; 
Институт проб\-лем информатики Федерального исследовательского центра 
<<Информатика и~управ\-ле\-ние>> Российской академии наук, 
\mbox{vkorolev@cs.msu.ru}}

\vspace*{-6pt}




\Abst{Рассматривается задача описания распределения доли суммы 
независимых случайных величин, которая составлена из слагаемых, 
превосходящих некоторый заданный порог. 
В~отличие от известных вариантов такой задачи, в~которых фиксируется
 число суммируемых крайних порядковых статистик, особенность 
 рассматриваемой здесь задачи состоит в~том, что заданный порог 
 может быть превзойден непредсказуемым числом элементов выборки. 
 В~статье в~терминах функции распределения отдельного слагаемого 
 формально представлен явный вид распределения отношения суммы 
 элементов выборки, превосходящих заданный порог, к~сумме всех наблюдений. 
 На эвристическом уровне выведены асимптотические и~предельные распределения 
 этого отношения при фиксированном пороге, удобные для использования в~качестве 
 асимптотических аппроксимаций в~практических вычислениях. 
 Рассмотрены ситуации, в~которых распределение слагаемых имеет легкий хвост 
 (конечны вторые моменты), и~ситуации, в~которых распределение слагаемых 
 имеет тяжелый хвост (принадлежит к~области притяжения устойчивого закона). 
 Во всех случаях описана нормировка отношения, гарантирующая невырожденность 
 предельного (при неограниченном увеличении числа слагаемых) распределения, 
 и~само предельное распределение (нормальное в~случае легких 
 хвостов и~устойчивое в~случае тяжелых хвостов).}

\KW{сумма независимых случайных величин; случайная сумма; 
биномиальное распределение; смесь распределений вероятностей; нормальное 
распределение; устойчивое распределение; экстремальная порядковая статистика}

\DOI{10.14357/19922264200305} 

\vspace*{-2pt}

\vskip 10pt plus 9pt minus 6pt

\thispagestyle{headings}

\begin{multicols}{2}

\label{st\stat}


\section{Постановка задачи}

\vspace*{-4pt}

В 1897~г.\ итальянский экономист и~социолог Вильфридо Парето 
выявил эмпирическую закономерность, заключающуюся в~том, что~80\%
дохода страны аккумулируются в~20\%
семей~\cite{Koch1998}. Эту закономерность многие пытались обосновать 
или опровергнуть. В~данной заметке предпринята попытка рассмотреть 
связанную с~этим принципом ве\-ро\-ят\-ност\-но-ста\-ти\-сти\-че\-скую 
задачу о~том, какую долю суммы наблюдений составляют наблюдения, 
превосходящие заданный порог.

Эта задача имеет большое значение не только для экономики, но и~для 
других областей знания. Например, прослеживая изменение во времени 
параметров распределения отношения суммы элементов выборки, 
превосходящих некоторый порог,\linebreak к~сумме всех элементов выборки при 
исследовании метеорологических данных (температура, осадки, теплообмен 
между атмосферой и~океаном) во времени (например, в~скользящем режиме, 
когда выборка~--- это <<окно>>, сдвигающееся в~направлении астрономического 
времени при исследовании соответствующего временн$\acute{\mbox{о}}$го ряда), 
можно получить информацию об особенностях проявления процесса изменения 
климата, в~просторечии называемого <<глобальным по\-теп\-ле\-ни\-ем>>.

С математической точки зрения эта задача тесно связана со 
статистикой цензурированных выборок. Известны разные варианты этой задачи.
Например, в~работах~[2--10] рассмотрена задача о~распределении
суммы заданного числа крайних порядковых статистик. 
Однако в~опубликованных на эту тему работах рассматривались либо
конкретные распределения, либо теоретические результаты типа законов 
больших чисел.

Пусть $n\hm\in\mathbb{N}$, $X_1,\ldots,X_n$~--- выборка, т.\,е.\ набор
независимых одинаково распределенных случайных величин с~общей
функцией распределения $F(x)\hm={\sf P}(X_1\hm<x)$, $x\hm\in\mathbb{R}$. Без
существенного ограничения общности для удобства будем считать, что~$F(x)$ 
непрерывна и~$F(0)\hm=0$, т.\,е.\ ${\sf P}(X_1\hm\ge0)\hm=1$. 
Пусть $u\hm\in(0,\infty)$~--- рассматриваемый порог.

Пусть $X_{(1)},\ldots,X_{(n)}$~--- вариационный ряд, 
построенный по выборке $X_1,\ldots,X_n$. В~качестве 
приближенного решения указанной задачи можно рассмотреть 
распределение случайной величины
$
{\sum\nolimits_{j=[nF(u)]}^nX_{(j)}}/{\sum\nolimits_{j=1}^nX_j}$.
Здесь в~числителе стоит сумма крайних порядковых статистик, 
начиная с~выборочной квантили порядка~$F(u)$, которая при $n\hm\to\infty$ 
сходится к~теоретической квантили порядка~$F(u)$, т.\,е.\ к~порогу~$u$.

Еще один подход к~решению общей задачи заключается в~учете того, 
что число $N_n(u)$ элементов выборки, превосходящих порог~$u$, 
является случайной величиной, имеющей биномиальное 
распределение с~параметрами~$n$ и~$1\hm-F(u)$. Тогда 
отношение суммы элементов выборки, превосходящих порог~$u$, к~сумме 
всех наблюдений принимает вид:
$
{\sum\nolimits_{j=n-N_n(u)+1}^nX_{(j)}}/{\sum\nolimits_{j=1}^nX_j}
$.
Для изучения распределения этой случайной величины в~некоторых частных 
случаях можно было бы применить результаты 
работ~\cite{3-kor, 5-kor, 7-kor, 9-kor, 12-kor}, 
но, к~сожалению, в~приведенных 
выше отношениях участвуют случайные величины, не являющиеся независимыми.

Чтобы обойти указанное препятствие, в~данной статье использован 
альтернативный подход. Отметим, что, в~отличие от упоминавшихся 
выше работ, в~рассматриваемой здесь задаче фиксируется не число 
суммируемых крайних порядковых статистик, а порог, который может быть 
пре\-взойден непредсказуемым числом элементов выборки. Ниже будет 
формально представлен явный вид распределения отношения суммы 
элементов выборки, превосходящих заданный порог, к~сумме всех 
наблюдений и~на эвристическом уровне будут выведены 
асимптотические и~предельные распределения этого отношения 
при фиксированном пороге, удобные для использования в~качестве 
асимптотических аппроксимаций в~практических вычислениях.

Обозначим $S_n\hm=X_1+\cdots+X_n$. Индикатор множества (события)~$A$
обозначим~$\mathbb{I}(A)$.

Пусть $u>0$ таково, что $0\hm<F(u)\hm<1$. Очевидно, 
$X_j\hm=X_j\mathbb{I}(X_j\hm<u)\hm+X_j\mathbb{I}(X_j\hm\ge u)$. Тогда
\begin{multline*}
S_n=\sum\limits_{j=1}^n X_j\mathbb{I}\left(X_j<u\right)+
\sum\limits_{j=1}^n X_j\mathbb{I}\left(X_j\ge u\right)\equiv{}\\
{}\equiv
S_n^{(<u)}+S_n^{(\ge u)}.
\end{multline*}

Основным объектом изучения будут распределения случайных 
величин $S_n^{(\ge u)}$ и~$S_n^{(< u)}$, а~в~первую очередь~--- 
распределение отношения
$R(u)\hm=S_n^{(\ge u)}/S_n$. Очевидно, что $0\hm\le R(u)\hm\le 1$ и

\noindent
$$
R(u)=\left(1+\fr{S_n^{(<u)}}{S_n^{(\ge u)}}\right)^{-1}.
$$
Поэтому если обозначить 
$$
\underline{W}_u(r)\hm={\sf P}(S_n^{(<u)}
\hm<rS_n^{(\ge u)})\,, \enskip r\hm\ge 0\,,
$$
то для $r\hm\in[0,1]$
$$
{\sf P}(R(u)<r)=1-\underline{W}_u\left(\fr{1}{r}-1\right).
$$
Так что сосредоточимся на описании функций распределения 
$\underline{W}_u(r)$ и~$\overline{W}_u(r)$:
$$\overline{W}_u(r)\hm\equiv {\sf P}(S_n^{(\ge u)}
\hm<rS_n^{(<u)})=1-\underline{W}_u\left(\fr{1}{r}\right)\,.
$$

\vspace*{-6pt}

\section{Точное решение}

\vspace*{-4pt}

Очевидно,

\vspace*{-2pt}

\noindent
$$
X_j\mathbb{I}(X_j\ge u)=
\begin{cases}0,& X_j< u;\\
X_j,& X_j\ge u.\end{cases}
$$
Если условиться считать равенство нулю индикатора $\mathbb{I}(X_j<u)$
<<неудачей>>, а противоположное событие~--- <<успехом>>, то число
$N_n(u)$ ненулевых слагаемых в~сумме~$S_n^{(\ge u)}$ будет случайной
величиной, имеющей биномиальное распределение с~па\-ра\-мет\-ра\-ми~$n$ 
и~$1\hm-F(u)$. Таким образом, для $x\hm\ge 0$
\begin{equation}
{\sf P}\left(S_n^{(\ge u)}<x\right)=
{\sf P}\left(\sum\limits_{j=0}^{N_n(u)}X_j^{(\ge
u)}<x\right),
\label{e1-kor}
\end{equation}
где случайные величины $X_1^{(\ge u)},X_2^{(\ge u)},\ldots$
независимы и~имеют одну и~ту же функцию распределения:

\noindent
\begin{multline*}
\!F^{(\ge u)}(x)\!\equiv\! {\sf P}\left(X_1^{(\ge u)}<x\right)=
1-{\sf P}\left(X_1^{(\ge u)}\ge x\right)={}\\
{}=
1-{\sf P}\left(X_1\mathbb{I}(X_1\ge u)\ge x|X_1\ge u\right)={}\\
{}=1-\fr{{\sf P}\left(X_1\ge x,\,X_1\ge u\right)}{{\sf P}\left(X_1\ge
u\right)}={}\\
{}=
\begin{cases}0, & x<u\,;\\
\displaystyle\fr{F(x)-F(u)}{1-F(u)}, & x\ge u
\end{cases}
={}\\
{}=\fr{F(\max\{x,u\})-F(u)}{1-F(u)}.
\end{multline*}

\vspace*{-2pt}

\noindent
При этом случайная величина $N_n(u)$ независима от
последовательности $X_1^{(\ge u)},X_2^{(\ge u)},\ldots$

Аналогично для $x\hm\ge 0$

\vspace*{3pt}

\noindent
\begin{equation}
{\sf P}\left(S_n^{(<u)}<x\right)={\sf
P}\left(\sum\limits_{j=0}^{n-N_n(u)}X_j^{(<u)}<x\right),
\label{e2-kor}
\end{equation}
где случайные величины $X_1^{(<u)},X_2^{(<u)},\ldots$ 
независимы и~имеют одну и~ту же функцию распределения:

\noindent
\begin{multline*}
F^{(<u)}(x)\equiv {\sf P}\left(X_1^{(<u)}<x\right)={}\\
{}={\sf P}\left(X_1\mathbb{I}(X_1<u)<x|X_1<u\right)={}
\\
{}=\fr{{\sf P}\left(X_1<x,\,X_1<u\right)}{{\sf P}(X_1<u)}=
\begin{cases}
1, & x\ge u;\\
\displaystyle\fr{F(x)}{F(u)}, & 0\le x< u
\end{cases}={}\\
{}=
\fr{F(\min\{x,u\})}{F(u)}.
\end{multline*}

%\vspace*{24pt}

При этом случайная величина $N_n(u)$ также независима от
последовательности $X_1^{(<u)},X_2^{(<u)},\ldots$ Более того,
случайные величины $X_1^{(\ge u)},X_2^{(\ge
u)},\ldots,X_1^{(<u)},X_2^{(<u)},\ldots$ независимы в~совокупности.

Таким образом, для $r\hm\ge 0$ по формуле полной вероятности имеем:
\begin{multline}
\underline{W}_u(r)={\sf P}\left(\sum\limits_{j=0}^{n-N_n(u)}X_j^{(<u)}<
r\sum\limits_{j=0}^{N_n(u)}X_j^{(\ge u)}\right) ={}
\\
{}=\sum\limits_{k=0}^nC_n^k\left(1-F(u)\right)^k(F(u))^{n-k}\times{}\\
{}\times {\sf P}
\left(\sum\limits_{j=0}^{n-k}X_j^{(<u)}<r\sum\limits_{j=0}^{k}X_j^{(\ge
u)}\right).
\label{e3-kor}
\end{multline}
Следует особо отметить, что в~последнем соотношении под 
знаком вероятности стоят суммы случайных величин,
\textit{независимых в~совокупности}.\linebreak
Имеем:
\begin{multline*}
\underline{W}_u(r)=\sum\limits_{k=0}^nC_n^k(1-F(u))^k(F(u))^{n-k}\times{}\\
{}\times
\int\limits_0^{\infty}B^{(<u)}_{n-k}(rz)\,dB^{(\ge u)}_k(z),
\end{multline*}
где для $m\ge0$
\begin{gather*} 
B^{(<u)}_m(x)=(F(u))^{-m}F^{\blacktriangledown m}(x),\\
F^{\blacktriangledown 0}(x)=
\begin{cases}
0, & x\le0;\\
1, &x>0,
\end{cases}\\  
F^{\blacktriangledown
(m+1)}(x)=\int\limits_{0}^{\min\{x,u\}}F^{\blacktriangledown m}(x-y)dF(y)\,;
\end{gather*}

%\columnbreak

\noindent
\begin{gather*}
B^{(\ge u)}_m(x)=(1-F(u))^{-m}F^{\blacktriangle m}(x),\\
F^{\blacktriangle 0}(x)=
\begin{cases}
0, & x\le0;\\
1, & x>0,
\end{cases}\\
F^{\blacktriangle
(m+1)}(x)=\!
%{}\\{}=
\int\limits_{u}^{\infty}\!
\left[F(\max\{x-y,u\})\hspace*{-.7pt}-\hspace*{-.7pt}F(u)\right]dF^{\blacktriangle m}(y).
\end{gather*}
Тогда
\begin{equation}
\underline{W}_u(r)=\sum\limits_{k=0}^n C_n^k\int\limits_0^{\infty}
F^{\blacktriangledown (n-k)}(rz)\,dF^{\blacktriangle k}(z).\label{e4-kor}
\end{equation}
Таким образом, в~принципе, зная функцию распределения~$F(x)$, по
формуле~(\ref{e4-kor}) можно вычислить функцию распределения~$\underline{W}_u(r)$.

Однако очевидно, что для непосредственного использования формулы~(\ref{e4-kor})
на практике придется применять громоздкие численные процедуры.\linebreak
Поэтому наиболее целесообразный подход заключается в~применении
асимптотических аппроксимаций, основанных на 
представлениях~(\ref{e1-kor})--(\ref{e3-kor}) и~получаемых с~помощью 
предельных теорем для сумм случайного числа случайных величин.

\section{Приближенные решения}


\subsection{Асимптотические аппроксимации для~распределения
от\-но\-ше\-ния $S_n^{(\ge u)}/S_n^{(< u)}$ при~фиксированном~$u$}

Поскольку случайные величины $S_n^{(\ge u)}$ и~$S_n^{(< u)}$ не
являются независимыми, изучение асимптотического распределения
отношения $S_n^{(\ge u)}/S_n^{(< u)}$\linebreak довольно трудоемко. Однако
если хвосты распределения случайных величин достаточно легки
(существуют математические ожидания), то при боль\-шом~$n$ некоторое
представление об этом отношении даст оно само, так как по закону
больших чисел это отношение будет стабилизироваться около некоторого
числа.

Для более аккуратных выводов, учитывающих вероятностные
 свойства рассматриваемого отношения, приведем более 
 подробные рассуждения.

Пусть $u$ фиксировано. Обозначим: 
$$
p\hm=p(u)\hm=1\hm-F(u)\,.
$$
Пусть $\epsilon\hm\in(0,1)$~--- малое число. Для $r\hm\ge 0$ по 
формуле полной вероятности имеем:

\noindent
\begin{multline*}
\underline{W}_u(r)={\sf P}\left(\sum\limits_{j=0}^{n-N_n(u)}
X_j^{(<u)}<r\sum\limits_{j=0}^{N_n(u)}X_j^{(\ge u)}\right)={}
\\
{}=\sum\limits_{k:\,\left\vert {k}/({np})-1\right\vert \le\epsilon}
C_n^k(1-F(u))^k(F(u))^{n-k}
\times{}\\
{}\times {\sf P}\left(
\sum\limits_{j=0}^{n-k}X_j^{(<u)}<r\sum\nolimits_{j=0}^{k}X_j^{(\ge u)}\right)+{}
\end{multline*}

\noindent
\begin{multline}
{}+\sum\limits_{k:\,\left\vert {k}/({np})-1\right\vert
>\epsilon}C_n^k(1-F(u))^k(F(u))^{n-k}
\times{}\\
{}\times {\sf P}\left(\sum\limits_{j=0}^{n-k}
X_j^{(<u)}<r\sum\limits_{j=0}^{k}X_j^{(\ge u)}\right)\equiv I_1+I_2. 
\label{e5-kor}
\end{multline}
Рассмотрим $I_2$. Очевидно, ${\sf E}N_n(u)\hm=np(u)$, ${\sf D}N_n(u)
\hm=np(u)\left(1\hm-p(u)\right)$. Тогда по неравенству Чебышёва имеем:
\begin{multline*}
I_2\le \sum\limits_{k:\,\left\vert {k}/({np})-1\right\vert
>\epsilon}C_n^k(1-F(u))^k(F(u))^{n-k}={}\\
{}={\sf P}\left(\left\vert \fr{N_n(u)}{np(u)}-1\right\vert >\epsilon\right)\le
\fr{1-p(u)}{np(u)\epsilon^2}={}\\
{}=\fr{F(u)}{n\left(1-F(u)\right)\epsilon^2}\,.
\end{multline*}
Таким образом, если $u$ фиксировано, то $I_2\hm=O(n^{-1})
\hm\longrightarrow 0$ при $n\hm\to\infty$.

Рассмотрим $I_1$. Если $k\hm\in\left\{k:\,\left\vert 
{k}/({np})-1\right\vert \le\epsilon\right\}$, то 
$np(1\hm-\epsilon)\hm\le k$ и, следовательно,
$$
\min\left\{k:\, \left\vert \fr{k}{np}-1\right\vert 
\le\epsilon\right\}\longrightarrow \infty
$$
при $n\to\infty$. Одновременно если 
$k\hm\in\left\{k:\,\left\vert {k}/({np})-1\right\vert \hm\le\epsilon\right\}$, 
то $k\hm\le np(1\hm+\epsilon)$ и, следовательно, $n-k
\hm\ge n[1\hm-p(1\hm+\epsilon)]$. Поэтому если
$$
\epsilon<\min\left\{1,\,\fr{F(u)}{1-F(u)}\right\},
$$
то $p(1+\epsilon)<1$ и~$$
\min\left\{n-k:\, \left\vert \fr{k}{np}-1\right\vert \le\epsilon\right\}
\longrightarrow \infty
$$
при $n\to\infty$. Таким образом, при надлежащем выборе~$\epsilon$ в~$I_1$ 
в~обеих суммах под знаком вероятности в~(\ref{e5-kor})
 число слагаемых неограниченно возрастает при $n\hm\to\infty$.

Случайные величины~$X_j^{(<u)}$ ограничены. Поэтому существуют 
${\sf E}X_j^{(<u)}\hm=\underline{a}_u$ и~${\sf D}X_j^{(<u)}
\hm=\underline{\sigma}^2_u$. При этом при больших~$n$ для 
$k\hm\in\left\{k:\,|{k}/({np})-1|\le\epsilon\right\}$
$$
{\sf P}\left(\sum\limits_{j=1}^{n-k}X_j^{(<u)}<x\right)
\approx\Phi\left(\fr{x-(n-k)\underline{a}_u}
{\underline{\sigma}_u\sqrt{n-k}}\right).
$$
Предположим, что функция распределения $F^{(\ge u)}(x)$ 
принадлежит к~области притяжения некоторого устойчивого 
распределения~$G_{\alpha}(x)$ с~характеристическим показателем
 $\alpha\hm\in(0,2]$. Это означает, что существуют числа~$c_k$ 
 и~$d_k\hm>0$ такие, что при больших~$n$ для 
 $k\hm\in\left\{k:\,|{k}/({np})-1|\le\epsilon\right\}$
 
\columnbreak

 \noindent
$$
{\sf P}\left(\sum\limits_{j=1}^{k}X_j^{(\ge u)}<x\right)
\approx G_{\alpha}\left(\fr{x-c_k}{d_k}\right).
$$
Тогда при $n\hm\to\infty$
\begin{multline}
\underline{W}_u(r)={\sf P}\left(\sum\limits_{j=0}^{n-N_n(u)}
X_j^{(<u)}<r\sum\limits_{j=0}^{N_n(u)}X_j^{(\ge u)}\right)={}
\\
{}=\sum\limits_{k:\,\left\vert
{k}/({np})-1\right\vert |\le\epsilon}
C_n^k(1-F(u))^k(F(u))^{n-k}\times{}\\
{}\times{\sf P}\left(
\sum\limits_{j=0}^{n-k}X_j^{(<u)}<r\sum\limits_{j=0}^{k}X_j^{(\ge u)}\right)+
O\left(\fr{1}{n}\right)\approx{}
\\
{}\approx\sum\limits_{k:\,\left\vert {k}/({np})-1\right\vert
\le\epsilon} C_n^k(1-F(u))^k(F(u))^{n-k}\times{}\\
{}\times \int\limits_{0}^{\infty}
\Phi\left(\fr{rz-(n-k)\underline{a}_u} {\underline{\sigma}_u\sqrt{n-k}}\right)dG_{\alpha}
\left(\fr{z-c_k}{d_k}\right)\approx{}
\\
{}\approx\sum\limits_{k=0}^nC_n^k(1-F(u))^k(F(u))^{n-k}\times{}\\
{}\times
\int\limits_{0}^{\infty}\Phi\left(\fr{r-(({n-k})/{z}) \underline{a}_u}
{\underline{\sigma}_u ({\sqrt{n-k}}/{z})}\right)dG_{\alpha}
\left(\fr{z-c_k}{d_k}\right),
\label{e6-kor}
\end{multline}
а если существует ${\sf D}X_j^{(\ge u)}\hm=\overline{\sigma}^2_u$, 
то $\alpha\hm=2$, $G_{\alpha}\hm=\Phi$, $c_k\hm=
k{\sf E}X_j^{(\ge u)}\hm\equiv k\overline{a}_u$, $d_k\hm=
\overline{\sigma}_u\sqrt{k}$ и~при $n\hm\to\infty$
\begin{multline}
\underline{W}_u(r)\approx \underline{W}_u^{(n)}(r)\equiv
\sum\limits_{k=0}^nC_n^k(1-F(u))^k(F(u))^{n-k}\times{}\\
{}\times \int\limits_{0}^{\infty}
\Phi\left(\fr{r-(({n-k})/{z})\underline{a}_u}
{\underline{\sigma}_u ({\sqrt{n-k}}/{z})}\right)d\Phi
\left(\fr{z-k\overline{a}_u}{\overline{\sigma}_u\sqrt{k}}\right),
\label{e7-kor}
\end{multline}
где для определенности сумма с~нулевым числом слагаемых полагается 
равной нулю. Таким образом, в~данном случае при больших~$n$ 
функция распределения~$\underline{W}_n(r)$ может быть приближена 
специальной сдвиг-мас\-штаб\-ной смесью нормальных законов, в~ ко\-то\-рой 
смешивающее распределение (вообще говоря, двумерное) может иметь 
тяжелый хвост по своей непрерывной переменной.

На практике можно использовать приближение функции 
распределения~$\underline{W}_n(r)$ конечной смесью нормальных законов, 
параметры которой статистически оцениваются, скажем, с~по\-мощью 
ЕМ~(expectation-maximization) ал\-го\-рит\-ма с~учетом возможности оценить значение~$F(u)$ 
соответствующим значением эмпирической функции распределения.

%\vspace*{-24pt}

\subsubsection{Случай легких хвостов}

Если объем выборки $n$ велик, то аккуратные вычисления по формулам~(\ref{e6-kor}) 
и~(\ref{e7-kor}) затруднены из-за возможных искажений окончательного
 результата вследствие накапливающихся погрешностей. Поэтому дальнейшей 
 целью будет получение аппроксимирующих распределений, не зависящих от~$n$ 
 и~адекватных при больших значениях объема выборки.

Пусть $\xi_0$ и~$\xi_1$~--- %независимые 
случайные величины с~одним и~тем же 
стандартным нормальным распределением, независимые от случайной 
величины~$N_n(u)$. Тогда функция распределения~$\underline{W}_u^{(n)}(r)$ 
(см.~(\ref{e7-kor})) соответствует случайной величине
\begin{multline*}
R_u^{(n)}=\fr{\underline{\sigma}_u\sqrt{n-N_n(u)}
\,\xi_0+(n-N_n(u))\underline{a}_u}
{\overline{\sigma}_u\sqrt{N_n(u)}\,\xi_1+N_n(u)\overline{a}_u}={}\\
{}=
\left({\sqrt{1-\fr{N_n(u)}{n}}\,\fr{1}{\sqrt{n}} \,
\underline{\sigma}_u \xi_0+\left(1-\fr{N_n(u)}{n}\right)\underline{a}_u}\right)\times{}\\
{}\times
\left(
{\sqrt{\fr{N_n(u)}{n}}\, \fr{1}{\sqrt{n}}\,
\overline{\sigma}_u \xi_1+ \fr{N_n(u)}{n}\, \overline{a}_u}\right)^{-1}
\longrightarrow
{}\\
{}\longrightarrow
\fr{F(u)}{1-F(u)}\, \fr{\underline{a}_u}{\overline{a}_u}=
\fr{\int\nolimits_{0}^{u}x\,dF(x)}{\int\nolimits_{u}^{\infty}x\,dF(x)}
\end{multline*}
по вероятности при $n\hm\to\infty$. Поэтому в~качестве статистической 
оценки математического ожидания отношения суммы наблюдений, меньших порога~$u$, 
к~сумме наблюдений, превосходящих этот порог, можно взять само это отношение: 
если $F_n(x)$~--- эмпирическая функция распределения, построенная по 
выборке $X_1,\ldots,X_n$, $F_n(x)\hm=({1}/{n})\sum\nolimits_{j=1}^n
\mathbb{I}(X_j<x)$, $x\hm\in\mathbb{R}$,
то
\begin{multline*}
\fr{\int\nolimits_{0}^{u}x\,dF(x)}{\int\nolimits_{u}^{\infty}x\,dF(x)}\approx
\fr{\int\nolimits_{0}^{u}x\,dF_n(x)}{\int\nolimits_{u}^{\infty}x\,dF_n(x)}={}\\
{}=
\fr{\sum\nolimits_{j=1}^nX_j\mathbb{I}(X_j<u)}
{\sum\nolimits_{j=1}^nX_j\mathbb{I}(X_j\ge u)}\,.
\end{multline*}

Для более точных выводов, выполняя очевидные преобразования и~применяя 
теорему Муав\-ра--Лап\-ла\-са, получаем:

\noindent
\begin{multline*}
\sqrt{n}\left(R_u^{(n)}-\fr{F(u)}{1-F(u)}\,
\fr{\underline{a}_u}{\overline{a}_u} \right)={}\\[6pt]
{}=
\fr{\sqrt{1-{N_n(u)}/{n}}\,\underline{\sigma}_u
\xi_0+\sqrt{n}(1-{N_n(u)}/{n})\underline{a}_u}
{\sqrt{{N_n(u)}/{n}}\,({1}/{\sqrt{n}}) 
\overline{\sigma}_u \xi_1+({N_n(u)}/{n})\overline{a}_u}
-{}
\end{multline*}

\noindent
\begin{multline*}
\hspace*{20mm}{}-\fr{\sqrt{n}F(u)}{1-F(u)}\,\fr{\underline{a}_u}{\overline{a}_u}={}
\\[6pt]
{}=\fr{\sqrt{1-{N_n(u)}/{n}}\,\underline{\sigma}_u \xi_0}
{\sqrt{{N_n(u)}/{n}}\,({1}/{\sqrt{n}})\overline{\sigma}_u
\xi_1+({N_n(u)}/{n}) \overline{a}_u}
+{}\\[6pt]
{}+\fr{\sqrt{n}(1-{N_n(u)}/{n})\underline{a}_u}
{\sqrt{{N_n(u)}/{n}}\,({1}/{\sqrt{n}})\overline{\sigma}_u
\xi_1+({N_n(u)}/{n})\overline{a}_u}
-{}\\[6pt]
{}-\fr{\sqrt{n}\,F(u)}{1-F(u)}\,\fr{\underline{a}_u}{\overline{a}_u}=
\fr{\sqrt{F(u)}\underline{\sigma}_u}
{(1-F(u))\overline{a}_u+o_P(1)}\,\xi_0+{}\\[6pt]
{}+
\fr{({n-N_n}/{\sqrt{n}})\underline{a}_u}
{(1-F(u))\overline{a}_u+o_P(1)}
-\fr{\sqrt{n}F(u)}{1-F(u)}\,
\fr{\underline{a}_u}{\overline{a}_u}+o_P(1)={}
\\[6pt]
{}=\fr{\sqrt{F(u)}\,\underline{\sigma}_u}{(1-F(u))\overline{a}_u+o_P(1)}\,
\xi_0+\fr{\sqrt{n}F(u)\underline{a}_u}
{(1-F(u))\overline{a}_u+o_P(1)}
-{}\\[6pt]
{}-\fr{\sqrt{n}F(u)}{1-F(u)}\,\fr{\underline{a}_u}{\overline{a}_u}-{}
\\[6pt]
{}-\fr{N_n-n(1-F(u))}{\sqrt{nF(u)(1-F(u))}}\,
\fr{\sqrt{F(u)}\,\underline{a}_u}
{\sqrt{1-F(u)}\,\overline{a}_u+o_P(1)}+{}\\[6pt]
{}+o_P(1)\tod
\fr{\sqrt{F(u)}\,\underline{\sigma}_u}{(1-F(u))\overline{a}_u} \xi_0+
\fr{\sqrt{F(u)}\,\underline{a}_u}
{\sqrt{1-F(u)}\,\overline{a}_u}\,\xi_1\eqd{}\\[6pt]
{}\eqd
\xi_0 \fr{1}{\overline{a}_u}\sqrt{\fr{F(u)}{1-F(u)}
\left(\fr{\underline{\sigma}^2_u}{1-F(u)}+
\underline{a}^2_u\right)},
\end{multline*}
где символом $o_P(1)$ обозначены разные случайные величины, 
сходящиеся к~нулю по вероятности при $n\hm\to\infty$, а символ~$\tod$ 
обозначает сходимость по распределению. Таким образом, в~случае
 легких хвостов распределения исходных случайных величин~$X_j$ (т.\,е.\
  существования дисперсий случайных величин~ $X_j^{(\ge u)}$) для любого~$x$
%\vspace*{-2pt}
\begin{multline*}
\lim\limits_{n\to\infty}{\sf P}\left(
\sqrt{n}\left(R_u^{(n)}-\fr{F(u)}{1-F(u)}\,
\fr{\underline{a}_u}{\overline{a}_u}\right)<x\right)={}\\
{}=
\Phi\left( x \overline{a}_u\left[\fr{F(u)}{1-F(u)}\left(
\fr{\underline{\sigma}^2_u}{1-F(u)}+
\underline{a}^2_u\right)\right]^{-1/2}\right).
\end{multline*}
Другими словами, в~указанных выше предположениях случайная 
величина~$R_u^{(n)}$ асимптотически нормальна с~математическим ожиданием
$$
\fr{F(u)}{1-F(u)}\, \fr{\underline{a}_u}{\overline{a}_u}=
\fr{\int\nolimits_{0}^{u}x\,dF(x)}{\int\nolimits_{u}^{\infty}x\,dF(x)}
$$

\vspace*{-9pt}

\pagebreak

\noindent
и дисперсией
\begin{multline*}
\fr{1}{\overline{a}_u^2n}\,\fr{F(u)}{1-F(u)}
\left(\fr{\underline{\sigma}^2_u}{1-F(u)}+
\underline{a}^2_u\right)={}\\
{}=\fr{1}{n}\,\fr{\int\nolimits_{0}^{u}
x^2\,dF(x)}{\left(\int\nolimits_{u}^{\infty}x\,dF(x)\right)^2}\,.
\end{multline*}

\subsubsection{Случай тяжелых хвостов}

Предположим, что распределение случайных величин $X_j^{(\ge u)}$ 
имеет тяжелый хвост, а~именно: будем считать, что это распределение 
принадлежит к~области нормального притяжения строго устойчивого 
распределения с~характеристическим показателем $\alpha\hm\in(0,2]$.

Сначала рассмотрим случай $\alpha\hm\in(1,2]$. 
В~такой ситуации у случайной величины~$X_j^{(\ge u)}$ существует 
математическое ожидание~$\overline{a}_u$ и~условие принадлежности ее 
распределения к~области нормального притяжения строго устойчивого
 распределения с~характеристическим показателем $\alpha\hm\in(1,2]$ 
 заключается в~существовании числа $b_u\hm\in(0,\infty)$, такого что
$$
\fr{1}{b_un^{1/\alpha}}\sum\limits_{j=1}^n\left(X_j^{(\ge u)}-
\overline{a}_u\right)\tod \zeta_{\alpha}
$$
при $n\to\infty$, где ${\sf P}(\zeta_{\alpha}\hm<x)\hm\equiv G_{\alpha}(x)$. 
Другими словами, в~(\ref{e6-kor}) $c_k\hm=k \overline{a}_u$ 
и~$d_k\hm=b_uk^{1/\alpha}$, так что функция распределения в~правой части~(\ref{e6-kor}) 
 соответствует случайной величине
$$
\bar{R}_u^{(n)}=\fr{\underline{\sigma}_u\sqrt{n-N_n(u)}\,\xi_0+
(n-N_n(u))\underline{a}_u}{b_u\left(N_n(u)\right)^{1/\alpha}
\zeta_{\alpha}+
N_n(u)\overline{a}_u}\,.
$$
Вместо случайной величины~$\bar{R}_u^{(n)}$ удобнее рассматривать 
обратную  случайную величину $\widetilde{R}_u^{(n)}
\hm=\left(\bar{R}_u^{(n)}\right)^{-1}$. Имеем:
\begin{multline*}
\widetilde{R}_u^{(n)}=\fr{b_uN_n(u)^{1/\alpha}\,\zeta_{\alpha}}
{\underline{\sigma}_u\sqrt{n-N_n(u)}\,\xi_0+(n-N_n(u))\underline{a}_u}+{}\\
{}+
\fr{({N_n(u)}/{n})\overline{a}_u}
{\underline{\sigma}_u\sqrt{({n-N_n(u)})/{n}}\,
({1}/{\sqrt{n}})\,\xi_0+
(1-{N_n(u)}/{n})\underline{a}_u}.\hspace*{-7.1pt}
\end{multline*}
При этом
\begin{multline*}
\fr{N_n(u)}{n}\,\overline{a}_u
\left(
\underline{\sigma}_u\sqrt{\fr{n-N_n(u)}{n}}\,
\fr{1}{\sqrt{n}}\,\xi_0+{}\right.\\
\left.\vphantom{\sqrt{\fr{n-N_n(u)}{n}}}{}+ 
\left(1-\fr{N_n(u)}{n}\right)\underline{a}_u\right)^{-1}
%\longrightarrow {}\\{}
\longrightarrow
\fr{(1-F(u))\overline{a}_u}{F(u)\underline{a}_u}
\end{multline*}
по вероятности при $n\hm\to\infty$. В~то же время
\begin{multline*}
n^{1-1/\alpha}\left(\widetilde{R}_u^{(n)}-\fr{(1-F(u))\overline{a}_u}
{F(u)\underline{a}_u}\right)={}\\
{}=
\fr{b_u\left({N_n(u)}/{n}\right)^{1/\alpha}\zeta_{\alpha}}
{\underline{\sigma}_u ({\sqrt{n-N_n(u)}}/{n})\xi_0+
\left(1-{N_n(u)}/{n}\right)\underline{a}_u}+{}\\
{}+o_P(1)={} \fr{b_u\left(1-F(u)\right)^{1/\alpha}\zeta_{\alpha}+o_P(1)}
{F(u)\underline{a}_u+o_P(1)}+o_P(1)\tod{}\\
{}\tod
\fr{b_u\left(1-F(u)\right)^{1/\alpha}}{F(u)
\underline{a}_u}\,\zeta_{\alpha}
\end{multline*}
при $n\to\infty$.

Пусть теперь $\alpha\in(0,1]$. В~такой ситуации у~слу\-чай\-ной 
величины~$X_j^{(\ge u)}$ отсутствует математическое ожидание. 
Будем считать, что распределение случайной величины~$X_j^{(\ge u)}$ 
принадлежит к~об\-ласти нормального притяжения \textit{одностороннего} 
строго устойчивого распределения с~характеристическим показателем $\alpha
\hm\in(0,1]$. Это означает, что существует такое число $b_u\hm>0$, что
\begin{equation}
\fr{1}{b_un^{1/\alpha}}\sum\limits_{j=1}^nX^{(\ge u)}\tod 
\zeta_{\alpha}\label{e8-kor}
\end{equation}
при $n\to\infty$, где ${\sf P}(\zeta_{\alpha}<x)\hm\equiv G_{\alpha}(x)$.
 Таким образом, можно считать, что в~(\ref{e6-kor}) 
 $c_k\hm=0$ и~$d_k\hm=b_uk^{1/\alpha}$. При этом функция распределения,
  стоящая в~правой части~(\ref{e6-kor}), соответствует случайной величине
$$
\widehat{R}_u^{(n)}=\fr{\underline{\sigma}_u
\left(n-N_n(u)\right)^{1/2} \xi_0+\left(n-N_n(u)\right)\underline{a}_u}
{b_uN_n^{1/\alpha}(u) \zeta_{\alpha}}\,.
$$
В то же время имеем:
\begin{multline*}
n^{1/\alpha-1}\,\widehat{R}_u^{(n)}=
\fr{\underline{\sigma}_u
({(n-N_n(u))^{1/2}}/{n^{1/\alpha}})\xi_0}
{b_u \left({N_n}/{n}\right)^{1/\alpha} \zeta_{\alpha}}
+{}\\
{}+\fr{(({n-N_n(u)})/{n^{1/\alpha}})\underline{a}_u}
{b_u \left({N_n}/{n}\right)^{1/\alpha} \zeta_{\alpha}}={}
\\
{}=n^{1/\alpha-1}
\fr{\underline{\sigma}_u \sqrt{({n-N_n(u)})/{n}}\,
({1}/{n^{1/\alpha-1/2}})\xi_0}
{b_u \left({N_n}/{n}\right)^{1/\alpha} \zeta_{\alpha}} +{}\\
{}+
\fr{n^{1-1/\alpha}F(u)\underline{a}_u}
{b_u \left({N_n}/{n}\right)^{1/\alpha} \zeta_{\alpha}}-{}\\
{}-\fr{\left((({N_n(u)-n(1-F(u))})/{\sqrt{n}})
({1}/{n^{1/\alpha-1/2}})\right)\underline{a}_u}
{b_u\left({N_n}/{n}\right)^{1/\alpha} \zeta_{\alpha}}={}
\\
{}=n^{1/\alpha-1}
\fr{n^{1-1/\alpha}F(u)\underline{a}_u+o_P(1)}
{b_u\left(1-F(u)\right)^{1/\alpha}\zeta_{\alpha}+o_P(1)}\tod {}\\
{}\tod
\fr{\underline{a}_uF(u)}
{b_u\left(1-F(u)\right)^{1/\alpha}\zeta_{\alpha}}
\end{multline*}
при $n\to\infty$. Следовательно, в~случае тяжелых хвостов 
(отсутствия дисперсии у~случайных величин~$X_j^{(\ge u)}$) 
имеет место сходимость:
$$
n^{1-1/\alpha}\fr{S_n^{(\ge u)}}{S_n^{(<u)}}\tod 
\fr{b_u\left(1-F(u)\right)^{1/\alpha}}{\underline{a}_uF(u)}\,
\zeta_{\alpha}\enskip (n\to\infty)\,.
$$
К такому выводу можно прийти и~другим, более строгим путем. 
Для этого понадобится следующее вспомогательное утверждение.

Рассмотрим последовательность случайных\linebreak величин
$\zeta_1,\zeta_2,\ldots$ Пусть $N_1,N_2,\ldots$~---
на\-ту\-раль\-но-знач\-ные случайные величины такие, что при 
каж\-дом~$n$ случайная величина~$N_n$ независима от последовательности
$\zeta_1,\zeta_2,\dots$ В~следующей лемме сходимость подразумевается
при $n\hm\to\infty$.



\smallskip

\noindent
\textbf{Лемма.} \textit{Предположим, что существуют неограниченно
возрастающая $($убывающая к~нулю$)$ последовательность положительных
чисел $\{A_n\}_{n\ge1}$ и~случайная величина~$\zeta$ такие, что}
\begin{equation}
A_n^{-1}\zeta_n\Longrightarrow\zeta\,.
\label{e9-kor}
\end{equation}
\textit{Если существуют неограниченно возрастающая $($убывающая к~нулю$)$
последовательность по\-ло\-жи\-тельных чисел $\{B_n\}_{n\ge1}$ и~случайная величина~$N$ такие, что}
\begin{equation}
B_n^{-1}A_{N_n}\Longrightarrow N,
\label{e10-kor}
\end{equation}
\textit{то}
\begin{equation}
B_n^{-1}\zeta_{N_n}\Longrightarrow \zeta N\,,\label{e11-kor}
\end{equation}
\textit{причем случайные сомножители в~правой части $(11)$ независимы. Если
дополнительно $N_n\hm\longrightarrow\infty$ по вероятности и~семейство
масштабных смесей функции распределения случайной величины $\zeta$
идентифициру\-емо, то условие}~(\ref{e10-kor})$~\textit{не только достаточно 
для}~(\ref{e11-kor})$, \textit{но и~необходимо}.

\smallskip

\noindent
Д\,о\,к\,а\,з\,а\,т\,е\,л\,ь\,с\,т\,в\,о\ \ см.\ в~\cite{Korolev1994} (случай
$A_n,B_n\hm\to\infty$) и~\cite{Korolev1995} (случай $A_n,B_n\hm\to 0$).

\smallskip

Очевидно представление:
\begin{multline}
n^{1-1/\alpha}\fr{S_n^{(\ge u)}}{S_n^{(<u)}}=
\fr{S_n^{(\ge u)}}{n^{1/\alpha}}\,\fr{n}{S_n^{(<u)}}={}\\
{}=\fr{1}{n^{1/\alpha}}\sum\limits_{j=0}^{N_n(u)}
\!X_j^{(\ge u)}\left(\fr{1}{n}\sum\limits_{j=0}^{n-N_n(u)}\!X_j^{(<u)}
\right)^{-1}.\label{e12-kor}
\end{multline}
В этом представлении перемножаемые суммы не являются независимыми. 
Рассмотрим второй множитель. В~нем случайная величина 
$\overline{N}_n(u)\hm=n\hm-N_n(u)$ имеет биномиальное распределение 
с~параметрами~$n$ и~$F(u)$ и~независима от случайных величин 
$X_1^{(<u)},X_2^{(<u)},\ldots$ При этом по усиленному закону больших чисел
\begin{equation}
\fr{1}{n}\sum\limits_{j=0}^{n}X_j^{(<u)}\longrightarrow \underline{a}_u
\label{e13-kor}
\end{equation}
с вероятностью единица, а значит, и~по распределению. 
В~то же время по усиленному закону больших чисел для бернуллиевых величин
\begin{equation}
n^{-1}\overline{N}_n(u)\longrightarrow F(u)\label{e14-kor}
\end{equation}
с вероятностью единица, а значит, и~по распределению. Применяя 
лемму с~$A_n=B_n=n$ и~соотношениями~(\ref{e13-kor}) и~(\ref{e14-kor}) 
в~роли~(\ref{e9-kor}) и~(\ref{e10-kor}) соответственно, получаем:
\begin{equation}
n^{-1}S_n^{(<u)}\tod \underline{a}_uF(u).\label{e15-kor}
\end{equation}
Поскольку константа независима от любой случайной величины, 
в~представлении~(\ref{e12-kor}) множители асимптотически независимы.

Теперь рассмотрим первый множитель в~правой части~(\ref{e12-kor}). 
В~нем случайная величина~$N_n(u)$ имеет биномиальное
 распределение с~параметрами~$n$ и~$1\hm-F(u)$ и~независима 
 от случайных величин $X_1^{(\ge u)},X_2^{(\ge u)},\ldots$ 
 По усиленному закону больших чисел для бернуллиевых величин
$$
n^{-1}N_n(u)\longrightarrow 1-F(u)
$$
с вероятностью единица, а~значит, и~по распределению. При этом, очевидно,
\begin{equation}
n^{-1/\alpha}N_n^{1/\alpha}(u)\longrightarrow 
\left(1-F(u)\right)^{1/\alpha}.
\label{e16-kor}
\end{equation}
Применяя лемму с~$A_n\hm=B_n\hm=n^{1/\alpha}$, соотношениями~(\ref{e8-kor}) 
и~(\ref{e16-kor}) в~роли~(\ref{e9-kor}) и~(\ref{e10-kor}) соответственно, получаем:
\begin{equation}
n^{-1/\alpha}S_n^{(\ge u)}\tod b_u\big(1-F(u)\big)^{1/\alpha}\zeta_{\alpha}.
\label{e17-kor}
\end{equation}
Используя соотношения~(\ref{e15-kor}) и~(\ref{e17-kor}), убеждаемся, что
\begin{multline*}
n^{1-1/\alpha}\fr{S_n^{(\ge u)}}{S_n^{(<u)}}=
\fr{S_n^{(\ge u)}}{n^{1/\alpha}}\,\fr{n}{S_n^{(<u)}}\tod{}\\
{}\tod
\fr{b_u\left(1-F(u)\right)^{1/\alpha}}{\underline{a}_uF(u)}\,\zeta_{\alpha}
\end{multline*}
при $n\to\infty$.

Таким образом, при указанной нормировке отношение суммы наблюдений, 
превосходящих порог~$u$, к~сумме наблюдений, меньших этого порога, 
асимптотически устойчиво с~тем же самым характеристическим показателем~$\alpha$.



{\small\frenchspacing
 {%\baselineskip=10.8pt
 \addcontentsline{toc}{section}{References}
%\vspace*{-3pt}


 \begin{thebibliography}{99}


\bibitem{Koch1998} %1
\Au{Koch R.} The 80/20 principle: The secret of achieving more with less.~--- 
London: Nicholas Brealey Publishing, 1998. 304~p. 
% (Русский перевод: \emph{Kox %Р.} Принцип 80/20. -- М.: Эксмо, 2012).

\bibitem{PrakasaRao1976} %2
\Au{Prakasa Rao B.\,L.\,S.} 
Limit theorems for sums of order statistics~// 
Z. Wahrscheinlichkeit., 1976. Vol.~33. No.\,4. P.~285--307.

\bibitem{Csorgoetal1986} %3
\Au{\mbox{Cs$\ddot{\mbox{o}}$rg{\!\!\ptb{\H{o}}}} S., Horv$\acute{\mbox{a}}$th~L., 
Mason~D.\,M.}
What portion of the sample makes a partial sum asymptotically stable or normal?~// 
Probab. Theory Rel., 1986. Vol.~72. P.~1--16.

\bibitem{Csorgoetal1988} %4
\Au{\mbox{Cs$\ddot{\mbox{o}}$rg{\!\!\ptb{\H{o}}}}~S., Haeusler~E., Mason~D.\,M.} 
The asymptotic distribution of trimmed
sums~// Ann. Probab., 1988. Vol.~16. P.~672--699.

\bibitem{Csorgoetal1988b} %5
\Au{\mbox{Cs$\ddot{\mbox{o}}$rg{\!\!\ptb{\H{o}}}}~S., Haeusler~E., Mason~D.\,M.} 
A~probabilistic approach to the asymptotic
distribution of sums of independent, identically distributed random variables~// 
Adv. Appl. Math., 1988. Vol.~9. P.~259--333.

\bibitem{Csorgo1989} %6
\Au{\mbox{Cs$\ddot{\mbox{o}}$rg{\!\!\ptb{\H{o}}}}~S.} 
Limit theorems for sums of order statistics~// 6th  Summer School
(International) in Probability theory and Mathematical Statistics.~--- 
Sofia: Publishing House Bulgarian Acad. Sci., 1989. P.~5--37.


\bibitem{Hahnetal1991} %7
Sums, trimmed sums and extremes~/ Eds. M.\,G.~Hahn,  D.\,M.~Mason, 
D.\,C.~Weiner.~--- Boston, MA, USA: Birkh$\ddot{\mbox{a}}$user, 1991. 428~p.

\bibitem{KestenMaller1992} %8
\Au{Kesten H., Maller~R.\,A.} Ratios of trimmed sums and order statistics~// 
Ann. Probab., 1992. Vol.~20. No.\,4. P.~1805--1842.

\bibitem{Janssen2000} %9
\Au{Janssen A.} Invariance principles for sums of extreme sequential 
order statistics attracted to Levy processes~// Stoch. Proc. Appl., 2000. 
Vol.~85. P.~255--277.

\bibitem{BenRachedetal2017} %10
\Au{Ben Rached N., Botev~Z., Kammoun~A., Alouini~M.-S., Tempone~R.} 
On the sum of order statistics and applications to wireless communication
systems performances~// 
IEEE T. Wirel. Commun., 2018. Vol.~17. No.\,11. P.~7801--7813. doi: 10.1109/twc.2018.2871201.

\bibitem{Korolev1994} %11
\Au{Королев~В.\,Ю.} 
Сходимость случайных последовательностей с~независимыми случайными индексами.~I~// 
Теория вероятн. и~ее примен., 1994. Т.~39. Вып.~2. С.~313--333. 

\bibitem{Korolev1995} %12
\Au{Королев~В.\,Ю.} 
Сходимость случайных последовательностей с~независимыми случайными индексами.~II~// 
Теория вероятн. и~ее примен., 1995. Т.~40. Вып.~4. С.~907--910. 

\end{thebibliography}

 }
 }

\end{multicols}




\vspace*{-6pt}

\hfill{\small\textit{Поступила в~редакцию 28.11.19}}

\vspace*{8pt}

%\pagebreak

%\newpage

%\vspace*{-28pt}

\hrule

\vspace*{2pt}

\hrule

%\vspace*{-2pt}

\def\tit{ON THE DISTRIBUTION OF~THE~RATIO OF~THE~SUM OF~SAMPLE ELEMENTS EXCEEDING 
A~THRESHOLD TO~THE~TOTAL SUM OF~SAMPLE ELEMENTS.~I}


\def\titkol{On the distribution of~the~ratio of~the~sum of~sample elements exceeding 
a~threshold to~the~total sum of~sample elements.~I}

\def\aut{V.\,Yu.~Korolev$^{1,2}$}

\def\autkol{V.\,Yu.~Korolev}

\titel{\tit}{\aut}{\autkol}{\titkol}

\vspace*{-9pt}

\noindent
$^1$Faculty of Computational Mathematics and Cybernetics, Lomonosov Moscow
State University, GSP-1, Leninskie\linebreak
$\hphantom{^1}$Gory, Moscow 119991, Russian Federation

\noindent
$^2$Federal Research Center ``Computer Science and Control'' 
of the Russian Academy of Sciences, 44-2~Vavilov\linebreak
$\hphantom{^1}$Str., 
Moscow 119333, Russian Federation


\def\leftfootline{\small{\textbf{\thepage}
\hfill INFORMATIKA I EE PRIMENENIYA~--- INFORMATICS AND
APPLICATIONS\ \ \ 2020\ \ \ volume~14\ \ \ issue\ 3}
}%
 \def\rightfootline{\small{INFORMATIKA I EE PRIMENENIYA~---
INFORMATICS AND APPLICATIONS\ \ \ 2020\ \ \ volume~14\ \ \ issue\ 3
\hfill \textbf{\thepage}}}

\vspace*{3pt} 


\Abste{The problem of description of the distribution of the 
ratio of the sum of sample elements exceeding a~threshold to 
the total sum of sample elements is considered. Unlike known 
versions of this problem in which the number of summed extreme 
order statistics is fixed, here, the specified threshold can be 
exceeded by an unpredictable number of sample elements. In the 
paper, in terms of the distribution function of a separate summand, 
the explicit form of the distribution of the ratio of the sum 
of sample elements exceeding a threshold to the total sum of 
sample elements is formally presented. The asymptotic and limit 
distributions are heuristically deduced for this ratio. These 
distributions are convenient for practical computations. The 
cases are considered in which the distributions of the summands 
have light tails (the second moments are finite) as well as 
the cases in which these distributions have heavy tails (belong 
to the domain of attraction of a stable law). In all cases, 
the normalization of the ratio is described that provides the
 existence of a nondegenerate limit (as the number of summands 
 infinitely increases) distribution as well as the limit distribution 
 itself (normal for the case of light tails and stable for the case 
 of heavy tails).}

\KWE{sum of independent random variables; random sum; binomial distribution; 
mixture of probability distributions; normal distribution; stable distribution;
 extreme order statistics}


\DOI{10.14357/19922264200305} 



%\vspace*{-20pt}

\Ack
\noindent
The research was conducted in accordance with the scientific program 
of the Moscow Center for Fundamental and Applied Mathematics and 
supported by the Russian Foundation for Basic Research (project 19-07-00914).



%\vspace*{6pt}



 \begin{multicols}{2}

\renewcommand{\bibname}{\protect\rmfamily References}
%\renewcommand{\bibname}{\large\protect\rm References}

{\small\frenchspacing
 {%\baselineskip=10.8pt
 \addcontentsline{toc}{section}{References}

 \begin{thebibliography}{99}

\bibitem{9-kor} %1
\Aue{Koch, R.} 1998.
\textit{The 80/20 principle: The secret of achieving more with less}. 
London: Nicholas Brealey Publishing. 304~p. 

\bibitem{12-kor} %2
\Aue{Prakasa Rao, B.\,L.\,S.} 1976.
Limit theorems for sums of order statistics.
%\textit{Zeitschrift fur Wahrscheinlichkeitstheorie und Verwandte Gebiete}
\textit{Z.~Wahrscheinlichkeit.} 33(4):285--307.

\bibitem{2-kor} %3
\Aue{\mbox{Cs$\ddot{\mbox{o}}$rg{\!\!\ptb{\H{o}}}}, S., L.~Horv$\acute{\mbox{a}}$th, 
and D.\,M.~Mason.} 1986.
What portion of the sample makes 
a~partial sum asymptotically stable or normal?
\textit{Probab. Theory Rel.} 72:1--16.

\bibitem{3-kor} %4
\Aue{\mbox{Cs$\ddot{\mbox{o}}$rg{\!\!\ptb{\H{o}}}}, S., E.~Haeusler, and D.\,M.~Mason.} 1988.
The asymptotic distribution of trimmed sums.
\textit{Ann. Probab.} 16:672--699.

\bibitem{4-kor} %5
\Aue{\mbox{Cs$\ddot{\mbox{o}}$rg{\!\!\ptb{\H{o}}}}, S., E.~Haeusler, and D.\,M.~Mason.} 1988.
A~probabilistic approach to the asymptotic
distribution of sums of independent, identically distributed random variables.
\textit{Adv. Appl. Math.} 9:259--333.

\bibitem{5-kor} %6
\Aue{\mbox{Cs$\ddot{\mbox{o}}$rg{\!\!\ptb{\H{o}}}}, S.} 1989. 
Limit theorems for sums of order statistics. 
\textit{6th Summer School (International) in Probability Theory and 
Mathematical Statistics.}
Sofia: Publishing House Bulgarian Acad. Sci. 5--37.

\bibitem{6-kor} %7
Hahn, M.\,G., D.\,M.~Mason, and D.\,C.~Weiner, eds. 1991.
\textit{Sums, trimmed sums and extremes.}
Boston, MA: Birkh$\ddot{\mbox{a}}$user. 428~p.

\bibitem{8-kor} %8
\Aue{Kesten, H., and R.\,A.~Maller.} 1992.
Ratios of trimmed sums and order statistics.
\textit{Ann. Probab.} 20(4):1805--1842.

\bibitem{7-kor} %9
\Aue{Janssen, A.} 2000.
Invariance principles for sums of extreme sequential order 
statistics attracted to Levy processes.
\textit{Stoch. Proc. Appl.} 85:255--277.

\bibitem{1-kor} %10
\Aue{Ben Rached,~N., Z.~Botev, A.~Kammoun, M.-S.~Alouini, and R.~Tempone.} 2018.
On the sum of order statistics and applications to wireless 
communication systems performances.
\textit{IEEE T. Wirel. Commun.} 17(11):7801--7813. doi: 10.1109/twc.2018.2871201.

\bibitem{10-kor} %11
\Aue{Korolev, V.\,Yu.} 1994.
Convergence of random sequences with the independent random indices.~I.
\textit{Theor. Probab. Appl.} 39(2):282--297.

\bibitem{11-kor} %12
\Aue{Korolev, V.\,Yu.} 1995.
Convergence of random sequences with independent random indices.~II.
\textit{Theor. Probab. Appl.} 40(4):770--772.

\end{thebibliography}

 }
 }

\end{multicols}

\vspace*{-6pt}

\hfill{\small\textit{Received November 28, 2019}}

%\pagebreak

%\vspace*{-24pt}

\Contrl


\noindent
\textbf{Korolev Victor Yu.} (b.\ 1954)~--- 
Doctor of Science in physics and
mathematics, professor, head of department, Faculty of 
Computational Mathematics and Cybernetics, and principal scientist, 
Moscow Center for Fundamental and Applied Mathematics, 
Lomonosov Moscow State University, GSP-1, Leninskie Gory, Moscow 119991, 
Russian Federation; leading scientist, Federal Research Center 
``Computer Science and Control'' of the Russian Academy of Sciences, 
44-2~Vavilova Str., Moscow 119333, Russian Federation; 
\mbox{vkorolev@cs.msu.ru}
\label{end\stat}

\renewcommand{\bibname}{\protect\rm Литература} 