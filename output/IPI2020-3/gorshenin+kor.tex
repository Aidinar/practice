
\def\stat{gorsh+kor}

\def\tit{СТАТИСТИЧЕСКОЕ ОЦЕНИВАНИЕ РАСПРЕДЕЛЕНИЙ СЛУЧАЙНЫХ КОЭФФИЦИЕНТОВ СТОХАСТИЧЕСКОГО 
ДИФФЕРЕНЦИАЛЬНОГО УРАВНЕНИЯ ЛАНЖЕВЕНА$^*$}

\def\titkol{Статистическое оценивание распределений случайных коэффициентов СДУ
%стохастического дифференциального уравнения 
Ланжевена}

\def\aut{А.\,К.~Горшенин$^1$, В.\,Ю.~Королев$^2$,  А.\,А.~Щербинина$^3$}

\def\autkol{А.\,К.~Горшенин, В.\,Ю.~Королев,  А.\,А.~Щербинина}

\titel{\tit}{\aut}{\autkol}{\titkol}

\index{Горшенин А.\,К.}
\index{Королев В.\,Ю.}
\index{Щербинина А.\,А.}
\index{Gorshenin A.\,K.}
\index{Korolev V.\,Yu.}
\index{Shcherbinina A.\,A.}
 

{\renewcommand{\thefootnote}{\fnsymbol{footnote}} \footnotetext[1]
{Исследования выполнены при частичной поддержке РФФИ (проект 19-07-00914), 
Стипендии Президента Российской Федерации молодым ученым 
и~аспирантам (СП-538.2018.5) и~в~соответствии с~программой Московского
 центра фундаментальной и~прикладной математики.}}


\renewcommand{\thefootnote}{\arabic{footnote}}
\footnotetext[1]{Федеральный исследовательский центр 
<<Информатика и~управление>> 
Российской академии наук, \mbox{agorshenin@frccsc.ru}}
\footnotetext[2]{Факультет вычислительной математики и~кибернетики
Московского государственного университета имени М.\,В.~Ломоносова;
Федеральный исследовательский центр <<Информатика и~управление>> 
Российской академии наук,
\mbox{vkorolev@cs.msu.ru}}
\footnotetext[3]{Факультет вычислительной математики и~кибернетики
Московского государственного университета имени М.\,В.~Ломоносова,
\mbox{shcherbinina.aa.97@gmail.com}}

%\vspace*{-6pt}

\Abst{Разработан метод статистического оценивания распределений 
случайных коэффициентов стохастических дифференциальных уравнений (СДУ) 
типа Ланжевена с~помощью техники скользящего разделения смесей (СРС). 
Предложены дискретные аппроксимации для оценок указанных распределений. 
С~целью изучения из\-мен\-чи\-вости распределений коэффициентов сдвига 
(дрейфа) и~диффузии СДУ во времени предложен алгоритм последовательной 
идентификации (определения локальной связ\-ности) компонент по\-лу\-ча\-емых смесей. 
В~его основу положена комбинация жад\-но\-го алгоритма для поиска чис\-ла 
компонент и~одного из методов кластеризации ($k$- или $c$-сред\-них). 
Применение метода иллюстрируется конкретными примерами анализа процесса 
теплообмена между атмосферой и~океаном для Гольфстрима и~тропиков.}

\KW{стохастические дифференциальные уравнения; смешанные распределения; 
локальная связанность; жад\-ный алгоритм; клас\-те\-ри\-зация}

\DOI{10.14357/19922264200301} 

%\vspace*{-6pt}

\vskip 10pt plus 9pt minus 6pt

\thispagestyle{headings}

\begin{multicols}{2}

\label{st\stat}

\section{Введение}

В физике СДУ Ланжевена принято называть сле\-ду\-ющее соотношение:
\begin{equation}
\label{LangevinEq}
dX(t)=a(t)\,dt+b(t)\,dW\,,
\end{equation}
определяющее случайный процесс~$X(t)$, где $W(t)$~--- винеровский процесс, 
а~коэффициенты $a(t)$ и~$b(t)$ случайны. Стохастические дифференциальные 
уравнения вида~\eqref{LangevinEq} 
широко используются, например, в~задаче ассимиляции данных 
при анализе\linebreak раз\-но\-мас\-штаб\-ной из\-мен\-чи\-вости геофизических 
пе\-ре\-менных~\cite{Belyaev2018}. В~финансовой математике из\-вест\-ны 
специальные вер\-сии урав\-не\-ния~\eqref{LangevinEq}. 
В~частности, весьма популярна модель геометрического броуновского 
движения вида:
\begin{equation}
\label{BrownMotionEq}
dX(t)=aX(t)\,dt+b X(t)\,dW\,,
\end{equation}
где $a\in\mathbb{R}$, $b\hm>0$. Известно много обобщений 
модели~\eqref{BrownMotionEq} c~конкретными видами за\-ви\-си\-мости~$a$ и~$b$ 
от~$X(t)$ и~других случайных процессов, например модели 
Леланда~\cite{Leland1985}, Барл\-са--Со\-не\-ра~\cite{BarlesSoner1998}, 
Хестона~\cite{Heston1993}, Кок\-са--Ин\-гер\-сол\-ла--Рос\-са~\cite{CoxIngersollRoss1985}, 
Хал\-ла--Уай\-та~\cite{HullWhite1987} и~другие так называемые модели 
стохастической волатильности (см.\ 
также~\cite{DermanKani1994, Dupire1994, Shiryaev2016}).

При отсутствии априорной информации о~<<физической>> структуре процесса~$X(t)$ 
для успешного изучения и~прогнозирования его эволюции первостепенную 
важ\-ность приобретает задача  статистического оценивания функциональных 
коэффициентов~$a(t)$ и~$b(t)$. В~силу их слу\-чай\-ности данная задача допускает 
как минимум две принципиально разные формулировки. Во-пер\-вых, можно 
пытаться найти (случайные же) оценки самих функций~$a(t)$ и~$b(t)$, т.\,е.\ 
найти их точечные аппроксимации, и,~во-вто\-рых, пытаться статистически 
оценить распределения случайных величин $a(t)$ и~$b(t)$. 
Во втором случае, зная ка\-кие-ли\-бо свойства этих коэффициентов, например 
структуру их функциональной зависимости от исходного процесса $X(t)$ 
(как в~упомянутых выше моделях Леланда, Барл\-са--Со\-не\-ра, Хестона, 
Кок\-са--Ин\-гер\-сол\-ла--Рос\-са или Беляева), мож\-но \mbox{найти} оценки 
чис\-ло\-вых па\-ра\-мет\-ров этих моделей.

В данной статье рассматривается вторая задача. 
Пусть $n\hm\geqslant1$ и~$t_0\hm=0\hm<t_1<\cdots<t_n$~--- 
моменты времени, в~которые наблюдается процесс~$X(t)$. 
Для простоты предположим, что $t_i\hm-t_{i-1}\hm=1$ для любого $i\hm\geqslant1$. 
Из уравнения~\eqref{LangevinEq} видно, что распределение приращения 
$X(t_i)\hm-X(t_{i-1})$ процесса~$X(t)$ можно аппроксимировать 
распределением вида:
\begin{equation}
\label{DiffApprox}
\mathbb{P}\left(X(t_i)-X(t_{i-1})<x\right)\approx \mathbb{E}\Phi\left(\fr{x-A_i}{B_i}\right),
\end{equation}
где $\Phi(x)$~--- стандартная нормальная функция распределения;
$A_i\hm\in\mathbb{R}$ и~$B_i\hm>0$~--- случайные величины. 
В~свою очередь, для распределений случайных величин~$A_i$ и~$B_i$, 
по отношению к~которым берется математическое ожидание 
в~формуле~\eqref{DiffApprox}, можно использовать дискретную аппроксимацию. 
Тогда вместо выражения~\eqref{DiffApprox} для распределения
 приращения $X(t_i)\hm-X(t_{i-1})$ можно применить приближение вида:

\noindent
\begin{equation}
\label{DiffDiscrApprox}
\mathbb{P}\left(X(t_i)-X(t_{i-1})<x\right)\approx\sum\limits_{k=1}^K
p_k\Phi\left(\fr{x-a_k}{b_k}\right),
\end{equation}
где $K\in\mathbb{N}$, $p_k\hm\geqslant0$, $k\hm=\overline{1,K}$, $\sum\nolimits_{k} p_k\hm=1$. 
Очевидно, параметры~$p_k$, $a_k$ и~$b_k$, формирующие так на\-зы\-ва\-емые 
динамические и~диффузионные компоненты~\cite{Korolev2011}, 
зависят также от~$i$ и~изменяются при переходе от~$t_i$ к~$t_{i+1}$.

Для статистического оценивания параметров~$p_k$, $a_k$ и~$b_k$ можно 
использовать %метод скользящего разделения смесей (
СРС-ме\-тод, описанный 
в~\cite{Korolev2011}. Статистические закономерности поведения 
рассматриваемых процессов $X(t)$, $a(t)$ и~$b(t)$ изменяются во 
времени, вообще говоря, нерегулярным образом, результатом чего 
является отсутствие универсального смешивающего закона. 
Поэтому изучение динамики изменения статистических 
закономерностей в~поведении ис\-сле\-ду\-емо\-го процесса проводится 
на последовательных интервалах времени. Тем самым па\-ра\-мет\-ры смесей 
(сдвига (дрейфа)~$a_k$, масштаба (диффузии) $b_k$ и~веса компонент~$p_k$) 
оцениваются как функции времени. С~этой целью при каждом положении 
скользящего окна используется ЕМ (expectation-maximization) 
ал\-го\-ритм или ка\-кие-ли\-бо 
из его модификаций~\cite{Korolev2011,Gorshenin2015a,Gorshenin2017}.

Для всеобъемлющего изучения закономерностей эволюции процесса~$X(t)$, 
задаваемого СДУ~\eqref{LangevinEq}, необходимо знать, как ведут 
себя случайные коэффициенты~$a(t)$ и~$b(t)$ во времени. 
Для этого, в~свою очередь, необходимо восстановить эволюцию 
во времени их распределений. В~рамках аппроксимации~\eqref{DiffDiscrApprox}
 по\-след\-няя задача сводится к~последовательной идентификации компонент 
 смеси~\eqref{DiffDiscrApprox}, т.\,е.\ к~определению соответствия 
 оценок параметров $p_k$, $a_k$ и~$b_k$, полученных на разных (например, 
 соседних) положениях сколь\-зя\-ще\-го окна. Другими словами, чтобы 
 определить эволюцию распределения случайных функций~$a(t)$ и~$b(t)$, 
 надо восстановить компоненты смеси~\eqref{DiffDiscrApprox} как 
 функции времени. 
 Таким образом, решается задача при\-бли\-жен\-но\-го 
 восстановления двумерного распределения 
 $$F_t(x,y)\hm\equiv 
 \mathbb{P}\left(a(t)<x,b(t)<y\right)\,.$$ 
 С~этой целью при каждом 
 $i\hm=\overline{1,n}$ ищется дискретная аппроксимация распределения 
 $F_{t_i}(x,y)$, а~затем осуществляется интерполяция этой функции для 
 остальных~$t$. Для интерполяции необходимо установить соответствие 
 между множествами возможных значений $\{a_k,b_k: k\hm=\overline{1,K}\}$ 
 случайных коэффициентов $a(t_i)$ и~$b(t_i)$ при разных~$i$, т.\,е.\
  восстановить эволюцию компонент аппроксимиру\-ющей 
  смеси~\eqref{DiffDiscrApprox} во времени.

В процессе шагов СРС-метода выделяются компоненты 
смеси~\eqref{DiffDiscrApprox}, которые эволюционируют 
при сдвигах скользящего окна. При этом достаточно слож\-но 
судить о~том, какая из оценок па\-ра\-мет\-ров соответствует тому 
или иному значению на предыдущем шаге. Обычно по\-доб\-ная взаимосвязь 
определяется визуально и,~очевидно, является достаточно субъективной.
 В~статье предложен подход к~автоматизации решения данной задачи на 
 основе комбинации жад\-но\-го алгоритма и~методов клас\-те\-ри\-за\-ции $k$- 
 или $c$-сред\-них~\cite{Steinhaus1956,MacQueen1967,David2007,Dunn1973}. 
 При этом на первом этапе определяется чис\-ло клас\-те\-ров для методов 
 машинного обучения, которые используются непосредственно для вы\-яв\-ле\-ния 
 связанных компонент. Данная процедура будет использована для получения 
 статистических оценок коэффициентов урав\-не\-ния~\eqref{LangevinEq} 
 на примере анализа опи\-сы\-ва\-емых им процессов переноса теп\-ла 
 между океаном и~атмосферой~\cite{Belyaev2018} в~ряде географических точек.

\section{Метод определения связности локальных компонент}
%\label{SecConnectivity}

Обозначим через $I^{(n)}$ набор индексов (номеров) компонент 
для шага с~номером~$n$ СРС-ме\-то\-да, т.\,е.\ $I^{(n)}
\hm=\{1,2,\ldots,k^{(n)}\}$, а через 
$J^{(n+1)}\hm=\{1,2,\ldots,k^{(n+1)}\}$~--- 
аналогичный набор для шага $(n+1)$. Через~$I_0$ и~$J_0$ 
обозначим множество индексов из первого и~второго наборов 
соответственно, для которых удалось найти ближайшую компоненту. 
Первоначально полагаем, что $I_0\hm=\emptyset$ и~$J_0\hm=\emptyset$. 
Для каждого фиксированного $J\hm\in J^{(n+1)}\setminus J_0$ 
находим наиболее близ\-кий номер~$I$ в~смыс\-ле решения сле\-ду\-ющей 
оптимизационной задачи:
\begin{multline}
\label{I}
I=\argmin\limits_{i\in I^{(n)} 
\setminus I_0}\left(\left|a_i^{(n)}-a_J^{(n+1)}\right|^p+{}\right.\\
\left.{}+\left|\sigma_i^{(n)}-\sigma_J^{(n+1)}\right|^p \right)^{1/p}\!.
\end{multline}

\renewcommand{\figurename}{\protect\bf Алгоритм}

\setcounter{figure}{0}
\begin{figure*}[b]
{\small \begin{center}
\textbf{Алгоритм 1} Динамическое определение числа локальных компонент\\
{\ }\\[-6pt]
\begin{tabular}{l}
\hline

\textbf{function} {\normalsize{N}}UM{\normalsize{G}}REEDY (Params, $ I^{(n)}$, $J^{(n+1)}$)\\

\hspace*{8mm}$I_0\gets \emptyset$, $J_0\gets \emptyset$, Comps$\gets \emptyset$; // 
\textit{Инициализация}\\
\hspace{8mm}\textbf{repeat} // \textit{Продолжающиеся или новые компоненты}
\\
 \hspace*{15mm}// \textit{Оптимизация выражения~\eqref{I} с~учетом условия~\eqref{Dist}}
 \\
\hspace*{15mm}[I, J]$\gets${\normalsize F}IND{\normalsize I}(Params, $J^{(n+1)}\setminus 
J_0$, $I^{(n)}\setminus I_0$);
\\
\hspace*{15mm}\textbf{if} I $\neq \emptyset$\ \textbf{then}\ // \textit{Найдена 
предшествующая $J$ компонента}
 \\   
\hspace*{20mm}$I_0\gets I_0\cup I$, $J_0\gets J_0\cup J$;
\\
\hspace*{15mm}\textbf{else} // \textit{Добавление новой компоненты}
\\
\hspace*{20mm}$J_0\gets J_0\cup J$;
\\
\hspace*{20mm}Comps $\gets$ {\normalsize A}DD{\normalsize N}EW{\normalsize C}OMP({Params, J});
\\
\hspace*{8mm}\textbf{until} ($J^{(n+1)}\setminus J_0 \neq \emptyset$)
\\
\hspace*{8mm}\textbf{return}\ Comps;\\
\hline
\end{tabular}
\end{center}}
\vspace*{-6pt}
\end{figure*}


В этом случае минимизируемое в~правой час\-ти выражение 
представляет собой $\ell^p$-нор\-му ($p\hm=1,2,\ldots$) соответствующего 
вектора разностей координат в~пространстве точек $(a,\sigma)$.

Полагая после этого $I_0\hm=I_0\cup I$ и~$J_0\hm=J_0\cup J$, необходимо 
повторить процедуру заново. Возможны следующие случаи.
\begin{enumerate}
\item $\left|I^{(n)}\right|=\left|J^{(n+1)}\right|$, т.\,е.\ 
$k^{(n)}\hm=k^{(n+1)}$. В~этом случае оба набора будут исчерпаны одновременно.
\item $\left|I^{(n)}\right|<\left|J^{(n+1)}\right|$, т.\,е.\
 $k^{(n)}<k^{(n+1)}$. В~этом случае процедура останавливается, 
 когда исчерпан набор $I^{(n)} \setminus I_0$. Оcтавшиеся 
 в~$J^{(n+1)}\setminus J_0$ элементы формируют новые компоненты, 
 которые появились только на $(n+1)$-м шаге.
\end{enumerate}

Случай $\left|I^{(n)}\right|>\left|J^{(n+1)}\right|$, т.\,е.\
 $k^{(n)}\hm>k^{(n+1)}$, не рассматривается, поскольку основная цель 
 данной процедуры~--- определение чис\-ла компонент за весь период 
 эволюции процесса, поэтому уменьшаться оно не может, даже если 
 на ка\-ком-то шаге произошло сокращение локального значения для чис\-ла 
 компонент. Отметим, что указанная процедура, очевидно, является жад\-ной.
  При этом, поскольку ее конечная цель со\-сто\-ит в~определении чис\-ла 
  клас\-те\-ров для сле\-ду\-юще\-го шага, данная особенность не пред\-став\-ля\-ет\-ся 
  критической.

Для точного определения чис\-ла компонент необходимо задавать 
некоторый допустимый порог бли\-зости $\varepsilon({\bf a}, 
{\boldsymbol \sigma})$:
\begin{multline}
\label{Dist}
\left(\left|a_I^{(n)}-a_J^{(n+1)}\right|^p+
\left|\sigma_I^{(n)}-\sigma_J^{(n+1)}\right|^p +{}\right.\\
\left.{}+\left|p_I^{(n)}-
p_J^{(n+1)}\right|^p\right)^{1/p} < \varepsilon({\bf a}, 
{\boldsymbol \sigma}).
\end{multline}

Такая проверка нужна для того, чтобы корректно определять 
ситуацию, при которой компоненты с~номерами~$I$ и~$J$ на $n$-м и~$(n+1)$-м 
шагах считались одинаковыми, и~не было не\-об\-хо\-ди\-мости создавать новую 
в~рамках жад\-но\-го алгоритма. Реализация метода определения чис\-ла 
компонент приведена в~алгоритме~1.

Предполагается, что данный алгоритм применяется для 
компонент с~положительными весами (нулевые значения соответствуют 
случаю уменьшения их чис\-ла на ка\-ком-ли\-бо шаге). Кроме того, для 
всех допустимых значений $i\hm\neq j$ и~$n$ должны существовать такие 
$\delta_a\hm>0$ и~$\delta_\sigma\hm>0$, что выполнено хотя бы одно из 
условий: 
$$
\left|a_i^{(n)}\hm-a_j^{(n)}\right|\hm>\delta_a\,; \quad
\left|\sigma_i^{(n)}\hm-\sigma_j^{(n)}\right|\hm>\delta_\sigma\,.
$$
Если же они оба нарушены, то необходимо объединять эти компоненты в~одну 
с~соответствующей корректировкой (суммированием) весов, т.\,е.\ 
предполагается, что все компоненты различны,~--- 
это гарантирует кор\-рект\-ность применения жад\-но\-го алгоритма.

Алгоритм~1 используется в~качестве важ\-ной 
со\-став\-ной час\-ти  метода формирования матрицы связ\-ности. 
Она пред\-став\-ля\-ет собой вспомогательную структуру, в~которой 
на каждом шаге скользящего окна сохраняется актуальное 
со\-сто\-яние всех выделенных к~текущему моменту компонент. 
Сначала ко всему ряду применяется метод EM-ти\-па для 
определения числа компонент на каждом шаге\linebreak (как было отмечено выше, 
оно не может убывать). Затем в~двухмерном пространстве 
$(\bf a, \boldsymbol \sigma)$ используется один из методов 
кластеризации с~полученным жад\-ным алгоритмом чис\-лом локальных 
ком\-по\-нент-клас\-те\-ров. Веса не учитываются, так как вклад компоненты 
в~смесь может изменяться, при этом па\-ра\-мет\-ры~--- 
математическое ожидание и~дис\-пер\-сия~--- варьи\-ру\-ют\-ся 
не слишком сильно, и~тогда считается, что это та же самая компонента. 
Со\-от\-вет\-ст\-ву\-ющая процедура 
пред\-став\-ле\-на в~алго\-ритме~2.

\begin{figure*}
{\small \begin{center}
\textbf{Алгоритм 2} Определение компонент связности в~СРС-методе\\
{\ }\\[-6pt]
\begin{tabular}{l}
\hline
\textbf{function} {\normalsize MSMC}OMPONENTS({Data, options})\\
\hspace*{5mm}Params $\gets${\normalsize EM}S(Data, options.EM) // \textit{СРС-метод}\\
\hspace*{5mm}// \textit{Инициализация числом ненулевых компонент на первом шаге}\\
\hspace*{5mm}Comps$^{(1)}$$\gets$Params.k$^{(1)}$;\\
\hspace*{5mm}\textbf{for} (n = 1:LENGTH(Params)-1) \textbf{do}\\
\hspace*{10mm}Comps$^{(n+1)}$ $\gets$ {\normalsize N}UM{\normalsize G}REEDY({Params, $ I^{(n)}$, $J^{(n+1)}$});\\
\hspace*{5mm}// \textit{Метки для каждого набора параметров, кластеризация}\\
\hspace*{5mm}Labels $\gets$ {\normalsize C}LUST(Params, {MAX}(Comps), options.ClustAlg);\\
\hspace*{5mm}// \textit{Матрица связности для компонент СРС-метода}\\
\hspace*{5mm}HistMatrConnect$\gets$ {\normalsize С}ONNECTIVITY({Params, Labels});  \\
\hspace*{5mm}{\normalsize P}LOT{\normalsize C}OMP(HistMatrConnect); // \textit{Визуализация результатов}\\
\hspace*{5mm}\textbf{return} HistMatrConnect;\\
\hline
\end{tabular}
\end{center}}
\vspace*{-6pt}
\end{figure*}

Очевидно, что данная процедура может быть использована и~в~случае 
смесей многомерных распределений. При этом в~формулу оптимизации~\eqref{I}
 долж\-ны быть добавлены все па\-ра\-мет\-ры со\-от\-вет\-ст\-ву\-юще\-го распределения 
 (с~соответствующей модификацией условия~\eqref{Dist}), 
 а~затем может быть про-\linebreak
 \vspace*{-12pt}
 
 \pagebreak
 
 \noindent
 ведена клас\-те\-ри\-за\-ция в~пространстве 
 переменных новой раз\-мер\-ности.

\section{Оценивание коэффициентов уравнения Ланжевена 
на~примере турбулентных потоков тепла между океаном и~атмосферой}

Для статистической оценки коэффициентов в~уравнении Ланжевена
воспользуемся алгоритмом~2, а~так\-же СРС-оцен\-ка\-ми, 
полученными при аппроксимации распределений приращений потоков теп\-ла 
между океаном и~атмосферой в~нескольких географических точках. Отметим, 
что СРС-метод %скользящего разделения смесей 
ранее использовался для 
статистического моделирования закономерностей в~подобного 
рода данных~\cite{Gorshenin2015b,Gorshenin2016}.

\setcounter{figure}{0}
\renewcommand{\figurename}{\protect\bf Рис.}
\begin{figure*} %fig1
\vspace*{9pt}

\vspace*{1pt}
 \begin{center}
 \mbox{%
 \epsfxsize=156mm 
 \epsfbox{gor-1.eps}
 }
 \end{center}
   \vspace*{-9pt}
\Caption{\label{FigComps_gulfstqe_dyn}
Оценки распределения сдвига (Гольфстрим): 
(\textit{a})~явные потоки;
(\textit{б})~скрытые потоки;
1--5~--- структурные компоненты}
\vspace*{9pt}
\end{figure*}


На рис.~1--4 %\ref{FigComps_gulfstqe_dyn}--\ref{FigComps_troptqh_diff} 
верхние графики демонстрируют статистическую структуру процесса 
теп\-ло\-об\-ме\-на между океаном и~атмосферой в~указанных географических 
точках для двух классов потоков~\cite{Perry1977}~---  скрытых 
%(далее обозначаются как \verb"qe") 
и~явных,~--- %(\verb"qh"),~--- 
яв\-ля\-ющих\-ся со\-став\-ля\-ющи\-ми теп\-ло\-во\-го баланса.

%\begin{figure*} %fig2
% \epsfbox{gor-2.eps}
%\end{figure*}

\begin{figure*} %fig3
\vspace*{1pt}
 \begin{center}
 \mbox{%
 \epsfxsize=156mm 
 \epsfbox{gor-3.eps}
 }
 \end{center}
   \vspace*{-9pt}
\Caption{\label{FigComps_gulfstqe_diff}
Оценки распределения коэффициента диффузии (Гольфстрим):
(\textit{a})~явные потоки;
(\textit{б})~скрытые потоки;
1--5~--- структурные компоненты}
\end{figure*}

%\begin{figure*} %fig4
% \epsfbox{gor-4.eps}
%\end{figure*}

\begin{figure*} %fig5
\vspace*{1pt}
 \begin{center}
 \mbox{%
 \epsfxsize=156mm 
 \epsfbox{gor-5.eps}
 }
 \end{center}
   \vspace*{-9pt}
\Caption{\label{FigComps_tropqe_dyn}
Оценки распределения сдвига (тропики):
(\textit{a})~явные потоки;
(\textit{б})~скрытые потоки;
1--5~--- структурные компоненты}
\end{figure*}

%\begin{figure*} %fig6
% \epsfbox{gor-6.eps}
%\end{figure*}

\begin{figure*} %fig7
\vspace*{1pt}
 \begin{center}
 \mbox{%
 \epsfxsize=156mm 
 \epsfbox{gor-7.eps}
 }
 \end{center}
   \vspace*{-9pt}
\Caption{\label{FigComps_troptqe_diff}
Оценки распределения коэффициента диффузии (тропики):
(\textit{a})~явные потоки;
(\textit{б})~скрытые потоки;
1--5~--- структурные компоненты}
\end{figure*}

%\begin{figure*} %fig8
% \epsfbox{gor-8.eps}
%\end{figure*}


Благодаря структурам, возвращаемым функцией \verb"MSMComponents",
 можно точно отследить, когда те или иные из компонент существовали, 
 прерывались и~возобновлялись, и~проанализировать их взаимосвязь с~реальными
  физическими процессами. На нижних графиках, содержащих СРС-оцен\-ки для 
  динамической и~диффузионной компонент, продемонстрирована эволюция весов, 
  т.\,е.\ вклад соответствующей структурной со\-став\-ля\-ющей в~общее 
  развитие процесса во времени.

Для аппроксимации были использованы четырехкомпонентные 
нормальные смеси, однако при выбранных настройках жад\-но\-го 
алгоритма~1 для всех рядов (за исключением
 явных потоков в~тропиках) были получены пять локальных
  компонент связ\-ности.


Видно, что общее чис\-ло компонент не слишком сильно изменяется, 
поэтому результаты автоматического определения с~по\-мощью 
жад\-но\-го алгоритма~1 варьируются от ряда к~ряду 
не очень существенно. Однако для лучшего учета локальных процессов 
полученное чис\-ло компонент (4--5) может быть расширено за счет 
повышения чув\-ст\-ви\-тель\-ности процедуры путем выбора меньшего 
порогового значения в~формуле~\eqref{Dist}.

\vspace*{-9pt}

\section{Заключение}
\vspace*{-3pt}

В работе описан метод статистического оценивания 
распределений случайных па\-ра\-мет\-ров СДУ
%стохастических дифференциальных уравнений 
типа Ланжевена с~по\-мощью СРС-техники.
% скользящего разделения смесей. 
Предложены дискретные аппроксимации для оценок указанных распределений. 
С~целью изучения из\-мен\-чи\-вости распределений коэффициентов СДУ 
во времени пред\-ло\-жен алгоритм последовательной идентификации 
(определения локальной связ\-ности) компонент получаемых смесей. 
В~его основу положена комбинация жад\-но\-го алгоритма для поиска чис\-ла 
компонент и~одного из методов клас\-те\-ри\-за\-ции ($k$- или $c$-сред\-них). 
Функциональные\linebreak па\-ра\-мет\-ры (компоненты распределения~\eqref{DiffDiscrApprox} 
как функции времени), полученные в~результате опи\-сы\-ва\-емых статистических 
процедур, могут быть\linebreak использованы при обучении интеллектуальных 
алгоритмов прогнозирования процессов, удовле\-тво\-ря\-ющих уравнениям 
типа~\eqref{LangevinEq}. Применение метода иллюстрируется конкретными 
примерами анализа процесса теп\-ло\-об\-ме\-на между атмосферой и~океаном.

\vspace*{-9pt}

{\small\frenchspacing
 {%\baselineskip=10.8pt
 \addcontentsline{toc}{section}{References}
\vspace*{-3pt}


 \begin{thebibliography}{99}
\bibitem{Belyaev2018} 
\Au{Belyaev~K., Kuleshov~A., Tuchkova~N., Tanajura~C.\,A.\,S.} 
An optimal data assimilation method and its application 
to the numerical simulation of the ocean dynamics~// 
Math. Comp. Model. Dyn., 2018. Vol.~1. Iss.~24. P.~12--25.

\bibitem{Leland1985} 
\Au{Leland~H.\,E.} Option pricing and replication with transactions costs~// 
J.~Financ., 1985. Vol. 40. P. 1283--1301.

\bibitem{BarlesSoner1998} 
\Au{Barles~G., Soner~H.\,M.} Option pricing with transaction 
costs and a nonlinear Black--Scholes equation~// 
Financ. Stoch., 1998. Vol.~2. P.~369--397.

\bibitem{Heston1993} 
\Au{Heston~S.\,L.} A~closed-form solution for options with stochastic
 volatility, with application to bond and currency options~// 
 Rev. Financ. Stud., 1993. Vol.~6. P.~327--343.

\bibitem{CoxIngersollRoss1985} 
\Au{Cox~J.\,C., Ingersoll~J.\,E., Ross~S.\,A.} 
A~theory of the term structure of interest rates~// 
Econometrica, 1985. Vol.~53. P.~385--407.

\bibitem{HullWhite1987} 
\Au{Hull~J., White~A.} The pricing of options on assets 
with stochastic volatilities~// J.~Financ., 1987. Vol.~42. P.~281--308.

\bibitem{Dupire1994} %7
\Au{Dupire~B.} 
Pricing with a~smile~// Risk, 1994. Vol.~7. P.~18--20.

\bibitem{DermanKani1994} 
\Au{Derman~E., Kani~J.} Riding on a smile~// Risk, 1994. Vol.~7. P.~32--39.

\bibitem{Shiryaev2016} %9
\Au{Ширяев~А.\,Н.} Основы стохастической финансовой математики.
 Т.~1. Факты. Модели.~--- М.: МЦНМО, 2016. 440~c.

\bibitem{Korolev2011} %10
\Au{Королев~В.\,Ю.} Ве\-ро\-ят\-ност\-но-ста\-ти\-сти\-че\-ские 
методы декомпозиции волатильности хаотических процессов.~--- 
М.: Изд-во Московского ун-та, 2011. 512~c.

\bibitem{Gorshenin2017} %11
\Au{Горшение А.\,К., Королев В.\,Ю., Турсунбаев А.\,М.}
Медианные модификации EM-алгоритма для разделения смесей вероятностных 
распределений и~их применение к~декомпозиции волатильности финансовых 
индексов~//
Статистические методы оценивания и~про\-вер\-ки гипотез, 2008. 
Т.~21. С.~169--195.

\bibitem{Gorshenin2015a} %12
\Au{Gorshenin~A.\,K.} 
On implementation of EM-type algorithms in the stochastic models for a matrix 
computing on GPU~// AIP Conf. Proc., 2015. Vol.~1648. Art. No.\,250008. 4~p.

\bibitem{Steinhaus1956} 
\Au{Steinhaus~H.} Sur la division des corps materiels en parties~// 
B. Acad. Pol. Sci.,1956. Vol.~4. Iss.~12. P.~801--804.

\bibitem{MacQueen1967} 
\Au{MacQueen~J.} Some methods for 
classification and analysis of multivariate observations~// 
5th Berkeley Symposium on Mathematical Statistics and Probability
Proceedings, 1967. P.~281--297.

\bibitem{Dunn1973} %15
\Au{Dunn~J.\,C.} A~fuzzy relative of the ISODATA process and its use in 
detecting compact well-separated clusters~// 
J.~Cybernetics, 1973. Vol.~3. Iss.~3. P.~32--57.

\bibitem{David2007} %16
\Au{David~A., Vassilvitskii~S.} K-means++: 
The advantages of careful seeding~// 18th Annual ACM-SIAM Symposium 
on Discrete Algorithms Proceedings.~--- ACM, 2007. P.~1027--1035.

\bibitem{Gorshenin2015b} 
\Au{Korolev~V.\,Yu., Gorshenin~A.\,K., Gulev~S.\,K., Belyaev~K.\,P.} 
Statistical modeling of air--sea turbulent heat fluxes 
by finite mixtures of Gaussian distributions~// Comm. Com. 
Inf. Sc., 2015. Vol.~564. P.~152--162.

\bibitem{Gorshenin2016} 
\Au{Горшенин~А.\,К.} 
Концепция он\-лайн-комп\-лек\-са для стохастического моделирования 
реальных процессов~// Информатика и~её применения, 2016. 
Т.~10. Вып.~1. C.~72--81.

\bibitem{Perry1977} 
\Au{Perry~A.\,H., Walker~J.\,M.} Ocean--atmosphere system.~--- 
Upper Saddle River, NJ, USA: Prentice Hall Press, 1977. 180~p.
\end{thebibliography}

 }
 }

\end{multicols}

\vspace*{-6pt}

\hfill{\small\textit{Поступила в~редакцию 15.07.20}}

\vspace*{8pt}

%\pagebreak

%\newpage

%\vspace*{-28pt}

\hrule

\vspace*{2pt}

\hrule

%\vspace*{-2pt}

\def\tit{STATISTICAL ESTIMATION OF~DISTRIBUTIONS OF RANDOM COEFFICIENTS 
IN~THE~LANGEVIN 
STOCHASTIC DIFFERENTIAL EQUATION}


\def\titkol{Statistical estimation of~distributions of random coefficients 
in~the~Langevin 
stochastic differential equation}

\def\aut{A.\,K.~Gorshenin$^{1}$, V.\,Yu.~Korolev$^{1,2}$, and~A.\,A.~Shcherbinina$^{2}$}

\def\autkol{A.\,K.~Gorshenin, V.\,Yu.~Korolev, and A.\,A.~Shcherbinina}

\titel{\tit}{\aut}{\autkol}{\titkol}

\vspace*{-9pt}


\noindent
$^1$Federal Research Center ``Computer Science and Control'' 
of the Russian Academy of Sciences, 44-2~Vavilov\linebreak
$\hphantom{^1}$Str., 
Moscow 119333, Russian Federation

\noindent
$^2$Faculty of Computational Mathematics and Cybernetics, Lomonosov Moscow
State University, GSP-1, Leninskie\linebreak
$\hphantom{^1}$Gory, Moscow 119991, Russian Federation


\def\leftfootline{\small{\textbf{\thepage}
\hfill INFORMATIKA I EE PRIMENENIYA~--- INFORMATICS AND
APPLICATIONS\ \ \ 2020\ \ \ volume~14\ \ \ issue\ 3}
}%
 \def\rightfootline{\small{INFORMATIKA I EE PRIMENENIYA~---
INFORMATICS AND APPLICATIONS\ \ \ 2020\ \ \ volume~14\ \ \ issue\ 3
\hfill \textbf{\thepage}}}

\vspace*{3pt} 



\Abste{A method is described for statistical estimation
 of the distributions of random coefficients of the 
 Langevin stochastic differential equation (SDE) 
 by the technique of moving separation of mixtures. 
 Discrete approximations are proposed for these distributions. 
 For the purpose of study of variability of the distributions 
 of the SDE coefficients in time, an algorithm is proposed for 
 sequential identification (determination of local connectivity) 
 of the components of the resulting mixture distributions. 
 This algorithm is based on combining a greedy algorithm for 
 the determination of the number of components with 
 a~lustering method ($k$- or $c$-means). 
 The application of the proposed method is illustrated by particular 
 examples of the analysis of processes of 
heat transfer between atmosphere and ocean.}


\KWE{mixture distribution; local connectivity; greedy algorithm; clustering}





\DOI{10.14357/19922264200301} 

%\vspace*{-20pt}

\Ack
\noindent
The research was partially supported by the Russian Foundation 
for Basic Research
(project~19-07-00914) and the RF Presidential scholarship program 
(project No.\,538.2018.5). The research was conducted in accordance 
with the Program of Moscow Center for Fundamental and Applied Mathematics.

%\vspace*{6pt}

 \begin{multicols}{2}

\renewcommand{\bibname}{\protect\rmfamily References}
%\renewcommand{\bibname}{\large\protect\rm References}

{\small\frenchspacing
 {%\baselineskip=10.8pt
 \addcontentsline{toc}{section}{References}

 \begin{thebibliography}{99}
\bibitem{1-gor-1}
\Aue{Belyaev,~K., A.~Kuleshov, N.~Tuchkova, and C.\,A.\,S.~Tanajura.}
 2018. An optimal data assimilation method and its application to the 
 numerical simulation of the ocean dynamics. \textit{Math.
 Comp. Model. Dyn.} 1(24):12--25.

\bibitem{2-gor-1}
\Aue{Leland,~H.\,E.} 
1985. Option pricing and replication with transactions costs. 
\textit{J.~Financ.} 40:1283--1301.

\bibitem{3-gor-1}
\Aue{Barles,~G., and H.\,M.~Soner.} 
1998. Option pricing with transaction costs and a nonlinear 
Black--Scholes equation. \textit{Financ. Stoch.} 2:369--397.

\bibitem{4-gor-1}
\Aue{Heston,~S.\,L.} 1993. A~closed-form solution for options
 with stochastic volatility, with application to bond and 
 currency options. \textit{Rev. Financ. Stud.} 6:327--343.

\bibitem{5-gor-1}
\Aue{Cox,~J.\,C., J.\,E.~Ingersoll, and  S.\,A.~Ross.}
 1985. A~theory of the term structure of interest rates. 
 \textit{Econometrica} 53:385--407.

\bibitem{6-gor-1}
\Aue{Hull,~J., and A.~White.} 1987. The pricing of options on assets 
with stochastic volatilities. \textit{J.~Financ.} 42:281--308.

\bibitem{9-gor-1} %7
\Aue{Dupire,~B.} 1994. Pricing with a smile. \textit{Risk} 7:18--20.

\bibitem{8-gor-1}
\Aue{Derman,~E., and J.~Kani.} 1994. Riding on a smile. \textit{Risk} 7:32--39.

\bibitem{7-gor-1} %9
\Aue{Shiryaev,~A.\,N.} 2016. 
\textit{Osnovy stokhasticheskoy finansovoy matematiki. T.~1. Fakty. Modeli} 
[Foundations of stochastic financial mathematics. Vol.~1. Facts. Models]. 
Moscow: MCCME. 440~p.

\bibitem{10-gor-1} %10
\Aue{Korolev,~V.\,Yu.} 2011. 
\textit{Veroyatnostno-statisticheskie metody dekompozitsii
volatil'nosti khaoticheskikh protsessov} [Probabilistic and statistical 
methods of decomposition of volatility of chaotic processes]. Moscow: 
Izd-vo Moskovskogo un-ta. 512~p.

\bibitem{12-gor-1} %11 
\Aue{Gorshenin,~A.\,K., V.\,Yu.~Korolev, and A.\,M.~Tursunbaev.}
 2017. Median modifications of the EM-algorithm for separation of
 mixtures of probability distributions and their applications 
 to the decomposition of volatility of financial indexes.
 \textit{J.~Math. Sci.} 227(2):176--195.

\bibitem{11-gor-1} %12
\Aue{Gorshenin,~A.\,K.} 2015. On implementation of EM-type algorithms 
in the stochastic models for a matrix computing on GPU. 
\textit{AIP Conf. Proc.} 1648:250008. 4~p.

\bibitem{13-gor-1}
\Aue{Steinhaus,~H.} 1956. Sur la division des corps materiels 
en parties. \textit{B. Acad. Pol. Sci.} 4(12):801--804.

\bibitem{14-gor-1}
\Aue{MacQueen,~J.} 1967. Some methods for classification and analysis 
of multivariate observations. \textit{5th Berkeley Symposium on Mathematical 
Statistics and Probability Proceedings}. 281--297.

\bibitem{16-gor-1} %15
\Aue{Dunn,~J.\,C.} 1973. A~fuzzy relative of the ISODATA process and 
its use in detecting compact well-separated clusters. 
\textit{J.~Cybernetics} 3(3):32--57.

\bibitem{15-gor-1} %16
\Aue{David,~A., and S.~Vassilvitskii.}
 2007. K-means++: The advantages of careful seeding. 
 \textit{18th Annual ACM-SIAM Symposium on Discrete Algorithms
 Proceedings}. ACM. 1027--1035.

\bibitem{17-gor-1}
\Aue{Korolev,~V.\,Yu., A.\,K.~Gorshenin, S.\,K.~Gulev, and K.\,P.~Belyaev.}
 2015. Statistical modeling of air--sea turbulent heat fluxes by
  finite mixtures of Gaussian distributions. \textit{Comm. Com. Inf.
Sc.} 564:152--162.

\bibitem{18-gor-1}
\Aue{Gorshenin,~A.\,K.} 2016. Kontseptsiya onlayn-kompleksa 
dlya stokhasticheskogo modelirovaniya real'nykh pro\-tses\-sov 
[Concept of online service for stochastic modeling of real processes]. 
\textit{Informatika i~ee Primeneniya~--- Inform. Appl.} 10(1):72--81.

\bibitem{19-gor-1}
\Aue{Perry,~A.\,H., and J.\,M.~Walker.} 1977. 
\textit{Ocean--atmosphere system}. Upper Saddle River, NJ: 
Prentice Hall Press. 180~p.
\end{thebibliography}

 }
 }

\end{multicols}

\vspace*{-6pt}

\hfill{\small\textit{Received July 15, 2020}}

%\pagebreak

%\vspace*{-24pt}

\Contr

\noindent
\textbf{Gorshenin Andrey K.} (b.\ 1986)~--- 
Candidate of Science (PhD) in physics and
mathematics, associate professor, leading scientist, Federal 
Research Center ``Computer Science and Control'' of the Russian 
Academy of Sciences, 44-2~Vavilov Str., Moscow 119333, Russian Federation; 
leading scientist, Faculty of Computational Mathematics and Cybernetics 
and Moscow Center for Fundamental and Applied Mathematics, Lomonosov 
Moscow State University, GSP-1, Leninskie Gory, Moscow 119991, 
Russian Federation; \mbox{agorshenin@frccsc.ru}

\vspace*{3pt}

\noindent
\textbf{Korolev Victor Yu.} (b.\ 1954)~--- 
Doctor of Science in physics and
mathematics, professor, head of department, Faculty of Computational 
Mathematics and Cybernetics, and principal scientist, 
Moscow Center for Fundamental and Applied Mathematics, 
Lomonosov Moscow State University, GSP-1, Leninskie Gory, Moscow 
119991, Russian Federation; leading scientist, Federal Research Center 
``Computer Science and Control'' of the Russian Academy of Sciences, 
44-2~Vavilov Str., Moscow 119333, Russian Federation; \mbox{vkorolev@cs.msu.ru}

\vspace*{3pt}

\noindent
\textbf{Shcherbinina Anastasia A.} (b.\ 1997)~--- Master student,  Faculty
of Computational Mathematics and Cybernetics, 
Lomonosov Moscow State University, GSP-1, Leninskie Gory, 
Moscow 119991, Russian Federation; \mbox{shcherbinina.aa.97@gmail.com}


\label{end\stat}

\renewcommand{\bibname}{\protect\rm Литература} 