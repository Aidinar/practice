
\documentclass[10pt]{book}
\usepackage[utf8]{inputenc}

\usepackage{latexsym,amssymb,amsfonts,amsmath,amsxtra,dsfont,
indentfirst,shapepar,%fleqn,%
picinpar,shadow,floatflt,enumerate,multicol,colortbl,moreverb,cite,ipi}

\usepackage{rotating}
\usepackage{mathrsfs}
\usepackage[noend]{algorithmic}
\usepackage{ulem}
\usepackage{graphicx}
%\usepackage{algorithm2e}
\usepackage[linesnumbered,boxed,ruled]{algorithm2e}
%\usepackage{xypic}
\usepackage{oldgerm}
\usepackage{epic}
\usepackage{eepic}

\SetAlgorithmName{Algorithm}{алгоритм}{Список алгоритмов}

%из Дюковой

\newcommand{\algKeyword}[1]{{\bf #1}}
\newcommand{\Proc}[1]{\text{\tt #1}}
\def\CALL{\algKeyword{call}~}

\newenvironment{AlgProcedure}[1]
{
    \small
    \medskip
    %    \hrule
    \medskip
    \algKeyword{PROCEDURE} #1
    \begin{algorithmic}[1]}
    {\end{algorithmic}
    %    \hrule
    \bigskip
}

\def\CALL{\algKeyword{call}~}

%конец для Дюковой

%\RequirePackage[ruled]{algorithm}


\input{epsf}

%\nofiles

%\includeonly{avtor}             %+pdf+
%\includeonly{obchak,avtor}
%\includeonly{pred}                 %+
%\includeonly{podgot-rus-site,podgot-eng-site}  
%\includeonly{podgot-rus,podgot-eng}  
%\includeonly{ocherk} 
%\includeonly{ipi-ind} 
%\includeonly{index13}
%\includeonly{toc-rus, toc-en}
%\includeonly{toc-rus}
%\includeonly{toc-en} 

%ИИЕП 2020-3
%\includeonly{gorshenin+kor}    %+01 pdf
%\includeonly{naumov}           %+02 pdf
%\includeonly{posypkin}         %+03 pdf
%\includeonly{razumchik}        %+04 pdf 
%\includeonly{korolev}          %+05 pdf+авт
%\includeonly{shestakov}        %+06 pdf+авт
%\includeonly{kudr}             %+07 pdf+авт
%\includeonly{betelin}          %+08 pdf+авт
%\includeonly{agasandyan}       %+09pdf
%\includeonly{grusho}           %+10 pdf+авт
%\includeonly{grusho-timonina}  %+11 pdf
%\includeonly{malashenko}       %+12 pdf
%\includeonly{bosov}            %+13pdf
%\includeonly{sopin}            %+14pdf
%\includeonly{shnurkov}         %+15pdf
%\includeonly{donskoy}          %+16pdf
%\includeonly{zatsman}          %+17
%\includeonly{krasnov}          %+18  pdf


%\includeonly{nekrolog-RB}      %+pdf


%\includeonly{obchak}
%\includeonly{rekl}
%\includeonly{rekl-1}
%\includeonly{reshal}  %
%\includeonly{cover3}

\usepackage{acad}
%\usepackage{courier}
\usepackage{decor}
\usepackage{newton}
\usepackage{pragmatica}
\usepackage{zapfchan}
\usepackage{petrotex}
\usepackage{bm}                     % полужирные греческие буквы
\usepackage{upgreek}                % прямые греческие буквы
\usepackage{eufrak}
\usepackage{verbatim}

\renewcommand{\bottomfraction}{0.99}
\renewcommand{\topfraction}{0.99}
\renewcommand{\textfraction}{0.01}

\setcounter{secnumdepth}{1} %здесь - 3 + chapter = 4

\arraycolsep=1.5pt

%\usepackage[pdftex]{graphicx}

%\usepackage{oz}

%NEW COMMANDS


\renewcommand*{\hm}[1]{#1\nobreak\discretionary{}%
            {\hbox{$\mathsurround=0pt #1$}}{}} %% Дублирует знаки операций
                               %при переносе в формуле (перед знаком, который
                               %надо продублировать ставится команда \hm)

%\newcommand{\endproof}{\hfill$\Box$}
\renewcommand{\r}{\mathbb{R}}
%\newcommand{\I}{{\rm I\hspace{-0.7mm}I}}
%\newcommand{\Ikl}{{\tt{1}}\hspace*{-1.44mm}\mathtt{1}}
\newcommand{\Ik}{\mbox{{\small \tt {1}}\hspace{-1.3mm}{\tt 1}}}
\newcommand{\argmin}{\mathop{\mathrm{arg}\,\mathrm{min}}}
\newcommand{\argmax}{\mathop{\mathrm{arg}\,\mathrm{max}}}
%\newcommand{\capr}{\mathop{\cap\,}}
%\newcommand{\cupr}{\mathop{\cup\,}}
%\def\argmin{\mathop{arg\,min}}

\def\vrp{\varphi}
\def\prt{\partial}
\def\mm{{\sf M}}
\def\modnop#1{\mathop{#1}\limits_{n}}
\def\eam{\mathbin{{\mathop{=}\limits^{\mathrm{def}}}}}
\def\dey#1#2{#1 (#2)}
\def\deyc#1#2{#1 \cdot  #2}
\def\ra#1{\;\mathop{\to}\limits^{#1}\;}
\def\raz#1{\;\mathop{\longrightarrow}\limits^{\!\!\!#1}\;}
\def\ral#1{\;\mathop{\longrightarrow}\limits^{#1}\;}

\newcommand{\Nor}{\mathcal{N}}
\newcommand{\T}{\mathbb{T}}
\newcommand{\Z}{\mathbb{Z}}



\newcommand{\il}[2]{\int\limits_{#1}^{#2}}%интеграл с пределами #1 и #2

\def\sm2{\mathop {\sum\limits^{n^\Theta}\sum\limits^{n^\Theta}}}
\def\sss{\sum\limits}
\def\tr{,\,\ldots\,,\,}
\def\rk{\right]}
\def\lk{\left[}
\def\rf{\right\}}
\def\lf{\left\{}
\def\lv{\,\left\vert}
\def\rv{\right\vert\,}
\def\iii{\int\limits}
\def\iin{\int\limits_{-\infty}^\infty}
\def\rrv{\right\vert}


\def\ee{{\cal E}}
\def\ww{{\cal W}}
\def\yy{{\cal Y}}
\def\vv{{\cal V}}

\newcommand{\R}{\mathbb R}
\newcommand{\E}{\mathbb E}
\newcommand{\N}{\mathbb N}

\renewcommand{\P}{\mathbb{P}}

\newcommand{\h}{{\bf H}}
\newcommand{\p}{{\sf P}}  % вероятность

\newcommand{\e}{{\sf E}}  % мат. ожидание
\newcommand{\D}{{\sf D}}  % дисперсия
\newcommand{\eps}{\varepsilon}
\newcommand{\vp}{{\mathbf p}}
\newcommand{\vz}{{\mathbf z}}
\newcommand{\vx}{{\mathbf x}}
\newcommand{\vf}{{\mathbf f}}
\newcommand{\F}{{\mathcal F}}
\def\ap{{\mathrm{ЭР}}}
\newcommand{\ud}{\Delta_n} %uniform ditance
\newcommand{\nud}{\Delta_n(x)}
%\renewcommand{\Re}{\mathrm{Re}\,}

\newcommand{\abs}[1]{\left\vert#1\right\vert}

\newcommand{\norm}[1]{\left\Vert#1\right\Vert}
\def\da{(\Delta_t,A)}

\newcommand{\corr}{\mathrm{corr}}

\newcommand{\cov}{\mathrm{cov}}
\newcommand{\Expect}{\mathbb{E}}

\def\w{\omega}
\def\W{\Omega}

\def\inh{\int\limits_{nh}^{(n+1)h}}

\def\sumin{\sum_{i=1}^N}


\def\bxt{(Y,t)}
\def\xt{(y,t)}

\def\ovth{{\fr{\tau-nh}{h}}}
\def\ov{\overline}
\def\tm{\tilde m}
\def\tl{\tilde\lambda}
\def\tB{\widetilde B}
\def\tb{\tilde b}
\def\ld{\ldots}
\def\cd{\cdots}


\DeclareMathOperator{\sign}{sign}

%\newcommand{\gr}{{\geqslant}}


\newcommand{\g}{\mbox{\textit{g}}}

\renewcommand{\la}{\lambda}
\newcommand{\si}{\sigma}
\newcommand{\alp}{\alpha}

\newcommand{\pto}{\stackrel{P}{\longrightarrow}} % сходимость по веpоятности

\newcommand{\eqd}{\stackrel{\mathrm{d}}{=}} % равенство по pаспpеделению
\newcommand{\eqdelta}{\stackrel{\triangle}{=}} % равенство по pаспpеделению

\def\be#1{\begin{equation}\label{#1}}
\def\ee{\end{equation}}
\def\re#1{(\ref{#1})}

\def\bn{\begin{enumerate}}
\def\en{\end{enumerate}}
\def\bi{\begin{itemize}}
\def\ei{\end{itemize}}
%\def\i{\item}

%\newcommand{\kp}{\kappa}
%\def\Q{{\cal Q}} \def\H{{\cal H}}
%\newcommand{\bet}{\beta_{2+\delta}}


%\newtheorem{definition}{Определение}
%\renewcommand{\thedefinition}{\arabic{definition}.}
%END NEW COMMANDS

%\renewcommand{\baselinestretch}{1.2}

%\pagestyle{myheadings}

\setlength{\textwidth}{167mm}      % 122mm
\setlength{\textheight}{658pt}
%\setlength{\textheight}{635.6pt}
\setlength{\columnsep}{4.5mm}

\setcounter{secnumdepth}{4}

%\addtolength{\headheight}{2pt}
%\addtolength{\headsep}{-2mm}

\addtolength{\topmargin}{-7mm}  % for printing


%\hoffset=-30mm  % From Yap
\hoffset=-23mm  % From Acrobat

%\voffset=0mm % From Yap
\voffset=-5mm   % From Acrobat

%\addtolength{\evensidemargin}{-2.5mm} % for printing
%\addtolength{\oddsidemargin}{2.5mm}  % for printing

\addtolength{\evensidemargin}{-12mm} % for printing
\addtolength{\oddsidemargin}{8mm}  % for printing

%\renewcommand{\thefootnote}{\fnsymbol{footnote}}
%\renewcommand{\thefootnote}{\arabic{footnote}}
\renewcommand{\figurename}{\protect\bf Рис.}
\renewcommand{\tablename}{\protect\bf Таблица}

\newcommand{\Caption}[1]{\caption{\protect\small %\baselineskip=2.5ex
#1}}

\renewcommand{\thefigure}{\arabic{figure}}
\renewcommand{\thetable}{\arabic{table}}
\renewcommand{\theequation}{\arabic{equation}}
\renewcommand{\thesection}{\arabic{section}}

\renewcommand{\contentsname}{СОДЕРЖАНИЕ}
\newcommand{\fr}[2]{\displaystyle\frac{\displaystyle #1\mathstrut}{\displaystyle #2\mathstrut}}

%\renewcommand{\thefootnote}{\fnsymbol{footnote}}
%\newcommand{\g}{\mbox{\textit{g}}}

%\newcommand{\Caption}[1]{\caption{\protect\small\baselineskip=2ex #1}}
\newcounter{razdel}
\setcounter{razdel}{0}

\def\god{2020}
\def\tom{14}
\def\vyp{3}


\newcommand{\titel}[4]{%
\

\vspace*{5pt}

\ifodd\therazdel {\raggedright\noindent\Large\textrm\textbf
 \lineskip .75em
  \baselineskip=3.2ex #1 \par}
\vskip 1em {\noindent\large\textrm\textbf #2 \par}
\addcontentsline{toc}{subsection}{{\textrm\textbf #1}\protect\newline #2}
\def\rightheadline{\underline{\noindent\hbox to \textwidth{\hfill\small\textrm{#4}
%\hfill \large\bf\thepage
}}}
\def\leftheadline{\underline{\noindent\parbox{\textwidth}{
%\raggedleft\large\bf\thepage \hfill
\small\textit{#3}\hfill}}}
\def\leftfootline{\small{\textbf{\thepage}
\hfill ИНФОРМАТИКА И ЕЁ ПРИМЕНЕНИЯ\ \ \ том~\tom\ \ \ выпуск~\vyp\ \ \ \god}
}%
 \def\rightfootline{\small{ИНФОРМАТИКА И ЕЁ ПРИМЕНЕНИЯ\ \ \ том~\tom\ \ \ выпуск~\vyp\ \ \ \god
\hfill \textbf{\thepage}}}
\vskip 2em \setcounter{figure}{0}
\setcounter{table}{0}
\setcounter{equation}{0}
\setcounter{section}{0}
\setcounter{subsection}{0}
\setcounter{subsubsection}{0}
\setcounter{footnote}{0}
\setcounter{razdel}{0}
%\end{flushleft}
\else {
 \raggedright\noindent\Large\textrm\textbf
 \lineskip .75em
\baselineskip=3.2ex #1 \par} \vskip 1em
%\begin{flushleft}
{\noindent\large\textrm\textbf #2 \par}
\addcontentsline{toc}{subsection}{{\textrm\textbf #1}\protect\newline #2}
\def\rightheadline{\underline{\noindent\hbox to \textwidth{\hfill\small\textrm{#4}
%\hfill \large\bf\thepage
}}}
\def\leftheadline{\underline{\noindent\parbox{\textwidth}{%\raggedleft\large\bf\thepage \hfill
\small\textit{#3}\hfill}}}
\def\leftfootline{\small{\textbf{\thepage}
\hfill ИНФОРМАТИКА И ЕЁ ПРИМЕНЕНИЯ\ \ \ том~\tom\ \ \ выпуск~\vyp\ \ \ \god}
}%
 \def\rightfootline{\small{ИНФОРМАТИКА И ЕЁ ПРИМЕНЕНИЯ\ \ \ том~14\ \ \ выпуск~3\ \ \ 2020
\hfill \textbf{\thepage}}} \vskip 2em \setcounter{figure}{0}
\setcounter{table}{0} \setcounter{equation}{0} \setcounter{section}{0}
\setcounter{subsection}{0} \setcounter{subsubsection}{0}
\setcounter{footnote}{0}
%\end{flushleft}
\fi}

\newcommand{\titelr}[2]{%
\

\vspace*{5pt}

\ifodd\therazdel {\raggedright\noindent%\Large\textrm\textbf
 \lineskip .75em
  \baselineskip=3.2ex #1 \par}
\vskip 1em {\noindent\normalsize\textrm\textbf #2 \par}
\else {
 \raggedright\noindent\Large\textrm\textbf
 \lineskip .75em
\baselineskip=3.2ex #1 \par} \vskip 1em
%\begin{flushleft}
{\noindent\large\textrm\textbf #2 \par
%\noindent\normalsize\textrm\textbf #2 \par
} \fi}

\newcommand{\titele}[5]{%
\

%\vspace*{5pt}

\ifodd\therazdel {\raggedright\noindent\large
\textrm\textbf
 \lineskip .75em
%  \baselineskip=3.2ex
#1 \par}
\vskip .5em {\noindent\large\textrm\textbf #2 \par}
\vskip .5em
 {\noindent\textrm #3 \par}
\addcontentsline{toc}{subsection}{{\textrm\textbf #1}\protect\newline #2}
\def\rightheadline{\underline{\noindent\hbox to \textwidth{\hfill\small\textrm{#4}
%\hfill \large\bf\thepage
}}}
\def\leftheadline{\underline{\noindent\parbox{\textwidth}{
%\raggedleft\large\bf\thepage \hfill
\small\textrm{#5}\hfill}}}
\def\leftfootline{\small{\textbf{\thepage}
\hfill ИНФОРМАТИКА И ЕЁ ПРИМЕНЕНИЯ\ \ \ том~14\ \ \ выпуск~3\ \ \ 2020}
}%
 \def\rightfootline{\small{ИНФОРМАТИКА И ЕЁ ПРИМЕНЕНИЯ\ \ \ том~14\ \ \ выпуск~3\ \ \ 2020
\hfill \textbf{\thepage}}} \vskip 1em \setcounter{figure}{0}
\setcounter{table}{0} \setcounter{equation}{0} \setcounter{section}{0}
\setcounter{subsection}{0} \setcounter{subsubsection}{0}
\setcounter{footnote}{0} \setcounter{razdel}{0}
%\end{flushleft}
\else {
 \raggedright\noindent\large
 \textrm\textbf
 \lineskip .75em
%\baselineskip=3.2ex
#1 \par} \vskip .5em
%\begin{flushleft}
{\noindent\large\textrm\textbf #2 \par} \vskip .5em
 {\noindent\textrm #3 \par}
\addcontentsline{toc}{subsection}{{\textrm\textbf #1}\protect\newline #2}
\def\rightheadline{\underline{\noindent\hbox to \textwidth{\hfill\small\textrm{#4}
%\hfill \large\bf\thepage
}}}
\def\leftheadline{\underline{\noindent\parbox{\textwidth}{%\raggedleft\large\bf\thepage \hfill
\small\textrm{#5}\hfill}}}
\def\leftfootline{\small{\textbf{\thepage}
\hfill ИНФОРМАТИКА И ЕЁ ПРИМЕНЕНИЯ\ \ \ том~14\ \ \ выпуск~3\ \ \ 2020}
}%
 \def\rightfootline{\small{ИНФОРМАТИКА И ЕЁ ПРИМЕНЕНИЯ\ \ \ том~14\ \ \ выпуск~3\ \ \ 2020
\hfill \textbf{\thepage}}} \vskip 1em \setcounter{figure}{0}
\setcounter{table}{0} \setcounter{equation}{0} \setcounter{section}{0}
\setcounter{subsection}{0} \setcounter{subsubsection}{0}
\setcounter{footnote}{0}
%\end{flushleft}
\fi}

\def\Abst#1{
\begin{center}\small\nwt
\parbox{150mm}{%\baselineskip=2.5ex
\textbf{Аннотация:}\ \
%\hspace*{\parindent}
#1}
\end{center}}
\def\Abste#1{
\begin{center}\small\nwt
\parbox{150mm}{%\baselineskip=2.5ex
\textbf{Abstract:}\ \
%\hspace*{\parindent}
#1}
\end{center}}

\def\DOI#1{
\begin{center}\small\nwt
\parbox{150mm}{%\baselineskip=2.5ex
\textbf{DOI:}\ \
%\hspace*{\parindent}
#1}
\end{center}}

\def\Abstend#1{
\begin{center}\small\nwt
\parbox{150mm}{%\baselineskip=2.5ex
%\hspace*{\parindent}
#1}
\end{center}}


\def\KW#1{
\begin{center}\small\nwt
\parbox{150mm}{%\baselineskip=2.5ex
\textbf{Ключевые слова:}\ \ #1}
\end{center}}

\def\KWE#1{
\begin{center}\small\nwt
\parbox{150mm}{%\baselineskip=2.5ex
\textbf{Keywords:}\ \ #1}
\end{center}}


\def\KWN#1{
%\begin{center}
%\small
%\parbox{150mm}\end{center}
}

\newcommand{\Avtors}[1]{%\smallskip
%\vspace*{.5pt}
\hangindent=23pt\noindent
%\nwt
{\bfseries#1}\
}


\renewcommand{\thesubsection}{\thesection.\arabic{subsection}\hspace*{-5pt}}
\renewcommand{\thesubsubsection}{\thesubsection\hspace*{5pt}.\arabic{subsubsection}\hspace*{-3pt}}

\newcommand{\Ack}{\section*{\protect\rmfamily Acknowledgments}\noindent}
\newcommand{\Contr}{\section*{\protect\rmfamily Contributors}\noindent}
\newcommand{\Contrl}{\section*{\protect\rmfamily Contributor}\noindent}

\makeindex


\begin{document}
\Rus

\nwt
%\ptb


%\renewcommand{\contentsname}{\protect\Large\bf Содержание}

\setcounter{tocdepth}{2}

%\tableofcontents

\renewcommand{\bibname}{\protect\rmfamily Литература}
  \def\Au#1{{\it #1}}
    \def\Aue#1{{#1}}

%\newcommand{\No}{№}
  \newcommand{\tg}{\,\mathrm{tg}\,}
    \newcommand{\ctg}{\,\mathrm{ctg}\,}
  \newcommand{\arctg}{\,\mathrm{arctg}\,}

\def\forallb{\mathop{\forall}}
\def\cupb{\mathop{\cup}}
\def\existsb{\mathop{\exists}}


\newpage
\addtocounter{razdel}{1}
%\def\razd{РЕГУЛИРУЕМЫЙ ЭЛЕКТРОПРИВОД ДЛЯ ЭЛЕКТРОЭНЕРГЕТИКИ}


\setcounter{page}{3}

%   { %\Large  
   { %\baselineskip=16.6pt
   
   \vspace*{-48pt}
   \begin{center}\LARGE
   \textit{Предисловие}
   \end{center}
   
   %\vspace*{2.5mm}
   
   \vspace*{25mm}
   
   \thispagestyle{empty}
   
   { %\small 

    
Вниманию читателей журнала <<Информатика и её применения>> предлагается 
очередной тематический выпуск <<Вероятностно-статистические методы и 
задачи информатики и информационных технологий>>. Предыдущие тематические 
выпуски журнала по данному направлению вышли в 2008~г.\ (т.~2, вып.~2), 
в 2009~г.\ (т.~3, вып.~3) и в 2010~г.\ (т.~4, вып.~2). 

Статьи, собранные в данном журнале, посвящены разработке новых вероятностно-статистических 
методов, ориентированных на применение к решению конкретных задач информатики и информационных 
технологий, а также~--- в ряде случаев~--- и других прикладных задач. Проблематика, охватываемая 
публикуемыми работами, развивается в рамках научного сотрудничества между Институтом проблем 
информатики Российской академии наук (ИПИ РАН) и Факультетом вычислительной математики и 
кибернетики Московского государственного университета им.\ М.\,В.~Ломоносова в ходе работ 
над совместными научными проектами (в том числе в рамках функционирования 
Научно-образовательного центра <<Вероятностно-статистические методы анализа рисков>>). 
Многие из авторов статей, включенных в данный номер журнала, являются активными участниками 
традиционного международного семинара по проблемам устойчивости стохастических моделей, 
руководимого В.\,М.~Золотаревым и В.\,Ю.~Королевым; регулярные сессии этого семинара 
проводятся под эгидой МГУ и ИПИ РАН (в 2011~г.\ указанный семинар проводится в октябре 
в Калининградской области РФ). 

Наряду с представителями ИПИ РАН и МГУ в число авторов данного выпуска журнала входят 
ученые из Научно-исследовательского института системных исследований РАН, Института 
проблем технологии микроэлектроники и особочистых материалов РАН, Института 
прикладных математических исследований Карельского НЦ РАН, Московского 
авиационного института, Вологодского государственного педагогического университета, 
НИИММ им.\ Н.\,Г.~Чеботарева, Казанского государственного университета, Дебреценского 
университета (Венгрия).

Несколько статей выпуска посвящено разработке и применению стохастических методов и 
информационных технологий для решения различных прикладных задач. В~работе В.\,Г.~Ушакова 
и О.\,В.~Шестакова рассмотрена задача определения вероятностных характеристик случайных 
функций по распределениям интегральных преобразований, возникающих в задачах эмиссионной 
томографии. В~статье Д.\,О.~Яковенко и М.\,А.~Целищева рассмотрены некоторые вопросы 
математической теории риска и предложен новый подход к диверсификации инвестиционных 
портфелей. Работа И.\,А.~Кудрявцевой и А.\,В.~Пантелеева посвящена построению и 
исследованию математической модели, описывающей динамику сильноионизованной плазмы. 
В~статье П.\,П.~Кольцова изучается качество работы ряда алгоритмов сегментации изображений. 
Статья А.\,Н.~Чупрунова и И.~Фазекаша посвящена вероятностному анализу числа без\-оши\-бочных 
блоков при помехоустойчивом кодировании; получены усиленные законы больших чисел для указанных 
величин.

В данном выпуске традиционно присутствует тематика, весьма активно разрабатываемая в течение 
многих лет специалистами ИПИ РАН и МГУ,~--- методы моделирования и управления для 
информационно-телекоммуникационных и вычислительных систем, в частности методы 
теории массового обслуживания. В~статье А.\,И.~Зейфмана с соавторами рассматриваются 
модели обслуживания, описываемые марковскими цепями с непрерывным временем в случае 
наличия катастроф. В~работе М.\,М.~Лери и И.\,А.~Чеплюковой рассматриваются случайные 
графы Интернет-типа, т.\,е.\ графы, степени вершин которых имеют степенные распределения; 
такие задачи находят применение при исследовании глобальных сетей передачи данных. 
Работа Р.\,В.~Разумчика посвящена исследованию систем массового обслуживания специального 
вида~--- с отрицательными заявками и хранением вытесненных заявок.

Ряд статей посвящен развитию перспективных теоретических 
вероятностно-статистических методов, которые находят широкое применение в различных 
задачах информатики и информационных технологий. В~работе В.\,Е.~Бенинга, А.\,К.~Горшенина 
и В.\,Ю.~Королева рассмотрена задача статистической проверки гипотез о числе компонент 
смеси вероятностных распределений, приводится конструкция асимптотически наиболее мощного 
критерия. Результаты этой работы найдут применение в ряде прикладных задач, использующих 
математическую модель смеси вероятностных распределений (в информатике, моделировании 
финансовых рынков, физике турбулентной плазмы и~т.\,д.). В~статье В.\,Ю.~Королева, 
И.\,Г.~Шевцовой и С.\,Я.~Шоргина строится новая, улучшенная оценка точности нормальной 
аппроксимации для пуассоновских случайных сумм; как известно, указанные случайные суммы 
широко используются в качестве моделей многих реальных объектов, в том числе в информатике, 
физике и других прикладных областях. Работа В.\,Г.~Ушакова и Н.\,Г.~Ушакова посвящена 
исследованию ядерной оценки плотности распределения; эти результаты могут применяться, 
в част\-ности, при анализе трафика в телекоммуникационных системах. Серьезные приложения 
в статистике могут получить результаты работы О.\,В.~Шестакова, в которой доказаны оценки 
скорости сходимости распределения выборочного абсолютного медианного отклонения к нормальному 
закону. 

\smallskip

Редакционная коллегия журнала выражает надежду, что данный тематический  выпуск 
будет интересен специалистам в области теории вероятностей и математической статистики 
и их применения к решению задач информатики и информационных технологий.
     
     %\vfill 
     \vspace*{20mm}
     \noindent
     Заместитель главного редактора журнала <<Информатика и её 
применения>>,\\
     директор ИПИ РАН, академик  \hfill
     \textit{И.\,А.~Соколов}\\
     
     \noindent
     Редактор-составитель тематического выпуска,\\
     профессор кафедры математической статистики факультета\\
      вычислительной математики и кибернетики МГУ им.\ М.\,В.~Ломоносова,\\
     ведущий научный сотрудник ИПИ РАН,\\ 
доктор физико-математических наук \hfill
      \textit{В.\,Ю.~Королев}
     
     } }
     }


\def\stat{gorsh+kor}

\def\tit{СТАТИСТИЧЕСКОЕ ОЦЕНИВАНИЕ РАСПРЕДЕЛЕНИЙ СЛУЧАЙНЫХ КОЭФФИЦИЕНТОВ СТОХАСТИЧЕСКОГО 
ДИФФЕРЕНЦИАЛЬНОГО УРАВНЕНИЯ ЛАНЖЕВЕНА$^*$}

\def\titkol{Статистическое оценивание распределений случайных коэффициентов СДУ
%стохастического дифференциального уравнения 
Ланжевена}

\def\aut{А.\,К.~Горшенин$^1$, В.\,Ю.~Королев$^2$,  А.\,А.~Щербинина$^3$}

\def\autkol{А.\,К.~Горшенин, В.\,Ю.~Королев,  А.\,А.~Щербинина}

\titel{\tit}{\aut}{\autkol}{\titkol}

\index{Горшенин А.\,К.}
\index{Королев В.\,Ю.}
\index{Щербинина А.\,А.}
\index{Gorshenin A.\,K.}
\index{Korolev V.\,Yu.}
\index{Shcherbinina A.\,A.}
 

{\renewcommand{\thefootnote}{\fnsymbol{footnote}} \footnotetext[1]
{Исследования выполнены при частичной поддержке РФФИ (проект 19-07-00914), 
Стипендии Президента Российской Федерации молодым ученым 
и~аспирантам (СП-538.2018.5) и~в~соответствии с~программой Московского
 центра фундаментальной и~прикладной математики.}}


\renewcommand{\thefootnote}{\arabic{footnote}}
\footnotetext[1]{Федеральный исследовательский центр 
<<Информатика и~управление>> 
Российской академии наук, \mbox{agorshenin@frccsc.ru}}
\footnotetext[2]{Факультет вычислительной математики и~кибернетики
Московского государственного университета имени М.\,В.~Ломоносова;
Федеральный исследовательский центр <<Информатика и~управление>> 
Российской академии наук,
\mbox{vkorolev@cs.msu.ru}}
\footnotetext[3]{Факультет вычислительной математики и~кибернетики
Московского государственного университета имени М.\,В.~Ломоносова,
\mbox{shcherbinina.aa.97@gmail.com}}

%\vspace*{-6pt}

\Abst{Разработан метод статистического оценивания распределений 
случайных коэффициентов стохастических дифференциальных уравнений (СДУ) 
типа Ланжевена с~помощью техники скользящего разделения смесей (СРС). 
Предложены дискретные аппроксимации для оценок указанных распределений. 
С~целью изучения из\-мен\-чи\-вости распределений коэффициентов сдвига 
(дрейфа) и~диффузии СДУ во времени предложен алгоритм последовательной 
идентификации (определения локальной связ\-ности) компонент по\-лу\-ча\-емых смесей. 
В~его основу положена комбинация жад\-но\-го алгоритма для поиска чис\-ла 
компонент и~одного из методов кластеризации ($k$- или $c$-сред\-них). 
Применение метода иллюстрируется конкретными примерами анализа процесса 
теплообмена между атмосферой и~океаном для Гольфстрима и~тропиков.}

\KW{стохастические дифференциальные уравнения; смешанные распределения; 
локальная связанность; жад\-ный алгоритм; клас\-те\-ри\-зация}

\DOI{10.14357/19922264200301} 

%\vspace*{-6pt}

\vskip 10pt plus 9pt minus 6pt

\thispagestyle{headings}

\begin{multicols}{2}

\label{st\stat}

\section{Введение}

В физике СДУ Ланжевена принято называть сле\-ду\-ющее соотношение:
\begin{equation}
\label{LangevinEq}
dX(t)=a(t)\,dt+b(t)\,dW\,,
\end{equation}
определяющее случайный процесс~$X(t)$, где $W(t)$~--- винеровский процесс, 
а~коэффициенты $a(t)$ и~$b(t)$ случайны. Стохастические дифференциальные 
уравнения вида~\eqref{LangevinEq} 
широко используются, например, в~задаче ассимиляции данных 
при анализе\linebreak раз\-но\-мас\-штаб\-ной из\-мен\-чи\-вости геофизических 
пе\-ре\-менных~\cite{Belyaev2018}. В~финансовой математике из\-вест\-ны 
специальные вер\-сии урав\-не\-ния~\eqref{LangevinEq}. 
В~частности, весьма популярна модель геометрического броуновского 
движения вида:
\begin{equation}
\label{BrownMotionEq}
dX(t)=aX(t)\,dt+b X(t)\,dW\,,
\end{equation}
где $a\in\mathbb{R}$, $b\hm>0$. Известно много обобщений 
модели~\eqref{BrownMotionEq} c~конкретными видами за\-ви\-си\-мости~$a$ и~$b$ 
от~$X(t)$ и~других случайных процессов, например модели 
Леланда~\cite{Leland1985}, Барл\-са--Со\-не\-ра~\cite{BarlesSoner1998}, 
Хестона~\cite{Heston1993}, Кок\-са--Ин\-гер\-сол\-ла--Рос\-са~\cite{CoxIngersollRoss1985}, 
Хал\-ла--Уай\-та~\cite{HullWhite1987} и~другие так называемые модели 
стохастической волатильности (см.\ 
также~\cite{DermanKani1994, Dupire1994, Shiryaev2016}).

При отсутствии априорной информации о~<<физической>> структуре процесса~$X(t)$ 
для успешного изучения и~прогнозирования его эволюции первостепенную 
важ\-ность приобретает задача  статистического оценивания функциональных 
коэффициентов~$a(t)$ и~$b(t)$. В~силу их слу\-чай\-ности данная задача допускает 
как минимум две принципиально разные формулировки. Во-пер\-вых, можно 
пытаться найти (случайные же) оценки самих функций~$a(t)$ и~$b(t)$, т.\,е.\ 
найти их точечные аппроксимации, и,~во-вто\-рых, пытаться статистически 
оценить распределения случайных величин $a(t)$ и~$b(t)$. 
Во втором случае, зная ка\-кие-ли\-бо свойства этих коэффициентов, например 
структуру их функциональной зависимости от исходного процесса $X(t)$ 
(как в~упомянутых выше моделях Леланда, Барл\-са--Со\-не\-ра, Хестона, 
Кок\-са--Ин\-гер\-сол\-ла--Рос\-са или Беляева), мож\-но \mbox{найти} оценки 
чис\-ло\-вых па\-ра\-мет\-ров этих моделей.

В данной статье рассматривается вторая задача. 
Пусть $n\hm\geqslant1$ и~$t_0\hm=0\hm<t_1<\cdots<t_n$~--- 
моменты времени, в~которые наблюдается процесс~$X(t)$. 
Для простоты предположим, что $t_i\hm-t_{i-1}\hm=1$ для любого $i\hm\geqslant1$. 
Из уравнения~\eqref{LangevinEq} видно, что распределение приращения 
$X(t_i)\hm-X(t_{i-1})$ процесса~$X(t)$ можно аппроксимировать 
распределением вида:
\begin{equation}
\label{DiffApprox}
\mathbb{P}\left(X(t_i)-X(t_{i-1})<x\right)\approx \mathbb{E}\Phi\left(\fr{x-A_i}{B_i}\right),
\end{equation}
где $\Phi(x)$~--- стандартная нормальная функция распределения;
$A_i\hm\in\mathbb{R}$ и~$B_i\hm>0$~--- случайные величины. 
В~свою очередь, для распределений случайных величин~$A_i$ и~$B_i$, 
по отношению к~которым берется математическое ожидание 
в~формуле~\eqref{DiffApprox}, можно использовать дискретную аппроксимацию. 
Тогда вместо выражения~\eqref{DiffApprox} для распределения
 приращения $X(t_i)\hm-X(t_{i-1})$ можно применить приближение вида:

\noindent
\begin{equation}
\label{DiffDiscrApprox}
\mathbb{P}\left(X(t_i)-X(t_{i-1})<x\right)\approx\sum\limits_{k=1}^K
p_k\Phi\left(\fr{x-a_k}{b_k}\right),
\end{equation}
где $K\in\mathbb{N}$, $p_k\hm\geqslant0$, $k\hm=\overline{1,K}$, $\sum\nolimits_{k} p_k\hm=1$. 
Очевидно, параметры~$p_k$, $a_k$ и~$b_k$, формирующие так на\-зы\-ва\-емые 
динамические и~диффузионные компоненты~\cite{Korolev2011}, 
зависят также от~$i$ и~изменяются при переходе от~$t_i$ к~$t_{i+1}$.

Для статистического оценивания параметров~$p_k$, $a_k$ и~$b_k$ можно 
использовать %метод скользящего разделения смесей (
СРС-ме\-тод, описанный 
в~\cite{Korolev2011}. Статистические закономерности поведения 
рассматриваемых процессов $X(t)$, $a(t)$ и~$b(t)$ изменяются во 
времени, вообще говоря, нерегулярным образом, результатом чего 
является отсутствие универсального смешивающего закона. 
Поэтому изучение динамики изменения статистических 
закономерностей в~поведении ис\-сле\-ду\-емо\-го процесса проводится 
на последовательных интервалах времени. Тем самым па\-ра\-мет\-ры смесей 
(сдвига (дрейфа)~$a_k$, масштаба (диффузии) $b_k$ и~веса компонент~$p_k$) 
оцениваются как функции времени. С~этой целью при каждом положении 
скользящего окна используется ЕМ (expectation-maximization) 
ал\-го\-ритм или ка\-кие-ли\-бо 
из его модификаций~\cite{Korolev2011,Gorshenin2015a,Gorshenin2017}.

Для всеобъемлющего изучения закономерностей эволюции процесса~$X(t)$, 
задаваемого СДУ~\eqref{LangevinEq}, необходимо знать, как ведут 
себя случайные коэффициенты~$a(t)$ и~$b(t)$ во времени. 
Для этого, в~свою очередь, необходимо восстановить эволюцию 
во времени их распределений. В~рамках аппроксимации~\eqref{DiffDiscrApprox}
 по\-след\-няя задача сводится к~последовательной идентификации компонент 
 смеси~\eqref{DiffDiscrApprox}, т.\,е.\ к~определению соответствия 
 оценок параметров $p_k$, $a_k$ и~$b_k$, полученных на разных (например, 
 соседних) положениях сколь\-зя\-ще\-го окна. Другими словами, чтобы 
 определить эволюцию распределения случайных функций~$a(t)$ и~$b(t)$, 
 надо восстановить компоненты смеси~\eqref{DiffDiscrApprox} как 
 функции времени. 
 Таким образом, решается задача при\-бли\-жен\-но\-го 
 восстановления двумерного распределения 
 $$F_t(x,y)\hm\equiv 
 \mathbb{P}\left(a(t)<x,b(t)<y\right)\,.$$ 
 С~этой целью при каждом 
 $i\hm=\overline{1,n}$ ищется дискретная аппроксимация распределения 
 $F_{t_i}(x,y)$, а~затем осуществляется интерполяция этой функции для 
 остальных~$t$. Для интерполяции необходимо установить соответствие 
 между множествами возможных значений $\{a_k,b_k: k\hm=\overline{1,K}\}$ 
 случайных коэффициентов $a(t_i)$ и~$b(t_i)$ при разных~$i$, т.\,е.\
  восстановить эволюцию компонент аппроксимиру\-ющей 
  смеси~\eqref{DiffDiscrApprox} во времени.

В процессе шагов СРС-метода выделяются компоненты 
смеси~\eqref{DiffDiscrApprox}, которые эволюционируют 
при сдвигах скользящего окна. При этом достаточно слож\-но 
судить о~том, какая из оценок па\-ра\-мет\-ров соответствует тому 
или иному значению на предыдущем шаге. Обычно по\-доб\-ная взаимосвязь 
определяется визуально и,~очевидно, является достаточно субъективной.
 В~статье предложен подход к~автоматизации решения данной задачи на 
 основе комбинации жад\-но\-го алгоритма и~методов клас\-те\-ри\-за\-ции $k$- 
 или $c$-сред\-них~\cite{Steinhaus1956,MacQueen1967,David2007,Dunn1973}. 
 При этом на первом этапе определяется чис\-ло клас\-те\-ров для методов 
 машинного обучения, которые используются непосредственно для вы\-яв\-ле\-ния 
 связанных компонент. Данная процедура будет использована для получения 
 статистических оценок коэффициентов урав\-не\-ния~\eqref{LangevinEq} 
 на примере анализа опи\-сы\-ва\-емых им процессов переноса теп\-ла 
 между океаном и~атмосферой~\cite{Belyaev2018} в~ряде географических точек.

\section{Метод определения связности локальных компонент}
%\label{SecConnectivity}

Обозначим через $I^{(n)}$ набор индексов (номеров) компонент 
для шага с~номером~$n$ СРС-ме\-то\-да, т.\,е.\ $I^{(n)}
\hm=\{1,2,\ldots,k^{(n)}\}$, а через 
$J^{(n+1)}\hm=\{1,2,\ldots,k^{(n+1)}\}$~--- 
аналогичный набор для шага $(n+1)$. Через~$I_0$ и~$J_0$ 
обозначим множество индексов из первого и~второго наборов 
соответственно, для которых удалось найти ближайшую компоненту. 
Первоначально полагаем, что $I_0\hm=\emptyset$ и~$J_0\hm=\emptyset$. 
Для каждого фиксированного $J\hm\in J^{(n+1)}\setminus J_0$ 
находим наиболее близ\-кий номер~$I$ в~смыс\-ле решения сле\-ду\-ющей 
оптимизационной задачи:
\begin{multline}
\label{I}
I=\argmin\limits_{i\in I^{(n)} 
\setminus I_0}\left(\left|a_i^{(n)}-a_J^{(n+1)}\right|^p+{}\right.\\
\left.{}+\left|\sigma_i^{(n)}-\sigma_J^{(n+1)}\right|^p \right)^{1/p}\!.
\end{multline}

\renewcommand{\figurename}{\protect\bf Алгоритм}

\setcounter{figure}{0}
\begin{figure*}[b]
{\small \begin{center}
\textbf{Алгоритм 1} Динамическое определение числа локальных компонент\\
{\ }\\[-6pt]
\begin{tabular}{l}
\hline

\textbf{function} {\normalsize{N}}UM{\normalsize{G}}REEDY (Params, $ I^{(n)}$, $J^{(n+1)}$)\\

\hspace*{8mm}$I_0\gets \emptyset$, $J_0\gets \emptyset$, Comps$\gets \emptyset$; // 
\textit{Инициализация}\\
\hspace{8mm}\textbf{repeat} // \textit{Продолжающиеся или новые компоненты}
\\
 \hspace*{15mm}// \textit{Оптимизация выражения~\eqref{I} с~учетом условия~\eqref{Dist}}
 \\
\hspace*{15mm}[I, J]$\gets${\normalsize F}IND{\normalsize I}(Params, $J^{(n+1)}\setminus 
J_0$, $I^{(n)}\setminus I_0$);
\\
\hspace*{15mm}\textbf{if} I $\neq \emptyset$\ \textbf{then}\ // \textit{Найдена 
предшествующая $J$ компонента}
 \\   
\hspace*{20mm}$I_0\gets I_0\cup I$, $J_0\gets J_0\cup J$;
\\
\hspace*{15mm}\textbf{else} // \textit{Добавление новой компоненты}
\\
\hspace*{20mm}$J_0\gets J_0\cup J$;
\\
\hspace*{20mm}Comps $\gets$ {\normalsize A}DD{\normalsize N}EW{\normalsize C}OMP({Params, J});
\\
\hspace*{8mm}\textbf{until} ($J^{(n+1)}\setminus J_0 \neq \emptyset$)
\\
\hspace*{8mm}\textbf{return}\ Comps;\\
\hline
\end{tabular}
\end{center}}
\vspace*{-6pt}
\end{figure*}


В этом случае минимизируемое в~правой час\-ти выражение 
представляет собой $\ell^p$-нор\-му ($p\hm=1,2,\ldots$) соответствующего 
вектора разностей координат в~пространстве точек $(a,\sigma)$.

Полагая после этого $I_0\hm=I_0\cup I$ и~$J_0\hm=J_0\cup J$, необходимо 
повторить процедуру заново. Возможны следующие случаи.
\begin{enumerate}
\item $\left|I^{(n)}\right|=\left|J^{(n+1)}\right|$, т.\,е.\ 
$k^{(n)}\hm=k^{(n+1)}$. В~этом случае оба набора будут исчерпаны одновременно.
\item $\left|I^{(n)}\right|<\left|J^{(n+1)}\right|$, т.\,е.\
 $k^{(n)}<k^{(n+1)}$. В~этом случае процедура останавливается, 
 когда исчерпан набор $I^{(n)} \setminus I_0$. Оcтавшиеся 
 в~$J^{(n+1)}\setminus J_0$ элементы формируют новые компоненты, 
 которые появились только на $(n+1)$-м шаге.
\end{enumerate}

Случай $\left|I^{(n)}\right|>\left|J^{(n+1)}\right|$, т.\,е.\
 $k^{(n)}\hm>k^{(n+1)}$, не рассматривается, поскольку основная цель 
 данной процедуры~--- определение чис\-ла компонент за весь период 
 эволюции процесса, поэтому уменьшаться оно не может, даже если 
 на ка\-ком-то шаге произошло сокращение локального значения для чис\-ла 
 компонент. Отметим, что указанная процедура, очевидно, является жад\-ной.
  При этом, поскольку ее конечная цель со\-сто\-ит в~определении чис\-ла 
  клас\-те\-ров для сле\-ду\-юще\-го шага, данная особенность не пред\-став\-ля\-ет\-ся 
  критической.

Для точного определения чис\-ла компонент необходимо задавать 
некоторый допустимый порог бли\-зости $\varepsilon({\bf a}, 
{\boldsymbol \sigma})$:
\begin{multline}
\label{Dist}
\left(\left|a_I^{(n)}-a_J^{(n+1)}\right|^p+
\left|\sigma_I^{(n)}-\sigma_J^{(n+1)}\right|^p +{}\right.\\
\left.{}+\left|p_I^{(n)}-
p_J^{(n+1)}\right|^p\right)^{1/p} < \varepsilon({\bf a}, 
{\boldsymbol \sigma}).
\end{multline}

Такая проверка нужна для того, чтобы корректно определять 
ситуацию, при которой компоненты с~номерами~$I$ и~$J$ на $n$-м и~$(n+1)$-м 
шагах считались одинаковыми, и~не было не\-об\-хо\-ди\-мости создавать новую 
в~рамках жад\-но\-го алгоритма. Реализация метода определения чис\-ла 
компонент приведена в~алгоритме~1.

Предполагается, что данный алгоритм применяется для 
компонент с~положительными весами (нулевые значения соответствуют 
случаю уменьшения их чис\-ла на ка\-ком-ли\-бо шаге). Кроме того, для 
всех допустимых значений $i\hm\neq j$ и~$n$ должны существовать такие 
$\delta_a\hm>0$ и~$\delta_\sigma\hm>0$, что выполнено хотя бы одно из 
условий: 
$$
\left|a_i^{(n)}\hm-a_j^{(n)}\right|\hm>\delta_a\,; \quad
\left|\sigma_i^{(n)}\hm-\sigma_j^{(n)}\right|\hm>\delta_\sigma\,.
$$
Если же они оба нарушены, то необходимо объединять эти компоненты в~одну 
с~соответствующей корректировкой (суммированием) весов, т.\,е.\ 
предполагается, что все компоненты различны,~--- 
это гарантирует кор\-рект\-ность применения жад\-но\-го алгоритма.

Алгоритм~1 используется в~качестве важ\-ной 
со\-став\-ной час\-ти  метода формирования матрицы связ\-ности. 
Она пред\-став\-ля\-ет собой вспомогательную структуру, в~которой 
на каждом шаге скользящего окна сохраняется актуальное 
со\-сто\-яние всех выделенных к~текущему моменту компонент. 
Сначала ко всему ряду применяется метод EM-ти\-па для 
определения числа компонент на каждом шаге\linebreak (как было отмечено выше, 
оно не может убывать). Затем в~двухмерном пространстве 
$(\bf a, \boldsymbol \sigma)$ используется один из методов 
кластеризации с~полученным жад\-ным алгоритмом чис\-лом локальных 
ком\-по\-нент-клас\-те\-ров. Веса не учитываются, так как вклад компоненты 
в~смесь может изменяться, при этом па\-ра\-мет\-ры~--- 
математическое ожидание и~дис\-пер\-сия~--- варьи\-ру\-ют\-ся 
не слишком сильно, и~тогда считается, что это та же самая компонента. 
Со\-от\-вет\-ст\-ву\-ющая процедура 
пред\-став\-ле\-на в~алго\-ритме~2.

\begin{figure*}
{\small \begin{center}
\textbf{Алгоритм 2} Определение компонент связности в~СРС-методе\\
{\ }\\[-6pt]
\begin{tabular}{l}
\hline
\textbf{function} {\normalsize MSMC}OMPONENTS({Data, options})\\
\hspace*{5mm}Params $\gets${\normalsize EM}S(Data, options.EM) // \textit{СРС-метод}\\
\hspace*{5mm}// \textit{Инициализация числом ненулевых компонент на первом шаге}\\
\hspace*{5mm}Comps$^{(1)}$$\gets$Params.k$^{(1)}$;\\
\hspace*{5mm}\textbf{for} (n = 1:LENGTH(Params)-1) \textbf{do}\\
\hspace*{10mm}Comps$^{(n+1)}$ $\gets$ {\normalsize N}UM{\normalsize G}REEDY({Params, $ I^{(n)}$, $J^{(n+1)}$});\\
\hspace*{5mm}// \textit{Метки для каждого набора параметров, кластеризация}\\
\hspace*{5mm}Labels $\gets$ {\normalsize C}LUST(Params, {MAX}(Comps), options.ClustAlg);\\
\hspace*{5mm}// \textit{Матрица связности для компонент СРС-метода}\\
\hspace*{5mm}HistMatrConnect$\gets$ {\normalsize С}ONNECTIVITY({Params, Labels});  \\
\hspace*{5mm}{\normalsize P}LOT{\normalsize C}OMP(HistMatrConnect); // \textit{Визуализация результатов}\\
\hspace*{5mm}\textbf{return} HistMatrConnect;\\
\hline
\end{tabular}
\end{center}}
\vspace*{-6pt}
\end{figure*}

Очевидно, что данная процедура может быть использована и~в~случае 
смесей многомерных распределений. При этом в~формулу оптимизации~\eqref{I}
 долж\-ны быть добавлены все па\-ра\-мет\-ры со\-от\-вет\-ст\-ву\-юще\-го распределения 
 (с~соответствующей модификацией условия~\eqref{Dist}), 
 а~затем может быть про-\linebreak
 \vspace*{-12pt}
 
 \pagebreak
 
 \noindent
 ведена клас\-те\-ри\-за\-ция в~пространстве 
 переменных новой раз\-мер\-ности.

\section{Оценивание коэффициентов уравнения Ланжевена 
на~примере турбулентных потоков тепла между океаном и~атмосферой}

Для статистической оценки коэффициентов в~уравнении Ланжевена
воспользуемся алгоритмом~2, а~так\-же СРС-оцен\-ка\-ми, 
полученными при аппроксимации распределений приращений потоков теп\-ла 
между океаном и~атмосферой в~нескольких географических точках. Отметим, 
что СРС-метод %скользящего разделения смесей 
ранее использовался для 
статистического моделирования закономерностей в~подобного 
рода данных~\cite{Gorshenin2015b,Gorshenin2016}.

\setcounter{figure}{0}
\renewcommand{\figurename}{\protect\bf Рис.}
\begin{figure*} %fig1
\vspace*{9pt}

\vspace*{1pt}
 \begin{center}
 \mbox{%
 \epsfxsize=156mm 
 \epsfbox{gor-1.eps}
 }
 \end{center}
   \vspace*{-9pt}
\Caption{\label{FigComps_gulfstqe_dyn}
Оценки распределения сдвига (Гольфстрим): 
(\textit{a})~явные потоки;
(\textit{б})~скрытые потоки;
1--5~--- структурные компоненты}
\vspace*{9pt}
\end{figure*}


На рис.~1--4 %\ref{FigComps_gulfstqe_dyn}--\ref{FigComps_troptqh_diff} 
верхние графики демонстрируют статистическую структуру процесса 
теп\-ло\-об\-ме\-на между океаном и~атмосферой в~указанных географических 
точках для двух классов потоков~\cite{Perry1977}~---  скрытых 
%(далее обозначаются как \verb"qe") 
и~явных,~--- %(\verb"qh"),~--- 
яв\-ля\-ющих\-ся со\-став\-ля\-ющи\-ми теп\-ло\-во\-го баланса.

%\begin{figure*} %fig2
% \epsfbox{gor-2.eps}
%\end{figure*}

\begin{figure*} %fig3
\vspace*{1pt}
 \begin{center}
 \mbox{%
 \epsfxsize=156mm 
 \epsfbox{gor-3.eps}
 }
 \end{center}
   \vspace*{-9pt}
\Caption{\label{FigComps_gulfstqe_diff}
Оценки распределения коэффициента диффузии (Гольфстрим):
(\textit{a})~явные потоки;
(\textit{б})~скрытые потоки;
1--5~--- структурные компоненты}
\end{figure*}

%\begin{figure*} %fig4
% \epsfbox{gor-4.eps}
%\end{figure*}

\begin{figure*} %fig5
\vspace*{1pt}
 \begin{center}
 \mbox{%
 \epsfxsize=156mm 
 \epsfbox{gor-5.eps}
 }
 \end{center}
   \vspace*{-9pt}
\Caption{\label{FigComps_tropqe_dyn}
Оценки распределения сдвига (тропики):
(\textit{a})~явные потоки;
(\textit{б})~скрытые потоки;
1--5~--- структурные компоненты}
\end{figure*}

%\begin{figure*} %fig6
% \epsfbox{gor-6.eps}
%\end{figure*}

\begin{figure*} %fig7
\vspace*{1pt}
 \begin{center}
 \mbox{%
 \epsfxsize=156mm 
 \epsfbox{gor-7.eps}
 }
 \end{center}
   \vspace*{-9pt}
\Caption{\label{FigComps_troptqe_diff}
Оценки распределения коэффициента диффузии (тропики):
(\textit{a})~явные потоки;
(\textit{б})~скрытые потоки;
1--5~--- структурные компоненты}
\end{figure*}

%\begin{figure*} %fig8
% \epsfbox{gor-8.eps}
%\end{figure*}


Благодаря структурам, возвращаемым функцией \verb"MSMComponents",
 можно точно отследить, когда те или иные из компонент существовали, 
 прерывались и~возобновлялись, и~проанализировать их взаимосвязь с~реальными
  физическими процессами. На нижних графиках, содержащих СРС-оцен\-ки для 
  динамической и~диффузионной компонент, продемонстрирована эволюция весов, 
  т.\,е.\ вклад соответствующей структурной со\-став\-ля\-ющей в~общее 
  развитие процесса во времени.

Для аппроксимации были использованы четырехкомпонентные 
нормальные смеси, однако при выбранных настройках жад\-но\-го 
алгоритма~1 для всех рядов (за исключением
 явных потоков в~тропиках) были получены пять локальных
  компонент связ\-ности.


Видно, что общее чис\-ло компонент не слишком сильно изменяется, 
поэтому результаты автоматического определения с~по\-мощью 
жад\-но\-го алгоритма~1 варьируются от ряда к~ряду 
не очень существенно. Однако для лучшего учета локальных процессов 
полученное чис\-ло компонент (4--5) может быть расширено за счет 
повышения чув\-ст\-ви\-тель\-ности процедуры путем выбора меньшего 
порогового значения в~формуле~\eqref{Dist}.

\vspace*{-9pt}

\section{Заключение}
\vspace*{-3pt}

В работе описан метод статистического оценивания 
распределений случайных па\-ра\-мет\-ров СДУ
%стохастических дифференциальных уравнений 
типа Ланжевена с~по\-мощью СРС-техники.
% скользящего разделения смесей. 
Предложены дискретные аппроксимации для оценок указанных распределений. 
С~целью изучения из\-мен\-чи\-вости распределений коэффициентов СДУ 
во времени пред\-ло\-жен алгоритм последовательной идентификации 
(определения локальной связ\-ности) компонент получаемых смесей. 
В~его основу положена комбинация жад\-но\-го алгоритма для поиска чис\-ла 
компонент и~одного из методов клас\-те\-ри\-за\-ции ($k$- или $c$-сред\-них). 
Функциональные\linebreak па\-ра\-мет\-ры (компоненты распределения~\eqref{DiffDiscrApprox} 
как функции времени), полученные в~результате опи\-сы\-ва\-емых статистических 
процедур, могут быть\linebreak использованы при обучении интеллектуальных 
алгоритмов прогнозирования процессов, удовле\-тво\-ря\-ющих уравнениям 
типа~\eqref{LangevinEq}. Применение метода иллюстрируется конкретными 
примерами анализа процесса теп\-ло\-об\-ме\-на между атмосферой и~океаном.

\vspace*{-9pt}

{\small\frenchspacing
 {%\baselineskip=10.8pt
 \addcontentsline{toc}{section}{References}
\vspace*{-3pt}


 \begin{thebibliography}{99}
\bibitem{Belyaev2018} 
\Au{Belyaev~K., Kuleshov~A., Tuchkova~N., Tanajura~C.\,A.\,S.} 
An optimal data assimilation method and its application 
to the numerical simulation of the ocean dynamics~// 
Math. Comp. Model. Dyn., 2018. Vol.~1. Iss.~24. P.~12--25.

\bibitem{Leland1985} 
\Au{Leland~H.\,E.} Option pricing and replication with transactions costs~// 
J.~Financ., 1985. Vol. 40. P. 1283--1301.

\bibitem{BarlesSoner1998} 
\Au{Barles~G., Soner~H.\,M.} Option pricing with transaction 
costs and a nonlinear Black--Scholes equation~// 
Financ. Stoch., 1998. Vol.~2. P.~369--397.

\bibitem{Heston1993} 
\Au{Heston~S.\,L.} A~closed-form solution for options with stochastic
 volatility, with application to bond and currency options~// 
 Rev. Financ. Stud., 1993. Vol.~6. P.~327--343.

\bibitem{CoxIngersollRoss1985} 
\Au{Cox~J.\,C., Ingersoll~J.\,E., Ross~S.\,A.} 
A~theory of the term structure of interest rates~// 
Econometrica, 1985. Vol.~53. P.~385--407.

\bibitem{HullWhite1987} 
\Au{Hull~J., White~A.} The pricing of options on assets 
with stochastic volatilities~// J.~Financ., 1987. Vol.~42. P.~281--308.

\bibitem{Dupire1994} %7
\Au{Dupire~B.} 
Pricing with a~smile~// Risk, 1994. Vol.~7. P.~18--20.

\bibitem{DermanKani1994} 
\Au{Derman~E., Kani~J.} Riding on a smile~// Risk, 1994. Vol.~7. P.~32--39.

\bibitem{Shiryaev2016} %9
\Au{Ширяев~А.\,Н.} Основы стохастической финансовой математики.
 Т.~1. Факты. Модели.~--- М.: МЦНМО, 2016. 440~c.

\bibitem{Korolev2011} %10
\Au{Королев~В.\,Ю.} Ве\-ро\-ят\-ност\-но-ста\-ти\-сти\-че\-ские 
методы декомпозиции волатильности хаотических процессов.~--- 
М.: Изд-во Московского ун-та, 2011. 512~c.

\bibitem{Gorshenin2017} %11
\Au{Горшение А.\,К., Королев В.\,Ю., Турсунбаев А.\,М.}
Медианные модификации EM-алгоритма для разделения смесей вероятностных 
распределений и~их применение к~декомпозиции волатильности финансовых 
индексов~//
Статистические методы оценивания и~про\-вер\-ки гипотез, 2008. 
Т.~21. С.~169--195.

\bibitem{Gorshenin2015a} %12
\Au{Gorshenin~A.\,K.} 
On implementation of EM-type algorithms in the stochastic models for a matrix 
computing on GPU~// AIP Conf. Proc., 2015. Vol.~1648. Art. No.\,250008. 4~p.

\bibitem{Steinhaus1956} 
\Au{Steinhaus~H.} Sur la division des corps materiels en parties~// 
B. Acad. Pol. Sci.,1956. Vol.~4. Iss.~12. P.~801--804.

\bibitem{MacQueen1967} 
\Au{MacQueen~J.} Some methods for 
classification and analysis of multivariate observations~// 
5th Berkeley Symposium on Mathematical Statistics and Probability
Proceedings, 1967. P.~281--297.

\bibitem{Dunn1973} %15
\Au{Dunn~J.\,C.} A~fuzzy relative of the ISODATA process and its use in 
detecting compact well-separated clusters~// 
J.~Cybernetics, 1973. Vol.~3. Iss.~3. P.~32--57.

\bibitem{David2007} %16
\Au{David~A., Vassilvitskii~S.} K-means++: 
The advantages of careful seeding~// 18th Annual ACM-SIAM Symposium 
on Discrete Algorithms Proceedings.~--- ACM, 2007. P.~1027--1035.

\bibitem{Gorshenin2015b} 
\Au{Korolev~V.\,Yu., Gorshenin~A.\,K., Gulev~S.\,K., Belyaev~K.\,P.} 
Statistical modeling of air--sea turbulent heat fluxes 
by finite mixtures of Gaussian distributions~// Comm. Com. 
Inf. Sc., 2015. Vol.~564. P.~152--162.

\bibitem{Gorshenin2016} 
\Au{Горшенин~А.\,К.} 
Концепция он\-лайн-комп\-лек\-са для стохастического моделирования 
реальных процессов~// Информатика и~её применения, 2016. 
Т.~10. Вып.~1. C.~72--81.

\bibitem{Perry1977} 
\Au{Perry~A.\,H., Walker~J.\,M.} Ocean--atmosphere system.~--- 
Upper Saddle River, NJ, USA: Prentice Hall Press, 1977. 180~p.
\end{thebibliography}

 }
 }

\end{multicols}

\vspace*{-6pt}

\hfill{\small\textit{Поступила в~редакцию 15.07.20}}

\vspace*{8pt}

%\pagebreak

%\newpage

%\vspace*{-28pt}

\hrule

\vspace*{2pt}

\hrule

%\vspace*{-2pt}

\def\tit{STATISTICAL ESTIMATION OF~DISTRIBUTIONS OF RANDOM COEFFICIENTS 
IN~THE~LANGEVIN 
STOCHASTIC DIFFERENTIAL EQUATION}


\def\titkol{Statistical estimation of~distributions of random coefficients 
in~the~Langevin 
stochastic differential equation}

\def\aut{A.\,K.~Gorshenin$^{1}$, V.\,Yu.~Korolev$^{1,2}$, and~A.\,A.~Shcherbinina$^{2}$}

\def\autkol{A.\,K.~Gorshenin, V.\,Yu.~Korolev, and A.\,A.~Shcherbinina}

\titel{\tit}{\aut}{\autkol}{\titkol}

\vspace*{-9pt}


\noindent
$^1$Federal Research Center ``Computer Science and Control'' 
of the Russian Academy of Sciences, 44-2~Vavilov\linebreak
$\hphantom{^1}$Str., 
Moscow 119333, Russian Federation

\noindent
$^2$Faculty of Computational Mathematics and Cybernetics, Lomonosov Moscow
State University, GSP-1, Leninskie\linebreak
$\hphantom{^1}$Gory, Moscow 119991, Russian Federation


\def\leftfootline{\small{\textbf{\thepage}
\hfill INFORMATIKA I EE PRIMENENIYA~--- INFORMATICS AND
APPLICATIONS\ \ \ 2020\ \ \ volume~14\ \ \ issue\ 3}
}%
 \def\rightfootline{\small{INFORMATIKA I EE PRIMENENIYA~---
INFORMATICS AND APPLICATIONS\ \ \ 2020\ \ \ volume~14\ \ \ issue\ 3
\hfill \textbf{\thepage}}}

\vspace*{3pt} 



\Abste{A method is described for statistical estimation
 of the distributions of random coefficients of the 
 Langevin stochastic differential equation (SDE) 
 by the technique of moving separation of mixtures. 
 Discrete approximations are proposed for these distributions. 
 For the purpose of study of variability of the distributions 
 of the SDE coefficients in time, an algorithm is proposed for 
 sequential identification (determination of local connectivity) 
 of the components of the resulting mixture distributions. 
 This algorithm is based on combining a greedy algorithm for 
 the determination of the number of components with 
 a~lustering method ($k$- or $c$-means). 
 The application of the proposed method is illustrated by particular 
 examples of the analysis of processes of 
heat transfer between atmosphere and ocean.}


\KWE{mixture distribution; local connectivity; greedy algorithm; clustering}





\DOI{10.14357/19922264200301} 

%\vspace*{-20pt}

\Ack
\noindent
The research was partially supported by the Russian Foundation 
for Basic Research
(project~19-07-00914) and the RF Presidential scholarship program 
(project No.\,538.2018.5). The research was conducted in accordance 
with the Program of Moscow Center for Fundamental and Applied Mathematics.

%\vspace*{6pt}

 \begin{multicols}{2}

\renewcommand{\bibname}{\protect\rmfamily References}
%\renewcommand{\bibname}{\large\protect\rm References}

{\small\frenchspacing
 {%\baselineskip=10.8pt
 \addcontentsline{toc}{section}{References}

 \begin{thebibliography}{99}
\bibitem{1-gor-1}
\Aue{Belyaev,~K., A.~Kuleshov, N.~Tuchkova, and C.\,A.\,S.~Tanajura.}
 2018. An optimal data assimilation method and its application to the 
 numerical simulation of the ocean dynamics. \textit{Math.
 Comp. Model. Dyn.} 1(24):12--25.

\bibitem{2-gor-1}
\Aue{Leland,~H.\,E.} 
1985. Option pricing and replication with transactions costs. 
\textit{J.~Financ.} 40:1283--1301.

\bibitem{3-gor-1}
\Aue{Barles,~G., and H.\,M.~Soner.} 
1998. Option pricing with transaction costs and a nonlinear 
Black--Scholes equation. \textit{Financ. Stoch.} 2:369--397.

\bibitem{4-gor-1}
\Aue{Heston,~S.\,L.} 1993. A~closed-form solution for options
 with stochastic volatility, with application to bond and 
 currency options. \textit{Rev. Financ. Stud.} 6:327--343.

\bibitem{5-gor-1}
\Aue{Cox,~J.\,C., J.\,E.~Ingersoll, and  S.\,A.~Ross.}
 1985. A~theory of the term structure of interest rates. 
 \textit{Econometrica} 53:385--407.

\bibitem{6-gor-1}
\Aue{Hull,~J., and A.~White.} 1987. The pricing of options on assets 
with stochastic volatilities. \textit{J.~Financ.} 42:281--308.

\bibitem{9-gor-1} %7
\Aue{Dupire,~B.} 1994. Pricing with a smile. \textit{Risk} 7:18--20.

\bibitem{8-gor-1}
\Aue{Derman,~E., and J.~Kani.} 1994. Riding on a smile. \textit{Risk} 7:32--39.

\bibitem{7-gor-1} %9
\Aue{Shiryaev,~A.\,N.} 2016. 
\textit{Osnovy stokhasticheskoy finansovoy matematiki. T.~1. Fakty. Modeli} 
[Foundations of stochastic financial mathematics. Vol.~1. Facts. Models]. 
Moscow: MCCME. 440~p.

\bibitem{10-gor-1} %10
\Aue{Korolev,~V.\,Yu.} 2011. 
\textit{Veroyatnostno-statisticheskie metody dekompozitsii
volatil'nosti khaoticheskikh protsessov} [Probabilistic and statistical 
methods of decomposition of volatility of chaotic processes]. Moscow: 
Izd-vo Moskovskogo un-ta. 512~p.

\bibitem{12-gor-1} %11 
\Aue{Gorshenin,~A.\,K., V.\,Yu.~Korolev, and A.\,M.~Tursunbaev.}
 2017. Median modifications of the EM-algorithm for separation of
 mixtures of probability distributions and their applications 
 to the decomposition of volatility of financial indexes.
 \textit{J.~Math. Sci.} 227(2):176--195.

\bibitem{11-gor-1} %12
\Aue{Gorshenin,~A.\,K.} 2015. On implementation of EM-type algorithms 
in the stochastic models for a matrix computing on GPU. 
\textit{AIP Conf. Proc.} 1648:250008. 4~p.

\bibitem{13-gor-1}
\Aue{Steinhaus,~H.} 1956. Sur la division des corps materiels 
en parties. \textit{B. Acad. Pol. Sci.} 4(12):801--804.

\bibitem{14-gor-1}
\Aue{MacQueen,~J.} 1967. Some methods for classification and analysis 
of multivariate observations. \textit{5th Berkeley Symposium on Mathematical 
Statistics and Probability Proceedings}. 281--297.

\bibitem{16-gor-1} %15
\Aue{Dunn,~J.\,C.} 1973. A~fuzzy relative of the ISODATA process and 
its use in detecting compact well-separated clusters. 
\textit{J.~Cybernetics} 3(3):32--57.

\bibitem{15-gor-1} %16
\Aue{David,~A., and S.~Vassilvitskii.}
 2007. K-means++: The advantages of careful seeding. 
 \textit{18th Annual ACM-SIAM Symposium on Discrete Algorithms
 Proceedings}. ACM. 1027--1035.

\bibitem{17-gor-1}
\Aue{Korolev,~V.\,Yu., A.\,K.~Gorshenin, S.\,K.~Gulev, and K.\,P.~Belyaev.}
 2015. Statistical modeling of air--sea turbulent heat fluxes by
  finite mixtures of Gaussian distributions. \textit{Comm. Com. Inf.
Sc.} 564:152--162.

\bibitem{18-gor-1}
\Aue{Gorshenin,~A.\,K.} 2016. Kontseptsiya onlayn-kompleksa 
dlya stokhasticheskogo modelirovaniya real'nykh pro\-tses\-sov 
[Concept of online service for stochastic modeling of real processes]. 
\textit{Informatika i~ee Primeneniya~--- Inform. Appl.} 10(1):72--81.

\bibitem{19-gor-1}
\Aue{Perry,~A.\,H., and J.\,M.~Walker.} 1977. 
\textit{Ocean--atmosphere system}. Upper Saddle River, NJ: 
Prentice Hall Press. 180~p.
\end{thebibliography}

 }
 }

\end{multicols}

\vspace*{-6pt}

\hfill{\small\textit{Received July 15, 2020}}

%\pagebreak

%\vspace*{-24pt}

\Contr

\noindent
\textbf{Gorshenin Andrey K.} (b.\ 1986)~--- 
Candidate of Science (PhD) in physics and
mathematics, associate professor, leading scientist, Federal 
Research Center ``Computer Science and Control'' of the Russian 
Academy of Sciences, 44-2~Vavilov Str., Moscow 119333, Russian Federation; 
leading scientist, Faculty of Computational Mathematics and Cybernetics 
and Moscow Center for Fundamental and Applied Mathematics, Lomonosov 
Moscow State University, GSP-1, Leninskie Gory, Moscow 119991, 
Russian Federation; \mbox{agorshenin@frccsc.ru}

\vspace*{3pt}

\noindent
\textbf{Korolev Victor Yu.} (b.\ 1954)~--- 
Doctor of Science in physics and
mathematics, professor, head of department, Faculty of Computational 
Mathematics and Cybernetics, and principal scientist, 
Moscow Center for Fundamental and Applied Mathematics, 
Lomonosov Moscow State University, GSP-1, Leninskie Gory, Moscow 
119991, Russian Federation; leading scientist, Federal Research Center 
``Computer Science and Control'' of the Russian Academy of Sciences, 
44-2~Vavilov Str., Moscow 119333, Russian Federation; \mbox{vkorolev@cs.msu.ru}

\vspace*{3pt}

\noindent
\textbf{Shcherbinina Anastasia A.} (b.\ 1997)~--- Master student,  Faculty
of Computational Mathematics and Cybernetics, 
Lomonosov Moscow State University, GSP-1, Leninskie Gory, 
Moscow 119991, Russian Federation; \mbox{shcherbinina.aa.97@gmail.com}


\label{end\stat}

\renewcommand{\bibname}{\protect\rm Литература}    %1
\def\stat{naumov}

\def\tit{О МАРКОВСКИХ И~РАЦИОНАЛЬНЫХ ПОТОКАХ 
СЛУЧАЙНЫХ СОБЫТИЙ.~II$^*$} % Часть~2$^*$}

\def\titkol{О марковских и рациональных потоках случайных 
событий. II} %Часть 2}

\def\aut{В.\,А.~Наумов$^1$, К.\,Е.~Самуйлов$^2$}

\def\autkol{В.\,А.~Наумов, К.\,Е.~Самуйлов}

\titel{\tit}{\aut}{\autkol}{\titkol}

\index{Наумов В.\,А.}
\index{Самуйлов К.\,Е.}
\index{Naumov V.\,A.}
\index{Samouylov К.\,Е.}


{\renewcommand{\thefootnote}{\fnsymbol{footnote}} \footnotetext[1]
{Исследование выполнено при финансовой поддержке РФФИ в рамках научного проекта №\,19-17-50126.}}


\renewcommand{\thefootnote}{\arabic{footnote}}
\footnotetext[1]{Исследовательский институт инноваций, г.~Хельсинки, Финляндия, 
\mbox{valeriy.naumov@pfu.fi}}
\footnotetext[2]{Российский университет дружбы народов; Институт проблем информатики Федерального 
исследовательского центра <<Информатика и~управ\-ле\-ние>> Российской академии наук, \mbox{samouylov-ke@rudn.ru}}

%\vspace*{6pt}

  \Abst{Статья представляет собой вторую часть обзора, выполненного в рамках проекта 
РФФИ 
  №\,19-17-50126. Цель обзора~--- ознакомление заинтересованных читателей с основами 
теории марковских потоков событий для более подробного изучения и облегчения 
применения этих моделей на практике. В~первой части приведены свойства общих 
марковских потоков событий и показана их связь с марковскими аддитивными процессами и 
процессами марковского восстановления. Во второй части обзора рассмотрены важные для 
приложений частные случаи таких потоков~--- подклассы марковских потоков событий, а~именно:
 простые и групповые потоки однородных и неоднородных событий. Показано, 
как свойства марковских потоков событий связаны с мультипликативностью стационарных 
распределений марковских систем. Обсуждаются  
мат\-рич\-но-экс\-по\-нен\-ци\-аль\-ные распределения и рациональные потоки событий, 
расширяющие возможности марковских потоков для моделирования сложных систем, при 
этом сохраняющие удобство их анализа с помощью вычислительной техники.}
  
  \KW{марковские процессы; марковские аддитивные процессы; потоки без последействия; 
  МС-по\-то\-ки}
  
\DOI{10.14357/19922264200406} 
  
\vspace*{6pt}


\vskip 10pt plus 9pt minus 6pt

\thispagestyle{headings}

\begin{multicols}{2}

\label{st\stat}


\section{Введение}

Настоящий обзор, состоящий из двух частей, включает изложение основ 
теории марковских потоков и снабжен ссылками на большое число работ, 
посвященных марковским и~рациональным потокам событий. Он начался с 
рассмотрения в первой части случайных величин фазового типа, определения 
марковских потоков общего вида и их связи с~марковскими аддитивными 
процессами и процессами марковского восстановления. Во второй части 
обзора  перейдем к важным для приложений подклассам марковских потоков 
однородных и неоднородных событий в разд.~2, а~в~завершение в~разд.~3 
обсудим  
мат\-рич\-но-экс\-по\-нен\-ци\-аль\-ные распределения и~в~разд.~4 
рациональные потоки событий, которые расширяют возможности марковских 
потоков для моделирования сложных систем и~при этом сохраняют удобство 
их анализа. 

Как и в первой части обзора, далее в работе жирные строчные буквы 
обозначают векторы, а~жирные прописные буквы обозначают матрицы. 
Кроме того, используются следующие обозначения: 
$$
\delta(i,j)= \begin{cases}
1, &\mbox{если } i=j\,;\\
0 & \mbox{в~противном\ случае};
\end{cases}
$$
 у~вектора~$\mathbf{e}_i$ 
$i$-я координата равна единице, а остальные равны нулю; $\mathbf{I}\hm= \left[ 
\delta(i,j)\right]$~--- единичная матрица; $\mathbf{u}$~---  
век\-тор-стол\-бец из единиц; $\boldsymbol{\mathcal{N}}^K$~--- множество 
неотрицательных целочисленных векторов длины~$K$, 
$\boldsymbol{\mathcal{N}}^K _0\hm= \boldsymbol{\mathcal{N}}^K \backslash 
\{\mathbf{0}\}$. Для краткости вмес\-то <<наступило $n_1$ событий типа~1, 
$n_2$ событий ти-\linebreak па~2,~\ldots , $n_K$ событий типа~$K$>> будем писать 
<<наступило $\mathbf{n}$ событий>>, где $\mathbf{n}\hm= \left( n_1, n_2, 
\ldots , n_K\right)$.

\section{Важные для~приложений частные случаи марковских потоков 
событий}

\subsection{Простой марковский поток однородных событий}

  Рассмотрим некоторый поток случайных неоднородных событий и 
обозначим через $N_k(t)$ чис\-ло событий типа~$k$, наступивших за время~$t$, 
$\mathbf{N}(t)\hm= (N_1(t), N_2(t), \ldots, N_K(t))$. Поток случайных событий 
называется марковским, если для некоторого случайного процесса~$X(t)$ с 
конечным \mbox{множеством} состояний $\boldsymbol{\mathcal{X}}\hm= \{1,2,\ldots , 
L\}$ процесс $\xi(t)\hm= (X(t), \mathbf{N}(t))$ является марковским процессом, 
однородным во времени и по второй компоненте, т.\,е.\ если для любых~$t, 
h\hm>0$ справедливы равенства
  \begin{multline*}
  {\sf P}\left(X(h+t)=j, \mathbf{N}(h+t)=\mathbf{k}+\mathbf{n}\vert X(h)=i, \right.\\
\left.\mathbf{N}(h)=\mathbf{k}\right)=p_{\mathbf{n}}(i,j,t)\,,\enskip
  \mathbf{k}, \mathbf{n} \in \boldsymbol{\mathcal{N}}^K,\enskip i,j\in 
\boldsymbol{\mathcal{X}}\,.
  \end{multline*}
Матрицы вероятностей переходов $\mathbf{P}_{\mathbf{n}}(t)\hm= 
[p_{\mathbf{n}}(i,j,t)]$ однозначно определяются матрицами интенсивностей 
переходов $\mathbf{A}_{\mathbf{n}}\hm= \left[ a_{\mathbf{n}}(i,j)\right]$, 
$\mathbf{n}\hm\geq \mathbf{0}$, где
\begin{align*}
a_{\mathbf{0}}(i,j) &=\lim\limits_{t\to0} \fr{1}{t}\left( p_{\mathbf{0}}(i,j,t)-\delta(i,j)\right)\,,\enskip
 i,j\in  \boldsymbol{\mathcal{X}}\,;\\
a_{\mathbf{n}}(i,j) &=\lim\limits_{t\to0} \fr{1}{t}\, p_{\mathbf{n}}(i,j,t)\,,\enskip i,j\in 
\boldsymbol{\mathcal{X}}\,,\enskip \mathbf{n}\in \boldsymbol{\mathcal{N}}^K_0,
\end{align*}
при этом фазовый процесс~$X(t)$ является однородным марковским 
процессом с матрицей интенсивностей переходов $\mathbf{A}\hm= 
\sum\nolimits_{\mathbf{n}\in \boldsymbol{\mathcal{N}}^K} 
\mathbf{A}_{\mathbf{n}}$.
  
  В первой части обзора определен процесс марковского восстановления 
$(X_l,\boldsymbol{\sigma}_l, \tau_l)$, где $X_l\hm= X(t_l)$~--- состояния 
фазового процесса~$X(t)$ марковского потока в моменты после наступления\linebreak 
событий потока, $X(t)\hm\in \boldsymbol{\mathcal{X}} \hm= \{1,2,\ldots ,L\}$, 
$0\hm< t_1\hm< t_2
  <\cdots$~--- моменты наступления событий, также называемые 
вызывающими моментами; $\tau_l\hm= t_l\hm- t_{l-1}$~--- длины интервалов 
между \mbox{моментами} наступления событий; $\boldsymbol{\sigma}_l$~--- вектор, 
$\boldsymbol{\sigma}_l\hm= (\sigma_{l,1}, \ldots , \sigma_{l,K})$, 
в~котором~$\sigma_{l,k}$ есть размер группы событий типа~$k$, наступивших 
в~момент~$t_l$, $l\hm=1, 2, \ldots$ Матрицы $\mathbf{G}_{\mathbf{n}}(x)\hm= 
[G_{\mathbf{n}}(i,j,x)]$, описывающие связанный с марковским потоком 
процесс марковского восстановления $(X_l, \boldsymbol{\sigma}_l, \tau_l)$, и 
их преобразования Лап\-ла\-са--Стилть\-еса имеют следующий вид:

\noindent
  \begin{align}
  \mathbf{G}_{\mathbf{n}}(x)&=\int\limits_0^x \exp 
(z\mathbf{A}_0)\mathbf{A}_{\mathbf{n}}\,dz={}\notag\\
&\hspace*{-10mm}{}=\left( \exp 
(x\mathbf{A}_{\mathbf{0}}))-\mathbf{I}\right)\mathbf{A}_0^{-1} \mathbf{A}_{\mathbf{n}}\,,\ \mathbf{n}\in 
\boldsymbol{\mathcal{N}}_0^K\,;
  \label{e1-nau}\\
  \int\limits_0^x e^{-\nu x}d\mathbf{G}_{\mathbf{n}}(x)&= (\nu\mathbf{I}-
\mathbf{A}_{\mathbf{0}})^{-1}\mathbf{A}_{\mathbf{n}}\,,\ \mathbf{n}\in 
\boldsymbol{\mathcal{N}}_0^K\,.
  \label{e2-nau}
  \end{align}
Используя матрицы $\mathbf{G}_{\mathbf{n}}(x)$, можно найти совместное 
распределение числа~$\boldsymbol{\sigma}_l$ наступивших событий и 
длин~$\tau_l$ интервалов между вызывающими моментами 
\begin{multline}
F_{\mathbf{k}_1, \mathbf{k}_2, \ldots , \mathbf{k}_m} \left(x_1, x_2, \ldots , 
x_m\right)={}\\
{}={\sf P}\left(
\boldsymbol{\sigma}_l=\mathbf{k}_l\,, \tau_l<x_l\,, l=1,2,\ldots, m\right)={}\\
{}=\bm{\alpha}\mathbf{G}_{\mathbf{k}_1}(x_1) \mathbf{G}_{\mathbf{k}_2}(x_2)\cdots 
\mathbf{G}_{\mathbf{k}_m}(x_m)\mathbf{u}\,,
\label{e3-nau}
\end{multline}
а также плотность этого распределения

\columnbreak

\noindent
\begin{multline}
f_{\mathbf{k}_1, \mathbf{k}_2, \ldots , \mathbf{k}_m} (x_1, x_2, \ldots , 
x_m)={}\\
{}=\bm{\alpha}\exp \left( x_1\mathbf{A}_{\mathbf{0}}\right) 
\mathbf{A}_{\mathbf{k}_1}\exp \left( x_2\mathbf{A}_{\mathbf{0}}\right) 
\mathbf{A}_{\mathbf{k}_2}\cdots\\
\cdots \exp \left( x_m\mathbf{A}_{\mathbf{0}}\right) 
\mathbf{A}_{\mathbf{k}_m}\mathbf{u}\,,\quad
\mathbf{k}_1, \mathbf{k}_2, \ldots , \mathbf{k}_m\in 
\boldsymbol{\mathcal{N}}_0^K\,,\\
 x_0, x_1, \ldots , x_m>0\,,\enskip m=1,2,\ldots
\label{e4-nau}
\end{multline}

\vspace*{-6pt}

\noindent
где $\bm{\alpha}$~--- начальное распределение фазового про\-цесса.


  
  Простой марковский поток однородных событий~--- это марковский поток 
событий одного типа, причем в каждый вызывающий момент наступает ровно 
одно событие. Он характеризуется двумя мат\-ри\-ца\-ми интенсивностей 
переходов $\mathbf{S}\hm= \mathbf{A}_0$ и~$\mathbf{R}\hm= \mathbf{A}_1$, 
а~остальные матрицы~$\mathbf{A}_k$, $k\hm\geq 2$, для такого потока~--- 
нулевые. Первыми работами, посвященными простым марковским потокам 
однородных событий, стали~[1--5]. Их применение к~решению задач теории 
телетрафика рассматривается  
в~\cite{6-nau, 7-nau}. Поток вызывающих моментов любого марковского 
потока~--- это простой марковский поток, характеризуемый матрицами 
$\mathbf{S}\hm= \mathbf{A}\hm-\mathbf{R}$ и~$\mathbf{R}\hm= 
\sum\nolimits_{\mathbf{n}\in \boldsymbol{\mathcal{N}}_0^K} 
\mathbf{A}_{\mathbf{n}}$. К~простым марковским потокам относятся также 
процессы восстановления фазового типа~\cite{8-nau}. Для таких потоков ранг 
матрицы~$\mathbf{R}$ равен единице и~она имеет вид $\mathbf{R}\hm= 
\mathbf{sq}$, где $\mathbf{s}\hm= -\mathbf{Su}$. Верно и~обратное~\cite{7-nau}. 
В~англоязычной литературе простые марковские потоки называют 
Markovian arrival process и~используют для их обозначения сокращение МАР 
или MArP.
  
  Простой марковский поток однородных событий является 
полумарковским, поскольку последовательность $(X_l, \tau_l)$, $l\hm=1, 
2,\ldots,$~--- процесс марковского восстановления. Из~(1) и~(2) вытекают 
следующие формулы для полумарковской матрицы $\mathbf{G}(x)\hm= \left[ 
G(i,j,x)\right]$ процесса $(X_l,\tau_l)$ марковского восстановления с 
элементами 

\vspace*{3pt}

\noindent
  $$
  G(i,j,x)={\sf P} \left( X_l=j,\ \tau_l<x\vert X_{l-1}=i\right)
  $$
  
  \vspace*{-1pt}
  
  \noindent
 и для ее преобразования Лап\-ла\-са--Стилть\-еса:
 
 \vspace*{2pt}
 
 \noindent
\begin{equation}
\left.
\begin{array}{rl}
\mathbf{G}(x)&=\left( \exp (x\mathbf{S})-\mathbf{I}\right) \mathbf{S}^{-
1}\mathbf{R}\,;\\
\displaystyle\int\limits_0^x e^{-\nu x}d\mathbf{G}(x)&=(\nu\mathbf{I}-\mathbf{S})^{-1}\mathbf{R}\,.
\end{array}
\right\}
\label{e5-nau}
\end{equation}

\vspace*{-2pt}
  
  Из~(\ref{e4-nau}) вытекает следующее выражение для плотности функции 
распределения длин интервалов~$\tau_l$ между моментами наступления 
событий простого марковского потока однородных событий:

\vspace*{-8pt}

\noindent
  \begin{multline}
  f\left( x_1, x_2, \ldots, x_m\right)={}\\
  {}=\bm{\alpha}\exp \left(x_1\mathbf{S}\right) 
\mathbf{R}\exp \left( x_2\mathbf{S}\right)\mathbf{R}\cdots \exp \left( 
x_m\mathbf{S}\right) \mathbf{Ru}\,,\\
  x_0, x_1,\ldots , x_m>0\,,\enskip m=1,2,\ldots
  \label{e6-nau}
  \end{multline}
  
  \vspace*{-2pt}
  
  Поскольку простой марковский поток является полумарковским, при 
анализе систем массового обслуживания с такими поступающими потоками 
можно использовать результаты, полученные для систем с полумарковским 
входящим потоком,  
например~[9--12].
   
  В первом разделе обзора указано, что стационарные распределения 
$\mathbf{q}\hm=[q(i)]$ и $\mathbf{q}_{\mathbf{n}}\hm= [q_{\mathbf{n}}(i)]$, 
$\mathbf{n}\hm\in \boldsymbol{\mathcal{N}}_0^K$, вложенных цепей 
Маркова~$X_l$ и~$(X_l, \boldsymbol{\sigma}_l)$ связаны со стационарным 
распределением~$\mathbf{p}$ фазового процесса~$X(t)$ следующими 
равенствами:
\begin{multline*}
  \mathbf{q}=\fr{1}{\lambda}\,\mathbf{p}\boldsymbol{\Lambda}\,,\
  \mathbf{p}=-\lambda \mathbf{q}\mathbf{A}_0^{-1}\,,\ 
  \mathbf{q}=\sum\limits_{\mathbf{n}\in \boldsymbol{\mathcal{N}}_0^K} 
\mathbf{q}_{\mathbf{n}}\,,\\
 \mathbf{q}_{\mathbf{n}}=\fr{1}{\lambda}\,\mathbf{p}
  \mathbf{A}_{\mathbf{n}}\,,\enskip \mathbf{n}\in \boldsymbol{\mathcal{N}}_0^K\,.
  \end{multline*}

  Если вектор из единиц~$\mathbf{u}$ является правым собственным 
вектором каждой из матриц~$\mathbf{A}_{\mathbf{n}}$ и выполняются 
равенства 
  \begin{equation}
  \mathbf{A}_{\mathbf{n}}\mathbf{u}=\lambda_{\mathbf{u}}\mathbf{u}\,,\quad
  \mathbf{n}\in \boldsymbol{\mathcal{N}}_0^K\,,
  \label{e7-nau}
  \end{equation}
то из~(\ref{e3-nau}) следует, что при любом начальном 
распределении~$\mathbf{s}$ марковский поток будет стационарным потоком 
без последействия. Аналогично, если вектор стационарных 
вероятностей~$\mathbf{p}$ является левым собственным вектором 
матриц~$\mathbf{A}_{\mathbf{n}}$ и выполняются равенства 
\begin{equation}
\mathbf{pA}_{\mathbf{n}}=\lambda_{\mathbf{n}}\mathbf{p}\,,\quad
\mathbf{n}\in \boldsymbol{\mathcal{N}}_0^K\,.
\label{e8-nau}
\end{equation}
    
Условия~(\ref{e7-nau}) и~(\ref{e8-nau}), достаточные для того чтобы 
марковский поток был пуассоновским, для простого марковского потока 
приобретают вид $\mathbf{Ru}\hm= \lambda\mathbf{u}$ и~$\mathbf{pR}\hm= 
\lambda\mathbf{p}$ соответственно, где $\lambda\hm= \mathbf{pRu}$~--- 
интенсивность потока. Проверка необходимых и~достаточных условий 
пуассоновости простого марковского потока более сложна и~требует знания 
собственных векторов матрицы~$\mathbf{S}$~\cite{13-nau}.
  
  Считающий процесс $N(t)$ стационарной версии простого марковского 
потока является асимптотически нормальным с~математическим ожиданием 
${\sf M}(t)\hm=\lambda t$ и дисперсией
  $$
  {\sf D}(t)=\left( 2\mathbf{d}_1\mathbf{s}-\lambda\right) t +2\left( 
\mathbf{d}_2\mathbf{s}-\lambda\right) +o(1)\,,
  $$
где векторы-столб\-цы~$\mathbf{d}_1$ и~$\mathbf{d}_2$~--- единственные 
решения систем линейных уравнений~\cite{2-nau}:
\begin{alignat*}{2}
\mathbf{d}_1\mathbf{A} &=\mathbf{p}(\lambda \mathbf{I}-\mathbf{R})\,,&\quad
\mathbf{d}_1\mathbf{u}&=1\,;\\
\mathbf{d}_2\mathbf{A}&=\mathbf{d}_1 -\mathbf{p}\,, &\quad
\mathbf{d}_2\mathbf{u}&=1\,.
\end{alignat*}
    
\subsection{Простой марковский поток неоднородных событий}

  Простой марковский поток неоднородных событий~--- это марковский 
поток событий нескольких типов, в каждый вызывающий момент которого 
наступает ровно одно событие. Такой поток характеризуется $K\hm+1$ 
матрицами интенсивностей переходов $\mathbf{S}\hm= \mathbf{A}_0$ и 
$\mathbf{R}_k\hm= \mathbf{A}_{\mathbf{e}_k}$, $k\hm=1,2,\ldots ,K$, 
а~остальные матрицы~$\mathbf{A}_{\mathbf{n}}$~--- нулевые. При этом поток 
событий одного типа, например типа~$i$, является простым марковским 
потоком однородных событий, описываемым матрицами 
$\mathbf{S}_i\hm=\mathbf{A}\hm- \mathbf{A}_{\mathbf{e}_i}$ 
и~$\mathbf{R}_i$. Первыми работами, посвященными прос\-тым марковским 
потокам неоднородных событий, считаются~[14--16]. В~англоязычной 
литературе такой поток называют Markovian Arrival Process with marked arrivals 
и~используют для его обозначения сокращение ММАР.  
Из~(\ref{e5-nau}) вытекает следующее выражение для плотности совместного 
распределения ${\sf P}(\omega_l=k_l,\tau_l<x_l, l\hm=1,2,\ldots ,m)$ типов 
$\omega_l$ событий, наступивших в~момент~$t_l$, и~длин~$\tau_l$ интервалов 
между вызывающими моментами: 
  \begin{multline}
  f_{{k}_1, {k}_2, \ldots , {k}_m}\left( x_1, x_2, \ldots , x_m\right)={}\\
  {}=\bm{\alpha}\exp \left( x_1\mathbf{S}\right)\mathbf{R}_{k_1}\exp\left( 
x_2\mathbf{S}\right) \mathbf{R}_{k_2}\cdots\\
\cdots \exp \left( x_m\mathbf{S}\right) 
\mathbf{R}_{k_m}\mathbf{u}\,,\quad
 1\leq k_1, k_2, \ldots , k_m\leq K\,,\\
x_0, x_1, \ldots , x_m>0\,,\quad  m=1,2,\ldots
 \label{e9-nau}
\end{multline}

\subsection{Марковский поток групп однородных событий}

  Марковский поток групп однородных событий~--- это марковский поток 
событий одного типа, в каждый вызывающий момент которого \mbox{может} 
наступить несколько событий. Такие марковские потоки впервые 
исследовались в~\cite{8-nau, 17-nau, 18-nau}, а их описание с помощью 
матриц~$\mathbf{A}_{\mathbf{n}}$ впервые появилось в~\cite{19-nau}. 
В~англоязычной литературе такой поток сейчас называют batch Markovian 
arrival process и используют для его обозначения сокращение BMAP. 
В~\cite{20-nau} получены формулы и асимптотики для первых двух моментов 
считающего процесса~$N(t)$, а~в~\cite{21-nau}~--- для старших моментов~$N(t)$.

\section{Матрично-экспоненциальные распределения}

  Функция распределения $F(t)$ неотрицательной случайной величины 
называется мат\-рич\-но-экс\-по\-нен\-ци\-аль\-ной, если $F(0)\hm<1$ и она 
представима в~виде 
  \begin{equation}
  F(t)=1-\mathbf{q}\exp (t\mathbf{S})\mathbf{u}
  \label{e10-nau}
  \end{equation}
с некоторым вектором~$\mathbf{q}$ и матрицей~$\mathbf{S}$, име\-ющей 
собственные числа лишь с отрицательными действительными частями. Для 
того чтобы функция распределения~$F(t)$ неотрицательной случайной 
величины была  
мат\-рич\-но-экс\-по\-нен\-ци\-аль\-ной, необходимо и достаточно, чтобы она 
имела рациональное преобразование 
Лап\-ла\-са--Стилть\-еса $\tilde{F}(\nu)$. Минимальный порядок 
матрицы~$\mathbf{S}$  
в~мат\-рич\-но-экс\-по\-нен\-ци\-аль\-ном представлении~(\ref{e10-nau}) равен 
чис\-лу полюсов функции $\tilde{F}(\nu)$ с учетом их кратности. Представление 
с~матрицей~$\mathbf{S}$ минимального порядка называется минимальным. 

  В некоторых работах по мат\-рич\-но-экс\-по\-нен\-ци\-аль\-ным функциям  
распределения~\cite{22-nau, 23-nau, 24-nau}, а~также в книгах~\cite{25-nau, 26-nau}, 
чтобы подчеркнуть аналогию с экспоненциальными функциями 
\mbox{распределения},  
вмес\-то~(\ref{e10-nau}) использовалось пред\-став\-ле\-ние $F(t)\hm= 1\hm - 
\mathbf{q}\exp (-t\mathbf{B})\mathbf{u}$ со знаком минус перед~$t$ 
и~мат\-ри\-цей~$\mathbf{B}$, име\-ющей собственные чис\-ла с~положительными 
действительными частями. В~настоящее\linebreak время используются только 
представления вида~(\ref{e10-nau}). Иногда допускается, что 
вектор~$\mathbf{u}$ в~(\ref{e10-nau}) может быть любым, а~не состоящим из 
единиц, как в~рас\-смат\-ри\-ва\-емом случае. Однако в~\cite{24-nau, 27-nau} 
было показано, что всегда можно подобрать мат\-рич\-но-экс\-по\-нен\-ци\-аль\-ное 
пред\-став\-ле\-ние с~$\mathbf{u}\hm=(1,1,\ldots , 1)$. 
  
  Идея матрично-экс\-по\-нен\-ци\-аль\-ных функций распределения восходит 
к работе~\cite{28-nau}, в которой показано, что рациональные преобразования  
Лап\-ла\-са--Стилть\-еса неотрицательных функций распределения 
представимы в виде:
  $$
  \tilde{F}(s)=p_0+\sum\limits^L_{l=1} q_0\cdots q_{l-1} p_l \prod\limits^l_{i=1} 
\fr{\lambda_i}{\lambda_{i}+s}\,,
  $$
где $p_i+q_i\hm=1$, $i\hm=1, \ldots , L$, $p_L\hm=1$, и~$-\lambda_i$, $i\hm=1, 
\ldots , L$,~--- полюсы~$\tilde{F}(s)$. Такое представление можно записать в 
мат\-рич\-но-экс\-по\-нен\-ци\-аль\-ном виде~(\ref{e10-nau}), полагая 
\begin{align*}
\mathbf{q}&=(1,0,\ldots ,0)\,;\\
\mathbf{S}&=\begin{bmatrix}
-\lambda_1&q_1\lambda_1&0&\cdots&0\\
0&-\lambda_2&q_2\lambda_2&\ddots &\vdots\\
0&0&\ddots& \ddots& 0\\
\vdots& \ddots& \ddots& -\lambda_{L-1}&q_{L-1}\lambda_{L-1}\\
0&\cdots & 0&0&-\lambda_L
\end{bmatrix}\,,
\end{align*}
%
  при этом элементы матрицы~$\mathbf{S}$ могут быть комплексными. 
В~\cite{22-nau} показано, что вектор~$\mathbf{q}$ и~мат\-ри\-ца~$\mathbf{S}$  
в~мат\-рич\-но-экс\-по\-нен\-ци\-аль\-ном  
пред\-ставлении~(\ref{e10-nau}) всегда могут быть выбраны действительными. 
  
  Из~(\ref{e10-nau}) вытекают формулы для начальных моментов
  \begin{equation*}
  \int\limits_0^\infty t^n dF(t)=n! \mathbf{q}(-\mathbf{S})^{-n}\mathbf{u}\,,\enskip 
n=1,2,\ldots
  %\label{e11-nau}
  \end{equation*}
и для преобразования Лап\-ла\-са--Стилть\-еса функции распределения~$F(t)$ 
\begin{multline*}
\tilde{F}(\nu)=\int\limits_0^\infty e^{-\nu t}dF(t)={}\\
{}=1-
\mathbf{q}\mathbf{u}+\mathbf{q}(\nu\mathbf{I}-\mathbf{S})^{-1} \mathbf{s}=1-
\nu\mathbf{q}(\nu\mathbf{I}-\mathbf{S})^{-1}\mathbf{u}\,,
%\label{e12-nau}
\end{multline*}
где $\mathbf{s}=-\mathbf{Su}$. Кроме того,  
мат\-рич\-но-экс\-по\-нен\-ци\-аль\-ные функции распределения обладают 
сле\-ду\-ющи\-ми свойствами~\cite{24-nau}.
\begin{enumerate}[1.]
\item Пусть $F_i(t)=1\hm- \mathbf{q}_i\exp(t\mathbf{S}_i) \mathbf{u}$, 
$i\hm=1,2$,~--- мат\-рич\-но-экс\-по\-нен\-ци\-аль\-ные функции 
распределения и $p_1\hm+p_2\hm=1$. Тогда
\begin{align*}
p_1F_1(t)+p_2F_2(t)&={}\\
&\hspace*{-15mm}{}=1-(p_1\mathbf{q}_1, p_2\mathbf{q}_2)\exp \left( 
t\begin{bmatrix} \mathbf{S}_1 & \mathbf{0}\\
\mathbf{0}& \mathbf{S}_2\end{bmatrix}
\right) \mathbf{u}\,; %\label{e13-naum}
\\
\left( F_1*F_2\right) (t) &={}\\
&\hspace*{-25mm}{}= 1-\left(\mathbf{q}_1,F_1(0) \mathbf{q}_2\right) \exp 
\left( t \begin{bmatrix}
\mathbf{S}_1 & -\mathbf{S}_1\mathbf{uq}_2\\
\mathbf{0} & \mathbf{S}_2\end{bmatrix} \right) \mathbf{u}\,.
%\label{e14-nau}
\end{align*}
\item Пусть $\tau$ и~$\gamma$~--- независимые неотрицательные случайные 
величины с функциями распределения~$F(t)$ и~$G(t)$ соответственно, 
причем~$F(t)$ имеет  
мат\-рич\-но-экс\-по\-нен\-ци\-аль\-ное представление~(\ref{e10-nau}). Тогда 
функция распределения~$H(t)$ случайной величины  $(\tau\hm-\gamma)^+$ 
имеет  
мат\-рич\-но-экс\-по\-нен\-ци\-аль\-ное пред\-став\-ление 
\begin{equation*}
H(t)=1-\mathbf{qU}\exp (t\mathbf{S})\mathbf{u}\,,
%\label{e15-nau}
\end{equation*}
где 
\begin{equation}
\mathbf{U}=\int\limits_0^\infty e^{t\mathbf{S}}dG(t)\,.
\label{e16-nau}
\end{equation}

\item Пусть $F(t)$ имеет мат\-рич\-но-экс\-по\-нен\-ци\-аль\-ное  
представление~(\ref{e10-nau}), а~у~квад\-рат\-ной мат\-ри\-цы~$\mathbf{V}$ 
все собственные числа имеют неотрицательные вещественные части. Тогда
\begin{multline*}
\int\limits_0^\infty e^{-t\mathbf{V}}dF(t)=(1-\mathbf{qu}) \mathbf{I}+
(\mathbf{q}\otimes \mathbf{I})\boldsymbol{\Psi}(\mathbf{Su}\otimes 
\mathbf{I})={}\\
{}=\mathbf{I}-(\mathbf{q}\otimes 
\mathbf{I})\boldsymbol{\Psi}(\mathbf{u}\otimes \mathbf{V})\,,
%\label{e17-nau}
\end{multline*}
где $\boldsymbol{\Psi}=(\mathbf{I}\otimes \mathbf{V}\hm- \mathbf{S}\otimes 
\mathbf{I})^{-1}$. 
  \end{enumerate}
  
  Последнее свойство можно использовать для вычисления 
матриц~$\mathbf{U}$ в~(\ref{e16-nau}) для мат\-рич\-но-экс\-по\-нен\-ци\-аль\-ных 
функций распределения~$G(t)$.
  
  Ясно, что функции распределения фазового типа являются  
мат\-рич\-но-экс\-по\-нен\-ци\-аль\-ны\-ми. Однако их  
мат\-рич\-но-экс\-по\-нен\-ци\-аль\-ные представления 
  \begin{multline*}
  F(t)-1-\mathbf{q}\exp (t\mathbf{S})\mathbf{u}\,,\enskip
  %\label{e18-nau}
    F(0)=1-\mathbf{qu}\,,\\
     \fr{d}{dt}\,F(t)= \mathbf{q}\exp 
(t\mathbf{S})\mathbf{s}\,,\ t>0\,,
  \end{multline*}
с ограничениями
\begin{equation}
\hspace*{-2mm}\left.
\begin{array}{rlrlrl}
\!\!\displaystyle 0<\sum\limits_{j\in \boldsymbol{\mathcal{X}}} q(j)&\leq 1\,,&\ q(i)&\geq0\,,&\ i&\in 
\boldsymbol{\mathcal{X}};
\\[9pt]
\!\!\displaystyle \sum\limits_{j\in \boldsymbol{\mathcal{X}}} \!\!s(i,j)&\leq 0\,,&\ s(i,j)&\geq 0\,,&\ 
i&\not= j\,,\ i, j\in \boldsymbol{\mathcal{X}},
\end{array}\!
\right\}\!\!\!\!
\label{e19-nau}
\end{equation}
где $\mathbf{S}=[s(i,j)]$, следует отличать от мат\-рич\-но-экс\-по\-нен\-ци\-аль\-ных 
представлений этих же функций, но без ограничений~(\ref{e19-nau}). 
Порядок  
мат\-рич\-но-экс\-по\-нен\-ци\-аль\-но\-го представления, удовлетворяющего 
ограничениям~(\ref{e19-nau}), будем называть числом этапов этого 
представления, а~порядок мат\-рич\-но-экс\-по\-нен\-ци\-аль\-но\-го 
представления, не удовлетворяющего этим ограничениям, следуя~\cite{28-nau}, 
будем называть\linebreak
 числом \textit{фиктивных} этапов. Необходимые и 
достаточные условия того, чтобы для функции распределения 
с~рациональным преобразованием Лап\-ла\-са--Стилть\-еса существовало 
представление, \mbox{удовлетворяющее} ограничениям~(\ref{e19-nau}), получены 
в~\cite{29-nau}. Для этого надо, чтобы (а)~функция распределения имела 
непрерывную положительную плотность на правой полуоси и~(б)~ее 
преобразование Лап\-ла\-са--Стилть\-еса имело единственный полюс 
с~максимальной вещественной частью. 

\section{Рациональные потоки событий}

  Рациональный поток групп неоднородных событий 
$(t_l,\boldsymbol{\sigma}_l)$, $l\hm=1,2,\ldots$, можно определить как поток, 
для которого совместное распределение чис\-ла~$\boldsymbol{\sigma}_l$ 
наступивших событий и~длин~$\tau_l$ интервалов между моментами~$t_l$ 
наступления событий дается формулами~(\ref{e1-nau}) и~(\ref{e3-nau}) 
с~матрицами~$\mathbf{A}_{\mathbf{n}}$, $\mathbf{n}\hm\in 
\boldsymbol{\mathcal{N}}^K $, обладающими следующими свойствами:
  \begin{enumerate}[(1)]
\item действительные части собственных чисел мат\-ри\-цы~$\mathbf{A}_{\mathbf{0}}$ 
отрицательны;
\item действительные части собственных чисел мат\-ри\-цы 
$\mathbf{A}\hm= \sum\nolimits_{\mathbf{n}\in 
\boldsymbol{\mathcal{N}}^K} \mathbf{A}_{\mathbf{n}}$ 
неположительны;
\item $\mathbf{Au}=\mathbf{0}$.
\end{enumerate}
  
  Для стационарных версий рациональных потоков дополнительно требуется, 
чтобы начальный вектор~$\bm{\alpha}$ совпадал с решением~$\mathbf{p}$ 
системы линейных уравнений $\mathbf{pA}\hm=0$, $\mathbf{pu}\hm=1$.
  
  Простой рациональный поток однородных событий, также называемый  
мат\-рич\-но-экс\-по\-нен\-циальным потоком~\cite{30-nau},~--- это поток 
событий одного типа, в каждый вызывающий момент которого наступает 
ровно одно событие и для которого плотность совместного распределения 
длин~$\tau_l$ интервалов между моментами наступления событий дается 
формулой~(\ref{e6-nau}) с матрицами~$\mathbf{S}$ и~$\mathbf{R}$, 
обладающими следующими свойствами~\cite{31-nau}:
\begin{itemize}
\item[(а)] вещественные части собственных чисел матрицы~$\mathbf{S}$ 
отрицательны;
\item[(б)] вещественные части собственных чисел матрицы 
$\mathbf{S}\hm+\mathbf{R}$ неположительны; 
\item[(в)] $(\mathbf{S}+\mathbf{R})\mathbf{u}=\mathbf{0}$. 
  \end{itemize}
  
  Примерами простых рациональных потоков однородных событий могут 
служить полумарковские потоки~\cite{22-nau} и процессы 
восстановления~\cite{27-nau}  
с~мат\-рич\-но-экс\-по\-нен\-ци\-аль\-ны\-ми функциями распределения длин 
интервалов между наступлениями событий.
  
  Рациональный поток неоднородных событий~--- это поток событий 
нескольких типов, в каждый вызывающий момент которого наступает ровно 
одно событие. Для такого потока совместное распределение типов 
наступивших событий~$\omega_l$ и длин~$\tau_l$ интервалов между 
моментами наступления событий дается формулой~(\ref{e9-nau}), а на 
матрицы~$\mathbf{S}$ 
и~$\mathbf{R}\hm=\mathbf{R}_1\hm+\mathbf{R}_2+\cdots  + \mathbf{R}_K$ 
накладываются перечисленные выше ограничения~(a)--(в)~\cite{32-nau}. 

\section{Заключение}

  Метод этапов Эрланга~\cite{33-nau} более 100~лет применяется при анализе 
стохастических систем. К~его широкому распространению привело открытие  
мат\-рич\-но-экс\-по\-нен\-ци\-аль\-но\-го представления для функций 
распределения фазового типа~\cite{34-nau} и моделей марковских потоков 
событий~\cite{1-nau, 17-nau}. Эти модели хорошо подходят для анализа 
стохастических систем с~по\-мощью вычислительной техники, 
приспособленной к обработке векторов и матриц, что привело к развитию 
специальных матричных методов анализа стохастических систем.
  
  Метод фиктивных этапов, предложенный в~\cite{28-nau}, позволил 
распространить метод Эрланга на любые распределения с рациональным 
преобразованием 
  Лап\-ла\-са--Стилть\-еса. Использование мат\-рич\-но-экс\-по\-нен\-ци\-аль\-ных 
  представлений для функций распределения~\cite{22-nau, 23-nau, 25-nau} 
  и~потоков случайных событий~\cite{31-nau} с произвольными рациональными 
преобразованиями 
  Лап\-ла\-са--Стилть\-еса упрощает применение метода фиктивных этапов. 
Формальное применение метода фиктивных этапов приводит\linebreak 
к~решению, 
в~котором вероятности, со\-от\-вет\-ст\-ву\-ющие фиктивным этапам, могут оказаться 
отрицательными, б$\acute{\mbox{о}}$льшими единицы или даже 
комплекс\-ны\-ми. Однако вероятности, соответствующие \mbox{не\-фик\-тив\-ным} 
состояниям, будут неотрицательными числами, не превосходящими единицы. 
Существуют различные интерпретации понятий отрицательных вероятностей 
и интенсивностей \mbox{переходов} \cite{35-nau, 36-nau, 37-nau, 38-nau}. Более 
детально ознакомиться с~{марковским} и~рациональным потоками событий, 
а~также с~матричными методами анализа стохастических систем можно 
 в~обзорах~\cite{39-nau, 40-nau, 41-nau, 42-nau, 43-nau, 44-nau}  
и~монографиях~[18, 25, 26, 45--57]. 

{\small\frenchspacing
 {%\baselineskip=10.8pt
 %\addcontentsline{toc}{section}{References}
 \begin{thebibliography}{99}

\bibitem{1-nau}
\Au{Наумов В.\,А.} О~независимой работе подсистем сложной системы~// Тр.~III 
Всесоюзной  
шко\-лы-се\-ми\-на\-ра по теории массового обслуживания.~--- 
М.: МГУ, 1976. №\,2. С.~169--177.
\bibitem{2-nau}
\Au{Бочаров П.\,П., Наумов В.\,А.} Анализ гиперэкспоненциальной двухфазной системы с 
ограниченным накопителем~// Информационные сети и их структура.~--- М.: Наука, 
1976.  
С.~168--180.
\bibitem{3-nau}
\Au{Наумов В.\,А.} Об обслуженной и избыточной нагрузках полнодоступного пучка с 
ограниченной очередью~// Численные методы решения задач математической физики и 
теории систем.~--- М.: УДН, 1977. С.~51--55.
\bibitem{4-nau}
\Au{Наумов В.\,А.} Исследование некоторых многофазных систем массового 
обслуживания: Дис.\ \ldots\ канд. физ.-мат. наук.~--- М.: УДН, 1978.  98~с.
\bibitem{5-nau}
\Au{Lucantoni D.\,M., Meier-Hellstern~K., Neuts M.\,F.} A~single-server queue with server 
vacations and a class of non-renewal arrival processes~// Adv. Appl. Probab., 1990. 
Vol.~22. Iss.~3. P.~676--705.
\bibitem{6-nau}
\Au{Башарин Г.\,П., Кокотушкин~В.\,А., Наумов~В.\,А.} О~методе эквивалентных замен 
расчета фрагментов сетей связи для ЦВМ~// Изв. АН \mbox{СССР}. Техническая кибернетика, 1979. №\,6. С.~92--99.
\bibitem{7-nau}
\Au{Basharin G., Naumov V.} Simple matrix description of peaked and smooth traffic and 
its applications~// 3rd ITC Specialist Seminar on Fundamentals of Teletraffic Theory.~--- M.: 
VINITI, 1984. P.~38--44. 
\bibitem{8-nau}
\Au{Neuts M.\,F.} Renewal processes of phase type~// Nav. Res. Logist.~Q., 1978. 
Vol.~25. Iss.~3. P.~445--454.
\bibitem{9-nau}
\Au{Cinlar E.} Queues with semi-Markovian arrivals~// J.~Appl. Probab., 1967. Vol.~4. Iss.~2.  
P.~365--379.
\bibitem{10-nau}
\Au{Franken P.} Erlangsche Formeln f$\ddot{\mbox{u}}$r semimarkowschen Eingang // 
Elektronische Informationsverarbeitung Kybernetik, 1968. Vol.~4. Iss.~3. P.~197--204.
\bibitem{11-nau}
\Au{Franken P., Kerstan~J.} Bedienungssysteme mit unendlich vielen Bedienungsapparaten~// 
Operationsforschung Mathematische Statistik.~--- Berlin: Akademie-Verlag, 1968. Vol.~I. 
P.~67--76.
\bibitem{12-nau}
\Au{Neuts M.\,F., Chen~S.-Z.} The infinite server queue with semi-Markovian arrivals and negative 
exponential services~// J.~Appl. Probab., 1972. Vol.~9. Iss.~1. P.~178--184.
\bibitem{13-nau}
\Au{Bean N.\,G., Green D.\,A., Taylor~P.\,G.} When is a MAP poisson?~// 2nd Australia--Japan 
Workshop on Stochastic Models in Engineering,
Technology 
and Management Proceedings~/ Eds. J.~Wilson, D.\,N.\,P.~Murthy, S.~Osaki.~--- 
Brisbane: Technology Management Center, University of Queensland, 1996. P.~34--43.
\bibitem{14-nau}
\Au{Наумов В.\,А.} Матричный аналог формулы Эрланга~// Модели распределения 
информации и методы их анализа.~--- М.: ВИНИТИ, 1988. C.~39--43.
\bibitem{15-nau}
\Au{He Q.-M.} Queues with marked customers~// Adv. Appl. Probab., 1996. Vol.~28. 
Iss.~2. P.~567--587.
\bibitem{16-nau}
\Au{He Q.-M., Neuts M.\,F.} Markov chains with marked transitions~// Stoch. Proc. 
Appl., 1998. Vol.~74. P.~37--52.
\bibitem{17-nau}
\Au{Neuts M.\,F.} A versatile Markovian point process.~--- Newark, DE: 
University of Delaware, Department of Statistics and Computer Science, 1977.
 Technical Report 77/13. 29~p.
\bibitem{18-nau}
\Au{Neuts M.\,F.} Structured stochastic matrices of $M/G/1$ type and their applications.~--- New 
York, NY, USA: Marcel Dekker, 1989. 512~p.
\bibitem{19-nau}
\Au{Lucantoni D.\,M.} New results on the single server queue with a batch Markovian arrival 
process~// Communications Statistics. Stochastic Models, 1991. Vol.~7. Iss.~1. P.~1--46. 
\bibitem{20-nau}
\Au{Narayana S., Neuts M.\,F.} The first two moment matrices of the counts for the Markovian 
arrival processes~// Communications Statistics. Stochastic Models, 1992. Vol.~8. Iss.~3. P.~459--477. 
\bibitem{21-nau}
\Au{Nielsen B.\,F., Nilsson L.\,A.\,F., Thygesen~U.\,H., Beyer~J.\,E.} Higher order moments and 
conditional asymptotics of the batch Markovian arrival process~// Stoch. Models, 2007. Vol.~23. 
Iss.~1. P.~1--26.
\bibitem{22-nau}
\Au{Бочаров П.\,П., Наумов В.\,А.} O~некоторых системах массового обслуживания 
конечной емкости~// Проб\-ле\-мы передачи информации, 1977. Т.~13. №\,4. С.~96--104.
\bibitem{23-nau}
\Au{Наумов В.\,А.} Об однолинейной системе с ограниченным накопителем и заявками 
нескольких видов~// Модели систем распределения информации и их анализ.~--- М.: 
Наука, 1982. C.~77--82.
\bibitem{24-nau}
\Au{Наумов В.\,А.} О~функциях распределения с рациональным преобразованием  
Лап\-ла\-са--Стилть\-еса~// Анализ информационно-вычислительных систем.~--- М.: 
УДН, 1986. С.~47--56.
\bibitem{25-nau}
\Au{Бочаров П.\,П., Печинкин~А.\,В.} Теория массового обслуживания.~--- М.: РУДН, 
1995. 528~с.
\bibitem{26-nau}
\Au{Bocharov P.\,P., D'Apice~C., Pechinkin~A.\,V., Salerno~S.} Queueing theory.~--- Utrecht--Boston: 
VSP, 2004. 446~p.
\bibitem{27-nau}
\Au{Asmussen S., Bladt M.} Renewal theory and queueing algorithms for matrix-exponential 
distributions~// Matrix-analytic methods in stochastic models~/
Eds. A.\,S.~Alfa, S.~Chakravarty.~--- New York, NY, USA: Marcel 
Dekker, 1996. P.~313--341.
\bibitem{28-nau}
\Au{Cox D.\,R.} A use of complex probabilities in the theory of stochastic processes~// Math. 
Proc. Cambridge, 1955. Vol.~51. Iss.~2. P.~313--319. 
\bibitem{29-nau}
\Au{O'Cinneide C.\,A.} Characterization of phase-type distributions~// Communications Statistics. 
Stochastic Models, 1990. Vol.~6. Iss.~1. P.~1--57.
\bibitem{30-nau}
\Au{Bodrog L., Horv$\acute{\mbox{a}}$th~A., Telek~M.} On the properties of moments of matrix 
exponential distributions and matrix exponential processes~// Dagstuhl Seminar Proceedings, 2008. 
Vol.~07461. Paper~1394.
\bibitem{31-nau}
\Au{Asmussen S., Bladt M.} Point processes with finite-dimensional conditional probabilities~// 
Stoch. Proc. \mbox{Appl.}, 1999. Vol.~82. Iss.~1. P.~127--142.
\bibitem{32-nau}
\Au{Horvath G., Telek M.} Acceptance-rejection methods for generating random variables from 
matrix exponential distribution and rational arrival processes~// Matrix-analytic methods in stochastic 
models~/ Eds. G.~Latouche, V.~Ramaswami, J.~Sethuraman, \textit{et al.}~--- 
New York, NY, USA: Springer, 2012. P.~123--144.
\bibitem{33-nau}
\Au{Erlang A.\,K.} \mbox{L{\!\ptb{\o}}sning} af nogle Problemer fra Sandsynlighedsregningen af 
Betydning for de automatiske Telefoncentraler~// Elektroteknikeren, 1917. Iss.~13. P.~5--13.
\bibitem{34-nau}
\Au{Neuts M.\,F.} Probability distribution of phase type~// Liber Amicorum Professor Emeritus 
H.~Florin.~--- Ottignies-Louvain-la-Neuve, Belgium: University of Louvain, Department of Mathematics, 
1975. P.~173--206.
\bibitem{35-nau}
\Au{Bartlett M.\,S.} Negative probability~// Math. Proc. Cambridge, 1945. Vol.~41. Iss.~1. P.~71--73.
\bibitem{36-nau}
\Au{Cox D.\,R.} The analysis of non-Markovian stochastic processes by the inclusion of 
supplementary variables~// Math. Proc. Cambridge, 1955. Vol.~51. Iss.~3. P.~433--441. 
\bibitem{37-nau}
\Au{Bladt M., Neuts M.\,F.} Matrix-exponential distributions: Calculus and interpretations via flows~// 
Stoch. Models, 2003. Vol.~19. Iss.~1. P.~113--124.
\bibitem{38-nau}
\Au{Khrennikov A.} Interpretations of probability.~--- 2nd ed.~--- Berlin: Walter de Gruyter, 2009. 
237~p.
\bibitem{39-nau}
\Au{Наумов В.\,А.} Марковские модели потоков требований~// Системы массового 
обслуживания и информатика.~--- М.: УДН, 1987. С.~67--73.
\bibitem{40-nau}
\Au{Asmussen S.} Matrix-analytic models and their analysis~// Scand. J.~Stat., 2000. 
Vol.~27. Iss.~2. P.~193--226.
\bibitem{41-nau}
\Au{Bladt M.} A~review on phase-type distributions and their use in risk theory~// ASTIN Bull., 
2005. Vol.~35. No.\,1. P.~145--161.
\bibitem{42-nau}
\Au{Artalejo J.\,R., G$\acute{\mbox{o}}$mez-Corral~A.} Markovian arrivals in stochastic 
modelling: A~survey and some new results~// SORT~--- Stat. Oper. Res.~T., 2010. 
Vol.~34. Iss.~2. P.~101--144.
\bibitem{43-nau}
\Au{Вишневский В.\,М., Дудин~А.\,Н.} Системы массового обслуживания с 
коррелированными входными потоками и их применение для моделирования 
телекоммуникационных сетей~// Автоматика и телемеханика, 2017. №\,8. С.~3--59.
\bibitem{44-nau}
\Au{Basharin G., Naumov~V., Samouylov~K.} On Markovian modelling of arrival processes~// 
Stat. Pap., 2018. Vol.~59. Iss.~4. P.~1533--1540. 
\bibitem{45-nau}
\Au{Neuts M.\,F.} Matrix-geometric solutions in stochastic models: An algorithmic approach.~--- 
Baltimore, MA, USA: The John Hopkins University Press, 1981. 332~p.
\bibitem{46-nau}
\Au{Latouche G., Ramaswami~V.} Introduction to matrix analytic methods in stochastic modeling.~--- 
Philadelphia, PA, USA: ASA \& SIAM, 1999. 334~p.
\bibitem{47-nau}
\Au{Asmussen S.} Applied probability and queues.~--- New York, NY, USA: Springer, 2003. 
438~p.
\bibitem{48-nau}
\Au{Breuer L., Baum D.} An introduction to queueing theory and matrix-analytic methods.~--- 
Dordrecht: Springer, 2005. 272~p.
\bibitem{49-nau}
\Au{Bini D.\,A., Latouche~G., Meini~B.} Numerical methods for structured Markov chains.~--- 
New York, NY, USA: Oxford University Press, 2005. 336~p.
\bibitem{50-nau}
\Au{Asmussen S., O'Cinneide~C.\,A.} Matrix-exponential distributions~// Encyclopedia of statistical 
sciences~/ Eds. S.~Kotz, C.\,B.~Read, N.~Balakrishnan, 
B.~Vidakovic, N.\,L.~Johnson.~--- Hoboken, NJ, USA: John Wiley \& Sons, 2006. Vol.~3. P.~1--5.
doi: 10.1002/0471667196.ess1092.
\bibitem{51-nau}
\Au{Li Q.-L.} Constructive computation in stochastic models with applications.~--- Berlin: Springer-Verlag, 2009. 650~p.
\bibitem{52-nau}
\Au{Lipsky L.} Queueing theory: A~linear algebraic approach.~--- 2nd ed.~--- New York, NY, 
USA: Springer, 2009. 548~p.
\bibitem{53-nau}
\Au{Alfa A.\,S.} Queueing theory for telecommunications.~--- New York, NY, USA: Springer, 2010. 
238~p.
\bibitem{54-nau}
\Au{He Q.-M.} Fundamentals of matrix-analytic methods.~--- New York, NY, USA: Springer, 2014. 
349~p.
\bibitem{55-nau}
\Au{Buchholz P., Kriege~J., Felko~I.} Input modeling with phase-type distributions and Markov 
models. Theory and applications.~--- New York, NY, USA: Springer, 2014. 127~p.
\bibitem{56-nau}
\Au{Наумов В.\,А., Самуйлов~В.\,А., Гайдамака~Ю.\,В.} Мультипликативные решения 
конечных цепей Маркова.~--- М.: РУДН, 2015. 159~с.
\bibitem{57-nau}
\Au{Bladt M., Nielsen B.\,F.} Matrix-exponential distributions in applied probability.~--- Boston, MA, USA: 
Springer, 2017. 736~p.
\end{thebibliography}

 }
 }

\end{multicols}

\vspace*{-12pt}

\hfill{\small\textit{Поступила в~редакцию 02.07.20}}

\vspace*{8pt}

%\pagebreak

\newpage

\vspace*{-28pt}

%\hrule

%\vspace*{2pt}

%\hrule

%\vspace*{-2pt}

\def\tit{ON MARKOVIAN AND RATIONAL ARRIVAL PROCESSES.~II}


\def\titkol{On Markovian and rational arrival processes.~II}


\def\aut{V.\,A.~Naumov$^1$ and~К.\,Е.~Samouylov$^{2,3}$}

\def\autkol{V.\,A.~Naumov and~К.\,Е.~Samouylov}

\titel{\tit}{\aut}{\autkol}{\titkol}

\vspace*{-11pt}


   \noindent
   $^1$Service Innovation Research Institute, 8A Annankatu, Helsinki 00120, Finland

\noindent
$^2$Peoples' Friendship University of Russia (RUDN University), 6~Miklukho-Maklaya Str., Moscow 
117198, Russian\linebreak
$\hphantom{^1}$Federation

\noindent
$^3$Institute of Informatics Problems, Federal Research Center ``Computer Science and Control'' 
of the Russian\linebreak
$\hphantom{^1}$Academy of Sciences, 44-2~Vavilov Str., Moscow 119333, Russian Federation

  

\def\leftfootline{\small{\textbf{\thepage}
\hfill INFORMATIKA I EE PRIMENENIYA~--- INFORMATICS AND
APPLICATIONS\ \ \ 2020\ \ \ volume~14\ \ \ issue\ 4}
}%
 \def\rightfootline{\small{INFORMATIKA I EE PRIMENENIYA~---
INFORMATICS AND APPLICATIONS\ \ \ 2020\ \ \ volume~14\ \ \ issue\ 4
\hfill \textbf{\thepage}}}

\vspace*{3pt} 
  
  
   
   
  \Abste{This article is the second part of the review carried out within the framework of the RFBR 
project No.\,19-17-50126. The purpose of this review is to get the interested readers familiar with the 
basics of the theory of Markovian arrival processes to facilitate the application of these models in practice 
and, if necessary, to study them in detail. In the first part of the review, the properties of the general 
Markovian arrival processes are presented and their relationship with Markov additive processes and 
Markov renewal processes is shown. In the second part of the review, the important for applications 
subclasses of Markovian arrival processes, i.\,e., simple and batch arrival processes of homogeneous and 
heterogeneous arrivals, are considered. It is shown how the properties of Markovian arrival processes are 
associated with the product form of stationary distributions of Markov systems. In conclusion, 
matrix-exponential distributions and rational arrival processes are discussed that expand the capabilities of 
Markovian arrival processes for modeling complex systems, while preserving the convenience of analyzing 
them using computations.}
  
  \KWE{Markov chain; Markovian arrival process; Markov additive process; MAP; MArP}
  
  
\DOI{10.14357/19922264200406} 

%\vspace*{-20pt}

  \Ack
  \noindent
  The reported study was funded by RFBR, project No.\,19-17-50126. 
  

%\vspace*{6pt}

  \begin{multicols}{2}

\renewcommand{\bibname}{\protect\rmfamily References}
%\renewcommand{\bibname}{\large\protect\rm References}

{\small\frenchspacing
 {%\baselineskip=10.8pt
 \addcontentsline{toc}{section}{References}
 \begin{thebibliography}{99}
  
  \bibitem{1-nau-1}
  \Aue{Naumov, V.\,A.} 1976. O~nezavisimoy rabote podsistem slozhnoy sistemy [About independent 
operation of subsystems of a complex system]. \textit{Tr. III Vsesoyuznoy shkoly-seminara po teorii 
massovogo obsluzhivaniya} [3th All-Russian School-Seminar of Queuing Theory Proceedings]. 
Moscow. 2:169--177.
  \bibitem{2-nau-1}
  \Aue{Bocharov, P.\,P., and V.\,A.~Naumov.} 1976. Analiz gipereksponentsial'noy dvukhfaznoy sistemy 
s~ogranichennym nakopitelem [Analysis of a hyperexponential two-phase system with a limited storage]. 
\textit{Informatsionnye seti i~ikh struktura} [Information networks and their structure]. Moscow: 
Nauka. 
  168--180.
  \bibitem{3-nau-1}
  \Aue{Naumov, V.\,A.} 1977. Ob obsluzhennoy i~izbytochnoy nagruzkakh polnodostupnogo puchka 
s~ogranichennoy ochered'yu [About serviced and excessive loads of a fully accessible bundle with a limited 
queue]. \textit{Chislennye metody resheniya zadach matematicheskoy fiziki i~teorii system} 
[Numerical methods for solving problems of mathematical physics and systems theory]. Moscow: UDN. 
51--55.
  \bibitem{4-nau-1}
  \Aue{Naumov, V.\,A.} 1978. Issledovanie nekotorykh mnogofaznykh sistem massovogo obsluzhivaniya 
[Research of some multiphase queuing systems].  Moscow: UDN.  PhD Thesis. 98~p.
  \bibitem{5-nau-1}
  \Aue{Lucantoni, D.\,M., K.~Meier-Hellstern, and M.\,F.~Neuts.} 1990. A single-server queue with 
server vacations and a~class of non-renewal arrival processes. \textit{Adv. Appl. Probab.} 
22(3):676--705.
  \bibitem{6-nau-1}
  \Aue{Basharin, G.\,P., V.\,A.~Kokotushkin, and V.\,A.~Naumov.} 1979. O~metode ekvivalentnykh 
zamen rascheta fragmentov setey svyazi dlya TsVM [On the method of equivalent substitutions for 
calculating fragments of communication networks for a central computer]. \textit{Engineering Cybernetics}
 6:92--99.
  \bibitem{7-nau-1}
  \Aue{Basharin, G.\,P., and V.\,A.~Naumov.} 1984. Simple matrix description of peaked and smooth 
traffic and its applications. \textit{3rd ITC Specialist Seminar on Fundamentals of Teletraffic Theory}. 
Moscow: VINITI. 
  38--44. 
  \bibitem{8-nau-1}
  \Aue{Neuts, M.\,F.} 1978. Renewal processes of phase type. \textit{Nav. Res. Logist.~Q.} 25(3):445--454.
  \bibitem{9-nau-1}
  \Aue{Cinlar, E.} 1967. Queues with semi-Markovian arrivals. \textit{J.~Appl. Probab.} 4(2):365--379.
  \bibitem{10-nau-1}
  \Aue{Franken, P.} 1968. Erlangsche Formeln f$\ddot{\mbox{u}}$r semimarkowschen Eingang. 
\textit{Elektronische Informationsverarbeitung  Kybernetik} 4(3):197--204.
  \bibitem{11-nau-1}
  \Aue{Franken, P., and J.~Kerstan.} 1968. Bedienungssysteme mit unendlich vielen 
Bedienungsapparaten. \textit{Operationsforschung Mathematische Statistik} 1:67--76.
  \bibitem{12-nau-1}
  \Aue{Neuts, M.\,F., and S.-Z.~Chen.} 1972. The infinite server queue with semi-Markovian arrivals 
and negative exponential services. \textit{J.~Appl. Probab.} 9(1):178--184.
  \bibitem{13-nau-1}
  \Aue{Bean, N.\,G., D.\,A.~Green, and P.\,G.~Taylor.} 1996. When is a MAP poisson? \textit{2nd 
  Australia--Japan Workshop on Stochastic Models in Engineering, Technology and Management 
Proceedings}. Eds. J.~Wilson, D.\,N.\,P.~Murthy, and S.~Osaki. 
Brisbane: Technology Management Center, University of Queensland. 34--43.
  \bibitem{14-nau-1}
  \Aue{Naumov, V.\,A.} 1988. Matrichnyy analog formuly Erlanga [The matrix analogue of a formula of 
Erlang]. \textit{Modeli raspredeleniya informatsii i~metody ikh analiza} [Information distribution 
models and methods for their analysis]. Moscow: VINITI. 39--43.
  \bibitem{15-nau-1}
  \Aue{He, Q.-M.} 1996. Queues with marked customers. \textit{Adv. Appl. Probab.} 
28(2):567--587.
  \bibitem{16-nau-1}
  \Aue{He, Q.-M., and M.\,F. Neuts.} 1998. Markov chains with marked transitions. \textit{Stoch. 
Proc. Appl.} 74:37--52.
  \bibitem{17-nau-1}
  \Aue{Neuts, M.\,F.} 1977. A~versatile Markovian point process.  
Newark, DE: University of Delaware, Department of Statistics and Computer Science.
Technical Report 77/13. 29~p.
  \bibitem{18-nau-1}
  \Aue{Neuts, M.\,F.} 1989. \textit{Structured stochastic matrices of $M/G/1$ type and their 
applications}. New York, NY: Marcel Dekker. 512~p.
  \bibitem{19-nau-1}
  \Aue{Lucantoni, D.\,M.} 1991. New results on the single server queue with a batch Markovian arrival 
process. \textit{Communications Statistics. Stochastic Models} 7(1):1--46. 
  \bibitem{20-nau-1}
  \Aue{Narayana, S., and M.\,F.~Neuts.} 1992. The first two moment matrices of the counts for the 
Markovian arrival processes. \textit{Communications Statistics. Stochastic Models} 8(3):459--477. 
  \bibitem{21-nau-1}
  \Aue{Nielsen, B.\,F., L.\,A.\,F.~Nilsson, U.\,H.~Thygesen, and J.\,E.~Beyer}. 2007. Higher order 
moments and conditional asymptotics of the batch Markovian arrival process. \textit{Stoch. Models} 
23(1):1--26.
  \bibitem{22-nau-1}
  \Aue{Bocharov, P.\,P., and V.\,A.~Naumov.} 1977. O~nekotorykh sistemakh massovogo 
obsluzhivaniya konechnoy emkosti [On some queueing systems of finite capacity]. \textit{Problemy 
peredachi informatsii} [Problems of Information Transmission] 13(4):96--104.
  \bibitem{23-nau-1}
  \Aue{Naumov, V.\,A.} 1982. Ob odnolineynoy sisteme s~ogranichennym nakopitelem i~zayavkami 
neskol'kikh vidov [About a single-line system with limited storage and multiple types of requests]. 
\textit{Modeli sistem raspredeleniya informatsii i~ikh analiz} [Models of information distribution 
systems and methods for their analysis]. Moscow: Nauka. 77--82.
  \bibitem{24-nau-1}
  \Aue{Naumov, V.\,A.} 1986. O~funktsiyakh raspredeleniya s~ratsio\-nal'nym preobrazovaniem  
Laplasa--Stilt'esa [On distribu\-tion functions with rational Laplace--Stiltjes transformation]. \textit{Analiz 
  informatsionno-vychislitel'nykh \mbox{system}}
   [\mbox{Analysis} of information and computing systems]. Moscow: 
UDN. 47--56.
  \bibitem{25-nau-1}
  \Aue{Bocharov, P.\,P., and A.\,V.~Pechinkin.} 1995. \textit{Teoriya massovogo obsluzhivaniya} 
[Queueing theory]. Moscow: RUDN. 528~p.
  \bibitem{26-nau-1}
  \Aue{Bocharov, P.\,P., C.~D'Apice, A.\,V.~Pechinkin, and S.~Salerno.} 2004. \textit{Queueing 
theory}. Utrecht--Boston: VSP. 446~p.
  \bibitem{27-nau-1}
  \Aue{Asmussen, S., and M.~Bladt}. 1996. Renewal theory and queueing algorithms for 
  matrix-exponential distributions. \textit{Matrix-analytic methods in stochastic models}. 
  Eds. A.\,S.~Alfa and 
S.~Chakravarty. New York, NY: Marcel Dekker. 313--341.
  \bibitem{28-nau-1}
  \Aue{Cox, D.\,R.} 1955. A~use of complex probabilities in the theory of stochastic processes. 
\textit{Math. Proc. Cambridge} 51(2):313--319.
  \bibitem{29-nau-1}
  \Aue{O'Cinneide, C.\,A.} 1990. Characterization of phase-type distributions. \textit{Communications 
Statistics. Stochastic Models} 6(1):1--57.
  \bibitem{30-nau-1}
  \Aue{Bodrog, L., A.~Horv$\acute{\mbox{a}}$th, and M.~Telek.} 2008. On the properties of 
moments of matrix exponential distributions and matrix exponential processes. 
\textit{Dagstuhl Seminar Proceedings} 07461:1394.
  \bibitem{31-nau-1}
  \Aue{Asmussen, S., and M.~Bladt.} 1999. Point processes with finite-dimensional conditional 
probabilities. \textit{Stoch. Proc. Appl.} 82(1):127--142.
  \bibitem{32-nau-1}
  \Aue{Horvath, G., and M.~Telek.} 2012. Acceptance-rejection methods for generating random 
variables from matrix exponential distribution and rational arrival processes. \textit{Matrix-analytic 
methods in stochastic models.} Eds. G.~Latouche, V.~Ramaswami, J.~Sethuraman, \textit{et al.} New York, NY: Springer. 123--144.
  \bibitem{33-nau-1}
  \Aue{Erlang, A.\,K.} 1917. \mbox{L{\!\ptb{\o}}sning} af nogle Problemer fra 
Sandsynlighedsregningen af Betydning for de automatiske Telefoncentraler. \textit{Elektroteknikeren} 
13:5--13.
  \bibitem{34-nau-1}
  \Aue{Neuts, M.\,F.} 1975. Probability distribution of phase type. \textit{Liber Amicorum Professor 
Emeritus H.~Florin}. Ottignies-Louvain-la-Neuve, Belgium: University of Louvain, Department of 
Mathematics.  
173--206.
  \bibitem{35-nau-1}
  \Aue{Bartlett, M.\,S.} 1945. Negative probability. \textit{Math. Proc. 
Cambridge} 41(1):71--73.
  \bibitem{36-nau-1}
  \Aue{Cox, D.\,R.} 1955. The analysis of non-Markovian stochastic processes by the inclusion of 
supplementary variables. \textit{Math. Proc. Cambridge} 
51(3):433--441.
  \bibitem{37-nau-1}
  \Aue{Bladt, M., and M.\,F.~Neuts.} 2003. Matrix-exponential distributions: Calculus and 
interpretations via flows. \textit{Stoch. Models} 19(1):113--124.
  \bibitem{38-nau-1}
  \Aue{Khrennikov, A.} 2009. \textit{Interpretations of probability}. 2nd ed. Berlin: Walter de 
Gruyter. 237~p.
  \bibitem{39-nau-1}
  \Aue{Naumov, V.\,A.} 1987. Markovskie modeli potokov trebovaniy [Markov models of demand 
flows]. \textit{Sistemy massovogo obsluzhivaniya i~informatika} [Queuing systems and computer 
science]. Moscow: UDN. 67--73.
  \bibitem{40-nau-1}
  \Aue{Asmussen, S.} 2000. Matrix-analytic models and their analysis. \textit{Scand. 
J.~Stat.} 27(2):193--226.
  \bibitem{41-nau-1}
  \Aue{Bladt, M.} 2005. A~review on phase-type distributions and their use in risk theory. \textit{ASTIN 
Bull.} 35(1):145--161.
  \bibitem{42-nau-1}
  \Aue{Artalejo, J.\,R., and A.~G$\acute{\mbox{o}}$mez-Corral.} 2010. Markovian arrivals in 
stochastic modelling: A~survey and some new results. \textit{SORT~--- Stat. Oper. Res.~T.}  
34(2):101--144.
  \bibitem{43-nau-1}
  \Aue{Vishnevskiy, V.\,M., and A.\,N.~Dudin.} 2017. Queueing systems with correlated arrival flows 
and their applications to modeling telecommunication networks. \textit{Automat. Rem. Contr.} 
78(8):1361--1403.
  \bibitem{44-nau-1}
  \Aue{Basharin, G., V.~Naumov, and K.~Samouylov.} 2018. On Markovian modelling of arrival 
processes. \textit{Stat. Pap.} 59(4):1533--1540.
  \bibitem{45-nau-1}
  \Aue{Neuts, M.\,F.} 1981. \textit{Matrix-geometric solutions in stochastic models: An algorithmic 
approach.} Baltimore, MA: The John Hopkins University Press. 332~p.
  \bibitem{46-nau-1}
  \Aue{Latouche, G., and V.~Ramaswami.} 1999. \textit{Introduction to matrix analytic methods in 
stochastic modeling}. Philadelphia, PA: ASA \& SIAM. 334~p.
  \bibitem{47-nau-1}
  \Aue{Asmussen, S.} 2003. \textit{Applied probability and queues}. New  York, NY: Springer. 
438~p.
  \bibitem{48-nau-1}
  \Aue{Breuer, L., and D.~Baum.} 2005. \textit{An introduction to queueing theory and 
  matrix-analytic methods.} Dordrecht: Springer. 272~p.
  \bibitem{49-nau-1}
  \Aue{Bini, D.\,A., G.~Latouche, and B.~Meini.} 2005. \textit{Numerical methods for structured 
Markov chains}. New  York, NY: Oxford University Press. 336~p.
  \bibitem{50-nau-1}
  \Aue{Asmussen, S., and C.\,A.~O'Cinneide}. 2006. Matrix-exponential distributions. 
\textit{Encyclopedia of statistical sciences.} Eds. S.~Kotz, C.\,B.~Read, N.~Balakrishnan, 
B.~Vidakovic, and N.\,L.~Johnson. Hoboken, NJ: John Wiley \&~Sons. 3:1--5. doi: 10.1002/0471667196.ess1092.pub2.
  \bibitem{51-nau-1}
  \Aue{Li, Q.-L.} 2009. \textit{Constructive computation in stochastic models with applications}. 
Berlin: 
  Springer-Verlag. 650~p.
  \bibitem{52-nau-1}
  \Aue{Lipsky, L.} 2009. \textit{Queueing theory: A~linear algebraic approach}. 2nd ed. New York, 
NY: Springer. 548~p.
  \bibitem{53-nau-1}
  \Aue{Alfa, A.\,S.} 2010. \textit{Queueing theory for telecommunications}. New York, NY: 
Springer. 238 p.
  \bibitem{54-nau-1}
  \Aue{He, Q.-M.} 2014. \textit{Fundamentals of matrix-analytic methods.} New York, NY: 
Springer. 349 p.
  \bibitem{55-nau-1}
  \Aue{Buchholz, P., J.~Kriege, and I.~Felko.} 2014. \textit{Input modeling with phase-type 
distributions and Markov models. Theory and applications.} New York, NY: Springer. 127~p.
  \bibitem{56-nau-1}
  \Aue{Naumov, V.\,A., K.\,E.~Samuylov, and Yu.\,V.~Gaidamaka.} 2015. \textit{Mul'tiplikativnye 
resheniya konechnykh tsepey Markova} [Multiplicative solutions of finite Markov chains]. Moscow: 
RUDN. 159~p.
  \bibitem{57-nau-1}
  \Aue{Bladt, M., and B.\,F.~Nielsen.} 2017. \textit{Matrix-exponential distributions in applied 
probability}. Boston, MA: Springer. 736~p.
\end{thebibliography}

 }
 }

\end{multicols}

\vspace*{-3pt}

\hfill{\small\textit{Received July 2, 2020}}

%\pagebreak

  %\vspace*{-24pt}
  
  \Contr
  
  \noindent
  \textbf{Naumov Valeriy A.} (b.\ 1950)~--- Candidate of Science (PhD) in physics and mathematics, 
scientific director, Service Innovation Research Institute, 8A~Annankatu, Helsinki 00120, Finland; 
\mbox{valeriy.naumov@pfu.fi}
  
  \vspace*{3pt}
  
  \noindent
  \textbf{Samouylov Konstantin E.} (b.\ 1955)~--- Doctor of Science in technology, professor, Head of 
Department,  Peoples' Friendship 
University of Russia (RUDN University), 6~Miklukho-Maklaya Str., Moscow 117198, Russian 
Federation; senior scientist, Institute of Informatics Problems, Federal Research Center ``Computer 
Science and Control'' of the Russian Academy of Sciences, 44-2~Vavilov Str., Moscow 119333, Russian 
Federation; 
  \mbox{samuylov-ke@rudn.university}
  
\label{end\stat}

\renewcommand{\bibname}{\protect\rm Литература} 
  
            %2
\def\stat{popov}

\def\tit{АППРОКСИМАЦИЯ МНОЖЕСТВА РЕШЕНИЙ СИСТЕМ НЕЛИНЕЙНЫХ НЕРАВЕНСТВ 
С~ИСПОЛЬЗОВАНИЕМ ГРАФИЧЕСКИХ УСКОРИТЕЛЕЙ$^*$}

\def\titkol{Аппроксимация множества решений систем нелинейных неравенств 
с~использованием графических ускорителей}

\def\aut{М.\,В.~Попов$^1$, М.\,А.~Посыпкин$^2$}

\def\autkol{М.\,В.~Попов, М.\,А.~Посыпкин}

\titel{\tit}{\aut}{\autkol}{\titkol}

\index{Попов М.\,В.}
\index{Посыпкин М.\,А.}
\index{Popov M.\,V.}
\index{Posypkin M.\,A.}
 

{\renewcommand{\thefootnote}{\fnsymbol{footnote}} \footnotetext[1]
{Работа выполнена при частичной поддержке РНФ (проект 16-19-00148).}}


\renewcommand{\thefootnote}{\arabic{footnote}}
\footnotetext[1]{Федеральный исследовательский центр <<Информатика и~управление>> Российской академии наук, 
\mbox{alvopim@gmail.com}}
\footnotetext[2]{Федеральный исследовательский центр <<Информатика и~управление>> Российской академии наук, 
\mbox{mposypkin@gmail.com}}

\vspace*{-6pt}

\Abst{Существует множество задач, сводящихся к~решению системы неравенств. Точное 
получение множества решений в~подобных задачах не всегда возможно, из-за чего 
прибегают к~различным методам аппроксимации данного множества. При повышении 
точности аппроксимации искомого множества увеличивается объем необходимых 
вычислений и, соответственно, время работы алгоритмов. В~работе для увеличения 
быстродействия алгоритмов поиска аппроксимируемого множества применяются 
параллельные вычисления на графических ускорителях. Приводится описание и~реализация 
последовательного метода аппроксимации системы неравенств и~предлагается параллельный 
гибридный алгоритм, сочетающий перебор на равномерной сетке и~идеи метода ветвей 
и~границ. Этот алгоритм хорошо подходит для реализации на графических ускорителях и~не 
приводит к~избыточному перебору. Приведено сравнение эффективности работы 
последовательного и~двух вариантов параллельного алгоритмов на примере прикладной 
задачи аппроксимации рабочей области робота. Рабочая область состоит из множества 
возможных положений инструмента и~служит одной из ключевых характеристик робота.}

\KW{оптимизация; параллельные вычисления; графический ускоритель, GPU; CUDA; 
нелинейные неравенства}

\DOI{10.14357/19922264200303} 
 
%\vspace*{-6pt}


\vskip 10pt plus 9pt minus 6pt

\thispagestyle{headings}

\begin{multicols}{2}

\label{st\stat}

\section{Введение}

     В статье рассматривается задача построения внутренней и~внешней 
аппроксимаций множества, задаваемого системой нелинейных неравенств. Для 
решения этой задачи разработаны подходы~[1], основной недостаток которых 
заключается в~высокой трудоемкости. В~данной работе показано, что 
упомянутая проблема может быть преодолена с~помощью методов 
высокопроизводительных вычислений. 
     
     В качестве платформы для высокопроизводительных вычислений 
рассматриваются современные графические ускорители (GPU, graphics processing units)~[2]. 
Данный тип 
устройств получил широкое распространение в~последнее время благодаря 
наилучшему соотношению це\-на/про\-из\-во\-ди\-тель\-ность. В~то же время 
разработка программ для этой платформы является нетривиальной задачей 
в~силу особенностей ее архитектуры. Современные графические ускорители 
обладают большим числом ядер, но возможности взаимодействия между 
потоками существенно ограничены по отношению к~центральному про\-цес\-сору.
      
     В работе предлагается гибридный алгоритм, сочетающий перебор на 
равномерной сетке и~идеи метода ветвей и~границ. Предложенный алгоритм 
хорошо подходит для реализации на графических ускорителях и~не приводит 
к~избыточному пе\-ре\-бору. 
     
\section{Последовательный алгоритм}

     Рассматривается задача нахождения множества решений системы 
неравенств вида:
     \begin{equation}
     \left.
     \begin{array}{rl}
     f_j(x)\leq 0\,, & j\in [1,m]\,;\\[6pt]
     a_i\leq x_i\leq b_i\,, & i\in [1,n]\,.
     \end{array}
     \right\}
     \label{e1.1-pos}
     \end{equation}
      
     Для поиска решения необходимо задать изначальные границы, внутри 
которых будет находиться искомая область~--- $n$-мер\-ный параллелепипед 
$X\hm\in R^n$. Далее задается точность аппроксимации~$d$ получаемого 
решения, представляющая собой минимальный размер аппроксимирующего 
параллелепипеда. Под размером будем понимать длину максимального 
измерения. После этого параллелепипед~${X}$ разбивается на 
меньшие параллелепипеды~$P_k$ равномерной сеткой. Число точек сетки по 
измерению~$j$  задается формулой:
 $$
 k_j= \left\lceil \fr{b_j- a_j}{d}\right\rceil\,.
 $$
 
  Для 
каждого из полученных параллелепипедов проверяется выполнение следующих 
условий:
     \begin{equation}
     \max\limits_{j\in [1,m]} \max\limits_{x\in P_k} f_j(x)<0\,;
     \label{e1.2-pos}
     \end{equation}
\begin{equation}
\max\limits_{j\in [1,m]} \min\limits_{x\in P_k} f_{x\in P_k}(x)>0\,.
\label{e1.3-pos}
\end{equation}

Можно выделить три основных случая:
\begin{enumerate}[(1)]
\item выполнено условие~(\ref{e1.2-pos})~--- параллелепипеды полностью 
лежат внутри искомой области;
\item выполнено условие~(\ref{e1.3-pos})~--- параллелепипед лежит вне 
искомой области;
\item оба условия не выполнены~--- параллелепипед классифицируется как 
<<граничный>>.
\end{enumerate}

     Метод перебора на равномерной сетке удобен тем, что можно заранее 
рассчитать число операций, необходимых для получения аппроксимирующего 
покрытия фигуры, образованной функциями из~(\ref{e1.1-pos}). 
     
     Для последовательного алгоритма перебора на равномерной сетке число 
рассматриваемых параллелепипедов определяется по формуле 
$\prod\nolimits^m_{j=1} k_j$. Для каждого параллелепипеда $P_k\hm\in X$ 
проводится проверка выполнения условий~(\ref{e1.2-pos}) и~(\ref{e1.3-pos}) для 
каждой функции~$f_j$. Время проверки~$t_c$ не отличается для разных 
параллелепипедов. Таким образом, общее время составляет $t_c  
\prod\nolimits^m_{j=1} k_j$. 
      
     Недостаток данного алгоритма состоит в~необходимости полного 
перебора всех параллелепипедов, число которых существенно возрастает 
с~уменьшением значения точности аппроксимации. Для преодоления 
указанного недостатка может применяться метод <<ветвей и~границ>>, 
адаптированный для данной задачи. Отличие заключается в~том, что сетка не 
задается сразу, а параллелепипеды формируются рекурсивно. Приведем 
описание алгоритма.
     \begin{enumerate}[I.]
\item В текущий список помещается $m$-мерный параллелепипед $X\hm\in 
R^m$, который полностью содержит в~себе искомую область.
\item Из текущего списка извлекается параллелепипед и~делится на две части 
по наибольшему из измерений.
\item Выполняется проверка условий~(2) и~(3) для полученных делением 
параллелепипедов и~в~за\-ви\-си\-мости от результата проверки проводится одно 
из следующих действий:
\begin{enumerate}[(1)]
\item если выполнено условие~(\ref{e1.2-pos}), то параллелепипед 
помещается в~список полученных прямоугольников для последующего 
отоб\-ра\-же\-ния;
 \item если выполнено условие~(\ref{e1.3-pos}), то параллелепипед 
далее не рассматривается;
\item если параллелепипед оказывается граничным, но его максимальное 
измерение превосходит величину точности аппрокси\-мации, то он помещается 
в~текущий список и~выполняется переход к~шагу~2. В~противном случае 
параллелепипед добавляется к~множеству граничных параллелепипедов.
 \end{enumerate}
 \end{enumerate}
     
     Данный метод не позволяет заранее рассчитать число получаемых 
параллелепипедов для покрытия искомой области, но, как показывает 
эксперимент, оно будет много меньше, чем при методе равномерной сетки. 
Минус данного метода~--- его рекурсивность, из-за чего данный алгоритм 
сложен для распараллеливания. 

\section{Параллельный алгоритм}

     Реализация метода равномерной сетки для параллельных вычислений на 
графическом ускорителе позволяет сократить число выполняемых операций 
и~время вычислений в~десятки раз. Главное\linebreak отличие параллельного алгоритма 
от последовательного состоит в~том, что обработка параллелепипедов 
выполняется одновременно разными потоками графической карты. 
     
     Алгоритм равномерной сетки (рис.~1) для параллельной аппроксимации 
множества решений системы неравенств состоит в~разбиении параллелепипеда, 
содержащего исследуемую область, на параллелепипеды меньшего размера 
с~их последующей параллельной обработкой. Параллельная\linebreak\vspace*{-12pt}

{ \begin{center}  %fig1
 \vspace*{12pt}
    \mbox{%
 \epsfxsize=79mm 
 \epsfbox{pos-1.eps}
 }

\vspace*{6pt}

\noindent
{{\figurename~1}\ \ \small{
Алгоритм равномерной сетки
}}
\end{center}}

%\vspace*{6pt}

{ \begin{center}  %fig2
 \vspace*{-2pt}
    \mbox{%
 \epsfxsize=79mm 
 \epsfbox{pos-2.eps}
 }

\vspace*{6pt}

\noindent
{{\figurename~2}\ \ \small{
Алгоритм двойной сетки
}}
\end{center}}


\vspace*{6pt}


\noindent
 обработка 
выполняется на графическом ускорителе~--- один поток обрабатывает 
несколько ячеек сетки. При этом координаты ячеек не пересылаются на GPU, 
а~генерируются исходя из номера потока.
     
      

     
     Алгоритм равномерной сетки идеально подходит для реализации на 
графических ускорителях, так как обработка ячеек равномерной сетки может 
выполняться абсолютно независимо. Независимость потоков создает 
предпосылки для массивного\linebreak параллелизма без синхронизации. Недостаток 
данного алгоритма~--- высокая ресурсоемкость: при\linebreak увеличении точности 
аппроксимации количество обрабатываемых параллелепипедов существенно 
возрастает и~превышает пределы графической памяти. 
     
     Преодолеть указанный недостаток позволяет алгоритм двойной сетки 
(рис.~2), сочетающий элементы метода ветвей и~границ и~перебора на 
равномерной сетке. Сначала на центральном процессоре (CPU,
central processing unit) проводится 
предварительная обработка ячеек крупной равномерной сетки. В~результате 
часть таких параллелепипедов классифицируется как внешние или внутренние, 
сохраняется и~далее не рассматривается. Граничные параллелепипеды 
передаются графическому ускорителю по одному, где они обрабатываются так 
же, как исходный параллелепипед в~алгоритме равномерной сетки.
     

     
     Результатом обработки равномерной сетки на GPU служит массив, 
содержащий значения~0, 1, 2 в~зависимости от принадлежности 
параллелепипеда к~искомому множеству аппроксимации. Далее в~зависимости 
от индекса элемента массива для него вычисляются координаты 
параллелепипеда.
     
     Для графического отображения полученного множества используется 
язык Python и~библиотека Matplotlib. При большом числе отображаемых 
параллелепипедов сильно возрастает время вывода изображения 
и~задействуется большой объем оперативной памяти. Для оптимизации 
процесса было принято решение объединять параллелепипеды, лежащие 
внутри искомого множества, в~крупные. Таким образом удалось существенно 
уменьшить затраты времени и~оперативной памяти. 

\section{Реализация и~экспериментальные исследования}

     Предложенные методы были реализованы с~помощью языка~C 
и~технологии CUDA (Compute Unified Device Architecture). 
В~качестве примера прикладного применения 
рассмотренных методов выбрана задача аппроксимации рабочей области 
(множества всех возможных положений рабочего инструмента) плоского 
параллельного робота. Роботы параллельной структуры характеризуются 
замкнутой кинематической цепью, сочетают высокую точность 
позиционирования и~жесткость конструкции~\cite{3-pos}. Для экспериментов 
выбран планарный робот с~двумя степенями свободы (\mbox{2-RPR}). Схематично 
данное устройство представлено на рис.~3. Рабочий инструмент робота, 
закрепленный в~точке~$X$, приводится в~движение двумя призматическими 
двигателями, которые изменяют длины штанг~$AX$ и~$BX$.



Рабочая область данного робота определяется следующей системой неравенств:
\begin{align*}
f_1\left( x_1, x_2\right) &=x_1^2+x_2^2-\left( l_1^{\max}\right)^2\leq0\,;\\
f_2\left(x_1, x_2\right)&= \left( l_1^{\min}\right)^2 -x_1^2-x_2^2\leq0\,;\\
f_3\left( x_1, x_2\right)&= \left(x_1-l_0\right)^2 +x_2^2-\left( 
l_2^{\max}\right)^2\leq0\,;\\
f_4\left(x_1, x_2\right)& = \left( l_2^{\min}\right)^2 -\left( x_1-l_0\right)^2-x_2^2\leq 
0\,,
\end{align*}
где
$$
-l_1^{\max}\leq x_1 \leq l_0+l_2^{\max}\,;\
0\leq x_2\leq \min\left( l_1^{\max}, l_2^{\max}\right)\,.
$$

\setcounter{figure}{3}
\begin{figure*}[b] %fig4
\vspace*{1pt}
 \begin{center}
 \mbox{%
 \epsfxsize=162.252mm 
 \epsfbox{pos-4.eps}
 }
 \end{center}
   \vspace*{-9pt}
\Caption{Искомые множества: (\textit{а})~точ\-ность $d\hm=0{,}1$;
(\textit{б})~точность $d\hm=0{,}01$}

\vspace*{6pt}

{\small
\begin{center}
\begin{tabular}{|c|c|c|c|c|}
\multicolumn{5}{c}{Время выполнения методов в~зависимости от точности аппроксимации 
(мс)}\\
\multicolumn{5}{c}{\ }\\[-6pt]
\hline
\tabcolsep=0pt\begin{tabular}{c}Точность\\ аппроксимации\end{tabular}&
\tabcolsep=0pt\begin{tabular}{c}Последовательный\\ метод равномерной\\ сетки\end{tabular}&
\tabcolsep=0pt\begin{tabular}{c}Метод\\ ветвей\\ 
и~границ\end{tabular}&
\tabcolsep=0pt\begin{tabular}{c}Параллельный\\ метод равномерной\\ сетки
\end{tabular}&
\tabcolsep=0pt\begin{tabular}{c}Параллельный\\ метод двойной\\ сетки\end{tabular}\\
\hline
0,1\hphantom{99}&\hphantom{9}13&\hphantom{99}8&47&54\\
0,05\hphantom{9}&\hphantom{9}41&\hphantom{9}27&47&54\\
0,01\hphantom{9}&945&889&92&54\\
0,005&4\,003\hphantom{9}&3\,950\hphantom{9}&140\hphantom{9}&54\\
0,001&99\,100\hphantom{99}&132\,300\hphantom{999}&510\hphantom{9}&102\hphantom{9}\\
\hphantom{9}0,0001&$>$2,5~ч&$>$2,5~ч&---&4\,400\hphantom{99}\\
\hline
\end{tabular}
\end{center}
}
%\end{table*}
\end{figure*}


{ \begin{center}  %fig3
 \vspace*{-2pt}
    \mbox{%
 \epsfxsize=79mm 
 \epsfbox{pos-3.eps}
 }

\vspace*{6pt}

\noindent
{{\figurename~3}\ \ \small{
Схема плоского робота 2-RPR
}}
\end{center}}


\vspace*{6pt}


\noindent
Значения $l_{1,2}$ изменяются в~диапазоне $8\hm\leq l_{1,2}\hm\leq 12$, т.\,е.\ 
$l_1^{\min}\hm= 8$; $l_1^{\max}\hm=12$. Расстояние между точками 
закрепления штанг составляет $l_0\hm=5$. Требуется аппроксимировать 
область, в~которой, исходя из конструктивных особенностей длин звеньев, 
может находиться точка $X(x_1,x_2)$ закрепления рабочего инструмента. Для 
получения гарантированных оценок минимума и~максимума функций в~левых 
частях неравенств применялись методы интервального  
анализа~\cite{4-pos, 5-pos}.
     
     Реализованные методы тестировались на персональном компьютере 
с~Intel Core~I7 с~базовой тактовой частотой 2,8~ГГц, 16~ГБ ОЗУ, видеокартой 
Nvidia GeForce~1060: 3~ГБ оперативной памяти, 1280~ядер с~базовой тактовой 
частотой от 1506~МГц. Тестирование проводилось для поставленной задачи со 
значениями точности аппроксимации, равными 0,1; 0,05; 0,01; 0,005; 0,001 
и~0,0001.
     
     Аппроксимации рабочей области для разных значений точности, 
полученные методом двойной сетки, приведены на рис.~4.




На рис.~4,\,\textit{б} убраны линии, выделяющие получаемые параллелепипеды, 
поскольку при высокой точности аппроксимации толщина отображаемых 
линий становится соизмерима с~высотой параллелепипедов, из-за чего линии 
сливаются. Таким образом, для точности $d\hm=0{,}01$ высота 
параллелепипедов, которые не объединены друг с~другом, слишком мала для 
графического отображения на целой фигуре. 
     
Время работы алгоритмов в~миллисекундах представлено в~таблице.




Проведенные эксперименты позволяют сделать следующие выводы.
\begin{enumerate}[1.]
\item При низкой точности аппроксимации (0,1 или~0,05) параллельные 
варианты работают с~невысокой эффективностью, что означает 
недостаточную загруженность вычислительных мощностей видеокарты. 
Применение GPU при таком объеме вычислений нецелесообразно.
\item При повышении точности возрастает объем вычислений 
и~загруженность ядер графической карты, что приводит к~увеличению 
эффективности расчетов. С~по\-мощью параллельного алгоритма 
равномерной сетки не удается получать решения при высокой точ\-ности 
аппроксимации в~силу превышения максимально допустимых значений 
числа обрабатываемых потоков или пределов локальной памяти. 
\item Принцип дополнительного разбиения исходной области, заложенный 
в~параллельном алгоритме двойной сетки, позволяет снизить объем 
вычислений и~данных, обрабатываемых на графической карте, благодаря 
чему возможно получение более высокой точности аппроксимации.
\end{enumerate}

\section{Заключение}
     
     В данной работе разработаны два параллельных алгоритма для численной 
аппроксимации решения системы нелинейных неравенств, ориентированные на 
графические ускорители: алгоритм равномерной сетки и~алгоритм двойной 
сетки. Результаты экспериментов показали, что параллельные методы 
с~вычислениями на графическом ускорителе оказались существенно 
эффективнее последовательных однопоточных методов на центральном 
процессоре. Параллельный алгоритм двойной сетки оказался наиболее 
эффективным по скорости работы и~наилучшим по точности аппроксимации. 
Этот метод сочетает в~себе аппроксимацию на равномерной сетке, которая 
хорошо распараллеливается, и~метод ветвей и~границ, позволяющий сократить 
число операций за счет эффективного отсева <<неперспективных>> областей 
на начальном этапе.
     
     В~дальнейшем планируется рассмотреть роботов более сложной 
конструкции~\cite{6-pos}, что неизбежно приведет к~необходимости 
привлечения большей вычислительной мощности для расчетов. Предполагается 
использование серверов, оснащенных несколькими графическими 
ускорителями, установленных в~центре обработки данных ФИЦ ИУ 
РАН~\cite{7-pos}.


{\small\frenchspacing
 {%\baselineskip=10.8pt
 \addcontentsline{toc}{section}{References}
 \begin{thebibliography}{9}
  \bibitem{1-pos}
  \Au{Evtushenko Yu., Posypkin~M., Rybak~L., Turkin~A.} Approximating a~solution set of 
nonlinear inequalities~// J.~Global Optim., 2018. Vol.~71. Iss.~1. P.~129--145. 
  \bibitem{2-pos}
  \Au{Cheng J., Grossman~M., McKercher~T.} Professional Cuda~C programming.~--- New 
York, NY, USA: John Wiley \& Sons, 2014. 499~p.
  \bibitem{3-pos}
  \Au{Merlet J.\,P.} Parallel robots.~--- Solid mechanics and its  
applications ser.~--- Springer Science \& Business Media, 2006. Vol.~128. 402~p.
  \bibitem{4-pos}
  \Au{Moore R.\,E., Bierbaum~F.} 
  Methods and applications of interval analysis.~---  SIAM studies 
in applied and numerical mathematics ser.~--- Soc. for 
Industrial \& Applied Math., 1979. 201~p.
  \bibitem{5-pos}
  \Au{Hansen~E., Walster~G.\,W.} Global optimization using interval 
  analysis: Revised and 
expanded.~--- Pure and applied mathematics ser.~--- CRC Press, 2003. 
Book~264. 728~p.
  \bibitem{6-pos}
  \Au{Malyshev D., Rybak~L., Behera~L., Mohan~S.} Workspace modelling of a~parallel robot 
with relative manipulation mechanisms based on optimization methods~// 
Robotics and mechatronics~/ Eds. C.\,H.~Kuo, P.\,C.~Lin, T.~Essomba, 
G.\,C.~Chen.~--- Cham: Springer, 2019. Vol.~78. P.~151--163.
  \bibitem{7-pos}
  \Au{Zatsarinny A.\,A., Gorshenin~A.\,K., Kondrashev~V.\,A., Volovich~K.\,I., Denisov~S.\,A.} 
Toward high performance solutions as services of research digital platform~// 
Procedia Comput. Sci., 2019. Vol.~150. P.~622--627.
\end{thebibliography}

 }
 }

\end{multicols}

\vspace*{-6pt}

\hfill{\small\textit{Поступила в~редакцию 08.10.19}}

\vspace*{8pt}

%\pagebreak

%\newpage

%\vspace*{-28pt}

\hrule

\vspace*{2pt}

\hrule

%\vspace*{-2pt}

\def\tit{APPROXIMATION OF~THE~SET OF~SOLUTIONS OF~SYSTEMS OF~NONLINEAR 
INEQUALITIES USING~GRAPHIC ACCELERATORS}


\def\titkol{Approximation of~the~set of~solutions of~systems of~nonlinear 
inequalities using~graphic accelerators}



\def\aut{M.\,V.~Popov and M.\,A.~Posypkin}

\def\autkol{M.\,V.~Popov and M.\,A.~Posypkin}

\titel{\tit}{\aut}{\autkol}{\titkol}

\vspace*{-9pt}


\noindent
Federal Research Center ``Computer Science and Control'' of the Russian Academy of Sciences, 
44-2~Vavilov Str., Moscow 119133, Russian Federation


\def\leftfootline{\small{\textbf{\thepage}
\hfill INFORMATIKA I EE PRIMENENIYA~--- INFORMATICS AND
APPLICATIONS\ \ \ 2020\ \ \ volume~14\ \ \ issue\ 3}
}%
 \def\rightfootline{\small{INFORMATIKA I EE PRIMENENIYA~---
INFORMATICS AND APPLICATIONS\ \ \ 2020\ \ \ volume~14\ \ \ issue\ 3
\hfill \textbf{\thepage}}}

\vspace*{3pt} 



\Abste{Solutions of certain problems can be reduced to the solution 
of some systems of inequalities. But the computation of the set of 
exact solutions may not be feasible. Thus, various methods for 
approximation of the solution set have been developed. The more 
accurate approximation is required, the bigger number of 
calculations must be performed and, consequently, the runtime of 
the algorithms increases. Nowadays, it is common to speed up algorithms 
by paralleling computations on graphics accelerators. The paper 
describes the serial method for approximation of the solution of systems 
of inequalities and proposes the parallel hybrid algorithm that combines 
iterations on the uniform grid and the branch and bound method. This 
algorithm is suited for direct implementation on graphics accelerators 
and does not suffer from the excessive enumeration of possible solution 
candidates. The\linebreak\vspace*{-12pt}}

\Abstend{sequential algorithm and the two versions of the 
parallel algorithm are compared through one example: the problem of 
approximation of the working area of the robot which consists of 
the set of robot's tool positions and is the key robot's characteristic.}

\KWE{optimization; parallel computing; graphics accelerator, GPU; CUDA; 
nonlinear inequalities}



\DOI{10.14357/19922264200303} 

%\vspace*{-20pt}

\Ack
\noindent
The paper was partially supported by the Russian Science Foundation  (project  16-19-00148).

%\vspace*{6pt}

 \begin{multicols}{2}

\renewcommand{\bibname}{\protect\rmfamily References}
%\renewcommand{\bibname}{\large\protect\rm References}

{\small\frenchspacing
 {%\baselineskip=10.8pt
 \addcontentsline{toc}{section}{References}
 \begin{thebibliography}{9}
\bibitem{1-pos-1}
\Aue{Evtushenko, Yu.\,G., M.\,A.~Posypkin, L.\,A.~Rybak, and A.\,V.~Turkin.} 2018. Approximating 
a~solution set of nonlinear inequalities. \textit{J.~Global Optim.} 71(1):129--145. 
\bibitem{2-pos-1}
\Aue{Cheng, J., M.~Grossman, and T.~McKercher.} 2014. \textit{Professional Cuda~C programming}. 
New York, NY: John Wiley \&~Sons. 499~p.
\bibitem{3-pos-1}
\Aue{Merlet, J.\,P.} 2006. \textit{Parallel robots}. 
Solid mechanics and its  
applications ser.
Springer Science \&~Business Media. Vol.~128. 402~p.
\bibitem{4-pos-1}
\Aue{Moore, R.\,E., and  F.~Bierbaum.} 1979. 
\textit{Methods and applications of interval analysis}.
 SIAM studies in applied and numerical mathematics ser.
{Soc. for Industrial \&~Applied Math.} 201~p.
\bibitem{5-pos-1}
\Aue{Hansen, E., and G.\,W.~Walster.} 2003. 
\textit{Global optimization using interval analysis: Revised 
and expanded.} Pure and applied mathematics ser.
CRC Press. Book~264. 728~p.
\bibitem{6-pos-1}
\Aue{Malyshev, D., L.\,A.~Rybak, L.~Behera, and S.~Mohan.} 2019. Workspace modelling of a parallel 
robot with relative manipulation mechanisms based on optimization methods. 
\textit{Robotics and 
mechatronics}. Eds. C.\,H.~Kuo, P.\,C.~Lin, T.~Essomba, 
and G.\,C.~Chen. Cham: Springer. 78:151--163.
\bibitem{7-pos-1}
\Aue{Zatsarinny, A.\,A., A.\,K.~Gorshenin, V.\,A.~Kondrashev, K.\,I.~Volovich, and S.\,A.~Denisov.} 
2019. Toward high performance solutions as services of research digital
 platform. \textit{Procedia Comput. 
Sci.} 150:622--627.
\end{thebibliography}

 }
 }

\end{multicols}

\vspace*{-6pt}

\hfill{\small\textit{Received October 8, 2019}}

%\pagebreak

%\vspace*{-24pt}

\Contr

\noindent
\textbf{Popov Mikhail V.} (b.\ 1995)~--- PhD student, Federal Research Center 
``Computer Science and Control'' of the Russian Academy of Sciences, 44-2~Vavilov Str., Moscow 119333, 
Russian Federation; \mbox{alvopim@gmail.com}

\vspace*{3pt}

\noindent
\textbf{Posypkin Mikhail A.} (b.\ 1974)~--- 
Doctor of Science in physics and mathematics, associate professor, principal scientist, 
Federal Research Center ``Computer Science 
and Control'' of the Russian Academy of Sciences, 44-2~Vavilov 
Str., Moscow 119333, Russian Federation; \mbox{mposypkin@gmail.com}

\label{end\stat}

\renewcommand{\bibname}{\protect\rm Литература}         %3

%\renewcommand{\figurename}{\protect\bf Figure}
\renewcommand{\tablename}{\protect\bf Table}

\def\stat{razum}


\def\tit{COMPARISON OF TWO ACTIVE QUEUE MANAGEMENT SCHEMES THROUGH THE~$M/D/1/N$ 
QUEUE}

\def\titkol{Comparison of two active queue management schemes through the $M/D/1/N$ 
queue}

\def\autkol{M.\,G.~Konovalov and R.\,V.~Razumchik}

\def\aut{M.\,G.~Konovalov$^1$ and R.\,V.~Razumchik$^2$}

\titel{\tit}{\aut}{\autkol}{\titkol}

%{\renewcommand{\thefootnote}{\fnsymbol{footnote}}
%\footnotetext[1] {The 
%research of Yuri Kabanov was done under partial financial support   of the grant 
%of  RSF No.\,14-49-00079.}}

\renewcommand{\thefootnote}{\arabic{footnote}}
\footnotetext[1]{Institute of Informatics Problems, Federal Research Center ``Computer Science and Control'' of the Russian Academy of Sciences,
44/2~Vavilov Str., Moscow 119333, Russian Federation, \mbox{mkonovalov@ipiran.ru}}
\footnotetext[2]{Institute of Informatics Problems, Federal Research Center 
``Computer Science and Control'' of the Russian Academy of Sciences,
44/2~Vavilov Str., Moscow 119333, Russian Federation; 
Peoples' Friendship University of Russia (RUDN University),
6~Miklukho-Maklaya Str., Moscow 117198, Russian 
Federation; 
\mbox{rrazumchik@ipiran.ru} %\mbox{razumchik\_rv@rudn.ru
}


\index{Konovalov M.\,G.}
\index{Razumchik R.\,V.}
\index{Коновалов М.\,Г.}
\index{Разумчик Р.\,В.}

\def\leftfootline{\small{\textbf{\thepage}
\hfill INFORMATIKA I EE PRIMENENIYA~--- INFORMATICS AND
APPLICATIONS\ \ \ 2018\ \ \ volume~12\ \ \ issue\ 4}
}%
 \def\rightfootline{\small{INFORMATIKA I EE PRIMENENIYA~---
INFORMATICS AND APPLICATIONS\ \ \ 2018\ \ \ volume~12\ \ \ issue\ 4
\hfill \textbf{\thepage}}}



\Abste{The paper focuses on giving the first in the literature numerical evidence
that the stationary performance characteristics of single-server queues
with the general renovation mechanism may be as good as of single-server queues
with the RED-type active queue management mechanisms
(AQM). Comparison is made in the queueing
theory context: the basic model is the $M/D/1/N$ queue. 
The characteristics reported are: the loss ratio, average system size, and 
average number of consecutive losses along 
with the standard deviations. Numerical results are based on the 
well-known facts and some new analytic results, presented in the paper.}

\KWE{queueing system; active queue management; RED; renovation}

\DOI{10.14357/19922264180402}


\vspace*{1pt}


\vskip 12pt plus 9pt minus 6pt

      \thispagestyle{myheadings}

      \begin{multicols}{2}

                  \label{st\stat}


\section{Introduction}

\noindent
A large number of AQM mechanisms have been developed
up to nowadays and quite a~lot of efforts have been devoted to the studies of
their efficiency. 
These mechanisms may be applicable in different contexts but historically, 
they are more often related to communication networks
in the context of mitigation of congestion and congestion avoidance.
This problem, as highlighted in the latest RFC~7567~\cite{RFC7567},
still does not have a~satisfying solution. 
An AQM mechanism is an advanced rejection discipline, 
when an arriving customer (packet, job, etc.) is lost randomly with a~probability 
that may depend on the (current, past, average, etc.) system state or performance.
The most popular class of AQM mechanisms seems to be the Random Early Detection (RED) and
its ramifications like GRED (Gentle RED), REM 
(Random Exponential Marking),
etc.\ (a~recent survey on the AQM can be found in~\cite{Adams}).
The goals of AQM are usually diverse and conflicting: 
prevent queues from growing too long, maintain high server (processor) utilization
and low variance of the queue size, ensure fairness among competing flows, 
and others. These are discussed in detail in~\cite{RFC7567} in the context
of communications network but most of the goals are applicable in other contexts as well
(buffer-bloat problems in data-center, etc.).

Besides simulation, analytic performance evaluation of systems with AQM is quite often
carried out in the queueing theory context (see,
for example,~\cite{Bonald,Chyd,Chyd2,oleg,hao,konnew} and references therein). 
Usually, the system with an AQM mechanism is modeled as a~queueing system or network
and then its performance characteristics are studied using known analytic techniques. 
Throughout the paper, we stay within the queueing theory context.

In the series of recent papers~\cite{Kreinin,Zaryadov2010,zarN1,zarN2,Zaryadov2009},
the authors have proposed the new type of AQM mechanism which they call 
\textit{renovation}. 
Roughly speaking, renovation implies that each customer, 
having received service, may remove some additional work from the system
(i.\,e., may renovate it). We will make this definition more precise in the next 
section 
but for now, note 
that queue management \textit{after service completions} is what makes the renovation
 different 
from the most known AQM schemes\footnote[3]{Indeed, renovation and most of the known AQM 
mechanisms
are conceptually different. One of the main goals of AQM mechanisms is to prevent 
queue from growing too large
leaving space for potential new arrivals.
In systems with renovation, the queue can become full (meaning that fewer customers are 
lost)
but after a~service completion, several customers may be removed from it. In this way, 
the content of the queue
can be preserved at a~certain average level but the loss pattern becomes intricate.}, 
in which the decisions are made \textit{upon arrivals}.  
To our best knowledge, there are no studies, 
which tell whether the performance of the systems with renovation is
better/same/worse than that of the same systems but with the implemented AQM mechanisms.
Thus, there is a~lack of bridge between available theoretical results for renovation and 
its practical perspective.

The scope of this paper is to give the first in the literature numerical evidence 
that the stationary performance of 
single-server queueing systems with the implemented renovation mechanism can
be as good as of 
the same single-server queues but the well-known packed dropping procedures like RED.
The emphasis is primarily on the reporting of this finding, complemented 
with some new insights into 
queueing systems with renovation. The relation to other 
AQM mechanisms like CoDel~\cite{RFC8289} is not discussed here. 
Moreover, in the numerical experiments presented here, 
we did not use any benchmarks to generate the traffic profiles 
but used the theoretical distributions instead.

The main stationary performance characteristics reported are: the loss ratio, 
the average number in the system
(average system size), and the average number of consecutive losses along with 
their standard deviations.
After introducing the renovation mechanism and the analytic setting,
in which  renovation mechanism is compared with RED, we give the new analytic results 
for computing system size moments and the loss ratio under the renovation mechanism.
The results presented in the numerical section are based on the analytic results. 
Monte-Carlo simulation is used only for the average (and standard deviation) 
number of consecutive losses in the system with renovation. 


\section{Settings and the Model}

\noindent
We follow the queueing theoretic approach and as the basic model, we use $M/D/1/N$ queue,
i.\,e., queue of finite capacity~$N$ fed at rate~$\lambda$ by a~Poisson flow of customers,
which are served on a~first-come-first-served basis  by a~single server with constant
service time $d>0$.
We assume that the system is in the steady state.
When an arriving customer sees that the queue is full,
it is lost. If no other type of losses occur in the system,
we say that the Tail Drop mechanism is implemented in it.

If an arriving customer is lost with probability~$d_n$
where~$n$ is the total number of customers 
it sees in the system on arrival, then we say that an AQM mechanism 
is implemented in the system. Various dropping functions can be obtained
by specifying the values of~$d_n$
(see, for example, RED dropping function in~\cite[Example 1]{Chyd}).
Important notice should be made here. In practice, RED-type mechanisms 
may use moving averages of the queue size instead of its instantaneous value. 
Thus, the way~$d_n$ introduced above is a~simplified way of thinking.
Yet, this trade-off is important because it allows to keep the mathematical 
models of RED-type AQM tractable.
Luckily, as noticed in~\cite[Section II.C]{Bonald}, such approximation may not
lead to significant bias, when the weight of the moving average scheme is small
(which is claimed to be the case sometimes in practice).

The renovation mechanism, which is implemented in a~system with Tail Drop,
works as follows. Define $N+1$ numbers, say, $q_i\ge 0$,
$0 \le i \le N$, satisfying \mbox{$\sum\nolimits_{i=0}^N q_i=1$}.
If upon service completion there are $i$, $1 \le i \le N$, customers
waiting in the queue, then the served customer leaves the system and
\begin{itemize}
\item with probability $q_0+Q_i$ nothing else happens, where 
$Q_i=q_i+q_{i+1}+\dots+ q_N$; and
\item with probability $q_j$, $0<j<i$, exactly $j$ customers
from the queue leave the system and those customers 
are chosen successively \textit{starting from the head of the queue}.
\end{itemize}
%\noindent
The served customer, which sees the empty queue, leaves the system.
Thus, after the renovation (if it happened), the system never becomes empty.
%what is appealing from the practical point of view. 

In the numerical section, we rank the systems with RED 
and renovation according to the stationary loss ratio, average system size,  
and average number of consecutive losses along with their standard deviations. 
The system with the Tail Drop is the standard $M/D/1/N$ queue, 
for which all these performance characteristics follow
from the classical results in queueing theory (see, for example,~\cite{Riordan1962}).
Analytic results for the systems of $M/G/1/N$ type with relatively 
arbitrary dropping functions are given in~\cite{Chyd}.
Yet, for the system with renovation, we need to derive these 
performance characteristics anew, since 
the available results in~\cite{Zaryadov2010,Zaryadov2009} 
are not valid for the renovation mechanism introduced above.
We briefly sketch the derivations in the next
section and omit most of the details
since they are based on the methodology, 
developed in~\cite{Zaryadov2010,Zaryadov2009},
and reviewed in~\cite{arxivRK}.

%Note that the above mentioned performance characteristics 
%do not depend on the order in which the customers
%are removed from the system; yet in the derivations we assume that the customers
%are chosen successively \textit{starting from the head of the queue}.

%The analytic results and parameters' values for 
%RED and REM are due to \cite{Chyd}.

\section{Performance Characteristics}

\noindent
Consider the $M/D/1/N$ queue with the renovation mechanism
introduced above. Since a~customer is served for constant service time
$d$, then for the cumulative distribution function $B(x)$ 
of its service time, one has: 
$$
B(x)=
\begin{cases}
0 & \mbox{if } x \le d\,;\\
1 & \mbox{if } x>d\,.
\end{cases}
$$
Let $N(t)$ be the total number of customers %\footnote{We assume that the system 
%starts empty, i.\,e., $N(0)=0$.} 
in the system at instant $t$ 
and $E(t)$ be the elapsed service time of the customer in server
(if there is one). 
In order to compute the stationary system size moments, 
one needs to know the stationary distribution:
\begin{equation*}
%\label{pn}
P_n=\lim\limits_{t \rightarrow \infty} \mathbf{P}\{ N(t)=n \},\enskip  0 \le n \le N+1\,.
\end{equation*} 
For the computation of the loss ratio,
due to the \mbox{PASTA} (Poisson Arrivals See Time Averages) 
property, it is sufficient to know
 the stationary probability densities
$p_n(x)=P'_n(x)$ where
\begin{multline*}
%\label{pnx}
P_n(x)=\lim\limits_{t \rightarrow \infty} \mathbf{P}\{ N(t)=n, E(t)<x \}, \\ 
1\le n \le N, \ x \in [0,d]\,.
\end{multline*}
Since we are dealing with the finite-capacity queue 
and work conserving service discipline, the
introduced stationary distributions exist.  
The probabilities~$P_n$ and 
the densities~$p_n(x)$ can be computed as follows. 
Let~$t_n$ denotes the $n$th service completion epoch 
and $N_n=N(t_n+0)$ denotes the total number of customers in the system. 
Clearly, $\{ N_n, \ n \ge 1\}$ is the finite-state Markov chain.
The entries of the transition probability matrix $\mathbb{P}=(p_{ij})$
of this chain have the form:
$$
p_{0j}=p_{1j}=
\begin{cases}
\beta_0, & \hspace*{-20mm}j=0;\\
\beta_j Q_j + \displaystyle\sum\limits_{k=j}^N \beta_k q_{k-j} +  B_N q_{N-j}, &\\
&\hspace*{-20mm} 1 \le j \le N-1\,;\\
(q_0 + q_N) B_{N-1}, & \hspace*{-20mm}j=N\,;
\end{cases}
$$
\begin{multline*}
\!\!p_{ij}=
\begin{cases}
0, & \hspace*{-38mm}j=0;\\
\sum\limits_{k=j-1}^{N-1} \beta_k q_{k-j+1} +  B_{N-1} q_{N-j}, & \\
& \hspace*{-38mm}1 \le j \le i-2\,;\\
\beta_{j-i+1} Q_j + \displaystyle\!\sum\limits_{k=j-1}^{N-1}\!\! \beta_k q_{k-j+1} + 
 B_{N-1} q_{N-j}, &\\
 &\hspace*{-38mm} i-1 \le j \le N-1;\\
(q_{0} +  q_{N})B_{N-i} , &\hspace*{-38mm} j=N\,,
\end{cases}
\\
 2 \le i \le N\,.
\end{multline*}
Here, $B_0=1-\beta_0$; $B_k=B_{k-1}-\beta_k$; and 
$\beta_k=[{(\lambda d)^k / k!}]e^{-\lambda d}$.
The matrix $\mathbb{P}$ does not have any special structure. 
So, the stationary distribution $\{P^+_n, \ 0 \le n\linebreak \le N\}$
may be found in a~straightforward manner by solving the system of linear algebraic 
equations 
$$
{\vec P}^+={\vec P}^+ \mathbb{P};\enskip 
{\vec P}^+ {\vec 1} =1
$$ 
where ${\vec P}^+= (P^+_0,\dots,P^+_N)$ and $\vec 1$ is the vector of ones. 
{\looseness=1

}

Once the probabilities $P^+_n$ are found,
the stationary distribution \mbox{$\{P_n, \ 0 \le n \le N+1\}$} 
is computed from the relation\footnote{This follows
from the well-known results for the Markov regenerative processes (see, for 
example,~\cite[Theorem 9.19]{kulk}).} 
$$
P_n=\sum\limits_{i=0}^N P^+_i \fr{f_{in}}{f^*}
$$
where $f_{in}$ is the average time during which there were $n$ customers in the system
provided that the system started with~$i$ customers in it; 
and~$f^*$ is the mean time between transitions of the embedded
Markov chain $\{ N_n, \ n \ge 1\}$.
{\looseness=1

}


Finally,the stationary probability densities $p_n(x)\linebreak =P'_n(x)$
can be computed using the fact that the relation for~$p_n(x)$ 
coincides with the relation for $p_n(x)$ in 
the standard $M/D/1/N$ queue.
Thus, $p_n(x)$ are given by (see, for example,~\cite[p.~72]{Riordan1962})

\noindent
\begin{multline}
\label{eq3}
p_n(x)=e^{-\lambda x} \left (1-B(x) \right ) 
\sum\limits_{k=0}^{n-1} p_{n-k}(0) 
\fr{(\lambda x)^k}{k!}\,, \\
1 \le n \le N\,,  \ x \in [0,d]\,.
\end{multline}
Even though~(\ref{eq3}) holds,
due to the presence of renovation, the boundary conditions $p_{n}(0)$ 
for the considered queue do not coincide 
with boundary conditions $p_{n}(0)$ for the standard $M/D/1/N$ queue.
By integration~(\ref{eq3}) from~0 to~$d$, one gets 
the following relation between~$p_{n}(0)$ and $P_n=\int\nolimits_0^d p_n(x) dx$: 
\begin{multline}
\label{eq3nn}
p_n(0)
= \fr{1}{B_0} \left (\lambda P_n- \sum\limits_{k=1}^{n-1} B_k p_{n-k}(0)
\right )\,, \\ 
1 \le n \le N\,.
\end{multline}
Since the stationary distribution \mbox{$\{P_n, \ 0 \le n \le N+1\}$} 
is already known, the values of $p_n(0)$ are computed recursively from~(\ref{eq3nn}).
The closed-form expressions for
 the average and the standard deviation of the system size are, in the most cases,
 not available and thus, they can be computed,
respectively, by $\sum\nolimits_{n=0}^{N+1} nP_n$ and  
$\sqrt{\sum\nolimits_{n=0}^{N+1} n^2P_n-(\sum_{n=0}^{N+1} nP_n)^2}$.

The computation of the loss ratio~$\pi$, i.\,e., the probability that the arriving customer is lost, 
is more involved. This is due to the fact that the accepted customer
may be lost either after the first service completion or the second, etc.
and the chance to be lost varies, depending on the number of
new customers that have arrived between successive service completions.
 
Let us introduce two quantities:
\begin{enumerate}[(1)]
\item $\gamma_{ij}$, $1 \le i \le N$, $j \ge 0$,~--- probability that the arriving customer
finds~$i$~customers in the system and until the next
service completion, exactly $j$ new customers arrive 
at the system; and
\item
$r_{ij}$, $0\le j \le N-1$, $0 \le i \le N-j-1$,~--- probability that the 
tagged customer waiting in the queue
\textit{will not} be served (i.\,e., will be lost), if there are~$j$~customers
 in front of it in the queue (excluding the one in server)
and~$i$ behind.
\end{enumerate}

Given that $\gamma_{ij}$ and $r_{ij}$ are known, the loss ratio~$\pi$  
can be computed as
\begin{multline*}
\pi =
P_{N+1}
 + 
\sum\limits_{i=1}^{N} \left [
\sum\limits_{j=0}^{N-i}
\gamma_{ij} \left ( \sum\limits_{k=0}^{i-2} q_k r_{j,i-2-k}
 +{}\right.\right.\\
\left. {}+ \sum\limits_{k=i}^{i+j-1} q_k  + Q_{j+i} r_{j,i-2} \right )
 + {}
\end{multline*}

\noindent
\begin{multline*}
{}+ \sum\limits_{j=N-i+1}^{\infty} \gamma_{ij} \left (
\sum\limits_{k=0}^{i-2} q_k r_{N-i,i-2-k}  + {}\right.\\
\left.\left.{}+\sum\limits_{k=i}^{N-1}
q_k  + Q_{N} r_{N-i,i-2} \right )
\right ]\,.
%\label{ploss2}
\end{multline*}

Due to the PASTA property of Poisson arrivals,
the expression for $\gamma_{ij}$ follows 
from the law of total probability:
\begin{multline*}
\gamma_{ij}
= \int\limits_0^d p_{i}(x) \fr{(\lambda (d- x))^j }{j!}\, e^{-\lambda (d-x)}\, dx\,,
\\
 1 \le i \le N\,, \ j \ge 0\,.
\end{multline*}
Again, by applying the law of total probability,
one gets the relations for the recursive computation of~$r_{ij}$, 
$0\le j \le N-1$, $0 \le i \le N-j-1$:
\begin{align*}
r_{i0}&= \sum\limits_{m=0}^{N-i-1}
\beta_m 
\sum\limits_{k=1}^{m+i} q_k +
B_{N-i-1} \sum\limits_{k=1}^{N-1} q_k\,;\\
r_{ij}&= \sum\limits_{m=i}^{N-1-j} \beta_{m-i} \left (
\vphantom{\sum\limits_{k=j+1}^{m+j} q_k + Q_{j+m+1} r_{m,j-1}}
\sum\limits_{k=0}^{j-1} q_k r_{m,j-1-k} +{}\right.\\
&\left.{}+
\sum\limits_{k=j+1}^{m+j} q_k + Q_{j+m+1} r_{m,j-1}
\right ) +{}
\\
&{}+ B_{N-j-i-1}
\left ( \vphantom{\sum\limits_{k=j+1}^{m+j} q_k + Q_{j+m+1} r_{m,j-1}}
\sum\limits_{k=0}^{j-1} q_k r_{N-j-1,j-1-k} +{}\right.\\
&\hspace*{16mm}\left.{}+\sum\limits_{k=j+1}^{N-1} q_k +Q_{N} r_{N-j-1,j-1}
\right ).
\end{align*}
The expressions above can be further simplified\footnote{There
are no principal difficulties in generalizing
the results for the $\mathrm{BMAP}/G/1/N$ queue.
Yet, this would obscure the goal of the paper and thus, 
we remain with the simple model.} by computing 
the integrals explicitly, but we do not dwell on it here.
For small and moderate values of~$d$, $N$, and~$\lambda$,
they can be directly used for numerical implementation.
In the numerical section, precisely these expressions are used to calculate
the loss ratio. The expressions for the average and the standard
deviation of consecutive losses are much harder to derive
and we leave it for a~separate study. The values of these 
two parameters were taken from the Monte-Carlo simulation. 

\section{Numerical Example}

\noindent
As the reference point, we have chosen the numerical results in~\cite{Chyd}
which are based on the analytic expressions and which show the 
performance characteristics 
of the $M/D/1/N$ queue with four different AQM mechanisms. 
Since RED scheme is one of the best among the four,
our goal here is to rank the $M/D/1/N$ queue with RED from~\cite{Chyd}
and the $M/D/1/N$ queue with renovation. Comparison is made 
according to the stationary loss ratio, average system size,  
and average number of consecutive losses along with 
their standard deviations.

The initial conditions are: 
the maximum queue size is $N=9$ and the service time is $d=1$. 
Thus, the offered load is $\rho=\lambda d$. 
The RED dropping function is given by (see~\cite[Eq. (59)]{Chyd}):
\begin{equation}
\label{df}
d_n=
\begin{cases}
0\,, & n\le 3\,;\\
0.11917n - 0.35752\,, & 4 \le n \le 9\,;\\
1\,, & n=10.
\end{cases}
\end{equation}
The performance of the $M/D/1/N$ queue with this RED dropping function 
is given in~\cite[Tables 1, 3, and~4]{Chyd}.
In order to find out whether there exists a~renovation 
mechanism under which the $M/D/1/N$ queue can perform at least as good as
under RED, one needs to perform exhaustive search over
the possible values of the renovation parameters $\{q_i, \ 0 \le i \le N\}$.
Since we are unaware of any analytic way of choosing these values,
adaptive search algorithms for partially observable
Markov decision processes from~\cite{kono1} were used instead.
Meta-heuristics (like particle swarm optimization), which are also applicable here,
were not used.

In Tables~1 and~2, one can see the numerical results for the four different
cases of the offered load\footnote{For the sake of reproducibility 
of the results 
presented in the paper, we also report the obtained values of the renovation 
probabilities: for $\rho=0.5$,
${\vec q}=(0.2544,0.0037,0.0053,0.0065,0.0122,0.0352,0.1108,0.1898,0.2129,0.1691)$;
for $\rho=1$, $q_0=0.0551$, $q_6=0.051$, $q_7=0.7166$, $q_8=0.0917$, and
$q_9=0.0856$;
for $\rho=2$, $q_0=0.1078$, $q_1=0.6374$, $q_4=0.0042$, $q_6=0.0084$, and
$q_9=0.2422$; and
for $\rho=3$, $q_1=0.4608$, and $q_2=0.5392$.}~$\rho$: $\rho=0.5$~--- underloaded system;
$\rho=1$~--- critically loaded system;
and $\rho=2$ and~3~--- overloaded system. The values displayed are the 
values from the numerical experiments rounded to three decimal digits.

%\noindent and in each case compute the stationary 
%loss ratio, average system content  
%and average number of consecutive losses along with
%their standard deviations. 


\begin{table*}\small
\begin{center}
\parbox{400pt}{\Caption{Performance characteristics of the $M/D/1/9$ system with the RED 
mechanism~(\ref{df}) and the $M/D/1/9$ the renovation mechanism (ren.)\
under the offered load $\rho=0.5$ and $\rho=1$}
}
%\label{my-label}
\vspace*{2ex}

\tabcolsep=8pt
\begin{tabular}{cc|c|c|c||c|c|c|}
\cline{3-8}
                                        &  & \multicolumn{3}{c||}{$\rho=0.5$} & \multicolumn{3}{c|}{$\rho=1$} \\ \cline{3-8} 
                                        &  &   Tail Drop      & RED     &    ren.   &    Tail Drop   &      RED &    ren.   \\ \hline
\multicolumn{2}{|c|}{loss ratio}    &   0    &   0.002        &   0.002  &    0.051   &    0.091        &   0.104    \\ \hline
\multicolumn{1}{|c|}{\textbf{system}} & average &   0.750    &   0.741        &    0.744    &    5.064   &      3.000       &   2.999  \\ \cline{2-8} 
\multicolumn{1}{|c|}{\textbf{size}} & standard deviation &   0.946    &    0.920      &    0.935      &    2.897   &   1.887       &   2.091      \\ \hline
\multicolumn{1}{|c|}{\textbf{consecutive}} &average  &  1.152     &  1.053        &   1.800     &    1.359   &    1.239       &  6.876         \\ \cline{2-8} 
\multicolumn{1}{|c|}{\textbf{losses}} & standard deviation &  0.403     &   0.240       &    1.260     &    0.647   &       0.561        &   0.852     \\ \hline
\end{tabular}
\end{center}
\vspace*{-6pt}
\end{table*}




\begin{table*}\small %tabl2
\begin{center}
\parbox{400pt}{\Caption{Performance characteristics of the $M/D/1/9$ system with the RED 
mechanism~(\ref{df}) and the $M/D/1/9$ the renovation mechanism (ren.)\
under the offered load $\rho=2$ and $\rho=3$}
}
%\label{my-label}
\vspace*{2ex}

\tabcolsep=8pt
\begin{tabular}{cc|c|c|c||c|c|c|}
\cline{3-8}
                                        &  & \multicolumn{3}{c||}{$\rho=2$} & \multicolumn{3}{c|}{$\rho=3$} \\ \cline{3-8} 
                                        &  &   Tail Drop      & RED     &    ren.   &    Tail Drop   &      RED &    ren.   \\ \hline
\multicolumn{2}{|c|}{loss ratio}    &   0.500    &    0.500        &   0.502     &    0.667   &    0.667       &     0.667      \\ \hline
\multicolumn{1}{|c|}{\textbf{system}} & average &   9.372    &    7.146       &   7.142      &  9.646     &      8.390        &    7.114  \\ \cline{2-8} 
\multicolumn{1}{|c|}{\textbf{size}} & standard deviation &   0.744    &     1.436       &  2.387      &    0.523   &     1.090        &    2.246  \\ \hline
\multicolumn{1}{|c|}{\textbf{consecutive}} &average  &  1.884     &   1.996         &  1.592       &     2.542  &     2.876        &   2.141    \\ \cline{2-8} 
\multicolumn{1}{|c|}{\textbf{losses}} & standard deviation &   1.069    &    1.366        &  1.100      &    1.454   &       2.064       &  1.236        \\ \hline
\end{tabular}
\end{center}
\end{table*}


Data is the tables show that with respect to the loss ratio,
renovation can perform as good as RED in the wide range of the offered load~$\rho$.
The only exception is the case $\rho=1$: here, renovation can keep
only the average system size at the same level as RED; other four 
performance characteristics are worse. 

If we fix the loss ratio, then the renovation mechanism
can guarantee at least the same value of the average system size as guaranteed by RED.
It is worth noticing that as the offered load increases, the average system size 
under the renovation mechanism becomes smaller than the average system size under RED.
Yet, renovation keeps the queue less stable than RED:
the standard deviation of system size is always smaller for RED.

If we fix the loss ratio and the average system size,
then the renovation mechanism spreads out the losses 
worse than RED when the system is underloaded and 
better than RED when it is overloaded.
This can be seen from the values of the averages and
standard deviations in the last two rows of Tables~1 and~2.

\vspace*{-6pt}

\section{Concluding Remarks}

\noindent
Even though the idea behind the renovation-type AQMs is completely 
different from the idea behind RED-type AQMs, 
renovation-type AQMs may allow one to achieve in some cases at least 
the same system performance level as guaranteed by RED-type AQMs. 
Although the comparison presented here is based only on a~single RED dropping 
function~(\ref{df}), 
our numerical experiments show that the results remain 
qualitatively the same for RED-type AQMs with other dropping functions.
Being defined by~$N$~parameters, the renovation mechanism is very flexible
and this constitutes its strength and weakness.
By varying the values of the renovation probabilities~$q_i$,
it is possible to carry out conditional optimization,
but good searching procedures are required here.

Implementation of the renovation as a~packet dropping mechanism
requires \textit{a~priori} tuning and/or operational configuration of its parameters.
Thus, whether it is appropriate to use renovation as a~packet dropping mechanism 
or not in practice heavily depends on the use case.
Although the tuning of the renovation parameters~$q_i$ can be made on the 
fly during operation, with respect to the recommendations of the RFC~7567~\cite{RFC7567},
renovation mechanism is not the proper choice for the network congestion control 
unless simple recommendations on how to set up the renovation parameters are given.
We believe this can be done based on more deep and insightful numerical experiments.

There remain a~large number of unresolved issues 
related to the renovation mechanism 
(e.\,g., can renovation ensure fairness among competing flows?
may the average queue size instead of its instantaneous value
increase the efficiency of the renovation mechanism?)
and this motivates its further analysis. 
Furthermore, evaluation of the renovation mechanism with parameters 
adapted to a~realistic router/switch use case
and/or evaluation which includes TCP feedback loops 
of several flows remains an open issue as well.

\vspace*{-6pt}


\Ack
  \noindent
   The reported study was partially funded by the Russian Foundation for 
Basic Research according to the research project No.\,18-07-00692.

The authors would like to thank the anonymous referees for their valuable comments 
which helped to improve the paper.
  
 \renewcommand{\bibname}{\protect\rmfamily References}

%\vspace*{-6pt}

\vspace*{-6pt}

{\small\frenchspacing
{\baselineskip=10.65pt
\begin{thebibliography}{99}
\bibitem{RFC7567} %1
\Aue{Baker, F., and G.~Fairhurst.} 2015.
IETF recommendations regarding active queue management.
Available at: {\sf https://tools.ietf.org/html/7567} (accessed October~4, 2018).



\bibitem{Adams} %2
\Aue{Adams, R.} 2013. 
Active queue management: A~survey. \textit{IEEE Commun. Surv.
Tut.} 15(3):1425--1476.

\bibitem{Bonald} %3
\Aue{Bonald, T., M.~May, and J.\,C.~Bolot.}
2000. Analytic evaluation of RED performance. 
\textit{IEEE Conference on Computer Communications Proceedings}
3:1415--1424.

\bibitem{hao} %4
\Aue{Hao, W., and Y.~Wei.} 2005.
An extended $GI^X/M/1/N$ queueing
model for evaluating the performance of AQM algorithms
with aggregate traffic.
\textit{Networking and mobile computing}.
Eds.\ Xicheng Lu and Wei Zhao.
{Lecture notes in computer science ser.} Springer. 3619:395--404.

\bibitem{Chyd} %5
\Aue{Chydzi$\acute{\mbox{n}}$nski, A., and L.~Chr$\acute{\mbox{o}}$st.} 
2011. Analysis of AQM queues with queue size based packet
dropping. \textit{Int. J.~Appl. Math. Comp.} 21(3):567--577.

\bibitem{Chyd2} %6
\Aue{Chydzi$\acute{\mbox{n}}$nski, A., and P.~Mrozowski.} 2016. 
Queues with dropping functions and general arrival
processes. \textit{PLoS ONE} 11(3):e0150702. Available at: 
{\sf https://\linebreak journals.plos.org/plosone/article?id=10.1371/journal.\linebreak pone.0150702} 
(accessed October~4, 2018).

\bibitem{oleg} %7
\Aue{Tikhonenko, O., and W.~Kempa.} 2016. Performance evaluation of 
an $M/G/n$-type queue
with bounded capacity and packet dropping. \textit{Int. J.~Appl.
Math. Comp.} 26(4):841--854.




\bibitem{konnew} %8
\Aue{Konovalov, M.\,G., and R.\,V.~Razumchik.} 2018. 
Numerical analysis of improved access restriction algorithms in a~$GI/G/1/N$
system. \textit{J.~Commun. Technol. El.} 63(6):616--625. 


\bibitem{Kreinin} %9
\Aue{Kreinin, A.\,Y.} 1997.
Queueing systems with renovation. 
\textit{J.~Appl. Math. Stochastic Analysis} 10(4):431--441.


\bibitem{zarN2} %10
\Aue{Zaryadov, I.\,S.} 2009. Queueing systems with general renovation.
\textit{Conference (International)
on Ultra Modern Telecommunications Proceedings}. 1--4.

\bibitem{Zaryadov2009} %11
\Aue{Zaryadov, I.\,S., and A.\,V.~Pechinkin.} 2009.
Stationary time characteristics of the ${GI/M/n/\infty}$
system with some variants of the generalized renovation discipline. \textit{Automat.
Rem. Contr.} 70(12):2085--2097.

\bibitem{Zaryadov2010} %12
\Aue{Zaryadov, I.\,S.}
2010. The ${GI/M/n/\infty}$ queuing system with generalized renovation.
\textit{Automat. Rem. Contr.} 71(4):663--671.

\bibitem{zarN1} %13
\Aue{Korolkova, A., and I.~Zaryadov.} 2010.
The mathematical model of the traffic transfer process with a~rate adjustable by {RED}.
\textit{Conference (International) on Ultra Modern Telecommunications Proceedings}. 
1046--1050.

\bibitem{RFC8289} %14
\Aue{Nichols, K., V.~Jacobson, A.~McGregor, and J.~Iyengar.} 2018.
Controlled delay active queue management.
Available at: {\sf https://datatracker.ietf.org/doc/rfc8289} (accessed October~4, 2018).

\bibitem{Riordan1962} %15
\Aue{Riordan, J.} 1962. \textit{Stochastic service systems}. 
SIAM ser. in applied mathematics. New York, NY: Wiley. 139~p.

\bibitem{arxivRK}  %16
\Aue{Konovalov, M.,  and R.~Razumchik.} 2017.
Queueing systems with renovation vs.\ queues with RED. Supplementary material. 
\textit{ArXiv e-prints}. Available at: {\sf https://arxiv.\linebreak org/abs/1709.01477/}
(accessed October~4, 2018).

\bibitem{kulk} %17
\Aue{Kulkarni, V.\,G.} 2016. \textit{Modeling and analysis of stochastic systems}. 
3rd ed. Chapman \&~Hall/CRC texts in statistical science ser.
Chapman \&~Hall/CRC. 606~p.

\bibitem{kono1} %18
\Aue{Konovalov, M.\,G.} 2007.
\textit{Metody adaptivnoy obrabotki informatsii i~ikh prilozheniya}
[Methods of adaptive information processing and their applications]. 
Moscow: Institute of Informatics Problems of RAS. 212~p.
\end{thebibliography} } }

\end{multicols}

\vspace*{-6pt}

\hfill{\small\textit{Received October 9, 2018}}

\vspace*{-18pt}
  

 \Contr

\noindent
\textbf{Konovalov Mikhail G.} (b.\ 1950)~--- 
Doctor of Science in technology, principal scientist, Institute of Informatics
Problems, Federal Research Center ``Computer Science and Control'' 
of the Russian Academy of Sciences, 44-2~Vavilov Str., Moscow 119333, 
Russian Federation; \mbox{mkonovalov@ipiran.ru}

\vspace*{3pt}


\noindent
\textbf{Razumchik Rostislav V.} (b.\ 1984)~--- 
Candidate of Science (PhD) in physics and mathematics, leading scientist,
Institute of Informatics Problems, Federal Research Center ``Computer 
Science and Control'' of the Russian
Academy of Sciences, 44-2~Vavilov Str., Moscow 119333, Russian Federation; 
associate professor, Peoples'
Friendship University of Russia (RUDN University), 
6~Miklukho-Maklaya Str., Moscow 117198, Russian
Federation; \mbox{rrazumchik@ipiran.ru} %; \mbox{razumchik\_rv@rudn.ru}

\vspace*{6pt}

\hrule

\vspace*{2pt}

\hrule

\vspace*{-2pt}

%\newpage

%\vspace*{-24pt}

\def\tit{СРАВНЕНИЕ ДВУХ МЕХАНИЗМОВ АКТИВНОГО УПРАВЛЕНИЯ ОЧЕРЕДЬЮ В~СИСТЕМЕ $M/D/1/N$$^*$}

\def\titkol{Сравнение двух механизмов активного управления очередью в~системе $M/D/1/N$}

\def\aut{М.\,Г.~Коновалов$^1$, Р.\,В.~Разумчик$^{1,2}$}

\def\autkol{М.\,Г.~Коновалов, Р.\,В.~Разумчик}

{\renewcommand{\thefootnote}{\fnsymbol{footnote}} \footnotetext[1]
{Исследование выполнено при частичной финансовой поддержке РФФИ (проект 18-07-00692).}}



\titel{\tit}{\aut}{\autkol}{\titkol}

\vspace*{-11pt}

\noindent
$^1$Институт проблем информатики Федерального исследовательского 
центра <<Информатика и управление>>\linebreak
$\hphantom{^1}$Российской академии наук

\noindent
$^2$Российский университет дружбы народов 

\vspace*{5pt}

\def\leftfootline{\small{\textbf{\thepage}
\hfill ИНФОРМАТИКА И ЕЁ ПРИМЕНЕНИЯ\ \ \ том\ 12\ \ \ выпуск\ 4\ \ \ 2018}
}%
 \def\rightfootline{\small{ИНФОРМАТИКА И ЕЁ ПРИМЕНЕНИЯ\ \ \ том\ 12\ \ \ выпуск\ 4\ \ \ 2018
\hfill \textbf{\thepage}}}

\vspace*{-3pt}


\Abst{Представлены некоторые результаты численных экспериментов, подтверждающие
следующее обстоятельство: параметры механизма обобщенного обновления
могут быть подобраны таким образом,\linebreak\vspace*{-12pt}}

\Abstend{что уровень производительности
однолинейных сис\-тем массового обслуживания с обобщенным обновлением
может быть не ниже уровня производительности
систем с RED-по\-доб\-ны\-ми механизмами активного управ\-ле\-ния очередями.
Механизмы сравниваются на примере сис\-те\-мы $M/D/1/N$
по стационарным значениям сле\-ду\-ющих характеристик:
вероятность потери заявки, среднее число заявок в сис\-те\-ме,
среднее чис\-ло последовательных потерь заявок 
и~их средние квадратические отклонения.
Расчеты основаны на известных фактах,
а~также на ряде новых аналитических результатов для систем
с~обобщенным обновлением, полученных в данной работе.}

\KW{система массового обслуживания; 
алгоритмы активного управления очередями; обобщенное обновление}

\DOI{10.14357/19922264180402}



%\vspace*{-3pt}


 \begin{multicols}{2}

\renewcommand{\bibname}{\protect\rmfamily Литература}
%\renewcommand{\bibname}{\large\protect\rm References}

{\small\frenchspacing
{%\baselineskip=10.8pt
\begin{thebibliography}{99}
%\vspace*{-3pt}

\bibitem{RFC7567-1} %1
\Au{Baker F., Fairhurst~G.}
IETF recommendations regarding active queue management, 2015.
{\sf https://tools.\linebreak ietf.org/html/7567}.



\bibitem{Adams-1} %2
\Au{Adams R.}
Active queue management: A~survey~// 
{IEEE Commun. Surv. Tut.}, 2013. Vol.~15. No.\,3. P.~1425--1476.

\bibitem{Bonald-1} %3
\Au{Bonald T., May M., Bolot~J.\,C.} Analytic evaluation of RED performance~//
{IEEE Conference on Computer Communications Proceedings}, 2000. 
Vol.~3. P.~1415--1424.

\bibitem{hao-1} %4
\Au{Hao W., Wei~Y.}
An extended $GI^X/M/1/N$ queueing
model for evaluating the performance of AQM algorithms
with aggregate traffic~// Networking and mobile computing~/
Eds. Xicheng Lu and Wei Zhao.~---
Lecture notes in computer science ser.~--- Springer, 2005. Vol.~3619. P.~395--404.

\bibitem{Chyd-1} %5
\Au{Chydzi$\acute{\mbox{n}}$ski A., Chr$\acute{\mbox{o}}$st~L.} 
Analysis of AQM queues with queue size based packet
dropping~// Int. J.~Appl. Math. Comp., 2011. Vol.~21. No.\,3. P.~567--577.

\bibitem{Chyd2-1} %6
\Au{Chydzi$\acute{\mbox{n}}$ski A.,  Mrozowski~P.}
 Queues with dropping functions and general arrival
processes~// PLoS ONE, 2016. Vol.~11. No.\,3. 
{\sf https://journals.plos.org/plosone/\linebreak article?id=10.1371/journal.pone.0150702}.

\bibitem{oleg-1} %7
\Au{Tikhonenko O., Kempa~W.} Performance evaluation of an $M/G/n$-type queue
with bounded capacity and packet dropping~// {Int. J.~Appl.
Math. Comp.}, 2016. Vol.~26. No.~4. P.~841--854.



\bibitem{konnew-1} %8
\Au{Konovalov M.\,G., Razumchik~R.\,V.}
Numerical analysis of improved access restriction algorithms in a~$GI/G/1/N$
system // {J.~Commun. Technol. El.}, 2018. Vol.~63. No.\,6. P.~616--625.

\bibitem{Kreinin-1} %9
\Au{Kreinin A.\,Y.}
Queueing systems with renovation //
{J.~Appl. Math. Stochastic Analysis}, 1997. Vol.~10. No.~4. P.~431--441.

\bibitem{zarN2-1} %10
\Au{Zaryadov~I.\,S.} Queueing systems with general renovation~//
{Conference (International) on Ultra Modern Telecommunications Proceedings}, 2009.
P.~1--4.

\bibitem{Zaryadov2009-1} %11
\Au{Зарядов И.\,С.,  Печинкин~А.\,В.}
Стационарные временные характеристики системы ${GI/M/n/\infty}$
с~некоторыми вариантами дисциплины обобщенного об\-нов\-ле\-ния~//
{Автоматика и~телемеханика}, 2009. Вып.~12. С.~161--174.

\bibitem{Zaryadov2010-1} %12
\Au{Зарядов И.\,С.} 
Система массового обслуживания $GI/M/n/\infty$ с~обобщенным об\-нов\-ле\-ни\-ем~//
{Автоматика и~телемеханика}, 2010. Вып.~4. С.~130--139.

\bibitem{zarN1-1} %13
\Au{Korolkova A., Zaryadov~I.}
The mathematical model of the traffic transfer process with a~rate adjustable by {RED}~//
{Conference (International) on Ultra Modern Telecommunications Proceedings}, 2010.
P.~1046--1050.

\bibitem{RFC8289-1} %14
\Au{Nichols K., Jacobson V., McGregor A., Iyengar J.}
Controlled delay active queue management, 2018.
{\sf https:// datatracker.ietf.org/doc/rfc8289}.


\bibitem{Riordan1962-1} %15
\Au{Riordan J.} {Stochastic service systems}.~--- 
SIAM ser. in applied mathematics.~--- New York, NY, USA: Wiley, 1962. 139~p.

\bibitem{arxivRK-1} %16
\Au{Konovalov M.,  Razumchik~R.}
Queueing systems with renovation vs.\ queues with RED. Supplementary material~//
{ArXiv e-prints}, 2017. {\sf https://arxiv.\linebreak org/abs/1709.01477/}.

\bibitem{kulk-1} %17
\Au{Kulkarni V.\,G.} Modeling and analysis of stochastic systems. 3rd ed.~--- 
Chapman \&~Hall/CRC texts in statistical science ser.~---
Chapman \& Hall/CRC, 2016. 606~p.

\bibitem{kono1-1}
\Au{Коновалов М.\,Г.} 
{Методы адаптивной обработки информации и~их приложения.}~--- 
М.: ИПИ РАН, 2007. 212~с.
\end{thebibliography}
} }

\end{multicols}

 \label{end\stat}

 \vspace*{-3pt}

\hfill{\small\textit{Поступила в~редакцию  09.10.2018}}


%\renewcommand{\bibname}{\protect\rm Литература}
%\renewcommand{\figurename}{\protect\bf Рис.}
\renewcommand{\tablename}{\protect\bf Таблица}       %4
\def\stat{kor-kor}



\def\tit{МОДИФИЦИРОВАННЫЙ СЕТОЧНЫЙ МЕТОД РАЗДЕЛЕНИЯ ДИСПЕРСИОННО-СДВИГОВЫХ
СМЕСЕЙ НОРМАЛЬНЫХ ЗАКОНОВ$^*$}



\def\titkol{Модифицированный сеточный метод разделения дисперсионно-сдвиговых
смесей нормальных законов}

\def\aut{В.\,Ю.~Королев$^1$,  А.\,Ю.~Корчагин$^2$}

\def\autkol{В.\,Ю.~Королев,  А.\,Ю.~Корчагин}

\titel{\tit}{\aut}{\autkol}{\titkol}

{\renewcommand{\thefootnote}{\fnsymbol{footnote}} \footnotetext[1]
{Работа поддержана Российским научным фондом (проект 14-11-00364).}}


\renewcommand{\thefootnote}{\arabic{footnote}}
\footnotetext[1]{Факультет
вычислительной математики и кибернетики Московского государственного
университета им.\ М.\,В.~Ломоносова; Институт проблем информатики
Российской академии наук; victoryukorolev@yandex.ru}
\footnotetext[2]{Факультет вычислительной математики и кибернетики
Московского государственного университета им.\ М.\,В.~Ломоносова;
sasha.korchagin@gmail.com}

%\vspace*{2pt}



\Abst{Описывается модифицированный двухэтапный
сеточный метод разделения дис\-пер\-си\-он\-но-сдви\-го\-вых смесей нормальных
законов, представляющий собой альтернативу чистому ЕМ (expectation-maximization)
ал\-го\-рит\-му. На
первом этапе этого алгоритма строится дискретная аппроксимация для
смешивающего распределения, на втором этапе подбирается абсолютно
непрерывное распределение из заранее заданного семейства, например,
обобщенных обратных гауссовских законов, ближайшее к~дискретному
распределению, полученному на первом этапе. Обсуждаются вопросы
сходимости этого двухэтапного алгоритма. Доказана монотонность
сеточного итерационного метода, используемого на первом этапе.
Подробно обсуждается вопрос оптимального выбора параметров метода,
прежде всего сетки, накидываемой на носитель смешивающего
распределения. С~этой целью предложены статистические оценки
квантилей смешивающего распределения. Эффективность метода
иллюстрируется примерами конкретных вычислений оценок параметров
обобщенных гиперболических распределений.}

\KW{смесь распределений вероятностей;
дис\-пер\-си\-он\-но-сдви\-го\-вая смесь нормальных законов; обобщенное
гиперболическое распределение; ЕМ-ал\-го\-ритм; сеточный метод
разделения смесей}

\vspace*{1pt}

%\vspace*{2pt}

\DOI{10.14357/19922264140402}


\vskip 12pt plus 9pt minus 6pt

\thispagestyle{headings}

\begin{multicols}{2}

\label{st\stat}

\section{Введение}

При {\it практическом} решении задачи моделирования и исследования
волатильности (изменчивости) хаотических стохастических процессов
ключевым этапом является статистическое разделение смесей
вероятностных распределений. Задача разделения смесей~---
статистического оценивания параметров смесей вероятностных
распределений~--- в~деталях разобрана, например, в~книге~\cite{k2011}.

Для решения задачи разделения смесей вероятностных распределений
традиционно используются итерационные процедуры типа ЕМ-ал\-го\-рит\-ма.
К~сожалению, классический ЕМ-ал\-го\-ритм обладает рядом серьезных
недостатков при его применении к~смесям нормальных законов, а~именно:
он демонстрирует крайнюю неустойчивость по отношению к~исходным
данным и~начальным приближениям.

Для преодоления этих недостатков
предложено много модификаций ЕМ-ал\-го\-рит\-ма (см., например,~\cite{k2011}).
Вместе с тем в~указанной книге предложен и~исследован
принципиально новый~--- сеточный~--- метод приближенного решения
задачи разделения смесей. В~работе~\cite{n2013} подробно исследованы
вопросы сходимости сеточных методов разделения смесей.

В соответствии с подходом к~статистическому анализу хаотических
стохастических процессов, в~частности к~решению задачи декомпозиции
волатильности таких процессов, развитом в~книге~\cite{k2011},
в~общем случае на практике приходится решать задачу разделения
конечных смесей нормальных законов с~произвольно большим числом
неизвестных параметров (параметров компонент и~их весов).
И~хотя в~большинстве приложений возникают смеси не более чем с~пятью--семью
компонентами, даже при использовании таких смесей, скажем, в~задачах
анализа и~прогнозирования финансовых рисков приходится моделировать
траекторию движения точки в~пространствах, размерность которых
соответственно лежит в~пределах от~14 (для пятикомпонентных смесей)
до~20 (для семикомпонентных смесей), что существенно увеличивает
вычислительные и~временн$\acute{\mbox{ы}}$е ресурсы, необходимые для практического
решения указанных задач.

Поскольку во многих ситуациях (например,
при прогнозировании на основе высокочастотных данных) эти задачи
необходимо решать в~режиме, близком к~реальному времени, для
создания эффективных методов статистического анализа на основе
смешанных моделей на первый план выходит проб\-ле\-ма снижения
размерности решаемой задачи, т.\,е.\ параметрического пространства.

Одним из возможных подходов к~снижению размерности является
априорное сужение классов допусти\-мых смесей. К~примеру, при решении
многих задач, связанных с~анализом процессов атмосферной или
плазменной турбулентности, а~так\-же процессов, описывающих эволюцию
различных финансовых индексов, высочайшую адекватность
продемонстрировали модели, основанные на дис\-пер\-си\-он\-но-сдви\-го\-вых
смесях нормальных законов. Класс таких смесей очень обширен
и,~в~част\-ности, включает в~себя обобщенные гиперболические распределения,
которые были введены О.-Е.~Барн\-дорфф-Ниль\-се\-ном в~1977--1978~гг.\ как
класс специальных сдвиг-мас\-штаб\-ных смесей нормальных законов~\cite{BN1977, BN1978}.
Пусть $\alpha\hm\in\r$, $\beta\hm\in\r$. Если
функцию распределения обобщенного гиперболического закона
с~параметрами~$\alpha$, $\beta$, $\nu$, $\mu$, $\lambda$ обозначить
$P_{GH}(x;\alpha,\beta,\nu,\mu,\lambda)$, то по определению
\begin{multline}
P_{GH}(x;\alpha,\beta,\nu,\mu,\lambda)={}\\
{}=
\int\limits_{0}^{\infty}\Phi\left(\fr{x-\beta-\alpha
z}{\sqrt{z}}\right)\,p_{GIG}(z;\nu,\mu,\lambda)\,dz\,,\\
x\in\r\,,
\label{e1-kor}
\end{multline}
где $\Phi(x)$~--- стандартная нормальная функция распределения:
$$
\Phi(x)=\int\limits_{-\infty}^{x}\varphi(z)\,dz\,,\enskip
\varphi(x)=\fr{1}{\sqrt{2\pi}}e^{-x^2/2}\,,\enskip  x\in\mathbb{R}\,;
$$
$p_{GIG}(x;\nu,\mu,\lambda)$~--- плот\-ность обобщенного обратного
гауссовского распределения:
\begin{multline*}
p_{GIG}(x;\nu,\mu,\lambda)={}\\
{}=\fr{\lambda^{\nu/2}}{2\mu^{\nu/2}
K_{\nu}\left(\sqrt{\mu\lambda}\right)}\,
x^{\nu-1}\exp\left\{-\fr{1}{2}\left(\fr{\mu}{x}+\lambda
x\right)\right\}\,,\\ x>0\,.
\end{multline*}
Здесь $\nu\in\r$;
$$
\begin{array}{lll}
\mu>0\,, & \lambda\geqslant0\,, & \mbox{если }\nu<0\,;\\[6pt]
\mu>0\,, & \lambda>0\,, & \mbox{если }\nu=0\,;\\[6pt]
\mu\geqslant0\,, & \lambda>0\,, & \mbox{если }\nu>0\,;
\end{array}
$$
$K_{\nu}(z)$~--- модифицированная бесселева функция третьего рода
порядка~$\nu$:

\noindent
\begin{multline*}
K_{\nu}(z)=\fr{1}{2}\int\limits_{0}^{\infty}y^{\nu-1}\exp
\left\{-\fr{z}{2}\left(y+\fr{1}{y}\right)\right\}\,dy\,,\\
z\in\mathbb{C}\,,\enskip \mathrm{Re}\,z>0\,.
\end{multline*}
Обратим внимание, что в~(1) смешивание происходит одновременно и~по
параметру сдвига, и~по параметру масштаба, но так как эти параметры
в~(1)  связаны жесткой зависимостью, так что параметр сдвига
смешиваемого распределения пропорционален его дисперсии, то
фактически смесь~(1) является {\it однопараметрической} и~поэтому
называется {\it дис\-пер\-си\-он\-но-сдви\-го\-вой} (см., например,~\cite{BN1982}).

Другим примером дис\-пер\-си\-он\-но-сдви\-го\-вых смесей нормальных законов
являются обобщенные дисперсионные гам\-ма-рас\-пре\-де\-ле\-ния, в~которых
смешивающими являются обобщенные гам\-ма-рас\-пре\-де\-ле\-ния~\cite{ks2012, zk2013}.

В указанных семействах смесей число неизвестных параметров равно
пяти или шести (если\linebreak учитывать неслучайный сдвиг). Вместе
с~тем у~подоб\-ных моделей имеются довольно серьезные тео\-ре\-ти\-че\-ские
обоснования: в~работах~\cite{zk2013, k2013} показано, что указанные
модели являются асимптотическими аппроксимациями в~простой
предельной схеме случайного суммирования и~потому могут успешно
применяться для анализа процессов типа остановленных случайных
блужданий. Эти выводы подтверждены статистическим анализом
вы\-со\-ко\-час\-тот\-ных финансовых данных, в~результате которого выявлен
синхронизированный характер изменения интенсивностей потоков заявок
в~сис\-те\-мах электронных торгов, что естественно приводит к~синхронизированному
поведению па\-ра\-мет\-ров сдвига и~диффузии в~соответствующих моделях вида смесей
нормальных законов~\cite{kckg2013}.

\section{Описание моди\-фи\-ци\-ро\-ван\-но\-го
сеточного ме\-то\-да разделения дисперсионно-сдвиговых смесей
нормальных законов и~его свойства}

Оказывается, что сеточные методы разделения смесей довольно
эффективны не только при разделении конечных смесей нормальных
законов, но и~при разделении произвольных дис\-пер\-си\-он\-но-сдви\-го\-вых
смесей нормальных законов. Поясним сказанное на примере задачи
оценивания па\-ра\-мет\-ров обобщенных гиперболических распределений.

Для решения задачи оценивания параметров обобщенных гиперболических
распределений традиционно используется метод, предложенный в~статье~\cite{p2004}
и~по сути являющийся классическим ЕМ-ал\-го\-рит\-мом,
приспособленным к~конкретной задаче, и,~соответственно, наследующий
присущие ЕМ-ал\-го\-рит\-мам недостатки.

Рассмотрим следующий альтернативный двухэтапный метод. На первом
этапе на поло\-жи\-тельной полупрямой выделим основную часть носителя
смешивающего распределения, т.\,е.\ \mbox{ограниченный} интервал,
вероятность которого, вычисленная в~соответствии со смешивающим
распределением, практически равна единице. На этот интервал накинем
конечную сетку, содержащую, возможно, очень много {\it известных}
узлов $u_1,\ldots,u_K$. Считая параметр сдвига~$\beta$ равным нулю,
приблизим искомое обобщенное гиперболическое распределение конечной
смесью нормальных законов:

\noindent
\begin{multline}
P_{GH}(x;\,\alpha,0,\nu,\mu,\lambda)\approx{}\\
{}\approx \sum\limits_{i=1}^K
p_i\Phi\left(\fr{x-\alpha u_i}{\sqrt{u_i}}\right)\,,\enskip
x\in\mathbb{R}\,.\label{e2-kor}
\end{multline}
В смеси, стоящей в~правой части соотношения~(2), неизвестными
являются только параметры $p_1,\ldots,p_{K-1}$ и~$\alpha$. Пусть
$x_1,\ldots,x_n$~--- анализируемая выборка значений случайной
величины с~оцениваемым обобщенным гиперболическим распределением.
Итерационный процесс, определяющий сеточный ЕМ-ал\-го\-ритм для данной
задачи, задается следующим образом. Пусть
$p_1^{(m)},\ldots,p_{K-1}^{(m)}$ и~$\alpha^{(m)}$~--- оценки параметров
$p_1,\ldots,p_{K-1}$ и~$\alpha$ на $m$-й итерации,
$p_K^{(m)}\hm=1\hm-p_1^{(m)}-\cdots-p_{K-1}^{(m)}$. Обозначим

\noindent
\begin{align*}
\varphi_{ij}^{(m)}&=\fr{1}{\sqrt{u_i}}\varphi\left(\fr{x_j-\alpha^{(m)}u_i}{\sqrt{u_i}}\right)\,;
\\
g_{ij}^{(m)}&=\fr{p_i^{(m)}\varphi_{ij}^{(m)}}{\sum\limits_{r=1}^K
p_r^{(m)}\varphi_{rj}^{(m)}}\,,\\
&\hspace*{14mm}i=1,\ldots,K\,;\enskip j=1,\ldots,n\,.
\end{align*}
Тогда, используя стандартные рассуждения, определяющие
вычислительные формулы EM-ал\-го\-рит\-ма для параметров конечной смеси
нормальных законов (см, например,~[1, разд.~5.3.7--5.3.8]),
следует положить

\noindent
\begin{equation}
p_i^{(m+1)}=\fr{1}{n}\sum\limits_{j=1}^n g_{ij}^{(m)}\,, \enskip
i=1,\ldots,K\,.\label{e3-kor}
\end{equation}
Обозначим $\overline{x}=(1/n)\sum\limits_{j=1}^nx_j$. Используя
соотношение~(5.3.24) в~\cite{k2011}, с~учетом очевидного равенства
$\sum\limits_{i=1}^K g_{ij}^{(m)}\hm=1$ можно заметить, что уточненная
оценка параметра~$\alpha$ имеет вид:

\columnbreak

\noindent
\begin{equation}
\alpha^{(m+1)}=\fr{\overline{x}}{\sum\limits_{i=1}^K u_ip_i^{(m+1)}}\,,
\label{e4-kor}
\end{equation}
т.\,е.\ равна отношению генерального выборочного среднего и~текущего
эмпирического среднего смешивающего распределения, что вполне
согласуется с~тем, что в~соответствии с~приводимым ниже соотношением~(\ref{e5-kor})
в~данном случае ${\sf E}X\hm=\alpha{\sf E}U$.

В силу монотонности классического ЕМ-ал\-го\-рит\-ма справедливо следующее
утверждение.

\smallskip

\noindent
\textbf{Теорема~1.} {\it Пусть узлы $u_1,\ldots,u_K$ сетки различны,
неотрицательны и~известны. Тогда итерационный процесс $(3)$--$(4)$
является монотонным, т.\,е.\ каждая его итерация не уменьшает
целевую сеточную функцию правдоподобия}
\begin{multline*}
L(p_1,\ldots,p_K,\alpha;x_1,\ldots,x_n)={}\\
{}=
\prod\nolimits_{j=1}^n\left[\sum\nolimits_{i=1}^K
\fr{p_i}{\sqrt{u_i}}\,\varphi\left(\fr{x_j-\alpha^{(m)}u_i}{\sqrt{u_i}}\right)\right].
\end{multline*}

\smallskip

\noindent
\textbf{Замечание~1.} В~разд.~5.7.4 книги~\cite{k2011} показано, что
при каждом фиксированном значении параметра~$\alpha$ сеточная
функция правдоподобия\linebreak
$L(p_1,\ldots,p_{K-1},\alpha;\,x_1,\ldots,x_n)$ вогнута по
аргументам $p_1,\ldots,p_{K-1}$. Поэтому на каждом шаге
итерационного процесса вместо соотношения~(3) можно\linebreak использо\-вать
любой более быстрый алгоритм максимизации функции
$L(p_1,\ldots,p_{K-1},\alpha^{(m)};\,x_1,\ldots$\linebreak $\ldots,x_n)$ по переменным
$p_1,\ldots,p_{K-1}$. Например, оценки весов $p_1,\ldots,p_K$ можно
искать методом условного градиента~\cite{k2011, kn2010}.

\smallskip

Таким образом, на первом этапе получаются оценки параметра~$\alpha$
и~весов всех узлов~$u_i$ конечной сетки, накинутой на носитель
смешивающего обобщенного обратного гауссовского распределения
$P_{\mathrm{GIG}}(z;\,\nu,\mu,\lambda)$.

На втором этапе остается применить ка\-кой-ли\-бо стандартный метод
подгонки обобщенного обратного гауссовского распределения
$P_{\mathrm{GIG}}(z;\,\nu,\mu,\lambda)$ к~эмпирическим данным типа
гистограммы $(u_1, p_1),\ldots, (u_K, p_K)$. Например, параметры~$\nu$,
$\mu$ и~$\lambda$ можно оценить, минимизируя соответствующую
статистику хи-квад\-рат. Или же, например, можно решить задачу
наименьших квад\-ратов:
\begin{multline*}
(\nu^*,\mu^*,\lambda^*)={}\\
{}=\arg\min\limits_{\nu,\mu,\lambda}\sum\limits_{i=1}^K
\left[p_i- \!\!\!\!\!
\int\limits_{(1/2)\left(u_{i-1}+u_i\right)}^{(1/2)(u_i+u_{i+1})}\!\!\!\!\!\!\!\!\!\!\!\!\!\!\!
p_{GIG}(u;\,\nu,\mu,\lambda)\,du\right]^2,
\end{multline*}
где $u_0=0$; $u_{K+1}\hm=\infty$.

На практике хорошие результаты показал подход с решением задачи
наименьших квадратов. Для поиска параметров использовался алгоритм
ns2sol, описанный в~книге~\cite{DSch1983}. Указанный алгоритм
доступен во многих статистических пакетах, отличается высоким
быстродействием и~возможностью при желании задавать разумные
интервалы для поиска параметров.

%\vspace*{-9pt}

\section{О практическом выборе сетки
на~первом этапе моди\-фи\-ци\-ро\-ван\-но\-го
сеточного метода разделения дисперсионно-сдвиговых смесей нормальных
законов}

Естественно, что при использовании указанного двухэтапного метода
в~динамическом режиме крайне важным становится вопрос о~выборе
наиболее эффективных и~быстродействующих численных процедур и~их
параметров. В~частности, исключительную важность приобретает
правильный выбор сетки на первом этапе. Рассмотрим этот вопрос
подробнее.

Формально рассматриваемая задача выглядит так: по наблюдаемым
значениям $x_1,\ldots,x_n$ требуется построить статистическую оценку
верхней границы квантилей заданного порядка сме\-ши\-ва\-юще\-го закона так,
чтобы как можно точнее оценить носитель смешивающего распределения.

В дальнейшем будем считать, что $x_1,\ldots,x_n$~--- независимые
реализации случайной величины $X\hm=Y\sqrt{U}+\alpha U$, где $Y$~---
случайная величина со стандартным нормальным распределением, а~$U$~---
независимая от нее случайная величина с~обобщенным обратным
гауссовским распределением. Тогда, очевидно, распределение случайной
величины~$X$ имеет вид~(1). Предположим, что у~случайной величины~$U$
существуют моменты первых двух порядков. Тогда, как несложно видеть,
\begin{equation}
{\sf E}X={\sf E}Y\cdot{\sf E}\sqrt{U}+\alpha{\sf E}U=\alpha{\sf
E}U\,.\label{e5-kor}
\end{equation}
При этом по усиленному закону больших чисел с~вероятностью единица
$\overline x\hm\longrightarrow {\sf E}X$ $(n\hm\to\infty)$, так что при
больших~$n$ справедливо приближенное равенство ${\sf E}X\hm\approx\overline x$
и~с учетом~(\ref{e5-kor})
\begin{equation}
{\sf E}U\approx\fr{\overline x}{\alpha}\,.\label{e6-kor}
\end{equation}
Далее, очевидно,

\columnbreak

\noindent
\begin{multline}
{\sf E}X^2={\sf E}Y^2\cdot{\sf E}U+2\alpha{\sf E}X\cdot{\sf E}U^{3/2}+{}\\
{}+
\alpha^2{\sf E}U^2={\sf E}U+\alpha^2{\sf E}U^2\,.
\label{e7-kor}
\end{multline}

\noindent
Поэтому, обозначив
$$
m^2=\fr{1}{n}\sum\limits_{i=1}^nx_i^2\,,
$$
получаем приближенное равенство ${\sf E}X^2\hm\approx m^2$, так что
с~учетом~(\ref{e6-kor}) и~(\ref{e7-kor}) имеем:
\begin{equation}
{\sf E}U^2\approx\fr{1}{\alpha^2}\left(m^2-\fr{\overline
x}{\alpha}\right)\,.\label{e8-kor}
\end{equation}
Если параметр~$\alpha$ известен, то для определения верхней границы~$u^*$
сетки, накидываемой на носитель распределения случайной
величины~$U$, можно задать малое положительное число~$\varepsilon$
и~воспользоваться требованием
\begin{equation}
{\sf P}(U\geqslant u^*)\leqslant\varepsilon\,.\label{e9-kor}
\end{equation}
А~для гарантированного выполнения требования~(\ref{e9-kor}) можно использовать
неравенство Маркова:
$$
{\sf P}(U\geqslant u^*)\leqslant\fr{{\sf E}U^2}{(u^*)^2}\leqslant \varepsilon\,,
$$
откуда с учетом~(\ref{e8-kor})
$$
(u^*)^2\geqslant\fr{{\sf E}U^2}{\varepsilon}\approx
\fr{1}{\alpha^2\varepsilon}\left( m^2-\fr{\overline x}{\alpha}\right)
$$
или
\begin{equation}
u^*\approx\fr{1}{\alpha\sqrt{\varepsilon}}\sqrt{m^2-
\fr{\overline x}{\alpha}}\,.\label{e10-kor}
\end{equation}

\begin{figure*}[b] %fig1
\vspace*{1pt}
 \begin{center}
 \mbox{%
 \epsfxsize=161.718mm
 \epsfbox{kor-1.eps}
 }
 \end{center}
 \vspace*{-9pt}
\Caption{Примеры применения модифицированного двухэтапного сеточного
ЕМ-ал\-го\-рит\-ма для подгонки обобщенного гиперболического распределения
к искусственным данным, $\beta\hm=0$: (\textit{a})~$n\hm=1000$, $\alpha\hm=0{,}3$,
$\nu\hm=1{,}3$, $\mu\hm=1{,}6$, $\lambda\hm=0{,}2$;
(\textit{б})~$n\hm=1000$, $\alpha\hm=0{,}5$, $\nu\hm=1$, $\mu\hm=1$,
$\lambda\hm=3$;
(\textit{в})~$n\hm=1000$, $\alpha\hm=3$,
 $\nu\hm=1{,}3$, $\mu\hm=1{,}6$, $\lambda\hm=2$;
(\textit{г})~$n\hm=10\,000$,
$\alpha\hm=0{,}3$, $\nu\hm=1{,}3$, $\mu\hm=1{,}6$, $\lambda\hm=0{,}2$}
\end{figure*}


Если же параметр~$\alpha$, определяющий асим\-мет\-рию распределения
случайной величины~$X$, неизвестен, то можно воспользоваться
следующими рассуждениями. Обозначим
$$
q_n=\fr{1}{n}\sum\limits_{i=1}^n{\bf 1}(x_i<0)\,,
$$
где ${\bf 1}(A)$~--- индикаторная функция множества (события)~$A$.
При этом по усиленному закону больших чисел с~вероятностью единица
$q_n\hm\longrightarrow {\sf P}(X\hm<0)$ $(n\hm\to\infty)$, так что при
больших~$n$ справедливо приближенное равенство
\begin{equation}
q_n\approx{\sf P}(X<0)\,.\label{e11-kor}
\end{equation}
Но
\begin{multline}
{\sf P}(X<0)=\int\limits_{0}^{\infty}\Phi
\left(-\alpha\sqrt{u}\right) p_{\mathrm{GIG}}(u;\nu,\mu,\lambda)\,du={}\\
{}=
{\sf E}\Phi\left(-\alpha\sqrt{U}\right)\,.\label{e12-kor}
\end{multline}

\pagebreak

\noindent
Предположим сначала, что $q_n\hm<1/2$. Если~$n$ достаточно велико,
то можно с~большой степенью
 уверенности утверж\-дать, что тогда
$\overline x\hm>0$ и~$-\alpha\hm<0$, т.\,е.
 $\alpha\hm>0$ и,~стало быть, на
положительной полуоси значений аргумента~$u$ функция $\Phi(\alpha u)$
вогнута, т.\,е.\ выпукла вверх. Тогда из~(\ref{e11-kor}) и~(\ref{e12-kor}), дважды
применяя неравенство Иенсена, в~силу монотонности функции~$\Phi$
получаем:
\begin{multline}
1-q_n\approx 1-{\sf E}\Phi\left(-\alpha\sqrt{U}\right)=
          {\sf E}\Phi\left(\alpha\sqrt{U}\right)\leqslant{}\\
          {}\leqslant\Phi
          \left(\alpha{\sf E}\sqrt{U}\right)\leqslant
          \Phi\left(\alpha\sqrt{{\sf E}U}\right)\,.\label{e13-kor}
\end{multline}
Если теперь для $t\hm\in(0,1)$ символом~$v_t$ обозначить $t$-кван\-тиль
стандартного нормального закона, то из~(\ref{e13-kor}) и~(\ref{e6-kor}) вытекает
<<приближенное неравенство>>
$$
v_{1-q_n}\hm\leqslant \alpha\sqrt{{\sf E}U}\,,
$$
т.\,е.
$$
\alpha\geqslant\fr{v_{1-q_n}}{\sqrt{{\sf E}U}}\approx
\fr{v_{1-q_n}\sqrt{\alpha}}{\sqrt{\overline x}}\,,
$$
откуда получаем, что при достаточно больших~$n$
\begin{equation}
\alpha\geqslant\fr{v_{1-q_n}^2}{\overline x}\,.\label{e14-kor}
\end{equation}
Если теперь задать малое положительное число~$\varepsilon$, то
для определения верхней границы~$u^*$ сетки, накидываемой на
носитель распределения случайной величины~$U$, можно воспользоваться
требованием~(\ref{e9-kor}), для гарантированного выполнения которого
с~учетом~(\ref{e6-kor}) и~(\ref{e14-kor}) можно использовать неравенство Маркова:
$$
{\sf P}(U\geqslant u^*)\leqslant \fr{{\sf E}U}{u^*}\approx\fr{\overline
x}{\alpha u^*}\leqslant \fr{(\overline x)^2}{v_{1-q_n}^2 u^*}\leqslant
\varepsilon\,,
$$
откуда окончательно вытекает оценка
\begin{equation}
u^*\approx\fr{(\overline x)^2}{v_{1-q_n}^2 \varepsilon}\,.\label{e15-kor}
\end{equation}

\begin{figure*}[b] %fig2
\vspace*{18pt}
 \begin{center}
 \mbox{%
 \epsfxsize=162.433mm
 \epsfbox{kor-3.eps}
 }
 \end{center}
 \vspace*{-9pt}
\Caption{Примеры применения модифицированного двухэтапного
сеточного ЕМ-ал\-го\-рит\-ма для подгонки обобщенного гиперболического
распределения к~искусственным данным, $n=10\,000$, $\beta\hm=0$:
(\textit{а})~$\alpha\hm=0{,}3$,
$\nu\hm=2$, $\mu\hm=2$, $\lambda\hm=2{,}5$;
(\textit{б})~$\alpha\hm=0{,}5$,  $\nu\hm=1$, $\mu\hm=1$, $\lambda\hm=3$;
(\textit{в})~$\alpha\hm=0{,}8$,
$\nu\hm=1{,}3$, $\mu\hm=1{,}6$, $\lambda\hm=2$;
(\textit{г})~$\alpha\hm=1{,}3$, $\nu\hm=2$, $\mu\hm=2$, $\lambda\hm=2{,}5$}
\end{figure*}



В случае $q_n\hm\geqslant1/2$, если $n$ достаточно велико, то можно
с~большой степенью уверенности утверж\-дать, что $\overline x\hm\leqslant 0$
и~$-\alpha\hm\geqslant 0$, т.\,е.\ на положительной\linebreak\vspace*{-12pt}

\pagebreak

%\end{multicols}


%\begin{multicols}{2}

\noindent
 полуоси значений аргумента~$u$
функция $\Phi(-\alpha u)$ вогнута, т.\,е.\ выпукла вверх. Тогда
из~(\ref{e11-kor}) и~(\ref{e12-kor}), дважды применяя неравенство Иенсена, в~силу
монотонности функции~$\Phi$ получаем
$$
q_n\approx {\sf E}\Phi\left(-\alpha\sqrt{U}\right)\leqslant
\Phi\left(-\alpha\sqrt{{\sf E}U}\right)\,,
$$
откуда вытекает <<приближенное неравенство>> $v_{q_n}\hm \leqslant
-\alpha\sqrt{{\sf E}U}$,
т.\,е.
$$
-\alpha\geqslant\fr{v_{q_n}}{\sqrt{{\sf E}U}}\approx
\fr{v_{q_n}\sqrt{|\alpha|}}{\sqrt{|\overline x|}}
$$
и при достаточно больших~$n$
\begin{equation}
|\alpha|\geqslant\fr{v_{q_n}^2}{|\overline x|}\,.\label{e16-kor}
\end{equation}
Для определения верхней границы~$u^*$ сетки, накидываемой на
носитель распределения случайной величины~$U$, снова зададим малое
положительное число~$\varepsilon$ и~потребуем, чтобы было
справедливо условие~(\ref{e9-kor}), для гарантированного выполнения которого
с~учетом~(\ref{e6-kor}) и~(\ref{e16-kor}) используем неравенство Маркова и~тот факт, что
$\mathrm{sign}\, \overline x\hm=\mathrm{sign}\,\alpha$ при достаточно
больших~$n$:
\begin{multline}
{\sf P}(U\geqslant u^*)\leqslant \fr{{\sf E}U}{u^*}\approx
\fr{\overline x}{\alpha u^*}=
\fr{|\overline x|}{|\alpha| u^*} \leqslant{}\\
{}\leqslant
\fr{(\overline x)^2}{v_{q_n}^2 u^*}\leqslant
\varepsilon\,.\label{e17-kor}
\end{multline}
В силу симметричности нормального распределения $v_{t}\hm=-v_{1-t}$ для
любого $t\hm\in(0,1)$, поэтому $v_{q_n}^2\hm=v_{1-q_n}^2$ и~в~случае
$q_n\hm\geqslant1/2$ соотношение~(\ref{e17-kor}) снова приводит к~оценке~(\ref{e15-kor}).

Справедливости ради необходимо отметить, что оценки~(\ref{e10-kor}) и~(\ref{e15-kor})
являются завышенными, но они гарантируют, что
$(1-\varepsilon)$-почти-весь носитель распределения случайной
величины~$U$ будет лежать внутри интервала $[0, u^*]$.

\section{Результаты численных экспериментов}

Приводимые в~данном разделе графики иллюстрируют качество работы
модифицированного сеточного метода разделения дис\-пер\-си\-он\-но-сдви\-го\-вых
смесей нормальных законов на примере его\linebreak применения к~оцениванию
параметров обоб\-щенных гиперболических распределений с~ис\-поль\-зованием
указанного алгоритма выбора сетки\linebreak с~умеренным чис\-лом узлов $K\hm=40$.
Для вы\-чис\-ле\-ний использовались искусственно сгенерированные выборки
объемов $n\hm=1000$ и~$n\hm=10\,000$ с~разными наборами параметров, значения
которых указаны на рисунках. На рис.~1 и~2 изображены гистограммы
(серые столбики) и~графики
истинной плот\-ности (штриховые линии), промежуточной
оценки, полученной сеточным ЕМ-ал\-го\-рит\-мом (пунктирные линии)
и~итоговой оценки (непрерывные линии). На рис.~1 и~2 так\-же указаны
значения полученных оценок параметров. Как видно из приводимых
рисунков, параметры~$\alpha$ оцениваются очень точно. Точность
оценок остальных параметров удовлетворительная и~может быть повышена
за счет использования более частых сеток и~более чувствительных
критериев остановки ЕМ-ал\-го\-рит\-ма на первом этапе. Следует отметить,
что даже в~тех случаях, в~которых наблюдаются заметные расхождения
оценок параметров и~их точных значений, оценки самих плотностей
довольно \mbox{точны}.




{\small\frenchspacing
 {%\baselineskip=10.8pt
 \addcontentsline{toc}{section}{References}
 \begin{thebibliography}{99}
\bibitem{k2011}
\Au{Королев В.\,Ю.} Ве\-ро\-ят\-но\-ст\-но-ста\-ти\-сти\-че\-ские методы
декомпозиции волатильности хаотических процессов.~--- М.: Изд-во
Московского ун-та, 2011.

\bibitem{n2013}
\Au{Назаров А.\,Л.} Приближенные методы разделения смесей
вероятностных распределений: Дисс.\ \ldots\  канд. физ.-мат. наук.~--- М.:
МГУ им.\ М.\,В.~Ломоносова, 2013.

\bibitem{BN1977}
\Au{Barndorff-Nielsen~O.-E.} Exponentially decreasing distributions
for the logarithm of particle size~// Proc. Roy. Soc. Lond.~A,
1977. Vol.~353. P.~401--419.

\bibitem{BN1978}
\Au{Barndorff-Nielsen~O.-E.} Hyperbolic distributions and
distributions of hyperbolae~// Scand. J. Statist., 1978. Vol.~5.
P.~151--157.

\bibitem{BN1982}
\Au{Barndorff-Nielsen~O.-E., Kent~J., S\!{\!\ptb{\!\o}}\,rensen~M.} Normal
variance-mean mixtures and $z$-distributions~// Int. Statist. Rev.,
1982. Vol.~50. No.\,2. P.~145--159.

\bibitem{ks2012}
\Aue{Королев В.\,Ю., Соколов И.\,А.} Скошенные распределения
Стьюдента, дисперсионные гам\-ма-рас\-пре\-де\-ле\-ния и~их обобщения как
асимптотические аппроксимации~// Информатика и~её применения, 2012.
Т.~6. Вып.~1. С.~2--10.

\bibitem{zk2013}
\Au{Закс Л.\,М., Королев В.\,Ю.} Обобщенные дисперсионные
гам\-ма-рас\-пре\-де\-ле\-ния как предельные для случайных сумм~// Информатика
и её применения, 2013. Т.~7. Вып.~1. С.~105--115.

\bibitem{k2013}
\Au{Королев В.\,Ю.} Обобщенные гиперболические
распределения как предельные для случайных сумм~// Тео\-рия
вероятностей и~ее применения, 2013. Т.~58. Вып.~1. С.~117--132.

\bibitem{kckg2013}
\Au{Королев В.\,Ю., Черток А.\,В., Корчагин~А.\,Ю.,
Горшенин~А.\,К.} Ве\-ро\-ят\-но\-ст\-но-ста\-ти\-сти\-че\-ское моделирование
информационных потоков в~сложных финансовых системах на основе
высокочастотных данных~// Информатика и~её применения, 2013. Т.~7.
Вып.~1. С.~12--21.

\bibitem{p2004}
\Au{Protassov R.\,S.} EM-based maximum likelihood parameter
estimation for a~multivariate generalized hyperbolic distribution
with fixed~$\lambda$~// Statistics Computing, 2004. Vol.~14.
P.~67--77.

\bibitem{kn2010}
\Au{Королев В.\,Ю., Назаров А.\,Л.} Разделение смесей
вероятностных распределений при помощи сеточных методов моментов и~максимального правдоподобия~//
Автоматика и~телемеханика, 2010. Вып.~3. С.~98--116.

\bibitem{DSch1983}
\Au{Dennis J.\,E., Schnabel R.\,B.} Numerical methods for
unconstrained optimization and nonlinear equations.~--- Englewood
Cliffs: Prentice-Hall, 1983. 378~p.
 \end{thebibliography}

 }
 }

\end{multicols}

\vspace*{-6pt}

\hfill{\small\textit{Поступила в редакцию 01.10.14}}

\newpage

%\vspace*{12pt}

%\hrule

%\vspace*{2pt}

%\hrule

%\vspace*{12pt}

\def\tit{A MODIFIED GRID METHOD FOR~STATISTICAL SEPARATION
OF~NORMAL VARIANCE-MEAN MIXTURES}

\def\titkol{A modified grid method for statistical separation
of~normal variance-mean mixtures}

\def\aut{V.\,Yu.~Korolev$^{1,2}$ and~A.\,Yu.~Korchagin$^1$}

\def\autkol{V.\,Yu.~Korolev and~A.\,Yu.~Korchagin}

\titel{\tit}{\aut}{\autkol}{\titkol}

\vspace*{-9pt}


\noindent
$^1$Faculty of Computational Mathematics and Cybernetics,
M.\,V.~Lomonosov Moscow State University,\linebreak
$\hphantom{^1}$1-52 Leninskiye Gory, GSP-1, Moscow 119991, Russian Federation


\noindent
$^2$Institute of Informatics Problems, Russian Academy of Sciences,
44-2~Vavilov Str., Moscow 119333, Russian\linebreak
$\hphantom{^1}$Federation

\def\leftfootline{\small{\textbf{\thepage}
\hfill INFORMATIKA I EE PRIMENENIYA~--- INFORMATICS AND
APPLICATIONS\ \ \ 2014\ \ \ volume~8\ \ \ issue\ 4}
}%
 \def\rightfootline{\small{INFORMATIKA I EE PRIMENENIYA~---
INFORMATICS AND APPLICATIONS\ \ \ 2014\ \ \ volume~8\ \ \ issue\ 4
\hfill \textbf{\thepage}}}

\vspace*{3pt}

\Abste{A~modified two-stage grid method for
statistical separation of normal variance-mean mixtures is described
as an alternative to a pure EM (expectation-maximization) algorithm.
At the first stage of this
algorithm, a~discrete approximation is constructed to the mixing
distribution. At the second stage, the obtained discrete
distribution is approximated by an absolutely continuous
distribution from a~predetermined family, say, by a generalized
inverse Gaussian distribution. The convergence of this two-stage
procedure is discussed. The monotonicity of the grid procedure used
at the first stage is proved. The problem of the optimal choice of
the parameters of the method is discussed in detail. First of all,
the problem of the optimal choice of the grid thrown on the support
of the mixing distribution is considered. Statistical estimators are
proposed for the quantiles of the mixing law. The efficiency of the
method is illustrated by examples of its application to the
estimation of the parameters of generalized hyperbolic
distributions.}

\smallskip

\KWE{mixture of probability distributions; normal
variance-mean mixture; generalized hyperbolic distribution;
EM-algorithm; grid method of separation of mixtures}

\DOI{10.14357/19922264140402}

\Ack
\noindent
The research was supported by the Russian Science Foundation (project 14-11-00364).

%\vspace*{3pt}

  \begin{multicols}{2}

\renewcommand{\bibname}{\protect\rmfamily References}
%\renewcommand{\bibname}{\large\protect\rm References}



{\small\frenchspacing
 {%\baselineskip=10.8pt
 \addcontentsline{toc}{section}{References}
 \begin{thebibliography}{99}
 \bibitem{k2011eng}
 \Aue{Korolev, V.\,Yu.} 2011.
\textit{Veroyatnostno-statisticheskie metody dekompozitsii
volatil'nosti khaoticheskikh protsessov}
[Probabilistic and statistical methods for the decomposition of volatility
of chaotic processes].
Moscow: Moscow University Press. 510~p.

\bibitem{n2013eng}
\Aue{Nazarov, A.\,L.} 2013.
{Priblizhennye metody razdeleniya smesey veroyatnostnykh raspredeleniy}
[Approximate methods for the decomposition of volatility of chaotic processes].
Ph.D. Thesis. Moscow: Moscow State University.

\bibitem{BN1977eng}
\Aue{Barndorff-Nielsen, O.\,E.} 1977.
Exponentially decreasing distributions for the logarithm of particle size.
\textit{Proc. Roy. Soc. Lond. A} 353:401--419.

\bibitem{BN1978eng}
\Aue{Barndorff-Nielsen, O.\,E.} 1978.
Hyperbolic distributions and distributions of hyperbolae.
\textit{Scand. J. Statist.} 5:151--157.

\bibitem{BN1982eng}
\Aue{Barndorff-Nielsen, O.\,E., J.~Kent, and M.~S\!{\ptb{\o}}rensen}. 1982.
Normal variance-mean mixtures and $z$-distributions.
\textit{Int. Statist. Rev.} 50(2):145--159.

\bibitem{ks2012eng}
\Aue{Korolev, V.\,Yu., and I.\,A. Sokolov}. 2012.
{Skoshennye raspredeleniya St'yudenta, dispersionnye
gam\-ma-ras\-pre\-de\-le\-niya i~ikh obobshcheniya kak asimptoticheskie
approksimatsii}
[Skewed Student's distributions, variance gamma distributions, and their
generalizations as asymptotic approximations].
\textit{Informatika i ee Primeneniya}~--- \textit{Inform. Appl.} 6(1):2--10.

\bibitem{zk2013eng}
\Aue{Korolev, V.\,Yu., and L.\,M.~Zaks}. 2013.
{Obobshchennye dispersionnye gam\-ma-ras\-pre\-de\-le\-niya kak
predel'nye dlya sluchaynykh summ}
[Generalized variance gamma distributions as limiting for random sums].
\textit{Informatika i ee Primeneniya}~--- \textit{Inform. Appl.} 7(1):105--115.

\bibitem{k2013eng} \Aue{Korolev, V.\,Yu.} 2013.
{Obobshchennye giperbolicheskie raspredeleniya kak predel'nye dlya sluchaynykh summ}
[Generalized hyperbolic distributions as limiting for random sums]
\textit{Theory Probab. Appl.} 58(1):117--132.

\bibitem{kckg2013eng}
\Aue{Korolev, V.\,Yu., A.\,V. Chertok, A.\,Yu.~Korchagin, and A.\,K.~Gorshenin}.
2013. {Ve\-ro\-yat\-no\-st\-no-sta\-ti\-sti\-che\-skoe
mo\-de\-li\-ro\-va\-nie informatsionnykh potokov v~slozhnykh finansovykh sistemakh
na osnove vysokochastotnykh dannykh}
[Probability and statistical modeling of information flows in complex
financial systems from high-frequency data].
\textit{Informatika i~ee Primeneniya}~--- \textit{Inform.  Appl.} 7(1):12--21.

\bibitem{p2004eng-1}
\Aue{Protassov, R.\,S.} 2004.
EM-based maximum likelihood parameter estimation for a multivariate
generalized hyperbolic distribution with fixed~$\lambda$.
\textit{Statistics Computing} 14:67--77.

\bibitem{kn2010eng-1}
\Aue{Korolev, V.\,Yu., and A.\,L.~Nazarov}. 2010.
{Razdelenie smesey veroyatnostnykh raspredeleniy pri pomoshchi
setochnykh metodov momentov i~maksimal'nogo pravdopodobiya}
[Separation of mixtures using grid moment-based methods and maximum likelihood].
\textit{Avtomatika i~Telemekhanika} [Automatics and Telemechanics] 3:98--116.

\bibitem{DSch1983eng}
\Aue{Dennis, J.\,E., and R.\,B.~Schnabel}. 1983.
\textit{Numerical methods for unconstrained optimization and nonlinear equations}.
Englewood Cliffs: Prentice-Hall. 378~p.


\end{thebibliography}

 }
 }

\end{multicols}

\vspace*{-6pt}

\hfill{\small\textit{Received October 01, 2014}}

\vspace*{-18pt}

\Contr

\noindent
\textbf{Korolev Victor Yu.} (b.\ 1954)~---
Doctor of Science in physics and mathematics, professor,
Department of Mathematical Statistics, Faculty of Computational Mathematics
and Cybernetics, M.\,V.~Lomonosov Moscow State University,
1-52 Leninskiye Gory, GSP-1, Moscow 119991, Russian Federation;
leading scientist, Institute of Informatics Problems,
Russian Academy of Sciences, 44-2~Vavilov Str., Moscow 119333, Russian
Federation; victoryukorolev@yandex.ru

\vspace*{3pt}

\noindent
\textbf{Korchagin Alexander Yu.} (b.\ 1989)~---
PhD student, Faculty of Computational Mathematics and Cybernetics,
M.\,V.~Lomonosov Moscow State University,
1-52 Leninskiye Gory, GSP-1, Moscow 119991, Russian Federation;
sasha.korchagin@gmail.com


\label{end\stat}

\renewcommand{\bibname}{\protect\rm Литература}         %5
\def\stat{shestakov+vor}

\def\tit{АСИМПТОТИЧЕСКАЯ НОРМАЛЬНОСТЬ И~СИЛЬНАЯ СОСТОЯТЕЛЬНОСТЬ ОЦЕНКИ РИСКА ПРИ~ИСПОЛЬЗОВАНИИ FDR-ПОРОГА В УСЛОВИЯХ СЛАБОЙ ЗАВИСИМОСТИ}

\def\titkol{Асимптотическая нормальность и~сильная состоятельность оценки риска при~использовании FDR-порога} % в~условиях слабой зависимости}

\def\aut{М.\,О.~Воронцов$^1$, О.\,В.~Шестаков$^2$}

\def\autkol{М.\,О.~Воронцов, О.\,В.~Шестаков}

\titel{\tit}{\aut}{\autkol}{\titkol}

\index{Воронцов М.\,О.}
\index{Шестаков О.\,В.}
\index{Vorontsov M.\,O.}
\index{Shestakov O.\,V.}


%{\renewcommand{\thefootnote}{\fnsymbol{footnote}} \footnotetext[1]
%{Работа 
%выполнена при поддержке Программы развития МГУ, проект №\,23-Ш03-03. При анализе 
%данных использовалась инфраструктура Центра коллективного пользования 
%<<Высокопроизводительные вычисления и~большие данные>> 
%(ЦКП <<Информатика>>) ФИЦ ИУ РАН (г.~Москва)}}


\renewcommand{\thefootnote}{\arabic{footnote}}
\footnotetext[1]{Московский государственный университет 
имени~М.\,В.~Ломоносова, факультет вычислительной математики и~кибернетики;  
Московский центр фундаментальной и~прикладной математики, \mbox{m.vtsov@mail.ru}}
\footnotetext[2]{Московский государственный университет 
имени М.\,В.~Ломоносова, факультет вычислительной математики и~кибернетики; 
Федеральный исследовательский центр <<Информатика и~управление>> Российской 
академии наук; Московский центр фундаментальной и~прикладной математики, 
\mbox{oshestakov@cs.msu.ru}}


\vspace*{-12pt}





\Abst{Рассматривается подход к~решению задачи удаления шума в~большом массиве 
разреженных данных, основанный на методе контроля средней доли ложных отклонений 
гипотез (False Discovery Rate, FDR). Данный подход эквивалентен процедурам 
пороговой обработки, обнуляющим компоненты массива, значения которых не 
превосходят некоторого заданного порога.  Наблюдения в~модели считаются слабо 
зависимыми. Для контроля степени зависимости используются ограничения на 
коэффициент сильного перемешивания и~максимальный коэффициент корреляции. 
В~качестве меры эффективности рассматриваемого подхода используется 
среднеквадратичный риск. Вычислить значение риска можно только на тестовых 
данных, поэтому в~работе рассматривается его статистическая оценка и~исследуются 
ее свойства. Показана асимптотическая нормальность и~сильная состоятельность 
оценки риска при использовании FDR-по\-ро\-га в~условиях слабой зависимости в~данных.}

\KW{пороговая обработка; множественная проверка гипотез; 
оценка риска}

\DOI{10.14357/19922264240309}{ZOQVTO}
  
%\vspace*{-6pt}


\vskip 10pt plus 9pt minus 6pt

\thispagestyle{headings}

\begin{multicols}{2}

\label{st\stat}



\section{Введение}

Во многих прикладных областях возникает задача обработки больших массивов 
зашумленных данных. Примерами служат задачи обработки изоб\-ра\-же\-ний с~высоким 
разрешением~\cite{FDRImage}, задачи множественной проверки гипотез, возникающие 
в~\mbox{исследованиях} в~об\-ласти генетики~\cite{MultipleTesting}, и~другие проб\-ле\-мы. 
В~связи с~этим рас\-смот\-рим модель
$$
x_i = \mu_i + z_i, \enskip i=\overline{1,n}\,,
$$
где $\mu_i\in\mathbb{R}$~--- <<полезные>> данные; $z_i \sim N(0,\sigma^2)$~--- 
шум. Задача заключается в~нахождении оценки неизвестного вектора $\mu \hm= 
(\mu_1,\ldots,\mu_n)$ как функции вектора $x \hm= (x_1,\ldots,x_n)$ и~может 
рассматриваться как задача множественной проверки гипотез о~равенстве нулю 
компонент вектора~$\mu$~\cite{AdaptingFDR}. При этом обычно предполагается, что 
вектор~$\mu$ имеет в~определенном смысле <<разреженную>> структуру, т.\,е.\ для 
<<полезных>> данных используется <<экономное>> представление.



В работе~\cite{AdaptingFDR} для решения рассматриваемой задачи в~условиях 
независимости компонент вектора~$x$ и~разреженности вектора~$\mu$ была 
предложена процедура построения оценки~$\hat{\mu}_F$ вектора~$\mu$, основанная 
на методе контроля средней доли ложных отклонений (FDR) 
гипотез при помощи алгоритма Бен\-жа\-ми\-ни--Хох\-бер\-га,
и~было проведено исследование асимптотики ее среднеквадратичного риска. 
В~работах~\cite{ZasShe17,Mathematics2020} была показана состоятельность 
и~асимптотическая нормальность оценки риска данной процедуры. Аналогичные 
результаты для других методов построения~$\hat{\mu}_F$ получены в~работах~\cite{Shestakov2021-1,Shestakov2021-2,Shestakov2022}.

В то же время в~определенных приложениях, например  при анализе полученных 
в~результате использования ДНК-мик\-ро\-чи\-пов данных~\cite{ResultsOnFDRUnderDependence}, исследовании геофизических процессов 
и~анализе помех\linebreak в~телекоммуникационных каналах, условие незави\-си\-мости компонент 
вектора $x$ может не выполняться. Ранее в~работах~\cite{VorontsovShestakov2023,Vorontsov2024} была \mbox{исследована} асимп\-то\-ти\-ка 
среднеквадратичного риска оценки~$\hat{\mu}_F$ \mbox{в~случае}, когда~$\mu$ принадлежит 
одному из классов разреженности
$$
l_0[\eta] = \left\{\mu\,:\, ||\mu||_0 \leq \eta n\right\}, \enskip \eta \in 
(0,1),
$$

\vspace*{-12pt}

\noindent
\begin{multline*}
m_p[\eta] \equiv{}\\
{}\equiv \left\{\mu \in \mathbb{R}^n : |\mu|_{(k)} \leq \eta n^{1/p} 
k^{-1/p},\ k=\overline{1,n}\right\}, \\
 p\in(0, 2),
\end{multline*}
а компоненты вектора~$x$ слабо зависимы~--- имеют достаточно быстро убывающий 
коэффициент сильного перемешивания~\cite{Bosq}

\noindent
\begin{multline*}
\alpha(k) = \sup\limits_{1\leq m\leq n}\alpha\left(\sigma(x_i, i\leq m), 
\sigma(x_i, i\geq m+k)\right), \\ 
k=\overline{1,n-1}\,,
\end{multline*}
где символом $\sigma(x_i, i\in I)$ обозначена сиг\-ма-ал\-геб\-ра, порожденная 
множеством случайных величин $\{x_i, i \hm\in I\}$, а~мера  $\alpha(\cdot, \cdot)$ 
близости двух сиг\-ма-ал\-гебр определяется как
$$
\alpha(\mathcal{B},\mathcal{C}) = \sup\limits_{B\in\mathcal{B}, 
C\in\mathcal{C}} \left|\p(BC)-\p(B)\p(C)\right|.
$$

В настоящей работе показана асимптотическая нормальность и~сильная 
состоятельность оценки риска при применении FDR-про\-це\-ду\-ры в~случае, когда 
компоненты вектора~$x$ слабо зависимы, а~$\mu$ принадлежит одному из классов 
раз\-ре\-жен\-ности: 
$l_0[\eta]$ или $m_p[\eta]$.


\section{Обработка вектора данных с~помощью FDR-процедуры}

Широким классом методов построения оценки~$\hat{\mu}$ стала пороговая обработка 
вектора~$x$ с~некоторым порогом~$T$. Различают жесткую пороговую обработку, при 
которой полагается
\begin{equation*}
\left(\hat{\mu}\right)_i  = p_H(x_i,T) \equiv
 \begin{cases}
   x_i, & |x_i| > T\,;\\
   0, & |x_i| \leq T\,,
 \end{cases}
\end{equation*}
и мягкую пороговую обработку, для которой
\begin{equation*}
(\hat{\mu})_i  = p_S(x_i,T) \equiv
 \begin{cases}
   x_i-T, & \hphantom{\vert\vert}x_i > T;\\
   x_i+T, & \hphantom{\vert\vert}x_i <- T;\\
   0, & |x_i| \leq T.
 \end{cases}
\end{equation*}
Среднеквадратичный риск подобных процедур определяется как
\begin{equation}
\label{riskDef}
R(T) = {\mathsf E} ||\hat{\mu}-\mu||^2 = \sum\limits_{i=1}^n {\mathsf E} \left((\hat{\mu})_i-
\mu_i\right)^2.
\end{equation}
Обозначим через~$T_m$ наилучшее значение порога:
$$
T_m : \, R(T_m) = \min\limits_{T} R(T).
$$

Предложенная в~\cite{AdaptingFDR} процедура заключается в~жесткой пороговой 
обработке компонент вектора~$x$ с~порогом $\hat{t}_F \hm= \hat{t}_F(x)$, и~ее 
результат~--- оценка $\hat{\mu}_F$ вектора~$\mu$ с~компонентами $(\hat{\mu}_F)_i  
\hm= p_H(x_i,\hat{t}_F)$, где
\begin{multline*}
\hat{t}_F = \sigma z\left(\fr{q \hat{k}_F}{2n}\right), \enskip
\hat{k}_F = \max 
\left\{k \, :\, |x|_{(k)} \geq t_k \right\}, \\
 t_k = \sigma z\left(\fr{q  k}{2n}\right);
\end{multline*}
$z(\alpha)$ --- квантиль уровня $(1\hm-\alpha)$ стандартного нормального 
распределения; $|x|_{(k)}$~--- $k$-й элемент вектора, получаемого в~результате 
упорядочения вектора~$|x|$ по невозрастанию:
$$
|x|_{(1)} \geq |x|_{(2)} \geq \cdots \geq |x|_{(n)};
$$
$q\in(0;1)$~--- управ\-ля\-ющий параметр FDR-ме\-то\-да.
Далее полагается, что $q\hm\equiv q_n$ зависит от~$n$. В~\cite{AdaptingFDR} 
показано, что эта процедура эквивалентна множественной проверке гипотез 
о~равенстве нулю компонент наблюдаемого вектора. Также показано, что с~помощью 
метода штрафных функций данную процедуру можно свести к~другим видам пороговой 
обработки, в~част\-ности к~мягкой пороговой обработке.

В работах~\cite{VorontsovShestakov2023, Vorontsov2024} была исследована 
асимптотика среднеквадратичного риска~$R(\hat{t}_F)$ описанной процедуры 
в~случае, когда компоненты вектора $x$ слабо зависимы, а $\mu$ принадлежит классу 
разреженности~$\Theta_n$, где~$\Theta_n$ есть~$l_0[\eta_n]$ или~$m_p[\eta_n]$. 
Было показано, что~$R(\hat{t}_F)$ асимптотически отличается от минимаксного 
риска
$\inf\nolimits_{\hat{\mu}\hm=\hat{\mu}(x)} \sup\nolimits_{\mu\in \Theta_n} {\mathsf E} 
||\hat{\mu}-\mu||^2$
на множитель не более чем логарифмического по\-рядка.

Отметим, что в~выражении для среднеквадратичного риска~(\ref{riskDef}) 
присутствуют неизвестные величины~$\mu_i$, а~потому вычислить~$R(T_m)$ и~$T_m$ 
не представляется возможным. На практике можно пользоваться, например, следующей 
оценкой среднеквадратичного риска~\cite{Mallat}:
$$
\hat{R}(T) = \sum\limits_{i=1}^n F[x_i, T],
$$
где  
\begin{multline*}
F[x_i, T] = {}\\[3pt]
{}=\!\begin{cases}
\left(x_i^2-\sigma^2\right) \Ik(|x_i|\leq T) + \sigma^2 \Ik\left(|x_i|>T\right) &\\[3pt]
&\hspace*{-53mm}\mbox{для\ жесткой\ пороговой\ обработки};\\[3pt]
\left(x_i^2-\sigma^2\right) \Ik\left(|x_i|\leq T\right) + (\sigma^2+T^2) 
\Ik \left(|x_i|>T\right) \hspace*{-11.21576pt}&\\[3pt]
&\hspace*{-51mm}\mbox{для\ мягкой\ пороговой\ обработки}.
\end{cases}\hspace*{-7.17859pt}
\end{multline*}


\noindent
\textbf{Замечание}.\ При пороговой обработке иногда также используется так 
называемый универсальный порог $T_U\hm = \sigma \sqrt{2\ln n}$, предложенный 
в~работе~\cite{spatialAdaptation}. Исследования в~\cite{AdaptingSURE, ExactRisk} 
показали, что порог~$T_U$ в~определенном смысле максимальный, и~рас\-смат\-ри\-вать 
пороги выше него не имеет смысла. Более того, нетрудно показать, что $t_k \hm< T_U$ 
для всех~$k$ и~всех достаточно больших~$n$, в~связи с~чем всюду далее полагаем, 
что порог~$\hat{t}_F$ выбирается на отрезке $[0; T_U]$.

\section{Вспомогательные утверждения}

Кроме коэффициента сильного перемешивания~$\alpha(\cdot)$ также понадобится 
следующее понятие~\cite{Bosq}.

\smallskip

\noindent
\textbf{Определение.} %\label{defRho}
Максимальным коэффициентом корреляции~$\rho(\cdot)$ компонент вектора~$x$ 
называется
\begin{multline*}
\rho (k) \equiv \rho_n (k) = {}\\
{}=\sup\limits_{1\leq m\leq n}\rho\left(\sigma(x_i, 
i\leq m), \sigma(x_i, i\geq m+k)\right), \\
 k=\overline{1,n-1}\,,
\end{multline*}
где мера $\rho(\cdot, \cdot)$ близости двух сиг\-ма-ал\-гебр определяется как
$$
\rho(\mathcal{B},\mathcal{C}) = \sup\limits_{\substack{\xi 
\in\mathcal{L}^2(\mathcal{B}) \\
 \eta \in\mathcal{L}^2(\mathcal{C})}} 
\left|\mathrm{corr}\,(\xi, \eta)\right|.
$$


Введем обозначения:
$$
T_1 = \sqrt{2\ln \eta_n^{-p}};  \,\gamma_n = \fr{1}{\ln\ln n}; \, \kappa_n 
= \fr{n \eta_n^p T_1^{-p}}{1 - q_n - \gamma_n}; 
$$
$$ 
\kappa_n^0 = \fr{[n \eta_n]}{1 - q_n - \gamma_n} ;\, \rho^\star (k) = 
\sup\limits_{n\geq k+1} \rho(k), k \in \mathbb{N} ;
$$
$$
t_{\kappa_n} = \sigma z\left(\fr{q_n \kappa_n }{2n}\right) , \,\, t_{\kappa_n^0} 
= \sigma z\left(\fr{q_n \kappa_n^0 }{2n}\right).
$$


Следующие два утверждения показывают, что случайный порог~$\hat{t}_F$ в~случае 
$\mu\hm\in m_p[\eta_n]$ (соответственно $\mu\hm\in l_0[\eta_n]$) с~большой 
вероятностью будет не меньше~$t_{\kappa_n}$ (соответственно~$ t_{\kappa_n^0}$). 
Их  доказательства приведены в~работах~\cite{VorontsovShestakov2023, Vorontsov2024}.

\smallskip

\noindent
%\begin{lem}\label{lem5}
\textbf{Лемма~1.}\ \textit{Пусть $n^{-\delta_1} \hm\leq \eta_n^p \hm\leq n^{-\delta_2}$, 
$0\hm<\delta_2\hm<\delta_1<1$, $\mathrm{lim\,inf} q_n \ln n \hm\geq C \hm> 0$, 
$m\hm\in[1;n/2]\cap\mathbb{N}$, а $\alpha(\cdot)$~--- коэффициент сильного 
перемешивания компонент вектора~$x$. Для некоторого $N\hm\in\mathbb{N}$ при $n \hm\geq 
N$ справедливо}
\begin{multline*}
\hspace*{-3pt}\sup\limits_{\mu\in m_p[\eta_n]} \p \left(\hat{k}_F \geq \kappa_n \right) \leq 
4 n \exp\left\{-\fr{m}{256n}  \kappa_n q_n \gamma_n^2    \right\}+{}\\
{}+ 22\left(1+\fr{8n}{\kappa_n q_n \gamma_n}\right)^{1/2} n m 
\alpha\left(\left[\fr{n}{2m}\right]\right).
\end{multline*}



\smallskip

\noindent
\textbf{Лемма 2.}\ 
%\label{lem1}
\textit{Пусть $\eta_n \hm\leq b\hm<1$, $m\in[1;n/2]\cap\mathbb{N}$, а~$\alpha(\cdot)$~--- 
коэффициент сильного перемешивания компонент вектора~$x$. Для некоторого 
$N\hm\in\mathbb{N}$ при $n \hm\geq N$ справедливо}
\begin{multline*}
\sup\limits_{\mu\in l_0[\eta_n]} \p \left(\hat{k}_F \geq \kappa_n^0 \right) 
\leq{}\\
{}\leq 4 n \exp\left\{-\fr{(1-b)m}{64n}\,  \kappa_n^0 q_n \gamma_n^2    
\right\}+{}\\
{}+ 22\left(1+\fr{4n}{(1-b)\kappa_n^0 q_n \gamma_n}\right)^{1/2} n m 
\alpha\left(\left[\fr{n}{2m}\right]\right).
\end{multline*}

Следующие два утверждения доказаны в~\cite{Bosq} и~представляют собой аналоги 
неравенств Хеффдинга и~Бернштейна для слабо зависимых случайных величин.


\smallskip

\noindent
\textbf{Лемма 3.}\
\textit{Пусть для набора действительных случайных величин $X_1, \ldots, X_n$ 
с~коэффициентом сильного перемешивания $\alpha(\cdot)$ выполняется ${\mathsf E} X_i \hm=0$, 
$|X_i|\hm\leq b$, $i\hm=\overline{1,n}$. Тогда для любого целого числа $m\hm\in[1; n/2]$ 
и~любого $\eps\hm>0$ справедливо}
\begin{multline*}
\p\left(\left|\sum\limits_{i=1}^n X_i\right| > n\eps \right) \leq 4 
\exp\left\{-\fr{\eps^2 m}{8 b^2}\right\}+ {}\\
{}+
22\left(1+\fr{4b}{\eps}\right)^{1/2} m\, 
\alpha\left(\left[\fr{n}{2m}\right]\right).
\end{multline*}


\smallskip

\noindent
\textbf{Лемма 4.}\
\textit{Пусть для набора действительных случайных величин $X_1, \ldots, X_k$ 
с~коэффициентом сильного перемешивания $\alpha(\cdot)$ выполняется ${\mathsf E} X_i \hm=0$, 
$|X_i|\hm\leq b$, $i\hm=\overline{1,k}$. Тогда для любого целого числа $m\hm\in[1; k/2]$ 
и~любого $\eps\hm>0$ справедливо}
\begin{multline*}
\p\left(\left|\sum\limits_{i=1}^k X_i\right| > \eps \right) \leq 4 
\exp\left\{-\fr{\eps^2 m}{8 v^2 k^2}\right\}+{}\\
{}+ 22\left(1+\fr{4bk}{\eps}\right)^{1/2} m\, 
\alpha\left(\left[\fr{k}{2m}\right]\right),
\end{multline*}
\textit{где $p = k/(2m)$}:
\begin{multline*}
v^2 =
 \fr{b \eps}{2k} + {}\\
 {}+\fr{2}{p^2} \,  \max\limits_{ j\in[0,\,2m-1]} 
{\mathsf E} \big( ([jp]+1-jp)X_{[jp]+1} + X_{[jp]+2}+{}\\
{}+ \cdots +  X_{[(j+1)p]} + ((j+1)p-[(j+1)p])X_{[(j+1)p+1]}\big)^2.
\end{multline*}

\noindent
\textbf{Замечание.}
Если существует такое число $S \hm> 0$, что сразу для всех $i\hm\in[1;k]$  выполняется 
${\mathsf E} X_i^2 \hm\leq S^2$, то в~качестве~$v^2$ можно взять
$$
v^2 = \fr{b \eps}{2k} + 8 S^2.
$$


Д\,о\,к\,а\,з\,а\,т\,е\,л\,ь\,с\,т\,в\,о\ \ сле\-ду\-юще\-го утверж\-де\-ния приведено в~работе~\cite{AdaptingFDR}.

\smallskip

\noindent
\textbf{Лемма 5.}\ 
\textit{Для $y\leq 0{,}01$ справедливы представления}
\begin{multline}
\label{lem1eq1}
z^2(y) = 2 \ln y^{-1} - \ln \ln y^{-1} - r_2(y), \\
 r_2(y) \in [1{,}8; 3];
\end{multline}

\noindent
\begin{equation}
\label{lem1eq2}
z(y) = \sqrt{2 \ln y^{-1}} - r_1(y), \, \, r_1(y) \in [0; 1{,}5].
\end{equation}


\section{Асимптотическая нормальность оценки риска при~применении FDR-процедуры в~условиях слабой зависимости}

Перейдем к~описанию достаточных условий для асимптотической нормальности оценки 
риска $\hat{R}(\hat{t}_F)$ в~случае $\mu \hm\in m_p[\eta_n]$.

\smallskip

\noindent
\textbf{Теорема~1.}\
\textit{Пусть $\mu \hm\in m_p[\eta_n],$ $\eta_n^p \hm\in[n^{-\delta_1}; n^{-\delta_2}],$ $1/2 \hm< 
\delta_2 \hm< \delta_1<1;$ имеются такие константы $c_1, c_2>0$, что для 
коэффициента сильного перемешивания $\alpha(\cdot)$ компонент вектора $x$ 
справедливо  $\alpha(k) \hm\leq c_1 k^{-1-(5/2)\delta_1/(1-\delta_1)-c_2},$ 
$k\hm=\overline{1,n-1};$ $q_n \hm< c_3 \hm< 1;$ $\mathrm{lim\,inf} q_n \ln n \hm= c_4 \hm> 0;$ и,~кроме того, 
для максимального коэффициента корреляции $\rho(\cdot)$ компонент вектора~$x$ 
справедливо}
$$
\sum\limits_{k = 1}^{\infty} \sup\limits_{n\geq k+1} \rho(k) \equiv 
\sum\limits_{k = 1}^{\infty}  \rho^\star (k) = c_5 < \infty. 
$$
\textit{Тогда при $n \to \infty$}
$$
\fr{\hat{R}(\hat{t}_F) - R(T_m)}{C_\rho \sqrt{2n}} \Rightarrow N(0, 1),
$$
\textit{где}
$$
C_\rho = \sigma^2\sqrt{1 +  \lim\limits_{n\to\infty} \fr{1}{n} \sum\limits_{j\neq i} \mathrm{corr}^2 (x_i, x_j)}.
$$

\noindent
Д\,о\,к\,а\,з\,а\,т\,е\,л\,ь\,с\,т\,в\,о\  \
 приводится для метода мягкой пороговой обработки; в~случае жесткой пороговой 
обработки доказательство аналогично. Обозначим
$$
U(T) = \hat{R}(T) -  \hat{R}(T_m) = \sum \limits_{i=1}^n H_i(T, T_m),
$$
где
$$
H_i(T, T_m) = F[x_i, T] - F[x_i, T_m].
$$
Имеем

\vspace*{-3pt}

\noindent
\begin{multline}
\label{D00}
\hat{R}(\hat{t}_F) - R(T_m) + \hat{R}(T_m) - \hat{R}(T_m) ={}\\
{}= \hat{R}(T_m) - 
R(T_m) + U(\hat{t}_F).
\end{multline}
Покажем, что
\begin{equation}
\label{D0}
\fr{\hat{R}(T_m) - R(T_m)}{C_\rho\sqrt{2n}} \Rightarrow N(0, 1).
\end{equation}


Повторяя рассуждения из~\cite{KuShe2016_1,KuShe2016_2,Jansen}, можно показать, 
что $T_m \hm\geq t_{\kappa_n}$. Учитывая также $T_m\hm \leq T_U$, имеем 
$$
C \sqrt{\ln n} \leq T_m \leq C^\prime \sqrt{\ln n}
$$ 
для некоторых положительных констант $C$ и~$C^\prime$.

\columnbreak

В случае мягкой пороговой обработки $\hat{R}(T_m)$ представляет собой 
несмещенную оценку~$R(T_m)$, а~при жесткой пороговой обработке и~выполнении 
условий теоремы смещение стремится к~нулю при делении на $\sqrt{n}$~\cite{Mallat}.

Для дисперсии числителя~(\ref{D0}) имеем:
\begin{multline*}
{\mathsf D} \left(\hat{R}(T_m) - R(T_m)\right) = \sum\limits_{i=1}^n {\mathsf D} F[x_i, T_m] + {}\\
{}+
\sum\limits_{i=1}^n\sum\limits_{\substack{j=1 \\  j\neq i}}^n \mathrm{cov}\left( F[x_i, T_m], F[x_j, 
T_m] \right).
\end{multline*}

Поскольку $\mu \in m_p[\eta_n]$,
\begin{equation}
\left.
\begin{array}{l}
 \displaystyle\sum\limits_{i: |\mu_i| > 1/T_1} {\mathsf D} F[x_i, T_m]  \leq{}\\
 \hspace*{15mm}{}\leq  4\left(\sigma^2 + T_m^2\right)^2 n \eta_n^p 
T_1^p = o(n);
\\[6pt]
\displaystyle \sum\limits_{\substack{{i,j: \max\{|\mu_i|, |\mu_j|\} > 1/T_1,}\\{j\neq i}}}  \hspace*{-12mm}\mathrm{cov}\,(F[x_i, 
T_m],F[x_j, T_m])  \leq{}\\
\hspace*{10mm}{}\leq 16\left(\sigma^2 + T_m^2\right)^2 n \eta_n^p T_1^p c_5 = o(n). 
\end{array}
\right\}    
\label{D2}
\end{equation}
Далее, учитывая что ${\mathsf D} x_i^2 \hm= 2\sigma^4 \hm+ 4\sigma^2 \mu_i^2$, нетрудно 
убедиться, что
\begin{multline}
\label{D3}
\sum\limits_{i: |\mu_i| \leq 1/T_1}\hspace*{-4mm} {\mathsf D} F[x_i, T_m] ={}\\
{}= \sum\limits_{i: |\mu_i| \leq 1/T_1} \hspace*{-4mm} {\mathsf D} 
x_i^2 + o(n) = 2\sigma^4 n + o(n).
\end{multline}


Введем обозначение 
$$
D_n = \left\{(i,j) : \max\left\{|\mu_i|, |\mu_j|\right\}  \leq \fr{1}{T_1}\,, \enskip j\hm\neq i\right\}.
$$
 Для суммы ковариаций аналогично~(\ref{D3}) получим
\begin{multline*}
\sum\limits_{(i,j)\in D_n} \hspace*{-2mm}\mathrm{cov}\left( F[x_i, T_m], F[x_j, T_m] \right) = {}\\
{}=
\sum\limits_{(i,j)\in D_n} \hspace*{-2mm}\mathrm{cov}\left( x_i^2, x_j^2 \right) + o(n).
\end{multline*}
Воспользуемся тождеством~\cite{Eroshenko}
$$
\mathrm{cov}\left (x_i^2, x_j^2\right) = 4 {\mathsf E} x_i {\mathsf E} x_j \mathrm{cov}\left(x_i, x_j\right) + 2 \mathrm{cov}^2 \left(x_i, x_j\right)
$$
для вектора $(x_i, x_j)$, имеющего двумерное нормальное распределение. Заметим, 
что
\begin{gather*}
 \sum\limits_{(i,j)\in D_n} 4 | {\mathsf E} x_i {\mathsf E} x_j \mathrm{cov}\left(x_i, x_j\right)| \leq 8 T_1^{-2} 
\sigma^2 n c_5 = o(n);
\\
\sum\limits_{(i,j)\in D_n} 2 \mathrm{cov}^2 (x_i, x_j)  = 2\sigma^4 \sum\limits_{(i,j)\in D_n} 
\mathrm{corr}^2 (x_i, x_j). 
\end{gather*}
Более того, поскольку  %< 4 \sigma^2 n c_5.$$
\begin{equation*}
\sum\limits_{\substack{{i,j: \max\{|\mu_i|, |\mu_j|\} > 1/T_1} \\ {j\neq i}}}
\hspace*{-10mm}\mathrm{corr}^2 (x_i, x_j)  
\leq  4 n \eta_n^p T_1^p c_5 =  o(n),
\end{equation*}
имеем
\begin{multline*}
\sum\limits_{(i,j)\in D_n} \mathrm{corr}^2 (x_i, x_j) ={}\\
{}= \sum\limits_{j\neq i} \mathrm{corr}^2 (x_i, x_j) 
+o(n)= c_6 n + o(n),
\end{multline*}
где
$$
c_6 = \lim\limits_{n\to\infty} \fr{1}{n} \sum\limits_{j\neq i} \mathrm{corr}^2 (x_i, x_j) 
\leq 2 c_5.
$$
Полагая $C_\rho \hm= \sigma^2\sqrt{1 + c_6}$, получим, наконец,
\begin{equation}
\label{D1}
{\mathsf D} \left(\hat{R}(T_m) - R(T_m)\right)  =  2 n C_\rho^2 + o(n).
\end{equation}
Заметим, что из~(\ref{D2}), (\ref{D3}) и~(\ref{D1}) следует, что
\begin{equation}
\label{D5}
\sup\limits_{n} \fr{\sum\nolimits_{i=1}^n {\mathsf D} F[x_i, T_m]}{V_n^2} < \infty\,,
\end{equation}
где 
$$
V_n^2 = {\mathsf D} \sum\limits_{i=1}^n \left(F[x_i, T_m] \hm- {\mathsf E} F[x_i, T_m]\right).
$$
Кроме того, поскольку $F[x_i, T_m]$ по модулю ограничены величиной $\sigma^2 \hm+ 
T_m^2$, выполнено условие Линдеберга: для любого $\eps\hm>0$ при $n \hm\to \infty$
\begin{multline}
\label{D6}
\!\!\!\fr{1}{V_n^2}\sum\limits_{i=1}^n {\mathsf E} \left( \!\left( F\left[x_i, T_m\right]\! -\! {\mathsf E} F\left[x_i, T_m\right]\right)^2 
\Ik \left(\vert F\left[x_i, T_m\right] -{}\right.\right.\hspace*{-2.69505pt}\\
\left.\left.{}- {\mathsf E} F\left[x_i, T_m\right]\vert >\eps V_n\right)\!
\vphantom{\left( F\left[x_i, T_m\right]\! -\! {\mathsf E} F\left[x_i, T_m\right]\right)^2}
\right) 
\to  0\,.
\end{multline}
Из~(\ref{D1})--(\ref{D6}), очевидного неравенства
$$ 
\lim\limits_{k\to\infty} \sup\limits_{n\geq k+1}\rho(k) \equiv 
\lim\limits_{k\to\infty} \rho^\star (k)  < 1
$$
 и~центральной предельной теоремы для сильно перемешанных случайных величин~\cite{Peligrad} следует~(\ref{D0}).

Перейдем к~доказательству того, что $U(\hat{t}_F) \, n^{-1/2} \overset{\, \p \, }{\to} 0$.
Всюду далее, не ограничивая общности, полагаем $\sigma=1$. 
Введем обозначения:

\noindent
\begin{align*}
S_1(T) &= \sum\limits_{i: |\mu_i| > 1/T_1} H_i(T, T_m); \\
S_2(T) &= \sum\limits_{i: |\mu_i| \leq 1/T_1} H_i(T, T_m); 
\\
N_1(a, b) &= \sum\limits_{i: |\mu_i| > 1/T_1} \Ik (a<|x_i|\leq b); \\ 
N_2(a, b) &= \sum\limits_{i: |\mu_i| \leq 1/T_1} \Ik (a<|x_i|\leq b);
\end{align*}

\noindent
\begin{align*}
Z_l(T) &= S_l(T) - {\mathsf E} S_l(T),\enskip l = 1,2\,; \\  
d_n &= \fr{T_U -  t_{\kappa_n}}{n};\\
T_j^{\prime} &= t_{\kappa_n}+j d_n,\enskip j = \overline{0,n-1}\,.
\end{align*} 

\vspace*{-3pt}

\noindent
Для произвольного $\eps>0$

\vspace*{-3pt}

\noindent
\begin{multline}
\p \left( \fr{|U(\hat{t}_F)|}{\sqrt{n}}> 4\eps \right) \leq 
\p\left(\hat{t}_F \leq t_{\kappa_n}\right) + {}\\
{}+\p \left(\fr{\sup\nolimits_{T\in 
[t_{\kappa_n}, T_U]} |U(T)|}{\sqrt{n}}>4\eps \right)\leq  {}\\
{}\leq \p\left(\hat{t}_F \leq t_{\kappa_n}\right) + \p\left(\fr{\sup\nolimits_{T\in 
[t_{\kappa_n}, T_U]} |{\mathsf E} U(T)|}{\sqrt{n}}>\eps\right)+{}\\
{}+ \p \left(\sup\limits_{T\in [t_{\kappa_n}, T_U]} |Z_1(T)| > 
\eps\sqrt{n}\right) +{}\\
{}+ \p \left(\sup\limits_{j \in [0, n-1]} |Z_2(T_j^{\prime})| > 
\eps\sqrt{n}\right) +{}\\
{}+ \p \left(\sup\limits_{\substack{j \in [0, n-1] \\
 T\in [T_j^{\prime},T_j^{\prime}+d_n]}} |Z_2(T)-Z_2(T_j^{\prime})| > \eps\sqrt{n}\right).
\label{M1}
\end{multline}
Заметим, что $\gamma_n\hm > \ln^{-1} n$, $\kappa_n\hm > n \eta_n^p \ln ^{-1} n \hm\geq 
n^{1-\delta_1} \ln ^{-1} n$ и~$q_n\hm > c_4 \ln ^{-1} n /2$ для всех достаточно 
больших~$n$.
Для первого слагаемого в~(\ref{M1}) по лемме~1 с~$m \hm= n^{\delta_1} \ln 
^7 n$ для  больших~$n$ имеем

\vspace*{-3pt}

\noindent
\begin{multline}
\label{M1next}
\p\left(\hat{t}_F \leq t_{\kappa_n}\right)  = \p \left(\hat{k}_F \geq \kappa_n 
\right) \leq 4 n e^{-\ln^2 n} + {}\\
{}+n^{1+(3/2)\,\delta_1} \ln^9 n \, 
\alpha\left(\left[\fr{n^{1-\delta_1}}{\ln^{7} n}\right]\right) = o(1)
\end{multline}
при $n\to\infty$. 
Для оценки второго слагаемого в~(\ref{M1}) заметим, что при $T \hm\in 
[t_{\kappa_n}, T_U]$ справедливо
\begin{equation}
\label{M2}
{\mathsf E} H_i(T, T_m) \leq T_U^2 + 1.
\end{equation}
Если же кроме $T \hm\in [t_{\kappa_n}, T_U]$ также выполнено $|\mu_i| \hm\leq T_1^{-1}$, то

\vspace*{-6pt}

\noindent
\begin{multline*}
|{\mathsf E} H_i (T, T_m)| \leq 2 T_U^2 \, \p \left(|x_i| > t_{\kappa_n}\right) \leq {}\\
{}\leq2 
T_U^2 \, \p \left(|x_i-\mu_i| > t_{\kappa_n}-T_1^{-1}\right) \leq{}\\
{}\leq 2 T_U^2  \exp\left\{ -\fr{1}{2} \left(t_{\kappa_n} - T_1^{-
1}\right)^2 \right\}  \leq{}\\
{}\leq
 4 (\ln n)  \exp\left\{ -\fr{1}{2} 
\left(z\left(\fr{q_n\kappa_n}{2n}\right)\right)^2 + t_{\kappa_n} T_1^{-
1}\right\},
\end{multline*}

\vspace*{-2pt}

\noindent
где использовано неравенство 

\noindent
$$
2(1-\Phi(x))\hm \leq \fr{e^{-x^2/2}}{x}
$$

\pagebreak


\noindent
 для $x\hm\geq 0$ 
($\Phi(x)$~--- функция распределения $N(0,1)$). Рас\-смот\-рим выражение 
в~экспоненте. Второе слагаемое не превышает $1\hm+o(1)$ при $n\hm\to\infty$, поскольку 
$t_{\kappa_n} \hm\leq T_1 (1+o(1))$ при $\sigma\hm=1$, что нетрудно получить из 
определения~$t_{\kappa_n}$, пред\-став\-ле\-ния~(\ref{lem1eq2}) и~ограничения на~$q_n$ 
из формулировки тео\-ре\-мы. Для первого слагаемого, используя пред\-став\-ле\-ние~(\ref{lem1eq1}) 
и~ограничения, наложенные на~$q_n$, при больших~$n$ получим
\begin{multline*}
-\fr{1}{2}\left(z\left(\fr{q_n \kappa_n}{2n}\right)\right)^2 \leq - \ln 
\fr{2n (1-q_n-\gamma_n)}{q_n n \eta_n^p T_1^{-p}} + {}\\
{}+\fr{1}{2} \ln 
\left((1+o(1)) \ln \eta_n^{-p}\right) + \fr{3}{2} \leq{}\\
{}\leq \ln \fr{c_3}{1-c_3} + \ln \eta_n^p + \ln T_1^{-p} + \ln T_1 + 
\fr{3}{2}+ o(1).
\end{multline*}
Из приведенных соотношений следует, что с~некоторой константой $c_7 = c_7(c_3, 
p, \delta_1, \delta_2, c_4)$
\begin{equation}\label{M3}
\sup\limits_{\substack{i: |\mu_i| \leq 1/T_1 \\ T\in [t_{\kappa_n}, T_U]}} |{\mathsf E} 
H_i (T, T_m)|  \leq c_7 (\ln n)^{(3-p)/2}\eta_n^p.
\end{equation}
Из (\ref{M2}) и~(\ref{M3}) с~учетом $\delta_2 \hm> 1/2$ следует
\begin{multline*}
\sup\limits_{T\in [t_{\kappa_n}, T_U]} |{\mathsf E} U(T)| \leq{}\\
{}\leq 
 n\eta_n^p T_1^p 
(T_U^2+1) + c_7 (\ln n)^{(3-p)/2} n \eta_n^p = o(\sqrt{n})
\end{multline*}
при $n\to\infty$, а следовательно, для любого $\eps\hm>0$ второе слагаемое в~(\ref{M1}) обращается в~ноль для всех достаточно больших~$n$.

Далее, поскольку при $T \hm\leq T_U$ и~$\sigma\hm=1$
$$
|H_i(T, T_m) - {\mathsf E} H_i(T, T_m)| \leq 2 (T_U^2 +2), \enskip i=\overline{1, n}\,,
$$
а число слагаемых в~$Z_1(T)$ не превосходит $n\eta_n^p T_1^p$, имеем
$$
\sup\limits_{T\in [t_{\kappa_n}, T_U]} |Z_1(T)|  \leq 2 n\eta_n^p T_1^p (T_U^2 
+2) = o(\sqrt{n})
$$
при $n\to\infty$, а следовательно, для любого $\eps\hm>0$ и~третье слагаемое в~(\ref{M1}) обращается в~ноль для всех достаточно больших~$n$.

Перейдем к~оценке четвертого слагаемого в~(\ref{M1}). Аналогично~(\ref{M3}) 
можно получить:
\begin{multline}
\label{M10}
\!\!\sup\limits_{\substack{i: |\mu_i| \leq 1/T_1 \\ T\in [t_{\kappa_n}, T_U]}} \!{\mathsf D} 
H_i (T, T_m)  \leq \!\sup\limits_{\substack{i: |\mu_i| \leq 1/T_1 \\ T\in 
[t_{\kappa_n}, T_U]}} \!{\mathsf E} \left(H_i (T, T_m)\right)^2  \leq{}\\
{}\leq 2 c_7 (\ln n)^{(5-p)/2} \eta_n^p.
\end{multline}
По лемме~4 с~$m \hm= \sqrt{n} (\ln n)^3$ и~$k \hm= n-[n\eta_n^p T_1^p]$ 
для четвертого слагаемого в~(\ref{M1}) имеем:

\noindent
\begin{multline}
\p \left(\sup\limits_{j \in [0, n-1]} |Z_2(T_j^\prime)| > \eps\sqrt{n}\right) 
\leq {}\\
{}\leq \sum\limits_{j \in [0, n-1]} \hspace*{-3mm}\p \left( |Z_2(T_j^\prime)| > \varepsilon\sqrt{n}\right)\leq{}\\
{}\leq 4 n \exp \left\{ - \fr{\eps^2 n^{3/2} (\ln n)^3}{n-[n\eta_n^p T_1^p]}\!\Bigg/\! \big( 8 (T_U^2+2)\eps\sqrt{n} +{}\right.\\
\left.{}+ 128 c_7 (\ln n)^{(5-p)/2} \eta_n^p  (n-
[n\eta_n^p T_1^p])\big) 
\vphantom{ \fr{\eps^2 n^{3/2} (\ln n)^3}{n-[n\eta_n^p T_1^p]}}
\right\} +{}\\
{}
+ 22 \left(1+\fr{8(T_U^2+2) (n-[n\eta_n^p T_1^p])}{\eps 
\sqrt{n}}\right)^{1/2}\times{}\\
{}\times n^{3/2} (\ln n)^3 \alpha\left(\left[\fr{n-[n\eta_n^p 
T_1^p]}{2 (\ln n)^3 \sqrt{n}}\right]\right).
\label{M5}
\end{multline}
Используя ограничения $n^{-\delta_1}\hm\leq \eta_n^p \leq n^{-\delta_2}$ 
и~$1/2\hm<\delta_2\hm<\delta_1\hm<1$, из~(\ref{M5}) получим для любого $\eps\hm>0$
$$
\p \left(\sup\limits_{j \in [0, n-1]} |Z_2(T_j^\prime)| > \eps\sqrt{n}\right) 
\to 0
$$
при $n \to \infty$.

Рассмотрим, наконец, пятое слагаемое в~(\ref{M1})). Заметим, что при $0\hm< a \hm< b$ 
справедливо
$$
|Z_2(b)-Z_2(a)| \leq 2 |N_2(a,b)-{\mathsf E} N_2(a,b)| + n (b^2-a^2).
$$
Полагая $a = T_j^\prime$, $b \hm= T \hm\in [T_j^\prime, T_j^\prime+d_n]$ для 
произвольного $j \hm\in [0, n-1]$ и~учитывая, что
$$
(T^2 - (T_j^\prime )^2) = (T - T_j^\prime)(T+ T_j^\prime ) \leq  2 d_n T_U < 2 
T_U^2 n^{-1}; 
$$

\vspace*{-12pt}

\noindent
\begin{multline*}
\p\left(T_j^\prime < |x_i| \leq T \right) \leq \p\left(T_j^\prime < |x_i| \leq 
T_j^\prime+d_n\right) <{}\\
{}< d_n < T_U n^{-1}, 
\end{multline*}
получим  оценку
$$
|Z_2(T)-Z_2(T_j^\prime)| \leq 2 N_2(T_j^\prime, T) +  3 T_U^2 .
$$
Далее, поскольку $N_2 (T_j^\prime, T) \hm\leq N_2 (T_j^\prime, T_j^\prime+d_n)$ и~${\mathsf E} N_2 (T_j^\prime, T_j^\prime+d_n) \hm< T_U^2$,
имеем
\begin{multline*}
\sup\limits_{T \in [T_j^\prime, T_j^\prime+d_n]} |Z_2(T)-Z_2(T_j^\prime)| \leq {}\\
{}\leq
2 \left|N_2 (T_j^\prime, T_j^\prime+d_n) - {\mathsf E} N_2 (T_j^\prime, 
T_j^\prime+d_n)\right| +  5 T_U^2 .
\end{multline*}
Аналогично~(\ref{M3}) показывается, что
\begin{multline}
\label{M11}
\sup\limits_{\substack{i : |\mu_i| \leq 1/T_1 \\ j \in [0, n-1]}} {\mathsf D} \Ik 
(T_j^\prime < |x_i| \leq T_j^\prime + d_n) <{}\\
{}< c_7 (\ln n)^{(1-p)/2} \eta_n^p.
\end{multline}
Пусть $n > N(\eps)$ настолько, что 
$$
\fr{\eps\sqrt{n} - 5 T_U^2}{2} > \fr{\eps \sqrt{n} }{4}\,.
$$
%
 Тогда для пятого слагаемого в~(\ref{M1}) по лемме~4 с~$m \hm= 
\sqrt{n} (\ln n)^2$ и~$k \hm= n\hm-[n\eta_n^p T_1^p]$ имеем
\begin{multline}
\p \left(\sup\limits_{\substack{j \in [0, n-1] \\ T\in 
[T_j^{\prime},T_j^{\prime}+d_n]}} |Z_2(T)-Z_2(T_j^{\prime})| > 
\eps\sqrt{n}\right) \leq{}\\
{}\leq  \sum\limits_{j \in [0, n-1]} \p \left(  \left|N_2 (T_j^\prime, 
T_j^\prime+d_n) -{}\right.\right.\\
\left.\left.{}- {\mathsf E} N_2 (T_j^\prime, T_j^\prime+d_n)\right| > \fr{\eps\sqrt{n}}{4} 
\right) \leq{}\\
{}\leq  4n \exp \left\{ -  \fr{\eps^2 n^{3/2} (\ln n)^2}{(n-[n\eta_n^p T_1^p])^{-1}}\Bigg/ 
\big( 16 \eps \sqrt{n} +{}\right.\\
\left.{}+ 64 c_7 (\ln n)^{(1-p)/2} \eta_n^p (n-[n\eta_n^p 
T_1^p]) \big) 
\vphantom{\fr{\eps^2 n^{3/2} (\ln n)^2}{(n-[n\eta_n^p T_1^p])^{-1}}}
\right\} +{}\\
{}+ 22 \left(1+\fr{16 (n-[n\eta_n^p T_1^p])}{\eps \sqrt{n}}\right)^{1/2}\times{}\\
{}\times 
n^{3/2} (\ln n)^2 \alpha\left(\left[\fr{n-[n\eta_n^p T_1^p]}{2 (\ln n)^2 
\sqrt{n}}\right]\right).
\label{M6}
\end{multline}
Используя ограничения $n^{-\delta_1}\hm\leq \eta_n^p\hm \leq n^{-\delta_2}$ 
и~$1/2\hm<\delta_2\hm<\delta_1<1$, из~(\ref{M6}) получим для любого $\eps\hm>0$
$$
\p \left(\sup\limits_{\substack{j \in [0, n-1] \\ T\in 
[T_j^{\prime},T_j^{\prime}+d_n]}} |Z_2(T)-Z_2(T_j^{\prime})| > 
\eps\sqrt{n}\right) \to 0
$$
при $n \to \infty$.

Таким образом, показано, что для любого $\eps>0$ все слагаемые в~(\ref{M1}) 
стремятся к~нулю при $n\to\infty$. Следовательно,
$$
\fr{|U(\hat{t}_F)|}{\sqrt{n}}  \overset{\, \p \, }{\to} 0 \,,
$$
что вместе с~(\ref{D0}) завершает доказательство тео\-ремы.~\hfill$\square$

\smallskip

Следующая теорема дает достаточные условия для асимптотической нормальности 
оценки риска $\hat{R}(\hat{t}_F)$ в~случае $\mu \hm\in l_0[\eta_n]$.

\smallskip

\noindent
\textbf{Теорема 2.}\ 
\textit{Пусть $\mu \hm\in l_0[\eta_n]$, $\eta_n\hm\in[n^{-\delta_1}, n^{-\delta_2}]$, $1/2\hm < 
\delta_2\hm < \delta_1\hm<1;$ имеются такие константы $c_1, c_2\hm>0$, что для 
коэффициента сильного перемешивания $\alpha(\cdot)$ компонент вектора~$x$ 
справедливо} 
\begin{gather*}
\alpha(k) \leq c_1 k^{-1-(5/2)\delta_1/(1\hm-\delta_1)\hm-c_2},\enskip 
k=\overline{1,n-1};\\
 q_n < c_3 < 1;\enskip \mathrm{lim\,inf} q_n \ln n = c_4 > 0;
\end{gather*}
\textit{для максимального коэффициента корреляции~$\rho(\cdot)$ компонент вектора~$x$ 
справедливо}
$$
\sum\limits_{k = 1}^{\infty} \sup\limits_{n\geq k+1} \rho(k) \equiv 
\sum\limits_{k = 1}^{\infty}  \rho^\star (k) = c_5 < \infty. 
$$
\textit{Тогда при $n \to \infty$}
$$
\fr{\hat{R}(\hat{t}_F) - R(T_m)}{C_\rho \sqrt{2n}} \Rightarrow N(0, 1),
$$
\textit{где}
$$
C_\rho = \sigma^2\sqrt{1 +   \lim\limits_{n\to\infty} \fr{1}{n} 
\sum\limits_{j\neq i} \mathrm{corr}^2 (x_i, x_j)}\,.
$$

\noindent
Д\,о\,к\,а\,з\,а\,т\,е\,л\,ь\,с\,т\,в\,о\  проводится аналогично доказательству теоремы~1. 
Переменная~$D_n$ теперь определяется как $D_n \hm= \{(i,j) : 
|\mu_i|\hm=|\mu_j|=0$, $j\hm\neq i\}$. Условия вида $|\mu_i|\hm<T_1^{-1}$ (вида 
$|\mu_i|\hm\geq T_1^{-1}$) заменяются условиями  $\mu_i\hm=0$ (соответственно 
$|\mu_i|\hm>0$).
Поскольку $\mu \hm\in l_0[\eta_n]$, количество~$i$ таких, что $|\mu_i|\hm>0$ 
(а~значит, и~число слагаемых в~$Z_1(T)$), не превышает~$[n \eta_n]$.

Для оценки первого слагаемого в~(\ref{M1}) используется лемма~2, 
в~которой можно взять, например, $b\hm=1/2$, а~для~$\kappa_n^0$ использовать оценку 
$\kappa_n^0 \hm> n \eta_n$. Формулы (\ref{M3}),  (\ref{M10}) и~(\ref{M11}) 
принимают вид соответственно
\begin{align*}
\sup\limits_{\substack{i: \mu_i =0 \\ T\in [t_{\kappa_n^0}, T_U]}} |{\mathsf E} H_i (T, 
T_m)| & \leq c_8 (\ln n)^{3/2} \eta_n ;
\\
\sup\limits_{\substack{i: \mu_i =0 \\ T\in [t_{\kappa_n^0}, T_U]}} {\mathsf D} H_i (T, 
T_m)  & \leq 2 c_8 (\ln n)^{5/2} \eta_n;
\\
\sup\limits_{\substack{i : \mu_i =0 \\ j \in [0, n-1]}} {\mathsf D} \Ik (T_j^\prime < 
|x_i| \leq T_j^\prime + d_n) &< c_8 (\ln n)^{1/2} \eta_n,
\end{align*}
где $c_8 = c_8(c_3,\delta_1, \delta_2, c_4)$. В~остальном доказательство 
аналогично.~\hfill$\square$

\section{Сильная состоятельность оценки риска при~применении FDR-процедуры 
в~условиях слабой зависимости}

Следующая теорема дает достаточные условия для сильной состоятельности оценки 
риска $\hat{R}(\hat{t}_F)$ в~случаях $\mu \hm\in m_p[\eta_n]$ и~$\mu \hm\in 
l_0[\eta_n]$.

\smallskip

\noindent
\textbf{Теорема 3.}
\textit{Пусть $\mu\hm \in m_p[\eta_n]$, $\eta_n^p\hm\in[n^{-\delta_1}, n^{-\delta_2}]$ либо 
$\mu \hm\in l_0[\eta_n]$, $\eta_n\hm\in[n^{-\delta_1}, n^{-\delta_2}]$; $0 \hm< \delta_2 
\hm< \delta_1<1$; имеются такие константы $c_1, c_2\hm>0$, что для коэффициента 
сильного перемешивания $\alpha(\cdot)$ компонент вектора~$x$ справедливо}  
$\alpha(k) \hm\leq c_1 k^{-2-(7/2)\delta_1/(1\hm-\delta_1)\hm-c_2}$, $k\hm=\overline{1,n-1}$; 
$q_n \hm< c_3 \hm< 1$; $\mathrm{lim\,inf} q_n \ln n \hm= c_4 \hm> 0$. \textit{Тогда при} $n \hm\to \infty$
$$
\fr{\hat{R}(\hat{t}_F) - R(T_m)}{n} \rightarrow 0 \, \, \,\textit{п.~в.}
$$


\noindent
Д\,о\,к\,а\,з\,а\,т\,е\,л\,ь\,с\,т\,в\,о\,.  Воспользуемся представлением~(\ref{D00}).

Покажем, что $(\hat{R}(T_m)-R(T_m))n^{-1}\hm \to 0$ п.~в.\ при $n\hm\to\infty$. 
При мягкой пороговой обработке ${\mathsf E} \hat{R}(T_m) \hm= R(T_m)$, а~при жесткой 
пороговой обработке
\begin{multline*}
\fr{\hat{R}(T_m)-R(T_m)}{n} = {}\\
{}=\fr{\hat{R}(T_m)-{\mathsf E} \hat{R}(T_m)}{n} 
+\fr{{\mathsf E}\hat{R}(T_m)-R(T_m)}{n}\,,
\end{multline*}
где второе слагаемое стремится к~нулю при $n\to\infty$ \cite{Mallat}. 
Следовательно, достаточно показать, что $(\hat{R}(T_m)\hm-{\mathsf E}\hat{R}(T_m))n^{-1} \hm\to 0$ п.~в.

Полагая в~лемме~3 $X_i \hm= F[x_i, T_m] \hm- {\mathsf E} F[x_i, T_m]$, $b \hm= 
2(\sigma^2\hm+T_m^2)$ и~$m \hm= n^{1/4}$ и~учитывая ограничения на $\alpha(\cdot)$ из 
условия, нетрудно убедиться, что для всех~$n$
$$
\p \left(\left| \fr{\hat{R}(T_m)-{\mathsf E} \hat{R}(T_m)}{n}\right| >\eps \right) 
\leq \fr{c_5}{n^{1+c_6}}\,, 
$$
где константы $c_5$, $c_6$ положительны. Отсюда
$$
\sum\limits_{n=1}^{\infty}\p \left(\left|\fr{\hat{R}(T_m)-{\mathsf E} 
\hat{R}(T_m)}{n}\right| >\eps \right) < \infty,
$$
и по теореме~1.3.4 из~\cite{Serfling2002} 
$$
\left(\hat{R}(T_m)-{\mathsf E}\hat{R}(T_m)\right)n^{-1} \to 0~\mbox{п.~в.}
$$



Покажем теперь, что  $U(\hat{t}_F) \, n^{-1}\hm \to 0$ п.~в. Доказательство 
проведено для $\mu \hm\in m_p[\eta_n]$, в~случае $\mu\hm \in l_0[\eta_n]$ 
доказательство аналогично.
Аналогично формуле~(\ref{M1}), для произвольного $\eps\hm>0$ в~терминах тео\-ре\-мы~1 имеем
\begin{multline*}
\p \left( \fr{|U(\hat{t}_F)|}{n}> 4\eps \right) \leq \p\left(\hat{t}_F 
\leq t_{\kappa_n}\right) +{}\\
{}+ \p\left(\fr{\sup\nolimits_{T\in [t_{\kappa_n}, T_U]} |{\mathsf E} 
U(T)|}{n}>\eps\right)+{}\\
{}+ \p \left(\sup\limits_{T\in [t_{\kappa_n}, T_U]} |Z_1(T)| > \eps n\right) +{}
\end{multline*}

\noindent
\begin{multline}
{}+ \p  \left(\sup\limits_{j \in [0, n-1]} |Z_2(T_j^{\prime})| > \eps n\right) +{}\\
{}+ \p \left(\sup\limits_{\substack{j \in [0, n-1] \\ T\in 
[T_j^{\prime},T_j^{\prime}+d_n]}} |Z_2(T)-Z_2(T_j^{\prime})| > \eps n\right).
\label{M1SC}
\end{multline}
Применяя рассуждения, аналогичные приведенным в~доказательстве теоремы~1, можно показать, что
$$
\sup\limits_{T\in [t_{\kappa_n}, T_U]} |{\mathsf E} U(T)| = o(n); \enskip
\sup\limits_{T\in [t_{\kappa_n}, T_U]} |Z_1(T)|  = o(n),
$$
откуда следует, что второе и~третье слагаемые в~(\ref{M1SC}) обращаются в~ноль 
для всех достаточно больших~$n$.

Для некоторых положительных констант  $c_7$ и~$c_8$ первое, четвертое и~пятое 
слагаемые  в~(\ref{M1SC}) не превышают $c_7 n^{-1-c_8}$ для всех достаточно 
боль\-ших~$n$, что можно показать с~помощью ограничения на $\alpha(\cdot)$ из 
условия и~рассуждений, аналогичных приведенным при выводе соответственно формул~(\ref{M1next}), (\ref{M5}) и~(\ref{M6}), с~тем отличием, что при применении 
леммы~4 полагается $m \hm= (\ln n)^3$.

Из доказанного следует, что
$$
\sum\limits_{n=1}^{\infty}\p \left( \fr{|U(\hat{t}_F)|}{n}> 4\eps \right) 
< \infty,
$$
и по теореме~1.3.4 из~\cite{Serfling2002} $U(\hat{t}_F) \, n^{-1} \to 0$ п.~в., 
что завершает доказательство теоремы.~\hfill$\square$



{\small\frenchspacing
 {\baselineskip=11.5pt
 %\addcontentsline{toc}{section}{References}
 \begin{thebibliography}{99}
\bibitem{FDRImage}
\Au{Krylov V.\,A., Moser~G., Serpico~S.\,B., Zerubia~J.}
False discovery rate approach to unsupervised image change detection~// IEEE 
T. Image Process., 2016. Vol.~25. No.\,10. P.~4704--4718. doi: 10.1109/TIP.2016.2593340.

\bibitem{MultipleTesting} %2
\Au{Menyhart~O., Weltz~B., Gyorffy~B.}
MultipleTesting.com: A~tool for life science researchers for multiple hypothesis 
testing correction~// PLoS One, 2021. Vol.~16. No.\,6. Art.~0245824. doi: 10.1371/journal.pone.0245824.

\bibitem{AdaptingFDR} %3
\Au{Abramovich~F., Benjamini~Y., Donoho~D., Johnstone~I.}
Adapting to unknown sparsity by controlling the false discovery rate~// Ann. Stat., 2006. Vol.~34. No.\,2. P.~584--653.
doi: 10.1214/009053606000000074.

\bibitem{ZasShe17} %4
\Au{Заспа~А.\,Ю., Шестаков~О.\,В.}
Состоятельность оценки риска при множественной проверке гипотез с~FDR-по\-ро\-гом~// 
Вестник ТвГУ. Сер. Прикладная математика, 2017. Вып.~1. С.~5--16.
doi: 10.26456/vtpmk119. EDN: YFYJXT.

\bibitem{Mathematics2020} %5
\Au{Palionnaya~S.\,I., Shestakov~O.\,V.}
Asymptotic properties of MSE estimate for the false discovery rate controlling 
procedures in multiple hypothesis testing // Mathematics, 2020. Vol.~8. No.~11. 
Art.~1913. 11~p. doi: 10.3390/ math8111913.

\bibitem{Shestakov2021-1} %6
\Au{Шестаков~О.\,В.}
Анализ несмещенной оценки среднеквадратичного риска метода блочной пороговой 
обработки~// Информатика и~её применения, 2021. Т.~15. Вып.~2. С.~30--35.
doi: 10.14357/19922264210205. EDN: DSQQAU.

\bibitem{Shestakov2021-2} %7
\Au{Шестаков~О.\,В.}
Пороговые функции в~методах подавления шума, основанных на вейв\-лет-раз\-ло\-же\-нии 
сигнала~// Информатика и~её применения, 2021. Т.~15. Вып.~3. С.~51--56.
doi: 10.14357/19922264210307. EDN: WSEAYG.

\bibitem{Shestakov2022} %8
\Au{Шестаков~О.\,В.}
Несмещенная оценка риска пороговой обработки с~двумя пороговыми значениями~// 
Информатика и~её применения, 2022. Т.~16. Вып.~4. С.~14--19.
doi: 10.14357/19922264220403. EDN: \mbox{DZBVLC}.

\bibitem{ResultsOnFDRUnderDependence} %9
\Au{Farcomeni~A.}
Some results on the control of the false discovery rate under dependence~// 
Scand. J. Stat., 2007. Vol.~34. No.\,2. P.~275--297.
doi: 10.1111/j.1467-9469.2006.00530.x.

\bibitem{VorontsovShestakov2023} %10
\Au{Воронцов~М.\,О., Шестаков~О.\,В.}
Среднеквадратичный риск FDR-про\-це\-ду\-ры в~условиях слабой за\-ви\-си\-мости~// 
Информатика и~её применения, 2023. Т.~17. Вып.~2. С.~34--40.
doi: 10.14357/19922264230205. EDN: AVJZDX.

\bibitem{Vorontsov2024} %11
\Au{Воронцов~М.\,О.}
Анализ среднеквадратичного риска при использовании методов множественной 
проверки гипотез для выбора параметров пороговой обработки в~условиях слабой 
зависимости~// Вестник Московского университета. Сер. 15: Вычислительная 
математика и~кибернетика, 2024. №\,2. С.~18--24.

\bibitem{Bosq} %12
\Au{Bosq~D.}
Nonparametric statistics for stochastic processes: Estimation and prediction.~--- 
Lecture notes in statistics ser.~--- New York, NY, USA: Springer, 1996. Vol.~110. 
188~p.

\bibitem{Mallat} %13
\Au{Mallat~S.}
A wavelet tour of signal processing.~--- New York, NY, USA: Academic Press, 1999. 
857~p.

\bibitem{spatialAdaptation} %14
\Au{Donoho~D., Johnstone~I.}
Ideal spatial adaptation via wavelet shrinkage~// Biometrika, 1994. Vol.~81. 
No.\,3. P.~425--455. doi: 10.1093/biomet/81.3.425.

\bibitem{AdaptingSURE} %15
\Au{Donoho D., Johnstone I.\,M.}
Adapting to unknown smoothness via wavelet shrinkage~// J.~Amer. Stat. Assoc., 
1995. Vol.~90. P.~1200--1224.

\bibitem{ExactRisk} %16
\Au{Marron J.\,S., Adak~S., Johnstone~I.\,M., Neumann~M.\,H., Patil~P.}
Exact risk analysis of wavelet regression~// J.~Comput. Graph. Stat., 1998. 
Vol.~7. P.~278--309. doi: 10.1080/ 10618600.1998.10474777.

\bibitem{Jansen} %17
\Au{Jansen~M.}
Noise reduction by wavelet thresholding.~-- Lecture notes in statistics ser.~--- 
New York, NY, USA: Springer, 2001. Vol.~161. 217~p.

\bibitem{KuShe2016_1} %18
\Au{Кудрявцев~А.\,А., Шестаков~О.\,В.}
Асимптотическое поведение порога, минимизирующего усредненную\linebreak вероятность ошибки 
вычисления вейв\-лет-ко\-эф\-фи\-ци\-ен\-тов~// Докл. Акад. наук, 2016. Т.~468. №\,5. 
С.~487--491.

\bibitem{KuShe2016_2} %19
\Au{Кудрявцев~А.\,А., Шестаков~О.\,В.}
Асимптотически оптимальная пороговая обработка вейв\-лет-ко\-эф\-фи\-ци\-ен\-тов в~моделях с~негауссовым распределением шума~// Докл. Акад. наук, 2016. Т.~471. №\,1. 
С.~11--15.



\bibitem{Eroshenko} %20
\Au{Ерошенко~А.\,А.}
Статистические свойства оценок сигналов и~изображений при пороговой обработке 
коэффициентов в~вейв\-лет-раз\-ло\-же\-ни\-ях: Дис.\ \ldots\ канд. физ.-мат. наук.~--- 
М.: МГУ, 2015. 82~с.

\bibitem{Peligrad} %21
\Au{Peligrad~M.}
On the asymptotic normality of sequences of weak dependent random variables~// 
J. Theor. Probab., 1996. Vol.~9. No.\,3. P.~703--715. doi: 10.1007/BF02214083.

\bibitem{Serfling2002} %22
\Au{Serfling~R.\,J.}
Approximation theorems of mathematical statistics.~--- New York, NY, USA: John Wiley \&~Sons, Inc., 2002. 371~p.

\end{thebibliography}

 }
 }

\end{multicols}

\vspace*{-6pt}

\hfill{\small\textit{Поступила в~редакцию 21.05.24}}

\vspace*{8pt}

%\pagebreak

%\newpage

%\vspace*{-28pt}

\hrule

\vspace*{2pt}

\hrule



\def\tit{ASYMPTOTIC NORMALITY AND STRONG CONSISTENCY\\ OF~RISK ESTIMATE WHEN USING THE~FDR THRESHOLD\\ UNDER WEAK DEPENDENCE CONDITION}


\def\titkol{Asymptotic normality and strong consistency of~risk estimate when using the~FDR threshold under weak dependence condition}


\def\aut{M.\,O.~Vorontsov$^{1,2}$ and~O.\,V.~Shestakov$^{1,2,3}$}

\def\autkol{M.\,O.~Vorontsov and~O.\,V.~Shestakov}

\titel{\tit}{\aut}{\autkol}{\titkol}

\vspace*{-13pt}


\noindent
$^{1}$Department of Mathematical Statistics, Faculty of Computational Mathematics and Cybernetics,
 M.\,V.~Lo\-mo-\linebreak
 $\hphantom{^1}$nosov Moscow State University, 1-52~Leninskie Gory, GSP-1, Moscow 119991, Russian Federation

\noindent
$^{2}$Moscow Center for Fundamental and Applied Mathematics, M.\,V.~Lomonosov Moscow State University,\linebreak
$\hphantom{^1}$1~Leninskie Gory, GSP-1, Moscow 119991, Russian Federation

\noindent
$^{3}$Federal Research Center ``Computer Science and Control'' of the Russian Academy of Sciences, 44-2~Vavilov\linebreak
$\hphantom{^1}$Str., Moscow 119333, Russian Federation


\def\leftfootline{\small{\textbf{\thepage}
\hfill INFORMATIKA I EE PRIMENENIYA~--- INFORMATICS AND
APPLICATIONS\ \ \ 2024\ \ \ volume~18\ \ \ issue\ 3}
}%
 \def\rightfootline{\small{INFORMATIKA I EE PRIMENENIYA~---
INFORMATICS AND APPLICATIONS\ \ \ 2024\ \ \ volume~18\ \ \ issue\ 3
\hfill \textbf{\thepage}}}

\vspace*{2pt}






\Abste{An approach to solving the problem of noise removal in a large array of sparse data is considered
 based on the method of controlling the average proportion of false hypothesis rejections (False Discovery Rate, FDR). 
 This approach is equivalent to threshold processing procedures that remove array components whose values do not exceed 
 some specified threshold. The observations in the model are considered weakly dependent. To control the\linebreak\vspace*{-12pt}}
 
 \Abstend{degree of dependence, 
 restrictions on the strong mixing coefficient and the maximum correlation coefficient are used. The mean-square risk is 
 used as a measure of the effectiveness of the considered approach. It is possible to calculate the risk value only on the test data;
  therefore, its statistical estimate is considered in the work and its properties are investigated. The asymptotic normality and
   strong consistency of the risk estimate are proved when using the FDR threshold under conditions of weak dependence in the data.}

\KWE{thresholding; multiple hypothesis testing; risk estimate}

\DOI{10.14357/19922264240309}{ZOQVTO}

%\vspace*{-12pt}


    
   %   \Ack

%\vspace*{-3pt}
%\noindent



  \begin{multicols}{2}

\renewcommand{\bibname}{\protect\rmfamily References}
%\renewcommand{\bibname}{\large\protect\rm References}

{\small\frenchspacing
 {\baselineskip=10.8pt
 \addcontentsline{toc}{section}{References}
 \begin{thebibliography}{99} 

%1
\bibitem{FDRImage-1}
\Aue{Krylov, V.\,A., G.~Moser, S.\,B.~Serpico, and J.~Zerubia.} 2016. 
False discovery rate approach to unsupervised image change detection. 
\textit{IEEE T. Image Process.} 25(10):4704--4718. doi: 10.1109/TIP.2016.2593340.

%2
\bibitem{MultipleTesting-1}
\Aue{Menyhart, O., B.~Weltz, and B.~Gyorffy.} 2021. 
MultipleTesting.com: A~tool for life science researchers for multiple hypothesis testing correction. 
\textit{PLoS One} 16(6):0245824. 
doi: 10.1371/journal.pone.0245824.

%3
\bibitem{AdaptingFDR-1}
\Aue{Abramovich, F., Y.~Benjamini, D.~Donoho, and I.\,M.~Johnstone.} 2006. 
Adapting to unknown sparsity by controlling the false discovery rate. 
\textit{Ann. Stat.} 34(2):584--653. 
doi: 10.1214/009053606000000074.


%4
\bibitem{ZasShe17-1}
\Aue{Zaspa, A.\,Yu., and O.\,V.~Shestakov.} 2017.
Sostoyatel'nost' otsenki riska pri mnozhestvennoy proverke gipotez s~FDR-porogom
 [Consistency of the risk estimate of the multiple hypothesis testing with the FDR threshold]. 
\textit{Vestnik TvGU. Ser.: Prikladnaya matematika} [Herald of Tver State University. Ser. Applied Mathematics] 1:5--16.
doi: 10.26456/vtpmk119. EDN: YFYJXT.

%5
\bibitem{Mathematics2020-1}
\Aue{Palionnaya, S.\,I., and O.\,V.~Shestakov.} 2020. 
Asymptotic properties of MSE estimate for the false discovery rate controlling procedures in multiple hypothesis testing. 
\textit{Mathematics} 8(11):1913. 11~p.
doi: 10.3390/math8111913.

%6
\bibitem{Shestakov2021-1-1}
\Aue{Shestakov, O.\,V.} 2021.
Analiz nesmeshchennoy otsenki srednekvadratichnogo riska metoda blochnoy po\-ro\-go\-voy obrabotki 
[Analysis of the unbiased mean-square risk estimate of the block thresholding method]. 
\textit{Informatika i~ee Primeneniya~--- Inform. Appl.} 15(2):30--35.
doi: 10.14357/19922264210205. EDN: DSQQAU.

%7
\bibitem{Shestakov2021-2-1}
\Aue{Shestakov, O.\,V.} 2021.
Porogovye funktsii v~metodakh podavleniya shuma, osnovannykh na veyvlet-razlozhenii signala 
[Thresholding functions in the noise suppression methods based on the wavelet expansion of the signal]. 
\textit{Informatika i~ee Primeneniya~--- Inform. Appl.} 15(3):51--56.
doi: 10.14357/19922264210307. EDN: WSEAYG.

%8
\bibitem{Shestakov2022-1}
\Aue{Shestakov, O.\,V.} 2022.
Nesmeshchennaya otsenka riska porogovoy obrabotki s dvumya porogovymi znacheniyami [Unbiased thresholding risk estimate with two threshold values]. 
\textit{Informatika i~ee Primeneniya~--- Inform. Appl.} 16(4):14--19.
doi: 10.14357/19922264220403. EDN: DZBVLC.

%9
\bibitem{ResultsOnFDRUnderDependence-1}
\Aue{Farcomeni, A.} 2007. Some results on the control of the false discovery rate under dependence. 
\textit{Scand. J. Stat.} 34(2):275--297. 
doi: 10.1111/j.1467-9469.2006.00530.x.

%10
\bibitem{VorontsovShestakov2023-1}
\Aue{Vorontsov, M.\,O., and O.\,V.~Shestakov.} 2023.
Sred\-ne\-kvad\-ra\-tich\-nyy risk FDR-protsedury v~usloviyakh slaboy za\-vi\-si\-mosti [Mean-square risk of the FDR procedure under weak dependence]. 
\textit{Informatika i~ee Primeneniya~--- Inform. Appl.} 17(2):34--40.
doi: 10.14357/19922264230205. EDN: AVJZDX.

%11
\bibitem{Vorontsov2024-1}
\Aue{Vorontsov, M.\,O.} 2024. 
RMS risk analysis when using multiple hypothesis testing select parameters of thresholding under conditions of weak dependence. 
\textit{Moscow University Computational Mathematics Cybernetics} 48:91--97. 
doi: 10.3103/S027864192470002X.

%12
\bibitem{Bosq-1}
\Aue{Bosq, D.} 1996. 
\textit{Nonparametric statistics for stochastic processes: Estimation and prediction}. 
Lecture notes in statistics ser. New York, NY: Springer Verlag. Vol.~110. 188~p.

%13
\bibitem{Mallat-1}
\Aue{Mallat, S.} 1999. 
\textit{A wavelet tour of signal processing}. New York, NY: Academic Press. 857~p.

%14
\bibitem{spatialAdaptation-1}
\Aue{Donoho, D., and I.\,M.~Johnstone.} 1994. 
Ideal spatial adaptation via wavelet shrinkage. 
\textit{Biometrika} 81(3):425--455. doi: 10.1093/biomet/81.3.425.

%15
\bibitem{AdaptingSURE-1}
\Aue{Donoho, D., and I.\,M.~Johnstone.} 1995. 
Adapting to unknown smoothness via wavelet shrinkage. 
\textit{J. Am. Stat. Assoc.} 90(432):1200--1224. doi: 10.1080/01621459. 1995.10476626.

%16
\bibitem{ExactRisk-1}
\Aue{Marron, J.\,S., S.~Adak, I.\,M.~Johnstone, M.\,H.~Neumann, and P.~Patil.} 1998. 
Exact risk analysis of wavelet regression. 
\textit{J.~Comput. Graph. Stat.} 7(3):278-309. doi: 10.1080/ 10618600.1998.10474777.

%17
\bibitem{Jansen-1}
\Aue{Jansen, M.} 2001. 
\textit{Noise reduction by wavelet thresholding}. Lecture notes in statistics ser. New York, NY: Springer Verlag. Vol.~161. 217~p.

%18
\bibitem{KuShe2016_1-1}
\Aue{Kudryavtsev, A.\,A., and O.\,V.~Shestakov.} 2016. 
Asymptotic behavior of the threshold minimizing the average probability of error in calculation of wavelet coefficients. 
\textit{Dokl. Math.} 93(3):295--299.
doi: 10.1134/S1064562416030212. EDN: WUMUEV. 

%19
\bibitem{KuShe2016_2-1}
\Aue{Kudryavtsev, A.\,A., and O.\,V.~Shestakov.} 2016. 
Asymptotically optimal wavelet thresholding in the models with non-Gaussian noise distributions. 
\textit{Dokl. Math.} 94(3):615--619.
doi: 10.1134/S1064562416060028. EDN: YUYVUP.




%20
\bibitem{Eroshenko-1}
\Aue{Eroshenko, A.\,A.} 2015. Statisticheskie svoystva otsenok signalov i~izobrazheniy pri porogovoy obrabotke ko\-ef\-fi\-tsi\-en\-tov 
v~veyvlet-razlozheniyakh 
[Statistical properties of signal and image estimates under thresholding of coefficients in wavelet decompositions]. Moscow: MSU. PhD Diss. 82~p.

%21
\bibitem{Peligrad-1}
\Aue{Peligrad, M.} 1996. 
On the asymptotic normality of sequences of weak dependent random variables. 
\textit{J. Theor. Probab.} 9(3):703--715. doi: 10.1007/BF02214083.

%22
\bibitem{Serfling2002-1}
\Aue{Serfling, R.\,J.} 2002. 
\textit{Approximation theorems of mathematical statistics}. New York, NY: John Wiley \&~Sons. 371~p.
\end{thebibliography}

 }
 }

\end{multicols}

\vspace*{-6pt}

\hfill{\small\textit{Received May 21, 2024}} 

%\vspace*{-18pt}

\Contr

\vspace*{-3pt}


\noindent
\textbf{Vorontsov Mikhail O.} (b.\ 1996)~--- PhD student, Department of Mathematical Statistics, 
Faculty of Computational Mathematics and Cybernetics, M.\,V.~Lomonosov Moscow State University, 1-52~Leninskie Gory, GSP-1, Moscow 119991, Russian Federation;  
mathematician, Moscow Center for Fundamental and Applied Mathematics, M.\,V.~Lomonosov Moscow State University, 1~Leninskie Gory, GSP-1, Moscow 119991, Russian Federation;
\mbox{m.vtsov@mail.ru}

\vspace*{6pt}

\noindent
\textbf{Shestakov Oleg V.} (b.\ 1976)~--- Doctor of Science in physics and mathematics, professor, Department of Mathematical Statistics,
 Faculty of Computational Mathematics and Cybernetics, M.\,V.~Lomonosov Moscow State University, 1-52~Leninskie Gory, GSP-1, Moscow 119991, Russian Federation; 
 senior scientist, Federal Research Center ``Computer Science and Control'' of the Russian Academy of Sciences, 44-2~Vavilov Str., Moscow 119333, 
 Russian Federation; leading scientist, Moscow Center for Fundamental and Applied Mathematics, M.\,V.~Lomonosov Moscow State University, 
 1~Leninskie Gory, GSP-1, Moscow 119991, Russian Federation; \mbox{oshestakov@cs.msu.su}


\label{end\stat}

\renewcommand{\bibname}{\protect\rm Литература}        %6
\def\stat{kudr}

\def\tit{ПРИБЛИЖЕННЫЕ МЕТОДЫ РЕШЕНИЯ ЗАДАЧИ ДИАГНОСТИКИ ПЛОСКИМ 
ЗОНДОМ СИЛЬНОИОНИЗОВАННОЙ ПЛАЗМЫ С~УЧЕТОМ КУЛОНОВСКИХ 
СТОЛКНОВЕНИЙ}

\def\titkol{Приближенные методы решения задачи диагностики плоским 
зондом сильноионизованной плазмы} %с~учетом Кулоновских  столкновений}

\def\autkol{И.\,А.~Кудрявцева, А.\,В.~Пантелеев}
\def\aut{И.\,А.~Кудрявцева$^1$, А.\,В.~Пантелеев$^2$}

\titel{\tit}{\aut}{\autkol}{\titkol}

%{\renewcommand{\thefootnote}{\fnsymbol{footnote}}\footnotetext[1]
%{Работа поддержана Российским фондом фундаментальных исследований
%(проекты 11-01-00515а и 11-07-00112а), а также Министерством
%образования и науки РФ в рамках ФЦП <<Научные и
%научно-педагогические кадры инновационной России на 2009--2013~годы>>.}}


\renewcommand{\thefootnote}{\arabic{footnote}}
\footnotetext[1]{Московский авиационный институт, irina.home.mail@mail.ru}
\footnotetext[2]{Московский авиационный институт, avpanteleev@inbox.ru}

\vspace*{-2pt}

\Abst{Сформирована математическая модель, описывающая динамику сильноионизованной 
плазмы с учетом столкновений заряженных частиц вблизи плоского зонда. Модель включает уравнение 
Фоккера--Планка и уравнение Пуассона. Предложено два подхода к решению задачи: на основе метода 
статистических испытаний Мон\-те-Кар\-ло и на основе композиции метода крупных частиц и метода 
расщепления.} 

\vspace*{-2pt}

\KW{телекоммуникационные системы; метод Монте-Карло; метод крупных частиц; метод 
расщепления; зонд; уравнение Фоккера--Планка; уравнение Пуассона} 

\vspace*{-4pt}

 \vskip 8pt plus 9pt minus 6pt

      \thispagestyle{headings}

      \begin{multicols}{2}
      
            \label{st\stat}

\section{Введение}

В настоящее время в области телекоммуникаций все более востребованными становятся 
информационные технологии, основанные на использовании математических моделей и численных 
методов физики плазмы. Поэтому особенно актуальным является решение разнообразных задач анализа 
поведения плазмы, включающих в себя формирование новых моделей и методов их исследования. 
Помимо этого, в разработке телекоммуникационного оборудования эффективно используются 
собственно физические свойства плазмы. В~частности, изготовлена антенна, работа которой основана 
на газовом разряде низкотемпературной плазмы~[1], интенсивно ведутся разработки по созданию и 
усовершенствованию источников бесперебойного питания на основе плазменных элементов~[2, 3]. 
      
      Одним из наиболее перспективных направлений для построения систем оптической 
беспроводной связи является использование лазеров~\cite{4-k, 5-k}. В~этой связи большое внимание 
уделяется использованию плазмы при разработке импульсных сильноточных коммутаторов~\cite{6-k}, 
так как практическое применение подобных разработок требует повышения уровня надежности и 
быстродействия лазерных систем.
      
      Исследования низкотемпературной плазмы также связаны с разработками в области дальней 
космической связи, так как моделирование процессов взаимодействия заряженного тела с верхними 
слоями атмосферы позволяет предлагать способы улучшения существующих систем радиосвязи с 
космическими летательными аппаратами~\cite{7-k}. 
      
      Наряду с этим актуальными также являются задачи диагностики плазмы, поскольку перспективы 
ее использования в области телекоммуникаций после более полного изучения физических свойств 
могут значительно расшириться. 

Для диагностики плазмы применяют зондовые методы исследования~[8--11]. Эти методы относятся к 
классу контактных методов; как следствие, возникает сложность в исследовании пристеночной области 
вблизи зонда, которая характеризуется достаточно сложным распределением потенциала и функциями 
распределения, отличными от максвелловских. 

Данная работа посвящена исследованию переходного режима обтекания заряженного тела плазмой. Для 
переходного режима выполняется следующее условие: длина свободного пробега иона до столкновения 
с нейтральным атомом или другим ионом невелика по сравнению с характерными размерами тела. 
В~этом случае возникает необходимость учета столкновений заряженных частиц с нейтральными 
атомами и кулоновских столкновений. В~работах~\cite{10-k, 11-k} подробно рассмотрена модель с 
учетом столкновений заряженных частиц с нейтральными атомами. В~настоящей статье представлена 
теоретическая модель, описывающая влияния ион-ионных и ион-элек\-т\-рон\-ных столкновений на 
измеряемые характеристики плазмы, что ранее детально не исследовалось.
      
      В~рамках данной работы предлагается модель, описывающая динамику сильноионизованной 
плазмы с учетом кулоновских столкновений. Эта модель учитывает такие процессы взаимодействия, 
как перенос частиц и столкновения между заряженными частицами типа <<ион--ион>> и 
      <<ион--электрон>> под влиянием макроскопического электрического поля. Перечисленные 
процессы описываются самосогласованной системой уравнений, включающей уравнение 
      Фок\-ке\-ра--План\-ка и уравнение Пуассона~[12].
      
      Вычислительная модель задачи строится на основе двух методов: метода статистических 
испытаний Мон\-те-Кар\-ло и композиции метода крупных частиц и метода расщепления. Приведены 
результаты численного моделирования, полученные с использованием вышеперечисленных методов.

\vspace*{-4pt}

\section{Постановка задачи}

\vspace*{-2pt}

Рассматривается следующая физическая постановка зондовой задачи~[11]. В~невозмущенную 
бесконечно протяженную плазму, состоящую из электронов и однозарядных ионов, внесена большая\linebreak 
заряженная до потенциала $\varphi_p$ плоскость. Плоскость, расположенная поперек потока плазмы, 
является идеально поглощающей для электронов. Ионы при ударе о плоскость нейтрализуются. 
Предполагается, что частицы в плазме движутся под действием внешнего электрического поля, 
магнитное поле отсутствует. Концентрации ионов $n_{i\infty}$ и электронов $n_{e\infty}$, а также 
температуры данных час\-тиц~$T_{i\infty}$ 
и~$T_{e\infty}$ в невозмущенной плазме заданы. За начальные 
функции распределения обоих типов час\-тиц принимаются функции распределения Максвелла. 
      
      Требуется с учетом столкновений между заряженными частицами найти напряженность 
самосогласованного электрического поля $\vec{E}(\vec{r},t)$, функции распределения однозарядных 
ионов $f_i(\vec{r}, \vec{v}, t)$ и электронов $f_e(\vec{r}, \vec{v}, t)$, 
а также их моменты (плотности 
токов ионов и электронов  $j_i(\vec{r},t)\hm
=q\int f_i(\vec{r}, \vec{v}, t)\vec{v}\,d\vec{v}$, $j_e(\vec{r},t) 
\hm={\sf e}\int f_e(\vec{r},\vec{v},t)\vec{v}\,d\vec{v}$, где $q=Z_i{\sf e}$, $Z_i=1$~--- заряд иона, ${\sf 
e}$~--- заряд электрона; концентрации ионов и электронов $n_i(\vec{r},t)\hm=\int 
f_i(\vec{r},\vec{v},t)\,d\vec{v}$, $n_e(\vec{r},t)\hm=\int f_e(\vec{r},\vec{v}, t)\,d\vec{v}$). 
Поведение частиц во 
времени~$t$ характеризуется ра\-ди\-ус-век\-то\-ром~$\vec{r}$ и вектором скорости~$\vec{v}$.
      
      Математическая модель, соответствующая данной физической постановке задачи, имеет 
вид~\cite{11-k, 13-k}:

\noindent
      \begin{equation}
      \left.
      \begin{array}{c}
      \fr{\partial f_\alpha (\vec{r},\vec{v},t)}{\partial t}+
      \vec{v}\fr{\partial f_\alpha (\vec{r},\vec{v},t)}{ 
\partial \vec{r}}+
\fr{\vec{F}_\alpha(\vec{r},t)}{m_\alpha}\times{}\\[4pt]
{}\times\fr{\partial f_\alpha(\vec{r},\vec{v},t)}{ \partial 
\vec{v}}=
\left(\fr{\partial f_\alpha(\vec{r},\vec{v},t)}{ \partial t}\right)_{\mathrm{с}}+S_\alpha 
(\vec{r},\vec{v},t)\,;\\[6pt]
      \Delta\varphi(\vec{r},t)=-\fr{{\sf e}}{\varepsilon_0}\left( n_i(\vec{r},t)-n_e(\vec{r},t)\right)\,;\\[6pt]
      \vec{E}(\vec{r},t)=-\nabla \varphi(\vec{r},t)\,.
      \end{array}\!\!
      \right\}\!\!
      \label{e1-k}
      \end{equation}
Здесь первое уравнение~--- уравнение Фок\-ке\-ра--План\-ка для частиц сорта~$\alpha$ ($\alpha=i,e$), 
второе~--- уравнение Пуассона для самосогласованного электрического поля; 
$f_\alpha(\vec{r},\vec{v},t)$~--- функция\linebreak
распределения час\-тиц сорта~$\alpha$; $(\partial 
f_\alpha(\vec{r},\vec{v},t)/\partial t)_{\mathrm{с}}$~--- 
оператор столкновений Фок\-ке\-ра--План\-ка; 
функция~$S_\alpha(\vec{r},\vec{v},t)$ описывает источники или стоки\linebreak
 час\-тиц; 
$\vec{F}_\alpha(\vec{r},t)=q_\alpha\vec{E}(\vec{r},t)$, где $\vec{E}(\vec{r},t)$~--- напряженность 
самосогласованного электрического поля, 
$$
q_\alpha =
\begin{cases}
-{\sf e}\,, & \alpha=e\,,\\
{\sf e}\,, & \alpha=i\,;
\end{cases}
$$
$\varphi(\vec{r},t)$~--- потенциал самосогласованного электрического поля; $n_\alpha(\vec{r},t)$ ($\alpha 
\hm=i,e$)~--- концентрация частиц сорта~$\alpha$; $m_\alpha$~--- масса частицы сорта~$\alpha$; 
$\varepsilon_0$~--- электрическая постоянная. 

Оператор столкновений Фок\-ке\-ра--План\-ка имеет вид~\cite{13-k, 14-k}
\begin{multline*}
\fr{1}{\Gamma_\alpha}\left( \fr{\partial f_\alpha}{\partial t}\right)_{\mathrm{с}} 
=\fr{1}{2}\,\nabla_v\nabla_v:\left(f_\alpha\nabla_v\nabla_vg_\alpha(\vec{r},\vec{v},t)\right)-{}\\
{}-
\nabla_v\cdot\left(f_\alpha\nabla_v h_\alpha\right)\,,
\end{multline*}
где $\nabla_v\nabla_v g_\alpha(\vec{r},\vec{v},t)$~--- ковариантная тензорная производная второго ранга, 
знак двоеточия ($:$) обозначает операцию двойного суммирования:
\begin{gather*}
\Gamma_\alpha=\fr{Z_\alpha^4 {\sf e}^4}{4\pi \varepsilon_0^2 m^2_\alpha}\,\ln D_\alpha\,;
\\
D_\alpha =\fr{12\pi\varepsilon_0 kT_{\alpha\infty}}{Z_\alpha^2 {\sf e}^2}\left( \fr{\varepsilon_0 k 
T_{e\infty}}{n_{e\infty} {\sf e}^2}\right)^{1/2}\,;\\
g_\alpha (\vec{r},\vec{v},t)=\sum\limits_{b=i,e}\left( \fr{Z_b}{Z_\alpha}\right) \int f_b 
(\vec{r},{\vec{v}}^{\,\prime},t)\left\vert \vec{v}-{\vec{v}}^{\,\prime}\right\vert\,d\vec{v}^{\,\prime}\,;\\
h_\alpha (\vec{r},\vec{v},t)=\sum\limits_{b=i,e} \fr{m_\alpha+m_b}{m_b} 
\left(\fr{Z_b}{Z_\alpha}\right)
\int
\fr{f_b(\vec{r},{\vec{v}}^{\,\prime}, t)}{\vert \vec{v}-{\vec{v}}^{\,\prime}\vert}
\,d{\vec{v}}^{\,\prime}\,;\\
Z_\alpha =1\,, \quad \alpha=i,e\,.
\end{gather*}
 
К системе уравнений~(\ref{e1-k}) необходимо добавить начальные и краевые условия:
\begin{equation}
\!\left.
\begin{array}{rrl}
t=0:\ & f_\alpha(\vec{r},\vec{v},0)&=f_\alpha^{\mathrm{maksv}}\,,\enskip \alpha=i,e;\\[9pt]
\vec{r}\in \Omega_p:\ & f_\alpha(\vec{r},\vec{v},t)\big\vert_{\vec{r}\in\Omega_p}&=0\,,\enskip \alpha=i,e\,;\\[9pt]
&\varphi(\vec{r},t)\big\vert_{\vec{r}\in\Omega_p}&=\varphi_p\,;\\[9pt]
\vec{r}\in\Omega_\infty:\ & 
f_\alpha(\vec{r},\vec{v},t)\big\vert_{\vec{r}\in\Omega_\infty}&= %{}\\[9pt]
f_\alpha^{\mathrm{maksv}}\,,\enskip \alpha=i,e\,;\\[9pt]
&\varphi(\vec{r},t)\big\vert_{\vec{r}\in\Omega_\infty}&=0\,,
\end{array}\!\!
\right\}\!\!\!\!
\label{e2-k}
\end{equation}
    где 
    
    \noindent
    \begin{multline*}
    f_\alpha^{\mathrm{maksv}}=n_{\alpha\infty}\left(\fr{m_\alpha}{2k\pi T_{\alpha\infty}}\right)^{3/2}\times{}\\
    {}\times
    \exp\left( -
\fr{m_\alpha}{2kT_{\alpha\infty}}\left\vert\vec{v}-\vec{v}_\infty\right\vert^2\right)\,,
\enskip \alpha=i, e\,;
\end{multline*} 
$\Omega_p$ и $\Omega_\infty$~--- множество радиус-векторов час\-тиц, концы которых принадлежат плоскости зонда и 
границе возмущенной зоны соответственно.

Для решения поставленной задачи введем декартову систему координат таким образом, чтобы 
заряженная плоскость совпала с плоскостью~$0xz$. Тогда положение частицы в пространстве будет 
определяться координатами $x,y,z$, а скорость~--- координатами $v_x, v_y, v_z$. В~силу того что 
плоскость является бесконечно большой в сравнении с характерным размером задачи, функции 
распределения частиц будут зависеть только от переменных $y, v_y, t$.

Поставленную задачу предлагается решать независимо двумя методами. Первый метод основывается на 
методе статистических испытаний Мон\-те-Кар\-ло, второй метод является композицией метода 
расщепления и метода крупных частиц.

\section{Применение метода Монте-Карло}

Запишем самосогласованную систему уравнений~(\ref{e1-k}) и~(\ref{e2-k}) в декартовой системе 
координат с учетом сделанных предположений:
\begin{equation}
\left.
\begin{array}{l}
\fr{\partial f_\alpha}{\partial t}+
v_y\fr{\partial f_\alpha}{\partial y}+\fr{F_y^\alpha}{m_\alpha}\,\fr{\partial 
f_\alpha}{\partial v_y}=\fr{1}{2}\,\fr{\partial^2 }{\partial [v_y]^2}\times{}\\
{}\times \left( 
f_\alpha\fr{\partial^2 g_\alpha  }{\partial [v_y]^2}\right) -
\fr{\partial}{\partial v_y}\left( f_\alpha\fr{\partial h_\alpha}{\partial v_y}\right)\,,
\enskip \alpha=i,e\,;\\[6pt]
    \fr{\partial^2\varphi}{\partial y^2} =-\fr{{\sf e}}{\varepsilon_0}\left(n_i-n_e\right)\,;
    \enskip E_y=-
\fr{\partial\varphi}{\partial y}\,;\\[6pt]
\hspace*{3.1mm}    t=0:\  \hspace*{2.6mm}f_\alpha(y,v_y,0)=f_\alpha^{\mathrm{maksv}}\,,\ \alpha=i,e\,;\\[9pt]
\hspace*{2.9mm} y=0:\ \hspace*{2.8mm}f_\alpha(0,v_y,t)=0\,,\ \alpha=i,e\,;\\[9pt]
\hspace*{24.3mm}\varphi(0,t)=\varphi_p\,;\\[9pt]
y=y_\infty:\ f_\alpha(y_\infty, v_y, t)=f_\alpha^{\mathrm{maksv}}\,,\ \alpha=i,e\,;\\[9pt]
\hspace*{21.5mm}\varphi(y_\infty, t)=0\,.
\end{array}
\right \}
\label{e3-k}
\end{equation}

В полученной системе уравнений~(\ref{e3-k}) перейдем к безразмерным величинам, применив 
соотношение $X=M_X \hat{X}$, где $M_X$~--- масштаб размерной величины~$X$, $\hat{X}$~--- 
безразмерная величина~$X$. В~качестве используемых масштабов были взяты следующие: радиус 
Дебая, скорость теплового движения частиц, концентрация частиц в невозмущенной плазме, потенциал, 
возникающий при разделении зарядов в дебаевской сфере, и производные от них величины.

Система безразмерных уравнений имеет следующий вид:
%\noindent
\begin{equation}
\left.
\begin{array}{l}
\fr{\partial 
\hat{f}_\alpha}{\partial\hat{t}}+A_\alpha\fr{\partial\hat{f}_\alpha}{\partial\hat{y}}+
B_\alpha\hat{E}_y\fr{\partial\hat{f}_\alpha}{\partial \hat{v}_y}={}\\
\!{}=
\fr{\partial^2}{\partial[\hat{v}_y]^2}\left(D_\alpha 
\hat{f}_\alpha\right)-\fr{\partial}{\partial\hat{v}_y}\left(K_\alpha \hat{f}_\alpha\right),\enskip 
\alpha=i,e;\\[9pt]
\fr{\partial^2\hat{\varphi}}{\partial\hat{y}^2}=-\left(\hat{n}_i-\hat{n}_e\right)\,;\enskip \hat{e}_y=-
\fr{\partial\hat\varphi}{\partial\hat{y}}\,;\\[9pt]
\hspace*{3.1mm}\hat{t}=0:\ \hspace*{2.6mm}\hat{f}_\alpha(\hat{y},\hat{v}_y,0)=\hat{f}_\alpha^{\mathrm{maksv}}\,,\enskip \alpha-i,e\,;\\[9pt]
\hspace*{2.9mm}\hat{y}=0:\ \hspace*{2.8mm}\hat{f}_\alpha(0,\hat{v}_y,\hat{t})=0\,,\enskip \alpha=i,e\,;\\[9pt]
\hspace*{24.3mm}\hat\varphi(0,\hat{t})=\hat{\varphi}_p\,;\\[9pt]
\hat{y}=\hat{y}_\infty:\ \hat{f}_\alpha(\hat{y}_\infty, \hat{v}_y, \hat{t})=\hat{f}^{\mathrm{maksv}}_\alpha\,,\enskip 
\alpha=i,e\,;\\[9pt]
\hspace*{21.5mm}\hat\varphi(\hat{y}_\infty,\hat{t})=0\,.
\end{array}
\right\}
\label{e4-k}
\end{equation}
Здесь 

\vspace*{-2pt}

\noindent
\begin{gather*}
A_\alpha=\sqrt{\delta_\alpha }\,\hat{v}_y\,;\enskip 
B_\alpha=\sqrt{\delta_\alpha}\,\fr{z_\alpha}{2\varepsilon_\alpha}\,;\\
\delta_\alpha=\fr{\varepsilon_\alpha}{\mu_\alpha}\,;\enskip 
\varepsilon_\alpha=\fr{T_{\alpha\infty}}{T_{i\infty}}\,;\\
\mu_\alpha=\fr{m_\alpha}{m_i}\,;\enskip 
D_\alpha=A_g^\alpha\fr{\partial^2\hat{g}_\alpha}{\partial  [\hat{v}_y]^2}\,;\\
K_\alpha=A_h^\alpha \fr{\partial \hat{h}_\alpha}{\partial \hat{v}_y}\,,\enskip \alpha=i,e\,,
\end{gather*}
где $A_g^\alpha$ и $A_h^\alpha$~--- коэффициенты, определяемые характерными параметрами 
задачи~\cite{15-k}.

Поиск решения самосогласованной системы уравнений~(\ref{e4-k}) осуществляется по следующей 
схе-\linebreak ме. Вначале находятся значения напряженности\linebreak
 электрического поля по значениям потенциала, 
полученным из граничной задачи для уравнения Пуассона. Далее, используя найденные значения 
напряженности, решается уравнение Фок\-ке\-ра--План\-ка путем перехода к стохастическому 
дифференциальному уравнению (СДУ) Ито:

\noindent
\begin{multline*}
d\Theta_\alpha(\hat{t}) = a_\alpha \left(\hat{t},\Theta_\alpha(\hat{t})\right)+{}\\
{}+\sigma\left(
\hat{t},\Theta_\alpha(\hat{t})\right)\,dW(\hat{t})\,,\quad \alpha=i,e\,,
%\label{e5-k}
\end{multline*}
где 

\noindent
\begin{align*}
\Theta_\alpha(\hat{t})&=\begin{bmatrix}
\hat{y}(\hat{t})\\ \hat{v}_y(\hat{t})
\end{bmatrix}\,;\\
a_\alpha\left(\hat{t},\Theta_\alpha(\hat{t})\right)&=\begin{bmatrix}
-A_\alpha\\ -K_\alpha -B_\alpha \hat{E}_y
\end{bmatrix}\,;\\
\sigma_\alpha\left(\hat{t},\Theta_\alpha(\hat{t})\right)\sigma_\alpha^{\mathrm{T}}\left( 
\hat{t},\Theta_\alpha(\hat{t})\right)&=D_\alpha\,,\enskip \alpha=i,e\,;
\end{align*} 
$W(\hat{t})$~--- стандартный винеровский случайный процесс.
\pagebreak

Для нахождения значений вектора состояния~$\Theta_\alpha(\hat{t})$ применим явную разностную 
схему стохастического метода Эйлера~\cite{16-k}:
\begin{multline*}
\Theta_\alpha^{n+1}=\Theta_\alpha^n +h_\tau a_\alpha \left( \hat{t}_n, \Theta_\alpha^n\right)+\sigma_\alpha 
\left( \hat{t}_n, \Theta_\alpha^n\right)\Delta W_n\,,\\ 
n=0,\ldots , N\,,\ \alpha=i,e\,,
%\label{e6-k}
\end{multline*}
где $\Theta_\alpha^n$, $n=0,\ldots , N$,~--- приближенное значение вектора 
состояния~$\Theta_\alpha(\hat{t})$, $\alpha=i,e$, в момент времени $\hat{t}\hm=\hat{t}_n$, 
$\hat{t}_n\hm=n h_\tau$, $n=0,\ldots , N$; $h_\tau$~--- достаточно малый шаг интегрирования; $\Delta 
W_n$, $n=0,\ldots ,N$,~--- величина приращения винеровского процесса~$W(\hat{t})$ на отрезке $\left[ 
\hat{t}_n,\,\hat{t}_{n+1}\right]$, по определению независимая от~$\Theta_\alpha^0$, 
$\Delta W_0,\ldots , 
\Delta W_{n-1}$: $\Delta W_n\hm=W(\hat{t}_{n-1})\hm-W(\hat{t}_n)$; $\Delta W_n\hm\sim N(0,\,h_\tau)$, 
т.\,е.\ $\Delta W_n$ представляют собой гауссовские случайные величины с нулевыми математическими 
ожиданиями и дисперсиями, равными шагу интегрирования; $\Theta_\alpha^0$~--- значение вектора 
состояния $\Theta_\alpha(\hat{t})$, $\alpha\hm=i,e$, в момент времени $\hat{t}=0$, 
$\Theta_\alpha^0\hm\sim \hat{f}_\alpha^{\mathrm{maksv}}$. 

Частные производные $\partial^2\hat{g}_\alpha/\partial[\hat{v}_y]^2$ и $\partial \hat{h}_\alpha/\partial 
\hat{v}_y$, являющиеся составляющими матрицы $\sigma_\alpha (\hat{t}_n, 
\Theta_\alpha^n)\sigma_\alpha^{\mathrm{T}}(\hat{t}_n,\Theta_\alpha^n)$ и вектора $a_\alpha(\hat{t}_n, 
\Theta_\alpha^n)$ соответственно, аппроксимируются со вторым порядком точности на трехточечном 
шаблоне на основе значений~$\hat{g}_\alpha$ и~$\hat{h}_\alpha$~\cite{17-k}.
      
      В выражения для функций~$\hat{g}_\alpha$ и~$\hat{h}_\alpha$ входят интегралы, которые 
вычисляются методом Мон\-те-Кар\-ло с использованием набора значений скоростной компоненты 
вектора состояния~$\hat{v}_y$, полученных из решения СДУ Ито:
      \begin{equation*}
      \int \hat{f}_\alpha \left\vert \hat{v}_y-
\hat{v}_y^\prime\right\vert\,dv_y^\prime=M\left(\zeta\left(\hat{V}_y\right)\right)\,,
\end{equation*}
где
$$
      \zeta\left(\hat{V}_y\right)=\left\vert \hat{v}_y-\hat{V}_y\right\vert\,,\enskip \hat{V}_y\sim 
\hat{f}_\alpha\,.
  $$
      
      Для вычисления напряженности самосогласованного электрического поля $\hat{E}_y=-
\partial\hat{\varphi}/\partial\hat{y}$, входящей в вектор $a_\alpha(\hat{t}_n, \Theta_\alpha^n)$, необходимо 
аналогично аппроксимировать со вторым порядком точности производную 
$\partial\hat{\varphi}/\partial\hat{y}$ на трехточечном шаблоне с использованием значений 
потенциала~$\hat{\varphi}$~\cite{17-k}. Значения потенциала~$\hat\varphi$ находятся из решения 
уравнения Пуассона. 
      
      Граничную задачу для уравнения Пуассона 
      \begin{align*}
      \fr{\partial^2 \hat\varphi}{\partial \hat{y}^2} & = -\left(\hat{n}_i-\hat{n}_e\right)\,;\\
      \hat{\varphi}\big|_{\hat{y}=0} &=\hat{\varphi}_p\,;\\
      \hat{\varphi}\big|_{\hat{y}_\infty=0} &=0
      \end{align*}
    предлагается решать путем перехода к конечно-разностной системе с последующим ее решением 
методом прогонки~\cite{17-k}:

\noindent
\begin{gather*}
\hat{\varphi}^n_{l-1}+2\hat{\varphi}_l^n+\hat{\varphi}^n_{l+1}=
h_y\hat{\delta}_l^n\,,\enskip l=1,\ldots , 
N_y\,;\\
\hat{\delta}_l^n=-\left( \hat{n}^n_{i,l}-\hat{n}^n_{e,l}\right)\,;\enskip 
\hat{\varphi}_0=\hat{\varphi}_p\,;\enskip \hat{\varphi}_{N_y}=0\,,
\end{gather*}
где $N_y$~--- число шагов по переменной~$\hat{y}$, $h_y$~--- величина шагов разбиения по~$\hat{y}$. 
      
      Концентрации $\hat{n}_\alpha$, $\alpha=i,e$, и плотности токов частиц на зонд~$\hat{f}_\alpha$, 
$\alpha=i,e$, вычисляются согласно описанному выше методу Мон\-те-Карло.

\section{Применение метода расщепления и~метода крупных~частиц}

Решение задачи в данном случае предлагается начать с записи правой части уравнения 
Фок\-ке\-ра--План\-ка в декартовой системе координат в виде:
$$
\mathbf{Q} f_\alpha = \fr{1}{2}\,\fr{\partial^2 f_\alpha}{\partial [v_y]^2}\,\fr{\partial^2 g_\alpha}{\partial 
[v_y]^2}+\fr{\partial f_\alpha}{\partial v_y}\,\fr{\partial C_\alpha}{\partial v_y}+H_\alpha\,,\enskip 
\alpha=i,e\,,
$$  
где 
\begin{align*}
C_\alpha(\vec{r},\vec{v},t)&=
\begin{cases}
\fr{1-\gamma}{Z_i^2}\int\fr{f_e(\vec{r},{\vec{v}}^{\,\prime},t)}{|\vec{v}-{\vec{v}}^{\,\prime} |}\,d{\vec{v}}^{\,\prime}\,, 
&\alpha=i\,;\\[9pt]
\fr{Z_i^2(\gamma-1)}{\gamma}\int \fr{f_i(\vec{r},{\vec{v}}^{\,\prime}, t)}
{|\vec{v}-{\vec{v}}^{\,\prime} 
|}\,d{\vec{v}}^{\,\prime}\,, &\alpha=e\,;
\end{cases} 
\\
H_\alpha&=
\begin{cases}
4\pi \left( \fr{\gamma f_e}{Z_i^2}+f_i\right)f_i\,, & \alpha=i\,;\\[9pt]
4\pi\left(\fr{Z_i^2 f_i}{\gamma}+f_e\right)f_e\,, &\alpha=e\,.
\end{cases}
\end{align*}
Тогда при переходе к безразмерным величинам (см.\ разд.~3) система~(\ref{e1-k}) запишется 
следующим образом:
      \begin{equation}
      \left.
\!\!\begin{array}{l}
      \fr{\partial 
\hat{f}_\alpha}{\partial\hat{t}}+A_\alpha\fr{\partial\hat{f}_\alpha}{\partial\hat{y}}+
B_\alpha  \hat{E}_y
\fr{\partial\hat{f}_\alpha}{\partial\hat{v}_\alpha}=\tilde{\mathbf{Q}}\hat{f}_\alpha\,,\enskip 
\alpha=i,e;\\[9pt]
      \fr{\partial^2\hat{\varphi}}{\partial\hat{y}^2}=-\left( \hat{n}_i-\hat{n}_e\right)\,,\enskip \hat{E}_y=-
\fr{\partial\hat\varphi}{\partial\hat{y}}\,,\\[9pt]
\hspace*{3.1mm}\hat{t}=0:\ \hspace*{2.6mm}\hat{f}_\alpha(\hat{y},\hat{v}_y, 0)=\hat{f}_\alpha^{\mathrm{maksv}}\,,\enskip \alpha=i,e\,,\\[9pt]
\hspace*{2.9mm} \hat{y}=0:\ \hspace*{2.8mm}\hat{f}_\alpha(0,\hat{v}_y,\hat{t})=0\,,\enskip \alpha=i,e\,;\\[9pt]
\hspace*{24.3mm}\hat\varphi(0,\hat{t})=\hat{\varphi}_p\,;\\[9pt]
      \hat{y}=\hat{y}_\infty:\ \hat{f}_\alpha(\hat{y}_\infty, 
\hat{v}_y,\hat{t})=\hat{f}_\alpha^{\mathrm{maksv}}\,,\enskip \alpha=i,e\,;\\[9pt]
\hspace*{21.5mm}\hat{\varphi}(\hat{y}_\infty,\hat{t})=0\,,\\[9pt]
    \end{array}
\right\}\!\!
\label{e7-k}
\end{equation}
где 
\begin{gather*}
\tilde{\mathbf{Q}} \hat{f}_\alpha=D_\alpha\fr{\partial^2\hat{f}_\alpha}{\partial 
[\hat{v}_y]^2}+K_\alpha\fr{\partial\hat{f}_\alpha}{\partial\hat{v}_y}+H_\alpha\,;\\
D_\alpha=A_g^\alpha\fr{\partial^2\hat{g}_\alpha}{\partial [\hat{v}_y]^2}\,;\enskip 
K_\alpha=A_h^\alpha \fr{\partial \hat{h}_\alpha}{\partial\hat{v}_y}\,,\ \alpha=i,e\,.
\end{gather*}

Для решения системы уравнений~(\ref{e7-k}) применяется модификация метода 
расщепления~\cite{17-k}, согласно которой исходная задача разбивается на две вспомогательные. Такое 
разбиение можно осуществить, переписав уравнение Фок\-ке\-ра--План\-ка в следующем виде:
$$
\fr{\partial\hat{f}_\alpha}{\partial\hat{t}} =
\tilde{\mathbf{Q}}_1\hat{f}_\alpha+\tilde{\mathbf{Q}}_2\hat{f}_\alpha\,,
$$
где 
\begin{align*}
\tilde{\mathbf{Q}}_1\hat{f}_\alpha &=-
\left(A_\alpha\fr{\partial\hat{f}_\alpha}{\partial\hat{y}}+
B_\alpha\fr{\partial\hat{f}_\alpha}{\partial\hat{y}}
\right)\,;\\
\tilde{\mathbf{Q}}_2\hat{f}_\alpha 
&=\left(D_\alpha\fr{\partial^2\hat{f}_\alpha}{\partial[\hat{v}_y]^2}+K_\alpha\fr{\partial 
\hat{f}_\alpha}{\partial\hat{v}_y}+H_\alpha\right)\,.
\end{align*}

      Правая часть уравнения Фок\-ке\-ра--План\-ка представляет собой сумму двух операторов, 
первый из которых отвечает за перенос частиц, второй~--- за столкновения заряженных частиц. 
В~результате образуются следующие задачи, которые решаются последовательно:
      \begin{itemize}
\item первая задача:
\begin{align*}
&\fr{\partial w_\alpha(\hat{y},\hat{v}_y,\hat{t})}{\partial\hat{t}} =\mathbf{Q}_1 
w_\alpha(\hat{y},\hat{v}_y,\hat{t})\,,\enskip \alpha=i,e\,;\\[9pt]
&\fr{\partial^2\hat\varphi}{\partial\hat{y}^2}=-\left(\hat{n}_i-\hat{n}_e\right)\,;\enskip
\hat{E}_y=-
\fr{\partial\hat\varphi}{\partial\hat{y}}\,;\\[9pt]
&w_\alpha(\hat{y},\hat{v}_y,\hat{t}^n)=\hat{f}_\alpha(\hat{y},\hat{v}_y,\hat{t}^n)\,,\enskip n=0,\ldots ,N-
1\,;\\[9pt]
&\hspace{2.9mm}\hat{y}=0:\ \hspace*{2.9mm}w_\alpha(0,\hat{v}_y,\hat{t})=0\,,\enskip \alpha=i,e\,;\\[9pt]
&\hspace*{25.1mm}\hat\varphi(0,\hat{t})=\hat{\varphi}_p\,;\\[9pt]
&\hat{y}=\hat{y}_\infty:\ w_\alpha(\hat{y}_\infty, \hat{v}_y, \hat{t})=
\hat{f}_\alpha^{\mathrm{maksv}}\,,\enskip 
\alpha=i,e\,;\\[9pt]
&\hspace*{22.5mm}\hat\varphi(\hat{y}_\infty,\hat{t})=0\,;
\end{align*}
\item вторая задача:
\begin{align*}
\!\!\!\!\!\!\!\fr{\partial s_\alpha(\hat{y},\hat{v}_y,\hat{t})}{\partial \hat{t}} &=\mathbf{Q}_2 
s_\alpha(\hat{y},\hat{v}_y,\hat{t})\,, & \alpha&=i,e\,;\\
\!\!\!\!\!\!\!s_\alpha (\hat{y},\hat{v}_y,\hat{t}^n) &=w_\alpha (\hat{y},\hat{v}_y, \hat{t}^{n+1}),& n&=0,\ldots ,N-
1.
\end{align*}
\end{itemize}

Первая задача представляет собой систему безразмерных уравнений Вла\-со\-ва--Пуас\-со\-на. Для ее 
решения применяется метод крупных частиц~\cite{18-k}. Согласно этому методу решение задачи 
осуществляется путем расщепления на два этапа: на первом этапе не учитываются конвективные члены 
и решение получается обычным интегрированием на неподвижной эйлеровой сетке, а на втором этапе 
рассматривается система, которая описывает перенос частиц в лагранжевой системе координат. Кроме 
того, на первом этапе необходимо решить уравнение Пуассона для получения значений потенциала 
самосогласованного электрического поля. Для этого применяется метод, описанный в разд.~3. 

Вторая задача решается путем перехода к ко\-неч\-но-раз\-ност\-ной сис\-те\-ме. При этом частные 
производные $\partial^2\hat{g}_\alpha/\partial[\hat{v}_y]^2$ и $\partial\hat{h}_\alpha/\partial\hat{v}_y$ 
аппроксимируются со вторым порядком точности с использованием трехточечного шаблона, а 
производная $\partial s_\alpha/\partial\hat{t}$ аппроксимируется на двухточечном шаблоне с первым 
порядком точности~\cite{16-k}. К~полученной системе разностных уравнений предлагается применить 
один из классических методов решения систем линейных уравнений, например метод 
Гаусса~\cite{19-k}.
      
      Решением первой задачи является функция $w_\alpha(\hat{y}, \hat{v}_y, \hat{t}^n)$, 
$n\hm=0,\ldots ,N$, , которая дает начальное условие для второй задачи. Решая вторую задачу, находим 
функцию $s_\alpha(\hat{y},\hat{v}_y,\hat{t}^n)\hm=\hat{f}_\alpha(\hat{y},\hat{v}_y,\hat{t}^n)$, 
$n=1,\ldots ,N$, $\alpha=i,e$, которая определяет решение $\hat{f}_\alpha(\hat{y},\hat{v}_y,\hat{t}^n)$, 
$\alpha=i,e$, исходной системы~(\ref{e7-k}) для рассматриваемых моментов времени $n=1,\ldots ,N$.

Моменты функций распределения $\hat{f}_\alpha$, $\alpha=i,e$, находятся с помощью методов 
численного интегрирования, например метода трапеций~\cite{19-k}.

\section{Результаты численного моделирования}

Для двух описанных выше методов реализованы две отдельные программы в среде {Matlab~7.0}. 
Эти программы позволяют по заданным значениям концентраций и температур частиц $n_{i\infty}$, 
$n_{e\infty}$, $T_{i\infty}$ и~$T_{e\infty}$ в невозмущенной плазме, а также потенциала~$\varphi_p$, 
подаваемого на зонд, изучить эволюцию во времени плотностей тока частиц~$j_i$ и~$j_e$, концентраций 
частиц~$n_i$  и~$n_e$ в произвольной точке пространства в возмущенной зоне, а также динамику 
изменения напряженности~$E_y$ самосогласованного электрического поля во времени и пространстве.

С использованием разработанных программ проведены серии расчетных экспериментов, в которых 
значение концентраций варьировалось в пределах $n_{i\infty} \hm = n_{e\infty}\hm =10^{18}\div 
10^{22}$~м$^{-3}$. Значение температур было выбрано неизменным и равным $T_{i\infty}\hm = 
T_{e\infty}\hm=3000$~K, а значения потенциала, подаваемого на зонд, изменялись в пределах 
$\varphi_p\hm=0\div 2{,}6$~В.

На рис.~1  и~2 приведены графики изменения напряженности самосогласованного электрического
 поля (см.\ рис.~1) и плотности токов ионов (см.\linebreak\vspace*{-12pt}

\pagebreak

\end{multicols}

\begin{figure} %fig1
\vspace*{1pt}
\begin{center}
\mbox{%
\epsfxsize=162.594mm
\epsfbox{kud-1.eps}
}
\end{center}
\vspace*{-9pt}
\Caption{Динамика изменения плотности тока ионов во времени в фиксированной точке возмущенной 
зоны для значений потенциала: \textit{1}~--- $\varphi_p=-6$; 
\textit{2}~--- $\varphi_p=-16$; \textit{3}~--- $\varphi_p=- 30$ 
в случае применения методов Монте-Карло~(\textit{а}) 
и крупных частиц~(\textit{б})}
\end{figure}

\begin{figure} %fig2
\vspace*{1pt}
\begin{center}
\mbox{%
\epsfxsize=162.713mm
\epsfbox{kud-2.eps}
}
\end{center}
\vspace*{-9pt}
\Caption{Динамика изменения напряженности электрического поля во времени в фиксированной точке 
возмущенной зоны для значений потенциала: 
\textit{1}~--- $\varphi_p=-6$; \textit{2}~--- $\varphi_p=-16$; 
\textit{3}~--- $\varphi_p=-30$ в случае применения методов Монте-Карло~(\textit{а}) и
крупных частиц~(\textit{б})
}
\end{figure}

\begin{multicols}{2}

\noindent
 рис.~2) во времени в фиксированной точке пространства 
возмущенной зоны в случае применения обоих разработанных алгоритмов.


На основании полученных результатов можно отметить похожее поведение зависимостей 
напряженности электрического поля и плотности тока от времени в двух рассматриваемых случаях. 
Графики кривых сначала убывают, затем начинают возрастать, выходя в некоторый момент 
времени~$t^\prime$ (момент установления) на стационарные значения. 

Одинаковое поведение 
напряженности и плот\-ности тока можно объяснить из следующих соображений: плотность тока ионов в 
данной области пространства равна произведению концентрации ионов на их направленную скорость и 
на заряд иона. Скорость ионов, в свою очередь, зависит от заряда, массы и напряженности 
электрического поля. 
%\columnbreak

При внесении в плазму отрицательно заряженного зонда возникает электрическое поле, которое 
нарушает квазинейтральность плазмы. Для того чтобы компенсировать действие внешнего 
электрического поля, ионы устремляются к зонду, а электроны~--- от зонда. Это приводит к дисбалансу 
концентраций вблизи зонда и, как следствие, к увеличению разности потенциалов; график 
напряженности электрического поля убывает. Вскоре разделение зарядов компенсирует внешнее 
электрическое поле; график выходит на стационарное значение. 

Также можно отметить, что значения 
напряженности электрического поля и плотности тока частиц на зонд в момент установления для двух 
методов совпадают. 

Момент установления~$t^\prime$ зависит от при\-ме\-ня\-емо\-го метода решения. В~случае метода 
Мон\-те-Кар\-ло $t^\prime=3{,}5\div 4$~ед., а для метода крупных частиц совместно с методом 
расщепления $t^\prime\hm=5\div 5{,}5$~ед. Используя ко\-неч\-но-раз\-ност\-ный метод, можно 
получить динамику изменения функций распределения частиц~$f_\alpha$, $\alpha=i,e$, во времени и 
пространстве. Функции распределения позволяют наглядно представить влияние на картину 
распределения частиц вблизи зонда самой поверхности зонда и электрического поля.

\section{Заключение}
      
      В работе найдено решение задачи диагностики плоским зондом сильноионизованной плазмы с 
учетом столкновений заряженных частиц. Разработана математическая модель исследуемого явления, 
описываемая уравнениями Фок\-ке\-ра--План\-ка и Пуассона. Решение получено двумя методами:\linebreak 
статистическим и ко\-неч\-но-раз\-ност\-ным на основе\linebreak сформированных алгоритмов. Приведены 
резуль-\linebreak таты численного моделирования при различных\linebreak характерных параметрах задачи.
 Из  проведенных 
вычислительных экспериментов вытекает, что искомые величины: напряженность 
электрического поля, плотности токов частиц на зонд, концентрации частиц вблизи зонда~--- как по 
характеру зависимости, так и по числовым значениям совпадают. При применении метода 
      Мон\-те-Кар\-ло момент установления наступает быстрее по сравнению с конечно-разностным 
методом, однако конечно-разностный метод позволяет получить более наглядные результаты.

{\small\frenchspacing
{%\baselineskip=10.8pt
\addcontentsline{toc}{section}{Литература}
\begin{thebibliography}{99}

\bibitem{1-k}
\Au{Alexeff I., Anderson T.}
Experimental and theoretical results with plasma antenna~// IEEE Trans. Plasma Sci., 2006. Vol.~34. 
No.\,2. P.~166--172.

\bibitem{2-k}
\Au{Сысун В.\,И.}
Сильноионизованная низкотемпературная плазма в приборах электронной техники: Методы 
исследования, свойства, применение. Дисс. \ldots д-ра физ.-мат. наук в форме науч. докл.: 
01.04.08.~--- Пет\-ро\-за\-водск, 1996.

\bibitem{3-k}
\Au{Тухас В.\,А.}
Методология создания средств измерений и испытаний на устойчивость к кондуктивным помехам~// 
Мат-лы VI Междунар. симп. по электромагнитной совместимости и 
электромагнитной экологии.~--- СПб., 2005. С.~231--234.

\bibitem{4-k}
\Au{Гудзенко Л.\,И., Яковленко С.\,И.}
Плазменные лазеры.~--- М.: Атомиздат, 1978.  256~с.

\bibitem{5-k}
\Au{Звелто О.}
Принципы лазеров.~--- М.: Мир, 1990.  560~с.

\bibitem{6-k}
\Au{Сысун В.\,И., Хромой Ю.\,Д.}
Расширение канала мощного импульсного разряда в парах ртути~// Электронная техника, 1974. 
Сер.~4. Вып.~10. С.~80--85. 

\bibitem{7-k}
\Au{Винклер Дж.\,Р.}
Искусственные пучки частиц в космической плазме.~--- М.: Мир, 1985.  451~с.

\bibitem{8-k}
\Au{Bernstein I.\,B., Rabinowitz I.\,N.}
Theory of electrostatic probes in low-density plasma~// Phys. Fluids, 1959. Vol.~2. No.\,2. P.~112--121. 

\bibitem{9-k}
\Au{Альперт Я.\,Л., Гуревич А.\,В., Питаевский~Л.\,П.}
Искусственные спутники в разреженной плазме.~--- М.: Наука, 1964.  282~с.

\bibitem{10-k}
\Au{Чан П., Тэлбот Л., Турян~К.}
Электрические зонды в неподвижной и движущейся плазме.~--- М.: Мир, 1978.  202~с.

\bibitem{11-k}
\Au{Алексеев Б.\,В., Котельников В.\,А.}
Зондовый метод диагностики плазмы.~--- М.: Энергоатомиздат, 1989.  240~с.

\bibitem{12-k}
\Au{Пантелеев А.\,В., Кудрявцева И.\,А.}
Формирование математической модели двухкомпонентной плазмы с учетом столкновений 
заряженных частиц в случае плоского зонда~// Теоретические вопросы вычислительной техники и 
программного обеспечения: Межвузовский сб. научн. тр.~--- М.: МИРЭА, 2006. С.~11--21.

\bibitem{13-k}
\Au{Олдер Б.}
Вычислительные методы в физике плазмы.~--- М.: Мир, 1974.  111~с.

\bibitem{14-k}
\Au{Montgomery D.\,C., Tidman D.\,A.}
Plasma kinetic theory.~--- New York, 1964. 

\bibitem{15-k}
\Au{Кудрявцева И.\,А., Пантелеев А.\,В.}
Применение метода Мон\-те-Кар\-ло для анализа поведения двухкомпонентной плазмы с учетом 
столкновений между заряженными частицами~// Теоретические вопросы\linebreak
вычислительной техники и 
программного обеспечения: Межвузовский сб. научн. тр.~--- М.: МИРЭА, 2008. С.~122--128. 

\bibitem{16-k}
\Au{Семенов В.\,В., Пантелеев А.\,В., Руденко~Е.\,А., Бор\-та\-ков\-ский~А.\,С.}
Методы описания, анализа и синтеза нелинейных систем управления.~--- М.: МАИ, 1993.  312~с.

\bibitem{17-k}
\Au{Киреев В.\,И., Пантелеев А.\,В.}
Численные методы в примерах и задачах.~--- М.: Высшая школа, 2006.  480~с.

\bibitem{18-k}
\Au{Белоцерковский О.\,М., Давыдов~Ю.\,М.}
Метод крупных частиц в газовой динамике. Вычислительный эксперимент.~--- М.: Наука, 
Физматгиз, 1982.

\label{end\stat}

\bibitem{19-k}
\Au{Вержбицкий В.\,М.}
Основы численных методов.~--- М.: Высшая школа, 2002.  840~с.
 \end{thebibliography}
}
}


\end{multicols}                   %7
\def\stat{betelin}

\def\tit{ОСНОВНЫЕ ПОНЯТИЯ ПРОГРАММИРОВАНИЯ 
В~ИЗЛОЖЕНИИ~ДЛЯ~ДОШКОЛЬНИКОВ$^*$}

\def\titkol{Основные понятия программирования в~изложении для 
дошкольников}

\def\aut{В.\,Б.~Бетелин$^1$, А.\,Г.~Кушниренко$^2$, А.\,Г.~Леонов$^3$}

\def\autkol{В.\,Б.~Бетелин, А.\,Г.~Кушниренко, А.\,Г.~Леонов}

\titel{\tit}{\aut}{\autkol}{\titkol}

\index{Бетелин В.\,Б.}
\index{Кушниренко А.\,Г.}
\index{Леонов А.\,Г.}
\index{Betelin V.\,B.}
\index{Kushnirenko A.\,G.}
\index{Leonov A.\,G.}
 

{\renewcommand{\thefootnote}{\fnsymbol{footnote}} \footnotetext[1]
{Работа выполнена по теме 0065-2019-0010 госзадания 2020~года 
в~отделе учебной информатики Федерального 
научного центра <<На\-уч\-но-ис\-сле\-до\-ва\-тель\-ский институт системных исследований>> Российской 
академии наук.}}


\renewcommand{\thefootnote}{\arabic{footnote}}
\footnotetext[1]{Федеральный научный центр <<На\-уч\-но-ис\-сле\-до\-ва\-тель\-ский институт системных исследований>> 
Российской академии наук, \mbox{betelin@niisi.msk.ru}}
\footnotetext[2]{Федеральный научный центр <<На\-уч\-но-ис\-сле\-до\-ва\-тель\-ский  институт системных исследований>> 
Российской академии наук, \mbox{agk\_@mail.ru}}
\footnotetext[3]{Московский государственный университет им.\ М.\,В.~Ломоносова; Федеральный научный центр  
<<На\-уч\-но-ис\-сле\-до\-ва\-тель\-ский институт системных исследований>> Российской академии наук; 
Московский педагогический государственный университет, \mbox{dr.l@vip.niisi.ru}}

%\vspace*{-6pt}



\Abst{Развитие информационных технологий сформировало социально-экономический 
запрос на снижение возраста знакомства детей с~программированием. В~результате 
шестилетних усилий авторам удалось разработать и~массово внедрить годовой курс 
программирования для дошкольников, построенный на метафоре программного управления. 
В~процессе развития курса удалось отобрать и~сформулировать набор основных понятий 
программирования, который может быть освоен дошкольниками возраста~6+  
в~дея\-тель\-ност\-но-иг\-ро\-вой форме. Этот набор понятий вводится на примерах программ 
управления движущимися и~неподвижными объектами с~интуитивно понятными, 
обозримыми системами команд. Курс строится на базе беcтекстовой пиктографической 
системы программирования <<ПиктоМир>> разработки ФНЦ НИИСИ РАН. Разработанное 
для курса про\-грам\-мно-ме\-то\-ди\-че\-ское наполнение позволяет каждому дошкольнику 
к~концу курса получить опыт составления и~отладки 120--150~простейших программ.}
  
  \KW{информатика; робот; программа; компьютер; язык программирования; дошкольник; 
<<ПиктоМир>>; пиктограмма}

\DOI{10.14357/19922264200308} 
 
%\vspace*{-6pt}


\vskip 10pt plus 9pt minus 6pt

\thispagestyle{headings}

\begin{multicols}{2}

\label{st\stat}
  
\section{Введение}

  Законодательными структурами власти России федерального уровня 
выдвигаются предложения по понижению возраста знакомства детей 
с~информатикой и~программированием на уровень системы дошкольного 
образования~[1]. Настоящая статья посвящена описанию конкретной методики 
такого\linebreak понижения и~суммирует шестилетний опыт разработки и~массового 
внедрения годового курса <<\mbox{Алгоритмика} для дошколят>>, проводимого 
совместными усилиями ФНЦ НИИСИ РАН и~Департамента образования 
администрации г.~Сур\-гу\-та. Начиная с~осени 2018~г.\ годовой курс проходят 
все выпускники всех подготовительных групп всех до\-школь\-но-об\-ра\-зо\-ва\-тель\-ных 
организаций г.~Сургу\-та~--- 
около 6~тыс.\ детей. В~курсе используется\linebreak
 бестекстовая учебная система 
программирования <<ПиктоМир>>, разработка которой была начата в~ФНЦ 
НИИСИ РАН около 10~лет назад~[2] и~продолжается сегодня в~работах по теме 
госзадания Минобрнауки РФ <<Разработка, реализация и~внедрение семейства 
интегрированных многоязыковых сред программирования$\ldots$>>~[3]. 
Система <<ПиктоМир>> и~методическое обеспечение курса~--- свободно 
распространяемые. Их можно загрузить с~сайта НИИСИ РАН или работать 
с~ними с~по\-мощью браузера через веб-ин\-тер\-фейс~[4].
  
  Информатизация дошкольного образования~--- очень широкая и~важная для 
современной цивилизации тема~[5]. В~данной статье, в~рамках более общей 
работы над 4-лет\-ним курсом программирования для дошкольников и~младших 
школьников, упор делается на конкретной задаче~--- организации первых 
занятий курса программирования для дошкольников. Этот вопрос особенно 
важен, поскольку именно на первых занятиях закладывается фундамент курса, 
осваивается набор основных понятий. 
  
  Накопленный опыт позволил авторам предложить набор основных понятий 
программирования, который может быть органично освоен дошкольниками 
в~дея\-тель\-ност\-но-иг\-ро\-вой форме. Этот набор понятий призван раскрыть 
одну из <<больших идей>> нашей цивилизации~--- {\bfseries\textit{принцип 
программного управ\-ления}}. 
  
  Согласно рубрикации А.\,Л.~Семенова~\cite[п.~12]{6-bet}, одной из целей 
курса информатики должно быть <<освоение информационной картины 
мира>>. Принцип программного управления и~строит такую картину 
в~терминах и~образах, понятных шестилетнему ребенку. Схематично принцип 
может быть объяснен так: \textbf{любую работу, которую человек может 
выполнить, командуя механическим по\-мощ\-ни\-ком-ро\-бо\-том, можно 
перепоручить компьютеру, если человеку удастся составить программу 
выполнения той деятельности, которую роботу надлежит выполнить}. 
  
  Методически правильным представляется двухэтапное изложение принципа 
программного управ\-ле\-ния, при котором на первом этапе излагается 
простейший вариант принципа~--- без обратной связи. Практика работы 
с~дошкольниками показала, что именно этот этап оказывается 
ос\-но\-во\-по\-ла\-гаю\-щим и~более трудным, чем второй этап, на котором вводится 
обратная связь.
  
\section{Принцип программного управления без~обратной связи}

  \textbf{Программа}~--- это план будущей деятельности,\linebreak
   в~процессе которой 
один объект~--- \textbf{компьютер}~-- управляет другим объектом~--- 
\textbf{роботом}~--- по программе, заранее составленной человеком~--- 
\textbf{программистом}~--- по заранее известным \textbf{правилам\linebreak 
составления программ} (\textbf{язык программирования}). Процесс 
\textbf{выполнения} программы компьютером состоит в~том, что компьютер, 
следуя программе, некоторым заранее установленным способом дает роботу 
\textbf{команды}, которые тот \textbf{исполняет}, докладывая компьютеру об 
окончании исполнения каждой команды и~готовности к~приему следующей 
команды. Чтобы компьютер мог выполнить программу, она должна быть ему 
предварительно сообщена (\textbf{загружена в~память} компьютера).
  
  Это описание принципа явно вводит 12~понятий: 
6~{\bfseries\textit{объектов}}, 1~{\bfseries\textit{субъект}} 
и~5~{\bfseries\textit{взаимодействий}} между объектами и~субъектом.
   
\textit{Объекты}: 

  \textbf{программа}; \textbf{компьютер}; \textbf{память компьютера}; 
\textbf{робот};  \textbf{правила составления программ} (\textbf{язык 
программирования}); \textbf{команда}.

\textit{Субъект}: 

  \textbf{программист}.

\textit{Взаимодействия}: 
\begin{itemize}
  \item программист \textbf{составляет} программу;
  \item компьютер \textbf{выполняет} программу, \textbf{давая} роботу 
команды;
\item получив команду, робот ее \textbf{исполняет} и~ждет поступления 
следующей   команды;
  \item компьютер \textbf{загружает в~свою память} сообщенную ему 
программу.
  \end{itemize}
  
  Разумеется, выбранный выше набор понятий и~акценты, сделанные 
в~объяснении принципа, могли бы быть другими. Авторы выбрали указанный 
набор из~12~понятий, исходя из сугубо практических соображений. Дело 
в~том, что на первых занятиях с~дошкольниками педагог должен одновременно 
решить две задачи:
  \begin{enumerate}[(1)]
  \item добиться интуитивного понимания детьми <<правил игры>>, 
интуитивного осознания детьми предложенной системы понятий;\\[-15pt]
  \item пополнить словарный запас детей, научить их использовать в~речевой 
практике термины, выражающие освоенные понятия.
  \end{enumerate}
  
    Предлагаемый выбор понятий позволяет педагогу вчерне решить обе 
задачи примерно за 10--15~получасовых групповых занятий и~добиться 
твердого усвоения системы понятий к~концу годового курса. Важно, что 
данные понятия образуют некоторую замкнутую систему. Скорее всего, для 
ребенка 6~лет, пришедшего на занятия алгоритмикой, данная система понятий 
окажется первой в~его жизни изученной взаимосвязанной \textbf{системой 
научных понятий}. И~это изучение должно быть организовано так, чтобы 
система понятий была понята до конца каждым ребенком.   
  
  Согласно Л.\,С.~Выготскому~\cite{7-bet}, осознание любого общего 
принципа требует комплексного освоения ребенком некоторой 
{\bfseries\textit{системы научных понятий}}: <<Научные понятия являются 
воротами, через которые осознанность входит в~царство детских понятий>>. 
  
  \textbf{Принцип программного управления может быть понят, осознан 
ребенком только после усвоения изложенной выше достаточно сложной 
системы из~12 научных понятий.}
   
  Разумеется, доведение перечисленных понятий до ребенка возраста 6--7~лет 
в~вербальной форме, путем устных объяснений взрослого, невозможно. 
Осознанное усвоение этих понятий станет возможным, только если ребенку 
будут предложены виды деятельности, позволяющие ему в~игровой форме 
<<вжиться>> \textbf{во все} эти 12~понятий, <<пропустить их через себя>>. 


  
  Подчеркнем, что принцип программного управ\-ле\-ния можно вводить 
 по-раз\-но\-му, варьируя набор понятий, которые преподносятся как основные, и~понятий, которые вводятся неявно. Предложенный выше набор 
из~12~основных понятий был подобран так, чтобы максимально облегчить 
ребенку освоение \textbf{каждого} из этих понятий и~\textbf{всей системы} 
понятий в~дея\-тель\-ност\-но-иг\-ро\-вой форме. 

  
  
  Авторам удалось построить первые занятия курса так, 
  чтобы каждый ребенок в~группе смог 
  про-\linebreak\vspace*{-12pt}
  
  \pagebreak
  
  \end{multicols}
  
  \begin{figure*} %fig1
\vspace*{1pt}
 \begin{center}
 \mbox{%
 \epsfxsize=163mm 
 \epsfbox{bet-1.eps}
 }
 \end{center}
   \vspace*{-9pt}
  \Caption{Виртуальные роботы~(\textit{а}) и~реальный робот-игрушка~(\textit{б})}
  %\end{figure*}
  %\begin{figure*} %fig2
\vspace*{18pt}
 \begin{center}
 \mbox{%
 \epsfxsize=149.668mm 
 \epsfbox{bet-3.eps}
 }
 \end{center}
   \vspace*{-9pt}
\Caption{Программа управления роботом, выложенная дошкольником 
из кубиков~(\textit{а}),
и~та же программа, составленная семиклассником~(\textit{б})}
\end{figure*}

\begin{multicols}{2}
  
  \noindent 
  ими\-ти\-ро\-вать выполнение \textbf{всех} пяти перечисленных
выше действий-взаимодействий, выполняя поочередно роль робота, 
компьютера и~программиста, и~смог поработать с~материальным воплощением 
программы, составляя ее и~загружая в~память компьютера и~сталкиваясь 
с~трудностями в~случае нарушения правил составления программы. Это 
удалось сделать за счет использования на первых занятиях следующих 
методических и~технических решений.
  
  Как было предложено Пейпертом еще полвека назад~\cite{8-bet}, дети 
работают не только с~виртуальными (экранными) роботами, но и~с~реальными  
ро\-бо\-та\-ми-иг\-руш\-ка\-ми, которые перемещаются по полу игровой 
комнаты, имитируя перемещения виртуальных роботов на экране планшета 
(рис.~1). 
  
  
   
Реальные роботы управляются звуковыми командами. Эти команды 
<<наблюдаемы>> (слышимы) детьми. 

  Программы составляются из материальных объектов, кубиков, 
с~нанесенными на их грани пикто\-грам\-ма\-ми команд, повторителей и~других 
конст\-рук\-ций языка программирования (рис.~2). \mbox{Функции} компьютера 
выполняет планшет.
  
    




  Загрузка программы в~память компьютера (планшета) состоит в~явно 
проводимом ребенком фотографировании камерой планшета выложенной на 
столе программы, т.\,е.\ некоторой конфигурации кубиков. За этим явным 
действием невидимо для ребенка, скрытно, следует <<понимание>> программы 
компьютером~--- распознавание фотографии с~кубиками процессором 
планшета с~помощью нейронных сетей. Результат этого понимания ребенок 
видит на экране планшета, а как это понимание происходит~--- с~ребенком не 
обсуждается.
  
  
  
  
  Первое синтаксическое правило составления программы из кубиков требует, 
чтобы кубики были выложены в~достаточно ровные ряды, ряды располагались 
друг под другом. Первое семантическое правило выполнения программы 
гласит: при выполнении программы пиктограммы в~рядах читаются слева 
направо, а~ряды читаются сверху вниз. 
  %
  При этом ребенок сталкивается с~тем, что компьютеру не удается <<понять>> 
программы, со\-став\-лен\-ные с~нарушением правил, т.\,е.\ компьютеру 
<<непонятны>> расположения кубиков на столе, в~которых трудно или 
невозможно мысленно разбить выложенные кубики на ряды 
(рис.~3)\footnote{Невыровненность кубиков может рассматриваться как непрерывный аналог 
<<синтаксической неправильности>> программы.}.
  
  Еще две группы правил описывают две конст\-рукции языка 
программирования: числовой повторитель и~подпрограмма с~однобуквенным 
именем.\linebreak Эти правила описывают и~синтаксис, и~семантику и~применяются 
в~ситуациях, когда ребенок, имитируя компьютер, пытается <<понять>> 
выложенную другим ребенком программу и~далее пытается <<понятую>> 
программу выполнить. Эти правила требуют, чтобы пиктограммы 
располагались в~рядах в~определенном порядке. Дети осваивают эти правила 
без затруднений.
  
  Рассматриваемый курс построен так, что на первых занятиях ребенок играет 
со сверстниками и~учебными пособиями (роботом и~набором кубиков 
с~пиктограммами команд) в~сюжетно-ролевые игры. \textbf{Компьютер на 
первых порах используется только по его главному назначению, для 
исполнения загруженных в~его память программ,} и~не используется для 
других целей: ни для составления программ, ни для генерации изображений 
виртуальных роботов и~виртуальных обстановок на экране. На первых занятиях 
курса \textbf{программа, робот и~обстановка, в~которой робот действует, 
являются реальными, а не виртуальными объектами,} и~все 
взаимодействия представляют собой реальные процессы с~участием 
материальных объектов. И~ребенок может осваивать роли объектов в~игре. 
Ребенок может выступать в~роли робота, исполняя звуковые команды, 
поступающие от компьютера или от другого ребенка, выступающего в~роли 
компьютера; ребенок может выступать в~роли компьютера, выполняя 
составленную другим ребенком программу и~командуя при этом третьим 
ребенком, играющим роль робота; ребенок может поработать программистом, 
составляя самостоятельно программу путем перемещения материальных 
кубиков на столе и~переходя позднее\linebreak\vspace*{-12pt}

{ \begin{center}  %fig3
 \vspace*{-1pt}
    \mbox{%
 \epsfxsize=79mm 
 \epsfbox{bet-5.eps}
 }

\end{center}

\noindent
{{\figurename~3}\ \ \small{
Иллюстрация художника Михаила Гладковского к~докладу А.\,П.~Ершова 
<<Программирование~--- вторая грамотность>>
}}}

\vspace*{9pt}




\noindent
 к~составлению программ путем 
псевдоматериального перемещения рукой пиктограмм на сенсорном экране 
планшета.
  
  Важно, чтобы по мере того, как основные по\-нятия программирования 
осваиваются детьми на\linebreak интуитивном уровне при работе с~реальными роботами 
на ковриках, кубиками на столе и~виртуальными роботами, ковриками 
и~программами на экранах планшетов, на бескомпьютерных половинах занятий 
происходил перевод этих интуитивных представлений в~вербальную форму. 
Под руководством воспитателя дети должны обсуждать значения слов 
\textit{программист}, \textit{робот}, \textit{программа}, значения фраз типа 
<<я~выполнил программу, которую составил Коля>>, <<программа составлена 
из 6~пиктограмм>>, <<робот выполнил 10~команд>>. Это пополнение 
словарного запаса детей и~развитие навыков монолога и~диалога 
с~использованием накопленного <<профессионального>> словарного запаса 
является столь же важной целью курса, как и~обретение навыков 
самостоятельного составления простейших программ в~учеб\-но-иг\-ро\-вой 
системе программирования.
  
\section{Принцип программного управления с~обратной связью}

  Б$\acute{\mbox{о}}$льшую часть курса <<Алгоритмика для дошколят>> 
занимает составление программ управления без обратной связи. Каждая такая 
программа решает одну задачу: обеспечивает нужное поведение робота  
в~од\-ной-един\-ст\-вен\-ной внешней обстановке, предъявляемой ребенку на 
полу в~игровой комнате или в~графической форме на экране планшета. 
Программы без обратной связи составляются с~использованием всего 
\textbf{двух явных} управляющих конструкций: числовой повторитель 
и~подпрограмма с~однобуквенным именем и~\textbf{одной неявной} 
конструкцией~--- последовательного выполнения линейного участка 
программы.
  
  В <<поминутной>> методичке годового курса <<Алгоритмика для 
дошколят>> описаны 30~занятий, еще 4~занятия предусмотрены как 
резервные. Последовательное выполнение линейного участка программы 
появляется на первом же занятии. Конструкция \textit{повторитель} впервые 
появляется на занятии №\,10 и~вводится как способ сокращения размера 
программы. Конструкция \textit{подпрограмма} впервые появляется на занятии 
№\,15 и~вводится как способ <<шифровки>> фрагментов программы. Позднее 
эта конструкция рассматривается еще и~как способ сокращения размера 
программы. Разумеется, выразительная сила двух выбранных управляющих 
конструкций невелика. Все программы, которые можно составить 
с~использованием этих конструкций, являются линейными. Однако эти 
линейные программы могут иметь достаточно сложную структуру управления. 
  
  Практика показала, что набора содержательных задач, решаемых 
с~использованием этих двух управ\-ля\-ющих конструкций, достаточно для 
удержания внимания детей в~течение года (первые 25~занятий из~30) при 
условии создания достаточного разнообразия роботов и~их графических 
представлений. В~настоящее время в~курсе <<Алгоритмика для дошколят>> 
используются 5~виртуальных роботов и~1~реальный робот (Ползун), 
имитирующий одного из виртуальных. Несмотря на весьма малую 
продолжительность компьютерной части каждого занятия курса~--- 
от~15~мин в~первом полугодии до~20~мин во втором~--- удается добиться 
того, что каждый ребенок на каждом занятии выполняет 4--5~заданий, т.\,е.\ 
\textbf{в~годовом курсе каждый дошкольник самостоятельно составляет 
120--150~простейших программ}. Самостоятельное выполнение более сотни 
упражнений представляется необходимым условием устойчивого освоения 
теоретического и~практического материала курса. 
  
  В конце первого года обучения, на последних занятиях, начинается (но не 
завершается) переход от управления без обратной связи к~управлению с~ее 
использованием:
  \begin{itemize}
  \item в~системы команд роботов вводятся ко\-ман\-ды-во\-про\-сы, 
и~в~предоставляемые системой <<ПиктоМир>> конструкции языка 
программирования добавляются ветвление и~повторение;
  \item в~систему основных понятий вводятся новый вид команды~--- 
\textbf{ко\-ман\-да-во\-прос} и~новый вид взаимодействия: робот 
\textbf{отвечает} на ко\-ман\-ду-во\-прос компьютера \textbf{да} или 
\textbf{нет}.
  \end{itemize}
  
  Параллельно с~введением понятия <<обратная связь>> в~систему основных 
понятий вводится и~понятие {\bfseries\textit{число}} (неотрицательное целое 
число). Для <<материализации>> понятия <<число>> вводится виртуальный 
исполнитель <<Волшебный кувшин с~камнями>>, играющий роль счетчика. 
Наличие счетчика позволяет с~помощью подсчета числа шагов решать задачи 
управления роботом типа <<дойти до ближайшей стены и~вернуться 
в~исходную точку>>. 
{\looseness=1

}
  
  Разумеется, введение новых понятий со\-про\-вож\-да\-ет\-ся играми. Исполняя роль 
робота, дети отвечают на ко\-ман\-ды-во\-про\-сы <<да>> или <<нет>>; 
исполняя роль кув\-ши\-на-счет\-чи\-ка, ребенок по команде до\-бав\-ля\-ет или 
удаляет камешек из реального кувшина и~отвечает на ко\-ман\-ды-во\-про\-сы 
<<кувшин пуст?>>, <<кувшин не пуст?>> и~<<сколько камней в~кувшине?>>. 
На введение обратной связи в~курсе <<Алгоритмика для дошколят>> требуется 
не менее~5~занятий. На устойчивое освоение этих понятий на следующем году 
обучения необходимо еще~15~занятий. 
{\looseness=1

}
  
   После введения обратной связи становится возможным давать задачи на 
составление <<универсальных>> программ, работающих не в~одной, 
а~в~нескольких разных обстановках. Эти задачи также даются в~графической 
форме, без словесного описания класса обстановок, в~которых должна работать 
программа. Просто на очередном уровне игры требуется составить программу, 
которая работает не в~одной, как раньше, а~в~двух или трех заданных 
обстановках. 

\vspace*{-10pt}
  
\section{Выводы}

\vspace*{-2pt}

  Опыт показал, что дети возраста 6--7~лет без труда и~с~энтузиазмом 
осваивают азы программирования с~использованием описанного выше подхода 
и~готовы продолжать занятия программированием в~школе. Авторы считают  
раннее освоение основ программирования в~описанном выше объеме
необходимым.

%\vspace*{-12pt}

{\small\frenchspacing
 {%\baselineskip=10.8pt
 \addcontentsline{toc}{section}{References}
 \begin{thebibliography}{9}
 
 %\vspace*{-4pt}
 
\bibitem{1-bet}
Глава профильного комитета Думы считает нужным ввести информатику в~дошкольную 
программу:  Мат-лы XVIII съезда <<Единой России>>~// ТАСС, 8~декабря 2018. {\sf 
https://tass.ru/obschestvo/5888487}.
\bibitem{2-bet}
\Au{Rogozhkina I., Kushnirenko~A.} PiktoMir: Teaching programming concepts to preschoolers with 
a~new tutorial environment~// Procd.  Soc. Behv., 2011. Vol.~28.  
P.~601--605. doi: 10.1016/j.sbspro.2011.11.114.
\bibitem{3-bet}
\Au{Бесшапошников Н.\,О., Кушниренко~А.\,Г., Леонов~А.\,Г., Малый~А.\,А.} Проект 
двуязыковой пик\-то\-грам\-мно-текс\-то\-вой учебной среды программирования ПиктоМир-К~// 
Свободное программное обеспечение в~высшей школе: Сб. тезисов XIV конф.~--- 
М.: МАКС Пресс, 2019. С.~64--66.
\bibitem{4-bet}
ПиктоМир: Стартовая страница проекта на сайте Федерального научного центра  
<<На\-уч\-но-ис\-сле\-до\-ва\-тель\-ский институт системных исследований>> Российской 
академии наук. {\sf https://www.niisi.ru/piktomir}.
\bibitem{5-bet}
\Au{Калаш И.} Возможности информационных и~коммуникационных технологий 
в~дошкольном образовании:\linebreak\vspace*{-12pt}

\columnbreak

\noindent
 Аналитический обзор~/ Пер. с~англ. под ред. 
А.\,Л.~Семенова.~--- Институт ЮНЕСКО по информационным технологиям 
в~дошкольном образовании, 2010. 176~с. {\sf 
https://iite.unesco.org/pics/publications/ru/files/\linebreak 3214673.pdf}.
(\Au{\mbox{Kaba{\!\!\ptb{\v{s}}}}, I.} 
Recognizing the potential of ICT in early childhood education: Analytical survey.~---
UNESCO Institute for Information Technologies in Education, 2010. 148~p. 
Available at: {\sf 
https://unesdoc.\linebreak unesco.org/ark:/48223/pf0000190433.pdf} 
(accessed September~2, 2020).)
\bibitem{6-bet}
\Au{Семёнов А.\,Л.} Концептуальные проблемы информатики, алгоритмики и~программирования 
в~школе~// Вестник кибернетики, 2016. №\,2(22). С.~12--16.
\bibitem{7-bet}
\Au{Выготский Л.\,С.} Мышление и~речь.~--- Изд. 5-е, испр.~--- М.: Лабиринт, 1999. Гл.~6. 
\bibitem{8-bet}
\Au{Пейперт С.} Переворот в~сознании: Дети, компьютеры и~плодотворные идеи~/ Пер. 
с~англ.~--- М.: Педагогика, 1989. 224~с.
(\Au{Papert~S.} Mindstorms: Children, computers and powerful ideas.~---  New York, NY, USA: Basic Books, 
1980. 252~p.)
\end{thebibliography}

 }
 }

\end{multicols}

\vspace*{-3pt}

\hfill{\small\textit{Поступила в~редакцию 20.08.19 (последняя правка 21.07.20)}}

\vspace*{10pt}

%\pagebreak

%\newpage

%\vspace*{-28pt}

\hrule

\vspace*{2pt}

\hrule

\vspace*{2pt}

\def\tit{BASIC CONCEPTS OF~PROGRAMMING EXPOUNDED FOR~PRESCHOOLERS}


\def\titkol{Basic concepts of~programming expounded for~preschoolers}

\def\aut{V.\,B.~Betelin$^1$, A.\,G.~Kushnirenko$^1$, and~A.\,G.~Leonov$^{1,2,3}$}

\def\autkol{V.\,B.~Betelin, A.\,G.~Kushnirenko, and~A.\,G.~Leonov}

\titel{\tit}{\aut}{\autkol}{\titkol}

\vspace*{-6pt}


\noindent
$^1$Federal Research Center ``Scientific Research Institute for System Analysis of the Russian Academy of 
Sciences,''\linebreak
$\hphantom{^1}$36-1~Nakhimovsky Prosp., Moscow 117218, Russian Federation

\noindent
$^2$M.\,V.~Lomonosov Moscow State University, 1~Leninskie Gory, GSP-1, Moscow 119991, Russian 
Federation

\noindent
$^3$Moscow Pedagogical State University, 1-1~Malaya Pirogovskaya Str., Moscow 119991, Russian 
Federation

\def\leftfootline{\small{\textbf{\thepage}
\hfill INFORMATIKA I EE PRIMENENIYA~--- INFORMATICS AND
APPLICATIONS\ \ \ 2020\ \ \ volume~14\ \ \ issue\ 3}
}%
 \def\rightfootline{\small{INFORMATIKA I EE PRIMENENIYA~---
INFORMATICS AND APPLICATIONS\ \ \ 2020\ \ \ volume~14\ \ \ issue\ 3
\hfill \textbf{\thepage}}}

\vspace*{6pt} 

\Abste{The development of information technology has formed 
a~socioeconomic demand for reducing the age of acquaintance 
of children with programming. As a~result of 6~years of efforts, 
the authors managed to develop and massively introduce an annual 
programming course for preschoolers built on the metaphor of program 
control. In the process of developing the course, the authors were 
able to select and formulate a~set of basic programming concepts 
which fully reveals this metaphor and, at the same time, can be 
mastered by preschool children age 6+ in an active-play form. 
This set of concepts is introduced using examples of control 
programs for moving and stationary objects with an intuitive, 
visible command system. At the beginning of the course, control 
without feedback is introduced, the concept of feedback is 
introduced and used only at the end of the course. As a~basic 
pedagogical software product, the PictoMir text-free pictographic 
system developed by the Federal Research Center 
``Scientific Research Institute for System Analysis
 of the Russian Academy of Sciences'' 
and its programmatic and methodological content is used, allowing 
each preschooler to gain experience in writing and debugging at 
least 120--150~simplest programs by the end of the course.}

\KWE{informatics; robot; program; computer; programming language; preschooler; 
PiktoMir; pictogram}

\DOI{10.14357/19922264200308} 

%\vspace*{-20pt}

\Ack
\noindent
The work was completed on the subject of the Government order 0065-2019-0010 
in 2020 in the Department 
of Educational Informatics, SRISA RAS.

%\vspace*{6pt}

 \begin{multicols}{2}

\renewcommand{\bibname}{\protect\rmfamily References}
%\renewcommand{\bibname}{\large\protect\rm References}

{\small\frenchspacing
 {%\baselineskip=10.8pt
 \addcontentsline{toc}{section}{References}
 \begin{thebibliography}{9}
\bibitem{1-bet-1}
Materialy TASS XVIII s''ezda ``Edinoy Rossii'' [TASS Materials of 
the 18th All-Russian political party 
``UNITED RUSSIA'' Congress]. Available at: {\sf https://tass.ru/
obschestvo/5888487} (accessed July~24, 
2020).
\bibitem{2-bet-1}
\Aue{Rogozhkina, I., and A.~Kushnirenko.} 2011. PiktoMir: Teaching programming concepts to 
preschoolers with a~new tutorial environment. 
\textit{Procd. Soc. Behv.}  28:601--605.  doi: 10.1016/j.sbspro.2011.11.114.
\bibitem{3-bet-1}
\Aue{Besshaposhnikov, N.\,O., A.\,G.~Kushnirenko,
 A.\,G.~Leonov, and A.\,A.~Malyy.} 2019. Proekt 
dvuyazykovoy piktogrammno-tekstovoy uchebnoy sredy programmirovaniya 
PiktoMir-K [The project of the 
bilingual pictogram-text educational environment for programming PictoMir-K]. 
\textit{Sbornik tezisov 
XIV konf. ``Svobodnoe 
programmnoe obespechenie v~vysshey shkole''} [14th Conference 
``Free Software in Higher Education'' Proceedings]. Moscow: MAKS Press. 64--66.
\bibitem{4-bet-1}
PiktoMir: Startovaya stranitsa proekta na sayte Fe\-de\-ral'\-no\-go 
nauchnogo tsentra  
``Nauchno-issledovatel'skiy\linebreak\vspace*{-12pt}

\columnbreak

\noindent
 institut sistemnykh issledovaniy'' Rossiyskoy akademii nauk [The start page of 
the PictoMir project on the website of the SRISA/NIISI RAS]. Available at: {\sf 
https://www.niisi.ru/piktomir/} (accessed July~24, 2020).
\bibitem{5-bet-1}
\Aue{\mbox{Kaba{\!\ptb{\v{s}}}}, I.} 2010.
Recognizing the potential of ICT in early childhood education: Analytical survey.
\mbox{UNESCO} Institute for Information Technologies in Education. 148~p. 
Available at: {\sf 
https://unesdoc.unesco.org/ark:/ 48223/pf0000190433.pdf} 
(accessed September~2, 2020).
\bibitem{6-bet-1}
\Aue{Semyonov, A.\,L.} 2016. Kontseptual'nye problemy informatiki, algoritmiki i~programmirovaniya 
v~shkole [Conceptual problems of teaching
 computer science, algorithm studies, and programming at school]. 
\textit{Vestnik kibernetiki} [Proceedings in Cybernetics] 2(22):11--15.
\bibitem{7-bet-1}
\Aue{Vygotskiy, L.\,S.} 1999. \textit{Myshlenie i~rech'}  
[Thinking and saying]. Moscow: Labirint. Ch.~6. 
\bibitem{8-bet-1}
\Aue{Papert, S.} 1980. \textit{Mindstorms: Children, computers and powerful 
ideas.} New York, NY: Basic 
Books. 252~p.
\end{thebibliography}

 }
 }

\end{multicols}

\vspace*{-6pt}

\hfill{\small\textit{Received August 20, 2019 (last revision July~21, 2020)}}

%\hfill{\small\textit{(last revision July~21, 2020)}}

%\pagebreak

%\vspace*{-24pt}

\Contr

\noindent
\textbf{Betelin Vladimir B.} (b.\ 1946)~--- 
Doctor of Science in physics and mathematics, professor, Academician of 
RAS, research advisor, Federal Research Center 
``Scientific Research Institute for System Analysis of the 
Russian Academy of Sciences,'' 36-1~Nakhimovsky Prosp., Moscow 117218, Russian Federation; 
\mbox{betelin@niisi.msk.ru}

\vspace*{3pt}

\noindent
\textbf{Kushnirenko Anatoliy G.} (b. 1944)~--- Candidate of Science (PhD) in 
physics and mathematics, Head of 
Department, Federal Research Center 
``Scientific Research Institute for System Analysis of the Russian 
Academy of Sciences,'' 36~b1~Nakhimovsky Prosp., Moscow 117218, Russian Federation; agk\_@mail.ru.

\vspace*{3pt}

\noindent
\textbf{Leonov Aleksandr G.} (b. 1961)~--- Candidate of Science (PhD) in physics and mathematics, associate 
professor, leading scientist, Department of Mechanics and Mathematics, 
M.\,V.~Lomonosov Moscow State 
University, 1~Leninskie Gory, GSP-1, Moscow 119991, Russian Federation; head of laboratory, Federal 
Research Center ``Scientific Research Institute for System Analysis of the Russian Academy of Sciences,''
 36-1~Nakhimovsky Prosp., Moscow 117218, Russian Federation; 
 professor, senior scientist, Moscow 
Pedagogical State University, 1-1~Malaya Pirogovskaya Str., Moscow 119991, Russian Federation; 
\mbox{dr.l@vip.niisi.ru}


\label{end\stat}

\renewcommand{\bibname}{\protect\rm Литература} 
             %8
\def\stat{agasandyan}

\def\tit{ВЫЧИСЛИТЕЛЬНЫЕ АСПЕКТЫ ПРИМЕНЕНИЯ CC-VaR НА~СОВОКУПНОСТИ РЫНКОВ$^*$}

\def\titkol{Вычислительные аспекты применения CC-VaR на совокупности рынков}

\def\aut{Г.\,А.~Агасандян$^1$}

\def\autkol{Г.\,А.~Агасандян}

\titel{\tit}{\aut}{\autkol}{\titkol}

\index{Агасандян Г.\,А.}
\index{Agasandyan G.\,A.}
 

{\renewcommand{\thefootnote}{\fnsymbol{footnote}} \footnotetext[1]
{Исследование выполнено при финансовой поддержке РФФИ в рамках
научного проекта №\,17-01-00816.}}


\renewcommand{\thefootnote}{\arabic{footnote}}
\footnotetext[1]{Вычислительный центр им.\ А.\,А.~Дородницына Федерального исследовательского 
центра <<Информатика и~управление>> Российской академии наук, 
\mbox{agasand17@yandex.ru}}

%\vspace*{-6pt}


  
  \Abst{Работа служит непосредственным продолжением предыдущей работы автора, 
посвященной применению континуального критерия VaR на совокупности нескольких 
рынков разных размерностей, связанных между собой базовыми активами. Исследование 
нацелено на приложение идей и~методов, развитых для теоретической континуальной 
модели, к~дискретным сценарным рынкам. В~модели совокупности одного двумерного 
и~двух одномерных рынков, а также ее усеченных вариантов приводятся конструкции 
оптимальных сценарных портфелей из базисных инструментов всех рынков совокупности 
с~применением рандомизации. Предлагаемые конструкции проверяются на числовых 
примерах с~использованием потенциально типичных двумерных расширений  
бе\-та-рас\-пре\-де\-ле\-ний для описания прогноза будущих цен базовых активов и~картины 
текущих цен базисных инструментов. Изложение сопровождается расчетами весовых 
коэффициентов базисных инструментов оптимальных портфелей и~иллюстрируется 
графиками портфельных доходов.}
  
  \KW{базовые активы; функция рисковых предпочтений; континуальный критерий VaR 
(CC-VaR); стоимостная и~прогнозная плотности; функция относительных доходов; 
процедура Ней\-ма\-на--Пир\-со\-на; комбинированный портфель; суррогатный портфель; 
идеалистичный портфель}

\DOI{10.14357/19922264200309} 
 
%\vspace*{-6pt}


\vskip 10pt plus 9pt minus 6pt

\thispagestyle{headings}

\begin{multicols}{2}

\label{st\stat}
  
   
  \section{Введение}
  
  Работа затрагивает круг проблем, связанных с~применением на финансовых 
рынках континуального критерия VaR (CC-VaR), который был введен 
и~изучался автором ранее~[1--5]. Исследование служит непосредственным 
продолжением работы~[6] и~использует ее идеи и~конструкции. В~работе~[6] 
последовательно излагалась методология применения CC-VaR сразу на 
совокупности трех теоретических континуальных рынков разных размерностей, 
связанных между собой базовыми активами, вводились правила переключения 
между рынками, формировались базисы из инструментов этих рынков 
с~привлечением механизма рандомизации, а также строились в~таких базисах 
оптимальные по CC-VaR комбинированные портфели. 
  
  В данной работе методология, развитая в~[6] для модели теоретических 
континуальных рынков, переносится на дискретные сценарные рынки с~\mbox{целью} 
проверки действенности модели на примерах и~иллюстрации результатов. 
  
  Вновь рассматривается совокупность трех рынков. Один из них является 
двумерным рынком, а два других~--- одномерными. При этом базовые активы 
одномерных рынков образуют пару базовых активов двумерного. Решение 
ищется в~форме совмещения трех сценарных портфелей на своих рынках. Оно 
основывается на анализе расхождений в~относительных доходах по сценариям, 
которые обусловлены расхождениями в~ценах на разных рынках. При этом 
наряду с~совокупностью из трех рынков для полноты исследования 
рассматриваются и~две усеченные, состоящие из пары рынков. Оптимальный 
комбинированный портфель строится с~применением идей и~техники 
рандомизации. 
  
  Предлагаемые конструкции иллюстрируются числовыми примерами 
и~графиками портфельных доходов с~использованием двумерных расширений 
стандартных бе\-та-рас\-пре\-де\-ле\-ний для описания прогноза будущих цен 
базовых активов и~картины текущих цен базисных инструментов.

\vspace*{-9pt} 
  
  \section{Сценарные рынки}
  
  \vspace*{-2pt}
  
  Рассматривается однопериодный тройственный рынок, состоящий из двух 
одномерных рынков \#X и~\#Y с~базовыми активами~$\boldsymbol{X}$ 
и~$\boldsymbol{Y}$ соответственно и~одного двумерного \#0 с~парой активов 
($\boldsymbol{X}, \boldsymbol{Y}$). Возможные цены базовых активов 
образуют\linebreak множества~{\sf X} и~{\sf Y}. Сценарная дискретизация\linebreak
 вводится 
разбиением множества ${\sf X} \hm=  [x_0, x_n)$ на~$n$~сценариев 
$S_i \hm= [x_{i-1}, x_i) \hm\subset {\sf X}$, $x_{i-1}\hm< x_i$, 
$i \hm\in I \hm= \{1, \ldots , n\}$, и~множества ${\sf Y}\hm = [y_0, y_m)$ 
на~$m$~сценариев $T_j\hm = [y_{j-1}, y_j) \hm\subset {\sf  Y}$, $y_{j-1}\hm<  y_j$, 
$j \hm\in J \hm= \{1, \ldots , m\}$. Прямым произведением одномерных 
сценариев разных рынков получаются $n\times m$ двумерных сценариев. 
Равномерное разбиение задается правилом $x_i \hm= x_0 \hm+ ih_1$, 
$h_1 \hm= (x_n \hm- x_0)/n$, $i \hm\in I$; $y_j\hm = y_0 + jh_2$, $h_2 \hm= (y_m \hm-
 y_0)/m$, $j\hm\in J$. Стоимость инструмента~$\boldsymbol{G}$ записывается 
как $\vert \boldsymbol{G}\vert$, а средний доход~--- $\| \boldsymbol{G}\|$.
  
  Связующим звеном теоретических и~сценарных рынков становятся 
инструментальные индикаторы сценариев. Их платежными функциями служат 
характеристические функции сценариев, и~они играют на сценарных рынках 
роль $\delta$-ин\-стру\-мен\-тов теоретического рынка. 
  
  Как и~для теоретического тройственного рынка, задается двумерная 
прогнозная плотность $p(x, y)$, $x\hm\in {\sf X}$, $y\hm\in {\sf Y}$. Ее 
маргинальные плотности~$p_1(x)$ и~$p_2(y)$ служат одновременно 
и~прогнозными плотностями для рынков \#X и~\#Y соответственно. Однако 
подобное свойство в~отношении стоимостных плотностей может и~не 
выполняться, и~для рынка \#0 задается стоимостная плотность $c(x, y)$, для 
рынка \#X~--- $c_{\mathrm{X}}(x)$, для рынка \#Y~--- $c_{\mathrm{Y}}(y)$, 
$x\hm\in {\sf  X}$, $y\hm\in {\sf Y}$. Отметим еще, что эти плотности вводятся 
для удобства формирования картины цен для сценарных рынков, хотя 
достаточно ограничиться дискретным набором стоимостей. 
  
  Индикаторы сценариев 
$\boldsymbol{D}_{ij}\hm = \boldsymbol{H}_{ij} (= \boldsymbol{H}\{S_i\hm\times T_j\})$ 
служат базисными инструментами на рынке \#0, 
$\boldsymbol{D}_{\mathrm{X};i}\hm = \boldsymbol{H}_{\mathrm{X};i}$~--- на рынке 
\#X, $\boldsymbol{D}_{\mathrm{Y};j}\hm = \boldsymbol{H}_{\mathrm{Y};j}$~--- на 
рынке~\#Y. Их цены и~вероятности (средние доходы) образуют 
векторы~$\boldsymbol{c}$ и~$\boldsymbol{p}$, $\boldsymbol{c}_{\mathrm{X}}$ 
и~$\boldsymbol{p}_{\mathrm{X}}$, $\boldsymbol{c}_{\mathrm{Y}}$ 
и~$\boldsymbol{p}_{\mathrm{Y}}$ с~компонентами соответственно ($i \hm\in I$, 
$j \hm\in J$):
  \begin{align*}
  c_{ij}=\left\vert  \boldsymbol{D}_{ij}\right\vert &= 
  \int\limits_{y_{j-1}}^{y_j} 
\int\limits^{x_i}_{x_{i-1}} c_{ij}(x,y)\,dxdy\,,\\
  p_{ij} &= \left\|  \boldsymbol{D}_{ij}\right\| = \int\limits_{y_{j-1}}^{y_j} 
\int\limits^{x_i}_{x_{i-1}} p_{ij}(x,y)\,dxdy\,;\\
  c_{\mathrm{X};i} =\left\vert \boldsymbol{D}_{\mathrm{X};i}\right\vert 
&=\int\limits_{x_{i-1}}^{x_i} {c}_{\mathrm{X}}(x)\,dx\,,\notag\\ 
  p_{\mathrm{X};i} &=\left\| \boldsymbol{D}_{\mathrm{X};i}\right\| 
=\int\limits_{x_{i-1}}^{x_i} \boldsymbol{p}_1(x)\,dx\,;%\label{e1-ags}
\\
  c_{\mathrm{Y};j} =\left\vert \boldsymbol{D}_{\mathrm{Y};j}\right\vert 
&=\int\limits_{y_{j-1}}^{y_j} {c}_{\mathrm{Y}}(y)\,dy\,,\notag\\ 
   p_{\mathrm{Y};j} &=\left\| \boldsymbol{D}_{\mathrm{Y};j}\right\| 
   =\int\limits_{y_{j-1}}^{y_j} p_2 (y)\,dy \,. %\label{e2-ags}
\end{align*}
  
  \section{Оптимизация на~сценарном тройственном рынке}
  
  Под оптимизацией на сценарном рынке понимается построение сценарного 
портфеля при\-ме\-нением дискретного алгоритма, получаемого про\-ецированием 
на сценарный рынок алгоритма\linebreak оптимизации для теоретического рынка. Как 
обычно в~задачах с~CC-VaR, в~качестве дискретного здесь используется 
\textit{стандартный дискретный алгоритм}~\cite{2-ags, 3-ags, 4-ags, 5-ags}, 
основанный на процедуре Ней\-ма\-на--Пир\-со\-на~[7]. В~алгоритме основные 
агрегаты, такие как~$\boldsymbol{p}$ и~$\boldsymbol{c}$, участвуют 
в~векторной форме (так же и~для двумерного рынка). 
  
  В краткой форме он записывается последовательностью операций: 
\begin{multline*}
\boldsymbol{\rho} = \fr{\boldsymbol{p}}{\boldsymbol{c}}\,,\enskip 
\boldsymbol{\xi} = \mathbf{O}(\boldsymbol{\rho})\,,\enskip
\boldsymbol{\eta}= \mathbf{O}(\boldsymbol{\xi})\,, \enskip
\boldsymbol{d} =\boldsymbol{p}(\boldsymbol{\xi})\,,\\
\mathbf{T} = \left\{
t_{ij} =
\begin{cases}
1,& i\leq j;\\
0,& i> j
\end{cases}\right\}
%\right\}
\,,\enskip 
\boldsymbol{\varepsilon}= \mathbf{T}\cdot \boldsymbol{d}\,;\\
\boldsymbol{b}= \phi(\boldsymbol{\varepsilon})\,,\enskip
\boldsymbol{g}= \boldsymbol{b}(\boldsymbol{\eta})\,, 
\end{multline*}
 где $\mathbf{O}$~---  
оператор упорядочения векторов в~порядке возрастания их компонент.
  
  Перед запуском его работы с~целью построения оптимального 
комбинированного портфеля в~базисе, составленном из инструментов трех 
рынков, потребуется специальная подготовка исходных данных. 
  
  \subsection{Правила замещения }
  
  Представления портфелей \#0, \#X и~\#Y с~весовыми векторами 
$\boldsymbol{g}$, $\boldsymbol{g}_{\mathrm{X}}$ 
и~$\boldsymbol{g}_{\mathrm{Y}}$ соответственно имеют вид:
\begin{align*}
  \boldsymbol{G}&=\sum\limits_{\substack{
  {i\in I}\\{j\in J}}} g_{ij}\boldsymbol{D}_{ij}\,;\\
  \boldsymbol{G}_{\mathrm{X}}&= \sum\limits_{i\in I} g_{\mathrm{X};i} 
\boldsymbol{D}_{\mathrm{X},i}\,;\\
 \boldsymbol{G}_{\mathrm{Y}} &=\sum\limits_{j\in J} g_{\mathrm{Y};j}
 \boldsymbol{D}_{\mathrm{Y};j}\,.
\end{align*}
  
  Применение CC-VaR предполагает анализ относительных доходов 
   $\rho_{ij}\hm\equiv p_{ij}/c_{ij}$, 
   $\rho_{\mathrm{X};i}\hm\equiv p_{1;i}/c_{\mathrm{X};i}$ 
и~$\rho_{\mathrm{Y};j}\hm\equiv p_{2;j}/c_{\mathrm{Y};j}$, $i\hm\in I$, $j\hm\in J$, 
для рынков \#0, \#X и~\#Y соответственно.

  С помощью этих характеристик основные \textit{правила замещения}~(4)--(6) 
в~[6], одновременно вводящие двумерные множества индексов (сценариев) 
$M_0$, $M_1$, $M_2 \hm\subset I\times  J$, теперь записываются в~виде:

\noindent
  \begin{align}
  (i,j)\in M_0 &\Leftrightarrow \left\{ \rho_{ij}\geq 
\rho_{\mathrm{X};i}\&\rho_{ij}\geq \rho_{\mathrm{Y};j}\right\}\,;\label{e3-ags}\\
  (i,j)\in M_1 &\Leftrightarrow \left\{ 
\rho_{\mathrm{X};i}>\rho_{ij}\&\rho_{\mathrm{X};i}\geq 
\rho_{\mathrm{Y};j}\right\}\,;\label{e4-ags}\\
  (i,j)\in M_2 &\Leftrightarrow  
\left\{ \rho_{\mathrm{Y};j}>\rho_{ij}\&\rho_{\mathrm{Y};j}> 
\rho_{\mathrm{X};i}\right\}\,.\label{e5-ags}
  \end{align}
  
  Множества $M_0$, $M_1$ и~$M_2$ взаимно не пересекаются 
и~в~объединении дают полное множество ${\sf X}\times {\sf Y}$. Функция 
замещения в~дискретном случае трансформируется в~\textit{матрицу 
замещений} (по\-ме\-ча\-ющую сценарии индексами~1, 2, 3):
  \begin{multline}
  \mathbf{A}=\left\| a_{ij}\right\|\,,\enskip a_{ij}=k\Leftrightarrow (i,j)\in M_k\,,\\
  k=0,1,2,\enskip i\in I\,,\enskip j\in J\,.
  \label{e6-ags}
  \end{multline}
  
  Для рынков \#X и~\#Y используются векторы $\boldsymbol{a}_{\mathrm{X}}$ 
и~$\boldsymbol{a}_{\mathrm{Y}}$ с~компонентами: 
  \begin{align}
  a_{\mathrm{X};i}&= \left.
  \begin{cases}
  1, & \exists j\in J: a_{ij}=1\,;\\
  0 & \mbox{иначе}
  \end{cases}\right\}\,,\ 
  i\in I\,;
  \label{e7-ags}
  \\
  a_{\mathrm{Y};j} &= \left.\begin{cases}
  1\,, & \exists i\in I: a_{ij}=2\,;\\
  0 & \mbox{иначе}
  \end{cases}\right\}\,,\
  j\in J\,.
  \label{e8-ags}
  \end{align}
  
  Первый выделяет сценарии замещения на рынке \#X, второй~--- на 
рынке~\#Y. 
  
  Рассматриваемую задачу, совмещающую все три рынка, назовем 
задачей~$A$. Аналогично могут быть рассмотрены и~две ее усеченные 
постановки~--- $B$ и~$C$. В~задаче~$B$ исключается рынок \#Y и~метка 
$k\hm = 2$ не используется, а ее правила замещения:
  \begin{align*}
  (i,j)\in M_0 &\Leftrightarrow \left\{ \rho_{ij}\geq \rho_{\mathrm{X};i}\right\}\,;\\
  (i,j)\in M_1 &\Leftrightarrow \left\{ \rho_{\mathrm{X};i}>\rho_{ij}\right\}\,.
  \end{align*}
  
  В задаче~$C$ исключается рынок \#0 и~не используется метка $k\hm = 0$, 
а~ее правила замещения: 
  \begin{align*}
  (i,j)\in M_1 &\Leftrightarrow \left\{ \rho_{\mathrm{X};i}\geq 
\rho_{\mathrm{Y};j}\right\}\,;\\
  (i,j) \in M_2 &\Leftrightarrow \left\{ \rho_{\mathrm{Y};j}> 
\rho_{\mathrm{X};i}\right\}\,.
  \end{align*}
  
  \subsection{Комбинированный портфель }
  
  По аналогии с~теоретической схемой и~в~соответствии с~правилами~(3)--(5) 
строится \textit{комбинированный} портфель замещением тех базисных 
инструментов рынка~\#0, для которых имеется базисный инструмент 
рынка~\#X или~\#Y с~б$\acute{\mbox{о}}$льшим относительным доходом. При 
этом также применяется рандомизация. 
  
  Формулы~(8)--(20) из~[6] для теоретического рынка переписываются 
в~несколько сокращенном виде применительно к~сценарному рынку 
с~естественной заменой переменных~$s$ и~$t$ индексами~$i$ и~$j$, 
а~множества~{\sf X} и~{\sf Y}~--- множествами индексов~$I$ и~$J$. 
  
  Вводятся множества $M_{1;i}\hm\subset J$ 
  и~$M_{2;j}\hm\subset I$ как максимальные 
подмножества множеств $M_1$ и~$M_2$ для фиксированных значений 
$i\hm\in I$ и~$j \hm\in J$ соответственно. Также определяются 
индикаторы~$\boldsymbol{M}_{1;i}$ и~$\boldsymbol{M}_{2;j}$ рынка~\#0 как 
объединение базисных инструментов~$\boldsymbol{D}_{ij}$ по $j \hm\in M_{1;i}$ 
и~$i\hm\in M_{2;j}$ соответственно: 
  \begin{multline}
   \boldsymbol{M}_{1;i}= \sum\limits_{j\in 
M_{1;i}}\boldsymbol{D}_{ij}=\boldsymbol{D}_{1;i}\times 
\boldsymbol{H}_2\left\{ M_{1;i}\right\}\,,\\
  \left\vert \boldsymbol{M}_{1;i}\right\vert = \sum\limits_{j\in M_{1;i}} 
c_{ij}\,,\enskip  \left\| \boldsymbol{M}_{1;i}\right\| =\sum\limits_{j\in M_{1;i}} 
p_{ij}\,;
  \label{e9-ags}
\end{multline}

\noindent
\begin{multline*}
  \boldsymbol{M}_{2;j}= \sum\limits_{i\in 
M_{2;j}}\boldsymbol{D}_{ij}=\boldsymbol{D}_{2;j}\times 
\boldsymbol{H}_1\left\{ M_{2;j}\right\}\,,\\
  \left\vert \boldsymbol{M}_{2;j}\right\vert= \sum\limits_{i\in M_{2;j}} 
c_{ij}\,,\enskip  \left\| \boldsymbol{M}_{2;j}\right\| =
\sum\limits_{i\in M_{2;j}} 
p_{ij}\,.
    \end{multline*}
  %
  Эти инструменты желательно было бы заместить <<гиб\-рид\-ны\-ми>>, 
совмещающими рынки \#0 и~\#X или \#0 и~\#Y, а~именно:
  \begin{equation}
  \left.
  \begin{array}{rl}
  \boldsymbol{M}_{\mathrm{X};i} &= \boldsymbol{D}_{\mathrm{X};i}\times 
\boldsymbol{H}_2\left\{ M_{1;i}\right\}\,;\\[6pt]
  \boldsymbol{M}_{\mathrm{Y};j} &= \boldsymbol{D}_{\mathrm{Y};j}\times 
\boldsymbol{H}_1\left\{ M_{2;j}\right\}\,.
  \end{array}
  \right\}
  \label{e10-ags}
  \end{equation}
  
  Поскольку таких инструментов на рассматриваемых рынках нет, для 
образования их реализуемого аналога на рынках \#X и~\#Y применяется 
рандомизация. 
  
  На рынке \#X для всех сценариев $i\hm\in I$ вводятся взаимонезависимые 
\textit{биномиальные} случайные величины $\vartheta_{\mathrm{X};i}$, 
вероятности \textit{успеха} (замещения) которых соответственно 
равны~$\theta_{\mathrm{X};i}$ и~в~совокупности образуют 
вектор~$\boldsymbol{\theta}_{\mathrm{X}}$:
  \begin{multline}
   \theta_{\mathrm{X};i} = {\sf P}\left\{ 
  M_{1;i}\vert i\right\} =\sum\limits_{j\in 
M_{1;i}} \fr{p_{ij}}{p_{1;i}}\,,\\
 \boldsymbol{\theta}_{\mathrm{X}}=\left\{ 
\theta_{\mathrm{X};i},\ i\in I\right\}\,.
  \label{e11-ags}
  \end{multline}
  
  Эта рандомизация встраивается непосредственно в~базисные инструменты 
одномерной части \#X комбинированного портфеля. Новые базисные 
инструменты становятся случайными, совпадающими с~инструментами 
$\boldsymbol{D}_{\mathrm{X};i}$ с~вероятностью~$\theta_{\mathrm{X};i}$ 
и~с~\textit{нулевыми} инструментами (с~нулевым доходом и~нулевой 
стоимостью) $\boldsymbol{N}_{\mathrm{X};i}$ с~вероятностью $1\hm- 
\theta_{\mathrm{X};i}$. Формально эти инструменты, их стоимость и~средний 
доход (также вероятность) соответственно равны: 
  \begin{multline}
   \boldsymbol{D}_{\mathrm{X};i}^{\mathrm{cmb}}=\vartheta_{\mathrm{X};i} 
\boldsymbol{D}_{\mathrm{X};i}\,,\enskip
  \left\vert  \boldsymbol{D}^{\mathrm{cmb}}_{\mathrm{X};i}\right\vert= 
c^{\mathrm{cmb}}_{\mathrm{X};i}=\theta_{\mathrm{X}} c_{\mathrm{X};i}\,,\\
  \left\| \boldsymbol{D}^{\mathrm{cmb}}_{\mathrm{X};i}\right\| =
  p^{\mathrm{cmb}}_{\mathrm{X};i} 
=\theta_{\mathrm{X};i} p_{1;i}\,,\enskip i\in I\,.
   \label{e12-ags}
  \end{multline}
  
  Назначение~(\ref{e11-ags}) параметров~$\theta_{\mathrm{X};i}$ уравнивает 
вероятности, связанные с~инструментами 
$\boldsymbol{M}_{\mathrm{X};i}$~(\ref{e10-ags}) и~$\boldsymbol{M}_{1;i}$~(\ref{e9-ags}). 
И~подтверждением этому служит именно третье соотношение  
в~(\ref{e12-ags}). 
  
  Аналогично для рынка \#Y вводятся случайные величины 
$\vartheta_{\mathrm{Y};j}$, $j\hm\in J$, с~вероятностью 
замещения~$\theta_{\mathrm{Y};j}$, и~тогда
  \begin{equation}
  \left.
  \begin{array}{c}
  \!\hspace*{-2.5mm}\boldsymbol{D}^{\mathrm{cmb}}_{\mathrm{Y};j}=\vartheta_{\mathrm{Y};j} 
\boldsymbol{D}_{\mathrm{Y};j}\,,\
  \left\vert \boldsymbol{D}^{\mathrm{cmb}}_{\mathrm{Y};j}\right\vert = 
c^{\mathrm{cmb}}_{\mathrm{Y};j}=\theta_{\mathrm{Y};j} c_{\mathrm{Y};j}\,,\\[6pt]
  \!\hspace*{-2.5mm}\left\| \boldsymbol{D}^{\mathrm{cmb}}_{\mathrm{Y};j} \right\| = 
p^{\mathrm{cmb}}_{\mathrm{Y};j}=\theta_{\mathrm{Y};j} p_{2;j}\,,\\[6pt]
  \!\hspace*{-2.5mm}\theta_{\mathrm{Y};j} ={\sf P}\left\{ M_{2;j}\vert j\right\} =
  \displaystyle\sum\limits_{i\in M_{2;j}} 
  \fr{p_{ij}}{p_{2;j}}\,,\\
  \!\hspace*{-2.5mm}\boldsymbol{\theta}_{\mathrm{Y}} = \left\{ 
\theta_{\mathrm{Y};j}\,,\ j\in J\right\}.
  \end{array}\!
  \right\}\!\!
  \label{e13-ags}
  \end{equation}
  
  Затем для нового базиса формируется новая \textit{единая функция} 
относительных доходов, и~к~ней применяется комбинированный алгоритм 
оптимизации, находящий весовые коэффициенты оптимального (случайного) 
портфеля. Имеем: 
  \begin{multline}
  \boldsymbol{G}^{\mathrm{cmb}}=\sum\limits_{a_{ij}=0} 
g_{ij}^{\mathrm{cmb}}\boldsymbol{D}_{ij} +
\sum\limits_{i\in I} g^{\mathrm{cmb}}_{\mathrm{X};i} 
\vartheta_{1;i} \boldsymbol{D}_{\mathrm{X};i}+ {}\\
{}+\sum\limits_{j\in J} 
g^{\mathrm{cmb}}_{\mathrm{Y};j} \vartheta_{2;j}\boldsymbol{D}_{\mathrm{Y};j}\,.
  \label{e14-ags}
  \end{multline}
%  
  Это представление порождает и~дискретную \textit{идеалистичную} версию 
комбинированного портфеля с~эквивалентной платежной функцией. Она 
получается заменой случайных базисных инструментов нереализуемыми на 
рынке детерминированными~(\ref{e10-ags}) (с~сохранением всех весовых 
коэффициентов):
{\looseness=-1

}

\noindent
\begin{multline*}
  \boldsymbol{G}^{\mathrm{idl}}={}\\
  {}=\sum\limits_{(i,j)\in M_0} g_{ij}^{\mathrm{cmb}} 
\boldsymbol{D}_{ij} +\sum\limits_{i\in I} 
g_{\mathrm{X};i}^{\mathrm{cmb}}\boldsymbol{M}_{\mathrm{X};i}+\sum\limits_{j\in J} 
g^{\mathrm{cmb}}_{Y;j}\boldsymbol{M}_{\mathrm{Y};j}.
  \end{multline*}
  
  Для графической иллюстрации платежной функции идеалистичного 
портфеля ее удобно рассчитывать по формуле:
  \begin{equation}
  \boldsymbol{\pi}_{ij}^{\mathrm{idl}}(x,y)=\max \left( \boldsymbol{\pi}_{0;ij}^{\mathrm{idl}} (x,y), 
\boldsymbol{\pi}_{\mathrm{X};i}^{\mathrm{idl}}(x), 
\boldsymbol{\pi}_{\mathrm{Y};j}^{\mathrm{idl}}(y)\right).\!\!
  \label{e15-ags}
  \end{equation}
  
  В задачах~$B$ и~$C$ соответственно: 
  \begin{align*}
  \boldsymbol{G}^{\mathrm{cmb}} &= \sum\limits_{a_{ij}=0} 
g_{ij}^{\mathrm{cmb}}\boldsymbol{D}_{ij} + \sum\limits_{i\in I} 
g^{\mathrm{cmb}}_{\mathrm{X};i}\vartheta_{1;i} \boldsymbol{D}_{\mathrm{X};j}\,;\\
  \boldsymbol{G}^{\mathrm{cmb}} &= \sum\limits_{i\in I} 
g_{\mathrm{X};i}^{\mathrm{cmb}}\vartheta_{1;i}\boldsymbol{D}_{\mathrm{X};i} + 
\sum\limits_{j\in J} g^{\mathrm{cmb}}_{\mathrm{Y};j}\vartheta_{2;j} 
\boldsymbol{D}_{\mathrm{Y};j}\,.
  \end{align*}
  %
   В этих двух представлениях сохраняются те же обозначения для весовых коэффициентов 
   и~величин $\vartheta_{\mathrm{X};i}$ и~$\vartheta_{\mathrm{Y};j}$, что и~в~задаче~$A$, но все они, 
вообще говоря, другие, так как получаются из иных \textit{правил замещений}. Также соответственно 

\noindent
  \begin{equation}
  \left.
  \begin{array}{l}
  \boldsymbol{G}^{\mathrm{idl}}=\displaystyle\sum\limits_{(i,j)\in M_0} 
g_{ij}^{\mathrm{cmb}}\boldsymbol{D}_{ij} +
\displaystyle\sum\limits_{i\in I} g^{\mathrm{cmb}}_{\mathrm{X};i} 
\boldsymbol{M}_{\mathrm{X};i}\,;\\[12pt]
  \boldsymbol{G}^{\mathrm{idl}}=\displaystyle \sum\limits_{i\in I} 
g_{\mathrm{X};i}^{\mathrm{cmb}}\boldsymbol{M}_{\mathrm{X};i} +\sum\limits_{j\in J} 
g^{\mathrm{cmb}}_{\mathrm{Y};j} \boldsymbol{M}_{\mathrm{Y};j}\,;\\[12pt]
  \boldsymbol{\pi}_{ij}^{\mathrm{idl}}(x,y)=\max\left( \boldsymbol{\pi}_{0;ij}^{\mathrm{idl}} (x,y), 
\boldsymbol{\pi}^{\mathrm{idl}}_{\mathrm{X};i}(x)\right);\\[9pt]
  \boldsymbol{\pi}_{ij}^{\mathrm{idl}}(x,y)=\max\left( 
\boldsymbol{\pi}_{\mathrm{X};i}^{\mathrm{idl}} (x), 
\boldsymbol{\pi}^{\mathrm{idl}}_{\mathrm{Y};j}(y)\right).
  \end{array}
  \right\}
  \label{e16-ags}
  \end{equation}
  
  \subsection{Суррогатный портфель }
  
  Наряду с~комбинированным строится и~портфель, называемый 
\textit{суррогатным}. Он получается в~результате формальной замены 
базисных инструментов для всех сценариев (по необходимости) новыми, для 
которых платежные функции и~вероятности те же, что и~для инструментов 
$\boldsymbol{D}_{ij}$, $i \hm\in I$, $j \hm\in J$, рынка~\#0, но цены 
корректируются соответственно правилам~(\ref{e3-ags})--(\ref{e5-ags}) 
и~ценам рынков~\#X и~\#Y: 
  \begin{equation}
  c_{ij}^{\mathrm{srg}}=%\left\{ 
  \begin{cases}
  c_{ij}, & (i,j)\in M_0\,;\\
  \fr{p_{ij}}{\rho_{\mathrm{X};i}},  &(i,j)\in M_1\,;\\
  \fr{p_{ij}}{\rho_{\mathrm{Y};j}}, &(i,j)\in M_2\,.
  \end{cases}
  %\right\}
  \label{e17-ags}
  \end{equation}
  %
  С этими ценами вновь образуется матрица относительных доходов, алгоритм 
находит матрицу новых весов портфеля, а~сам портфель имеет вид: 
  \begin{equation}
  \boldsymbol{G}^{\mathrm{srg}}=\sum\limits_{i\in I, j\in J} g_{ij}^{\mathrm{srg}} 
\boldsymbol{D}_{ij}^{\mathrm{srg}}\,.
  \label{e18-ags}
  \end{equation}
  
  На сценарном рынке числовые показатели для суррогатного портфеля из-за 
замены части базисных инструментов будут, как правило, слегка отличаться от 
комбинированного (и идеалистичного) портфеля. Тем не менее относительно 
просто конструируемый \textit{суррогатный} портфель можно использовать 
как дополнительное средство проверки всех расчетов (и графиков доходов). 
  
  В задачах~$B$ и~$C$ иными становятся цены: 
  \begin{align*}
  c_{ij}^{\mathrm{srg}} &= %\left\{ 
  \begin{cases}
  c_{ij}, &(i,j)\in M_0;\\
  \fr{p_{ij}}{\rho_{\mathrm{X};i}}, & (i,j)\in M_1\,;
    \end{cases}%\right\}
    \\
  c_{ij}^{srg} &= 
  %\left\{ 
  \begin{cases}
  \fr{p_{ij}}{\rho_{\mathrm{X};i}},& (i,j)\in M_1;\\
\fr{p_{ij}}{\rho_{\mathrm{Y};j}}, &(i,j)\in M_2\,.
\end{cases}
%\right\}
  \end{align*}
  %
  По этим ценам пересчитываются матрицы относительных доходов и~новых 
весов портфелей, а~для построения портфелей применяется та же 
формула~(\ref{e18-ags}). 
  
  \section{Алгоритм построения комбинированного портфеля}
   
  Построение комбинированного портфеля (\textit{комбинированный} 
алгоритм) и~его идеалистичной и~суррогатной версий основано на применении 
\textit{стандартного} алгоритма~\cite{2-ags, 3-ags, 4-ags, 5-ags}, краткая запись 
которого приводится в~начале разд.~3. Для его запуска требуется 
предварительная (до применения \textit{стандартного} алгоритма) 
трансформация мат\-риц~$\boldsymbol{c}$, $\boldsymbol{p}$ 
и~$\boldsymbol{\rho}$ для рынка \#0 и~векторов $\boldsymbol{c}_{\mathrm{X}}$, 
$\boldsymbol{p}_{\mathrm{X}}$ и~$\boldsymbol{\rho}_{\mathrm{X}}$ для 
рынка~\#X и~$\boldsymbol{c}_{\mathrm{Y}}$, $\boldsymbol{p}_{\mathrm{Y}}$ 
и~$\boldsymbol{\rho}_{\mathrm{Y}}$ для рынка~\#Y. 
  
  Схема \textit{комбинированного} алгоритма приводится без технических 
подробностей. Она состоит из нескольких последовательных блоков. 
  \begin{enumerate}[1.]
  \item  Нахождение матрицы замещений~$\mathbf{A}$ и~по\-стро\-ение матриц 
$\boldsymbol{c}_{0\mathrm{N}}$, $\boldsymbol{p}_{0\mathrm{N}}$ 
и~$\boldsymbol{\rho}_{0\mathrm{N}}$ операцией обнуления  
в~мат\-ри\-цах~$\boldsymbol{c}$, $\boldsymbol{p}$ и~$\boldsymbol{\rho}$ всех 
элементов $(i,j)$, для которых $a_{ij}\not= 0$. 
  \item Образование матриц $\boldsymbol{p}_{\mathrm{XN}}$ 
и~$\boldsymbol{p}_{\mathrm{YN}}$ обнулением всех элементов  
в~мат\-ри\-це~$\boldsymbol{p}$, для которых соответственно $a_{ij}\not=1$ 
и~$a_{ij}\not= 2$. 
  \item  Определение векторов суммарной вероятности замещения 
$\boldsymbol{p}_{\mathrm{XM}}\hm = \left\{ {\sf P}\{M_{1;i}\},\ i \hm\in I\right\}$ 
и~$\boldsymbol{p}_{\mathrm{YM}}\hm = \left\{ {\sf P}\{ M_{2;j}\},\ j \hm\in J\right\}$ 
на основе результатов п.~2, а~также векторов 
$\boldsymbol{\theta}_{\mathrm{X}}\hm = \left\{\theta_{\mathrm{X};i}, 
i \hm\in I\right\}$ 
и~$\boldsymbol{\theta}_{\mathrm{Y}}\hm = \left\{\theta_{\mathrm{Y};j}, 
j \hm\in J\right\}$ согласно (\ref{e11-ags}) и~(\ref{e13-ags}) соответственно. 
  \item  Вычисление (в~дополнение к~$\boldsymbol{p}_{\mathrm{XM}}$ 
и~$\boldsymbol{p}_{\mathrm{YM}}$) векторов 
$\boldsymbol{c}_{\mathrm{XM}}\hm= \{a_{\mathrm{X};i} 
\theta_{\mathrm{X};i}c_{\mathrm{X};i},\ i\hm\in I\}$,
  $\boldsymbol{c}_{\mathrm{YM}}\hm= \{a_{\mathrm{Y};j} 
\theta_{\mathrm{Y};j}c_{\mathrm{Y};j},\  
j\hm\in J\}$~(\ref{e11-ags})--(\ref{e13-ags}) 
и~$\boldsymbol{\rho}_{\mathrm{XM}}\hm= \{ a_{\mathrm{X};i} 
\rho_{\mathrm{X};i},\ i\hm\in I\}$, $\boldsymbol{\rho}_{\mathrm{YM}}\hm= \{ 
a_{\mathrm{X};i}\rho_{\mathrm{Y};j}, \ j\hm\in J\}$ (параметры 
$a_{\mathrm{X};i}$~(\ref{e7-ags}) и~$a_{\mathrm{Y};j}$~(\ref{e8-ags}) 
используются во избежание деления на нуль).
  \item Преобразование матриц $\boldsymbol{c}_{\mathrm{0N}}$, 
$\boldsymbol{p}_{\mathrm{0N}}$ и~$\boldsymbol{\rho}_{\mathrm{0N}}$ 
в~векторы (без изменения обозначений) и~формирование комбинированной 
тройки векторов ($\boldsymbol{c}_{\mathrm{F}}$, 
$\boldsymbol{p}_{\mathrm{F}}$ и~$\boldsymbol{\rho}_{\mathrm{F}}$) 
размерности $nm \hm+ n \hm+ m$ каждый операцией конкатенации: 
$\boldsymbol{c}_{\mathrm{F}}\hm= (\boldsymbol{c}_{\mathrm{0N}}, 
\boldsymbol{c}_{\mathrm{XM}}, \boldsymbol{c}_{\mathrm{YM}})$; 
$\boldsymbol{p}_{\mathrm{F}}\hm = (\boldsymbol{p}_{\mathrm{0N}}, 
\boldsymbol{p}_{\mathrm{XM}}, \boldsymbol{p}_{\mathrm{YM}})$; 
$\boldsymbol{\rho}_{\mathrm{F}}\hm = (\boldsymbol{\rho}_{\mathrm{0N}}, 
\boldsymbol{\rho}_{\mathrm{XM}}, \boldsymbol{\rho}_{\mathrm{YM}})$. 
  \item Применение \textit{стандартного} алгоритма оптимизации к~тройке 
векторов ($\boldsymbol{c}_{\mathrm{F}}$, $\boldsymbol{p}_{\mathrm{F}}$, 
$\boldsymbol{\rho}_{\mathrm{F}}$) для нахождения 
вектора~$\boldsymbol{g}_{\mathrm{F}}$ весовых коэффициентов 
оптимального комбинированного портфеля с~разбиением на три подвектора: 
$\boldsymbol{g}_{\mathrm{F}}\hm= (\boldsymbol{g}_{\mathrm{0F}}, 
\boldsymbol{g}_{\mathrm{XF}}, \boldsymbol{g}_{\mathrm{YF}})$. 
  
  {\small \textit{Примечание.} В~наборах, получаемых в~п.~5, в~результате обнуления в~п.~1 
и~п.~2 содержится, как правило, значительная доля нулевых троек. Но это не сказывается на 
окончательном результате п.~6, так как этим тройкам придается нулевой вес. Зато такой 
вариант алгоритма упрощает вычисления. Тем не менее из соображений строгости и~чистоты 
алгоритма, а также буквального следования представлению~(\ref{e14-ags}), но уже 
с~векторами $\boldsymbol{g}^{\mathrm{cmb}}$, 
$\boldsymbol{g}^{\mathrm{cmb}}_{\mathrm{X}}$ 
и~$\boldsymbol{g}_{\mathrm{Y}}^{\mathrm{cmb}}$, все подвергшиеся обнулению элементы нужно 
было бы удалить из наборов с~запоминанием их положения для последующего выхода на 
представление~(\ref{e14-ags}).}
  
  
  \item  Вычисление записей основных результатов инвестиции 
$\boldsymbol{L}\hm= \langle A, R, y\rangle$ ($A$~--- инвестиционная сумма; 
$R$~--- средний доход; $y$~--- средняя доходность) и~построение графиков 
платежных функций 
 портфелей~(\ref{e13-ags})--(\ref{e16-ags}) как по частям, так и~совместно 
(в~варианте идеалистичного портфеля). 
  \item Построение суррогатного портфеля путем формирования стоимостного 
$\boldsymbol{c}^{\mathrm{srg}}$ по формуле~(\ref{e17-ags}) и~прогнозного 
$\boldsymbol{p}^{\mathrm{srg}} (\equiv \boldsymbol{p}$) векторов размерности $n\times 
m$, образованием из них вектора относительных 
доходов~$\boldsymbol{\rho}^{\mathrm{srg}}$ той же размерности с~последующим 
применением \textit{стандартного} алгоритма. 
  \end{enumerate}
  
  \section{Иллюстративные примеры}
  
  Рассматриваются примеры с~${\sf X}\hm= {\sf Y}\hm= [0,1)$, при этом $x, 
s\hm\in {\sf X}$, $y,t\hm\in {\sf Y}$. В~них используется функция рисковых 
предпочтений инвестора $\phi(\varepsilon)\hm=\varepsilon^2$, 
$\varepsilon\hm\in [0, 1]$, а при задании стоимостных и~вероятностных 
характеристик существенная роль отводится двухпараметрическому 
бе\-та-рас\-пре\-де\-ле\-нию $\mathrm{Be}(u, v)$ с~плотностью
  $$
  \beta(x;u,v)\hm=\fr{x^{u-1}(1-x)^{v-1}}{B(u,v)} \,,\enskip u, v>0\,,
  $$
где 
     $B(u,v)\hm= \int\nolimits_0^1 x^{u-1} (1\hm-x)^{v-1}\,dx 
     \hm= \Gamma(u)\Gamma(v)/\Gamma(u\hm+v)$
и~$\Gamma(u)\hm= \int\nolimits_0^{\infty} x^{u\hm-1} \exp(-x)\,dx$~--- 
соответственно бе\-та- и гам\-ма-функ\-ции. 

  Для рынка \#0 с~помощью бе\-та-плот\-ности предварительно строится 
нормированная двумерная плотность $f(x,y; u,v)$: 
 \begin{equation}
 \left.
 \begin{array}{rl}
  f^\prime (x,y; u,v)&=
%\left\{ 
\begin{cases} 
x^{u-1}(1-y)^{v-1}\,, & x\leq y;\\
   y^{u-1}(1-x)^{v-1}\,,& x>y;\end{cases}
   %\right\}
  \\[12pt]
  N(u,v)&=\displaystyle\int\limits_0^1 \int\limits_0^1 f^\prime (x,y;u,v)\,dxdy\,;\\[6pt]
  f(x,y;u,v)&=\fr{f^\prime(x,y; u,v)}{N(u,v)}\,,
  \end{array}
  \right\}
  \label{e19-ags}
  \end{equation}
  %
  а затем стоимостную и~прогнозную плотности для этого рынка задаем 
соответственно в~виде вы\-пук\-лых комбинаций: 
  \begin{multline}
  c(x,y)=\omega_c f(x,y;2{,}8;1{,}8)+{}\\
  {}+(1-\omega_c)\beta(x;3;3)\beta(y;3;3)\,; 
\label{e20-ags}
  \end{multline}
  
  %\vspace*{-12pt}
  
  { \begin{center}  %fig1
 \vspace*{-1pt}
    \mbox{%
 \epsfxsize=76.396mm 
 \epsfbox{aga-1.eps}
 }


\vspace*{6pt}

\noindent
{{\figurename~1}\ \ \small{
Плотность $p(1 - x, y)$
}}
\end{center}}

  
     
  \vspace*{-3pt}
  
  \noindent
  \begin{multline}
  \!\!\!p(x,y)=\omega_p f(x,y;3;2)+(1-\omega_p)\beta(x;3;3)\beta(y;3;3)\,,\\
  \omega_p=0{,}5\,,\enskip \omega_c=0{,}5\,.
  \label{e21-ags}
  \end{multline}
  
  \vspace*{-2pt}
  
  \noindent
(С~ростом параметра $\omega_p\hm\in [0,1]$ корреляция между базовыми 
активами увеличивается.) На рис.~1 приводится график плотности $p(x, y)$. 


  
  Для большей наглядности график изображается с~заменой $x\hm\to 1\hm - x$. 
Различие между плотностями~(\ref{e20-ags}) и~(\ref{e21-ags}) весьма 
незначительно, и~потому график $c(x,y)$ не приводится. Отметим лишь, что 
отличие графика плотности $c(x,y)$ проявляется в~меньшем угле излома на 
диагонали квадрата $[0, 1]\times[0, 1]$ и~снижении максимальных значений. 
Такое соотношение плотностей говорит о~большем разбросе стоимостной 
плотности в~сравнении с~прогнозной. Подобная задача на языке финансового 
рынка означает \textit{продажу волатильности}. 
  
  Для стоимостных плотностей рынков \#X и~\#Y принимается также 
  
  \vspace*{-5pt}
  
\noindent
  \begin{multline}
  c_{\mathrm{X}}(x)=\chi_{\mathrm{X}} \left( \omega_x c_1(x)+(1-
\omega_x)\beta(x;2;2)\right),\\[-0.5pt]
  \omega_x=0{,}9\,;
  \label{e22-ags}
  \end{multline}
  
  \vspace*{-14pt}
  
  \noindent
  \begin{multline}
  c_{\mathrm{Y}}(y)=\chi_{\mathrm{Y}} \left( \omega_y c_2(y)+(1-
\omega_y)\beta(y;2;2)\right),\\[-0.5pt]
  \omega_y=0{,}9\,.
  \label{e23-ags}
  \end{multline}
  
  \vspace*{-4pt}
  
  \noindent
При $\chi_{\mathrm{X}}=\chi_{\mathrm{Y}}\hm=1$ эти функции приобретают 
все свойства плотности вероятности. 
  
  В отличие от стоимостных плотностей про\-гноз\-ное распределение 
вероятности, как и~в~[6], едино для всех рассматриваемых рынков, и~потому
  $p_{\mathrm{X}}(x)\hm\equiv p_1(x)$; $p_{\mathrm{Y}}(y)\hm\equiv p_2(y)$. 
  
  Дискретизация множества {\sf X} проводится при $n\hm=13$, а~{\sf Y}~--- 
при $m \hm= 12$, порождая тем самым $n\times m\hm= 156$ двумерных 
сценариев, при этом\linebreak
 стоимостные и~прогнозные характеристики опре\-деляются 
для всех сценариев, одномерных и~дву\-мерных, интегрированием 
функций~(\ref{e19-ags})--(\ref{e23-ags}).\linebreak Отметим лишь, что для дискретной 
модели при
 нахождении маргинальных цен и~вероятностей од-\linebreak\vspace*{-12pt}

\columnbreak

\noindent
номерных 
сценариев достаточно ограничиться однократным суммированием по $j\hm\in 
J$ и~$i\hm\in I$ соответственно цен и~вероятностей для двумерных сценариев. 
  
  Заметим еще, что при интегрировании плот\-ности типа 
 бе\-та-рас\-пре\-де\-ле\-ния в~пределах крайних сценариев при некоторых 
значениях параметров из интервала $(-1, 0)$ приходится иметь дело 
с~интегрируемой особенностью в~точке $x\hm= 1$. Для обеспечения 
корректности счета имеет смысл выделять из плотности <<чистую>> 
особенность, допускающую аналитическое интегрирование. Остаточный 
интеграл особенностей уже не содержит и~без труда приближенно 
интегрируется. Так, если $v\hm\in (-1,0)$ и~$u\hm>0$, то при $0\hm<\tau \hm< 1$ 
имеет место:

\vspace*{-12pt}

\noindent
  \begin{multline*}
  \int\limits^1_\tau x^u(1-x)^v \,dx={}\\[-12pt]
  {}=\int\limits_\tau^1(1-x)^v \,dx+\int\limits_\tau^1 
(x^u-1)(1-x)^v \,dx={}\\[-2pt]
  {}=\fr{1-\tau^{v+1}}{v+1} +\int\limits^1_\tau (x^u-1)(1-x)^v \,dx\,.
  \end{multline*}
  
  \vspace*{-12pt}
  
  По полученным вероятностям и~ценам находятся относительные доходы для 
рынков \#0, \#X и~\#Y, необходимые для проведения операций замещения. При 
этом двумерные массивы трансформируются в~векторы c лексикографическим 
упорядочением компонент.
  
  Применением стандартного дискретного алгоритма оптимизации ко всем 
исходным рынкам по отдельности можно находить для них векторы весов 
оптимальных портфелей и~доходности инвестиции. Так, для рынков \#0, \#X 
и~\#Y при $\chi_{\mathrm{X}}\hm=1$ и~$\chi_{\mathrm{Y}}\hm=1$ они 
соответственно равны 0{,}0423358, 0{,}0313433 и~0{,}0312055. Их можно 
использовать в~качестве ориентиров для сравнения результатов. 
  
  
  При построении комбинированного портфеля используется алгоритм разд.~4. 
С~фиксацией $\chi_{\mathrm{X}}\hm=1$ и~$\chi_{\mathrm{Y}}\hm=1$ 
находится матрица замещений~$\mathbf{A}$~(\ref{e6-ags}):

\noindent
  $$
  \left\|
  \begin{array}{cccccccccccc}
0&0&2&2&2&2&2&2&2&2&2&2\\
0&0&0&2&2&2&2&2&2&2&2&1\\
1&0&0&0&2&2&2&2&2&2&2&1\\
1&0&0&0&0&2&2&2&2&2&1&1\\
1&1&0&0&0&0&2&2&2&1&1&1\\
1&1&1&0&0&0&0&2&2&1&1&1\\
1&1&1&1&0&0&0&0&1&1&1&1\\
1&1&1&1&1&0&0&0&1&1&1&1\\
1&1&1&1&1&1&0&0&0&1&1&1\\
1&1&1&1&1&2&2&0&0&0&1&1\\
1&1&1&1&2&2&2&2&2&0&1&1\\
1&1&2&2&2&2&2&2&2&2&0&1\\
1&2&2&2&2&2&2&2&2&2&2&2
\end{array}
\right\|
\vspace*{1pt}
$$

\pagebreak

\noindent
(в~ней 35~нулей, 60~единиц и~61~двойка), а также векторы~(\ref{e7-ags}) 
и~(\ref{e8-ags}):
\begin{align*}
\boldsymbol{a}_{\mathrm{X}} &= \{0, 1, 1, 1, 1, 1, 1, 1, 1, 1, 1, 1, 1\};\\ 
\boldsymbol{a}_{\mathrm{Y}} &= \{0, 1, 1, 1, 1, 1, 1, 1, 1, 1, 1, 1\}.
\end{align*}
  
  По матрице~$\mathbf{A}$ и~распределению вероятностей на 
сценариях~(\ref{e21-ags}) (двумерных и~одномерных) определяются 
необходимые для построения комбинированного портфеля 
векторы~$\boldsymbol{\theta}_{\mathrm{X}}$ 
и~$\boldsymbol{\theta}_{\mathrm{Y}}$ параметров рандомизации, состоящие из 
вероятностей переключения на одномерные части портфеля и~вы\-чис\-ля\-емые по 
формулам~(\ref{e11-ags}) и~(\ref{e13-ags}): 
  \begin{align*}
  \boldsymbol{\theta}_{\mathrm{X}} &= \{0; 0{,}0062; 0{,}0139; 
  0{,}0434; 0{,}1385;    0{,}2017;\\
  &\hspace*{-5mm}0{,}4047;  0{,}532; 0{,}539; 0{,}309; 0{,}2489; 0{,}0788; 0{,}0012\};\\
  \boldsymbol{\theta}_{\mathrm{Y}} &= \{0; 0{,}0051; 0{,}0353; 0{,}0617; 
  0{,}1669; 0{,}3336;\\
    & 0{,}4396;  0{,}4511; 0{,}4556; 0{,}1992; 0{,}0842; 0{,}1241\}. 
  \end{align*}
  
  Результатом реализации п.~5 алгоритма оптимизации становится тройка 
векторов ($\boldsymbol{c}_{\mathrm{F}}$, $\boldsymbol{p}_{\mathrm{F}}$, 
$\boldsymbol{\rho}_{\mathrm{F}}$) размерности $nm \hm+ n \hm+ m \hm= 181$ 
каждый, а~п.~6~--- весовой вектор~$\boldsymbol{g}_{\mathrm{F}}$ той же 
длины с~сохранением всех элементов обнуления. В~них первые $nm$ 
элементов отвечают рынку \#0, следующие~$n$~--- рынку \#X, 
последние~$m$~--- рынку \#Y. Разбиение 
вектора~$\boldsymbol{g}_{\mathrm{F}}$ на такие три блока дает векторы весов 
для трех час\-тей комбинированного портфеля с~представлением,  
аналогичным~(\ref{e14-ags}) и~фактически эквивалентным ему (см.\ 
примечание к~п.~6). 
  
  
  С точностью до 10$^{-4}$ для рынков \#0, \#X и~\#Y соответственно: 
  
  \vspace*{3pt}
  
 \noindent
  $\boldsymbol{g}_{\mathrm{0F}}$\;=\;\{0; 0; 0; 0; 0; 0; 0; 0; 0; 0; 0; 0; 0; 0,0006; 0,0007; 0; 
0; 0; 0; 0; 0; 0; 0; 0; 0; 0,0011; 0,0162; 0,0107; 0; 0; 0; 0; 0; 0; 0; 0; 0; 0,001; 0,0363; 0,6184; 
0,0454; 0; 0; 0; 0; 0; 0; 0; 0; 0; 0,0096; 0,5743; 0,7013; 0,1704; 0; 0; 0; 0; 0; 0; 0; 0; 0; 0,0411; 
0,6762; 0,7346; 0,5057; 0; 0; 0; 0; 0; 0; 0; 0; 0; 0,158; 0,8035; 0,8754; 0,5568; 0; 0; 0; 0; 0; 0; 0; 0; 
0; 0,6035; 0,9589; 0,9157; 0; 0; 0; 0; 0; 0; 0; 0; 0; 0; 0,6483; 1; 0,7662; 0; 0; 0; 0; 0; 0; 0; 0; 0; 0; 
0,529; 0,8348; 0,1819; 0; 0; 0; 0; 0; 0; 0; 0; 0; 0; 0; 0,1073; 0; 0; 0; 0; 0; 0; 0; 0; 0; 0; 0; 0; 0,0005; 
0; 0; 0; 0; 0; 0; 0; 0; 0; 0; 0; 0; 0\}; 

  \vspace*{3pt}
  
\noindent
  $\boldsymbol{g}_{\mathrm{XF}}$\;=\;\{0; 0; 0; 0,0008; 0,0083; 0,034; 0,1451; 0,4798; 
0,2442; 0,0257; 0,0034; 0,0001; 0\}; 
  
    \vspace*{3pt}
    
\noindent
  $\boldsymbol{g}_{\mathrm{YF}}$\;=\;\{0; 0; 0,0001; 0,0015; 0,0151; 0,099; 0,3114; 
0,3862; 0,0724; 0,006; 0,0002; 0\}. 

  \vspace*{3pt}
  
  В векторе $\boldsymbol{g}_{0\mathrm{F}}$ содержатся~35~подвергшихся 
обнулению элементов, относящихся к~множеству~$\mathbf{M}_0$. Наличие в~нем 
прочих нулевых элементов есть результат округления; они вносят свой вклад 
в~общий доход комбинированного портфеля, хотя и~с незначительным весом. 
  
\columnbreak
  
    { \begin{center}  %fig2
 \vspace*{-2pt}
    \mbox{%
 \epsfxsize=73.463mm 
 \epsfbox{aga-2.eps}
 }


\vspace*{6pt}

\noindent
{{\figurename~2}\ \ \small{
Доходы идеалистичного портфеля
}}
\end{center}}

 \vspace*{12pt} 
 
 
   Запись результатов для комбинированного портфеля: 
       $$
  \boldsymbol{L}^{\mathrm{cmb}}=\langle 0{,}335345; 0{,}351642; 0{,}048599\rangle\,, 
  $$
а график его платежной функции изображен на рис.~2 в~идеалистичной версии.
  

  С помощью матрицы~$\mathbf{A}$ находится и~вектор весовых 
коэффициентов для \textit{суррогатного} портфеля~(\ref{e18-ags}) и~его запись 
результатов: 

\vspace*{3pt}
  
  \noindent
  $g^{\mathrm{srg}}$\;=\;\{0; 0; 0,0001; 0,0011; 0,0108; 0,0726; 0,2446; 0,3317; 0,0455; 0,0035; 
0,0001; 0; 0; 0,0006; 0,0007; 0,0015; 0,0113; 0,0914; 0,247; 0,3342; 0,0724; 0,0036; 0,0001; 0; 0; 
0,0011; 0,0162; 0,0107; 0,0126; 0,076; 0,253; 0,3402; 0,0473; 0,0039; 0,0001; 0,0000, 0,0007; 
0,001; 0,0363; 0,6184; 0,0454; 0,082; 0,2632; 0,3505; 0,0503; 0,0045; 0,0008; 0,0008; 0,0071; 
0,0076; 0,0096; 0,5743; 0,7013; 0,1704; 0,2781; 0,3653; 0,0547; 0,007; 0,0082; 0,0083; 0,0259; 
0,0268; 0,0292; 0,0411; 0,6762; 0,7346; 0,5057; 0,3293; 0,0606; 0,0321; 0,0336; 0,034; 0,1095; 
0,1092; 0,1142; 0,1223; 0,158; 0,8035; 0,8754; 0,5568; 0,1328; 0,1401; 0,144; 0,1451; 0,3867; 
0,3902; 0,3983; 0,412; 0,4316; 0,6035; 0,9589; 0,9157; 0,4537; 0,4691; 0,4775; 0,4798; 0,2442; 
0,1839; 0,1886; 0,1966; 0,2081; 0,223; 0,6483; 1; 0,7662; 0,235; 0,2418; 0,2438; 0,022; 0,0167; 
0,0178; 0,0197; 0,025; 0,0892; 0,2935; 0,529; 0,8348; 0,1819; 0,0219; 0,0257; 0,0015; 0,0016; 
0,0018; 0,0022; 0,0141; 0,0964; 0,3114; 0,3788; 0,0668; 0,1073; 0,0031; 0,0034; 0; 0; 0,0001; 
0,0013; 0,0149; 0,099; 0,2991; 0,3862; 0,0704; 0,0056; 0,0005; 0,0001; 0; 0; 0,0001; 0,0013; 
0,0151; 0,0898; 0,3006; 0,3313; 0,0715; 0,006; 0,0002; 0\};

\vspace*{3pt}
  
  \noindent
  $\boldsymbol{L}^{\mathrm{srg}} = \langle 0{,}324923; 0{,}340849; 0{,}0490131\rangle$. 
  
  \vspace*{3pt}
  
  Здесь не приводится график доходов \textit{суррогатного} портфеля, так как 
он весьма похож на график рис.~2 (до степени смешения), хотя и~не тождествен 
ему. Расхождения объясняются тем, что в~суррогатном портфеле (в~отличие от 
идеалистичного) все двумерные сценарии рассматриваются по отдельности. 
И~даже при совпадении относительных доходов на нескольких сценариях (что 
часто происходит после замещений) алгоритм придает им разные веса, в~то 
время как в~комбинированном портфеле таким сценариям приписывается 
единый вес. Близость графиков служит подтверждением корректности 
расчетов. 
  
  В задаче~$B$ при $\chi_{\mathrm{X}}\hm= 1{,}02^{-1}
  \hm\approx 0{,}980392$ и~$\chi_{\mathrm{Y}}\hm = 1$:
  
  \vspace*{3pt}
  
 \noindent
  $\boldsymbol{\theta}_{\mathrm{X}}$\;=\;\{0; 0,0062; 0,0434; 0,1054; 0,1385; 0,3047; 
0,2934; 0,295; 0,3108; 0,4238; 0,4302; 0,3053; 0,4165\}. 

  \vspace*{3pt}
  
  В задаче~$C$ при $\chi_{\mathrm{X}}\hm = 1$ и~$\chi_{\mathrm{Y}}\hm = 1$:
  
    \vspace*{3pt}
  
  \noindent
  $\boldsymbol{\theta}_{\mathrm{X}}$\;=\;\{0; 0,0199; 0,0616; 0,1623; 0,3246; 0,4458; 
0,6904; 1; 0,6843; 0,4372; 0,2489; 0,0788; 0,0012\};

  \vspace*{3pt}
  
  \noindent
  $\boldsymbol{\theta}_{\mathrm{Y}}$\;=\;\{0,0273; 0,0715; 0,1536; 0,2337; 0,3867; 0,5895; 
0,8467; 0,8484; 0,5965; 0,3364; 0,2241; 0,1241\}. 
  
  \vspace*{3pt}
    
  Для задач~$B$ и~$C$ матрицы замещений соответственно равны: 
  $$
  \left\|
  \begin{array}{cccccccccccc}
0&0&0&0&0&0&0&0&0&0&0&0\\
0&0&0&0&0&0&0&0&0&0&0&1\\
1&0&0&0&0&0&0&0&0&0&1&1\\
1&0&0&0&0&0&0&0&0&1&1&1\\
1&1&0&0&0&0&0&0&0&1&1&1\\
1&1&1&0&0&0&0&0&1&1&1&1\\
1&1&1&1&0&0&0&0&0&1&1&1\\
1&1&1&1&0&0&0&0&0&1&1&1\\
1&1&1&1&1&0&0&0&0&0&1&1\\
1&1&1&1&1&1&0&0&0&0&1&1\\
1&1&1&1&1&1&0&0&0&0&1&1\\
1&1&1&1&1&1&0&0&0&0&0&1\\
1&1&1&1&1&1&1&0&0&0&0&1
\end{array}
\right\|; \ 
%$$
%$$
\left\|
\begin{array}{cccccccccccc}
2&2&2&2&2&2&2&2&2&2&2&2\\
1&2&2&2&2&2&2&2&2&2&2&1\\
1&1&2&2&2&2&2&2&2&2&2&1\\
1&1&1&2&2&2&2&2&2&2&1&1\\
1&1&1&1&2&2&2&2&2&1&1&1\\
1&1&1&1&1&2&2&2&2&1&1&1\\
1&1&1&1&1&1&2&2&1&1&1&1\\
1&1&1&1&1&1&1&1&1&1&1&1\\
1&1&1&1&1&1&2&2&1&1&1&1\\
1&1&1&1&1&2&2&2&2&1&1&1\\
1&1&1&1&2&2&2&2&2&2&1&1\\
1&1&2&2&2&2&2&2&2&2&2&1\\
1&2&2&2&2&2&2&2&2&2&2&2
\end{array}
\right\|.
$$
  
  Выбором в~задаче~$B$ параметра $\chi_{\mathrm{X}} (= 1{,}02^{-1})$ 
подчеркивается необязательность равенства его единице. К~тому же при 
получении графической картинки играют свою роль также соображения 
эстетического характера. 
  
  Записи результатов $\boldsymbol{L}^{\mathrm{cmb}}$ для задач~$B$ и~$C$ 
соответственно: 
 \begin{gather*}
  \langle 0{,}329105; 0{,}343464; 0{,}0436319\rangle\,;\\
  \langle 
0{,}374383; 0{,}38801; 0{,}0363967\rangle\,. 
  \end{gather*}
  
  Графики платежных функций комбинированных портфелей для них 
изображены на рис.~3 в~идеалистичной версии.

  \vspace*{-6pt}
  
  \section{Заключение}
  
  \vspace*{-2pt}
  
  Работа завершает намеченное исследование\linebreak применимости CC-VaR 
к~совокупности трех рынков разной размерности. Для теоретических 
и~сценарных рынков предложен подход, связанный\linebreak\vspace*{-12pt}




{ \begin{center}  %fig3
 \vspace*{-1pt}
    \mbox{%
 \epsfxsize=74.293mm 
 \epsfbox{aga-3.eps}
 }

\end{center}

\noindent
{{\figurename~3}\ \ \small{
Доходы идеалистичных портфелей в~задачах~$B$~(\textit{а}) и~$C$~(\textit{б})
}}
}

\vspace*{12pt}


\noindent
 с~замещением менее 
доходных базисных инструментов двумерного рынка более доходными 
ба\-зисными инструментами соответствующих одномерных рынков. Для 
реализуемости замещения применяется методология рандомизации. 
Проведенные для сценарных рынков расчеты и~построенные графики 
свидетельствуют об эффективности модели.

%\vspace*{-6pt}

\vspace*{-6pt} 
  
{\small\frenchspacing
 {%\baselineskip=10.8pt
 \addcontentsline{toc}{section}{References}
 \begin{thebibliography}{9}
  \bibitem{1-ags}
  \Au{Agasandian G.\,A.} Optimal behavior of an investor in option market~// 
  Joint 
Conference (International) on Neural Networks Proceedings.~--- Honolulu, HI,
USA:  IEEE,  2002. P.~1859--1864. 
  \bibitem{2-ags}
  \Au{Агасандян Г.\,А.} Применение континуального критерия VaR на финансовых 
рынках.~--- М.: ВЦ РАН, 2011. 299~с. 
  \bibitem{3-ags}
  \Au{Агасандян Г.\,А.} Континуальный критерий VaR на многомерных рынках  
опционов.~--- М.: ВЦ РАН, 2015. 297~с. 
  \bibitem{4-ags}
  \Au{Агасандян Г.\,А.} Континуальный критерий VaR на сценарных рынках~// 
Информатика и~её применения, 2018. Т.~12. Вып.~1. С.~32--40.

 
  \bibitem{5-ags}
  \Au{Агасандян Г.\,А.} Континуальный критерий VaR и~оптимальный портфель 
инвестора~// Управ\-ле\-ние большими сис\-те\-ма\-ми, 2018. Вып.~73. С.~6--26.

\columnbreak

  \bibitem{6-ags}
  \Au{Агасандян Г.\,А.} Теоретические основы оптимизации по CC-VaR на совокупности 
рынков~// Информатика и~её применения, 2019. Т.~13. Вып.~4. С.~36--41.
  \bibitem{7-ags}
  \Au{Крамер Г.} Математические методы статистики~/ Пер. с~англ.~--- М.: Мир, 1975. 
750~с. (\Au{Cramer~H.} Mathematical methods of statistics.~--- Princeton, NJ, USA: Princeton 
University Press, 1946. 575~p.)
 \end{thebibliography}

 }
 }

\end{multicols}

\vspace*{-9pt}

\hfill{\small\textit{Поступила в~редакцию 21.11.19}}

\vspace*{8pt}

%\pagebreak

%\newpage

%\vspace*{-28pt}

\hrule

\vspace*{2pt}

\hrule

\vspace*{-4pt}

\def\tit{COMPUTATIONAL ASPECTS OF~OPTIMIZATION ON~CC-VaR 
IN~A~COMPLEX OF MARKETS}


\def\titkol{Computational aspects of~optimization on CC-VaR 
in~a~complex of markets}

\def\aut{G.\,A.~Agasandyan}

\def\autkol{G.\,A.~Agasandyan}

\titel{\tit}{\aut}{\autkol}{\titkol}

\vspace*{-15pt}


\noindent
A.\,A.~Dorodnicyn Computing Center, Federal Research Center ``Computer Science 
and Control'' of the Russian Academy of Sciences, 40~Vavilov Str., Moscow 119333, 
Russian Federation


\def\leftfootline{\small{\textbf{\thepage}
\hfill INFORMATIKA I EE PRIMENENIYA~--- INFORMATICS AND
APPLICATIONS\ \ \ 2020\ \ \ volume~14\ \ \ issue\ 3}
}%
 \def\rightfootline{\small{INFORMATIKA I EE PRIMENENIYA~---
INFORMATICS AND APPLICATIONS\ \ \ 2020\ \ \ volume~14\ \ \ issue\ 3
\hfill \textbf{\thepage}}}

\vspace*{3pt}

\Abste{The work is the direct continuation of the previous author's 
investigation on using continuous VaR-criterion (CC-VaR) in a~set 
of markets of different dimensions, which are mutually connected by 
their underliers. The exposition is aimed at the application of ideas 
and methods developed for the theoretical continuous model to discrete 
scenarios markets. In a~typical model case of a~collection of one 
two-dimensional market and two one-dimensional markets, a~rule of 
constructing a~combined portfolio in these markets is submitted. 
This rule gives a~necessary and sufficient condition of portfolio 
optimality in the weighted composition of basis instruments. The 
condition is founded on misbalance in returns relative between 
markets with maintaining optimality on CC-VaR. The optimal combined 
portfolio with three components is constructed. Also, the idealistic 
and surrogate versions of this combined portfolio, which are useful 
in testing all algorithmic calculations and in graphic illustrating 
portfolio's payoff functions, are adduced. The model can be extended 
without difficulties, theoretic anyway, on markets of greater dimensions.}

\KWE{underlie; risk preferences function; continuous VaR-criterion; 
cost and forecast densities; return relative function; Newman--Pearson 
procedure; combined portfolio; surrogate portfolio}
 
\DOI{10.14357/19922264200309} 

\vspace*{-20pt}

 \Ack
 \noindent
  The reported study was funded by RFBR, project 
No.\,17-01-00816.

%\vspace*{6pt}

 \begin{multicols}{2}

\renewcommand{\bibname}{\protect\rmfamily References}
%\renewcommand{\bibname}{\large\protect\rm References}

{\small\frenchspacing
 {%\baselineskip=10.8pt
 \addcontentsline{toc}{section}{References}
 \begin{thebibliography}{9}
  \bibitem{1-ags-1}
  \Aue{Agasandian, G.\,A.} 2002. Optimal behavior of an investor in option market. 
\textit{Joint Conference (International) on Neural Networks Proceedings}. 
Honolulu, HI: IEEE. 
1859--1864. 
  \bibitem{2-ags-1}
  \Aue{Agasandyan, G.\,A.} 2011. \textit{Primenenie kontinual'nogo kriteriya VaR 
na finansovykh rynkakh} [Application of continuous VaR-criterion in financial 
markets]. Moscow: CC RAS. 299~p. 
  \bibitem{3-ags-1}
  \Aue{Agasandyan, G.\,A.} 2015. \textit{Kontinual'nyy kriteriy VaR na 
mnogomernykh rynkakh optsionov} [Continuous VaR-criterion in multidimensional 
option markets]. Moscow: CC RAS. 297~p. 
  \bibitem{4-ags-1}
  \Aue{Agasandyan, G.\,A.} 2018. Kontinual'nyy kriteriy VaR na stsenarnykh 
rynkakh [Continuous VaR-criterion in scenario markets]. \textit{Informatika i~ee 
Primeneniya~--- Inform. Appl.} 12(1):32--40. 
  \bibitem{5-ags-1}
  \Aue{Agasandyan, G.\,A.} 2018. Kontinual'nyy kriteriy VaR i~optimal'nyy portfel' 
investora [Continuous VaR-criterion and investor's optimal portfolio]. 
\textit{Upravleniye bol'shimi sistemami} [Large-Scale Systems Control] 73:6--26.
  \bibitem{6-ags-1}
  \Aue{Agasandyan, G.\,A.} 2018. Teoreticheskie osnovy opti\-mi\-za\-tsii po 
kontinual'nomu kriteriyu VaR na sovokupnosti rynkov [Theoretical foundations of 
continuous VaR optimization in the collection of markets]. \textit{Informatika i~ee 
Primeneniya~--- Inform. Appl.} 13(4):36--41. 
  \bibitem{7-ags-1}
  \Aue{Cramer, H.} 1946. \textit{Mathematical methods of statistics}. Princeton, NJ: 
Princeton University Press. 575~p.
\end{thebibliography}

 }
 }

\end{multicols}

\vspace*{-9pt}

\hfill{\small\textit{Received October 21, 2019}}

%\pagebreak

\vspace*{-20pt}
  
  \Contrl
  
  \vspace*{-4pt}
  
  \noindent
  \textbf{Agasandyan Gennady A.} (b.\ 1941)~--- Doctor of Science in physics and 
mathematics, leading scientist, A.\,A.~Dorodnicyn Computing Center, Federal 
Research Center ``Computer Science and Control'' of the Russian Academy of 
Sciences, 40~Vavilov Str., Moscow 119333, Russian Federation; 
\mbox{agasand17@yandex.ru}
  

   
\label{end\stat}

\renewcommand{\bibname}{\protect\rm Литература} 
         %9
\def\stat{grusho}

\def\tit{АРХИТЕКТУРНЫЕ РЕШЕНИЯ В~ЗАДАЧЕ ВЫЯВЛЕНИЯ МОШЕННИЧЕСТВА ПРИ~АНАЛИЗЕ 
ИНФОРМАЦИОННЫХ ПОТОКОВ В~ЦИФРОВОЙ ЭКОНОМИКЕ$^*$}

\def\titkol{Архитектурные решения в~задаче выявления мошенничества при~анализе 
информационных потоков в
%~цифровой 
экономике}

\def\aut{А.\,А.~Грушо$^1$, М.\,И.~Забежайло$^2$, Н.\,А.~Грушо$^3$, 
Е.\,Е.~Тимонина$^4$}

\def\autkol{А.\,А.~Грушо, М.\,И.~Забежайло, Н.\,А.~Грушо, 
Е.\,Е.~Тимонина}

\titel{\tit}{\aut}{\autkol}{\titkol}

\index{Грушо А.\,А.}
\index{Забежайло М.\,И.}
\index{Грушо Н.\,А.}
\index{Тимонина Е.\,Е.}
\index{Grusho A.\,A.}
\index{Zabezhailo M.\,I.}
\index{Grusho N.\,A.}
\index{Timonina E.\,E.}


{\renewcommand{\thefootnote}{\fnsymbol{footnote}} \footnotetext[1]
{Работа частично поддержана РФФИ (проекты 18-29-03081 и~18-07-00274).}}


\renewcommand{\thefootnote}{\arabic{footnote}}
\footnotetext[1]{Институт проблем информатики Федерального исследовательского центра <<Информатика и~управление>> 
Российской академии наук, grusho@yandex.ru}
\footnotetext[2]{Институт проблем информатики Федерального исследовательского центра <<Информатика и~управление>> 
Российской академии наук, m.zabezhailo@yandex.ru}
\footnotetext[3]{Институт проблем информатики Федерального исследовательского центра <<Информатика и~управление>> 
Российской академии наук, info@itake.ru}
\footnotetext[4]{Институт проблем информатики Федерального исследовательского центра <<Информатика и~управление>> 
Российской академии наук, eltimon@yandex.ru}

\vspace*{-12pt}
   

 
  
  \Abst{Cформулирован подход к~исследованию некоторых видов мошенничества в~цифровой 
экономике с~использованием причинно-следственных связей. Во всех видах рассматриваемых 
мошенничеств должно наблюдаться несоответствие между целями финансовых транзакций 
и~реальной стоимостью достижения этих целей. Данные о транзакциях можно собирать, 
наблюдая информационные потоки, в~которых отражаются эти транзакции. Архитектура сбора 
данных и~их анализа может быть организована с~помощью распределенных реестров 
с~централизованным консенсусом, что позволяет создать аналог электронной бухгалтерской 
книги, фиксирующей финансово-экономическую деятельность субъектов цифровой экономики в~регионе. 
  Рассматриваемые методы выявления мошенничества основаны на противоречиях 
между действиями, описанными в~транзакциях, и~информацией, содержащейся в~планах, 
стандартах, прецедентах и~др. Рассмотрен метод, основанный на некоторой упрощенной схеме 
реализации абстрактного проекта. Для выявления противоречий необходимо проводить анализ 
от следствия к~причине, т.\,е.\ искать аномалии в~информации, описывающей порождение 
наблюдаемых следствий. 
  Показано, как в~реализации проекта можно выделять простые <<необходимые условия>> 
нарушения при\-чин\-но-след\-ст\-вен\-ных связей, т.\,е.\ множество <<необходимых условий>>, 
нарушение которых свидетельствует о наличии мошенничества. Это множество <<необходимых 
условий>> можно назвать метаданными для контроля проекта на выявление мошенничества.} 
 
 
  \KW{цифровая экономика; информационные потоки; при\-чин\-но-след\-ст\-вен\-ные связи; 
выявление мошеннических схем} 

\DOI{10.14357/19922264190204}
  
\vspace*{-4pt}


\vskip 10pt plus 9pt minus 6pt

\thispagestyle{headings}

\begin{multicols}{2}

\label{st\stat}

\section{Введение}

\vspace*{3pt}

  В работе сформулирован подход к~исследованию некоторых видов 
мошенничества в~цифровой экономике с~использованием  
при\-чин\-но-след\-ст\-вен\-ных связей. Рассматриваются три вида мошенничества, 
а именно:
  \begin{enumerate}[(1)]
\item отмыв денег; 
\item обман при выполнении договорных обязательств при реализации 
технических проектов (строительные проекты и~др.); 
\item незаконный вывод денег. 
\end{enumerate}

  Названные виды мошенничества могут быть сведены к~решению одного типа 
задач. Для отмывания денег источник должен заключать фиктивные контракты, 
в~соответствии с~которыми будут переводиться средства за заведомо ненужную 
работу и~материалы. 
  
  Мошенничество, связанное с~невыполнением договорных обязательств, связано 
со снижением качества услуг, качества и~количества закупаемых 
материалов, выполнением работ с~ненадлежащим качеством. 
  
  Вывод денег связан с~переводом средств фир\-мам-од\-но\-днев\-кам, которые 
заведомо не могут выполнить обязательства по контрактам, за которые им 
переводятся средства. 
  
  Таким образом, во всех трех видах рассматриваемых мошенничеств должно 
наблюдаться несоответствие между целями финансовых транзакций и~реальной 
стоимостью достижения этих целей. Данные о транзакциях можно собирать, 
наблюдая информационные потоки, в~которых отражаются эти транзакции. 
  
  Однако для наблюдения таких информационных потоков необходимо создавать 
архитектуру\linebreak телекоммуникационной системы, позволяющей перехватывать 
и~собирать данные о всех транзакциях. Например, такая архитектура может быть 
организована с~помощью распределенных реестров с~централизованным 
консенсусом, т.\,е.\ все информационные потоки, сформированные в~цифровой 
экономике и~несущие информацию о транзакциях, проходят через некоторый 
центральный узел, запоминающий их в~форме распределенного реестра. Такие 
реестры могут дублироваться в~аналогичных центрах различных регионов, что 
позволяет создать аналог электронной бухгалтерской книги, фиксирующей 
фи\-нан\-со\-во-эко\-но\-ми\-че\-скую деятельность субъектов цифровой экономики. Такой 
подход предложено реализовать на базе системы ситуационных центров, что 
отражено в~работах~[1, 2].
  
  Собранная из информационных потоков информация о~транзакциях, т.\,е.\ 
о~контрактах, договорах, платежах, отчетах, закупленных материалах, 
характеристиках исполнителей работ и~др., собирается в~базе данных в~указанном 
центре. Согласно теории интеллектуальных сис\-тем~[3], эту базу данных можно 
называть базой фактов (БФ). Базу фактов можно представить как бинарную мат\-ри\-цу, 
строки которой описывают характеристики, входящие в~транзакции, а столбцы 
нумеруются характеристиками. Строки матрицы будем называть 
\textit{объектами}~[4, 5]. 
  
  Рассматриваемые в~работе методы выявления мошенничества будут основаны 
на противоречиях между действиями, описанными в~транзакциях, и~информацией, 
содержащейся в~планах, стандартах, прецедентах и~др. Для нахождения 
противоречий в~архитектуре центра предусмотрена другая база данных~--- база 
знаний (БЗ)~\cite{3-gr, 6-gr}, которая устроена так же, как БФ. 
  
  Информация в~БЗ собирается на основе положительного опыта или расчетов. 
Используя БЗ, можно выводить факты нарушения при\-чин\-но-след\-ст\-вен\-ных 
связей. Нарушения при\-чин\-но-след\-ст\-вен\-ных связей будем называть 
\textit{аномалиями}. 
  
  Для упрощения дальнейшее изложение будет вестись в~рамках поиска 
противоречий при выполнении некоторого абстрактного проекта. Выявление 
аномалий будет происходить на основе фактов из БФ с~помощью знаний из БЗ 
методами искусственного интеллекта и~интеллектуального анализа 
данных~\cite{6-gr}. 

\vspace*{-10pt}
  
  \section{Модели}
  
  \vspace*{-3pt}
  
  Наиболее сложная из рассмотренных выше задач~--- выявление противоречий, 
т.\,е.\ использование БЗ для получения новых знаний и~выявление аномалий из 
полученных фактов. 
  
  Все способы выявления противоречий основаны на определении 
  причинно-следственных связей. При этом противоречия в~параметрах транзакций по 
отношению к~требуемым в~БЗ составляют сущность аномалий. 
  
   Далее будет рассмотрен метод, основанный на некоторой упрощенной схеме 
реализации абстрактного проекта. 
  
  Каждый проект имеет цель: например, цель представляет собой построение 
некоторой системы. Воспользуемся структурным подходом, который позволяет 
строить проект на основе разбиения системы на подсистемы и~определения 
взаимодействий подсистем~\cite{7-gr}. При этом каждая подсистема также 
представима структурной моделью. 
  
  Как сама система, так и~каждая ее подсистема имеют свой функционал 
и~спецификацию, па\-ра\-мет\-ры настройки и~домены параметров настройки. Кроме 
этих характеристик существует множество характеристик, связанных 
с~<<жизненным циклом>> создания системы. Сюда входят работы, ресурсы, 
сроки выполнения работ по созданию подсистем и~самой системы, стоимости 
компонентов и~материалов, стоимости работ, схемы поставок, договорные 
обязательства и~др. Все характеристики связаны между собой, поэтому можно 
говорить о стоимости и~времени изготовления структурных компонентов системы. 
  
  Одной из важнейших характеристик является смета (система смет для 
подсистем). Смета сопоставляет каждому компоненту системы стоимость его 
изготовления и~настройки. 
  
  Схема построения системы может быть пред\-став\-ле\-на диаграммой, 
изображенной на рис.~1. 

{ \begin{center}  %fig1
 \vspace*{9pt}
   \mbox{%
 \epsfxsize=79mm 
 \epsfbox{gru-1.eps}
 }


\vspace*{9pt}


\noindent
{{\figurename~1}\ \ \small{Диаграмма достижения цели}}
\end{center}
}

\vspace*{9pt}

\addtocounter{figure}{1}
  
  


  Представленная на рис.~1 диаграмма позволяет описать основные классы 
возможных противоречий при достижении цели. Противоречия возникают, когда 
данные БФ не соответствуют требуемым характеристикам. 
  
  
  \section{Потенциальные классы аномалий при~достижении цели}
  
  Выделим четыре потенциальных класса противоречий, которые показывают, 
каким образом нужно искать эти противоречия.
  
 
  Противоречие цели и~проекта (рис.~2) возникает при отсутствии обоснования 
или в~случае логического противоречия между возможностями проектируемого 
функционала и~целью системы. Отметим, что в~проект входят сроки, перечень 
работ, материалы, настройки, которые описываются соответствующими 
параметрами и~допустимыми значениями этих параметров. Проект формируется 
на основе БЗ и~расчетов, исходя из информации, полученной по аналогии 
с~другими проектами и~решениями, которые считаются апробированными. 
  
  Отметим, что цель порождает проект и~в этом смысле является причиной 
проекта. Однако для анализа противоречий необходимо двигаться по штриховой 
стрелке диаграммы (см.\ рис.~2) от проекта к~цели. В~самом деле, любой компонент 
проекта направлен на теоретическое достижение цели. Цель~--- сложный объект, 
поэтому в~проекте могут возникнуть характеристики, противоречащие хотя бы 
некоторым характеристикам цели. Это делает проект противоречивым, но вывод 
об этом может быть сделан только на уровне описания цели. 
  

  Противоречия между проектом и~его реализацией, исключая настройки 
(рис.~3), могут возникать, например, при закупке исполнителем материалов более 
низкого качества по более низким ценам, при попытках достижения требуемых 
сроков работы за счет снижения качества выполнения работ, за счет нахождения 
<<объективных>> причин для увеличения сроков работы и,~следовательно, 
увеличения цены реализации проекта. 


  Для выявления указанных противоречий необходимо двигаться по диаграмме 
(см.\ рис.~3) в~обратную сторону в~соответствии со~штриховыми стрелками. 
Действительно, выявить противоречия между характеристиками закупленных 
материалов и~требуемыми по проекту можно только при обращении к~проекту 
и~его спецификациям. Манипуляции со сроками работы также можно выявить 
только при обращении к~соответствующим расчетам в~проекте. Задержки в~сроках 
работы, связанные с~поставками материалов, можно определить только на 
предыдущем этапе диаграммы (см.\ рис.~3) в~описании проекта. 


  


  Противоречия между реализацией проекта и~его настройкой (рис.~4) возникает, 
когда не удается добиться требуемых значений параметров функционала, не 
удается обеспечить необходимый уровень\linebreak\vspace*{-12pt}

{ \begin{center}  %fig2
 \vspace*{-6pt}
   \mbox{%
 \epsfxsize=16mm 
 \epsfbox{gru-2.eps}
 }


\vspace*{6pt}


\noindent
{{\figurename~2}\ \ \small{Противоречия цели и~проекта}}
\end{center}
}

%\vspace*{9pt}

\addtocounter{figure}{1}

{ \begin{center}  %fig3
 \vspace*{6pt}
    \mbox{%
 \epsfxsize=79mm 
 \epsfbox{gru-3.eps}
 }


\end{center}

\vspace*{-2pt}


\noindent
{{\figurename~3}\ \ \small{Противоречия проекта и~его реализации (без настройки)}}
}

\vspace*{6pt}

\addtocounter{figure}{1}

{ \begin{center}  %fig4
 \vspace*{1pt}
   \mbox{%
 \epsfxsize=54.5mm 
 \epsfbox{gru-4.eps}
 }


\end{center}


\noindent
{{\figurename~4}\ \ \small{Противоречия реализации проекта и~его на\-стройки}}
}

%\vspace*{9pt}

\addtocounter{figure}{1}

{ \begin{center}  %fig5
 \vspace*{5pt}
    \mbox{%
 \epsfxsize=79mm 
 \epsfbox{gru-5.eps}
 }


\end{center}



\noindent
{{\figurename~5}\ \ \small{Противоречия цели и~достигнутой реализации проекта}}
}

\vspace*{6pt}

\addtocounter{figure}{1}

\noindent
 качества реализации проекта. Для 
определения противоречия в~настройках надо опять же двигаться по диаграмме 
(см.\ рис.~4) в~обратную сторону по штриховым стрелкам, так как для выявления 
характеристик результатов работы, которые не дают возможности реализации 
определенного функционала, необходимо иметь информацию о результатах этой 
работы. 


  



  Противоречие между целью и~достигнутой реализацией проекта (рис.~5) 
возникает, когда реализованная система не позволяет достичь цели. В~этом случае 
опять противоречие нужно искать, двигаясь от цели к~реальному достигнутому 
функционалу по штриховой стрелке (см.\ рис.~5).
  
  Суммируя положения, изложенные в~данном разделе, приходим к~выводу, что 
для выявления противоречий необходимо проводить анализ от следствия 
к~причине, т.\,е.\ искать аномалии в~информации, описывающей порождение 
наблюдаемых следствий. 
  
  
  \section{Связь противоречий и~причин}
  
  Прежде чем построить связь между причинами и~противоречиями, кратко 
опишем простейшую модель связи этих понятий. Причины и~противоречия будут 
сформулированы для представления компонентов системы как объектов, 
обладающих наборами известных характеристик~\cite{4-gr, 5-gr}. 
  
  Пусть $U\hm=\{\alpha, \beta, \ldots\}$~--- совокупность характеристик 
(пространство характеристик). Согласно~\cite{4-gr} \textit{объектом}~$O$ 
называется любое подмножество характеристик $O\hm\subseteq U$. Рассмотрим 
последовательность объектов, возможно в~различных пространствах 
характеристик. 
  
  \smallskip
  
  \noindent
  \textbf{Определение~1.}\ Объект~$P$ с~числом характеристик, большим или 
равным~2, является \textit{причиной} объекта (\textit{свойства})~$B$ в~цепочке 
наблюдаемых объектов тогда и~только тогда, когда выполнены следующие 
условия:
  \begin{enumerate}[(1)]
\item для каждого объекта~$C$, если $P\hm\subseteq C$, то $C\mapsto B$, где 
$C\mapsto B$ означает, что объект~$B$ присутствует в~объекте, следующем за 
объектом~$C$;
\item объект~$P$ является минимальным объектом, удовлетворяющим 
условию~1, а~именно: $\forall \alpha\hm\in P$ объект~$P\backslash \{\alpha\}$ 
не является причиной, т.\,е.\ $\exists C:\ \alpha\not\in C$, $P\backslash 
\{\alpha\}\hm\subseteq C$ и~$C\not\mapsto B$, где $C\not\mapsto B$ означает, 
что~$B$ не может содержаться в~объекте, следующем за объектом~$C$. 
\end{enumerate}

  Приведенное определение причины является упрощением причин, 
возникающих в~реальном мире. Например, реальные причины могут возникать\linebreak 
как совокупность характеристик из разных пространств. Одно следствие может 
порождаться разными причинами или возникать из внешних\linebreak и~ненаблюдаемых 
характеристик. Однако пред\-став\-лен\-ная далее формализация позволяет доступно 
изложить при\-чин\-но-след\-ст\-вен\-ные истоки противоречий, которые 
инициируют в~дальнейшем глубокое исследование рассматриваемых процессов.
  
  Будем считать, что для любого интересующего нас свойства~$B$ существует 
причина. Тогда справедлива следующая теорема.
  
  \smallskip
  
  \noindent
  \textbf{Теорема~1.}\ \textit{Для любого свойства~$B$ существует 
единственная причина}. 
  
  \smallskip
  
  \noindent
  Д\,о\,к\,а\,з\,а\,т\,е\,л\,ь\,с\,т\,в\,о\,.\ \ Доказательство будем вести от противного, 
т.\,е.\ предположим, что существуют две причины свойства~$B$: $P$ 
и~$P^\prime$, $P\hm\not= P^\prime$. Тогда существует $\alpha\hm\in U$, которое 
удовлетворяет одному из двух условий:
  \begin{itemize}
\item[(а)] $\alpha\in P$, $\alpha\notin P^\prime$;
\item[(б)] $\alpha\notin P$, $\alpha \in P^\prime$.
\end{itemize}

  Пусть выполняется условие~(б). Тогда $P^\prime\backslash \{\alpha\}$ не 
является причиной по условию~2 определения~1, т.\,е.\ $\exists C$ такое, что 
$\alpha\notin C$, $P^\prime\backslash \{\alpha\}\hm\subseteq C$ и~$C\not\mapsto B$. 
Но если~$B$ произошло и~$P$ его причина, то $C\mapsto B$, что противоречит 
предположению. Теорема~1 доказана.
  
  \smallskip
  
  \noindent
  \textbf{Лемма.} \textit{Если $P$~--- причина появления свойства~$B$, то 
объект~$B$ определяет существование свойства~$P$ в~объекте, 
предшествующем~$B$. }
  
  \smallskip
  
  \noindent
  Д\,о\,к\,а\,з\,а\,т\,е\,л\,ь\,с\,т\,в\,о\,.\ \ Из предположения, что у~каж\-до\-го 
свойства~$B$ есть причина, и~условия, что~$P$ является причиной~$B$, следует, 
что при появлении в~данных свойства~$B$ объект~$C$, предшествующий 
появлению~$B$, содержит как часть объект~$P$. Это следует из теоремы~1 
и~определения причины. 
  
  Докажем принцип <<необходимого условия>>, который, несмотря на простоту 
доказательства, будет играть в~дальнейшем существенную роль.
  
  \smallskip
  
  \noindent
  \textbf{Теорема~2.} \textit{Если~$P$~--- причина появления свойства~$B$ 
и~$A\hm\subseteq P$, то объект~$B$ определяет наличие свойства~$A$ 
в~объекте, предшествующем~$B$}. 
  
  \smallskip
  
  \noindent
  Д\,о\,к\,а\,з\,а\,т\,е\,л\,ь\,с\,т\,в\,о\,.\ \ Пусть в~данных имеется объект~$B$ 
и~$P\mapsto B$, тогда в~силу существования и~единственности причины~$B$ 
в~данных должен существовать объект~$C$, предшествующий~$B$ 
и~содержащий причину~$P$. Поскольку $A\hm\subseteq P$ и~$B$ содержит 
причину~$P$, то $B\mapsto A$. С~учетом леммы теорема~2 доказана.
  
  \smallskip
  
  Пусть даны пространства $U_1, U_2,\ldots$ и~имеется последовательность 
данных (процесс выполнения этапов проекта в~соответствии с~рис.~1) $A, B, 
\ldots$, где каждый объект является подмножеством некоторого 
пространства~$U_i$, $i\hm=1,\ldots$ Тогда в~объекте~$A$ присутствует 
причина~$P$ появления интересующего нас свойства~$C$ в~объекте~$B$. Пусть 
$P\hm\subseteq A$, тогда по теореме~2 $\forall \alpha\hm\in P$:  
$C\mapsto \{\alpha\}$, т.\,е.\ из появления~$C$ следует появление 
характеристики~$\alpha$ в~предшествующем объекте. Это необходимое условие 
того, что~$C$ удовлетворяет причинно-следственным связям развития процесса 
выполнения проекта. Если для~$C$ нет характеристики~$\alpha$, которую можно 
отнести к~причине~$C$, то можно считать, что нарушена  
при\-чин\-но-след\-ст\-вен\-ная связь и~$C$~--- аномальный объект. 
  
  \smallskip
  
  \noindent
  \textbf{Пример.} Если объект~$C$ состоит в~получении суммы~$a$ 
фирмой~$K$, то согласно теореме~2 в~пред\-шест\-ву\-ющем объекте должна 
существовать причина перевода суммы~$a$ на фирму~$K$. Если эта причина 
в~проекте отсутствует, то это можно считать признаком мошеннической схемы. 
Все проекты по предположению собираются из <<кубиков>>, содержащихся в~БЗ. 
Тогда можно сравнить цену объекта~$C$, породившего получение суммы~$a$, 
и~сумму, присутствующую в~смете проекта. Если разница велика, то это либо 
ошибка проекта, либо признак мошеннической схемы.
  
  \section{Поиск противоречий на~основе~принципа <<необходимых~условий>>}
   
  Как было показано в~разд.~3, нахождение противоречий соответствуют 
движению от следствия к~причине. Для каждого объекта в~наблюдаемых данных 
выявление причин его появления является трудоемкой задачей. Кроме того, при 
реализации контроля соблюдения при\-чин\-но-след\-ст\-вен\-ных связей на 
большом множестве участников экономической деятельности задача анализа 
причин становится трудоемкой. Поэтому процедуру контроля необходимо разбить 
на два этапа, где первый этап состоит в~анализе простых <<необходимых 
условий>> проявления мошенничества, когда используется хотя бы одна 
известная характеристика причины. Второй этап (в~режиме офлайн) состоит 
в~выявлении причин, позволяющих провести анализ источников мошеннических 
схем. 
  
  Один из подходов к~выбору <<необходимых условий>> состоит в~построении 
множества подцелей исходной цели проекта (структурный метод построения 
проекта~\cite{7-gr}). Каждая подцель описывается диаграммой на рис.~1, 
и~реализации подцелей должны образовывать полный функционал цели. Это 
является необходимым, но не достаточным условием достижения цели, так как 
при таком подходе отсутствует компонент согласования всех подцелей в~единую 
систему. Однако такой подход значительно упрощает анализ выполнения проекта 
на предмет поиска мошенничества. Если признаки мошенничества будут 
обнаружены в~реализации хотя бы одной из подцелей, то это значит, что 
мошенничество присутствует в~реализации всего проекта. 
  
  Аналогично в~реализации каждого этапа в~любой из подцелей можно выделять 
простые <<необходимые условия>> нарушения при\-чин\-но-след\-ст\-венн\-ых 
связей. 
  
  Таким образом, получается множество <<необходимых условий>>, нарушение 
которых свидетельствует о наличии мошенничества. Это множество 
<<необходимых условий>> можно назвать метаданными~[8, 9] для контроля 
проекта на выявление мошенничества. 
  
  
  \section{Заключение }
  
  В поиске противоречий необходимо от транзакций, соответствующих 
следствиям при\-чин\-но-след\-ст\-вен\-ных связей, переходить к~анализу причин 
наблюдаемых следствий. Это сложная задача, которая связана с~описанием причин 
определенных свойств. 
  
  В работе представлена модель, позволяющая строить множество необходимых 
условий соответствия наблюдаемого следствия вызвавшей его причине. Этот 
подход делает поиск противоречий вполне вычислимой задачей, но не гарантирует 
успех. 
  
  {\small\frenchspacing
 {%\baselineskip=10.8pt
 \addcontentsline{toc}{section}{References}
 \begin{thebibliography}{9}
\bibitem{1-gr}
\Au{Грушо А.\,А., Зацаринный~А.\,А., Тимонина~Е.\,Е.} Блокчейны цифровой экономики на базе 
системы ситуационных центров и~централизованного консенсуса~// Радиолокация, навигация, 
связь: Мат-лы XXV Междунар. научн.-технич. конф.~---
Воронеж: Издательский дом ВГУ, 2019. Т.~6. С.~183--191. 
\bibitem{2-gr}
\Au{Grusho A., Zatsarinny~A., Timonina~E.} A~system approach to information security in 
distributed ledgers on the situational centers platform.~---
Lecture notes in computer science ser.~--- Springer, 2019 
(in press).
\bibitem{3-gr}
\Au{Финн В.\,К.} Искусственный интеллект: Методология, применения, философия.~--- М.: 
Красанд, 2011. 448~с.

\bibitem{5-gr} %4
\Au{Аншаков~О.\,М., Фабрикантова~Е.\,Ф.} ДСМ-ме\-тод автоматического порождения 
гипотез: Логические и~эпистемологические основания.~--- М.: Либроком, 2009. 432~с.

\bibitem{4-gr} %5
\Au{Poelmans J., Elzinga~P., Viaene~S., Dedene~G.} Formal concept analysis in knowledge 
discovery: A~survey~// Conceptual structures: From information to intelligence~/ Eds.\ M.~Croitoru, 
S.~Ferr$\acute{\mbox{e}}$, and D.~Lukose.~--- Lecture notes in computer science 
ser.~--- Berlin--Heidelberg: Springer, 2010. Vol.~6208.  P.~139--153.

\bibitem{6-gr}
\Au{Панкратова~Е.\,С., Финн~В.\,К.} Автоматическое по\-рож\-де\-ние гипотез в~интеллектуальных 
системах.~--- М.: Либроком, 2009. 528~с. 
\bibitem{7-gr}
\Au{Денисов А.\,А., Колесников~Д.\,Н.} Теория больших систем управления.~--- Л.: Энергоиздат, 1982. 488~с.

\bibitem{9-gr}
\Au{Грушо А.\,А., Грушо Н.\,А., Забежайло~М.\,И., Смирнов~Д.\,В., Тимонина~Е.\,Е.} 
Параметризация в~прикладных задачах поиска эмпирических причин~// Информатика и~её 
применения, 2018. Т.~12. Вып.~3. С.~62--66.

\bibitem{8-gr}
\Au{Грушо А.\,А., Грушо Н.\,А., Левыкин~М.\,В., Тимонина~Е.\,Е.} Методы идентификации 
захвата хоста в~распределенной ин\-фор\-ма\-ци\-он\-но-вы\-чис\-ли\-тель\-ной сис\-те\-ме, 
защищенной с~помощью метаданных~// Информатика и~её применения, 2018. Т.~12. Вып.~4. 
С.~41--45.

 \end{thebibliography}

 }
 }

\end{multicols}

\vspace*{-3pt}

\hfill{\small\textit{Поступила в~редакцию 03.04.19}}

%\vspace*{8pt}

%\pagebreak

\newpage

\vspace*{-28pt}

%\hrule

%\vspace*{2pt}

%\hrule

%\vspace*{-2pt}

\def\tit{ARCHITECTURAL DECISIONS IN~THE~PROBLEM 
OF~IDENTIFICATION OF~FRAUD IN~THE~ANALYSIS 
OF~INFORMATION FLOWS IN~DIGITAL ECONOMY\\[-5pt]}


\def\titkol{Architectural decisions in~the~problem 
of~identification of~fraud in~the~analysis 
of~information flows in~digital economy}

\def\aut{A.\,A.~Grusho, M.\,I.~Zabezhailo, N.\,A.~Grusho, and~E.\,E.~Timonina}

\def\autkol{A.\,A.~Grusho, M.\,I.~Zabezhailo, N.\,A.~Grusho, and~E.\,E.~Timonina}

\titel{\tit}{\aut}{\autkol}{\titkol}

\vspace*{-13pt}


 \noindent
   Institute of Informatics Problems, Federal Research Center ``Computer Sciences and 
Control'' of the Russian Academy of Sciences; 44-2~Vavilov Str., Moscow 119133, 
Russian Federation

\def\leftfootline{\small{\textbf{\thepage}
\hfill INFORMATIKA I EE PRIMENENIYA~--- INFORMATICS AND
APPLICATIONS\ \ \ 2019\ \ \ volume~13\ \ \ issue\ 2}
}%
 \def\rightfootline{\small{INFORMATIKA I EE PRIMENENIYA~---
INFORMATICS AND APPLICATIONS\ \ \ 2019\ \ \ volume~13\ \ \ issue\ 2
\hfill \textbf{\thepage}}}

\vspace*{3pt}


   
     
   \Abste{An approach to a~research of some types of fraud in digital economy with the usage of relationships of 
cause and effect is formulated. In all types of the considered frauds, the discrepancy between the 
purposes of financial transactions and actual cost of achievement of these purposes
has to be observed. Data on 
transactions can be collected by observing information flows in which these transactions are reflected. 
The architecture of data collection and their analysis can be organized by means of the distributed 
ledgers with the centralized consensus that allows creating an analog of the electronic account book 
fixing financial and economic activity of subjects of digital economy in the region. 
   The methods of fraud identification considered are based on the contradictions 
between actions described in transactions and information, which is contained in plans, standards, 
precedents, etc. 
   The method based on a~simplified scheme of implementation of the abstract project is considered. 
For identification of contradictions, it is necessary to carry out the analysis from the effect to the cause, 
i.\,e., to look for anomalies in information describing the generation of the observed effects. 
   It is shown how in implementation of the project it is possible to allocate simple ``necessary 
conditions'' of violation of cause and effect relationships, i.\,e., a~set of ``necessary conditions'' 
violation of which demonstrates fraud existence. It is possible to call this set of "necessary conditions" 
by metadata for control of the project for fraud identification.} 
   
   \KWE{digital economy; information flows; relationships of reason and effect; detection of 
fraudulent schemes}
   
  

 \DOI{10.14357/19922264190204}

\vspace*{-20pt}

 \Ack
   \noindent
   The work was partially supported by the Russian Foundation for Basic Research (projects  
18-29-03081 and 18-07-00274).



%\vspace*{6pt}

  \begin{multicols}{2}

\renewcommand{\bibname}{\protect\rmfamily References}
%\renewcommand{\bibname}{\large\protect\rm References}

{\small\frenchspacing
 {\baselineskip=10.5pt
 \addcontentsline{toc}{section}{References}
 \begin{thebibliography}{9}
\bibitem{1-gr-1}
\Aue{Grusho, A.\,A., A.\,A.~Zatsarinny, and E.\,E.~Timonina.} 2019. Blokcheyny tsifrovoy ekonomiki 
na baze sistemy situatsionnykh tsentrov i~tsentralizovannogo konsensusa [Blockchains of digital 
economy on the basis of the system of the situational centres and the centralized consensus]. 
\textit{25th Scientific and Technical Conference (International) ``Radar-Location, Navigation, 
Communication'' Proceedings}. Voronezh: VSU Publs. 6:183--191.
\bibitem{2-gr-1}
\Aue{Grusho, A., A.~Zatsarinny, and E.~Timonina.} 2019 (in press). 
A~system approach to information security 
in distributed ledgers on the situational centers platform. 
Lecture notes in computer science ser. Springer.
\bibitem{3-gr-1}
\Aue{Finn, V.\,K.} 2011. \textit{Iskusstvennyy intellekt: Metodologiya, primeneniya, filosofiya} 
[Artificial intelligence: Methodology, applications, philosophy]. Moscow: KRASAND. 448~p.

\bibitem{5-gr-1}
\Aue{Anshakov, O.\,M., and E.\,F.~Fabrikantova}. 2009. \textit{DSM-metod avtomaticheskogo porozhdeniya gipotez: Logicheskie 
i~epistemologicheskie osnovaniya} [JSM-method of automatic hypothesis generation: Logical and 
epistemological]. Moscow: KD LIBROKOM. 432~p.
\bibitem{4-gr-1} %5
\Aue{Poelmans, J., P.~Elzinga, S.~Viaene, and G.~Dedene.} 2010. Formal concept analysis in 
knowledge discovery: A~survey. \textit{Conceptual structures: From information to intelligence}. 
Eds.\ M.~Croitoru, S.~Ferr$\acute{\mbox{e}}$, and D.~Lukose. Lecture notes in 
computer science ser. Berlin--Heidelberg: Springer. 6208:139--153.

\bibitem{6-gr-1}
\Aue{Pankratov, E.\,S., and V.\,K.~Finn}. 
2009. \textit{Avtomaticheskoe porozhdenie gipotez v~intellektual'nykh 
sistemakh} [Automatic hypotheses generation in intelligent systems]. Moscow: KD 
\mbox{LIBROKOM}.  528~p. 
\bibitem{7-gr-1}
\Aue{Denisov, A.\,A., and D.\,N.~Kolesnikov.} 1982. \textit{Teoriya bol'shikh 
sistem upravleniya} [Theory of big control systems]. Leningrad: Energoizdat. 488~p.

\bibitem{9-gr-1}
\Aue{Grusho, A.\,A., N.\,A.~Grusho, M.\,I.~Zabezhailo, D.\,V.~Smirnov, and 
E.\,E.~Timonina.} 2018. 
Parametrizatsiya v~prikladnykh zadachakh poiska empiricheskikh prichin 
[Parametrization in applied 
problems of search of the empirical reasons]. 
\textit{Informatika i~ee Primeneniya~--- 
Inform. Appl.} 12(3):62--66.

\bibitem{8-gr-1}
\Aue{Grusho, A.\,A., N.\,A.~Grusho, M.\,V.~Levykin, and E.\,E.~Timonina.} 2018. Metody 
identifikatsii zakhvata khosta v~raspredelennoy informatsionno-vychislitel'noy sisteme, 
zashchishchennoy s~pomoshch'yu metadannykh [Methods of identification of host capture 
in the  distributed information system which is protected on the base of meta data].
\textit{Informatika i~ee 
Primeneniya~--- Inform. Appl.} 12(4):41--45.
{ %\looseness=1

}

\end{thebibliography}

 }
 }

\end{multicols}

\vspace*{-12pt}

\hfill{\small\textit{Received April 3, 2019}}

%\pagebreak

%\vspace*{-18pt}

\Contr

\noindent
\textbf{Grusho Alexander A.} (b.\ 1946)~--- Doctor of Science in physics and 
mathematics, professor, principal scientist, Institute of Informatics Problems, 
Federal Research Center ``Computer Sciences and Control'' of the Russian 
Academy of Sciences; 44-2~Vavilov Str., Moscow 119133, Russian Federation; 
\mbox{grusho@yandex.ru} 

\vspace*{3pt}

\noindent
\textbf{Zabezhailo Michael I.} (b.\ 1956)~--- Doctor of Science in physics and 
mathematics, principal scientist, Institute of Informatics Problems, Federal Research 
Center ``Computer Sciences and Control'' of the Russian Academy of Sciences;  
44-2~Vavilov Str., Moscow 119133, Russian Federation; 
\mbox{m.zabezhailo@yandex.ru} 

\vspace*{3pt}


\noindent
\textbf{Grusho Nikolai A.} (b.\ 1982)~--- Candidate of Science (PhD) in physics 
and mathematics, senior scientist, Institute of Informatics Problems, Federal 
Research Center ``Computer Sciences and Control'' of the Russian Academy of 
Sciences; 44-2~Vavilov Str., Moscow 119133, Russian Federation; 
\mbox{info@itake.ru} 

\vspace*{3pt}


\noindent
\textbf{Timonina Elena E.} (b.\ 1952)~--- Doctor of Science in technology, 
professor, leading scientist, Institute of Informatics Problems, Federal Research 
Center ``Computer Sciences and Control'' of the Russian Academy of Sciences;  
44-2~Vavilov Str., Moscow 119133, Russian Federation; 
\mbox{eltimon@yandex.ru} 

\label{end\stat}

\renewcommand{\bibname}{\protect\rm Литература}            %10
\def\stat{gr+timon}

\def\tit{ИСПОЛЬЗОВАНИЕ МЕТАДАННЫХ ДЛЯ~РЕАЛИЗАЦИИ ТРЕБОВАНИЙ ПОЛИТИКИ 
БЕЗОПАСНОСТИ MLS$^*$}

\def\titkol{Использование метаданных для~реализации требований политики 
безопасности MLS}

\def\aut{А.\,А.~Грушо$^1$,  Н.\,А.~Грушо$^2$, 
Е.\,Е.~Тимонина$^3$}

\def\autkol{А.\,А.~Грушо, Н.\,А.~Грушо, 
Е.\,Е.~Тимонина}

\titel{\tit}{\aut}{\autkol}{\titkol}

\index{Грушо А.\,А.}
\index{Грушо Н.\,А.} 
\index{Тимонина Е.\,Е.}
\index{Grusho A.\,A.}
\index{Grusho N.\,A.}
\index{Timonina E.\,E.}



{\renewcommand{\thefootnote}{\fnsymbol{footnote}} \footnotetext[1]
{Работа частично поддержана РФФИ (проект 18-07-00274).}}


\renewcommand{\thefootnote}{\arabic{footnote}}
\footnotetext[1]{Институт проблем информатики Федерального исследовательского центра <<Информатика и~управление>> 
Российской академии наук, \mbox{grusho@yandex.ru}}
\footnotetext[2]{Институт проблем информатики Федерального исследовательского центра <<Информатика и~управление>> 
Российской академии наук, info@itake.ru}
\footnotetext[3]{Институт проблем информатики Федерального исследовательского центра <<Информатика и~управление>> 
Российской академии наук, eltimon@yandex.ru}

\vspace*{2pt}

   


  \Abst{Рассматривается распределенная информационная система, объекты которой содержат 
как ценную информацию (или сами являются ценными), так и~открытую (не ценную) 
информацию. Для защиты ценной информации используется политика безопасности (ПБ) MLS
(Multilevel Security), 
которая запрещает информационные потоки от объектов с~ценной информацией к~объектам 
с~открытой информацией. Объекты с~ценной информацией образуют класс объектов уровня High, 
а~объекты c~открытой информацией образуют класс объектов уровня Low. 
  Метаданные (МД) создаются для управления соединениями в~сетях. Метаданные являются 
упрощением математических моделей биз\-нес-про\-цес\-сов и~служат основой разрешительной 
системы для соединений хостов в~распределенной 
ин\-фор\-ма\-ци\-он\-но-вы\-чис\-ли\-тель\-ной сис\-те\-ме
(РИВС).
  В~работе сформулированы правила ПБ MLS и~на основе 
инфраструктуры, связанной с~МД, показана возможность реализации этой 
ПБ в~РИВС. 
Единственный доверенный процесс, необходимый для реализации ПБ MLS, 
функционирует на уровне управления соединениями. Этот уровень не связан с~плос\-костью 
передачи данных и~может быть изолирован с~\mbox{целью} обеспечения его информационной 
безопасности.}
  
  \KW{политика безопасности MLS; информационные потоки; метаданные}
  
  \DOI{10.14357/19922264190414} 
  
%\vspace*{1pt}


\vskip 10pt plus 9pt minus 6pt

\thispagestyle{headings}

\begin{multicols}{2}

\label{st\stat}

\section{Введение}

  Политика безопасности в~компьютерной системе и~сети~--- это набор 
требований по ограничению доступа, хранению и~распределению 
информации~[1]. Обычно ПБ определяет требования по защите 
конфиденциальности, целостности и~доступности информации. Политика
безопас\-ности опирается на 
четкую классификацию ценных информационных ресурсов и~открытых 
информационных ресурсов. 
  
  Один из общих подходов к~описанию требований информационной 
безопасности к~конкретному информационному ресурсу~--- это неотделимая 
привязка к~информационному объекту вектора характеристик, определяющих 
обращение с~этой информацией. Каждый такой вектор содержит результаты 
классификации объекта, а~именно: требования по конфиденциальности, 
целостности и~доступности информации. 
  
  Для простоты будем классифицировать информацию как конфиденциальную 
  и~как открытую. Объекты, содержащие конфиденциальную информацию (ценную 
информацию), будем помечать символом~$(*)$. Это ограничение не умаляет 
общности, так как легко обобщается на требования защиты целостности 
и~доступности. 
  
  Для защиты конфиденциальности широко используется политика 
  MLS~[1], которая запрещает информационные потоки от 
объектов~$(*)$ к~объектам, не помеченным~$(*)$. Объекты с~меткой~$(*)$ 
образуют класс объектов уровня High, а объекты без метки образуют класс 
объектов уровня Low. 
  
  Метаданные создаются для управления соединениями в~сетях~[2]. Как 
правило, МД являются упрощением математических моделей  
биз\-нес-про\-цес\-сов и~служат основой разрешительной системы для соединений 
хостов в~РИВС. Разрешительная система строится на основе порядка взаимодействий 
задач, реализующих бизнес-процесс. Для ее работы вводятся две специальные 
задачи~$\mathcal{M}$ и~$\mathcal{N}$. Задача~$\mathcal{M}$ распределяет 
задачи по хостам, т.\,е.\ определяет бинарное отношение $H(A)$, где~$A$~--- 
задача, а~$H$~--- хост сети. Задача~$\mathcal{N}$ реализует разрешительную 
систему, которая разрешает и~организует соединения хостов $H(A)$ и~$H(B)$, 
если в~МД отражена необходимость инициализации или взаимодействия 
задач~$A$ и~$B$. 
  
  Как было показано в~[3, 4], МД не несут информации о значениях данных. 
Поэтому требования ПБ в~МД отражаются косвенно, т.\,е.\ МД могут содержать 
только информацию о том, куда нельзя направлять информационный поток. 
  
  Как правило, информационная технология (ИТ) представима в~виде составной 
задачи~[5], а~так\-же может описываться DAG (Directed Acyclic Graph)~[6], 
в~котором вершины~--- это задачи ИТ, а~дуги указывают направления 
информационных потоков, передающих исходные данные сле\-ду\-ющим задачам ИТ. 
В~таком представлении возможно существование дуг извне DAG как некоторых 
внешних информационных потоков с~исходными данными и~дуг, выходящих из 
DAG, но не входящих в~задачи ИТ (внешнее распределение информации). 
Передача ценных информационных ресурсов также осуществляется через дуги 
DAG. Поэтому передача ценного информационного ресурса соответствует 
метке~$(*)$ на соответствующей дуге. 
  
\section{Отражение требований MLS в~графах задач и~метаданных} 
  
  Правила MLS могут быть выражены следующим образом. Если в~вершину 
входит хоть одна дуга, помеченная~$(*)$, то эта вершина уже имеет метку~$(*)$ 
и~далее все дуги, выходящие из этой вершины, приобретают метку~$(*)$. 
Возможно, что ценная информация может порождаться в~результате решения 
задачи. Тогда эта задача помечается~$(*)$. 
  
  Справедливо следующее утверждение. 
  
  \smallskip
  
  \noindent
  \textbf{Утверждение~1.}\ \textit{При выполнении правил расстановки 
меток~$(*)$ в~данной ИТ выполняется политика MLS. }
  
  \smallskip
  
  \noindent
  Д\,о\,к\,а\,з\,а\,т\,е\,л\,ь\,с\,т\,в\,о\,.\ \ Допустим, что существует 
информационный поток с~уровня High на уровень Low. Тогда вершина, из которой 
исходит данный поток, имеет метку~$(*)$. По определению любая дуга, 
выходящая из вершины, помеченной~$(*)$, также имеет метку~$(*)$. Но такая 
дуга может входить только в~вершину, которая уже помечена (*). Но вершины 
уровня Low не могут иметь таких меток. Следовательно, предположение о 
существовании информационного потока с~уровня High на уровень Low неверно. 
Утверждение~1 доказано.
  
  \smallskip
  
  Метаданные содержат порядок решения задач, определяемый DAG. Поэтому 
если задачи в~этом порядке используют или порождают ценную информацию, то 
они имеют метку~$(*)$. Тогда все дальнейшие задачи также имеют такую метку. 
При таком дополнении МД несут информацию о требованиях~ПБ. 
  
  Ясно, что появление ценой информации означает дополнительные требования 
  к~хосту, на котором решается эта задача. Поэтому хост, на котором решается задача 
с~меткой~$(*)$, также должен иметь метку~$(*)$. Задача должна иметь 
метку~$(*)$, если она будет содержать ценную информацию. Отметим, что 
протокол управления соединениями в~РИВС основан на криптографии~[7] 
и~является безопасным. 
  
  Особенности хоста с~меткой~$(*)$ основаны на том, что на такой хост должен 
быть загружен <<чис\-тый>> образ операционной сис\-те\-мы, безопасный агент хоста для связи 
с~задачей~$\mathcal{N}$, <<чис\-тое>> программное
обеспечение для задач с~метками~$(*)$, и~на нем 
реализована процедура доверенной загрузки. В~MLS разрешены информационные 
потоки от уровня Low к~уровню High. Для предотвращения попадания на 
хост~$H^*$ вредоносного кода с~уровня Low необходимо обеспечить безопасный 
однонаправленный канал~[8] с~уровня Low на уровень High. Ясно, что на 
хосте~$H^*$ могут решаться различные (доверенно загруженные) задачи 
с~меткой~$(*)$. 
  
  Рассмотрим процедуру решения задачи~$A$ на уровне Low. В~этом случае 
возможны два варианта:
  \begin{enumerate}[(1)]
\item задача~$\mathcal{M}$ устанавливает задачу~$A$ на хосте, к~которому не 
допускается ни один информационный поток с~меткой~$(*)$;
\item на хосте $H^*(A^*)$ реализуется ИТ очистки от информации 
с~меткой~$(*)$ (новая доверенная загрузка). 
\end{enumerate}
  
\section{Отражение требований MLS в~инфраструктуре метаданных}

  Поскольку ИТ может использовать информацию с~меткой~$(*)$, то в~МД задача 
имеет метку~$(*)$. Если ИТ не использует информацию, помеченную~$(*)$, то 
в~МД отсутствуют задачи с~такой меткой. 
  
  Как было отмечено ранее, для передачи ценной информации задаче она должна 
иметь метку~$(*)$. Тогда хост, на котором находится эта задача, также должен 
иметь метку~$(*)$. Отсюда получается простейшее решение 
задачи~$\mathcal{M}$, обеспечивающее выполнение ПБ
MLS. Так как заранее известны все задачи с~меткой~$(*)$, то 
задача~$\mathcal{M}$ размещает их на хостах, помеченных~$(*)$, и~по правилам 
MLS задача~$\mathcal{N}$ реализует разрешительную систему на данных хостах. 
Это означает, что в~задаче~$\mathcal{N}$ выделяется подзадача~$\mathcal{N}_H$, 
реа\-ли\-зу\-ющая взаимодействие только на хостах с~меткой~$(*)$. Остальные хосты 
относятся к~уровню Low, и~задача~$\mathcal{N}$ реализует разрешительную 
систему только этих хостов. Таким образом, в~задаче~$\mathcal{N}$ выделяется 
независимая подзадача~$\mathcal{N}_L$, реализующая разрешительную систему 
взаимодействия хостов на уровне Low. 
  
  Отметим, что задачи~$\mathcal{N}_H$ и~$\mathcal{N}_L$ могут 
рассматриваться как два экземпляра задачи~$\mathcal{N}$, функционирующие на 
непересекающихся доменах РИВС. 
  
  Наиболее сложный вопрос состоит в~безопасной передаче информации с~уровня 
Low на уровень High. Согласно модели Bell-LaPadula~\cite{9-tt}, такую передачу 
можно осуществить только с~по\-мощью доверенного процесса, реализующего 
од\-но\-на\-прав\-лен\-ный канал с~уровня Low на уровень High. Такой однонаправленный 
канал можно реализовать на базе задачи~$\mathcal{N}$. Этот доверенный канал 
может быть организован на основе инфраструктуры метаданных следующим 
образом. 
  
  Пусть задача~$A$ запрашивает соединение с~задачей~$B^*$ через 
задачу~$\mathcal{N}_L$. Исходя из метаданных, задача~$\mathcal{N}$ 
определяет необходимость передачи исходных данных из задачи~$A$ 
в~задачу~$B^*$. При получении разрешения эти данные передаются из задачи~$A$ в~задачу~$\mathcal{N}_L$. После проверки их безопасности данные из 
задачи~$\mathcal{N}_L$ передаются в~задачу~$\mathcal{N}_H$ для дальнейшей 
передачи данных в~задачу~$B^*$. Отметим, что непосредственного соединения 
хоста уровня Low с~хостом уровня High не происходит. Отсюда следует 
утверждение. 
  
  \smallskip
  
  \noindent
  \textbf{Утверждение~2.}\ \textit{Пусть выполняются следующие условия}:
  \begin{enumerate}[(1)]
  \item \textit{ сформирована подсистема РИВС уровня High с~помощью задач 
  и~хостов с~метками~$(*)$ и~разрешительная система на основе МД 
и~задачи}~$\mathcal{N}_H$;
\item \textit{сформирована подсистема РИВС уровня Low с~помощью задач 
и~хостов без меток~$(*)$ и~разрешительная система на основе метаданных 
и~задачи}~$\mathcal{N}_L$;
  \item \textit{взаимодействие уровней Low и~High осуществляется только через 
однонаправленное взаимодействие задач~$\mathcal{N}_L$ 
и~~$\mathcal{N}_H$}. 
  \end{enumerate}
  \textit{Тогда в~РИВС выполняется политика MLS}.
  
  \smallskip
  
  \noindent
  Д\,о\,к\,а\,з\,а\,т\,е\,л\,ь\,с\,т\,в\,о\,.\ \ Метаданные
   и~задача~$\mathcal{N}_H$ не 
допускают взаимодействия уровня High с~уровнем Low. Аналогично МД и~задача 
NL не допускают непосредственного взаимодействия уровня Low с~уровнем High. 
Безопасный интерфейс уровня Low с~уровнем High реализован 
однонаправленным каналом между задачами~$\mathcal{N}_L$ 
и~$\mathcal{N}_H$. Функционал, реализующий этот канал, как и~вся 
задача~$\mathcal{N}$, могут быть изолированы от остального функционала РИВС и~поэтому могут считаться доверенным субъектом. Таким образом, все условия 
реализации политики MLS выполнены. Утверждение~2 доказано. 
  
  \section{Поиск информации с~уровня High на~уровне Low}
  
  Для решения задач с~меткой~$(*)$ может возникнуть необходимость 
дополнительного поиска информации в~памяти о предшествующих задачах. 
Метаданные сохраняют информацию о цепочке решенных задач, и~обращение 
к~ним на уровне High не представляет сложности. 
  
  Если задаче~$A^*$ необходимо найти дополнительную информацию на уровне 
Low, то тогда также можно использовать доверенное взаимодействие между 
задачами~$\mathcal{N}_H$ и~$\mathcal{N}_L$. Задача~$\mathcal{N}_H$ 
обращается к~задаче~$\mathcal{N}_L$ с~запросом на поиск данных в~решенных на 
уровне Low задачах. Задача~$\mathcal{N}_L$, используя обратный обзор 
решенных на уровне Low задач, ищет искомую информацию. В~данном случае 
возможен скрытый канал с~уровня High на уровень Low в~задании поиска 
информации на уровне Low~\cite{8-tt}. Этот канал можно перекрыть с~помощью 
последовательного опроса задач на уровне Low и~выявления признаков искомой 
информации уже на уровне задачи~$\mathcal{N}_L$. В~случае появления 
необходимых признаков задача~$\mathcal{N}_L$ передает 
задаче~$\mathcal{N}_H$ данные для задачи~$A^*$. 
  
  Данный способ не является доказательством перекрытия скрытого канала. 
Однако обращение с~уровня High на уровень Low считается запрещенным 
информационным потоком в~политике MLS, и~реализация такого поиска с~учетом 
возможности скрытого канала является сложной задачей~\cite{10-tt}. 
  
  \section{Заключение }
  
  В работе рассмотрена РИВС, в~которой управ\-ле\-ние соединениями 
осуществляется с~помощью МД. Показана возможность реализации 
ПБ MLS в~рассматриваемой РИВС на основе 
инфраструктуры, связанной с~МД. Единственный доверенный процесс, 
необходимый для реализации ПБ MLS, функционирует на 
уровне управ\-ле\-ния соединениями. Этот уровень не связан с~плос\-костью передачи 
данных и~может быть изолирован с~\mbox{целью} обеспечения его информационной 
безопас\-ности. 
  
  В работе рассмотрена только одна ИТ, для которой необходимо выполнить 
требования политики безопасности MLS. Однако рассмотренный метод легко 
обобщается на случай множества ИТ.
  

{\small\frenchspacing
 {%\baselineskip=10.8pt
 \addcontentsline{toc}{section}{References}
 \begin{thebibliography}{99}
\bibitem{1-tt}
Department of Defense trusted computer system evaluation criteria.~--- U.S.\ National Institute of 
Standards and Technology, Department of Defense, 1985. {\sf 
http://csrc.nist.gov/publications/history/dod85.\linebreak pdf}. 
\bibitem{2-tt}
\Au{Grusho A., Grusho N., Zabezhailo~M., Zatsarinny~A., Timonina~E.} Information security of SDN 
on the basis of meta data~// Computer 
network security~/ Eds. J.~Rak, J.~Bay, I.~Kotenko, \textit{et al.}~---
Lecture notes in computer science ser.~--- Springer, 2017. 
Vol.~10446.  P.~339--347. doi: 10.1007/978-3-319-65127-9\_27.
\bibitem{3-tt}
\Au{Грушо А.\,А., Грушо~Н.\,А., Левыкин~М.\,В., Тимонина~Е.\,Е.} Методы идентификации 
захвата хоста в~распределенной ин\-фор\-ма\-ци\-он\-но-вы\-чис\-ли\-тель\-ной системе, 
защищенной с~по\-мощью метаданных~// Информатика и~её применения, 2018. Т.~12. Вып.~4. 
С.~41--45.
\bibitem{4-tt}
\Au{Grusho A.\,A., Grusho~N.\,A., Timonina~E.\,E.} 
Information flow control on the basis of meta 
data~// Distributed computer and communication networks~/ 
Eds. V.\,M.~Vishnevskiy, K.\,E.~Samouylov, D.\,V.~Kozyrev.~--- 
Lecture notes
in computer science ser.~--- Springer,
 2019. Vol.~11965. P.~548--562.

\bibitem{5-tt}
\Au{Грушо А.\,А., Тимонина~Е.\,Е., Шоргин~С.\,Я.} Иерархический метод порождения 
метаданных для управления сетевыми соединениями~// Информатика и~её применения, 2018. 
Т.~12. Вып.~2. С.~44--49.
\bibitem{6-tt}
\Au{Грушо А.\,А., Зацаринный~А.\,А., Тимонина~Е.\,Е.} Электронная бухгалтерская книга на базе 
ситуационных центров для цифровой экономики~// Системы и~средства информатики, 2019. 
Т.~29. №\,2. С.~4--11.
\bibitem{7-tt}
\Au{Grusho A.\,A., Timonina~E.\,E., Shorgin~S.\,Ya.} Modelling for ensuring information security of 
the distributed information systems~// 31th European Conference on Modelling and Simulation 
Proceedings.~--- Dudweiler, Germany: Digitaldruck Pirrot GmbH, 2017. P.~656--660. {\sf 
http://www.scs-europe.net/dlib/2017/\linebreak  
ecms2017acceptedpapers/0656-probstat\_ECMS2017\_ 0026.pdf}.
\bibitem{8-tt}
\Au{Тимонина Е.\,Е.} Анализ угроз скрытых каналов и~методы построения гарантированно 
защищенных распределенных автоматизированных систем: Дис.\ \ldots\ д-ра техн. наук.~--- 
М., 2004. 204~с.
\bibitem{9-tt}
\Au{Грушо А.\,А., Применко~Э.\,А., Тимонина~Е.\,Е.} Теоретические основы компьютерной 
безопасности.~--- М.: Академия, 2009. 272~с.
\bibitem{10-tt}
\Au{Grusho A.\,A., Grusho~N.\,A., Zabezhailo~M.\,I., Timonina~E.\,E.} Protection of valuable 
information in public information space~// Communications of the ECMS: 33th European Conference 
on Modelling and Simulation Proceedings.~--- 
Dudweiler, Germany: Digitaldruck Pirrot GmbH, 2019. 
Vol.~33. No.\,1. P.~451--455. 
{\sf 
http://www.scs-europe.net/dlib/2019/ecms2019acceptedpapers/0451\_\linebreak pstat\_ecms2019\_0018.pdf}.
 \end{thebibliography}

 }
 }

\end{multicols}

\vspace*{-6pt}

\hfill{\small\textit{Поступила в~редакцию 13.10.19}}

\vspace*{8pt}

%\pagebreak

%\newpage

%\vspace*{-28pt}

\hrule

\vspace*{2pt}

\hrule

%\vspace*{-2pt}

\def\tit{USING METADATA TO~IMPLEMENT MULTILEVEL SECURITY POLICY 
REQUIREMENTS}


\def\titkol{Using metadata to~implement multilevel security policy 
requirements}

\def\aut{A.\,A.~Grusho, N.\,A.~Grusho, and~E.\,E.~Timonina}

\def\autkol{A.\,A.~Grusho, N.\,A.~Grusho, and~E.\,E.~Timonina}

\titel{\tit}{\aut}{\autkol}{\titkol}

\vspace*{-11pt}


 \noindent
   Institute of Informatics Problems, Federal Research Center ``Computer Sciences and 
Control'' of the Russian Academy of Sciences; 44-2~Vavilov Str., Moscow 119133, 
Russian Federation

\def\leftfootline{\small{\textbf{\thepage}
\hfill INFORMATIKA I EE PRIMENENIYA~--- INFORMATICS AND
APPLICATIONS\ \ \ 2019\ \ \ volume~13\ \ \ issue\ 4}
}%
 \def\rightfootline{\small{INFORMATIKA I EE PRIMENENIYA~---
INFORMATICS AND APPLICATIONS\ \ \ 2019\ \ \ volume~13\ \ \ issue\ 4
\hfill \textbf{\thepage}}}

\vspace*{3pt}  
 
  
   \Abste{A distributed information computing system which objects contain both valuable 
information (or are themselves valuable) and open (non-valuable) information is considered. To protect 
valuable information, multilevel  security (MLS) policy is used that prohibits information flows from objects with 
valuable information to objects with open information. Objects with valuable information form a~class 
of high-level objects, and objects with open information form a class of low-level objects.
   Metadata is created to manage network connections. Metadata is a simplification of mathematical 
models of business processes and is the basis of a permission system for host connections in 
a~distributed information computing system.
   The paper constructs MLS security policy rules, and based on metadata-related infrastructure, 
shows the ability to implement this security policy in the distributed information computing system. 
The only trusted process required to implement the MLS security policy is at the connection 
management level. This layer is unrelated to the data plane and can be isolated to ensure its 
information security.}
    
   \KWE{MLS security policy; information flows; metadata}
   
   
   

\DOI{10.14357/19922264190414} 

%\vspace*{-14pt}

 \Ack
   \noindent
   The paper was partially supported by the Russian Foundation for Basic Research (project  
18-07-00274).


%\vspace*{-6pt}

  \begin{multicols}{2}

\renewcommand{\bibname}{\protect\rmfamily References}
%\renewcommand{\bibname}{\large\protect\rm References}

{\small\frenchspacing
 {%\baselineskip=10.8pt
 \addcontentsline{toc}{section}{References}
 \begin{thebibliography}{99}
\bibitem{1-tt-1}
U.S.\ National Institute of Standards and Technology, Department of Defence. 1985.
Department of Defense trusted computer system evaluation criteria. Available at: {\sf 
http://csrc.nist.gov/publications/history/dod85.pdf} (accessed October~6, 2019).
\bibitem{2-tt-1}
\Aue{Grusho, A., N.~Grusho, M.~Zabezhailo, A.~Zatsarinny, and E.~Timonina.} 2017. Information 
security of SDN on the basis of meta data. 
\textit{Computer network security}. Eds. J.~Rak, J.~Bay, I.~Kotenko, 
\textit{et al.}
Lecture notes in computer science ser. Springer. 
10446:339--347. doi: 10.1007/978-3-319-65127-9\_27.
\bibitem{3-tt-1}
\Aue{Grusho, A.\,A., N.\,A.~Grusho, M.\,V.~Levykin, and E.\,E.~Timonina.} 2018. Metody 
identifikatsii zakhvata khosta v~raspredelennoy informatsionno-vychislitel'noy sisteme, 
zashchishchennoy s~pomoshch'yu metadannykh [Methods of identification of host capture in the 
distributed information system which is protected on the base of meta data]. \textit{Informatika i~ee 
Primeneniya~--- Inform. Appl.} 12(4):41--45. 
\bibitem{4-tt-1}
\Aue{Grusho, A.\,A., N.\,A.~Grusho, and E.\,E.~Timonina.} 2019. Information flow control 
on the basis of meta data. \textit{Distributed computer and communication networks}.
 Eds. V.\,M.~Vishnevskiy, 
K.\,E.~Samouylov, and D.\,V.~Ko\-zy\-rev. 
Lecture notes
in computer science ser. Springer. 11965:548--562.
\bibitem{5-tt-1}
\Aue{Grusho, A.\,A., E.\,E.~Timonina, and S.\,Ya.~Shorgin.} 2018. Ierarkhicheskiy metod 
porozhdeniya metadannykh dlya upravleniya setevymi soedineniyami [Hierarchical method of meta 
data generation for control of network connections]. \textit{Informatika i~ee Primeneniya~--- Inform. 
Appl.} 12(2):44--49.
\bibitem{6-tt-1}
\Aue{Grusho, A.\,A., A.\,A.~Zatsarinny, and E.\,E.~Timonina.} 2019. Elektronnaya bukhgalterskaya 
kniga na baze situatsionnykh tsentrov dlya tsifrovoy ekonomiki [The electronic ledger on the basis of 
the situational centers for digital economy]. \textit{Sistemy i~Sredstva Informatiki~--- Systems and 
Means of Informatics} 29(2):4--11.
\bibitem{7-tt-1}
\Aue{Grusho, A.\,A., E.\,E.~Timonina, and S.\,Ya.~Shorgin.} 2017. Modelling for ensuring 
information security of the distributed information systems. \textit{31th European Conference on 
Modelling and Simulation Proceedings}. Dudweiler, Germany: Digitaldruck Pirrot GmbH. 656--660. 
Available at: {\sf  
http://www.scs-europe.net/dlib/2017/\linebreak ecms2017acceptedpapers/0656-probstat\_ECMS2017\_ 0026.pdf} (accessed 
October~6, 2019).
\bibitem{8-tt-1}
\Aue{Timonina, E.\,E.} 2004. Analiz ugroz skrytykh kanalov i~metody postroeniya garantirovanno 
zashchishchennykh raspredelennykh avtomatizirovannykh sistem [The analysis of threats of covert 
channels and methods of creation of guaranteed protected distributed automated 
systems]. Moscow.  D.Sc. Diss.  204~p.
\bibitem{9-tt-1}
\Aue{Grusho, A., E.~Primenko, and E.~Timonina.} 2009. \textit{Teoreticheskie osnovy 
komp'yuternoy bezopasnosti} [Theoretical bases of computer security]. Moscow: Academy. 272~р.
\bibitem{10-tt-1}
\Aue{Grusho, A.\,A., N.\,A.~Grusho, M.\,I.~Zabezhailo, and E.\,E.~Timonina.} 2019. Protection of 
valuable information in public information space. \textit{Communications of the ECMS:  33th 
European Conference on Modelling and Simulation Proceedings}. 
Dudweiler, Germany: Digitaldruck Pirrot GmbH.
33(1):451--455. Available at: {\sf 
http://www.scs-europe.net/dlib/2019/ecms2019acceptedpapers/0451\_\linebreak pstat\_ecms2019\_0018.pdf} (accessed 
October~6, 2019).
\end{thebibliography}

 }
 }

\end{multicols}

\vspace*{-6pt}

\hfill{\small\textit{Received October 13, 2019}}

%\pagebreak

%\vspace*{-22pt}


\Contr


\noindent
\textbf{Grusho Alexander A.} (b.\ 1946)~--- Doctor of Science in physics and 
mathematics, professor, principal scientist, Institute of Informatics Problems, Federal 
Research Center ``Computer Sciences and Control'' of the Russian Academy of 
Sciences; 44-2~Vavilov Str., Moscow 119133, Russian Federation;  
\mbox{grusho@yandex.ru}

\vspace*{3pt} 

\noindent
\textbf{Grusho Nikolai A.} (b.\ 1982)~--- Candidate of Science (PhD) in physics 
and mathematics, senior scientist, Institute of Informatics Problems, Federal 
Research Center ``Computer Sciences and Control'' of the Russian Academy of 
Sciences;  
44-2~Vavilov Str., Moscow 119133, Russian Federation; info@itake.ru 

\vspace*{3pt}

\noindent
\textbf{Timonina Elena E.} (b.\ 1952)~--- Doctor of Science in technology, 
professor, leading scientist, Institute of Informatics Problems, Federal Research 
Center ``Computer Sciences and Control'' of the Russian Academy of Sciences;  
44-2~Vavilov Str., Moscow 119133, Russian Federation; 
\mbox{eltimon@yandex.ru}
\label{end\stat}

\renewcommand{\bibname}{\protect\rm Литература}   %11
\def\stat{malashenko}

\def\tit{ПОСЛЕДОВАТЕЛЬНЫЙ АНАЛИЗ И~МЕТРИЧЕСКИЕ ОЦЕНКИ ПРЕДЕЛЬНЫХ
РАСПРЕДЕЛЕНИЙ МЕЖУЗЛОВЫХ ПОТОКОВ В~МНОГОПОЛЬЗОВАТЕЛЬСКОЙ СЕТИ}

\def\titkol{Последовательный анализ и~метрические оценки предельных
распределений межузловых потоков в %~многопользовательской 
сети}

\def\aut{Ю.\,Е. Малашенко$^1$}

\def\autkol{Ю.\,Е. Малашенко}

\titel{\tit}{\aut}{\autkol}{\titkol}

\index{Малашенко Ю.\,Е.}
\index{Malashenko Yu.\,E.}


%{\renewcommand{\thefootnote}{\fnsymbol{footnote}} \footnotetext[1]
%{Исследование выполнено при финансовой поддержке Российского научного фонда (проект 
%<<Информатика>> ФИЦ ИУ РАН, Москва).}}


\renewcommand{\thefootnote}{\arabic{footnote}}
\footnotetext[1]{Федеральный исследовательский центр <<Информатика и~управление>> Российской академии 
\mbox{mala-yur@yandex.ru}}


%\vspace*{-6pt}



\Abst{Для оценки функциональных возможностей
многопользовательской сети связи аналилизируется множество векторов межузловых потоков при предельных распределениях ресурсов
сети. В~рамках многопродуктовой модели про\-пуск\-ные спо\-соб\-ности ребер рас\-смат\-ри\-ва\-ют\-ся 
как компоненты вектора ресурсов различных
типов, которые требуются для передачи потоков различных видов.
При проведении вычислительных экспериментов на каждой итерации вычисляются нормы векторов совместно допустимых межузловых
потоков, при передаче которых полностью используется пропускная спо\-соб\-ность всех ребер сети. Полученные последовательности
метрических оценок позволяют анализировать особенности множества до\-сти\-жи\-мости и~эф\-фек\-тив\-ность использования ресурсов сети при
уравнительном распределении про\-пуск\-ной спо\-соб\-ности между корреспондентами.}

\KW{многопродуктовая потоковая сетевая
модель; множество достижимых межузловых потоков; предельные
распределения пропускной способности}

\DOI{10.14357/19922264220306} 
  
%\vspace*{-3pt}


\vskip 10pt plus 9pt minus 6pt

\thispagestyle{headings}

\begin{multicols}{2}

\label{st\stat}

\section{Введение}

Данная работа продолжает исследования функциональных характеристик
сетевых сис\-тем связи~[1]. В~настоящее время математические модели
передачи многопродуктового потока применяются для поиска
распределений потоков и~ресурсов в~многопользовательских
телекоммуникационных\linebreak сетях~[2--4]. Разрабатываются методы анализа
с~учетом вектора требований всех \textit{равноправных} 
и~невзаимозаменяемых корреспондентов~[5]. С~позиций\linebreak методологии
исследования операций изучаются справедливые распределения потоков
и~ресурсов~[6].

Соответствующие \textit{недискриминирующие} правила управления
потоками являются решениями задач на максмин и/или получаются 
в~результате использования процедур последовательной
лексикографически упорядоченной оптимизации~[7].

В~настоящей работе пути соединения корреспондентов прокладываются
через со\-от\-вет\-ст\-ву\-ющие минимальные разрезы. Указанный метод\linebreak \mbox{можно}
рассматривать как возможный вариант решения задачи о~построении
SPLIT-марш\-ру\-тов~[8,~9]. В~рамках вычислительных экспериментов\linebreak на
многопродуктовой модели анализируются распределения межузловых
потоков  и~пропускной способ\-ности сети.  Для оценки функциональных
возможностей многопользовательской сети используется вектор
совместно допустимых межузловых потоков. Под ресурсом, выделяемым
некоторой паре узлов-кор\-рес\-пон\-ден\-тов,  понимается суммарное
значение тре\-бу\-емых пропускных способностей на всех ребрах,
расположенных на всех маршрутах при прохождении межузлового\linebreak потока
данного вида.  Сумма соответствующих реберных потоков трактуется
как полная нагрузка на сеть, возникающая при передаче заданного
межузлового потока. Рас\-смат\-ри\-ва\-ют\-ся распределения пропускной
способности и~межузловых потоков при предельной загрузке сети.
При проведении вычислительных экспериментов на каждой  итерации
вычисляется норма  вектора совместно допустимых межузловых
потоков.   Для оценки величины требуемых ресурсов при соединении
корреспондентов по различным путям для каж\-дой пары узлов
определяется максимальный однопродуктовый поток. Марш\-ру\-ты передачи
всех допустимых межузловых потоков  проходят по ребрам
соответствующих минимальных разрезов. Вычислительные эксперименты
проводились  для получения последовательности  мет\-ри\-че\-ских оценок
векторов межузловых потоков, принадлежащих множеству до\-сти\-жи\-мости
многопользовательской сети.

\section{Математическая модель}

В качестве математической модели для описания
многопользовательской сетевой системы используется следующая
формальная запись условий и~ограничений, которые должны
выполняться при одновременной передаче потоков различных видов
между всеми парами улов-корреспондентов:

Сеть $G(\mathbf{d})$ задается множествами $\langle V,
R,U,P\rangle$:
\begin{itemize}
\item  узлов (вершин) сети 
$$
V=\left \{v_{1}, v_{2},\dots,v_{n},\dots,v_{N}\right\};
$$
\item  неориентированных ребер 
$$
R=\left\{r_{1}, r_{2}, \dots, r_{k}, \dots,
r_{E}\right\}.
$$
\end{itemize}

Ребро $r_{k}$ \textit{соединяет} концевые вершины~$v_{n_k}$ и~$v_{j_k}$. 
Реб\-ру~$r_{k}$ ставятся в~соответствие две
ориентированные дуги $\{u_{k},u_{k+E}\}$ из множества
ориентированных дуг $U\hm=\{u_{1}, u_{2}, \dots, u_{k}, \dots,
u_{2E}\}$. Дуги $\{u_{k}, u_{k+E}\}$ определяют прямое и~обратное
на\-прав\-ле\-ние передачи потока по реб\-ру~$r_{k}$ между концевыми
вершинами $\{v_{n_k}, v_{j_k}\}$.

В многопользовательской сети~$G(\mathbf{d})$ рассматривается
$M\hm=N(N\hm-1)$ независимых, невзаимозаменяемых и~равноправных потоков
различных видов, которые передаются между уз\-ла\-ми-кор\-рес\-пон\-ден\-та\-ми
из множества 
$$
P=\left\{p_{1}, p_{2}, \dots, p_{M}\right\}.
$$

По определению, каждой паре уз\-лов-кор\-рес\-пон\-ден\-тов~$p_{m}$
соответствуют:
\begin{itemize}
\item вершина-ис\-точ\-ник с~номером~$s_{m}$, через которую входной поток
$m$-го вида~$z_{m}$ поступает в~сеть;
\item  вершина-при\-ем\-ник с~номером~$t_{m}$, из которой поток $m$-го
вида~$z_{m}$ покидает сеть.
\end{itemize}

В множестве~$P$ выделяется подмножество $P(R^{+})$ пар
уз\-лов-кор\-рес\-пон\-ден\-тов, расположенных в~концевых вершинах
ребра~$r_{k}$, $k\hm=\overline{1,E}$. Вводятся сле\-ду\-ющие обозначения:
пусть ребро~$r_{k}$  с~номером~$k$ соединяет вершины с~номерами~$n$ и~$j$ такими, что $n\hm< j$. Для со\-от\-вет\-ст\-ву\-ющей пары
уз\-лов-кор\-рес\-пон\-ден\-тов~$p_{k}$, расположенных в~узлах $\{v_{n},
v_{j}\}$, узел~$v_{n}$ считается источником, а узел~$v_{j}$~---
приемником потока $z_{k}$ $k$-го вида, который передается из узла
c номером~$n$ в~узел с~номером~$j$ для пары~$p_{k}$ с~номером~$k$.
Для пары $p^{}_{k+E} \Longleftrightarrow \{v_{j},v_{n}\}$ узел~$v_{j}$ 
считается источником~$s_{k+E}$, а~узел $v_m$~--- приемником~$t_{k+E}$ для пары с~номером~$p_{k+E}$. Формируется
$R^+\hm=\{1,2,3,\dots,E,E+1,\dots,2E\}$~--- список номеров смежных
пар.

Пары $p_{k}$ из подмножества~$P(R^{+})$ называются
\textit{смежными} уз\-ла\-ми-кор\-рес\-пон\-ден\-та\-ми. Все остальные
\textit{несмежные} пары уз\-лов-кор\-рес\-пон\-ден\-тов относятся к~множеству~$P(R^{-})$:
\begin{equation*}
P=P(R^{+})\cup P(R^{-});\quad
P(R^{+}) \cap P(R^{-}) = \emptyset.
\end{equation*}

Введем обозначения:
\begin{description}
\item[\,]
$z_{m}$~--- величина \textit{межузлового} потока $m$-го вида,
который поступает в~сеть из узла с~номером~$s_{m }$ и~покидает из
узла с~номером~$t_{m}$;
\item[\,]
$S(v_{n})$~--- множество номеров исходящих дуг, по которым поток
покидает узел~$v_{n}$;
\item[\,]
$T(v_{n})$~--- множество номеров входящих дуг, по которым поток
поступает в~узел~$v_{n}$.
\end{description}

Во всех узлах $v_{n}\in V$, $n\hm=\overline{1,N}$, для всех видов
потоков должны выполняться условия сохранения потоков:
\begin{multline}
\label{eq1} 
\sum\limits_{i\in S(v_n )} x_{mi}-\sum\limits_{i\in T(v_n )} x_{mi}
={}\\
{}=\begin{cases}
z_m, & \mbox{если } v=v^{}_{S_m}; \\
-z_m,&\mbox{если } v=v_{t_m}; \\
0&\mbox{в остальных случаях}, \\
\end{cases}
\end{multline}
$n=\overline{1,N}$, $m\hm=\overline{1,M}$, $x_{mi}\hm\ge 0$,
$z_{m}\hm\ge0$.

Величина {z}$_{m}$ равна входному потоку $m$-го вида, который
пропускается от источника к~приемнику пары $p_{m}$ при
распределении потоков $x_{mi}$ по дугам сети.

Каждому ребру $r_{k}\hm\in R$ приписывается неотрицательное число~$d_{k}$, 
определяющее суммарный предельно допустимый поток,
который можно передать по реб\-ру~$r_{k}$ в~обоих на\-прав\-ле\-ни\-ях. 
В~исходной сети компоненты вектора про\-пуск\-ных способностей
$\mathbf{d}\hm=(d_{1}, d_{2},\dots, d_{k}, \dots, d_{E})$~--- наперед
заданные положительные числа $d_{k}
\hm> 0$. Вектором $\mathbf{d}$ определяются сле\-ду\-ющие ограничения на сумму
дуговых потоков всех видов, пе\-ре\-да\-ва\-емых по реб\-ру~$r_{k}$:
\begin{multline}
\sum\limits_{m=1}^M (x_{mk}+x_{m(k+E)}) \le d_k,\\
 x_{mk}\ge 0\,,\enskip
 x_{m(k+E)}\ge 0\,, \enskip k=\overline {1,E}\,.
 \label{eq2} 
\end{multline}
В рамках данной модели пропускная спо\-соб\-ность ребер сети~--- вектор~$\mathbf{d}$~--- трактуется как <<\textit{ресурсное ограничение}>>,
а~сумма дуговых
 потоков рас\-смат\-ри\-ва\-ет\-ся как показатель использования
<<\textit{ресурсов}>> сети при передаче межузловых потоков
различных видов.

Для всех $z_{m}$ и~$x_{mi}$, удовлетворяющих
условиям~\eqref{eq1} и~\eqref{eq2}, вычисляются суммарные потоки:
\begin{equation}
 y_{m }=\sum\limits_{i=1}^{2E} {x}_{mi},\quad
m=\overline{1,M}\,.
\label{eq3}
\end{equation}

Суммарный реберный поток~$y_{m}$ характеризует
<<\textit{нагрузку}>> на сеть при передаче межузлового потока
величины $z_{m}$ из уз\-ла-ис\-точ\-ни\-ка~$s_{m}$ в~узел-при\-ем\-ник~$t_{m}$. 
Величина~$y_{m}$ показывает, какой суммарный
\textit{ресурс}~-- пропускная спо\-соб\-ность сети~-- требуется для
передачи межузлового потока~$z_{m}$, а~отношение
$w_{m}\hm={y_m}/{z_m}$,  $m\hm=\overline{1,M},$
показывает, какие \textit{ресурсы} необходимы для передачи
единичного потока $m$-го вида между узлами~$s_{m}$ и~$t_{m}$.

Ограничения~\eqref{eq1}--\eqref{eq3} задают подмножество
допустимых значений компонент вектора межузловых потоков
$\mathbf{z}\hm=\left(z_{1}, z_{2},\dots,z_{m},\dots,z_{M}\right)$:
\begin{equation*}
{Z}(\mathbf{d})=\left\{\mathbf{z} \ge 0 \mid
(\mathbf{z},\mathbf{x},\mathbf{y}) \ \mbox{удовлетворяют~\eqref{eq1}--\eqref{eq3}}
\right\}\!,
\!\!
%\label{eq4} 
\end{equation*}
а все допустимые распределения ресурсов принадлежат подмножеству
\begin{equation*}
{Y}(\mathbf{d})=\left\{\mathbf{y} \ge 0 \mid
(\mathbf{z},\mathbf{x},\mathbf{y}) \ \mbox{удовлетворяют~\eqref{eq1}--\eqref{eq3}}\right\}\!.
%\!\!\!\label{eq5}
\end{equation*}


\section{Метрические оценки предельных распределений}

Для оценки функциональных возможностей сис\-те\-мы рассматриваются
допустимые распределения межузловых потоков при предельных
загрузках ребер сети.

В рамках данного модельного описания монопольным режимом
называется способ управления, при котором все ресурсы сети
используются для передачи потока одной выделенной пары
уз\-лов-кор\-рес\-пон\-ден\-тов $p_{a}\hm\in P(R^-)$, а для всех
остальных потоки полагаются равными нулю.

Предельно допустимый поток, который можно передать между
фиксированной парой уз\-лов-кор\-рес\-пон\-ден\-тов $p_{a}$ в~монопольном
режиме, является решением стандартной, в~данном случае
однопродуктовой, задачи о~максимальном потоке.

\smallskip

\noindent
\textbf{Задача 1.} Найти
$z_a^0\hm=\max\limits_{\langle z,x\rangle \in Z(d)} z_a
$
при условии $z_{i}=0$, $i\hm=\overline{1,M}$, $i\hm\ne a$.

При решении задачи~1 для пары $p_{a}$ вы\-чис\-ля\-ют\-ся: межузловой
поток~$z_a^0$; дуговые потоки $\{x^{0}_{ak};x^{0}_{a(k+E)}\}$,
$k\hm=\overline{1,E}$; суммарное значение реберного
потока~$y_{a}^{0}\hm=\sum\nolimits_{i=1}^{2E} {x}_{ai}^{0}$.

Поток величины $z_a^0$ является \textit{максимальным потоком},
пе\-ре\-да\-ва\-емым в~\textit{монопольном режиме} для пары
уз\-лов-кор\-рес\-пон\-ден\-тов~$p_{a}$, $p_{a}\hm\in P(R^-)$, в~сети~$G(d)$.

Задача~1 решается последовательно для всех $p_{m}\in P(R^-)$,
вы\-чис\-ля\-ют\-ся значения $z_{m}^{0}(t)$.

При проведении вычислительных экспериментов использовалась
итерационная процедура для оценки функциональных возможностей
сис\-те\-мы при передаче межузловых потоков по нескольким маршрутам.
На предварительном этапе шага~$t$ в~сети~$G(t)$ при заданных
значениях пропускной спо\-соб\-ности ребер~$d_k(t)$ для каждой \mbox{пары}
уз\-лов-кор\-рес\-пон\-ден\-тов $p_m\hm\in P(R^-)$ определяется максимально
допустимый однопродуктовый поток~$z^0_m(t)$, со\-от\-вет\-ст\-ву\-ющие
дуговые потоки $(x_{mk}^0(t),x_{m(k+E)}^0(t))$, $p_m\hm\in P(R^-)$, и~коэффициенты нормировки
$\xi_m^0(t)\hm={1}/{z_m^0(t)}$ для всех  $p_m\hm \in P(R^-)$,
таких что $z^0_m(t)\hm>0$, $y_m^0(t)\hm>0$.
Коэффициенты~$\xi_m^0(t)$ используются для поиска текущих
совместно допустимых квот на передачу потоков одновременно между
всеми парами $p_m\in P(R^-)$.

\smallskip

\noindent
\textbf{Задача 2.} Найти $\alpha^*(t)=\max\limits_\alpha \alpha$
при условиях
$$
\alpha\!\!\sum\limits_{m\in R^-}\! \xi_m^0\left(x_{mk}^0(t)+x_{m(k+E)}^0(t)\right)\le d_k(t),\enskip
k=\overline{1,E}\,.
$$

На основании $\alpha^*(t)$ вычисляются совместно допустимые
дуговые потоки:
\begin{multline*}
x_{mk}^*(t)=\alpha^*(t)\xi^0_m(t)x^0_{mk}(t),\\
x^*_{m(k+E)}(t)=\alpha^*(t)\xi^0_m(t)x^0_{m(k+E)}(t),
\\
m=\overline{1,M}\,,\enskip k=\overline{1,E}\,,
\end{multline*}
и остаточная пропускная способность ребер в~сети $G(t+1)$:
\begin{multline*}
d_k(t+1)=d_k(t)-\sum_{m\in R^-} (x_{mk}^*(t)+x_{m(k+E)}(t)),\\
k=\overline{1,E}\,,\enskip p_m\in P(R^-).
\end{multline*}
Формируется вектор допустимых межузловых потоков:
\begin{align*}
z_k^+(t)&=d_k(t+1),\enskip p_k\in P(R^+),\enskip k=\overline{1,E}\,;
\\
z_m^-(t)&=\sum\limits_{\tau=1}^t\alpha^*(\tau)\xi_m^0(\tau) z_m^0(\tau), \enskip p_m\in P(R^-).
\end{align*}

По построению, на шаге~$t$ при передаче вектора межузлового потока
$\mathbf{z}(t)=\{\mathbf{z}^+(t), \mathbf{z}^-(t)\}$ достигается
предельная загрузка, и~пропускная способность всех ребер  сети
используется полностью.

Точка с~координатами $\mathbf{z}(t)$ принадлежит множеству~$Z(d)$.

Расстояние точки от начала координат определяется как норма
соответствующего вектора:
\begin{align*}
\rho^+(t)&=\|\mathbf{z}^+(t)\|=
\left[\,\sum\limits_{k=1}(\mathbf{z}^+(t))^2\right]^{1/2};
\\
\rho^-(t)&=\|\mathbf{z}^-(t)\|= \left[\sum\limits_{p_m\in P(R^-)}(\mathbf{z}_m^-(t))^2\right]^{1/2}.
\end{align*}

Если при выполнении шага $(t+1)$ окажется, что $z_m^0(t+1)=0$ для
всех $p_m\in P(R^-)$, то про\-изойдет останов и~сформируются
массивы финальных данных:
\begin{align*}
z_m^-(T)&=\sum\limits_{\tau=1}^t \alpha^*(\tau)\xi_m^0(\tau) z_m^0(\tau),\enskip 
p_m\in P(R^-),\\
z_k^+(T)&=d_k(t+1),\enskip p_k\in P(R^+),\enskip k=\overline{1,E}\,.
\end{align*}

Вышеописанная вычислительная процедура далее обозначается как
MFPL-про\-це\-ду\-ра (от англ.\ \textit{max-flow-peak-load}).

При проведении второй серии вычислительных экспериментов
MFPL-про\-це\-ду\-ра использовалась для оценки функциональных
характеристик сис\-те\-мы при \textit{уравнительном} поэтапном
распределении пропускной способности между всеми
па\-ра\-ми-кор\-рес\-пон\-ден\-тами.

При реализации MFPL-процедуры выполнение каждого шага разбивается
на несколько этапов. На предварительном этапе шага~$t$ 
в~сети~$G(t)$ при заданных значениях пропускной способности ребер~$d_k(t)$ 
для каждой пары уз\-лов-кор\-рес\-пон\-ден\-тов $p_m\hm\in P(R^-)$
определяется максимально допустимый однопродуктовый
поток~$z_m^0(t)$, соответствующие дуговые потоки
$\left(x_{mk}^0(t),x_{m(k+E)}^0(t)\right)$, $p_m\hm\in P(R^-)$, и~суммарная
реберная нагрузка
$$
y_m^0(t)=\sum\limits_{k=1}^E (x_{mk}^0(t),x_{m(k+E)}^0(t)),\enskip p_m\in P(R^-).
$$

Для каждой пары $p_m\hm\in P(R^-)$ вычисляются коэффициенты
нормировки
$\theta_m^0(t)\hm={1}/{y_m^0(t)}$ для всех  
$p_m\hm\in P(R^-)$, таких что  $z^0_m(t)\hm>0$,
$y_m^0(t)\hm>0$.
Коэффициенты~$\theta_m^0(t)$ используются для поиска совместно
допустимых дуговых потоков для всех $p_m\hm\in P(R^-)$.

\smallskip

\noindent
\textbf{Задача 3.} Найти $\beta^*(t)=\max\nolimits_\beta \beta$ при
условиях
$$
\beta\!\!\!\!\sum\limits_{p_m\in P(R^-)}\!\!
\theta_m^0(x_{mk}^0(t)+x_{m(k+E)}^0(t))\le d_k(t),\enskip
k=\overline{1,E}\,.
$$

 С помощью $\beta^*(t)$ (решения задачи~3) вычисляются текущие допустимые значения дуговых потоков:
\begin{multline*}
x_{mk}^*(t)=\beta^*(t)\theta^0_m(t)x^0_{mk}(t),\\
x^*_{m(k+E)}(t)=\beta^*(t)\theta^0_m(t)x^0_{m(k+E)}(t), \enskip
k=\overline{1,E},
\end{multline*}
и реберных нагрузок при одновременной передаче межузловых потоков:

\noindent
\begin{multline*}
y_m^*(t)=\sum\limits_{i=1}^E
\left[x_{mi}^*(t)+x^*_{m(i+E)}(t)\right]={}\\
{}= \fr{\beta^*(t)}{y_m^0(t)} \sum\limits_{i=1}^E
\left[x_{mi}^0(t)+x^0_{m(i+E)}(t)\right]=\beta^*(t), \\
 p_m\in P(R^-).
\end{multline*}
Таким образом на каждом шаге определенная часть имеющегося ресурса
(пропускной спо\-соб\-ности) делится строго по\-ров\-ну меж\-ду всеми
корреспондентами $p_m\in P(R^-)$, для которых существует путь
передачи в~$G(t)$.

Формируется вектор допустимых межузловых потоков:
\begin{gather*}
\hspace*{-30mm}z_k^{++}(t)=d_k(t+1)={}\hspace*{10mm}\\
{}=d_k(t)-\!\!\! \sum\limits_{p_m\in P(R^-)}\!\!\!
\left(x_{mk}^*(t)+x_{m(k+E)}(t)\right),\\
\hspace*{35mm}k=\overline{1,E}, \enskip
p_k\in P(R^+);\\
z_m^{(=)}(t)\overset{\Delta}{=}\sum\limits_{\tau=1}^t\beta^*(\tau)
\theta_m^0(\tau) z_m^0(\tau), \enskip p_m\in P(R^-).
\end{gather*}

\noindent
Определяются расстояния:
\begin{align*}
\rho^{++}(t)&=\|\mathbf{z}^{++}(t)\|\overset{\Delta}{=}
\left[\sum\limits_{k=1}^E\left(d_k(t+1)\right)^2\right]^{1/2};\\
\rho^{(=)}(t)&=\|\mathbf{z}^{=}(t)\|= \left[\sum\limits_{p_m\in
P(R^-)}\left(z_m^{(=)}(t)\right)^2\right]^{1/2}.
\end{align*}

Если на предварительном этапе на шаге $(t+1)$ окажется, что в~сети~$G(t+1)$ для всех $p_m\hm\in P(R^-)$ все значения
$z_m^0(t+1)\hm=0$, то произойдет останов и~сформируются финальные
массивы:
\begin{align*}
z_k^{(++)}(T)&=d_k(t+1), \enskip
p_k\in P(R^+), \enskip k=\overline{1,E};
\\
z_m^{(=)}(t)&=\sum\limits_{\tau=1}^{t+1}\beta^*(\tau)
\theta_m^0(\tau) z_m^0(\tau), \enskip p_m\in P(R^-).
\end{align*}



\section{Вычислительный эксперимент}

Результаты вычислительных экспериментов, описанные ниже, служат
продолжением исследований, начатых в~[1]. Вычислительные
эксперименты проводились на моделях сетевых сис\-тем, пред\-став\-лен\-ных
на рис.~1 и~2. В~каждой сети~69~узлов. Пропускные спо\-соб\-но\-сти
ребер~-- значения $d_k$~-- выбирались случайным образом из отрезка
$[900,999]$ и~совпадали для ребер, при\-сут\-ст\-ву\-ющих в~обеих сетях.
В~кольцевой сети пропускная спо\-соб\-ность каждого из добавленных
ребер равнялась~900.

\begin{figure*} %fig1
\vspace*{1pt}
\begin{minipage}[t]{80mm}
  \begin{center}  
    \mbox{%
\epsfxsize=69.408mm
\epsfbox{mal-1.eps}
}

\end{center}
\vspace*{-6pt}
\Caption{Базовая сеть}
\end{minipage}
%\end{figure*}
\hfill
%\begin{figure*} %fig2
\vspace*{1pt}
\begin{minipage}[t]{80mm}
  \begin{center}  
    \mbox{%
\epsfxsize=69.408mm
\epsfbox{mal-2.eps}
}

\end{center}
\vspace*{-6pt}
\Caption{Кольцевая сеть}
\end{minipage}
\end{figure*}

\begin{table*}[b]\small %tabl1
\vspace*{-12pt}
\begin{center}

%\renewcommand{\arraystretch}{1.1}
\Caption{Базовая сеть}
\vspace*{2ex}

\begin{tabular}{|c||c|c|c||c|c|c|} 
\hline
&&&&&&\\[-9pt]
$t$  & $\rho^{-}(t)$ & $\rho^{+}(t)$ & $d^{+}(t+1)$ &
$\rho^{=}(t)$ & $\rho^{++}(t)$&  $d^{++}(t+1)$ \\ 
\hline
\hphantom{99}0  & \hphantom{99}0   & 8048&  68256&  \hphantom{9}0   &  8048&   68256\\
1  & \hphantom{9}63  & 4182&  26544&  \hphantom{9}95  &  3881&   24476\\
$\cdots$  & $\cdots$   & $\cdots$   &  $\cdots$    &  $\cdots$   &  $\cdots$   &   $\cdots$\\
11 & \hphantom{9}70  & 3975&  21469&  \hphantom{9}101\hphantom{9} &  3707&   20155\\
$\cdots$& $\cdots$   & $\cdots$   &  $\cdots$    & $\cdots$   &  $\cdots$   &  $\cdots$\\
22 & \hphantom{9}83  & 3861&  19623&  \hphantom{9}122\hphantom{9} &  3586&   18260\\
$\cdots$ & $\cdots$  & $\cdots$   &  $\cdots$   &  $\cdots$   &  $\cdots$  &   $\cdots$\\
33 & \hphantom{9}103\hphantom{9} & 3778&  18827&  \hphantom{9}139\hphantom{9} &  3522&   17601\\
$\cdots$ &$\cdots$  &$\cdots$  & $\cdots$  & $\cdots$   &  $\cdots$  &  $\cdots$\\
44 & \hphantom{9}\bf 190\hphantom{9} & \bf3553&  \bf17503&  \hphantom{9}\bf203\hphantom{9} &  \bf3285&   \bf16201\\
45 & \hphantom{9}\bf1452\hphantom{99}& \bf2166&  \hphantom{9}\bf7069 &  \hphantom{9}\bf1376\hphantom{99}&  \bf2020&   \hphantom{9}\bf6584\\
46 & \hphantom{9}\bf1498\hphantom{99}& \bf2158&  \hphantom{9}\bf6707 &  \hphantom{9}\bf1388\hphantom{99}&  \bf2017&   \hphantom{9}\bf6483\\
$\cdots$ & $\cdots$   & $\cdots$   &  $\cdots$    & $\cdots$   &  $\cdots$   &  $\cdots$\\
52 & \hphantom{9}1535\hphantom{99}& 2155&  \hphantom{9}6413 & \hphantom{9}1442\hphantom{99} &  2011&   \hphantom{9}6059\\
\hline
\end{tabular}
\end{center}
 %\end{table*}
% \begin{table*}\small %tabl2
\begin{center}
\Caption{Кольцевая сеть}
\vspace*{2ex}


\begin{tabular}{|c||c|c|c||c|c|c|} 
\hline
&&&&&&\\[-9pt]
$t$  & $\rho^{-}(t)$ & $\rho^{+}(t)$ & $d^{+}(t+1)$ &
$\rho^{=}(t)$ & $\rho^{++}(t)$&  $d^{++}(t+1)$ \\
 \hline
\hphantom{9}0  &\hphantom{99}0    & 8440  & 75456   &\hphantom{9}0      &8440   &75456\\
\hphantom{9}1  &\hphantom{9}68   & 5317  & 43038   &92     &5045   &40716 \\ 
$\cdots$ &$\cdots$    & $\cdots$     & $\cdots$   &$\cdots$      &$\cdots$      &$\cdots$      \\
11 &\hphantom{9}95   & 3608  & 20459   &124    &3397   &19080  \\
$\cdots$ &$\cdots$   & $\cdots$    & $\cdots$      &$\cdots$     &$\cdots$     &$\cdots$   \\
22 &\hphantom{9}101\hphantom{9}  & 3540  & 19530   &130    &3350   &18338 \\
$\cdots$ &$\cdots$  & $\cdots$   &$\cdots$      &$\cdots$     &$\cdots$   &$\cdots$    \\
33 &\hphantom{9}135\hphantom{9}  & 3346  & 17561   &154    &3220   &17003 \\
$\cdots$  &$\cdots$   & $\cdots$    & $\cdots$      &$\cdots$     &$\cdots$    &$\cdots$    \\
44 &\hphantom{9}234\hphantom{9}  & 3094  & 14881   &269    &2918   &13848 \\
$\cdots$ &$\cdots$   & $\cdots$    &$\cdots$      &$\cdots$     &$\cdots$     &$\cdots$    \\
50 &\hphantom{9}\bf 413\hphantom{9}  & \bf2770  & \bf12901   &\bf329    &\bf2792   &\bf13079 \\
51 &\hphantom{9}\bf1040\hphantom{99} & \bf2299  & \hphantom{9}\bf8801    &\bf334    &\bf2784   &\bf13034 \\
52 &\hphantom{9}\bf1062\hphantom{99} & \bf2297  & \hphantom{9}\bf8672    &\bf974    &\bf2262   &\hphantom{9}\bf8768  \\
$\cdots$ &$\cdots$   &$\cdots$    & $\cdots$      &$\cdots$      &$\cdots$     &$\cdots$    \\
55 &\hphantom{9}1069\hphantom{99} & 2297  & \hphantom{9}8630    &1010\hphantom{9}   &2259   &\hphantom{9}8553  \\
\hline
 \end{tabular}
\end{center}
 \end{table*}




Для базовой сети исходная сумма пропускных способностей:
$D^+(0)\hm=68\,256$, а~для кольцевой сети $D^{++}(0)=75\,456$.
Соответствующие значения $\rho^+(0)$ и~$\rho^{++}(0)$ указаны в~<<нулевой>> строке 
в~табл.~1 и~2, где собраны результаты
вычислительных экспериментов. В~ходе эксперимента при
уравнительном распределении остаточных ресурсов соблюдается
\textit{равномерное} убывание остаточной пропускной спо\-соб\-ности и~<<\textit{длины}>> вектора~$\rho^+(t)$. 
Однако между 44--46
итерациями для базовой и~50--52 для кольцевой сети наблюдается
резкий скачок величин~$\rho^-(t)$, $\rho^{=}(t)$ и~$d^+(t)$,
$d^{++}(t)$.

На указанных шагах полностью используется пропускная способность
ребер в~центральной час\-ти сети. Сеть \textit{распадается} на
несвязные компоненты, и~для $80\%$ корреспондентов пропадают пути
соединения, а~остаточный ресурс распределяется поровну между
оставшимися парами узлов.

Анализ результатов показал, что почти равные значения потоков
достигаются для~80\% корреспондентов и~требуют 60\%--70\%
ресурсов. Однако для~2\% смежных  пар узлов межузловые потоки на
два порядка выше медианных значений, а~затраты пропускной
способности  со\-став\-ля\-ют~20\%--30\%.








\section{Заключение}

Предложенный метод и~проведенные вычислительные эксперименты
показали, что уравнительное поэтапное распределение   приводит 
к~неравномерному  распределению   потоков  для разных групп\linebreak
корреспондентов.    Метрические оценки, полученные  в~ходе
экспериментов, продемонстрировали\linebreak \textit{деформацию} множества
достижимых потоков. В~рамках модели   предполагалось, что  все
корреспонденты  равноправны, а~потоки невзаимозаменяемы,  однако
при уравнительном предельном  распределении  смежные  пары узлов
оказывались в~привилегированном положении при использовании
остаточной пропускной способности. Пропускные способности  ребер
рассматривались  как вектор   ресурсов  различных типов,  которые
распределяются между корреспондентами   при передаче  потоков
различных видов.  По построению, на каж\-дом шаге норма вектора
смежных   межузловых    потоков численно равна   модулю вектора
остаточных  пропускных способностей.   Полученные мет\-ри\-че\-ские
значения  можно использовать  для   оценки функциональных
возможностей сети  в~режиме  предельной загрузки.

{\small\frenchspacing
 {%\baselineskip=10.8pt
 %\addcontentsline{toc}{section}{References}
 \begin{thebibliography}{9}

\bibitem{1-mal}
\Au{Малашенко Ю.\,Е., Назарова И.\,А.} Неоднородность
распределения   потоков при предельной  загрузке
многопользовательской сети~//  Известия РАН. Теория и~сис\-те\-мы
управления,  2022. №\,3. С.~81--96.

\bibitem{4-mal} %2
\Au{Luss H.} Equitable resource allocation: Models,
algorithms, and applications.~--- Hoboken, NJ, USA: John Wiley \& Sons, 2012.
420~p.

\bibitem{2-mal} %3
\Au{Ogryczak W., Luss~H., Pioro~M., Nace~D., Tomaszewski~A.}   Fair
optimization and networks: A~aurvey~// J.~Appl. Math., 2014. Vol.~2014. Art.~ID~612018. 25~p. doi: 10.1155/ 2014/612018.

\bibitem{3-mal} %4
\Au{Salimifard K., Bigharaz~S.} The multicommodity network
flow problem: State of the art classification, applications, and
solution methods~// J.~Oper. Res., 2020. Vol.~18. Iss.~3. P.~1--47.



\bibitem{5-mal}
\Au{Balakrishnan A., Li~G., Mirchandani~P.}  Optimal
network design with end-to-end service requirements~// Oper. Res.,
2017. Vol.~65. Iss.~3. P.~729--750.

\bibitem{6-mal}
\Au{Nace D., Doan~L.\,N., Klopfenstein~O., Bashllari~A.} Max-min
fairness in multicommodity flows~// Comput. Oper. Res., 2008.
Vol.~35. Iss.~2. P.~557--573.

\bibitem{7-mal}
\Au{Ros-Giralt J., Tsai~W.\,K.} A~lexicographic optimization
framework to the flow control problem~// IEEE T.
Inform. Theory, 2010. Vol.~56. Iss.~6. P.~2875--2886.

\bibitem{8-mal}
\Au{Baier G., Kohler~E., Skutella~M.}  The \mbox{k-splittable}
flow problem~//  Algorithmica, 2005. Vol.~42. Iss.~3-4.
P.~231--248.

\bibitem{9-mal}
\Au{Bialon P.\,A.} Randomized rounding approach to 
a~\mbox{k-splittable} multicommodity flow problem with lower path flow
bounds affording solution quality guarantees~// Telecommun. Syst.,
2017. Vol.~64. Iss.~3. P.~525--542.
\end{thebibliography}

 }
 }

\end{multicols}

\vspace*{-6pt}

\hfill{\small\textit{Поступила в~редакцию 10.06.22}}

\vspace*{8pt}

%\pagebreak

%\newpage

%\vspace*{-28pt}

\hrule

\vspace*{2pt}

\hrule

%\vspace*{-2pt}

\def\tit{SEQUENTIAL ANALYSIS AND METRIC ESTIMATES\\ OF~PEAK LOAD FLOWS IN~THE~MULTIUSER NETWORK}


\def\titkol{Sequential analysis and metric estimates of~peak load flows in~the~multiuser network}


\def\aut{Yu.\,E.~Malashenko}

\def\autkol{Yu.\,E.~Malashenko}

\titel{\tit}{\aut}{\autkol}{\titkol}

\vspace*{-8pt}


\noindent
Federal Research Center ``Computer Science and Control'' of the Russian Academy of Sciences, 
44-2~Vavilov Str., Moscow 119333, Russian Federation



\def\leftfootline{\small{\textbf{\thepage}
\hfill INFORMATIKA I EE PRIMENENIYA~--- INFORMATICS AND
APPLICATIONS\ \ \ 2022\ \ \ volume~16\ \ \ issue\ 3}
}%
 \def\rightfootline{\small{INFORMATIKA I EE PRIMENENIYA~---
INFORMATICS AND APPLICATIONS\ \ \ 2022\ \ \ volume~16\ \ \ issue\ 3
\hfill \textbf{\thepage}}}

\vspace*{3pt} 



\Abste{The set of vectors of internodal flows in a~multiuser communication network under peak load is analyzed. Within the framework of
 the multicommodity model, the throughput capacities of edges are considered as the components of a~vector of resources of various types that 
 are required for the transmission of various kinds of
 flows. When conducting computational experiments, at each iteration, the
  norms of vectors of jointly permissible internodal flows are calculated, during the transmission of which the capacity of 
  all network edges is fully used.\linebreak\vspace*{-12pt}}
 
 \Abstend{The proposed method and computational experiments have shown that the equalizing phased 
  distribution leads to an uneven distribution of flows for different groups of correspondents. Metric values obtained during experiments 
  indicate deformation of the sets of accessible flows. Within the framework of the model, all correspondents are tantamount 
  and the flows are noninterchangeable; however, in the case of an equalizing peak load distribution, adjacent pairs 
  of nodes are in a privileged position when using residual capacity. The obtained metric values can be used to 
  evaluate the functional characteristics of the transmission network in the finite capacity loading mode.}

\KWE{multicommodity flow network model; set of achievable internodal flows; peak load distribution}


\DOI{10.14357/19922264220306} 

%\vspace*{-16pt}

%\Ack
%\noindent



%\vspace*{4pt}

  \begin{multicols}{2}

\renewcommand{\bibname}{\protect\rmfamily References}
%\renewcommand{\bibname}{\large\protect\rm References}

{\small\frenchspacing
 {%\baselineskip=10.8pt
 \addcontentsline{toc}{section}{References}
 \begin{thebibliography}{9}
\bibitem{1-mal-1}
\Aue{Malashenko, Yu.\,E., and I.\,A.~Nazarova.}
2022. Heterogeneous flow distribution at the peak load in the multiuser network. \textit{J.~Comput. Sys. Sc. Int.} 61:372--387.

\bibitem{4-mal-1} %2
\Aue{Luss, H.} 2012. \textit{Equitable resource allocation: Models, algorithms, and applications}.
Hoboken, NJ: John Wiley \& Sons. 420~p.

\bibitem{2-mal-1} %3
\Aue{Ogryczak, W., H.~Luss, M.~Pioro, D.~Nace, and A.~Tomaszewski.}
 2014. Fair optimization and networks: A~survey. \textit{J.~Appl. Math.} 2014:612018. 25~p. doi: 10.1155/ 2014/612018.
\bibitem{3-mal-1} %4
\Aue{Salimifard, K., and S.~Bigharaz.}
 2020. The multicommodity network flow problem: State of the art classification, applications, and solution methods. 
 \textit{J.~Oper. Res.} 18(3):\linebreak 1--47.

\bibitem{5-mal-1}
\Aue{Balakrishnan, A., G.~Li, and P.~Mirchandani.} 2017. Optimal network design with end-to-end service requirements. 
\textit{Oper. Res.} 65(3):729--750.
\bibitem{6-mal-1}
\Aue{Nace, D., L.\,N.~Doan, O.~Klopfenstein, and A.~Bashllari.} 2008. Max-min fairness in multicommodity flows. 
\textit{Comput. Oper. Res.} 35(2):557--573.
\bibitem{7-mal-1}
\Aue{Ros-Giralt, J., and W.\,K.~Tsai.} 2010. A~lexicographic optimization framework to the flow control problem. 
\textit{IEEE T.~Inform. Theory} 56(6):2875--2886.
\bibitem{8-mal-1}
\Aue{Baier, G., E.~Kohler, and M.~Skutella.}
 2005. The k-splittable flow problem. \textit{Algorithmica} 42(3-4):231--248.
\bibitem{9-mal-1}
\Aue{Bialon, P.} 2017. A~randomized rounding approach to a~\mbox{k-splittable} multicommodity flow problem with lower path flow bounds affording solution quality guarantees. 
\textit{Telecommun. Syst.} 64(3):525--542.
 \end{thebibliography}

 }
 }

\end{multicols}

\vspace*{-6pt}

\hfill{\small\textit{Received June 10, 2022}}

\Contrl

\noindent
\textbf{Malashenko Yuri E.} (b.\ 1946)~--- 
Doctor of Science in physics and mathematics, principal scientist, Federal Research Center ``Computer Science and Control'' 
of the Russian Academy of Sciences, 44-2~Vavilov Str., Moscow 119333, Russian Federation; \mbox{malash09@ccas.ru} 


\label{end\stat}

\renewcommand{\bibname}{\protect\rm Литература}        %12
\def\stat{bosov+stef}

\def\tit{УПРАВЛЕНИЕ ВЫХОДОМ СТОХАСТИЧЕСКОЙ ДИФФЕРЕНЦИАЛЬНОЙ СИСТЕМЫ 
ПО~КВАДРАТИЧНОМУ КРИТЕРИЮ. I.~ОПТИМАЛЬНОЕ РЕШЕНИЕ МЕТОДОМ 
ДИНАМИЧЕСКОГО ПРОГРАММИРОВАНИЯ$^*$}

\def\titkol{Управление выходом стохастической дифференциальной системы 
по~квадратичному критерию. I}
%.~Оптимальное решение методом 
%динамического программирования}

\def\aut{А.\,В.~Босов$^1$, А.\,И.~Стефанович$^2$}

\def\autkol{А.\,В.~Босов, А.\,И.~Стефанович}

\titel{\tit}{\aut}{\autkol}{\titkol}

\index{Босов А.\,В.}
\index{Стефанович А.\,И.}
\index{Bosov A.\,V.}
\index{Stefanovich A.\,I.}




{\renewcommand{\thefootnote}{\fnsymbol{footnote}} \footnotetext[1]
{Работа выполнена при частичной поддержке РФФИ (проект 16-07-00677).}}


\renewcommand{\thefootnote}{\arabic{footnote}}
\footnotetext[1]{Институт проблем информатики Федерального исследовательского центра <<Информатика 
и~управление>> Российской академии наук, \mbox{AVBosov@ipiran.ru}}
\footnotetext[2]{Институт проблем информатики Федерального исследовательского центра <<Информатика 
и~управление>> Российской академии наук, \mbox{AStefanovich@frccsc.ru}}

%\vspace*{8pt}



  
  \Abst{Решается задача оптимального управления для диффузионного процесса 
Ито и~линейного управ\-ля\-емо\-го выхода. Рассматриваемая постановка близка 
к~классической ли\-ней\-но-квад\-ра\-тич\-ной гауссовской задаче управления 
(linear-quadratic Gaussian (LQG) control). Отличия состоят в~том, что состояние описывается нелинейным 
дифференциальным уравнение Ито $dy_t\hm= A_t(y_t) \,dt\hm+ \Sigma_t(y_t)\,dv_t$ 
и~не зависит от управ\-ле\-ния~$u_t$, оптимизации подлежит управ\-ля\-емый 
линейный выход $dz_t\hm= a_t y_t\,dt\hm+ b_t z_t \,dt\hm+ c_t u_t \,dt\hm+ \sigma_t\, 
dw_t$. Дополнительные обобщения внесены в~квад\-ра\-тич\-ный критерий качества 
с~целью воз\-мож\-ности постановки таких задач, как отслеживание выходом 
состояния или управ\-ле\-ни\-ем~--- линейной комбинации состояния и~выхода. Для 
решения используется метод динамического программирования. Функцию 
Беллмана позволяет найти предположение о~ее структуре вида $V_t(y,z)\hm= 
\alpha_t z^2\hm+ \beta_t(y)z \hm+\gamma_t(y)$. Решение дают три 
дифференциальных уравнения для коэффициентов~$\alpha_t$, $\beta_t(y)$ 
и~$\gamma_t(y)$. Эти уравнения со\-став\-ля\-ют оптимальное решение 
рас\-смат\-ри\-ва\-емой задачи.}
  
  \KW{стохастическое дифференциальное уравнение; оптимальное управ\-ле\-ние; 
динамическое программирование; функция Беллмана; уравнение Риккати; 
линейные уравнения параболического типа}

\DOI{10.14357/19922264180314}
  
%\vspace*{4pt}


\vskip 10pt plus 9pt minus 6pt

\thispagestyle{headings}

\begin{multicols}{2}

\label{st\stat}

\section{Введение}

     Ключевые результаты в~области оптимизации стохастических 
динамических систем, со\-став\-ля\-ющие классическую теорию управления, 
получены более~40~лет назад (такова работа~[1] в~отношении задачи 
управ\-ле\-ния ли\-ней\-но-гаус\-сов\-ски\-ми стохастическими сис\-те\-ма\-ми по 
квад\-ра\-тич\-но\-му критерию). К~классической тео\-рии следует относить 
линейные модели стохастических сис\-тем и~квадратичный критерий качества. 
Это исходный базис, на котором основано множество успешно 
исследованных и~решенных задач стохастического управ\-ле\-ния 
и~оптимизации. 

Дальнейшее развитие~--- это новые модели и~критерии, но 
прежде всего это новые методы: от тео\-рии линейных регуляторов, метода 
динамического программирования и~принципа максимума к~адаптивному 
и~минимаксному подходу, импульсному управ\-ле\-нию и~т.\,д. Множество 
инноваций как в~час\-ти моделей, так и~в~час\-ти математического аппарата, 
имевших мес\-то в~по\-сле\-ду\-ющие годы, существенно обогатили тео\-рию 
управ\-ле\-ния. Но и~до настоящего времени линейные модели и~квадратичный 
критерий, несмотря на всю справедливую критику в~отношении их 
аде\-кват\-ности и~гиб\-кости, сохраняют исследовательский интерес и~находят 
современные области приложения.
     
     Не претендуя на сколь\-ко-ни\-будь полное обосно\-ва\-ние последнего 
тезиса, приведем несколько примеров, показавшихся наиболее ин\-те\-рес\-ными. 

Так, в~[2] решается ли\-ней\-но-квад\-ра\-тич\-ная за\-да\-ча в~игровой 
постановке с~запаздыванием. В~близ\-кой по модели работе~[3] задача 
управ\-ле\-ния ставится в~терминах $H_\infty$-ро\-баст\-ности. Точнее \mbox{называть} 
эту тематику $H_2/H_\infty$-управ\-ле\-ни\-ем, и~работ по этой теме очень 
много. Аккуратности ради следует уточнить, что под линейными 
понимаются модели с~мультипликативными по состоянию воз\-му\-ще\-ниями. 

Совсем другой класс моделей, особо популярных в~по\-след\-ние годы, 
составляют скачкообразные процессы. Например, линейные уравнения 
в~сочетании с~пуассоновскими скачками в~[4] используются в~моделях, 
описывающих различные показатели функционирования сетевых протоколов 
передачи данных транспортного уровня. Телекоммуникации представляют 
в~последние годы самый популярный прикладной материал для 
исследований, работ по этой проб\-ле\-ма\-ти\-ке множество, математические 
техники привлекаются самые разные и~самые современные, но и~линейным 
моделям место находится. Еще один любопытный пример исследования 
скачкообразного процесса и~оптимизации на основе квад\-ра\-тич\-но\-го критерия 
можно найти в~[5] применительно к~задаче инвестирования на финансовом 
рынке. Наконец, упомянем еще работу~[6], подводящую итог исследований 
в~отношении классической детерминированной  
ли\-ней\-но-квад\-ра\-тич\-ной задачи с~использованием техники матричных 
неравенств.
     
     В данной работе также эксплуатируются привлекательные свойства 
линейных моделей и~квад\-ра\-тич\-но\-го критерия, причем в~стохастической 
постановке. На\-прав\-ле\-ни\-ем для обобщения \mbox{выбрана} модель динамики 
сис\-те\-мы: основные усилия на\-прав\-ле\-ны на то, чтобы сделать ее нелинейной. 
Кроме того, пред\-став\-лен\-ная постановка может рас\-смат\-ри\-вать\-ся и~как 
обобщение ранее решенной задачи в~дискретном времени~[7, 8] на время 
непрерывное. В~упомянутых работах помимо собственно модельной 
постановки важна еще и~привлекаемая прикладная об\-ласть~--- 
функционирование сложных программных сис\-тем. Результатов, 
ориентированных непосредственно на такие приложения, к~настоящему 
времени пренебрежимо мало, поэтому~[7, 8]~--- это еще и~прикладное 
обоснование рас\-смат\-ри\-ва\-емой далее задачи.
     
     Оптимизируемая динамическая сис\-те\-ма описывается двумя 
уравнениями. Состояние задается нелинейным стохастическим 
дифференциальным уравнением Ито, не содержащим управ\-ля\-емой 
переменной. Возмущение здесь описывается стандартным винеровским 
процессом, накладываются простые условия существования 
и~един\-ст\-вен\-ности решения. Поскольку состояние не управ\-ля\-ет\-ся, то уместно 
его интерпретировать как слож\-ное внешнее возмущение. Вторая 
переменная~--- управ\-ля\-емый выход~--- задается линейным стохастическим 
дифференциальным уравнением. Цель оптимизации выхода формируется 
квадратичным критерием общего вида. Формальная постановка задачи 
приведена в~сле\-ду\-ющем разделе.
     
     Для решения задачи используется метод динамического 
программирования, решается уравнение Беллмана~[9]. Соответственно, 
в~результате получаются аналитические выражения и~для оптимального 
управ\-ле\-ния, и~для значения функционала качества. Технически 
традиционный, стандартный подход к~задаче обременен, пожалуй, 
единственной проблемой~--- поиском верного пред\-став\-ле\-ния структуры 
функции Беллмана. Справиться с~этой проблемой в~большей степени удается 
за счет результата, полученного при решении дискретного по времени 
аналога рассматриваемой постановки~\cite{8-bos}. Конечные соотношения 
для оптимального решения, как и~во всех подобных задачах, включая 
классическую ли\-ней\-но-квад\-ра\-тич\-ную, содержат решения 
определенных дифференциальных уравнений (обыкновенных и~в~частных 
производных). Вывод этих уравнений и~со\-став\-ля\-ет содержание первой час\-ти 
данной работы. Во второй части будет обсуждаться их приближенное 
чис\-лен\-ное решение и~компьютерные эксперименты.
     
     Кратко обозначим основные положения, при\-вле\-ка\-емые далее 
к~решению задачи, следуя в~основном обозначениям 
и~терминологии~\cite{9-bos}, а~именно: будем рассматривать задачу 
оптимального управления в~стохастической динамической сис\-те\-ме по полной 
информации, применяя метод динамического программирования. В~качестве 
целевого функционала, опре\-де\-ля\-юще\-го качество управ\-ле\-ния $U_0^T\hm= \{ 
u_t,\ 0\leq t\leq T\}$, выступает
     \begin{equation}
     J\left(U_0^T\right)={\sf E}\left\{ \int\limits_0^T L_t \left(x_t, u_t\right)\,dt+ 
l\left(x_T\right)\right\}\,.
     \label{e1-bos}
     \end{equation}
Здесь ${\sf E}\{\cdot\}$~--- оператор математического ожидания; $x_t$~--- 
случайный процесс, описываемый стохастическим дифференциальным 
уравнением Ито
     \begin{equation}
     dx_t=m_t\left( x_t, u_t\right) dt+ \sigma_t\left( x_t\right)dW_t\,,\enskip 
x_0=X\,,
     \label{e2-bos}
     \end{equation}
где $W_t$~--- стандартный винеровский процесс подходящей раз\-мер\-ности; 
$X$~--- случайный вектор.

     $U_0^T$ будем выбирать из класса допустимых неупреждающих (по 
отношению к~$W_t$) управлений~\cite{9-bos}. Соответственно, 
относительно функций сноса и~диффузии~$m_t$ и~$\sigma_t$  
в~(\ref{e2-bos}) будем предполагать выполненными ка\-кие-ли\-бо условия 
существования сильного решения для заданного до\-пус\-ти\-мо\-го управ\-ле\-ния. 
Например, для управ\-ле\-ния с~обратной связью $u_t\hm= u_t(x_t)$ будем 
считать, что $m_t(x,u_t(x))$ и~$\sigma_t(x)$ удовлетворяют условию 
линейного рос\-та и~локальному условию Липшица по~$x$ равномерно 
по~$t$ (т.\,е.\ условиям Ито).
     
     Для поиска оптимального управления, минимизирующего $J(U_0^T)$, 
рас\-смат\-ри\-ва\-ет\-ся функция Беллмана
     \begin{equation}
     V_t(x)=\left.\mathop{\mathrm{inf}}\limits_{U_t^T} {\sf E} \left\{ \int\limits_t^T 
L_t \left( x_t, u_t\right)\,dt+l\left( x_T\right) \right\vert \mathcal{F}_t^x\right\}\,,
     \label{e3-bos}
     \end{equation}
где $\mathcal{F}_t^x$~--- $\sigma$-ал\-геб\-ра, по\-рож\-ден\-ная~$x_\tau$, 
$0\hm\leq \tau\hm\leq t$, ${\sf E}\{\cdot\vert \mathcal{F}\}$~--- оператор условного 
математического ожидания относительно~$\mathcal{F}$. Соответственно, 
в~качестве достаточного условия оп\-ти\-маль\-ности воспользуемся уравнением 
динамического программирования
\begin{multline}
\fr{\partial V_t(x)}{\partial t} +\fr{1}{2}\sum\limits^n_{i,j=1} \sigma^2_{t_{ij}}
\fr{\partial^2 V_t(x)}{\partial x_i \partial x_j}+{}\\
{}+\min\limits_u\left[  
\sum\limits^n_{i=1} m_{t_i} \fr{\partial V_t(x)}{\partial x_i} + L_t(x,u)\right] 
=0\,,\\
V_T(x)=l(x)\,,
\label{e4-bos}
\end{multline}
где $m_{t_i}$~--- $i$-й элемент век\-тор-функ\-ции~$m_t(x,u)$; 
$\sigma^2_{t_{ij}} \hm= \sum\nolimits^m_{k=1} 
\sigma_{t_{ik}}\sigma_{t_{ki}}$, $\sigma_{t_{ij}}$~--- $i$-й по строке, $j$-й 
по столб\-цу элемент мат\-рич\-ной функции~$\sigma_t(x)$; $n$ и~$m$~--- 
размерности~$x_t$ и~$W_t$ соответственно.

     Традиционно в~рамках применения метода динамического 
программирования будем предполагать, что функции~$L_t$, $l$, $m_t$ 
и~$\sigma_t$ обеспечивают существование хотя бы одного решения 
уравнения~(\ref{e4-bos}), а~следовательно, и~оптимального 
управления~$u_t^*$, $0\hm\leq t\hm\leq T$, до\-став\-ля\-юще\-го минимум 
целевому функционалу~(\ref{e1-bos}). Задача оптимизации далее получается 
путем указания конкретных выражений для~$L_t$, $l$, $m_t$ и~$\sigma_t$.

\section{Постановка задачи управления выходом}

     Рассматриваемые далее случайные функции будут предполагаться 
скалярными. Такое упрощение позволит разгрузить выкладки и~итоговые 
выражения от не самых существенных деталей.
     
     Рассмотрим стохастическую дифференциальную сис\-те\-му, со\-сто\-яние 
которой представляет диффузи\-он\-ный процесс~$y_t$, описываемый 
нелинейным стохастическим дифференциальным уравнением Ито
     \begin{equation}
     dy_t=A_t\left( y_t\right) dt +\Sigma_t \left( y_t\right) dv_t\,,\enskip 
y_0=Y\,,
     \label{e5-bos}
     \end{equation}
где $v_t$~--- стандартный (одномерный) винеровский процесс; $Y$~--- 
случайная величина с~конечным вторым моментом; функции~$A_t$ 
и~$\Sigma_t$ удовлетворяют условиям Ито:
\begin{equation*}
\left\vert A_t(y)\right\vert +\left\vert \Sigma_t(y)\right\vert \leq C(1+\vert y\vert )\ 
\mbox{для\ всех } 0\leq t\leq T\,;
\end{equation*}

\vspace*{-12pt}

\noindent
\begin{multline*}
\hspace*{-2.10051pt}\left\vert A_t\left(y_1\right) -A_t \left( y_2\right) \right\vert +\left\vert 
\Sigma_t\left( y_1\right) -\Sigma_t \left(y_2\right)\right\vert \leq
C\left\vert y_1-y_2\right\vert\\
 \mbox{для\ всех\ } 0\leq t\leq T\ \mbox{и } 
y_1,y_2\in \mathbb{R}^1\,,
\end{multline*}
обеспечивающим существование единственного сильного (потраекторного) 
решения уравнения.
     
     Будем считать, что~$y_t$ описывает состояние некоторой 
динамической системы. Соответственно, поведение этой сис\-те\-мы опишем 
выходом, линейно связанным с~со\-сто\-янием:
     \begin{equation}
     dz_t=a_t y_t \,dt+ b_t z_t \,dt+ c_t u_t \,dt+\sigma_t \,dw_t\,,\enskip
     z_0=Z\,.
     \label{e6-bos}
     \end{equation}
Здесь $w_t$~--- не зависящий от~$v_t$, $Y$ и~$Z$ стандартный (одномерный) 
винеровский процесс; $Z$~--- случайная величина с~конечным вторым 
моментом; $u_t$~--- допустимое неупреждающее управ\-ле\-ние, качество 
которого определяется целевым функционалом следующего вида:
\begin{multline}
\!\hspace*{-3.98538pt}J\left( U_0^T\right) ={\sf E}\left\{ \int\limits_0^T \!\left( S_t\left( s_ty_t-g_t z_t -h_t 
u_t\right)^2 +G_t z_t^2+{}\right.\right.\\
\left.\left.{}+ H_t u_t^2
\vphantom{S_t\left( s_ty_t-g_t z_t -h_t 
u_t\right)^2}
\right) dt+S_T\left( s_T y_T -g_T 
z_T\right)^2+G_T z_T^2
\vphantom{\int\limits_0^T}\right\}\,,
\label{e7-bos}
\end{multline}
где $S_t$, $G_t$ и~$H_t$~--- неотрицательные функции\linebreak
$0\hm\leq t\hm\leq T$. 
Такой критерий отражает физический смысл задачи распределения ресурсов 
со\-глас\-но аналогичной~(\ref{e5-bos})--(\ref{e7-bos}) задаче для дис\-крет\-но\-го 
времени, рас\-смот\-рен\-ной в~\cite{7-bos}. В~част\-ности,  
функци\-онал~(\ref{e7-bos}) поз\-во\-ля\-ет ставить задачи отслеживания
 выходом 
со\-сто\-яния сис\-те\-мы, используя сла\-га\-емое $(y_t\hm- z_t)^2$, или 
управлением~--- линейной комбинации со\-сто\-яния и~выхода, сла\-га\-емое типа\linebreak 
$(y_t\hm+ z_t\hm- u_t)^2$. Поскольку задача формулируется 
в~предположении наличия пол\-ной информации о~со\-сто\-янии~$y_t$ 
и~выходе~$z_t$ (соответствующую $\sigma$-ал\-геб\-ру 
обозначим~$\mathcal{F}_t^{y,z}$), то допустимое управ\-ле\-ние ищется 
в~классе~$\mathcal{F}_t^{y,z}$-из\-ме\-ри\-мых неупреждающих функций 
(и,~как будет показано далее, оказывается управ\-ле\-ни\-ем с~обратной связью).

     Функции~$a_t$, $b_t$, $c_t$ и~$\sigma_t$ будем предполагать 
ограниченными: $\vert a_t\vert \hm+ \vert b_t\vert \hm+\vert c_t\vert \hm+ \vert 
\sigma_t \vert \hm\leq C$ для всех $0\hm\leq t\hm\leq T$, процесс  
управления~--- допустимым не\-упреж\-да\-ющим~\cite{9-bos}, обеспечивая, 
таким образом, существование сильного решения урав\-не\-ния~(\ref{e6-bos}) 
для любого допустимого управ\-ления.
     
     Задачу составляет поиск~$u_t^*$~--- допустимого управ\-ле\-ния, 
доставляющего минимум квад\-ра\-тич\-но\-му функционалу~$J(U_0^T)$.
      
     Поставленная задача очевидным образом формулируется в~терминах 
введенных выше в~(\ref{e1-bos})--(\ref{e3-bos}) обозначений, а~именно: 
     требуется обозначить
     \begin{gather*}
      x_t=\begin{pmatrix}
     y_t\\ z_t\end{pmatrix};\quad  m_t(x_t, u_t)=\begin{pmatrix}
     A_t(y_t)\\ a_t y_t +b_t z_t +c_t u_t\end{pmatrix};\\
     \sigma_t(x_t)= \begin{pmatrix}
     \Sigma_t(y_t)& 0\\
     0& \sigma_t\end{pmatrix};\quad W_t=\begin{pmatrix}
     v_t \\ w_t\end{pmatrix}
     %     \label{e8-bos}
     \end{gather*}
для записи уравнения со\-сто\-яния типа~(\ref{e2-bos}) и
\begin{align*}
L_t(x,u)&= L_t(y,z,u) ={}\\
&\hspace*{3mm}{}=S_t\left( s_t y-g_t z -h_t u\right)^2 +G_t z^2 +H_t  u^2\,;\\
l(x)&= l(y,z) =S_T \left( S_T y-g_T z\right)^2 +G_T z^2
%\label{e9-bos}
\end{align*}
для записи целевого функционала в~виде~(\ref{e1-bos}).

     Функция Беллмана~(\ref{e3-bos}) принимает вид 
     $V_t(x)\hm= V_t(y,z)$. Для записи со\-от\-вет\-ст\-ву\-юще\-го~(\ref{e4-bos}) 
уравнения Беллмана для~$V_t(y,z)$ заметим, что
     $$
     \left( \sigma^2_{t_{ij}}\right)_{i,j=1,2}= \begin{pmatrix}
     \Sigma_t^2(y) & 0\\
     0 & \sigma_t^2\end{pmatrix}\,.
     $$
     
     С~учетом перечисленных обозначений урав\-не\-ние динамического 
программирования~(\ref{e4-bos}) принимает вид:
     \begin{multline}
     \fr{\partial V_t(y,z)}{\partial t} +\fr{1}{2}\left( \Sigma_t^2(y) \fr{\partial^2 
V_t(y,z)} {\partial y^2}+\sigma_t^2\fr{\partial^2 V_t(y,z)} {\partial 
z^2}\right)+{}\\
    {}+\min\limits_u\! \left[ A_t(y) \fr{\partial V_t(y,z)}{\partial y}+\left( a_t 
y+b_t z+c_t u\right) \fr{\partial V_t(y,z)}{\partial z} +{}\right.\hspace*{-3pt}\\
\left.{}+ S_t\left( s_t y-g_t z-h_t 
u\right)^2+G_t z^2+H_t u^2
     \vphantom{\fr{\partial V_t(y,z)}{\partial y}}\right] =0\,,\\
     V_T(y,z)=S_T\left( s_T y-g_T z\right)^2+G_T z^2\,.
     \label{e10-bos}
     \end{multline}
     Это и~есть то самое уравнение, которое требуется решить: 
существование решения данного урав\-не\-ния суть достаточное условие 
оптимальности; оптимальное управ\-ле\-ние при этом~--- точ\-ка минимума 
со\-от\-вет\-ст\-ву\-юще\-го сла\-га\-емого.
     
\section{Динамическое программирование и~оптимальное 
управление}

     В рассматриваемой постановке линейность\linebreak выхода и~квадратичность 
критерия дают те же преимущества, что и~в~классической  
ли\-ней\-но-квад\-ра\-тич\-ной задаче управ\-ле\-ния~\cite{1-bos}, а~именно: 
позволяют сразу определить вид оптимального управ\-ле\-ния и~фактические 
условия его существования. Действительно, со\-хра\-няя в~(\ref{e10-bos}) под 
знаком $\min\nolimits_u$ только члены, зависящие от~$u$, получаем
     \begin{multline*}
     \fr{\partial V_t(y,z)}{\partial t} +\fr{1}{2}\left( \Sigma_t^2(y) \fr{\partial^2 
V_t(y,z)} {\partial y^2}+\sigma_t^2\fr{\partial^2 V_t(y,z)} {\partial 
z^2}\right)+{}\\
     {}+A_t(y)\fr{\partial V_t(y,z)}{\partial y}+\left( a_t y+b_t z\right) 
\fr{\partial V_t(y,z)}{\partial z}+{}\\
{}+S_t\left( s_t y-g_t z\right)^2 +G_t z^2+{}
\end{multline*}

\noindent
\begin{multline*}
     {}+\min\limits_u \left[ \left( c_t \fr{\partial V_t(y,z)}{\partial z}-2S_t \left( 
s_t y-g_t z\right) h_t\right)u +{}\right.\\
\left.{}+\left( S_t h_t^2+H_t\right) u^2
\vphantom{\fr{\partial V_t(y,z)}{\partial z}}
\right]=0\,,
     %\label{e11-bos}
     \end{multline*}
откуда в~предположении $S_t h_t^2\hm+ H_t\hm>0$ следует, что существует 
оптимальное управ\-ле\-ние, которое определяется равенством
\begin{multline}
u_t^* = u_t^*(y,z)=-\fr{1}{2}\left( S_t h_t^2 +H_t\right)^{-1} \left( c_t 
\fr{\partial V_t(y,z)}{\partial z}-{}\right.\\
\left.{}-2S_t\left( s_t y-g_t z\right) h_t
\vphantom{\fr{\partial V_t(y,z)}{\partial z}}
\right)
\label{e12-bos}
\end{multline}
и доставляет минимум соответствующему сла\-га\-емо\-му в~урав\-не\-нии Беллмана, 
равный
$-\left( S_t h_t^2\hm+\right.$\linebreak
$\left.{}+H_t\right)^{-1} \left( c_t 
{\partial V_t(y,z)}/{\partial 
z}\hm-2S_t\left( s_t y \hm-g_t z\right) h_t \right)^2/4.
$ 
     
     Отметим, что, как и~в~классической ли\-ней\-но-квад\-ра\-тич\-ной 
задаче, управ\-ле\-ние из класса до\-пус\-ти\-мых не\-упреж\-да\-ющих получилось 
управ\-ле\-ни\-ем с~обратной связью.
     
     Таким образом, функция Беллмана описывается сле\-ду\-ющим 
дифференциальным уравнением:
     \begin{multline}
     \fr{\partial V_t(y,z)}{\partial t} +\fr{1}{2}\left( \Sigma_t^2(y) \fr{\partial^2 
V_t(y,z)} {\partial y^2}+\sigma_t^2\fr{\partial^2 V_t(y,z)} {\partial 
z^2}\right)+{}\\
     {}+ A_t(y) \fr{\partial V_t(y,z)}{\partial y}+\left( a_t y+b_t z\right) 
\fr{\partial V_t(y,z)}{\partial z}+{}\\
{}+ S_t \left( s_t y- g_t z\right)^2 +G_t z^2-
 \fr{1}{4}\left( S_t h_t^2+H_t\right)^{-1}\times{}\\
 {}\times \left( c_t \fr{\partial V_t(y,z)} 
{\partial z}-2S_t\left( s_t y -g_t z\right) h_t \right)^2=0\,.
     \label{e13-bos}
     \end{multline}
     
     Возводя в~квадрат по\-след\-нее сла\-га\-емое в~(\ref{e13-bos}), перепишем 
его в~виде:
     \begin{multline}
     \fr{\partial V_t(y,z)}{\partial t} +\fr{1}{2}\left( \Sigma_t^2(y) \fr{\partial^2 
V_t(y,z)} {\partial y^2}+\sigma_t^2\fr{\partial^2 V_t(y,z)} {\partial 
z^2}\!\right)+{}\\
{}+A_t(y) \fr{\partial V_t(y,z)}{\partial y}
+ \left( 
\vphantom{\left( S_t h_t^2 +H_t\right)^{-1}}
a_t y+b_t z+{}\right.\\
\left.{}+\left( S_t h_t^2 +H_t\right)^{-1}
 c_t S_t \left( s_t y-g_t z\right) h_t
\right) 
     \fr{\partial V_t(y,z)}{\partial z}+{}\\
     {}+\left( S_t-\left( S_t h_t^2 +H_t\right)^{-1} S_t^2 h_t^2\right)\left( s_t y -
g_t z\right)^2+{}\\
     \!\!{}+
     G_t z^2 -\fr{1}{4}\left( S_t h_t^2+H_t\right)^{-1}\! c_t^2
     \left(\fr{\partial V_t(y,z)}{\partial z}\right)^{\!2}=0\,.\!\!
     \label{e14-bos}
     \end{multline}
     
     Рассматривая полученное уравнение, заметим, что его решение может 
быть пред\-став\-ле\-но в~виде:
   \begin{equation}
     V_t(y,z)= \alpha_t z^2+\beta_t(y) z +\gamma_t(y)\,,
     \label{e15-bos}
     \end{equation}
т.\,е.\ будем искать решение при дополнительном предположении 
о~квад\-ра\-тич\-ности функции Белл\-ма\-на по переменной~$z$, и~сведем, таким 
образом, поиск оптимального решения к~уравнениям относительно функций 
$\alpha_t$, $\beta_t(y)$ и~$\gamma_t(y)$. Отметим сразу, что явный вид 
функции~$\gamma_t(y)$ для реализации оптимального управ\-ле\-ния не 
требуется, однако далее будет предложен вариант вы\-чис\-ле\-ния и~этой 
функции, что пред\-став\-ля\-ет\-ся небесполезным, поскольку позволит выполнять 
расчет минимума целевого функционала. Источником для 
предложения~(\ref{e15-bos}) является уже упоминавшаяся аналогичная 
задача для случая дис\-крет\-но\-го времени~\cite{7-bos, 8-bos}. В~той задаче 
выражение для функции Беллмана получается формально без 
дополнительных усилий. При этом форма~(\ref{e15-bos}) обнаруживается 
как свойство оптимального решения. В~рассматриваемом случае 
непрерывного времени~(\ref{e15-bos}) постулируется, а~пра\-виль\-ность 
постулата под\-тверж\-да\-ет\-ся далее ре\-зуль\-ти\-ру\-ющи\-ми уравнениями 
для~$\alpha_t$, $\beta_t(y)$ и~$\gamma_t(y)$ Кроме того, данное 
предположение пред\-став\-ля\-ет\-ся вы\-те\-ка\-ющим из линейной структуры задачи 
в~отношении переменной~$z$, в~част\-ности, тем фактом, что такой вид 
функции Беллмана обеспечивает линейность оптимального 
управ\-ле\-ния~(\ref{e12-bos}) по~$z$.

     Граничное условие при выбранном предположении~(\ref{e15-bos}) 
принимает вид:

\noindent
     \begin{multline*}
     V_T(y,z)= S_T\left( s_T y- g_T z\right)^2+G_T z^2 ={}\\[-0.5pt]
     {}=\alpha_T z^2 
+\beta_T(y) z +\gamma_T(y)\,,
    \end{multline*}
т.\,е.

\noindent
\begin{align*}
\alpha_T&= S_T g_T^2 +G_T\,;\\[-0.5pt]
\beta_T(y)&=-2S_T s_T g_T y\,;\\[-0.5pt]
\gamma_T(y)&=S_T s_T^2 y^2\,.
%\label{e16-bos}
\end{align*}
          При этом само оптимальное управ\-ле\-ние, определенное 
выражением~(\ref{e12-bos}), оказывается управ\-ле\-ни\-ем с~обратной связью 
по~$y_t$ и~$z_t$:

\noindent
     \begin{multline}
     u_t^*=u_t^*(y,z) ={}\\[-0.5pt]
     {}=
     -\fr{1}{2}\left( S_t h_t^2 +H_t\right)^{-1}
     \left( c_t \left( 2\alpha_t z +\beta_t(y)\right) +{}\right.\\[-0.5pt]
    \left. {}+2S_t\left( s_t y-g_t z\right) 
h_t\right)\,.
     \label{e17-bos}
     \end{multline}
          Подставляем $V_t(y,z)\hm= \alpha_t z^2 \hm+ \beta_t(y) 
z\hm+\gamma_t(y)$ в~(\ref{e14-bos}):

\noindent
     \begin{multline*}
     \fr{\partial \alpha_t}{\partial t}\, z^2 +
     \fr{\partial \beta_t(y)}{\partial t}\,z +
     \fr{\partial \gamma_t(y)}{\partial t}+{}\\[-0.5pt]
     {}+\fr{1}{2}\left( \Sigma_t^2(y) \left( 
\fr{\partial^2\beta_t(y)}{\partial y^2}\,z +\fr{\partial^2 \gamma_t(y)}{\partial 
y^2}\right) +2\sigma_t^2\alpha_t\right)+{}\\[-0.5pt]
 {}+A_t(y)\left(\fr{\partial \beta_t(y)}{\partial y}\,z + \fr{\partial 
\gamma_t(y)}{\partial y}\right) +{}\\[-0.5pt]
\hspace*{-0.22987pt}{}+\left( a_t y+b_t z+\left( S_t h_t^2 +H_t\right)^{-1} c_t S_t \left( s_t y-
g_t z\right) h_t\right)\times{}
\end{multline*}

\noindent
\begin{multline*}
         {}\times \left( 2\alpha_t z+\beta_t(y)\right)+{}\\
     {}+\left( S_t-\left( S_t h_t^2 +H_t\right)^{-1} S_t^2 h_t^2\right)\left( s_t y-
g_t z\right)^2+{}\\
     {}+ G_t z^2 -\fr{1}{4}\left( S_t h_t^2 +H_t\right)^{-1} c_t^2 \left( 
2\alpha_t z+\beta_t(y)\right)^2=0\,.
     \end{multline*}
          Далее выделяем слагаемые при~$z^2$, $z$ и~$z^0$
          
          \noindent
     \begin{multline*}
     \fr{\partial \alpha_t}{\partial t}\, z^2 +\fr{\partial \beta_t(y)}{\partial t}\,z +
     \fr{\partial \gamma_t(y)}{\partial 
t}+\fr{1}{2}\,\Sigma_t^2(y)\fr{\partial^2\beta_t(y)}{\partial y^2}\,z+ {}\\
{}+
\fr{1}{2}\,\Sigma_t^2(y)\fr{\partial^2\gamma_t(y)}{\partial 
y^2}+\sigma_t^2\alpha_t+A_t(y)\fr{\partial \beta_t(y)}{\partial y}\,z +{}\\
{}+A_t(y) \fr{\partial 
\gamma_t(y)}{\partial y}+{}\\
{}+ 2\alpha_t \left( b_t -\left( S_t h_t^2+H_t\right)^{-1} c_t 
S_t h_t g_t \right)z^2+{}\\
     {}+
     \left( 2\alpha_t\left( \alpha_t+\left( S_t h_t^2+H_t\right)^{-1} c_t S_t h_t 
s_t\right)y +{}\right.\\
\left.{}+\beta_t(y) \left( b_t-\left( S_t h_t^2+H_t\right)^{-1} c_t S_t h_t 
g_t\right) \right) z+{}\\
     {}+\beta_t(y)\left( a_t +\left( S_t h_t^2+H_t\right)^{-1} c_t S_t h_t s_t\right) 
y+{}\\
{}+ \left( S_t -\left( S_t h_t^2+H_t\right)^{-1} S_t^2 h_t^2\right) g_t^2 z^2-{}\\
     {}- 2\left( S_t -\left( S_t h_t^2+H_t\right)^{-1} S_t^2 h_t^2\right) s_t g_t yz 
+{}\\
{}+
     \left( S_t-\left( S_t h_t^2+H_t\right)^{-1} S_t^2 h_t^2\right) s_t^2 y^2+{}\\
     {}+G_t z^2 -\left( S_t h_t^2 +H_t\right)^{-1} c_t^2 \alpha_t^2 z^2 -{}\\
     {}-\left( 
S_t h_t^2+H_t\right)^{-1} c_t^2 \alpha_t \beta_t(y) z-{}\\
{}-
\fr{1}{4}\left( S_t h_t^2+H_t\right)^{-1}  c_t^2 \beta_t^2(y)=0\,,
     \end{multline*}
группируем их и~получаем сле\-ду\-ющие уравнения:
\begin{itemize}
\item  для~$\alpha_t$:

\noindent
\begin{multline}
\fr{\partial\alpha_t}{\partial t}+2\alpha_t\left( b_t-\left( S_t h_t^2+H_t\right)^{-1} c_t 
S_t h_t g_t\right)+{}\\
{}+ \left( S_t- \left( S_t h_t^2+H_t\right)^{-1} S_t^2 h_t^2\right) 
g_t^2+G_t-{}\\
\hspace*{-8mm}{}-\left( S_t h_t^2+H_t\right)^{-1} c_t^2 \alpha_t^2 =0\,,\enskip \alpha_T=S_T 
g_t^2+G_T\,;\!\!
\label{e18-bos}
\end{multline}
\item для $\beta_t$:

\noindent
\begin{multline}
\fr{\partial\beta_t(y)}{\partial 
t}+\fr{1}{2}\,\Sigma_t^2(y)\fr{\partial^2\beta_t(y)}{\partial y^2} 
+A_t(y)\fr{\partial \beta_t(y)}{\partial y}+{}\\
{}+ 2\alpha_t\left( a_t +\left( S_t h_t^2+H_t\right)^{-1} c_t S_t h_t s_t\right) y+{}\\
{}+
\beta_t(y)\left( b_t -\left( S_t h_t^2 +H_t\right)^{-1} c_t S_t h_t g_t\right)-{}\\
{}-2\left( S_t-\left( S_t h_t^2+H_t\right)^{-1} S_t^2 h_t^2\right) s_t g_t y-{}
\\
{}-
\left( S_t h_t^2+H_t\right)^{-1} c_t^2 \alpha_t \beta_t(y)=0\,,\\
\beta_T(y)=-2S_T s_T g_T y\,;
\label{e19-bos}
\end{multline}
\item  для $\gamma_t$:
\begin{multline}
\hspace*{-0.8pt}\fr{\partial \gamma_t(y)}{\partial t}+\fr{1}{2}\,\Sigma_t^2(y)
\fr{\partial^2 \gamma_t(y)}{\partial y^2} +\sigma_t^2 \alpha_t +A_t(y)
\fr{\partial \gamma_t(y)}{\partial y}+{}\\
{}+ \beta_t(y)\left( a_t +\left( S_t h_t^2+H_t\right)^{-1} c_t S_t h_t s_t\right) y+{}\\
{}+
\left( S_t-\left( S_t h_t^2+H_t\right)^{-1} S_t^2 h_t^2\right)  s_t^2 y^2-{}\\
{}-\fr{1}{4}\left( S_t h_t^2+H_t\right)^{-1} c_t^2 \beta_t^2(y) =0\,,\\
\gamma_T(y)=S_T s_T^2 y^2\,.
\label{e20-bos}
\end{multline}
\end{itemize}
     
     Уравнение~(\ref{e18-bos}), легко заметить, является уравнением 
Риккати, которое в~силу сформулированного выше условия   
имеет единственное неотрицательное решение для всех $0\hm\leq t\hm\leq T$. 
Этот факт требует дополнительного комментария. Для получения 
уравнения~(\ref{e18-bos}) рас\-смот\-рим исходную задачу при дополнительных 
условиях $a_t\hm=0$ и~$s_t\hm=0$ для всех $0\hm\leq t\hm\leq T$. Нетрудно 
видеть, что эти условия рассматриваемую по\-ста\-нов\-ку сводят фактически 
к~классической ли\-ней\-но-квад\-ра\-тич\-ной задаче. Имеющуюся 
в~рассматриваемой формулировке чуть более общую форму целевой 
функции (принципиального значения это обобщение, конечно, не имеет) 
сведем к~классической еще одним предположением: $S_t\hm=0$ для всех 
$0\hm\leq t\hm\leq T$. Теперь уравнение для~$\alpha_t$ принимает хорошо 
известный вид:
     \begin{equation}
     \fr{\partial \alpha_t}{\partial t}+2\alpha_t b_t +G_t- H_t^{-1} c_t^2 
\alpha_t^2=0\,,\enskip \alpha_T=G_T\,.
     \label{e21-bos}
     \end{equation}

     В таком случае, как известно~\cite{10-bos}, существует единственное 
оптимальное управление~--- линейное с~обратной связью по выходу~$z_t$, 
с~коэффициентом усиления, опи\-сы\-ва\-емым уравнением  
Риккати~(\ref{e21-bos}). Именно этот результат дают  
уравнения~(\ref{e18-bos})--(\ref{e20-bos}) и~описываемая ими функция 
Беллмана~(\ref{e15-bos}), так как из $a_t\hm=0$ и~$s_t\hm=0$ немедленно 
следует, что $\beta_t(y)\hm=0$, откуда, в~свою очередь, с~учетом 
не\-за\-ви\-си\-мости решения от~$y_t$ следует, что $\gamma_t(y)\hm=\gamma_t$, 
т.\,е.\ не зависит от~$y$ и~задается уравнением: 
     $$
     \fr{\partial \gamma_t(y)}{\partial t} +\sigma^2_t \alpha_t=0\,,\enskip 
\gamma_T=0\,.
     $$ 
     Оптимальное управ\-ле\-ние при этом 
     $$
     u_t^*= -H_t^{-1} c_t \alpha_t z_t\,,
     $$
      т.\,е.\ все полностью совпадает с~известным классическим решением.
     
     С уравнениями~(\ref{e19-bos}) и~(\ref{e20-bos}) ситуация, естественно, 
обстоит сложнее. Это линейные уравнения второго порядка параболического 
типа, поскольку\linebreak
 $\Sigma_t^2(y)\hm>0$. Фактически отсутствуют 
конструктивные условия, гарантирующие существование их\linebreak
 решений 
(требовать, чтобы все фигурирующие в~уравнениях коэффициенты были 
представлены аналитическими функциями на всем пространстве значений, 
вряд ли целесообразно), поэтому далее будем предполагать, что данные 
уравнения имеют на рас\-смат\-ри\-ва\-емом интервале $0\hm\leq t\hm\leq T$ хотя 
бы одно ограниченное решение и~именно эти условия будем рас\-смат\-ри\-вать 
как достаточные условия существования оптимального решения 
рассматриваемой задачи.
     
     Таким образом, доказана следующая тео\-рема.
     
     \smallskip
     
     \noindent
     \textbf{Теорема.}\ \textit{Пусть для диффузионного 
процесса}~(\ref{e5-bos}) \textit{выполнены условия Ито, для 
     процесса}~(\ref{e6-bos})~--- \textit{ограничены коэффициенты, 
уравнения}~(\ref{e18-bos})--(\ref{e20-bos}) \textit{имеют ограниченные 
решения для $0\hm\leq t\hm\leq T$. Тогда минимум  
функционалу}~(\ref{e7-bos}) \textit{доставляет оптимальное 
управ\-ле\-ние}~(\ref{e17-bos}), \textit{где} $y\hm= y_t$; $z\hm=z_t$.
     
\section{Заключение}

     Рассмотренная задача оптимизации в~целом близка и~по модели, и~по 
критерию к~классической ли\-ней\-но-квад\-ра\-тич\-ной постановке. 
Принципиальным отличием является нелинейная модель для описания 
со\-сто\-яния динамической сис\-те\-мы, в~которой отсутствует управ\-ля\-ющее 
воздействие.\linebreak
 Такую модель наряду с~традиционной интер\-пре\-тацией  
<<со\-сто\-яние--вы\-ход>> мож\-но понимать как\linebreak модель неконтролируемого 
слож\-но\-го внешнего воздействия. Небольшое дополнительное отличие дает 
предложенная форма квад\-ра\-тич\-но\-го критерия, поз\-во\-ля\-ющая, в~част\-ности, 
ставить такие задачи, как отслеживание выходом или управ\-ле\-ни\-ем со\-сто\-яния 
сис\-те\-мы или ее выхода.
     
     Поскольку обсуждать возможности точного решения уравнений, 
определяющих оптимальное управ\-ле\-ние, не имеет смыс\-ла, наиболее 
актуальной далее является задача их приближенного чис\-лен\-но\-го решения 
и~анализа воз\-мож\-ности практической реализации. Этому посвящена вторая 
часть данной работы, пла\-ни\-ру\-емая к~выходу в~ближайшее время.

{\small\frenchspacing
 {%\baselineskip=10.8pt
 \addcontentsline{toc}{section}{References}
 \begin{thebibliography}{99}
\bibitem{1-bos}
\Au{Athans M.} Editorial on the LQG problem~// IEEE~T. Automat. Contr., 1971. Vol.~16. 
No.\,6. P.~528--552. doi: 10.1109/TAC.1971.1099845.
\bibitem{2-bos}
\Au{Wu Z.} Forward-backward stochastic differential equations, linear quadratic stochastic 
optimal control and nonzero sum differential games~// J.~Syst. Sci. Complex., 2005. Vol.~18. 
No.\,2. P.~179--192.
\bibitem{3-bos}
\Au{Chen B.\,S., Zhang~W.} Stochastic H2/H1 control with state-dependent noise~// IEEE 
T.~Automat. Contr., 2004. Vol.~49. No.\,1. P.~45--56. doi: 10.1109/TAC.2003.821400.
\bibitem{4-bos}
\Au{Bohacek S.} A~stochastic model of TCP and fair video transmission~// IEEE 
INFOCOM, 2003. Vol.~2. P.~1134--1144. doi: 10.1109/INFCOM.2003.1208950.
\bibitem{5-bos}
\Au{Домбровский В.\,В., Объедко~Т.\,Ю.} Управление с~прогнозированием системами 
с~марковскими скачками при ограничениях и~применение к~оптимизации 
инвестиционного портфеля~// Автомат. телемех., 2011. №\,5. С.~96--112. doi: 
10.1134/S0005117911050079.
\bibitem{6-bos}
\Au{Баландин Д.\,В., Коган~М.\,М.} Оптимальное линейно-квад\-ра\-тич\-ное управление: от 
матричных уравнений к~линейным матричным неравенствам~// Автомат. телемех., 2011. 
№\,11. С.~60--69. doi: 10.1134/ S0005117911110038.
\bibitem{7-bos}
\Au{Босов А.\,В.} Обобщенная задача распределения ресурсов программной системы~// 
Информатика и~её применения, 2014. Т.~8. Вып.~2. С.~39--47. doi: 
10.14357/19922264140204.
\bibitem{8-bos}
\Au{Босов А.\,В.} Управление линейным выходом дискретной стохастической системы по 
квадратичному критерию~// Изв. РАН. Теория и~системы управления, 2016. №\,3.  
С.~19--35. doi: 10.1134/S1064230716030060.
\bibitem{9-bos}
\Au{Флеминг У., Ришел~Р.} Оптимальное управление детерминированными 
и~стохастическими системами~/ Пер. с~англ.~--- М.: Мир, 1978. 316~с. 
(\Au{Fleming~W.\,H., Rishel~R.\,W.} Deterministic and stochastic optimal control.~--- New 
York, NY, USA: Springer-Verlag, 1975. 222~p.)
\bibitem{10-bos}
\Au{Девис М.\,Х.\,А.} Линейное оценивание и~стохастическое управление~/ Пер. с~англ.~--- 
М.: Наука, 1984. 206~с. (\Au{Davis~M.\,H.\,A.} Linear estimation and stochastic control.~--- 
London: Chapman and Hall, 1977. 224~p.)

 \end{thebibliography}

 }
 }

\end{multicols}

\vspace*{-6pt}

\hfill{\small\textit{Поступила в~редакцию 30.03.18}}

\vspace*{4pt}

%\newpage

%\vspace*{-24pt}

\hrule

\vspace*{2pt}

\hrule

\vspace*{-2pt}


\def\tit{STOCHASTIC DIFFERENTIAL SYSTEM OUTPUT CONTROL 
BY~THE~QUADRATIC CRITERION.~I.~DYNAMIC\\ PROGRAMMING 
OPTIMAL SOLUTION}


\def\titkol{Stochastic differential system output control 
by~the~quadratic criterion. I.~Dynamic programming 
optimal solution}

\def\aut{A.\,V.~Bosov and~A.\,I.~Stefanovich}

\def\autkol{A.\,V.~Bosov and~A.\,I.~Stefanovich}

\titel{\tit}{\aut}{\autkol}{\titkol}

\vspace*{-11pt}


\noindent
Institute of Informatics Problems, Federal Research Center ``Computer Science 
and Control'' of the Russian Academy of Sciences, 44-2~Vavilov Str., Moscow 
119333, Russian Federation


\def\leftfootline{\small{\textbf{\thepage}
\hfill INFORMATIKA I EE PRIMENENIYA~--- INFORMATICS AND
APPLICATIONS\ \ \ 2018\ \ \ volume~12\ \ \ issue\ 3}
}%
 \def\rightfootline{\small{INFORMATIKA I EE PRIMENENIYA~---
INFORMATICS AND APPLICATIONS\ \ \ 2018\ \ \ volume~12\ \ \ issue\ 3
\hfill \textbf{\thepage}}}

\vspace*{3pt}



\Abste{The problem of optimal control for the Ito diffusion 
process and a~controlled linear output is solved. The considered 
statement is close to the classical linear-quadratic Gaussian 
control  (LQG control) problem. Differences consist in the fact 
that the state is described by the nonlinear differential Ito equation  $dy_y = A_t(y_t) 
\,dt+\Sigma_t(y_t)\,dv_t$ and does not depend on the control~$u_t$, 
optimization subject is controlled linear output 
 $dz_t=a_ty_t\,dt +b_tz_t\,dt +c_t u_t\,dt +\sigma_t \,dw_t$. 
Additional generalizations are included in the quadratic 
quality criterion for the purpose of statement such problems 
as state tracking by output or a linear combination of state 
and output tracking by control. The method of dynamic programming 
is used for the solution. 
The assumption about Bellman function in the form  $V_t(y,z)= \alpha_t 
z^2+\beta_t(y) z+\gamma_t(y)$ allows one to find it. 
Three differential equations for the coefficients $\alpha_t$,  $\beta_t(y)$,
and $\gamma_t(y)$ give the solution. 
These equations constitute the optimal solution of the problem under consideration.}

\KWE{stochastic differential equation; optimal control; dynamic programming; 
Bellman function; Riccati equation; linear differential equations of parabolic type}


\DOI{10.14357/19922264180314}

\vspace*{-12pt}

\Ack
\noindent
This work was partially supported by the Russian Science Foundation (grant  
16-07-00677).



%\vspace*{6pt}

  \begin{multicols}{2}

\renewcommand{\bibname}{\protect\rmfamily References}
%\renewcommand{\bibname}{\large\protect\rm References}

{\small\frenchspacing
 {%\baselineskip=10.8pt
 \addcontentsline{toc}{section}{References}
 \begin{thebibliography}{99}
\bibitem{1-bos-1}
\Aue{Athans, M.} 1971. Editorial on the LQG problem. \textit{IEEE~T. 
Automat. Contr.} 16(6):528--552. doi: 10.1109/ TAC.1971.1099845.
\bibitem{2-bos-1}
\Aue{Wu, Z.} 2005. Forward-backward stochastic differential equations, linear 
quadratic stochastic optimal control and\linebreak\vspace*{-12pt}

\columnbreak

\noindent
 nonzero sum differential games. 
\textit{J.~Syst. Sci. Complex.} 18(2):179--192.
\bibitem{3-bos-1}
\Aue{Chen, B.\,S. and W.~Zhang.} 2004. Stochastic H2/H1 control with  
state-dependent noise. \textit{IEEE~T. Automat. Contr.} 49(1):45--56.
doi: 10.1109/TAC.2003.821400.
\bibitem{4-bos-1}
\Aue{Bohacek, S.} 2003. A~stochastic model of TCP and fair video 
transmission. \textit{IEEE INFOCOM}. 2:1134--1144.
doi: 10.1109/INFCOM.2003.1208950.
\bibitem{5-bos-1}
\Aue{Dombrovskii, V.\,V., and T.\,Yu.~Ob''edko.} 2011. Predictive control of 
systems with Markovian jumps under constraints and its application to the 
investment portfolio optimization. \textit{Automat. Rem. Contr.}  
72(5):989--1003.
\bibitem{6-bos-1}
\Aue{Balandin, D.\,V., and M.\,M.~Kogan.} 2011. Optimal linear-quadratic 
control: From matrix equations to linear matrix inequalities. \textit{Automat. 
Rem. Contr.} 72(11):2276--2284.
\bibitem{7-bos-1}
\Aue{Bosov, A.\,V.} 2014. Obobshchennaya zadacha raspredeleniya resursov 
programmnoy sistemy [The generalized problem of software system resources 
distribution]. \textit{Informatika i~ee Primeneniya~--- Inform. Appl.}  
8(2):39--47. doi: 
10.14357/19922264140204.
\bibitem{8-bos-1}
\Aue{Bosov, A.\,V.} 2016. Discrete stochastic system linear output control 
with respect to a quadratic criterion. \textit{J.~Comput. Syst. Sc. 
Int.} 55(3):349--364.
\bibitem{9-bos-1}
\Aue{Fleming, W.\,H., and R.\,W.~Rishel.} 1975. \textit{Deterministic and 
stochastic optimal control.} New York, NY: Springer-Verlag. 222~p.
\bibitem{10-bos-1}
\Aue{Davis, M.\,H.\,A.} 1977. \textit{Linear estimation and stochastic 
control.} London: Chapman and Hall. 224~p.
\end{thebibliography}

 }
 }

\end{multicols}

\vspace*{-6pt}

\hfill{\small\textit{Received March 30, 2018}}

%\pagebreak

%\vspace*{-18pt}
     
     \Contr
     
       \noindent
       \textbf{Bosov Alexey V.} (b.\ 1969)~--- Doctor of Science in technology, 
principal scientist, Institute of Informatics Problems, Federal Research 
Center ``Computer Science and Control'' of the Russian Academy of Sciences, 
44-2~Vavilov Str., Moscow 119333, Russian Federation; 
\mbox{AVBosov@ipiran.ru}
       
       \vspace*{3pt}
       
       \noindent
       \textbf{Stefanovich Alexey I.} (b.\ 1983)~--- principal specialist, 
Institute of Informatics Problems, Federal Research Center ``Computer Science 
and Control'' of the Russian Academy of Sciences, 44-2~Vavilov Str., Moscow 
119333, Russian Federation; \mbox{AStefanovich@frccsc.ru}
\label{end\stat}

\renewcommand{\bibname}{\protect\rm Литература}       

                 %13

\def\stat{sopin}

\def\tit{АНАЛИЗ МЕХАНИЗМОВ НАРЕЗКИ СЕТИ С УЧЕТОМ ГАРАНТИЙ ДЛЯ РАЗЛИЧНЫХ 
ТИПОВ ТРАФИКА$^*$}

\def\titkol{Анализ механизмов нарезки сети с~учетом гарантий для различных 
типов трафика}

\def\aut{К.\,А.~Агеев$^1$, Э.\,С.~Сопин$^2$, Н.\,В.~Яркина$^3$, 
К.\,Е.~Самуйлов$^4$, С.\,Я.~Шоргин$^5$}

\def\autkol{К.\,А.~Агеев, Э.\,С.~Сопин, Н.\,В.~Яркина и др.} 
%К.\,Е.~Самуйлов$^4$, С.\,Я.~Шоргин$^5$ и~др.}

\titel{\tit}{\aut}{\autkol}{\titkol}

\index{Агеев К.\,А.}
\index{Сопин Э.\,С.}
\index{Яркина Н.\,В.} 
\index{Самуйлов К.\,Е.}
\index{Шоргин С.\,Я.}
\index{Ageev K.\,A.}
\index{Sopin E.\,S.}
\index{Yarkina N.\,V.}
\index{Samouylov K.\,Е.}
\index{Shorgin S.\,Ya.}
 

{\renewcommand{\thefootnote}{\fnsymbol{footnote}} \footnotetext[1]
{Исследование выполнено при поддержке Программы РУДН <<5-100>> и~при частичной финансовой 
поддержке РФФИ в~рамках научных проектов №\,19-07-00933 и~№\,19-37-90147.}}


\renewcommand{\thefootnote}{\arabic{footnote}}
\footnotetext[1]{Российский университет дружбы народов, ageev-ka@rudn.ru}
\footnotetext[2]{Российский университет дружбы народов; Институт проблем информатики Федерального исследовательского 
центра <<Информатика и~управ\-ле\-ние>> Российской академии наук, \mbox{sopin-es@rudn.ru}}
\footnotetext[3]{Российский университет дружбы народов, yarkina-nv@rudn.ru}
\footnotetext[4]{Российский университет дружбы народов; Институт проб\-лем информатики Федерального 
исследовательского центра <<Информатика и~управ\-ле\-ние>> 
Российской академии наук,  
\mbox{samouylov-ke@rudn.university}}
\footnotetext[5]{Институт проблем информатики Федерального исследовательского центра <<Информатика  
и~управ\-ле\-ние>> Российской академии наук, \mbox{sshorgin@ipiran.ru}}

\vspace*{-5pt}

 
  
  \Abst{Нарезка радиоресурсов сети (network slicing)~--- это одна из ключевых 
возможностей современных сетей, позволяющая нескольким виртуальным мобильным 
операторам использовать ресурсы одной базовой станции. Это дает возможность 
операторам, владельцам ресурсов, предоставлять в~аренду и~управлять несколькими 
выделенными логическими сетями с~определенной функциональностью, реализуемой поверх 
общей инфраструктуры. Каждая из этих логических сетей называется слайсом сети и~может 
быть адаптирована для обеспечения определенного поведения системы, чтобы наилучшим 
образом поддерживать определенные показатели качества услуг. В~работе построена модель 
механизма нарезки радиоресурсов, распределяющего ресурс по слайсам, и~проведен анализ 
этой модели методом имитационного моделирования.}
  
  
  \KW{имитационное моделирование; система массового обслуживания; ограниченные 
ресурсы; нарезка сети}

\DOI{10.14357/19922264200314} 
 
%\vspace*{-6pt}


\vskip 10pt plus 9pt minus 6pt

\thispagestyle{headings}

\begin{multicols}{2}

\label{st\stat}
  
\section{Введение}

\vspace*{-2pt}

  Нарезка радиоресурсов сети (англ.\ \textit{network slicing}) дает возможность 
оператору мобильной связи предоставлять выделенные логические сети 
в~аренду виртуальным сетевым операторам в~виде сетевых слайсов 
с~функциями, специфичными для клиента. Слайс сети, который охватывает все 
сегменты сетевой инфраструктуры, может быть выделен для конкретных видов 
услуг нескольким виртуальным операторам, предоставляющим схожие услуги, 
либо отдельно для каждого виртуального оператора~[1,~2].
  
  Для каждого слайса сети выделяются ресурсы (например, 
виртуализированные сетевые функции, пропускная способность сети и~др.), 
и~ошибки или неисправность, возникающие в~одном слайсе, не влияют на 
обеспечение показателей качества обслуживания QoS (Quality of Service) 
в~других слайсах; иными словами, гарантируется изоляция слайса для 
обеспечения гарантированного качества обслуживания. При этом алгоритм 
нарезки радиоресурсов должен обеспечивать эффективное использование 
ресурсов соты с~учетом гарантированного объема ресурсов, выделенного для 
каждого слайса~[3].
  
  Данная тематика в~последнее время привлекает повышенное внимание 
исследователей. В~\cite{4-sop} представлена гибкая модель нарезки сети 
радиодоступа (Radio Access Network, RAN). Основные цели заключаются 
в~определении уровня изоляции производительности между операторами 
виртуальных сетей (Virtual Network Operator, VNO), которые выступают 
в~качестве арендаторов сети, с~тем чтобы гарантировать, что их соглашения об 
уровне обслуживания (Service Level Agreement, SLA) не будут затронуты 
изменением различных параметров сети, и~в~то же время оптимизировать 
использование инфраструктуры RAN путем динамического распределения 
радиоресурсов между различными сегментами справедливым образом. 
  
  В~\cite{5-sop} также рассматривается система управления виртуальными 
радиоресурсами (Virtual radio resource management, VRRM), которая 
обеспечивает оптимальное использование виртуализированных ресурсов 
поставщика инфраструктуры между несколькими операторами виртуальной 
сети. В~статье представлена архитектура инструмента моделирования VRRM 
в~терминах систем массового обслуживания (СМО). С~по\-мощью разработанного 
инструмента проводится анализ практического сценария с~тремя поставщиками 
и~различными типами SLA и~исследуются показатели производительности при 
изменении нагрузки на трафик и~SLA.
  
  В~работах~\cite{6-sop, 7-sop} рассматривается теоретическая основа для 
многооператорного планирования (Multi-Operator Scheduling, MOS). Благодаря 
динамической адаптации к~каналу и~нагрузке централизованный подход 
максимизирует спектральную эффективность для нескольких операторов 
с~полным контролем над гарантиями совместного использования. 
  
  В данной работе рассматривается сценарий функционирования одной соты 
беспроводной сети связи, в~которой активировано~$S$~слайсов, модуль 
нарезки делит между ними~$C$~единиц радиоресурсов. Каждый слайс 
предоставляет пользователям услугу связи, предполагающую непрерывную 
передачу данных с~определенным выделенным ресурсом, скоростью передачи, 
не менее $a_s\hm\geq 0$ и~не более $b_s\hm\geq a_s$, $s\hm\in S$. 
При этом скорость передачи является переменной: в~каждый момент времени 
она пересчитывается и~зависит от числа активных сессий в~каждом слайсе. 
Предполагается, что ресурс в~рамках одного слайса распределяется поровну 
между пользователями. В~работе описана модель в~виде 
СМО, предложен алгоритм разделения радиоресурсов, описана 
работа средства имитационного моделирования, проведен численный 
эксперимент и~анализ полученных результатов.

\section{Математическая модель}

  Пусть в~многолинейную СМО
поступает~$S$~потоков заявок, соответствующих запросам на передачу 
данных от пользователей~$S$~различных слайсов. Потоки являются 
пуассоновскими с~интенсивностями~$\lambda_s$, $s\hm\in S$. Объемы  
заявок~--- независимые случайные величины, распределенные по 
экспоненциальному закону с~параметрами~$1/\mu_s$, $s\hm\in S$, а~скорость 
обслуживания заявки определяется объемом выделенного ей ресурса. 
  
  Пусть общий объем ресурсов СМО для обслуживания заявок равен~$C$. 
Количество ресурсов, выделяемых заявке, принятой на обслуживание, зависит 
от состояния системы и~может варьироваться в~диапазоне $[a_s, b_s]$, $s\hm\in 
S$. При этом после каж\-до\-го поступления либо ухода заявки происходит 
перераспределение ресурсов между слайсами.
  
  Определим случайный процесс $X(t)\hm= \{ m_1(t), m_2(t), \ldots , m_S(t)\}$, 
где~$m_s$, $s\hm\in S$,~--- число заявок в~слайсе в~момент времени~$t$, 
причем 
  $$
  m_s\in \left\{ 0,1,\ldots , \left\lceil \fr{C}{a_s}\right\rceil \right\}\,,\enskip s\in S\,.
  $$
  %
  Тогда пространство возможных состояний процесса имеет вид:
  $$
  \mathrm{X}=\left\{ \left( m_1, m_2, \ldots , m_s\right)\,,\ \sum\limits^S_{s=1} 
m_s a_s\leq C\right\}\,.
  $$
  
  Обозначим через~$r_s$, $s\hm\in S$, количество выделенного ресурса одной 
заявке в~слайсе~$s$, $s\hm\in S$. Тогда интенсивности обслуживания заявок 
соответствующих слайсов определятся как~$r_s\mu_s$, $s\hm\in S$.
  
  Для обеспечения изоляции слайсов обозначим через~$\overline{R}_s$ объем 
ресурсов, который гарантированно выделен слайсу~$s$, $s\hm\in S$, а~через 
$\overline{M}_s\hm= \overline{R}_s/a_s$~--- число заявок, которое 
гарантированно может быть принято в~слайсе~$s$, $s\hm\in S$. Слайсы, число 
заявок в~которых превышает гарантированное значение, будем называть 
нарушителями. В~случае нехватки ресурсов и~наличия  
слай\-сов-на\-ру\-ши\-те\-лей поступившая заявка другого слайса может 
вытеснить одну или несколько заявок слай\-сов-на\-ру\-ши\-те\-лей. В~случае 
нехватки ресурсов и~отсутствия нарушителей, а~также в~случае когда все 
слайсы нарушают, поступающая заявка будет сброшена.
  
  Рассмотрим подробнее возможные события при поступлении сессий 
в~состоянии $(m_1, m_2, \ldots , m_s)\hm\in \mathrm{X}$. Пусть в~систему 
поступает заявка слайса~$s$, $s\hm\in S$. Тогда возможны следующие случаи:
  \begin{itemize}
  \item $(m_1, m_2, \ldots , m_s+1, \ldots , m_S)\hm\in \mathrm{X}$, т.\,е.\ 
предо\-став\-ле\-ние минимального количества ресурса для поступающей заявки 
возможно; в~этом случае заявка встает на обслуживание, а~случайный процесс 
переходит в~состояние  $(m_1, m_2, \ldots , m_s+1,\ldots , m_S)$;
  \item  $(m_1, m_2, \ldots , m_s+1, \ldots , m_S)\not= \mathrm{X}$, т.\,е.\ не 
гарантируется предоставление минимально тре\-бу\-емо\-го количества ресурса, 
тогда:
\begin{itemize}
  \item  если $m_s\hm< \overline{M}_s$ и~$m_k\hm> \overline{M}_k$, $k\hm\in S$, $k\not= 
s$, то выполняется освобождение ресурсов слай\-са-на\-ру\-ши\-те\-ля. 
Алгоритм сброса заявок приведен в~разд.~3;
  \item  в~остальных случаях поступающая заявка будет сброшена.
  \end{itemize}
  \end{itemize}
  
  Таким образом, множество состояний сброса заявок при поступлении:
  \begin{multline*}
  D_s=\left\{\vphantom{\left(\overline{M}_k\right)}\left (m_1, m_2, \ldots , m_S\right): {}\right.\\
{}:  \left( \left(m_1, m_2, \ldots , m_s+1, \ldots , m_S\right)\not= 
\mathrm{X}\right)\cap{}\\ 
\left.{}\cap\left( \left( m_s\geq \overline{M}_s\right) \cup \left( m_k\leq 
\overline{M}_k,\ k\in S,\ k\not=s\right)\right)
  \right\}\,.
\end{multline*}
  
  Множество состояний прерывания обслуживания:
\begin{multline*}
  B_k=\left\{ \vphantom{\left(\overline{M}_k\right)}
  \left( m_1, m_2, \ldots , m_S\right) :{}\right.\\
  {}: \left( \left( m_1, m_2, \ldots , m_s+1, \ldots , 
m_S\right) \not=\mathrm{X}\right) \cap {}\\
\left.{}\cap\left( \left( m_s<\overline{M}_s\right) \cup 
\left( m_k>\overline{M}_k,\ k\in S\,,\ k\not= s\right)\right)\right\}\,.
\end{multline*}

\section{Алгоритм выбора сбрасываемой заявки}

  Для выбора сбрасываемой заявки при наличии нескольких  
слай\-сов-на\-ру\-ши\-те\-лей введем понятие веса слайса, показывающего, 
насколько сильно слайс нарушает границы других слайсов. Ниже приведены 
три возможные формулы для вычисления весов: 
\begin{align*}
 w_s^{(1)}&=\begin{cases}
  1\,, & m_s\leq \overline{M}_s\,;\\
  \fr{1}{m_s-\overline{M}_s+1}\,, & m_s>\overline{M}_s\,;
  \end{cases}\\
  w_s^{(2)}&=\begin{cases}
  1\,, & m_s\leq \overline{M}_s\,;\\
  \fr{\overline{M}_s}{m_s}\,, & m_s>\overline{M}_s\,;
  \end{cases}\\
  w_s^{(3)}&=\begin{cases}
  1\,, & m_s\leq \overline{M}_s\,;\\
  w=const<1, & m_s>\overline{M}_s\,.
  \end{cases} %\label{e3-sop}
\end{align*}
 
  Освобождение ресурсов требуется в~случае, если заявка поступает в~слайс, 
который не является нарушителем, при этом она не может быть принята 
в~сис\-те\-му из-за нехватки свободного ресурса: 
\begin{multline*}
\left\{ \left( \left( m_1, m_2, \ldots , m_s+1, \ldots , m_S\right) \notin 
\mathrm{X}\right) :{}\right.\\
\left.{}: \left( m_s+1\right)a_s\leq R_s\right\}\,.
\end{multline*}

 \textbf{Шаг 1:} ищем слайс с~наименьшим весом, т.\,е.\ $s^*\hm\in S$: $w_{s^*}\hm= 
\min \{ w_r, r\hm\in S\}$.
  
  \textbf{Шаг 2:} сбрасываем заявку найденного слайса $(m_1, m_2, \ldots , m_{s^*}-1, 
\ldots , m_S)$.
  
  Шаги 1 и~2 выполняются, пока не будет удовле\-тво\-ре\-но условие $C\hm- 
\sum\nolimits^S_{i=1} m_i a_i\hm\geq a_s$. Отметим, что в~случае, когда 
у~нескольких слайсов вес минимален, выбор слайса для сброса заявки 
происходит случайным образом.
  
  
  \section{Распределение ресурсов}
  
  В состоянии избытка ресурсов 
  $$
  \mathrm{X}_0= \left(\! \left(m_1, m_2, \ldots , 
m_s+1, \ldots, m_S\right): \ \!\sum\limits^S_{s=1}\! m_s b_s \hm\leq C\!\right)
\hspace*{-0.94673pt}
$$ 
всем  заявкам во всех слайсах будет выделен максимальный объем ресурса~$b_s$.
  
  В состоянии ограниченных ресурсов $\mathrm{X}_1\hm= 
\mathrm{X}\backslash \mathrm{X}_0$ для справедливого и~эффективного 
распределения ресурсов решается задача оптимизации. Пусть скорости 
передачи данных в~слайсах имеют функцию полезности $U_s(r_s)\hm= \ln \left( 
r_s\right)$. Тогда задача выглядит следующим образом:
  \begin{gather}
  \sum\limits_{s\in S} w_s(m_s) m_s U_s(r_s)\to \max\,,\enskip s\in S\,;\notag\\
  \sum\limits_{s\in S} m_s r_s=C\,;\label{e5-sop}\\
  P=\left\{ \mathbf{r}\in \mathbf{R}^S:\ a_s\leq r_s\leq b_s\,,\ s\in S\right\}\,,
  \label{e6-sop}
  \end{gather}
где вектор~$\mathbf{r}$ имеет вид $\mathbf{r}\hm= (r_1, r_2, \ldots , r_S)$. 
Решение задачи выполняется с~по\-мощью при\-бли\-жен\-но\-го метода 
проецирования градиента, в~котором итеративная процедура поиска максимума 
описывается соотношениями:
\begin{gather*}
\mathbf{d}_k=\mathbf{P}\nabla f(\mathbf{x}_k)\,;\\
\mathbf{x}_{k+1} =\mathbf{x}_k+\tau_k \mathbf{d}_k\,;\\
\tau_k>0:\ \mathbf{x}_{k+1}\in P\,,
\end{gather*}
где вектор~$\mathbf{x}$ является решением, а~$\mathbf{P}$~--- матрица 
проецирования на гиперплоскость~(\ref{e5-sop}):
$$
\mathbf{P}=\mathbf{I}-\mathbf{m}\left(\mathbf{m}
\mathbf{m}^{\mathrm{T}}\right)^{-1}\mathbf{m} = \mathbf{I} -\fr{1}{\sum\nolimits_{s\in S} 
m_s^2}\,\mathbf{m}^{\mathrm{T}}\mathbf{m}\,.
$$
    Здесь вектор $\mathbf{m}\hm= (m_1, m_2, \ldots , m_S)$~--- текущее 
состояние системы. Градиент функции полезности представляет собой  
век\-тор-стол\-бец:
  $$
  \nabla f(\mathbf{x}_k)=\left( \fr{w_s m_s}{r_s}\right)_{s\in S}\,.
  $$
  
  Длину шага~$\tau_k$ выбираем таким образом, чтобы не выйти за пределы 
области~$P$, задаваемой прямыми ограничениями задачи~(\ref{e6-sop}). 
В~качестве начального приближения~$\mathbf{x}_0$ удобно взять точку 
пересечения диагонали координатного параллелограмма~$P$, соединяющей 
точки $(a_1, a_2, \ldots , a_S)$ и~$(b_1, b_2, \ldots , b_S)$ 
с~гиперплоскостью~(\ref{e5-sop}). Данная точка находится при решении 
системы линейных уравнений:
  \begin{equation*}
  \left.
  \begin{array}{c}
  (b_S-a_S)x_1-(b_1-a_1)x_S=a_1b_S-a_Sb_1\,;\\[6pt]
  (b_S-a_S)x_2-(b_2-a_2)x_S=a_2b_S-a_Sb_2\,;\\[6pt]
  \cdots\\[6pt]
  \hspace*{-8mm}(b_S-a_S)x_{S-1}-(b_{S-1}-a_{S-1})x_S={}\\[6pt]
  \hspace*{32mm}{}=a_{S-1}b_S - a_Sb_{S-1}\,;\\[6pt]
  \hspace*{-5mm}m_1x_1+\cdots + m_S x_S=C\,.
  \end{array}
  \right.
  \end{equation*}
  
  \begin{figure*}[b] %fig1
\vspace*{-4pt}
 \begin{center}
 \mbox{%
 \epsfxsize=159.488mm 
 \epsfbox{sop-1.eps}
 }
 \end{center}
   \vspace*{-9pt}
\Caption{Схема работы инструмента имитационного моделирования}
\end{figure*}
  
  Итеративная процедура обеспечивает движение по  
гиперплоскости~(\ref{e5-sop}) в~направлении возрастания функции полезности 
до границы области~$P$, где и~будет найдено решение.

\vspace*{-6pt}
  
  \section{Описание работы средства имитационного моделирования}
  
  \vspace*{-3pt}
  
  Общая схема работы имитатора изображена на рис.~1. На шаге 
инициализации входных па\-ра\-мет\-ров задаются начальные параметры: 
$\lambda_s$, $\mu_s$, $\overline{M}_s$, $a_s$, $b_s$, $s\hm\in S$, $C$, $\max$~--- 
максимальное число принятых заявок в~системе. Затем выполняется запуск 
имитационного моделирования. Далее определяется ближайшее событие: 

\begin{enumerate}[(1)]
\item если это поступление, то выполняется проверка на достаточность ресурсов:
\begin{enumerate}[({1}.1)] 
\item если ресурсов хватает, то добавляется заявка, пересчитываются ресурсы, 
и~определяется продолжение моделирования; 
\item если ресурсов не хватает, то 
проверяется, является ли слайс нарушителем: 
\begin{enumerate}[({1.2.}1)]
\item если является, то входящая 
заявка считается заблокированной; 
\item если не является, то запускается 
процесс поиска нарушителя и~освобождение ресурсов, после чего происходит 
добавление заявки, пересчитываются ресурсы и~определяется продолжение 
моделирования;
\end{enumerate}
\end{enumerate}
\item если ближайшее событие~--- это обслуживание, то 
выполняется освобождение и~перерасчет ресурсов.
\end{enumerate}

\vspace*{-8pt}

  
  \section{Численный эксперимент}
  
  \vspace*{-2pt}
  
  Для численного эксперимента рассматривается диапазон 10~МГц в~сети LTE
  (long-term evolution). 
Минимально допустимая скорость в~таком диапазон составляет 0,75~Мбит/с 
$(a_s\hm=0{,}75$, $s\hm\in S)$,\linebreak максимальная~--- 79,9~Мбит/с ($b_s\hm= 
79{,}9$, $s\hm\in S$). Максимальный объем ресурса, который может 
единовременно предоставлять базовая станция в~диапазоне 10~МГц,~--- 
450~Мбит/с, для расчетов используются несколько значений объема ресурсов: 
$$
C= [325,350,375,400,425,450]\,.
$$
 Данный ресурс будет разделен на трех 
($S\hm=3$) виртуальных операторов с~гарантированным выделенным ресурсом 
$\overline{R}_1\hm=225$, $\overline{R}_2\hm= 150$ и~$\overline{R}_3\hm= 75$. 
Интенсивности поступлений $\lambda_1\hm= \lambda_2\hm= \lambda_3\hm=30$, 
интенсивности обслуживания $\mu_1\hm= \mu_2\hm= \mu_3\hm= 30$ оставим 
равными для всех операторов.
  
  На рис.~2,\,\textit{а} видно, что когда ресурсов не хватает для обеспечения 
гарантированной скорости для всех операторов, то и~вероятность блокировки 
выше. Далее при увеличении ресурса видно, что для оператора, которому 
предоставлено больше гарантированных ресурсов, уменьшение вероятности 
блокировки происходит быстрее.
  

  На рис.~2,\,\textit{б} видно, что для оператора, которому предоставляется больший 
гарантированный ресурс, увеличивается средний размер слайса, в~то время как 
для оператора, которому гарантируется меньший ресурс, этот параметр 
меняется незначительно. Это связано со способом решения оптимального 
распределения ресурса, описанного в~данной работе.
  
  На рис.~2,\,\textit{в} отображен график, показывающий изменение доли времени, когда 
оператор является нарушителем. В~связи с~тем что общий размер \mbox{ресурса} 
растет, гарантированный объем остается неизменным, а~интенсивности 
поступления~--- фиксированные для каждого оператора, можно наблюдать 
резкое увеличение значений этого параметра для оператора~3.
  
   
\vspace*{-8pt}

  
  \section{Заключение}
  
  \vspace*{-2pt}
  
  В работе описана СМО с~ограниченными 
ресурсами с~распределением ресурсов в~зависимости от
веса слайса сети. 
Предложен и~реализован алгоритм освобождения ресурсов  
слай\-са\-ми-на\-ру\-ши\-те\-ля\-ми. Предложен и~реализован алгоритм 
распределения\linebreak\vspace*{-12pt}

{ \begin{center}  %fig2
 \vspace*{-3pt}
   \mbox{%
 \epsfxsize=79.374mm 
 \epsfbox{sop-2.eps}
 }

\end{center}

\noindent
{{\figurename~2}\ \ \small{
Вероятность блокировки~(\textit{а}),  
средний размер слайса~(\textit{б}) 
и~доля времени, когда слайс находится в~состоянии нарушителя~(\textit{в})
в~зависимости от объема ресурсов: \textit{1}~--- VNO~1; 
\textit{2}~--- VNO~2; \textit{3}~--- VNO~3
}}}

\vspace*{12pt}





\noindent
 ресурсов на основе весов слайсов. Разработано средство 
имитационного моделирования механизма распределения ресурсов. Проведен 
численный эксперимент для трех слайсов. В~дальнейшем планируется 
разработать модель, в~которой перераспределение слайсов происходит не при 
каждом изменении состояния процесса, а~при выполнении некоторого 
критерия.

\vspace*{-8pt}
  
{\small\frenchspacing
 {%\baselineskip=10.8pt
 \addcontentsline{toc}{section}{References}
 \begin{thebibliography}{9}
 
 \vspace*{-1pt}
 
 
\bibitem{1-sop}
3GPP TS 23.501 V15.4.0. System architecture for the 5G System, 2018. {\sf 
http://www.3gpp.org/ftp//Specs/ archive/23\_series/23.501/23501-f40.zip}.
\bibitem{2-sop}
ITU-T Rec. Y.3101. Requirements of the IMT-2020 network, 2018. {\sf 
https://www.itu.int/rec/dologin\_pub.asp?\linebreak lang=e\&id=T-REC-Y.3101-201801-I!!PDF-E\&type= items}.
\bibitem{3-sop}
\Au{P$\acute{\mbox{e}}$rez-Romero~J., 
Sallent~O., Ferr$\acute{\mbox{u}}$s~R., 
\mbox{Agust{\!\!\ptb{$\acute{\mbox{\i}}$}}}~R.} On 
the configuration of radio resource management in a~sliced RAN~// IEEE/IFIP 
Network Operations and Management Symposium.~--- IEEE, 2018. P.~1--6. doi: 
10.1109/ NOMS.2018.8406280.
\bibitem{4-sop} 
\Au{Rouzbehani B., Correia~L.\,M., Caeiro~L.} An SLA-based method for radio resource slicing 
and allocation in virtual %\linebreak\vspace*{-12pt}
%\columnbreak 
%\noindent
 RANs~// IRACON 7th MC and Technical Meeting.~--- 
Cartagena, Spain, 2018. Cost Action 15104 TD(18)07034. P.~1--7.
\bibitem{5-sop} 
\Au{Ageev K., Garibyan~A., Golskaya~A., Gaidamaka~Yu., Sopin~E., Samouylov~K., Correia~L.} 
Modelling of virtual radio resources slicing in 5G networks~// 
Information technologies and 
mathematical modelling. Queueing theory and applications~/
Eds. A.~Dudin, A.~Nazarov, A.~Moiseev.~--- Communications in 
computer and information science ser.~--- Cham: Springer, 2019.  
Vol.~1109. P.~150--161. doi: 10.1007/978-3-030-33388-1\_13.
\bibitem{6-sop} 
\Au{Malanchini I., Valentin~S., Aydin~O.} An analysis of\linebreak gen\-er\-al\-ized 
resource sharing for multiple 
operators in cel\-lu\-lar networks~// 25th Annual Symposium
 (In\-ter\-na\-tion\-al) on Personal, Indoor, 
and Mobile Radio Com\-mu\-ni\-ca\-tion.~--- IEEE, 2014. P.~1157--1162. doi: 
10.1109/\linebreak \mbox{PIMRC}.2014.7136342.
\bibitem{7-sop}
\Au{Malanchini I., Valentin~S., Aydin~O.} Wireless resource sharing for multiple operators: 
Generalization, fairness, and the value of prediction~// Comput. Netw., 2016. Vol.~100. 
P.~110--123. doi: 10.1016/j.comnet.2016.02.014.
\end{thebibliography}

 }
 }

\end{multicols}

\vspace*{-6pt}

\hfill{\small\textit{Поступила в~редакцию 15.07.20}}

\vspace*{8pt}

%\pagebreak

%\newpage

%\vspace*{-28pt}

\hrule

\vspace*{2pt}

\hrule

%\vspace*{-2pt}

\def\tit{ANALYSIS OF~THE~NETWORK SLICING MECHANISMS WITH~GUARANTEED ALLOCATED 
RESOURCES\\ FOR~VARIOUS TRAFFIC TYPES}


\def\titkol{Analysis of~the~network slicing mechanisms with~guaranteed allocated 
resources for~various traffic types}

\def\aut{K.\,A.~Ageev$^1$, E.\,S.~Sopin$^{1,2}$, N.\,V.~Yarkina$^1$, K.\,E.~Samouylov$^{1,2}$, 
and~S.\,Ya.~Shorgin$^2$}

\def\autkol{K.\,A.~Ageev, E.\,S.~Sopin, N.\,V.~Yarkina, 
%K.\,E.~Samouylov$^{1,2}$, and~S.\,Ya.~Shorgin$^2$ 
et al.}

\titel{\tit}{\aut}{\autkol}{\titkol}

\vspace*{-9pt}


\noindent
$^1$Peoples' Friendship University of Russia (RUDN University), 6~Miklukho-Maklaya Str., Moscow 
117198, Russian\linebreak
$\hphantom{^1}$Federation

\noindent
$^2$Institute of Informatics Problems, Federal Research Center ``Computer Sciences and Control'' of the 
Russian\linebreak
$\hphantom{^1}$Academy of Sciences; 44-2~Vavilov Str., Moscow 119133, Russian Federation

\def\leftfootline{\small{\textbf{\thepage}
\hfill INFORMATIKA I EE PRIMENENIYA~--- INFORMATICS AND
APPLICATIONS\ \ \ 2020\ \ \ volume~14\ \ \ issue\ 3}
}%
 \def\rightfootline{\small{INFORMATIKA I EE PRIMENENIYA~---
INFORMATICS AND APPLICATIONS\ \ \ 2020\ \ \ volume~14\ \ \ issue\ 3
\hfill \textbf{\thepage}}}

\vspace*{3pt} 




\Abste{Network slicing is one of the key capabilities of
 modern networks, allowing several virtual mobile operators 
 to use the physical resources of one base station. 
 This allows operators and resource owners (tenants) to 
 lease and manage several dedicated logical networks with 
 specific functionality working on top of a~common infrastructure. 
 Each of these logical networks is called a~network slice and can 
 be adapted to provide certain system behavior to maintain 
 a~specified level of quality of service indicators. The paper
  describes the developed mathematical framework of the network
   slicing mechanisms 
and analyzes it by means of extensive simulations.}

\KWE{simulation modeling; queuing system; limited resources; network slicing}

\DOI{10.14357/19922264200314} 

\vspace*{-20pt}

\Ack
\noindent
The reported study was funded by the ``RUDN University Program 5-100'' and in
part by RFBR, projects  
Nos.\,19-07-00933 and 19-37-90147.

\vspace*{4pt}

 \begin{multicols}{2}

\renewcommand{\bibname}{\protect\rmfamily References}
%\renewcommand{\bibname}{\large\protect\rm References}

{\small\frenchspacing
 {%\baselineskip=10.8pt
 \addcontentsline{toc}{section}{References}
 \begin{thebibliography}{9}
 
 \vspace*{-2pt}
 
\bibitem{1-sop-1}
3GPP TS 23.501 V15.4.0. 2018. System architecture for the 5G System. 
Available at: {\sf 
http://www.3gpp.org/ftp// Specs/archive/23\_series/23.501/23501-f40.zip} 
(accessed June~15, 2020).
\bibitem{2-sop-1}
ITU-T Rec. Y.3101. 2018. Requirements of the IMT-2020 network. Available at: {\sf 
https://www.itu.int/rec/ dologin\_pub.asp?lang=e\&id=T-REC-Y.3101-201801-I!!PDF-E\&type=items} 
(accessed June~15, 2020).
\bibitem{3-sop-1}
\Aue{P$\acute{\mbox{e}}$rez-Romero, J., O.~Sallent,
 R.~Ferr$\acute{\mbox{u}}$s, and 
R.~\mbox{Agust{\!\!\ptb{$\acute{\mbox{\i}}$}}}}. 2018. On the configuration of radio resource management in a sliced 
RAN. \textit{IEEE/IFIP Network Operations and Management Symposium}.
 IEEE. 1--6.  doi: 10.1109/ NOMS.2018.8406280.
\bibitem{4-sop-1}
\Aue{Rouzbehani, B., L.\,M.~Correia, and L.~Caeiro.} 2018. An SLA-based method for radio resource 
slicing and allocation in virtual RANs. \textit{IRACON 7th MC and Technical Meeting}. Cartagena. 
TD(18)07034. 1--7.
\bibitem{5-sop-1}
\Aue{Ageev, K., A.~Garibyan, A.~Golskaya, Yu.~Gaidamaka, E.~Sopin, K.~Samouylov, and L.~Correia.} 
2019. Modelling of virtual radio resources slicing in 5G networks. 
\textit{Information technologies and 
mathematical modelling. Queueing theory and applications}. Eds. 
A.~Dudin, A.~Nazarov, and A.~Moiseev. 
Communications in computer and information science ser. Cham:
Springer. 1109:150--161. 
\bibitem{6-sop-1}
\Aue{Malanchini, I., S.~Valentin, and O.~Aydin.} 2014. An analysis of generalized resource sharing for 
multiple operators in cellular networks. \textit{25th Annual Symposium (International)
on Personal,  Indoor, and Mobile Radio Communication}. IEEE. 1157--1162. 
doi:  10.1109/PIMRC.2014.7136342.
\bibitem{7-sop-1}
\Aue{Malanchini, I., S.~Valentin, and O.~Aydin.} 2016. Wireless resource sharing for multiple operators: 
Generalization, fairness, and the value of prediction. \textit{Comput. Netw.} 100:110--123.
doi: 10.1016/j.comnet.2016.02.014.

\end{thebibliography}

 }
 }

\end{multicols}

\vspace*{-6pt}

\hfill{\small\textit{Received July 15, 2020}}

%\pagebreak

%\vspace*{-24pt}



\Contr

\noindent
\textbf{Ageev Kirill A.} (b.\ 1993)~--- PhD student, Peoples' Friendship University of Russia (RUDN 
University), 6~Miklukho-Maklaya Str., Moscow 117198, Russian Federation; \mbox{ageev-ka@rudn.ru}

\vspace*{3pt}

\noindent
\textbf{Sopin Eduard S.} (b.\ 1987)~--- Candidate of Science in physics and mathematics, associate 
professor, Peoples' Friendship University of Russia (RUDN University), 6~Miklukho-Maklaya Str., 
Moscow 117198, Russian Federation; senior scientist, Institute of Informatics Problems, Federal Research 
Center ``Computer Sciences and Control'' of the Russian Academy of Sciences; 44-2 Vavilov Str., Moscow 
119133, Russian Federation; \mbox{sopin-es@rudn.ru}

\vspace*{3pt}

\noindent
\textbf{Yarkina Natalia V.} (b.\ 1979)~--- Candidate of Science in physics and mathematics, associate 
professor, Peoples' Friendship University of Russia (RUDN University), 6~Miklukho-Maklaya Str., 
Moscow 117198, Russian Federation; \mbox{yarkina-nv@rudn.ru}

\vspace*{3pt}

\noindent
\textbf{Samouylov Konstantin E.} (b.\ 1955)~--- Doctor of Science in technology, professor, Head of 
Department, Peoples' Friendship University of Russia (RUDN University), 6~Miklukho-Maklaya Str., 
Moscow 117198, Russian Federation; senior scientist, Institute of Informatics Problems, Federal Research 
Center ``Computer Sciences and Control'' of the Russian Academy of Sciences; 44-2~Vavilov Str., Moscow 
119133, Russian Federation; \mbox{samuylov\_ke@rudn.university}

\vspace*{3pt}

\noindent
\textbf{Shorgin Sergey Ya.} (b.\ 1952)~--- Doctor of Science in physics and mathematics, professor, 
principal scientist, Institute of Informatics Problems, Federal Research Center ``Computer Sciences and 
Control'' of the Russian Academy of Sciences, 44-2~Vavilov Str., Moscow 119333, Russian Federation; 
\mbox{sshorgin@ipiran.ru}
\label{end\stat}

\renewcommand{\bibname}{\protect\rm Литература}            %14

\def\stat{shnurkov}

\def\tit{АНАЛИТИЧЕСКОЕ РЕШЕНИЕ ЗАДАЧИ ОПТИМАЛЬНОГО УПРАВЛЕНИЯ ПОЛУМАРКОВСКИМ ПРОЦЕССОМ\\ 
С~КОНЕЧНЫМ МНОЖЕСТВОМ СОСТОЯНИЙ$^*$}

\def\titkol{Аналитическое решение задачи оптимального управления полумарковским 
процессом} %с~конечным множеством состояний}

\def\aut{П.\,В.~Шнурков$^1$, А.\,К.~Горшенин$^2$, В.\,В.~Белоусов$^3$}

\def\autkol{П.\,В.~Шнурков, А.\,К.~Горшенин, В.\,В.~Белоусов}

\titel{\tit}{\aut}{\autkol}{\titkol}

\index{Шнурков П.\,В.}
\index{Горшенин А.\,К.}
\index{Белоусов В.\,В.}
\index{Shnurkov P.\,V.}
\index{Gorshenin A.\,K.}
\index{Belousov V.\,V.}


{\renewcommand{\thefootnote}{\fnsymbol{footnote}} \footnotetext[1]
{Работа выполнена при частичной поддержке РФФИ (проект 15-07-05316).}}


\renewcommand{\thefootnote}{\arabic{footnote}}
\footnotetext[1]{Национальный исследовательский университет <<Высшая школа экономики>>, 
\mbox{pshnurkov@hse.ru}}
\footnotetext[2]{Институт проблем информатики Федерального исследовательского центра <<Информатика 
и~управ\-ле\-ние>> Российской академии наук, \mbox{agorshenin@frccsc.ru}}
\footnotetext[3]{Институт проблем информатики Федерального исследовательского центра <<Информатика 
и~управление>> Российской академии наук, \mbox{vbelousov@ipiran.ru}}

%\vspace*{-6pt}

\Abst{Настоящее исследование посвящено теоретическому обоснованию нового метода 
нахождения оптимальной стратегии управления полумарковским процессом с~конечным 
множеством состояний. Рассматриваются марковские рандомизированные стратегии 
управления, определяемые конечным набором вероятностных мер, соответствующих 
каждому состоянию. Характеристикой качества управления служит стационарный 
стоимостной показатель. Данный показатель представляет собой дроб\-но-ли\-ней\-ный 
интегральный функционал от набора вероятностных мер, задающих стратегию управления. 
Для этого функционала известны явные аналитические представления подынтегральных 
функций числителя и~знаменателя. Дальнейшие результаты основываются на новой 
усиленной и~обобщенной форме теоремы об экстремуме дроб\-но-ли\-ней\-но\-го интегрального 
функционала. Доказывается, что проблемы существования оптимальной стратегии управления 
полумарковским процессом и~ее нахождения сводятся к~задаче численного исследования 
на глобальный экстремум заданной функции от конечного числа вещественных переменных.}

\KW{оптимальное управление полумарковским процессом; стационарный стоимостной 
показатель качества управления; дроб\-но-ли\-ней\-ный интегральный функционал}

\DOI{10.14357/19922264160408} 

\vspace*{9pt}


\vskip 10pt plus 9pt minus 6pt

\thispagestyle{headings}

\begin{multicols}{2}

\label{st\stat}

\section{Введение}

Теория оптимального управления марковскими и~полумарковскими случайными 
процессами интенсивно развивается с~начала 1960-х~гг. Еще в~первых 
основополагающих исследованиях рассматривались не только проблемы существования 
оптимальных стратегий управления, но и~способы нахождения этих стратегий. 

Для решения таких проблем, имеющих алгоритмическое содержание, использовались 
открытые незадолго до этого мощные методы прикладной математики: линейное 
программирование и~динамическое программирование. Отметим, прежде всего, 
классическую работу Р.~Ховарда~\cite{1}, в~которой метод динамического 
программирования был применен для решения проблемы оптимального управления 
марковским процессом с~непрерывным временем. В~дальнейшем В.\,В.~Рыков~\cite{2} 
доказал, что для аналогичной модели управления марковским процессом с~учетом 
переоценки оптимальной стратегией также является стационарная.

Важную роль в~развитии теории управления случайными процессами сыграла работа 
В.~Джевелла~\cite{3}, в~которой были впервые рассмотрены полумарковские модели 
управления для вариантов с~переоценкой и~без переоценки. Данная работа была 
переведена на русский язык и~послужила основой для многих последующих работ 
отечественных и~зарубежных специалистов. В~частности, Б.~Фокс показал~\cite{4}, 
что оптимальной стратегией управления полумарковским процессом в~варианте без 
переоценки является стационарная; аналогичные результаты были получены Э.~Денардо 
и~для варианта с~переоценкой~\cite{5}.

Среди последующих исследований алгоритмической направленности отметим работы 
Р.~Ховарда~\cite{6}, Б.~Фокса~\cite{4}, а также С.~Осаки и~Х.~Майна~\cite{7}. 
В~этих работах для нахождения оптимальных стратегий управления полумарковскими 
процессами использовался метод линейного программирования.

В 1970~г.\ была опубликована фундаментальная монография Х.~Майна и~С.~Осаки~\cite{8}, 
переведенная на русский язык в~1977~г., в~которой были систе\-ма\-ти\-зи\-ро\-ва\-ны и~изложены 
основные результаты по теории оптимального управления марковскими и~полумарковскими 
случайными процессами. Фактически данная книга стала итогом исследований по проблемам 
стохастического управления\linebreak
 за~10~лет. Отметим, что в~этой монографии рас\-смат\-ри\-ва\-лись 
марковские и~полумарковские модели управления с~конечными множествами состояний 
и~допустимых решений, принимаемых \mbox{в~каждом} состоянии. Были получены принципиальные 
тео\-ре\-ти\-че\-ские результаты, заключающиеся в~том, что оптимальные стратегии управ\-ле\-ния 
для основных видов рас\-смат\-ри\-ва\-емых моделей с~переоценкой и~без переоценки являются 
детерминированными и~стационарными. Были разработаны и~обоснованы процедуры нахождения 
оптимальных стратегий управления. В~частности, для модели управления полумарковским 
процессом без переоценки, когда множество со\-сто\-яний образует один эргодический класс, 
а~показатель качества управления пред\-став\-ля\-ет собой стационарный средний удельный 
доход (см.~[8, гл.~5, п.~5.5]), процедура поиска оптимальной рандомизированной 
стратегии осуществлялась методом линейного программирования. Обратим особое внимание 
на данный результат, поскольку аналогичная модель управления полумарковским 
процессом будет рассмотрена в~настоящей работе.

Принципиальную роль в~развитии теории стохастического управления сыграла 
монография И.\,И.~Гихмана и~А.\,В.~Скорохода~\cite{9}. В~этой книге были впервые 
систематически изложены основы теории оптимального управления случайными процессами 
с~дискретным и~непрерывным временем, включая теорию управления процессами, которые 
описываются стохастическими дифференциальными уравнениями. Отдельно были рас\-смот\-ре\-ны 
проблемы управления марковскими процессами с~дискретным временем и~скачкообразными 
марковскими процессами с~непрерывным временем. Роли множеств состояний и~допустимых 
управ\-ле\-ний играли пространства весьма общей структуры. Для широких классов функционалов 
качества управ\-ле\-ния (так называемых эволюционных функционалов в~марковских моделях 
с~дискретным временем и~интегральных функционалов накопления в~марковских моделях 
с~непрерывным временем) были доказаны теоремы о~существовании и~формах пред\-став\-ле\-ния 
оптимальных стратегий управ\-ле\-ния. Было установлено, что для однородных марковских 
моделей оптимальные стратегии управ\-ле\-ния существуют, являются стационарными 
и~детерминированными. Иначе говоря, такие стратегии задаются детерминированными 
функциями, аргументом которых является со\-сто\-яние сис\-те\-мы в~момент принятия решения, 
и~не зависящими от самого момента принятия решения. Что же касается важного вопроса 
о~формах представления этих функций, то их можно охарактеризовать следующим образом. 
Были найдены функциональные уравнения, осложненные условием экстремума, которым 
удовле\-тво\-ря\-ют упомянутые функции. По существу эти соотношения пред\-став\-ля\-ют собой 
уравнения Беллмана для соответствующих динамических стохастических моделей.

Особо отметим, что в~монографии~\cite{9} не рас\-смат\-ри\-ва\-лись проблемы управления 
полумарковскими процессами. Однако дальнейшее развитие общей теории управления 
такими процессами шло по пути, идейно намеченному в~указанной книге.

В последующие годы развитие теории управ\-ле\-ния полумарковскими процессами 
осуществля-\linebreak лось по направлению усложнения моделей и~обобщения исходных предположений. 
Например,\linebreak в~работах~\cite{10, 11} рассмотрены управляемые по\-лумарковские процессы при 
весьма общих предположениях относительно характера пространств состояний и~управлений. 
Проблемы управления исследовались по отношению к~различным видам целевых показателей, 
обобщающих упомянутый выше стационарный показатель средней удельной прибыли. В~этих 
работах доказывается, что оптимальная стратегия управления по отношению к~каж\-до\-му из 
показателей существует и~является одной и~той же стационарной детерминированной 
стратегией, определяемой некоторой функцией, заданной на множестве со\-сто\-яний процесса. 
Об этой функции известно лишь то, что она удовлетворяет некоторому интегральному 
уравнению, которое по содержанию пред\-став\-ля\-ет собой уравнение Бел\-лма\-на для 
соответствующей задачи управ\-ления.

Среди исследований, предшествовавших настоящему, отметим работу 
В.\,А.~Каштанова~[12, гл. 13]. В этом разделе коллективной монографии~\cite{12} 
автором была рассмотрена проблема оптимального управления полумарковским 
процессом с~конечным множеством состояний и~множеством возможных решений, 
которое представляет собой произвольный интервал множества вещественных чисел. 
Модель относится к~виду моделей без переоценки, показателем качества управления 
служит стационарное значение среднего удельного дохода, определяемое аналогично 
классическим работам~\cite{3, 8}. Рандомизированное управление в~каждом состоянии 
определяется в~соответствии с~вероятностным распределением, совокупность которых 
задает\linebreak
 стратегию управления. В.\,А.~Каш\-та\-но\-вым было\linebreak сформулировано утверждение о том, 
что стацио\-нарное значение среднего удельного дохода представляет собой 
дроб\-но-ли\-ней\-ный интегральный функционал от набора вероятностных распределений, 
образующих стратегию управления. При этом\linebreak ранее~[12, гл.~10; 13] было уста\-нов\-ле\-но, 
что дроб\-но-ли\-ней\-ный функционал достигает экстремума на вырожденных распределениях. 
Отсюда естест-\linebreak венно следует, что оптимальная стратегия управ\-ле-ния является 
детерминированной и~должна\linebreak определяться точкой экстремума функции, представляющей 
собой отношение подынтегральных функций чис\-ли\-те\-ля и~знаменателя данного 
дроб\-но-ли\-ней\-но\-го функционала. Однако в~\cite{12} не были получены явные 
представления для указан-\linebreak ных функций. Кроме того, приведенный в~гл.~10 
монографии~\cite{12} вариант теоремы об экстремуме дроб\-но-ли\-ней\-но\-го 
интегрального функционала требовал проверки выполнения условия существования 
этого экстремума. Такие условия указаны не были. В~связи с~этими обстоятельствами 
использовать полученные в~\cite{12} результаты для доказательства существования 
оптимальной детерминированной стратегии управ\-ле\-ния полумарковским процессом и~для 
строгого обоснования способа нахождения такой стратегии оказалось невозможным.

Настоящее исследование посвящено теоретическому обоснованию нового метода 
нахождения\linebreak оптимальной стратегии управления полумарковским процессом с~конечным 
множеством со\-сто\-яний. Рассматриваются марковские рандомизи\-рованные стратегии 
управления, определяемые конеч\-ным набором вероятностных мер, соответствующих 
каждому состоянию. Показателем качества управления служит уже упоминавшийся 
классический  показатель~--- стационарное значение средней удельной прибыли. 
Доказано, что этот показатель представляет собой дроб\-но-ли\-ней\-ный интегральный 
функционал от набора вероятностных мер, задающих стратегию управления. При этом, 
в~отличие от~\cite{12}, получены явные аналитические представления для подынтегральных 
функций числителя и~знаменателя этого дроб\-но-ли\-ней\-но\-го\linebreak
 функционала. Дальнейшие 
результаты основываются на новой усиленной и~обобщенной форме\linebreak
 теоремы об экстремуме 
дроб\-но-ли\-ней\-но\-го интегрального функционала, впервые опубликованной 
в~работе П.\,В.~Шнуркова~\cite{14}. Согласно\linebreak
 утверж\-де\-нию этой теоремы, если 
существует глобальный экстремум так называемой основной функции дроб\-но-ли\-ней\-но\-го 
функционала, которая пред\-став\-ля\-ет собой отношение подынтегральных функций чис\-ли\-те\-ля 
и~знаменателя, то существует безусловный экстремум самого дроб\-но-ли\-ней\-но\-го 
функционала, который достигается на наборе вырожденных вероятностных распределений, 
сосредоточенных в~точке глобального экстремума. В~этом случае оптимальная стратегия 
управ\-ле\-ния существует, является стационарной и~детерминированной и~определяется точкой, 
в~которой основная\linebreak функция достигает глобального экстремума. Таким\linebreak образом, проблемы 
существования оптимальной стратегии управ\-ле\-ния полумарковским процессом и~ее 
нахождения сводятся к~задаче чис\-лен\-но\-го исследования на глобальный экстремум 
заданной функции от конечного чис\-ла вещественных переменных.

\section{Общее описание модели управления полумарковским случайным процессом}

Построим модель управления полумарковским случайным процессом, следуя общему 
подходу, принятому в~классических работах~\cite{3, 8}. Пусть $\xi(t)$~--- 
случайный полумарковский процесс с~конечным множеством состояний
$X\hm=\{1,2,\ldots, N\}$, $N\hm< \infty$. Обозначим через~$t_n$, $n=0,1,2,\ldots$, 
$t_0\hm=0$, случайные моменты изменения состояний данного процесса, 
$\theta_n\hm=t_{n+1}-t_n$, $n\hm=0,1,2,\ldots$, $\xi_n\hm=\xi(t_n)\hm=\xi(t_n+0)$, 
$n\hm=0,1,2,\ldots$ (предполагается, что траектории процесса~$\xi(t)$ 
непрерывны справа). Случайная последовательность~$\{\xi_n\}$
образует цепь Маркова, вложенную в~полумарковский процесс~$\xi(t)$.
Зададим набор измеримых пространств\linebreak $(U_1, \mathscr{B}_1), 
(U_2, \mathscr{B}_2), \ldots, (U_N, \mathscr{B}_N)$, где $U_i$~--- 
множество возможных допустимых управ\-ле\-ний, $\mathscr{B}_i$~--- $\sigma$-ал\-геб\-ра 
подмножеств множества~$U_i$, вклю\-ча\-ющая в~себя все одноточечные подмножества\linebreak  
множества~$U_i$, т.\,е.\ если $u_i \hm\in U_i$, то $\{u_i\} \hm\in \mathscr{B}_i$, 
$i\hm=1,2,\ldots, N$.
Пусть $\Gamma_i$~--- некоторое множество всевозможных вероятностных мер $\Psi_i 
\hm \in \Gamma_i$, заданных на $\sigma$-ал\-геб\-ре~$\mathscr{B}_i$, $i\hm=1,2,\ldots,N$.

Поскольку идейное содержание и~свойства вероятностных мер существенно используются 
в~данной работе, укажем на некоторые фундаментальные издания, в~которых 
изложена соответствующая тео\-рия. Понятие и~основные свойства вероятностной 
меры определены и~подробно проанализированы в~книге А.\,Н.~Ширяева~\cite[гл.~II]{15}. 
Глубокое изложение основ теории вероятностных мер имеется также в~книге 
А.\,А.~Боровкова~\cite{16}. Заметим попутно, что в~книге~\cite{16} имеются разделы, 
посвященные изложению основ теории полумарковских и~регенерирующих случайных процессов. 
Из зарубежных изданий отметим фундаментальную работу П.~Хеннекена и~А.~Тортра~\cite{17}, 
основная часть которой посвящена изложению математических основ теории вероятностей.

Введем специальное понятие вырожденной вероятностной меры, которое будет часто 
использоваться в~дальнейшем. Пусть $(U_0, \mathscr{B}_0)$~--- некоторое измеримое 
пространство, $\mathscr{B}_0$~--- $\sigma$-ал\-геб\-ра подмножеств множества~$U_0$, 
включающая в~себя все одноточечные подмножества этого множества.

\medskip

\noindent
\textbf{Определение 1.}\ Вероятностная мера~$\Psi^*$, заданная 
на $\sigma$-ал\-геб\-рe~$\mathscr{B}_0$, называется вырожденной, если существует 
такой элемент $u^* \hm\in U_0$, для которого выполняются условия $\Psi^*(\{u^*\})\hm=
1$, $\Psi^*(U_0 \setminus \{u^*\})\hm=0$, где $\{u^*\}=u^*$~--- 
множество, состоящее из единственной точки $u^* \hm\in U_0$. Соответствующая 
точка $u^* \hm\in U_0$ будет называться точкой сосредоточения вырожденной 
вероятностной меры~$\Psi^*$.
Таким образом, всякая вырожденная вероятностная мера~$\Psi^*$ определяется 
своей точкой сосредоточения~$u^*$. В~дальнейшем будем использовать 
обозначение~$\Psi_{u^*}^{*}$, имея в~виду, что вырожденная вероятностная мера~$\Psi^*$ 
сосредоточена в~точке~$u^*$.
Отметим также, что вырожденная вероятностная мера~$\Psi_{u^*}^{*}$ соответствует 
детерминированной величине, которая принимает фиксированное значение $u\hm=u^*$ 
с~вероятностью, равной единице.

\medskip

Обозначим через $\Gamma_0$ множество всех  вероятностных мер, заданных 
на измеримом пространстве ($U_0, \mathscr{B}_0$), а через~$\Gamma_0^*$~--- 
множество всех вырожденных вероятностных мер, заданных на этом пространстве, 
$\Gamma_0^*\hm\in \Gamma_0$. Аналогичные обозначения будут использоваться 
и~в~дальнейшем. Заметим, что множество~$\Gamma_0^*$ находится во взаимно
 однозначном соответствии с~множеством точек сосредоточения вырожденных 
 вероятностных мер, т.\,е.\ с~множеством~$U_0$.

Пусть $\Gamma_i^{*}$~--- множество всех вырожденных мер, заданных на 
$\sigma$-ал\-геб\-ре~$\mathscr{B}_i$, $\Gamma_i^{*}\hm\subset \Gamma_i$.
Произвольная вероятностная мера~$\Psi_i$ описывает случайную величину, 
принимающую значения в~$U_i$, а вырожденная мера~$\Psi_i^*$, сосредоточенная 
в~точке~$u_i^*$, соответствует детерминированной величине $u_i^*\hm\in U_i$.
Предполагается, что соответствующие конструкции определены на всех измеримых 
пространствах управлений $(U_1, \mathscr{B}_1), (U_2, \mathscr{B}_2), \ldots, 
(U_N,\mathscr{B}_N)$.

Предположим, что управления случайным полумарковским процессом~$\xi(t)$ 
осуществляются в~моменты времени~$t_n,$ $n\hm=0,1,2,\ldots,$
непосредственно после изменения состояния процесса. Если\linebreak 
$\xi_n\hm=\xi(t_n)\hm=i \hm\in X$, то значение управления представляет 
собой случайную величину~$u_n$, принимающую значения в~множестве допустимых 
управ\-ле\-ний~$U_i$ и~описываемую вероятностной мерой (распределе\-ни\-ем 
вероятностей) $\Psi_i \hm\in \Gamma_i$.
Будем предполагать, что при фиксированном условии $\xi_n\hm=\xi(t_n)=i$ 
управ\-ле\-ние определяется независимо от прошлого поведения процесса~$\xi(t)$ 
и~вероятностная мера~$\Psi_i$,
описывающая стохастическое управление~$u_n$, зависит только от состояния $i\hm\in X$.
Тогда выбор управ\-ле\-ний в~моменты изменения состояний $\{t_n, n\hm=0,1,2,\ldots \}$ 
описывается набором вероятностных мер (распределений вероятностей) 
$(\Psi_1, \Psi_2,\ldots, \Psi_N)$, 
$\Psi_i \hm\in \Gamma_i$, $i\hm=1,2,\ldots,N$.
Назовем любой такой набор стратегией управ\-ле\-ния полумарковским процессом~$\xi(t)$. 
По своим свойствам такая стратегия является марковской, однородной 
и~рандомизированной.

Следуя классической монографии П.~Халмоша~\cite[гл.~VII]{18}, 
рассмотрим декартово произведение пространств $U\hm=U_1\times U_2\times \cdots\times U_N$ 
и~соответствующих $\sigma$-ал\-гебр $\mathscr{B}\hm=\mathscr{B}_1\times \mathscr{B}_2
\times \cdots \times\mathscr{B}_N$. Обозначим через $\Psi\hm=\Psi_1\times \Psi_2\times \cdots
\times \Psi_N$ вероятностную меру на~$(U,\mathscr{B})$, определяемую как 
произведение мер $\Psi_1,\Psi_2, \ldots , \Psi_N$, где $\Psi_i \hm\in \Gamma_i$, 
$i\hm=1,2,\ldots,N$. Обозначим также через~$\Gamma$ множество вероятностных мер~$\Psi$, 
заданных на~$(U,\mathscr{B})$, которые пред\-став\-ля\-ют собой произведение 
мер $\Psi_1,\Psi_2, \ldots , \Psi_N$, где $\Psi_i \hm\in \Gamma_i$, $i\hm=1,2,\ldots,N$.
Множество~$\Gamma$ можно отож\-де\-ст\-вить с~множеством всех стратегий управ\-ле\-ния 
полумарковским процессом~$\xi(t)$.

Проблема оптимального управления полумар\-ковским процессом~$\xi(t)$ будет в~дальнейшем 
сформулирована в~виде задачи безусловного экстремума некоторого функционала 
$I(\Psi)\hm=I(\Psi_1,\Psi_2, \ldots , \Psi_N)$, заданного на множестве 
допустимых стратегий управления. Содержание показателя качества управления~$I(\Psi)$, 
аналитическое представление для него, а~также описание множества допустимых 
стратегий управления будут приведены в~последующих разделах данной работы.

Для получения дальнейших результатов потребуются различные вероятностные 
характеристики управляемого полумарковского процесса~$\xi(t)$. Как известно из
 общей теории полумарковских процессов~\cite{19, 20}, 
 основной вероятностной характеристикой такого процесса является так называемая 
 полумарковская функция. Определим эту функцию для процесса с~управлением 
 (см.~\cite[гл.~5]{8}):
\begin{multline}
Q_{ij}(t,u)=
{\sf P}\left(\xi_{n+1}=j,\theta_n<t \mid \xi_n=i, u_n=u\right)\,,\\
t\in [0,\infty)\,,\ u\in U_i\,;\ i,j\in X=\{1,2,\ldots,N\}\,. \label{e1}
\end{multline}
Используя полумарковские функции, можно получить вероятности перехода 
управляемой цепи Маркова~$\{\xi_n\}$:
\begin{multline}
p_{ij}(u)={\sf P}\left(\xi_{n+1}=j \mid \xi_n=i, u_n=u\right)= {}\\
{}=
\lim\limits_{t\rightarrow\infty}Q_{ij}(t,u)\,,\enskip
u\in U_i\,;\enskip i,j\in X\,, 
\label{e2}
\end{multline}
а также функции распределения длительностей пребывания полумарковского 
процесса~$\xi(t)$ в~соответствующих состояниях:

\noindent
\begin{multline}
H_{i}(t,u)={\sf P}\left(\theta_n<t \mid \xi_n=i, u_n=u\right)={}\\
{}=
\sum\limits_{j\in X}Q_{ij}(t,u)\,,\enskip
t\in [0,\infty)\,,\  u\in U_i\,; \  i\in X\,. 
\label{e3}
\end{multline}

Обозначим через
\begin{multline}
T_{i}(u)=\mathbf{E}\left[\theta_n \mid \xi_n=i, u_n=u\right]={}\\
{}=
\int\limits_0^{\infty}\left[1-H_i(t,u)\right]\,dt\,,\enskip
u\in U_i\,,\ i\in X\,, 
\label{e4}
\end{multline}
математические ожидания длительностей пребывания полумарковского процесса~$\xi(t)$ 
в~каждом из состояний.

Введенные выше характеристики~(1)--(4) определены для случая, когда 
в~момент изменения состояния~$t_n$ процесс оказывается в~состоянии~$i$ 
и~принимается решение $u\hm\in U_i$. При заданной стратегии управления 
$\Psi\hm=\left(\Psi_1,\Psi_2, \ldots , \Psi_N\right)$ можно записать 
соответствующие вероятностные характеристики без условия на управление, а~именно:
\begin{multline*}
Q_{ij}(t)={\sf P}\left(\xi_{n+1}=j,\theta_n<t \mid \xi_n=i\right)={}\\
{}=
\int\limits_{U_i}Q_{ij}(t,u) \,d\Psi_i(u)\,,\enskip 
t\in [0,\infty)\,,\ i,j\in X\,; %\label{e5}
\end{multline*}

\vspace*{-12pt}

\noindent
\begin{multline}
p_{ij}={\sf P}\left(\xi_{n+1}=j \mid \xi_n=i\right)=
\int\limits_{U_i}p_{ij}(u)\, d\Psi_i(u)\,,\\  
i,j\in X\,; 
\label{e6}
\end{multline}

\vspace*{-9pt}

\noindent
\begin{equation}
T_{i}=\mathbf{E}\left[\theta_n \mid \xi_n=i\right]=
\int\limits_{U_i}T_{i}(u)\,d\Psi_i(u)\,,\enskip i\in X\,. 
\label{e7}
\end{equation}
В дальнейшем будем предполагать, что для рас\-смат\-ри\-ва\-емой 
полумарковской модели заданы вероятностные характеристики 
$p_{ij}(u)$, $u\hm\in U_i$, $i,j\hm\in X$, и~$T_i(u)$, $u\hm\in U_i$, $i\hm\in X$, 
определяемые соотношениями~(\ref{e2}) и~(\ref{e4}). 
Для фиксированной стратегии управления $\Psi\hm=(\Psi_1, \Psi_2,\ldots, \Psi_N)$ 
соответствующие вероятностные характеристики~$p_{ij}$ и~ $T_i$, $i,j\hm\in X,$ 
определены равенствами~(\ref{e6}) и~(\ref{e7}) без условий на управление.

\section{Стационарный стоимостной показатель качества управления}

Определим некоторый стоимостной аддитивный функционал, связанный 
с~рассматриваемым полумарковским процессом~$\xi(t)$. По содержанию этот функционал 
представляет собой случайный\linebreak доход или прибыль, накопленную за период времени $[0,t]$. 
Определения такого функционала приведены в~основополагающих работах~[3; 8, гл.~5].\linebreak 
Обозначим через $\widetilde{v}(t)$, $t\hm\geq 0$, значение этого аддитивного 
функционала в~момент времени~$t$; $\widetilde{v}_n\hm=\widetilde{v}(t_n\hm+0)$~--- 
соответствующее значение непосредственно после очередного момента изменения\linebreak 
состояния~$t_n$, $n\hm=0,1,2,\ldots$; $\widetilde{v}_0\hm=v_0$~--- 
заданное начальное значение в~момент $t\hm=0$. Рассмотрим величину
\begin{multline}
d_i(u)=\mathbf{E}\left[\widetilde{v}_{n+1}-\widetilde{v}_n \mid \xi_n=i\,, 
u_n=u\right]\,,\\
u\in U_i\,, \enskip i\in X\,, \label{e8}
\end{multline}
представляющую собой математическое ожидание приращения стоимостного 
аддитивного функционала за период времени между последовательными 
изменениями состояния полумарковского процесса~$\xi(t)$. Тогда соответствующее 
математическое ожидание, вычисляемое без условия на решение, 
принимаемое в~момент времени~$t_n$, представляется в~виде:
\begin{equation*}
d_i=\mathbf{E}\left[\widetilde{v}_{n+1}-\widetilde{v}_n \mid \xi_n=i\right]=
\!\int\limits_{U_i}\!d_i(u)\,d\Psi_i(u)\,,\ i\in X \,. %\label{e9}
\end{equation*}

Предположим, что для заданной стратегии управ\-ле\-ния 
$\Psi\hm=(\Psi_1,\Psi_2,\ldots,\Psi_N)$ вложенная цепь Маркова~$\{\xi_n\}$ 
имеет ровно один класс возвратных положительных состояний (по терминологии, 
принятой в~\cite{8}, такое множество состояний называется эргодическим классом). 
Как известно~\cite[гл.~VIII]{15}, данное условие является необходимым 
и~достаточным для существования единственного\linebreak стационарного распределения. 
Обозначим это стационарное распределение цепи Маркова~$\{\xi_n\}$ через 
$\pi\hm=(\pi_1, \pi_2,\ldots, \pi_N)$. Заметим, что данное\linebreak распределение зависит  
от стратегии управления $\Psi\hm=(\Psi_1,\Psi_2,\ldots,\Psi_N)$. При указанном 
условии имеет место следующий результат, называемый эргодической теоремой 
для аддитивного стоимостного функционала:
\begin{equation}
I=\lim\limits_{t\rightarrow\infty}\fr{\mathbf{E}\widetilde{v}(t)}{t}=
\fr{\sum\nolimits_{i=1}^N d_i\pi_i}{\sum\nolimits_{i=1}^N T_i\pi_i}\,. 
\label{e10}
\end{equation}

Соотношение~(\ref{e10}) доказано в~работе~\cite[гл.~5]{8}. Заметим, что аналогичные 
результаты имеют мес\-то для гораздо более общих полумарковских моделей~\cite{10, 11}.

По своему прикладному содержанию величина, определяемая соотношением~(\ref{e10}), 
представляет собой
среднюю удельную прибыль, связанную с~эволюцией системы в~стационарном
режиме. Кроме того, величина~$I$ представляет собой функционал от
набора вероятностных распределений~$\Psi_{i}$, $i\hm\in\lbrace 1,\ldots
,N\rbrace $, определяющих стратегию управле-\linebreak\vspace*{-12pt}

\pagebreak

\noindent
ния системой. 
В~дальнейшем будем рассматривать стационарный стоимостной функционал 
$I\hm=I(\Psi_{1},\Psi_{2},\ldots , \Psi_{N})$ как
показатель качества управ\-ле\-ния системой и~построенным полумарковским
процессом~$\xi (t)$.

\section{Представление стационарного показателя в~форме
дробно-линейного интегрального функционала}

В данном разделе будет приведено утверждение об аналитическом
представлении стационарного стоимостного функционала~(\ref{e10}), 
служащего критерием качества управления в~рассматриваемой задаче управления 
полумарковским процессом.

\smallskip

\noindent
\textbf{Теорема 1.} \textit{Стационарный стоимостной показатель, 
определяемый равенством}~(\ref{e10}), \textit{представляет собой дроб\-но-ли\-ней\-ный
функционал от вероятностных распределений~$\Psi_{i}(u_{i})$,
$i\hm\in\{1,\dots,N\}$. Данный функционал задается
аналитически следующей формулой:}
\begin{multline}
I=I(\Psi_{1},\ldots, \Psi_{N})={}\\
\hspace*{-2mm}{}=\!
\fr{\int\nolimits_{U_1}\!{\cdots\! 
\int\nolimits_{U_N}\!{A(u_{1},\ldots ,u_{N})d\Psi_{1}(u_{1})\cdots
\,d\Psi_{N}(u_{N})}}}{\int\nolimits_{U_1}{\!\cdots\! \int\nolimits_{U_N}\!{B(u_{1},\ldots ,u_{N})
\,d\Psi_{1}(u_{1})\ldots
d\Psi_{N}(u_{N})}}},\!\!\! \label{e11}
\end{multline}
\textit{где подынтегральные функции числителя и~знаменателя выражаются
соотношениями}:
\begin{align}
A(u_{1},\ldots
,u_{N})&={}\notag\\
&\hspace*{-20mm}{}=\sum\limits_{i=1}^{N}{d_{i}(u_{i})}{\widehat{D}}^{(i)}(u_{1}, \ldots
,u_{i-1},u_{i+1}, \ldots , u_{N})\,;  \label{e12}\\
 B(u_{1},\ldots
,u_{N})&={}\notag\\
&\hspace*{-20mm}{}=\sum\limits_{i=1}^{N}{T_{i}(u_{i})}{\widehat{D}}^{(i)}(u_{1}, \ldots
,u_{i-1},u_{i+1}, \ldots , u_{N})\,.  \label{e13}
\end{align}
\textit{В свою очередь, функции} ${\widehat{D}}^{(i)}(u_{1}, \ldots
,u_{i-1},u_{i+1}, \ldots$\linebreak $\ldots , u_{N})$, $i\hm\in\{1,\dots,N\}$, 
\textit{входящие в~правые части формул}~(\ref{e12}) и~(\ref{e13}), 
\textit{определяются следующим образом:}

\noindent
\begin{multline}
{\widehat{D}}^{(i)}(u_{1}, \ldots ,u_{i-1},u_{i+1}, \ldots , u_{N})={}
\\
{}=(-1)^{N+i+2}\sum\limits_{\alpha ^{(N),i}}{(-1)}^{\delta (\alpha
^{(N),i}) }{\widehat{D}}_{0}^{(i)}\left(\alpha ^{(N),i},u_{1}, \ldots\right.\\
\left.\ldots , u_{i-1},u_{i+1}, \ldots , u_{N}\right)\,. \label{e14}
\end{multline}
\textit{Здесь} $\alpha ^{(N),i}=(\alpha _{1}, \ldots , \alpha _{i-1},\alpha_{i+1}, \ldots , 
\alpha _{N})$~\textit{--- произвольная
перестановка чисел }$(1, \ldots , i-1, i+1, \ldots , N)$;
$\delta
(\alpha ^{(N),i})$~\textit{--- число инверсий в~перестановке} 
$\alpha ^{(N),i}$;

\noindent
\begin{multline}
{\widehat{D}}_{0}^{(i)}(\alpha ^{(N),i},u_{1}, \ldots ,u_{i-1},u_{i+1},
\ldots , u_{N})={}\\
{} ={\widetilde{p}}_{1,\alpha _{1}}\left(u_{1}\right)\cdots {\widetilde{p}}_{i-1,\alpha
_{i-1}}\left(u_{i-1}\right){\widetilde{p}}_{i+1,\alpha _{i+1}}\left(u_{i+1}\right)\cdots\\
\cdots
{\widetilde{p}}_{N,\alpha _{N}}\left(u_{N}\right)\,, 
\label{e15}
\end{multline}
где
\begin{multline}
 {\widetilde{p}}_{k,\alpha _{k}}(u_{k})=
\begin{cases}
p_{kk}(u_{k})-1,\  & \alpha _{k}=k\,; \\
p_{k,\alpha _{k}}(u_{k}),\  & \alpha _{k}\ne k, \\
\end{cases}\\
 k=1, \ldots , i-1, i+1, \ldots ,N\,. \label{e16}
 \end{multline}
\textit{Функции $p_{ij}(u_i)$, $T_{i}(u_{i})$ и~$d_{i}(u_{i})$,
$u_i\hm\in U_i$, $i,j\hm\in \{1,2,\ldots,N\}$, 
входящие в~соотношения}~(\ref{e12})--(\ref{e16}), 
\textit{определяются равенствами}~(\ref{e2}), (\ref{e4}) \textit{и}~(\ref{e8}) \textit{соответственно.}

\smallskip

\noindent
Д\,о\,к\,а\,з\,а\,т\,е\,л\,ь\,с\,т\,в\,о\ теоремы~1 
в~весьма сжатой форме приведено в~работе~\cite{21}. Читателю, интересующемуся 
более подробным обоснованием данного результата, порекомендуем обратиться к~тексту 
кандидатской диссертации А.\,В.~Иванова~\cite[гл.~3]{22}.

\smallskip

Итак, теорема~1 позволяет получить явное аналитическое представление 
для стационарного стоимостного показателя вида~(\ref{e10}) в~форме 
дроб\-но-ли\-ней\-но\-го интегрального функционала от набора\linebreak вероятностных мер 
$\Psi\hm=(\Psi_{1},\Psi_{2},\ldots , \Psi_{N})$, за\-да\-ющих стратегию управления 
полумарковским процессом~$\xi(t)$. При этом подынтегральные функции числителя 
и~знаменателя задаются формулами~(\ref{e12}), (\ref{e13}) 
и~вспомогательными равенствами~(\ref{e14})--(\ref{e16}). Таким образом, функция
\begin{equation}
C\left(u_1, u_2,\ldots, u_N\right)=\fr{A(u_1, u_2,\ldots, u_N)}{B(u_1, u_2,\ldots, u_N)}\,,
\label{e17}
\end{equation}
которая в~дальнейшем будет называться основной функцией дроб\-но-ли\-ней\-но\-го 
интегрального функционала~(\ref{e11}) и~которая будет играть важную роль 
в~дальнейшем исследовании, также явно определяется формулами~(\ref{e17}), 
(\ref{e12}), (\ref{e13}).

\section{Формальная постановка оптимизационной задачи 
и~условия существования оптимальной стратегии управления полумарковским процессом}

Будем рассматривать проблему управления полумарковским процессом~$\xi(t)$ в~форме 
экстремальной задачи
\begin{multline}
I(\Psi)=I\left(\Psi_1, \Psi_2,\ldots,\Psi_N\right)\rightarrow \mathrm{extr}\,,
\\
\Psi=\left(\Psi_1, \Psi_2,\ldots,\Psi_N\right)\in\Gamma\,. \label{e18}
\end{multline}
При этом показатель качества управления~$I(\Psi)$ представляет собой 
дроб\-но-ли\-ней\-ный интегральный функционал вида~(\ref{e11}).

Для решения экстремальной задачи~(\ref{e18}) воспользуемся некоторым утверждением 
об экстремуме дроб\-но-ли\-ней\-но\-го интегрального функционала. Прежде 
чем сформулировать данное утверждение, отметим, что в~теории оптимизации 
хорошо известны задачи, в~которых целевая функция представляет собой 
отношение двух линейных отображений, а имеющиеся ограничения также линейны. 
Такой раздел называется дроб\-но-ли\-ней\-ным программированием. Основные
 теоретические результаты данного направления изложены в~работе~\cite{23},
  там же приведена подробная библиография. В~дальнейшем потребуется некоторый 
  специальный результат о безусловном экстремуме дроб\-но-ли\-ней\-но\-го 
  интегрального функционала вида~(\ref{e11}), который был впервые сформулирован 
  в~работе~\cite{14}. Заметим, что для использования этого результата необходимо, 
  чтобы выполнялись некоторые предварительные условия, которые в~данном случае 
  можно сформулировать следующим образом:
\begin{enumerate}[1.]
\item Интегральные выражения
\begin{align*}
I_1(\Psi)&=I_1\left(\Psi_1,\Psi_2,\ldots,\Psi_N\right)={}&\\
&\hspace*{-13mm}{}=\int\limits_{U_1}\!\cdots\!
\int\limits_{U_N}\!\!A\left(u_1,\ldots ,u_N\right)\,
d\Psi_1\left(u_1\right) %d\Psi_2\left(u_2\right)
\cdots
 d\Psi_N\left(u_N\right)\,;
\\
I_2(\Psi)&=I_2\left(\Psi_1,\Psi_2,\ldots,\Psi_N\right)={}&\\
&\hspace*{-13mm}{}=\int\limits_{U_1}\!\cdots\!\int\limits_{U_N}\!\!
B\left(u_1,\ldots,u_N\right)\,
d\Psi_1\left(u_1\right)% d\Psi_2\left(u_2\right)\cdots\\
\cdots d\Psi_N\left(u_N\right)
\end{align*}
определены для всех стратегий управления $\Psi\hm=(\Psi_1, \ldots,\Psi_N)
\hm\in \Gamma$.

\item Функционал $I_2(\Psi)=I_2(\Psi_1, \ldots,\Psi_N)\hm\neq 0$ 
для всех $\Psi\hm=(\Psi_1, \ldots,\Psi_N)\hm\in \Gamma$.

\item Множество $\Gamma$ включает в~себя множество всех вырожденных 
вероятностных мер: $\Gamma^* \hm\subset \Gamma$.
\end{enumerate}

Сделаем несколько важных замечаний по поводу введенных предварительных условий.

\smallskip

\noindent
\textbf{Замечание~1.}\ Из условия~2 следует, что функция $B(u_1, u_2,\ldots, u_N)$ 
не может принимать значения разных знаков. С~учетом условия~3 
получаем, что указанная функция должна обладать \mbox{свойством} строгой 
знакопостоянности на всем множестве~$U$. С~другой стороны, если выполняется 
условие строгой знакопостоянности функции $B(u_1, u_2,\ldots, u_N), 
(u_1, u_2,\ldots, u_N)\hm\in U$, то условие~2 выполняется автоматически.

\smallskip

\noindent
\textbf{Замечание~2.}\ Если рассматривать в~качестве целевого функционала 
$I(\Psi_1, \Psi_2,\ldots,\Psi_N)$ экстремальной задачи~(\ref{e18}) 
стационарный стоимостной пока\-затель~(\ref{e10}), то функция $B(u_1,u_2,\ldots,u_N)$ 
имеет\linebreak следующее теоретическое содержание. Данная функция представляет собой условное 
математическое ожидание длительности периода времени между соседними моментами 
изменения со\-сто\-яния полумарковского процесса~$\xi(t)$ при условии, что стратегия 
его управ\-ле\-ния является детерминированной и~задается набором значений аргументов 
$(u_1,u_2,\ldots,u_N)$. Тогда условие строгой положительности функции 
$B(u_1,u_2,\ldots,u_N)$ при всех $(u_1,u_2,\ldots,u_N)\hm\in U$ является естественным 
и~фактически означает, что при любой заданной детерминированной стратегии 
управ\-ле\-ния процесс~$\xi(t)$ не имеет мгновенных со\-сто\-яний, длительность пребывания 
в~которых равна нулю.

\smallskip

\noindent
\textbf{Замечание~3.}\ Сделаем некоторые замечания, связан\-ные с~подынтегральной 
функцией числителя дроб\-но-ли\-ней\-но\-го интегрального функционала~(\ref{e11}). 
Как и~ранее, будем рассматривать в~качестве целевого функционала $I(\Psi_1, \Psi_2,\ldots,\Psi_N)$\linebreak 
экстремальной задачи~(\ref{e18}) стационарный стоимостной показатель~(\ref{e10}). 
Тогда для любого фиксированного набора значений аргументов $(u_1,u_2,\ldots,u_N)\hm\in U$ 
значение функции $A(u_1,u_2,\ldots\linebreak \ldots,u_N)$ представляет собой условное математическое
 ожидание приращения рассматриваемого стоимостного функционала, 
 происшедшее за время пребывания полумарковского процесса~$\xi(t)$ в~некотором 
 фиксированном  состоянии при условии, что стратегия управления является 
 детерминированной и~задается указанным набором $(u_1,u_2,\ldots,u_N)\hm\in U$. 
 Отметим, что в~теореме об экстремуме дроб\-но-ли\-ней\-но\-го интегрального 
 функционала, доказанной в~работе~\cite[гл.~10]{12}, 
 на подынтегральную функцию числителя накладываются условия ограниченности на 
 всем множестве значений аргумента. Для многих математических моделей и~связанных 
 с~ними задач оптимального управления такое условие является излишне ограничительным. 
 В~качестве примера можно привести модели оптимального управления запасом непрерывного 
 продукта, рассмотренные в~работах~\cite{27, 28}. 
 В~настоящем исследовании на функцию $A(u_1,u_2,\ldots,u_N)$ накладывается только 
 условие интегрируемости по любому заданному набору вероятностных мер 
 $\Psi\hm=(\Psi_1, \Psi_2,\ldots,\Psi_N)$, образующему стратегию управления 
 полумарковским процессом~$\xi(t)$ (условие~1 системы предварительных условий).

\smallskip

\noindent
\textbf{Замечание~4.} Условия~1--3 являются необходимыми для корректной 
постановки задачи безусловного экстремума дроб\-но-ли\-ней\-но\-го интегрального 
функционала. Если этот функционал служит показателем качества в~задаче оптимального 
управления случайным процессом, то необходимо добавить к~этим условиям дополнительное, 
связанное с~некоторой регулярностью самого управляемого процесса, а~именно: некоторый 
содержательный показатель, связанный с~поведением этого процесса, должен существовать 
и~быть представимым в~виде дроб\-но-ли\-ней\-но\-го интегрального функционала. 
Если потребовать, чтобы выполнялось эргодическое соотношение~(\ref{e10}), 
то можно использовать\linebreak теорему~1 и~сформулировать задачу оптимального управ\-ле\-ния 
в~виде~(\ref{e18}) для дроб\-но-ли\-ней\-но\-го\linebreak интегрального функционала~(\ref{e11}). 
Таким образом, необходимо ввести условие, обеспечивающее существование единственного 
стационарного распределения вложенной цепи Маркова и~выполнение\linebreak соотношения~(\ref{e10}). 
По аналогии с~[8, гл.~5] сформулируем это дополнительное условие в~следующем виде:
\begin{enumerate}
\setcounter{enumi}{3}
\item Для любой рассматриваемой стратегии управ\-ле\-ния $\Psi\hm=
(\Psi_1, \Psi_2,\ldots,\Psi_N)\hm\in \Gamma$ вложенная цепь Маркова 
полумарковского процесса $\xi(t)$ имеет ровно один класс возвратных 
положительных состояний.
\end{enumerate}

Теперь определим понятие допустимой стратегии управления полумарковским процессом 
с~конечным множеством состояний.

\smallskip

\noindent
\textbf{Определение~2.}\ Назовем стратегию управления 
$\Psi\hm=(\Psi_1, \Psi_2,\ldots,\Psi_N)$ 
допустимой в~данной задаче, если она удовлетворяет условиям~1--4.


\smallskip

\noindent
\textbf{Замечание~5.}\ Как следует из замечания~1, если потребовать, 
чтобы функция $B(u_1, u_2,\ldots,u_N)$ являлась строго знакопостоянной при 
всех $(u_1, u_2,\ldots,u_N)\hm\in U$, то можно считать допустимыми стратегии 
$(\Psi_1, \Psi_2,\ldots,\Psi_N)$, удовлетворяющие условиям~1, 3, 4. С~учетом замечания~2 
о~естественном характере условия строгой знакопостоянности функции $B(u_1,u_2,\ldots,u_N)$ 
при всех значениях аргументов $(u_1, u_2,\ldots,u_N)\hm\in U$ будем требовать 
выполнения этого условия в~формулировке приводимой в~дальнейшем основной 
теоремы об оптимальной стратегии управления полумарковским процессом.

\smallskip

\noindent
\textbf{Замечание~6.}\ Ниже будет сформулирована и~доказана основная 
теорема об оптимальной стра\-тегии управления полумарковским процессом с~конеч\-ным 
множеством состояний. Будем формулировать эту теорему по отношению к~экстремальной 
задаче~(\ref{e18}), в~которой целевой функционал $I(\Psi_1, \Psi_2,\ldots,\Psi_N)$ 
имеет вид дроб\-но-ли\-ней\-но\-го интегрального функционала. 
Это обстоятельство связано с~тем, что целевой функционал в~задаче 
оптимального управления необязательно должен иметь характер стационарного 
стоимостного показателя вида~(\ref{e10}). В~частности, еще в~1983~г.\ П.\,В.~Шнурковым 
было установлено~\cite{24}, что ряд показателей, связанных 
с~временем пребывания управляемого полумарковского процесса в~заданном конечном 
подмножестве состояний, имеет структуру дроб\-но-ли\-ней\-но\-го интегрального 
функционала от набора вероятностных мер, определяющих стратегию управления. 
Таким образом, рассматриваемая задача управления имеет более общий характер, 
чем задача, в~которой целевой функционал представляет собой стационарный 
стоимостной показатель вида~(\ref{e10}).






\smallskip

\noindent
\textbf{Замечание~7.}\ Если рассматривать задачу оптимального управления 
полумарковским процессом, в~кото\-рой целевой функционал не совпадает 
со стационарным стоимостным показателем~(\ref{e10}), то возможно, что могут 
потребоваться другие дополнительные условия, обеспечивающие существование этого 
показателя и~его представление в~форме~(\ref{e11}). В~связи с~этим в~формулировке 
основной теоремы будем использовать термин допустимые стратегии в~широком смысле, 
имея в~виду выполнение всех необходимых условий для каждого рассмат\-ри\-ва\-емо\-го 
показателя качества управления.

\smallskip


\noindent
\textbf{Замечание 8.} Множество допустимых стратегий может 
не совпадать с~множеством всех возможных стратегий управления. 
В~частности, допустимые стратегии могут состоять только из дискретных вероятностных 
мер $\Psi_1, \Psi_2,\ldots,\Psi_N$, т.\,е.\ таких, которые сосредоточены на дискретных 
множествах точек пространств $U_1, U_2,\ldots,U_N$.

\section{Теоретическое решение задачи оптимального управления}

Перейдем к~формулировке и~доказательству тео\-ре\-мы об 
оптимальной стратегии управ\-ле\-ния полумарковским процессом с~конечным 
множеством состояний.

\smallskip

\noindent
\textbf{Теорема~2.} \textit{Рассмотрим проблему оптимального управ\-ле\-ния 
полумарковским процессом~$\xi(t)$ в~виде экстремальной задачи}~(\ref{e18}), 
\textit{определенной на множестве допустимых стратегий $\Gamma$, 
для дроб\-но-ли\-ней\-но\-го 
функционала}~(\ref{e11}). \textit{Пусть функция $B(u_1,u_2,\ldots,u_N)$, 
входящая в~определение функционала}~(\ref{e11}),
\textit{является строго знакопостоянной (строго положительной или строго отрицательной) 
при всех значениях аргументов $(u_1,u_2,\ldots,u_N)\hm\in U$.
Тогда справедливы сле\-ду\-ющие утверждения}:
\begin{enumerate}[1.]
\item \textit{Если функция} $C(u_1,u_2,\ldots,u_N)\hm=A(u_1,u_2,\ldots$\linebreak
$\ldots,u_N)/{B(u_1,u_2,\ldots,u_N)}$ 
\textit{ограничена сверху или снизу и~достигает глобального экст\-ре\-му\-ма на множестве
$U\hm=U_1\times U_2\times \cdots \times U_N$ (максимума или минимума), 
то оптимальная стратегия управления полумарковским процессом~$\xi(t)$ существует, 
является детерминированной и~определяется
вырожденной вероятностной мерой $\Psi^*\hm\in \Gamma^*$, сосредоточенной в~точке, 
в~которой достига\-ет соответствующего экстремума функция $C(u_1,u_2,\ldots,u_N)$,
и~при этом выполняются соотношения}:
\begin{multline}  %{\substack{{i=\overline{1,n}}\\ {j=\overline{1,l}}}}
\max\limits_{\Psi \in \Gamma} I(\Psi)=
\max\limits_{\substack{{\Psi_i \in \Gamma_i\,,}\\ 
{i=\overline{1,N}}}}
I\left(\Psi_1,\Psi_2,\ldots,\Psi_N\right)={}\\
{}=
\max\limits_{\substack{{\Psi_i^* \in \Gamma_i^*,}\\ 
{i=\overline{1,N}}}}
 I\left(\Psi_1^*,\Psi_2^*,\ldots,\Psi_N^*\right)={}\\
{}=\max\limits_{(u_1,u_2,\ldots,u_N)\in U}\fr{A(u_1,u_2,\ldots,u_N)}
{B(u_1,u_2,\ldots,u_N)}\,; \label{e19}
\end{multline}

\vspace*{-12pt}

\noindent
\begin{multline*}
\min\limits_{\Psi \in \Gamma} I(\Psi)=
\min\limits_{\substack{{\Psi_i \in \Gamma_i\,,}\\ 
{i=\overline{1,N}}}} I\left(\Psi_1,\Psi_2,\ldots,\Psi_N\right)={}\\
{}=
\min\limits_{\substack{{\Psi_i^* \in \Gamma_i^*,}\\ 
{i=\overline{1,N}}}}
I\left(\Psi_1^*,\Psi_2^*,\ldots,\Psi_N^*\right)={}\\
{}=\min\limits_{(u_1,u_2,\ldots,u_N)\in U}\fr{A(u_1,u_2,\ldots,u_N)}
{B(u_1,u_2,\ldots,u_N)}\,. %\label{e20}
\end{multline*}
\item \textit{Если функция $C(u_1,u_2,\ldots,u_N)\hm=
{A(u_1,u_2,\ldots,u_N)}/{B(u_1,u_2,\ldots,u_N)}$ ограничена сверху или снизу, 
но не достигает глобального экстремума на множестве $U\hm=U_1\times U_2\times\cdots
\times U_N$,
то для любого $\varepsilon\hm > 0$ можно выбрать $\varepsilon$-оп\-ти\-маль\-ную 
детерминированную стратегию управления полумарковским процессом~$\xi(t)$, 
которая определяется вырожденной
вероятностной мерой $\Psi^{*(+)}(\varepsilon)\hm\in \Gamma^*$ или вырожденной
вероятностной мерой $\Psi^{*(-)}(\varepsilon)\hm\in \Gamma^*$, в~зависимости от 
вида экстремума (максимума или минимума) в~задаче}~(\ref{e18}). 
\textit{При этом вероятностная мера $\Psi^{*(+)}(\varepsilon)\hm\in \Gamma^*$ может быть 
сосредоточена в~любой точке $\left(u_1^{(+)}(\varepsilon),u_2^{(+)}(\varepsilon),\ldots,
u_N^{(+)}(\varepsilon)\right)$, удовлетворяющей соотношению}:
\begin{multline}
\sup\limits_{(u_1,u_2,\ldots,u_N) \in U}
\fr{A(u_1,u_2,\ldots,u_N)}{B(u_1,u_2,\ldots,u_N)}-\varepsilon <{}\\
{}<
\fr{A\left(u_1^{(+)}(\varepsilon),u_2^{(+)}(\varepsilon),\ldots,u_N^{(+)}
(\varepsilon)\right)}
{B\left(u_1^{(+)}(\varepsilon),u_2^{(+)}(\varepsilon),\ldots,u_N^{(+)}
(\varepsilon)\right)}<{}\\
{}<\sup\limits_{(u_1,u_2,\ldots,u_N) \in U}
\fr{A(u_1,u_2,\ldots,u_N)}{B(u_1,u_2,\ldots,u_N)}<\infty\,, 
\label{e21}
\end{multline}
\textit{если функция $C(u_1,u_2,\ldots,u_N)$ ограничена сверху 
и~экстремальная задача}~(\ref{e18}) 
\textit{представляет собой задачу на максимум. Аналогично вероятностная мера 
$\Psi^{*(-)}(\varepsilon)\hm\in \Gamma^*$ может быть сосредоточена в~любой точке 
$\left(u_1^{(-)}(\varepsilon),u_2^{(-)}(\varepsilon),\ldots,u_N^{(-)}(\varepsilon)
\right)$, удовлетворяющей соотношению}:

\noindent
\begin{multline*}
-\infty<\inf\limits_{(u_1,u_2,\ldots,u_N) \in U}\fr{A(u_1,u_2,\ldots,u_N)}
{B(u_1,u_2,\ldots,u_N)} <{}\\
{}<
\fr{A\left(u_1^{(-)}(\varepsilon),u_2^{(-)}
(\varepsilon),\ldots,u_N^{(-)}(\varepsilon)\right)}
{B\left(u_1^{(-)}(\varepsilon),u_2^{(-)}(\varepsilon),\ldots,
u_N^{(-)}(\varepsilon)\right)}<{}\\
{}<\inf\limits_{(u_1,u_2,\ldots,u_N) \in U}
\fr{A(u_1,u_2,\ldots,u_N)}{B(u_1,u_2,\ldots,u_N)}+\varepsilon\,, 
%\label{e22}
\end{multline*}
\textit{если функция $C(u_1,u_2,\ldots,u_N)$ ограничена снизу и~экстремальная 
задача}~(\ref{e18})  \textit{представляет собой задачу на минимум}.
\item \textit{Если функция $C(u_1,u_2,\ldots,u_N)\hm=
{A(u_1,u_2,\ldots,u_N)}/{B(u_1,u_2,\ldots,u_N)}$ не ограничена сверху 
или снизу, то оптимальной стратегии управления в~смысле
соответствующей экстремальной задачи не существует. 
При этом найдется такая последовательность вырожденных вероятностных
мер~$\Psi^{*(+)}(n)$, сосредоточенных в~точках 
$\left(u_1^{(+)}(n),u_2^{(+)}(n),\ldots,u_N^{(+)}(n)\right)$, $n\hm=1,2,\dots $, 
для которых выполняется соотношение}:
\begin{multline*}
I\left(\Psi^*(n)\right)={}\\
{}=
I\left(\Psi_1^{*(+)}(n),\Psi_2^{*(+)}(n),\ldots,\Psi_N^{*(+)}(n)\right)={}\\
{}=\fr{A\left(u_1^{(+)}(n),u_2^{(+)}(n),\ldots,u_N^{(+)}(n)\right)}
{B\left(u_1^{(+)}(n),u_2^{(+)}(n),\ldots,u_N^{(+)}(n)\right)}\to 
\infty\\
\mbox{при}\ n\to\infty\,, 
%\label{e23}
\end{multline*}
\textit{если функция $C(u_1,u_2,\ldots,u_N)$ не ограничена сверху. 
Аналогично найдется такая последовательность вырожденных вероятностных
мер~$\Psi^{*(-)}(n)$, сосредоточенных в~точках 
$\left(u_1^{(-)}(n),u_2^{(-)}(n),\ldots,u_N^{(-)}(n)\right)$, 
$n\hm=1,2,\dots $, для которых выполняется соотношение}:
\begin{multline*}
I\left(\Psi^{*(-)}(n)\right)={}\\
{}= I
\left(\Psi_1^{*(-)}(n),\Psi_2^{*(-)}(n),\ldots,\Psi_N^{*(-)}(n)\right)={}\\
{}=\fr{A\left(u_1^{(-)}(n),u_2^{(-)}(n),\ldots,u_N^{(-)}(n)\right)}
{B\left(u_1^{(-)}(n),u_2^{(-)}(n),\ldots,u_N^{(-)}(n)\right)}\to 
-\infty\\
\mbox{при}~~n\to\infty\,,  
%\label{e24}
\end{multline*}
\textit{если функция $C(u_1,u_2,\ldots,u_N)$ не ограничена \mbox{снизу}}.
\end{enumerate}
\textit{При этом сформулированные утверждения каждого пункта теоремы~$2$ 
могут выполняться как по отдельности, для одного из двух
видов экстремума, так и~совместно, для обоих видов экстремума.}

\smallskip

Прежде чем непосредственно доказывать теорему~2, докажем некоторые 
вспомогательные утверждения.

\smallskip

\noindent
\textbf{Лемма~1.}\ 
\textit{Рассмотрим дроб\-но-ли\-ней\-ный интегральный функционал 
$I(\Psi_1, \Psi_2,\ldots, \Psi_N)$ вида}~(\ref{e11}), 
\textit{заданный на некотором множестве наборов вероятностных мер 
$\Psi\hm=(\Psi_1, \Psi_2,\ldots, \Psi_N)\hm \in \Gamma$. Предположим, что на 
множестве~$\Gamma$ выполняется условие~$1$ из набора предварительных условий 
и~функция $B(u_1, u_2,\ldots, u_N)$  обладает свойством строгой знакопостоянности 
при всех $(u_1, u_2,\ldots, u_N) \hm\in U$. Тогда справедливы следующие утверждения}:
\begin{enumerate}[1.]
\item \textit{Если основная функция 
$C(u_1, u_2,\ldots, u_N)\hm={A(u_1, u_2,\ldots, u_N)}/{B(u_1, u_2,\ldots, u_N)}$ 
ограничена сверху, т.\,е.\ выполняется условие}
\begin{multline}
C\left(u_1, u_2,\ldots, u_N\right)=
\fr{A(u_1, u_2,\ldots, u_N)}{B(u_1, u_2,\ldots, u_N)}\leq {}\\
{}\leq
c_0^{(+)}<\infty \,, \enskip \left(u_1, u_2,\ldots, u_N\right) \in U\,, \label{e25}
\end{multline}
\textit{то имеет место неравенство}:
\begin{equation}
I\left(\Psi_1, \Psi_2,\ldots, \Psi_N\right)\leq c_0^{(+)} 
\label{e26}
\end{equation}
\textit{для всех} $(\Psi_1, \Psi_2,\ldots, \Psi_N) \in \Gamma$.
\item \textit{Если основная функция 
$C(u_1, u_2,\ldots, u_N)\hm={A(u_1, u_2,\ldots, u_N)}/{B(u_1, u_2,\ldots, u_N)}$ 
ограничена снизу, т.\,е.\ выполняется условие}
\begin{multline*}
C\left(u_1, u_2,\ldots, u_N\right)=\fr{A(u_1, u_2,\ldots, u_N)}{B(u_1, u_2,\ldots, 
u_N)}\geq{}\\
{}\geq c_0^{(-)}>-\infty \,, 
\left(u_1, u_2,\ldots, u_N\right) \in U\,, 
%\label{e27}
\end{multline*}
\textit{то имеет место неравенство}:
\begin{equation*}
I\left(\Psi_1, \Psi_2,\ldots, \Psi_N\right)\geq c_0^{(-)} 
%\label{e28}
\end{equation*}
\textit{для всех} $(\Psi_1, \Psi_2,\ldots, \Psi_N) \hm\in \Gamma$.
\end{enumerate}

\noindent
Д\,о\,к\,а\,з\,а\,т\,е\,л\,ь\,с\,т\,в\,о\ \ леммы~1.\ 
Докажем первое утверждение леммы. Предположим сначала, 
что функция $B(u_1, u_2,\ldots,  u_N)$ строго положительна:
\begin{equation}
B\left(u_1, u_2,\ldots, u_N\right)>0\,,\enskip
\left(u_1, u_2,\ldots, u_N\right)\in U\,. \label{e29}
\end{equation}
Заметим, что в~таком случае по свойству интеграла~\cite[гл.~V]{18}
\begin{multline}
\hspace*{-2mm}\int\limits_{U_1}\!\!\cdots\! \!\int\limits_{U_N}\!\!B(u_1, \ldots,u_N) \,
d\Psi_1(u_1)%d\Psi_2(u_2)\cdots\\
\cdots d\Psi_N(u_N)>0 \!\!\!\!\label{e30}
\end{multline}
для любого фиксированного набора $\Psi\hm=(\Psi_1, \ldots, \Psi_N)\hm\in \Gamma$.
Из неравенства~(\ref{e25}) с~уче\-том~(\ref{e29}) получаем:
\begin{multline}
\hspace*{-4mm}A\left(u_1,\ldots, u_N\right)\leq{}\\
\hspace*{-4mm}{}\leq c_0^{(+)} B\left(u_1, \ldots, u_N\right)\,, 
\left(u_1, \ldots, u_N\right)\in U\,. \label{e31}
\end{multline}
В свою очередь, из неравенства~(\ref{e31}) и~свойств интеграла следует:
\begin{multline}
\int\limits_{U_1}\!\!\cdots\! \!\int\limits_{U_N}\!\!A(u_1,\ldots, u_N) \,
d\Psi_1\left(u_1\right)%d\Psi_2\left(u_2\right)\cdots\\
\cdots d\Psi_N\left(u_N\right)\leq\\
\hspace*{-24pt}\leq 
c_0^{(+)}\!\!\int\limits_{U_1}\!\!\cdots\!\! \int\limits_{U_N}\!\!\!B\!\left(u_1,\ldots, u_N\right)
 d\Psi_1\!\left(u_1\right)\!%d\Psi_2\left(u_2\right)\cdots\\
 \cdots d\Psi_N\!\left(u_N\right)\!\! 
 \label{e32}
\end{multline}
для любого фиксированного набора $\Psi\hm=(\Psi_1, \ldots, \Psi_N)\hm\in \Gamma$. 
Но тогда из~(\ref{e32}) с~учетом~(\ref{e30}) получаем:
\begin{multline}
I(\Psi_1, \ldots, \Psi_N)={}\\
{}=
\fr{\int\nolimits_{U_1}\!\cdots\! \int\nolimits_{U_N}\!\!A\left(u_1, \ldots, u_N\right)\,
 d\Psi_1(u_1)\cdots d\Psi_N(u_N)}{
\int\nolimits_{U_1}\!\cdots\! \int\nolimits_{U_N}\!\!B\left(u_1, \ldots, u_N\right)\,
 d\Psi_1(u_1)
 \cdots d\Psi_N(u_N)}\leq{}\\
 {}\leq c_0^{(+)} 
 \label{e33}
\end{multline}
для любого фиксированного набора $(\Psi_1, \ldots\linebreak\ldots, \Psi_N)\hm\in \Gamma$.

Предположим теперь, что функция $B(u_1,\ldots, u_N)$ строго отрицательна:
\begin{equation}
B(u_1,\ldots, u_N)<0 \quad \left(u_1, \ldots, u_N\right)\in U\,. 
\label{e34}
\end{equation}
Тогда
\begin{multline}
\hspace*{-6pt}\int\limits_{U_1}\!\!\cdots\!\! \int\limits_{U_N}\!\!B\!\left(u_1,\ldots, u_N\right)\!
 d\Psi_1(u_1) \cdots d\Psi_N(u_N)<0 \!\!\!
 \label{e35}
\end{multline}
для любого фиксированного набора $(\Psi_1, \ldots\linebreak \ldots, \Psi_N)\hm\in \Gamma$.

Как и~ранее, будем исходить из неравенства~(\ref{e25}). 
При выполнении условий~(\ref{e34}) и~(\ref{e35}) характер неравенств~(\ref{e31}) 
и~(\ref{e32}) меняется на противоположный, но характер неравенства~(\ref{e33}) 
остается неизменным. Таким образом, для любой функции 
$B(u_1, u_2,\ldots, u_N)$, обладающей свойством строгой знакопостоянности, 
из условия~(\ref{e25}) следует выполнение неравенства~(\ref{e33}), 
которое совпадает с~(\ref{e26}). Первое утверждение леммы~1 доказано. 
Второе утверждение доказывается аналогично. Лемма~1 доказана.

\smallskip

\noindent
\textbf{Лемма 2.} \textit{Рассмотрим дроб\-но-ли\-ней\-ный интегральный функционал 
$I(\Psi_1, \Psi_2,\ldots, \Psi_N)$ вида}~(\ref{e11}), 
\textit{заданный на некотором множестве наборов вероятностных мер 
$\Psi\hm=(\Psi_1, \Psi_2,\ldots, \Psi_N)\hm\in \Gamma$. Предпо\-ложим, что на 
множестве~$\Gamma$ выполняется условие~$1$ из набора предварительных условий 
и~функция $B(u_1, u_2,\ldots, u_N)$ обладает свойством строгой знакопостоянности 
при всех $(u_1, u_2,\ldots, u_N)\hm\in U$. Тогда справедливы следующие утверждения}:
\begin{enumerate}[1.]
\item \textit{Если основная функция $C(u_1, u_2,\ldots, u_N)\hm=
{A(u_1, u_2,\ldots, u_N)}/{B(u_1, u_2,\ldots, u_N)}$ ограничена сверху, 
но не достигает своего максимального 
значения, то имеет место неравенство}:
\begin{multline}
I\left(\Psi_1, \Psi_2,\ldots, \Psi_N\right)<{}\\
{}< \sup\limits_{(u_1, u_2,\ldots, u_N)\in U}
 C\left(u_1, u_2,\ldots, u_N\right)<\infty \label{e36}
\end{multline}
\textit{для всех} $(\Psi_1, \Psi_2,\ldots, \Psi_N)\in \Gamma$.
\item \textit{Если основная функция $C(u_1, u_2,\ldots, u_N)\hm=
{A(u_1, u_2,\ldots, u_N)}/{B(u_1, u_2,\ldots, u_N)}$ ограничена снизу, 
но не достигает своего минимального значения, то имеет место неравенство}:
\begin{multline*}
I\left(\Psi_1, \Psi_2,\ldots, \Psi_N\right)>{}\\
{}> \inf\limits_{(u_1, u_2,\ldots, u_N)\in U} 
C\left(u_1, u_2,\ldots, u_N\right)>-\infty 
%\label{e37}
\end{multline*}
\textit{для всех} $(\Psi_1, \Psi_2,\ldots, \Psi_N)\hm\in \Gamma$.
\end{enumerate}

\noindent
Д\,о\,к\,а\,з\,а\,т\,е\,л\,ь\,с\,т\,в\,о\ \ леммы~2. 
Докажем первое утверждение леммы. Поскольку множество значений 
основной функции $C(u_1, u_2,\ldots, u_N)$ ограничено сверху, оно имеет конечную 
верхнюю грань:
$$
\exists \sup\limits_{(u_1, u_2,\ldots, u_N)\in U} 
C\left(u_1, u_2,\ldots, u_N\right)<\infty
$$
(см.~\cite[гл.~1, \S3, п.~3.4, теорема~1]{25}).

По условию функция $C(u_1, u_2,\ldots, u_N)$ не достигает своего максимального 
значения. Следовательно, выполняется неравенство:
\begin{multline}
C(u_1, u_2,\ldots, u_N)=\fr{A(u_1, u_2,\ldots, u_N)}{B(u_1, u_2,\ldots, u_N)}<{}\\
{}< 
\sup\limits_{(u_1, u_2,\ldots, u_N)\in U} C(u_1, u_2,\ldots, u_N)<\infty\,, 
\\
\left(u_1, u_2,\ldots, u_N\right)\in U\,.
\label{e38}
\end{multline}
Взяв за основу строгое неравенство~(\ref{e38}), проведем рассуждения, аналогичные тем, 
которые были проведены в~лемме~1 по отношению к~неравенству~(\ref{e25}). 
В~результате получим строгое неравенство~(\ref{e36}).

Второе утверждение леммы~2 доказывается аналогично. Лемма~2 доказана.

\noindent
Д\,о\,к\,а\,з\,а\,т\,е\,л\,ь\,с\,т\,в\,о\ 
\ теоремы~2.
Начнем с~доказательства утверждения~1. Предположим сначала, что основная 
функция $C(u_1, u_2,\ldots, u_N)={A(u_1, u_2,\ldots, u_N)}/{B(u_1, u_2,\ldots, u_N)}$ 
ограничена сверху и~достигает глобального максимума на множестве~$U$ 
в~некоторой точке $u^{(+)}\hm=\left(u^{(+)}_1,u^{(+)}_2,\ldots,u^{(+)}_N\right)\hm\in U$,
а~именно:
\begin{multline*}
\max\limits_{(u_1, u_2,\ldots, u_N)\in U} C\left(u_1, u_2,\ldots, u_N\right) = {}\\
{}=
C\left(u^{(+)}_1,u^{(+)}_2,\ldots,u^{(+)}_N\right)<\infty\,.
\end{multline*}
Тогда выполняется соотношение:
\begin{multline}
C(u_1, u_2,\ldots, u_N)=\fr{A(u_1, u_2,\ldots, u_N)}{B(u_1, u_2,\ldots, u_N)}
\leq{}\\
{}\leq C\left(u^{(+)}_1,u^{(+)}_2,\ldots,u^{(+)}_N\right)<\infty\,, 
\\
\left(u_1, u_2,\ldots, u_N\right)\in U\,.
\label{e39}
\end{multline}
Условия леммы~1 выполнены, и~можно воспользоваться ее утверждениями. 
Согласно первому из них, если выполняется неравенство~(\ref{e39}), 
то имеет место соотношение:
\begin{equation*}
I(\Psi_1, \Psi_2,\ldots, \Psi_N)\leq 
C\left(u^{(+)}_1,u^{(+)}_2,\ldots,u^{(+)}_N\right)<\infty 
%\label{e40}
\end{equation*}
для всех стратегий управления $\Psi\hm=(\Psi_1, \Psi_2,\ldots\linebreak
\ldots, \Psi_N)\hm\in \Gamma$.

Таким образом, множество значений дроб\-но-ли\-ней\-но\-го интегрального 
функционала $I(\Psi_1, \Psi_2,\ldots, \Psi_N)$ ограничено сверху при всех 
$\Psi\hm=(\Psi_1, \Psi_2,\ldots, \Psi_N)\hm\in \Gamma$. Тогда существует верхняя 
грань этого множества и~выполняется неравенство:
\begin{multline}
\sup\limits_{(\Psi_1, \Psi_2,\ldots, \Psi_N)\in \Gamma} 
I\left(\Psi_1, \Psi_2,\ldots, \Psi_N\right)\leq {}\\
{}\leq
C\left(u^{(+)}_1,u^{(+)}_2,\ldots,u^{(+)}_N\right). \label{e41}
\end{multline}
Рассмотрим детерминированную стратегию управ\-ле\-ния 
$\Psi^{*(+)}\hm=\left(\Psi_1^{*(+)}, \Psi_2^{*(+)},\ldots, \Psi_N^{*(+)}\right)$, 
в~которой каждая вероятностная мера~$\Psi_i^{*(+)}$ является вы\-рож\-ден\-ной 
и~сосредоточена в~точке $u_i^{(+)}$, $i\hm=\overline{1, N}$.
По свойству интеграла
\begin{multline}
I\left(\Psi_1^{*(+)}, \Psi_2^{*(+)},\ldots ,\Psi_N^{*(+)}\right)={}\\
{}=
C\left(u^{(+)}_1,u^{(+)}_2,\ldots,u^{(+)}_N\right). \label{e42}
\end{multline}
Из соотношений~(\ref{e41}) и~(\ref{e42}) получаем:
\begin{multline}
\sup\limits_{(\Psi_1, \Psi_2,\ldots, \Psi_N)\in \Gamma} 
I\left(\Psi_1, \Psi_2,\ldots, \Psi_N\right)\leq{}\\
{}\leq
 I\left(\Psi_1^{*(+)}, 
\Psi_2^{*(+)},\ldots, \Psi_N^{*(+)}\right). \label{e43}
\end{multline}
Заметим дополнительно, что выполняются отношения принадлежности:
\begin{equation}
\Psi^{*(+)}=\left(\Psi_1^{*(+)}, \Psi_2^{*(+)},\ldots, \Psi_N^{*(+)}\right) 
\in \Gamma^* \subset \Gamma\,. \label{e44}
\end{equation}
Из~(\ref{e44}) и~свойства верхней грани следует:
\begin{multline}
\sup\limits_{\left(\Psi_1^{*}, \Psi_2^{*},\ldots, \Psi_N^{*}\right) \in \Gamma^*} 
I\left(\Psi_1^{*}, \Psi_2^{*},\ldots, \Psi_N^{*}\right)\leq {}\\
{}\leq
\sup\limits_{\left(\Psi_1, \Psi_2,\ldots, \Psi_N\right) 
\in \Gamma} I\left(\Psi_1, \Psi_2,\ldots, \Psi_N\right)\,. 
\label{e45}
\end{multline}
Объединяя~(\ref{e42}), (\ref{e43}) и~(\ref{e45}), получаем соотношение:
\begin{multline}
\sup\limits_{\left(\Psi_1^{*}, \Psi_2^{*},\ldots, \Psi_N^{*}\right) 
\in \Gamma^*} I\left(\Psi_1^{*}, \Psi_2^{*},\ldots, 
\Psi_N^{*}\right)\leq{}\\
{}\leq \sup\limits_{\left(\Psi_1, \Psi_2,\ldots, \Psi_N\right) 
\in \Gamma} I\left(\Psi_1, \Psi_2,\ldots, \Psi_N\right)\leq{}\\
{}\leq I\left(\Psi_1^{*(+)}, \Psi_2^{*(+)},\ldots, \Psi_N^{*(+)}\right)={}\\
{}=
\fr{A\left(u^{(+)}_1,u^{(+)}_2,\ldots,u^{(+)}_N\right)}{B\left(u^{(+)}_1,u^{(+)}_2,
\ldots,u^{(+)}_N\right)}\,.
 \label{e46}
\end{multline}
Из соотношения~(\ref{e46}) с~учетом~(\ref{e44}) получаем, что максимум 
функционала $I(\Psi_1, \Psi_2,\ldots, \Psi_N)$ на множестве допустимых стратегий 
$\Psi\hm=(\Psi_1, \Psi_2,\ldots, \Psi_N)\hm\in \Gamma$ существует и~достигается 
на детерминированной стратегии $\left(\Psi_1^{*(+)}, \Psi_2^{*(+)},\ldots, 
\Psi_N^{*(+)}\right)$.

Кроме того, выполняются соотношения~(\ref{e19}). Таким образом, утверждение~1 
в~случае, когда основная функция $C(u_1, u_2,\ldots, u_N)$ достигает глобального 
максимума, доказано. Соответствующее утверждение в~случае, когда основная функция 
$C(u_1, u_2,\ldots, u_N)$ достигает глобального минимума, доказывается аналогично. 
При этом используется второе утверждение леммы~1.

\smallskip

Перейдем к~доказательству второго утверждения теоремы~2. Предположим, что основная 
функция $C(u_1, u_2,\ldots, u_N)\hm=A(u_1, u_2,\ldots$\linebreak
$\ldots, u_N)/{B(u_1, u_2,\ldots, u_N)}$ 
ограничена сверху, но не достигает глобального максимума на множестве 
$U \hm= U_1 \times U_2 \times \cdots \times U_N$. Тогда множество значений 
основной функции имеет конечную верхнюю грань:

\noindent
\begin{multline*}
C\left(u_1, u_2,\ldots, u_N\right)=\fr{A(u_1, u_2,\ldots, u_N)}
{B(u_1, u_2,\ldots, u_N)}<{}\\
{}<
\sup\limits_{(u_1, u_2,\ldots, u_N)\in U} \fr{A(u_1, u_2,\ldots, u_N)}
{B(u_1, u_2,\ldots, u_N)}<\infty\,, 
\\
\left(u_1, u_2,\ldots, u_N\right)\in U\,.
%\label{e47}
\end{multline*}
По определению верхней грани для любого фиксированного $\varepsilon \hm>0$ 
существует точка $(u_1^{(+)}(\varepsilon), u_2^{(+)}(\varepsilon),\ldots, 
u_N^{(+)}(\varepsilon))$ такая, что выполняется двойное неравенство~(\ref{e21}) 
(см.~\cite[гл.~1, \S\,3, п.~3.4]{25}). Иначе говоря, значение основной функции 
в~указанной точке лежит в~левой \mbox{$\varepsilon$-окрест}\-ности верхней грани. 
Рассмотрим детерминированную стратегию управления 
$\Psi^{*(+)}(\varepsilon)\hm=\!\left(\Psi_1^{*(+)}(\varepsilon), 
\Psi_2^{*(+)}(\varepsilon),\ldots, \Psi_N^{*(+)}(\varepsilon)\!\right)$, компонентами\linebreak 
которой являются вырожденные вероятностные меры $\Psi_1^{*(+)}(\varepsilon), 
\Psi_2^{*(+)}(\varepsilon),\ldots, \Psi_N^{*(+)}(\varepsilon)$, причем вырожденная 
мера~$\Psi_i^{*(+)}(\varepsilon)$ сосредоточена в~точке~$u_i^{(+)}(\varepsilon)$,
$i\hm=1,2,\ldots,N$.

По свойству интеграла
\begin{multline}
I\left(\Psi_1^{*(+)}(\varepsilon), \Psi_2^{*(+)}(\varepsilon),\ldots,
 \Psi_N^{*(+)}(\varepsilon)\right)={}\\
 {}=
 C\left(u_1^{(+)}(\varepsilon), u_2^{(+)}(\varepsilon),\ldots, 
 u_N^{(+)}(\varepsilon)\right)\,. 
 \label{e48}
\end{multline}
Из соотношения~(\ref{e48}) с~учетом указанного свойства основной функции получаем:
\begin{multline}
\sup\limits_{(u_1, u_2,\ldots, u_N)\in U} \fr{A(u_1, u_2,\ldots, u_N)}
{B(u_1, u_2,\ldots, u_N)}-\varepsilon<{}\\
{}< I\left(\Psi_1^{*(+)}(\varepsilon), 
\Psi_2^{*(+)}(\varepsilon),\ldots, \Psi_N^{*(+)}(\varepsilon)\right)<{}
\\
{}< \sup\limits_{(u_1, u_2,\ldots, u_N)\in U} \fr{A(u_1, u_2,\ldots, u_N)}
{B(u_1, u_2,\ldots, u_N)}<\infty\,. 
\label{e49}
\end{multline}
Заметим также, что в~рассматриваемом случае выполнены условия леммы~2. 
Воспользуемся первым утверждением этой леммы, а~именно соотношением~(\ref{e36}):
\begin{multline}
I(\Psi_1, \Psi_2,\ldots, \Psi_N)< {}\\
{}<\sup\limits_{(u_1, u_2,\ldots, u_N)
\in U} \fr{A(u_1, u_2,\ldots, u_N)}{B(u_1, u_2,\ldots, u_N)}<\infty 
\label{e50}
\end{multline}
для всех $(\Psi_1, \Psi_2,\ldots, \Psi_N)\in\Gamma$.

Из соотношений~(\ref{e49}) и~(\ref{e50}) следует, что детерминированная стратегия 
$\Psi^{*(+)}(\varepsilon)\hm=\left(\Psi_1^{*(+)}(\varepsilon), \Psi_2^{*(+)}(\varepsilon),
\ldots, \Psi_N^{*(+)}(\varepsilon)\right)$, опре\-де\-ля\-емая набором вырожденных 
вероятностных мер, сосредоточенных в~соответствующих точках 
$\left(u_1^{(+)}(\varepsilon), u_2^{(+)}(\varepsilon),\ldots, 
u_N^{(+)}(\varepsilon)\right)$, является $\varepsilon$-оп\-ти\-маль\-ной. 
Вторая часть утверждения~2 теоремы~2, связанная со свойствами нижней грани, 
доказывается аналогично.

Докажем третье утверждение теоремы~2. Предположим, что множество значений 
основной функции $C(u_1, u_2,\ldots, u_N)\hm=
A(u_1, u_2,\ldots$\linebreak $\ldots, u_N)/{B(u_1, u_2,\ldots, u_N)}$
не является ограниченным сверху на множестве $U\hm=U_1\times U_2 \times \cdots $\linebreak
$\cdots \times U_N$.
Тогда существует последовательность\linebreak точек $\left(u_1^{(+)}(n), u_2^{(+)}(n),
\ldots,u_N^{(+)}(n)\right)\hm\in U$, $n\hm=1,2,\ldots$, для которой
\begin{multline}
C\left(u_1^{(+)}(n), u_2^{(+)}(n),\ldots,u_N^{(+)}(n)\right)={}\\
{}=
\fr{A\left(u_1^{(+)}(n), u_2^{(+)}(n),\ldots,u_N^{(+)}(n)\right)}
{B\left(u_1^{(+)}(n), u_2^{(+)}(n),\ldots,u_N^{(+)}(n)\right)}
\longrightarrow \infty \,,\\
n\rightarrow \infty\,.
\label{e51}
\end{multline}
Зафиксируем некоторую последовательность точек $\left(u_1^{(+)}(n), u_2^{(+)}(n),
\ldots,u_N^{(+)}(n)\right)\hm\in U$, $n\hm=1,2,\ldots$, обладающих указанным свойством, 
и~рассмотрим последовательность детерминированных  стратегий управления 
$\Psi^{*(+)}(n)\hm=\left(\Psi_1^{*(+)}(n), \Psi_2^{*(+)}(n),\ldots, 
\Psi_N^{*(+)}(n)\right)$, $n\hm=1,2,\ldots$, определяемых набором вырожденных 
вероятностных мер, сосредоточенных в~соответствующих точках 
$\left(u_1^{(+)}(n), u_2^{(+)}(n),\ldots,u_N^{(+)}(n)\right)$, $n\hm=1,2,\ldots$ 
По свойству интеграла для любого фиксированного значения $n=1,2,\ldots$ 
выполняется равенство:
\begin{multline}
I \left(\Psi^{*(+)}(n)\right)={}\\
{}=I\left(\Psi_1^{*(+)}(n), \Psi_2^{*(+)}(n),\ldots,
 \Psi_N^{*(+)}(n)\right)={}\\
{}=\fr{A\left(u_1^{(+)}(n), u_2^{(+)}(n),\ldots,u_N^{(+)}(n)\right)}
{B\left(u_1^{(+)}(n), u_2^{(+)}(n),\ldots,u_N^{(+)}(n)\right)}\,. 
\label{e52}
\end{multline}
Из соотношений~(\ref{e51}) и~(\ref{e52}) следует, что
\begin{multline}
I\left(\Psi^{*(+)}(n)\right)={}\\
{}=I\left(\Psi_1^{*(+)}(n), \Psi_2^{*(+)}(n),\ldots, 
\Psi_N^{*(+)}(n)\right)\longrightarrow\infty\,,\\ 
n \rightarrow\infty\,.
 \label{e53}
\end{multline}
Соотношение~(\ref{e53}) означает, что множество значе\-ний дроб\-но-ли\-ней\-но\-го 
интегрального функциона\-ла $I(\Psi_1, \Psi_2,\ldots, \Psi_N)$ вида~(\ref{e11}) 
не ограничено сверху\linebreak на множестве наборов вырожденных вероятностных мер 
$\left(\Psi_1^{*(+)}(n), \Psi_2^{*(+)}(n),\ldots, \Psi_N^{*(+)}(n)\right)\hm\in\Gamma^*$, 
а~следовательно, и~на более широком\linebreak множестве наборов вероятностных 
мер $(\Psi_1, \Psi_2,\ldots$\linebreak $\ldots, \Psi_N)\hm\in\Gamma$. В~таком случае решения экстремальной 
задачи~(\ref{e18}) в~форме задачи на максимум не существует. Соответствующее утвержде\-ние 
для варианта, когда множество значений основной функции $C(u_1, u_2,\ldots,u_N)
\hm=A(u_1, u_2,\ldots$\linebreak $\ldots,u_N)/{B(u_1, u_2,\ldots,u_N)}$ 
не является ограниченным снизу, доказывается аналогично. Третье утверж\-де\-ние теоремы~2 
доказано. Тем самым тео\-ре\-ма~2 доказана полностью.

\smallskip

Применим теорему~2 для решения поставленной задачи оптимального управления. 
Из утверждения этой теоремы следует, что для доказательства су-\linebreak ществования 
оптимального управ\-ле\-ния и~его нахождения необходимо исследовать на 
глобальный экстремум основную функцию дроб\-но-ли\-ней\-но\-го интегрального 
функционала $C(u_1,u_2,\ldots,u_N)$, определяемую формулой~(\ref{e17}) с~учетом 
равенств~(\ref{e12})--(\ref{e16}). В~некоторых случаях, например когда основной 
процесс~$\xi(t)$ является регенерирующим, а~стоимостные характеристики 
модели задаются линейными функциями, такое исследование можно провести 
аналитически. Однако для подавляющего большинства полумарковских моделей 
для этого необходимо использовать численные методы.

\section{Заключение}

В заключительной части работы приведем \mbox{краткое} описание теоретической 
основы метода решения задачи оптимального управления полумарковским 
процессом с~конечным множеством состояний.

\begin{enumerate}[1.]
\item Исходная проблема оптимального управления формулируется в~виде 
экстремальной задачи~(\ref{e18}). Целевым показателем качества управ\-ле\-ния в~данной задаче 
служит величина~(\ref{e10}), которая имеет характер средней удельной прибыли.
\item Доказывается, что стационарный показатель~(\ref{e10}) представим в~виде 
дроб\-но-ли\-ней\-но\-го интегрального функционала~(\ref{e11}), для которого явно 
определяются подынтегральные функции числителя и~знаменателя, а~следовательно, 
и~основная функция данного функционала.
\item Используется теорема об экстремуме дроб\-но-ли\-ней\-но\-го интегрального 
функционала. На основании утверждений этой теоремы уста\-нав\-ли\-ва\-ет\-ся, что 
исходная задача оптимального управления сводится к~исследованию на глобальный 
экстремум основной функции этого функционала, для которой получено явное 
аналитическое представление.
\end{enumerate}

Заметим, что такое исследование задач оптимального управления 
стохастическими системами фактически уже было проведено в~ряде работ П.\,В.~Шнуркова 
и~его соавторов. В~частности, в~работе~\cite{26} была рассмотрена модель 
управления для обрывающегося процесса восстановления, описывающего функционирование 
некоторой технической системы. Задача управления решалась для различных показателей 
эффективности и~надежности этой системы, имеющих структуру дроб\-но-ли\-ней\-но\-го 
интегрального функционала.

В работах~\cite{27, 28} рассматривались модели регенерирующих процессов 
для исследования сис\-тем управления запасами. Различные показатели качества 
управления были представлены в~форме дроб\-но-ли\-ней\-ных интегральных функционалов. 
Основные функции этих функционалов были найде\-ны в~явной форме и~исследовались 
на глобальный экстремум. В~работах~\cite{21,29} рассматривалась достаточно 
сложная полумарковская модель с~конечным множеством состояний, описывающая 
сис\-те\-му управления запасом непрерывного продукта. Показатели качества управления в~этой 
модели также имели структуру дроб\-но-ли\-ней\-ных интегральных функционалов, 
для основных функций которых были найдены явные аналитические представления. 
Упомянем также работы~\cite{30, 31}, в~которых была исследована полумарковская 
модель с~дис\-крет\-но-не\-пре\-рыв\-ным фазовым пространством. Показатели 
качества управления в~этой  модели были найдены в~явной форме как функции от 
двух непрерывных параметров управления.

Фактически во всех упомянутых работах уже был использован метод решения задачи 
оптимального управления регенерирующим или полумарковским случайным процессом, 
основанный на исследовании экстремальных свойств основной функции соответствующего 
дроб\-но-ли\-ней\-но\-го интегрального функционала. Из соображений, изложенных 
во\linebreak введении, следует, что в~период написания и~пуб\-ли\-кации этих работ данный метод 
не имел стро\-гого обоснования. Однако после публикации\linebreak работы~\cite{14} и~настоящего 
исследования можно утверж\-дать, что полученные в~них результаты полностью теоретически 
обоснованы.

Таким образом, изложенный выше метод решения проблемы оптимального управления 
полумарковскими процессами с~конечными множествами состояний может быть успешно 
реализован для многих задач, рассматриваемых в~различных областях прикладной 
теории вероятностей.

Практическая реализация численной процедуры поиска оптимального решения на примере\linebreak 
полумарковской модели управления запасом непрерывного продукта (подробнее 
см.~\cite{21, 29}), ба\-зи\-ру\-юща\-яся на изложенных выше результатах (в~частности, 
теореме~1), была осуществлена А.\,К.~Горшениным и~соавторами 
в~статье~\cite{Gorshenin2015}. Коротко опишем наиболее важные аспекты этой работы.

Для решения поставленной задачи опти\-мального управления была создана 
специальная программа \verb"Inventory" на встроенном языке программирования 
пакета \verb"MATLAB", ее возможности\linebreak кратко представле\-ны в~упомянутой ранее 
\mbox{статье}~\cite{Gorshenin2015}. В~программе \verb"Inventory" реализованы функции 
для оценивания через заданные исходные параметры вероятностных и~стоимостных 
характеристик модели, которые в~дальнейшем используются для поиска значений 
основной функции дроб\-но-ли\-ней\-но\-го функционала~(\ref{e17}). Точка глобального 
экстремума этой функции и~определяет оптимальное управление.

В качестве начальных данных необходимо задание следующих параметров:
\begin{itemize}
\item спрос и~вместимость склада;
\item разбиение множества значений объема запаса;
\item вероятностные характеристики, описывающие модель пополнения запаса;
\item условные математические ожидания длительностей задержек пополнения запаса;
\item функции для характеризации затрат и~доходов.
\end{itemize}

По итогам работы программы \verb"Inventory" ряд вспомогательных функций 
представляется в~аналитической форме (в частности, с~использованием аппарата 
символьных вычислений  \verb"Symbolic Toolbox"\linebreak пакета \verb"MATLAB"), выводится 
точка глобального экстремума функции нескольких вещественных переменных~(\ref{e17}), 
найденная с~помощью применения численных и~при\-бли\-жен\-но-ана\-ли\-ти\-че\-ских\linebreak 
аппроксимаций. 
Также формируются графики оценок значений ве\-ро\-ят\-ност\-но-сто\-и\-мост\-ных 
характеристик 
и~основной функции дроб\-но-ли\-ней\-но\-го функционала~(\ref{e17}), либо трехмерных 
сечений в~случае наличия более трех параметров управления (переменных).

Функциональность пакета \verb"Inventory" может быть расширена для практической 
реализации метода решения задачи поиска оптимального управ\-ле\-ния полумарковскими 
процессами с~конечными множествами состояний, рассмотренного в~данной статье.


 {\small\frenchspacing
 {%\baselineskip=10.8pt
 \addcontentsline{toc}{section}{References}
 \begin{thebibliography}{99}
 \bibitem{1}
\Au{Ховард Р.} Динамическое программирование и~марковские процессы~/ 
Пер. с~англ.~--- М.: Сов. радио, 1964. 189~с.
(\Au{Howard~R.\,A.} Dynamic programming and Markov processes.~--- 
Cambridge, MA, USA: MIT Press, 1960. 136~p.)
\bibitem{2} 
\Au{Рыков В.\,В.} Управляемые марковские процессы с~конечными пространствами 
состояний и~управлений~// Теория вероятностей и~ее применения, 1966. Т.~11. 
Вып.~2. С.~343--351.
\bibitem{3} 
\Au{Джевелл В.} Управляемые полумарковские процессы~// Кибернетич. сборник.~--- 
М.: Мир, 1967. Вып.~4. С.~97--134.
%{\em Jewell W.\,S.} Markov-renewal programming~// Operations Research, 1963. Vol.~11. P.~938--971.
\bibitem{4} 
\Au{Fox B.} Markov renewal programming by linear fractional programming~// 
SIAM J.~Appl. Math., 1966. Vol.~14. P.~1418--1432.
\bibitem{5} 
\Au{Denardo E.\,V.} Contraction mappings in the theory underlying dinamic programming~// 
SIAM Rev., 1967. Vol.~9. P.~165--177.

\bibitem{6} 
\Au{Howard R.\,A.} Research in semi-Markovian decision structures~// 
J.~Oper. Res. Soc. Japan, 1963. Vol.~6. P.~163--199.
\bibitem{7} 
\Au{Osaki S., Mine H.} Linear programming algorithms for Markovian decision processes~//
 J.~Math. Anal. Appl., 1968. Vol.~22. P.~356--381.
\bibitem{8} 
\Au{Майн Х., Осаки С.} Марковские процессы принятия решений~/ Пер. с~англ.~--- 
М.: Наука, 1977. 176~с.
(\Au{Mine~H., Osaki~S.} 
Markovian decision processes.~--- New York, NY, USA: 
American Elsevier Publishing Co., 1970. 142~p.)
\bibitem{9} 
\Au{Гихман И.\,И., Скороход А.\,В.} Управляемые случайные процессы.~--- 
Киев: Наукова думка, 1977. 251~с.
\bibitem{10} 
\Au{Luque-Vasquez F., Herndndez-Lerma~О.} Semi-Markov control models with average costs~// 
Appl. Math., 1999. Vol.~26. No.\,3. P.~315--331.
\bibitem{11} 
\Au{Vega-Amaya O., Luque-Vasquez~F.} Sample-path average cost optimality for 
semi-Markov control processes on Borel spaces: Unbounded costs and mean holding times~// 
Appl. Math., 2000. Vol.~27. No.\,3. P.~343--367.
\bibitem{12} 
Вопросы математической теории надежности~/ Под ред. Б.\,В. Гнеденко.~--- 
М.: Радио и~связь, 1983. 376~с.
\bibitem{13} 
\Au{Барзилович Е.\,Ю., Каштанов~В.\,А.} Некоторые математические вопросы теории 
обслуживания сложных систем.~---  М.: Сов. радио, 1971. 272~с.
\bibitem{14} 
\Au{Шнурков П.\,В.} О~решении проблемы безусловного экстремума для 
дроб\-но-ли\-ней\-но\-го интегрального функционала на множестве вероятностных мер~// 
Докл. РАН. Сер. Математика, 2016. Т.~470. №\,4. C.~387--392.
\bibitem{15} 
\Au{Ширяев А.\,Н.}  Вероятность.~--- М.:~МЦНМО, 2011. Кн.~1. 552~с. Кн.~2. 968~с.
\bibitem{16} 
\Au{Боровков А.\,А.} Теория вероятностей.~--- М.: Либроком, 2009. 656~c.
\bibitem{17} 
\Au{Хеннекен П.\,Л., Тортра А.} Теория вероятностей 
и~некоторые ее приложения.~--- М.: Наука, 1974. 472~c.
\bibitem{18} 
\Au{Халмош П.} Теория меры~/ Пер. с~англ.~--- М.: ИЛ, 1953. 282~c.
(\Au{Halmos~P.} Measure theory.~--- Litton Educational Publishing, Inc. 1950. 304~p.)
\bibitem{19} 
\Au{Королюк В.\,С., Турбин~А.\,Ф.} Полумарковские процессы и~их приложения.~--- 
Киев:~Наукова думка, 1976. 184~с.
\bibitem{20} 
\Au{Janssen J., Manca R.} Applied semi-Markov processes.~--- New York,
NY, USA: Springer, 2006. 309~p.
\bibitem{21} 
\Au{Шнурков П.\,В., Иванов~А.\,В.} Анализ дискретной полумарковской модели
 управления запасом непрерывного продукта при периодическом прекращении потребления~// 
 Дискретная математика, 2014. Т.~26. Вып.~1. С.~143--154.
\bibitem{22} 
\Au{Иванов~А.\,В.} Анализ дискретной полумарковской модели
 управления запасом непрерывного продукта при периодическом прекращении 
 потребления.~--- М.: НИУ ВШЭ, 2014.  Дисс.\ \ldots\ канд. физ.-мат. наук. 120~с.
\bibitem{23}  %23
\Au{Bajalinov~E.\,B.} Linear-fractional programming. 
Theory, methods, applications and software.~--- 
Boston/\linebreak Dordrecht/London: Kluwer Academic Publs., 2003. 423~p.

\bibitem{27} %27
\Au{Шнурков П.\,В., Мельников~Р.\,В.} Оптимальное управление запасом 
непрерывного продукта в~модели регенерации~// Обозрение прикладной 
и~промышленной математики, 2006. Т.~13. Вып.~3. С.~434--452.
\bibitem{28} 
\Au{Шнурков П.\,В., Мельников~Р.\,В.} 
Исследование проб\-ле\-мы управления запасом непрерывного продукта при детерминированной 
задержке поставки~// Автоматика и~телемеханика, 2008. Т.~10. С.~93--113.


\bibitem{24}  %26
\Au{Шнурков П.\,В.} Методы исследования задач оптимального обслуживания 
в~математической теории надежности.~--- 
М.: МИЭМ, 1983.  Дисс.\ \ldots\ канд. физ.-мат. наук.

 \bibitem{25}  %25
\Au{Кудрявцев Л.\,Д.} Курс математического анализа. Т.~1.~--- 
М.: Дрофа, 2006. 704~с.

\bibitem{26} %24
\Au{Шнурков П.\,В.} Оптимальное обслуживание на периоде 
до первого отказа системы~// Применение аналитических методов в~вероятностных
 задачах.~--- Киев: Институт математики АН УССР, 1986. С.~121--129.

\bibitem{29} 
\Au{Шнурков П.\,В., Иванов~А.\,В.} Исследование задачи оптимизации в~дискретной 
полумарковской модели управления непрерывным запасом~// Вестник МГТУ им.\ 
Н.\,Э. Баумана. Сер.\ Естественные науки, 2013. Т.~3. Вып.~50. С.~62--87.
\bibitem{30} 
\Au{Shnourkoff P.\,V.} The two-element system with one 
restoring device optimum maintenance~// Stoch. Anal. Appl., 1997. 
Vol.~15. No.\,5. P.~823--837.
\bibitem{31} 
\Au{Shnourkoff P.\,V.} The two-element system optimum maintenance tills the first fail~// 
Stoch. Anal. Appl., 2001. Vol.~19. No.\,6. P.~1005--1024.
\bibitem{Gorshenin2015} 
\Au{Gorshenin~A.\,K., Belousov~V.\,V., Shnourkoff~P.\,V.,
Ivanov~A.\,V.} Numerical research of the optimal control problem in the semi-Markov 
inventory model~// AIP Conference Proceedings, 2015. Vol.~1648. {250007}. 4~p.
%\bibitem{33} {\em Горшенин А.\,К., Белоусов В.\,В., Шнурков П.\,В.} 2016. Система управления запасами на основе стохастических полумарковских моделей. Свидетельство о государственной регистрации программы для ЭВМ \textnumero 2016619021.
 \end{thebibliography}

 }
 }

\end{multicols}

\vspace*{-6pt}

\hfill{\small\textit{Поступила в~редакцию 15.07.16}}

%\vspace*{8pt}

\newpage

\vspace*{-24pt}

%\hrule

%\vspace*{2pt}

%\hrule

%\vspace*{8pt}


\def\tit{ANALYTICAL SOLUTION OF~THE~OPTIMAL CONTROL TASK OF~A~SEMI-MARKOV 
PROCESS WITH~FINITE SET OF~STATES}

\def\titkol{Analytical solution of~the~optimal control task of~a~semi-Markov 
process with~finite set of~states}

\def\aut{P.\,V.~Shnurkov$^{1}$, A.\,K.~Gorshenin$^{2}$, and~V.\,V.~Belousov$^{2}$}

\def\autkol{P.\,V.~Shnurkov, A.\,K.~Gorshenin, and~V.\,V.~Belousov}

\titel{\tit}{\aut}{\autkol}{\titkol}

\vspace*{-9pt}


    
\noindent
$^1$National Research University Higher School of Economics, 34~Tallinskaya Str., 
Moscow, 123458, Russian\linebreak
$\hphantom{^9}$Federation

\noindent
$^2$Institute of Informatics Problems, Federal Research Center 
``Computer Science and Control'' of the Russian\linebreak
$\hphantom{^9}$Academy of Sciences, 44-2~Vavilova Str., 
Moscow 119333, Russian Federation



\def\leftfootline{\small{\textbf{\thepage}
\hfill INFORMATIKA I EE PRIMENENIYA~--- INFORMATICS AND
APPLICATIONS\ \ \ 2016\ \ \ volume~10\ \ \ issue\ 4}
}%
 \def\rightfootline{\small{INFORMATIKA I EE PRIMENENIYA~---
INFORMATICS AND APPLICATIONS\ \ \ 2016\ \ \ volume~10\ \ \ issue\ 4
\hfill \textbf{\thepage}}}

\vspace*{3pt}


\Abste{The theoretical verification of the new method of finding 
the optimal strategy of control of a~semi-Markov process with finite set of states is 
presented. The paper considers Markov randomized strategies of control, determined by 
a~finite collection of probability measures, corresponding to each state. The quality 
characteristic is the stationary cost index. This index is a~linear-fractional integral 
functional, depending on collection of probability measures, giving the strategy of control. 
Explicit analytical forms of integrands of numerator and denominator of this 
linear-fractional integral functional are known. The basis of consequent results is 
the new generalized and strengthened form of the theorem about an extremum of 
a~linear-fractional integral functional. It is proved that problems of existence 
of an optimal control strategy of a~semi-Markov process and finding this strategy 
can be reduced to the task of numerical analysis of global extremum for 
the given function, depending on finite number of real arguments.}

\KWE{optimal control of a~semi-Markov process; stationary cost index of quality control; 
linear-fractional integral functional}




\DOI{10.14357/19922264160408} 

\vspace*{-16pt}

\Ack
\noindent
The research was partially supported by the Russian Foundation 
for Basic Research (project 15-07-05316).



%\vspace*{3pt}

  \begin{multicols}{2}

\renewcommand{\bibname}{\protect\rmfamily References}
%\renewcommand{\bibname}{\large\protect\rm References}

{\small\frenchspacing
 {%\baselineskip=10.8pt
 \addcontentsline{toc}{section}{References}
 \begin{thebibliography}{99}
\bibitem{1-1}
\Aue{Howard,~R.\,A.} 1960. \textit{Dynamic programming and Markov processes}. 
Cambridge, MA: MIT Press. 136~p.
\bibitem{2-1}
\Aue{Rykov,~V.\,V.} 1966. Upravlyaemye markovskie protsessy 
s~konechnymi prostranstvami sostoyaniy i~upravleniy 
[Controlled Markov processes with finite spaces of states and controls ]. 
\textit{Teoriya veroyatnostey i~ee primeneniya} 
[Theory of Probability and Its Applications] 11(2):343--351.
\bibitem{3-1}
\Aue{Jewell,~W.\,S.} 1963. Markov-renewal programming. 
\textit{Oper. Res.} 11:938--971.
\bibitem{4-1}
\Aue{Fox,~B.} 1966. Markov renewal programming by linear fractional programming. 
\textit{SIAM J.~Appl. Math.} 14:1418--1432.
\bibitem{5-1}
\Aue{Denardo, E.\,V.} 1967. Contraction mappings in the theory underlying dinamic 
programming. \textit{SIAM Rev.} 9:165--177.
\bibitem{6-1}
\Aue{Howard,~R.\,A.} 1963. Research in semi-Markovian decision structures. 
\textit{J.~Oper. Res. Soc. Japan} 6:163--199.
\bibitem{7-1}
\Aue{Osaki,~S., and H.~Mine.} 1968. Linear programming algorithms 
for Markovian decision processes. \textit{J.~Math. Anal. Appl.} 22:356--381.
\bibitem{8-1}
\Aue{Mine,~H., and S.~Osaki.} 1970. 
\textit{Markovian decision processes}. New York, NY: Elsevier. 142~p.
\bibitem{9-1}
\Aue{Gikhman,~I.\,I., and A.\,V.~Skorokhod.} 1977. 
\textit{Upravlyaemye sluchaynye protsessy} 
[Controlled random processes]. Kiev: Naukova Dumka. 251~p.
\bibitem{10-1}
\Aue{Luque-Vasquez,~F., and О.~Herndndez-Lerma.} 1999. 
Semi-Markov control models with average costs. \textit{Appl. Math.} 26(3):315--331.
\bibitem{11-1}
\Aue{Vega-Amaya,~O., and  F.~Luque-Vasquez.} 2000.  
Sample-path average cost optimality for semi-Markov control processes on Borel spaces: 
Unbounded costs and mean holding times. \textit{Appl. Math.} 27(3):343--367.
\bibitem{12-1}
Gnedenko,~B.~V., ed. 1983. 
\textit{Voprosy matematicheskoy teorii nadezhnosti} 
[Problems of the mathematical theory of reliability].  Moscow: Radio i~svyaz'. 376~p.
\bibitem{13-1}
\Aue{Barzilovich,~E.\,Yu., and V.\,A.~Kashtanov.} 1971. 
\textit{Nekotorye matematicheskie voprosy teorii obsluzhivaniya slozhnykh sistem}  
[Some mathematical questions in theory of complex systems maintenance]. 
Moscow: Sovetskoe radio. 272~p.
\bibitem{14-1}
\Aue{Shnurkov,~P.\,V.} 2016. Solution of the unconditional extremum problem for 
a~linear-fractional 
integral functional on a~set of probability measures. 
\textit{Dokl. Math.} 94(2):550--554.
\bibitem{15-1} %14
\Aue{Shiryaev,~A.\,N.} 2016. 
\textit{Probability-1}. Graduate texts in mathematics ser.
New York, NY: Springer. Vol.~95. 503~p.;
2017. \textit{Probability-2.} Vol.~900. 500~p.
\bibitem{16-1}
\Aue{Borovkov,~А.\,А.} 2009. 
\textit{Teoriya veroyatnostey} [Probability theory]. Moscow: Librokom. 656~p.
\bibitem{17-1}
\Aue{Khenneken,~P.\,L., and A.~Tortra.} 1974. 
\textit{Teoriya veroyatnostey i~nekotorye ee prilozheniya} 
[Probability theory and some of its applications]. Moscow: Nauka. 472~p.
\bibitem{18-1}
\Aue{Halmos,~P.} 1950. \textit{Measure theory}. Litton Educational Publishing. 304~p.
\bibitem{19-1}
\Aue{Korolyuk, V.\,S., and A.\,F.~Turbin.} 1976. 
\textit{Polumarkovskie protsessy i~ikh prilozheniya} 
[Semi-Markov processes and their applications]. Kiev: Naukova Dumka. 184~p.
\bibitem{20-1}
\Aue{Janssen,~J., and R.~Manca.} 2006. 
\textit{Applied semi-Markov processes}. New York, NY: Springer. 309~p.
\bibitem{21-1}
\Aue{Shnurkov,~P.\,V, and A.\,V~Ivanov.} 2015. Analysis of a~discrete semi-Markov model of continuous inventory 
control with periodic interruptions of consumption. 
\textit{Discrete Math. \mbox{Appl}.} 25(1):59--67.
\bibitem{22-1} %21
\Aue{Ivanov,~A.\,V.} 2014. Analiz diskretnoy polumarkovskoy modeli upravleniya 
zapasom nepreryvnogo produkta pri periodicheskom prekrashchenii potrebleniya 
[Analysis of a~discrete semi-Markov control model of continuous product inventory 
in a~periodic cessation of consumption].  
Moscow: Natsional'nyy Issledovatel'skiy Universitet ``Vysshaya Shkola Ekonomiki.'' 
PhD Thesis. 120~p.
\bibitem{23-1} %22
\Aue{Bajalinov,~E.\,B.} 2003. 
\textit{Linear-fractional programming. Theory, methods, applications and software}. 
Boston/\linebreak Dordrecht/London: Kluwer Academic Publs. 423~p.
\bibitem{26-1} %24
\Aue{Shnurkov,~P.\,V., and R.\,V.~Mel'nikov.} 2006. Optimal'noe upravlenie 
zapasom nepreryvnogo produkta v modeli regeneratsii [Optimal control of 
a~continuous product inventory in the regeneration model]. 
\textit{Obozrenie prikladnoy i~promyshlennoy matematiki} [Rev. Appl. Ind. Math.]
13(3):434--452.

\bibitem{25-1} %25
\Aue{Shnurkov,~P.\,V., and R.\,V.~Mel'nikov.} 2008. 
Analysis of the problem of continuous-product inventory control under deterministic 
lead time. \textit{Automat. Rem. Contr.} 69(10):1734--1751.

\columnbreak

\bibitem{24-1} %26
\Aue{Shnurkov,~P.\,V.} 1983. Metody issledovaniya zadach optimal'nogo obsluzhivaniya 
v~matematicheskoy teorii nadezhnosti [Research methods of optimal service problems 
in the mathematical theory of reliability].  
Moscow: Moskovskiy Institut Elektronnogo Mashinostroeniya.  PhD Thesis. 


\bibitem{27-1} %27
\Aue{Kudryavtsev,~L.\,D.} 2006. 
\textit{Kurs matematicheskogo analiza} 
[A~course of mathematical analysis]. Vol.~1. Moscow: Drofa. 704~p.

\bibitem{28-1}
\Aue{Shnurkov,~P.\,V.} 1986. Optimal'noe obsluzhivanie na periode do 
pervogo otkaza sistemy [The optimum service period until the first system failure]. 
\textit{Primenenie analiticheskikh metodov v~veroyatnostnykh zadachakh} 
[The application of analytical methods in probabilistic tasks]. Kiev:
Institute of Mathematics of the Academy of Sciences of the USSR. 121--129.

\bibitem{29-1}
\Aue{Shnurkov,~P.\,V., and A.\,V.~Ivanov.} 2013. Issledovanie zadachi optimizatsii 
v~diskretnoy polumarkovskoy modeli upravleniya nepreryvnym zapasom 
[Study of the optimization problem in discrete semi-Markov model of continuous 
inventory control]. \textit{Vestnik MGTU im.\ N.\,E.~Baumana. Ser. 
Estestvennye nauki} [Vestnik of MSTU named after N.\,E.~Bauman. Ser. Natural sciences] 
3(50):62--87.
\bibitem{30-1}
\Aue{Shnourkoff,~P.\,V.} 1997. The two-element system with one restoring device 
optimum maintenance.  \textit{Stoch. Anal. Appl.} 15(5):823--837.
\bibitem{31-1}
\Aue{Shnourkoff,~P.\,V.} 2001. The two-element system optimum maintenance tills 
the first fail. \textit{Stoch. Anal. Appl.} 19(6):1005--1024.
\bibitem{32-1}
\Aue{Gorshenin,~A.\,K., V.\,V.~Belousov, P.\,V.~Shnourkoff, and A.\,V.~Ivanov.}
2015. Numerical research of the optimal control problem in the semi-Markov 
inventory model. \textit{AIP Conference Proceedings} 1648:250007.
\end{thebibliography}

 }
 }

\end{multicols}

\vspace*{-3pt}

\hfill{\small\textit{Received July 15, 2016}}

\Contr

\noindent
\textbf{Shnurkov Peter V.} (b.\ 1953)~---
 Candidate of Science (PhD) in physics and mathematics, 
 associate professor, National Research University Higher School of Economics, 
 34~Tallinskaya Str., Moscow 123458, Russian Federation; \mbox{pshnurkov@hse.ru} 
 
 \vspace*{3pt}
 
 \noindent
\textbf{Gorshenin Andrey K.}  (b.\ 1986)~---
Candidate of Science (PhD) in physics and mathematics, leading scientist, 
Institute of Informatics Problems, Federal Research Center ``Computer Science 
and Control'' of the Russian Academy of Sciences, 44-2~Vavilov Str., Moscow 119333, 
Russian Federation; associate professor, Federal State Budget Educational 
Institution of Higher Education ``Moscow Technological University,'' 
78~Vernadskogo Ave., Moscow 119454, Russian Federation;
\mbox{agorshenin@frccsc.ru}

\vspace*{3pt}

\noindent
\textbf{Belousov Vasiliy V.} (b.\ 1977)~---
Candidate of Science (PhD) in technology, senior scientist, Institute of 
Informatics Problems, Federal Research Center ``Computer Science and Control'' 
of the Russian Academy of Sciences, 44-2~Vavilov Str., Moscow 119333, Russian 
Federation; \mbox{VBelousov@ipiran.ru}
\label{end\stat}


\renewcommand{\bibname}{\protect\rm Литература}         %15
\include{donskoy}         %16
\def\stat{zatsman}

\def\tit{ТРАНСФОРМАЦИИ ОБЪЕКТОВ ПЕРВОГО И~ВТОРОГО ПОРЯДКА 
В~ЛЕКСИКОГРАФИЧЕСКОЙ ИНФОРМАЦИОННОЙ СИСТЕМЕ$^*$}

\def\titkol{Трансформации объектов первого и~второго порядка 
в~лексикографической информационной системе}

\def\aut{И.\,М.~Зацман$^1$}

\def\autkol{И.\,М.~Зацман}

\titel{\tit}{\aut}{\autkol}{\titkol}

\index{Зацман И.\,М.}
\index{Zatsman I.\,M.}


{\renewcommand{\thefootnote}{\fnsymbol{footnote}} \footnotetext[1]
{Исследование выполнено в~ФИЦ ИУ РАН за счет гранта Российского научного фонда №\,24-18-00155, {\sf 
https://rscf.ru/project/24-18-00155}. Работа выполнялась с~использованием инфраструктуры Центра 
коллективного пользования <<Высокопроизводительные вычисления и~большие данные>> (ЦКП 
<<Информатика>>) ФИЦ ИУ РАН (г.\ Москва).}}


\renewcommand{\thefootnote}{\arabic{footnote}}
\footnotetext[1]{ Федеральный исследовательский центр <<Информатика и~управление>> Российской академии наук, 
\mbox{izatsman@yandex.ru}}

\vspace*{-12pt}


  
  \Abst{Рассматриваются теоретические основания проектирования информационных 
технологий (ИТ) интеграции двуязычных словарей и~параллельных корпусов. Дано описание 
первых результатов создания третьего уровня классификации трансформаций объектов 
предметной области информатики, которую предполагается использовать при создании 
концепции лексикографической информационной системы, обеспечивающей интеграцию. 
Все сущности информатики в~статье разделены на два глобальных класса: объекты и~их 
трансформации. Для каждого такого класса конструируется своя классификация. Ранее были 
описаны два верхних уровня классификации трансформаций объектов предметной области. 
В~данной статье рассматривается третий уровень этой классификации. Основанием для 
построения самого верхнего ее уровня служило деление предметной области информатики 
на среды (ментальная, сенсорно воспринимаемая, цифровая и~ряд других сред), каждая из 
которых по определению включает объекты одной природы. Основанием для построения 
второго уровня классификации трансформаций объектов служила типология знаковых  
сис\-тем А.~Соломоника. Цель статьи состоит в~систематизации трансформаций первого 
и~второго порядка объектов предметной области на третьем уровне этой классификации. 
Основанием для систематизации служит средовая версия иерархии Акоффа.}
  
  \KW{объекты предметной области; трансформации объектов; классификация; данные; 
информация; знание; лексикографическая информационная сис\-тема}

\DOI{10.14357/19922264240211}{VZTGVV}
  
\vspace*{3pt}


\vskip 10pt plus 9pt minus 6pt

\thispagestyle{headings}

\begin{multicols}{2}

\label{st\stat}
  
\section{Введение}

\vspace*{-9pt}

  Возникновение параллельных корпусов, в~которых предложениям 
оригинального текста со\-по\-став\-ле\-ны предложения его перевода, обеспечило 
возможность контрастивного лингвистического\linebreak \mbox{анализа} на принципиально 
новом уровне полноты и~точности, недостижимом в~докорпусную эпоху. 
Пионерскими в~этой области стали работы \mbox{1990-х~гг}. Стига Йоханссона  
с~анг\-ло-нор\-веж\-ским корпусом~[1]. В России параллельные корпусы стали 
формироваться в~начале XXI~века в~рамках Национального корпуса русского 
языка~[2].
  
  Создатели двуязычных словарей используют параллельные корпусы для 
сбора материала и~эмпирической проверки своих гипотез, касающихся 
межъязы\-ко\-вой эквивалентности. Ценность параллельных корпусов 
определяется тем, что в~лингвистике этап сбора исходного материала считается 
наиболее трудоемким и~наименее творческим, а~параллельные корпусы 
позволяют значительно сэкономить время и~силы для творческого этапа 
создания словарей~[3].
 % 
  При этом двуязычные словари, создаваемые на основе исходного материала, 
извлеченного из параллельных корпусов, сейчас формируются без связей с~их 
текстами. Другими словами, онлайновые связи созданных словарей 
с~параллельными корпусами, которые служили источниками исходного 
материала, отсутствуют. 

Параллельные корпусы постоянно пополняются 
новыми текстами, в~предложениях которых можно обнаружить новые значения 
слов и~устойчивых словосочетаний. Однако при этом отсутствуют методы 
и~средства оперативного обновления словарей по корпусным данным. 
В~настоящее время проблема установления связей между двуязычными 
словарями и~параллельными корпусами (далее~--- проблема интеграции) 
находится на стадии поиска концептуальных подходов к~их интеграции на 
уровне значений.
  
  Подход к~решению проблемы интеграции, предлагаемый в~статье, учитывает 
  и~появление новых значений слов и~устойчивых словосочетаний, и~динамику 
смысловых значений, которая обусловлена развитием и~пополнением знания 
лингвистов, фиксирующих эти значения в~результате семантического анализа 
пополняемых корпусных данных. Проведенные эксперименты показали, что 
обнаружение нового лингвистического знания обусловливает и~формирование 
дефиниций новых значений, и~пересмотр уже существующих дефиниций~[4, 5].
  
  Например, в~проведенных экспериментах с~использованием ЦКП 
<<Информатика>> ФИЦ ИУ РАН фиксировалась эволюция значений немецких 
модальных глаголов, исходное состояние значений которых было описано 
в~не\-мец\-ко-рус\-ском словаре. В~экспериментальном массиве текстов как 
потенциальных источниках нового знания 16\,268 предложений содержали 
немецкие модальные глаголы и~в~2041 из них встречался глагол sollen. 
В~начале эксперимента в~словаре были описаны~12~значений этого модального 
глагола. По окончании эксперимента лингвисты обнаружили два новых его 
значения, согласовали их дефиниции и~описали эволюцию дефиниций~[6, 7].
  
  Таким образом, для решения проблемы интеграции требуется фиксировать 
новое знание, обнаруженное лингвистами в~текстовых данных параллельных 
корпусов, отслеживать эволюцию знания, представленного в~виде дефиниций 
значений слов и~устойчивых словосочетаний, и,~соответственно, 
актуализировать электронные двуязычные словари. Предлагаемый 
концептуальный подход к~интеграции, который планируется реализовать 
в~процессе проектирования лексикографической информационной сис\-те\-мы, 
фиксирующей эволюцию лингвистического знания, основан на решении 
следующих задач:\\[-14pt]
  \begin{itemize}
  \item категоризация трех базовых понятий информатики, включенных 
  в~иерархию Акоффа~[8] (данные, информация, знание), на объекты 
проектируемой сис\-те\-мы, которая необходима, чтобы фиксировать 
<<кванты>> нового знания и~отслеживать его эволюцию в~этой сис\-теме;\\[-15pt]
  \item  систематизация трансформаций объектов этой сис\-темы.\\[-14pt]
  \end{itemize}
  
  Цель статьи и~состоит в~решении двух задач: категоризации трех базовых 
понятий информатики на объекты лексикографической информационной  
сис\-те\-мы и~сис\-те\-ма\-ти\-за\-ции трансформаций первого и~второго порядка 
ее объектов.
  
  Трансформациями первого порядка, о которых сказано в~формулировке цели 
статьи, называются взаимные преобразования между двумя объектами  
сис\-те\-мы одной природы. Например, перевод в~сис\-те\-ме текста с~русского 
языка на английский относится к~ним. Трансформациями второго порядка 
и~выше называются взаимные преобразования между двумя и~более объектами 
разной природы. Например, кодирование символов текс\-та компьютерными 
кодами и~их декодирование относятся по определению к~трансформациям 
второго порядка.

%\vspace*{-9pt}
  
\section{Процессы трансформаций в~информатике}

%\vspace*{-3pt}

Процессы трансформаций, рассматриваемые в~статье, относятся к~теоретическому ядру информатики, а~не 
только к~проектированию лексикографической информационной сис\-те\-мы. Например, из трех основных 
подходов к~описанию предметной об\-ласти информатики\footnote{В статье предметная область информатики 
трактуется согласно концепции полиадического компьютинга Пола Розенблума~\cite{9-zac}.} (объектный, 
трансформационный и~синтетический) сис\-те\-ма\-ти\-за\-ция трансформаций ближе всего ко второму 
подходу. Примерами первого подхода, в~рамках которого основное внимание уделяется объектам предметной 
области информатики и~в~меньшей степени отношениям\linebreak между ними, могут служить  
работы~\cite{8-zac, 10-zac, 11-zac}; \mbox{примерами} второго подхода, в~рамках которого основное внимание 
уделяется трансформациям и~в~меньшей степени трансформируемым объектам,~---  
работы~\cite{12-zac, 13-zac}; примерами третьего, синтетического подхода, в~котором уделяется внимание 
и~объектам предметной об\-ласти информатики, и~отношениям между ними, могут служить работы~\cite{14-zac, 
15-zac, 16-zac, 17-zac, 18-zac}.

  Таким образом, для описания трансформаций объектов лексикографической 
информационной\linebreak системы предпочтительнее всего трансформационный 
подход, который упоминается и~в определениях информатики. Например, 
в~2009~г.\ П.~Деннинг и~П.~Розенблум сформулировали суть \mbox{информатики} как 
компьютинга следующим образом: <<$\ldots$информатика~--- это не просто 
алгоритмы и~структуры данных; это преобразования [трансформации] 
представлений>>~\cite{12-zac}. Чуть позже, в~контексте краткого описания 
парадигмы информатики как компьютинга, П.~Деннинг и~П.~Фриман изменили 
эту формулировку на такую: <<Центральный объект внимания в~информатике 
можно определить как информационные процессы~--- \textit{естественные или 
искусственные процессы, преобразующие информацию} (курсив мой~--- 
И.\,З.)>>~\cite{13-zac}. Согласно парадигме, предлагаемой авторами этой 
статьи, на начальном этапе проектирования автоматизированных систем 
базовыми элементами моделей их функционирования служат 
\textit{информационные про\-цессы}.
  
  Однако если 15~лет назад в~формулировке из работы~\cite{13-zac} шла речь 
о~процессах, преобразующих информацию, то в~последние~10~лет в~спектр 
процессов трансформаций все чаще стали включать процессы, преобразующие 
не только информацию, но также и~другие объекты автоматизированных 
систем, в~первую очередь данные и~знания~[19--21]. Например, Виктория 
Стодден, позиционируя науку о~данных как одну из дисциплин информатики, 
говорит, что центральный объект исследований в~науке о~данных~--- это 
<<изучение обобщаемого извлечения знания из данных>>~\cite{21-zac}. 
Увеличение и~чис\-ла объектов, и~спект\-ра процессов их трансформаций 
в~автоматизированных сис\-те\-мах обуслов\-ли\-ва\-ет не\-об\-хо\-ди\-мость 
систематизации и~объектов, и~процессов их трансформаций на начальном этапе 
проектирования сис\-тем.
  
  Для создания концепции лексикографической информационной сис\-те\-мы 
и~проектирования ИТ, обеспечивающих интеграцию 
двуязычных словарей и~параллельных корпусов, сначала выполним 
категоризацию на объекты этой сис\-те\-мы трех базовых понятий информатики 
(данные, информация, знание) в~контексте построения классификаций 
сущностей ее предметной об\-ласти.
  
  Необходимость использования классификаций информатики в~процессе 
создания концепции проиллюстрируем, используя иерархию  
Акоффа~\cite{8-zac}. Он использовал принцип их вертикального размещения 
в~иерархии снизу вверх: данные, информация и~знание. Еще в~ней есть термин 
<<мудрость>>, который в~статье не рассматривается. Такое размещение Акофф 
прокомментировал так: <<Каждое из пе\-ре\-чис\-лен\-ных понятий [кроме данных] 
содержит в~себе нижестоящие$\ldots$>>~\cite{8-zac}.
  
  Этому принципу размещения и~комментарию Акоффа свойственны 
недостатки, проанализированные, в~частности, в~работе~\cite{10-zac}. Главный 
вывод, к~которому пришла Роули после изучения иерархии Акоффа, 
заключается в~следующем: <<$\ldots$информация определяется в~терминах 
данных, знание~--- в~терминах информации$\ldots$ но существует меньше 
консенсуса в~описании трансформаций, которые преобразуют сущности, 
расположенные ниже в~иерархии, в~те, которые находятся над ними, что 
приводит к~их терминологической неопределенности>>~\cite{10-zac}. Причина 
этой неопределенности, скорее всего, в~том, что базовые понятия информатики 
включены в~иерархию Акоффа изолированно от общего контекста 
классификаций сущностей ее предметной об\-ласти.

%\vspace*{-9pt}
  
\section{Классификации сущностей информатики}


%\vspace*{-2pt}

  Все сущности предметной области информатики в~работах~[22, 23] 
разделены на два глобальных класса: ее объекты и~их трансформации. Для 
каждого такого класса была предложена своя классификация. 
В~работе~\cite{22-zac} дано описание классификации объектов предметной 
области информатики, первый уровень которой содержит базовые понятия ее 
предметной области (данные, информация, знания и~др.).  
В~работе~\cite{23-zac} дано описание двух верхних уровней классификации 
трансформаций объектов предметной об\-ласти (см.\ рисунок 
в~работе~\cite{23-zac}). Основанием для построения самого верхнего ее уровня послужило деление 
предметной области информатики на среды\footnote{В~работе~\cite{24-zac} дано описание пяти сред 
предметной области информатики (ментальная; сенсорно воспринимаемая, или информационная; 
цифровая; нейро- и~ДНК-среда), каждая из которых по определению включает объекты одной и~той же 
природы.} и~степень разнообразия природы объектов, вовлеченных в~трансформации:
\begin{itemize}
\item  первый класс верхнего уровня классификации включает 
трансформации объектов в~пределах среды только одной природы 
(трансформации первого порядка);
\item  второй класс включает трансформации объектов, относящихся 
к~двум средам разной природы (трансформации второго порядка);
\item третий и~последующие классы включают трансформации объектов, 
относящихся к~трем и~более средам разной природы (трансформации 
третьего и~более высоких порядков).
\end{itemize}

  В работе~\cite{23-zac} были приведены примеры для трех первых классов 
трансформаций, включая пример трансформаций объектов, относящихся 
к~двум средам разной природы (компьютерное кодирование символов текстов 
с~по\-мощью таб\-лиц Unicode).
  
Основанием для построения второго уровня классификации трансформаций объектов послужила типология 
знаковых сис\-тем А.~Соломоника~\cite[c.~131]{25-zac}: естественные знаковые сис\-те\-мы, образные,  
ес\-тест\-вен\-но-язы\-ко\-в$\acute{\mbox{ы}}$е,  
вер\-баль\-но-не\-сло\-вес\-ные сис\-те\-мы записи\footnote{Под системой записи понимается знаковая 
система, сочетающая вербальные знаки с~несловесными (языки нотной записи, карт, таблиц и~др.).} 
и~формализованные знаковые сис\-те\-мы, включая математические. Введем понятие обобщенного текста~--- 
это текст, который может быть создан в~любой из перечисленных знаковых систем. Тогда обобщенные тексты 
могут быть естественными, образными, ес\-тест\-вен\-но-язы\-ко\-в$\acute{\mbox{ы}}$\-ми,  
вер\-баль\-но-не\-сло\-вес\-ны\-ми и~формализованными. Второй уровень классификации трансформаций 
охватывает не все виды объектов предметной  
об\-ласти информатики, а~только перечисленные~5~видов текс\-тов и~их представления, вовлеченные 
в~процессы трансформаций в~одной или более средах вместе с~данными, знанием и~его концептами.

\begin{figure*}[b] %fig1
\vspace*{6pt}
      \begin{center}
     \mbox{%
\epsfxsize=121.191mm 
\epsfbox{zac-1.eps}
}
\end{center}
\vspace*{-6pt}
\Caption{Средовая версия иерархии Акоффа}
\end{figure*}

\section{Классификация трансформаций: построение~третьего 
уровня}

  Основанием для систематизации трансформаций первого и~второго порядка 
на третьем уровне этой классификации служит иерархия Акоффа~\cite{8-zac}, 
на основе которой и~была создана ее средов$\acute{\mbox{а}}$я версия~[26, 
27]. Для создания средов$\acute{\mbox{о}}$й версии была выполнена 
категоризация трех базовых понятий информатики (данные, информация, 
знания) на объекты лексикографической информационной сис\-те\-мы 
в~процессе создания ее концепции\linebreak (рис.~1).
  


  В отличие от классической иерархии Акоффа, в~ее 
средов$\acute{\mbox{о}}$й версии различаются три вида данных: сенсорно 
воспринимаемые, цифровые и~те данные, которые генерируются 
искусственными нейронными сетями (ИНС) в~системах искусственного интеллекта 
(далее~--- ИИ-дан\-ные). Последний вид данных необходим, например, для 
различения входа и~выхода процесса применения обученной 
ИНС в~цифровой модели генерации знания, описанию которой 
посвящена работа~\cite{27-zac}.
  
  Также предлагается различать два вида информации: сенсорно 
воспринимаемая и~цифровая. Кроме знания в~средов$\acute{\mbox{у}}$ю 
версию добавлены концепты и~ментальные образы сенсорно воспринимаемых 
данных. Последние служат промежуточной сущностью между сенсорно 
воспринимаемыми данными и~генерируемым знанием при описании процессов 
извлечения знания из текстовых данных лексикографической информационной 
системы. Описание объектов средов$\acute{\mbox{о}}$й версии иерархии 
Акоффа (см.\ рис.~1) и~отношений между ними дано в~работах~\cite{26-zac, 28-zac}.
  
  В средов$\acute{\mbox{о}}$й версии число объектов равно восьми. Если 
учитывать направления трансформаций, то между восемью объектами на 
рис.~1 она включает~16 их видов (трансформации на границе между сенсорно 
воспринимаемыми данными и~информацией, обозначенные символом~<<?>>, 
в~статье не рас\-смат\-ри\-ва\-ют\-ся). В~будущем число объектов 
в~средов$\acute{\mbox{о}}$й версии, которая выбрана как основание для 
сис\-те\-ма\-ти\-за\-ции трансформаций первого и~второго порядка, может быть 
увеличено. Для построения классификации трансформаций 
важ\-но не возможное увеличение числа объектов 
и~трансформаций между ними, а то, что их виды в~средов$\acute{\mbox{о}}$й 
версии распределены между трансформациями первого и~второго порядка. Из 
16~видов на рис.~1 шесть относятся к~трансформациям первого порядка, это\linebreak 
виды с~номерами~7, 8, 13--16 (далее~--- типология трансформаций первого 
порядка), а~десять~--- к~трансформациям второго порядка, это виды 
с~\mbox{номерами}~1--6 и~9--12 (далее~--- типология трансформаций второго 
порядка). Разместим обе типологии на третьем уровне классификации (см.\ ее 
схему на рис.~2). Перечислим виды трансформаций первой типологии, вводя 
в~скобках их краткие названия, используемые ниже на рис.~3:
  \begin{description}
  \item[\,] 7~--- членение знания на концепты с~помощью одной или нескольких 
знаковых систем (далее~--- членение знания);
  \item[\,] 8~--- формирование знания на основе концептов (формирование 
знания);
  \item[\,] 13~--- обучение ИНС;
  \end{description}
  
  \vspace*{-6pt}
  
  \pagebreak
  
  \end{multicols}
  
  \begin{figure*} %fig2
\vspace*{1pt}
      \begin{center}
     \mbox{%
\epsfxsize=127.513mm 
\epsfbox{zac-2.eps}
}
\end{center}
\vspace*{-9pt}
\Caption{Схема трех верхних уровней классификации трансформаций объектов (объединены 
по три слоя и~для второго, и~для третьего уровней этой классификации)}
\end{figure*}
  
  \begin{multicols}{2}
  
  \noindent
  \begin{description}
  \item[\,] 14~--- восстановление обучающей информации на основе 
содержания обученной ИНС (обращение ИНС);
  \item[\,] 15~--- использование обученной ИНС (использование ИНС);



  \item[\,] 16~--- восстановление исходных данных, соответствующих 
полученным результатам работы обучен\-ной ИНС (восстановление исходных данных 
по результатам ИНС).
  \end{description}
  
  
  Не все виды трансформаций 13--16 поддерживаются в~конкретных системах 
искусственного интеллекта, но с~теоретической точки зрения все их 
предлагается включить в~первую типологию для полноты спектра видов 
трансформаций.
  
  Перечислим виды трансформаций второй типологии:
  \begin{description}
  \item[\,] 1~--- декодирование цифровых данных в~компьютерных системах 
(декодирование данных);
  \item[\,]  2~--- кодирование сенсорно воспринимаемых данных (кодирование 
данных);
  \item[\,] 3~--- ментальное копирование сенсорно воспринимаемых данных 
(ментальное копирование);
  \item[\,] 4~--- восстановление сенсорно воспринимаемых данных по 
ментальным образам (восстановление по образам);
  \item[\,] 5~--- смысловая интерпретация без деления на концепты ментальных 
образов сенсорно воспринимаемых данных (смысловая интерпретация);
  \item[\,] 6~--- восстановление ментальных образов (восстановление образов);
  \item[\,] 9~--- представление концептов в~виде сенсорно воспринимаемой 
информации, например текс\-та\-ми, формулами, таблицами, рисунками и~т.\,д.\ 
(представление концептов);
  \item[\,] 10~--- понимание смысла сенсорно воспринимаемой информации 
(понимание смысла);
  \item[\,] 11~--- кодирование сенсорно воспринимаемой информации 
(кодирование информации);
\end{description}

\vspace*{-6pt}

\pagebreak

\end{multicols}

\begin{figure*} %fig3
\vspace*{1pt}
      \begin{center}
     \mbox{%
\epsfxsize=163mm 
\epsfbox{zac-3.eps}
}
\end{center}
\vspace*{-9pt}
\Caption{Схема частного случая классификации трансформаций объектов (трансформации 
пронумерованы согласно рис.~1)}
\end{figure*}

\begin{multicols}{2}

\noindent
\begin{description}

  \item[\,] 12~--- декодирование цифровой информации (декодирование 
информации).
  \end{description}
  
  Отметим, что в~существующих ИТ
  и~компьютерных системах наиболее часто используются виды 
трансформаций~13 и~15 типологии первого порядка и~1, 2, 11 и~12 типологии 
второго порядка. На рис.~2 в~первом слое третьего уровня классификации 
показаны типологии первого порядка без указания числа трансформаций в~них 
и~без детализации трансформируемых объектов.
  
  Во втором слое третьего уровня классификации условно (без названий) 
показаны типологии второго порядка. Также на рис.~2 в~третьем слое третьего 
уровня классификации условно (также без названий) показаны типологии 
третьего порядка, которые планируется рассмотреть в~отдельной статье. По 
определению они должны включать трансформации между тремя объектами 
разной природы, но средов$\acute{\mbox{а}}$я версия иерархии Акоффа 
включает трансформации только между двумя объектами разной природы. 
Поэтому потребуется другое основание для их систематизации (ранее были 
рассмотрены отдельные примеры трансформаций третьего 
порядка\footnote{Далеко не всегда трансформации третьего и~более высоких порядков можно 
рассматривать как последовательность трансформаций второго порядка. Примером этого могут 
служить трансформации в~процессе обучения пациента пользованию роботизированной рукой, 
охватывающие личностные концепты пациента, релевантные его намерениям, сигналы активности 
мозга как объекты нейросреды и~компьютерные коды~\cite{29-zac}.}~\cite{29-zac}).

\section{Классификация трансформаций: частный~случай}

  Выше было отмечено, что в~будущем число объектов 
в~средов$\acute{\mbox{о}}$й версии иерархии Акоффа может быть увеличено. 
Это означает, что увеличатся и~чис\-ло объектов, и~чис\-ло трансформаций между 
ними в~классификации трансформаций, так как эта средов$\acute{\mbox{а}}$я 
версия служит по определению основанием для систематизации 
трансформаций первого и~второго порядка. Поэтому на третьем уровне рис.~2 
указаны типологии без детализации объектов и~без указания числа 
трансформаций в~каждой из них. С~одной стороны, при таком подходе 
получаем достаточно общий вид этой классификации, так как она не зависит от 
числа объектов в~том или ином варианте средов$\acute{\mbox{о}}$й версии 
(и~это существенно упрощает рис.~2). С~другой стороны, на третьем уровне 
такой общей классификации подразумевается, но не эксплицируется природа 
трансформируемых объектов и~их возможные сочетания в~трансформациях. 

При проектировании лексикографической информационной системы важно 
эксплицировать природу трансформируемых объектов и~их возможные 
сочетания.
  %
  Поэтому в~парадигму информатики~\cite{30-zac} кроме общей 
классификации трансформаций предлагается включать и~ее частные случаи, 
эксплицирующие природу трансформируемых объектов. 

В~этом разделе 
рассмотрим один частный случай, когда используются только естественные 
знаковые сис\-те\-мы из типологии А.~Соломоника~\cite{25-zac} вместе 
с~данными, знанием и~его концептами. Чис\-ло естественных языков при этом не 
ограничено. И~этот частный случай классификации включает только три 
класса природных трансформаций (первого, второго и~третьего порядка, см.\ 
схему классификации на рис.~3).
  
  Первый и~второй уровни схемы общей классификации (см.\ рис.~2) можно 
объединить в~один уровень в~этом частном случае. Ниже этого уровня 
приведено содержание типологий первого и~второго порядка без содержания 
типологий третьего по\-рядка.




  Наполнение типологий первого и~второго порядка соответствует 
средов$\acute{\mbox{о}}$й версии иерархии Акоффа на рис.~1, содержащей 
6~видов трансформаций типологии первого порядка и~10~видов 
трансформаций типологии второго порядка (на рис.~3 стрелки указывают 
направления трансформаций согласно средов$\acute{\mbox{о}}$й версии на рис.~1).
  
  Таким образом, частный случай классификации содержит для этих двух 
типологий 16~теоретически возможных трансформаций, 6 из которых 
в~настоящее время в~существующих ИТ применяются наиболее часто: виды 
трансформаций~1, 2, 11 и~12 типологии второго порядка реализуются 
с~помощью тех или иных методов ко\-ди\-ро\-ва\-ния/де\-ко\-ди\-ро\-ва\-ния 
(например, с~использованием таблиц Unicode), а~виды трансформаций~13 и~15
 в~типологии первого порядка реализуются полностью с~по\-мощью процессов 
цифровой обработки компьютерами.
  
  Остальные виды трансформаций или применяются намного реже (это 
виды~3, 5, 7, 9 и~10), или находятся в~стадии поиска и~разработки (14 и~16) или 
в~настоящее время носят только теоретический характер, обеспечивая полноту 
первой и~второй типологий (4, 6 и~8). Знаком~<<?>> обозначены те виды 
трансформаций, которые по определению не существуют в~используемой 
парадигме информатики~\cite{30-zac}. Однако возможно, что в~других 
будущих подходах к~построению ее парадигмы эти виды трансформаций будут 
существовать.
  
\section{Заключение}

  На сегодняшний день процесс построения классификаций объектов 
предметной области информатики~\cite{22-zac} и~их  
трансформаций~\cite{23-zac} еще не завершен. Однако первые результаты их 
построения уже используются для создания концепции лексикографической 
информационной сис\-те\-мы, обеспечивающей интеграцию двуязычных 
словарей и~параллельных корпусов.
  
  \bigskip
  
  
  Автор признателен рецензентам за помощь в~улучшении статьи.
  
{\small\frenchspacing
 { %\baselineskip=10.6pt
 %\addcontentsline{toc}{section}{References}
 \begin{thebibliography}{99}
\bibitem{1-zac}
\Au{Aijmer K., Altenberg~B.} Advances in corpus-based contrastive linguistics. Studies in honour 
of Stig Johansson.~--- Amsterdam: John Benjamins, 2013. 295~p.  doi: 10.1075/scl.54.
\bibitem{2-zac}
\Au{Добровольский Д.\,О., Кретов~А.\, А., Шаров~С.\,А.} Корпус параллельных текстов~// 
Научная и~техническая информация. Сер.~2: Информационные процессы и~сис\-те\-мы, 2005. 
№\,6. С.~16--27.
\bibitem{3-zac}
\Au{Добровольский Д.\,О.} Корпус параллельных текстов и~сопоставительная 
лексикология~// Труды Института русского языка им.\ В.\,В.~Виноградова, 2015. №\,6. 
С.~413--449. EDN: VJQBHP.
\bibitem{4-zac}
\Au{Гончаров А.\,А., Зацман~И.\,М., Кружков~М.\,Г.} Эволюция классификаций 
в~надкорпусных базах данных~// Информатика и~её применения, 2020. Т.~14. Вып.~4. 
С.~108--116. doi: 10.14357/19922264200415.  
EDN: \mbox{GKWBZT}.
\bibitem{5-zac}
\Au{Гончаров А.\, А., Зацман И. \,М., Кружков~М.\, Г}. Представление новых 
лексикографических знаний в~динамических классификационных сис\-те\-мах~// 
Информатика и~её применения, 2021. Т.~15. Вып.~1. С.~86--93.  doi: 10.14357/19922264210112. EDN: OPEFXW.
\bibitem{6-zac}
\Au{Zatsman I.} Finding and filling lacunas in linguistic typologies~// 15th Forum (International) 
on Knowledge Asset Dynamics Proceedings.~--- Matera, Italy: Institute of Knowledge Asset 
Management, 2020. P.~780--793.
\bibitem{7-zac}
\Au{Zatsman I.} Three-dimensional encoding of emerging meanings in AI-systems~// 21st 
European Conference on Knowledge Management Proceedings.~--- Reading, U.K.: Academic 
Publishing International Ltd., 2020. P.~878--887.
\bibitem{8-zac}
\Au{Ackoff R.} From data to wisdom~// J.~Applied Systems Analysis, 1989. Vol.~16. No.\,1. P.~3--9.
\bibitem{9-zac}
\Au{Rosenbloom P.\,S.} On computing: The fourth great scientific domain.~--- Cambridge, MA, 
USA: MIT Press, 2013. 307~p.
\bibitem{10-zac}
\Au{Rowley J.} The wisdom hierarchy: Representations of the DIKW hierarchy~// J.~Inf. 
Sci., 2007. Vol.~33. Iss.~2. P.~163--180. doi: 10.1177/0165551506070706.
\bibitem{11-zac} 
\Au{Frick$\acute{\mbox{e}}$~M.\,H.} Data--Information--Knowledge--Wisdom (DIKW) pyramid, 
framework, continuum~// Encyclopedia of big data~/ Eds. L.~Schintler, C.~McNeely.~--- Cham: 
Springer, 2018. 4~p. doi: 10.1007/978-3-319-32001-4\_331-1.
\bibitem{12-zac}
\Au{Denning P., Rosenbloom~P.} Computing: The fourth great domain of science~// Commun. 
ACM, 2009. Vol.~52. Iss.~9. P.~27--29.
\bibitem{13-zac}
\Au{Denning P., Freeman~P.} Computing's paradigm~// Commun.  ACM, 2009. Vol.~52. 
Iss.~12. P.~28--30. doi: 10.1145/ 1610252.1610265.
\bibitem{17-zac} %14
\Au{Farradane J.} Knowledge, information, and information science~// J.~Inf. Sci., 
1980. Vol.~2. Iss.~2. P.~75--80. doi: 10.1177/01655515800020020.

\bibitem{15-zac}
\Au{Шрейдер Ю.\,А.} Информация и~знание~// Сис\-тем\-ная концепция информационных 
процессов.~--- М.: ВНИИСИ, 1988. С.~47--52.
\bibitem{16-zac}
\Au{Ingwersen P.} Information and information science~// Enclyclopaedie of library and 
information science~/ Eds. J.\,D.~McDonald, 
M.~Levine-Clark.~--- New York, NY, USA: Marcel Dekker Inc., 1992. Vol.~56. Sup.~19. 
P.~137--174.

\bibitem{14-zac} %17
Информатика как наука об информации: Информационный, документальный, 
технологический, экономический, социальный и~организационный аспекты~/ Под ред. 
Р.\,С.~Гиляревского.~--- М.: Фаир-Пресс, 2006. 592~с.

\bibitem{18-zac}
\Au{Hjorland B.} Library and information science: practice, theory, and philosophical basis~// 
Inform. Process. Manag., 2000. Vol.~36. Iss.~3. P.~501--531. doi:  
10.1016/S0306-\mbox{4573(99)00038-2}.
\bibitem{19-zac}
Deep shift~--- technology tipping points and societal impact.~--- Geneva: WE Forum, 2015. 44~p. 
{\sf http://www3.weforum.org/docs/WEF\_GAC15\_ Technological\_Tipping\_Points\_report\_2015.pdf}.
\bibitem{20-zac}
\Au{Berman F., Rutenbar~R., Hailpern~B., Christensen~H., Davidson~S., Estrin~D., 
Franklin~M., Martonosi~M., Raghavan~P., Stodden~V., Szalay~A.\,S.} Realizing the potential of 
data science~// Commun.  ACM, 2018. Vol.~61. Iss.~4. P.~67--72. doi: 10.1145/3188721.

\bibitem{21-zac}
\Au{Stodden V.} The data science life cycle: A~disciplined approach to advancing data science as 
a~science~// Commun.  ACM, 2020. Vol.~63. Iss.~7. P.~58--66. doi: 10.1145/ 3360646.


\bibitem{23-zac} %22
\Au{Зацман И.\,М.} Научная парадигма информатики: классификация трансформаций 
объектов предметной об\-ласти~// Системы и~средства информатики, 2023. Т.~33. №\,4. 
С.~126--138. doi: 10.14357/08696527230412. EDN: ZIKUWO.

\bibitem{22-zac} %23
\Au{Зацман И.\,М.} Научная парадигма информатики: классификация объектов предметной  
об\-ласти~// Информатика и~её применения, 2023. Т.~17. Вып.~4. С.~96--103. doi: 
10.14357/19922264230413. EDN: FIUQAT.

\bibitem{24-zac}
\Au{Зацман И.\,М.} О~научной парадигме информатики: верхний уровень классификации 
объектов ее предметной об\-ласти~// Информатика и~её применения, 2022. Т.~16. Вып.~4. 
С.~73--79. doi: 10.14357/ 19922264220411. EDN: XZNKVI.

\bibitem{25-zac}
\Au{Соломоник А.\,Б.} Философия знаковых систем и~язык.~--- М.: ЛКИ, 2011. 408~с.
\bibitem{26-zac}
\Au{Зацман И.\,М.} Трансформация иерархии Акоффа в~научной парадигме информатики~// 
Информатика и~её применения, 2023. Т.~17. Вып.~3. С.~107--113. doi: 
10.14357/19922264230315. EDN: UMVRRV.

\bibitem{27-zac}
\Au{Zatsman I.} Building digital spiral models of knowledge generation~// 19th Forum 
(International) on Knowledge Asset Dynamics Proceedings.~--- Matera, Italy: Arts for Business 
Institute, 2024. P.~2185--2196.
\bibitem{28-zac}
\Au{Zatsman I.} Digital spiral model of knowledge creation and encoding its dynamics~// 18th 
Forum (International) on Knowledge Asset Dynamics Proceedings.~--- Matera, Italy: Arts for 
Business Institute, 2023. P.~581--596.
\bibitem{29-zac}
\Au{Зацман И.\,М.} Интерфейсы третьего порядка в~информатике~// Информатика и~её 
применения, 2019. Т.~13. Вып.~3. С.~82--89. doi: 10.14357/19922264190312. EDN: 
EHRQLF.

\bibitem{30-zac}
\Au{Зацман И.\,М.} Научная парадигма информатики как третьей культуры~//  
На\-уч\-но-тех\-ни\-че\-ская информация. Сер.~1: Организация и~методика информационной 
работы, 2023. №\,11. С.~1--14.

\end{thebibliography}

 }
 }

\end{multicols}

\vspace*{-9pt}

\hfill{\small\textit{Поступила в~редакцию 14.04.24}}

\vspace*{4pt}

%\pagebreak

%\newpage

%\vspace*{-28pt}

\hrule

\vspace*{2pt}

\hrule



\def\tit{OBJECT TRANSFORMATIONS OF~THE~FIRST AND~SECOND ORDER
IN~A~LEXICOGRAPHIC INFORMATION SYSTEM\\[-5pt]}


\def\titkol{Object transformations of~the~first and~second order
in~a~lexicographic information system}


\def\aut{I.\,M.~Zatsman}

\def\autkol{I.\,M.~Zatsman}

\titel{\tit}{\aut}{\autkol}{\titkol}

\vspace*{-13pt}


\noindent
Federal Research Center ``Computer Science and Control'' of the Russian Academy of Sciences, 
44-2~Vavilov Str., Moscow 119133, Russian Federation


\def\leftfootline{\small{\textbf{\thepage}
\hfill INFORMATIKA I EE PRIMENENIYA~--- INFORMATICS AND
APPLICATIONS\ \ \ 2024\ \ \ volume~18\ \ \ issue\ 2}
}%
 \def\rightfootline{\small{INFORMATIKA I EE PRIMENENIYA~---
INFORMATICS AND APPLICATIONS\ \ \ 2024\ \ \ volume~18\ \ \ issue\ 2
\hfill \textbf{\thepage}}}

\vspace*{2pt}



\Abste{The theoretical foundations of the design of information technologies used for 
the integration of bilingual dictionaries and parallel corpora are considered. The 
description of the first outcomes of the creation of the third\linebreak\vspace*{-12pt}}

\Abstend{ level of object 
transformations classification in the subject domain of informatics, which is supposed 
to be used
in creating the lexicographic information system providing integration, is 
given. All the entities of informatics are divided into two global classes: objects and 
their transformations. For each such class, its own classification is constructed. 
Previously, the two upper levels of the object transformation classification in the subject 
domain have been described. The present paper discusses the third level of this classification. The 
basis for the construction of its highest level was the division of the subject domain of 
informatics into media (mental, sensory, digital, and a~number of other media), each 
of which by definition includes objects of the same nature. The Solomonick's 
typology of sign systems served as the basis for constructing the second level of the 
object transformation classification. The aim of the paper is to systematize object 
transformations of the first and second orders at the third level of this classification. 
The basis for systematization is the medium version of the Ackoff's hierarchy.}

\KWE{subject domain objects; object transformations; classification; data; 
information; knowledge; lexicographic information system}


\DOI{10.14357/19922264240211}{VZTGVV}

\vspace*{-12pt}

\Ack

\vspace*{-3pt}


\noindent
The reported study was funded by the Russian Science Foundation, project  
No.\,24-18-00155, {\sf 
https://rscf.ru/project/24-18-00155}. The research was carried out using the infrastructure of the Shared 
Research Facilities ``High Performance Computing and Big Data'' (CKP 
``Informatics'') of FRC CSC RAS (Moscow) .
   


  \begin{multicols}{2}

\renewcommand{\bibname}{\protect\rmfamily References}
%\renewcommand{\bibname}{\large\protect\rm References}

{\small\frenchspacing
 {%\baselineskip=10.8pt
 \addcontentsline{toc}{section}{References}
 \begin{thebibliography}{99} 
\bibitem{1-zac-1}
\Aue{Aijmer, K., and B.~Altenberg.} 2013. \textit{Advances in corpus-based 
contrastive linguistics. Studies in honour of Stig Johansson}. Amsterdam: John 
Benjamins. 295~p. doi: 10.1075/scl.54.
\bibitem{2-zac-1}
\Aue{Dobrovolskiy, D.\,O., A.\,A.~Kretov, and S.\,A.~Sharov.} 2005. Korpus 
parallel'nykh tekstov [Corpus of parallel texts]. \textit{Nauchnaya i~tekhnicheskaya 
informatsiya. Ser. 2. Informatsionnye protsessy i~sistemy} [Scientific and Technical 
Information. Ser.~2: Information Processes and Systems] 6:16--27.
\bibitem{3-zac-1}
\Aue{Dobrovolskiy, D.\,O.} 2015. Korpus parallel'nykh tekstov i~sopostavitel'naya 
leksikologiya [The corpus of parallel texts and contrastive lexicology]. \textit{Trudy 
Instituta russkogo yazyka im. V.\,V.~Vinogradova} [Proceedings of the 
V.\,V.~Vinogradov Russian Language Institute] 6:413--449. EDN: VJQBHP.
\bibitem{4-zac-1}
\Aue{Goncharov, A.\,A., I.\,M.~Zatsman, and M.\,G.~Kruzhkov.} 2020. Evolyutsiya 
klassifikatsiy v~nadkorpusnykh ba\-zakh dannykh [Evolution of classifications in 
supracorpora databases]. \textit{Informatika i~ee Primeneniya~--- Inform. \mbox{Appl.}}  
14(4):108--116. doi: 10.14357/19922264200415.  
EDN: GKWBZT.
\bibitem{5-zac-1}
\Aue{Goncharov, A.\,A., I.\,M.~Zatsman, and M.\,G.~Kruzhkov.} 2021. 
Predstavlenie novykh leksikograficheskikh znaniy v~dinamicheskikh 
klassifikatsionnykh sistemakh [Representation of new lexicographical knowledge in 
dynamic classification systems]. \textit{Informatika i~ee Primeneniya~--- Inform. 
Appl.} 15(1):86--93. doi: 10.14357/19922264210112. EDN: OPEFXW.
\bibitem{6-zac-1}
\Aue{Zatsman, I.} 2020. Finding and filling lacunas in linguistic typologies. 
\textit{15th Forum (International) on Knowledge Asset Dynamics Proceedings}. 
Matera, Italy: Institute of Knowledge Asset Management. 780--793.
\bibitem{7-zac-1}
\Aue{Zatsman, I.} 2020. Three-dimensional encoding of emerging meanings in  
AI-systems. \textit{21st European Conference on Knowledge Management 
Proceedings}. Reading, U.K.: Academic Publishing International Ltd. 878--887.
\bibitem{8-zac-1}
\Aue{Ackoff, R.} 1989. From data to wisdom. \textit{J.~Applied Systems Analysis} 
16(1):3--9.
\bibitem{9-zac-1}
\Aue{Rosenbloom, P.\,S.} 2013. \textit{On computing: The fourth great scientific 
domain}. Cambridge, MA: MIT Press. 307~p.
\bibitem{10-zac-1}
\Aue{Rowley, J.} 2007. The wisdom hierarchy: Representations of the DIKW 
hierarchy. \textit{J.~Inf. Sci.} 33(2):163--180. doi: 10.1177/0165551506070706.
\bibitem{11-zac-1}
\Aue{Frick$\acute{\mbox{e}}$, M.\,H.} 2018.  
Data-Information-Knowledge-Wisdom (DIKW) pyramid, framework, continuum. 
\textit{Encyclopedia of big data}. Eds. L.~Schintler and C.~McNeely. Cham: 
Springer. 4~p. doi: 10.1007/978-3-319-32001- 4\_331-1.
\bibitem{12-zac-1}
\Aue{Denning, P., and P.~Rosenbloom.} 2009. Computing: The fourth great domain 
of science. \textit{Commun. ACM} 52(9):27--29.
\bibitem{13-zac-1}
\Aue{Denning, P., and P.~Freeman.} 2009. Computing's paradigm. \textit{Commun. 
ACM} 52(12):28--30. doi: 10.1145/ 1610252.1610265.

\bibitem{17-zac-1} %14
\Aue{Farradane, J.} 1980. Knowledge, information, and information science. 
\textit{J.~Inf. Sci.} 2(2):75--80. doi: 10.1177/ 01655515800020020.

\bibitem{15-zac-1}
\Aue{Shreyder, Yu.\,A.} 1988. Informatsiya i~znanie [Information and knowledge]. 
\textit{Sistemnaya kontseptsiya in\-for\-ma\-tsi\-on\-nykh protsessov} [System concept of 
information processes]. Moscow: VNIISI. 47--52.
\bibitem{16-zac-1}
\Aue{Ingwersen, P.} 1995. Information and information science. 
\textit{Encyclopedia of library and information science}. Eds. J.\,D.~McDonald and 
M.~Levine-Clark. New York, NY: Marcel Dekker Inc. 56(19):137--174.

\bibitem{14-zac-1} %17
Gilyarevskiy, R.\,S., ed. 2006. \textit{Informatika kak nauka ob informatsii: 
informatsionnyy, dokumental'nyy, tekh\-no\-lo\-gi\-che\-skiy, ekonomicheskiy, sotsial'nyy 
i~organizatsionnyy aspekty} [Informatics as information science: Informational, 
documentary, technological, economic, social, and organizational dimensions]. 
Moscow: FAIR-PRESS. 592~p.

\bibitem{18-zac-1}
\Aue{Hjorland, B.} 2000. Library and information science: Practice, theory, and 
philosophical basis. \textit{Inform. Process. Manag.} 36(3):501--531. doi:  
10.1016/S0306-\mbox{4573(99)00038-2}.
\bibitem{19-zac-1}
Deep shift~--- technology tipping points and societal impact. 2015. \textit{World Economic 
Forum}. Geneva. 44~p. Available at: {\sf 
http://www3.weforum.org/docs/WEF\_ GAC15\_Technological\_Tipping\_Points\_report\_2015.pdf} (accessed May~20, 
2024).
\bibitem{20-zac-1}
\Aue{Berman, F., R.~Rutenbar, B.~Hailpern, H.~Christensen, S.~Davidson, 
D.~Estrin, M.~Franklin, M.~Martonosi, P.~Raghavan, V.~Stodden, and 
A.\,S.~Szalay.} 2018. Realizing the potential of data science. \textit{Commun. ACM} 
61(4):67--72. doi: 10.1145/3188721.
\bibitem{21-zac-1}
\Aue{Stodden, V.} 2020. The data science life cycle: A~disciplined approach to 
advancing data science as a~science. \textit{Commun. ACM} 
 63(7):58--66. doi: 10.1145/3360646.

\bibitem{23-zac-1} %22
\Aue{Zatsman, I.\,M.} 2023. Nauchnaya paradigma informatiki: klassifikatsiya 
transformatsiy ob''ektov predmetnoy oblasti [Scientific paradigm of informatics: 
Transformation classification of domain objects]. \textit{Sistemy i~Sredstva 
Informatiki~--- Systems and Means of Informatics} 33(4):126--138. doi: 
10.14357/08696527230412. EDN: ZIKUWO.

\bibitem{22-zac-1} %23
\Aue{Zatsman, I.\,M.} 2023. Nauchnaya paradigma informatiki: klassifikatsiya 
ob''ektov predmetnoy oblasti [Scientific paradigm of informatics: Classification of 
domain objects]. \textit{Informatika i~ee Primeneniya~--- Inform. Appl.} 
 17(4):96--103. doi: 10.14357/19922264230413. EDN: FIUQAT.
 
\bibitem{24-zac-1}
\Aue{   Zatsman, I.\,M.} 2022. O nauchnoy paradigme informatiki: verkhniy uroven' 
klassifikatsii ob''ektov ee predmetnoy oblasti [On the scientific paradigm of 
informatics: The classification high level of its objects]. \textit{Informatika i~ee 
Primeneniya~--- Inform. Appl.} 16(4):73--79. doi: 10.14357/19922264220411. EDN: 
XZNKVI.
\bibitem{25-zac-1}
\Aue{Solomonick, A.\,B.} 2011. \textit{Filosofiya znakovykh system i~yazyk} 
[Philosophy of sign systems and language]. Moscow: LKI. 408~p.
\bibitem{26-zac-1}
\Aue{Zatsman, I.\,M.} 2023. Transformatsiya ierarkhii Akoffa v~nauchnoy 
paradigme informatiki [Transformation of the Ackoff's hierarchy in the scientific 
paradigm of informatics]. \textit{Informatika i~ee Primeneniya~--- Inform. \mbox{Appl.}} 
17(3):107--113. doi: 10.14357/19922264230315. EDN: UMVRRV.
\bibitem{27-zac-1}
\Aue{Zatsman, I.} 2024. Building digital spiral models of knowledge 
generation. \textit{19th Forum (International) on Knowledge Asset Dynamics 
Proceedings}. Matera, Italy: Arts for Business Institute. 2185--2196.
\bibitem{28-zac-1}
\Aue{Zatsman, I.} 2023. Digital spiral model of knowledge creation and encoding its 
dynamics. \textit{18th Forum (International) on Knowledge Asset Dynamics 
Proceedings}. Matera, Italy: Arts for Business Institute. 581--596.
\bibitem{29-zac-1}
\Aue{Zatsman, I.\,M.} 2019. Interfeysy tret'ego poryadka v~informatike 
 [Third-order interfaces in informatics]. \textit{Informatika i~ee Primeneniya~--- 
Inform. Appl.} 13(3):82--89. doi: 10.14357/19922264190312. EDN: EHRQLF.
\bibitem{30-zac-1}
\Aue{Zatsman, I.} 2023. Scientific paradigm of informatics as a~third culture. 
\textit{Scientific Technical Information Processing} 50(4):246--258. doi: 
10.3103/S0147688223040111. EDN: CKHMYS.

\end{thebibliography}

 }
 }

\end{multicols}

\vspace*{-6pt}

\hfill{\small\textit{Received April 14, 2024}} 


\vspace*{-12pt}


\Contrl

\vspace*{-3pt}

\noindent
\textbf{Zatsman Igor M.} (b.\ 1952)~--- Doctor of Science in technology, head of 
department, Federal Research Center ``Computer Science and Control'' of the 
Russian Academy of Sciences, 44-2~Vavilov Str., Moscow 119333, Russian 
Federation; \mbox{izatsman@yandex.ru}





\label{end\stat}

\renewcommand{\bibname}{\protect\rm Литература}          %17
\include{krasnov}         %18




%%%%%%%%%%%%%%%%%%%%%%%%%%%%%%%%%%%%%%%%

%\def\stat{rez}
{%\hrule\par
%\vskip 7pt % 7pt
\raggedleft\Large \bf%\baselineskip=3.2ex
Р\,Е\,Ц\,Е\,Н\,З\,И\,И \vskip 17pt
    \hrule
    \par
\vskip 6pt plus 6pt minus 3pt }

%\thispagestyle{headings} %с верхним колонтитулом
%\thispagestyle{myheadings} %с нижним колонтитулом, но в верхнем РЕЦЕНЗИИ

\def\tit{НОВАЯ КНИГА И.\,Н.~СИНИЦЫНА, А.\,С.~ШАЛАМОВА <<ЛЕКЦИИ ПО ТЕОРИИ 
ИНТЕГРИРОВАННОЙ ЛОГИСТИЧЕСКОЙ ПОДДЕРЖКИ>> (М.: ТОРУС ПРЕСС, 2012. 624~с.)}

%1
\def\aut{Д.ф.-м.н., профессор С.\,Я.~Шоргин}

\def\auf{\ }

\def\leftkol{\ % РЕЦЕНЗИИ
}

\def\rightkol{ \ } 

%\def\leftkol{\ } % ENGLISH ABSTRACTS}

%\def\rightkol{\ } %ENGLISH ABSTRACTS}

%\def\leftkol{РЕЦЕНЗИИ}

%\def\rightkol{РЕЦЕНЗИИ}

\titele{\tit}{\aut}{\auf}{\leftkol}{\rightkol}
\vspace*{-18pt}


     \label{st\stat}

     \begin{multicols}{2}
     {\small
     {\baselineskip=10.1pt
     

      В книге представлено системное изложение теоретических основ одного из новейших 
направлений в \mbox{об\-ласти} экономики послепродажного обслуживания изделий наукоемкой 
продукции (ИНП) длительного пользования~--- интегрированной логистической поддержки
(ИЛП). 
{\looseness=1

}

Приведены также результаты новых работ, выполненных в Институте проблем информатики 
Российской академии наук в рамках научного направления <<Информационные технологии и 
анализ сложных сис\-тем>>.
 {%\looseness=1

}
     
      Излагаемые в книге научные подходы позво\-ляют карди\-наль\-но реформировать 
существующие системы производства и эксплуатации ИНП путем создания и внед\-ре\-ния 
методов рационального и оптимального управ\-ле\-ния процессами расходования 
вре\-мен\-н$\acute{\mbox{ы}}$х, 
мате\-ри\-аль\-ных, трудовых и других ресурсов на всех стадиях жизненного цикла изделий (ЖЦИ) по 
критериям экономической целесообразности и эф\-фек\-тив\-ности.
  {\looseness=1

}
    
      В книге приведен краткий обзор причин возник\-новения и
      развития CALS-методологии как основы 
современных международных стандартов по созданию и функционированию глобальных 
ин\-фор\-ма\-ци\-он\-но-ком\-му\-ни\-ка\-ци\-он\-ных систем, ее ключевых возможностей и эффективности 
результатов ее использования. 
Авторы %\linebreak 
предлагают ряд научных обоснований для разработки 
единой теории проектирования и управления систем ИЛП для полноценного использования 
преимуществ %\linebreak
 суще\-ст\-ву\-ющей методологии, определяют \mbox{общую} структурную схему 
комплексной системы <<ИНП-СППО>> и необходимость разработки для ее описания 
гибридных стохастических моделей.
{%\looseness=1

}

%\columnbreak
      
      Книга состоит из пяти частей, где последовательно излагается материал по каждой из 
следующих тем: <<Интегрированная логистическая поддержка>>, <<Теория гибридных 
стохастических систем и компьютерная поддержка исследований и разработок>>, <<Основы 
математического моделирования, анализа и синтеза систем послепродажного обслуживания>>, 
<<Определение и анализ показателей экспортного потенциала ИНП при проектировании>>, 
<<Задачи управления поддержкой послепродажного обслуживания>>, а также 
<<Моделирование инвестиционных процессов ИЛП в условиях неравновесных финансовых 
рынков>>. 
   
      В конце каждой главы приведены выводы и даны вопросы и задания для 
самоконтроля. В~приложениях содержатся основные определения по программам работ по 
анализу ИЛП, логистическим базам данных и компьютерным решениям, эквивалентной статистической 
линеаризации нелинейных преобразований ИЛП, справочный материал, а также развернутые 
уравнения для вероятностных характеристик.


      \def\leftkol{РЕЦЕНЗИИ}

\def\rightkol{РЕЦЕНЗИИ} 

      
      Книга заинтересует широкий круг специалистов и может быть использована научными 
проектными организациями в сфере промышленного производства ИНП. Большое количество 
иллюстраций, примеров и вопросов, обращенных к читателю, позволяет использовать книгу 
также в качестве учебного пособия для студентов и аспирантов машиностроительных, 
транспортных и~других специальностей, а также для самостоятельного изучения. 
{%\looseness=-1

}

Книга 
представляет несомненный интерес для специалистов и студентов в области прикладной 
математики и информатики.
    

}

}
\end{multicols}

%\newpage

\def\stat{authorsrus}
{%\hrule\par
%\vskip 7pt % 7pt
\raggedleft\Large \bf%\baselineskip=3.2ex
О\,Б\ \ А\,В\,Т\,О\,Р\,А\,Х \vskip 17pt
    \hrule
    \par
\vskip 21pt plus 8pt minus 4pt }


\def\tit{\ }

\def\aut{\ }

\def\auf{\ }

\def\leftkol{\ } % ENGLISH ABSTRACTS}

\def\rightkol{ОБ АВТОРАХ} %ENGLISH ABSTRACTS}

\titele{\tit}{\aut}{\auf}{\leftkol}{\rightkol}
      
            \label{st\stat}



\vspace*{24pt}

\begin{multicols}{2}




\noindent
\textbf{Архипов Олег Петрович} (р.\ 1948)~---
кандидат технических наук, директор Орловского филиала Института проб\-лем информатики
Российской академии наук
%302025, г.Орел, Московское шоссе, д.137

\vspace*{3pt}

\noindent
\textbf{Бирюкова Татьяна Константиновна} (р.\ 1968)~---
кандидат фи\-зи\-ко-ма\-те\-ма\-ти\-че\-ских наук, старший научный сотрудник Института проб\-лем информатики
Российской академии наук

\vspace*{3pt}

\noindent 
\textbf{Бобков  Сергей Геннадьевич} (р.\ 1955)~---
доктор технических наук,  заведующий отделением На\-уч\-но-ис\-сле\-до\-ва\-тель\-ско\-го 
института системных исследований Российской академии наук
%117218, Москва, Нахимовский просп., 36, к.1 

\vspace*{3pt}

\noindent \textbf{Васильев Николай Семенович} (р.\ 1952)~--- доктор 
фи\-зи\-ко-ма\-те\-ма\-ти\-че\-ских наук, профессор, 
МГТУ им.\ Н.\,Э.~Баумана 
%, Москва 105005, 2-я Бауманская ул., д.~5,

\vspace*{3pt}

\noindent
\textbf{Гершкович Максим Михайлович} (р.\ 1968)~---
старший научный сотрудник Института проб\-лем информатики
Российской академии наук

\vspace*{3pt}

\noindent 
\textbf{Дьяченко Юрий Георгиевич} (р.\ 1958)~--- кандидат технических наук, 
старший научный сотрудник Института проб\-лем информатики
Российской академии наук

\vspace*{3pt}

\noindent 
\textbf{Ерошенко Александр Андреевич} (р.\ 1989)~--- аспирант кафедры 
математической статистики факультета вычисли\-тельной математики и кибернетики 
Московского государственного университета им.\ М.\,В.~Ломоносова
%119991, Москва ГСП-1, Ленинские горы, д.\ 1, стр. 52

\vspace*{3pt}
 
\noindent 
\textbf{Захаров Виктор Николаевич} (р.\ 1948)~--- 
доктор технических наук, доцент, ученый секретарь Института проб\-лем информатики
Российской академии наук

\vspace*{3pt}

\noindent
\textbf{Зейфман Александр Израилевич} (р.\ 1954)~---
доктор фи\-зи\-ко-ма\-те\-ма\-ти\-че\-ских наук, профессор, 
заведующий кафедрой Вологодского государственного университета; 
старший научный сотрудник Института проб\-лем информатики
Российской академии наук; главный научный сотрудник ИСЭРТ Российской академии наук

\vspace*{3pt}

\noindent
\textbf{Зыкин Сергей Владимирович} (р.\ 1959)~--- 
доктор технических наук, профессор, заведующий лабораторией Института математики 
им.\ С.\,Л.~Соболева Сибирского отделения Российской академии наук, Новосибирск 
%630090, пр.\ ак.\ Коптюга, 4 

\vspace*{4pt}

\noindent
\textbf{Киреев Владимир Иванович} (р.\ 1938)~---
доктор фи\-зи\-ко-ма\-те\-ма\-ти\-че\-ских наук, профессор Московского 
государственного горного университета
%Адрес: Россия, 119991, г. Москва, Ленинский проспект, д. 6

%\columnbreak

\vspace*{4pt}

\noindent
\textbf{Козеренко Елена Борисовна} (р.\ 1959)~---
кандидат филологических наук, заведующая лабораторией Института проб\-лем информатики
Российской академии наук

\vspace*{4pt}

\noindent
\textbf{Королев Виктор Юрьевич} (р.\ 1954)~--- доктор
фи\-зи\-ко-ма\-те\-ма\-ти\-че\-ских наук, профессор кафедры математической 
статистики факультета вычисли\-тельной математики и кибернетики 
Московского государственного университета; 
ведущий научный сотрудник Института проб\-лем информатики
Российской академии наук

\vspace*{4pt}

\noindent
\textbf{Коротышева Анна Владимировна} (р.\ 1988)~---
старший преподаватель Вологодского государственного университета

\vspace*{4pt}

\noindent 
\textbf{Кун Де Турк} (р.\ 1981)~--- научный сотрудник 
исследовательской группы SMACS факультета телекоммуникаций и обработки информации
Университета Гента, Бельгия
%В-9000 Гент, Бельгия

\vspace*{4pt}

\noindent
\textbf{Лупенцов Олег Сергеевич} (р.\ 1986)~---
аспирант Омского государственного института сервиса
%Омск 644043, ул.\ Певцова 13

\vspace*{4pt}

\noindent
\textbf{Лучко Олег Николаевич} (р.\ 1961)~---
кандидат педагогических наук, профессор, заведующий кафедрой 
Омского государственного института сервиса
%Омск 644043, ул.\ Певцова 13

\vspace*{4pt}

\noindent
\textbf{Малашенко Юрий Евгеньевич} (р.\ 1946)~---
доктор фи\-зи\-ко-ма\-те\-ма\-ти\-че\-ских наук, заведующий сектором 
Вычислительного центра им.\ А.\,А.~Дородницына Российской академии наук
%Адрес: 119333, Москва, ул. Вавилова, 40,

\vspace*{4pt}

\noindent
\textbf{Маньяков Юрий Анатольевич} (р.\ 1984)~---
кандидат технических наук, научный сотрудник Орловского филиала Института проб\-лем информатики
Российской академии наук
%302025, г.Орел, Московское шоссе, д.137

\vspace*{4pt}

\noindent
\textbf{Маренко Валентина Афанасьевна} (р.\ 1951)~---
кандидат технических наук, доцент, старший научный сотрудник 
Института математики им.\ С.\,Л.~Соболева Сибирского отделения Российской академии наук
%Новосибирск 630090, пр. ак. Коптюга, 4 

\vspace*{3pt}

\noindent 
\textbf{Морозов Евсей Викторович} (р.\ 1947)~--- доктор 
фи\-зи\-ко-ма\-те\-ма\-ти\-че\-ских, профессор, ведущий научный сотрудник 
Института прикладных математических исследований Карельского научного центра Российской
академии наук; 
%%185910 Россия, Республика Карелия, г.\ Петрозаводск, ул.\ Пушкинская, 11
профессор Петрозаводского государственного университета, Петрозаводск
%185910 Россия, Республика Карелия, г.\ Петрозаводск, пр.\ Ленина, 33

%\pagebreak

\vspace*{3pt}

\noindent
\textbf{Назарова Ирина Александровна} (р.\ 1966)~---
кандидат фи\-зи\-ко-ма\-те\-ма\-ти\-че\-ских наук, 
научный сотрудник Вычислительного центра им.\ А.\,А.~Дородницына Российской академии наук 
%Адрес: 119333, Москва, ул. Вавилова, 40

\vspace*{3pt}

\noindent
\textbf{Павлов Игорь Валерианович} (р.\ 1945)~--- 
доктор фи\-зи\-ко-ма\-те\-ма\-ти\-че\-ских наук, профессор МГТУ им.\ Н.\,Э.~Баумана 
%Москва 105005, 2-я Бауманская ул., д.~5 

%\pagebreak

\vspace*{3pt}

\noindent 
\textbf{Потахина Любовь Викторовна} (р.\ 1989)~--- аспирантка
Института прикладных математических исследований Карельского научного центра
Российской академии наук; 
%%185910 Россия, Республика Карелия, г.\ Петрозаводск, ул.\ Пушкинская, 11
инженер Петрозаводского государственного университета, Петрозаводск
%185910 Россия, Республика Карелия, г.\ Петрозаводск, пр.\ Ленина, 33

\vspace*{3pt}

\noindent 
\textbf{Рождественский Юрий Владимирович} (р.\ 1952)~--- 
кандидат технических наук, заведующий сектором Института проб\-лем информатики
Российской академии наук

\vspace*{3pt}

\noindent 
\textbf{Синицын Игорь Николаевич} (р.\ 1940)~--- доктор технических наук,
профессор, заслуженный деятель\linebreak\vspace*{-12pt}

\columnbreak

\noindent
 науки РФ, заведующий отделом Института проб\-лем информатики
Российской академии наук

\vspace*{7pt}


\noindent
\textbf{Сиротинин Денис Олегович} (р.\ 1984)~---
кандидат технических наук, научный сотрудник Орловского филиала Института проб\-лем информатики
Российской академии наук
%302025, г.Орел, Московское шоссе, д.137

\vspace*{7pt}

%\columnbreak

\noindent 
\textbf{Соколов  Игорь Анатольевич} (р.\ 1954)~--- академик (действительный член) Российской 
академии наук, доктор технических наук, директор Института проб\-лем информатики
Российской академии наук

\vspace*{7pt}

\noindent
\textbf{Степченков Юрий Афанасьевич} (р.\ 1951)~---
кандидат технических наук, заведующий отделом Института проб\-лем информатики
Российской академии наук

\vspace*{7pt}

\noindent
\textbf{Сурков Алексей Викторович} (р.\ 1978)~--- 
старший научный сотрудник На\-уч\-но-ис\-сле\-до\-ва\-тель\-ско\-го 
института системных исследований Российской академии наук
%117218, Москва, Нахимовский просп., 36, к.1 

\vspace*{7pt}

\noindent 
\textbf{Шестаков Олег Владимирович} (р.\ 1976)~--- доктор 
фи\-зи\-ко-ма\-те\-ма\-ти\-че\-ских, доцент кафедры математической статистики 
факультета вычисли\-тельной математики и кибернетики Московского 
государственного университета им.\ М.\,В.~Ломоносова; 
%119991, Москва ГСП-1, Ленинские горы, д.\ 1, стр. 52
старший научный сотрудник Института проб\-лем информатики
Российской академии наук
%, Москва 119333, ул. Вавилова, д.~44, корп.~2

\vspace*{7pt}

\noindent 
\textbf{Шоргин Сергей Яковлевич} (р.\ 1952.)~--- доктор
фи\-зи\-ко-ма\-те\-ма\-ти\-че\-ских наук, профессор, заместитель директора Института 
проб\-лем информатики Российской академии наук





%%%%%%%%%%%%%%%%%%%%%%%%%%%%%%%%%%%%%%%%%%%%%%%%%%%%%%%%%%%%%%%%%%%%%%%%%%%%%%%




%\def\rightkol{ОБ АВТОРАХ}
%\def\leftkol{ОБ АВТОРАХ}

 \label{end\stat}





%\def\leftfootline{\small{\textbf{\thepage}
%\hfill ИНФОРМАТИКА И ЕЁ ПРИМЕНЕНИЯ\ \ \ том~7\ \ \ выпуск~1\ \ \ 2013}
%}%
% \def\rightfootline{\small{ИНФОРМАТИКА И ЕЁ ПРИМЕНЕНИЯ\ \ \ том~7\ \ \ выпуск~1\ \ \ 2013
%\hfill \textbf{\thepage}}}


%\thispagestyle{myheadings}



\end{multicols}

\newpage  

%\def\stat{cont}
{%\hrule\par
%\vskip 7pt % 7pt
\raggedleft\Large \bf%\baselineskip=3.2ex
А\,В\,Т\,О\,Р\,С\,К\,И\,Й\ \ У\,К\,А\,З\,А\,Т\,Е\,Л\,Ь\ \ З\,А\ \ 2\,0\,0\,7 г. \vskip 17pt
    \hrule
    \par
\vskip 21pt plus 6pt minus 3pt }

\label{st\stat}

\def\tit{\ }

\def\aut{\ }
\def\auf{\ }

\def\leftkol{\ } % ENGLISH ABSTRACTS}

\def\rightkol{\ } %ENGLISH ABSTRACTS}

\titele{\tit}{\aut}{\auf}{\leftkol}{\rightkol}


\contentsline {chapter}{\ }{Выпуск \quad Стр.} 
\contentsline {section}{\textbf{Батракова Д.\,А., Королев В.\,Ю., Шоргин С.\,Я.}\ \ Новый метод вероятностно-ста\-ти\-сти\-че\-ско\-го анализа информационных потоков в\nobreakspace {}телекоммуникационных сетях}{\qquad 1 \qquad 40} 
\contentsline {section}{\textbf{Борисов А.\,В.}\ \ Байесовское оценивание в системах наблюдения с\nobreakspace {}марковскими скачкообразными процессами: игровой подход}{\qquad 2 \qquad 65}
\contentsline {section}{\textbf{Босов А.\,В., Иванов А.\,В.}\ \ Программная инфраструктура информационного Web-пор\-тала}{\qquad 2 \qquad 50}
\contentsline {section}{\textbf{Захаров В.\,Н., Калиниченко Л.\,А., Соколов И.\,А., Ступников С.\,А.}\ \ Конструирование канонических информационных моделей для интегрированных информационных систем}{\qquad 2 \qquad 15}
\contentsline {section}{\textbf{Захаров В.\,Н., Козмидиади В.\,А.}\ \ Средства обеспечения отказоустойчивости при\-ло\-жений}{\qquad 1 \qquad 14} 
\contentsline {section}{\textbf{Иванов А.\,В.}\ \ см. Босов А.\,В.\hfill\hfill\hfill\hfill\hfill\hfill\hfill\hfill\hfill\hfill\hfill\hfill\hfill\hfill\hfill\hfill\hfill\hfill\hfill\hfill\hfill\hfill\hfill\hfill\hfill\hfill\hfill\hfill\hfill\hfill\hfill\hfill\hfill\hfill\hfill}{\ }
\contentsline {section}{\textbf{Ильин В.\,Д., Соколов И.\,А.}\ \ Символьная модель системы знаний информатики в\nobreakspace {}че\-ло\-ве\-ко-автоматной среде}{\qquad 1 \qquad 66} 
\contentsline {section}{\textbf{Калиниченко Л.\,А.}\ \ см. Захаров В.\,Н.\hfill\hfill\hfill\hfill\hfill\hfill\hfill\hfill\hfill\hfill\hfill\hfill\hfill\hfill\hfill\hfill\hfill\hfill\hfill\hfill\hfill\hfill\hfill\hfill\hfill\hfill\hfill\hfill\hfill\hfill\hfill\hfill\hfill\hfill\hfill}{\ }
\contentsline {section}{\textbf{Козеренко Е.\,Б.}\ \ Лингвистическое моделирование для систем машинного перевода и обработки знаний}{\qquad 1 \qquad 54} 
\contentsline {section}{\textbf{Козмидиади В.\,А.}\ \ см. Захаров В.\,Н.\hfill\hfill\hfill\hfill\hfill\hfill\hfill\hfill\hfill\hfill\hfill\hfill\hfill\hfill\hfill\hfill\hfill\hfill\hfill\hfill\hfill\hfill\hfill\hfill\hfill\hfill\hfill\hfill\hfill\hfill\hfill\hfill\hfill\hfill\hfill }{\ } 
\contentsline {section}{\textbf{Королев В.\,Ю.}\ \ см. Батракова Д.\,А.\hfill\hfill\hfill\hfill\hfill\hfill\hfill\hfill\hfill\hfill\hfill\hfill\hfill\hfill\hfill\hfill\hfill\hfill\hfill\hfill\hfill\hfill\hfill\hfill\hfill\hfill\hfill\hfill\hfill\hfill\hfill\hfill\hfill\hfill\hfill}{\ } 
\contentsline {section}{\textbf{Кудрявцев А.\,А., Шоргин С.\,Я.}\ \ Байесовский подход к\nobreakspace {}анализу систем массового обслуживания и\nobreakspace {}показателей надежности}{\qquad 2 \qquad 76}
\contentsline {section}{\textbf{Печинкин А.\,В., Соколов И.\,А., Чаплыгин В.\,В.}\ \ Многолинейная система массового обслуживания с конечным накопителем и ненадежными приборами}{\qquad 1 \qquad 27} 
\contentsline {section}{\textbf{Печинкин А.\,В., Соколов И.\,А., Чаплыгин В.\,В.}\ \ Стационарные характеристики многолинейной\nobreakspace {}системы массового обслуживания с\nobreakspace {}одновременными отказами приборов}{\qquad 2 \qquad 39}
\contentsline {section}{\textbf{Синицын И.\,Н.}\ \ Корреляционные методы построения аналитических информационных моделей флуктуаций полюса Земли по априорным данным}{\qquad 2 \qquad \hphantom{9}2}
\contentsline {section}{\textbf{Синицын И.\,Н.}\ \ Развитие теории фильтров Пугачева для оперативной обработки информации в стохастических системах}{{\qquad 1 \qquad \hphantom{9}3}} 
\contentsline {section}{\textbf{Соколов И.\,А.}\ \ см. Захаров В.\,Н.\hfill\hfill\hfill\hfill\hfill\hfill\hfill\hfill\hfill\hfill\hfill\hfill\hfill\hfill\hfill\hfill\hfill\hfill\hfill\hfill\hfill\hfill\hfill\hfill\hfill\hfill\hfill\hfill\hfill\hfill\hfill\hfill\hfill\hfill\hfill}{\ }
\contentsline {section}{\textbf{Соколов И.\,А.}\ \ см. Ильин В.\,Д.\hfill\hfill\hfill\hfill\hfill\hfill\hfill\hfill\hfill\hfill\hfill\hfill\hfill\hfill\hfill\hfill\hfill\hfill\hfill\hfill\hfill\hfill\hfill\hfill\hfill\hfill\hfill\hfill\hfill\hfill\hfill\hfill\hfill\hfill\hfill}{\ } 
\contentsline {section}{\textbf{Соколов И.\,А.}\ \ см. Печинкин А.\,В.\hfill\hfill\hfill\hfill\hfill\hfill\hfill\hfill\hfill\hfill\hfill\hfill\hfill\hfill\hfill\hfill\hfill\hfill\hfill\hfill\hfill\hfill\hfill\hfill\hfill\hfill\hfill\hfill\hfill\hfill\hfill\hfill\hfill\hfill\hfill}{\ } 
\contentsline {section}{\textbf{Соколов И.\,А.}\ \ см. Печинкин А.\,В.\hfill\hfill\hfill\hfill\hfill\hfill\hfill\hfill\hfill\hfill\hfill\hfill\hfill\hfill\hfill\hfill\hfill\hfill\hfill\hfill\hfill\hfill\hfill\hfill\hfill\hfill\hfill\hfill\hfill\hfill\hfill\hfill\hfill\hfill\hfill}{\ }
\contentsline {section}{\textbf{Ступников С.\,А.}\ \ см. Захаров В.\,Н.\hfill\hfill\hfill\hfill\hfill\hfill\hfill\hfill\hfill\hfill\hfill\hfill\hfill\hfill\hfill\hfill\hfill\hfill\hfill\hfill\hfill\hfill\hfill\hfill\hfill\hfill\hfill\hfill\hfill\hfill\hfill\hfill\hfill\hfill\hfill}{\ }
\contentsline {section}{\textbf{Чаплыгин В.\,В.}\ \ см. Печинкин А.\,В.\hfill\hfill\hfill\hfill\hfill\hfill\hfill\hfill\hfill\hfill\hfill\hfill\hfill\hfill\hfill\hfill\hfill\hfill\hfill\hfill\hfill\hfill\hfill\hfill\hfill\hfill\hfill\hfill\hfill\hfill\hfill\hfill\hfill\hfill\hfill}{\ } 
\contentsline {section}{\textbf{Чаплыгин В.\,В.}\ \ см. Печинкин А.\,В.\hfill\hfill\hfill\hfill\hfill\hfill\hfill\hfill\hfill\hfill\hfill\hfill\hfill\hfill\hfill\hfill\hfill\hfill\hfill\hfill\hfill\hfill\hfill\hfill\hfill\hfill\hfill\hfill\hfill\hfill\hfill\hfill\hfill\hfill\hfill}{\ }
\contentsline {section}{\textbf{Шоргин С.\,Я.}\ \ см. Батракова Д.\,А.\hfill\hfill\hfill\hfill\hfill\hfill\hfill\hfill\hfill\hfill\hfill\hfill\hfill\hfill\hfill\hfill\hfill\hfill\hfill\hfill\hfill\hfill\hfill\hfill\hfill\hfill\hfill\hfill\hfill\hfill\hfill\hfill\hfill\hfill\hfill}{\ } 
\contentsline {section}{\textbf{Шоргин С.\,Я.}\ \ см. Кудрявцев А.\,А.\hfill\hfill\hfill\hfill\hfill\hfill\hfill\hfill\hfill\hfill\hfill\hfill\hfill\hfill\hfill\hfill\hfill\hfill\hfill\hfill\hfill\hfill\hfill\hfill\hfill\hfill\hfill\hfill\hfill\hfill\hfill\hfill\hfill\hfill\hfill}{\ }
%\thispagestyle{myheadings}
\def\leftfootline{\small{\textbf{\thepage}
\hfill ИНФОРМАТИКА И ЕЁ ПРИМЕНЕНИЯ\ \ \ том~1\ \ \ выпуск~2\ \ \ 2007}
}%
 \def\rightfootline{\small{ИНФОРМАТИКА И ЕЁ ПРИМЕНЕНИЯ\ \ \ том~1\ \ \ выпуск~2\ \ \ 2007
 \hfill \textbf{\thepage}}}
 \label{end\stat} 
                     
%\def\stat{cont-e}
{%\hrule\par
%\vskip 7pt % 7pt
\raggedleft\Large \bf%\baselineskip=3.2ex
2\,0\,0\,7\ \ A\,U\,T\,H\,O\,R\ \ I\,N\,D\,E\,X \vskip 17pt
    \hrule
    \par
\vskip 21pt plus 6pt minus 3pt }

\label{st\stat}

\def\tit{\ }

\def\aut{\ }
\def\auf{\ }

\def\leftkol{\ } % ENGLISH ABSTRACTS}

\def\rightkol{\ } %ENGLISH ABSTRACTS}

\titele{\tit}{\aut}{\auf}{\leftkol}{\rightkol}


\contentsline {chapter}{\ }{Issue \quad Page} 
\contentsline {subsection}{\textbf{Batrakova D.\,A., Korolev V.\,Yu., Shorgin S.\,Ya.}\ \ A New Method for the Probabilistic and Statistical Analysis of Information Flows in Telecommunication Networks}{\qquad 1 \qquad 40} 
\contentsline {subsection}{\textbf{Borisov A.\,V.}\ \ Bayesian Estimation in\nobreakspace {}Observation Systems with\nobreakspace {}Markov Jump Processes: Game-Theoretic Approach}{\qquad 2 \qquad 65} 
\contentsline {subsection}{\textbf{Bosov A.\,V., Ivanov A.\,V.}\ \ Linguistic Simulation for Machine Translation and Knowledge Management Systems}{\qquad 2 \qquad 50} 
\contentsline {subsection}{\textbf{Chaplygin V.\,V.} see Pechinkin A.\,V.\hfill\hfill\hfill\hfill\hfill\hfill\hfill\hfill\hfill\hfill\hfill\hfill\hfill\hfill\hfill\hfill\hfill\hfill\hfill\hfill\hfill\hfill\hfill\hfill\hfill\hfill\hfill\hfill\hfill\hfill\hfill\hfill\hfill\hfill\hfill}{\ }
\contentsline {subsection}{\textbf{Chaplygin V.\,V.} see Pechinkin A.\,V.\hfill\hfill\hfill\hfill\hfill\hfill\hfill\hfill\hfill\hfill\hfill\hfill\hfill\hfill\hfill\hfill\hfill\hfill\hfill\hfill\hfill\hfill\hfill\hfill\hfill\hfill\hfill\hfill\hfill\hfill\hfill\hfill\hfill\hfill\hfill}{\ }
\contentsline {subsection}{\textbf{Ilyin V.\,D., Sokolov I.\,A.}\ \ The Symbol Model of Informatics Knowledge System in Human-Automaton Environment}{\qquad 1 \qquad 66} 
\contentsline {subsection}{\textbf{Ivanov A.\,V.} see Bosov A.\,V.\hfill\hfill\hfill\hfill\hfill\hfill\hfill\hfill\hfill\hfill\hfill\hfill\hfill\hfill\hfill\hfill\hfill\hfill\hfill\hfill\hfill\hfill\hfill\hfill\hfill\hfill\hfill\hfill\hfill\hfill\hfill\hfill\hfill\hfill\hfill}{\ }
\contentsline {subsection}{\textbf{Kalinichenko L.\,A.} see Zakharov V.\,N.\hfill\hfill\hfill\hfill\hfill\hfill\hfill\hfill\hfill\hfill\hfill\hfill\hfill\hfill\hfill\hfill\hfill\hfill\hfill\hfill\hfill\hfill\hfill\hfill\hfill\hfill\hfill\hfill\hfill\hfill\hfill\hfill\hfill\hfill\hfill}{\ }
\contentsline {subsection}{\textbf{Korolev V.\,Yu.} see Batrakova D.\,A.\hfill\hfill\hfill\hfill\hfill\hfill\hfill\hfill\hfill\hfill\hfill\hfill\hfill\hfill\hfill\hfill\hfill\hfill\hfill\hfill\hfill\hfill\hfill\hfill\hfill\hfill\hfill\hfill\hfill\hfill\hfill\hfill\hfill\hfill\hfill}{\ }
\contentsline {subsection}{\textbf{Kozerenko E.\,B.}\ \ Linguistic Simulation for Machine Translation and Knowledge Management Systems}{\qquad 1 \qquad 54} 
\contentsline {subsection}{\textbf{Kozmidiady V.\,A.} see Zakharov V.\,N.\hfill\hfill\hfill\hfill\hfill\hfill\hfill\hfill\hfill\hfill\hfill\hfill\hfill\hfill\hfill\hfill\hfill\hfill\hfill\hfill\hfill\hfill\hfill\hfill\hfill\hfill\hfill\hfill\hfill\hfill\hfill\hfill\hfill\hfill\hfill}{\ }
\contentsline {subsection}{\textbf{Kudryavtsev A.\,A., Shorgin S.\,Ya.}\ \ Bayesian Approach to Queueing Systems and Reliability Characteristics}{\qquad 2 \qquad 76} 
\contentsline {subsection}{\textbf{Pechinkin A.\,V., Sokolov I.\,A., Chaplygin V.\,V.}\ \ Multichannel Queuing System with Finite Buffer and Unreliable Servers}{\qquad 1 \qquad 27} 
\contentsline {subsection}{\textbf{Pechinkin A.\,V., Sokolov I.\,A., Chaplygin V.\,V.}\ \ Stationary Characteristics of a Multichannel Queueing System with\nobreakspace {}Simultaneous Refusals of Servers}{\qquad 2 \qquad 39} 
\contentsline {subsection}{\textbf{Shorgin S.\,Ya.} see Batrakova D.\,A.\hfill\hfill\hfill\hfill\hfill\hfill\hfill\hfill\hfill\hfill\hfill\hfill\hfill\hfill\hfill\hfill\hfill\hfill\hfill\hfill\hfill\hfill\hfill\hfill\hfill\hfill\hfill\hfill\hfill\hfill\hfill\hfill\hfill\hfill\hfill}{\ }
\contentsline {subsection}{\textbf{Shorgin S.\,Ya.} see Kudryavtsev A.\,A.\hfill\hfill\hfill\hfill\hfill\hfill\hfill\hfill\hfill\hfill\hfill\hfill\hfill\hfill\hfill\hfill\hfill\hfill\hfill\hfill\hfill\hfill\hfill\hfill\hfill\hfill\hfill\hfill\hfill\hfill\hfill\hfill\hfill\hfill\hfill}{\ }
\contentsline {subsection}{\textbf{Sinitsyn I.\,N.}\ \ Correlational Methods for Analytical Informational Models of the Earth Pole Fluctuations Design Based on a priori Data}{\qquad 2 \qquad \hphantom{9}2}
\contentsline {subsection}{\textbf{Sinitsyn I.\,N.}\ \ Development of Pugachev Filtering for Stochastic Systems}{\qquad 1 \qquad \hphantom{9}3}
\contentsline {subsection}{\textbf{Sokolov I.\,A.} see Ilyin V.\,D.\hfill\hfill\hfill\hfill\hfill\hfill\hfill\hfill\hfill\hfill\hfill\hfill\hfill\hfill\hfill\hfill\hfill\hfill\hfill\hfill\hfill\hfill\hfill\hfill\hfill\hfill\hfill\hfill\hfill\hfill\hfill\hfill\hfill\hfill\hfill}{\ }
\contentsline {subsection}{\textbf{Sokolov I.\,A.} see Pechinkin A.\,V.\hfill\hfill\hfill\hfill\hfill\hfill\hfill\hfill\hfill\hfill\hfill\hfill\hfill\hfill\hfill\hfill\hfill\hfill\hfill\hfill\hfill\hfill\hfill\hfill\hfill\hfill\hfill\hfill\hfill\hfill\hfill\hfill\hfill\hfill\hfill}{\ }
\contentsline {subsection}{\textbf{Sokolov I.\,A.} see Pechinkin A.\,V.\hfill\hfill\hfill\hfill\hfill\hfill\hfill\hfill\hfill\hfill\hfill\hfill\hfill\hfill\hfill\hfill\hfill\hfill\hfill\hfill\hfill\hfill\hfill\hfill\hfill\hfill\hfill\hfill\hfill\hfill\hfill\hfill\hfill\hfill\hfill}{\ }
\contentsline {subsection}{\textbf{Sokolov I.\,A.} see Zakharov V.\,N.\hfill\hfill\hfill\hfill\hfill\hfill\hfill\hfill\hfill\hfill\hfill\hfill\hfill\hfill\hfill\hfill\hfill\hfill\hfill\hfill\hfill\hfill\hfill\hfill\hfill\hfill\hfill\hfill\hfill\hfill\hfill\hfill\hfill\hfill\hfill}{\ }
\contentsline {subsection}{\textbf{Stupnikov S.\,A.} see Zakharov V.\,N.\hfill\hfill\hfill\hfill\hfill\hfill\hfill\hfill\hfill\hfill\hfill\hfill\hfill\hfill\hfill\hfill\hfill\hfill\hfill\hfill\hfill\hfill\hfill\hfill\hfill\hfill\hfill\hfill\hfill\hfill\hfill\hfill\hfill\hfill\hfill}{\ }
\contentsline {subsection}{\textbf{Zakharov V.\,N., Kalinichenko L.\,A., Sokolov I.\,A., Stupnikov S.\,A.}\ \ Development of Canonical Information Models for Integrated Information Systems}{\qquad 2 \qquad 15} 
\contentsline {subsection}{\textbf{Zakharov V.\,N., Kozmidiady V.\,A.}\ \ Means Providing Applications Fault Tolerance}{\qquad 1 \qquad 14} 
\def\leftfootline{\small{\textbf{\thepage}
\hfill ИНФОРМАТИКА И ЕЁ ПРИМЕНЕНИЯ\ \ \ том~1\ \ \ выпуск~2\ \ \ 2007}
}%
 \def\rightfootline{\small{ИНФОРМАТИКА И ЕЁ ПРИМЕНЕНИЯ\ \ \ том~1\ \ \ выпуск~2\ \ \ 2007
 \hfill \textbf{\thepage}}}
 \label{end\stat} 


%\end{document}

%
\def\stat{rekl}
%\label{preobr}

%\def\tit{АКАДЕМИК ПУГАЧЁВ  ВЛАДИМИР СЕМЁНОВИЧ\\
%25.03.1911--25.03.1998}


%   \vspace*{-48pt}
%   \begin{center}\LARGE
%Академик Пугачёв  Владимир Семёнович\\ (25.03.1911--25.03.1998)
%   \end{center}

   %\vspace*{2.5mm}

   \begin{center}

{\prgsh\LARGE
ЮБИЛЕИ}

\end{center}
%\hrule

\vspace*{6pt}


   \vspace*{8mm}

   \thispagestyle{empty}


%\def\stat{emel}


\section*{К 70-летию заместителя директора ИПИ РАН,\\ члена редколлегии журнала
<<Информатика и её применения>>\\ доктора технических наук В.\,И.~Будзко}

\vspace*{18pt}




          \begin{multicols}{2}

%            \label{st\stat}

\begin{center}
\vspace*{1pt}
\mbox{%
\epsfxsize=78mm
\epsfbox{bud-1.eps}
}
\end{center}

\vspace*{12pt}

      14 августа 2014~г.\ исполнилось 70~лет за\-мес\-ти\-те\-лю директора ИПИ РАН по
научной работе доктору технических наук Владимиру Игоревичу Будзко.

      Владимир Игоревич Будзко родился в г.~Москве. Высшее образование получил на факультете
элект\-рон\-но-вы\-чис\-ли\-тель\-ных устройств в Московском
ин\-же\-нер\-но-фи\-зи\-че\-ском институте
(МИФИ), который он окончил в 1968~г., после чего был на\-прав\-лен для прохождения
службы в одну из войс\-ко\-вых частей, где прошел путь от инженера до первого заместителя
командира войсковой части.

      С приходом В.\,И.~Будзко в ИПИ РАН (2001~г.)\ в институте
сформировалось новое научное на\-прав\-ле\-ние теоретических исследований~--- <<Постро\-ение
ин\-фор\-ма\-ци\-он\-но-те\-ле\-ком\-му\-ни\-ка\-ци\-он\-ных\linebreak сис\-тем
высокой до\-ступ\-ности>>. В~рамках этого
направления выполнен широкий круг фундаментальных исследований по поиску подходов и
определению принципов построения средств обеспечения доступности, конфиденциальности
и целостности современных крупномасштабных
ин\-фор\-ма\-ци\-он\-но-те\-ле\-ком\-му\-ни\-ка\-ци\-он\-ных
сис\-тем (ИТС). Разработаны основные сис\-тем\-но-тех\-ни\-че\-ские принципы и базовые
архитектурные решения построения перспективных для условий России ИТС с
централизованной обработкой и хранением информации, сочетающих в себе свойства
высокой доступности, отказо- и катастрофоустойчивости, информационной защищенности.
Определены принципы, методы и математические основы рационального построения и
оптимизации средств восстановления функционирования центров обработки данных (ЦОД)
после возникновения отказов и катастроф, передачи и хранения данных, обеспечения
информационной безопасности при достижении минимальной совокупной стоимости
владения такими системами. Результаты нашли практическое воплощение при реализации
проектов в интересах ряда отечественных государственных и негосударственных
организаций, таких как Банк России (БР), Внешторгбанк, ОАО <<ГМК <<Норильский Никель>>,
<<Газпром>>, Минэкономразвития России, Правительство Москвы, а также ряд силовых
ведомств.

      Под руководством В.\,И.~Будзко начиная с 2001~г.\ выполнен комплекс
      на\-уч\-но-ис\-сле\-до\-ва\-тель\-ских и
      опыт\-но-кон\-ст\-рук\-тор\-ских работ (свыше 100~проектов),
направленных на развитие электронной информационной технологии БР.
Разработаны концепции развития ИТС БР сначала до 2008~г., а затем до 2013~г., которые
были приняты в качестве основы проведения технической политики. За реализацию проекта
<<Катастрофоустойчивая тер\-ри\-то\-ри\-аль\-но-рас\-пре\-де\-лен\-ная
      ин\-фор\-ма\-ци\-он\-но-те\-ле\-ком\-му\-ни\-ка\-ци\-он\-ная сис\-те\-ма централизованной
обработки банковской информации>> В.\,И.~Будзко удостоен Премии Правительства РФ в
области науки и техники за 2010~г.

      В.\,И.~Будзко возглавлял и возглавляет работы по ряду других прикладных проектов,
связанных с созданием, совершенствованием и развитием крупномасштабных ИТС.

      В.\,И.~Будзко~--- генерал-майор, доктор технических наук, член-кор\-рес\-пон\-дент
Академии криптографии РФ, известный ученый в области информатики и применения
информационных технологий при построении территориально распределенных ИТС
различного назначения. Является автором свыше 250~научных работ, опубликованных в
на\-уч\-но-тех\-ни\-че\-ских и специальных изданиях.

    \thispagestyle{empty}

      В.\,И.~Будзко уделяет большое внимание подготовке научных кадров. Под его
руководством защищено 6~диссертаций на соискание ученой степени кандидата
технических наук. Свыше 30~лет он читает лекции в ИКСИ Академии ФСБ, профессор
кафедры НИЯУ МИФИ. Является членом двух диссертационных советов, главным
редактором журнала <<Системы высокой доступности>> и членом редколлегии журнала
<<Информатика и её применения>>.

      \bigskip

      Редакционный совет и Редакционная коллегия журнала <<Информатика и её
применения>> сердечно поздравляют Владимира Игоревича Будзко с 70-ле\-ти\-ем и желают
крепкого здоровья и новых научных достижений.

\end{multicols}

%%Информатика и её применения
%Том 13 Выпуск 1-4 Год 2019

\def\stat{cont}
{%\hrule\par
%\vskip 7pt % 7pt
\raggedleft\Large \bf%\baselineskip=3.2ex
А\,В\,Т\,О\,Р\,С\,К\,И\,Й\ \ У\,К\,А\,З\,А\,Т\,Е\,Л\,Ь\ \ З\,А\ \ 2\,0\,1\,9 г. \vskip 17pt
 \hrule
 \par
\vskip 21pt plus 6pt minus 3pt }

\label{st\stat}

\def\tit{\ }

\def\aut{\ }
\def\auf{\ }

\def\leftkol{\ } % ENGLISH ABSTRACTS}

\def\rightkol{\ } %АВТОРСКИЙ УКАЗАТЕЛЬ ЗА 2019 г.} %ENGLISH ABSTRACTS}

\titele{\tit}{\aut}{\auf}{\leftkol}{\rightkol}
\addcontentsline{toc}{subsection}{\textrm\textbf Авторский указатель за 2019 г.}

%\vspace*{-12pt}

\noindent
{\tabcolsep=3pt
\begin{tabular}{p{397pt}cc}
&\textbf{Вып.} & \textbf{Стр.}\\[6pt]
\Avtors{Абгарян~К.\,К., Осипова~В.\,А.} Применение методов поддержки принятия решений для\linebreak
\\[-12pt]
\hspace*{23pt}многокритериальной задачи отбора многомасштабных композиций&2&47--53\\
\Avtors{Агаларов~Я.\,М., Коновалов~М.\,Г.} Доказательство унимодальности целевой функции\linebreak
\\[-12pt]
\hspace*{23pt}в~задаче порогового управления нагрузкой на~сервер&2&2--6\\
\Avtors{Агаларов~Я.\,М., Ушаков~В.\,Г.} Об унимодальности функции дохода системы массового\linebreak
\\[-12pt]
\hspace*{23pt}обслуживания типа $G|M|s$ с~управляемой очередью&1&55--61\\
\Avtors{Агасандян~Г.\,А.} Вычисление показателей оптимальных по CC-VaR портфелей на~рынках\linebreak
\\[-12pt]
\hspace*{23pt}опционов&3&72--81\\
\Avtors{Агасандян~Г.\,А.} Теоретические основы оптимизации по континуальному критерию VaR на совокупности рынков&4&36--41\\
\Avtors{Анашин~В.\,С.} О теоретико-автоматных моделях блокчейн-среды&2&29--36\\
\Avtors{Аникеев~Д.\,А., Пенкин~Г.\,О., Стрижов~В.\,В.} Классификация физической активности\linebreak
\\[-12pt]
\hspace*{23pt}человека с~помощью локальных аппроксимирующих моделей&1&40--48\\
\Avtors{Арутюнов~Е.\,Н., Кудрявцев~А.\,А., Титова~А.\,И.} Байесовские модели баланса факторов, \linebreak
\\[-12pt]
\hspace*{23pt}имеющих априорные распределения Вейбулла и~Накагами&2&71--75\\
\Avtors{Бахтеев~О.\,Ю.} см.\ Грабовой~А.\,В.&&\\
\Avtors{Бондаренко~Н.\,Н.} см.\ Журавлев~Ю.\,И.&&\\
\Avtors{Борисов~А.\,В.} Численные схемы фильтрации марковских скачкообразных процессов по\linebreak
\\[-12pt]
\hspace*{23pt}дискретизованным наблюдениям~I: характеристики точности&4&68--75\\
\Avtors{Босов~А.\,В., Миллер~Г.\,Б.} О развитии концепции условно-минимаксной нелинейной\linebreak
\\[-12pt]
\hspace*{23pt}фильтрации: модифицированный фильтр и~его анализ&2&\hphantom{1}7--15\\
\Avtors{Босов~А.\,В., Мхитарян~Г.\,А., Наумов~А.\,В., Сапунова~А.\,П.} Использование модели гамма-распределения в~задаче формирования ограниченного по времени теста в~системе\linebreak
\\[-12pt]
\hspace*{23pt}дистанционного обучения&4&11--17\\
\Avtors{Босов~А.\,В., Стефанович~А.\,И.} Управление выходом стохастической дифференциальной системы по~квадратичному критерию. II.~Численное решение уравнений динами-\linebreak
\\[-12pt]
\hspace*{23pt}ческого программирования&1&\hphantom{1}9--15\\
\Avtors{Босов~А.\,В., Стефанович~А.\,И.} Управление выходом стохастической дифференциальной системы по квадратичному критерию. III.~Анализ свойств оптимального управ-\linebreak
\\[-12pt]
\hspace*{23pt}ления&3&41--49\\
\Avtors{Бурлуцкий~В.\,В., Якимчук~А.\,В., Мельников~А.\,В., Царегородцев~А.\,Л., Волошин~С.\,В.} Разработка метода формирования признакового пространства и~модели для оценки и~прогнозирования антропогенного влияния на окружающую среду (на примере\linebreak
\\[-12pt]
\hspace*{23pt}лесного фонда нефтедобывающего региона)&3&131--136\\
\Avtors{Вахтанов~Н.\,А.} см.\ Шнурков~П.\,В.&&\\
\Avtors{Вахтанов~Н.\,А.} см.\ Шнурков~П.\,В.&&\\
\Avtors{Виноградов~А.\,П.} см.\ Журавлев~Ю.\,И.&&\\
\Avtors{Волошин~С.\,В.} см.\ Бурлуцкий~В.\,В.&&\\
\Avtors{Вышинский~Л.\,Л., Курьянский~М.\,К., Флеров~Ю.\,А.} Цифровая модель весового паспорта\linebreak
\\[-12pt]
\hspace*{23pt}летательного аппарата&4&\hphantom{1}3--10\\
\Avtors{Гайдамака~А.\,А., Чухно~Н.\,В., Чухно~О.\,В., Самуйлов~К.\,Е., Шоргин~С.\,Я.} Формализация метода ранжирования альтернатив для процесса группового принятия решений при\linebreak
\\[-12pt]
\hspace*{23pt}анализе социальных сетей&3&63--71\\
\Avtors{Гайдамака~Ю.\,В.} см.\ Горбунова~А.\,В.&&\\
\end{tabular}
}

\pagebreak

\def\leftkol{АВТОРСКИЙ УКАЗАТЕЛЬ ЗА 2019 г.} % ENGLISH ABSTRACTS}

\def\rightkol{АВТОРСКИЙ УКАЗАТЕЛЬ ЗА 2019 г.} %ENGLISH ABSTRACTS}

%\thispagestyle{myheadings}
\def\leftfootline{\small{\textbf{\thepage}
\hfill ИНФОРМАТИКА И ЕЁ ПРИМЕНЕНИЯ\ \ \ том~13\ \ \ выпуск~4\ \ \ 2019}
}%
 \def\rightfootline{\small{ИНФОРМАТИКА И ЕЁ ПРИМЕНЕНИЯ\ \ \ том~13\ \ \ выпуск~4\ \ \ 2019
 \hfill \textbf{\thepage}}}


\noindent
{\tabcolsep=3pt
\begin{tabular}{p{394pt}cc}
&\textbf{Вып.} & \textbf{Стр.}\\[3pt]
\Avtors{Гольская~А.\,А.} см.\ Маркова~Е.\,В.&&\\
\Avtors{Гончаров~А.\,А., Зацман~И.\,М., Кружков~М.\,Г.} Темпоральные данные в~лексикографиче-\linebreak
\\[-12pt]
\hspace*{23pt}ских базах знаний&4&90--96\\
\Avtors{Гончаров~А.\,А., Инькова~О.\,Ю.} Методика поиска имплицитных логико-семантических\linebreak
\\[-12pt]
\hspace*{23pt}отношений в~тексте&3&\hphantom{1}97--104\\
\Avtors{Горбунова~А.\,В., Наумов~В.\,А., Гайдамака~Ю.\,В., Самуйлов~К.\,Е.} Ресурсные системы\linebreak
\\[-12pt]
\hspace*{23pt}массового обслуживания с~произвольным обслуживанием&1&\hphantom{1}99--107\\
\Avtors{Горшенин~А.\,К., Кузьмин~В.\,Ю.} Оптимизация гиперпараметров нейронных сетей с~ис-\linebreak
\\[-12pt]
\hspace*{23pt}пользованием высокопроизводительных вычислений для~предсказания осадков&1&75--81\\
\Avtors{Горшенин~А.\,К., Кузьмин~В.\,Ю.} Применение рекуррентных нейронных сетей для\linebreak
\\[-12pt]
\hspace*{23pt}прогнозирования моментов конечных нормальных смесей&3&114--121\\
\Avtors{Горшенин~А.\,К., Мартынов~О.\,П.} Гибридные модели экстремального градиентного\linebreak
\\[-12pt]
\hspace*{23pt}бустинга для восстановления пропущенных значений в~данных об~осадках&3&34--40\\
\Avtors{Грабовой~А.\,В., Бахтеев~О.\,Ю., Стрижов~В.\,В.} Определение релевантности параметров\linebreak
\\[-12pt]
\hspace*{23pt}нейросети&2&62--70\\
\Avtors{Гринченко~С.\,Н.} О генезисе информационного общества: информатико-кибернетиче-\linebreak
\\[-12pt]
\hspace*{23pt}ское модельное представление&2&100--108\\
\Avtors{Грушо~А.\,А., Грушо~Н.\,А., Тимонина~Е.\,Е.} Использование метаданных для реализации\linebreak
\\[-12pt]
\hspace*{23pt}требований политики безопасности MLS&4&85--89\\
\Avtors{Грушо~А.\,А., Грушо~Н.\,А., Тимонина~Е.\,Е.} Методы выявления <<слабых>> признаков\linebreak
\\[-12pt]
\hspace*{23pt}нарушений информационной безопасности&3&3--8\\
\Avtors{Грушо~А.\,А., Забежайло~М.\,И., Грушо~Н.\,А., Тимонина~Е.\,Е.} Архитектурные решения в~задаче выявления мошенничества при анализе информационных потоков\linebreak
\\[-12pt]
\hspace*{23pt}в~цифровой экономике&2&22--28\\
\Avtors{Грушо~А.\,А., Забежайло~М.\,И., Грушо~Н.\,А., Тимонина~Е.\,Е.} Формирование концептов\linebreak
\\[-12pt]
\hspace*{23pt}на основе малых выборок&4&81--84\\
\Avtors{Грушо~Н.\,А.} см.\ Грушо~А.\,А.&&\\
\Avtors{Грушо~Н.\,А.} см.\ Грушо~А.\,А.&&\\
\Avtors{Грушо~Н.\,А.} см.\ Грушо~А.\,А.&&\\
\Avtors{Грушо~Н.\,А.} см.\ Грушо~А.\,А.&&\\
\Avtors{Гудкова~И.\,А.} см.\ Маркова~Е.\,В.&&\\
\Avtors{Дзантиев~И.\,Л.} см.\ Маркова~Е.\,В.&&\\
\Avtors{Докукин~А.\,А.} см.\ Журавлев~Ю.\,И.&&\\
\Avtors{Дулин~С.\,К., Дулина~Н.\,Г., Кожунова~О.\,С.} Синтез геоданных в пространственных\linebreak
\\[-12pt]
\hspace*{23pt}инфраструктурах на~основе связанных данных&1&82--90\\
\Avtors{Дулина~Н.\,Г.} см.\ Дулин~С.\,К.&&\\
\Avtors{Дюкова~Е.\,В., Масляков~Г.\,О., Прокофьев~П.\,А.} О числе максимальных независимых\linebreak
\\[-12pt]
\hspace*{23pt}элементов частичных порядков (случай цепей)&1&25--32\\
\Avtors{Журавлев~Ю.\,И., Сенько~О.\,В., Бондаренко~Н.\,Н., Рязанов~В.\,В., Докукин~А.\,А., Виноградов~А.\,П.} Исследование возможности прогнозирования изменения финансового\linebreak
\\[-12pt]
\hspace*{23pt}состояния кредитной организации на основе публикуемой отчетности&4&30--35\\
\Avtors{Забежайло~М.\,И.} см.\ Грушо~А.\,А.&&\\
\Avtors{Забежайло~М.\,И.} см.\ Грушо~А.\,А.&&\\
\Avtors{Захарова~Т.\,В., Тархов~А.\,А.} Оценка уровня значимости критерия Шуирманна для\linebreak
\\[-12pt]
\hspace*{23pt}проверки гипотезы биоэквивалентности при наличии пропущенных данных&3&58--62\\
\Avtors{Зацаринный~А.\,А., Коротков~В.\,В., Матвеев~М.\,Г.} Моделирование процессов сетевого планирования портфеля проектов с~неоднородными ресурсами в~условиях нечет-\linebreak
\\[-12pt]
\hspace*{23pt}кой информации&2&92--99\\
\Avtors{Зацман~И.\,М.} Интерфейсы третьего порядка в~информатике&3&82--89\\
\Avtors{Зацман~И.\,М.} Кодирование концептов в~цифровой среде&4&\hphantom{1}97--106\\
\Avtors{Зацман~И.\,М.} Целенаправленное развитие систем лингвистических знаний: выявление\linebreak
\\[-12pt]
\hspace*{23pt}и~заполнение лакун&1&91--98\\
\Avtors{Зацман~И.\,М.} см.\ Гончаров~А.\,А.&&\\
\end{tabular}
}

\pagebreak

\def\leftkol{АВТОРСКИЙ УКАЗАТЕЛЬ ЗА 2019 г.} % ENGLISH ABSTRACTS}

\def\rightkol{АВТОРСКИЙ УКАЗАТЕЛЬ ЗА 2019 г.} %ENGLISH ABSTRACTS}

%\thispagestyle{myheadings}
\def\leftfootline{\small{\textbf{\thepage}
\hfill ИНФОРМАТИКА И ЕЁ ПРИМЕНЕНИЯ\ \ \ том~13\ \ \ выпуск~4\ \ \ 2019}
}%
 \def\rightfootline{\small{ИНФОРМАТИКА И ЕЁ ПРИМЕНЕНИЯ\ \ \ том~13\ \ \ выпуск~4\ \ \ 2019
 \hfill \textbf{\thepage}}}


\noindent
{\tabcolsep=3pt
\begin{tabular}{p{394pt}cc}
&\textbf{Вып.} & \textbf{Стр.}\\[3pt]
\Avtors{Зейфман~А.\,И., Сатин~Я.\,А., Киселева~К.\,М.} Об оценках скорости сходимости для некоторых моделей массового обслуживания с~неполно заданными интенсивно-\linebreak
\\[-12pt]
\hspace*{23pt}стями&3&14--19\\
\Avtors{Инькова~О.\,Ю., Кружков~М.\,Г.} Сочетаемость логико-семантических отношений: коли-\linebreak
\\[-12pt]
\hspace*{23pt}чественные методы анализа&2&83--91\\
\Avtors{Инькова~О.\,Ю.} см.\ Гончаров~А.\,А.&&\\
\Avtors{Кириков~И.\,А.} см.\ Румовская~С.\,Б.&&\\
\Avtors{Киселева~К.\,М.} см.\ Зейфман~А.\,И.&&\\
\Avtors{Ковалёв~Д.\,Ю., Тарасов~Е.\,А.} Виртуальные эксперименты в~исследованиях с~интенсив-\linebreak
\\[-12pt]
\hspace*{23pt}ным использованием данных&2&117--125\\
\Avtors{Кожунова~О.\,С.} см.\ Дулин~С.\,К.&&\\
\Avtors{Колесников~А.\,В., Листопад~С.\,В.} Протокол гетерогенного мышления гибридной интеллектуальной многоагентной системы для решения проблемы восстановления\linebreak
\\[-12pt]
\hspace*{23pt}распределительной электросети&2&76--82\\
\Avtors{Коновалов~М.\,Г., Разумчик~Р.\,В.} Комплексное управление в~одном классе систем\linebreak
\\[-12pt]
\hspace*{23pt}с~параллельным обслуживанием&4&54--59\\
\Avtors{Коновалов~М.\,Г.} см.\ Агаларов~Я.\,М.&&\\
\Avtors{Коротков~В.\,В.} см.\ Зацаринный~А.\,А.&&\\
\Avtors{Кривенко~М.\,П.} Выбор модели данных в~задачах медицинской диагностики&4&27--29\\
\Avtors{Кружков~М.\,Г.} см.\ Гончаров~А.\,А.&&\\
\Avtors{Кружков~М.\,Г.} см.\ Инькова~О.\,Ю.&&\\
\Avtors{Кудрявцев~А.\,А.} Априорное обобщенное гамма-распределение в~байесовских моделях\linebreak
\\[-12pt]
\hspace*{23pt}баланса&3&27--33\\
\Avtors{Кудрявцев~А.\,А.} О представлении 
гамма-экспоненциального и~обобщенного отрица-\linebreak
\\[-12pt]
\hspace*{23pt}тельного биномиального распределений&4&76--80\\
\Avtors{Кудрявцев~А.\,А., Палионная~С.\,И., Шоргин~В.\,С.} Априорные Фреше и масштабированное\linebreak
\\[-12pt]
\hspace*{23pt}обратное хи-распределение в~байесовских моделях баланса&1&62--66\\
\Avtors{Кудрявцев~А.\,А.} см.\ Арутюнов~Е.\,Н.&&\\
\Avtors{Кузьмин~В.\,Ю.} см.\ Горшенин~А.\,К.&&\\
\Avtors{Кузьмин~В.\,Ю.} см.\ Горшенин~А.\,К.&&\\
\Avtors{Курьянский~М.\,К.} см.\ Вышинский~Л.\,Л.&&\\
\Avtors{Ланге~M.\,M.} О~сравнительной эффективности схем классификации данных на~ансамбле\linebreak
\\[-12pt]
\hspace*{23pt}источников с~использованием средней взаимной информации&4&18--26\\
\Avtors{Лебедев~А.\,В.} Нетранзитивные триплеты непрерывных случайных величин и~их прило-\linebreak
\\[-12pt]
\hspace*{23pt}жения&3&20--26\\
\Avtors{Листопад~С.\,В.} см.\ Колесников~А.\,В.&&\\
\Avtors{Логачев~О.\,А., Сукаев~А.\,А., Федоров~С.\,Н.} Об одном методе решения систем 
квад\-ра\-тич\-ных булевых уравнений, использующем локальные аффинности булевых\linebreak
\\[-12pt]
\hspace*{23pt}функций&2&37--46\\
\Avtors{Логачев~О.\,А., Сукаев~А.\,А., Федоров~С.\,Н.} Полиномиальные алгоритмы вычисления\linebreak
\\[-12pt]
\hspace*{23pt}локальных аффинностей квадратичных булевых функций&1&67--74\\
\Avtors{Лукашенко~О.\,В., Морозов~Е.\,В., Пагано~М.} Гауссовская аппроксимация процесса\linebreak
\\[-12pt]
\hspace*{23pt}распределенных вычислений&2&109--116\\
\Avtors{Малашенко~Ю.\,Е., Назарова~И.\,А., Новикова~Н.\,М.} Анализ уязвимости многополюсных\linebreak
\\[-12pt]
\hspace*{23pt}сетей при~структурных повреждениях&1&33--39\\
\Avtors{Маркова~Е.\,В., Гольская~А.\,А., Дзантиев~И.\,Л., Гудкова~И.\,А., Шоргин~С.\,Я.} Сравнительный анализ показателей эффективности модели беспроводной сети меж\-ма\-шин\-ного взаимодействия, работающей в~рамках двух политик разделения радиоре-\linebreak
\\[-12pt]
\hspace*{23pt}сурсов&1&108--116\\
\Avtors{Мартынов~О.\,П.} см.\ Горшенин~А.\,К.&&\\
\Avtors{Масляков~Г.\,О.} см.\ Дюкова~Е.\,В.&&\\
\Avtors{Матвеев~М.\,Г.} см.\ Зацаринный~А.\,А.&&\\
\Avtors{Мейханаджян~Л.\,А., Разумчик~Р.\,В.} Система массового обслуживания Geo$/G/1/\infty$\linebreak
\\[-12pt]
\hspace*{23pt}синверсионным порядком обслуживания и~ресамплингом в~дискретном времени&4&60--67\\
\end{tabular}
}

\pagebreak

\def\leftkol{АВТОРСКИЙ УКАЗАТЕЛЬ ЗА 2019 г.} % ENGLISH ABSTRACTS}

\def\rightkol{АВТОРСКИЙ УКАЗАТЕЛЬ ЗА 2019 г.} %ENGLISH ABSTRACTS}

%\thispagestyle{myheadings}
\def\leftfootline{\small{\textbf{\thepage}
\hfill ИНФОРМАТИКА И ЕЁ ПРИМЕНЕНИЯ\ \ \ том~13\ \ \ выпуск~4\ \ \ 2019}
}%
 \def\rightfootline{\small{ИНФОРМАТИКА И ЕЁ ПРИМЕНЕНИЯ\ \ \ том~13\ \ \ выпуск~4\ \ \ 2019
 \hfill \textbf{\thepage}}}


\noindent
{\tabcolsep=3pt
\begin{tabular}{p{394pt}cc}
&\textbf{Вып.} & \textbf{Стр.}\\[3pt]
\Avtors{Мельников~А.\,В.} см.\ Бурлуцкий~В.\,В.&&\\
\Avtors{Миллер~Г.\,Б.} см.\ Босов~А.\,В.&&\\
\Avtors{Морозов~Е.\,В.} см.\ Лукашенко~О.\,В.&&\\
\Avtors{Мхитарян~Г.\,А.} см.\ Босов~А.\,В.&&\\
\Avtors{Назарова~И.\,А.} см.\ Малашенко~Ю.\,Е.&&\\
\Avtors{Наумов~А.\,В.} см.\ Босов~А.\,В.&&\\
\Avtors{Наумов~В.\,А.} см.\ Горбунова~А.\,В.&&\\
\Avtors{Новикова~Н.\,М.} см.\ Малашенко~Ю.\,Е.&&\\
\Avtors{Нуриев~В.\,А.} Архитектура системы нейронного машинного перевода&3&90--96\\
\Avtors{Осипова~В.\,А.} см.\ Абгарян~К.\,К.&&\\
\Avtors{Павлов~Ю.\,Л.} Об асимптотике кластерного коэффициента конфигурационного графа\linebreak
\\[-12pt]
\hspace*{23pt}с~неизвестным распределением степеней вершин&3&\hphantom{1}9--13\\
\Avtors{Пагано~М.} см.\ Лукашенко~О.\,В.&&\\
\Avtors{Палионная~С.\,И.} см.\ Кудрявцев~А.\,А.&&\\
\Avtors{Панов~А.\,И.} см.\ Смирнов~И.\,В.&&\\
\Avtors{Пенкин~Г.\,О.} см.\ Аникеев~Д.\,А.&&\\
\Avtors{Прокофьев~П.\,А.} см.\ Дюкова~Е.\,В.&&\\
\Avtors{Разумчик~Р.\,В.} см.\ Коновалов~М.\,Г.&&\\
\Avtors{Разумчик~Р.\,В.} см.\ Мейханаджян~Л.\,А.&&\\
\Avtors{Румовская~С.\,Б., Кириков~И.\,А.} Методы моделирования и~визуального представления\linebreak
\\[-12pt]
\hspace*{23pt}конфликта в~малом коллективе экспертов, решающих проблемы (обзор)&3&122--130\\
\Avtors{Рыбаков~К.\,А.} Об одном классе задач фильтрации на многообразиях&1&16--24\\
\Avtors{Рязанов~В.\,В.} см.\ Журавлев~Ю.\,И.&&\\
\Avtors{Самуйлов~К.\,Е.} см.\ Гайдамака~А.\,А.&&\\
\Avtors{Самуйлов~К.\,Е.} см.\ Горбунова~А.\,В.&&\\
\Avtors{Сапунова~А.\,П.} см.\ Босов~А.\,В.&&\\
\Avtors{Сатин~Я.\,А.} см.\ Зейфман~А.\,И.&&\\
\Avtors{Сейфуль-Мулюков~Р.\,Б.} Законы информатики и~синергетики в~познании сложных\linebreak
\\[-12pt]
\hspace*{23pt}систем&4&107--113\\
\Avtors{Сенько~О.\,В.} см.\ Журавлев~Ю.\,И.&&\\
\Avtors{Синицын~И.\,Н.} Интерполяционное аналитическое моделирование распределений\linebreak
\\[-12pt]
\hspace*{23pt}в~сложных стохастических системах&1&2--8\\
\Avtors{Скрынник~А.\,А.} см.\ Смирнов~И.\,В.&&\\
\Avtors{Смирнов~И.\,В., Панов~А.\,И., Скрынник~А.\,А., Чистова~Е.\,В.} Персональный когнитивный\linebreak
\\[-12pt]
\hspace*{23pt}ассистент: концепция и~принципы работы&3&105--113\\
\Avtors{Стефанович~А.\,И.} см.\ Босов~А.\,В.&&\\
\Avtors{Стефанович~А.\,И.} см.\ Босов~А.\,В.&&\\
\Avtors{Стрижов~В.\,В.} см.\ Аникеев~Д.\,А.&&\\
\Avtors{Стрижов~В.\,В.} см.\ Грабовой~А.\,В.&&\\
\Avtors{Сукаев~А.\,А.} см.\ Логачев~О.\,А.&&\\
\Avtors{Сукаев~А.\,А.} см.\ Логачев~О.\,А.&&\\
\Avtors{Сучков~А.\,П.} Научный результат как информационный объект в~контексте системы\linebreak
\\[-12pt]
\hspace*{23pt}управления научными сервисами&3&137--144\\
\Avtors{Тарасов~Е.\,А.} см.\ Ковалёв~Д.\,Ю.&&\\
\Avtors{Тархов~А.\,А.} см.\ Захарова~Т.\,В.&&\\
\Avtors{Тимонина~Е.\,Е.} см.\ Грушо~А.\,А.&&\\
\Avtors{Тимонина~Е.\,Е.} см.\ Грушо~А.\,А.&&\\
\Avtors{Тимонина~Е.\,Е.} см.\ Грушо~А.\,А.&&\\
\Avtors{Тимонина~Е.\,Е.} см.\ Грушо~А.\,А.&&\\
\Avtors{Титова~А.\,И.} см.\ Арутюнов~Е.\,Н.&&\\
\Avtors{Ушаков~В.\,Г., Ушаков~Н.\,Г.} Выходящие потоки в~однолинейной системе с~относитель-\linebreak
\\[-12pt]
\hspace*{23pt}ным приоритетом&4&42--47\\
\Avtors{Ушаков~В.\,Г.} см.\ Агаларов~Я.\,М.&&\\
\Avtors{Ушаков~Н.\,Г.} см.\ Ушаков~В.\,Г.&&\\
\end{tabular}
}

\pagebreak

\def\leftkol{АВТОРСКИЙ УКАЗАТЕЛЬ ЗА 2019 г.} % ENGLISH ABSTRACTS}

\def\rightkol{АВТОРСКИЙ УКАЗАТЕЛЬ ЗА 2019 г.} %ENGLISH ABSTRACTS}

%\thispagestyle{myheadings}
\def\leftfootline{\small{\textbf{\thepage}
\hfill ИНФОРМАТИКА И ЕЁ ПРИМЕНЕНИЯ\ \ \ том~13\ \ \ выпуск~4\ \ \ 2019}
}%
 \def\rightfootline{\small{ИНФОРМАТИКА И ЕЁ ПРИМЕНЕНИЯ\ \ \ том~13\ \ \ выпуск~4\ \ \ 2019
 \hfill \textbf{\thepage}}}


\noindent
{\tabcolsep=3pt
\begin{tabular}{p{394pt}cc}
&\textbf{Вып.} & \textbf{Стр.}\\[3pt]
\Avtors{Федоров~С.\,Н.} см.\ Логачев~О.\,А.&&\\
\Avtors{Федоров~С.\,Н.} см.\ Логачев~О.\,А.&&\\
\Avtors{Флеров~Ю.\,А.} см.\ Вышинский~Л.\,Л.&&\\
\Avtors{Царегородцев~А.\,Л.} см.\ Бурлуцкий~В.\,В.&&\\
\Avtors{Чистова~Е.\,В.} см.\ Смирнов~И.\,В.&&\\
\Avtors{Чухно~Н.\,В.} см.\ Гайдамака~А.\,А.&&\\
\Avtors{Чухно~О.\,В.} см.\ Гайдамака~А.\,А.&&\\
\Avtors{Шестаков~О.\,В.} Обращение однородных операторов с помощью стабилизированной\linebreak
\\[-12pt]
\hspace*{23pt}жесткой пороговой обработки при неизвестной дисперсии шума&1&49--54\\
\Avtors{Шестаков~О.\,В.} Свойства вейвлет-оценок сигналов, регистрируемых в~случайные\linebreak
\\[-12pt]
\hspace*{23pt}моменты времени&2&16--21\\
\Avtors{Шестаков~О.\,В.} Среднеквадратичный риск нелинейной регуляризации задачи обраще-\linebreak
\\[-12pt]
\hspace*{23pt}ния линейных однородных операторов при случайном объеме выборки&4&48--53\\
\Avtors{Шнурков~П.\,В., Вахтанов~Н.\,А.} Исследование проблемы оптимального управления запасом дискретного продукта в~стохастической модели регенерации с~непрерывно\linebreak
\\[-12pt]
\hspace*{23pt}происходящим потреблением и~случайной задержкой поставки&2&54--61\\
\Avtors{Шнурков~П.\,В., Вахтанов~Н.\,А.} О~решении проблемы оптимального управления запасом дискретного продукта в~стохастической модели регенерации с непрерывно\linebreak
\\[-12pt]
\hspace*{23pt}происходящим потреблением&3&50--57\\
\Avtors{Шоргин~В.\,С.} см.\ Кудрявцев~А.\,А.&&\\
\Avtors{Шоргин~С.\,Я.} см.\ Гайдамака~А.\,А.&&\\
\Avtors{Шоргин~С.\,Я.} см.\ Маркова~Е.\,В.&&\\
\Avtors{Якимчук~А.\,В.} см.\ Бурлуцкий~В.\,В.&&\\
\end{tabular}
}

%\thispagestyle{myheadings}
\def\leftfootline{\small{\textbf{\thepage}
\hfill ИНФОРМАТИКА И ЕЁ ПРИМЕНЕНИЯ\ \ \ том~13\ \ \ выпуск~4\ \ \ 2019}
}%
 \def\rightfootline{\small{ИНФОРМАТИКА И ЕЁ ПРИМЕНЕНИЯ\ \ \ том~13\ \ \ выпуск~4\ \ \ 2019
 \hfill \textbf{\thepage}}}

 \label{end\stat}

\newpage

\def\stat{cont-e}
{%\hrule\par
%\vskip 7pt % 7pt
\raggedleft\Large \bf%\baselineskip=3.2ex
2\,0\,1\,9\ \ A\,U\,T\,H\,O\,R\ \ I\,N\,D\,E\,X \vskip 17pt
 \hrule
 \par
\vskip 21pt plus 6pt minus 3pt }

\label{st\stat}

\def\tit{\ }

\def\aut{\ }
\def\auf{\ }

\def\leftkol{\ } %2019 AUTHOR INDEX} % ENGLISH ABSTRACTS}

\def\rightkol{\ } %2019 AUTHOR INDEX} %ENGLISH ABSTRACTS}

\titele{\tit}{\aut}{\auf}{\leftkol}{\rightkol}
\addcontentsline{toc}{subsection}{\textrm\textbf 2019 Author Index}

\def\leftfootline{\small{\textbf{\thepage}
\hfill INFORMATIKA I EE PRIMENENIYA~--- INFORMATICS AND APPLICATIONS\ \ \ 2019\
\ \ volume~13\ \ \ issue\ 4}
}%
 \def\rightfootline{\small{INFORMATIKA I EE PRIMENENIYA~--- INFORMATICS AND APPLICATIONS\ \ \ 2019\ \ \ volume~13\ \ \ issue\ 4
\hfill \textbf{\thepage}}}

%\vspace*{-12pt}

\noindent
{\tabcolsep=3pt
\begin{tabular}{p{396pt}cc}
&\textbf{Issue} & \textbf{Page}\\[6pt]
\Avtors{Abgaryan~K.\,K.\ and Osipova~V.\,A.} Application of decision support methods for the multicriterial\linebreak
\\[-12pt]
\hspace*{23pt}selection of multiscale compositions&2&47--53\\
\Avtors{Agalarov~Ya.\,M.\ and Konovalov~M.\,G.} Proof of the unimodality of the objective function in\linebreak
\\[-12pt]
\hspace*{23pt}$M/M/N$ queue with threshold-based congestion control&2&2--6\\
\Avtors{Agalarov~Ya.\,M.\ and Ushakov~V.\,G.} On the unimodality of the~income function of a~type $G|M|s$\linebreak
\\[-12pt]
\hspace*{23pt}queueing system with controlled queue&1&55--61\\
\Avtors{Agasandyan~G.\,A.} Performance estimations for optimal-on-CC-VaR portfolios in option markets&3&72--81\\
\Avtors{Agasandyan~G.\,A.} Theoretical foundations of~continuous VaR criterion optimization in~the~col-\linebreak
\\[-12pt]
\hspace*{23pt}lection of~markets&4&36--41\\
\Avtors{Anashin~V.\,S.} On automata models of blockchain&2&29--36\\
\Avtors{Anikeyev~D.\,A., Penkin~G.\,O., and Strijov~V.\,V.} Local approximation models for~human physical\linebreak
\\[-12pt]
\hspace*{23pt}activity classification&1&40--48\\
\Avtors{Arutyunov~E.\,N., Kudryavtsev~A.\,A., and Titova~A.\,I.} Bayesian models of factors balance with\linebreak
\\[-12pt]
\hspace*{23pt}\textit{a~priori} Weibull and Nakagami distributions&2&71--75\\
\Avtors{Bakhteev~O.\,Yu.} see Grabovoy~A.\,V.&&\\
\Avtors{Bondarenko~N.\,N.} see Zhuravlev~Yu.\,I.&&\\
\Avtors{Borisov~A.\,V.} Numerical schemes of markov jump process filtering given discretized observa-\linebreak
\\[-12pt]
\hspace*{23pt}tions~I:~Accuracy characteristics&4&68--75\\
\Avtors{Bosov~A.\,V.\ and Miller~G.\,B.} On the conditionally minimax nonlinear filtering concept\linebreak
\\[-12pt]
\hspace*{23pt}development: Filter modification and analysis&2&\hphantom{1}7--15\\
\Avtors{Bosov~A.\,V., Naumov~A.\,V., Mkhitaryan~G.\,A., and Sapunova~A.\,P.} Using the model of~gamma\linebreak
\\[-12pt]
\hspace*{23pt}distribution in~the~problem of~forming a~time-limited test in~a~distance learning system&4&11--17\\
\Avtors{Bosov~A.\,V.\ and Stefanovich~A.\,I.} Stochastic differential system output control by~the~quadratic\linebreak
\\[-12pt]
\hspace*{23pt}criterion. II.~Dynamic programming equations numerical solution&1&\hphantom{1}9--15\\
\Avtors{Bosov~A.\,V.\ and Stefanovich~A.\,I.} Stochastic differential system output control by~the~quadratic\linebreak
\\[-12pt]
\hspace*{23pt}criterion. III.~Optimal control properties analysis&3&41--49\\

\Avtors{Burlutskiy~V.\,V., Yakimchuk~A.\,V., Melnikov~A.\,V., Tsaregorodtsev~A.\,L., and Voloshin~S.\,V.} Development of a method for the formation of~attribute space and a~model for~the~assessment and prediction of anthropogenic influence on~the~environment (on~the~example of~the~forest fund of the~oil-producing region)&3& 131--136\\
\Avtors{Chistova~E.\,V.} see Smirnov~I.\,V.&&\\
\Avtors{Chukhno~N.\,V.} see Gaidamaka~A.\,A.&&\\
\Avtors{Chukhno~O.\,V.} see Gaidamaka~A.\,A.&&\\
\Avtors{Djukova~E.\,V., Maslyakov~G.\,O., and Prokofyev~P.\,A.} On the number of maximal independent\linebreak
\\[-12pt]
\hspace*{23pt}elements of~partially ordered sets (the case of~chains)&1&25--32\\
\Avtors{Dokukin~A.\,A.} see Zhuravlev~Yu.\,I.&&\\
\Avtors{Dulin~S.\,K., Dulina~N.\,G., and Kozhunova~O.\,S.} Synthesis of geodata in spatial infrastructures\linebreak
\\[-12pt]
\hspace*{23pt}based on related data&1&82--90\\
\Avtors{Dulina~N.\,G.} see Dulin~S.\,K.&&\\
\Avtors{Dzantiev~I.\,L.} see Markova~E.\,V.&&\\
\Avtors{Fedorov~S.\,N.} see Logachev~O.\,A.&&\\
\Avtors{Fedorov~S.\,N.} see Logachev~O.\,A.&&\\
\end{tabular}
}
\pagebreak

\def\leftfootline{\small{\textbf{\thepage}
\hfill INFORMATIKA I EE PRIMENENIYA~--- INFORMATICS AND APPLICATIONS\ \ \ 2019\
\ \ volume~13\ \ \ issue\ 4}
}%
 \def\rightfootline{\small{INFORMATIKA I EE PRIMENENIYA~---
INFORMATICS AND APPLICATIONS\ \ \ 2019\ \ \ volume~13\ \ \ issue\ 4
\hfill \textbf{\thepage}}}

\def\leftkol{2019 AUTHOR INDEX} % ENGLISH ABSTRACTS}

\def\rightkol{2019 AUTHOR INDEX} %ENGLISH ABSTRACTS}


\noindent
{\tabcolsep=3pt
\begin{tabular}{p{395.48108pt}cc}
&\textbf{Issue} & \textbf{Page}\\[6pt]
\Avtors{Flerov~Yu.\,A.} see Vyshinsky~L.\,L.&&\\
\Avtors{Gaidamaka~A.\,A., Chukhno~N.\,V., Chukhno~O.\,V., Samouylov~K.\,E., and Shorgin~S.\,Ya.} Formalization of the alternatives ranking method for group decision making in social net-\linebreak
\\[-12pt]
\hspace*{23pt}works&3&63--71\\
\Avtors{Gaidamaka~Yu.\,V.} see Gorbunova~A.\,V.&&\\
\Avtors{Golskaia~A.\,A.} see Markova~E.\,V.&&\\
\Avtors{Goncharov~A.\,A.\ and Inkova~O.\,Yu.} Methods for identification of implicit logical-semantic\linebreak
\\[-12pt]
\hspace*{23pt}relations in~texts&3&\hphantom{1}97--104\\
\Avtors{Goncharov~A.\,A., Zatsman~I.\,M., and Kruzhkov~M.\,G.} Temporal data in~lexicographic databases&4&90--96\\
\Avtors{Gorbunova~A.\,V., Naumov~V.\,A., Gaidamaka~Yu.\,V., and Samouylov~K.\,E.} Resource queuing\linebreak
\\[-12pt]
\hspace*{23pt}systems with general service discipline&1&\hphantom{1}99--107\\
\Avtors{Gorshenin~A.\,K.\ and Kuzmin~V.\,Yu.} Application of recurrent neural networks to~forecasting\linebreak
\\[-12pt]
\hspace*{23pt}the~moments of~finite normal mixtures&3&114--121\\
\Avtors{Gorshenin~A.\,K.\ and Kuzmin~V.\,Yu.} Optimization of hyperparameters of neural networks using\linebreak
\\[-12pt]
\hspace*{23pt}high-performance computing for prediction of precipitation&1&75--81\\
\Avtors{Gorshenin~A.\,K.\ and Martynov~O.\,P.} Hybrid extreme gradient boosting models to~impute\linebreak
\\[-12pt]
\hspace*{23pt}the~missing data in~precipitation records&3&34--40\\
\Avtors{Grabovoy~A.\,V., Bakhteev~O.\,Yu., and Strijov~V.\,V.} Estimation of the relevance of the neural\linebreak
\\[-12pt]
\hspace*{23pt}network parameters&2&62--70\\
\Avtors{Grinchenko~S.\,N.} On the genesis of the information society: Informatics-cybernetic model\linebreak
\\[-12pt]
\hspace*{23pt}representation&2&100--108\\
\Avtors{Grusho~A.\,A., Grusho~N.\,A., and Timonina~E.\,E.} Methods of identification of ``weak'' signs of\linebreak
\\[-12pt]
\hspace*{23pt}violations of information security&3&3--8\\
\Avtors{Grusho~A.\,A., Grusho~N.\,A., and Timonina~E.\,E.} Using metadata to~implement multilevel security\linebreak
\\[-12pt]
\hspace*{23pt}policy requirements&4&85--89\\
\Avtors{Grusho~A.\,A., Zabezhailo~M.\,I., Grusho~N.\,A., and Timonina~E.\,E.} Architectural decisions in the problem of identification of~fraud in~the~analysis of~information flows in~digital eco-\linebreak
\\[-12pt]
\hspace*{23pt}nomy&2&22--28\\
\Avtors{Grusho~A.\,A., Zabezhailo~M.\,I., Grusho~N.\,A., and Timonina~E.\,E.} Concepts forming on~the~basis\linebreak
\\[-12pt]
\hspace*{23pt}of~small samples&4&81--84\\
\Avtors{Grusho~N.\,A.} see Grusho~A.\,A.&&\\
\Avtors{Grusho~N.\,A.} see Grusho~A.\,A.&&\\
\Avtors{Grusho~N.\,A.} see Grusho~A.\,A.&&\\
\Avtors{Grusho~N.\,A.} see Grusho~A.\,A.&&\\
\Avtors{Gudkova~I.\,A.} see Markova~E.\,V.&&\\
\Avtors{Inkova~O.\,Yu.\ and Kruzhkov~M.\,G.} Compatibility of logical semantic relations: Methods\linebreak
\\[-12pt]
\hspace*{23pt}of~quantitative analysis&2&83--91\\
\Avtors{Inkova~O.\,Yu.} see Goncharov~A.\,A.&&\\
\Avtors{Kirikov~I.\,A.} see Rumovskaya~S.\,B.&&\\
\Avtors{Kiseleva~K.\,M.} see Zeifman~A.\,I.&&\\
\Avtors{Kolesnikov~A.\,V.\ and Listopad~S.\,V.} Heterogeneous thinking protocol of hybrid intelligent\linebreak
\\[-12pt]
\hspace*{23pt}multiagent system for~solving distributional power grid recovery problem&2&76--82\\
\Avtors{Konovalov~M.\,G.\ and Razumchik~R.\,V.} Mixed policies for~online job allocation in~one class\linebreak
\\[-12pt]
\hspace*{23pt}of~systems with~parallel service&4&54--59\\
\Avtors{Konovalov~M.\,G.} see Agalarov~Ya.\,M.&&\\
\Avtors{Korotkov~V.\,V.} see Zatsarinny~A.\,A.&&\\
\Avtors{Kovalev~D.\,Y.\ and Tarasov~E.\,A.} Virtual experiments in data intensive research&2&117--125\\
\Avtors{Kozhunova~O.\,S.} see Dulin~S.\,K.&&\\
\Avtors{Krivenko~M.\,P.} Data model selection in~medical diagnostic tasks&4&27--29\\
\Avtors{Kruzhkov~M.\,G.} see Goncharov~A.\,A.&&\\
\Avtors{Kruzhkov~M.\,G.} see Inkova~O.\,Yu.&&\\
\Avtors{Kudryavtsev~A.\,A.} \textit{A priori} generalized gamma distribution in Bayesian balance models&3&27--33\\
\Avtors{Kudryavtsev~A.\,A.} On the representation of gamma-exponential and~generalized negative\linebreak
\\[-12pt]
\hspace*{23pt}binomial distributions&4&76--80\\
\end{tabular}
}
\pagebreak

\def\leftfootline{\small{\textbf{\thepage}
\hfill INFORMATIKA I EE PRIMENENIYA~--- INFORMATICS AND APPLICATIONS\ \ \ 2019\
\ \ volume~13\ \ \ issue\ 4}
}%
 \def\rightfootline{\small{INFORMATIKA I EE PRIMENENIYA~---
INFORMATICS AND APPLICATIONS\ \ \ 2019\ \ \ volume~13\ \ \ issue\ 4
\hfill \textbf{\thepage}}}

\def\leftkol{2019 AUTHOR INDEX} % ENGLISH ABSTRACTS}

\def\rightkol{2019 AUTHOR INDEX} %ENGLISH ABSTRACTS}


\noindent
{\tabcolsep=3pt
\begin{tabular}{p{395.48108pt}cc}
&\textbf{Issue} & \textbf{Page}\\[6pt]
\Avtors{Kudryavtsev~A.\,A., Palionnaia~S.\,I., and Shorgin~V.\,S.} \textit{A priori} Frechet and~scaled inverse chi\linebreak
\\[-12pt]
\hspace*{23pt}distribution in~Bayesian balance models&1&62--66\\
\Avtors{Kudryavtsev~A.\,A.} see Arutyunov~E.\,N.&&\\
\Avtors{Kuryansky~M.\,K.} see Vyshinsky~L.\,L.&&\\
\Avtors{Kuzmin~V.\,Yu.} see Gorshenin~A.\,K.&&\\
\Avtors{Kuzmin~V.\,Yu.} see Gorshenin~A.\,K.&&\\
\Avtors{Lange~M.\,M.} On comparative efficiency of classification schemes in an ensemble of data\linebreak
\\[-12pt]
\hspace*{23pt}sources using average mutual information&4&18--26\\
\Avtors{Lebedev~A.\,V.} Nontransitive triplets of continuous random variables and their applications&3&20--26\\
\Avtors{Listopad~S.\,V.} see Kolesnikov~A.\,V.&&\\
\Avtors{Logachev~O.\,A., Sukayev~A.\,A., and Fedorov~S.\,N.} On local affinity based method of solving\linebreak
\\[-12pt]
\hspace*{23pt}systems of quadratic Boolean equations&2&37--46\\
\Avtors{Logachev~O.\,A., Sukayev~A.\,A., and Fedorov~S.\,N.} Polynomial algorithms for~constructing local\linebreak
\\[-12pt]
\hspace*{23pt}affinities of~quadratic Boolean functions&1&67--74\\
\Avtors{Lukashenko~O.\,V., Morozov~E.\,V., and Pagano~M.} A~Gaussian approximation of~the~distributed\linebreak
\\[-12pt]
\hspace*{23pt}computing process&2&109--116\\
\Avtors{Malashenko~Yu.\,E., Nazarova~I.\,A., and Novikova~N.\,M.} Vulnerability analysis of multipolar\linebreak
\\[-12pt]
\hspace*{23pt}networks after structural damages&1&33--39\\
\Avtors{Markova~E.\,V., Golskaia~A.\,A., Dzantiev~I.\,L., Gudkova~I.\,A., and Shorgin~S.\,Ya.} Comparative analysis of performance measures for a wireless machine-to-machine network model\linebreak
\\[-12pt]
\hspace*{23pt}operating within two radio resource management policies&1&108--116\\
\Avtors{Martynov~O.\,P.} see Gorshenin~A.\,K.&&\\
\Avtors{Maslyakov~G.\,O.} see Djukova~E.\,V.&&\\
\Avtors{Matveev~M.\,G.} see Zatsarinny~A.\,A.&&\\
\Avtors{Melnikov~A.\,V.} see Burlutskiy~V.\,V.&&\\
\Avtors{Meykhanadzhyan~L.\,A.\ and Razumchik~R.\,V.} Discrete-time $\mathrm{GEO}/G/1/\infty$ LIFO queue with\linebreak
\\[-12pt]
\hspace*{23pt}resampling policy&4&60--67\\
\Avtors{Miller~G.\,B.} see Bosov~A.\,V.&&\\
\Avtors{Mkhitaryan~G.\,A.} see Bosov~A.\,V.&&\\
\Avtors{Morozov~E.\,V.} see Lukashenko~O.\,V.&&\\
\Avtors{Naumov~A.\,V.} see Bosov~A.\,V.&&\\
\Avtors{Naumov~V.\,A.} see Gorbunova~A.\,V.&&\\
\Avtors{Nazarova~I.\,A.} see Malashenko~Yu.\,E.&&\\
\Avtors{Novikova~N.\,M.} see Malashenko~Yu.\,E.&&\\
\Avtors{Nuriev~V.\,A.} Architecture of a~machine translation system&3&90--96\\
\Avtors{Osipova~V.\,A.} see Abgaryan~K.\,K.&&\\
\Avtors{Pagano~M.} see Lukashenko~O.\,V.&&\\
\Avtors{Palionnaia~S.\,I.} see Kudryavtsev~A.\,A.&&\\
\Avtors{Panov~A.\,I.} see Smirnov~I.\,V.&&\\
\Avtors{Pavlov~Yu.\,L.} On the asymptotics of clustering coefficient in~a~configuration graph with unknown\linebreak
\\[-12pt]
\hspace*{23pt}distribution of~vertex degrees&3&\hphantom{1}9--13\\
\Avtors{Penkin~G.\,O.} see Anikeyev~D.\,A.&&\\
\Avtors{Prokofyev~P.\,A.} see Djukova~E.\,V.&&\\
\Avtors{Razumchik~R.\,V.} see Konovalov~M.\,G.&&\\
\Avtors{Razumchik~R.\,V.} see Meykhanadzhyan~L.\,A.&&\\
\Avtors{Rumovskaya~S.\,B.\ and Kirikov~I.\,A.} Methods of modeling and visual representation of~a~conflict\linebreak
\\[-12pt]
\hspace*{23pt}in~a~small collective of experts solving problems (review)&3&122--130\\
\Avtors{Ryazanov~V.\,V.} see Zhuravlev~Yu.\,I.&&\\
\Avtors{Rybakov~K.\,A.} On a class of filtering problems on~manifolds&1&16--24\\
\Avtors{Samouylov~K.\,E.} see Gaidamaka~A.\,A.&&\\
\Avtors{Samouylov~K.\,E.} see Gorbunova~A.\,V.&&\\
\Avtors{Sapunova~A.\,P.} see Bosov~A.\,V.&&\\
\Avtors{Satin~Y.\,A.} see Zeifman~A.\,I.&&\\
\end{tabular}
}
\pagebreak

\def\leftfootline{\small{\textbf{\thepage}
\hfill INFORMATIKA I EE PRIMENENIYA~--- INFORMATICS AND APPLICATIONS\ \ \ 2019\
\ \ volume~13\ \ \ issue\ 4}
}%
 \def\rightfootline{\small{INFORMATIKA I EE PRIMENENIYA~---
INFORMATICS AND APPLICATIONS\ \ \ 2019\ \ \ volume~13\ \ \ issue\ 4
\hfill \textbf{\thepage}}}

\def\leftkol{2019 AUTHOR INDEX} % ENGLISH ABSTRACTS}

\def\rightkol{2019 AUTHOR INDEX} %ENGLISH ABSTRACTS}


\noindent
{\tabcolsep=3pt
\begin{tabular}{p{395.48108pt}cc}
&\textbf{Issue} & \textbf{Page}\\[6pt]
\Avtors{Sen'ko~O.\,V.} see Zhuravlev~Yu.\,I.&&\\
\Avtors{Seyful-Mulyukov~R.\,B.} Understanding of~complex systems using~the~laws of~synergetics\linebreak
\\[-12pt]
\hspace*{23pt}and~informatics&4&107--113\\
\Avtors{Shestakov~O.\,V.} Inversion of homogeneous operators using stabilized hard thresholding with\linebreak
\\[-12pt]
\hspace*{23pt}unknown noise variance&1&49--54\\
\Avtors{Shestakov~O.\,V.} Properties of wavelet estimates of signals recorded at random time points&2&16--21\\
\Avtors{Shestakov~O.\,V.} The mean square risk of~nonlinear regularization in~the~problem of~inversion\linebreak
\\[-12pt]
\hspace*{23pt}of~linear homogeneous operators with~a~random sample size&4&48--53\\
\Avtors{Shnurkov~P.\,V.\ and Vakhtanov~N.\,A.} On the solution of the optimal control problem of inventory of~a~discrete product in~the~stochastic model of~regeneration with continuously\linebreak
\\[-12pt]
\hspace*{23pt}occuring consumption&3&50--57\\
\Avtors{Shnurkov~P.\,V.\ and Vakhtanov~N.\,A.} Research of the optimal control problem of~inventory of~a~discrete product in~the~stochastic regeneration model with continuously\linebreak
\\[-12pt]
\hspace*{23pt}occuring consumption and random delivery delay&2&54--61\\
\Avtors{Shorgin~S.\,Ya.} see Gaidamaka~A.\,A.&&\\
\Avtors{Shorgin~S.\,Ya.} see Markova~E.\,V.&&\\
\Avtors{Shorgin~V.\,S.} see Kudryavtsev~A.\,A.&&\\
\Avtors{Sinitsyn~I.\,N.} Interpolatonal analytical modeling in~complex stochastic systems&1&2--8\\
\Avtors{Skrynnik~A.\,A.} see Smirnov~I.\,V.&&\\
\Avtors{Smirnov~I.\,V., Panov~A.\,I., Skrynnik~A.\,A., and Chistova~E.\,V.} Personal cognitive assistant: \linebreak
\\[-12pt]
\hspace*{23pt}Concept and key principals&3&105--113\\
\Avtors{Stefanovich~A.\,I.} see Bosov~A.\,V.&&\\
\Avtors{Stefanovich~A.\,I.} see Bosov~A.\,V.&&\\
\Avtors{Strijov~V.\,V.} see Anikeyev~D.\,A.&&\\
\Avtors{Strijov~V.\,V.} see Grabovoy~A.\,V.&&\\
\Avtors{Suchkov~A.\,P.} The scientific result as~the~information object in~the~context of~the~scientific\linebreak
\\[-12pt]
\hspace*{23pt}services system management&3&137--144\\
\Avtors{Sukayev~A.\,A.} see Logachev~O.\,A.&&\\
\Avtors{Sukayev~A.\,A.} see Logachev~O.\,A.&&\\
\Avtors{Tarasov~E.\,A.} see Kovalev~D.\,Y.&&\\
\Avtors{Tarkhov~A.\,A.} see Zakharova~T.\,V.&&\\
\Avtors{Timonina~E.\,E.} see Grusho~A.\,A.&&\\
\Avtors{Timonina~E.\,E.} see Grusho~A.\,A.&&\\
\Avtors{Timonina~E.\,E.} see Grusho~A.\,A.&&\\
\Avtors{Timonina~E.\,E.} see Grusho~A.\,A.&&\\
\Avtors{Titova~A.\,I.} see Arutyunov~E.\,N.&&\\
\Avtors{Tsaregorodtsev~A.\,L.} see Burlutskiy~V.\,V.&&\\
\Avtors{Ushakov~N.\,G.} see Ushakov~V.\,G.&&\\
\Avtors{Ushakov~V.\,G.\ and Ushakov~N.\,G.} The output streams in~the~single server queueing system\linebreak
\\[-12pt]
\hspace*{23pt}with~a~head of~the~line priority&4&42--47\\
\Avtors{Ushakov~V.\,G.} see Agalarov~Ya.\,M.&&\\
\Avtors{Vakhtanov~N.\,A.} see Shnurkov~P.\,V.&&\\
\Avtors{Vakhtanov~N.\,A.} see Shnurkov~P.\,V.&&\\
\Avtors{Vinogradov~A.\,P.} see Zhuravlev~Yu.\,I.&&\\
\Avtors{Voloshin~S.\,V.} see Burlutskiy~V.\,V.&&\\
\Avtors{Vyshinsky~L.\,L., Kuryansky~M.\,K., and Flerov~Yu.\,A.} Digital model of the aircraft's weight\linebreak
\\[-12pt]
\hspace*{23pt}passport&4&\hphantom{1}3--10\\
\Avtors{Yakimchuk~A.\,V.} see Burlutskiy~V.\,V.&&\\
\Avtors{Zabezhailo~M.\,I.} see Grusho~A.\,A.&&\\
\Avtors{Zabezhailo~M.\,I.} see Grusho~A.\,A.&&\\
\Avtors{Zakharova~T.\,V.\ and Tarkhov~A.\,A.} Evaluation of the significance level in schuirmann's test for\linebreak
\\[-12pt]
\hspace*{23pt}checking the~bioequivalence hypothesis in~missing data conditions&3&58--62\\
\end{tabular}
}
\pagebreak

\def\leftfootline{\small{\textbf{\thepage}
\hfill INFORMATIKA I EE PRIMENENIYA~--- INFORMATICS AND APPLICATIONS\ \ \ 2019\
\ \ volume~13\ \ \ issue\ 4}
}%
 \def\rightfootline{\small{INFORMATIKA I EE PRIMENENIYA~---
INFORMATICS AND APPLICATIONS\ \ \ 2019\ \ \ volume~13\ \ \ issue\ 4
\hfill \textbf{\thepage}}}

\def\leftkol{2019 AUTHOR INDEX} % ENGLISH ABSTRACTS}

\def\rightkol{2019 AUTHOR INDEX} %ENGLISH ABSTRACTS}


\noindent
{\tabcolsep=3pt
\begin{tabular}{p{395.48108pt}cc}
&\textbf{Issue} & \textbf{Page}\\[6pt]
\Avtors{Zatsarinny~A.\,A., Korotkov~V.\,V., and Matveev~M.\,G.} Modeling the process of network planning\linebreak
\\[-12pt]
\hspace*{23pt}of~a~portfolio of~projects with heterogeneous resources under fuzziness&2&92--99\\
\Avtors{Zatsman~I.\,M.} Digital encoding of~concepts&4&\hphantom{1}97--106\\
\Avtors{Zatsman~I.\,M.} Goal-oriented development of~linguistic knowledge systems: Identifying and\linebreak
\\[-12pt]
\hspace*{23pt}filling of~lacunae&1&91--98\\
\Avtors{Zatsman~I.\,M.} Third-order interfaces in informatics&3&82--89\\
\Avtors{Zatsman~I.\,M.} see Goncharov~A.\,A.&&\\
\Avtors{Zeifman~A.\,I., Satin~Y.\,A., and Kiseleva~K.\,M.} On the bounds of the rate of convergence for\linebreak
\\[-12pt]
\hspace*{23pt}some queueing models with incompletely defined intensities&3&14--19\\
\Avtors{Zhuravlev~Yu.\,I., Sen'ko~O.\,V., Bondarenko~N.\,N., Ryazanov~V.\,V., Dokukin~A.\,A., and Vinogradov~A.\,P.} Research of~the~possibility to~forecast changes in~financial state of~a~credit\linebreak
\\[-12pt]
\hspace*{23pt}organization on~the~basis of~public financial statements&4&30--35\\
\end{tabular}
}

%\thispagestyle{myheadings}
\def\leftfootline{\small{\textbf{\thepage}
\hfill INFORMATIKA I EE PRIMENENIYA~--- INFORMATICS AND APPLICATIONS\ \ \ 2019\
\ \ volume~13\ \ \ issue\ 4}
}%
 \def\rightfootline{\small{INFORMATIKA I EE PRIMENENIYA~---
INFORMATICS AND APPLICATIONS\ \ \ 2019\ \ \ volume~13\ \ \ issue\ 4
\hfill \textbf{\thepage}}}

 \label{end\stat}

\newpage

   \vspace*{-48pt}

\begin{center}
\vspace*{6pt}
\mbox{%
%\epsfxsize=50mm %56.519mm  
%\epsfbox{smu-1.eps} 

\epsfxsize=50mm %46.402 mm
\epsfbox{nec-rb.eps}
}
%\end{center}

\vspace*{9pt} %Академик


%   \begin{center}
\fbox{\large\textbf{Рустем Бадриевич Сейфуль-Мулюков}}\\[6pt]
\textbf{\large 1928--2020}
   \end{center}


   %\vspace*{2.5mm}

   \vspace*{5mm}

   \thispagestyle{empty}

%\

%\vspace*{-12pt}

  
      Редакция журнала <<Информатика и~её применения>> с глубоким 
      прискорбием сообщают, что 17~марта 2020~г.\ на 93-м~году жизни 
      скончался заведующий редакцией журнала, главный научный сотрудник Федерального исследовательского центра <<Информатика и~управление>> Российской академии наук
      Рустем Бадриевич Сейфуль-Мулюков.
           
     Всю свою жизнь Рустем Бадриевич посвятил служению науке. Закончив в~1956~г.\ аспирантуру Московского ордена Трудового Красного знамени Нефтяного института им.\ академика
     И.\,М.~Губкина, он прошел путь от заведующего отделом Института геологии зарубежных стран Министерства геологии СССР до заместителя директора ВИНИТИ
     АН СССР, доктора гео\-ло\-го-ми\-не\-ра\-ло\-ги\-че\-ских наук, профессора.
     
     С марта 2002~г.\ Рустем Бадриевич успешно применял свои знания и~организационный талант в ИПИ
     РАН (в~дальнейшем~--- ФИЦ ИУ РАН), в~котором руководил лабораторией и~отделом, занимающимися вопросами технологий информационной технической деятельности. 
Р.\,Б.~Сейфуль-Мулюков, являясь автором значительного количества научных трудов и~монографий по геологии, информационным технологиям и~теоретической информатике, осуществлял организацию издания монографий ИПИ РАН и~ФИЦ ИУ РАН, библиографий научных сотрудников Центра.
     
     Р.\,Б.~Сейфуль-Мулюков являлся заведующим редакцией журналов <<Информатика и~её применения>> и~<<Системы и~средства информатики>>, членом редколлегии журнала <<Системы и~средства информатики>>. Он вложил огромный вклад в становление и~развитие этих журналов, организацию их регистрации, функционирования, редактуры и~издания. Включение этих журналов в ряд отечественных и~зарубежных информационных баз и~систем цитирования во многом является его личной заслугой.
     
     На всех занимаемых должностях Рустем Бадриевич отличался высоким профессионализмом, преданностью делу и~вниманием к коллегам.
     
     \thispagestyle{empty}
     
     Рустема Бадриевича отличали доброта, отзывчивость, неиссякаемый
      оптимизм, простота и~сердечность.
     
     Коллеги Рустема Бадриевича запомнят его как многогранного в~своих увлечениях человека, живописца,
     эрудита и~энциклопедиста, интересующегося историей, литературой и~искусством.
     
     Выражаем глубокое
     соболезнование семье, родственникам, друзьям и~коллегам по работе в~связи с~тяжелой невосполнимой утратой.
     Светлый образ Рустема Бадриевича навсегда сохранится в~нашей памяти.
     

      

%\def\stat{cont}
{%\hrule\par
%\vskip 7pt % 7pt
\raggedleft\Large \bf%\baselineskip=3.2ex
А\,В\,Т\,О\,Р\,С\,К\,И\,Й\ \ У\,К\,А\,З\,А\,Т\,Е\,Л\,Ь\ \ З\,А\ \ 2\,0\,1\,0 г. \vskip 17pt
    \hrule
    \par
\vskip 21pt plus 6pt minus 3pt }

\label{st\stat}

\def\tit{\ }

\def\aut{\ }
\def\auf{\ }

\def\leftkol{\ } % ENGLISH ABSTRACTS}

\def\rightkol{\ } %АВТОРСКИЙ УКАЗАТЕЛЬ ЗА 2010 г.} %ENGLISH ABSTRACTS}

\titele{\tit}{\aut}{\auf}{\leftkol}{\rightkol}

\vspace*{-12pt}

{\tabcolsep=3pt
\begin{tabular}{p{388pt}rr}
&\textbf{Выпуск} & \textbf{Стр.}\\[6pt]
\hangindent=23pt\noindent\textbf{Арутюнян~А.\,Р.} Моделирование влияния деформаций отпечатков пальцев на 
точность\linebreak
\vspace*{-12pt}\\
\hspace*{23pt}дактилоскопической идентификации$\dotfill$&1&51\\
\hangindent=23pt\noindent\textbf{Архипов~О.\,П., Зыкова~З.\,П.} Интеграция гетерогенной информации о цветных 
пикселях\linebreak
\vspace*{-12pt}\\
\hspace*{23pt}и их цветовосприятии$\dotfill$&4&15\\
\hangindent=23pt\noindent\textbf{Баранов~С.\,И., Френкель~С.\,Л., Захаров~В.\,Н.} Полуформальная верификация 
цифрового устройства с конвейером, основанная на использовании алгоритмических машин\linebreak
\vspace*{-12pt}\\
\hspace*{23pt}состояния$\dotfill$&4&49\\
\textbf{Бекетова~И.\,В.} см.~Каратеев~С.\,Л.&&\\
\textbf{Белоусов~В.\,В.} см.~Синицын~И.\,Н.&&\\
\hangindent=23pt\noindent\textbf{Бенинг~В.\,Е., Королев~Р.\,А.} О предельном поведении мощностей критериев в 
случае\linebreak
\vspace*{-12pt}\\
\hspace*{23pt}распределения Лапласа$\dotfill$&2&63\\
\hangindent=23pt\noindent\textbf{Бенинг~В.\,Е., Сипина~А.\,В.} Асимптотическое разложение для мощности 
критерия,\linebreak
\vspace*{-12pt}\\
\hspace*{23pt}основанного на выборочной медиане, в случае распределения Лапласа$\dotfill$&1&18\\
\textbf{Бондаренко~А.\,В.} см.~Каратеев~С.\,Л.&&\\
\hangindent=23pt\noindent\textbf{Бородина~А.\,В., Морозов~Е.\,В.} Об оценивании асимптотики вероятности 
большого\linebreak
\vspace*{-12pt}\\
\hspace*{23pt}уклонения стационарной регенеративной очереди с одним прибором$\dotfill$&3&29\\
\hangindent=23pt\noindent\textbf{Бунтман~Н.\,В., Минель~Ж.-Л., Ле~Пезан~Д., Зацман~И.\,М.} Типология и 
компьютерное\linebreak
\vspace*{-12pt}\\
\hspace*{23pt}моделирование трудностей перевода$\dotfill$&3&77\\
\textbf{Визильтер~Ю.\,В.} см.~Каратеев~С.\,Л.&&\\
\hangindent=23pt\noindent\textbf{Гавриленко~С.\,В.} Оценки скорости сходимости распределений случайных сумм с 
безгранично делимыми индексами к нормальному закону$\dotfill$&4&81\\
\hangindent=23pt\noindent\textbf{Григорьева~М.\,Е., Шевцова~И.\,Г.} Уточнение неравенства 
Каца--Берри--Эссеена$\dotfill$&2&75\\
\hangindent=23pt\noindent\textbf{Грушо~А.\,А., Грушо~Н.\,А., Тимонина~Е.\,Е.} Поиск конфликтов в политиках 
безопасности: модель случайных графов$\dotfill$&3&38\\
\textbf{Грушо~Н.\,А.} см.~Грушо~А.\,А.&&\\
\hangindent=23pt\noindent\textbf{Гудков~В.\,Ю.} Математические модели изображения отпечатка пальца на основе 
описания линий$\dotfill$&1&58\\
\textbf{Гуртов~А.\,В.} см.~Лукьяненко~А.\,С.&&\\
\textbf{Желтов~С.\,Ю.} см.~Каратеев~С.\,Л.&&\\
\hangindent=23pt\noindent\textbf{Захаров~А.\,А., Серебряков~В.\,А.} Система управления электронной библиотекой 
LibMeta$\dotfill$&4&2\\
\textbf{Захаров~В.\,Н.} см.~Баранов~С.\,И.&&\\
\textbf{Захарова~Т.\,В.} см.~Матвеева~С.\,С.&&\\
\hangindent=23pt\noindent\textbf{Зацаринный~А.\,А., Чупраков~К.\,Г.} Некоторые аспекты выбора технологии для 
постро-\linebreak
\vspace*{-12pt}\\
\hspace*{23pt}ения систем отображения информации ситуационного центра$\dotfill$&3&59\\
\textbf{Зацман~И.\,М.} см.~Бунтман~Н.\,В.&&\\
\hangindent=23pt\noindent\textbf{Зейфман~А.\,И., Коротышева~А.\,В., Сатин~Я.\,А., Шоргин~С.\,Я.} Об 
устойчивости нестаци-\linebreak
\vspace*{-12pt}\\
\hspace*{23pt}онарных систем обслуживания с катастрофами$\dotfill$&3&9\\
\textbf{Зыкова~З.\,П.} см.~Архипов~О.\,П.&&\\
\hangindent=23pt\noindent\textbf{Илюшин~Г.\,Я., Соколов~И.\,А.} Организация управляемого доступа пользователей 
к\linebreak
\vspace*{-12pt}\\
\hspace*{23pt}разнородным ведомственным информационным ресурсам$\dotfill$&1&24\\
\hangindent=23pt\noindent\textbf{Кавагучи~Ю., Ульянов~В.\,В., Фуджикоши~Я.} Приближения для статистик, 
описывающих\linebreak
\vspace*{-12pt}\\
\hspace*{23pt}геометрические свойства данных большой размерности, с оценками 
ошибок$\dotfill$&1&12\\
\hangindent=23pt\noindent\textbf{Каратеев~С.\,Л., Бекетова~И.\,В., Ососков~М.\,В., Князь~В.\,А., 
Визильтер~Ю.\,В., Бондаренко~А.\,В., Желтов~С.\,Ю.} Автоматизированный контроль 
качества цифровых\linebreak
\vspace*{-12pt}\\
\hspace*{23pt}изображений для персональных документов$\dotfill$&1&65\\
\end{tabular}
}

\pagebreak

\def\leftkol{АВТОРСКИЙ УКАЗАТЕЛЬ ЗА 2010 г.} % ENGLISH ABSTRACTS}

\def\rightkol{АВТОРСКИЙ УКАЗАТЕЛЬ ЗА 2010 г.} %ENGLISH ABSTRACTS}

{\tabcolsep=3pt
\begin{tabular}{p{388pt}rr}
&\textbf{Выпуск} & \textbf{Стр.}\\[3pt]
\hangindent=23pt\noindent\textbf{Козеренко~Е.\,Б.} Лингвистические фильтры в статистических моделях машинного\linebreak
\vspace*{-12pt}\\
\hspace*{23pt}перевода$\dotfill$&2&83\\
\hangindent=23pt\noindent\textbf{Козеренко~Е.\,Б., Кузнецов~И.\,П.} Когнитивно-лингвистические представления в 
систе-\linebreak
\vspace*{-12pt}\\
\hspace*{23pt}мах обработки текстов$\dotfill$&3&69\\
\textbf{Князь~В.\,А.} см.~Каратеев~С.\,Л.&&\\
\hangindent=23pt\noindent\textbf{Колесников~А.\,В., Солдатов~С.\,А.} Алгоритм координации для гибридной 
интеллектуальной системы решения сложной задачи оперативно-производственного\linebreak
\vspace*{-12pt}\\
\hspace*{23pt}планирования$\dotfill$&4&61\\
\hangindent=23pt\noindent\textbf{Коновалов~М.\,Г.} О планировании потоков в системах вычислительных 
ресурсов$\dotfill$&2&3\\
\textbf{Конушин~А.\,С.} см.~Конушин~В.\,С.&&\\
\hangindent=23pt\noindent\textbf{Конушин~В.\,С., Кривовязь~Г.\,Р., Конушин~А.\,С.} Алгоритм распознавания людей 
в видео-\linebreak
\vspace*{-12pt}\\
\hspace*{23pt}последовательности по одежде$\dotfill$&1&74\\
\textbf{Корепанов~Э.\, Р.} см.~Синицын~И.\,Н.&&\\
\textbf{Королев~В.\,Ю.} см.~Соколов~И.\,А.&&\\
\textbf{Королев~Р.\,А.} см.~Бенинг~В.\,Е.&&\\
\textbf{Коротышева~А.\,В.} см.~Зейфман~А.\,И.&&\\
\hangindent=23pt\noindent\textbf{Кривенко~М.\,П.} Непараметрическое оценивание элементов байесовского 
клас\-си-\linebreak
\vspace*{-12pt}\\
\hspace*{23pt}фикатора$\dotfill$&2&13\\
\textbf{Кривовязь~Г.\,Р.} см.~Конушин~В.\,С.&&\\
\textbf{Крылов~А.\,С.} см.~Павельева~Е.\,А.&&\\
\hangindent=23pt\noindent\textbf{Крылов~В.\,А.} Моделирование и классификация многоканальных дистанционных\linebreak
\vspace*{-12pt}\\
\hspace*{23pt}изображений с использованием копул$\dotfill$&4&34\\
\hangindent=23pt\noindent\textbf{Крючин~О.\,В.} Разработка параллельных эвристических алгоритмов подбора 
весовых\linebreak
\vspace*{-12pt}\\
\hspace*{23pt}коэффициентов искусственной нейтронной сети$\dotfill$&2&53\\
\hangindent=23pt\noindent\textbf{Кудрявцев~А.\,А., Шоргин~С.\,Я.} Байесовские модели массового обслуживания и 
надеж-\linebreak
\vspace*{-12pt}\\
\hspace*{23pt}ности: характеристики среднего числа заявок в системе $M\vert M \vert 1\vert 
\infty$$\dotfill$&3&16\\
\hangindent=23pt\noindent\textbf{Кузнецов~А.\,А.} Связь между временными и структурно-топологическими 
характери-\linebreak
\vspace*{-12pt}\\
\hspace*{23pt}стиками диаграмм ритма сердца здоровых людей$\dotfill$&4&39\\
\textbf{Кузнецов~И.\,П.} см.~Козеренко~Е.\,Б.&&\\
\textbf{Ле~Пезан~Д.} см.~Бунтман~Н.\,В.&&\\
\hangindent=23pt\noindent\textbf{Лукьяненко~А.\,С., Морозов~Е.\,В., Гуртов~А.\,В.} Анализ сетевого протокола с общей 
функ-\linebreak
\vspace*{-12pt}\\
\hspace*{23pt}цией расширения окна передачи сообщения при конфликтах$\dotfill$&2&46\\
\hangindent=23pt\noindent\textbf{Лямин~О.\,О.} О предельном поведении мощностей критериев в случае обобщенного\linebreak
\vspace*{-12pt}\\
\hspace*{23pt}распределения Лапласа$\dotfill$&3&47\\
\hangindent=23pt\noindent\textbf{Маркин~А.\,В., Шестаков~О.\,В.} Асимптотики оценки риска при пороговой 
обработке\linebreak
\vspace*{-12pt}\\
\hspace*{23pt}вейвлет-вейглет коэффициентов в задаче томографии$\dotfill$&2&36\\
\hangindent=23pt\noindent\textbf{Матвеева~С.\,С., Захарова~Т.\,В.} Сети массового обслуживания с наименьшей 
длиной\linebreak
\vspace*{-12pt}\\
\hspace*{23pt}очереди$\dotfill$&3&22\\
\hangindent=23pt\noindent\textbf{Матюшенко~С.\,И.} Стационарные характеристики двухканальной системы 
обслужива-\linebreak
\vspace*{-12pt}\\
\hspace*{23pt}ния с переупорядочиванием заявок и распределениями фазового типа$\dotfill$&4&68\\
\textbf{Минель~Ж.-Л.} см.~Бунтман~Н.\,В.&&\\
\textbf{Морозов~Е.\,В.} см.~Бородина~А.\,В.&&\\
\textbf{Морозов~Е.\,В.} см.~Лукьяненко~А.\,С.&&\\
\textbf{Ососков~М.\,В.} см.~Каратеев~С.\,Л.&&\\
\hangindent=23pt\noindent\textbf{Павельева~Е.\,А., Крылов~А.\,С.} Поиск и анализ ключевых точек радужной 
оболочки\linebreak
\vspace*{-12pt}\\
\hspace*{23pt}глаза методом преобразования Эрмита$\dotfill$&1&79\\
\textbf{Печинкин~А.\,В.} см.~Френкель~С.\,Л.,&&\\
\hangindent=23pt\noindent\textbf{Протасов~В.\,И.} Составление субъективного портрета с использованием 
эволюционно-\linebreak
\vspace*{-12pt}\\
\hspace*{23pt}го морфинга и квалиметрия метода$\dotfill$&1&83\\
\hangindent=23pt\noindent\textbf{Рудаков~К.\,В., Торшин~И.\,Ю.} Вопросы разрешимости задачи распознавания 
вторичной\linebreak
\vspace*{-12pt}\\
\hspace*{23pt}структуры белка$\dotfill$&2&25\\
\textbf{Сатин~Я.\,А.} см.~Зейфман~А.\,И.&&\\
\hangindent=23pt\noindent\textbf{Сейфуль-Мулюков~Р.\,Б.} Нефть как носитель информации о своем 
происхождении,\linebreak
\vspace*{-12pt}\\
\hspace*{23pt}структуре и эволюции$\dotfill$&1&41\\
\end{tabular}
}

{\tabcolsep=3pt
\begin{tabular}{p{388pt}rr}
&\textbf{Выпуск} & \textbf{Стр.}\\[6pt]
\textbf{Семендяев~Н.\,Н.} см.~Синицын~И.\,Н.&&\\
\textbf{Серебряков~В.\,А.} см.~Захаров~А.\,А.&&\\
\textbf{Синицын~В.\,И.} см.~Синицын~И.\,Н.&&\\
\hangindent=23pt\noindent\textbf{Синицын~И.\,Н., Синицын~В.\,И., Корепанов~Э.\, Р., Белоусов~В.\,В., 
Семендяев~Н.\,Н.} Оперативное построение информационных моделей движения полюса 
Земли\linebreak
\vspace*{-12pt}\\
\hspace*{23pt}методами линейных и линеаризованных фильтров$\dotfill$&1&2\\
\textbf{Сипина~А.\,В.} см.~Бенинг~В.\,Е.&&\\
\hangindent=23pt\noindent\textbf{Соколов~И.\,А.} О работах заслуженного деятеля науки Российской Федерации 
И.\,Н.~Синицына в области информационных технологий и автоматизации (к 70-летию\linebreak
\vspace*{-12pt}\\
\hspace*{23pt}со дня рождения)$\dotfill$&3&84\\
\textbf{Соколов~И.\,А.} см.~Илюшин~Г.\,Я.&&\\
\hangindent=23pt\noindent\textbf{Соколов~И.\,А., Королев~В.\,Ю.} Предисловие$\dotfill$&2&2\\
\textbf{Солдатов~С.\,А.} см.~Колесников~А.\,В.&&\\
\hangindent=23pt\noindent\textbf{Степанов~С.\,Ю.} Использование координатного метода фрагментации 
коммутаторной\linebreak
\vspace*{-12pt}\\
\hspace*{23pt}нейронной сети для сокращения трафика$\dotfill$&2&57\\
\textbf{Тимонина~Е.\,Е.} см.~Грушо~А.\,А.&&\\
\textbf{Торшин~И.\,Ю.} см.~Рудаков~К.\,В.&&\\
\textbf{Ульянов~В.\,В.} см.~Кавагучи~Ю.&&\\
\textbf{Фазекаш~И.} см.~Чупрунов~А.\,Н.&&\\
\textbf{Френкель~С.\,Л.} см.~Баранов~С.\,И.&&\\
\hangindent=23pt\noindent\textbf{Френкель~С.\,Л., Печинкин~А.\,В.} Оценка времени самовосстановления в 
цифровых\linebreak
\vspace*{-12pt}\\
\hspace*{23pt}системах после сбоев, вызываемых переходными помехами$\dotfill$&3&2\\
\textbf{Фуджикоши~Я.} см.~Кавагучи~Ю.&&\\
\hangindent=23pt\noindent\textbf{Цискаридзе~А.\,К.} Математическая модель и метод восстановления позы человека 
по\linebreak
\vspace*{-12pt}\\
\hspace*{23pt}стереопаре силуэтных изображений$\dotfill$&4&27\\
\hangindent=23pt\noindent\textbf{Чупраков~К.\,Г.} К вопросу о размещении коллективных средств отображения в 
ситуа-\linebreak
\vspace*{-12pt}\\
\hspace*{23pt}ционном зале с заданными параметрами$\dotfill$&4&89\\
\textbf{Чупраков~К.\,Г.} см.~Зацаринный~А.\,А.&&\\
\hangindent=23pt\noindent\textbf{Чупрунов~А.\,Н., Фазекаш~И.} Законы повторного логарифма для числа 
безошибочных\linebreak
\vspace*{-12pt}\\
\hspace*{23pt}блоков при помехоустойчивом кодировании$\dotfill$&3&42\\
\textbf{Шевцова~И.\,Г.} см.~Григорьева~М.\,Е.&&\\
\hangindent=23pt\noindent\textbf{Шестаков~О.\,В.} Аппроксимация распределения оценки риска пороговой 
обработки вейвлет-коэффициентов нормальным распределением при использовании 
выбо-\linebreak
\vspace*{-12pt}\\
\hspace*{23pt}рочной дисперсии$\dotfill$&4&73\\
\textbf{Шестаков~О.\,В.} см.~Маркин~А.\,В.&&\\
\textbf{Шоргин~С.\,Я.} см.~Зейфман~А.\,И.&&\\
\textbf{Шоргин~С.\,Я.} см.~Кудрявцев~А.\,А.&&\\
\end{tabular}
}

%\thispagestyle{myheadings}
\def\leftfootline{\small{\textbf{\thepage}
\hfill ИНФОРМАТИКА И ЕЁ ПРИМЕНЕНИЯ\ \ \ том~4\ \ \ выпуск~4\ \ \ 2010}
}%
 \def\rightfootline{\small{ИНФОРМАТИКА И ЕЁ ПРИМЕНЕНИЯ\ \ \ том~4\ \ \ выпуск~4\ \ \ 2010
 \hfill \textbf{\thepage}}}
 \label{end\stat}
%
%Том 10 Выпуск 1-4 Год 2016

\def\stat{cont-e}
{%\hrule\par
%\vskip 7pt % 7pt
\raggedleft\Large \bf%\baselineskip=3.2ex
2\,0\,1\,6\ \ A\,U\,T\,H\,O\,R\ \ I\,N\,D\,E\,X \vskip 17pt
 \hrule
 \par
\vskip 21pt plus 6pt minus 3pt }

\label{st\stat}

\def\tit{\ }

\def\aut{\ }
\def\auf{\ }

\def\leftkol{\ } %2016 AUTHOR INDEX} % ENGLISH ABSTRACTS}

\def\rightkol{\ } %2016 AUTHOR INDEX} %ENGLISH ABSTRACTS}

\titele{\tit}{\aut}{\auf}{\leftkol}{\rightkol}

\def\leftfootline{\small{\textbf{\thepage}
\hfill INFORMATIKA I EE PRIMENENIYA~--- INFORMATICS AND APPLICATIONS\ \ \ 2016\
\ \ volume~10\ \ \ issue\ 4}
}%
 \def\rightfootline{\small{INFORMATIKA I EE PRIMENENIYA~--- INFORMATICS AND APPLICATIONS\ \ \ 2016\ \ \ volume~10\ \ \ issue\ 4
\hfill \textbf{\thepage}}}

\vspace*{-12pt}
\vspace*{-18pt}

{\tabcolsep=2.8pt
\begin{tabular}{p{382pt}cc}
&\textbf{Issue} & \textbf{Page}\\[6pt]
\Avtors{Agalarov~M.\,Ya.} see~Agalarov~Ya.\,M.&&\\
\Avtors{Agalarov~Ya.\,M., Agalarov~M.\,Ya., and
Shorgin~V.\,S.} About the optimal threshold of queue\linebreak
\\[-12pt]
\hspace*{23pt}length in a~particular problem of profit maximization
in the $M/G/1$ queuing system&2&70--79\\
\Avtors{Alexeyevsky~D.\,A.} BioNLP ontology extraction from 
a~restricted language corpus with\linebreak
\\[-12pt]
\hspace*{23pt}context-free grammars&1&119--128\\
\Avtors{Andreev~S.\,D.} see~Gaidamaka~Yu.\,V.&&\\
\Avtors{Andreev~S.\,D.} see~Ometov~A.\,Ya.&&\\
\Avtors{Arkhipov~O.\,P., Arkhipov~P.\,O., and Sidorkin~I.\,I.} The
option to create a~local coordinate\linebreak
\\[-12pt]
\hspace*{23pt}system for synchronization of selected images&3&91--97\\
\Avtors{Arkhipov~P.\,O.} see~Arkhipov~O.\,P.&&\\
\Avtors{Belousov~V.\,V.} see~Shnurkov~P.\,V.&&\\
\Avtors{Belousov~V.\,V.} see~Shnurkov~P.\,V.&&\\
\Avtors{Bening~V.\,E.} Calculation of~the~asymptotic deficiency
of~some statistical procedures based\linebreak
\\[-12pt]
\hspace*{23pt}on~samples with~random sizes&4&34--45\\
\Avtors{Borisov~A.\,V., Bosov~A.\,V., and Miller~G.\,B.} Modeling and
monitoring of VoIP connection&2&\hphantom{1}2--13\\
\Avtors{Bosov~A.\,V.} see~Borisov~A.\,V.&&\\
\Avtors{Briukhov~D.\,O.} see~Stupnikov~S.\,A.&&\\
\Avtors{Callaos~N.\,K.\ and Seyful-Mulyukov~R.\,B.} Complexity and
its information content&1&129--139\\
\Avtors{Chertok~A.\,V., Kadaner~A.\,I., Khazeeva~G.\,T., and
Sokolov~I.\,A.} Regime switching detection\linebreak
\\[-12pt]
\hspace*{23pt}for~the~Levy driven
Ornstein--Uhlenbeck process using CUSUM methods&4&46--56\\
\Avtors{Chichagov~V.\,V.} Asymptotic expansions of mean absolute
error of uniformly minimum variance unbiased and maximum likelihood
estimators on the one-parameter exponential\linebreak
\\[-12pt]
\hspace*{23pt}family model of lattice distributions&3&66--76\\
\Avtors{Danishevsky~V.\,I.} see~Kolesnikov A.\,V.&&\\
\Avtors{Fazliev~A.\,Z.} see~Kalinichenko~L.\,A.&&\\
\Avtors{Fedoseev~A.\,A.} What is behind the concept of ``knowledge in
small packages''&3&105--110\\
\Avtors{Gaidamaka~Yu.\,V., Andreev~S.\,D., Sopin~E.\,S.,
Samouylov~K.\,E., and Shorgin~S.\,Ya.} Interference analysis
of~the~device-to-device communications model with~regard to~a~signal\linebreak
\\[-12pt]
\hspace*{23pt}propagation environment&4&\hphantom{1}2--10\\
\Avtors{Gasilov~A.\,V.} see~Yakovlev~O.\,A.&&\\
\Avtors{Goncharov~A.\,V.\ and Strijov~V.\,V.} Metric time series
classification using weighted dynamic\linebreak
\\[-12pt]
\hspace*{23pt}warping relative to centroids of classes&2&36--47\\
\Avtors{Gordov~E.\,P.} see~Kalinichenko~L.\,A.&&\\
\Avtors{Gorshenin~A.\,K.} Concept of online service for stochastic
modeling of real processes&1&72--81\\
\Avtors{Gorshenin~A.\,K.} see~Shnurkov~P.\,V.&&\\
\Avtors{Gorshenin~A.\,K.} see~Shnurkov~P.\,V.&&\\
\Avtors{Grusho~A.\,A., Grusho~N.\,A., Zabezhailo~M.\,I., and
Timonina~E.\,E.} Integration of statistical and\linebreak
\\[-12pt]
\hspace*{23pt}deterministic methods for
analysis of information security&3&2--8\\
\Avtors{Grusho~A.\,A., Zabezhailo~M.\,I., and Zatsarinny~A.\,A.} On
the advanced procedure to reduce\linebreak
\\[-12pt]
\hspace*{23pt}calculation of Galois closures&4&\hphantom{1}96--104\\
\Avtors{Grusho~N.\,A.} see~Grusho~A.\,A.&&\\
\Avtors{Havanskov~V.\,A.} see~Minin~V.\,A.&&\\
\Avtors{Inkova~O.\,Yu.} see~Zatsman~I.\,M.&&\\
\Avtors{Isachenko~R.\,V.\ and Strijov~V.\,V.} Metric learning in
multiclass time series classification\linebreak
\\[-12pt]
\hspace*{23pt}problem&2&48--57\\
\end{tabular}
}
\pagebreak

\def\leftfootline{\small{\textbf{\thepage}
\hfill INFORMATIKA I EE PRIMENENIYA~--- INFORMATICS AND APPLICATIONS\ \ \ 2016\
\ \ volume~10\ \ \ issue\ 4}
}%
 \def\rightfootline{\small{INFORMATIKA I EE PRIMENENIYA~---
INFORMATICS AND APPLICATIONS\ \ \ 2016\ \ \ volume~10\ \ \ issue\ 4
\hfill \textbf{\thepage}}}

\def\leftkol{2016 AUTHOR INDEX} % ENGLISH ABSTRACTS}

\def\rightkol{2016 AUTHOR INDEX} %ENGLISH ABSTRACTS}


{\tabcolsep=2.83pt
\begin{tabular}{p{382pt}cc}
&\textbf{Issue} & \textbf{Page}\\[6pt]
\Avtors{Kadaner~A.\,I.} see~Chertok~A.\,V.&&\\[.255pt]
\Avtors{Kalinichenko~L.\,A., Volnova~A.\,A., Gordov~E.\,P.,
Kiselyova~N.\,N., Kovaleva~D.\,A., Malkov~O.\,Yu., Okladnikov~I.\,G.,
Podkolodnyy~N.\,L., Pozanenko~A.\,S., Ponomareva~N.\,V.,
Stupnikov~S.\,A.,} \textbf{and Fazliev~A.\,Z.} Data access challenges for data
intensive\linebreak
\\[-12pt]
\hspace*{23pt}research in Russia&1& 2--22\\[.255pt]
\Avtors{Karasikov~M.\,E.\ and Strijov~V.\,V.} Feature-based
time-series classification&4&121--131\\[.255pt]
\Avtors{Khazeeva~G.\,T.} see~Chertok~A.\,V.&&\\[.255pt]
\Avtors{Khokhlov~Yu.\,S.} Multivariate fractional Levy motion and its
applications&2&\hphantom{1}98--106\\[.255pt]
\Avtors{Kirikov~I.\,A., Kolesnikov~A.\,V., Listopad~S.\,V., and
Rumovskaya~S.\,B.} Fine-grained hybrid\linebreak
\\[-12pt]
\hspace*{23pt}intelligent systems. Part 2:
Bidirectional hybridization&1&\hphantom{1}96--105\\[.255pt]
\Avtors{Kirikov~I.\,A., Kolesnikov~A.\,V., Listopad~S.\,V., and
Rumovskaya~S.\,B.} ``Virtual council''~---\linebreak
\\[-12pt]
\hspace*{23pt}source environment
supporting complex diagnostic decision making&3&81--90\\[.255pt]
\Avtors{Kiselyova~N.\,N.} see~Kalinichenko~L.\,A.&&\\[.255pt]
\Avtors{Kolesnikov A.\,V., Listopad~S.\,V., Rumovskaya~S.\,B., and
Danishevsky~V.\,I.} Informal axiomatic\linebreak
\\[-12pt]
\hspace*{23pt}theory of~the~role visual models&4&114--120\\[.255pt]
\Avtors{Kolesnikov~A.\,V.} see~Kirikov~I.\,A.&&\\[.255pt]
\Avtors{Kolesnikov~A.\,V.} see~Kirikov~I.\,A.&&\\[.255pt]
\Avtors{Kolin~K.\,K.} Humanitarian aspects of information
security&3&111--121\\[.255pt]
\Avtors{Konovalov~M.\,G.\ and Razumchik~R.\,V.} Dispatching
to~two parallel nonobservable queues using\linebreak
\\[-12pt]
\hspace*{23pt}only static
information&4&57--67\\[.255pt]
\Avtors{Korchagin~A.\,Yu.} see~Korolev~V.\,Yu.&&\\[.255pt]
\Avtors{Korchagin~A.\,Yu.} see~Korolev~V.\,Yu.&&\\[.255pt]
\Avtors{Korepanov~E.\,R.} see~Sinitsyn~I.\,N.&&\\[.255pt]
\Avtors{Korepanov~E.\,R.} see~Sinitsyn~I.\,N.&&\\[.255pt]
\Avtors{Korolev~V.\,Yu., Korchagin~A.\,Yu., and Zeifman~A.\,I.} The
Poisson theorem for Bernoulli trials\linebreak
\\[-12pt]
\hspace*{23pt}with~a~random probability
of~success and~a~discrete analog of~the~Weibull distribution&4&11--20\\[.255pt]
\Avtors{Korolev~V.\,Yu., Zeifman~A.\,I., and Korchagin~A.\,Yu.}
Asymmetric Linnik distributions as~limit\linebreak
\\[-12pt]
\hspace*{23pt}laws for~random sums
of~independent random variables with~finite variances&4&21--33\\[.255pt]
\Avtors{Koucheryavy~E.\,A.} see~Ometov~A.\,Ya.&&\\[.255pt]
\Avtors{Kovaleva~D.\,A.} see~Kalinichenko~L.\,A.&&\\[.255pt]
\Avtors{Kovalyov~S.\,P.} Metaprogramming to increase
manufacturability of large-scale software-\linebreak
\\[-12pt]
\hspace*{23pt}intensive systems&1&56--66\\[.255pt]
\Avtors{Krivenko~M.\,P.} Significance tests of feature selection for
classification&3&32--40\\[.255pt]
\Avtors{Kruzhkov~M.\,G.} see~Zalizniak~Anna~A.&&\\[.255pt]
\Avtors{Kruzhkov~M.\,G.} see~Zatsman~I.\,M.&&\\[.255pt]
\Avtors{Kudryavtsev~A.\,A.} Bayesian queueing and reliability models:
\textit{A~priori} distributions with\linebreak
\\[-12pt]
\hspace*{23pt}compact support&1&67--71\\[.255pt]
\Avtors{Kudryavtsev~A.\,A.} Characteristics dependent on the balance
coefficient in Bayesian models\linebreak
\\[-12pt]
\hspace*{23pt}with compact support of \textit{a priori}
distributions&3&77--80\\[.255pt]
\Avtors{Kudryavtsev~A.\,A.\ and Palionnaia~S.\,I.} Bayesian recurrent
model of reliability growth:\linebreak
\\[-12pt]
\hspace*{23pt}Parabolic distribution of parameters&2&80--83\\[.255pt]
\Avtors{Kudryavtsev~A.\,A.\ and Titova~A.\,I.} Bayesian queuing
and~reliability models: Degenerate-\linebreak
\\[-12pt]
\hspace*{23pt}Weibull case&4&68--71\\[.255pt]
\Avtors{Leontyev~N.\,D.\ and Ushakov~V.\,G.} Analysis of a queueing
system with autoregressive arrivals\linebreak
\\[-12pt]
\hspace*{23pt}and nonpreemptive priority&3&15--22\\[.255pt]
\Avtors{Listopad~S.\,V.} see~Kirikov~I.\,A.&&\\[.255pt]
\Avtors{Listopad~S.\,V.} see~Kirikov~I.\,A.&&\\[.255pt]
\Avtors{Listopad~S.\,V.} see~Kolesnikov A.\,V.&&\\[.255pt]
\Avtors{Malkov~O.\,Yu.} see~Kalinichenko~L.\,A.&&\\[.255pt]
\Avtors{Markov~A.\,S., Monakhov~M.\,M., and
Ulyanov~V.\,V.} Generalized Cornish--Fisher expansions\linebreak
\\[-12pt]
\hspace*{23pt}for distributions of statistics based on samples
of random size&2&84--91\\[.255pt]
\Avtors{Melnikov~A.\,K.\ and Ronzhin~A.\,F.} Generalized statistical
method of~text analysis based\linebreak
\\[-12pt]
\hspace*{23pt}on~calculation of~probability distributions
of~statistical values&4&89--95\\
\end{tabular}
}
\pagebreak

\def\leftfootline{\small{\textbf{\thepage}
\hfill INFORMATIKA I EE PRIMENENIYA~--- INFORMATICS AND APPLICATIONS\ \ \ 2016\
\ \ volume~10\ \ \ issue\ 4}
}%
 \def\rightfootline{\small{INFORMATIKA I EE PRIMENENIYA~---
INFORMATICS AND APPLICATIONS\ \ \ 2016\ \ \ volume~10\ \ \ issue\ 4
\hfill \textbf{\thepage}}}

\def\leftkol{2016 AUTHOR INDEX} % ENGLISH ABSTRACTS}

\def\rightkol{2016 AUTHOR INDEX} %ENGLISH ABSTRACTS}


{\tabcolsep=3pt
\begin{tabular}{p{381pt}cc}
&\textbf{Issue} & \textbf{Page}\\[6pt]
\Avtors{Meykhanadzhyan~L.\,A.} Stationary characteristics of the finite
capacity queueing system with\linebreak
\\[-12pt]
\hspace*{23pt}inverse service order and generalized
probabilistic priority&2&123--131\\[.23pt]
\Avtors{Miller~G.\,B.} see~Borisov~A.\,V.&&\\[.23pt]
\Avtors{Minin~V.\,A., Zatsman~I.\,M., Havanskov~V.\,A., and
Shubnikov~S.\,K.} Intensity of citation of scientific publications in
inventions on information and computer technologies patented\linebreak
\\[-12pt]
\hspace*{23pt}in Russia by domestic and foreign applicants&2&107--122\\[.23pt]
\Avtors{Monakhov~M.\,M.} see~Markov~A.\,S.&&\\[.23pt]
\Avtors{Naumov~V.\,A.\ and Samouylov~K.\,E.} On relationship
between queuing systems with resources\linebreak
\\[-12pt]
\hspace*{23pt}and Erlang networks&3&\hphantom{1}9--14\\[.23pt]
\Avtors{Okladnikov~I.\,G.} see~Kalinichenko~L.\,A.&&\\[.23pt]
\Avtors{Ometov~A.\,Ya., Andreev~S.\,D., Turlikov~A.\,M., and
Koucheryavy~E.\,A.} Performance analysis of\linebreak
\\[-12pt]
\hspace*{23pt}a wireless data
aggregation system with contention for contemporary sensor
networks&3&23--31\\[.23pt]
\Avtors{Palionnaia~S.\,I.} see~Kudryavtsev~A.\,A.&&\\[.23pt]
\Avtors{Podkolodnyy~N.\,L.} see~Kalinichenko~L.\,A.&&\\[.23pt]
\Avtors{Ponomareva~N.\,V.} see~Kalinichenko~L.\,A.&&\\[.23pt]
\Avtors{Popkova~N.\,A.} see~Zatsman~I.\,M.&&\\[.23pt]
\Avtors{Pozanenko~A.\,S.} see~Kalinichenko~L.\,A.&&\\[.23pt]
\Avtors{Razumchik~R.\,V.} see~Konovalov~M.\,G.&&\\[.23pt]
\Avtors{Ronzhin~A.\,F.} see~Melnikov~A.\,K.&&\\[.23pt]
\Avtors{Rumovskaya~S.\,B.} see~Kirikov~I.\,A.&&\\[.23pt]
\Avtors{Rumovskaya~S.\,B.} see~Kirikov~I.\,A.&&\\[.23pt]
\Avtors{Rumovskaya~S.\,B.} see~Kolesnikov A.\,V.&&\\[.23pt]
\Avtors{Samouylov~K.\,E.} see~Gaidamaka~Yu.\,V.&&\\[.23pt]
\Avtors{Samouylov~K.\,E.} see~Naumov~V.\,A.&&\\[.23pt]
\Avtors{Serebryanskii~S.\,M.} see~Tyrsin~A.\,N.&&\\[.23pt]
\Avtors{Seyful-Mulyukov~R.\,B.} see~Callaos~N.\,K.&&\\[.23pt]
\Avtors{Shestakov~O.\,V.} Statistical properties of the denoising method
based on the stabilized hard\linebreak
\\[-12pt]
\hspace*{23pt}thresholding&2&65--69\\[.23pt]
\Avtors{Shestakov~O.\,V.} The strong law of large numbers for the risk
estimate in the problem of\linebreak
\\[-12pt]
\hspace*{23pt}tomographic image reconstruction from
projections with a correlated noise&3&41--45\\[.23pt]
\Avtors{Shestakov~O.\,V.} see~Zakharova~T.\,V.&&\\[.23pt]
\Avtors{Shnurkov~P.\,V., Gorshenin~A.\,K., and Belousov~V.\,V.}
Analytical solution of~the~optimal control\linebreak
\\[-12pt]
\hspace*{23pt}task of~a~semi-Markov
process with~finite set of~states&4&72--88\\[.23pt]
\Avtors{Shnurkov~P.\,V., Zasypko~V.\,V., Belousov~V.\,V., and
Gorshenin~A.\,K.} Development of the algorithm of numerical solution
of the optimal investment control problem\linebreak
\\[-12pt]
\hspace*{23pt}in the closed dynamical model of three-sector economy&1&82--95\\[.23pt]
\Avtors{Shorgin~S.\,Ya.} see~Gaidamaka~Yu.\,V.&&\\[.23pt]
\Avtors{Shorgin~V.\,S.} see~Agalarov~Ya.\,M.&&\\[.23pt]
\Avtors{Shubnikov~S.\,K.} see~Minin~V.\,A.&&\\[.23pt]
\Avtors{Sidorkin~I.\,I.} see~Arkhipov~O.\,P.&&\\[.23pt]
\Avtors{Sinitsyn~I.\,N.} Analytical modeling of processes in stochastic
systems with complex fractional\linebreak
\\[-12pt]
\hspace*{23pt}order Bessel nonlinearities&3&55--65\\[.23pt]
\Avtors{Sinitsyn~I.\,N.} Orthogonal supoptimal filters for nonlinear
stochastic systems on manifolds&1&34--44\\[.23pt]
\Avtors{Sinitsyn~I.\,N.\ and Korepanov~E.\,R.} Normal Pugachev
conditionally-optimal filters and extra-\linebreak
\\[-12pt]
\hspace*{23pt}polators for state linear stochastic systems&2&14--23\\[.23pt]
\Avtors{Sinitsyn~I.\,N.\ and Sinitsyn~V.\,I.} Analytical modeling of
distributions in stochastic systems on\linebreak
\\[-12pt]
\hspace*{23pt}manifolds based on ellipsoidal approximation&1&45--55\\[.23pt]
\Avtors{Sinitsyn~I.\,N., Sinitsyn~V.\,I., and
Korepanov~E.\,R.} Ellipsoidal suboptimal filters for nonlinear\linebreak
\\[-12pt]
\hspace*{23pt}stochastic systems on manifolds&2&24--35\\[.23pt]
\Avtors{Sinitsyn~V.\,I.} see~Sinitsyn~I.\,N.&&\\[.23pt]
\Avtors{Sinitsyn~V.\,I.} see~Sinitsyn~I.\,N.&&\\[.23pt]
\Avtors{Skvortsov~N.\,A.} see~Stupnikov~S.\,A.&&\\[.23pt]
\Avtors{Sokolov~I.\,A.} see~Chertok~A.\,V.&&\\
\end{tabular}
}
\pagebreak

\def\leftfootline{\small{\textbf{\thepage}
\hfill INFORMATIKA I EE PRIMENENIYA~--- INFORMATICS AND APPLICATIONS\ \ \ 2016\
\ \ volume~10\ \ \ issue\ 4}
}%
 \def\rightfootline{\small{INFORMATIKA I EE PRIMENENIYA~---
INFORMATICS AND APPLICATIONS\ \ \ 2016\ \ \ volume~10\ \ \ issue\ 4
\hfill \textbf{\thepage}}}

\def\leftkol{2016 AUTHOR INDEX} % ENGLISH ABSTRACTS}

\def\rightkol{2016 AUTHOR INDEX} %ENGLISH ABSTRACTS}


{\tabcolsep=3pt
\begin{tabular}{p{382pt}cc}
&\textbf{Issue} & \textbf{Page}\\[6pt]
\Avtors{Sopin~E.\,S.} see~Gaidamaka~Yu.\,V.&&\\
\Avtors{Strijov~V.\,V.} see~Goncharov~A.\,V.&&\\
\Avtors{Strijov~V.\,V.} see~Isachenko~R.\,V.&&\\
\Avtors{Strijov~V.\,V.} see~Karasikov~M.\,E.&&\\
\Avtors{Stupnikov~S.\,A., Briukhov~D.\,O., and Skvortsov~N.\,A.}
Co-lending systemic risk analysis over\linebreak
\\[-12pt]
\hspace*{23pt}heterogeneous data collections&1&23--33\\
\Avtors{Stupnikov~S.\,A.} see~Kalinichenko~L.\,A.&&\\
\Avtors{Suchkov~A.\,P.} see~Zatsarinny~A.\,A.&&\\
\Avtors{Timonina~E.\,E.} see~Grusho~A.\,A.&&\\
\Avtors{Titova~A.\,I.} see~Kudryavtsev~A.\,A.&&\\
\Avtors{Turlikov~A.\,M.} see~Ometov~A.\,Ya.&&\\
\Avtors{Tyrsin~A.\,N.\ and Serebryanskii~S.\,M.} Recognition of
dependences on the basis of inverse\linebreak
\\[-12pt]
\hspace*{23pt}mapping&2&58--64\\
\Avtors{Ulyanov~V.\,V.} see~Markov~A.\,S.&&\\
\Avtors{Ushakov~V.\,G.} Queueing system with working vacations and
hyperexponential input stream&2&92--97\\
\Avtors{Ushakov~V.\,G.} see~Leontyev~N.\,D.&&\\
\Avtors{Volnova~A.\,A.} see~Kalinichenko~L.\,A.&&\\
\Avtors{Yakovlev~O.\,A.\ and Gasilov~A.\,V.} Speeded-up stereo
matching using geodesic support weights&3&\hphantom{1}98--104\\
\Avtors{Zabezhailo~M.\,I.} see~Grusho~A.\,A.&&\\
\Avtors{Zabezhailo~M.\,I.} see~Grusho~A.\,A.&&\\
\Avtors{Zakharova~T.\,V.\ and Shestakov~O.\,V.} Precision analysis of
wavelet processing of aerodynamic\linebreak
\\[-12pt]
\hspace*{23pt}flow patterns&3&46--54\\
\Avtors{Zalizniak~Anna~A.\ and Kruzhkov~M.\,G.} Database
of~Russian impersonal verbal constructions&4&132--141\\
\Avtors{Zasypko~V.\,V.} see~Shnurkov~P.\,V.&&\\
\Avtors{Zatsarinny~A.\,A.\ and Suchkov~A.\,P.} Systems engineering
approaches to~the~establishment of\linebreak
\\[-12pt]
\hspace*{23pt}a~system for~decision support based
on~situational analysis&4&105--113\\
\Avtors{Zatsarinny~A.\,A.} see~Grusho~A.\,A.&&\\
\Avtors{Zatsman~I.\,M., Inkova~O.\,Yu., Kruzhkov~M.\,G., and
Popkova~N.\,A.} Representation of cross-\linebreak
\\[-12pt]
\hspace*{23pt}lingual knowledge about
connectors in supracorpora databases&1&106--118\\
\Avtors{Zatsman~I.\,M.} see~Minin~V.\,A.&&\\
\Avtors{Zeifman~A.\,I.} see~Korolev~V.\,Yu.&&\\
\Avtors{Zeifman~A.\,I.} see~Korolev~V.\,Yu.&&\\
\end{tabular}
}

%\thispagestyle{myheadings}
\def\leftfootline{\small{\textbf{\thepage}
\hfill INFORMATIKA I EE PRIMENENIYA~--- INFORMATICS AND APPLICATIONS\ \ \ 2016\
\ \ volume~10\ \ \ issue\ 4}
}%
 \def\rightfootline{\small{INFORMATIKA I EE PRIMENENIYA~---
INFORMATICS AND APPLICATIONS\ \ \ 2016\ \ \ volume~10\ \ \ issue\ 4
\hfill \textbf{\thepage}}}

 \label{end\stat}

\newpage

%\def\stat{rekl}
%\label{preobr}

%\def\tit{АКАДЕМИК ПУГАЧЁВ  ВЛАДИМИР СЕМЁНОВИЧ\\
%25.03.1911--25.03.1998}


%   \vspace*{-48pt}
%   \begin{center}\LARGE
%Академик Пугачёв  Владимир Семёнович\\ (25.03.1911--25.03.1998)
%   \end{center}
   
   %\vspace*{2.5mm}
   
   \begin{center}

{\prgsh\LARGE
ОБЪЯВЛЕНИЯ О КОНФЕРЕНЦИЯХ}

\end{center}
%\hrule

\vspace*{6pt}

   
   \vspace*{10mm}
   
   \thispagestyle{empty}

\noindent
\begin{tabular}{cc}
%\begin{center}
\multicolumn{1}{c}{\raisebox{-40pt}[0pt][0pt]{\mbox{%
\epsfxsize=33mm
\epsfbox{vspu.eps}
}}}
%\end{center}
&
\tabcolsep=0pt\begin{tabular}{c}
{\prg{\Large\textbf{XII Всероссийское совещание}}}\\[6pt]
{\prg{\Large\textbf{по проблемам управления}}}\\[12pt]
{\prg{\large 16--19 июня 2014~г.}}\\[6pt] 
{\prg{\large Институт проблем управления имени В.\,А.~Трапезникова РАН}}\\[6pt]
{\prg{\large Москва, Россия}}
\end{tabular}
\end{tabular}

\vspace*{60pt}

     
 { %\large    
 XII Всероссийское совещание по проблемам управления (ВСПУ XII), посвященное 75-летию 
Института проблем управления (ИПУ) имени В.\,А.~Трапезникова РАН, проводится 16--19~июня 
2014~г.\ 
в ИПУ РАН (г.~Москва, Россия). ВСПУ XII организуется ИПУ РАН при поддержке РФФИ, Отделения 
энергетики, машиностроения, механики и процессов управления Российской академии наук, 
Российского 
национального комитета по автоматическому управлению, Академии навигации и управ\-ле\-ния 
движением, 
Научного совета РАН по комплексным проблемам управления и автоматизации, Совета по 
мехатронике и робототехнике РАН. Официальный язык Совещания~--- русский.

\vspace*{24pt}
     
     \textbf{Направления работы}
     \begin{enumerate}[1.]
\item Теория систем управления
\item Управление подвижными объектами и навигация
\item Интеллектуальные системы управления
\item Управление в промышленности, транспортом и логистикой
\item Управление системами междисциплинарной природы
\item Средства измерения, вычислений и контроля в управлении
\item Системный анализ и принятие решений в задачах управления
\item Информационные технологии в управлении
\item Проблемы образования в области управления: современное содержание и технологии обучения
\end{enumerate}

\vspace*{24pt}

     Подробная информация о Совещании находится на сайте {\sf http://vspu2014.ipu.ru}. Срок 
окончательной подачи докладов через систему подачи докладов на сайте~--- \textbf{30~ноября} 
2013~г.
}

%\include{rekl-1}

%\end{document}

%\include{nekrolog-rb}


%\end{document}

%\include{IPPM-25}

\def\stat{cont-rus}
{%\hrule\par
%\vskip 7pt % 7pt
\vspace*{-24pt}
\raggedleft\Large \bf%\baselineskip=3.2ex
Правила подготовки рукописей  для публикации в журнале
<<Информатика~и~её~применения>> \vskip 8pt
    \hrule
    \par
\vskip 14pt plus 6pt minus 3pt }

\label{st\stat}

\def\tit{\ }

\def\aut{\ }
\def\auf{\ }

\def\leftkol{\ }
% Правила подготовки рукописей  для публикации в журнале
%<<Информатика и её применения>>

\def\rightkol{\ }
%Правила подготовки рукописей  для публикации в журнале
%<<Информатика и её применения>>}


\titele{\tit}{\aut}{\auf}{\leftkol}{\rightkol}


\vspace*{-60pt}
{ %\small

Журнал <<Информатика и её применения>>
публикует теоретические, обзорные и дискуссионные статьи,
посвященные научным исследованиям и разработкам в области
информатики и ее приложений.

Журнал издается на русском языке. По специальному решению
редколлегии отдельные статьи могут печататься на английском языке.

Тематика журнала охватывает следующие направления:
\begin{itemize}
\item теоретические основы информатики;\\[-15pt]
      \item
математические методы исследования сложных систем и процессов;\\[-15pt]
           \item
информационные системы и сети;\\[-15pt]
                \item
информационные технологии;\\[-15pt]
                     \item
архитектура и программное обеспечение вычислительных комплексов и сетей.\\[-15pt]
\end{itemize}


\noindent
\begin{enumerate}[1.]
\item В журнале печатаются статьи, содержащие результаты, ранее не опубликованные и
не предназначенные к одновременной публикации в других изданиях.

%Публикация не должна нарушать закон об авторских правах.
Публикация предоставленной автором(ами) рукописи не должна нарушать 
положений глав~69, 70 раздела~VII части~IV Гражданского кодекса, 
которые определяют права на результаты интеллектуальной деятельности 
и~средства индивидуализации, в~том числе авторские права, в~РФ.

Ответственность за нарушение авторских прав, в~случае предъявления претензий к~редакции журнала,  
несут авторы статей.



Направляя рукопись в редакцию, авторы сохраняют свои права на данную
рукопись и при этом передают учредителям и редколлегии журнала неисключительные права на
издание статьи на русском языке 
(или на языке статьи, если он отличен от рус\-ско\-го) и~на перевод ее на английский
язык, а~также на
ее распространение в России и за рубежом. 
Каждый автор должен представить в~редакцию подписанный 
с~его стороны <<Лицензионный договор о~передаче неисключительных прав 
на использование произведения>>, текст которого размещен по адресу 
{\sf http://www.ipiran.ru/publications/licence.doc}. 
Этот договор может быть пред\-став\-лен в~бумажном (в~2-х экз.)\ 
или в~электронном виде (отсканированная копия заполненного и~подписанного документа).




Редколлегия вправе запросить у авторов экспертное заключение о возможности
пуб\-ли\-ка\-ции пред\-став\-лен\-ной статьи в открытой печати.\\[-13.5pt]

\item К статье прилагаются данные автора (авторов) (см.\ п.~8). При наличии нескольких
авторов указывается фамилия автора, ответственного за переписку с редакцией.\\[-13.5pt]

\item Редакция журнала осуществляет экспертизу присланных статей в соответствии с
принятой в журнале процедурой рецензирования.

Возвращение рукописи на доработку не означает ее принятия к печати.

Доработанный вариант с ответом на замечания рецензента необходимо прислать в
редакцию.\\[-13.5pt]

\item Решение редколлегии о публикации статьи или ее отклонении сообщается авторам.

Редколлегия может также направить авторам текст рецензии на их статью. Дискуссия по
поводу отклоненных статей не ведется.\\[-13.5pt]

%\pagebreak

\item Редактура статей высылается авторам для просмотра. Замечания к редактуре должны
быть присланы авторами в кратчайшие сроки.\\[-13.5pt]

\item Рукопись предоставляется в электронном виде в форматах MS WORD (.doc или
.docx) или \LaTeX\  (.tex), дополнительно~--- в формате .pdf, на дискете, лазерном диске
или электронной почтой. Предоставление бумажной рукописи необязательно.\\[-13.5pt]

\item При подготовке рукописи в MS Word рекомендуется использовать следующие
настройки.

Параметры страницы:
формат~--- А4; ориентация~--- книжная; поля (см): внутри~--- 2,5, снаружи~--- 1,5,
сверху~--- 2, снизу~--- 2, от края до нижнего колонтитула~--- 1,3.

Основной текст: стиль~--- <<Обычный>>, шрифт~--- Times New Roman, размер~---
14~пунк\-тов, абзацный отступ~--- 0,5~см, 1,5~интервала, выравнивание~--- по ширине.

\pagebreak

\def\leftkol{Правила подготовки рукописей  для публикации в журнале
<<Информатика и её применения>>}

\def\rightkol{Правила подготовки рукописей  для публикации в журнале
<<Информатика и её применения>>}



Рекомендуемый объем рукописи~--- не свыше 10~страниц указанного формата.
При превышении указанного объема редколлегия вправе потребовать от 
автора сокращения объема рукописи.


Сокращения слов, помимо стандартных, не допускаются. Допускается минимальное
количество аббревиатур.


Все страницы рукописи нумеруются.

Шаблоны оформления представлены в интернете:

\noindent
 {\sf
http://www.ipiran.ru/journal/template\_iiep\_ssi\_2024.zip}\\[-14pt]

\item Статья должна содержать следующую информацию на {\bfseries\textit{русском и
английском языках}}:\\[-16pt]

\begin{itemize}
\item название статьи;\\[-15pt]
\item Ф.И.О.\ авторов, на английском можно только имя и фамилию;\\[-15pt]
\item место работы, с указанием почтового адреса организации и электронного адреса каждого
автора;\\[-15pt]
\item сведения об авторах, в соответствии с форматом, образцы которого
представлены на страницах:



\def\leftfootline{\small{\textbf{\thepage}
\hfill ИНФОРМАТИКА И ЕЁ ПРИМЕНЕНИЯ\ \ \ том\ 18\ \ \ выпуск\ 3\ \ \ 2024}
}%
 \def\rightfootline{\small{ИНФОРМАТИКА И ЕЁ ПРИМЕНЕНИЯ\ \ \ том\ 18\ \ \ выпуск\ 3\ \ \ 2024
\hfill \textbf{\thepage}}}



{\sf http://www.ipiran.ru/journal/issues/2013\_07\_01/authors.asp} и

{\sf http://www.ipiran.ru/journal/issues/2013\_07\_01\_eng/authors.asp};
\item аннотация (не менее 100~слов на каждом из языков). Аннотация~--- это краткое
резюме работы, которое может публиковаться отдельно. Она является основным
источником информации в~ин\-фор\-ма\-ци\-он\-ных системах и базах данных. Английская
аннотация должна быть оригинальной, может не быть дословным переводом русского
текста и должна быть написана хорошим английским языком. В~аннотации не должно
быть ссылок на литературу и, по возможности, формул;\\[-15pt]
\item ключевые слова~--- желательно из принятых в мировой
на\-уч\-но-тех\-ни\-че\-ской литературе тематических тезаурусов. Предложения не
могут быть ключевыми словами;\\[-15pt]
\item источники финансирования работы (ссылки на гранты, проекты,
поддерживающие организации и~т.\,п.).
\end{itemize}



%\pagebreak

\item  Требования к спискам литературы.\\[-14pt]

Ссылки на литературу в тексте статьи нумеруются (в квадратных скобках) и
располагаются в каждом из списков литературы в порядке  первых упоминаний. Если источник имеет DOI и/или EDN,
то их необходимо указывать.

Списки литературы представляются в двух вариантах:\\[-14pt]


\noindent
\begin{enumerate}[(1)]
\item \textbf{Список литературы к русскоязычной части}. Русские и английские
работы~---  на языке и в алфавите оригинала;\\[-14.5pt]
\item  \textbf{References}. Русские работы и работы на других языках~--- в латинской
транслитерации с переводом на английский язык; английские работы и работы на других
языках~--- на языке оригинала.
\end{enumerate}

Необходимо для составления списка ``References'' пользоваться размещенной на сайте
{\sf http://www. translit.net/ru/bgn/} бесплатной программой транслитерации русского
 текста в~латиницу. %, при этом в~за\-клад\-ке <<варианты\ldots>> следует выбратьопцию BGN.

Список литературы ``References'' приводится полностью отдельным блоком, повторяя все
позиции из списка литературы к русскоязычной части, независимо от того, имеются или
нет в нем иностранные источники. Если в списке литературы к русскоязычной части есть
ссылки на иностранные публикации, набранные латиницей, они полностью повторяются в
списке ``References''.

Ниже приведены примеры ссылок на различные виды публикаций в списке ``References''.

\def\leftfootline{\small{\textbf{\thepage}
\hfill ИНФОРМАТИКА И ЕЁ ПРИМЕНЕНИЯ\ \ \ том\ 18\ \ \ выпуск\ 3\ \ \ 2024}
}%
 \def\rightfootline{\small{ИНФОРМАТИКА И ЕЁ ПРИМЕНЕНИЯ\ \ \ том\ 18\ \ \ выпуск\ 3\ \ \ 2024
\hfill \textbf{\thepage}}}

{\small

\noindent
\textbf{Описание статьи из журнала:}

\Aue{Zagurenko, A.\,G., V.\,A.~Korotovskikh, A.\,A.~Kolesnikov, A.\,V.~Timonov, and D.\,V.~Kardymon}. 2008.
Tekhniko-ekonomicheskaya optimizatsiya dizayna gidrorazryva plasta [Technical and
economic optimization of the design
of hydraulic fracturing]. \textit{Neftyanoe hozyaystvo} [\textit{Oil Industry}] 11:54--57.

\Aue{Zhang, Z., and D.~Zhu}. 2008. Experimental research on the localized
electrochemical micromachining. \textit{Russ. J.~Electrochem.}  44(8):926--930.
{\sf doi:10.1134/S1023193508080077}.

\noindent
\textbf{Описание статьи из электронного журнала:}

\Aue{Swaminathan, V., E.~Lepkoswka-White, and B.\,P.~Rao}. 1999. Browsers or buyers in cyberspace? An
investigation of electronic factors influencing electronic exchange. \textit{JCMC}
5(2). Available at: {\sf http://www.ascusc.org/jcmc/vol5/issue2/} (accessed April~28, 2011).

\def\leftkol{Правила подготовки рукописей  для публикации в журнале
<<Информатика и её применения>>}

\def\rightkol{Правила подготовки рукописей  для публикации в журнале
<<Информатика и её применения>>}


\noindent
\textbf{Описание статьи из продолжающегося издания (сборника трудов):}

\Aue{Astakhov, M.\,V., and T.\,V.~Tagantsev}. 2006. Eksperimental'noe
issledovanie prochnosti soedineniy ``stal'--kompozit'' [Experimental study of
the strength of joints ``steel--composite'']. \textit{Trudy MGTU
``Matematicheskoe modelirovanie slozhnykh tekh\-ni\-che\-skikh sistem''}
[\textit{Bauman MSTU ``Mathematical Modeling of Complex Technical
Systems'' Proceedings}]. 593:125--130.


\pagebreak



\noindent
\textbf{Описание материалов конференций:}

\Aue{Usmanov, T.\,S., A.\,A.~Gusmanov, I.\,Z.~Mullagalin, R.\,Ju.~Muhametshina, A.\,N.~Chervyakova, and
A.\,V.~Sveshnikov}. 2007. Osobennosti proektirovaniya razrabotki mestorozhdeniy
s primeneniem gidrorazryva
plasta [Features of the design of field development with the use of hydraulic fracturing].
\textit{Trudy 6-go
Mezhdu\-na\-rod\-no\-go Simpoziuma ``Novye resursosberegayushchie tekhnologii nedropol'zovaniya i povysheniya
neftegazootdachi''} [\textit{6th  Symposium (International) ``New Energy Saving Subsoil Technologies and
the Increasing of the Oil and Gas Impact'' Proceedings}]. Moscow. 267--272.



\def\leftfootline{\small{\textbf{\thepage}
\hfill ИНФОРМАТИКА И ЕЁ ПРИМЕНЕНИЯ\ \ \ том\ 18\ \ \ выпуск\ 3\ \ \ 2024}
}%
 \def\rightfootline{\small{ИНФОРМАТИКА И ЕЁ ПРИМЕНЕНИЯ\ \ \ том\ 18\ \ \ выпуск\ 3\ \ \ 2024
\hfill \textbf{\thepage}}}



\noindent
\textbf{Описание книги (монографии, сборники):}



Lindorf, L.\,S., and L.\,G.~Mamikoniants, eds. 1972.
\textit{Ekspluatatsiya turbogeneratorov s neposredstvennym
okhlazhdeniem} [\textit{Operation of turbine generators with direct cooling}].
Moscow: Energy Publs. 352~p.


\Aue{Latyshev, V.\,N.} 2009. \textit{Tribologiya rezaniya. Kn.~1: Friktsionnye protsessy
pri rezanii metallov}
[\textit{Tribology of cutting. Vol.~1: Frictional processes in metal cutting}]. Ivanovo: Ivanovskii
State Univ. 108~p.

\def\leftkol{Правила подготовки рукописей  для публикации в журнале
<<Информатика и её применения>>}

\def\rightkol{Правила подготовки рукописей  для публикации в журнале
<<Информатика и её применения>>}

\noindent
\textbf{Описание переводной книги}
(в списке литературы к русскоязычной части необходимо указать:~/ Пер.\ с англ.~---
после названия книги, а в конце ссылки указать оригинал книги в круглых скобках):
\begin{enumerate}[1.]
\item  В русскоязычной части:

\def\leftfootline{\small{\textbf{\thepage}
\hfill ИНФОРМАТИКА И ЕЁ ПРИМЕНЕНИЯ\ \ \ том\ 18\ \ \ выпуск\ 3\ \ \ 2024}
}%
 \def\rightfootline{\small{ИНФОРМАТИКА И ЕЁ ПРИМЕНЕНИЯ\ \ \ том\ 18\ \ \ выпуск\ 3\ \ \ 2024
\hfill \textbf{\thepage}}}

\Au{Тимошенко С.\,П., Янг Д.\,Х., Уивер~У.}
Колебания в инженерном деле~/ Пер.\ с англ.~--- М.: Машиностроение, 1985. 472~с.
(\Au{Timoshenko~S.\,P., Young~D.\,H., Weaver~W.}
Vibration problems in engineering.~--- 4th ed.~--- New York, NY, USA: Wiley, 1974. 521~p.)\\[-13.5pt]
\item  В англоязычной части:

\Aue{Timoshenko, S.\,P., D.\,H.~Young, and W.~Weaver}.
1974. \textit{Vibration problems in engineering}. 4th ed. New York: 
Wiley. 521~p.
\end{enumerate}

\vspace*{-3pt}


\noindent
\textbf{Описание неопубликованного документа:}


\Aue{Latypov, A.\,R., M.\,M.~Khasanov, and V.\,A.~Baikov}.
2004 (unpubl.). Geologiya i~dobycha (NGT GiD) [Geology and production (NGT GiD)]. Certificate on official registration of the computer program
No.\,2004611198. 

\noindent
\textbf{Описание интернет-ресурса:}


Pravila tsitirovaniya istochnikov [Rules for the citing of sources]. Available at: {\sf
http://www.scribd.com/doc/1034528/} (accessed February~7, 2011).

%\pagebreak

\noindent
\textbf{Описание диссертации или автореферата диссертации:}

\Aue{Semenov, V.\,I.}
2003. Matematicheskoe modelirovanie plazmy v sisteme kompaktnyy tor [Mathematical
modeling of the plasma in the compact torus].  Moscow.  D.Sc.\ Diss. 272~p.

\Aue{Kozhunova, O.\,S.} 2009. Tekhnologiya razrabotki semanticheskogo
slovarya informatsionnogo monitoringa [Technology of development of
semantic dictionary of information monitoring system].  Moscow: IPI RAN. PhD Thesis. 23~p.


\noindent
\textbf{Описание ГОСТа:}

GOST 8.586.5-2005. 2007. Metodika vypolneniya izmereniy. Izmerenie raskhoda i~kolichestva zhidkostey i~gazov
s~pomoshch'yu standartnykh suzhayushchikh ustroystv [Method of measurement.
Measurement of flow rate and volume of liquids and gases by means of orifice devices]. Moscow:
Standardinform  Publs. 10~p.

\noindent
\textbf{Описание патента:}

\Aue{Bolshakov, M.\,V., A.\,V.~Kulakov, A.\,N.~Lavrenov, and M.\,V.~Palkin}.
2006. Sposob orientirovaniya po krenu letatel'nogo
apparata s opti\-che\-skoy golovkoy
samonavedeniya [The way to orient on the roll of aircraft with optical homing head].
Patent RF No.\,2280590.
}

\item Присланные в редакцию материалы авторам не возвращаются.\\[-13.5pt]

\item При отправке файлов по электронной почте просим придерживаться следующих
правил:
\begin{itemize}
\item указывать в поле subject (тема) название журнала и фамилию автора;\\[-13.5pt]
\item указывать в тексте письма название статьи, авторов и~журнал, в~который направляется статья;\\[-13.5pt]
\item использовать attach (присоединение);\\[-13.5pt]
\item в состав электронной версии статьи должны входить: файл, содержащий текст
статьи, и файл(ы), содержащий(е) иллюстрации.\\[-13.5pt]
\end{itemize}

\item Журнал <<Информатика и её применения>> является некоммерческим изданием.
Плата за публикацию не взимается, гонорар авторам не выплачивается.
\end{enumerate}



\def\leftfootline{\small{\textbf{\thepage}
\hfill ИНФОРМАТИКА И ЕЁ ПРИМЕНЕНИЯ\ \ \ том\ 18\ \ \ выпуск\ 3\ \ \ 2024}
}%
 \def\rightfootline{\small{ИНФОРМАТИКА И ЕЁ ПРИМЕНЕНИЯ\ \ \ том\ 18\ \ \ выпуск\ 3\ \ \ 2024
\hfill \textbf{\thepage}}}


\vspace*{-1mm}

\begin{center}

\textbf{Адрес редакции журнала <<Информатика и её применения>>:} \\




Москва 119333, ул.~Вавилова, д.~44, корп.~2, ФИЦ ИУ РАН\\[-10pt]

\

Тел.: +7\,(499)\,135-86-92\ \ Факс:  +7\,(495)\,930-45-05\\[-10pt]

 \

e-mail:   {\sf iiep@frccsc.ru} (Стригина Светлана Николаевна)\\[-10pt]

\

{\sf http://www.ipiran.ru/journal/issues/}
\end{center}
}


\def\leftkol{Правила подготовки рукописей  для публикации в журнале
<<Информатика и её применения>>}

\def\rightkol{Правила подготовки рукописей  для публикации в журнале
<<Информатика и её применения>>}


\def\leftfootline{\small{\textbf{\thepage}
\hfill ИНФОРМАТИКА И ЕЁ ПРИМЕНЕНИЯ\ \ \ том\ 18\ \ \ выпуск\ 3\ \ \ 2024}
}%
 \def\rightfootline{\small{ИНФОРМАТИКА И ЕЁ ПРИМЕНЕНИЯ\ \ \ том\ 18\ \ \ выпуск\ 3\ \ \ 2024
\hfill \textbf{\thepage}}} 
\def\stat{podg-e}
{%\hrule\par
%\vskip 7pt % 7pt
\vspace*{-24pt}
\raggedleft\Large \bf%\baselineskip=3.2ex
Requirements for manuscripts submitted to Journal
``Informatics~and~Applications'' \vskip 8pt
    \hrule
    \par
\vskip 21pt plus 6pt minus 3pt }

\label{st\stat}

\def\tit{\ }

\def\aut{\ }
\def\auf{\ }

\def\leftkol{\ }

\def\rightkol{\ }
%Requirements for manuscripts submitted to Journal
%``Informatics~and~Applications''}

\titele{\tit}{\aut}{\auf}{\leftkol}{\rightkol}

\def\leftfootline{\small{\textbf{\thepage}
\hfill INFORMATIKA I EE PRIMENENIYA~--- INFORMATICS AND APPLICATIONS\ \ \ 2019\
\ \ volume~13\ \ \ issue\ 4}
}%
 \def\rightfootline{\small{INFORMATIKA I EE PRIMENENIYA~--- INFORMATICS AND APPLICATIONS\ \ \ 2019\ \ \ volume~13\ \ \ issue\ 4
\hfill \textbf{\thepage}}}

\vspace*{-60pt}

{\small

\noindent
Journal ``Informatics and Applications'' (Inform.\ Appl.)
publishes theoretical, review, and discussion
articles on the research and development in the
field of informatics and its applications.

The journal is published in Russian.
By a special decision of the editorial
board, some articles can be published in English.


The topics covered include the following areas:
\begin{itemize}
               \item
     theoretical fundamentals of informatics; \\[-14pt]
\item
mathematical methods for studying complex systems and processes; \\[-14pt]
\item
information systems and networks;\\[-14pt]
\item
information technologies; and \\[-14pt]
\item
architecture and software of computational complexes and networks. \\[-14pt]
\end{itemize}

\noindent
\begin{enumerate}[1.]
\item The Journal publishes original articles which have not been published before and are not
intended for simultaneous publication in other editions. An article submitted to the Journal must not violate the
Copyright law. Sending the manuscript to the Editorial Board, the authors retain all rights of the
owners of the manuscript and transfer the nonexclusive rights to publish the article in Russian
(or the language of the article, if not Russian) and its distribution in Russia and abroad to the
Founders and the Editorial Board. Authors should submit a letter to the Editorial Board in the
following form:

{\bfseries\textit{Agreement on the transfer of rights to publish:}}

``\textit{We, the undersigned authors of the manuscript ``\ldots'', pass to the
Founder and the Editorial Board of the Journal ``Informatics and Applications''
the nonexclusive right to publish the manuscript of the article in Russian (or
in English) in both print and electronic versions of the Journal. We affirm
that this publication does not violate the Copyright of other persons or
organizations.}

\textit{Author(s) signature(s): (name(s), address(es), date).}

This agreement should be submitted in paper form or in the form of a scanned copy (signed by
the authors).


%The Editorial Board has the right to request from the authors an official expert conclusion that
%the submitted article has no secret data prohibited for publication. \\[-13.5pt]
\item
A submitted article should be attached with \textbf{the data on the author(s)} (see item~8). If
there are several authors, the contact person should be indicated who is responsible for
correspondence with the Editorial Board and other authors about revisions and final approval
of the proofs.\\[-13.5pt]

\item The Editorial Board of the Journal examines the article according to the established
reviewing procedure. If the authors receive their article for correction after reviewing, it does not
mean that the article is approved for publication. The corrected article should be sent to the
Editorial Board for the subsequent review and approval.\\[-13.5pt]

\item The decision on the article publication or its rejection is communicated to the authors. The
Editorial Board may also send the reviews on the submitted articles to the authors. Any
discussion upon the rejected articles is not possible.\\[-13.5pt]

\item The edited articles will be sent to the authors for proofread. The comments of the authors
to the edited text of the article should be sent to the Editorial Board as soon as possible.\\[-13.5pt]

\item The manuscript of the article should be presented electronically in the MS WORD (.doc or
.docx) or \LaTeX\ (.tex) formats, and additionally in the .pdf format. All documents
 may be sent
by e-mail or provided on a CD or diskette. A~hard copy submission is not necessary.\\[-13.5pt]

\item The recommended typesetting instructions for manuscript.

Pages parameters: format A4, portrait orientation, document margins (cm): left~--- 2.5, right~---
1.5, above~--- 2.0, below~--- 2.0, footer 1.3.

Text: font~---Times New Roman, font size~--- 14, paragraph indent~--- 0.5, line spacing~--- 1.5,
justified alignment.

The recommended manuscript size: not more than 15~pages of the specified format.
If the specified size exceeded, the editorial board is entitled to require the author
to reduce the manuscript.

Use only standard abbreviations. Avoid  abbreviations in the title and
abstract. The full term for which an abbreviation stands should precede
its first use in the text unless it is a standard unit of measurement.

All pages of the manuscript should be numbered.

The templates for the manuscript typesetting are presented on site: {\sf
http://www.ipiran.ru/journal/template.doc}.\\[-13.5pt]


%\def\leftkol{Requirements for manuscripts submitted to Journal
%``Informatics~and~Applications''}

\item The articles should enclose data both in \textbf{Russian and English}:
\begin{itemize}
\item title;\\[-13.5pt]
\item author's name and surname;\\[-13.5pt]
\item affiliation~--- organization, its address with ZIP code, city, country, and
official e-mail address;\\[-13.5pt]
\item data on authors according to the format: (see site)

{\sf http://www.ipiran.ru/journal/issues/2013\_07\_01/authors.asp}  and

{\sf  http://www.ipiran.ru/journal/issues/2013\_07\_01\_eng/authors.asp};\\[-13.5pt]

\pagebreak

\def\leftfootline{\small{\textbf{\thepage}
\hfill INFORMATIKA I EE PRIMENENIYA~--- INFORMATICS AND APPLICATIONS\ \ \ 2019\
\ \ volume~13\ \ \ issue\ 4}
}%
 \def\rightfootline{\small{INFORMATIKA I EE PRIMENENIYA~--- INFORMATICS AND APPLICATIONS\ \ \ 2019\ \ \ volume~13\ \ \ issue\ 4
\hfill \textbf{\thepage}}}


%\def\leftkol{Requirements for manuscripts submitted to Journal
%``Informatics~and~Applications''}

%\def\rightkol{Requirements for manuscripts submitted to Journal
%``Informatics~and~Applications''}



\item abstract (not less than 100 words) both in Russian and in English. Abstract is a short
summary of the article that can be published separately. The abstract is the
main source of information on the article and it could be included in leading information
systems and data bases. The abstract in English has to be an original text and should
not be an exact translation of the Russian one. Good English is required.
In abstracts, avoid references and formulae;\\[-13.5pt]
\item indexing is performed on the basis of keywords. The use of keywords from the
internationally accepted thematic Thesauri is recommended.

%\def\leftkol{Requirements for manuscripts submitted to Journal
%``Informatics~and~Applications''}

%\def\rightkol{Requirements for manuscripts submitted to Journal
%``Informatics~and~Applications''}

Important! Keywords must not be sentences;
\item Acknowledgments.
\end{itemize}

\item References. Russian references have to be presented both in English translation and Latin
transliteration (refer {\sf http://www.translit.net/ru/bgn/}).

Please take into account the following examples of Russian references appearance:

\noindent
\textbf{Article in journal:}

\Aue{Zhang, Z., and D.~Zhu}. 2008. Experimental research on the localized electrochemical
micromachining.
\textit{Rus. J.~Electrochem.}  44(8):926--930. {\sf doi:10.1134/S1023193508080077}.


\noindent
\textbf{Journal article in electronic format:}

\Aue{Swaminathan, V., E.~Lepkoswka-White, and B.\,P.~Rao}. 1999. Browsers or buyers in
cyberspace? An
investigation of electronic factors influencing electronic exchange. \textit{JCMC}
5(2). Available at: {\sf http://www.ascusc.org/jcmc/vol5/issue2/} (accessed April~28, 2011).




\noindent
\textbf{Article from the continuing publication (collection of works, proceedings):}

\Aue{Astakhov, M.\,V., and T.\,V.~Tagantsev}. 2006. Eksperimental'noe
issledovanie prochnosti soedineniy ``stal'--kompozit'' [Experimental study of
the strength of joints ``steel--composite'']. \textit{Trudy MGTU
``Matematicheskoe modelirovanie slozhnykh tekh\-ni\-che\-skikh sistem''}
[\textit{Bauman MSTU ``Mathematical Modeling of Complex Technical
Systems'' Proceedings}]. 593:125--130.

\def\leftfootline{\small{\textbf{\thepage}
\hfill INFORMATIKA I EE PRIMENENIYA~--- INFORMATICS AND APPLICATIONS\ \ \ 2019\
\ \ volume~13\ \ \ issue\ 4}
}%
 \def\rightfootline{\small{INFORMATIKA I EE PRIMENENIYA~--- INFORMATICS AND APPLICATIONS\ \ \ 2019\ \ \ volume~13\ \ \ issue\ 4
\hfill \textbf{\thepage}}}

\def\leftkol{Requirements for manuscripts submitted to Journal
``Informatics~and~Applications''}

\def\rightkol{Requirements for manuscripts submitted to Journal
``Informatics~and~Applications''}

\noindent
\textbf{Conference proceedings:}

\Aue{Usmanov, T.\,S., A.\,A.~Gusmanov, I.\,Z.~Mullagalin, R.\,Ju.~Muhametshina,
A.\,N.~Chervyakova, and
A.\,V.~Sveshnikov}. 2007. Osobennosti proektirovaniya razrabotki mestorozhdeniy
s primeneniem gidrorazryva
plasta [Features of the design of field development with the use of hydraulic fracturing].
\textit{Trudy 6-go
Mezhdu\-na\-rod\-no\-go Simpoziuma ``Novye resursosberegayushchie tekhnologii
nedropol'zovaniya i povysheniya
neftegazootdachi''} [\textit{6th  Symposium (International) ``New Energy Saving Subsoil
Technologies and
the Increasing of the Oil and Gas Impact'' Proceedings}]. Moscow. 267--272.


\noindent
\textbf{Books and other monographs:}




Lindorf, L.\,S., and L.\,G.~Mamikoniants, eds. 1972.
\textit{Ekspluatatsiya turbogeneratorov s neposredstvennym
okhlazhdeniem} [\textit{Operation of turbine generators with direct cooling}].
Moscow: Energy Publs. 352~p.


%\Aue{Latyshev, V.\,N.} 2009. \textit{Tribologiya rezaniya. Kn.~1: Frikcionnye prosessy
%pri rezanii metallov}
%[\textit{Tribology of cutting. Vol.~1: Frictional processes in metal cutting}]. Ivanovo: Ivanovskii
%State Univ. 108~p.


%\noindent
%\textbf{Unpublished material:}

%\Aue{Latypov, A.\,R., M.\,M.~Khasanov, and V.\,A.~Baikov}.
%2004. Geology and production (NGT GiD). Certificate on official registration of the computer
%program
%No.\,2004611198. (In Russian, unpubl.)

%\noindent
%\textbf{Internet-source:}

%APA Style. 2011. Available at: {\sf http://www.apastyle.org/apa-style-help.aspx} (accessed
%February~5, 2011).

%Pravila citirovaniya istochnikov [Rules for the citing of sources]. Available at: {\sf
%http://www.scribd.com/doc/1034528/} (accessed February~7, 2011).


\noindent
\textbf{Dissertation and Thesis:}

%\Aue{Semenov, V.\,I.}
%2003. Matematicheskoe modelirovanie plazmy v sisteme kompaktnyy tor. [Mathematical
%modeling of the plasma in the compact torus]. D.Sc.\ Diss. Moscow. 272~p.

\Aue{Kozhunova, O.\,S.} 2009. Tekhnologiya razrabotki semanticheskogo
slovarya informatsionnogo monitoringa [Technology of development of
semantic dictionary of information monitoring system]. PhD Thesis. Moscow: IPI RAN. 23~p.


\noindent
\textbf{State standards and patents:}

GOST 8.586.5-2005. 2007. Metodika vypolneniya izmereniy. Izmerenie raskhoda i~kolichestva
zhidkostey i gazov 
s~pomoshch'yu standartnykh suzhayushchikh ustroystv [Method of measurement.
Measurement of flow rate and volume of liquids and gases by means of orifice devices]. M.:
Standardinform
Publs. 10~p.

%\noindent
%\textbf{Patent:}

\Aue{Bolshakov, M.\,V., A.\,V.~Kulakov, A.\,N.~Lavrenov, and M.\,V.~Palkin}.
2006. Sposob orientirovaniya po krenu letatel'nogo
apparata s opti\-che\-skoy golovkoy
samonavedeniya [The way to orient on the roll of aircraft with optical homing head].
Patent RF No.\,2280590.

References in Latin transcription are presented in the original language.

References in the text are numbered according to the order of their
first appearance; the number is
placed in square brackets. All items from the reference list should be
cited.\\[-13.5pt]

\item Manuscripts and additional materials are not returned to Authors by the Editorial Board.\\[-13.5pt]

\item Submissions of files by e-mail must include:\\[-13.5pt]
\begin{itemize}
\item   the journal title and author's name in the ``Subject'' field; \\[-13.5pt]
\item   an article and additional materials have to be attached using the ``attach'' function;\\[-13.5pt]
\item   an electronic version of the article should contain the file with the text and a separate file
with figures.\\[-13.5pt]
\end{itemize}

\item ``Informatics and Applications'' journal is not a profit publication. There are no
charges for the authors as well as there are no royalties.\\[-13.5pt]
\end{enumerate}

\def\leftfootline{\small{\textbf{\thepage}
\hfill INFORMATIKA I EE PRIMENENIYA~--- INFORMATICS AND APPLICATIONS\ \ \ 2019\
\ \ volume~13\ \ \ issue\ 4}
}%
 \def\rightfootline{\small{INFORMATIKA I EE PRIMENENIYA~--- INFORMATICS AND APPLICATIONS\ \ \ 2019\ \ \ volume~13\ \ \ issue\ 4
\hfill \textbf{\thepage}}}

\def\leftkol{Requirements for manuscripts submitted to Journal
``Informatics~and~Applications''}

\def\rightkol{Requirements for manuscripts submitted to Journal
``Informatics~and~Applications''}


%\vspace*{5mm}


\begin{center}
\textbf{Editorial Board address:} \\

%ABOUT AUTHORS



FRC CSC RAS, 44, block~2, Vavilov Str., Moscow 119333, Russia\\[-10pt]

\

Ph.: +7\,(499)\,135\,86\,92,\ \ Fax: +7\,(495)\,930\,45\,05\\[-10pt]

\

 e-mail: {\sf rust@ipiran.ru} (to Prof.\ Rustem Seyful-Mulyukov)\\[-10pt]

\

 {\sf http://www.ipiran.ru/english/journal.asp}
\end{center}
 }
%\thispagestyle{myheadings}

\def\leftkol{Requirements for manuscripts submitted to Journal
``Informatics~and~Applications''}

\def\rightkol{Requirements for manuscripts submitted to Journal
``Informatics~and~Applications''}

\def\leftfootline{\small{\textbf{\thepage}
\hfill INFORMATIKA I EE PRIMENENIYA~--- INFORMATICS AND APPLICATIONS\ \ \ 2019\
\ \ volume~13\ \ \ issue\ 4}
}%
 \def\rightfootline{\small{INFORMATIKA I EE PRIMENENIYA~--- INFORMATICS AND APPLICATIONS\ \ \ 2019\ \ \ volume~13\ \ \ issue\ 4
\hfill \textbf{\thepage}}}

 \label{end\stat}

\newpage

%\vspace*{-60pt} {\small
{\baselineskip=9.1pt
\section*{Правила подготовки рукописей статей для публикации в журнале
<<Информатика и её применения>>}

\thispagestyle{empty}

 Журнал <<Информатика и её применения>> публикует
теоретические, обзорные и дискуссионные статьи, посвященные научным
исследованиям и разработкам в области информатики и ее приложений. Журнал
издается на русском языке. По специальному решению редколлегии отдельные статьи,
в виде исключения, могут печататься на английском языке.
Тематика журнала охватывает следующие направления:
\begin{itemize}
\item теоретические основы информатики; %\\[-13.5pt]
\item математические методы исследования сложных систем и процессов; %\\[-13.5pt]
\item информационные системы и сети; %\\[-13.5pt]
\item информационные технологии; %\\[-13.5pt]
\item архитектура и программное
обеспечение вычислительных комплексов и сетей.
\end{itemize}
\begin{enumerate}
\item В журнале печатаются результаты, ранее не
опубликованные и не предназначенные к одновременной публикации в других
изданиях. Публикация не должна нарушать закон об авторских правах. Направляя
свою рукопись в редакцию, авторы автоматически передают учредителям и
редколлегии неисключительные права на издание данной статьи на русском языке и
на ее распространение в России и за рубежом. При этом за авторами сохраняются
все права как собственников данной рукописи. В связи с этим авторами должно
быть представлено в редакцию письмо в следующей форме:
Соглашение о передаче права на публикацию:

\textit{<<Мы, нижеподписавшиеся, авторы рукописи <<$\qquad\qquad$>>, передаем
учредителям и редколлегии журнала <<Информатика и её применения>>
неисключительное право опубликовать данную рукопись статьи на русском языке как
в печатной, так и в электронной версиях журнала. Мы подтверждаем, что данная
публикация не нарушает авторского права других лиц или организаций. Подписи
авторов: (ф.\,и.\,о., дата, адрес)>>.}

Указанное соглашение может быть представлено 
как в бумажном виде, так и в виде отсканированной копии (с подписями авторов).


Редколлегия вправе запросить у авторов экспертное заключение о возможности
опубликования представленной статьи в открытой печати. %\\[-13.5pt]
\item Статья
подписывается всеми авторами. На отдельном листе представляются данные автора
(или всех авторов): фамилия, полные имя и отчество, телефон, факс, e-mail,
почтовый адрес. Если работа выполнена несколькими авторами, указывается фамилия
одного из них, ответственного за переписку с редакцией. %\\[-13.5pt]
\item Редакция журнала
осуществляет самостоятельную экспертизу присланных статей. Возвращение рукописи
на доработку не означает, что статья уже принята к печати. Доработанный вариант
с ответом на замечания рецензента необходимо прислать в редакцию. %\\[-13.5pt]
\item Решение
редакционной коллегии о принятии статьи к печати или ее отклонении сообщается
авторам. Редколлегия не обязуется направлять рецензию авторам отклоненной
статьи. %\\[-13.5pt]
\item Корректура статей высылается авторам для просмотра. Редакция
просит авторов присылать свои замечания в кратчайшие сроки. %\\[-13.5pt]
\item При
подготовке рукописи в MS Word рекомендуется использовать следующие настройки.
Параметры страницы: формат~--- А4; ориентация~--- книжная; поля (см): внутри~---
2,5, снаружи~--- 1,5, сверху~--- 2, снизу~--- 2, от края до нижнего
колонтитула~--- 1,3. Основной текст: стиль~--- <<Обычный>>: шрифт Times New
Roman, размер 14~пунктов, абзацный отступ~--- 0,5~см, 1,5 интервала,
выравнивание~--- по ширине. Рекомендуемый объем рукописи~--- не свыше
25~страниц указанного формата. Ознакомиться с шаблонами, содержащими примеры
оформления, можно по адресу в Интернете:
\textsf{http://www.ipiran.ru/journal/template.doc}.
\item К рукописи, предоставляемой в 2-х
экземплярах, обязательно прилагается электронная версия статьи (как правило, в
форматах MS WORD (.doc) или \LaTeX\ (.tex), а также~--- дополнительно~--- в
формате .pdf) на дискете, лазерном диске или по электронной почте. Сокращения
слов, кроме стандартных, не применяются. Все страницы рукописи должны быть
пронумерованы. %\\[-13.5pt]
\item Статья должна содержать следующую информацию на русском и
английском языках: название, Ф.И.О. авторов, места работы авторов и их
электронные адреса, подробные сведения об авторах, оформленные в соответствии с форматом, 
определяемым файлами {\sf http://www.ipiran.ru/journal/issues/2011\_05\_01/authors.asp} и 
{\sf http://www.ipiran.ru/journal/issues/2011\_01\_eng/authors.asp},
аннотация (не более 100~слов), ключевые слова. Ссылки на
литературу в тексте статьи нумеруются (в квадратных скобках) и располагаются в
порядке их первого упоминания. В~списке литературы не должно быть позиций, на которые нет ссылки в тексте статьи.
Все фамилии авторов, заглавия статей, названия
книг, конференций и~т.\,п.\ даются на языке оригинала, если этот язык
использует кириллический или латинский алфавит. %\\[-13.5pt]
\item Присланные в редакцию материалы авторам не возвращаются.
\item При отправке файлов по электронной
почте просим придерживаться следующих правил:
\begin{itemize}
\item указывать в поле subject (тема) название журнала и фамилию автора; %\\[-13.5pt]
\item использовать attach (присоединение); %\\[-13.5pt]
\item в случае больших объемов информации возможно
использование общеизвестных архиваторов (ZIP, RAR); %\\[-13.5pt]
\item в состав электронной версии статьи должны входить: файл, содержащий текст статьи, и файл(ы),
содержащий(е) иллюстрации. %\\[-13.5pt]
\end{itemize}
\item Журнал <<Информатика и её применения>> является некоммерческим изданием. 
Плата за публикацию с авторов не взимается, гонорар авторам не выплачивается.
\end{enumerate}
\thispagestyle{empty}
\textbf{Адрес редакции:} Москва 119333,
ул.~Вавилова, д.~44, корп.~2, ИПИ РАН\\
\hphantom{\textbf{Адрес редакции:} }Тел.: +7 (499) 135-86-92\ \
Факс:  +7 (495) 930-45-05\ \  E-mail:   rust@ipiran.ru }
}

%\include{ipi-ind}

%\tableofcontents

\end{document}

%\tableofcontents

%\end{document}

%\tableofcontents


\end{document}

\newcommand{\Ack}{\subsection*{\protect\large\bf Acknowledgments}}

\vphantom{\int\limits_0^T }

{ \begin{center}  %fig1
 \vspace*{3pt}
    \mbox{%
 \epsfxsize=79mm 
 \epsfbox{gru-1.eps}
 }

\end{center}

\noindent
{{\figurename~1}\ \ \small{
Временные зависимости данные 
}}}

\vspace*{6pt}

\setcounter{figure}{1}

$\acute{\mbox{о}}$