\documentclass[14pt,a4paper]{extarticle}

\usepackage{nccsect}
\usepackage[utf8]{inputenc}
\usepackage[english,russian]{babel}
\usepackage{amssymb,amsmath,amsthm}
\usepackage{epsfig}
\usepackage{hyperref}
\hypersetup{
colorlinks=true,
allcolors = blue
}


\usepackage{color}
\usepackage{cite}
\usepackage[a4paper, margin=2cm]{geometry}



\sloppy


\begin{document}







\section*{On the distribution of the ratio of the sum of sample elements exceeding a threshold to the total sum of sample elements. I}

\large{\bfseries V.\,Yu.~Korolev}
\bigskip

\small


$^1$Faculty of Computational Mathematics and Cybernetics, Lomonosov Moscow
State University, GSP-1, Leninskie Gory, Moscow, 119991, Russian Federation

$^2$Federal Research Center ``Computer Science and Control'' of the Russian Academy of Sciences, 44-2 Vavilov Str., Moscow 119333, Russian Federation


\bigskip

\bigskip

{\bf Abstract:} The problem of description of the distribution of the ratio of the sum of sample elements exceeding a threshold to the total sum of sample elements is considered. Unlike known versions of this problem in which the number of summed extreme order statistics is fixed, here the specified threshold can be exceeded by an unpredictable number of sample elements. In the paper, in terms of the distribution function of a separate summand, the explicit form of the distribution of the ratio of the sum of sample elements exceeding a threshold to the total sum of sample elements is formally presented. The asymptotic and limit distributions are heuristically deduced for this ratio. These distributions are convenient for practical computation. The cases are considered in which the distributions of the summands have light tails (the second moments are finite), as well as the cases in which these distributions have heavy tails (belong to the domain of attraction of a stable law). In all cases, the normalization of the ratio is described that provides the existence of a non-degenerate limit (as the number of summands infinitely increases) distribution as well as the limit distribution itself (normal for the case of light tails and stable for the case of heavy tails).



\medskip

{\bf Keywords}: sum of independent random variables; random sum; binomial distribution; mixture of probability distributions; normal distribution; stable distribution; extreme order statistic



\section*{Acknowledgments}
The research was conducted in accordance with the scientific program of the Moscow Center for Fundamental and Applied Mathematics and supported by the Russian Foundation for Basic Research (project 19-07-00914).


\section*{References}
\begin{enumerate}
\renewcommand{\labelenumi}{\theenumi.}
\setlength{\itemsep}{1pt}
\setlength{\parskip}{1pt}


\item
Ben Rached, N., Z. Botev, A. Kammoun, M.-S. Alouini, and R. Tempone. 2017.
On the sum of order statistics and applications to wireless communication systems performances.
{\it arXiv.org} Available at: https://arxiv.org/abs/1711.04280 
(accessed July 30, 2020).

\item
Csorgo, S., L. Horvath, and D.M. Mason. 1986.
What portion of the sample makes a partial sum asymptotically stable or normal?
{\it Probab. Theory Related Fields}. 72: 1--16.

\item
Csorgo, S., E. Haeusler, and D.M. Mason. 1988.
The asymptotic distribution of trimmed sums.
{\it Ann. Probab.} 16: 672--699.

\item
Csorgo, S., E. Haeusler and D.M. Mason. 1988.
A probabilistic approach to the asymptotic
distribution of sums of independent, identically distributed random variables.
{\it Adv. Appl. Math.} 9: 259--333.

\item
Csorgo, S. 1989. 
Limit theorems for sums of order statistics. 
{\it Sixth international summer school in probability theory and mathematical statistics.}
Varna. 5--37.

\item
Hahn, M. G., D. M. Mason and D.C. Weiner eds. 1991.
{\it Sums, trimmed sums and extremes.}
Boston: Birkhauser. 428 p.

\item
Janssen, A. 2000.
Invariance principles for sums of extreme sequential order statistics attracted to Levy processes.
{\it Stoch. Process. Appl.} 85: 255--277.

\item
Kesten, H., and R.A. Maller. 1992.
Ratios of trimmed sums and order statistics.
{\it Ann. Probab.} 20(4): 1805--1842.

\item
Koch, R. 1998.
{\it The 80/20 principle: the secret of achieving more with less}. 
London: Nicholas Brealey Publishing. 304 p. 

\item
Korolev, V.Yu. 1994.
Convergence of random sequences with independent random indexes. I.
{\it Theor. Probab. Appl.} 39(2): 313--333.

\item
Korolev, V.Yu. 1995.
Convergence of random sequences with independent random indexes. II.
{\it Theor. Probab. Appl.} 40(4): 770--772.

\item
Prakasa Rao, B. L. S. 1976.
Limit theorems for sums of order statistics.
{\it Zeitschrift fur Wahrscheinlichkeitstheorie und Verwandte Gebiete}
33(4): 285--307.


\end{enumerate}




\section*{╤тхфхэш  юс ртЄюЁрї}

\small

{\bfseries ╩юЁюыхт ┬шъЄюЁ ▐Ё№хтшў} (Ё. 1954) -- фюъЄюЁ Їшчшъю-ьрЄхьрЄшўхёъшї эрєъ, яЁюЇхёёюЁ, чртхфє■∙шщ ърЇхфЁющ ьрЄхьрЄшўхёъющ ёЄрЄшёЄшъш Їръєы№ЄхЄр т√ўшёышЄхы№эющ ьрЄхьрЄшъш ш ъшсхЁэхЄшъш ш уыртэ√щ эрєўэ√щ ёюЄЁєфэшъ ╠юёъютёъюую ЎхэЄЁр ЇєэфрьхэЄры№эющ ш яЁшъырфэющ ьрЄхьрЄшъш ╠юёъютёъюую уюёєфрЁёЄтхээюую єэштхЁёшЄхЄр шьхэш ╠.\,┬.~╦юьюэюёютр; тхфє∙шщ эрєўэ√щ ёюЄЁєфэшъ ╘хфхЁры№эюую шёёыхфютрЄхы№ёъюую ЎхэЄЁр <<╚эЇюЁьрЄшър ш єяЁртыхэшх>> ╨юёёшщёъющ рърфхьшш эрєъ


{\bfseries Korolev Victor Yu.} (b. 1954) -- Doctor of Science in physics and
mathematics, professor, head of department, Faculty of Computational Mathematics and Cybernetics, and principal scientist, Moscow Center for Fundamental and Applied Mathematics, Lomonosov Moscow State University, GSP-1, Leninskie Gory, Moscow, 119991, Russian Federation; leading scientist, Federal Research Center ``Computer Science and Control'' of the Russian Academy of Sciences, 44-2 Vavilova Str., Moscow 119333, Russian Federation; \url{vkorolev@cs.msu.ru}


\end{document} 