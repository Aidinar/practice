\def\stat{grusho-tim}

\def\tit{ВЫЯВЛЕНИЕ АНОМАЛИЙ С ПОМОЩЬЮ МЕТАДАННЫХ$^*$}

\def\titkol{Выявление аномалий с~помощью метаданных}

\def\aut{А.\,А.~Грушо$^1$, Е.\,Е.~Тимонина$^2$, Н.\,А.~Грушо$^3$, И.\,Ю.~Терехина$^4$}

\def\autkol{А.\,А.~Грушо, Е.\,Е.~Тимонина, Н.\,А.~Грушо, И.\,Ю.~Терехина}

\titel{\tit}{\aut}{\autkol}{\titkol}

\index{А.\,А.~Грушо, Е.\,Е.~Тимонина, Н.\,А.~Грушо, И.\,Ю.~Терехина}
\index{A.\,A.~Grusho, E.\,E.~Timonina, N.\,A.~Grusho, 
and~I.\,Yu.~Teryokhina}
 

{\renewcommand{\thefootnote}{\fnsymbol{footnote}} \footnotetext[1]
{Работа частично поддержана РФФИ (проект 18-07-00274).}}


\renewcommand{\thefootnote}{\arabic{footnote}}
\footnotetext[1]{Институт проблем информатики Федерального исследовательского центра <<Информатика и~управ\-ле\-ние>> 
Российской академии наук, \mbox{grusho@yandex.ru}}
\footnotetext[2]{Институт проблем информатики Федерального исследовательского центра <<Информатика и~управ\-ле\-ние>> 
Российской академии наук, \mbox{eltimon@yandex.ru}}
\footnotetext[3]{Институт проблем информатики Федерального исследовательского центра <<Информатика и~управ\-ле\-ние>> 
Российской академии наук, \mbox{info@itake.ru}}
\footnotetext[4]{Факультет вычислительной математики и~кибернетики Московского государственного университета им.\ 
М.\,В.~Ломоносова, \mbox{irina\_teryokhina@mail.ru}}

\vspace*{6pt}
  

  
  \Abst{Рассматривается проблема контроля безопасности информационных 
  технологий (ИТ) на 
основе данных компьютерного аудита. Эти данные представляют собой последовательность 
малых выборок, каж\-дая из которых описывает передачу информации от одних преобразований 
к~другим. Информационные технологии представляются математическими моделями в~виде 
ориентированных ациклических графов. Такие графы, описывающие передачу данных, в~статье 
называются метаданными (МД). 
  В~единых данных компьютерного аудита могут одновременно присутствовать следы 
выполнения нескольких ИТ, описываемые своими графами. Это 
обстоятельство затрудняет распознавание информационных потоков
(ИП), которые соответствуют 
дугам разных графов. В~работе введено понятие легальных ИП, которые 
соответствуют передаче данных всех выполняемых ИТ. 
Информационные потоки, не соответствующие выполнению действующих 
ИТ, называются нелегальными, или аномалиями. Такие ИП могут 
возникать из-за враждебной деятельности инсайдеров или из-за ошибок действий 
пользователей.
  В~\mbox{статье} на основе МД решена задача эффективного выявления легальных 
ИП и~аномалий. }
  
  \KW{информационная безопасность; информационный поток; аномалии; метаданные; 
системы различных представителей}

\DOI{10.14357/19922264200311}
 
\vspace*{6pt}


\vskip 10pt plus 9pt minus 6pt

\thispagestyle{headings}

\begin{multicols}{2}

\label{st\stat}
  
\section{Введение}

  В работе рассмотрена проблема выявления нелегальных
ИП с~по\-мощью МД. Пусть распределенная 
ин\-фор\-ма\-ци\-он\-но-вы\-чис\-ли\-тель\-ная сис\-те\-ма (РИВС) реализует 
ряд ИТ, поддерживающих работу организации, например банка. Каждая 
ИТ формируется на основе биз\-нес-про\-цес\-са, и~в~\mbox{РИВС}\linebreak одновременно 
функционируют несколько ИТ. В~работах~[1, 2] модель произвольной ИТ была 
представлена в~виде ориентированного ациклического графа (DAG~--- Directed 
Acyclic Graph). В~этих моделях вершины обозначаются греческими буквами 
и~соответствуют преобразованиям информации (блокам преобразований) вместе 
с~входными и~выходными данными. Преобразования обозначаются строчными 
латинскими буквами, а~входные и~выходные данные~--- прописными 
латинскими буквами. Дуги DAG соответствуют передаче данных блокам от 
предыдущих блоков, т.\,е.\ дуга выходит из выходных данных выполненного 
преобразования и~входит во входные данные для сле\-ду\-юще\-го преобразования. 
Существуют дуги, входящие в~вершины DAG как исходные данные извне ИТ, 
и~существуют выделенные дуги, идущие вне DAG к~потребителям результатов 
ИТ. Все указанные дуги фиксированы и~считаются легальными. Определенные 
таким образом модели ИТ будем называть метаданными.
  
  Одной из важных угроз организации пред\-став\-ля\-ет\-ся возможность 
существования ИП, которые используются для утечки ценной информации из ИТ. 
Далее ИП, не участвующие в~функционирующих ИТ, но исходящие из данных ИТ 
или входящие в~них, будем называть нелегальными ИП, или аномалиями. 
В~работе поиск аномальных ИП проводится исходя из противоречий с~МД, 
определяемых с~помощью DAG. Исходными данными для анализа служат 
результаты компьютерного аудита~[3, 4], в~которых отражены имена субъектов, 
инициировавших ИП, идентификаторы получателей ИП и~время производимых 
действий. Таким образом, последовательность легальных ИП соответствует МД. 
  
  Проблема восстановления структуры ИП по данным 
логов рассматривалась в~ряде научных работ, например в~[5, 6]. В~качестве 
модели структур, восстанавливаемых по логам, предлагались сети Петри. Однако 
модели сетей Петри представляют собой сложные структуры, и~поэтому их 
вос\-ста\-нов\-ле\-ние требует сложных алгоритмов. Кроме того, для сетей Петри не 
найдены алгоритмы иерархической декомпозиции. В~то же время иерархическая 
декомпозиция DAG~\cite{1-gt} позволяет отслеживать не все ИП (часть ИП 
невозможно отследить). 
  
  В работе рассматривается задача восстановления множества DAG, 
соответствующих дей\-ст\-ву\-ющим в~данный период времени ИТ. Отметим связь 
между МД, рассматривавшимися в~предыдущих работах~[7--9], касавшихся 
контроля соединений в~сети. Если ИП проходил через сеть, то его содержание 
могло быть зашифровано, т.\,е.\ определялись входные и~выходные блоки сети, где 
формировались сообщения для передачи по сети, и~в этих блоках использовались 
криптографические средства. Протокол организации сетевого обмена мог 
опираться на несколько удаленных субъектов, однако это учитывалось 
в~организации МД~[10]. 
  
  Отметим, что при определении легальности ИП необходимо учитывать 
присутствие дуг в~других ИТ, а также возможность кратного использования одной 
ИТ. В этом множестве ИП необходимо найти нелегальные ИП. 
  
  В работе рассматривается проведение анализа по поиску нелегальных ИП 
в~режиме офлайн с~использованием следующей идеи. Для восстановления всех 
действующих в~данное время ИТ предлагается использовать метод 
трансверсалей~[11, 12]. Трансверсали (системы различных представителей, СРП) 
позволяют быстро и~однозначно вос\-ста\-нав\-ли\-вать DAG ИТ, выполняемых в~данное 
время, т.\,е.\ по СРП вос\-ста\-нав\-ли\-ва\-ют\-ся МД, в~соответствии с~которыми 
поступают данные аудита. После восстановления МД из данных аудита убирается 
информация о~легальных ИП. Тогда оставшиеся ИП оказываются аномалиями, 
которые подлежат дальнейшему анализу службы безопасности. 
  
  \section{Выявление нелегальных информационных потоков при~наличии системы
  различных представителей}
  
  Пусть в~рассматриваемых данных функционировали 
ИТ $\mathrm{IT}_1, \ldots , \mathrm{IT}_L$. Этим технологиям соответствуют 
МД 
$G_1,\ldots , G_L$, при этом множеству DAG $G_1,\ldots , G_L$ соответствуют 
множества вершин $V_1, \ldots, V_L$ и~множества дуг $E_1,\ldots , E_L$. В~тех 
случаях, когда в~ИТ вносятся внешние данные, можно формально определить 
вершину~$\alpha_0$, из которой выходит дуга в~соответствующую вершину из 
множества~$V$, причем эту дугу можно включить в~множество~$E$. Аналогично 
если предусматривается легальный ИП из~$G$, то можно ввести формальную 
вершину~$\alpha_1$, в~которую из соответствующей вершины графа~$G$ направлена 
дуга с~выходящим из~$G$ ИП. 
  
  Воспользуемся определением СРП и~формулировкой теоремы Ф.~Холла из 
работы~\cite{12-gt}. Пусть~$E$~--- множество всех дуг графов $G_1,\ldots , G_L$, 
$\mathbf{P}(E)$~--- множество подмножеств множества~$E$, $D\hm= (e_1,\ldots 
,e_L)$~--- выборка из множества~$E$.
  
  \smallskip
  
  \noindent
  \textbf{Определение.} Пусть~$L$ элементов выборки~$D$ отличны друг от 
друга и~$e_i\hm\in E_i$, $i\hm=1,\ldots , L$. Тогда~$D$ называется \textit{системой 
различных представителей} множеств $E_1,\ldots , E_L$. 
  
  \smallskip
  
  Условия существования СРП получены в~теореме Ф.~Холла~\cite{12-gt}.
  
  \smallskip
  
  \noindent
  \textbf{Теорема~1}~\cite{12-gt}. \textit{Множества $E_1,\ldots , E_L$ имеют 
СРП тогда и~только тогда, когда для каждого $k\hm=1,\ldots , L$ и~всех 
сочетаний индексов $i_1,\ldots , i_k$ выполняются условия}:
  $$
  \left\lfloor  \bigcup\limits_{j=1}^k E_{i_j}\right\rfloor \geq k\,.
  $$
  
  Пусть множества $E_1, \ldots , E_L$ удовлетворяют тео\-ре\-ме~1. 
  
  \smallskip
  
  \noindent
  \textbf{Теорема~2.} \textit{В~последовательности данных обязательно 
встретятся все элементы $e_1,\ldots , e_s$ из СРП тогда и~только тогда, когда 
одновременно функционируют различные} $\mathrm{IT}_{i_1}, \ldots , \mathrm{IT}_{i_s}$.
  
  \smallskip
  
  \noindent
  Д\,о\,к\,а\,з\,а\,т\,е\,л\,ь\,с\,т\,в\,о\,. \textit{Достаточность}. При анализе данных 
в~режиме офлайн в~последовательности данных все запущенные ИТ можно 
считать завершившимися. Тогда в~каждом графе~$G_{i_j}$, $j\hm=1, \ldots , s$, 
должны быть пройдены все дуги. Значит, должны быть пройдены все дуги, 
принадлежащие СРП. 
  
  \textit{Необходимость}. Если в~последовательности данных встретился элемент 
СРП, то однозначно определен граф, которому принадлежит эта дуга. Это 
означает, что однозначно определена ИТ, моделью которого служит этот граф. 
Теорема доказана.
  
  \smallskip
  
  \noindent
  \textbf{Следствие.} Если в~последовательности данных встретился элемент 
СРП, то в~данных должны встретиться все дуги, принадлежащие 
соответствующему графу модели ИТ. 
  
  Отсюда следует алгоритм определения легальных ИП. Сначала в~данных 
находятся все элементы СРП, по ним восстанавливаются соответствующие DAG 
и~ИТ. Далее выявляются все дуги этих DAG в~данных. Построенное множество 
ИП является легальным для всех восстановленных DAG. 
  
  Как было отмечено выше, одновременно могут функционировать несколько 
копий одной ИТ. В~условиях существования СРП выявление легальных ИП 
в~этом случае не выглядит сложным. Пусть одна ИТ запущена~$r$~раз, тогда 
элемент СРП соответствующего DAG встретится~$r$~раз. Это означает, что 
необходимо собрать~$r$~экземпляров данного DAG. Очевидно, что в~этих графах 
дуги могут быть переставлены и~относиться к~разным экземплярам DAG, но 
несмотря на это совокупность вос\-ста\-нов\-лен\-ных графов будет правильно 
определять легальные ИП. 
  
  В случае существования СРП удаление легальных ИП из последовательности 
данных определяет множество нелегальных ИП, анализировать которые 
необходимо отдельно. Отсюда следует, что СРП значительно упрощает поиск 
нелегальных ИП. 
  
  \section{Определение легальных информационных потоков 
  в~случае отсутствия системы различных представителей}
  
  Рассмотрим случай, когда в~системе множеств $E_1, \ldots , E_L$ нет СРП. Это 
означает, что существует набор подмножеств дуг $E_{i_1}, \ldots , E_{i_k}$, не 
удовлетворяющих теореме~1, т.\,е.
$$
\left\lfloor \bigcup\limits^k_{j=1} E_{i_j} \right\rfloor 
\hm <k\,.
$$

 Тогда имеет место следующая цепочка неравенств:
  \begin{multline*}
  k> \left\vert E_{i_1} \cup \cdots \cup E_{i_k}\right\vert \geq{}\\
  {}\geq \left\vert
  E_{i_1}\cup\cdots\cup E_{i_{k-1}}\right\vert \geq \cdots \geq \left\vert 
E_{i_1}\right\vert  =t>0\,.
  \end{multline*}
В этой цепочке~$k$ неравенств, из которых строгими могут быть не более $k-t$, 
остальные~$t$ неравенств являются равенствами. Пусть для некоторого $1\hm< 
s\hm< k$ справедливо равенство:
$$
\left\vert E_{i_1} \cup\cdots \cup E_{i_s}\right\vert =\left\vert E_{i_1} \cup\cdots\cup 
E_{i_{s-1}}\right\vert\,.
$$
Отсюда следует, что 
$$
E_{i_s}\subseteq E_{i_1}\cup\cdots\cup E_{i_{s-1}}\,,
$$
т.\,е.\ дуги графа~$G_{i_s}$ являются частью объединения дуг других DAG.

  Найденные условия позволяют реализовать другой подход к~использованию 
СРП. Избавляясь от таких включений, разобьем множество $E_1,\ldots , E_L$ 
на~$M$~групп $E_1^{(1)},\ldots , E^{(1)}_{L_1}$, $E_1^{(2)},\ldots , 
E^{(2)}_{L_2}$, \ldots , $E_1^{(M)},\ldots , E^{(M)}_{L_M}$, в~каждой из 
которых есть СРП. Тогда встреча дуги в~$i$-й СРП позволяет восстановить один 
из DAG этой группы. Если эта дуга встречается в~нескольких СРП, то в~каждой 
группе она восстанавливает DAG. Если дуга из данного СРП не позволила 
восстановить DAG, то эта дуга может принадлежать другой группе DAG. Тогда 
СРП этой другой группы позволит восстановить DAG, в~который войдет 
указанная дуга. Если дуга $i$-й СРП не породила из данных аудита DAG этой 
группы и~эта дуга не попала ни в~один из DAG других групп, то это~--- 
нелегальный ИП. 
  
  Основной критерий завершенности выявления легальных ИП~--- это полное 
восстановление всех DAG, элементы которых возникли из разных СРП 
с~включенными в~них дугами СРП других групп. Данный метод повышает 
эффективность анализа данных, если групп мало. Однако следующий пример 
показывает, что метод работает при любом чис\-ле групп. 
  
  \smallskip
  
  \noindent
  \textbf{Пример.} Пусть ИТ порождают графы $G_1,\ldots , G_L$ и~множества 
дуг $E_1, \ldots , E_L$. Пусть каждое множество дуг есть отдельная группа, тогда 
каждая дуга может быть взята как СРП в~своей группе. Если встретилась дуга из 
$E_i$ и~граф~$G_i$ был восстановлен из данных, то множества ИП дуг 
графа~$G_i$ являются легальными. Если эта же дуга принадлежит другому 
графу~$G_j$ со своим СРП и~граф~$G_j$ полностью восстановлен, то все его 
дуги также порождают легальные ИП. Если дуга не попала ни в~один из 
восстановленных графов~$G_i$, то она представляет нелегальный ИП. После 
восстановления~$G_i$ нельзя исключать дуги из рассмотрения, так как другой 
граф может оказаться недостроенным. Поэтому дуги, соответствующие 
восстановленным легальным ИП, до окончания работы алгоритма необходимо 
<<раскрашивать>>.  
  
  \section{Заключение }
  
  В множестве ИТ, модели которых представимы в~виде DAG (МД), 
можно ввести понятие легальных и~нелегальных ИП. Тогда каждая ИТ может 
быть представлена множеством дуг соответствующего DAG. Легальность ИП 
может быть проверена на основе СРП, по которым сами DAG восстанавливаются 
однозначно. Нелегальные ИП определяются после устранения всех легальных ИП 
из данных компьютерного аудита. 
  
  Этот метод может быть распространен на случай, когда в~исходной 
  информационной системе не  существует СРП. 
  
{\small\frenchspacing
 {%\baselineskip=10.8pt
 \addcontentsline{toc}{section}{References}
 \begin{thebibliography}{99}

\bibitem{2-gt}
\Au{Самуйлов К.\,Е., Чукарин~А.\,В., Яркина~Н.\,В.} Биз\-нес-про\-цес\-сы и~информационные 
технологии в~управлении телекоммуникационными компаниями.~--- М.: Альпина Паблишерс, 
2009. 442~с.

\bibitem{1-gt} %2
\Au{Грушо Н.\,А., Грушо~А.\,А., Забежайло~М.\,И., Тимонина~Е.\,Е.} Методы нахождения 
причин сбоев в~информационных технологиях с~помощью метаданных~// Информатика и~её 
применения, 2020. Т.~14. Вып.~2. С.~33--39.
 

\bibitem{4-gt} %3
Department of Defense Trusted Computer System Evaluation Criteria.
DoD 5200.28-STD.~--- U.S.\ National Institute of 
Standards and Technology, Department of Defense, 1985. {\sf 
http://csrc.nist.gov/\linebreak publications/history/dod85.pdf}.

\bibitem{3-gt} %4
\Au{Грушо А.\,А., Применко~Э.\,А., Тимонина~Е.\,Е.} Теоретические основы компьютерной 
безопасности.~--- М.: Академия, 2009. 272~с.

\bibitem{5-gt}
\Au{Aalst W., Weijters~T., Maruster~L.} Workflow mining: Discovering process models from event 
logs~// IEEE~T. Knowl. Data En., 2004. Vol.~16. Iss.~9. P.~1128--1142. 
\bibitem{6-gt}
\Au{Bezerra F., Weiner~J.} Algorithms of anomaly detection of traces in logs of process aware 
information systems~// Inform. Syst., 2013. Vol.~38. Iss.~1. P.~33--44. 
\bibitem{7-gt}
\Au{Grusho A., Grusho~N., Timonina~E.} Detection of anomalies in non-numerical data~// 8th 
Congress (International) on Ultra Modern Telecommunications and Control Systems and Workshops 
Proceedings.~--- Piscataway, NJ, USA: IEEE, 2016. P.~273--276.
\bibitem{8-gt}
\Au{Грушо А.\,А., Тимонина~Е.\,Е., Шоргин~С.\,Я.} Иерархический метод порождения 
метаданных для управления сетевыми соединениями~// Информатика и~её применения, 2018. 
Т.~12. Вып.~2. С.~44--49.
\bibitem{9-gt}
\Au{Grusho A., Grusho~N., Timonina~E.} Information flow control on the basis of meta data~//  
Distributed computer and communication networks, 22nd International Conference, Revised Selected 
Papers~/ Eds. V.\,M.~Vishnevskiy, K.\,E.~Samouylov, and D.\,V.~Kozyrev.~---
Lecture notes in computer science ser.~--- Springer, 2019,  Vol.~11965. P.~548--562.
\bibitem{10-gt}
\Au{Grusho A.\,A., Timonina~E.\,E., Shorgin~S.\,Ya.} Modelling for ensuring information security of 
the distributed information systems~// 31th European Conference on Modelling and Simulation 
Proceedings.~--- Dudweiler, Germany: Digitaldruck Pirrot GmbH, 2017. P.~656--660.


\bibitem{12-gt}
\Au{Райзер Г.\,Дж.} Комбинаторная математика~/ Пер. с~англ. К.\,А.~Рыбникова.~--- М.: Мир, 
1966. 154~с. (\Au{Ryser~H.\,J.} {Combinatorial mathematics.}~--- New York, NY, USA: Wiley, 1963. 
154~p.)

\bibitem{11-gt}
\Au{Холл М.} Комбинаторика~/ Пер. с~англ. С.\, А.~Широковой.~--- М.: Мир, 1970. 424~с.
(\Au{Hall~M., Jr.} {Combinatorial theory.}~---
 New York, NY, USA: Wiley, 1967. 424~p.)

\end{thebibliography}

 }
 }

\end{multicols}

\vspace*{-6pt}

\hfill{\small\textit{Поступила в~редакцию 15.07.20}}

\vspace*{8pt}

%\pagebreak

%\newpage

%\vspace*{-28pt}

\hrule

\vspace*{2pt}

\hrule

%\vspace*{-2pt}

\def\tit{IDENTIFYING ANOMALIES USING METADATA}

\def\titkol{Identifying anomalies using metadata}

\def\aut{A.\,A.~Grusho$^1$, E.\,E.~Timonina$^1$, N.\,A.~Grusho$^1$, 
and~I.\,Yu.~Teryokhina$^2$}

\def\autkol{A.\,A.~Grusho, E.\,E.~Timonina, N.\,A.~Grusho, 
and~I.\,Yu.~Teryokhina}

\titel{\tit}{\aut}{\autkol}{\titkol}

\vspace*{-9pt}


\noindent
$^1$Institute of Informatics Problems, Federal Research Center 
``Computer Sciences and Control'' of the Russian\linebreak
$\hphantom{^1}$Academy of Sciences, 
44-2~Vavilov Str., Moscow 119133, Russian Federation


\noindent
$^2$Faculty of Computational Mathematics and Cybernetics, 
M.\,V.~Lomonosov Moscow State University, 
1-52~Lenin-\linebreak
$\hphantom{^1}$skiye Gory, GSP-1, Moscow 119991, Russian Federation

\def\leftfootline{\small{\textbf{\thepage}
\hfill INFORMATIKA I EE PRIMENENIYA~--- INFORMATICS AND
APPLICATIONS\ \ \ 2020\ \ \ volume~14\ \ \ issue\ 3}
}%
 \def\rightfootline{\small{INFORMATIKA I EE PRIMENENIYA~---
INFORMATICS AND APPLICATIONS\ \ \ 2020\ \ \ volume~14\ \ \ issue\ 3
\hfill \textbf{\thepage}}}

\vspace*{3pt} 


\Abste{The paper discusses the problem of information technology 
security control based on computer audit data. These data are the sequence 
of small samples, each of which describes the transmission of information 
from one transformation to another. Information technologies are 
represented by mathematical models in the form of oriented acyclic 
graphs. In the article, such graphs describing data transmission are 
called metadata. In integrated computer audit data, there may be at the 
same time traces of the execution of several information technologies 
described by their graphs. This makes it difficult to recognize 
information flows that correspond to arcs of different graphs. The 
concept of legal information flow is introduced in the paper, which 
corresponds to the transfer of data of all information technologies 
being performed. Information flows that do not correspond to the 
execution of existing information technologies are called illegal 
or anomalies. Such information flows can occur due to hostile activities 
of insiders or due to errors in user actions. The article solves 
the problem of effective identification of legal information flows 
and anomalies on the basis of metadata.}

\KWE{information security; information flow; anomalies; metadata; 
systems of different representatives}

\DOI{10.14357/19922264200311}

%\vspace*{-20pt}

\Ack
\noindent
The paper was partially supported by the Russian Foundation 
for Basic Research (project 18-07-00274).

%\vspace*{6pt}

 \begin{multicols}{2}

\renewcommand{\bibname}{\protect\rmfamily References}
%\renewcommand{\bibname}{\large\protect\rm References}

{\small\frenchspacing
 {%\baselineskip=10.8pt
 \addcontentsline{toc}{section}{References}
 \begin{thebibliography}{99}

\bibitem{2-gt-1}
\Aue{Samuylov, K.\,E., A.\,V.~Chukarin, and N.\,V.~Yarkina.} 2009. \textit{Biznes-protsessy 
i~informatsionnye tekhnologii v~upravlenii telekommunikatsionnymi kompaniyami} [Business 
processes and information technologies in management of the telecommunication companies]. 
Moscow: Alpina Pabls. 442~p.

\bibitem{1-gt-1}
\Aue{Grusho, N.\,A., A.\,A.~Grusho, M.\,I.~Zabezhailo, and E.\,E.~Timonina.} 2020. Metody 
nakhozhdeniya prichin sboev v informatsionnykh tekhnologiyakh s pomoshch'yu metadannykh 
[Methods of finding the causes of information technology failures by means of meta data]. 
\textit{Informatika i~ee Primeneniya~--- Inform. Appl.} 14(2):33--39.


\bibitem{4-gt-1}
DoD 5200.28-STD. 1985. Department of Defense Trusted Computer System Evaluation Criteria. 
Available at: {\sf http://csrc.nist.gov/publications/history/dod85.pdf} (accessed July~14, 2020).

\bibitem{3-gt-1}
\Aue{Grusho, A.\,A., E.\,A.~Primenko, and E.\,E.~Timonina.} 2009. \textit{Teoreticheskie osnovy 
komp'yuternoy bezopasnosti} [Theoretical bases of computer security]. Moscow: Academy. 272~р.

\bibitem{5-gt-1}
\Aue{Aalst, W., T.~Weijters, and L.~Maruster}. 2004. Workflow mining: Discovering process models 
from event logs. \textit{IEEE~T. Knowl. Data En.} 16(9):1128--1142.
\bibitem{6-gt-1}
\Aue{Bezerra, F., and J.~Weiner.} 2013. Algorithms of anomaly detection of traces in logs of process 
aware information systems. \textit{Inform. Syst.} 38(1):33--44.
\bibitem{7-gt-1}
\Aue{Grusho, A., N.~Grusho, and E.~Timonina.} 2016. Detection of anomalies in non-numerical data. 
\textit{8th Congress (International) on Ultra Modern Telecommunications and Control Systems and 
Workshops Proceedings}. Piscataway, NJ: IEEE. 273--276.
\bibitem{8-gt-1}
\Aue{Grusho, A.\,A., E.\,E.~Timonina, and S.\,Ya.~Shorgin.} 2018. Ierarkhicheskiy metod 
porozhdeniya metadannykh dlya upravleniya setevymi soedineniyami [Hierarchical method of meta 
data generation for control of network connections]. \textit{Informatika i~ee Primeneniya~--- Inform. 
Appl.} 12(2):44--49.
\bibitem{9-gt-1}
\Aue{Grusho, A.\,A., N.\,A.~Grusho, and E.\,E.~Timonina.} 2019. Information flow control on the 
basis of meta data. \textit{Distributed computer and communication networks, 22nd International 
Conference, Revised Selected Papers.} Eds. V.\,M.~Vishnevskiy, K.\,E.~Samouylov, and 
D.\,V.~Kozyrev. Lecture notes in computer science ser. Springer. 11965:548--562.
\bibitem{10-gt-1}
\Aue{Grusho, A.\,A., E.\,E.~Timonina, and S.\,Ya.~Shorgin.} 2017. Modelling for ensuring 
information security of the distributed information systems. \textit{31th European Conference on 
Modelling and Simulation Proceedings}. Dudweiler, Germany: Digitaldruck Pirrot GmbH. 656--660. 

\bibitem{12-gt-1}
\Aue{Ryser, H.\,J.} 1963. \textit{Combinatorial mathematics.} New York, NY: Wiley. 154~p.

\bibitem{11-gt-1}
\Aue{Hall, M., Jr.} 1967. \textit{Combinatorial theory.} New York, NY: Wiley. 424~p.

\end{thebibliography}

 }
 }

\end{multicols}

\vspace*{-6pt}

\hfill{\small\textit{Received July 15, 2020}}

%\pagebreak

%\vspace*{-24pt}

\Contr

\noindent
\textbf{Grusho Alexander A.} (b.\ 1946)~--- Doctor of Science in physics and mathematics, professor, 
principal scientist, Institute of Informatics Problems, Federal Research Center ``Computer Sciences 
and Control'' of the Russian Academy of Sciences, 44-2~Vavilov Str., Moscow 119133, Russian 
Federation;  \mbox{grusho@yandex.ru}

\vspace*{3pt}

\noindent
\textbf{Timonina Elena E.} (b.\ 1952)~--- Doctor of Science in technology, professor, leading scientist, 
Institute of Informatics Problems, Federal Research Center ``Computer Sciences and Control'' of the 
Russian Academy of Sciences, 44-2 Vavilov Str., Moscow 119133, Russian Federation; 
\mbox{eltimon@yandex.ru}

\vspace*{3pt}

\noindent
\textbf{Grusho Nikolai A.} (b.\ 1982)~--- Candidate of Science (PhD) in physics and mathematics, 
senior scientist, Institute of Informatics Problems, Federal Research Center ``Computer Sciences and 
Control'' of the Russian Academy of Sciences, 44-2~Vavilov Str., Moscow 119133, Russian 
Federation; \mbox{info@itake.ru}

\vspace*{3pt}

\noindent
\textbf{Teryokhina Irina Yu.} (b.\ 1994)~--- PhD student, Faculty of Computational Mathematics and 
Cybernetics, M.\,V.~Lomonosov Moscow State University, 1-52~Leninskiye Gory, GSP-1, Moscow 
119991, Russian Federation; \mbox{irina\_teryokhina@mail.ru}
\label{end\stat}

\renewcommand{\bibname}{\protect\rm Литература} 