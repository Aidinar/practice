

%\newcommand{\e}{\mathrm{e}}
%\renewcommand{\i}{\mathrm{i}}

\def\stat{razumchik}

\def\tit{ОДНОЛИНЕЙНАЯ СИСТЕМА МАССОВОГО ОБСЛУЖИВАНИЯ 
С~ИНВЕРСИОННЫМ ПОРЯДКОМ ОБСЛУЖИВАНИЯ 
С~ВЕРОЯТНОСТНЫМ ПРИОРИТЕТОМ, ГРУППОВЫМ ПУАССОНОВСКИМ ПОТОКОМ 
И~ФОНОВЫМИ ЗАЯВКАМИ$^*$}

\def\titkol{Однолинейная система массового обслуживания 
с~инверсионным порядком обслуживания} 
%с~вероятностным приоритетом, групповым пуассоновским потоком 
%и~фоновыми заявками}

\def\aut{Т.\,А.~Милованова$^1$, Р.\,В.~Разумчик$^2$}

\def\autkol{Т.\,А.~Милованова, Р.\,В.~Разумчик}

\titel{\tit}{\aut}{\autkol}{\titkol}

\index{Милованова Т.\,А.}
\index{Разумчик Р.\,В.}
\index{Milovanova T.\ A.}
\index{Razumchik R.\,V.}
 

{\renewcommand{\thefootnote}{\fnsymbol{footnote}} \footnotetext[1]
{Исследование выполнено при финансовой поддержке РФФИ (проект №\,20-07-00804).}}


\renewcommand{\thefootnote}{\arabic{footnote}}
\footnotetext[1]{Российский
университет дружбы народов, \mbox{milovanova-ta@rudn.ru}}
\footnotetext[2]{Институт проблем информатики Федерального исследовательского
центра <<Информатика и~управление>> Российской академии наук,
\mbox{rrazumchik@ipiran.ru}}

%\vspace*{-6pt}


\Abst{Рассматривается система массового обслуживания~(СМО) с~одним прибором, групповым 
пуассоновским потоком основных заявок и~насыщенным потоком фоновых заявок, запас 
которых не ограничен. Основные заявки имеют относительный приоритет перед заявками 
фонового потока. Таким образом, прерывание обслуживания заявки второго типа не 
допускается и~процесс обслуживания фоновых заявок начинается только тогда, когда 
после выхода с~прибора основной заявки очередь основных заявок пуста. Времена 
обслуживания основных и~фоновых заявок независимы и~имеют произвольные распределения. 
В~системе реализована дисциплина инверсионного порядка обслуживания с~вероятностным 
приоритетом. Предложен метод расчета (в терминах преобразований) основных 
стационарных показателей функционирования системы, включая стационарные 
распределения времени ожидания начала обслуживания и~времени пребывания заявок 
основного потока в~системе.}


\KW{инверсионный порядок обслуживания; вероятностный приоритет;
групповой поток; фоновые заявки}

\DOI{10.14357/19922264200304} 
 
%\vspace*{-6pt}


\vskip 10pt plus 9pt minus 6pt

\thispagestyle{headings}

\begin{multicols}{2}

\label{st\stat}

\section{Введение}

Статья посвящена развитию результатов работ \cite{R6,new1,R3,R4}
по исследованию стационарных харак\-те\-ри\-стик %систем массового обслуживания~(
СМО с~так называемыми фоновыми заявками, находящих применение при 
разработке энерго\-сбе\-ре\-га\-ющих схем в~сенсорных беспроводных \mbox{сетях}
(см., например,~\cite{R9}). Рассматривается СМО с~одним прибором, 
инверсионным порядком обслуживания с~вероятностным приоритетом 
и~двумя входящими потоками заявок разных классов: групповым 
пуассоновским потоком основных заявок и~ординарным потоком фоновых 
заявок. Источником заявок фонового потока служит так называемый бункер, 
в~котором запас заявок не ограничен: если нет очереди основных заявок, 
то начинается процесс обслуживания фоновых  заявок\footnote[3]{Такой 
поток заявок иногда называют насыщенным.}.
Заявки основного потока имеют относительный приоритет перед заявками 
фонового потока. С~помощью специального метода, введенного в~\cite{n4}, 
здесь решена задача отыскания (в терминах преобразований) стационарных 
вероятностей состояний, \mbox{а~также} стационарных распределений времени 
ожидания начала обслуживания и~времени пребывания заявок основного 
потока в~системе. Заметим, что рассмотренная система может быть трактована 
как система с~прогулками (отключением прибора) при опустошении системы 
от заявок основного потока (см., например, \cite{R5,R7}). Таким образом, 
полученные здесь результаты обобщают некоторые из известных для этого 
типа систем результаты и,~в~частности, позволяют находить стационарное 
распределение вероятностей состояний при прямом порядке обслуживания 
(т.\,е.\ при дисциплине FIFO~--- first in, first out).

\vspace*{-9pt}

\section{Описание системы}
\vspace*{-2pt}

Далее основные (фоновые) заявки будем называть заявками первого (второго) 
типа. Рассмотрим СМО %систему массового обслуживания 
с~одним прибором 
и~инверсионным порядком обслуживания с~вероятностным приоритетом,
в~которую поступают заявки двух типов. Заявки первого типа поступают в~систему 
группами в~соответствии с~пуассоновским потоком с~параметром~$\lambda$.
Вероятность поступления~$k$, $k\hm\ge 1$, заявок в~группе обозначим через~$c_k$, 
а~средний размер группы~--- через $\overline{c}\hm= \sum\nolimits_{k=1}^\infty k c_k$. 
Заявки первого типа обслуживаются по одной, и~их времена обслуживания (длины) 
являются независимыми случайными величинами с~функцией распределения~$B(x)$ 
и~средним значением $\overline{b} \hm=\int\nolimits_0^\infty (1-B(x))\,dx \hm< \infty$. 
Заявки второго типа поступают из накопителя бесконечной емкости, и~их 
длины тоже независимы с~функцией распределения~$G(x)$ и~средним 
значением $\overline{g}\hm=\int\nolimits_0^\infty (1-G(x))\, dx \hm< \infty$. 

Для простоты изложения будем полагать\footnote{Как и~в~системе с~ординарным 
потоком заявок первого типа (см.~\cite{new1}), случай произвольных распределений 
длин заявок можно трактовать в~терминах обобщенных функций.}, 
что существуют плотности $b(x)\hm=B'(x)$ и~$g(x)\hm=G'(x)$.
Заявки первого типа имеют относительный приоритет перед заявками второго типа, 
т.\,е.\ поступление на прибор заявки второго типа происходит только в~том случае, 
если в~системе отсутствуют заявки первого типа (прерывание обслуживания заявок 
второго типа не допускается). Общее число заявок первого типа в~системе 
ограничено числом~$N$, $N\hm\le\infty$. При $N\hm<\infty$ считается, что 
поступающая группа теряется целиком, если в~момент поступления для хотя бы 
одной из заявок в~группе не хватает места в~очереди.

Принятый в~системе инверсионный порядок обслуживания с~вероятностным 
приоритетом заключается в~следующем:
\begin{enumerate}[(1)]
\item если в~момент поступления группы заявок первого типа в~системе 
обслуживается заявка первого типа, то длина~$x$ первой из по\-сту\-па\-ющей 
группы заявки сравнивается с~остаточной длиной~$y$ заявки, 
находящейся на приборе. С~вероятностью~$v(y,x)$ поступающая 
группа занимает первые места в~очереди, а~заявки, находившиеся в~очереди 
до поступления группы, становятся за ними с~учетом порядка. 
С~дополнительной вероятностью $\overline{v}(y,x)\hm=1\hm-v(y,x)$ 
первая заявка из поступающей группы становится на прибор, остальные 
заявки из поступающей группы занимают первые места в~очереди, заявка с~прибора 
встает за ними. Остальные заявки, находившиеся в~очереди до поступления 
новой группы, становятся после этой заявки с~сохранением порядка;

\item
если группа заявок первого типа в~момент поступления застает на 
приборе заявку второго типа, то длина~$x$ первой из поступающей группы 
заявки сравнивается с~длиной~$y$ заявки, стоящей на первом месте в~очереди. 
С~вероятностью~$w(y,x)$ заявка длины~$y$ остается на первом месте в~очереди, 
поступающая группа занимает места в~очереди начиная со второго, а~остальные 
заявки, имевшиеся в~очереди до поступления новой группы, становятся за 
ними с~сохранением порядка. С~дополнительной вероятностью 
$\overline{w}(y,x)\hm=1\hm-w(y,x)$ поступа\-ющая группа заявок 
занимает первые места в~очереди, заявка длины~$y$ 
становится за поступившей группой заявок, а~заявки, находившиеся в~очереди 
до момента поступления группы, становятся за ней с~учетом порядка.
\end{enumerate}

\vspace*{-6pt}

\section{Стационарное распределение числа заявок в~системе}

Рассмотрим $\nu(t)$~--- общее число заявок в~сис\-те\-ме в~момент~$t$ 
и~$\vec\xi(t)\hm=(\xi_{1}(t),\ldots,\xi_{\nu(t)}(t))$~--- 
вектор, координатой~$\xi_{1}(t)$ которого
является остаточное время обслуживания
заявки, находящейся в~этот момент на приборе,
$\xi_{2}(t)$~--- первой заявки в~очереди$,\ldots,$ $\xi_{\nu(t)}(t)$~---
последней заявки в~очереди.
Тогда если обозначить через~$\chi(t)$~--- 
величину, равную в~момент~$t$ единице, когда прибор занят 
обслуживанием заявки первого типа,
и~нулю в~противном случае,
то процесс $(\chi(t),\nu(t),\vec\xi(t))$ будет марковским.
Введем необходимые для дальнейшего изложения функции:
\begin{description}
\item[\,]
$q_k(t,x)$, $0 \le k \le N-1$,~--- стационарная 
плотность\footnote{Определяется как $q_k(t,x)\hm=
\partial^2 Q_{k}(t,x)/(\partial t \partial x)$, 
где $Q_{k}(t,x)\hm=\lim\nolimits_{T\to\infty}
{\bf P}\{\chi(T)\hm=0, \nu(T)=k+1, \xi_{1}(T)\hm<t,
\xi_{2}(T)<x\}$. При $k\hm=0$ аргумент~$x$ опускается. Можно показать, 
что эта и~другие введенные плотности существуют.} вероятности того, 
что на приборе обслуживается заявка второго типа длины~$t$ и~в~очереди 
находятся~$k$~заявок первого типа, причем заявка, стоящая в~очереди 
первой, имеет длину~$x$;
\item[\,]
$q^*_N(t,x,y)$ --- стационарная плотность 
вероят\-ности того, что на приборе обслуживается заявка 
второго типа длины~$t$ и~в~очереди находятся~$n$~заявок 
первого типа, причем длины первых двух заявок в~очереди~$x$ и~$y$;
\item[\,]
$p_k(x,y)$, $1 \le k \hm\le N-1$,~--- стационарная плотность вероятности 
того, что на приборе обслуживается заявка первого типа длины~$x$ и~в~очереди 
находятся $k-1$~заявок первого типа, причем заявка, стоящая 
в~очереди первой, имеет длину~$y$;
\item[\,]
$p^*_N(x,y)$~--- стационарная плотность вероятности того, 
что на приборе обслуживается заявка первого типа длины~$x$ и~в~очереди 
находятся $N-1$ заявок первого типа, причем заявка, стоящая 
в~очереди первой, имеет длину~$y$.
\end{description}

Обозначим через 
%\vspace*{2pt}

\noindent
$$P_k\hm=\int\limits_0^\infty \int\limits_0^\infty 
p_k(x,y)\,dy dx\,, 1 \hm\le k \hm\le N-1\,;
$$

\pagebreak

\noindent
$$P^*_N
\hm=\int\limits_0^\infty \int\limits_0^\infty p^*_N(x,y)\,dy dx
$$ 
стационарные вероятности
того, что на приборе обслуживается заявка первого типа и~в~очереди 
находятся $k-1$ заявок первого типа. Соответственно, 
\begin{align*}
Q_k&=\int\limits_0^\infty \int\limits_0^\infty q_k(t,x)\,dx dt\,, \enskip
0 \hm\le k \hm\le N-1\,;\\
Q^*_N&=
\int\limits_0^\infty \int\limits_0^\infty \int\limits_0^\infty 
q^*_N(t,x,y)\,dy dx dt
\end{align*}
суть стационарные вероятности того, 
что в~системе находятся~$k$~заявок первого типа и~обслуживается заявка 
второго типа.

Для нахождения стационарного распределения числа заявок в~системе 
воспользуемся методом производящих функций (ПФ).
Для этого, не связывая систему с~конкретным значением~$N$, 
рассмотрим набор систем с~различными $N\hm>1$ и~введем %производящие функции
ПФ\footnote{Введенная ПФ $P(s,x)$ имеет смысл обычной %производящей функции 
ПФ только при загрузке системы заявками первого типа
$\lambda \overline{c} \overline{b} \hm<1$.}
\begin{align}
\label{pf1}
P(s,x)&=\sum\limits_{k=1}^\infty p_k(x)s^k\,;
\\
\label{pf2}
Q(s,t,x)&=\sum\limits_{k=1}^\infty q_k(t,x)s^k\,;
\\
\label{pf3}
Q^*(s,t,z,x)&=\sum\limits_{k=1}^\infty q^*_{k+1}(t,z,x)s^k\,,
\end{align}
где 
$$p_k(x) = \int\limits_0^\infty p_k(x,y)\,dy\,.
$$
Начнем с~нахождения ПФ~\eqref{pf2} и~\eqref{pf3}. 
Система уравнений Кол\-мо\-го\-ро\-ва--Чеп\-ме\-на 
для стационарных плотностей $q_k(t,x)$ и~$q^*_N(t,x,y)$ 
имеет вид\footnote{Для сокращения
записи здесь и~далее используются соглашения, что $\sum\nolimits_{j=i}^{i-1} 
\hm\equiv 0$ и~$\prod\nolimits_{j=i}^{i-1} \hm\equiv 1$.}:
\begin{align}
\label{q0}
-q'_0(t)&=-\lambda q_0(t) +  g(t) Q\,;
\\
\label{q1}
-\fr{dq_1(t,x)}{dt}
&= -\lambda q_1(t,x) + \lambda c_1 b(x) q_0(t)\,;
\end{align}

\noindent
\begin{multline}
\label{qk} 
\hspace*{-0.202pt}\!-\fr{dq_k(t,x)}{dt} = -\lambda q_k(t,x) + \lambda c_k  b(x) q_0(t)
+ \!\sum\limits_{m=1}^{k-1} \!\!\lambda c_{k-m}\times{}\\
{}\times \int\limits_0^\infty \left (
b(y) w(x,y) q_m(t,x) + b(x) \overline{w}(y,x) q_m(t,y) \right )
dy, \\ 2 \le k \le N-1\,;
\end{multline}

\vspace*{-12pt}

\noindent
\begin{multline}
\label{q*n}
-\fr{dq^*_N(t,z,x)}{dt} = \lambda c_N  b(z) b(x) q_0(t) +{}\\
{}+ \lambda c_{1} \left (
b(x) w(z,x) q_{N-1}(t,z) + {}\right.\\
\left.{}+b(z) \overline{w}(x,z) q_{N-1}(t,x)
\right )
+{}
\\
{}+ \sum\limits_{m=1}^{N-2}
\lambda c_{N-m} \int\limits_0^\infty \left (
b(y) w(z,y) q_m(t,z) \delta(y-x) +{}\right.\\
\left.{}+
b(z) b(x) \overline{w}(y,z) q_m(t,y)
\right )\,dy,
\end{multline}
где $Q=q_0(0)+p_1(0)$~--- постоянная, которая, как будет 
видно из дальнейшего, имеет смысл нормирующего множителя, а~$\delta$ 
здесь и~далее~--- дель\-та-функ\-ция Дирака.
Решая уравнения~\eqref{q0}--\eqref{qk}, получаем, что при $1 \hm\le k \hm\le 
N\hm-1$ плотности $q_k(t,x)$ представимы в~виде 

\noindent
$$q_k(t,x) \hm=\sum\limits_{i=1}^k a^{(k)}_i A_i(x)q_i(t)\,,
$$
причем сомножители определяются рекуррентным образом:
\begin{align*}
%\label{soln1}
q_0(t)
&=
Q e^{\lambda t} \int\limits_t^\infty g(y) e^{- \lambda y}\,dy\,,\quad
q_k(t) ={}\\
{}&= \lambda e^{\lambda t} \int\limits_t^\infty q_{k-1}(y) e^{- \lambda y}\,dy\,; 
\enskip 1 \le k \le N-1\,;
\\
a^{(k)}_1
&=c_k\,, \quad  a^{(k)}_i = \sum\limits_{m=i-1}^{k-1}
a^{(m)}_{i-1} c_{k-m}, \enskip 2 \le i \le k\,;
\\ 
A_1(x) &=
b(x)\,, \quad
A_i(x)=
\int\limits_0^\infty \left ( b(y) w(x,y)
 A_{i-1}(x) +{}\right.\\
& \hspace*{25mm}\left.{}+ b(x) \overline{w}(y,x)  A_{i-1}(y) \right )\,
dy.
\end{align*}
Отсюда путем стандартных преобразований находим, что
ПФ $Q(s,t,x)$ имеет вид\footnote{Заметим, что в~некоторых случаях для
определения $Q(s,t,x)$ может быть удобно 
воспользоваться соответствующим интегральным уравнением, 
которое следует из~\eqref{q0}--\eqref{qk}.}:
$$Q(s,t,x)\hm= \sum\limits_{i=1}^\infty A_i(x) q_i(t) 
\left (C(s) \right )^{i}\,,
$$
где $C(s)$~--- ПФ размера группы. Интегрируя выражение для~$Q(s,t,x)$ 
по всем возможным значениям~$t$ и~$x$ 
и~учитывая, что $\int\nolimits_0^\infty A_i(x)\, dx\hm=1$, получаем 
ПФ стационарных вероятностей ${\{ Q_k, \ k \hm\ge 0 \}}$:
\begin{equation}
\label{Qs}
Q(s)+Q_0= Q \fr{ 1 - \gamma (\lambda - \lambda C(s) )}
{\lambda  - \lambda C(s)}\,,
\end{equation}
где $\gamma(s)$~--- преобразование Лапласа плотности~$g$ в~точке~$s$. 
Далее, как видно из~\eqref{q*n}, после умножения\linebreak
\vspace*{-12pt}

\pagebreak 

\noindent
левой и~правой частей на~$s^N$ 
и~суммирования по $N\hm>1$, ПФ~$Q^*(s,t,z,x)$ задается формулой:
\begin{multline}
\label{Q*s}
Q^*(s,t,z,x) ={}\\
{}= Q \lambda \left (\!
\fr{C(s)}{s} - c_1 \!\right ) b(z) b(x) 
\!\int\limits_t^\infty\!\! e^{\lambda u}\, du \!
\int\limits_{u}^\infty \!\!g(y) e^{- \lambda y}\,dy +{}
\\
{}+\lambda c_{1} \int\limits_t^\infty \left (
b(x) w(z,x) Q(s,u,z)+{}\right.\\
\left.{}+ b(z) \overline{w}(x,z) Q(s,u,x) \right )\, du
+ \lambda
\left ( \fr{C(s)}{s} - c_1\right )\times{}\\
{}\times \int\limits_t^\infty du
\int\limits_0^\infty \left (
b(y) w(z,y) Q(s,u,z) \delta(y-x) +{}\right.\\
\left.{}+ b(z) b(x) \overline{w}(y,z) Q(s,u,y)
\right )\,dy.
\end{multline}
Интегрируя теперь~\eqref{Q*s} по всем~$t$, $z$ и~$x$,
с~учетом вида ПФ $Q(s,t,x)$ в~\eqref{Qs}, получаем ПФ 
стационарных вероятностей $\{ Q^*_N, \ N \ge 2 \}$:
\begin{multline}
\label{Q*final}
Q^*(s) = \fr{Q}{\lambda} \left (
\fr{C(s)}{s} \left (  \fr{\lambda \overline{g} }{1-C(s)} 
- {}\right.\right.\\
\left.\left.{}-\fr{1-\gamma(\lambda- \lambda C(s))}{(1-C(s))^2 }
\right )
- c_1 \left (
\lambda \overline{g} - 1 + \gamma(\lambda) \right ) \right ). 
\end{multline}

Перейдем к~нахождению ПФ~\eqref{pf1}. Система уравнений 
Кол\-мо\-го\-ро\-ва--Чеп\-ме\-на для стационарных плотностей~$p_k(x)$ 
и~$p^*_N(x,y)$ имеет вид:
\begin{equation}
\label{np1}
-\fr{dp_1(x) }{dx} = -\lambda p_1(x) + q_1(0,x) + p_2(0,x)\,;
\end{equation}

\vspace*{-12pt}

\noindent
\begin{multline}
\label{npk}
-\fr{dp_k(x)}{ dx} = -\lambda p_k(x) + q_k(0,x) + p_{k+1}(0,x)
+{}\\
{}+
\sum\limits_{m=1}^{k-1} \lambda c_{k-m} \int\limits_0^\infty
\left ( b(u) v(x,u) p_m(x) +{}\right.\\
\left.{}+ b(x) \overline{v}(u,x) p_m(u) \right )
\,du\,, \enskip 2 \le k \le N-1\,;
\end{multline}

\vspace*{-12pt}

\noindent
\begin{multline}
\label{npn*}
-\fr{dp^*_N(x,y)}{dx} = q^*_N(0,x,y) +{}\\
{}+ \lambda c_1
\left (
b(y) v(x,y) p_{N-1}(x) + b(x) \overline{v}(y,x) p_{N-1}(y)
\right )
+{}\\
{}+
\sum\limits_{m=1}^{N-2}
\lambda c_{N-m}
\int\limits_0^\infty
\left ( b(u) v(x,u) p_m(x)\delta(u-y) +{}\right.\\
\left.{}+ b(x) b(y) \overline{v}(u,x) p_m(u)
\right )du\,.
\end{multline}

\noindent 
Из метода исключения состояний 
(см., например, \cite{n1,n3,n4,n5} и~\cite[с.~75]{new1})
следует, что плотности $p_k(x)$ и~$p_k(x,y)$ совпадают 
с~точ\-ностью до постоянного множителя, в~качестве которого, 
с~учетом найденных выше ПФ~\eqref{Qs} и~\eqref{Q*final}, можно выбрать~$Q$.
Поэтому в~\eqref{npk} вместо $p_{k+1}(0,x)$
можно подставить $p^*_{k+1}(0,x)$, предварительно
проинтегрировав~\eqref{npn*}.
После этого, умножая левые и~правые части~\eqref{np1} и~\eqref{npk} 
на~$s^k$ и~суммируя по $k \hm\ge 1$, приходим к~уравнению для
расчета ПФ $P(s,x)$:
\begin{multline*}
-\fr{dP(s,x) }{dx} =
-\lambda (1-c_1) P(s,x) + \lambda c_1 b(x) P(s) +{}\\
{}+ Q(s,0,x)
+ \int\limits_0^\infty Q^*(s,0,u,x)du
+ {}\\
{}+\lambda \left ( C(s) - c_1 \right )
\int\limits_0^\infty \left (
b(u) v(x,u) P(s,x) +{}\right.\\
\left.{}+ b(x) \overline{v}(u,x) P(s,u) \right ) du
+
\lambda \left ( \fr{C(s)}{s} - c_1 \right )\times{}\\
{}\times
\int\limits_0^\infty dz
\int\limits_0^\infty \left (
b(u) v(z,u) P(s,z) \delta(u-x) +{}\right.\\
\left.{}+ b(z) b(x) \overline{v}(u,z) P(s,u)
\right ) du.
\end{multline*}

\noindent 
Наконец, интегрируя~\eqref{npn*} и~вводя ПФ %производящую функцию
$$P^*(s,x,y)\hm=\sum\limits_{N=2}^\infty p^*_{N}(x,y)s^N\,,
$$
находим:
\begin{multline*}
P^*(s,x,y)= s\int\limits_x^\infty Q^*(s,0,u,y)\,du +{}\\
{}+
\lambda c_1 s \!\int\limits_x^\infty \!\left ( b(y) v(u,y) P(s,u) +
b(u) \overline{v}(y,u) P(s,y)\right ) du
+{}\\
{}+
\lambda \left(C(s)-c_1 s\right) \!\int\limits_x^\infty\!
dz \!\int\limits_0^\infty \!\left (
b(u) v(z,u) P(s,z) \delta(u-y) +{}\right.\\
\left.{}+ b(z) b(y) \overline{v}(u,z)  P(s,u)
\right )du\,.
\end{multline*}

Таким образом, в~рассматриваемой системе стационарные 
вероятности~$Q_k$, ${0 \hm\le k \hm\le N-1}$, и~$P_k$, 
${1 \hm\le k \hm\le N-1}$, определяются
как коэффициенты при~$s^k$ разложения в~ряд по степеням~$s$ 
соответственно функций $Q(s)\hm+Q_0$ 
и~${P(s)=\int\nolimits_0^\infty P(s,x)\,dx}$. 
Стационарные вероятности~$Q^*_N$ и~$P^*_N$ 
определяются как коэффициенты при~$s^N$ разложения в~ряд по степеням~$s$ 
соответственно функций~$sQ^*(s)$
и~$P^*(s)\hm=\int\nolimits_0^\infty \int_0^\infty P^*(s,x,y)\,dydx$.
Постоянная~$Q$ определяется из условия нормировки
$$Q_0\hm+\sum\limits_{k=1}^{N-1} (Q_k+P_k) \hm+Q^*_N \hm+ P^*_N\hm=1$$
или, в~случае ${N\hm=\infty}$, из соотношения 
$${Q_0\hm+P(1)\hm+Q(1)\hm=1}\,.$$


Зная стационарное распределение, в~силу пуассоновости 
входящего потока можно вычислить и~стационарную вероятность~$\pi$ 
потери заявки:
$$
\pi= P^*_{N} + Q^*_{N} + \sum\limits_{k=2}^{N}
\fr{k c_k }{\overline{c}} \sum\limits_{j=N-k+1}^{N-1} (P_{j}+Q_{j}).
$$

\noindent 
Моменты стационарного распределения числа заявок в~системе
вычисляются путем дифференцирования соответствующих ПФ 
и~последующего решения получившихся уравнений.
Заметим, что при $N\hm<\infty$ загрузка системы заявками первого типа, 
равная $(\sum\nolimits_{k=1}^{N-1} P_k \hm+ P^*_N)/(1\hm-\pi)$,
также служит одним из показателей ее производительности.

Ситуация с~решением интегральных уравнений аналогична ситуации
в случае системы с~ординарным потоком: в~некоторых частных 
случаях\footnote{Например, когда длины заявок первого типа принимают 
только конечное число значений, или когда для функций $v(x,y)$ 
известна сепарабельная аппроксимация (см.~\cite{n5,n6,n7,n8}). 
Подробнее см.~\cite[с.~78]{new1}.} решения могут быть выписаны в~явном виде.
Необходимо также отметить, что некоторые эффекты, которые наблюдаются 
в~сис\-те\-ме с~ординарным потоком, исчезают в~системе с~групповым потоком. 
В~част\-ности (при фиксированной загрузке заявками первого типа), 
стационарное распределение числа заявок в~рассматриваемой системе
 при инверсионном порядке обслуживания с~прерыванием (т.\,е.\
  $w(y,x)\hm=v(y,x)\hm=0$) уже не является инвариантным относительно 
  вида распределения~$B(x)$.

%(ср. с~\cite[С.~79]{new1}).

\section{Стационарное распределение времени ожидания 
и~времени пребывания заявки в~системе}

Для нахождения временн$\acute{\mbox{ы}}$х характеристик поступающих 
в~систему заявок первого типа необходимо предварительно определить 
несколько вспомогательных величин.
Обозначим через~$u_m(s;x)$, $1\hm \le m\hm \le N$, 
преобразование Лап\-ла\-са--Стилть\-еса (ПЛС) функции распределения
времени до того момента,
когда в~системе останется $(m\hm-1)$ заявок при условии,
что на приборе начала обслуживаться заявка первого типа длины~$x$
и~в~системе было~$m$~заявок.
Поскольку в~соответствии с~описанием системы
группа заявок, застающая при поступлении в~системе~$n$~заявок 
первого типа, теряется, то $u_{N}(s;x)\hm=e^{- s x}$.
Для $1 \hm\le m \hm\le N-1$, рассматривая все возможные события 
и~воспользовавшись свойствами ПЛС, получаем:

\noindent
\begin{multline*}
u_{m}(s;x) = e^{- (\lambda+s) x} +
\sum\limits_{k=1}^{N-m} c_k
\int\limits_0^x \lambda e^{- (\lambda+s) t} \, dt\times{}\\
{}\times
\int\limits_0^\infty \overline{v}(x-t,y) u_{m}(s;x-t) 
u_{m+k}(s;y) \times{}\\
{}\times \prod\limits_{j=2}^{k} \! u_{m+k+1-j}(s) b(y)\,dy
+\!\sum\limits_{k=1}^{N-m} \!c_k\!
\int\limits_0^x \!\lambda e^{- (\lambda+s) t} \, dt\times{}\\
{}\times
\int\limits_0^\infty v(x-t,y) u_{m+k}(s;x-t) u_{m+k-1}(s;y) \times{}\\
{}\times
\prod\limits_{j=2}^{k} u_{m+k-j}(s) b(y)\,dy.
\end{multline*}


\noindent 
Здесь и~далее 
$$u_m(s)\hm=\int\limits_0^\infty u_m(s;x)b(x)\,dx\,.
$$ 
Решения этих интегральных уравнений в~явном виде
при $N\hm<\infty$ выписать не удается\footnote{Однако при $N\hm=\infty$
функции $u_{m}(s;x)$ и~$u_{m}(s)$ не зависят от~$m$ и,~как показано 
в~\cite[с.~14]{R1}, могут быть найдены путем решения
одного-единственного уравнения. Аналогично обстоит дело и~с~ПЛС 
периода занятости системы заявками первого типа.}.
Но при любом фиксированном~$s$ они могут быть найдены численно,
причем расчет необходимо вести в~сле\-ду\-ющем
порядке: $u_{N-1}(s;x)$, $u_{N-1}(s)$, $u_{N-2}(s;x)$,
$u_{N-2}(s)\dots$

Пусть теперь в~системе находится всего~$m_1$~заявок первого типа,
первая заявка в~очереди имеет длину~$x_1$,
а на приборе обслуживается заявка второго типа.
Выделим заявку в~очереди: пусть она находится
на $i_1$-м месте. Обозначим через $h_{(m_1,i_1),(m_2,i_2)}(x_1,x_2)$
(условную) плот\-ность вероятности того, что после очередного поступления
в~сис\-те\-ме окажутся~$m_2$, ${\min(m_1+1,N)\hm\le m_2 \hm\le N}$, заявок,
выделенная заявка окажется на $i_2$-м,  ${i_1 \hm\le i_2 \hm\le m_2}$,  месте,
а длина первой заявки в~очереди будет~$x_2$.
Доопределим функцию $h$ в~точках $m_2\hm=m_1$
следующим образом (здесь и~далее~$\mathbf{1}_{(A)}$~--- индикатор события~$A$):
\begin{multline}
\label{h0}
h_{(m_1,i_1),(m_1,i_2)}(x_1,x_2)
={}\\
{}=
\begin{cases}
0, & m_1 \neq N;\\
\delta(x_1-x_2) \mathbf{1}_{(i_1=i_2)}, & m_1=N.
\end{cases}
\end{multline}

\noindent
Учитывая, что поступающая группа заявок может либо занять
первые места в~очереди, либо встать в~очередь
сразу позади первой заявки, по формуле полной вероятности
получаем:
\pagebreak

\noindent
\begin{multline}
\label{h}
h_{(m_1,i_1),(m_2,i_2)}(x_1,x_2)
={}\\
{}= c_{m_2-m_1} \int\limits_0^\infty
\left (\overline{w}(x_1,y)
\delta(y-x_2) \mathbf{1}_{(i_2=i_1+m_2-m_1)}
+{}\right. \\
{}+ w(x_1,y) \delta(x_1-x_2) \mathbf{1}_{(i_2=1)} \mathbf{1}_{(i_1=1)}
+ w(x_1,y)\times{}\\
\left.
{}\times
\delta(x_1-x_2) \mathbf{1}_{(i_2=i_1+m_2-m_1)}
\mathbf{1}_{(i_1\neq 1)} \right ) b(y)\,dy.
\end{multline}

\noindent 
На основе~\eqref{h0} и~\eqref{h}
рассчитываются плотности $h^{(n)}_{(m_1,i_1),(m_2,i_2)}(x_1,x_2)$,
 $n \hm\ge 2$, вероятностей
переходов за~$n$ последовательных поступлений:
\begin{multline*}
h^{(n)}_{(m_1,i_1),(m_2,i_2)}(x_1,x_2)
={}\\
{}=
\begin{cases}
\displaystyle
\sum\limits_{m=m_1+1}^{m_2}
\sum\limits_{i=i_1}^{i_2}
\int\limits_{0}^\infty
h^{(n-1)}_{(m_1,i_1),(m,i)}(x_1,y)\times{}&\\
{}\hspace*{8mm}\times h_{(m,i),(m_2,i_2)}(y,x_2) dy, & \hspace*{-5mm}m_1 \neq N;
\\
h_{(N,i_1),(m_2,i_2)}(x_1,x_2), & \hspace*{-5mm}m_1=N.
\end{cases}
\end{multline*}

Нахождение распределения времени ожидания начала обслуживания удобно разбить
на два этапа: один соответствует случаю, когда поступа\-ющая группа застала
на приборе заявку первого типа, другой~--- случаю, когда на приборе оказалась
заявка второго типа.
Определим ПЛС~$\omega^{(1)}_{k1}(s;x)$, $1 \hm\le k\hm \le N\hm-1$, 
функции распределения
времени ожидания начала обслуживания принятой заявки,
заставшей на приборе заявку первого типа, при условии,
что она поступила в~группе размера~$k$, была на первом месте в~группе
и имела длину~$x$. Ее время ожидания равно нулю, если она
 заняла место заявки на приборе.
Если же она застала в~сис\-те\-ме~$m$, $1 \hm\le m \hm\le N\hm-k$, 
заявок первого типа,
на приборе~--- заявку длины~$y$ и~не заняла ее место,
то время ожидания совпадает с~периодом времени, необходимым для
уменьшения длины очереди на единицу. В~терминах ПЛС имеем:
\begin{multline}
\label{o1} 
\omega^{(1)}_{k1}(s;x)
= \sum\limits_{m=1}^{N-k} \int\limits_0^\infty \!p_m(y)
\left ( \overline{w}(y,x) +{}\right.\\
\left.{}+ w(y,x) u_{m+k}(s;y)
\right )\,dy.
\end{multline}

\noindent Предположим теперь, что выделенная заявка оказалась на $i$-м месте
в~группе. Тогда ее время ожидания зависит от длин заявок, стоящих перед ней, 
а~также от того, произошла ли смена заявки на приборе. 
Поскольку длины заявок в~группе независимы и~не зависят от размера группы, 
то ПЛС $\omega^{(1)}_{ki}(s;x)$, $1 \le k \le N-1$, $2 \le i \le k$,
стационарного (условного) распределения ее времени ожидания начала 
обслуживания равно:

\noindent
\begin{multline}
\label{o2}
\omega^{(1)}_{ki}(s;x)
= {}\\
{}=\!\sum\limits_{m=1}^{N-k}
\int\limits_0^\infty p_m(y)\!
\int\limits_0^\infty\!\!
\left (\! \overline{w}(y,z) u_{m+k}(s;z) 
\prod\limits_{j=1}^{i-2} \!u_{m+k-j}(s) \right.
+{}
\\
\left.
{}+
w(y,z) u_{m+k}(s;y) u_{m+k-1}(s;z) \prod\limits_{j=2}^{i-1}
u_{m+k-j}(s) \right )\times{}\\
{}\times b(z)\,dzdy\,.
\end{multline}

\noindent Усредняя $\omega^{(1)}_{ki}(s;x)$, полученные 
в~\eqref{o1} и~\eqref{o2},
по распределению длины выделенной заявки и~ее мес\-ту в~поступающей группе,
получаем безусловное ПЛС~$\omega^{(1)}(s)$ функции распределения времени
ожидания начала обслуживания принятой заявки, заставшей при поступлении
на приборе заявку первого типа:
\begin{equation}
\label{w3}
\omega^{(1)}(s)
=\sum\limits_{k=1}^{N-1} \sum\limits_{i=1}^k
\fr{c_k}{\overline{c}} \int\limits_0^\infty
\omega^{(1)}_{ki}(s;x) b(x)\,dx\,.
\end{equation}


Перейдем ко второму этапу.
Обозначим через $\theta_{(k,i)(l,j)}(y,z,x)$,
${k \hm\le l \hm\le N}$, ${1\hm \le i\hm \le k}$,
вероятность того, что
выделенная заявка займет $j$-е место в~очереди,
первая заявка в~очереди будет иметь длину~$z$,
всего в~системе будет~$l$~заявок первого типа
и на приборе~--- заявка второго типа длины~$y$, при условии,
что выделенная заявка поступила в~группе размера~$k$, была в~ней на $i$-м
месте и~имела длину~$x$.
Выражения для ненулевых вероятностей $\theta_{(k,i)(l,j)}(y,z,x)$
находятся по формуле полной вероятности:
\begin{multline*}
\theta_{(k,i)(k,i)}(y,z,x)={}\\
{}=q_0(y) \delta(x-z) \mathbf{1}_{(i=1)}
+ q_0(y) b(z) \mathbf{1}_{(i\neq 1)};
\end{multline*}
\vspace*{-12pt}

\noindent
\begin{multline*}
\theta_{(k,i)(l,i)}(y,z,x)={}\\
{}=q_{l-k}(y,z) 
\overline{w}(z,x) \delta(x-z) \mathbf{1}_{(i=1)}+{}\\
{}+ b(z)\int\limits_{0}^\infty q_{l-k}(y,t) 
\overline{w}(t,z) \,dt \mathbf{1}_{(i\neq 1)}\,;
\end{multline*}
\vspace*{-12pt}

\noindent
\begin{multline*}
\theta_{(k,i)(l,i+1)}(y,z,x)
=q_{l-k}(y,z) w(z,x) \mathbf{1}_{(i=1)}+ {}\\
{}+q_{l-k}(y,z)\int\limits_{0}^\infty 
b(t) w(z,t) \,dt \mathbf{1}_{(i\neq 1)}.
\end{multline*}

Оказавшись в~очереди, выделенная заявка до поступления 
на прибор должна ожидать окончания обслуживания заявки на 
приборе, впереди стоящих заявок (и их потомков), 
а~также окончания обслуживания тех заявок (и~их потомков), 
которые могут поступить в~систему (за время обслуживания заявки 
второго типа на приборе) и~встать перед ней. Нетрудно видеть, что
вероятность ${\tilde \theta}_{(k,i)(l,j)}(y,z,x)$,
${k \hm\le l\hm \le N}$, ${i\hm \le j\hm\le l}$, того, что
(при прочих условиях, остающихся неизменными)
через время $y$ выделенная заявка окажется на $j$-м месте
в~очереди, равна\footnote{Отметим, что
при расчетах по формуле~\eqref{oo1} приходится суммировать
конечное число слагаемых, поскольку из принятого правила
постановки новых заявок в~очередь (и того, что $N\hm<\infty$) следует,
что плотности $h^{(n)}_{(m_1,i_1),(m_2,i_2)}(x_1,x_2)$ 
(при фиксированных~$m_1$ и~$m_2$)
совпадают начиная с~некоторого~$n$.}:
\begin{multline}
\label{oo1}
{\tilde \theta}_{(k,i)(l,j)}(y,z,x) ={}\\
{}= e^{-\lambda y} \theta_{(k,i)(l,j)}(y,z,x) +
%{} \\{}+
\sum\limits_{n=1}^\infty \fr{(\lambda y )^n }{n!}\,
e^{-\lambda y} \times{}\\
\!\times \!\!\sum\limits_{l_1=k}^{l-1} \sum\limits_{j_1=i}^{j}
\int\limits_0^\infty \!\theta_{(k,i)(l_1,j_1)}(y,t,x)
h^{(n)}_{(l_1,j_1),(l,j)}(x,t) dt,\!\!\!
\end{multline}
причем
$${\tilde \theta}_{(N,i)(N,j)}(y,z,x)\hm = \theta_{(N,i)(N,j)}(y,z,x) 
\mathbf{1}_{(j=i)}$$ 
при $1 \hm\le i\hm \le N$.
Тогда, с~учетом независимости длин заявок в~группе,
ПЛС $\omega^{(2)}_{ki}(s;x)$, $1 \hm\le k\hm \le N$, $1\hm \le i \hm\le k$, 
стационарного (условного) распределения времени ожидания начала обслуживания
заявки, заставшей при поступлении на приборе заявку второго типа, равно:
\begin{multline}
\label{oo3}
\omega^{(2)}_{ki}(s;x) =
\sum\limits_{l=k}^{N} \sum\limits_{j=i}^{l} \int\limits_0^\infty
\int\limits_0^\infty e^{-s y}
{\tilde \theta}_{(k,i)(l,j)}(y,z,x)
\times {}\\
\!\!{}\times
\left ( \mathbf{1}_{(j=1)} +
u_l(s;z) \prod\limits_{m=1}^{j-2} u_{l-m}(s) \mathbf{1}_{(j\neq 1)}
\right )\, dz dy, \!\!
\end{multline}
усредняя которое, как в~\eqref{w3}, получаем безусловное ПЛС~$\omega^{(2)}(s)$. 
Таким образом, стационарное распределение времени ожидания начала 
обслуживания принятой в~систему заявки имеет ПЛС
 $(\omega^{(1)}(s)\hm+\omega^{(2)}(s))(1\hm-\pi)^{-1}$.

Аналогичным образом находится и~ПЛС ста\-цио\-нарного распределения 
времени пребывания заявки первого типа в~системе. Действительно,
 поскольку это время  складывается из времени ожидания начала 
 обслуживания и~времени  пребывания на приборе с~учетом возможных 
 прерываний,
то каж\-дое слагаемое в~\eqref{o1}, \eqref{o2} и~\eqref{oo3} достаточно 
домножить на ПЛС времени до уменьшения на единицу числа заявок первого
 типа в~системе, т.\,е.\ на~$u_m(s;x)$ при соответствующем~$m$.

Заметим, что распределение времени пребывания в~системе 
заявок второго типа не представляет интереса: оно равно~$G(x)$.
 Что касается периодов времени между моментами начала поступления заявок 
 второго типа на прибор, то они
совпадают по распределению с~циклом занятости системы заявками 
первого типа.


\section{Заключение}

Дифференцируя полученные в~предыдущем разделе 
формулы необходимое число раз, можно \mbox{найти} моменты стационарных 
распределений времени ожидания начала обслуживания
и времени пребывания заявки в~системе. Для нахождения средних 
значений также можно (и удобнее) пользоваться формулой Литтла.

На основе полученных результатов находятся и~различные совместные 
стационарные распределения (например, периода занятости заявками 
первого типа и~числа обслуженных и~потерянных на нем заявок и~т.\,п.). 
Кроме того, из результатов разд.~4 следует, что стационарные 
временн$\acute{\mbox{ы}}$е характеристики могут быть найдены в~случае, когда порядок 
постановки в~очередь заявок первого типа в~момент обслуживания заявки
 второго типа произвольный. В~дальнейшем интерес представляет обобщение 
 предложенного метода на случай более общего входного потока (как, например, 
 в~\cite{R1,n0}), нескольких классов (с~различными функциями распределения 
 длин) заявок второго типа и~других правил начала обслуживания заявок 
 первого типа (например, $N$-по\-ли\-ти\-ки~\cite{R9}).


{\small\frenchspacing
 {%\baselineskip=10.8pt
 \addcontentsline{toc}{section}{References}
 \begin{thebibliography}{99}

\bibitem{R6}
\Au{Lee T.\,T.} ${M/G/1/N}$ queue with vacation time and 
exhaustive service discipline~// Oper. Res., 1984. Vol.~32. P.~774--785.

\bibitem{new1} 
\Au{Печинкин А.\,В.} Двухприориетная система массового 
обслуживания с~инверсионным порядком обслуживания~// 
Техника средств связи. Сер. СС, 1985. Вып.~1. С.~72--81.

\bibitem{R3}
\Au{Бочаров П.\,П.,  Шлумпер~Л.\,О.}
Однолинейная система массового обслуживания с~фоновыми заявками~// 
Автомат. телемех., 2005. №\,6. С.~74--88.
%; Autom. Remote Control, 66:6 (2005), 917-930

\bibitem{R4}
\Au{Razumchik R.} Two-priority queueing system with LCFS
service, probabilistic priority and batch arrivals~// 
AIP Conf. Proc., 2019. Vol.~2116. No.\,1. 
P.~090011-1--\mbox{090011-3.}

\bibitem{R9}
\Au{Kempa W.\,M.} Analytical model of a wireless sensor 
network (WSN) node operation with a modified threshold-type 
energy saving mechanism~//
Sensors Basel, 2019. Vol.~19. Iss.~14. Art.\ ID: 3114.

\bibitem{n4} 
\Au{Печинкин А.\,В.} Об одной
инвариантной системе массового обслуживания~//
Math.\ Operationsforsch.\ Statist.
Ser.\ Optimization, 1983. Vol.~14. No.\,3. P.~433--444.

\bibitem{R5}
\Au{Kim C., Dudin~A., Dudina~O., Klimenok~V.}
Analysis of queueing system with non-preemptive 
time limited service and impatient customers~// 
Methodol. Comput. \mbox{Appl.,} 2020. Vol.~22. P.~401--432.

\bibitem{R7}
\Au{Kempa W.\,M., Marjasz~R.}
Distribution of the time to buffer overflow in the ${M/G/1/N}$-type queueing 
model with batch arrivals and multiple vacation policy~//
J.~Oper. Res. Soc., 2020. Vol.~71. Iss.~3. P.~447--455.

\bibitem{n3} %9
\Au{Нагоненко В.\,А.}
О характеристиках одной нестандартной системы массового обслуживания.~I, II~//
Изв.\ АН СССР. Технич.\ кибернет., 1981. №\,1. С.~187--195; №\,3. С.~91--99.

\bibitem{n5} %10
\Au{Милованова~Т.\,А., Печинкин~А.\,В.}
Стационарные характеристики системы обслуживания с~инверсионным порядком 
обслуживания, вероятностным приоритетом и~гистерезисной политикой~//
Информатика и~её применения, 2013. Т.~7. Вып.~1. С.~22--36.

\bibitem{n1} %11
\Au{Мейханаджян Л.\,А., Милованова~Т.\,А., Печинкин~А.\,В., Разумчик~Р.\,В.}
Стационарные вероятности состояний в~системе обслуживания с~инверсионным 
порядком обслуживания и~обобщенным вероятностным приоритетом~// Информатика 
и~её применения, 2014. Т.~8. Вып.~3. С.~16--26.

\bibitem{n6} %12
\Au{Поспелов В.\,В.}
О~погрешности приближения функции двух переменных суммами
произведений функций одного переменного~//
Ж.~вычисл. мат. мат. физ., 1978. Т.~18. Вып.~5. С.~1307--1308.

\bibitem{n8} %13
\Au{Uschmajew A.} Regularity of tensor product approximations to square
integrable functions~// Constr. Approx., 2011. Vol.~34. Iss.~3. P.~371--391.

\bibitem{n7} %14
\Au{Townsend A., Trefethen~L.\,N.} An extension of chebfun to two dimensions~// 
SIAM J.~Sci. Comput., 2013. Vol.~35. Iss.~6. P.~495--518.

\bibitem{R1}
\Au{Разумчик Р.\,В.} 
Стационарные характеристики системы обслуживания с~инверсионным порядком
 обслуживания, вероятностным приоритетом и~групповым поступлением 
 разнородных заявок~// Информатика и~её применения, 2017. Т.~11. Вып.~4. С.~10--18.

\bibitem{n0}
\Au{Razumchik R.} 
On ${M/G/1}$ queue with state-dependent heterogeneous batch arrivals,
 inverse service order and probabilistic priority~// AIP Conf. 
 Proc., 2017. Vol. 1863. No.\,1. P.~090006-1--09006-3.
\end{thebibliography}

 }
 }

\end{multicols}

\vspace*{-3pt}

\hfill{\small\textit{Поступила в~редакцию 15.07.20}}

\vspace*{8pt}

%\pagebreak

%\newpage

%\vspace*{-28pt}

\hrule

\vspace*{2pt}

\hrule

%\vspace*{-2pt}

\def\tit{A~SINGLE-SERVER QUEUEING SYSTEM WITH~LIFO SERVICE, PROBABILISTIC PRIORITY,
BATCH POISSON ARRIVALS, AND~BACKGROUND CUSTOMERS}


\def\titkol{A~single-server queueing system with~LIFO service, probabilistic priority,
batch Poisson arrivals, and~background customers}

\def\aut{T.\,A.~Milovanova$^1$ and R.\,V.~Razumchik$^2$}

\def\autkol{T.\,A.~Milovanova and R.\,V.~Razumchik}

\titel{\tit}{\aut}{\autkol}{\titkol}

\vspace*{-5pt}


\noindent
$^1$Peoples' Friendship University of Russia (RUDN University), 
6~Miklukho-Maklaya Str., Moscow 117198, Russian\linebreak
$\hphantom{^1}$Federation

\noindent
$^2$Institute of Informatics Problems, Federal Research Center 
``Computer Science and Control'' of the Russian\linebreak
$\hphantom{^1}$Academy of Sciences, 44-2~Vavilov Str., Moscow 119333, Russian Federation

\def\leftfootline{\small{\textbf{\thepage}
\hfill INFORMATIKA I EE PRIMENENIYA~--- INFORMATICS AND
APPLICATIONS\ \ \ 2020\ \ \ volume~14\ \ \ issue\ 3}
}%
 \def\rightfootline{\small{INFORMATIKA I EE PRIMENENIYA~---
INFORMATICS AND APPLICATIONS\ \ \ 2020\ \ \ volume~14\ \ \ issue\ 3
\hfill \textbf{\thepage}}}

\vspace*{9pt} 



\Abste{Consideration is given to the single-server queueing system with two 
independent flows of customers: a~batch Poisson flow of (primary) customers
and a~saturated flow of background customers.
Primary customers have relative priority over background customers,
i.\,e., the service of a~background customer cannot be interrupted.
A~background customer is instantly taken for service every time the buffer 
for primary customers is empty upon the service completion.
The service times of primary and background customers are independent and
are allowed to be generally distributed. 
The implemented service policy is LIFO (last in, first out) with the probabilistic priority.
The method and analytic expressions for the computation (in terms of transforms) 
of the system's main stationary performance characteristics, including 
the stationary distribution of the waiting and sojourn times
of the primary customers, are presented.}


\KWE{queueing system; LIFO service; probabilistic priority; 
batch arrivals; background customers}




\DOI{10.14357/19922264200304} 

%\vspace*{-18pt}

\Ack
\noindent
The reported study was funded by RFBR, project No.\,20-07-00804.


\pagebreak

 \begin{multicols}{2}

\renewcommand{\bibname}{\protect\rmfamily References}
%\renewcommand{\bibname}{\large\protect\rm References}

{\small\frenchspacing
 {%\baselineskip=10.8pt
 \addcontentsline{toc}{section}{References}
 \begin{thebibliography}{99}
\bibitem{XR6-1}
\Aue{Lee, T.\,T.} 1984.
${M/G/1/N}$ queue with vacation time and exhaustive service discipline.
\textit{Oper. Res.} 32:~774--785.

\bibitem{Xnew1-1} 
\Aue{Pechinkin, A.\,V.} 1985.
Dvukhpriorietnaya sistema massovogo obsluzhivaniya 
s~inversionnym poryadkom obsluzhivaniya [Two-priority queuing 
system with inversion order of service].
\textit{Tekhnika sredstv svyazi Ser. SS} [Communication technology. 
Communication systems series] 1:72--81.

\bibitem{XR3-1}
\Aue{Bocharov, P.\,P., and L.\,O.~Shlumper.} 2005.
%Odnolineynaya sistema massovogo obsluzhivaniya s~fonovymi zayavkami
A single-server queueing system with background customers.
%\textit{Avtomat. telemekh.} 
\textit{Automat. Rem. Contr.} 66(6):917--930.

\bibitem{XR4-1}
\Aue{Razumchik, R.} 2019. Two-priority queueing system with LCFS
service, probabilistic priority and batch
arrivals. \textit{AIP Conf. Proc.}  2116(1):090011-1--090011-3.

\bibitem{XR9-1}
\Aue{Kempa, W.\,M.} 2019.
Analytical model of a~wireless sensor network (WSN) node 
operation with a~modified threshold-type energy saving mechanism.
\textit{Sensors Basel} 19(14):3114.

\bibitem{Xx4-1}
\Aue{Pechinkin, A.\,V.} 1983.
Ob odnoy invariantnoy sisteme massovogo obsluzhivaniya
[On an invariant queuing system].
\textit{Math.\ Operationsforsch.\ Statist. Ser.\ Optimization} 14(3):433--444.

\bibitem{XR5-1}
\Aue{Kim, C., A.~Dudin, O.~Dudina, and V.~Klimenok.} 2020.
Analysis of queueing system with non-preemptive 
time limited service and impatient customers.
\textit{Methodol. Comput. Appl.} 22:401--432.

\bibitem{XR7-1}
\Aue{Kempa, W.\,M., and R.~Marjasz.} 2020. 
Distribution of the time to buffer overflow in the ${M/G/1/N}$-type 
queueing model with batch arrivals and multiple vacation policy. 
\textit{J.~Oper. Res. Soc.} 71(3):447--455.

\bibitem{Xx3-1} %9
\Aue{Nagonenko, V.\,A.} 1981.
O~kharakteristikakh odnoy nestandartnoy sistemy massovogo obsluzhivaniya
[On the characteristics of one non-standard queuing system].~I, II
%\textit{Izv.\ AN SSSR. Tekhnich.\ kibernet}
%[Proceedings of the Academy of Sciences of the USSR. Technical Cybernetics]
\textit{J.~Comput. Sys. Sc. Int.}
1:187--195; 3:91--99.

\bibitem{Xx5-1} %10
\Aue{Milovanova, T.\,A., and A.\,V.~Pechinkin}. 2013.
Statsionarnye kharakteristiki sistemy obsluzhivaniya
s~inversionnym poryadkom obsluzhivaniya,
veroyatnostnym prio\-ri\-te\-tom i~gisterezisnoy politikoy
[Stationary characteristics of queuing system with
an inversion procedure service probabilistic priority
and hysteresis policy]
\textit{Informatika i~ee Primeneniya~--- Inform. Appl.} 7(1):22--35.

\bibitem{Xx1-1} %11
\Aue{Meykhanadzhyan, L.\,A., T.\,A.~Milovanova, A.\,V.~Pe\-chin\-kin, and 
R.\,V.~Razumchik.} 2014.
Statsio\-nar\-nye veroyatnosti sostoyaniy v~sisteme obsluzhivaniya 
s~inversionnym poryadkom obsluzhivaniya i~obobshchennym veroyat\-no\-st\-nym
prioritetom
[Stationary distribution in a~queueing system with inverse service 
order and generalized probabilistic priority].
\textit{Informatika i~ee Primeneniya~--- Inform. Appl.} 8(3):16--26.

\bibitem{Xn6-1}
\Aue{Pospelov, V.\,V.} 1978.
O~pogreshnosti priblizheniya funktsii dvuh peremennykh summami
proizvedeniy funktsiy odnogo peremennogo
[The error of approximation of a~function of two variables by sums 
of the products of functions of one variable].
\textit{ %Zh. vichisl. matem. matem fiz.} 
USSR Comp. Math. Math.} %Phys. 
18(5):1307--1308.

\bibitem{Xn8-1} %13
\Aue{Uschmajew, A.}
 2011. Regularity of tensor product approximations to square
integrable functions. \textit{Constr. Approx.} 34(3):371--391.

\bibitem{Xn7} %14
\Aue{Townsend, A., and L.\,N.~Trefethen.} 
2013. An extension of chebfun to two dimensions.
\textit{SIAM J.~Sci. Comput.} 35(6):495--518.

\bibitem{XR1-1} 
\Aue{Razumchik, R.\,V.} 2017.
Statsionarnye kharakteristiki sistemy obsluzhivaniya 
s~inversionnym poryadkom obsluzhivaniya, veroyatnostnym prioritetom 
i~gruppovym postupleniyem raznorodnykh zayavok
[${M/G/1}$ queue with state-dependent heterogeneous batch arrivals, 
inverse service order, and probabilistic priority].
\textit{Informatika i~ee Primeneniya~--- Inform. Appl.} 8(3):16--26.


\bibitem{Xn0-1}
\Aue{Razumchik, R.} 2017. 
On ${M/G/1}$ queue with state-dependent heterogeneous batch arrivals,
 inverse service order and probabilistic priority. 
 \textit{AIP Conf. Proc.} 1863(1):090006-1--090006-3.

\end{thebibliography}

 }
 }

\end{multicols}

\vspace*{-6pt}

\hfill{\small\textit{Received July 15, 2020}}

%\pagebreak

%\vspace*{-24pt}

\Contr

\noindent
\textbf{Milovanova Tatiana A.} (b.\ 1977)~---
Candidate of Science (PhD) in physics and mathematics, senior lecturer,
Department of Applied Informatics and Probability Theory,
Peoples' Friendship University of Russia (RUDN University), 
6~Miklukho-Maklaya Str., Moscow 117198, Russian Federation; 
\mbox{tmilovanova77@mail.ru}

\vspace*{4pt}

\noindent
\textbf{Razumchik Rostislav V.} (b.\ 1984)~---
Candidate of Science (PhD) in physics and mathematics, leading scientist,
Institute of Informatics Problems, Federal Research Center 
``Computer Science and Control'' of the Russian Academy of Sciences, 
44-2~Vavilov Str., Moscow 119333, Russian Federation; \mbox{rrazumchik@ipiran.ru}
\label{end\stat}

\renewcommand{\bibname}{\protect\rm Литература} 

\renewcommand{\figurename}{\protect\bf Рис.}