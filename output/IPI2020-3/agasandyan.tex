\def\stat{agasandyan}

\def\tit{ВЫЧИСЛИТЕЛЬНЫЕ АСПЕКТЫ ПРИМЕНЕНИЯ CC-VaR НА~СОВОКУПНОСТИ РЫНКОВ$^*$}

\def\titkol{Вычислительные аспекты применения CC-VaR на совокупности рынков}

\def\aut{Г.\,А.~Агасандян$^1$}

\def\autkol{Г.\,А.~Агасандян}

\titel{\tit}{\aut}{\autkol}{\titkol}

\index{Агасандян Г.\,А.}
\index{Agasandyan G.\,A.}
 

{\renewcommand{\thefootnote}{\fnsymbol{footnote}} \footnotetext[1]
{Исследование выполнено при финансовой поддержке РФФИ в рамках
научного проекта №\,17-01-00816.}}


\renewcommand{\thefootnote}{\arabic{footnote}}
\footnotetext[1]{Вычислительный центр им.\ А.\,А.~Дородницына Федерального исследовательского 
центра <<Информатика и~управление>> Российской академии наук, 
\mbox{agasand17@yandex.ru}}

%\vspace*{-6pt}


  
  \Abst{Работа служит непосредственным продолжением предыдущей работы автора, 
посвященной применению континуального критерия VaR на совокупности нескольких 
рынков разных размерностей, связанных между собой базовыми активами. Исследование 
нацелено на приложение идей и~методов, развитых для теоретической континуальной 
модели, к~дискретным сценарным рынкам. В~модели совокупности одного двумерного 
и~двух одномерных рынков, а также ее усеченных вариантов приводятся конструкции 
оптимальных сценарных портфелей из базисных инструментов всех рынков совокупности 
с~применением рандомизации. Предлагаемые конструкции проверяются на числовых 
примерах с~использованием потенциально типичных двумерных расширений  
бе\-та-рас\-пре\-де\-ле\-ний для описания прогноза будущих цен базовых активов и~картины 
текущих цен базисных инструментов. Изложение сопровождается расчетами весовых 
коэффициентов базисных инструментов оптимальных портфелей и~иллюстрируется 
графиками портфельных доходов.}
  
  \KW{базовые активы; функция рисковых предпочтений; континуальный критерий VaR 
(CC-VaR); стоимостная и~прогнозная плотности; функция относительных доходов; 
процедура Ней\-ма\-на--Пир\-со\-на; комбинированный портфель; суррогатный портфель; 
идеалистичный портфель}

\DOI{10.14357/19922264200309} 
 
%\vspace*{-6pt}


\vskip 10pt plus 9pt minus 6pt

\thispagestyle{headings}

\begin{multicols}{2}

\label{st\stat}
  
   
  \section{Введение}
  
  Работа затрагивает круг проблем, связанных с~применением на финансовых 
рынках континуального критерия VaR (CC-VaR), который был введен 
и~изучался автором ранее~[1--5]. Исследование служит непосредственным 
продолжением работы~[6] и~использует ее идеи и~конструкции. В~работе~[6] 
последовательно излагалась методология применения CC-VaR сразу на 
совокупности трех теоретических континуальных рынков разных размерностей, 
связанных между собой базовыми активами, вводились правила переключения 
между рынками, формировались базисы из инструментов этих рынков 
с~привлечением механизма рандомизации, а также строились в~таких базисах 
оптимальные по CC-VaR комбинированные портфели. 
  
  В данной работе методология, развитая в~[6] для модели теоретических 
континуальных рынков, переносится на дискретные сценарные рынки с~\mbox{целью} 
проверки действенности модели на примерах и~иллюстрации результатов. 
  
  Вновь рассматривается совокупность трех рынков. Один из них является 
двумерным рынком, а два других~--- одномерными. При этом базовые активы 
одномерных рынков образуют пару базовых активов двумерного. Решение 
ищется в~форме совмещения трех сценарных портфелей на своих рынках. Оно 
основывается на анализе расхождений в~относительных доходах по сценариям, 
которые обусловлены расхождениями в~ценах на разных рынках. При этом 
наряду с~совокупностью из трех рынков для полноты исследования 
рассматриваются и~две усеченные, состоящие из пары рынков. Оптимальный 
комбинированный портфель строится с~применением идей и~техники 
рандомизации. 
  
  Предлагаемые конструкции иллюстрируются числовыми примерами 
и~графиками портфельных доходов с~использованием двумерных расширений 
стандартных бе\-та-рас\-пре\-де\-ле\-ний для описания прогноза будущих цен 
базовых активов и~картины текущих цен базисных инструментов.

\vspace*{-9pt} 
  
  \section{Сценарные рынки}
  
  \vspace*{-2pt}
  
  Рассматривается однопериодный тройственный рынок, состоящий из двух 
одномерных рынков \#X и~\#Y с~базовыми активами~$\boldsymbol{X}$ 
и~$\boldsymbol{Y}$ соответственно и~одного двумерного \#0 с~парой активов 
($\boldsymbol{X}, \boldsymbol{Y}$). Возможные цены базовых активов 
образуют\linebreak множества~{\sf X} и~{\sf Y}. Сценарная дискретизация\linebreak
 вводится 
разбиением множества ${\sf X} \hm=  [x_0, x_n)$ на~$n$~сценариев 
$S_i \hm= [x_{i-1}, x_i) \hm\subset {\sf X}$, $x_{i-1}\hm< x_i$, 
$i \hm\in I \hm= \{1, \ldots , n\}$, и~множества ${\sf Y}\hm = [y_0, y_m)$ 
на~$m$~сценариев $T_j\hm = [y_{j-1}, y_j) \hm\subset {\sf  Y}$, $y_{j-1}\hm<  y_j$, 
$j \hm\in J \hm= \{1, \ldots , m\}$. Прямым произведением одномерных 
сценариев разных рынков получаются $n\times m$ двумерных сценариев. 
Равномерное разбиение задается правилом $x_i \hm= x_0 \hm+ ih_1$, 
$h_1 \hm= (x_n \hm- x_0)/n$, $i \hm\in I$; $y_j\hm = y_0 + jh_2$, $h_2 \hm= (y_m \hm-
 y_0)/m$, $j\hm\in J$. Стоимость инструмента~$\boldsymbol{G}$ записывается 
как $\vert \boldsymbol{G}\vert$, а средний доход~--- $\| \boldsymbol{G}\|$.
  
  Связующим звеном теоретических и~сценарных рынков становятся 
инструментальные индикаторы сценариев. Их платежными функциями служат 
характеристические функции сценариев, и~они играют на сценарных рынках 
роль $\delta$-ин\-стру\-мен\-тов теоретического рынка. 
  
  Как и~для теоретического тройственного рынка, задается двумерная 
прогнозная плотность $p(x, y)$, $x\hm\in {\sf X}$, $y\hm\in {\sf Y}$. Ее 
маргинальные плотности~$p_1(x)$ и~$p_2(y)$ служат одновременно 
и~прогнозными плотностями для рынков \#X и~\#Y соответственно. Однако 
подобное свойство в~отношении стоимостных плотностей может и~не 
выполняться, и~для рынка \#0 задается стоимостная плотность $c(x, y)$, для 
рынка \#X~--- $c_{\mathrm{X}}(x)$, для рынка \#Y~--- $c_{\mathrm{Y}}(y)$, 
$x\hm\in {\sf  X}$, $y\hm\in {\sf Y}$. Отметим еще, что эти плотности вводятся 
для удобства формирования картины цен для сценарных рынков, хотя 
достаточно ограничиться дискретным набором стоимостей. 
  
  Индикаторы сценариев 
$\boldsymbol{D}_{ij}\hm = \boldsymbol{H}_{ij} (= \boldsymbol{H}\{S_i\hm\times T_j\})$ 
служат базисными инструментами на рынке \#0, 
$\boldsymbol{D}_{\mathrm{X};i}\hm = \boldsymbol{H}_{\mathrm{X};i}$~--- на рынке 
\#X, $\boldsymbol{D}_{\mathrm{Y};j}\hm = \boldsymbol{H}_{\mathrm{Y};j}$~--- на 
рынке~\#Y. Их цены и~вероятности (средние доходы) образуют 
векторы~$\boldsymbol{c}$ и~$\boldsymbol{p}$, $\boldsymbol{c}_{\mathrm{X}}$ 
и~$\boldsymbol{p}_{\mathrm{X}}$, $\boldsymbol{c}_{\mathrm{Y}}$ 
и~$\boldsymbol{p}_{\mathrm{Y}}$ с~компонентами соответственно ($i \hm\in I$, 
$j \hm\in J$):
  \begin{align*}
  c_{ij}=\left\vert  \boldsymbol{D}_{ij}\right\vert &= 
  \int\limits_{y_{j-1}}^{y_j} 
\int\limits^{x_i}_{x_{i-1}} c_{ij}(x,y)\,dxdy\,,\\
  p_{ij} &= \left\|  \boldsymbol{D}_{ij}\right\| = \int\limits_{y_{j-1}}^{y_j} 
\int\limits^{x_i}_{x_{i-1}} p_{ij}(x,y)\,dxdy\,;\\
  c_{\mathrm{X};i} =\left\vert \boldsymbol{D}_{\mathrm{X};i}\right\vert 
&=\int\limits_{x_{i-1}}^{x_i} {c}_{\mathrm{X}}(x)\,dx\,,\notag\\ 
  p_{\mathrm{X};i} &=\left\| \boldsymbol{D}_{\mathrm{X};i}\right\| 
=\int\limits_{x_{i-1}}^{x_i} \boldsymbol{p}_1(x)\,dx\,;%\label{e1-ags}
\\
  c_{\mathrm{Y};j} =\left\vert \boldsymbol{D}_{\mathrm{Y};j}\right\vert 
&=\int\limits_{y_{j-1}}^{y_j} {c}_{\mathrm{Y}}(y)\,dy\,,\notag\\ 
   p_{\mathrm{Y};j} &=\left\| \boldsymbol{D}_{\mathrm{Y};j}\right\| 
   =\int\limits_{y_{j-1}}^{y_j} p_2 (y)\,dy \,. %\label{e2-ags}
\end{align*}
  
  \section{Оптимизация на~сценарном тройственном рынке}
  
  Под оптимизацией на сценарном рынке понимается построение сценарного 
портфеля при\-ме\-нением дискретного алгоритма, получаемого про\-ецированием 
на сценарный рынок алгоритма\linebreak оптимизации для теоретического рынка. Как 
обычно в~задачах с~CC-VaR, в~качестве дискретного здесь используется 
\textit{стандартный дискретный алгоритм}~\cite{2-ags, 3-ags, 4-ags, 5-ags}, 
основанный на процедуре Ней\-ма\-на--Пир\-со\-на~[7]. В~алгоритме основные 
агрегаты, такие как~$\boldsymbol{p}$ и~$\boldsymbol{c}$, участвуют 
в~векторной форме (так же и~для двумерного рынка). 
  
  В краткой форме он записывается последовательностью операций: 
\begin{multline*}
\boldsymbol{\rho} = \fr{\boldsymbol{p}}{\boldsymbol{c}}\,,\enskip 
\boldsymbol{\xi} = \mathbf{O}(\boldsymbol{\rho})\,,\enskip
\boldsymbol{\eta}= \mathbf{O}(\boldsymbol{\xi})\,, \enskip
\boldsymbol{d} =\boldsymbol{p}(\boldsymbol{\xi})\,,\\
\mathbf{T} = \left\{
t_{ij} =
\begin{cases}
1,& i\leq j;\\
0,& i> j
\end{cases}\right\}
%\right\}
\,,\enskip 
\boldsymbol{\varepsilon}= \mathbf{T}\cdot \boldsymbol{d}\,;\\
\boldsymbol{b}= \phi(\boldsymbol{\varepsilon})\,,\enskip
\boldsymbol{g}= \boldsymbol{b}(\boldsymbol{\eta})\,, 
\end{multline*}
 где $\mathbf{O}$~---  
оператор упорядочения векторов в~порядке возрастания их компонент.
  
  Перед запуском его работы с~целью построения оптимального 
комбинированного портфеля в~базисе, составленном из инструментов трех 
рынков, потребуется специальная подготовка исходных данных. 
  
  \subsection{Правила замещения }
  
  Представления портфелей \#0, \#X и~\#Y с~весовыми векторами 
$\boldsymbol{g}$, $\boldsymbol{g}_{\mathrm{X}}$ 
и~$\boldsymbol{g}_{\mathrm{Y}}$ соответственно имеют вид:
\begin{align*}
  \boldsymbol{G}&=\sum\limits_{\substack{
  {i\in I}\\{j\in J}}} g_{ij}\boldsymbol{D}_{ij}\,;\\
  \boldsymbol{G}_{\mathrm{X}}&= \sum\limits_{i\in I} g_{\mathrm{X};i} 
\boldsymbol{D}_{\mathrm{X},i}\,;\\
 \boldsymbol{G}_{\mathrm{Y}} &=\sum\limits_{j\in J} g_{\mathrm{Y};j}
 \boldsymbol{D}_{\mathrm{Y};j}\,.
\end{align*}
  
  Применение CC-VaR предполагает анализ относительных доходов 
   $\rho_{ij}\hm\equiv p_{ij}/c_{ij}$, 
   $\rho_{\mathrm{X};i}\hm\equiv p_{1;i}/c_{\mathrm{X};i}$ 
и~$\rho_{\mathrm{Y};j}\hm\equiv p_{2;j}/c_{\mathrm{Y};j}$, $i\hm\in I$, $j\hm\in J$, 
для рынков \#0, \#X и~\#Y соответственно.

  С помощью этих характеристик основные \textit{правила замещения}~(4)--(6) 
в~[6], одновременно вводящие двумерные множества индексов (сценариев) 
$M_0$, $M_1$, $M_2 \hm\subset I\times  J$, теперь записываются в~виде:

\noindent
  \begin{align}
  (i,j)\in M_0 &\Leftrightarrow \left\{ \rho_{ij}\geq 
\rho_{\mathrm{X};i}\&\rho_{ij}\geq \rho_{\mathrm{Y};j}\right\}\,;\label{e3-ags}\\
  (i,j)\in M_1 &\Leftrightarrow \left\{ 
\rho_{\mathrm{X};i}>\rho_{ij}\&\rho_{\mathrm{X};i}\geq 
\rho_{\mathrm{Y};j}\right\}\,;\label{e4-ags}\\
  (i,j)\in M_2 &\Leftrightarrow  
\left\{ \rho_{\mathrm{Y};j}>\rho_{ij}\&\rho_{\mathrm{Y};j}> 
\rho_{\mathrm{X};i}\right\}\,.\label{e5-ags}
  \end{align}
  
  Множества $M_0$, $M_1$ и~$M_2$ взаимно не пересекаются 
и~в~объединении дают полное множество ${\sf X}\times {\sf Y}$. Функция 
замещения в~дискретном случае трансформируется в~\textit{матрицу 
замещений} (по\-ме\-ча\-ющую сценарии индексами~1, 2, 3):
  \begin{multline}
  \mathbf{A}=\left\| a_{ij}\right\|\,,\enskip a_{ij}=k\Leftrightarrow (i,j)\in M_k\,,\\
  k=0,1,2,\enskip i\in I\,,\enskip j\in J\,.
  \label{e6-ags}
  \end{multline}
  
  Для рынков \#X и~\#Y используются векторы $\boldsymbol{a}_{\mathrm{X}}$ 
и~$\boldsymbol{a}_{\mathrm{Y}}$ с~компонентами: 
  \begin{align}
  a_{\mathrm{X};i}&= \left.
  \begin{cases}
  1, & \exists j\in J: a_{ij}=1\,;\\
  0 & \mbox{иначе}
  \end{cases}\right\}\,,\ 
  i\in I\,;
  \label{e7-ags}
  \\
  a_{\mathrm{Y};j} &= \left.\begin{cases}
  1\,, & \exists i\in I: a_{ij}=2\,;\\
  0 & \mbox{иначе}
  \end{cases}\right\}\,,\
  j\in J\,.
  \label{e8-ags}
  \end{align}
  
  Первый выделяет сценарии замещения на рынке \#X, второй~--- на 
рынке~\#Y. 
  
  Рассматриваемую задачу, совмещающую все три рынка, назовем 
задачей~$A$. Аналогично могут быть рассмотрены и~две ее усеченные 
постановки~--- $B$ и~$C$. В~задаче~$B$ исключается рынок \#Y и~метка 
$k\hm = 2$ не используется, а ее правила замещения:
  \begin{align*}
  (i,j)\in M_0 &\Leftrightarrow \left\{ \rho_{ij}\geq \rho_{\mathrm{X};i}\right\}\,;\\
  (i,j)\in M_1 &\Leftrightarrow \left\{ \rho_{\mathrm{X};i}>\rho_{ij}\right\}\,.
  \end{align*}
  
  В задаче~$C$ исключается рынок \#0 и~не используется метка $k\hm = 0$, 
а~ее правила замещения: 
  \begin{align*}
  (i,j)\in M_1 &\Leftrightarrow \left\{ \rho_{\mathrm{X};i}\geq 
\rho_{\mathrm{Y};j}\right\}\,;\\
  (i,j) \in M_2 &\Leftrightarrow \left\{ \rho_{\mathrm{Y};j}> 
\rho_{\mathrm{X};i}\right\}\,.
  \end{align*}
  
  \subsection{Комбинированный портфель }
  
  По аналогии с~теоретической схемой и~в~соответствии с~правилами~(3)--(5) 
строится \textit{комбинированный} портфель замещением тех базисных 
инструментов рынка~\#0, для которых имеется базисный инструмент 
рынка~\#X или~\#Y с~б$\acute{\mbox{о}}$льшим относительным доходом. При 
этом также применяется рандомизация. 
  
  Формулы~(8)--(20) из~[6] для теоретического рынка переписываются 
в~несколько сокращенном виде применительно к~сценарному рынку 
с~естественной заменой переменных~$s$ и~$t$ индексами~$i$ и~$j$, 
а~множества~{\sf X} и~{\sf Y}~--- множествами индексов~$I$ и~$J$. 
  
  Вводятся множества $M_{1;i}\hm\subset J$ 
  и~$M_{2;j}\hm\subset I$ как максимальные 
подмножества множеств $M_1$ и~$M_2$ для фиксированных значений 
$i\hm\in I$ и~$j \hm\in J$ соответственно. Также определяются 
индикаторы~$\boldsymbol{M}_{1;i}$ и~$\boldsymbol{M}_{2;j}$ рынка~\#0 как 
объединение базисных инструментов~$\boldsymbol{D}_{ij}$ по $j \hm\in M_{1;i}$ 
и~$i\hm\in M_{2;j}$ соответственно: 
  \begin{multline}
   \boldsymbol{M}_{1;i}= \sum\limits_{j\in 
M_{1;i}}\boldsymbol{D}_{ij}=\boldsymbol{D}_{1;i}\times 
\boldsymbol{H}_2\left\{ M_{1;i}\right\}\,,\\
  \left\vert \boldsymbol{M}_{1;i}\right\vert = \sum\limits_{j\in M_{1;i}} 
c_{ij}\,,\enskip  \left\| \boldsymbol{M}_{1;i}\right\| =\sum\limits_{j\in M_{1;i}} 
p_{ij}\,;
  \label{e9-ags}
\end{multline}

\noindent
\begin{multline*}
  \boldsymbol{M}_{2;j}= \sum\limits_{i\in 
M_{2;j}}\boldsymbol{D}_{ij}=\boldsymbol{D}_{2;j}\times 
\boldsymbol{H}_1\left\{ M_{2;j}\right\}\,,\\
  \left\vert \boldsymbol{M}_{2;j}\right\vert= \sum\limits_{i\in M_{2;j}} 
c_{ij}\,,\enskip  \left\| \boldsymbol{M}_{2;j}\right\| =
\sum\limits_{i\in M_{2;j}} 
p_{ij}\,.
    \end{multline*}
  %
  Эти инструменты желательно было бы заместить <<гиб\-рид\-ны\-ми>>, 
совмещающими рынки \#0 и~\#X или \#0 и~\#Y, а~именно:
  \begin{equation}
  \left.
  \begin{array}{rl}
  \boldsymbol{M}_{\mathrm{X};i} &= \boldsymbol{D}_{\mathrm{X};i}\times 
\boldsymbol{H}_2\left\{ M_{1;i}\right\}\,;\\[6pt]
  \boldsymbol{M}_{\mathrm{Y};j} &= \boldsymbol{D}_{\mathrm{Y};j}\times 
\boldsymbol{H}_1\left\{ M_{2;j}\right\}\,.
  \end{array}
  \right\}
  \label{e10-ags}
  \end{equation}
  
  Поскольку таких инструментов на рассматриваемых рынках нет, для 
образования их реализуемого аналога на рынках \#X и~\#Y применяется 
рандомизация. 
  
  На рынке \#X для всех сценариев $i\hm\in I$ вводятся взаимонезависимые 
\textit{биномиальные} случайные величины $\vartheta_{\mathrm{X};i}$, 
вероятности \textit{успеха} (замещения) которых соответственно 
равны~$\theta_{\mathrm{X};i}$ и~в~совокупности образуют 
вектор~$\boldsymbol{\theta}_{\mathrm{X}}$:
  \begin{multline}
   \theta_{\mathrm{X};i} = {\sf P}\left\{ 
  M_{1;i}\vert i\right\} =\sum\limits_{j\in 
M_{1;i}} \fr{p_{ij}}{p_{1;i}}\,,\\
 \boldsymbol{\theta}_{\mathrm{X}}=\left\{ 
\theta_{\mathrm{X};i},\ i\in I\right\}\,.
  \label{e11-ags}
  \end{multline}
  
  Эта рандомизация встраивается непосредственно в~базисные инструменты 
одномерной части \#X комбинированного портфеля. Новые базисные 
инструменты становятся случайными, совпадающими с~инструментами 
$\boldsymbol{D}_{\mathrm{X};i}$ с~вероятностью~$\theta_{\mathrm{X};i}$ 
и~с~\textit{нулевыми} инструментами (с~нулевым доходом и~нулевой 
стоимостью) $\boldsymbol{N}_{\mathrm{X};i}$ с~вероятностью $1\hm- 
\theta_{\mathrm{X};i}$. Формально эти инструменты, их стоимость и~средний 
доход (также вероятность) соответственно равны: 
  \begin{multline}
   \boldsymbol{D}_{\mathrm{X};i}^{\mathrm{cmb}}=\vartheta_{\mathrm{X};i} 
\boldsymbol{D}_{\mathrm{X};i}\,,\enskip
  \left\vert  \boldsymbol{D}^{\mathrm{cmb}}_{\mathrm{X};i}\right\vert= 
c^{\mathrm{cmb}}_{\mathrm{X};i}=\theta_{\mathrm{X}} c_{\mathrm{X};i}\,,\\
  \left\| \boldsymbol{D}^{\mathrm{cmb}}_{\mathrm{X};i}\right\| =
  p^{\mathrm{cmb}}_{\mathrm{X};i} 
=\theta_{\mathrm{X};i} p_{1;i}\,,\enskip i\in I\,.
   \label{e12-ags}
  \end{multline}
  
  Назначение~(\ref{e11-ags}) параметров~$\theta_{\mathrm{X};i}$ уравнивает 
вероятности, связанные с~инструментами 
$\boldsymbol{M}_{\mathrm{X};i}$~(\ref{e10-ags}) и~$\boldsymbol{M}_{1;i}$~(\ref{e9-ags}). 
И~подтверждением этому служит именно третье соотношение  
в~(\ref{e12-ags}). 
  
  Аналогично для рынка \#Y вводятся случайные величины 
$\vartheta_{\mathrm{Y};j}$, $j\hm\in J$, с~вероятностью 
замещения~$\theta_{\mathrm{Y};j}$, и~тогда
  \begin{equation}
  \left.
  \begin{array}{c}
  \!\hspace*{-2.5mm}\boldsymbol{D}^{\mathrm{cmb}}_{\mathrm{Y};j}=\vartheta_{\mathrm{Y};j} 
\boldsymbol{D}_{\mathrm{Y};j}\,,\
  \left\vert \boldsymbol{D}^{\mathrm{cmb}}_{\mathrm{Y};j}\right\vert = 
c^{\mathrm{cmb}}_{\mathrm{Y};j}=\theta_{\mathrm{Y};j} c_{\mathrm{Y};j}\,,\\[6pt]
  \!\hspace*{-2.5mm}\left\| \boldsymbol{D}^{\mathrm{cmb}}_{\mathrm{Y};j} \right\| = 
p^{\mathrm{cmb}}_{\mathrm{Y};j}=\theta_{\mathrm{Y};j} p_{2;j}\,,\\[6pt]
  \!\hspace*{-2.5mm}\theta_{\mathrm{Y};j} ={\sf P}\left\{ M_{2;j}\vert j\right\} =
  \displaystyle\sum\limits_{i\in M_{2;j}} 
  \fr{p_{ij}}{p_{2;j}}\,,\\
  \!\hspace*{-2.5mm}\boldsymbol{\theta}_{\mathrm{Y}} = \left\{ 
\theta_{\mathrm{Y};j}\,,\ j\in J\right\}.
  \end{array}\!
  \right\}\!\!
  \label{e13-ags}
  \end{equation}
  
  Затем для нового базиса формируется новая \textit{единая функция} 
относительных доходов, и~к~ней применяется комбинированный алгоритм 
оптимизации, находящий весовые коэффициенты оптимального (случайного) 
портфеля. Имеем: 
  \begin{multline}
  \boldsymbol{G}^{\mathrm{cmb}}=\sum\limits_{a_{ij}=0} 
g_{ij}^{\mathrm{cmb}}\boldsymbol{D}_{ij} +
\sum\limits_{i\in I} g^{\mathrm{cmb}}_{\mathrm{X};i} 
\vartheta_{1;i} \boldsymbol{D}_{\mathrm{X};i}+ {}\\
{}+\sum\limits_{j\in J} 
g^{\mathrm{cmb}}_{\mathrm{Y};j} \vartheta_{2;j}\boldsymbol{D}_{\mathrm{Y};j}\,.
  \label{e14-ags}
  \end{multline}
%  
  Это представление порождает и~дискретную \textit{идеалистичную} версию 
комбинированного портфеля с~эквивалентной платежной функцией. Она 
получается заменой случайных базисных инструментов нереализуемыми на 
рынке детерминированными~(\ref{e10-ags}) (с~сохранением всех весовых 
коэффициентов):
{\looseness=-1

}

\noindent
\begin{multline*}
  \boldsymbol{G}^{\mathrm{idl}}={}\\
  {}=\sum\limits_{(i,j)\in M_0} g_{ij}^{\mathrm{cmb}} 
\boldsymbol{D}_{ij} +\sum\limits_{i\in I} 
g_{\mathrm{X};i}^{\mathrm{cmb}}\boldsymbol{M}_{\mathrm{X};i}+\sum\limits_{j\in J} 
g^{\mathrm{cmb}}_{Y;j}\boldsymbol{M}_{\mathrm{Y};j}.
  \end{multline*}
  
  Для графической иллюстрации платежной функции идеалистичного 
портфеля ее удобно рассчитывать по формуле:
  \begin{equation}
  \boldsymbol{\pi}_{ij}^{\mathrm{idl}}(x,y)=\max \left( \boldsymbol{\pi}_{0;ij}^{\mathrm{idl}} (x,y), 
\boldsymbol{\pi}_{\mathrm{X};i}^{\mathrm{idl}}(x), 
\boldsymbol{\pi}_{\mathrm{Y};j}^{\mathrm{idl}}(y)\right).\!\!
  \label{e15-ags}
  \end{equation}
  
  В задачах~$B$ и~$C$ соответственно: 
  \begin{align*}
  \boldsymbol{G}^{\mathrm{cmb}} &= \sum\limits_{a_{ij}=0} 
g_{ij}^{\mathrm{cmb}}\boldsymbol{D}_{ij} + \sum\limits_{i\in I} 
g^{\mathrm{cmb}}_{\mathrm{X};i}\vartheta_{1;i} \boldsymbol{D}_{\mathrm{X};j}\,;\\
  \boldsymbol{G}^{\mathrm{cmb}} &= \sum\limits_{i\in I} 
g_{\mathrm{X};i}^{\mathrm{cmb}}\vartheta_{1;i}\boldsymbol{D}_{\mathrm{X};i} + 
\sum\limits_{j\in J} g^{\mathrm{cmb}}_{\mathrm{Y};j}\vartheta_{2;j} 
\boldsymbol{D}_{\mathrm{Y};j}\,.
  \end{align*}
  %
   В этих двух представлениях сохраняются те же обозначения для весовых коэффициентов 
   и~величин $\vartheta_{\mathrm{X};i}$ и~$\vartheta_{\mathrm{Y};j}$, что и~в~задаче~$A$, но все они, 
вообще говоря, другие, так как получаются из иных \textit{правил замещений}. Также соответственно 

\noindent
  \begin{equation}
  \left.
  \begin{array}{l}
  \boldsymbol{G}^{\mathrm{idl}}=\displaystyle\sum\limits_{(i,j)\in M_0} 
g_{ij}^{\mathrm{cmb}}\boldsymbol{D}_{ij} +
\displaystyle\sum\limits_{i\in I} g^{\mathrm{cmb}}_{\mathrm{X};i} 
\boldsymbol{M}_{\mathrm{X};i}\,;\\[12pt]
  \boldsymbol{G}^{\mathrm{idl}}=\displaystyle \sum\limits_{i\in I} 
g_{\mathrm{X};i}^{\mathrm{cmb}}\boldsymbol{M}_{\mathrm{X};i} +\sum\limits_{j\in J} 
g^{\mathrm{cmb}}_{\mathrm{Y};j} \boldsymbol{M}_{\mathrm{Y};j}\,;\\[12pt]
  \boldsymbol{\pi}_{ij}^{\mathrm{idl}}(x,y)=\max\left( \boldsymbol{\pi}_{0;ij}^{\mathrm{idl}} (x,y), 
\boldsymbol{\pi}^{\mathrm{idl}}_{\mathrm{X};i}(x)\right);\\[9pt]
  \boldsymbol{\pi}_{ij}^{\mathrm{idl}}(x,y)=\max\left( 
\boldsymbol{\pi}_{\mathrm{X};i}^{\mathrm{idl}} (x), 
\boldsymbol{\pi}^{\mathrm{idl}}_{\mathrm{Y};j}(y)\right).
  \end{array}
  \right\}
  \label{e16-ags}
  \end{equation}
  
  \subsection{Суррогатный портфель }
  
  Наряду с~комбинированным строится и~портфель, называемый 
\textit{суррогатным}. Он получается в~результате формальной замены 
базисных инструментов для всех сценариев (по необходимости) новыми, для 
которых платежные функции и~вероятности те же, что и~для инструментов 
$\boldsymbol{D}_{ij}$, $i \hm\in I$, $j \hm\in J$, рынка~\#0, но цены 
корректируются соответственно правилам~(\ref{e3-ags})--(\ref{e5-ags}) 
и~ценам рынков~\#X и~\#Y: 
  \begin{equation}
  c_{ij}^{\mathrm{srg}}=%\left\{ 
  \begin{cases}
  c_{ij}, & (i,j)\in M_0\,;\\
  \fr{p_{ij}}{\rho_{\mathrm{X};i}},  &(i,j)\in M_1\,;\\
  \fr{p_{ij}}{\rho_{\mathrm{Y};j}}, &(i,j)\in M_2\,.
  \end{cases}
  %\right\}
  \label{e17-ags}
  \end{equation}
  %
  С этими ценами вновь образуется матрица относительных доходов, алгоритм 
находит матрицу новых весов портфеля, а~сам портфель имеет вид: 
  \begin{equation}
  \boldsymbol{G}^{\mathrm{srg}}=\sum\limits_{i\in I, j\in J} g_{ij}^{\mathrm{srg}} 
\boldsymbol{D}_{ij}^{\mathrm{srg}}\,.
  \label{e18-ags}
  \end{equation}
  
  На сценарном рынке числовые показатели для суррогатного портфеля из-за 
замены части базисных инструментов будут, как правило, слегка отличаться от 
комбинированного (и идеалистичного) портфеля. Тем не менее относительно 
просто конструируемый \textit{суррогатный} портфель можно использовать 
как дополнительное средство проверки всех расчетов (и графиков доходов). 
  
  В задачах~$B$ и~$C$ иными становятся цены: 
  \begin{align*}
  c_{ij}^{\mathrm{srg}} &= %\left\{ 
  \begin{cases}
  c_{ij}, &(i,j)\in M_0;\\
  \fr{p_{ij}}{\rho_{\mathrm{X};i}}, & (i,j)\in M_1\,;
    \end{cases}%\right\}
    \\
  c_{ij}^{srg} &= 
  %\left\{ 
  \begin{cases}
  \fr{p_{ij}}{\rho_{\mathrm{X};i}},& (i,j)\in M_1;\\
\fr{p_{ij}}{\rho_{\mathrm{Y};j}}, &(i,j)\in M_2\,.
\end{cases}
%\right\}
  \end{align*}
  %
  По этим ценам пересчитываются матрицы относительных доходов и~новых 
весов портфелей, а~для построения портфелей применяется та же 
формула~(\ref{e18-ags}). 
  
  \section{Алгоритм построения комбинированного портфеля}
   
  Построение комбинированного портфеля (\textit{комбинированный} 
алгоритм) и~его идеалистичной и~суррогатной версий основано на применении 
\textit{стандартного} алгоритма~\cite{2-ags, 3-ags, 4-ags, 5-ags}, краткая запись 
которого приводится в~начале разд.~3. Для его запуска требуется 
предварительная (до применения \textit{стандартного} алгоритма) 
трансформация мат\-риц~$\boldsymbol{c}$, $\boldsymbol{p}$ 
и~$\boldsymbol{\rho}$ для рынка \#0 и~векторов $\boldsymbol{c}_{\mathrm{X}}$, 
$\boldsymbol{p}_{\mathrm{X}}$ и~$\boldsymbol{\rho}_{\mathrm{X}}$ для 
рынка~\#X и~$\boldsymbol{c}_{\mathrm{Y}}$, $\boldsymbol{p}_{\mathrm{Y}}$ 
и~$\boldsymbol{\rho}_{\mathrm{Y}}$ для рынка~\#Y. 
  
  Схема \textit{комбинированного} алгоритма приводится без технических 
подробностей. Она состоит из нескольких последовательных блоков. 
  \begin{enumerate}[1.]
  \item  Нахождение матрицы замещений~$\mathbf{A}$ и~по\-стро\-ение матриц 
$\boldsymbol{c}_{0\mathrm{N}}$, $\boldsymbol{p}_{0\mathrm{N}}$ 
и~$\boldsymbol{\rho}_{0\mathrm{N}}$ операцией обнуления  
в~мат\-ри\-цах~$\boldsymbol{c}$, $\boldsymbol{p}$ и~$\boldsymbol{\rho}$ всех 
элементов $(i,j)$, для которых $a_{ij}\not= 0$. 
  \item Образование матриц $\boldsymbol{p}_{\mathrm{XN}}$ 
и~$\boldsymbol{p}_{\mathrm{YN}}$ обнулением всех элементов  
в~мат\-ри\-це~$\boldsymbol{p}$, для которых соответственно $a_{ij}\not=1$ 
и~$a_{ij}\not= 2$. 
  \item  Определение векторов суммарной вероятности замещения 
$\boldsymbol{p}_{\mathrm{XM}}\hm = \left\{ {\sf P}\{M_{1;i}\},\ i \hm\in I\right\}$ 
и~$\boldsymbol{p}_{\mathrm{YM}}\hm = \left\{ {\sf P}\{ M_{2;j}\},\ j \hm\in J\right\}$ 
на основе результатов п.~2, а~также векторов 
$\boldsymbol{\theta}_{\mathrm{X}}\hm = \left\{\theta_{\mathrm{X};i}, 
i \hm\in I\right\}$ 
и~$\boldsymbol{\theta}_{\mathrm{Y}}\hm = \left\{\theta_{\mathrm{Y};j}, 
j \hm\in J\right\}$ согласно (\ref{e11-ags}) и~(\ref{e13-ags}) соответственно. 
  \item  Вычисление (в~дополнение к~$\boldsymbol{p}_{\mathrm{XM}}$ 
и~$\boldsymbol{p}_{\mathrm{YM}}$) векторов 
$\boldsymbol{c}_{\mathrm{XM}}\hm= \{a_{\mathrm{X};i} 
\theta_{\mathrm{X};i}c_{\mathrm{X};i},\ i\hm\in I\}$,
  $\boldsymbol{c}_{\mathrm{YM}}\hm= \{a_{\mathrm{Y};j} 
\theta_{\mathrm{Y};j}c_{\mathrm{Y};j},\  
j\hm\in J\}$~(\ref{e11-ags})--(\ref{e13-ags}) 
и~$\boldsymbol{\rho}_{\mathrm{XM}}\hm= \{ a_{\mathrm{X};i} 
\rho_{\mathrm{X};i},\ i\hm\in I\}$, $\boldsymbol{\rho}_{\mathrm{YM}}\hm= \{ 
a_{\mathrm{X};i}\rho_{\mathrm{Y};j}, \ j\hm\in J\}$ (параметры 
$a_{\mathrm{X};i}$~(\ref{e7-ags}) и~$a_{\mathrm{Y};j}$~(\ref{e8-ags}) 
используются во избежание деления на нуль).
  \item Преобразование матриц $\boldsymbol{c}_{\mathrm{0N}}$, 
$\boldsymbol{p}_{\mathrm{0N}}$ и~$\boldsymbol{\rho}_{\mathrm{0N}}$ 
в~векторы (без изменения обозначений) и~формирование комбинированной 
тройки векторов ($\boldsymbol{c}_{\mathrm{F}}$, 
$\boldsymbol{p}_{\mathrm{F}}$ и~$\boldsymbol{\rho}_{\mathrm{F}}$) 
размерности $nm \hm+ n \hm+ m$ каждый операцией конкатенации: 
$\boldsymbol{c}_{\mathrm{F}}\hm= (\boldsymbol{c}_{\mathrm{0N}}, 
\boldsymbol{c}_{\mathrm{XM}}, \boldsymbol{c}_{\mathrm{YM}})$; 
$\boldsymbol{p}_{\mathrm{F}}\hm = (\boldsymbol{p}_{\mathrm{0N}}, 
\boldsymbol{p}_{\mathrm{XM}}, \boldsymbol{p}_{\mathrm{YM}})$; 
$\boldsymbol{\rho}_{\mathrm{F}}\hm = (\boldsymbol{\rho}_{\mathrm{0N}}, 
\boldsymbol{\rho}_{\mathrm{XM}}, \boldsymbol{\rho}_{\mathrm{YM}})$. 
  \item Применение \textit{стандартного} алгоритма оптимизации к~тройке 
векторов ($\boldsymbol{c}_{\mathrm{F}}$, $\boldsymbol{p}_{\mathrm{F}}$, 
$\boldsymbol{\rho}_{\mathrm{F}}$) для нахождения 
вектора~$\boldsymbol{g}_{\mathrm{F}}$ весовых коэффициентов 
оптимального комбинированного портфеля с~разбиением на три подвектора: 
$\boldsymbol{g}_{\mathrm{F}}\hm= (\boldsymbol{g}_{\mathrm{0F}}, 
\boldsymbol{g}_{\mathrm{XF}}, \boldsymbol{g}_{\mathrm{YF}})$. 
  
  {\small \textit{Примечание.} В~наборах, получаемых в~п.~5, в~результате обнуления в~п.~1 
и~п.~2 содержится, как правило, значительная доля нулевых троек. Но это не сказывается на 
окончательном результате п.~6, так как этим тройкам придается нулевой вес. Зато такой 
вариант алгоритма упрощает вычисления. Тем не менее из соображений строгости и~чистоты 
алгоритма, а также буквального следования представлению~(\ref{e14-ags}), но уже 
с~векторами $\boldsymbol{g}^{\mathrm{cmb}}$, 
$\boldsymbol{g}^{\mathrm{cmb}}_{\mathrm{X}}$ 
и~$\boldsymbol{g}_{\mathrm{Y}}^{\mathrm{cmb}}$, все подвергшиеся обнулению элементы нужно 
было бы удалить из наборов с~запоминанием их положения для последующего выхода на 
представление~(\ref{e14-ags}).}
  
  
  \item  Вычисление записей основных результатов инвестиции 
$\boldsymbol{L}\hm= \langle A, R, y\rangle$ ($A$~--- инвестиционная сумма; 
$R$~--- средний доход; $y$~--- средняя доходность) и~построение графиков 
платежных функций 
 портфелей~(\ref{e13-ags})--(\ref{e16-ags}) как по частям, так и~совместно 
(в~варианте идеалистичного портфеля). 
  \item Построение суррогатного портфеля путем формирования стоимостного 
$\boldsymbol{c}^{\mathrm{srg}}$ по формуле~(\ref{e17-ags}) и~прогнозного 
$\boldsymbol{p}^{\mathrm{srg}} (\equiv \boldsymbol{p}$) векторов размерности $n\times 
m$, образованием из них вектора относительных 
доходов~$\boldsymbol{\rho}^{\mathrm{srg}}$ той же размерности с~последующим 
применением \textit{стандартного} алгоритма. 
  \end{enumerate}
  
  \section{Иллюстративные примеры}
  
  Рассматриваются примеры с~${\sf X}\hm= {\sf Y}\hm= [0,1)$, при этом $x, 
s\hm\in {\sf X}$, $y,t\hm\in {\sf Y}$. В~них используется функция рисковых 
предпочтений инвестора $\phi(\varepsilon)\hm=\varepsilon^2$, 
$\varepsilon\hm\in [0, 1]$, а при задании стоимостных и~вероятностных 
характеристик существенная роль отводится двухпараметрическому 
бе\-та-рас\-пре\-де\-ле\-нию $\mathrm{Be}(u, v)$ с~плотностью
  $$
  \beta(x;u,v)\hm=\fr{x^{u-1}(1-x)^{v-1}}{B(u,v)} \,,\enskip u, v>0\,,
  $$
где 
     $B(u,v)\hm= \int\nolimits_0^1 x^{u-1} (1\hm-x)^{v-1}\,dx 
     \hm= \Gamma(u)\Gamma(v)/\Gamma(u\hm+v)$
и~$\Gamma(u)\hm= \int\nolimits_0^{\infty} x^{u\hm-1} \exp(-x)\,dx$~--- 
соответственно бе\-та- и гам\-ма-функ\-ции. 

  Для рынка \#0 с~помощью бе\-та-плот\-ности предварительно строится 
нормированная двумерная плотность $f(x,y; u,v)$: 
 \begin{equation}
 \left.
 \begin{array}{rl}
  f^\prime (x,y; u,v)&=
%\left\{ 
\begin{cases} 
x^{u-1}(1-y)^{v-1}\,, & x\leq y;\\
   y^{u-1}(1-x)^{v-1}\,,& x>y;\end{cases}
   %\right\}
  \\[12pt]
  N(u,v)&=\displaystyle\int\limits_0^1 \int\limits_0^1 f^\prime (x,y;u,v)\,dxdy\,;\\[6pt]
  f(x,y;u,v)&=\fr{f^\prime(x,y; u,v)}{N(u,v)}\,,
  \end{array}
  \right\}
  \label{e19-ags}
  \end{equation}
  %
  а затем стоимостную и~прогнозную плотности для этого рынка задаем 
соответственно в~виде вы\-пук\-лых комбинаций: 
  \begin{multline}
  c(x,y)=\omega_c f(x,y;2{,}8;1{,}8)+{}\\
  {}+(1-\omega_c)\beta(x;3;3)\beta(y;3;3)\,; 
\label{e20-ags}
  \end{multline}
  
  %\vspace*{-12pt}
  
  { \begin{center}  %fig1
 \vspace*{-1pt}
    \mbox{%
 \epsfxsize=76.396mm 
 \epsfbox{aga-1.eps}
 }


\vspace*{6pt}

\noindent
{{\figurename~1}\ \ \small{
Плотность $p(1 - x, y)$
}}
\end{center}}

  
     
  \vspace*{-3pt}
  
  \noindent
  \begin{multline}
  \!\!\!p(x,y)=\omega_p f(x,y;3;2)+(1-\omega_p)\beta(x;3;3)\beta(y;3;3)\,,\\
  \omega_p=0{,}5\,,\enskip \omega_c=0{,}5\,.
  \label{e21-ags}
  \end{multline}
  
  \vspace*{-2pt}
  
  \noindent
(С~ростом параметра $\omega_p\hm\in [0,1]$ корреляция между базовыми 
активами увеличивается.) На рис.~1 приводится график плотности $p(x, y)$. 


  
  Для большей наглядности график изображается с~заменой $x\hm\to 1\hm - x$. 
Различие между плотностями~(\ref{e20-ags}) и~(\ref{e21-ags}) весьма 
незначительно, и~потому график $c(x,y)$ не приводится. Отметим лишь, что 
отличие графика плотности $c(x,y)$ проявляется в~меньшем угле излома на 
диагонали квадрата $[0, 1]\times[0, 1]$ и~снижении максимальных значений. 
Такое соотношение плотностей говорит о~большем разбросе стоимостной 
плотности в~сравнении с~прогнозной. Подобная задача на языке финансового 
рынка означает \textit{продажу волатильности}. 
  
  Для стоимостных плотностей рынков \#X и~\#Y принимается также 
  
  \vspace*{-5pt}
  
\noindent
  \begin{multline}
  c_{\mathrm{X}}(x)=\chi_{\mathrm{X}} \left( \omega_x c_1(x)+(1-
\omega_x)\beta(x;2;2)\right),\\[-0.5pt]
  \omega_x=0{,}9\,;
  \label{e22-ags}
  \end{multline}
  
  \vspace*{-14pt}
  
  \noindent
  \begin{multline}
  c_{\mathrm{Y}}(y)=\chi_{\mathrm{Y}} \left( \omega_y c_2(y)+(1-
\omega_y)\beta(y;2;2)\right),\\[-0.5pt]
  \omega_y=0{,}9\,.
  \label{e23-ags}
  \end{multline}
  
  \vspace*{-4pt}
  
  \noindent
При $\chi_{\mathrm{X}}=\chi_{\mathrm{Y}}\hm=1$ эти функции приобретают 
все свойства плотности вероятности. 
  
  В отличие от стоимостных плотностей про\-гноз\-ное распределение 
вероятности, как и~в~[6], едино для всех рассматриваемых рынков, и~потому
  $p_{\mathrm{X}}(x)\hm\equiv p_1(x)$; $p_{\mathrm{Y}}(y)\hm\equiv p_2(y)$. 
  
  Дискретизация множества {\sf X} проводится при $n\hm=13$, а~{\sf Y}~--- 
при $m \hm= 12$, порождая тем самым $n\times m\hm= 156$ двумерных 
сценариев, при этом\linebreak
 стоимостные и~прогнозные характеристики опре\-деляются 
для всех сценариев, одномерных и~дву\-мерных, интегрированием 
функций~(\ref{e19-ags})--(\ref{e23-ags}).\linebreak Отметим лишь, что для дискретной 
модели при
 нахождении маргинальных цен и~вероятностей од-\linebreak\vspace*{-12pt}

\columnbreak

\noindent
номерных 
сценариев достаточно ограничиться однократным суммированием по $j\hm\in 
J$ и~$i\hm\in I$ соответственно цен и~вероятностей для двумерных сценариев. 
  
  Заметим еще, что при интегрировании плот\-ности типа 
 бе\-та-рас\-пре\-де\-ле\-ния в~пределах крайних сценариев при некоторых 
значениях параметров из интервала $(-1, 0)$ приходится иметь дело 
с~интегрируемой особенностью в~точке $x\hm= 1$. Для обеспечения 
корректности счета имеет смысл выделять из плотности <<чистую>> 
особенность, допускающую аналитическое интегрирование. Остаточный 
интеграл особенностей уже не содержит и~без труда приближенно 
интегрируется. Так, если $v\hm\in (-1,0)$ и~$u\hm>0$, то при $0\hm<\tau \hm< 1$ 
имеет место:

\vspace*{-12pt}

\noindent
  \begin{multline*}
  \int\limits^1_\tau x^u(1-x)^v \,dx={}\\[-12pt]
  {}=\int\limits_\tau^1(1-x)^v \,dx+\int\limits_\tau^1 
(x^u-1)(1-x)^v \,dx={}\\[-2pt]
  {}=\fr{1-\tau^{v+1}}{v+1} +\int\limits^1_\tau (x^u-1)(1-x)^v \,dx\,.
  \end{multline*}
  
  \vspace*{-12pt}
  
  По полученным вероятностям и~ценам находятся относительные доходы для 
рынков \#0, \#X и~\#Y, необходимые для проведения операций замещения. При 
этом двумерные массивы трансформируются в~векторы c лексикографическим 
упорядочением компонент.
  
  Применением стандартного дискретного алгоритма оптимизации ко всем 
исходным рынкам по отдельности можно находить для них векторы весов 
оптимальных портфелей и~доходности инвестиции. Так, для рынков \#0, \#X 
и~\#Y при $\chi_{\mathrm{X}}\hm=1$ и~$\chi_{\mathrm{Y}}\hm=1$ они 
соответственно равны 0{,}0423358, 0{,}0313433 и~0{,}0312055. Их можно 
использовать в~качестве ориентиров для сравнения результатов. 
  
  
  При построении комбинированного портфеля используется алгоритм разд.~4. 
С~фиксацией $\chi_{\mathrm{X}}\hm=1$ и~$\chi_{\mathrm{Y}}\hm=1$ 
находится матрица замещений~$\mathbf{A}$~(\ref{e6-ags}):

\noindent
  $$
  \left\|
  \begin{array}{cccccccccccc}
0&0&2&2&2&2&2&2&2&2&2&2\\
0&0&0&2&2&2&2&2&2&2&2&1\\
1&0&0&0&2&2&2&2&2&2&2&1\\
1&0&0&0&0&2&2&2&2&2&1&1\\
1&1&0&0&0&0&2&2&2&1&1&1\\
1&1&1&0&0&0&0&2&2&1&1&1\\
1&1&1&1&0&0&0&0&1&1&1&1\\
1&1&1&1&1&0&0&0&1&1&1&1\\
1&1&1&1&1&1&0&0&0&1&1&1\\
1&1&1&1&1&2&2&0&0&0&1&1\\
1&1&1&1&2&2&2&2&2&0&1&1\\
1&1&2&2&2&2&2&2&2&2&0&1\\
1&2&2&2&2&2&2&2&2&2&2&2
\end{array}
\right\|
\vspace*{1pt}
$$

\pagebreak

\noindent
(в~ней 35~нулей, 60~единиц и~61~двойка), а также векторы~(\ref{e7-ags}) 
и~(\ref{e8-ags}):
\begin{align*}
\boldsymbol{a}_{\mathrm{X}} &= \{0, 1, 1, 1, 1, 1, 1, 1, 1, 1, 1, 1, 1\};\\ 
\boldsymbol{a}_{\mathrm{Y}} &= \{0, 1, 1, 1, 1, 1, 1, 1, 1, 1, 1, 1\}.
\end{align*}
  
  По матрице~$\mathbf{A}$ и~распределению вероятностей на 
сценариях~(\ref{e21-ags}) (двумерных и~одномерных) определяются 
необходимые для построения комбинированного портфеля 
векторы~$\boldsymbol{\theta}_{\mathrm{X}}$ 
и~$\boldsymbol{\theta}_{\mathrm{Y}}$ параметров рандомизации, состоящие из 
вероятностей переключения на одномерные части портфеля и~вы\-чис\-ля\-емые по 
формулам~(\ref{e11-ags}) и~(\ref{e13-ags}): 
  \begin{align*}
  \boldsymbol{\theta}_{\mathrm{X}} &= \{0; 0{,}0062; 0{,}0139; 
  0{,}0434; 0{,}1385;    0{,}2017;\\
  &\hspace*{-5mm}0{,}4047;  0{,}532; 0{,}539; 0{,}309; 0{,}2489; 0{,}0788; 0{,}0012\};\\
  \boldsymbol{\theta}_{\mathrm{Y}} &= \{0; 0{,}0051; 0{,}0353; 0{,}0617; 
  0{,}1669; 0{,}3336;\\
    & 0{,}4396;  0{,}4511; 0{,}4556; 0{,}1992; 0{,}0842; 0{,}1241\}. 
  \end{align*}
  
  Результатом реализации п.~5 алгоритма оптимизации становится тройка 
векторов ($\boldsymbol{c}_{\mathrm{F}}$, $\boldsymbol{p}_{\mathrm{F}}$, 
$\boldsymbol{\rho}_{\mathrm{F}}$) размерности $nm \hm+ n \hm+ m \hm= 181$ 
каждый, а~п.~6~--- весовой вектор~$\boldsymbol{g}_{\mathrm{F}}$ той же 
длины с~сохранением всех элементов обнуления. В~них первые $nm$ 
элементов отвечают рынку \#0, следующие~$n$~--- рынку \#X, 
последние~$m$~--- рынку \#Y. Разбиение 
вектора~$\boldsymbol{g}_{\mathrm{F}}$ на такие три блока дает векторы весов 
для трех час\-тей комбинированного портфеля с~представлением,  
аналогичным~(\ref{e14-ags}) и~фактически эквивалентным ему (см.\ 
примечание к~п.~6). 
  
  
  С точностью до 10$^{-4}$ для рынков \#0, \#X и~\#Y соответственно: 
  
  \vspace*{3pt}
  
 \noindent
  $\boldsymbol{g}_{\mathrm{0F}}$\;=\;\{0; 0; 0; 0; 0; 0; 0; 0; 0; 0; 0; 0; 0; 0,0006; 0,0007; 0; 
0; 0; 0; 0; 0; 0; 0; 0; 0; 0,0011; 0,0162; 0,0107; 0; 0; 0; 0; 0; 0; 0; 0; 0; 0,001; 0,0363; 0,6184; 
0,0454; 0; 0; 0; 0; 0; 0; 0; 0; 0; 0,0096; 0,5743; 0,7013; 0,1704; 0; 0; 0; 0; 0; 0; 0; 0; 0; 0,0411; 
0,6762; 0,7346; 0,5057; 0; 0; 0; 0; 0; 0; 0; 0; 0; 0,158; 0,8035; 0,8754; 0,5568; 0; 0; 0; 0; 0; 0; 0; 0; 
0; 0,6035; 0,9589; 0,9157; 0; 0; 0; 0; 0; 0; 0; 0; 0; 0; 0,6483; 1; 0,7662; 0; 0; 0; 0; 0; 0; 0; 0; 0; 0; 
0,529; 0,8348; 0,1819; 0; 0; 0; 0; 0; 0; 0; 0; 0; 0; 0; 0,1073; 0; 0; 0; 0; 0; 0; 0; 0; 0; 0; 0; 0; 0,0005; 
0; 0; 0; 0; 0; 0; 0; 0; 0; 0; 0; 0; 0\}; 

  \vspace*{3pt}
  
\noindent
  $\boldsymbol{g}_{\mathrm{XF}}$\;=\;\{0; 0; 0; 0,0008; 0,0083; 0,034; 0,1451; 0,4798; 
0,2442; 0,0257; 0,0034; 0,0001; 0\}; 
  
    \vspace*{3pt}
    
\noindent
  $\boldsymbol{g}_{\mathrm{YF}}$\;=\;\{0; 0; 0,0001; 0,0015; 0,0151; 0,099; 0,3114; 
0,3862; 0,0724; 0,006; 0,0002; 0\}. 

  \vspace*{3pt}
  
  В векторе $\boldsymbol{g}_{0\mathrm{F}}$ содержатся~35~подвергшихся 
обнулению элементов, относящихся к~множеству~$\mathbf{M}_0$. Наличие в~нем 
прочих нулевых элементов есть результат округления; они вносят свой вклад 
в~общий доход комбинированного портфеля, хотя и~с незначительным весом. 
  
\columnbreak
  
    { \begin{center}  %fig2
 \vspace*{-2pt}
    \mbox{%
 \epsfxsize=73.463mm 
 \epsfbox{aga-2.eps}
 }


\vspace*{6pt}

\noindent
{{\figurename~2}\ \ \small{
Доходы идеалистичного портфеля
}}
\end{center}}

 \vspace*{12pt} 
 
 
   Запись результатов для комбинированного портфеля: 
       $$
  \boldsymbol{L}^{\mathrm{cmb}}=\langle 0{,}335345; 0{,}351642; 0{,}048599\rangle\,, 
  $$
а график его платежной функции изображен на рис.~2 в~идеалистичной версии.
  

  С помощью матрицы~$\mathbf{A}$ находится и~вектор весовых 
коэффициентов для \textit{суррогатного} портфеля~(\ref{e18-ags}) и~его запись 
результатов: 

\vspace*{3pt}
  
  \noindent
  $g^{\mathrm{srg}}$\;=\;\{0; 0; 0,0001; 0,0011; 0,0108; 0,0726; 0,2446; 0,3317; 0,0455; 0,0035; 
0,0001; 0; 0; 0,0006; 0,0007; 0,0015; 0,0113; 0,0914; 0,247; 0,3342; 0,0724; 0,0036; 0,0001; 0; 0; 
0,0011; 0,0162; 0,0107; 0,0126; 0,076; 0,253; 0,3402; 0,0473; 0,0039; 0,0001; 0,0000, 0,0007; 
0,001; 0,0363; 0,6184; 0,0454; 0,082; 0,2632; 0,3505; 0,0503; 0,0045; 0,0008; 0,0008; 0,0071; 
0,0076; 0,0096; 0,5743; 0,7013; 0,1704; 0,2781; 0,3653; 0,0547; 0,007; 0,0082; 0,0083; 0,0259; 
0,0268; 0,0292; 0,0411; 0,6762; 0,7346; 0,5057; 0,3293; 0,0606; 0,0321; 0,0336; 0,034; 0,1095; 
0,1092; 0,1142; 0,1223; 0,158; 0,8035; 0,8754; 0,5568; 0,1328; 0,1401; 0,144; 0,1451; 0,3867; 
0,3902; 0,3983; 0,412; 0,4316; 0,6035; 0,9589; 0,9157; 0,4537; 0,4691; 0,4775; 0,4798; 0,2442; 
0,1839; 0,1886; 0,1966; 0,2081; 0,223; 0,6483; 1; 0,7662; 0,235; 0,2418; 0,2438; 0,022; 0,0167; 
0,0178; 0,0197; 0,025; 0,0892; 0,2935; 0,529; 0,8348; 0,1819; 0,0219; 0,0257; 0,0015; 0,0016; 
0,0018; 0,0022; 0,0141; 0,0964; 0,3114; 0,3788; 0,0668; 0,1073; 0,0031; 0,0034; 0; 0; 0,0001; 
0,0013; 0,0149; 0,099; 0,2991; 0,3862; 0,0704; 0,0056; 0,0005; 0,0001; 0; 0; 0,0001; 0,0013; 
0,0151; 0,0898; 0,3006; 0,3313; 0,0715; 0,006; 0,0002; 0\};

\vspace*{3pt}
  
  \noindent
  $\boldsymbol{L}^{\mathrm{srg}} = \langle 0{,}324923; 0{,}340849; 0{,}0490131\rangle$. 
  
  \vspace*{3pt}
  
  Здесь не приводится график доходов \textit{суррогатного} портфеля, так как 
он весьма похож на график рис.~2 (до степени смешения), хотя и~не тождествен 
ему. Расхождения объясняются тем, что в~суррогатном портфеле (в~отличие от 
идеалистичного) все двумерные сценарии рассматриваются по отдельности. 
И~даже при совпадении относительных доходов на нескольких сценариях (что 
часто происходит после замещений) алгоритм придает им разные веса, в~то 
время как в~комбинированном портфеле таким сценариям приписывается 
единый вес. Близость графиков служит подтверждением корректности 
расчетов. 
  
  В задаче~$B$ при $\chi_{\mathrm{X}}\hm= 1{,}02^{-1}
  \hm\approx 0{,}980392$ и~$\chi_{\mathrm{Y}}\hm = 1$:
  
  \vspace*{3pt}
  
 \noindent
  $\boldsymbol{\theta}_{\mathrm{X}}$\;=\;\{0; 0,0062; 0,0434; 0,1054; 0,1385; 0,3047; 
0,2934; 0,295; 0,3108; 0,4238; 0,4302; 0,3053; 0,4165\}. 

  \vspace*{3pt}
  
  В задаче~$C$ при $\chi_{\mathrm{X}}\hm = 1$ и~$\chi_{\mathrm{Y}}\hm = 1$:
  
    \vspace*{3pt}
  
  \noindent
  $\boldsymbol{\theta}_{\mathrm{X}}$\;=\;\{0; 0,0199; 0,0616; 0,1623; 0,3246; 0,4458; 
0,6904; 1; 0,6843; 0,4372; 0,2489; 0,0788; 0,0012\};

  \vspace*{3pt}
  
  \noindent
  $\boldsymbol{\theta}_{\mathrm{Y}}$\;=\;\{0,0273; 0,0715; 0,1536; 0,2337; 0,3867; 0,5895; 
0,8467; 0,8484; 0,5965; 0,3364; 0,2241; 0,1241\}. 
  
  \vspace*{3pt}
    
  Для задач~$B$ и~$C$ матрицы замещений соответственно равны: 
  $$
  \left\|
  \begin{array}{cccccccccccc}
0&0&0&0&0&0&0&0&0&0&0&0\\
0&0&0&0&0&0&0&0&0&0&0&1\\
1&0&0&0&0&0&0&0&0&0&1&1\\
1&0&0&0&0&0&0&0&0&1&1&1\\
1&1&0&0&0&0&0&0&0&1&1&1\\
1&1&1&0&0&0&0&0&1&1&1&1\\
1&1&1&1&0&0&0&0&0&1&1&1\\
1&1&1&1&0&0&0&0&0&1&1&1\\
1&1&1&1&1&0&0&0&0&0&1&1\\
1&1&1&1&1&1&0&0&0&0&1&1\\
1&1&1&1&1&1&0&0&0&0&1&1\\
1&1&1&1&1&1&0&0&0&0&0&1\\
1&1&1&1&1&1&1&0&0&0&0&1
\end{array}
\right\|; \ 
%$$
%$$
\left\|
\begin{array}{cccccccccccc}
2&2&2&2&2&2&2&2&2&2&2&2\\
1&2&2&2&2&2&2&2&2&2&2&1\\
1&1&2&2&2&2&2&2&2&2&2&1\\
1&1&1&2&2&2&2&2&2&2&1&1\\
1&1&1&1&2&2&2&2&2&1&1&1\\
1&1&1&1&1&2&2&2&2&1&1&1\\
1&1&1&1&1&1&2&2&1&1&1&1\\
1&1&1&1&1&1&1&1&1&1&1&1\\
1&1&1&1&1&1&2&2&1&1&1&1\\
1&1&1&1&1&2&2&2&2&1&1&1\\
1&1&1&1&2&2&2&2&2&2&1&1\\
1&1&2&2&2&2&2&2&2&2&2&1\\
1&2&2&2&2&2&2&2&2&2&2&2
\end{array}
\right\|.
$$
  
  Выбором в~задаче~$B$ параметра $\chi_{\mathrm{X}} (= 1{,}02^{-1})$ 
подчеркивается необязательность равенства его единице. К~тому же при 
получении графической картинки играют свою роль также соображения 
эстетического характера. 
  
  Записи результатов $\boldsymbol{L}^{\mathrm{cmb}}$ для задач~$B$ и~$C$ 
соответственно: 
 \begin{gather*}
  \langle 0{,}329105; 0{,}343464; 0{,}0436319\rangle\,;\\
  \langle 
0{,}374383; 0{,}38801; 0{,}0363967\rangle\,. 
  \end{gather*}
  
  Графики платежных функций комбинированных портфелей для них 
изображены на рис.~3 в~идеалистичной версии.

  \vspace*{-6pt}
  
  \section{Заключение}
  
  \vspace*{-2pt}
  
  Работа завершает намеченное исследование\linebreak применимости CC-VaR 
к~совокупности трех рынков разной размерности. Для теоретических 
и~сценарных рынков предложен подход, связанный\linebreak\vspace*{-12pt}




{ \begin{center}  %fig3
 \vspace*{-1pt}
    \mbox{%
 \epsfxsize=74.293mm 
 \epsfbox{aga-3.eps}
 }

\end{center}

\noindent
{{\figurename~3}\ \ \small{
Доходы идеалистичных портфелей в~задачах~$B$~(\textit{а}) и~$C$~(\textit{б})
}}
}

\vspace*{12pt}


\noindent
 с~замещением менее 
доходных базисных инструментов двумерного рынка более доходными 
ба\-зисными инструментами соответствующих одномерных рынков. Для 
реализуемости замещения применяется методология рандомизации. 
Проведенные для сценарных рынков расчеты и~построенные графики 
свидетельствуют об эффективности модели.

%\vspace*{-6pt}

\vspace*{-6pt} 
  
{\small\frenchspacing
 {%\baselineskip=10.8pt
 \addcontentsline{toc}{section}{References}
 \begin{thebibliography}{9}
  \bibitem{1-ags}
  \Au{Agasandian G.\,A.} Optimal behavior of an investor in option market~// 
  Joint 
Conference (International) on Neural Networks Proceedings.~--- Honolulu, HI,
USA:  IEEE,  2002. P.~1859--1864. 
  \bibitem{2-ags}
  \Au{Агасандян Г.\,А.} Применение континуального критерия VaR на финансовых 
рынках.~--- М.: ВЦ РАН, 2011. 299~с. 
  \bibitem{3-ags}
  \Au{Агасандян Г.\,А.} Континуальный критерий VaR на многомерных рынках  
опционов.~--- М.: ВЦ РАН, 2015. 297~с. 
  \bibitem{4-ags}
  \Au{Агасандян Г.\,А.} Континуальный критерий VaR на сценарных рынках~// 
Информатика и~её применения, 2018. Т.~12. Вып.~1. С.~32--40.

 
  \bibitem{5-ags}
  \Au{Агасандян Г.\,А.} Континуальный критерий VaR и~оптимальный портфель 
инвестора~// Управ\-ле\-ние большими сис\-те\-ма\-ми, 2018. Вып.~73. С.~6--26.

\columnbreak

  \bibitem{6-ags}
  \Au{Агасандян Г.\,А.} Теоретические основы оптимизации по CC-VaR на совокупности 
рынков~// Информатика и~её применения, 2019. Т.~13. Вып.~4. С.~36--41.
  \bibitem{7-ags}
  \Au{Крамер Г.} Математические методы статистики~/ Пер. с~англ.~--- М.: Мир, 1975. 
750~с. (\Au{Cramer~H.} Mathematical methods of statistics.~--- Princeton, NJ, USA: Princeton 
University Press, 1946. 575~p.)
 \end{thebibliography}

 }
 }

\end{multicols}

\vspace*{-9pt}

\hfill{\small\textit{Поступила в~редакцию 21.11.19}}

\vspace*{8pt}

%\pagebreak

%\newpage

%\vspace*{-28pt}

\hrule

\vspace*{2pt}

\hrule

\vspace*{-4pt}

\def\tit{COMPUTATIONAL ASPECTS OF~OPTIMIZATION ON~CC-VaR 
IN~A~COMPLEX OF MARKETS}


\def\titkol{Computational aspects of~optimization on CC-VaR 
in~a~complex of markets}

\def\aut{G.\,A.~Agasandyan}

\def\autkol{G.\,A.~Agasandyan}

\titel{\tit}{\aut}{\autkol}{\titkol}

\vspace*{-15pt}


\noindent
A.\,A.~Dorodnicyn Computing Center, Federal Research Center ``Computer Science 
and Control'' of the Russian Academy of Sciences, 40~Vavilov Str., Moscow 119333, 
Russian Federation


\def\leftfootline{\small{\textbf{\thepage}
\hfill INFORMATIKA I EE PRIMENENIYA~--- INFORMATICS AND
APPLICATIONS\ \ \ 2020\ \ \ volume~14\ \ \ issue\ 3}
}%
 \def\rightfootline{\small{INFORMATIKA I EE PRIMENENIYA~---
INFORMATICS AND APPLICATIONS\ \ \ 2020\ \ \ volume~14\ \ \ issue\ 3
\hfill \textbf{\thepage}}}

\vspace*{3pt}

\Abste{The work is the direct continuation of the previous author's 
investigation on using continuous VaR-criterion (CC-VaR) in a~set 
of markets of different dimensions, which are mutually connected by 
their underliers. The exposition is aimed at the application of ideas 
and methods developed for the theoretical continuous model to discrete 
scenarios markets. In a~typical model case of a~collection of one 
two-dimensional market and two one-dimensional markets, a~rule of 
constructing a~combined portfolio in these markets is submitted. 
This rule gives a~necessary and sufficient condition of portfolio 
optimality in the weighted composition of basis instruments. The 
condition is founded on misbalance in returns relative between 
markets with maintaining optimality on CC-VaR. The optimal combined 
portfolio with three components is constructed. Also, the idealistic 
and surrogate versions of this combined portfolio, which are useful 
in testing all algorithmic calculations and in graphic illustrating 
portfolio's payoff functions, are adduced. The model can be extended 
without difficulties, theoretic anyway, on markets of greater dimensions.}

\KWE{underlie; risk preferences function; continuous VaR-criterion; 
cost and forecast densities; return relative function; Newman--Pearson 
procedure; combined portfolio; surrogate portfolio}
 
\DOI{10.14357/19922264200309} 

\vspace*{-20pt}

 \Ack
 \noindent
  The reported study was funded by RFBR, project 
No.\,17-01-00816.

%\vspace*{6pt}

 \begin{multicols}{2}

\renewcommand{\bibname}{\protect\rmfamily References}
%\renewcommand{\bibname}{\large\protect\rm References}

{\small\frenchspacing
 {%\baselineskip=10.8pt
 \addcontentsline{toc}{section}{References}
 \begin{thebibliography}{9}
  \bibitem{1-ags-1}
  \Aue{Agasandian, G.\,A.} 2002. Optimal behavior of an investor in option market. 
\textit{Joint Conference (International) on Neural Networks Proceedings}. 
Honolulu, HI: IEEE. 
1859--1864. 
  \bibitem{2-ags-1}
  \Aue{Agasandyan, G.\,A.} 2011. \textit{Primenenie kontinual'nogo kriteriya VaR 
na finansovykh rynkakh} [Application of continuous VaR-criterion in financial 
markets]. Moscow: CC RAS. 299~p. 
  \bibitem{3-ags-1}
  \Aue{Agasandyan, G.\,A.} 2015. \textit{Kontinual'nyy kriteriy VaR na 
mnogomernykh rynkakh optsionov} [Continuous VaR-criterion in multidimensional 
option markets]. Moscow: CC RAS. 297~p. 
  \bibitem{4-ags-1}
  \Aue{Agasandyan, G.\,A.} 2018. Kontinual'nyy kriteriy VaR na stsenarnykh 
rynkakh [Continuous VaR-criterion in scenario markets]. \textit{Informatika i~ee 
Primeneniya~--- Inform. Appl.} 12(1):32--40. 
  \bibitem{5-ags-1}
  \Aue{Agasandyan, G.\,A.} 2018. Kontinual'nyy kriteriy VaR i~optimal'nyy portfel' 
investora [Continuous VaR-criterion and investor's optimal portfolio]. 
\textit{Upravleniye bol'shimi sistemami} [Large-Scale Systems Control] 73:6--26.
  \bibitem{6-ags-1}
  \Aue{Agasandyan, G.\,A.} 2018. Teoreticheskie osnovy opti\-mi\-za\-tsii po 
kontinual'nomu kriteriyu VaR na sovokupnosti rynkov [Theoretical foundations of 
continuous VaR optimization in the collection of markets]. \textit{Informatika i~ee 
Primeneniya~--- Inform. Appl.} 13(4):36--41. 
  \bibitem{7-ags-1}
  \Aue{Cramer, H.} 1946. \textit{Mathematical methods of statistics}. Princeton, NJ: 
Princeton University Press. 575~p.
\end{thebibliography}

 }
 }

\end{multicols}

\vspace*{-9pt}

\hfill{\small\textit{Received October 21, 2019}}

%\pagebreak

\vspace*{-20pt}
  
  \Contrl
  
  \vspace*{-4pt}
  
  \noindent
  \textbf{Agasandyan Gennady A.} (b.\ 1941)~--- Doctor of Science in physics and 
mathematics, leading scientist, A.\,A.~Dorodnicyn Computing Center, Federal 
Research Center ``Computer Science and Control'' of the Russian Academy of 
Sciences, 40~Vavilov Str., Moscow 119333, Russian Federation; 
\mbox{agasand17@yandex.ru}
  

   
\label{end\stat}

\renewcommand{\bibname}{\protect\rm Литература} 
   