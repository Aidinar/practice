\def\stat{grusho}

\def\tit{МЕТОДЫ МАТЕМАТИЧЕСКОЙ СТАТИСТИКИ\\ В~ЗАДАЧЕ ПОИСКА ИНСАЙДЕРА$^*$}

\def\titkol{Методы математической статистики в~задаче поиска инсайдера}

\def\aut{А.\,А.~Грушо$^1$, М.\,И.~Забежайло$^2$, Д.\,В.~Смирнов$^3$, 
Е.\,Е.~Тимонина$^4$, С.\,Я.~Шоргин$^5$}

\def\autkol{А.\,А.~Грушо, М.\,И.~Забежайло, Д.\,В.~Смирнов и~др.}

\titel{\tit}{\aut}{\autkol}{\titkol}

\index{Грушо А.\,А.}
\index{Забежайло М.\,И.}
\index{Смирнов Д.\,В.}
\index{Тимонина Е.\,Е.}
\index{Шоргин С.\,Я.}
\index{Grusho N.\,A.}
\index{Zabezhailo M.\,I.}
\index{Smirnov D.\,V.}
\index{Timonina E.\,E.}
\index{Shorgin S.\,Ya.}
 

{\renewcommand{\thefootnote}{\fnsymbol{footnote}} \footnotetext[1]
{Работа частично поддержана РФФИ (проект 18-29-03081).}}


\renewcommand{\thefootnote}{\arabic{footnote}}
\footnotetext[1]{Институт проблем информатики Федерального исследовательского центра <<Информатика и~управ\-ле\-ние>> 
Российской академии наук, \mbox{grusho@yandex.ru}}
\footnotetext[2]{Вычислительный центр им.\ А.\,А.~Дородницына Федерального исследовательского центра <<Информатика 
и~управ\-ле\-ние>> Российской академии наук, \mbox{m.zabezhailo@yandex.ru}}
\footnotetext[3]{ПАО Сбербанк России, Департамент кибербезопасности,
\mbox{dvlsmirnov@sberbank.ru}}
\footnotetext[4]{Институт проблем информатики Федерального исследовательского центра <<Информатика и~управ\-ле\-ние>> 
Российской академии наук, \mbox{eltimon@yandex.ru}}
\footnotetext[5]{Институт проблем информатики Федерального исследовательского центра <<Информатика и~управ\-ле\-ние>> 
Российской академии наук, \mbox{sshorgin@ipiran.ru}}

\vspace*{-9pt}

  
  
  \Abst{Исследованы подходы к~выявлению враждебных инсайдеров организации, 
использующих сговор. Проблема выявления организованной группы нарушителей 
информационной безопас\-ности~--- одна из самых сложных задач обеспечения 
безопас\-ности  организации. 
  Исходное множество данных для анализа состоит из множества малых выборок, 
описывающих функционал информационных технологий 
(ИТ) организации. Это множество можно 
считать большими данными. Для сокращения объема исходных данных использован метод 
кластеризации.  Это позволило эффективно использовать методы математической статистики, 
т.\,е.\ выявить малые выборки, несущие информацию о враждебных инсайдерах. Сложность 
задачи заключалась в~том, чтобы как можно меньше потерять искомых малых выборок. 
Найдены условия, когда в~схеме серий вероятность выявления инсайдеров, использующих 
сговор, стремится к~1.  }
  
  \KW{выявление организованной группы враждебных инсайдеров; малые выборки; большие 
данные; математическая статистика}

\DOI{10.14357/19922264200310} 
 
\vspace*{-5pt}


\vskip 10pt plus 9pt minus 6pt

\thispagestyle{headings}

\begin{multicols}{2}

\label{st\stat}
  
\section{Введение }

  Сбор ценной информации сотрудниками, использующими личные связи,~--- 
серьезная проблема информационной безопасности в~государственных 
и~коммерческих организациях. При этом часто нарушитель информационной 
безопасности сам имеет доступ к~части ценной информации
(ЦИ), но не имеет права 
получать ее целиком. Примером такой ситуации может служить разделение общей 
проблемы для ее решения в~разных подразделениях организации. Тогда 
враждебный инсайдер (далее~--- инсайдер) попытается узнать недостающую 
ЦИ от своих друзей в~других подразделениях или войти в~сговор 
о~сборе и~продаже~ЦИ. 
  
  Исследованию проблемы выявления инсайдеров посвящено множество 
научных работ. 
  
  В работе~[1] приведен ряд вызовов, исследование которых авторы данного 
обзора считают важными, но трудными проблемами. Вызов~4 (см.~[1]) связан 
с~проб\-ле\-мой выявления инсайдеров, организованных с~помощью преступного 
сговора. 
  
  Выявление инсайдеров в~течение ряда лет финансировалось американским 
управлением исследованиями Министерства обороны США (DARPA, Defence Advanced Research
Projects Agency). Основное 
направление этих исследований связано с~выявлением аномалий в~больших 
данных. В~част\-ности, в~работе~[2] разработан язык описания\linebreak аномалий 
и~представлена информация по его использованию. 
  
  В работах~[3, 4] для выявления слабо выраженных аномалий, порожденных 
инсайдерами, предложено использование информации из нескольких 
информационных пространств. 
  
  Важной проблемой остается определение условий, при которых выявление 
инсайдера возможно. Все методы в~данной области строятся на основе некоторых 
предположений, выполнение которых необходимо для их работоспособности.  
Далее рассматривается проблема выявления инсайдера в~банковской сфере.
  
  В статье предполагается, что цели инсайдера состоят в~сборе и~продаже ЦИ
   о~клиентах банка. Как правило, такой информацией выступают 
персональные данные вместе с~ЦИ о~счетах и~движении 
денежных средств. Для того чтобы эти данные могли дать постоянный доход 
инсайдеру, они должны собираться в~достаточно большом объеме. 
{\looseness=1

}
  
  Для защиты от таких инсайдеров можно разделить информацию, например на 
персональные данные и~ЦИ на счетах. При этом работать 
с~каж\-дым из этих блоков данных могут только разные менеджеры (сотрудники 
банка). Отсюда возникает задача выявления возможного сговора ка\-ких-то 
менеджеров, занимающихся либо персональными данными, либо информацией 
о~счетах.  Статья посвящена анализу условий выявления такого сго\-вора.
  
  \section{Формальная модель сговора инсайдеров}
  
  Для исследования задачи выявления инсайдеров, использующих сговор, 
рассмотрим сле\-ду\-ющую модель. Обозначим через~$V$ множество\linebreak клиентов, 
использующих сервисы банка, через~$U$~--- множество менеджеров 
(пользователей банковской сис\-те\-мы и~реализующих ИТ). Хранилище данных, 
которое используют менеджеры, устроено сле\-ду\-ющим 
образом. Все данные заносятся в~прямоугольную таб\-ли\-цу, столб\-цы которой 
нумеруются атрибутами $A_0, A_1,\ldots$\,, а~строки содержат записи данных 
и~результатов действий с~ними. Строки не изменяются, но при изменениях 
данных появляются новые строки с~новыми данными. В~отличие от 
традиционных реляционных баз данных записи не обязательно содержат 
значения всех атрибутов и~поиск по таб\-ли\-це осуществляется по сложному 
ключу, зависящему от типа данных в~строке. Однако каждая строка содержит 
данные о~ме\-нед\-же\-ре, осуществляющем обращение, идентификаторе экземпляра 
ИТ, которая связана с~обращением, и~санкционированием текущей транзакции от 
предыдущих шагов ИТ. Часть атрибутов~$\vec{A}_1$ соответствует содержанию 
персональных данных, нужных для авторизации транзакций, часть 
атрибутов~$\vec{A}_2$ соответствует ЦИ (данные счетов, 
переводы и~др.).  
  
  Политика безопасности запрещает одному менеджеру иметь доступ 
одновременно к~$\vec{A}_1$ и~к~$\vec{A}_2$. Поэтому строки, 
соответствующие использованию атрибутов~$\vec{A}_2$, обладают индексами, 
скрывающими данные, соответствующие атрибутам~$\vec{A}_1$. Кроме того, 
выделены пользователи~$U_1$, которые могут иметь доступ к~информации 
с~атрибутами~$\vec{A}_1$, и~пользователи~$U_2$, не имеющие доступа 
к~данным с~атрибутами~$\vec{A}_1$, но работающие с~данными, 
соответствующими атрибутам~$\vec{A}_2$.
  
  Интересная для инсайдеров ЦИ может быть\linebreak получена тогда и~только тогда, 
когда известны значения атрибутов~$\vec{A}_1$ и~соответствующие им значения 
атрибутов~$\vec{A}_2$. Для получения такой информации необходим сговор 
какого-либо пользователя из множества~$U_1$ с~каким-либо пользователем из 
множества~$U_2$. Задача состоит в~выявлении таких инсайдеров. 
  
  Пусть $u_1\hm\in U_1$  и~$u_2\hm\in U_2$ образуют такую пару $(u_1, u_2)$ 
инсайдеров, использующих сговор. 
  
  Сделаем дополнительные предположения о том, что пара  $(u_1, u_2)$ может 
так управлять потоком заявок клиентов на сервисы, что пара $(u_1, u_2)$ 
появляется чаще, чем все остальные пары из множества $U_1\times U_2$. Тогда 
можно пытаться использовать методы математической статистики для выявления 
пары $(u_1, u_2)$. 
  
  Однако для большого значения числа элементов в~множестве $U_1\times U_2$ 
выявление пары $(u_1, u_2)$ упирается в~проблему многих малых выборок~[5, 6]. 
  
  Пусть статистические критерии выявления $(u_1, u_2)$ имеют ошибки 
$\alpha\hm>0$, $\beta\hm>0$ где~$\alpha$~---  вероятность <<ложной>> тревоги; 
$\beta$~--- вероятность пропуска $(u_1, u_2)$.  Тогда при многих малых выборках 
анализ <<ложных>> тревог становится трудоемкой задачей, а вероятность 
выявления пары $(u_1, u_2)$ ограничена снизу константой (приблизительно~1/3). 
Поэтому для эффективного применения статистических методов необходимо 
снижать объем множества малых выборок. Иными словами, необходимо 
применять статистические методы в~рамках некоторых кластеров ограниченного 
набора малых выборок. 
  
  В~данной статье используется параметр, управ\-ля\-ющий объемом множества 
малых выборок и~поз\-во\-ля\-ющий повышать эффективность статистического 
анализа. Этим параметром служит объем множества клиентов, в~интересах 
которых проводятся транзакции, интересные для $(u_1, u_2)$, т.\,е.\ вмес\-те 
с~каждой парой $(u, u^\prime)$, $u\hm\in U_1$, $u^\prime\hm\in U_2$, необходимо 
рассматривать параметр~$v$ со значениями в~$V$, идентифицирующий клиента, 
который инициирует транзакцию. Обозначим эту тройку $(u,u^\prime)\vert v$. 
  
  Инсайдеров $(u_1, u_2)$ интересуют клиенты из множества~$V$, продажа 
информации о которых имеет высокую стоимость. Вместе с~тем если инсайдеров 
интересуют только такие клиенты (обозначим множество таких клиентов 
через~$V_1$) и~их транзакции, то количество троек $(u, u^\prime)\vert v$, где 
$v\hm\in V_1$, для анализа сокращается, что позволяет повысить эффективность 
статистических методов.

  \vspace*{-4pt}
  
  \section{Анализ выявляемости сговора инсайдеров}
  
  \vspace*{-3pt}
  
  Пусть транзакция обрабатывается случайной парой менеджеров $(u, u^\prime)$, 
$u\hm\in U_1$, $u^\prime\hm\in U_2$. Положим $\vert U_1\vert \hm= n_1$, $\vert 
U_2 \vert \hm= n_2$, $\vert U_1\times U_2 \vert \hm= n_1  n_2 \hm= n$. Тогда 
вероятность появления пары $(u, u^\prime)$ равна~$1/n$. 
  
  Если $(u_1, u_2)$~--- инсайдеры, использующие сговор, и~клиент~$v$ 
представляет для них интерес, то вероятность того, что эта пара будет 
обслуживать этого клиента равна 
  $$
  {\sf P}\left(\left(u_1,u_2\right)\vert v\right)=p\,,
  $$
  а в~случае любой другой пары $(u, u^\prime) \not= (u_1, u_2)$ эта вероятность 
равна
  $$
  {\sf P}\left( \left( u, u^\prime\right) v\right)= (1-p)\fr{1}{n-1}\,.
  $$
  
  С помощью методов кластеризации можно выделить множество~$V_1$ тех 
клиентов, которые представляют интерес для инсайдеров, использующих сговор. 
Ясно, что мощность множества~$V_1$ много меньше, чем мощность 
множества~$V$. Как было отмечено выше, это позволяет избежать эффекта малых 
выборок~\cite{6-gr}. 
  
  Пусть $C_1$~--- среднее число транзакций у клиентов из множества~$V_1$. 
Тогда средний объем данных для клиентов из множества~$V_1$ равен $\vert 
V_1\vert  C_1\hm=m$. Далее считаем~$m$ известным параметром схемы. 
Тогда соотношение параметров~$m$, $n$ и~$p$ определяет возможность выявления 
инсайдеров, использующих сговор. Инсайдеры $(u_1, u_2)$ не могут переключить 
на себя весь поток клиентов из множества~$V_1$. Поэтому для данных~$\xi$, 
которые они получили, можно предложить модель биномиального распределения: 
  $$
 {\sf P}\left( \xi\left ( u_1, u_2\right)=k\right) =
  \begin{pmatrix}
  m\\ k
  \end{pmatrix} p^k (1-p)^{m-k}\,.
  $$
  
  Тогда для любой пары $(u, u^\prime) \not= (u_1, u_2)$ распределение числа 
случаев предоставления сервисов описывается следующим образом:
 \begin{multline*}
{\sf P}\left(\xi\left( u,u^\prime\right)=r\right) =\sum\limits^m_{k=0} \begin{pmatrix}
  m\\ k\end{pmatrix} p^k (1-p)^{m-k}\times{}\\
  {}\times \begin{pmatrix}
  m-k\\ r\end{pmatrix}
  \left(\fr{1}{n-1}\right)^r \left( 1-\fr{1}{n-1}\right)^{m-k-r}\,.
  \end{multline*}
  
  Поскольку числа~$m$ и~$n$ являются большими, то рассмотрение задачи 
выявляемости инсайдеров $(u_1, u_2)$ в~множестве~$V_1$ будет вестись 
в~терминах асимптотических распределений вероятностей в~схеме серий, т.\,е.\ 
в~предположении, что $m\hm\to \infty$, $n\hm\to \infty$ и~$p\hm\to 0$. Задача 
состоит в~поиске значений вероятности~$p$, при которых вероятность выявления 
инсайдеров $(u_1, u_2)$ стремится к~1. 
  
  В практическом плане интересен случай, когда $m\hm\to \infty$, $n\hm\to \infty$ 
и~$\alpha\hm= m/n$ удовлетворяет условию $\alpha/\ln(n) \hm\to 0$. Это условие 
соответствует преобладанию числа предоставления сервисов для клиентов из 
множества~$V_1$ над числом пар менеджеров $n_1 n_2\hm=n$, которые 
проводят это обслуживание. В~каком-то смысле множитель~$\ln n$ можно 
интерпретировать как число обслуживаний на одну пару менеджеров, не 
являющихся инсайдерами $(u_1, u_2)$, использующими сговор. 
  
  Рассмотрим случайную величину~$\eta_{(n)}$, равную максимальному числу 
обслуживаний клиентов из множества~$V_1$  по всем парам менеджеров $(u, 
u^\prime)$. Обозначим 
  $$
  \alpha=\fr{m}{n}\,;\quad p_k=\fr{\alpha^k}{k!}\,,\enskip k=0,1,\ldots
  $$
  
  В~работе~\cite{7-gr} приведена следующая теорема.
  
  \smallskip
  
  \noindent
  \textbf{Теорема}~\cite{7-gr}. \textit{Если $m, n \hm\to \infty$, $\alpha/ \ln(n)\hm\to 0$ и~$r\hm= 
r(\alpha, n)$ выбрано так, что $r\hm> \alpha$ и~$np_r\hm\to \lambda$, где~$\lambda$~--- 
положительная постоянная, то} 
  $$
  {\sf P}\left\{ \eta_{(n)} = r-1\right\}\to e^{-\lambda}\,;\enskip 
  {\sf P}\left\{ \eta_{(n)}=r\right\} \to 1 -  
e^{-\lambda}\,.
  $$
  
  Рассмотрим простой пример применения этой теоремы. Пусть
  $$
  n\fr{\alpha^3}{6}\,e^{-\alpha} \to \lambda >0\,,
  $$
т.\,е.\ $r\hm= r(\alpha, n)\hm=3$. Тогда максимальное число обслуживаний 
клиентов из множества~$V_1$  по всем парам менеджеров $(u, u^\prime)\not= 
(u_1, u_2)$ равно~3 с~вероятностью, стремящейся к~1. 

  Превышение числа появлений пары $(u_1, u_2)$, начиная с~4, однозначно 
идентифицирует эту пару инсайдеров с~вероятностью, стремящейся к~1. 
Вероятность того, что пара $(u_1, u_2)$ в~определенной выше схеме появится не 
больше трех раз, равна
  \begin{multline*}
  (1-p)^m+mp(1-p)^{m-1}+\fr{m(m-1)}{2}\,p^2 (1-p)^{m-2}+{}\\
  {}+ \fr{m(m-1)(m-2)}{6}\,p^3(1-p)^{m-3}\,.
  \end{multline*}
  
  Отсюда следует, что при  $p\hm\leq \ln (m)/m$ пара инсайдеров $(u_1, u_2)$, 
использующих сговор, выявляется с~вероятностью, стремящейся к~1. 
  
  \section{Заключение }
  
  В статье показано, как сочетать методы клас\-те\-ри\-за\-ции и~методы 
математической статистики для выявления инсайдеров, использующих сговор, 
а~именно: для эффективного применения математической статистики на 
множестве разнородных данных, которые состоят из множества малых выборок, 
необходимо приближенное выполнение соотношения между параметрами схемы. 
  
  Однако согласование параметров возможно, когда существует возможность 
управления объемами данных. Для такого управления можно использовать методы 
кластеризации.
  
  В рассмотренной задаче можно выделить сис\-те\-му вложенных кластеров 
данных, повышающих\linebreak шансы выявления инсайдеров, использующих сговор. 
Выделение таких кластеров позволило доказать пренебрежимо малую 
вероятность превышения построенного порога для числа случайных\linebreak повторений 
встречаемости пар менеджеров, не являющихся инсайдерами, использующими 
сговор. 
  
  Вместе с~тем при полученной отсюда оценке вероятности встречаемости 
инсайдеров, использующих сговор, их идентификация происходит 
с~вероятностью, стремящейся к~1. 
  
{\small\frenchspacing
 {%\baselineskip=10.8pt
 \addcontentsline{toc}{section}{References}
 \begin{thebibliography}{9}
\bibitem{1-gr}
\Au{Gheyas I., Abdallah~A.} Detection and prediction of insider threats to cyber security: 
A~systematic literature review and meta-analysis~// Big Data Anal., 2016. 
Vol.~1. P.~1--29. doi:  10.1186/s41044-016-0006-0.
\bibitem{2-gr}
\Au{Memory A., Goldberg H.\,G., Senator~T.\,E.} 
Context-aware insider threat detection~// 
Workshops at 27th AAAI Conference on Artificial Intelligence, 2013. 
P.~44--47. {\sf 
https:// pdfs.semanticscholar.org/04aa/e6d97900ba62e90b07ac6\linebreak 82fb7bd8c2e1029.pdf}.


\bibitem{4-gr} %3
\Au{Грушо А.\,А., Забежайло~М.\,И., Смирнов~Д.\,В., Тимонина~Е.\,Е.} Модель множества 
информационных пространств в~задаче поиска инсайдера~// Информатика и~её применения, 
2017. Т.~11. Вып.~4. С.~65--69.

\bibitem{3-gr} %4
\Au{Grusho A., Grusho~N., Timonina~E.} Method of several information spaces for identification of 
anomalies~// Intelligent distributed computing~XIII~/ Eds. I.~Kotenko, C.~Badica, V.~Desnitsky, 
D.~El Baz, M.~Ivanovic.~--- Studies in computational intelligence ser.~---   
Springer, 2020.  Vol.~868. P.~515--520. doi: 10.1007/978-3-030-32258-8\_60.

\bibitem{5-gr}
\Au{Axelsson S.} The base-rate fallacy and 
the difficulty of intrusion detection~// ACM~T. 
Inform. Syst. Se., 2000. Vol.~3. Iss.~3. P.~186--205.
\bibitem{6-gr}
\Au{Grusho A., Grusho~N., Timonina~E.} The bans in finite probability spaces and the problem of 
small samples~// Distributed computer and communication networks~/ Eds. V.~Vishnevskiy, 
K.~Samouylov, D.~Kozyrev.~--- Lecture notes in computer science ser.~--- Springer, 2019.  
Vol.~11965. P.~578--590.
\bibitem{7-gr}
\Au{Колчин В.\,Ф., Севастьянов~Б.\,А., Чистяков~В.\,П.} Случайные размещения.~--- М.: Наука, 
1976. 224~с.
\end{thebibliography}

 }
 }

\end{multicols}

\vspace*{-6pt}

\hfill{\small\textit{Поступила в~редакцию 02.06.20}}

\vspace*{8pt}

%\pagebreak

%\newpage

%\vspace*{-28pt}

\hrule

\vspace*{2pt}

\hrule

%\vspace*{-2pt}

\def\tit{MATHEMATICAL STATISTICS IN~THE~TASK OF~IDENTIFYING HOSTILE INSIDERS}


\def\titkol{Mathematical statistics in~the~task of~identifying hostile insiders}

\def\aut{N.\,A.~Grusho$^1$, M.\,I.~Zabezhailo$^2$, D.\,V.~Smirnov$^3$, E.\,E.~Timonina$^1$, 
and~S.\,Ya.~Shorgin$^1$}

\def\autkol{N.\,A.~Grusho, M.\,I.~Zabezhailo, D.\,V.~Smirnov, et al.}

\titel{\tit}{\aut}{\autkol}{\titkol}

\vspace*{-9pt}


\noindent
$^1$Institute of Informatics Problems, Federal Research Center ``Computer Sciences and Control'' of 
the Russian\linebreak
$\hphantom{^1}$Academy of Sciences, 44-2~Vavilov Str., Moscow 119133, Russian Federation

\noindent
$^2$A.\,A.~Dorodnicyn Computing Center, Federal Research Center ``Computer Science and Control'' 
of the Russian\linebreak
$\hphantom{^1}$Academy of Sciences, 40~Vavilov Str., Moscow 119333, Russian Federation

\noindent
$^3$Sberbank of Russia, 19~Vavilov Str., Moscow 117999, Russian Federation

\def\leftfootline{\small{\textbf{\thepage}
\hfill INFORMATIKA I EE PRIMENENIYA~--- INFORMATICS AND
APPLICATIONS\ \ \ 2020\ \ \ volume~14\ \ \ issue\ 3}
}%
 \def\rightfootline{\small{INFORMATIKA I EE PRIMENENIYA~---
INFORMATICS AND APPLICATIONS\ \ \ 2020\ \ \ volume~14\ \ \ issue\ 3
\hfill \textbf{\thepage}}}

\vspace*{3pt} 


\Abste{The paper explores approaches to identifying hostile insiders
 of the organization using collusion. The problem of identifying the 
 organized group of information security violators is one of the most 
 complex tasks of ensuring the security of organization. The set 
 of source data for analysis consists of many small samples describing 
 the functionality of the organization's information technologies. 
 This set can be considered as big data. The clustering method is 
 used to reduce the amount of source data that made it possible to 
 use mathematical statistics efficiently, i.\,e., to identify small 
 samples carrying information about hostile insiders. The difficulty 
 of the task was to lose as little as possible the needed small samples. 
The conditions have been found where in the series scheme, the probability of 
identifying insiders using collusion tends to~1.}

\KWE{identification of the organized group of hostile insiders; 
small samples; big data; mathematical statistics}



\DOI{10.14357/19922264200310} 

%\vspace*{-20pt}

\Ack
\noindent
The work was partially supported by the Russian Foundation 
for Basic Research (project 18-29-03081).

%\vspace*{6pt}

 \begin{multicols}{2}

\renewcommand{\bibname}{\protect\rmfamily References}
%\renewcommand{\bibname}{\large\protect\rm References}

{\small\frenchspacing
 {%\baselineskip=10.8pt
 \addcontentsline{toc}{section}{References}
 \begin{thebibliography}{9}
\bibitem{1-gr-1}
\Aue{Gheyas, I., and A.~Abdallah.} 2016. Detection and prediction of insider threats to cyber security: 
A~systematic literature review and meta-analysis. 
\textit{Big Data Anal.} 1:1--29.
doi: 10.1186/s41044-016-0006-0.
\bibitem{2-gr-1}
\Aue{Memory, A., H.\,G.~Goldberg, and T.\,E.~Senator.} 2013. Context-aware insider threat detection. 
\textit{Workshops at 27th Conference on Artificial Intelligence}. 44--47.
Available at:
 {\sf 
https://pdfs.semanticscholar.org/04aa/e6d97900ba62e9\linebreak 0b07ac682fb7bd8c2e1029.pdf}
(accessed August~13, 2020).

\bibitem{4-gr-1}
\Aue{Grusho, A.\,A., M.\,I.~Zabezhailo, D.\,V.~Smirnov, and E.\,E.~Timonina.} 2017. Model' 
mnozhestva informatsionnykh prostranstv v~zadache poiska insaydera [The model of the set of 
information spaces in the problem of insider detection]. \textit{Informatika i~ee Primeneniya~--- 
Inform. Appl.} 11(4):65--69.

\bibitem{3-gr-1}
\Aue{Grusho, A., N.~Grusho, and E.~Timonina.} 2020. Method of several information spaces for 
identification of anomalies. \textit{Intelligent distributed computing XIII}. Eds. I.~Kotenko, 
C.~Badica, V.~Desnitsky, D.~El Baz, and M.~Ivanovic. Studies in computational intelligence ser. 
Springer. 868:515--520.

\bibitem{5-g-1r}
\Aue{Axelsson, S.} 2000. The base-rate fallacy and its implications for the difficulty of intrusion 
detection. \textit{ACM~T. Inform. Syst. Se.} 3(3):186--205.
\bibitem{6-gr-1}
\Aue{Grusho, A., N.~Grusho, and E.~Timonina.} 2019. The bans in finite probability spaces and the 
problem of small samples. \textit{Distributed computer and communication networks}. Eds. 
V.\,M.~Vishnevskiy, K.\,E.~Samouylov, and D.\,V.~Kozyrev. Lecture notes in computer science ser. 
Springer. 11965:578--590.
\bibitem{7-gr-1}
\Aue{Kolchin, V.\,F., B.\,A.~Sevastyanov, and V.\,P.~Chistyakov.} 1976. 
\textit{Sluchaynye 
razmeshcheniya} [Random allocations]. Moscow: Nauka. 224~p.
\end{thebibliography}

 }
 }

\end{multicols}

\vspace*{-6pt}

\hfill{\small\textit{Received June 2, 2020}}

%\pagebreak

%\vspace*{-24pt}

\Contr

\noindent
\textbf{Grusho Alexander A.} (b.\ 1946)~--- Doctor of Science in physics and mathematics, professor, 
principal scientist, Institute of Informatics Problems, Federal Research Center ``Computer Sciences 
and Control'' of the Russian Academy of Sciences, 44-2~Vavilov Str., Moscow 119133, Russian 
Federation;  \mbox{grusho@yandex.ru}

\vspace*{3pt}

\noindent
\textbf{Zabezhailo Michael I.} (b.\ 1956)~--- Doctor of Science in physics and mathematics, principal 
scientist, A.\,A.~Dorodnicyn Computing Center, Federal Research Center ``Computer Science and 
Control'' of the Russian Academy of Sciences, 40~Vavilov Str., Moscow 119333, Russian Federation; 
\mbox{m.zabezhailo@yandex.ru}

\vspace*{3pt}

\noindent
\textbf{Smirnov Dmitry V.} (b.\ 1984)~--- business partner
 for IT security department, Sberbank of 
Russia, 19~Vavilov Str., Moscow 117999, Russian Federation; \mbox{dvlsmirnov@sberbank.ru}

\vspace*{3pt}

\noindent
\textbf{Timonina Elena E.} (b.\ 1952)~--- Doctor of Science in technology, professor, leading scientist, 
Institute of Informatics Problems, Federal Research Center ``Computer Sciences and Control'' of the 
Russian Academy of Sciences, 44-2~Vavilov Str., Moscow 119133, Russian Federation; 
\mbox{eltimon@yandex.ru}

\vspace*{3pt}

\noindent
\textbf{Shorgin Sergey Ya.} (b.\ 1952)~--- Doctor of Science in physics and mathematics, professor, 
principal scientist, Institute of Informatics Problems, Federal Research Center ``Computer Sciences 
and Control'' of the Russian Academy of Sciences, 44-2~Vavilov Str., Moscow 119133, Russian 
Federation; \mbox{sshorgin@ipiran.ru}





\label{end\stat}

\renewcommand{\bibname}{\protect\rm Литература} 